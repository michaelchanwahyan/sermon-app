\documentclass{book}
%\usepackage[letterpaper, portrait, margin=1cm]{geometry}
%\usepackage[letterpaper, bindingoffset=0.2in, left=1in,right=1in,top=.5in,bottom=.5in,footskip=.25in,marginparwidth=5em]{geometry}
\usepackage[letterpaper, left=1in,right=1in,top=.5in,bottom=.5in,footskip=.25in,marginparwidth=1cm]{geometry}
% ---------------------
% mini-table-of-content
% ---------------------
\usepackage{minitoc}
\setcounter{minitocdepth}{1}
\setlength{\mtcindent}{24pt}
\setcounter{secnumdepth}{-2}
%\renewcommand{\mtcfont}{\small\rm}
%\renewcommand{\mtcSfont}{\small\bf}
%\usepackage{setspace}
%\usepackage{tocloft}
%\setlength\cftparskip{-1.2pt}
%\setlength\cftbeforesecskip{1.3pt}
%\setlength\cftaftertoctitleskip{2pt}
%\renewcommand{\cftsecafterpnum}{\hspace*{02.0em}}
%\renewcommand{\cftsubsecafterpnum}{\hspace*{02.0em}}

% ---------------------------
% Chinese Characters Packages
% ---------------------------
\usepackage{fontspec} 
\usepackage{xeCJK}
\setmainfont{Times}
\setCJKmainfont{BiauKai}
\newfontfamily\sblgoodhebrew{SBL BibLit}[Script=Hebrew,Contextuals=Alternate]
\newfontfamily\sblgoodgreek{SBL BibLit}[Script=Greek,Contextuals=Alternate]

\usepackage{ifpdf,cite,algorithmic,url,tikz}
\usepackage[cmex10]{amsmath}

% ---------------------------
% Hebrew Characters Packages
% ---------------------------
\usepackage{polyglossia}
\setmainfont{Times New Roman}

% -------
% General
% -------
\usepackage{multicol}
\usepackage{multirow}
\usepackage{color,colortbl}
\usepackage{xparse}
\usepackage{pbox}
\usepackage{stackengine}
\usepackage{titlesec}% http://ctan.org/pkg/titlesec
\usepackage{tabularx}
\usepackage{xltabular}
\usepackage{titlesec}
\usepackage{makecell}
\newcommand{\sectionbreak}{\clearpage}

\author{
  Editor, Michael Chan\\
  \texttt{michaelchan\_wahyan@yahoo.com.hk}
}
\usepackage{tocloft}

\usepackage{hyperref}
\hypersetup{
    colorlinks=true, % set true if you want colored links
    linktoc   =all , % set to all if you want both sections and subsections linked
    linkcolor =blue, % choose some color if you want links to stand out
}

% ----------
% Afterword
% ----------
\usepackage{marginnote}
\usepackage{sectsty}
\usepackage{ragged2e}
\usepackage{lineno}
\usepackage{xcolor}
\usepackage{paracol}

\begin{document}

\clearpage
%% temporary titles
% command to provide stretchy vertical space in proportion
\newcommand\nbvspace[1][3]{\vspace*{\stretch{#1}}}
% allow some slack to avoid under/overfull boxes
\newcommand\nbstretchyspace{\spaceskip0.5em plus 0.25em minus 0.25em}
% To improve spacing on titlepages
\newcommand{\nbtitlestretch}{\spaceskip0.6em}
\pagestyle{empty}
\begin{center}
\bfseries
\nbvspace[1]
\Huge
{%\nbtitlestretch
\Large
\textbf{粵語講道逐字稿 2020-present \\
       Youtube Channel: 流堂 Flow Church
       }}

\nbvspace[1]

{\large
Editor: Michael\\
\texttt{michaelchan\_wahyan@yahoo.com.hk}
}

\nbvspace[1]

{\large
Revision: \texttt{v1.1}\\
Last Update: \today
}


\vfill
\begin{tikzpicture}
    %remove comment for OT cover%\node (0,0) [opacity=0.03]{\includegraphics[width=15cm]{../bible_out/ot_frontcover.png}} ;
    %remove comment for NT cover%\node (0,0) [opacity=0.03]{\includegraphics[width=15cm]{../bible_out/christ_on_cross.png}} ;
    %remove comment for Bible cover%\node (0,0) [xshift=0.8cm, yshift=+2cm, opacity=0.03]{\includegraphics[width=10cm]{./christ_on_cross.png}} ;
    %remove comment for Bible cover%\node (0,0) [              yshift=-2cm, opacity=0.03]{\includegraphics[width=14cm]{./ot_frontcover.png}} ;
\end{tikzpicture}
\vfill

\end{center}

\newpage

\setcounter{tocdepth}{0}
\dominitoc
\begin{multicols}{3}
\addtocontents{toc}{\protect\hypertarget{toc}{}}
\tableofcontents
\end{multicols}

\large
%\twocolumn

% the color definition syntax is as follow:
% \definecolor{name}{system}{definition}
% example: a mono-channel color can be defined as
%          \definecolor{Gray}{gray}{0.9}
% example: an rgb-3-channel color can be defined as
%          \definecolor{LightCyan}{rgb}{0.88,1,1}
%          \definecolor{pink}{rgb}{0.68,0,0.68}

\definecolor{CUV1LightRed}{rgb}{1,0.75,0.75}     % for CUV1
\definecolor{LZZVLightGray}{rgb}{0.9,0.9,0.9}    % for LZZ
\definecolor{KJVVLightGreen}{rgb}{0.75,1,0.85}   % for KJV
\definecolor{CUV2LightYellow}{rgb}{1,1,0.75}     % for CUV2
\definecolor{CNVVLightBrown}{rgb}{1,0.85,0.7}    % for CNV
\definecolor{NRSVLightBlue}{rgb}{0.75,1,1}       % for NRSV
\definecolor{WENLLightPurple}{rgb}{0.95,0.85,0.9}% for WENL
\definecolor{TCV19PaleGreen}{rgb}{0.85,1,0.95}   % for TCV19
\definecolor{MSGVLightWhite}{rgb}{0.98,0.98,0.98}% for MSGV
\definecolor{NETSLightRed}{rgb}{1,0.75,0.75}     % for NETS
\definecolor{JPS1917LightYellow}{rgb}{1,1,0.75}  % for JPS1917
\definecolor{SBLGNTPaleRed}{rgb}{1,0.85,0.80}    % for SBLGNT

\section{目錄\small{(順時)}}
\label{sec:index_chronic}
{ \scriptsize


\begin{xltabular}{\textwidth}{|p{0.15\textwidth} p{0.6\textwidth}|p{0.07\textwidth} p{0.1\textwidth}|}
\hline
腓利門書 1:1-25 & \hyperref[sec:J3EQacUFDFI]{【網上崇拜】再見…再相見 | 腓利門書1\_1-25 | 20210206 [J3EQacUFDFI]} & 2021-02-06 & \href{https://youtube.com/watch?v=J3EQacUFDFI}{\texttt{ J3EQacUFDFI}} \\
    & \hyperref[sec:GmLFDCSkId4]{【這是最好的時代:給香港基督徒的神學八課】第1講:亂世中才明白甚麼是「基督徒」|20210522 [GmLFDCSkId4]} & 2021-05-22 & \href{https://youtube.com/watch?v=GmLFDCSkId4}{\texttt{ GmLFDCSkId4}} \\
    & \hyperref[sec:OTk7WEa_w50]{【這是最好的時代:給香港基督徒的神學八課】第2講:究竟好消息有多好?|20210619 [OTk7WEa-w50]} & 2021-06-19 & \href{https://youtube.com/watch?v=OTk7WEa-w50}{\texttt{ OTk7WEa-w50}} \\
    & \hyperref[sec:dNWjC8vnhS0]{【這是最好的時代:給香港基督徒的神學八課】第3課: Let’s flow|20210726 [dNWjC8vnhS0]} & 2021-07-26 & \href{https://youtube.com/watch?v=dNWjC8vnhS0}{\texttt{ dNWjC8vnhS0}} \\
    & \hyperref[sec:d_aSxcuQPus]{【這是最好的時代:給香港基督徒的神學八課】第4課:亂世的靈性修持|20210822 [d\_aSxcuQPus]} & 2021-08-22 & \href{https://youtube.com/watch?v=d_aSxcuQPus}{\texttt{ d\_aSxcuQPus}} \\
    & \hyperref[sec:akT8yKiTNTo]{【這是最好的時代:給香港基督徒的神學八課】第5課:一根刺的人|20210918 [akT8yKiTNTo]} & 2021-09-18 & \href{https://youtube.com/watch?v=akT8yKiTNTo}{\texttt{ akT8yKiTNTo}} \\
    & \hyperref[sec:K2_OK28IM68]{【這是最好的時代:給香港基督徒的神學八課】第6課: Passion|20211018 [K2\_OK28IM68]} & 2021-10-18 & \href{https://youtube.com/watch?v=K2_OK28IM68}{\texttt{ K2\_OK28IM68}} \\
    & \hyperref[sec:hq6PGyJ3aBs]{【這是最好的時代:給香港基督徒的神學八課】第7課:今日的「我」推翻昨日的「我」|20211121 [hq6PGyJ3aBs]} & 2021-11-21 & \href{https://youtube.com/watch?v=hq6PGyJ3aBs}{\texttt{ hq6PGyJ3aBs}} \\
    & \hyperref[sec:HS1KRCnzG5o]{【這是最好的時代:給香港基督徒的神學八課】第8課:跟隨耶穌的一百萬個可能|20211218 [HS1KRCnzG5o]} & 2021-12-18 & \href{https://youtube.com/watch?v=HS1KRCnzG5o}{\texttt{ HS1KRCnzG5o}} \\
撒母耳記下 11:1-12:31-20230225 & \hyperref[sec:lsdGk_BkHa8]{【流堂崇拜】天光請開眼|撒母耳記下11\_1-12\_31|20230225 [lsdGk-BkHa8]} & 2023-02-25 & \href{https://youtube.com/watch?v=lsdGk-BkHa8}{\texttt{ lsdGk-BkHa8}} \\
    & \hyperref[sec:VfT5ldcLjqQ]{《致餘民及流散者:給香港基督徒的神學八課》第二季第1課|20230227 [VfT5ldcLjqQ]} & 2023-02-27 & \href{https://youtube.com/watch?v=VfT5ldcLjqQ}{\texttt{ VfT5ldcLjqQ}} \\
    & \hyperref[sec:7ZGXT0f30Z0]{《致餘民及流散者:給香港基督徒的神學八課》第二季第2課|20230325 [7ZGXT0f30Z0]} & 2023-03-25 & \href{https://youtube.com/watch?v=7ZGXT0f30Z0}{\texttt{ 7ZGXT0f30Z0}} \\
馬太福音 26:30-34-20230401 & \hyperref[sec:8KdYgVn_hzk]{【流堂崇拜】有關跌倒前的三件事|馬太福音26\_30-34|20230401 [8KdYgVn-hzk]} & 2023-04-01 & \href{https://youtube.com/watch?v=8KdYgVn-hzk}{\texttt{ 8KdYgVn-hzk}} \\
路加福音 23:32-43-20230408 & \hyperref[sec:v4hE6GM4QsI]{【流堂崇拜】我們都有錯|路加福音23\_32-43|20230408 [v4hE6GM4QsI]} & 2023-04-08 & \href{https://youtube.com/watch?v=v4hE6GM4QsI}{\texttt{ v4hE6GM4QsI}} \\
哈該書 1:1-15-20230422 & \hyperref[sec:S0X_1Lh_dHA]{【流堂崇拜】你先?我先?我地可能都會搞錯!|哈該書1\_1-15|20230422 [S0X-1Lh\_dHA]} & 2023-04-22 & \href{https://youtube.com/watch?v=S0X-1Lh_dHA}{\texttt{ S0X-1Lh\_dHA}} \\
以斯帖記 1:1-22-20230429 & \hyperref[sec:VNZbDAiXlG0]{【流堂崇拜】她和她最後的倔強|以斯帖記1\_1-22|20230429 [VNZbDAiXlG0]} & 2023-04-29 & \href{https://youtube.com/watch?v=VNZbDAiXlG0}{\texttt{ VNZbDAiXlG0}} \\
    & \hyperref[sec:gRf39gjSNbM]{《致餘民及流散者:給香港基督徒的神學八課》第二季第3課|20230512 [gRf39gjSNbM]} & 2023-05-12 & \href{https://youtube.com/watch?v=gRf39gjSNbM}{\texttt{ gRf39gjSNbM}} \\
    & \hyperref[sec:RYCxV16hfwM]{《致餘民及流散者:給香港基督徒的神學八課》第二季第4課|20230528 [RYCxV16hfwM]} & 2023-05-28 & \href{https://youtube.com/watch?v=RYCxV16hfwM}{\texttt{ RYCxV16hfwM}} \\
    & \hyperref[sec:I6Z1WA7E0RA]{《致餘民及流散者:給香港基督徒的神學八課》第二季第5課|20230625 [I6Z1WA7E0RA]} & 2023-06-25 & \href{https://youtube.com/watch?v=I6Z1WA7E0RA}{\texttt{ I6Z1WA7E0RA}} \\
    & \hyperref[sec:2QyWxsVtL8E]{《致餘民及流散者:給香港基督徒的神學八課》第二季第6課|20230924 [2QyWxsVtL8E]} & 2023-09-24 & \href{https://youtube.com/watch?v=2QyWxsVtL8E}{\texttt{ 2QyWxsVtL8E}} \\
    & \hyperref[sec:JKdFzjAsLZY]{《致餘民及流散者:給香港基督徒的神學八課》第二季第7課|20231101 [JKdFzjAsLZY]} & 2023-11-01 & \href{https://youtube.com/watch?v=JKdFzjAsLZY}{\texttt{ JKdFzjAsLZY}} \\
啟示錄 12:1-17-20231125 & \hyperref[sec:2LIl7VilU18]{【網上崇拜】細路仔唔識世界|啟示錄12\_1-17|20231125 [2LIl7VilU18]} & 2023-11-25 & \href{https://youtube.com/watch?v=2LIl7VilU18}{\texttt{ 2LIl7VilU18}} \\
    & \hyperref[sec:w1NzLUX2_GE]{《致餘民及流散者:給香港基督徒的神學八課》第二季第8課|20231126 [w1NzLUX2\_GE]} & 2023-11-26 & \href{https://youtube.com/watch?v=w1NzLUX2_GE}{\texttt{ w1NzLUX2\_GE}} \\
詩篇 90:1-17-20231202 & \hyperref[sec:lfg8MyM5M04]{【網上崇拜】大人者,不失其赤子之心者也|詩篇90\_1-17|20231202 [lfg8MyM5M04]} & 2023-12-02 & \href{https://youtube.com/watch?v=lfg8MyM5M04}{\texttt{ lfg8MyM5M04}} \\
路加福音 21:25-28-34-36-20231209 & \hyperref[sec:0oiGMpkgXB8]{【網上崇拜】超級耶穌基督,驚奇!|路加福音21\_25-28,34-36|20231209 [0oiGMpkgXB8]} & 2023-12-09 & \href{https://youtube.com/watch?v=0oiGMpkgXB8}{\texttt{ 0oiGMpkgXB8}} \\
馬太福音 8:1-4-20231216 & \hyperref[sec:sKBDQD8UIMg]{【網上聖餐崇拜】年少多好|馬太福音8\_1-4|20231216 [sKBDQD8UIMg]} & 2023-12-16 & \href{https://youtube.com/watch?v=sKBDQD8UIMg}{\texttt{ sKBDQD8UIMg}} \\
路加福音 2:10-12-49-20231223 & \hyperref[sec:dT3dN2jF8BQ]{【網上崇拜】天選的細路|路加福音2\_10-12,49|20231223 [dT3dN2jF8BQ]} & 2023-12-23 & \href{https://youtube.com/watch?v=dT3dN2jF8BQ}{\texttt{ dT3dN2jF8BQ}} \\
耶利米書 1:1-8-20231230 & \hyperref[sec:9ztySs_vnP4]{【網上崇拜】一道風景,一種心靈|耶利米書1\_1-8|20231230 [9ztySs-vnP4]} & 2023-12-30 & \href{https://youtube.com/watch?v=9ztySs-vnP4}{\texttt{ 9ztySs-vnP4}} \\
\end{xltabular}
}
\newpage

\section{目錄\small{(順卷)}}
\label{sec:index_scriptual}
{ \scriptsize


\begin{xltabular}{\textwidth}{|p{0.15\textwidth} p{0.6\textwidth}|p{0.07\textwidth} p{0.1\textwidth}|}
\hline
撒母耳記下 11:1-12:31-20230225 & \hyperref[sec:lsdGk_BkHa8]{【流堂崇拜】天光請開眼|撒母耳記下11\_1-12\_31|20230225 [lsdGk-BkHa8]} & 2023-02-25 & \href{https://youtube.com/watch?v=lsdGk-BkHa8}{\texttt{ lsdGk-BkHa8}} \\
耶利米書 1:1-8-20231230 & \hyperref[sec:9ztySs_vnP4]{【網上崇拜】一道風景,一種心靈|耶利米書1\_1-8|20231230 [9ztySs-vnP4]} & 2023-12-30 & \href{https://youtube.com/watch?v=9ztySs-vnP4}{\texttt{ 9ztySs-vnP4}} \\
哈該書 1:1-15-20230422 & \hyperref[sec:S0X_1Lh_dHA]{【流堂崇拜】你先?我先?我地可能都會搞錯!|哈該書1\_1-15|20230422 [S0X-1Lh\_dHA]} & 2023-04-22 & \href{https://youtube.com/watch?v=S0X-1Lh_dHA}{\texttt{ S0X-1Lh\_dHA}} \\
詩篇 90:1-17-20231202 & \hyperref[sec:lfg8MyM5M04]{【網上崇拜】大人者,不失其赤子之心者也|詩篇90\_1-17|20231202 [lfg8MyM5M04]} & 2023-12-02 & \href{https://youtube.com/watch?v=lfg8MyM5M04}{\texttt{ lfg8MyM5M04}} \\
以斯帖記 1:1-22-20230429 & \hyperref[sec:VNZbDAiXlG0]{【流堂崇拜】她和她最後的倔強|以斯帖記1\_1-22|20230429 [VNZbDAiXlG0]} & 2023-04-29 & \href{https://youtube.com/watch?v=VNZbDAiXlG0}{\texttt{ VNZbDAiXlG0}} \\
馬太福音 8:1-4-20231216 & \hyperref[sec:sKBDQD8UIMg]{【網上聖餐崇拜】年少多好|馬太福音8\_1-4|20231216 [sKBDQD8UIMg]} & 2023-12-16 & \href{https://youtube.com/watch?v=sKBDQD8UIMg}{\texttt{ sKBDQD8UIMg}} \\
馬太福音 26:30-34-20230401 & \hyperref[sec:8KdYgVn_hzk]{【流堂崇拜】有關跌倒前的三件事|馬太福音26\_30-34|20230401 [8KdYgVn-hzk]} & 2023-04-01 & \href{https://youtube.com/watch?v=8KdYgVn-hzk}{\texttt{ 8KdYgVn-hzk}} \\
路加福音 2:10-12-49-20231223 & \hyperref[sec:dT3dN2jF8BQ]{【網上崇拜】天選的細路|路加福音2\_10-12,49|20231223 [dT3dN2jF8BQ]} & 2023-12-23 & \href{https://youtube.com/watch?v=dT3dN2jF8BQ}{\texttt{ dT3dN2jF8BQ}} \\
路加福音 21:25-28-34-36-20231209 & \hyperref[sec:0oiGMpkgXB8]{【網上崇拜】超級耶穌基督,驚奇!|路加福音21\_25-28,34-36|20231209 [0oiGMpkgXB8]} & 2023-12-09 & \href{https://youtube.com/watch?v=0oiGMpkgXB8}{\texttt{ 0oiGMpkgXB8}} \\
路加福音 23:32-43-20230408 & \hyperref[sec:v4hE6GM4QsI]{【流堂崇拜】我們都有錯|路加福音23\_32-43|20230408 [v4hE6GM4QsI]} & 2023-04-08 & \href{https://youtube.com/watch?v=v4hE6GM4QsI}{\texttt{ v4hE6GM4QsI}} \\
腓利門書 1:1-25 & \hyperref[sec:J3EQacUFDFI]{【網上崇拜】再見…再相見 | 腓利門書1\_1-25 | 20210206 [J3EQacUFDFI]} & 2021-02-06 & \href{https://youtube.com/watch?v=J3EQacUFDFI}{\texttt{ J3EQacUFDFI}} \\
啟示錄 12:1-17-20231125 & \hyperref[sec:2LIl7VilU18]{【網上崇拜】細路仔唔識世界|啟示錄12\_1-17|20231125 [2LIl7VilU18]} & 2023-11-25 & \href{https://youtube.com/watch?v=2LIl7VilU18}{\texttt{ 2LIl7VilU18}} \\
    & \hyperref[sec:VfT5ldcLjqQ]{《致餘民及流散者:給香港基督徒的神學八課》第二季第1課|20230227 [VfT5ldcLjqQ]} & 2023-02-27 & \href{https://youtube.com/watch?v=VfT5ldcLjqQ}{\texttt{ VfT5ldcLjqQ}} \\
    & \hyperref[sec:7ZGXT0f30Z0]{《致餘民及流散者:給香港基督徒的神學八課》第二季第2課|20230325 [7ZGXT0f30Z0]} & 2023-03-25 & \href{https://youtube.com/watch?v=7ZGXT0f30Z0}{\texttt{ 7ZGXT0f30Z0}} \\
    & \hyperref[sec:gRf39gjSNbM]{《致餘民及流散者:給香港基督徒的神學八課》第二季第3課|20230512 [gRf39gjSNbM]} & 2023-05-12 & \href{https://youtube.com/watch?v=gRf39gjSNbM}{\texttt{ gRf39gjSNbM}} \\
    & \hyperref[sec:RYCxV16hfwM]{《致餘民及流散者:給香港基督徒的神學八課》第二季第4課|20230528 [RYCxV16hfwM]} & 2023-05-28 & \href{https://youtube.com/watch?v=RYCxV16hfwM}{\texttt{ RYCxV16hfwM}} \\
    & \hyperref[sec:I6Z1WA7E0RA]{《致餘民及流散者:給香港基督徒的神學八課》第二季第5課|20230625 [I6Z1WA7E0RA]} & 2023-06-25 & \href{https://youtube.com/watch?v=I6Z1WA7E0RA}{\texttt{ I6Z1WA7E0RA}} \\
    & \hyperref[sec:2QyWxsVtL8E]{《致餘民及流散者:給香港基督徒的神學八課》第二季第6課|20230924 [2QyWxsVtL8E]} & 2023-09-24 & \href{https://youtube.com/watch?v=2QyWxsVtL8E}{\texttt{ 2QyWxsVtL8E}} \\
    & \hyperref[sec:JKdFzjAsLZY]{《致餘民及流散者:給香港基督徒的神學八課》第二季第7課|20231101 [JKdFzjAsLZY]} & 2023-11-01 & \href{https://youtube.com/watch?v=JKdFzjAsLZY}{\texttt{ JKdFzjAsLZY}} \\
    & \hyperref[sec:w1NzLUX2_GE]{《致餘民及流散者:給香港基督徒的神學八課》第二季第8課|20231126 [w1NzLUX2\_GE]} & 2023-11-26 & \href{https://youtube.com/watch?v=w1NzLUX2_GE}{\texttt{ w1NzLUX2\_GE}} \\
    & \hyperref[sec:GmLFDCSkId4]{【這是最好的時代:給香港基督徒的神學八課】第1講:亂世中才明白甚麼是「基督徒」|20210522 [GmLFDCSkId4]} & 2021-05-22 & \href{https://youtube.com/watch?v=GmLFDCSkId4}{\texttt{ GmLFDCSkId4}} \\
    & \hyperref[sec:OTk7WEa_w50]{【這是最好的時代:給香港基督徒的神學八課】第2講:究竟好消息有多好?|20210619 [OTk7WEa-w50]} & 2021-06-19 & \href{https://youtube.com/watch?v=OTk7WEa-w50}{\texttt{ OTk7WEa-w50}} \\
    & \hyperref[sec:dNWjC8vnhS0]{【這是最好的時代:給香港基督徒的神學八課】第3課: Let’s flow|20210726 [dNWjC8vnhS0]} & 2021-07-26 & \href{https://youtube.com/watch?v=dNWjC8vnhS0}{\texttt{ dNWjC8vnhS0}} \\
    & \hyperref[sec:d_aSxcuQPus]{【這是最好的時代:給香港基督徒的神學八課】第4課:亂世的靈性修持|20210822 [d\_aSxcuQPus]} & 2021-08-22 & \href{https://youtube.com/watch?v=d_aSxcuQPus}{\texttt{ d\_aSxcuQPus}} \\
    & \hyperref[sec:akT8yKiTNTo]{【這是最好的時代:給香港基督徒的神學八課】第5課:一根刺的人|20210918 [akT8yKiTNTo]} & 2021-09-18 & \href{https://youtube.com/watch?v=akT8yKiTNTo}{\texttt{ akT8yKiTNTo}} \\
    & \hyperref[sec:K2_OK28IM68]{【這是最好的時代:給香港基督徒的神學八課】第6課: Passion|20211018 [K2\_OK28IM68]} & 2021-10-18 & \href{https://youtube.com/watch?v=K2_OK28IM68}{\texttt{ K2\_OK28IM68}} \\
    & \hyperref[sec:hq6PGyJ3aBs]{【這是最好的時代:給香港基督徒的神學八課】第7課:今日的「我」推翻昨日的「我」|20211121 [hq6PGyJ3aBs]} & 2021-11-21 & \href{https://youtube.com/watch?v=hq6PGyJ3aBs}{\texttt{ hq6PGyJ3aBs}} \\
    & \hyperref[sec:HS1KRCnzG5o]{【這是最好的時代:給香港基督徒的神學八課】第8課:跟隨耶穌的一百萬個可能|20211218 [HS1KRCnzG5o]} & 2021-12-18 & \href{https://youtube.com/watch?v=HS1KRCnzG5o}{\texttt{ HS1KRCnzG5o}} \\
\end{xltabular}
}
\newpage



\section{腓利門書 1:1-25}
\label{sec:J3EQacUFDFI}
\textbf{【網上崇拜】再見…再相見 | 腓利門書1\_1-25 | 20210206 [J3EQacUFDFI]}
\newline
\newline
連結: \href{https://youtube.com/watch?v=J3EQacUFDFI}{\texttt{ https://youtube.com/watch?v=J3EQacUFDFI}} ~~~~ 語音日期: 2021-02-06 
\newline
\newline
\hyperref[sec:OfpcASuB51M]{\small{< < < PREV SERMON < < <}}
~
\hyperref[sec:index_chronic]{\small{[返順時目]}}
~
\hyperref[sec:index_scriptual]{\small{[返順卷目]}}
~
\hyperref[sec:yt29IwiNPNk]{\small{> > > NEXT SERMON > > >}}
\newline
\newline
腓利門書 1:1-25
\newline
\begin{longtable}{cl}
\hline
\hline
章節 & 經文 (和合本修訂版)\\
\hline
1:1 & \begin{tabularx}{0.7\textwidth}{X} 為基督耶穌被囚的保羅,同弟兄提摩太,寫信給我們所親愛的同工腓利門、 \end{tabularx} \\ \\ \relax
1:2 & \begin{tabularx}{0.7\textwidth}{X} 亞腓亞姊妹,和我們的戰友亞基布,以及在你家裡的教會。 \end{tabularx} \\ \\ \relax
1:3 & \begin{tabularx}{0.7\textwidth}{X} 願恩惠、平安從我們的父神和主耶穌基督歸給你們! \end{tabularx} \\ \\ \relax
1:4 & \begin{tabularx}{0.7\textwidth}{X} 我在禱告中記念你的時候,常為你感謝我的神, \end{tabularx} \\ \\ \relax
1:5 & \begin{tabularx}{0.7\textwidth}{X} 因聽說你對眾聖徒的愛心,和你對主耶穌的信心。 \end{tabularx} \\ \\ \relax
1:6 & \begin{tabularx}{0.7\textwidth}{X} 願你與人分享信心的時候,能產生功效,讓人知道我們所行的各樣善事都是為基督做的。 \end{tabularx} \\ \\ \relax
1:7 & \begin{tabularx}{0.7\textwidth}{X} 弟兄啊,由於你的愛心,我得到極大的快樂和安慰,因為眾聖徒的心從你得到舒暢。 \end{tabularx} \\ \\ \relax
1:8 & \begin{tabularx}{0.7\textwidth}{X} 雖然我靠著基督能放膽吩咐你做該做的事, \end{tabularx} \\ \\ \relax
1:9 & \begin{tabularx}{0.7\textwidth}{X} 可是像我這上了年紀的保羅,現在又是為基督耶穌被囚的,寧可憑著愛心求你, \end{tabularx} \\ \\ \relax
1:10 & \begin{tabularx}{0.7\textwidth}{X} 就是為我在捆鎖中所生的兒子阿尼西謀求你。 \end{tabularx} \\ \\ \relax
1:11 & \begin{tabularx}{0.7\textwidth}{X} 從前他與你沒有益處,但如今與你我都有益處。 \end{tabularx} \\ \\ \relax
1:12 & \begin{tabularx}{0.7\textwidth}{X} 我現在打發他回到你那裡去,他是我心肝。 \end{tabularx} \\ \\ \relax
1:13 & \begin{tabularx}{0.7\textwidth}{X} 我本來有意將他留下,在我為福音所受的捆鎖中替你伺候我。 \end{tabularx} \\ \\ \relax
1:14 & \begin{tabularx}{0.7\textwidth}{X} 但不知道你的意見,我不願意這樣做,好使你的善行不是出於勉強,而是出於自願。 \end{tabularx} \\ \\ \relax
1:15 & \begin{tabularx}{0.7\textwidth}{X} 他暫時離開你,也許是要讓你永遠得著他, \end{tabularx} \\ \\ \relax
1:16 & \begin{tabularx}{0.7\textwidth}{X} 不再是奴隸,而是高過奴隸,是親愛的弟兄;對我確實如此,何況對你呢!無論在肉身或在主裡更是如此。 \end{tabularx} \\ \\ \relax
1:17 & \begin{tabularx}{0.7\textwidth}{X} 所以,你若以我為同伴,就接納他,如同接納我一樣。 \end{tabularx} \\ \\ \relax
1:18 & \begin{tabularx}{0.7\textwidth}{X} 他若虧負你,或欠你甚麼,都算在我的賬上吧, \end{tabularx} \\ \\ \relax
1:19 & \begin{tabularx}{0.7\textwidth}{X} 我必償還。這是我—保羅親筆寫的。我並不用對你說,甚至你自己也虧欠我呢! \end{tabularx} \\ \\ \relax
1:20 & \begin{tabularx}{0.7\textwidth}{X} 弟兄啊,希望你使我在主裡因你得益處,讓我的心在基督裡得到舒暢。 \end{tabularx} \\ \\ \relax
1:21 & \begin{tabularx}{0.7\textwidth}{X} 我寫信給你,深信你必順服,知道你所要做的,必過於我所說的。 \end{tabularx} \\ \\ \relax
1:22 & \begin{tabularx}{0.7\textwidth}{X} 此外,還請給我預備住處,因為我盼望藉著你們的禱告,必蒙恩回到你們那裡去。 \end{tabularx} \\ \\ \relax
1:23 & \begin{tabularx}{0.7\textwidth}{X} 為基督耶穌與我一同坐監的以巴弗問候你。 \end{tabularx} \\ \\ \relax
1:24 & \begin{tabularx}{0.7\textwidth}{X} 我的同工馬可、亞里達古、底馬、路加也都問候你。 \end{tabularx} \\ \\ \relax
1:25 & \begin{tabularx}{0.7\textwidth}{X} 願主耶穌基督的恩與你們的靈同在。 \end{tabularx} \\ \\
[1ex]
\hline
\hline
\end{longtable}
$^{1}$各位姐妹平安.
不知道你們是否感受到現場的upbeat.
因為一連幾首快歌.
我就唱不完這麼多首快歌.
因為我沒有氣.
怕趕到的時候不夠氣.
我感受到勁敝隊帶給我們那種向前的感覺.
我今天的訊息都是希望有一個向前的動力.
或者在訊息當中和大家去想想.
新的一年新的開始的時候.
move on是甚麼回事呢.
今天我的訊息是說「肥你門」.
「肥你門」書裡面的內容是一段書信.
一會兒會和大家詳細了解內容.
其實今天選「肥你門」的訊息.
其實想呼應上個星期「做音目者」.
是說關於舊的事情做一個終結.
或者say goodbye.
或者當再要去面對舊.
面對曾經的時候.
其實下一步應該可以做些甚麼呢.
所以今天講題是「再見」.
一個是past tense.
但是如果再相見的時候.
是一個現在式或者future tense的時候.
對我們來說我們應該要預備些甚麼呢.
我就選了「肥你門」書.
「肥你門」書是一個比較短的書信.
對於你來說可能你很久之前看過.
因為很短.
一章裡面只有25節就完結了整件事.
很快別人問你讀了多少章書信.
你就讀完一卷.
很快就讀完一卷.
所以如果你沒看過不要緊.
一會兒有聲音導航.
用廣東話口語去讀一次「肥你門」書.
四分多鐘而已.
那把聲音都挺好聽的.
但不是我錄的.

$^{41}$希望在一直看著經文的時候.
經文是新日本的文字.
但用廣東話口語的方法去看這篇書信.
我們先聽聽讀經的內容.
「肥你門」書第一章.
為基督耶穌的緣故.
而成為囚犯的保羅.
和我們的弟兄提摩泰.
寫信給我們親愛的弟兄和同工肥你門.
還有姐妹阿肥亞和我們的戰友阿基布.
以及在你家裡追集的教會.
願恩典和平安.
從我們的父神和主耶穌基督臨到你們.
我在祈禱當中看到你的時候.
因為我聽說你對神的指紋是有愛心.
和對主耶穌是有信心.
願你能夠有效地和人分享信仰.
更深明白我們所做的一切美善的事.
都是為了基督.
弟兄啊.
所有的聖徒的心既然因為你而得到暢快.
我也因為你的愛心得到更大的喜樂和安慰.
我雖然靠著基督可以大膽地吩咐你.
去做那些你所應該做的事.
但是我情願憑著愛心來懇求你.
看在我這個上了年紀的.
現在又為了基督耶穌的緣故.
成為了囚犯的保羅的份上.
就是為了我在困所當中所生的兒子.
阿尼謀懇求你.
他過去對你來說是沒有什麼益處.
但是現在對你和對我來說.
都是有益處的.
我現在叫他自己.
即是我的心肝回到你那裡.
其實我本來是想將他留在身邊.
等他在我為福音被困所的時候.
代替你來服侍我.
不過,未得你的同意.
我並不願意這樣做.

$^{81}$叫你向我作出的任何幫助.
都不是出於勉強.
而是出於自願.
或者,他之前暫時離開你.
就只是為了讓你可以永遠地得到他.
不再是一個奴隸.
而是遠遠超過一個奴隸.
是作為一個親愛的兄弟.
他對我來說尤其是這樣.
更何況是對你呢?.
無論是在身體上.
還是在主裡面.
都是這樣.
所以,你如果還當我是你的拍檔.
就請你接納他.
好像你接納我一樣.
如果他有什麼得罪了你.
還是有什麼欠了你.
請你全部都算在我的身上.
我是一定會還給你的.
這句是我保羅親手寫的.
其實,不用我說你都知道.
連你自己都欠了我.
兄弟,我真的希望.
你使我在主裡面因你得益.
讓我的心在基督裡面得到安慰.
我寫信給你.
深信你一定會照著這樣做.
而你所要做的.
一定是會超過我所要求的.
另外還有一件事.
就是請你為我預備住的地方.
因為我盼望藉著你們的祈禱.
我可以蒙恩到你們那裡去.
為基督耶穌的緣故.
和我一起被囚禁的以巴忽問候你.
和我一起同宮的馬可.
阿彌達古.
底瑪.
路加.

$^{121}$也都問候你.
願我們主耶穌基督的恩典.
跟你的靈同在.
阿們.
(音樂結束).
不知道大家聽了廣東話的口語旁述.
是不是很親切呢.
當有時候讀經有些乏味.
覺得文字很生硬的時候.
你試一下讀經用我們本土語.
香港廣東話去讀.
其實感覺會來得親切.
我自己小小分享.
很多時候讀書信的時候.
都用對話的形式.
我想那種感覺你就會明白.
受書人或者寫書的人.
其實那種情懷是什麼.
有時候我們讀經.
可能會著重一些字面.
或者一些字意上的分析.
但其實你記住.
當初寫信的時候.
是將整個信仰的反省.
或者在信仰當中的那種得著.
和受書的人明白.
《肥理門》就是一本這樣的書信.
如果你看保羅的書信當中.
監獄書信有四卷.
是哥羅西書,爾忽所書.
《肥理門》書和《肥立比》書.
四卷當中《肥理門》書是最短的.
但我自己覺得這卷書.
能夠納入在新約的全書中.
有很濃厚的神學信息.
而當中能夠應用在我們生活中.
剛才我不知道你聽旁述的時候.
你感受到什麼.
我很喜歡這個表達方法.
因為它將那種情心告訴你.

$^{161}$譬如他用了心肝.
又或者在語氣的時候.
那種意境受書人會明白到.
兄弟啊,我知道這樣寫.
你也會照做的.
正正就是很熟悉受書人和他那種關係.
而將最底層的底牌告訴受書人.
整本書剛才你看下去的時候.
就是一個人做錯事.
離開了《肥理門》.
然後出走的一個奴隸阿尼西姆.
整件事就是去到保羅之後.
阿尼西姆他自己經歷了福音的改變.
輾轉就去到保羅的身邊.
成為協助保羅的其中一人.
詳細的情況這卷書上沒有提及.
但是在西書第四章裡面.
這個叫推基古和阿尼西姆的名字就出現.
因為保羅在獄中寫了的書信.
其中有兩卷書.
一卷是哥羅西書和《肥理門》書.
這兩卷書就透過這兩個人送達去.
推基古就帶著阿尼西姆.
去到哥羅西達地方.
因為當我們知道哥羅西書裡面的描述.
就出現了肥理門這個人.
知道他是來自哥羅西教會的人.
所以保羅就猜推基古和阿尼西姆.
就回到哥羅西達地方.
就將哥羅西書帶給哥羅西教會.
同樣都將阿尼西姆帶回去給肥理門.
就是肥理門書.
所以整件事肥理門就和阿尼西姆相遇.
接著阿尼西姆就停留在肥理門家.
推基古就再帶著哥羅西書去哥羅多教.
繼續他的行程.
第三卷書就帶去爾弗所教會.
《監獄書信》就是這樣完成了這三卷的帶動.
精妙的地方就是.
為什麼保羅要阿尼西姆.

$^{201}$要回去他原先離開的肥理門家.
整件事就是.
奴隸出走當然有他的原因.
奴隸出走有原因的時候.
現在奴隸出走了.
去了一個新地方.
有一個新生活.
但保羅為什麼要將阿尼西姆帶回去肥理門.
這個就是整卷書要上澄清的地方.
對於你來說.
既然都有一個新恩一葉.
為什麼還要回去舊的地方.
我們曾經在Flo Church開堂的時候.
應該是一百年.
當我們辦了兩個大型聚會的時候.
就是回魂夜和解慰之後.
我們中間都做了一些大型的團契.
分別有不同的人可以聚集.
一起去聊聊天.
去到開堂的時候.
都被人問一個問題.
當人們來到Flo Church之後.
你們有聚會.
到最後他們療完傷.
你們是不是會推回原本的教會.
如果這樣你會不會來Flo Church.
你應該會轉頭吧.
我剛剛才離開一個地方.
你現在就推回去那裡.
Flo Church是不是一個療傷的地方呢.
但我們都跟他說.
那件事不是那麼簡單.
從來離開一個地方有它的原因.
在整本書上是否沒有解釋.
當日亞歷西伯是因為什麼事離開菲利門.
沒有解釋.
很難揣測.
當然有些聖經書或者作者解釋.
其實可能保羅是想這件事.
去煽動奴隸制度.

$^{241}$但我不覺得這件事是主打.
反而就是沒有著墨的地方.
是不是不重要呢.
我不覺得不是重要.
之後保羅如何處理亞歷西伯和他的舊宿主.
即是菲利門之間的一種關係.
我今天重點就是.
前關係是沒有了.
但再見面的時候.
我們的心態是怎樣.
我今天想跟大家一起去想這個問題.
不代表一個訊息是可以完成解決的方法.
但我覺得我們很多時候要面對我們的曾經.
你舊有很熟悉的朋友.
你斷了關係.
你在街上見到他的時候.
你會點個頭問個安.
之後就走.
還是當看不到立即看電話.
還是怎樣呢.
還是你會很主動和他聊天呢.
如果那個是普通朋友可能可以.
很熟的朋友你會發現有些嫌隙離開.
如果那個是你的前度呢.
你是否活得比你好.
立即整理好一點.
整理好一點髮型.
是要讓你看到我現在多魅力.
你是用什麼心態去再相遇呢.
整件事對我們來說.
是很複雜的事.
亞歷西伯在文字裡面的描述不多.
不知道他做過什麼.
可能他真的偷了錢.
人們說他偷了錢.
因為保羅要給他錢.
或者交帳.
但是不是偷錢呢.
不知道.
但總之結果就是他出走了.

$^{281}$出走了一個地方的時候.
他就信了主.
然後就認識了被囚的保羅.
保羅就adopt了他.
成為他的helper.
然後就繼續去form工作.
就差他不和他form事工.
保羅剛才在經文裡面說.
原本想收回來自己用.
但因為我覺得不是我的.
我要將他交回給你.
所以亞歷西伯做了什麼.
是不知道.
他們中間的糾結地方是不知道.
另外一邊想看看肥李門.
肥李門是什麼人呢.
經文裡面開頭說的一件事就是.
其實他有愛心.
一開始經文就說他愛心很好.
很有名聲.
而他家裡就成為家庭教會的聚腳點.
他是有錢.
一個patron.
他可以促養奴隸.
他一定是有一定的財力.
所以他有錢又有愛心.
教會又在這裡參與的時候.
其實他的名聲很好.
那你可能就會說.
他名聲這麼好.
走得那個奴隸當然是頑皮.
就是不識貨.
很快就會順理成章.
他走就是他不對.
他一定是偷錢走的.
有時候我們聽別人的故事.
或者我們了解案情的時候.
我自己很多時候都說.
聽一邊那個一定是苦主.
就因為在淑齡的角度.

$^{321}$我們覺得人的雜性.
凡是那件事要靠近身體追討的時候.
一定會推卸.
就不覺得自己是自身是內.
一定是自身是外.
想排他那種追討.
如果這樣看.
阿彌西武是不OK的.
但是肥李門是不是因為這樣說就很OK呢.
我們都不知道.
因為不是實質的案情.
問題就是第三個.
保羅.
保羅有什麼事要做這個中介呢.
其實出走了就OK.
那他就給他一個新生活.
他離開了.
就算是《生命記》第23章都在說.
如果一個奴隸是離開了.
他的主人的時候出走了.
能夠接待他的人.
應該要給他一個安全屋住.
不應該送他回去.
舊約是這樣教的.
但是整件事保羅沒有接待他.
應該說保羅有接待他.
但是他沒有一直和保羅住.
保羅明白了之後.
反而送他回去.
那會不會推阿彌西武回去性死呢.
或者其實是令到阿彌西武很尷尬.
當我再見到我的宿主的時候.
我應該要什麼呢.
我可能帶著一些虧欠.
帶著一些對他來說的不適.
再一次面對他的時候.
是很尷尬的.
我現在和你隔著.
屏幕可能不太感受到.
但可能面對面的時候.

$^{361}$你就難搞一點.
你自己問一問自己.
你們很多都是適婚年齡.
現在疫情不能喝酒.
如果去喝酒的時候.
當然我不是叫你去X那裡喝酒.
我的point就是.
你的朋友會結婚的.
你們有common friend的.
你會問他去不去.
他去我不會去.
不要浪費我的位置.
我不做人情的.
整件事就是你不想再見到他.
這個就是point.
你一次可以不見.
但是突然而來在街上碰到的時候.
又不是回到剛才的東西.
回到剛才的東西就是扮作不到.
這個是一個很親密的關係.
如果相對沒有那麼親密.
但是有很多共事的關係.
就是舊教會的人.
舊教會的人.
你和他合作了很久.
又一起有很多服侍的空間.
大家相處日子都不少.
你再見回舊教會的人.
你是用什麼方式呢.
你有沒有重整過自己的心態.
以至再面對舊友的關係呢.
還是你覺得我們沒有交集的人生當中.
我們老死不相往來.
就當不認識這個人呢.
你是會用選擇這個方式.
我無意淡化當中你們受到的傷害.
你們面對的嚴規.
或者中間很糾纏的那種來來回回的對話.
這些完全是真實的.
我明白.

$^{401}$但是在過去Full Church開堂這兩年.
我聽到最近這些故事.
很多.
而Cold Call的時候.
每個星期我都會打電話給一些留名.
62745377那些.
即是我們的電話號碼.
我都和他們談的時候.
都聽一次他們的故事.
我說所有故事都是真實來經歷的.
但是故事的全部都是真實的.
但是故事之後.
今天要面對同樣都真實的.
你來到Full Church可能是療傷.
你來到Full Church可能是一個過渡.
但是如果你真的再面對舊有的教會的人.
你會用什麼心態.
又或者你用什麼形式呢.
這個就是我希望在《肥理門書》裡和大家去高呼.
其實保羅做了什麼呢.
保羅他要了解什麼呢.
以至他覺得他這麼大膽.
他居然推翻一個能夠出走的人.
回到他自己的宿主那裡.
其實他沒理由要推翻他.
你看看《肥理門》又不缺錢.
少一個奴隸什麼所謂.
其實不需要爭一個人.
反而有人在幫你.
豈不是更好.
他可以不用帶他.
所以帶他回去的時候.
冷嘲熱諷又不是.
自知不理又不是.
其實保羅在做什麼.
其實是好心做壞事.
但是從另一個角度來說.
我想問你.
你有沒有試過抱一個人.
不是抱那種抱.

$^{441}$是因為那個人你很認識他.
有些事他做錯.
他現在悔改了.
他想回頭你抱他.
你力戰他.
以至他不要緊我支持你.
我跟你一起去.
你有沒有試過抱一個人.
如果你試過.
你知道通常都是中招.
是不是.
出街都是很正常的.
中招是廣東話.
你會發覺一件事.
你要包底.
你要多熟悉這個人.
你有多熟悉這個人.
其實你又不是很熟悉.
但你又要抱他.
你憑什麼覺得你很肯定要抱他.
還有一個大前提.
可能他不是知道事實的全部.
你一直都聽著亞歷西姆說.
為什麼要出走的時候.
其實相反是怎樣.
其實保羅不是不認識菲利門.
你明白嗎.
剛才書信都說到菲利門是他認識的.
所以兩邊廂都知道.
保羅就做一樣東西.
就是將亞歷西姆帶回去給菲利門.
這個位置是有些問題的.
問題是還有很多細節位.
可能都要想一下.
如果真的要抱這個.
要推他回去.
或者帶他回去的時候.
是不是融合到呢.
你需要打底嗎.
所以保羅就寫一封信給亞歷西姆.

$^{481}$和推基股一起.
如果是亞歷西姆.
就去見菲利門.
你先看這封信.
你先不要罵我.
你先看這封信.
接著他讀完這封信的時候.
就等菲利門的反應.
整個過程就是這樣.
但我想同一樣說.
在剛才我開初的時候.
這封信帶去的時候.
其實保羅都同樣是帶哥羅西書和伊弗所書去.
哥羅西書裡面說了一些內容.
想和大家做一個平衡.
我們看看哥羅西書第一章的內容.
在螢幕上你會看到.
哥羅西書第一章第二十字.
既然藉著他在十字架上所流的血成就了和平.
便藉著他叫萬有.
無論是地上的天上的都叫自己和好.
你們從前與上帝隔絕.
因著惡行心裡與他為敵.
如今他藉著基督的肉身受死.
叫你們與自己和好.
都成了聖潔.
沒有瑕疵.
無可責備.
把你們引到自己面前.
保羅在哥羅西書第一章說的一件事就是.
我們以往是不可以的.
不可以的意思就是我們純粹因為罪的緣故.
我們背逆.
違背了上帝.
但耶穌基督十架寶血流出.
以致我們能夠透過這個中寶的角色.
成為一個橋樑.
可以與自己.
你不要再憎恨自己.
你可以與自己和好.

$^{521}$而可以在上帝面前有一個無可責備的開始.
同樣第二卷.
以弗所書第二章.
就是什麼?.
同樣是《加護書信》說到的.
第十六節.
「基督十字架滅了冤仇別藉著這十字架使兩下歸為一體.
與上帝和好了.
並且來傳和平的福音給遠處的人.
也給近處的人.
因為我們兩下藉著他被一個聖靈所感得而進到父面前」.
這段紅色的字說的是什麼?.
是十字架滅了冤仇.
不會因為我們大家的嚴拘.
大家所謂怨恨因為十字架.
就可以兩下.
即是成為一個橋樑.
大家能夠拼合一起.
而這個拼合是因為聖靈在當中做事.
保羅強調一件事.
不是靠仁義.
靠Lobbying.
大家兩邊做一個中介.
做一個協談.
那個不是沒用.
最重要的不是靠這件事.
最重要的是.
你知不知道自己要為自己負責.
十字架就掩飾了這件事.
第二件事就是.
你們兩個走在一起.
是有十字架在當中做一條和.
同樣聖靈在當中也有做事.
大家能不能感受在聖靈當中怨言虛偽.
聖靈的督責告訴你.
當初你為什麼說這句話.
這句話實在是傷害了對方.
而這句話你說的時候.
你沒有顧及對方的感受.
而你做的時候.

$^{561}$都沒有了解後果的嚴重度.
以致你說了你收不回.
之後你更加沒有跟進.
這完全都是雙方決裂.
又或者第一支箭未處理.
又再射第二支箭的問題.
聖靈就是不斷用這些方式去提醒.
彼此.
以致大家再見面的時候.
你願不願意做這個復和.
上帝既然已經跟我們開始復和.
而我們又願不願意當中去多走一步呢.
所以去到以弗所知第二章第十一節.
他說什麼.
所以你們當紀念.
你們從前按肉體示愛幫人.
也稱為未受割禮的.
這明願是那些憑人手在肉身上.
稱為受割禮的人所起的.
那時你們與基督無關.
在以色列民以外.
在所應許的諸約上是局外人.
並且活在世上沒有子望沒有上帝.
你們從前遠離上帝的人.
如今卻在基督裡靠著他的血.
已經得親近了.
這裡說的是什麼呢.
其實是耶穌基督做了個中介.
你們以前是沒有關係.
因為兩個都是不同的人.
但因為你們同領基督之後.
你跟基督有關.
你已經是同一個血緣.
你同一個血緣的時候.
你們就彼此親近.
你還記得剛才讀經的時候.
讀到一個點就是.
他已經不是一個奴隸.
他是你的弟兄.
大家都是同一個弟兄的時候.

$^{601}$大家就開始要走近一點.
要學習彼此接納.
剛才那封信就是這樣說的.
在這裡我們停一停.
剛才說的《玉加木書信》有三卷.
除了《肥立比》以外.
其實《哥羅西書》和《爾忽所書》中所說的訊息.
跟今天肥利門所說的內容是很相近的.
意思是什麼呢.
就是《哥羅西書》和《爾忽所書》.
我們都稱為是教會藍本的書卷.
是說一個教會的墓會.
那個墓養的神學立論.
或者那個教義上.
怎麼看基督的死.
基督的代屬.
基督的能力.
基督所謂.
不是所謂.
基督要我們去明白.
上帝要做中保的角色是什麼.
肥利門書正正就告訴我們.
那件事不是留在頭腦.
實際上生活我們都要經歷的.
實際上生活當中.
我們都學習到.
大家本身都是來自兩個個體.
但因著基督的緣故.
我們能夠彼此相近.
是基督的血覆蓋了我們的錯.
覆蓋了大家的錯.
跟著大家學習彼此相近.
是基督的血做了我們的中介.
以致我們能夠聯合.
這個就是肥利門正正經歷著這件事.
保羅不是不知道舊約的教導.
要給他一個更新的機會.
不要再回到舊的地方.
不是.
他的反應就是.

$^{641}$我們不再需要受前約的捆綁.
我們要面對新的關係建立.
我們要面對就是.
要看回自己的過去.
面對自己的將來.
我知道有些人舊的傷痛之大.
是很難一下子去平息的.
無論是身體上的傷痛.
心靈上的傷痛.
關係上的破裂.
這些很難有一個比較.
但是我想說的是.
那個已經是事實.
那個都是曾經.
聽過很多弟兄姊妹的分享.
很坦白說.
你見過輔導.
你見過教睦.
你見過你的閨密.
你的老闆.
說過很多次.
說完之後.
你是得到舒緩.
但是我都很希望.
要說多少次.
但是每次說的時候.
其實你都是不斷不開心.
正如我上一個月的訊息就是.
那你想怎樣重整呢?.
你想怎樣不再受核制呢?.
很多時候.
有弟兄姊妹找我聊天的時候.
或者是在.
他重述他離開教會的難處的時候.
我都說.
實際上我沒有什麼可以幫到你.
你願意告訴我.
我一定會聽.
但是聽完之後.
我都會和你輕輕分析一下.

$^{681}$或者有些情況.
我可能會建議一下.
但是我說最主要就是.
你怎樣再看這些關係.
你怎樣再看這些傷痛.
如果不是.
你下一次再說.
其實對你來說.
幫助在哪裡呢?.
你說當然有幫助.
起碼我知道.
有人會明白.
我說是呀.
我說不止你.
很多人都會有這樣的經歷.
換了同路人.
但是都要再一次面對自己困難的時候.
我就說.
其實主耶穌都知道.
但是主耶穌.
其實最想和你說.
其實我覆蓋了.
我覆蓋了.
就讓我覆蓋了的東西.
你就開了新一頁.
如果不是,你想一想.
上帝不是白祖嗎?.
還是你不相信上帝會覆蓋?.
還是你一直都不想覆蓋?.
我相信.
好像我上個月說的訊息就是.
其實靜下來的時候.
你想處理的.
你經常自己一個哭.
你不會開心的.
你和別人說的時候.
也不一定有出路的.
所以保羅.
既然他知道肥利門是一個什麼人.
是一個有愛心的人.

$^{721}$他知道他說.
其實我可以賜老賣老的.
但是他沒有這樣做.
他知道肥利門是一個什麼人.
他也知道亞歷西姆是一個什麼人.
而亞歷西姆現在已經信了主.
他改變了.
他能夠成為保羅的同工.
他就說.
不如我和你回去吧.
好不好?.
你也要面對這件事情的.
亞歷西姆就回去了.
因為他經歷了聖靈的更新.
親愛的兄弟姐妹.
你的傷痛仍然是.
傷疤仍然有.
但是傷疤成為你的核心.
還是什麼呢?.
其實要自己再認清這個關係.
和那個過去.
在經文裡面要提一個東西.
就是.
我們再看看下一頁.
就是第十一節經文.
他從前與你無益.
但如今與你我都有益處.
我現在打發他親自回里那去.
他是我心上的人.
就正如剛才我說.
他們雙方面.
保羅是雙方面都知道.
他現在就叫他們回去.
從前與你無益的意思是什麼呢?.
從前與你無益就是.
其實亞歷西姆的意思就是有益處.
就是有益處.
名字的意思就是有益處.
從前與你無益處.
現在有了.

$^{761}$可能以前的關係是不OK的.
但現在他已經成為一個新做的人.
他已經信了主.
他已經是一個好的福音的協助者.
既然你在家裡做教會的話.
他以前對你無益.
他現在對你有益了.
所以我就打發他回去幫你.
這個就是保羅想跟費利門說得清楚.
這個再下去就是.
以福所書第二章裡面的經文.
因他使我們和睦.
將兩下合而為一.
拆毀中間隔斷的牆.
而且以自己身體廢掉冤仇.
就是那記載法律法上的規條.
並為要將兩下直接自己做成一個新人.
如此便成就了和睦.
你看到嗎?.
在以福所書裡面說的情況就是費利門的情況.
因他使我們和睦.
其實原文的意思就是.
他是我們的和睦.
意思是基督就是我們的和睦.
如果沒有基督.
如果亞里斯多德沒有母信主.
故事並不是這樣的.
如果上帝沒有做事.
冤仇仍然存在.
所以正正是因為上帝做了事.
他就拆了中間的牆.
用自己身體廢掉冤仇.
就是記載法律上的規條.
從來有些東西是綁住了.
但因為基督的緣故就拆解了.
所以成就了和睦.
就是再一次面對舊的關係.
我去到這裡仍然很希望你明白.
我無意淡化或是小看你過去受的傷痛.
但你已經受夠了.

$^{801}$其實上帝已經掩飾了.
你說上帝沒有.
上帝沒有掩飾.
我不知道.
但上帝說.
你煩惱苦擔重擔.
你就去他那裡吧.
我不相信沒有事是上帝不能承擔的.
但重點是你有多少事交給上帝呢?.
還是我一定要生氣.
我一定要看死他.
還是怎樣?.
但你自己的心靈狀態.
精神狀態都不好.
又何苦呢?.
其實最不開心的.
應該是主耶穌.
其實我掩飾你.
你又不接受我掩飾.
你又不明白.
其實你可以有新的開始.
為何你不嘗試呢?.
我相信聖靈仍然會在當中提醒我們.
剛才說到.
總會聽到很多弟兄姊妹的難處和故事.
我都很坦白說.
有些事我沒有實際幫助.
但我通常和弟兄姊妹說.
我不是說你的故事很普通.
但我說你經歷的教會問題.
你自身的問題.
其實Flow Church有很多弟兄姊妹.
都有相近的經歷.
我的意思是.
你其實不乏同路人.
但他們為何會來Flow Church.
或者為何Flow Church會開展.
因為他們相信上帝已經掩飾了.
而他們可以選擇開啟新的頁面.
為何不兩者都一樣呢?.

$^{841}$就是接受上帝的掩飾.
然後開啟新的頁面.
正視舊有的關係.
我相信保羅提醒亞歷西姆.
面對舊有做的不是.
同樣就是面對舊有的關係.
我和你一起回去吧.
這樣就回去了.
所以你會看到.
其實保羅不是隨便的.
他雖然在那25節的經文中.
其實有些事是很認真的.
你會看到下面.
下面的時刻就是第15節.
保羅和他說.
「他暫時離開你」.
「或者叫你永遠得著他」.
保羅說的一個很簡短的濃縮版.
就是他曾經走過的.
就是他曾經走過.
但他現在回來了.
這個我們不知道為甚麼.
但上帝就給我們機會再次證明.
他就回來了.
我不知道你怎樣看一些.
舊有和好如初的關係.
和好如初.
我經常覺得這個字很迷.
是否真的沒有嚴規呢.
你再看回他.
你會有15,16猜想.
他會否像以前那樣對我呢.
你跟他說話是否還可以沒有隔膜呢.
有時我都不容易的.
我都被人騙過.
有時有人說潘Sir你很好.
可能很多人對你很好.
但我被人騙的時候.
難道我又要你吃苦閒罰嗎.
我不是比較.

$^{881}$但在過程當中.
有些事情你可能都會有些猜想.
有些猜測.
保羅說一個訊息.
或者我們在經文中聯想一個情況.
就是當再見到那個人的時候.
你可能有很多想法.
但正正那個想法對你來說.
一個提示就是.
不要再走舊有對大家的方法.
這個很重要.
就是吵架之後.
我相信總有吵過架.
吵架之後你會發覺有些說話.
是不可以在這個人面前再說的.
那你就會被記.
這個就是尊重.
這個就是我上一個月想認識的體面.
不夠體面的月花加級體面就是.
你懂得避重就輕.
最重要是你懂得欣賞人.
這個很重要.
從前他有益的時候.
現在就從事無益.
現在有益了.
他暫時離開你.
叫你永遠得著他.
其實都是說一個情況.
就是在那個時候學不到.
現在你學到了.
你之後就可以和他在一起.
整件事不容易去了解的.
但總會有些人經一事長一智.
學習怎樣和人相處.
在經文下去的時候.
你會看到.
就會是第十六節.
他現在不是再是奴僕了.
他高過奴僕是親愛的弟兄.
就剛才讀過那段經文.

$^{921}$因為他同樣是蒙基督的血所灑.
但大家都是因為基督的血.
能夠拆了那個圍牆.
大家能夠成為以合為一.
這個是上帝給他的恩典.
所以第十七節.
你若以我為同伴就收納他.
如同收納我一樣.
收納這個字又再出現了.
在保羅的信息裡面.
我不知道大家有沒有記得.
在保羅書上這個字.
其實出現了兩個段落.
一個段落就是.
在我講閃避球.
即是十月第二次講的信息裡面.
就是在講Nabano這個字.
就是接納.
接納什麼意思呢?.
就是你要學習.
懂得明白對方的難處.
以至你懂得接納.
保羅在羅馬書裡面.
十四章在講.
有人說吃肉可以.
有人說吃肉不行.
但吃肉可以吃肉不行.
對我們來說.
那個是自由的選取.
是信心的問題.
信心從來都有先後信心.
有人有信心先.
有人有信心後.
有人信心大.
有人信心小.
有人信心重.
有人信心輕.
從來都是一個.
即是場景上的比較.
沒有絕對的標準.

$^{961}$你就接納他有先後.
有能力的差異.
有大小的問題.
即是你接納.
這個字Nabano.
保羅這次仍然是用.
仍然是在講一個主動.
主動Pros Nabano.
好像羅馬書第十五章.
你們接納我.
如同基督接納我們一樣.
是主人蘇主動去接納我們每一個.
是主人蘇主動掩蓋我們所有的人.
保羅這次是在講.
如果你接納我是你的同工.
你就再主動接納他.
就是這樣.
保羅是要求或者是懇求.
讓你們明白到.
你就主動接納他.
他現在回來了.
你就主動接納他.
我看這裡的時候.
感受到保羅想起自己.
保羅在《少林傳》開初的時候.
他是一個很差的人.
沒有人會這樣和他一起.
見到他就走.
怕他被逼迫.
鬥死他.
當巴拿巴要抱著保羅.
去引戰使徒的時候.
人們說巴拿巴傻的.
但巴拿巴說不是的.
我領受了上帝.
但沒有人會理解.
但巴拿巴就抱著保羅.
所以今天保羅抱著阿彌西姆.
其實就是我曾經被人接納.
我也很希望這個回轉.

$^{1001}$同樣歸在基督名下的弟兄.
他也被你接納.
你走一步吧.
你願不願意再正視這個關係.
這是雙向的.
我開頭也說.
保羅是認識菲利門.
保羅也是和阿彌西姆同工.
兩個都認識.
保羅就做這個醜人.
做一個中介去做協調這件事.
我過去很坦白說.
有一些教務同工問我.
志剛,我有些會友去了你那裡.
我就說,哪一個是你的會友.
我也不認識.
我說真的.
我認識不少人.
但我真的不認識哪一個是.
因為事實上.
我們很坦白說.
每一個在Flow Church參與的弟兄姊妹.
我們都沒有問.
你的戶籍在哪裡.
你的組織在哪裡.
你從哪間跳過來.
我們從來都沒有這樣問.
我不知道現在留言打什麼.
我一整晚回去看都害怕.
但我仍然很坦白說.
其實.
有些是我知道的.
我也跟他說.
最大的問題不是在哪裡.
我們跟Flow Church的弟兄姊妹說.
這個不存在大家互通消息.
是互通消息的.
這個不存在台底交易.
不是這個意思.
但是.

$^{1041}$我也跟他說.
那個同工問我的原因.
不是想把你帶回去.
不是.
也知道不能把你帶回去.
但我認真跟大家說.
這個案件很敏感.
我也不會說名字.
但我想告訴你.
那個同工問我的原因.
他其實不是想你回去.
他是告訴我.
我見過他.
他現在比以前開心了很多.
我覺得.
真的很不同.
那種傷痛其實是相輔相成的.
是兩個都有不開心的.
我希望你不要覺得.
以片蓋全.
覺得每個人都是這樣.
我不是這個意思.
我又不無意說情.
告訴你一定是.
令到他的情緒好像很激昂.
我不是這個意思.
我只是告訴你.
那個是真實的.
有些教務同工是知道.
自己的同學有些限制.
有些原因 有些前因後果.
是很複雜.
以致你離開了.
但其實他知道你活得好.
其實他也很安心.
那又何苦你仍然帶著.
我說弟兄姊妹.
又何苦你仍然帶著舊的傷痛.
而一直都不復原呢?.
為什麼不做兩者呢?.

$^{1081}$就是你接受了.
那個已經是曾經.
當你再面對那種關係的時候.
上帝已經覆蓋了.
你開始了新的頁面.
保羅很想我們明白到.
聖靈一直在我們心裡動工.
開我們的眼睛.
看到有新的轉向.
同樣告訴我們.
耶穌基督已經做了那個中介.
我不會迷信一個信息.
就能夠改變太多.
但我希望你想想.
其實保羅寫哥羅西書.
以法索書給教會.
讓他們明白到基督的重要性.
肥利門書就告訴他們.
實務上我們怎樣運用這件事.
以致我們在關係上可以建立.
最後保羅很認真的.
保羅認真講什麼呢?.
第十八節.
他不是想著轉過去就算的.
如果他真的虧負了你呢.
他欠你什麼呢?.
我包底.
他真的知道亞歷西武有些事.
可能做得不對的.
所以他會跟他說.
他會負責任.
這個承擔的角色.
他一定會償還.
他會證明這是他的親筆寫.
所以最重要的就是.
那件事不是回到他手上.
說一句話就算了.
不是這個意思.
重點是.
保羅講得很清楚.

$^{1121}$仍然要對這件事處理好.
不要讓這件事重蹈覆轍.
也不要讓這件事.
笑笑口就過去.
從來都沒有看輕這件事.
一定要面對.
第二十節就是.
上帝會令我們有Rejoice喜樂.
這是耶穌基督的工作.
就是他的功能.
到最後有一節.
我寫給你.
他深信你會信服.
其實保羅很清楚.
大家雙方的為人.
大家處事的情況.
而他不是.
他不是就這樣.
做一個中介就算了.
你看到二十二節.
他說我會去的.
他會去家房的.
他會跟進那件事.
其實之後大家的關係是怎樣.
正正如在Folk Church.
我們所有的牧者可以說.
我們其實在小組.
花很大的氣勁.
去處理舊教會.
或者一些破碎了.
在其他層面的關係.
不要讓那些事情再核製你.
以致你得不到.
基督給我們豐盛的生命.
所以我們很多牧者.
都會做跟進的工作.
做個人的關心.
如果不準備就慢慢來.
但就不可以置之不理.
有時候大家太過迷信時間.

$^{1161}$以為放下時間慢慢去調探.
就會好.
但其實在時間當中不做事.
其實那件事是不會好的.
最後跟大家說再見.
在再見一種合作關係.
大家可能不會再合作.
真的可能不會再合作.
不會再合作.
就算了.
但如果再相見的時候.
我們仍然有一個弟兄姊妹的關係.
因為仍然是主內的弟兄姊妹.
我不是奢求大家要重修舊好.
一如以往那樣.
繼續可以無縫銜接.
做到事.
有時候這些太過天真.
那就停在合作關係上.
但在相見的時候仍然是主內的弟兄姊妹.
因為耶穌基督給我們一個聯合.
保持適當的社交距離.
大家在當中停在那裡.
但不要再被冤仇去克制你.
成為你每次見到的時候.
你又睡不著.
你又不開心.
你又怨自己.
都說不去.
又去.
所以整件事.
我希望大家明白.
緋天門面對亞里西姆是怎樣呢?.
我不知道.
但如果.
給大家一份功課試試做.
如果你回信給保羅.
你會怎樣呢?.
亞里西姆帶著這封信去見緋天門.
緋天門看完這封信的時候.

$^{1201}$亞里西姆在那裡.
接著保羅不在那裡.
如果你回信給保羅.
你會寫些什麼呢?.
有什麼選項呢?.
大家一起說說吧.
你也想想吧.
選項一.
保羅我不行.
你收回這個我不要.
選項二.
保羅我明白的.
但我接受不了.
你給我一點時間吧.
還是選項三.
保羅我真的明白到.
原來.
呼音真的會改變一個人的.
當初他出走.
有他的原因.
但呼音改變了他.
他能夠成為你的同工.
今天你帶他回來成為我的同工.
我感受到.
呼音本是上帝的大能.
是要救一切相信的.
我感受到上帝給我的能力.
我就接收他.
在呼音的工作.
繼續下去.
多謝你啊保羅.
三個選項.
你仍然可以選擇.
但你會選擇哪一個呢?.
願意在重整這個訊息裡.
你想想.
你選擇哪個選項來重整.
2021年已經去到二月了.
六分一了.
不要再等了.

$^{1241}$上帝與我們同在.
願主祝福你.
\newpage



\section{}
\label{sec:GmLFDCSkId4}
\textbf{【這是最好的時代:給香港基督徒的神學八課】第1講:亂世中才明白甚麼是「基督徒」|20210522 [GmLFDCSkId4]}
\newline
\newline
連結: \href{https://youtube.com/watch?v=GmLFDCSkId4}{\texttt{ https://youtube.com/watch?v=GmLFDCSkId4}} ~~~~ 語音日期: 2021-05-22 
\newline
\newline
\hyperref[sec:hx9eq2tkbx4]{\small{< < < PREV SERMON < < <}}
~
\hyperref[sec:index_chronic]{\small{[返順時目]}}
~
\hyperref[sec:index_scriptual]{\small{[返順卷目]}}
~
\hyperref[sec:k_0V9RXlAGE]{\small{> > > NEXT SERMON > > >}}
\newline
\newline
$^{1}$(這段影片由於拍攝時鏡頭不足,所以不得不重複).
(香港,香港,你永遠是塵夢鄉).
(香港,香港,你那些笑容).
(山頂看小島水裡頭,處處換上新裝).
(看看那海鷗飛過自由港).
(海邊看小島充滿裝,處處搖眼風光).
(這個市區的吸引沒法擋).
(日日聲香港,香港,太有望童年夢想).
(香港,香港,叫我不以為望).
(香港,我身心的抱憾,這裡讓我增長).
(有我喜歡的親友共陽光).
(路上人在跑過馬桿,看見我欣賞).
(這裡有許多好處沒法講).
(日日聲香港,香港,你永遠是塵夢鄉).
(香港,香港,你那些笑容).
(香港,太有望童年夢想).
(香港,香港,叫我不以為望).
(香港,香港,你永遠是塵夢鄉).
每個年代都有每個年代的神學.
作為土生土長的香港人.
我們似乎正在經歷一個最差的年代.
不過往往在最差的年代.
我們才能夠經歷福音信仰的最好.
就讓我們一起從聖經裡學習.
如何做這個年代裡的香港基督徒.
這是最好的時代給香港基督徒的神學百科.
(神學百科).
粉絲們晚安.
歡迎大家參加我們Flo Church的神學講座.
這是最好的時代給香港基督徒的神學百科.
這是一個什麼呢?.
我稱之為一個有道具有服裝的主學.
或者叫做一個錄了的神學講座.
這個也是我們Flo Church裡的一個新嘗試.
我們希望能夠有一個神學教導的百科講座.
但我們想認真一點.
大家有現場的頂字幕可以聽.
也有網上的頂字幕可以聽.
所以我覺得這個很突出.
平時都是用Zoom來講.

$^{41}$但這個有道具可以清除.
這個就是我們新嘗試.
有些頂字幕問.
這個是什麼呢?.
Flo Church現在變了.
Flo Church現在是否要做主學,門訓呢?.
是否我們要和其他教會一樣.
要做門徒訓練呢?.
我會說是Yes and No.
不是那些.
不是你一般認識的門訓.
不是你以前裝備過的門訓.
就像今天的題目.
我們不是要講門徒.
而是講基督徒.
所以最多叫做機訓.
我們要思考的是什麼呢?.
其實香港這十年有很多門訓.
不知道大家有沒有參加過門訓?.
有沒有試過被門訓過?.
都有吧?.
門訓不知為何成為了教會很流行的東西.
門徒成為了一個更加高層次的東西.
我聽過一些人說.
教會需要的不是參崇拜基督徒.
而是我們要更多門徒.
我們不是做基督徒.
而是做門徒.
門徒更加厲害.
我覺得不是的.
門徒只是其中一個名字.
所以今天我想講的是一個字.
叫做基督徒.
反而我要強調.
我們對於門徒的執著.
很明顯是來自於香港教會這幾年.
很多時候大家一起.
傳聞大練訓.
大家一起去訓練門徒.
好像一開始甚麼都不是.

$^{81}$就變成門徒才是.
我覺得這是教會以前很事工化的方式.
所以Folk Church不是做這些東西.
Folk Church不是純粹做那種門訓.
門訓是一種課程.
一種教會增長的方式.
一個事工.
門徒當然是很重要.
門徒是跟隨耶穌的人.
門徒是一群背十字架的人.
去學習耶穌基督的人.
不過我覺得門徒不可以純粹參加一個訓練.
或者聽完一些東西就算了.
反而我覺得更加要思考的是.
門徒本身是一個很特別的意思.
你知道潘Sir以前是教基教的.
基督教教育.
他說過一個很有意思的說話.
他說門訓是甚麼呢?.
門訓是一個身份價值的教育.
去告訴我們我們是甚麼.
我們是甚麼.
我們應該做甚麼.
這是一個很重要的學生.
不是一個套餐.
不是一個教會增長的訓練.
所以如果是這樣的話.
Folk Church不是做這種門訓.
不過如果作為一種身份價值的教導.
我覺得Folk Church的弟兄姊妹.
是很需要有這樣的認識和實踐.
我覺得如果有門訓的話.
門訓其實是一種永遠帶著時限的東西.
門訓是帶著一個時期.
在某個年代某個時代裡面.
去面對如何去做門徒.
即是說門徒是會過期的.
可能你以前學過的甚麼三寶.
或者是甚麼價值觀重整之甚麼.
那些是很重要.

$^{121}$但那些東西其實應該是按著不同的年代.
我們重新來去再學習和裝備過.
因為我們跟隨主耶穌.
跟隨主耶穌的時候.
我們就在不同的年代裡面.
在這個年代裡面.
即是說在一個後光光法的年代裡面.
我們學習如何去做一個門徒.
因此門徒不是一個永久有效的東西.
如果是的話.
它就不是一個很新的東西.
門徒是帶著時限.
舉個例子.
1934年潘福華.
他面對納粹德國的時候.
他說納粹主義將會改變.
即是帶來德國教會中建.
他就開始想到一個想法.
不如我用登山補訓來做一個門訓.
後來這個想法成為了甚麼書.
就是跟隨基督這本書.
所以每個年代每個時空的基督徒.
我們都重新來去思考.
我們應該如何來做門徒.
如何來做基督徒.
而不是純粹一套套裝.
不是一套裝出來.
大量生產一套套裝.
這是我們Full Church.
為何會想做一套門訓.
而是我們想面對國安法的時候.
我們如何能夠在這個年代.
重新來學習思考做基督徒.
所以基督徒的身份.
這個成為我們今天的主題.
但事實上初期教會都面對著這樣的主題.
初期的一班猶太人.
跟隨耶穌的人.
當耶穌升天之後.
這班人其實本身是甚麼人.

$^{161}$是一班猶太教徒.
一班猶太人.
所以他們出現了一個很特別的身份危機.
他們要分別出.
他們不是一班以前過往.
聖殿時期的猶太教.
而是一班一個嶄新的信仰.
一個群體.
一班相信基督徒的群體.
所以你問 Who am I?.
即是我是誰.
所以這個身份價值.
正成為初期教會裡面.
很重要的一個題目.
如果你們看《釋經》的話.
你會知道.
《釋經》裡面有很多不同的字眼.
關乎於一個門徒的自稱.
不知道大家是否認識.
第一個是甚麼.
第一個叫Adelphoi.
即是弟兄.
當然有姊妹.
這個是我們經常見到.
聖經裡面出現的一個門徒.
自我身份的一個名稱.
第二個是甚麼.
就是Pistero.
一個信徒.
強調一班相信的人.
一班相信耶穌基督.
復活.
是上帝神的兒子的一班人.
第三個就是Hagios.
即是一個聖徒.
強調一班分別為聖的人.
他們是不同於世界裡面的一班聖徒.
第四是Ecclesi.
大家都認識.
就是教會.

$^{201}$他們用這個字眼.
來稱呼自己的群體.
我們被招聚在一起去結集的一班人.
第五個當然是大家認識的.
Mathetis.
就是門徒.
這班基督徒經常出現.
是最基本.
最常見的一班人.
稱呼自己的字眼.
我們是學生.
我們是門徒.
最後我稱之為甚麼呢.
The Way.
即是道.
我稱之為道友.
即是這班人是學習真道的人.
相信真道的人.
無道友.
尋道友.
甚麼道友都好.
而今天要講的.
就是第七個字.
這個字在聖經裡面.
出現了三次.
一個很特別的字眼.
就是叫做基督徒.
Sorry.
不是.
不好意思.
這是別人這樣稱呼我們.
改了.
不會這樣稱呼.
耶L.
不是這個.
下一個.
是叫做基督徒.
基督徒不是一種.
今天所稱呼的佛教徒.
道教徒.

$^{241}$伊斯蘭教徒.
基督徒是一個Biblical的字眼.
即是這個字眼出現在你去辨別.
不同宗教.
信甚麼教.
基督教.
基督徒.
不是的.
我懷疑大家都誤會了意思.
基督徒的字眼在聖經裡面.
出現了三次.
雖然不多.
但是一個很特別的神學意義.
不是泛指任何相信基督教的人.
就是基督徒.
甚麼是基督徒呢.
沒事.
我想讀一讀.
大家一起讀.
現場觀眾一起讀.
就是叫做.
Christianos.
我一直讀成這樣.
Christianos.
基督徒.
我們是Christianos.
所以這個字在聖經裡面.
賦予我們第七個身份價值.
除了信徒,聖徒,門徒之外.
是一個很特別的身份價值.
今天我們會討論這個題目.
這個字是怎樣來的呢.
這個字其實是一個拉丁文化了的字.
即是它是將一個拉丁文grammar.
這樣去辨別了.
假設我們說有姜濤.
我不是搞爛笑話.
姜濤不是基督徒的朋友.
姜濤的追隨者叫甚麼名字.
叫姜粉.

$^{281}$(姜糖).
Sorry.
不好意思.
我不是.
叫姜糖.
我還未說不是.
姜糖.
如果按以前就叫姜濤濤.
因為是一群追隨.
屬於永死忠心不變的跟隨姜濤的人.
叫姜糖.
以前也是這樣.
這個字是一個辯字.
所以將這個字.
第一個字是希律.
根據馬克方的字.
叫做同希律一黨的人.
這個字其實是來自於一個.
不是那麼長的字眼.
就是Helodys.
Helodynus.
所以從希律這個字.
加上一個ion.
就變成了希律黨的人.
其實是一個比較短的字眼.
就是姜濤和姜糖.
所以用這個邏輯的話.
Christos.
就是基督這個字.
一群一樣.
你當他們是姜濤一樣.
即是誓死永不變的愛.
誓死跟從.
屬他的.
即是跟隨他的.
這樣一群人.
所以就叫基督徒.
Christianos.
所以基督徒的字就是這樣來的.
一個這樣的字眼.

$^{321}$這個字其實本身帶著一種政治意味.
因為只有凱撒和希律會這樣稱呼.
所以當人.
特別是一些不信基督的人才會這樣說.
即是一群外面的人才會叫他們做.
一個這樣的基督徒字眼.
有一個軍事意味.
一個政治意味.
即是希律,凱撒和自稱維尼塞亞的基督.
下一個故事我們今天就開始看.
我們看三段經文.
在聖地牢那三段有關出現過基督徒的字眼.
打段不熟一點.
就是小人傳第十一章第25,26節.
巴拿巴又往大掃羅.
掃出了就帶他到安提阿去.
他們足有一年的功夫和教會一同聚集.
教訓了許多人門徒稱為基督徒.
是從安提阿起手的.
這個就是第一段用到基督徒這個字的一段經文.
很明顯你見到這個群體是被外面的人來稱呼.
這個從來都不是他們自稱.
他們自稱做大英姐妹.
自稱做信徒或者門徒.
外面的人用政治術語.
用這個稱呼來稱呼這群人.
並且見到聖經這樣說.
門徒就稱之為基督徒.
所以我們問究竟門徒厲害一點還是基督徒厲害一點.
大家今天覺得門徒當然厲害一點.
門徒跟隨耶穌.
其實不是.
反而門徒是一個比較普通的字眼.
我覺得門徒基本上是.
最基本的東西.
不是什麼高峰層面.
不過當門徒變成基督徒的時候.
當門徒這群人升到基督徒的層次的時候.
其實一個很特別的東西就在當中不同了.
出現了第三者.

$^{361}$即是當一群人去見到他們的時候.
而他們覺得這群人這麼似他.
並且這麼願意在他面前見到他的時候.
這群安提安的信徒就被稱之為基督徒.
所以你見到這是一個不是基督徒的自稱.
而是別人永遠出現了第三身.
一個他者的自稱.
一個稱呼.
這群基督徒被別人掛著基督的名字.
從別人口中稱之為基督徒.
第二段就是《小人傳》第26章第24至29節.
保羅在審判中的一個自辯.
當保羅在自辯中的時候.
他講了一大輪很多的見證.
他自己在大馬士革中被耶穌的光弄得盲了.
被呼召並且受苦.
當保羅講了一大輪之後.
他開始說:保羅,你癲狂了嗎?.
你的學問太大反對你癲狂嗎?.
保羅說:非斯都大人,我不是癲狂.
我說的乃是真實明白話.
王也曉得這些事.
所以我向王放膽直言.
也深信這些事沒有一件向王隱藏的.
因都不是在背地裡做的.
接著我們聽到一個回應.
阿基伯王說:你信先知嗎?.
我知道你是信的.
當保羅這樣問他的時候.
阿基伯就跟保羅說:.
你想小微一勸.
你不要勸我了.
你省一點吧.
你想我做基督徒嗎?.
所以這是一句很戲語的說話.
都是來自一個不信主的人的說話.
基督徒在初期的時候.
永遠來自一個不信耶穌的人的口.
出現在保羅的見證裡.
當保羅不斷強調自己受苦.

$^{401}$在他面前見證基督耶穌的時候.
外面的人就被他的行動所回應.
就說:你想我做基督徒嗎?.
記住這個字不是我們今天說的基督徒.
而是帶著政黨的意味.
一種軍事的意味.
一種當時文化的背景.
你是否想我成為一個基督黨的黨員?.
所以當時保羅所做的.
其實是向一個人見證耶穌基督.
並且在一個政權底下.
讓其他人都見到耶穌基督.
在羅馬政權的一個官的底下.
見到耶穌基督這個王.
是這樣的一個事情.
所以基督徒永遠都是帶著一個見證.
並且是從他人口中的見證的結局.
到了第三段.
這段是彼得前書的經文.
第四章第十二至十六節.
到了後期的時間.
當外面的人不斷去稱呼這班人為基督黨員.
基督徒的時候.
教會裡的人都開始用到這個字眼.
不過有一個特別的角度.
他說:親愛的弟兄!.
有火煽煉到你,燃到你們.
不要以為奇怪,不要以為要歡喜.
因為你們是與基督一同受苦的.
你們是一起受苦的.
所以你們在榮耀顯現的時候.
可以歡喜快樂.
你們若為基督的名受辱罵便是有福的.
因為上帝榮耀的靈常駐在你們身上.
你們中間卻不可有人因為殺人,偷竊,作惡.
或管閒事而受苦.
若為了基督徒受苦.
卻不要羞恥.
都要因者明歸天上的上帝.
當基督徒第一次用到基督徒這個字去自稱的時候.

$^{441}$是帶著一種受苦的概念.
來稱呼基督徒這個字眼.
而受苦永遠都是來自於對外公共的見證.
在政權底下,這班人被稱為基督徒.
甚至乎在政權的高官底下.
指控他們為基督徒.
然後他們慢慢接受了這個字.
在受苦的情況下.
稱自己為基督徒.
所以基督徒在這三段經文中.
告訴我們一個很重要的意義.
與門徒不同.
門徒是一班學者.
一個學習耶穌的人物,一個學生.
但當你稱之為基督徒的時候.
你會有一個更重要的意味.
你在社會裡成為一個基督的見證.
並且願意隨時受到很多苦難.
我再看後面的文獻.
這是在聖經以後的教會文獻.
有一個文獻叫《十二兆遺訓》.
這也是早期的字眼.
我們搜尋一下所有有關基督徒的字眼.
第十二章有關接待客人的教導.
他用了兩個很接近的字眼.
情繪說,如果有人有客人的話.
你就去接待他.
大概是這個意思.
如果他是基督徒的話.
你就不要單單叫他去偷難.
因為基督徒是不會偷難的.
應該是更好的回報.
而不是純粹的自我開放.
但如果他是自我開放的話.
他就不是基督徒了.
他就變成了假托基督詐取私利的人.
中文看不明白是甚麼意思.
我們看下一頁.
下一頁就說更加重要的字眼.
一個就叫Christiano.

$^{481}$就是基督徒.
一群接待基督徒.
另一個就叫Christian Poros.
很相似的兩個字眼.
前者就叫基督徒.
或者基督黨羽.
或者基督的跟隨者.
一個就叫靠基督謀生的人.
這個字眼出現了很多次.
就是靠著基督耶穌的名字來謀生.
即是甚麼呢?.
我來去黐飲黐食.
或者用基督的名字.
來為自己利益的一群人.
所以稱之為基督玩家.
更加好一點的字眼.
很會玩的一群人.
即是掛著基督的名字.
來為自己的著數而生存.
我不知道你是否一個這樣的人.
如果基督徒不是單單是耶穌那麼簡單.
而是一種見證.
一種為主願意受苦的意思的時候.
我們掛著基督的名字的時候.
究竟我們是為了自己.
好會玩的樣子.
還是真真正正的去做一個基督徒呢?.
所以我想說的是.
基督徒這三個字真的不是開玩笑的.
不是純粹一個最普通的字眼.
反而是一個最不普通的字眼.
因為這個字正正是意味著甚麼呢?.
意味著我們基督徒本身的一個公共性.
當你被耶穌基督呼召為基督徒的時候.
這個名稱帶著一個很重要的社會公共性.
我們看看其他的文獻.
另一個就是約瑟夫.
可能大家都聽過.
一個猶太歷史學家.
他在公元95年96年的時間.

$^{521}$他就去到這樣的地方.
這群人全部都不信耶穌.
所以基督徒這字眼在教外文獻中比性更多.
這群基督徒.
基督黨語.
星之維.
這群所謂一直以基督為名的人.
這群不信的人會這樣理解.
政權就會這樣理解這群人.
一群以基督為名的人.
至今仍未消失.
當教會.
當這群門徒.
這群聖徒.
仍然在社會裡.
存在被不信的人看到的時候.
基督徒這字眼又再次出現.
當然.
更可怕的是.
基督徒這字眼成為了.
尼祿迫害教會裡的字眼.
當時羅馬的一個官員.
一個歷史學家.
用了Christianos這字眼.
來形容這群人.
是六四年羅馬大火的滋長者.
因為這群基督徒的緣故.
令羅馬起火.
因此從此之後.
基督徒就被迫害.
所以基督徒被迫害.
基督徒這字眼.
本來就是一個這樣的context.
這不是一個普遍的字眼.
而是一個政權.
來迫害基督徒的字眼.
基督徒在社會裡.
成為了公共層面意義的時候.
一個這樣的字眼.
亦成為了一個受苦的字眼.

$^{561}$後來整個羅馬帝國.
迫害基督徒三百多年.
其實大家有沒有想過.
基督徒是因為犯了什麼法?.
基督徒需要什麼來被捕?.
當然不是國安法.
是什麼?.
一個很簡單的法律.
叫做基督徒之名.
Norman Christian.
單單是因為你自稱為基督徒的時候.
單單這個事情.
就足以令到你犯法.
我就能夠來行政的權力.
來逮捕你.
迫害你.
甚至殺死你.
如果你再次強調.
如果你忘記了.
一個很普通的基督徒的意思的時候.
基督徒正正永遠都是承擔著這些的.
因為基督徒之名.
因為你在我面前作見證.
當時的社會就要逮捕這些基督徒.
因此我想說的就是.
基督徒永遠都是帶著一個.
很重要的社會性.
我們Full Church這套所謂的門訓.
為什麼會用這套課來開始呢?.
因為你看看我們很多時候市面上的材料.
什麼什麼價值觀.
或者是真金法.
基督徒什麼的.
全部都是用個人出發開始的.
你有沒有想過.
你今天如果上天堂的話.
你行不行?.
這是第一課第一句問題.
基督徒對於價值觀.
都是很自我開始的中心.

$^{601}$這是可以的.
我不是反對.
但是我想開始我們所謂的課程.
我們不如重新來思考.
我們不用我們個人的得救來開始.
因為基督徒這三個字.
不是從我們得不得救來開始.
而是從別人的眼中.
別人的口中來開始.
我們的存在.
正正是受苦的見證來開始.
所以我們這套所謂的門訓的起始點.
或者是我們整個課程的起始點.
永遠我們說.
不如我們寫一套.
不以個人做開始.
而是重新去認識我們的身份.
我們是門徒.
我們是信徒.
我們是教會.
我們是聖徒.
不過一個很重要的身份.
就是我們都是基督徒.
當你背負著這個名字的時候.
基督徒這個名字並不是說玩的.
古佛華在他被監禁臨死前十五個月.
寫了一首詩叫做《Christen und Heiden》.
翻譯為《與非基督徒》的一首詩.
一首獄中寫出來的詩.
這首詩我覺得寫得很有意思.
大家留意它吧.
因為它本身德文也有些押韻.
我讀出來.
.
當人來到上帝面前.
在他的困苦裡面.
這群人是任何人.
所有的人.
祈求的是甚麼?.
祈求都是一些很基本的東西.

$^{641}$麵包,幸福,快樂.
尋求的是甚麼?.
都是一些很基本的.
很普通人所尋求的東西.
在疾病裡面.
在聚債裡面.
在面對死亡裡面的拯救.
《與非基督徒》其實都是一樣的.
我們沒有甚麼特別.
我們都是有些甚麼事.
都是祈禱,尋求幫助.
我們人生都是尋求麵包,快樂,幸福.
這樣的東西.
不過第二節就給了我們一個很特別的分岔.
它說.
.
第二節裡面說到人去到上帝面前.
不過那個他字你見到嗎?.
那個他字令了另一個他.
不是一個普通人的那個受苦.
而是哪個他?.
是耶穌基督的那個他.
這群人是去到耶穌基督的苦難裡面.
發現耶穌基督在十字架裡面的貧窮羞辱.
同樣都是沒有麵包.
沒有一個棲息之地.
親眼目睹耶穌基督被凌辱.
在軟弱的裡面,在罪惡的裡面,在死亡的裡面.
唯有基督徒同行在這個苦難裡面.
通常說真正能夠去分別出基督徒和非基督徒的.
不是甚麼得救的名單.
並不是你缺了自媒.
並不是你是不是將來會上天堂.
並不是你是不是預定被揀選的人.
是否你得了救恩,是否得了拯救?.
不是.
真正去define基督徒和非基督徒的是甚麼?.
是基督徒願意和基督耶穌一起去經歷這個苦難.
親眼看見這個苦難並願意同行在這個苦難裡面.
第三節.

$^{681}$.
第三節,一個很大的逆轉.
上帝來到人的面前.
同樣這個人並不是分基督徒或非基督徒.
上帝仍然用餅來填滿我們心靈的滿足.
記住我們是所有的人.
為世上所有的人釘在十字架上.
基督徒,非基督徒.
同樣上帝都是去憐憫,愛惜.
所以你看見潘福華臨死前的十五個月裡.
一個很大的感觸.
我做了這麼久基督徒.
其實真正的基督徒.
不是因為我上了慕道班裡的第一課.
或德國聖經裡的第一課.
因為我能上天堂.
而是因為我願意去參與基督的苦難.
因為我在世界裡願意作基督的見證.
所以我想說的就是.
基督徒並不是一個名字這麼簡單.
基督徒是一個動詞.
「弟子妹,你今天基督徒了嗎?」.
「你今天下班後有沒有基督徒到?」.
不是說你是否基督徒.
而是你有沒有基督徒到.
你有沒有基督徒的父母?.
你有沒有基督徒的同事?.
你有沒有在香港社會裡.
做一個基督徒.
去成為一個基督徒?.
這是我們全聖教裡.
很想每一個弟子妹.
重新去聆聽的問題.
我們有沒有在香港社會.
在這個年代裡去做基督徒?.
去做基督徒?.
上帝呼召我們.
主要不是神學院的呼召.
不是更加高層次的呼召.
耶穌基督從來都是一次次地呼召我們.

$^{721}$就在你十多年前.
幾年前信主的時候的呼召.
耶穌基督呼召我們去成為基督徒.
不是單單說給你一份救恩.
一開始就搭上了.
或者說一開始就帶著這種召命.
基督徒這三個字.
已經是一個很高層次的呼召.
全聖教弟子妹我們去學習的.
這個開始說很簡單的第一課.
它是一個非常嚴肅的第一課.
耶穌基督的呼召.
是從你剛剛信主那一刻的開始.
呼召你去成為他的見證.
我自己的母會的牧師很有趣.
他有一個很特別的見證.
他說他在一個陪靈會裡信耶穌.
然後他在一個報道會裡納至去侍奉上帝.
剛好相反.
一般來說都是在報道會裡信耶穌.
陪靈會來去決止.
來去呼召成為傳道人.
他剛好相反.
在報道會裡被蒙召.
在陪靈會裡信主.
我也是這樣想的.
回想十幾年前的報道會的呼召.
那個決止.
其實同樣是耶穌基督呼召我們.
來去做他的見證.
所以這是我們從第一天接受的呼召.
我今天想和大家進行一個很特別的活動.
就是和大家重新來決止.
待會我們吃飯.
我講一句,你講一句.
在褲子裡說自己的名字就可以了.
我們一起來.
雖然是小玩,但不是玩的.
認真的.
這個決止的文是不同的.

$^{761}$不是純粹個人得教的決止.
因為決止是什麼呢?.
決止就是決定立志.
所以不是純粹得救.
而決止是一個立定志向.
我們作為Fold Church的弟兄姊妹.
願意在社會裡.
在這個年代裡去見證主耶穌.
開始祈禱.
我講一句,你講一句就可以了.
親愛的主耶穌.
我,陳偉安.
承認我有罪.
的罪釘在十字架上.
死而復活.
呼召我一生去跟隨你.
成為一個基督徒.
一生去見證你.
奉主耶穌基督的名求.
阿們.
恭喜你,你成為基督徒了.
喂喂喂.
你見到圖案了嗎?.
這麼久還沒到,講了九個字.
通常決而至都會有禮物的.
難得你這麼開心.
先喝一杯吧.
剛才講的都有點硬.
因為我們Fold Church是認真的.
今天很多頂姐妹再次重新思考.
以往覺得門徒是高階一點.
以往教會的門徒訓練是高階課程.
不是每個人都能讀到.
反而今天有一個新的觀念.
我覺得不是只做門徒.
門徒是基本的.
門徒是學校耶穌,跟隨耶穌.
但去到見證,去到社會.
是我們更加需要思考的東西.
在座頂姐妹有什麼反題可以問?.

$^{801}$或者有什麼不太清楚的字.
或者相關的內容都可以討論一下.
可以隨便發問.
主要是學來的.
可以問問題的,不用擔心.
不單單是一個訊息,或者是一個專題.
不清楚的,或者相關的內容大家可以討論一下.
有問題嗎?.
好,那邊.
給他咪高峰.
中間那個,謝謝.
(聽到之後想起《Imitation of Christ》這本書).
(或者這方面).
(我想看看老師有沒有補充).
你說什麼?一開始聽不到.
(《Imitation of Christ》).
(好像有一本這樣的書籍).
(所以《孝法基督》是一本中世紀的靈修書).
(我稱之為神秘主義的書).
(其實是一種強調我們和個人關係).
(因為中世紀那種靈修是很強調).
(我們怎樣能夠和耶穌有團契和契合).
(所以《Imitation of Christ》是一本).
(強調透過我們在生命中實踐).
(當然是一種有靈性和行為的實踐).
(來學習耶穌基督).
(所以這是很重要的).
(我們說跟隨耶穌).
(你覺得跟隨耶穌比較厲害還是學學耶穌比較厲害?).
(有沒有說哪一個比較厲害?).
(其實是兩個不同的說法).
(跟隨耶穌,還是去模仿).
(就像你所說的,他做什麼,你做什麼).
(你分心分心去模仿他).
(耶穌是一個模仿的字眼).
(這是我們靈性上的操練傳統).
(今天我們嘗試說的不是純粹靈命).
(雖然我們知道有一部會說靈命).
(但我們說我們整個神學思考是從第三者出發).
(不是從修身齊家自我平天下).

$^{841}$(首先從個人層面做好).
(靈性好一點,其他OK).
(慢慢做下去,不是這樣).
(而我們整個基督徒的行動).
(其實可以從這點開始出發).
(我覺得我們需要去打破).
(或者用一種新的思考去認識自己).
(不是先搞定自己,永遠都搞不定自己).
(永遠都做到不是最好).
(但我們全靈學).
(我們希望能夠從這樣的角度).
(開始我們的第一課).
(所以這個課程).
(可以說是給初信人).
(或者給我們從商俗人).
(很多年輕人都可以).
(最重要是我們從社會的角度).
(去思考我們的身份).
(其他呢?後面).
(不如我不在教會工作).
(我正在做自己的工作).
(我就不算是靠著基督去謀生的人).
(我何時才知道).
(牧者是靠著基督去謀生的人).
這個問題真是懂得問.
Bruce你回答吧.
我當然是靠基督去謀生.
我當然不斷宣揚基督和去見證基督.
我剛才覺得這個詞.
真的可以是一個專題.
繼續去討論一下.
因為其實John可以補充.
其實那班人.
你見到前面說的.
其實有些事情是不務正業的.
不應該做的事情他做了.
或者應該做的事情他不做.
即是叫做是狐假虎威.
我想牧者就不會是狐假虎威.
起碼Folk Church的牧者不是.

$^{881}$我想這個都是很認真去處理.
其實心態和行動的相符是重要的.
是的,這個我覺得是大家都值得一起去想.
前者基督徒是為主受苦.
但基督賺錢的人是用基督.
大家都carry著基督的名字.
其實大家都是見證.
因為基督徒都是被人看上基督.
那些賺錢的人都是被人看上基督.
但他們的做法是為了自己的利益去著想.
所以他可以是傳道人.
可以是現在問的傳道人.
或者已經是在做傳道人.
或者信徒.
我們今天用Bear這個名字的時候.
其實我們整件事是.
究竟你是想自己.
還是真的去為他作見證呢?.
所以我覺得這個不是純粹傳道人尊重的.
雖然傳道人很容易會犯這個錯.
因為他究竟是賺錢的.
但重點是我們是否去尋求益處.
靠著基督耶穌來尋求自己益處.
所以我覺得大家不妨在小組裡面討論一下這個terms.
自己是否一個很懂得玩基督玩家.
做了基督徒之後.
是否純粹用這個名字來為自己好處去做人呢?.
你好,我叫Yuki.
我剛才聽到的message就是.
基督徒是為基督受苦的人.
基督徒在仕途行傳的年代都會有一些受苦的經歷.
甚至乎是.
總之你自認自己是基督徒你就犯法.
剛剛信基督的人.
通常得到的message就會是.
信耶穌你就得永生.
你會有救恩.
你怎樣去解讀.
原來基督徒就已經是會受苦.
但其實又會說.

$^{921}$但又有一個救恩.
又會是恩聚德赦.
你怎樣解讀這兩個中間的落差?.
落差其實是同時.
我想說同時間一次過.
都是的.
因為聖經裡面有七個不同的稱呼.
我們是聖徒,是信徒,是門徒.
信徒和基督徒.
我想說這七個都是同時重要的.
我不是說其他不重要.
大公說當你從第一天信耶穌之後.
恭喜你你已經得救了.
這當然是真的.
我當然不會說是假的.
不過其實同時間這個呼召已經是開始了.
已經是叫我們去做基督徒.
這個見證.
基督徒當然不是拿苦來生.
我想說這個補充不是受苦的基督徒.
而是他願意為耶穌在社會裡面作見證.
從而肯定會受苦.
並且會承受這些迫害.
這個就是基督徒的定義.
如果問我我會說.
這些東西是同時間出現的.
我們是天父上帝的兒女.
這是我們同神關係裡面.
我們有一課會說這個話題.
我想說同時我們也領受了一個使命.
我們比較少去強調這個使命.
我覺得去到第十堂或是Extra才說.
不如你宣教 傳道人 傳福音.
但其實這個是從第一天開始.
我們的身份價值.
通常說我們的身份價值.
這是我們是什麼東西的問題.
多於說你做了這些就得救了.
或者已經成為教會裡面的頂姐妹.
才去想是不是見證耶穌.

$^{961}$我們嘗試去想.
不如我們從這一點開始去想.
我們Fold Church.
從這一點開始去想我們Fold Church是什麼呢.
我在網上也呼籲了弟兄姐妹問一些問題.
有幾個問題我覺得和見證有幾個關係.
其中一個也有很多人附和.
他們想知道多一點關於基督徒的公共性.
在現在的政治環境下.
怎樣可以實踐到 應用出來.
因為可能在現在這個時代.
很多事情都很無力.
怎樣可以實踐到 見證出來.
有沒有什麼可以再多說一點.
這是一個很好的問題.
我覺得這個也是大家一起去想.
當然我在神樂院是說得多一點.
因為是Offline.
當然在這個時勢裡面.
我們所做到的公共性有多少呢.
但我覺得如果看回聖經裡面的定義.
這個公共性其實不是一定要去到.
我們要去到公共領域層面 政策層面.
這些我們沒有了.
我們立法會也沒有了.
但反而我們能夠讓人去真的見到.
從聖經裡面所說.
在安提阿教會裡面.
安提阿裡面城市的人能夠見到我們.
甚至乎保羅在那個官面前能夠見到我們.
所以在我們的生活裡面.
我們生活裡面能夠讓人見到耶穌基督.
當然所說的不是純粹親戚父母那麼簡單.
而是我們在社會裡面仍然可以去說.
起碼在可以說的時候可以說.
所以我們封出一個崇拜的Online.
即使我們的宣講仍然是在說一些公共領域的東西.
所以這個我覺得我們FourShare都希望能夠做到.
而在底子母層面裡面.
起碼在我們的工作裡面.

$^{1001}$在我們的生活裡面能夠做到這件事.
所以重點是有第三者出現.
讓人見到我們在社會裡面仍然存在.
這個其實我覺得有很多想像.
起碼對我作為一個港道的人.
傳道人來說.
起碼我們FourChurch在一個網絡裡面.
仍然在回應一些社會的事.
這個我們能夠做到其中一個很微小的事.
但底子母所做的.
如果整個群體一起去做的話.
整件事是可以很寬闊.
所以雖然我覺得是世界艱難.
但我都覺得我們讓人見到的.
是讓人探望的.
讓人見到的.
是讓人認識到自己.
所以大家一起去做.
是一件能夠做到的事.
(MC) 這邊有問題.
(Feronica) 你好,我是Feronica.
剛剛看到一張slide很大感觸.
就是vocation呼召.
我看到vocation放假.
基督徒的價值.
就是時刻警醒.
天天背負自己十價.
有時回看現在社會發生的事.
看到有些基督徒.
這樣都可以做到基督徒.
其實怎樣面對自己.
真的沒有了一個vocation的基督徒.
時時刻刻都做到vocation.
我覺得是一個很大的難題.
(MC) 回答歡sir.
其實我剛才想用公共性去回應vocation.
因為其實在我們的信仰層面.
剛才John都說到.
基督徒是一個行動.
不是一個term.

$^{1041}$基督徒是行動的時候.
行動就是回應公共性一個很重要的表達.
譬如我們.
小組或者在我們信息裡.
我們會用我們的崇拜時間.
為一些社會有需要的群體.
或者一些意見去發聲去禱告.
做回我們基督徒.
我們禱告最大的權柄.
另外就是一些鼓勵.
可能用你的財幹.
那些基本上都會說.
但是可能更加具體公共性就是.
現在很多時候金錢的重新分配.
我們都會想我們的錢會放在哪裡.
支援什麼群體.
或者是什麼經濟圈.
或者是什麼對於你來講.
你可以重新去認定一些什麼價值.
這個都是我們公共性的參與.
所以你說那個vocation.
呼召未必真的要在一個崗位上要盡忠.
譬如好像.
我仍然是說.
我們fortune上面沒有內部事工.
一定要你involve的.
但我們常常強調就是.
我們的教會的出口是向著社會的.
所以我們仍然鼓勵.
在一個星期你聽完的訊息.
你在小組上面的支援.
或者是分享.
最重要就是我們回到社區裡面.
可以做到我們從教導上得到的那種價值觀.
的行動參與是什麼.
那個就是fortune上再強調的公共性.
所以如果真的好像今天聽到John講的那個訊息就是.
基督徒不是一個terms.
而是真的將那個行動走出去的話.
基本上你不會是一個vocation放假.

$^{1081}$而你每組的參與都是一個vocation.
這邊也有問題.
Hello.
我想問剛才說的那個教徒的terms.
是由第三者去灌給我們.
好像過往門徒那方面.
好像有很多指標告訴我們.
怎樣做到門徒.
我們怎樣知道自己是在實踐.
做一個基督徒呢.
如果這件事是別人賦予我們的話.
起碼有第三者.
我覺得是真的.
起碼你的生命裡面有一個.
能夠可以讓他看到耶穌基督的人.
這個好像有問.
究竟我自己在一個星期裡.
究竟我做人的時候有沒有一個第三者.
一個第三身的人.
可以看著我.
能夠看到耶穌基督呢.
我想這個很簡單的思維.
已經幫到我們去想.
我們怎樣能夠去做基督徒.
其實有時我們都不覺.
都忙著做一份工作.
或者很多的事.
都想不到這些事.
但當你一刻想到.
其實我是成為那個人眼中的人.
我在想怎樣能夠去.
將基督耶穌的探望.
那份救恩告訴他.
說的不是全副音這麼簡單.
而是能夠你所做的事.
在這個社會裡面.
社區也好 社會也好.
讓人看到.
這個意識是重要的.
永遠帶著一種有第三身的角度.

$^{1121}$來看著自己的行徑.
這個意識.
我們來到一個很特別的地方.
如果你說天父的兒女.
是一個看著天父上帝的靈性.
而我們的生活裡面.
是永遠帶著一個.
有一個人在我心裡展現的人.
你會有這個心上人.
然後你就可以慢慢地.
帶著基督徒的身份.
我可以說我小時候信耶穌之後.
但我其實是信主的見證.
當然我不是說家庭.
但我信耶穌之後.
兩三年內就帶著全家人信耶穌.
那時候我是中五.
我信耶穌之後.
我第一次帶我兩妹回教會.
那時候是我的生日.
他們說今年不收信.
就送給我吧.
不如今天回教會算了.
我說好啊.
然後就帶我回教會.
然後就信耶穌了.
我爸媽也是類似這樣帶著.
三兄妹一起上學.
上學後不如就做好一點.
回家就做好一點兒女.
會看到很不一樣.
然後我爸就信了耶穌.
所以說不是全方位技巧的問題.
而是我們在社會裡面.
我們不可以全方位地說.
我見證耶穌之後也好.
寫到神面也好.
說闊一點也好.
我們是帶著這樣的眼界.
去做人的時候.

$^{1161}$我覺得正正我們就是.
發揮基督徒應該要做的事情.
無論是社會也好.
或者是見證基督真理也好.
公義發聲也好.
我們是帶著這樣的思維去行動.
我回答一下.
剛才John說.
大家可能覺得有些笑話.
就好像他帶家人信主.
很簡單.
其實最主要是你看到改變.
別人會不會察覺你的改變.
第三者有沒有察覺你的改變.
是很重要的.
或者第三者覺得你行動的存在.
是否感受到基督的影響.
近來我身邊也有不少旁聽師.
其中有一位前同工是牧師.
他經常去旁聽.
為出庭的人祈禱.
他表明他是牧師身份.
你介不介意.
不知不覺有些人在當中感受到.
在他最不安的時候.
基督的平安臨到.
在過程當中他表明.
希望你結果如何.
也願意在當中.
上帝的平安.
給你一刻的觸動和平安.
我想這就是第三者.
去看我們在做甚麼.
而我們真的要主動去迎向第三者.
這正正就是.
仍然是那句.
基督徒就是一個行動.
你說不去做旁聽.
我們旁邊的弟兄姊妹.
也有擺放一些東西在身上.

$^{1201}$例如擺放口罩,飯券,現金券.
看到有需要的人就跟他們共享.
我們在紅土區也有社會參與.
有些弟兄姊妹也聽過.
你們基督徒真是很好.
這件事正如我常常說.
做了就是做了.
零和一的分別就是.
一做了就讓人感受到基督的行動.
有沒有網上的弟兄姊妹.
有問題嗎?.
哦.
好,我們八課當中的第一講就到這裡了.
我們看那邊.
網上的弟兄姊妹.
密切留意我們下一個月.
下一個月是什麼時候?.
是六月的第三個星期.
下個星期就沒有了.
一個月之後就沒有了.
好像崇拜一樣.
下個月見.
我們一個月一講.
先跟網上的弟兄姊妹說聲再見.
再見.
《香港》 歌手:陳靜茹 作詞:陳靜茹 作曲:陳靜茹.
香港我心真的國鄉.
這裡讓我生長.
有我喜歡的親友共陽光.
路上人在跑過他過.
幹勁令我欣賞.
這裡有許多好處沒發覺.
食一聲香港香港.
你永遠是曾夢鄉.
香港香港你哪識了哪我.
山頂看小島水雷塘.
處處換上新裝.
看看那海鷗飛過自由港.
海邊看小島處萬丈.
處處搖眼新光.

$^{1241}$這個市區的吸引沒發覺.
食一聲香港香港.
再有我童年夢想.
香港香港叫我不以為望.
香港我心真的國鄉.
這裡讓我生長.
有我喜歡的親友共陽光.
路上人在跑過他過.
幹勁令我欣賞.
這裡有許多好處沒發覺.
食一聲香港香港.
你永遠是曾夢鄉.
香港香港你哪識了哪我.
香港香港再有我童年夢想.
香港香港叫我不以為望.
香港香港你永遠是曾夢鄉.
香港香港你哪識了哪我.
Zither Harp.
\newpage



\section{}
\label{sec:OTk7WEa_w50}
\textbf{【這是最好的時代:給香港基督徒的神學八課】第2講:究竟好消息有多好?|20210619 [OTk7WEa-w50]}
\newline
\newline
連結: \href{https://youtube.com/watch?v=OTk7WEa-w50}{\texttt{ https://youtube.com/watch?v=OTk7WEa-w50}} ~~~~ 語音日期: 2021-06-19 
\newline
\newline
\hyperref[sec:adSMNfhKn24]{\small{< < < PREV SERMON < < <}}
~
\hyperref[sec:index_chronic]{\small{[返順時目]}}
~
\hyperref[sec:index_scriptual]{\small{[返順卷目]}}
~
\hyperref[sec:rjndxjSXpt8]{\small{> > > NEXT SERMON > > >}}
\newline
\newline
$^{1}$我只想知道.
你到底是什麼意思.
我只想知道.
你到底是什麼意思.
我只想知道.
你到底是什麼意思.
我只想知道.
你到底是什麼意思.
我只想知道.
你到底是什麼意思.
我只想知道.
你到底是什麼意思.
我只想知道.
你到底是什麼意思.
我只想知道.
你到底是什麼意思.
我只想知道.
你到底是什麼意思.
我只想知道.
你到底是什麼意思.
我只想知道.
你到底是什麼意思.
我只想知道.
你到底是什麼意思.
我只想知道.
你到底是什麼意思.
我只想知道.
你到底是什麼意思.
我只想知道.
你到底是什麼意思.
我只想知道.
你到底是什麼意思.
我只想知道.
你到底是什麼意思.
我只想知道.
你到底是什麼意思.
我只想知道.
你到底是什麼意思.
我只想知道.
你到底是什麼意思.

$^{41}$我只想知道.
你到底是什麼意思.
我只想知道.
你到底是什麼意思.
我只想知道.
你到底是什麼意思.
我只想知道.
你到底是什麼意思.
我只想知道.
你到底是什麼意思.
我只想知道.
你到底是什麼意思.
我只想知道.
你到底是什麼意思.
我只想知道.
你到底是什麼意思.
我只想知道.
你到底是什麼意思.
我只想知道.
你到底是什麼意思.
我只想知道.
你到底是什麼意思.
我只想知道.
你到底是什麼意思.
我只想知道.
你到底是什麼意思.
我只想知道.
你到底是什麼意思.
我只想知道.
你到底是什麼意思.
我只想知道.
你到底是什麼意思.
我只想知道.
你到底是什麼意思.
我只想知道.
你到底是什麼意思.
我只想知道.
你到底是什麼意思.
我只想知道.
你到底是什麼意思.

$^{81}$我只想知道.
你到底是什麼意思.
我只想知道.
你到底是什麼意思.
我只想知道.
你到底是什麼意思.
我只想知道.
你到底是什麼意思.
我只想知道.
你到底是什麼意思.
我只想知道.
你到底是什麼意思.
我只想知道.
你到底是什麼意思.
我只想知道.
你到底是什麼意思.
我只想知道.
你到底是什麼意思.
我只想知道.
你到底是什麼意思.
我只想知道.
你到底是什麼意思.
我只想知道.
你到底是什麼意思.
我只想知道.
你到底是什麼意思.
我只想知道.
你到底是什麼意思.
我只想知道.
你到底是什麼意思.
我只想知道.
你到底是什麼意思.
我只想知道.
你到底是什麼意思.
我只想知道.
你到底是什麼意思.
我只想知道.
你到底是什麼意思.
我只想知道.
你到底是什麼意思.

$^{121}$我只想知道.
你到底是什麼意思.
我只想知道.
你到底是什麼意思.
我只想知道.
你到底是什麼意思.
我只想知道.
你到底是什麼意思.
我只想知道.
你到底是什麼意思.
我只想知道.
你到底是什麼意思.
我只想知道.
你到底是什麼意思.
我只想知道.
你到底是什麼意思.
我只想知道.
你到底是什麼意思.
我只想知道.
你到底是什麼意思.
我只想知道.
你到底是什麼意思.
我只想知道.
你到底是什麼意思.
我只想知道.
你到底是什麼意思.
我只想知道.
你到底是什麼意思.
我只想知道.
你到底是什麼意思.
我只想知道.
你到底是什麼意思.
我只想知道.
你到底是什麼意思.
我只想知道.
你到底是什麼意思.
我只想知道.
你到底是什麼意思.
我只想知道.
你到底是什麼意思.

$^{161}$我只想知道.
你到底是什麼意思.
我只想知道.
你到底是什麼意思.
我只想知道.
你到底是什麼意思.
我只想知道.
你到底是什麼意思.
我只想知道.
你到底是什麼意思.
我只想知道.
你到底是什麼意思.
我只想知道.
你到底是什麼意思.
我只想知道.
你到底是什麼意思.
我只想知道.
你到底是什麼意思.
我只想知道.
你到底是什麼意思.
我只想知道.
你到底是什麼意思.
我只想知道.
你到底是什麼意思.
我只想知道.
你到底是什麼意思.
我只想知道.
你到底是什麼意思.
我只想知道.
你到底是什麼意思.
我只想知道.
你到底是什麼意思.
我只想知道.
你到底是什麼意思.
我只想知道.
你到底是什麼意思.
我只想知道.
你到底是什麼意思.
我只想知道.
你到底是什麼意思.

$^{201}$我只想知道.
你到底是什麼意思.
我只想知道.
你到底是什麼意思.
我只想知道.
你到底是什麼意思.
我只想知道.
你到底是什麼意思.
我只想知道.
你到底是什麼意思.
我只想知道.
你到底是什麼意思.
我只想知道.
你到底是什麼意思.
我只想知道.
你到底是什麼意思.
我只想知道.
你到底是什麼意思.
我只想知道.
你到底是什麼意思.
我只想知道.
你到底是什麼意思.
我只想知道.
你到底是什麼意思.
我只想知道.
你到底是什麼意思.
我只想知道.
你到底是什麼意思.
我只想知道.
你到底是什麼意思.
我只想知道.
你到底是什麼意思.
我只想知道.
你到底是什麼意思.
我只想知道.
你到底是什麼意思.
我只想知道.
你到底是什麼意思.
我只想知道.
你到底是什麼意思.

$^{241}$我只想知道.
你到底是什麼意思.
我只想知道.
你到底是什麼意思.
我只想知道.
你到底是什麼意思.
我只想知道.
你到底是什麼意思.
我只想知道.
你到底是什麼意思.
我只想知道.
你到底是什麼意思.
我只想知道.
你到底是什麼意思.
我只想知道.
你到底是什麼意思.
我只想知道.
你到底是什麼意思.
我只想知道.
你到底是什麼意思.
我只想知道.
你到底是什麼意思.
我只想知道.
你到底是什麼意思.
我只想知道.
你到底是什麼意思.
我只想知道.
你到底是什麼意思.
我只想知道.
你到底是什麼意思.
我只想知道.
你到底是什麼意思.
我只想知道.
你到底是什麼意思.
我只想知道.
你到底是什麼意思.
我只想知道.
你到底是什麼意思.
我只想知道.
你到底是什麼意思.

$^{281}$我只想知道.
你到底是什麼意思.
我只想知道.
你到底是什麼意思.
我只想知道.
你到底是什麼意思.
我只想知道.
你到底是什麼意思.
我只想知道.
你到底是什麼意思.
我只想知道.
你到底是什麼意思.
我只想知道.
你到底是什麼意思.
我只想知道.
你到底是什麼意思.
我只想知道.
你到底是什麼意思.
我只想知道.
你到底是什麼意思.
我只想知道.
你到底是什麼意思.
我只想知道.
你到底是什麼意思.
我只想知道.
你到底是什麼意思.
我只想知道.
你到底是什麼意思.
我只想知道.
你到底是什麼意思.
我只想知道.
你到底是什麼意思.
我只想知道.
你到底是什麼意思.
我只想知道.
你到底是什麼意思.
我只想知道.
你到底是什麼意思.
我只想知道.
你到底是什麼意思.

$^{321}$我只想知道.
你到底是什麼意思.
我只想知道.
你到底是什麼意思.
我只想知道.
你到底是什麼意思.
我只想知道.
你到底是什麼意思.
我只想知道.
你到底是什麼意思.
我只想知道.
你到底是什麼意思.
我只想知道.
你到底是什麼意思.
我只想知道.
你到底是什麼意思.
我只想知道.
你到底是什麼意思.
我只想知道.
你到底是什麼意思.
我只想知道.
你到底是什麼意思.
我只想知道.
你到底是什麼意思.
我只想知道.
你到底是什麼意思.
我只想知道.
你到底是什麼意思.
我只想知道.
你到底是什麼意思.
我只想知道.
你到底是什麼意思.
我只想知道.
你到底是什麼意思.
我只想知道.
你到底是什麼意思.
我只想知道.
你到底是什麼意思.
我只想知道.
你到底是什麼意思.

$^{361}$我只想知道.
你到底是什麼意思.
我只想知道.
你到底是什麼意思.
我只想知道.
你到底是什麼意思.
我只想知道.
你到底是什麼意思.
我只想知道.
你到底是什麼意思.
我只想知道.
你到底是什麼意思.
我只想知道.
你到底是什麼意思.
我只想知道.
你到底是什麼意思.
我只想知道.
你到底是什麼意思.
我只想知道.
你到底是什麼意思.
我只想知道.
你到底是什麼意思.
我只想知道.
你到底是什麼意思.
我只想知道.
你到底是什麼意思.
我只想知道.
你到底是什麼意思.
我只想知道.
你到底是什麼意思.
我只想知道.
你到底是什麼意思.
我只想知道.
你到底是什麼意思.
我只想知道.
你到底是什麼意思.
我只想知道.
你到底是什麼意思.
我只想知道.
你到底是什麼意思.

$^{401}$我只想知道.
你到底是什麼意思.
我只想知道.
你到底是什麼意思.
我只想知道.
你到底是什麼意思.
我只想知道.
你到底是什麼意思.
我只想知道.
你到底是什麼意思.
我只想知道.
你到底是什麼意思.
我只想知道.
你到底是什麼意思.
我只想知道.
你到底是什麼意思.
我只想知道.
你到底是什麼意思.
我只想知道.
你到底是什麼意思.
我只想知道.
你到底是什麼意思.
我只想知道.
你到底是什麼意思.
我只想知道.
你到底是什麼意思.
我只想知道.
你到底是什麼意思.
我只想知道.
你到底是什麼意思.
我只想知道.
你到底是什麼意思.
我只想知道.
你到底是什麼意思.
我只想知道.
你到底是什麼意思.
我只想知道.
你到底是什麼意思.
我只想知道.
你到底是什麼意思.

$^{441}$我只想知道.
你到底是什麼意思.
我只想知道.
你到底是什麼意思.
我只想知道.
你到底是什麼意思.
我只想知道.
你到底是什麼意思.
我只想知道.
你到底是什麼意思.
我只想知道.
你到底是什麼意思.
我只想知道.
你到底是什麼意思.
我只想知道.
你到底是什麼意思.
我只想知道.
你到底是什麼意思.
我只想知道.
你到底是什麼意思.
我只想知道.
你到底是什麼意思.
我只想知道.
你到底是什麼意思.
我只想知道.
你到底是什麼意思.
我只想知道.
你到底是什麼意思.
我只想知道.
你到底是什麼意思.
我只想知道.
你到底是什麼意思.
我只想知道.
你到底是什麼意思.
我只想知道.
你到底是什麼意思.
我只想知道.
你到底是什麼意思.
我只想知道.
你到底是什麼意思.

$^{481}$我只想知道.
你到底是什麼意思.
我只想知道.
你到底是什麼意思.
我只想知道.
你到底是什麼意思.
我只想知道.
你到底是什麼意思.
我只想知道.
你到底是什麼意思.
我只想知道.
你到底是什麼意思.
我只想知道.
你到底是什麼意思.
我只想知道.
你到底是什麼意思.
我只想知道.
你到底是什麼意思.
我只想知道.
你到底是什麼意思.
我只想知道.
你到底是什麼意思.
我只想知道.
你到底是什麼意思.
我只想知道.
你到底是什麼意思.
我只想知道.
你到底是什麼意思.
我只想知道.
你到底是什麼意思.
我只想知道.
你到底是什麼意思.
我只想知道.
你到底是什麼意思.
我只想知道.
你到底是什麼意思.
我只想知道.
你到底是什麼意思.
我只想知道.
你到底是什麼意思.

$^{521}$我只想知道.
你到底是什麼意思.
我只想知道.
你到底是什麼意思.
我只想知道.
你到底是什麼意思.
我只想知道.
你到底是什麼意思.
我只想知道.
你到底是什麼意思.
我只想知道.
你到底是什麼意思.
我只想知道.
你到底是什麼意思.
我只想知道.
你到底是什麼意思.
我只想知道.
你到底是什麼意思.
我只想知道.
你到底是什麼意思.
我只想知道.
你到底是什麼意思.
我只想知道.
你到底是什麼意思.
我只想知道.
你到底是什麼意思.
我只想知道.
你到底是什麼意思.
我只想知道.
你到底是什麼意思.
我只想知道.
你到底是什麼意思.
我只想知道.
你到底是什麼意思.
我只想知道.
你到底是什麼意思.
我只想知道.
你到底是什麼意思.
我只想知道.
你到底是什麼意思.

$^{561}$我只想知道.
你到底是什麼意思.
我只想知道.
你到底是什麼意思.
我只想知道.
你到底是什麼意思.
我只想知道.
你到底是什麼意思.
我只想知道.
你到底是什麼意思.
我只想知道.
你到底是什麼意思.
我只想知道.
你到底是什麼意思.
我只想知道.
你到底是什麼意思.
我只想知道.
你到底是什麼意思.
我只想知道.
你到底是什麼意思.
我只想知道.
你到底是什麼意思.
我只想知道.
你到底是什麼意思.
我只想知道.
你到底是什麼意思.
我只想知道.
你到底是什麼意思.
我只想知道.
你到底是什麼意思.
我只想知道.
你到底是什麼意思.
我只想知道.
你到底是什麼意思.
我只想知道.
你到底是什麼意思.
我只想知道.
你到底是什麼意思.
我只想知道.
你到底是什麼意思.

$^{601}$我只想知道.
你到底是什麼意思.
我只想知道.
你到底是什麼意思.
我只想知道.
你到底是什麼意思.
我只想知道.
你到底是什麼意思.
我只想知道.
你到底是什麼意思.
我只想知道.
你到底是什麼意思.
我只想知道.
你到底是什麼意思.
我只想知道.
你到底是什麼意思.
我只想知道.
你到底是什麼意思.
我只想知道.
你到底是什麼意思.
我只想知道.
你到底是什麼意思.
我只想知道.
你到底是什麼意思.
我只想知道.
你到底是什麼意思.
我只想知道.
你到底是什麼意思.
我只想知道.
你到底是什麼意思.
我只想知道.
你到底是什麼意思.
我只想知道.
你到底是什麼意思.
我只想知道.
你到底是什麼意思.
我只想知道.
你到底是什麼意思.
我只想知道.
你到底是什麼意思.

$^{641}$我只想知道.
你到底是什麼意思.
我只想知道.
你到底是什麼意思.
我只想知道.
你到底是什麼意思.
我只想知道.
你到底是什麼意思.
我只想知道.
你到底是什麼意思.
我只想知道.
你到底是什麼意思.
我只想知道.
你到底是什麼意思.
我只想知道.
你到底是什麼意思.
我只想知道.
你到底是什麼意思.
我只想知道.
你到底是什麼意思.
我只想知道.
你到底是什麼意思.
我只想知道.
你到底是什麼意思.
我只想知道.
你到底是什麼意思.
我只想知道.
你到底是什麼意思.
我只想知道.
你到底是什麼意思.
我只想知道.
你到底是什麼意思.
我只想知道.
你到底是什麼意思.
我只想知道.
你到底是什麼意思.
我只想知道.
你到底是什麼意思.
我只想知道.
你到底是什麼意思.

$^{681}$我只想知道.
你到底是什麼意思.
我只想知道.
你到底是什麼意思.
我只想知道.
你到底是什麼意思.
我只想知道.
你到底是什麼意思.
我只想知道.
你到底是什麼意思.
我只想知道.
你到底是什麼意思.
我只想知道.
你到底是什麼意思.
我只想知道.
你到底是什麼意思.
我只想知道.
你到底是什麼意思.
我只想知道.
你到底是什麼意思.
我只想知道.
你到底是什麼意思.
我只想知道.
你到底是什麼意思.
我只想知道.
你到底是什麼意思.
我只想知道.
你到底是什麼意思.
我只想知道.
你到底是什麼意思.
我只想知道.
你到底是什麼意思.
我只想知道.
你到底是什麼意思.
我只想知道.
你到底是什麼意思.
我只想知道.
你到底是什麼意思.
我只想知道.
你到底是什麼意思.

$^{721}$我只想知道.
你到底是什麼意思.
我只想知道.
你到底是什麼意思.
我只想知道.
你到底是什麼意思.
我只想知道.
你到底是什麼意思.
我只想知道.
你到底是什麼意思.
我只想知道.
你到底是什麼意思.
我只想知道.
你到底是什麼意思.
我只想知道.
你到底是什麼意思.
我只想知道.
你到底是什麼意思.
我只想知道.
你到底是什麼意思.
我只想知道.
你到底是什麼意思.
我只想知道.
你到底是什麼意思.
我只想知道.
你到底是什麼意思.
我只想知道.
你到底是什麼意思.
我只想知道.
你到底是什麼意思.
我只想知道.
你到底是什麼意思.
我只想知道.
你到底是什麼意思.
我只想知道.
你到底是什麼意思.
我只想知道.
你到底是什麼意思.
我只想知道.
你到底是什麼意思.

$^{761}$我只想知道.
你到底是什麼意思.
我只想知道.
你到底是什麼意思.
我只想知道.
你到底是什麼意思.
我只想知道.
你到底是什麼意思.
我只想知道.
你到底是什麼意思.
我只想知道.
你到底是什麼意思.
我只想知道.
你到底是什麼意思.
我只想知道.
你到底是什麼意思.
我只想知道.
你到底是什麼意思.
我只想知道.
你到底是什麼意思.
我只想知道.
你到底是什麼意思.
我只想知道.
你到底是什麼意思.
我只想知道.
你到底是什麼意思.
我只想知道.
你到底是什麼意思.
我只想知道.
你到底是什麼意思.
我只想知道.
你到底是什麼意思.
我只想知道.
你到底是什麼意思.
我只想知道.
你到底是什麼意思.
我只想知道.
你到底是什麼意思.
我只想知道.
你到底是什麼意思.

$^{801}$我只想知道.
你到底是什麼意思.
我只想知道.
你到底是什麼意思.
我只想知道.
你到底是什麼意思.
我只想知道.
你到底是什麼意思.
我只想知道.
你到底是什麼意思.
我只想知道.
你到底是什麼意思.
我只想知道.
你到底是什麼意思.
我只想知道.
你到底是什麼意思.
我只想知道.
你到底是什麼意思.
我只想知道.
你到底是什麼意思.
我只想知道.
你到底是什麼意思.
我只想知道.
你到底是什麼意思.
我只想知道.
你到底是什麼意思.
我只想知道.
你到底是什麼意思.
我只想知道.
你到底是什麼意思.
我只想知道.
你到底是什麼意思.
我只想知道.
你到底是什麼意思.
我只想知道.
你到底是什麼意思.
我只想知道.
你到底是什麼意思.
我只想知道.
你到底是什麼意思.

$^{841}$我只想知道.
你到底是什麼意思.
我只想知道.
你到底是什麼意思.
我只想知道.
你到底是什麼意思.
我只想知道.
你到底是什麼意思.
我只想知道.
你到底是什麼意思.
我只想知道.
你到底是什麼意思.
我只想知道.
你到底是什麼意思.
我只想知道.
你到底是什麼意思.
我只想知道.
你到底是什麼意思.
我只想知道.
你到底是什麼意思.
我只想知道.
你到底是什麼意思.
我只想知道.
你到底是什麼意思.
我只想知道.
你到底是什麼意思.
我只想知道.
你到底是什麼意思.
我只想知道.
你到底是什麼意思.
我只想知道.
你到底是什麼意思.
我只想知道.
你到底是什麼意思.
我只想知道.
你到底是什麼意思.
我只想知道.
你到底是什麼意思.
我只想知道.
你到底是什麼意思.

$^{881}$我只想知道.
你到底是什麼意思.
我只想知道.
你到底是什麼意思.
我只想知道.
你到底是什麼意思.
我只想知道.
你到底是什麼意思.
我只想知道.
你到底是什麼意思.
我只想知道.
你到底是什麼意思.
我只想知道.
你到底是什麼意思.
我只想知道.
你到底是什麼意思.
我只想知道.
你到底是什麼意思.
我只想知道.
你到底是什麼意思.
我只想知道.
你到底是什麼意思.
我只想知道.
你到底是什麼意思.
我只想知道.
你到底是什麼意思.
我只想知道.
你到底是什麼意思.
我只想知道.
你到底是什麼意思.
我只想知道.
你到底是什麼意思.
我只想知道.
你到底是什麼意思.
我只想知道.
你到底是什麼意思.
我只想知道.
你到底是什麼意思.
我只想知道.
你到底是什麼意思.

$^{921}$我只想知道.
你到底是什麼意思.
我只想知道.
你到底是什麼意思.
我只想知道.
你到底是什麼意思.
我只想知道.
你到底是什麼意思.
我只想知道.
你到底是什麼意思.
我只想知道.
你到底是什麼意思.
我只想知道.
你到底是什麼意思.
我只想知道.
你到底是什麼意思.
我只想知道.
你到底是什麼意思.
我只想知道.
你到底是什麼意思.
我只想知道.
你到底是什麼意思.
我只想知道.
你到底是什麼意思.
我只想知道.
你到底是什麼意思.
我只想知道.
你到底是什麼意思.
我只想知道.
你到底是什麼意思.
我只想知道.
你到底是什麼意思.
我只想知道.
你到底是什麼意思.
我只想知道.
你到底是什麼意思.
我只想知道.
你到底是什麼意思.
我只想知道.
你到底是什麼意思.

$^{961}$我只想知道.
你到底是什麼意思.
我只想知道.
你到底是什麼意思.
我只想知道.
你到底是什麼意思.
我只想知道.
你到底是什麼意思.
我只想知道.
你到底是什麼意思.
我只想知道.
你到底是什麼意思.
我只想知道.
你到底是什麼意思.
我只想知道.
你到底是什麼意思.
我只想知道.
你到底是什麼意思.
我只想知道.
你到底是什麼意思.
我只想知道.
你到底是什麼意思.
我只想知道.
你到底是什麼意思.
我只想知道.
你到底是什麼意思.
我只想知道.
你到底是什麼意思.
我只想知道.
你到底是什麼意思.
我只想知道.
你到底是什麼意思.
我只想知道.
你到底是什麼意思.
我只想知道.
你到底是什麼意思.
我只想知道.
你到底是什麼意思.
我只想知道.
你到底是什麼意思.

$^{1001}$我只想知道.
你到底是什麼意思.
我只想知道.
你到底是什麼意思.
我只想知道.
你到底是什麼意思.
我只想知道.
你到底是什麼意思.
我只想知道.
你到底是什麼意思.
我只想知道.
你到底是什麼意思.
我只想知道.
你到底是什麼意思.
我只想知道.
你到底是什麼意思.
我只想知道.
你到底是什麼意思.
我只想知道.
你到底是什麼意思.
我只想知道.
你到底是什麼意思.
我只想知道.
你到底是什麼意思.
我只想知道.
你到底是什麼意思.
我只想知道.
你到底是什麼意思.
我只想知道.
你到底是什麼意思.
我只想知道.
你到底是什麼意思.
我只想知道.
你到底是什麼意思.
我只想知道.
你到底是什麼意思.
我只想知道.
你到底是什麼意思.
我只想知道.
你到底是什麼意思.

$^{1041}$我只想知道.
你到底是什麼意思.
我只想知道.
你到底是什麼意思.
我只想知道.
你到底是什麼意思.
我只想知道.
你到底是什麼意思.
我只想知道.
你到底是什麼意思.
我只想知道.
你到底是什麼意思.
我只想知道.
你到底是什麼意思.
我只想知道.
你到底是什麼意思.
我只想知道.
你到底是什麼意思.
我只想知道.
你到底是什麼意思.
我只想知道.
你到底是什麼意思.
我只想知道.
你到底是什麼意思.
我只想知道.
你到底是什麼意思.
我只想知道.
你到底是什麼意思.
我只想知道.
你到底是什麼意思.
我只想知道.
你到底是什麼意思.
我只想知道.
你到底是什麼意思.
我只想知道.
你到底是什麼意思.
我只想知道.
你到底是什麼意思.
我只想知道.
你到底是什麼意思.

$^{1081}$我只想知道.
你到底是什麼意思.
我只想知道.
你到底是什麼意思.
我只想知道.
你到底是什麼意思.
我只想知道.
你到底是什麼意思.
我只想知道.
你到底是什麼意思.
我只想知道.
你到底是什麼意思.
我只想知道.
你到底是什麼意思.
我只想知道.
你到底是什麼意思.
我只想知道.
你到底是什麼意思.
我只想知道.
你到底是什麼意思.
我只想知道.
你到底是什麼意思.
我只想知道.
你到底是什麼意思.
我只想知道.
你到底是什麼意思.
我只想知道.
你到底是什麼意思.
我只想知道.
你到底是什麼意思.
我只想知道.
你到底是什麼意思.
我只想知道.
你到底是什麼意思.
我只想知道.
你到底是什麼意思.
我只想知道.
你到底是什麼意思.
我只想知道.
你到底是什麼意思.

$^{1121}$我只想知道.
你到底是什麼意思.
我只想知道.
你到底是什麼意思.
我只想知道.
你到底是什麼意思.
我只想知道.
你到底是什麼意思.
我只想知道.
你到底是什麼意思.
我只想知道.
你到底是什麼意思.
我只想知道.
你到底是什麼意思.
我只想知道.
你到底是什麼意思.
我只想知道.
你到底是什麼意思.
我只想知道.
你到底是什麼意思.
我只想知道.
你到底是什麼意思.
我只想知道.
你到底是什麼意思.
我只想知道.
你到底是什麼意思.
我只想知道.
你到底是什麼意思.
我只想知道.
你到底是什麼意思.
我只想知道.
你到底是什麼意思.
我只想知道.
你到底是什麼意思.
我只想知道.
你到底是什麼意思.
我只想知道.
你到底是什麼意思.
我只想知道.
你到底是什麼意思.

$^{1161}$我只想知道.
你到底是什麼意思.
我只想知道.
你到底是什麼意思.
我只想知道.
你到底是什麼意思.
我只想知道.
你到底是什麼意思.
我只想知道.
你到底是什麼意思.
我只想知道.
你到底是什麼意思.
我只想知道.
你到底是什麼意思.
我只想知道.
你到底是什麼意思.
我只想知道.
你到底是什麼意思.
我只想知道.
你到底是什麼意思.
我只想知道.
你到底是什麼意思.
我只想知道.
你到底是什麼意思.
我只想知道.
你到底是什麼意思.
我只想知道.
你到底是什麼意思.
我只想知道.
你到底是什麼意思.
我只想知道.
你到底是什麼意思.
我只想知道.
你到底是什麼意思.
我只想知道.
你到底是什麼意思.
我只想知道.
你到底是什麼意思.
我只想知道.
你到底是什麼意思.

$^{1201}$我只想知道.
你到底是什麼意思.
我只想知道.
你到底是什麼意思.
我只想知道.
你到底是什麼意思.
我只想知道.
你到底是什麼意思.
我只想知道.
你到底是什麼意思.
我只想知道.
你到底是什麼意思.
我只想知道.
你到底是什麼意思.
我只想知道.
你到底是什麼意思.
我只想知道.
你到底是什麼意思.
這就是我認識的福音.
不過我想說.
福音這個詞.
其實是一個.
在舊裡面出現過一次.
麻煩下一張.
舊裡面.
唯一一次出現.
Evangelion.
在哪裡呢.
其實是有的.
看回希利文版本的舊約.
就是七十字本的話.
你會發現.
有一句經文.
就是當時大衛.
對當人殺死掃挪王的時候.
那些說話.
他說.
從前有人報告我說.
掃挪死了.
他自以為報好消息.

$^{1241}$我就拿著他.
將他殺在西格拉.
這就作為他報好消息的賞詞.
所以舊裡面.
唯一一次出現了.
Good news這個字.
Evangelion這個字就在這裡.
其實沒有關係.
當時大衛就是說.
Good news.
就是將掃挪.
看為一個Good news.
就這樣看.
大衛就殺了他.
所以基本上.
Evangelion這個字.
基本上.
不關一個什麼福音的事.
這個字本身是一個很普通的字眼.
一個就是Good news的意思.
一個很平常的字眼.
所以你會發現.
如果你這樣看的時候.
我們就會知道.
其實這個字.
麻煩夏業.
我們就更加好理解.
不要當作福音的意思.
因為我們太過慣性.
就像Dragon.
大家重新理解.
這個Gospel.
這個福音.
其實就是一個好消息.
所以我們重新來思考.
不如我們重新來學習這個字.
其實Evangelion這個字.
其實就是一個好消息的意思.
一個好的消息.
什麼是好消息呢?.

$^{1281}$就是簡單.
假設你今天突然患了癌症.
醫生突然說你沒事了.
這是一個Good news.
就是這麼簡單.
所以當2000年前教會.
嘗試去講一個Gospel的時候.
這個不是什麼獨特的字眼.
他和世人說.
我有一個Good news給你們.
有一個好的消息給你們.
就是這麼簡單.
所以我們一起看下去.
所以你會發現.
其實這個字.
早在教會用這個字之前.
我們問.
究竟什麼時候開始用這個字呢?.
早在教會用這個字之前.
其實當時候的羅馬帝國.
就一早用這個字.
大家看到這個碑文嗎?.
這個是當時羅馬帝國的寫實.
一些歌頌皇帝的墓碑.
一些碑文.
你會發現有一個字.
不知道大家能不能看到.
我先把這個放大.
這樣放大了.
誒?.
不要不要.
請你放大鏡頭.
是了.
我放大了.
看到這個字嗎?.
是了.
看到這個字叫Evangelia嗎?.
是了.
不要動.
這個Evangelia這個字.

$^{1321}$是了.
不要動.
Evangelia這個字是福音的字.
所以這個字不要叫福音.
叫做Good news就對了.
我現在有一個Good news給你們聽.
凱撒的生日就是Good news.
當時的皇帝是很用這個字的.
我現在告訴你們.
我們有一個好消息.
我們現在可以打針了.
這個是Good news.
我們現在可以通關了.
這個也是Good news.
所以當時的Good news是一個縱數.
這個Gospel是加一個s字的.
所以當時羅馬帝國不斷地去宣傳很多的Good news.
News有一個s字.
其實是一個縱數.
也就是說.
其實這個是很平常的字.
有很多不同的Good news給當時的人.
在這樣的情況下.
當時的教會大概在公元30年左右.
耶穌過世真無多久之後.
教會就開始用這個字.
不過是用一個單數的字.
The only one gospel.
Evangelion.
不是一個縱數的字.
所以很明顯不同.
當時羅馬帝國是用一個縱數的Good news.
很多很多不同的信息.
很多很多很零碎的Good news給大家.
但當時的教會偏偏宣傳一個唯一的Good news.
所以整件事我們就說.
所謂的Gospel.
其實大家要忘記這個屬靈的字眼.
我們問究竟這個Good news是有多麼Good.
關乎於什麼.

$^{1361}$和我們有什麼關係.
在什麼的Context下被宣傳.
請下一張.
所以當時你發覺.
聖經裡面有兩個很不同的版本.
或者兩個不同的意味的Gospel.
第一個就是保羅的福音.
如果你不看福音書的時候.
你單單看保羅的書信.
保羅所說的Good news.
其實關於什麼呢.
大家看一看.
第一個就是.
甚至我從耶路撒冷轉到以來.
猜測直到傳基督的福音.
看到是一個基督的福音.
有關尼賽亞的Good news.
第二張.
然而我們沒有用過者權並.
倒凡事忍受免得基督的福音被阻隔.
要留意這個語言.
他用基督的語言.
就會有關基督的Good news的福音.
第三張.
這個更加重要.
因為上帝的義正在這福音上顯明出來.
這義事本於信而至於信.
如經所記.
義人必因信得新.
這個福音是關乎於叫人得救的.
叫人因著信的緣故.
能夠得到這個義.
所以這個福音是關乎於.
這個Good news是關乎於耶穌基督.
和我們的救恩.
能夠讓我們得到永生.
是一個十字架的福音.
然後這個更加重要的定義.
因為保祿說過.
在臨前十五章裡面.

$^{1401}$他說.
「你們我如今把先前所傳給你們的福音.
告訴你們知道.
這福音你們領受了.
有靠著.
暫立得住.
並且以你們若不是徒然相信.
能以持守我所傳給你們的.
就必因這福音得救」.
這是一個得救福音.
是什麼.
第一就是基督照聖經所說.
「為我們的罪死了.
而且埋葬了」.
又照聖經所說.
「第三天復活了」.
這是一個很典型的.
金羅安桑方派所講的福音.
因為它關乎於得救的福音.
就是耶穌基督為你得十字架.
然後復活.
你就能夠得到永生的福音.
最後一句.
這個福音是什麼.
甚至加上說.
保羅強調.
你不可以來到它搞錯.
這個是沒有其他別的福音.
那些其他錯的福音.
例如受國禮的.
國禮得救的.
都不是真正的福音.
唯有基督的福音.
我之前傳給你們的.
才是那個純正的福音.
這個就是保羅的福音.
我們今天基本上福音派.
或者基督派所講的福音.
其實是保羅的福音.
是一個經過保羅神學化.

$^{1441}$消化了之後.
因為基督耶穌受難.
十字架緣故.
他重新來到.
再一次的來到.
去理解這個福音的內容.
不過我們發現.
其實當你看福音書的時候.
耶穌也講福音.
耶穌也講福音這件事.
特別你看下一章.
《馬可》的時候.
第一章第一節.
有一個很特別的一句.
這是整本書的名稱.
其實這不是第一句.
而是整本書的書名也好.
或者整個的名稱也好.
他說什麼呢?.
順德的疑似.
耶穌基督福音的起頭.
如果你把滑鼠放在中文字上.
有東西可以看.
請看.
是.
看到這個嗎?.
這個是根據原文裡面的次序.
第一個就是阿肯.
就是從Eugenium.
就是福音的起頭.
整個書就是福音的起頭.
然後是耶穌基督.
就是耶穌基督.
然後是神的頤指.
所以整個名稱的原文.
如果直譯的話.
是什麼意思呢?.
就是福音的起頭.
耶穌基督.
神的頤指.

$^{1481}$如果我們相信.
馬可福音是第一本福音書的時候.
其實馬可正是在.
羅馬帝國的時間裡面.
在羅馬城市.
他嘗試去做一個新的genre.
一個新的體創.
當時沒有福音書這件事.
他是第一本福音書.
所以他嘗試去寫.
耶穌基督的生平和教導.
並且他要以福音來命名.
所以這是一個很突破的想法.
記住.
是沒有福音的這個Souling Dragon.
這個叫做好消息.
所以當馬可在羅馬帝國裡面的核心.
羅馬.
他嘗試去講一個.
整個耶穌的故事給人聽.
他就說.
這個就是福音的起頭.
就是說一個好消息的起頭.
就是耶穌基督.
所以你發覺.
當時他們嘗試做什麼呢.
他們嘗試去稱呼耶穌基督.
才是真正的good news.
而不是當時羅馬帝國的文宣.
當時羅馬帝國很多皇帝不斷在度假.
不喜歡出文宣來.
有什麼good news什麼的.
但是他說.
真正的good news.
其實只有一個.
就是耶穌基督自己.
而當你去看回福音書的時候.
你發覺福音書.
特別是婦女福音.
馬太.

$^{1521}$馬可路加的福音書.
所講的耶穌.
其實福音是怎麼樣的.
我們看一下.
其實耶穌也講福音的.
但是耶穌的福音從來都沒有講.
自己會得十字架.
不會這樣講.
耶穌講什麼福音.
我們看一下字眼.
耶穌就走遍加里里.
在國會堂里教訓人.
傳天國的福音.
你看到形容詞是不一樣的.
不是基督的福音.
而是天國的福音.
第二個.
你們去把所聽見所看見的事告述若寒.
就是乞子看見.
瓊子行走.
長大或風得竭症.
龍子聽見死人復活.
窮人有福音傳給他們.
這個很明顯是和當時的困境有關係.
耶穌所傳揚的好消息.
是關乎貧窮人.
關乎社會邊緣上的人.
一些蠻的,聾的,跛的,不方便的人.
耶穌所傳的好消息.
不是我要得十字架.
然後你信了就能得永生.
而是一個很具體的.
就是一個社會關懷的福音.
耶穌所傳的福音.
是告訴人現在就有好消息.
你不用等死去上天堂.
而是說你能夠得到真真正正的好消息.
第三段.
耶穌來到加里里.
宣傳神的福音.

$^{1561}$說日期滿了.
聖律的國近了.
你們當悔改信福音.
記住耶穌所講的福音.
不是叫你信耶穌.
而是叫你去知道天國近了.
上帝的國度將要來臨.
所以那些人可以得到釋放.
那些人能夠得到解放.
所以你們要悔改.
要信這個好消息.
就是一個關乎天國神的福音.
然後.
第一個重要.
就是耶穌在加伯倫會堂所講的福音.
當耶穌站在會堂里.
開始宣講.
要讀《以塞雅書特羅亞章》的經文.
主的靈在我身上.
因為祂用高高我.
叫我全福音級及貧窮的人.
被老的得釋放.
黑眼的得看見.
受壓制的得自由.
報告上帝月立人的欺凌.
這句話其實是一個非常解放的福音.
耶穌所講的就是.
尼塞雅將要來臨.
並且尼塞雅來臨之後.
那些貧窮人能夠得到這個好消息.
被老的人得到釋放.
黑眼的得看見.
受壓制的人得到自由.
然後你們記得.
在經文裡面是怎樣的.
耶穌被人打.
因為祂說什麼.
我就是那個福音的內容.
我的來到正正就帶來這個好消息.
所以你發覺耶穌所講的福音.

$^{1601}$是完全與保祿海嘯不同的.
保祿所講的是基督的福音.
耶穌所講的福音是一個關乎於天國.
是說受欺壓得到解放的福音.
最後請.
耶穌祖先們說.
我必須再辟承全上帝國的福音.
因為我奉差原是為此.
所以耶穌所講的福音是關乎於天國來臨.
地上的國度.
他們將要面對上帝國度來臨.
而我就是尼賽亞.
就是那位君王.
所以你發現.
整個福音基本上是根據.
以賽亞書.
請下一張.
就是那個經文.
請下一句.
整個以賽亞書裡面.
很重要的經文.
就是內章裡面.
主耶和華的靈在我身上.
因為耶和華用高我.
叫我傳好消息給貧窮的人.
差見我這致傷心的人.
報告我.
報告被勞的得釋放.
被囚的得出監牢.
報告耶和華的恩憐.
和我們上帝保守的日子.
我安慰一切悲傷的人.
所以整個福音是建基於舊日裡面所講.
一個以賽亞書內章所講的經文.
所以整件事是關乎於.
現世裡面.
一個很真實地尼賽亞.
去解放人民的福音.
所以我們嘗試做一個比較.
試一下,請.

$^{1641}$所以發覺是有兩個很不同的福音.
似乎很不一樣.
但兩個是不同的版本.
一個就是保羅嘗試去詮釋.
耶穌所講的Good News.
關乎於耶穌的十字架的意義.
十字架能夠叫罪.
能夠得赦免.
然後凡是信福音的.
就能夠罪得赦免.
並且能夠生復活.
這是我們所謂的.
傳統教會所講的福音.
而耶穌的福音是關乎於行動.
耶穌所講的是一個.
我們知道上帝的國度來臨.
我們知道現在的人.
受欺壓的人.
他們能夠得到釋放.
是一個傳給貧窮人的福音.
所以其實兩個都很重要.
你發覺有兩個不同的向度.
保羅的版本和方書的版本.
其實都是互相來捕捉著.
我們不能單單偏向其中一個.
單單看著保羅的版本.
你就會忘記方書所講的福音.
耶穌所講的福音.
其實是關乎於天國的來臨.
關乎於我們如何面向世界.
而耶穌的版本.
正正都是不能夠沒有保羅的版本.
因為保羅是很好的.
在整合整件事.
耶穌的來臨.
不是單單去幫助當事人.
而是來講整個世界.
如何終結人.
如何得到最後的釋放.
所以我想說的是.

$^{1681}$其實兩個福音都是要明白.
是互相捕捉的.
單單有Evangelium.
一個所謂保羅的版本的話.
就變成了一個離地的假想.
只是想著將來如何得救.
如何上天堂.
這個不是福音的全部.
而單單一個所謂關懷社會福音.
這不是整件事的全部.
我們知道上帝的請求.
不是純粹關乎於現世.
而是關乎於整個世界的終末.
和整個人類的未來.
所以就是說.
根據這幾年裡面的討論.
其實我們呼出的這些福音.
其實是一個.
我們不能忽略這兩邊.
我們所謂得救是重要的.
但卻不能忽略社會上的影響.
所以我覺得有一個很好的整合.
請下一張.
大家如果看Anti-Right的一本書.
其實是一本很好的書.
中文翻譯也有.
叫做Simply Good News.
Anti-Right是一個很簡單.
用了百多頁寫出來的福音書.
有福音內容的教導.
一個非常好的解釋.
究竟什麼是Good News.
他說耶穌所宣講的好消息.
和我們後來所講的好消息.
其實是一個相同的信息.
兩者其實不是矛盾.
是一個很重要的互相補充.
所以好消息是什麼呢.
就是獨一真實的上帝.
要透過耶穌和祂受死的復活.

$^{1721}$並在一切裡面.
即是他執掌權並統治世界.
即是耶穌的死和復活.
是要改變的.
是關乎於整個世界的事.
以色列人自古以來的盼望.
已經實現了.
可是那實現的方式.
是他們意想不到的.
神在拯救世界的計劃.
終於啟動了.
他按照他的應許.
以新的方式來管治大地.
解決世界的問題.
他並以他的榮譽和公義.
充滿世界.
但是他成就這一切的方法.
完全消乏所有先知的想象.
所以就說.
原來從前所說的尼賽亞.
正要來到了.
不過那方法.
不是我們所想象的那麼簡單.
聖約瑛他要去重新.
用新的方法來管治大地.
去解決整個世界的問題.
不是單單個人靈魂的問題.
福音所謂的Good News.
是關乎於整個世界的.
自古以來世界病了.
世上的人都病了.
其實大家都有病.
人個體自然病.
人的命運是病了.
但是世界都病了.
所以現在上帝要把疾病醫治好.
使世界和人都能重獲新生.
有生命力將江河一般流出來.
悉心不惜地注入世界.
這種生命力是一種新的力量.

$^{1761}$他就是愛的大能.
好消息是.
以上一切已經在耶穌的身上發生.
並且借著耶穌成就了.
所以耶穌來到是要改變的.
不單單是靈魂的得救.
而是整個世界.
有關國度上的改變和更新.
總不久將來.
整個受造的世界有同樣的變化.
整體人類.
我們每個人.
不論是誰.
都要得著更新改變.
這就是基督教福音.
所以將來整個世界將會改變.
整個人類的命運都要改變.
這就是好消息.
如果你補充好.
這個好消息.
當然是好消息.
但這些好消息不需要傳給別人.
不關我事.
如果日報贏了.
當然是好消息.
對我來說.
但可能不關你事.
唯一一個值得.
向全世界人說的好消息是什麼.
就是世界將會改變.
世界將會改變.
所以縮音的向度.
似乎是一個很重要的內容.
我們不單單關乎一個個人的得救.
而是關乎整個更加寬闊.
更加遠大的世界觀的改變.
下一張,謝謝.
所以你試著想想.
你反省一下.
我們認識福音之後.

$^{1801}$究竟這個比較寬闊的方觀.
對我們有什麼意義呢.
謝謝下一張.
以前我們很簡單.
福音就是一個很簡單的五分鐘內容.
我們都會說.
我們以前對著.
我們有罪.
然後就怎麼樣.
對不起,他應該是想創造世界.
然後我們有罪.
他都來了.
十字架裡面.
為我們釘死.
你只要認罪悔改.
就能得到福音.
然後大力決治.
這是一個很簡單的福音.
福音今天其實很簡單.
我們基本上可以說.
大亂引賊.
我們不斷地說.
信仰述德永生.
不斷地重復.
重復這個很公式化的福音.
不斷地教人家說.
不斷地說.
方成為了一個天堂的入場卷.
這張快速通過.
只要你按按鈕就能拿到快速通過.
方成為了一個得救的途徑.
一個很特別的方程式.
只要你能夠相信.
你就能夠上天堂.
這個福音是一個比較傾向簡化.
和沒有關係的東西.
你問面對今天的香港.
我和你學習的這個福音.
有什麼意義呢.
我面對這樣的世界的時候.

$^{1841}$你跟我說.
想快速通過可以死去上天堂.
是好的.
但這個福音似乎和我們沒有什麼關係.
都是一個比較遠的故事.
但我們說福音其實不是一個罐頭.
福音是一個關乎於整個世界.
關乎於上帝國度的改變.
關乎於我們的未來.
這個福音就是關乎於我們整個人.
你不能夠只是單單聽著故事來做人.
你要整個人去投入下去.
這個是關乎於我們.
你怎樣能夠全個人來真誠地貫徹這個故事.
你只能夠跟著整個故事走下去.
所以第一件事就是.
福音其實是一種世界觀.
當你去信福音的時候.
福音其實是說你怎樣去理解這個世界.
當你面對著今天的世界的時候.
你的世界觀其實是改變了.
就算你面對著今天的香港.
福音其實是一個關乎於一個Good News的時候.
這個Good News就是說.
上帝的國度已經開始離林.
世界已經是轉變了.
這個是關乎於你怎樣去看香港.
你怎樣去看香港的未來.
怎樣去理解現在的生活.
當你有這個所謂Good News的時候.
其實你是否真的有.
你是否真的帶著一種好消息的態度來看著這個世界.
這個世界和福音沒有什麼關係.
我們有一個拿著的故事去講.
耶穌帶我們去信耶穌.
然後十字架.
怎樣能夠得到永生.
但當我們看著香港的時候.
我們仍然覺得很悲傷.
仍然覺得很絕望.

$^{1881}$其實這不是一個對劇本的人生觀.
當你真正信福音的時候.
福音其實是一種完全真真正正的世界觀.
你怎樣去看著今天的世界和香港很重要的改變.
福音是什麼.
這個許詩就是說耶穌的徒將要和已經是更新的世界.
改變的不是我的靈魂.
不是我做得聖變那麼簡單.
而是整個的世界已經是改變.
所以我能夠理解今天發生了什麼事情.
因為我知道這個世界是有一個好消息.
所以今天經常說.
今天我們說國安法之後我們怎樣傳福音.
我想說的是.
其實今天我們更加明白這個好消息是多麼好.
因為這個好消息正正告訴你.
我們面對著這個世界.
我們知道的消息就是這裡.
你明白那個興奮的地方在哪裡嗎.
如果我們真的去明白這個好消息的時候.
今天我們更加能夠真正去明白這個好消息的真正深度.
所以這個好消息是讓我們能夠.
所謂叫做好消息地活下去.
我們能夠懷著這種好消息來做人.
帶著一個好消息的態度來做人.
因為我們真正是信福音的.
我們真正去明白這個世界將會是怎麼走的時候.
我們就拿著這個福音來好好做人.
福音再不是一個純粹簡單的演繹.
一本書,一個福音橋就完了.
而是和我們的生活有很大關係.
當時福音就是這樣.
當我們面對著很多困境的時間.
我們就知道這個好消息究竟有多好.
和我們這個世界有什麼關係.
不過下一章.
所以說了這麼久.
什麼叫傳福音呢.
當我們知道福音的內容之後.
我們的福音是怎麼傳的呢.

$^{1921}$我們有沒有一套東西去傳呢.
或者我們是怎麼去傳呢.
如果福音是關乎於我們去看世界的視野的時候.
我們的福音的傳揚.
首先就是我們要這樣做人.
按著福音的世界觀來做人.
你知道這個是好消息.
你就帶著一種很開心.
或者說是一個好消息的態度去活下去.
從而你也會和別人說.
其實你面對著香港.
今天面對著蘋果被人這樣搞的時候.
我們仍然要去.
正正在這個位置.
你也要和別人說.
我有一個好消息要告訴你.
就算蘋果被人這樣搞.
好消息是這樣的.
因為到了最後的時間.
黑暗的勢力將會滅亡.
世界將會被更新.
福音是能夠和你今天所見到的新聞.
是有點接點的.
每一個六月發生的事情.
你都可以用一個好消息來延續下去.
因為這些新聞.
這些事情都不是最後的完結.
而是能夠用這個好消息來和別人說.
雖然是這樣.
雖然那個人被判無罪.
但我們仍然告訴他.
我有一個好消息要和你說.
這個好消息正正和你有關係.
是真真正正和你有關係.
不是一個二千年前的故事.
而是關乎於你今天身處在香港裡面.
仍然可以去堅持下去的一個好消息.
所以這個福音.
我們首先不是一種很簡單的故事.
而是我們的生活態度.

$^{1961}$你只能夠懷著這種生活態度.
來延續這個福音的內容.
然後向你身邊的人.
來告訴你這個好消息.
所以下次你和別人說的時候.
你不需要拿著什麼東西來和別人說.
你和別人說.
就算面對著今天這樣一個新聞的時候.
我都可以和你說一個好消息.
因為耶穌的來臨.
世界最終可以這樣這樣這樣.
世界是會改變的.
縱然這樣的時候.
這個好消息正正就是二千年前.
信那幫門徒和別人說的好消息.
縱然羅馬帝國是這樣的時候.
我正正和你說過.
The only one gospel.
一個這樣的好消息給別人聽.
當然我們仍然會有一個很特別的迷戀.
我們有一個福音五色珠.
以前大家有沒有玩過.
大家都試過都訓練過.
我們都有一個這樣的教導.
簡單的五色珠.
講給別人聽.
來講福音.
這個福音是有些缺陷的.
因為他只看那些書.
只講到你的靈魂得救就完了.
大聖耶穌你就能夠缺陷.
所以我自己就重新來做一個新的.
叫做Full Church 五色珠.
(笑聲).
這個五色珠是五色珠.
麻煩你幫我按按.
我們試一下這樣做.
都有五個步驟和別人講.
假設你個出信者.
未信的人和你說.

$^{2001}$我就試一下講.
第一個.
我們都是綠色.
綠色很簡單.
差不多和舊的那個.
綠色你會想到什麼呢.
會想到就是.
可能是一些樹葉,高山.
甚至生命.
是的.
因為枯葉並不是綠色的.
神造人原意是讓人得到生命的.
並且享受一個和平,美善,豐盛的生命.
這個是上帝原創的旨意.
這個版本是一樣的.
不過加了和平的字.
一個很簡單的創造.
我和你講.
這個是我們FourTruck的五色珠.
好 第一個.
第二個就是罪.
很可惜.
黑色代表什麼.
當然是代表.
黑色的東西.
(笑聲).
看這個社會.
充滿黑暗和罪惡.
可以這樣和我們講.
這是現代版本的五色珠.
因為人有罪.
身處在一個不公義和欺壓的世界.
充滿著謊言.
你可以加上一些新聞例子.
黑警,裸二等等.
(笑聲).
這樣講吧.
黑色的東西.
人充滿著罪.
所以每個人都是罪人.

$^{2041}$當我們面對著黑暗的世代的時候.
我們是罪人.
都是被罪者.
每個人都是犯罪的.
都是被罪所困擾的人.
我們是被罪所影響,被欺壓.
同時也成為了罪人.
因此聖經說.
世人都犯了罪.
無論是個人,社會,政權.
都不怕負擔神的心意.
所以這個罪不單單是個人的問題.
而是整個社會,整個世界的問題.
當然包括很多不公義的政權.
所以世界充滿著黑暗,荒謬,罪惡.
所以這是我們第二個顏色.
就是罪.
這個罪是關乎於更加闊諧.
一個社會上的罪.
一個不公義和黑暗的世代.
然後就是紅色.
其實也是粉紅色.
耶穌是尼賽亞.
萬國的君王.
我們可以說.
耶穌就是各國的君王.
是萬國之君.
耶穌在世的時候.
宣告上帝國道的好消息.
記得耶穌的反部福音.
一個負力福音的反部福音.
事實上,祂就是人類好消息的起頭.
馬克思第一章第一節里所講的理由.
他來到世上.
代表著上帝國道正要降臨.
關乎於整個國道的理能.
耶穌在十字架上的死.
毀滅了世界上一切黑暗的權勢.
人類得著解放.
所以十字架不單單是來叫我們聚得聖明.

$^{2081}$而是來消滅黑暗的勢力.
死亡的毒勾.
黑暗的勢力都能夠被破壞.
所以不是純粹個人的罪.
而是整個魔鬼的角度.
一個邪惡勢力都被毀滅.
耶穌的復活正是新創造的開始.
復活不是單單暗示了我們能夠得到永生的復活.
而是整個世界將要被更新.
而這已經是開始了.
上帝國道將要圓滿的臨到.
世界上所有不義的政權將要終結.
世界被更新.
並且帶來終極的公義和和平.
這就是十字架的意義.
一個更加完整的意義.
十字架不單單流血為我們死.
更加能夠勝過世界的黑暗.
這就是正式所講的.
然後就是白色.
白色傳統來說就是叫你信.
然後就是缺志.
我們就把它稱之為盼望.
基本上是信望外三個很重要的元素.
你跟他說你信不信這個好消息.
當你去講這個世界觀的時候.
如果你是相信這個世界觀的時候.
你就是開始去信這個福音.
如果你相信的話.
你也是在黑暗時代里成為一個好消息的部分.
你更加要成為這個宗教的部分.
因為耶穌基督要呼召你.
你用生死的態度來活出這個世界觀.
並且成為一個基督徒去傳揚盼望.
這就是第一堂所講的.
我們第一天開始就是被呼召去傳揚這個福音.
所以當你來到去信這個世界觀.
這個世界的將來的時候.
你就要好好地用生死的態度來活得像這個好消息一樣.
帶著這份好消息來做人.

$^{2121}$在上帝的角度上.
在你完全離開之前.
用愛來幫助世上所有需要的人.
並且懷著盼望面對黑暗的時代.
這個正正就是我們所謂大人所謂缺志.
那個很重要的元素.
不是單單信.
而是你要帶著盼望.
當你有這份福音的時候.
你就有這份盼望來看著這個世界.
並且用愛來幫助.
好像耶穌一樣.
在在世里幫助貧窮的人.
這十幾億的人.
並且活得好像好消息一樣.
最後一句.
請稱之為光復.
你會看到最後都是黃色.
上帝的國度將要離臨.
創世之初的公義和平友愛國度.
終要被光復.
即從前被黑暗勢力所影響的世界.
將要去過去.
上帝公義和平國度.
被恢復出來.
到那一天.
人類將要和諧共處.
共同活在沒有死亡.
沒有謊言.
沒有欺壓.
真正自由的國度裡面.
耶穌的祝福在我們的王子里.
永遠永遠.
所以這就是我們的.
不識諸.
當然將來不會教大家做這些事.
但起碼我們知道.
如果我們用榜樣來說.
我們這樣來教我們福音.
福音是關乎於更加闊的東西.

$^{2161}$我們能夠帶著福音.
帶著盼望的做人.
來和別人宣講.
當你有些朋友.
被現在的香港.
很多的形勢打倒的時候.
我們會和他說一個好消息.
這個好消息.
是真真正正能夠幫助他.
面對今天的世界.
今天的社會.
所以最後.
我和大家分享一段經文.
就是保羅所說的.
他說我不以福音為恥.
福音本是神的大能.
就是.
保羅所說的一個好消息.
其實是一個很荒謬的好消息.
因為這個好消息.
其實是沒有什麼人覺得可信的.
猶太人不信這個福音.
面對著當時的政權.
當他說耶穌基督的國度.
是一個很難以想象的事情.
今天我們也是一樣.
今天我們也是用福音來做.
我們人生很重要的生活態度.
我們不以福音為恥.
雖然現在面對世界.
和我們的好消息.
好像還未能對得上.
但我們確實仍然以福音.
來做我們生活每一刻的嚮導.
因為整件事情.
是上帝的大能.
來到香港裡面.
這個大能.
正正就是福音的根源.
讓我們能夠經歷上帝.

$^{2201}$在二千年前已經開始工作的大難.
這個好消息正正已經來了.
讓我們能夠懷著這份盼望.
來學習,來過每一天.
我只能祈禱.
在你知道你是那位尼賽亞.
你昔日來到世界裡面.
宣講神國福音的時候.
要告訴你這個世界的歷史已經改變.
這個世界的國度將要毀滅.
因為你的國度將要離林.
我們相信這份好消息.
並且我們將要以這份好消息.
作為我們人生最重要的嚮導.
當我們面對每一天香港的新聞的時候.
我們知道這個好消息.
正正是能夠幫助我們.
來面對這些事情.
因為我們知道這個好消息.
正正就是這個世界每一天的轉變.
最終的結局.
讓我們一群付出的人.
能夠帶著這份好消息來過活.
更加成為一個傳福音的人.
一個傳揚你好消息的門徒.
能夠讓每個人因為好消息.
這個福音能夠得以得著盼望.
得著改變.
求主你幫助我們.
奉主命求.
阿們.
江仔.
你都講了一個小時了.
是嗎?.
但是你還沒有叫食物.
是嗎?.
今天有什麼吃的?.
今天有沒有五色炒麵?.
我以為你只喝一杯.
相計我就是好消息.

$^{2241}$是嗎?.
其實我聽到你說了這麼久.
那個福音很多東西都可以包底的.
很多東西都可以搭到嘴巴的.
其實應該是.
如果包不到底就不是福音.
我看見你說用福音台聊天.
那個世界觀.
其實很方便.
我們都可以融入到人群.
是的,可以試一下.
我做茶餐廳的.
我接觸很多客人.
我可不可以用這個位置去接觸客人.
你試一下想想.
你可以怎麼做呢?.
一來就問他.
我識過五色豬.
我跟你講一下.
我想要牛腩河.
加凍奶茶.
你試一下試一下.
怎樣能夠講到福音呢?.
世界觀可以怎樣改變呢?.
人是要吃東西的.
是.
但是有些東西不是吃東西可以解決的.
是不是這樣呢?.
然後呢?.
所以人有沒有東西不是靠吃東西解決.
而是要一些更厲害的東西可以解決到人以外的東西.
是不是這樣呢?.
可以嗎?.
就是這樣.
吃著吃著就發現那些東西不公義.
哦,我知道.
不公義之後就發現耶穌更新了世界.
就是東西被人弄丟.
甚至我經常被人拆牌.
是.

$^{2281}$我開什麼都被人拆.
是不是這個意思?.
然後就說不要緊.
因為有好消息可以告訴你.
因為將來就會得到更加好的公義的審判.
哦,有點難.
我開茶餐廳.
很多街坊有很多不同的環境.
大家覺得什麼是好消息呢?.
對他們來說,今天聽完之後.
或者你覺得好消息有什麼困難呢?.
是,請說.
教會有教會的包袱.
可以多說一點什麼意思嗎?.
上一輩的那種傳福音.
你這樣又對,你這樣又不對.
一想起就想起大教會.
我們中國人已經少了教會的包袱.
哎,站起來,不可以,不用站起來.
我怕被人….
先說一句.
我想傳統的….
自己長大也有接觸過這些教會.
自己是基督徒也會覺得反感.
不是基督徒的時候.
我瞭解為什麼別人會覺得反感.
變成了要有一個位置.
跳過他們心理的關口.
就像我來樓堂.
我之前對教會很反感.
要來就要迫他們很大很大的能量.
要過來.
你叫一個不是基督徒的人去聽.
你一說基督徒想說基督徒的話.
他就想走,走開,彈開.
明白,對.
剛才瞭解你舊有的包袱是什麼呢?.
例如會是比較性的包袱.
哦,就是你有罪,你一定要….
對,罪人,你不下….

$^{2321}$其實真的很….
你不信神,落眼地獄.
只有這個神是對的.
你信佛就迷信.
我雖然沒見過這個神,但我不是迷信.
那一種很傳統就是.
我告訴你這個就是神.
你下地獄,你明明可以選擇上天堂.
你還是要選擇地獄.
那就是罪人.
那種背景的話.
這個是不是剛才說的那個比較狹窄的觀念?.
以前的那種看法.
那是難傳的,在這個年代里.
即時間變成跟人說地獄.
我也不知道怎麼回事.
所以要闊一點.
講一下政局.
講一下新聞,講政治.
這件事是能夠接到福音的內容裡面.
因為這個福音就是關乎於.
那個世界的發展之後的東西.
就多過一個神話.
講一下天堂,地獄的話.
這個就好像不是跟現在的一些….
除非那個人很興趣.
但其實應該是關我們現在發生的事.
就好像你剛才說到耶穌.
平時的日常就是去不同地方.
去瞭解人間的問題.
其他人呢?.
後面有.
剛剛說到福音.
其中一件事就是天國的來到.
這個世界上將會被改變,被更新.
我這樣聽下去是一個對於將來的盼望.
假設是對於一些不是去教會.
或者不是信基督教的朋友來說.
就是怎麼會令他們去相信這件事.
可能是我們是信開的.

$^{2361}$或者是我們是去教會的.
我們會傾向比較相信這一套.
但如果是說一些將來會發生的改變.
是由未發生的時候.
而又可能每天看著社會的情況.
其實說起來的時候.
怎麼可以令他們容易相信我們這一套.
我覺得是相信的.
因為現在的人都沒有出路.
你看著現在這樣的社會.
個個都不知道怎麼辦.
但你跟人家說.
現在的政權會怎樣結局.
其實人們是會想聽的.
比我叫你不要理會.
不如你上天堂.
因為兩個是很不同的.
都是福音來的.
但我們所說的福音.
是真的在說現在的處境.
現在很多人.
我覺得福出應該全部都是福音.
因為很多人是前線的人.
他們是很絕望.
當然移民是一個方法.
但我們去面對這些的時候.
有什麼出路.
很多人當他沒有信仰的時候.
其實他沒有出路.
而我們的信仰不是叫他.
不如你死去.
之後上天堂.
而是能夠說一個.
關乎於這個世界的事情的時候.
當你帶著這個好事來過活.
其實是你知道如何面對這件事.
剛才說白色那段.
就是盼望那段.
我們可以懷著盼望.
去面對這個世界.

$^{2401}$所以白色那段其實不是信耶穌.
而是把盼望放在我們比較重要的位置.
不是信完就行.
而是我們能夠懷著盼望.
去看著這個福音.
或者從另一個角度來看.
好像是將來式的.
現在處理不了.
好像將來式.
但其實也是一個過去式.
因為其實在不同年代都會遇到一些困境.
或者在不同年代當中.
都會有沒有出路的情況.
但是在不同的年代當中.
福音或者好信息.
都是帶動那些年代的人.
去經過那些日子.
所以歷史是不斷地重演.
也是重現在那個年代發生的事.
好像過去這段日子.
有不同的聚會或者主題都在說.
在文革時期的基督徒是怎麼生活.
或者在一個大政權之下.
基督徒或者基督教信仰.
怎麼可以延續對人的盼望的好信息.
所以這件事就.
到現在好像是將來式.
但其實一直以來可能會是過去都發生的事.
成為一個借鏡.
我們可以去想這個.
回應剛才所說的盼望.
嗯.
如果我們都知道有些事是要盼望.
以前都是.
摩西死的時候都見不到那些人去加拿大.
我們是明白的.
我都會覺得可能我死之前.
都未必會見到政權倒下來.
但是對於那些未信的人來說.
他好像如果我們這樣說的時候.

$^{2441}$他會不會覺得很壓縮精神.
你只是想.
我事實上就見到.
不停這樣的事.
每一天都會比每一天都更加厲害.
你喜歡做什麼就做什麼.
可能我.
看一份報紙都有罪.
我扔報紙都會有罪.
我覺得怎麼去解決.
當我們去傳福音的時候.
那些人他這樣去.
說的時候我們怎麼去回應呢.
嗯.
當然我們不是這樣去完.
你見到我們很多今天.
很多被人抓了的都是基督徒.
或者是青少年.
他們其實.
都是懷著很深層次的信仰的力量.
那個就是福音的內容.
他們會知道這個世界怎麼完.
才會到今天做事.
所以我們不是說Q就沒事了.
但很多時候我們.
今天去面對.
去改變一些東西.
雖然被人抓了.
但他們心底里很重要的一份力量.
就是來自於這份所謂的壓縮精神.
其實不是.
是信仰那個Good News.
當你連Good News都沒有.
你根本就沒什麼可能會去.
不容易去.
今天來去行動.
所以我們.
當然有些人信了很正常.
不是每個人都信的.
這個就是.

$^{2481}$世界上總有些人不信這個Good News.
不過很多人都是不信那個Good News.
不信那個天堂版本的福音.
你叫我天堂地獄.
我不是很信.
我都不在乎這件事.
但你和他說這個世界.
其實這些政權或者今天.
這個社會公義怎麼能夠得到圓滿.
得到一個比較圓滿的解決.
這件事起碼能夠和他今天身處的有關係.
當然會有些人不信.
但是我們.
當我們有這份Good News的時候.
我們就能夠去做當下的行動.
很多的今日政治家都是信仰的人.
正是因為這樣的原因.
有些人會覺得很奇怪.
但這個也是動力.
是你摸不到將來的力量.
如果你覺得阿Q是一些很不實在.
或者是個人FF的東西.
我覺得反而現在這個Good News.
就正正不是阿Q.
因為我們現在眼見現實的環境.
那種不公義.
那種讀灰.
那種指責的文化.
其實我們打開馬太豐第十章的時候.
正正耶穌說.
你們出去傳福音的時候.
你就會遇到這樣的情況.
你就會遇到家人的摒棄.
你就會遇到很多人帶你去會堂去公審.
這件事就更加讓我明白到.
聖經說的話對於末世是真實的.
聖經說的話其實在二千年前.
已經是成書告訴我們.
這個世界的終局會是這樣的.
我現在叫做有幸.

$^{2521}$去見證到聖經說的話的真實.
切切就不是阿Q.
所以聖經在告訴我們.
那個就是Good News.
這個也是讓我們明白到.
其實有一本書一直告訴我們.
我們可以看著人與人之間的分離.
那種終局.
那種盼望的失去.
但是有耶穌當中能夠成就這件事.
這個就不是阿Q.
這個正正就是我們有機會.
有法可醫的時候.
我們更加要學習耶穌.
那種福音的闊點.
就是做好我們的日常.
可以做到周祭.
關心.
接待.
和瞭解人需要.
那個責任和本分.
對人來說就是Good News.
後面呢.
聽到大家問這些問題.
我想分享一下.
今天剛剛和我兩個未信的同事.
講起政治的問題.
我是基督徒.
我真的和他們說.
我說我們有信仰是好一點的.
因為相信上帝始終都會審判.
那些不公義的行為.
我心裡面都會有.
剛才大家提起的那種掙扎.
就是他們不信.
聽到的時候.
他們會有什麼想法呢.
但我說出來的時候.
我知道這個真的是我作為基督徒.
我的相信.

$^{2561}$我真的真的相信.
上帝始終會去審判.
人類歷史里一切不公義的事.
因為我的同事都是很.
很追求公義的.
因為發生了蘋果日報的事件.
我們從2014年講到2019年.
講到今年.
真的有很多感受.
去到大家都很灰心.
講到大家覺得根本改變不了.
社會的任何東西.
由我們有機會爭取到.
連爭取的機會都沒有了.
但去到最後.
我還是很掙扎地告訴他們.
我說我相信就算我們今時今日.
沒有任何事情可以做到.
就算我們改變不了.
香港現在的局面.
但我始終都相信.
他們現在在為所有人.
做過不公義的事.
他們一定會承受他們自己的後果.
我其中一個同事.
他們兩個都不是基督徒.
我其中一個同事都說.
是的,他們會受他們應有的報應.
我都跟他們說.
雖然有生之年我們可能看不到這些.
但是.
我就是將我心裡面.
相信的那件事講了出來.
我會覺得是.
就算他們不相信也好.
但至少我要知道我所信的上帝.
是最公義的審判者.
無論我們現在看到.
在法律界裡面.
有多少.

$^{2601}$有多少立場很偏頗的律師.
或者是裁判官.
或者是法官都好.
但是我們還有一個最公義的法官.
我當時也是這樣跟我兩個未信的同事說.
我不知道他們會覺得是.
假Q還是什麼的東西.
但是.
事實是不變的.
不變的就是我們的上帝.
就是不會變的,祂就是公義.
有時候我覺得.
在面對這樣的情況.
我再跟姐妹分享.
有時候面對這樣的情況.
當我們.
怎樣才叫做全福音呢?.
未必一定會跟她這樣說.
有時候面對這樣的情況.
有一種淡定的態度很重要.
因為我知道這個好消息的時候.
我是能夠知道怎樣面對這件事.
這種態度其實是很舒服的.
因為.
我想補充.
好消息除了將要做事.
其實除了做事.
2000年前十字架的耶穌舉動是做了事.
耶穌舉動已經是消滅了黑暗力量.
做了,不過是未完全地來到去.
去實現整個國度的理能.
所以我們那種拍摩不是單單等它做事.
而是它做了事了.
所以你看著今天的政權或者黑暗勢力.
你是淡定的.
因為你已經玩完了.
就好像不得之拳.
你已經死了.
沒看過.
你已經死了.

$^{2641}$你看著它已經死了.
它好像很囂張.
但它已經死了.
這種淡定是很重要的.
所以我們跟別人宣講福音.
就是這種心路態度.
我們福音的態度.
當你真的相信這個好消息的時候.
你雖然是投入.
但你仍然可以.
那種盼望就在那時候.
我們能夠知道世界是怎樣完.
以及上帝已經做了事.
那我們就能夠去安慰.
或者跟別人一起走前面的路.
雖然大家都很害怕.
雖然大家都不知道怎樣前面走.
但你又不是完全不知道怎樣走.
因為你知道上帝已經做了事.
以及將來會怎樣.
還有每個人都會有自己的信念.
我都很認同剛才梓梅說.
在審判的觀念上.
我自己都跟自己說.
如果真的沒有審判.
其實做好人和做壞人是沒有分別的.
還有你做什麼.
其實沒有一件事去制裁.
或者做一件事有後果.
其實我為什麼這麼辛苦呢.
今天最困難的一件事.
就是做好人的負代價更加多.
而做好人更加責任.
對自己的要求高.
所以我們背後就知道.
在終末總會有一件事要平反.
還有終末會有一件事.
會跟你去看你的清單.
其實在做什麼.
所以我們真的相信.

$^{2681}$上帝是會做事的.
在將來他會再回來的時候.
這是我們很相信的.
其實很多未信的人.
或者對信仰不深的.
未認識很深的人.
其實底層都相信.
有些事是現在未行.
但將來不代表沒有.
不過在他的系統里.
在他的世界觀里.
沒有所謂可以解釋到.
但我們在對話當中.
就可以告訴你.
我們的信仰告訴你.
好消息可以幫他明白到.
將來審判上帝會做些什麼.
因為上帝之前已經做了.
就是敗壞一個掌死權.
就是政治和人.
即壞人可以殺死耶穌的身體.
但耶穌可以復活.
所以有些事情是.
過去人以為這樣就解決到.
但上帝會用第二個方法去解決.
這就是我們的信仰.
而那樣東西就真的做了.
所以上帝會再做其他東西.
去整治這個混亂的世代.
或者是歪曲的世代.
甚至有罪惡的世代.
這個講得最多就是.
我們等待的那一天來.
這個就是我們的盼望.
所以有審判.
其實不同人都有不同的程度理解.
但我們對審判的理解更加深入.
剛才聽到Kim Sir講.
其實在這個時代.
我們做什麼是很重要的.

$^{2721}$我記得2019年.
有一晚很多基督徒.
在正宗那裡唱.
Sing Hallelujah to the Lord.
我也是一個連燈粉絲.
我記得在那一晚之前.
其實很多連燈都叫基督徒.
都有一個名字叫做耶L.
但我記得那一晚.
所有基督徒在正宗門口.
唱了那首Sing Hallelujah.
然後我們去了灣仔警察總部.
唱了Sing Hallelujah.
我記得那一晚連燈.
很多人都說.
不要再叫他們耶L了.
我們證明叫他們基督徒了.
我覺得是在這些.
我不知道誰會受審判.
因為我們不是神.
但我很相信上帝.
一定有公義的審判對每一個人.
那我們可以做些什麼呢.
我講一個我覺得挺有趣的.
原來很多普通的人.
不是信徒會看你基督徒是怎樣的.
我前幾天去了一個朋友的餐廳吃飯.
他知道我是基督徒.
他說我跟你講一個很有趣的事.
因為他的餐廳在附近.
他說你是否回附近的教會.
我說不是.
不過那裡有個地方.
他說會不會是你們教會的人.
我說我不知道.
他說他們幾個是唱詩的.
他們在那裡不停地說.
三個唱詩都不合音.
他就說最合音的是誰.
然後有一個人回答最合音的是琴.

$^{2761}$然後他就覺得.
你們的基督徒都挺有趣的.
他說其實我很期待.
他們再來我這家餐廳吃飯.
我很想再聽他們講的東西.
其實很多外面的人.
是不知道你基督教是怎樣的.
可能因為比以前很多.
弱定俗成或固定的印象去看.
然後他聽到你們基督徒.
會講這麼有趣的東西.
其實他們覺得很有趣.
我覺得如果是好消息.
就是我們告訴他們.
其實在這個境況里.
我們是怎樣的.
其實我想我們的好消息.
就是告訴他們我們有信仰.
我們做人可以是怎樣.
然後給他們看.
然後我相信如果他們看到你真的好.
我相信他們會自己走來.
去問你基督教是怎樣的.
所以我的朋友也會問.
為什麼他們不合音.
為什麼不齊.
為什麼是琴.
反而是琴最合音.
我想知道多一點.
他反過來問我.
我覺得我們現在的信徒.
是否可以傳出好消息.
就是我們做好自己.
然後我們將我們所知的.
或者他們有問題.
我們能夠解決的.
去帶給他們.
我覺得是否可以做這些.
我想說最後的總結和補充.
其實今下的世代里.

$^{2801}$每一次的政治話題.
都是所謂的福音契機.
這個福音契機.
這個詞是很章的.
福音契機.
就是講好消息的一個機會.
就是一個接到嘴的時候.
你想想.
如果我們每一次講政治的時候.
其實都能夠接到嘴.
能夠去講好消息給別人聽.
這個正是一個最厲害的時候.
我們福音教會.
或者我們香港教會里.
能夠講福音契機非常多.
因為太多政治事件.
因為每一個事件.
都是能夠我們去講好消息給別人聽.
這就很舒服.
所以我覺得就是這樣.
你不需要找一次報道會.
講誰是藝人有癌症.
又是怎樣怎樣.
或者是什麼.
每一次這些社會話題.
都成為了我們講好消息給別人聽的機會.
就是這麼自然.
還有沒有其他?.
有沒有好消息?.
在後面.
剛才大家都講到關於.
大家都講到關於政治那邊.
我就在想我們怎樣可以向弱勢傳福音.
譬如見到一個執子婢的婆婆.
八十多歲,跟著她住tong 房.
她有拿過那些三國金.
但是都不夠交租.
又或者是一些無家者.
我們見到她跟她講傳福音.
可能都探了很多次.

$^{2841}$但是她都會跟你說.
你不要再跟我講福音了.
因為她的處境令到她.
曾經有人跟她講過.
但是對於她來講這件事是很.
對她們來講好像很遙遠.
那我們可以怎樣將這個福音帶給她們呢?.
就是怎樣可以講到對於她們來講是.
真的可以入到她們的心裡.
或者我跟你分享我們做社官.
或者做輪捨服務的做法.
我們的重點就是.
那個不是上而下.
不是說她是弱勢我們去幫她.
也不是說她沒有什麼我們要供應給她.
反而是一個平等.
她是我們社區的一部分.
我們本身是輪捨.
在輪捨過程當中.
有什麼是可以共享呢?.
因為她也是我們整個社區的共同參與者.
只不過她在她能力擁有的空間.
她沒有我們那麼多資源.
所以我們對她的互動過程當中.
她有物資我們可以分享.
比如我們有相關的.
一些可以支援的東西.
我們就讓她知道她要不要.
簡單來講就是.
如果過去有參與過平等分享行動.
一個社區輪捨的互動.
你就會發覺很多參與平等分享行動的人.
就會打開袋子讓她選擇內容.
而不是我覺得你需要藥油我就給你藥油.
而不是我覺得你需要高布我就給你高布.
你打開之後.
我們很多時候都會發現.
當我們做一些輪捨互動的時候.
打開的時候.
你會感受到一件事.

$^{2881}$她不是貪心.
她不是什麼都要.
她只是按她自己所需要的東西.
她只是拿她自己的一部分.
這個共享的文化.
對她來說.
就是我們能夠和她接點的好消息.
我不是硬要提出好消息.
正正就是她有些東西.
你覺得她需要她未必一定需要.
但你能夠和她共享.
那就是一個好消息.
所以可能要調整一下.
參與過程當中.
不是我們比較優越.
去向下服事.
而是和她平衡.
所以我自己在過去的日子.
和頂智妹一起去做.
我不斷和頂智妹分享.
我們落區做相關的參與.
我們只不過是.
恢復回耶穌基督的日常.
耶穌基督的日常就是周遊四方.
行善善 醫治各樣的病症.
見到有需要的人就伸出援手.
因此這個字就是.
屬靈果之第五個特質.
Kindness.
就是有需要的時候.
伸出援手.
就好像幫一幫她.
幫一幫她.
知道她有需要的時候.
我們能不能夠補充支援.
這對她來說就是好消息.
但剛才你分享的情況就是.
很多時候過去教會.
做這些社會服務.
或者社會參與的時候.

$^{2921}$就好像從上而下這樣做.
最後還要放一張單張進去.
甚至有個五色豬.
給她做手繩.
這些很多時候就是用福音.
打包這件事.
其實就曲解了參與.
這個我覺得就是滑蛇添足.
還有很多時候做的過程當中.
就太多一次性.
其實福音就是關係.
或者好消息是跟人接觸的關係.
或者瞭解到人的接點空間是重要的.
所以那個婆婆覺得反感是正常的.
因為好像要做完這件事.
才可以受惠於你.
其實是一件很煩厭的事.
各位你好.
我覺得耶穌有很多盼望.
好像香港市道差.
很難找到谷.
我覺得都是透過不同的心裡面去探望的.
做義工能夠得到很大的盼望的.
謝謝.
剛才說到.
剛才的問題.
我想說一點.
所以就說不要福音罐頭.
方單章的問題就是將福音變成了一段文字.
剛才說的只是玩玩而已.
不是要將我們那套東西變成文字.
這些都會變成罐頭.
所以我們的福音如果純粹變成了一個信息.
那就真的沒意思.
純粹加多一張單章下去.
那福音就不重要了.
所以對於那個婆婆來說.
怎麼能夠真正是一個好消息呢.
這個只能夠是你整個人投入.
你能夠在她面前帶給她.

$^{2961}$多過說成為一個純粹的.
弱化了的公式或者單章.
所以福音罐頭不會這麼做.
我們肯定不會派單章.
我們就是單章.
我們的關係就是單章.
我們的行動就是福音.
我想問有時這些社關.
可能你一路一路做.
就會去到一個瓶頸位.
就是我用不用再和他說一次福音呢.
如果我完全不提及福音.
就好像只是社關.
好像又不知如何是好.
但是如果你和他再提及福音.
會不會又回到以前的位置.
我只是傳福音給你.
人家會想你只是想我相信福音.
有時會面對瓶頸位.
可能久不久去到某個位置.
就會想要不要再和他說一次福音.
如果面對這些情況.
應該可以怎麼做呢.
我先把經驗分享一下.
通常你慢慢和他接觸一兩次之後.
他會對你有一個研究.
就會問你.
其實你沒有工作嗎.
為什麼你這麼有空呢.
你會告訴他.
其實我不是沒有工作.
我是特意抽時間做.
他會問你很有錢嗎.
為什麼你有這麼多物資呢.
我們會告訴他.
其實我們想資源重新分配.
有些人有更多資源和渠道的時候.
我們就願意和你分享.
他會問你很多問題.
為什麼你會這樣做.

$^{3001}$我們就會問他們.
就好像有人問你們心中盼望的緣由.
你就常常準備以溫柔敬畏的心回答他.
這是比特前書里的信息.
你刻意和他分享.
他會覺得分享完之後.
我就拿你的東西.
越打越挨.
我拿你的東西就聽你的.
但你不和他分享.
他就會有興趣問.
為什麼你們會這樣做.
因為這個世界不是這樣的.
但你就告訴他.
我們做的不是這個世界要做的事.
我們真的反世界而行的事.
他就會覺得.
你們是什麼人.
當然有些人很清楚教會的定位.
或者我們的做法.
他就會持續地拿.
他拿完就走.
他就會和你.
我又不會美化這件事.
他真的會和你無瓜葛.
他想快點完成這件事.
但是他知道這是一個渠道.
到時他會問.
或者我們也會面對的情況就是.
當我們見到他.
有些情況我們會問.
我們可不可以為你祈禱.
或者有什麼你想分享.
我們可以聽你.
其實人的心.
在當時就會有個觸動.
最後那個slide裡面說.
這個福音是上帝的大能.
我相信在福音的契機當中.
適時就會有那個工作.

$^{3041}$這個不是很遠的事.
但是最重點就是.
不是一次性可以見到一件事.
是我們遲之後可能成為我們的日常.
成為我們的生活態度.
這個好的信息就慢慢.
讓人可以參與.
至於會不會成為一個樽頸.
我覺得未必.
但是他想問我們相關的內容的時候.
我們都會告訴他.
其實福音是什麼.
或者我們的信仰是什麼.
可能都要幫他重新瞭解.
到時他接受的程度.
或者理解的範圍會多一些.
這個很重要.
我覺得.
什麼叫一次福音.
一次福音其實是很奇怪的東西.
一次福音是頭尾整個故事講出來.
教是可以教的一次福音.
不是一個定義.
但是我就不是這麼相信.
我不是聽一次福音這麼相信.
可能我上教會的時候.
很溫暖.
看一段聖經.
一起祈禱.
一群人這樣生活.
那個東西是一個累積而成的東西.
通常聽一次福音已經相信了.
重新聽一次.
整個原理版本.
但首先他相信了.
想聽一次的意思.
對他來說.
福音其實是一個.
不斷地慢慢滲入.
那個見證.

$^{3081}$所以我都不需要.
有些人很反對.
做社會肯定不能說福音.
其實不需要那麼硬來.
說還是不說.
因為福音從來都不是一次半次.
講出來的東西.
就是你跟他說.
跟他祈禱.
跟他說耶穌.
這個人.
這個事情.
加起來就是你整個的見證.
所以我覺得不是一次福音.
還是兩次福音的問題.
福音不是一次兩次.
這樣講出來的故事.
不是單章.
所以我們所傳揚的.
正正就是一個你行動裡面.
不斷來到他.
累積出來的關係和見證.
所以當你想到一次福音時.
已經是.
一個故事形式的福音.
其實就不需要一定這樣.
想起上課是需要講一次給他聽的.
但這個已經不是.
信耶穌那件事了.
聖經裡面用.
撒種來處理那種.
多次性的情況.
撒種裡面的教導就說.
那個種子需要時間去.
醞釀發芽.
所以仍然是講福音.
是那種關係性.
如何讓他理解.
生命可以有另一個出口.
或者生命有另一個轉態.

$^{3121}$這個是關鍵.
有時候我不懂得如何去.
向我的朋友解釋.
我的朋友可能會問我.
如果這個神真的那麼.
外來那麼厲害的話.
為什麼他還會容許.
那麼不公義的事情發生呢.
為什麼好像連一些聲音.
都不容許呢.
我知道.
之前潘Sir可能會說過.
聖經其實是一個人生的菜單.
很多東西都可以在那裡找到.
是一個人生的指南.
但是不信的人.
又如何去聽我引經據典.
去講這些呢.
我如何貼地地.
去和他解釋這個好消息呢.
我都說了.
你要知道.
其實神是做了事情.
又會完全讓你見到.
祂做了的事情.
所以我們說.
怎麼說呢.
對於上帝.
如何去看待世界呢.
我經常覺得世界的開始和終結.
是一個很永恆的小息.
永恆到永恆中間.
一個很小的小息.
這就是世界歷史.
所以上帝不是不做事.
而是對人的短暫時間.
就是一個所謂的黑暗勢力.
不好的東西.
但是對上帝來說.
是一個很微不足道的事情.

$^{3161}$當然我們覺得很不容易.
也很難過.
因為時間很長.
但是對上帝來說.
這是一個很輕而易舉.
能夠解決的事情.
我們叫做不可能的可能性.
其實祂在神面前.
已經不存在.
都說祂已經死了.
但是對我們來說.
好像是很漫長.
所以上帝不是不做事.
其實上帝已經做了事.
對我們人的眼光來說.
仍然是一個很微小的角度.
當然是很不容易過渡的.
我這樣理解.
當然我們可以想象.
一個能夠即時破滅.
殺死所有的東西的神.
其實這是我們人所想象.
一個比較弱的超級英雄.
能夠做到的事情.
但是上帝是終極地解決問題.
所以我們有時要有眼光和軟度.
來看世界的黑暗.
我會這樣去說.
我曾經在《射傑美》的時候.
說過一篇信息.
都是說關於.
那些人很期望上帝.
即時做出審判或整治.
我當時其中一個解讀是.
其實我們人很短視和心急.
很想那些壞人在你面前撲倒.
但是你會發覺.
很多時候都沒有那麼.
天氣不似預期.
但是最後.

$^{3201}$是否因為不能滿足你的期望.
或者你的時間表.
你就覺得上帝不正.
或者不厲害呢.
我會回到剛才所說的.
上帝是在做事.
和上帝也在做事.
就如剛才John所說.
因為上帝在做事.
因為有審判.
上帝就在看這群人不斷發生什麼事.
上帝是在做事.
因為之前都會有歷史.
重復這些類近的情況.
但是當你回顧歷史的時候.
上帝其實已經對那些人.
有那時候的懲治.
只不過人在現在的進行當中.
看不到上帝工作的時刻.
這個情況.
所以我記得我和兩個兒子.
在家裡看電視的時候.
我的兒子就說.
餵,他們又說謊了.
然後我說是.
他說,傳導人你怎麼看這些事.
然後,那時候是七八月的時候.
19年.
每次四點鐘的時候都說謊.
然後我看著.
我說,我的答案就是.
我和他說.
你看多少,上帝也看多少.
在聖經的角度來說.
他們所做的.
是在上帝面前.
即使上帝對他們的憤怒.
這個是聖經說得很清楚.
所以對於上帝正在做事.
或者上帝將會做事.

$^{3241}$那個是將來式.
但仍然是現在進行.
但不是按我們的劇本.
或者按我們的時間.
這個在彼得前書也在說.
就是上帝不願意人沈淪.
願意萬人悔改.
這個在上帝計劃當中有時間表.
所以回應你朋友提問的時候.
我們會有些無奈.
因為上帝不是我們說.
要做什麼就做什麼.
但我們看回過去歷史和聖經的教導.
其實上帝是過去做事.
現在做事.
將來也會做事.
這個就是我們所相信.
那個有真有活的上帝.
我想問一個問題.
這個很宏觀.
關於世界的一個好消息.
其實和神愛世人.
這個概念有沒有些抵觸呢.
因為以往那個罐頭福音.
其實是很直接.
人犯罪.
然後信耶穌就上天堂.
神愛世人.
你信了就上天堂.
這個很直接.
所以比較受港豬歡迎.
但是這套宏觀的好消息.
其實是講世界觀.
去到最後.
第五隻豬.
其實是.
感覺像是死得人多.
其實如何可以.
與一些未信主的人聯繫.
去聽的時候.

$^{3281}$會覺得這個信仰.
是和自己有關係.
或者將來.
釘了之後有些關係.
可以如何去講.
去到最後的時候.
其實神愛世人的原文.
其實是什麼.
如果你看過.
就是什麼.
就是Cosmos.
就是神愛世界.
所以其實.
神愛世界是不足夠的.
在中文翻譯裡面.
其實神不單單愛世人.
更加不是愛基督徒.
而是愛整個世界所有的東西.
所以上帝.
在《藥王方經》中所講的愛.
其實是更加.
闊於我們所想的.
不是說我們有罪.
然後就能夠得救.
或者我們信仰得救.
上帝的愛是關乎於.
整個世界.
能夠得到改變.
所以基督徒環保也有關係.
因為我們所講的.
那個世界將會被更新.
所以這件事反而.
更加闊於我們所理解.
所以上帝愛這個世界的時候.
其實不是說熟悉這個世界.
而是很愛這個世界.
更新這個世界.
所以上帝的福音是說.
這個世界的歷史和改變.
多於純粹人的事情.

$^{3321}$所以上帝不單單愛.
一些信許人.
上帝愛所有的人.
所以這個所謂的福音.
是關乎於全人類.
雖然你可以不相信.
但不代表上帝不愛你.
上帝的愛的救贖.
也包括在一些.
所謂不相信這個計劃的人身上.
所以我覺得反而.
是闊於我們以前所理解的.
神話世人.
一個月才來一次.
你會點什麼吃呢?.
還沒點什麼吃.
過了幾個小時都沒來.
那就要等下一次才點.
我已經關門了.
好啊.
到下個月才見.
下次是什麼呢?.
我先想想.
下次好像是講教會.
那就是要叫多些不是教會的人來.
還是叫多些教會的人來呢?.
我真的要多點位置.
好啊.
叫多些人來可以試下一次.
那就下個月見了.
拜拜.
《香港》 作詞:陳汝佳 作曲:陳汝佳.
香港 我心中的國鄉.
這裡讓我生長.
有我喜歡的親友共陽光.
路上人在跑 過他港.
感驚為我欣賞.
這裡有許多好處沒發覺.
食一聲香港 香港.
你永遠是塵埃香.

$^{3361}$香港 香港.
你那色調那望.
山頂看小島水淚淌.
處處換上新裝.
看看那海鷗飛過自由港.
海邊看小島處萬丈.
處處搖曳新光.
這個市區的吸引沒法擋.
食一聲香港 香港.
再有我童年夢想.
香港 香港.
叫我不以為望.
香港 我心中的故鄉.
這裡讓我生長.
有我喜歡的親友共陽光.
路上人在跑 過他港.
感驚為我欣賞.
這裡有許多好處沒發覺.
食一聲香港 香港.
你永遠是塵埃香.
香港 香港.
你那色調那望.
香港 香港.
再有我童年夢想.
香港 香港.
QQ我吧.
\newpage



\section{}
\label{sec:dNWjC8vnhS0}
\textbf{【這是最好的時代:給香港基督徒的神學八課】第3課: Let’s flow|20210726 [dNWjC8vnhS0]}
\newline
\newline
連結: \href{https://youtube.com/watch?v=dNWjC8vnhS0}{\texttt{ https://youtube.com/watch?v=dNWjC8vnhS0}} ~~~~ 語音日期: 2021-07-26 
\newline
\newline
\hyperref[sec:Gv9ZCQJGqmE]{\small{< < < PREV SERMON < < <}}
~
\hyperref[sec:index_chronic]{\small{[返順時目]}}
~
\hyperref[sec:index_scriptual]{\small{[返順卷目]}}
~
\hyperref[sec:Gprv_Nw0Oi4]{\small{> > > NEXT SERMON > > >}}
\newline
\newline
$^{1}$我只想知道.
你到底是什麼意思.
我只想知道.
你到底是什麼意思.
我只想知道.
你到底是什麼意思.
我只想知道.
你到底是什麼意思.
我只想知道.
你到底是什麼意思.
我只想知道.
你到底是什麼意思.
我只想知道.
你到底是什麼意思.
我只想知道.
你到底是什麼意思.
我只想知道.
你到底是什麼意思.
我只想知道.
你到底是什麼意思.
我只想知道.
你到底是什麼意思.
我只想知道.
你到底是什麼意思.
我只想知道.
你到底是什麼意思.
我只想知道.
你到底是什麼意思.
我只想知道.
你到底是什麼意思.
我只想知道.
你到底是什麼意思.
我只想知道.
你到底是什麼意思.
我只想知道.
你到底是什麼意思.
我只想知道.
你到底是什麼意思.
我只想知道.
你到底是什麼意思.

$^{41}$我只想知道.
你到底是什麼意思.
我只想知道.
你到底是什麼意思.
我只想知道.
你到底是什麼意思.
我只想知道.
你到底是什麼意思.
我只想知道.
你到底是什麼意思.
我只想知道.
你到底是什麼意思.
我只想知道.
你到底是什麼意思.
我只想知道.
你到底是什麼意思.
我只想知道.
你到底是什麼意思.
我只想知道.
你到底是什麼意思.
我只想知道.
你到底是什麼意思.
我只想知道.
你到底是什麼意思.
我只想知道.
你到底是什麼意思.
我只想知道.
你到底是什麼意思.
我只想知道.
你到底是什麼意思.
我只想知道.
你到底是什麼意思.
我只想知道.
你到底是什麼意思.
我只想知道.
你到底是什麼意思.
我只想知道.
你到底是什麼意思.
我只想知道.
你到底是什麼意思.

$^{81}$我只想知道.
你到底是什麼意思.
我只想知道.
你到底是什麼意思.
我只想知道.
你到底是什麼意思.
我只想知道.
你到底是什麼意思.
我只想知道.
你到底是什麼意思.
我只想知道.
你到底是什麼意思.
我只想知道.
你到底是什麼意思.
我只想知道.
你到底是什麼意思.
我只想知道.
你到底是什麼意思.
我只想知道.
你到底是什麼意思.
我只想知道.
你到底是什麼意思.
我只想知道.
你到底是什麼意思.
我只想知道.
你到底是什麼意思.
我只想知道.
你到底是什麼意思.
我只想知道.
你到底是什麼意思.
我只想知道.
你到底是什麼意思.
我只想知道.
你到底是什麼意思.
我只想知道.
你到底是什麼意思.
我只想知道.
你到底是什麼意思.
我只想知道.
你到底是什麼意思.

$^{121}$我只想知道.
你到底是什麼意思.
我只想知道.
你到底是什麼意思.
我只想知道.
你到底是什麼意思.
我只想知道.
你到底是什麼意思.
我只想知道.
你到底是什麼意思.
我只想知道.
你到底是什麼意思.
我只想知道.
你到底是什麼意思.
我只想知道.
你到底是什麼意思.
我只想知道.
你到底是什麼意思.
我只想知道.
你到底是什麼意思.
我只想知道.
你到底是什麼意思.
我只想知道.
你到底是什麼意思.
我只想知道.
你到底是什麼意思.
我只想知道.
你到底是什麼意思.
我只想知道.
你到底是什麼意思.
我只想知道.
你到底是什麼意思.
我只想知道.
你到底是什麼意思.
我只想知道.
你到底是什麼意思.
我只想知道.
你到底是什麼意思.
我只想知道.
你到底是什麼意思.

$^{161}$我只想知道.
你到底是什麼意思.
我只想知道.
你到底是什麼意思.
我只想知道.
你到底是什麼意思.
我只想知道.
你到底是什麼意思.
我只想知道.
你到底是什麼意思.
我只想知道.
你到底是什麼意思.
我只想知道.
你到底是什麼意思.
我只想知道.
你到底是什麼意思.
我只想知道.
你到底是什麼意思.
我只想知道.
你到底是什麼意思.
我只想知道.
你到底是什麼意思.
我只想知道.
你到底是什麼意思.
我只想知道.
你到底是什麼意思.
我只想知道.
你到底是什麼意思.
我只想知道.
你到底是什麼意思.
我只想知道.
你到底是什麼意思.
我只想知道.
你到底是什麼意思.
我只想知道.
你到底是什麼意思.
我只想知道.
你到底是什麼意思.
我只想知道.
你到底是什麼意思.

$^{201}$我只想知道.
你到底是什麼意思.
我只想知道.
你到底是什麼意思.
我只想知道.
你到底是什麼意思.
我只想知道.
你到底是什麼意思.
我只想知道.
你到底是什麼意思.
我只想知道.
你到底是什麼意思.
我只想知道.
你到底是什麼意思.
我只想知道.
你到底是什麼意思.
我只想知道.
你到底是什麼意思.
我只想知道.
你到底是什麼意思.
我只想知道.
你到底是什麼意思.
我只想知道.
你到底是什麼意思.
我只想知道.
你到底是什麼意思.
我只想知道.
你到底是什麼意思.
我只想知道.
你到底是什麼意思.
我只想知道.
你到底是什麼意思.
我只想知道.
你到底是什麼意思.
我只想知道.
你到底是什麼意思.
我只想知道.
你到底是什麼意思.
我只想知道.
你到底是什麼意思.

$^{241}$我只想知道.
你到底是什麼意思.
我只想知道.
你到底是什麼意思.
我只想知道.
你到底是什麼意思.
我只想知道.
你到底是什麼意思.
我只想知道.
你到底是什麼意思.
我只想知道.
你到底是什麼意思.
我只想知道.
你到底是什麼意思.
我只想知道.
你到底是什麼意思.
我只想知道.
你到底是什麼意思.
我只想知道.
你到底是什麼意思.
我只想知道.
你到底是什麼意思.
我只想知道.
你到底是什麼意思.
我只想知道.
你到底是什麼意思.
我只想知道.
你到底是什麼意思.
我只想知道.
你到底是什麼意思.
我只想知道.
你到底是什麼意思.
我只想知道.
你到底是什麼意思.
我只想知道.
你到底是什麼意思.
我只想知道.
你到底是什麼意思.
我只想知道.
你到底是什麼意思.

$^{281}$我只想知道.
你到底是什麼意思.
我只想知道.
你到底是什麼意思.
我只想知道.
你到底是什麼意思.
我只想知道.
你到底是什麼意思.
我只想知道.
你到底是什麼意思.
我只想知道.
你到底是什麼意思.
我只想知道.
你到底是什麼意思.
我只想知道.
你到底是什麼意思.
我只想知道.
你到底是什麼意思.
我只想知道.
你到底是什麼意思.
我只想知道.
你到底是什麼意思.
我只想知道.
你到底是什麼意思.
我只想知道.
你到底是什麼意思.
我只想知道.
你到底是什麼意思.
我只想知道.
你到底是什麼意思.
我只想知道.
你到底是什麼意思.
我只想知道.
你到底是什麼意思.
我只想知道.
你到底是什麼意思.
我只想知道.
你到底是什麼意思.
我只想知道.
你到底是什麼意思.

$^{321}$我只想知道.
你到底是什麼意思.
我只想知道.
你到底是什麼意思.
我只想知道.
你到底是什麼意思.
我只想知道.
你到底是什麼意思.
我只想知道.
你到底是什麼意思.
我只想知道.
你到底是什麼意思.
我只想知道.
你到底是什麼意思.
我只想知道.
你到底是什麼意思.
我只想知道.
你到底是什麼意思.
我只想知道.
你到底是什麼意思.
我只想知道.
你到底是什麼意思.
我只想知道.
你到底是什麼意思.
我只想知道.
你到底是什麼意思.
我只想知道.
你到底是什麼意思.
我只想知道.
你到底是什麼意思.
我只想知道.
你到底是什麼意思.
我只想知道.
你到底是什麼意思.
我只想知道.
你到底是什麼意思.
我只想知道.
你到底是什麼意思.
我只想知道.
你到底是什麼意思.

$^{361}$我只想知道.
你到底是什麼意思.
我只想知道.
你到底是什麼意思.
我只想知道.
你到底是什麼意思.
我只想知道.
你到底是什麼意思.
我只想知道.
你到底是什麼意思.
我只想知道.
你到底是什麼意思.
我只想知道.
你到底是什麼意思.
我只想知道.
你到底是什麼意思.
我只想知道.
你到底是什麼意思.
我只想知道.
你到底是什麼意思.
我只想知道.
你到底是什麼意思.
我只想知道.
你到底是什麼意思.
我只想知道.
你到底是什麼意思.
我只想知道.
你到底是什麼意思.
我只想知道.
你到底是什麼意思.
我只想知道.
你到底是什麼意思.
我只想知道.
你到底是什麼意思.
我只想知道.
你到底是什麼意思.
我只想知道.
你到底是什麼意思.
我只想知道.
你到底是什麼意思.

$^{401}$我只想知道.
你到底是什麼意思.
我只想知道.
你到底是什麼意思.
我只想知道.
你到底是什麼意思.
我只想知道.
你到底是什麼意思.
我只想知道.
你到底是什麼意思.
我只想知道.
你到底是什麼意思.
我只想知道.
你到底是什麼意思.
我只想知道.
你到底是什麼意思.
我只想知道.
你到底是什麼意思.
我只想知道.
你到底是什麼意思.
我只想知道.
你到底是什麼意思.
我只想知道.
你到底是什麼意思.
我只想知道.
你到底是什麼意思.
我只想知道.
你到底是什麼意思.
我只想知道.
你到底是什麼意思.
我只想知道.
你到底是什麼意思.
我只想知道.
你到底是什麼意思.
我只想知道.
你到底是什麼意思.
我只想知道.
你到底是什麼意思.
我只想知道.
你到底是什麼意思.

$^{441}$我只想知道.
你到底是什麼意思.
我只想知道.
你到底是什麼意思.
我只想知道.
你到底是什麼意思.
我只想知道.
你到底是什麼意思.
我只想知道.
你到底是什麼意思.
我只想知道.
你到底是什麼意思.
我只想知道.
你到底是什麼意思.
我只想知道.
你到底是什麼意思.
我只想知道.
你到底是什麼意思.
我只想知道.
你到底是什麼意思.
我只想知道.
你到底是什麼意思.
我只想知道.
你到底是什麼意思.
我只想知道.
你到底是什麼意思.
我只想知道.
你到底是什麼意思.
我只想知道.
你到底是什麼意思.
我只想知道.
你到底是什麼意思.
我只想知道.
你到底是什麼意思.
我只想知道.
你到底是什麼意思.
我只想知道.
你到底是什麼意思.
我只想知道.
你到底是什麼意思.

$^{481}$我只想知道.
你到底是什麼意思.
我只想知道.
你到底是什麼意思.
我只想知道.
你到底是什麼意思.
我只想知道.
你到底是什麼意思.
我只想知道.
你到底是什麼意思.
我只想知道.
你到底是什麼意思.
我只想知道.
你到底是什麼意思.
我只想知道.
你到底是什麼意思.
我只想知道.
你到底是什麼意思.
我只想知道.
你到底是什麼意思.
我只想知道.
你到底是什麼意思.
我只想知道.
你到底是什麼意思.
我只想知道.
你到底是什麼意思.
我只想知道.
你到底是什麼意思.
我只想知道.
你到底是什麼意思.
我只想知道.
你到底是什麼意思.
我只想知道.
你到底是什麼意思.
我只想知道.
你到底是什麼意思.
我只想知道.
你到底是什麼意思.
我只想知道.
你到底是什麼意思.

$^{521}$我只想知道.
你到底是什麼意思.
我只想知道.
你到底是什麼意思.
我只想知道.
你到底是什麼意思.
我只想知道.
你到底是什麼意思.
我只想知道.
你到底是什麼意思.
我只想知道.
你到底是什麼意思.
我只想知道.
你到底是什麼意思.
我只想知道.
你到底是什麼意思.
我只想知道.
你到底是什麼意思.
我只想知道.
你到底是什麼意思.
我只想知道.
你到底是什麼意思.
我只想知道.
你到底是什麼意思.
我只想知道.
你到底是什麼意思.
我只想知道.
你到底是什麼意思.
我只想知道.
你到底是什麼意思.
我只想知道.
你到底是什麼意思.
我只想知道.
你到底是什麼意思.
我只想知道.
你到底是什麼意思.
我只想知道.
你到底是什麼意思.
我只想知道.
你到底是什麼意思.

$^{561}$我只想知道.
你到底是什麼意思.
我只想知道.
你到底是什麼意思.
我只想知道.
你到底是什麼意思.
我只想知道.
你到底是什麼意思.
我只想知道.
你到底是什麼意思.
我只想知道.
你到底是什麼意思.
我只想知道.
你到底是什麼意思.
我只想知道.
你到底是什麼意思.
我只想知道.
你到底是什麼意思.
我只想知道.
你到底是什麼意思.
我只想知道.
你到底是什麼意思.
我只想知道.
你到底是什麼意思.
我只想知道.
你到底是什麼意思.
我只想知道.
你到底是什麼意思.
我只想知道.
你到底是什麼意思.
我只想知道.
你到底是什麼意思.
我只想知道.
你到底是什麼意思.
我只想知道.
你到底是什麼意思.
我只想知道.
你到底是什麼意思.
我只想知道.
你到底是什麼意思.

$^{601}$我只想知道.
你到底是什麼意思.
我只想知道.
你到底是什麼意思.
我只想知道.
你到底是什麼意思.
我只想知道.
你到底是什麼意思.
我只想知道.
你到底是什麼意思.
我只想知道.
你到底是什麼意思.
我只想知道.
你到底是什麼意思.
我只想知道.
你到底是什麼意思.
我只想知道.
你到底是什麼意思.
我只想知道.
你到底是什麼意思.
我只想知道.
你到底是什麼意思.
我只想知道.
你到底是什麼意思.
我只想知道.
你到底是什麼意思.
我只想知道.
你到底是什麼意思.
我只想知道.
你到底是什麼意思.
我只想知道.
你到底是什麼意思.
我只想知道.
你到底是什麼意思.
我只想知道.
你到底是什麼意思.
我只想知道.
你到底是什麼意思.
我只想知道.
你到底是什麼意思.

$^{641}$我只想知道.
你到底是什麼意思.
我只想知道.
你到底是什麼意思.
我只想知道.
你到底是什麼意思.
我只想知道.
你到底是什麼意思.
我只想知道.
你到底是什麼意思.
我只想知道.
你到底是什麼意思.
我只想知道.
你到底是什麼意思.
我只想知道.
你到底是什麼意思.
我只想知道.
你到底是什麼意思.
我只想知道.
你到底是什麼意思.
我只想知道.
你到底是什麼意思.
我只想知道.
你到底是什麼意思.
我只想知道.
你到底是什麼意思.
我只想知道.
你到底是什麼意思.
我只想知道.
你到底是什麼意思.
我只想知道.
你到底是什麼意思.
我只想知道.
你到底是什麼意思.
我只想知道.
你到底是什麼意思.
我只想知道.
你到底是什麼意思.
我只想知道.
你到底是什麼意思.

$^{681}$我只想知道.
你到底是什麼意思.
我只想知道.
你到底是什麼意思.
我只想知道.
你到底是什麼意思.
我只想知道.
你到底是什麼意思.
我只想知道.
你到底是什麼意思.
我只想知道.
你到底是什麼意思.
我只想知道.
你到底是什麼意思.
我只想知道.
你到底是什麼意思.
我只想知道.
你到底是什麼意思.
我只想知道.
你到底是什麼意思.
我只想知道.
你到底是什麼意思.
我只想知道.
你到底是什麼意思.
我只想知道.
你到底是什麼意思.
我只想知道.
你到底是什麼意思.
我只想知道.
你到底是什麼意思.
我只想知道.
你到底是什麼意思.
我只想知道.
你到底是什麼意思.
我只想知道.
你到底是什麼意思.
我只想知道.
你到底是什麼意思.
我只想知道.
你到底是什麼意思.

$^{721}$我只想知道.
你到底是什麼意思.
我只想知道.
你到底是什麼意思.
我只想知道.
你到底是什麼意思.
我只想知道.
你到底是什麼意思.
我只想知道.
你到底是什麼意思.
我只想知道.
你到底是什麼意思.
我只想知道.
你到底是什麼意思.
我只想知道.
你到底是什麼意思.
我只想知道.
你到底是什麼意思.
我只想知道.
你到底是什麼意思.
我只想知道.
你到底是什麼意思.
我只想知道.
你到底是什麼意思.
我只想知道.
你到底是什麼意思.
我只想知道.
你到底是什麼意思.
我只想知道.
你到底是什麼意思.
我只想知道.
你到底是什麼意思.
我只想知道.
你到底是什麼意思.
我只想知道.
你到底是什麼意思.
我只想知道.
你到底是什麼意思.
我只想知道.
你到底是什麼意思.

$^{761}$我只想知道.
你到底是什麼意思.
我只想知道.
你到底是什麼意思.
我只想知道.
你到底是什麼意思.
我只想知道.
你到底是什麼意思.
我只想知道.
你到底是什麼意思.
我只想知道.
你到底是什麼意思.
我只想知道.
你到底是什麼意思.
我只想知道.
你到底是什麼意思.
我只想知道.
你到底是什麼意思.
我只想知道.
你到底是什麼意思.
我只想知道.
你到底是什麼意思.
我只想知道.
你到底是什麼意思.
我只想知道.
你到底是什麼意思.
我只想知道.
你到底是什麼意思.
我只想知道.
你到底是什麼意思.
我只想知道.
你到底是什麼意思.
我只想知道.
你到底是什麼意思.
我只想知道.
你到底是什麼意思.
我只想知道.
你到底是什麼意思.
我只想知道.
你到底是什麼意思.

$^{801}$我只想知道.
你到底是什麼意思.
我只想知道.
你到底是什麼意思.
我只想知道.
你到底是什麼意思.
我只想知道.
你到底是什麼意思.
我只想知道.
你到底是什麼意思.
我只想知道.
你到底是什麼意思.
我只想知道.
你到底是什麼意思.
我只想知道.
你到底是什麼意思.
我只想知道.
你到底是什麼意思.
我只想知道.
你到底是什麼意思.
我只想知道.
你到底是什麼意思.
我只想知道.
你到底是什麼意思.
我只想知道.
你到底是什麼意思.
我只想知道.
你到底是什麼意思.
我只想知道.
你到底是什麼意思.
我只想知道.
你到底是什麼意思.
我只想知道.
你到底是什麼意思.
我只想知道.
你到底是什麼意思.
我只想知道.
你到底是什麼意思.
我只想知道.
你到底是什麼意思.

$^{841}$我只想知道.
你到底是什麼意思.
我只想知道.
你到底是什麼意思.
我只想知道.
你到底是什麼意思.
我只想知道.
你到底是什麼意思.
我只想知道.
你到底是什麼意思.
我只想知道.
你到底是什麼意思.
我只想知道.
你到底是什麼意思.
我只想知道.
你到底是什麼意思.
我只想知道.
你到底是什麼意思.
我只想知道.
你到底是什麼意思.
我只想知道.
你到底是什麼意思.
我只想知道.
你到底是什麼意思.
我只想知道.
你到底是什麼意思.
我只想知道.
你到底是什麼意思.
我只想知道.
你到底是什麼意思.
我只想知道.
你到底是什麼意思.
我只想知道.
你到底是什麼意思.
我只想知道.
你到底是什麼意思.
我只想知道.
你到底是什麼意思.
我只想知道.
你到底是什麼意思.

$^{881}$我只想知道.
你到底是什麼意思.
我只想知道.
你到底是什麼意思.
我只想知道.
你到底是什麼意思.
我只想知道.
你到底是什麼意思.
我只想知道.
你到底是什麼意思.
我只想知道.
你到底是什麼意思.
我只想知道.
你到底是什麼意思.
我只想知道.
你到底是什麼意思.
我只想知道.
你到底是什麼意思.
我只想知道.
你到底是什麼意思.
我只想知道.
你到底是什麼意思.
我只想知道.
你到底是什麼意思.
我只想知道.
你到底是什麼意思.
我只想知道.
你到底是什麼意思.
我只想知道.
你到底是什麼意思.
我只想知道.
你到底是什麼意思.
我只想知道.
你到底是什麼意思.
我只想知道.
你到底是什麼意思.
我只想知道.
你到底是什麼意思.
我只想知道.
你到底是什麼意思.

$^{921}$我只想知道.
你到底是什麼意思.
我只想知道.
你到底是什麼意思.
我只想知道.
你到底是什麼意思.
我只想知道.
你到底是什麼意思.
我只想知道.
你到底是什麼意思.
我只想知道.
你到底是什麼意思.
我只想知道.
你到底是什麼意思.
我只想知道.
你到底是什麼意思.
我只想知道.
你到底是什麼意思.
我只想知道.
你到底是什麼意思.
我只想知道.
你到底是什麼意思.
我只想知道.
你到底是什麼意思.
我只想知道.
你到底是什麼意思.
我只想知道.
你到底是什麼意思.
我只想知道.
你到底是什麼意思.
我只想知道.
你到底是什麼意思.
我只想知道.
你到底是什麼意思.
我只想知道.
你到底是什麼意思.
我只想知道.
你到底是什麼意思.
我只想知道.
你到底是什麼意思.

$^{961}$我只想知道.
你到底是什麼意思.
我只想知道.
你到底是什麼意思.
我只想知道.
你到底是什麼意思.
我只想知道.
你到底是什麼意思.
我只想知道.
你到底是什麼意思.
我只想知道.
你到底是什麼意思.
我只想知道.
你到底是什麼意思.
我只想知道.
你到底是什麼意思.
我只想知道.
你到底是什麼意思.
我只想知道.
你到底是什麼意思.
我只想知道.
你到底是什麼意思.
我只想知道.
你到底是什麼意思.
我只想知道.
你到底是什麼意思.
我只想知道.
你到底是什麼意思.
我只想知道.
你到底是什麼意思.
我只想知道.
你到底是什麼意思.
我只想知道.
你到底是什麼意思.
我只想知道.
你到底是什麼意思.
我只想知道.
你到底是什麼意思.
我只想知道.
你到底是什麼意思.

$^{1001}$我只想知道.
你到底是什麼意思.
我只想知道.
你到底是什麼意思.
我只想知道.
你到底是什麼意思.
我只想知道.
你到底是什麼意思.
我只想知道.
你到底是什麼意思.
我只想知道.
你到底是什麼意思.
我只想知道.
你到底是什麼意思.
我只想知道.
你到底是什麼意思.
我只想知道.
你到底是什麼意思.
我只想知道.
你到底是什麼意思.
我只想知道.
你到底是什麼意思.
我只想知道.
你到底是什麼意思.
我只想知道.
你到底是什麼意思.
我只想知道.
你到底是什麼意思.
我只想知道.
你到底是什麼意思.
我只想知道.
你到底是什麼意思.
我只想知道.
你到底是什麼意思.
我只想知道.
你到底是什麼意思.
我只想知道.
你到底是什麼意思.
我只想知道.
你到底是什麼意思.

$^{1041}$我只想知道.
你到底是什麼意思.
我只想知道.
你到底是什麼意思.
我只想知道.
你到底是什麼意思.
我只想知道.
你到底是什麼意思.
我只想知道.
你到底是什麼意思.
我只想知道.
你到底是什麼意思.
我只想知道.
你到底是什麼意思.
我只想知道.
你到底是什麼意思.
我只想知道.
你到底是什麼意思.
我只想知道.
你到底是什麼意思.
我只想知道.
你到底是什麼意思.
我只想知道.
你到底是什麼意思.
我只想知道.
你到底是什麼意思.
我只想知道.
你到底是什麼意思.
我只想知道.
你到底是什麼意思.
我只想知道.
你到底是什麼意思.
我只想知道.
你到底是什麼意思.
我只想知道.
你到底是什麼意思.
我只想知道.
你到底是什麼意思.
我只想知道.
你到底是什麼意思.

$^{1081}$我只想知道.
你到底是什麼意思.
我只想知道.
你到底是什麼意思.
我只想知道.
你到底是什麼意思.
我只想知道.
你到底是什麼意思.
我只想知道.
你到底是什麼意思.
我只想知道.
你到底是什麼意思.
我只想知道.
你到底是什麼意思.
我只想知道.
你到底是什麼意思.
我只想知道.
你到底是什麼意思.
我只想知道.
你到底是什麼意思.
我只想知道.
你到底是什麼意思.
我只想知道.
你到底是什麼意思.
我只想知道.
你到底是什麼意思.
我只想知道.
你到底是什麼意思.
我只想知道.
你到底是什麼意思.
我只想知道.
你到底是什麼意思.
我只想知道.
你到底是什麼意思.
我只想知道.
你到底是什麼意思.
我只想知道.
你到底是什麼意思.
我只想知道.
你到底是什麼意思.

$^{1121}$我只想知道.
你到底是什麼意思.
我只想知道.
你到底是什麼意思.
我只想知道.
你到底是什麼意思.
我只想知道.
你到底是什麼意思.
我只想知道.
你到底是什麼意思.
我只想知道.
你到底是什麼意思.
我只想知道.
你到底是什麼意思.
我只想知道.
你到底是什麼意思.
我只想知道.
你到底是什麼意思.
我只想知道.
你到底是什麼意思.
我只想知道.
你到底是什麼意思.
我只想知道.
你到底是什麼意思.
我只想知道.
你到底是什麼意思.
我只想知道.
你到底是什麼意思.
我只想知道.
你到底是什麼意思.
我只想知道.
你到底是什麼意思.
我只想知道.
你到底是什麼意思.
我只想知道.
你到底是什麼意思.
我只想知道.
你到底是什麼意思.
我只想知道.
你到底是什麼意思.

$^{1161}$我只想知道.
你到底是什麼意思.
我只想知道.
你到底是什麼意思.
我只想知道.
你到底是什麼意思.
我只想知道.
你到底是什麼意思.
我只想知道.
你到底是什麼意思.
我只想知道.
你到底是什麼意思.
我只想知道.
你到底是什麼意思.
我只想知道.
你到底是什麼意思.
我只想知道.
你到底是什麼意思.
我只想知道.
你到底是什麼意思.
我只想知道.
你到底是什麼意思.
我只想知道.
你到底是什麼意思.
我只想知道.
你到底是什麼意思.
我只想知道.
你到底是什麼意思.
我只想知道.
你到底是什麼意思.
我只想知道.
你到底是什麼意思.
我只想知道.
你到底是什麼意思.
我只想知道.
你到底是什麼意思.
我只想知道.
你到底是什麼意思.
我只想知道.
你到底是什麼意思.

$^{1201}$我只想知道.
你到底是什麼意思.
我只想知道.
你到底是什麼意思.
我只想知道.
你到底是什麼意思.
我只想知道.
你到底是什麼意思.
我只想知道.
你到底是什麼意思.
我只想知道.
你到底是什麼意思.
我只想知道.
你到底是什麼意思.
我只想知道.
你到底是什麼意思.
我只想知道.
你到底是什麼意思.
我只想知道.
你到底是什麼意思.
我只想知道.
你到底是什麼意思.
我只想知道.
你到底是什麼意思.
我只想知道.
你到底是什麼意思.
我只想知道.
你到底是什麼意思.
我只想知道.
你到底是什麼意思.
我只想知道.
你到底是什麼意思.
我只想知道.
你到底是什麼意思.
我只想知道.
你到底是什麼意思.
我只想知道.
你到底是什麼意思.
我只想知道.
你到底是什麼意思.

$^{1241}$我只想知道.
你到底是什麼意思.
我只想知道.
你到底是什麼意思.
我只想知道.
你到底是什麼意思.
我只想知道.
你到底是什麼意思.
我只想知道.
你到底是什麼意思.
我只想知道.
你到底是什麼意思.
我只想知道.
你到底是什麼意思.
我只想知道.
你到底是什麼意思.
我只想知道.
你到底是什麼意思.
我只想知道.
你到底是什麼意思.
我只想知道.
你到底是什麼意思.
我只想知道.
你到底是什麼意思.
我只想知道.
你到底是什麼意思.
我只想知道.
你到底是什麼意思.
我只想知道.
你到底是什麼意思.
我只想知道.
你到底是什麼意思.
我只想知道.
你到底是什麼意思.
我只想知道.
你到底是什麼意思.
我只想知道.
你到底是什麼意思.
我只想知道.
你到底是什麼意思.

$^{1281}$我只想知道.
你到底是什麼意思.
我只想知道.
你到底是什麼意思.
我只想知道.
你到底是什麼意思.
我只想知道.
你到底是什麼意思.
我只想知道.
你到底是什麼意思.
我只想知道.
你到底是什麼意思.
我只想知道.
你到底是什麼意思.
我只想知道.
你到底是什麼意思.
我只想知道.
你到底是什麼意思.
我只想知道.
你到底是什麼意思.
我只想知道.
你到底是什麼意思.
我只想知道.
你到底是什麼意思.
我只想知道.
你到底是什麼意思.
我只想知道.
你到底是什麼意思.
我只想知道.
你到底是什麼意思.
我只想知道.
你到底是什麼意思.
我只想知道.
你到底是什麼意思.
我只想知道.
你到底是什麼意思.
我只想知道.
你到底是什麼意思.
我只想知道.
你到底是什麼意思.

$^{1321}$我只想知道.
你到底是什麼意思.
我只想知道.
你到底是什麼意思.
我只想知道.
你到底是什麼意思.
我只想知道.
你到底是什麼意思.
我只想知道.
你到底是什麼意思.
我只想知道.
你到底是什麼意思.
我只想知道.
你到底是什麼意思.
我只想知道.
你到底是什麼意思.
我只想知道.
你到底是什麼意思.
我只想知道.
你到底是什麼意思.
我只想知道.
你到底是什麼意思.
我只想知道.
你到底是什麼意思.
我只想知道.
你到底是什麼意思.
我只想知道.
你到底是什麼意思.
我只想知道.
你到底是什麼意思.
我只想知道.
你到底是什麼意思.
我只想知道.
你到底是什麼意思.
我只想知道.
你到底是什麼意思.
我只想知道.
你到底是什麼意思.
我只想知道.
你到底是什麼意思.

$^{1361}$我只想知道.
你到底是什麼意思.
我只想知道.
你到底是什麼意思.
我只想知道.
你到底是什麼意思.
我只想知道.
你到底是什麼意思.
我只想知道.
你到底是什麼意思.
我只想知道.
你到底是什麼意思.
我只想知道.
你到底是什麼意思.
我只想知道.
你到底是什麼意思.
我只想知道.
你到底是什麼意思.
我只想知道.
你到底是什麼意思.
我只想知道.
你到底是什麼意思.
我只想知道.
你到底是什麼意思.
我只想知道.
你到底是什麼意思.
我只想知道.
你到底是什麼意思.
我只想知道.
你到底是什麼意思.
我只想知道.
你到底是什麼意思.
我只想知道.
你到底是什麼意思.
我只想知道.
你到底是什麼意思.
我只想知道.
你到底是什麼意思.
我只想知道.
你到底是什麼意思.

$^{1401}$我只想知道.
你到底是什麼意思.
我只想知道.
你到底是什麼意思.
我只想知道.
你到底是什麼意思.
我只想知道.
你到底是什麼意思.
我只想知道.
你到底是什麼意思.
我只想知道.
你到底是什麼意思.
我只想知道.
你到底是什麼意思.
我只想知道.
你到底是什麼意思.
我只想知道.
你到底是什麼意思.
我只想知道.
你到底是什麼意思.
我只想知道.
你到底是什麼意思.
我只想知道.
你到底是什麼意思.
我只想知道.
你到底是什麼意思.
我只想知道.
你到底是什麼意思.
我只想知道.
你到底是什麼意思.
我只想知道.
你到底是什麼意思.
我只想知道.
你到底是什麼意思.
我只想知道.
你到底是什麼意思.
我只想知道.
你到底是什麼意思.
我只想知道.
你到底是什麼意思.

$^{1441}$我只想知道.
你到底是什麼意思.
我只想知道.
你到底是什麼意思.
我只想知道.
你到底是什麼意思.
我只想知道.
你到底是什麼意思.
我只想知道.
你到底是什麼意思.
我只想知道.
你到底是什麼意思.
我只想知道.
你到底是什麼意思.
我只想知道.
你到底是什麼意思.
我只想知道.
你到底是什麼意思.
我只想知道.
你到底是什麼意思.
我只想知道.
你到底是什麼意思.
我只想知道.
你到底是什麼意思.
我只想知道.
你到底是什麼意思.
我只想知道.
你到底是什麼意思.
我只想知道.
你到底是什麼意思.
我只想知道.
你到底是什麼意思.
我只想知道.
你到底是什麼意思.
我只想知道.
你到底是什麼意思.
我只想知道.
你到底是什麼意思.
我只想知道.
你到底是什麼意思.

$^{1481}$我只想知道.
你到底是什麼意思.
我只想知道.
你到底是什麼意思.
我只想知道.
你到底是什麼意思.
我只想知道.
你到底是什麼意思.
我只想知道.
你到底是什麼意思.
我只想知道.
你到底是什麼意思.
我只想知道.
你到底是什麼意思.
我只想知道.
你到底是什麼意思.
我只想知道.
你到底是什麼意思.
我只想知道.
你到底是什麼意思.
我只想知道.
你到底是什麼意思.
我只想知道.
你到底是什麼意思.
我只想知道.
你到底是什麼意思.
我只想知道.
你到底是什麼意思.
我只想知道.
你到底是什麼意思.
我只想知道.
你到底是什麼意思.
我只想知道.
你到底是什麼意思.
我只想知道.
你到底是什麼意思.
我只想知道.
你到底是什麼意思.
我只想知道.
你到底是什麼意思.

$^{1521}$我只想知道.
你到底是什麼意思.
我只想知道.
你到底是什麼意思.
我只想知道.
你到底是什麼意思.
我只想知道.
你到底是什麼意思.
我只想知道.
你到底是什麼意思.
我只想知道.
你到底是什麼意思.
我只想知道.
你到底是什麼意思.
我只想知道.
你到底是什麼意思.
我只想知道.
你到底是什麼意思.
我只想知道.
你到底是什麼意思.
我只想知道.
你到底是什麼意思.
我只想知道.
你到底是什麼意思.
我只想知道.
你到底是什麼意思.
我只想知道.
你到底是什麼意思.
我只想知道.
你到底是什麼意思.
我只想知道.
你到底是什麼意思.
我只想知道.
你到底是什麼意思.
我只想知道.
你到底是什麼意思.
我只想知道.
你到底是什麼意思.
我只想知道.
你到底是什麼意思.

$^{1561}$我只想知道.
你到底是什麼意思.
我只想知道.
你到底是什麼意思.
我只想知道.
你到底是什麼意思.
我只想知道.
你到底是什麼意思.
我只想知道.
你到底是什麼意思.
我只想知道.
你到底是什麼意思.
我只想知道.
你到底是什麼意思.
我只想知道.
你到底是什麼意思.
我只想知道.
你到底是什麼意思.
我只想知道.
你到底是什麼意思.
我只想知道.
你到底是什麼意思.
我只想知道.
你到底是什麼意思.
我只想知道.
你到底是什麼意思.
我只想知道.
你到底是什麼意思.
我只想知道.
你到底是什麼意思.
我只想知道.
你到底是什麼意思.
我只想知道.
你到底是什麼意思.
我只想知道.
你到底是什麼意思.
我只想知道.
你到底是什麼意思.
我只想知道.
你到底是什麼意思.

$^{1601}$我只想知道.
你到底是什麼意思.
我只想知道.
你到底是什麼意思.
我只想知道.
你到底是什麼意思.
我只想知道.
你到底是什麼意思.
我只想知道.
你到底是什麼意思.
我只想知道.
你到底是什麼意思.
我只想知道.
你到底是什麼意思.
我只想知道.
你到底是什麼意思.
我只想知道.
你到底是什麼意思.
我只想知道.
你到底是什麼意思.
我只想知道.
你到底是什麼意思.
我只想知道.
你到底是什麼意思.
我只想知道.
你到底是什麼意思.
我只想知道.
你到底是什麼意思.
我只想知道.
你到底是什麼意思.
我只想知道.
你到底是什麼意思.
我只想知道.
你到底是什麼意思.
我只想知道.
你到底是什麼意思.
我只想知道.
你到底是什麼意思.
我只想知道.
你到底是什麼意思.

$^{1641}$我只想知道.
你到底是什麼意思.
我只想知道.
你到底是什麼意思.
我只想知道.
你到底是什麼意思.
我只想知道.
你到底是什麼意思.
我只想知道.
你到底是什麼意思.
我只想知道.
你到底是什麼意思.
我只想知道.
你到底是什麼意思.
我只想知道.
你到底是什麼意思.
我只想知道.
你到底是什麼意思.
我只想知道.
你到底是什麼意思.
我只想知道.
你到底是什麼意思.
我只想知道.
你到底是什麼意思.
我只想知道.
你到底是什麼意思.
我只想知道.
你到底是什麼意思.
我只想知道.
你到底是什麼意思.
我只想知道.
你到底是什麼意思.
我只想知道.
你到底是什麼意思.
我只想知道.
你到底是什麼意思.
我只想知道.
你到底是什麼意思.
我只想知道.
你到底是什麼意思.

$^{1681}$我只想知道.
你到底是什麼意思.
一個無形教會.
其實是奧古斯丁發明的字.
在四世是形而上的.
這是Plato的字學.
一個最完美的燈.
是在天上的一個理型的燈.
如果讀過歷史的人.
真實的燈總是會爛.
還有很多不同的形狀.
總是會消逝.
總是會爛壞.
真正最完美的東西.
是在天上.
所以姜濤也不夠最帥.
是天上的一個姜濤的想法.
才是最帥的.
意思就是.
任何在地上的東西都不是完美的.
所以奧古斯丁就用了這些概念.
來解釋.
教會其實是地上的教會.
是髒髒的.
他這麼說是有背景的.
因為當時有些人叛教.
有些人以前被迫離開教會.
終於教會就不好.
奧古斯丁就嘗試去平衡這兩件事.
我們看著教會.
他們不行.
他們曾經不信耶穌.
他們曾經甚至離棄信仰.
他們很軟弱.
他們甚至說到很悶.
經常遲到.
總之.
奧古斯丁為瞭解釋地上教會的問題.
發明瞭一個字.
真正的教會.

$^{1721}$不要失望.
因為真正的教會是一個無法見證的教會.
上帝所預定的一群人.
是一個見不到的群體.
但他們是真正的教會.
地上的教會.
其實是什麼.
他們叫物子和敗子.
我經常忘記怎麼讀.
敗子.
所以將來審判的時候.
那些才是真正的基督徒.
這就是此類的看法.
所以奧古斯丁嘗試去解釋.
為什麼地上的教會那麼髒髒的.
其實不要緊.
是髒髒的.
因為我們說真正的教會.
真正上帝屬靈神聖的基督的身體.
其實是形而上的.
無法見證的.
這個說法好還是不好.
當然他能夠解釋.
能夠用無法見證的教會去解釋.
我們這群有形群體.
其實都是軟弱的.
都是有限的.
也不一定是真的.
最真最完美的.
其實在無法見證里.
所以大家是沒有對話的.
大家OK的.
因為教會是這樣的.
這個就是奧古斯丁的解法.
但其實當我們這樣做的時候.
其實是做什麼.
其實是不斷避免.
去解釋為什麼地上的教會髒髒的.
你只不過不斷推薦.
地上的教會其實不是真的.

$^{1761}$不是最真實的.
最真實的不是最真實的.
只不過是堂會.
不是教會.
很多這些說法.
但這個不是最好的方法.
去解釋現在的問題.
當然如果看回歷史裡面.
東正教天主教.
其實都嘗試去強調這個無法見證教會.
東正教會他們會嘗試.
當無法見證教會執行聖餐的時候.
聖餐就令到一間髒髒的教會.
重新成為一間教會.
這些不過好心不說了.
總之我們嘗試不斷地解釋用不同的神理論.
去將地上有形的髒髒的教會.
去合法化.
我們不要只是將完美放在天上.
這個是卡爾巴克稱之為教會的幻影論.
不知道你是不是這樣.
很多香港基督徒.
可能在網上對教會的批評者.
可能都是這樣.
他拿著教會一個完美的點法.
教會應該有愛心的程度.
教會的人應該彼此相愛.
教會應該不貪錢.
教會應該很關心我們每個人的需要.
教會應該民主.
很多這些不同的看法.
其實他們只是拿著理想的教會的圓形.
去對地上的問題比較.
這樣永遠會找到很多問題.
你拿著完美的教會點法.
去和見到的教會比較.
總會找到很多不同的問題.
這個我稱之為教會的幻影論.
幻影論這個字本身來自於耶穌的詮釋.
耶穌是一個幻影.

$^{1801}$認為上帝不會受到任何的問題.
所以他是一個幻影.
沒有特意來到地上.
這是一個異端的說法.
如果把這個看法去到教會的問題.
同樣也是.
教會是一個嘗試將地上的教會去否定它.
真正的教會是天上的.
最終的方法也是否定地上的教會.
從而去承認天上的教會才是唯一的教會.
這樣是不妥當的.
因為真正的教會就是我們現在看到的教會.
就是一群人.
怎麼可能去否定地上的教會呢.
所以今天我們嘗試去想.
究竟怎麼能夠解決這個問題.
可能這個就是大家以前的問題.
回到自己以前的誤會的時候.
很多的甩樓.
很多的問題.
很多的不理解.
很多令你覺得很激氣的地方.
我怎麼來理解地上的教會.
甚至Full Church也是.
Full Church也是一間地上的教會.
Full Church也不是一間完美的教會.
所以我怎麼能夠去平衡這件事呢.
這是我自己嘗試在這幾年里.
重新建構教會論.
我們稱之為教會的軟弱的本相.
我們嘗試不要將一些完美的東西.
作為一個default.
反過來我們嘗試將教會的軟弱.
成為我們的default.
我舉個例子.
問大家一個問題.
你也知道鋼琴裡面.
黑鍵和白鍵.
你覺得黑鍵還是白鍵是default的.
就是default的位置.

$^{1841}$你覺得黑鍵是降低了key的白鍵.
還是白鍵是升了key的黑鍵.
誰才是default的.
誰才是鋼琴原本default的按鈕.
後來按鈕是加進去的.
你明白我的意思嗎.
黑鍵還是白鍵.
你可以說是白鍵.
白鍵是default的.
後來人們發覺不夠音.
要sharp一點要fat一點.
就加上黑鍵.
其實是相對的.
黑鍵其實是default的.
白鍵只不過是黑鍵的sharp.
所以我們就用類似的思維.
就是說我們.
(問問觀眾).
所以我們嘗試用黑鍵來做default的狀態.
不是白鍵.
教會的default狀態.
其實不是白色而是黑色.
教會應該是凹凸不平的.
你不要把default看為教會是完美的.
黑鍵那些不好的.
是一些缺陷或者是意外.
我們嘗試反過來去想.
這個正正就是卡爾巴特在《羅馬書》裡面.
一個很好的一句話.
讀給大家聽.
今天看不到.
他說教會是雅各的教會.
只能發生在神跡裡面.
否則教會從來都是以素的教會.
卡爾巴特用了兩個概念來說教會.
一個叫以素的教會.
一個叫雅各的教會.
用教會語來比喻雅各和以素.
這裡麻煩大家.
這個在《羅馬書》時期.

$^{1881}$1922年裡面.
卡爾巴特嘗試去講教會論.
《羅馬書》是一本非常dialectic的書.
用一些很浪漫主義的手法來寫神學.
以素和雅各的教會.
就是說誰是奸誰不是奸.
很明顯吧.
以素是奸的.
以素是差一點.
雅各是好一點.
我們先當是這樣吧.
他說雅各的教會和以素的教會.
他說教會同時是雅各的教會.
同時是以素的教會.
他說教會是雅各的教會.
只能在神跡裡面.
否則教會從來都是以素的教會.
我們將黑厭看作是default.
教會就是一群人.
教會就是一群被呼召成為基督徒的人.
這個群體的行動就是教會.
這個人就是人.
就是一群人.
不過我們沒有否定過.
這個教會可以彼此相愛.
可以很溫暖.
可以做一些很正義的事情.
可以在正宗裡面.
轉化為法律.
不過這些事情.
是在神跡裡面的時候.
這些事情是一個.
唯有靠著恩典才能夠發生的事情.
如果沒有恩典的時候.
教會從來都是以素的教會.
卡邦尼的說話其實是重新把黑厭和白厭調轉.
不是說教會應該是很完美的.
為什麼突然間那個人不愛我.
為什麼那麼壞.
不要這麼想 反過來.

$^{1921}$這群人本來就很壞.
但竟然我們這群脾脾肋骨的人.
都能夠聚在一起.
做一些有意義的事情.
能夠在一起和諧共處.
都能夠有愛.
其實是神跡.
所以這是一個很大的教會觀.
我們沒有放棄過地上的教會是教會.
我們不會說這間是堂會.
我們不會否定這個教會只是一個.
visible church.
不是教會.
它仍然是真正的教會.
它仍然是上帝的教會.
不過它的好東西.
只能夠在禱告和恩典里.
竟然可以那麼好.
我們今天這群人聚在一起.
或者網上的人聚在一起.
我們都那麼有心來學神學.
是挺好的 大家都挺好的.
但這是神跡.
我經常說我在長洲裡面教神學.
是一個很棒的經驗.
我們山上有百多個神學山住在一起.
百多個好人.
全都人都不是好人.
有些是壞的.
起碼仇山是OK的.
仇山還是比較幼稚.
百多個好人住在一起.
是一件很難精靈的事.
所以是可以發生的.
我們說彼此相愛 法莫共用.
我們能夠一起有使命 有意象.
去感動 是可以的.
但這不是default.
這是因點.
所以我們嘗試這樣來理解教會是什麼.

$^{1961}$教會的本體就是從我們的軟弱開始.
之前的頁我先show剛才那一頁.
之前的頁.
所以我說教會的軟弱的本相.
我們用church的weakness來作為我們整個本體的開始點.
我們承認我們教會本身是軟弱的.
這個不需要驚訝.
這是我們會有的東西.
我們能夠大家不同意見都可以在一起.
這個反而是因點.
去到後面 去剛才那一頁.
剛才我們有個總結.
去到後面那一頁.
OK 下一頁.
所以我說教會的軟弱不是異數.
而是作為教會本體.
尤其當我們不嘗試為教會的有形無形作一個不必要的區分的時候.
教會的軟弱是一個非常重要的課題.
我們不再嘗試用有形無形來區分.
好的東西歸給無形.
不好的東西歸給有形.
而是我們見到真實地這間教會.
這個有形群體它就是教會.
只不過它們的好東西仍然是有的.
不過在因點裡面.
所以這個就是我們來思考教會的很重要的課題.
好 下一個.
所以 下一張 麻煩你.
所以教會的那個.
下一個就是教會的基督的離去和同在.
同樣都是.
基督的升天和基督的再來中間的一段時間.
這個正正就是教會時期.
當基督耶穌升天都住在了反向的時間.
這段居間期正正就是教會地上存在的時間.
這段時間是很特別的.
因為我經常都說過.
很多年前我教授都說過這篇道.
就是基督的離開.
我們經常都強調基督的同在.

$^{2001}$但其實基督同時的升天.
很重要的意義就是基督耶穌不在.
不是真實的在.
起碼就是.
所以我們要知道我們很多仍然是軟弱的.
我們只能夠靠住.
你看到耶穌不是一下子下來.
突然間幫你很多問題.
然後耶穌就說了.
我們只能夠在同在和離開的那種緊張裡面.
約翰福音14章裡面.
我們經常都在裡面.
祂在我們裡面.
但同時也是一個離別之言.
同時耶穌也是正正是祂的門徒.
耶穌將要離開.
所以耶穌的升天標誌著什麼.
就是祂的離開.
但祂的離開不是完全沒有做事.
是什麼呢.
祂會將祂在天上為我們代求.
所以我們只能夠在這樣的狀態裡面.
我們是基督的身體.
我們是基督的身父.
上帝的子民.
這些詞很威風.
不過我們確實也不是那麼行.
門徒.
一旦耶穌離開就軟到爆.
所以在這樣的狀態裡面.
我們只能夠得到一些很真實.
但又不是完全讓你理解到的東西.
基督耶穌會在天上為我們代求.
耶穌是在我們身邊.
不過是祂的靈在我們身邊.
所以在這樣的曖昧.
有些行有些不行.
這樣的狀態正正就是我們教會的狀態.
所以怎麼辦呢.
下一章請.

$^{2041}$怎麼辦呢.
所以說更新是一個必然的途徑.
既然我們說教會的軟弱.
正正就是我們的本體的時候.
更新就絕對不是一個純屬的偶發的事情.
教會的改變.
教會的更新.
成為教會的本體很重要的一環.
它不是偶發的.
它必定會間中轉變一遍.
所以我都說沒有東西是完美的.
唯有不斷更新和改變才是完美的.
所以教會也是一樣.
沒有一個完美的教會.
Full Church也不是.
但如果Full Church能夠不斷地更新.
不斷地改變.
不斷地按照我們的時代需要去改變.
我經常說.
今天我推翻昨天的我這件事不是不好的.
是很重要的.
你需要去打倒昨天的我.
因為你重新去聆聽上帝的聲音.
重新去明白我們應該怎麼做.
需要不斷地去改變和更新.
所以這是我們很重要的.
教會的更新是我們很重要的一環.
今天不說了.
可以再談談這個問題.
所以教會是不斷地需要更新和改變.
所以Full Church就這樣開始.
在三年前的時間里.
Full Church是一間教會.
是一間有形的教會.
是有運作的教會.
不過我們當然嘗試有一個新的方法運作.
一個所謂的Simple Church.
有時候也說重新推翻所有外在的東西.
抓緊那個essence.
針對那個essence來回應這個時代.

$^{2081}$所以我們仍然是做教會應該做的事.
不過不是用傳統的方法.
因為這些方法是可以變的.
但essence是我們不會變的.
所以Full Church的出現.
就是仍然是做一間.
知道是有軟弱的教會.
我們不會寄望這間教會沒有軟弱.
但是都很警惕軟弱的問題.
怎麼能夠在運作上.
重新減少軟弱的問題之類的.
這些會是上Info Group也聽過的.
所以Full Church的概念.
正正是從今天所講的教會觀來做開始.
下一張請.
我想多說一些教會的使命.
Full Church的mission.
所以教會Full Church在這個年代.
正正是做教會應該做的事.
從凝聚了一群人之後.
我們如何能夠見證基督呢.
這個群體如何能夠發生呢.
如何能夠藉著這群人做一些事來發生呢.
因為當我們做事的時候.
才出現教會這件事.
所以Flow字就是這個意思.
當我們這群人在行動的時候.
Flow Church才突然出現了.
當大家散會的時候就沒有了這教會.
我這樣理解的時候是會.
當我們聚集在一起.
當然敬拜就不可能了.
我們Full Church只要是在做事的時間.
包括見證基督耶穌.
在香港社會裡面.
我們是一個敬拜群體.
當我們彼此相愛的時候.
Full Church就形成了.
然後當我們小組結束了就散了.
Full Church就不出現了.

$^{2121}$我平時是這樣理解Full Church這件事.
所以教會是一聚一聚就在散.
這個的發生是很重要.
而我們今天Full Church正是想做這件事.
在香港裡面來秉持我們應有的使命.
就是去大堂上說話.
做一個見證基督的群體 基督徒.
這個是我們在香港裡面很重要的課題.
下一張.
所以回到實際問題.
究竟是否需要教會呢.
因為其實這個問題.
這一年裡面很多人都問這個問題.
特別是如果你是二十多歲的話.
你覺得教會.
我們這個無堂會.
很明顯Full Church不是一間無堂會的教會.
為什麼我不會做一間無堂會的教會呢.
因為無堂會是一個很好的理想.
但其實我們說Full Church.
教會本身作為地上的群體.
如果我們不否定它的話.
Full Church本身是一間教會.
它無可避免有這些東西.
它肯定是有運作上的東西.
有一些人的計劃的東西.
當然它可以是一個.
我們去短孫吧 不如我們走.
這個早點.
但其實我們如果想長遠運作的時候.
有堂會 其實這不是一件壞事.
問題是如何能夠有堂會.
能夠做到的事.
我們不想做了一間.
兩年之後就沒有了.
能否維持下去.
將使命 將教導.
所以每一方是一個很理想的想法.
但當我們說教會是地上的教會的時候.
我們就有一些很實際的東西要去想.

$^{2161}$所以這幾年裡面.
我從一個神學人.
慢慢變成一個教會人.
其實就是將一些神學的東西.
變成一些很實際的東西.
如何能夠顧及一些實際的東西.
這個是重要的.
所以我們不是覺得教會是不需要運作.
教會仍然是需要這些東西.
但我們想要減化掉.
或者變成一些有意義的東西.
而不是完全否定一些實際上的東西.
所以簡單來說.
對於無堂會法的問題.
我覺得這個是可以試試的.
但是我仍然覺得是有需要.
是有一定程度的運作的東西.
這個才能夠真正地.
解決一些實際的問題.
下一個 最後兩個問題.
所以這個時勢還需要教會嗎.
不知道大家的答案是什麼.
這個時勢其實是需要教會的.
應該有的 下一個 謝謝.
我給答案出來 是需要的.
為什麼需要.
因為其實教會的目的.
正正就是去私行宣揚盼望.
耶穌基督的福音的盼望.
所以這時候更加需要教會.
更加需要一個能夠存在香港的教會.
當然很多人都問.
究竟將來怎麼辦.
將來如果打倒賣尼的時候怎麼辦.
不過起碼我都說.
能夠存在就繼續存在.
能夠聚在一起就聚在一起.
只不過我們知道.
我們不是為了聚在一起而聚在一起.
帶著使命 能夠改變.

$^{2201}$帶著希望去做這件事.
其實現在很多人是需要耶穌的.
我們科學學院裡面.
也有發現有些不信耶穌的人.
來到教會裡面.
他們真的很需要耶穌基督的盼望.
來支持自己.
怎麼面對這個世界.
所以這個時勢更加需要教會.
更加需要我們這群.
更加明白什麼是教會的人.
第二條.
教會有什麼用呢.
有沒有用這個問題.
是一個很特別的問題.
有沒有用.
你覺得自己有沒有用.
我不知道今天有沒有用.
但我都覺得有沒有用這個問題.
其實這個問題.
都是問得很特別.
我忘了怎麼回答.
有沒有用.
應該說有用.
有用的.
因為我們是全是耶穌基督.
多說一句.
這個時候更加有需要.
當然我們不是問教會.
沒有需要 沒有用.
但我都覺得這些東西.
正正就是這樣.
你發覺很多東西你都做不到的時候.
我們說的這些耶穌的東西.
就變得更加特別有用.
以前你覺得都會法律援助.
或者出版一些什麼.
都是實際的東西.
我們做傳道人這行是最沒用的.
基本上.

$^{2241}$說的都是耶穌.
說的都是天國的東西.
但在這個年代裡面.
當你發覺做的那些東西.
都是沒什麼用的時候.
我們這些沒什麼用的東西.
都挺有用的.
我發覺都是會.
因為就只能夠說一些更加軟的東西.
就只能夠說一些.
就是生命的東西.
不是說一些那麼實際的東西.
但正正就是我們這些基督徒.
就是說這些東西.
宣揚一個比較軟的眼光.
宣揚一些看不到的東西.
這個有形的群體.
去說一些無形的東西.
正正就是我們教會.
香港教會.
對於香港這個年代裡面.
很重要的一個用處.
正正就是說一些沒用的東西.
好像沒用的東西.
但這些沒用的東西.
在什麼都沒用的時間裡面.
是最有用的.
所以大概就說到這裡.
我口渴了.
請給我一杯奶茶.
我經常看到你.
很多時候都說耶穌.
說教會.
其實這裡附近有很多教會.
我想問.
其實和我開茶餐廳.
或者開茶几.
有沒有分別.
深的問題.
就是什麼.

$^{2281}$就是有沒有分別.
我自己覺得.
開個檔.
剛才聽你.
其實不同人去不同教會.
其實不同人去不同茶几.
都差不多.
是不是.
當然是.
肯定是沒有一間茶几是獨市.
但有沒有一些.
每個人都要.
每一間開茶的人.
都覺得自己的奶茶最好喝.
對嗎.
我這間才是最正宗.
最好喝的那間.
這個自信我是有的.
OK.
但有沒有一些.
你每次來都是說不叫東西的.
有些人回教會.
就是只聽不做事.
是嗎.
會不會.
不如問一下他們.
對.
你們都回過很多教會.
剛才都說了.
你們過去可能回教會.
都有不同的經歷.
跟你今天聽他說完之後.
關於教會的感覺.
或者理解的.
其實是什麼.
大家可以說一下.
這間是第一間教會.
是不是初信.
或者選教會.
其實會選什麼.

$^{2321}$有沒有什麼指引.
我們一定會看到一個有型的教會.
很有型.
你這間教會是不是很有型.
要說一下.
要說一下這間教會是怎樣.
是嗎.
大家選教會.
或者是否像選茶餐廳.
試過覺得可以.
不是的.
是,後面有一個人問.
還有這個成分.
就是會去那間教會瞭解一下.
然後去知道多一點.
但是其實.
剛才和John說的.
有些地方都是相符的.
就是說.
一間教會是否真的做實事.
這件事是要有的.
因為很多坊間的教會.
都是會比較.
他們會集中在概念上.
但是實際的行動.
或者是可能.
例如落地.
你去做一些事情也好.
其實這件事是.
有些教會可能是欠縫.
我想有些討論一下.
什麼叫概念或者實事.
因為我們剛才說的.
就好像.
選茶餐廳.
他就說奶茶到位正.
就吸引到人來.
那會有什麼叫做.
是實事可以吸引到一群人來這間教會呢.
關心社會事實.

$^{2361}$其他人呢.
就是選教會有什麼.
考慮因素.
或者對你來說.
剛才聽完說.
教會其實是一件什麼事呢.
後面.
剛才在想John你提到的問題.
我們回教會究竟是找一個.
是同一群人來回教會.
還是回一個建築的教會.
我自己也有看過不同的教會.
也回過不少國家的教會.
也看過.
其實我自己也會想.
如果回到剛才John的問題.
究竟我去選一間教會.
或者我去找那間教會的時候.
究竟我是去找那間教會在做的事情多一些.
還是我找那間教會.
可能我這間大一點.
或者這間剛才是Hillsong之類的.
究竟我是應該如何去選擇教會.
第二個問題就是.
有些人會問.
如果教會是純粹是一群人.
而不是一個建築的時候.
如果我一大群人聚在一起的時候.
是不是代表我可以.
就這樣變成一個教會.
不需要回一間實體的教會.
我想我為什麼會想到handicap呢.
因為今天所說的話.
其實和選擇教會是有些.
當然是很實際的.
選擇教會是一個很實際的問題.
但是發覺剛才說了大量的東西.
其實那個縱深.
反而是大家能夠明白教會是什麼呢.
其實很多事情你應該是會做的.

$^{2401}$懂得如何去行動.
如果你知道原來教會是這樣的.
教會的重點是在於那個event.
那個行動.
其實那些東西.
你可以說是幫你選擇教會.
其實我覺得問題.
我就不會說今天付出是如何選擇教會.
但是問題是.
發覺因為問題有點.
真的不像茶几.
因為茶几是消費的.
因為餐廳是在消費.
你給錢買杯奶茶就要好喝.
所以這個問題.
我始終都不懂得這樣放在一起去談.
反而就是說.
如果我們是.
想起第一科.
我們是一個基督徒.
你被呼召.
基督徒來到去要做一些事的時候.
有點像你現在打手游.
你多些加入哪個工會.
你去做一些事.
我加入哪個工會去做哪些事.
我想選擇哪個工會來到去實踐使命.
多過我來到去哪裡好些.
所以是能夠可以問.
我想問哪個教會.
我想和哪些人一起來實踐使命.
我覺得這個會是.
我想金塘跟我的思路回答多些.
所以正正這個都是回到最基本.
今天確實是這個年代裡面.
你說上年那條法例或者疫情之類.
香港教會亂七彩.
現在重新再回頭問的時候.
其實應該也問這個問題.
我們應該是去.

$^{2441}$想哪個群體一起去行.
這個是重要的.
其他群體不代表行不通.
一起而已.
沒有說多少人一起.
沒有說什麼style一起.
所以我覺得是這樣去想.
你能夠明白這個問題之後.
你自然而然就會選擇.
我應該.
我選擇了群體.
選擇之後不是終極的重點.
你選擇之後要做些什麼.
所以.
不過我也覺得Full Trust是值得行的.
因為是大家好.
跟教會好不好和Facebook好不好一樣.
發現你的Facebook很悶.
為什麼.
因為你的朋友很悶.
就是這個原因.
所以跟教會好不好.
其實都是關乎教會的人.
他們是否有心實踐使命.
自然而然這群人就聚在一起.
這個也是我覺得.
為什麼Full Trust這兩年.
做的事有些什麼做對了.
其實都不是關乎這件事.
而是大家聚在一起.
正正就像我們說的.
那些水散了.
回到一起.
大家一樣.
大家都在實踐一樣東西.
這個正正就是我們Full Trust最美的地方.
就是大家.
大家能夠好的話.
這間教會不就好了.
其他人呢.

$^{2481}$以實際例子去回應John.
剛才說的一群人聚集.
就一起做一些他們覺得值得的事.
其實Full Trust在不同派對的參與.
其實都很多元.
有些弟兄姐妹群體.
一起聚集的時候.
會想一些平時覺得很習以為常的經文.
但當轉個向.
成為一個大家再深思.
就變成弟兄你錯了的一條片子的製作.
另外就是有一群弟兄姐妹.
很關心社區的論社.
他們就聚集.
就是服務紅土區的.
有低收入家庭.
有SEN小朋友的家庭.
就在這裡做親子媽媽的社區服務.
那個都是來自不同小組.
甚至是播放的弟兄姐妹.
就一起聚集.
每次都是召集活動去服侍.
另外有些弟兄姐妹就想.
不是用影像.
是用聲音.
所以他們每個星期有兩天回來錄Podcast.
去做不同的製作節目.
就是想用聲音去做宣揚.
用聲音做見證.
這個就是回應John說的.
Flow Church有不同群體的結聚.
就在做一個活動.
而那個活動是教會的行動.
也是我們的表徵.
大家有沒有其他關於.
過去教會的參與經歷.
或者是.
是的 後面.
我可能說得有點題外話.
不過我也分享一下.

$^{2521}$其實我覺得.
剛才John說到的.
怎麼說呢.
剛才我也很欣賞你們.
Flow Church的事工.
我也有留意過.
我覺得在堂會聚在一起.
其實就是給大家.
有個.
對我來說.
教會有個空間給你去.
怎麼說呢.
是沒什麼逃脫的.
只是說可以做到.
你平時的生活限制了你.
社會限制的價值.
那套主流價值.
限制了你去做的一些.
可以做的事.
甚至是對教會的定型.
去限制了.
例如說.
沒人想過你在.
你的主流堂會有人.
會有人拍片去.
想重演模擬當時聖經的場景.
那.
那就是.
而我在.
我曾經見過另一間.
就是關於John說的.
那個無堂會化的事.
雖然他的確是理想.
但其實在現在的情況.
我覺得不是不可行的.
整件事.
因為.
因為其實你們有沒有聽過.
新罪教會的YouTube頻道.
這個Riordan Church.

$^{2561}$其實它就是一間.
無堂會的教會.
為什麼這樣呢.
其實它只是一個YouTube頻道.
而它只是有.
兩至三位.
其實是不滿意.
一些堂會體制的人.
那些神職人員走出來.
就開了.
但是.
雖然.
他做的時候是.
去拍一些片.
去分享他們對一些.
基督信仰的新的反思.
理論上.
但其實.
他私底下也有一個.
TG group.
給我們.
給人去裡面討論.
那.
我當然有參加那個group.
其實是.
就是說去裡面.
真的可以和某些.
幾位弟兄姊妹.
真的有心交.
每晚都可以有些.
拋一些真心的.
拿很多.
以前在堂會沒有怎麼探討的信仰.
拿出來討論.
大家都是有得著.
有造就.
生命是有成長.
其實這些都算我們做了一切.
我想這個世代.
這個時代底下.

$^{2601}$其實.
其實真的.
用flow check.
flow的概念就是.
其實水.
不是一池的死水.
是可以.
是好像流動的水.
我們在香港這個地方.
是屬於海洋的人.
是屬於海洋的.
既然你這個地方.
沒辦法令.
實現你想的信仰.
你怎麼看的信仰.
其實你就可以.
其實你就不需要.
只留在原本的comfort zone.
你可以走出去.
用新的形式.
或者是去.
新的.
新的形式.
去新的地方去發展.
就是發展.
其實.
總之而言.
就是說.
我覺得不需要定一套很明確的.
就是標準去說.
教育監教會.
怎麼怎麼你才對.
總之.
這個是很personal的.
總之這間教會就是.
對你來說.
是真的對你的信仰.
是有成長.
那你也有那幫弟妹.
是可以form一個很近.

$^{2641}$你既然一次.
還可以留下最近一起去聯繫.
有一個很bonding.
那這間我相信.
是適合你可以去留下的地方.
那.
所以.
就是這樣.
多謝你分享.
其他呢.
所以我不反對這些教會.
我都說.
我都年紀不小.
所以我都覺得.
有些東西我想不到.
所以.
現在很多二十多歲的人.
想起我的東西.
我自己覺得.
我自己不會做的.
但其實有很多人是做.
其實是很重要的.
我都說不斷更新.
所以我.
我十幾歲的時候.
我都會去.
所以我.
我十幾年之後.
我都不是.
年紀都不小.
所以就慢慢fade out.
大家有沒有想過.
十幾年之後.
這幅畫怎麼辦.
如果還有的話.
所以其實很多時候.
反而重要的就是這個moment.
十幾年.
在這幾年裡面.
能夠做到的東西.

$^{2681}$就做這些東西.
有些新的東西.
新到我都覺得.
get不到的.
不重要的.
正正我現在的東西.
有些人都get不到的.
但都支持的.
有些前輩都會.
所以.
只要能夠work就行了.
就是work不work呢.
其實是能見到的.
能不能夠.
有一班人能夠支持到.
剛才說.
有TG調教心覺.
就work了.
就是能夠有些.
跟到耶穌的.
知識上的.
什麼都做到.
就work了.
所以.
當然.
能夠work多久.
是個問題.
或者work不work.
或者怎麼work.
是個問題.
但總之.
你試試.
試到work.
就真的work.
這個work不work.
就是很明顯.
有得prove的.
所以我覺得.
不反對的.
很多新類型.

$^{2721}$我沒想過的教會.
都可以出現.
我小小回應.
其實.
什麼是work.
因為.
我不是咬文嚼字.
因為.
應用神學.
或者實踐神學.
裡面有一個東西.
要去面對.
就是.
work out.
就是那樣東西.
做出來的時候.
John剛才說得對.
是要發生一些東西.
發生的時候.
你要看果效.
有兩個字.
常常掛在我自己的唇邊.
就是output和outcome.
是什麼.
output就是一些.
facts.
data.
statistic.
是可以量化.
可以評估的.
可能teach you group有多少人.
有多少人看我們現在live.
有多少人在這裡聚會.
這是一些.
實際上的東西.
但如果追數字.
如果還記得.
在flowchurch做登記崇拜的時候.
其實我們四分十幾秒就爆了崇拜.
其實可以直到永遠.

$^{2761}$但我們不是停在數字追求.
就是我們想.
其實有多少新朋友.
想來flowchurch.
是沒有機會來.
這是我們要正視的.
所以我們要upset原本的報名方法.
用第二個方式.
去令更多人可以帶新朋友來.
因為我們不是追output的數字.
我們是要處理outcome.
我們要change.
outcome是change.
要改變.
我們的信仰是.
更新而變化.
有很多人在teach group討論.
或者很多人在群體討論.
是好的.
是可行的.
但他要轉化什麼.
我想回應John最後的問題.
是在傳揚基督嗎.
教會的使命是在傳揚基督.
還是他圍內在討論教會呢.
這就不是實踐神學.
或者是想要帶到的信息.
是要轉化我們的生命.
以至我們可以見證基督教會存在的價值.
就是讓事情發生.
這是重要的.
你好.
我不知道接下來的主題會不會有些離題.
剛才提及了很多關於一些概念.
whatever.
想知道在牧養方面.
又有什麼概念可以分享.
因為都明白教會是希望一班人.
可以一起去做一些事情去宣揚福音.
所以大家都有很多不同的ideas.

$^{2801}$轉化一些action.
去社會的群體去做.
至於弟兄姊妹之間的牧養.
怎麼可以保持一定程度的屬靈狀態.
以至可以令福音的廣傳更加闊.
謝謝.
牧養嘛.
你先說主題.
我先說.
在infogroup里會帶來一個很重要的信息.
就是教會不是一個屬靈的家.
一個面向.
教會同樣是一個teaching institute.
一個教育學院.
或者當school.
不同階段有不同進程.
有不同進程當中有教學的內容或提升.
正正就是希望在不同階段的時候.
讓他們有轉變.
和知道自己的成長方向是什麼.
這個就是統稱叫做牧養.
在牧養過程當中.
每個小組里都會有一個牧者.
去埋身和瞭解全組的需要.
或者個別組員的需要.
就去里定.
或者去瞭解他實際上過去有什麼需要處理.
或者是開發他有興趣的地方.
過去處理的東西就包括一些舊有的堂會的經歷.
或者他對人事物當中可能有些難處.
就希望能夠梳理.
反而正面的就是開發他可以認識的渠道.
用學校的概念就是.
Flow Church是一所學校.
一所學校有不同的Club.
不同的Club就是他有興趣就去加入.
舉個例子.
有些弟兄姊妹都知道我們有Coffee Corner.
Coffee Corner就是我們將來會發展的Club.
有些弟兄姊妹正在受訓做一個Barista.

$^{2841}$可以將來做一些在這裡服務弟兄姊妹的工作.
有些就像現在在拍攝的那兩位姐妹.
都是一個Production Team.
他們都是小組組員.
他們參加Production Team的Club.
有些是Podcast 有些是拍片.
有些是做現場的收音.
其實都是不同弟兄姊妹的興趣.
或者他有興趣想發展成為他可以侍奉的空間.
所以牧羊在Flow Church的概念.
不僅僅是一些知識上的提升.
或者重溫一些比較重要的聖經教導.
同樣都是在你的生命有個行動.
令到那件事件在你生命當中出現.
這是重要的.
所以我先點題就是說回.
讓每一位願意加入Flow Church的弟兄姊妹.
入組之前要瞭解的內容.
我補充一下.
跟這位旁邊的女士對於牧羊的看法.
我想起其中一個想法.
教會的存在也好運作也好.
其實不可以只依賴牧者.
我不是說加入或者潘Sir的牧者存在不重要.
而是說信徒也要學會自發和組織性.
因為剛才潘Sir的說法.
Club這東西本身也是由人自發.
有興趣就去開創.
大家要有這樣的自發性和組織性的創業.
才可以令到教會本身都是百花齊放.
這樣的教會才可以不斷.
像水一樣生生不息.
萬一像阿John或者潘Sir的牧者.
突然不在你身邊.
或者他不幸地有事而不見了.
你的信徒如果不懂自發.
重新組織自己.
大家去組織維繫教會.
是不是就這樣就會被教會分散.
不是不應該是這樣.

$^{2881}$所以之後的教會應該是.
你要如何去繼續.
其實存在不重要.
在一個有形的物質世界來說.
就好像沒有大台.
但不可以沒有組織的一個.
一個這樣的屬靈的.
怎麼說呢?概念或者一個社區.
這一點我少少補充.
我認同弟兄的內容.
不過在Flow Church的做法是.
Flow Church的做法就是.
沒有人就沒有事工.
當然是需要人的.
所以我要讓你瞭解我們的做法.
我不是說你做的不對.
但我說Flow Church的做法就是.
沒有那種參與的人就不會有事工.
意思就是如果沒有人願意學咖啡.
我們就不會開一個咖啡事工.
重點就是他有興趣想用這個方法去服事.
或者讓他更加接觸到社區.
我們就開展這個事工的內容.
正正就是可能和其他教會經歷不同.
其他教會經歷就開了一個事工出來.
然後就邀請人去fill up那個事工.
如果那群弟兄姐妹發現.
他已經過了他想服事的時間.
那事工仍然存在.
每年都要邀請人去fill up那個事工.
就使事情不斷地要邀請人.
我們的做法就是.
那個事工是因為有那群弟兄姐妹做.
如果那群弟兄姐妹沒有了.
其實都證明那個事工是有效的日期.
在Flow Church的事工運作就是.
不會開了不能關的.
我們會知道有些事工是有效的日期.
做了繼續做.
就好像剛才回應John的做法.

$^{2921}$如果有些事件看到有果效.
又吸引到其他人的時候.
就會不斷地flow那些人來運作那個事工.
這個就看到當市場需要.
或者人弟兄姐妹需要的時候.
就能夠知道那件事就到位了.
這個是很直接去感受到.
這個也是Flow Church去扮.
如果你當在學校的club.
就算你放在booth.
那個club是否吸引人都沒有用.
會斷樁的.
但是這個就讓人明白到.
不是為了一個崗位去run那件事.
而是為了人去做那件事.
都是以人為本.
本身應該是可以尊重弟兄姐妹的需要.
或者他們有什麼專長.
有沒有其他呢?.
(觀眾:有).
我想問剛才提到教會有很多event.
其實本身事工的重要性.
是否和教會是同等的呢?.
教會....
譬如這部琴作一個比喻.
這部琴本身的存在是否也有價值呢?.
還是它要不斷地在彈很多歌.
那些歌才有價值呢?.
還是兩樣都重要呢?.
我想那個意思.
首先event和事工是兩回事.
event和事工不是同等的.
這個概念正正是一個行動driven的概念.
所以對我來說.
教會是一個行動driven的概念.
這個講得比較抽象.
比較conceptual.
教會是什麼呢?.
教會不是先有教會的being.
才有教會的不同行動.

$^{2961}$對我來說.
因為你做一些教會的東西.
這些東西就構成了教會.
對我來說是這樣.
當然教會是一個很複雜的概念.
因為信徒基督徒當然是一個人.
但教會就是一群基督徒.
一群信耶穌的人.
他們一起去做一些事.
這樣的事就構成了教會.
我先拋開教會的觀念.
教會不是一件東西.
是這群人行動了才構成教會.
我舉兩個例子.
我那時候在德國去過蘇黎世.
在那裡辭運了一個教會.
很大一間教會.
不知道是什麼姓姓母的教會.
那間教會只是一間景點.
已經沒有崇拜了.
那間是不是教會呢?.
當然不是教會.
那間教堂.
那次我看的時候.
旁邊有一群人在傳福音.
大聲的大叫福音.
後來有警察抓了他.
因為他很吵.
很大的對比.
不是因為教會一個人在那裡.
就做一些教會的事.
而是當你做一些教會應該做的事.
那個才叫教會.
教會是一個活動的意思.
活動是一個發生的事情.
而不是一個有本體的事情.
有時候是這樣說的.
所以不是教會什麼都不做.
那就是教會.
我經常說.

$^{3001}$開這個話題.
牧師也是.
按了牧就成牧師.
你星期天放假.
你也是牧師.
但牧師其實是建基於牧養的行動.
不是因為你做牧師才去牧養.
所以我覺得這是很新教的概念.
行動主要是有本體.
多於有本體才做出行動.
也是一樣.
我們是基督徒.
我們做一些基督徒的事.
才去構成我們的基督徒身份.
多於我已經得救了.
我才去想想怎麼做.
所以是一個很強的行動主要概念.
那所以.
不好意思想問一問.
有關於一個個人的信徒和教會的關係.
因為剛才聽到是以一個事件.
或者以一個行動的形式去參與一個教會.
那我舉一個例子.
如果一個信徒.
他在參與某一間教會.
他有一個很有passion的.
就假設是可能是中咖啡.
他又有另一個passion.
可能在另一方面.
他參加可能在另一間教會的.
一個這樣的event.
可能他也是在做一個.
一個教會性質的event在發生.
但他同時參與兩間堂會.
那其實這樣算不算是一個健康還是不健康.
當教會的觀念都傾向於一個事件發生化.
可能是沒有以前好像很傳統那種.
很固化或者是一個很堂會式的.
很恆常很規律.
或者是一個信徒通常都會在某一個堂會裡面.

$^{3041}$去完成他整個教會生活的那部分.
那現在如果有這樣的情況.
或者是你剛才提出的這個比較.
就是和平時教會的這個觀念比較flow一點.
就是闊一點或者碎片化一點的concept.
那會是一個健康還是不健康.
你怎麼看呢.
謝謝你的問題.
幫我補充我整個教會論的看法.
剛才我說行動driven.
但堂會.
我沒有機會去講堂會.
就是堂會正正是什麼呢.
堂會正正是去facilitate這些行動的一個組織.
你看馬保羅的書也這麼說.
堂會真的是一間公司.
這句話不是負面的.
堂會真的是一個purpose來製造很多行動的組織.
堂會就是這樣.
堂會搞崇巴.
堂會搞段宣.
堂會搞騎勞會.
都是去做這些事.
所以我覺得有堂會是重要的.
因為是需要有一個這樣的.
大也好小也好的組織.
來去人為地做這些事出來的.
所以我覺得我不贊成沒有堂會就是這個意思.
因為我覺得是有一些人為組織的東西.
不是無為的.
但現在這個世代其實是很多這樣的.
就是那個堂會A可能你也去過.
就是堂會A正在縱視.
但堂會B就有一個班友friend.
然後回復出崇拜之類的.
現在這個世代就是這樣.
這件事我覺得也沒有問題.
因為我都說了.
我回某間教會品牌的時工.
那個崇拜的教會.

$^{3081}$那個搞的騎勞會.
搞的段宣.
那些人.
但這樣的教會是不包那些東西的.
但我仍然是有一個.
就是所謂實踐基督徒要做的事.
我有敬拜.
有喜相愛.
有行動有使命.
不過是不同品牌的組織的人.
去搞的東西出來的.
這個似乎是那個趨勢.
在沒有疆界和一個後疫情時代.
現在是越來越有助於這樣.
而這件事其實又無損教會論.
因為剛才說過.
有行動有使命.
不過是不同品牌組織出來的活動.
這個我自己覺得是可以的.
似乎趨勢也是這樣.
所以我的教會論就會覺得.
是這樣的就可以了.
因為我們說行動為本.
正正不是唐會品牌為本.
所以我們有一個真正的行動.
這行動是有意思的.
不過是不同品牌出來的行動.
所以就是為什麼.
是更加有用的.
不過我仍然覺得是很多東西.
沒有機會講得那麼深.
我們有很多東西.
一個弟妹成長很多東西.
當然有崇拜.
一個很重要的就是一個mentor.
這個正正是牧者很重要的關係.
牧者的牧養最重要是.
能夠有人跟你一起去走.
需要很多選擇.
你可以去參加某個名道社的班.

$^{3121}$神流院的課程.
這些都是有的.
但是一個牧者牧養的關係.
這件事也是很重要的.
所以又不是純粹組織活動.
最能夠滿足的事情.
所以我也覺得.
為什麼我們仍然花很多資源.
去保護牧養.
雖然現在說一點.
雖然說Folkshot好像不是很簡單.
其實不是.
Folkshot仍然保護牧養的簡單.
這部分仍然是很簡單.
不過其他東西其實有很多不同的.
你現在在直播.
當然不簡單.
但是牧養仍然是最簡單的.
牧者也不需要碰那些東西.
去幫助牧養維持這種關係.
用Folkshot的例子.
或者是運作去回應.
Info Group也提過這個例子.
當Folkshot仍然在沒有限聚的情況下.
實體聚會大概有五百多人.
每個崇拜.
但一直看著數字.
整個會眾的比率.
或者成分是怎樣.
其實有三分一是有小組的.
另外三分一就是.
每個月大概來兩次崇拜.
另外有三分一是.
每個月來一次.
或者是新朋友.
其中有每個月來兩次的弟兄姊妹.
曾經跟我談.
他很認真地說.
潘Sir,其實我每個月來兩次崇拜.
因為我星期日.

$^{3161}$還要在堂會服事.
如果不用服事.
星期日我星期六就可以來.
如果星期日要服事.
我星期六就要錄音.
所以我就來不了.
其實Folkshot這樣聚會是否可行.
我問為什麼不行.
他說你不覺得我去兩次堂會不可以嗎.
我說我可以.
不過我不知道你教會是否可以.
他問為什麼Folkshot覺得可以.
因為我很認真地跟他說.
如果你覺得在這裡崇拜.
是可以讓你充滿.
而你在這一班人服事.
另一個群體而我碰不到.
你是一個教會的延伸.
本身是一個教會的一體形容.
這就是很重要的.
因為我能夠慰養你.
而你能夠慰養其他人.
這不是國安國職的一體化表徵嗎.
這也是剛才說的School的概念.
如果他自己堂會未能夠慰養.
或者他覺得要多點位置.
他在其他地方能夠吃得到.
或者吃得飽的話.
其實彼此配搭.
這也是需要的.
當然我仍然認同John所說的.
他一定要有一個教育.
否則他慢慢都會枯乾.
或者慢慢的那個慰養概念.
就變成了為了做而做的活動.
就沒有了那個.
慰養就是慰養.
所以慰者的角色仍然是很重要的.
而我們的教會觀不是闊.
是因為我們知道.

$^{3201}$應該不要用堂會去定義一個位置.
沒有了互為肢體配搭.
彼此協作的那種尾線.
十點了,我們收到報告了.
下次你就早點命令.
下次見,我們在食堂.
拜拜.
請不吝點贊 訂閱 轉發 打賞支持明鏡與點點欄目.
\newpage



\section{}
\label{sec:d_aSxcuQPus}
\textbf{【這是最好的時代:給香港基督徒的神學八課】第4課:亂世的靈性修持|20210822 [d\_aSxcuQPus]}
\newline
\newline
連結: \href{https://youtube.com/watch?v=d_aSxcuQPus}{\texttt{ https://youtube.com/watch?v=d\_aSxcuQPus}} ~~~~ 語音日期: 2021-08-22 
\newline
\newline
\hyperref[sec:gexfrTB3Ccc]{\small{< < < PREV SERMON < < <}}
~
\hyperref[sec:index_chronic]{\small{[返順時目]}}
~
\hyperref[sec:index_scriptual]{\small{[返順卷目]}}
~
\hyperref[sec:VhMBgPBkDi8]{\small{> > > NEXT SERMON > > >}}
\newline
\newline
$^{1}$我只想知道.
你到底是什麼意思.
我只想知道.
你到底是什麼意思.
我只想知道.
你到底是什麼意思.
我只想知道.
你到底是什麼意思.
我只想知道.
你到底是什麼意思.
我只想知道.
你到底是什麼意思.
我只想知道.
你到底是什麼意思.
我只想知道.
你到底是什麼意思.
我只想知道.
你到底是什麼意思.
我只想知道.
你到底是什麼意思.
我只想知道.
你到底是什麼意思.
我只想知道.
你到底是什麼意思.
我只想知道.
你到底是什麼意思.
我只想知道.
你到底是什麼意思.
我只想知道.
你到底是什麼意思.
我只想知道.
你到底是什麼意思.
我只想知道.
你到底是什麼意思.
我只想知道.
你到底是什麼意思.
我只想知道.
你到底是什麼意思.
我只想知道.
你到底是什麼意思.

$^{41}$我只想知道.
你到底是什麼意思.
我只想知道.
你到底是什麼意思.
我只想知道.
你到底是什麼意思.
我只想知道.
你到底是什麼意思.
我只想知道.
你到底是什麼意思.
我只想知道.
你到底是什麼意思.
我只想知道.
你到底是什麼意思.
我只想知道.
你到底是什麼意思.
我只想知道.
你到底是什麼意思.
我只想知道.
你到底是什麼意思.
我只想知道.
你到底是什麼意思.
我只想知道.
你到底是什麼意思.
我只想知道.
你到底是什麼意思.
我只想知道.
你到底是什麼意思.
我只想知道.
你到底是什麼意思.
我只想知道.
你到底是什麼意思.
我只想知道.
你到底是什麼意思.
我只想知道.
你到底是什麼意思.
我只想知道.
你到底是什麼意思.
我只想知道.
你到底是什麼意思.

$^{81}$我只想知道.
你到底是什麼意思.
我只想知道.
你到底是什麼意思.
我只想知道.
你到底是什麼意思.
我只想知道.
你到底是什麼意思.
我只想知道.
你到底是什麼意思.
我只想知道.
你到底是什麼意思.
我只想知道.
你到底是什麼意思.
我只想知道.
你到底是什麼意思.
我只想知道.
你到底是什麼意思.
我只想知道.
你到底是什麼意思.
我只想知道.
你到底是什麼意思.
我只想知道.
你到底是什麼意思.
我只想知道.
你到底是什麼意思.
我只想知道.
你到底是什麼意思.
我只想知道.
你到底是什麼意思.
我只想知道.
你到底是什麼意思.
我只想知道.
你到底是什麼意思.
我只想知道.
你到底是什麼意思.
我只想知道.
你到底是什麼意思.
我只想知道.
你到底是什麼意思.

$^{121}$我只想知道.
你到底是什麼意思.
我只想知道.
你到底是什麼意思.
我只想知道.
你到底是什麼意思.
我只想知道.
你到底是什麼意思.
我只想知道.
你到底是什麼意思.
我只想知道.
你到底是什麼意思.
我只想知道.
你到底是什麼意思.
我只想知道.
你到底是什麼意思.
我只想知道.
你到底是什麼意思.
我只想知道.
你到底是什麼意思.
我只想知道.
你到底是什麼意思.
我只想知道.
你到底是什麼意思.
我只想知道.
你到底是什麼意思.
我只想知道.
你到底是什麼意思.
我只想知道.
你到底是什麼意思.
我只想知道.
你到底是什麼意思.
我只想知道.
你到底是什麼意思.
我只想知道.
你到底是什麼意思.
我只想知道.
你到底是什麼意思.
我只想知道.
你到底是什麼意思.

$^{161}$我只想知道.
你到底是什麼意思.
我只想知道.
你到底是什麼意思.
我只想知道.
你到底是什麼意思.
我只想知道.
你到底是什麼意思.
我只想知道.
你到底是什麼意思.
我只想知道.
你到底是什麼意思.
我只想知道.
你到底是什麼意思.
我只想知道.
你到底是什麼意思.
我只想知道.
你到底是什麼意思.
我只想知道.
你到底是什麼意思.
我只想知道.
你到底是什麼意思.
我只想知道.
你到底是什麼意思.
我只想知道.
你到底是什麼意思.
我只想知道.
你到底是什麼意思.
我只想知道.
你到底是什麼意思.
我只想知道.
你到底是什麼意思.
我只想知道.
你到底是什麼意思.
我只想知道.
你到底是什麼意思.
我只想知道.
你到底是什麼意思.
我只想知道.
你到底是什麼意思.

$^{201}$我只想知道.
你到底是什麼意思.
我只想知道.
你到底是什麼意思.
我只想知道.
你到底是什麼意思.
我只想知道.
你到底是什麼意思.
我只想知道.
你到底是什麼意思.
我只想知道.
你到底是什麼意思.
我只想知道.
你到底是什麼意思.
我只想知道.
你到底是什麼意思.
我只想知道.
你到底是什麼意思.
我只想知道.
你到底是什麼意思.
我只想知道.
你到底是什麼意思.
我只想知道.
你到底是什麼意思.
我只想知道.
你到底是什麼意思.
我只想知道.
你到底是什麼意思.
我只想知道.
你到底是什麼意思.
我只想知道.
你到底是什麼意思.
我只想知道.
你到底是什麼意思.
我只想知道.
你到底是什麼意思.
我只想知道.
你到底是什麼意思.
我只想知道.
你到底是什麼意思.

$^{241}$我只想知道.
你到底是什麼意思.
我只想知道.
你到底是什麼意思.
我只想知道.
你到底是什麼意思.
我只想知道.
你到底是什麼意思.
我只想知道.
你到底是什麼意思.
我只想知道.
你到底是什麼意思.
我只想知道.
你到底是什麼意思.
我只想知道.
你到底是什麼意思.
我只想知道.
你到底是什麼意思.
我只想知道.
你到底是什麼意思.
我只想知道.
你到底是什麼意思.
我只想知道.
你到底是什麼意思.
我只想知道.
你到底是什麼意思.
我只想知道.
你到底是什麼意思.
我只想知道.
你到底是什麼意思.
我只想知道.
你到底是什麼意思.
我只想知道.
你到底是什麼意思.
我只想知道.
你到底是什麼意思.
我只想知道.
你到底是什麼意思.
我只想知道.
你到底是什麼意思.

$^{281}$我只想知道.
你到底是什麼意思.
我只想知道.
你到底是什麼意思.
我只想知道.
你到底是什麼意思.
我只想知道.
你到底是什麼意思.
我只想知道.
你到底是什麼意思.
我只想知道.
你到底是什麼意思.
我只想知道.
你到底是什麼意思.
我只想知道.
你到底是什麼意思.
我只想知道.
你到底是什麼意思.
我只想知道.
你到底是什麼意思.
我只想知道.
你到底是什麼意思.
我只想知道.
你到底是什麼意思.
我只想知道.
你到底是什麼意思.
我只想知道.
你到底是什麼意思.
我只想知道.
你到底是什麼意思.
我只想知道.
你到底是什麼意思.
我只想知道.
你到底是什麼意思.
我只想知道.
你到底是什麼意思.
我只想知道.
你到底是什麼意思.
我只想知道.
你到底是什麼意思.

$^{321}$我只想知道.
你到底是什麼意思.
我只想知道.
你到底是什麼意思.
我只想知道.
你到底是什麼意思.
我只想知道.
你到底是什麼意思.
我只想知道.
你到底是什麼意思.
我只想知道.
你到底是什麼意思.
我只想知道.
你到底是什麼意思.
我只想知道.
你到底是什麼意思.
我只想知道.
你到底是什麼意思.
我只想知道.
你到底是什麼意思.
我只想知道.
你到底是什麼意思.
我只想知道.
你到底是什麼意思.
我只想知道.
你到底是什麼意思.
我只想知道.
你到底是什麼意思.
我只想知道.
你到底是什麼意思.
我只想知道.
你到底是什麼意思.
我只想知道.
你到底是什麼意思.
我只想知道.
你到底是什麼意思.
我只想知道.
你到底是什麼意思.
我只想知道.
你到底是什麼意思.

$^{361}$我只想知道.
你到底是什麼意思.
我只想知道.
你到底是什麼意思.
我只想知道.
你到底是什麼意思.
我只想知道.
你到底是什麼意思.
我只想知道.
你到底是什麼意思.
我只想知道.
你到底是什麼意思.
我只想知道.
你到底是什麼意思.
我只想知道.
你到底是什麼意思.
我只想知道.
你到底是什麼意思.
我只想知道.
你到底是什麼意思.
我只想知道.
你到底是什麼意思.
我只想知道.
你到底是什麼意思.
我只想知道.
你到底是什麼意思.
我只想知道.
你到底是什麼意思.
我只想知道.
你到底是什麼意思.
我只想知道.
你到底是什麼意思.
我只想知道.
你到底是什麼意思.
我只想知道.
你到底是什麼意思.
我只想知道.
你到底是什麼意思.
我只想知道.
你到底是什麼意思.

$^{401}$我只想知道.
你到底是什麼意思.
我只想知道.
你到底是什麼意思.
我只想知道.
你到底是什麼意思.
我只想知道.
你到底是什麼意思.
我只想知道.
你到底是什麼意思.
我只想知道.
你到底是什麼意思.
我只想知道.
你到底是什麼意思.
我只想知道.
你到底是什麼意思.
我只想知道.
你到底是什麼意思.
我只想知道.
你到底是什麼意思.
我只想知道.
你到底是什麼意思.
我只想知道.
你到底是什麼意思.
我只想知道.
你到底是什麼意思.
我只想知道.
你到底是什麼意思.
我只想知道.
你到底是什麼意思.
我只想知道.
你到底是什麼意思.
我只想知道.
你到底是什麼意思.
我只想知道.
你到底是什麼意思.
我只想知道.
你到底是什麼意思.
我只想知道.
你到底是什麼意思.

$^{441}$我只想知道.
你到底是什麼意思.
我只想知道.
你到底是什麼意思.
我只想知道.
你到底是什麼意思.
我只想知道.
你到底是什麼意思.
我只想知道.
你到底是什麼意思.
我只想知道.
你到底是什麼意思.
我只想知道.
你到底是什麼意思.
我只想知道.
你到底是什麼意思.
我只想知道.
你到底是什麼意思.
我只想知道.
你到底是什麼意思.
我只想知道.
你到底是什麼意思.
我只想知道.
你到底是什麼意思.
我只想知道.
你到底是什麼意思.
我只想知道.
你到底是什麼意思.
我只想知道.
你到底是什麼意思.
我只想知道.
你到底是什麼意思.
我只想知道.
你到底是什麼意思.
我只想知道.
你到底是什麼意思.
我只想知道.
你到底是什麼意思.
我只想知道.
你到底是什麼意思.

$^{481}$我只想知道.
你到底是什麼意思.
我只想知道.
你到底是什麼意思.
我只想知道.
你到底是什麼意思.
我只想知道.
你到底是什麼意思.
我只想知道.
你到底是什麼意思.
我只想知道.
你到底是什麼意思.
我只想知道.
你到底是什麼意思.
我只想知道.
你到底是什麼意思.
我只想知道.
你到底是什麼意思.
我只想知道.
你到底是什麼意思.
我只想知道.
你到底是什麼意思.
我只想知道.
你到底是什麼意思.
我只想知道.
你到底是什麼意思.
我只想知道.
你到底是什麼意思.
我只想知道.
你到底是什麼意思.
我只想知道.
你到底是什麼意思.
我只想知道.
你到底是什麼意思.
我只想知道.
你到底是什麼意思.
我只想知道.
你到底是什麼意思.
我只想知道.
你到底是什麼意思.

$^{521}$我只想知道.
你到底是什麼意思.
我只想知道.
你到底是什麼意思.
我只想知道.
你到底是什麼意思.
我只想知道.
你到底是什麼意思.
我只想知道.
你到底是什麼意思.
我只想知道.
你到底是什麼意思.
我只想知道.
你到底是什麼意思.
我只想知道.
你到底是什麼意思.
我只想知道.
你到底是什麼意思.
我只想知道.
你到底是什麼意思.
我只想知道.
你到底是什麼意思.
我只想知道.
你到底是什麼意思.
我只想知道.
你到底是什麼意思.
我只想知道.
你到底是什麼意思.
我只想知道.
你到底是什麼意思.
我只想知道.
你到底是什麼意思.
我只想知道.
你到底是什麼意思.
我只想知道.
你到底是什麼意思.
我只想知道.
你到底是什麼意思.
我只想知道.
你到底是什麼意思.

$^{561}$我只想知道.
你到底是什麼意思.
我只想知道.
你到底是什麼意思.
我只想知道.
你到底是什麼意思.
我只想知道.
你到底是什麼意思.
我只想知道.
你到底是什麼意思.
我只想知道.
你到底是什麼意思.
我只想知道.
你到底是什麼意思.
我只想知道.
你到底是什麼意思.
我只想知道.
你到底是什麼意思.
我只想知道.
你到底是什麼意思.
我只想知道.
你到底是什麼意思.
我只想知道.
你到底是什麼意思.
我只想知道.
你到底是什麼意思.
我只想知道.
你到底是什麼意思.
我只想知道.
你到底是什麼意思.
我只想知道.
你到底是什麼意思.
我只想知道.
你到底是什麼意思.
我只想知道.
你到底是什麼意思.
我只想知道.
你到底是什麼意思.
我只想知道.
你到底是什麼意思.

$^{601}$我只想知道.
你到底是什麼意思.
我只想知道.
你到底是什麼意思.
我只想知道.
你到底是什麼意思.
我只想知道.
你到底是什麼意思.
我只想知道.
你到底是什麼意思.
我只想知道.
你到底是什麼意思.
我只想知道.
你到底是什麼意思.
我只想知道.
你到底是什麼意思.
我只想知道.
你到底是什麼意思.
我只想知道.
你到底是什麼意思.
我只想知道.
你到底是什麼意思.
我只想知道.
你到底是什麼意思.
我只想知道.
你到底是什麼意思.
我只想知道.
你到底是什麼意思.
我只想知道.
你到底是什麼意思.
我只想知道.
你到底是什麼意思.
我只想知道.
你到底是什麼意思.
我只想知道.
你到底是什麼意思.
我只想知道.
你到底是什麼意思.
我只想知道.
你到底是什麼意思.

$^{641}$我只想知道.
你到底是什麼意思.
我只想知道.
你到底是什麼意思.
我只想知道.
你到底是什麼意思.
我只想知道.
你到底是什麼意思.
我只想知道.
你到底是什麼意思.
我只想知道.
你到底是什麼意思.
我只想知道.
你到底是什麼意思.
我只想知道.
你到底是什麼意思.
我只想知道.
你到底是什麼意思.
我只想知道.
你到底是什麼意思.
我只想知道.
你到底是什麼意思.
我只想知道.
你到底是什麼意思.
我只想知道.
你到底是什麼意思.
我只想知道.
你到底是什麼意思.
我只想知道.
你到底是什麼意思.
我只想知道.
你到底是什麼意思.
我只想知道.
你到底是什麼意思.
我只想知道.
你到底是什麼意思.
我只想知道.
你到底是什麼意思.
我只想知道.
你到底是什麼意思.

$^{681}$我只想知道.
你到底是什麼意思.
我只想知道.
你到底是什麼意思.
我只想知道.
你到底是什麼意思.
我只想知道.
你到底是什麼意思.
我只想知道.
你到底是什麼意思.
我只想知道.
你到底是什麼意思.
我只想知道.
你到底是什麼意思.
我只想知道.
你到底是什麼意思.
我只想知道.
你到底是什麼意思.
我只想知道.
你到底是什麼意思.
我只想知道.
你到底是什麼意思.
我只想知道.
你到底是什麼意思.
我只想知道.
你到底是什麼意思.
我只想知道.
你到底是什麼意思.
我只想知道.
你到底是什麼意思.
我只想知道.
你到底是什麼意思.
我只想知道.
你到底是什麼意思.
我只想知道.
你到底是什麼意思.
我只想知道.
你到底是什麼意思.
我只想知道.
你到底是什麼意思.

$^{721}$我只想知道.
你到底是什麼意思.
我只想知道.
你到底是什麼意思.
我只想知道.
你到底是什麼意思.
我只想知道.
你到底是什麼意思.
我只想知道.
你到底是什麼意思.
我只想知道.
你到底是什麼意思.
我只想知道.
你到底是什麼意思.
我只想知道.
你到底是什麼意思.
我只想知道.
你到底是什麼意思.
我只想知道.
你到底是什麼意思.
我只想知道.
你到底是什麼意思.
我只想知道.
你到底是什麼意思.
我只想知道.
你到底是什麼意思.
我只想知道.
你到底是什麼意思.
我只想知道.
你到底是什麼意思.
我只想知道.
你到底是什麼意思.
我只想知道.
你到底是什麼意思.
香港 香港 天涯思念的夢.
香港 香港 再有我同你夢想.
香港 香港 叫我不已快樂.
香港 香港 你永遠是曾夢想.
每一個年代都有每一個年代的神學.
作為土生土長的香港人.

$^{761}$我們似乎正在經歷一個最差的年代.
不過往往在最差的年代.
我們才能夠經歷福音信仰的最好.
就是我們一起從聖經裡面學習.
如何做這個年代裡面的香港基督徒.
這是最好的時代.
給香港基督徒的神學百科.
各位弟兄姊妹晚安.
我們來到第四堂.
香港基督徒神學百科的第四堂.
開始我們會從福音基督徒教會之外.
我們會講一些比較應用性的東西.
我們會講一下靈修或者靈性.
題目叫亂世的靈性修持.
我自己喜歡這個名字.
因為修持這個名字在教會裡面比較少用.
修持是修得來又要持.
保持.
我挺喜歡這個.
這個名字似乎是更加盡心的題目.
比起所謂靈修.
靈修的字就是道士打坐.
修行或者修煉.
基督徒的靈性.
我自己回到香港之後這麼多年.
都不斷有寫或者有講.
如果你有看我以前的書.
講靈命靈修.
或者如果你有上神學院的課程.
我自己也有講到有關禱告或者靈性的題目.
所以今天我自己這一堂有很多東西要講.
關於我們整個基督徒.
如何理解我們的靈性靈命.
我自己那本書還沒寫完.
其中一件事就是講這件事.
這本書還沒寫完.
所以我很多東西可以和大家分享.
當然加上一個字就是亂世.
特別在這個年代裡面.
我們如何理解我們的亂世.

$^{801}$所以我們怎麼開始呢.
不如講一下我自己的靈情.
我自己是十八歲信耶穌.
可能都講過.
當時初信書的時候.
都是和大家一樣.
都是被教導我們要去靈修.
以前我們那時候年代是有那些.
Form of 金泉.
那些靈修書.
每個月都會派一本給我們去做.
有些很厚的靈修書.
我們會去做.
我初信書的時候很乖的.
都是一直跟著做.
都是一個很典型的.
很上進的初信者.
那時候就跟著靈修書做.
甚至乎給自己一些.
extra 的靈修功夫.
那時候我屬靈九果.
每天都會選一個果子來做.
今天是喜樂日.
就做一些喜樂的事情.
今天是忍耐就忍耐.
很多這些計劃給自己.
很快的我就讀神學.
讀神學的時候.
是我自己思考靈修.
變成一個很重要的階段.
那時候我住長洲宿舍.
那時候試過很多不同的方法.
來思考靈修這件事.
試過用很多不同的方法去靈修.
那時候試過看一本叫做.
揭露出來的能力的書.
是藤木思的.
那本書是一本很好的.
給自己屬靈反省的書.
那時候就寫到.

$^{841}$每天早上要跟著以前的屬靈的人.
五點鐘起床祈禱靈修.
那時候我跟著.
你不要幾點鐘睡.
起碼五點鐘起床靈修.
那時候剛學完文就用完文靈修.
試過用馮錦坤的羅馬書.
逐一靈修.
試過跑步靈修.
試過上天台裡面祈禱.
試過不靈修.
所以那時候正正就是一個.
不斷來思考靈修是一回事.
我自己以前是一個挺勤人.
以前.
現在是另一種方式.
以前是跟著靈修.
敬勤的方法來思考這件事.
後來我發覺我挺喜歡一個方法.
就是跑步靈修.
不知道大家有沒有試過.
跑步的時候一直祈禱.
甚至帶著這封.
聽聽廣道.
聽聽靈修分享.
跑步是一個跟上帝最親近的時間.
因為跑步的時候.
是我自己一直在呼吸.
完全是一個受眾物的狀態.
整個人心跳.
每分鐘跳130多下.
喘氣流汗.
那個狀態是一個最.
好像最貼近自己做一個.
本相就是一個被上帝所創造的人.
這樣的狀態.
後來我發覺.
原來靈修不是一定要在這裡.
靈修很多時候是一件事.
我發現靈修是我的生活裡面.

$^{881}$那時候我開始打算去外國讀書.
你知道我是研究巴特.
那時候我研究巴特的原因是這個主題.
他寫了一句很特別的說話.
祈禱和當基督徒是一樣的事情.
那時候就被這句說話吸引了.
為什麼呢.
什麼叫祈禱和當基督徒是一樣的事情呢.
開始發現.
祈禱不是純屬是閉上眼.
不是純屬靜下來.
或者做一個閱讀理解.
然後這樣的靜態活動.
而是我的生活.
我的整個人.
就是一種和上帝的關係.
然後我就去德國.
去德國的時候.
我每天四十五分鐘就去圖書館裡面工作.
看神學書.
看聖經.
預備講章.
那時候的生活是一至五.
跑步.
然後做運動.
去圖書館做論文.
回家.
一個很極度像作家的生活.
完全是很規律的生活.
那時候我開始慢慢衝破一個心理障礙.
這個大家先不要學.
要聽完我整個講話.
你才去反省這件事.
那時候我就開始開所謂的不靈修.
就是一般華人教會所定義的那種靈修.
我沒有做到.
這個是有點勇氣的.
因為你要突然衝破某些人叫你這樣做.
要這樣讀那本靈修書.
你也知道我現在是寫意道節的靈修學者.

$^{921}$我自己寫給別人看.
但我自己也不看自己的東西.
當然我會講更加詳細的進程.
那時候我就開始沒有這樣的靈修.
因為那時候我天天都在看聖經.
天天都有思想上帝.
天天都是來到.
整個生活都是一種.
我這樣理解.
整個生活都是和上帝的關係.
特別是卡爾巴特.
如果看卡爾巴特的話.
德文基本上沒有什麼靈修的字眼.
翻譯為Spirituality.
英文轉過來的德文字.
沒有一個具體的靈修的字眼.
我們會叫做敬虔性.
德文其實沒有一個靈性.
或者是這樣的字.
都是將英文的字變成德文.
所以開始就是跟著那個說法.
To pray and to be Christian.
是同一件事.
禱告或者做基督徒是一件事.
我當時就嘗試將靈修或者靈命.
融入我的教會.
在我的生活裡面.
我回來香港的2013年.
是懷著這樣的態度.
那時候我很恨那些靈修人.
心裡面不是恨.
我經常看別人的方法教靈修.
我會抗拒那些.
抗拒那些靈修大師.
通常靈修大師會說話很慢.
然後會很婦道形式地跟你說話.
我不是那種人.
但我自己讀的博士論文是做這些事的.
基本上都是一種禱告.
禱告基本上都是一種類似靈修的事.

$^{961}$我有一種反靈修的方法.
來理解靈修這件事.
那時候我懷著這種神學觀.
在建度裡面教書.
整個的生命就是靈修.
整個生活就是我的靈性的彰顯.
如果你看我的粉紅書.
其實都是大概會說這些東西的.
只不過我沒有說得那麼坦白.
不過我後來就去.
因為我在建度裡面教.
就教教會歷史.
教了大概.
基本上都教了六七年.
都會教教會歷史.
所以特別對於中世紀.
我稱之為神秘主義者.
我特別有興趣.
我就開始迷上神秘主義這兩樣東西.
神秘主義不是一些.
你今天好像很迷信的東西.
神秘主義反而是一些很知性的人.
他們覺得因為上帝是超越我們的.
所以他更加不能用我們的理性.
或者用我們的語言來理解他.
我發覺這幫中世紀的神秘主義.
特別是女性神秘主義.
一幫來自法國,德國,英國.
這些神秘主義的女性.
是一幫很前衛的人.
今天她們是用一些本土語言來寫書.
以前是拉丁文.
她們那時候就用一些法文,英文來寫東西.
以前是女性沒有什麼地位.
就用一些自己的語言.
來寫一些和上帝關係的文字出來.
我發覺原來這些東西是像我的東西.
因為你不知道大家有沒有聽過十格約翰.
就是十六世紀一個很出名的西班牙.
一個所謂的靈修的修道士.

$^{1001}$今天我們叫他靈修.
其實就是一個比較好聽的.
全部靈修的人都是神秘主義者.
今天你找一個在讀靈修博士的人.
他都是在研究那些人.
那些人正正就是神秘主義者.
我開始發覺原來我都喜歡這些東西.
因為你發覺.
剛才說十八藥王.
他就是用一些很.
你想想你要去說出和上帝關係的時候.
其實你用的語言是很不容易的.
不知道大家有沒有試過這樣.
聽到的時候.
你試過聽一些很精彩的讀法.
你抄下那些筆記.
你聽完之後發覺.
聽的時候會覺得很感動.
抄下的時候會發覺第一點.
愛主近身.
第二點多多祈禱.
第三 獻身基督.
發覺原來讀出來都沒有什麼特別.
但聽的時候發覺是很感動.
所以發覺和上帝的關係.
是一種只能夠用一些超越我們語言的方法.
才能夠記錄這種關係.
所以有一點點是私人或者神學的方式.
十八藥王只是一個西班牙私人.
如果你看回今天.
他在西班牙文學裡面是一個很重要的位置.
因為他用一些很特別的文學方式.
來形容這種和上帝的關係.
說到越南人.
除了和上帝的關係是一種生命的靈修之外.
我們更加要去找出一種和上帝的特別的時刻和關係.
而今天我們還教會靈修.
其實不夠神秘主義.
不夠靈修.
所以變成了一種閱讀理解.

$^{1041}$你發現靈修不過是閱讀理解.
甚至是詞語閱讀理解.
一堆文字叫你反省.
我經常覺得是很填鴨式的.
因為基本上我們去.
你想想什麼叫靈修.
就是你將某些靈修作者以前和上帝關係的東西.
人家講完你再討論一次.
少次人家將食物吃完再拿出來給你吃.
你將人家的東西來成為你的經歷.
這個可行的.
因為這個是上帝和你的關係.
但其實都是靠著人家的反省來成為你的反省.
所以今天其實很多東西要講.
因為可以講整個靈修的歷史.
神秘主義的一些重點.
如何奈若拉.
就是伊納爵.
如何將神秘主義規範化.
變成了今天所講的量產化的靈修之類.
但不講那麼遠.
今天我們想講的就是.
基本上我們想想.
如何重新來思考今天我們靈修這件事.
特別是全教會.
今天我們要講全教會.
如何理解我們的靈性靈修等等的主題.
所以我們首先要講一件事.
就是基本上我們的靈修或靈明.
這兩個是聖經沒有的字.
靈修和靈明.
基本上是聖經沒有的字.
這兩個字好像是過字.
可以嗎.
所以靈修和靈明是聖經沒有的.
不過我們經常會用到這兩個字.
我們和上帝的關係.
基本上是分兩部分.
如果我們用皇馬書來理解.
皇馬書頭十一章是什麼.

$^{1081}$就是上帝的工作.
上帝如何拆遣耶穌都不得到世上.
有聖靈來幫助我們.
如何拯救我們.
如何支持我們.
頭十一章就是上帝對我們的工作.
十二章開始是什麼.
我都講過很多次.
是人的倫理部分.
人如何去回應上帝.
所以我們和上帝的關係.
基本上是分開兩部分.
就是上帝如何在我們身上工作.
和我們如何去回應他.
所以基本上是雞和豉油的關係.
上帝出雞給我們所有東西.
而我們的靈性.
我們對上帝的回應.
就是這一點點的部分.
所以基本上我們的靈修.
是在因典里的關係.
所以今天首先要打明白.
靈修不是一種跑數.
基本上你能靈修的事.
已經暗示你在因典的關係里.
我們永遠都能夠懷著快樂.
懷著因典來回應上帝.
至於你有沒有回應.
如何回應.
這是一個極微小的部分.
上帝在基督里的工作.
已經是源源全全的.
主導整個的關係.
就算你今天沒有靈修.
這個因典仍然存在.
不過當然我們要問.
我們的回應是什麼意思.
我們如何回應上帝.
所以明顯有兩個不同的部分.
一個是客觀的部分.

$^{1121}$上帝已經做了.
你做不做也好.
上帝已經做了.
而我們所說的所謂的基督生活.
或者我們的靈性.
或者我們的靈修等等.
是一些我們回應上帝.
很微不足道的事物.
這是我們很重要的部分.
而對於我們對上帝的回應是怎麼樣.
基本上我們可以如何回應上帝.
如果將我們能夠回應上帝.
分為三大類的話.
基本上是分三大類.
這個在我們平時的講道里.
他將平時的講道分為三大類的話.
基本上離不開這三個東西.
是什麼呢.
我稱之為.
第一個就是遵從.
叫你做.
你能夠回應上帝.
就是跟從.
聽話的去遵從.
去做.
這個就是我們很大部分.
叫你傳福音.
叫你孝順父母.
叫你喜樂.
叫你敬拜.
這些全部都是遵從.
你能夠回應上帝.
可以做的事情.
有時有些東西不是只有做.
有時是一個信心的回應.
你信得過他.
你能夠用信心來回應他.
這些不關談為事.
所以我們基本上.
我們大部分華人教會.

$^{1161}$對於上帝的回應.
其實都是離不開這兩個.
遵從和信仰.
兩個部分.
你見到我還沒寫完第三個.
祈禱.
但我們發覺這三個東西.
其實都是一種.
互相重疊的關係.
即是說.
信仰等同於遵從.
遵從等同於祈禱.
祈禱等同於信仰.
三個是大家互為的一種關係.
所以我們通常都這麼理解.
你怎樣理解祈禱.
就是一種什麼.
就是一種遵從.
上帝叫你祈禱.
你便祈禱.
這個是對的.
因為你是聽從上帝的命令.
所以你便願意去回去祈禱會.
你願意去祈禱.
祈禱都是一種信心的表達.
我們願意相信祂.
所以你就會祈禱.
這個其實是卡巴神學.
但卡巴神學有一個很重要的補充.
其實反過來也是.
當我們願意去奉行上帝的時候.
當我們去嘗試去服從的時候.
其實這個本身也是一種祈禱.
我們願意去相信祂的時候.
其實這個也是一種祈禱.
什麼意思呢.
就是你願意去跟從耶穌.
去回應上帝的命令.
其實你帶著一種懇求的方式.
來去回應上帝.

$^{1201}$你願意去回教會.
不可停止聚會.
但你也帶著一種禱告的態度.
來去做這個命令.
因為發覺自己做不到.
當我們信的時候也是一樣.
我們願意信耶穌.
但我們只能夠說求主你幫助我.
因為我的信不足.
所以我們的信心是重要的.
但信心也成為了一種祈禱.
一種懇求的態度來去回應上帝.
所以你發覺.
任何事都可以是一種奉行.
任何事都可以是一種信心的表現.
任何事都可以成為一種禱告.
所以禱告可以這麼說.
正正是一種我們稱之為.
一切基督徒行動的底蘊.
一種任何的行動.
其實都是離不開禱告的本質.
包括你靈修.
意思不是叫你那一種祈禱.
不是叫靈修前祈禱.
或者靈修後祈禱那種祈禱.
而是你整個的生命.
你的靈命本身.
都是帶著一種祈禱的態度.
來去回應上帝.
主要是我願意去做.
但我做不到.
求你幫助我.
我願意相信你.
但我信心不夠.
求你幫助我.
基督徒其實最基本的一種行動.
就是這樣的祈禱.
不是一種聽話的祈禱.
或者命令的回應.
而是一種反過來.

$^{1241}$任何的奮鬥.
任何的信仰行動.
都是一種的禱告.
是當中的那樣.
所以我們的靈命.
是一種這樣的關係.
不知道你今天有沒有靈修.
不知道你這個星期有沒有靈修.
或者一個很軟弱的人.
他對於上帝那種的關係.
我覺得很多時候.
和上帝的關係.
是一種浪子回頭的關係.
或者是瑞利在聖殿裡面.
認罪的那種關係.
就是發覺你是all or nothing.
當你發覺自己無的時候.
突然間你可以去借著禱告.
回旋來親近他的時候.
你就是all.
你就是所有.
所以我們用這種理解來.
理解我們這種靈命態度.
因為我們華人教會.
常常都是會.
將靈命和靈修這個字扭曲了.
不知道大家怎麼看靈命.
你有沒有靈命.
你覺得靈命存不存在.
你說存在在哪裡.
靈命在哪裡.
靈命是不是在肚子里.
靈命是不是一些能吃的東西.
例如全對問你.
你這段時間靈命怎麼樣.
靈命是一些拿不出來看的東西.
但是我們發覺靈命是一種.
我用一種叫做唯命論的概念.
它不是真的有些東西叫靈命.
我們華人教會將靈命.

$^{1281}$看成一種.
有點大不慘的.
就像養鬼仔.
你養靈命在你生命裡面.
你要餵它.
你要養它.
你不餵它就會死.
靈命就會枯乾.
就像你養了貪婪歌詞.
你要養著它.
你靈命不好就靈修.
靈命就會加五.
它會培靈本就會加十.
整件事是百家組織分.
靈命是我們覺得可以存放的東西.
一樣東西叫靈命.
你要去養著它.
你要去培靈它.
要去修它.
所以發覺全對人的靈命好些.
因為它存放得很健康.
不知道等級一百多.
所以覺得靈命是積分.
我存放了很多.
所以都不存放.
其實這不是真正的靈命概念.
以前的人叫你不斷靈修.
就是為了好些靈命.
是對的.
靈修是跟你靈命有關係.
但不是積分的關係.
不是能夠存放到區塊鏈來換多些東西.
所以我們就想究竟我們所謂靈修.
我們平時所做的那些屬靈事情.
和我們靈命是什麼.
更加基本上是靈命.
如果靈命不是真實的鬼.
靈命其實是一種唯命論.
就等於熱氣.
熱氣是什麼.

$^{1321}$熱氣是在這裡嗎.
我熱氣.
熱氣是什麼.
就是一些氣.
也不是你肚子里的東西.
或者濕熱也是.
濕熱是不是肚子里黑色的東西.
濕熱只不過是我們給它一個名字.
把一些症狀給它一個名字.
所以靈命也一樣.
靈命不是一些我們來到.
覺得它真的存在一些實質的東西.
而是在量度你和上帝的關係.
所以基本上我們說.
靈命這個字是沒有的.
但其實它所說的東西是有的.
所謂靈命的靈字是什麼.
靈字其實就是性靈.
性靈在你生命裡面.
或者你的靈和上帝的靈之間的那種關係.
所以我很喜歡這兩段經文.
其實都是保羅在早期和後期.
早期在迦太叔.
後期在羅馬書所講的同樣的說話.
他說上帝就差他兒子的靈進入你們的心.
呼叫阿巴夫.
這是一個三一關係.
上帝的兒子差顯他們的靈進入你心裡.
同時你也會呼叫阿巴夫.
所以我們說呼叫是一種討告.
是一種呼喊.
我們成為上帝的兒女.
最基本的開始就是什麼.
就是你願意親口叫天父上帝做爸爸.
你就叫阿巴夫.
而阿巴夫是一種討告.
是一種生命的討告.
從此以後你任何事情.
整個生命.
你就願意用潛藏在生命裡面的阿巴夫的討告.

$^{1361}$去呼喊上帝.
它不是一個討告.
一個聽到聲音的討告.
而是你整個生命的態度.
從此以後就稱之為上帝的兒女.
所以這個是一個.
繼基督徒之後.
我們打堂講基督徒.
一個很重要的身份就是我們天父的兒女們.
我們是天父的兒女.
因為我們願意用我們的生命.
用一種討告來呼喊爸爸.
所以當我們願意呼喊他的時候.
我們的靈命就開始了.
因為上帝就差他兒子的靈進入你生命裡面.
你就出現了你有靈命這件事.
因為所謂靈命不是屬靈部分的生命.
而是一種你整個性靈在你生命裡面的生命.
所以所謂的靈命就是性靈在你生命裡面的一切.
如果你用這種看法來理解靈命的時候.
靈命就不是在你靈修那一刻.
靈命也不是在你上教會或者有沒有奉獻那一刻.
而是你整個的生命.
整個性靈在你生命裡面那種彰顯.
我以前也寫書寫過.
靈命是在你吸了那本靈修書之後開始.
你早上咬著三文治搭地鐵.
靈修之後.
你吸了那本聖經開始上班的時候.
那一刻正是你靈命的開始.
你怎麼做一個老闆.
你怎麼做一個員工.
你怎麼做一個女婿.
你怎麼做一個媳婦.
這些位置全部都是你靈命的彰顯.
所以這個就是我早期的靈命觀.
整個人的生命就是我的靈命.
我的靈命很壞.
想想一個很屬靈的人.
但是他很不環保.

$^{1401}$很奇怪.
他是一個很屬靈的人.
但是他發脾氣.
你覺得是不對勁的一件事.
因為所謂靈命好.
靈命好就是人品自然好.
你整個人.
你的人是好的應該會.
他不是經常祈禱過他.
但是他發脾氣.
他很玻璃心.
這樣就叫靈命不好.
因為不是一部分.
不是一個宗教的一部分.
而是你整個人的生命.
你怎麼去做一個好的下屬.
怎麼能夠盡忠職守在你生命里.
你見到前面那個人過閘.
你都不發脾氣罵他.
這些都是屬靈生命很重要的彰顯.
我經常說.
所謂屬靈九果.
怎麼能夠得到忍耐的果子.
方法不是靈修.
不是靈修這麼簡單.
而是真實地.
請你忍耐一下.
真實地有忍耐的生活和行動.
所以靈修或者你的宗教生活.
是離不開你整個的生命.
這是我回來香港之前的靈修觀.
整個人的生命就是靈修.
所以我開始沒有所謂的靈修.
其實不是不靈修.
而是將靈修融在我的生活里.
你發覺那些奴魂神父都是一樣的.
奴魂神父教的是什麼.
最後的究竟是什麼.
在修道院裡掃地都是靈修.
其實是一樣的道理.

$^{1441}$因為發覺靈修不是某種神聖時間.
而是你整個人生命的擴展.
如果你要靠你每天十五分鐘都沒有的靈修時間.
作為你靈修的量度單位.
你肯定靈命不好.
因為你最多是什麼意思.
24小時減15分鐘之後全部都沒有靈修.
就算我們這些傳道人.
有了聖經也好.
其實都是有限的時間.
你只能將你的靈命擴闊到整個生命里.
所以我就要說.
今天你聽不明白就算了.
這個活例用了很多年.
因為龍珠第三十三期裡面.
其中一個令我覺得很有啟發性的一幅圖畫.
如果你有看的話.
明白不明白就算了.
我不解釋那麼多了.
第一期就講到.
悟空和悟凡就是一個時間仿走出來的.
你會發現他們兩個人有什麼不同.
你會發現他們將超級殺人的狀態.
變成一個恆常的狀態.
如果你看龍珠就知道.
以前龍珠要爆炸就突然很厲害.
他說不要了.
我們將超級殺人的狀態變成一個恆常的狀態.
不是我要爆炸那一下才爆炸.
而是成為了生活的習慣.
所以他當時的想法.
他的比喻就類似這樣.
我們將上帝和上帝的關係.
不要局限在我們所謂的二道自建.
還是方法金船的靈修.
那十五分鐘都沒有的時間裡面.
而是你的靈明就是你整個的生活.
整個的生命的開始.
所以我們今天很想大家.
起碼這個我都是全職的教會來理解.

$^{1481}$我們如何理解我們靈明的其中一半部分.
這個一半.
因為後來我都有自己的反省.
我們的生命就是我們整個的宿命.
如何來在生活.
生命裡面來彰顯靈明.
才是關鍵.
不是那十五分鐘的靈修.
也不是那一個禮拜裡面.
一個小時回來的靈明.
這些是重要的.
但整個的靈明.
不單單在這些特定的時間裡面.
然後我們就說一點點.
深的那麼多的東西.
就是一種叫做關係性定論.
當我們去理解靈明的時候.
我們就要去理解性靈是什麼.
你要那麼多的深入的識別.
來認識什麼叫做性靈.
性靈是一些很不容易說的東西.
因為性靈是不能夠去摸到.
也不能夠那麼容易去論述它.
靈因派的好處就是將性靈的工作.
很具體化下去.
很容易明白.
性靈在你旁邊.
性靈感動我.
叫你怎樣怎樣怎樣.
什麼靈什麼區域邪靈.
怎樣搞到我們.
這就是將性靈的東西.
變成一些很具體化.
好處是很具體化.
但就很簡單.
太簡單.
我們不是這樣去看性靈.
性靈是一種我稱為關係的性靈.
性靈其實發覺.
性裡面每一段有關性靈.

$^{1521}$或上帝的靈經文.
其實都不是獨立出現的.
它總是會和性父或者性子.
或者和我們之間有關係.
因為性靈稱之為一個關係的建立者.
這裡我跳過了一些深入的東西.
不說了.
性靈可以是父和子之間的關係.
想想.
上帝是性靈感人的新耶穌.
耶穌洗禮的時候.
性靈就像甲子一樣降下.
性靈成為了.
上帝就拆遍性靈.
叫基督耶穌復活等等.
任何父和子的關係.
其實都是在性靈裡面的關係.
這種性靈關係正是很重要的.
而你發覺.
性靈裡面任何基督和我們的關係.
其實都在性靈裡面.
若不借助性靈.
就沒有人能夠稱耶穌為主.
所謂性靈其實就是基督耶穌的靈.
在我們的生命裡面等等.
性父也一樣.
天父上帝和世界的關係.
都是在靈裡面的關係.
而不是在靈裡面.
這裡我可以詳細地說.
上課時就不說這麼多.
總之性靈就是一種.
我們和天父.
或者父和子之間的關係.
他是任何關係的建立者.
包括人和人之間的關係都一樣.
我們同感一靈等等.
所以說來.
性靈就不是一些.
獨立於任何東西的一塊.

$^{1561}$我們嘗試不用靈因派的語言來說.
有時候都會問人.
你覺得性靈充滿更厲害.
還是跟隨基督更厲害.
誰說高級一點.
其實沒法想.
哪有比較.
因為性靈充滿就跟隨基督.
很難想象一個.
跟隨基督但不是性靈充滿.
很難想象他性靈充滿.
但不是跟隨耶穌.
所以我們嘗試將這些.
靈因派的很簡單的性靈語言.
變成一些能夠具體化的.
你不要問你自己有沒有性靈充滿.
這不是一種感覺那麼簡單.
問你有沒有跟隨耶穌.
我們是否效法基督.
還是說有一個所謂的成性靈.
其實是一樣的.
性靈叫你成性但同時也叫你效法耶穌.
一個客觀的毀身的行動.
正正是跟你主觀的.
那種屬靈經驗.
其實是分不開的.
就是說.
我再說一句.
你很難自我陶醉.
一個屬靈美滿的狀態.
而你很恨不恨.
你可以很愛主但你很恨不恨.
我不接受你這一套.
如果你是和神關係好的話.
我都說了.
靈命好人品自然好.
所以你一些很客觀的東西.
成為了我們能夠量度靈明的東西.
多於一些很簡單的屬靈觀念的語言.
所以我想說的是.

$^{1601}$今天我們去嘗試去說靈明的時候.
其實一些很具體的東西.
不是問你靈修多少次.
靈修多少次肯定會說什麼.
有什麼意義.
但靈修的意思並不是這樣能夠量度到.
可能你以前回教會的傳道.
問你這個禮拜有多少次靈修.
這個好像能夠將事情量化了.
但你究竟的生命行為是怎麼樣.
你個人的狀態是怎麼樣.
這個正是你的靈明的重要一點.
而更加重要的就是.
今天開始我人生後半段的靈明觀念.
我稱之為生命的靈.
這個在羅馬書第八章第二次裡面有.
本來說是生命的靈.
但生命其實不單單是一種教會神聖的靈.
它更加像我們的生命.
上帝的靈在我們的生命裡面.
我們就有氣息.
所以這個靈正是我們同一位的生命.
生命是叫我們有生命靈.
我們其中一本很重要的書.
我在美國那年翻譯的那本莫特曼的書.
就叫做生命的上帝.
聖經裡面有很多次數說到.
耶和華是永活的神.
Living God.
永活神不單單說他有永生或者永遠的生命.
而是一種生命的上帝.
因為上帝是生命的上帝.
而我們好像一個魚鹿.
學無溪水的奶奶.
一個飢渴的小鹿.
是學無上帝的生命靈.
所以我們的生命正是需要去追求這個Living God.
我們飢渴這位生命的上帝.
耶穌基督是叫人得生命.
並且得更豐盛.

$^{1641}$所以我們那種靈命.
其實不是計你有沒有躲起來靈修多少次.
而是你的生命.
如何來經歷很多不同生命裡面的事情.
我不知道你的靈命.
有些Milestone是如何去理解自己的靈命.
可能有些人的靈命.
可能是在一次車禍裡面出現.
然後轉變他的生命.
有些人在癌症裡面突然有很大的反省.
有些人寫過翻山書.
就有很多不同的東西.
有些人失戀.
然後突然發現上帝很需要等等.
我們的靈命永遠在我們的生命.
這些關口位裡面顯得重要.
因為我們的生命本身.
就是我們整個靈命一個很重要的場所.
所以我們嘗試去.
如果你說性靈是生命的靈的時候.
我們嘗試用一種.
另一種說法.
不是單單叫Spirituality.
而是叫Vitality.
就是我們的生命力.
你如何去面對你自己的人生.
面對現在移民的一個關口.
面對家裡人的疾病.
或者男女朋友之間的吵架.
或者夫妻之間的關係等等.
這些生命的那種量度和力度.
或者承受痛苦的力度.
正是我們屬靈生命裡面很重要的東西.
這些就很道地了.
這些就是非常具體的東西.
靈命不是一些你靈修次數.
或者你和神的關係是怎麼樣.
而是很具體地.
你和神的關係就在這個生命裡面.
如何面對這次的困難.

$^{1681}$這次的逆境.
這種生命的那種油韌力.
這種力量.
能夠抵抗很多傷口或者傷痛等等.
正是我們靈命的一種向度.
所以說亂世的靈性修持.
基本上我們呼出這大半年里.
其實都是在說類似的東西.
只不過是將這些的向度.
變成不同的主題裡面.
面對著這樣的時代.
我們的靈命是怎麼樣呢.
你以為你能夠靈修就沒事嗎.
你以為你靈得幾次修.
就可以面對今天的困難嗎.
或者當中上帝會和你說話.
但更重要的是.
你怎麼面對這個社會.
這種氣氛.
這種本身就是你的靈命.
生命靈.
正是你的性靈在你生命裡面.
怎麼能夠面對著你每一天.
每一刻裡面所面對的事情.
所以我們說.
當然如果幾年前我們在運動的期間裡面.
有很多不同的靈性操練.
其實都一樣.
就是你怎麼在生命裡面.
你在街上.
在中環.
在金鐘裡面.
怎麼能夠彰顯你的靈性.
到今天沒有了.
全部都沒有了.
但你的生命力仍然是面對著這個局勢.
只不過方法不同了.
可能會變成休閒.
可能變成減肥.
可能會變成減躁.

$^{1721}$等等不同的方面.
但其實都是一樣的.
就是你的生命.
怎麼在性靈裡面.
能夠面對這些一個一個的困難.
所以具體來說.
這種生命的靈性.
我們強調我們怎麼藉著性靈.
基督耶穌自己.
來面對著我們的生命.
耶穌也說過.
祂來叫人得生命.
豐盛生命.
這正正就是我們靈命的意思.
所以我們說.
我們嘗試去打破身體和靈魂的異原.
靈修不是只是靜.
閉上眼睛去大自然.
這些東西.
而是你整個的身體.
包括你自己的人是怎麼樣.
不是只是處理著屬靈的部分.
而是你整個人.
因為身體和靈魂是分不開的.
我們就不只是去陪一些屬靈部分的自己.
而是整個人.
你有沒有熬夜.
吃得好不好.
還有沒有胃痛.
這些全部都是屬靈生命的東西.
不要把它分開.
當然受苦主義是有意思的.
所以我說減肥.
這是另一種意思.
但其實是兩個不同的向度.
熱愛生命.
自己生命和別人的生命.
對於生命的熱愛.
對於人的熱愛.
正是我們這種熱愛.

$^{1761}$是一個很重要的素質.
另外就是.
精神學的感覺.
我們嘗試打開我們的五感.
來經歷這種屬靈部分.
也就是說.
不只是閉上眼睛.
我們的屬靈部分.
我們的禱告也不是只是閉上眼睛.
而是嘗試張開你的五感.
嘗試去看多一點.
聽多一點.
接受多一點.
思考多一點這個世界.
你這個靈性.
其實是和世界建立.
一種連結的方法.
所以靈修.
大家可能會來長洲.
因為長洲對你來說是一個新的地方.
但對我來說.
我不是在長洲靈修.
因為長洲是一個我自己住的地方.
可能我會去旺角靈修.
我們如何能夠連結這個世界.
如何能夠借助一種新的五感的衝擊.
嘗試重新去認識天賦的世界等等.
都是一些我們可以思考的課題.
這些全部都可以說很多東西的.
我今天很簡單的版本.
所以我們的生命.
我們的生活.
正正就是我們屬靈的場所.
Live energy.
可以翻譯做生命.
生命就是你生活.
所以你的靈性.
就是你如何生活.
你的生活.
你的壓力.

$^{1801}$你如何調節自己的生命節奏.
有沒有休閒等等.
這些全部都是我們靈性裡面.
可以經歷的課題.
最後我想說一點.
最後一些具體的修持部分.
我說了.
七年前我回來香港的時候.
我就說我的生活就是靈修.
但我發現神秘主義者.
我告訴他們.
我們在當中仍然是值得.
有一些特別的moment.
而這些moment正正是我們的洞見.
我們打破我們恆常裡面的生活.
突然你靜一靜下來.
去反思自己的生命.
這一下正正就是靈修的本意.
五百年前伊納爵.
他將一些很難跟的神秘主義的物觀.
變成一些很規範化的物觀主題.
譬如你寫一本書叫《神操》.
不是跑馬那些.
而是special exercise.
他說今天你就物想謙卑.
謙卑就是你今天物想的重心.
方法不重要.
誰寫的書都不重要.
而是你真的去想東西那一下才重要.
所以經常都覺得靈修是一種慰藏.
你有沒有能夠在你生活里抽出來.
能夠反省一下自己.
讓上帝的說話.
讓上帝那一下的安靜.
讓你能夠重新抽離一下.
從而去反省自己的生命.
這一下正正是靈修的重點.
所以不知道弟兄有沒有這樣的moment.
有時候有露台.
突然夜晚吃完飯.

$^{1841}$拿著一罐啤酒在露台裡邊.
吹吹風喝一口啤酒.
呃一下出來之後.
那一下的反省是很relaxing.
不是relaxing.
是將你那個生命抽出來.
那一下反省很重要.
所以這一下是重要的.
這一下才是靈修的本意.
相反那本看靈修書的方法.
不是不好.
因為總是有一些隱瞞方法.
不是每個人都能夠這麼容易反省自己.
看靈修書好像能夠幫助你反省.
但是過度依賴它的時候.
如果你能夠不靠它的話.
想東西是很重要的.
所以我想說.
找一個你自己合適的方法.
能夠帶給你自己有生命的refresh.
和那種rethink.
正正是我們靈修的重點.
如果你不是看書的人.
如果你不是理解很好的那些人.
不妨.
所以一定要跟那些.
要自建的那些.
你可以試一下.
用另一個方法.
能夠重新讓自己能夠articulate到生命.
為什麼人們這麼流行畫畫靈修.
還是書法靈修.
我自己也喜歡書法靈修.
我曾經一段時間也練書法.
就是覺得這一下是能夠讓你用一些不能.
因為都說了.
語言是一種很難articulate到上帝.
Art藝術是能夠幫我們articulate到.
自己和上帝的關係和感受.
這種的自省.

$^{1881}$可能是唱歌.
可能是畫畫.
可能是做手工等等.
可以成為你靈修的部分.
如果你是art的人.
如果你是運動人.
跑步.
游泳.
可能是一種方法.
只要你的生命能夠給你一種.
稱之為新.
一種新的衝擊的時候.
如果你發覺你靈修了二十年.
都沒有什麼起色.
不妨轉一種方法.
不用一定要跟那些靈修東西.
嘗試去找一種能夠和上帝親近.
一種你能夠articulate到和上帝說話的方法.
當然,就算是動有動的靈性.
靜有靜的靈性.
整個的重點就是你能夠給自己一個反省的時候.
今天香港人其實是很不懂得反省.
很不懂得能夠抽出來.
每一天去反省一下自己.
而我在中國大陸就太多反省了.
是過多.
我們起碼有一段時間能夠從.
當你是那種多層線感覺就會發覺太過反省.
想太多,想多了.
沒辦法想那麼多.
但你平時一個普通的頂智妹.
可能很忙的時候.
找一個給自己反省的moment.
這一下正正就是我們的生命開始.
重點不是你靈得多就自然好.
而是你那一下的refreshment.
那一下的更新正正就是我們每一次的意義.
所以不需要去問我這次有沒有得著.
還是有沒有學到什麼.
現在靈修不是學東西.

$^{1921}$而是純粹的去沈澱反省自己的生命.
想的未必都一定是你自己所謂的屬靈的topic.
純粹是你生命裡面遇見的topic.
不過在上帝裡面.
重新來重整自己的生命.
我們先講到這裡.
我們有很多問題可以和大家談.
特別是一些很具體的屬靈的情況.
可以和大家談談.
我們先談到這裡.
餵.
這杯酒今天很棒.
當然了.
剛才看你口渴.
剛才你說到有些地方.
和我做生意來說都很不同.
你說不要做那麼多.
但我們做餐廳最重要的是套餐.
單點人們覺得不值.
套餐會更值.
剛才你說不要做那麼多.
最重要是專一點適合自己.
這件事怎麼告訴別人.
不要做那麼多.
而單點會更加有得著.
不是不要做那麼多.
而是不要以為自己做得多.
就一定是好.
不是來得多就叫好.
而是吃得對才對.
自來也沒有什麼喜悅.
沒有什麼快樂.
不如找另一間.
我覺得這些都可以.
其實大家怎麼知道多或不多呢.
大家可以談談.
你做得多或少.
可能你的少對他來說也很多.
不過這麼說.
我很怕外面的人.

$^{1961}$陳茵安叫別人不要零收.
其實不關事.
因為大家很著重那些優先.
那就麻煩了.
你一年看不看完整本聖經.
夫子今周要多少次零收才對.
不要問這些.
重點不在這裡.
零收一個星期.
但每次看完就算了.
很貼合式的.
其實也沒有什麼意思.
大家怎麼看呢.
大家有什麼靈命上的奇難雜症.
可以和大家談談.
或者分享一下.
平時你怎麼零收.
或者你零收不到的原因.
這可能共鳴大點.
幫朋友問也可以.
因為這些事情大家都會有機會.
也會經歷.
其實零收可以很開放.
可以任何時候任何地方.
你做的事什麼都可以.
只不過是同神掛勾.
去關心神裡面.
或者教訓.
或者神的話.
甚至他喜不喜悅這件事.
他怎麼看.
其實是不是這樣呢.
我想回應一下.
廣義來說就是我回香港早期的看法.
沒錯.
你的生命就是你的靈命.
是一樣東西.
你怎麼做人.
怎麼做你的靈修.
或者你的靈性.

$^{2001}$是一樣東西.
你不能分開.
你最怕我零收就叫零元.
其他東西就好像分割.
應該融合在一起.
但是那時候又不代表.
都是這樣做.
那是不是真的不需要零收呢.
那個particular是重要的.
每一天.
為什麼人們不來長洲.
因為你特別特別.
我特意來這裡.
還有些東西.
所謂神聖之地.
或者神聖的時間.
這種東西是重要的.
因為它是在打破你的恆常.
從而令到你可以.
對你的恆常去作出一些反省.
所以兩樣東西加在一起.
就是我們所說的.
所謂的零收.
或者屬靈操練.
因為屬靈操練.
有一種我們基督教.
新教的.
其實是兩頭不到岸.
天主教最強.
就是因為我們說的那種.
沒有停過的.
那種contemplative.
500年前的一幫reformer.
他們重新delete了所有的東西.
而我們開始慢慢去發展.
一些清教徒式的零收.
就是一種discipline.
discipline是好的.
因為habit is power.
你的習慣是一種力量.

$^{2041}$所以你繼續做某些東西.
是有一定程度的力量.
但還教會就將這種.
habitual power.
這種習慣變成了一種很律法式的東西.
我們以為這樣跟著做.
就是了.
其實又不可以錯的.
你教一個小朋友.
當然叫他每天都溫書.
但溫書本身就不是.
你學習的東西.
是動見的.
是insight.
零收是整個生命.
但有時候我們那一刻的insight.
或者那種reflection.
那種反省.
是一種particular的時間.
我自己推翻自己.
七年前所講.
不是生命就是零收.
就不需要零收.
所謂零收或者那種particular的時間.
是有它的重要性.
以前的神秘主義者.
就是這樣.
親近上帝之類.
這個就是另一個topic.
因為講一些本體問題.
我們的靈魂.
如何跟上帝相遇.
這些很不同的東西.
基督教就不講這些.
我們是聚人.
所以就不講這些靈魂.
如何去上帝面前契合.
我們用聖靈來articulate這件事.
但不是純屬一種行為.
不是純屬清教徒這樣做.

$^{2081}$就算有了.
大概兩部分.
或者通常零收的困難是什麼呢?.
大家認為是什麼呢?.
沒有什麼特別困難.
沒有什麼特別零收.
後面.
因為剛才說anytime anywhere.
好像有不同形式都可以零收.
其實是不是不一定有讀經這件事在裡面.
就算我有默想到神.
因為傳統教會教的.
其實是零收.
讀經是一個foundation.
開始.
然後透過讀經去默想神的話語.
有些應用.
我覺得因為我們.
用聖經是最穩定的.
用聖經就不用說不會錯.
但我想說也不是一定.
昨天我說不要叫人用聖經零收.
不是這個意思.
我們大概是這樣.
上帝跟你說話.
當然有很多不同方式.
你以前有沒有試過通聖識.
揭開那一頁.
上帝和你看什麼.
為什麼不行.
你上帝和你看什麼.
為什麼不行.
你上帝和你看什麼.
為什麼不行.
當然可以.
我不反對這一套.
你用通聖識揭開.
今天上神和我說什麼.
你不看吧.
當然不好.

$^{2121}$上帝有很多不同方式和我們說話.
詩歌,反省,聖經.
甚至你和朋友聊天之後有些反省也是.
當然你是不是這樣就叫零收呢.
我都說反過來.
你的生命本來就是零收.
所以你不需要問那些是不是零收.
你本身整個人的生命就是上帝的零和你一起的方法.
問題是長遠來說你用什麼方法.
當然長遠來說是不是每次都沒有聖經.
當然這個不建議.
但是你說是不是一定要有.
沒有就不叫零收呢.
我就不是這樣看.
大家平時會怎麼零收.
有什麼想不通的地方.
最難在哪裡拍.
我接下去剛才John說的關於聖經的參與.
因為剛才有一個powerpoint是關於關係性的正靈的部分.
其實很多時候零收的時候我們都想尋求上帝的介入和指引.
但是很多時候都會和弟子妹妹去討論.
你有多敏感聖靈的提醒.
這就牽扯到你和聖靈的關係.
聖靈幫我們做分辨.
聖靈提醒我們.
很多時候過去接觸弟兄姐妹.
問如何明白上帝的旨意.
不外乎都是兩至三個處境.
一個處境就是我應不應該轉供.
第二個就是我應不應該和他開始拍拖.
第三個就是應不應該有比人生重要的決定.
那時候就尋求聖經的指引.
那時候就會遇到一個困難.
就是聖經說的一些東西不關於聖經.
但是在關係建立或者敏感度的時候.
這個才是你靈明要反映的.
對於我們來說.
平時聖經說什麼.
讓我們認識上帝是一個什麼上帝.
上帝想期望我們的人生.

$^{2161}$或者我們的信仰表達是什麼.
但是如果你平時沒有聖經的支持.
或者瞭解聖經對我們的提示的時候.
有什麼引領你去做分辨呢.
這個就是一個問號.
所以我覺得更加好就是.
把讀經習慣和靈修分開.
你覺得讀經是加起來的.
不是說靈修讀經.
讀經就是讀經.
讀經就是看多一點聖經.
看熟一點 想想.
但是靈修真的不是一種那麼量化的東西.
反而是有一種生命的反省.
和上帝反省自己和世界之間三者的關係.
這個比你學習一些東西還要重要.
還有問題嗎.
我想兩位可不可以分享一下.
香港人現在.
我自己在靈修上有困難.
每天都很忙 很趕.
其實靈修是一種靜態的東西.
需要靜下來.
我覺得是一個靜態的東西才能做到.
兩位有沒有分享有什麼方法.
讓香港人多一種途徑去靈修.
除了說.
唯一可以靜下來的就是.
每個星期的主日崇拜.
最長的時間靜下來.
其餘的時間.
其實都有個難度.
是嗎 比如說晚上臨睡前.
那些是祈禱時間.
不夠 而且你又會很累.
我覺得祈禱和靈修是不夠的.
祈禱時期都可以睡著.
說到底都是不夠時間.
太累 太忙.
整天困在這個狀態里.

$^{2201}$我覺得兩位是傳道人.
一個是神學院教授.
我覺得你們的時間是很足夠的.
說到底是時間.
時間這東西.
我經常落在這個光景里.
有時候很難.
比如說退休.
是很正的一件事.
起碼要兩三天.
你先要起來.
然後你要靜下來.
才能進入那個狀態.
但當你覺得時間足夠.
你又回到一個現實的世界里.
周而復始都是往返.
剛才John說到伊臘教的精神訓練.
數年初年.
我教基督教里的靈命培育課程.
都會用伊臘教的精神訓練.
其實當中最後的教學點.
其實都是和現代教學方法.
五感教育相近.
不說得太複雜.
其實人有不同的學習方式.
一個是視覺.
一個是聽.
第三個是做.
有些人是文字敏感.
看,預度理解會快一點.
有些人看不太快.
但聽人說的時候.
就馬上上腦.
就已經有圖案在腦子里.
有些人聽和看都不敏感.
你告訴他怎麼做.
他做完一次.
他就會變通.
其實要找到自己的方法接收.
這個是基本單元.

$^{2241}$所以剛才說的.
嘗試去尋找自己一個可以接收.
或者對你來說是互通的方法.
第二件事就是牽涉到時間的參與.
我從事教育.
就是你找到一個方法之後.
就開始要花時間.
每個訓練是要用練習.
所以你會明白.
一個叫做練習讓它完美的意思.
就是你不嘗試不斷去打磨自己的方法.
這樣的話很多時候你都不會靈活.
你牽涉到多少時間.
有些人用短時間就能夠做到.
有些人用長時間就能做到.
我自己的做法就是.
我自己視覺和聽都是敏感的.
因為有時看的時間就限制了.
剛才說要靜的時候才看得快和看得好.
我大部分時間都在街上.
或者要去不同的地方.
我會選擇走路的.
我每天走路的時間超過一個多小時.
有時我不坐車.
選擇走回家.
或者其他事情.
我走的時候就會聽我想聽的東西.
譬如聽podcast.
聽廣東話的聖經.
或者聽一些我覺得有質素的文章.
或者是一些講道的時候.
那就是我自己的靈修時間.
不一定一次過走一個半小時.
可能有時我回家要洗碗.
我都會在洗碗的15分鐘里聽一個podcast.
那個podcast對我來說是一些信息.
我有興趣想融入的.
就將一個regular或者particular的時間.
就斬件在不同的地方.
去融入我收集的或者我靈修的方式里.

$^{2281}$讓自己有調整的空間.
這個都是融會了.
就是選擇一個你自己接收的舒服和比較efficient的渠道.
再將一些時間散在不同的地方.
讓那件事成為你生活的一部分.
所以這個都是我和John相近的靈修方法.
還有我都覺得回到我們講第一部分.
就是你整個生命就是你的靈性.
這種觀念都是很重要的.
如果是那些趁你工作以外或者很忙以外.
就肯定是少的.
肯定少於傳道人.
少於教會的人.
但其實我們要的最基本的方法就是.
你不要覺得那些東西不是屬靈的東西.
我經常都寫不完那本書.
那本書的名字就想到了.
就是張開眼睛去禱告.
其實你打開眼睛的那個世界.
其實就是你禱告的世界.
不是你閉上眼睛那一刻才是天父那一刻的世界.
你面對現在.
面對你的工作.
那個辦公室.
那些人事.
或者很日常的那些excel那些東西.
那裡其實都是上帝在的.
所以我們需要練的就是練你那個生命的動策.
那裡都看到一些反省的位置.
都看到人與人之間有些不同.
那些未必只是看到屬靈的東西.
但那些其實都是屬靈的東西.
就是人與人之間的關係.
或者是一些不同的reflection.
所以重點都是.
你與其你squeeze時間.
不如你學習一下在這些位置裡面去洞察一下上帝.
當然這一刻是不容易的.
我們覺得總是帶著自然是容易一點發現上帝的.
但我們會正正就是在一些.

$^{2321}$外表不像有上帝的地方去找上帝.
這一刻是對城市人來說是重要的.
如果你不是你怎麼辦.
你這麼多年在城市裡面生活.
如果你去瑞士的話.
瑞士真的那些半山那些.
海底那些.
那些真的很好.
在那裡長大的人.
以前我在歐洲的時候.
每年去一次瑞士開會.
那裡真的那些藍天白雲.
有些很漂亮的草地.
那些蒼蠅都慢一點.
那些蒼蠅可以被你拍死.
那些人不習慣那些蒼蠅不會被人殺死.
那些地方很棒.
但我們就需要用另一個方法去找上帝.
上帝就不一定在那些地方裡面出現.
在你的小廚房裡.
我以前就覺得.
我很喜歡去信和那些地方.
信和也能找上帝.
你能夠在那裡.
很多人很稠密的地方去找上帝.
這一刻是我們去操練的地方.
網上也有些問題.
我們就抽了一條問.
就是Cliff Tom 問.
如何判斷在行動中的他正在靈修.
進入與神同在.
例如當他在跑步的時候.
聽到靈修經文或者祈隱.
其實他有時候會覺得他在跑步.
而祈隱是在走路.
在浪費自己.
令他自己沒有覺得那麼辛苦.
這個是牽涉到一個很大的範疇.
我們叫專屬.
專屬其實我也想找機會講道講.

$^{2361}$專屬是一個很重要的靈性或者做人.
所以跑步的時候專注在那裡.
反而我覺得是容易的.
跑步沒有東西做.
反而你繼續去繼續.
按照靈修歷史來說.
以前的沙漠教父都是這樣.
他們以前不住的禱告.
不住的禱告就是不斷重復一段的禱文.
就是這樣說.
所以有這些傳統.
所以跑步的時候可以不斷重復一些禱文.
其實也可以.
重點不是在禱文那裡.
而是你整個人的專注在某個上帝那裡.
我以前剛剛初信的時候.
那時候真的很勤奮.
平時都在想上帝.
開會都在想上帝.
那就不行了.
你會分心.
但意思不是這樣.
這個是要粗連的.
就是理解你日常的生活是一個天賦世界.
你去理解.
你幻想一下耶穌就在你前面.
當然這個是要練的.
但我們覺得就是說.
你生活裡面能夠變成你能夠專注的熟能生活.
你有沒有補充.
這個問題是很大的一個課題.
從教育來說.
米哈利有一個教育理論叫flow.
中文叫做心流.
在台灣翻譯是2020年2月初版.
很厚的一本書.
其實flow的過程是什麼.
不是flowchurch的flow.
是真的flow.
時間過得很快.

$^{2401}$他的感覺就是.
當你學習一件事.
專注過程當中.
時間過得很快.
而你不覺得那件事是受時間限制.
而你是投入在其中的.
就像我講到的時候.
講過進入多個zone.
你能夠掌握那個方式.
去讓自己更大的吸收.
我自己都享受跑步的時候.
聽討論或者聽東西的時候.
不知不覺就這樣跑了幾k.
因為是浪費了.
你覺得沒那麼辛苦.
但你就投入了在那個經文或者信息思考裡面.
因為你進入了那個心流的狀態.
所以重點是.
你能夠找到一件事可以專注.
你可能覺得不專注.
其實你慢慢習慣了那個方式的時候.
你會投入那個專注.
是不是還有網上的問題.
第一排都有一個.
第一排.
我是真的幫人問.
因為有個大兄想問.
他也在看直播.
他說想瞭解一下.
就靈修方面如何分辨出思考的想法.
並不是自我的想法.
而是主跟我說的對話.
他有個例子.
就是當我走路靈修的時候.
忽然想到一些東西.
思考到一些東西.
怎樣才知道這件事不是他自己的想法.
是上帝在跟我說呢.
所以為什麼我會說這個關係性靈論.
因為如果按照性靈論來看.

$^{2441}$性靈做的事情不是在我們之外.
其實性靈是雙方的.
它是建立在兩邊的.
我們說靈修和基督耶穌的關係.
就是在性靈裡面.
它是連結我們和耶穌基督自己.
所以我們就不像靈因派.
靈因派是什麼.
就是性靈要激怒我就要跌下來.
就好像我沒得選擇.
性靈其實是一些我們和上帝之間的關係的時候.
當你有那種想法的時候.
其實這件事本身不是單單出於你.
怎樣說呢.
性靈感動你的話.
你不會覺得很不滿.
你會成為你的想法.
明白嗎.
所以性靈不是完全是第三者.
它是我們的一個subjective god.
就是它讓我們能夠有這樣的感受和想法.
所以當然你敬畏地去想.
這個是不是上帝的想法呢.
這個是好的.
但其實上帝讓你這樣想.
這個不是偶然的事.
只要你懷著一種尋求討告的心.
來開放給上帝去求問.
這些所謂自己想法的事.
其實就是經過時間引證.
這些都是上帝的想法.
這些問題是不是讀神學.
是不是蒙照.
一樣的.
上帝讓你這樣想.
不要覺得這些事.
是突然從天而降.
聽到聲音對我說.
很多時候我們這樣想.
其實是必然的過程.

$^{2481}$因為上帝感動你.
你必然會想到一些事.
所以這些是很個別的.
我會再和他談談那些情況.
我會很多時候和他再討論.
那個是不是他自己想法的時候.
我會回到《約翰·加爾文》.
基督教義義卷一.
第一章第一段落.
裡面說的是.
那個想法或者那個瞭解.
有沒有認識上帝多些.
有沒有認識自己多些.
有沒有認識神和人之間的關係.
應該不會有衝突.
而那樣東西應該會豐富你.
對上帝或者對你.
剛才John說到.
在那個生活處境當中.
應該是參與在其中.
我仍然相信那句話.
就是聖靈一直都看著那件事.
不是偶發性的.
仍然不斷地發生在我們生命當中.
我都很少說那些.
我不是這樣說的.
聖靈神叫我這樣說的.
這些我最怕.
叫你付出.
我沒試過這樣說.
神就叫我怎樣怎樣.
因為其實中間我們去翻譯.
你這樣的想法.
其實帶你都有祈禱的感覺.
其實都是神讓我這樣想的.
很少說這個完全是屬於原始的他者.
我是完全不想的.
其實神叫我來的.
或者我不會這樣說.
網上是不是也有問題.

$^{2521}$來自Cheng Tech Team的留言.
各位有事目者都會問你的靈明有沒有成長.
想問靈明成長會不會和相信耶穌的連支有掛勾呢?.
例如你相信耶穌的時間是多久呢?.
但是上教會很少.
讀聖經很少.
又會不會和靈明成長有關係呢?.
或者上教會上得很長.
又會不會代表他的靈明比其他人高呢?.
所以有事目者問這個問題就覺得很大疑惑.
我就覺得不是的.
我覺得基本上一方面不是.
意思就是說他沒有關係.
不過人做了很久.
其實會有進步.
這樣也是.
很複雜的那件事.
這個很大題.
花了很多時間很多元素.
我都說不是做積分.
所以不是說做基督徒越久你的靈明就好了.
不過你做基督徒越久.
你又確實可以好了.
因為你發覺和基督的關係深厚了.
你又會有更多的經驗等等.
但是否必然呢?當然不是.
很多人越做越差.
我昨天才發了個Facebook post.
不是這樣的.
很多人初信是很好.
但是你做人做久了.
其實你會有很多人生經驗.
很多體會也會多了.
你學的東西也會多了.
但是還是那句.
看你人有沒有反省.
你個人是有反省的話.
那你就會突然浪子回頭.
那你那一刻就會突然好了.
但浪子回家後會不會比第二次好呢?.

$^{2561}$不一定.
所以我覺得沒有直接關係.
人生就是這樣.
人生裡面有很多這些.
你可以走錯路.
和上帝的關係一樣.
很多時候初信的時候.
多數是all nothing.
要麼就有在基督裡面.
要麼就不在基督裡面.
那些可能是你的聖經知識加深.
可能是你禱告多了經文.
花款更多.
或者是你熟悉語言多了.
或者是侍奉技巧好了.
或者人生的成苦心了.
但是是不是靈明好了呢?.
我覺得不一定是.
我嘗試用一個例子.
看看大家會不會應對到這個情況.
有健身生會籍這個問題.
有很多人有參與健身院的會籍.
是不是去多健身室就會有好的體態呢?.
其實你知道不是的.
有很多人經常去.
但是身體是沒有變的.
因為他在裡面沒有做事.
還有是不是懂得很多健身理論就會有的身體呢?.
都不是的.
因為他懂得但他不做.
相仿的情況就是.
回教會是否很靈明呢?.
都未必的.
是不是懂得很多查經書的資料.
或者讀經書代表很靈明呢?.
都未必的.
有時就是他做了一個口語.
所以我以往一段時間.
長期都在健身室.
都有很多靈修的提醒.

$^{2601}$有很多人都問我如何操馬甲線.
但我說你不要問我.
因為我沒有做過.
不如你找一個真的操馬甲線.
他告訴你.
如何熬過頭六個星期的熱身過程.
我就熬不了.
我操不了.
但你找一個操得到的人.
那個精神訓練同樣很重要.
不是時間坐得久就行.
在時間當中沒有工作是不會解決問題的.
所以不要迷信時間.
我們有一堂會講這個題目.
就是講成性的問題.
因為我們說.
Science of Certification.
就是成性的科學.
就是你借助靈修.
什麼靈性的增長.
就令你能夠越來越好.
所以叫成性的Science.
這個說法是真是假.
這個我們會再探討.
但我覺得這是一個成性的問題.
我們會如何越來越像耶穌呢.
這個問題是一個值得我們再花多一堂去思考的課題.
現場有沒有其他問題.
網上有沒有其他問題.
有一個來自Cliff Thomas的回應.
他說若然行動的目的是為了令人專注.
專注可以是看劇,工作,打球等等.
其實是否只要是專注的訓練.
而不是尋求神的同在呢.
還有他還有一個問題.
若然專注的方向是進入神同在.
除了鬆動狀況之外.
有沒有明顯的指標.
可以令他判斷自己是在靈修而不是專注呢.
其實剛才潘SIR說到流動.

$^{2641}$最容易流動的活動是什麼呢.
就是打遊戲.
打遊戲很容易流動.
因為視覺,聽覺都很專注在那一方面.
容易被吸引到那裡.
我覺得什麼叫專注上帝呢.
這個是很深的議題.
但我覺得當我們專注在某些東西的時候.
你都可以享受上帝某些東西.
當你理解整個生命是你的靈明的時候.
當你專注在上帝的世界的時候.
意思是上帝的世界不是俗世負面的世界.
這些東西都可以成為你去享受生命.
就是上帝給你生命的那種世界.
這就叫生命的靈性.
因為生命本身就是上帝賜予的.
所以這種方法其實都是一種去專注上帝的方法.
上帝,世界和我三個的關係.
而打遊戲是差一點的.
因為打遊戲和上帝的世界是比較遠一點的.
所以打遊戲你會發覺人是溫溫的.
然後又好像有點心虛.
但是當你專注在一些活動.
例如行山或者是世界上的工作.
你會感覺到上帝的多一點.
所以我覺得專注上帝可以是一些物觀.
純粹想上帝.
但是當你在生命裡面去注意的時候.
其實都是其中一個方式.
當然這不是唯一和全部.
但起碼你去跑步.
你去能夠在世界裡面來享受世界生命.
我覺得這是其中一種方法.
我不知道有沒有理解錯.
剛才聽回應問題的時候.
就會感覺到好像是怕在坊間的場景.
或者一些情況會好像是.
投入的時候就未必是很靈修的做法.
但是我自己的看法就未必是這樣的.
所以大前提就是.

$^{2681}$我相信這是天賦世界.
每一點都是上帝的管理範圍之內.
而我覺得從事基督教教育.
面對最大的困難就是.
過去教會的教導很多時候都有不同的規範.
例如一些經文的背誦.
或者一些形式的教導方式.
令到你可以進級.
但最大的問題就是.
在經文的背誦.
或者不同的課程進級的時候.
缺乏很多場景題.
場景題就是會友或者信徒.
遇到那個場景的時候.
他的反應,他的靈明.
如何可以支持他作出那個反應.
或者如何作出那個判斷.
甚至如何用聖經的原則去做好那件事.
其實是少了一些場景題去應對.
所以回到回應今天的內容.
我們希望靈明散居於不同的生活場景.
從而提升我們面對場景題的挑戰.
或者場景遇到困難的時候.
我們都會問.
如果我們有信仰的人.
有基督信仰的人.
我們如何應對那件事呢.
這個思考過程是重要的.
回應剛才John內容.
在一個亂世當中做一個反省.
就要想一想.
我自己是一個喜歡想場景題的人.
譬如見到有人撞車.
我都會去看看.
那裡有人吵架,我就會去聽.
有時過程當中.
如果我真的面對這樣的情況.
我應該怎麼辦呢.
信仰很多時候是很突發.
但如果能夠在當中有個先驗.

$^{2721}$或者先理解一個處境的時候.
其實會幫助你去應對自己的情節.
有沒有其他問題.
在後面.
我想問一下.
怎麼說呢.
我看過這個Re-Order Church.
他們經常說.
在靈修放空很危險.
又說很多靈因派的弟兄姊妹.
可能是遇到鬼.
我就開始有點害怕.
其實我是一個很容易放空的人.
因為我很喜歡發呆.
但是當你一放空.
一去想.
好像內心有些聲音.
我就會害怕.
咦,是什麼呢.
我會有個疑問.
就是.
我現在會看聖經.
也會祈禱.
但是.
一說要靜下來.
自己去安靜獨處.
去放空自己的時候.
我就開始有點害怕.
會不會有些邪靈入侵之類的.
因為有時候.
你會看到一些靈因派的弟兄姊妹.
很虔誠很敬虔.
但是你又會看到一些靈因派的朋友.
不只是靈因派.
有些人會看到一些很奇怪的行為出現.
我又分不清.
所以.
這件事我想問一下.
會不會有些方法可以.
令到自己很肯定.

$^{2761}$這個是聖靈的.
就是我.
在獨處想的時候.
那個交談的對象.
不是一些奇奇怪怪的東西呢.
其實是這樣的.
就是.
如果.
你是一個人.
你叫陶.
然後你想去.
你的心,你的靈.
你的靈魂,你的靈.
去尋求上帝.
你去做一些事.
那.
怎麼可能會遇見邪靈呢.
因為你的心是想尋求上帝.
如果.
我不知道怎麼理解放空.
我都有.
我沒記得聽YouTube的那些東西.
但是我覺得.
我自己那個就.
我自己對於這個.
不是他.
就是放空就會遇到邪靈.
這種說法.
什麼叫空.
因為我們是有聖靈在我們心.
在《加泰書第四章》那裡說.
聖靈就在你心靈裡面.
就成為基督徒.
你天父的兒女.
怎麼會空到連那個都沒有呢.
所以我們是天父的兒女.
怎麼空都不會空.
反而.
反而我都有一點像.
太久沒靜.

$^{2801}$就突然一靜就.
一靜就靜不下來.
突然就想很多東西.
所以我就不覺得會那麼容易.
空到連聖靈都沒有.
或者你連同神的關都沒有.
你是天父的兒女.
你在這裡.
你的土告生命裡面.
都在依靠著他,都去找他.
怎麼會撥錯線去到邪靈那裡呢.
聖靈上帝不會那麼差.
不會讓你找到.
所以我就不太相信.
靈恩派的說法.
怎麼邪靈就怎麼搞我們.
我不是不相信他們的存在.
但我更加相信上帝的掌權.
所以我覺得就不需要怕.
當你.
當然是怕的.
但我經常都覺得.
你相信有神.
更加相信.
多過那些東西.
所以就真的不需要擔心.
怕放空空到地步會被邪靈.
這個我們以前回教會.
都聽過這樣說.
什麼瑜伽空到什麼.
我不覺得我們空到.
因為我們不是空在我們生命里.
我簡短的.
我都不是很接受到.
那個放空.
是什麼一回事.
因為.
你不是真的放空.
因為你總有些事情是做的.
我自己具體的說.

$^{2841}$那段時間你會有經文.
那時候你會有一些.
所謂的靈修資料.
去填滿.
你說你法外的話.
就真的好像剛才John.
說的那樣.
你法外的時候.
最後你都會有聖靈在你心裡.
這個是上帝的.
肯定來的.
還有他是.
沒有離開過當中.
我真的很相信.
就是.
那些鬼.
那些靈.
或者邪靈.
看到我們.
應該是他們怕我們.
而不是我們怕他們.
不過我們可能都會怕.
因為我們沒有試過.
那樣東西那麼厲害.
所以怕鬼.
很多時候都是被.
看了鬼片.
怕了.
所以.
我最記得.
我兩個兒子的時候.
我就問.
我第一件事問他們.
小時候.
你信不信有鬼.
信啊電視有的.
我說信鬼不是因為電視告訴你有鬼.
是因為聖經裡面說到有鬼.
那有鬼怕不怕.
我不怕.

$^{2881}$因為有耶穌在裡面.
我小時候.
他小時候我已經告訴你.
你怕是很多人都會的.
但是你記住.
耶穌在裡面.
聖靈在裡面.
這個很重要.
讓他們明白.
從小就教他們.
聖靈內在他們心裡.
這個是一個很重要的憑據.
武器.
這個很重要.
好.
今天差不多了.
下個月有什麼.
這個月是青嶺特飲.
因為是夏天.
下個月就九月了.
九月就秋天了.
可能有.
下個月見.
拜拜.
鄭處女.
好吧.
現在我們就給您熱 packets.
好的.
那個.
冷卻一下.
好.
那好.
哪個.
老頓.
你.
是.
好了.
好.
吧.
好.

$^{2921}$你來吧.
好.
那我走了.
好.
那我走了.
好.
那我走了.
好.
那我走了.
好.
那我走了.
好.
那我走了.
好.
那我走了.
好.
那我走了.
好.
那我走了.
好.
那我走了.
好.
那我走了.
好.
那我走了.
好.
那我走了.
好.
那我走了.
好.
那我走了.
好.
那我走了.
好.
那我走了.
好.
那我走了.
好.
那我走了.
好.

$^{2961}$那我走了.
好.
那我走了.
好.
那我走了.
好.
那我走了.
好.
那我走了.
好.
那我走了.
好.
那我走了.
好.
那我走了.
好.
那我走了.
好.
那我走了.
好.
那我走了.
好.
\newpage



\section{}
\label{sec:akT8yKiTNTo}
\textbf{【這是最好的時代:給香港基督徒的神學八課】第5課:一根刺的人|20210918 [akT8yKiTNTo]}
\newline
\newline
連結: \href{https://youtube.com/watch?v=akT8yKiTNTo}{\texttt{ https://youtube.com/watch?v=akT8yKiTNTo}} ~~~~ 語音日期: 2021-09-18 
\newline
\newline
\hyperref[sec:Wv0tPAVEIA8]{\small{< < < PREV SERMON < < <}}
~
\hyperref[sec:index_chronic]{\small{[返順時目]}}
~
\hyperref[sec:index_scriptual]{\small{[返順卷目]}}
~
\hyperref[sec:TgQ5_ITPOW8]{\small{> > > NEXT SERMON > > >}}
\newline
\newline
$^{1}$(第1集).
(第2集).
(第3集).
(第1集).
(第2集).
(第3集).
(第4集).
(第5集).
(第6集).
(第7集).
(第8集).
(第9集).
(第10集).
(第11集).
(第12集).
(第13集).
(第14集).
(第16集).
(第17集).
(第18集).
(第19集).
(第20集).
(第21集).
(第22集).
(第23集).
(第24集).
(第25集).
(第26集).
(第27集).
(第28集).
(第29集).
(第30集).
(第31集).
(第32集).
(第33集).
(第34集).
(第35集).
(第36集).
(第38集).
(香港我心心的故鄉).

$^{41}$(這律讓我生長).
(有我喜歡的朋友共陽光).
(路上人在跑過 逃過).
(幹勁靜默欣賞).
(這律有許多可圈換法案).
(日日星光 香港 香港).
(你永遠是曾夢開).
(香港 香港 你那小娘娘).
(山頂看小島 水裡流浪).
(似是玩得很開心).
(看向那海鷗飛過自由港).
(海邊看小島 全萬種).
(處處搖眼星光).
(這個市區的吸引無法擋).
(日日星光 香港 香港).
(所有我憑念夢想).
(香港 香港 叫我不以為望).
(香港我真心的抱憾).
(這律讓我伸張).
(有我喜歡的親友共陽光).
(路上人在跑過 逃過).
(幹勁靜默欣賞).
(這律有許多可圈換法案).
(日日星光 香港 你永遠是曾夢開).
(香港 香港 你那小娘娘).
(香港 香港 再有我同你吻人).
(香港 香港 叫我不以為望).
(香港 香港 你永遠是曾夢開).
每個年代都有每個年代的神學.
作為土生土長的香港人.
我們似乎正在經歷一個最差的年代.
不過往往在最差的年代.
我們才能夠經歷福音信仰的最好.
就是我們一起從聖經裡面學習.
怎樣做這個年代裡面的香港基督徒.
這是最好的時代.
給香港基督徒的神學百科.
各位弟兄姊妹晚安.
無論是在現場的.
或者我們在YouTube裡面的.

$^{81}$來到我們神學百科的第幾課.
第五課.
這課是很特別.
如果之前我們看過我們那四課的時候.
我們都講過什麼是福音.
我們什麼是基督徒.
講到教會.
講到我們的靈性.
去到金堂.
我覺得我們特別是加進去的.
如果你看一般教會的門訓材料.
或者是一些初信的百科.
其實都沒有這個主題.
我覺得我們作為全聖教會.
我們在新的年代裡面.
我們做一些最基本的東西.
我很想加進去.
作為我們每一個全聖教會的弟兄姊妹.
都可以首先去思考的課題.
就是一根刺人.
我們今天會講一些比較具體的東西.
我們會講一下我們跟人的關係.
講一下我們跟教會的弟兄姊妹.
當我們去群體相處的時候.
我們會遭遇到的課題.
其實這個課題是很重要的.
基本上我們在教會裡面.
大家都是這樣的經驗.
作為全聖教會的弟兄姊妹.
可能都試過很多不同.
在教會群體裡面的相處.
當中是牽涉到很多東西.
一些是藝術.
都牽涉到一些基本的聖經神學的理解.
所以這個課題.
我自己很想去特意去講的一課.
如果我們說去延續.
我們第三課的時候所講的教會觀.
大家記不記得.
我們所講的教會觀.

$^{121}$是一個稱之為黑暗教會觀.
我們從鋼琴鍵裡面的黑字來做定義.
我們覺得教會群體裡面.
我們人和人之間.
我們做一個教會群體.
其實可以充滿著很多的盲點.
充滿著很多不理想的狀態.
但這個是我們人.
或者教會在地上一個最基本的狀態.
教會是一個美視群體.
都是我們的信仰.
這個永遠都在恩典裡面.
所以延續著這種教會觀.
我們知道我們的教會.
甚至地上的教會.
其實是它的本相.
永遠都是一班有問題的人.
我們這樣開始的時候.
我們就很具體想到.
我們在教會裡面.
作為基督徒.
怎樣能夠將這種軟弱.
這種人和人之間很多的不同的摩擦.
成為我們所謂的門訓裡面的其中一課.
這個是很重要的.
很多門訓.
很多不同的材料都會講那個理想出來.
應該要怎樣做.
但我覺得其實在那個應該.
那個理想和實際之間的落差.
其實我們更加要探討的就是那個落差.
所以這一課我們會一起去探討.
當我們成為基督徒之後.
當我們願意去決志成為一個見證基督的人.
我們去明白福音的意義.
明白教會.
明白我們的靈性之後.
我們回到我們一個很具體的生活的處境裡面.
怎樣和基督徒去相處呢.
所以今天我就會去用我們之前.

$^{161}$我之前都經常講的題目.
就是一根刺.
當然保羅在哥倫多後書裡所講.
他說你為這事我三次求過主.
叫這刺離開我.
他對我說我的恩典夠你用的.
因為我的能力是在人的軟弱上顯得完全.
所以我更喜歡誇自己的軟弱.
好叫基督的能力復辟我.
我為基督的緣故就以軟弱.
凌辱.
急難.
迫不.
困苦為何喜樂的.
因我什麼時候軟弱.
什麼時候就剛強了.
保羅在教會裡面所經歷的.
我之前講到都講過.
那根刺似乎更加的痛.
不是因為身體的軟弱.
而是人和人之間的關係.
就算是在哥倫多教會裡面.
當保羅去寫聖經書卷的時候.
人事的關係.
在教會裡面.
人事關係已經出現了很實際的狀況.
所以對保羅來說.
這根刺其實是一根很不容易拔掉的刺.
因為這根刺不單單是關於他自己.
而是關乎於他整個面對的教會群體.
當時教會出現了很多不同的人.
大家都是基督徒.
可能大家都有這樣的經驗.
最難處理的可能就是基督徒和基督徒之間的關係.
很多時候我發覺很特別.
明明大家都是好人.
為什麼會出現這麼多問題呢.
如果其中一方是壞人.
就很容易理解這件事.
因為他是壞人.

$^{201}$所以就有這個問題.
或者我是壞人.
自然就會出現問題.
但是當兩個都願意做好人的人.
或者願意覺得自己是好人的人.
當他們相處的時候.
竟然會出現很多很多問題的時候.
這就是我們今天會思考的問題.
所以當中我們會嘗試.
簡單地講一些最基本的神學原因.
今天不會講太多神學.
但基本上都會講一下.
我們對這個情況的一些理解.
當然我們都回到罪的題目.
我覺得我們在這個年代裡.
我們很多時候都會講罪.
但其實都不是很真正.
能夠好好地去講這個課題.
如果你要講一個罪觀.
這個都是神學課題.
罪論.
基本上可以很簡單地講.
由所謂的原罪.
或者是次組犯罪.
我們人人都有罪.
我們很多年前的決戰時期.
我都明白.
人人都犯了罪.
不過我們這個罪的題目.
其實不能單單純粹作為一個神學理論.
作為一個知識去理解.
我們真的嘗試將這個罪的課題.
放在我們的生活裡面.
特別是在我們教會和弟子妹相處的裡面.
所以我們今天會花十分鐘時間.
講一下少少罪論.
今天是有些厭煩的.
可能對一些做了很久的基督徒來說.
又會講罪.
但我覺得我們將這個罪的知識.

$^{241}$需要一個很正確的拿捏和定位.
我們很多年前講了太多的罪.
報道會,培靈會都叫你認罪悔改.
以致我們這十年八載都比較少提這個題目.
但我覺得這個題目其實永遠都是重要的.
特別是不知道你知不知道.
如果看回初期教會的時間.
看回整個罪觀的發展的時候.
你都知道天主教有一個懺悔.
有告解這件事.
天主教反而是比較容易處理.
當它有告解和懺悔.
因為很實際.
你犯罪你就告解.
這件事告解完之後就清了.
真的清了.
一件錯事就去神父那裡告解.
上帝藉著教會的權柄去免去你的罪.
我們新教有些搞事的.
新教是沒有告解禮.
基本上我們說耶穌基督是十字架.
一次過去免去我們一生的罪.
但這樣是令我們很強調罪之餘.
又很不強調罪.
變成我們的罪的問題.
好像永遠都不能夠有一個很具體的方案.
能夠去解決它.
特別是初期教會.
初期教會不知大家知不知道.
當未有告解保屬禮之前.
其實初期教會的信徒他們是很怕犯罪的.
因為他們以前還沒有一套很完整的神學.
去說基督徒犯罪之後究竟有什麼解決方案.
究竟有多少個階段.
以前覺得只有三次階段.
你想想理想主義只有三次犯罪機會.
犯罪之後就沒有了.
再犯罪就會落地獄.
所以初期教會他們對於罪的很審慎.
所以你看到頭兩三百年.

$^{281}$基督徒是非常堅貞他們的信仰.
他們都是非常委神.
因為其中一個原因是他們覺得不可以犯罪.
他們對於罪是非常小心和謹慎.
因為只有三次階段.
後來出現了保屬禮.
懺悔告解才成為一種解救.
所以以前你可能聽過.
君薩丁皇帝以前是一個隨從的洗禮神父.
因為他是臨死才洗禮的.
他要臨死才洗禮.
為什麼要這樣做.
因為他不想用高塔.
所以要拖到洗禮的客人臨死前.
馬上急救幫他洗完禮就死了.
所以這是一個這樣的方法.
所以對於罪這個課題.
其實我們基督徒是很曖昧的.
可能大家聽過.
馬丁路德我們新教的一個很重要的發起人.
一個很重要的神託.
對於我們基督徒來說.
同時是罪人.
同時是義人.
這個可能大家都聽過的神託.
這是一個很辯證的看法.
路德是一個很嚴謹的人.
是一個對自己非常高要求的人.
所以對於罪這個來說.
是一個很存在性的理解.
一個很追求完美的人.
但發現自己一點都不完美.
所以他才會很痛苦.
他發現自己很多時候仍然會得罪上帝.
得罪了人.
所以今天我們做基督徒的時候.
其實背著同時是罪人.
同時是義人這個神託.
其實是可能模糊的.
你和你說你是罪人.

$^{321}$你會覺得我是罪人.
大家都是罪人.
但我同時也是義人.
所以這個似乎是你不能夠很容易消化的.
天主教是很不同的.
是很均真的.
犯了罪就是罪人.
告完解沒有事就重新再來.
重新再來做好一點.
做錯了就真的要去懺悔.
所以對於罪這個課題.
我們其實還不夠真正的去認識.
特別是在去年.
當我寫《庚次》的時間裡.
我覺得對罪這個課題.
特別有很深的理解.
我覺得在社會上很多所謂的罪人.
甚至比基督徒更加好.
因為罪人往往是很複雜的.
今天我會說幾個不同的神學概念.
不想說得那麼複雜.
因為很多時候我們發覺有幾樣東西.
一個叫做罪者和被罪者.
這個是我們香港的神學人馮煒文先生提出來的概念.
所以這是很土產本土的概念.
被罪者.
他說我們每個人不單單是罪人.
更加是一個被罪者.
就是說我們同時間是罪人.
不過我們也不要忘記.
我們同時也是被罪所傷害的人.
所以今天我們在群體裡.
無論是在你家裡.
或者是社會.
或者是教會.
我們每個人同時也是罪人.
同時也是被罪傷害的人.
所以我很喜歡我自己寫的那一句.
我們無論是多麼的病態.
在這兩條十字架.

$^{361}$我們每個人都是或多或少都有一定程度的病態.
因為是被罪所影響.
可能是父母.
可能是成長.
可能是社會的壓抑.
令到我們都是被罪所影響的人.
從而成為一個罪人.
不要說自己說別人.
所以當我們發現在教育群體裡.
那個人是一個罪人的時候.
他同時也是被罪所捆綁.
被罪所深深傷害的人.
所以令到這個課題更加複雜.
當我們說罪人的時候.
我們不單單是說一個理論上.
阿當的後裔所以犯罪.
我們同時也是被罪傷害.
從而做出一些很錯的行為.
所以人和人之間的傷害其實是很複雜.
不是純粹一個壞人和一個好人那麼簡單.
而是往往更加深層次.
而另一個古著的概念就是Sin of Omission.
和Sin of Commission.
所謂Sin of Commission是甚麼意思.
原來那個罪不是單單是你做了.
我們不單單說犯罪.
而是當你沒有做到應該要做的事的時候.
這個也成為了一個罪.
最明顯就是.
這個故事很失望.
頭兩個人基本上是沒有做事.
其實這個冷漠.
這個不加以幫助.
或者是沒有行動.
這個也是一種罪.
所以更加複雜.
原來罪不單單是你做了甚麼.
而是在你應該要做的時候.
而你沒有做到的時候.
這個也在群體裡面構成了一個罪.

$^{401}$所以你會發覺.
很多時候我們不是有意的.
當我們在群體生活裡面的時候.
我們傷害別人的時間裡.
可能只是沒有做一個應該要做的事.
可能是沒有說一句話.
可能是沒有回一個訊息.
可能是沒有說早安.
可能是沒有做一個應該要做的安慰.
這個也可能構成一些傷害.
這個就成為了一種罪.
所以當我們重新在群體裡面.
作為一個神學課題去想到罪的時間.
罪不是一個單單的理論.
而是在一個很具體的生活裡面.
都是唯一.
唯有在群體生活裡面.
你才能夠去明白和認識罪.
所以這個很具體.
你離開了群體的時間.
你沒有辦法真正實踐出.
我不犯罪的操練.
一個人是減少犯罪機會的.
不言不說.
一個人在獨處的時間裡面.
確實犯罪機會是少很多的.
但你只能夠在群體裡面.
在教會裡面.
在社會裡面.
才能夠真正去具體.
讓自己去跟隨耶穌基督.
就是對抗罪.
所以這是第二個比較.
令我們今天很不容易的課題.
就是一個Sin of Omission的課題.
不為.
純粹的無為.
都是一些罪.
第三個更加複雜.
我稱之為無知.

$^{441}$其實這是一個頗有趣的課題.
我們下課會說Passion.
下課就會說Passion的課題.
天主教一個很重要的神奴大師.
Thomas Aquinas.
他對於罪是一個很有系統的分析.
大家有沒有聽過七宗罪.
七宗罪.
你可能看過那套電影.
原來我最近才知道.
七宗罪其實應該是七罪宗.
七宗罪的意思不是.
Seven Cases of Sin.
宗字不是解作Case.
不是七宗罪.
而是罪宗本身的一個term.
什麼意思呢.
原來他認為.
罪是有七個不同的源頭.
一切的罪都來自這七個不同的源頭.
什麼呢.
貪心.
驕傲.
還有說謊.
這些罪其實是一切罪惡的源頭.
還有影響.
所以七宗罪其實不是七宗Case的罪.
而是七個不同的源頭.
Thomas Aquinas說到.
他更深層次地說.
原來罪是有分內和外的原因.
今天不說外.
外來自於什麼.
撒旦.
人.
試探之類.
或者是我們的意志.
但是他說.
原來罪是有三個內在的原因.
這個值得我們去認識.

$^{481}$原來一個人犯罪是來自三個不同的原因.
今天說幾個.
其中一個就是passion.
因為你衝動.
情.
情就是解作你的情感.
或者你的情緒.
這個我們下課會再說.
因為衝動.
衰衝動.
所以就會犯罪.
所以這是內在原因.
這個就是惡意.
就是邪惡的意志.
想壞事.
這個很明顯.
原來除了衝動和惡意之外.
一個很重要的內在原因.
就是ignorant.
就是無知.
天主教翻譯為愚昧.
但我覺得用無知更加直接.
因為ignorant這個字.
很多時候我們犯罪.
不是我們.
我講個人的.
人犯罪.
其實是出於無知.
就是說我們對事情的真相.
是不足夠的理解.
這是一個很重要的因素.
很多時候不是因為你有惡意.
也不是因為你有衝動.
因為人的理性能夠可以壓衝動.
令你不去犯罪.
但其中一個很重要的原因.
就是出於ignorance.
對於事情的真相.
是沒有足夠的理解.
這些事情其實會導致人犯罪.

$^{521}$當然不是什麼都知道.
湯姆·桑瓦尼講得很清楚.
他說不是什麼都要知道.
但你應該要知道的.
而你不知道.
這下就是罪.
所以你不需要什麼都很厲害.
或者知道整盤棋子怎麼走.
或者整個的東西.
但是當我們有些基本的資訊.
或者對於事情沒有足夠的理解.
而導致的罪.
一些錯誤的事情.
這個是罪.
所以說起來.
很多的罪其實不是理想的.
因為你仍然以為自己有足夠的理性.
和足夠的善意.
不過是因為出冤.
我們沒有足夠的對於事實的認清.
所以我覺得基督徒和基督徒之間.
很多時候是因為這個問題.
我們對於一個別人的看法.
沒有足夠的理解.
對於事情的真相.
沒有足夠的認識.
對於一個客觀的事實.
沒有足夠的知識的時候.
這個ignorance.
在聖經裡面就是愚昧人.
這種的愚昧.
是令到我們犯罪的.
所以去到這樣的層次的時候.
一個人犯罪.
其實是更加多層面的.
很多時候人和人之間的傷害.
往往在這些位置是更加容易走出來.
更加傷害的.
我經常都覺得.
特別這幾年.

$^{561}$一個惡人.
一個很明顯的罪人.
更加容易更加好.
但很多時候基督徒.
因為出於ignorance的緣故.
所做出來的傷害.
是更加深.
所以這個令人很沮喪.
因為原來很多時候的傷害.
是因為這樣的緣故.
所以我曾經在Facebook上寫過.
明目張膽去做一個罪人.
可能會更加舒服.
當一個基督徒很多時候.
出於無知的緣故去傷害人.
這個反而是一個更大的傷害.
所以我們不要去低估.
在教會裡面.
人和人之間這種罪的傷害.
教會裡面有沒有壞人呢.
有沒有充滿惡意的人呢.
我覺得是有的.
但不是很多.
衝動的也不少.
但無知的更多.
對於事情的真相沒有足夠的理解.
懷著善意.
懷著理性.
但不足夠的知識.
會造成很多不同類型的罪和傷害.
這是我們教群體裡面.
很重要的具體處境.
所以我很喜歡.
我也在不同的場合提過.
一個美國神學家叫拉奎尼布.
他是一個非常真實的人.
一個基督教現實主義者.
所以他對罪從來都不會很吝嗇.
他考慮罪的因素在整個神學論述裡面.
在倫理裡面.

$^{601}$他寫很少太多理想的東西.
相反他寫了很多實際的東西.
來探討我們要如何考慮.
罪的元素在我們的社會裡面.
我們要如何做一個好行為出來.
所以他介紹了三本不同的書.
大家可以去讀.
一本叫做Moral Man and Immoral Society.
Study in Ethics and Politics.
這裡說到一個Moral.
道德的人與不道德的社會.
他說原來一群有道德的人.
聚在一起就變成了一個不道德的社會.
很奇怪的.
如果將這個課題壓縮到變成一個群體.
其實都是一樣的.
大家都是好人.
但聚在一起.
卻變成一個充滿著很多問題的群體.
所以這本書是非常精彩.
說到如何從個人去到社群裡面.
即是那個之間.
如何從大家都懷著好意.
同時導致很多惡的問題.
第二本就是一個很重要的書.
被譽為20本最好的書之一.
即是非洲書裡面.
The Nature and Descent of Man.
一本很簡單.
重提基督教的罪觀.
在這個世界裡面.
當時是1939年.
即是二戰的時間.
當飛機飛來飛去.
炸來炸去的時候.
尼泊爾就在英國裡面.
他給了一個很重要的Nature.
就是關於人的罪.
這個罪觀.
所謂很老土的論述.

$^{641}$在二戰當中是令人發生心醒的.
亦都說到一個問題.
即是去探討到人的罪.
究竟是一件甚麼事情.
第三本更加喜歡的.
就是The Truth of Light and the Truth of Darkness.
光明之子與黑暗之子.
他說.
用回耶穌的那個比喻.
即是你們的光明之子.
要去參考今世之治.
你們要學習今世之治這樣做人.
他說基督徒.
很多的光明之子.
不是壞的.
是蠢的.
即是你不夠那些黑暗之子那麼聰明.
所以他說.
我們一班光明之子.
更加需要去學習.
來明白到這個世界黑暗之子的玩法.
所以他從來都很強調.
這個世界和社會的黑暗.
不過這個不是悲觀的.
這些看法仍然是說.
我們怎樣能夠在這麼多盲點裡.
能夠活出我們應該有的倫理.
所以我希望大家都是.
在這個作為我們Flow Church的其中一個門訓課程裡.
大家去重視這個課題.
我們不單單去說理想群體應該怎樣做.
更加想我們實際上應該怎樣去面對赤裸裸.
活生生的一個群體.
怎樣去相處.
所以你問有什麼可以做呢.
其實都沒有什麼可以做.
唯有上主光照我們.
真的.
唯有怎樣去發現這些盲點呢.
我很喜歡彼得這段故事.

$^{681}$當耶穌去天神跡.
在海裡叫西門彼得.
打了很多雨回來的時候.
你猜都猜不到彼得突然爆了一句話出來.
完全不對劇本.
打完雨突然爆了一句話出來.
在耶穌的膝前說主啊離開我,我是個罪人.
這句話是完全沒有什麼context.
前面後面都沒有什麼直接關乎於說到罪.
還是什麼問題.
但是當耶穌去顯現他的榮耀的時間裡.
我喜歡這樣說.
當彼得去發現上帝的時間裡.
這是一個很重要的illumination,一個光照.
唯有耶穌能夠在我們的生命裡光照我們.
我們能夠發現自己是一個真正的罪人.
甚至乎可以說是一個PK.
是一個壞人,是一個賤人.
有時候我不知道大家怎樣理解自己.
當你做基督徒做了很久.
有沒有發現到偶爾望著鏡子.
我真的是一個PK.
或者你很熟悉的人.
你這一下真的很PK.
偶爾去想一下自己.
是否一個PK都很重要.
這種光照.
唯有上帝讓我們看到自己的錯誤和盲點.
是非常重要的.
所以我覺得這是我們可能很重要的一個方法.
來解救.
我之前也提過.
我覺得一間健康的教會.
很簡單.
基本上每一間教會都有這些問題.
就是人與人之間的關係出事.
就是唐主任和傳道人.
傳道人和鄧小姐.
鄧小姐與唐主任.
很多不同的組合.

$^{721}$總是會出事.
所以一間健康的教會.
是不能說不存在.
但一個動態過程.
你只能夠不斷地去減低苦頭.
減低自己對人的傷害.
第二就是強化玻璃心.
這是一個防洪工程.
加強我們的心.
我不是說大玻璃心.
而是減低一些受傷心靈的問題.
減少苦頭釋放出來的機會.
真人.
我們很多時候在教會裡面.
說一句話.
做一些事情.
一個舉動.
其實往往會傷害到人.
這個我自己也經常出事.
我今天活到41歲.
仍然在學.
不過確實這幾年是好一點.
苦頭是少釋放了.
所以我想大家也是值得去問.
如何能夠在人與人之間相處.
如何能夠減少苦頭釋放出來的機會.
不斷地去查看.
我今天有沒有不小心說了一些話.
傷害了人.
某個舉動會不會令人不開心.
有些比較說笑的話.
可不可以不需要說.
這是我們第一件事的工程.
在教會裡面要減少苦頭.
第二就是強化玻璃心.
不過大家不要誤會.
我不是批評玻璃心.
基本上任何心都是玻璃.
心與苦頭永遠不能比較.
一旦碰上去心就會爆裂.

$^{761}$強化玻璃心不是叫你不要傷害.
不要有任何感受.
而是當你被人劈到之後.
可以令到你不會在痛楚裡面太久.
甚至可以快點復原.
當然這是一個理論.
如何能夠做到很多不是在課堂上說到的事.
如何能夠讓心臟更加強大.
我們能夠面對弟兄姊妹的說話.
不會太過玻璃心受到傷害.
我都說過我跟你吃過火鍋.
我跟你說雞翼未熟你都不開心.
這樣就糟糕了.
別人不同意你你都不開心.
這其實很不容易.
所以如何能夠仍然.
其中一個位置就是相信弟兄姊妹的善意.
起碼第一次是.
起碼第一次的時候.
去相信弟兄姊妹是有善意的.
我很喜歡Serendi的一首歌.
這首歌叫做《屍》.
《我不知道你是否一個賤人》.
這首歌很有趣.
大家搜尋一下.
基本上我只是懷疑你是一個賤人.
我不會太快認定你是一個賤人.
我跟朋友談過.
他都說你可能是一個賤人.
這很重要.
不要太快去認定對方是奸惡著.
不是說不要.
我不是叫你永遠相信阿門.
因為他是好人.
起碼你第一次.
我都給一個信任.
你是好人來的.
我覺得你有善意的.
第二三次才慢慢去評價原來你是賤人.
這很重要.

$^{801}$不要太快去玻璃心.
心臟更加強大.
當然我這樣說是沒有責任的.
永遠是耶穌基督的恩典.
能夠幫助我們.
能夠面對這些問題.
所以我覺得客觀來說.
只有兩個方法.
弟兄姊妹之間相處.
減少苦頭的出現.
和去強化我們心臟.
是一個很重要.
能夠讓全教會健康.
兩個很重要的工程.
所以我覺得弟兄姊妹之間的相處.
都是兩個原則.
我自己都是激的人.
說話都不是很審慎的人.
不像潘Sir.
我很珍惜坦誠.
坦誠是我們基督徒很重要的美德.
在橫教會裡面.
基本上我希望Fortress能夠成為我們的DNA.
能夠去坦誠.
不需要裝模作樣.
不需要裝成基督徒.
不需要裝成自己很好的傳道人.
意思是真的算是好的.
不需要裝出來.
大家之間說話能夠說成實話.
這是我自己一向都覺得很重要.
無論是在我們侍奉裡面.
或者是在我們的群體裡面.
能夠將心裡面的說話坦誠地說.
不偽裝.
是很重要的事情.
不過我自己在35歲之後.
覺得這句說話其實是有下一句的.
坦誠是作為很重要的美德.
是一個第一步.

$^{841}$大家要去堅持的東西.
不過卻是要有第二步.
我最近很喜歡這句經文.
羅馬書的經文.
保羅大家都很認識的.
十二章裡面.
愛人不可虛假.
惡要厭惡.
善要親人.
愛弟兄要彼此親熱.
恭敬人要彼此推養.
我以前是很不喜歡後半句的.
要恭敬人要彼此推養.
就想起以前的橫教會的東西.
牧師牧師牧師牧師.
不好意思不好意思.
你那些.
大家你先你先你先.
我經常想起這些課題.
我都很怕橫教會的文化.
很有禮貌.
都是棉女針那些.
最恐怖.
但其實原文不是這個意思.
恭敬不是純粹基督教文化的禮貌.
愛人當然不可以虛假.
惡要厭惡.
善要親近.
這個是大家要坦誠.
我們要表裡如一去做.
但保羅後面那兩個很重要.
原文裡面就說.
兄弟的愛.
不是單單叫愛弟兄.
而是一個真正的兄弟的愛.
我們要用兄弟的愛來彼此對待.
大家真的在群體裡面.
要真是一個brotherly的關係.
彼此以尊榮.
在原文裡面就是說.

$^{881}$你要去honor.
給一個尊榮來看待對方.
所以這個是其實後面那兩個說話.
只是補充頭那兩句說話.
沒錯我們是要惡要厭惡.
善要親近.
但你仍然要怎樣.
仍然要真是視對方為一個兄弟.
family的對待.
並且是要honor他.
你要遷就他.
你要去愛著他.
這一下是重要的.
怎樣能夠可以去.
在我們的言語上.
仍然在每一刻裡面.
去honor對方.
去遷就對方.
去說出你不高興.
你認為錯的事情.
所以我覺得人長大了.
發覺原來我們弟子妹之間.
這些事情是很聚集的.
當然我們不會回到.
以前華人教會的那些位置.
但是在惡要厭惡.
善要親近.
和彼此去尊重和honor.
那個拿捏位置是很重要.
去拿那個智慧.
怎樣能夠不會被你的passion反抬.
不會因為ignorance.
去傷害對方.
我覺得大家都沒有惡意.
但怎樣仍然honor對方.
是一個很重要的學習.
很快我們說完下面那兩個.
所以這個我都說.
tolerance似乎是一個很重要的觀念.
我們比較少提的觀念.

$^{921}$tolerance不是解造容忍.
不是去忍受一些錯的事情.
而是一種寬容.
一種心胸裡面能夠容納更多的東西.
相信對方的美善.
你的心胸能夠擴大.
能夠嘗試去接納.
嘗試去包容對方.
嘗試釋出善意.
起碼在第一,二步時.
是一個很重要的東西.
第二就是復和.
我們怎樣能夠主動去復和.
這都是我們基督徒很重要的學習.
最後我想說的是.
潘復華在Nag Foger.
《聖經基督徒》這本書裡.
提出一個很重要的觀念.
他整本書裡有一部分是說登山補訓的詮釋.
他提到當耶穌教我們.
愛仇敵.
被人刮完一巴之後再刮多一巴.
這一下是怎樣解釋的.
他說是一個超乎自然的.
他說愛其實是超乎自然的.
愛本身或者耶穌所叫我們群體裡的愛.
是超乎自然的.
什麼意思呢.
你自然而然是不會這樣做的.
正正常常你是不會愛仇敵的.
你都不會去復和的.
你都不會去不計較的.
你都不會去不寬容的.
你都不會寬容那個人的.
因為是非對錯是人之常情.
但古佛說這種的愛是超乎自然.
是擺明不正常的.
是你正常的行為裡面你不會做的事.
所以這些行為是唯有靠著耶穌基督去幫助你.
去永遠的多你幾步.

$^{961}$的多你幾格.
逼你多走一兩步去嘗試.
是奇蹟地去做出來的行為.
所以頂尖末之間都是這樣.
我們自然而然有人得罪我.
我是不會喜歡他的.
但這種的愛正正在基督徒裡面.
是有可能去做到.
這個不是要大家說你不做就不是基督徒.
或者不做就是壞人.
而是永遠都是基督的愛激勵我們.
去逼我們用這份的愛去回應.
所以這份的愛是超乎我們能夠做到的事.
自然的愛我們就會喜歡一些我們喜歡的人.
群一些我們喜歡的人.
得罪我我就自然不會去任何的善待他.
但是愛受的這個命令告訴他什麼叫真正的愛.
一個真正耶穌基督所說的那種愛.
是超乎我們自然.
我們不可能做到的.
最後再說兩句.
有人打你右邊臉.
左邊臉也被人打了.
這句話怎麼解釋.
這句話我傾向這樣解釋.
這句話說出來不是叫你做不到的.
是嘗試叫你做的.
是可以讓你做的.
我會說為什麼.
其實這部分這句話很重要.
就是七章十二節.
所謂叫做Golden Rule.
就是你願意人怎樣對你.
你就怎樣去對人.
這句話跟孔子所說的完全不同.
跟說己所不欲勿施於人是完全不同道理.
你想人怎樣對你.
你就先去怎樣去對人.
我想潘Sir請我吃飯.
我會怎樣.

$^{1001}$我就嘻嘻.
不如我請你吃飯吧.
不要這樣.
怎樣能夠在別人的Facebook上加你.
就加他吧.
怎樣可以.
你太太當你好像皇帝一樣服侍.
你先當她是皇后.
所以這樣對你.
先用別人想對你的方法去對別人.
所以這是一個你主動的行為.
以眼還眼以牙還牙是被動的.
別人有仇敵對我.
別人以眼我就還眼.
以牙就還牙.
這是對的.
這是公義.
這是一個律法很重要的精神.
公平公正公義.
但耶穌跟他說.
當我們在群體相處的時間.
我們要超越自然的事情.
當別人對你有傷害的時間.
我們可以嘗試扭轉局面.
我嘗試給一個例子.
很多年前我去德國讀書的時候.
我就要買飛機票去德國.
當時我由香港飛去德國.
買了一張單程機票.
因為我不知道什麼時候回來.
我去德國這麼多年.
就買了一張單程機票.
後來我發現原來我回來香港都會回來.
我會結婚或探親.
放假我都會回來.
原來這些人是這樣的買法.
他們會在香港讀書的時候.
先買一張來回機票.
走過程就買一張沒有寫日期的來回機票.
拿著去程去德國.

$^{1041}$拿著回程機票就不用.
保留在這裡.
回來香港探親才用那張機票回來.
再買一張新的來回機票.
原來這樣買法便宜一點.
我怎樣可以買回這些便宜的機票.
這其實是一條智商題.
我怎樣可以買回這些便宜的機票.
當時我已經在德國.
我怎樣可以買回一張從香港出發的來回機票.
沒理由不認識這些.
Jet Soul的東西沒理由不認識.
我怎樣做.
當時我已經在德國.
我怎樣做.
怎樣才可以變回我在香港買一張去來回的機票.
德國香港的來回機票.
YouTube的朋友知道嗎.
怎樣.
如果你聰明的話.
我明知道是貴的.
明知道是虧本給國泰的.
我都買了一張貴的單程機票回來香港.
我就可以從此買回香港的來回機票.
我先買了一張貴的單程機票.
明知道是不划算的.
明知道是貴的.
不過我可以從此享受一張永遠是便宜的關係.
所以耶穌教的一樣.
被人打完右邊碟再打左邊碟不是白癡.
不是被虐狂.
而是我嘗試用愛來回應別人的恨.
明知道是傻子.
明知道是不公平的.
但我這樣可以去終結一段不理想的關係.
當你願意這樣去付出的時候.
你能夠終結一段不理想的來回關係.
重新開始一段以愛還愛的關係.
別人以眼你還眼.
這是一個被動的.

$^{1081}$永遠都是被動的.
你被人拉著鼻子走.
但你嘗試用愛去還人家的眼.
這就可以終止一段關係.
所以耶穌的意思不是叫你做傻子.
而是你有可能去這樣選擇.
去做主動地.
因為你想別人怎樣對你.
你先去怎樣去對人.
下星期一你嘗試上班.
你對著你很恨的老闆.
你嘗試去愛他.
試試吧.
試試對著那個客戶或弟兄姊妹.
你嘗試的去這樣做.
總括來說.
別人以眼你還眼.
別人以牙你還牙.
下次可以試試.
當別人再以眼的時候.
你想想.
我是上過神學百科的.
耶穌也說過.
我嘗試突然用愛去對待他.
他會突然說.
這個人是基督徒.
突然嘗試這樣做.
然後他就會這樣去愛.
大家就可以成為朋友.
你很快會問.
是不是可以的.
是不是這麼天真.
是不是安利.
可能是的.
可能這些想法是天真的.
不過我想最少在教會裡面.
是值得去試的.
外面我不知道.
外面很多人會利用我們的愛心.
但在教會頂姐妹裡面.

$^{1121}$我覺得值得這樣去試.
當別人對方踩你一腳的時候.
你嘗試用愛來回應.
嘗試扭轉這種恨的時間裡面.
就可能有上帝的聖靈幫助.
大家可以復和.
所以我想這是一個很具體的.
頂姐妹之間.
我們復和的寬容.
能夠在破碎的關系裡面.
勇敢地走出第一步.
這個的主動.
我希望Full Church能夠有這樣的DNA.
或者希望有這樣的理解.
嘗試能夠建立我們這個群體.
將這次成為我們成長裡面.
或者我們門訓裡面的一個很重要的課題.
用上帝的愛來衝破一些.
不可能我們做到的事.
這是一個冒險.
這是一個未必有回報的事.
但值得我們去嘗試.
用這份愛重新建立這個群體.
作為一個大家流散過.
回到Full Church裡面的頂姐妹.
我覺得我們都要學這個功課.
如何能夠用上帝賜給我們的愛.
來對待我們的頂姐妹.
這次好像有湯喝.
請問.
有,不過湯涼了.
我怕你喝完之後會肚痛.
我原本是打算給你喝湯.
但因為我最後聽到你說.
要用愛去對待別人的時候.
我其實還有些菜.
一會兒拿回來給你吃.
我都會愛你的,沒問題.
不過其實我剛才有一個很入心的內容.
就是主動去做一步.

$^{1161}$我們搞食肆.
其實食物經常給一些特價.
真的吸引多些人.
對我們的感官和關係都好一點.
因為我們基督徒應該主動去做一些事.
不是回應,回應是被人拉住鼻子走.
所以這是我們很重要的東西.
但我有一個問題.
我自己一個做,是很難的.
有時覺得自己經常吃虧.
有沒有什麼方法.
或者是如何令到多些人都覺得這件事是正的.
我覺得是教會裡面是可以的.
特別是Full Church裡面可以試一下.
公司真的難說.
甚至乎在網絡世界裡面.
真的有.
謝謝阿妹.
都說了,沒騙你.
我做大了.
我覺得教會是可以實踐這件事.
在外面職場就很難說.
或者在網絡世界裡面.
很多事情不是那麼簡單.
都有很多ignorance.
人家都還沒明白真相.
很多時候都化解不了.
但我們在教會裡面是值得這樣做.
大家都聽過這課.
大家都明白這件事的時候.
希望大家都可以嘗試被聖靈光照.
去嘗試做這件事.
我覺得Full Church是值得的.
大家有沒有什麼主動的經歷.
或者是你覺得在課堂裡面.
覺得不明白的地方.
因為我剛才聽到.
ignorance 無知都是很糟糕的東西.
大家有沒有什麼想問.
怕自己無知呢?.

$^{1201}$有,那裡.
後面.
其實我想問.
可不可以再說多一點點.
關於復和.
就是.
你知道.
其實可能傳統華人教會.
教人家的是假復和.
就是你要去包容.
去原諒.
去饒恕.
但可能你中間是沒有真相的.
是沒有彼此認罪的.
其實不是想像之中那麼容易的事.
如果你做主動嘗試去復和.
但是結果不是你想像之中那樣.
其實是更加受傷.
如果一個群體.
暫時都沒有復和得到.
那你怎樣去.
繼續去面對大家.
跟著去彼此坦誠地相處呢?.
第二個問題就是.
我想問怎樣去面對.
一些.
無知的人.
即是就算你.
可能你想跟他說一些東西.
或者說一些事實的某一部份.
他未知的畫面.
但其實你發現你告訴他.
其實他都好像.
聽完都沒有甚麼.
或者他甚至沒有興趣知道.
有時可能.
一些群體的無知.
其實都會繼續傷害到.
堂會裡面的其他弟兄姊妹.
其實都好像是一個困局.

$^{1241}$怎樣去.
去坦誠.
或者很真誠地去相處.
我覺得是很struggle.
還有很多傷害在裡面.
是的.
我們是真的.
不可以忽視那種傷害.
剛才說的絕對不是想.
純粹擺一個理論出來.
就叫大家做.
我都知道.
我自己都經歷了很多.
所以都知道.
這些事不是說完之後就行了.
剛才你說復和那裡.
我覺得.
復和是mutual.
同時間.
但我覺得復和是一個過程.
那個過程是一個.
一個.
很多很多.
想起很多東西.
可能是爸爸媽媽和自己之間.
或者是在社會裡面.
或者是在一個關係裡面.
很多不同種的復和.
那個復和從開始到完滿.
是一個很長的過程.
真的.
沒有了大家的認罪.
大家的真相.
其實是談不上復和.
談不上復和的意思是很vague的.
是因為談不上一個完滿的復和.
但我覺得我們剛才所說.
如果耶穌和我們說愛受敵.
或者是十字架裡面.
耶穌和旁邊那兩個被釘十字架的.

$^{1281}$赦免你們.
這種的.
我都稱之為復和.
是復和的開始.
我們在整個過程裡面.
很多時候.
譬如其中那本書.
Michael Wolfe.
寫了一本書叫做復和的書.
我稍後再說.
很精彩的.
就是這個問題.
怎樣能夠說復和的神學.
在南非裡面.
在一個政治裡面的復和.
真相 彼此認罪 道歉.
這些是一個很重要的復和.
我們都很明白.
沒有這些是談不上復和.
但我這個說話是值得再edit的.
是談不上完滿的復和.
過程是未完結的.
但我覺得我們一開.
即是耶穌教鼓勵我們.
先拋一個石頭出來.
這一步其實都是一個復和的開始.
我覺得會是.
當中是充滿著冒險的.
因為我們拋完出來.
可能是沒有回應.
可能更加多的傷害.
可能是別人覺得你很傻.
但我覺得.
我經常說.
全個地球裡面.
最有本錢去先拋一個石頭的人.
就是我們基督徒.
因為我們是先被上帝復和了的一群人.
我們有本錢去嘗試冒險去拋這個.
可能沒有回報的石頭出來.

$^{1321}$所以耶穌鼓勵我們做這一步.
這個鼓勵其實不是一種命令.
或者純粹是一種硬梆梆的條件.
做完就復和了.
不是這個意思.
但總是有第一步.
例如很多複雜的情況.
例如父子關係.
我們跟爸爸媽媽的關係.
可能耶穌給我們勇氣力量.
嘗試再拋多一步出來.
可能已經很多年了.
但我們又嘗試去說一聲早晨.
或者WhatsApp問他喝茶.
可能都是傷害.
可能都是沒有回應.
但聖靈給我們又一次勇氣去先做這個力量.
所以這個談不上復和的圓滿.
但這個都是復和的開始.
沒有這個就沒有了.
大家是停在這裡.
我們似乎是基督徒作為復和的使者.
能夠嘗試實踐.
我自己很喜歡和平這首歌.
和平是指這件事.
無論是獅子和羊.
或者人和人之間.
終極地就是耶穌的和平.
祂復和了這個世界.
祂才叫我們成就一切的復和.
所以我們可以嘗試做這件事.
所以我覺得是語言上的澄清.
這個復和不是圓滿復和.
但都是復和的一步.
亦都是重要的一步.
那Ignorance那裡.
反而認識了Ignorance就懂得放下.
因為我們知道這個世界裡.
唯有在天國裡才有無限長的時間能夠對話.
如果你看一些哈巴馬斯克的說話.

$^{1361}$叫做Discourse Ethic.
即是對話的倫理.
如果一個社會是理性的話.
有足夠的空間去對話的話.
是很重要的.
但我們沒有足夠的空間.
不夠時間去看別人的文章.
不夠時間去面對面對對談.
所以現在這個世代我們沒有足夠的空間.
真相去對話.
所以我們只能夠得到某些程度的真正滿足的對話.
但發覺社會裡面越忙.
我們越來越少這樣的空間對話.
可能夫妻都已經不同意.
大家真誠的對話.
所以我們確實對Ignorance是很嚴重的傷害.
所以我覺得世界裡面仍然充滿著這些問題.
因為Ignorance是解決不了的.
所以比較實際的.
就是你明白對方是Ignorance.
你就知道不需要那麼上心.
這就是我這幾年在Facebook的經驗.
所以也不需要太介懷別人的無知.
反而我覺得無知是對自己說.
自己可能是無知.
也不明白對方.
可能也會有錯的機會.
這樣去做人.
不知道潘Sir有沒有其他補充.
我覺得如果看復活.
剛才John說的那是一部分.
我覺得說復活之先應該是說饒恕.
就是饒恕是第一步.
我認識的饒恕或者我自己個人做的饒恕.
不需要期望對方有什麼回應.
是我自己主動去做的那部分.
意思就是我怎樣看待那件事.
他再做什麼對我來說.
已經不會再觸動我或者觸怒我.
舉一個例子.

$^{1401}$我那天請John去我家吃飯.
他去廁所.
走過我的走廊的時候.
這是例子不是真的.
不是真的.
走過我的走廊的時候.
心急.
手袖碰到花瓶.
掉在地上爛了.
他很急去完廁所出來.
跟我說不好意思.
打爛了你的花瓶.
那一刻他就跟我說.
不好意思打爛了你的花瓶.
但他不知道花瓶對我的意義是多麼重要.
事實上花瓶是爛了.
我不能再有那個花瓶.
他真的跟我說對不起.
我沒辦法了都爛了.
我就原諒了他.
但是每一次我回家的時候.
經過我的走廊.
原本我很喜歡的花瓶在那裡.
但已經沒有了.
但我每一次看到沒有花瓶的時候.
我就會想.
總之不要請他來吃飯.
不請他吃飯就不會爛了那個花瓶.
他一句對不起就弄了我的花瓶.
我沒有了.
很不開心.
但我可以做些什麼呢.
我可以做些什麼.
我唯有想想.
其實他也是不經意.
我最主要的就是.
主動去饒恕他那個無知.
或者不在意的情況下.
做了這件事.
這就是我做些什麼.

$^{1441}$而不是他做些什麼.
可能對你來說.
這也不是解脫.
但是我跟自己去處理這個情緒.
或者這個過程的時候.
每一次看到花瓶沒有了.
那個空的桌子的時候.
我都告訴自己.
那件事已經過去了.
不要再讓那件事再核製了.
是不開心的.
如果那個花瓶是很有意義的話.
你更加會對你的傷痛.
或者那個執著更加大.
但是那件事已經過去了.
你願不願意讓自己放下呢.
我自己在成長過程當中.
有幾次都要學習這個大課題.
讓自己明白.
有些事情你再堅持下去.
一來不會回頭.
二來又會令到那件事.
不斷地在自己的情緒勒索自己.
而對我再看回這個人的時候.
我就會更加對他有一種隔膜.
每次見到他的時候.
就想起沒有了那花瓶.
那我跟他的關係又怎樣可以繼續呢.
這個都是我們要正視的.
所以這個就會影響到那種復和.
或者那種關係可以繼續下去.
這個可以再有討論空間可以對話.
第二個就是在說無知那件事.
我想2019年開始之後.
我們周遭的環境.
很多人看的視覺都不同.
收的資訊都不同.
或者表達的立場.
或者是描述事件發生的經過都很不同.
我們都覺得為甚麼人們.

$^{1481}$不多點角度去看.
我自己從石劍未講到的時候.
都講過我接觸過的家庭.
在那半年.
很多都是兩代人之間的溝通關係.
或者視覺上的問題.
但我都跟那些年輕人說.
如果你信得過.
或者我的能力.
我都會跟你爸爸媽媽說.
我都讓他們.
我那時候講到都是.
我實際上都帶父母去現場.
看一下他們的視覺.
也是讓他們明白到.
有些事情他們在電視機看不到.
真實不是在電視機看到那個.
所以至於你說.
我可以回應.
無知其實是一個時間性的過程.
當我們在經驗學習過.
經驗經歷過.
或者在嘗試讓自己開放.
無知就是要開放.
開放自己.
給自己多點角度去看.
或者多點參與的時候.
就縮短那個無知的時間.
我用聖經的說話.
就是耶穌說.
「父啊赦免他們.
因為他們所做的他們不曉得」.
人是看不明白耶穌那時候.
為甚麼要上十字架.
但當過後的時候.
就明白.
其實無知是很多仕途.
當時或者門徒當時來說.
是不明白的.
如果在科幻書裡面.

$^{1521}$很多時候有兩個天.
科幻書裡面說.
有兩位身穿白衣的人.
跟他們說.
第一次出現.
就是在耶穌復活的第一天.
就是「加利利人啊.
你們為甚麼在死人中找活人呢.
他不在這裡.
當紀念他在加利利所說的話」.
就是那時候他們不明白.
耶穌必須受死埋葬.
第三天復活.
但現在你們明白了.
你們要看了.
第二次.
還有兩個身穿白衣的人.
跟那班門徒說.
就是《司徒行傳》第一章第八節.
對不起 第十二節說的訊息就是.
「加利利人啊.
你們為甚麼站著望天呢.
他怎樣上去.
亦會怎樣下來.
當紀念他在加利利對你們所說的話」.
其實很多時候.
人在無知的情況下.
看不到上帝的工作.
或看不到周邊的工作.
但等著時間過去.
或等著時間中慢慢去揭示.
看到整個群體的參與.
我們會慢慢.
我們求上帝讓我們看到.
一些我們要學的東西.
所以無知就是要經歷那個時間.
我們願不願意.
有沒有一個主動性.
讓自己不要單一視角.
或是要學習多點經歷.

$^{1561}$其他人嗎?.
網上的問題.
網上的Cliff Thomas問.
我想有很多離開了園堂會的朋友.
原本都是懷著這種寬容.
但這種寬容只能單方面終止一次性.
一次的聚群環.
卻無法終止.
其他人去一次又一次.
去開展一個新的聚群環.
聽起來都有點蒼白無力.
第一面對這種情況.
對人的信任.
或許比不信任的人更低.
信任.
對人的信任.
或許比不信主的人更低.
在這種單方面去走一步的狀況.
我們有沒有什麼出路.
還有第二.
有沒有一些可以增加大家的承諾的方法.
我覺得承諾是對的.
我覺得要補充.
剛才說承諾,復和這些字眼.
我正在想的反而是一個具體群體.
剛才的情況.
我們離開了群體或離開了那群人.
有時切割是重要的.
你活在那個群體或社會裡.
本身就是一個社會學的字眼.
一個現代的社會裡.
一個多文化的社會裡.
就需要有承諾.
所以這個很重要.
如何去說一句「合力」.
人有息相關.
大家如何學習.
可以有承諾.
這是一個很重要的精神.
在美國多文化裡就需要有承諾.

$^{1601}$很小的時候就學習.
大家有不同的文化.
因為他們在一個社會裡.
不會離得開.
承諾是說.
你在一個身處不可逃的環境裡.
你需要有承諾.
但有時我們也會學會.
我要離開.
我要割捨.
不要讓那個東西繼續在我心裡.
所以是兩個不同的東西.
你很難去承諾一些敢對你的人.
不是說你不應該赦免他.
而是具體的傷害.
所以我覺得是兩種不同的情況.
我們還未去到那個階段.
例如Future的弟姐妹.
在一個小組裡.
大家有不同的想法.
或者不同的取向.
這是要承諾.
所以大家要學習.
一起去相處.
但在另一方面.
我都覺得.
剛才你說一次又一次.
你說一次做到的.
但在一個序的循環裡.
我都覺得不是適應的問題.
而是另一個問題.
所以我都說.
今天說的話.
在一個全面的教會裡.
大家都學習去做.
是會容易些去做.
而且是可達的.
而在外面其實是很複雜的.
其實我不是第一次說這個話題.
但很多時候.

$^{1641}$有人問我在職場裡怎麼辦.
我都不斷利用我的愛心.
這是另一個課題.
但我覺得在教會裡.
大家都這樣去學習.
是容易做.
而且是值得去做.
所以在某些位置.
不要讓別人繼續傷害.
不要讓別人繼續在那些位置裡.
最少你要離開.
不要讓別人在那個循環裡這樣做.
我認同.
意思是.
有些位置.
我補充一句.
不是未試過的.
如果它是日復日.
或者是不斷地.
最循環地這樣做的時候.
你中間要說的話都說過.
你做的事都做過.
它都是用原原這個方法去對付你.
或者是對待你.
我覺得就要抽離.
因為不可以不斷地.
給這些情況去傷害自己.
但我認同.
既然你遇到這樣的經歷.
而你又找到一班相關的弟兄姊妹.
其實剛才說到Flow Church.
Flow Church有很多弟兄姊妹.
有相近的經歷.
我自己教Info Group的時候.
我在第三第四課都說過.
Flow Church的小組.
有些最大公因數.
雖然大家來自不同的堂會.
但是侍奉.
或者小組經歷.

$^{1681}$或者對人相處.
或者在教會上有些難處的時候.
都有相近的地方.
你一分享的時候.
就知道原來只有我這麼慘.
原來有些人都跟我差不多慘.
我的答案就是說.
我們盡可能不要將這些慘.
再在Flow Church的小組出現.
因為這個盡可能抽離.
不要再讓這件事再核製我們.
或者再重演出來.
我不是戴頭盔的.
我那時候也說.
但是我們要學習.
因為很多時候我們離開了群體的時候.
因為我們在那個環境太久了.
我們需要好不容易提醒自己.
剛才那件事.
你曾經被人這樣對待過的時候.
你就不要用那個方式再這樣對待人.
所以在小組當中.
應該用一個新的形態.
新的方式去建立一個.
比較好的小組的溝通方法.
小組的動態.
和表達方式的時候.
怎樣可以再貼近基督徒的本份.
所以我認同.
我們應該要產生一個新的群體的表達方式.
不要再重蹈舊的.
你曾經不開心的經歷.
我想說一些簡單的問題.
這個問題就是強化玻璃心.
想從個人層面出發.
譬如說.
很多人可能比較用心去做一件事.
很努力地在教會進行一些的侍奉.
或者他不能夠.
不是和大家想法比較相同.

$^{1721}$他很全心全意去維繫一個關係.
但是徒勞無功.
然後他就很容易傷害到自己的心.
可能其他人是不察覺的.
有些人可以說流童是一個.
很多受傷害的心靈.
在其他教會聚集在一起的人.
其實很多被受傷的心靈.
去到其他教會.
可能他已經有一個不好的經歷.
於是他已經不能夠像以前.
一開始很火熱地.
去拿出他的所有的熱誠去服侍主.
就好像去到一個點.
知道了和大家不太相同.
應該是去禱告主.
我要向主認罪.
我要去認同這裡.
否則我又要再離開.
莫非我也去到一個點.
我不能夠再受傷害.
我又再去另一個地方.
到最後我究竟是屬於哪個地方呢.
神啊 你要我去哪裡呢.
所以我很想從個人層面方面.
如何在哪些屬靈書籍.
或者哪一卷聖經.
或者哪一類型的參考.
可以強化自己的內在.
在我們可以重新出發.
重新在一個地方.
主說這個應許是我在這個教會.
我去全心全意地擺出我自己.
而我也可以真正地強化我的心.
即使我這次再受傷害.
我仍然能夠站立起來.
有沒有一些書籍可以給我們參考.
謝謝.
我覺得看書也沒有什麼用.
因為這個不是知識問題.

$^{1761}$當然看聖經.
但是反而我聽姐妹們的分享.
我想起剛才我說強化玻璃心.
意思是.
記住不是可以當那些東西沒事.
而是能夠快點沒事.
能夠不會在那裡.
那怎樣呢.
反而我覺得是需要.
曾經經歷那麼多傷害之後.
我覺得反而是何時可以強化.
就是找到一群很疼愛你的人.
那群人足夠在一段時間裡.
有群人來支持你.
還有一段時間後.
你才能夠重新侍奉.
或者做一些比較高危冒險的.
有機會破碎心靈的事.
我想起我女兒.
因為怎樣能夠叫到我女兒.
她的父母親育.
你去愛她愛得足夠.
她就會有自信.
她不會怕失敗.
反而你經常罵她.
她就會很怕失敗.
所以怎樣能夠強化一個人的心臟.
其實經歷很多的愛.
才能夠被強化.
所以我覺得不一定是看書.
而是真的可以在群體裡.
如果真的在群體裡.
我們經常說群體裡的弟妹.
不需要立即侍奉.
你在群體裡找到一群戰友的時候.
你才可以慢慢去想這些事.
有一定程度的同行的人.
才能夠幫你強化心.
因為其實你的心不會被強化.
你的心是受傷之後很快被復原.

$^{1801}$因為藉著很多弟妹的愛.
我去年年尾的經歷.
發現原來很多人是愛我的.
愛過的人比恨過的人多.
這件事是有用的.
是可以復原的.
所以我覺得強化身體的意志.
其實是一個愛的群體.
如果說書的話.
我突然想起一本書.
那本書是台灣校園出版社出版的.
叫做《玉子團契》.
那個玉子不是日本人吃的那種.
那是那裡的玉子酒吧的玉子.
那本書《玉子團契》.
是John Albert這位牧者寫的.
台灣翻譯成中文.
中文名叫做奧伯格.
如果作者.
《玉子團契》書裡面說什麼呢.
其實就是在說群體裡很多的不是.
就是不好的東西.
但是正正在很多不好的東西當中.
有不同的聖經教導我們.
要去接納和欣賞.
因為.
籠統一點說.
上帝接納我們.
這群不同類型的人.
走在一起的時候.
我們本身就是亂七八糟的.
我們亂七八糟的時候.
在一個群體當中.
總有很多需要協調.
需要很多調整.
需要很多磨合.
其中有一課裡面就是在說.
那些豪豬.
豪豬就是那些賤豬.
那些賤豬很想去接近人的時候.

$^{1841}$但是牠會刺到人.
人們經常跟牠保持距離的時候.
牠覺得很疏離.
牠怎麼可以在當中跟人相處呢.
又可以有那種關係呢.
就好像我們本身.
如果我們越是跟人接近的時候.
就是越是將自己不好的東西放大給人看.
那我們怎麼去跟人相處呢.
那本書就是在不同的題目.
就是在說.
在一個真誠的群體.
就是剛才那個課堂內容.
越是坦誠的時候.
越是要讓我們去懂得欣賞對方.
因為他願意將最真誠的一面去揭示給你.
就是我就是一個這樣的人.
我沒有裝假的.
而我過程當中.
有些東西是我不是你期望想的東西.
但是這個就是我.
我自己說到.
在Flow Church有一課是說十二體面.
就是說哥倫多前書十二章裡面.
有人體面的越化.
加急去體面的時候.
就是要懂得欣賞對方.
欣賞群體當中個別的特長.
或者是個別一些我沒有你有的東西.
很多時候我們人就看到.
一些你這麼厲害.
我不行.
但是很多時候就是.
看到別人很漂亮的東西.
就看不到一些.
可以欣賞一些其實你沒有.
但其實他也不是特別好.
這個過程當中其實.
是懂不懂得互為欣賞呢.
互為補足呢.

$^{1881}$保羅用很長的篇幅說.
不需要比較的.
但我們需要學習去欣賞對方.
我覺得.
我認同阿John說的.
心本身就是心.
肉心就是肉心.
就是你始終都會.
你會強化到他.
但復原那個resilient的power是重要的.
就是怎樣可以令到那件事.
回到本位呢.
就是一個群體.
就讓那件事可以快速回到本位.
就是可以再來過.
有一個網上問題.
來自風Sir的問題.
作為基督徒.
真的要為每一個傷害自己的人.
做寬恕.
例如曾經施虐自己的人.
問號.
我認同阿Poon Sir.
對於寬恕的定義.
我也是用這個看法.
有些東西不是說.
對方沒有道歉.
就不能夠寬恕.
我自己不是這樣理解寬恕的.
例如十字架裡面.
耶穌也不懂ignorance.
就是ignorance.
祂也不懂.
但我也故意寬恕他們.
我比較贊同耶穌.
當我們說寬恕.
其實寬恕不是一種權柄.
而是寬恕真正的受惠者.
不是寬恕那個人.
而是寬恕別人的人.

$^{1921}$所以寬恕就像寬恕說的.
是讓自己不會再受到傷害.
你已經被人傷害了.
而當中的.
耶穌寬恕別人.
是讓我們被這些問題釋放.
所以寬恕是能夠放下那件事.
所以世上只有一個人.
能夠有權柄去免去別人的罪.
這就是我們的上主.
我們寬恕別人不是免去別人的罪.
也沒有這樣的能力.
也不需要去免去別人的罪.
這與我們無關.
我們寬恕是讓我們.
不會再受到別人的加害.
很慘的.
你已經被別人傷害了.
還要在那個困境中.
走不出來.
所以寬恕是一種.
耶穌給我們一種恩典.
讓我們能夠離開這種.
雙重的影響.
所以寬恕最大的重要性是寬恕者.
他能夠不會被那個罪的問題.
來困鎖了他.
所以剛才的問題.
被別人侵犯寬恕.
所以寬恕永遠都不能叫別人寬恕別人.
因為寬恕是自己能夠從上恩典中.
可以離開這種罪債.
所以是一種恩典的出路.
讓我們不會被這個罪債所困擾.
因為基督耶穌已經能夠.
可以消除這個問題.
所以回到和平的那首歌.
十字架是一切復和的開始.
因為他滅掉一切的冤仇.
所以十字架讓我們能夠釋放.

$^{1961}$能夠被這個罪的問題.
仇恨所釋放.
不是你赦免他.
而是你真正被他的罪困鎖住.
所以寬恕讓我們能夠離開這個夢鴉.
我補充一點.
在談寬恕這個課題裡.
我剛才說過.
有些成長過程中都學習這個功課.
都看過很多不同的書.
其實真的有不同的.
你當作是作者或學派.
說有些不同的方式.
但我自己覺得.
真的你自己主動去做.
剛才我說的內容.
不要再讓事情核實自己.
所以剛才網上的朋友問到.
是否要寬恕所有的人.
我覺得具體一點.
你要知道那個步驟.
是不是一定John.
我主動去寬恕John.
是不是一定John有反應或回應.
或者是否去到John面前跟他說.
我多生你的氣.
是不是需要這樣.
不同人有不同的看法.
但我自己做木樣的過程中.
如果選擇一定要去當事人.
要對質 要他做.
你真的要為打爛我的花瓶道歉.
你知道花瓶對我的.
要他說到我很不開心.
要他明白 要他認同的話.
這個其實都不容易.
我為什麼這樣說呢.
你要自己跟自己去做一個評估.
當你講到很重要的東西.
如果對方的反應都不是你的期望.

$^{2001}$你會更傷心.
你明白我的意思嗎.
因為我曾經真的跟我的職稱說.
我真的要這樣跟他說.
我已經跟他預先說過.
如果他說原來我傷害了你嗎.
那我會說你更慘.
你會很生氣自己.
我生氣了這麼久你都不知道.
原來我自己生氣自己.
你更難過的日子.
你明白我的意思嗎.
有些不同的人用這些方法.
我就不是說對與錯.
我要期望一件事.
因為你要別人的回應.
如果別人的回應跟你的預期有落差.
那個都是第二個傷害.
我不是說一個方法.
但我在過程當中.
要說你自己能不能夠從你困所的環境釋放出來.
而你真的在當中讓自己明白到你曾經受傷害.
而那件事不會再觸碰你.
是需要時間的.
因為那件事越是複雜越是難的話.
你需要等待醫治的時間.
或者需要等待再看到不再令你情緒繃緊.
其實不是說我有什麼特異功能.
但真的要慢慢讓自己有空間.
有沒有其他問題?.
現場?.
沒有的話就來湯凍吧.
那就到下個月了.
下個月就可以令這碗湯熱起來.
下個月就說熱情.
那我長時間要拿個火鍋出來嗎?.
下個月差不多就可以弄個羊腩煲.
草樣?黑草樣?.
有機會的.
弟子妹希望大家都能夠經歷聖帝聖靈的愛.

$^{2041}$讓我們更加有愛去對待不同的人.
下個月見.
再見.
ball.
晚安.
早安.
\newpage



\section{}
\label{sec:K2_OK28IM68}
\textbf{【這是最好的時代:給香港基督徒的神學八課】第6課: Passion|20211018 [K2\_OK28IM68]}
\newline
\newline
連結: \href{https://youtube.com/watch?v=K2_OK28IM68}{\texttt{ https://youtube.com/watch?v=K2\_OK28IM68}} ~~~~ 語音日期: 2021-10-18 
\newline
\newline
\hyperref[sec:xbbmUvNItM8]{\small{< < < PREV SERMON < < <}}
~
\hyperref[sec:index_chronic]{\small{[返順時目]}}
~
\hyperref[sec:index_scriptual]{\small{[返順卷目]}}
~
\hyperref[sec:S_0UcgQqbRI]{\small{> > > NEXT SERMON > > >}}
\newline
\newline
$^{1}$(廣播中).
各位靚姐妹晚安.
我們來到神學八課的第六課.
很快我們經歷了六堂的時間.
接下來的七,八課就完結了整個系列.
香港神學,我們基督徒神學八課.
其實我們很想在這八課裡面去教神學.
讓我們能夠面對這個時代.
所以今天上課也是.
我們慢慢從一些宏觀的福音,基督徒,教會.
去到我們個人的裡面.
我們怎樣去面對自己的靈性.
我們怎樣去看我們的軟弱.
今堂是很特別的.
今堂其實不是平時你會聽過的題目.
叫做Passion.
為甚麼叫Passion呢.
我覺得這個字是很有意思的.
大家可以先問一下.
這個是甚麼呢.
我想問一下.
這個是甚麼呢.
我們香港人叫做熱情果.
原來熱情果Passion fruit.
這個字其實不是這樣的.
如果你看一些典故.
你會發現Passion fruit本身的熱情跟熱情沒有甚麼關係.
Passion fruit的原因.
簡單介紹一下.
它是來自於南美洲的巴西.
在17世紀傳到歐洲.
在新南海時代的時間.
有人拿著Passion fruit的花.
在西班牙的船駕時.
看到這個花就覺得很特別.
這個花其實有一個十字架的形具.
如果你詳細看的話.
它裡面有三個分裂出來的.
好像是三根釘.
整個花就像耶穌的荊棘觀念.

$^{41}$它加上五個黃色的花瓣.
就像耶穌的五個傷痕.
所以整件事的起點.
它本身先取了這個花.
就叫做Passion flower.
Passion flower本身應該解作甚麼呢.
就叫做壽難花.
原來整件事的Passion.
不是解作熱情.
而是壽難.
其實Passion有很多意思.
所以從Passion flower.
壽難花到壽難果.
來到香港就變成了熱情果.
就像熱情一樣.
其實完全不相關.
跟熱情沒有關係.
反而是壽難.
其實你也知道.
你去看很多不同的字的時候.
你會發現.
我們叫甚麼呢.
壽難曲裡面的英文叫.
Passion of the Christ.
原來英文裡面的Passion.
除了解作你的熱情之外.
它更加是解作壽難.
所以Mill Gibson拍了一套電影.
叫做壽難曲.
其實這個節奏很久了.
如果你有聽Classic Music的話.
J.S. Bach.
寫了兩個非常有名的.
就叫做St. Matthew Passion.
和St. John Passion.
兩個都是從.
這個黑西瑪利園開始講述.
整個耶穌受難的故事.
所以當我們.
特別是信仰來說.

$^{81}$那個Passion.
其實我們是.
首先去解作壽難這個字.
所以你會發現.
很特別.
當你去提到Passion的時候.
我們看回原本的字義.
你會發現其實.
Passion裡面.
有兩個很不同的希臘文.
一個是名詞.
就叫做Pathema這個字.
動詞就叫做Puzzle這個字.
兩個字其實是解作什麼意思呢.
看回字典裡就說.
第一個就是.
就是解作一個受苦.
就是一個受苦的意思.
然後它才解作.
所謂的Emotion.
一個很強烈的Inward Emotion.
一個內在的情感.
所以從來都是.
從聖經的字義.
到我們整個2000年裡面.
有關Passion的字.
其實它都是.
有兩個很不同的意思.
一個就是解作我們的熱忱.
我們的那份情.
那份熱忱.
另外它是解作受難.
所以今天我們就講.
當我們去想.
我們怎樣去形成.
我們一群Folk Church弟兄姊妹.
如果我們這八課裡面.
如果有一課是講我們.
我們Folk Church弟兄姊妹.
一個很獨有的DNA的話.

$^{121}$Passion成為我們一個.
很重要大家去思考和學習的課題.
都是的.
這十年裡面.
我們經常都開始慢慢聽人說.
做Growth.
首先做一個人.
我們開始就混淆.
究竟我們做基督徒和做人.
有什麼分別.
做一個有血有肉的人.
和做基督徒.
兩者之間究竟是一回什麼事.
我們經常說我們做一個人.
就是怕就怕.
我們面對著恐懼就恐懼.
當我們面對著悲痛就悲痛.
做一個有血有女人.
這是一個很真實的事情.
所以我們今天就探討問題.
究竟Passion這件事情.
對我們做基督徒有什麼意義.
所以今天我就會特別看一句字.
叫做Lydon Sharpe.
很有趣的.
德文裡面的Lydon Sharpe.
都是一個很有趣的字.
和Passion一樣.
如果你是一個很有熱忱.
去做一件事.
就叫做Lydon Sharple.
但Lydon本身是一個苦難解.
當你很喜歡砌模型.
就會叫做Lydon Sharple.
這樣做一些事.
所以當我很有熱忱去做一件事.
甚至和受苦有關係.
很奇怪.
這麼多年來.
受苦和熱忱都是同樣的字根出現的字眼.

$^{161}$今天我們就嘗試去認識一下.
究竟我們整個基督教裡面.
我們怎樣看Passion這件事.
回想到最早期.
大概在2000年前的希臘哲學的時候.
斯多瓦派很崇尚一種叫做.
Empathy.
即是無情.
你見到現在的A在前面.
即是一個沒有情感的意思.
古希臘哲學裡面的上帝.
更加超越更加高尚的上帝.
他們認為是一個沒有情感的上帝.
換言之.
情感Passion.
本身是一個負面的字.
即是我們稱之為凡夫俗子.
人世間的人才是這麼低等的.
才會有這種情感.
會受到一些Emotion影響.
所以2000年前的希臘哲學.
他們認為神明或上帝應該是沒有情感.
最高尚的最形而上的神.
是沒有情感.
所以我們基督教也是.
在初期的時候.
也很受到希臘哲學所影響.
我們所想的上帝.
都偏向一個沒有情感的上帝.
他是不被動搖.
不會被人動搖.
我們覺得很Moving.
很Touching.
無論是Touch或Move.
都好像被人搞到一樣.
所以上帝他們覺得.
不應該被人Move或Touch到.
上帝是高高在上.
不會被任何事影響到自己.
所以他們認為一個更加真正的上帝.

$^{201}$一個高階基督徒.
是一個沒有情感.
沒有任何情緒的人.
所以我們看回一些沙漠教父.
當時有一個John Cassian.
一個很出名的撫修主義者.
當時一些最早期的靈修傳統裡.
他們的撫修.
正正是嘗試透過這些撫修.
去磨練自己的靈魂.
不會被任何事影響.
所以這些人就是這樣.
在沙漠裡不斷地去撫修和操練.
他們認為要和靈魂裡的情感去搏鬥.
靈是最高上的.
魂.
即是我們的情緒.
或者是我們的魂魄.
是比較次等的.
所以他們認為.
大家可能聽過七宗罪.
最早期有八宗罪.
有八個非常不好的罪.
都是被一個人裡面的靈魂的情感所影響.
其中有貪吃,淫亂,虛榮,驕傲,憤怒,貪婪,懶惰.
其中一個是憂傷.
原來以前覺得憂傷是一個不好的事情.
我一個熟靈人的時候.
我是不會感到任何憂傷.
不會被任何事情影響我的情緒.
所以反過來.
快樂也不是最厲害的情感.
快樂這麼容易被人家迫起.
這麼開開心心也不是好事.
真正真正最高層次是怎樣的.
是無情.
是一個完整的情況.
所以我們摸一摸.
所以他們認為.
上帝也是一樣.

$^{241}$上帝是不能受苦.
也是沒有感情.
一個無法受苦和無情.
上帝不單止不會讓苦難去教導他.
更加不會被任何情緒去動搖.
他們初時後都認為.
究竟上帝是不是真正的十字架.
不過釘的是一個人子的耶穌.
還是神子的耶穌.
他們認為上帝是不會被十字架所影響.
釘的只不過是一個人的耶穌.
這是一個很早期的神學.
他們認為上帝是不會受苦.
言下之意也不會被受苦當中的情感影響.
所以這就是2000年前的傳統上帝.
不過我們所相信的耶穌並不是這樣.
耶穌是真的經歷十字架.
並且他是被情感所影響.
他會流淚.
他會被其他人去動搖.
這份我們遲些會講得很詳細.
愛的緣故是他甘願開放自己去動搖.
從而被情緒所影響.
所以我們發現.
通常最神聖的樣子就是這個.
不知道大家知不知道.
四月清史中最神聖的就是這個沙加.
因為他永遠都閉上眼.
閉上眼是一個完全不會被任何的外來事物.
看到最神聖的樣子就是這個.
閉上眼.
所以這正正就是我們一直以來所想像出來的.
上帝的模樣.
不被任何的情緒和事情影響.
所以你會發現這是我們初期教會的看法.
不過到了中世紀的時間.
多瑪斯·奎納就嘗試去想多一點.
所謂的passion是一回事.
阿夫娜給了一個非常全面和中性的理解.
我們就開始開展出來.

$^{281}$我們講一些神學.
有關passion的神學.
passion對於多瑪斯·奎納來說.
他寫得很詳細.
在《神大傳》1.2中的第22條到48條.
全部都講述有關passion這個課題.
對我們來說passion是什麼呢.
passion是我們靈魂裡必然會有的情感.
沒錯 靈是很厲害的.
我們有靈性.
不過我們作為一個人.
我們是不會沒有情感.
所以passion就像今天所說的emotion.
或者是我們的情緒.
我們的情緒 我們的情懷.
我們的感動 流落的情感.
正正就是passion這個字.
阿夫娜就問 究竟靈魂有沒有情呢.
或者作為一個人.
我們有沒有一個情感.
我們有沒有一個passion在當中呢.
究竟passion是好還是不好呢.
多瑪斯·奎納給了一個非常好的理解.
我覺得passion我們作為一個有靈魂的人.
流落的passion就像一條海一樣.
我們的河流是可以繁繁盛盛.
水能載舟 羊腹奏.
它可以成為一個非常可怕的江河.
一個氾濫的河流.
當我們爆發的時候.
當我們受到情感所影響的時候.
如果你記得上一課我講過.
上一課我講過犯罪.
犯罪其中一個外來因素是什麼呢.
就是passion.
所以我們作為一個人.
我們確實有時候會被我們的passion所影響.
甚至會令我們犯罪.
我們的情緒 我們的動怒.
我們的情慾.

$^{321}$所以passion就像一條河一樣.
它可以是很恐怖的.
它可以非常影響到我們整個人的生命.
不過作為一條河流.
它也可以有非常重要性.
它可以成為一個非常重要的.
譬如說水力發電.
它可以成為一個人的水塘.
也可以成為樹木的休養地方.
凡是河流流過的地方.
它都成為一個非常大的動力.
所以多半受訪的人認為.
passion其實它可以是好.
可以是不好的.
它基本上是忠誠的.
只要它能夠讓我們能夠藉著這些一道道的閘去閘住它.
好能夠適當地來控制我們的passion的時候.
那我就能夠去善用我們的passion.
所以我們作為人是不可以沒有passion的.
這是詩篇所說的.
詩篇第84篇第二次.
心腸的肉體向永生上帝歡呼.
My heart and my flesh sing for joy to the living God.
這裡說原來我們對上帝的歌頌.
或者我們的頌讚.
不是純粹我們的心靈.
而是我們的flesh.
所以我們作為一個人.
一個有靈魂有心的人.
我們對於上帝的那種流露.
我們不單單是屬靈的.
所以我們是屬靈之餘.
我們會牽涉到我們整個的身體和我們的所謂的情感.
所以我們是應該要去運用我們全人.
從我們的靈到我們的情感或者是魂.
去到我們的身體.
來做任何事情.
所以想想一個基督徒如果是沒有passion.
沒有情感的人.
超級屬靈.

$^{361}$是一個很奇怪的人.
說話沒有表情.
不被任何世事所影響到他.
這個不是一個人.
所以你會發現一個真正的基督徒.
應該是一個帶動從心靈到整個人到靈魂.
是整個人的全人的向度.
才是一個真正的人.
所以通常來說我們的passion是重要的.
問題是如何能夠好好擺放這個passion.
接著說一些比較深的東西.
Fernand說其實有兩種不同的passion.
一個叫做欲情.
一個叫做憤情.
一個叫做irrational.
什麼來的呢?有點複雜.
中文和英文都很難明白.
第一個我們先來解釋什麼是欲情.
欲情是什麼呢?.
就是我們一個sensible的那種情感.
那種passion.
是關乎於我們對於任何事情的那種感受.
分好和不好.
good and evil.
當我們去面對一些美善的事.
我們會怎樣呢?.
我們就會愛.
這個很特別.
Fernand給了一個非常複雜和詳細的哲學給我們聽.
當我們.
普遍來說我們叫愛.
愛一個手機.
愛一個人.
愛一個事情.
都是愛.
愛當我們還沒有得到的時候.
叫做什麼呢?.
就叫做desire.
當我們很想要的時候.
就叫做desire.

$^{401}$當我們得到的時候.
就叫做pleasure.
當我們已經得到的時候.
我們就能夠享受它.
並且得到快樂.
所以當我們愛.
我們愛任何的東西.
愛上帝.
愛任何的物件都是一樣的.
所以愛就分成了三種不同的passion.
愛和渴望和享受那種快樂.
這是我們人對於美善的事情.
三種不同的情感.
這是第一個.
分支.
第二個.
就是對一些不好的事情.
對一些evil.
對一些惡事.
那些事情是不能夠成全你的.
那些不是在你的perfection裡面有關係的.
所以你會怎樣呢?.
你會去珍惠它.
Generally你會珍惠它.
不過珍惠不是首先的.
我們首先有愛.
愛這件事.
我們反過來講就會珍惜某件事.
我們愛護真理.
就自然會珍惜那些虛假.
所以另一批的情感.
情緒.
Passion.
就是這三種.
一個是hatred.
一個是珍惠.
珍惠.
當我們看到一些.
當我們還沒有的時候.
就叫做Aversion.

$^{441}$就是一個名字.
就是討厭.
我想遠離它.
我想離開它.
任何的惡事.
譬如你嫁給一個很臭的.
或者是那些.
那個人你不喜歡他.
或者一些很討厭的國士.
都想遠離他.
例如香港那些人覺得是.
所以當我們發現一些不好的事情的時候.
我們會遠離它.
這是跟design剛好相反.
Design是當我們還沒有得到的時候.
就會想渴望它.
這個Aversion就是正確的.
我們想遠離它.
當我們去.
不能不去面對一些惡事.
跟它同在的時候會怎樣呢.
我就感到痛苦和悲傷.
所以悲傷和痛苦.
正正就是這批的情感.
這就跟Pleasure剛好相反.
當我們不是跟一些美善的事情在一起的時候.
當我們跟一些不好的事情在一起的時候.
我們就會出現到.
Sorrow.
就是痛苦的Passion.
這個就是第一批.
就是有關Sensible裡面的情感.
接著就是粉情.
希望大家能夠理解.
這些Passion是非常仔細的.
第二批就是什麼叫做粉情呢.
就是一些對於.
未能夠得到的東西.
一些將來的東西.
一些這樣的情緒.

$^{481}$或者是一些Passion.
叫做Irresistible.
就是粉情.
有些什麼呢.
它有分幾個.
一個就叫做Hope.
一個就叫Desperate.
Hope.
盼望.
中文翻譯成期盼.
你不一定要盼望一些很厲害的東西.
就是耶穌再回來.
耶穌再度過這些東西.
你盼望今晚不下雨.
都是一種盼望.
盼望今晚有宵夜吃.
都是一種盼望.
所以對於一些好的東西.
一些Goodness的東西.
你未擁有它.
但你仍然懷著一種盼望.
去想它.
這就叫做盼望.
這是一種人的情感.
相反.
就是我們的絕望.
我覺得明天一定會下雨.
你很悔.
這種Desperate.
正正就是一種這樣的情感.
這是一種Passion.
這是一種情感.
是一種情緒.
所以對於一些惡事.
我們對於將來那種絕望.
正正就是一種這樣的情感.
接著.
這些是一些好事的看法.
另外就是一些.
很惡的事.

$^{521}$很難的事.
第一個就叫什麼.
就叫做Dare.
這個就叫做大膽.
或者叫做勇敢.
你不怕那些東西.
對於一些非常大的困難.
你感覺到.
你是不會害怕.
你是有大膽去面對它.
當然這個不一定是好事.
你大膽到開大膽車.
大膽到你做一些惡事都可以.
所以這個不同於一個美善的勇敢.
這個純粹大膽才可以.
另外就是Fear.
就是恐懼.
恐懼也是.
可以恐懼上帝.
恐懼一些不好的事情.
都可以恐懼.
那你就不做了.
但它都可以出於恐懼.
純粹是一個害怕.
這些都是一些未及之事.
對於一些惡事.
對於將來的情緒表達.
最後一個就是Anger.
Anger就不是分好和壞.
因為好和壞都會Anger.
對一些正義的事.
你感覺到很憤怒.
你覺得那個人不對.
你就覺得很憤怒.
反過來.
你仍然對某些事都很憤怒.
所以這些情緒.
這些Passion成為了.
Thomas Aquanain非常重要的一個刻畫.
一個描寫.

$^{561}$所以簡單來說.
我有一個小總結.
Passion是分開兩個不同的情感.
一個叫做.
Conception.
一個叫Irresistible.
一個叫Sensible.
一個叫將來.
每個人都有一個好事和壞事.
這些好事包括什麼.
Love, Desire, and Pleasure.
接著一些惡事.
我會有一個Avoidance.
想避免它.
Hate, Pain, and Sorrow.
另外.
我們對於好事有一個Hope.
Desperate.
接著還有一個Evil.
有一個Dear和Fear.
最後就是一個中立的Anger.
這11個正面.
就是Thomas Aquanain所說的11個的Passion.
作為一個人.
作為一個靈魂.
必然會擁有的10個Passion.
它是有可以好和壞的.
我可以純粹Desire一些不好的東西.
Desire一些純粹物質的東西.
你可以去Hate一些不對的事情等等.
所以我想說這麼多是什麼呢.
這11個情緒.
是不是你這兩年裡.
正正面對著這個世界裡的情緒.
或者Passion.
所謂做一個有Passion的人.
其實正正都是這些東西.
當我們帶著Passion去做基督徒.
我們會有很多不同的愛.
渴望.

$^{601}$享受.
更加多的可能是痛苦.
或者是憎惡.
或者是討厭.
有時候我們會有盼望.
有時候會絕望.
有時候我們會勇敢.
大膽.
有時候我們會覺得恐懼.
有時候會覺得很憤怒.
這11個Passion.
作為一個有Passion的人.
是一個非常自然的事情.
是自然不會流露出來的事情.
不過它好還是不好呢.
記得,這要視乎什麼呢.
就是視乎你能不能夠用.
一個很好的東西來駕馭它.
所以剛才說了.
這11個不同的情.
Passion的意思.
愛,憎惡,怨望,逃避.
喜樂,哀愁.
期望,失望,畏懼,勇猛,憤怒.
這11個成為了我們很重要的Passion.
不過這11個Passion.
你能不能夠把它成為好事.
就在乎於什麼呢.
阿輝就告訴我.
就在乎這個Virtual.
Virtual正正是能夠.
好像一個提靶一樣.
來控制著.
或者有效地監管著你的Passion.
讓你的Passion不會太過厲害.
過度紅河氾濫.
成為一個災難.
得勝Virtual是一個非常重要的天主教觀念.
Virtual是一個上帝賜給我們的尾線.
信望,愛.

$^{641}$上帝給我們的聖靈.
正正可以令我們這些Passion.
成為一些非常重要的力量.
我們不可以沒有Passion.
因為沒有Passion.
我們就是一個沒有生氣的人.
我們懷著Passion去做人.
不過他要有這個得勝.
這個Virtual.
好好地去善用我們的Passion.
所以今天我們沒有機會詳細講Virtual.
因為這個是天主教的概念.
不過意思就是說.
我們仍然是需要上帝.
需要一些尾線的事.
來流露我們這11個Passion.
憤怒,哀愁.
這些不是差的東西.
只要我們為著一些正確的事.
只要我們懷著尾線的得幸去做.
就是一個美好的Passion.
令我們更加流露我們得幸的Passion.
就像這提法一樣.
我剛才講了.
講了這麼久.
講了很久.
好像不太關13世紀的神學.
關我們什麼事呢.
我想到四點和大家分享.
當我們去問.
我們基督徒要做一群有血有肉的人.
其實會是什麼事情呢.
第一就是我們.
一個叫做生命熱情.
既然Passion是我們一個很重要上帝.
給我們很自然的力量的時候.
我們本身對於生命.
是應該充滿著熱情的.
我們是懷著一種非常之.
享受和熱愛生命的激情.

$^{681}$來做人.
不知道大家有沒有看過這部電影.
《靈魂奇境》大家有看過嗎.
在這部電影裡面.
很特別.
整部電影是一個非常基督教的電影.
當中記得.
這部電影是說一個靈魂.
在這個世界裡面不斷去尋找.
基本上是兩個主角.
主角要去幫這個靈魂.
去投胎.
他也要去拿著貼紙.
那個Divine Spark.
這個靈魂BB.
一直都不能夠做人.
因為他找不到那個貼紙.
這個令到他.
是一個非常之有Passion.
做的事情.
他覺得有試過打球 唱歌.
什麼都試過.
但都激起不了他的Passion.
所以做不夠貼紙.
來做一個真正的人.
後來他成為了真正的人.
是什麼原因呢.
就是他能夠真正的來享受生命本身.
重點不是一定是說.
你能夠做到什麼目標.
記得那個主角.
這個Jazz Player.
他以為可以做一個出名的Jazz Player.
之後能夠成為一個最人生高峰.
但原來不是.
只是每天都做同樣的事情.
重點是什麼呢.
就是我們是享受生命本身.
不是純粹追求某個生命目標.
或者某一件事.

$^{721}$所以這種熱忱.
不是單單某種人生的高峰.
或者是我們生命中某種狀態.
而是單單回到我們生命本身.
純粹帶著一種Passion去享受.
和面對我們生命.
我們活著.
為你活著而活著.
我們去學習.
不是因為外面的環境.
或者我們的際遇.
才有熱忱或沒有熱忱.
而是單單的生命.
已經讓我們能夠充滿熱忱去做人.
這是第一點.
我們做Galto.
做一個人.
最基本最基本.
就是懷著一種Passion.
上帝給我們生命的Passion.
來面對我們的生命.
接著才面對我們人生裡.
香港這個年代很多不同的事情.
所以這是第一點.
我們找回我們生命本身的熱忱.
這不是一個.
被這個年代的時勢改變的東西.
我上個月和我的中學同學吃飯.
我們就說開了.
因為上學都是最後一次見面.
因為很快之後.
有一個去英國.
一個就回到美國.
一個就在中國上海工作很久.
其中一個同學就說.
我真的很想來到現在.
Dream House.
就是我60歲的時候.
在澳洲找一間很漂亮的海邊小屋.
白色的.

$^{761}$很漂亮的.
這就是我們的生活目標.
但其實.
我們如果將這個成為我們的生活目標.
這樣也很糟糕.
那就是說什麼呢.
就是說你現在這20年.
基本上是一個.
instrumentalized的生命.
純粹是為了達到目標而過的生活.
不是.
我們生命本身就是我們生命的熱心.
不是我們某個生命的高峰.
才是我們的熱心.
所以這是我們第一點.
我們懷著這種passion.
來做人.
第二.
剛剛相反.
就是我們去拒絕犬儒.
犬儒正正是.
一個passion相反.
他可以充滿智慧.
他可以充滿高見.
但他有一種對於事實.
對於reality.
其實是放棄你的態度.
因為他害怕牽涉在裡面.
所以他抽身.
單單在Facebook裡面.
觀看世事.
不敢.
也不願意將自己的生命.
投放在這個時代裡面.
就是一個犬儒主義者.
他可能是沒有被傷害的.
因為他害怕被傷害.
所以就不牽涉在當中.
冷眼旁觀.
去看這個世界.

$^{801}$這是一個犬儒主義的問題.
不知道大家喜不喜歡play safe.
近來我們很喜歡這個.
安心play safe.
play safe永遠是最好的.
play safe是一個最安全的做法.
不如大家都.
play safe大家就戴口罩.
不要出街.
play safe就不要做任何take risk的事.
教會不如我們怕有人感染.
怕被人搞.
我們就不要怎樣怎樣做.
我們可以做一個play safe的人.
做一個play safe的教會.
做一個play safe的生命.
但是play safe正正是一個非常安全的原理踏步.
當你的生命不去面對任何風險的時候.
這個好像很安全.
但其實這個非常.
開始已經是投降.
已經輸了.
所以這個我們不要去.
我覺得play safe是一個很好聽.
一個很好的教會說話.
但我們過分地play safe.
其實是一個不是太好的決定.
第二就是defeatism.
就是失敗主義.
因為我們過分害怕失敗.
所以我們就不牽涉下去.
我們就不嘗試去參與.
不嘗試去全情投入下去.
記住passion是同時解作甚麼.
解作受苦.
當你有熱忱去做人的時候.
是會經歷很多很多不同的.
被傷害的可能.
但play safe不是你的出路.
失敗主義也不是.

$^{841}$因為我過分害怕失敗.
所以我就寧願不去全情投入去參與.
這個是我們面對犬儒主義的問題.
第三既然passion是解作我們.
要將我們的靈魂身體完全獻上.
盡心盡性盡義.
正是我們一個非常非常重要的基督態度.
上帝呼召我們.
要去盡心盡性盡義.
來去敬拜去愛我們的神.
和好好的去扶人.
這種盡心盡性盡義.
全情投入獻上活祭.
完全的立志擺上.
正正就是passion的重要生命.
這個我只是想.
Fold Church的姐妹們認真思考的一點.
我覺得Fold Church如果是DNA的話.
其中一個很重要的DNA就是這樣.
我們是一個追求全情投入的群體.
你看到我們搞神學講座也可以搞到這樣.
敬拜隊,爆風球也可以出來出隊.
每一個位台前幕後都是完全到去盡心盡性盡義.
去追求最好.
這個就是我們passion.
passion不是沒有成本的.
當我們願意去做好一件事的時候.
是會擺上很多的事情.
但這個我覺得這個就是我們Fold Church很重要.
第六課要講的事情.
我們懷著passion來做基督徒.
來思考我們的教會.
一篇講道,一個powerpoint,一個拍攝.
都是我們盡心盡性盡義的獻上最好.
當中擺上我們是可以有的.
這個我覺得是很值得我們.
可以好好投入Fold Church這樣的教會.
所以我們懷著passion來做基督徒.
來獻上我們自己.
最後我要給大家一個德文.

$^{881}$第四就是Leidenbereit.
什麼意思呢?因為很難翻譯.
什麼叫Leidenbereit呢?.
Leiden怎麼解釋?.
就是suffering,Leiden.
Leidenbereit是什麼意思呢?.
就是一個…怎麼說呢?.
bereit就是預備的意思.
既然我們是做一個有passion的人.
我們就要Leidenbereit.
簡單來說就是這樣.
我們是準備受苦或者面臨傷痛.
這樣不代表我們會面對傷痛或者受苦.
但我們既然有passion的時候.
我們就自然地.
預備好我們隨時會有相對的傷痛或者受苦.
愛本身就是受苦.
當你愛一個人,當你闖開心去愛人的時候.
你就預約給別人.
一支箭來傷害你.
你可以不受苦的,你就收起自己.
你就不參與,你就不獻上.
但當我們既然是做一個有passion的人的時候.
我們就會預備好自己隨時會面對這個passion.
這個正正是基督徒受難的意思.
所以passion很明顯兩個意思就是這樣.
我們全程鋒利地去做人.
同時也預備好迎接生命裡可以出現的受苦.
但我們不怕,因為我們願意去做.
所以我們總結一下,四點.
就是生命熱情.
我們來對生命,最基本對生命充滿我們的passion.
不需要人生的高峰或者某個階段.
生命本身就是我們passion的對象.
第二就是拒絕犬語.
我們不願意play safe.
我們不願意純粹去害怕失敗.
第三,我們要盡心盡性盡義.
做好我們每一件事.
當中隨時去預備好我們要付出的.

$^{921}$第四,lighten the light.
我們預備好我們可以受苦的.
所以基督徒當然要做一個人.
但我們more than一個人.
我們會有血有肉之餘.
我們仍然不會說我們是一個人.
當然會這樣.
我們仍然會做一個基督徒.
嘗試來獻上最好.
這個就是我想和大家分享的第六課.
passion.
一點思想時間.
你想一下你自己是不是一個passionate的人.
是不是一個預備好自己.
有passion去做基督徒.
面對著這個年代.
我們的passion在哪裡.
我們怎樣能夠得著更多的熱忱.
面對這個世代.
我們一起祈禱好嗎.
祝福我們求你幫助我們.
當中我們每一個參與的頂尖妹.
讓我們能夠可以善用你給我們的passion.
我們的情.
我們對於你的愛.
對於很多惡事的憎恨.
我們的憤怒.
我們的憂傷.
我們的盼望等等.
求主你讓我們仍然有一個美好的virtue.
一個德行.
來好好管治我們這份情.
讓我們做一個有血有肉的人.
卻是成為一個能夠有美善的人.
將我們這份情懷.
這份passion.
好好來善用.
我們這個full church交託給你.
求主你幫我們這份教會.
成為一個有熱情的教會.

$^{961}$去獻上最好的給你.
夢主命求.
阿們.
沒想過你又來.
因為其實打了兩天風.
我就沒東西吃.
是嗎.
所以你看到我們只有水.
是這樣嗎.
其實你在賣什麼.
其實我們就.
公仔麵也有.
不是,我們是阿勒卡.
你想要什麼我們都可以做給你.
即是深夜吃糖果那些.
應該你來多幾次就知道了.
不過剛才我聽你說的內容.
我覺得我們做這間食店.
是有這個passion.
你有熱情果食.
我們沒有這些這麼西方的東西.
但我覺得我有這個passion.
就是因為我們很多時候都all in.
人家要什麼我們就大家去配合.
但其實對於普遍的街坊來說.
有沒有一些日常可以具體的例子.
可以用得著.
你覺得呢.
我自己覺得.
通常我和街坊都說.
你想吃什麼.
譬如我們擺生日飯.
就問他們有什麼想做.
看想吃飯.
我們就用他們的錢.
就幫他們做得好看.
但對於我們普遍街坊來說.
能力可能不是很多.
剛才你介紹了四點.
有些生命的熱情.

$^{1001}$就是說他自己的命.
或者能力可以做到什麼.
其實是怎樣可以用得著.
在日常生活裡.
我覺得這個反而是一個最基本的東西.
因為生命的熱情.
我覺得是人越大.
慢慢會明白這個東西.
純粹對於生命是一種的熱情.
不是生命裡面某些東西的熱情.
我們可以很喜歡沖咖啡.
對咖啡很有熱情.
你喜歡玩模型.
對模型很有熱情.
但你會發覺那些東西都會淡.
都會覺得我已經做了.
我已經做了某個位置.
但如果你對於生命沒有熱情的話.
你做什麼其實都是浪費的.
所以我覺得靈魂奇遇是很好看的.
因為不是在乎於你的際遇.
而是對於你活著這種的情.
那種的熱情.
那種的興奮.
那種的熱愛.
我覺得是首先要的.
基督徒我覺得是要去找回.
最根本對於生命的熱情.
這個我覺得才是最重要的東西.
我了解你來說.
我們盡量去探索一下.
然後就會找到一些我們覺得值得追的東西.
是不是這個意思.
可能就算是.
都不是.
我沒有看過《套戲制》.
我沒有.
《套戲制》說.
最後那個人就說原來我單單活著.
都是很好.

$^{1041}$原來我站起來.
原來我站起來看到外面的風景.
我能夠繼續做人.
都已經是值得我們去懷著熱情去做的東西.
所以你就可以很甘於我今天.
我未必一定要做什麼成功的大事.
而是我單單對於生命本身的熱愛.
這種正正是我覺得是一個很重要的第一點.
然後才去問.
我懷著這種熱情去做什麼呢.
不知道大家怎麼看.
大家有沒有看過那部電影.
每一天都是一個新的一天.
可以感受到那種熱情呢.
或者你現在每天有沒有熱情呢.
好像有點難吃.
或者你感受不到熱情.
有沒有想過什麼原因呢.
我自己覺得在受教育過程當中.
很少講感受.
很多時候都講自性.
講自性的發展.
講成績的表現.
講那件事能不能做到.
其實不是很講個人感受.
就好像你剛才所說的.
就是不起眼於食.
不容易被人觀察到你自己的情緒.
這樣就算是型.
不知道大家有沒有覺得.
在過去其實不是很懂得講自己的感受.
學校又沒有教.
而在過程當中很著重那種表現.
表現到你有什麼能力之處.
大家是怎樣成長出來的呢.
我再講一下第一個問題.
生命的熱情.
剛剛10月9日不是颳颱風嗎.
那天是我的生日.
其實我那天早上已經預計會回到Folk Church.

$^{1081}$總之就是出來吃飯.
中午的時候.
我想吃完飯就出去.
做完飯就打完.
就退下.
我覺得每年我都會寫一些東西.
今年我沒有寫.
因為我發覺.
開始人到了那個年紀.
原來颳颱風是趕走我開始的心境.
原來不需要有什麼.
不知道那天颳颱風是做什麼的.
由早上到晚上.
當時我忍到他出門.
發現他全天都在家.
不知道他悶不悶.
但原來你可以純粹在家裡.
當你燃起了這份對於生命的熱情的時候.
你不需要有內心.
不需要有那些東西.
我十幾歲的時候很喜歡去玩.
但你發現到了三個年紀.
你發覺我不需要去充實自己的生命.
而生命本身就是一個可以令你有熱情.
然後你才嘗試去做一些你心裡的東西.
不知道你有沒有這樣的想法.
我有啊.
你大我有吧.
我有啊.
為什麼我說我有呢.
我喜歡做運動.
但我老婆不喜歡做運動.
她現在沒有看應該不會的.
我老婆完全不喜歡運動.
連車都不會追.
我做運動的時候會流汗.
回家就會臭.
她問我為什麼這麼喜歡運動.
我說我覺得流汗是我能夠感受自己活著的感覺.
她說那你就當我是一個死人就行了.

$^{1121}$(笑聲).
她問我為什麼流汗是活著的感覺.
我說死的生物是不會流汗的.
然後她說那你就當我是死人就行了.
我覺得能夠有生命氣息.
能夠流汗就是一種活著的感覺.
我覺得這就是你所說的生命的熱情.
不是做什麼的.
流汗我已經覺得很開心.
我們要抓緊這件事才能面對這個世界.
十幾二十幾歲的時候我們是相反的.
還沒有自我的時候.
就開始去找一些東西去補充自己的生命.
去玩或者去目標.
讀大學.
但人是很空的.
還不知道生命是什麼.
當你發現上耶穌.
你知道上帝給你生命的時候.
我們用我們的心腸和肉體去歌頌我們永遠的上帝.
我們就發現我們正正就是可以單單活著.
來享受這個活著.
因為我自己也是十幾歲的時候.
我為什麼要上耶穌.
因為我覺得不知道為什麼要做人.
我覺得死了之後不知道要去哪裡.
然後就很想找一個人生目標.
什麼是我的人生目標.
我是一個很目標主義的人.
但是發現原來現在是不需要有這個目標.
正正一個很粗略的.
就是叫做「我們活著就是為了活著而活著」.
我們活著不是為了一些超級厲害的目標.
成為一個偉大音樂家.
或者成為一個怎樣怎樣.
買了幾層很大的房子.
而是單單當你有一個位置.
去到我純粹活著都能夠為了活著而感到有個熱誠的時候.
這樣你才能再做一個人.
才能面對今天的香港.

$^{1161}$或者面對今天面對其他的東西.
有些懸念.
但是我覺得這個很重要.
對於生命的熱忱.
如果你對生命沒有那種熱誠的話.
其他的熱誠是有的.
但是你會失望的.
你會發現做到了.
但你會發現原來都是這樣.
做不到你會覺得很灰心.
所以我覺得這點是很重要的.
對於生命的熱忱.
後面.
你好啊 我叫Amy.
其實我自己很有共鳴.
我猜大家不說話.
可能他們年紀不夠大.
因為人慢慢長大了.
我自己也少了定一些目標.
這個不知道是不是以前的教會氛圍.
每年有中,短,長期目標.
在整個教會設定.
都會很想去最後有些慶祝.
有些成果,收穫.
那就成為我們一個很有熱誠去追求的東西.
但是長大了之後.
慢慢覺得過了之後又怎樣呢.
我的人生還要繼續.
那是不是單純是我生命的那一點呢.
所以這是有共鳴的.
相信這裡在座的太年輕了.
但是有另一件事.
就是我想問你.
怎樣去感受自己的那種溫度呢.
你一個人很敷衍.
還有一個人很有熱誠.
那怎樣去分呢.
你說沒有目標.
也不可以完全沒有目標.
完全沒有那種向前的動力.

$^{1201}$我記得有一個年輕人.
跟我說他人生的座右銘就是.
世上沒有難事.
只要肯放棄.
(笑).
你可以去到每一天都不需要定目標.
去到可以很敷衍.
很沒有方向.
但是有時候教會或者.
以前的傳達教我們就是.
不可以太敷衍.
要有熱誠.
但是你剛才說的那種.
就是一種很自在.
可能去到某一個境況.
才可以去到這個狀態.
又不會介乎太敷衍信仰.
他也可以用這個態度去擺信仰.
不讀經 不領修 沒有目標.
一年讀完都是這樣.
那你自己怎樣去感受那種.
自己仍然有那種熱情.
怎樣去保持那種狀態.
大概是這樣.
其實你很有熱誠.
我覺得自己也不錯.
但我不太明白你怎樣去量度.
或者我怎樣知道.
或者弟兄姊妹我怎樣去教他們.
太多目標又好像太追求某些東西.
但沒有的話.
他真的跟你說躺平的態度.
那又怎樣呢.
我想這裡有兩個不同層面.
我覺得剛才第一個是生命的熱忱.
躺平是一點點.
躺平是不是好事.
我沒想過這個位置.
躺平是不是沒有熱忱呢.
我覺得可以有很多不同的躺平.

$^{1241}$我們.
躺平可以是一種失敗主義.
我投降了.
我其實已經犬喻了.
但也可以成為.
我仍然是享受.
或者我是熱忱地去活著.
所以我覺得有兩種不同的情況.
先說完.
這裡有兩種不同熱忱的定義.
我剛才聽完Emmy說的時候.
我也接觸過.
弟兄姊妹她也會回應.
可以吃的話不會動.
但對我來說我是挺闊的人.
我對弟兄姊妹的要求不算高.
有時別人覺得我很脾氣.
或者覺得包容度很大.
某程度上我覺得也是.
但我自己看科幻書.
我感受到耶穌不是一個很推動的人.
特別是在跟門徒相處的過程當中.
但如果真的要跟耶穌.
他就會有要求.
但要跟耶穌相處的時候.
耶穌大部分時間都是帶著門徒去看東西.
去經歷.
在過程中跟他們問問題.
做互動.
最後你看到耶穌也是用free end的方法.
讓他們想想.
你自己想清楚.
你也看到.
或者你也經歷過.
其實你在想些什麼呢.
所以剛才在信息裡.
有一個位置我覺得是很大的提醒.
當我們知道很多定義.
當我們知道很多別人想過的東西.
其實對於你來說是什麼回事呢.

$^{1281}$這是最重要的.
過去教會的教導很著重自成.
提醒經民.
但其實那些經民有多靠近呢.
其實從來都是自己去經歷.
和自己去用頭身那段經文的要求的時候.
你才會知道那件事對你來說有多大的passion.
你是不是受苦.
你有要求.
你就要去熬.
還是你自己去從那些經文裡.
或者教導裡找到自己的生命意義.
你一定要自己去嘗試.
不是別人告訴你.
也不是說要訂立一些條款.
做了就是了.
所以我自己覺得.
我大部分時間都是和弟兄姊妹一起去.
給他看見.
給他經歷.
或者給他問.
其實最大的.
你有興趣的地方是什麼.
所以無論對於你去想什麼叫做passion的話.
其實我仍然認同.
回到John的第一個本位.
就是其實你活著.
你每天一早起床.
你告訴自己我是怎樣生活.
這個是最根本的地方.
不然就很糟糕.
我覺得我和潘Sir談18課的內容.
其中一課初時叫做all in.
但他不是叫all in.
我想究竟full church有什麼DNA.
我們是DNA出來的.
我們是弟兄姊妹來說出來.
大家可以這樣去追求.
或者去想.
因為我們覺得基督徒是一個.

$^{1321}$八課裡面的一課想說的.
類似之前叫all in的名字.
後來我們就叫passion.
我覺得.
是也是的.
其實都幾all in.
不要犬儒.
盡心盡意.
又要承受苦難.
都幾是那些.
剛才Evan所說的那種.
但大前提就是第一點.
這件事不是要逼自己.
也不是一件很慘的事.
passion就是兩個意思.
就是熱忱和受苦.
那個熱忱是你很喜歡.
這樣去活出來的.
但同時間會遇到一些.
任何事情的一些事情.
因為這個字相反就是.
你去折迷.
就是犬儒.
我拒絕受傷.
我怕受傷怕失敗.
我就不去passion地做人.
我收起自己.
收起自己的情感.
不敢說話.
我覺得我們很想的就是.
當你真的看不到第一點的時候.
我們就自然而然去流露.
那二三四點出來.
當我們未找到最基本的.
對於上帝給我們的生命.
那種passion的時候.
其他的都是一種很律法的規條.
你一定要去盡.
一定要做到最好.
追求出月.

$^{1361}$其實不是這樣.
任何人.
就像銀紙比喻.
你多少錢的銀紙都能夠.
all in到的.
是法律的all in到的.
不是要強行逼著.
因為教會叫你釋放就做.
而是我帶著熱忱去參與.
這裡也是.
很多弟兄姊妹去參與.
福祉去釋放都是這樣.
我從來都沒有叫別人去做什麼.
但每個人都可以帶著熱忱去做.
是辛苦的.
但是快樂的.
因為我是享受的.
所以passion這次就是將享受.
和少許困難.
和我的生命.
混在一起的狀態.
就是我們做基督徒.
是一個很重要的特徵.
網上有問題嗎?.
Liz Mok問.
熱情是否需要一些載體.
才能展現出來呢?.
載體是否即是要一些行動的意思?.
我不明白.
應該就是了.
載體.
因為載體這個詞其實很空泛.
不過如果照我的理解來說.
熱情是需要有一個載體.
就是在場景上.
要有一個場景.
要有一個互動的時候.
才能感受到熱情.
不說其他,說自己吧.
有些弟兄姊妹經常問.

$^{1401}$我怎樣運用時間.
他們覺得我很忙.
我自己覺得我不是很忙.
我會和他們分享我的時間表.
或我的工作安排.
他們說我這樣也不算忙.
那我就是不知所措.
我說不用比較.
但我很享受.
我沒有一天很….
簡單來說.
他們看我的時間表.
很多時候都是見人的.
我說是呀,我很喜歡見人.
然後他們問我見人不累嗎?.
我說見人怎會不累呢?.
但重點不是我自己的感受.
重點是.
每當我聽弟兄姊妹的故事的時候.
我就感受到上帝在她生命當中的工作.
或者感受到上帝在她生命當中做著一些事.
所以你問我的熱情.
我的熱情是見到弟兄姊妹.
和她相處和見面的時候.
那個場景就是我的熱情的載體.
所以我從頭到尾都覺得自己是一個教會人.
我喜歡在教會的環境當中存在.
這就是我每天最期盼的熱情場景.
我用我自己的例子去了解熱情是載體的地方.
(字幕提供:Johnny).
(前面那裡,Jacky).
剛才聽到Johnny提到.
在熱情之前應該會有些熱愛.
剛才我聽到Denise的問題.
我在想是否需要熱愛才會開始有熱情在裡面.
假設我對熱情的愛可能消失了.
例如我突然玩了一個遊戲.
可能玩到厭惡了,不喜歡了.
我就不會再有熱情去玩這個遊戲.
或者去做某些事.

$^{1441}$如果是這樣的話.
我就好像沒有了載體.
我怎樣可以繼續有載體呢?.
這是一個很好的問題.
回到剛才討論過的問題.
我們….
就如Thomas Aquinas所說.
有十一個熱情.
這是我們人必然會有的東西.
其中一點我剛才沒有說.
任何熱情最根本都是愛.
他也這樣說.
因為Hatred也是愛的相反.
首先因為愛而衍生成Desire.
或者是Pleasure.
玩模型.
所以這件事.
當然你玩遊戲怎會玩到厭惡.
但我覺得可以的.
你不就玩到厭惡.
但我覺得最基本的就是.
對於生命的熱愛.
因為生命就是載體.
生命就是最基本.
如果我們連生命都沒有熱愛.
來開始玩模型的話.
明白嗎?.
模型可以玩厭惡.
但我覺得我做人做到很厭惡.
這就很糟糕.
這是我們很多人.
我自己也是.
原來我自己也不是很熱愛生命.
曾經那段時間也是.
所以我覺得.
這是我們最最基本的熱愛對象.
就是生命本身.
所以這是我們需要去燃起的第一點.
我們熱愛生命.
然後才去玩任何的東西.

$^{1481}$其他東西你可以有不同的轉變.
其實我最近很喜歡我女兒的舉動.
因為她做了一個很熱愛的人.
就是熱愛生命.
她整個人的喜怒哀樂都是很熱愛的.
她很開心就很瘋狂.
然後又會不開心,生氣.
其實我們也是.
我們發覺人長大了.
反而我們會收起這些熱愛.
但最基本的就是對於生命的熱愛.
如果你對於生命的熱愛.
你其實不太害怕其他人怎樣看你.
另一個部分我想說的是.
其實這些都是環教會的問題.
很多時候我們都很藏起來.
做教務就是要你循規蹈矩.
這是一個很無情的.
你開心就不要太瘋狂.
不開心就不要讓別人看.
但其實當我們熱愛生命的時候.
我們可以做出這樣的一個樣子.
我覺得這些東西是可以沒有的.
但當你捉到生命的熱愛.
其他東西其實都不是大大問題.
這是你活著很重要的東西.
我不做就做其他東西.
就像風琴那天.
在家裡也可以.
這是一個很重要的東西.
否則當你不是生命.
你做什麼都會覺得好像一個感覺不到的洞.
但其實你做了很多事情.
你覺得自己是一個很好的人.
因為你說其實罪的其中一個來源是熱愛.
其實我們壓低自己的熱情.
剛才你說教務的案件.
其實你不知道你投放熱情出去的時候.
你不知道是不是在投放斧頭.
在投放人.

$^{1521}$當你不知道你那個.
剛才你講了一句.
你沒有發展下去.
當我們自己不知道自己的斧頭是否可以.
當你嘗試盡心盡力去愛人.
或者怎樣都好的時候.
你發覺自己中招了.
犯了罪.
然後你救不回來.
沒得救.
當然你可以找神.
但無論如何.
你都是想儘量避免犯罪.
你避一避.
例如我們講回投放斧頭的案件.
你不想投放人.
最簡單的就是你見不到人就投放不了人.
(收起了).
收起了.
將所有的情感收起了.
你就不會無緣無故.
本來我打算很開放地跟你說話.
然後我很熱情地跟你說話.
原來那個人是接受不到那種熱情的.
他就覺得你在傷害他.
但其實你是很嘗試地跟你的朋友相處.
但其實你是不停地傷害他.
但你還是喜歡他.
你不是有心傷害他.
而是對他很熱情.
當你慢慢長大的時候.
你就會發覺.
有些事情是不可以太開放的.
你不開放反而是對人好.
你少投放一些斧頭.
對方就會自在一點.
慢慢就會覺得.
有很多事情是不應該做的.
可能小朋友有很多事情都可以自在地發表.
但你長大之後就沒有了.

$^{1561}$所以慢慢就會變成.
你不要騷擾我.
很被動.
但怎樣可以避免這個情況.
我明知我的熱情可能是罪的來源.
但我仍然可以繼續有熱情地做.
這是一個很好的問題.
其實正正就是.
為何很多人長大後會這樣.
就是因為有問題.
所以寧願收起.
所以那個方法.
正正不是收起.
而是增加那個熱情.
那個方法不是減低自己的熱情.
而是讓那個熱情更加有效地發揮出來.
令你不需要收起你的熱情.
就像河流.
重點不是不要留著河.
而是要更加控制住這條河.
那個控制.
不是控制自己.
這個名字好像很藏起來.
建立一些德行.
其實今天是一個很厲害的天主教概念.
一個.
最基本的信望愛.
Theological virtue 信望愛.
另外就是智慧.
這些東西正正可以幫我們.
將我們的熱情變得更加美麗.
譬如智慧.
我們說話是有智慧的.
不過我想說的是.
我們基督徒做人.
有血有淚的人.
我覺得一半對一半不對.
我們有血有淚是對的.
但有血有淚會傷害別人.
我只是說真話.

$^{1601}$這些是人話.
但人話也會傷害別人.
所以需要一個智慧.
需要有智慧去判斷.
我什麼時候說什麼時候不說.
我需要有愛心去面對這些東西.
所以重點是不要藏起自己的熱情.
而是問如何增加自己的德行.
這個有很多.
譬如是可以培養出來的.
Habitus.
有些德行可以培養出來.
信望愛是神給你的.
我們的重點是要增加.
基督徒應該有的智慧.
多於藏起這些熱情.
這個就是理論上的答案.
不需要藏起.
而是要好好增加自己的德行.
我做了40年.
慢慢發現自己以前也很藏起.
但發現當我建立多些Virtual.
這些東西我可以不需要藏起.
可以坦然無懼地將熱情流露出來.
也是好的.
不知道Person有沒有補充.
其實回應那個內容.
我回想起彼得後書第一章.
講關於生命和敬虔的教導.
第一章是講這個.
有了生命敬虔的表達.
就是有了信心就有德行.
有德行就有知識.
有知識就有節制.
節制加上忍耐.
基本上那幾節的經文裡.
一直慢慢提升.
到最後就是要有愛弟兄的心.
其實一個屬靈生命的追求過程中.
不斷地慢慢經歷完.

$^{1641}$上帝會加添.
而加添過程中.
上帝會幫你補滿.
你一直在過程中.
發掘了更豐盛的可能性.
到最後所有東西加起來.
就是令你能夠愛弟兄的心.
或者愛眾人的心.
這就是在行為上讓事情可以延展下去.
我想你過去可能也聽過.
或者早前也流行.
愛的反面不是恨.
恨也會花力氣.
就像剛才John所說.
Hatred其實是另一方面的愛表現.
他也會留意一舉一動.
才可以釘住他的字.
但是重點就是.
有些人現在愛的反面不是恨.
愛的反面是無情.
是不理會你.
對他來說.
這不是他最賞心的事.
反而我覺得這件事更難處理.
因為他已經沒有下一步.
對他來說那件事是什麼呢.
不是不重要.
不過你跟他說也無關痛癢.
不想再注視那件事情.
我覺得反而是難處理的.
另外我一點回應就是.
可以看一本書.
這本書比較舊.
是David Benner寫的.
《The Gift of Being Yourself》.
正主翻譯了那本書的中文名.
《天賦給我的禮物》.
是一本薄薄的書.
他是基督徒的心理學家.
也是一位牧者.

$^{1681}$他寫了這本書.
是一本很薄的書.
但這本書正正反映了一件事.
就是你的本相是一個怎樣的人.
其實看這本書是不斷去.
評價自己的本相.
評價自己作為一個人.
用我們今天的詞語.
作為一個人你的熱誠是什麼.
招牌沒有拆掉.
熱誠是什麼.
所以對我們來說.
《The Gift of Being Yourself》.
我中括了兩句話.
就成為這本書對我最大的提醒.
第一,Be Yourself.
做回你是一個怎樣的人.
上帝做你.
這個世界只有一個潘志剛.
同名同姓同性別都好.
但都不是同一個人.
所謂Be Yourself.
但如果你說Be Yourself.
我喜歡做什麼都可以.
亂來都可以.
David Benner 說.
Behave yourself.
就好像剛才提到.
彼得後書一章裡說.
有不同的東西加添.
當你Be Yourself的時候.
上帝會不斷加添.
補滿.
然後讓你最後可以有愛的展現出來.
這個就是我相信.
Passion 而有的生命的豐富.
謝謝.
我想回應剛才弟兄提出的.
我覺得如果用自己的熱情做事.
但傷害了別人.

$^{1721}$其實我們不是放棄.
而是從錯誤中學習.
因為我覺得沒有一樣東西.
或很少機會是你第一次做.
而你做得非常成功.
我覺得熱情的事.
是一樣值得你花時間去做.
如果你覺得愛人的時候.
你做的事是傷害了他.
即是你做的不是他需要的事.
你繼續去愛的時候.
你就會明白.
原來他要的不是這樣.
你就會一直不停地改善的時候.
你會做得更加好.
而你真的會做到你想做的事.
但如果因為錯了一次就放棄的話.
我覺得很難成功做到一件事.
是一個很好的回應.
因為愛從來都是經驗學習.
你會發覺我們與生俱來.
不是很懂得了解自己的取態.
很多時候你會看到小朋友.
都有一個表現.
我看我兩個兒子成長的時候.
我發覺只要是那個歲數.
就會有那個表現.
還沒有讀書.
意思是甚麼呢.
看他們過年的時候.
我就教會他們甚麼是「傳合」.
很多人是「糖果」.
所以他們主打供人「糖果庫」.
我都會看到小朋友是怎樣呢.
他口吃一件東西.
手拿著一件東西.
眼睛就看著那件東西.
全部都是給自己的.
人很多時候都是從自我出發.
去了解自己的需要.

$^{1761}$很想拿取一些東西.
但我們的信仰教愛人如己.
推己及人.
或者非以人乃於人.
是對外的時候.
其實這些全部都要學的.
不是我們知識上知道我們懂得做.
所以剛才姐妹回應一件事.
你有時傷害一個人.
你才知道說話有分寸.
你傷害一個人才知道.
懂得看別人那種反應的時候.
避重就輕.
這個對我來說.
都是一個比較有熱誠.
對生命有熱誠.
願意互相造就的過程.
如果回到剛才我說的.
如果他無情.
他就說「雖然理得你」.
「我已經說完我的事了」.
「完」.
那件事就會越搞越糟.
是不是網上有問題?.
菲姐問兩條問題.
第一條問題是.
她以前是一個很熱情愛主愛人.
現在變得很麻木.
甚至冷血.
怎樣才可以回復以前的狀態?.
第二條問題是.
以馬五師的門徒聽到耶穌講解聖經之後.
就心裡火熱.
為何現在的教會沒有這個效果?.
每間教會不同.
有些教會有的.
關掉冷氣.
我想第一個問題.
其實不是一時一刻就解決到.
我覺得是上次開學崇拜所說的.

$^{1801}$我都說臨彼祈禱.
求神給我們受傷的靈魂.
能夠回到初心.
其實初心就是那份熱誠.
我們小時候.
其實熱誠是每個人都有的.
重點不是問你怎樣有熱誠.
熱誠是自然的.
最自然就是有的.
不過我們因為受傷了.
或者我們長大了.
我們會去掩蓋它.
所以問題不是有或沒有熱誠.
而是你怎樣能夠在受傷的過程中.
能夠放膽將最初的那件事.
活出來.
從信仰到你怎樣看自己.
或者做人.
怎樣理解生命之類的東西.
這些全部都是我們本身有的.
上帝創造我們小時候.
小朋友就已經很有熱誠.
對於生命,對於世界,對於上帝.
只不過我們長大了,受傷了.
才會有很多這些加在上面.
所以我都覺得要慢慢來求上帝幫助我們.
去醫治我們.
我想這個有很多不同的情況.
不是一下子就沒事了.
所以我覺得這個重點不是行為.
不是我們靠做些什麼.
才能夠有熱誠.
而是問你本身都有.
怎樣能夠回復.
不妨聽聽我們在哪裡開學崇拜.
我這個都是求上帝.
讓我們重現這份熱誠和初心.
我的看法就是.
剛才都說到熱誠.
或者愛是一個經驗學習.

$^{1841}$所以有些事情不是一步到位.
或者很快就看到那個果效.
正如我開初都說.
我們的受教育過程當中.
常常都很著重表現.
就是有些東西出來讓人看到.
沒有的話你就會失望.
我們看結果多於看結果.
結果就是數據,數據,數據.
有東西可以讓你看到的.
有些東西是可以讓你看到的.
但結果是什麼?結果是改變.
改變是需要時間的.
時間就在過程當中.
改變很慢和不明顯.
要很久才能看到改變.
但我們對自己的要求都很高的時候.
就會很迷惘.
正如John所說.
他的篇幅就是.
他現在都覺得追不上.
他以前二十多歲的自己.
在這個過程當中.
總會有很多東西不斷想去.
重現原先的熱誠.
但我自己的看法就是.
我喜歡馬太福音第五章的結尾.
你們要完全將你們的天賦完全一樣.
我很喜歡完全這個字.
完全這個字希臘文是Teleos.
Teleos這個字根是T-E-L-E.
Tele是什麼?.
拍照就知道Tele.
鏡子是遠的,看遠的東西.
我們看遠的東西是什麼?.
就是看遠鏡.
就是Telescope.
Scope就是View.
我們用望遠鏡看遠的東西的時候.
那東西是否在那裡?.

$^{1881}$那東西是在那裡的.
在很遠的那裡.
在不在這裡?.
不在這裡.
在那裡,在很遠的那裡.
但那東西是在那裡的.
不過現在還未到.
我們朝著望遠鏡的方向走的時候.
最後我們會去到望遠鏡.
看到遠處那裡.
現在還未到.
其實我們的完全都是.
我們現在完全還未到?.
你們現在完全還未到?.
還未到.
你們已經完全到.
因為耶穌基督已經洗淨我們的罪.
但你們完全還未到?.
你們還未完全到.
因為你們還會犯罪.
但我們聖靈會提醒我們.
少犯罪.
我們會更新.
我們會提醒自己.
我們會慢慢越來越少.
但我們朝著一個更好的方向走.
這就是你們要完全像你們天父一樣.
那種進程過程.
Tenelos裡面的字是說.
我們的性情.
我們的熱情.
我們的做法.
會慢慢朝著那個方向走.
但我們現在不行.
所以我的回應就是.
我們不要看我們很多不行的地方.
我們就會frustrated.
那個情緒會有的.
但我們知道.
聖靈會提醒我們.

$^{1921}$聖靈會告訴我們.
你又犯錯了.
下次要再聰明一點.
下次不要欺騙聖靈的提醒.
心有責心的時候.
就停一停.
想一想.
不要太快去決定.
在過程中一直朝著那個方向走.
不要太快就斷纜.
不要太快就覺得沒有用.
我覺得這個進展過程當中.
就是我們成性.
或者我們在生命當中.
仍然感受到有再一次的機會.
我覺得passion是其中一樣東西.
讓我們感受到有再一次.
我感受到有新開始.
這個也是.
剛才聽回那個問題.
我感受那個問題.
弟兄或姐妹.
那個感覺就是沒有了.
我希望鼓勵他們.
其實有的.
現在可能你覺得.
不知道怎樣開始.
反而朝著那個方向走.
或者朝著想一開始.
你想追的過程.
其實用甚麼方法去追.
你再嘗試重新觸碰那個過程.
(記者:還有沒有回應?).
前面有一個.
我想嘗試一下.
是不是這樣說.
對熱情來說.
可能我們會定一些.
外在很高不可攀的目標.
其實是不是.

$^{1961}$回到單純的選擇.
而那個選擇就是我們本身.
本相其實是.
去投入到的.
就好像潘Sir說的.
去投入到的一種的場景.
選擇那個場景來投入.
來對生命的熱情.
在那裡去發揮.
還有我想其實那個管理方法.
其實我覺得會不會用普通一句.
譬如說.
當然很多時候我們都說.
隨遇而安.
有些事情改變了.
我們就會順著那個勢會改變.
但屬靈上.
其實就是說.
等候神帶領去下一個選擇.
或者遇到的下一個場景.
有時候不是我們.
我們意志上可以去.
定了目標就做得到.
或者我是這個場景.
我就一直在這裡發展.
其實也未必的.
是不是這樣就會.
放開一點呢.
我的意思是不會.
又會帶來自己很多失望.
同時也可以去.
找回原本那個.
自己投入的那個場景.
我先說.
是的.
意思就是.
有些事情是要.
保持著那個勢力去做.
你才會看到下一步有甚麼改變.
就好像我.

$^{2001}$上一個月說到.
《企硬講ye 》那篇.
裡面就是.
保羅其實他的馬其頓回應呼聲.
那個宣教旅程.
從來都不順利.
覺得上帝要開路.
上帝不是開這樣的路給他走.
但你看到他在過程當中.
他沒有記下他發願.
他看到原來那件事過了之後.
他看到上帝有些事情要他去做.
但他最重點就是.
他清楚他的宣教旅程為了甚麼.
就是讓外邦人得聞福音.
讓希臘文化有一個新的.
接受福音的向導.
這是很清楚他做了甚麼.
至於他自己的經歷.
被人打,坐牢,被人趕.
被人追殺.
這些過程對他來說.
他覺得不是最賞心或最需要處理.
反而是他在做他覺得自己應該做的事情.
我覺得對於我們做信徒來說.
問得最多的是.
我們現在做的事.
其實是不是上帝最喜歡.
或是最想我們要做的事.
做牧者最難回應的就是.
要和弟兄姊妹一起去經歷.
分享情況.
一起去談.
所以怎樣談呢?.
你會看到我們常常鼓勵.
你有一個屬靈群體.
你自己一個的時候.
你和牧者談.
可能只有一個一起去想.
但你看到其他弟兄姊妹有相訪的情況.

$^{2041}$你就知道上帝在這個群體當中正在做事.
用保羅回應.
在很多教會中提及一個很重要的訊息.
你的經歷會成為別人的祝福.
這個就看到整個群體.
會有上帝希望對這個群體當中的影響力.
所以我回應到這一點.
你的經歷會和其他人有出入.
但其他人的經歷會成為你的借鏡.
但大家一直在做一些以善的方向去做.
其實你會看到上帝在這個群體當中的教導是甚麼.
最後我講一下為甚麼我會用滑板作背景.
因為其實上年12月一更次之後.
我重拾生命的熱忱.
我就去了學滑板.
所以我覺得滑板是我打PASSION出現的圖畫.
所以我希望大家都可以重拾.
懷著生命的熱忱去開展你的生命.
無論是滑板還是任何事情.
都可以令你好好地去活.
我可以問問題嗎?.
那幅照片的主角是誰?.
不是我吧?.
我都想問.
網民問那幅照片是不是你.
希望大家喜歡今晚的課堂.
我們今晚到此為止.
下個月見.
再見.
(音樂播放).
\newpage



\section{}
\label{sec:hq6PGyJ3aBs}
\textbf{【這是最好的時代:給香港基督徒的神學八課】第7課:今日的「我」推翻昨日的「我」|20211121 [hq6PGyJ3aBs]}
\newline
\newline
連結: \href{https://youtube.com/watch?v=hq6PGyJ3aBs}{\texttt{ https://youtube.com/watch?v=hq6PGyJ3aBs}} ~~~~ 語音日期: 2021-11-21 
\newline
\newline
\hyperref[sec:BozY0a8wlNg]{\small{< < < PREV SERMON < < <}}
~
\hyperref[sec:index_chronic]{\small{[返順時目]}}
~
\hyperref[sec:index_scriptual]{\small{[返順卷目]}}
~
\hyperref[sec:gSBEvA3qrgQ]{\small{> > > NEXT SERMON > > >}}
\newline
\newline
$^{1}$再共你相.
請不吝點贊訂閱轉發打賞支持明鏡與點點欄目.
歡迎收看訂閱轉發打賞訂閱轉發打賞.
明鏡與點點欄目.
歡迎大家來到香港基督徒的神學八課第七課.
第七課快要完結了.
快完成了我非常完美的八課.
我們回看過去的六課.
其實我們都很複雜的心情.
從頭開始說基督徒,福音,教會,靈性,一根刺,熱情.
這幾課特別是中後段的時段.
其實很不容易去劃分.
我們八課中,到中後段的時間.
其實一直在想一個我們基督徒群體.
希望能夠一起塑造的模樣.
但這確實不能夠這麼容易地定下來.
所以想了很久,又有熱情.
希望這群人能夠看到自己的一根刺.
明白自己的罪.
這課跟之前的兩課都很相似.
我自己認為都是一些很個人的靈性.
或者我們如何做基督徒的議題.
大家都知道,我寫的書全部都放在基督徒生活裡.
基本上都是我對基督徒的生活.
如何做一個好的基督徒.
我自己是很有興趣去思考.
簡單來說,如果用神學的擺位.
金堂其實是一個叫「誠聖觀」的課題.
誠聖觀這個課題,好像很深,很悶.
誠聖觀是一個很簡單的課題.
如何做一個好的基督徒.
甚麼才算是一個好的基督徒.
整件事情是一個甚麼過程.
或者反過來,我們Fold Church的課題.
對我們Fold Church來說.
我們如何去定義一個好的基督徒.
如何去達成這個目標.
當然前面的全部都很重要.
如何在亂世操練一個這樣的時代的靈性.
我們懷著Passion,生命力去做人.

$^{41}$懷著一根刺,但仍然不斷地面對自己的罪等等.
金堂的名字是「今日的我推翻昨的我」.
很明顯,我想說的是一個有關更新的課題.
我們有兩個主題去討論.
一開始會先去認識一下「誠聖」這個課題.
「誠聖」這個課題,如果從神學的角度來說.
我們都一定程度要先認識它.
然後我會說,其實所謂的「誠聖」.
是一個甚麼的「誠聖」.
在21世紀,香港這樣的社會裡.
所謂的聖人模式是怎樣的.
我們如何做一個聖徒.
所以今天嘗試去說一個不容易去定義的課題.
我們先說「誠聖」.
「誠聖」其實是「誠聖」.
這個詞,如果我們用傳統的神學教義來說.
我們一個叫做「Order of Salvation」的裡面.
一個基督徒從他整個人的得救開始.
他有甚麼步驟會經過.
一開始我們會稱義.
我們被稱為義.
然後去成聖.
然後會繼續不斷地被差遣等等.
在一個基督徒成長或信主之後的過程.
往往很多人會將「誠聖」看為稱義之後的部分.
「誠聖」是一個如何能夠越來越像上帝.
越來越接近上帝.
甚至東方教會稱之為Deification.
即是更加神化.
越來越和上帝相似.
對我們新教徒來說.
這個課題是一點都不容易的.
反過來說,天主教是更加簡單.
我只是說簡單的意思,不是說它容易做.
起碼你懂得怎樣做.
如果今天你是天主教徒的話.
你的「誠聖」要做的事情很簡單.
我給你一個清單就能做到.
基本上你會接受水禮.
不是普通的水禮.

$^{81}$而是一個能夠洗去原罪的有功效的水禮.
然後你偶爾犯罪之後.
你可以去神父教堂告解.
告解完之後,你便洗完.
然後每個月你會領聖體.
聖體是一個幫助你靈明.
或是你的「誠聖」不斷更新的方法.
基本上就像吃藥一樣.
每個月一次,你就見效了.
或者有點像脫髮那些.
你一直做,就會收回頭髮.
所以有些事情是可以做到的.
而你一直做是能夠有效的.
甚至我們會說的事情是「德行」.
基本上是一個好基督徒.
做多一點好事,做少一點壞事.
減少犯罪,做多一點好事.
這就是基督徒或天主教徒可以做到的事情.
整件事雖然可以說是不容易.
但你都可以跟進.
你去大列的清單,你都會做.
你用心去做,你就做到.
因為天主教有一個觀念叫做「Habitus」.
一個習性的概念.
你做得多,你就自然會有好處.
你就不會有壞處.
所以你這樣做,你會覺得穩妥.
你覺得做基督徒,你誠信一步一步走.
你作為一個天主教徒,你就死了.
然後就有少許憐慮,上天堂了.
其實我越來越羨慕天主教的方法.
因為作為一個傳道人,我覺得容易教你們.
有什麼事就過來告解我.
有什麼事就來做,繼續做.
我就保證在我的教會裡面.
大家會一步一步走上誠信之路.
但基督教很不同.
五百年前,馬丁路德提出一個很矛盾.
又很真實的觀念.
這也是他自己人生裡面的掙扎.

$^{121}$同是事,二人,又是罪人.
這是一個很矛盾的觀念.
聖母Eustace在帕特卡拒絕了聖人模式.
馬丁路德是一個非常追求.
用心做好基督徒的人.
他覺得自己做不來.
然後他發覺無論如何.
他仍然是一個和上帝很遠的一個人.
不配領受恩典的一個人.
所以我們基督教,我們新教這樣開始.
我們作為基督徒.
我們面對誠信這件事就很矛盾.
究竟我們能否去誠信.
有沒有一些事是可以保證做得到.
來成為一個好的基督徒.
不知道是不是你的掙扎.
你可能做了很多年基督徒.
仍然覺得自己很浮沉.
或者很多事都做不到,做不好.
好像沒有一個方法可以保證做得到.
這似乎是我們面對的矛盾.
甚至我們會掛念很多的罪名.
或者過往做的事一直掛在心上.
假設一個姐妹離婚的時候.
離婚的一個人在教會裡永遠都是離婚.
起碼是離婚的一個地位.
掛念你一生.
再結婚也是再婚.
所以這些事情.
在基督教裡雖然叫做耶穌洗血的罪.
但這些內容似乎更難洗得走.
什麼罪是最難洗得走.
就是早已被赦免的罪.
天主教很簡單.
我真的去告解.
我就真的可以得到完全的赦罪.
我就懷著重新開始過.
但基督教很容易來說我們是一個義人.
但我們同時也是一個罪人.
所以這種矛盾.

$^{161}$希望大家能夠明白.
體會到這種複雜性.
我們基督徒去說誠信.
去說自己是一個好基督徒.
其實我覺得是一件很複雜的事情.
不知道大家有沒有唱過這首歌.
或者你也聽過這些說法.
誠信需要功夫.
其實我沒有怎麼唱過這首歌.
是一首很舊的詩歌.
但這句說話是很刻骨銘心的.
從小我就聽到這句說話.
誠信需要功夫.
我們華人教會會不斷說.
誠信需要功夫.
所以大家要用心追求誠信.
基本上不外乎都是那些.
都是靈修班教會的事情.
所以不知道大家是否覺得.
我們是否這樣就能成為一個好基督徒.
不斷很毀身地追求教會.
給我們設定的誠信功夫清單.
不過你會發現其實不是的.
誠信究竟需要功夫嗎?.
如果你看原文的字.
這個叫hagiazu.
hagiazu是一個動詞.
如果你找一個動詞.
大概就是這個字.
hagiazu就是hagia.
就是聖人.
使某人成為聖人.
是這個意思.
使某人成為聖.
所以這個字是一個吸脈動詞.
記得嗎?.
我們看過VT和VI.
甚麼是VT?就是吸脈動詞.
你需要一個object在後面.
我們有甚麼例子?.

$^{201}$在英文裡面.
I kill you.
I kill you是你是object.
你被我殺了.
所以是一個有subject有object的字眼.
誠信也是這樣.
其實不是我誠信.
我不是subject.
其實我是甚麼?.
我是object.
我是被誠信.
某某某誠信了我.
所以我們看約翰福音第十七章裡面.
耶穌說.
求你用真理使他們誠信.
你的道就是真理.
你怎樣猜我到世上.
我也猜他們到世上.
我為他們的緣故自己分別為聖.
叫他們也因真理誠信.
我們看見.
在哈爾亞訴這一節裡.
其實耶穌從來都說.
誠信的概念不是要你去誠信.
而是上天父上帝已經誠信了你.
所以你從來都不是誠信裡面的subject.
你不是主要的object.
我吃西瓜你是西瓜.
是subject.
所以耶穌說.
原來我們去探索誠信的時候.
其實從來都不是你的努力.
我這樣說.
這不是你的努力.
上帝是要使你們成為聖.
上帝已經在基督裡面.
使你們成為聖.
所以這是我們很重要的大前提.
誠信是在基督耶穌裡面客觀的基礎.
如果你以前是回宣道會.

$^{241}$我現在也是宣道會.
宣道會所說.
耶穌是誠信的主.
所以其實叫我們誠信的是上帝.
藉著耶穌基督叫我們誠信.
所以誠信需要用功夫這個字.
其實是我們要重新思考這個字的意思.
我不是說這個字是錯的.
而是我們要如何理解誠信需要用功夫.
容許我講一個聖誕樹的故事.
這是真的.
這是我在德國讀書的時候.
這棵聖誕樹旁邊的是我老婆.
聖誕節她穿著綠色的衣服.
這棵樹在德國的好處是什麼呢?.
你可以買一棵真的聖誕樹.
不單是真的聖誕樹.
更是一棵活生生的聖誕樹.
它還未死的.
還要用泥土養著它.
它不是被人砍下來的.
不是塑膠.
而是一棵樹.
所以我們買回來的時候很重.
下面的盆栽和泥土.
整棵都搬上來.
放在家裡很漂亮.
而且很香.
原來一棵未死的聖誕樹很香.
有很多特別的香味.
就放在家裡.
過了聖誕節25,26日.
過完新年.
發覺那棵聖誕樹很重.
搬它不容易.
那時候因為懶惰.
或者正在讀書.
沒什麼時間去做家裡的工作.
因為不是隨便放在門口.
要開車去某個地方.

$^{281}$所以那時候.
放在家裡的露台.
放到復活節.
整棵樹就一直放到復活節.
差不多放了半年在家裡.
有一天我坐在沙發上.
懶洋洋地看著那棵聖誕樹.
覺得不錯.
不如長期放在家裡.
下年搬出來.
下年就不用買一棵新的.
而且會變大.
說不定你坐著放十年八載.
好像海港城一樣大.
也不奇怪.
覺得不錯.
原來聖誕樹.
大家知道聖誕樹平日是做什麼的嗎?.
聖誕節以外的時間.
他們在那裡做什麼?.
他們仍然是做聖誕樹.
雖然他們已經是聖誕樹.
但他們仍然要成為聖誕樹.
你明白嗎?.
我想說這件事.
他們要成長.
他們要繼續變成大棵.
沒錯,他們是聖誕樹.
他們已經成為聖誕樹.
但他們仍然要繼續成為聖誕樹.
這個道理就是我們今天面對的情況.
你們已經成為聖.
在基督耶穌裡面.
在上帝十字架裡面.
你已經被稱義.
被成聖.
不過.
你要去做你應該有.
和已經有的事情.
基督徒永遠都是這樣.

$^{321}$你們去奮鬥的那件事.
不是你沒有的.
而是你一向都有的.
你是為著你有的事情去奮鬥.
無論是上帝的國度.
或是成聖的身份.
我們已經在基督裡面成為聖.
所以我們就要去成長.
我們就要慢慢去改變.
越來越大棵.
越來越漂亮.
越來越茂盛.
這就是我們今天.
起碼去談及成聖這個話題.
很重要的前提.
我們所謂要做的事.
不是你不做就不是聖人.
或者做了才是聖人.
不過你仍然要做.
做的意義不代表.
你做完之後有什麼.
而是你本身就是.
然後你去追求.
你去成長.
所以這是我們今天.
作為一個基督徒.
去思考成聖裡面.
一個很重要的課題.
我們開始思考成聖的科學.
Science of Sanification.
其實這就是我們所謂的靈性.
Define Spirituality.
就是Science of Sanification.
我們思考我們的靈性,靈命.
就是一個有關成聖的科學.
怎樣去走這條道路.
我們的成長.
我們從初信到活了的時候.
我們做基督徒是一個什麼路程.
是一條什麼路.

$^{361}$是不斷向上還是怎樣.
這就是我們嘗試去思考的事情.
怎樣能夠達至成長,成聖的道路.
有很多不同的理論.
今天很簡單,很籠統.
用圖片來做解釋.
很籠統的,不是有很多東西說的.
但很簡單地說出來.
我們看看路德的成聖觀.
基本上路德是沒有成聖觀的.
他說同時是聚人,同時是異人.
所以如果成聖是一條古事的線的話.
它是長期在一個低點.
你沒有所謂向上的可能.
為什麼呢?.
因為你同時是異人,同時是聚人.
你怎樣做都不斷掙扎在低點.
不要成為聚人.
不斷和聚搏鬥到你死的一刻為止.
這就是所謂的成聖觀.
基本上是沒有成聖的.
但最後仍然是一個極大的聚人.
這樣去掙扎著.
所以這是一些困難.
因為當時天主教最美麗的地方.
就是一些靈修傳統.
而路德是沒有什麼靈修可言.
因為你只能夠同序掙扎.
沒有什麼正面或積極的可能.
去談及做得更好的基督徒.
所以這就是路德的成聖觀.
不斷掙扎在低點的看法.
另一個我上次提及的極端.
就是楊惠思理的成聖觀.
當然楊惠思理和路德相差一段200年的時間.
不過楊惠思理對於恩信清義.
是更加強烈的經驗.
所以他會提及一個Christian Perfection的概念.
他覺得只要我們去追求的時候.
我們是有可能.

$^{401}$雖然不是每個人都可能.
但我們有可能去達至Christian Perfection的可能性.
你會跑到最後很高.
可以跑到很接近耶穌.
甚至是完美的perfection.
所以只要你追求.
只要你毀身.
只要你和上帝耶穌基督願意親近的時候.
你的線雖然可以是反覆的.
但是一個牛市.
是一個反覆向上的圖.
是會穩步上揚的.
雖然會有高低起跌.
但仍然在上升軌的裡面.
這就是他的比較樂觀的看法.
他認為基督徒是可以越來越好.
這是另一極的看法.
引申楊惠思理之後的相似經驗.
我特意帶來講.
就是十九世紀聖潔運動的經驗.
Holiness Movement的經驗.
他們是叫甚麼.
他們都是楊惠思理的發展.
他們認為有一個稱之為.
爆洩的情況.
突然間有一個叫second blessing.
一個第二次祝福.
因為聖靈的緣故.
你初初缺志的時候是一個起點.
但基督徒裡面會突然間經歷爆洩.
可能你聽過一些見證.
以前信主都不是很追求的.
突然間有一次車禍的時候.
他就重新再反省.
甚至動物神學中有很多人.
很多這些經驗.
稱之為第二次的祝福.
這是基督徒的爆洩.
我們先不要理會它的意思.
但起碼他認為仍然是一個向上的看法.

$^{441}$但需要聖靈一次的助力.
突然間聖靈的降臨和高末.
成為第二次爆洩的階段.
後來就形成了靈魂運動.
靈魂運動正正就是一個這樣的看法.
聖靈的秋雨降臨之類的.
有謊言.
所以我覺得那條路線比較不同.
但仍然是一個向上的看法.
第四個就是嘉義民的看法.
嘉義民的看法是非常不容易的.
簡單來說就像食神一樣.
寫一個「店」字.
就像基督一樣.
其實基督是畫不到的.
是畫不到那條線出來的.
總之就是你何時能夠去參與在基督的裡面.
你何時就是一個最好的狀態.
你何時不在基督耶穌的裡面.
你就沒有了.
所以稱之為 Participatio Christi.
Participation of Christ.
即是在基督裡面.
參與基督裡面.
只要你常常在基督裡面.
你就能夠是一個最好的狀態.
但當你離開了.
你就什麼都不是.
不過這樣說是對的.
但這樣說是什麼意思呢.
怎樣叫做在基督裡面.
怎樣叫做不在基督裡面.
怎樣叫做參與基督呢.
這個就似乎是比較抽象的.
所以你畫不到那條線出來.
你不知道你人生裡面何時叫做參與.
何時叫做不參與.
那條線是向上還是向下.
你不知道的.
不過他就說得很對的一件事.

$^{481}$就是在基督裡面.
所以這個就是.
總的來說.
我們是有四個比較去classify.
四種不同的成性觀.
去得出這四種的結論.
問題就是我們怎樣看呢.
或者我怎樣看呢.
我們Folk Church怎樣去理解.
這個成性的道理呢.
當然四個都有道理的.
我認為.
四個其實都是描繪著某些基督徒的狀態.
不過我想說的不是這些東西.
剛才那些成性觀的背景我們說完了.
我擔心的是另一些東西.
其實你也看到我們今天副題所說.
既然是成性.
倒不如更新自己.
為什麼這樣說呢.
因為其實你發現.
在教會裡面出現了這樣的狀況.
我稱之為丁蟹基督徒.
我寫過一篇文章.
這個題材我寫了一篇文章.
什麼叫丁蟹基督徒.
可能你看過大時代.
那你就知道誰叫丁蟹.
如果你沒看過就追溯一下.
丁蟹是一個什麼樣的角色.
丁蟹是一個很棒的角色.
丁蟹其實是一個好人.
他是正義的.
他覺得自己是正義的.
他是做好事的.
他其實是想做好事的.
他一直都在做一件他認為是對的事和好的事.
不過做錯了的意思.
他一直都以為.
藍潔英是喜歡他.

$^{521}$所以他才會這樣.
藍潔英死後.
你不喜歡我你早說.
打倒人家二十年都說了很多次.
他一直都不明白.
不知道.
但對丁蟹自己來說.
他一直都堅持著自己的良善來做人.
其實他真的不是詭詐.
他是單純地不知道自己做錯事.
所以.
所謂丁蟹基督徒是什麼意思.
他們是一群仍然追求誠信的基督徒.
仍然在自己的誠信裡.
去盡心甚至深化.
去追求耶穌和誠信的人.
不過客觀的原來是.
他有很多的限制和缺陷.
他們不知道.
所以我想說.
這個可能是更值得去提的一個課題.
我們在做好基督徒的時候.
我們更加會面對到很多的盲點.
有幾個案例.
第一個丁蟹基督徒的案例.
簡單來說就是缺乏知識.
如果你是很敬虔.
但缺乏知識的話.
你去追求其實都很糟糕.
如果你是很追求信仰.
但變成迷信的話.
當你缺乏知識的時候.
其實你的追求誠信都很有限制.
甚至你會變成很偏見的人.
你的神學沒有很好的聖經裝備.
沒有嘗試去擴闊自己的眼界.
去讀多點神學.
或者去看多點不同的信仰框架的時候.
你只是在你掛的那套進心.
當你缺乏這些知識的時候.

$^{561}$你可能會變成一個很虔誠.
但會看錯很多東西.
我在網上見過很多這種情況.
他們可能是很好的.
但他們未必有這樣的知識去嘗試.
讓他們好是更加好的.
所以我覺得呼出丁子妹.
很需要有這樣的知識.
敬虔愛主之愛.
追求之愛.
其實我們要有一定程度的知識.
去深入我們的靈明.
我自己讀神學的時候.
正正是這樣體會.
知識正正是幫助我們的靈性.
或者我們怎樣做基督徒.
因為當你知道.
你做得好之前.
你要知道要做些什麼.
你怎樣面對這個世界.
怎樣做好基督徒.
你知道之後.
或者你一定程度的深度之後.
你才能好好做好.
當然得知識沒有行為是錯的.
但單單有敬虔和追求和誠信.
而沒有知識.
仍然是一個很丁蟹的基督徒.
第二個情況就是缺乏包容.
有些基督徒其實是真的很好.
但當他缺乏一種愛心.
或者對於做不到的人.
是很嚴苛的時候.
這個就變成了另一種丁蟹基督徒.
他仍然是追求誠信.
仍然在這條路上.
是很熱切地對自己有很高的要求.
但他對其他做不到的人.
一旦嚴苛.
其實都會很麻煩.

$^{601}$變成了一些很不理想的人.
當然理性上是不會的.
當你有足夠的愛的時候.
其實你又會明白體諒其他人.
所以我覺得這個要自己引以為鑑.
當我們有屬靈的驕傲.
當我們做得好的時候.
我們缺乏包容的時候.
我們仍然是一個完全的.
第三就是過分主觀.
雖然這一點好像很濕性.
但我覺得是很真實的.
很多人是很主觀的.
他看事情太快.
去論定某些事情.
雖然他仍然是很追求.
和熱切地去執行信仰.
但有時他看事情太過.
覺得那些事情是看別人錯的.
當他的想法和價值不同的時候.
這就是錯.
我講得很抽象.
但具體我不想講.
因為有些人我認識.
很多人確實是很屬靈.
但其實是很主觀.
對於像我這樣的人.
就變得覺得不能客觀地看事情.
所以在這件事上.
遇到很多很好的基督徒.
但其實他過分主觀的話.
就不能理解很多其他事情.
所以在我缺乏慧子的情況下.
可能大家可以討論一下.
有什麼情況是有共鳴的.
丁蟹基督徒在你遇過的情況.
大家可以一會兒討論一下.
所以要避免成為一個很好的丁蟹基督徒.
要怎麼做呢?.
其實答案是很老套的.

$^{641}$就是更新.
所以我還是不容易講的.
後半段的話題.
因為其實答案是很簡單的.
我們就是要去更新自己.
我們要不斷地讓自己重新檢視自己.
我覺得這個更加重要.
相比起成性.
其實更新是更加讓我們能夠向前行一個必備的條件.
我們成性好像是一個很大課題.
但你對於自己的更新.
是一個似乎更具體的事情.
我們說教會的更新.
我們之前說教會的時候.
我們說教會不斷地改變.
不斷地改革.
當我們認同教會不斷地去改革的時候.
更加同樣道理.
教會裡面的人都不斷地需要改革.
所以大家想想.
大家有沒有去改變.
在這幾年裡.
最基本我們有沒有去更新我們的看法.
昨天的我和今天的我.
是不是不同了呢?.
今天我們會將這句話看成是一個負面的.
今天的我打倒了昨天的我.
你就會恥笑他.
但你想想.
如果你這幾年沒有改變.
其實這件事情反而更加詭異.
今天的你和十年前的你是一樣的.
一樣的想法.
一樣的體會.
如果一樣的話.
除了恭喜你之外.
其實會更加為你擔心.
所以我們問我們有沒有更新.
什麼更新了呢?.
特別是在《羅馬書》裡面.

$^{681}$當我們看到保羅說我們一個很重要的要求的時候.
我講過.
不要效法世界.
不要再困在世界的意識形態裡面.
只能夠透過聖靈所帶來的轉變.
這個更新心意變化是來自於聖靈的.
你不斷地來改變你自己.
不斷地改變你自己.
這個就是我們很重要的更新的意思.
當然另一個說話就是來自於.
這個歌頌前出第五七十節.
若有人作基督律.
他就是新造的人.
舊事已過都變成新的了.
這是我們之前在福音書的洗禮的時候.
我送給弟兄姊妹的經文.
但你想想你洗禮多久了?.
你洗禮可能已經有十幾二十年.
甚至乎三十年.
問題是你是否一個舊的新造的人.
你想想你回教會這麼久.
你對於新造的人有什麼感覺?.
理論上我叫新造的人.
但我已經回教會三十年了.
我是一個舊的新造的人.
所以我們所說的不是舊的新造的人.
我們仍然要歷久常新.
新造的人上帝是沒有舊的東西的.
所以我們所謂的新造人.
不是說你三十年前洗禮那麼簡單.
而是你不斷地每天都被更新.
在上帝裡面成為一個新的人.
新的想法.
新的關係.
所以你會發覺很多經文在聖經裡面.
每天早晨都是新的.
上帝是一個新的上帝.
新的字是一個很重要的神學字.
我們拉丁文叫做Novum.
上帝的屬性本來就是新的.

$^{721}$你會發覺有很多新的字.
新天新地新耶路撒冷.
新生命新的城新造的人.
所以上帝是沒有舊的東西的.
意思不是說他不斷地換東西.
而是他的本性就是新的時候.
我們只能不斷地有新的看法在裡面.
所以想想.
當我們如何能夠作為一個人.
能夠不斷地新.
你已經上了沙年.
你會發覺自己是舊的.
很多東西.
你唱一首詩.
我們聖餐那首《銅餅銅杯》.
都叫新歌.
都已經唱了兩年多.
但當你發覺.
當你唱了十年八載的時候.
這首已經變成舊歌了.
我們如何能夠在一些所謂舊的東西裡面.
仍然有新的東西呢.
仍然都是每個星期六.
星期六八點鐘崇拜.
我們開始發覺.
很多流出的新的東西都是舊的.
這個地方用著用著都是舊的.
當你聽了十年後.
發覺都是那些東西.
其實很難的.
一個人說了十年.
其實都明白他的笑話是從哪裡來的.
他的家底都聽過了.
所以重點是.
我們如何能夠在所謂.
舊的東西裡面.
能夠去更新.
這個更新是一個內在的更新.
如何能夠在自己的靈明.
在我們的生命裡面.

$^{761}$仍然保持一個新狀態.
我們第四課的時候說過.
叫我們找找你的靈性生活.
找一個新的靈修方法.
不知道你有沒有找過.
但你發覺總是要找找.
這個做得差不多了.
但你又要重新找新的.
這不奇怪.
所以發覺原來沒有一個完美的東西.
任何的活動任何的事情.
當你久了.
發覺總是會從新意分離.
慢慢變成一些你發覺很舊.
沒有常態的東西.
但這個時候你需要慢慢更新它.
我舉個例子.
我們今天常常用手機去說更新.
基本上更新是必須的.
你發覺沒有一個App.
沒有一個公司能夠寫出一個完美的軟件.
Microsoft Apple Google都是這樣說.
沒有一個完美的軟件.
真正的完美是什麼.
真正的完美就是不斷地去找出問題.
然後去整走它.
更新它.
不斷地這樣做.
所以與其說找一個誠信的基督徒.
不如說找一個願意更新的基督徒.
你在地上找不到一個聖人.
你只能找到一個願意不斷地去改變的基督徒.
我覺得這個很重要.
如果我們Full Church.
大家都能夠不斷地去更新自己的心態.
想法和靈性的時候.
我們沒有人完美.
我們都是一根刺的人.
但我們就是一個不斷地去更新的群體.
這個就是所謂上帝的身.

$^{801}$在聖靈裡不斷地去轉變我們自己.
這個就是我們所謂香港裡的聖徒.
我上次講道也講過.
這個年代是不需要英雄.
我們有很多的反英雄.
我想屬靈也是一樣.
我們這個年代不需要任何一個屬靈偉人.
一個完美的屬靈偉人.
全部東西都是完美的.
行事為人又端正.
有沒有不良嗜好.
其實都有的.
但總是有缺陷的.
我們不需要去建立一個完美的屬靈偉人.
任何屬靈的人其實都是不屬靈的.
不過他是願意去更新改變自己.
這個反而是更值得我們追求的地方.
所以你明白為什麼我們這個課題.
叫做與其成性都不如不斷地更新自己.
所謂捨己.
捨己可以說是一個很屬靈的情操.
不斷地去毀身.
不斷地去犧牲自己.
這個其實未必是英雄.
所說的不是sacrifice.
其實原文裡寫的是reject yourself.
這個人不是去犧牲自己.
而是不斷地去拒絕自己.
當然這個意思可以很多.
所謂打低自己的老我.
打低自己的舊我.
不過我覺得在我們這個context裡的意思就是說.
我們不斷地嘗試去拒絕自己.
去更新自己.
以前我這樣想可能是錯的.
以前我這樣去看可能需要被改變.
這個拒絕似乎是一種更新的意味.
所以我們Full Church裡的捨己.
可能就是這種態度.
不單單是為教會和上帝犧牲這麼簡單.

$^{841}$而是更加地不斷地去更新自己.
丁蟹基督徒也會捨己.
也會犧牲.
可能做了一些錯事.
但我們不斷地去反省自己.
說了這麼久.
我有幾點具體的說.
待會潘Sir來到我們會再多談一些具體的情況.
先說手.
為何你會選擇手來做這次課題的背景呢?.
我搜尋圖的時候.
搜尋這個Revolution.
當我搜尋Revolution的時候.
就給了我這幅圖.
這幅圖正正就是一個很好的象徵.
象徵著我們去革自己的命.
所謂更新正正是一種可能會血腥的.
血腥的是在你自己裡面.
有些東西你是這樣看.
有些東西你是這樣做人.
甚至乎這是你的性格.
不斷地去更新改變.
它是會痛苦的.
很多東西不斷去拒絕自己.
不斷地重新審視自己的想法和性格.
不是厭世.
不是不喜歡自己.
而是重新將自己不好的地方不斷更新.
怎樣做呢?.
我們有三點可以討論.
第一個就是嘗試開放.
我想第一點也是這樣.
嘗試去開放自己出來去被改變.
起碼assume自己可能是錯的.
就算你覺得自己對也好.
這是很矛盾的.
沒有人會做錯的事.
沒有人會做對的事.
但不斷去assume自己可能是錯的.
然後去尋求可能性.

$^{881}$學習更多的東西.
開放自己去學習.
一個人學習是真的.
你發覺自己成熟後.
學習是會慢了和差了.
我和洪禮方寫的書.
開始覺得自己越來越難改變.
因為我開始教人做人.
開始寫回應人做人.
當你成為老師的時候.
就越來越難成為學生.
但仍然讓我們去開放.
去準備自己去被改變.
這是一個很重要的態度.
第二就是反省.
都是很老套的.
發現自己的盲點.
這個不容易的.
可以回到小組裡.
或者今晚結婚和老公老婆聊天.
雖然你可能說了很多次.
我有什麼缺點.
問問你的好朋友.
你有什麼缺點可以改變.
這是很重要的.
不斷去看到自己看不到的東西.
所謂的盲點.
說是容易.
但我們仍然要去發現自己很多的盲點.
我會繼續聊.
很多這些課題.
有很多具體的情況可以跟大家聊.
發現自己更多的盲點是很重要的事情.
你的朋友.
你的組員.
你的另一半.
正正是幫你去發現盲點的一個很重要的位置.
第三就是對話.
我想特別是在這個年代裡.
我們真的要和一些和我們不同看法的人多些對話.

$^{921}$有句很好的說話.
就是說一個小小的真理的相反.
就是錯.
一個真理的相反就是錯.
但一個極大真理的相反是什麼.
是另一個極大的真理.
我覺得這是一個很受用的說話.
很多時候我們去看到盲點.
或者我們不能夠明白看法的時候.
往往就是我們缺乏和另一種體系的人去對話.
當我們嘗試去更新我們自己的時候.
嘗試去多些聽不同人不同看法的對話.
特別是我們可以說說這個話題.
有關我們的政治話題.
有關我們在教會裡.
不同立場和做事方法的問題.
如果上帝把一個和你完全相反性格的人放在你工作裡.
好幾年的時候.
這是一個很值得我們珍惜的機會.
你會爆炸.
但同時你會好好更新自己的方法.
一個physis, antithesis.
變成sympathies.
就會昇華.
你就可以學到更多東西.
這幾年我是這樣.
上帝把一個和我完全相反性格的人放在我身邊.
在我的生命裡.
當你克服了之後.
確實你能夠看得闊了很多.
能夠成長了很多.
記住我們是一棵聖誕樹.
聖誕樹是需要成長的.
雖然我們已經成為聖.
但我們往往需要不斷地去更新.
去走這條神聖的道路.
我們有一個討告時間.
無論你是在YouTube或在場.
我們一起討告.
求聖靈去臨到我們.

$^{961}$讓我們能夠提醒我們.
知道我們怎樣得到更新.
我們一起討告.
主我們求你去提醒我們.
讓我們能夠看見.
看見我們一些我們看不到的東西.
看見我們生命裡的盲點.
一些我們的缺點.
主我們每個人都有我們的缺點.
可能我們的性格.
可能我們的體見.
我們的立場.
求主你幫我們面對很多.
敵黨或相反的人的時候.
讓我們能夠有耐性.
開放自己去對話.
從而去更新.
主我們不願保留我們今天的狀態.
我們所想,所做的.
我們願意的不斷地去被改變.
我們願意今天的我.
打倒昨日的我.
願意成為一個更好的我.
求主你幫助我們.
願你去切割提醒我們保護弟妹.
生命裡可以被改變的地方.
具體的能夠去實踐這個成性的路.
馮春永求,阿門.
原來最後兩堂課了.
是啊.
開始有點掛念你了.
下期到你了.
OK.
今天你覺得這個課題怎麼樣?.
對我來說是很適合的.
因為我們做食肆.
我們經常去行間吃東西.
更新菜單.
還有自己想一些花樣出來.
吸引一些新客.

$^{1001}$所以更新對我來說是很重要的.
還有問問人.
東西行不行.
看看自己有沒有缺點.
所以後面那幾點對我來說.
開放一點.
自己除了一些手下小菜之外.
還要讓自己去嘗試多一些不同的元素.
食材.
是很棒的.
其實有些難實踐.
或者說個容易的.
大家都知道要更新的.
不過有沒有一些具體的經驗呢?.
我通常都是.
拿個話題出來.
跟別人去碰碰運氣.
我自己通常都喜歡一個方式.
不要太快就定了格.
還有問多一些可能性.
剛才我說主觀的人.
主觀的丁蟹.
正正是這個問題.
當你是很主觀的話.
你就很快在對話裡面.
就已經覺得我是對的.
他是錯的.
或者他就是這樣.
太快去主觀地看.
你就會變成沒有空間去開放.
還有你的對話都變成了.
已經定了角色.
他已經是反派.
我就是一個好人.
你是好人.
但其實別人可能是另一種看法的人.
這是我幾年來很多這樣的經驗.
我自己就經常都會有口頭禪.
當有一個問題的時候.
就不要太快覺得是一個問題.

$^{1041}$人們經常都想解決問題.
或者解決問題.
反而我覺得問題.
不如就質疑一下.
為什麼會有那樣東西出現.
或者質疑一下.
那樣東西對我來說.
其實觸動了我什麼呢?.
我仍然覺得留一條尾巴.
去繼續討論空間.
是比很快就定了格.
還有覺得應該是這樣的.
所以教會裡面吵架都是這樣.
教會吵架其實都是.
另一種大家都很追求的人.
你去吵架.
通常那個人不愛主不追求.
他就走了.
留下來和他吵架的人.
甚至乎吵到火紅火綠的人.
正正都是很追求的人.
都是願意成為聖的人.
很多時候問題就在這裡.
有些像你剛才說的丁夏.
其中一個重點就是包容性.
我覺得很多弟兄姊妹.
對誠性都有自己的觀念.
他未必很清楚.
我是屬於嘉義文.
還是衛斯理.
但他自己覺得在成長過程當中.
他已經對誠性有一個要求.
和一個他自己的看法.
就用那個看法主導了.
對其他人的標準.
這件事都是一個很普遍的現象.
是的.
所以不知道Future弟兄姊妹.
怎麼看呢?.
大家有什麼丁夏的feedback?.

$^{1081}$或者對誠性這個課題.
過去這幾年對自己的提醒.
有沒有什麼反省?.
(記者:你和教會說到火紅火綠).
(不代表我不想再更新).
(你是……).
聽得不太清楚,不好意思.
我說不再和教會的人.
罵到火紅火綠.
不等於我不想再更新.
只不過是我對我的組織.
我自己的根深柢固.
當然最健康的就是.
大家罵完之後.
繼續可以有一個關係在裡面.
但如果比如說社會撕裂.
你說教會裡面的人.
根本完全看不到年輕人的看法.
或者就算不認同也好.
至少和他們同行.
教會如果完全不想做這件事.
其實沒有意思再和教會說.
但這樣也不能夠說.
放棄再說.
就是一個人的靈明有沒有更新.
其實現在我覺得.
這個complexity可能大了很多.
為什麼?.
為什麼你覺得complexity大了很多?.
什麼意思?.
以前沒有這麼多問題發生的時候.
我覺得在我以前信主的途徑.
雖然可能有.
好像你剛才說.
要成性的那段時間很長.
大家有不同的看法.
怎樣才算成性.
但是沒有很多社會的動盪.
不會有很多這些問題.
去貼身地去明白這件事.

$^{1121}$當天下太平的時候.
我覺得大家的靈明.
最多是賺錢多一點.
賺錢少一點.
結婚離婚.
有工作沒工作.
進大學也進不了大學.
可能只是這些問題.
其實也算是一些舒服的掙扎.
但是當社會撕裂.
去到一些年輕人.
我不是年輕人.
但是有沒有希望在這個社會上.
又是一個再深一層的問題.
譬如以前教會不需要說.
以前的教會經常說.
我們很和平的.
信耶穌就很和平.
好像是新約裡的神才是真.
教會裡的神不是真.
這些話題完全不用說的時候.
走那條路很舒服.
變成我覺得可以面紅耳赤地爭辯.
也是一條健康的路.
但是當現在的撕裂這麼大的時候.
中間原來那個紅溝.
簡直是從這裡走到那裡.
那個空隙是完全沒有一個橋可以隔開的時候.
我看到我這邊的橋.
對方也不是在看到.
因為橋是大家一起面對.
去見的時候.
我是健康地去取得.
我覺得也是需要的.
同意.
所以我在想.
能不能夠在五年之後.
大家又再見面的時候.
大家又會走多了.
或者改變了.

$^{1161}$期望是一個過程.
我們說更新肯定是一個過程.
現在大家是一個很遠的距離.
如果大家都是一個更新中的基督徒.
五年之後原來大家都會改變了.
這麼多的改變可能是很多不同的改變.
但起碼會.
如果是嘗試去理論上.
其實可以拉近了.
我也是很渴想這個圖畫.
你說世代之爭也好.
或者是建制和年輕人之間也好.
我曾經想過會不會是年紀大的人難改變.
年輕人容易改變.
我不知道大家怎麼看.
不知道大家覺得.
所謂的OC food的人難改變.
還是年輕人容易改變.
不知道.
經常聽人說年輕人都很萌塞.
或者很多人都可以不斷被改變.
所以我想大家都是嘗試去.
好像一個途徑.
比較遙遠容易被上帝塑造.
來開放自己去做人.
這一刻是有差距的.
這一刻是很多不被接納的.
但自己是在遙遠的時候.
也是那一句.
上帝帶領我們更加明白.
明白別人為什麼會拒絕你.
我自己在這些日子感受到.
因為我不是接觸過很多外語教會.
大部分時間都是華語教會.
我就很感受那種愛的表達方式的差異.
特別很多時間都在香港的教會出入的時候.
聽過大家表達愛的方式.
那種從上而下.
我覺得這樣是適合他的.
於是他用那種方式主導了.

$^{1201}$我食鹽比你食米多.
用這種方式去教導他.
或者這樣做的時候.
打者愛也,愛者打多幾下.
那些管教的,但很主導性的時候.
我覺得不容易有溝通.
或者令事情現在來說複雜了.
就像今天的課題來說.
誠信其實是比較多向度.
不同年代大家追求的方式差異很大.
就像剛才說到第四堂關於靈修.
關於靈敏塑造的時候的方式.
以前只用文字.
用文本,用字,看讀經.
但現在可以聽,看影片.
有很多方式.
其實追求過程可以多向度一點.
因為方式也不同了.
所以回到現在爭論到面紅耳熱.
或者因為這個年代.
大家對道德標準.
或者對何謂善.
追求的步伐其實差異也很大.
就像剛才John說到年紀是一個問題.
我也見過一些年青人都很萌失.
我見過一些年長很開通.
一個地方就是歸根究底.
大家經驗過來就會感受到過程.
他經過年歲的時候就知道.
就像John說的會柔軟了.
這個也是生命歷練.
慢慢就會經歷到我現在這個階段.
不會很快就定格.
或者很快就覺得應該是這樣.
其實我覺得有點迷茫.
迷茫的地方是其實成性這件事.
剛才一開始提到.
是個人或是復興教會.
就是一個群體在成性或成長中究竟是怎樣.
我記得N年前的成性很簡單.

$^{1241}$就是分別為性.
我們是跟著世俗隨波逐流.
我們以聖經的教導為標準.
而我們是活出一個.
我們認為聖經應該叫我們這樣做.
我們就這樣做.
以前可以說是一個太平盛世的時候就這樣做.
我記得到了十多年前.
那段時間中東或者非洲很多地方很混亂.
而回教的狂熱有時會威脅基督徒的生命.
有時遊擊隊會去到一個地方.
「是基督徒嗎?」.
「是就殺,不是就放走」.
那時我會想.
為何現在文明社會還有這些事呢?.
基督徒有時會在突發情形面對.
那不是我們平時生活的獨特標準.
而是好像考驗你對基督教的信仰.
因為你在本能上.
在那一刻.
如果你是很害怕的.
你不承認是基督徒.
那就沒事了.
之後怎樣處理.
可能就像彼得那樣.
三次被認主之後.
然後就再懺悔.
我就在想.
其實到了現在.
好像是時代在不斷地變化.
因為我們面對的環境又變了.
我們可以面對不同的課題.
我就在想.
所以我有一個迷茫.
我們在追求怎樣的東西呢?.
所謂在信仰上的成長.
梁俊賢:所以就說不能夠用以前的方法.
因為你會發覺某個在七十年代.
追求成性的人.
會拒絕這一代的人.

$^{1281}$說他們不是基督徒.
那個情況就是這樣.
所以就說原來我的成性.
不是我做一馬嘴上壓.
不是我做得很多.
做得很好就能夠做到.
因為問題是你做得很闊.
原來我日讀經日靈修.
一定是好基督徒.
但原來這種方法.
你未能夠回應到很多.
其他時代的議題的時候.
都不是議題問題.
而是你要更新自己.
更新你的知識.
更新你的眼界.
不要太主觀.
看事物的時候是需要的.
我這樣說不是說你不.
剛才說的都跳過了.
仍然要做好基督徒.
那些事是大家在做的.
但這個之餘更加要更新.
不只是80年代那套觀念.
就叫做好基督徒.
其他就不是了.
不可行的.
我自己也試過很多這樣的經歷.
被很多前輩說我不是好基督徒.
但他們是很好的.
但他們有些東西好像不足夠.
反過來說.
我自己也更新了.
我33歲回來香港教書的時候.
正正是很多前輩覺得.
陳永安就是這個樣子.
33,34歲的時候.
寫的東西很激烈.
又怎樣怎樣說.
但其實我已經8年後了.

$^{1321}$很多東西已經改變了.
更新了.
但我印象就是在那個時候.
人是需要變.
也需要不是停留在那個裡面.
所以無論牧師或我.
誰都一樣.
大家都需要不斷更新.
開闊自己的眼界.
重新看事物.
不知道答不答到你的問題.
但我覺得迷惘的地方就是這樣不足夠.
所以更加需要更新.
大家對更新有什麼想法嗎?.
我想問一下更新.
即將到來的我們自己.
圍內小組也想面對2022年.
怎樣運作.
因為我們現在還在建立小組的模式.
我想問一下更新的意思.
代表以前回教會.
收集了其他組員的經歷.
通常有祈禱,崇拜,查經.
或是分享,見證.
更新代表這些元素都抹去嗎?.
或是基於這些元素.
轉變成另一種模式.
出現在現在FoldChurch的小組裡?.
我先簡單說.
更新不是代表抹去之前.
大家覺得是好的東西.
用食肆為例.
你會發覺食材不是雞,牛,羊的肉類.
其實也不會突然有什麼.
除非你說什麼分子料理.
其實是在吃羊而不是羊.
不是說這些.
反而是說基督徒成性.
讀經,祈禱,靈修的基本.
可以吸收的方法.

$^{1361}$其實不是不好的.
反而你說團體更新的時候.
我覺得.
我想回應更新和熱誠的連結.
你的小組接下來的運作.
有什麼可以令小組十多人裡.
都有熱誠.
回來小組享受.
這個才是重要.
因為我和學生說.
過去教會編週會.
大部分時間都用填充式編週會.
什麼叫填充式?.
第一週是查經週.
第二週是專題週.
第三週是祈禱會.
第四週是其他特別聚會.
很快就填充了.
因為你覺得屬靈的基本單元.
要全部填充.
但其實你沒有很大的熱誠.
想回來.
如果你不喜歡查經.
可能你不想.
第一週是查經週.
想填充就不回來.
你找不到熱誠回來.
反而我覺得你們十多人.
收了不同喜好的時候.
大家會否找到最大公因素.
或者覺得有什麼.
令你回來小組有熱誠.
這很重要.
對於我們來說.
好像Info Group.
我當中也和弟姐妹說.
你對小組有什麼期望.
有什麼令你想在小組中.
大家一起成長.
一起去過程.

$^{1401}$那個熱誠是很重要的.
所以對於你們來說.
小組裡面是說查經.
查經只是一個形式.
但關鍵是什麼呢?.
查什麼經呢?.
什麼主題呢?.
什麼內容呢?.
你們小組出來.
令大家有個向度.
有個熱誠回來.
這個才是最重要的.
不是說煮食的方法.
是煮什麼食物.
怎樣煮.
令你覺得.
很美味.
我想回來.
這個重新是重要的.
(包括有被辱罵的感覺).
(有辱罵的事是對大家的).
營養這個字也很浮誇.
因為營養有時是有品味的.
但重點是.
過去大家都試過.
你自己帶過那個周會.
或者準備那個周會的時候.
你自己最容易被那個周會得著.
那個過程當中.
不是你帶過那一刻.
是你準備那一刻.
已經在養你的人.
你自己已經在概念上開放了.
在內容上.
知識上更新了.
你帶的時候.
跟別人混的時候.
你就會多了一些技巧聆聽.
那個已經被牧養了.
不只是來自牧者牧養.

$^{1441}$你是透過教導.
透過彼此分享時.
就在牧養整個群體.
包括自己.
無論你是小組內容更新.
或者教會都一樣.
其實兩樣東西要執著.
哪些要 哪些不要.
哪些要拆 哪些不拆.
Flow Church本身就是教會改革.
我們說有些東西要改變.
當時拿著什麼來改變呢.
重新砍掉所有東西.
還是什麼呢.
就是說你很需要知道那個essence.
你知道那個essence.
你就知道其他東西可以拆掉.
我知道大家這個群體.
是需要實踐彼此相愛的.
我們是需要建立好關係.
我們要學習聖經.
這些是essence.
但怎樣做呢.
查經還是看影片.
還是什麼都好.
這些是方法.
抓住那個essence.
這是第一點.
拆掉.
有些人說.
我以前團體一定要搞美食烹飪周.
這些是form.
很迷戀這個烹飪周.
就是這個迷戀.
但烹飪周的目的就是.
大家彼此有興趣生活.
建立關係.
所以有些教會是這樣的.
有些教會20年都是做這些.
大家只是教會.

$^{1481}$一開始是ice breaking遊戲.
正正就是那個form變成了傳統.
不可砍掉的傳統.
大家30歲還玩ice breaking遊戲.
變成這樣的情況.
所以我們會抓住essence.
第一點.
第二就是你需要因應那個現狀.
一個小組裡面.
現在我的情況.
我們有什麼需要.
正如Poulson所說.
有什麼需要.
有什麼situation出現.
我們就是拼在一起.
essence和現狀需要去想新的東西.
更新就是這樣.
因為明年可能需要不同.
所以就重新去想.
所以確實沒有什麼是不可砍掉的.
烹飪周.
但其實essence就在這裡.
大家要分享的時間.
大家有彼此認識多一點的時間.
都有一段大家在靈明上進深的時間.
這個就是模樣.
所以我想essence拿著.
按著不同的時候需要去吻合.
這個就是我們不斷去更新.
要抓住兩個很重要的元素.
其他東西.
傳統一定要唱團呼呼團奮.
這個就不一定要.
你就可以砍掉它.
你拿著一些東西去決定.
所以你很知道essence的需要.
很多電影節目還不夠認識essence.
以為這個是必須的.
甚至Full Shot也是.
有些人說Full Shot好像改變了.

$^{1521}$以前沒有這些東西.
為什麼呢?.
因為那個情況不同了.
如果我說我做足十年都是這樣.
這個不就是傳統了.
我很但願Full Shot的傳統就是沒有傳統.
可以不斷地去更新自己.
按著不同的需要.
抓住教會essence去不斷地改變.
不斷地轉變.
這個就是我們很想做的事.
所以這個不是.
你今天的我打倒了昨天的我.
就是這樣.
就是需要更新改變.
我想問不斷更新的意思.
你也認同人是要不斷處於上升軌.
持續更新就是剛才說的一套.
要不斷上升.
更加越來越近耶穌.
另外也想問.
當我們擴闊視野和認識更多的時候.
很多時候都會接觸到很多不同的看法.
例如剛才了解到.
路德這樣看.
嘉義文這樣看.
好像每個人都有些道理.
反省自己.
我有什麼可以去學.
或者去吸收.
或者裡面有什麼要調教.
這個過程可否多說一些怎樣做.
因為有時越看得多.
又發覺自己不懂的原來有更多.
有時不太懂得怎樣建立自己的看法.
這方面有點深.
剛才那些圖表是很籠統的說法.
如果你問我.
我有點像嘉義文.
就是沒得說.

$^{1561}$沒得定.
因為你不一定只會在下面.
也不一定會向上面.
不過你問我當然想不想.
當然想向上面.
所以只不過是我想.
不是說我必定會保證.
所以我們當然是想向上.
更加好.
但那個方法就不是保證.
正正都不是說你保證怎樣做就怎樣好.
而是不斷更新.
不斷去更新.
因為人是會跌的.
我第一本書也說.
靈命就是一些東西.
你今天很好靈明.
第二天你有事.
突然崩潰就跌到地上.
但地上不代表你以後都衰逝.
你會好好的.
所以這個正正就是我們可以去跟隨.
去更新我們的情況.
所以有點像嘉義文.
就是你什麼時候在基督裡面.
什麼時候被更新改變.
你就可以走得高一點.
這個情況就是你沒有確定性.
但你只能不斷更新自己.
去尋找一個可以做基督徒的方法.
這個經歷我不知道怎樣分享.
潘Sir有沒有其他看法.
我自己.
這個問題很好.
還有很多東西可以說.
但我想先慢慢收窄.
我自己看自己的做法.
有點像John Wesley的專業.
我覺得會上下都要保持上下.
我覺得成性的觀念都是這樣.

$^{1601}$因為我覺得在聖經裡面.
人是有這樣的模式.
譬如彼得就是這樣.
他躲起來也好.
他都有上下.
但最後也有上下.
讓他更加知道.
他決定跟隨耶穌.
他不會回頭.
甚至像一個耶穌很喜愛的門徒約翰.
如果你看到路加福音第九章.
就是耶穌和門徒經過撒米爾村莊的時候.
那些撒米爾村民不接待耶穌.
約翰和雅各就跟耶穌說.
夫子我們需要求天火燒死那些人嗎.
這些是很差的東西.
那時候小主的註釋也說.
他是一個雷子.
性情很剛烈.
不可一世.
很威風.
他見過耶穌的真身.
你看到這個有多完美.
有多成性呢.
但你會看到這個被稱為這麼喜愛的門徒.
他見到自己的夫子釘上十字架之後.
他的性情就變了.
變了的地方就是.
最明顯看到的就是約翰一書三章十六節.
主為我們寫明.
我們從此就知道何為愛.
我們也當為弟兄寫明.
當初一個人說要燒死不接待耶穌的人.
但當見到夫子那種犧牲的愛的時候.
在他晚年的時候,他的性情就改變了.
他的願意犧牲,願意接納更多.
願意在當中有一些.
從年輕到老年的時候.
性情的改變.
看到他的專業是上下走.

$^{1641}$即使是上下走也好.
他也會有一個基督形象的表達方式.
你問我如何具體做呢.
我覺得上一課的passion.
可以在你自己的生命當中.
繼續提醒自己的方式.
或者讓自己繼續可以存活的心智.
我想早上年還是前年.
在網上很多時候都傳一句說話.
就是一個日本小說家.
叫做本間九雄說一句說話.
很多人三十歲就死亡.
八十歲才埋葬.
就是因為三十歲過後.
已經沒有人生生活的那種passion.
每天上班下班,做基本事.
但沒有什麼特別意義.
這一定不會是基督徒的成性觀.
或者基督徒的生活態度.
我們不是在意短暫時間的up and down track.
反而是我們在乎每天有沒有經歷.
或者在做信仰要表達的方式.
又或者信仰對我們自己的提醒.
我感受到有些東西是我覺得能力有限.
我需要別人提醒我.
或者我不懂的時候.
可以問我身邊的buddy.
所以在剛才還沒開始的時候.
跟John再談一下.
成性對我們來說一定不是個人性.
其實是一個群體.
和彼此的一個partnership去做這件事.
令這件事可以更加方便.
或者多些程度的不同層次.
(記者:大家對成性有沒有什麼困擾?).
(記者:還是覺得已經可以了?).
我想問是否純服性的帶領.
開始踏上成性的階段呢?.
因為很多聖賢都是純服性.
然後他們成為很多的牧者.

$^{1681}$或者是一些很好的傳道人.
我想問一下Flow Church有沒有異象呢?.
我不懂回答.
當然我們可以很簡單回答.
當然是聖靈帶領我們可以做基督徒.
就可以做好基督徒.
這樣就完結了.
但這是上帝自己的perspective去看.
但有時人的層面.
可能我們下第二季.
如果有的話我們會再說聖靈的topic.
因為聖靈是一個很abstract的topic.
什麼叫聖靈帶領你.
什麼叫有聖靈帶領你.
當然我們知道是有.
但問題是我們在一個人的層面去說.
其實不想停在這個句子就完結.
如果你說到聖靈帶領你.
大家就不知道拿著什麼回家.
所以我們仍然有些事情可以做.
都要去想.
所以為什麼會說和聖的東西.
當然我們說聖靈.
基督的靈.
作為一個成性的靈很重要.
但我們作為人都要有些事情去想和做.
所以如果單單說聖靈.
我是一個不負責任的教導者.
很重要.
但不單單說聖靈就完結.
有關這個topic大概是這個意思.
因為聖靈是工作.
聖靈是一個subjective god.
在我們生命裡.
所以我們要做什麼.
都是一個很重要的topic.
所以我們關心的.
我們具體在聖靈裡面怎樣做.
另一個話題.
很多時候大家都是.

$^{1721}$很多人在教會上吵架.
大家都是好基督徒.
難道你說他沒有聖靈.
這很主觀.
當我吵架的時候.
你沒有聖靈.
這樣很糟糕.
你說人家沒有聖靈.
我沒有聖靈.
所以這個很不容易說.
但我覺得是.
在一個時段來看.
大家要不斷被更新.
聖靈會帶領我們一起.
從一個不咬弦.
或者矛盾裡面.
慢慢去有合一的聖靈.
所以我覺得這個topic.
我大概想這樣說.
Full Church的vision很簡單.
就是我們在這個年代裡面.
為基督作耶穌的見證.
就是這麼簡單.
所以我們是一間.
去見證耶穌的教會.
特別是在這個年代裡面.
能夠用一種.
能夠容易到地明白的方式.
去宣揚基督耶穌的盼望.
我剛才在想的位置.
就是怎樣可以容易.
articulate那個層面.
通常我講聖靈.
我都會主要在門徒訓練的內容.
講那個身份教育.
因為剛才開初.
我講聖靈的帶領.
因為我們一路.
由第一天開始做基督徒.
聖靈就內在你的心裡.

$^{1761}$它一直都在你那裡.
所以我自己的教導層面.
常常都說.
聖靈叫保衛師.
又叫訓衛師.
就是它會圍著跟你說話.
它跟你說甚麼呢?.
其實就是提醒你一個.
上帝而來的身份.
你是一個基督徒.
你的身份而有的行為表現.
就是一個應該要有的東西.
用回聖經的說話.
就是一個新造的人有三個特質.
就是已分所書四章二十四節.
就是真理仁義聖潔.
在這個身份上.
你能不能夠去更加學悟.
和了解上帝的真理呢?.
在你的法則上.
你的判斷上.
有沒有上帝的公義呢?.
你的行為上有沒有一個.
聖潔的表現.
讓人感受到你是一個.
信主的人.
有那種應該要有的表達方式.
和行為原則呢?.
所以你說成聖的話.
或者聖靈的帶領過程當中.
就不斷重提你有那種身份.
而有最基本的三個向導的表現.
所以你剛才說到聖靈帶領的話.
我覺得先從你自己信徒身份.
或者基督徒身份當中.
去了解你的行為表現.
我覺得會比較具體一點.
不要只側重聖靈有沒有特別跟你說話.
或者一些方言.
或者其他比較抽象的.

$^{1801}$不容易去了解或看到的表現.
下一課就完結了.
下一課我們作為總結.
是關於跟隨主耶穌.
不重要了.
但我們這裡A類.
一台只有兩個人.
六點後沒有東西吃.
下次我們收店就吃一餐好一點.
下次我們收店再說吧.
這個很重要.
今天收店對我們來說.
第七課完結了.
下個月見了.
希望下個月見到大家.
好,拜拜.
\newpage



\section{}
\label{sec:HS1KRCnzG5o}
\textbf{【這是最好的時代:給香港基督徒的神學八課】第8課:跟隨耶穌的一百萬個可能|20211218 [HS1KRCnzG5o]}
\newline
\newline
連結: \href{https://youtube.com/watch?v=HS1KRCnzG5o}{\texttt{ https://youtube.com/watch?v=HS1KRCnzG5o}} ~~~~ 語音日期: 2021-12-18 
\newline
\newline
\hyperref[sec:cYPXvL44u1Y]{\small{< < < PREV SERMON < < <}}
~
\hyperref[sec:index_chronic]{\small{[返順時目]}}
~
\hyperref[sec:index_scriptual]{\small{[返順卷目]}}
~
\hyperref[sec:49X8yc0UC2g]{\small{> > > NEXT SERMON > > >}}
\newline
\newline
$^{1}$(廣播中).
《側徑》是蘇恩配以前寫的小說.
蘇恩配很久以前出生.
在七十年代時期突破很重要的人.
後來影響了蔡元雲,梁家麟等前輩.
蘇恩配姐妹寫這本小說《側徑》.
是講述他那個年代如何跟隨耶穌的方法.
後來這首詩歌叫《只有祝福》.
可能大家都聽過.
後來我在《Fourchurch》第四封家書也寫過.
大家有沒有記得這本書的內容.
它講述一個「子」字.
很有趣,如何用這個「子」字呢?.
這個「子」字是解作什麼呢?.
是解作「神明」的意思.
只有祝福,沒有奏作.
不是純粹解作「祝福Only」.
沒有任何奏作.
這個「子」字同時解作「上帝」的意思.
即是上帝仍然有祝福.
我想是一意相關.
基督徒一生中.
如果將時間線延長到永恆的時候.
確實是得到祝福的.
因為這是一個完完全全是上帝恩典的時間.
但基督徒一生也不只是寄望著那個終末.
而是此時此刻這個年代裡我們首要的關懷.
因此所謂祝福的意思是什麼呢?.
「子」字是解作「創造天地的上帝」.
祂是一個勝過世代的權勢的上帝.
雖然罪的勢力仍然存在.
死亡仍然存在.
不過我們仍然去認順.
世上只有祝福.
是上帝的恩典.
所以這個說話也成為了我們很大的提醒.
雙倍也說過.
與其就著黑暗不如燃燒自己.
這個說話在我們今天仍然是用得著.
在這個年代裡我們仍然走一條窄路.

$^{41}$這個窄路是昔日雙倍姐妹.
都嘗試去跟隨耶穌的方法.
可能方法不同.
但路是一樣的.
都是在黑暗裡嘗試跟隨.
嘗試去找一條可以走的道路.
所以受告裡面說.
只有祝福因我已踏上那則徑.
只有祝福因我已突破那罪的勢力.
只有祝福因我已勝過那死亡.
沒有就坐.
只有祝福.
這也成為我們全聖經頂尖輩嘗試去實踐的東西.
你問我今天說的也很虛偽.
沒有什麼特別的具體建議.
確實是的.
前面是沒有什麼具體的聰明智慧.
來去怎麼走.
但我想就是說.
我們確認承認.
我們有很多不同的走的方法.
可能有些人是不認同的.
可能有些人覺得我們是很另類的.
可能覺得我們是反傳統的.
但這個都是我們跟隨耶穌的方法.
很值得去尋求.
我們怎麼可以去跟隨耶穌.
最後這個部分我自己做了一下也挺感動的.
我嘗試去.
弄了雙倍之後就多加一些人出來.
全部都是死了的人.
嘗試加一些死了的人在這裡.
找回以前的屬靈前輩.
我們年輕的時候的樣子.
其實我們這個年紀.
三十多歲四十多歲的年紀的人.
每一代的我們今天稱之為.
我們很敬重的屬靈的前輩.
都做過年輕人.
都做過青年人.

$^{81}$他們都在那個年代裡.
嘗試去尋找他們那個年代裡.
跟隨耶穌的方法.
後來才成為了被定義為屬靈的傳統.
即是突破.
當時是一個很突破的雜誌.
今天也是.
這些新的事情.
很多不同年代的人.
他們都做過青年人.
嘗試去尋找一些跟隨耶穌的方法.
所以我特意把Full Church的標誌放在這裡.
我想如果你這樣看的時候.
你會發覺我們Full Church很快想到.
如果許可的話.
三十年後四十年後.
其實大家都老.
大家都成為了一些老人家.
或者一些退休人士.
但我們仍然是值得去問.
我們接下來的二十多三十年.
我們可以怎樣來到香港裡面.
繼續來生存下去.
跟隨耶穌下去.
所以我希望我們在這個年代裡.
懷著勇氣.
踏上未知的跟隨主的責勁.
責勁是什麼呢?.
就是一條很狹窄的道路.
狹窄有很多意思.
一來是艱苦.
二來就是很難找.
你需要去尋找出來.
第三就是很少人走.
所以這個正是我們基督徒.
我們去尋找耶穌裡面走的路徑.
如果這條路是這樣走.
有地圖.
有所有的起點和終點.
有中間的checkpoint的話.

$^{121}$未必是耶穌叫你走的道路.
是一條你前面都找不到.
我也不知道怎樣走的道路.
但這個成為我們今天.
大家一起來去共勉.
Full Church的目標仍然是那一句.
大家在這年代裡.
跟隨耶穌基督.
上師來尋找耶穌基督的足跡.
這樣走下去.
好,今天想和大家多聊天.
所以大家可以.
待會有聊天時間.
我們先祈禱.
好,我們先祈禱.
然後才有聊天時間.
祝我們去求你幫助我們.
因為我們在這一刻裡.
真的有很多未知之數.
我們Full Church的姐妹.
我們都不知道怎樣可以.
靠著你走過去.
如果我們能夠去效法你.
去跟隨你.
去尋找你的足跡.
我們都能夠在這個年代裡.
去找出我們繼續去跟隨你的方法.
無論是我們的目者.
我們一班的頂尖妹.
我們願意去有心的時候.
讓我們藉著不同的討論.
都能夠找得到我們去跟隨你的方法.
可能有些頂尖妹在外國裡.
有些頂尖妹在香港裡.
求主你成為我們這樣的一個主.
去告訴我們怎樣去跟隨你.
去有勇敢的去走一些新的事情.
讓我們能夠去找到一些.
可以知道你心意的方法.
藉著你的靈的帶領來幫助我們.

$^{161}$讓我們教會能夠成為一個更加全面的教會.
奉主命求,阿們.
喂,今天有什麼吃的?.
有花生,吃花生了.
有些花生米.
有杯飲料.
真的有7\%.
不用賣廣告都可以.
謝謝.
你最後一堂想說什麼?.
有什麼花生米?.
你覺得有什麼花生米?.
感動米.
我們的設定已經是很花生米.
我經常都說我們這個主學是有POP市的.
下一季的POP市會不會成雞?.
想一下,可能是花茶.
花茶可能是令女群看的.
我覺得感受上是很大的.
因為我覺得整個系列,八堂裡面最主要都是一些.
我自己覺得是一些氣質.
怎樣可以在身份上表達一些很實在的東西.
雖然你剛才最後的結束.
覺得好像不是一些很實在的東西.
其實我自己看就是.
其實呼應第一堂關於門徒訓練.
關於門徒這個身份其實是什麼一回事.
我覺得第一堂和第八堂.
最主要都是想帶出.
條命是什麼.
其實是你自己怎樣認受你自己是一個什麼身份的人.
今天這個訊息.
或者是最後這個PowerPoint裡面.
每一代的人其實都在尋找自己的身份.
可以做些什麼.
我覺得寄望Folk Church的弟兄姊妹都是.
不是一些很爆的東西.
不是一些很特別的與人不同的事情.
反而是在自己的崗位.
在自己的身份當中.

$^{201}$做到一些你可以做的東西.
這個是最重要的.
我自己在Folk Church三年.
認識很多不同的弟兄姊妹.
不過他們都很厲害.
很多弟兄姊妹其實都很厲害.
怎樣能夠可以在.
Folk Church不只是一個他們每個星期去敬拜的地方.
但是他們自己在外面是更加多可能.
我覺得是.
或者他們是一個很普通的牧者.
但是他們更加大的潛力.
我覺得Folk Church是一個很多不同的人聚在一起的地方.
但是他們自己都有不同的故事.
這個更加突破.
更加力量大的地方.
是呀.
我有時都跟一些組員或者弟兄姊妹說.
回到Folk Church有時會很尊重潘Sir,牧者.
但是我都常常強調.
在教會.
回到是一個彼此尊重的地方.
但是就不需要特別去調整自己的能力.
有時回到教會很謙卑.
很謙讓.
很多人說不行的.
但其實他在公司裡面是管理人員.
是有識之士.
回到教會好像就不敢說自己一些特別專長.
我覺得其實就不用.
回到大家就各行各職.
或者你有些什麼.
你又可以用到的地方.
大家彼此去欣賞.
彼此去配合.
這個是重要的.
講得俗一點.
你回到公司可能你IQ84.
你回到教會不用只有40.
有時就好像自貶身價.

$^{241}$或者有些東西是過份去謙讓.
其實我覺得不需要.
我和你都是在教會裡面的人.
我們那個群組都是這麼多.
最多都是外面多一點.
都是教會.
但其實等於每期更加闊一點.
我們暫時都是純粹讓教會裡面有些東西是能夠得到.
但更加大的場所就在教會外面.
所以為什麼FoodCenter都強調.
不要在教會裡面這麼多侍奉.
意思就是說不要只是在這裡.
我們說跟隨人數不要只是在教會裡面.
我們沒有一條路讓你跟隨人數走下去.
這樣是不好的.
可能會有更加大的需要.
特別是在這個年代裡.
教會其實不需要很多東西.
反而是外面更加多東西.
我常常都說我們提供一個平台給弟兄姊妹一起去分享.
這個位置就是.
我很感受保羅在《哥林多後書》說的.
每個人的經歷有前後.
我只不過是走你之前走過的路.
有個空間一起分享這個是重要的.
所以大家經歷了這麼多課.
有沒有什麼可以分享.
或者你期望在這些課堂內容.
你有什麼再想聽.
都可以大家談一下.
有,中間有.
我今天聽完十個八課之後.
我覺得這八課裡面的講道都比較詳盡和精彩.
我期望下一次的十個八課.
都可以加插一些不同的生活圖畫.
令故事裡面都很有趣.
都有開心的事情發生.
即是要製作一些場景劇.
即是說可以用故事都可以的.
不需要做劇.

$^{281}$其他呢.
有沒有什麼話題想將來談一下.
或者想知道多一點都可以.
網上有一個人問.
可能大家都談一下.
剛才我不是說到.
我們不是純粹.
所謂的根除.
成為了我們詮釋的問題.
很多不同的人對於耶穌有不同的詮釋.
所以我就說.
似乎不是純粹根據一個過去了.
或者這樣去做.
這個我想是對的.
我想大家沒有說誰錯.
可能有些教會.
他們的詮釋的耶穌.
會比較沒有那麼政治關懷.
或者沒有那麼多這些.
我們可能不是這樣看.
但我覺得這個正正就是很需要的.
因為大家有不同的詮釋.
之餘大家又不是反對對方.
我經常說我們Folk Church.
不是反對任何其他的傳統.
而是我們只不過是.
開放其他不同的可能.
所以我們正正就是要根除的.
是一個我們在前面的耶穌.
嘗試可以去尋求一些新的方式.
來根除耶穌.
所以這個我覺得不是純粹去拒絕.
過去或者其他的傳統.
而是嘗試去尋求一下.
我們在這個年代裡.
怎樣去做基督徒.
耶穌會怎樣做呢.
這個問題是值得我們用信心去尋求的.
在這一點我回應或者說一些看法.
我自己教基督教教育.

$^{321}$很多時候都被人問到.
課程設計怎樣可以令到弟兄姊妹.
在信徒的靈命培育上.
可以有一些準則或者進程.
但是我很多時候和一些主要學校校長.
或者一些信徒去領袖去談這些課題內容的時候.
我都先和他們說一件事.
就是某程度上現在的信徒培育.
都會受工業革命影響.
因為工業革命就影響到教育有一個模式.
怎樣可以有一個標準.
怎樣可以保證那件事是合格的.
但是我覺得信徒培育正正就未必一定要用這個方法.
因為我們傳統成長的學習階段.
都會有很多所謂門檻或者標準.
但是我常常都覺得.
基督教教育就沒有一個特定的標準一定要通過.
這個可能你可以不認同的.
但是我覺得聖經對於初信的.
對於年長的.
對於進心的其實有不同人.
有不同的領袖或者可以學習的內容.
所以不一定.
還有沒有限時間要多久讀完一本聖經.
其實就不一定要跑一個syllabus.
正正就是.
有時好像John.
我覺得最後那三句金句.
對他來說是他可以賴以過世的經文.
其實都可以的.
那個正正就是他每天提醒自己.
他的身份.
他的行為表現.
他對人對事的準則.
這個已經是很重要的.
反而不是說懂多少金句.
上了多少課程.
這個對於我們來說.
就不需要用課程內容.
或者上了多少課程套進去.

$^{361}$就說我是一個信徒領袖.
或者已經懂了這些東西.
你好.
我想問一個問題.
就是有很多不同的教會.
都有一些不同的側徑的路向.
有時候在這個側徑裡面.
本身一開始的一班人.
他們分了出來.
認為自己是一個新路.
然後但這班人走了一段時間的時候.
他們對聖靈的感動.
對印證在聖經上的明白.
和對可能大家一起交通過之後.
這個側徑明顯地有兩個東西的分別.
然後也有些教會.
可能本身在小的時候.
因為他可以是一種很自由的方式去明白主.
但因為這個側徑真的吸引到很多的.
或者我們的信徒.
去加入這個教會.
這樣就變大了.
變大了的時候.
我們就公式化.
我們就有一套準則的機制.
為了有效率去做事.
有時候這個側徑.
那個初衷已經是不同了.
或者那個路不知道怎麼去走.
有一些服務的弟兄姊妹.
或者參與的弟兄姊妹.
有時候很想保持那種合一.
保守那種合一.
但是他又在這裡.
他不能夠達到他們.
可能一個圈子裡面.
可能二十個人.
他們都有這個心.
他們做不到的時候.
他們退了下來.

$^{401}$我不服侍了.
我都會仍然留在教會.
甚至有一些就說.
我認為抓住一些主的說話.
我自己就再創另一條側徑.
但是其實經過了幾十年後.
有時候可能會發現.
原來所謂的分裂.
是一種彰顯主不同的榮美.
其實我們只是21世紀.
不知道還有多少個世紀.
我們才去到面見到神.
其實我很想知道.
當我們真的去到一個教會.
我很想保守合一.
我想繼續侍奉.
有一些就說.
有一人一票的這種情況.
可以去投票.
然後再去選擇我們前面的路向.
但當變成一個很大的教會的時候.
可能真的會有一些限制.
或者是當我們面對這個世代.
我們都很有抱負.
很有雄心.
認為我們會做得很好的.
這個側徑是會變大的.
現在是一條窄路.
將來我們怎樣可以.
繼續保持著這種初衷.
然後繼續去指望將來.
我們會有著主的榮美.
我們不會分裂.
我們永遠合一.
可否給我一些實際的例子.
我參考一下.
謝謝.
剛才你說的是一個很具體的故事.
很具體的.
這是你很真實的經驗.

$^{441}$或者是一些看過的東西.
我覺得重點不是.
不要將Foltrace 當成一條徑.
我經常說.
Foltrace 不是一條船.
不要想著Foltrace 要成為甚麼.
我經常跟Foltrace說.
Foltrace是一個大家聚在一起的地方.
但舞台在你們外面.
所以不需要幫Foltrace 想一些側徑或不側徑.
Foltrace 只是一個大家可以學習跟隨耶穌的方法.
但跟隨的實境和實踐不是在Foltrace裡面.
所以你說是否合一.
其實我覺得.
所以剛才那些是很虛弱的.
因為全部是沒有甚麼具體的方法.
如果是一些比較洗腦的方法.
不如叫你怎樣怎樣.
大家就做吧.
但其實是沒有的.
大家都是找自己去跟隨耶穌的方法.
而跟隨的方法不是去幫Foltrace.
去想之後的發展.
所以我唯一要說的.
大家一定要知道.
有很多不同的跟隨方法.
跟隨方法不是在Foltrace裡面.
而是在Foltrace外面.
這個才是真正能夠讓Foltrace去更新的地方.
因為Foltrace之前也說過.
Foltrace唯一一條不能改變的.
就是不斷要改變.
不斷更新就是我們Foltrace最要掌握的東西.
我今天是我 明天是我.
我們Foltrace本身也不是一個讓大家去發展的地方.
所以從來都沒有想過大家要去幫Foltrace發展.
而是成為一個平台.
讓大家在外面去跟隨耶穌.
而那個是側徑.
不是一個人走的就只有你.

$^{481}$不是大家一起走 走到很大.
而是大家有很多不同的方法.
所以這個明顯是不合一的.
大家都有不同的方法去實踐和去想和詮釋.
我覺得這個OK的.
東和西完全相反都OK的.
只是很多時候別人不OK我們.
但是我們是OK的.
我們是有不同的想法.
所以我覺得我們是需要這樣去做基督徒.
不知道潘Sir有沒有補充.
不是補充 我想大家相輔相成的地方就是.
我認同John剛才姐妹說的那些地方都是很具體的.
不過我覺得要有什麼例子.
我覺得真的不同人有不同的例子.
不是一個戴頭盔的說法.
因為耶穌和被追問的時候.
耶穌都沒有什麼具體例子.
人們問要怎麼做的時候.
耶穌就從他重述兩個大的綱要.
第一就是你要盡心盡意盡力去愛主你的神.
向上.
你知道你在敬拜的是誰.
其次有一雙方就是愛人如己.
其實耶穌都沒有說這些很特別的場景.
當然在馬太福音第五章至八章裡面.
有一些很日常的事情.
他會說一個律法.
一個有天國子民特質的氣質的人.
他應該在具體上怎麼做.
但是重點就是你碰到的人就是你要碰到的人.
你碰到的人不是我碰到.
所以我不可以告訴你.
你應該碰到的人怎麼做.
因為你最熟悉那些人.
瑞士理論社就是你最清楚誰是理論社.
我講得多麼學術也好.
多麼大包圍也好.
我不是跟那個人住.
是你才跟那個人住.

$^{521}$所以應該是你最清楚怎麼去對應那個人.
而不是我講所有的點.
我告訴你應該怎麼做.
因為教書在香港的教育方法就是.
什麼都講.
考不考的.
你去補習就會問這些考不考的.
不考就不要教了.
因為我會給錢的.
重點就是你常常都想拿一些精讀.
但是我覺得信仰就不是精讀.
信仰是一個通識.
通識的意思就是我應用神學就是.
你要懂得去探索場景是什麼.
第二件事是懂得分辨場景是什麼.
以至你做了神學反省.
你才知道這班人我用什麼方式.
這個處境我用什麼態度.
然後就是這件事有沒有違反我的聖經原則.
就是這樣.
所以很難是one size fits all.
或者是apply to all ages.
這個情況.
左邊這個.
不好意思.
想問兩個問題.
第一個問題可能被最後一幕觸動到我.
我也知道在過去的時候.
很多屬靈的前輩和長輩.
在當時都開創了一些新的路出來.
譬如黃明道或者黎作星.
在過去火紅的年代.
為中國的信徒帶來了一個屬靈的視野.
但是事後看回來.
似乎他們的屬靈派或不信派.
造成教會的基要主義.
或者教會的分裂.
回過頭來看.
是不是真的這麼對.
我們今天去走一條新的路的時候.

$^{561}$或者去探索不同的可能性的時候.
我們如何確保我們走上這條路.
或者開拓這條路.
會在現實的場景裡.
或者事後看回來.
不是走錯路.
這條路是真的對的路.
是上帝喜悅的路.
這是第一個問題.
反省我們應該如何去做.
這是第一件事.
第二件事我想回應.
你用蘇欣沛姐妹的小說《側徑》.
但我記得《側徑》裡有一幕是說.
當時的女主角選擇留在美國.
還是回台灣工作的時候.
她選擇了.
當時很多留學生不願意回到落後的東亞地區.
而願意留在美國繼續生活.
去服侍.
因為當時美國比較繁榮富庶.
台灣比較落後.
事實上最後蘇欣沛選擇了去台灣服侍.
當然後來她的癌症就是後話.
再回來香港發展突破.
這些全部都是後話.
但她說的側徑似乎不是純粹可能性的問題.
而是她看到.
這個抉擇是誰比較好走.
誰不太好走.
哪個代價的問題.
甚至哪個比較有好處.
坦白說.
似乎是那個面量會多一點.
我覺得你用她的側徑來詮釋.
會不會好像有一點不是那樣.
不好意思.
是的.
其實我喜歡這個字.
因為這個側徑的字.

$^{601}$我覺得和窄路的字比較優雅.
還有我喜歡這首歌.
剛才所說的這首歌.
但我自己去認識蘇欣沛的人生的時候.
我覺得有點鼓勵.
其實是不同的意思.
我都說側徑有三個意思.
一個是難走的.
一個是不容易找的.
一個是少人走的.
對我來說更加大意義.
譬如金塘那邊特別大意義.
就是要去尋找側徑.
不是那麼明顯.
還有一點另類.
所以我覺得在我們跟隨耶穌那邊.
當然不是為另類而另類.
但我自己這幾年裡.
感受就是多走一些另類的東西.
今天有人覺得Future還是有點奇怪.
所以我覺得還是有些另類.
但我們覺得.
我們是跟隨耶穌.
我們知道我們要.
只是不同的方式.
所以重點是.
我們正正是用不同的方式.
和一些不是既定的方法.
來跟隨耶穌.
這就是我自己對於側徑的意思.
對於這個問題.
我反而不覺得有點困難.
因為我都說任何東西都在對的地方.
它只不過是錯在時間裡面.
基要派是好東西.
不過是在什麼時間裡.
去練習基要信仰.
所以我們都可以成為基要派.
一百年之後.
任何信仰當你不去跟隨的時候.

$^{641}$就成為基要派.
所以我覺得重點不是內容對還是錯.
而是時間對還是錯.
所以為什麼說要跟隨.
我們要找的耶穌是前面的耶穌.
而不是純粹看回.
這個意思就在這裡.
如果純粹看回我們的傳統.
如果神學百科運用二十年.
我都會懷疑這些問題.
所以今天我們是好東西.
新的 去更新的.
但這套東西都不可能用二十年.
我想我們付出的正正就是.
一個法則.
就是要不斷反映和更新.
沒有什麼是不可以被改變的.
不過很深度的問題和分享.
我都想回應第一個.
怎樣才知道自己會不會錯.
或者怎樣去檢驗一下自己的想法.
對我來說.
信仰從來都是經驗學習的.
經一事象一智.
而在過程當中知道自己的錯漏.
或者知道自己的分辨機制.
或者能力在哪裡.
以至怎樣去不斷修正自己的路.
我覺得上帝給人理性最優美的地方.
就不是一個機械人.
就是可以在自由觀中不斷提醒自己.
這個都是我覺得在過去.
我自己認識教會比較少做的.
做思辨訓練或是一個場景題的情況.
因為很多時候教會教導的時候.
都是很著重經文的詮釋.
對生活的道德法則意義.
但其實我覺得要做多一件事.
就是要提出一個場景題.
就好像回應John今天說的.

$^{681}$如果現在在這個現況當中.
我們要面對這些做決定的時候.
我們的信仰和神學.
給我們一個反省的能力去到哪裡.
這個就在當中討論.
我也認同John剛才說的.
基要派對那個年代的人來說.
是一個頗重要的信仰教導.
可能可以保存當時的道德和信仰範疇.
是很實在的.
正如舊約的人.
其實律法本身是沒有錯的.
只不過是因為律法是讓人知道自己做錯事.
或者令人知罪.
律法本身是沒有錯.
本來在羅馬書講得很清楚.
只不過他要守律法成為別人的重擔.
那就是扭曲了那件事件.
律法本身的原意.
所以當一件事不斷地加添的時候.
就會偏離了原先的那件事.
這個正正就是我們Flow Church的彈性.
是好的.
就是不斷地去檢視我們在做的事.
在這個情況回應剛才的姊妹.
Flow Church很多事都可以改的.
我們那些團隊沒有很強烈的持續性.
我們會拆散重整.
不會說你做了很久.
然後就推倒重來.
又再叫一個團隊去做.
這個都是我們想讓那些弟兄姊妹.
不斷地有不同的更新.
或者對我們來說不要太過固定.
首先多謝John.
因為這八堂.
我自己的感覺就是.
其實John將在教會歷史裡面.
一些相對重要的信仰觀念.
一些人物的思想帶給我們看.

$^{721}$讓我們去看.
其實在過往那裡.
究竟在每個不同的時期.
他們對信仰是怎樣詮釋.
怎樣去實踐出來.
對我們是會有幫助的.
我一直回Flow Church.
或者聽八堂的時候.
我的感覺是.
我們應該從一些很大的宏觀的東西.
走回實際生活那裡.
去看我們人生裡面一些很細微的東西.
其實如果我們去真正去看.
我們會發現一些東西.
但對於我的感覺.
就好像很多年前去看.
一些叫做心靈雞湯的文章.
我覺得相對是比較主觀的.
主觀就是.
每個人去看同樣的事情之後.
大家可能有不同的看法.
有不同的領袖.
有些人會被鼓勵多一點.
有些人聽了會有一時的感動.
但未必會持續下去.
但方向是對的.
因為大家都是會向好的方向去做.
或者是向信仰去追尋去做.
但我的感覺是.
有時就好像沒辦法捉到.
如果用道家的說法.
有些東西不可以嚴存.
只可以意會.
我覺得在教導上有時會有困難.
例如我們經常說.
我們做一個人在最高基督徒.
我們想在我們的人生過後.
究竟在其他人心目中留下了多少回憶.
留下了多少影響.
我一定是對的.

$^{761}$但問題是.
這樣有多少影響呢?.
或者我們怎樣可以做得到呢?.
如果純粹是一心向著信仰去做好.
總之有人因為我的生存.
因為曾經跟我一起經歷過.
受到影響.
這樣就可以了.
這個說法也是對的.
但就好像沒辦法去量度.
或者沒辦法去令大家.
因為是一種教育的感覺.
還有在另一方面.
基督教很多時候我們除了強調.
大家彼此謙卑和諧.
去包容.
或者去明白對方.
大家去放長時間去看.
例如大家對信仰的傳息有不同的看法.
但可能過了幾十年之後.
大家看回的時候.
大家是有少許不同的做法.
但其實我們都是一樣.
會令到神的角度去擴展.
這是其中一樣.
但其實在基督教的信仰裡面.
有一條路就是我們有一個.
對真理的堅持.
或者是一個捍衛.
這就好像在《百科》裡面.
我覺得好像沒有說過這件事.
另外就是.
我想會不會其實在現在的年代.
我們除了去自身做好之外.
其實在基督教的信仰裡面.
我們會有一些情況.
會面對一些衝突.
我們要站出來.
有些事要說.
這就好像在《百科》裡面.

$^{801}$我們沒有特別提過.
我想我是這種語調.
我個人不是這麼衝突.
因為我覺得這個年代裡面.
當然可以學術研究一下這個問題.
真理是不是抗衡出來的問題.
這個可以討論一下.
我自己當然因為很多原因.
都不是這麼絕對.
我和神教授都是這樣.
說完之後.
大家都不會說這個就是答案.
都是一些大家想一下.
或者大家討論一下.
又不一定的.
會有什麼衝突呢.
都是這樣.
我自己都prefer.
因為都是我自己限制的.
我自己prefer在這個年代裡面的真理.
當然有些堅守.
堅守都不是靠一些defend.
或者我要這個才對.
這個就是對.
是容易的.
有時候都需要.
但有時候我們是.
需要的是反而是.
我都說了.
那幾堂課我都說過.
包容.
又說tolerance.
又說要更新.
這些字眼都不是很絕對的字眼.
確實是.
那個路徑確實是.
現在太多人覺得.
我們這裡才是正路.
所以我們要做的.
很多時候都是變成另類.

$^{841}$所以我又變了不是太強.
我不想說.
我這個才是正路.
其他就不是.
正路就比較差.
或者沒有那麼對.
我不會這樣說這些.
這樣.
所以我自己覺得.
我自己都覺得.
因為這個年代裡面.
太多這些太過絕對化了自己的東西.
特別都是.
當你讀了很多歷史.
和很多不同的觀點.
就發覺其實又不是那麼絕對.
我經常說一個.
小規模的true.
相反是false.
但一個極大的true.
相反是另一個true.
所以我覺得.
特別是耶穌基督裡面.
那個信仰的真理.
其實我覺得那個真理.
不是確定出來才有的東西.
而是大家來到去.
交流擁抱.
發現大家都有的東西.
是一些共享的東西.
不過是.
我要承擔.
因為我要抵抗.
因為你不這樣.
這些東西.
但我都會接受的.
所以潘Sir來到.
坐下來就把東西concrete一點.
變得大一點.
實際一點.

$^{881}$但我又覺得.
這些東西是需要.
有些wake 一點的.
因為實際上.
如果太不wake的話.
確實是變成.
大概就是這樣.
跟著來做.
很容易做.
但這個只不過是.
其中一種方法.
我都是要拉平衡.
但我都是嘗試去.
反映一些橫教會的傳統.
不是執行上那麼清楚的東西.
因為確實是.
規限了我們很多的方法.
我覺得這個題目挺好說的.
正正就是.
做教育來說.
什麼叫做標準.
或者什麼是要.
syllabus 或者要說的東西.
我想我.
get到其中一個point.
就是要怎樣提出那件事.
我們要再宣告.
或者再重申.
這個是我們的神學立場.
或者是我們的信仰起點.
我反而有一些疑問.
就是其實這些東西.
不用我們說.
你都知道基督信仰是什麼.
不用我們說.
其實有些東西你都知道.
但我們是否要再說一次呢.
是.
但是不是在課堂上說呢.
我覺得反而要想一下.

$^{921}$因為某程度上.
可能解讀錯誤.
不過我覺得有個情況.
我自己經歷過.
就是好像有些東西.
要有權威.
或者有title.
或者有能力的人說.
那件事才特別要加持.
其實如果是true.
一加一等於二.
這件事其實是小朋友說.
都是true.
不一定要是一個大學教授說.
才是true.
我們基督信仰有很多東西.
基本上是不會有妥協的.
譬如三位體是上帝.
基督的神人異性.
這些不用John說.
才覺得我們一定要說一次.
或者我們要confirm一些東西.
譬如同性戀.
或者其他一些.
可能很多所謂一個.
比較現代一些.
有爭議性的東西.
我覺得我們可以說的.
但又不一定要特別說得.
很嚴正其詞地去說一次.
反而有時有些難處理.
都會被弟兄姊妹問到一件事.
就是應該怎樣做.
好像現在網上都問.
應該怎樣做.
我自己經歷過的教導.
或者弟兄姊妹訓練的時候.
常常他們自己經歷一件事.
就是有得讓他選擇.
他都選錯.

$^{961}$我問你們.
你們不用回答我.
你們做了這麼多年人.
有幾樣東西你選對了.
有什麼不適合你選擇.
你都回答過.
不要說得很大是大非.
老婆老公不是那些.
說到人生生活.
有多少東西可以讓你選擇.
你會選擇對的.
其實你會發覺.
你很緊張選錯.
你不想選錯.
但我常常都覺得.
信仰就是經歷一個對與錯.
是與非的一個經驗學習.
連耶穌都沒有一個detail.
或者govern他回答的答案.
你看科林書裡面.
耶穌和他相遇的人.
耶穌基本上是沒有理會他的答案是什麼.
也不主導他回答什麼答案.
直接是你自己想.
你都不用告訴我最後是怎樣.
你看到尼哥底姆.
撒邁爾鄭棠婦人.
少年紀官.
凡是多敘述耶穌和他對話的人.
耶穌就是重點.
我呈現所有的fact給你知道.
我告訴你結果是怎樣.
但最後你怎樣想和決定.
你不要告訴我.
你就做吧.
你自己選擇吧.
這是耶穌和人的對話.
但反而你現在.
我不是在說什麼問題.
反而很多頂姐妹問.

$^{1001}$你告訴我應該怎樣做.
我很難告訴你怎樣做.
就如我剛才說.
我不是你熟悉那班你接點的人.
其實應該是你最熟悉.
你應該要想想.
你怎樣去articulate那件事.
在做的方法.
而不是我告訴你.
你三碗水煲兩碗喝了就行.
不是這樣.
做prescription是很難的.
因為很難有些東西.
好像吃藥一粒藥就搞定了.
特效藥就是對那件事有效.
其他東西沒有效.
我們都有一些suppose.
大家上完錄像或現場.
大家在小組裡再談.
談完再一起去想實踐.
這件事是整個環境多於答案.
我們就是想這樣.
實踐是一起實踐.
大家一起去想怎樣做.
但這不是說出來的答案.
答案都很簡單.
怎樣做都是做.
反而我覺得重要的是.
有些聖經基礎.
有些神學的background.
你知道這些東西.
你學了這些知識.
但實踐就不是單向講座.
能夠說出來.
很多不同教會的講座.
都是問怎樣做.
是很重要的.
因為是很具體的東西.
但又不是在這些場合.
能夠說得出來的東西.

$^{1041}$因為實踐不是講座.
能夠提供的東西.
是在群體裡.
大家一起去分享.
一起去走一些東西.
所以不是單聽講座.
大家一起去做一些材料.
在群體裡一起去做的東西.
前面.
開了嗎?.
開了.
喂喂喂.
OK OK.
其實我不是沒有問題.
不過我想回應一下.
我覺得Voltage去搞神學百科.
是很有心的.
在這個時代.
可能大家理解.
一般成長班.
我們可能看很多經卷.
或者很多茶經.
或者我想這個.
可能大家.
你們開這個平台.
就是想給大家.
在這個時代.
可能針對這個時代.
去有一些.
可能怎樣扣住.
我們這個年代.
和我們的神學.
其實有什麼關係.
其實可能想做一些這樣的事情.
我沒有上完百堂.
但是我發現你們在講道的時候.
有些閱題.
和你們這些主題.
是有些扣連的.
我覺得其實是挺好的.

$^{1081}$因為都有點深化了.
可能你說聽完之後.
無論你是追溯那個主題的東西.
或者是再解說的時候.
我都覺得是理解又多了.
又看得多了的面向.
我也同意.
有些時候.
有些題目未必可以講得這麼淋漓盡致.
或者講得有多深.
因為可能大家的理解全息.
可能講的時候.
那個情況.
大家都不是很一樣.
但是我想.
我會覺得.
可能你們剛才會問的.
未來下一季又會是怎樣.
我都會覺得是期待.
就是.
就好像你們剛剛說.
可能你們是經過很多的反省.
很多的反思.
然後再構思每一季的主題.
就好像今季.
我都覺得是很有心的.
因為其實我都.
因為這些主題.
這些內容.
我都很想去報.
我會覺得是.
正正可能像寬Sir剛才說的.
可能正正因為因應著我們那些場景.
而我們設立了這些主題.
這些題目.
但是.
在這些題目.
可能其實裡面說的神學.
都是我們最基本最核心的那些東西.
可能我們在教會學的都是這些東西.

$^{1121}$但是只不過是轉了一些場景.
其實大家再看看.
原來.
那些神學的訊息.
其實和我們今天是有關係.
其實可能讓我們更加知道.
究竟我們現在這個時候.
可能要怎樣走.
我.
也可能剛才聽到大家的分享.
就覺得.
我們其實都很容易落入一個位置.
就是我們聽了很多教授.
或者我們聽了很多講座.
其實我們經常都很想知道答案.
對於那種找不到答案.
又有些不安全感的狀態.
其實都明白.
我們其實經常都會很想.
很怕錯的情況.
但是都會覺得.
是呀.
好像剛剛這一篇.
可能這個分享裡面說的.
就是我們有些時候.
其實好像耶穌在福音書.
或者聖經裡面.
其實祂給我們的答案.
真的不會說是.
你要走一二三四五六.
有很多的規條.
或者是你一定有一個很準確的答案.
但是意思是一個很.
祂有一個原則.
可能或者祂已經告訴你.
答案是怎樣.
但是原來上帝真的會給我們.
有自由去走那些路徑.
而那些路徑其實是.
好像剛剛那樣說.

$^{1161}$有些心意或者有些上帝的話.
未必是一板一眼.
我們會很的確地知道.
但是其實.
你有這個心去追隨的時候.
或者是.
剛剛那樣在碟碰碰.
去試去撞的時候.
你知道有些路是走不通的.
有些路是可以走的.
那你繼續走下去.
看看是怎樣.
是呀.
多謝你.
後面有一個.
我再說一點.
多謝你看到我的心思.
還有回應剛才.
有些內容我忘記了說.
就是.
在教育上.
我們受教了很多年.
讀了很多年書.
有時候某程度上我們迷信了課程.
和迷信了教育.
為什麼這樣說呢.
但是對於一個人.
有時候有些東西不是學校教你的.
不是課程教你的.
如果是一個人的本意.
你會發覺最核心的東西就是良善.
守時.
認錯.
盡責.
這些待人處事的時候.
你會發覺都不是學校教的.
學校只不過提供一個場景給你.
很多這些東西.
在你三歲讀幼稚園之前.
你爸爸媽媽已經教你.

$^{1201}$錯了要認.
要守時.
或者要待人怎樣.
見到人會叫人的禮貌.
全部都是和你的生活場景.
和你最有密切關係的人.
去做那件事.
去教你那件事.
都不是學校教的.
學校只不過延伸那條場景.
延伸多些空間.
或者是調理化.
或者規範了那件事.
什麼叫做好.
這些都不是.
所以我們Flow Church.
去說這些課題的時候.
不是要redefine什麼叫做.
基督徒的syllabus.
也不是redefine什麼叫做.
good or bad.
反而是告訴你.
我們要用我們的能力.
去在我們的場景.
我們接觸的人當中.
去展現基督是什麼.
或者是展現基督信仰是什麼.
這個是重要的.
所以多謝你欣賞我們的課程.
後面是不是有一個.
終於聽完第一季的神學百課.
多謝John和潘Sir.
給我們上了總共八個月的神學百課.
我覺得這八堂非常有意思.
因為我第一次聽.
真的有人會將神學的東西.
基本的信仰確認.
那些神學的東西.
是普及到我們這麼多信徒的群體當中.
至於其實我見前面有幾位觀眾.

$^{1241}$都有提到的一些.
關於現在教會普遍的問題.
太過直接focus.
要去找答案那樣.
我會覺得對我來講.
問題會出於.
有時我們問的東西.
是連方向都錯了.
換言之是說.
有時我們太容易.
跳過了過程直接去到結論.
當然我會相信一些.
當這些東西是真的.
比如說其實很多信仰的概念.
基本的概念.
例如罪,救恩.
罪,救恩或者信心.
甚至是基督耶穌神真理本身.
無論在這八堂來說.
給我們知道一些.
關於這一方的概念.
其他前人神學家有什麼詮釋.
還是對我個人來說.
重新去看聖經.
其實發現是.
其實很多時候我們.
我們之所以這麼依賴去找答案.
是因為.
其實重點不是在答案本身.
因為答案是什麼其實不重要.
要走怎樣走其實不重要.
真正要知道怎樣走.
其實不是要靠這個.
而是靠的是.
你要找到你跟隨著什麼.
你為著什麼去做.
你清楚你跟隨著什麼.
在我們的情況下.
你跟隨著耶穌後繼是什麼樣的人.
又是什麼樣的神.

$^{1281}$你跟清楚.
自然就會.
你搞清楚整個的印象.
你就大概知道你應該怎樣去跟隨.
所以其實是.
所以我覺得.
潘Sir有幾個重點.
都是很反映我現在這一刻.
信仰教會的想法.
就是有時我們相處之間.
其實太多時間是專注在.
我覺得太少去做一些.
信仰的反映.
明明這是我們.
將我們賴以為生的一個必要的事.
但偏偏我們太多時間.
很快就跳過了這些是怎樣被正義.
然後直接就當它是一個.
好像當成了金波玉律.
但很少我們去.
重新想想去找找.
究竟這些是怎樣.
慢慢怎樣形成我們這些信念.
其實這個形成的過程.
是先是重要的.
而耶穌其實都說得出.
耶穌祂是當初所.
祂主要對人做的.
不是說答案.
要怎樣跟隨.
去過一個聖潔生活.
而是讓他知道這個世界的真相.
真理就是世界的真相.
這個無論放在什麼年代.
都是一樣不會變的.
所以其實重點在於.
怎樣過的時候.
就是要找回真相.
而這個求真的.
這個風氣的特質.

$^{1321}$在這個年代.
其中一個基督徒最需要重新培養.
就是群體裝備的特質.
所以這些百科其實.
定位都是一些基本的東西.
因為都是.
好像一些教會的基本百科.
得來其實又不是基本的東西.
因為都有一些.
重塑我們這麼多年想做基督徒.
因為大家都不是初信.
所以是讓大家去.
一些不是初信的人.
學一些初信的東西.
但那些東西其實都是.
正正是大家可以重新去學習的東西.
剛剛我才聊的時候.
我就想到我下一季叫什麼名字.
叫做神學百科之No Way Home.
你吃得很快.
其實都是預告.
我們下一季.
我們是想做一些.
關於海外基督徒.
和留德留基督徒.
兩批人不同的處境.
所以小初步在想.
下一季會分開兩個分支.
一班就是給海外的.
離開香港的基督徒.
還有怎樣可以去家裡.
怎樣可以.
那些課程都希望能夠可以去.
對於那個處境的幫助.
重塑一些所謂流散神學的東西.
對於留下來的基督徒.
繼續去延續一些.
再實踐多一些的東西.
所以如果這八課是基本的話.
下一個季度的東西.

$^{1361}$就是基本以上.
更加具體和更加.
要求的東西.
會去想一些.
剛才都有幾個關於.
怎樣做或者怎樣思考.
我自己都沒有想過.
我做信徒培訓.
或者課程設計的時候.
這些弟兄姊妹.
為什麼會有這樣的思路呢?.
我都認同.
楊醫他的神學反省.
環教會都很著重救贖神學.
很著重救贖進路.
得救或者是成聖.
或者在信徒成長過程當中的操練.
是否達標.
或者是否做到.
很著重那個訓練.
反而忽略了創造神學的平衡.
創造神學其中有幾個關鍵.
對於我覺得現在的基督徒來說.
或者作為一個結語的時候.
我覺得都可以給弟兄姊妹去想想.
創造神學其中一件事就是欣賞.
欣賞你現在可以做的空間.
或者你擁有的東西.
另外就是承擔.
即是有一樣東西在這件事當中交給你.
你能不能夠承擔到呢?.
第三件事就是護理.
你怎樣可以管理好你自己擁有的東西.
這都是創造神學會提醒我們.
我們現在無論你在海外或者在香港都好.
你做基督徒身份.
已經不是在說你是否得熟那件事.
是你現在在這個環境空間.
你怎樣去欣賞你可以擁有的空間.
你怎樣去承擔你可以做的東西.

$^{1401}$你怎樣去護理你可以擁有的資源.
這都是.
我覺得香港做神學反省的時候.
我自己常常都覺得.
不是對與錯的問題.
是有沒有做的問題.
我覺得都是希望在當中和大家一起去思考.
好啊.
下個場景.
過了八個月.
那些翻身位就差不多了.
多謝各位參與.
希望我們可以下季再見.
拜拜.
\newpage



\section{撒母耳記下 11:1-12:31-20230225}
\label{sec:lsdGk_BkHa8}
\textbf{【流堂崇拜】天光請開眼|撒母耳記下11\_1-12\_31|20230225 [lsdGk-BkHa8]}
\newline
\newline
連結: \href{https://youtube.com/watch?v=lsdGk-BkHa8}{\texttt{ https://youtube.com/watch?v=lsdGk-BkHa8}} ~~~~ 語音日期: 2023-02-25 
\newline
\newline
\hyperref[sec:4Dll86a7b18]{\small{< < < PREV SERMON < < <}}
~
\hyperref[sec:index_chronic]{\small{[返順時目]}}
~
\hyperref[sec:index_scriptual]{\small{[返順卷目]}}
~
\hyperref[sec:VfT5ldcLjqQ]{\small{> > > NEXT SERMON > > >}}
\newline
\newline
$^{1}$(麥美娟: 請問你對於「同工」的看法是怎樣的?).
剛才我自己也很投入.
因為勁霸隊實在太肉緊了.
我自己也是一個經常很肉緊的人.
所以我的同工經常都很喜歡看我說話.
說話很肉緊 做事很肉緊.
我玩遊戲也很肉緊.
有些同工在搖頭.
我看到這個講題.
我想起一次跟牧者玩過一個類似狼人殺的桌遊.
有沒有人是未玩過的?.
狼人殺.
未玩過的 旁邊的人教一下他.
又未玩過.
一會兒找人教一下他.
類似是那些.
分好人和壞人的陣型.
有些人做壞人 有些人做好人.
在最緊張的時候.
那一次最緊張的時候有個警示出現了.
這個警示是從一位牧者的錶而來的.
他的錶響起.
就說「現在是九十分貝了」.
因為現場實在大家太肉緊了.
大家不斷在叫「是九十分貝」.
這九十分貝都主要來自他旁邊的那位牧者.
那位牧者就很肉緊地叫.
「他?一定是壞人啊」.
「你剛才看見他把尾巴挾起來嗎?」.
「喂!我們才是一隊的」.
「一隊啊」.
「你現在想揭穿我?不是吧?」.
「我剛才做過甚麼 你不是看不見我的吧?」.
「我是好人來的」.
「你相信我嗎?」.
「相信啊 有些人相信」.
但可惜那位叫到九十分貝的牧者.
也有人不相信他.
其實是…不要揭穿他了.
因為無論現實或遊戲.

$^{41}$很多時候都真假難辨.
今天我就帶大家參與一場宮廷版的狼人殺.
待會我們閉上眼.
不用怕的 閉上眼.
不用怕的.
閉上眼一起感受一下光明和黑暗的角力.
魔法.
上去吧.
請不要看著.
我們一起閉上眼.
天黑請閉眼.
這一刻我們來到一個宮廷.
天黑了.
有一隻獵物就在狼人的眼前.
狼人很想殺牠.
但很奇怪.
這個狼人對獵物的態度非常友善.
非常關心.
當獵物退下.
狼人就開始行動.
他寫了一封信.
他寫這封信交給他的下屬.
他指示他這樣做.
「你和我派他去戰場裡面最危險的地方」.
「然後你們就撤退」.
「讓他被敵人自然地殺死」.
狼人的下屬收到這封指令.
他心想.
「大家都撤退」.
「只剩下他一個」.
「這不就很明顯了嗎」.
「所有人都知道他們是狼人」.
於是他就將獵物和周圍的士兵.
都派到最危險的戰場上.
天光請開眼.
大家可以打開眼.
這次死的是烏莉亞和幾個士兵.
我們一起讀《薩姆二記》下十一章十六至十七節.
有交叉的部分可以暫時跳過.
好 預備開始.

$^{81}$「約阿偵測城的時候」.
「知道敵人那裡有勇士」.
「就派烏莉亞到那地方」.
「城裡的人出來和約阿打仗」.
「僕人中有幾個士兵被殺」.
「嚇人烏莉亞也死了」.
大家都很熟悉狼人殺.
知不知道致命傷是甚麼.
通常死是怎樣死.
(死是死).
是 有些人聽到了.
「信錯隊友」.
「你以為隔壁那個人幫你」.
「他可能真的幫你」.
「但可能是隔壁那個人在害你」.
「認錯隊友的烏莉亞」.
「她望著曾經和他出生入死的隊友」.
「誰知到最後」.
「是派她去戰場最危險的地方」.
「想她死的那個」.
「但我們好心知肚明」.
「背後還有指使她的那位主腦」.
「誰交信給約翰」.
「誰就是主腦」.
我們再一起讀.
《撒謬義記》下十一章14至15節.
「早晨,大衛寫信給約翰」.
「交烏莉亞親手帶去」.
「他在信裡寫著說」.
「要派烏莉亞到戰爭激烈的前線去」.
「然後你們撤退離開她」.
「使她被擊殺而死」.
原來主腦就是當時已經成為了軍王.
以色列軍王的大衛.
還要很特別.
你看不看到.
那封信交給了誰.
原來這封死亡密令.
是先交給烏莉亞.
她從來沒有想過她手裡握著的.

$^{121}$是一封殺害自己的死亡密令.
不過就算她揭破又如何.
可不可以改變呢.
要殺她的是以色列的掛事人.
烏莉亞暗地裡被殺害.
其實都是源於一件暗地裡發生的事.
請聽我讀出《撒謬義記》下十一章.
原來之前發生了一件這樣的事.
在有一天黃昏的時候.
大衛就起床.
在皇宮上的平頂散步.
他在平頂上看見一個婦人沐浴.
這個婦人非常漂亮.
大衛就派人打聽她是誰.
有人就說認識的.
她是米爾年的女兒.
赫人烏莉亞的妻子拔士巴.
大衛就派使者將婦人接回來.
來到大衛那裡.
這個婦人月經剛剛結症.
大衛跟她同寢.
他就回家.
那個婦人之後懷孕了.
派人告訴大衛.
我懷孕了 有了嬰兒.
雖然這件事我們看到.
黃昏的時候即是入黑.
在晚上的時候.
將軍的老婆被人召入宮.
都是大件事.
在經文沒有提及過.
她被人包著頭 蒙著頭.
很鬼鬼祟祟的.
沒有掩飾.
但是相信除了大衛的婦人.
都有其他人看見.
為什麼烏莉亞被人搶了老婆.
都懵然不知.
都不走出來說一句話.
這是因為當時烏莉亞在哪裡.

$^{161}$她被人派去打仗.
就是被大衛派去前線打仗.
而一向和以色列人出入沙場的大衛.
這次沒有去到 一起征戰.
大衛遠離了刀光劍影的戰場.
但是他自己留在宮裡.
都上演了一個內心的小劇場.
一個天使與魔鬼的小劇場.
去計算著自己如何保住光明的形象.
遮掩了他犯奸淫的罪行.
為求目的達到.
大衛就急召了烏莉亞回來.
急召她回來之後.
他還用了一個招數.
發動了一個招數.
就是灌醉她.
灌醉她.
讓她以酒亂性.
她就會回家與老婆親近.
那肚子就沒有人知道是誰.
不過烏莉亞有很多理由.
其中一個理由就是.
大家都在打仗.
她如何可以回去與老婆親近.
堅決回去.
這就令大衛想下一個計謀.
就是借刀殺人.
我們拉遠一點 看整個局勢.
剛才說大衛是狼人的陣營.
他的對手是誰.
是否在戰場上的那一班人.
大衛曾經強調過.
以色列軍是永生上主的軍隊.
現在他將自己軍隊的勇士除去.
其實就是轉投了黑暗的陣營.
整個局勢是怎樣的.
他現在與上主對抗.
站在上主的對面.
形勢有這麼大的改變.
我們很多時候看這段經文都會說.

$^{201}$一定是色字頭上一把刀.
在座的男士有沒有人用這段經文來引誡你們.
小心一點 可能漢奸就出事了.
不要看那麼多東西.
這個又不能全錯.
很複雜 你們問那些藍屋姐.
這個色字頭上一把刀.
是不是就是這段經文的總結呢.
這把刀確實是.
確實是變節的開始.
但是令到他陰謀越發越大.
背後有更重要的原因.
一個關鍵的因素.
這個原因就是一個字.
「派」字.
這個「派」是猜拳.
是與一個人連繫了.
可能我派你去幫我做一些事.
你的上司會派你去出差等等.
在整段經文裡面.
大衛不斷做的就是猜牌.
大衛猜牌約壓.
派約壓率領神僕和以色列的士兵去打仗.
去對前線打仗.
他派人打聽婦人是誰.
派人去接那個婦人回來.
之後他的計謀想開始的時候.
他又派人叫約壓命令布里亞回來.
派布里亞去送死.
派人將拔士巴接入宮裡面.
一連串猜牌的行動.
大衛不斷行使他君王的權力.
猜牌不同的人達成自己的陰謀.
有權力的人才可以猜牌.
而大衛做每一個惡行之前.
都是先猜牌.
他為了掩飾而猜牌.
為了姦淫而猜牌.
他為了殺人而猜牌.
他的惡行是因為他濫用猜牌這個權力.

$^{241}$並且忘記當初猜牌他管理國家的是誰.
當初猜牌他管理以色列國的是上主.
有一次我和一位街坊叔叔.
看著街上聊天.
聊開一些社會議題.
他很教我.
來看看這裡誰擺什麼勢力.
講到最後講很多社會議題.
最後他就咬牙切齒地說.
權力越大貪污一定越大.
阿妹你知不知道.
我就咬牙切齒地說.
權力死人腐敗我們常常聽.
我們基督徒是否迴避了它.
我不要有權就行了.
我形容很簡單.
不要那麼多權力就行了.
如果是這樣基督徒就很難搞了.
因為你不能升職不能做老闆不能請工人.
因為你都有機會派人去工作.
生孩子都很大試探.
今天有小朋友在.
一不小心有些父母可能會用權力操控子女.
做他想做的事走他想走的路.
其實從聖經時代直到今天.
權力都存在在不同的關係裡面.
就算是剛才我們看不斷猜派人的是大衛.
我們看經文其實約翰和拔士巴都曾經有猜派.
曾經都運用他們的力量權力去猜派.
就算當時體為很弱勢的拔士巴.
他都派人去通知大衛.
而在生活裡面我們有時會收到不同的角色牌.
我們在不同的場合就是不同角色.
我們都要思考怎樣去運用權力.
先合乎上主的心意.
我自己曾經收過一個角色牌.
就是什麼呢.
很厲害的.
可以影響人的成長.
在還沒讀神學之前.

$^{281}$我是一位幼稚園老師.
其中有一位相熟的家長和他聚會的時候.
他經常都說當年一件事.
就是當年我面試他的兒子的時候那件事.
他就跟我說.
黃老師你知不知道.
那時候我多麼徬徨.
剛剛搬屋來到這一區.
到處去找學校.
誰知道每間學校都說我的兒子.
話又不懂說.
多句.
很多東西都不懂.
英文又不擅長.
很難追到進度的.
每個人都說回家等通知.
沒有一間學校肯收他.
我來見你.
已經打好書數了.
我自己都說我的兒子很多東西都不懂.
誰知道你跟我說.
你說慢慢教吧.
教小朋友是老師的責任.
你放心吧.
我每次聽完他說.
我都被當時青春的自己所感動.
原來我年少這麼有抱負.
感動啊.
你都感動了.
因為上主當時給我有些微的影響力.
我可以決定收不收他.
當時我就收了一位.
被很多學校放棄.
很多準則裡面都不會選他的小朋友.
但我們知道上主的眼睛不是這樣看.
信徒要思考的是我們怎樣.
有時限制自己的權力.
以免做了上帝恨惡的事.
有時我們又要好好運用我們的權力.
可以去做上主喜悅的事.

$^{321}$特別是我們上教會的人.
很特別的.
教會這個場景.
這裡有些不同.
可能我們以前上教會.
很多權力.
你做組長有權力.
有導師的權力.
招待都有權力.
我們怎樣運用權力.
都是上主看重的事.
有一次我就帶一位無家者.
去到某教會參與主日崇拜.
很好.
期了一段時間肚子餓.
他又肯跟我去.
我們約了就去了.
怎料那一次就吃閉門羹.
這個閉門羹嚴格來說.
只是給那位無家者吃.
他沒有拒絕我進去.
理由他們給我的.
就是我們這個教會沒有這個事工.
那裡崇拜不適合他.
甚至乎他好像有些味道.
可能會影響到會友.
試想像如果當日站在門口的你.
你有權接待或者不接待他.
你又會怎樣回應呢.
你的回應可能只有你知道.
但是大衛他面對事情的反應.
大家都看到.
宮裡的人看到.
他的親信都看到.
其實都是有些重要的蛛絲馬跡.
被人看穿了他的底牌.
其實在過往他有幾次運用權力的情況下.
我們做一些比對.
就發現他整個人都變了.
他的態度.

$^{361}$他的反應有很大很大的變化.
好像換了一個人.
我們一起整理一下.
其中一件事.
上次港獨都有分享過.
大衛對著米菲波切的時候.
米菲波切是腳都震了.
他不知道大衛怎樣對待他.
但是大衛第一句跟他說的是什麼.
你不用怕.
一個君王跟他說你不用怕.
之後還派人照顧他.
常常跟他吃飯.
非常之關顧.
第二個事件我們可以看到.
昔日的大衛是怎樣呢.
這件事發生是因為大衛差派一些神僕去慰問別國.
但是人家不領情.
不領情之下他們做了什麼呢.
他們就將神僕的鬍鬚剃了一半.
又割斷他們下面的袍子.
令他們露出下體.
然後就放他們走.
當這班神僕被人用一個好像收獄戰犯的方式對待的時候.
大衛即時的反應是派人迎接他們.
派人迎接這班受獄的士兵.
讓他們有安身之所.
還可以讓他們慢慢來.
等你們的鬍鬚長了.
有體面.
你才回來面對大家.
這兩件事大衛都有行使他軍權的力量去幫助弱勢的人.
而從他之後的行動和反應.
更加看到他怎樣去看待人.
他對待米菲波切的時候.
當日的他.
有上主同在的他.
細心安慰.
願意去照顧.
不分你我.

$^{401}$不會覺得自己是君王.
就不會同桌吃飯.
當他對著那班受獄的神僕.
有上主同在的他.
體恤照顧他們.
維護他們.
尊嚴.
非常之體貼.
但是當他轉頭了黑暗的陣營的他.
對烏尼亞和那班士兵.
他說了什麼.
他說了一句.
刀劍有時會吞滅這個人.
有時會吞滅那個人.
即是說白一點.
打得仗有時會死這個.
有時會死那個.
他竟然對人命無動於衷.
還是他熟悉的戰士.
假仁假義的他.
還叫人不要那麼傷心.
你認為他身邊的人.
會不會察覺到他這個變化呢.
權力不單止令到大衛對罪埋沒.
對上主視而不見.
更加令他看東西不同了.
看什麼不同了.
他看人不同了.
他覺得自己高人一等.
自己的命比別人優越.
他輕視其他的生命.
這種看人的角度.
其實都出現在我們的社區.
有時我和清潔工聊天的時候.
都會聽到他們很多在職場被欺壓的狀況.
在他們的角度.
他們說沒辦法.
上級有權.
甚至一個懂投訴的路人都有權.
他們經常被路人的權力所影響.

$^{441}$有時就算是因為別人攬權.
而令到他受到無理的對待或要求.
他們都沒能力或不懂得怎樣拒絕.
甚或乎是因為害怕.
要承受一些後果.
所以他都不可以為自己發聲.
有個清潔伯伯就說.
我們都是無權無勢的.
他們被人輕看.
有時在他們的言談之間.
他們覺得是必然.
有一次新年那段時間.
我帶兒子和二公去探望一位清潔伯伯.
我們在公廁門外聊天.
聊得很開心.
聊天的時候.
有位工友都認識.
走過來想捉住兒子的手.
玩玩.
那位伯伯就很緊張.
嚇住他.
喂!不要搞.
不行的.
接著我就說.
捉住兒子的手.
開心玩玩.
他就說.
不好.
我們這些人.
不方便的.
我們這些人.
我聽完之後.
非常不開心.
原來有時社會將他們輕看.
自己習慣被輕看.
自己都會輕看自己.
當然我們和他們的接觸.
就是告訴他們.
什麼是正常的關係.
沒有誰比誰更高尚.

$^{481}$輕看人命的大衛.
在這一場宮廷版的狼人殺入面.
其實沿路都有很多很多指示給他看到.
其實很多指示都可以告訴他.
你可以由黑暗那邊走出來.
走回上主的陣營裡面.
我們看看那些指示是什麼.
第一個指示就是約季.
當烏尼亞拒絕大衛的時候.
他劈頭就說起約季.
約季是代表神的同在.
是神和以色列人納約的象徵.
大衛本身都非常重視約季.
非常重視神與人同在.
這番話是出自一個外邦人的口.
一個歸順了以色列國的外邦人的口.
其實這個指示已經足夠.
讓大衛感到羞愧和醒覺.
第二個指示就是烏尼亞的名字.
雖然他是外邦人.
但他有希伯來的名字.
他的名字的意思是上主是光.
所以大衛每一次與烏尼亞的對話.
看到烏尼亞.
他應該都會想起上主就是光.
給了機會和時間.
但大衛都無視.
不過上主一直都看到.
而且他看到大衛是以色列人的領袖.
如果他轉投了黑暗的陣營.
後面那班以色列民會怎樣.
可能全部都會一起跟過去.
所以上主為了挽回他和那班子民.
他就出了一張牌.
這張牌是甚麼牌呢.
很厲害.
這張牌叫做牌先知.
他就派了先知出去.
他的功能是甚麼.
先知的功能就是揭開大衛藏在籠底的惡行.

$^{521}$令他見光死.
不能再扮君子.
先知更宣佈.
上主會在日光之下宣佈報應大衛.
讓他承受惡果.
認清主權在哪裡.
主權在上主那裡.
上主這一回合很成功.
當大衛的底牌被揭開的時候.
他都立刻醒一醒.
轉投回上主的陣營.
願意悔改離開黑暗.
今日我們生命的主權是屬於誰呢.
我們生命的主權是屬於耶穌.
是屬於誰呢.
屬於耶穌的門徒.
主的光一直會指引我們.
我們一起讀約翰福音三章十九至二十一節.
預備開始.
光來到世上.
世人因自己的行為示惡的.
不愛光.
倒愛黑暗.
這就定了他們的罪.
凡作惡的人都恨護光.
不來接近光.
恐怕他的行為被暴露.
但實行真理的人就來接近光.
為要顯明他的行為是靠神而行的.
耶穌進入黑暗的世界.
是要揭示黑暗陣營的底牌.
為那些被封鎖在黑暗裡面的人.
提供逃生的出路.
提供力量去面對黑暗.
我們接受光就可以離開黑暗.
最近耶穌的光都照著我.
去年有一個很多年沒見的幼教同學.
看到我兒子的照片就說.
探望你兒子吧.
很可愛.

$^{561}$但又有防疫措施.
總之約都約不成.
有一天剛才的同學A.
有一個同學B就說.
今天見到同學A.
他見到他發生的事之後.
就令我很想快點約同學A.
他說什麼呢.
他說今天收到同學A的卡片.
原來他做了校長.
我又看一看那間學校.
我們很多同學都做校長.
那間學校是我心儀兒子想讀的幼稚園.
同一個系列的.
當刻我自然就這樣想.
這次可以了.
找他去踏路.
一定容易很多.
其實真的.
很現實.
校長說一句不收你嗎.
沒理由吧.
可以有特權或者加分.
之後就無限FF.
我的兒子前途一片光明.
想得很遠.
在我想約他的時候.
我就想起當年我怎樣面試人.
當年那個滿腔熱誠.
剛才仍然在感動我的自己.
我可以運用權力去幫助弱勢的學生.
但今天當有機會來到的時候.
我竟然想借助別人的權力.
讓其他應該公平競爭的學生成為弱勢.
真的很真實.
可能我們每一天都面對一些生命的抉擇.
有時權力未必在我們手中.
但我們都受到引誘.
想透過穿針引線.
透過別人的權力得到好處.

$^{601}$或者遮蓋我們一些惡行.
當我們去擔當不同角色的時候.
我們有不同的看見.
我們可能擁有不同的權力.
又或者我們今天被別人的權力牽扯著.
令我們動彈不得.
但很願意上主的光成為我們生命唯一的指引.
成為我們的力量.
又願我們Flow Church.
我們的弟兄姊妹.
我們看見弱勢的時候.
可能弱勢就存在在你身邊.
有時可能我們看不見.
或者選擇看不見.
當他們受到不公平對待的時候.
我們能夠提供實際的幫助.
並且指引他們.
見到上主的光.
一會兒我都會有一些時間.
邀請大家祈禱.
你為自己祈禱.
你為身邊見到的弱勢群體去祈禱.
你為需要迴轉的人去祈禱.
我們一起去禱告.
耶穌你就是真光.
能夠認識到你.
能夠接受你.
我們的生命就可以離開黑暗.
這個世界有太多的誘惑.
罪惡充滿在我們周圍的時候.
每一刻我們都需要去選擇.
我們選擇就近光.
接近光.
還是離開上主的指引.
求上主去指引我們.
這一刻.
你都圍到我們的城市去禱告.
會不會有一些群體.
是你特別有感動.
祈求主讓他們在.

$^{641}$可能沒有那麼多權力.
真的很客觀的限制.
令到他們成為弱勢的時候.
他們怎樣可以看見上主的光.
他們怎樣可以得到實際的幫助.
你為一個有感動的群體去祈禱.
我們又為自己去禱告.
今天你的角色是什麼.
你的狀態是什麼.
你已經認識了耶穌一段時間.
你可能離開了祂一段時間.
又或者你還不是很認識祂.
還沒有相信祂 還沒有接受祂.
無論你在什麼階段.
你都可以禱告.
求上主讓你看到光.
讓你看到現在身處的黑暗.
讓你看到那些指示.
祈求上主讓你有力量.
可以走出來跟隨祂.
或許有一些自己不為人知的話.
說不出任何人聽.
很需要去認罪.
這一刻你都跟上主說.
我們為自己的生命去祈禱.
親愛的天父.
我們懇切祈求你.
幫助我們在黑暗當中.
我們更多看到上主給我們的指示.
更多看到你的光去引領我們.
幫助我們.
更多讓我們經歷出黑暗入光明.
我們很多的試探.
很多的試煉.
求主都給我們力量.
讓我們看見.
上主你想我們看見的.
愛你所愛 恨你所恨.
求主憐憫我們.
我們的祈禱交託.

$^{681}$是奉主耶穌的聖名致以求.
Amen.
\newpage



\section{}
\label{sec:VfT5ldcLjqQ}
\textbf{《致餘民及流散者:給香港基督徒的神學八課》第二季第1課|20230227 [VfT5ldcLjqQ]}
\newline
\newline
連結: \href{https://youtube.com/watch?v=VfT5ldcLjqQ}{\texttt{ https://youtube.com/watch?v=VfT5ldcLjqQ}} ~~~~ 語音日期: 2023-02-27 
\newline
\newline
\hyperref[sec:lsdGk_BkHa8]{\small{< < < PREV SERMON < < <}}
~
\hyperref[sec:index_chronic]{\small{[返順時目]}}
~
\hyperref[sec:index_scriptual]{\small{[返順卷目]}}
~
\hyperref[sec:dLJdySFiu9c]{\small{> > > NEXT SERMON > > >}}
\newline
\newline
$^{1}$(拍攝中).
(廣播聲).
(攝影機搭載音樂).
我們是時代之子.
是被上帝揀選.
在這個年代見證耶穌的基督徒.
我們身於這個年代.
被這個年代分散.
有人流散到海外尋覓理想.
有人繼續在本土奮鬥下去.
但又如何?.
基督徒仍然需要作基督徒.
香港人仍然是香港人.
究竟上帝的旨意如何?.
我們應該如何生活?.
我們應該如何為主而活?.
無論你身處何處.
只要你是香港人.
邀請你和我們一起思想這個流散年代的信仰.
致愚民與流散者.
給香港基督徒的神學百科.
(音樂結束).
各位弟兄姊妹晚安.
很高興今天和大家一起上香港基督徒的神學百科第二季.
我小時候看《四十名穿梭機》.
有一首歌叫做.
我也能做到.
所以今天.
這一季我會做地勤.
第二季我們會延續第一季.
我們課程系列是全聖教的基因.
希望能塑造流淌群體的屬靈信仰.
最基本的原則.
第一季是基本.
第二季我們會延續到一些處境.
特別是在我們這幾年流散的情況.
所以第二季我們會在流散的情況中.
一起思考我們的信仰.
今季我們也很特別.
每一季的應用都會分開兩邊.

$^{41}$一個是愚民.
香港的基督徒如何面對我們的課題.
如何能夠在香港繼續盡心信仰.
另一方面是我們在海外流散的弟兄姊妹.
所以今天我們每一課.
同一課題都會同樣地.
讓兩邊弟兄姊妹去思考信仰.
所以一個同閉.
兩邊的應用.
第一課我們叫做.
從此我們分離愚民與流散者.
我想這是一個引言.
去討論一下流散的課題.
流散的課題.
可能大家都面對著處境.
如果我們回看我們的字.
可能大家都見過的字.
Diaspora.
這個字是解作流散.
或是外地移居移鄉的人.
這個字在學術上有很多層面應用.
無論是人類學,社會學,歷史研究.
信仰都是一樣.
所以Diaspora這個字是指本地外國人的情況.
無論是香港的南亞裔.
或是美國的華僑等等.
都是一個很重要的研究.
所以當我們回看聖經的時候.
Diaspora這個字是基督教神學中.
一個很重要的字眼.
實際上以色列文的歷史.
從來都是流散歷史.
所以當我們研究Diaspora這個字眼.
不單單是我們香港人的情況.
我們香港人的民族.
我們去不同的地方.
飄流到海外.
但這個字本身是我們信仰中一個很重要的課題.
所以我們就一起來看看.
我們一班香港人.

$^{81}$我們去思考我們的處境.
同時也去認識聖經中.
我們有關流散的信仰.
這就是我們今天的目的.
要去研究這個字.
我們發現它本身可以解作兩個字眼.
這個字是一個來自希臘文的字眼.
叫Diaspora.
但如果你看舊約的時候.
舊約的希臘文的本.
就是74頁本.
正正就是用Diaspora這個字去翻譯那個秘魯的字.
如果你明白那個基礎.
74頁本來就是一班流散了猶太人的年代寫成的.
當時猶太人已經分散到不同地方.
主前大國是幾百年時間.
他們重新用希臘文.
重新去認識和詮釋他們古道的信仰.
所以想一想.
一班用英文的人.
去明白香港一樣.
所以是一個第二代第三代的人.
他們用希臘文.
去翻譯希伯來文的字眼.
所以Diaspora本身是一定程度的流散處境.
以色列人是一班離開了他們的本族父家的地方.
在整個帝國裡面.
從而用希臘文去明白摩西五境.
或者以前的阿伯拉罕信仰.
所以這個字眼是直接用希伯來文裡面.
秘魯的Lut這個字眼.
所以這個字眼可以解作.
秘魯或者散居或流散的字眼.
其實可以有很多研究.
如果你去探討Diaspora這個字眼.
我們可以知道.
本身猶太人.
為什麼說Diaspora是一個很重要的信仰根基呢.
因為本身猶太人的歷史就是流散歷史.
無論是從阿伯拉罕開始.

$^{121}$阿伯拉罕要離開他的本族父家.
漂流到海外.
一個這樣的處境.
到後來秘魯的斯里文.
都是Diaspora的處境.
然後在耶穌的年代裡.
他們散居在整個的帝國裡面.
然後整個的中世紀.
就是中世紀歐洲.
他們都仍然散居在整個的不同的國內.
到二戰的時候都是這樣.
所以猶太人的歷史本身就是Diaspora的歷史.
一本書叫Encyclopedia of Jewish Diaspora.
是千頁厚的.
只是說猶太人的Diaspora的輸入.
都足夠成為一本百科全書.
所以是一個非常流散的.
猶太人是第一名.
第二名可能是印度和中國人.
所以我們看聖經裡的時候.
Diaspora本身是一個非常重要的舊學課題.
當然大家讀聖經的時候.
如果你上了教育課.
都大概知道一些基本的知識.
就是猶太人和以色列人.
他們是曾經被擄.
他們首先是在北國.
在公元前721年被亞述國擄去.
然後南國猶太.
他們是在巴比倫.
公元前586年來到去被擄.
所以被擄的歷史本身就是一個流散的事情.
記住Diaspora是解作.
被擄也可以這樣解釋.
當時他們這班人離開他們自己的地方.
有很多不同的原因.
首先就是被擄.
就是被人把他們抓去監獄.
也可以是一些生活上的原因.
有些人沒有被抓去.

$^{161}$純粹因為要工作.
就要去到不同地方居住.
這也是當時猶太人的情況.
有些人是發覺自己的地方.
經濟或城市被毀.
就要離開家園.
所以以前的被擄.
也可以是一些流散的成份.
他們不是純粹被人抓去這麼簡單.
可能也是因為不同的原因.
就離開了自己的家.
所以如果我們去留意一些舊約裡面.
希伯來文裡面.
一些被擄相關的字眼是很多的.
就是Diaspora是一個希臘文.
一個泛指很多時候流散被擄的希臘文.
但我們回看希伯來文的時候.
其實有很多不同的字眼.
都是說差不多的意思.
可能大家最熟悉的就是Galut.
我們有一個IG專頁叫Galut.
就是流散者們.
Galut這個字是解釋什麼呢.
原來這個字不是一個技術性的字眼.
這個字不是純粹技術性的.
是在說一個以色列人被擄的情況.
而是一個因著移動導致的視線改變.
這個很特別.
就是說一個因為被擄走了.
所以就看不到.
這樣很簡單的意思.
所以它可以解釋為一個位置上的轉移.
就是被人Galut這樣做.
就是被擄走了看不到.
所以它也解釋為一個消失或者失去.
總之你看不到他.
這個就是Galut的意思.
所以它最基本的意思不是一個純粹的秘魯.
不是巴比倫或者亞述國的秘魯.
而是一個很簡單的因為不在所以看不到的情況.

$^{201}$接著第二個字就是解作Shabah.
Shabah這個字是解作戰爭下的俘虜.
古境東裡面的戰爭是很流行的.
總之我打完一場仗的時候.
我就自然會俘虜對方的士兵.
所以這是一個很慣常的近東的做法.
但是當面對一些很大型的戰爭的時候.
一群人就被擄走了.
那就不是解作戰俘那麼簡單.
如果香港輸了.
不是那些士兵被人俘虜.
而是整個香港人都被人抓了的話.
那就叫做Deportation.
一個很大型的俘虜的意思.
所以Shabah這個字是解作一個這樣的意思.
就是在戰爭下的移居的人被人抓了.
所以是一種大規模的遷徙.
因為戰爭的緣故.
因為戰敗了.
所以就要離開家園被人抓了.
這也是和某些機密裡面和我們的自由有關係.
就是一個救贖或者賣贖的原因.
一個經濟上的賤額.
是這樣的意思.
所以另一個字就是這個.
就是Shabah這個字.
是一個戰爭下的俘虜的意思.
第三個就叫做Nada.
Nada這個字是講述物理上的移動.
所以看到很多的聲音裡面都有很簡單的字.
譬如拋.
Wing.
《九龍生命記》第十個章第五節裡面.
拋來拋去.
或者是Wing走了.
這樣都是一種物理上的移動.
Nada.
有一個譬如C篇第六二篇第五節.
就是拉下來.
都是一種動作上的移動.

$^{241}$另外就是有撞開的字.
所以原來當我們發現聖經裡面.
有Nada的字都是有時候被用來做一個流散.
或者是被擄相關的字眼.
意思就是人被人抓走了.
撞開了.
或者是拋來拋去.
所以這個就是第三個.
跟我們流散或者被擄相關的希伯來文.
然後就是分散.
Pasul這個字.
這個字其實是有一個水字在裡面.
其實就是散水.
所以當水散開了.
正正跟我們flow出一樣.
所以它的水是散開了.
無論是在一個真實上的水散開.
或者是在一種隱喻上都一樣.
例如泛藍都一樣.
洪水泛藍.
或者是水散開了.
都是一種跟流散有關係的字眼.
另外一個字眼就是戰爭蔓延.
戰火蔓延的蔓延.
正正是慢慢擴散開去.
都是這個意思.
所以你會發現獵王記上面.
有一個字眼叫入門四散.
就是用這個字眼.
就是這個Pasul這個字.
就是人們流散了.
另外當然有殺種的字.
就是散開了種子.
所以這個就是第四個.
有關流散裡面.
希伯來文裡面的字眼.
第五個很有趣的.
就是那個.
大色毒藥的字是什麼.
就是筲箕.

$^{281}$就是一個筲箕.
它就是一個筲箕的動作.
筲箕就是篩走東西.
篩走了那些人.
所以這個本身是堪披.
聖經裡面有堪披的字.
就散走了它.
最重要的經文就是這個.
神要埃及成為最荒涼.
最荒蕪.
最荒廢.
而且也要被擄.
散在列國.
那個原文裡面的散字.
其實就是這個筲箕的字.
筲箕的字.
想想伊斯利文.
是一個很生動的字眼.
上帝就用筲箕.
散開了.
篩走了伊斯利文.
所以這個就是第五個.
有關流散裡面的字眼.
非常生動的字眼.
筲箕.
動詞.
最後就是這個菩薩.
就是這個物件上的四散.
比如耶利米茲書所說.
伊斯利是打散的羊.
是被獅子趕出來的.
首先是亞述王將牠吞滅.
末後是巴比倫王來殺王.
將牠的骨頭截斷.
所以這個被人打散的羊.
就用了這個菩薩這個字.
來形容伊斯利文.
被亞述.
被巴比倫趕走的意思.
所以發現原來在舊約裡面.

$^{321}$不只是秘魯.
不只是Galut這個字.
也不只是Diaspora.
而是有很多不同的經文.
都是說相似的字眼.
就是筲箕.
分散.
或者是蔓延.
這些字眼其實都是嘗試.
來形容一個很重要的情況.
就是伊斯利文.
他們是分散了.
他們是住在不同的地方.
可能是因為秘魯的緣故.
可能是因為當時候.
秘魯的緣故相關的很多經濟活動.
迫著要在那裡做奴隸.
或者是要甘願離開自己的地方.
來生存.
這個就是當時候一些舊約.
我們看到的字眼.
為什麼要認識這麼多的英文呢.
我們發現原來.
所謂的秘魯.
或者流散.
我們問是不是一樣的字眼呢.
今天香港人流散到海外.
我們能不能夠用秘魯來形容自己的處境呢.
今天我們去將信仰.
去理解我們今天的處境的時候.
我們怎樣去理解自己呢.
我們是秘魯了的一個群族.
還是一個因為什麼原因而分散呢.
這個就是今天想和大家去談的課題.
如果我們去總結整本舊約的時候.
有一本書今天不拿出來說了.
是非常厲害的書.
非常之.
剛才那些字眼研究的結論.
就是這樣.

$^{361}$八百多頁的一本德文書.
他就去研究究竟.
Diespora這個字.
和Glut.
或者很多相關的字眼裡面.
究竟是一個什麼的理解.
舊約的人在經文裡面.
怎樣去理解這些處境.
作者就這樣有七個不同的階段.
或者是一個stage.
他將整本的舊約.
分成七個category.
來嘗試歸納出七個不同的狀況.
或者七個不同.
在秘魯流散的處境裡面.
他怎樣去理解自己.
甚至乎是一種沉迷.
有點像我們今天那些.
人遇見了.
哀傷的那五個階段那些.
大家看我們一起去吧.
第一個.
首先是秘魯作為一種上帝的審判.
這個大家都很熟悉的.
以色文秘魯或者流散.
因為他們做錯事.
他們因為被上帝懲罰.
因為他們是潑逆.
所以上帝就要他們秘魯.
這個是其中一個.
但不是唯一一個.
是一個很重要的category.
即是聖經裡面有很多經文.
都是嘗試來將秘魯流散.
看為一種上帝的審判.
因為他們做錯事.
所以他們就要流散.
所以就要被人去俘虜.
第二個.
秘魯作為一種無可避免的現實.

$^{401}$他們就發現這是一個.
不可以避免的情況.
無論是耶利米書的教導.
你肯定會流散.
肯定會被捕.
這是肯定會出現的事情.
即是發現原來他們開始要接受.
去接受這樣的處境.
這是一個上帝既定的事情.
你需要去面對.
才會預言.
大家必定會被俘虜.
這是第二個情況.
是一種第二個階段.
對於流散和秘魯的理解.
第三個.
甚至乎在科學經文裡面.
後來出現了一種盼望.
這本經文是盼望他們能夠歸回.
甚至乎是重視將要歸回的那種英雄.
都有經文的.
今天我們不展示經文.
因為太多了.
到了第三個階段.
這本經文開始會在秘魯裡面出現盼望.
知道上帝和華相會叫他們歸回.
我們能夠回到耶路撒冷裡.
安山等等的情況.
這是第三種層面.
開始不是純粹的審判.
而審判之後上帝會歸回.
會拯救.
仍然會幫助.
第四.
是一種流散者綜活的團聚.
這個叫綜活.
發覺很多先知書裡面的預言.
以色列文在綜活的時間.
能夠回到自己的家鄉.
能夠大家歸回上帝.

$^{441}$這個不單是一種能夠盼望的事.
更加是一種綜活的事實.
這就是第四種情況.
一種綜活性的團聚.
第五.
開始接受流散的處境.
最出名的經文是什麼.
就是那句.
你們為者城求平安.
你們開始要為自己在異鄉裡.
為那個城市求平安.
你們會住那裡很久.
所以你們要開始接受流散的處境.
買樓的買樓.
大家在那裡慢慢找工作.
慢慢來生活.
接受這個流散的處境.
這是第五種現實.
可以接受了.
第六.
甚至強調上帝在流散和秘魯中的同在.
上帝不單在耶路撒冷.
不僅僅在聖殿裡.
更加在秘魯的國度裡.
最明顯的就是以色列的書.
以色列的書在異鄉河裡.
看到上帝的榮耀.
是一個外邦的異象.
耶和聖帝在外邦的天空裡出現的異象.
所以強調上帝也會同在.
在秘魯當中.
最後.
更加強調流散和萬民救贖的關聯.
以色列成為了列邦萬國的祝福.
因為他們流散出去.
所以他們更加成為了上帝使用的工具.
來祝福萬國.
這是七個類別.
如果我們把整本舊約.
把所有的經文分類.

$^{481}$把剛才所說的字眼分類.
這樣就有七個不同種類的經文.
甚至是有一個層底.
最開頭的時間裡.
是純粹審判.
因為我們做錯事.
所以流散 秘魯.
慢慢地發覺 接受這個事實.
慢慢地有盼望 有應許.
慢慢地強調能夠綜合性的團聚.
然後開始要在那裡生活.
深刻強調上帝的同在.
最後 甚至覺得流散是一種祝福.
對其他人是祝福.
所以我們很籠統 很快地.
把整本舊約.
有關流散的資訊.
知道 所以發覺.
秘魯和流散.
不是絕對傳言是負面的.
不只是因為我們做錯了.
被人罰.
而是有很多很多不同的意思在當中.
這是第一個.
第二個就是我們看到的.
一個很重要的字眼.
叫做Shakina.
如果你在家的話.
可以一起讀.
Shakina 這個字眼.
這是一個很重要的猶太教的字眼.
中文叫做赦金納.
不過南部人讀Shakina算了.
Shakina 為什麼呢.
就是解作一個上帝的同住.
上帝住在那裡.
或者上帝休息在那裡.
所以上帝就是在世界上的同在.
我們覺得上帝同在不是很特別.
都是同在.

$^{521}$但其實我們今天將這個同住看得太藍.
因為上帝無處不在.
反正哪裡都在.
但是耶和華說上帝的同住.
是一個很不簡單的事情.
上帝住在哪裡.
上帝你可能聽過經文.
是在天庭裡.
或者在聖殿裡.
或者天上的天都不容上帝居住.
這些經文.
所以當以色列人強調.
超越耶和華上帝住在當中的時候.
其實是一個很不簡單的事情.
或者我們用聖靈來解釋.
聖靈的同在.
但是當以前的猶太人覺得.
耶和華的同住是一個很重要的字.
不是隨便來的同在.
上帝是否同住.
是一個不容易解釋的便宜的事情.
上帝在哪裡呢.
譬如經文有一個叫做.
在初學者的經文裡.
我要住在以色列人中間.
作他們的上帝.
他們必知道我是耶和華他們的神.
是將他們從埃及地領出來的.
我要住在他們中間.
我是耶和華他們的神.
這個就是上帝的應許.
當一群埃及人.
以色列人被人追趕的時候.
上帝是住在他們中間.
這個字其實是一個很重要的字眼.
因為上帝竟然跟著他們走路.
經過紅海再在曠野裡.
不斷跟著他們走.
所以這是一個很特別的字.
所以上帝住在這裡.

$^{561}$當然我們發覺早期裡.
我們用這個來梳理.
就是這個聖神巷.
當所羅門去見聖殿之後.
上帝就住在聖殿裡.
上帝在聖殿當中.
所羅門說.
耶和華神說我必住在幽暗之處.
我已經建造殿與在你的居所.
為你永遠的住處.
所以所羅門說這是你的殿.
這是你永遠的住處.
你就住在這裡.
所以問上帝在哪裡.
上帝就在聖殿裡.
這是一個很重要的Temple Theology.
強調上帝住在聖殿裡.
這個圖就是古代裡的理解.
上帝是藉著聖殿與世界去關聯.
當然你發覺後來所羅門也這樣說.
天上的天都不足以居所.
你住在天上的天.
你不是住在聖殿裡.
所以是很弔詭的.
沒錯聖殿是上帝的居所.
但後來也強調聖殿不足以成為他的居所.
這幅圖是這麼說的.
我們任何人都經過聖殿去找到上帝.
以色列人是怎樣.
以前被擄了.
仍然往回猶太地方去朝見上帝.
溫哥華是第一眼.
要往回香港的地方去祈禱.
所以發現原來這樣的聖神學.
上帝就住在聖殿裡.
我們不詳細說聖神學.
後面就更加精彩.
但當以色列人被擄後怎麼辦.
他們不是在猶太.
聖殿被毀了.

$^{601}$那怎麼辦呢.
上帝怎會住呢.
所以出現了這個Shakina.
雖然聖殿被毀.
但上帝的同在就轉移到一群人身上.
上帝不是住在神聖的殿裡.
而是住在以色列人的中間.
這是一個很重要的概念.
因為上帝不是跟著死物或聖物.
而是跟著人.
人流散到哪裡.
上帝就跟著去哪裡.
因為這樣的上帝.
所以我就住在他們的中間直到永遠.
上帝不再是天上的居住的居所.
而是在以色列人的中間的位置.
就是一個Shakina的概念.
所以在以色列的書中.
人子也說這是我寶座之地.
是我腳掌所踏之地.
我要住在這裡.
在以色列人的中間直到永遠.
上帝說我要住在你們當中.
直到永遠.
直到你們歸回回去的時候.
所以你會發現.
如果拉到很遠的時候.
這個是猶太教信仰.
當然斯列文猶太人.
他們在中世紀的時間裡.
仍然相信.
雖然他們不在猶太人居住.
在歐洲不同的地方.
但上帝就跟他們同在.
所以有這樣的流散的信仰.
所以發覺對猶太人來說.
聖殿神學是重要的.
不過更加重要的是流散神學.
因為他們在世上經歷了幾千年.
都是流散的歷史比較多.

$^{641}$他們住在外國.
但上帝就跟他們同在.
是這樣的信仰.
接著我們看到.
到了新約時間也是一樣.
原來耶穌的說話.
因為無論在哪裡.
有兩三個人奉安的名聚會.
我們就在他們中間.
這是耶穌所說的.
這個經文其實源自於.
猶太教的一句名言.
一句密西那格言.
有兩個人用托拉一起聚會.
我們就有席琴那在他們中間.
所以經文其實是來自猶太的席琴那概念.
耶穌就是那個席琴那.
兩個人一起聚會.
今天在英國,多倫多.
或者在不同地方.
只要一起聚會.
就有上帝的席琴那.
就有上帝的同住.
他就住在當中.
所以這個就是到了新約時間.
原來是借用了這樣的概念.
去創造席琴那的神學.
當然我們知道.
最終極的席琴那是什麼.
就是耶穌基督.
他就在我們中間.
這個住也是這個字.
都是同住.
上帝的兒子就住在世人當中.
有恩典有真理.
所以我們是對的.
我們強調耶穌.
強調聖靈.
我們就有上帝同住.
哪裡有聚會.

$^{681}$在希伯倫堂也好.
在紅磡也好.
這個就是我們上帝的同住.
所以你會發覺.
這個就是我們在教裡面.
很重要的概念.
從聖殿到流散.
到席琴那.
上帝的同住.
去新約.
新約裡面其實有很多的經文.
都是在說一些這樣的情況.
都仍然在用despair這個字眼.
新約裡面其實不是特別多.
despair這個字眼.
但是當他在說despair的時候.
其實有三種不同的情況.
第一是什麼呢.
就是當時候.
真的散居在羅馬帝國猶太人.
當時候猶太人已經亡國了.
就散在不同的羅馬帝國地方.
保羅就是其中一個.
保羅就是其中一個猶太人.
散居在羅馬地方的公民.
所以第一個意思.
很明顯就是散居了的意思.
一群散居了的猶太人.
第二就是流散基督徒.
因為被迫的緣故.
大家記不記得.
當時是反被迫害的時候.
然後因為被迫害.
就要散居在不同地方.
所以despair就是這樣的情況.
就是一群因為迫害的緣故.
而迫著要散居分散基督徒.
不過我們發現.
當我們看《釋經史》的時間.
despair不單單是一種處境性.

$^{721}$或者是地理上的意思.
更加是一種神學上的流散概念.
就是說流散不單單是說.
當時猶太人或者基督徒的處境.
更加是基督徒作為基督徒.
教做為教會應該有的質素.
所以我們就會說一下這一點.
為什麼會成為一個神學上的字眼.
despair.
第一個就是亞國書.
亞國書第一章這樣說.
作神和追溯基督僕人的雅各.
請散處十二支派人的安.
這個散居了的十二支派是什麼意思呢.
是解作真的散居了十二支派的人.
還是一種symbolic meaning.
亞國書說的是.
這群十二支派是不是真的literally.
十二支派.
還是一種比喻上猶太人的十二支派.
是一種symbolic meaning.
還是真正是十二支派的猶太人呢.
所以如果你覺得是後者的時候.
如果是一種純粹象徵意義的話.
這種散居其實不一定是那種.
真的散在不同地方.
而是一種信仰上很重要的意味在當中.
另外就是希伯來書.
更加強調.
亞伯拉罕在信夢照的時候就遵命出去.
往將來要得的基業地方去.
出去的時候還不知往哪裡去.
因為信他就住在應許之地作客.
好在在異地居住帳棚.
與那夢和夢一個應許的二十二個一樣.
他們就是在異地裡居住.
這些人都存信心死的.
並沒有得著所應許的.
卻從遠處按見.
且歡喜迎接.

$^{761}$有時仍在世上是客裡是寄居的.
所以這個很重要的身份就是說.
基督徒在地上的那種客裡神學.
我們在地上是客裡.
我們經常上一首歌.
我們是客裡我們是寄居的.
這個寄居不是真實上的地理意味.
而是一種信仰的意味.
我們就是留在地上的人.
無論你住在香港也好.
在外國也好 元朗也好.
都是一種客裡的身份.
都是一種散居的身份.
所以看到希伯來書所說的散居.
就不是真的分散在不同地方.
是一種基督徒的本質.
我們就是一種客裡的身份.
我們就是在地上的一個人.
我們的家鄉在哪裡.
我們的家鄉在天上 天家的概念.
所以我們學某一個近美的家鄉.
因為在天上.
如果我們的家鄉在天上的時候.
我們今天在地上就是什麼.
就是一個迪斯珀拉 就是這個意思.
所以發現新約是慢慢將迪斯珀拉.
變成一個神學概念.
沒錯他們是被迫不留散的.
或者是散居了.
但是更重要的是什麼.
他們真正的家鄉其實就在天上.
所以每個人在基督徒.
都是一個這樣流散的人.
最後更重要的經文就是這個.
就是這個彼得前書.
這個非常重要.
都是打一句.
一個有名的文案.
耶穌基督的使徒彼得.
寫信給那分散在本都.

$^{801}$加拉太 加帕多加 亞細亞.
被推來寄居的.
就是招父臣的先見被揀選.
我刻了三個字.
其實原文裡面是這三個publicable.
去形容這班人.
三個形容詞.
首先就是他們是分散了.
他們是異鄉客 一個寄居的人.
他們被揀選.
所以彼得就用了這三個形容詞.
去形容基督徒.
他們是被上帝揀選的.
一個異鄉客 一個分散了的人.
所以彼得前書更加將流散這個字眼.
不是一種純粹地理上的問題.
而是一種基督徒身份問題.
他們是上帝揀選的子民.
上帝是他們的constant.
雖然他們是分散.
但他們仍然是被上帝揀選的一班基督徒.
他們是一班異鄉的人.
他們的方式和生活習慣是跟其他人不同的.
這就是異鄉的意思.
所以今天在香港裡面也是一樣.
仍然是有一個異鄉的身份.
基督徒跟其他人是不同的.
這就是異鄉的意思.
第三就是散居.
dyspora 他們是分散的.
所以如果我們去理解這個經文的時候.
散居不是說你去加拿大或英國那麼簡單.
而每個基督徒都需要去領受這三個形詞.
他們是被上帝所分散的.
這是上帝的命令.
我們甘願去被分散.
不是因為客觀的元素.
不是因為某個政府的強權.
而是因為我們基督徒就會被分散.
我們被上帝揀選.

$^{841}$我們是一個異鄉的角色.
所以你會發覺.
從舊弱到新弱.
你會發現原來被擄或流散.
對我們來說不是純粹一個客觀的政治現實.
不是純粹因為某一年出現某條法例.
而是一個我們基督徒應該要有的質素.
所以你會發覺.
我會嘗試去講這一點.
這三個形詞.
這個我不講了.
就是擄散作為我們教會的使命.
原來擄散如果不是客觀的政治現實.
而是我們基督徒作為教會的重要的確具.
我們作為教會的觀念.
擄散是我們的教會使命.
更好的形詞.
我們稱之為「踩險」.
大使明這麼說.
你們要去.
我們小時候就說你們要去.
這個去本身就是一種什麼.
就是一種擄散.
大家要分散去不同地方.
我們實踐了大使明.
所以你會發覺整個的擄散.
可以成為我們很重要的觀念.
剛好我們叫flow church.
所以我們flow church本身就成為一個這樣的觀念.
flow church和擄散這幾年我們都用了這些keywords.
be water我們都用了這些keywords.
無論是be water也好.
flow也好.
擄散也好.
其實都是一個我們核心的價值.
所以我上次去說去形容.
怎樣去理解我們作為flow church.
怎樣理解擄散這件事情.
所以擄散是我們教會的使命.
是我們一件帶著使命去做的事情.

$^{881}$一種我們教會的本質.
我們理解教會的概念.
這樣去理解這個擄散.
第一就是擄散和教會的差遣.
這個差遣這個字.
其實正正就是一種叫做mission.
一種叫做使命的字眼.
其實我經常說mission day.
或者叫做missionary.
好像叫做宣教或者叫做使命.
其實mission本來就是差遣.
上帝自己來差遣一個人.
這是最基本的意思.
所以不單單是宣教或者某些使命.
而是上帝去差遣.
上帝差遣他的獨生兒子來到世界上.
這是第一層的差遣.
然後在永福特色七章裡面說什麼.
耶穌是差他的門徒去到世界上.
這是上帝耶穌基督差遣門徒的差遣.
然後我們教會會差遣宣教士和差遣人.
所以是這樣的理解.
所以教會本身正正就是一個這樣的群體.
我們教會有幾個很重要的步驟.
我們教會是會招聚.
Ecclesia就是教會的聚集.
留堂在四年前是這樣.
重新來招聚一群無教會基督徒.
我們重新來聚集.
然後我們重新來建立我們的群體成長等等.
但是第三就是差遣.
我們不單單是聚的.
也不單單是彼此有個團契或者模樣的.
而是我們有使命的.
上帝差我們去世界裡面.
所以是關乎我們教會和世界的關係.
我們怎樣和世界建立關係.
我們怎樣來見證上帝.
就是差遣的元素.
所以教會本身的分散是帶著這樣的差遣.

$^{921}$每個禮拜崇拜裡面.
我們都有一個這樣的差遣的意味.
我們今天聚完聽完道.
唱散會詩.
大家就走出去差遣.
帶著使命的去見證上帝.
所以流散不單單是給流散者.
如果你是在看YouTube的.
你在海外的時候.
當然這個流散是與你無關的.
因為本身你是一個流散的處境.
大家看我們在這裡像在座的都一樣.
今天我們在香港.
我們所謂的漁民.
就是留下來的人.
我們仍然帶著這樣的使命.
去做同樣的事.
所以我們作為流塘.
一個流散的教會.
一個流散群體.
這是我們每個人都應該思考的問題.
怎樣能夠在我們今天的生活裡.
在香港 在海外.
怎樣能夠來見證上帝呢.
這是我們第一個很重要要問的問題.
第二就是這個.
那時候教會是甚麼呢.
我就稱之為教會是任何形狀.
你聽過一次我這樣說過.
如果我們說流散是我們一個教會觀的時候.
一種帶著使命的教會觀的時候.
教會的形狀是甚麼呢.
教會就不再是一個建築物或一群人.
而是一個我稱之為liquid church的概念.
其實有一本書叫做liquid church.
就是將教會看成一種liquid.
可以滲透在不同的地方裡.
我們不是一件東西.
我們是流塘.
我覺得這些全都是keyword.

$^{961}$流塘正正是這樣的群體.
一個液化教會.
它是一個流散在世界不同角落裡.
滲透在世界裡的群體.
所以這是比無牆教會更加重要.
以前我們說甚麼無牆教會.
無牆教會只不過是一個無牆教會.
人們進來不需要牆就能進來.
但我們更加進一步.
我們不是無牆.
更加是liquid church.
我們的教會是沒有任何形狀的.
你無法定義流塘的群體是怎樣.
因為你會發覺.
剛剛做了數據.
有七成是香港人在流塘崇拜.
有三成是海外的.
所以發覺原來流塘的群體.
不是純粹那麼容易畫出來的.
它是一個liquid.
它可以任何形狀.
它可以任何的形態.
它可以是一個floam的石光群體.
一個坐在一起學習的群體.
所以這是我們很重要的流塘的概念.
既然我們是流散的時候.
我們自然而然很強調我們是流散出去.
但我們會有很多不同的方式去連結.
藉著我們崇拜去聚集.
但這不是永遠的事情.
我們是會分散出去.
成為一個很重要的使命.
在不同的角落滲透.
這也是關我們事的.
關原民事的.
當你在香港住的時候也是一樣.
我們作為基督徒.
在你的工作裡面.
在你的公司裡面.
作為一個這樣的水滴.

$^{1001}$一個這樣的liquid.
去應驗.
在教會延伸.
所以今天我們來到.
現在教會的時候特別流塘.
我們是一個比其他地方更不像舊的東西.
我們沒有舊的東西.
很強調就是大家的那種使命.
在不同的地方成為水滴.
來見證耶穌.
所以這是我們第二個很重要的形容.
無論你是在海外還是在香港.
我們成為教會的一部分.
我們未必一定有一個很concrete的形狀.
畫出來一個十字架.
一個三角形屋頂的教會.
但我們是一個liquid church.
一個很flexible.
很高流動性.
能夠滲透在世界裡面.
見證上帝的一個群體.
第三.
我想這是一個很重要的一點.
就是我們所說.
就是一個.
我們的流散.
今天我們分離.
我想大家都沒有人猜到.
當我們在流塘開始的時候.
沒有人猜到我們會離開香港.
我想說.
這個分離.
其實走和留在這裡都是很痛苦的.
我們被告別.
我們告別人.
這種分離令到我們流塘被分散.
有些人去了英國.
有些人去了加拿大.
有些人回到香港.
我們好像很傷.

$^{1041}$好像被人激散了.
好像被人篩走了.
但如果我們今天回頭看.
今天想說什麼.
這個流散.
這個despair.
絕對不是因為一個客觀因素.
而是我們作為基督徒.
一個很重要的使命.
我們正正是一個被揀選群體.
一個異鄉客群體.
一個流散群體.
這個不是一個客觀因素弄走我們.
而是我們知道上帝帶著命令去猜險我們.
上帝帶著命令.
要我們帶著使命.
這樣去面對我們的處境.
所以如果你是在海外.
頂姐妹.
不要覺得因為哪個國家令到你要走.
而是因為上帝早在你出生的時候.
祂知道今天你在海外.
我們記得孖孖順是揀選這群人.
順便強調一件事.
我們是被流散.
是異鄉.
但被揀選.
我們每個人的生命裡面.
你住在哪裡.
決定住在哪裡.
這個是上帝的命定.
所以我們嘗試去思考上帝的命定和召命.
來想想我們自己的情況.
基督徒只能夠帶著召命去做人.
我們流散是意味著一種召命.
我們怎樣能夠帶著上帝在我們人生裡面.
要我們做的事.
去到不同地方裡面.
來延續下去.
今天你可能離開海外.

$^{1081}$很多基督徒.
傳道人.
你都知道我經常去海外.
這兩年裡.
見到很多基督徒.
可能暫時不知道做什麼.
甚至傳道人都不知道做什麼.
但一定要找回自己的召命.
上帝要我來這裡.
我不是因為落難.
不是因為要逃避某些事.
要被迫出走.
而是我是正面帶著召命.
來面對我的處境.
上帝知道的.
上帝揀選.
我們流散是一種使命.
是差遣.
同樣香港人都一樣.
香港人都一樣.
我們都仍然是一個流散群體.
我們走來這裡.
但仍然是一個被流散這個詞形容的群體.
我們今天在我們的工作裡.
仍然在香港的時候.
都是帶著召命.
知道今天我們怎樣可以來到香港.
繼續延續上帝的召命.
這是我們上一季的第一堂所說.
基督徒的意思.
不是純粹宗教徒的身份.
而是我們怎樣見證上帝基督徒的意思.
或者見證基督群體.
所以第二季我們加上一個處境.
我們雖然這幾年裡面.
面對這個情況.
但我們是帶著這樣的一個身份.
去面對我們的處境.
不知道在場的弟妹妹.
怎樣去想.

$^{1121}$你怎樣去理解這幾年發生的事情.
怎樣去想你的未來.
你是否帶著一種召命去想自己的未來.
你是否將「留山」這個字.
看為一種使命.
我想這個就是我們留堂裡面.
一個很重要的價值.
我們是沒有形狀的教會.
它是一個很堅定帶著使命的教會.
我們一起祈禱吧.
然後我們請Captain Poon上來.
因為你讓我們留堂.
帶著「留」這個字.
你在早在四年前.
你都知道我們作為一個留堂的時候.
一個Flow Church的時候.
你叫我們去實踐這個「留山」的召命.
我們留堂不是被人擊打分散.
而是帶著一種更加強烈的使命.
來理解我們的處境.
無論我們在英國,在加拿大,在澳洲.
我們仍然來思想我們留堂.
作為一個這樣的教會.
一個很奇怪.
有時在YouTube,有時在Facebook.
有時在實體裡面.
好像分散它能夠不同方式來招聚的教會.
求主你這樣來幫助我們.
將我們每個人作為留堂的分子.
都能夠作為一個真正的基督徒.
來思考我們的召命.
我們求主你幫助我們.
讓我們能夠在金堂裡面.
更加認清聖經裡面很多很多的基礎.
我們知道這個流散,被擄.
不是純粹一個客觀政治現實.
而是一個對於萬國萬民的心意.
更加是你對我們的呼召.
求主你這樣幫助我們.
奉主命求,阿門.

$^{1161}$喂.
這個人次真的不一樣.
哇,真是.
很忙啊.
你坐下來吧.
你剛剛從多倫多回來嗎?.
你工作辛苦了.
辦公時間不要喝東西.
但現在不是辦公時間.
謝謝.
今天聽了一個內容.
可能對弟兄姊妹來說都有點新.
但其實剛才你所說的過程當中.
其實都有很多歷史源流.
過去教會很少特別提及.
或者是很認真去說.
在場很久沒見過機場有這麼多人.
不知道大家.
很多棒隊都在等你出來.
應該在等你工作.
應該可能都有些問題可以相關大家一起討論.
特別這個.
我想過去教會比較少討論的主題.
有沒有等待登機的乘客.
有些問題想參與一下呢?.
不用登記也可以問的.
我想問一下.
你怎樣想自己的未來?.
你怎樣想這幾年後的未來?.
你覺得有什麼做?.
可以分享一下.
剛才我看網上的時候.
看到有些弟兄姊妹問.
不要只說加拿大,英國.
其實真的有很多地方都涵蓋.
因為我相信周邊有很多弟兄姊妹.
總有不同朋友去的地方.
從來都沒聽過.
但事實上去了.
帶著不同的情感去.

$^{1201}$或者原因去.
對於今天在內容裡再一次提醒.
可能有些事情我們這一刻還不明白.
但其實都帶著一些使命.
可能未必覺得.
沒想過用使命這個原因去.
不過對於大家來說.
今天可能參與現場的你.
你在想什麼呢?.
有問題嗎?.
有些神學的問題想問一下.
第一點是.
在地老之前是聖殿神學.
就是上帝住在聖殿裡.
地老之後就流散在以色列人中間.
你覺得上帝有沒有搬過房子呢?.
我的意思是.
祂本身住在聖殿裡.
然後我躲在山樓裡搬了房子.
第二個問題也和這個有關.
因為本身地老流放是一個懲罰.
我想知道這個轉變是人的心態改變.
還是一開始由懲罰成為祝福.
這個變化究竟是一件怎樣的事呢?.
因為如果最後變成祝福.
就不是罰了.
沒有罰.
反而我反過來想.
你們這麼壞.
你們準備好可以出去了.
就放你們出去.
好像很奇怪.
我問你.
我想其實不是大家互相反對.
不是大家有矛盾.
我想是同一件事上.
祂肯定是懲罰.
因為很清楚寫明是上帝審判.
因為他們犯罪博弈.
但同時原來這件事有很多向度去理解.

$^{1241}$在審判之餘.
原來上帝有憐憫.
有盼望.
甚至會和他們同在.
這種事其實不會有矛盾.
沒錯他們是很慘的.
但同時也可以成為人的祝福.
所以舊約裡面.
我一本書強調.
這七個類別裡面.
是七個不同的角度去理解同一件事.
早期覺得沒有得罰.
很慘.
流淚就走了.
但當他們注入十八個愛.
原來也不是純粹是一種傳言的審判.
也有加上是上帝的盼望.
也會歸回.
也會有上帝同在.
所以我們今天面對的情況是很慘的.
但同時也有上帝的使命.
也有上帝的臨界.
所以剛才說的Shakina.
所以這是一種很重要的.
聖經也是.
我也不是強調聖經在聖殿裡面.
聖經是一個通達聖帝的地方.
所以聖經也是在天上的天裡面.
所以聖經只是一個中心點.
來將我們敬拜.
將我們的禱告.
如《伊武信》般升上去.
所以聖經是一個我們朝向的方向.
這是一個這樣的理解.
就算在那個時期裡面.
聖經也不是一個真正的上帝同住的地方.
而是一個象徵性的上帝名字的地方.
所以當後來聖經被毀的時候.
更加一樣.
原來上帝是可以不單單是天上的天.

$^{1281}$更加會跟著人們落難走.
這是一個很強大的愛.
因為耶和華上帝不是生神.
用教育的概念來講.
不是在山上.
不是巴黎.
但是他甘願跟著這班被人迫害的猶太人.
去到不同地方跟他一起走.
所以很強.
我們說得很低俗.
上帝同在.
上帝總是同在.
他經常說.
但是對於當時來說.
同在是一個很重要的愛.
因為他願意跟著一些落魄的人.
去到不同地方.
跟著他們在後面走.
所以這是一個更加深層的意義.
還有一個更加長遠的歷史.
聖經是說幾百年.
但是整個猶太歷史幾千年.
都是流散.
所以更加重要的是.
他是流散的神學.
一個上帝願意跟著這些人.
想想二戰的時候的猶太人.
在集中營裡.
上帝跟著他們在集中營裡.
所以上帝就是這樣.
所以我們作為基督徒.
更加要知道.
我們流散是一件什麼事情.
我們今天的處境.
沒有否定過.
也不是阿Q覺得不是一種災難.
但是這個之餘.
上帝仍然在當中跟他們同在.
還有帶著使命跟他們走.
用應用神學的人設來回應.

$^{1321}$就好像大家熟悉的約瑟.
約瑟當初被他的兄弟賣的時候.
原是惡的.
但是當他回想的時候.
他就看到上帝的美意.
而用同一個切合角度.
就是在約瑟當中的Sakina.
就是他要解夢的時候.
他要去執政的時候.
他就感受到耶和華神與他同在.
這個你就會明白到.
那個過程當中的相近.
有些事情想請教.
剛才提到有七個.
我用一個點來解釋.
因為我始終不太掌握到.
那個七個點是七個不同的階段.
還是七個不同的認知的觀點.
如果是七個不同的階段的時候.
它是有什麼刺激呢.
是由一個階段轉去另一個階段呢.
如果是七個不同的認知.
或者一個世界觀的時候.
是不是在當中的時候.
是同一時間在流散群體裡.
其實是有不同的群體.
有不同的認知.
但是如果是有的時候.
是什麼原因引致不同的群體有不同的認知.
我最想掌握的就是.
它那七點.
我看到其實好像是.
對於我們現在的基督徒來說.
第七點好像是.
較為安慰的訊息.
或者是一個盼望的訊息.
開頭那幾個點.
可能都是較為困難的認知.
如果用階段認知也好.
都是較為困難.

$^{1361}$怎樣在當中的時候.
作者去提.
在頭幾個階段認知也好.
怎樣去承載他們繼續走下去.
我覺得這是我自己.
對於當下的時候.
其實是較為關心和想掌握到.
謝謝.
這個其實是那本八百多本書.
最後的章節的結論.
所以那本書裡面就是將所有的經文字研究.
做歸納的研究結果.
所以這是一個很客觀的分析.
總之教育裡面有七個category.
將經文全部放在一起.
分七組.
七組不同意味的經文.
所以本身是一個很客觀的step.
原來教育裡面有七個不同的category.
有沒有一些階段性呢.
其實是一種詮釋.
有沒有呢? 我覺得是有的.
因為看到它們很明顯是從一個過得被罰.
慢慢慢慢接受.
好像一些階段性.
所以我覺得這七個.
其實是純粹經文裡面有記載.
以適當去理解自己的秘魯的回應.
我覺得是有階段性的.
但是怎樣到下一個階段.
我應該就沒有那麼清楚.
純粹我們知道教育裡面有不同的情況.
可能後期一點就慢慢發現.
原來上帝會同在.
所以我覺得又不是最後只有七個才是樂觀.
基本上是第一個審判之外.
那六個其實全部都是一些.
在困難裡面.
來發現一些好事的經文.
接受了宗教的團契.

$^{1401}$或者是上帝同在.
所以其實是沒有一種很容易的理解.
法國聖經裡面是這樣.
七個不同的經文.
講同一件事情.
所以我覺得很難給一個很簡單的答案.
原來只要這樣做就下一個階段了.
又不是這樣.
所以我覺得這件事就是.
教育裡面講秘魯.
是一個很明顯的痛苦現實.
但是會有這樣的分類.
所以我覺得我們又不容易很簡單.
將它簡化成為一個七個流程.
但這是一個事實.
聖經裡面有七個不同的.
描述講秘魯的經文.
(記者:劉先生,如果我們留下來).
(記者:我們在香港的教育).
(記者:劉先生,如果我們).
(記者:基督徒的一個使命).
(記者:我們應該視他為一個使命).
(記者:但如果).
(記者:好像很清楚我們在亂流中).
(記者:應該做些什麼).
(記者:但如果我不知道).
(記者:我知道神給我一個照明).
(記者:但我不知道應該).
(記者:擺自己在什麼位置或方向).
(記者:我們可以怎樣做?).
(記者:我正在看網上).
(記者:他問怎樣做).
(記者:怎樣找到自己的照明).
(記者:如果有人知道的話).
(記者:我看照明這個詞).
(記者:其實不是很高的東西).
(記者:我常常覺得上帝).
(記者:他會給你一個照明).
(記者:但我看照明).
(記者:其實不是很高的東西).

$^{1441}$(記者:我常常覺得上帝).
(記者:他會給你一個照明).
(記者:但我看照明).
(記者:他會給你一個照明).
(記者:我常常覺得上帝).
(記者:做我們出來).
(記者:有他自己給我們的).
(記者:那種心意).
(記者:意思是).
(記者:假設你).
(記者:當你真的做一個泥公仔).
(記者:那你自己的比例).
(記者:你做的姿勢是什麼).
(記者:其實你自己也有一個想法).
(記者:然後你想成品是什麼).
(記者:其實照明就是).
(記者:我自己看).
(記者:上帝給我的照明就是).
(記者:上帝應該有一個).
(記者:事情要我做).
(記者:對我來說).
(記者:我常常說的人設).
(記者:你的gift和你的talent).
(記者:你的經歷).
(記者:還有你自己的心意).
(記者:有沒有一些群體).
(記者:有沒有一些特別的場景).
(記者:你喜歡).
(記者:對於我來說).
(記者:你在成長過程當中).
(記者:在信仰生命當中).
(記者:不斷去了解).
(記者:自己的過程).
(記者:發生了什麼事).
(記者:具體來說).
(記者:過去在神學院).
(記者:會接觸到很多).
(記者:想進來讀神學的弟兄姊妹).
(記者:他們告訴我).
(記者:他們感受到上帝的呼召).

$^{1481}$(記者:我想了解一下).
(記者:他們的呼召具體是什麼).
(記者:就像我教應用神學).
(記者:就是有沒有特別的群體).
(記者:他們特別上心).
(記者:他們的技能).
(記者:或者他們的進路).
(記者:是用什麼方式).
(記者:然後大概怎樣用到).
(記者:他們將來出來服侍的方法).
(記者:所以你問我).
(記者:怎樣可以了解).
(記者:留下或出去的弟兄姊妹).
(記者:他們的呼召或使命).
(記者:其實就).
(記者:我就會朝著這個方向去想).
(記者:這樣讓自己知道).
(記者:我不是排他).
(記者:是他了上帝史無變有的能力).
(記者:就是突然間沒有).
(記者:他突然給你一次).
(記者:我不是這樣).
(記者:但大多上帝都會用).
(記者:一直以來的方式).
(記者:就好像).
(記者:你看很多聖經人物).
(記者:我經常都說).
(記者:那個例子就好像).
(記者:大衛能夠打死哥利亞).
(記者:不是因為他單單靠上帝的能力).
(記者:因為他自己都說).
(記者:當時蘇羅給了他一把劍).
(記者:他說他用不到的).
(記者:這件衣服我穿是動不了的).
(記者:但我一直都是).
(記者:對於野獸對於熊的方式).
(記者:就是這樣).
(記者:那就是上帝一直給他).
(記者:還有大衛是一個詩人).
(記者:他是唱歌).

$^{1521}$(記者:他放羊他悶就做這些事).
(記者:我想未必是一個).
(記者:他突然爆出來的事).
(記者:就是這樣).
(記者:我們跟我老婆聊天).
(記者:就說如果我們).
(記者:十幾二十年前).
(記者:沒有回應夫子要讀神學).
(記者:會怎樣呢).
(記者:我講到這個位置).
(記者:我應該和大家一樣).
(記者:沒有回教會).
(記者:沒有回到外國).
(記者:不知道去哪裡).
(記者:我老婆說).
(記者:因為他都是傳能).
(記者:如果沒有回應召命).
(記者:突然爆出來).
(記者:上帝應該給我召命).
(記者:如果不回應A).
(記者:上帝會給他召命).
(記者:這樣也可以嗎).
其實很有趣.
我不知道什麼召命.
但一個有召命的人.
我知道如果沒有召命的人會怎樣.
我覺得自己有召命已經很足夠.
明不明白.
我不知道是什麼.
但我知道自己有召命.
這個信念很重要.
如果沒有召命去活.
我只會去英國.
然後看看怎樣做.
雖然我不知道是什麼.
但我知道我有.
但找不到.
但我不知道是什麼.
都是在家裡找工作.
但我知道我有召命的話.

$^{1561}$是一個完全不同的態度去活.
所以你問我我召命什麼.
我可能是傳達人.
但總之我知道我有.
所以很重要.
無論在香港還是海外.
我們是被揀選.
很強調這個字.
我們是異鄉的客女.
是被揀選的.
這個字是一個很重要的字眼.
因為我們知道雖然我們散了.
收了又流散了.
但我們知道我們有上帝的命.
在當中的活.
是不能分開三個字.
我們是流散.
不過我們被揀選.
所以我們流散之餘.
我們知道自己.
是有一個事要去做.
雖然我不知道是什麼.
是可以的.
但我知道我是有多過無.
最怕是當一個有召命的人.
去到那裡之後.
突然沒有了一個召命.
這樣去活.
我不是為了做安全導人.
而是他沒有召命.
他沒有這樣做人.
是很痛苦的.
純粹在那裡.
因為這個緣故我就走了.
但你沒有召命.
他繼續下去.
你離開是可以的.
但你離開之後.
你真的繼續去找.
深信我是有召命的.

$^{1601}$你面對可能做一個家庭主婦.
在加拿大做某個工作.
但你知道是有的.
是重要的.
其實是揀選.
所以這個心態是重要的.
可能第一次會卡的一個位置.
我不知道什麼叫召命.
可能會卡在一個位置.
傳導人或傳職侍奉.
這個比較普遍容易掛單.
我打份工.
或者我在一個崗位上.
我有什麼召命.
這些位置會卡的.
我以前用過一本書.
跟弟兄姊妹講解一下.
你的角色是什麼.
那本書是一個退休.
呼蘭神學院的老師.
他叫做Robert J. Clinton.
他寫的書叫做.
The Making of a Leader.
裡面中立了很多聖經裡面.
一些領袖上帝怎樣去教導他.
或者培育他.
其中要帶出的訊息就是.
當初他未做一個領袖之前.
其實就是他在不同單位的職位.
怎樣做好他崗位上的職份.
慢慢他做到一樣.
又再加一點給他.
慢慢加多一點的時候.
不知不覺就能夠完成上帝給他做領袖的職位.
所以過程是一個慢慢的.
你說做不到那怎麼辦.
Clinton的說話就是.
學習過程失敗了.
那就要重做.
即是要再做.

$^{1641}$不過用另一個方式.
或者他不回應的時候會怎樣呢.
上帝用另一個方法.
或者用另一個途徑.
讓他再試那個方式.
好像若拿一樣.
即是上帝擺在每一個人生命當中.
都有一個使命.
我們人生的過程當中.
就是不斷去認證上帝給我們的使命.
不一定是全職.
是在你的職位裡面怎樣領導.
以致你可以成為一個領導職位.
所以整件事就是.
這件事就沒有地位限制.
留在香港也好.
或者散居其他地方.
你只不過是換了場景.
上帝仍然在你的生命當中有他的心意.
以致你在不同崗位當中.
去lead out 流出.
這就是你的照明.
剛才聽到你提及七個stage.
一個是接受流散的處境.
舊約的人被人擄走.
之後那個國家又變得非常強大.
擄走那些國家.
他們在那裡怎樣面對一個矛盾.
就是他們受害者.
即是仇人.
亞歷山河因為他身處在那裡.
他也要祝福一下.
因為那裡不好.
自己也不好.
以前的人怎樣面對這個.
舊約的人怎樣面對這個矛盾.
技術上我們不是說技術上.
但你會發覺流散.
或者被擄其實不是相當差.
亞述國是好很多.

$^{1681}$亞述國被擄是會好一點.
亞述國被擄是會.
一擄就會.
那時候好像很.
一擄是一家人一起擄的.
這樣是好的.
即是一起移民去那裡.
巴布倫是一個很壞的擄國.
他們會只擄男人走.
很凄利子散.
所以其實他們和.
都是會有很多不同的情況.
有時是會是.
擄工人一起走.
有時是一擄的時候.
所謂的一擄是一個deputation.
一個遷移.
所以就不打算只掛著監牢.
不一定這樣.
是一種遷移的情況.
所以其實亞述國是會將.
這些人和人搬來搬去.
令他們分散.
因為他們不能團結.
分散了.
所以不是一種純粹是.
太差的.
雖然都是差的.
所以不是說.
生活.
今天這樣.
今天都很差.
但都是生活.
所以所謂接受.
算是一種生活.
去找工作.
去生活.
去有信仰.
所以又未必是一個破家亡的情況.
是有的.

$^{1721}$但又不是全部都是這樣.
平民其實都是很普通.
這樣就被人遷移.
所以有這樣的情況.
這個是很technical.
technical的意思是.
在神目上怎樣運作.
或者怎樣去感受.
但我們之後的課堂都會講.
因為我自己覺得.
回應剛才的提問.
其實我們華人教會.
或者華人群體.
很少去學習什麼叫做分離.
我不知道大家成長過程當中.
有沒有教一個.
情緒教很少教的.
其實分離這件事.
在你的成長當中.
可能都沒有怎樣接觸.
所以要怎樣面對分離.
還有家這個觀念.
在我們華人.
特別是華人教會來說.
是很濃烈的.
你有什麼情況會離開這個屬靈的家.
離不開這三個原因.
如果你去讀神學.
你就帶著祝福離開這個屬靈的家.
但如果你說.
你結婚也是帶著祝福的.
不如就他離.
就不要你過去.
都還可以的.
但你說你搬遠了.
轉教會.
想轉這個屬靈的家.
其實香港有多大.
都是一個多小時車程.
但你搬到天水圍回柴灣.

$^{1761}$都很麻煩.
這些分離的感覺.
很不容易解決.
特別是.
我只是說.
雖然剛才聖經說了很多方式.
令人去分離.
或者整個群體分離.
但起碼我們香港.
或者過去成長這麼多年.
很少說這些話題.
或者沒有告訴你.
有一個離開的計劃.
所以我們要認真了解.
這個聖經的教導.
在現在的實務上.
怎樣去理解和執行.
後來也知道.
先知後繼先知.
要人們回到耶路撒冷.
人們不肯離開.
不肯回到自己的家鄉.
因為那邊太過安穩.
不敢回到耶路撒冷.
重建城牆之類.
所以發覺.
在外邦生活也不差.
他們的經濟.
或者文化.
適應了下一代.
就像我們今天的處境.
很多人都有不同的情況.
反而是.
問題是怎樣生活.
我們香港也一樣.
怎樣能夠在這樣的環境.
繼續去接受.
這樣的處境.
不是同意,但仍然有這樣的生活.
來開展.

$^{1801}$這是我們下一步要說的.
怎樣有新生活,新生命.
怎樣在這樣的環境裡面有新的開展.
甚麼是新呢?.
想問一個.
跟趙明有關的問題.
你剛才說用了藥室做例子.
你說可能被擄或留散.
可能跟趙明是相關的.
是同意的.
但我想問一個問題.
是先後次序的問題.
有時候可能你是被擄.
或留散了.
之後回頭看才發現.
原來趙明是神安排的一部分.
又或者另一個時候.
神給你趙明.
你去離開.
到底是哪一樣東西.
會比較多.
或者兩者都有.
還是怎樣去分呢?.
主要是想問.
關於這兩者之間的關係.
和先後的事.
我會看到.
可能是我們這樣.
或者是.
我不知道可不可以用你們兩個做例子.
例如你們有趙明建立一個.
領類.
Flow Church的教會.
還是其實Flow Church建立完.
之後慢慢發現.
原來上帝一開始叫我們這樣建立.
這間教會是有這樣的一個趙明呢?.
謝謝.
我先回答.
我回答的問題.

$^{1841}$先不是賣關子的.
不過又不要覺得是搞笑.
就是.
你怎樣認識女孩子的?.
你怎樣選擇女朋友?.
你是.
看完這麼多女孩子的標準.
或者你本身已經設定好標準.
我有五個標準.
中了這五個.
這個就可以做我女朋友.
還是你覺得.
這個女孩子不錯.
然後認識.
認識之後發現都不是我的杯茶.
我想說的意思就是.
怎樣才算是一個趙明呢?.
有時候有些東西.
有一個很具體的圖畫.
然後完全贊同下去.
但有些東西.
就是你做著做著.
你會對自己的趙明清晰.
我.
我自己比較喜歡.
去.
覺得信仰是一個.
談感情的過程.
信仰就是.
當然可以很神學.
Faith seeking understanding.
你帶著信心去求認識都可以.
但我們上帝.
我自己一直理解.
都是很幽默的.
在一個過程當中.
一個情投意合.
我靠近他多一點.
他又給我認識多一點.
我抽離的時候.

$^{1881}$他又不會不理我.
這個沉迷過程是重要.
我不是要淡化這麼嚴肅的問題.
不過你背後問的東西.
其實是想有一個方程式.
但我們正正就很難有一個方程式.
我和John都不會同意.
但我們兩個.
更加未必適合你用.
但我們會將一些可能性.
和你分享.
信仰的多面性.
或者豐富.
正正就是.
不說你不知道.
說你都知道不完.
但說多一點就知道多一點.
這個就是我.
在啟示上.
上帝啟示都是這樣.
不斷地上帝啟示.
讓子民知道.
到耶穌基督.
來到的時候.
約翰用得很好.
從來沒有人見過父上帝.
只有父懷裡的獨生子.
將祂表明出來.
約翰一二三書的時候.
他在晚年講得清楚.
這是我們親手摸過.
親眼見過和親眼聽過.
他就將所有東西.
第一身去表達.
過程當中.
有圖畫.
但又可以讓你互動.
沉覓一下.
希望你明白我的意思.
十居其九.

$^{1921}$都是不知道.
你會去了.
才知道.
靈鵪派就知道了.
收到一些東西.
阿伯拉罕.
都是憑信就走.
所以十居其九.
都是這樣的模式.
坦白說.
剛才我說要做明日進.
因為知道有東西.
有的時候.
有這個.
有這個就去.
有些人會拿著東西.
會拿著東西去.
但那東西是否真的.
你都不會知道.
但都可以拿著東西去.
神叫我這樣做.
但最後回頭看.
更加重要.
因為那是事實.
是上帝的作為.
像呼吸.
我只是摸摸.
現在就這樣做.
所以很多時候.
不妨試試拿著東西.
但又知道.
最後回頭看.
才會更加實際.
拿著東西的原因.
是你確信有.
神會帶領.
有一個召命.
就這樣做.
所以我經常都拿著東西.
你說神叫你去.

$^{1961}$我會說對.
但是否我們不需要.
多數都不知道.
不確定.
不妨說是.
反正大家容易去想.
但又不要太絕對.
留一些空間給上帝.
去改.
以前我在德國讀書時.
有幾個同學來過德國.
然後又回到香港.
讀不完或讀不成.
都是這樣的情況.
但儘管來.
都是這樣看神怎樣開路.
回頭看原來神不是帶我去德國讀書.
就是這樣的情況.
(記者:第一班機的乘客還有沒有其他登機前要問的問題?).
好像是茶會.
(記者:現在沒有茶會了).
現在不玩團了.
(記者:準備上機了).
麻煩你了.
我下班了.
我們下個月再見.
拜拜.
多謝大家收看.
\newpage



\section{}
\label{sec:7ZGXT0f30Z0}
\textbf{《致餘民及流散者:給香港基督徒的神學八課》第二季第2課|20230325 [7ZGXT0f30Z0]}
\newline
\newline
連結: \href{https://youtube.com/watch?v=7ZGXT0f30Z0}{\texttt{ https://youtube.com/watch?v=7ZGXT0f30Z0}} ~~~~ 語音日期: 2023-03-25 
\newline
\newline
\hyperref[sec:y5NJfoAjRCI]{\small{< < < PREV SERMON < < <}}
~
\hyperref[sec:index_chronic]{\small{[返順時目]}}
~
\hyperref[sec:index_scriptual]{\small{[返順卷目]}}
~
\hyperref[sec:8KdYgVn_hzk]{\small{> > > NEXT SERMON > > >}}
\newline
\newline
$^{1}$我只想知道.
你到底是什麼意思.
我只想知道.
你到底是什麼意思.
我只想知道.
你到底是什麼意思.
我只想知道.
你到底是什麼意思.
我只想知道.
你到底是什麼意思.
我只想知道.
你到底是什麼意思.
我只想知道.
你到底是什麼意思.
我只想知道.
你到底是什麼意思.
我只想知道.
你到底是什麼意思.
我只想知道.
你到底是什麼意思.
我只想知道.
你到底是什麼意思.
我只想知道.
你到底是什麼意思.
我只想知道.
你到底是什麼意思.
我只想知道.
你到底是什麼意思.
我只想知道.
你到底是什麼意思.
我只想知道.
你到底是什麼意思.
我只想知道.
你到底是什麼意思.
我只想知道.
你到底是什麼意思.
我只想知道.
你到底是什麼意思.
我只想知道.
你到底是什麼意思.

$^{41}$我只想知道.
你到底是什麼意思.
我只想知道.
你到底是什麼意思.
我只想知道.
你到底是什麼意思.
我只想知道.
你到底是什麼意思.
我只想知道.
你到底是什麼意思.
我只想知道.
你到底是什麼意思.
我只想知道.
你到底是什麼意思.
我只想知道.
你到底是什麼意思.
我只想知道.
你到底是什麼意思.
我只想知道.
你到底是什麼意思.
我只想知道.
你到底是什麼意思.
我只想知道.
你到底是什麼意思.
我只想知道.
你到底是什麼意思.
我只想知道.
你到底是什麼意思.
我只想知道.
你到底是什麼意思.
我只想知道.
你到底是什麼意思.
我只想知道.
你到底是什麼意思.
我只想知道.
你到底是什麼意思.
我只想知道.
你到底是什麼意思.
我只想知道.
你到底是什麼意思.

$^{81}$我只想知道.
你到底是什麼意思.
我只想知道.
你到底是什麼意思.
我只想知道.
你到底是什麼意思.
我只想知道.
你到底是什麼意思.
我只想知道.
你到底是什麼意思.
我只想知道.
你到底是什麼意思.
我只想知道.
你到底是什麼意思.
我只想知道.
你到底是什麼意思.
我只想知道.
你到底是什麼意思.
我只想知道.
你到底是什麼意思.
我只想知道.
你到底是什麼意思.
我只想知道.
你到底是什麼意思.
我只想知道.
你到底是什麼意思.
我只想知道.
你到底是什麼意思.
我只想知道.
你到底是什麼意思.
我只想知道.
你到底是什麼意思.
我只想知道.
你到底是什麼意思.
我只想知道.
你到底是什麼意思.
我只想知道.
你到底是什麼意思.
我只想知道.
你到底是什麼意思.

$^{121}$我只想知道.
你到底是什麼意思.
我只想知道.
你到底是什麼意思.
我只想知道.
你到底是什麼意思.
我只想知道.
你到底是什麼意思.
我只想知道.
你到底是什麼意思.
我只想知道.
你到底是什麼意思.
我只想知道.
你到底是什麼意思.
我只想知道.
你到底是什麼意思.
我只想知道.
你到底是什麼意思.
我只想知道.
你到底是什麼意思.
我只想知道.
你到底是什麼意思.
我只想知道.
你到底是什麼意思.
我只想知道.
你到底是什麼意思.
我只想知道.
你到底是什麼意思.
我只想知道.
你到底是什麼意思.
我只想知道.
你到底是什麼意思.
我只想知道.
你到底是什麼意思.
我只想知道.
你到底是什麼意思.
我只想知道.
你到底是什麼意思.
我只想知道.
你到底是什麼意思.

$^{161}$我只想知道.
你到底是什麼意思.
我只想知道.
你到底是什麼意思.
我只想知道.
你到底是什麼意思.
我只想知道.
你到底是什麼意思.
我只想知道.
你到底是什麼意思.
我只想知道.
你到底是什麼意思.
我只想知道.
你到底是什麼意思.
我只想知道.
你到底是什麼意思.
我只想知道.
你到底是什麼意思.
我只想知道.
你到底是什麼意思.
我只想知道.
你到底是什麼意思.
我只想知道.
你到底是什麼意思.
我只想知道.
你到底是什麼意思.
我只想知道.
你到底是什麼意思.
我只想知道.
你到底是什麼意思.
我只想知道.
你到底是什麼意思.
我只想知道.
你到底是什麼意思.
我只想知道.
你到底是什麼意思.
我只想知道.
你到底是什麼意思.
我只想知道.
你到底是什麼意思.

$^{201}$我只想知道.
你到底是什麼意思.
我只想知道.
你到底是什麼意思.
我只想知道.
你到底是什麼意思.
我只想知道.
你到底是什麼意思.
我只想知道.
你到底是什麼意思.
我只想知道.
你到底是什麼意思.
我只想知道.
你到底是什麼意思.
我只想知道.
你到底是什麼意思.
我只想知道.
你到底是什麼意思.
我只想知道.
你到底是什麼意思.
我只想知道.
你到底是什麼意思.
我只想知道.
你到底是什麼意思.
我只想知道.
你到底是什麼意思.
我只想知道.
你到底是什麼意思.
我只想知道.
你到底是什麼意思.
我只想知道.
你到底是什麼意思.
我只想知道.
你到底是什麼意思.
我只想知道.
你到底是什麼意思.
我只想知道.
你到底是什麼意思.
我只想知道.
你到底是什麼意思.

$^{241}$我只想知道.
你到底是什麼意思.
我只想知道.
你到底是什麼意思.
我只想知道.
你到底是什麼意思.
我只想知道.
你到底是什麼意思.
我只想知道.
你到底是什麼意思.
我只想知道.
你到底是什麼意思.
我只想知道.
你到底是什麼意思.
我只想知道.
你到底是什麼意思.
我只想知道.
你到底是什麼意思.
我只想知道.
你到底是什麼意思.
我只想知道.
今天我們嘗試去做一個舊的題目.
但是我們用一個新的角度去理解.
特別是能夠回應我們時代的一個向度裡面.
所以我們今天就會去探討.
講一些我自己信主的時候的一些見證.
那時候我是十八歲信耶穌.
那時候我很記得我信耶穌之前.
我經常都有一個習慣.
就是我想重新再開始.
我是從五年級開始信耶穌.
那時候很明顯很記得這句話.
有人做基督徒.
他就是新來的人.
舊事已過都變成新的了.
這是每一個信主的人.
信主的時候很重要的經文.
周培的人跟你說.
你現在是信耶穌了.
基督裡面.

$^{281}$你有一個新的生命.
你有新的開始.
那時候我有一個習慣.
就是每次我很不開心的時候.
我就會去洗澡.
不知道為什麼.
你是不是很習慣.
洗完澡之後.
我重新很清新的感覺.
我好像重新開始一樣.
就是一個很舒服.
洗完頭洗完身都乾透了.
有一個很新的開始的感覺.
但發現原來是不行的.
那時候很大的感覺就是.
當我去信耶穌之後.
我就是一個重新的開始.
我是一個罪人.
我以前是一個很不開心的人.
一個失喪的人.
我現在在基督裡面.
又重新開始.
這是我自己在很多年前.
信主的時候一個很大的震撼.
就是我知道在基督裡面.
我重新再開始.
但不知道這個是不是你的感覺.
當你信主這麼久的時候.
你是不是一個新做的人呢.
可能你信主十幾年.
信主二十年.
二十年前剛剛決志的你.
和今天的你.
其實是不是都這麼新呢.
如果二十年前的新做的人.
今天是不是都已經舊了呢.
這個其實我今天很有興趣問這個問題.
什麼叫做新做的人.
什麼叫做一個重新開始的人.
當我們想主這麼久.

$^{321}$當我們面對著自己的信仰.
這麼多年在教會裡面經驗的時候.
我們怎麼來保持.
或者去認識我們在聖經裡面的一個英雄.
就是那個《後書聯經文》所說的.
一個新做的人.
所以今天我們就會思考一個.
很簡單的課題.
都是一個大家聽過的課題.
就是有關新做的人這個字.
如果我們去看回聖經的時候.
你發現聖經裡面這個新做的人.
其實你發現希臘文裡面.
就是Kanei,就是Christie.
這個字就是解作新的創造.
所以如果你是認真看經文的時候.
那個原文裡面其實就不是.
正確地和本所說的新做的人.
我發覺某一段的中文聖經.
是比較貼切的.
因為基本上是沒有人字在裡面.
有個新字.
和這個Christie這個字是解作創造.
就是Creation.
所以如果你是嚴格去認識這個創造.
新做的人的話.
其實更好的翻譯是新的創造.
它叫做New Creation.
所以NIV的版本其實翻譯得挺好的.
所以如果我們去校正方向.
用New Creation來理解經文.
或者我們的身份的時候.
今天我們就會發覺一個.
看得很多的東西.
更加闊的概念.
來理解我們這個新做的人的狀況.
我們就會這樣.
說的是一方面面對一個新環境.
我們面對的香港.
或者是海外裡面移民之後.

$^{361}$面對一個新生活.
這個也是和新字有關係的.
所以我們所謂的新生活.
就是用一個新做人.
New Creation來重新思考這個課題.
如果你去看保羅的神學的時候.
特別是這個學者.
Hans Dezsibis.
這個我很喜歡的一個.
美國籍的德裔的聖經學者.
他很簡單地概括了保羅.
在特別是加拿大書裡面.
或者是整個神裡面.
一個很重要的拯救的四個步驟.
很明顯第一個.
Death and Resurrection of Christ.
基督耶穌的死和復活.
這個是整件事情的第一個步驟.
第二.
第二就是.
Putting on of and dying and rising with Christ in baptism.
這裡所說的是披帶基督的意思.
就是剛才所說的.
我們和基督同死同復活.
這個是使用洗禮.
來重新去carry住基督耶穌的生命.
死過再活過來.
是這樣的開始.
然後第三.
就是有關聖靈的尼林.
The gift of the spirit of God.
就是聖靈的尼林.
第四個步驟.
就是一個更加長遠的.
就是Living in a new creation.
這個也是大家和耶穌的關係裡面.
或者有關德教的秩序.
我們如何去理解.
我們在保羅的世界裡面所說的.
德教的秩序.

$^{401}$有什麼步驟會高呼過他呢.
在整個德教的事件上.
很明顯第四個.
一個是非常長遠的.
從你決志之後.
洗禮之後.
都是這樣的生活.
就是Living in a new creation.
你可以叫做上帝的言論的新生命.
所以那本書是很對的.
這個新生命.
New life.
新的生命.
新生命.
不僅是重生.
不僅是在基督裡面.
有一個重生的New creation.
更加是你開展一個新的生命.
生活.
所以Live的字是解作生活.
生活也可以.
所以不僅是擁有一個新的生命.
更加是要按著這個新的生命.
重新來生活.
這個就是我想大概.
在那本綠色的書裡面.
那八個來源.
就是要去.
決志.
奉獻.
有團契.
敬拜之類的.
今天我們.
當我們新屬二十年之後.
我們很熟悉那些東西.
初順.
成人百貨.
新生命新生活那些東西.
我們都很熟悉.
我們如何去理解.

$^{441}$當下我們的新生命.
新生活呢.
特別是面對著我們.
很contextualize的處境.
今天我們香港.
或者你離開了香港之後.
一個新的環境裡面.
其實是有關係的.
所以當我們去.
看得出.
這樣來理解這個新生命的時候.
你就會發覺.
在我們的信仰當中.
是一個這樣的角色.
有關這個新族的人.
或者這個新生命經文.
其實有兩段.
達比蘇奇這段.
是否說不出後書的第五章經文.
有人在基督里.
他就是新造的人.
舊事已過都變成新的了.
一切都是出於上帝.
他借著基督使我們與他和好.
又將勸人與他和好的積分.
遲及我們.
這就是上帝在基督里.
叫世人與自己和好.
不將他們的過犯歸到他們身上.
並且將自己和好的道理托付了我們.
這些大家很熟悉的經文.
在基督裡面的人.
就是新造的人.
舊事已過了.
都變成新的了.
這句話可能在大家.
在《說人書》之前.
是一個非常大的震撼.
很多以前不好的事情.
未信主之前舊的事情.

$^{481}$遺憾事.
很多犯罪的事.
都能夠移過一個新的事情.
奈何當你信主之後.
發覺很多新的事.
都變成一些舊的事.
你曾經.
在教會很多很雀躍的地方.
一個很新的體驗.
靈命.
信仰.
崇拜.
很多都是一些新的東西.
但發覺回到20年.
10年之後.
很多新的東西.
就再也不是新的.
我們會探討這個課題.
究竟這個新字是什麼意思呢.
這是大家比較熟悉的經文.
其實有一段的.
大家知不知道.
另外一段有關新造的.
有一段經文.
就是在這個.
加泰書的經文.
加泰書第六章.
最後一章的經文.
但我不斷以別的誇口.
只誇我們耶穌基督的十字架.
因這十字架就我而論.
世界已經釘在十字架上.
就世界而論.
我已經釘在十字架上.
受過禮不受過禮都無關要緊.
要的事就是作新造的人.
可能大家不是很記得這些經文.
只記得之前的一些.
世界被釘在十字架上.
這些經文.

$^{521}$後面保羅做了一個總結.
一個application.
God like不God like不重要.
重要的是你要做一個新造的人.
你需要是一個.
God和不God無關.
But new creation.
這個就是保羅在初期裡面.
提出new creation這個概念.
一個這樣的經文.
所以我今天就會去.
詳細地去思考.
什麼叫做new creation.
new creation對我們今天.
面對著自己的屬靈生命.
或者作為full church群體.
我們面對著今天香港.
或者我們海外.
開展新生活.
我們有什麼意義呢.
好了,我們就來看看.
不過如果你去.
認真地去想.
new creation.
其實就不是那個新造的人.
不是完全一樣.
new creation是一個更加.
廣闊一點的神的概念.
他在說的是上帝新的創造.
當你說到創造的時候.
他就不是純粹在說.
你重新做人.
所以新造的人不是.
純粹告訴你.
你重新做人.
這個對的.
這個不單止是這樣.
上帝的救贖工作.
其實是一個更加宏觀的工作.
他不單單是全新的你.

$^{561}$重新來過.
這樣開始做人.
而是說整個創造的重新開始.
所以這個字你會想起什麼呢.
就想起創世紀.
創世紀裡面有一個creation.
創造.
六天的創造.
而去到新約的時候.
在基督裡面.
有一個new creation.
所以你會發覺.
是這樣的對比.
所以新造的人.
不是重新做人.
沒錯.
你二十年前重新做人.
你出了字.
做基督徒想做人.
但是所牽涉的.
不單單是你重新再做人.
再來過.
會講這麼簡單.
不是一個純粹個人層面的事情.
而是一個上帝對於這個世界.
的一個救贖工作.
所以我們今天就去看看.
new creation.
究竟我們用什麼的.
當詞去理解new creation呢.
其中一個就是這樣.
你發現.
就是這個.
它關乎於猶太的啟示文學.
讓我先弄弄這個.
好.
OK,不要弄它.
在這裡.
其實保羅.
當他在他的.

$^{601}$出版本裡面.
講new creation的時候.
其實是關於一些猶太.
猶太教.
在兩天之間.
或者在教育裡面.
所講的一些啟示文學的根據.
其實猶太教一直都有講new creation的概念.
最少在以前的書裡面都提過.
這是比較在正典裡面出現的經文.
我們來看一下.
他說看啊,我做生天生地.
從前的事不再被紀念.
也不再追想.
你們當因我所做的永遠歡喜快樂.
因為我做夜宵蘭為人所喜.
做其餘的居民為人所樂.
大家都聽過.
我當初都講過幾次.
新夜老撒冷.
新出埃及.
我要做一件新事.
開新的江河.
其實裡面都很多這些經文.
上帝要做一個新天新地.
一個new creation.
不再是舊約.
創世紀以前的creation.
而是.
耶穌說上帝會重新來更新這個世界.
重新來做一個新天新地.
所以是一些中末的味道.
有一些天啟文學的味道.
如果你再看多一些時候.
看多一些猶太教的小小的.
其他的經卷的時候.
你會發現.
譬如在這個以色列的十六章.
雖然在聖經裡面沒有.
但都是一些猶太教經典.

$^{641}$這裡就沒有中文翻譯.
他說.
We are done after death.
As soon as everyone of us.
yield up his soul.
We shall be kept in rest.
Until those times come.
When we renew the creation.
所以耶和臥上是會重新來更新這個創造.
或者在八六書裡面.
都說.
And a new world is coming.
所以當保羅說到新造人的時候.
其實不是說我們怎麼重新再來過這麼簡單.
而是說整個世界.
從舊約開始.
耶和臥上在以色列文所應許的.
他將要更新這個世界.
重新來創造一次.
所以這個我今天不說太深.
究竟是一個restoration.
還是一個重新再來過.
還是一個什麼的創造.
但發現當保羅說到new creation的時候.
其實不是純屬一個個人層面的事情.
是一個從舊約開始.
一直都有這樣的背景的事情.
所以當保羅說到new creation的時候.
其實羅馬書也是這樣說的.
當羅馬書說.
我想現在的苦楚.
要比起將來要顯得我們的榮耀.
就不足為價而了.
受造之物.
切忘等候上帝眾子的顯出來.
因為受造之物伏在胸之下.
得向上帝榮耀.
這個受造之物就是那個creation.
一切的creation.
將要等到上帝將來的臨臨.

$^{681}$那個榮耀的時候.
就會被更新.
所以無論是在哥林多後書第五章.
還是在加拉太書第六章.
還是在羅馬書第八章也好.
這三個問題告訴我們.
保羅所說的new creation.
一個更加宏觀.
一個更加有猶太教背景.
一個綜活裡面這樣的一種視野.
所以說了這麼久.
究竟這個有什麼關係呢.
我們經常說.
十字架不是一個個人得救的方法.
當基督在十字架上.
祂是叫整個世界得以改變.
這是一個更加完整的福音.
一個更加準確的美術福音.
得救的不是你自己.
以前怎麼不開心.
然後怎麼可以被拯救.
而是整個世界都借著基督十字架.
去被更新改變.
譬如這句經文.
叫我們知道他知的奧秘.
要照他所安排的.
在日期滿足的時候.
在天上地上.
一切所有的都在基督裡面同歸於一.
一切全部的都在基督耶穌裡面.
重新來到再被拯救.
如果你有聽我們播出那首和平.
Evangelist的詩歌.
他講的一樣東西.
十字架全部的和平.
都在基督裡面同歸於一.
所以當我們去理解的時候.
其實基督耶穌的十字架.
基督女.
你成為新生命的人.

$^{721}$New Creation.
不是說你重新再開始.
做基督徒.
一個浪子的比喻.
這樣的方法.
而是整個世界的改變.
所以我們應該是這樣看的.
我們用一個世界的水平.
來去明晰這個叫做New Creation.
這個神話概念.
所以你都知道.
基督耶穌的復活.
基督耶穌的死.
然後復活.
這個復活是叫整個世界.
成為一個New Creation的開始.
他是一個初熟的果子.
一個投生的果子.
他是第一個復活的.
而一切基督裡面都成為復活裡面的開始.
所以復活節應該是開心的.
因為他講的是一個.
全世界能夠重新被更新被創造.
所以發現復活節是在春天出現的.
他特意強調這個世界得以一個.
重新的開始.
一個New Creation.
任何人在基督裡面都能夠不再被罪.
和被這個世界的權並和死亡來克制.
一個重新的創造.
所以我們拿著這種視野.
我們就明白當我們去理解新造的人的時候.
其實你應該從一個世界來做對比.
保羅也是.
當他每一次講New Creation的時候.
其實他是在講一個世界和New Creation之間的張力和改變.
所以為什麼會說當你今天面對香港的時候.
你是應該或者可以用New Creation來理解今天的香港.
或者理解今天你面對一個新的移民國家.
你的新生活.

$^{761}$Anyway都是世界.
如果說在基督裡面是一個新的創造的時候.
不在基督裡面是什麼.
就是一個舊的創造.
一個Old Creation.
Not in Christ.
In the Old Creation.
世界不是邪惡的.
你都知道楊方所講的.
上帝愛世界.
所以世界本身是美好的.
所以世界是有秩序的.
是美麗的.
不過在保羅的文獻裡面.
世界被形容為一個罪惡和死亡的一個展示.
所以保羅的文獻裡面.
世界是一些負面的字詞.
本來是好的.
但它是充滿著罪惡.
和被死亡.
承載著死亡和罪惡.
所以這樣說.
無論在《光陰厚書》也好.
《網絡學書》也好.
世界都是一些不好的東西.
但有證據的.
大家一起看看.
如果我們去看經文的時候.
我們很多時候都是忽視了.
基督裡面就是新造的人.
一個New Creation.
其實後面的話.
上帝在基督里.
我特意改了翻譯.
和本叫世人與自己和好.
記住原文裡面不是世人.
是Cosmos.
就是世界.
所以和本是很搞事的.
全部把人都出來了.

$^{801}$其實不是世人是世界.
所以保羅說.
我們是一個New Creation.
我們作為New Creation.
在基督裡面叫世界與自己和好.
所以保羅一直在比較什麼.
在比較世界和New Creation.
兩個都是非常龐大的事物.
Old Creation 世界 Cosmos.
和New Creation.
所以你看到.
我特意要把它作為一個比較.
先不說這個.
就是一個Cosmos.
一個紅色的有罪的世界.
還有保羅說.
在基督裡面一個新的創造.
一個新的世界.
兩個這樣對比.
這個也是.
無論是在哥倫多後書的第五章.
還是在卡塔遜的第六章.
都是一樣.
都是世界.
保羅說什麼.
我是一個在基督裡面.
十字架的緣故.
對我來說.
世界已經被釘在十字架上.
對世界來說.
我已經被釘在十字架上.
所以保羅當他在談論這個世界的時候.
世界是一個被釘在十字架上的物體.
然後說我們是一個新的創造.
一個New Creation.
所以明白.
每一次保羅說New Creation的時候.
都是在比較什麼.
就是那個Cosmos.
就是那個世界.

$^{841}$OK.
我們要明白這一點.
因為你沒有那麼宏觀.
你就不明白後面的那些應用是說香港的.
所以我們明白到保羅的New Creation的視野.
是需要和Cosmos的視野不相比.
那你就明白.
究竟是什麼意思.
一本書就是這樣說.
一本書就是將保羅和政治和New Creation.
去分析出來.
羅馬帝國.
當時保羅身處的羅馬帝國.
正正就是Cosmos的代表.
這個世界是邪惡的.
羅馬帝國也是邪惡.
我不是說羅馬帝國是邪惡.
因為它在世界的一部分裡面.
而羅馬帝國是世界上最象徵意義的代表.
所以保羅究竟是怎麼去看羅馬帝國呢.
這個很多人都在討論.
保羅對於羅馬帝國的討論.
他不是要去推翻它.
也不是去信服它.
明不明白.
保羅不是打算推翻它.
但不打算信服它.
因為羅馬帝國正正是屬於世界裡面.
一個很重要的部分的時候.
其實它也是被罪和死亡所克制住.
這個就是保羅對於羅馬帝國的立場.
羅馬帝國是一個最典型.
最象徵意義的世界的一個政權.
一個這樣的物體.
所以保羅不是個人而是很憎恨羅馬帝國.
但保羅說的是一個更加宏觀的問題.
就是一個有關cosmos的問題.
所以另外也是.
上帝的角度也是一樣.
上帝的角度從來都不是要去和羅馬帝國對抗.

$^{881}$上帝是沒法比的.
上帝的角度真正要對抗的是這個世界.
anyway 世界都是上帝所愛的.
所以世界都是需要被renew.
需要被更新.
所以明不明白整個的視野和比較.
上帝的角度是.
到很大程度是完全可以更新這個世界.
這個世界本身是好的.
所以要在十字架裡面洗去它的罪和死亡.
而羅馬帝國保羅所面對的政權或者社會.
其實在這個很渺小的地位.
是屬於世界的一部分.
或者是一個世界的代表而已.
所以保羅要去.
他所說的renew.
一個new creation的時候.
其實都是衝著羅馬帝國而講.
不過其實感動的時候就發覺是很渺小的.
這個政權雖然有很多不好的東西.
但是他說其實已經搞定了.
因為基督裡面的十字架.
已經可以叫做有一個new creation.
完全是超過了.
推翻了整個世界的問題.
這個是能夠讓我們.
藉著一個好像不太關事的.
新做的人和香港有什麼關係.
其實是有關係的.
所以如果2000年前.
保羅這樣來理解羅馬帝國的時候.
他用new creation來面對當時的政權的時候.
因為每次在講世界的時候都講new creation的時候.
就明白我們怎麼來理解.
這個我之前講過.
我那篇國安法的那一部我都提過.
我們不要效法這個世界.
不要被這個世界困住.
你不要去confirm它.
你不要聽它說.

$^{921}$你不要fix在那裡.
停不停地讓它.
你不要跟它.
你不要抄它.
你不要學它.
但是你要它更新你的心意.
在政治裡面.
所以同樣一個差不多的道理.
面對著這個世界的時候.
你不要被它改變你.
你不要跟它.
但是你要尋求一個transformation.
你要去做一些新的事情.
因為你是一個new creation.
我們最大的問題是什麼.
其實我們今天是new creation.
不過我們是一個活在舊世界裡面的new creation.
到哪一天.
一個完全中的來臨的時候.
全世界會被更新.
但是作為基督徒.
我們是new creation.
但是我們在一個舊世界裡面.
仍然出現一個這樣的處境.
所以我們不要跟它.
當你在這個世界裡面.
如果你沒有去發現自己的身的時候.
你就會被這個世界跟隨著.
就會被同化了.
好了,我不說了.
大家聽回哪裡.
這個講回哪裡.
有關這個不要confirm這個世界.
和一個更新的問題.
所以我們如果去這樣講的時候.
有幾點我們可以值得來一個總結.
第一.
就是我們可以重新來定義.
什麼叫做新生活.
什麼叫舊,什麼叫新.

$^{961}$我們用一個更加高的層次去理解今天的新香港.
什麼叫做新,什麼叫舊.
你會發現真正的新.
其實不是在這個世界裡面出現.
這個世界是一個羅馬帝國的old creation.
這個世界有多少新鮮滾熱辣的新常態都好.
其實這些都是舊的東西.
在神眼中.
在神眼中真正的新.
不是這些所謂新鮮的事情.
不是一些新的轉變.
或者新的政權移交.
或者你去了新的地方裡面生活.
而是那個新字在你的生命裡面.
多於在生命的外面生活.
這個很重要.
這個雖然是把definition這麼說.
你其實都感覺不到的.
但是你要知道這個definition.
就是真正的新不是在這個我們面對的新環境.
新處境.
而是那個新.
在神的眼中.
在我們基督裡面的新創造.
那些是舊的東西.
正如我所說.
所謂日光之下無新事.
這個也是傳說中所說.
我在日光之下.
沒有什麼是新的事情.
政權又是一個政權.
古代很多的政權的問題.
今天仍然存在.
很多事情來來去去都是這樣.
所以你發現.
這是一些新的改變.
但發現那些都是一些舊的東西.
那些是日光之下的東西.
沒有什麼特別新的現象出現.
所以我們可以prepare to be surprised.

$^{1001}$我們是surprised的.
我都知道.
當我要用VPN我都會surprised.
不知為什麼.
沒有那麼多香港人都用VPN.
但這些事情又發覺.
不是太令你那麼驚訝.
我們知道.
這些都是一些舊的東西.
舊的板斧.
都是為了操控.
都是為了某些.
以前出現過的某些問題.
所以縱然是新發現.
但都是一些舊有的歷史.
重復又重復.
所以這些是我們要的基本盤.
我們去重新定義.
什麼叫新什麼叫舊.
我們知道今天所發現的新的東西.
都不是一些新的事情.
反而.
什麼叫新.
新是一個很獨有上帝專有的詞語.
新的事情.
是上帝就是有新的事情.
因為上帝是一個新的上帝.
不是重新來一個吸出來新的意思.
這個新是不會被時間沖淡或流逝.
回答一個問題.
20年前你做了新壽的人.
今天你是否舊了.
想想你自己是否舊了.
對於以前的信仰.
今天是否已經.
起了壽.
已經發覺很多的.
已經滑了牙.
或者已經不同了.
很多以前新的驚喜.

$^{1041}$很多新的事情.
發生了.
唱了新的詩歌都算舊了.
以前我第一次唱贊給傳承人.
你也開心.
今天你不會唱這些歌.
所以不關事.
上帝的新不是被時間的新出現.
而是上帝裡面是歷久常新.
因為新本身就是上帝的屬性.
所以上帝的新.
只能夠在上帝裡面.
不關你一個的時間問題.
所以今天你仍然是一個新的創造.
不是因為你對了決志.
而是因為你仍然是一個新的創造.
為什麼.
新的創造.
新是一個上帝的專有名詞.
新出埃及.
新耶路撒冷.
我要做一件新事.
譬如說唱新歌.
全部都是上帝專有的詞語.
所以將來在天堂裡面.
你不會沈悶.
因為全部都是新的.
新不是因為你沒玩過.
而是因為一個這樣的狀態.
無論如何我也不知道是怎樣.
是一個很新的狀態.
所以今天我們要問的問題就是.
你面對著這個世界.
你面對著今天的香港.
或者你面對著將要移民的新地方.
你是一個新基督徒.
面對著一個舊世界.
還是說一個新世界的舊基督徒.
有很多反省的地方.
究竟你面對著這個世界的時候.

$^{1081}$哪一個是新一點的.
一個不許你去新香港.
但是是一個舊的你去面對這個新香港.
或者說我是一個新的創造.
面對著一些舊的世界.
Cosmos 和新的創造.
很有趣的.
究竟我們面對著這個世界.
是怎樣去理解它呢.
這些所謂的新事.
其實都是舊的東西.
重要的是我們本身一個新的創造.
都是一樣.
就算是移民也好.
面對著一個新的環境.
其實重要的不是這一點.
而是你自己是一個新的創造.
所以就要形容我們.
我們一些比較具體的思考.
我這次的思考會分開兩邊.
一邊就是給我們來香港的人.
一邊就是留下來的人.
因為發覺都是面對著一些新的環境.
同樣一個課題.
兩邊的人可以怎樣面對呢.
第一個是我們面對著新常態.
留下來的人面對著新常態.
我們說這些所謂新的常態.
其實這些不是新的事情.
就算你是面對著一個新生活也好.
你去移民.
你是一個新的環境.
其實這些環境新是對的.
你當然要適應.
但從一個上來說.
我不是教你怎樣去適應新生活.
而是你要記得我們真正新的事情.
不單單是這些新的環境.
而是我們本身一個新的生命.
當你懷著這個新的生命.

$^{1121}$去面對這些新的事情的時候.
你才能夠應對.
所以重點不要放在眼前的環境.
無論是這個香港.
或者是你面對的新國家.
重點是你自己內心的新是怎樣.
所以我們都會說.
我們香港這邊就說不要習慣.
意思就是我們不要被它確定.
一方面我們提醒自己.
不要習慣這個新的香港.
不要覺得這些事情成為了我們日常的事情.
不要成為新的常態.
我們就慢慢習慣.
同時我們在海外的時候.
我們都說要適應環境.
我們要適應新的生活.
要接受當地的社會.
重點是一樣.
重點不是你習慣還是不習慣.
或者是適不適應的問題.
而是你如何來到裡面去面對這個世界.
當然你要習慣一下這些事情.
但這些不是上課題.
我們要習慣的說.
這些不是我最擅長的說話.
但作為一個聖經的人.
我們要知道的就是.
面對這個環境.
你習慣還是不習慣.
重點都不是這些生活問題.
而是你自己內心的生命.
是否有一個新的力量去面對它.
所以重點在這裡.
無論這個世界如何轉變.
那個持續性其實是你的信仰.
那個持續性是你的信仰.
這個在你內心的新生命.
是一個最重要的持續性.
當我們有這個新的眼界的時候.

$^{1161}$我們就能夠去應對很多不同的變化.
因為那個真正的新來自於我們的生命裡面.
所以說得很像.
我今天只有一個命題.
就是我們所謂的新生活.
我們說的基督徒新生活.
不是說你如何去重新適應今天的香港.
這麼簡單.
也不是說你移民之後.
如何去適應新生活.
而是你內心的生命.
是否一個新的創造.
那是什麼呢?這個新創造.
有什麼可以簡單地去描述一下呢?.
這個也是我翻譯的那本書.
默德曼的《豐盛的上帝》那本書.
有一句我講得很好的.
有一個字叫做.
Incipit Vita Nova.
一個叫做開戰新生命.
默德曼說.
什麼叫自由.
有一個很有趣的定義.
自由.
他說自由不是我們能夠做到多少事情.
也不是我們能夠有多少主權.
有多少可能性.
而他的自由是在於.
是你能不能夠引發一些新事物出來.
當你能夠有能力去引發一些新事物的時候.
這個就是真正的自由.
這是一個很特別的定義.
我覺得這個很有啟發性.
當我們去面對著香港.
或者是面對著新的生活的時候.
你的身心的創造就是什麼呢?.
就是上帝給你一種仍然可以引發新事物的力量.
今天的香港好像很多都沒怎麼變化.
或者很多時候都已經是一個緊局.
重點不是你有多少自由可以去做些什麼.

$^{1201}$起碼這個不是定義.
定義就是你有多少能力去做一些新的事情.
這種創造性.
這種在沒有的有的力量.
才是那種自由.
他說每個人都是被誕生者.
每個都是一個被創造.
一個新生的人.
這些未誕生的人都是一個新生命的起始者.
一個新生命的開始者.
每個人的生命都是獨特的.
無論怎麼被基因影響.
無論這個社會怎麼限制.
生命的開端.
卻讓生命有別於任何前來與之後的生命.
生命仍然可以去做一件新事.
在香港裡面.
當然在海外也一樣.
不是說你適不適應.
還是能不能確定.
而是你能不能開始一件新事.
充滿著創意地去開始一件新事.
我想這個可以很實際地去量度.
今天我們是不是一個新創造.
在很多很局限制的事情裡面.
我們可能能夠做新事.
因為我覺得翻牆也是一件新事.
我們可以突破今天的限制.
比較有創意地去做新事.
想不到今天我們會講翻牆這個話題.
所以新生命就是這樣.
我們仍然有一種力量.
去無視外界的限制.
去創造一件新的事情.
最後一個總結.
所謂基督徒的新生活.
不是在乎世界環境是怎樣.
因為這個世界是一些舊的東西.
所以這個世界是怎樣的新常態也好.
或者新環境.

$^{1241}$新的移民國家也好.
這個不是重點.
這個影響不了你的新生活.
我們的新生活.
也不是20年前的那個缺席.
而是什麼呢.
是基督裡面的那種力量.
因為你每天都有一個新的生命.
它能夠讓你對這個世界保持一個新的觀點.
一個新的眼光.
保持一個新的可能.
這個是我們今天在基督裡面.
新生命裡面.
去面對世界一個很重要的思考.
很喜歡這句話.
愛歌的經文.
就是每天精神都是新的.
這個是上帝給我們一個很大的應許.
在基督裡面一個新的生命力量.
我們就能夠有新的觀點.
新的眼光.
新的可能.
我們就祈禱吧.
因為你給我們Full Church弟姐妹.
在這裡今天去思考.
你給我們一個新的生命.
我知道我們這個新的生命.
不是在乎於我們信主連日有多少.
而是我們在你裡面.
得到這份生命力量.
所以我們此刻很多可能還沒有去.
尋索思考和發問的時候.
但是你求你的靈當中去跟隨我們.
讓我們能夠對於今天的環境.
無論是香港還是國外.
我們面對的生活.
我們自己的生命.
讓我們有一個新的開始.
讓我們能夠可以.
有一個力量去想.

$^{1281}$和去開始一件新事.
讓我們能夠不會被時間.
或者是社會的規限所限制.
在這裡永遠發現一個新的事情.
來展現我們在這裡的新生命.
幫助我們Full Church弟姐妹.
都能夠有這個做新事的力量.
奉主命求.
阿們.
Captain Poon.
阿莫基.
我今天聽了很多個新字.
你也是個新屋.
是新邊.
我不知道正在等待上機的人.
聽了很多新字.
有沒有什麼問題.
或者是覺得有些東西棘棘的.
大家可以隨時問問題.
我們看到了.
今天講了很多在舊的世界.
等一下.
好 謝謝.
今天也講了很多在舊的世界.
有什麼新的觀點.
新的眼光.
但是其實我還沒有很明白.
是真的應用.
我相信新的東西不是說.
我們在舊的世界找新的餐廳.
看新的書.
現在還沒有去到那裡.
香港的環境不允許教會的團聚.
用一些新的方法去有團體生活.
其實今天的香港.
或者是一些流散到外地的人.
那些新的東西.
其實在應用上.
有沒有一些實質的例子.
剛才我說了.

$^{1321}$比如面對著香港.
或者說一個比較遠的例子.
這個我曾經說過.
柏林被人分開東德和西德.
東柏林和西柏林.
當時柏林的人民.
其實為了自由.
因為當時蘇聯封了整個東柏林.
那些人就去.
我記得那時候去過柏林.
博物館就說.
這件事是叫人激發創意的一個.
最大的方法.
那時候博物館就展示很多.
人們怎麼從東柏林走到西柏林的方法.
有些人是在車後面車尾.
就運過去西柏林.
有些人就坐中興氣球.
有些人就在對面那裡.
用床單綁一條繩子就滑過去.
有些人就扮警察走過去.
那博物館是很有趣的.
他說當人失去自由的時候.
是最能激發起創意的方法.
所以就說面對著這個Old World.
這個羅馬帝國的時候.
重點不是一些新的常態.
或者一些新的形勢.
新的法律.
而是我們能夠可以去用我們的.
上帝給我們的創意.
新事來做出新事.
重點不是服從還是反抗.
明不明白.
兩個都不是.
是新的事情.
你要麼就要服從這個舊的世界.
或者你走一次反過去.
其實都不是.
而是你做一件新的事.

$^{1361}$一條第三道路.
一個third way去面對它.
所以這個就是精神.
就是我們嘗試能夠去突破框框去做事.
不是服從還是反抗.
而是突破.
所以這是一個例子.
所以我們可以說.
面對著這個世界的時候.
就是說根據信仰上.
將社會和新的創造.
新的創造來連在一起.
所以新的創造.
就是讓我們能夠有這個.
發揮做新事的一種能力.
大家可以試著想想.
什麼可以做出來的事.
無論是教會.
你面對的生活處境.
或者是怎樣都好.
可以有很多的例子.
還有問題嗎.
那邊有.
有問題嗎.
你好.
剛才看到關於這個新創造.
然後你又引用了以蔡阿書.
然後我看到新業路剎那.
所以不止想起創世紀.
想起啟示錄.
然後我就有一個問題.
因為你一直說.
新的創造.
似乎是耶穌基督的死和復活之後.
就已經出現了.
那我理解為.
以蔡阿書說的新天新地.
好像已經出現了.
就是我們已經在新的創造裡面.
那是不是在事實裡面.

$^{1401}$所說的新天新地.
其實都已經出現了.
我就想著.
我就一頭問號.
所以就在想.
我這樣理解是不是有問題.
沒問題.
其實都是最經典的.
已經出現了但還沒出現的那些.
明白嗎.
就是所謂新創造不是新世界.
已經是彰顯.
但又未完全彰顯的狀態.
所以我們有這樣的緊張.
我們的新創造.
是可以有能力.
可以做新事的.
但它在一個古代的世界裡面.
這個就是那個緊張.
又未完全是.
新生地所說的新創造.
或者是所說的天堂的環境.
但又是一個能夠去彰顯.
這一詞最難的.
是彰顯還是呈現.
還是活出.
我們是和新創造有關係的.
我們有可能去.
生活在新創造的生活里.
有一個已然未然的狀態.
就是這樣.
所以就是這種.
這種中活性的緊張.
這樣說的時候.
或者你可以用一個.
道地點.
或者可以用一個.
另一個方向去想的方法.
就是.
我們很多時候比較到.

$^{1441}$我們會上天堂.
是上天堂.
但當我們悼念主土文的時候.
會提到一件事.
願禰的國降臨的時候.
我們這個是主耶穌教導我們的土文.
是上帝的國降臨的時候.
上帝的國.
剛才The Kingdom of God.
也說了一個信息.
就是上帝的國會臨到.
在我們當中去彰顯的時候.
其實不是我們上去.
是上帝的國來到我們當中.
去將祂的能力彰顯.
所以用剛才羅馬書.
帶出的經文的時候.
你會看到.
他叫我們不要確定在這個世界里.
而有的行為.
所以第十四章.
第十五章的羅馬書里.
保羅基本上用了.
羅馬書差不多五分一的篇章.
說一個信徒.
有這個身份而有的生活.
應該怎麼做.
他用了不同的例子去說.
既然你這群不是猶太裔的基督徒.
在一個希羅文化.
羅馬政權之下.
要活出一個新的形態.
的生活形式當中.
我們怎麼可以做到一些事.
所以回應剛才John說的信息.
其實不能夠拿一個方程式.
告訴你.
在哪個地方就做些什麼.
反而就是在你的處境當中.
你應該會有一個新的向度.

$^{1481}$新的mindset去過你的新的生活.
這個就是他在回應上帝.
給我的新的創造.
不知道你能否理解這個向度.
我想問.
我自己覺得香港是一個.
大家很要走同一條路的地方.
如果你有些不同.
很難走.
我想問基督徒的新生活.
如果以個人去做一些事.
或者回應來說.
好像力量很薄弱.
我想問.
在個人和群體的新生活中.
可以如何理解去做到.
對,本來是說群體.
他說群體能夠扶持.
去做一些新的事情.
或者群體裡面.
和這個羅馬帝國有些不同的做法.
有個不同的方式去活.
譬如我們當時Fold Church.
我們整個群體一起.
做一些新的方法.
面對著疫情的時候.
就是這些例子.
所以是一個群體的事情.
大家一起去.
又不是去confirm他.
但又不是去推翻他.
而是嘗試一些新的方法.
去走第三條路.
我們不會跟.
但又不是不跟.
跟一些不跟一些.
這些是一個很好的例子.
所以群體是說大家去做這些事.
所以不是說你自己.
要和社會做些什麼.

$^{1521}$這群人是一起來活一個社會.
教會就是一個新的群體.
和這個世界價值是不同的.
這才是力量.
所以不是你自己去做一些新的行動.
一個人做叫做action.
一個行為.
一群人做就是一個社會.
一個群體.
這樣才能有互動.
更加容易行出來.
一個人做就是一個社會.
有些奇怪的人在地球上.
不知道在做什麼.
但一群人做就是一個社會.
就可以不需要在社會裡面.
不是,是一個小社會.
當時羅伯特說.
教會就是一個小的社群.
他們一群人是能夠生活的.
所以有些小圈子.
叫做教派主義的看法.
但這種教會群體.
正正是大家能夠work到.
生活到.
前面這裡.
其實也有點像.
我不知道是我見識少還是什麼.
我覺得香港有一個特別的情況.
就是香港人的工作.
應該是世界問鼎的其中一位.
我很明白那種.
如何讓自己的心智.
仍然保持著新.
不要磨滅.
但我也很深信.
可能有很多弟兄姊妹.
甚至包括自己也是.
有些位置就是.
你的日常生活已經被佔據得太多.

$^{1561}$以至你仍然能夠.
和某些弟兄姊妹圍爐.
再分享,再希望那團火不要熄滅.
但好像實質上.
真的沒有這樣的空間.
讓你讓自己的生命.
不自達盡於世界當中.
分別出來.
我就想.
是不是以前.
如果說以前在教育里.
那種是他們的.
呼求的方式.
是我們差一點理解.
我們應該怎麼去渴求.
還是我們還沒得到.
那種所謂的new creation.
我們還差什麼.
還有我們可以怎麼在.
我們現在這些信徒群體里.
不只是互相支持.
是真的有一些東西.
在香港這個情況做出來.
我說的情況.
雖然可能局勢是一樣.
但我覺得另一樣就是.
那種工作對我們的生命的捆綁.
在香港是特別一點.
我覺得.
我先回答.
我就稱之為身份教育.
就是你多麼認受.
這是你的身份.
和你多麼認受.
是你的身份而有那種行為表現.
舉個例子.
我怎麼教兩個兒子.
去做好一個學生的身份.
小時候和他們去旅行的時候.
他說爸爸你為什麼還不快點看.

$^{1601}$我快要下機了.
這部戲快要完了.
快點看吧.
因為我在填紙.
下機的時候要填新報表.
我就說這些你自己填吧.
然後我兒子就問我.
occupation在這裡寫什麼.
我說你平時做什麼多.
你平時最主要的時間做什麼.
學生.
你就寫學生.
在這個過程中.
我就教他一件事.
其實你的身份是學生.
所以做功課是你要投入的地方.
在這個過程中.
讓他明白身份而有的行為表現.
剛才你說到環境很差.
或者環境很不就的過程中.
其實我們要做這件事.
是很不容易的.
反而我就會用身份的角色去看.
上帝把我們放在香港.
把我們放在其他地方.
那也是你的身份.
你去到不同地方.
雖然環境改變了.
但你的身份沒有改變.
你應該在你的環境中.
活回你的身份而有的表現.
所以你要有時間和職能.
在當中想辦法去做出來.
對於我來說.
應該要融入在你的日常生活中.
反而如果日常生活中沒有改變.
其實你做哪裡都好.
有沒有改變都好.
其實你未必做到基督徒的本份.
所以我其中一句口頭禪就是.

$^{1641}$「似」即是「不是」.
你似一個基督徒還是不是一個基督徒.
就看你平時有沒有做基督徒應該有的事.
不只是取決於環境.
也不是取決於你的信仰在哪裡.
反而是你自己怎麼去持守.
因為耶穌的緣故.
你成為一個新做的人.
因為耶穌的緣故.
你再一次重拾上帝兒女的身份.
那種職能.
你好,如果有一個基督徒.
他本身很壓抑.
經常都很希望自己去愛神.
但神就告訴他.
你要放鬆自己.
你要做一個新的人.
你要心思跟身而變化.
那他就成為一個很輕鬆很愉快的人.
因為耶穌就提醒他.
凡事都可行.
大不都有益處.
但他就真的很希望在2023年.
決定做自己很有天分.
很喜歡做的事.
但其實這件事.
是世上的小學.
其實都是老舊世界出來的某些東西.
其實作為一個比較熟悉的基督徒.
嚴格來說.
可能有些人就覺得他不是跟身而變化.
而是你去愛你自己的舊我.
去到這裡.
其實他真的很想做一個跟身而變化.
隔身的自己.
其實他已經放下了.
他不再服侍主.
他很開心.
他也很少看聖經.
但也會禱告.

$^{1681}$也感受到神的同在.
他說其實這個真的神給我在2023年.
成為一個隔身的自己.
但其他人不認同他.
反而又為他禱告.
很擔心他的情況出現.
要繼續牧養這個迷途的基督徒.
我想請問.
他其實內心真的覺得神讓他跟身而變化.
但身邊的人這樣說的時候.
他很迷茫.
你會不會給一個建議.
這個迷茫的基督徒.
完全認同.
因為.
我這樣說.
我是很明白這些情況.
剛才這樣.
跟身不代表完美.
我剛才認為不是完美的基督徒.
很多時候仍然是很多的軟弱.
所以這個就是所謂.
你要renew一個new creation的意思.
因為new creation不是從你一個以前的罪人.
變成完美的人.
new creation正正就是一個這樣的狀態.
你需要不斷的去被更新和改變.
而你更新完的那一刻.
其實是一個beta版.
0.幾個version來的.
是新的.
但又不是final的.
所以每個人都是這樣.
更新不代表好了.
是好一點.
或者是另一個perspective去看這件事.
可能都是差的.
也不出奇.
所以更新不代表好了很多.
我們人生裡面就是不斷的被transform.

$^{1721}$最怕的是你已經停在那個位置.
以為我已經完美了.
就是這樣.
所以其實我想.
很多fortune 的人的掙扎.
不只是剛才所說的人的心路歷程.
每個人都是這樣面對自己的生命.
不斷問自己我是否一個新做的人.
好像還是很不行.
其實很多舊的東西.
剛才說的比較多的是社會層面.
但說到屬靈層面多一點也是一樣.
我們仍然有很多問題.
所以我們這個新的創造的意思.
正正不是一個perfect的創造.
所以是基督裡面不斷的被更新.
所以很體會剛才所說的情況.
我的看法是.
姐妹剛才提到的尋求幫助的人.
我通常覺得要多聽幾次之余.
還有不同的人聽聽他的表達方式.
因為很多時候我的反省是.
自己說的都是自己的苦主.
其他人一起聽的時候會給他更多角度.
重點說的是.
如果自己是基督徒是很孤單.
是很不容易的.
不是說一定要圍圍圍大群人一起.
才是一群好的團體.
反而是你將你的難處和其他人分擔.
和其他人分享.
其他人給你的看法.
這個在當中就是聖經提到的.
兩三人聚集的時候上帝就在當中.
就好像上次我們第一堂裡面說的.
上帝與人同在Shakina的意思.
那種會幫我們去分辨.
我們這個群體團結一起的時候.
有什麼向度要我們一起走.
或者是我們跌倒的時候.

$^{1761}$怎麼可以提醒.
所以我覺得剛才的提問是有種苦澀.
或者是不安的.
但是我相信在慢慢一直去接觸.
和不同的人一起去守望或者聽這件事.
其實慢慢會看到有一些共通性.
是大家都可以做的.
說得好像很凶狠.
但是流動的場景是有這些群體慢慢去結集.
然後透過不同的活動或者聚會的時候.
就讓訴說這些狀況的弟妹感受到不孤單.
其實就會看到一個群體慢慢被凝聚.
就是這樣.
另外我想說的是新的視野.
剛才華文說新的視野.
新的眼界去看這個世界.
也是看自己.
所以永遠保持一個新的眼界.
去看自己和信仰是很重要的.
如果多些應用的話.
希望Footswap的人都能夠多些去接受其他人的看法.
從而大家可以更加更新自己的盲點和看法.
這個也是重要的.
對於我們Footswap群體來說.
其實很需要去不斷去.
不要覺得自己是一個最後的產品.
這個就是一個新創作很重要的意思.
不然就是一個舊的完美的產品.
其實都是一些舊的東西.
所以更新的眼光和視野是很重要的.
有沒有其他問題?.
這邊.
我想問一下新創作在.
如果個人生命的應用層面.
其實是否只單單在乎於.
一個跟從基督的道德規範生活的人.
和一些不跟從基督道德規範生活的人.
這麼簡單.
還是其實規範不只是在這方面.
是否跟隨這方面呢?.

$^{1801}$這個問題可以有很多不同的答案.
當然是.
沒理由不跟隨耶穌.
當然是跟隨耶穌.
但什麼叫跟隨耶穌呢?.
我今天所說的就是比較強調將來的可能性.
不單單是回望.
看回方文秀所說的耶穌所做的事.
這當然是.
但很多時候我們所面對的世界.
或者所說的耶穌.
其實不單單是以前.
2000年前在加利利海一帶的耶穌.
而是將來做生事的耶穌.
所以當我們這樣說的時候.
我們今天強調的就是新的可能性.
這都是一些跟隨.
上一集也說過.
跟隨耶穌的方法.
不是回望耶穌.
而是將來的耶穌.
跟隨他也是一種方法.
所以做生事也是跟隨耶穌的方法之一.
所以緊守耶穌的教導當然是.
同時也會因為這樣的緣故.
去做一些生事.
所以這兩個不是懷疑我的事情.
而是兩個都有.
總的來說當然是跟隨耶穌.
或者是一些法則,規則.
但有時候規則不是在聖經裡面寫出來的.
你需要在聖經裡面去跟隨耶穌的話.
就不單單是聖經寫過的東西那麼簡單.
那些是聖經的原則.
但很多事情都是更加在世界裡面.
你需要去面對的處境.
這樣去面對.
而不是在很律法的方法裡面.
就算數了.
或者可以從.

$^{1841}$我們看哥羅西書裡面.
關於基督論的一些內容.
因為哥羅西書裡面帶出一個信息.
就是說我們不分什麼類型的人.
無論是希利里人,維魯人,自主的人.
但接著保羅就提到.
不同身份的人.
我們都會有一些不同的.
所謂的創作.
所以說到要傳揚基督的方法.
就是用詩歌,頌慈靈歌.
不同的方法去做一些方式去傳頌.
這個就不僅僅是一個道德規範的嚴謹.
或者是一個單一向度.
反而就是我們這個身份而有.
我們如何傳揚它的時候.
有很多創作的空間.
或者新的生活形式.
讓人去明白到.
生生活而有的多面性.
有沒有其他問題?.
其實潘Sir每一科都有做教材.
如果你們是在小學就知道.
無論是在網上參加也好.
或者是想有更多具體應用上的問題.
想討論或者談.
其實在那些材料都有.
教材或者今天講的比較理論神學多一些.
但如果很多想討論的應用.
其實在教材裡面.
是有些問題.
讓大家去思考大家的生活和生命.
和分享.
所以都建議大家.
都可以去參加或者用這些資料.
差不多登機了.
預告一下.
下次是四月.
都是四月最後一星期.
下一次就是我們的航班.

$^{1881}$就是FC4.
FC4是多少?.
FC428.
FC428航班.
邀請大家登機時間是八點鐘.
今天的題目就是.
是他也是你和我.
基督徒的身份認同.
我們下個星期下個月再見.
拜拜.
字幕志願者:劉文英.
主持人:劉文英.
(字幕志願者:劉文英).
\newpage



\section{馬太福音 26:30-34-20230401}
\label{sec:8KdYgVn_hzk}
\textbf{【流堂崇拜】有關跌倒前的三件事|馬太福音26\_30-34|20230401 [8KdYgVn-hzk]}
\newline
\newline
連結: \href{https://youtube.com/watch?v=8KdYgVn-hzk}{\texttt{ https://youtube.com/watch?v=8KdYgVn-hzk}} ~~~~ 語音日期: 2023-04-01 
\newline
\newline
\hyperref[sec:7ZGXT0f30Z0]{\small{< < < PREV SERMON < < <}}
~
\hyperref[sec:index_chronic]{\small{[返順時目]}}
~
\hyperref[sec:index_scriptual]{\small{[返順卷目]}}
~
\hyperref[sec:v4hE6GM4QsI]{\small{> > > NEXT SERMON > > >}}
\newline
\newline
馬太福音 26:30-34-20230401
\newline
\begin{longtable}{cl}
\hline
\hline
章節 & 經文 (和合本修訂版)\\
\hline
26:30 & \begin{tabularx}{0.7\textwidth}{X} 他們唱了詩,就出來往橄欖山去。 \end{tabularx} \\ \\ \relax
26:31 & \begin{tabularx}{0.7\textwidth}{X} 那時,耶穌對他們說:「今夜,你們為我的緣故都要跌倒。因為經上記著:『我要擊打牧人,羊就分散了。』 \end{tabularx} \\ \\ \relax
26:32 & \begin{tabularx}{0.7\textwidth}{X} 但我復活以後,要在你們之前往加利利去。」 \end{tabularx} \\ \\ \relax
26:33 & \begin{tabularx}{0.7\textwidth}{X} 彼得回答他說:「即使眾人為你的緣故跌倒,我也絕不跌倒。」 \end{tabularx} \\ \\ \relax
26:34 & \begin{tabularx}{0.7\textwidth}{X} 耶穌說:「我實在告訴你,今夜雞叫以前,你要三次不認我。」 \end{tabularx} \\ \\ \relax
26:35 & \begin{tabularx}{0.7\textwidth}{X} 彼得說:「我就是必須和你同死,也絕不會不認你。」所有的門徒都是這樣說。 \end{tabularx} \\ \\ \relax
26:36 & \begin{tabularx}{0.7\textwidth}{X} 耶穌和門徒來到一個地方,名叫客西馬尼。他對他們說:「你們坐在這裡,我到那邊去禱告。」 \end{tabularx} \\ \\ \relax
26:37 & \begin{tabularx}{0.7\textwidth}{X} 於是他帶著彼得和西庇太的兩個兒子同去。他憂愁起來,極其難過, \end{tabularx} \\ \\ \relax
26:38 & \begin{tabularx}{0.7\textwidth}{X} 就對他們說:「我心裡非常憂傷,幾乎要死;你們留在這裡,和我一同警醒。」 \end{tabularx} \\ \\ \relax
26:39 & \begin{tabularx}{0.7\textwidth}{X} 他就稍往前走,俯伏在地,禱告說:「我父啊,如果可能,求你使這杯離開我。然而,不是照我所願的,而是照你所願的。」 \end{tabularx} \\ \\ \relax
26:40 & \begin{tabularx}{0.7\textwidth}{X} 他回到門徒那裡,見他們睡著了,就對彼得說:「怎麼樣?你們不能同我警醒一小時嗎? \end{tabularx} \\ \\ \relax
26:41 & \begin{tabularx}{0.7\textwidth}{X} 總要警醒禱告,免得陷入試探。你們心靈固然願意,肉體卻軟弱了。」 \end{tabularx} \\ \\ \relax
26:42 & \begin{tabularx}{0.7\textwidth}{X} 他第二次又去禱告說:「我父啊,這杯若不能離開我,必須我喝,就願你的旨意成全。」 \end{tabularx} \\ \\ \relax
26:43 & \begin{tabularx}{0.7\textwidth}{X} 他又來,見他們睡著了,因為他們的眼睛困倦。 \end{tabularx} \\ \\ \relax
26:44 & \begin{tabularx}{0.7\textwidth}{X} 耶穌又離開他們,第三次去禱告,說的話跟先前一樣。 \end{tabularx} \\ \\ \relax
26:45 & \begin{tabularx}{0.7\textwidth}{X} 然後他來到門徒那裡,對他們說:「現在你們仍在睡覺安歇嗎?看哪,時候到了,人子被出賣在罪人手裡了。 \end{tabularx} \\ \\ \relax
26:46 & \begin{tabularx}{0.7\textwidth}{X} 起來,我們走吧!看哪,那出賣我的人快來了。」 \end{tabularx} \\ \\ \relax
26:47 & \begin{tabularx}{0.7\textwidth}{X} 耶穌還在說話的時候,十二使徒之一的猶大來了,還有一大群人帶著刀棒,從祭司長和百姓的長老那裡跟他同來。 \end{tabularx} \\ \\ \relax
26:48 & \begin{tabularx}{0.7\textwidth}{X} 那出賣耶穌的給了他們一個暗號,說:「我親誰,誰就是。你們把他抓住。」 \end{tabularx} \\ \\ \relax
26:49 & \begin{tabularx}{0.7\textwidth}{X} 猶大立刻進前來對耶穌說:「拉比,你好!」就跟他親吻。 \end{tabularx} \\ \\ \relax
26:50 & \begin{tabularx}{0.7\textwidth}{X} 耶穌對他說:「朋友,你來要做的事,就做吧。」於是那些人上前,下手抓住耶穌。 \end{tabularx} \\ \\ \relax
26:51 & \begin{tabularx}{0.7\textwidth}{X} 忽然,有一個和耶穌一起的人伸手拔出刀來,把大祭司的僕人砍了一刀,削掉了他一隻耳朵。 \end{tabularx} \\ \\ \relax
26:52 & \begin{tabularx}{0.7\textwidth}{X} 耶穌對他說:「收刀入鞘吧!凡動刀的,必死在刀下。 \end{tabularx} \\ \\ \relax
26:53 & \begin{tabularx}{0.7\textwidth}{X} 你想我不能求我父,現在為我差遣比十二營還多的天使來嗎? \end{tabularx} \\ \\ \relax
26:54 & \begin{tabularx}{0.7\textwidth}{X} 若是這樣,經上所說事情必須如此發生的話怎麼應驗呢?」 \end{tabularx} \\ \\ \relax
26:55 & \begin{tabularx}{0.7\textwidth}{X} 就在那時,耶穌對眾人說:「你們帶著刀棒出來抓我,如同拿強盜嗎?我天天坐在聖殿裡教導人,你們並沒有抓我。 \end{tabularx} \\ \\ \relax
26:56 & \begin{tabularx}{0.7\textwidth}{X} 但這整件事的發生,是要應驗先知書上的話。」那時,門徒都離開他,逃走了。 \end{tabularx} \\ \\ \relax
26:57 & \begin{tabularx}{0.7\textwidth}{X} 抓耶穌的人把他帶到大祭司該亞法那裡去,文士和長老已經在那裡聚集。 \end{tabularx} \\ \\ \relax
26:58 & \begin{tabularx}{0.7\textwidth}{X} 彼得遠遠地跟著耶穌,直到大祭司的院子,進到裡面,就和警衛同坐,要看結局怎樣。 \end{tabularx} \\ \\ \relax
26:59 & \begin{tabularx}{0.7\textwidth}{X} 祭司長和全議會尋找假見證控告耶穌,要處死他。 \end{tabularx} \\ \\ \relax
26:60 & \begin{tabularx}{0.7\textwidth}{X} 雖然有好些人來作假見證,總找不到實據。最後有兩個人前來, \end{tabularx} \\ \\ \relax
26:61 & \begin{tabularx}{0.7\textwidth}{X} 說:「這個人曾說:『我能拆毀神的殿,三日內又建造起來。』」 \end{tabularx} \\ \\ \relax
26:62 & \begin{tabularx}{0.7\textwidth}{X} 大祭司就站起來,對耶穌說:「這些人作證告你的事,你甚麼都不回答嗎?」 \end{tabularx} \\ \\ \relax
26:63 & \begin{tabularx}{0.7\textwidth}{X} 耶穌卻不言語。大祭司對他說:「我指著永生神命令你起誓告訴我們,你是不是基督—神的兒子?」 \end{tabularx} \\ \\ \relax
26:64 & \begin{tabularx}{0.7\textwidth}{X} 耶穌對他說:「你自己說了。然而,我告訴你們,此後你們要看見人子坐在權能者的右邊,駕著天上的雲來臨。」 \end{tabularx} \\ \\ \relax
26:65 & \begin{tabularx}{0.7\textwidth}{X} 大祭司就撕裂衣服,說:「他說了褻瀆的話,我們何必再要證人呢?現在你們已經聽見他這褻瀆的話了。 \end{tabularx} \\ \\ \relax
26:66 & \begin{tabularx}{0.7\textwidth}{X} 你們的意見如何?」他們回答:「他該處死。」 \end{tabularx} \\ \\ \relax
26:67 & \begin{tabularx}{0.7\textwidth}{X} 他們就吐唾沫在他臉上,用拳頭打他,也有打他耳光的, \end{tabularx} \\ \\ \relax
26:68 & \begin{tabularx}{0.7\textwidth}{X} 說:「基督啊,向我們說預言吧!打你的是誰?」 \end{tabularx} \\ \\ \relax
26:69 & \begin{tabularx}{0.7\textwidth}{X} 彼得在外面院子裡坐著,有一個使女前來,說:「你素來也是同那加利利人耶穌一起的。」 \end{tabularx} \\ \\ \relax
26:70 & \begin{tabularx}{0.7\textwidth}{X} 彼得在眾人面前卻不承認,說:「我不知道你說的是甚麼!」 \end{tabularx} \\ \\ \relax
26:71 & \begin{tabularx}{0.7\textwidth}{X} 他出去,到了門口,又有一個使女看見他,就對那裡的人說:「這個人是同拿撒勒人耶穌一起的。」 \end{tabularx} \\ \\ \relax
26:72 & \begin{tabularx}{0.7\textwidth}{X} 彼得又不承認,起誓說:「我不認得那個人。」 \end{tabularx} \\ \\ \relax
26:73 & \begin{tabularx}{0.7\textwidth}{X} 過了不久,旁邊站著的人進前來,對彼得說:「你的確是他們一夥的,你的口音把你顯露出來了。」 \end{tabularx} \\ \\ \relax
26:74 & \begin{tabularx}{0.7\textwidth}{X} 彼得就賭咒發誓說:「我不認得那個人。」立刻雞就叫了。 \end{tabularx} \\ \\ \relax
26:75 & \begin{tabularx}{0.7\textwidth}{X} 彼得想起耶穌所說的話:「雞叫以前,你要三次不認我。」他就出去痛哭。 \end{tabularx} \\ \\
[1ex]
\hline
\hline
\end{longtable}
$^{1}$頂姐妹平安.
在網上的頂姐妹平安.
這篇道是我以前曾經說過的道.
如果在很多年前.
大概在六七年前 七八年前.
我曾經作為外來港元.
去過理教會講道的話.
可能我曾經在這裡說過這篇道.
當然這篇道我重新再寫過一次.
是一個Full Church的Remake版本.
重新說這篇道的原因.
是因為這星期是受難節前的一個星期.
今天的講道經文正正就是.
在耶穌受難前的一段故事.
一段有關彼得三次不認主.
發生之前的一段故事.
一段耶穌和彼得之間的對話.
所以這篇道很適合在受難節前的一個星期.
來說的篇道.
這篇道當然是有關Sorry這個題目.
正如上星期梁國權老師所說.
Sorry可以分為兩大類.
一個是跟人說的Sorry.
一個是跟上帝說的Sorry.
今次我們會說有關跟上帝說的Sorry.
特別是跟耶穌說的Sorry.
不知道大家有沒有試過跟耶穌說Sorry.
可能比較少.
跟耶穌說Sorry好像有點奇怪.
一般來說我們跟神說Sorry.
我們稱之為認罪.
認罪祈禱大家都應該試過.
跟上帝說對不起.
或者認罪對我們新教徒來說.
其實是一件很抽象的事.
我們都知道耶穌已經赦免了我們的罪.
所以我們每一次禱告認罪的時候.
似乎都是一件不太功能的事情.
天主教會覺得這件事是懺悔告解.
真的有用的.

$^{41}$懺悔完告解之後.
你的罪能夠得到赦免.
會比較荒謬.
我們基督教新教認罪祈禱.
就不是那麼功能性.
我覺得有時候是我們的功能性比較好.
起碼我們覺得懺悔告解之後.
你去潘相面前告解.
有人去寬恕你.
你覺得會比較實在.
所以我們會一起去思想.
跟耶穌說Sorry.
我們跟神的關係.
透過彼得和耶穌說的一段說話.
一段真論.
來反省我們跟耶穌的關係.
我們一起去看一段經文.
就是《馬利福音》第26章30-34節經文.
我們一起去聽我讀.
他們唱了詩就出來往橄欖山去.
那時耶穌對他們說.
今夜你們為我的緣故都要跌倒.
因為經上說我要擊打木人.
揚州分散鳥.
但我復活以後要在你們耳先往加利利去.
彼得說眾人雖然為你的緣故跌倒.
我卻永不跌倒.
耶穌說我實在告訴你.
今夜雞叫耳先.
你要三次不認我.
彼得說我就是必須和你同死.
也縱不能不認你.
眾門徒都是這樣說.
我們一起祈禱.
祝福我們一起來敬拜你.
剛才我們全聚的弟子妹在網上.
我們在實體敬拜裡面.
我們都獻上我們真誠的頌讚.
求主你切剋對我們每一個心靈說話.
讓我們每一個全聚弟子妹的生命.

$^{81}$再一次來得以醒察.
我們在受難前的一個星期裡.
我們預備好自己來迎接你受苦的日子.
好讓我們的生命能夠行事為人.
配這個蒙召恩相請.
能夠做一個合理心意的兒女.
求你幫助我們.
幫助孩子.
孩子不配.
求你使用.
蒙主命,求.
阿們.
今天的經文是發生在.
下西馬尼園和最後晚餐.
兩個段落中間的一段事情.
在耶穌被埋的那一夜.
耶穌和他的門徒在耶路撒冷裡.
過了最後一個的雨節.
耶穌和門徒一同坐著.
要說的臨別說話都說了.
晚飯都吃了.
要唱的詩歌都唱了.
因此耶穌和門徒就離開他們晚飯的地方.
一同去下西馬尼園那邊.
正正就在最後晚餐和下西馬尼園中間.
就發生了今天我們看的經文.
就在這個時候.
整個最後晚餐的最後.
耶穌就和他的門徒.
在最後晚餐裡.
說了最後晚餐裡的最後一段話.
這段跟整段的最後晚餐的氣氛.
是很不配合的一段話.
耶穌就和門徒說.
今夜你們為我的緣故都要跌倒.
是一句不容易消化的說話.
聲裡面跌倒這個字.
其實在風書裡面是一個經常用的字眼.
我們都知道你們要跌倒就不是真的跌倒.
跌倒是一個比喻.

$^{121}$跌倒這個字英文就叫做stumble.
就是失去平衡.
幾乎要跌倒的狀態.
不過其實跌倒這個字的希臘文.
其實在風書裡面是一個很強烈的字眼.
給個例子大家.
馬科第五章裡面說過.
耶穌說若是你的右眼叫你跌倒.
就挖出來丟掉.
寧可失去白體中的一體.
不叫全身丟在地獄裡.
十八章第六節.
耶穌也這麼說.
凡使者信我的一個小字跌倒的.
倒不如把大磨石.
撿在人的頸上.
沉在深海裡.
基本上跌倒是一件落地獄的事情.
甚至是推下海這麼大這麼嚴重的事情.
所以在風書裡面.
每一段經文提到跌倒這個字.
其實不是簡單說你跌倒.
或者是說你不小心做錯事.
而是一個失去救恩.
或者是一個非常嚴重的落地獄的程度.
跟耶穌不單止是關係變差了.
而是沒有關係.
是斷絕關係的意思.
所以今天我們聽這篇道之前.
我們要調整好這個字.
跌倒是一個比較淑女的童話.
但跌倒這個字不是純粹說輕輕跌倒.
做錯一點點事.
所以再做好一點.
而是在你生命裡面的一個屬靈危機.
一個很重要很艱難的地步.
所以門徒將要面對這樣的情況.
在今夜.
大概在最後晚餐三四個小時之後.
三年跟耶穌同行建立的信心的關係.

$^{161}$一夜之間就沒有了.
門徒將要翻天覆地推翻他們的信仰.
所以耶穌說今夜你們每一個緣故都要跌倒.
是一句很嚴重的話.
如果你細心留意的話.
耶穌引用了舊約.
撒該利亞書第十三章第七節經文.
來講這句話.
耶穌說我要擊打木人.
羊就分散了.
這裡耶穌將原本撒該利亞書經文.
萬君之和華.
改成「我」這個字.
上帝自己親口的命令.
我要粉碎我的木人.
然後他的羊就分散了.
今晚將要發生的是.
就好像羊沒有了木人一樣.
耶穌首先被擊碎.
然後跟隨他的門徒流離.
跌倒離棄他們的信仰.
就好像一個沒有木人的羊一樣.
我們不是很明白沒有木人的羊是什麼情況.
我們在香港也不太見到羊.
我們經常看到羊在煮飯.
但是羊就很少見.
想想像一下.
在商場裡一個嬰兒推著推著.
突然父母不在.
嬰兒車就在商場三樓停在那裡.
這是一個非常嚴重危機的事情.
所以耶穌說你們每個人今晚將要跌倒.
你們將要失去你們的信仰.
就好像沒有了木人的羊一樣.
耶穌說你們將要為我的緣故跌倒.
是一句不是很適合當時的氣氛.
最少不是很適合當時的farewell.
最後晚餐的說話.
這兩年大家也吃了不少farewell飯.
想想林彪之前你跟鄧兄姊妹的朋友.

$^{201}$在移民之前你們就吃飯.
很開心大家聊天吃飯.
說以前的往事.
吃飯拍照自拍.
林彪突然間的朋友說.
其實我走了之後你也不會再找我了.
這是一段很不適合farewell的說話.
所以耶穌是這樣說.
耶穌在最後晚餐的最後.
說了一句非常不適合最後晚餐的說話.
不是一些珍重道別或是勸勉.
而是一句你們將要離開我.
信仰跌倒的說話.
為什麼耶穌在最後會說一句這麼令人詫異的說話呢.
所以這時候彼得很忍不住.
開口就跟耶穌說了幾句.
開始研究一下耶穌和彼得之間的那段說話.
No, No, No, No, 不會的.
這些事情絕對不會發生的.
眾人雖然為你懸固跌倒.
我卻永不跌倒.
彼得似乎很清楚自己的性格.
也對的,世上最明白自己的就是自己.
我絕對不是這樣的人.
絕對不會得罪上帝.
不會成為渣男水人的那種人.
彼得這樣說.
其實彼得沒有說謊.
最少在那個時候是沒有的.
你問彼得他會不會想這樣做呢.
不認主,出賣主,這些都是不會的.
不會的,我不會跌倒的.
我不會願意跌倒的.
不過不願意是一件事.
但是不是真的跌倒也是另一件事.
所以從這個對話裡面.
我們看到今天我們所說的第一件事.
一個跌倒的人.
在跌倒之前發生的第一件事.
就是不覺得自己會跌倒.

$^{241}$不過也很合理.
一個跌倒的人在跌倒之前.
應該不會覺得自己會跌倒.
PK總是意外.
你想想,如果跌倒之前也跌倒過.
就不會跌倒了.
這是很正常的事情.
不過彼得知道自己會跌倒.
耶穌警告了他.
不過他也跌倒了.
至少彼得不願意這樣做.
我想我們也一樣.
我想沒有哪個人願意做一個非常糟糕的人.
事實上每一個很糟糕的情況.
不是一朝一夕發生的事情.
如果有一條線叫做上帝的標準的話.
我們跟這個上帝標準.
不是一夜之間就突然滑落.
而是我們每一天慢慢地變化.
這樣去遠離這個標準.
離開神,犯罪,得罪上帝.
或者成為一個上帝眼中不合格的人.
得罪我們身邊的人.
我們沒有人願意變成這樣.
至少這不是我們的理想.
世上沒有人立志成為一個討人厭的人.
在我們人生中.
我們往往跟朋友.
跟家人,父母的關係.
往往不知不覺地去到某個境地.
我們成為別人眼中的壞人.
一個不喜歡的人.
得罪了上帝,得罪了人.
為什麼公司旁邊的部門的人.
會當我做仇人.
為什麼連打招呼都變得這麼困難.
為什麼我跟父母的關係.
已經回到一個只回去吃飯的關係.
這是一個長年累月的問題.
甚至我們自己知道.

$^{281}$很多自己生命中的一些問題.
一些的罪.
開始的時候我們都有一點點的罪疚感.
慢慢就開始變得沒有感覺.
然後開始理解為一件正常的事情.
繼續這樣去順.
這都不是我們一開始想的那個計劃.
這些不是我們想要的.
但我們是做了的.
所以有人這樣說.
有人做人的原則這樣說.
其實對得起別人.
對得起自己.
其實單單憑這句說話.
作為我們人生的座右銘.
其實是不足夠的.
每個人都覺得自己對得起自己.
很少人明知道那件事是不對的.
都是這樣做下去.
通常人們明知道自己是錯的.
就不會做.
很少人明知道錯都會做.
所以每個人都對得起自己.
歧視別人的人.
永遠都覺得自己是沒有問題的.
他歧視別人是有問題的.
自私人都是一樣.
覺得自私總是有原因和理由的.
有問題的人.
總覺得是別人的問題.
一句傷害別人的話.
永遠都是說出來之後.
才發現原來這麼大破壞力.
我們每個人都是這樣.
所以經文裡面還有一句很有趣.
他說眾門徒都是這樣算.
每個門徒都是這樣.
我們每個人都是這樣.
雖然所有的門徒都是這樣說.
但彼得的說話.

$^{321}$有另一個令我們反思的地方.
他說眾人雖然為你的緣故跌倒.
我卻永不跌倒.
眾人都會跌倒.
但彼得是一個例外.
在彼得和耶穌的爭論裡.
彼得開始選擇退而求其次.
彼得作出讓步.
他同意耶穌的命題.
但他否定了命題.
他身上的有效性.
即是說.
你對,但不關我的事.
你說的會發生.
但我沒有這樣做.
彼得同意跌倒這件事.
但他不同意這件事在他身上會發生.
他覺得自己是一個例外.
人都是這樣.
總覺得自己是一個例外.
有人說我們眼睛是長在自己頭上.
你看什麼都好.
你都是中間.
所以你覺得任何事.
都似乎是自己是一個例外.
你看不到自己.
你看不到世界.
所以斯人徐志摩寫了一段很有趣的說話.
他說誰都以為自己是例外.
在後悔之外.
誰都以為擁有的感情也是例外.
在變淡之外.
誰都以為戀愛的對象剛好也是例外.
在改變之外.
然而最終發現除了變化.
無一例外.
我們都很想成為那個例外.
但這件事的想法.
是我們在想像中很危險的想法.
一方面你高估了自己的能力.

$^{361}$覺得自己永遠站在有道理的那一邊.
覺得自己比其他人更加清醒.
覺得自己是在問題中單方面的受害者.
覺得自己對得起天父上帝.
這是一個很危險的想法.
這是一個很真實的屬靈道理.
當一個人覺得自己處於屬靈高峰的時候.
這就是屬靈低潮的開始.
或者你不是這樣看.
或者你說不是的.
我永遠都是屬靈低谷的人.
快點關心我吧,牧師.
大家都謙卑的.
大家都不覺得自己有多厲害.
可能你沒有高估你的能力.
不過你可能低估了這個世界的環境.
但變成都是這樣.
當耶穌真的被捉的時候.
當其他門徒都躲起來的時候.
他才發現原來一個人在大祭司家裡.
在大喊的時候.
每個人都在審耶穌的時候.
那一刻他真的會害怕.
那一刻他才知道自己是這樣的.
事實上在我們這個世界裡.
很多時候我們都處於這樣的環境裡.
任何事情都可以叫我們成為犯罪的理由.
能夠叫你犯罪的事情.
不知不覺之間就臨到.
大家知道大家出來工作這麼多年.
外面有很多的試探.
讓我這些做教會的人更加明白.
任何事情都可以成為一個試探.
任何組合都可以.
兩個人去旅行可以成為一個試探.
一個人出差可以成為一個試探.
一部電腦可以成為一個試探.
侍奉都可以成為一個試探.
所以說一個故事.
一個五歲的小朋友.

$^{401}$在一個糖果店外面走來走去.
眼睛不停地看著糖果店的糖果.
好像忍不住就想拿.
然後老闆看著這個小朋友.
這麼久都看著.
就問他想偷我的糖果嗎.
然後那個小朋友怎麼說.
他說我不是想偷你的糖果.
我是想忍住不偷你的糖果.
這是一個很真實的事情.
不要笑.
我想說這幾年裡我們最大的試探是什麼.
可能大家沒有想過.
這幾年我發覺是一個很大的試探.
就是在我們面前有一個很大的惡.
在我們面前有一個很大的惡.
唯願公義如滔滔江河.
唯願公平如大水滾滾.
灰飛煙滅.
這些東西.
當一個極大的惡在我們面前的時候.
當我們這幾年不斷嘗試去回應和去反抗.
逃避這個極大的惡的時候.
一個極大的惡在你面前.
你就看不到自己的惡.
真的.
為什麼我特意在這兩個月要說「對不起」這個話題.
我很想全面宣傳.
不過問題的重點都不是彼得是否準確地評價自己.
或者去評估這個世界.
整個問題.
最大的問題基本上不是在跌倒這件事裡.
跌倒是一個問題.
但這不是耶穌和彼得說話的重點.
其實我們一直被彼得的說話誤導了.
我們不小心被彼得的說話牽著鼻子走.
我們沒有聽清楚耶穌要和彼得說的話.
大家留心看耶穌說了什麼.
耶穌說什麼.
「今夜你們為我的緣故都要跌倒.

$^{441}$因為經上記著說.
我要擊打木人,揚就分散了.
但是我復活以後.
要在你們以先往加里尼去.
耶穌想和彼得說的是.
重點不是你將會跌倒.
而是你跌倒之後.
我會在哪裡等你.
我會在加里尼等你.
但彼得沒有將這句話聽進耳朵裡.
彼得只是將重點放在自己那裡.
我會跌倒,我不會跌倒.
彼得要辯論.
彼得和耶穌爭辯自己會否跌倒.
眾人雖然為了緣故跌倒.
但他永不跌倒.
然後耶穌看到他這麼極力反駁.
他就說出了那句話.
我實在告訴你們.
今夜皆要以先你要三次不認.
但彼得仍然不服氣.
仍然要爭辯.
他反駁說我是不會跌倒.
我必須要和死種不能不認你.
我是不會跌倒的.
所以聖經留下了一個很特別.
一個很深深的沉默.
耶穌沒有再和彼得爭辯下去.
耶穌只是留下一個沉默.
耶穌沉默.
因為整個討論已經是一個錯誤的方向.
整個討論只留於彼得會否跌倒這個題目.
但這不是耶穌要講的重點.
再爭辯下去.
彼得會否跌倒.
這就是耶穌的重點.
因為耶穌說.
如果我跌倒的話.
我會跌倒.
這就是耶穌要講的重點.

$^{481}$再爭辯下去.
再討論下去.
只會讓討論越來越錯.
耶穌不是要和彼得討論.
他能否跌倒.
而是你能否做到.
所以耶穌沒有出聲.
因為耶穌一開始就不想和彼得討論.
彼得會否跌倒這個問題.
他會跌倒還是不會跌倒.
這不是重點.
耶穌要和彼得討論什麼.
耶穌和彼得說.
當你真的下一次跌倒的時候.
你要知道我會在哪裡等你.
這才是耶穌和我們要說話的重點.
這才是耶穌整段說話的意思.
是最後晚餐最後一句說話的重點.
當你跌倒之後.
我會永遠在那個地方等你.
就是在你認識我的那個地方等你.
我記得很多年前.
小時候.
媽媽帶我去香港那些.
拉屁股的公園玩.
如果你和我差不多年紀就知道.
以前的公園是真的.
那些鋼架那些.
爬的那些.
你沒玩過那些嗎.
那些叫爬鋼架.
全部都是鐵造的.
不是膠來的.
是鐵的.
那時候香港的鐵架是那些.
那麼高一層.
像一個球那樣的.
鐵架是這樣爬.
很高的.
我記得那時候很高.

$^{521}$五六歲的時候都很高.
要爬的那些鋼架.
那時候媽媽帶我去公園玩.
我一去到就爬.
因為都挺高的.
就爬.
媽媽就坐在公園椅子上.
看著我.
每次我爬到一格的時候.
我就立刻扶著.
回頭看著我媽媽.
試一試爬到.
然後媽媽就這樣.
然後我就繼續爬.
爬到一格之後.
她就再看著我.
然後媽媽就這樣.
媽媽就微笑.
媽媽就點頭.
現在我做人爸爸.
我就明白這個點頭.
這刻微笑是什麼意思.
媽媽點頭.
媽媽微笑其實.
她不太關心我爬到多高.
她關心什麼.
她關心的就是.
如果你跌倒的話.
你要知道媽媽在這裡.
媽媽就在你下面.
記住.
我就來幫你.
弟子們.
今天的敬拜到今天這裡.
或者整個兩個月的月替.
都很想和你們說.
我們知道.
每個基督徒.
總是有很多軟弱的時候.
總是有犯罪的時候.

$^{561}$總是有很多生命的問題的時候.
全家都是弟子們.
都是一樣.
我們要記得.
耶穌基督在這裡等我們.
我們跌倒.
或者這一刻沒有跌倒.
這個不是重點.
反正沒有人可以說.
我永不跌倒.
我會.
你的目者會.
潘Sir都會.
每個人都會.
重點不是在你跌不跌倒的問題.
而是當你下一次.
真的遇到一個很嚴重的時候.
當你遇到一個.
聖經所說跌倒.
這樣的情況的時候.
或者你之前跌倒了.
今天你在網上看Full Church.
或者現在開始重建.
重拾的時候.
我們要記得.
耶穌就在你下一個跌倒的地方.
等著我們.
最後晚餐.
最後一句話.
耶穌不是要預言什麼.
耶穌不是要預言.
讓你三次不認主.
而是跟彼得說.
你跌倒之後.
你可以怎樣做.
你記得.
在那個地方找我.
基督徒或者我們Full Church.
最危險的地方是什麼.
不是跌倒.

$^{601}$我們知道我們有很多.
不同的背景弟子們在這裡.
最危險是我們跌倒之後.
我們總是有很多不同的原因.
很漂亮的.
當那件事沒有發生.
沒有人知道.
只有你自己知道.
問題是當你跌倒之後.
什麼時候.
怎樣.
在哪裡.
重新去找回我們的主耶穌基督.
真的想想自己.
順著這麼多年.
從你自己的誤會.
到你今天這個教會.
很多的傷害.
自己的問題.
我們同神好像很近.
又好像很遠.
回到來.
我不知道你現在的情況怎麼樣.
當我預備.
道歉的時候.
我發現.
道歉是什麼意思.
跟神說對不起是什麼意思.
道歉其實是一個關係的延續.
任何一句真而眾知的道歉.
其實都是我們期待著.
這段關係能夠繼續下去.
你才會說對不起.
剛好都是神給我的經歷.
上星期我和我女兒吵架.
一件很少發生的事情.
因為我女兒只有九歲.
我很少和我女兒吵架.
事緣都是一件很無聊的事情.
每晚我和我太太會和我女兒一起看靈修書.

$^{641}$那本靈修書叫《我與天父chat chat》.
是一本中文幾口的書.
一方面靈修一方面看中文.
家長就是這樣.
一方面讀一下中文.
又可以看靈修.
每晚睡前她都和我一起讀.
差不多有一晚十點多.
差不多睡覺.
她就過來說要靈修.
我說好吧.
那時候我在用Steam Deck的彈機.
那時候我立刻收.
也不拖她.
我先save遊戲.
讓我走走.
當我save遊戲的時候.
她突然在玩.
她在戳我的畫面.
你知道Steam最雜的是什麼.
就是按錯按鈕.
她在戳我的畫面.
就save錯了遊戲.
我立刻罵她.
不要這樣.
罵了一下.
她就很不爽.
被人罵了一下.
然後她就不理我了.
我和她一起靈修.
她就不理我了.
然後她就說自己靈修了.
但是她會隨便靈修.
隨便看.
然後我就和她越來越僵.
一個在床頭一個在床尾.
兩個就在那裡很憤怒.
弄了很久.
然後她媽就走進來.
她說不要再弄了你們.

$^{681}$明天再說.
因為你女兒已經受不了.
因為她睏了.
睏到沒什麼理性.
她已經硬睡了.
然後她就回到自己的房間.
關燈睡覺.
然後我和師母聊了幾句.
我忍不住回到房間.
在床邊說對不起.
早點睡.
當我想有明天早上的時候.
我就想今晚就說對不起算了.
當你願意說對不起是什麼意思.
你願意和那個人有將來.
能夠有第二天.
能夠延續這個關係下去.
所以耶穌和彼得這個故事裡面.
沒有說過任何對不起.
不過其實你知道.
耶穌在彼得跌倒之前.
甚至知道自己跌倒之前.
就一早已經預備了這樣的關係.
彼得在不認罪之前.
甚至知道自己有問題之前.
耶穌一早就已經為我們.
預備了這樣的關係.
耶穌和彼得說.
當我復活以後.
要在你們的門前先往加利利去.
加利利是什麼地方.
如果你去加利利的話.
加利利是一個海邊的地方.
一個非常大的海.
海邊有漁船.
一個風和日麗.
氣候很溫和的地方.
不過這個不是重點.
加利利是彼得和耶穌認識的地方.
加利利是彼得認識主耶穌的地方.

$^{721}$加利利是彼得起初的信仰建立的地方.
加利利是昔日彼得很快樂.
很有熱誠地跟隨著的地方.
耶穌在加利利等著我們.
還記不記得你剛剛坐上主的時候的樣子.
在中學生團契那裡.
傻乎乎地在做團長.
在團契那裡跟團友吃到地的那種.
很有熱誠地在侍奉那個樣子.
耶穌在加利利等著我們.
從各個他的十字架.
復活的空墳墓.
再回到昔日的加利利.
耶穌邀請我們回到昔日的地方.
那個你起初很有熱誠.
跟隨著耶穌基督的地方.
在那裡,復活的大能.
賜人生命的力量等著我們.
我們發現耶穌等待彼得並沒有確實的時間.
經文裡面好像沒有這樣寫.
你說這樣也有的,這樣等人也有的.
什麼時候?.
耶穌什麼時候約了彼得?.
耶穌約了彼得的時間.
就是彼得願意回到加利利的時間.
彼得願意什麼時候回去.
耶穌什麼時候就在那裡等他.
所以你記得是不是.
《約翰福音》第二章.
彼得回到加利利的時候的經文.
大家記得在漁船裡面.
然後在岸邊,耶穌就在那裡煮早餐.
大家試想一下一個完美的星期日早上.
當你起床的時候.
你床上突然聽到廚房那些「叉叉叉」的聲音.
一陣煎香腸的味道.
對不起,還有一些奄列的味道.
當你走出你的床,走到飯桌的時候.
發現奄列旁邊還有一些剛煮熟的麵包仔.
旁邊還有一些切好的青瓜和番茄仔.

$^{761}$奄列切出來的蛋汁流出來.
很漂亮,很漂亮,很漂亮.
還有一些火腿絲在裡面.
當然還有一個非常完美的咖啡.
然後你看到幫你煮早餐的不是你媽媽.
也不是你高人姐姐.
煮早餐給你的是主耶穌.
主耶穌在等著你.
等你回到你昔日的模樣.
就在這個完美的星期日早上.
你抹乾你的眼淚.
再一次回到主耶穌的身邊.
對不起.
我們回到這間教堂正正就是期盼這個再一次.
我們可以有再一次.
(音樂播放).
\newpage



\section{路加福音 23:32-43-20230408}
\label{sec:v4hE6GM4QsI}
\textbf{【流堂崇拜】我們都有錯|路加福音23\_32-43|20230408 [v4hE6GM4QsI]}
\newline
\newline
連結: \href{https://youtube.com/watch?v=v4hE6GM4QsI}{\texttt{ https://youtube.com/watch?v=v4hE6GM4QsI}} ~~~~ 語音日期: 2023-04-08 
\newline
\newline
\hyperref[sec:8KdYgVn_hzk]{\small{< < < PREV SERMON < < <}}
~
\hyperref[sec:index_chronic]{\small{[返順時目]}}
~
\hyperref[sec:index_scriptual]{\small{[返順卷目]}}
~
\hyperref[sec:3PY1nwdp_0k]{\small{> > > NEXT SERMON > > >}}
\newline
\newline
路加福音 23:32-43-20230408
\newline
\begin{longtable}{cl}
\hline
\hline
章節 & 經文 (和合本修訂版)\\
\hline
23:32 & \begin{tabularx}{0.7\textwidth}{X} 另外有兩個犯人也被帶來和耶穌一同處死。 \end{tabularx} \\ \\ \relax
23:33 & \begin{tabularx}{0.7\textwidth}{X} 到了一個地方,名叫髑髏地,他們就在那裡把耶穌釘在十字架上,又釘了兩個犯人:一個在右邊,一個在左邊。 \end{tabularx} \\ \\ \relax
23:34 & \begin{tabularx}{0.7\textwidth}{X} 〔 這時,耶穌說:「父啊!赦免他們,因為他們所做的,他們不知道。」〕士兵就抽籤分他的衣服。 \end{tabularx} \\ \\ \relax
23:35 & \begin{tabularx}{0.7\textwidth}{X} 百姓站在那裡觀看。官長也嘲笑他,說:「他救了別人,他若是基督,是神所揀選的,救救他自己吧!」 \end{tabularx} \\ \\ \relax
23:36 & \begin{tabularx}{0.7\textwidth}{X} 士兵也戲弄他,上前拿醋送給他喝, \end{tabularx} \\ \\ \relax
23:37 & \begin{tabularx}{0.7\textwidth}{X} 說:「你若是猶太人的王,救救你自己吧!」 \end{tabularx} \\ \\ \relax
23:38 & \begin{tabularx}{0.7\textwidth}{X} 在耶穌上方有一個牌子寫著:「這是猶太人的王。」 \end{tabularx} \\ \\ \relax
23:39 & \begin{tabularx}{0.7\textwidth}{X} 同釘的犯人中有一個譏笑他,說:「你不是基督嗎?救救你自己和我們吧!」 \end{tabularx} \\ \\ \relax
23:40 & \begin{tabularx}{0.7\textwidth}{X} 另一個就應聲責備他,說:「你是一樣受刑的,還不怕神嗎? \end{tabularx} \\ \\ \relax
23:41 & \begin{tabularx}{0.7\textwidth}{X} 我們是應得的,因為我們是自作自受,但這個人沒有做過一件不對的事。」 \end{tabularx} \\ \\ \relax
23:42 & \begin{tabularx}{0.7\textwidth}{X} 他對耶穌說:「耶穌啊,你進入你國的時候,求你記念我。」 \end{tabularx} \\ \\ \relax
23:43 & \begin{tabularx}{0.7\textwidth}{X} 耶穌對他說:「我實在告訴你,今日你要同我在樂園裡了。」 \end{tabularx} \\ \\ \relax
23:44 & \begin{tabularx}{0.7\textwidth}{X} 那時大約是正午,全地都黑暗了,直到下午三點鐘, \end{tabularx} \\ \\ \relax
23:45 & \begin{tabularx}{0.7\textwidth}{X} 太陽變黑了,殿的幔子從當中裂為兩半。 \end{tabularx} \\ \\ \relax
23:46 & \begin{tabularx}{0.7\textwidth}{X} 耶穌大聲喊著說:「父啊,我將我的靈交在你手裡!」他說了這話,氣就斷了。 \end{tabularx} \\ \\ \relax
23:47 & \begin{tabularx}{0.7\textwidth}{X} 百夫長看見所發生的事,就歸榮耀給神,說:「這人真是個義人!」 \end{tabularx} \\ \\ \relax
23:48 & \begin{tabularx}{0.7\textwidth}{X} 聚集觀看這事的眾人,見了所發生的事,都捶著胸回去了。 \end{tabularx} \\ \\ \relax
23:49 & \begin{tabularx}{0.7\textwidth}{X} 所有與耶穌熟悉的人,和從加利利跟著他來的婦女們,都遠遠地站著,看這些事。 \end{tabularx} \\ \\ \relax
23:50 & \begin{tabularx}{0.7\textwidth}{X} 有一個人名叫約瑟,是個議員,為人善良正直, \end{tabularx} \\ \\ \relax
23:51 & \begin{tabularx}{0.7\textwidth}{X} 卻沒有附從別人的所謀所為。他是猶太的亞利馬太城人,素常盼望著神的國。 \end{tabularx} \\ \\ \relax
23:52 & \begin{tabularx}{0.7\textwidth}{X} 這人去見彼拉多,請求要耶穌的身體。 \end{tabularx} \\ \\ \relax
23:53 & \begin{tabularx}{0.7\textwidth}{X} 他把耶穌的身體取下來,用細麻布裹好,安放在鑿巖而成的墳墓裡;那墳墓從來沒有葬過人。 \end{tabularx} \\ \\ \relax
23:54 & \begin{tabularx}{0.7\textwidth}{X} 那日是預備日,安息日快到了。 \end{tabularx} \\ \\ \relax
23:55 & \begin{tabularx}{0.7\textwidth}{X} 那些從加利利和耶穌同來的婦女跟在後面,看見了墳墓和他的身體怎樣安放。 \end{tabularx} \\ \\ \relax
23:56 & \begin{tabularx}{0.7\textwidth}{X} 她們就回去,預備了香料香膏。在安息日,她們遵照誡命安息了。 \end{tabularx} \\ \\
[1ex]
\hline
\hline
\end{longtable}
$^{1}$弟子妹平安.
歡迎參加Flo Church在受難與復活之間的崇拜.
我們堂會是星期六聚會.
所以每年在受難的日子當中.
星期六是我們崇拜的時間.
今天仍然是選擇講十字架的訊息.
其實也掙扎會講復活的訊息.
但是回應和配合堂會Sorry這個月提.
我覺得要再次重提十字架訊息對我們來說也是很重要.
很高興今天看到很多弟兄姊妹在現場和我們一起崇拜.
不知道你們這兩天在街上的感覺如何.
我覺得清閒了很多.
我今天應該是看到這兩天最多人的地方.
希望你不要覺得你崇拜來這裡好像不能去旅行.
但我相信你是擺了對的位置來這裡.
特別紀念身體有缺陷.
周邊也有很多人流感.
所以戴口罩是必須的.
剛才詩歌投入過程中我自己很受感動.
因為其實我今天三點鐘已經開始聽這些歌.
但是在這麼多弟兄姊妹一起唱的時候.
特別是後面的聲音很澎湃.
不單是這邊後面的.
這邊後面也很澎湃.
是很被觸動.
但我相信基督教不是動之以情.
同樣是說之以理.
今天選的經文是路加福音的十字架經文.
我們一起重讀上帝的說話.
我們在路加福音第23章第32節開始.
又有兩個犯人和耶穌一同帶來處死.
到了一個地方名叫築留地.
就在那裡耶穌釘在十字架上.
又釘了兩個犯人.
一個在左邊一個在右邊.
當下耶穌說.
父啊赦免他們.
因為他們所做的他們不曉得.
丁丁就尖叩翻他的衣服.
百姓站在那裡觀看.

$^{41}$官府也恥笑他說.
他救了別人.
他若是基督上帝所揀選的.
可以救自己吧.
丁丁也戲弄他.
上前拿槍在他後說.
你若是猶太人的王.
可以救自己吧.
在耶穌耳上有一個牌子寫著.
這是猶太人的王.
那和丁丁的兩個犯人.
有一個譏笑他說.
你不是基督嗎.
可以救自己和我們吧.
那一個就應聲責備他說.
你既是一樣受刑的.
還不怕上帝嗎.
我們是應該的.
因我們所受的.
與我們所做的相稱.
贊這人沒有做過一件不好的事.
就說耶穌啊.
你得過降臨的時候.
求你紀念我.
耶穌對他說.
我實在告訴你.
今天你要和我在樂園裡.
我們一起禱告.
天上帝每當我們打開你的說話.
再重溫當日的情景的時候.
求主你今天營救我們.
對我們的說話.
以至我們再一次明白到.
你掛在木頭上.
死在十字架的緣由.
是為了什麼.
而我們每一個面對十字架的時候.
同樣要做出抉擇.
同樣要面對當中要決定的事情.
求主你幫助.

$^{81}$開我們的心.
開我們的耳.
以至我們明白你的心意.
我們祈禱奉耶穌的名求.
阿們.
十字架我相信大家不陌生的.
有很多電影節目可能現場.
你戴的飾物都是有十字架.
或者是你周遭環境當中都不難見到.
今天不是說十字架的飾物的symbolic meaning.
是實際上當日的十字架要面對的什麼.
你看到各個他山上有三個十字架.
最少你看到有三個十字架在當中.
第一個十字架就是耶穌基督已經被掛在當中.
另外就有兩個銅釘在十字架的人.
你會看到很多壁畫.
或者很多基督教關於復活節的畫.
當中都有三個十字架.
三個十字架有三個不同的意思.
三個不同的對答.
所以從文字上要處理的時候.
下一張請.
你會看到在經文抽一些對話當中.
你就不難理解其實說話帶出的力量是什麼.
當我們要跟別人聊天的時候.
其實你都會想了解他說什麼.
英文有一句說話.
You read my mind? Do you read me?.
在read的過程當中其實是在讀你的字.
其實是在讀你的說話.
很多時候你會發覺為什麼跟一些人不容易溝通.
因為他每個字你都聽得到.
但你不太聽得懂他整句意思是什麼.
因為是不是沒有邏輯.
還是有些用詞你不太明白.
其實都有點複雜.
但今天你會看到在選了這段經文當中.
有不同的對話其實要表達一個比較重要的訊息.
我們看下一張powerpoint的時候.
你會看到耶穌開頭的時候說了一句話.

$^{121}$就是父啊 聖啊 他們 因為他們所做的他們不曉得.
其實今天我們自己做人做了一段時間.
有些很小的.
但我相信大部分都是年輕人.
或者已經有一段年齡的弟兄姊妹.
其實你是不是很清楚自己每一個決定在做什麼.
其實你是不是很清楚你自己每一個決定的前因和後果.
而你還沒有承擔計算過之後承擔的後果.
真的 我相信這幾年在香港我們一起經歷的過程當中.
你會發覺你每一天出街.
你每一個網上的comment.
你每一個你要做的表達.
其實你都有心思熟慮過.
你都會想一下值不值得做.
還有是不是這樣做.
還有做完之後其實那件事是不是你可以控制到呢.
從來你會發覺可能過去你本身是一個很好的環境.
但時態轉了 身份轉了.
還有你的心境轉了的時候.
你會發覺很不容易.
但都是當下要做決定.
十字架的環境今天要面對的情況就是.
耶穌掛在木頭上.
祂望著下面的人.
祂第一句說話是說.
父啊赦免他們.
因為他們所做的他們不孝得.
這個是很重要.
其實很多人不知道自己為什麼要這樣做.
因為跟風這樣做.
因為我不做我有問題.
我怕有問題我這樣做.
我知道這樣做.
但我選擇不抗拒我照做.
其實我相信你過去這些日子.
你不難見到這樣的狀況.
我不知道你這兩天看了什麼貼文.
你都會見到一些貼文.
就是當初要做了什麼.
現在要承擔結果.

$^{161}$可能要倒閉.
可能要重新洗底.
我想問一件事.
其實當初要做的時候.
他自己知不知道呢.
我這幾年常常都問自己.
和跟自己說.
每一個做的決定.
每一句說話.
和要表達是什麼.
讓人知道的訊息.
我是要想清楚.
但我不是Play Safe的做法.
我是清楚別人問我的時候.
我是告訴當時的人聽.
我為什麼做這個決定.
這個是很不容易的.
我在當中都曾經和大家說過.
在過去日子特別是很困難.
大家香港很困難的日子的時候.
每天早上起來看著鏡子拍一拍自己.
就是今天要醒目的一個人.
我相信在這些日子當中.
我今天下午和一個教務員傾談的時候.
都說當坊間說復常的時候.
教會是不是都是復常呢.
是復常在之前.
很笑聲滿載溫馨的日子.
還是真的重新面對新的環境.
新的香港環境之下.
如何默會呢.
這個是很真實的事情.
耶穌說.
「父啊,世面貪門,因為貪門所執,貪門不曉得.」.
但我們今天就是問我們做事.
我們知不知道.
你會發覺有些人是這樣的.
官府就笑他.
他救了自己?.
他救了別人?.

$^{201}$如果你是基督的話.
你是上帝揀選的.
你救自己吧.
丁丁就笑他.
如果你真的是猶太人的王.
你救自己吧.
你會看到他們是忽視無視上帝的能力.
很多人問為什麼要相信耶穌呢.
你覺得耶穌很厲害.
其實是不是覺得.
或者知道耶穌很厲害的人就會相信耶穌呢.
未必的.
是不是知道基督教是很好的人.
人們就會相信呢.
未必的.
但真正你問問自己.
你是因什麼原因相信耶穌.
你是因什麼原因覺得基督教是可信的.
你願意花時間.
花你的生命在當中追隨.
真的要問這個問題.
其實在這段經文.
你會看到那些笑的人.
其實他不是未見過.
如果他說得出你是基督.
你是上帝揀選.
你是猶太人的王.
這些對話當中.
其實他一直都在見耶穌之前做的事.
我沒有讀《約翰福音》.
但《約翰福音》在第十二章第一節的時候.
是在說.
雨戰前六天耶穌再去伯大利.
伯大利是什麼地方呢.
即是上星期六.
如果說日子.
上星期六的時候.
耶穌再去伯大利.
然後他就進聖城.
然後撒冷準備進城.

$^{241}$就是這個星期的日子.
耶穌再去伯大利的意思是什麼呢.
就是去見一次瑪大,瑪利亞和拉撒路.
這三個弟弟.
在《約翰福音》第十一章是在說什麼呢.
就是拉撒路他死而復活的經文.
耶穌在臨進城之前.
再去過瑪大利那家人的時候.
再見他們一家三口.
然後就用真拿大香膏去告耶穌.
但《約翰福音》第十一章.
說拉撒路的事件的時候.
你會見到他們死了的弟弟拉撒路.
他們很傷心.
但耶穌說他們是睡著了.
但他們都不是很相信.
他說你相信復活吧.
耶穌說了一句話.
「復活在我,生命也在我,信我的人,雖然死了也必復活.」.
但瑪大的答案是什麼呢.
瑪大的答案是.
「主啊,我信在末後的日子是會復活的.」.
是信了,末後嘛.
但耶穌不是這樣.
耶穌沒有和他們爭辯.
耶穌說拉撒路在哪裡.
你帶我去吧.
他們就帶拉撒路去.
不要去,已經是四天了.
已經可以發臭了,不要去.
他們照樣帶耶穌去.
耶穌就救了拉撒路.
拉撒路就復活了.
很多人看到的事件.
但是不是很多人看到復活的神蹟.
是不是很多人看到耶穌說得出做得到的事情的時候.
那些人就會相信耶穌.
經文第56節說一句話.
「從那日起,他們就相已要殺害耶穌.」.
不是每個人看到神蹟.

$^{281}$看到復活這麼厲害的神蹟.
就會相信耶穌.
不是每個人看到神蹟的騎士.
大家就會覺得那件事是屬於我.
我會贊成下去.
不是的.
你會看到有些人是這樣更加不相信.
因為他令到他地位上.
能力上受到威脅.
他都是在想自己.
所以你會看到.
今天我們看到耶穌釘在十字架上.
他們的說話要表達.
如果你真的這麼厲害.
意思就是你是基督.
基督是猶太人期盼等待的彌賽亞.
要救他們的位置.
如果你真的基督是上帝的子孫.
你救自己吧.
你展示給我看.
你跳下來吧.
你將你的能力彰顯吧.
你再行一個神蹟出來吧.
冰釘也說.
如果你真的是猶太人的王.
你救自己吧.
在希臘的文化.
在羅馬的政權之下.
每一個王都為自己的王權爭戰.
你如果是猶太人的王.
你就救自己吧.
你會發覺每一個人去面對耶穌的時候.
耶穌釘在十字架上.
他沒有跟祂抗爭.
沒有跟祂做任何的申辯.
耶穌說了一句話.
父啊.
赦免他們.
其實他們自己在做什麼.
他們自己不知道.

$^{321}$親弟姐妹.
如果你面對耶穌的時候.
你跟耶穌說了什麼話.
相信在座很多人都做了基督徒一段時間.
如果你身邊都遇到一些不相信耶穌的人.
或者是你跟他分享了福音很久.
他都沒有回應的時候.
其實他當中不相信什麼呢.
當中他不明白的是什麼呢.
其實可以慢慢繼續去討論.
每個人面對耶穌.
他覺得基督教都是很感性的.
我就不是很感性.
我是很理性的.
其實我覺得不是分感性理性.
上帝做人就有左右腦.
右腦是一個情感.
左腦是一個邏輯.
我們每個人都可以有這個.
其實不是分感性理性.
其實就是他不認清耶穌是什麼.
或者他不接受的耶穌是什麼.
這個就是大家可以繼續對話下去.
耶穌沒有跟他們去抗爭,抗辯.
耶穌選擇沉默.
下一章你會看到彼得怎樣看待這個情景呢.
彼得前書第二章21節是這樣說的.
你們夢照原是為此.
因為基督也為你們受過苦.
給你們留下榜樣.
叫你們跟隨他的腳蹤行.
他並沒有犯罪.
口裡也沒有鬼.
他被罵不還口.
受害不說威嚇的話.
只將自己交託.
拿按公義審判人的主.
其實耶穌很清楚.
他自己掛在木頭上的恩由和目的是什麼.
彼得怯怯面散居在不同地方的基督徒.

$^{361}$你們也會遇到這些逼迫.
這些苦待.
這些挑戰.
這些不解.
但是彼得提醒你們夢照原是為此.
因為耶穌基督掛在木頭上那一天.
用什麼反應.
他留下榜樣給我們看到.
就是他沒有錯.
他沒有騙人.
你罵我.
你會看到受苦的經文裡.
要帶出耶穌對話當中.
很多耶穌都沒有抗辯.
但當彼得問你是猶太人的皇媽.
你說的是.
身份上耶穌一定會確認.
有工作上耶穌一定會表達得清楚.
但是其他東西.
耶穌不多談.
因為上帝天父看著他所做的事情.
身體的前面我們也是.
我們這幾年生命生活都改變很多.
但如果你真的每天都知道自己在做什麼的時候.
上帝都知道你在做什麼.
不需要被論斷.
你不需要論斷自己.
有關論斷這個課題.
我在上一個月有兩講.
我都說過.
包括上帝的視覺和演員的道德修養.
裡面都說過論斷這個內容.
今天很多人會留言.
在你的媒體.
在你的工作上有平頭品足.
但是你作為一個跟隨上帝的人.
你清楚自己在做什麼.
就繼續做吧.
天父都看到.
下一段經文我們看的是.

$^{401}$你會看到那些犯人開始一句一句地說.
但是旁邊那兩個十字架的人強盜.
他們的對話就成為第二段的焦點.
我們再下一張.
你會看到兩個犯人的對話.
他們自己說.
第一個犯人說.
你不是基督嗎.
你可以救自己和我們吧.
這次就跟耶穌說.
如果你真的救了主.
你就救我們兩個吧.
我們意味著不如三個一起跳下去吧.
就離開吧.
整件事是希望.
如果真的這麼厲害.
不如走一個神蹟就便宜我們兩個吧.
但是那個囚犯被另一個囚犯的說話罵他.
他就說.
你何時一樣受刑還不怕上帝嗎.
這是第一句說話.
這句說話我不知道你感受如何.
但是我聽下去和你會看到文字上.
你都一起受刑.
你都不怕上帝.
我不知道你怕不怕上帝.
現在沒有什麼對話空間.
你問問自己你怕不怕上帝.
有些人怕有些人不怕.
因為你會問.
什麼是怕.
怕我就不敢回教會.
什麼是怕上帝.
你要問我怕不怕上帝.
如果我怕上帝我應該不敢犯罪.
我怕上帝會擊殺我.
說個笑話.
我們每個星期都會把台上的椅子搬到下面.
那你每張椅子下面都有本什麼.
後面的勁敗隊在笑.

$^{441}$勁敗隊搬的椅子很珍而重之.
因為每張椅子裡面都有本聖經.
一怕掉了本聖經就說上帝會擊殺你.
你糟蹋了上帝的話在地上.
這個笑話在勁敗隊裡面每個星期都會笑一笑.
如果你這麼看重上帝的話你又不靈修.
我舉個例子.
這個氣氛應該是嚴肅一點.
如果你不怕上帝的話.
如果你這麼怕上帝的話.
你應該每個星期都很準時崇拜.
但其實生活中你又不是這麼怕上帝.
上帝很寬容我的.
上帝很有大愛的.
你還不怕上帝媽這個說話其實可圈可點.
你問你自己有多怕上帝.
如果你這麼怕上帝的話.
你很多事情都很謹慎.
如果你不怕上帝的話.
你每天就擺上帝在哪裡呢.
上帝我去打機你先在這裡休息.
外面很危險的.
還是上帝我要喝一杯東西.
外面的人說很多髒話你不要聽.
你回家等我.
是不是這樣呢.
我希望好像有一點點嬉笑.
但我希望你明白到.
其實我們是不是真的這麼怕上帝呢.
可能不是的.
還是你心中覺得有問題就找上帝.
沒有問題就不要打擾上帝.
上帝隨傳隨到的.
但真的怕上帝的人.
其實都不是一個生命比較穩定的環境.
可能大家知道我以前在醫院工作.
其實上下病房的時候.
久不久都會撞到軟木.
我通常撞到軟木都不會打擾軟木.
因為軟木上了病房.

$^{481}$都是有些事情要做的.
都會聊聊天.
不一定每次都要搶救.
靈魂啊靈魂.
都會聊聊天.
但你會看到.
其實軟木很多時候跟病人聊天的時候.
你會看到不同病人真的有不同的反應.
有些人仍然覺得.
我聽過很多次了.
多謝你.
不如找下一個吧.
但有些人就會在把握那一刻.
就跟上帝說.
他要缺志.
你不是那個階段.
你不知道自己的生命去到哪個位置.
但人真的去到生命一個位置的時候.
他無力.
他無助的時候.
他面對自己生命的時候.
生命主的出現.
他會不會把握.
那時候他什麼叫做害怕.
你明白我的意思嗎.
你真的明白嗎.
可能你都還沒有明白.
因為我見過一些弟兄姊妹.
真的死過活生生的時候.
之後很愛主的.
我不是要大家死過活生生才愛主.
但不要等到你很害怕上帝的時候才會轉.
其實一直有機會有恩典.
不是必然的.
我自己在2006年的時候.
去過外地一個地方上課程.
應該是2004,2005年那段時間.
去過外地上一個課程.
那時候因為還沒有正式全時間服侍的時候.
很輕鬆.

$^{521}$但我見到有弟兄姊妹.
她什麼筆記都抄.
上課的時候抄了筆記.
我見了一天兩天.
我說其實是講普通話的.
我問她為什麼你經常要抄那麼多筆記.
她說我要記住要講的內容.
我說其實不用抄的.
如果你想上這堂錄音可以訂的.
不用抄的.
她又給你一張光碟.
她說謝謝.
第三天我見她還在抄的時候.
我心想她是聽不懂我的普通話.
還是她看不到別人說外面可以錄.
其實熟了三天.
她見到這個人那麼煩.
她回答我.
我是幾條村供她一個人去馬來西亞上課.
幾條村教會供她一個人去馬來西亞上課.
我自己一個人就自己上課.
我才明白到原來她沒有錢買那些CD.
她是抄了所有的.
變成了逐字藤稿.
寫下所有老師說的話.
附帶手帕.
回去她要教我幾條村的東西.
那種敬畏.
我就馬上俯伏在她面前.
能夠見到上帝的時候.
或者能夠見到那件事是真而眾知的時候.
不是每個人都懂得做這個俯伏.
幾條村的人供她一個.
她就回去教那班弟兄姊妹.
人生要去到一個位置的時候.
你怕不怕上帝呢?.
這個囚犯他怕上帝.
我們活該的.
因為我們受的和我們所做的是相稱的.
但這個人連一件錯事都沒有做過.

$^{561}$他很清楚一件事就是耶穌不應該在這裡.
但是他在這裡.
他很清楚一件事就是.
他們所受的刑罰是應當的.
因為我用了我的選擇方法.
過過生活.
做過那件事.
我現在受的刑罰是叫做履行他罪有應得的結果.
所以他很清楚一件事就是.
每一個人要帶出去上色的人.
每一個人去到上帝面前.
他都要拓出他所有的事情.
我相信大家信主的時候.
都會說過一件事叫做.
耶穌會再來的.
耶穌再來的時候會有審判的.
基督教其實很清楚要讓每一個信主和不信主的人都知道.
會有審判的.
不是代表我們不用.
上主不用.
我們都會.
我們都會在白色大寶座前面將我們所做的事情.
但我們所做的錯事.
耶穌基督已經覆蓋了.
不是到那時候才.
但不代表我們所做的事.
我們不用跟上帝說.
剛才在敬拜的時候都在說一件事.
就是相投.
我們做過不同的事.
信了主之後我們都會犯不同的錯.
信了主之後我們都會有錯誤的決定.
我們都會求上帝的憐憫.
我們求上帝的寬恕.
這是很重要的事情.
你能夠將你所說的事情說給別人聽.
很難說的.
有時候你不想讓我知道.
但你又會不會選擇將你的錯跟上帝說.
但你又不想跟上帝說.

$^{601}$因為不想上帝認罪.
好像很複雜.
上星期John說的訊息的時候.
我們不像天主教.
天主教會有些告解.
但其實基督教在禮儀的教會裡都有.
例如禮賢會 崇禎會 信義會 聖公會.
他們在禮敘裡都會有一個赦罪的環節.
禮敘當中.
就是主席會跟頂子妹一起有認罪討問.
當頂子妹一起跟主席有認罪討問的時候.
主席就會宣洩.
你們若說自己無罪.
便是自欺真理不在你們心裡.
你們若認自己的罪.
上帝是信實的 是公義的.
必要赦免你的罪.
洗清你一切的不義.
這是主席的宣洩.
接著他會說主與你們同在.
會中說也與你們同在.
主的平安尚與你們同在.
上帝將他的宣洩賦予給每一個願意認罪的頂子妹.
上帝的平安臨到當中.
主席就會下台跟頂子妹握手.
我以前做主禮的時候很喜歡這個環節.
因為很感受到握手的過程當中的親密.
不過現在可能不要握手了.
主禮完結了這個問安環節的時候.
才進入聖道禮聽上帝的話.
面對上帝的時候.
他很清楚我該死.
這是我做的事我應得的.
但是我去到你的面前.
因為你是沒有做那件事.
耶穌被掛在木頭上.
是承擔了眾人的罪.
不同人的罪.
而每一個去到上帝面前的我們.
你有沒有和上帝和盤托出你所犯的罪.

$^{641}$你求上帝饒恕.
很多時候我們都會和上帝爭辯.
這些小事上帝不介意應該不用站在這裡.
有些事情大的時候不知道怎樣和上帝說.
所以最後最後沒有說.
小的又覺得不用說.
大的又不知道怎樣說.
於是就沒有說.
但是我希望大家面對你自己的十字架.
你會不會求上帝饒恕.
所以他最後說一句.
你得國降臨的時候求你紀念我.
我不知道大家怎樣看他的說話.
但是我看他的說話的時候.
我覺得這個強度其實是超強.
他有一個透視.
有一個遠像看到.
他真的相信上帝的國.
是在耶穌傳道當中展現出來.
耶穌在地上周遊四方.
除了行善事.
醫治各樣的病症.
就是傳講天國的福音.
上帝的國在其中.
這個強度可能在中間聽過.
但是他最後面對耶穌那一刻.
他說了最後的說話.
耶穌啊 你得國降臨的時候.
求你紀念我.
他相信耶穌的國會展現.
這是信心.
我們都相信上帝的國在當中.
我們相信上帝會.
耶穌會再次回來.
但是當日第一個跟耶穌說的.
就是這個強度.
很希望我們再重看這段經文的時候.
你會更加明白到.
其實這個強度.
他有三樣事情他做了.

$^{681}$第一樣事情就是.
他知道他生命不在他掌握當中.
他面對難處.
最後關頭的時候.
他擺一機會.
他面對上帝.
說回第二樣事情.
說回他自己所做的錯事.
他願意悔改.
第三樣.
他求寬恕之外.
他求上帝的國讓他進入.
這個就是那個強度.
他背著他的十字架.
被釘過程當中.
他面對這三個決定.
同樣都是我們.
耶穌給了他最後的應許.
我實在告訴你.
今天你要跟我再落遠了.
我實在告訴你.
這句話我相信大家都不陌生.
在科幻書裡面這句話通常是.
我鄭重地告訴你.
或者我很謹慎地告訴你.
今天你就跟我再落遠了.
不知道你看這句話也好.
我經常都問.
看了經文大家的反應是什麼.
哇 那就正了.
這是第一個被上帝肯定入落遠.
這個落遠.
你知道這個落遠不是那邊.
大嶼山的落遠.
你會看到那個落遠就是.
上帝肯定他一定會.
verbal這樣confirm的.
是不是.
你會看到.
哇 真的好得無比.

$^{721}$你也想好成這樣.
是不是.
但有些人看這句話的時候.
就一般般.
就是說.
哇 搞錯.
這麼兒戲.
這句話.
或者這麼不公平.
前兩天跟我老婆逛街.
經過一間餃子店.
我老婆說想吃.
我就陪她進去吃.
她在選餃子的時候.
選完就坐下來吃.
選完就在等.
我老婆說這麼多姐姐在包餃子.
四個就很多工資.
就算一下算一下人手.
我就聽到那四個包餃子的姨姨在說話.
就在說這段經文.
哈哈 o下.
o下 這樣.
不是基督教開的.
不是 她說了一段經文.
其實我是聽不到她說什麼.
但當我要預備講章的時候.
這些經文都上路了.
突然間她說了一句.
哪有這麼兒戲和這麼不公平.
你有沒有聽過.
那個囚犯最後在耶穌旁邊說了一句.
耶穌就跟他說.
今天就在天堂.
哇 多麼沒天理.
這些經文都上路了.
她說了一句.
我當然是質疑而聽.
她就說多麼沒天理.
我聽不到其他三個姨姨有回她.

$^{761}$她就繼續說.
你說是不是.
如果那個囚犯做到最後的時候.
十惡不赦突然間一個說.
我今天就跟你去天堂.
有沒有搞錯.
起碼都要罰回去.
這樣不公平.
是不是.
為什麼不是.
當然是.
罪有應得這件事一定.
從阿姐的角度來說是合理的.
哪有這麼大隻甲拿.
現在我們不說甲拿.
哪有這麼便宜.
按他們的常理來說.
因為要判刑.
講一句就可以超脫.
這樣就很划算.
你會看到這是人慣常的思考模式.
就是要論功行賞.
要按功德.
但是你會看到今天這個是救恩.
什麼叫救恩.
救就是挽回.
他已經死了.
我就挽回他.
什麼叫恩.
恩典是.
是恩典.
就是不配得.
配得那些叫功架.
不配得那些叫恩典.
救恩就是要挽回他.
令他得著不配得的事情.
你看到這個.
他知道自己罪有應得.
他看到這個.
他看到上帝的國會在這裡.

$^{801}$他看到這個.
他有能力.
但他選擇不用這個能力救自己.
他為其他人的罪釘在十字架.
耶穌的時候.
這個是信心.
所以耶穌看得出.
我鄭重告訴你.
你今天就和我在樂園裡.
祂不是隨便的.
弟兄姊妹我們能否做到這三件事.
你懼怕上帝嗎.
你做錯事有沒有跟上帝說對不起.
你看到上帝的國.
你願意上帝的國在其中.
你有沒有參與上帝的國.
那個囚犯做了.
耶穌給他肯定.
他肯定了耶穌的國.
耶穌的國也肯定他.
今天你就和我在樂園.
親弟兄姊妹.
十字架是什麼地方呢.
十字架是一個沒有神蹟的地方.
但最大的神蹟就是上帝的兒子.
降世為人釘在十字架上.
十字架是一個沒有光的地方.
是一個很幽暗的地方.
但是那光是真光.
照亮一切身在世上的人.
那種那光就在十字架當中出現.
十字架是什麼地方.
十字架是一個沒有自己辯護的地方.
耶穌是避麻不還口.
但是十字架就彰顯.
最大上帝的能力的地方.
就是救恩的出現.
我盼望大家.
順著主的好來.
你要從述和重溫.

$^{841}$十字架不是一個普通的地方.
十字架是我們救恩的出處.
十字架也是我們上帝的大能.
所以保祿很清楚.
我不傳別的.
我只誇耶穌基督並他釘十字架.
這是保祿傳道最核心的訊息.
很希望每年我們能夠有經歷復活節的日子.
除了我們享受最長的假期之外.
十字架也是我們思想上帝的說話最核心所在.
十字架從來都是一個很大張力的地方.
我希望今天的訊息不僅僅是動之以情.
我希望你更加明白上帝的說話是說之易理.
每一個人面對十字架的時候.
你選擇是一個強度去挑戰耶穌.
還是選擇另一個強度去歸向耶穌.
而耶穌在其中就彰顯十架的大能.
救恩的開始.
我們一起祈禱.
主耶穌 多謝你愛我們.
多謝你給我們機會認識你.
多謝你呼召我們成為你的門徒.
多謝你揀選我們能夠有福分.
在我們有生命氣息的時候去接受這個救恩.
今天我們打開你的說話.
你營救對我們說話.
因為這個古老的十架仍然發揮效力.
雖然歷史已經成就.
但今天營救對我們是很重要的提醒.
我們眾弟姊妹都是罪人.
但耶穌因為你愛我們的緣故.
你死在十字架上.
這也是你告訴我們.
你透過十字架向眾人發出的邀請.
願意我們當中有未認識上帝的.
又或者我們在網絡上轉發.
讓更多人知道耶穌基督的心意.
這也是我們希望他們能夠在有生的日子.
能夠接受耶穌基督.
多謝主你讓我們一起崇拜.

$^{881}$願主你坐直.
享受你當德的榮耀.
我們祈禱感恩.
奉耶穌基督的名求.
阿們.
在今天崇拜開始的時候.
用了啟示錄的宣召經文.
在結束今天信息的時候.
我同樣用啟示錄的信息.
跟大家說一句結束.
啟示錄第三章20節是這樣說的.
看那我站在門內叩門.
若有聽見我聲音就開門的.
我要進他那裡去.
我與他 他與我一同坐直.
讓我們一起踏進光明之向.
\newpage



\section{哈該書 1:1-15-20230422}
\label{sec:S0X_1Lh_dHA}
\textbf{【流堂崇拜】你先?我先?我地可能都會搞錯!|哈該書1\_1-15|20230422 [S0X-1Lh\_dHA]}
\newline
\newline
連結: \href{https://youtube.com/watch?v=S0X-1Lh_dHA}{\texttt{ https://youtube.com/watch?v=S0X-1Lh\_dHA}} ~~~~ 語音日期: 2023-04-22 
\newline
\newline
\hyperref[sec:3PY1nwdp_0k]{\small{< < < PREV SERMON < < <}}
~
\hyperref[sec:index_chronic]{\small{[返順時目]}}
~
\hyperref[sec:index_scriptual]{\small{[返順卷目]}}
~
\hyperref[sec:VNZbDAiXlG0]{\small{> > > NEXT SERMON > > >}}
\newline
\newline
哈該書 1:1-15-20230422
\newline
\begin{longtable}{cl}
\hline
\hline
章節 & 經文 (和合本修訂版)\\
\hline
1:1 & \begin{tabularx}{0.7\textwidth}{X} 大流士王第二年六月初一,耶和華的話藉哈該先知向撒拉鐵的兒子猶大省長所羅巴伯和約撒答的兒子約書亞大祭司傳講,說: \end{tabularx} \\ \\ \relax
1:2 & \begin{tabularx}{0.7\textwidth}{X} 「萬軍之耶和華如此說,這百姓說,建造耶和華殿的時候還沒有到。」 \end{tabularx} \\ \\ \relax
1:3 & \begin{tabularx}{0.7\textwidth}{X} 耶和華的話藉哈該先知傳講,說: \end{tabularx} \\ \\ \relax
1:4 & \begin{tabularx}{0.7\textwidth}{X} 「這殿荒涼,你們自己還住天花板的房屋嗎? \end{tabularx} \\ \\ \relax
1:5 & \begin{tabularx}{0.7\textwidth}{X} 現在,萬軍之耶和華如此說,你們要省察自己的行為。 \end{tabularx} \\ \\ \relax
1:6 & \begin{tabularx}{0.7\textwidth}{X} 你們撒的種多,收的卻少;你們吃,卻不得飽;喝,卻不得足;穿衣服,卻不得暖;領工錢的,領了工錢卻裝入有破洞的袋中。 \end{tabularx} \\ \\ \relax
1:7 & \begin{tabularx}{0.7\textwidth}{X} 「萬軍之耶和華如此說,你們要省察自己的行為。 \end{tabularx} \\ \\ \relax
1:8 & \begin{tabularx}{0.7\textwidth}{X} 你們要上山取木料,建造這殿,我就因此喜樂,且得榮耀。這是耶和華說的。 \end{tabularx} \\ \\ \relax
1:9 & \begin{tabularx}{0.7\textwidth}{X} 你們盼望多得,看哪,所得的卻少;你們收到家中,我就吹去。這是為甚麼呢?因為我的殿荒涼,你們各人卻只為自己的房屋奔走。這是萬軍之耶和華說的。 \end{tabularx} \\ \\ \relax
1:10 & \begin{tabularx}{0.7\textwidth}{X} 所以,因你們的緣故,天不降甘露,地也不出土產。 \end{tabularx} \\ \\ \relax
1:11 & \begin{tabularx}{0.7\textwidth}{X} 我命令乾旱臨到土地、山岡、五穀、新酒、新油和地上的出產,也臨到人和牲畜,以及一切人手勞碌得來的。」 \end{tabularx} \\ \\ \relax
1:12 & \begin{tabularx}{0.7\textwidth}{X} 那時,撒拉鐵的兒子所羅巴伯、約撒答的兒子約書亞大祭司,和所有倖存的百姓都聽從耶和華-他們神的話,就是哈該先知奉耶和華他們神差遣所說的話;百姓在耶和華面前存敬畏的心。 \end{tabularx} \\ \\ \relax
1:13 & \begin{tabularx}{0.7\textwidth}{X} 耶和華的使者哈該奉耶和華差遣對百姓說:「我與你們同在。這是耶和華說的。」 \end{tabularx} \\ \\ \relax
1:14 & \begin{tabularx}{0.7\textwidth}{X} 耶和華激發撒拉鐵的兒子猶大省長所羅巴伯、約撒答的兒子約書亞大祭司,和所有倖存百姓的心,他們就來為萬軍之耶和華-他們神的殿做工。 \end{tabularx} \\ \\ \relax
1:15 & \begin{tabularx}{0.7\textwidth}{X} 這是在大流士王第二年六月二十四日。 \end{tabularx} \\ \\
[1ex]
\hline
\hline
\end{longtable}
$^{1}$頂姐妹平安,在網上送拜的頂姐妹平安.
剛才謙明說得很對,可以來到神日殿一起敬拜祂不是必然的事.
另一樣我領受的是可以在神日殿中侍奉祂更加不是必然的事.
因為在這個星期一,我起床的時候,我發覺我喉嚨很痛,聲沙咳出痰很濃.
接著不久,我小兒子就告訴我,爸爸我兩條線.
我馬上就進去廁所,看看我多少條線,幸好那兩條線是黏在一起的.
但是今天都是一條線,難保明天不是,最怕是星期五才來兩條線怎麼辦呢.
我馬上就WhatsApp給潘Sir,潘Sir潘Sir,有沒有什麼指引呢?.
因為萬一都要穩定,後補後備,我們有很多同工在場.
除了清心之外,清心是失聲的.
潘Sir就說,現在這些只是風土病,只要你沒有失聲就行了.
但是我心寒寒,當時我聲沙咳,我不敢告訴他.
我跟他說潘Sir,幫我祈禱吧.
所以這幾天我一直在喘氣,跟老婆都聊笑了,跟兒子都說笑了.
昨天去踢球都走了半個小時,怕真的聲音不夠.
所以大家幫我祈禱吧,我希望今天可以有足夠的氣,講完整個行動.
不知道大家還記不記得上次我站在台上和大家分享的那個行動.
在去年十月,我記得當時和大家用七千人這個故事去思考.
我們在暴政下所產生的那種無力感,我們應該怎樣面對.
這次我就很想帶大家去看看,性機裡面另一班更加大群的人.
他們每次都會作一點點的內在反思,思考一下我們和神之間的關係.
思考一下我們生命中某些優先次序.
這些既細微又重要,但是又很容易搞錯的事情.
所以今天我很想和大家分享,我跟潘Sir說.
不行,就算我中了我也要回來,如果萬一失聲你可不可以幫我讀.
幸好今天沒有找他,所以開始分享之前我就不讀了.
邀請大家幫我讀這十五字經文.
上次我用七千人去和大家作思考.
今天我打算用五萬人和大家一起去思考剛才所說的優先次序.
或者大家聽完之後會覺得,剛才我讀的裡面,怎樣讀都找不到五字或五萬字.
但如果熟悉猶太人回歸歷史的弟兄姊妹都會知道.
除了我們在哈蓋書第一章有記載他們的事情之外.
我們在以色列記和利希米記都有相關的記載.
而我剛才所說的五萬人群體就出於這兩本書.
而五萬這個數字就在以色列記第二章和利希米記第七章裡面計算出來.
怎樣計算呢?我們一起看看吧.
根據以色列記第一章的記載.
耶和華親自去激動波斯王古列的心.
叫他頒下禹旨讓猶太人回到耶路撒冷.
去重建家園,重建聖殿.

$^{41}$在所羅巴巴和耶穌大人的帶領之下.
聖經在以色列記第二章64-65節裡面記載.
當時願意回歸的人有回眾四萬二千三百六十人.
木被有七千多人,又有唱歌的男女.
所以將他們加起來就大約有四萬九千八百多人.
我將他們四寫五入之後就簡稱為五萬人.
正如以色列記第一章一節所提供的資料.
五萬人生活在波斯王古列年代.
當時大概是主前538年.
距離猶太人第一次被擄.
即是主前608年來說已經足足有70年.
如果以第三次被擄的時間去計算的話.
即是在主前586年來計算的話.
這班人當時已經有50年之久.
所以我們有理由相信這五萬人當中.
大部分的人都是在被擄期間在巴比倫出生.
他們可算是巴比倫土生土長的猶太人.
另一方面我們一般人對這個群體都有個錯覺.
因為這班猶太人就像以色列人一樣.
是很苦的.
他們流浪異鄉的二等公民一定是很苦.
生活一定像以色列人一樣.
被奴役,被迫,被苦待,被歧視.
但其實實際的情況並不是我們所想像中那麼差.
因為如果我們對比其他經文.
我們會看到這班猶太人在當時的生活.
無論在經濟,宗教和社會層面上.
都是得到一定程度的自由和尊重.
我們回看社會方面.
我們可以在耶利米書29章一節中.
看到耶利米先知寄信給被擄猶太人.
經文中看到巴比倫政府容讓被擄到巴比倫的猶太人.
和留在耶路撒冷的猶太人之間.
可以有書信的自由來往.
所以我們可以看到他們之間的通訊自由.
並不會被人說是勾結外國勢力.
更加不會把他們當作外國代理人去看待.
在宗教方面我們可以看到.
根據《撒爾利書》六章十字記載.
當時的猶太人某程度上還可以享有自己的獨經,禱告和敬拜生活.

$^{81}$他們的宗教自由似乎都沒有被巴比倫政府有很大的限制.
第三方面在經濟方面.
雖然他們在經文中沒有直接看到.
巴比倫政府給猶太人享有什麼程度的耕田和經商的自由.
但我們可以肯定的是.
這五萬人在回歸耶路撒冷的時候.
他們的經濟能力是有一定的能力的.
因為我們在以色列的二章六十四至六十七節看到.
這班回歸的猶太人帶著七千個腹皮.
當時四萬多人帶著七千個腹皮.
五個人有一個腹皮.
將雷,馬,駱駝加起來有八千多隻.
雖然當中有部分是當地人捐獻給他們.
但我們可以肯定.
他們不是窮到窿的那種.
所以以上三方面我們可以看到.
這五萬人一定不是像埃及的那班以色列人.
因為在巴比倫生活得很苦.
被迫去巴比魯拍攝.
沒有一口好食.
所以被迫要回耶路撒冷.
相反他們可能在巴比倫已經有很好的適應.
甚至發展得不錯.
他們可能有自己的房屋.
有自己的田地.
甚至有自己的事業.
有自己的群體.
更加有相對的社會地位.
其實這五萬人大可以繼續留在巴比倫.
馬照跑,舞照跳,鼓照炒.
賺多點錢.
但他們願意回歸.
讓我們看到他們不是那種等次的人.
他們不是只顧自己的利益.
馬照跑,舞照跳,鼓照炒.
賺多點錢.
而將神的話語置之不理.
他們不是那種只要賺多點錢.
就可以將公義,憐憫拋諸腦後.
他們不是那種只要賺多點錢.

$^{121}$就可以不再追求公義,真理,民主,自由的普世價值.
我相信他們不是那種只要賺多點錢.
就可以不理會社會上被壓迫,被迫迫,被無理囚禁的人的需要.
我相信他們不是那種只要賺多點錢.
就可以泯滅良心,出賣人性的人.
弟兄姊妹,有沒有想過.
這一班人他們被譽為成長在一個背叛,混亂,墮落,不信的巴比倫國度裡.
他們為什麼可以出於污泥而不染呢?.
我相信是他們父母很有關係.
我相信他們父母在他們成長當中.
不斷地將神的話語,聖經教導,人性的道德底線.
將猶太人的歷史鉅細無遺地教導他們.
以致他們有正確的價值觀.
以致他們有正確的人生的優先詞綴.
以致他們可以出於污泥而不染.
所以上星期六嘉Sir提醒我們這班50+的人.
要幫助90後,00後的年青人.
今天我在這裡都很想鼓勵我們當中的90後,00後的年青人.
尤其是在當中以為人父母的或者將會成為人父母的弟兄姊妹.
我們要好好地幫助我們下一代.
好好地教導我們的孩子.
幫助他們建立一個正確的價值觀.
教導他們有一個正確的生命優先詞綴.
將神的話語,聖經教導,人性應有的道德底線.
將香港的歷史,將香港所發生的事情.
鉅細無遺地去教導他們.
讓他們正確地認識真理,認識歷史.
鉅細無遺地去教導他們.
以致他們不會沾染上這些馬照跑,舞照跳,鼓照炒.
賺多點錢的世俗思想.
親愛的爸爸媽媽們,親愛的準父母們.
在小朋友成長過程的時期裡.
家庭和學校是對他們影響最大的.
亦都是他們最信任的地方.
亦都是他們最願意開放自己去接受被塑造的地方.
但是當我們回看我們現在的教育制度.
我想我們是時候站起來.
擔起更多的責任.
在家庭當中作教導的工作.
以致他們有能力去辨別是非.

$^{161}$有能力去抗衡那些謊言.
有能力去分辨那些荒謬的事情.
親愛的爸爸媽媽們.
我們所處的這個世界.
那種背叛,那種混亂,那種不信,那種墮落.
真是不下於巴比倫,是嗎?.
我們孩子真是需要你們.
將教導他們的時間.
配伴他們成長的時間.
去擺在優先的位置上.
你們願意嗎?.
另一方面,這五萬人響應回歸.
回去耶路撒冷重建聖殿.
除了要扶上剛才所說的.
可能房屋,田地,社交圈子,社會江湖地位.
還要面對很多困難和挑戰.
例如回歸屠城的風險.
回去耶路撒冷面對那些頹垣敗瓦.
面對人生路不熟.
要面對重新來過的挑戰.
但是他們心智依然堅定.
可見他們當時是將神的利益,神的心意.
擺在自己的優先位置上.
所以,親愛的弟兄姊妹們.
我們今天回到教會當中.
我們在跟隨主的路途上.
我們將誰的心意擺在最前面呢?.
我們記得3月18日.
大家有沒有印象?.
潘Sir在台上凝需.
他說自己信心少.
大家還記得他凝需的樣子是怎樣嗎?.
今天潘Sir不在,大家可以放膽說.
哎,不行,立仔幫幫忙.
童工說我信心少.
我承認.
童工說我信心少.
不要趁潘Sir不在,我們作大反.
一次就夠了.
大家看到這段片.

$^{201}$你看看凝需的樣子.
凝需的樣子.
潘Sir不好意思.
凝需的樣子.
他多滿足,多開心.
我從未見過有人凝需得這麼開心.
但我相信.
潘Sir的樣子是很滿足.
為什麼呢?.
因為他看到我們Flow Church這個群體.
絕大部分的弟兄姊妹.
都是在跟隨主的路途上.
願意去付上代價.
絕大多數的弟兄姊妹.
當時沒有被這107級的樓梯所嚇倒.
更加沒有被這些夾攝.
不方便,不懂路,不懂去的念頭去阻礙.
而不去崇拜.
我想可能弟兄姊妹會說.
你會不會說得太大?107級而已.
有什麼大事?.
其實我覺得我和潘Sir都看到了一種起點.
107級是一種起點.
我們看到原來我們身邊有這麼好的弟兄姊妹.
都是願意去學習.
去操練自己向跟隨主的路途上.
去學習付上代價.
學習以神為優先.
以神的心意為優先.
就好像那五萬人一樣.
是不是?.
去到以色列的第三章.
尤其在這五萬人.
回到耶路撒冷之後沒多久.
他們就很齊心地聚集在耶路撒冷.
以色列的第三章一節.
是這麼說的.
如果.
不好意思,我聽到.
他們是如同一人這樣聚集在耶路撒冷.

$^{241}$他們如同一人這樣聚集在耶路撒冷做什麼呢?.
經文在三章二至五節進一步帶出.
這群猶太人聚集在一起.
要為神去祝壇.
向神去獻祭.
同時間他們又出錢.
又出力地開始動工去建造聖殿.
因此他們在回歸短短一年之內.
完成了聖殿的奠基工程.
可見他們對神的敬畏.
對信仰的認真.
雖然耶路撒冷是他們的故鄉.
但當中大部分的人都是巴比倫土生土長的人.
他們可以算是一群初度貴境.
除了要適應新環境之外.
他們應該還有很多生活上的事情要處理.
例如他們要找地方安頓自己的家人.
預備建屋,買建材.
甚至乎要替別人拿回被霸佔的農地,地方.
很多事情需要放上心思.
但他們依然堅持將神的事情放在優先位置上.
為神的事情放上心思,負少,時間和金錢.
我記得在2021年年尾.
我們Flow Church推出了一個Outflow Mission.
是鼓勵留下在外地的教務同工和弟兄姊妹.
一起齊心在當地去共建一個敬拜上帝的群體.
去共建一個信仰上同行的群體.
我記得當時只是短短十天.
就有超過一百位弟兄姊妹和教務回應.
他們大多數都是初度貴境.
他們都是人生路不熟.
有些剛剛到達還沒有租屋.
還是住在Airbnb.
有些就算已經租到屋.
都還要為找工作,考車牌去傷腦筋.
但他們依然很熱心地,很齊心地回應Outflow Mission.
跟他們傾談的時候.
我很感受到他們對神的敬畏.
對重建同行群體的渴望.
對自己的敬拜,心靈的重視.

$^{281}$雖然當中有些弟兄姊妹因為某些原因.
譬如交通或者其他特別原因.
最終都未能夠在Outflow Mission裡面參與.
但我知道他們大部分都很積極地.
在自己的家附近找Local Church.
去建立自己的信仰上同行群體.
有些住得較遠的.
連Local Church都找不到的.
所以他們現在都迫在網上跟我們一起崇拜.
我相信神都一定可以看到你們的擺相.
你們的優先次序.
並且閱立了你們的敬拜.
我們再看聖經.
我們看到好景不常.
在第四章當五萬人慶高彩烈為奠基儀式慶賀的時候.
聖經記載他們遇上難阻,遇上很多挑戰.
經文說有些住在當地的外族居民來難阻他們.
他們千方百計去擾亂他們,恐嚇他們,阻攔他們.
甚至有人用錢不斷收買當地的官員來難阻他們.
消磨他們建建的心智.
在這些惡劣的環境下.
在這些外來的壓力,威脅下.
甚至有些政權的威脅下.
這五萬人就開始退縮.
他們的手就放軟下來.
結果他們在回歸第二年.
大約是主前535年建電工程就完全停頓下來.
一停就停了15年.
時間一斬過了15年之後.
我們就看到經文的下蓋書.
於是神就差派下蓋先知去到當中傳講神的話語.
叫他們重新啟動重建聖殿.
在這裡.
在這裡,《華本聖經》在五章一節中.
描述神叫下蓋先知去到他們去跟他們講勸勉的話.
而呂鎮忠亦本亦說鼓勵他們.
不過如果我們參考原文.
《聖經》記載.
我們會看到新譯本的記載.
會貼近多些.

$^{321}$因為原文真的沒有勸勉,鼓勵這些字眼.
純粹就好像新譯本這樣翻譯.
就說下蓋先知去到跟猶太人傳講信息而言.
另一方面,如果我們再看回剛才我們所讀的.
《哈該書》一至四節中.
我會發覺神差派下蓋先知去跟這五萬人講.
不單是沒有這種鼓勵,勸勉的語氣.
相反,他是帶著那種雜備.
因為神在《哈該書》第二節中.
稱呼這五萬人為「這百姓」.
他不再用那種親切的角度.
「我的百姓」,這些稱呼他們.
如果我們比對當時神在《以賽亞書》中.
稱呼那些勃逆的以色列人.
是稱呼他們「這百姓」的時候.
我們就會看到神當時的心情.
當時真的去雜備這五萬人.
神要下蓋先知去雜備這五萬人.
雜備他們什麼呢?.
經文在第四節中展示神對他們不滿的地方.
經文說:「這殿盈盈荒涼,你們自己還住天花板的房屋嗎?」.
當中的天花板房屋原文是指有蓋的意思.
他指向的不單是有屋頂的房子這麼簡單.
更是指向他們用很高貴的木材去遮蓋房屋.
或者去裝飾牆壁.
來建造一個非常華麗的房子.
就好像新年代本中所翻譯的一樣.
「我的聖殿盈盈荒涼,你們卻在建築華麗的房子裡」.
下蓋先知批評他們不是批評他們住屋過份華麗.
下蓋先知責備他們的地方是.
他們既然有時間有金錢去裝潢自己的家.
為什麼他們任憑聖殿去荒涼去荒廢呢?.
他們這一刻到底是將神放在什麼位置上呢?.
神叫下蓋先知指出這五萬人在這十幾年裡.
因為被外來的社會環境,被政治,政權的威脅.
被經濟的影響.
不知不覺地將生命的優先次序放錯了.
放亂了.
不再像回歸初期那樣.
將神的事情,心意放在優先位置上.

$^{361}$取而代之的是將自己的喜好,享受,事情放在神的事情,心意之先.
就像經文第九節所說.
「我的殿荒涼,你們各人卻顧自己的房屋」.
弟兄姊妹,這幾年我們都經歷了很多社會的變遷.
經歷了漫長的政治,疫情,經濟上的衝擊.
我們有沒有像這五萬人一樣.
不知不覺地放錯了我們的生命的優先次序呢?.
最近我在網上看到一個有關香港基督徒在疫情期間參與網上崇拜的調查報告.
報告當中發現原來有大約4\%的人參加超過五個教會的網上崇拜.
參加兩至五間的有17\%.
參與一間教會的網上崇拜有53\%.
完全沒有參與網上崇拜的有大約25\%.
根據這個比例,到了疫情的中期,即大約2021年底.
香港每四位基督徒當中有一個信徒沒有參與網上崇拜.
究竟有多少人因為環境和疫情衝擊而改變了他們的優先次序.
不再看重敬拜的生活?.
我相信一定不會全部.
因為我知道有些長者不懂得用這些數碼.
但我相信當中也有不少.
因為相信大家都遇過聽過身邊的人.
尤其是在修例期間.
很多信徒看到很多不公義的事情.
他們懷疑上帝的公義.
以至有些人生氣上帝.
有些懷疑上帝有沒有存在.
之後就沒有再去崇拜,去教會.
可能連網上崇拜也不去.
求主憐憫,求主憐憫這班弟兄姊妹.
幫助他們去重建他們的敬拜生命.
回到神的懷抱當中,重新投入教會生活.
另一方面,可能弟兄姊妹也會問.
在復常的情況下,我們留在家裡網上崇拜又如何看呢?.
在這方面,我們也可以用優先次序去審視.
首先,崇拜沒有一個最好的形式.
也沒有一個最好的地點,地方,處所.
就好像耶穌基督和井旁婦人所說.
你們拜父也不在這山,也不在耶路撒冷.
真正的敬拜並不在於任何的地點和形式.
真正的敬拜是在於我們內心的態度和焦點.
在你們拜父這四個字裡.

$^{401}$我們可以看到兩方面和崇拜有關的教導.
首先,拜父這兩個字帶出我們敬拜的中心是上帝.
所以我們在崇拜當中,每一個環節,每一個人.
都應該圍繞著神,以神為優先.
以神得榮耀為優先,以神為中心.
而你們這兩個字是一個眾數.
某程度上,看到是多於一個人的意思.
某程度上帶出崇拜的群體性.
而事實上,我們在很多聖經其他地方.
我們都看到神是閱立我們集體的敬拜和教導.
所以我很同意有目者在網上說.
崇拜是神指引集體朝見神的約會.
一群得救重生的信徒聚集在一起朝見神.
靠著聖靈的能力和引導,將神當得的榮耀歸給神.
因此,我相信如果我們一群人聚集在一個地方.
在一個家中,用網上崇拜去敬拜神.
去崇拜,神都必定閱立.
不過如果我們因為身體不適,要照顧家人.
可能有些特別的原因,需要個別留在家中.
去網上崇拜,我相信神都會明白,祂都會閱立.
但如果我們考量純粹以自己的方便和舒適.
可以放棄雙腳去唱詩歌,去喝杯咖啡去崇拜.
我們可能要想一想,我們那次的崇拜.
我們是否以神為中心,以神為優先.
最近,我在Outflow Mission中.
認識一位姐妹,和她聊電話.
我知道她是幾年前剛剛移居英國.
到Po 後,她很努力尋找教會去崇拜.
她加入教會侍奉,融入教會.
不過,最近因為教會有教務.
禁止她選擇一首詩歌去崇拜,敬拜詩歌.
她仍毅然離開服侍多年的教會.
之後她嘗試在自己附近,去找其他教會崇拜.
但都發覺,沒有一間可以容納到她對公義念文的渴求.
但很感恩,她在朋友的推介介紹之下.
她知道流唐有網上崇拜,於是在Outflow Mission回應.
很可惜,我們在該區還沒有聚會點.
所以她現在每個星期都迫於在網上以流唐的YouTube來崇拜.
話說回頭,大家可能會覺得.
選一首詩歌不讓,都要離開?.

$^{441}$大家知不知道這首詩歌叫什麼名字?.
Alex聽著了.
是榮譽降下.
我說,這首詩歌只是為香港祈禱,這樣都不行?.
兄弟姐妹,我們看到原來那個人在教會裡.
回到教會當中,可能站在侍奉崗位上.
所以他們的心可能都不是將神放在優先的.
他們在做審查,做自己喜好的事情.
他將自己的喜好,將自己的政見放在敬拜上帝的優先上.
其實在我們成長過程裡.
我們都可能試過不知不覺地搞錯這幾方面的次序.
我們可能曾經,不知道大家有沒有試過.
又或者大家有沒有聽過身邊的人這樣說.
就是我看看誰先講完講到,才決定回不回教會.
我有聽過,有些姐妹我親耳聽過.
今天在裡面崇拜,我一首歌都不唱.
我問他為什麼?.
我沒有一首喜歡唱的.
我們在敬拜當中,都很可能像他們這樣.
不知不覺地以自己為中心.
以滿足自己為優先.
不單止在敬拜,在崇拜當中.
就算在侍奉,在奉獻,我們都可能會這樣.
我不知道在這段時間,可能那種社會對我們的經濟都很大衝擊.
不知道會不會聽到有些人會說.
或者心裡想,讓我夠用才奉獻吧.
讓我有時間才侍奉吧.
我不知道我們大家有沒有這種掙扎.
我們這些掙扎會不會在這個環境當中.
在疫情當中,在社會當中被放大了呢?.
甚至被扭曲了呢?.
或者又被打亂了呢?.
盼望我們這次趁在這個服償日子裡.
我們一起檢視一下我們自己在這方面的態度.
校正我們的優先次序.
以致我們每個星期都可以獻上輕香的祭給上帝.
我們每一天都可以將自己獻上作為活祭.
去榮耀神,去尊崇神.
我們一起禱告.
親愛的天父上帝,你是各位尊貴榮耀的獨一真神.

$^{481}$你是配得我們全心全意的敬拜.
你是配得我們一切至高無上的重讚.
願你自己在我們當中得著應得的榮耀.
得著當得的讚美.
祈禱奉耶穌得勝名球.
\newpage



\section{以斯帖記 1:1-22-20230429}
\label{sec:VNZbDAiXlG0}
\textbf{【流堂崇拜】她和她最後的倔強|以斯帖記1\_1-22|20230429 [VNZbDAiXlG0]}
\newline
\newline
連結: \href{https://youtube.com/watch?v=VNZbDAiXlG0}{\texttt{ https://youtube.com/watch?v=VNZbDAiXlG0}} ~~~~ 語音日期: 2023-04-29 
\newline
\newline
\hyperref[sec:S0X_1Lh_dHA]{\small{< < < PREV SERMON < < <}}
~
\hyperref[sec:index_chronic]{\small{[返順時目]}}
~
\hyperref[sec:index_scriptual]{\small{[返順卷目]}}
~
\hyperref[sec:D8sOzznkhGg]{\small{> > > NEXT SERMON > > >}}
\newline
\newline
以斯帖記 1:1-22-20230429
\newline
\begin{longtable}{cl}
\hline
\hline
章節 & 經文 (和合本修訂版)\\
\hline
1:1 & \begin{tabularx}{0.7\textwidth}{X} 這事發生在亞哈隨魯的時代,亞哈隨魯從印度直到古實統治一百二十七個省, \end{tabularx} \\ \\ \relax
1:2 & \begin{tabularx}{0.7\textwidth}{X} 就是亞哈隨魯王在書珊城堡中坐國度王位的那些日子。 \end{tabularx} \\ \\ \relax
1:3 & \begin{tabularx}{0.7\textwidth}{X} 他在位第三年,為所有官員和臣僕擺設宴席,有波斯和瑪代的權貴,各省的貴族與領袖在他面前。 \end{tabularx} \\ \\ \relax
1:4 & \begin{tabularx}{0.7\textwidth}{X} 他把他榮耀國度的豐富和他偉大威嚴的尊貴給他們看了許多日子,共一百八十天。 \end{tabularx} \\ \\ \relax
1:5 & \begin{tabularx}{0.7\textwidth}{X} 這些日子滿了,王又為所有住書珊城堡的百姓,無論大小,在御花園的院子裡擺設宴席七日。 \end{tabularx} \\ \\ \relax
1:6 & \begin{tabularx}{0.7\textwidth}{X} 院子裡有白色棉和藍色線,用細麻繩、紫色繩繫在白玉石柱的銀環上,又有金銀的床榻擺在紅、白、黃、黑大理石鑲嵌的地上。 \end{tabularx} \\ \\ \relax
1:7 & \begin{tabularx}{0.7\textwidth}{X} 用金器皿盛酒,有很多不同的器皿,照王的厚意提供豐富的御酒。 \end{tabularx} \\ \\ \relax
1:8 & \begin{tabularx}{0.7\textwidth}{X} 飲酒有規定,不准勉強人,因為王吩咐宮裡所有的臣宰,讓人各隨己意。 \end{tabularx} \\ \\ \relax
1:9 & \begin{tabularx}{0.7\textwidth}{X} 瓦實提王后在亞哈隨魯王的宮內也為婦女擺設宴席。 \end{tabularx} \\ \\ \relax
1:10 & \begin{tabularx}{0.7\textwidth}{X} 第七日,亞哈隨魯王飲酒,心中快樂,就吩咐在他面前侍立的七個太監米戶幔、比斯他、哈波拿、比革他、亞拔他、西達、甲迦, \end{tabularx} \\ \\ \relax
1:11 & \begin{tabularx}{0.7\textwidth}{X} 請瓦實提王后頭戴王后的冠冕到王面前,讓各民族和官員觀看她的美貌,因為她容貌美麗。 \end{tabularx} \\ \\ \relax
1:12 & \begin{tabularx}{0.7\textwidth}{X} 瓦實提王后卻不肯遵照太監所傳的王命前來,所以王非常憤怒,怒火中燒。 \end{tabularx} \\ \\ \relax
1:13 & \begin{tabularx}{0.7\textwidth}{X} 按王的常規,辦事必先詢問知例明法的人。那時,王詢問通達時務的智慧人, \end{tabularx} \\ \\ \relax
1:14 & \begin{tabularx}{0.7\textwidth}{X} 就是在王左右常見王面、在國中坐高位的波斯和瑪代的七個大臣,甲示拿、示達、押瑪他、他施斯、米力、瑪西拿、米慕干: \end{tabularx} \\ \\ \relax
1:15 & \begin{tabularx}{0.7\textwidth}{X} 「瓦實提王后不遵照太監所傳的王命,照例應當怎樣辦理呢?」 \end{tabularx} \\ \\ \relax
1:16 & \begin{tabularx}{0.7\textwidth}{X} 米慕干在王和眾官長面前回答說:「瓦實提王后這事,不但得罪王,並且有害於亞哈隨魯王各省的臣民。 \end{tabularx} \\ \\ \relax
1:17 & \begin{tabularx}{0.7\textwidth}{X} 因為王后這事必傳到眾婦人那裡,她們就會藐視自己的丈夫,說:『亞哈隨魯王吩咐瓦實提王后到王面前,她卻不來。』 \end{tabularx} \\ \\ \relax
1:18 & \begin{tabularx}{0.7\textwidth}{X} 今日波斯和瑪代的眾夫人聽見王后這事,必向王所有的官長照樣說,如此必造成無數的藐視和憤怒。 \end{tabularx} \\ \\ \relax
1:19 & \begin{tabularx}{0.7\textwidth}{X} 王若以為好,請降諭旨,寫在波斯和瑪代人的條例中,永不更改,不准瓦實提再到亞哈隨魯王面前,把她王后的位分賜給比她更好的妃子。 \end{tabularx} \\ \\ \relax
1:20 & \begin{tabularx}{0.7\textwidth}{X} 王的諭旨一傳遍全國,國土縱然遼闊,凡作妻子的,無論丈夫是尊貴或卑賤,都必尊敬他。」 \end{tabularx} \\ \\ \relax
1:21 & \begin{tabularx}{0.7\textwidth}{X} 王和眾官長都以這話為美,王就照米慕干的建議去做。 \end{tabularx} \\ \\ \relax
1:22 & \begin{tabularx}{0.7\textwidth}{X} 王下詔書,用各省的文字、各族的語言通知各省,使凡作丈夫的在家中作主,各說本地的語言。 \end{tabularx} \\ \\
[1ex]
\hline
\hline
\end{longtable}
$^{1}$《一字二十二字》 詞:陳曦 曲:陳曦.
阿哈徐老作王.
從印度直到古實.
統管一百二十七省.
阿哈徐老王在書山城的宮登基.
在位第三年.
為他一切首領臣服設擺筵席.
有波斯王馬袋的權貴.
就是各省的貴就與首領在他面前.
他把他榮耀之國的豐富.
和他美好威嚴的尊貴.
給他們看了許多日.
就是一百八十日.
這日子滿了.
又為所有住書山城的大小人民.
在御園的院子.
屢次擺筵席七日.
有白色,綠色,藍色的杖子.
用細麻繩,紫色繩.
從銀環內繫在白玉石柱上.
有金銀的床塔.
擺在紅,白,黃,黑玉石的鋪石地上.
用金器皿賜酒.
器皿各有不同.
御酒神多.
足顯王的厚意.
喝酒有禮.
不准勉強人.
因王吩咐宮內的一切神材.
讓人覺取己意.
皇后雅實堤在阿哈徐老王的宮內.
也為婦女設擺筵席.
第七日.
阿哈徐老王喝酒.
心中快樂.
就吩咐在他面前.
侍納的七個太監.
米胡曼,比斯他,哈波娜,比格他.
阿拔他,西達,甲家.
請皇后雅實堤.

$^{41}$投帶皇后的冠冕到王面前.
使各等臣民看她的美貌.
因為她容貌神美.
皇后雅實堤卻不肯.
遵太監所傳的皇命而來.
所以皇臣發怒.
心如火燒.
那時.
在王左右上常見王面.
國中坐高位的.
有波斯皇馬代的七個大臣.
就是甲斯娜,士達,阿瑪他.
他斯斯,米力,瑪西娜,米姆干.
都是達時務的名節人.
按皇的常規.
辦事先順民之禮明法的人.
皇問他們說.
皇后雅實堤不遵太監所傳的皇命.
照例應當怎樣辦理呢.
米姆干在王和眾首領面前回答說.
皇后雅實堤這事.
不但得罪王.
並且有害於王國省的臣民.
因為皇后這事必傳到眾婦人的耳中.
說哈徐老王吩咐皇后雅實堤到王面前.
她卻不來.
她們就藐視自己的丈夫.
今日波斯王馬代的眾婦人聽見皇后這事.
必向王的大臣照樣行.
從此必大開藐視和憤怒之端.
皇若以為美.
就降旨寫在波斯王馬代的禮中.
永不更改.
不准雅實堤再到王面前.
將她皇后的位分賜給她還好的人.
所降的旨意全片通過.
所有的婦人.
無論丈夫貴賤都必尊敬她.
王和眾首領都以米姆干的話為尾.
王就照這話去行.

$^{81}$發詔書用各省的文字.
各族的謊言通知各省.
使為丈夫的在家中作主.
各說本地的謊言.
上個月我知道粵題是Sorry的時候.
當我構思講章的時候.
我和一個英文不太好的小朋友.
溫習的時候.
他就見到一個這樣的題目.
請PowerPoint的弟兄幫我按一下.
看到了嗎?.
按吧.
看到一個英文題目.
就是「Sorry for your loss」.
那個小朋友就立刻回答.
「Say! Say sorry」.
然後我就跟他說.
Sorry不一定是say.
可以是feel.
可以是work to be.
今天是粵題Sorry的最後一講.
我想到的是.
一個人無論是和上帝說Sorry.
和別人或自己說Sorry都好.
是需要經過一番自我的反省.
才可以很油衝地說一句Sorry.
所以今天就藉著《耳私貼記》第一章.
我們去想想個人內省的課題.
我不知道當提起《耳私貼記》的時候.
大家第一印象是什麼?.
當然很容易想到《耳私貼記》.
因為書卷就是以它命名.
還有很萬能的經文.
總之有什麼情況就跟你說.
「焉知你得了皇后的位分」.
「不是為了現今的機會嗎?」.
就叫你去做.
除此之外.
大家對《耳私貼記》還想起什麼呢?.
我不知道大家.

$^{121}$高高為我們讀《耳私貼記》第一章的時候.
我不知道過往有沒有很用心地去看過這章經文.
大家有沒有留意到當中提及的皇后雅實題呢?.
皇后雅實題只是很短暫出現在《耳私貼記》第一章.
她因為拒絕阿哈徐魯的召喚.
拒絕出席王的筵席.
被王和一班王室顧問褫奪了她皇后的位分.
勒令她永遠不可以再見王.
很快就被罪事者寫出整個《耳私貼記》的罪事.
當然如果我們有稍稍讀過《耳私貼記》的話.
我們可能會被書卷裡面的哈曼.
即是一個惡人.
或是一個英雄的耳私貼去吸引.
無疑他們兩個的形象是很典型.
自然會令人留下深刻的印象.
但雅實題呢?.
我想說雅實題對於我們基督教或基督徒來說.
是一個很有趣的課題.
為什麼這樣說呢?.
傳統教會通常去解讀我們剛才所讀的經文的時候.
會認為雅實題拒絕亞哈徐魯的要求.
他們第一個反應是甚麼呢?.
他們不聽話.
是一個不服從的表現.
他們是應該被廢的.
所以很多時候教會對這個敘述有一個很片面的看待.
又或者我們再看《耳私貼記》的時候.
很多時候她都是一個很順從.
不說話,默默做事的一個女性.
不過當我再看《耳私貼記》第一章的時候.
發現這個幾乎被人遺忘的角色.
其實有很多讓我們反省和思考的地方.
今天就透過雅實題看看對我們有甚麼啟發.
首先介紹一下雅實題.
根據猶太拉比對聖經的傳息.
雅實題是一個甚麼身份呢?.
雅實題是巴比倫王室的後裔.
是尼布格尼薩爾薩的元孫.
白沙薩的孫女.
年紀輕輕的雅實題就在但爾利牆上的文字.

$^{161}$即是那晚預言白沙薩被殺的那一晚.
事件就記載在但爾利書第五章.
她就被當時的馬代人放過了.
即是沒有留下活口,沒有殺了她.
但劉氏王,即是但爾利書第六章.
那個扔但爾利入獅子坑的那一位.
就利用雅實題作為鞏固自己政治的一個資產.
利用它做甚麼?.
將她嫁給自己一個很忠實的臣子.
一個前途無可限量的年輕軍人.
瓦哈徐魯.
為甚麼要安插一個人在瓦哈徐魯身邊?.
就好像放個善人在他身邊.
以斯帖記一章一節這樣說.
「這是發生在瓦哈徐魯的時代.
瓦哈徐魯從印度直到古實統治127個省.
這裡沒有稱呼他為王.
所以猶太拉比估計.
當時瓦哈徐魯還未正式成為王.
他的確是大樓市一個很驍勇善戰的一個將帥.
不過給我們看到的是.
他的行軍策略已經促成了波斯帝國的一個初型.
已經有127個省的規模.
而在這次大規模擴張之後.
帝國裡面出現了一場爭奪權力的鬥爭.
當然既然剛才我所說.
瓦哈徐魯是一個很能打的將帥.
他的軍事成就自然就是成為了.
統治波斯帝國的一個合理候選人.
又例如就成如我之前所說.
瓦哈徐魯和瓦什提其實只是一宗政治婚姻.
當瓦哈徐魯的勢力未穩固的時候.
他只可以容讓瓦什提在他身邊.
但當他慢慢手握軍政權的時候.
他就想剷除前朝軍王安插在他身邊的人.
他很希望可以從皇后手中取回所有的控制權.
希望可以集中所有的事務在他手中.
一章三節說他在位第三年就擺設賢直.
這個賢直其實是想將皇后剷除其中一個計劃付諸實行.
經文講到就是剛才我們讀經文的時候.

$^{201}$就看到第一個賢直為期長達六個月.
講的是他宴請帝國裡面一些很重要的人物.
包括波斯的軍官貴族地區長官等等.
在裡面向他們展示自己的財富和軍政權力等等.
然後緊接著第一個賢直之後就有第二個賢直.
就是向書山城大概是首都首府的所有人.
去開放自己宮殿的花園.
第八節講的是讓他們可以肆無忌憚地去喝酒.
為甚麼呢?.
是要確保所有人包括阿哈徐老自己都喝夠酒.
然後可以方便將之後發生的事情歸咎於醉酒的影響.
而不是單純一個精心策劃了很多日的宮廷政變.
一章九節都講給我們聽.
就是在王擺設賢直的時候.
其實皇后雅實緹都在宮裡面招待一班婦女.
她當時其實是有事要負責的.
不過我們再看的時候.
醉使者將婦女的賢直和王的賢直平衡地放在一起.
但是我想我們剛才聽的時候.
發覺醉使者花了很多筆墨去描述王所擺的賢直.
講了很多的佈置等等.
但是皇后的賢直卻只是很簡單一句地交代.
這裡給我們看到一個很強烈的對比.
我們看到這裡的時候會問.
為甚麼波斯男女要分開喝酒,吃飯呢?.
不是單純因為男女之隔等等.
而是曾經有學者對當時的波斯文化有一個這樣的研究.
他說當時代的波斯人有幾個特點.
第一,他們會喝酒.
第二,他們很喜歡搶東西.
第三,波斯男人通常都是好色之徒.
這樣的安排,為甚麼男女要分開喝酒,吃飯呢?.
是要令女性不會在賢直期間.
被那些喝酒的男人去借醉行兇,被侵犯.
這就是男女分開喝酒,吃飯的其中一個原因.
到了第十至十一節就說.
喝酒喝到第七天.
王就很興奮.
阿哈徐魯就將瓦實緹召喚到男性的賢直裡面.
《經文》說美其名是想向眾人炫耀皇后的美貌.

$^{241}$不過如果我們再聽一些背景的時候.
是否純粹想炫耀皇后的美貌呢?.
或者說得白一點,其實他想皇后來是甚麼呢?.
是要她去招呼一班被大量酒精影響的男人.
至於是怎樣的招呼呢?.
這是值得我們去思考的.
不過我們再看《經文》,究竟是想她怎樣招待呢?.
《經文》說是請七個太監.
然後請瓦實緹投帶皇后的冠冕到王面前.
瓦實緹聽到的是投帶皇后的冠冕.
猶太拉比的詮釋是.
要求皇后只是帶皇后的冠冕來到這班男性的賢直.
簡單來說就是叫她帶著後冠.
然後在王和所有醉酒的賓客面前赤身露體.
我想說這個要求不單是要她將自尊放低.
更加是對一個女性一個很大的侮辱.
我們再看《經文》,七這個數字其實在第一章三次重複出現.
首先第一次是第七天.
然後王吩咐七個太監傳召皇后瓦實緹到來.
然後在皇后要想怎樣懲處瓦實緹的時候.
她也是向身邊七個智慧人諮詢.
而瓦實緹這個名字在這章也七次被提及.
七這個數字對猶太人來說是一個很特別的數字.
但對瓦實緹來說也對她的命運起了一個很關鍵的作用.
醉使者特別提到王的心因酒而歡喜.
除了因為酒,我也相信是因為賓客面對著王的富有和慷慨.
對他有很多的讚美,阿諛奉承等等.
王在賓客面前不斷的吹捧下.
阿哈徐老慢慢的自我膨脹到一個地步.
要求七個太監領帶著皇冠的皇后瓦實緹到王面前.
向民眾和領袖展示她的美貌.
希望用這個作為筵席的壓軸戲.
我想說阿哈徐老這個做法其實就是將瓦實緹當作一件戰利品.
除了炫耀自己在軍政權的成就之外.
還想藉著皇后贏得一班大臣對她的效忠,支持,擁戴.
我想說這個做法就像將瓦實緹當作其中一個財產.
她在物化瓦實緹.
將她和她的身體當作娛樂醉酒賓客的玩物.
我想說當下瓦實緹可以有兩個選擇.
第一就是失去尊嚴.

$^{281}$她可以去,但她在皇后的日子可以繼續做皇后.
第二就是不去.
她會失去皇后的位份,一切的享受.
但她可以保持著作為人的尊嚴.
那一刻瓦實緹心裡究竟在想甚麼呢?.
如果她被虛榮心去充分投老.
或者她習慣了享受福縣廣大的國家所帶給她的榮華富貴.
她可以將這種侮辱視作等閒.
她可以跟自己說.
其實她只是展示自己有多漂亮.
她可以這樣說,經文也是這樣說.
她可以跟自己說,其實一群人都醉了.
延籍過後發生了甚麼事.
大家都可能不記得了.
或者瓦實緹也可以跟自己說.
那個時代就是鼓勵女人不加思索地去順從男人,聽男人的話.
當時很多女性都是默默無聲地去遵從這樣的要求.
有時候很多人覺得順服是一個讓人得到提升,攀升的機會.
有些人因而選擇背棄了自己的信念.
瓦實緹可以給自己很多很多的原因去游說自己.
去吧.
不過,瓦實緹面對著眼前一個絕對的權力.
我想她經過一番反省和思考.
她最後選擇了甚麼.
她沒有屈服,她選擇了抗爭.
她拒絕迎合黃,不合理的要求.
她拒絕成為男人的性對象和玩物.
她沒有將她自己本身作為皇后的既得利益放在一個不能動搖的位置.
她寧可面對因為反抗而帶來的失去或危機.
她也致力維護她作為人的尊嚴.
她選擇不做一個順民的皇后.
而是用行動去反抗這個無理的要求.
她跟黃說不,她say no.
我想說她的拒絕是顛覆了男人對女人物化的願望.
她也行使了她作出選擇的自由.
瓦實緹給我們看到的是.
她不只是外表漂亮.
而是她能夠權衡外面複雜的情況.
並且能夠作出決定,是一個很自主的女性.
我想強調的是.

$^{321}$瓦實緹那種堅持或抗爭並不是廉價的.
女性的反抗在當時是沒有人能猜到的.
並且當你反抗的時候.
男人會視為一個很大的威脅.
所以經文給我們看到的是.
一群有權有勢,但醉酒不理智的男人要怎樣?.
他決定要將皇后瓦實緹從她的位置拉下來.
她的反抗令她失去皇后的地位.
但她卻願意承擔這個後果.
願意犧牲自己皇后的位份.
也不願意輕易拋下自己作為人.
一個很重要的核心價值.
儘管她放棄的是什麼?.
放棄的是一個從印度到埃塞俄比亞.
遍及127個省的王國.
她的自尊和榮譽感比整個王國都要高.
都要重要.
雖然瓦實緹從此在《以斯帖記》中不再被提及.
缺席了.
不過,她的缺席我們可以看為是一種留白.
她仍然在發聲.
她用行動去表明.
她沒有屈服在不合理的強權下.
19世紀英國有位著名的貴官詩人這樣說.
他說一個人的自我反省.
自我認識.
自我控制.
這三者可以將一個人的生命.
引向一個很大的力量.
我想說這個力量的大.
可以令瓦實緹用行動.
向當時握有權力的男性.
去表達她的立場.
她拒絕為醉酒的丈夫.
放下她作為女性的尊嚴.
她就是因為她敢於表達立場.
而被記錄下來.
我們想想.
如果瓦實緹就像當時其他女人一樣.
隨波逐流.

$^{361}$男人叫她做什麼就做.
不想想是否合理.
不反抗.
只是做一個順命的皇后.
或許歷史很快就忘記她.
因為她只是一個為利益隨波逐流的女人.
不過瓦實緹沒有盲目地服從不合理的當權者.
或許她那一刻已經做好了.
隨時做好準備死的準備.
的確我們看到.
她打算死是一個很合理的推測.
因為她所認識的瓦剋隨路是什麼.
她的反抗激怒了她.
原文形容她怎樣生氣.
氣得好像被火燒一樣.
帝國裡也沒有人站在她那邊.
如果我們再看的時候.
在整個敘事.
有一個皇室貴就向她伸出援手.
大概就是皇的意志已經凌駕了全國人的常識.
縱然大家都知道這是一個完全不合理的命令.
不過她在現實裡看到的是.
有七個皇室顧問.
她是趨炎附勢.
他們把握機會向皇提議.
為瓦剋提議度身訂做一個法例.
怎樣呢?.
做低她.
趕走她.
過往去讀《以斯帖記》的時候.
我們都會視《以斯帖記》,《末底改》.
是普爾哲的英雄.
因為的確是他們兩個合力拯救猶太人.
免遭滅族之難.
不過在我看來.
我覺得瓦剋提也是一個英雄.
她拒絕在皇和一群醉酒的人面前貶低自己.
她選擇重視自己固有的信念.
不願意屈服在丈夫無理的要求.
不願意接受那些示意對她的踐踏.

$^{401}$我會形容瓦剋提為個性很倔強的人.
她不是利用自己的美貌.
女性的身份去為自己謀利益.
不是利用這些去找著數.
相反她很清楚她是一個有原則.
有所為有所不為的女人.
瓦剋提的反抗令我想起五月天的《倔強》.
我想大家看看下一張PowerPoint.
《倔強》這首歌其實已經推出了很久.
在五月天,當她在滾石唱片年代.
這首歌的歌名亦叫做《Stubborn》.
不過當她轉到上一張圖片.
即是她轉到《相信音樂》即是現在的公司.
歌名亦叫做《Persistence》.
頂姐妹,我不知道你怎樣看「固執」這個字.
固執看起來是一個很負面的詞彙.
不過中國人有句說話叫做「擇善而固執之」.
重要的是我們為了甚麼去固執呢?.
其中有幾句歌詞很觸動我.
我不會唱歌,放心.
它是這樣說的.
「當我和世界不一樣,那就讓我不一樣」.
「堅強對我來說,就是以剛克剛」.
「如果我們對自己妥協,如果對自己說謊」.
「即使別人原諒,我也不能原諒」.
還有一句.
「逆風的方向更適合飛翔」.
「我不怕千萬人阻擋,只怕自己投降」.
我想說這堆歌詞.
很有種「道之所在,雖千萬人吾往矣」的氣魄.
即是我堅持的是真理的話.
縱然有千萬人阻擋我.
但我都一樣繼續去.
瓦薩提的倔強,瓦薩提的堅持.
是因為他重視他心裡一個很重要的價值觀.
他寧可冒著失去地位的風險.
都要維護作為女性為人的尊嚴.
拒絕王,不公義,不文明的要求.
我想說瓦薩提的拒絕行動.
被他當時代的人指責是一種傲慢,不明智.

$^{441}$又或者是現在大男人主義的教會.
都用相似的力度去抨擊他.
不過我想說通過他反抗屈服在男性手下.
他其實是在為一些無法為自己發聲的女性.
做了一個榜樣.
即使這樣做要付上很高昂的代價.
但因為他有自己所重視的信念.
這個信念勝過世界主流的價值觀.
他就是身體力行地實踐對女性的解放.
縱然革命尚未成功.
不過他至少都走了第一步.
所謂的希望不是確信事情一定會有好的結果.
而是無論結局都肯定自己所做的每一度微小.
都有他自己的價值.
瓦薩提就是用他一人之力去抵抗整個波斯的體制.
看起來好像徒勞無功.
不過我想說他的行動.
卻是為這個荒謬的世代賦予多一點意義.
頂姐妹,今天上帝將我們擺在此時此地.
或者我們都面對著很多荒謬的要求.
我們可能會面對著一些無理的指責和批評.
當外人對我們不明白,不解,鄙視.
或者是迫不得已的時候.
我們能不能像瓦薩提一樣去堅持自己的信念呢?.
堅持上帝置放在我們心中的良心和真理呢?.
作為記憶徒,我們能否在艱難的處境.
仍然堅持進行上帝頒布給我們的誡命呢?.
我們願不願意去踐行上帝給我們的良心呢?.
在瓦薩提的敘述中,不知道大家看到些甚麼呢?.
有沒有看到你自己的影子?.
我們是否處於一個窒礙我們體驗神完全榮耀的境況呢?.
昔日我們可能對一些不好的環境.
或者對我們靈性有破壞的處境.
我們可能會捉緊就命.
我們可能會給自己很多理由.
一人在江湖,身不由己,不行的.
我們可能曾經默許過這些事情加諸在我們身上.
但今天我們會否因為上主的緣故.
而對這些對我們具破壞性的環境說NO呢?.
我覺得如果我們曾經順命,認命,捉緊就命.

$^{481}$我們仍然可以.
所以今天要祂說NO,永遠都不會太遲.
我想說,作為上主的門徒.
縱然處身在艱難的現實.
但我深信亦無改上帝賜給我們.
靈性,正直,公義的特質.
祂就是藉著我們,使用我們在此時此地.
在這個時代,在這個世代去彰顯.
我相信,如果我們選擇堅持上帝的真理的時候.
我相信總有一天,會證明我們是站在上帝那一邊.
如果我們選擇為上帝而做的時候.
我相信總有一天,上帝會為我們所作真誠的行動,抉擇.
對我們作出肯定.
總有一天,我們會被上帝所認同.
求主幫助我們,讓我們一同低頭祈禱.
親愛的上主,我們有時也要向你承認.
在現在的世代,在現在的處境.
要堅持對的信念,持守真理,真的很不容易.
特別是在現在荒謬,顛倒是非,邪惡的世代.
大主你將真理,靈性,良心賜給我們.
讓我們知道何謂善,何謂美.
主呀,求你加能賜力給我們.
讓我們在這裡仍然能夠散發人性美好的光輝.
也求你讓我們對你仍然有盼望.
我們深信上主你公義的審判.
有一天會臨到這個地上.
惡人會遭報,義人會被主你所肯定和掌上.
求主就是這樣幫助我們.
在這條路上繼續走下去.
願意我們走得好,終結得好.
求你就是這樣幫助我們.
多謝你聽我們祈禱.
奉耶穌基督的聖名自祈求.
阿們.
主呀,求你加能賜力給我們.
願意我們走得好,終結得好.
阿們.
主呀,求你加能賜力給我們.
願意我們走得好,終結得好.
阿們.

$^{521}$多謝您收睇時局新聞,再會!.
\newpage



\section{}
\label{sec:gRf39gjSNbM}
\textbf{《致餘民及流散者:給香港基督徒的神學八課》第二季第3課|20230512 [gRf39gjSNbM]}
\newline
\newline
連結: \href{https://youtube.com/watch?v=gRf39gjSNbM}{\texttt{ https://youtube.com/watch?v=gRf39gjSNbM}} ~~~~ 語音日期: 2023-05-12 
\newline
\newline
\hyperref[sec:D8sOzznkhGg]{\small{< < < PREV SERMON < < <}}
~
\hyperref[sec:index_chronic]{\small{[返順時目]}}
~
\hyperref[sec:index_scriptual]{\small{[返順卷目]}}
~
\hyperref[sec:u6GL1Cm7cwU]{\small{> > > NEXT SERMON > > >}}
\newline
\newline
$^{1}$(第四章 基督徒).
我們是時代之子.
是被上帝揀選.
在這個年代見證耶穌的基督徒.
我們身於這個年代.
被這個年代分散.
有人流散到海外尋覓理想.
有人繼續在本土奮鬥下去.
但又如何.
基督徒仍然需要作基督徒.
香港人仍然是香港人.
究竟上帝的旨意如何.
我們應該如何生活.
我們應該如何為主而活.
無論你身處何處.
只要你是香港人.
邀請你跟我們一起思想.
這個流散年代的信仰.
致願民與流散者.
給香港基督徒的神學百科.
.
不過有一點關係.
勉強也說得通.
今天我們會討論一個主題.
就是身份認同.
就是身份認同這個詞語.
當我們想到這個詞語時.
便想到上帝和我們之間的身份.
坦白說,我跟潘老師預備的時候.
這不是我自己提出的主題.
因為我不懂得說.
這個詞不是我常用的詞語.
很少提到身份認同這個詞語.
這個詞也不算是神學詞語.
應該是一些心理學,社會學的詞語.
所以我很少用這個詞語說話.
平時很少提到身份認同的問題.
當我預備這課的時候.
我發覺對自己也有很大幫助.
雖然我很少用這個詞語說話.

$^{41}$但我有這些問題.
我有身份認同的問題.
我覺得自己也面對這些問題.
所以今天也是….
可能潘老師稍後問答時.
我們可以多說一點.
怎樣看我們身份認同這個詞語.
我覺得這個詞語.
中文翻譯出來也很特別.
身份,身份證.
用香港人的身份.
即是我們香港人對於我們一個群體.
一個很獨特,很獨特的身份.
就是我們香港人.
當這個社會不能滿足.
某個群體的身份認同的時候.
簡單來說,當香港人的身份價值.
還未在社會中被重視的時候.
因為多元,對吧?.
特別是香港社會是多元的.
當多元的緣故不被重視的時候.
他們就會強烈渴求,渴望,有尊嚴.
並且被重視.
重視自己的身份.
如何理解我們一群香港人.
所以這成為了一個政治訴求.
這就是政治最基本的原則.
但今天的課是神學課.
除了我們內裡的心理和成長之外.
除了我們說的社會政治之外.
我們更加在想.
我們作為教會群體.
如何理解我們的身份.
特別是留堂.
今天我也會嘗試去定義一下.
留堂的身份.
如果要說的時候.
這也是我們很少去想的問題.
所以今天我們會說一些.
我不太熟悉的題目.

$^{81}$因為這不算是很神學的題目.
起碼上一部劇是這樣.
甚麼是身份呢?.
最早提出身份認同的人.
大家可能都知道.
就是艾瑞爾·艾森.
一個很有名的心理學家.
他用了「身份」這個字來提出理論.
來說我們人的成長.
我想這是大家熟悉的東西.
甚麼是intimacy.
或是alignment.
因為人在不同的成長期裡.
都有不同的階段去達成.
這也是一個人累積的經驗.
或是他的信念,價值和回憶.
去訴出他如何看待自己.
所以我們人的心理成長.
或是人的成長.
都是很重要的.
我們如何看待自己.
我們在這個世界裡存在的時候.
在不同的人生階段裡.
我們如何理解自己.
他更說.
如果我們實踐不到.
就會出現成長的問題.
這些都是很普遍的心理學的看法.
另一個就是社會學上.
我們稱為collective identity.
原來身份認同不單單是個人層面.
社會學家更覺得.
這是一種群體裡的身份認同.
這個認同不單單是一個人.
而是一群人.
他們如何互動.
如何各自理解自己.
形成一個群體身份.
所以這個身份.
這個群體的身份.

$^{121}$會令我們收到這幾年的政治.
這是一位很有名的學者.
Francis Fukuyama.
是一位很有名的社會政治學家.
這本書很新的.
2018年才出版.
他提出了一個身份政治的問題.
我覺得他能夠用這個理論.
去回顧我們過去幾年做了什麼.
他說原來我們一個群體的身份.
其實我們一群人.
是有一個很強烈的需求.
先用香港人的身份.
我們香港人對於我們一個群體.
有一個很獨特,很獨特的身份.
就是我們香港人.
當這個社會不能滿足.
某個群體的身份認同的時候.
簡單來說,當香港人的身份價值.
還沒有在社會中被重視的時候.
因為多元,特別是在香港社會是多元的.
當因為多元的緣故.
而不被重視的時候.
他們就有一個很強烈的學求.
來去渴望,有尊嚴.
來去被重視.
重視自己的身份.
如何去理解我們一群香港人.
所以這就成為了一個政治訴求.
這就是所謂的政治最基本的原則.
就是因為一個多元的緣故.
這群人很想更加被尊重.
他們本身的存在.
所以這件事同時也被近代的全球化.
以及社交媒體的興起.
更加促進了這件事.
我們在社交媒體中.
更加容易去視察到我們的身份.
這件事未必是靠Facebook,IG,社交媒體.
來看到我們香港人是怎樣的.

$^{161}$我們的身份價值很容易在當中交流.
以及被認同.
所以當社會制度.
不能夠滿足群體的尊嚴的時候.
就成為了一個很強烈的民主運動.
當然香港人的成果是失敗了.
但福克斯媽媽所說的.
如果成功的話.
其實又會回到另一個極端.
就是會成為一種民族主義.
或者是成為一個很強大的政治力量.
這可以是一個很正面或負面的事情.
所以我們要嘗試去平衡.
以及我們作為一個人和一個群體.
如何平衡個人的權利和自由.
以及集體身份這件事.
我們不單單是個人.
我們更加是一個集體的身份.
同時我們也會尊重自己個人的認同.
所以這就是我們.
很簡單,我也不是這方面的人.
但也簡單地說一下.
身份在心理學,社會學和政治裡面.
有幾個我們都會觸碰到的字眼.
當然我們要說的.
我自己特別有很大的感受.
就是身份危機.
這個字我聽過很多次.
但我們發現原來這個字眼.
其實也不是….
原來是發生的.
我自己也說過,我有中年危機.
我35歲的時候.
是一個完全合法地….
本來35歲至65歲是一個中年危機的開始.
我剛好35歲的時候就有中年危機.
是甚麼呢?原來就是….
一個人到了那個階段的時候.
你會發覺你開始不知道自己在做甚麼.
那時候會覺得很迷惘,很迷失.

$^{201}$或者是有點虛空.
原來這樣的感受.
就是身份認同的危機問題.
原來「identity」這字眼不是那麼遙遠.
我們怎樣看自己.
我們怎樣去反映自己的現狀.
還有這個世界裡面.
你的生活或社會的世界.
和你自己的身份之間的差距.
正成為了身份認同危機的問題.
所以就算你未過35歲也好.
你仍然有不同的階段裡面.
可能會面對的問題.
特別是在這幾年裡.
當我們面對著….
可能你是移民的,留在香港的.
這幾年裡發生的事情.
戴口罩或轉教會.
這個也是我們….
今天我們會說的問題.
就是這個世界的轉變.
令我們對於自己的身份被動搖.
或者是不被滿足.
或者需要尋求更新的時間.
我自己看了一些資料.
我覺得有一個不錯的分類.
是這樣的.
其中一個身份認同危機叫「displacement」.
稱為與世界和過去中斷.
你會發現這個世界突然轉變了.
不再是你以前認識的世界.
這個就是「displacement」最簡單的情形.
你發現你所熟悉的過去.
已經不能有一個「continuity」.
你發現過去的香港,過去的社會.
不再是這樣的時候.
你就出現了這樣的「displacement」.
你覺得自己活在一個….
不是在繼續活在自己的世界裡.
所以很多人有不同的做法.

$^{241}$你會發現自己沒有根.
你會想尋根,卻尋不到.
你無法追溯一些你能夠認識的日子和世界.
所以很多人開始想懷舊.
或者回想以前.
或者去看某些共同回憶.
有很多不同的方法.
老人家說舊時.
這也是一種身份認同危機.
他發現現在的社會發展得太快.
不能夠適應.
或者我們不認識香港.
我們就開始想如何算數.
所以任何情況.
當我們發現我們所認識的過去.
和現在有一個差別的時候.
這就是出現了我們其中一個.
身份認同危機的問題.
我們不知道如何回應.
這是第一個我們所說的身份認同現象.
不知道你們有沒有這樣的情況.
無論在你的教會裡.
或者在香港裡.
或者你在外國裡.
你都可能有這樣的情況出現.
發覺無法支撐.
你發覺你的過去好像突然被剝奪了.
第二個稱為「建構」.
就是本身的字解作「退縮」.
或者是困囚.
我們發覺無論是身體被困住.
或者是身心靈被困住.
你發覺無法實踐自己.
這個環境不能夠活出真正的你.
或者你以前覺得可以.
突然覺得真正的不是你.
所以當你不能夠活出.
你想活出的自己的時候.
你就會發覺有這樣的問題.
其實真的有很多都會發生.

$^{281}$我35歲的時候就是這樣.
我覺得自己不想愁遠教書.
教一輩子,就這樣重複下去.
有些事情我是想做的.
但我做不到,就覺得很迷茫.
我自己的夢境也是這樣.
我21歲的夢境也是這樣的情況.
發覺我浪費了大半年時間.
原來我內心有些事情想做.
但我做不到.
或者我未能實踐我想做的事情.
所以無論是你內心的想法.
或者是性別.
或者是你對自己的渴望.
你發覺你很想去做.
但你被很多環境….
我聽過很多時候.
一些媽媽結婚多年.
她發現自己被困在家裡.
很多這種情況都是這樣.
所以當我們的生活.
越來越成為一種束縛的時候.
你就很想活出你真正的自己.
你發覺你每天起來的時候.
發現越來越不像自己.
做一份工作,可能大家都是這樣.
轉工的原因.
或者是你自己甚至離開香港的原因.
都是這樣.
就是因為發覺現在的生活.
不再能夠成為自己.
「我不屬於這裡」.
「我要去別的地方」.
「我要成為別人」.
不知道你們有沒有試過.
想成為另一個人.
認真地想成為另一個人.
可能不再是這樣的生活方法.
很想去一個完全不同的方法生活.
這是我很多次在行屍裡.

$^{321}$經常想著很想去過另一種生活.
所以我們總是發覺這種轉變.
因為這個環境不能讓你活出自己的時候.
就會出現這樣的身分認同問題.
雖然字好像很複雜.
但很簡單,就是我們迷失或迷惘.
第三個.
剛才那兩個基本上有甚麼方法解決.
問題在哪裡?.
無論是當你被人發現世界的擾亂.
當你迷失的時候.
或者當你發覺你被囚禁在生活裡的時候.
我們都很自然地會尋求自由.
很想打破這個僵局.
尋找你過去熟悉的世界或生活.
或者去實踐你真正的自我.
這種對於自由的渴望.
這也是我們性中很少提及的重要課題.
上帝給我們追求自由的重要性.
就是我們作為一個人.
我們很想去實踐自我.
因為我們的選擇.
上帝給我們生命.
我們來實踐我們的生命.
是一個很重要的元素.
所以當我們無論是面對迷失或擾亂.
我們都很想憑著一些方法去尋找自由.
但當我們面對尋找突破的時候.
我們會出現第三種問題.
就是「壓抑自由」.
當我們一下去尋求自由的時候.
不知道你們有沒有試過這樣.
當你一下去突然離開舊的階段.
但那一刻你突然間太自由.
你還未能定義自己的下一個步驟的時候.
那一刻也是另一種危機.
你會迷失自己.
你不知道如何定義自己是誰.
太多的自由,太多的可能性.
令我們不知道如何去成為自己.

$^{361}$所以這是我們尋求身分認同時.
其中一個反動力.
當我們尋求自由的時候.
我們反而會面對另一種迷失的情況.
我們不知道如何去做.
可能你會很自由到某個地步.
你會失去很多東西.
人會不知道如何去站立.
所以這很簡單.
我們可以說是用這三種情況去嘗試.
嘗試去想想我們不同的狀況.
無論是我們所說的「迷失」.
「壓抑自由」或「壓抑自由」.
嘗試去想想我們有沒有遇到這種情況.
第一個,我們說說移民大姐妹.
當我們離開香港的時候.
其實可能也是面對這種情況.
我們用這套理論去理解自己的情況.
可能我們發覺香港不再是香港.
香港的社會價值已經不再是以前那樣.
所以我們就要離開.
離開本身就是一種尋求過去的東西.
所以發覺很多移民國家都是這樣.
唐人街是在那個年代的.
80年代香港在哪裡?就是加拿大.
張學友或周國榮的歌曲仍然在播放.
教會的詩歌也是一樣.
仍然在播放那個年代的詩歌.
所以我們在樓下也是一樣.
如果你突然在樓下.
過去的話也會在榮耀廈播放.
所以都是這種情況.
我們會嘗試尋找我們的過去.
所以這是一個很自然的現狀.
但當我們移民的時候.
我們同時面對著一種.
overwhelming freedom的情況.
我們有太多的可能性.
我不知道自己應該怎樣去定義自己.
我去哪個城市都可以.

$^{401}$我做什麼工作都可以.
但我接下來應該怎樣做人呢?.
我怎樣去定義自己呢?.
反而是有一種迷失在這裡.
所以對於流散的鄧小平來說.
他們很明顯地尋求自由.
嘗試不要被一些東西去困擾.
去尋求這種新的方向.
但我們仍然面對著很多認同問題.
香港人是一個很好的collective identity.
我們有一種固定的共同體.
但我們仍然面對著這種情況.
我們未必能夠在一剎那裡.
找到我們的身分認同.
另外,我們一群留在香港的姐妹.
我們發現香港越來越不熟悉的時候.
我們過去的香港不再熟悉.
我們該怎麼辦呢?.
這很值得大家回到小時候去討論.
我們現在是在甚麼階段?.
我們在發現以前的香港不是這個時候.
我們沒有一個比較可行的方向.
去理解自己.
我們只是一個過客嗎?.
我們純粹對這個世界,這個社會.
沒有任何積極的期盼嗎?.
我們只想賺錢,生活好一點就算了.
我們怎樣去理解自己在香港?.
我們同時也要活出我們的真我.
我們如何在香港的社會裡.
去做回我們自己.
我們是否有些東西被…….
無論在法律上或恐懼中.
我們不能真正地活出來.
當然,我再說一次.
那個collective identity是重要的.
那個群體是重要的.
我們仍然很需要那個群體.
來承載我們這些有價值觀念的東西.
所以當我們發現世界不同了.

$^{441}$我們更加需要一個collective identity.
保留在香港.
我們會否…….
又是另一種overwhelming freedom.
我們會否覺得現在反正沒有甚麼可以留戀.
我們反而過分地去找一些我們想做的事.
但我們找不到呢?.
所以問題是…….
我再回應一句.
我們有沒有身份的台階呢?.
以前的時間.
你真正的自己.
有沒有被這幾年裡打敗了呢?.
第三個我想特別說的就是教會.
我們作為留堂人.
我們其實也面對這三種問題.
我特意加了一個隱許度.
因為我很少說「留堂人」這個詞.
很少稱呼大家為「留堂人」.
我一向都很強制自己不要說這些詞.
因為這些詞我都很…….
該怎麼說呢?我都未有認同危機.
我不是很認同我們這個群體是怎樣的.
我不是叫「Full Church人」.
但我覺得好像很奇怪.
我不想叫大家「Full Church人」.
甚麼「留堂人」之類.
因為我不想特意要給大家一個名字.
所以今天我們討論的問題.
我們這些留堂人.
其實我們也有這種問題出現.
我們有沒有好好地去處理呢?.
我們跟過去的信仰生活.
是否已經中斷了?.
它可以是好,也可以是壞.
我們怎樣跟以前的教會生活.
是否有一種分離?.
可能是因為我們離開教會.
這裡有很多離開教會的人.
你離開教會的時候.

$^{481}$其實你是否發現.
你以前信主的情況是沒有了.
有一個很大的分離.
你嘗試尋求一個找回以前教會.
或者信主的美好來留堂這裡.
還是你嘗試去實踐自己的信仰.
但以前教會實踐不到.
所以我嘗試去打破心靈的囚禁.
去尋回真正的自己.
同時我們又會怎樣?.
有超越自由.
反正我們星期天不用去崇拜.
我們有太多的自由.
令我們突然離開前教會.
好像很多事情都已經完全不同了.
所以我覺得這是一個頗有信心的問題.
對我們這群人來說.
因為我們似乎都是面對著這樣的心靈問題.
因為留堂一直都是一個很奇怪的體系.
留堂無法定義它是多少人的教會.
它不是普通的教會.
因為YouTube裡的人很多.
很多人都看留堂.
但我們又好像是一個很模糊的群體.
我們似乎不是定義自己的群體.
我們一起想想.
我們怎樣定義自己的群體.
在一個真正的社群的時候.
我們怎樣去看待自己.
我們會否稱自己為留堂人.
還是其他名字.
我們怎樣去理解自己的群體.
可能你的組別會比較好.
比較容易看到.
但對整個全群教會來說.
我們好像很不容易定義自己.
這些都是一些問題.
我覺得值得我們去思考.
無論你是離開香港.
或者你在香港生活.

$^{521}$或者你回到FourShot.
我們都值得用這個角度.
去輕輕想想.
我們面對著的每一個問題.
接著就說一些我比較熟悉的事情.
一些比較聖經上的事情.
我們在聖經裡怎樣去說身份認同.
或者從我們的信仰裡.
我們怎樣去好好地回應.
剛才所說的很多問題.
其中一個我覺得很簡單.
我們基督徒有甚麼身份認同.
我們有甚麼身份.
其實不難說的.
聖經裡有很多身份.
你是甚麼.
上帝所愛的兒女.
你是門徒.
你是神的僕人.
你是弟兄姊妹.
你是天國的子民.
你是神的形象.
有很多這些.
但我想說的是.
這些說出來其實很簡單.
但你知道這些代表甚麼.
你知道你是神的兒女.
是否等於沒有甚麼問題.
好像似乎不是這樣.
雖然我們有很多.
在聖經裡給予我們的身份.
我們有很多不同的角色和崗位.
你是門徒,你是跟隨耶穌的人.
有很多不同的身份.
身份是不缺的,是我們的名字.
但我想我們怎樣好好地.
將這些身份.
聖經裡的身份更有說服力地.
去幫助我們面對這些生命危機的事情.
跟大家看一首詩.

$^{561}$一首潘福華的詩.
這首詩叫作「我是誰」.
「我是誰」.
這首詩是潘福華在監獄裡寫的.
當時他在柏林的車叫監獄裡.
坐牢坐了好幾年.
他開始有一個很明顯的身份認同危機.
他開始對我這個身份.
對自己整個的生命迷失.
或者是一種真正的危機.
所以他寫了這首詩.
我們可以一起唸這首詩.
他說「我是誰」.
「他們常對我說」.
「當我步出牢房」.
「從容,愉快,堅定的」.
「彷似一位從城堡出來的國族一般」.
這也是潘福華在外表上.
很多人都看到他仍然是一個很有氣質的人.
仍然是一個很信靠神的人.
在監獄裡生活著.
「我是誰」.
「他們常對我說」.
「當我和肉觀說話」.
「輕鬆,友善,清醒的」.
「彷如我正在發洩好另一半」.
仍然可以對人有很多不同.
有禮貌和非常淡定的舉止.
「我是誰」.
「他們常對我說」.
「當我在不幸的日子」.
「平靜,微笑,高傲」.
「彷如一個慣常勝利的人」.
仍然是一樣.
面對著這麼艱難的監獄日子.
仍然可以保持鎮定,微笑.
面對著一切.
「我真是他們所說的那樣嗎」.
「還是只有我自己心裡明白」.
「不安,饑渴,病態」.

$^{601}$「彷如一隻籠中鳥」.
「為呼吸而掙扎」.
「彷如被人捏住喉嚨一半」.
這就是他們很深刻地體會.
自己內裡的情況.
這也是一個建構.
他們是被人困住.
不能活出自己真正想做的人生.
無論在心靈或身體裡都是這樣.
「我想念色彩,花朵,鳥語」.
「我奢望安慰的話語和同門的愛戀」.
「我痛恨獨裁和深空狹窄」.
「我而不定,期待大事降臨」.
「思念千里之隔的朋友」.
「卻只能無力地顛簸」.
「我的禱告,思想和舉動都令人煩厭」.
「都是虛空」.
「我虛弱地想隨時告別一切」.
這很明顯是對於過去的扭曲.
他似乎對過往的日子很掛念.
他仍然覺得現在的生活.
他的狀態完全割裂了.
所以他找不到自己的狀態.
我想說,坐牢坐得好並不容易.
坐牢是覺得自己已經坐牢.
其實是一個不容易的想法.
「闖明」中有很多不同的.
對於外面的世界,對於自己的實踐.
對於過去有很多不同的思考.
特別是有很多時間.
「我是誰?這個還是那個?」.
究竟是外表的那個,還是內在的那個呢?.
「我是誰?今天是這樣的人」.
「明天是這個人嗎?」.
「兩個又是我嗎?」.
「人前的偽君子,在自己面前的可憐軟弱者」.
「還是在我裡內是一個被打敗的軍隊」.
「在已經贏得勝的情況下,無序地後退」.
他就問「我是誰?」.
他似乎不是很確定.

$^{641}$我想說這是一個掙扎.
我不是說他的外表不是真的.
他仍然是一個很有禮貌的人.
仍然是一個微笑,有信念的人.
這不是假的.
但他似乎不知道應該怎樣定位自己.
面對著這種情況.
他不知道怎樣活出自己.
他怎樣跟現狀有甚麼關聯.
他怎樣在現在的情況下做人.
他完全不知道.
以這個證據來看.
他放大到香港的時候也是一樣.
如果以證據來說.
香港正面對著監牢.
我不是說監牢,而是監牢.
在這種情況下.
我都遇到一個問題.
我們是否真正能夠活出真我呢?.
我們是否知道怎樣跟現時生活的世界相處呢?.
這是我們今天可以問的問題.
他最後開始有一點曙光.
他說我是否在贏了的情況下無序地後退呢?.
當然這是很明顯的.
說的是基督耶穌得勝.
即是耶穌已經贏了.
所以就算他坐在監牢裡.
他也是在贏了的情況下.
像是打輸了的軍隊一樣.
所以這是一個很奇怪的風事.
但也很明白的情況.
雖然他知道耶穌已經贏了.
但他就像輸了的人一樣.
他正在面對自己.
所以有一個問題.
他究竟是誰?.
他究竟用了甚麼.
用了哪把尺子來開始想他自己呢?.
明白嗎?.
我想我們的問題是.

$^{681}$我們不是不知道自己做了甚麼.
問題是我們不知道誰懂得.
我們應該開始理解自己的起點.
我們是用甚麼尺度.
或是用甚麼量度單位來看自己.
才是對的.
所以兩者都是真的.
不是外表是假的,內在才是真.
而是我們應該怎樣去看自己呢?.
最後很重要的,一句很簡單的結束.
他說我是誰.
這個孤獨的問題嘲笑著我.
無論我是誰,你認識我.
我屬於你,上帝.
這正是整首詩最後的總結.
對於「我是誰」的問題.
這個問題不斷嘲笑著自己.
但最後的討告.
也是一個很重要的形象.
無論我誰也好.
他還未弄清楚.
他的身分認同危機的問題.
他不是知道自己的身分.
才會離開監獄或有出路.
不妨在還未掌握自己是誰的狀態下.
來結束這首詩.
所以我們面對著身分認同危機時.
我們的答案不一定是我們知道自己是誰.
我們是一種生命問題.
可能到了某個階段,我們會有點迷失.
但答案不一定是我們知道自己是誰.
我知道自己是誰,就做人.
我們永遠都有一種距離.
我和自己永遠都有一個差距.
你嘗試拉近這個差距是對的.
但你不一定要完全消滅這個差距.
我仍然有一個問題是留白的.
我未必知道這一刻我完全是誰.
或我的身分是怎樣.
但我知道甚麼?.

$^{721}$無論是誰,只要你認識我.
我屬你,上帝.
如果看德文,這首詩叫「Wer bin ich」.
「Wer」即是「Who」.
「Bin ich」即是「Who am I」.
但最後的回應是「Dein bin ich」.
這字跟「Wer bin ich」的字很相似.
不是「Wer」.
但「Dein bin ich」在語法上有點特別.
它不說我是誰,而是說我是你.
「你」在語法上是指屬於你的意思.
德文有個字叫「受格」.
「你」不是「You」.
而是屬於你的「You」.
所以就說「Dein bin ich」.
所以這字回應是「我屬於你」.
但最後的回應是「Wer bin ich」.
即是我誰,我就是甚麼.
我就是我所不知的誰.
但我屬於你.
所以這很大的回應是.
我們有時未必知道自己是誰.
但我知道我屬於誰.
所以在這個比較闊的亮光之下.
我們可以不斷尋找和實踐自己是誰.
但我們未必有一個很具體的.
這一刻我們是誰.
這就是我對這首詩的看法.
彭佛華用了「Dein bin ich」.
作為整個生物形態的總結.
即是我屬於上帝.
所以我們發覺有趣.
經文中有時會有這樣的經文.
它說「我現在活著的不再是我」.
似乎基督教中很少叫你不要做自己.
你不是自己,你是基督,耶穌等等.
但意思不是叫你不要做自己.
而是基督在你內活著.
所以經文中不是叫你不做自己.
也不是不實踐自己.

$^{761}$而是你需要在基督內好好實踐自己.
所以更清楚地辨認.
你們是歸入基督的,是披萊基督的.
不分猶太人,希臘人,自主的,遺奴的.
男的,女的.
這些都是認同.
你們有不同的身分.
無論是性別或是社會身分.
你可以有很多不同的身分.
但你們是屬乎基督的.
這就能夠確定.
你可以有很多不同的身分.
但總的來說,你是屬於基督,耶穌的.
這就是我們最重要的答案.
雖然不是身分.
這不是身分.
但是比身分更闊的範疇.
只要我們屬於基督.
我們就可以在這種狀態下.
不斷尋找更相似的身分和價值.
我們有些總結.
第一句話是「Dine bin ich」.
今天學得慢了,不好意思.
「Dine bin ich」.
意思是「我屬於你」.
「你」是「dein」.
「Dine bin ich」.
我們一起記得.
「Dine bin ich」,我是屬於你的.
這是一種禱告,我是屬於你的.
這是我們很重要的神學答案.
聖經沒有給我們唯一的身分認同.
有很多不同的生活狀態和狀況.
但只要我們屬於上帝.
我們可以有很多不同的空間.
尋找我們的身分認同.
所以我們人生中不需要有完美的和諧.
沒有一刻完全完美.
找到自己的目標.
這是我的理想和目標.

$^{801}$我們未必只是這樣的狀態.
我們多數是在尋找的狀態.
尋找,仍然有點迷惘.
我們會嘗試減輕迷惘.
但我們不可能完美地沒有迷惘.
這是我們很重要的想法.
沒有完全的正確性在過去.
所以我們不斷尋找,不斷拉近.
但我們在「Dine bin ich」的情況下.
尋找我們的身分.
第二,我們的教會群體.
這是我們很重要的幫助.
當我們在整個群體中.
我們就更容易地有身分認同.
流淌是一個甚麼樣的群體呢?.
我們經常說流淌是一個不被定型群體.
流動不被定型群體.
我們很不容易去定義自己的群體.
不過我們正正是一群人.
我們不斷地尋求我們對上帝的實踐.
我們就是一群這樣的人.
在這種情況下,我們反而會尋找自己.
我們就是一群這樣的人.
我們分明不是正常教會,大家都知道.
但我們是基督徒,我們屬於基督耶穌.
這是我們很重要的肯定.
所以大家做基督徒.
特別是今天做香港人也好.
或者面對著很多離開教會的朋友也好.
教會群體很重要.
我們不能過於自由.
不要太自由.
任何事都像基督徒一樣,沒所謂.
反正我們都可以不回教會.
或者做基督徒就算了.
不是這樣的.
我們仍然是在一個群體中做基督徒.
其中一個我覺得很重要的就是聖餐.
我們也說了一些聖餐的事.
聖餐是我們整個教會.

$^{841}$來定義我們自己的時刻.
我們是BYOB,即是Bel C.
我們帶了一杯回來.
這種狀態,行徑正正是定義我們整個教會.
我們有很多不同的人,不同類型的杯.
但我們回到教會,我們一起成為一體.
所以我們的聖餐是很好的體現我們的身分的機會.
聖餐更加是一個基督徒的參與.
所以我們覺得聖餐從來都不是受洗.
因為任何基督徒,只要他信耶穌.
他就是我們一起的一個群體.
所以我們就可以一起藉著聖餐.
去成為我們的教會群體.
所以Full Church可以說是有很多不同的形態.
可以有小組,有不同的MFC.
但聖餐這一刻令我們流唐.
成為一個很顯而易見的群體.
大家是多元,不同的故事.
但一起去領聖餐.
所以這是我們整個教會很重要的時刻.
那次我們不需要領洗禮.
因為我們知道聖餐是給所有基督徒的.
所以是一個傳統的習慣.
但對我來說,任何缺志基督徒都能領聖餐.
所以今天的總結就是.
上帝,我屬你.
無論我是任何人,還是未成為任何人.
我都是屬於耶穌基督的.
無論我是誰,我如何,我未能如何.
我就是你的兒女.
所以既然我們知道我們是屬於上帝的時候.
我們的身分可以浮動.
可以沉覓中,可以有些迷惘.
但我們知道無論如何.
我們仍然是屬於上帝自己的.
這就是我今天簡單的分享.
我們一起討穀笑,好嗎?.
接著我們有Q and A時間.
我們都知道我們是屬於你的.
我們崇賢迷失.

$^{881}$我們對於自己的身分.
對於我們身處的社會.
身處的狀態.
可能仍然有很多不確定的事.
主佑大愛求主你,來幫助我們.
面對我們不同的生活狀況.
無論是工作,教會群體或生活.
我們求主你幫我們在當中找到你.
當我們知道是屬於你的時候.
我們就可以做很多事情.
來沉覓你給我們的生活.
在當中讓我們看到我們應有的身分.
求主你幫助我們.
逢尊命求,阿門.
你檢查好身分上機了嗎?.
差不多完成了.
檢查身分很重要.
不適合身分上不了機.
大家都有登機證,對吧?.
可以吧?.
有沒有帶護照?.
只是檢查身分而已.
今天這個課題很重要.
身分,大家本身有身分危機嗎?.
開始有點嘴巴彎彎的.
應該有的,所以說時也會點頭.
是嗎?.
不是他所說的中年危機吧?.
大家應該不是這個想法.
青春期危機吧?.
容貌危機,大家覺得是.
這個課題大家思考過程中.
今天除了學到一句德文外.
還有沒有身分問題.
大家覺得有點卡嗎?.
你說就行了,後面大家會一起參與.
沒有身分問題,沒有身分危機.
有的,可能太深入的問題.
太深入? 太深入.
大家在香港,網上等於不會參與.

$^{921}$我說一個真實的個案.
以前的組員去了英國.
他已經過了半年.
當初他也不太急於找工作.
但去了英國後.
他覺得沒有工作.
他不太清楚自己在做甚麼.
聽到這裡,你不知道有甚麼想法.
但我在過程中.
他出發時,我跟他說.
「你放下香港的專業去到」.
「你也說不用做你的專業」.
「你說倒垃圾也可以」.
「或者找其他人做倉務也可以」.
「我說你不行的」.
「你覺得可以放下」.
「以前在香港的專業」.
「去到你也不太緊張薪金」.
「你隨便找工作也可以」.
「我說你不行的」.
然後他說「先試試吧」.
「我說你很難的」.
為何我看得通呢?.
其實是在工作上找不到自己.
你聽得懂嗎?.
不知道你在香港會否這樣.
你的工作能否找到自己.
或者你做基督徒的期間.
能否找到自己.
這也是我們要考慮的.
(記者: 請問教會是否有教會的弟兄姐妹?).
其實現在更深入了.
(記者: 請問教會是否有教會的弟兄姐妹?).
前面那位弟兄.
第二行.
(記者: 剛才沒有承接你那句).
不要緊.
(記者: 因為很多教會).
(強調自己在教會裡的身份認同).
(當有教會的弟兄姐妹離開時).

$^{961}$(他們就會說自己離開教會).
(那種感覺是).
(他們把普世教會和地方教會).
(很混亂地).
(給人一種感覺是).
(隔離教會的).
(都是表兄弟姐妹這件事).
(該如何處理呢?).
其實所謂的堂會.
或者是社群.
是一個真實的群體.
雖然我們說有普世無形教會.
但我們需要有一個.
具體,有形的群體.
所以身份應該是那個身份.
不是說我屬於普世教會.
這是對的.
但在現實來說.
我們需要有一個具體的群體.
多於一個「反正全世界都是教會」.
所以我會這樣看.
我們不難.
你想主宰的就是大教會.
但在我們的生活中.
我們需要有一個具體的群體.
一個實體教會.
或者一個堂會.
一個群體的堂會.
所以我從來不建議人說.
「這個普世教會就是我的教會」.
這是對的.
但我們需要有一個實體的群體.
來成為你的身份象徵.
和身份認同.
所以我覺得兩者都重要.
我可以問你一個問題嗎?.
我不是很明白表兄弟姐妹的意思.
問得好,但其實我也不明白.
當他說到你離開某一個堂會時.
你就是離開了教會群體基督的本身.

$^{1001}$有另一間教會的兄弟姐妹是不同的.
他們是他們,我們是我們.
我明白每個堂會有自己獨特的地方.
但當過分強調一件事時.
很容易產生疏離感.
「別人的教會是別人的教會」.
當然他會再補充一句.
「我們不要單顧自己的事」.
「Bilibili」,會有另一堆事.
但這很奇怪.
我覺得會有很強烈的排他性.
因為這樣推遠一點.
整件事都會很異端.
不要說推遠一點,放大一點.
其他宗派可能都用不下.
宗派中神學和聖餐的方式.
禮儀方式不同,所以有不同的宗派出現.
但如果用親屬關係來說.
即是跟一個爸爸.
但有不同兒子的面向的宗派表達.
如果這樣說,其他宗派的做法未必可以.
不心夠他的表兄弟的想法如何.
但我想這與身分無關.
我覺得這樣會太過專制.
令那件事好像只有我對.
其他人都不太對,這樣就不太可以.
回到身分方面,大家覺得有甚麼困難.
對於你們來說.
今天的課題是關於身分認同這件事.
(無聲).
還是覺得沒有甚麼大難處?.
那裡有點口,是否有問題?.
(有,在那裡) 在後面.
其實我覺得我不是有很大問題.
可能你不用特別回答.
但我純粹好奇的是….
因為在說身分的流動性.
特別在流動教會裡.
我覺得大家好像不太想定義.
例如我們是甚麼群體.

$^{1041}$我們也很模糊,是一條界線.
所以永遠在尋找的時候.
我覺得比較難去看.
我們在個人裡面.
究竟我們是否有一套….
或者有很多…因為太專注.
所以在議題上.
我也不知道如何去看自己與身分的關係.
特別是…我不提及具體的議題.
也不是特別問你們.
你們有甚麼看法?也不是.
我純粹有時候在想.
如果別人問我.
我作為「Folk Show」大型節目.
我是不能夠為他回答的.
因為始終如果在議題上.
我們有一個很光譜的看法.
未必是立場.
但有一種聖經的呈現.
我也不能替他回答.
不能替教會回答的時候.
我有時候會在想.
這種流動性,其實是否….
我也不太能拿捏.
這種永遠在流變的對議題看法.
或者對教會,宏觀的事情的看法.
應該是可以如何應對的.
所以我會用自己對社會的理解去回答.
當然不是代表「Flow Church」.
但有時候也很想看看.
有沒有方式可以從聖經角度.
了解更多「Flow」是怎樣的.
所以這是我們….
說得很準確地說出我對「Flow Church」的看法.
剛才說流淌人.
我自己也很害怕這樣說.
或者覺得不太有關.
其實也不是不行的.
但我不想像以前我們中學時期.
穿著一件寫生衣.

$^{1081}$然後有很多不同的卡,身分證.
我也想過,先做一張「Flow Church」證.
大家有一個分析分析.
掛在這裡,也挺有型的.
我們可以有很多這些東西.
但我都說這些不重要.
但重要的是甚麼?.
重要的是我們的信念.
所以我們「18課」就是這樣解釋.
「18課」就是我們希望….
雖然大家可以有很多不同款式.
和不同樣子的「Flow Church」人.
但我們也不需要有「Flow Church」家.
或「Flow Church」衣服.
或甚麼來定義我們自己.
但信念是一樣的.
所以我們在首兩季,下一季,三季.
對某些議題,我們的想法是一樣的.
我們是重視這些想法的.
這就是我們「Flow Church」的信念.
所以縱然我們未必有很硬件.
或物件上的身分認同共同點.
但想法,相信的態度是一樣的.
但我們覺得是不同的.
「Flow Church」的人的想法.
或我們的神學,信仰.
是和「出面教會」有一定的不同.
不是說是大公正的東西.
而是我們的質地是不同的.
所以「18課」想我們可以展示出來.
我們可以有不同的東西.
我們說靈修.
我們說我們怎樣看基督徒.
怎樣看跟隨耶穌.
怎樣看流散.
這就是我們「Flow Church」的頂尖姊妹.
獨特之處.
這就是我們的身分認同.
(陳健波)我用教會運作的做法回應.
「Flow Church」到了第四年.

$^{1121}$由石硤美年代開始參與崇拜的頂尖姊妹.
加入小組.
或是很穩定和我們相聚的頂尖姊妹.
我也被不同的人問過.
「Flow Church」沒有甚麼活動.
我說我們有很多活動.
他們說你們有活動.
但不是像部門般參與.
因為他們覺得.
除了疫情令我們很多聚會封鎖.
其實過去在教會.
如何有一個所謂的「Church Member」身分.
就是你屬於甚麼部.
你屬於栽培部,傳道部.
新朋友部,會友部.
那些部就是他的身分.
他覺得那裡認同了這間教會的參與.
但「Flow Church」不是用這個運作方式.
他一直不太容易擺明自己的位置.
所以他覺得.
正如你剛才提到的.
如何介紹或如何在「Flow Church」中.
表達一些東西.
不容易具體地帶出來.
所以這也是我們的運作狀況.
但我們不是要有部讓你掛單.
讓你覺得有安處.
沒有世界混亂的情況.
反而我們覺得有些題目.
是大家一起要合作,要帶出來的.
這正正是說甚麼是基督徒.
早期教會被稱為.
「他們就是基督徒」.
為何安提拉那班人會被稱為基督徒.
基督徒不是要承認自己是基督徒.
基督徒是要宣揚基督.
所以他被稱為基督徒.
這句說話很重要.
我們要承認自己是基督徒.
但你沒有宣揚基督.

$^{1161}$這是重要的.
因為那班人從早期一直宣揚基督.
所以他被稱為基督徒.
但我們沒有宣揚基督.
但你承認自己是基督徒.
我覺得你名不符實.
這就是你沒有自己的基督徒身分.
所以我們一個不被定型的群體.
去演化或延續出去.
其實我們在哪裡都能夠宣揚基督.
我們才能夠運作出自己的身分.
這就是流動性.
你明白我的意思.
所以這是很重要的.
不是我們要在教會裡運作出什麼.
去認識自己是基督徒.
去服侍教會的身分.
但教會從來不應該是向內的.
教會是向外的.
所以我們被猜出去宣揚基督.
我們就是把基督徒的身分向外展現.
這就是我們應該要有.
從信仰而有的身分.
這就是重點.
(記者:請問你會在教會做什麼?).
我會做一些教會的事.
例如教會的教會.
我會做一些教會的事.
例如教會的教會.
我會做一些教會的事.
例如教會的教會.
我會做一些教會的事.
例如教會的教會.
我會做一些教會的事.
例如教會的教會.
我會做一些教會的事.
例如教會的教會.
我會做一些教會的事.
例如教會的教會.
我會做一些教會的事.

$^{1201}$例如教會的教會.
其實很開心的.
轉眼間發現看到很多不同的東西.
有一次在一些工作活動當中.
會與不同的群體交際.
我遇到一些我以前工作的那類人.
看到他們在做自己的事.
那一刻我出現了一種.
「我在做什麼呢?」.
我很嚮往以前那種.
如果我還在那個地方.
我就是在做那些事的人.
那一刻是很傷心的.
真的有一個危機.
直到跟同事分享.
自己再想.
真的能夠知道.
原來那個移動.
那個工作崗位的轉變.
仍然我看得出是上帝帶我離開.
所以那個時刻.
不是因為我不是以前崗位的職位.
而是因為我仍然是上帝的兒子.
他帶我去這個崗位.
甚至是將來我是.
應該說我頗確定.
這個未必是我永久的崗位.
但無論我再轉換,再沉覓也好.
知道我們的根源是與上帝連繫的時刻.
那就是我真正的身分.
因為浮世間的職位.
定形不了我們的身分.
那個東西永遠都是短暫.
但最不會變的就是你是上帝的人.
那是最重要的,所以很能夠反映.
想分享一下自己經歷過這件事.
謝謝.
因為我經常提醒自己.
形式是一件事.
形式有很多不同的形式.

$^{1241}$但你的本質是甚麼是最重要的.
我以前經常說笑.
穿醫生袍的人不一定是醫生.
精神病人穿醫生袍也會扮醫生.
除了醫生袍,你仍然是醫生.
因為你的本身的色彩就是醫生的能力.
所以不是著重你的外觀.
或是你的所謂的「Title」.
其實你的能力是表現出你的功能.
以及你可以有身分.
即是表達出來的工作,這是很重要的.
所以仍然是那句話.
你「Proclaim Christ」.
你才是一個基督徒.
所以無論你探病也好,探監也好.
探訪也好,你探甚麼也好.
你帶著你的基督信仰去祝福他人.
就是有身分而有工作.
(陳零九)我也加一句.
在這個時代,越來越多這樣的場合.
正如你剛才所說.
你明明是醫生,但不是以醫生的身分做醫生.
對吧?現在穿警察服的.
未必是警察的那種.
特別是傳道人也是.
傳道人很多時候更不是傳道人.
方式就是傳道人.
所以很多時候,世界叫你.
未必用一個典型的消防員,警察,醫生的身分做人.
有時候很浮動,是否像slash一樣.
基本上是slash的.
沒有一個很強烈的身分.
但最重要的是你知道你在神面前.
你是一個怎樣的人,做甚麼事.
那些外加的狀態是一些很外表的東西.
也不是最重要的東西.
最重要的是你是一個怎樣的人,做甚麼事.
多於外面的人怎樣看你.
我們要習慣了這樣做人.
(陳零九)還有身分.

$^{1281}$裡面有一個分類是自我身分.
你自己是否意識到自己身分而有的能力.
以及表達方式是很重要的.
剛才說到像不像.
我自嘲說一件事.
就是我初出來做傳道人的時候.
被年輕人說.
「潘先生,你也不像傳道人」.
我說「我不像嗎?我哪一部分不像?」.
很緊張.
然後他想了想.
其實傳道人是不遮蓋頭的.
然後我心想.
遮蓋了頭就不是傳道人了.
我說甚麼不….
(陳零九)那些是用髮蠟的.
對,我說唐主任也是遮蓋了頭.
不過他用髮蠟,我用髮泥.
其實也是遮蓋了頭的.
然後他說「但你真的不像」.
我自己的口頭禪是.
「像」即不是.
我是一個傳道人,我不用像.
這件事讓他明白到.
其實有些人會帶著舊有或社會的觀念.
去看待身分.
但你如何展現自己的自我身分.
其實你一定要清楚.
不要因為社會的規則.
而主導了你的看法.
這是我們….
特別是我們在上一年的神學百科裡.
也想說的八個核心.
我們現在在這個環境當中.
做基督徒時.
我們如何展現八個特質也是重要的.
所以有一個這樣的模式.
穿著這件衣服,這樣返教會化.
就是基督徒.
但我們不是這樣.

$^{1321}$我們是說你們可以有不同的基督徒樣貌.
但只要你們是屬於上帝就可以.
所以羅唐就不是有這樣的版本.
你跟隨這個版本,就是基督徒.
剛才阿寬Sir說的類似.
前面有一個,剛才….
我想問,作為基督徒.
剛才說我們對自己的身分認定.
例如,如果用醫生的身分.
他有否通過實驗,是否真的醫生.
我們在心裡有否認清.
我們順序去上帝的時候的過程.
但我們做也是.
就是「To have to be」和「To do」.
我們「Be」了.
我覺得自己像剛才那位師一樣.
屬於神.
但我們也有「To do」.
以前教會有很多說法.
你們做基督徒要這樣做….
這些是你們要做的做法.
有些人會說.
可能你平日要拿著「Four Steps to Christ」.
到處跟別人說.
「你信神嗎?不信神就去地獄」.
這樣說的.
那些是其中一個「To do」.
你去宣教,其中一樣就是你的「To do」.
有些人會說.
「不是這樣的,你平日愛人」.
「誠實地做事」.
「這些都可以讓人看見」.
「你如何分開自己」.
「你是一個基督徒」.
「讓人看見你在神裡的特質」.
我經常想.
如果別人要看我的特質.
而他相信有神.
他們應該覺得沒有神的存在.
我覺得這也是很困難的.

$^{1361}$例如,問回身分.
如果對我個人來說.
我是否只是讀經.
我相信神是愛我的.
那我就是基督徒.
如果在Flo Church.
不太著重我們去侍奉.
那我們的「做」在哪裡?.
我不知道這是否很中國人的事.
或是很香港人的事.
我經常都要做.
才能證明我是那樣的人.
梁國雄:當然要是.
我經常說行動是重要的.
特別是新教.
新教是用行動來定義自己.
我經常說.
牧師不是因為牧師的身分.
才做牧師的事.
你是做牧師的事.
才被人確定你是牧師的身分.
基督徒也一樣.
基督徒.
如果想回到第一課的第一課.
基督徒的意思是甚麼?.
基督徒的意思是.
我們去傳揚基督的人.
多於是一個身分.
所以基督徒.
每一個聖經中的名詞.
本身都是一個動詞.
基督徒是傳揚基督的人.
門徒是跟隨耶穌的人.
頂姐妹或神二女.
就是一個天父,爸爸的人.
所以每一個身分.
都來自於一個行動.
只是行動不是那些教會的行動.
不是叫你去回收甚麼.
報甚麼,侍奉甚麼.

$^{1401}$而是你做基督徒的事.
有很多的.
我們會說敬拜.
中文碼頭會說敬拜.
我們會叫你去回一個實體的群體.
在社會上彰顯上帝的公義.
每個碼頭都會說的.
這類行動.
不是教會的活動.
一定是那些才定義為基督徒.
不是用教會的東西來定義自己.
而是做基督徒.
在社會中做基督徒.
這就是我們做的行動.
這也是我對於流唐.
為何沒有那麼多侍奉.
因為你將你的時間和工作.
可以放在外面的社會中.
不是叫你去學放假去玩.
而是去見證基督耶穌.
這就是我們為何少侍奉的原因.
不是叫你去玩.
而是叫你將時間去服侍外面的人.
(梁繼昌)用John剛才說的.
Erikson的身分理論.
每一個身分轉變的時候.
都有一個責任.
他能否做到的事情.
以至身分中的確立.
基督徒也有相似的情況.
有些時候是你有身分.
而沒有相關的能力去幫助.
舉個例子.
你有老師的身分.
但沒有盡老師的身分.
而去想一些你可以做到的事情.
例如我本身是老師.
我用我的方式來指.
讓我要教的東西.
是通達,清楚,能夠幫助別人.

$^{1441}$這就是我要為我的教學對象做好這部分.
這就是我要想的事情.
正如我們之前在Info Group.
或在不同訊息中也說過.
特別是我們《Voice for Church》的泥膠片中.
提到第三部分,社會參與.
我們基本上是散去在香港不同地方.
居住和工作的群體.
但你碰到的人,我碰不到.
但你熟悉的人,我不熟悉.
但你就想方式讓他們接觸到基督信仰.
這就是你的職業任務.
就是你要做的工作.
不一定是定讀經,祈禱.
或其他教會崗位的事情.
這一定要透過你去思考,去做.
這才是正確的.
正確的意思是能夠對應呼應需要.
這其實是保羅所說的.
我在哪裡就做什麼人.
我面對什麼群體就用什麼方式對待它.
無論是希利利人,其他人,法外人.
我想這就是我們需要以身份對應的工作表現.
(梁繼昌)網上的.
不過我剛才回應John所說的.
Flow Church,流塘人.
我也認同是很好的,要包圍它.
因為之前John按摩的時候也說過.
其實Flow Church沒有細菌相.
很多人聚集的.
因為Flow Church這個群體是很廣泛的.
有Zone A的就是實體現場崇拜加小組.
Zone B的就是武俠小組,但有現場崇拜.
Zone C的就是網絡上.
很多人偶爾也會參與Flow Church的網上活動.
所以要定義流塘人是一件不容易.
或者是不需要定義的事情.
不過對於我們來說.
就是如何可以一起參與.
在教會群體當中建立一個相聚的形態.

$^{1481}$我覺得這個身份對我們來說很重要.
(陳克勤)我們下課就會多說這一點.
我們下課的主題是「家」.
所以就說教會作為一個家的一種看法.
所以我們下課就叫做Homeless.
就是回家這個字.
所以我們都會探討.
無論是流散還是….
他們對教會有甚麼重要性.
我們作為流塘人.
如何看待所謂的教會「家」.
如何理解.
我覺得是一個挺好玩的主題.
教會和家的關係.
(陳克勤)明天我們會有戶外崇拜.
我覺得流塘人.
我們想延伸這個沒有界限.
去接觸更多外界的人.
我覺得這個延伸是我們可以做到的.
還有希望做得更加遠.
讓教會不只是四面牆.
或者是一個普通的聚會時段.
我們真的在不同社區.
或者是機會可以接觸更多.
可以認識教會群體的人.
好,最後有沒有其他問題?.
好,可以登機了,準備.
我們下次是這個月底,對嗎?.
對,這個月底是第四課.
OK,遲些見.
再見.
謝謝大家.
\newpage



\section{}
\label{sec:RYCxV16hfwM}
\textbf{《致餘民及流散者:給香港基督徒的神學八課》第二季第4課|20230528 [RYCxV16hfwM]}
\newline
\newline
連結: \href{https://youtube.com/watch?v=RYCxV16hfwM}{\texttt{ https://youtube.com/watch?v=RYCxV16hfwM}} ~~~~ 語音日期: 2023-05-28 
\newline
\newline
\hyperref[sec:OcD6qni0UQE]{\small{< < < PREV SERMON < < <}}
~
\hyperref[sec:index_chronic]{\small{[返順時目]}}
~
\hyperref[sec:index_scriptual]{\small{[返順卷目]}}
~
\hyperref[sec:40Zpw7rWZSQ]{\small{> > > NEXT SERMON > > >}}
\newline
\newline
$^{1}$香港人仍然是香港人.
究竟上帝的旨意如何?.
我們應該如何生活?.
我們應該如何為主而活?.
無論你身處何處,只要你是香港人,.
邀請你和我們一起思想這個流散年代的信仰..
致愚民與流散者,給向我基督徒的神學百科..
各位頂尖會員,大家好!.
歡迎大家來到我們神學百科第二季的第四課,.
Homeless 回家的人生..
我覺得這課是很好玩的..
首先,這課是很切合我們這個年代的需要..
一方面是我們很多頂尖會員移民到海外重新找教會..
教會成為了一個很重要的家..
這也是我以前在德國讀書的經驗..
我在德國柏林生活了六年..
德國柏林教會是我的家..
我的家常常被人迫使我睡覺..
很多頂尖會員來我家過夜,睡沙發..
我們會煮飯給他們吃..
每逢星期五晚上,我會煮十多碗飯給他們吃..
我懂得煮飯,煮飯給他們吃..
所以是一個很小的群體,三十多人的群體..
送機,接機,去家裡玩..
很多這些事情,特別在外國也是這樣..
中秋節,新年這些,在外國是找不到人跟你一起過的..
教會的頂尖會員就跟你一起過..
以前你在家裡,新年,拜年,中秋節,.
教會成為了一個很重要的功能..
所以我覺得海外的教會,海外的頂尖會員,.
他們在英國,加拿大,在不同的地方,.
他們重新來重拾他們的上群體..
家裡這個概念,很值得我們花一點時間,.
用聖經,用神學來思考..
這方面就算在香港也是,.
香港裡面,我們橫教會,.
我們很流行用家字來指稱,.
以前沒有會,你自己在教會叫什麼家,.
不知道是不是這樣稱呼..
大家會穿同一件T恤,.

$^{41}$然後回到家裡,這是家事報告,.
要有家書之類..
我們橫教會也很喜歡用家字來成為教會一個很重要的歸屬感的地方..
當然我們知道,Full Church不是很流行,.
我挺在意不去說這些東西..
我一向都不太強調Full Church人,.
Full Church家,還是大家回到這裡..
不過很有趣的是,Full Church的Banner,.
那個Elijah寫什麼呢?.
就是歡迎回家,.
但實際上又是突然叫大家回家..
所以今天我們會說一個很特別的題目,.
也跟我們Full Church很重要的DNA有關,.
我們怎麼看教會作為家這件事呢?.
事實上我們知道這幾年來,.
很多人對於教會家的觀念是有些批判的..
我在網上看過一些文章,.
很強烈地說不應該把教會當成家..
你知道以前的神學,牧師,.
用什麼聖經來背後他們的說法,.
為什麼說教會是家?.
其中一個很重要的就是,.
教會是神的家,.
這些是神的家事,.
諸如此類..
所以今天我們會花一點時間去看聖經,.
究竟我們如何理解神的家這個概念呢?.
有些人反對,覺得那個經文不是這樣解釋,.
諸如此類..
所以今天我們會去探討這個課題,.
很多人都說我在柏林,.
這裡是柏林的我,.
我覺得今天是一個很開心的日子,.
今天我仍然跟那些頂次妹非常親密,.
因為我們是共同生活了幾年的弟兄姊妹..
所以當時海外的教會往往都成為了一個同鄉會的功能,.
這個沒有什麼貶義的,.
當然是有一種功能的,.
海外裡,你在外國裡,.
一群香港人,一群華人聚在一起,.

$^{81}$這是一個很重要的概念..
所以今天如果你在看我們百貨裡面的海外頂次妹,.
我想你也很同意吧,.
我們的海外教會,.
我們Full Mission,.
我們Full Church這個群體,.
其實都是一個成為了你很有歸屬感的地方..
所以今天我們可以來思考這個課題,.
其實我們華人教會對於家的觀念似乎是很重要的,.
這個是關乎我們對於堂會忠誠的文化,.
往往,如果很負面去理解的話,.
教會用了家這個觀念來鎖住你,不讓你走,.
有些人這樣說過,.
你不要離家出走,.
你離開家裡是很嚴重的事情,.
他們覺得你不應該去其他地方去謀生,.
你應該回到自己家裡,.
所以一旦堂會稱自己為家,.
你應該對這個地方,對這個堂會忠誠,.
你要像一家人一樣,.
你要有責任,.
你要有一個承諾在家裡,.
你要給家用,.
你要幫忙去建設家,.
但似乎我們流唐並不是這樣,.
我一開始也說過,流唐似乎不是用這個方式去理解,.
但我們如何去理解我們四教會的教會,.
如何去看待家,.
究竟我們是贊成還是反對呢?.
我會去想一想,.
大家知道這首歌嗎?.
回家,Eternity Girls,.
我稱之為基督教第一隊女團,.
這首歌是大家很熟悉的詩歌,.
回家,.
我們往往會將我們的信仰和回家連上關係,.
我們試試研究歌詞,.
報佛是否感覺疲倦了,.
得得轉轉,沒了沒完,.
所以當我們信耶穌的時候,.

$^{121}$其實就是回家,.
其實有根據的,.
浪子比喻本身就是這樣的比喻,.
浪子回頭,回到家裡,.
所以我們不能否定,.
在我們的信仰裡,.
回家似乎是一個很重要的觀念,.
你回轉歸向上帝,.
正正是回到家裡,.
而當中我們信主時候的教會,.
我們就會自然地看為一個家,.
所以大家想想,.
以前的教會,.
或者所謂的母會,.
你稱它為母會,.
就是有媽媽的成分,.
所以你會發覺,.
我們信仰裡往往會很強調回家,.
都是一個家的字,.
當然我們華人教會特別,.
華人文化裡面,.
特別重視家,.
這也是我們根深蒂固的一件事情,.
我們對於家庭的觀念,.
似乎比西方文化更加重要,.
所以我們就去理解,.
究竟我們教會是不是神的家呢?.
我們怎樣去理解神的家呢?.
如果你看回經文的時候,.
我們這個神的家,.
是叫House of God,.
House of God,.
Echo Theou,.
就是一個上帝的家的意思,.
是什麼經文呢?.
最基本來說,.
我們知道提摩太前書第三章十五節,.
教會是永生神的家,.
這個大家都熟悉的經文,.
保羅這樣寫,.

$^{161}$倘若我擔言日久,.
你也可知道在神的家當中怎樣行,.
這家就是永生上帝的教會,.
真理的基石和注釋和根基,.
所以我們很多時候,.
我們環教會就用這段經文來強調,.
教會就是神的家,.
實際上,.
如果我們真的去明白這段經文,.
或者保羅很多連串裡面,.
很多這些神的家的經文,.
你會發現,.
Echo那一字,.
其實就是解作家,.
不過你會發現,.
這字可以有兩個不同的解釋,.
一個是解作House,.
就是解作Home的意思,.
一個是概念上的家,.
可以解作Building,.
一個是House的意思,.
所以這字其實有兩個不同的意思,.
一方面是Home,.
就是House的意思,.
一個是概念上的家,.
一個是建築物上的屋,.
如果是這樣的話,.
經文裡面有兩個可以不同的解法,.
教會是永生神的屋,.
還是神的家呢?.
這樣就可以有兩個很明顯的意思,.
如果是一個上帝的屋,.
是一個殿的概念,.
一個Temple of God,.
一個House of God,.
如果是Home的話,.
如果是一個家庭的意思的話,.
是另一個解法,.
所以我們可以發覺,.
可以有很多不同的思考,.

$^{201}$但如果我們看回經文裡面,.
看回提摩太前書第三章的話,.
看回上下文的時候,.
第三章裡面似乎有些暗示,.
譬如第四到五節這樣說,.
好好管理自己的家,.
這個家字也是個House,.
可以解作屋,.
或者是個Home,.
好好管理這個House,.
使兒女凡事端莊順服,.
人若不知道管理自己的家,.
焉能照顧上帝的教會呢?.
似乎這段經文裡面暗示了什麼呢?.
似乎是有關子女,兒女,家庭的事務,.
這樣說的話,.
這段經文裡面似乎是說,.
教會真的是一個Family,.
一個家的觀念,.
有兒女,有父親管理之類的,.
這樣說的話,.
似乎是神的家,.
神的Home,.
是這樣的意思,.
不過,.
再看15節之後的話,.
你會發現,.
那個字眼裡面說到有Pillar,.
有個基石和根基的話,.
又似乎是一個建築的概念,.
因為說有根基,.
有基石的話,.
這個家是說Building多一點,.
所以其實是不容易去肯定,.
經文裡面是說一個Family,.
Home的概念,.
還是說一個Building的概念,.
這是第一個,.
提摩太前書裡面所說的,.
似乎是可以有兩種不同的解法,.

$^{241}$可以是一種Family,.
Home的概念,.
家的感覺,.
也可以解作一個純粹的Building的意思,.
所以究竟教會是神的Building,.
還是純粹一個Family呢?.
我們還是未知道,.
我們再看另一段經文,.
就是希伯來書第三章和第十章的經文,.
有一次這樣出現,.
希伯來書裡面曾經有這樣的理論去強調,.
他比摩西算是更配多的榮耀,.
好像建造房屋的比房屋更加的尊榮,.
因為房屋都是必有人建造的,.
但建造的都是上帝,.
摩西為僕人,.
在上帝的全家承言尊重,.
為要證明將來必傳說的事,.
但基督為兒子,.
治理神的國家,.
若將可誇的盼望和膽量堅持到底,.
便是他的家了..
這個價值仍然可以兩不解法,.
Home Family又可以,.
Building又可以,.
所以究竟經文裡面是說一個Family,.
Home,.
還是Building呢?.
似乎我們都不容易理解,.
這裡說的是一個建造房,.
似乎是Building的比喻,.
不過又說到一個僕人,.
在一個全家進宗,.
又好像是一種Family的意思..
我們看回另一段經文,.
帖本流出第十章,.
又有大祭司治理神的家,.
被我們心中天良的虧欠已經灑去,.
身體用清水洗淨,.
就當存誠心和忠心的信心,.

$^{281}$來到神面前..
似乎在這裡,.
治理神的家裡面,.
我們都不太能夠肯定,.
究竟是Building還是House的意思..
第三段,.
就是前書第四章的經文,.
因為時候到了,.
審判要從神的家起手,.
又是我們不容易理解的..
這個經文似乎所說的是什麼意思呢?.
神的家起手審判的時候,.
起碼我們知道,.
這個審判所說的神的家是什麼意思,.
似乎未必,.
就算我們說教會也好,.
未必是說堂會的意思..
他所說的神的家,.
不是純粹說某一間的堂會,.
而是整體上帝的子民,.
審判是由整體的上帝子民開始說起..
所以我們可以知道,.
就算經文所說,.
可以說是教會作為神的家也好,.
是Building也好,.
他最少所說的不是一個堂會,.
是一個家的意思..
他所說的是一種整體基督徒群體宏觀的教會,.
是一個神的家的意思..
你不能不否定,.
我們稱之為弟兄姊妹,.
稱之為天父爸爸做爸爸,.
是一個很Family的概念..
聖經裡面說到弟兄們,.
Brother and Sister,.
天父上帝,.
耶穌是我們的兄長,.
我們是弟兄姊妹,.
Children of God,.
我們是上帝的兒女,.

$^{321}$因為我們是上帝的兒女,.
所以我們可以說是一個家的意思..
我們可以說是一個家的意思..
教會是整體教會的縮影..
當我們說我們是一個普世教會的時候,.
我們不能忽視我們實際是一個見得到的教會,.
一個很具體的群體成為我們對教會的投射..
所以沒錯,我們整體來說是宏觀的一間教會,.
但我們是有一個堂會概念,.
起碼有一個群體概念..
這個群體是你實踐教會理想的地方..
任何教會的實踐理想都應該放在這個具體的群體裡面來實踐..
所以將堂會理解為一個家,其實並不是完全錯誤..
但我們要如何理解這個家的概念?.
是不是那種要你完全忠誠,.
只是這樣的概念呢?我覺得並不是..
所以我們想說,我們作為全聖教群體,.
我們對於家的觀念並不是反對,.
但我們要重新思考究竟我們應該如何理解作為神的家的意思..
事實上,我們回看中國華人的教會,.
為什麼我們會強調這個家字呢?.
你會發現其實有很多宣教的原因在這裡..
十九世紀裡面出現了一些稱為基督教的天家小說..
事實上,因為中國人很強調家,.
一個是孔子的文化,一個是儒家的思想..
所以當西教士來到中國傳播的時候,.
他們是用了這種重視家庭的觀念來傳教..
所以他們是用了很多西方的小說來作為傳教的第一步..
你會發現,這些可能大家不知道懂不懂,.
我們華人很強調稱天堂為天家,.
雖然西人也有一個叫做Heavenly Home的字,.
但他們沒有我們強調,我們很少叫天堂,.
我們會叫天家多一點..
西人也會說Heaven,.
但比較少用Heavenly Home這個字..
所以我們對於死去的地方,我們稱之為翻天家..
我們也很強調,我們用的都是家字..
所以我們的信仰似乎一開始就把家庭的觀念吃進去了..
無論是天羅律程,無論是後面所說的忍家,當道,或安樂家,.
這些是西教士來到傳播的時候,他們翻譯的小說..

$^{361}$這些小說成為了很好的報導工具,.
他們把西方基督教的文學翻譯成中文..
這些小說全部都說一件事,.
就是原來信仰是一個回家的過程..
大家有沒有看過天羅律程?.
可能很舊了,如果你是比較資深的就看過..
我們的信仰是一個天路,回天家的天路,.
一個朝聖之旅..
所以他理解為我們在塵世裡慢慢回到天家的朝聖之旅..
所以天羅律程是一個這樣的概念,.
我們在地上生活是一種朝向朝聖之旅,.
我們在塵世裡是一個寄居的生活,.
我們最後會回到天家裡..
這就是天羅律程最重要的故事框架..
另外一本就叫做忍家當道,.
他也很強調家庭背景的故事..
故事說什麼都不重要,.
但最後還是要回到家庭觀念作為一個結束..
他覺得家庭是一個很重要的家庭,.
家庭裡大家如何實踐信仰就是一個這樣的故事..
最後一本叫做安樂家的書,.
叫做Home Sweet Home,.
也是用一個這樣的比喻,.
他將家比喻為天堂,.
將天堂比喻為家,.
用了哥羅西書的經文,.
天堂就是天家的基業,.
我們將來有一個榮耀的城,.
一個安樂家,.
就是我們所學求的地方..
所以這些在晚清時的翻譯,.
其實是慢慢地進入我們華人教會對信仰的理解,.
我們信仰裡似乎有一個很根深蒂固的,.
將家字放在信仰裡,.
我們知道天堂是我們的天家,.
我們在信仰裡是一個塵世隔離的人生..
所以我們來到這裡理解的時候,.
確實是有一些經文的意思在裡面,.
譬如用第十四章裡面,.
耶穌說什麼呢?.

$^{401}$「在我父的家裡有很多的住處,.
若是沒有,我早已告訴你們,.
我去元兆位你們預備地方.」.
所以耶穌也這樣說,.
祂強調將來是一個父的家,.
耶穌為我們預備的就是父的家,.
所以你知道你的家就是將來一個天家裡面的生活,.
我們正正是一個邁向回天家的旅程..
所以我們這班人,.
我稱之為我們那班流散的人,.
我們是一班被上帝揀選的人,.
我們漂泊在塵世裡面的人..
家這個字,.
其實也是一個很值得我們重視的觀念,.
不過那個家不是真正的家,.
不是我們的堂妹,.
不是我們以前回過的家,.
而真正的家是什麼?.
就是我們流散的旅途裡面,.
最終極的天家..
我們知道我們人生裡面是一個邁向回家的旅程..
所以我們這種神學,.
我稱之為客女神學,.
我們很強調,.
在塵世裡面,.
世界裡面,.
只不過是一個客女,.
我們在地上沒有一個真正的家,.
我們是一個邁向天家的過程..
這個神學其實我們不是陌生人,.
我們在傳統環境裡面也很強調,.
這種客女人生,.
大家不知道會不會唱這首歌,.
大家都知道這本是什麼,.
是青年聖歌,.
我們很傳統的一些聖詩..
這首歌叫做什麼?.
叫做《這世界非我家》,.
This world is not my home,.
這首歌本身是一首20世紀初的時候,.

$^{441}$一首英文的美國的詩歌,.
This world is not my home,.
這個世界不是我家..
但是我們在幾十年前的環教會來說,.
我們就很強調,.
這種世界不是我家,.
甚至我們對世界有一種反感,.
我覺得世界是一些貪愛世界,.
世界是和上帝敵黨的地方,.
所以我們不要碰太多這個世界,.
因為我們家就在我們的天堂裡面..
我們看回我們的中文翻譯詩歌,.
這世界非我家,.
我停留如客女,.
我積財寶在天,.
時刻仰望我主,.
天門為我大開,.
天使呼召迎也,.
故我不再貪愛,.
這世界為我家..
可能大家會唱這首歌..
如果我們看回這首歌的英文原文,.
你會發覺有一個很特別的地方,.
原文裡面是說.
I can't feel at home in this world anymore,.
就是說我對於這個世界再感覺不到at home的感覺,.
這是原文的歌詞,.
I can't feel at home in this world anymore,.
不過中文的翻譯似乎比英文更加強烈,.
不單覺得這個世界沒有家的感覺,.
更加是覺得不再貪愛這個世界,.
不再覺得這個世界是我的家..
有些差異在這裡,.
我不覺得這個世界是我的家,.
不代表我是反對這個世界..
所以發覺中文在我們橫教會裡面,.
對於這種客女神學,.
我們加強了一種對世界反感的.
一些很古舊的思想,.
以前的基督徒都是這樣,.

$^{481}$不要看電影,.
不要唱卡拉OK,.
這些世界的東西全部都是不好的,.
這就是客女神學..
我們在家裡,世界裡,只是客女,.
所以不要關心社會,.
不要理會這個世界,.
因為我們是客女..
我們全家人不是這樣,.
所以我們重構我們全家人,.
怎樣去理解我們怎樣看客女神學..
當我們看回這首詩歌的後來翻譯,.
後來有很多不同的人都嘗試重新翻譯這首歌,.
譬如你看到,.
不再寫這句話,.
叫我不再貪愛世界,.
即是世界為我駕馭,.
後來的翻譯者都嘗試重新翻譯這首歌詞,.
你會發覺這首歌詞全部都避免了.
這麼反對世界的神學,.
即是神是愛伴我,陪伴我,.
走擇路天交往,.
明白世界已經不再是我心所盼,.
容我滿有望盼,.
不感到孤單,.
全部都不敢將客女神學翻譯出來,.
原本是說,.
我再也不想再感到孤單,.
但這群人都不想保留對世界不是我家的感覺,.
所以你會發覺很有趣,.
我們很強調,.
我們都知道客女神學是一個很重要的,.
不能否定神學,.
我們是天家,.
耶穌是我們預備的地方,.
我們真正的家是天家,.
這個世界是流散了的世界,.
我們是一個漂泊的人生,.
我們對於這個世界是一個暫時的過程,.
但我們全聖教的客神學是甚麼呢?.

$^{521}$沒錯,我們強調是客女,.
我們是流動群體,流散群體,.
但我們不是對這個世界有任何反感,.
我們很重要,.
其中一個神學家叫做Johannes Baptiste Metz,.
他是世界的神學,.
即是說原來世界對於上帝來說並不是一件負面的事情,.
我經常說,.
用第三章十六節,.
上帝愛甚麼?.
愛世界,.
原文上帝愛世界,.
心得獨生子,.
且恃有貪玩,.
所以上帝所愛的不是世人,.
而是整個世界,.
當耶穌降世的時候,.
當祂來到世界的時候,.
祂就將世界包在祂身上,.
所以作為基督徒,.
我們再也不覺得這個世界是一種反感的感覺,.
因為天父擁抱這個世界,.
我們在客女人生中,.
我們仍然會擁抱這個世界,.
我們不覺得像以前的橫教會一樣,.
客女等同於對抗這個世界,.
不參與這個世界,.
我們是一個流散群體,.
但我們是擁抱這個世界,.
我們流散是為了在世界裡作我們的見證,.
所以我想說甚麼呢?.
今天我稱之為,.
說了大堆好像不太容易消化的事情,.
我想說的是我們對於回家這個字的看法,.
我們作為一個班基督徒,.
流淌的頂尖妹,.
我們怎樣看流淌,.
我們怎樣理解我們在地上的流散過程,.
所以今天我們會說的字叫做空明,.
空明這個字其實在一年前左右,.

$^{561}$我去討論神學的時候,.
我發覺這個字是很有意思的字,.
因為我們在今天香港裡面,.
或者你在外國裡面,.
其實我們是一個空明的過程,.
甚麼是空明?.
正正是一個漫長的回家的過程,.
就好像那些季候鳥一樣,.
我們是一個很長途的回家的過程,.
所以我們是一個朝向著一個家的旅程,.
我們知道地上沒有真正的家,.
我們不想將任何一個的堂會,.
看為我們終極的奮鬥目標,.
聽說很多這樣的故事,.
特別是台灣教會,.
台灣教會其實很強調,.
你是不可以轉教會的,.
大家記得嗎?.
之前有一個台灣的報紙才說這件事,.
在批評一些轉教會的人,.
發覺很多時候我們環教會都是這樣,.
我覺得這個堂會就是你的家,.
所以你不要離開,.
你的終身事業,.
你的侍奉,.
你的遺失都在堂會裡面,.
這個堂會的家就是你最終極的目標,.
Full Church並不是這樣,.
我們知道我們不想大家.
只是為一個堂會去賣命,.
但我們也強調,.
我們其實不是反對家這件事,.
我們覺得我們是一個回家的旅程,.
我們是一個朝向天家的過程,.
但在我們這個流散的過程裡面,.
家這個概念仍然是重要的,.
因為我們仍然有一個落腳的地方,.
我們仍然有一段很深厚的關係,.
我們仍然有弟兄姊妹,.
我們仍然有一個大家彼此好像家庭的關係,.

$^{601}$所以我們Full Church並不是反對家庭,.
我們很強調,.
我們是一班彼此相愛的弟兄姊妹,.
我們只不過不想大家.
單單為一個堂會獻上所有,.
這個就是你的終身事業的目標,.
所以我們是一個回家的過程,.
這個home字是一個動詞,.
這個home字不是一個名詞,.
因為我們強調這個home不是地上某一個地方,.
而是天上的地方,.
但我們home名是很重要的,.
我們每個星期寫著歡迎大家回家,.
是一個真心的說法,.
希望你能夠在回來崇拜裡面,.
真的能夠有一個關係,.
這個家不是你賣命的堂會,.
但是一個真實的關係,.
希望大家能夠回來教會裡面,.
回來小組裡面,.
都能夠有一個回到家的感覺,.
這個是你一個季後鳥裡面,.
一個短暫中途停下的地方,.
能夠不斷來到樂極的地方,.
我自己很喜歡武明裡面的一個角色叫斯力奇,.
斯力奇是一個很有趣的故事,.
他是偶爾會回來的,.
他會飾演一個旅行者的角色,.
吹著一支笛子,.
突然很有型地來到,.
他就像我一樣,.
不太會說話,.
然後回來就站在那裡,.
每次他回到武明谷都是一個回家的過程,.
雖然他一年回來一次,.
不是叫大家一年回來一次,.
但起碼每次回來都有一個真正的家的感覺,.
雖然是一個漂泊的旅程,.
但他每次回到武明谷都是和一班人很好的關係,.
大家一起很有詩意地看月亮,.

$^{641}$然後一起聊天,.
然後第二天他就走了,.
所以這種感覺,.
這種圖畫,.
就是我們所說的,.
我們在一個空明的過程,.
天家是我們真正的家,.
但每個星期你們回來教會裡面,.
戰戰兢兢,.
就像斯力奇回來一樣,.
大家聚在一起,.
那份關係是重要的,.
你不需要將堂會成為你終日毀身,.
不要走,.
鎖死的地方,.
但每次你回來,.
每個星期你回來,.
都是一個這樣的概念,.
所以我們不可以忽略家這個字,.
很有趣的,.
我們有天國這個字,.
我們會將天國稱為天家,.
但我們中國人更加將國,.
country,.
或是state,.
加上了家字,.
明明就國就國,.
為何叫國家呢?.
所以我們發覺,.
我們真的很難抽起那個家字,.
雖然我知道,.
今天我們呼出群體,.
家有很多必然的,.
家是一個回來喝湯的地方,.
未必有一個完整的家庭,.
或是你的家人生活,.
未必是一般的,.
但重點不是模樣或模式,.
而是真正的關係,.
我們希望能夠和弟兄姊妹,.

$^{681}$和你目者,.
有一份真正的家庭關係,.
這是我們Folk Church很想大家擁有的地方,.
甚至乎你是需要去承諾下去,.
我想說,.
我們不是像那些堂會一樣,.
將Folk Church稱為Folk Church家,.
然後你就是這樣貢獻自己進去Folk Church的發展,.
但又相反地說,.
當我們知道Folk Church是一個地上落腳的家,.
我們都可以去毀身貢獻這個家,.
只不過不是那種那麼病態地貢獻,.
所以如果你是海外的弟兄姊妹,.
我們有一首歌叫《默念》,.
是關於一個在外國生活的瑣碎事,.
寫一封家書給他的家人,.
所以這種關係是我們流堂很想擁有的事情,.
無論你在海外還是在香港,.
我們需要在海外建立一份這樣的家庭關係,.
真的像一個家庭一樣的關係..
所以家作為一個動詞,.
教會家,我們是家,.
只不過我們不會忘卻我們流散的身份和使命,.
我們不會將教會家,.
我們唯一的目標,.
而是我們知道我們在世界裡,.
要作見證,.
我們有我們的身份在世界裡,.
所以我覺得,.
今天我和Poon Sir談了很久,.
我們想大家做些什麼呢?.
我想大家其實都是一樣,.
我們Full Church是不是一個家呢?.
我會說Full Church是,.
也可以說不是,.
它是一個真正的家,.
因為你是一個有關係的地方,.
它是一個你和弟兄姊妹有關係的地方,.
它都要求你彼此相愛,.
去為生的地方,.

$^{721}$但我們Full Church不是將這個家,.
視為一個你信仰唯一的目標,.
它是一個像史力奇一樣,.
每次回來都會去貢獻,.
你都會去奉獻,.
你都會去得著力量的地方,.
但不是唯一的地方,.
所以我覺得很想大家,.
將來在組裡面,.
來談談的事情,.
不妨去談談,.
你在Full Church裡面,.
你回到這個地步,.
你覺得你自己是不是,.
覺得在Full Church成為你的家呢?.
我們不會想你去,.
鎖在家裡不讓你走,.
但你是不是一個,.
其中一個的家園,.
來發揮一個弟兄姊妹的身份,.
來為其他人,.
來貢獻自己呢?.
所以我們說,.
大家都很,.
今日這個課題我覺得很重要,.
因為我們Full Church經常都是,.
說真一句,.
如果我們要大家去,.
賣命的話,.
我過兩天都會說的,.
大家都應該很想,.
commit(承諾)下去,.
但我經常都不說的,.
經常都不說,.
大家要怎樣侍奉,.
大家要怎樣做,.
我不說,.
但不代表這個流堂,.
不是我們地上,.
一個這樣的群體,.

$^{761}$我們是需要,.
來思考,.
我們怎樣來將你的生命,.
放一些時間在這裡,.
去怎樣在你小組裡面,.
彼此去建立家呢?.
起碼你的小組是你的家,.
因為這是一個真正的教會,.
真正的群體就在這裡,.
所以想想你怎樣回小組,.
你怎樣來在這個群體裡面,.
成為一個很重要的一員,.
來貢獻自己,.
今日有些不容易結束的地方,.
但希望大家可以在現場節目裡面,.
可以多些來談談,.
你怎樣看流堂作為你的家,.
怎樣看你以前的教會,.
怎樣看我們流散的使命,.
怎樣看我們空明這個概念呢?.
潘Sir應該差不多回來了,.
我先叫他過來,.
他剛剛下機,.
何時吃你煮菜的技巧?.
我煮菜?哦,我好像沒有吃過,.
我以前是煮甚麼,.
我忘記了煮甚麼,.
你煮番茄,.
番茄蛋,.
今日講家這個觀念,.
應該大家都有不少,.
應該可以把空間給多些弟兄姊妹分享,.
因為當時講流堂的弟兄姊妹,.
流堂不是第一個你回教會的,.
所以大家應該有相關的經驗,.
大家可以說一下,.
他收到麥克風了,.
當中聽的內容有沒有甚麼內容,.
覺得跟你以前接收的很不同?.
網上有沒有人問呢?.

$^{801}$有,好像有,.
如果教會是一個家的概念,.
大家每次都是回家,.
但沒有一個固定的教會,.
是否代表每個人都可以多一個家呢?.
這個很好,.
我們可以談談這個問題,.
我們可以有幾個住家,.
他說現在作為家中的小三,.
又應該如何處理這個家的衛生問題,.
躲在衣櫃裡面,.
有人回來,快躲起來,.
我自己在infogroup的時候,.
都說過弟兄姊妹如何回教會,.
其中有一個真實個案,.
不是回答過一次,.
是兩個不同的人都問相關的問題,.
因為他說,.
潘Sir,我一個月回教會兩次,.
我說這麼有趣,.
如果我星期日不需要服事,.
星期六我就會回來flowchurch,.
因為星期日他要服事,.
星期六他要rehearsal,.
所以他就要回教會預習,.
如果他不用服事,.
星期六他就會回來,.
所以一個月他有兩次回來flowchurch,.
有兩次就不回來,.
接著他就問,.
潘Sir,這樣是不是吃兩家茶禮,.
不是這麼好,.
flowchurch是如何看待的?.
我說flowchurch沒有問題,.
你不覺得這樣好像有兩個家不是這麼好嗎?.
接著我說我不覺得,.
我說如果你覺得flowchurch的聚會能夠charge up你,.
或者你能夠在當中敬拜,.
你是能夠在當中得著的話,.
你就繼續可以服事你另外的群體,.

$^{841}$which is我們是服事不到的,.
因為flowchurch的教會觀念是one church的觀念,.
你服事的那群是我們服事不到,.
而你在flowchurch可以得著,.
可以繼續服事其他人的時候,.
這個彼此協作,.
豈不是一個不要單顧自己的事,.
也要顧別人的事的教會觀念做法嗎?.
接著他就明白了,.
他說這樣也不錯,.
但你也要問一問你服事那群堂會的目者,.
如何看待這個做法,.
因為我們是沒有所謂的,.
但有些觀念大家未必是協同的話,.
大家一起去了解會好一點,.
所以好像剛才網上問的問題,.
是不是多一個家呢?.
我們是OK的,.
對我們來說沒有衝突的..
我經常這樣說,.
這個home不是一個固定的概念,.
不如我們把它變成一個動詞,.
不如我們叫做homeing,.
為什麼呢?.
因為我覺得這是流堂一個很特別的地方,.
我們不會叫你不要再回你自己的舞會,.
或者怎樣脫手,.
但就是回到這裡,.
你侍奉在這裡,.
全部都在這裡,.
我們不會像以前的教會一樣,.
這個就是你的家,.
所以你不要離開,.
只把你的信仰實踐就算了..
我們很強調,.
家是一種流動的概念,.
我們現代人都是這樣,.
我們現在還有一個現代人,.
年輕人,都市人,.
我們的家是怎樣呢?.

$^{881}$很多時候都不是這麼簡單,.
早餐吃一個有雞蛋,香腸,很經典的早餐,.
然後上學,上班,回來,爸爸媽媽都在,.
一個這麼經典的家,.
但家是一個我們能夠信任的地方,.
一個可以承載自己的地方,.
所以這件事我覺得是有的,.
應該有的,.
流堂這個地方,.
我們的小組,.
特別強調這個小組,.
就是希望能夠建立到一種關係的地方,.
它可以是很多不同的外揚,.
即是形式,.
但當中的東西是一樣的,.
有一次我喜歡一首歌,.
不知道大家有沒有聽過,.
就是陳以貞的家這首歌,.
如果你聽過的話,.
大家回去搜尋一下,.
有一首歌叫家,.
歌詞寫得很好,.
輕輕牽著你的手,.
慢慢長路一直走,.
哪裡都是我們的家,.
我覺得只要牽著那個人的手,.
我們一起走,.
哪裡都是我們的家,.
所以我想這件事就是這樣,.
我們流堂的那個家,.
不是一個很簡單的堂會機構,.
你不斷去發大財來搞,.
而是這裡的人,.
可以有很多不同的關係,.
可能在網上,.
可能是不同的身份,.
重點不是有多少個家,.
不是有多少個住家,.
而是你在這裡能夠有一個.
很重要的家庭關係,.

$^{921}$有真正家庭的人,.
和你一起走這個地方,.
你可能好像史力奇一樣,.
偶爾你就離開這個地方一會兒,.
但這種關係是重要的,.
當然更加理想,.
可以去承諾這份關係下去,.
不只是拿走,.
所以這就是我們嘗試去強調,.
我們不想鎖死你,.
但這也是你要去承諾的地方,.
因為這正是一個真實的彼此相愛的群體,.
一個家的關係,.
所以我想大概是這樣,.
我們重點不是有多少個住家,.
而是當你來到這裡的時候,.
你不要只是拿東西走,.
而是真實關係的,.
一個彼此建立的身份..
有沒有問題?.
在後面..
剛才聽了也很好,.
兩位牧者都很大方,.
覺得全聖教的人可以很自由地,.
不需要完全全聖教的侍奉,.
或者必須要留下來,.
但全聖教也有家務,.
回看其他堂會,.
他們也會困著大家,.
我們一起來聚會,.
星期三我們就祈禱會,.
星期四就崇拜,.
練師,.
有很多東西玩,.
很多東西搞,.
你們又會怎樣看這些衡量?.
我們想你自動自覺,.
家務就是這樣,.
大家都知道小時候爸爸媽媽都叫做家務,.
要麼就哭哭啥,.

$^{961}$要麼就用很多不同的方法,.
威逼理由,.
但其實一個家庭就應該自動自覺,.
大家很舊的電影,.
以前那套電影,.
陀奧斯汀下海,.
陀奧斯汀在劇集,.
大家雖然不是這樣的一家人,.
但大家的自動自覺,.
去埋伏,.
去貢獻,.
就構成了一個家,.
我要去勒索你,.
你回來一定要怎樣做,.
不然你就不是家人,.
是沒有意思的,.
只是很多破碎的家庭就是這樣,.
但真正的家就是不用出聲,.
潘Sir也說我們預備課程,.
很多時候很多掙扎,.
你問就對了,.
我們不會主動叫你做家務,.
但家務當然要做,.
我們想你自動自覺做,.
不是想你不做就不能回來,.
不做就不是家人,.
最美麗的家庭就是大家自動自覺回來,.
不成功就埋伏,.
就抹掉,.
這才是一個理想的家,.
因為我們成長的家就是這樣,.
是一個底子,.
主動扶持的地方,.
我和弟兄姊妹回應,.
提問這些過程當中,.
其實家這個觀念在過去的堂會是很好用的,.
譬如家用,家規,.
大家有功能性去做到那件事,.
是好用的,.
我剛才說的轉堂會,.

$^{1001}$剛才John說的台灣或香港都很不容易,.
通常你轉教會離不開那兩至三個原因,.
第一是你嫁了或娶了,.
跟了人去,.
你帶著祝福他,.
然後你去讀神學,.
他都期望你會回來,.
然後你搬了轉教會,.
香港有多大?.
只坐一個小時車,.
不是很遠,.
你離開了,.
沒有做家務,.
家務原本你做,.
現在要找人去頂,.
這些地方是不容易協調的,.
或者會令到像現在這樣說的話題,.
會有些爭端,.
但我們都希望讓弟兄姊妹說,.
疫情下其實正正就是顯了很多東西,.
這三年香港教會在疫情下,.
其實沒有了很多聚會,.
很多功能侍奉組別都停工了,.
教會會不會死呢?.
其實教會真的不會死,.
教會只要維持崇拜,.
有一個牧養的過程當中,.
大家每個星期在崇拜當中有一個synergy,.
其實教會是繼續運作到,.
但突顯了一件事,.
就是你沒有了那些服侍,.
或者沒有了那些崗位之後,.
你發覺可能跟教會沒有什麼network,.
就是沒有什麼聯繫,.
突顯了一件事,.
教會過去可能著重很多functional approach,.
或者functional domain的東西,.
而忽略了relational domain,.
關係,聯繫那件事,.
我不知道你過去跟弟兄姊妹,.

$^{1041}$或者牧者之間的關係是怎樣,.
很多時候你看回check history,.
很多都是功能性,.
交代你什麼時候回來,什麼時候合作,.
其實靠近的東西不是很多,.
交心的東西也不是很多,.
但疫情突顯了,.
我們再重新去想一下,.
其實教會take out了這些功能性的東西,.
其實剩下什麼呢?.
空名最重要,帶出一個訊息,.
就是那種關係最純粹的,.
所以fortune就真的,.
fortune是不會沒有事情做的,.
教會也沒有事情做,.
但重點就是那些事情是否必要做,.
但如果大家一起見到需要的話,.
一起做就是最賞心,.
我想大家都有做家務,.
很難說,.
但家務這件事就是,.
你做不覺,不做就很覺,.
不做就很骯髒,.
但每人做一點點就是做了那件事,.
所以好像說得很遠,.
不過最重要是大家投心,.
這樣就ok了..
後面那邊有一個,前面有一個,.
後面先可以,.
Hello各位,.
我是在海外年紀的,.
我是昨天才回來的,.
我回來之前就很開心,.
收到訊息有今天的課,.
覺得很合聽,.
我可以分享一下,.
我在海外的年紀,.
我覺得我在英國的時候,.
我不覺得那裡是教,.
在周邊都是很多外國人的樣子,.

$^{1081}$很少人說廣東話,.
會覺得我究竟去了哪裡,.
很飄泊的感覺是有的,.
還有跟香港的時差,.
基本上你想找朋友,.
或者等於知道有些困難,.
就算我留個訊息給他,.
可能他第二天才回覆,.
我覺得在這裡,.
個人很不穩定,很不紮緊,.
剛去到又不是認識了很多朋友,.
初初我有回教會,.
後來覺得教會不適合,.
去了兩個月就沒有回,.
一直沒有再玩教會,.
所以覺得那裡人很飄泊,.
剛才我聽的時候,.
其中有一個powerpoint是說,.
homing是guiding to a destination,.
那一刻給我一種安慰,.
無論我在海外,.
或者現在我回到香港,.
其實我回來的時候,.
我都有想,.
究竟我回來香港是旅行,.
還是回家呢?.
我都有這樣的疑問,.
Facebook有很多人留言給我,.
說你是回家的,.
我調整了心態,.
對,我也可能是回家的,.
但剛才那句guiding to a destination,.
開始給我一個啟發,.
就是我不需要著重,.
我究竟在哪個地方生活,.
我最終的目標就是天價,.
我覺得這樣想對我來說會舒服些,.
因為我覺得我去年的時候,.
我經常都很想在英國找一個家的感覺,.
我想除了我沒有回教會,.

$^{1121}$認識的人不多,.
剛才John說,.
回教會是跟別人有一個關係,.
我覺得我在去年沒有跟別人有關係,.
所以我不覺得那個地方是一個家,.
雖然我那裡有工作,.
有我自己同事,.
但你只可以說那些是朋友,.
不像以前在flowchurch少了的時候,.
你可以分享自己的事,.
而對方的feedback,.
就算回教會崇拜,.
大家也會約會,.
或者之前會一起吃東西,.
你會覺得跟這些人有關係,.
那家的感覺會重些,.
我這次回來一兩天,.
除了約朋友,.
也很感恩flowchurch,.
幫我聯繫一下弟兄姊妹,.
他們也很熱烈,.
也會出來跟我吃飯,.
我就開始覺得,.
好像找回家人一樣,.
我今天再進來這個地方,.
我覺得感覺很熟悉,.
以前在這裡開小組,.
可能我分享得不是很好,.
不過我覺得很明白,.
你要跟別人有一個關係,.
也有一個家的感覺,.
多謝你分享..
現在,尤其在教會,.
對於多數講家的觀念很重,.
通常有些門檻可能很小,.
可能要求班足一年穩定的聚會,.
才可以讓你洗禮,.
進來做一些事夢,.
去擔任一些崗位,.
要洗禮後才可以做,.

$^{1161}$其實以前一直流傳,.
這些傳統究竟應該怎樣面對呢?.
剛才也說了,.
我們沒有一個固定的教會,.
其他是怎樣呢?.
我們在流動的過程中,.
我想你的問題,.
我會分開來說,.
他回家多久,.
跟他工作和洗禮,.
我就不會找唯一的,.
例如你剛才說洗禮後才可以擔任崗位,.
以flowchurch來說,.
就不要說其他,.
flowchurch也不會一加入flowchurch,.
回了flowchurch就立即工作,.
對我們來說,.
以洗禮為例,.
要加入小組才會跟他進行洗禮,.
原理就是,.
我們覺得洗禮不是一個魔法,.
洗禮是一種關係的建立,.
讓他有牧羊根進,.
跟他開展熟齡新的一頁,.
跟侍奉是不掛勾的,.
洗完禮他也不一定要一人一侍奉,.
所以我想回應你,.
工作不是因為洗禮的門檻,.
工作也不是因為洗禮後續要有什麼委身,.
我想釐清這個沒有連帶關係,.
至於關於多久可以工作,.
我們沒有時限,.
對我們來說,.
他是否看到需要,.
是否願意委身才一起參與,.
不是你剛才的思路或想法,.
好像要有門檻做了才參與,.
你明白我的意思嗎?.
(何君堯:如果我們回想起什麼角度看呢?).
這不是Flow Church的想法,.

$^{1201}$不是侍奉和洗禮掛勾,.
你不洗禮就有多厲害也不做,.
不是這些思路想法,.
(何君堯:剛才有舉手).
我來Flow Church沒多久,.
其實這些主題,.
剛才說的客女或回家,.
都是很傳統到現代也有的討論主題,.
不過時間關係,.
只分享少少,.
關於教會或Flow Church,.
剛好過去一年或幾個月,.
經歷了很多更新,.
以前我比較傳統,.
不喜歡弟兄姊妹不每個星期回來,.
以前我是這種,.
覺得很辛苦,.
為何人們當是Staycation,.
不理解,.
後來我自己也受不了,.
我完全不回來,.
因為太多事情處理不了,.
然後放下了,.
最後我覺得神令我明白,.
祂想我因為祂的我而回來,.
不要因為被迫要回來,.
不要覺得我不回來就很頑皮,.
而是一種我想回來,.
和其他人一起敬拜神,.
想和大家一起那樣東西而回來,.
所以我覺得家是,.
如果我想回來就有家的感覺,.
如果被迫就沒有,.
另外一件事是,.
有時可能靈唱的弟兄姊妹,.
或是講道,.
當她說一些信念我很認同時,.
其實也會有家的感覺,.
因為有時人數多的church,.
很難認識這麼多人,.

$^{1241}$或是很難和大家很熟,.
但如果台上的人說的信念很認同,.
其實也會有一種連繫感,.
就是這樣..
我想大家很值得看一套日劇,.
同一承下,.
我想我們劉唐想做的家是這樣的,.
不是傳統中國家春秋那些,.
家庭成為一種捆索,負擔,政治,.
一種包袱,責任,.
而是好像同一承下一樣,.
大家自願自負成為一個家庭,.
是一種自由,.
但在自由之下又願意付出,.
這個我們正想營造,.
不容易的,.
因為需要大家一起參與付出,.
才成為同一承下,.
早餐法子加起來,.
大家一起付出,.
去關心,.
才成為這樣的家,.
而不是一種傳統教人吃不消的家庭..
就是這樣..
有沒有其他對於家的想法,觀念,.
或者大家對於投入家的程度,.
有不同的分享?.
如果沒有的話,.
如果你收看我們YouTube的話,.
歡迎你,.
不是歡迎,.
我們都是在小組裡面,.
繼續分享,.
這個也是一個很重要的題目,.
希望你們都能夠分享,.
你怎樣理解自己的教會生活,.
自己對於群體的貢獻,.
還有看同一承下,.
這是一個很好看的日劇..
我們就來吧,.

$^{1281}$下個月再見吧..
今天YouTube也有很多人留意..
是啊..
好吧,.
我們就回家吧..
好吧,.
再見..
謝謝大家.
\newpage



\section{}
\label{sec:I6Z1WA7E0RA}
\textbf{《致餘民及流散者:給香港基督徒的神學八課》第二季第5課|20230625 [I6Z1WA7E0RA]}
\newline
\newline
連結: \href{https://youtube.com/watch?v=I6Z1WA7E0RA}{\texttt{ https://youtube.com/watch?v=I6Z1WA7E0RA}} ~~~~ 語音日期: 2023-06-25 
\newline
\newline
\hyperref[sec:XixhhdfEXw8]{\small{< < < PREV SERMON < < <}}
~
\hyperref[sec:index_chronic]{\small{[返順時目]}}
~
\hyperref[sec:index_scriptual]{\small{[返順卷目]}}
~
\hyperref[sec:J_OpyaPLYIE]{\small{> > > NEXT SERMON > > >}}
\newline
\newline
$^{1}$我只想知道.
你到底是什麼意思.
我只想知道.
你到底是什麼意思.
我只想知道.
你到底是什麼意思.
我只想知道.
你到底是什麼意思.
我只想知道.
你到底是什麼意思.
我只想知道.
你到底是什麼意思.
我只想知道.
你到底是什麼意思.
我只想知道.
你到底是什麼意思.
我只想知道.
你到底是什麼意思.
我只想知道.
你到底是什麼意思.
我只想知道.
你到底是什麼意思.
我只想知道.
你到底是什麼意思.
我只想知道.
你到底是什麼意思.
我只想知道.
你到底是什麼意思.
我只想知道.
你到底是什麼意思.
我只想知道.
你到底是什麼意思.
我只想知道.
你到底是什麼意思.
我只想知道.
你到底是什麼意思.
我只想知道.
你到底是什麼意思.
我只想知道.
你到底是什麼意思.

$^{41}$我只想知道.
你到底是什麼意思.
我只想知道.
你到底是什麼意思.
我只想知道.
你到底是什麼意思.
我只想知道.
你到底是什麼意思.
我只想知道.
你到底是什麼意思.
我只想知道.
你到底是什麼意思.
我只想知道.
你到底是什麼意思.
我只想知道.
你到底是什麼意思.
我只想知道.
你到底是什麼意思.
我只想知道.
你到底是什麼意思.
我只想知道.
你到底是什麼意思.
我只想知道.
你到底是什麼意思.
我只想知道.
你到底是什麼意思.
我只想知道.
你到底是什麼意思.
我只想知道.
你到底是什麼意思.
我只想知道.
你到底是什麼意思.
我只想知道.
你到底是什麼意思.
我只想知道.
你到底是什麼意思.
我只想知道.
你到底是什麼意思.
我只想知道.
你到底是什麼意思.

$^{81}$我只想知道.
你到底是什麼意思.
我只想知道.
你到底是什麼意思.
我只想知道.
你到底是什麼意思.
我只想知道.
你到底是什麼意思.
我只想知道.
你到底是什麼意思.
我只想知道.
你到底是什麼意思.
我只想知道.
你到底是什麼意思.
我只想知道.
你到底是什麼意思.
我只想知道.
你到底是什麼意思.
我只想知道.
你到底是什麼意思.
我只想知道.
你到底是什麼意思.
我只想知道.
你到底是什麼意思.
我只想知道.
你到底是什麼意思.
我只想知道.
你到底是什麼意思.
我只想知道.
你到底是什麼意思.
我只想知道.
你到底是什麼意思.
我只想知道.
你到底是什麼意思.
我只想知道.
你到底是什麼意思.
我只想知道.
你到底是什麼意思.
我只想知道.
你到底是什麼意思.

$^{121}$我只想知道.
你到底是什麼意思.
我只想知道.
你到底是什麼意思.
我只想知道.
你到底是什麼意思.
我只想知道.
你到底是什麼意思.
我只想知道.
你到底是什麼意思.
我只想知道.
你到底是什麼意思.
我只想知道.
你到底是什麼意思.
我只想知道.
你到底是什麼意思.
我只想知道.
你到底是什麼意思.
我只想知道.
你到底是什麼意思.
我只想知道.
你到底是什麼意思.
我只想知道.
你到底是什麼意思.
我只想知道.
你到底是什麼意思.
我只想知道.
你到底是什麼意思.
我只想知道.
你到底是什麼意思.
我只想知道.
你到底是什麼意思.
我只想知道.
你到底是什麼意思.
我只想知道.
你到底是什麼意思.
我只想知道.
你到底是什麼意思.
我只想知道.
你到底是什麼意思.

$^{161}$我只想知道.
你到底是什麼意思.
我只想知道.
你到底是什麼意思.
我只想知道.
你到底是什麼意思.
我只想知道.
你到底是什麼意思.
我只想知道.
你到底是什麼意思.
我只想知道.
你到底是什麼意思.
我只想知道.
你到底是什麼意思.
我只想知道.
你到底是什麼意思.
我只想知道.
你到底是什麼意思.
我只想知道.
你到底是什麼意思.
我只想知道.
你到底是什麼意思.
我只想知道.
你到底是什麼意思.
我只想知道.
你到底是什麼意思.
我只想知道.
你到底是什麼意思.
我只想知道.
你到底是什麼意思.
我只想知道.
你到底是什麼意思.
我只想知道.
你到底是什麼意思.
我只想知道.
你到底是什麼意思.
我只想知道.
你到底是什麼意思.
我只想知道.
你到底是什麼意思.

$^{201}$我只想知道.
你到底是什麼意思.
我只想知道.
你到底是什麼意思.
我只想知道.
你到底是什麼意思.
我只想知道.
你到底是什麼意思.
我只想知道.
你到底是什麼意思.
我只想知道.
你到底是什麼意思.
我只想知道.
你到底是什麼意思.
我只想知道.
你到底是什麼意思.
我只想知道.
你到底是什麼意思.
我只想知道.
你到底是什麼意思.
我只想知道.
你到底是什麼意思.
我只想知道.
你到底是什麼意思.
我只想知道.
你到底是什麼意思.
我只想知道.
你到底是什麼意思.
我只想知道.
你到底是什麼意思.
我只想知道.
你到底是什麼意思.
我只想知道.
你到底是什麼意思.
我只想知道.
你到底是什麼意思.
我只想知道.
你到底是什麼意思.
我只想知道.
你到底是什麼意思.

$^{241}$我只想知道.
你到底是什麼意思.
我只想知道.
你到底是什麼意思.
我只想知道.
你到底是什麼意思.
我只想知道.
你到底是什麼意思.
我只想知道.
你到底是什麼意思.
我只想知道.
你到底是什麼意思.
我只想知道.
你到底是什麼意思.
我只想知道.
你到底是什麼意思.
我只想知道.
你到底是什麼意思.
我只想知道.
你到底是什麼意思.
我只想知道.
你到底是什麼意思.
我只想知道.
你到底是什麼意思.
我只想知道.
你到底是什麼意思.
我只想知道.
你到底是什麼意思.
我只想知道.
你到底是什麼意思.
我只想知道.
你到底是什麼意思.
我只想知道.
你到底是什麼意思.
我只想知道.
你到底是什麼意思.
我只想知道.
你到底是什麼意思.
我只想知道.
你到底是什麼意思.

$^{281}$我只想知道.
你到底是什麼意思.
我只想知道.
你到底是什麼意思.
我只想知道.
你到底是什麼意思.
我只想知道.
你到底是什麼意思.
我只想知道.
你到底是什麼意思.
我只想知道.
你到底是什麼意思.
我只想知道.
你到底是什麼意思.
我只想知道.
你到底是什麼意思.
我只想知道.
你到底是什麼意思.
我只想知道.
你到底是什麼意思.
我只想知道.
你到底是什麼意思.
我只想知道.
你到底是什麼意思.
我只想知道.
你到底是什麼意思.
我只想知道.
你到底是什麼意思.
我只想知道.
你到底是什麼意思.
我只想知道.
你到底是什麼意思.
我只想知道.
你到底是什麼意思.
我只想知道.
你到底是什麼意思.
我只想知道.
你到底是什麼意思.
我只想知道.
你到底是什麼意思.

$^{321}$我只想知道.
你到底是什麼意思.
我只想知道.
你到底是什麼意思.
我只想知道.
你到底是什麼意思.
我只想知道.
你到底是什麼意思.
我只想知道.
你到底是什麼意思.
我只想知道.
你到底是什麼意思.
我只想知道.
你到底是什麼意思.
我只想知道.
你到底是什麼意思.
我只想知道.
你到底是什麼意思.
我只想知道.
你到底是什麼意思.
我只想知道.
你到底是什麼意思.
我只想知道.
你到底是什麼意思.
我只想知道.
你到底是什麼意思.
我只想知道.
你到底是什麼意思.
我只想知道.
你到底是什麼意思.
我只想知道.
你到底是什麼意思.
我只想知道.
你到底是什麼意思.
我只想知道.
你到底是什麼意思.
我只想知道.
你到底是什麼意思.
我只想知道.
你到底是什麼意思.

$^{361}$我只想知道.
你到底是什麼意思.
我只想知道.
你到底是什麼意思.
我只想知道.
你到底是什麼意思.
我只想知道.
你到底是什麼意思.
我只想知道.
你到底是什麼意思.
我只想知道.
你到底是什麼意思.
我只想知道.
你到底是什麼意思.
我只想知道.
你到底是什麼意思.
我只想知道.
你到底是什麼意思.
我只想知道.
你到底是什麼意思.
我只想知道.
你到底是什麼意思.
我只想知道.
你到底是什麼意思.
我只想知道.
你到底是什麼意思.
我只想知道.
你到底是什麼意思.
我只想知道.
你到底是什麼意思.
我只想知道.
你到底是什麼意思.
我只想知道.
你到底是什麼意思.
我只想知道.
你到底是什麼意思.
我只想知道.
你到底是什麼意思.
我只想知道.
你到底是什麼意思.

$^{401}$我只想知道.
你到底是什麼意思.
我只想知道.
你到底是什麼意思.
我只想知道.
你到底是什麼意思.
我只想知道.
你到底是什麼意思.
我只想知道.
你到底是什麼意思.
我只想知道.
你到底是什麼意思.
我只想知道.
你到底是什麼意思.
我只想知道.
你到底是什麼意思.
我只想知道.
你到底是什麼意思.
我只想知道.
你到底是什麼意思.
我只想知道.
你到底是什麼意思.
我只想知道.
你到底是什麼意思.
我只想知道.
你到底是什麼意思.
我只想知道.
你到底是什麼意思.
我只想知道.
你到底是什麼意思.
我只想知道.
你到底是什麼意思.
我只想知道.
你到底是什麼意思.
我只想知道.
你到底是什麼意思.
我只想知道.
你到底是什麼意思.
我只想知道.
你到底是什麼意思.

$^{441}$我只想知道.
你到底是什麼意思.
我只想知道.
你到底是什麼意思.
我只想知道.
你到底是什麼意思.
我只想知道.
你到底是什麼意思.
我只想知道.
你到底是什麼意思.
我只想知道.
你到底是什麼意思.
我只想知道.
你到底是什麼意思.
我只想知道.
你到底是什麼意思.
我只想知道.
你到底是什麼意思.
我只想知道.
你到底是什麼意思.
我只想知道.
你到底是什麼意思.
我只想知道.
你到底是什麼意思.
我只想知道.
你到底是什麼意思.
我只想知道.
你到底是什麼意思.
我只想知道.
你到底是什麼意思.
我只想知道.
你到底是什麼意思.
我只想知道.
你到底是什麼意思.
我只想知道.
你到底是什麼意思.
我只想知道.
你到底是什麼意思.
我只想知道.
你到底是什麼意思.

$^{481}$我只想知道.
你到底是什麼意思.
我只想知道.
你到底是什麼意思.
我只想知道.
你到底是什麼意思.
我只想知道.
你到底是什麼意思.
我只想知道.
你到底是什麼意思.
我只想知道.
你到底是什麼意思.
我只想知道.
你到底是什麼意思.
我只想知道.
你到底是什麼意思.
我只想知道.
你到底是什麼意思.
我只想知道.
你到底是什麼意思.
我只想知道.
你到底是什麼意思.
我只想知道.
你到底是什麼意思.
我只想知道.
你到底是什麼意思.
我只想知道.
你到底是什麼意思.
我只想知道.
你到底是什麼意思.
我只想知道.
你到底是什麼意思.
我只想知道.
你到底是什麼意思.
我只想知道.
你到底是什麼意思.
我只想知道.
你到底是什麼意思.
我只想知道.
你到底是什麼意思.

$^{521}$我只想知道.
你到底是什麼意思.
我只想知道.
你到底是什麼意思.
我只想知道.
你到底是什麼意思.
我只想知道.
你到底是什麼意思.
我只想知道.
你到底是什麼意思.
我只想知道.
你到底是什麼意思.
我只想知道.
你到底是什麼意思.
我只想知道.
你到底是什麼意思.
我只想知道.
你到底是什麼意思.
我只想知道.
你到底是什麼意思.
我只想知道.
你到底是什麼意思.
我只想知道.
你到底是什麼意思.
我只想知道.
你到底是什麼意思.
我只想知道.
你到底是什麼意思.
我只想知道.
你到底是什麼意思.
我只想知道.
你到底是什麼意思.
我只想知道.
你到底是什麼意思.
我只想知道.
你到底是什麼意思.
我只想知道.
你到底是什麼意思.
我只想知道.
你到底是什麼意思.

$^{561}$我只想知道.
你到底是什麼意思.
我只想知道.
你到底是什麼意思.
我只想知道.
你到底是什麼意思.
我只想知道.
你到底是什麼意思.
我只想知道.
你到底是什麼意思.
我只想知道.
你到底是什麼意思.
我只想知道.
你到底是什麼意思.
我只想知道.
你到底是什麼意思.
我只想知道.
你到底是什麼意思.
我只想知道.
你到底是什麼意思.
我只想知道.
你到底是什麼意思.
我只想知道.
你到底是什麼意思.
我只想知道.
你到底是什麼意思.
我只想知道.
你到底是什麼意思.
我只想知道.
你到底是什麼意思.
我只想知道.
你到底是什麼意思.
我只想知道.
你到底是什麼意思.
我只想知道.
你到底是什麼意思.
我只想知道.
你到底是什麼意思.
我只想知道.
你到底是什麼意思.

$^{601}$我只想知道.
你到底是什麼意思.
我只想知道.
你到底是什麼意思.
我只想知道.
你到底是什麼意思.
我只想知道.
你到底是什麼意思.
我只想知道.
你到底是什麼意思.
我只想知道.
你到底是什麼意思.
我只想知道.
你到底是什麼意思.
我只想知道.
你到底是什麼意思.
我只想知道.
你到底是什麼意思.
我只想知道.
你到底是什麼意思.
我只想知道.
你到底是什麼意思.
我只想知道.
你到底是什麼意思.
我只想知道.
你到底是什麼意思.
我只想知道.
你到底是什麼意思.
我只想知道.
你到底是什麼意思.
我只想知道.
你到底是什麼意思.
我只想知道.
你到底是什麼意思.
我只想知道.
你到底是什麼意思.
我只想知道.
你到底是什麼意思.
我只想知道.
你到底是什麼意思.

$^{641}$我只想知道.
你到底是什麼意思.
我只想知道.
你到底是什麼意思.
我只想知道.
你到底是什麼意思.
我只想知道.
你到底是什麼意思.
我只想知道.
你到底是什麼意思.
我只想知道.
你到底是什麼意思.
我只想知道.
你到底是什麼意思.
我只想知道.
你到底是什麼意思.
我只想知道.
你到底是什麼意思.
我只想知道.
你到底是什麼意思.
我只想知道.
你到底是什麼意思.
我只想知道.
你到底是什麼意思.
我只想知道.
你到底是什麼意思.
我只想知道.
你到底是什麼意思.
我只想知道.
你到底是什麼意思.
我只想知道.
你到底是什麼意思.
我只想知道.
你到底是什麼意思.
我只想知道.
你到底是什麼意思.
我只想知道.
你到底是什麼意思.
我只想知道.
你到底是什麼意思.

$^{681}$我只想知道.
你到底是什麼意思.
我只想知道.
你到底是什麼意思.
我只想知道.
你到底是什麼意思.
我只想知道.
你到底是什麼意思.
我只想知道.
你到底是什麼意思.
我只想知道.
你到底是什麼意思.
我只想知道.
你到底是什麼意思.
我只想知道.
你到底是什麼意思.
我只想知道.
你到底是什麼意思.
我只想知道.
你到底是什麼意思.
我只想知道.
你到底是什麼意思.
我只想知道.
你到底是什麼意思.
我只想知道.
你到底是什麼意思.
我只想知道.
你到底是什麼意思.
我只想知道.
你到底是什麼意思.
我只想知道.
你到底是什麼意思.
我只想知道.
你到底是什麼意思.
我只想知道.
你到底是什麼意思.
我只想知道.
你到底是什麼意思.
我只想知道.
你到底是什麼意思.

$^{721}$我只想知道.
你到底是什麼意思.
我只想知道.
你到底是什麼意思.
我只想知道.
你到底是什麼意思.
我只想知道.
你到底是什麼意思.
我只想知道.
你到底是什麼意思.
我只想知道.
你到底是什麼意思.
我只想知道.
你到底是什麼意思.
我只想知道.
你到底是什麼意思.
我只想知道.
你到底是什麼意思.
我只想知道.
你到底是什麼意思.
我只想知道.
你到底是什麼意思.
我只想知道.
你到底是什麼意思.
我只想知道.
你到底是什麼意思.
我只想知道.
你到底是什麼意思.
我只想知道.
你到底是什麼意思.
我只想知道.
你到底是什麼意思.
我只想知道.
你到底是什麼意思.
我只想知道.
你到底是什麼意思.
我只想知道.
你到底是什麼意思.
我只想知道.
你到底是什麼意思.

$^{761}$我只想知道.
你到底是什麼意思.
我只想知道.
你到底是什麼意思.
我只想知道.
你到底是什麼意思.
我只想知道.
你到底是什麼意思.
我只想知道.
你到底是什麼意思.
我只想知道.
你到底是什麼意思.
我只想知道.
你到底是什麼意思.
我只想知道.
你到底是什麼意思.
我只想知道.
你到底是什麼意思.
我只想知道.
你到底是什麼意思.
我只想知道.
你到底是什麼意思.
我只想知道.
你到底是什麼意思.
我只想知道.
你到底是什麼意思.
我只想知道.
你到底是什麼意思.
我只想知道.
你到底是什麼意思.
我只想知道.
你到底是什麼意思.
我只想知道.
你到底是什麼意思.
我只想知道.
你到底是什麼意思.
我只想知道.
你到底是什麼意思.
我只想知道.
你到底是什麼意思.

$^{801}$我只想知道.
你到底是什麼意思.
我只想知道.
你到底是什麼意思.
我只想知道.
你到底是什麼意思.
我只想知道.
你到底是什麼意思.
我只想知道.
你到底是什麼意思.
我只想知道.
你到底是什麼意思.
我只想知道.
你到底是什麼意思.
我只想知道.
你到底是什麼意思.
我只想知道.
你到底是什麼意思.
我只想知道.
你到底是什麼意思.
我只想知道.
你到底是什麼意思.
我只想知道.
你到底是什麼意思.
我只想知道.
你到底是什麼意思.
我只想知道.
你到底是什麼意思.
我只想知道.
你到底是什麼意思.
我只想知道.
你到底是什麼意思.
我只想知道.
你到底是什麼意思.
我只想知道.
你到底是什麼意思.
我只想知道.
你到底是什麼意思.
我只想知道.
你到底是什麼意思.

$^{841}$我只想知道.
你到底是什麼意思.
我只想知道.
你到底是什麼意思.
我只想知道.
你到底是什麼意思.
我只想知道.
你到底是什麼意思.
我只想知道.
你到底是什麼意思.
我只想知道.
你到底是什麼意思.
我只想知道.
你到底是什麼意思.
我只想知道.
你到底是什麼意思.
我只想知道.
你到底是什麼意思.
我只想知道.
你到底是什麼意思.
我只想知道.
你到底是什麼意思.
我只想知道.
你到底是什麼意思.
我只想知道.
你到底是什麼意思.
我只想知道.
你到底是什麼意思.
我只想知道.
你到底是什麼意思.
我只想知道.
你到底是什麼意思.
我只想知道.
你到底是什麼意思.
我只想知道.
你到底是什麼意思.
我只想知道.
你到底是什麼意思.
我只想知道.
你到底是什麼意思.

$^{881}$我只想知道.
你到底是什麼意思.
我只想知道.
你到底是什麼意思.
我只想知道.
你到底是什麼意思.
我只想知道.
你到底是什麼意思.
我只想知道.
你到底是什麼意思.
我只想知道.
你到底是什麼意思.
我只想知道.
你到底是什麼意思.
我只想知道.
你到底是什麼意思.
我只想知道.
你到底是什麼意思.
我只想知道.
你到底是什麼意思.
我只想知道.
你到底是什麼意思.
我只想知道.
你到底是什麼意思.
我只想知道.
你到底是什麼意思.
我只想知道.
你到底是什麼意思.
我只想知道.
你到底是什麼意思.
我只想知道.
你到底是什麼意思.
我只想知道.
你到底是什麼意思.
我只想知道.
你到底是什麼意思.
我只想知道.
你到底是什麼意思.
我只想知道.
你到底是什麼意思.

$^{921}$我只想知道.
你到底是什麼意思.
我只想知道.
你到底是什麼意思.
我只想知道.
你到底是什麼意思.
我只想知道.
你到底是什麼意思.
我想說這個.
Thomas Aquinas的目的是做什麼.
他想嘗試去看著這個世界.
從而去推論這個世界背後的上帝.
所以他覺得這個世界和上帝之間.
是有一定程度的結連.
你看著這個世界.
是可以找回上帝的足跡.
這是一個很重要的進路.
你看著世界里的一些事情.
你能夠找到一些理性的原因.
跡象去找回上帝的存在.
但問題是.
究竟上帝存在是什麼意思.
上帝存在是不是我們能夠理解的東西.
上帝存在和你媽媽存在是不是一回事.
上帝存在的意思是不是普通的肉眼看到.
或者是起到的意思.
這是一個很深的問題.
所以有些人覺得上帝不存在.
因為上帝不是我們平時所想的那一切的存在.
所以上帝可以說是不存在的.
因為不是按著人世間的存在理解去存在.
所以我們不要嘗試用一般我們平時理解的方式.
去理解上帝的存在.
他未必一定有一個手摸到你.
或者未必一定有一個緊緊的擁抱抱著你.
或者是一把聲音讓你聽到.
這樣的存在.
所以上帝不是那一種的存在.
所以如果我們用我們平時所理解的存在方法.
對的上帝是不存在的.

$^{961}$不是那一種的存在.
所以我們今天想說的是.
我們可以想多一層.
究竟對你來說上帝起到是什麼意思.
上帝起到因為我今天追巴士的時候被我追到.
然後你就覺得這是上帝的恩典.
你可以這樣去論證他的存在.
我很肚痛,今天上帝神父去證實我.
所以我就要認罪了.
這也是一種你自己去理解的.
所以上帝存在的意思.
如果用一般的理解可能不是這樣.
所以今天純粹想上帝的起到.
上帝的absence.
我們嘗試不是用一般你去理解一個人在哪裡.
或者一件事物在哪裡的方式存在.
因為上帝不是這個世界.
所以我們不能用世界的事情存在的方式.
來理解上帝的存在.
接著想說的是潘福華.
潘福華雖然是一生裡面去對抗希特勒.
但其實他在坐牢的時候.
寫了幾封信去討論一些很有深度的問題.
他寫給了他的朋友.
有幾封信談論到將來基督教的發展.
如果他沒有被處死刑的時候.
信仰是一回什麼事情.
基督教是一回什麼事情.
是一句很震撼的話.
其實我之前也聽過.
所謂的吸靈的世界.
或者叫做非宗教的教程.
我也說過.
但當我預備這一課的時候.
我再看那封信原本的資料的時候.
其實他所說的話會更加深入.
他說上帝告訴我們.
我們必須以沒有上帝的方式去生活.
與我們同在的上帝是離棄我們的上帝.
所以他想說的是.

$^{1001}$我們去到20世紀的德國.
去到20世紀的世界.
這個世界是一個現代世界.
他稱之為一個吸靈.
是一個成熟的世界.
我們這個世界似乎開始不需要靠上帝生活.
你不是原始人.
你不是一些澳洲的土著.
我們的世界.
我否認開始去到某個位置.
你開始能夠認知很多的東西.
你知道原來下雨是一回事.
你不會求雨神.
你知道下雨是因為天文台有個低壓槽.
下雨你會知道那個科學背後是什麼事情.
我說過Virus也是一樣.
Virus是這八年才發現的.
一千年前的人看到病毒的時候.
他們以為是瘟疫.
瘟疫是一個配合某些神秘的東西.
所以我們這個世界去到這個位置.
是開始去到吸靈的世界.
我們其實不需要有一個上帝來解釋很多的東西.
當然你現在有很多東西是需要解釋的.
解釋不了的.
譬如說癌症的出現.
你都會求一些奇跡神跡來幫助你.
有些東西你仍然解釋不了.
譬如說你的手機壞了你會祈禱.
除非你是IT人.
IT人就會重新安排你的手機.
有些東西你解釋不了的時候.
你就會開始去找這位上帝.
問題是我們這個年代.
我們越來越少的東西解釋不了.
我們越來越多.
我們明白當中背後的運作.
很多東西你都會開始發現.
原來背後某些道理某些真相是這樣.
而潘福華是怎麼看這件事呢.

$^{1041}$他說其實是上帝知道的.
上帝告訴他.
我們開始需要以一個沒有上帝去生活的方式生活.
這不是一件冒犯的事情.
人類的知識到了某個水平的時候.
我們開始不需要有上帝作為解釋原因.
而我們同時可以相信上帝.
這正是困難的地方.
我不知道你怎麼相信耶穌.
可能是某些你覺得很解決不了的事情.
所以你有一刻相信耶穌.
但有一刻有一些事情是解釋不了.
或者是處理不了.
唯有靠神去幫助你.
但是當你處理到這些事情.
那你怎麼辦呢.
所以潘福華說.
上帝是開始以一種.
好像不需要有上帝去生活的方式.
就現在這樣的年代來生活.
和我們同在的上帝.
同時是離棄的上帝.
然後潘福華就曲了十字架的道理.
所謂離棄上帝是一位被釘十字架的上帝.
我們所相信的是一位這樣的上帝.
接著下面有一句.
我在另一些講座也有說過這句話.
但原來不知道兩句話是連在一起的.
他說讓我們再沒有上帝工作的假設下生活的上帝.
就是我們不斷要面對的上帝.
我們開始面對著這樣的生活方法.
好像你不需要上帝的方式來生活.
我們在上帝面前.
和上帝同在同時.
卻在一個沒有上帝的情況下生活.
不知道你明白這句話的意思嗎.
你明白上帝同在.
但你上班.
你上班的時候不需要靠上帝來生活.
坦白說.

$^{1081}$你不做就會被人罵.
你開會有些事情要處理.
你祈禱是需要.
但你確實有些事情要做.
你開車會被人拍照.
被警察抓了就會被抓了.
這些祈禱是沒用的.
你會祈禱祈禱上帝.
但你不會有成績.
我講過很多次這個例子.
我曾經在長洲.
一大早就乘船出來.
我講過這個例子.
我去長洲碼頭乘船.
突然發現我的錢包不見了.
那時候我繼續祈禱.
剛才發現原來不是我錢包不見了.
而是我沒有帶錢包.
那我會怎麼祈禱.
如果你是我的話.
我講船.
還有五分鐘就要開船.
那我怎麼辦.
我可以有很多方法去祈禱.
但我會選擇祈禱.
可能遇到一個同事借錢給我乘船.
但你不會祈禱.
錢包會從山裡飛下來.
你不會祈禱.
瞬間轉移去中環.
你可以這樣祈禱.
但你不會這樣祈禱.
所以你會用一個現代人.
去處理生活世界的方式去祈禱.
明白我的意思嗎.
所以我們在這樣的世界生活.
你是一個很普通的人.
你會用一些世界的方式去處理世界.
這絕對不是不屬靈的事.
教會要開會.

$^{1121}$師班要練師.
講道義 社會上的講章.
全部都是很重要的事.
所以我們是以一種.
照著世界的運作方法去過生活.
同時你相信上帝的同在.
不過卻被要求.
在一個沒有上帝的情況下去生活.
上帝讓自己被驅逐出世界.
被釘在十字架上.
上帝在世界是無能和軟弱的.
正是這樣.
他才是與我們同在並幫助我們.
十字架那段經文.
基督不是憑借祂的全能來幫助.
而是憑借祂的軟弱和受苦.
這是與所有宗教的關鍵區別.
人的宗教性將他在困境中指向上帝.
在世界中的權能.
上帝是機械中的神.
即是說人的宗教性.
往往是讓我們去發明一個這樣的神.
簡單來說就是黃大仙的神.
我們人的宗教.
一個人為的宗教.
就會期盼一個這樣的神.
即是有突然間某些事情可以幫助你.
一個突然間的聖徒來解救你.
為什麼這個叫做Deus Ex Machina.
就是在以前希臘的悲劇里.
最後會有一個用機械調出來的神.
我就是神.
我幫你解決問題.
然後就完.
是一個很差的本來的.
即是說一個超烏雞的神出現就完了.
一個這樣的神.
一個大團結的神.
所以我們宗教似乎是期待一個這樣的上帝.
一個能夠來一個什麼事都是上帝解決完.

$^{1161}$完美結局的上帝.
但是他說聖經上帝不是這樣.
聖經上帝是十字架上被釘死的上帝.
是一個完全不上帝的上帝.
上帝在受苦的上帝里才能夠幫助我們.
他說正正是因為這位十字架的上帝.
才在一個靠近的世界裡面是有意義的.
不知道你明不明白我的意思.
一個靠近的世界.
一個Modern World.
不再需要一個這樣的宗教.
因為很多東西你都能解釋得到.
不需要一個上帝.
反過來說.
一個在十字架裡面受苦的上帝.
才能夠繼續在靠近的世界裡面存在.
所以十字架本身就是一個.
去對抗一切宗教的一回事.
這幾段時間我經常提這個詞.
我在Full Church講道裡面都說.
我不想Full Church成為一個搞得很大的宗教.
純粹去滿足一些宗教人士的宗教需要.
安慰你,疼愛你,寶寶.
這樣的一個上帝.
上帝在一個靠近的世界裡面.
不是純粹滿足人的宗教需要.
所以潘福華最後的結論.
就是一個非宗教的基督教.
一個再沒有宗教包裝外殼的基督教.
我覺得起碼對Full Church來說.
我是一種這樣的圖畫.
你看到我們很小事.
講道不穿西裝.
這件事本身也是這樣的事.
不講那麼多宗教儀式.
我叫穆斯普,但穆家族做的沒穿過 呼出裡面我沒穿過穆斯普的.
基本上是想將宗教的味道脫去 希望你平時所生活的香港就是你上教會的香港.
不會是兩個世界來的 所謂呼出能夠希望貼地.
希望我們的信仰能夠和大家生活一樣的信仰.
不是一個脫離香港政治困擾的地方 而是可以讓你發覺是同一個地方.

$^{1201}$所以我希望有一個非宗教的基督教.
可以和大家討論一下 怎樣是一個非宗教的基督教 不再有一種迷信任何半點人為宗教氣氛的基督教.
而是一個真真正正的基督教.
接著我們就會說 今天說的內容比較深.
說完潘福牙之後說一下莫特曼.
莫特曼在一本書 《丁十字架的上帝》裡面 其實這本書是說十字架神學.
有一個很有意思的點子 說十字架是一個對抗一切有神論信仰的事情.
什麼意思呢 十字架正正就是一個推翻人世間宗教信仰的一回事情.
所以我很喜歡聖經的一句話.
不知道大家有沒有留意過這句很有型的說話 聖經裡面保羅說的.
在《迦太史》裡面寫的.
「弟兄們,我若仍舊傳國禮,為什麼還不受逼迫呢?若是這樣,那十字架討厭的地方就沒有了.」.
這是和本翻譯.
和本不知道是不是令人感動 將這個字眼翻譯成十字架討厭的地方.
很多年前 蔡震清院長也有一本書叫《十字架討厭的地方》.
這個字本身的意思是 大家猜到的 是scandal的字.
其實就是scandal的字.
原文的意思是十字架的scandal 十字架的醜聞 十字架叛島人的地方.
所以有些翻譯成信仰的阻礙 十字架冒犯人的地方 十字架眼中釘.
總之是一個這樣的氣人的地方 用我們朝議的十字架的氣的地方.
它會卡住你 會令你跌倒 會令你不舒服.
但和本也很有意思 很有風格.
十字架討厭的地方.
所以十字架本來就是這樣的東西.
所以保羅有一個這樣的意思 如果我們仍然去傳國禮的話.
這樣十字架討厭的地方就沒有了.
保羅似乎是想保留十字架討厭的地方.
他想保留十字架很奇怪的地方 十字架令我們很不舒服的地方.
但我們這二千年來 十字架是甚麼來的.
十字架是水晶來的 紀念品來的 項鏈來的 T-shirt來的.
十字架是我們最宗教的東西來的 十字架成為了一個最有代表性的宗教記號.
就算你淘寶也是一樣 淘寶你買不到聖經的 但淘寶可以買到十字架.
請問.
你試過問 如果你不信教可以買十字架.
因為是一個很普通的宗教記號.
但我們似乎忘記了十字架的原意就是要人生厭.
本來就是要你覺得很不舒服.
十字架是一個凝聚 古代羅馬死刑的方法.
以色列人把耶穌釘在十字架上是一件很羞澀的事.
尼塞亞被釘是一件大不道的事 你明不明白.

$^{1241}$其實你不明白 因為你感覺不到.
因為你已經感覺不到十字架有多羞澀.
我們試試改一改 不要用十字架來講.
我們試試用古代中國的清末民初年代.
瑪利亞是文女 十字架是最後晚餐是最後晚餐.
當耶穌和門徒吃完最後一杯茶 被清兵捉完之後.
然後被人捉去浸豬籠 你明不明白.
耶穌被人浸豬籠 旁邊有一對姦夫淫婦在他兩旁.
所以有三個浸豬籠在那裡 死刑.
最後耶穌也被人浸豬籠死了.
早期教會說 耶穌基督被人浸豬籠是一個好消息.
並且說當你要跟隨耶穌的人要背上你的豬籠.
是很有型的事情 背上豬籠是不有型的.
十字架是一件很奇怪的事情.
一件很褻瀆的事情 一件很不型的事情.
甚至是最沒有上帝的事情.
所以無論是潘福華還是莫特曼 或者是普羅多瓦都強調.
十字架正正是要你覺得討厭.
要你覺得這正正不像有上帝的方式.
就是上帝的方式.
十字架要推翻一切宗教.
十字架是最不宗教的方法.
奈何我們人就將十字架捧上天.
變成一個宗教最強的記號 變成一種宗教.
所以我覺得很不想全聖教成為另一個宗教.
或者今天很多基督教都是一種宗教.
純粹是耶穌基督版本的宗教.
客觀來說基督教是一個高級宗教.
我們有教義 我們有文獻 我們有很多不同的體制.
是一個很強勁的宗教.
但這也是一個宗教.
十字架正正是要推翻一切宗教的外表.
所以我想說的是.
我們嘗試去實踐一個非宗教的基督教.
去活出一個非宗教信仰的信仰的時候.
正正就是將一些人為的東西去除掉.
或者說十字架是一切宗教的照妖鏡.
今天你純粹是來尋求安慰.
尋求上帝疼愛你的寶寶的時候.
實際上不是 你要背十字架.

$^{1281}$十字架是令你痛苦的.
這個信仰是令你受苦的.
如果你想尋求敬拜的時候有些良好的感覺.
或者是很在乎小眾目的關不關心你.
這些事情的時候.
這些是宗教的需求.
但我經常覺得我們嘗試活出一個非宗教的基督教.
做回一個人 做回一個社會裡的人.
對人好 有人應該要對的態度.
我們仍然去持守基督耶穌要求我們做的事情.
多於上帝這樣的喜悼.
回到那句話題 上帝喜不喜悼.
對我來說 全聖教怎麼看上帝喜不喜悼呢.
不是那種喜悼.
今天我想說的是靈恩派教會.
我不是想插兩派教會.
我只不過是想用一個例子來說不是什麼.
我們所想的上帝不是什麼.
上帝喜悼不是這樣的意思.
所以靈恩派的優點是什麼.
正正就是將上帝喜悼變得很具體.
上帝喜悼是很明顯的.
因為有聖靈的充滿.
聖靈的動作 方言等等.
參加這些教會的人是很明顯地看到上帝喜悼.
很強調上帝喜悼.
因為他們用很多這樣的方式.
去將上帝在這裡展現出來給你看到.
區域邪靈也是一樣.
這裡有一隻懶惰的靈在.
所以你這麼懶惰.
他們將懶惰變成一個很具體的屬靈現象.
你只要弄走靈就不懶惰了.
這正正就是一個不是所說的吸靈的世界.
是一個哈利波特的世界.
將這個世界變成這樣是很簡單的.
你只要多點收靈就會弄走鬼.
多點上教會就會沒事了.
這是一個這樣的上帝喜悼的方式.
我不是想拆他們 我是很尊重他們.

$^{1321}$我想說我們不是這樣的.
上帝喜悼不是這樣的.
我們很少說上帝昨天這樣對我.
那次我聽到上帝這樣對我.
我不會這樣說的.
福祉律沒有目的這樣說.
不會說上帝昨天對我說.
我禱告了二十年.
神物上帝聽我祈禱.
不是這樣做的.
不是上帝不做這些.
但我不會將上帝起到這麼活生生的.
跟我聊天的.
我希望你能夠明白什麼叫上帝喜悼.
不是那種靈恩派教會的上帝喜悼.
所以我想說的是.
我希望你明白我想說什麼.
上帝不在場是OK的.
那種不在場.
不需要每次都在你旁邊.
當然他可以這樣做.
但他不需要每次都是這樣.
為什麼說我希望大家能夠成熟一點.
能夠理解那個上帝.
剛才那個第四格的上帝.
你未必見到他.
你未必每天早上聽到他說話.
但你知道他在.
但你未必見到他.
你仍然跟隨耶穌要你做的事.
做一個人.
做一個很好的人.
將你所有的熱心忠於上主.
但不需要像好歷人派的喜悼.
所以我講了不是的.
我也講了什麼是.
我們怎樣看聖靈.
如果不是靈恩派那種.
起碼我覺得也要處理.
所以我也處理不完.

$^{1361}$我覺得.
如何看上帝的同座.
因為聖靈正正就是上帝在.
以前我們也說過.
耶穌在前團體要給他一張椅子.
這張椅子是耶穌坐的.
不要讓他坐.
強調耶穌在.
強調耶穌的靈在就是聖靈.
所以聖靈是我們今天能夠.
講上帝同在的意思.
耶穌基督在.
全靈父上帝的右邊.
耶穌的靈.
聖靈.
天父的靈也在同在.
這種同在.
所以這種同在的意思.
主會再回來.
什麼叫主再來.
就是他不能起床.
他不能起床才能回來.
所以耶穌基督今天不是那種人.
他的靈.
在那裡.
我們理解父子靈就是這樣.
很公正的三角形.
What if 聖靈其實是一種.
更加強調是一種很動態的聖靈.
他是父子之間的關係的建立者.
父子兩個的關係當中的橋樑.
就是聖靈.
不單止是這樣.
更加是上帝和世界的關係的建立者.
我們在說聖靈的時候.
他不一定是靈恩派那種.
一種很強烈的第三者的存在.
明不明白.
不是上身那種.
不一定要是.

$^{1401}$有時候說.
聖經不一定要寫東西.
我做了什麼.
聖物是寶羅在監獄裡寫信.
不需要上身.
我寫了什麼 羅拔書.
不需要這樣叫物是聖經.
物是是一個很普通的寫信過程.
我們的聖經也是這樣.
是一個平時生活上和你一起的聖靈.
但不需要有第三者.
去侵入你的聖靈.
所以你的生活.
你的世界.
你的一切.
是有聖靈在.
但都是你.
你被人解僱了.
不要說魔鬼要你.
因為你做得不好.
或者那個人不好.
有些世界的方式可以解釋到的.
但仍然有上帝在.
仍然有聖靈在.
不需要有第三者的存在.
仍然我們用一個.
及靈世的方法去明白這個世界.
所以不說那麼深.
我們覺得聖靈不是一些很奇怪的東西.
上星期清心的篇章說了這件事.
聖靈的感動.
未必是一些很突如其來.
很奇怪的感動.
而是一些很普通的.
一些平常的東西.
都可以是聖靈的感動.
所以我們嘗試去.
活在一個現代世界裡.
去理解上帝.
最後我有幾個應用.

$^{1441}$我們作為一個.
及靈世界的基督徒.
一個不迷信的基督徒.
一個非宗教的信仰的基督徒.
我們有幾樣東西可以大家一起想想.
這個回小組大家可以討論一下.
第一個.
行動和禱告.
我們很強調.
禱告很重要.
但是禱告從來都沒有和你的行動有任何的矛盾.
你仍然要做你要做的事情.
你仍然需要禱告.
所以不是那些.
你經常聽到的謠言.
什麼韓國教會祈禱八個小時.
然後禱告就復興了教會.
你就會看到我們很不熟悉的.
經常都不禱告.
我們有禱告,但不是那種禱告.
不是說我禱完八個小時就能夠復興教會.
而是我們很重視行動和禱告.
我們作為基督徒.
我們很需要禱告.
更加想重視.
作為一個現代人我們需要行動.
很多東西你知道.
是可以解釋得到的.
我們要開會的.
做事要預備.
很多事情都是這樣.
我們不會抹殺我們禱告的重要.
當然單單只是行動而不禱告不是我們的事情.
但我覺得兩者是並行的.
我們很重視行動.
很重視禱告.
第二.
我想這個也是按需要.
很多人經常問的問題.
答完了.

$^{1481}$如何明白上帝的心意呢.
當然我們很多時候.
我們可以尋求上帝的心意.
聽聽有沒有上帝跟你說話.
不知道你以前是怎樣聽.
但我想你也未必是突然間.
保羅式的光射.
讓你瞎了三天.
然後聽到上帝跟你說話.
上帝跟你說話的方法可能都是.
不知道你有沒有試過.
揭聖經那些.
揭到大紀元看看那些.
這些我們也OK的.
但也不是你長期會用的方法.
你可能會尋求人的意見.
用理性去明白判斷.
再將事情禱告交給上帝.
你會用一些及齡的方法去明白.
你想移民也好.
麻煩填表格申請簽證.
你不會祈禱那麼簡單.
祈禱不會突然間簽簽證給你.
你會做事的.
你尋求神心意也是一樣.
你會期待著上帝跟你說話.
但你仍然會預備.
仍然會去判斷.
不過更多時候可能就是.
上帝未必是這樣跟你說話.
很多時候都是你先做了一件事之後.
才發現原來這是上帝的心意.
這個也是上帝的方式.
聖經也有很多這樣的例子.
伯來河也不知道去哪裡.
走到哪裡才回頭看.
原來上帝要來這裡.
所以我們也一樣.
我們不是不聽上帝的話.
但不是靈因派去聽上帝的話.

$^{1521}$不是這樣去尋求一些奇怪的東西.
才能明白上帝的心意.
第三點我仍然強調.
雖然是這樣.
雖然今天說了這麼多.
什麼靈的世界.
什麼非宗教的基督教.
作為一個人.
我想我們跟非基督徒不同的地方.
就是我們仍然相信奧秘.
相信神秘.
我們知道世界上有很多東西.
是可以這樣解釋的.
但仍然相信上帝有些東西是神秘.
有些東西是無法參透的.
你仍然懷有一份奧秘.
仍然有一份期盼.
雖然覺得這樣的情況好像不能拔.
但我仍然相信上帝會做事.
只是你不知道怎麼做.
也不是依靠靈因派那種去做.
但你仍然相信上帝仍然有些空間.
仍然超過你解釋得到的東西.
所以我自己是這樣去理解.
這這麼多年我這樣理解上帝的存在.
上帝的存在不是那種.
抱歉,是靈因派那種.
我用它做例子.
不是那種.
好像是第四格漫畫那種.
我今天早上也見不到神.
抱歉,雖然我是牧師.
但我今天也見不到神.
未必聽到祂的說話.
但我仍然相信上帝在這裡.
但不是那種在這裡.
我也不需要去依賴那種在這裡.
而是更加要警醒自己.
不要用這麼多宗教的方式去明白你的信仰.
用十字架來做那個照妖鏡.

$^{1561}$十字架正正是推翻一切人為的構想出來的上帝.
一個最不像上帝的上帝.
在十字架裡面.
我們一起祈禱.
因為你在十字架裡面.
來表明你是上帝.
我們知道你和祂同在.
但我們知道你不是這樣的同在.
我們需要知道你怎樣.
這樣去構想你是一種昔日亞倫他們拜金網讀的事情.
他們假信一個上帝並敬拜他.
求主讓我們能夠認清你的同在.
更加去拆掉很多宗教.
我們更加去毀身於你.
不是毀身一個宗教.
而是單單去毀身於你自己.
求你讓我們能夠有你成人當中去分辨.
讓我們能夠真真正正地去跟隨你.
相信你.
帶領我們全家人.
能夠有更多的人能夠起來.
成為跟隨你的人.
馮春永求.
阿們.
餵.
今天真的.
又在這裡又不在這裡.
在這裡還是不在這裡呢.
你在這裡就對了.
你在這裡就可以說完.
但我也想問一下.
其實現場也有很多人在.
你那個上帝在還是不在.
大家應該可以有很多聯想或者感受.
可以揮揮手或者同工會給咪高峰大家分享一下.
聽完之後.
你那四格漫畫當中.
你覺得你是哪一格多一點呢.
剛才我沒有聽錯.
剛才那四格應該都有你的表達空間.

$^{1601}$那個漫畫.
我第一眼看到就是覺得.
為什麼我們一定要用這個方式去看漫畫.
可以反過來看.
莫特曼他在《盼望神話》也說.
上帝看時間.
或者潘林博.
其實也是在終末看現在.
即使我們很多時候都在第二第三格中間的時候.
但其實今天.
我們所理解這個線性的時間.
其實在上帝的眼中.
原來這個現實是由終末所斷定.
潘林博說終末是決定現在.
我覺得也很啟發.
其實我們經常用過去去證明上帝.
我們想從昨天來試驗經歷.
但原來上帝是在未來奔跑過來.
我記得之前John也在港島看過.
好像是否龍珠.
其實上帝是未來奔跑過來.
可能在場和不在場之間.
就好像莫特曼所說.
其實上帝是來臨中的上帝.
歷史是上帝自我啟示.
這個不是一個已經預設上帝存在.
而是上帝存在中.
I am.
I am是一個過程.
一直會慢慢M到最後.
會爆炸性地.
原來我就是.
是一個過程.
所以我們今天理解的.
究竟有沒有上帝.
很多時候都是有或沒有.
有一點像Process Theology.
其實是一個過程.
我們站在時間點裡.
上帝已經啟示了自己.

$^{1641}$但他還沒有完全啟示自己.
所以我們很吊詭一時間.
不過我想可能我們一生.
就是徘徊在那四格裡.
一時我們覺得.
上帝是一個完全的他者.
我們在上帝面前.
我們很軟弱很卑微的時候.
我們在第一格.
我們是一個完全的他者.
可以被我們依賴.
但可能慢慢成長的時候.
又去到第四格.
我們原來是一個活在.
一個心下God is dead的時格裡.
我們是在上帝面前.
成為我自己所是.
上帝面前的一個被造.
純粹被罪人.
不過可能去到我們年老的時候.
又發覺原來有回歸到.
上帝造人吹口氣的時候.
又是真的有一個這麼實在的上帝.
在我衰老的時候.
可以被我完全依賴.
所以我想是一個很有趣的過程.
我們的觀眾越來越高集.
我覺得很好.
認同.
沒錯,是一個already but not yet的過程.
我們說耶穌主宰來.
是一個圓滿的同在.
將來我們跟祂同在.
今天我們的同在.
總是有些不是你那種.
這裡知道那種落差.
所以上次說要等待上帝.
等候是其中一種方式.
後面.
我就是你剛才所說的.

$^{1681}$曾經是你所說的那些靈恩派的那種同在.
OD.
我自己覺得靈恩派的那種同在.
當然現在會比較落地一點.
用一些所謂世界的眼光去解釋身邊的事.
但是靈恩派的那種同在.
其實我自己會覺得是一種.
理解聖經有很多超自然和神互動之間的.
那些經文的一些比較好的體驗.
那種體驗會讓我們更加明白到.
原來耶穌去打仗.
神會因為祂的禱告.
所以會讓那一格退一格.
會讓太陽倒轉走.
在我們的經歷裡面.
就會體驗到原來不是真的長期消失.
是真的有這樣的時候.
我就會覺得靈恩派的那種經歷.
其實可能會比較容易讓我們明白.
如何實際去理解聖經和活用聖經.
就不會太多停留在理論裡面.
就會在一個更加實際裡面.
但是最容易的就好像你剛才所說.
會走過龍.
好像會依賴了那種感覺.
所以我覺得無論是靈恩派.
或者是所謂靈恩派,方恩派.
其實我覺得是各自都可能是一個.
我自己理解就是大家面對著那隻大象.
大家看著鼻子和尾巴的時候.
其實各自都有一些好處.
和各自看到一個部份.
但其實是需要平衡的.
但是我會覺得靈恩派就是解釋不了.
為什麼剛才說馬可福音14章35節.
主啊主啊你為什麼離棄我.
在十字架上耶穌說這句話的時候.
其實回想以前是沒有解釋到.
為什麼上帝會離世.
但是剛才看回原來潘博華這個解釋.

$^{1721}$其實才是整個信仰的核心.
我就補充得到.
其實沒有了那種所謂神的同在的感覺的時候.
那個信仰生活的核心.
究竟是什麼樣的本質呢.
我就會覺得整個故事就完成了.
無論是很實際的,很近身的感覺.
一些活用真理的部份.
或者是一個很理論性的真理.
和一個比較貼地的信仰層面.
整個故事我就覺得可以組合在一起.
所以剛才聽完之後.
感覺就好像將這四格裡面.
就像剛才弟兄所說的.
其實是一個循環.
其實是一個完整的故事.
一個循環.
好,謝謝你.
多謝兩位弟兄的分享.
因為當初這一堂的時候.
我和John Chit Chat就是怕弟兄姐妹.
不是很明白想帶的訊息.
聽到兩位弟兄的分享也很好.
因為說得很好.
靈感派教會.
我自己的觀察和認識.
都是太過現象學.
就是從現象當中去主導.
然後就用自己解經的方法去套路.
讓人明白到那個現象是可以解釋的.
我覺得這件事有點太過了.
方法有時候太過實用.
很著重實用性.
覺得是否容易用一個做了一些事.
未必心安理得.
但就是要做一些事.
成為那件事覺得.
我在其中.
覺得那種感覺是很實在.
但你會看到今天的訊息.

$^{1761}$不是在說現象.
也不是在說要做些什麼.
其實現象如何.
上帝也在.
現象不在.
沒有的時候也不代表上帝不在.
但過程當中也不在乎我們做什麼.
和不做什麼.
這個過程是我們最難去接受和跳過的.
再說一點分享.
剛才弟兄說的那件事.
就是那四個漫畫的心態或者感受.
我自己最大的感受.
很多時候過了之後回想.
才感受到那個時刻其實上帝在.
我反而是過了那一刻才感受到.
在做的時候不覺.
因為真的用剛才John說的話.
我自己用我的熟悉.
或者我的想法.
或者我的能力去做.
按我自己的能力去做那件事.
即是切切實實做那件事.
但回想過後.
那一刻又感受到.
好像是上帝令我想到一些東西.
所以可能是過後才回想.
從一個綜合角度去看那件事.
發生過程當中上帝參與.
所以我自己的信仰經歷.
常常和弟兄姊妹分享.
你嘗試做多一點回想.
你常覺主因的滋味就更當如此.
回想那種喝茶的回甘.
回甘那一刻.
你就回答了.
曾經那口茶是過後那一刻.
其實感覺是很實在的.
好像很虛,但感覺就是這樣.
兩位先.

$^{1801}$剛才去解釋十字架和漫畫.
對我理解現在的社會狀況.
都是一個安慰.
但其實推到極致的時候.
這個信徒他自己一個.
徬彿神是和自己同在.
或者在心裡的狀態.
其實是否和沒有信仰的人.
都可以很相似呢?.
怎樣可以區分到自己.
我做的事是否真的沒有犯罪.
或者正在走在神裡.
你們怎樣看呢?.
我第一個反應是覺得.
其實應該要像外面的人.
用另一派的例子.
太不像就不要太當.
你應該做外面又像的人.
不過我想有些質素正正就是不同.
他們覺得你多了一份盼望.
盼望其實不是純粹盼望.
因為你有你的信仰.
有一種力量.
你相信耶穌都會幫助你.
這種盼望好像沒什麼特別.
但其實這就是最大的差別.
所以不會突然我會說方言.
你不會說.
或者我突然有什麼能夠聽到你聽不到.
其實大家都不知道.
大家都好像對前面很迷惘.
但為何我們付出這幾年.
或者很多時候信仰這幾年這麼重要呢?.
盼望這個話題.
或者良心這些話題.
或者堅持這些話題.
很多人覺得信仰的力量就在這裡.
那力量不是純粹心理的力量.
而是一種從聖靈而來的力量.
這正正就是差別.

$^{1841}$但差別不應該太過誇張.
我們完全好像聖人模式.
或者完全不同他們.
所以我覺得想想像的基督徒就是應該這樣.
因為及齡世界基督徒的彭富華.
在這樣的亂世裡面.
仍然有一些價值上是持守住.
但又不是.
對不起再說一次.
你想不到很靈的彭富華.
就是這個差別.
我的看法就是.
你剛才說的是.
信主和未信主的人的差異.
我仍然相信聖靈的參與.
因為信了主之後.
聖經說得很清楚.
就是聖靈就內在我們心裡.
所以聖靈是一個會對我們說話的上帝.
祂會做一些啟發.
讓我們感受到那件事.
是否合乎聖經或耶穌教導的事情.
反而你說未信主的人.
他們是否會沒有這個觸動呢?.
我覺得未必完全沒有.
因為普遍恩典裡面都說的訊息就是.
保羅在羅馬書裡說的良知.
上帝會啟發人的良知去做分辨.
所以在過程當中你會見到.
如果沒有信主的人.
他的做事可能比我們信主的人更好.
這就是回應John所說的.
基督徒應該會有更多的改變.
但有時基督徒對信仰都沒有太大改變.
他只不過是接收了信仰.
我是一個基督徒.
但其實他接收了宗教.
而不是接收了耶穌的改變.
所以那個身份而有的能力.
應該是聖靈的改變.

$^{1881}$最不想的就是多了一份宗教習慣.
我會純粹犯善祈禱.
或者會不去教會.
或者去教會.
這些是宗教的練習.
但我想你應該有更加重要的東西在裡面.
潘福華寫了一首詩.
非宗教和宗教人.
大家都是需要一些很普通的東西.
麵包,平安,健康.
都要求主憐憫.
但唯有信主的人.
會為基督耶穌而受苦.
所以不是問你能夠得到甚麼.
而是你會做甚麼.
你會為主耶穌和他一起去受苦.
這就是差別.
我覺得另一種方法去理解這四格密碼.
就是因為神的不存在才凸顯了傳說.
就好像你每天都很開心.
那種就不叫開心.
是要有不開心才能感到開心的感覺.
所以再套用這四格密碼.
就是因為有第四格神的不存在.
才凸顯第一格神的存在.
很開心今天聽到兩位講這四格的密碼.
我的經歷就真的好像這四格密碼.
年輕的時候可能會依賴很多教會給你的訊息.
或者你已經調到宗教的模式.
你會有些依賴.
覺得你會開心一點.
但當你隨著年紀慢慢增長的時候.
你會發覺有些價值可能會和周圍的東西碰撞.
我舉個例子.
我以前總是舉例子.
好像你的車壞了.
其實有幾個方法去處理.
如果你懂得做就自己做.
你不懂就叫人做.
叫人來修.

$^{1921}$如果你有信心.
你按手在車上.
奉主耶穌基督的命令.
讓它可以重開.
其實很多東西是可以.
好像是一切.
但互相矛盾.
我覺得在我信仰的歷程裡.
我看到的教會.
無論是靈恩,灰恩派.
傳統的教會.
其實最大的問題在哪裡呢.
剛才所說的宗教問題.
其實源自於自我迷信.
我不知道各位.
我年紀比較大.
我出年也等老了.
曾經有些電視劇.
《無名無姓》.
很久以前.
模擬了一些基督徒.
突然打地鐵衝進門口.
還是你口頭禪說出來呢.
我覺得我們最大的問題就是.
在傳統的教會裡.
一直以來.
都是這樣的迷信.
我一直在人生裡.
其實我現在已經去到第四格.
我不再聽傳統教會說話.
我會用自己的經歷去感受神.
我的存在在哪裡.
以前辛苦的地方是.
當你有這種意識的時候.
在傳統教會.
我以前很喜歡說.
我自己變成一個.
我是一個異教徒.
在一份半教的基督徒裡.
我的模式就是這樣.

$^{1961}$是很辛苦的.
當然你要靠自己去思索.
去想一些問題.
我剛才也回過一些靈驗派的小組.
很簡單.
我感受到他們去山上祈禱.
叫得很大聲.
說他們所謂的方言.
然後去感受到聖靈.
其實我自己的邏輯很簡單.
如果聖靈是真的.
為什麼他們感受得到.
我感受不到.
一個簡單的邏輯.
就算我不明白他們說什麼.
如果聖靈真的在那裡.
我應該感受得到.
甚至我見過他們有些.
很奇怪.
和黃大仙沒有什麼分別.
他們有一個叫.
我不知道你們知不知道.
有一個app叫做AJAR.
靈驗派很喜歡用.
想不到什麼問題的時候.
他就揭開那個.
好像通勝一樣.
就指出今天他說的話.
是,我真的很中.
他就會去做.
我想說的是.
我們有很多痛苦.
是源自於我們的自我迷信.
不只是一個宗教的框框條條.
而是我們有沒有去.
開放我們自己去接受一些新的東西.
包括.
好像早前在互聯網說得很流行.
有些人說.
耶穌年輕的時候.

$^{2001}$其實去了印度學佛法.
我們有沒有一個批判的精神去想.
和去接受.
或者在這件事上去看.
究竟哪些東西是真哪些東西是假.
或者就算所有人都說是假.
如果你憑你自己的感受覺得是真的.
你會不會願意去.
去堅持.
好像今天福音教會那樣.
我相信一樣是真的.
但所有人都說是假.
甚至我是異端.
我是否繼續願意走下去呢.
我覺得當今基督徒.
當代基督徒很重要的一點就是要懂得思考.
不要用屁股去想.
這是一個我經常要說的東西.
包括香港教育局那時.
很多東西都是我們沒有了自我批判.
沒有了自我解除迷信的方法.
思考的方法.
回應剛才展會所說.
其實他提出的問題原意的核心在哪裡呢.
我們被傳統的教會誣陷了一樣東西.
基督徒和普通人應該有分別.
我們是分別成性的.
所以我們做的事情和他們一樣.
那就有問題.
邏輯其實就是這樣.
但你可以反過來看.
信仰做人本來給了一個.
創世紀元說了什麼.
給了我們一個管理萬物的能力.
這個能力是普遍的.
基督徒也有.
我們不要害怕一件事.
傳統教會告訴我們要分別為性.
其實我們要做的事情和他們不一樣.
其實很多東西應該是一樣的.

$^{2041}$正如剛才所說的良知.
是一樣的東西.
我覺得我們要在今天這一堂課.
除了剛才所說的宗教框框條條.
我希望大家抱著一個開放的心.
不要自我迷信.
你可以自我質疑自己.
去考慮很多事情.
可能你會看的路不一樣.
這四部漫畫其實就是在說這件事.
這是我今天要和大家分享的東西.
謝謝大家分享.
聽大家分享的時候.
我發現今天所說的原來大家都明白.
也有共鳴.
希望大家都能夠跨到成熟的地步.
我覺得這是成熟.
不是第一部漫畫不重要.
而是你起初的時候很需要.
這樣去理解.
但沒有了就不代表.
這不是我們的信仰.
所以不是否定第一部漫畫.
大家起初的時候不用教.
反而更加強調.
有時面對這樣的情況.
我們怎樣能夠知道上帝同在.
或是怎樣去理解呢.
最後給個牌子.
後面.
我想大家都明白.
但我覺得第一位弟兄和姊妹都說得很好.
我回想起來想問.
我想說我想說的東西.
為何有人說上帝不在場呢.
其實在這四部漫畫裡.
在第三部漫畫.
那個人就開始忘了上帝不在.
就是那一部才會問.
為何我聽到很多人.

$^{2081}$不論信不信.
說為何這時間上帝不在.
不在的我就不相信.
很多時候就是因為第三部.
第三部已經很堅定.
我看不到他也在.
我就跳下去沒問題.
如果你能夠提出這個題目.
我覺得第三部是重要的.
我明明學到的東西.
我知道聖靈應該這樣做.
但世界不是這樣.
怎麼辦呢.
我相信19年大家都經歷了很多.
20年之後都經歷了很多.
甚至聖經也有很多例子.
我想起.
想也想到幾個.
阿伯拉罕無端端叫.
阿伯拉罕獻了以撒出來.
然後士司祭.
不是士司祭.
總之就是有一段時間.
我去到加拿大.
我要殺光全地所有的人.
一個都不可以.
小朋友都不可以.
有慈悲之心留一個都會出事.
當你和你.
上帝要你.
那時候是上帝擺明要你做.
你都不可以質疑上帝不在.
現在是你靠感覺的.
靠感動的.
靠你自己對神的認知.
當你世界出現的東西.
和你所想像的神不同的時候.
你怎樣去自處呢.
我相信.
但正常的想法就是.

$^{2121}$你等吧.
上帝還沒做事.
不在場的上帝.
其實那些人問.
不在場其實是上帝沒做事.
或者沒做他覺得上帝應該要做的事.
當不做事的時候.
我們怎樣呢.
我說等.
等出來的結果不是.
那又怎樣呢.
我之前.
有時候會聽.
土青年圖老師.
他們經常說基督教.
他們說.
如果上帝真的在場.
你做出來的人應該是這樣.
他也是說這件事.
他心目中的上帝應該要做這些事.
但這個世界不是這樣.
我們怎樣去處理這個問題呢.
你說等.
很可能最後就是.
看到結果就安慰自己.
如果結果是好的.
那還算是自圓其說.
但如果不是呢.
其實很多人就是因為這樣.
就證明瞭我不在場.
沒有.
假的.
我覺得是.
我們相信的.
我們會有第四教出現的.
我們很堅定的相信上帝在場.
我們就走下去.
但怎樣去處理這個.
明明我覺得上帝應該要這樣做.
這些事是不跟隨聖經教導的.

$^{2161}$但依然在場的情況.
可能你回去再聽一聽.
今天想說的東西.
其實值得再想多一點.
其實正正在說第四教的情況下.
我們仍然是OK的.
剛才Boofer那段話.
值得我們再思考.
正正在十字架上.
受苦的上帝.
才是上帝.
剛才你所說.
看著他心意想像的上帝.
這只是宗教.
宗教就是這樣.
宗教不一定是迷信.
拜黃大仙那些.
這幾年裡.
很多現代人都有自己的宗教上帝.
上帝一定要是我那種.
黃的上帝.
民主,資源普選.
都是一種宗教.
所以上帝只是超過我們所說的.
一切世界的解救.
明白嗎.
一切的解救.
在其他世界裡.
正正是受苦的上帝.
才是出路.
是一個軟弱的上帝.
無能的上帝.
在這幾年裡.
香港的情況下.
是一個無能的上帝.
才是一個跟人同苦的上帝.
這句話很有深度.
正正不是要求那種.
救到或應驗到某些東西.
結局並不是說.

$^{2201}$Yes 第四格裡沒有了.
我想說這件事.
所以結局不是.
能夠應驗或等到.
而是真真正正知道.
第四格裡所謂不存在.
也是我們的上帝.
所以我想說.
那個 absence.
不是一個 code and code absence.
真是可以 absence.
所謂的 absence.
是你那種存在的方法.
仍然是我們所信的上帝.
所以潘博雅說.
今天那個同在的上帝.
是一個離棄的上帝.
這句話好像很吊詭.
但正正是我們今天可以去想的那種.
很多人因為這樣.
所以去宗教的時候就走了.
因為他想要的是這樣.
我回應一下.
剛才說的第三格.
可能他很 puzzle 的狀況.
我自己通常都會喜歡用.
信仰是一個經驗學習的過程.
經驗學習當中.
就視乎大家的經歷.
你是有什麼取態.
有些人很著重經歷當中的感受.
有些人很著重經歷當中的批判.
我就覺得.
信仰在一個經驗學習過程當中.
你的感受和批判都是一致的.
因為在悟性和感性當中.
你怎樣去調整.
其實就是你怎樣去認識自己.
怎樣去認識這個信仰.
其實不是很複雜.

$^{2241}$其實就是過程當中.
如果你沒有了.
你只用感性.
你沒有批判的話.
你很容易就跌進迷信.
別人跟著做.
你就做得很漂亮.
跟著練習去做.
從感性當中.
剛才用了幾個例子.
打開聖經.
我以前帶青年小組的時候.
都有人告訴我.
那天我剛開始打開那頁.
就是在說.
是就這樣成了.
他就覺得是就這樣成了.
我就說.
我開第一頁就是.
後來他死了怎麼辦.
這個感受上.
就是按你自己需要的話.
剛開那頁靈修.
他覺得是靈修的時候.
看了那頁.
他就覺得那件事.
他就承諾上帝給他.
我說這些感受.
不是很批判性.
或者不是很理性去衡量那件事.
所以如果說第三格那格.
就好像上帝不在.
他很不熟的話.
我覺得就是我們在這幾年的過程.
或過去信仰過程當中.
你遇到要做抉擇.
你遇到要做一些分辨的時候.
你就想一下.
你平時信仰的建構的時候.
你是著重理性,感性.

$^{2281}$還是你真的複雜了.
所以剛才值得回去聽聽.
今天的內容.
John在後面有兩個連結.
第一個就是.
講關於祈禱與行動.
我們相信上帝的意旨與人的回應.
或者上帝的意旨與人的行動.
第二就是講如何明白上帝的旨意.
你用感性去明白上帝的旨意.
還是看上帝在我們當中的互動.
或者回想上帝在我們生命當中.
參與的過程.
你如何作出判斷呢.
我覺得這件事比較實在.
不要說實在.
你會慢慢看到自己如何建立你的信仰.
或者我講得比較快一點.
我覺得這幅圖很好幫助大家去思想.
去想想和神的關係.
或者神的存在.
但我覺得這幅圖也有一點.
是很相似的.
但同一時間.
那四幅圖的主題是.
My walk with God.
所以是那個人對神的感覺.
但如果你看這幅圖的時候.
你會看到是一個神?三個神?.
還是你真的看到四個神?.
好像剛才John一開始已經說了.
The presence of God.
只不過是Presence.
不是我們預期的Presence.
而不是神真的有一個時刻.
是Absent了.
神是Always there.
只不過在這四幅圖裡面.
主題的出發點是.
由我們自己的個人出發.

$^{2321}$我們如何在第三幅圖或第四幅圖.
acknowledge神的Presence.
才是我們如何去.
作為一個成熟的基督徒.
應該可以達到目標.
我想說我們以前聽到.
有些很悶.
有些不悶.
有些是奇蹟牧師.
不知道有沒有聽過.
很多神蹟奇事的故事.
告訴你.
在建堂的時候欠六千元.
然後就祈禱.
然後發現錢是六千元.
這些奉獻建堂神蹟的故事.
我想說其實很多都不是完全對的.
不是整個Fashionable.
背後很多東西.
其實他沒有告訴你.
收拾過.
說的好像很奇妙.
我們聽了太多這些.
其實初期是可以的.
但聽了太多.
你會發現一來不是那麼真.
反而有反效果.
信仰變成只有這些.
所以我們看世界.
不是靠神蹟的故事.
我不是不相信這些故事.
而是不能只靠這些故事信仰.
你知道背後很多東西.
其實都是.
上次這樣說.
原來有借錢的.
原來有怎樣的.
有些努力都做出來的.
我們就需要去瞭解這方面.
那個Presence.

$^{2361}$就是要包括在這裡.
不只是那種神蹟的Presence.
這個我自己很重視.
我都聽過這些神蹟.
但我從來都沒有經歷過.
我是那些經常送車的人.
就算App說還有兩分鐘.
到我去到看到車子就走.
我是很實幹的人.
所以我都很清楚.
我自己的能力去到什麼就做什麼.
不過四格漫畫.
或者剛才說到My Wall with God.
這個感受.
我自己覺得.
可能例子未必是最近的.
但我想你能夠瞭解.
如果你爸爸或者家人.
帶你去樂園.
主題樂園玩.
你未必一定要你家人.
長期陪你坐Ride.
可能你自己進去坐.
他在外面等你.
那你就自己坐.
或者他整個樂園.
他說你去玩吧.
我在哪裡等你.
那你就自己去排Fast Pass.
什麼都好.
吃飯的時候才會和他一起吃.
我覺得上帝好像放在我們樂園.
或者在我們現在生活的環境當中.
你未必每次都感受到.
上帝真的在你旁邊.
但上帝事實上就在我們旁邊.
這些可能你會用你的方法去產生類比.
但我覺得大家都是討論一個方向.
就是你相信上帝會在其中.
只不過在不同的感受當中.

$^{2401}$你敢不敢受到祂的存在.
但這個我不可以說跳到滑.
這個就是你的信心.
我不是想這樣說.
不過你的信仰在經歷過程當中.
你就會慢慢得到你自己所信的上帝.
其實我自己覺得你是經歷過程當中.
你會有感受.
有情感.
你會有批判.
有分辨.
這個信仰是真實的.
盡可能都是想拿開一些太過宗教性的行為.
去主導了.
好像做了這些會心安理得.
做了這些就好像會靈一點.
做了這些就好像上帝會喜歡一點.
我覺得這個外衣其實不應該是我們要守著的東西.
接下來我們七八月放暑假.
真的飛了一次.
所以我們七八月就會放暑假.
放暑假.
第六課就九月.
就會有我們第六課.
大家都可以一起去放暑假.
OK 遲些見.
OK 拜拜.
拜拜.
\newpage



\section{}
\label{sec:2QyWxsVtL8E}
\textbf{《致餘民及流散者:給香港基督徒的神學八課》第二季第6課|20230924 [2QyWxsVtL8E]}
\newline
\newline
連結: \href{https://youtube.com/watch?v=2QyWxsVtL8E}{\texttt{ https://youtube.com/watch?v=2QyWxsVtL8E}} ~~~~ 語音日期: 2023-09-24 
\newline
\newline
\hyperref[sec:5EgvGimlwXk]{\small{< < < PREV SERMON < < <}}
~
\hyperref[sec:index_chronic]{\small{[返順時目]}}
~
\hyperref[sec:index_scriptual]{\small{[返順卷目]}}
~
\hyperref[sec:JxHW7ujVbSI]{\small{> > > NEXT SERMON > > >}}
\newline
\newline
$^{1}$我只想知道.
你到底是什麼意思.
我只想知道.
你到底是什麼意思.
我只想知道.
你到底是什麼意思.
我只想知道.
你到底是什麼意思.
我只想知道.
你到底是什麼意思.
我只想知道.
你到底是什麼意思.
我只想知道.
你到底是什麼意思.
我只想知道.
你到底是什麼意思.
我只想知道.
你到底是什麼意思.
我只想知道.
你到底是什麼意思.
我只想知道.
你到底是什麼意思.
我只想知道.
你到底是什麼意思.
我只想知道.
你到底是什麼意思.
我只想知道.
你到底是什麼意思.
我只想知道.
你到底是什麼意思.
我只想知道.
你到底是什麼意思.
我只想知道.
你到底是什麼意思.
我只想知道.
你到底是什麼意思.
我只想知道.
你到底是什麼意思.
我只想知道.
你到底是什麼意思.

$^{41}$我只想知道.
你到底是什麼意思.
我只想知道.
你到底是什麼意思.
我只想知道.
你到底是什麼意思.
我只想知道.
你到底是什麼意思.
我只想知道.
你到底是什麼意思.
我只想知道.
你到底是什麼意思.
我只想知道.
你到底是什麼意思.
我只想知道.
你到底是什麼意思.
我只想知道.
你到底是什麼意思.
我只想知道.
你到底是什麼意思.
我只想知道.
你到底是什麼意思.
我只想知道.
你到底是什麼意思.
我只想知道.
你到底是什麼意思.
我只想知道.
你到底是什麼意思.
我只想知道.
你到底是什麼意思.
我只想知道.
你到底是什麼意思.
我只想知道.
你到底是什麼意思.
我只想知道.
你到底是什麼意思.
我只想知道.
你到底是什麼意思.
我只想知道.
你到底是什麼意思.

$^{81}$我只想知道.
你到底是什麼意思.
我只想知道.
你到底是什麼意思.
我只想知道.
你到底是什麼意思.
我只想知道.
你到底是什麼意思.
我只想知道.
你到底是什麼意思.
我只想知道.
你到底是什麼意思.
我只想知道.
你到底是什麼意思.
我只想知道.
你到底是什麼意思.
我只想知道.
你到底是什麼意思.
我只想知道.
你到底是什麼意思.
我只想知道.
你到底是什麼意思.
我只想知道.
你到底是什麼意思.
我只想知道.
你到底是什麼意思.
我只想知道.
你到底是什麼意思.
我只想知道.
你到底是什麼意思.
我只想知道.
你到底是什麼意思.
我只想知道.
你到底是什麼意思.
我只想知道.
你到底是什麼意思.
我只想知道.
你到底是什麼意思.
我只想知道.
你到底是什麼意思.

$^{121}$我只想知道.
你到底是什麼意思.
我只想知道.
你到底是什麼意思.
我只想知道.
你到底是什麼意思.
我只想知道.
你到底是什麼意思.
我只想知道.
你到底是什麼意思.
我只想知道.
你到底是什麼意思.
我只想知道.
你到底是什麼意思.
我只想知道.
你到底是什麼意思.
我只想知道.
你到底是什麼意思.
我只想知道.
你到底是什麼意思.
我只想知道.
你到底是什麼意思.
我只想知道.
你到底是什麼意思.
我只想知道.
你到底是什麼意思.
我只想知道.
你到底是什麼意思.
我只想知道.
你到底是什麼意思.
我只想知道.
你到底是什麼意思.
我只想知道.
你到底是什麼意思.
我只想知道.
你到底是什麼意思.
我只想知道.
你到底是什麼意思.
我只想知道.
你到底是什麼意思.

$^{161}$我只想知道.
你到底是什麼意思.
我只想知道.
你到底是什麼意思.
我只想知道.
你到底是什麼意思.
我只想知道.
你到底是什麼意思.
我只想知道.
你到底是什麼意思.
我只想知道.
你到底是什麼意思.
我只想知道.
你到底是什麼意思.
我只想知道.
你到底是什麼意思.
我只想知道.
你到底是什麼意思.
我只想知道.
你到底是什麼意思.
我只想知道.
你到底是什麼意思.
我只想知道.
你到底是什麼意思.
我只想知道.
你到底是什麼意思.
我只想知道.
你到底是什麼意思.
我只想知道.
你到底是什麼意思.
我只想知道.
你到底是什麼意思.
我只想知道.
你到底是什麼意思.
我只想知道.
你到底是什麼意思.
我只想知道.
你到底是什麼意思.
我只想知道.
你到底是什麼意思.

$^{201}$我只想知道.
你到底是什麼意思.
我只想知道.
你到底是什麼意思.
我只想知道.
你到底是什麼意思.
我只想知道.
你到底是什麼意思.
我只想知道.
你到底是什麼意思.
我只想知道.
你到底是什麼意思.
我只想知道.
你到底是什麼意思.
我只想知道.
你到底是什麼意思.
我只想知道.
你到底是什麼意思.
我只想知道.
你到底是什麼意思.
我只想知道.
你到底是什麼意思.
我只想知道.
你到底是什麼意思.
我只想知道.
你到底是什麼意思.
我只想知道.
你到底是什麼意思.
我只想知道.
你到底是什麼意思.
我只想知道.
你到底是什麼意思.
我只想知道.
你到底是什麼意思.
我只想知道.
你到底是什麼意思.
我只想知道.
你到底是什麼意思.
我只想知道.
你到底是什麼意思.

$^{241}$我只想知道.
你到底是什麼意思.
我只想知道.
你到底是什麼意思.
我只想知道.
你到底是什麼意思.
我只想知道.
你到底是什麼意思.
我只想知道.
你到底是什麼意思.
我只想知道.
你到底是什麼意思.
我只想知道.
你到底是什麼意思.
我只想知道.
你到底是什麼意思.
我只想知道.
你到底是什麼意思.
我只想知道.
你到底是什麼意思.
我只想知道.
你到底是什麼意思.
我只想知道.
你到底是什麼意思.
我只想知道.
你到底是什麼意思.
我只想知道.
你到底是什麼意思.
我只想知道.
你到底是什麼意思.
我只想知道.
你到底是什麼意思.
我只想知道.
你到底是什麼意思.
我只想知道.
你到底是什麼意思.
我只想知道.
你到底是什麼意思.
我只想知道.
你到底是什麼意思.

$^{281}$我只想知道.
你到底是什麼意思.
我只想知道.
你到底是什麼意思.
我只想知道.
你到底是什麼意思.
我只想知道.
你到底是什麼意思.
我只想知道.
你到底是什麼意思.
我只想知道.
你到底是什麼意思.
我只想知道.
你到底是什麼意思.
我只想知道.
你到底是什麼意思.
我只想知道.
你到底是什麼意思.
我只想知道.
你到底是什麼意思.
我只想知道.
你到底是什麼意思.
我只想知道.
你到底是什麼意思.
我只想知道.
你到底是什麼意思.
我只想知道.
你到底是什麼意思.
我只想知道.
你到底是什麼意思.
我只想知道.
你到底是什麼意思.
我只想知道.
你到底是什麼意思.
我只想知道.
你到底是什麼意思.
我只想知道.
你到底是什麼意思.
我只想知道.
你到底是什麼意思.

$^{321}$我只想知道.
你到底是什麼意思.
我只想知道.
你到底是什麼意思.
我只想知道.
你到底是什麼意思.
我只想知道.
你到底是什麼意思.
我只想知道.
你到底是什麼意思.
我只想知道.
你到底是什麼意思.
我只想知道.
你到底是什麼意思.
我只想知道.
你到底是什麼意思.
我只想知道.
你到底是什麼意思.
我只想知道.
你到底是什麼意思.
我只想知道.
你到底是什麼意思.
我只想知道.
你到底是什麼意思.
我只想知道.
你到底是什麼意思.
我只想知道.
你到底是什麼意思.
我只想知道.
你到底是什麼意思.
我只想知道.
你到底是什麼意思.
我只想知道.
你到底是什麼意思.
我只想知道.
你到底是什麼意思.
我只想知道.
你到底是什麼意思.
我只想知道.
你到底是什麼意思.

$^{361}$我只想知道.
你到底是什麼意思.
我只想知道.
你到底是什麼意思.
我只想知道.
你到底是什麼意思.
我只想知道.
你到底是什麼意思.
我只想知道.
你到底是什麼意思.
我只想知道.
你到底是什麼意思.
我只想知道.
你到底是什麼意思.
我只想知道.
你到底是什麼意思.
我只想知道.
你到底是什麼意思.
我只想知道.
你到底是什麼意思.
我只想知道.
你到底是什麼意思.
我只想知道.
你到底是什麼意思.
我只想知道.
你到底是什麼意思.
我只想知道.
你到底是什麼意思.
我只想知道.
你到底是什麼意思.
我只想知道.
你到底是什麼意思.
我只想知道.
你到底是什麼意思.
我只想知道.
你到底是什麼意思.
我只想知道.
你到底是什麼意思.
我只想知道.
你到底是什麼意思.

$^{401}$我只想知道.
你到底是什麼意思.
我只想知道.
你到底是什麼意思.
我只想知道.
你到底是什麼意思.
我只想知道.
你到底是什麼意思.
我只想知道.
你到底是什麼意思.
我只想知道.
你到底是什麼意思.
我只想知道.
你到底是什麼意思.
我只想知道.
你到底是什麼意思.
我只想知道.
你到底是什麼意思.
我只想知道.
你到底是什麼意思.
我只想知道.
你到底是什麼意思.
我只想知道.
你到底是什麼意思.
我只想知道.
你到底是什麼意思.
我只想知道.
你到底是什麼意思.
我只想知道.
你到底是什麼意思.
我只想知道.
你到底是什麼意思.
我只想知道.
你到底是什麼意思.
我只想知道.
你到底是什麼意思.
我只想知道.
你到底是什麼意思.
我只想知道.
你到底是什麼意思.

$^{441}$我只想知道.
你到底是什麼意思.
我只想知道.
你到底是什麼意思.
我只想知道.
你到底是什麼意思.
我只想知道.
你到底是什麼意思.
我只想知道.
你到底是什麼意思.
我只想知道.
你到底是什麼意思.
我只想知道.
你到底是什麼意思.
我只想知道.
你到底是什麼意思.
我只想知道.
你到底是什麼意思.
我只想知道.
你到底是什麼意思.
我只想知道.
你到底是什麼意思.
我只想知道.
你到底是什麼意思.
我只想知道.
你到底是什麼意思.
我只想知道.
你到底是什麼意思.
我只想知道.
你到底是什麼意思.
我只想知道.
你到底是什麼意思.
我只想知道.
你到底是什麼意思.
我只想知道.
你到底是什麼意思.
我只想知道.
你到底是什麼意思.
我只想知道.
你到底是什麼意思.

$^{481}$我只想知道.
你到底是什麼意思.
我只想知道.
你到底是什麼意思.
我只想知道.
你到底是什麼意思.
我只想知道.
你到底是什麼意思.
我只想知道.
你到底是什麼意思.
我只想知道.
你到底是什麼意思.
我只想知道.
你到底是什麼意思.
我只想知道.
你到底是什麼意思.
我只想知道.
你到底是什麼意思.
我只想知道.
你到底是什麼意思.
我只想知道.
你到底是什麼意思.
我只想知道.
你到底是什麼意思.
我只想知道.
你到底是什麼意思.
我只想知道.
你到底是什麼意思.
我只想知道.
你到底是什麼意思.
我只想知道.
你到底是什麼意思.
我只想知道.
你到底是什麼意思.
我只想知道.
你到底是什麼意思.
我只想知道.
你到底是什麼意思.
我只想知道.
你到底是什麼意思.

$^{521}$我只想知道.
你到底是什麼意思.
我只想知道.
你到底是什麼意思.
我只想知道.
你到底是什麼意思.
我只想知道.
你到底是什麼意思.
我只想知道.
你到底是什麼意思.
我只想知道.
你到底是什麼意思.
我只想知道.
你到底是什麼意思.
我只想知道.
你到底是什麼意思.
我只想知道.
你到底是什麼意思.
我只想知道.
你到底是什麼意思.
我只想知道.
你到底是什麼意思.
我只想知道.
你到底是什麼意思.
我只想知道.
你到底是什麼意思.
我只想知道.
你到底是什麼意思.
我只想知道.
你到底是什麼意思.
我只想知道.
你到底是什麼意思.
我只想知道.
你到底是什麼意思.
我只想知道.
你到底是什麼意思.
我只想知道.
你到底是什麼意思.
我只想知道.
你到底是什麼意思.

$^{561}$我只想知道.
你到底是什麼意思.
我只想知道.
你到底是什麼意思.
我只想知道.
你到底是什麼意思.
我只想知道.
你到底是什麼意思.
我只想知道.
你到底是什麼意思.
我只想知道.
你到底是什麼意思.
我只想知道.
你到底是什麼意思.
我只想知道.
你到底是什麼意思.
我只想知道.
你到底是什麼意思.
我只想知道.
你到底是什麼意思.
我只想知道.
你到底是什麼意思.
我只想知道.
你到底是什麼意思.
我只想知道.
你到底是什麼意思.
我只想知道.
你到底是什麼意思.
我只想知道.
你到底是什麼意思.
我只想知道.
你到底是什麼意思.
我只想知道.
你到底是什麼意思.
我只想知道.
你到底是什麼意思.
我只想知道.
你到底是什麼意思.
我只想知道.
你到底是什麼意思.

$^{601}$我只想知道.
你到底是什麼意思.
我只想知道.
你到底是什麼意思.
我只想知道.
你到底是什麼意思.
我只想知道.
你到底是什麼意思.
我只想知道.
你到底是什麼意思.
我只想知道.
你到底是什麼意思.
我只想知道.
你到底是什麼意思.
我只想知道.
你到底是什麼意思.
我只想知道.
你到底是什麼意思.
我只想知道.
你到底是什麼意思.
我只想知道.
你到底是什麼意思.
我只想知道.
你到底是什麼意思.
我只想知道.
你到底是什麼意思.
我只想知道.
你到底是什麼意思.
我只想知道.
你到底是什麼意思.
我只想知道.
你到底是什麼意思.
我只想知道.
你到底是什麼意思.
我只想知道.
你到底是什麼意思.
我只想知道.
你到底是什麼意思.
我只想知道.
你到底是什麼意思.

$^{641}$我只想知道.
你到底是什麼意思.
我只想知道.
你到底是什麼意思.
我只想知道.
你到底是什麼意思.
我只想知道.
你到底是什麼意思.
我只想知道.
你到底是什麼意思.
我只想知道.
你到底是什麼意思.
我只想知道.
你到底是什麼意思.
我只想知道.
你到底是什麼意思.
我只想知道.
你到底是什麼意思.
我只想知道.
你到底是什麼意思.
我只想知道.
你到底是什麼意思.
我只想知道.
你到底是什麼意思.
我只想知道.
你到底是什麼意思.
我只想知道.
你到底是什麼意思.
我只想知道.
你到底是什麼意思.
我只想知道.
你到底是什麼意思.
我只想知道.
你到底是什麼意思.
我只想知道.
你到底是什麼意思.
我只想知道.
你到底是什麼意思.
我只想知道.
你到底是什麼意思.

$^{681}$我只想知道.
你到底是什麼意思.
我只想知道.
你到底是什麼意思.
我只想知道.
你到底是什麼意思.
我只想知道.
你到底是什麼意思.
我只想知道.
你到底是什麼意思.
我只想知道.
你到底是什麼意思.
我只想知道.
你到底是什麼意思.
我只想知道.
你到底是什麼意思.
我只想知道.
你到底是什麼意思.
我只想知道.
你到底是什麼意思.
我只想知道.
你到底是什麼意思.
我只想知道.
你到底是什麼意思.
我只想知道.
你到底是什麼意思.
我只想知道.
你到底是什麼意思.
我只想知道.
你到底是什麼意思.
我只想知道.
你到底是什麼意思.
我只想知道.
你到底是什麼意思.
我只想知道.
你到底是什麼意思.
我只想知道.
你到底是什麼意思.
我只想知道.
你到底是什麼意思.

$^{721}$我只想知道.
你到底是什麼意思.
我只想知道.
你到底是什麼意思.
我只想知道.
你到底是什麼意思.
我只想知道.
你到底是什麼意思.
我只想知道.
你到底是什麼意思.
我只想知道.
你到底是什麼意思.
我只想知道.
你到底是什麼意思.
我只想知道.
你到底是什麼意思.
我只想知道.
你到底是什麼意思.
我只想知道.
你到底是什麼意思.
我只想知道.
你到底是什麼意思.
我只想知道.
你到底是什麼意思.
我只想知道.
你到底是什麼意思.
我只想知道.
你到底是什麼意思.
我只想知道.
你到底是什麼意思.
我只想知道.
你到底是什麼意思.
我只想知道.
你到底是什麼意思.
我只想知道.
你到底是什麼意思.
我只想知道.
你到底是什麼意思.
我只想知道.
你到底是什麼意思.

$^{761}$我只想知道.
你到底是什麼意思.
我只想知道.
你到底是什麼意思.
我只想知道.
你到底是什麼意思.
我只想知道.
你到底是什麼意思.
我只想知道.
你到底是什麼意思.
我只想知道.
你到底是什麼意思.
我只想知道.
你到底是什麼意思.
我只想知道.
你到底是什麼意思.
我只想知道.
你到底是什麼意思.
我只想知道.
你到底是什麼意思.
我只想知道.
你到底是什麼意思.
我只想知道.
你到底是什麼意思.
我只想知道.
你到底是什麼意思.
我只想知道.
你到底是什麼意思.
我只想知道.
你到底是什麼意思.
我只想知道.
你到底是什麼意思.
我只想知道.
你到底是什麼意思.
我只想知道.
你到底是什麼意思.
我只想知道.
你到底是什麼意思.
我只想知道.
你到底是什麼意思.

$^{801}$我只想知道.
你到底是什麼意思.
我只想知道.
你到底是什麼意思.
我只想知道.
你到底是什麼意思.
我只想知道.
你到底是什麼意思.
我只想知道.
你到底是什麼意思.
我只想知道.
你到底是什麼意思.
我只想知道.
你到底是什麼意思.
我只想知道.
你到底是什麼意思.
我只想知道.
你到底是什麼意思.
我只想知道.
你到底是什麼意思.
我只想知道.
你到底是什麼意思.
我只想知道.
你到底是什麼意思.
我只想知道.
你到底是什麼意思.
我只想知道.
你到底是什麼意思.
我只想知道.
你到底是什麼意思.
我只想知道.
你到底是什麼意思.
我只想知道.
你到底是什麼意思.
我只想知道.
你到底是什麼意思.
我只想知道.
你到底是什麼意思.
我只想知道.
你到底是什麼意思.

$^{841}$我只想知道.
你到底是什麼意思.
我只想知道.
你到底是什麼意思.
我只想知道.
你到底是什麼意思.
我只想知道.
你到底是什麼意思.
我只想知道.
你到底是什麼意思.
我只想知道.
你到底是什麼意思.
我只想知道.
你到底是什麼意思.
我只想知道.
你到底是什麼意思.
我只想知道.
你到底是什麼意思.
我只想知道.
你到底是什麼意思.
我只想知道.
你到底是什麼意思.
我只想知道.
你到底是什麼意思.
我只想知道.
你到底是什麼意思.
我只想知道.
你到底是什麼意思.
我只想知道.
你到底是什麼意思.
我只想知道.
你到底是什麼意思.
我只想知道.
你到底是什麼意思.
我只想知道.
你到底是什麼意思.
我只想知道.
你到底是什麼意思.
我只想知道.
你到底是什麼意思.

$^{881}$我只想知道.
你到底是什麼意思.
我只想知道.
你到底是什麼意思.
我只想知道.
你到底是什麼意思.
我只想知道.
你到底是什麼意思.
我只想知道.
你到底是什麼意思.
我只想知道.
你到底是什麼意思.
我只想知道.
你到底是什麼意思.
我只想知道.
你到底是什麼意思.
我只想知道.
你到底是什麼意思.
我只想知道.
你到底是什麼意思.
我只想知道.
你到底是什麼意思.
我只想知道.
你到底是什麼意思.
我只想知道.
你到底是什麼意思.
我只想知道.
你到底是什麼意思.
我只想知道.
你到底是什麼意思.
我只想知道.
你到底是什麼意思.
我只想知道.
你到底是什麼意思.
我只想知道.
你到底是什麼意思.
我只想知道.
你到底是什麼意思.
我只想知道.
你到底是什麼意思.

$^{921}$我只想知道.
你到底是什麼意思.
我只想知道.
你到底是什麼意思.
我只想知道.
你到底是什麼意思.
我只想知道.
你到底是什麼意思.
我只想知道.
你到底是什麼意思.
我只想知道.
你到底是什麼意思.
我只想知道.
你到底是什麼意思.
我只想知道.
你到底是什麼意思.
我只想知道.
你到底是什麼意思.
我只想知道.
你到底是什麼意思.
我只想知道.
你到底是什麼意思.
我只想知道.
你到底是什麼意思.
我只想知道.
你到底是什麼意思.
我只想知道.
你到底是什麼意思.
我只想知道.
你到底是什麼意思.
我只想知道.
你到底是什麼意思.
我只想知道.
你到底是什麼意思.
我只想知道.
你到底是什麼意思.
我只想知道.
你到底是什麼意思.
我只想知道.
你到底是什麼意思.

$^{961}$我只想知道.
你到底是什麼意思.
我只想知道.
你到底是什麼意思.
我只想知道.
你到底是什麼意思.
我只想知道.
你到底是什麼意思.
我只想知道.
你到底是什麼意思.
我只想知道.
你到底是什麼意思.
我只想知道.
你到底是什麼意思.
我只想知道.
你到底是什麼意思.
我只想知道.
你到底是什麼意思.
我只想知道.
你到底是什麼意思.
我只想知道.
你到底是什麼意思.
我只想知道.
你到底是什麼意思.
我只想知道.
你到底是什麼意思.
我只想知道.
你到底是什麼意思.
我只想知道.
你到底是什麼意思.
我只想知道.
你到底是什麼意思.
我只想知道.
你到底是什麼意思.
我只想知道.
你到底是什麼意思.
我只想知道.
你到底是什麼意思.
我只想知道.
你到底是什麼意思.

$^{1001}$我只想知道.
你到底是什麼意思.
我只想知道.
你到底是什麼意思.
我只想知道.
你到底是什麼意思.
我只想知道.
你到底是什麼意思.
我只想知道.
你到底是什麼意思.
我只想知道.
你到底是什麼意思.
我只想知道.
你到底是什麼意思.
我只想知道.
你到底是什麼意思.
我只想知道.
你到底是什麼意思.
我只想知道.
你到底是什麼意思.
我只想知道.
你到底是什麼意思.
我只想知道.
你到底是什麼意思.
我只想知道.
你到底是什麼意思.
我只想知道.
你到底是什麼意思.
我只想知道.
你到底是什麼意思.
我只想知道.
你到底是什麼意思.
我只想知道.
你到底是什麼意思.
我只想知道.
你到底是什麼意思.
我只想知道.
你到底是什麼意思.
我只想知道.
你到底是什麼意思.

$^{1041}$我只想知道.
你到底是什麼意思.
我只想知道.
你到底是什麼意思.
我只想知道.
你到底是什麼意思.
我只想知道.
你到底是什麼意思.
我只想知道.
你到底是什麼意思.
我只想知道.
你到底是什麼意思.
我只想知道.
你到底是什麼意思.
我只想知道.
你到底是什麼意思.
我只想知道.
你到底是什麼意思.
我只想知道.
你到底是什麼意思.
我只想知道.
你到底是什麼意思.
我只想知道.
你到底是什麼意思.
我只想知道.
你到底是什麼意思.
我只想知道.
你到底是什麼意思.
我只想知道.
你到底是什麼意思.
我只想知道.
你到底是什麼意思.
我只想知道.
你到底是什麼意思.
我只想知道.
你到底是什麼意思.
我只想知道.
你到底是什麼意思.
我只想知道.
你到底是什麼意思.

$^{1081}$我只想知道.
你到底是什麼意思.
我只想知道.
你到底是什麼意思.
我只想知道.
你到底是什麼意思.
我只想知道.
你到底是什麼意思.
我只想知道.
你到底是什麼意思.
我只想知道.
你到底是什麼意思.
我只想知道.
你到底是什麼意思.
我只想知道.
你到底是什麼意思.
我只想知道.
你到底是什麼意思.
我只想知道.
你到底是什麼意思.
我只想知道.
你到底是什麼意思.
我只想知道.
你到底是什麼意思.
我只想知道.
你到底是什麼意思.
我只想知道.
你到底是什麼意思.
我只想知道.
你到底是什麼意思.
我只想知道.
你到底是什麼意思.
我只想知道.
你到底是什麼意思.
我只想知道.
你到底是什麼意思.
我只想知道.
你到底是什麼意思.
我只想知道.
你到底是什麼意思.

$^{1121}$我只想知道.
你到底是什麼意思.
我只想知道.
你到底是什麼意思.
我只想知道.
你到底是什麼意思.
我只想知道.
你到底是什麼意思.
我只想知道.
你到底是什麼意思.
我只想知道.
你到底是什麼意思.
我只想知道.
你到底是什麼意思.
我只想知道.
你到底是什麼意思.
我只想知道.
你到底是什麼意思.
我只想知道.
你到底是什麼意思.
我只想知道.
你到底是什麼意思.
我只想知道.
你到底是什麼意思.
我只想知道.
你到底是什麼意思.
我只想知道.
你到底是什麼意思.
我只想知道.
你到底是什麼意思.
我只想知道.
你到底是什麼意思.
我只想知道.
你到底是什麼意思.
我只想知道.
你到底是什麼意思.
我只想知道.
你到底是什麼意思.
我只想知道.
你到底是什麼意思.

$^{1161}$我只想知道.
你到底是什麼意思.
我只想知道.
你到底是什麼意思.
我只想知道.
你到底是什麼意思.
我只想知道.
你到底是什麼意思.
我只想知道.
你到底是什麼意思.
我只想知道.
你到底是什麼意思.
我只想知道.
你到底是什麼意思.
我只想知道.
你到底是什麼意思.
我只想知道.
你到底是什麼意思.
我只想知道.
你到底是什麼意思.
我只想知道.
你到底是什麼意思.
我只想知道.
你到底是什麼意思.
我只想知道.
你到底是什麼意思.
我只想知道.
你到底是什麼意思.
我只想知道.
所以前纖是一種嘗試.
將上帝的國度.
耶穌基督回來的東西.
去看為一些.
時間化了它.
或者將它歷史化了.
這是一個問題.
我會說為什麼.
總之是這三種的縱末論.
我想說就是.
前纖是嘗試.

$^{1201}$將耶穌基督回來的這種盼望.
看為一種將來.
某年某月某日.
發生的事情.
而你就等待那天回來.
而毛遷就覺得.
其實就是沒有這種.
聖經的方法.
這樣理解.
當然奧斯丁就是毛遷欺的.
他不覺得聖經裡面.
的契捨祿經文.
是嘗試去撲出一個時間表.
將耶穌基督回來.
看為一個時間表.
是有些問題的.
我會說為什麼.
簡單來說.
縱末論裡面有三個不同的看法.
後纖是不用理會的.
因為是一百年前.
烏托邦的看法.
前纖是比較福音派.
字面保守解經的方法.
嘗試.
那時候聽過.
極端一點就是.
那些時代論.
波斯灣戰爭.
末世來到.
覺得耶穌回來前有大災難.
嘗試去撲出一條.
歷史時間.
而毛遷就覺得.
其實是沒有這些.
這些的偏派.
當然我們要說的是.
面對這種.
終末.
我們身處在一個.

$^{1241}$什麼狀態裡面呢.
可能大家聽過.
中文叫以言未言.
Already but not yet.
就是說.
好像雖然已經是.
發生了.
但又未來到.
這種很矛盾.
很吊詭的張力裡面.
耶穌基督已經在.
不過又未完全在.
上帝的國度已經來臨.
但又未完全來臨.
你很喜樂.
但又沒什麼喜樂.
這種很吊詭的狀態.
為什麼會這樣呢.
正是我們身處.
這種狀態裡面.
我們正是在一個.
現在的時間.
和將來的時間.
兩種矛盾裡面.
之前我講道講過.
我講那篇.
關乎於道.
因為講這件事.
但就是說.
我們其實.
還教會的盼望.
其實正是建基於.
剛才講前千的那種.
看法裡面.
他將耶穌基督.
回來.
看為一種什麼.
一種將來會發生的事.
即是今天.
還沒發生的事情.

$^{1281}$所以他覺得.
什麼叫盼望耶穌.
盼望耶穌.
就是一種我們.
純粹等住再來.
你的盼望.
就是住再來.
你對於今天.
這個世界.
我們以前唱的那些詩歌.
基本上.
到哪天.
耶穌就回來了.
今天我們就等他回來了.
這種去到極端化.
是一種像UFO那些教.
我們整個基督徒.
就不需要再理現在的事.
我們就純粹等.
耶穌回來.
很多異端都是這樣.
從奧姆真理教.
到很多其他的.
看法都是這樣.
他們覺得盼望.
就是住末日的時間裡面.
所以我們今天的世代邪惡.
我們就不要理現在的世界.
你就不要.
甚至極端點.
就不要做事.
你就辭職那份工作.
等UFO回來.
接你走.
這種是很極端的版本.
就是說.
你是越指望末日來臨的時候.
你就越對現今的世界.
是沒有任何參與.
或者改變的思想.

$^{1321}$所以當然環球會.
不比他們那麼極端.
但環球會是有這樣的傾向.
我們以前那些.
一講到朱末侖的時候.
就是什麼.
現在這個世道黑暗.
又攻打民.
我們就快點等朱末侖來.
所以這種盼望是.
盼望是什麼.
沒有的.
現在沒有盼望.
你只能等耶穌回來.
就是盼望.
所以這種盼望是對於.
現今的世界.
是沒有任何的改變.
或者是任何的參與.
他會叫你.
要不就等朱末侖來.
這種盼望是什麼.
就是你回教會.
回教會崇拜最好.
回教會崇拜.
敬拜很好.
回到外面就黑暗世代.
就不要理他.
回教會敬拜.
這種是一種方法.
所以這種叫做.
如果你很強調這種前遷.
當然我怕被人罵.
前遷不一定是問題.
前遷引出來的是一種.
對於耶穌回來.
一種完全擺在歷史上的將來的時候.
那你就和他沒有關係.
所以今天我們所說.
其實我們基督徒的盼望.

$^{1361}$並不是這樣.
不是純粹等待一個遙遠的將來.
那一天到來.
而是和我們的現在.
和你的生活有關係.
所以我們對於耶穌的盼望.
不是純粹回來.
而是希望能夠更加多的東西.
當然我們盼望有很多不同的方式.
今天可能你在小組裡面談也是.
對於盼望有很多不同的理解.
我們叫做false hope.
假的盼望.
有些盼望是建基於我們的理性.
因為我覺得.
我對於香港社會的預測是怎麼樣.
我對於整個世界的經濟預測是怎麼樣.
從而得出一個結論.
這種盼望是建基於人的理性.
我對於世界怎麼理解的時候.
我就有盼望.
計算不清就沒有盼望.
這個並不是真正的盼望.
起碼不是信仰的盼望.
信仰盼望不是純粹一種人類的計算.
確實是這樣的.
面對現在我們面臨的社會的時候.
確實是你怎麼計算.
你也計算不了有盼望.
正如剛才所說.
盼望並不是在乎於我們自己所可見的事情.
所以盼望不是一種人的計算.
也不是一種脫離現實.
純粹寄居於耶穌回來的那種終末的盼望.
而是什麼呢.
我們就要去看看.
要說盼望就不能不說莫特曼.
莫特曼寫了一本書叫《來臨中的上帝》.
The Coming of God.
他用了一句很有意思的經文.

$^{1401}$就是這個啟示錄里的第一章.
他說:今在,昔在,來臨中的上帝.
很奇怪的.
發現在啟示錄里的和本不是這樣翻譯的.
他說的是今在,昔在,而又永在.
但原文里不是永在.
而是coming的字.
回來的意思.
所以原來我們所說的上帝耶穌.
他不是一個純粹今在,昔在,永在.
而是一個正在回來的上帝.
coming的上帝.
而這個回來是什麼意思呢.
這個回來又不是那種.
除了說1984年6月9日那天回來.
而是耶穌不斷地在回來.
在未來裡面.
從未來跑回來現在裡面.
所以我喜歡畫這幅圖.
我展示裡面.
昔在,以前,今在,即是present.
而將來.
耶穌也是在將來那裡跑回來.
所以對於將來的時間.
仍然是充滿著可能性.
仍然是充滿著那種可塑性.
將來並不是和我們沒有關係.
更加不是完全定的.
將來仍然是一個充滿著可變的方法.
而耶穌正正是那位來臨中的上帝.
所以我們今天對於我們.
無論是香港的將來.
或者你生命的將來都一樣.
當我們說盼望的時候.
我們不是等主回來接我們走.
而是對於這個世界的將來.
仍然是充滿著那種可塑性的盼望.
我們希望能夠改變它.
我們知道耶穌正正是掌管著這個未來.
所以對莫特曼來說.

$^{1441}$或者對我來說.
盼望耶穌是一件很實際的事情.
不是純粹等UFO回來.
等主再來.
而是我們對於今天的世界里.
仍然是充滿著那種行動和改變.
這就是我們想強調的.
不是傳統華人教會所說.
那種終末災難性的主再來.
而是今天我們對於生活的力量.
我們對於面對黑暗實在變化.
我們如何來參與和改變.
當然我們剛才說.
當我們預見著將來的時候.
是充滿著很多的張力.
這個我講過的.
我都講過一次.
特別是在哥倫多前說里.
弟兄們我對你們說.
時間減少了.
這個時間減少了一個很特別的字眼.
就是發覺一個很弔詭的現象.
從此以後乃有七字的.
就好像每七字一樣.
哀哭的要像沒哭.
快樂的要像不快樂.
自慢的要像沒有所得.
用世物的要像不用世物.
一個很弔詭的狀態.
有老婆就等於沒老婆.
哭也不哭 快樂也不快樂.
什麼意思呢.
因為我們正正是處在兩個時間點裡面.
一個是我們熟悉的時間.
一個是我們知道上帝的時間.
所以我說過.
這個時間的意思是解作什麼呢.
時間減少了.
是解作時間的那種壓縮.
就像一個疊疊了的時間一樣.

$^{1481}$我們同時身處在兩個時間當中.
一個是世界所見到的時間.
2023年.
一個是上帝的時間.
終末的時間.
所以我們今天.
already but not yet.
我們基督徒正正是這樣.
似乎開始面對一個很艱難的世代.
同時我們又知道.
上帝的結局.
上帝的永恆.
已經在今天的裡面.
所以我上次也說過.
我們基督徒戴了兩只手上的標.
一個是我們普通時間的標.
一個是上帝的標.
所以基督徒是這樣.
你又知道今天好像沒什麼指望.
但我們仍然有一種盼望在當中.
這種正正就是我所說的.
這個盼望的開始.
所以我們見到.
當我們面對這個世界的時候.
這個世界的表象正在過去.
雖然它現在是很真實的.
但知道它正在fade away.
這個就是《仰仰書所》的經文.
真光已經照耀.
黑暗卻漸漸過去.
這個是不是我們經驗那種的經驗.
比如我們回家一開燈.
晚上回家一開燈.
一開就立刻光了.
但這個經文是什麼呢.
真光已經照耀.
黑暗是慢慢慢慢過去.
所以這個就是這麼奇怪.
一方面我們面對的是耶穌的真光.
但這個真光又不是立刻一止.

$^{1521}$一下子就光了.
這個正正就是.
already but not yet的意思.
正正是兩只手上的標的意思.
所以我們的盼望.
是建基於這個時間的觀念裡面.
我們的盼望不是純粹我們的計算.
也不是一種等待著來的看法.
而是我們知道今天黑暗.
但我們仍然可以看到.
仍然相信.
仍然是期盼.
知道這種上帝的盼望.
所以看到這種盼望是有點.
霍神經的.
不是很說得出來的.
因為確實是你解釋不了的.
正正是保祿所說的.
他在毫無盼望的時候.
仍懷著盼望而相信.
這個是保祿在羅馬書裡面所說的經文.
而經文裡面.
正正是一個很有趣的英文詞.
就叫做hope against hope.
這個盼望正正是和一切人的盼望.
是對敵的.
是相反的.
就是人所能夠期盼的.
所依靠的.
無論是靠那種理性分析的形式.
或者大局.
或者是那種純粹是宗教的盼望.
我們基督徒的盼望.
其實是一切盼望的相反.
我們不是去看現在.
面對的事情有多樂觀和悲觀.
而是我們純粹那兩只手.
根據著我們知道上帝留給我們的手.
一個終末的手.
我們身處在兩個時代裡面.

$^{1561}$所以我們仍然能夠知道我們仍然有盼望.
所以簡單來說.
任何便宜的盼望.
任何easy hope.
都不是真正的盼望.
所以就像傻強所說.
總之任何說自己是盼望的.
那些都不是盼望.
任何容易得著的盼望.
那些都不是真正的盼望.
所以布魯德說.
得救是在乎盼望.
只是所看見的盼望不是盼望.
誰還盼望所他看見得見的呢.
所以盼望的課題其實是很不容易的.
我預查的時候都和牧者談過這些話題.
說盼望很容易.
但要等到能夠掌握到這種盼望.
其實又不容易.
因為任何你會覺得很容易的盼望.
這些都未必是信仰裡面所說的盼望.
所以去年我們的研究所的那套劇很好.
他想說絕望是什麼.
絕望是很難的.
完全的絕望是很難的.
因為我們知道在任何情況之下.
完全絕望是很不容易的.
我只能這麼說.
但是那種盼望你只能夠好好地去經驗出來.
我不能夠傳給你的.
這個就是這樣,你拿吧.
盼望你只能夠慢慢地在絕境裡面掌握出來.
所以這個就是我們今天嘗試知道.
盼望是關乎於上帝的時間.
盼望是一切盼望的相反.
我們不是去依靠一切其他人類的可能性.
所以既然這麼難.
究竟有什麼可以實踐呢.
我們想說一種能夠實踐的盼望.
特地我用了背景的圖.

$^{1601}$不知道大家記不記得這幅圖.
這是幾年了.
兩三年前一場比賽.
利物浦對巴塞隆拿上半場.
上半場輸了3比0.
下半場就贏了4比0.
這個我覺得是一個很經典的盼望的例子.
其實那天我沒有看這場比賽.
我也是沒有盼望.
我預料會輸,所以沒有看這場比賽.
就是這樣.
原來我們的盼望就是這些東西.
我們嘗試去實踐一些盼望.
這種盼望其實是嘗試去變成一場球.
很容易在我們生活裡面出現.
會是嘗試.
因為我們要知道盼望其實不是等待的東西.
不是等待住在外面那麼簡單.
比特厚書說得好.
他說是切切仰望上帝日子的來到.
這種仰望,這種等待上帝回來.
沒錯,你只能等待.
你好像沒有什麼能夠做到上帝回來.
一個字很重要,就是切切.
語文裡面的切切就是hastening.
就是很急速,很催促.
一方面是等待,好像很被動.
但其實他知道我們要很迫切地等待上帝.
什麼意思呢?.
就是我們仍然用很多方法將盼望變成一些行動.
沒錯,我們不能夠逼上帝回來.
也不能夠做些什麼去改變這個世界的未來.
不過我們確實是嘗試用盡我們的心思和方法.
去做一些事,去改變一些事.
所以我想說,這個我寫了一篇課讀.
當我說實踐盼望的時候,我有兩個風格要說.
第一,我們需要有一些事是可以做的.
既然是實踐盼望,我嘗試一些事是可以做的.
這是第一點,我們不是純粹等待,不是等待耶穌回來.
我們起碼要付出盼望,希望能夠是主動的,去做的.

$^{1641}$但你知道做是一種等待,你做不了什麼,改變不了什麼.
但你也要做些事,這是第一點.
我們的應用實踐上,第一我們是希望能夠做些事出來.
第二,就是練習的意義是什麼呢?.
今天想說,以下所說的練習,所謂的應用.
其實做了不代表有的,你想說.
你不明白什麼叫練習嗎?.
練習就是說,其實你是去嘗試去做.
但那些東西其實跟後果沒什麼關係.
好像一個體操運動員,我們會去跳鞍馬,兩個半空翻落地.
你會不斷地練習出來,這個練習是有意義的.
因為你去練習,其實整件事是有關係的.
不過是否說你練完這些東西就能夠做到呢?.
其實又不一定.
所以這個練習,不是說以下的就叫做盼望.
盼望不等於我做完就有盼望.
不過盼望往往都有這些東西做出來.
明白嗎?因為我想說,這就是困難的地方.
因為你去嘗試去實踐的時候,你會不斷地重復去做這些東西.
有什麼例子呢?.
譬如說,我說的就是開電單車的例子.
你開車,你學一個運動也好,你學車也好.
你會不斷地重復一些動作行為.
轉幾個指,然後你會扭車輪,踩油門這些東西.
你狂練這些東西出來,其實不代表你會開車.
但會開車肯定會會做這些東西.
所以你不斷去練習這些東西是和盼望有關的.
但不代表你做足都有盼望.
有沒有盼望呢?其實真的沒得教的.
盼望是關乎什麼?就是上帝時間.
你對於上帝時間的那只標有多大的體會.
你看到今天香港,你能夠知道上帝的光輝已經照耀了.
你就有盼望,就不關那些動作事.
但如果要教你的話,我只能夠說盼望就是會做這些東西.
就是這一下這些東西,明不明白?.
所以實踐就是這樣做,但這些東西不等同於盼望.
只不過有盼望的人會做這些東西.
所以我說我們會嘗試去做這些東西.
嘗試去實踐這些東西出來.
第二就是我們嘗試知道這些東西不需要包括.

$^{1681}$第一個就是想象和計劃.
一個盼望的人,其實他正正是對於將來仍然有計劃和想象的人.
我說過了,我說過了.
如果你發現在凶案現場有個死屍拿著一張鏡頭會飛的話.
他應該是他殺的,不是自殺的.
因為他仍然對於將來有盼望.
所以我們嘗試去實踐我們的盼望的時候.
不妨對於你的將來,或者對於香港社會的將來.
或者這個世界的將來,你仍然可以去計劃一些東西.
透過這種計劃,其實是慢慢去學習對於我們的將來.
是可以有一些想象的.
所以這個第一個,你發現兩個字很特別.
一個字就是計劃,一個比較理性的思考.
這兩個都是從無到有的,都是你腦子里的東西來的.
將來的,未發生的.
所以對於將來,我們仍然可以嘗試去計劃或者想象.
大家不知道這幅圖是一個什麼歷史事件.
是,柏林圍牆.
柏林圍牆是人類歷史里最激發人想象力的一個事情.
為什麼呢?.
因為當突然一天之間,你知道歷史是什麼嗎?.
蘇聯突然一天之間建了一座牆,封鎖了整個柏林,分成兩邊.
一天之間,東柏林和西柏林是分開了兩邊.
人們不能夠翻牆去另一邊.
所以這個封鎖,這個牆,這個失去的自由,令人有很多的想象.
如果你去柏林的話,柏林有一個叫柏林圍牆博物館的.
有一個很有意思的說話,他說人類因為尋找自由的緣故,激發無限的想象力.
當時有些人為了要越過這一面牆,有很多不同很有創意的想象.
有人試過用家裡的床單,編織一條繩子就爬過去.
有人試過弄輕氣球飛過去.
有人試過在車尾箱里躲起來就開車過去.
有人試過扮警察走過去.
原來我們對於將來的盼望,其實和我們的想象有很大關係.
一個沒有盼望的人,對於將來是沒有任何的想象.
或者不願意去想象,也不有意不去想象任何的計劃.
所以,我想說一次,有這些東西是不代表你有盼望.
但如果你想操練一下盼望的話,你可以嘗試去做這些東西.
如果你覺得自己今天對於你自己任何一個方面沒有任何盼望的時候.
不妨嘗試去想象,不妨嘗試去計劃.
任何對於將來的想法都是一種盼望的練習.

$^{1721}$這是我想說的第一點,你嘗試不妨去思考將來.
第二個就是行善,我覺得是.
這個也是我們在《牧者御茶》的時候提出的一個很重要的點.
特別是面對著今天這個社會.
當你對於這個社會仍然有盼望的時候.
在一個黑暗的時代裡面你仍然有盼望的時候.
你做一件好事出來,這個已經是一個最好的盼望的名證.
你仍然願意在當中做一件好事出來.
這個就是盼望的起點.
我說過的也是,以前有一部電影叫做《熔爐》.
可能大家聽過,整個《熔爐》的電影是一個很黑暗的結局.
因為一個法律的不公平,整個不公義.
最後其實是沒有好結果的.
主角裡面的小朋友全部都得到不公義的對待.
其實是這樣結束的.
但是導演怎麼去結束這麼悲傷的電影故事呢?.
最後就是這群人雖然輸了.
仍然面對著很多不公義的對待.
但這群人一起在這裡過聖誕節,一起在這裡吃東西.
在黑暗裡面你仍然做一些很微小的好事出來.
沒錯,你改變不了這個世界.
但你仍然去嘗試去行善的時候.
這個是我們一些盼望的信號.
記住我說過一次,有些事情不代表得盼望.
但你去嘗試,仍然堅持自己去做美善的事.
這個是我們對於盼望的一些實踐的操練.
這幅圖我們會在以後首播.
我們一些片段.
所以我們教會仍然希望將我們的社關放在我們很重要的位置裡面.
因為這是我們作為盼望的群體.
一個很重要的實踐出來的一個表徵.
第三個我想講的仍然是一個很老土的題目.
就是土告.
土告為什麼和盼望這麼大的關係呢?.
這個我都提過.
有一個很好的天才哲學家Joseph Pieper有提過.
他說prayer is nothing other than the voicing of hope.
土告正正是我們盼望的聲音.
因為每一個土告.
全部都是你和將來去打交道的聲音.

$^{1761}$當然感恩土告關乎於過去.
但任何一個祈求.
今晚不要下雨.
到你的子女能夠出去讀書.
能夠平安到任何的東西.
任何的土告都是和將來打交道的.
你將你的將來.
你將香港社會將來放在土告裡面的時候.
這個就是最直接來將盼望的聲音放在你的面前.
這個我覺得是最重要的.
剛才所講的可能都是一些做了不代表有的事情.
但土告正正是因基於我們的相信.
更加是對於我們對於未來.
置放在我們的上帝面前.
所以仍然堅持來為未來去土告.
特別是為香港的未來土告.
我不知道你還有沒有為香港未來土告.
真的.
可能已經有兩三年沒有為香港未來土告.
當我們連土告都沒有做到的時候.
當我們連土告都不敢覺得是那些值得相信的時候.
這個正正是我們沒有盼望的開始.
或者是一個沒有盼望的表徵.
所以不妨重新來去取回這個土告.
為香港的未來去土告.
這個我覺得是我們留堂仍然很堅持的東西.
所以看到潘Sir每個月裡面月土.
很多時候都是這樣.
仍然堅持為香港去土告.
是講的不是純粹過去的事.
而是我們整個香港社會的未來.
仍然相信香港或者這個世界仍然是可以被改變的.
我們不是等到自己死了能夠回到天家.
那就叫做最好的結局.
我們仍然覺得我們的世界是此刻當下仍然可以被改變.
這個就是我們的盼望.
最後我想講Full Church是一個盼望的群體.
因為這個是我自己很特別很想提出來的東西.
當然Full Church是一個有愛群體.
一個有信仰群體.

$^{1801}$但我覺得在這個年代裡面.
Full Church更加需要被稱作是一個盼望群體.
首先每個頂尖輩是真的擁有盼望.
沒有的話立刻去操練他.
提醒自己如何可以有.
在諸如神學聖經裡面.
在土告裡面.
在實踐裡面.
令自己有盼望.
從而把這份盼望給你身邊的人.
這個我覺得是Full Church最需要.
為什麼要存在的原因.
我覺得教會是一個見證耶穌基督的群體.
我已經講過很多次了.
而我覺得盼望正正是這個年頭裡面.
最重要告訴人們.
需要人認識耶穌的那個向度.
我很期望更加多人在Full Church裡面.
看到盼望.
或者因為我們能夠將耶穌的盼望帶給他.
所以他能夠信耶穌.
所以這個我覺得是很重要的.
人們認識耶穌不是因為上了天堂.
還是因為癌症可以被醫治.
而是耶穌的盼望能夠幫助他.
所以這個我覺得是很想.
我們Full Church的教會能夠可以掌握的東西.
我們先講這麼多.
一會兒有些問答的時間.
希望大家能夠去問一下.
在你心目中如何能夠實踐盼望呢.
對於盼望的房地有什麼看法.
可以和大家談談.
很期待和大家談.
我們有祈禱時間.
我們將教會交托給你.
最後求主讓你在當中.
無論是今天直播.
或者將來每一個頂智媒去看第六科.
讓他能夠掌握到.

$^{1841}$盼望你給我們這麼大的恩賜.
因為你的緣故你給我們有盼望.
讓我們能夠在無論黑暗裡.
都仍然有一個不是人所能夠促成的盼望.
而是單單因為你的緣故.
我們所掌握的盼望.
求主幫我們這個教會.
能夠成為一個有盼望的教會.
一個能夠全陽盼望的教會.
讓更多人實踐盼望的教會.
求主幫助我們.
奉主命求.
阿們.
盼望來到了.
放了兩個月的假期.
有沒有休息.
都休息了很多.
希望教會盼望.
其實都是最困難的.
是不是也不容易.
和弟子妹說盼望的主題.
單單以最後你這張照片.
在戶外崇拜就真的很不容易.
大家都希望在公共空間繼續敬拜.
在公共空間繼續見證.
其實都是我們整個信仰群體.
想實踐這個盼望.
我為什麼會選這張照片呢.
因為這張照片正正就是.
這個我們月頭想做的事.
就是我們一班人一起去做一些事.
這件事其實.
做完之後有什麼果後是不知道的.
但我們仍然對於將來是有想像的.
對將來是有計劃的.
我們就做了.
這正正是我們去實踐盼望.
很重要的一個例子.
正正是這樣.
正正是一班人.

$^{1881}$無論是多無聊.
多不知道有沒有功用也好.
仍然去堅持去行善.
這正正是我們希望能夠做的事.
對呀.
不知道大家整晚聽了很多盼望這兩個字.
對你們來說.
這一刻你的盼望是什麼.
或者你本身有沒有什麼盼望的想法呢.
或者大家可以分享下.
和大家互動下.
或者盼望對大家難不難呀.
後面好像有些聲音.
哈哈.
是不是已經很難了.
很難有盼望了.
不是的.
OK.
或者大家可不可以分享你怎麼去有盼望的.
想問一下.
剛才提到有些操練的辦法.
但是很強調的就是.
做了那些事不代表自己有盼望.
為什麼要強調做了那些事都不等於有盼望呢.
什麼時候才知道自己很有盼望呢.
所以就說一個運動員例子.
你明白到一件事.
其實你重復那件事是有用的.
有幫助.
但是不是等同於就叫做會的.
所以一個盼望的人就有這些事.
我們嘗試有盼望就做多些這類的事.
所以我覺得是.
因為盼望不是很容易實現.
我也不知道怎麼能夠重復到有.
但是我覺得能夠說的就是.
就做這些.
接著做這些事.
你就會深刻地就能夠.
就能夠理解到.

$^{1921}$但這種理解不是說.
一些機械性的.
你做幾次就有.
只能夠你去不斷重復去摸索出來.
因為我覺得其實.
因為在《牧者會》裡面我們也談過這個課題.
真的發覺.
怎麼叫等於沒有盼望呢.
這個真的是不容易的課題.
但是我覺得起碼可以試一下做這些事.
試一下對於將來有想象計劃去行線去討告.
這些我覺得是可以試一下來到去嘗試去.
透過重復去好像肌肉記憶那樣.
令到我們能夠可以慢慢掌握到背後的道理.
其實重點就是背後那種的.
叮一聲那一下.
我覺得會是.
當然不一定是性靈.
但是如果是實踐出來的話.
我覺得又不是純粹那麼一無縹緲的.
應該是有些事可以試一下來到去嘗試去做的.
我想問一下你現在會不會還收街邊的單張呢.
都有的.
有些都不會收的.
如果你派的是呼音單張.
那些人會不會收呢.
我自己試過派單張的.
派單張的人會看一看單張的後面有沒有一些虛線.
有虛線就是證明有QPON就可能會收的.
如果是一張廣告的東西他通常都不會收的.
我一個例子.
其實都和我有一次在一間教會有一段時間參與的時候.
一個對我自己很大的提醒.
那間教會每個主日都會.
他是屋村教會來的.
每個主日他都會派呼音單張的.
但是屋村來來去去都是那些人的.
都沒有什麼生客的.
但是沒有人會收的.
但是他都繼續每個禮拜都會派單張的.

$^{1961}$不是有新朋友進來.
但是他們堅持一件事就是.
仍然告訴人家這裡還有一間教會.
這個是一個很重要的信息來的.
就是應用剛才John說的那樣東西.
做了就沒有那一刻不做好像都不是很覺.
但是有些人堅持做的時候就告訴人家.
那裡仍然是有教會.
那間教會是有開門的.
不派單張那天就是教會不開門.
只要教會開門都有人派單張.
這個就是很重要的信息.
那樣東西的存在對我們來說.
沒有加上沒有減少.
但是那樣東西繼續做就讓人有個盼望.
有個存在的意思.
我都說一下上年那套劇.
我們和Jan談最簡單的那種想法.
就是整套劇就是想去說這件事.
就是怎樣叫有盼望.
發覺是很難.
但是發覺說絕望是怎樣.
當然是容易一點.
絕望大概就是這些東西.
所以我就說只能夠反過來說.
既然絕望是這些東西的話.
那你不要做絕望的那些東西.
那你就嘗試去領悟盼望是什麼.
所以我覺得領悟盼望是一種領悟.
或者是一種從神而來給你的一些恩典.
或者是信心.
但是這些很普普票票的.
叫你信叫你禱告.
相信耶穌都會回來.
但是我們不要做那些絕望的東西.
嘗試做一些絕望相反的東西.
嘗試去希望能夠丁一聲明白什麼是盼望.
或者從神而來得到什麼是盼望.
所以這是金螃的pose.
強調好像很奇怪.

$^{2001}$什麼叫實踐什麼叫練習.
因為真的你仍然是可以沒有什麼盼望的.
你對於你會去旅行.
你會去有計劃看Mirror.
你會仍然生活.
但是不代表你有盼望.
因為你仍然可能對於這個世界.
對一般人來說不是你有期盼.
你是對於將來有一些的想法.
因為你是人來的.
但是這樣是不是有盼望呢.
我覺得真的不容易去找得出來.
但是我覺得如果真的要說到底的話.
總之你不夠就是.
總之你發現自己有絕望的時候就不要去做.
嘗試去尋找上帝的盼望.
其實我最後想說一件事.
原來盼望是一種領悟.
我們做夢研究所原來都是.
發現就是想人去領悟到什麼叫做盼望.
但是這個說不出來給你聽.
你不能夠跟著那個餐單.
跟著什麼方法什麼去做找到盼望.
所以就是這樣去尋找出來.
網上就問到當禱告的時候.
好像以讀不回.
有什麼方法可以繼續維持盼望呢.
我自己的看法就是用剛才的信息.
一個基督徒戴著兩只手錶.
你自己祈禱可能是按你自己的時間表去祈禱.
或者是表達你的意願.
帶上帝有上帝的時間表去工作.
這件事其實都不是新的東西.
你看回舊約的先知.
很多時候他都去做上帝要他做的工作.
但是他說的審判.
他說的上帝的追討.
甚至一些上帝的工作的時間.
都不是他說了就立刻做.
上帝都有他的主權和他做事的方法.

$^{2041}$有時甚至說完之後都不一定會做.
因為那些人會轉了.
所以我覺得那個過程當中.
盼望不是建立了剛才一直說的我們做了什麼.
盼望是說我們知不知道自己在做什麼.
這個是很重要.
我們Folk Church是希望弟兄姊妹看到我們.
帶領整個群體在做什麼.
而我們在做什麼都是教會存在的重要地方.
讓人得聞福音.
讓人認識上帝.
讓人感受到生命中有愛.
讓群體當中有社會參與.
這些都是我們覺得我們不斷存在.
在做一些事.
但是這個不是說你討告.
好像堅持很久都沒有果效.
反而就好像沒有盼望.
我覺得這仍然是看你自己一隻手標.
就看不到另一隻手標的東西.
不知道有沒有其他問題.
後面.
我有點生活上的分享.
不知道可不可以回應今天.
我覺得得著的點.
很體會你所說的那種行善.
是基於我在神里領受到愛.
去感受到這個世界有愛的時候.
覺得不貪大步只是小步.
看到小步自己可以做.
而帶出神的愛.
和人之間的愛在流動.
我就覺得這是很有意義的事情.
而在這個事情里.
我是不是一個想象還是計劃.
其實有時候不是想得太多.
但是我又在想.
例如我很喜歡看到的圖片.
我明明看到一個未信的親友或者朋友.
在臨終還未信.

$^{2081}$我可以做些什麼呢.
就去探望他.
在他身邊嘗試過.
他未必明白.
但是我就是帶著這個祝福.
我不是真的去想.
那個臨終的人將來會是怎樣.
我當然希望他相信.
但是那一刻他真的不相信.
我也要尊重他的選擇.
但是我去做這件事.
我也在感染身邊看著我們.
陪伴他的人.
他經歷神依然愛他.
我就去感染這種愛.
我想說的.
你想說的那種.
我沒有特別好計劃.
但是我就看到一個想象.
我就是將神期望的愛.
新天地有愛的新天地.
在地上要活出來.
要讓身邊人都感染到.
沒有人被捨棄.
那種對他的不離不棄.
這些是不是你所說的那種想象.
但我不是很實際地計劃.
我覺得做得到就去做.
我只是去按自己的那個微小的行動.
去嘗試讓身邊的人多一份愛的流動.
是不是你所說的那種.
我就有些生活上的分享.
還有剛才提到的土告.
我也試過土告落空.
是不是就沒有盼望.
我在想一個小朋友和爸爸聊天.
爸爸你為什麼不給我.
我想要那樣東西.
我想要你為什麼不給我.
可能是哭著說.

$^{2121}$但同時也會說.
你給我吧 你給我吧.
是不是我叫他給我他就會給我呢.
有時候我自己也要學習一個信服的功課.
就是天父爸爸會知道什麼適合我.
或者我們暫時要等什麼.
可能不是等一個很美的天地.
而是等我一個很美的素質.
去有一個更堅穩的信心.
相信神會帶我有更美的地方.
是在他心裡面.
不是在地裡面.
不是地上.
因為心裡面是天價的.
這些就是我所說的盼望嗎.
所以我覺得任何事情都可以.
我們很適合這個月的主題.
一起去做一些事.
任何事你肯做.
就已經是盼望的開始.
我近來對香港教會沒什麼盼望.
不想做.
不想再理不想再碰.
因為我近來在預備這些課.
我都沒什麼盼望對香港教會.
對於那件事.
沒盼望或者絕望只是無動於衷.
你無動於衷.
他影響不到你.
你又沒什麼興趣對他.
所以我覺得.
但你仍然去做一些事的時候.
那件事不在乎你計算得到是否成功.
或者是否行得通.
但你仍然肯去做的時候.
這正正就是一種盼望的開始.
所以我覺得不妨想想.
你現在的生命裡面.
有什麼仍然有心頭的事.
很想去做的事.

$^{2161}$那就一起去做.
一起結黨去做.
這正正是我們的盼望.
很重要的實踐.
所以我覺得我們的盼望是一些行動.
不是純粹等最後來.
不知道你發不發覺周邊的弟兄姐妹.
或者你自己都有些初老的症狀.
什麼叫初老的症狀呢.
你聽John說他自己曾經說.
他自己說有中年危機.
我那次聽他說的時候.
我比你大.
我覺得我做中年應該比他快.
初老或者中年危機其中一些症狀就是.
經常想舊事.
舊事衝了.
還想什麼呢.
重點又回到.
可能對自己做的事沒有什麼觸動.
沒有什麼想象.
甚至很多事都會覺得已經過了有效日期定格.
沒有什麼動力繼續做.
可能籠統地說.
這些就好像沒有了盼望.
所以如果你的歲數還沒有到中年.
就不應該有初老的跡象.
但可能環境或者周邊的聲音.
令你可能不想再想象.
我自己通常覺得.
你試一下去近不同的人.
看多一點.
很多時候你突破了平時接觸的人群體.
看多一點.
看闊一點的時候.
你會發覺你還有很多想象空間.
如果說實務一點的做法.
我自己的經歷就是.
我自己情緒覺得好像.
沒有什麼空間或者很困逼的時候.

$^{2201}$我很多時候都會約很多人.
每個星期都約不同的人見面.
其實我很喜歡聽故事.
或者我看人訪.
看見證的時候就發覺.
其實這個世界充斥了很多人不同的經歷.
其實你看不同人經歷的時候.
你就會見到很多人在自己的環境當中.
做回他自己的事.
其實你就會見到每個人都有自己的盼望.
可以繼續到.
不是要比較.
但你會發覺.
其實你有沒有看到你自己可以.
發展的空間或者做到的事呢.
這個就是由你而來的盼望.
我希望你不要覺得很遠.
但你會發覺看多一點.
看闊一點的時候.
你就會覺得.
是哦,其實我也可以的.
我們有時候都會分享喵喵的那些片子.
你會見到喵喵很多人訪.
哇,這麼有活力.
我們基督徒都沒有這麼有活力.
你會發覺故事會成為激勵.
我們的見證其實.
不一定很成功才能說出來.
失敗見證其實都可以.
成為一個很重要的能力來源.
說一個例子.
我暑假坐飛機的時候.
看一部電影.
Tom Hanks那部電影.
他很萌室的老伯伯.
我的名字是Auto.
這部電影其實很有趣.
他想自殺.
然後一個鄰居叫他幫忙.
不知道做什麼.

$^{2241}$正正就是這件事.
他想自殺.
不過他要幫人家弄那部車.
還是什麼的.
他想做完才自殺的意思.
這種做一些事.
這正正就是對將來的計劃或參與.
我都要先做完這件事.
我都起碼先做了.
做的時候不是很大件事.
不是很大件事.
但是你的生命裡面.
就正正是有些事你要完成.
有些事你要在那些事做完.
這件事就讓我們生命裡面.
仍然有一種對於將來.
仍然有一些事你要做.
你要實踐.
你要去完成.
你要去計劃.
其實人生就是這樣.
你不斷去讓自己.
能夠在將來裡面.
仍然有一些參與.
有一些想象.
那條命就這樣救回來了.
很多次幫人家弄東西.
弄一下燈.
弄一下東西.
我仍然.
原來我對將來.
仍然是有一些參與.
這個就是生命裡面的盼望.
用John這個劇目的例子.
每次Otto他自己想自殺的時候.
有些人找他做事就死不去.
但是他又聯想起.
他太太其實.
旁邊有時候碎碎念.
其實都是欣賞他這些事.

$^{2281}$做事的地方.
這個就是他的人設.
可以有這個空間.
可以繼續做些事.
我們很多時候.
沒有欣賞自己.
或者欣賞旁邊的人.
其實大家每人做一點點事.
都是一個很實際.
給對方一個盼望.
或者存在的意義.
你好.
我都覺得大家說得很好.
我想問一個問題.
盼望其實都需要長期的練習.
這個一定是一個可實踐的盼望.
因為我們有永生.
我們有時候經常與神和好.
經常回到主裡面.
就是一個很簡單的盼望.
不過因為人.
都有些罪性和貪婪.
我們可不可以利用他們來實踐這種盼望呢.
這個貪婪如何去運用一個好處.
我這樣想.
人都是想進到天國的時候.
神就讓他管很多座城池.
就不只是做一個普通的藝文.
你僅僅是能夠得救.
你在外面哭泣像一千年之後才回來.
你之前在地上僅僅是能夠得救.
你又沒有實踐盼望.
有時候我沒有應允你的禱告.
你就走了去外世界才回來.
我就想了一個方法.
不知道你們覺得可不可行.
就是當我沒有盼望很失望的時候.
我就和神說.
我這次一定要實踐盼望.
繼續去傳福音.

$^{2321}$雖然可能傳了三十年.
親戚暫時沒有興趣.
但是我要繼續.
因為我每一次去做.
每一次禱告去傳的時候.
我做了這件事.
神已經給了信用.
到時我走了的時候.
離開了這個世界.
我都有想去的.
很開心.
我繼續去實踐盼望.
但是這裡有點困難.
其實你只是想要將來的相似.
但是我覺得這個推動力.
有多符合人性呢.
去推使自己.
即使在一個很絕望的環境當中.
都去工作.
就好像上班有錢.
不打工就沒有錢.
沒有收入.
我在我很失望的時候.
這樣去處處逢生.
請問兩位覺得.
會不會有些偏.
還是你覺得都挺好的.
不如明天我們一起開始吧.
我想說.
我一向都不是看將來的相似.
我不太相信相似的東西.
因為我覺得不是.
我現在做多少有多少相似吧.
我不太理解經文裡面的相似.
是這樣相對的掌握.
所以我覺得.
今天我們做一些事情.
不是為了相似.
而是相反的.
應該是因為我們有上帝.

$^{2361}$給我們的相似恩恩典.
所以才去做.
所以我覺得.
我們應該先打破這件事.
不是一種很相似的看法.
我們的行為應該是因為.
耶穌基督的恩典.
我們去回應.
多於我們回應了之後.
做了之後就相似.
我不覺得將來牧師傳導人.
會多相似一些.
只不過我們有些人是.
願意去回應上帝的恩典多一些.
我們就去做.
都是一些無常的事情.
都是一些自己甘心情願去做的事情.
多於一些爭取相似.
雖然這個說法是一些.
以前的教會經常說的.
相似的那些.
我們找次講一下這個話題.
或者相似這個問題.
這個反而要想一想.
不關相似的事.
因為我們很軟弱.
經常靠著主的恩典.
那些恩典.
有時低谷去到最低.
那些人很功利.
特別是香港地.
我就在想.
香港有些某些功利的人.
可不可以用這個方法.
來激勵他們.
當他們很絕望的時候.
就跟他們說這個.
但其實最後上次都是神判斷的.
我這個立場只是告訴你.
可能會好一點.

$^{2401}$是不是都不是那麼好.
不是那麼好.
因為hope against hope.
這個盼望正正不是那些東西.
不是那些好處.
或者眼見的好處.
上次都是眼見的好處.
所以我們信仰所說的盼望.
應該不是那些.
不需要這樣去想.
都應該有的.
我本來還想來到這裡.
今天來到這裡.
我沒有什麼盼望.
來到這裡我開心.
我又多一點盼望.
神都很開心.
你來教會都是一件好事.
你繼續努力.
結果是怎麼樣.
我覺得很開心.
因為經過暑假之後.
大家都很絕望.
因為你看到天都下雨.
下到十號風球.
哇 所有東西都倒下.
今天風平浪靜.
神祝福第六科.
因為這是一種可實踐的盼望.
我比較少用這個方式去表達信仰.
因為這樣說很容易被人套路.
人家覺得會有功有利.
這個盼望.
這個邏輯.
其實信仰從來都不是用功利的方式去說.
所以你現在做多少.
我認同.
我都不是阿John的方法.
現在不是和一百人傳福音.
將來會比你和一個人傳福音.

$^{2441}$獎賞會多一點.
因為整件事.
我們接觸的人都不同.
你怎麼比較.
所以我仍然覺得信仰是一個很好的方法.
因為信仰是一個很好的方法.
我仍然覺得信仰的表達和傳揚給弟兄姊妹的過程當中.
都不是用你功與利的.
好像做生意的思維.
這樣逼的.
這樣逼的過程當中.
就很容易被人拿捏了.
就好像以前天主教那些.
就是你做不夠你可以買來補償.
這個補償的過程當中.
就會令到人會給下去.
其實信仰從來都不是我們做多少去賺回來的.
這個以弗所說講得很清楚.
得救是本福音.
借著信.
整件事都不是我們的工作.
做了多少東西出來.
而是上帝的恩典.
所以整件事.
這個思路就不要再想下去了.
不可行的.
我想問一個可能.
敏感一點的範疇.
這裡是講.
流散.
愚民的一個.
過程.
去到一個盼望的位置.
我就會想起.
因為很多人在走與流之間.
有很多掙扎.
很多想法.
而在很多這些掙扎.
想法裡面都牽涉盼望.
你對這個地方.

$^{2481}$還有沒有希望呢.
你對於這個地方還有沒有想象.
你想象這個地方會.
繼續更差 更糟.
那你可能會選擇離開.
或者怎麼樣.
那怎樣在.
當中去看.
無論是在.
對於信仰上面.
的一些盼望.
或者對於在地上面.
可能現在的實況.
對於地上的一些.
無論是人士還是權勢.
的一些沒有盼望.
或者還有沒有盼望的一些.
問題.
怎樣可以在我們信仰裡面.
或者實踐裡面.
去影響我們.
對於走或者流.
的一些不同的思考.
關於這方面的一些.
可以再分享多一點.
正正是等這些問題問.
是很需要問.
因為我也知道會有人問.
所以我也沒說到.
這個主題.
本身也想說.
有關移民的一個主題.
特別是流散的第二季.
我覺得移民本身.
是一件盼望.
有盼望的行動.
我想說的定義是什麼.
當你願意去計劃.
計劃移民是一個最大計劃.
是將來希望來的.

$^{2521}$去準備.
不是叫大家移民.
而是我們.
對於將來.
仍然有想象.
仍然有一種.
的嘗試.
去改變行動.
或者你去計劃.
這個就是移民.
其中一個可以實踐.
所以沒錯的.
移民的人可能對香港沒什麼盼望.
但沒問題的.
對香港是有盼望的.
但對我們對生命是有盼望的.
對世界是有盼望的.
所以我覺得.
移民不是一個有問題的.
也不是說你對香港沒盼望就不好.
這樣.
事實上你問我.
我對香港沒什麼盼望.
按照字面來看.
只不過因為我對生命有盼望.
所以我覺得我香港住.
我也是沒問題.
在香港住我也是有盼望.
我女兒也是有盼望.
對她來說.
所以我才這樣理解.
所以我覺得一個人有盼望.
可以選擇移民.
也可以選擇不移民.
兩個是沒關係的.
所以我都說.
移民只是一個嘗試去籌劃.
去計劃將來的事情.
所以你會這麼做.
當然對於.

$^{2561}$看回這一集的移民.
你也要問.
究竟我移民.
是純屬出於恐懼.
是不是.
出於絕望.
還是我嘗試去做一些.
新的事情去改變.
我生命的東西.
我想這是一個關鍵的問題.
純屬對於生命的絕望.
所以我離開.
還是我嘗試去尋找一個新的方向.
生日生活.
這個也提過了.
所以你嘗試去做一些新的事情.
也是一個盼望的表現.
所以我覺得對於移民.
對你來說.
盼望的課題.
正正就是在一個新環境里.
如何能夠去開展.
無論是為你的小朋友.
或是為你自己也好.
嘗試去.
不是等著再來.
而是可以在新的方向裡面去實踐信仰.
去尋找.
耶穌基督的豐盛.
不過是說.
耶穌的豐盛不等於移民.
但是移民是可以在一種實踐.
尋找上帝豐盛的一個方法.
所以我覺得.
兩件事是沒有衝突的.
我自己就.
看.
這個.
問題的過程當中.
回到最簡單的.

$^{2601}$就是其實我自己覺得.
每個選擇.
都是一個盼望.
你.
不要說移民這麼大步.
就算你去吃飯.
你選A餐和B餐.
你選了A餐也要當A餐有盼望.
你應該盼望A餐比B餐好吃.
你才選A餐.
所以選擇本身就是一個盼望.
我一直以來的想法.
移民這個套路.
或者這個場景是複雜了.
因為牽涉到你身份的重建.
所以你就會.
好像壓著很多東西.
認同John的做法.
和我們.
有些東西一直都在見證.
就是選擇在一個.
新的環境生活的弟兄姊妹.
其實是對.
他的選擇是有盼望的.
所以這個也是一個.
在過程當中建立.
但是.
難搞就是那種情意結.
那種情意結就牽涉到.
身份,複雜的地方就是身份.
大家好像身份不同了.
又或者好像有些東西.
情感上有些分離的時候.
就令到那件事.
感覺不好.
所以好像趕身.
移民都好像很.
敏感的話題.
但是我想你都聽過.
有些人移民之後.

$^{2641}$其實他不代表.
好過香港生活.
然後就說他自己選的.
所以過程當中.
要處理的過程當中.
不是說有沒有盼望.
這個話題那麼簡單.
所以我自己覺得那件事.
有沒有盼望或者那件事.
怎麼看盼望呢?回到你的身份.
是做什麼.
我自己的盼望仍然是.
通常都用基督徒的身份.
來處理的,就是明天會更好.
我自己覺得明天不會好.
因為這個世界會越來越差.
耶穌說過回來之前.
有很多問題,但是明天會更好.
明天會更好.
因為明天耶穌會回來.
這個好像很吊詭的方式就是.
我的身份做了決定.
我仍然沒好沒好一天.
我每一天期盼耶穌會回來.
這個是我的盼望,但是我不會盼望這個世界好.
因為世界本身就不會好.
所以移民也好.
或者做決定也好.
你去到那個地方做新的身份的時候.
就在那個新的身份當中.
做到盼望.
生活,一個基督徒原則.
做好本份就夠了.
既然你去到新的身份.
又何苦要想回舊的身份呢?.
想回舊身份的時候.
你一定很多時候會不當初.
這個就是.
手扶著泥巴往後看的人.
就是一個問題.

$^{2681}$我希望大家明白我的觀點.
有沒有問題?.
最後一個.
沒有就班機開就等下個月了.
下個月我們就會說什麼?.
說上班.
這個也是我們第二季裡面.
其中一個很實用的題目.
香港人很喜歡上班.
我們下個月再見.
拜拜.
\newpage



\section{}
\label{sec:JKdFzjAsLZY}
\textbf{《致餘民及流散者:給香港基督徒的神學八課》第二季第7課|20231101 [JKdFzjAsLZY]}
\newline
\newline
連結: \href{https://youtube.com/watch?v=JKdFzjAsLZY}{\texttt{ https://youtube.com/watch?v=JKdFzjAsLZY}} ~~~~ 語音日期: 2023-11-01 
\newline
\newline
\hyperref[sec:BDg16RM34JI]{\small{< < < PREV SERMON < < <}}
~
\hyperref[sec:index_chronic]{\small{[返順時目]}}
~
\hyperref[sec:index_scriptual]{\small{[返順卷目]}}
~
\hyperref[sec:_cxLnHL_TWQ]{\small{> > > NEXT SERMON > > >}}
\newline
\newline
$^{1}$我只想知道.
你到底是什麼意思.
我只想知道.
你到底是什麼意思.
我只想知道.
你到底是什麼意思.
我只想知道.
你到底是什麼意思.
我只想知道.
你到底是什麼意思.
我只想知道.
你到底是什麼意思.
我只想知道.
你到底是什麼意思.
我只想知道.
你到底是什麼意思.
我只想知道.
你到底是什麼意思.
我只想知道.
你到底是什麼意思.
我只想知道.
你到底是什麼意思.
我只想知道.
你到底是什麼意思.
我只想知道.
你到底是什麼意思.
我只想知道.
你到底是什麼意思.
我只想知道.
你到底是什麼意思.
我只想知道.
你到底是什麼意思.
我只想知道.
你到底是什麼意思.
我只想知道.
你到底是什麼意思.
我只想知道.
你到底是什麼意思.
我只想知道.
你到底是什麼意思.

$^{41}$我只想知道.
你到底是什麼意思.
我只想知道.
你到底是什麼意思.
我只想知道.
你到底是什麼意思.
我只想知道.
你到底是什麼意思.
我只想知道.
你到底是什麼意思.
我只想知道.
你到底是什麼意思.
我只想知道.
你到底是什麼意思.
我只想知道.
你到底是什麼意思.
我只想知道.
你到底是什麼意思.
我只想知道.
你到底是什麼意思.
我只想知道.
你到底是什麼意思.
我只想知道.
你到底是什麼意思.
我只想知道.
你到底是什麼意思.
我只想知道.
你到底是什麼意思.
我只想知道.
你到底是什麼意思.
我只想知道.
你到底是什麼意思.
我只想知道.
你到底是什麼意思.
我只想知道.
你到底是什麼意思.
我只想知道.
你到底是什麼意思.
我只想知道.
你到底是什麼意思.

$^{81}$我只想知道.
你到底是什麼意思.
我只想知道.
你到底是什麼意思.
我只想知道.
你到底是什麼意思.
我只想知道.
你到底是什麼意思.
我只想知道.
你到底是什麼意思.
我只想知道.
你到底是什麼意思.
我只想知道.
你到底是什麼意思.
我只想知道.
你到底是什麼意思.
我只想知道.
你到底是什麼意思.
我只想知道.
你到底是什麼意思.
我只想知道.
你到底是什麼意思.
我只想知道.
你到底是什麼意思.
我只想知道.
你到底是什麼意思.
我只想知道.
你到底是什麼意思.
我只想知道.
你到底是什麼意思.
我只想知道.
你到底是什麼意思.
我只想知道.
你到底是什麼意思.
我只想知道.
你到底是什麼意思.
我只想知道.
你到底是什麼意思.
我只想知道.
你到底是什麼意思.

$^{121}$我只想知道.
你到底是什麼意思.
我只想知道.
你到底是什麼意思.
我只想知道.
你到底是什麼意思.
我只想知道.
你到底是什麼意思.
我只想知道.
你到底是什麼意思.
我只想知道.
你到底是什麼意思.
我只想知道.
你到底是什麼意思.
我只想知道.
你到底是什麼意思.
我只想知道.
你到底是什麼意思.
我只想知道.
你到底是什麼意思.
我只想知道.
你到底是什麼意思.
我只想知道.
你到底是什麼意思.
我只想知道.
你到底是什麼意思.
我只想知道.
你到底是什麼意思.
我只想知道.
你到底是什麼意思.
我只想知道.
你到底是什麼意思.
我只想知道.
你到底是什麼意思.
我只想知道.
你到底是什麼意思.
我只想知道.
你到底是什麼意思.
我只想知道.
你到底是什麼意思.

$^{161}$我只想知道.
你到底是什麼意思.
我只想知道.
你到底是什麼意思.
我只想知道.
你到底是什麼意思.
我只想知道.
你到底是什麼意思.
我只想知道.
你到底是什麼意思.
我只想知道.
你到底是什麼意思.
我只想知道.
你到底是什麼意思.
我只想知道.
你到底是什麼意思.
我只想知道.
你到底是什麼意思.
我只想知道.
你到底是什麼意思.
我只想知道.
你到底是什麼意思.
我只想知道.
你到底是什麼意思.
我只想知道.
你到底是什麼意思.
我只想知道.
你到底是什麼意思.
我只想知道.
你到底是什麼意思.
我只想知道.
你到底是什麼意思.
我只想知道.
你到底是什麼意思.
我只想知道.
你到底是什麼意思.
我只想知道.
你到底是什麼意思.
我只想知道.
你到底是什麼意思.

$^{201}$我只想知道.
你到底是什麼意思.
我只想知道.
你到底是什麼意思.
我只想知道.
你到底是什麼意思.
我只想知道.
你到底是什麼意思.
我只想知道.
你到底是什麼意思.
我只想知道.
你到底是什麼意思.
我只想知道.
你到底是什麼意思.
我只想知道.
你到底是什麼意思.
我只想知道.
你到底是什麼意思.
我只想知道.
你到底是什麼意思.
我只想知道.
你到底是什麼意思.
我只想知道.
你到底是什麼意思.
我只想知道.
你到底是什麼意思.
我只想知道.
你到底是什麼意思.
我只想知道.
你到底是什麼意思.
我只想知道.
你到底是什麼意思.
我只想知道.
你到底是什麼意思.
我只想知道.
你到底是什麼意思.
我只想知道.
你到底是什麼意思.
我只想知道.
你到底是什麼意思.

$^{241}$我只想知道.
你到底是什麼意思.
我只想知道.
你到底是什麼意思.
我只想知道.
你到底是什麼意思.
我只想知道.
你到底是什麼意思.
我只想知道.
你到底是什麼意思.
我只想知道.
你到底是什麼意思.
我只想知道.
你到底是什麼意思.
我只想知道.
你到底是什麼意思.
我只想知道.
你到底是什麼意思.
我只想知道.
你到底是什麼意思.
我只想知道.
你到底是什麼意思.
我只想知道.
你到底是什麼意思.
我只想知道.
你到底是什麼意思.
我只想知道.
你到底是什麼意思.
我只想知道.
你到底是什麼意思.
我只想知道.
你到底是什麼意思.
我只想知道.
你到底是什麼意思.
我只想知道.
你到底是什麼意思.
我只想知道.
你到底是什麼意思.
我只想知道.
你到底是什麼意思.

$^{281}$我只想知道.
你到底是什麼意思.
我只想知道.
你到底是什麼意思.
我只想知道.
你到底是什麼意思.
我只想知道.
你到底是什麼意思.
我只想知道.
你到底是什麼意思.
我只想知道.
你到底是什麼意思.
我只想知道.
你到底是什麼意思.
我只想知道.
你到底是什麼意思.
我只想知道.
你到底是什麼意思.
我只想知道.
你到底是什麼意思.
我只想知道.
你到底是什麼意思.
我只想知道.
你到底是什麼意思.
我只想知道.
你到底是什麼意思.
我只想知道.
你到底是什麼意思.
我只想知道.
你到底是什麼意思.
我只想知道.
你到底是什麼意思.
我只想知道.
你到底是什麼意思.
我只想知道.
你到底是什麼意思.
我只想知道.
你到底是什麼意思.
我只想知道.
你到底是什麼意思.

$^{321}$我只想知道.
你到底是什麼意思.
我只想知道.
你到底是什麼意思.
我只想知道.
你到底是什麼意思.
我只想知道.
你到底是什麼意思.
我只想知道.
你到底是什麼意思.
我只想知道.
你到底是什麼意思.
我只想知道.
你到底是什麼意思.
我只想知道.
你到底是什麼意思.
我只想知道.
你到底是什麼意思.
我只想知道.
你到底是什麼意思.
我只想知道.
你到底是什麼意思.
我只想知道.
你到底是什麼意思.
我只想知道.
你到底是什麼意思.
我只想知道.
你到底是什麼意思.
我只想知道.
你到底是什麼意思.
我只想知道.
你到底是什麼意思.
我只想知道.
你到底是什麼意思.
我只想知道.
你到底是什麼意思.
我只想知道.
你到底是什麼意思.
我只想知道.
你到底是什麼意思.

$^{361}$我只想知道.
你到底是什麼意思.
我只想知道.
你到底是什麼意思.
我只想知道.
你到底是什麼意思.
我只想知道.
你到底是什麼意思.
我只想知道.
你到底是什麼意思.
我只想知道.
你到底是什麼意思.
我只想知道.
你到底是什麼意思.
我只想知道.
你到底是什麼意思.
我只想知道.
你到底是什麼意思.
我只想知道.
你到底是什麼意思.
我只想知道.
你到底是什麼意思.
我只想知道.
你到底是什麼意思.
我只想知道.
你到底是什麼意思.
我只想知道.
你到底是什麼意思.
我只想知道.
你到底是什麼意思.
我只想知道.
你到底是什麼意思.
我只想知道.
你到底是什麼意思.
我只想知道.
你到底是什麼意思.
我只想知道.
你到底是什麼意思.
我只想知道.
你到底是什麼意思.

$^{401}$我只想知道.
你到底是什麼意思.
我只想知道.
你到底是什麼意思.
我只想知道.
你到底是什麼意思.
我只想知道.
你到底是什麼意思.
我只想知道.
你到底是什麼意思.
我只想知道.
你到底是什麼意思.
我只想知道.
你到底是什麼意思.
我只想知道.
你到底是什麼意思.
我只想知道.
你到底是什麼意思.
我只想知道.
你到底是什麼意思.
我只想知道.
你到底是什麼意思.
我只想知道.
你到底是什麼意思.
我只想知道.
你到底是什麼意思.
我只想知道.
你到底是什麼意思.
我只想知道.
你到底是什麼意思.
我只想知道.
你到底是什麼意思.
我只想知道.
你到底是什麼意思.
我只想知道.
你到底是什麼意思.
我只想知道.
你到底是什麼意思.
我只想知道.
你到底是什麼意思.

$^{441}$我只想知道.
你到底是什麼意思.
我只想知道.
你到底是什麼意思.
我只想知道.
你到底是什麼意思.
我只想知道.
你到底是什麼意思.
我只想知道.
你到底是什麼意思.
我只想知道.
你到底是什麼意思.
我只想知道.
你到底是什麼意思.
我只想知道.
你到底是什麼意思.
我只想知道.
你到底是什麼意思.
我只想知道.
你到底是什麼意思.
我只想知道.
你到底是什麼意思.
我只想知道.
你到底是什麼意思.
我只想知道.
你到底是什麼意思.
我只想知道.
你到底是什麼意思.
我只想知道.
你到底是什麼意思.
我只想知道.
你到底是什麼意思.
我只想知道.
你到底是什麼意思.
我只想知道.
你到底是什麼意思.
我只想知道.
你到底是什麼意思.
我只想知道.
你到底是什麼意思.

$^{481}$我只想知道.
你到底是什麼意思.
我只想知道.
你到底是什麼意思.
我只想知道.
你到底是什麼意思.
我只想知道.
你到底是什麼意思.
我只想知道.
你到底是什麼意思.
我只想知道.
你到底是什麼意思.
我只想知道.
你到底是什麼意思.
我只想知道.
你到底是什麼意思.
我只想知道.
你到底是什麼意思.
我只想知道.
你到底是什麼意思.
我只想知道.
你到底是什麼意思.
我只想知道.
你到底是什麼意思.
我只想知道.
你到底是什麼意思.
我只想知道.
你到底是什麼意思.
我只想知道.
你到底是什麼意思.
我只想知道.
你到底是什麼意思.
我只想知道.
你到底是什麼意思.
我只想知道.
你到底是什麼意思.
我只想知道.
你到底是什麼意思.
我只想知道.
你到底是什麼意思.

$^{521}$我只想知道.
你到底是什麼意思.
我只想知道.
你到底是什麼意思.
我只想知道.
你到底是什麼意思.
我只想知道.
你到底是什麼意思.
我只想知道.
你到底是什麼意思.
我只想知道.
你到底是什麼意思.
我只想知道.
你到底是什麼意思.
我只想知道.
你到底是什麼意思.
我只想知道.
你到底是什麼意思.
我只想知道.
你到底是什麼意思.
我只想知道.
你到底是什麼意思.
我只想知道.
你到底是什麼意思.
我只想知道.
你到底是什麼意思.
我只想知道.
你到底是什麼意思.
我只想知道.
你到底是什麼意思.
我只想知道.
你到底是什麼意思.
我只想知道.
你到底是什麼意思.
我只想知道.
你到底是什麼意思.
我只想知道.
你到底是什麼意思.
我只想知道.
你到底是什麼意思.

$^{561}$我只想知道.
你到底是什麼意思.
我只想知道.
你到底是什麼意思.
我只想知道.
你到底是什麼意思.
我只想知道.
你到底是什麼意思.
我只想知道.
你到底是什麼意思.
我只想知道.
你到底是什麼意思.
我只想知道.
你到底是什麼意思.
我只想知道.
你到底是什麼意思.
我只想知道.
你到底是什麼意思.
我只想知道.
你到底是什麼意思.
我只想知道.
你到底是什麼意思.
我只想知道.
你到底是什麼意思.
我只想知道.
你到底是什麼意思.
我只想知道.
你到底是什麼意思.
我只想知道.
你到底是什麼意思.
我只想知道.
你到底是什麼意思.
我只想知道.
你到底是什麼意思.
我只想知道.
你到底是什麼意思.
我只想知道.
你到底是什麼意思.
我只想知道.
你到底是什麼意思.

$^{601}$我只想知道.
你到底是什麼意思.
我只想知道.
你到底是什麼意思.
我只想知道.
你到底是什麼意思.
我只想知道.
你到底是什麼意思.
我只想知道.
你到底是什麼意思.
我只想知道.
你到底是什麼意思.
我只想知道.
你到底是什麼意思.
我只想知道.
你到底是什麼意思.
我只想知道.
你到底是什麼意思.
我只想知道.
你到底是什麼意思.
我只想知道.
你到底是什麼意思.
我只想知道.
你到底是什麼意思.
我只想知道.
你到底是什麼意思.
我只想知道.
你到底是什麼意思.
我只想知道.
你到底是什麼意思.
我只想知道.
你到底是什麼意思.
我只想知道.
你到底是什麼意思.
我只想知道.
你到底是什麼意思.
我只想知道.
你到底是什麼意思.
我只想知道.
你到底是什麼意思.

$^{641}$我只想知道.
你到底是什麼意思.
我只想知道.
你到底是什麼意思.
我只想知道.
你到底是什麼意思.
我只想知道.
你到底是什麼意思.
我只想知道.
你到底是什麼意思.
我只想知道.
你到底是什麼意思.
我只想知道.
你到底是什麼意思.
我只想知道.
你到底是什麼意思.
我只想知道.
你到底是什麼意思.
我只想知道.
你到底是什麼意思.
我只想知道.
你到底是什麼意思.
我只想知道.
你到底是什麼意思.
我只想知道.
你到底是什麼意思.
我只想知道.
你到底是什麼意思.
我只想知道.
你到底是什麼意思.
我只想知道.
你到底是什麼意思.
我只想知道.
你到底是什麼意思.
我只想知道.
你到底是什麼意思.
我只想知道.
你到底是什麼意思.
我只想知道.
你到底是什麼意思.

$^{681}$我只想知道.
你到底是什麼意思.
我只想知道.
你到底是什麼意思.
我只想知道.
你到底是什麼意思.
我只想知道.
你到底是什麼意思.
我只想知道.
你到底是什麼意思.
我只想知道.
你到底是什麼意思.
我只想知道.
你到底是什麼意思.
我只想知道.
你到底是什麼意思.
我只想知道.
你到底是什麼意思.
我只想知道.
你到底是什麼意思.
我只想知道.
你到底是什麼意思.
我只想知道.
你到底是什麼意思.
我只想知道.
你到底是什麼意思.
我只想知道.
你到底是什麼意思.
我只想知道.
你到底是什麼意思.
我只想知道.
你到底是什麼意思.
我只想知道.
你到底是什麼意思.
我只想知道.
你到底是什麼意思.
我只想知道.
你到底是什麼意思.
我只想知道.
你到底是什麼意思.

$^{721}$我只想知道.
你到底是什麼意思.
我只想知道.
你到底是什麼意思.
我只想知道.
你到底是什麼意思.
我只想知道.
你到底是什麼意思.
我只想知道.
你到底是什麼意思.
我只想知道.
你到底是什麼意思.
我只想知道.
你到底是什麼意思.
我只想知道.
你到底是什麼意思.
我只想知道.
你到底是什麼意思.
我只想知道.
你到底是什麼意思.
我只想知道.
你到底是什麼意思.
我只想知道.
你到底是什麼意思.
我只想知道.
你到底是什麼意思.
我只想知道.
你到底是什麼意思.
我只想知道.
你到底是什麼意思.
我只想知道.
你到底是什麼意思.
我只想知道.
你到底是什麼意思.
我只想知道.
你到底是什麼意思.
我只想知道.
你到底是什麼意思.
我只想知道.
你到底是什麼意思.

$^{761}$我只想知道.
你到底是什麼意思.
我只想知道.
你到底是什麼意思.
我只想知道.
你到底是什麼意思.
我只想知道.
你到底是什麼意思.
我只想知道.
你到底是什麼意思.
我只想知道.
你到底是什麼意思.
我只想知道.
你到底是什麼意思.
我只想知道.
你到底是什麼意思.
我只想知道.
你到底是什麼意思.
我只想知道.
你到底是什麼意思.
我只想知道.
你到底是什麼意思.
我只想知道.
你到底是什麼意思.
我只想知道.
你到底是什麼意思.
我只想知道.
你到底是什麼意思.
我只想知道.
你到底是什麼意思.
我只想知道.
你到底是什麼意思.
我只想知道.
你到底是什麼意思.
我只想知道.
你到底是什麼意思.
我只想知道.
你到底是什麼意思.
我只想知道.
你到底是什麼意思.

$^{801}$我只想知道.
你到底是什麼意思.
我只想知道.
你到底是什麼意思.
我只想知道.
你到底是什麼意思.
我只想知道.
你到底是什麼意思.
我只想知道.
你到底是什麼意思.
我只想知道.
你到底是什麼意思.
我只想知道.
你到底是什麼意思.
我只想知道.
你到底是什麼意思.
我只想知道.
你到底是什麼意思.
我只想知道.
你到底是什麼意思.
我只想知道.
你到底是什麼意思.
我只想知道.
你到底是什麼意思.
我只想知道.
你到底是什麼意思.
我只想知道.
你到底是什麼意思.
我只想知道.
你到底是什麼意思.
我只想知道.
你到底是什麼意思.
我只想知道.
你到底是什麼意思.
我只想知道.
你到底是什麼意思.
我只想知道.
你到底是什麼意思.
我只想知道.
你到底是什麼意思.

$^{841}$我只想知道.
你到底是什麼意思.
我只想知道.
你到底是什麼意思.
我只想知道.
你到底是什麼意思.
我只想知道.
你到底是什麼意思.
我只想知道.
你到底是什麼意思.
我只想知道.
你到底是什麼意思.
我只想知道.
你到底是什麼意思.
我只想知道.
你到底是什麼意思.
我只想知道.
你到底是什麼意思.
我只想知道.
你到底是什麼意思.
我只想知道.
你到底是什麼意思.
我只想知道.
你到底是什麼意思.
我只想知道.
你到底是什麼意思.
我只想知道.
你到底是什麼意思.
我只想知道.
你到底是什麼意思.
我只想知道.
你到底是什麼意思.
我只想知道.
你到底是什麼意思.
我只想知道.
你到底是什麼意思.
我只想知道.
你到底是什麼意思.
我只想知道.
你到底是什麼意思.

$^{881}$我只想知道.
你到底是什麼意思.
我只想知道.
你到底是什麼意思.
我只想知道.
你到底是什麼意思.
我只想知道.
你到底是什麼意思.
我只想知道.
你到底是什麼意思.
我只想知道.
你到底是什麼意思.
我只想知道.
你到底是什麼意思.
我只想知道.
你到底是什麼意思.
我只想知道.
你到底是什麼意思.
我只想知道.
你到底是什麼意思.
我只想知道.
你到底是什麼意思.
我只想知道.
你到底是什麼意思.
我只想知道.
你到底是什麼意思.
我只想知道.
你到底是什麼意思.
我只想知道.
你到底是什麼意思.
我只想知道.
你到底是什麼意思.
我只想知道.
你到底是什麼意思.
我只想知道.
你到底是什麼意思.
我只想知道.
你到底是什麼意思.
我只想知道.
你到底是什麼意思.

$^{921}$我只想知道.
你到底是什麼意思.
我只想知道.
你到底是什麼意思.
我只想知道.
你到底是什麼意思.
我只想知道.
你到底是什麼意思.
我只想知道.
你到底是什麼意思.
我只想知道.
你到底是什麼意思.
我只想知道.
你到底是什麼意思.
我只想知道.
你到底是什麼意思.
我只想知道.
你到底是什麼意思.
我只想知道.
你到底是什麼意思.
我只想知道.
你到底是什麼意思.
我只想知道.
你到底是什麼意思.
我只想知道.
你到底是什麼意思.
我只想知道.
你到底是什麼意思.
我只想知道.
你到底是什麼意思.
我只想知道.
你到底是什麼意思.
我只想知道.
你到底是什麼意思.
我只想知道.
你到底是什麼意思.
我只想知道.
你到底是什麼意思.
我只想知道.
你到底是什麼意思.

$^{961}$我只想知道.
你到底是什麼意思.
我只想知道.
你到底是什麼意思.
我只想知道.
你到底是什麼意思.
我只想知道.
你到底是什麼意思.
我只想知道.
你到底是什麼意思.
我只想知道.
你到底是什麼意思.
我只想知道.
你到底是什麼意思.
我只想知道.
你到底是什麼意思.
我只想知道.
你到底是什麼意思.
我只想知道.
你到底是什麼意思.
我只想知道.
你到底是什麼意思.
我只想知道.
你到底是什麼意思.
我只想知道.
你到底是什麼意思.
我只想知道.
你到底是什麼意思.
我只想知道.
你到底是什麼意思.
我只想知道.
你到底是什麼意思.
我只想知道.
你到底是什麼意思.
我只想知道.
你到底是什麼意思.
我只想知道.
你到底是什麼意思.
我只想知道.
你到底是什麼意思.

$^{1001}$我只想知道.
你到底是什麼意思.
我只想知道.
你到底是什麼意思.
我只想知道.
你到底是什麼意思.
我只想知道.
你到底是什麼意思.
我只想知道.
你到底是什麼意思.
我只想知道.
你到底是什麼意思.
我只想知道.
你到底是什麼意思.
我只想知道.
你到底是什麼意思.
我只想知道.
你到底是什麼意思.
我只想知道.
你到底是什麼意思.
我只想知道.
你到底是什麼意思.
我只想知道.
你到底是什麼意思.
我只想知道.
你到底是什麼意思.
我只想知道.
你到底是什麼意思.
我只想知道.
你到底是什麼意思.
我只想知道.
你到底是什麼意思.
我只想知道.
你到底是什麼意思.
我只想知道.
你到底是什麼意思.
我只想知道.
你到底是什麼意思.
我只想知道.
你到底是什麼意思.

$^{1041}$我只想知道.
你到底是什麼意思.
我只想知道.
你到底是什麼意思.
我只想知道.
你到底是什麼意思.
我只想知道.
你到底是什麼意思.
我只想知道.
你到底是什麼意思.
我只想知道.
你到底是什麼意思.
我只想知道.
你到底是什麼意思.
我只想知道.
你到底是什麼意思.
我只想知道.
你到底是什麼意思.
我只想知道.
你到底是什麼意思.
我只想知道.
你到底是什麼意思.
我只想知道.
你到底是什麼意思.
我只想知道.
你到底是什麼意思.
我只想知道.
你到底是什麼意思.
我只想知道.
你到底是什麼意思.
我只想知道.
你到底是什麼意思.
我只想知道.
你到底是什麼意思.
我只想知道.
你到底是什麼意思.
我只想知道.
你到底是什麼意思.
我只想知道.
你到底是什麼意思.

$^{1081}$我只想知道.
你到底是什麼意思.
我只想知道.
你到底是什麼意思.
我只想知道.
你到底是什麼意思.
我只想知道.
你到底是什麼意思.
我只想知道.
你到底是什麼意思.
我只想知道.
你到底是什麼意思.
我只想知道.
你到底是什麼意思.
我只想知道.
你到底是什麼意思.
我只想知道.
你到底是什麼意思.
我只想知道.
你到底是什麼意思.
我只想知道.
你到底是什麼意思.
我只想知道.
你到底是什麼意思.
我只想知道.
你到底是什麼意思.
我只想知道.
你到底是什麼意思.
我只想知道.
你到底是什麼意思.
我只想知道.
你到底是什麼意思.
我只想知道.
你到底是什麼意思.
我只想知道.
你到底是什麼意思.
我只想知道.
你到底是什麼意思.
我只想知道.
你到底是什麼意思.

$^{1121}$我只想知道.
你到底是什麼意思.
我只想知道.
你到底是什麼意思.
我只想知道.
你到底是什麼意思.
我只想知道.
你到底是什麼意思.
我只想知道.
你到底是什麼意思.
我只想知道.
你到底是什麼意思.
我只想知道.
你到底是什麼意思.
我只想知道.
你到底是什麼意思.
我只想知道.
你到底是什麼意思.
我只想知道.
你到底是什麼意思.
我只想知道.
你到底是什麼意思.
我只想知道.
你到底是什麼意思.
我只想知道.
你到底是什麼意思.
我只想知道.
你到底是什麼意思.
我只想知道.
你到底是什麼意思.
我只想知道.
你到底是什麼意思.
我只想知道.
你到底是什麼意思.
我只想知道.
你到底是什麼意思.
我只想知道.
你到底是什麼意思.
我只想知道.
你到底是什麼意思.

$^{1161}$我只想知道.
你到底是什麼意思.
我只想知道.
你到底是什麼意思.
我只想知道.
你到底是什麼意思.
我只想知道.
你到底是什麼意思.
我只想知道.
你到底是什麼意思.
我只想知道.
你到底是什麼意思.
我只想知道.
你到底是什麼意思.
我只想知道.
你到底是什麼意思.
我只想知道.
你到底是什麼意思.
我只想知道.
你到底是什麼意思.
我只想知道.
你到底是什麼意思.
我只想知道.
你到底是什麼意思.
我只想知道.
你到底是什麼意思.
我只想知道.
你到底是什麼意思.
我只想知道.
你到底是什麼意思.
我只想知道.
你到底是什麼意思.
我只想知道.
你到底是什麼意思.
我只想知道.
你到底是什麼意思.
我只想知道.
你到底是什麼意思.
我只想知道.
你到底是什麼意思.

$^{1201}$我只想知道.
你到底是什麼意思.
我只想知道.
你到底是什麼意思.
我只想知道.
你到底是什麼意思.
我只想知道.
你到底是什麼意思.
我只想知道.
你到底是什麼意思.
我只想知道.
你到底是什麼意思.
我只想知道.
你到底是什麼意思.
我只想知道.
你到底是什麼意思.
我只想知道.
你到底是什麼意思.
我只想知道.
你到底是什麼意思.
我只想知道.
你到底是什麼意思.
我只想知道.
你到底是什麼意思.
我只想知道.
你到底是什麼意思.
我只想知道.
你到底是什麼意思.
我只想知道.
你到底是什麼意思.
我只想知道.
你到底是什麼意思.
我只想知道.
你到底是什麼意思.
我只想知道.
你到底是什麼意思.
我只想知道.
你到底是什麼意思.
我只想知道.
你到底是什麼意思.

$^{1241}$我只想知道.
你到底是什麼意思.
我只想知道.
你到底是什麼意思.
我只想知道.
你到底是什麼意思.
我只想知道.
你到底是什麼意思.
我只想知道.
你到底是什麼意思.
我只想知道.
你到底是什麼意思.
我只想知道.
你到底是什麼意思.
我只想知道.
你到底是什麼意思.
我只想知道.
你到底是什麼意思.
我只想知道.
你到底是什麼意思.
我只想知道.
你到底是什麼意思.
我只想知道.
你到底是什麼意思.
我只想知道.
你到底是什麼意思.
我只想知道.
你到底是什麼意思.
我只想知道.
你到底是什麼意思.
我只想知道.
你到底是什麼意思.
我只想知道.
你到底是什麼意思.
我只想知道.
你到底是什麼意思.
我只想知道.
你到底是什麼意思.
我只想知道.
你到底是什麼意思.

$^{1281}$我只想知道.
你到底是什麼意思.
我只想知道.
你到底是什麼意思.
我只想知道.
你到底是什麼意思.
我只想知道.
你到底是什麼意思.
我只想知道.
你到底是什麼意思.
我只想知道.
你到底是什麼意思.
我只想知道.
你到底是什麼意思.
我只想知道.
你到底是什麼意思.
我只想知道.
你到底是什麼意思.
我只想知道.
你到底是什麼意思.
我只想知道.
你到底是什麼意思.
我只想知道.
你到底是什麼意思.
我只想知道.
你到底是什麼意思.
我只想知道.
你到底是什麼意思.
我只想知道.
你到底是什麼意思.
我只想知道.
你到底是什麼意思.
我只想知道.
你到底是什麼意思.
我只想知道.
你到底是什麼意思.
我只想知道.
你到底是什麼意思.
我只想知道.
你到底是什麼意思.

$^{1321}$我只想知道.
你到底是什麼意思.
我只想知道.
你到底是什麼意思.
我只想知道.
你到底是什麼意思.
我只想知道.
你到底是什麼意思.
我只想知道.
你到底是什麼意思.
我只想知道.
你到底是什麼意思.
我只想知道.
你到底是什麼意思.
我只想知道.
你到底是什麼意思.
我只想知道.
你到底是什麼意思.
我只想知道.
你到底是什麼意思.
我只想知道.
你到底是什麼意思.
我只想知道.
你到底是什麼意思.
我只想知道.
你到底是什麼意思.
我只想知道.
你到底是什麼意思.
我只想知道.
你到底是什麼意思.
我只想知道.
你到底是什麼意思.
我只想知道.
你到底是什麼意思.
我只想知道.
你到底是什麼意思.
我只想知道.
你到底是什麼意思.
我只想知道.
你到底是什麼意思.

$^{1361}$我只想知道.
你到底是什麼意思.
我只想知道.
你到底是什麼意思.
我只想知道.
你到底是什麼意思.
我只想知道.
你到底是什麼意思.
我只想知道.
你到底是什麼意思.
我只想知道.
你到底是什麼意思.
我只想知道.
你到底是什麼意思.
我只想知道.
你到底是什麼意思.
我只想知道.
你到底是什麼意思.
我只想知道.
你到底是什麼意思.
我只想知道.
你到底是什麼意思.
我只想知道.
你到底是什麼意思.
我只想知道.
你到底是什麼意思.
我只想知道.
你到底是什麼意思.
我只想知道.
你到底是什麼意思.
我只想知道.
你到底是什麼意思.
我只想知道.
你到底是什麼意思.
我只想知道.
你到底是什麼意思.
我只想知道.
你到底是什麼意思.
我只想知道.
你到底是什麼意思.

$^{1401}$我只想知道.
你到底是什麼意思.
我只想知道.
你到底是什麼意思.
我只想知道.
你到底是什麼意思.
我只想知道.
你到底是什麼意思.
我只想知道.
你到底是什麼意思.
我只想知道.
你到底是什麼意思.
我只想知道.
你到底是什麼意思.
我只想知道.
你到底是什麼意思.
我只想知道.
你到底是什麼意思.
我只想知道.
你到底是什麼意思.
我只想知道.
你到底是什麼意思.
我只想知道.
你到底是什麼意思.
我只想知道.
你到底是什麼意思.
我只想知道.
你到底是什麼意思.
我只想知道.
你到底是什麼意思.
我只想知道.
你到底是什麼意思.
我只想知道.
你到底是什麼意思.
我只想知道.
你到底是什麼意思.
我只想知道.
你到底是什麼意思.
我只想知道.
你到底是什麼意思.

$^{1441}$我只想知道.
你到底是什麼意思.
我只想知道.
你到底是什麼意思.
我只想知道.
你到底是什麼意思.
我只想知道.
你到底是什麼意思.
我只想知道.
你到底是什麼意思.
我只想知道.
你到底是什麼意思.
我只想知道.
你到底是什麼意思.
我只想知道.
你到底是什麼意思.
我只想知道.
你到底是什麼意思.
我只想知道.
你到底是什麼意思.
我只想知道.
你到底是什麼意思.
我只想知道.
你到底是什麼意思.
我只想知道.
你到底是什麼意思.
我只想知道.
你到底是什麼意思.
我只想知道.
你到底是什麼意思.
我只想知道.
你到底是什麼意思.
我只想知道.
你到底是什麼意思.
我只想知道.
你到底是什麼意思.
我只想知道.
你到底是什麼意思.
我只想知道.
你到底是什麼意思.

$^{1481}$我只想知道.
你到底是什麼意思.
我只想知道.
你到底是什麼意思.
我只想知道.
你到底是什麼意思.
我只想知道.
你到底是什麼意思.
我只想知道.
你到底是什麼意思.
我只想知道.
你到底是什麼意思.
我只想知道.
你到底是什麼意思.
我只想知道.
你到底是什麼意思.
我只想知道.
你到底是什麼意思.
我只想知道.
你到底是什麼意思.
我只想知道.
你到底是什麼意思.
我只想知道.
你到底是什麼意思.
我只想知道.
你到底是什麼意思.
我只想知道.
你到底是什麼意思.
我只想知道.
你到底是什麼意思.
我只想知道.
你到底是什麼意思.
我只想知道.
你到底是什麼意思.
我只想知道.
你到底是什麼意思.
我只想知道.
你到底是什麼意思.
我只想知道.
你到底是什麼意思.

$^{1521}$我只想知道.
你到底是什麼意思.
我只想知道.
你到底是什麼意思.
我只想知道.
你到底是什麼意思.
我只想知道.
你到底是什麼意思.
我只想知道.
你到底是什麼意思.
我只想知道.
你到底是什麼意思.
我只想知道.
你到底是什麼意思.
我只想知道.
你到底是什麼意思.
我只想知道.
你到底是什麼意思.
我只想知道.
你到底是什麼意思.
我只想知道.
你到底是什麼意思.
我只想知道.
你到底是什麼意思.
我只想知道.
你到底是什麼意思.
我只想知道.
你到底是什麼意思.
我只想知道.
你到底是什麼意思.
我只想知道.
你到底是什麼意思.
我只想知道.
你到底是什麼意思.
我只想知道.
你到底是什麼意思.
我只想知道.
你到底是什麼意思.
我只想知道.
你到底是什麼意思.

$^{1561}$我只想知道.
你到底是什麼意思.
我只想知道.
你到底是什麼意思.
我只想知道.
你到底是什麼意思.
我只想知道.
你到底是什麼意思.
我只想知道.
你到底是什麼意思.
我只想知道.
你到底是什麼意思.
我只想知道.
你到底是什麼意思.
我只想知道.
你到底是什麼意思.
我只想知道.
你到底是什麼意思.
我只想知道.
你到底是什麼意思.
我只想知道.
你到底是什麼意思.
我只想知道.
你到底是什麼意思.
我只想知道.
你到底是什麼意思.
我只想知道.
你到底是什麼意思.
我只想知道.
你到底是什麼意思.
我只想知道.
你到底是什麼意思.
我只想知道.
你到底是什麼意思.
我只想知道.
你到底是什麼意思.
我只想知道.
你到底是什麼意思.
我只想知道.
你到底是什麼意思.

$^{1601}$我只想知道.
你到底是什麼意思.
我只想知道.
你到底是什麼意思.
我只想知道.
你到底是什麼意思.
我只想知道.
你到底是什麼意思.
我只想知道.
你到底是什麼意思.
我只想知道.
你到底是什麼意思.
我只想知道.
你到底是什麼意思.
我只想知道.
你到底是什麼意思.
我只想知道.
你到底是什麼意思.
我只想知道.
你到底是什麼意思.
我只想知道.
你到底是什麼意思.
我只想知道.
你到底是什麼意思.
我只想知道.
你到底是什麼意思.
我只想知道.
你到底是什麼意思.
我只想知道.
你到底是什麼意思.
我只想知道.
你到底是什麼意思.
我只想知道.
你到底是什麼意思.
我只想知道.
你到底是什麼意思.
我只想知道.
你到底是什麼意思.
我只想知道.
你到底是什麼意思.

$^{1641}$我只想知道.
你到底是什麼意思.
我只想知道.
你到底是什麼意思.
我只想知道.
你到底是什麼意思.
我只想知道.
你到底是什麼意思.
我只想知道.
你到底是什麼意思.
我只想知道.
你到底是什麼意思.
我只想知道.
你到底是什麼意思.
我只想知道.
你到底是什麼意思.
我只想知道.
你到底是什麼意思.
我只想知道.
你到底是什麼意思.
我只想知道.
你到底是什麼意思.
我只想知道.
你到底是什麼意思.
我只想知道.
你到底是什麼意思.
我只想知道.
你到底是什麼意思.
我只想知道.
你到底是什麼意思.
我只想知道.
你到底是什麼意思.
我只想知道.
你到底是什麼意思.
我只想知道.
你到底是什麼意思.
我只想知道.
你到底是什麼意思.
我只想知道.
你到底是什麼意思.

$^{1681}$我只想知道.
你到底是什麼意思.
我只想知道.
你到底是什麼意思.
我只想知道.
你到底是什麼意思.
我只想知道.
你到底是什麼意思.
我只想知道.
你到底是什麼意思.
我只想知道.
你到底是什麼意思.
我只想知道.
你到底是什麼意思.
我只想知道.
你到底是什麼意思.
我只想知道.
你到底是什麼意思.
我只想知道.
你到底是什麼意思.
我只想知道.
你到底是什麼意思.
我只想知道.
你到底是什麼意思.
我只想知道.
你到底是什麼意思.
我只想知道.
你到底是什麼意思.
我只想知道.
你到底是什麼意思.
我只想知道.
你到底是什麼意思.
我想我們今天回到.
最基本的內容.
我們看回一篇經文.
有關創世紀的.
上帝叫你勞動的經文.
他說什麼呢.
地必為你的猿固受咒助.
你必終身勞苦才能從地裡得吃.

$^{1721}$地必給你長出荊棘和泥澤來.
你也要吃田間的菜蔬.
你必汗流滿面才得糊口.
直到你歸了土.
因為你是出於土而出的.
這是上帝對於.
阿當和夏娃的懲罰.
因為犯罪墮落的緣故.
所以阿當夏娃面對這樣的情況.
是什麼呢.
就是你要有勞動才能得吃.
以前沒有的.
犯罪前阿當夏娃.
基本上人類的狀況.
你不需要勞動.
你都能夠糊口.
你不需要賺錢.
你都能夠得到食物.
所以他所說的不是純粹工作辛苦了.
而是說你沒有錢.
或者你沒有資源.
如果你有資源.
你不需要賺錢.
你就不需要工作.
所以我想重點就是.
經文不是嘗試去.
給予一個工作的神聖理由.
相反.
它不是一個賦予工作.
一種神聖意義的經文.
它只不過是間接地.
消極地解釋.
為什麼你要工作.
為什麼要工作.
先問問現場觀眾.
工作有什麼原因.
最重要的是什麼.
就是錢.
工作的定義是什麼.
你不需要辦公室.

$^{1761}$你不需要老闆.
你不需要任何放假.
有錢就行.
有錢就是工作.
沒有錢不是工作.
沒有錢就叫做義工.
沒有錢就叫做興趣.
最能夠定義的工作就是錢.
所謂工作就是.
你有回報的一個交易.
今天我想講的不是市區.
而是工作本身的定義就是這樣.
工作就是一個.
讓你能夠得到金錢回報的一件事.
所以黃子博講得很好.
就是賺錢.
賺錢或者以前固定的叫戶口.
戶口或者賺錢.
這正是工作的核心定義.
如果沒有錢的話.
這不是工作.
工作就是賺錢的一件事.
所以基民說為什麼要賺錢.
為什麼要工作.
因為你不能夠隨便得食.
你需要透過勞動.
能夠換取.
今天所講的金錢.
以前不是.
換取金錢才能夠得食.
所以就叫做賺錢或者戶口.
所以賺錢或者戶口的工作.
有沒有意義呢.
或者它是否必須有一個屬靈的意義呢.
我就問這個問題.
或者是什麼情況.
所以劉基民告訴我.
不是去嘗試去給予工作一個定義.
或者一個屬靈的定義.
而是去反面或者間接地.

$^{1801}$解釋為什麼要工作.
因為有缺陷.
因為你不是什麼都有.
因為你沒有資源.
因為你家裡沒有東西吃.
所以你需要工作來換取資源.
這個就是世界的法則.
這個也是當紅夏娃犯罪之後出現的法則.
所以簡單來講.
這個基民告訴我.
當人類墮落之後.
這個大地被咒坐之後.
工作就是間接地隨之而生.
沒錯.
工作未必一定是一個上帝直接的懲罰.
但是工作是一個間接上出現的東西.
因為上帝吸予大地被咒坐之後.
你會出現scarcity的問題.
你有資源短缺的問題.
因此你要間接地工作.
找食物戶口來換取資源.
所以關係就是這樣.
因為犯罪緣故.
你不是必然有資源.
因此你要工作.
所以你才有資源.
所以當然有錢的人是不需要的.
因為他本身有資源.
他不一定工作.
但是很多人都是這樣.
他需要工作來找食物.
所以這就是我們所面對的情況.
你要工作來換取金錢.
所謂工作的時候.
我之前也提過.
工作永遠都是帶著工具性目的性.
為什麼你要上班.
因為你為了某些目的.
就是為了賺錢.
不是說你沒有使命.

$^{1841}$不一定是你沒有屬靈意義.
你是需要來找食物賺錢.
所以當你做一件事.
為了賺錢的時候.
你會面對很多問題.
你會做一些你不想做的事.
做一些很無聊的事.
做一些你覺得沒什麼貢獻的事.
工作是否有很多貢獻呢.
我想十個裡面的工作.
其實都是為了幫老闆賺錢.
意義就是幫老闆賺錢.
所以就是這樣.
不否定工作是有意義的.
但是我們不能夠將這些變成定義.
不能夠說工作的定義就是能夠幫助人.
因為很多時候工作都幫不了人.
幫了你老闆賺錢.
就是那個人.
老闆就是那個人.
所以我想我們要讀穿這件事.
很多工作都不一定有貢獻和意義.
這個情況是荒謬的.
但正正就是因為人類墮落之後的情況.
因為大罪的緣故.
所以我們面對一個很荒謬的情況.
因為純粹是來找食物.
做一些很沒有意義的事.
一些很不合理的事.
但你能夠得到回報.
所以天堂有沒有工作呢.
天堂是沒有工作的.
意思不是說你不用工作.
而是你不需要為了賺錢換取東西去工作.
所以天堂是有無限資源.
或者沒有stress的問題.
所以我們工作裡面不需要來到去.
因為想找資源去做一些不想做的事.
這個就是天堂.
所以大家很羨慕我們的社會.

$^{1881}$這個就是我們來坦白的事情.
我認為工作是大地被咒坐之後人類的狀況.
因為資源短缺.
你被逼要做一些純粹賺錢的事.
當然不否定.
不否定你的工作有意義和使命.
這樣說不是否定一件事.
但這不是必然的.
我們不要把這看為工作的定義.
它未必一定是上帝對你的呼召.
未必是一些有意義的事.
未必是一些你很想做的事.
當然有人會這樣做.
我做一個老師.
我真的很想做一個老師.
我做一些我想做的事.
但這不是一個定義.
你不能將它看為定義.
因為很多人不是這樣.
當然我們盡可能都想做這些事.
既然我們都要上班.
為什麼不找一些你想做的事.
為什麼不找一些我自己呼召的事.
為什麼不做一些我有意義的事.
這絕對是理想.
但我們不能將理想成為一個工作的定義.
所以我嘗試分開.
不用路德那套看為天職.
或者將工作看為本身有意義的事.
那怎麼辦呢?好像很灰.
問題是.
雖然我們說工作不一定有意義.
工作不一定是上帝對你的召命.
但我想說我們的人生是可以有意義的.
我們的人生是可以有上帝的召命和呼召的.
所以我們有些人.
很多不同的情況.
有些人的工作是上帝對他的呼召.
他做了一輩子.
這是其中一個例子.

$^{1921}$有些人或者你.
上帝對你的呼召和理想.
不是在工作里.
我的工作是為了賺錢.
9至9晚51至55.
我出賣我的靈魂.
然後我賺錢回來.
我晚上可以做我想做的事.
太多這些例子.
我想說這個不是一個.
我認為這個不是一個不正常的情況.
如果你說工作是一個就在後間結果的時候.
你早上九晚五做一些沒有意義的事.
其實你就會明白為什麼.
因為這是一個不可領略的情況.
這個世界的狀況.
但你仍然要尋找你的理想.
尋找你自己想做的事.
或者你活出你生命的意義.
即是說你的工作不一定能夠實踐你當下意義的時候.
你仍然要相信和知道.
我的生命是有意義的.
我的工作未必能夠幫我這樣做.
但它能夠讓我有錢.
能夠做其他事.
所以我們嘗試將工作的意義和生命的意義來區分.
你的生命意義未必一定是實踐在工作里.
重復一次.
如果是是很好的.
但如果不是的話.
不代表你不正常.
或者你的生命沒有意義.
最近我看了一部Netflix的電影.
先講完這裡.
其實有一個很寶貴的字眼.
就是Voluntary Work.
叫做義工.
如果工作的定義是有錢的時候.
你會發現人生里有很多工作是沒有錢的.
但你很喜歡做.

$^{1961}$那些不被賠償的工作是很辛苦的.
你很喜歡去行山.
行山是很辛苦的.
但你是願意去做的.
教會的敬拜服侍.
你很辛苦來到去預備.
但你仍然覺得很滿足.
所以我想說.
工作不一定是用辛苦來定義工作.
我用賠償和不賠償來定義工作.
但一些不賠償的工作.
Voluntary的工作.
你仍然很樂於去做.
譬如你是媽媽.
媽媽是一個最強大的Voluntary Work.
你不收錢的.
但你是會願意去做的.
很多的事情你都會願意去做.
所以反而我們生命上的意義.
往往很多人都是在Voluntary Work裡面.
可能我晚上.
或者Batman.
Batman晚上做Voluntary Work.
那些不收錢的.
但很有意義的.
平時做Bruce Willis的時候.
他賺了錢.
但這不是他的生命意義.
明白我的意思嗎.
很多人都是這樣.
工作以外才是真正的生命實踐.
所以我想再重復一次.
你可以正式做意義的事.
但未必一定是這樣.
所以更加要強調.
義工其實都很重要.
重點是很多事情都辛苦.
但你想做的事是否重點.
是否收費都不是那麼重要.
而是你自己想做什麼.

$^{2001}$所以往往你為了支持.
你做Voluntary Work.
你會做一些沒有意義的工作.
近期我看了一部戲.
原來沒有了.
沒有了一張圖片.
但我看過近期Netflix有一部.
叫The Wonderful Story of Henry Su.
沒有了圖片.
上回給大家看.
故事是講一個很有錢的人.
基本上他都是無憂無慮.
不需要工作的有錢人.
他無端端看了一本書.
是一些印度瑜伽大師.
故事是講他有些方法.
可以讓他看穿背後的東西.
可以透視.
有錢人花了好幾年時間.
不斷去練透視的方法.
練成功了.
他就可以去賭場里.
透視看啤牌背後的牌.
那一刻他才開始他的生命.
記住他不是為了賺錢.
因為他已經有很多錢.
所以他才開始做他生命意義的事.
他每天都去賭場里.
賺夠好.
賭錢看穿牌.
每次賺兩萬英鎊就走.
怕被人打.
每天都很勤力.
很日常地賺兩萬五千英鎊.
然後就離開.
把錢捐給人開兒童院和醫院.
他從那時起才開始上班.
他每天都很勤力地上班.
在賭場里賺錢.
然後把錢捐給別人.

$^{2041}$我覺得這是一個尋找生命意義的故事.
一個嘗試用一個角度.
去理解什麼是工作的故事.
其實我們工作的意義.
本身更加要問的是你的生命意義在哪裡.
工作是你實現生命的其中一部分.
它可以是一個很好的部分.
但是那時候.
voluntary work才是你生命裡面更加重要的一部分.
所以我們嘗試這麼說.
上帝給我們的照明.
未必一定在工作裡面.
但是我們是有照明的.
上帝給我們的照明是有工作的.
那些工作未必是要付的工作.
但是是工作來的.
一個你可能會流汗.
你要辛苦的工作.
但是你願意做的工作.
所以我嘗試重寫這幾點.
如果我們這麼看的時候.
什麼是工作呢.
Work is an indirect result of the form.
它未必是直接因為墮落的緣故.
但是一個間接的原因.
因為墮落的緣故.
所以你的工作.
你是一個很不理想.
很荒謬的世界裡面.
做一些很不想做的事情.
所以工作裡面的那個基本質.
Work is a place to make money.
這個是名正言順.
都有道理的定義.
工作就是為了make money.
好讓你能夠生活下去.
或者你的子女能夠生活下去.
第三.
一些神聖的呼召.
不比其他的職業更加崇高.

$^{2081}$但是想說一件事.
Work is not necessarily a calling.
工作不一定是calling來的.
但是你的生命是calling的.
但是工作不一定是calling的實踐.
第四.
Redemption extends to all of life.
only in this sense including our world.
因為我們的生命是被救贖的時候.
所以工作才是救贖的部分.
所以你要看的不是工作.
而是什麼.
而是你的生命.
所以我會改後面那幾句.
There are indications that some of our lives will be present in the new heaven and the new earth.
你的生命會彰顯新天新地的模樣.
就此言言,你的工作才會這樣彰顯出來.
We are called to glorify God in our life.
所以不要把定義狹雜地放在工作里.
而是放在你的生命里.
因為你的生命是有意義的.
要榮耀上帝的.
是值得彰顯上帝的生命的.
這樣來看你的工作.
所以更加宏觀些,闊些,遠些去看你的生命.
最後我想說一些工作倫理.
我想其實有很多很困難的倫理.
公司賺錢有些不太誠實.
或者你所做的東西似乎沒有什麼特別的貢獻.
或者你遇到很多倫理問題.
我想說,今天處理不了的.
你可以嘗試在小組裡面打聽一下.
這些兩難.
工作的開單似乎又不太正確.
你發現自己在幫JPES賺錢.
我不知道.
很多這些困難.
今天不說這些.
如果我們說工作倫理是一個基督徒倫理.
就是說如果工作是一個你本身做人應該的倫理.

$^{2121}$其實工作倫理也不是太複雜.
你怎麼做人,就是在工作裡面怎麼做工作.
所以我嘗試寫一些比較無聊的三點.
工作有什麼倫理呢?基督徒.
第一,就是要負責任.
就是做事要負責任.
第二,有交代.
第三,Facebook和WhatsApp.
其實沒什麼特別的.
我看到有基督徒是基督徒.
但是不Facebook和WhatsApp.
我也是.
基督徒,但沒什麼交代.
香港就是這麼簡單.
既然工作是一個合約,是一個交換.
那你就應該做好這件事.
就是這麼簡單.
不說你是不是要在公司裡面唱歌.
還是怎麼建設上帝.
但請你先做好工作.
這個就是基督徒最基本要做的事.
應該稱職地負責工作的責任.
這些事,走的時候記住抹上水餅.
做好事才讓別人做.
這些很簡單的事.
可能太多,是基督徒來的.
但做事是要射箭的.
或者不斷地推來推去.
所以做好這些基本的倫理.
就是做好工作里應該做的表現.
再在我們的生命裡面彰顯我們的信仰.
不是說你在公司裡面唱歌不對.
是要做的.
但沒有一個工作就是為了傳福音的倫理.
因為你的人就是傳福音.
所以你自然在工作裡面會建立基督.
所以沒有一個很具體的工作倫理.
因為每一個工作倫理都是我們生命倫理的處境.
當然有很多很特別的工作處境.
像剛才說的跑數,說謊是很難處理的.

$^{2161}$但我們人生裡面的倫理其實都是很簡單的.
你既然做了基督徒.
回到達哥第一科裡面.
基督徒的呼召就是去見證耶穌.
所以我們基本上是懷著這種召命和身份去做人.
不過你會上班的.
你會做一些不同的事.
你會遇到不同的人.
所以這樣去理解的時候.
我覺得沒有一套很特別要讀的工作倫理.
而是如何做一個人.
你會很簡單地去想.
當然我也有很多問題要談.
特別是大家有很多奇能合智.
大家會問工作裡面遇到不同的困難.
怎麼辦.
當然我們可以談一下.
但我想更重要的是.
做一個好的基督徒之前.
大家要做一個好的人.
在職場裡面.
做一個正常的老闆.
或者正常的員工.
你不一定要做得特別好.
但請你負責工作裡面的責任.
等等這樣.
我們有些對話空間.
看看大家有沒有問題想問.
有關工作上的事.
我先叫上一位.
潘sir.
(潘sir的工作人員).
已經上班了.
很羨慕你的工作.
飛來飛去.
應該對於各位.
有沒有人今天不用上班.
有.
你是特別一點.
剛剛下班.

$^{2201}$就來到這裡.
應該很累了.
如果不用上班.
大家會不會來呢.
都會來的.
很多人下班就很累.
就去吃喝快樂.
去做回饋.
但有時我自己在小組討論.
如果上班這麼累的話.
我會不會來呢.
如果上班這麼累的話.
會不會選一份工作.
不要那麼累.
很多小組隊員說不要.
所以今天應該.
關於上班的題目.
你們會想瞭解些什麼.
剛剛聽完之後.
有沒有人帶了問題想討論.
應該很多.
很多不同工作上的難題.
都會答一下.
大家可以回小組裡面談一下.
今天也可以和大家談一下.
還是幫朋友問吧.
有朋友的.
(笑).
(看電視).
看下來.
(看電視).
後面有一個.
(看電視).
Hello 潘Sir.
我是John.
關於上班的時候.
我突然想起.
剛才你們說工作倫理.
我想問.
以前教會教導.

$^{2241}$如果你不要在賭場上班.
因為賭場.
沒有什麼意義.
或者負面的意義.
我之前認識一個.
大兄姐妹.
她在那地方.
找工作比較難.
外國有一間.
是賭場.
的機構.
請她.
當初我和大兄姐妹.
激烈地討論.
好像幫.
賭場的.
公司工作.
不是一個好的工作倫理.
但當時她也沒有什麼選擇.
因為她找了幾個月工.
當時也比較難找工作.
我想知道.
你們會不會.
不建議還是你們怎麼看.
她不是直接做荷官.
但她幫公司設計IT系統.
你們會怎麼看呢?.
也有些問題.
你試試回答.
我沒有在外國.
我也不太知道.
我就以.
香港賽馬會為例.
我曾經回應過一些問題.
也有些教會.
大兄姐妹.
也關於這些職場的.
倫理選擇.
我就和她說.
如果.

$^{2281}$真的是.
馬會染指的.
項目的話.
我想香港很多工作都不能做.
香港賽馬會.
徵收了.
博彩稅之後.
其實她拿到很多錢.
也做很多社會服務.
就以我們後面.
那幾間NGO.
去做紅土區的社區服務.
如果你說.
凡是馬會的錢就不打工.
其實很多人都失業.
這樣說好像.
不對題.
但其實.
我會用另一個角度去看.
雖然那些錢.
好像覺得不應該這樣賺.
或者這樣.
好像有些人.
挑釁生態,博彩.
但其實這個做法.
是資源重新分配.
因為那些錢拿出來.
是目標社會福利.
讓一些低收入.
受助人可以在.
金錢上.
得到重新分配之下.
可以得到他的生活保障.
我覺得這個也是.
工作的輔助.
我覺得這件事.
是一個.
可圈可點.
用倫理的字來看.
倫就是關係.

$^{2321}$理就是分辨.
在關係之下.
你如何作出分辨.
你想不想和他建立一個關係.
你想不想在這個關係中染指.
如果你不想在這個關係中染指.
你做了分辨就不做.
你不用在馬會工作.
那些錢.
你可以做其他都可以.
但如果你覺得.
這個關係不是.
那麼差.
或者那麼要排除.
在分辨程度下.
你把資源重新分配.
可以用回那些錢.
在正確的社會福利渠道.
當中.
用得其所.
做回應該有的決定.
我會這樣去看.
我就覺得.
馬會和.
行生是不同的.
賭場是行生.
你幫他賺錢.
我都覺得可以.
不是可以.
你不應該問我.
因為太多不同的處境.
我們不應該.
這樣做就算對或不對.
起碼不是JPES的.
個案.
JPES很明顯是騙人的.
你都做就很明顯是騙人.
但很多情況下.
其實這個是倫理問題.
我們如何去判斷倫理.

$^{2361}$我們很不容易.
不應該隨便.
去一刀切.
這樣就不對.
特別澳門人.
我都講過.
我那時候在澳門教會.
有一堂是星期日七點半崇拜.
知道為什麼嗎.
因為是方便人們崇拜後.
可以上班.
如果你是澳門人.
基本上你是沒得選擇的.
大部分人都是在賭場工作.
所以很難說.
一刀切就算對或不對.
這樣的問題.
是不應該這樣問.
所以是沒有答案的.
但我覺得.
我們作為基督徒.
在不同的生命里.
有不同的處境來判斷.
你都說了.
要賺錢.
你沒錢.
小孩沒得吃飯.
或者有很大差別.
做到可以就不做.
很多這樣的處境.
多於問應該.
做還是不做.
因為太多不同的故事.
太多不同的處境去處理.
所以除非很明顯.
做賊.
或者追賊.
很明顯是不對的.
但太多這些很模糊的例子.
我們不能夠.

$^{2401}$答到這樣就對或不對.
所以都說.
在一個墮落的世界里.
幫韓生賺錢.
但沒辦法.
這個不是呼叫.
不需要呼叫.
純粹為了賺錢.
網上那個不知道是否問你.
為什麼買散水餅.
是工作倫理的一部分.
什麼邏輯.
因為你做了一個應該要做的事情.
工作裡面.
很多例子.
做了很久的工作.
做了工作的人.
應該做的事情.
我不知道是不是買散水餅.
現在可能不是買散水餅.
多了不同的東西.
另外就是.
問關於.
是.
如果可以.
有人.
過了.
過多幾年.
就會轉工甚至轉行.
是不是代表神在不同的年和日.
當中有不同的工作.
照明.
我都說工作沒有照明.
但你想說.
生命有照明.
所以為什麼會轉工.
你不能.
用工作照明.
就解釋不了.
呼叫你做這行.

$^{2441}$但人生有上帝.
給你的呼叫帶領.
我想說的例子.
用這套工作神學.
是能夠解釋的.
上帝對一些移民了的基督徒.
有一個心意和人生方向.
所以你不做香港的護士.
去那裡就做其他的事.
拿著茶.
但那個照明不是在你的工作裡面.
所以我就說.
你轉工不是上帝給你轉了照明.
而是你本身.
上帝是你的照明.
所以就不需要將工作和照明.
去拍得那麼緊.
但我們人生裡面.
有不同的方向.
有時這十年就做這件事.
二十年之後就做這件事.
這就是我們人生裡面很多不同的生命方向.
這是我們值得去尋索的.
甚至有些.
擺明就是.
一個曠野.
這五年裡面.
很多人都沒事做.
或者很迷茫.
這不代表上帝對你沒呼叫.
但這種迷茫和曠野.
其實都是一個過程.
所以未必每一刻.
都是很明顯有意義的事.
但我們知道我們的旅程.
一個progress.
關於轉工這件事.
我曾經在教會.
散會的時候.
被家長問我.

$^{2481}$他問.
潘先生你帶開的職稱.
多不多轉工.
我說也有.
你也支持他們經常轉工嗎.
我說不是支持.
不過現在很難一份工作.
做得久.
我家裡的經常轉工.
這些情景我都不是一次見到.
不過我都跟家長說.
調整一下工作觀念.
因為上一代.
或者我也是97年前.
已經入職的人.
我們的工可以打過世.
97年前入職的炒不了你.
但我說現在不同了.
2000年.
03年和06年.
其實香港換了幾次.
簽contract的文化.
其實就算是政府工.
都是三年簽一次.
簽了兩次三年才有perm.
整件事是複雜了很多.
有機會我都跟家長說.
不要再用舊的觀念.
去說子女經常轉工.
好跳.
或者為什麼不忠於.
打一份工.
都不容易讓家長明白.
不過他們都知道.
打工辛苦.
我們年輕人都捱的.
言下之意.
好像自己子女.
不是很捱.
令父母之間的張力.

$^{2521}$都很大.
所以都給在座職稱.
知道如果你爸媽.
暗暗的你.
都可以跟他們調整.
觀念.
或者有機會.
跟你爸媽聊天.
今天這個talk.
能不能解釋.
你現在的處境.
能不能合理化.
或者叫做.
解釋現在做的事情.
令大家上班都很無奈.
剛才說的都是為了錢.
前面有.
前面都有.
最前面那位.
我想問.
很多一般人的想法.
上班就是.
有一天不用上班.
我記得John有一次.
有提過.
如果我們的夢想.
就是有一天在海邊小屋.
那天才達成就很大件事.
應該是一個.
持續在實踐中.
如果工作是這樣.
是否也可以呢.
我們上班.
都是可以為了將來.
不用上班.
第二個我想問.
在現在現今的社會.
其實工作是.
都有少少建構.
我們自己身份的一部分.

$^{2561}$如果我們不用工作.
我們用照明.
這個代入.
會是一件怎樣的事呢.
我們用照明來建立自己身份.
很好的問題.
因為我的工作.
很多時候都不用上班.
我的工作是.
如果說放假的話.
我在十二月.
我沒有出去跟別人說.
基本上.
我由六月尾放假.
放到九月頭.
十二月整個月.
都是放假.
之類的.
今天可以來做這些.
我想說是很可靠.
很多人都是這樣.
很多人是這樣.
他們不用很多obligation.
要去做很多你不想做的事.
很自由.
但你那個暑假.
或者你那個.
不用上班也好.
其實你想做事的.
只不過不是pay.
你會想收拾一下屋.
或者想無聊到找些事來搞一下.
甚至乎我打機.
打機都很辛苦.
打得好其實都不容易.
你要做.
如果是一個.
你要去做的事.
人生裡面你要做的一些事.
不過不是pay.

$^{2601}$例子.
有錢人本身都不用做事.
不過他找到自己的意義.
就好像我上班都在賭場賭錢.
賺錢回來.
所以重點.
就是這樣的意思.
當你能夠賺到什麼創造自由.
即是退休之後.
你其實仍然.
要去想下照明.
想做些什麼.
只不過你現在不需要賺錢.
去令自己養活.
你不需要這個問題.
但你仍然要問我能不能做些什麼.
所以當你賺錢.
賺到某個位置你不用上班的時候.
你就要問你人生有什麼意義.
想做些什麼.
所以都是那個做事的字.
但定義就不是賺錢的做事.
就是voluntary work.
很多人.
很多有錢人或者很多人.
都仍然找這件事.
他不需要有財務上的壓力.
但他都要問我做些什麼.
不然真的很無聊.
所以這件事.
不一定需要在你工作.
退休之後.
而是在你公寓之外.
你可以有些什麼做.
所以回到問題.
你的身份是什麼.
值得問.
我在上帝裡面.
想做些什麼.
如果上班不是的時候.

$^{2641}$某個公司裡面的會計人.
不是我的照明.
但我人生裡面.
有些什麼想做呢.
可以是一個照顧子女的媽媽.
可以是一個教會侍奉的人.
可以是一個.
對於游讀.
很有熱忱的人.
很喜歡行山的人.
很喜歡養花的人.
這些不是叫懶.
不是叫射波.
而是你的生命裡面.
上帝給你的方向就是這些.
這些不是在工作裡面.
過去十年裡面.
太多人告訴你工作就是你的照明.
而太多人不是.
其實大家都可以實踐自己的生命和理想.
我想身份正正是.
可能在工作以外裡面.
去尋索你的身份.
因為今天太多複雜的東西.
我不再是一個天職.
做一個漁夫或者消防員.
但我的天職.
仍然是一個職位.
我的照明是做什麼呢?.
可能是複雜的.
可能每年去泰國短川.
一個短川發燒友.
我就做這件事.
工作只是我能夠去生活.
值得想這件事.
每個人都有這件事.
生命才有意思.
這個問題是.
問得好.
但都相當複雜.

$^{2681}$因為牽涉到工作.
身份和照明三個很大的詞.
如果混合在一起.
其實我都在想有什麼例子.
其中一個例子就是.
以前認識了一些在北區.
做農夫的.
工作的人.
他本身不是做農夫.
但我認識他的時候.
他工作是農夫.
但他本身的職業是律師.
和精算師.
但.
他的身份.
現在是農夫.
但他又不是不做.
律師的工作.
因為他本身是律師的專業.
他幫了很多.
北區的農民.
去看文件.
以致他不要被人.
閹割了地.
和權益.
精算師就幫人.
看了一些數字.
如果收地的時候.
數字應該是這樣計算.
所以你問我.
那是他的工作嗎?.
那不是他的工作,是農夫.
但他的身份是律師.
他的照明是將他的專業.
成為一些.
可能沒有相關能力的人.
能夠得到應有的報酬.
或者是他的待遇.
所以你問我三件事.
其實來說.

$^{2721}$沒有遺和感.
大家是可以融合的.
但未必用.
像John所說.
用傳統的工作定義身份.
一定要做到這件事.
才會發揚光大.
但現在工作層面是多.
我覺得不要分得那麼細.
我自己也是.
我自己有醫學訓練.
其實我現在每個星期.
見很多弟兄姐妹的時候.
很多時候都會問我去哪裡做檢查.
還有哪裡比較划算.
應不應該這樣做.
其實我每個星期都會回答不同的.
這些相關的查詢.
但我已經離開醫院十幾年了.
但這些就是我所懂的東西.
對我來說.
我的工作是教書和木匪.
但我的身份仍然是有之前工作的能力.
所以我覺得這件事.
不一定要分割.
其他弟兄姐妹呢?.
有沒有在前面?.
(有).
讓我變聲.
因為我認識一個中學同學.
他仍然是五入歧途.
他現在的處境是.
他無法選擇自己做什麼工作.
因為他已經離不開那個位置.
他可以如何面對他這個生命的處境呢?.
可能跟這個話題未必有直接關係.
但說起工作.
有些人是無法選擇.
他的工作是幫他糊口.
但他的工作內容.

$^{2761}$可能被他的上司強行放進去.
即是他在做臥底嗎?.
臥底不會說自己是臥底的.
開玩笑的.
即是如何面對?.
譬如他跟黑社會有關.
他現在離不開了.
他無法還錢.
做一輩子工作都無法還錢.
可能今天去幫人開船.
第二天幫人派牌.
第三天幫人去游泳.
是這樣的.
無法選擇.
他的工作.
他的生活被困.
他如何面對?.
他有無辦法尋求幫助?.
簡單來說.
很多苦難的人.
其實很多這些人.
這些都是很大的話題.
我想都不只是工作的問題.
而是當人生面對這樣的情況下.
如何去面對?.
我最近看了很久的那部.
《大紅燈和菊菇掛》.
不知道大家有沒有看過.
很多年前.
那部電影基本上是說.
一個大學女孩.
因為父母無錢.
就嫁給了一個有錢人.
人生裡面就是做人家的四太太.
這部電影是說四位太太的鬥爭.
最後她瘋了.
就是這樣.
人生裡面很多人會面對.
這種情況下.
他的工作就是做人家的太太.

$^{2801}$特別是以前的年代.
文初時期.
在封建裡面做太太.
沒有什麼自由.
所以很多人都是這樣的情況.
你問我.
我都是回到耶穌給我們.
生命封城這件事.
似乎我們.
如果更加想強調.
就是說你的生命.
不是被你的工作限制住.
這個生命.
其實都可以在當中.
不能夠很精彩.
但起碼是能夠幫助我們.
在工作以外.
能夠有些出路.
是很困難的.
不會有任何好結局.
但實際上.
如果他能夠信耶穌.
我相信在這樣的情況下.
他仍然可以活得有意義.
或者快樂.
這就是我們.
所說的福音的意思.
在困難中仍然能夠活得.
有一點順利給我們的祝福和快樂.
是,扭轉不了這些.
所以就說這個工作.
這個關係.
因為我要賺錢.
所以被迫做這些.
真的不能夠稱之為照明.
不能夠稱之為順利給我們.
但我們就說.
嘗試去推高一格.
不在工作里賦予意義.
但在人生里.

$^{2841}$可以有意義.
這樣說是很風涼話.
這樣說好像很好.
但我唯有這樣理解.
我有個案.
不過不能在YouTube說.
所以我就不說那個.
那個就真的近身一點.
不過我說那個.
之後再和你談.
但暑假.
暑假就和牧者去了台灣.
我在講道也說過.
我們去了一個紅燈區.
在台北.
西門町後面的.
一個紅燈區.
探一個侍工.
那個侍工.
最後我們買了一本書回來.
我自己剛看完那本書.
叫《茶室女人心》.
我自己在講道的時候也看過那本書.
裡面記錄.
茶室的阿姨.
她的大半生.
前幾十年的人生.
其實也像你這樣說.
她從小就被人賣了.
在記載上.
或者養大了就開始陪茶.
整件事.
她人生是一片黑.
她沒得選擇.
但當孫教士去探訪她們.
和她們說.
其實你可以選擇這件事.
她們也不選擇.
她覺得.
她們的人生也是這樣.

$^{2881}$也沒什麼出路.
但孫教士一而再再而三.
去探訪她們.
她們才覺得.
其實也可以試試.
她可以試試.
於是就成為.
生命當中的一點光.
然後慢慢讓她自己發覺.
原來人生也有選擇.
如果聽你剛才講.
你朋友的個案.
她不是沒得選擇.
但她可能不想選擇.
又或者.
她覺得一選擇就要.
轉變她整個生活形態.
可能她又不想.
她為生了很久.
也是靠這個方式.
可能她轉回.
相對我們這些普通人.
她覺得太悶.
但我覺得仍然有一個網絡.
她也和你有聯繫.
你是信耶穌的人.
你就繼續給她一點光.
她希望朝著光走.
我覺得這也是.
我們在她生命當中.
的一個幫助.
所以我覺得這件事.
未必是聯繫到工作.
反而是她的人生觀.
和她的取態.
對福音的回應.
或者對於上帝的揮手.
會不會影響.
這些位置.
剛才提到.

$^{2921}$那些自願工作.
其實.
假設有個人.
對某些事物.
很有興趣.
當是去旅行.
但那件事.
其實未必.
雖然她很喜歡做.
但未必一定是能幫助人的事.
如果在這種情況下.
其實.
是否代表那件.
自願工作.
未必一定是.
照明的層次.
會不會還有其他東西呢?.
我想.
照明有幾個層面.
第一堂就說.
基督徒本身就是一個.
帶著照明的名詞.
見證耶穌.
這是一個照明.
剛才也說得對.
其實也叫做興趣.
興趣就是去旅行.
不太關我自願工作.
或照明的事.
不過我想.
剛才我說的自願工作比較闊.
不是純粹說做義工.
而是有些事.
是會放一些心思.
甚至是.
一些努力的事.
未必說得很偉大.
不是做義工.
但可能是.
我喜歡行山.

$^{2961}$經常約人行山或是搞手.
搞手也有很多事要做.
約人來.
即是說.
這些小事.
都值得我們實踐.
是否不做.
幫人的事呢?.
當然不是.
基督徒的身份就是.
幫人.
這是我們要做的事.
所以是不同層面的事.
一個是實踐自己生命中.
想做的事.
可以是興趣.
可以是你想做的事.
另一個是很大的命令.
就是你需要.
去幫助人.
是嗎?.
有些是基督徒.
去見證耶穌的事.
所以有幾個不同層面的事.
我們不會只做這件事.
但我覺得一個人.
也沒理由只是侍奉.
也有些興趣.
有些你喜歡做的事.
這些都重要.
所以我不會覺得一個人.
只做義工幫人就夠了.
不幫就不好了.
這幾件事都是很重要的.
我沒什麼想法.
因為做義工.
我會去做的.
反而我兩個兒子.
他在學校也有.
義工的事數.

$^{3001}$我不知道你學校有沒有.
我經常都擔心.
其實做十個小時.
十五個小時.
三十個小時.
其實對於想達到的目標.
我其實也很懷疑.
我覺得做義工.
應該是讓你自己去探索.
你自己是否想達到什麼目的.
所以我覺得.
很多義工其實都不是.
表面上的義工.
好像作品裡面.
很多都是義工.
不付費的東西.
我不覺得是義工.
這些都是想做的事.
父母也是一樣.
父母也是義工.
但你不會問.
為了做義工就讓你.
人裡面很多理想.
都是義工的性質.
不付費的.
但不是那些很表面的.
工程團的義工.
那些反而是很表面的東西.
很多很深層次的.
不是被稱為義工.
但不付費的東西.
後面有一個.
前輩.
我有一個很無聊的問題.
哪一個發明瞭上班?.
第二個就是.
義工和業餘是怎樣分辨的?.
他是義工還是興趣?.
有些人打遊戲是興趣.
但電競也是一個職業.

$^{3041}$誰發明瞭上班?.
應該是李老闆吧.
應該是人類吧.
人類發明瞭上班這個制度.
簡單來說就是.
以前原始人打獵.
是否叫上班呢?.
不是,只是為了賺錢.
後來發現.
不如你幫我照顧小孩.
我給你一隻羊.
那就可以上班了.
應該是這樣吧.
但應該不是上帝.
因為上帝只是給我們.
資源有限的狀況.
其他人的待羅.
所以我說不是上帝.
因為上帝沒有把工作.
成為他自己對創造心意的一種心意.
那與義工有關的問題.
我也不懂得回答.
因為我剛才說的問題是一樣的.
義工不是那些義工.
只是不付費的意思.
所以很多時候人生里.
有些是付費的,有些是不付費的.
有些事是不付費的.
可以是業餘.
可以是幫人的事.
可以是你想做的事.
可以是理想,可以是幫朋友.
所以這些全部都是義工.
所以不要太不信任.
任何不付費的事.
都是值得留意的.
就像新年的時候幫媽媽做蘿蔔糕.
這是義工.
這就是你的生命和媽媽的關係.
這就是這樣.

$^{3081}$可能也是你的業餘.
但你喜歡烹飪.
可能有點不關事.
剛才說工作上有些不是屬於你的照明.
用人生作基礎去想這件事.
我在想如果照明出事了怎麼辦呢?.
我認識一些朋友.
他至少很想做警察.
小時候參加訓練警訓.
他父母全部都是警察世家.
他很想去保安.
但你知道這段時間.
做警察有多大的掙扎.
他也知道.
有幾個這種朋友.
有些是知道這件事不好.
他就不做.
但他又忍不住.
有些經常去做護警.
但他真的很喜歡這件事.
他覺得自己很適合.
也很應該去做這件事.
又或者有些不理會.
繼續做警察.
但有些他覺得不對的就不理會.
但如果這樣就做不到.
剛才說的三件事.
負責人又用WhatsApp.
又交貨帶.
平時有些時間你會沒有交貨帶.
因為你覺得不應該要跟隨規則.
第二,有多少部隊.
第一要跟從上司的指令.
有這些矛盾的時候.
會怎麼辦呢?.
我想到一個這樣的例子.
可能這件事會比較入肉.
他自己是否這樣看?.
自己是否覺得這樣是不想做.
或者不接受?.

$^{3121}$有些是.
有幾個警察朋友.
有些是他覺得不可以.
但群眾壓力之下.
或者他覺得自己是.
或者他本身整個人.
本身的生活就是想要做這件事.
他從來沒有想過要做其他事.
或者他覺得自己真的適合.
就算我看他的人.
他的性格都真的很適合.
但他面對著這樣的情況.
怎麼辦呢?.
我想警察本身不是一件什麼.
本身警察都是一件好事.
很多事情都是需要警察來做.
都說JPEG.
警察會抓人.
所以我覺得其實.
有點像剛才賭場的情況.
有些情況下.
不是自己想做的事.
怎麼辦?.
其實都是很處境化的事.
就是掙扎.
今天才和弟兄聊天.
他以前是做保險的.
但他的保險是很不適合.
他不喜歡保險的事.
做了幾年就不做了.
所以都是一種過程.
他可能某個位置不認同警察某些事.
自然就會掙扎.
但暫時未來.
不代表他會全然敗壞.
都是一種過程.
所以很難判斷一個人.
或者判斷一個行業.
本身是怎樣的.
好還是不好.

$^{3161}$很難來到.
那麼容易去畫界線.
我的重點都是覺得.
不是爭持在那個行業上.
因為那個行業.
剛好這次有些事故.
或者事件.
就覺得那個行業不好.
我是不是要跳船.
每個行業都有不好的地方.
就正如醫生都有很多不好的醫生.
剛才張先生說保險.
我想有些人保險是好的.
有些人保險是不好的.
每次見到你都是sell down.
可能你都會負面.
我想重點未必是.
側重在那個行業上.
回到剛才John follow你.
他問你.
他本身的人設是否對做那份工作.
這個是重要的.
因為人設不對做那份工作.
一來他做得辛苦.
二來他都在問自己日日上班做什麼.
這個都很重要.
我經常都說身體是很誠實的.
我以往.
就算我自己家人都是這樣.
剛才說到97年前入職.
很多東西都是基本法保障.
不會炒到你的.
但當慢慢升職升到一個位置.
很大壓力.
工時很長的時候.
就頭髮不斷掉.
病又不好.
都會跟家人說.
錢買到的東西其實可以不要.
有些東西是錢買不到的.

$^{3201}$如果你在這份工作.
給到這筆錢.
但你沒有了家人相處的時間.
沒有身體.
沒有東西的時候.
你覺得是否還值得.
我覺得那份工作考慮的因素.
是否符合你那個階段的人設.
如果你的人設定是剛剛做的.
我覺得那個衡量是比較重要的.
但不是因為工作在那個階段.
突然被污名.
或者有些東西是違反了.
你自己接受不了.
你就選擇不做.
但如果你覺得不是那個想法.
你覺得你可以堅守到一些原則.
你就繼續做下去.
那牽涉到職業道德.
你在工作上有沒有違反職業道德.
道德就是道就是法規.
德就是行為.
在法規之下你有相應的行為.
那你就合乎道德.
但如果你說做警察.
你的職業道德你去包庇.
那就是違反職業道德.
我覺得不要用職業來.
那個環境因素影響到他的選擇.
反而他自己是否符合.
和他有沒有做回職業道德.
這個是重要的.
差不多了.
我們還有一堂課.
多飛一會就完結了.
那下一段有什麼期望呢.
下一段我們最後會講基督徒的隱藏性.
先不說這些.
大家下次來的時候就知道.
如何去理解隱藏性的題目.

$^{3241}$下個星期 下次見.
下一段 再見.
再見.
下個星期見.
\newpage



\section{啟示錄 12:1-17-20231125}
\label{sec:2LIl7VilU18}
\textbf{【網上崇拜】細路仔唔識世界|啟示錄12\_1-17|20231125 [2LIl7VilU18]}
\newline
\newline
連結: \href{https://youtube.com/watch?v=2LIl7VilU18}{\texttt{ https://youtube.com/watch?v=2LIl7VilU18}} ~~~~ 語音日期: 2023-11-25 
\newline
\newline
\hyperref[sec:2LJqqGa1zFo]{\small{< < < PREV SERMON < < <}}
~
\hyperref[sec:index_chronic]{\small{[返順時目]}}
~
\hyperref[sec:index_scriptual]{\small{[返順卷目]}}
~
\hyperref[sec:w1NzLUX2_GE]{\small{> > > NEXT SERMON > > >}}
\newline
\newline
啟示錄 12:1-17-20231125
\newline
\begin{longtable}{cl}
\hline
\hline
章節 & 經文 (和合本修訂版)\\
\hline
12:1 & \begin{tabularx}{0.7\textwidth}{X} 天上出現了一個大兆頭:有一個婦人身披太陽,腳踏月亮,頭戴十二顆星的冠冕; \end{tabularx} \\ \\ \relax
12:2 & \begin{tabularx}{0.7\textwidth}{X} 她懷了孕,在生產的陣痛中疼痛地喊叫。 \end{tabularx} \\ \\ \relax
12:3 & \begin{tabularx}{0.7\textwidth}{X} 天上又出現了另一個兆頭:有一條大紅龍,有七個頭十個角;七個頭上戴著七個冠冕。 \end{tabularx} \\ \\ \relax
12:4 & \begin{tabularx}{0.7\textwidth}{X} 牠的尾巴拖拉著天上星辰的三分之一,把它們摔在地上。然後龍站在那將要生產的婦人面前,等她生產後要吞吃她的孩子。 \end{tabularx} \\ \\ \relax
12:5 & \begin{tabularx}{0.7\textwidth}{X} 婦人生了一個男孩子,就是將來要用鐵杖管轄萬國的;她的孩子被提到神和他寶座那裡去。 \end{tabularx} \\ \\ \relax
12:6 & \begin{tabularx}{0.7\textwidth}{X} 婦人就逃到曠野,在那裡有神給她預備的地方,使她在那裡被供養一千二百六十天。 \end{tabularx} \\ \\ \relax
12:7 & \begin{tabularx}{0.7\textwidth}{X} 天上發生了爭戰。米迦勒同他的使者與龍作戰,龍同牠的使者也起來應戰, \end{tabularx} \\ \\ \relax
12:8 & \begin{tabularx}{0.7\textwidth}{X} 牠們都打敗了,天上再也沒有牠們的地方。 \end{tabularx} \\ \\ \relax
12:9 & \begin{tabularx}{0.7\textwidth}{X} 大龍就是那古蛇,名叫魔鬼,又叫撒但,是迷惑普天下的;牠被摔在地上,牠的使者也一同被摔下去。 \end{tabularx} \\ \\ \relax
12:10 & \begin{tabularx}{0.7\textwidth}{X} 我聽見在天上有大聲音說:「我神的救恩、能力、國度,和他所立的基督的權柄現在都來到了。因為那個在我們神面前、晝夜控告我們弟兄的,已經被摔下去了。 \end{tabularx} \\ \\ \relax
12:11 & \begin{tabularx}{0.7\textwidth}{X} 弟兄勝過那條龍是因羔羊的血,和因自己所見證的道。雖然至於死,他們也不惜自己的性命。 \end{tabularx} \\ \\ \relax
12:12 & \begin{tabularx}{0.7\textwidth}{X} 所以,諸天和住在其中的,你們都快樂吧!只是地和海有禍了!因為魔鬼知道自己的時候不多,就氣憤憤地下到你們那裡去了。」 \end{tabularx} \\ \\ \relax
12:13 & \begin{tabularx}{0.7\textwidth}{X} 龍見自己被摔在地上,就迫害那生男孩子的婦人。 \end{tabularx} \\ \\ \relax
12:14 & \begin{tabularx}{0.7\textwidth}{X} 於是有大鷹的兩個翅膀賜給婦人,讓她能飛到曠野,到自己的地方,躲避那蛇。她在那裡受供養一載二載半載。 \end{tabularx} \\ \\ \relax
12:15 & \begin{tabularx}{0.7\textwidth}{X} 蛇在婦人背後,從口中噴出水來,像河一樣,要將婦人沖走。 \end{tabularx} \\ \\ \relax
12:16 & \begin{tabularx}{0.7\textwidth}{X} 地卻幫助了婦人,開口吞了從龍口噴出來的水。 \end{tabularx} \\ \\ \relax
12:17 & \begin{tabularx}{0.7\textwidth}{X} 於是龍向婦人發怒,去與她其餘的兒女作戰,就是與那些遵守神命令、為耶穌作見證的。那時龍站在海邊沙灘上。 \end{tabularx} \\ \\
[1ex]
\hline
\hline
\end{longtable}
$^{1}$.
因為啟示了這段經文.
一來有很多人說不明白.
二來剛才有人跟我說.
原來在YouTube也有問題.
就說為甚麼這段經文會有這個講題呢.
為了讓大家對這段經文可以有深刻的印象.
所以希望透過剛才這個演繹.
大家可以對於經文的描述可以立體一點.
但最重要我們也是求聖靈去幫助我們.
希望我們可以明白聖經的說話.
也讓我們可以開眼界.
看得見祂要我們去看的東西.
我們一起禱告吧.
因為你是坐在寶座上的.
你是掌管天下的.
你的恩典是與我們同在的.
今天我們去看你的話語的時候.
求主你去幫助我們.
特別求聖靈去成為我們每一個人的老師.
去教導我們明白我們經常說看不明白的啟示錄.
也親自去感動我們.
讓我們看見主你的說話那種能力.
也讓我們看見主你是常常與我們同在.
以致我們縱使面對很多困難.
我們今天可以靠著你的力量去面對.
我們將聽你的話語的時間交給你.
願神你與我們一起幫助我們.
引導我們.
我們這樣禱告奉耶穌基督的聖名祈求.
阿們.
有一晚崇拜完結後.
我通常會去買宵夜吃的.
當天我去買魚肉碗仔翅.
如果你去過這間大圍小吃.
我真的去這間大圍小吃買的.
如果你去過這間大圍小吃.
就知道大圍小吃分三條隊.
最左邊是買魚蛋燒賣.
中間是買魚肉碗仔翅.

$^{41}$最左邊是買類似煎釀三寶的.
在排隊中.
前面有個媽媽和大約五歲的女兒.
排著排著就到了.
突然間旁邊排煎釀三寶的大叔.
突然和前面的媽媽和女兒打招呼.
原來認識的.
很開心,妹妹跟他玩.
他們兩個都到了.
去叫他們的食物,也在等.
然後妹妹就知道大叔想買煎釀三寶.
就拿了幾支竹籤.
想拿給大叔.
這個叔叔.
還沒看到他拿著竹籤的時候.
這個妹妹的媽媽就看到他拿著竹籤.
在座的媽媽.
如果你看到你的女兒拿著竹籤.
你會想說什麼呢.
這個媽媽.
不知道你聽不懂她說什麼.
這個媽媽就說,幹什麼.
拿著竹籤幹什麼.
玩得嗎.
這麼危險.
多一隻眼那怎麼辦.
妹妹就含著兩泡眼淚.
看著媽媽.
說,不玩得.
不玩得,那你還拿.
知道不可以玩,還哭.
然後呢.
這個叔叔就搞定了.
他的煎釀三寶.
轉身就看到妹妹.
哭了,無端端的.
就馬上走了.
好了.
我呢,其實呢,看到妹妹哭的時候.
很不忍心.

$^{81}$很慘.
我簡直有股衝動.
跟那個姐姐說.
媽媽,不是啊,其實妹妹.
想拿竹籤給叔叔.
不是想玩啊.
正當我.
真的很想拍拍她的時候.
我就想起我丈夫經常叫我不要那麼多事.
然後我就.
決定繼續排隊.
等我這一碗魚肉碗仔翅.
始終人家是教女兒,我這三姑六婆.
出聲幹什麼呢.
妹妹看著.
買煎釀三寶的叔叔.
終於想幫她.
拿竹籤,誰知道被媽媽罵.
媽媽看到妹妹.
拿著竹籤.
下意識覺得她貪玩很危險.
所以罵爆她.
似乎罪魁禍首的叔叔.
只顧著看著煎釀三寶.
可以了嗎.
完全不知道媽媽罵妹妹.
也不知道妹妹為什麼哭.
完全不知道發生什麼事.
而我,這個無關的.
三姑六婆.
看得很通,看得很透整件事.
但我選擇沉默.
不要那麼多事.
用不同角度看事情.
會有不同看法.
對嗎.
我不知道大家.
習慣坐地鐵的人.
會不會覺得香港.
用地鐵路線圖去理解香港.

$^{121}$曾經有人回答我.
他住綠色線.
知不知道綠色線.
是坐在哪裡,即是住在哪裡.
我再看下一頁.
他說綠色線.
原來住在觀塘.
明明綠色線很長.
你還說住在觀塘.
因為他叫觀塘線.
有人會覺得.
荃灣去大圍.
很遠.
因為他覺得要轉很多程車.
所以覺得很遠.
但其實如果你不坐地鐵.
坐巴士或是.
開車.
你會覺得荃灣去大圍很方便.
因為有城門隧道.
但最重點是.
如果我們用地鐵路線圖.
去理解香港.
其實很多地方.
是沒有地鐵站的.
我按不到.
為甚麼我每次都按不到.
我再按多一下.
要看著哪裡.
OK.
哎呀.
這裡.
很小的字.
這些地方沒有地鐵站.
但都是香港.
如果我們的腦.
用了地鐵路線圖去理解香港.
我們便會不知道.
原來這些地方都屬於香港.
這些地方都需要資源.

$^{161}$都可能要保育.
一件事的發生.
用不同眼光看.
而我們的思維.
習慣都會限制我們對一個人.
一件事.
甚至對整個世界的認識.
2023年快要過了.
我們過去這一年.
用甚麼眼光去認識自己.
認識香港.
甚至認識這個世界.
我們的眼光和思維.
有沒有限制我們去認識.
這個天賦的世界.
所以我今天講題.
叫「小朋友不懂世界」是這個意思.
謝謝Youtube的朋友.
或者我們不單單是小朋友.
不懂世界.
其實我們大人都不太懂.
剛才已經出了.
「白日之下」.
OK.
我都交給你按.
好嗎?.
「白日之下」.
大家有沒有看.
其實我沒有看.
所以你放心我不會劇透.
但就算不劇透.
我不知道,只是用戲名.
就算沒有看.
其實都大約猜到「白日之下」.
單憑名字都不會是.
開心,陽光的戲.
因為「白日之下」的.
香港大部份都是.
荒謬,難過.
令人無奈的東西.

$^{201}$如果我們用「白日之下」的.
角度去看這個世界.
我們會看得很灰的.
我們會看得很失望的.
所以我今天和大家看啟示錄.
大家轉一轉角度.
我們向上.
看我們「白日之上」的世界.
以致我們在「白日之下」的生活.
除了失望,灰心.
都可以有不同的看法.
不少人和我說.
他們看不明白啟示錄.
主要原因是因為.
經文用了很多符號,圖像.
但這些符號,圖像.
對當時的讀者來說.
其實可以一看就明白.
可以幻想.
今天我們知道小熊維尼.
不單單是小熊維尼.
粉紅色的秋我們都知道.
不是季節.
大約是這個意思.
777和721.
我們知道是數字.
但不單單是數字.
777是代表一個人.
721是一個日子.
但這些數字除了代表人物和日子.
同時我們這個年代的人.
會明白背後的代表.
或隱藏著.
一個更大的意義和意思.
今天我們看啟示錄12章.
可能剛才聽的時候.
大家對這些細節會很好奇.
可能你會想.
這個婦人為何可以.
披著太陽都不燒死.

$^{241}$龍紅色是否代表某個國家.
七頭十角究竟是甚麼樣子.
但請留意.
這些字面的意思.
不是作者想我們去斟酌.
或去強調的東西.
因為對當時的人來說.
全部都心神領會.
畫公仔不用畫出腸.
大部分的東西一聽.
大家當時已經能夠意會.
能夠明白.
啟示錄可以分為兩部分.
第一部分是第一至第十二節.
是講天上的增減.
第一節它說天上有個徵兆.
我都要下一張.
有個徵兆.
它說.
這裡有個孕婦.
很痛苦地在生育.
有一條很大的紅色的龍.
咬著這個女人.
是想等這個女人.
一生育就吃掉她.
這個小孩是誰呢.
第五節回答了我們.
它說要用鐵匠管萬國.
其實這句是引至.
詩篇第二章第九節.
講尼塞亞的王權.
所以這個小孩是在講耶穌基督.
因為耶穌基督.
才有權去管治萬國.
不過婦人.
不單單是在講耶穌.
育身的母親瑪利亞.
大部分聖經學者認為.
這個婦人是象徵神的子民.
不過當第五節.

$^{281}$這個小孩一出生的時候.
小孩就說被提到神的寶座.
這一句其實已經.
指到耶穌從死裡復活.
以至高升到神的寶座.
所以第五節可以說是.
一句講完耶穌的生平.
他的降生.
死亡,復活和升天.
所以這句同時是表示.
耶穌釘十字架.
升天,復活.
是打贏龍的原因.
也是打贏龍的基礎.
婦人在這個時間.
也到了一個荒地.
去躲避大龍.
第七節.
講到天上發生了一場戰爭.
就是米加勒,米加勒是天使長.
他和其他天使.
和龍和他的小孩.
一起去打仗.
第八節就說龍不夠打.
天上沒有他們的地方.
沒有屬於龍的地方.
龍和他的小孩.
可以說,如果用今天這樣說.
真的不是持份者.
所以第九節米加勒.
很不客氣地就一下子.
把他趕下地.
天上的戰爭.
龍被趕下地.
是表示龍是徹徹底底的失敗.
這條龍是誰呢.
他告訴我們.
這條龍是古蛇.
魔鬼,又叫撒旦.
迷惑全地.

$^{321}$用古蛇來形容.
我們大約都會聯想到創世紀.
引誘夏娃的蛇.
這套片我們有個感覺.
原來.
這個撒旦.
這條蛇,這個魔鬼.
由創世的時候.
直到現在末世.
他的攻擊,他的誘惑.
是不會停止的.
不過撒旦對耶穌基督.
是少少動搖和影響力都沒有.
耶穌仍然是.
很穩定地坐在寶座上.
所以第十節到第十二節.
如果你剛才聽的時候.
你會聽到背景音樂很大聲.
是因為他有一個宣告.
宣告就是.
耶穌基督的國已經建立.
已經響道.
是一個宣言.
告訴我們基督的國度已經響道.
第十三節到第十七節.
第十三節到第十七節是說另一場戰爭.
是在地上的戰爭.
這條龍掉在地上.
想撿回砸沙.
所以他被人一扔就掉在地上.
他殺不死耶穌.
他就唯有向他的跟隨者下手.
第十七節就說.
龍要逼迫女人.
和其餘的子女去開戰.
聖經很清楚說到.
其餘的子女是哪些.
就是那些遵守神的誡命.
有耶穌的見證的人.
龍.

$^{361}$由創世到今天.
他要做的事其實只有一個.
就是要迷惑.
引誘拉攏.
本身那些是跟隨神的子民.
拉攏他們.
改變陣營.
改變陣營去跟隨大龍.
其實啟授十二章.
很容易明白.
就是在說打仗.
說一場正邪大戰.
原因是當時的弟兄姊妹.
在信仰上.
都好像在打一場正邪大戰.
可能大家聽過.
在約翰社啟述的時候.
羅馬皇帝就近迫迫.
可能要流行拜皇帝.
基督徒不拜.
是很受迫迫的.
我們經常聽到.
不拜就慘了.
之類的.
但其實你想深一層.
明刀明槍要你拜皇帝.
要你宣誓.
其實反而沒那麼難決定.
都是做和不做.
但難在的不只是這樣.
原來當時的羅馬帝國.
沒有立例要人們拜皇帝.
相反是不同的小城市.
為了想羅馬皇帝.
見到他們.
為了想羅馬皇帝.
可以寵幸這些小城市.
所以他們要自己爭上位.
啟授就提到亞細亞有七個教會.
我們都知道.

$^{401}$寫信給七教會.
七教會其實就是七個地方的教會.
所以這七個教會.
那些城市都不例外.
他們都很想得到.
羅馬皇帝的寵幸.
要怎樣做.
就是效忠.
要讓羅馬皇帝知道.
我們這個城市對你忠心耿耿.
其中一樣東西很特別.
原來他們曾經試過.
不同城市要入紙申請.
要求你去.
建一座廟.
廟裡要有一個羅馬皇帝.
在那裡.
讓百姓去敬拜.
當時.
士美那城和薩迪.
這兩個城市.
曾經爭奪.
爭奪要建一座廟.
給提比留斯皇帝.
最終士美那這個城.
贏了.
贏了不是代表有座廟.
你明不明白.
贏了代表.
我這個羅馬皇帝寵幸你這個城市.
你這個城市真乖.
所以我現在可以給你資源.
建橋給你.
給你錢發展.
全世界都會見到你.
投放多點資源.
自然我們這個城市就有錢.
想得到資源.
有洞路賺錢做生意.
自然就要識埋堆.

$^{441}$經濟圈要搞得起經濟.
自然就要有一班.
識人好過識字的朋友圈.
其實和今天我們信徒.
面對的掙扎和處境.
幾相似.
我們不少信徒都面對一些很貼地的挑戰.
不單單是.
拜不拜皇帝.
放不放國旗.
上不上課.
而是政治社會信仰.
經濟文化教育.
等等.
每一個範疇都混在一起.
環環相扣.
以至我們這一班.
跟隨耶穌的人.
其實要解決生活困難.
和選擇.
我們都覺得搞不定,很大壓力.
還說要在信仰上.
堅持繼續有.
生命的見證.
是更加困難.
甚至可能會和神.
越走越遠.
作者就是想透過.
這一場天上的戰爭.
讓我們看見.
我們肉眼看不見.
白日之上的真相.
他想鼓勵我們.
有勇氣去面對.
在地上不同的挑戰.
和困難.
作者想我們看清一件事.
其實是.
這條龍不堪一擊.
在地上.

$^{481}$你見到他似乎是.
張牙舞爪,任意妄為.
也問也無.
但我們看見聖經很粗俗地說.
這條龍是垃圾.
是嗎?.
經文一開始形容龍很厲害.
我們會害怕,第三節.
他說龍既紅色又大條.
很有霸氣.
紅色會有人形容.
無人性,血腥,殘暴.
而且還說.
他帶著七個皇冠.
哪些人.
可以帶皇冠?.
皇帝當權者.
才可以帶皇冠.
是身份和權力的象徵.
代表他擁有.
一些權力,擁有實權.
這條龍帶出七個.
七個.
全方位他都有權勢.
全方位他都有影響力.
啟示六十二章.
告訴我們.
帶紅龍同時有很多種身份.
他是古蛇,他是魔鬼.
他是撒旦,也是迷惑人.
控告人.
目的是要我們遠離神.
漸遠,甚至轉陣營.
今天我們有什麼.
令我們和神的關係.
漸遠漸遠.
甚至明知道.
得罪神我們也會照做.
他可以化身.
做很多不同的形象.

$^{521}$可以用權力.
金錢,慾望,明星.
等等等等.
去迷惑我們.
如果我們只是看著地上的世界.
其實這條紅龍.
是很吸引的.
因為他是錢.
他是名,他是成就.
他是名氣.
他是權力.
這些對於我們來說很有安全感.
很實在.
很有渣拿.
但如果我們向上.
看天上的世界.
我們看清真相.
是紅龍只有樣貌.
暫時.
垃圾.
他一下子.
被神扔到地上.
沒有定期.
今天我們還要不要.
抱著垃圾.
當作寶.
當我們今天在地上.
看見很多蠢人.
壞人,說是.
自恃自己有權有勢,隻手遮天的時候.
我們可能.
同時也會覺得灰心絕望.
但天上的.
真相讓我們知道.
他們只是垃圾.
真正的勝利.
是屬於為我們釘十字架.
流出補血.
頭戴冠冕.
坐在寶座上的耶穌基督.

$^{561}$其實在地上.
冠冕一點也不吸引.
誰會戴冠冕.
你知不知道.
是香港小姐.
香港小姐會戴冠冕.
但你會,而且現在香港小姐的冠冕.
似乎是不值錢.
以前聽說有位港姐.
賣了一個官,有錢的,但現在似乎也不值錢.
香港小姐戴冠冕.
我們知道她甚麼也沒有.
那個官也不值錢.
做了港姐,不代表有戲拍.
也不代表她.
靚.
其實我不知道港姐現在還有沒有港姐.
Sorry,我不知道.
這樣嗎?還是我們這裡會有一個港姐?.
我們Full Church甚麼人都有.
Sorry,Sorry,我看的那些.
應該不是你那一屆.
不值錢的後官.
官冕是沒有權的.
虛榮.
地上我們是這樣看的.
但天上的世界.
讓我們認清.
官冕比皇冠.
是更大能力.
更值得我們依靠.
因為頭戴官冕的耶穌基督.
才是真正擁有.
天上,地上.
和地下的權柄.
今天我們的生命.
想戴皇冠.
還是戴官冕?.
我們想跟紅龍的陣營.
抱著垃圾皇冠當作寶.

$^{601}$還是我們跟隨.
耶穌.
跟隨港姐冠軍.
戴回官冕.
戴回真正.
得勝的官冕.
天上的徵戰.
也讓我們看清第二個真相.
原來當我們落在苦難的時候.
神不是坐在寶座上.
按按腳.
不理我們.
神不會放棄我們.
神不會不理我們.
祂看見.
而且祂一直眷顧我們.
第六節.
他說婦人走到曠野.
應該有個powerpoint.
婦人走到曠野.
在那裡有神為她預備的地方.
好讓她在那裡受供養.
1260天.
第十四節也是這樣說.
其實說同一件事.
去自己的地方供養一年兩年半年.
這裡其實一年兩年半年.
和1260天是一樣的.
都是說三年半.
三年半這個日期.
其實在第十一章.
和第十三章.
後面那章也有說到這個時間.
第十一章是說到.
教會受外邦人踐踏.
第十三章是說到.
當國勢力任意妄為的時候.
時間也是三年半.
三年半是什麼意思呢.
不是真的三年半.

$^{641}$因為我們知道啟祖不是這樣說.
七是代表完美完全的數字.
三年半.
是七的一半.
意思是讓我們知道.
受苦是有時限的.
受苦是會終止的.
受苦是不會到永遠的.
而最重要的是.
當我們.
經歷困難挑戰的時候.
神是沒有停止他的工作的.
苦人去到神.
為他預備的曠野.
有神的眷顧和保守.
十四節再次是這樣說.
他得到神的供養.
第十五節更加說蛇又出現了.
他用另一個形象.
想噴水出來淹死女人.
但是地開口吞了.
這個景象.
讓我們想起紅海.
紅海分為旱地.
神的帶領和保守.
無論如何.
這個天上的景象.
再次強調給我們這群人看見.
神是不會放棄每一個.
跟隨他的人.
神會帶領人去到他所預備的曠野.
得到他的供應.
甚至會叫地開口.
讓人避過危險.
用一些我們想不到的方式.
去為我們開路.
在困難裡面.
神的保守和眷顧.
是會停止的.
鄧小姐我不知道今天.

$^{681}$你的生命正在遇上什麼困難.
正在遇上什麼挑戰或迷茫.
可能是身體軟弱.
可能是.
心靈很沉重.
很孤單.
可能是政治環境令你感到很壓迫.
工作前景.
不明朗.
家庭婚姻可能出問題.
擔心子女的教育.
照顧父母的壓力.
人際關係.
帶給你傷害.
過去的事.
讓你很內疚自責.
不安等等.
等等.
我們眼看見白日之下.
全部都是問題.
很多的困難.
很辛苦.
很大壓力.
但聖經提醒我們.
我們向上望.
向上望這個.
白日之上天上的世界.
耶穌釘十字架的愛.
和祂同在.
是成為我們的動力.
是成為我們的勇氣.
今天我的講題是.
小朋友不懂世界.
因為我小時候.
大約是月體這張相.
這個年紀.
下一張.
有一晚.
我經過紅磡.
看到街上很多車.

$^{721}$車好像是真的.
不要那麼快開.
是我不對.
我不應該這樣說.
開出來讓他們看.
很多車我覺得很有趣.
明明是假的.
但我覺得很真.
我那時問我媽.
為何街上那麼多車.
我媽說.
因為街上很多車.
很多車.
為何街上那麼多車.
很可愛 我想坐.
我覺得我自己能坐.
因為我小時候.
我想坐 媽媽我可不可以去坐.
然後我媽說.
呸.
別亂說話 這些車不是讓你坐的.
然後我還問誰坐.
沒有理我.
後來長大了.
才知道原來這些車叫紙紮車.
你們有沒有見過.
很有趣 你們應該見過.
當年我說要坐這些車.
有些可能.
迷信的長輩.
會說一句有怪莫怪.
小朋友不懂世界.
這句話.
代表什麼.
代表他們認為.
靈界 鬼神的世界.
是真實存在.
而且比我們更有能力.
我這些小朋友不懂事.
才說要坐車.

$^{761}$所以要有怪莫怪.
這個意思是.
如果這班人.
其實真的有.
鬼怪世界.
是有的.
如果這班人對鬼神的世界.
的存在和能力.
是那麼深信不疑.
我們一班跟隨耶穌的人.
是不是理應.
應該比他們.
更加相信.
白日之上 天上的世界.
是不是應該比他們更加認定.
耶穌基督.
坐在寶座上 安定在天.
坐著為王.
是可以掌管我們.
是擁有絕對的權柄.
擁有能力.
如果你看見天上的世界.
又相信天上的世界.
相信祂的主權.
相信神的同在.
相信神的能力.
今天我們就可以更加有.
力量和勇氣.
白日之下.
全部都是困難.
但如果我們看見又相信.
天上的世界.
是真實存在的.
比我們更有能力.
我們今天就有勇氣去面對.
前幾天是感恩節.
不知道大家有沒有留意我們出了一個帖文.
過往一年在流淌.
不同方面付出過的.
不同單位.

$^{801}$感恩是提醒我們.
要多點向上望的操練.
感恩是提醒我們.
要多點向上望的操練.
當我們只是看著.
自己的困難.
我們會看不見神的眷顧.
但每當我們.
向上望.
我們就會發覺困難是很多.
但恩典.
是夠用的.
有時可能只是一句.
前幾天我開組的時候.
有組員分享.
他近來很累.
工作很忙.
放八天.
不計OT.
加上他在診所工作.
每天要照顧很多不同的病人.
他付出的不單是.
時間.
更是他的心力.
他的腦和他的勞力.
誰知這段這麼忙碌的時間.
突然不舒服.
他下班照顧完自己的病人.
他上班照顧完自己的病人.
他下班.
更加要下手下腳.
照顧自己的媽媽.
還要買菜.
照顧她的起居飲食.
但他在這段時間裡.
他覺得媽媽很不體諒他.
上班已經很疲倦.
還覺得他很老.
要照顧他.
有一天他如常上班.

$^{841}$有個伯伯來看.
他幫伯伯.
減輕了很多痛楚.
然後無理到又要看下一個症.
然後剛好.
放飯的時候.
聽到伯伯說要找他.
根據他過往經驗.
人們找他主要是問他.
要怎麼做.
他就走出去問.
原來這個伯伯說.
我很想跟你說一句謝謝.
你真的幫我減輕了很多痛楚.
他很忙.
不用了不用了.
回到家.
他當晚回到崇拜.
他說這一幕.
不停在他腦裡出現.
然後我的組員說.
他明白到.
這句謝謝雖然是出自病人的口中.
但他看見.
這句謝謝.
其實是天父跟他說的.
因為天父.
知道他需要這一句謝謝.
其實他需要媽媽.
跟他說一句謝謝.
天父知道他照顧媽媽.
的辛苦.
所以天父跟他說謝謝.
去肯定.
他照顧媽媽的付出.
一句謝謝.
沒有減輕到他.
面對的困難.
但他看見天父的看顧.
有時可能只是一份禮物.

$^{881}$我呢.
一向對收禮物.
沒有特別感覺.
大家如果想送禮物給我.
是可以的.
但你可能會覺得.
我很冷漠.
因為我從來.
從小到大.
沒有送和收禮物.
不代表你可以不送給我.
如果每個目者都有的話.
請你都送一份給我.
如果你只是送給一個.
就不用送給我了.
其實都可以不用送.
真的完全不用.
因為我本身的愛語不是收禮物.
所以其實.
別人送禮物給我.
我會覺得謝謝你.
但沒有禮物收.
我其實覺得很理所當然.
沒有問題.
所以我一向不太著重禮物.
有就有,沒有就沒有.
前陣子有個組員.
他送了一份禮物給我.
其實我一開始.
很冷漠,謝謝你.
然後就沒有了.
真的這樣.
有一天突然做了一件事.
我覺得很煩.
我要處理的事很煩.
又不知道怎樣做,怎樣解決.
覺得很頭痛,不想搞.
又想找潘Sir.
突然間我看到一份禮物.
我突然間看到一份禮物.

$^{921}$然後我就想.
為何他要把那份禮物送給我.
其實那份禮物.
他不送給我是更好的.
如果他送給他的家人.
送給他的朋友.
那份禮物對他來說.
和那份禮物來說.
總之更好,我不說那份禮物是甚麼.
但他竟然選擇送給我.
在那一刻.
突然間我看到.
其實是天父送給我的禮物.
我從來對禮物.
是沒有感動的.
但那一刻我覺得.
我很感動.
因為天父送的那份禮物.
讓我知道原來我眼前.
不只是那些很頭痛,不知怎樣搞.
很煩,不如只找潘Sir算了.
而是我身邊還有很多.
很好的人.
支持我和鼓勵我.
我從來都不會.
因為收到一份禮物.
感動到.
或是很真心.
感動到想哭.
但我想那一刻.
我就真的覺得.
天父是送了這份禮物.
透過我的組員.
去鼓勵我.
2023年.
就快過了.
你試一下向上望.
我們現在.
我們想一想.
天父今年讓你遇到什麼人.

$^{961}$讓你發生什麼事.
可能你想起的.
都是困難多的.
但有時我們向上望.
哪怕只有一個恩典.
但這個恩典.
其實就足夠了.
這個一個恩典.
就足以讓我們繼續有勇氣.
有力量.
去繼續面對我們的生活.
2024年就快到了.
弟兄姊妹.
我們不如.
由現在開始操練.
我們多點向上望.
多點看見.
天父給我們的恩典.
以致我們在生活上.
即使遇見困難.
仍然認定.
耶穌基督是那位最有能力.
亦是常常與我們同在的主.
我們一起禱告.
因為你真是.
坐著為王.
擁有最大權柄的.
那位統治者.
今天我們看見地上.
很多困難.
很多奇怪的事.
但我們看見.
當我們向上望的時候.
我們同樣有恩典.
讓我們看見你是為我們.
釘十字架.
流出補血.
求你讓我們認定.
天上的世界是真實存在.
而且比我們更有能力.

$^{1001}$以致我們今天在地上.
即使面對很多困難.
很多挑戰.
我們仍然有勇氣.
去過我們的生活.
有勇氣去面對這些挑戰.
我們這樣的祈禱.
是奉耶穌基督的名義祈求.
阿們.
\newpage



\section{}
\label{sec:w1NzLUX2_GE}
\textbf{《致餘民及流散者:給香港基督徒的神學八課》第二季第8課|20231126 [w1NzLUX2\_GE]}
\newline
\newline
連結: \href{https://youtube.com/watch?v=w1NzLUX2_GE}{\texttt{ https://youtube.com/watch?v=w1NzLUX2\_GE}} ~~~~ 語音日期: 2023-11-26 
\newline
\newline
\hyperref[sec:2LIl7VilU18]{\small{< < < PREV SERMON < < <}}
~
\hyperref[sec:index_chronic]{\small{[返順時目]}}
~
\hyperref[sec:index_scriptual]{\small{[返順卷目]}}
~
\hyperref[sec:lfg8MyM5M04]{\small{> > > NEXT SERMON > > >}}
\newline
\newline
$^{1}$我只想知道.
你到底是什麼意思.
我只想知道.
你到底是什麼意思.
我只想知道.
你到底是什麼意思.
我只想知道.
你到底是什麼意思.
我只想知道.
你到底是什麼意思.
我只想知道.
你到底是什麼意思.
我只想知道.
你到底是什麼意思.
我只想知道.
你到底是什麼意思.
我只想知道.
你到底是什麼意思.
我只想知道.
你到底是什麼意思.
我只想知道.
你到底是什麼意思.
我只想知道.
你到底是什麼意思.
我只想知道.
你到底是什麼意思.
我只想知道.
你到底是什麼意思.
我只想知道.
你到底是什麼意思.
我只想知道.
你到底是什麼意思.
我只想知道.
你到底是什麼意思.
我只想知道.
你到底是什麼意思.
我只想知道.
你到底是什麼意思.
我只想知道.
你到底是什麼意思.

$^{41}$我只想知道.
你到底是什麼意思.
我只想知道.
你到底是什麼意思.
我只想知道.
你到底是什麼意思.
我只想知道.
你到底是什麼意思.
我只想知道.
你到底是什麼意思.
我只想知道.
你到底是什麼意思.
我只想知道.
你到底是什麼意思.
我只想知道.
你到底是什麼意思.
我只想知道.
你到底是什麼意思.
我只想知道.
你到底是什麼意思.
我只想知道.
你到底是什麼意思.
我只想知道.
你到底是什麼意思.
我只想知道.
你到底是什麼意思.
我只想知道.
你到底是什麼意思.
我只想知道.
你到底是什麼意思.
我只想知道.
你到底是什麼意思.
我只想知道.
你到底是什麼意思.
我只想知道.
你到底是什麼意思.
我只想知道.
你到底是什麼意思.
我只想知道.
你到底是什麼意思.

$^{81}$我只想知道.
你到底是什麼意思.
我只想知道.
你到底是什麼意思.
我只想知道.
你到底是什麼意思.
我只想知道.
你到底是什麼意思.
我只想知道.
你到底是什麼意思.
我只想知道.
你到底是什麼意思.
我只想知道.
你到底是什麼意思.
我只想知道.
你到底是什麼意思.
我只想知道.
你到底是什麼意思.
我只想知道.
你到底是什麼意思.
我只想知道.
你到底是什麼意思.
我只想知道.
你到底是什麼意思.
我只想知道.
你到底是什麼意思.
我只想知道.
你到底是什麼意思.
我只想知道.
你到底是什麼意思.
我只想知道.
你到底是什麼意思.
我只想知道.
你到底是什麼意思.
我只想知道.
你到底是什麼意思.
我只想知道.
你到底是什麼意思.
我只想知道.
你到底是什麼意思.

$^{121}$我只想知道.
你到底是什麼意思.
我只想知道.
你到底是什麼意思.
我只想知道.
你到底是什麼意思.
我只想知道.
你到底是什麼意思.
我只想知道.
你到底是什麼意思.
我只想知道.
你到底是什麼意思.
我只想知道.
你到底是什麼意思.
我只想知道.
你到底是什麼意思.
我只想知道.
你到底是什麼意思.
我只想知道.
你到底是什麼意思.
我只想知道.
你到底是什麼意思.
我只想知道.
你到底是什麼意思.
我只想知道.
你到底是什麼意思.
我只想知道.
你到底是什麼意思.
我只想知道.
你到底是什麼意思.
我只想知道.
你到底是什麼意思.
我只想知道.
你到底是什麼意思.
我只想知道.
你到底是什麼意思.
我只想知道.
你到底是什麼意思.
我只想知道.
你到底是什麼意思.

$^{161}$我只想知道.
你到底是什麼意思.
我只想知道.
你到底是什麼意思.
我只想知道.
你到底是什麼意思.
我只想知道.
你到底是什麼意思.
我只想知道.
你到底是什麼意思.
我只想知道.
你到底是什麼意思.
我只想知道.
你到底是什麼意思.
我只想知道.
你到底是什麼意思.
我只想知道.
你到底是什麼意思.
我只想知道.
你到底是什麼意思.
我只想知道.
你到底是什麼意思.
我只想知道.
你到底是什麼意思.
我只想知道.
你到底是什麼意思.
我只想知道.
你到底是什麼意思.
我只想知道.
你到底是什麼意思.
我只想知道.
你到底是什麼意思.
我只想知道.
你到底是什麼意思.
我只想知道.
你到底是什麼意思.
我只想知道.
你到底是什麼意思.
我只想知道.
你到底是什麼意思.

$^{201}$我只想知道.
你到底是什麼意思.
我只想知道.
你到底是什麼意思.
我只想知道.
你到底是什麼意思.
我只想知道.
你到底是什麼意思.
我只想知道.
你到底是什麼意思.
我只想知道.
你到底是什麼意思.
我只想知道.
你到底是什麼意思.
我只想知道.
你到底是什麼意思.
我只想知道.
你到底是什麼意思.
我只想知道.
你到底是什麼意思.
我只想知道.
你到底是什麼意思.
我只想知道.
你到底是什麼意思.
我只想知道.
你到底是什麼意思.
我只想知道.
你到底是什麼意思.
我只想知道.
你到底是什麼意思.
我只想知道.
你到底是什麼意思.
我只想知道.
你到底是什麼意思.
我只想知道.
你到底是什麼意思.
我只想知道.
你到底是什麼意思.
我只想知道.
你到底是什麼意思.

$^{241}$我只想知道.
你到底是什麼意思.
我只想知道.
你到底是什麼意思.
我只想知道.
你到底是什麼意思.
我只想知道.
你到底是什麼意思.
我只想知道.
你到底是什麼意思.
我只想知道.
你到底是什麼意思.
我只想知道.
你到底是什麼意思.
我只想知道.
你到底是什麼意思.
我只想知道.
你到底是什麼意思.
我只想知道.
你到底是什麼意思.
我只想知道.
你到底是什麼意思.
我只想知道.
你到底是什麼意思.
我只想知道.
你到底是什麼意思.
我只想知道.
你到底是什麼意思.
我只想知道.
你到底是什麼意思.
我只想知道.
你到底是什麼意思.
我只想知道.
你到底是什麼意思.
我只想知道.
你到底是什麼意思.
我只想知道.
你到底是什麼意思.
我只想知道.
你到底是什麼意思.

$^{281}$我只想知道.
你到底是什麼意思.
我只想知道.
你到底是什麼意思.
我只想知道.
你到底是什麼意思.
我只想知道.
你到底是什麼意思.
我只想知道.
你到底是什麼意思.
我只想知道.
你到底是什麼意思.
我只想知道.
你到底是什麼意思.
我只想知道.
你到底是什麼意思.
我只想知道.
你到底是什麼意思.
我只想知道.
你到底是什麼意思.
我只想知道.
你到底是什麼意思.
我只想知道.
你到底是什麼意思.
我只想知道.
你到底是什麼意思.
我只想知道.
你到底是什麼意思.
我只想知道.
你到底是什麼意思.
我只想知道.
你到底是什麼意思.
我只想知道.
你到底是什麼意思.
我只想知道.
你到底是什麼意思.
我只想知道.
你到底是什麼意思.
我只想知道.
你到底是什麼意思.

$^{321}$我只想知道.
你到底是什麼意思.
我只想知道.
你到底是什麼意思.
我只想知道.
你到底是什麼意思.
我只想知道.
你到底是什麼意思.
我只想知道.
你到底是什麼意思.
我只想知道.
你到底是什麼意思.
我只想知道.
你到底是什麼意思.
我只想知道.
你到底是什麼意思.
我只想知道.
你到底是什麼意思.
我只想知道.
你到底是什麼意思.
我只想知道.
你到底是什麼意思.
我只想知道.
你到底是什麼意思.
我只想知道.
你到底是什麼意思.
我只想知道.
你到底是什麼意思.
我只想知道.
你到底是什麼意思.
我只想知道.
你到底是什麼意思.
我只想知道.
你到底是什麼意思.
我只想知道.
你到底是什麼意思.
我只想知道.
你到底是什麼意思.
我只想知道.
你到底是什麼意思.

$^{361}$我只想知道.
你到底是什麼意思.
我只想知道.
你到底是什麼意思.
我只想知道.
你到底是什麼意思.
我只想知道.
你到底是什麼意思.
我只想知道.
你到底是什麼意思.
我只想知道.
你到底是什麼意思.
我只想知道.
你到底是什麼意思.
我只想知道.
你到底是什麼意思.
我只想知道.
你到底是什麼意思.
我只想知道.
你到底是什麼意思.
我只想知道.
你到底是什麼意思.
我只想知道.
你到底是什麼意思.
我只想知道.
你到底是什麼意思.
我只想知道.
你到底是什麼意思.
我只想知道.
你到底是什麼意思.
我只想知道.
你到底是什麼意思.
我只想知道.
你到底是什麼意思.
我只想知道.
你到底是什麼意思.
我只想知道.
你到底是什麼意思.
我只想知道.
你到底是什麼意思.

$^{401}$我只想知道.
你到底是什麼意思.
我只想知道.
你到底是什麼意思.
我只想知道.
你到底是什麼意思.
我只想知道.
你到底是什麼意思.
我只想知道.
你到底是什麼意思.
我只想知道.
你到底是什麼意思.
我只想知道.
你到底是什麼意思.
我只想知道.
你到底是什麼意思.
我只想知道.
你到底是什麼意思.
我只想知道.
你到底是什麼意思.
我只想知道.
你到底是什麼意思.
我只想知道.
你到底是什麼意思.
我只想知道.
你到底是什麼意思.
我只想知道.
你到底是什麼意思.
我只想知道.
你到底是什麼意思.
我只想知道.
你到底是什麼意思.
我只想知道.
你到底是什麼意思.
我只想知道.
你到底是什麼意思.
我只想知道.
你到底是什麼意思.
我只想知道.
你到底是什麼意思.

$^{441}$我只想知道.
你到底是什麼意思.
我只想知道.
你到底是什麼意思.
我只想知道.
你到底是什麼意思.
我只想知道.
你到底是什麼意思.
我只想知道.
你到底是什麼意思.
我只想知道.
你到底是什麼意思.
我只想知道.
你到底是什麼意思.
我只想知道.
你到底是什麼意思.
我只想知道.
你到底是什麼意思.
我只想知道.
你到底是什麼意思.
我只想知道.
你到底是什麼意思.
我只想知道.
你到底是什麼意思.
我只想知道.
你到底是什麼意思.
我只想知道.
你到底是什麼意思.
我只想知道.
你到底是什麼意思.
我只想知道.
你到底是什麼意思.
我只想知道.
你到底是什麼意思.
我只想知道.
你到底是什麼意思.
我只想知道.
你到底是什麼意思.
我只想知道.
你到底是什麼意思.

$^{481}$我只想知道.
你到底是什麼意思.
我只想知道.
你到底是什麼意思.
我只想知道.
你到底是什麼意思.
我只想知道.
你到底是什麼意思.
我只想知道.
你到底是什麼意思.
我只想知道.
你到底是什麼意思.
我只想知道.
你到底是什麼意思.
我只想知道.
你到底是什麼意思.
我只想知道.
你到底是什麼意思.
我只想知道.
你到底是什麼意思.
我只想知道.
你到底是什麼意思.
我只想知道.
你到底是什麼意思.
我只想知道.
你到底是什麼意思.
我只想知道.
你到底是什麼意思.
我只想知道.
你到底是什麼意思.
我只想知道.
你到底是什麼意思.
我只想知道.
你到底是什麼意思.
我只想知道.
你到底是什麼意思.
我只想知道.
你到底是什麼意思.
我只想知道.
你到底是什麼意思.

$^{521}$我只想知道.
你到底是什麼意思.
我只想知道.
你到底是什麼意思.
我只想知道.
你到底是什麼意思.
我只想知道.
你到底是什麼意思.
我只想知道.
你到底是什麼意思.
我只想知道.
你到底是什麼意思.
我只想知道.
你到底是什麼意思.
我只想知道.
你到底是什麼意思.
我只想知道.
你到底是什麼意思.
我只想知道.
你到底是什麼意思.
我只想知道.
你到底是什麼意思.
我只想知道.
你到底是什麼意思.
我只想知道.
你到底是什麼意思.
我只想知道.
你到底是什麼意思.
我只想知道.
你到底是什麼意思.
我只想知道.
你到底是什麼意思.
我只想知道.
你到底是什麼意思.
我只想知道.
你到底是什麼意思.
我只想知道.
你到底是什麼意思.
我只想知道.
你到底是什麼意思.

$^{561}$我只想知道.
你到底是什麼意思.
我只想知道.
你到底是什麼意思.
我只想知道.
你到底是什麼意思.
我只想知道.
你到底是什麼意思.
我只想知道.
你到底是什麼意思.
我只想知道.
你到底是什麼意思.
我只想知道.
你到底是什麼意思.
我只想知道.
你到底是什麼意思.
我只想知道.
你到底是什麼意思.
我只想知道.
你到底是什麼意思.
我只想知道.
你到底是什麼意思.
我只想知道.
你到底是什麼意思.
我只想知道.
你到底是什麼意思.
我只想知道.
你到底是什麼意思.
我只想知道.
你到底是什麼意思.
我只想知道.
你到底是什麼意思.
我只想知道.
你到底是什麼意思.
我只想知道.
你到底是什麼意思.
我只想知道.
你到底是什麼意思.
我只想知道.
你到底是什麼意思.

$^{601}$我只想知道.
你到底是什麼意思.
我只想知道.
你到底是什麼意思.
我只想知道.
你到底是什麼意思.
我只想知道.
你到底是什麼意思.
我只想知道.
你到底是什麼意思.
我只想知道.
你到底是什麼意思.
我只想知道.
你到底是什麼意思.
我只想知道.
你到底是什麼意思.
我只想知道.
你到底是什麼意思.
我只想知道.
你到底是什麼意思.
我只想知道.
你到底是什麼意思.
我只想知道.
你到底是什麼意思.
我只想知道.
你到底是什麼意思.
我只想知道.
你到底是什麼意思.
我只想知道.
你到底是什麼意思.
我只想知道.
你到底是什麼意思.
我只想知道.
你到底是什麼意思.
我只想知道.
你到底是什麼意思.
我只想知道.
你到底是什麼意思.
我只想知道.
你到底是什麼意思.

$^{641}$我只想知道.
你到底是什麼意思.
我只想知道.
你到底是什麼意思.
我只想知道.
你到底是什麼意思.
我只想知道.
你到底是什麼意思.
我只想知道.
你到底是什麼意思.
我只想知道.
你到底是什麼意思.
我只想知道.
你到底是什麼意思.
我只想知道.
你到底是什麼意思.
我只想知道.
你到底是什麼意思.
我只想知道.
你到底是什麼意思.
我只想知道.
你到底是什麼意思.
我只想知道.
你到底是什麼意思.
我只想知道.
你到底是什麼意思.
我只想知道.
你到底是什麼意思.
我只想知道.
你到底是什麼意思.
我只想知道.
你到底是什麼意思.
我只想知道.
你到底是什麼意思.
我只想知道.
你到底是什麼意思.
我只想知道.
你到底是什麼意思.
我只想知道.
你到底是什麼意思.

$^{681}$我只想知道.
你到底是什麼意思.
我只想知道.
你到底是什麼意思.
我只想知道.
你到底是什麼意思.
我只想知道.
你到底是什麼意思.
我只想知道.
你到底是什麼意思.
我只想知道.
你到底是什麼意思.
我只想知道.
你到底是什麼意思.
我只想知道.
你到底是什麼意思.
我只想知道.
你到底是什麼意思.
我只想知道.
你到底是什麼意思.
我只想知道.
你到底是什麼意思.
我只想知道.
你到底是什麼意思.
我只想知道.
你到底是什麼意思.
我只想知道.
你到底是什麼意思.
我只想知道.
你到底是什麼意思.
我只想知道.
你到底是什麼意思.
我只想知道.
你到底是什麼意思.
我只想知道.
你到底是什麼意思.
我只想知道.
你到底是什麼意思.
我只想知道.
你到底是什麼意思.

$^{721}$我只想知道.
你到底是什麼意思.
我只想知道.
你到底是什麼意思.
我只想知道.
你到底是什麼意思.
我只想知道.
你到底是什麼意思.
我只想知道.
你到底是什麼意思.
我只想知道.
你到底是什麼意思.
我只想知道.
你到底是什麼意思.
我只想知道.
你到底是什麼意思.
我只想知道.
你到底是什麼意思.
我只想知道.
你到底是什麼意思.
我只想知道.
你到底是什麼意思.
我只想知道.
你到底是什麼意思.
我只想知道.
你到底是什麼意思.
我只想知道.
你到底是什麼意思.
我只想知道.
你到底是什麼意思.
我只想知道.
你到底是什麼意思.
我只想知道.
你到底是什麼意思.
我只想知道.
你到底是什麼意思.
我只想知道.
你到底是什麼意思.
我只想知道.
你到底是什麼意思.

$^{761}$我只想知道.
你到底是什麼意思.
我只想知道.
你到底是什麼意思.
我只想知道.
你到底是什麼意思.
我只想知道.
你到底是什麼意思.
我只想知道.
你到底是什麼意思.
我只想知道.
你到底是什麼意思.
我只想知道.
你到底是什麼意思.
我只想知道.
你到底是什麼意思.
我只想知道.
你到底是什麼意思.
我只想知道.
你到底是什麼意思.
我只想知道.
你到底是什麼意思.
我只想知道.
你到底是什麼意思.
我只想知道.
你到底是什麼意思.
我只想知道.
你到底是什麼意思.
我只想知道.
你到底是什麼意思.
我只想知道.
你到底是什麼意思.
我只想知道.
你到底是什麼意思.
我只想知道.
你到底是什麼意思.
我只想知道.
你到底是什麼意思.
我只想知道.
你到底是什麼意思.

$^{801}$我只想知道.
你到底是什麼意思.
我只想知道.
你到底是什麼意思.
我只想知道.
你到底是什麼意思.
我只想知道.
你到底是什麼意思.
我只想知道.
你到底是什麼意思.
我只想知道.
你到底是什麼意思.
我只想知道.
你到底是什麼意思.
我只想知道.
你到底是什麼意思.
我只想知道.
你到底是什麼意思.
我只想知道.
你到底是什麼意思.
我只想知道.
你到底是什麼意思.
我只想知道.
你到底是什麼意思.
我只想知道.
你到底是什麼意思.
我只想知道.
你到底是什麼意思.
我只想知道.
你到底是什麼意思.
我只想知道.
你到底是什麼意思.
我只想知道.
你到底是什麼意思.
我只想知道.
你到底是什麼意思.
我只想知道.
你到底是什麼意思.
我只想知道.
你到底是什麼意思.

$^{841}$我只想知道.
你到底是什麼意思.
我只想知道.
你到底是什麼意思.
我只想知道.
你到底是什麼意思.
我只想知道.
你到底是什麼意思.
我只想知道.
你到底是什麼意思.
我只想知道.
你到底是什麼意思.
我只想知道.
你到底是什麼意思.
我只想知道.
你到底是什麼意思.
我只想知道.
你到底是什麼意思.
我只想知道.
你到底是什麼意思.
我只想知道.
你到底是什麼意思.
我只想知道.
你到底是什麼意思.
我只想知道.
你到底是什麼意思.
我只想知道.
你到底是什麼意思.
我只想知道.
你到底是什麼意思.
我只想知道.
你到底是什麼意思.
我只想知道.
你到底是什麼意思.
我只想知道.
你到底是什麼意思.
我只想知道.
你到底是什麼意思.
我只想知道.
你到底是什麼意思.

$^{881}$我只想知道.
你到底是什麼意思.
我只想知道.
你到底是什麼意思.
我只想知道.
你到底是什麼意思.
我只想知道.
你到底是什麼意思.
我只想知道.
你到底是什麼意思.
我只想知道.
你到底是什麼意思.
我只想知道.
你到底是什麼意思.
我只想知道.
你到底是什麼意思.
我只想知道.
你到底是什麼意思.
我只想知道.
你到底是什麼意思.
我只想知道.
你到底是什麼意思.
我只想知道.
你到底是什麼意思.
我只想知道.
你到底是什麼意思.
我只想知道.
你到底是什麼意思.
我只想知道.
你到底是什麼意思.
我只想知道.
你到底是什麼意思.
我只想知道.
你到底是什麼意思.
我只想知道.
你到底是什麼意思.
我只想知道.
你到底是什麼意思.
我只想知道.
你到底是什麼意思.

$^{921}$我只想知道.
你到底是什麼意思.
我只想知道.
你到底是什麼意思.
我只想知道.
你到底是什麼意思.
我只想知道.
你到底是什麼意思.
我只想知道.
你到底是什麼意思.
我只想知道.
你到底是什麼意思.
我只想知道.
你到底是什麼意思.
我只想知道.
你到底是什麼意思.
我只想知道.
你到底是什麼意思.
我只想知道.
你到底是什麼意思.
我只想知道.
你到底是什麼意思.
我只想知道.
你到底是什麼意思.
我只想知道.
你到底是什麼意思.
我只想知道.
你到底是什麼意思.
我只想知道.
你到底是什麼意思.
我只想知道.
你到底是什麼意思.
我只想知道.
你到底是什麼意思.
我只想知道.
你到底是什麼意思.
我只想知道.
你到底是什麼意思.
我只想知道.
你到底是什麼意思.

$^{961}$我只想知道.
你到底是什麼意思.
我只想知道.
你到底是什麼意思.
我只想知道.
你到底是什麼意思.
我只想知道.
你到底是什麼意思.
我只想知道.
你到底是什麼意思.
我只想知道.
你到底是什麼意思.
我只想知道.
你到底是什麼意思.
我只想知道.
你到底是什麼意思.
我只想知道.
你到底是什麼意思.
我只想知道.
你到底是什麼意思.
我只想知道.
你到底是什麼意思.
我只想知道.
你到底是什麼意思.
我只想知道.
你到底是什麼意思.
我只想知道.
你到底是什麼意思.
我只想知道.
你到底是什麼意思.
我只想知道.
你到底是什麼意思.
我只想知道.
你到底是什麼意思.
我只想知道.
你到底是什麼意思.
我只想知道.
你到底是什麼意思.
我只想知道.
你到底是什麼意思.

$^{1001}$我只想知道.
你到底是什麼意思.
我只想知道.
你到底是什麼意思.
我只想知道.
你到底是什麼意思.
我只想知道.
你到底是什麼意思.
我只想知道.
你到底是什麼意思.
我只想知道.
你到底是什麼意思.
我只想知道.
你到底是什麼意思.
我只想知道.
你到底是什麼意思.
我只想知道.
你到底是什麼意思.
我只想知道.
你到底是什麼意思.
我只想知道.
你到底是什麼意思.
我只想知道.
你到底是什麼意思.
我只想知道.
你到底是什麼意思.
我只想知道.
你到底是什麼意思.
我只想知道.
你到底是什麼意思.
我只想知道.
你到底是什麼意思.
我只想知道.
你到底是什麼意思.
我只想知道.
你到底是什麼意思.
我只想知道.
你到底是什麼意思.
我只想知道.
你到底是什麼意思.

$^{1041}$我只想知道.
你到底是什麼意思.
我只想知道.
你到底是什麼意思.
我只想知道.
你到底是什麼意思.
我只想知道.
你到底是什麼意思.
我只想知道.
你到底是什麼意思.
我只想知道.
你到底是什麼意思.
我只想知道.
你到底是什麼意思.
我只想知道.
你到底是什麼意思.
我只想知道.
你到底是什麼意思.
我只想知道.
你到底是什麼意思.
我只想知道.
你到底是什麼意思.
我只想知道.
你到底是什麼意思.
我只想知道.
你到底是什麼意思.
我只想知道.
你到底是什麼意思.
我只想知道.
你到底是什麼意思.
我只想知道.
你到底是什麼意思.
我只想知道.
你到底是什麼意思.
我只想知道.
你到底是什麼意思.
我只想知道.
你到底是什麼意思.
我只想知道.
你到底是什麼意思.

$^{1081}$我只想知道.
你到底是什麼意思.
我只想知道.
你到底是什麼意思.
我只想知道.
你到底是什麼意思.
我只想知道.
你到底是什麼意思.
我只想知道.
你到底是什麼意思.
我只想知道.
你到底是什麼意思.
我只想知道.
你到底是什麼意思.
我只想知道.
你到底是什麼意思.
我只想知道.
你到底是什麼意思.
我只想知道.
你到底是什麼意思.
我只想知道.
你到底是什麼意思.
我只想知道.
你到底是什麼意思.
我只想知道.
你到底是什麼意思.
我只想知道.
你到底是什麼意思.
我只想知道.
你到底是什麼意思.
我只想知道.
你到底是什麼意思.
我只想知道.
你到底是什麼意思.
我只想知道.
你到底是什麼意思.
我只想知道.
你到底是什麼意思.
我只想知道.
你到底是什麼意思.

$^{1121}$我只想知道.
你到底是什麼意思.
我只想知道.
你到底是什麼意思.
我只想知道.
你到底是什麼意思.
我只想知道.
你到底是什麼意思.
我只想知道.
你到底是什麼意思.
我只想知道.
你到底是什麼意思.
我只想知道.
你到底是什麼意思.
我只想知道.
你到底是什麼意思.
我只想知道.
你到底是什麼意思.
我只想知道.
你到底是什麼意思.
我只想知道.
你到底是什麼意思.
我只想知道.
你到底是什麼意思.
我只想知道.
你到底是什麼意思.
我只想知道.
你到底是什麼意思.
我只想知道.
你到底是什麼意思.
我只想知道.
你到底是什麼意思.
我只想知道.
你到底是什麼意思.
我只想知道.
你到底是什麼意思.
我只想知道.
你到底是什麼意思.
我只想知道.
你到底是什麼意思.

$^{1161}$我只想知道.
你到底是什麼意思.
我只想知道.
你到底是什麼意思.
我只想知道.
你到底是什麼意思.
我只想知道.
你到底是什麼意思.
我只想知道.
你到底是什麼意思.
我只想知道.
你到底是什麼意思.
我只想知道.
你到底是什麼意思.
我只想知道.
你到底是什麼意思.
我只想知道.
你到底是什麼意思.
我只想知道.
你到底是什麼意思.
我只想知道.
你到底是什麼意思.
我只想知道.
你到底是什麼意思.
我只想知道.
你到底是什麼意思.
我只想知道.
你到底是什麼意思.
我只想知道.
你到底是什麼意思.
我只想知道.
你到底是什麼意思.
我只想知道.
你到底是什麼意思.
我只想知道.
你到底是什麼意思.
我只想知道.
你到底是什麼意思.
我只想知道.
你到底是什麼意思.

$^{1201}$我只想知道.
你到底是什麼意思.
我只想知道.
你到底是什麼意思.
我只想知道.
你到底是什麼意思.
我只想知道.
你到底是什麼意思.
我只想知道.
你到底是什麼意思.
我只想知道.
你到底是什麼意思.
我只想知道.
你到底是什麼意思.
我只想知道.
你到底是什麼意思.
我只想知道.
你到底是什麼意思.
我只想知道.
你到底是什麼意思.
我只想知道.
你到底是什麼意思.
我只想知道.
你到底是什麼意思.
我只想知道.
你到底是什麼意思.
我只想知道.
你到底是什麼意思.
我只想知道.
你到底是什麼意思.
我只想知道.
你到底是什麼意思.
我只想知道.
你到底是什麼意思.
我只想知道.
你到底是什麼意思.
我只想知道.
你到底是什麼意思.
我只想知道.
你到底是什麼意思.

$^{1241}$我只想知道.
你到底是什麼意思.
我只想知道.
你到底是什麼意思.
我只想知道.
你到底是什麼意思.
我只想知道.
你到底是什麼意思.
我只想知道.
你到底是什麼意思.
我只想知道.
你到底是什麼意思.
我只想知道.
你到底是什麼意思.
我只想知道.
你到底是什麼意思.
我只想知道.
你到底是什麼意思.
我只想知道.
你到底是什麼意思.
我只想知道.
你到底是什麼意思.
我只想知道.
你到底是什麼意思.
我只想知道.
你到底是什麼意思.
我只想知道.
你到底是什麼意思.
我只想知道.
你到底是什麼意思.
我只想知道.
你到底是什麼意思.
我只想知道.
你到底是什麼意思.
我只想知道.
你到底是什麼意思.
我只想知道.
你到底是什麼意思.
我只想知道.
你到底是什麼意思.

$^{1281}$我只想知道.
你到底是什麼意思.
我只想知道.
你到底是什麼意思.
我只想知道.
你到底是什麼意思.
我只想知道.
你到底是什麼意思.
我只想知道.
你到底是什麼意思.
我只想知道.
你到底是什麼意思.
我只想知道.
你到底是什麼意思.
我只想知道.
你到底是什麼意思.
我只想知道.
你到底是什麼意思.
我只想知道.
你到底是什麼意思.
我只想知道.
你到底是什麼意思.
我只想知道.
你到底是什麼意思.
我只想知道.
你到底是什麼意思.
我只想知道.
你到底是什麼意思.
我只想知道.
你到底是什麼意思.
我只想知道.
你到底是什麼意思.
我只想知道.
你到底是什麼意思.
我只想知道.
你到底是什麼意思.
我只想知道.
你到底是什麼意思.
我只想知道.
你到底是什麼意思.

$^{1321}$我只想知道.
你到底是什麼意思.
我只想知道.
你到底是什麼意思.
我只想知道.
你到底是什麼意思.
我只想知道.
你到底是什麼意思.
我只想知道.
你到底是什麼意思.
我只想知道.
你到底是什麼意思.
我只想知道.
你到底是什麼意思.
我只想知道.
你到底是什麼意思.
我只想知道.
你到底是什麼意思.
我只想知道.
你到底是什麼意思.
我只想知道.
你到底是什麼意思.
我只想知道.
你到底是什麼意思.
我只想知道.
你到底是什麼意思.
我只想知道.
你到底是什麼意思.
我只想知道.
你到底是什麼意思.
我只想知道.
你到底是什麼意思.
我只想知道.
你到底是什麼意思.
我只想知道.
你到底是什麼意思.
我只想知道.
你到底是什麼意思.
我只想知道.
你到底是什麼意思.

$^{1361}$我只想知道.
你到底是什麼意思.
我只想知道.
所以這就是我們要大概知道的框框.
潘博華說登山寶訓的時候.
其實簡單來說就是跟隨耶穌.
呼召十字架這些關鍵詞是有關係的.
然後你看到當潘博華說跟隨基督的時候.
他就很強調.
當一個人來跟隨基督的時候.
當他願意去信服基督的命令的時候.
這個命令是一個很單純的信服.
就是說你不想太多.
就像一個小朋友一樣.
你去做一個很簡單而單純的跟隨和信服.
而這個跟隨必然是可見的.
這個行動一定會被人看到.
這句話是這樣說的.
這句話的相反是這句.
我特別改出來.
我覺得是寫得挺好的一句話.
他說逃避.
就是逃避去一個不可見的地方.
當你去隱藏或者去離開人的眼光.
被人看不到的時候.
其實這個正正是否定上帝的呼召.
所以對潘博華來說.
當耶穌去呼召我們做基督徒的時候.
本身這件事情就是可見到的.
必定可見到的.
這個也是我們回到第一課.
第一課第一課我們說什麼.
就是我們作為基督徒.
如果你有上我們的八課的話.
什麼叫基督徒.
基督徒不是一個得救名單.
也不是一個宗教徒的名稱.
而是一個有意義的基督徒.
在社會裡面見證耶穌的一群人.
所以這個跟潘博華的想法是一樣的.

$^{1401}$當我們自稱做基督徒.
或者成為基督徒之後.
你就是可見的.
當你願意去跟隨耶穌的時候.
這個行徑必定會被人看到.
這是第一個.
相反來說你逃避去隱藏.
被人看不到.
這個是違反你對於上帝耶穌基督呼召你的照明.
所以你會發覺.
他整個的神學或者是教會和世界的關係.
最重要的主線就是這一條.
就是跟隨耶穌的呼召.
和信服基督的命令.
當我們去看.
當潘博華在這條主線裡面.
開始講解登山寶藏的時候.
他很簡單地將第五章和第六章.
是兩個完全不同的章節.
其實我講道講過.
我以前講第五章的時候.
如果你們有聽我編導講.
這個基督的意義.
我也提過這個點.
之後當潘博華或者是耶穌.
耶穌在講你們身上的炎和光的時候.
這個正正就是在講這個事情.
你們是身上的炎.
你們身上的光.
所以你們的光應當這樣照在人前.
叫他們看見你們的好行為.
變成榮耀.
歸咎於你們身上的福.
當耶穌在登山寶藏裡面.
講你們是炎和光的時候.
已經是在講基督徒在世界上是可見的.
當你不可見就是什麼.
正正就是鬥底下的光.
大家知道的東西.
你明明是光.

$^{1441}$但你被人收買或自我收買的時候.
這個正正是在違反你自己的本質.
不過很有趣的.
當你細心想想炎和光的時候.
你會發覺這是一個很好的比喻.
為什麼呢?.
因為炎是什麼樣的炎?.
炎是什麼樣的炎?.
炎是不能夠假裝炎的.
炎就是炎.
炎不是假裝出來的.
也不是要做出來的.
炎本身就是炎.
所以耶穌提出了一個很特別的假設.
就是說炎虐失了味.
如果炎失去味道.
那怎麼辦呢?.
怎麼會再咸呢?.
這裡有一些毒理科.
炎怎麼能夠失去咸味?.
炎怎麼可以不咸呢?.
其實我問過一些化學博士.
炎怎麼可以不咸呢?.
炎能不能不咸呢?.
或者作為智力題.
炎怎麼可以不咸呢?.
你會發覺炎就是.
我問過一些問題.
你可以把它的元素抽走.
炎是什麼元素呢?.
什麼?.
氯化鈉.
氯化鈉,對吧?.
Sodium Chloride,對吧?.
我完全不懂文科.
如果把它的元素抽走.
那它就不咸了.
但問題是.
如果把它的元素抽走.
那它就是炎了.

$^{1481}$它就不是炎了.
所以炎是堅強的.
你要摸就是.
弄到它不是炎.
但如果它是炎.
它就一定會咸.
後來有些人對這個問題有不同的答法.
他說那些工業用炎的.
可能就不咸.
但工業用炎是什麼呢?.
可能成績會死掉.
所以你也不需要.
你只需要把它抽走.
你不需要把它抽走.
你只需要把它抽走.
那些是會死掉的.
所以你也不知道是不是不咸.
因為你真的沒吃過.
所以唯一一個.
如果是智商題就會答的.
什麼炎是不咸的?.
作為智商題的話.
不是專業的那些.
但是什麼炎是不咸的?.
就是放在一個坑裡面的炎.
為什麼那些炎不咸?.
就是因為它沒被人品嘗過.
所以要它不咸.
只能夠是智商題.
就是說你只能在沒被人品嘗過的炎.
就會不咸.
正如光也一樣.
當我們放在鬥底的光會不光.
其實這個也是假設.
其實它仍然是在光的.
不過你見不到的意思.
所以這個正正就是耶穌一個很巧妙的比喻.
就是那個見證.
那個被人看到.
品嘗到的行動.

$^{1521}$其實是與生俱來的.
就是你作為基督徒.
你本身就是.
帶著這種見證在身上.
除非你將這種見證的能力.
收起來.
藏起來.
放在一個炎坑裡面.
或者是在鬥底下面.
所以炎和光的比喻正正是.
一個行動的問題.
你就是那個炎和光.
這個是你的being.
作為基督徒的being.
但是這個being同時是你的行動.
兩者是分不開的.
而且你本身就是要去.
就是會做這些事.
炎就自然會咸.
他不需要特意去想一些咸的東西.
就是不需要特意去做一些咸的東西.
就是這樣.
光就一樣.
光就是會照射出去.
他不需要特意去想一些plan.
想一些intentional的東西.
動機去這樣去過.
所以當我們去想這個比喻的時候.
撲克語也有這樣說.
不過他說得比較足夠.
會說得比較優雅.
就是炎和光就是這樣的一個本質.
他本質上就是一個見證的東西.
他不需要他特意去做一些什麼.
而那些東西是可見的.
必定可見.
當他.
就像潘鳳華說的.
逃離到不可見的地方的時候.
其實是什麼呢.

$^{1561}$就是違反了他的呼召.
違反了他的呼召.
所以這個就是潘鳳華在第五章裡面.
特別說到.
他仍然跟隨耶穌的思想來說這件事情.
當一個人來跟隨耶穌.
聽見耶穌的呼召.
做基督徒.
為他受苦的時候.
這樣的一個舉動.
必然是可見的.
他不需要特意去想一大餐.
如何見到或是什麼問題.
他就會被人見到.
所以這一話.
整個第五章.
潘鳳華就稱之為.
基督徒生活的超平凡.
這個我講到講過.
叫做Ause Authenticum.
這個字是一個很特別的字眼.
整個第五章.
想想第五章說什麼.
第五章說什麼.
就是說你們被人打完又如何.
再打錯又如何.
愛仇敵.
這些行徑.
你們要完全.
像我們天父完全一樣.
這些超級難度高的行徑.
因為跟隨耶穌的緣故.
基督徒一群門徒.
願意去這樣做.
這些事情.
愛仇敵.
為神禱告.
幫我犯姦淫.
更加不會動人臉.
這些事情.

$^{1601}$他稱之為基督徒生活的超平凡.
Ause Authenticum.
這個字我都說了.
是一個很奇怪的字眼.
Authenticum就是平凡.
Ause就是超過.
所以中文翻譯為超平凡.
基督徒生活是超平凡.
就是說是一個不正常的生活.
是一個很.
我都說了.
我那時候講到.
是一個超乎你的發揮水準.
超水準的.
超過你平時發揮的.
而今天所講的.
這些行徑.
其實是很出位的.
是很並肩的.
是一個很.
一個光影會看著.
你想想.
被人打完右邊.
就打左邊.
愛仇敵.
這些全部都是一些很.
很顯眼的東西.
明白嗎.
所以基督徒在世上.
特別當他願意跟隨耶穌的時候.
這些行徑.
整個第五章所講的.
正正就是一些超乎平凡人所做的行為.
這些行為肯定是可見的.
肯定會被人見到.
不會見不到的.
所以基督徒在世上的生活.
就是這樣.
願意跟隨耶穌.
簡單的聽從.

$^{1641}$就做出一些很radical.
很超級吸睛的行動.
肯定會讓人看到.
從而看到耶穌基督在那裡.
這就是那個原意.
所以這是第五章所講的.
今天我們不是講第五章.
是講第六章.
但這是背景.
你知道第五章整個超平凡的行徑.
是可見的時候.
然後Don Puff說.
講第六章.
是一個180度的轉向.
雖然是這樣.
但第五章所講的.
耶穌在讀中一講就反過來.
他說你們要小心.
不可將善事行在人的面前.
故意叫他們看見.
若是這樣你們就沒有天賦上前.
這是第六章裡頭十幾節所講的.
後面有三個行徑.
一個是施捨.
一個是禱告.
一個是禁食.
三個都是猶太人所講的好事.
好行為.
一些敬虔的舉動.
所以Don Puff就用了這段經文.
來講基督徒的隱藏性.
所以今天的主題叫基督徒的不可.
隱藏性和隱藏性.
就是這兩個章節所講的嚮導.
耶穌就這樣說.
耶穌就說你們的善事.
你們的好行為.
不要讓別人看到.
不可將善事行在人的面前.
故意叫他們看見.

$^{1681}$所以發覺第六章所講的.
似乎又和第五章剛剛相反.
今天我們就開始講多一些.
沿著這本書裡面的講法.
這個隱藏的意.
Don Puff說第六章所講的.
是什麼呢.
大家不用看這裡.
可以回去看YouTube.
如果想看的話.
這個字.
所以你看到就會被你看到.
總之開頭就講這件事.
第五章講了這麼多這些東西.
這些超越的行為.
一些超頻繁的行為.
一些跟隨耶穌緣故.
這麼出位的行為.
這麼好的東西.
其實會有很多的.
side effect 出來.
什麼是side effect.
就是有些人會將這些行為.
變成了一些敬虔的驕傲.
一個屬靈的驕傲.
我被人打完左邊.
左邊踩右邊臉.
一件很吸睛的事.
一件很屬靈的事.
可以成為一件很驕傲的事.
我為我的仇人討告.
這些一方面成為了一個見證.
同時也成為了某種.
另一種的見證.
我似乎是在宣傳.
或是在高舉某些東西.
所以這個就是.
在開頭那一段裡面.
帶出來的問題.
其實這些東西.

$^{1721}$很容易會讓人覺得.
他稱之為宗教狂熱者.
或者是.
今日的靈派.
意思其實就是這些.
我不是說靈派不好.
而是說當時的字.
其實有點像宗教狂熱者.
他們會將那些行為.
變成了一些高舉.
很明顯的東西.
所以這個是.
怎麼翻譯這些言著.
今日裡面其實有很多.
我稍後再說.
所以這些很敬虔的行為.
這些跟隨耶穌裡面.
很吸睛的行為.
很容易會變成這些東西.
屬靈的驕傲.
成為了眾人的焦點.
甚至偏離了.
他跟隨耶穌的目的.
所以他就說.
這句話大家可以一起看.
堅持了基督門徒的能見性.
那個visibility.
當然有他的根據.
但這個visibility.
不是他的目的.
如果成為了我們的目的.
我們就失去了我們最初的目的.
就是跟隨耶穌.
而且我們只要做過一次.
我們就決不能再在我們的別的地方.
重新再起來做.
既然我們去選擇跟隨耶穌的時候.
我們就不能夠本末倒置.
或者失去了我們的初心.
明白嗎?.

$^{1761}$即是有人去跟隨耶穌.
然後跟隨得很吸睛.
成為了焦點.
被人見到.
然後失去了跟隨耶穌的初心.
或者本質.
所以這就是Pulver所說的.
我們的行動是可見的.
但卻不是為了叫別人看見而行.
所以這就是他開始慢慢說出的分別.
基督徒在香港.
我們的見證肯定是可見的.
但被人見到不是我們的目的.
Pulver也這麼說.
不是我們的目的.
那如果不是目的呢?.
這就是他開始慢慢梳理的問題.
其中一個小小的點.
下面我也寫多一點.
其實就是要隱藏.
所以為什麼耶穌說你們要隱藏.
就是這個意思.
其實是矛盾的.
既是要被人見.
但又要隱藏.
隱藏什麼呢?.
整個第五章和第六章之間有什麼差別.
如何能夠融合兩篇經文呢?.
Pulver提出了三點.
他說兩個章例如何疏解.
第一個問題是.
我們作門徒的能見性是向誰隱藏呢?.
既然我們被人見到.
我們隱藏什麼呢?.
我們不被人見到.
是被誰不見到呢?.
那必然不是別人.
我們不是不被別人見到.
因為我們就是光.
我們就是室內的光.

$^{1801}$是被人見到的.
我們是向自己隱藏.
這個是值得我們去想的.
他說我們是向自己隱藏.
我們的工作就是跟隨.
我們就是去仰望那位耶穌.
我們必須不覺得自己是異.
而只是仰望耶穌當中.
他就好像不是超平常的.
而是十分平常又自然的了.
第五六行.
他什麼意思呢?.
就是說當我們向自己隱藏的時候.
我們做著做著去跟隨耶穌的時候.
我們不覺得自己在做什麼特別的.
我們所謂的超平凡.
其實當你向自己隱藏的時候.
你是會不覺得自己有什麼超平凡.
你只是在跟隨耶穌.
你是在做著平常又自然的事.
雖然後果是很超平凡.
但你作為一個跟隨耶穌的人.
你應該是不覺得有什麼特別.
所以簡單來說就是那個我.
是沒有了的.
我是向自己隱藏了.
如果那個超平凡成為了重點的話.
那我們就變成了吸睛的地方.
當你覺得自己跟隨耶穌.
做些很好的事情的時候.
你覺得很好的時候.
你開始想想那些人看到你有多好的時候.
其實這個就已經偏離了那個的意思.
變成了那個超平凡的事情成為了焦點.
而不是跟隨耶穌.
所以這個說法說得有點深.
但其實意思是大概這樣.
就是說當我們向自己隱藏的時候.
其實我們會做著做著.
我們不覺得自己有什麼特別.

$^{1841}$你應該這樣想.
我們只不過是跟隨耶穌.
當你考慮著嘗試用第三個角度.
看回自己的時候.
這個就是問題的開始.
不知道你值得想一想.
你平時做基督徒或者你做了一件好事.
或者你是作見證的時候.
當你見證耶穌的時候.
你會不會突然從第三個角度.
看回自己.
想想這個畫面有多美.
這個畫面有多像見證耶穌.
當你用第三個鏡頭去看回自己的時候.
其實這件事正正就是開始偏離了.
跟隨耶穌的方向.
因為你是沒有向自己隱藏.
所以這段書講得很有意思.
就是基督徒的生活的本質.
最後那一段.
是超平常的.
同時也是平常自然的.
看不出來的.
因為他也看不到自己在做什麼.
只不過跟隨著耶穌.
所以這個就是布木佛羅第一個點.
他嘗試梳理兩個的張力.
他說其實隱藏是向自己隱藏.
第二他就說.
作門徒的能見性和不能見性.
怎樣能夠結合在一起呢.
怎樣能夠成為一件事呢.
他說.
這個我也有點心.
因為他寫得很短.
所以我也沒什麼機會多理解.
他說超平常的能見性和不能見性.
正正就是那個十字架.
十字架正正就是能見和不能見的中心.
這個中文翻譯有點問題.

$^{1881}$譯得不太好.
他說十字架是必須的.
看不出的.
同時也能看見的.
他就是超平常之處.
但原文裡面的字其實更加簡單.
他說十字架.
Dark Christ.
下面是一大堆字.
所以十字架就是必然的東西.
隱藏的東西.
可見的東西.
和超平凡的東西.
所以.
大家可以看十字架和跟隨的意思.
所以唯有背負十字架.
才能讓我們同時.
真正被看見的十字架.
隱藏的也是十字架.
這部分我也覺得不容易理解.
十字架成為了我們可見和不可見的.
最核心的地方.
第三.
第五六章的矛盾.
如何疏離呢.
他說答案在作門徒的意義裡面.
作門徒的意義是什麼.
就是專一歸順他.
他所含義的就是仰望主和跟隨耶穌.
所以你做這件事很簡單.
你就是跟隨耶穌.
這正正就是我們要做的事.
所以如果他只是仰望基督徒生活的超平常特質.
他就不再是跟隨基督.
這和剛才說的差不多.
但是跟隨耶穌的本質就在這個位置.
所以你問自己.
是不是跟隨耶穌.
這就是最重要的問題和答案.
其他的後果.

$^{1921}$跟隨耶穌之後有什麼好事發生.
有什麼影響.
有什麼效力.
這些全部都是不重要的.
或者不能夠取代.
單純跟隨耶穌這件事.
所以這就是唯一我們能夠自然做的事.
所以回到那個位置.
就是簡單的聽從和跟隨.
成為了我們要做的事.
最後我們有個總結.
其實我們所做的事.
是基督的德行.
那些好事.
其實只能夠是因為耶穌基督的緣故.
所以作門徒的德行.
只要你完全不覺得.
只要你不察覺的時候才能做出來.
就是當你去察覺自己在做好事的時候.
這件事就會開始偏離了本質.
這個道理是簡單.
但是對教會來說可以有很多的用處.
我們現在會談談這件事.
當我們不察覺的時候做出來.
才是一個最純潔的東西.
當你察覺甚至有部鏡頭看到自己的樣子.
甚至想象到後果的時候.
這個正正就是我們要小心的東西.
任何你察覺到的好事.
都開始有些問題.
因為任何的好事都是你不察覺到的.
有一些例子.
假如我們想知道我們的善恨和愛的時候.
那就不是愛.
甚至對我們愛仇敵的愛都不是察覺到的.
我覺得這個故事是真的.
他說什麼叫愛仇敵.
就是當你去愛仇敵的時候.
其實你已經不覺得是仇敵了.
你愛一個人.

$^{1961}$你愛到已經忘記他是仇敵.
當你還很生氣.
我想要仇敵.
你仍然在看自己的第三個鏡頭.
看著愛仇敵的圖畫.
所以很特別的.
用了一個盲目的狀態.
當你自願的盲目.
當你看不到自己的時候.
其實這個正正就是隱藏了自己的東西.
愛仇敵.
你已經愛到一個對象.
你已經忘記他是仇敵.
那就是愛仇敵.
就是我有個自拍.
有個看著鏡頭看著愛仇敵的圖畫.
有多漂亮.
所以那個見證就是這樣.
那個見證是一個.
不含一種第三身的動機.
你甚至忘卻了自己在做這個見證.
你只是在跟隨耶穌.
所以這個就是顧佛的意思.
所以在最後寫得很好.
一個自願的盲目.
其實是基督所照的眼目.
只是讓他確知道.
在生活裡面正正是在隱藏著自己.
所以這個就是潘福華的神堂.
嘗試去理解整個的見證問題.
我們在世界上肯定是被可見的.
但是我們這個被人們看到不是我們的目的.
當我們嘗試將它成為一個目的.
甚至是一個營運的方針的時候.
可能這就變成了不簡單跟隨耶穌.
可能都是好的事情.
但這就不是一個好的.
真正最單純跟隨耶穌的行徑.
就是很簡單跟隨耶穌的時候.
你會忘記了這些事情會被人看到.

$^{2001}$而這樣你會被人看到.
今天就講了大扎潘福華.
這些1937年的這本書.
對今天我們留堂有什麼關係呢.
我們留堂這間教會.
你如何做一間見證耶穌的教會呢.
當然我們回到第一課.
第一課我們說當我們被呼召做基督徒的時候.
其實我們就已經在世界和社會裡面.
所以用一個炎和光的比喻.
你本身的呼召就是跟隨耶穌.
而在世界裡面去見證耶穌.
這件事是你整個人的行動.
第一我想講的是.
見證是一個行動.
不是一篇作出來一段一千至五分鐘的講稿或者故事.
而是你整個人的行動.
都說了你回想失見證就會想到什麼叫作見證.
失見證是什麼呢.
就是你的行動.
你不是故意很自然地做出來的.
失見證就是你生命里不需要刻意做出來的.
你不需要去失見證.
最是你流露出來的.
所以相反.
作見證也是一樣.
是你的行動.
是你整個人跟隨耶穌的時候.
整個的舉動.
所以這是我們第一樣要知道的東西.
從個別來講.
我們在流唐經常強調.
我們很想大家在社會裡面去作見證.
所以不需要特意有很多的侍奉.
而你在社會裡面的見證.
就是一個很重要的你要做的事情.
很多東西.
從你的家庭到你的公司到社會裡面.
都是我們希望大家能夠做的見證.
而所謂做.

$^{2041}$其實都是跟隨耶穌.
老實說.
都是一個這麼簡單的要求.
就是在世界裡面去跟隨耶穌.
這個就是我們最首要要做的見證.
那件好行為就在世界裡面做出來.
而我們不是特意要做給別人看.
不過是會被人看到的.
你明白我的意思嗎?.
就是他肯定會被人看到.
但這個不是你的目的.
你的目的就是做回應該你要做的事情.
我想這個是我們每弟子每妹都要去做.
和都記得我們要做的事情.
當然作為教會肯定不是只有這個.
教會有一個群體的方向.
這個是個人的.
你自己作為一個基督徒.
你要做的事情.
但作為一間教會.
肯定是會有一個不同程度的層次和考慮.
我們在牧者會談的時候都是一樣.
你教一個初訓的作見證.
總有一些教導成分.
總有一些所謂.
你就做這件事.
那就做了.
那就學會了去作見證.
作為群體.
我們是會有一些一起去做的事情.
那些事情就不會沒有計劃.
你不會突然間碰到就做.
總有一些計劃.
總有一些特意要做的事情.
但重點不是.
教會始終是有策略的.
有策劃.
但不代表這些事情是特意要做給別人看的.
所以教會的群體見證肯定會有一些計劃.
有些Emmy大家幫忙的事情.

$^{2081}$這些肯定是會預先來計劃好的事情.
所以第二個我們可以參與的.
就是我們整個教會群體的工作.
我們會有flown.
這些都是我們希望能夠大家一起去做的事情.
這個也是見證來的.
這個也是一些好事.
大家記住我們不是特意要去sell耶穌.
但也不是特意要avoid.
不講耶穌.
簡單地做一件耶穌叫我們做的事情.
我們要關心論捨.
我們見到有人需要就幫助他.
就是這樣.
最後.
這一段不重要.
大家都知道flown是我們很重要的一個元素.
第三就是我們的群體生活.
你發覺我們是沒有很多很特意要做出來的事情.
希望大家能夠透過大家的群體生活.
能夠讓人看見.
從而去見證耶穌.
所以那些生活.
首先問你生活是不是一些被人見到的生活.
是不是一個作門徒的生活.
是不是我們班基督徒所做的事情.
是不是作門徒的生活.
有兩個層次.
第一個就是我們整個敬拜群體.
可以叫人參加.
讓人看見.
從而讓人認識耶穌.
第二就是我們的club.
club絕對不是我們一群人打乒乓球.
或者行山就玩完很開心的東西.
而是我們都很想透過這些生活.
能夠讓人看得見我們本身的生命.
從而認識耶穌.
所以這件事是我們很重要的目的.
搞club其實是為了讓大家可以.

$^{2121}$更多的機會向人看見我們的生命.
讓一群行山友.
透過行山能夠認識我們的群體.
看見我們的生活.
讓一群咖啡友.
或者其他不同的block game友.
都能夠這樣來見到耶穌.
這些是我們的生命流露.
不是我們純粹要去sell什麼.
目的就是這樣做.
我想這兩件事是分不開的.
我們本身的群體.
就是我們整個展現的內容.
所以我想我們流行都開了五年.
我們經常強調.
我們要在香港裡面見證耶穌.
是時候要見證耶穌.
不是大家聚在一起.
不是純粹依次.
而是我們很需要想想.
如何能夠自然地.
透過我們的生命來見證耶穌.
這就是我們想做的事.
事後不如叫潘Sir出來.
我們可以一起有些問題和大家談談.
大家對於見證這件事有什麼看法呢.
或者你如何理解教會在基督徒.
在香港見證是怎麼樣呢.
我們一起談談.
最後一次call time.
對,還有一個.
最後的缺字.
我都講了第一堂所講.
我們說要新見證耶穌.
這是我們第一堂所講的內容.
就是叫大家的呼召.
正正就是來見證耶穌.
所以大家可以一起來.
這是他思想的事情.
這次應該很靠近了.

$^{2161}$我想第一天回教會就會講這件事.
你成為基督徒了.
要有好見證.
對於大家聽了那麼久.
或者經歷了那麼多年.
可以分享一下嗎.
舉手就是了.
我們有個麥給你.
謝謝.
對於....
是,後面.
是,Testing.
聽到John這樣講.
我會覺得是一個真正的謙虛.
可能是人家看到你一些好行為.
或者是見證的時候.
你會說.
人家會說你很好.
你就會說一句.
也算是這樣吧.
不是的.
然後他就會說一句.
你謙虛而已.
這個時候你就不是.
尖尖自喜地覺得自己很謙虛.
而是真的一份.
覺得自己還有不足的謙虛.
這樣聽下去.
不過我還有一個問題.
如果是一個長期那麼真正謙虛.
或者是一個世界那麼追求自我實現.
想建立自信.
建立多一點我們的自我價值的時候.
長期在一個那麼自我毀滅的狀態.
如何能夠有滿足感繼續下去呢.
我只能夠答就是跟隨耶穌.
如果是奔赴黃毛的話.
他就會覺得你就是在看前面的耶穌.
這樣跟隨.
少一點看自己.

$^{2201}$我剛才說的.
少一點用第三部鏡頭看著自己的畫面.
這個是他所說的.
我們有機會一起弄個讀書會.
看這本書.
就是說跟隨與個人.
就是跟隨與個別者.
他就說這件事.
其實你正正就是一個簡單的跟從.
就是這樣.
將那件事單純地去思考.
放下其他的考慮.
這個也是我們在樓堂里.
徐牧師經常說的.
就是這樣的感覺.
那種很簡單的思想.
跟隨耶穌.
其他的東西放下.
這個正是我們最關鍵的核心.
當然牧羊裡面我們有很多東西.
潘sir也說了很多這些.
自我價值.
K-Club的東西.
用Pull for Pull的話就是跟隨.
有沒有多一點.
我不是鑽著你的字.
剛才有個字是說自我抗爭.
其實我就不是太喜歡用這個字.
因為事實上你做得到.
而是做得好.
別人是欣賞你.
所以就不用說不是的.
其實你是的.
你真的是.
因為你真的做得到.
做得好.
別人欣賞你.
或者欣賞到這件事的時候.
那不就是對你一個肯定.
on the right track.

$^{2241}$繼續做下去.
你沒理由說.
你煮了些東西.
其他人說其實一般般.
不是我排到出街.
個個都等著吃這些東西.
你不要說.
隨便吧不是的.
有客觀標準.
這個很重要.
今天教會裡面.
剛才我說了很多application.
今天教會最大問題是什麼.
就是我在想客人有多少.
還有我在想客人有什麼反應.
而不是想我煮的食物.
怎麼煮食物煮得好.
今天教會最大問題就是這裡.
想很多的後果.
想很多marketing.
或者想很多其他的人.
怎麼想和怎麼多人來.
所謂多少人來.
但不想最簡單的.
就是我怎麼煮好食物.
專注在你的烹飪裡面.
才是你的目標.
當你煮得好.
都是專注在自然多人來.
明白嗎.
所以那個意思就在這裡.
今天我們教會見證就是這樣.
我們應該是專注在我們.
去跟隨耶穌裡面.
自然就有好的行為.
吸睛的一些甚至人數.
但這不是我們應該要想的東西.
我們想怎麼去做好教會.
怎麼去跟隨耶穌.
應該做什麼.

$^{2281}$應該要做的.
而不是想教會的後果.
那些都有成功的.
今天方安派教會的定義就是.
將教會變成一個生意.
這樣做多少人會來.
有什麼需要就煮什麼給他們吃.
但完全沒有想煮什麼.
完全沒有專注在烹飪裡面.
我想這就是布福華想說的東西.
問:你好.
我想問一點.
因為是關於群體和隱藏性和不能隱藏性.
我覺得作為一個基督徒.
就會覺得我要做好這個見證.
做好見證代表他欣賞我這件事.
但我們做事工某程度上也涉及這件事.
但我們不能讓別人知道我們的矛盾差異.
我就比較疑惑隱藏性和不能隱藏性和事工上的東西.
問:再說幾句事工是怎樣的.
可能我們是做一些關心社群的事工.
我們是關心社群的事工.
但也可能會被人標籤.
其實你只是想做一個見證給別人看.
我會疑惑如何不隱藏性又隱藏性的比例.
我說應該不是太想這件事.
現在很多人都不想這件事.
剛剛Emmy才發了一張照片給我看.
就是說他們和一群基督小學生考完試就請他們吃炸雞.
這很明顯不是想讓別人看到的事.
是一個愛的行為想讓他們成長.
這是一個很好的見證.
因為他不是想讓別人看到他有多好或做什麼好事.
而是從一個初心出發.
就是想幫他們 愛他們.
但這些事也不需要隱藏.
但你做的時候的動機.
你知道是為了什麼.
不是為了讓別人看到我耶穌有多好.
或者見證耶穌做了什麼好事.

$^{2321}$所以動機和次序要分得清楚.
但做完之後不妨讓別人看到.
也不妨讓人認識耶穌是可以的.
直接的例子就是.
你可能沒有見過或聽過.
我親眼見過.
有些群體去掃街.
餵 等一下.
這個位置可以嗎?清清楚楚.
這樣的.
拍了位置才開始掃.
掃完就走了.
海邊掃垃圾也是這樣.
你知道那些位置是為了那些來做.
我想重點仍然是出發.
最初用什麼方式去做到那件事.
結果是什麼其實不是我們最關心的.
有兩個問題或想法.
第一個就是.
我平時有探討基層家庭.
剛才你提到有些團體.
喜歡隱藏自己基督徒的身份.
有些打著旗號說我是基督徒.
你會不會回教會.
我記得最近有一次探討的經驗.
就是.
目者說.
其實我都探討了你一段很長的時間.
其實是否有什麼.
令你不能夠放開心去接受耶穌呢?.
我感覺你又不是抗拒.
但你又好像不太接受.
他不斷地教導.
我因為第一次接觸這個家庭.
其實我都不太認識.
但那一刻我自己都覺得.
突然間很感激.
他本身是關心他最近的需要.
工作的需要.
突然間就說這些.

$^{2361}$比較基督的內容.
其實對於我來說.
我都不舒服.
但我又覺得.
這些目者.
其實我又很尊敬.
他亦都是.
雖然我聽起來都覺得不舒服.
但我又覺得.
他又好像很有經驗去做這些事.
他自己又覺得.
他平時都是這樣.
我又覺得.
一方面我又覺得有些壓力.
另一方面我又覺得.
如果不是有些目者這樣去傳.
可能未必傳得成一個整杵的福音.
但剛才聽你這樣說.
我就會有些迷失.
我真的在那個處境里.
其實我可以怎樣做呢.
我是否.
因為你不斷地探討.
其實你每一次都是重復.
你最近怎樣.
重復禁放.
究竟應該怎樣做呢.
何時隱藏何時不隱藏.
或者目者去傳福音的方式.
或者我自己覺得.
我怎樣傳.
怎樣去作見證的方式.
是真的不同的.
但其實是否一定要說.
誰是好誰是不好呢.
因為他又真是我尊敬的人.
我覺得又不是一個錯的表達.
只不過是那一刻.
我自己都覺得.
如果我是受眾.

$^{2401}$我自己都覺得很大壓力.
第二個問題就是.
因為我最近從德國回來.
我正在想你剛才說的.
其實有時代性的.
二戰的時候.
說的是你每一天都在想.
你究竟明天會不會還在生.
都未到說要表明自己身份.
即是我明天能否吃東西.
或者明天我家人.
是否在自己旁邊都不知道的時候.
即是可能我表明自己是基督徒.
我已經被人殺了.
其實是否當時.
潘博華去寫的時候.
會不會他都顧及到.
這些考量去跟基督徒說.
其實你是由心底去散髮基督的香氣.
你不需要特別去表明.
但現在香港的社會又未去到.
我一定要隱藏我基督徒的身份.
我覺得會不會二戰的情況.
跟現在香港的情況有少少差距呢.
不是少少,是仍然有差距.
會不會都影響到我們.
如何去詮釋隱藏性的揣摩.
先回答你第二個問題.
這個比較技術性.
德國的情況就不是這樣.
因為德國的情況.
基本上每個人都是基督徒.
Hila都是基督徒.
所以不是非基督徒.
基督徒一個那麼大的問題.
不過他們當時的背景是這樣的.
這本書是寫給一群神學生.
那群就是去送死那群人.
一群地下神學院裡面傳道人的人.
所以都挺強調受苦.

$^{2441}$在困難黑暗裡面去送死.
我小時候看這本書.
覺得死是一個比喻.
我越看越覺得.
這可能不是純屬比喻那麼簡單.
而是在背景裡面.
當基督呼召你去死.
這件事不是純屬象徵性的事.
可能是一些受落生的人都要面對的事.
所以是否說要我們去受苦呢.
還是說在背景裡面.
都有這樣的背景讓他們實在地受苦呢.
我覺得是後者都有很大的機會.
其實跟香港今天不是差太遠.
不是說誰厲害些誰差些.
但肯定不是說他們沒有了基督徒和非基督徒的困難.
大家都面對著某一種信仰是很真實的情況.
我都說兩年前.
我們講《國安法》之後.
好處就是聖經寫的東西越來越真實.
很具體的逼迫或者盼望.
這些全部都不是象徵性的.
而是很真實的東西.
這就會跟聖經相似.
跟德國的時候相似.
所以我覺得時代的某些東西是相似的.
跟那種危難的時候相似.
跟內瑪帝國的基督徒.
德國的二戰基督徒和香港.
香港差一點.
香港沒有那麼危險.
但背景是類似的.
第一個問題是.
(問問德先生).
第一個問題對我來說.
問得很好.
很典型的.
有什麼時間去講.
或者什麼時間去要受眾有適量的回應呢?.
我的做法就是.

$^{2481}$用你剛才講的目者的做法.
是否很好.
或者做了之後令到很緊張.
好像令到參與的人很凶.
一定要做決定.
好像推卸責任.
我覺得都看目者對於探訪過程當中.
他在做什麼.
當然你說你都尊重目者.
我自己的做法就是.
在探訪過程當中.
當然我不會推卸責任.
但我都想瞭解一下.
有什麼問題令到他很緊張.
因為我們很多時候在探訪過程當中.
關心他全人.
除了物質.
我們有些物資供應給他.
其實心靈上有什麼可以聆聽.
在聆聽過程當中瞭解一下.
其實有些地方.
如果我們不再推一推.
要他去想多一點的時候.
他可能都灑了一個就完了.
推一推手就沒有了那件事.
所以我理解目者就是.
有些事可能他曾經發出邀請.
或者他曾經.
不過他每次都是推了去.
或者覺得遲些再說.
或者等待過程.
所以如果你覺得.
目者有點像很推卸責任.
逼他做決定的時候.
我就覺得如果時候還沒有到.
就再放下.
因為我自己都試過.
有叫做推過一下.
有些人反彈.
我通常仍然會保持關係.

$^{2521}$但是我試過有一次.
不要說幾次.
我試過推一推的時候.
再推多一點的時候.
那一下有弟兄姊妹按鍵.
他就問多了回應.
事就這樣成了.
所以我覺得這個有點不容易拿捏.
一下推那一下.
是大力勢力還是時間.
所以如果你剛才說.
那個目者都探了很久的時候.
我覺得某程度上.
目者都覺得.
存感情牌存得差不多.
當然我相信他不是帶著.
一定要探十次之後.
就要他決志的心態.
但是我覺得整件事.
是怎樣可以讓那件事.
是雙方都感受.
有一定程度的信任去做決定.
這個都是要想清楚.
或者拿捏一下大家之間的關係.
因為探訪.
我可能都不是像社工那樣.
要計算探多少次.
這個位置.
所以這個位置有點難拿捏.
但是我自己做法.
都是會讓他明白到.
探訪過程當中給他感受.
其中我自己做得最多就是.
如果我不在.
或者找不到我的時候.
或者有什麼家裡.
晚上有什麼需要的時候.
你都可以鼓勵一下他.
可以祈禱.
或者是告訴他.

$^{2561}$我們為你祈禱.
留下那個網絡.
讓他知道真的有個心.
除了分享物資之外.
都是知道帶著一個關懷.
和一個代禱的心意在當中.
不是要他決志.
這個像是最終的結果.
這樣做法.
其他人有沒有?.
不好意思.
想追問一下.
其實當講到隱藏性和不能隱藏性的時候.
我們會有一些例子.
我們做一些科研工作.
是否真的要打正旗號.
說我們是基督徒的時候.
我會看會不會是.
我們最終的目的.
是一個心甘樂意的信仰.
或者自然流露的信仰.
而不是刻意去否定隱藏性.
或者不能隱藏性.
因為我在想.
人有一個潛力.
想展示自己.
所以耶穌就叫我們要隱藏一點.
但是去到潘博華的句子.
他說如果你隱藏得太多.
你就是逃避神的召命.
其實兩邊都不是最終的目的.
而是一個自然流露的信仰.
我這麼想.
那個結論應該是這樣.
就是基督徒是不需要向人隱藏.
沒記錯吧?.
是向自己隱藏.
所以這麼說.
其實是不需要刻意隱藏.
不告訴別人你是基督徒.

$^{2601}$這個很明顯.
但是同時也不會讓別人知道.
你是基督徒成為首要的目標.
所以兩個很重要的原則.
我們不需要隱藏.
但也不會將那件事成為首要目的.
這兩個原則就會讓我們知道該怎麼做.
所以你問我是否做社官.
要刻意不說耶穌.
我覺得不是.
因為潘博華說不是.
他說向自己隱藏.
不需要隱藏.
但如果目標是為了推銷而做好事.
就不對了.
所以就做好事.
做回你要做的社官.
探訪的時候去關心他.
但不需要否定或隱藏你的身份.
甚至介意地不告訴他.
但就不是成為了首要目標.
但你會跟他說.
所以在這個規則下.
很多人有不同的看法.
我見到的老師.
我們見到有報道團.
我們這些學生有些是.
怎麼說呢?.
(讀答口上).
是報道精.
他們很開心.
在船上帶著傳呼音.
他們是很有心的人.
我不覺得那些人有問題.
他現在成為了牧師.
成為了報道會牧師.
所以這種熱情.
如果是出於自然流露.
我覺得沒問題.
所以你覺得很勉強.

$^{2641}$就不需要這樣做.
我姐妹覺得這樣做.
好像不太和她們性格一樣.
就做回你自己去跟隨耶穌.
和見證耶穌的方法.
但有些人可能是以前的神後.
或者覺得我是這樣做的.
這件事他不是推銷自己.
他不是將目的變成.
讓別人看到他多好.
就做這些事.
所以我覺得不同人有不同的.
好行為的方法.
我不覺得單純地說.
四律報道的福音書.
那些是錯的.
最重要是不要將那件事.
成為了一個純粹.
一個硬銷的目的.
我帶你來自由失魂書.
其實純粹是想告訴你.
這就變成了.
你將那件目的.
不是好行為.
所以不同人有不同的.
跟隨耶穌的方法.
跟隨耶穌的時候.
我每天都帶報道.
和人家說福音的.
我覺得這些.
不會有什麼問題.
最不可以.
兩個avoid.
avoid就是要.
取代跟隨耶穌的目的.
和要隱藏自己.
所以這兩個是.
要避免的.
我們做地區服務.
都是和地方的NGO合作.

$^{2681}$然後他們就轉介.
然後我們就去探訪.
和去分享物資.
我們過程當中.
都是做一個協調.
和一路參與在群體當中.
我們也沒有宣傳我們是教會.
我們有不同的NGO參與的時候.
後來他們問.
為什麼你們經常來.
或者你們是什麼人.
然後他們問我的時候.
我們就說我們是.
我們不是聖雅各福林會.
我們是一間教會.
叫Flow Church.
就是說流塘.
什麼流塘.
什麼來的.
人家問你心中盼望的緣由.
你就常作準備.
你就以溫柔敬畏的心回答.
就是這樣.
就算我們定期都會分享物資.
我們也不會放一張封.
放一張單張.
我們也不會做這些.
分享物資就分享物資.
探訪就是探訪.
所以整件事就是.
你的光照在人前的時候.
人家因為你的好行為.
他就會問.
那你就歸榮耀彼在天上的父.
這也是耶穌去做行徑的時候.
一個正面的反饋.
因為你做的時候.
就是做你本身光的特質.
你做的行為就是.
做你本身炎柚的影響力.

$^{2721}$所以你就做你自己的本相.
就像剛才John說的.
你用什麼形態就是什麼形態.
你用什麼形態就做你自己的形態.
適合你的東西.
所以我認同那個.
有人隨手.
我也認識有牧者.
隨時袋子里都有一些福音工具.
五色珠也好.
五色筆也好.
什麼都好.
但是小弟就什麼都沒有.
那嘴巴算不算.
就是這樣.
形態是很重要.
你自己適合自己的方式.
Hello.
其實也是和剛才大兄提問的問題有關.
有些衰竭.
就是技術上有些分別.
例如我們知道.
其實我們做侍奉最重要就是.
我們侍奉目的.
例如去派飯.
探訪.
都是為了關心有需要的人.
回想起之前有時候教會可能.
在節日會有一些小冊子.
可能寫著耶穌愛你.
或者和探訪者說.
我們教會來探你.
我在想.
如果我們說教會來的.
或者我們基督徒.
當然我們心裡不是說.
是為了侍奉自己.
而是將榮耀交給神的群體.
例如教會.
或者將榮耀交給神.

$^{2761}$這個行徑又算不算是為了自己.
就是看到自己呢.
我想會不會有少少分別呢.
因為我為什麼這麼問呢.
就是說.
很多信或者不信的人.
都不斷地做善事.
特別是很多.
其實這些侍奉.
當然長期的侍奉會.
受侍奉的對象會慢慢認識你.
會問你.
你們是哪個群體的.
我們就自然介紹.
也有很多短期的.
可能幾個月探訪一次.
大家都沒見過大家.
有時就會說.
我們是教會來的.
又或者.
如果心態純粹是.
我現在想將這件.
做善事的背後目的.
我真的想幫助你.
也同時想將.
做善事的.
類似將榮耀.
歸給教會或者神.
如果我們特地這樣介紹.
又可以嗎.
在這個情況.
你問我也會說.
那個先後在哪裡.
做事的人自己心想.
其實很難說.
有些人每個人都不一樣.
同一件事有兩個人做.
但是有兩個人的動機.
其實很少說這麼單一.
單單純粹想.

$^{2801}$榮耀神而不想幫助你.
很少這樣想.
也想幫助你.
但主要是想.
我覺得是次序.
那件事是想純粹一個.
對於sell.
為什麼對於sell這麼抗拒.
所謂sell福音就是因為.
你只是sell福音而已.
後面是沒有任何的愛心.
或者是很想幫助目標.
所以這件事是會轟動的.
所以我想要避免的就是.
把目標不能取代.
跟隨耶穌的行為的時候.
當你自己覺得是這樣想的時候.
這樣就行了.
所以我想和那件事.
很多人的看法是不同的.
剛才打卡那些.
拍照那些.
似乎是想讓人知道.
想sell多一點.
我們大約不需要去到.
抗拒我們完全.
不去提自己是基督徒.
不知道是否能回答你的問題.
Hello.
剛才也聽你的分享.
其中一件強調的事.
也是那種自然的流露.
譬如鹽本身就應該有那種特質.
咸的那種特質.
但是如果譬如.
因為我們.
剛才說到.
關於一些很超自然的事.
我猜可能.
對很多人來說.

$^{2841}$未必是一個.
很立刻就做到的一些事.
假設有個例子.
我可能真的還.
我想問一下.
如果有一些情況下.
我覺得我有些事還沒做到.
但是那些事是好的.
究竟其實我是應該.
根據我的自然狀態.
我還沒做到.
我就應該不要.
純粹是為了希望.
那個人對基督教.
或者基督徒有好印象而去做.
我應該不做.
還是.
不是.
因為有些人會說.
你都要嘗試一下.
踏出你的舒適區.
你不習慣就不做.
也做了.
這個位置.
我想問一下是如何拿捏.
對於那些你覺得.
不完全是你自然流露的事.
是應該你都為了神的緣故.
繼續去做.
還是既然我這個.
我現在這個位置.
我覺得還沒做到.
我就先不做.
是這樣.
怎樣處理這個問題呢.
所以我看FourTruck不是說了.
那篇道叫不要尋常.
就是正正講到.
這個第五章.
所以你先聽回.

$^{2881}$我們永遠都是一個.
逼高自己的位置.
就是說.
你平常你不會做的.
這些好行為.
所以永遠都是逼高自己.
你願意去做.
我想這樣去做.
我都嘗試去做.
在這些超發揮.
超水準策略的情況下.
就做了.
做到了.
所以我們基督徒的行徑.
就是這些位置.
所以你永遠都不會等同於.
在自然所做到出來.
剛才Mufan講到很擊中的地方就是.
當你嘗試去.
只是想嘗試去做的時候.
就忘記了自己能不能做到.
那一刻只是自然.
所以一向都是不自然的事.
但當你忘記了自己.
show off 或者什麼好行為的時候.
那些就很自然地做了出來.
不知你明不明白我的意思.
很複雜的位置.
就是你只是想著.
我如何做得好一點.
為了耶穌.
嘗試去做這些好事出來.
那些是不自然的.
你逼自己做.
甚至是超過自己水平.
做到出來的東西.
但當你專注去做的時候.
就帶來了一些影響.
那些人就見到了.
但你忘記了這件事.

$^{2921}$就會帶來一些影響.
或者會見到很多見證.
很多吸睛的東西.
所以問題就是.
是不自然的事.
等到自然就做不到.
我用回保羅的說話.
就是行善是要學的.
行善是一個很不自然的事.
因為你不用再選擇.
通常都是愛自己.
所以基督徒要學習行善.
是一個經驗學習.
我自己以往在.
九龍城區的幕會的時候.
我就帶著群體去做平等分享行動.
這個社會認識和參與.
總有一些弟兄姐妹.
如果你做過平等分享行動.
這個形式的時候.
總有一些弟兄姐妹就覺得.
我有潔癖的.
我不能下區的.
但我會支持物資給大家分享.
她真的每次都支持到物資.
給我們分享.
她就祝福我們出去順利.
她做回她的部分.
但她等我們回來.
久而久之她就問我一件事.
就是我這樣做.
是不是沒有突破自己的舒適區呢.
我覺得你自己去到.
不是很舒服的話.
你就做到你的部分.
但她說.
那我需要突破一下自己嗎.
我說我的原則就是.
不要驚動她.
讓她情願.

$^{2961}$她的位置就是.
我有一天說.
你出去都不用像我們這樣.
跟別人靠近.
你可以遠遠地看.
感受一下.
你覺得原來要是這樣的距離.
不用這樣摟頭摟脖子.
和你聊天.
你慢慢去感受一下.
她有一次就跟我們出去看看.
她就遠遠地看看我們原來是怎樣的.
因為她自己發覺.
我不可以走得太近.
因為我見她走得太近.
她身體就覺得很痕跡.
因為是她自己的感受.
這是真實的.
她慢慢看多幾次的時候.
就會發覺.
其實我想聽聽你們說什麼.
她越走越近.
就好像剛才John那件事.
不知不覺其實坐在旁邊.
她越做那件事就越忘記那件事.
忘記自己的身痕.
就是這樣.
所以行善是一個經驗學習.
也是違反我們常態的東西.
也是在做的過程中.
你會在迷茫的過程中.
你就不知道.
告訴你這次要做些什麼.
因為你在做那件事的時候.
你就不用告訴自己我在做什麼.
那個理解.
潘鳳婭的意思就是這樣.
所以.
那件事不是你告訴自己要突破舒適區.
是在過程中你慢慢去學習.

$^{3001}$如何可以.
或者是那件事當中.
你如何可以讓自己.
在另一個角度試一下.
不是一定用那個道路去做那件事.
我想問一問這個.
淑靈的驕傲這種情緒.
就是它好像是一種.
我們不良隱藏性或者不平凡生活.
必然會引出來的一種情緒.
那它是怎樣的性質.
和我們應該如何處置.
它是好是壞.
還是一個很正常的一件事.
很正常的失敗.
我記得我自己初信主.
第一次淑靈驕傲是什麼事.
那時候我都說我初信主.
我那時候是很敬虔的.
每天都看聖經.
每天都靈修.
每天都這樣.
第一次覺得自己是挺好的.
那時候我都說.
當你覺得自己淑靈挺好的時候.
就是淑靈低落的開始.
那你就會開始出事.
所以是很正常的失敗.
沒有了.
就是這樣.
沒什麼好說的.
總是會有這種情況.
如何避免呢.
然後你就知道自己不是挺好的.
下次你也不是挺好的.
這些不是永遠的事.
我覺得每個人的靈命靈情.
都是會不同.
開始我就會不想這些問題.
不想會不會有問題.

$^{3041}$反正都不是挺好的.
很少去衡量自己.
不會覺得自己是挺好的.
不是謙卑地說.
而是都擺明知道不是挺好的.
所以就慢慢忘卻了這件事.
當你拆掉第三個鏡頭.
當你做基督徒的時候.
經常有第三個鏡頭.
看回自己好像很謙卑的那種.
或者是.
我這次聖誕時哭了.
經常會有第三個鏡頭看著自己.
所以少一點這種鏡頭.
多一點去不要看自己.
這個也是挺實際的.
我不知道什麼叫肅靈驕傲.
真正的.
因為肅靈驕傲.
我覺得這個詞是很堆砌的.
驕傲就不會是肅靈.
什麼叫肅靈驕傲呢.
因為總會.
我覺得自己讀書厲害.
有些人不用讀書都這麼厲害.
那你就不會驕傲.
可能我身邊太多厲害的人.
所以從來都不會覺得值得驕傲.
但是你說肅靈驕傲.
我就想了想.
我身邊也有些人說.
你讀了多少次聖經.
我說讀了多少次.
整本讀了多少次.
我說一次.
他說一次.
我讀了四次.
我說聖經不是說讀多少次.
是說做了多少.
整件事你就知道他錯了.

$^{3081}$驕傲的人就會錯重點.
你錯重點.
其實我不太懂得回答這個問題.
不過我想說.
你上第三季講.
肅靈驕傲.
其實整件事是錯重點.
他根本不是做要做的事.
你看到我們今天經常都會說.
見證其實要知道焦點是什麼.
見證就是知道你自己.
你做什麼和你所在.
你知道自己是being.
你就懂得做你的doing.
這個過程是需要你認真.
你就會看到焦點.
你的being是一個question.
我之前在Q and A的時候也說.
我們不是claim自己是基督徒.
我們是proclaim自己是基督的門徒.
你做的事讓人感受到你.
認識基督.
這個是重要的.
因為那個是你的being.
以至你的doing在做的那件事.
這個是.
所以驕傲的人都不會肅靈.
那最後一個.
最後一個.
多謝各位.
我們on board.
我自己每次都沒有看自己.
教書的樣子.
但很期待第三季.
我也覺得有第三季.
因為第三季其實就是想盡心一點.
我想是更加基本的東西.
剛才大兄說得很好.
會不會第三季講一下這些題目呢.
講一下獻碑.

$^{3121}$講一下什麼和神關係多一點的東西.
或者是健議.
或者是衛生的東西.
都希望能夠.
應該最多三季都會.
我們都是一個進階.
希望大家能夠從神學到生命.
到信仰反省裡面.
都能夠是一個很有思考的人.
也有言行的人.
我們一起祈禱.
主佑多謝你.
讓我們能夠神學八科裡面.
完成了我們第二季.
當中有很多弟兄姐妹一起去參與.
他們真的每次去實體裡面參與.
或者是在網上重看也好.
我們都知道他們都是有心.
來更加去反省他們的信仰.
讓他們能夠在裡面.
做一個更加理所喜悅的門徒.
我們樓堂都願意成為一間.
更加合理心意的教會.
從牧者到弟兄姐妹.
都是一起來成長.
求主讓我們.
記得你去呼召我們.
你呼召我們成為一個基督徒.
一群見證你的群體.
讓我們一間這樣的教會.
能夠認真單純地跟隨你.
單單地愛你.
單單地在香港這個社會裡面.
在不同的地方裡面.
都能夠真真正正地來見證你.
用我們的生命.
用我們的言行來見證你.
求主你這樣去兼顧我們這間教會.
奉主命求.
拜拜.

$^{3161}$我們可以出來一起拍張照.
可以.
謝謝.
字幕志願者:劉文英.
優優獨播劇場——YoYo Television Series Exclusive.
\newpage



\section{詩篇 90:1-17-20231202}
\label{sec:lfg8MyM5M04}
\textbf{【網上崇拜】大人者,不失其赤子之心者也|詩篇90\_1-17|20231202 [lfg8MyM5M04]}
\newline
\newline
連結: \href{https://youtube.com/watch?v=lfg8MyM5M04}{\texttt{ https://youtube.com/watch?v=lfg8MyM5M04}} ~~~~ 語音日期: 2023-12-02 
\newline
\newline
\hyperref[sec:w1NzLUX2_GE]{\small{< < < PREV SERMON < < <}}
~
\hyperref[sec:index_chronic]{\small{[返順時目]}}
~
\hyperref[sec:index_scriptual]{\small{[返順卷目]}}
~
\hyperref[sec:0oiGMpkgXB8]{\small{> > > NEXT SERMON > > >}}
\newline
\newline
詩篇 90:1-17-20231202
\newline
\begin{longtable}{cl}
\hline
\hline
章節 & 經文 (和合本修訂版)\\
\hline
90:1 & \begin{tabularx}{0.7\textwidth}{X} 主啊,你世世代代作我們的居所。 \end{tabularx} \\ \\ \relax
90:2 & \begin{tabularx}{0.7\textwidth}{X} 諸山未曾生出,地與世界你未曾造成,從亙古到永遠,你是神。 \end{tabularx} \\ \\ \relax
90:3 & \begin{tabularx}{0.7\textwidth}{X} 你使人歸於塵土,說:「世人哪,你們要歸回。」 \end{tabularx} \\ \\ \relax
90:4 & \begin{tabularx}{0.7\textwidth}{X} 在你看來,千年如已過的昨日,又如夜間的一更。 \end{tabularx} \\ \\ \relax
90:5 & \begin{tabularx}{0.7\textwidth}{X} 你叫他們如水沖去,他們如睡一覺。早晨,他們如生長的草; \end{tabularx} \\ \\ \relax
90:6 & \begin{tabularx}{0.7\textwidth}{X} 早晨發芽生長,晚上割下枯乾。 \end{tabularx} \\ \\ \relax
90:7 & \begin{tabularx}{0.7\textwidth}{X} 我們因你的怒氣而消滅,因你的憤怒而驚惶。 \end{tabularx} \\ \\ \relax
90:8 & \begin{tabularx}{0.7\textwidth}{X} 你將我們的罪孽擺在你面前,將我們的隱惡擺在你面光之中。 \end{tabularx} \\ \\ \relax
90:9 & \begin{tabularx}{0.7\textwidth}{X} 我們經過的日子,都在你震怒之下,我們度盡的年歲,好像一聲嘆息。 \end{tabularx} \\ \\ \relax
90:10 & \begin{tabularx}{0.7\textwidth}{X} 我們一生的年日是七十歲,若是強壯可到八十歲;但其中所矜誇的不過是勞苦愁煩,轉眼即逝,我們便如飛而去。 \end{tabularx} \\ \\ \relax
90:11 & \begin{tabularx}{0.7\textwidth}{X} 誰曉得你怒氣的權勢?誰因著敬畏你而曉得你的憤怒呢? \end{tabularx} \\ \\ \relax
90:12 & \begin{tabularx}{0.7\textwidth}{X} 求你指教我們怎樣數算自己的日子,好叫我們得著智慧的心。 \end{tabularx} \\ \\ \relax
90:13 & \begin{tabularx}{0.7\textwidth}{X} 耶和華啊,我們要等到幾時呢?求你轉回,憐憫你的僕人們。 \end{tabularx} \\ \\ \relax
90:14 & \begin{tabularx}{0.7\textwidth}{X} 求你使我們早早飽得你的慈愛,好叫我們一生一世歡呼喜樂。 \end{tabularx} \\ \\ \relax
90:15 & \begin{tabularx}{0.7\textwidth}{X} 求你照著你使我們受苦的日子,和我們遭難的年歲,使我們喜樂。 \end{tabularx} \\ \\ \relax
90:16 & \begin{tabularx}{0.7\textwidth}{X} 願你的作為向你僕人們顯現,願你的榮耀向他們子孫顯明。 \end{tabularx} \\ \\ \relax
90:17 & \begin{tabularx}{0.7\textwidth}{X} 願主-我們神的恩寵歸於我們身上。願你堅立我們手所做的工,我們手所做的工,願你堅立。 \end{tabularx} \\ \\
[1ex]
\hline
\hline
\end{longtable}
$^{1}$丁師妹平安.
大家聽的一首歌曲叫《細佬歌》.
這首歌曲是林海峰先生在2005年出品的一首唱片.
這首唱片的名字叫《三字頭》.
不知道大家有沒有聽過這首唱片.
可能大家有聽過吧.
你們應該都是四字頭吧.
當年我聽這首唱片的時候.
大概也是二字頭吧.
我很喜歡這首唱片.
這首唱片是討論一個人三十多歲的時候.
那個心境和文化.
所以大家當中如果是三十多歲.
很值得聽這首CD.
很精彩的CD.
不知道你什麼時候第一次覺得自己長大了.
你覺得自己長大了沒有.
我二十四歲的時候.
也不覺得自己長大了.
那時候我在讀神學.
在教會裡面實習.
背著一套西裝.
在教會廁所裡面對著鏡子.
看來看去都覺得自己像小朋友背著西裝.
我什麼時候開始覺得自己長大了呢.
當然是當爸爸的時候.
這是一個長大了的記號.
不過我覺得這也不是我最深刻的時刻.
發掘出來的也不是那個時刻.
我發現那個時刻是什麼呢.
就是我開始有老花.
這是我新的眼鏡.
漸進式眼鏡.
這個時刻我覺得自己長大了.
甚至和年老牽上關係.
我不是一個年輕人.
極其量只是一個年輕人.
今天我們會看C篇90篇.
思想成長和年歲的題目.
或者看一看這段C篇90篇的經文.

$^{41}$主啊,你世世代代作我們的居所.
朱山未曾伸出.
地與世界你未曾造成.
從僅古到永遠,你是神.
你使人歸於塵土說.
你們世人要歸回.
在你看來,千年如已過的昨日.
如如夜更的一更.
你叫他們如水沖去.
他們如睡一覺.
早晨他們如生長的草.
早晨發芽生長.
晚上割下枯乾.
我們因你的怒氣而消滅.
因你的憤怒而驚惶.
你將我們的罪孽擺在你面前.
朱山未曾伸出.
地與世界未曾造成.
從僅古到永遠,你是神.
你使人歸於塵土說.
你們世人要歸回.
(字幕組:我忘記了).
不過如果我們去.
(字幕組:我忘記了).
我們人生的年歲是七十歲.
若是強壯可到八十歲.
但其中所驚誇的不過是.
勞苦受煩,轉眼成空.
我們便如煙飛而去.
誰曉得我們怒氣的權勢.
誰安著你該受的敬畏.
隨著你憤怒.
求你指教我們.
(字幕組:我忘記了).
好在我們得著智慧的心.
不過如果你細心閱讀詩篇九十篇的話.
你會發現詩篇九十篇不單單是.
講述生命的短暫,時間的流逝.
人生將要過去.
而是更加講述一個人生命的年歲.

$^{81}$一個人如何渡過他的年歲.
他的歲月究竟是怎樣的流逝,怎樣的過去.
我的意思是,如果你細心閱讀這篇詩篇.
你會發現這篇詩篇不單單是講述.
上帝向千年如同一日.
我們的生命過得很快,我們的生命很短暫.
我們很快便會年老歸去.
更加具體講述的是.
我們的年歲過得很慢,而且過得不太好.
親愛的兄弟姐妹.
如果你這篇詩篇是講述有關年老末世的時候.
這首詩歌不單單是感嘆一生的短暫.
更加是對於一個人年紀漸長的一種成熟的控訴.
甚至乎他講述詩篇九十篇.
其實正是去同步人生的歲月.
對不起,我不是特別想用英文.
找不到一個更加貼切的中文去形容這個同步的字眼.
簡單來說,詩篇九十篇是對成年長大年長的一種控訴.
詩篇九十篇不單單是講述時間過得很快.
人在世上的日子很有限.
就算是讓你活到幾十歲,五十歲也好,七十歲也好.
其實都是很廢的.
不知道當中年長的弟兄姊妹或年輕的弟兄姊妹.
在讀書的時候有沒有一種感覺.
當你面對一些比你年長的長輩的時候.
有沒有一種很敬畏,恐懼的感覺.
我們叫做權威恐懼症.
我以前也有一點權威恐懼症.
當我們面對權威的時候.
你會發現面對一些人的時候.
你會發現這些人突然間覺得很敬畏.
你會覺得他們是一班大人.
你會覺得他們是一個比你年長有智慧的人.
一個很成熟的人.
當我們這樣發現的時候.
不好意思,今天真的老化了.
讓我再檢查好一點.
你會覺得他們是一班大人.
他們很有智慧,我們要尊敬他們.
跟他們差很遠.

$^{121}$但當我自己成為了四十多歲的人的時候.
我發覺原來四十多歲也不是什麼.
也可以是一班很敷衍的人.
年紀大,其實只是年紀大.
不知道你們有沒有聽過.
也是一首舊歌.
陳昇一首的歌.
叫做《像我們這樣的人》.
歌詞正正是說一班年紀大的人.
越是年紀大就越是敗壞的一種無奈.
歌詞裡面寫.
我一天上了多少網.
我一天用了多少個塑膠袋.
我一天走了多少的路.
我一天找了多少的夢.
我一天說了多少個謊言.
我一天撒了多少個諾言.
一千個縮小的痛苦.
也換不了真心的悔悟.
我軟弱去跟好人吐槽.
他更嘔心得要人低頭.
像我們這樣的人.
是不是應該在上個世紀就死去.
詩篇大概就是一篇這樣的詩篇.
詩人去記述年老摩西.
一個八十歲被呼召的老人家.
在礦業裡面四十年.
一百二十歲離不開礦業.
而去離開世界的老人家.
當他生命將要終結的時候.
他發現年紀大其實不代表什麼.
當我們去大概想想.
整篇九十篇裡面的說話.
譬如第五第六節.
經文這樣說.
早晨他們如新長的草.
早晨發芽生長.
晚上閣下敷肝.
經文告訴我們.
人的生就像草一樣的短暫.

$^{161}$不過當我們細味經文的時候.
經文不單單是說新長的草.
生命短暫.
而是說新長的草一直在生長.
早上發芽生長.
晚上閣下敷肝.
早上仍然是小孩子.
中午就變成成年人.
晚上就步入晚年而死去.
經文不斷在描述各式各樣的生長.
不斷在描述歲月的事情.
究竟四篇九十篇如何去理解歲月呢.
我們看看下面的經文.
第七到第九節.
詩人說我們因你的怒氣而消滅.
因你的憤怒而驚惶.
你將我們的罪業擺在你面前.
將我們的忍惡擺在你面光之中.
我們經過的日子都在你怒怒之下.
我們渡盡的年歲好像在星探式.
對詩人來說.
人的生就在上帝的震怒之下.
在上帝的怒氣之中.
人類的罪業和忍惡都擺在上帝的面前.
不過經文所強調的那種罪業不是一般的罪業.
而是先強調是一個人隨著他的年歲之後積累出來的罪業.
第九節這樣寫.
我們經過的日子都在你震怒之下.
我們渡盡的年歲好像在星探式.
詩人說我們一生經過的日子都在上帝的震怒之下.
不是嗎?.
就是年紀越大.
經過的日子越多.
你經過顛覆的罪業.
必然隨著你走過的日子越久.
就會越來越多.
就好像方書裡面.
用石頭打死婦人的情景一樣.
你的年紀越是增長.
你越來越大.

$^{201}$你的罪業就必然會越重.
所以詩經這樣說.
第十節.
我們一生的年日是七十歲.
若是強壯可到八十歲.
但其中所驚誇的不過是勞苦受煩.
轉眼成空我們便如飛如去.
你以為你自己很厲害嗎?.
做了那麼多年人.
你所驚誇的都只不過是勞苦受煩.
大家不要誤會.
我不是不敬老.
事實上我不是在說老人家.
而是我們要認真反省的時候.
我們如何去理解成長這回事情.
《詩歌十篇》正正是一篇.
控訴年紀歲月長大的一篇詩篇.
你覺得自己長大了沒有?.
或者問你覺得怎樣才叫成長?.
不知道你有沒有懷疑過我們這個月的題目.
即是小孩.
小孩是否一件好事?.
上帝不是叫我們成熟一點嗎?.
不是叫我們不要做小孩子嗎?.
做一個人難道我們不成長嗎?.
我們是要做大人還是小孩呢?.
問題關鍵是你如何成長?.
你是一個怎樣的長大?.
怎樣的大人?.
我們去推一把步去思考這個問題.
當然長大是好事.
上帝創造我們.
我們有上帝形象.
理論上我們越是長大成人.
就越來越像上帝.
我們越是成熟就越是做一個更加好的人.
我們就遠離小孩子的階段.
成為一個尚有所喜悅的人.
理論上成長是好事.
成長當然是一件好事.

$^{241}$但我們在人生裡面往往不是成長.
我們只不過是勉勉強強地長大.
經過一些年.
就好像雞,豬一樣.
明明沒有足夠的空間去成長.
像吹菊打口針.
肉是長多了.
肌肉是爆出來的.
外表是大了的.
好像是成熟長大了.
但裡面其實沒有實質的內容.
我稱之為強迫長大.
面對一個問題.
你可以用某個捷徑.
某個方法繞過它.
你長大了.
你成為了一個大人.
但你並沒有真正的去成長.
在這個年頭裡面.
這個社會裡面不斷地逼出一個大人.
就好像差點我不會飛的歌詞.
求時間變慢.
不想逼於成長.
我想這是對於社會.
對於生命一種崇高的奇緣.
你叫一個小朋友長大有很多方法.
但不是每一個都是健康的方法.
我們用很多很容易的方法.
很快速的方法.
很便宜的方法.
但在我們眼中.
這未必是真正成長的方法.
我想近日大家看的電影.
即是《年少日記》.
或者是《百日之夏》.
都讓我們去反思這個課題.
真人一個人長大了.
做事變聰明.
經驗多了.
不代表真正的成長了.

$^{281}$最少你看公司的那些O.C.Food.
什麼叫O.C.Food.
O.C.Food就是一些很有經驗豐富.
很老練的前輩.
Book in a warm way.
他們的豐富經驗在哪裡.
他們的老練在哪裡.
他們的厲害在哪裡.
他們的厲害在走姐面.
射博,politics,賺好處.
這些東西裡面.
所以真正的問題是.
你是怎樣的長大.
你是一個怎樣的大人.
最近我聽一個曾經做保險的弟兄分享.
他說保險裡面有個行家的說法.
就是一年保險三年人.
不知道大家有沒有聽過.
一年保險三年人.
這句說話是什麼意思.
即是當你做了一年保險之後.
你所經驗到的人生百態.
做人處事,言談技巧.
豐富了,厲害了.
你那個人的誠苦深了.
說話圓滑了.
但是這樣是不是真的叫做成長呢.
即是一年保險三年人.
他做了保險兩年.
做了六年人.
最後他就不做了.
看不順,看不過眼.
不是做自己.
天之末日我們做了這麼多年人.
我們作為一個大人.
比起小時候的你.
你想一下你自己有什麼不同了.
說話自圓其說了.
懂得隱藏自己多了.
公關笑容漂亮了.

$^{321}$推銷自己推銷得多了.
吹吹吹得厲害了.
究竟你是圓滑還是虛偽.
自私還是自愛.
懂得疼愛自己.
還是過分放縱自己.
我們作為大人.
我們經常都有一句很厲害的話.
小朋友才有選擇.
問題是不願選擇的你.
究竟是怎樣選擇做一個大人呢.
我知道女兒今年九歲.
基本上她發覺她的成熟程度.
比我們更吃驚.
一年前我才分享過她踩滾輪.
撲噴牙的照片.
今年她是一個少女.
以前她喜歡粉紅色.
粉紫色的冰冰珠片的衣服.
Unicorn那些.
今年她買新衣服已經買黑色的.
完全是兩個不同的品味.
當然隨著年紀的長大.
都看到很多少女的煩惱衝擊.
不滿憤怒陰謀論.
有一次她回來說.
老師說謊騙我們.
原來今天的活動取消了.
她就一群人陰謀論.
老師騙她們.
不偏心.
小女孩那些.
我自己是這樣祈禱的.
求神叫她不要踩得太快.
主要踩得對.
踩得對比踩得快更重要.
寧願保持少許天真單純.
都不要踩得太快.
今天我們的講題叫.
大人者不失其赤子之心者也.

$^{361}$一個可能大家過份文就咒的講題.
大人者不失其赤子之心者也.
一句出自萬子.
即是尼留篇的一句說話.
文字的說話是什麼意思呢.
他說所謂大人.
就是一個仍然沒有丟棄赤子之心的人.
一個某方面仍然像一個小朋友的人.
不過我要下一個注釋.
其實這裡所說的所謂大人.
不是一般的大人.
所謂大人是一些能夠成就大事的人.
一些偉大的人.
即是中文法所說的君子.
用我們基督教的信仰來說.
就是一個上帝所喜悅的人.
並不是每一個大人都是真正的大人.
並不是每一個長大的人都是神所喜悅的人.
詩篇其實正正是教我們怎樣做一個大人.
雖然詩篇不斷對歲月年紀成長控訴.
不過詩篇裡最後一句說話.
教導我們怎樣做一個大人.
最後一句說話12節說.
詩人說求你指教我們怎樣數算自己的日子.
好叫我們得著智慧的心.
求你指教我們怎樣數算自己的日子.
好叫我們得著智慧的心.
怎樣數算自己的日子.
為何詩人要求主去教導他.
數算自己的日子.
怎樣才算數日子.
聽到一個很不爭的事實.
就是我們每天都在長大.
你今天度過了一天 你就長大多一天.
你的生命過多了一天 你就長大多一天.
這個說話比非洲每一分鐘有60秒過去更加真實.
一天過了 你就長大了一天.
所以詩人在冥冥道理當中的道理裡.
他去禱告上帝.
叫他可以好好去審查每一天.

$^{401}$審查他怎樣來長大.
我們要很有智慧.
你要懂得怎樣來長大.
不要隨便長大.
一些未能長大的地方.
不要隨便長大.
寧願保留一些笨拙的地方.
那些很笨拙的地方.
正正可能就是你仍然善良.
正義仍然保留的地方.
我寧願笨拙地善良.
也不想成熟地失去自己.
每個人都有自己笨拙的地方.
但不要緊.
寧願保持這份笨拙.
也不要急速地跨過長大.
我想這正正就是哲學心的意思.
我自己不是一個很會說話的人.
口齒不伶俐.
也不懂得說話.
所以有時候我會突然之間輕聲.
會突然之間的空氣.
因為我說話是我笨拙的地方.
不過對我來說.
沉默寡言也好.
空氣也好.
最少我想的是真誠地說話.
也不想言過其實.
也不想純粹討好別人.
我想這正正就是心.
往往就是這些笨拙的東西.
好好地去數算自己怎樣過日子.
好好地去選擇每一天怎樣長大.
它是在你信仰裡面.
拒絕妥協.
拒絕廉價成長的記號.
在你這些尚未成長的地方裡面.
保持著小孩子的狀態.
也不要輕易做一個廉價的大人.
如果你記得我們每次來到去.

$^{441}$寧聖餐的時候.
我們唱同餅同杯的時候.
第七節.
其實當中寫著赤子之心這個字.
赤子之心眼淚盈眶.
讓我講一下我寫這句歌詞的經歷.
當我寫到這句歌詞的時候.
我就開始想第七節最後的一節.
到底要寫什麼呢.
就是49735203.
你知道填廣東話的詞一定要對音.
所以我們就知道.
這49735203要填什麼好.
後來我就出茅招.
上了填詞網.
你知道有個網可以.
你打數字進去.
可以給你很多可能性.
對音的數字.
我就打了這個字.
49735.
那個網就給了我很多對音的四成語.
遠走高飛.
眾志之的馬可福音無功德心.
放虎歸山脆比叉燒.
今次吸煙.
很多這些字出來.
後來我就看到這個很漂亮的字.
赤子之心.
後來我就再加上.
眼淚盈眶這四個字.
赤子之心眼淚盈眶.
正正就是我們將來.
見天父時候的情景.
頂智培揚我們堅持做一個赤子之心的大人.
雖然是有些笨拙.
雖然有些單純.
雖然好像沒什麼好處.
但是當我們在明日再見到我們的上帝的時候.
我們以最珍貴最終極的身份.

$^{481}$天父的小朋友.
天父的孩子.
去迎接我們的天父.
你眼淚盈眶.
因為你知道.
你一生的堅持都是值得的.
你的成長.
都不是徒然的.
\newpage



\section{路加福音 21:25-28-34-36-20231209}
\label{sec:0oiGMpkgXB8}
\textbf{【網上崇拜】超級耶穌基督,驚奇!|路加福音21\_25-28,34-36|20231209 [0oiGMpkgXB8]}
\newline
\newline
連結: \href{https://youtube.com/watch?v=0oiGMpkgXB8}{\texttt{ https://youtube.com/watch?v=0oiGMpkgXB8}} ~~~~ 語音日期: 2023-12-09 
\newline
\newline
\hyperref[sec:lfg8MyM5M04]{\small{< < < PREV SERMON < < <}}
~
\hyperref[sec:index_chronic]{\small{[返順時目]}}
~
\hyperref[sec:index_scriptual]{\small{[返順卷目]}}
~
\hyperref[sec:sKBDQD8UIMg]{\small{> > > NEXT SERMON > > >}}
\newline
\newline
路加福音 21:25-28-34-36-20231209
\newline
\begin{longtable}{cl}
\hline
\hline
章節 & 經文 (和合本修訂版)\\
\hline
21:25 & \begin{tabularx}{0.7\textwidth}{X} 「日月星辰要顯出預兆,地上的邦國也有困苦,因海中波浪的響聲而惶惶不安。 \end{tabularx} \\ \\ \relax
21:26 & \begin{tabularx}{0.7\textwidth}{X} 人想到那要臨到世界的事,就都嚇得魂不附體,因為天上的萬象都要震動。 \end{tabularx} \\ \\ \relax
21:27 & \begin{tabularx}{0.7\textwidth}{X} 那時,他們要看見人子帶著能力和大榮耀駕雲來臨。 \end{tabularx} \\ \\ \relax
21:28 & \begin{tabularx}{0.7\textwidth}{X} 一有這些事,你們就當挺身昂首,因為你們得救贖的日子近了。」 \end{tabularx} \\ \\ \relax
21:29 & \begin{tabularx}{0.7\textwidth}{X} 耶穌對他們講了一個比喻說:「你們看無花果樹和各樣的樹, \end{tabularx} \\ \\ \relax
21:30 & \begin{tabularx}{0.7\textwidth}{X} 樹葉一長出來,你們看了自然就知道夏天近了。 \end{tabularx} \\ \\ \relax
21:31 & \begin{tabularx}{0.7\textwidth}{X} 同樣,當你們看見這些事發生,就知道神的國近了。 \end{tabularx} \\ \\ \relax
21:32 & \begin{tabularx}{0.7\textwidth}{X} 我實在告訴你們,這世代還沒有過去,一切都要發生。 \end{tabularx} \\ \\ \relax
21:33 & \begin{tabularx}{0.7\textwidth}{X} 天地要廢去,我的話卻絕不廢去。」 \end{tabularx} \\ \\ \relax
21:34 & \begin{tabularx}{0.7\textwidth}{X} 「你們要謹慎,免得被貪食、醉酒和今生的憂慮壓住你們的心,那日子就忽然臨到你們, \end{tabularx} \\ \\ \relax
21:35 & \begin{tabularx}{0.7\textwidth}{X} 如同羅網一樣,因為那日子要臨到所有居住在地面上的人。 \end{tabularx} \\ \\ \relax
21:36 & \begin{tabularx}{0.7\textwidth}{X} 你們要時時警醒,常常祈求,使你們能逃避這一切要來的事,得以站立在人子面前。」 \end{tabularx} \\ \\ \relax
21:37 & \begin{tabularx}{0.7\textwidth}{X} 耶穌每日在聖殿裡教導人,每夜出城到橄欖山住宿。 \end{tabularx} \\ \\ \relax
21:38 & \begin{tabularx}{0.7\textwidth}{X} 眾百姓清早上聖殿,到耶穌那裡聽他講道。 \end{tabularx} \\ \\
[1ex]
\hline
\hline
\end{longtable}
$^{1}$我只想知道.
你到底是什麼意思.
我只想知道.
你到底是什麼意思.
我只想知道.
你到底是什麼意思.
我只想知道.
你到底是什麼意思.
我只想知道.
你到底是什麼意思.
我只想知道.
你到底是什麼意思.
我只想知道.
你到底是什麼意思.
我只想知道.
你到底是什麼意思.
我只想知道.
你到底是什麼意思.
我只想知道.
你到底是什麼意思.
我只想知道.
你到底是什麼意思.
我只想知道.
你到底是什麼意思.
我只想知道.
你到底是什麼意思.
我只想知道.
你到底是什麼意思.
我只想知道.
你到底是什麼意思.
我只想知道.
你到底是什麼意思.
我只想知道.
你到底是什麼意思.
我只想知道.
你到底是什麼意思.
我只想知道.
你到底是什麼意思.
我只想知道.
你到底是什麼意思.

$^{41}$我只想知道.
你到底是什麼意思.
我只想知道.
你到底是什麼意思.
我只想知道.
你到底是什麼意思.
我只想知道.
你到底是什麼意思.
我只想知道.
你到底是什麼意思.
我只想知道.
你到底是什麼意思.
我只想知道.
你到底是什麼意思.
我只想知道.
你到底是什麼意思.
我只想知道.
你到底是什麼意思.
我只想知道.
你到底是什麼意思.
我只想知道.
你到底是什麼意思.
我只想知道.
你到底是什麼意思.
我只想知道.
你到底是什麼意思.
我只想知道.
你到底是什麼意思.
我只想知道.
你到底是什麼意思.
我只想知道.
你到底是什麼意思.
我只想知道.
你到底是什麼意思.
我只想知道.
你到底是什麼意思.
我只想知道.
你到底是什麼意思.
我只想知道.
你到底是什麼意思.

$^{81}$我只想知道.
你到底是什麼意思.
我只想知道.
你到底是什麼意思.
我只想知道.
你到底是什麼意思.
我只想知道.
你到底是什麼意思.
我只想知道.
你到底是什麼意思.
我只想知道.
你到底是什麼意思.
我只想知道.
你到底是什麼意思.
我只想知道.
你到底是什麼意思.
我只想知道.
你到底是什麼意思.
我只想知道.
你到底是什麼意思.
我只想知道.
你到底是什麼意思.
我只想知道.
你到底是什麼意思.
我只想知道.
你到底是什麼意思.
我只想知道.
你到底是什麼意思.
我只想知道.
你到底是什麼意思.
我只想知道.
你到底是什麼意思.
我只想知道.
你到底是什麼意思.
我只想知道.
你到底是什麼意思.
我只想知道.
你到底是什麼意思.
我只想知道.
你到底是什麼意思.

$^{121}$我只想知道.
你到底是什麼意思.
我只想知道.
你到底是什麼意思.
我只想知道.
你到底是什麼意思.
我只想知道.
你到底是什麼意思.
我只想知道.
你到底是什麼意思.
我只想知道.
你到底是什麼意思.
我只想知道.
你到底是什麼意思.
我只想知道.
你到底是什麼意思.
我只想知道.
你到底是什麼意思.
我只想知道.
你到底是什麼意思.
我只想知道.
你到底是什麼意思.
我只想知道.
你到底是什麼意思.
我只想知道.
你到底是什麼意思.
我只想知道.
你到底是什麼意思.
我只想知道.
你到底是什麼意思.
我只想知道.
你到底是什麼意思.
我只想知道.
你到底是什麼意思.
我只想知道.
你到底是什麼意思.
我只想知道.
你到底是什麼意思.
我只想知道.
你到底是什麼意思.

$^{161}$我只想知道.
你到底是什麼意思.
我只想知道.
你到底是什麼意思.
我只想知道.
你到底是什麼意思.
我只想知道.
你到底是什麼意思.
我只想知道.
你到底是什麼意思.
我只想知道.
你到底是什麼意思.
我只想知道.
你到底是什麼意思.
我只想知道.
你到底是什麼意思.
我只想知道.
你到底是什麼意思.
我只想知道.
你到底是什麼意思.
我只想知道.
你到底是什麼意思.
我只想知道.
你到底是什麼意思.
我只想知道.
你到底是什麼意思.
我只想知道.
你到底是什麼意思.
我只想知道.
你到底是什麼意思.
我只想知道.
你到底是什麼意思.
我只想知道.
你到底是什麼意思.
我只想知道.
你到底是什麼意思.
我只想知道.
你到底是什麼意思.
我只想知道.
你到底是什麼意思.

$^{201}$我只想知道.
你到底是什麼意思.
我只想知道.
你到底是什麼意思.
我只想知道.
你到底是什麼意思.
我只想知道.
你到底是什麼意思.
我只想知道.
你到底是什麼意思.
我只想知道.
你到底是什麼意思.
我只想知道.
你到底是什麼意思.
我只想知道.
你到底是什麼意思.
我只想知道.
你到底是什麼意思.
我只想知道.
你到底是什麼意思.
我只想知道.
你到底是什麼意思.
我只想知道.
你到底是什麼意思.
我只想知道.
你到底是什麼意思.
我只想知道.
你到底是什麼意思.
我只想知道.
你到底是什麼意思.
我只想知道.
你到底是什麼意思.
我只想知道.
你到底是什麼意思.
我只想知道.
你到底是什麼意思.
我只想知道.
你到底是什麼意思.
我只想知道.
你到底是什麼意思.

$^{241}$我只想知道.
你到底是什麼意思.
我只想知道.
你到底是什麼意思.
我只想知道.
你到底是什麼意思.
我只想知道.
你到底是什麼意思.
我只想知道.
你到底是什麼意思.
我只想知道.
你到底是什麼意思.
我只想知道.
你到底是什麼意思.
我只想知道.
你到底是什麼意思.
我只想知道.
你到底是什麼意思.
我只想知道.
你到底是什麼意思.
我只想知道.
你到底是什麼意思.
我只想知道.
你到底是什麼意思.
我只想知道.
你到底是什麼意思.
我只想知道.
你到底是什麼意思.
我只想知道.
你到底是什麼意思.
我只想知道.
你到底是什麼意思.
我只想知道.
你到底是什麼意思.
我只想知道.
你到底是什麼意思.
我只想知道.
你到底是什麼意思.
我只想知道.
你到底是什麼意思.

$^{281}$我只想知道.
你到底是什麼意思.
我只想知道.
你到底是什麼意思.
我只想知道.
你到底是什麼意思.
我只想知道.
你到底是什麼意思.
我只想知道.
你到底是什麼意思.
我只想知道.
你到底是什麼意思.
我只想知道.
你到底是什麼意思.
我只想知道.
你到底是什麼意思.
我只想知道.
你到底是什麼意思.
我只想知道.
你到底是什麼意思.
我只想知道.
你到底是什麼意思.
我只想知道.
你到底是什麼意思.
我只想知道.
你到底是什麼意思.
我只想知道.
你到底是什麼意思.
我只想知道.
你到底是什麼意思.
我只想知道.
你到底是什麼意思.
我只想知道.
你到底是什麼意思.
我只想知道.
你到底是什麼意思.
我只想知道.
你到底是什麼意思.
我只想知道.
你到底是什麼意思.

$^{321}$我只想知道.
你到底是什麼意思.
我只想知道.
你到底是什麼意思.
我只想知道.
你到底是什麼意思.
我只想知道.
你到底是什麼意思.
我只想知道.
你到底是什麼意思.
我只想知道.
你到底是什麼意思.
我只想知道.
你到底是什麼意思.
我只想知道.
你到底是什麼意思.
我只想知道.
你到底是什麼意思.
我只想知道.
你到底是什麼意思.
我只想知道.
你到底是什麼意思.
我只想知道.
你到底是什麼意思.
我只想知道.
你到底是什麼意思.
我只想知道.
你到底是什麼意思.
我只想知道.
你到底是什麼意思.
我只想知道.
你到底是什麼意思.
我只想知道.
你到底是什麼意思.
我只想知道.
你到底是什麼意思.
我只想知道.
你到底是什麼意思.
我只想知道.
你到底是什麼意思.

$^{361}$我只想知道.
你到底是什麼意思.
我只想知道.
你到底是什麼意思.
我只想知道.
你到底是什麼意思.
我只想知道.
你到底是什麼意思.
我只想知道.
你到底是什麼意思.
我只想知道.
你到底是什麼意思.
我只想知道.
你到底是什麼意思.
我只想知道.
你到底是什麼意思.
我只想知道.
你到底是什麼意思.
我只想知道.
你到底是什麼意思.
我只想知道.
你到底是什麼意思.
我只想知道.
你到底是什麼意思.
我只想知道.
你到底是什麼意思.
我只想知道.
你到底是什麼意思.
我只想知道.
你到底是什麼意思.
我只想知道.
你到底是什麼意思.
我只想知道.
你到底是什麼意思.
我只想知道.
你到底是什麼意思.
我只想知道.
你到底是什麼意思.
我只想知道.
你到底是什麼意思.

$^{401}$我只想知道.
你到底是什麼意思.
我只想知道.
你到底是什麼意思.
我只想知道.
你到底是什麼意思.
我只想知道.
你到底是什麼意思.
我只想知道.
你到底是什麼意思.
我只想知道.
你到底是什麼意思.
我只想知道.
你到底是什麼意思.
我只想知道.
你到底是什麼意思.
我只想知道.
你到底是什麼意思.
我只想知道.
你到底是什麼意思.
我只想知道.
你到底是什麼意思.
我只想知道.
你到底是什麼意思.
我只想知道.
你到底是什麼意思.
我只想知道.
你到底是什麼意思.
我只想知道.
你到底是什麼意思.
我只想知道.
你到底是什麼意思.
我只想知道.
你到底是什麼意思.
我只想知道.
你到底是什麼意思.
我只想知道.
你到底是什麼意思.
我只想知道.
你到底是什麼意思.

$^{441}$我只想知道.
你到底是什麼意思.
我只想知道.
你到底是什麼意思.
我只想知道.
你到底是什麼意思.
我只想知道.
你到底是什麼意思.
我只想知道.
你到底是什麼意思.
我只想知道.
你到底是什麼意思.
我只想知道.
你到底是什麼意思.
我只想知道.
你到底是什麼意思.
我只想知道.
你到底是什麼意思.
我只想知道.
你到底是什麼意思.
我只想知道.
你到底是什麼意思.
我只想知道.
你到底是什麼意思.
我只想知道.
你到底是什麼意思.
我只想知道.
你到底是什麼意思.
我只想知道.
你到底是什麼意思.
我只想知道.
你到底是什麼意思.
我只想知道.
你到底是什麼意思.
我只想知道.
你到底是什麼意思.
我只想知道.
你到底是什麼意思.
我只想知道.
你到底是什麼意思.

$^{481}$我只想知道.
你到底是什麼意思.
我只想知道.
你到底是什麼意思.
我只想知道.
你到底是什麼意思.
我只想知道.
你到底是什麼意思.
我只想知道.
你到底是什麼意思.
我只想知道.
你到底是什麼意思.
我只想知道.
你到底是什麼意思.
我只想知道.
你到底是什麼意思.
我只想知道.
你到底是什麼意思.
我只想知道.
你到底是什麼意思.
我只想知道.
你到底是什麼意思.
我只想知道.
你到底是什麼意思.
我只想知道.
你到底是什麼意思.
我只想知道.
你到底是什麼意思.
我只想知道.
你到底是什麼意思.
我只想知道.
你到底是什麼意思.
我只想知道.
你到底是什麼意思.
我只想知道.
你到底是什麼意思.
我只想知道.
你到底是什麼意思.
我只想知道.
你到底是什麼意思.

$^{521}$我只想知道.
你到底是什麼意思.
我只想知道.
你到底是什麼意思.
我只想知道.
你到底是什麼意思.
我只想知道.
你到底是什麼意思.
我只想知道.
你到底是什麼意思.
我只想知道.
你到底是什麼意思.
我只想知道.
你到底是什麼意思.
我只想知道.
你到底是什麼意思.
我只想知道.
你到底是什麼意思.
我只想知道.
你到底是什麼意思.
我只想知道.
你到底是什麼意思.
我只想知道.
你到底是什麼意思.
我只想知道.
你到底是什麼意思.
我只想知道.
你到底是什麼意思.
我只想知道.
你到底是什麼意思.
我只想知道.
你到底是什麼意思.
我只想知道.
你到底是什麼意思.
我只想知道.
你到底是什麼意思.
我只想知道.
你到底是什麼意思.
我只想知道.
你到底是什麼意思.

$^{561}$我只想知道.
你到底是什麼意思.
我只想知道.
你到底是什麼意思.
我只想知道.
你到底是什麼意思.
我只想知道.
你到底是什麼意思.
我只想知道.
你到底是什麼意思.
我只想知道.
你到底是什麼意思.
我只想知道.
你到底是什麼意思.
我只想知道.
你到底是什麼意思.
我只想知道.
你到底是什麼意思.
我只想知道.
你到底是什麼意思.
我只想知道.
你到底是什麼意思.
我只想知道.
你到底是什麼意思.
我只想知道.
你到底是什麼意思.
我只想知道.
你到底是什麼意思.
我只想知道.
你到底是什麼意思.
我只想知道.
你到底是什麼意思.
我只想知道.
你到底是什麼意思.
我只想知道.
你到底是什麼意思.
我只想知道.
你到底是什麼意思.
我只想知道.
你到底是什麼意思.

$^{601}$我只想知道.
你到底是什麼意思.
我只想知道.
你到底是什麼意思.
我只想知道.
你到底是什麼意思.
我只想知道.
你到底是什麼意思.
我只想知道.
你到底是什麼意思.
我只想知道.
你到底是什麼意思.
我只想知道.
你到底是什麼意思.
我只想知道.
你到底是什麼意思.
我只想知道.
你到底是什麼意思.
我只想知道.
你到底是什麼意思.
我只想知道.
你到底是什麼意思.
我只想知道.
你到底是什麼意思.
我只想知道.
你到底是什麼意思.
我只想知道.
你到底是什麼意思.
我只想知道.
你到底是什麼意思.
我只想知道.
你到底是什麼意思.
我只想知道.
你到底是什麼意思.
我只想知道.
你到底是什麼意思.
我只想知道.
你到底是什麼意思.
我只想知道.
你到底是什麼意思.

$^{641}$我只想知道.
你到底是什麼意思.
我只想知道.
你到底是什麼意思.
我只想知道.
你到底是什麼意思.
我只想知道.
你到底是什麼意思.
我只想知道.
你到底是什麼意思.
我只想知道.
你到底是什麼意思.
我只想知道.
你到底是什麼意思.
我只想知道.
你到底是什麼意思.
我只想知道.
你到底是什麼意思.
我只想知道.
你到底是什麼意思.
我只想知道.
你到底是什麼意思.
我只想知道.
你到底是什麼意思.
我只想知道.
你到底是什麼意思.
我只想知道.
你到底是什麼意思.
我只想知道.
你到底是什麼意思.
我只想知道.
你到底是什麼意思.
我只想知道.
你到底是什麼意思.
我只想知道.
你到底是什麼意思.
我只想知道.
你到底是什麼意思.
我只想知道.
你到底是什麼意思.

$^{681}$我只想知道.
你到底是什麼意思.
我只想知道.
你到底是什麼意思.
我只想知道.
你到底是什麼意思.
我只想知道.
你到底是什麼意思.
我只想知道.
你到底是什麼意思.
我只想知道.
你到底是什麼意思.
我只想知道.
你到底是什麼意思.
我只想知道.
你到底是什麼意思.
我只想知道.
你到底是什麼意思.
我只想知道.
你到底是什麼意思.
我只想知道.
你到底是什麼意思.
我只想知道.
你到底是什麼意思.
我只想知道.
你到底是什麼意思.
我只想知道.
你到底是什麼意思.
我只想知道.
你到底是什麼意思.
我只想知道.
你到底是什麼意思.
我只想知道.
你到底是什麼意思.
我只想知道.
你到底是什麼意思.
我只想知道.
你到底是什麼意思.
我只想知道.
你到底是什麼意思.

$^{721}$我只想知道.
你到底是什麼意思.
我只想知道.
你到底是什麼意思.
我只想知道.
你到底是什麼意思.
我只想知道.
你到底是什麼意思.
我只想知道.
你到底是什麼意思.
我只想知道.
你到底是什麼意思.
我只想知道.
你到底是什麼意思.
我只想知道.
你到底是什麼意思.
我只想知道.
你到底是什麼意思.
我只想知道.
你到底是什麼意思.
我只想知道.
你到底是什麼意思.
我只想知道.
你到底是什麼意思.
我只想知道.
你到底是什麼意思.
我只想知道.
你到底是什麼意思.
我只想知道.
你到底是什麼意思.
我只想知道.
你到底是什麼意思.
我只想知道.
你到底是什麼意思.
我只想知道.
你到底是什麼意思.
我只想知道.
你到底是什麼意思.
我只想知道.
你到底是什麼意思.

$^{761}$我只想知道.
你到底是什麼意思.
我只想知道.
你到底是什麼意思.
我只想知道.
你到底是什麼意思.
我只想知道.
你到底是什麼意思.
我只想知道.
你到底是什麼意思.
我只想知道.
你到底是什麼意思.
我只想知道.
你到底是什麼意思.
我只想知道.
你到底是什麼意思.
我只想知道.
你到底是什麼意思.
我只想知道.
你到底是什麼意思.
我只想知道.
你到底是什麼意思.
我只想知道.
你到底是什麼意思.
我只想知道.
你到底是什麼意思.
我只想知道.
你到底是什麼意思.
我只想知道.
你到底是什麼意思.
我只想知道.
你到底是什麼意思.
我只想知道.
你到底是什麼意思.
我只想知道.
你到底是什麼意思.
我只想知道.
你到底是什麼意思.
我只想知道.
你到底是什麼意思.

$^{801}$我只想知道.
你到底是什麼意思.
我只想知道.
你到底是什麼意思.
我只想知道.
你到底是什麼意思.
我只想知道.
你到底是什麼意思.
我只想知道.
你到底是什麼意思.
我只想知道.
你到底是什麼意思.
我只想知道.
你到底是什麼意思.
我只想知道.
你到底是什麼意思.
我只想知道.
你到底是什麼意思.
我只想知道.
你到底是什麼意思.
我只想知道.
你到底是什麼意思.
我只想知道.
你到底是什麼意思.
我只想知道.
你到底是什麼意思.
我只想知道.
你到底是什麼意思.
我只想知道.
你到底是什麼意思.
我只想知道.
你到底是什麼意思.
我只想知道.
你到底是什麼意思.
我只想知道.
你到底是什麼意思.
我只想知道.
你到底是什麼意思.
我只想知道.
你到底是什麼意思.

$^{841}$我只想知道.
你到底是什麼意思.
我只想知道.
你到底是什麼意思.
我只想知道.
你到底是什麼意思.
我只想知道.
你到底是什麼意思.
我只想知道.
你到底是什麼意思.
我只想知道.
你到底是什麼意思.
我只想知道.
你到底是什麼意思.
我只想知道.
你到底是什麼意思.
我只想知道.
你到底是什麼意思.
我只想知道.
你到底是什麼意思.
我只想知道.
你到底是什麼意思.
我只想知道.
你到底是什麼意思.
我只想知道.
你到底是什麼意思.
我只想知道.
你到底是什麼意思.
我只想知道.
你到底是什麼意思.
我只想知道.
你到底是什麼意思.
我只想知道.
你到底是什麼意思.
我只想知道.
你到底是什麼意思.
我只想知道.
你到底是什麼意思.
我只想知道.
你到底是什麼意思.

$^{881}$我只想知道.
你到底是什麼意思.
我只想知道.
你到底是什麼意思.
我只想知道.
你到底是什麼意思.
我只想知道.
你到底是什麼意思.
我只想知道.
你到底是什麼意思.
我只想知道.
你到底是什麼意思.
我只想知道.
你到底是什麼意思.
我只想知道.
你到底是什麼意思.
我只想知道.
你到底是什麼意思.
我只想知道.
你到底是什麼意思.
我只想知道.
你到底是什麼意思.
我只想知道.
你到底是什麼意思.
我只想知道.
你到底是什麼意思.
我只想知道.
你到底是什麼意思.
我只想知道.
你到底是什麼意思.
我只想知道.
你到底是什麼意思.
我只想知道.
你到底是什麼意思.
我只想知道.
你到底是什麼意思.
我只想知道.
你到底是什麼意思.
我只想知道.
你到底是什麼意思.

$^{921}$我只想知道.
你到底是什麼意思.
我只想知道.
你到底是什麼意思.
我只想知道.
你到底是什麼意思.
我只想知道.
你到底是什麼意思.
我要富豪得主.
家力前行.
就算遇到風浪.
不需抖震.
就算遇到困惑事.
並有著你的火柱.
困處忍耐.
我要跟你走出.
各個地域.
流淚與不理應.
去一生共贏.
看最廣闊的風景.
盡到你榮耀的國.
看最廣闊.
看最廣闊的風景.
盡到你榮耀的國.
看最廣闊.
看最廣闊的風景.
盡到你榮耀的國.
看最廣闊.
看最廣闊的風景.
盡到你榮耀的國.
看最廣闊.
看最廣闊的風景.
盡到你榮耀的國.
看最廣闊.
看最廣闊的風景.
盡到你榮耀的國.
看最廣闊.
看最廣闊的風景.
盡到你榮耀的國.
看最廣闊.

$^{961}$看最廣闊的風景.
盡到你榮耀的國.
看最廣闊.
看最廣闊的風景.
盡到你榮耀的國.
看最廣闊.
看最廣闊的風景.
盡到你榮耀的國.
看最廣闊.
看最廣闊的風景.
看最廣闊.
看最廣闊的風景.
盡到你榮耀的國.
看最廣闊.
看最廣闊的風景.
盡到你榮耀的國.
看最廣闊.
看最廣闊的風景.
盡到你榮耀的國.
看最廣闊.
看最廣闊的風景.
盡到你榮耀的國.
看最廣闊.
看最廣闊的風景.
盡到你榮耀的國.
看最廣闊.
看最廣闊的風景.
盡到你榮耀的國.
看最廣闊.
看最廣闊的風景.
盡到你榮耀的國.
看最廣闊.
看最廣闊的風景.
盡到你榮耀的國.
看最廣闊.
看最廣闊的風景.
盡到你榮耀的國.
看最廣闊.
看最廣闊的風景.
盡到你榮耀的國.

$^{1001}$看最廣闊.
看最廣闊的風景.
盡到你榮耀的國.
看最廣闊.
看最廣闊的風景.
盡到你榮耀的國.
看最廣闊.
看最廣闊的風景.
盡到你榮耀的國.
看最廣闊.
看最廣闊的風景.
盡到你榮耀的國.
看最廣闊的風景.
盡到你榮耀的國.
看最廣闊的風景.
盡到你榮耀的國.
看最廣闊的風景.
盡到你榮耀的國.
看最廣闊的風景.
盡到你榮耀的國.
看最廣闊的風景.
盡到你榮耀的國.
看最廣闊的風景.
盡到你榮耀的國.
看最廣闊的風景.
盡到你榮耀的國.
看最廣闊的風景.
盡到你榮耀的國.
看最廣闊的風景.
盡到你榮耀的國.
看最廣闊的風景.
盡到你榮耀的國.
看最廣闊的風景.
盡到你榮耀的國.
看最廣闊的風景.
盡到你榮耀的國.
看最廣闊的風景.
盡到你榮耀的國.
看最廣闊的風景.
盡到你榮耀的國.

$^{1041}$看最廣闊的風景.
盡到你榮耀的國.
看最廣闊的風景.
盡到你榮耀的國.
看最廣闊的風景.
盡到你榮耀的國.
看最廣闊的風景.
盡到你榮耀的國.
看最廣闊的風景.
盡到你榮耀的國.
看最廣闊的風景.
盡到你榮耀的國.
看最廣闊的風景.
盡到你榮耀的國.
看最廣闊的風景.
盡到你榮耀的國.
看最廣闊的風景.
盡到你榮耀的國.
看最廣闊的風景.
盡到你榮耀的國.
看最廣闊的風景.
盡到你榮耀的國.
看最廣闊的風景.
盡到你榮耀的國.
看最廣闊的風景.
盡到你榮耀的國.
看最廣闊的風景.
盡到你榮耀的國.
看最廣闊的風景.
盡到你榮耀的國.
看最廣闊的風景.
盡到你榮耀的國.
看最廣闊的風景.
盡到你榮耀的國.
看最廣闊的風景.
盡到你榮耀的國.
看最廣闊的風景.
盡到你榮耀的國.
看最廣闊的風景.
盡到你榮耀的國.

$^{1081}$看最廣闊的風景.
盡到你榮耀的國.
看最廣闊的風景.
盡到你榮耀的國.
看最廣闊的風景.
盡到你榮耀的國.
看最廣闊的風景.
盡到你榮耀的國.
看最廣闊的風景.
盡到你榮耀的國.
看最廣闊的風景.
盡到你榮耀的國.
看最廣闊的風景.
盡到你榮耀的國.
看最廣闊的風景.
盡到你榮耀的國.
看最廣闊的風景.
盡到你榮耀的國.
看最廣闊的風景.
盡到你榮耀的國.
看最廣闊的風景.
盡到你榮耀的國.
看最廣闊的風景.
盡到你榮耀的國.
看最廣闊的風景.
盡到你榮耀的國.
看最廣闊的風景.
盡到你榮耀的國.
看最廣闊的風景.
盡到你榮耀的國.
看最廣闊的風景.
盡到你榮耀的國.
以跳舞代替沮喪.
在絕地綻放希望.
WOW.
你要恨歧視.
會暗著放膽宣告.
在活後亦有指望.
WOW.
You are my miracle.

$^{1121}$WOW.
You are my miracle.
恨歧視的神.
充滿著大能.
世界的光暗.
也在你手.
神能令祂近.
神未必降臨.
斷開鎖鏈解開毒咒.
來看守於我.
濾著渴睡的生命.
看到神永遠不曾於性命.
自食掠奪的本性.
以跳舞代替沮喪.
在絕地綻放希望.
WOW.
你要恨歧視.
會暗著放膽宣告.
在活後亦有指望.
WOW.
You are my miracle.
以跳舞代替沮喪.
在絕地綻放希望.
WOW.
你要恨歧視.
會暗著放膽宣告.
在活後亦有指望.
WOW.
You are my miracle.
來看守於我.
濾著渴睡的生命.
看到神永遠不曾於性命.
自食掠奪的本性.
以跳舞代替沮喪.
在絕地綻放希望.
WOW.
你要恨歧視.
會暗著放膽宣告.
在活後亦有指望.
WOW.

$^{1161}$You are my miracle.
以跳舞代替沮喪.
在絕地綻放希望.
WOW.
你要恨歧視.
會暗著放膽宣告.
在活後亦有指望.
WOW.
You are my miracle.
我一齊宣告.
You are my miracle.
所以呢佢就係我地Miracle.
就係我地Superhero.
我地繼續停留.
墮落係一個勁派當中.
我地去高舉佢ge 榮耀.
我地繼續停留.
墮落係一個勁派當中.
我地繼續停留.
墮落係一個勁派當中.
我地繼續停留.
墮落係一個勁派當中.
我地繼續停留.
墮落係一個勁派當中.
我地繼續停留.
墮落係一個勁派當中.
我地繼續停留.
墮落係一個勁派當中.
我地繼續停留.
墮落係一個勁派當中.
我地繼續停留.
墮落係一個勁派當中.
我地繼續停留.
墮落係一個勁派當中.
我地繼續停留.
墮落係一個勁派當中.
我地繼續停留.
墮落係一個勁派當中.
我地繼續停留.
墮落係一個勁派當中.

$^{1201}$我地繼續停留.
墮落係一個勁派當中.
我地繼續停留.
墮落係一個勁派當中.
我地繼續停留.
墮落係一個勁派當中.
我地繼續停留.
墮落係一個勁派當中.
我地繼續停留.
墮落係一個勁派當中.
我地繼續停留.
墮落係一個勁派當中.
我地繼續停留.
墮落係一個勁派當中.
我地繼續停留.
墮落係一個勁派當中.
我地繼續停留.
墮落係一個勁派當中.
我地繼續停留.
墮落係一個勁派當中.
我地繼續停留.
墮落係一個勁派當中.
我地繼續停留.
墮落係一個勁派當中.
我地繼續停留.
墮落係一個勁派當中.
我地繼續停留.
墮落係一個勁派當中.
我地繼續停留.
墮落係一個勁派當中.
我地繼續停留.
墮落係一個勁派當中.
我地繼續停留.
墮落係一個勁派當中.
我地繼續停留.
墮落係一個勁派當中.
我地繼續停留.
墮落係一個勁派當中.
我地繼續停留.
墮落係一個勁派當中.

$^{1241}$我地繼續停留.
墮落係一個勁派當中.
我地繼續停留.
墮落係一個勁派當中.
我地繼續停留.
墮落係一個勁派當中.
我地繼續停留.
墮落係一個勁派當中.
我地繼續停留.
墮落係一個勁派當中.
我地繼續停留.
墮落係一個勁派當中.
我地繼續停留.
墮落係一個勁派當中.
我地繼續停留.
墮落係一個勁派當中.
我地繼續停留.
墮落係一個勁派當中.
我地繼續停留.
墮落係一個勁派當中.
我地繼續停留.
墮落係一個勁派當中.
我地繼續停留.
墮落係一個勁派當中.
我地繼續停留.
墮落係一個勁派當中.
我地繼續停留.
墮落係一個勁派當中.
我地繼續停留.
墮落係一個勁派當中.
我地繼續停留.
墮落係一個勁派當中.
我地繼續停留.
墮落係一個勁派當中.
我地繼續停留.
墮落係一個勁派當中.
我地繼續停留.
墮落係一個勁派當中.
我地繼續停留.
墮落係一個勁派當中.

$^{1281}$我地繼續停留.
墮落係一個勁派當中.
我地繼續停留.
墮落係一個勁派當中.
我地繼續停留.
墮落係一個勁派當中.
我地繼續停留.
墮落係一個勁派當中.
我地繼續停留.
墮落係一個勁派當中.
我地繼續停留.
墮落係一個勁派當中.
我地繼續停留.
墮落係一個勁派當中.
我地繼續停留.
墮落係一個勁派當中.
我地繼續停留.
墮落係一個勁派當中.
我地繼續停留.
墮落係一個勁派當中.
我地繼續停留.
墮落係一個勁派當中.
我地繼續停留.
墮落係一個勁派當中.
我地繼續停留.
墮落係一個勁派當中.
我地繼續停留.
墮落係一個勁派當中.
我地繼續停留.
墮落係一個勁派當中.
我地繼續停留.
墮落係一個勁派當中.
我地繼續停留.
墮落係一個勁派當中.
我地繼續停留.
墮落係一個勁派當中.
我地繼續停留.
墮落係一個勁派當中.
我地繼續停留.
墮落係一個勁派當中.

$^{1321}$我地繼續停留.
墮落係一個勁派當中.
我地繼續停留.
墮落係一個勁派當中.
我地繼續停留.
墮落係一個勁派當中.
我地繼續停留.
墮落係一個勁派當中.
我地繼續停留.
墮落係一個勁派當中.
我地繼續停留.
墮落係一個勁派當中.
我地繼續停留.
墮落係一個勁派當中.
我地繼續停留.
墮落係一個勁派當中.
我地繼續停留.
墮落係一個勁派當中.
我地繼續停留.
墮落係一個勁派當中.
我地繼續停留.
墮落係一個勁派當中.
我地繼續停留.
墮落係一個勁派當中.
我地繼續停留.
墮落係一個勁派當中.
我地繼續停留.
墮落係一個勁派當中.
我地繼續停留.
墮落係一個勁派當中.
我地繼續停留.
墮落係一個勁派當中.
我地繼續停留.
墮落係一個勁派當中.
我地繼續停留.
墮落係一個勁派當中.
我地繼續停留.
墮落係一個勁派當中.
我地繼續停留.
墮落係一個勁派當中.

$^{1361}$我地繼續停留.
墮落係一個勁派當中.
我地繼續停留.
墮落係一個勁派當中.
我地繼續停留.
墮落係一個勁派當中.
我地繼續停留.
墮落係一個勁派當中.
我地繼續停留.
墮落係一個勁派當中.
我地繼續停留.
墮落係一個勁派當中.
我地繼續停留.
墮落係一個勁派當中.
我地繼續停留.
墮落係一個勁派當中.
我地繼續停留.
墮落係一個勁派當中.
我地繼續停留.
墮落係一個勁派當中.
我地繼續停留.
墮落係一個勁派當中.
我地繼續停留.
墮落係一個勁派當中.
我地繼續停留.
墮落係一個勁派當中.
我地繼續停留.
墮落係一個勁派當中.
我地繼續停留.
墮落係一個勁派當中.
我地繼續停留.
墮落係一個勁派當中.
我地繼續停留.
墮落係一個勁派當中.
我地繼續停留.
墮落係一個勁派當中.
我地繼續停留.
墮落係一個勁派當中.
我地繼續停留.
墮落係一個勁派當中.

$^{1401}$我地繼續停留.
墮落係一個勁派當中.
我地繼續停留.
墮落係一個勁派當中.
我地繼續停留.
墮落係一個勁派當中.
我地繼續停留.
墮落係一個勁派當中.
我地繼續停留.
墮落係一個勁派當中.
我地繼續停留.
墮落係一個勁派當中.
我地繼續停留.
墮落係一個勁派當中.
我地繼續停留.
墮落係一個勁派當中.
我地繼續停留.
墮落係一個勁派當中.
我地繼續停留.
墮落係一個勁派當中.
我地繼續停留.
墮落係一個勁派當中.
我地繼續停留.
墮落係一個勁派當中.
我地繼續停留.
墮落係一個勁派當中.
我地繼續停留.
墮落係一個勁派當中.
我地繼續停留.
墮落係一個勁派當中.
我地繼續停留.
墮落係一個勁派當中.
我地繼續停留.
墮落係一個勁派當中.
我地繼續停留.
墮落係一個勁派當中.
我地繼續停留.
墮落係一個勁派當中.
我地繼續停留.
墮落係一個勁派當中.

$^{1441}$我地繼續停留.
墮落係一個勁派當中.
我地繼續停留.
墮落係一個勁派當中.
我地繼續停留.
墮落係一個勁派當中.
我地繼續停留.
墮落係一個勁派當中.
我地繼續停留.
墮落係一個勁派當中.
我地繼續停留.
墮落係一個勁派當中.
我地繼續停留.
墮落係一個勁派當中.
我地繼續停留.
墮落係一個勁派當中.
我地繼續停留.
墮落係一個勁派當中.
我地繼續停留.
墮落係一個勁派當中.
我地繼續停留.
墮落係一個勁派當中.
我地繼續停留.
墮落係一個勁派當中.
我地繼續停留.
墮落係一個勁派當中.
我地繼續停留.
墮落係一個勁派當中.
我地繼續停留.
墮落係一個勁派當中.
我地繼續停留.
墮落係一個勁派當中.
我地繼續停留.
墮落係一個勁派當中.
我地繼續停留.
墮落係一個勁派當中.
我地繼續停留.
墮落係一個勁派當中.
我地繼續停留.
墮落係一個勁派當中.

$^{1481}$我地繼續停留.
墮落係一個勁派當中.
我地繼續停留.
墮落係一個勁派當中.
我地繼續停留.
墮落係一個勁派當中.
我地繼續停留.
墮落係一個勁派當中.
我地繼續停留.
墮落係一個勁派當中.
我地繼續停留.
墮落係一個勁派當中.
我地繼續停留.
墮落係一個勁派當中.
我地繼續停留.
墮落係一個勁派當中.
被你所賄 榮耀遍地.
兩節我們一起禱告.
親愛的主 親愛的主.
我們知道.
眾言生活當中.
很多事情我們看到的.
八成 九成.
都是一些黑暗的事情.
但我們知道.
你就是榮耀的主.
你就是審判的主.
將來你會降臨.
將來你會帶來勝利.
因此我們的盼望.
只是在於你身上.
因此我們拒絕.
任何的絕望.
拒絕任何的失望和無力.
因為我們知道.
單單我們尋找你.
單單我們跟隨你的時候.
你就是我們生命中.
唯一的盼望.
而你是真實的盼望.

$^{1521}$我們將這一切的眾讚.
這一切的禱告.
獻情給你.
奉主耶穌基督.
得聖名字.
祈求.
阿們.
弟兄姊妹請坐.
請坐.
敬禮.
請坐.
弟兄姊妹平安.
剛才我很嘮叨.
一首慢歌都不能平靜.
很緊張的心情.
因為今天有兩個崇拜.
一個崇拜在小朋友那邊.
見到很多小朋友都在那裡.
應該在看劇.
有個劇目.
來到的時候就見到.
天使得得B.
見到排球少年.
見到網球王子.
見到小飛俠.
見到很多大家都熟悉的東西.
今天的小朋友崇拜.
在外面有攤位.
都有今天限定的.
今天第一天出的貼紙.
基本上參與過程當中.
你就會有一張貼紙.
希望和大家一起去參與.
在崇拜當中一起投入.
參與之後大家可以留下來聊天.
也可以一起去玩.
今天崇拜當中.
想和大家看看張琳琦的訊息.
可以.
出到沒有.

$^{1561}$是不是要那裡.
好.
不好意思 等一下.
還沒出就先走.
先出.
謝謝.
恰恰就打到我腦袋里了.
我還沒把手放開呢.
章林奇通常在聖誕節.
數頭四周的主日開始.
預備等待耶穌基督的降生.
或者紀念耶穌的降生.
同樣提醒我們.
等待耶穌基督的應許.
祂再一次降臨.
章林奇的經文也有很豐富的.
其中你可能會聽到.
關於耶穌基督再降臨之前.
面對的情況.
或者是一些風雷雨電.
我以前在信主的過程當中.
看三代經題.
包括章林奇的經文的時候.
我就覺得整件事很超級英雄.
因為你看到整個世界.
翻天覆地.
變了天.
環境當中很明顯就知道.
一出這些環境就知道.
是打大佬的.
整件事就應該是很準備.
我不知道你帶著什麼期望.
去看那些超級英雄的戲.
或者是有些人不看的.
但是當我帶著兩個兒子.
去看這些超級英雄的戲的時候.
他們當然很期盼.
我最記得一次他們兩個小時候.
就帶著他們去海運那裡.
你知道去海運的電影院.

$^{1601}$有些凳子會動的.
我兩個兒子都很醒目.
他說爸爸為什麼買三張票這麼貴.
比平時貴那麼多.
我說進去你就會知道.
當然他們很小的時候.
我還要教他們.
我說不行.
很震.
教到一.
一就是沒用.
因為他們小的時候.
大人坐的位置不太對.
所以震的位置對他們來說不好.
最後就是.
其實就是沒震那張凳子.
但不要緊.
一會兒會不會很緊張.
不緊張.
一定贏的.
我不知道你有沒有一個小朋友的心境.
你知道打大佬最後一定是贏的.
你的心態會不會是知道邪不能勝正.
小朋友是這樣想的.
但是當我兒子大了的時候.
他看的戲越多的時候.
他突然出一個結論.
其實沒有主角光環.
沒有主角光環是什麼意思.
主角會死的.
直到他去看Iron Man的時候就知道.
於是回頭Iron Man就完事了.
我想張臨其對於你來說.
我希望今天的信息.
和大家一起去想想.
其實你相信耶穌基督再來嗎.
這是一個很真實的.
在問自己.
你頭腦里知道還是情感會接受呢.
這是我們的信仰很靠近的.

$^{1641}$其實在新約時期的教會的弟兄姊妹.
其實都在懷疑這件事.
懷疑這件事就是.
耶穌是否說得出做得到.
而我們現在的情況.
就真的未如理想.
我們有沒有這個期盼呢.
所以今天我選擇的.
是路加福音的經文.
是關於張臨其.
耶穌基督再來之前的現象.
下一章.
我們一起讀路加福音第21章第一段的經文.
請.
好 謝謝.
你見到剛才那兩節的經文當中.
在路加要提這段經文.
其實是第二次提.
第一次提的時候.
路加福音第17章都說過.
關於人子再來的時候的場景是什麼.
這次是第二次提.
其實路加在記述的過程當中.
跟馬可福音的源頭其實很相似.
好像是在耶穌受難之前提醒監牢門徒.
就是說我再回來的時候.
或者是耶穌在這個世界終結的時候.
會出現什麼兆頭.
那幫門徒都問過.
但在這次也一章再提的時候.
很明顯這件事很重要.
而重要的一點就是.
那件事一定會出現.
而出現的重點就是.
你們如何準備這件事.
再提你如何準備這件事.
其實我不知道你過去教會成長.
你自己讀經的過程當中.
你多看重那件事是否能埋身.
我是一個很實務的人.

$^{1681}$當那件事我要做.
或者我要參與過程當中.
我已經在想我應該如何做那件事.
或者我看到這件事的時候.
我就在想下一步應該如何執行.
那件事對我來說.
我要參與就要知道埋身的程度.
我要出多少力.
或者我要預備多少時間.
就正正好像你知道耶穌會回來.
你知道和你預備過程當中.
你如何準備這件事.
在剛才的經文當中.
好像剛才提到.
可以出下一張.
再下一張.
你見到要看見人子有能力.
其實路加想提醒一件事.
就是我們相信的人子是有能力的.
他能力是讓我們明白到.
他會整頓這個世界.
整件事就好像上次.
在《小孩不懂世界》.
阿廷講的那篇信息的時候.
其實一切已經成定局.
是差在時間何時去執行.
而一切已經是會發生的時候.
我們雖然見到眼前的不是如我們所願的.
但結局都會是如技術一樣.
而重點就是這個人子是有能力.
而這個人子是有什麼能力呢.
而我們如何認識這個能力呢.
所以剛才說路加和馬可很相似的時候.
其實馬可是如何讓我們明白到.
人子的能力在哪裡.
我們下一張.
你會見到在馬可福音第四至六章的時候.
耶穌是這樣將神跡慢慢地彰顯出來.
首先見到他在第二三章的時候.
你見到他醫治了大麻瘋的人.

$^{1721}$人們就知道他有能力慢慢去擁擠他.
然後去到第四章的時候.
或者第五章開始.
你見到耶穌就平靜了風浪.
然後由過了胡之後去到落地的時候.
就能夠趕出那個格拉森的鬼.
之後就醫治了一個十二年血流的女人.
然後再繼續第六章.
就是伍秉儀的神跡.
但其中一件事就是耶穌回到鄉下.
他回到鄉下的時候.
就是回到拿撒勒的時候.
他那群村民見到耶穌的時候.
其實他的反應和我們可能會不同.
當一路走來的時候.
見到耶穌平靜風浪.
趕走污鬼.
醫治血流的人.
餵飽了五千人的時候.
但那群一直和他在玩弄的人.
其實他就說.
他不是木匠的兒子.
他的兄弟和妹妹我都認識.
其實他是誰呢.
我認識了你三十年.
你現在出去轉一轉的時候.
回頭就這樣.
其實他們接受不了.
這個人子耶穌的轉變.
他們卡在這裡.
其實為什麼這個人突然間不同了呢.
就好像剛才Alex說.
你看超級英雄的電影的時候.
其實有很多伏筆.
或者在外面的人是看不到的.
但他什麼時候會有些東西突然間轉變呢.
當他有個使命出現的時候.
他的人設就轉變了.
當然我不是說耶穌就是超級英雄.
那個等同.

$^{1761}$但你要見到.
其實這幾個神跡有他的意思.
我們下一章.
你會見到.
平靜風浪是一個掌管自然.
你會見到.
連風和海也聽從他的時候.
那個情況很有趣.
其實那群門徒都不是很熟悉耶穌.
雖然是跟了耶穌.
但是他們的環境就是.
風浪很大.
做漁夫的都很害怕.
於是就去到船尾.
向著枕頭睡覺的耶穌就說.
夫子我們放送命啊.
你還不顧嗎.
耶穌第一件事就是.
不是罵他們.
不是說.
其實最危險就是船尾.
我睡覺最危險的地方都不怕.
你們怕什麼.
我做木匠.
你們打魚的.
你們見慣風浪.
不是說這些.
耶穌什麼都不跟他們爭拗.
他反而去到船頭就說.
向著風說住了吧.
靜了吧.
風就大大的指住了.
轉過頭對他們說.
你們這些小信的人.
你們還沒有信心嗎.
耶穌在那個環境當中.
你見到整件事一說就停了.
那幫門徒最後說.
這是什麼人呢.
連風和海也聽從他.

$^{1801}$其實他有很多奇妙.
很多驚奇.
其實這個是什麼人.
為什麼有這樣的能力.
可以平靜風浪呢.
而且是立刻停了.
然後船就下到一個地方.
叫加拉森.
下船的時候.
假設是一起下船.
一行十三人下了地的時候.
有個人從墳墓走出來.
就喊叫.
跑到耶穌的根前說.
至高神的兒子.
我與你有何乾呢.
你想想整件事是怎樣.
整件事就是.
十三個人一起下去.
那時沒有照片.
什麼都好.
十三個當中.
如果假設剛才台上.
我們敬拜的十三人.
加上我十四個.
當十三個.
一下去的時候.
叫你找個潘志剛出來.
但你從來沒有見過潘志剛.
你也是十三分之一.
你很大機會錯.
但你看到那個人.
從墳墓走出來的時候.
沒有走到其他人面前.
就是走到耶穌面前.
就說至高神的兒子.
因為鬼認得耶穌.
你看到.
那個被鬼附加拉森的人.
困在加拉森一個偏遠的地方.

$^{1841}$連家人都不接濟他的情況下.
其實他是被隔絕的.
他沒有溝通.
但那個人去到耶穌面前的時候.
他就說至高神的兒子.
我與你何乾.
如果你是同行的那十二個門徒.
你看到這個情況的時候.
你在想什麼.
他剛才說什麼.
什麼名字.
至高神的兒子.
剛剛見完平靜風浪.
接著又在加拉森的情況.
接著你會見到.
去到血流的時候.
一個被隔絕的女人.
她因為血流不方便出門的時候.
也是行經期間.
定為不潔.
她不能夠那麼容易與人接觸.
她碰到的東西都成為不潔的時候.
她就自我隔離.
長期做這個自我隔離的時候.
但她本著去得醫治的心態.
你就讓我摸一摸.
就用這個小小的盼望.
去希望摸一摸根治的時候.
事就這樣成了.
耶穌是upset了宗教上.
什麼是聖潔.
也是將一個難處.
在生命當中被捆索.
將它挪開.
所以無論控制了自然.
靈界.
甚至宗教的層面.
耶穌都重新去更新了.
也是讓跟隨的人明白.
他的能力在哪裡.

$^{1881}$最後五餅二魚就很明顯看到.
史無變有.
上帝公認的那種創造.
所以你看到馬可夫音.
其中一個轉接.
在那幾章裡面.
他要讓人明白到.
今天你看到一個人子耶穌.
他是人子耶穌基督.
他有救贖的能力.
更新和拯救的能力.
這個就是希望被人更新.
但和他玩了三十年.
那群村民或舊鄉里.
其實我全家人都認識他.
他不能夠卡住.
不能轉.
他不能夠透過那件事情.
去看到耶穌的另外身份.
就是耶穌基督真正出現的身份.
這個同樣是靠近.
不知道你信主多久.
我信主超過三十年.
但過程中不是論資排輩.
不是坐得久教會.
信仰從來都是靠那種生命的經歷.
你認真去反復去想.
其實那件事是不是這樣.
不是說自己理性多厲害.
但過程中你都想想.
同一件事為什麼有不同的反應.
你看到那群門徒.
其實不斷地問.
其實是什麼事呢.
不斷地問.
其實我們跟隨的是誰呢.
所以你看到看馬克思的後面.
他在山上.
耶穌變相的時候.
他們就說竹譚.

$^{1921}$但他更加聽到一個更重要的信息.
就是這是我的愛子.
你們要聽他.
其實信仰從來都不是看見多少神跡.
信仰從來就是在見過當中.
見了當中.
你有沒有轉念.
有沒有轉化.
所以上教會不是說你上多少次崇拜.
是說崇拜當中你如何去開放去經歷.
去轉念.
去看到上帝的說話.
如何去更新你或者提醒你.
或者另外就是如何去迎受.
那個說話在你生命當中.
成為一個什麼行事的准則.
所以馬可就是看到.
由人子耶穌和人子耶穌基督的重要性.
所以用迴路加的時候.
就是說這個人子就是有能力.
在眾人面前彰顯出來.
他一定會再回來.
而他回來的時候.
就是如他所說的.
從開初他會更新了各種能力.
他就回來了.
所以我們常常都說.
特別是我自己很喜歡說.
就是在散會結束的時候.
一個星期的崇拜就結束了.
策顯崇拜的時候.
我仍然是等待著我主榮耀中再來.
因為不知道那個星期.
上帝是否會再回來.
但他回來的時候.
一定是帶著人子的榮耀回到當中.
但我們會怎麼做呢.
這個是不容易的.
在落義生powerpoint之前.
有一次我和一群年輕人.

$^{1961}$中一,二,三的年輕人上主教學.
事源是因為我那時候住堂的時候.
有位同工主教老師病了.
那天我八點半知道我要去代課.
他告訴我代那群人.
我說那群是什麼人.
他說了那些學生的時候.
我就說要認真一點.
為什麼要認真一點呢.
因為那裡有同主任的兒子.
有執事會副主席的女兒.
有主學校長的兒子.
全部都是從孩童就回教會.
說得清楚.
我不是忌惟他父母是什麼人.
正正就是我知道我要教什麼.
我要教五餅二魚的神跡.
我說OMG.
他們家裡已經聽過很多次.
我還要教五餅二魚.
因為課程是講道的.
課程是定格的.
不可以改的.
於是我就考功力.
於是我就去.
進去看到那六個年輕人的時候.
我就說今天我來代課的.
很明顯看得出.
平時都不是你教的.
第一句話就是這樣.
所以我今天也是跟著課程教.
教完.
不過他們都沒有看.
他說一個小時的主要學期間.
我給你45分鐘講.
給你15分鐘我打機行不行.
哈哈.
我就說.
看你能不能用15分鐘.
我說我開始講了.

$^{2001}$他說今天講五餅二魚.
立刻就反我眼.
我每人就派一張紙給他.
派一張紙給他的時候.
那怎麼辦呢.
重點是什麼.
就是有四格的.
我就說大家都很熟悉五餅二魚.
那你就將你熟悉的五餅二魚.
用四格漫畫去反映出來.
跟著他就說.
不做行不行.
你又說給我45分鐘.
哈哈.
於是就畫畫畫畫.
跟著畫的時候.
有些畫得真的很漂亮.
畫得很漂亮.
跟著第一格就有很多人.
跟著第二格有個小朋友.
拿了五個餅兩個魚.
跟著第三格不講了.
第四格有12個男子.
那很清楚這個modal answer.
是這樣評分這四格.
追包的是第三格.
第三格是畫什麼.
猜猜.
肚子餓.
第三格他畫了一個耶穌.
跟著前面有個盤.
跟著就這樣.
哈哈.
跟著就在boom.
boom.
跟著我說這格解釋一下.
他說畫得這麼好都要解釋.
我說我只知道boom是什麼.
很明顯就是耶穌變.
史無變有.

$^{2041}$變了boom.
我美式一點就有個boom.
跟著我就說.
你覺得耶穌是什麼.
耶穌很明顯就是一個魔術師.
我說耶穌是一個魔術師.
那魔術師可不可以是耶穌.
他們開始不出聲.
如果耶穌是史無變有.
是一個magic的話.
是否會用magic的就是耶穌呢.
他們就停在這.
你會發覺他那段經文很熟.
但我想象他明白到.
不是每件事反轉了都是一樣的.
就好像我的年代.
有叔就是你老爸.
那些想法.
這些邏輯很基本.
所以我跟初中那班.
回教會很久的弟兄姊妹.
年輕人說一句.
你習以為常的東西.
你的生命沒有轉念是很重要的.
仍然是那句.
當我們知道耶穌再回來的時候.
你知道自己要預備.
但我問你有沒有預備的時候.
除了是一個practical要預備之外.
你的心智上沒有預備.
所以在這段經文的時候.
就講到一個重點.
就是.
是不是我按得到.
就是你要謹慎.
恐怕因貪食醉酒.
今生的思慮.
你們的時候.
你要站立.
得穩這個很重要.

$^{2081}$其實你知不知道自己如何站得定.
或者覺得到.
好 我們.
是不是我按得到.
是不是.
所以我就用.
其實剛才講到.
貼守羅來加教會都遇到一個問題.
貼守羅來加教會保羅離開的時候.
保羅又寫封信給他們.
其實第一件事是贊他們.
因為他們見證很好.
很多人因為他們見證信心的時候.
就稱贊他們.
他們很開心.
但其實貼守羅來加前書第四章的時候.
就有些弟兄姐妹所愛的人死了.
他們覺得上帝說回來.
但回來幾十年都沒有.
其實是不是真的回來.
又問到一件事.
如果死了的話.
他死了的狀態會是怎樣.
所以保羅寫信就讓他們明白到.
上帝如何去看這件事.
或者上帝曾經已經告訴我們.
那件事是怎樣會發生和經歷.
在第五章的時候.
保羅都提醒一次.
就是那個日期是怎樣.
沒有人知道.
環境是怎樣會變化.
但是上帝讓我們明白到.
他預定我們不是要受刑.
預定我們直住耶穌基督得救.
他替我們死.
叫我們無論醒著睡著都與他同活.
再一次讓受輸的人明白到.
我們在什麼狀況都好.
上帝都清楚.

$^{2121}$雖然你現在生命已經暫停了.
但是你只是在上帝當中睡著.
抹後的時候他會再一次回來.
讓我們一起去和他同享受天國.
這個這一刻是證實不了的.
保羅說這是我們深信.
因為耶穌從來都是說得出做得到.
所以你們要彼此勸慰.
互相建立.
正如你們素尚所行的.
這個字素尚所行.
我覺得可圈可點.
什麼是素尚所行.
你今天素尚所行是什麼.
你還記得早前Obi目者說.
素尚對於彈爾利是什麼回事.
就是仍然在他難處當中.
做他敬拜的生活.
做他每日如常去守節.
或者是去默想上帝的生活.
對你來說什麼是素尚所行.
今天你素尚所行是什麼.
這個現在這麼問你.
不會這麼回答我.
但是你問一問自己.
平時你信仰的素尚所行.
和你生活的素尚所行有多接近.
你每日佔據的時間有多少.
你每日用的時間用在哪裡.
多些你自己在遇到困難的時候.
你找誰.
剛剛星期四小組的時候.
我和頂姐妹說.
當你有困難的時候.
我們找誰.
那個人在當中.
我可以問你.
我經常這麼問.
當你有困難的時候.
你第一個打電話會打給誰.

$^{2161}$你的腦子里可能有個人.
如果那個人彈了出來的時候.
你想掃苦.
你打給那個人.
那個人是不是在你小組里.
如果他不是你小組的人.
他是不是在你教會里.
如果他不是你教會的人.
他是不是基督徒.
如果你想有問題.
想掃苦.
想找人談的時候.
那個人又不是你小組的人.
又不是你教會的人.
又不是基督徒.
我仍然是問.
其實和有共同信仰.
能不能夠彼此勸惟.
彼此建立.
這個是很真實的.
我當然不是說.
我的朋友沒有信主的.
支持不了我.
但是我仍然覺得.
我們是有共同信仰.
有共同經歷.
在教會有不同分享的弟兄姊妹.
那種建立是很真實的.
這個也是.
這個純純都是你花多少時間.
花多少時間去分出去.
你和未信主的朋友一起.
或者和已信主的朋友一起.
我不是叫你break even.
計數不是那些.
你總會有很多.
大家一起去建立關係.
是真的.
這個就是很真實.
如何成為一個互為肢體.

$^{2201}$互為support.
是一個很重要的.
保羅仍然提醒.
貼上來的教會.
就是你們繼續彼此勸惟.
彼此建立.
就好像你們面對困難的時候.
面對失落的時候.
其實有很多見證人.
在你身邊當中出現.
他們如何經歷信仰.
其實是彼此分享.
彼此承擔.
就好像你平時那樣做.
聽我講道或者信息的時候.
你都會聽過.
其實我很少講成功建政的牧者.
因為我覺得失敗建政才是最受用的.
知道獲了就叫人不要跟著做.
你都聽過一句話.
成功沒有方程式.
失敗一定有原因.
重點就是成功是你無法複述的.
因為他的人設是可以的.
那就可以做到.
但你不行.
照板轉換都不行.
但失敗一定是有原因.
為什麼失敗的時候.
就告訴大家.
有些事是不需要做.
或者不要做.
可能他的失敗對你不是失敗.
但起碼過程當中.
人家給了你一個參考.
這個就是在當中一起去參長.
我說我們當中不一定每個人都行.
但我們一起談你不行的地方.
就是讓我們面對在等待耶穌之間的困難.
這個素尚所行是很重要的.

$^{2241}$Falsehood有很多東西是不行的.
Falsehood有很多東西是試下做下.
數下錯下.
錯下當中又試下.
一定是try an error.
走到今年第五年.
我們素尚所行就是keep trying.
keep doing.
這個就是我們所做的事.
有些什麼是重要.
弟兄姐妹之間一起投入一個信仰群體.
那個是最重要.
這個就是教會存在.
在地上一定要做到的地方.
所以我們不是追數字.
但我們緊張的就是.
你每個星期有沒有崇拜.
你來不了現場的話.
你要看回.
每個星期都一定要崇拜.
崇拜就是你每個星期預留時間去敬拜上帝.
這個是你素尚所行的事情.
我經常說求害求分數.
分數是反映你沒有做.
你得10分就因為.
你100分你得10分就因為你沒有運輸.
所以你做了那麼多就那麼多.
對我們來說素尚所行是什麼呢.
貼瘦羅爾加教會就是.
他一直勸勉一起咬實牙筋.
等耶穌基督的回來.
經文在張臨其過程當中.
我希望我們都是等耶穌基督的第二次降臨.
等的過程當中我們真的不知道.
但我仍然和頂尖妹說.
我們不斷地倒數耶穌的回來.
就像神學百科的時候.
John也說耶穌基督在遠方跑來.
我們在跑去迎接祂.
我們等相遇的那一刻.

$^{2281}$這個就是我們繼續做.
我們做回我們平時的練習.
這個就是我小時候很喜歡看的卡通片.
對很多年輕人來說.
可能沒有看過.
你看的可能是Dragon's Set.
但我仍然喜歡小時候的龍珠.
吳空由龜仙人開始長大.
到他自己打贏了龜仙人.
他把自己的衣服從龜變成了悟字之外.
我就很喜歡.
吳空其實整個班底就是.
很多team來的,很多隊員.
大家一起參與,一起很開心.
厲害和不厲害都好.
都可以一起去玩.
很著重那種群體關係.
對我來說,我真的很喜歡這個卡通.
一個很重要的人設.
但整件事你會看到他不斷地成長.
你會看到由吳空跟龜仙人.
到自己贏了武林大會.
然後開始慢慢長大.
開始變超西.
開始有不同的變化.
然後你會發現.
由一個變一次身變兩次身.
就會發現一直在變身.
但你會看到那套路又不是一直在變身.
其實你會,你可能不會,我會.
我會看到每次變身的時候.
都有些誘因.
令他在傷痛當中經歷成長.
因為我自己很喜歡吳空一個角色.
他的路是否很順利都不差.
但他一定有傷痛.
每次傷痛當中.
他就在交友公司帶著口罩.
然後泡著水.
然後好起來的時候就很厲害.

$^{2321}$每一次好起來之後.
他整個人就厲害了很多.
如果你記得他第一次變超級賽亞人的時候.
是什麼原因.
哇,立刻感覺到.
回來了.
我雖然不知道是誰.
就是他從小玩到大.
一起參加武林大會的無限.
被菲利殺死了.
然後.
然後他就.
然後看到他在轉身.
轉身的頭髮的時候.
你會發現人生會經歷傷痛.
人生會經歷分離.
人生會經歷失去的時候.
那種生命的突破.
那種毀滅的轉變.
是很重要的.
整件事對於那個做法.
是令到他成長一個很重要的突破的位置.
今天我們沒有傷痛嗎.
今天我們沒有分離嗎.
今天有沒有聽判刑.
我們在那個難處當中.
會不會經歷那個突破位呢.
很多年之後才可以再出來的時候.
你能不能感受到那種苟潔.
但是不是現在可以立刻出來.
立刻可以突破.
不是的.
很多戰鬥會繼續打的.
但是悟空的生命力讓我們明白到.
其實很多東西不是即時.
但是我們在過程當中養傷.
生命在不斷成長的時候.
會給我們很多韌力.
又給我們突破.
第一次變超西.

$^{2361}$第二次變超西.
第三次變超西.
然後不斷越強越強.
你可能說這些只是沒有辦法的卡通片.
哪有那麼神的.
但是它讓我們明白到.
其實和我們一起看這些的人很多.
在我預備講章的時候.
我就找人幫忙.
我就問人借了一件衣服.
你見到穿悟空的衣服.
就是我的好幫手.
很漂亮的Jazz.
在他讓我選擇那麼多件衣服當中.
很多件.
選擇那麼多件當中.
我就是選擇這件.
因為這件是小時候的悟空和無限.
就是這兩個.
我很喜歡這個款式.
在過程當中讓我感受到.
這個就是最原創的那種小時候的心態.
悟空和無限.
其實無限一定是不懂打架的.
但是你會發覺.
他說一起玩一起玩.
其實不是比高低.
就是那種赤子之心.
就是那種同行的感覺.
很重要.
在比拼當中一起去參與.
是很重要.
認識我的人都知道.
我很喜歡看天.
下一張.
我每次看天的時候.
我就很感受到.
就是我自己覺得有空間.
有好好.
在我看悟空很多場比賽的時候.

$^{2401}$你會看到很多風輪.
打得起起含含的.
但是通常那些都會輸.
但是耶穌.
我看到一出一個很漂亮的天空的時候.
我知道耶穌一定會贏.
不是耶穌.
悟空一定會贏.
就是悟空的時候已經打到沒有了.
當他龜波氣功都沒有用的時候.
他就會向著天.
舉高兩只手.
是怎樣的.
是的.
那些人就會了.
好了.
頂姐妹可不可以借手給我.
哇.
是哦.
我們一起來.
我們試一下.
頂姐妹.
Float出來頂姐妹.
可不可以舉起手.
分點力量給我.
不要下來.
別收.
元氣彈是一定要慢慢地收集的.
你明白嗎.
我們一個人是不行的.
你會看到.
是不斷地將我們的氣集在一起的時候.
然後悟空收集好氣的時候.
然後就將元氣彈扎下去.
好.
看到了.
手可以放下.
你會看到的.
我看悟空的時候.
我很感受到那種集氣.

$^{2441}$當他發覺自己所有的不身的絕學都打不贏那個壞人的時候.
他會發現一件事.
我一個人贏不了.
全世界都會一起幫我們.
這個對我來說.
那時候我已經回教會了.
那時候我很熟悉.
那套卡通片是12點C看的.
但我就感受到.
其實那件事很教會.
如果你覺得你有一萬塊.
如果你覺得你自己有一萬塊.
教會是要一萬個一塊.
而不是一萬個一萬塊.
你覺得你很厲害.
打完所有人是一萬塊.
一萬個power.
但其實每一個power是集在一起.
所以元氣彈是我最喜歡的東西.
而且元氣彈是沒法輸的.
最後就一定贏.
接著就天浪氣清.
我希望弟子妹.
今天大家都舉了手出來.
謝謝你.
在過程當中.
我希望教會是每個人都拿一點點出來.
不知不覺就一起去做點事.
如素尚所行一樣.
我們認識的耶穌基督.
是真真實實的人子耶穌基督.
祂會再回來.
當我們望天的時候.
你看看.
你會見到超級的耶穌基督.
那種驚奇.
那種期盼.
祂會在當中.
是真的.
我自己很不開心.

$^{2481}$或者很困難的時候.
我經常都嘆天望日.
我經常都和自己說.
或者有一天.
我會看著耶穌下來.
不知道.
但我信我主.
榮耀中在立.
你信嗎.
我願我主在來.
I am ready.
你ready嗎.
我第一次祈禱.
我們需要集氣.
集氣不是.
現實就有一個人.
我們要立刻打敗.
集氣是因為我們知道.
我們不孤單.
我們有一群同行者.
我們彼此結連.
彼此勸慰.
我們失敗的經驗.
可以彼此去支持.
讓我們可以檢討.
可以同行.
我們同行的經驗.
可以成為別人的幫助.
讓他明白到.
他不孤單.
我們需要集氣.
這是上帝讓我們看到.
也是上帝給我們的能力.
求主讓我們有更多焦聚的能力.
讓弟子們一起參與.
我們每次去敬拜禰的時候.
感受到上帝的同在.
我們也在當中望天.
我們也ready去等待禰.
再一次榮耀中在立.

$^{2521}$求主禰繼續對我們說話.
祈禱奉耶穌的名求.
\newpage



\section{馬太福音 8:1-4-20231216}
\label{sec:sKBDQD8UIMg}
\textbf{【網上聖餐崇拜】年少多好|馬太福音8\_1-4|20231216 [sKBDQD8UIMg]}
\newline
\newline
連結: \href{https://youtube.com/watch?v=sKBDQD8UIMg}{\texttt{ https://youtube.com/watch?v=sKBDQD8UIMg}} ~~~~ 語音日期: 2023-12-16 
\newline
\newline
\hyperref[sec:0oiGMpkgXB8]{\small{< < < PREV SERMON < < <}}
~
\hyperref[sec:index_chronic]{\small{[返順時目]}}
~
\hyperref[sec:index_scriptual]{\small{[返順卷目]}}
~
\hyperref[sec:dT3dN2jF8BQ]{\small{> > > NEXT SERMON > > >}}
\newline
\newline
馬太福音 8:1-4-20231216
\newline
\begin{longtable}{cl}
\hline
\hline
章節 & 經文 (和合本修訂版)\\
\hline
8:1 & \begin{tabularx}{0.7\textwidth}{X} 耶穌下了山,有一大群人跟著他。 \end{tabularx} \\ \\ \relax
8:2 & \begin{tabularx}{0.7\textwidth}{X} 這時,一個痲瘋病人前來拜他,說:「主啊,你若肯,你能使我潔淨。」 \end{tabularx} \\ \\ \relax
8:3 & \begin{tabularx}{0.7\textwidth}{X} 耶穌伸手摸他,說:「我肯,你潔淨了吧!」他的痲瘋病立刻就潔淨了。 \end{tabularx} \\ \\ \relax
8:4 & \begin{tabularx}{0.7\textwidth}{X} 耶穌對他說:「你要注意,不可告訴任何人,只要去,讓祭司為你檢查,並獻上摩西所吩咐的祭物,作為證據給眾人看。」 \end{tabularx} \\ \\ \relax
8:5 & \begin{tabularx}{0.7\textwidth}{X} 耶穌進了迦百農,有一個百夫長進前來,求他, \end{tabularx} \\ \\ \relax
8:6 & \begin{tabularx}{0.7\textwidth}{X} 說:「主啊,我的僮僕癱瘓了,躺在家裡,非常痛苦。」 \end{tabularx} \\ \\ \relax
8:7 & \begin{tabularx}{0.7\textwidth}{X} 耶穌說:「我去醫治他。」 \end{tabularx} \\ \\ \relax
8:8 & \begin{tabularx}{0.7\textwidth}{X} 百夫長回答:「主啊,你到舍下來,我不敢當;只要你說一句話,我的僮僕就會痊癒。 \end{tabularx} \\ \\ \relax
8:9 & \begin{tabularx}{0.7\textwidth}{X} 因為我在人的權下,也有兵在我以下。我對這個說:『去!』他就去;對那個說:『來!』他就來;對我的僕人說:『做這事!』他就去做。」 \end{tabularx} \\ \\ \relax
8:10 & \begin{tabularx}{0.7\textwidth}{X} 耶穌聽了就很驚訝,對跟從的人說:「我實在告訴你們,這麼大的信心,就是在以色列,我也沒有見過。 \end{tabularx} \\ \\ \relax
8:11 & \begin{tabularx}{0.7\textwidth}{X} 我又告訴你們,從東從西,將有許多人來,在天國裡與亞伯拉罕、以撒、雅各一同坐席; \end{tabularx} \\ \\ \relax
8:12 & \begin{tabularx}{0.7\textwidth}{X} 本國的子民反而被趕到外邊黑暗裡去,在那裡要哀哭切齒了。」 \end{tabularx} \\ \\ \relax
8:13 & \begin{tabularx}{0.7\textwidth}{X} 耶穌對百夫長說:「你回去吧!照你的信心成全你了。」就在那時,他的僮僕好了。 \end{tabularx} \\ \\ \relax
8:14 & \begin{tabularx}{0.7\textwidth}{X} 耶穌到了彼得家裡,見彼得的岳母正發燒躺著。 \end{tabularx} \\ \\ \relax
8:15 & \begin{tabularx}{0.7\textwidth}{X} 耶穌一摸她的手,燒就退了,於是她起來服事耶穌。 \end{tabularx} \\ \\ \relax
8:16 & \begin{tabularx}{0.7\textwidth}{X} 傍晚的時候,有人帶著許多被鬼附的來到耶穌跟前,他只用一句話就把邪靈都趕出去,並且治好了一切有病的人。 \end{tabularx} \\ \\ \relax
8:17 & \begin{tabularx}{0.7\textwidth}{X} 這是要應驗以賽亞先知所說的話:「他代替了我們的軟弱,擔當了我們的疾病。」 \end{tabularx} \\ \\ \relax
8:18 & \begin{tabularx}{0.7\textwidth}{X} 耶穌見許多人圍著他,就吩咐渡到對岸去。 \end{tabularx} \\ \\ \relax
8:19 & \begin{tabularx}{0.7\textwidth}{X} 有一個文士進前來對他說:「老師,你無論往哪裡去,我都要跟從你。」 \end{tabularx} \\ \\ \relax
8:20 & \begin{tabularx}{0.7\textwidth}{X} 耶穌說:「狐狸有洞,天空的飛鳥有窩,人子卻沒有枕頭的地方。」 \end{tabularx} \\ \\ \relax
8:21 & \begin{tabularx}{0.7\textwidth}{X} 又有一個門徒對耶穌說:「主啊,容許我先回去埋葬我的父親。」 \end{tabularx} \\ \\ \relax
8:22 & \begin{tabularx}{0.7\textwidth}{X} 耶穌說:「讓死人埋葬他們的死人。你跟從我吧!」 \end{tabularx} \\ \\ \relax
8:23 & \begin{tabularx}{0.7\textwidth}{X} 耶穌上了船,門徒跟著他。 \end{tabularx} \\ \\ \relax
8:24 & \begin{tabularx}{0.7\textwidth}{X} 海裡忽然起了猛烈的風暴,以致船幾乎被波浪淹沒,耶穌卻睡著了。 \end{tabularx} \\ \\ \relax
8:25 & \begin{tabularx}{0.7\textwidth}{X} 門徒去叫醒他,說:「主啊,救命啊,我們快沒命啦!」 \end{tabularx} \\ \\ \relax
8:26 & \begin{tabularx}{0.7\textwidth}{X} 耶穌說:「你們這些小信的人哪,為甚麼膽怯呢?」於是他起來,斥責風和海,風和海就大大平靜了。 \end{tabularx} \\ \\ \relax
8:27 & \begin{tabularx}{0.7\textwidth}{X} 眾人驚訝地說:「這是怎樣的一個人?連風和海都聽從他。」 \end{tabularx} \\ \\ \relax
8:28 & \begin{tabularx}{0.7\textwidth}{X} 耶穌渡到對岸去,到加大拉人的地區,有兩個被鬼附的人從墳墓迎著他走來。他們極其兇猛,甚至沒有人敢從那條路經過。 \end{tabularx} \\ \\ \relax
8:29 & \begin{tabularx}{0.7\textwidth}{X} 他們喊著說:「神的兒子,你為甚麼干擾我們?時候還沒有到,你就上這裡來叫我們受苦嗎?」 \end{tabularx} \\ \\ \relax
8:30 & \begin{tabularx}{0.7\textwidth}{X} 離他們很遠,有一大群豬正在吃食。 \end{tabularx} \\ \\ \relax
8:31 & \begin{tabularx}{0.7\textwidth}{X} 鬼就央求耶穌,說:「若要把我們趕出去,就打發我們進入豬群吧!」 \end{tabularx} \\ \\ \relax
8:32 & \begin{tabularx}{0.7\textwidth}{X} 耶穌對他們說:「去吧!」鬼就出來,進入豬群。一轉眼,整群豬都闖下山崖,投進海裡,淹死了。 \end{tabularx} \\ \\ \relax
8:33 & \begin{tabularx}{0.7\textwidth}{X} 放豬的就逃進城去,把這一切事和被鬼附的人所遭遇的都告訴眾人。 \end{tabularx} \\ \\ \relax
8:34 & \begin{tabularx}{0.7\textwidth}{X} 全城的人都出來迎見耶穌,見了他以後,就央求他離開他們的地區。 \end{tabularx} \\ \\
[1ex]
\hline
\hline
\end{longtable}
$^{1}$好,頂姐妹晚安.
聖誕節這幾年都很怕說馬大福音一二章和奴家福音一二章.
可以說的都說過.
凡是很怕這些日子說一些聖誕的訊息.
已經很難說了.
特別為在聖誕節裡有不同需要的人特別紀念.
無論是留下了不同的頂姐妹在不同的地方.
或者在艱難容易困所裡.
都特別紀念到不同人的需要.
所以今天只穿了一件聖誕樹的衣服.
多謝贊助商的人好像說在迪士尼Fosen那裡買的.
牌子是迪士尼的.
我猜應該是.
所以選的經文我們不用聖誕節的經文.
我們今天用馬大福音第八章.
多謝網上的頂姐妹很聰明.
她以為會說的是十八章.
因為十八章的主題是關於小子.
但是我們今天是想說第八章.
為什麼會說第八章呢.
其實也很關乎我自己最近對福音書其中一個很大的難題.
其實我想有些時候了解一下.
其實我經常問的問題就是.
為什麼耶穌記載的事情那麼多.
為什麼會選某些事情要記載.
某些事情不記載.
這是我看福音書的時候經常想問的問題.
為什麼那些東西要寫在這裡.
不寫下去其他很多很值得寫的應該不寫.
為什麼要寫一些這樣的東西.
今天我們可以看四字經文.
其實我們看完之後.
講這四字經文的機會很少.
不過我想解釋一下大麻風這個「伊」字.
其實是什麼.
或者想講這個故事背後.
其實馬太在交代一件什麼事情.
不複雜的.
耶穌下了山.
有很多人跟從祂.

$^{41}$有一個長大麻風的人來拜祂.
主約肯就必能叫我潔淨了.
耶穌伸手摸了他說.
我肯你潔淨了吧.
他的大麻風立即就潔淨了.
耶穌對他說.
你切不可告訴人.
只要去把身體給祭司擦汗.
獻上摩西所吩咐的禮物.
對眾人作證據.
這個故事其實很容易理解.
不複雜.
耶穌摸了一個大麻風的人.
通常這個故事的應用就是.
去了德蘭修女.
通常這個故事講完的時候.
我們的結論或者總結.
或者可以應用到的就是.
請我們做一個德蘭修女.
摸一些有需要的人.
很骯髒的人.
或者一些很不可愛的人.
請你擁抱他.
但實際上這個故事講完之後.
我們得出的結論是這樣的.
但我今天想問多一點.
除了這個結論之外.
會不會有多一些.
這個故事寫出來的時候.
不一樣的結論和想像呢?.
所以我們今天希望能夠從一個角度去看.
這四節聖經其實是在講一件什麼事情.
複雜的事情我們不是很想討論.
第一節.
耶穌下了山.
有很多人跟從著祂.
這個句子.
我們不要說一些希臘文複雜的東西.
這個句子背後其實想交代一件事情是什麼呢?.
5至7章是耶穌刻意在山上.

$^{81}$做了一個叫登山補糞的事情.
登山補糞就是祂在上邊.
交代所謂八福.
所謂主禱文.
被人打完左邊右邊.
講了很多很多不同的事情.
收集在一起.
叫做登山補糞.
登山補糞的故事是怎麼來的呢?.
大致上很多學者都會認為.
耶穌在那裡做了一件事.
就是重新詮釋摩西的律法.
等於很簡單的意思就是說.
如果昔日摩西在山上頒布律法的話.
耶穌做的事情很相似.
都在山上重新詮釋摩西的律法.
你會說不要殺人.
其實不是的.
心裡說別人蠢材死蠢.
你還是殺了人吧.
所以耶穌在做的事情.
很像摩西在山上領受律法一樣.
不過耶穌這次不同.
祂重新詮釋摩西的律法是什麼.
所以如果你這樣理解.
五至七章很像摩西的話.
第八章的故事.
或者第八章第九章的故事.
其實就是想表達一件事.
耶穌所做的所有事情.
尤其是在第八章第九章裡.
不是隨機地去選擇一些事情.
這個故事我記得耶穌這樣做過.
那件事耶穌做過很出名.
我又放進來了.
如果那是關於耶穌重新演繹摩西頒布律法的話.
第八至九章就是說.
耶穌如何在實際行動裡.
演繹在五至七章登山補訓裡.
所說的每一件事.

$^{121}$所以你會發覺五至七章登山補訓所說的.
其實我們很難應用.
你不會被人打完左邊右邊.
你不會看到一些你不應該看的東西的時候.
你挖自己的眼睛.
斬自己的手.
但是上登山補訓.
很難去應用.
你聽到別人說他笨一點.
你說他死蠢.
難道你又落地獄之火.
難免你就會受到.
所以登山補訓的事情.
其實是說完之後.
不知道如何實體去應用.
所以馬太她選的八至九章.
所有的故事.
其實是用來回應.
耶穌在五至七章登山補訓.
說完那些事情之後.
他找一些故事來.
告訴我們.
登山補訓其實是如何應用的.
其中登山補訓這一句也很有名.
他不是來廢掉律法.
而是承傳律法.
所以耶穌在八至九章所做的故事裡.
第一個故事.
就是他下山之後.
他遇到大麻風的人.
他治好了他.
其實這個的醫治.
正正代表耶穌要來.
承傳律法的故事.
聽到這裡.
其實我們可能不是很習慣聽這些.
你覺得耶穌如何承傳律法.
例如最近猶太人慶祝節期.
光明節 夏律卡.
你會說我都不慶祝猶太人的節期.

$^{161}$為什麼耶穌要來承傳律法.
他承傳什麼律法呢.
如果我們想多理解和明白的話.
一會兒你再聽下去的時候.
你會知道耶穌在承傳律法裡.
他在做什麼改革與更新.
這是在利美記十四章第二節.
他說長大麻風得潔淨日子.
其禮乃是者曰.
要帶他去見濟世.
其實在利美記裡也挺特別的.
利美記裡有兩張聖經說大麻風.
兩張聖經.
其中一句我們很出名的.
在這四節經文裡.
我們通常會用的.
那句是什麼.
如果大麻風來的時候.
你就要跟別人說不潔淨了.
大家快點走.
通常都要引用利美記的經文.
但你會發現利美記裡有兩張聖經.
十三十四章的兩張聖經.
專門說大麻風.
你會發現大麻風其實.
坦白說你看完利美記.
看到利美記的時候.
你會發現利美記的大麻風.
由於七天之後好一點.
再等七天.
再等七天又給濟世看.
看完之後你的身體怎麼樣.
那種煩惱複雜.
好像與愛無關.
所以聖經其實沒什麼用.
尤其是利美記說的東西.
對我們完全沒有什麼關係.
但你不能忽略的是.
這兩張聖經裡.
它說大麻風的時候.

$^{201}$重點是什麼.
其實整個大麻風.
這兩張聖經.
記載的只有一個重點.
如果你看.
你可以現在滾手機.
不要介意.
你看十三十四章.
大部分的內容.
說的是一件事情.
就是說怎樣接納大麻風的人.
在自己群體裡.
我再來一次.
我的意思是.
利美記這兩張好像很悶的聖經.
很無聊的聖經.
什麼七天七天.
又怎樣檢查它.
又要放在哪裡.
又要找祭司等七天.
肉又要看看怎樣.
七天之後又要回來.
如果沒事就獻祭.
然後就抓他回來.
它搞這麼長的篇幅.
就是想說.
凡是有大麻風的人.
人都會將他放在一旁.
將他放在最遠的地方.
利美記十三十四章寫這麼長的東西.
就是想表達一件事情.
就是我們常常將他放在一旁.
而不理會的人.
聖經執著的是.
要將他回到自己群體裡.
花很多的時間和氣力.
花很多的精神和時間.
祭司要做很多的事情.
確保這個人能夠回到自己群體裡.
所以利美記其實是說什麼呢.

$^{241}$真正接納大麻風這類型的人.
回到自己群體裡.
如果要回到今天.
不一定要去嘉義國塔.
像德蘭修女那樣.
接觸那些很骯髒很骯髒的人.
其實今天應該在說什麼呢.
其實今天這個故事.
要想說的或是想表達的是什麼呢.
如果你看回.
在一世紀的時候.
你記不記得耶穌和那些罪人吃飯.
耶穌和那些罪人吃飯的時候.
那些法利塞人.
西門家裡的法利塞人的朋友.
他們罵耶穌什麼呢.
罵耶穌是你為什麼和那些罪人吃飯.
當耶穌要趕鬼的時候.
耶穌被人罵什麼呢.
你只不過是靠著鬼王別世不趕鬼.
奉係上帝的兒子耶穌.
接納那些罪人.
接納那些鬼婦的人.
回到自己群體的時候.
其實大部分的人就不喜歡.
其實大部分的人就不知道.
突然間就說.
這班大麻瘋的人是上帝就坐的.
是上帝不喜歡的.
所以在拉比的文獻裡.
曾經形容過大麻瘋是什麼.
說他們是活著的死人.
其實根本是死的.
不過是活著的.
所以我們不用理會他.
但事實上他活著.
但他差不多死了.
用宗教的條例.
將一些人家不滿的人.
人家不喜歡的人.

$^{281}$將他放在一旁.
只接納一些同聲同氣.
好像自己談得來的人.
這是當時耶穌要突破的東西.
其實我不知道大家是否發現.
在我們的圈子裡.
我們越來越少見很多.
我應該不要這樣說.
應該怎麼說.
很特別的人.
好像我們圈子裡.
就有些很類似的人.
譬如以前.
在教會裡.
我們做人格的時候.
你發現教會裡最多的是什麼.
二號仔.
或者我這類的九號仔.
二號,九號在教會裡很多.
多到不能再多.
你知道教會裡最少的是哪類型嗎.
我的資料不準確.
八號仔是最少見的.
如果是八分之五的話.
八號仔就是意見領袖.
五號仔是思考者.
全部用腦和說話的人.
在教會裡很少出現.
教會裡多的是像你這樣.
很乖的聽道的人.
你不用想.
總之你說什麼就是.
八號和五號是教會很絕的生物.
所以一輪.
八號仔進到教會.
我就很想留住他.
希望他生存在教會裡出現.
六號仔我就不理他.
你知道六號仔是忠誠者.
你不理他.

$^{321}$他都會坐在那裡.
他很忠誠.
他一定坐在那裡.
他回來就是了.
所以我不是想說那些.
很骯髒的人.
是我們教會某個形式模式裡.
有些人在那裡生存不了.
這是我們的現況.
更重要的是什麼.
我們以為我們這樣生存是對的.
我們以為就這樣.
就這樣做.
我們就要這樣走.
這樣就適合某些人就OK了.
但需不需要你那些.
未必常在的人.
他一定要邀請他回來.
或者他容易在那裡生存得久.
這些慢慢不成為我們心裡.
要想的想法.
耶穌是一個.
整個人生裡.
祂做一件事我懷疑祂.
在很多錯誤解釋的真理.
在很多人將那個真理扭曲了一些.
表面上好像很對.
但背後其實是很錯的東西.
祂全部將那些東西扑出來.
這是耶穌想做的事.
以致能夠容納到更多不同的群體.
能夠進到上帝家裡.
全面的教會.
希望是一個什麼樣的群體.
無論是三千八個奇人怪狀.
你定義一下自己是否這樣.
我是不是這樣.
那些奇人怪狀和三千八個的人.
你都可以覺得你被歡迎.
你被接納.

$^{361}$最慘的是什麼.
當一個群體以為自己可以很穩定.
很平穩的時候.
我們慢慢就成為了很多的界線.
成為很多的牆.
將很多跟我們好像不同的人.
放在外面.
教會最難的是什麼.
是永遠都不知道自己.
在做的事其實是將一些人.
放在外面.
並且以為自己在做的事.
是很正確的.
我們不要說教會.
好像很複雜的題目.
我們說我八月的時候.
我講道壽的時候說過.
我兒子在英國的時候很煩.
我罵他一頓.
記不記得.
我還跟他說.
我要跟兒子找時間道歉.
好像八月的時候我這樣說過.
其實我沒有交代.
其實九月的時候.
我真的跟我兒子道歉了.
為什麼會道歉呢.
其實我都忘記了我要道歉.
因為我女兒聽到我去福爾摩教會講道.
(笑).
好像一家人吃飯.
吃飯的時候我女兒無聲無息地說一句.
爸爸你不是說要跟我兒子道歉嗎.
她說沒辦法.
你不道歉就好像很壞.
道歉也不知道在做什麼.
很少這樣.
然後我真的很恭敬地.
在兒子旁邊吃飯.
說兒子對不起.

$^{401}$爸爸不是很明白你那一次.
希望你原諒爸爸.
說完以為我兒子九歲十歲就當沒事.
繼續吃飯.
我就覺得這樣就完了.
可以跟你們交代.
我做了 多好.
我收到劇本大約是這樣寫.
誰知道我兒子突然.
撲到我大腿上.
整塊臉撲到大腿上.
然後嚎哭.
哭得很厲害.
哭完之後他一起來.
我的褲子兩條濕的淚痕.
鼻涕加上什麼都齊了.
我不忘記我兒子哭完之後.
他說了一句.
他說爸爸 你知不知道被人冤枉.
很慘的.
到今天我還在學什麼呢.
我現在罵他.
我也想我沒有冤枉他.
起碼我到今天.
我要很生氣去罵他的時候.
我也想我這次沒有冤枉他.
起碼我學.
所以聽完節目你會發覺.
耶穌來到世間上.
祂所做的事是什麼.
是很多我們以為我們很合得來的時候.
祂突然說了一些話.
其實你想一些事不一定是這樣.
其實祂好像登山佈墳一樣.
上次告訴你.
你以為守了規矩就可以.
搞定那些事就成為正常的基督徒.
祂告訴你不是.
祂明明可以不摸那個人.
他就可以潔淨.

$^{441}$他怎樣都要摸一摸.
祂這個動作想告訴你.
大家都覺得很礙眼.
你摸他的時候你自己都不潔淨.
你馬上要找個祭司來.
看看你有沒有發作.
要關他七天.
看看麻瘋病有沒有傳染給你.
但耶穌都不介意.
要做一些很奇怪的事.
告訴人家.
我要接納這個人回來我這裡.
在2024年很快就來.
我自己對自己一個想法.
我想給自己多一些.
以往覺得很適合的東西.
重新想一想那些很適合的東西.
是不是很適合.
基督教好像越來越是一班.
覺得自己很適合的人.
走在一起做一些好像很適合的事.
但我們很少去留意反省一下.
到底我們經常覺得我們很適合的東西.
是不是真的很適合.
2024年是一個很大的分水嶺.
分水嶺的意思是.
要麼我們繼續以往的信仰模式的群體.
要麼我們嘗試反省完.
想完.
自己和一班人也好.
不知怎樣也好.
嘗試在以往很慣常覺得很適合的裡面.
你抽出來嘗試做一些人家會說鬼你的事.
人家會說你幹嘛碰他大麻瘋.
嘗試在這個年代和不容易的環境裡.
做一些人家未必覺得你適合的事.
但你冒險告訴自己.
我想有一個不一樣的信仰表達.
如果那時候耶穌看著大麻瘋的人.
每個人都是說讓祭司來搞.

$^{481}$我們不用怎麼搞.
放在祂一邊.
祂們只是活著的死人.
都不關我們事.
耶穌就在宗教上覺得大家都適合的裡面.
就要告訴所有人.
祂碰完祂 潔淨了祂.
祂做了一件人家覺得匪夷所思的事.
所以無論祂在安息人家醫病也好.
或是安息在裡面澆墨水也好.
祂所做的每一個動作.
祂由出生到成長到傳道那幾年.
祂正在挑戰的是.
是否我們覺得我們想的事都正正正正.
如果那班二千年前的人.
是很聰明的人 很厲害的人.
在宗教上玩得很好的人的時候.
我們是否真的會比他們厲害多了.
2024年我不知道你有什麼想法.
或者2023年尾聲的時候.
當我們見到耶穌出生的時候.
祂不純粹是一個嬰兒.
祂不純粹是一個聖嬰孩.
僅此而已.
祂出來的目的.
為了要將很多人語為對的宗教的東西.
祂拆了它.
祂建立了一個叫做.
跟從耶穌群體的一班人出現了.
秋天兄姊妹.
Fold Church是一個群體.
不是因為我們有什麼獨特的東西.
如果純粹是因為Fold Church很獨特而來的話.
這只不過是說.
這些獨特的東西有一天會不再獨特下去.
Fold Church就完了.
所以獨特的東西完了一年兩年三年四年五年.
這次我在這裡說了多少年.
五年還是六年?五年吧.
好像是這樣.

$^{521}$已經說了五年.
我們沒有獨特的了.
聽了五年一個人說道.
我做了每個人一次.
沒有什麼好聽的了.
有些人不再獨特的時候.
什麼是定義Fold Church這個群體獨特的呢?.
不是Fold Church出來做的東西很獨特.
是Fold Church的群體裡.
每一個人都獨特的.
是每一個人都找了一樣東西.
是他以往覺得對的.
突然間他抹掉.
其實不一定是這樣.
我們都為上帝做了一些不同的事.
好像耶穌摸了大麻風一樣.
讓不同群體的人能夠回來我們裡面.
真正吸引人的不是Fold Church本身運作的獨特.
Fold Church獨特的是每一個個體頂尖會參與的人.
他不只是聽完明白完一些東西.
不同的東西就完了.
我希望2024年.
我自己有些東西不同.
我在2024年裡都在想一些東西是不同的.
我已經2023年準備了一年.
希望2024年有些東西會看起來是不同的.
試一下信仰不行禮如儀.
試一下不要純粹因為Fold Church.
好像很獨特而來到一個很獨特的群體.
這些獨特的已經過了幾年不獨特了.
明年也是John說到.
明年又是他又是潘Sir和我.
沒有什麼特別的.
獨特的是個別,個體.
在這個地方裡.
你可以被鼓勵,被相信,被肯定.
你可以做一些不同的事.
跟你以往宗教行為模式的表現不同的東西.
聖誕節是一個開心的日子.
很歡喜的日子.

$^{561}$你怎麼名字都可以.
我期望這個聖誕節.
認識主耶穌基督降臨的時候.
他可以將整個宗教系統.
轉變成反轉的.
今天香港教會需要一班人起來.
將我們以往覺得要這樣信耶穌.
這樣表達信仰的東西.
可以轉變成反轉.
讓人看到.
像上次所說.
這班是一班搞亂天下的人.
就好像耶穌摸過大麻蜂一樣.
不要死心.
不要覺得信仰就是這樣.
仍然對這個世界有想像.
對我們的主有盼望.
在極艱難,不容易的2024年裡.
找什麼抗衡那些艱難.
似非而是,似是而非.
連欺騙都要值得慶祝的群體.
因為我們可以做一些.
令人驚訝的信仰行為表達.
耶穌五至七章的登山部分.
不是純粹說的.
他是用八章,九章.
他所說的內容和表達的故事行為.
告訴我們.
他如何顛覆人的想像.
如何搞亂天下的人.
最後Fold Church的頂姐妹.
包括海外的頂姐妹.
我們需要一班搞亂天下的人.
你發不發覺.
小朋友經常喜歡找真理.
我兒子覺得我在冤枉他.
他很深信我在冤枉他.
你發不發覺他青少年時有種墮落感.
我做父親的.
不斷冤枉他,冤枉到他十二,三歲.

$^{601}$他不會再記在心裡.
他會覺得這個爸爸是這樣.
我懶得跟你說任何真理.
成人為何會有種墮落感.
因為青少年已經墮落了.
你發覺跟人說真理是沒有用的.
跟人說正常的事.
世界的人聽不明白.
他還會說你回頭.
所以我不理你.
為何要回去做一個小朋友.
為何天國進入要像小朋友一樣.
因為只有小朋友才會肆無忌憚.
不耐煩.
不理任何人的眼光.
覺得自己愚蠢與否.
我想說這是我還在堅持的事.
希望這個聖誕節.
少在信仰裡耕耘.
不要跟自己說其實差不多.
信仰來到這個階段,就這樣就夠了.
這就像我女兒.
她跟你多說一句,你也會浪費氣力.
她會聽你寫這句.
她也不跟你說,你也不明白她.
她跟你說完,你也不知道她在做甚麼.
她說BTS十週年.
有甚麼好慶祝的.
那五百多元的東西買來做甚麼.
我都不明白.
她跟我說完.
買了那套東西,五百多元.
CD,現在沒有人用CD機了.
買CD做甚麼?.
她是傻的.
她說我才是傻.
但她不告訴我,她藏在裡面.
但小朋友不同的是.
他會堅持.
他會把他覺得要相信的東西.

$^{641}$告訴你.
今天我需要一班小朋友.
在信仰群體裡有這樣的表達.
我一起來禱告.
天父,我求你幫我們.
要突破自己.
要告訴自己,自己所相信的東西不夠.
所相信的表達很微弱.
甚至有時候會走錯路.
都好像很艱難.
我們要再走多一步.
將我們心裡所相信的東西走出來.
更困難.
所以我祈求的是.
在聖誕節.
在這個時機,我們要在你面前.
不只是開心.
不只是吃聖誕大餐.
不只是看無人機的燈火表演.
我求你祝福我們.
在我們面對2024更艱難的日子.
我們可以實質在這個世代.
做一些不同的事.
我求天父你組合我們.
把我們的人在不同的位置組合在一起.
可以在這個世代裡.
做我們覺得應該要做的事.
Fortune有獨特的群體.
但Fortune延伸出來的.
後會有更多獨特的群體.
我求天父你保守大靈主.
聽我們在面前的祈禱.
奉禱耶穌保衛命運.
Amen.
謝謝大家.
謝謝大家收看 再見.
\newpage



\section{路加福音 2:10-12-49-20231223}
\label{sec:dT3dN2jF8BQ}
\textbf{【網上崇拜】天選的細路|路加福音2\_10-12,49|20231223 [dT3dN2jF8BQ]}
\newline
\newline
連結: \href{https://youtube.com/watch?v=dT3dN2jF8BQ}{\texttt{ https://youtube.com/watch?v=dT3dN2jF8BQ}} ~~~~ 語音日期: 2023-12-23 
\newline
\newline
\hyperref[sec:sKBDQD8UIMg]{\small{< < < PREV SERMON < < <}}
~
\hyperref[sec:index_chronic]{\small{[返順時目]}}
~
\hyperref[sec:index_scriptual]{\small{[返順卷目]}}
~
\hyperref[sec:9ztySs_vnP4]{\small{> > > NEXT SERMON > > >}}
\newline
\newline
路加福音 2:10-12-49-20231223
\newline
\begin{longtable}{cl}
\hline
\hline
章節 & 經文 (和合本修訂版)\\
\hline
2:10 & \begin{tabularx}{0.7\textwidth}{X} 那天使對他們說:「不要懼怕!看哪!因為我報給你們大喜的信息,是關乎萬民的: \end{tabularx} \\ \\ \relax
2:11 & \begin{tabularx}{0.7\textwidth}{X} 因今天在大衛的城裡,為你們生了救主,就是主基督。 \end{tabularx} \\ \\ \relax
2:12 & \begin{tabularx}{0.7\textwidth}{X} 你們要看見一個嬰孩,包著布,臥在馬槽裡,那就是給你們的記號。」 \end{tabularx} \\ \\ \relax
2:13 & \begin{tabularx}{0.7\textwidth}{X} 忽然,有一大隊天兵同那天使讚美神說: \end{tabularx} \\ \\ \relax
2:14 & \begin{tabularx}{0.7\textwidth}{X} 「在至高之處榮耀歸與神!在地上平安歸與他所喜悅的人!」 \end{tabularx} \\ \\ \relax
2:15 & \begin{tabularx}{0.7\textwidth}{X} 眾天使離開他們,升天去了。牧羊人彼此說:「我們往伯利恆去,看看所成的事,就是主所告訴我們的。」 \end{tabularx} \\ \\ \relax
2:16 & \begin{tabularx}{0.7\textwidth}{X} 他們急忙去了,找到馬利亞和約瑟,還有那嬰孩臥在馬槽裡。 \end{tabularx} \\ \\ \relax
2:17 & \begin{tabularx}{0.7\textwidth}{X} 他們看見,就把天使論這孩子的話傳開了。 \end{tabularx} \\ \\ \relax
2:18 & \begin{tabularx}{0.7\textwidth}{X} 聽見的人都詫異牧羊人對他們所說的話。 \end{tabularx} \\ \\ \relax
2:19 & \begin{tabularx}{0.7\textwidth}{X} 馬利亞卻把這一切的事存在心裡,反覆思考。 \end{tabularx} \\ \\ \relax
2:20 & \begin{tabularx}{0.7\textwidth}{X} 牧羊人回去了,因所聽見所看見的一切事,正如天使向他們所說的,就歸榮耀於神,讚美他。 \end{tabularx} \\ \\ \relax
2:21 & \begin{tabularx}{0.7\textwidth}{X} 滿了八天,他們就給孩子行割禮,又給他起名叫耶穌;這是他還沒有在母腹裡成胎以前天使所起的名。 \end{tabularx} \\ \\ \relax
2:22 & \begin{tabularx}{0.7\textwidth}{X} 按摩西律法滿了潔淨的日子,他們就帶著孩子上耶路撒冷去,要把他獻給主。 \end{tabularx} \\ \\ \relax
2:23 & \begin{tabularx}{0.7\textwidth}{X} 正如主的律法上所記:「凡頭生的男子必歸主為聖」; \end{tabularx} \\ \\ \relax
2:24 & \begin{tabularx}{0.7\textwidth}{X} 又要照主的律法上所說,用一對斑鳩,或用兩隻雛鴿獻祭。 \end{tabularx} \\ \\ \relax
2:25 & \begin{tabularx}{0.7\textwidth}{X} 那時,在耶路撒冷有一個人,名叫西面;這人又公義又虔誠,素常盼望以色列的安慰者來到,又有聖靈在他身上。 \end{tabularx} \\ \\ \relax
2:26 & \begin{tabularx}{0.7\textwidth}{X} 他得了聖靈的啟示,知道自己未死以前必看見主所立的基督。 \end{tabularx} \\ \\ \relax
2:27 & \begin{tabularx}{0.7\textwidth}{X} 他受了聖靈的感動,進入聖殿,正遇見耶穌的父母抱著孩子進來,要照律法的規矩而行。 \end{tabularx} \\ \\ \relax
2:28 & \begin{tabularx}{0.7\textwidth}{X} 西面就把他抱過來,稱頌神說: \end{tabularx} \\ \\ \relax
2:29 & \begin{tabularx}{0.7\textwidth}{X} 「主啊,如今可以照你的話,容你的僕人安然去世; \end{tabularx} \\ \\ \relax
2:30 & \begin{tabularx}{0.7\textwidth}{X} 因為我的眼睛已經看見你的救恩, \end{tabularx} \\ \\ \relax
2:31 & \begin{tabularx}{0.7\textwidth}{X} 就是你在萬民面前所預備的: \end{tabularx} \\ \\ \relax
2:32 & \begin{tabularx}{0.7\textwidth}{X} 是啟示外邦人的光,是你民以色列的榮耀。」 \end{tabularx} \\ \\ \relax
2:33 & \begin{tabularx}{0.7\textwidth}{X} 孩子的父母因論耶穌的這些話就驚訝。 \end{tabularx} \\ \\ \relax
2:34 & \begin{tabularx}{0.7\textwidth}{X} 西面給他們祝福,又對孩子的母親馬利亞說:「這孩子被立,是要叫以色列中許多人跌倒,許多人興起;又要成為毀謗的對象, \end{tabularx} \\ \\ \relax
2:35 & \begin{tabularx}{0.7\textwidth}{X} 叫許多人心裡的意念顯露出來;你自己的心也要被劍刺透。」 \end{tabularx} \\ \\ \relax
2:36 & \begin{tabularx}{0.7\textwidth}{X} 又有位女先知,名叫亞拿,是亞設支派法內力的女兒,年紀已經老邁,從童女出嫁,同丈夫住了七年, \end{tabularx} \\ \\ \relax
2:37 & \begin{tabularx}{0.7\textwidth}{X} 就寡居了,現在已經八十四歲。她不離開聖殿,禁食祈求,晝夜事奉神。 \end{tabularx} \\ \\ \relax
2:38 & \begin{tabularx}{0.7\textwidth}{X} 正當那時,她進前來感謝神,對一切盼望耶路撒冷得救贖的人講論這孩子的事。 \end{tabularx} \\ \\ \relax
2:39 & \begin{tabularx}{0.7\textwidth}{X} 約瑟和馬利亞照主的律法辦完了一切的事,就回加利利,到自己的城拿撒勒去了。 \end{tabularx} \\ \\ \relax
2:40 & \begin{tabularx}{0.7\textwidth}{X} 孩子漸漸長大,強健起來,充滿智慧,又有神的恩典在他身上。 \end{tabularx} \\ \\ \relax
2:41 & \begin{tabularx}{0.7\textwidth}{X} 每年逾越節,他父母都上耶路撒冷去。 \end{tabularx} \\ \\ \relax
2:42 & \begin{tabularx}{0.7\textwidth}{X} 當他十二歲的時候,他們按著過節的規矩上去。 \end{tabularx} \\ \\ \relax
2:43 & \begin{tabularx}{0.7\textwidth}{X} 守滿了節期,他們回去,孩童耶穌仍舊在耶路撒冷。他的父母並不知道, \end{tabularx} \\ \\ \relax
2:44 & \begin{tabularx}{0.7\textwidth}{X} 以為他在同行的人中間,走了一天的路程才在親屬和熟悉的人中找他, \end{tabularx} \\ \\ \relax
2:45 & \begin{tabularx}{0.7\textwidth}{X} 既找不著,就回耶路撒冷去找他。 \end{tabularx} \\ \\ \relax
2:46 & \begin{tabularx}{0.7\textwidth}{X} 過了三天,他們發現他在聖殿裡,坐在教師中間,一面聽,一面問。 \end{tabularx} \\ \\ \relax
2:47 & \begin{tabularx}{0.7\textwidth}{X} 凡聽見他的人都對他的聰明和應對感到驚奇。 \end{tabularx} \\ \\ \relax
2:48 & \begin{tabularx}{0.7\textwidth}{X} 他父母看見就很驚奇。他母親對他說:「我兒啊,為甚麼對我們這樣做呢?看哪,你父親和我很焦急,到處找你!」 \end{tabularx} \\ \\ \relax
2:49 & \begin{tabularx}{0.7\textwidth}{X} 耶穌對他們說:「為甚麼找我呢?難道你們不知道我應當在我父的家裡嗎?」 \end{tabularx} \\ \\ \relax
2:50 & \begin{tabularx}{0.7\textwidth}{X} 他所說的這話,他們不明白。 \end{tabularx} \\ \\ \relax
2:51 & \begin{tabularx}{0.7\textwidth}{X} 他就同他們下去,回到拿撒勒,並且順從他們。他母親把這一切的事都存在心裡。 \end{tabularx} \\ \\ \relax
2:52 & \begin{tabularx}{0.7\textwidth}{X} 耶穌的智慧和身量,並神和人喜愛他的心,都一齊增長。 \end{tabularx} \\ \\
[1ex]
\hline
\hline
\end{longtable}
$^{1}$大家知道嗎?.
三千八百元座位的你們知道嗎?.
第一句想說的話是不是說聖誕快樂呢?.
我都想了一會.
我這樣說會不會很離地?.
所以我輕輕搜尋了一下,麻煩你拍拍手.
前兩年Full Church的聖誕崇拜.
說完到底是怎麼說的?.
大家有沒有印象?.
原來John是一個很好的教師.
他教了很多人.
大家有沒有印象?.
原來John是沒有說聖誕快樂的.
他一開始就介紹他的introvert 2000 pro.
Pawn Sir也是沒有說聖誕快樂的.
他說的是很開心我們終於有場地.
不受限制,可以100\%讓大家一起聚會.
到旁邊那個,大家認不認得?.
今天不在的,天目者.
他開頭第一句都是說,弟兄姊妹平安.
其實不說聖誕快樂.
已經是我們Full Church的傳統.
有人跟我說,報佳音還唱We Wish You a Merry Christmas.
我都不好好想一想.
他說大家想報一個怎樣的福音?.
我不是第一年聽到這個問題.
這個真的是一個很值得思考的東西.
所以今天.
其實我是不是應該選擇馬太福音更加適合呢?.
應是被追殺.
一個不平安的聖誕.
似乎這些.
是跟我們現在的生活更加相近.
起碼不像路加,是嗎?.
大家有沒有看過那段經文?.
他筆下的耶穌降生是揚日著歡喜快樂的.
有天使報大喜的訊息.
有牧羊人見證.
有一大隊天兵加入去到炸尾神.
之後還有寫兩個老人家.

$^{41}$有西冕和阿娜.
他在聖殿裡面稱頌著.
這個是路加福音所說的耶穌降生.
但是弟兄姊妹大家記不記得?.
又或者大家有沒有剛剛好.
在昨天的Discord裡面的靈修遊戲室.
又一起靈修了嗎?.
路加福音一開始就表示.
他的寫成是按著次序.
他說他為的是要讓人因此更加確信.
耶穌基督.
所以今天我想問的問題是.
路加這樣去記載耶穌降生.
在2023年的聖誕節.
我們怎樣看?.
我們今天會看兩小段經文.
第一段在路加福音2章10至12節.
那部分是.
經文一開始.
天使跟一群牧羊人說.
不要害怕.
因為那個時候有主的使者站在他們旁邊.
主的榮光四面照著他們.
牧羊人見到上帝的榮光.
他們就很害怕.
如果我們記得的話.
每一次救藥的人見到上帝的時候.
他們都是害怕的.
上帝一顯現他們就害怕.
大家有沒有印象?.
二菜啊,是不是?.
二菜啊,見到主坐在寶座上的時候.
他害怕得要死.
他說我有禍了,我要死了.
而這次是一群牧羊人見到主的使者.
他們就很害怕.
他們一群人都是非常害怕.
但弟兄姊妹,這次的害怕.
加多了一重的意思.
對於牧羊人來說.

$^{81}$他們可能怎樣也想不到.
原來有生之年.
可以有這番奇遇.
因為畢竟上帝的聲音.
彷彿停了四百年.
大家記不記得?.
繼先知馬拉基之後.
已經很久很久沒有先知出來.
再傳達上帝的話.
也很久很久沒有見到上帝的工作.
沉默了幾代人的日子之後.
牧羊人見到上帝再次出手工作.
十至十二節是這樣說的.
他說:天使華康啊.
我報給你們大喜的消息.
是關於萬民的.
今天在大衛的城裡為你們生了救主.
就是主基督.
你們要找到一個鷹鞋.
花雀布,我像馬槽裡.
那就是記號了.
這群牧羊人一聽到大衛的城.
一聽到救主,一聽到主基督這些關鍵字.
他們的天線立刻豎起了.
因為他們和所有的以色列一樣.
他們一直在等.
他們一直在等上帝會實現.
曾經藉著先知所說的英許.
那個英許大家都知道.
在大衛教會有一個苗裔.
成為他們永遠的王.
成為以色列人永遠的王.
所以當天使天兵離開了之後.
我們看十五至十七節.
牧羊人就立刻跑去伯尼行.
去找到馬槽中的那個鷹鞋.
當他們找到之後.
他們就說這件事是真的.
他fact check了.
上帝真的為我們成就了這件事.

$^{121}$他們就出去跟很多人說.
弟兄姊妹.
他們見證到的是.
上帝是一個說得出做得到的上帝.
他曾經的英許.
現在真的實現了.
他們又見證到的是.
上帝工作的時代.
隔了幾百年.
他們又再一次來到了.
上帝再一次出手工作.
這個我們都很明白.
我們已經聽了幾十年.
應該沒有的.
十年?幾年而已.
有些報佳音的事情.
但我發覺可能我們都有一個迷思.
覺得耶穌降生是二千多年前的事.
是上帝二千多年前的工作.
聖誕節紀念的是什麼?.
就是紀念二千多年前耶穌降生.
聖誕節對我們現在有什麼意義?.
除了我們可以連續放假之外.
熟悉一點說的.
我們有些事是熟悉一點說的.
熟悉一點說的就是.
紀念神很愛我們.
曾經在很多很多年前.
為神的兒子.
應該說是讓神的兒子.
為全人類降生.
所以一到聖誕我們就會想.
不如我們回教會.
大家千萬不要對它合作.
在我心裡.
沒有一個畫面出現.
沒有一個人出現.
最多是我媽媽.
因為以前我經常用聖誕聚餐來理由我媽媽.
不如你一年都回來一次.

$^{161}$聖誕節好像是什麼呢?.
好像是用來紀念的東西.
紀念曾經發生過的事.
紀念一些過去的事情.
一些事情我們不想遺忘.
所以我們有聖誕節.
我們紀念它.
就好像.
我今天本身拿了兩個磁石貼過來.
我忘記拿.
那些磁石貼就是我每一次去旅行的時候.
我都會去記住那趟旅程.
例如一個是澳洲的.
一個是英國的.
在我的袋子裡.
但不要緊.
因為最重要的是什麼呢?.
我會再去.
我下一次旅行去哪裡?.
我下一次旅行是何時?.
同樣地.
頂尖我們活在此時此刻.
不是活在二千多年前.
我們活在此時此刻的時候.
我們會繼續去祈禱等候.
我們等待上帝下一次出手的工作是什麼?.
我們最有信心的應該是.
我們盼望將來的耶穌.
第二次來臨.
再回來.
我們最有信心的就是.
我們會有一個幕後的審判.
另外的就是我們生活當中的事情.
我們祈禱上帝出手.
還有這幾年.
我們有很多很多集體祈禱的時間.
我們同心為大家一起經歷的事情祈禱.
至於上帝有多少出手呢?.
都是看上帝.
看上帝的主權.

$^{201}$看上帝的時間.
這是我們這幾年一直學習的功課.
是一個很好的.
是一個馴服等待的態度.
是很好.
因為上帝才是上帝.
我們是人.
但其實當我思考.
天使報很消失的這段經文的時候.
我是有一種想法浮現.
想跟大家分享一下.
其實上帝二千多年前的一次出手.
派耶穌來.
就已經足夠了未來所有時間的人.
包括現在的我們.
我再說一次.
就是上帝二千多年前的一次出手.
派耶穌來.
就已經足夠了未來所有時間的人.
包括我們和未來的人.
足夠了的意思是什麼呢?.
足夠的意思就是.
其實我們不用等再多都可以.
不需要更多都可以.
上帝那一次出手.
就跨時代地救了我們.
直到耶穌再次回來.
第一節我們想到保羅.
他說過一段很刻骨銘心的自白.
是《肥拿比書》三章.
其中第八節是這麼說的.
他說「不但如此」.
麻煩拍一下.
「不但如此,我又將萬事當作有損的,因我已認識我主基督耶穌為至寶,我為他已經凋棄萬事,看作糞土為要得著基督.」.
我剛剛說的是.
有耶穌這一手就足夠了.
但保羅更厲害.
他說其他東西都是垃圾.
只要認識耶穌就是至寶.
之後他解釋.

$^{241}$在三章十至十一節.
「使我認識基督,曉得他復活的大能,並且曉得和他一同受苦後發他的死,或我都得以從死裡復活.」.
我要解釋一下.
對他來說.
因為有耶穌降生這一手.
讓他可以認識到耶穌是怎樣的.
認識了之後.
他才明白兩件很重要的事.
第一,耶穌復活的大能.
第二,他知道要和耶穌一同受苦,後發他的死.
他說唯有這樣才有機會在死裡復活.
弟兄姊妹,我們現在記住這件事.
我們從另一個角度來看保羅.
大家會不會問一個問題.
保羅有沒有見過上帝會再一次出手.
有.
在《史獨行傳》中記錄.
他說被困在被囚.
被兩條鐵鏈捆綁住.
然後天使拍拍他.
鐵鏈就掉下來.
然後鐵門就像《多拉伊夢》的自動門.
自動打開.
然後徐燕門.
不知道是什麼門.
神蹟一樣很自由地可以放出來.
所以保羅是知道的.
他也知道上帝要出手的時候就會出手.
但對他來說.
這卻不是他所看的重點.
弟兄姊妹,有一樣很重要的事.
我想分享的是.
他經歷被打失去自由的時候.
他不是依靠上帝的特別出手.
他是依靠早就被耶穌基督.
在《肥納比書》他自己說.
只要認識耶穌.
就明白復活的大能是怎樣的.
他說單單耶穌來了.
自己可以認識到祂.

$^{281}$他就明白自己是要和祂一起受苦的.
他保羅自己是要效法耶穌的死.
這是他寫給教會的心聲自白.
說到這裡我分享一個小故事.
每一次我講flowchurch的道.
我都會先跟老公講.
這次他也有聽我說.
然後講到這一部分的時候.
他就說「好吧,有耶穌就夠了」.
「好耶,不用回教會了」.
「有耶穌就行了」.
「我也有耶穌,那還好吧」.
他就這樣跟我說.
所以因為他.
我就覺得會不會大家也有這些想法呢.
所以補充一句.
我要補充多一件事.
這也是保羅流著淚補充給教會聽的.
就在下一節,三章十二節.
保羅說.
「這並不是說我已經得著了」.
「已經完全了」.
「而是竭力追求」.
「好像我可以得著」.
「基督耶穌要我得著」.
保羅是覺得自己不夠多認識耶穌.
不是一個耶穌不夠.
而是自己不夠認識耶穌.
所以他說他要竭力追求.
弟兄姊妹我們這幾年其實也有很多信仰反思.
我相信每一個來Flow Church的弟兄姊妹.
可能他們會有更多.
你們會有更多的信仰反思.
所以你們會來到這裡.
所以現在我問一問大家.
我也問一問我自己.
耶穌對於我們來說.
是否已經足夠.
我們的信仰的心聲自白.
會是甚麼.

$^{321}$我們看看今天的第二段經文.
我再告訴大家.
我的心聲自白是甚麼.
第二段經文.
是寫耶穌作為小孩的時候.
終於踢到他小孩了.
因為之前其實也是嬰兒.
我看過網上的留言.
都是嬰兒.
因為他寫到小孩的時候.
但非常特別的是.
為何路加會寫耶穌降生的時候.
是寫到他12歲的時候才寫完.
我會問為甚麼.
為甚麼會這樣.
大家有沒有見過我們做耶穌降生的劇的時候.
除了塑膠嬰兒之外.
還有一個12歲的小孩會出來做這個角色.
我是沒有見過.
但作者路加是按次序的.
所以他寫12歲的耶穌的時候.
一定有他的意思.
他是怎樣寫的.
他說這個小孩跟著父母去聖殿的時候.
他不跟他們說話.
自己留在聖殿裡.
要父母瑪利亞和約瑟.
很擔心地找了他們三天.
其實在原文裡三天不是真的三天.
是說一段時間.
路加就是用這件事去完成耶穌降生這個記載.
我們一起分析一下.
耶穌失蹤這件事其實有兩個高潮.
一個就是找到耶穌之後.
發現他竟然坐在聖殿的老師當中對答.
就好像這張圖.
這個高潮是想說耶穌有超乎12歲小孩的智慧.
甚至是超乎在場所有老師的智慧.
這張46節是這樣記載的.
他說耶穌坐在教師中間.

$^{361}$一面聽一面聞.
不是每個人都可以坐.
我本身也可以坐但最後沒有坐.
因為我覺得自己不配.
不是每個人都可以坐.
本來是老師才可以坐.
現在的老師才這麼慘不能坐.
在課堂上只能站.
經文很仔細地形容.
一面聞一面聽.
他在說耶穌不是受教的學生.
而是在跟他們交流.
甚至是在教這班作為老師的人.
大家有沒有曾經腦中問過一個問題.
就是他們這樣圍著圈子.
他們在說什麼.
中文老師應該是在說中文.
IT老師應該是在說電腦程式.
在聖殿的老師其實應該是在說上帝的律法.
所以最後這張47節路加寫.
所有聽見黑的人都稀奇他的聰明和應對.
對於猶太人來說.
13歲是大人.
但12歲還是個小孩.
路加上等的高潮是.
耶穌這個還是12歲的小孩.
就已經超級有智慧.
是世界上最明白上帝律法的人.
我們記住這一點.
而另一個高潮.
應該說是最最高潮的就是.
耶穌出生之後說的第一句話.
也是他小時候唯一被路加記載的說話.
我們經常都學一件事.
上帝的說話很重要.
是腳前的燈是路上的光.
創世紀說.
神說要有光就要有了光.
而路加記載.
耶穌的第一句說話是.

$^{401}$為什麼你們要來找我.
你們不知道我必須以我父的事為念嗎.
以章49節.
他說.
為什麼要來找我.
第一個高潮.
突顯耶穌的智慧.
所以我們會知道.
這裡不是想批評耶穌態度差博咀.
反而是一直住這一刻.
你猜都猜不到他會這樣說的話.
去衝擊所有聽到的人和讀者.
所以這句話.
其實是路加記載最高潮的.
耶穌說.
我必須要以我父的事為念.
第一我們會知道.
耶穌要告訴瑪利亞和約瑟.
自己真正的父是誰.
不是約瑟你.
不是你.
是父上大人.
這個我們明白的.
誰是我們真正的父.
當我們信主的時候.
我們就知道.
我們是以上帝為我們的父.
第二件事.
有少許原文避不了.
要輕輕解釋一下.
我們手頭上可能有兩個譯本.
一個是譯作我必須在我父的家裡.
另一個是我必須要以我父的事為念.
因為原文當中.
我不想改得太.
我的「的」是一個 plural form.
所以我們認為.
必須以我父的事為念.
是更有說服力的.
如果大家想了解一下.

$^{441}$可以之後再問.
但反而今天想問的是什麼.
是父的事是什麼事.
耶穌所說的必須.
又是什麼必須.
有什麼是必須想念著的事.
這個才是我們真正要著緊的.
路加到最後.
都要說一個十二歲的耶穌出來.
在這個高潮.
我們就要問.
祂說了什麼.
我們看看路加福音.
記載著什麼是耶穌的必須.
因為原來在路加的筆下.
必須是一個很重要的詞語.
經文有些多.
我們一起進入耶穌的一生.
看看祂是如何實踐.
祂的孩童時候.
說過那句說話.
那句必須.
當耶穌行了很多神蹟之後.
眾人都想留住祂.
在路加福音4:43.
耶穌就跟他們說.
我必須再別承傳上帝國的福音.
因為我奉差緣而為此.
當彼得指耶穌是神的基督的時候.
9:22 耶穌說.
人子必須受許多的苦.
被長老濟斯掌王民氏氣絕.
並且被殺第三日復活.
到了中期.
當耶穌解釋神的國是怎樣的時候.
17:25 耶穌說.
只是他必須先受許多苦.
又被這世代氣絕.
到了最後晚餐的時候.
22:37 耶穌說.

$^{481}$我告訴你們.
經常寫著說.
他被列在罪犯之中.
這話必應驗在我身上.
因為那關係我的事.
必然成就.
到復活之後.
天使讓耶穌屍體的婦女回想起.
耶穌說過的一句話.
24:7 他說.
人子必須被交在罪人手裡.
釘在十字架上.
第三日復活.
到了最後.
在二萬五師的路上.
復活了的耶穌說.
24:26-27 他說.
基督怎樣受害.
又進入他的榮耀.
豈不是應當的嗎.
於是從摩西和眾仙之起.
凡經上所指著自己的說話.
都幫他們講解明白.
原來十二歲的耶穌說.
你們不知道.
我必須以我父的事為念嗎.
我的必須.
我的應當.
是他要傳上帝國的福音.
是他要先受害.
再進入他的榮耀.
弟兄姊妹.
他由降生.
就知道自己的命是怎樣過的.
但我們好像不知道.
我們的命是怎樣過的.
一部份我們會很想知道.
未來的自己會是怎樣的.
我聽過就算是.
我們作為基督徒.

$^{521}$有時都會看看星座.
我們會看看上座.
做下心理測驗等等各樣.
有一部份我們會很害怕.
很害怕未來.
讓人恐懼的社會.
從來都未試過.
禁不下2024年來到.
從來都未試過.
禁不下新的一年來到.
往年總會有一些.
對未來期望的問卷調查.
調查我們對未來.
有多少分的希望.
還是盼望指數.
但今年去到年尾.
都好像不覺有.
我還未收到這些電話.
因為我們知道他一定不會做.
因為做了一定會破身低.
所以究竟我們的命是怎樣.
第一節我們一起看看.
《徹行傳》兩段的經文.
因為除了耶穌的必需.
路加同樣記載了.
屬於我們的必需.
9章15至16節.
主德阿拿尼亞說.
你只管去.
你是我揀選的器皿.
要在愛邦人和君王.
並以色列人面前宣揚我的名.
我也要指示他.
為我的名必須受許多苦難.
這個不用怕.
是跟保羅說的.
14章22節.
就是跟我們所有人說.
堅固門徒的心.
勸他們行守所信的道.

$^{561}$有說避不了.
我們進入上帝的國.
必須經歷許多艱難.
這個是跟我們說的.
為基督耶穌的名.
必須受許多苦難.
進入上帝的國.
必須經歷許多艱難.
我們不需要對未來感到不確定.
因為這個就是我們的必需.
這個是聖經跟我們說的.
我們有時候可能會有僥倖的心態.
未必是我的.
特別是當我們不祈求上帝的經歷.
我們就不會遇到甚麼事.
有時候我們可能會有逃避心態.
我們很不想有這些艱難.
上學院的時候有一堂.
說到人總是很想去shake off.
shake off所有的痛苦和suffering.
當我的痛苦和艱難.
好像一堆堆黃色的面膜紙.
貼到我們全身都是的時候.
我們就很想甩開它.
用盡全力甩開這些黃色的貼紙.
除非有自虐傾向.
我們也不是M底.
我們自然也不會想有艱難.
我們祈禱通常都是祈求主幫我們脫離災禍困境.
但是弟兄姊妹.
這些苦難和艱難本來就是基督的必需.
如果我們能接受它.
我們就不是祈求上帝再一次幫我們.
出手甩開它.
而是祈求耶穌基督幫我們經歷過這些艱難.
焉知道這些痛苦可以成就的是甚麼事情.
因為到現在我仍然看到.
只是經歷艱難的人的安分力量.
就好像有人叫我們繼續寫信給他一樣.
因為他知道自己為了甚麼而受苦.

$^{601}$就像每一個母親.
作為媽媽為了生孩子.
她受十級的苦都可以.
保羅面對一切的鞭打苦待.
他沒有祈求上帝再一次出手.
他教導教會的是認定認識耶穌基督就夠.
救他一生知道怎樣走.
有力量去走.
救他救到自己的命.
息經息到這裡.
大家可以聽我分享我的感受.
我們這幾年失去了很多.
有很多新聞Google現在也搜尋不到.
也搜尋不到.
有很多人事物很珍貴很珍貴的東西沒有了.
損害了.
剛剛說到一個很怕2024年來臨的人其實就是我.
我自己30歲的時候都沒有怕過這個關口.
但是明年2024年我就怕.
我不想面對.
我很怕去經歷.
但是無論我們失去了甚麼也好.
耶穌基督是永遠永遠都不會失去.
除非是我們拋棄了他.
如果不是的話.
這個最重要最重要的耶穌基督.
只要我們抓緊了.
就夠了.
這個就是我的大喜訊息.
我一生當中沒有東西比這個訊息更令我大喜.
不是結婚.
Sorry Ryan.
Sorry老老公.
不是有一份好工作.
Sorry Flow Church.
因為耶穌降生.
意志的是我們可以在一個擁有耶穌的時代裡面.
一個擁有耶穌的時代.
救我們面對香港的新時代.
這個救是因為我們能夠以必須受的苦.

$^{641}$去面對我們不想受的苦.
我多謝12歲的耶穌.
他給我生命裡面最大的智慧和方向.
人生不知為何亂過一通.
短短幾十載.
不知活成什麼樣子就死去了.
人生真的很快.
這是我睡不著覺思考這段經文的時候想的東西.
以負的事為念這樣去走.
其實都不會走很久.
但如果我們真的這樣走的話.
我們就真的抓得住這個大喜的訊息.
這個小孩告訴我了.
他的路是怎麼過.
降生時候的一句話.
就是他的一生.
弟兄姊妹.
耶穌降生的意義並不是.
或者並不只是聖誕節前後那幾天.
紀念他為人類降生.
不是24,25號.
而是人生當中的每一個時刻.
我們怎麼活.
我們有時候都會陷進一個想法裡面.
包括我自己也是.
祈禱的時候求上帝幫助出手.
工作同在.
這幾年大家在Full Church聽了幾個篇.
回應這些問題.
我們又唱了幾首歌.
敬拜裡面我們向神呼喊.
我們坦承我們的懼怕.
之後我們又堅定地相信.
不過今天的聖誕崇拜.
講耶穌降生的時候.
老家告訴我們.
這個給普天下人的大喜訊息.
上帝已經工作了.
而不是我們現在.
只是單單等待上帝的工作.

$^{681}$上帝拯救已經賜下了.
救人生命的主都給了.
那條生命的路.
耶穌都已經向我們表明了.
聖靈都賜下給我們了.
稱得上為普天下的大喜訊息.
都應該只有這樣.
打機來說.
已經齊備了裝備.
那我們還要些什麼呢.
一切都夠了.
救我們面對一切現實和未來的風浪.
救我們過我們夢照的必需.
讓我們一起祈禱.
天父上帝多謝你.
多謝你賜下耶穌基督.
來到這個世界上.
這是一份絕大的恩典.
和一份大喜的訊息.
是由我們深深處的當中去覺得.
在我們生命裡面.
沒有一樣東西是比耶穌基督更加寶貴的.
所以究竟什麼叫做有耶穌就夠.
求主你教導我們.
什麼叫做有耶穌就夠.
好像保羅一樣.
看其他東西都是有損的.
只以認識耶穌基督為至寶.
如果我們還沒經歷過.
主求你教導我們.
求你幫我們這樣去經歷.
因為我們已經得到了.
我們就不想白白好像沒有得到過一樣.
所以天父上帝求你幫我們.
我們禱告奉主耶穌基督.
得勝明志祈求.
阿們.
\newpage



\section{耶利米書 1:1-8-20231230}
\label{sec:9ztySs_vnP4}
\textbf{【網上崇拜】一道風景,一種心靈|耶利米書1\_1-8|20231230 [9ztySs-vnP4]}
\newline
\newline
連結: \href{https://youtube.com/watch?v=9ztySs-vnP4}{\texttt{ https://youtube.com/watch?v=9ztySs-vnP4}} ~~~~ 語音日期: 2023-12-30 
\newline
\newline
\hyperref[sec:dT3dN2jF8BQ]{\small{< < < PREV SERMON < < <}}
~
\hyperref[sec:index_chronic]{\small{[返順時目]}}
~
\hyperref[sec:index_scriptual]{\small{[返順卷目]}}
~
\hyperref[sec:code]{\small{> > > NEXT SERMON > > >}}
\newline
\newline
耶利米書 1:1-8-20231230
\newline
\begin{longtable}{cl}
\hline
\hline
章節 & 經文 (和合本修訂版)\\
\hline
1:1 & \begin{tabularx}{0.7\textwidth}{X} 這些是便雅憫地亞拿突城的祭司,希勒家的兒子耶利米的話。 \end{tabularx} \\ \\ \relax
1:2 & \begin{tabularx}{0.7\textwidth}{X} 亞們的兒子猶大王約西亞在位第十三年,耶和華的話臨到耶利米。 \end{tabularx} \\ \\ \relax
1:3 & \begin{tabularx}{0.7\textwidth}{X} 從約西亞的兒子猶大王約雅敬在位的時候,直到約西亞的兒子猶大王西底家在位的末年,就是第十一年五月間耶路撒冷被擄時,耶和華的話也常臨到耶利米。 \end{tabularx} \\ \\ \relax
1:4 & \begin{tabularx}{0.7\textwidth}{X} 耶利米說,耶和華的話臨到我,說: \end{tabularx} \\ \\ \relax
1:5 & \begin{tabularx}{0.7\textwidth}{X} 「我尚未將你造在母腹中,就已認識你;你未出母胎,我已將你分別為聖,派你作列國的先知。」 \end{tabularx} \\ \\ \relax
1:6 & \begin{tabularx}{0.7\textwidth}{X} 我就說:「唉!主耶和華,看哪,我不知道怎麼說,因為我年輕。」 \end{tabularx} \\ \\ \relax
1:7 & \begin{tabularx}{0.7\textwidth}{X} 耶和華對我說:「不要說『我年輕』,因為我差遣你到誰那裡去,你都要去;我吩咐你說甚麼話,你都要說。 \end{tabularx} \\ \\ \relax
1:8 & \begin{tabularx}{0.7\textwidth}{X} 你不要怕他們,因為我與你同在,要拯救你。這是耶和華說的。」 \end{tabularx} \\ \\ \relax
1:9 & \begin{tabularx}{0.7\textwidth}{X} 於是耶和華伸手按住我的口,對我說:「看哪,我已將我的話放在你口中。 \end{tabularx} \\ \\ \relax
1:10 & \begin{tabularx}{0.7\textwidth}{X} 我今日立你在列邦列國之上,為要拔出,拆毀,毀壞,傾覆,又要建立,栽植。」 \end{tabularx} \\ \\ \relax
1:11 & \begin{tabularx}{0.7\textwidth}{X} 耶和華的話臨到我,說:「耶利米,你看見甚麼?」我說:「我看見一根杏樹枝。」 \end{tabularx} \\ \\ \relax
1:12 & \begin{tabularx}{0.7\textwidth}{X} 耶和華對我說:「你看得不錯;因為我要看守我的話,使它實現。」 \end{tabularx} \\ \\ \relax
1:13 & \begin{tabularx}{0.7\textwidth}{X} 耶和華的話第二次臨到我,說:「你看見甚麼?」我說:「我看見一個水燒開的鍋,從北而傾。」 \end{tabularx} \\ \\ \relax
1:14 & \begin{tabularx}{0.7\textwidth}{X} 耶和華對我說:「必有災禍從北方發出,臨到這地所有的居民。 \end{tabularx} \\ \\ \relax
1:15 & \begin{tabularx}{0.7\textwidth}{X} 看哪,我要召北方列國的萬族。這是耶和華說的。他們要來,各安寶座在耶路撒冷的城門口,周圍攻擊城牆,又要攻擊猶大的一切城鎮。 \end{tabularx} \\ \\ \relax
1:16 & \begin{tabularx}{0.7\textwidth}{X} 這民離棄我,向別神燒香,跪拜自己手所造的,我要針對這一切惡行,向他們宣讀我的判決。 \end{tabularx} \\ \\ \relax
1:17 & \begin{tabularx}{0.7\textwidth}{X} 所以你當束腰,起來,將我所吩咐你的一切話都告訴他們;不要因他們驚惶,免得我使你在他們面前驚惶。 \end{tabularx} \\ \\ \relax
1:18 & \begin{tabularx}{0.7\textwidth}{X} 看哪,我今日使你成為堅城、鐵柱、銅牆,對抗全地和猶大的君王、官長、祭司,並這地的百姓。 \end{tabularx} \\ \\ \relax
1:19 & \begin{tabularx}{0.7\textwidth}{X} 他們要攻擊你,卻不能勝過你,因為我與你同在,要拯救你。這是耶和華說的。」 \end{tabularx} \\ \\
[1ex]
\hline
\hline
\end{longtable}
$^{1}$我們從歲首到年終都敬拜主.
上次跟大家說到,不知道大家記不記得.
我不是回來追數.
你們許了的願應該自己去還.
上次跟大家思想薩姆爾.
薩姆爾是上帝興起的先知.
他是準備建立以色列國的.
今天跟大家思想另外一個先知.
是耶利米.
他是上帝興起準備宣告.
以色列要亡國.
一頭一尾,上帝都興起先知.
耶利米先知大概二十歲左右蒙召.
他見證著國家的興衰,戰敗.
經歷了四十多年多.
他到六十歲的時候.
他與他的國民一同被擄.
這個先知又經歷人生另一個階段.
今天我會透過導讀的方式.
我們會找幾節經文.
嘗試掌握耶利米書.
甚至是這位先知的人生.
耶利米的蒙召我們無法模仿.
因為他是很特別的蒙召.
但他經歷過的事.
他如何走過這些艱難的路.
我們是可以學習的.
今天想跟大家思想一個主題.
有人說人生好像坐在一部列車一樣.
環繞我們的風景是不斷轉換.
我們的經歷,我們的際遇都不斷變.
風景轉了.
你的心境,你的心靈有沒有跟著轉呢?.
今天是最後的一天.
在敬拜裡面.
我都希望在2023年結束的時候.
跟大家思想一道風景,一種心靈.
我們先讀經文.
經文應該在PowerPoint裡面大家看到.
我們會讀這八節經文.

$^{41}$如果大家都看到的.
都請你為我讀出.
預備,1,2,3.
我們今天會先理解經文.
然後在第二個階段.
會盡心去想一想.
究竟什麼叫一道風景,一種心靈.
最後我會就著這段經文對我自己的提醒.
成為彼此的鼓勵.
我們先理解經文.
剛才經文的一至三節是全卷耶利米書的引言.
耶利米先知經歷了40年的侍奉的履歷.
裡面的一至三節有幾樣東西我們需要知道.
有幾個背景的資料.
第一,耶利米出生在阿拿德.
這是祭司之城.
是利美人所居住的地方.
上帝所選擇的是出自正統的.
耶利米是出自正統的祭司血統.
第二就是耶利米這個名字.
如果你有看舊約聖經的話.
你會發現這個名字很普遍.
大概出現了十次不同的耶利米.
大概就像阿強一樣.
有很多阿強在.
這個名字的意思就是願神尋得他的居所.
是神同在的另一種表達.
耶利米的意思就是神同在.
在場的另一種表達.
然後第三個資料.
我們會發現耶利米先知侍奉了最少有三代的王朝.
第一代就是約西亞.
大概他二十歲初蒙召的時候.
如果你熟悉聖經.
你知道約西亞是一個好君王.
他打算扭轉國家的命運.
除去偶像大興土木.
耶利米出道的時候.
他遇到一個好上司.
正是他發揮的時候.

$^{81}$不過耶利米書沒有記載這段他稍為光輝的日子.
反而他集中記載.
約西亞的兩個兒子.
約亞勁和西底加.
這兩個君王是國家最敗壞的時候.
上帝就要在這段日子宣告.
以色列國將要滅亡.
他要人民接受一個不能夠接受的信息.
正是耶利米書所記載大部分的信息.
去到三十七章裡面有兩句更加這樣說.
他說當時的人民百姓全部都不聽耶和華籍耶利米先知所說的話.
即是過去四十年他的侍奉都沒有人聽.
這個正是耶利米書所記載的一個主要內容.
然後我們看到另外一個撤婚的時段.
就是十一年五月國家戰敗.
耶利米又要成為一個被擄的人.
如果我們概括去看耶利米的人生.
二十歲到六十歲.
侍奉了四十年.
由一個好的君王.
打算大興土木.
即是一展拳腳.
到很快轉為兩位壞的君王.
他需要向人民宣告一個沉重的信息.
然後國家戰敗被擄.
他要向人民說你們要用心.
七十年而已忍一下.
這樣就回歸了.
不同的環境令到這位先知所宣告的都變得不一樣.
如果我們看耶利米書.
你會發現常有一句說話成為書卷的分割.
就是耶和華的話臨到耶利米.
每一次上帝出聲.
這位先知的人生也要作新的改變.
配合上帝做新的事.
一至三節是整本書的引言.
第四至八節是回到他第一次蒙召.
即是二十歲的時候.
神怎樣第一次呼召他.
如果我們熟悉聖經.

$^{121}$就是神每一次向人呼召.
都是很轟烈的.
耶利米也不例外.
神向耶利米呼召.
跟他說了幾句說話.
是甚麼.
神做他的.
神曉得的.
神分別他為聖.
神派他的.
即是說他不能夠擁有自己的人生.
他從母福出來開始.
他已經被分別出來.
去做神的事.
對於一個二十歲的年輕人.
他聽見這麼沉重的呼召.
他就好像很多先知一樣.
就馬上撒手.
說我不行的.
耶利米推搪的原因是甚麼.
我年幼.
我未有這樣的資歷.
未有這樣的經歷.
去面對一個這麼大的呼召.
但如果我們看得夠多.
你就知道.
所有合理的理由.
在上帝的呼召裡面.
都不成理由.
摩西說我絕口不說.
耶利米說我年幼.
所有的理由合理的.
在人間裡面是正確的.
但對上帝來說.
這些推搪的.
都不構成理由.
所以七至八節.
神再次跟他說.
是因為我猜你.
我吩咐你.

$^{161}$耶利米擔心的就是.
神給他的東西.
他做不來.
搞不定.
神怎樣跟他說.
我就是根本給了一樣.
你一生都做不來的東西.
給你做.
你一生都說不服那班人.
我呼召你.
就去做一件.
你一生都不能夠完成的事.
這個是耶利米的呼召.
神不是要先知.
為他做些甚麼.
不是神有些東西搞不定.
耶利米你好像厲害一點.
交給你幫忙.
不是.
從撒姆爾到耶利米.
我們都清楚一件事.
上帝是呼召一個聽命的人.
遠多於一個幫他完成.
某件事的人.
聽命是最重要的.
不是那件事能不能夠做到.
讓你有甚麼成就感.
呼召的本意是.
你願不願意服在上帝面前.
哪管你做不到.
你都願意聽.
這個是呼召的本意.
上帝能夠用得到的人.
這個就是最成功.
就是這麼簡單.
經文看完了.
我嘗試為這段經文做一個小結.
然後我們進入思考.
究竟甚麼是風景.
甚麼是心靈.

$^{201}$這段經文提醒我們一些事情.
第一就是.
每當耶和華的話臨到耶利米.
他人生就要轉一個場景.
他要面對一個新的局面.
他要用一種新的心態.
去應對他周圍的事情.
他要很敏銳上帝的話.
這是幾節的經文提醒我們.
第二個就是.
耶利米大部分的人生.
都在人間看來是沉重和失敗.
甚至連他自己都跟神說.
可不可以不要這樣下去.
這些生活太艱難.
但他自己又回答自己.
如果我不做神的事.
我心就好像被火燒一樣.
他為在兩難之間.
是這個先知人生所經歷.
然後還有一樣.
在他大部分的侍奉日子.
他都是勸服當時的人民.
請你們轉換一個新的心靈.
我們現在是被擄的.
我們是過艱難的日子.
他要說服周圍的人.
去接受這種環境.
這是大概整卷耶利米書給我們的提醒.
經文講完了.
我們又嘗試想一想.
在這麼劇烈的世代裡.
這麼多環境的轉變裡.
我們又在耶利米先知裡.
學到什麼呢.
首先來說一下什麼叫風景.
一道風景.
正如我們開頭所說.
人生就像坐在一部列車.
眼見的風景,經歷,際遇,友誼.

$^{241}$不斷地改變.
除了這個時代的巨輪改變.
就好像耶利米先知.
經歷,戰敗,被擄.
時代的改變.
其實在人生裡面.
你會否發現還有另外幾種改變.
是常在我們中間出現的.
第一就是順其自然的成長.
古語有云三十而立.
心理學也說我們有嬰兒期,青春期.
電視也說人生有多少個十年.
你可能還有很多個十年.
我就倒數剩下多少個十年.
我們的人生是不斷地改變.
這是自然的成長.
由升學提醒我們.
到轉坐兩元車又提醒我們.
我們已經不同了.
這是客觀的真實.
除了順其自然的改變之外.
現在的世代也鼓勵我們做人生規劃.
你也可以規劃一下自己.
在不同的年紀,不同的階段去做不同的事.
在學的時候有老師教我們人生規劃.
到了差不多退休.
我們也計算一下.
有多少老本可以繼續活下去.
還要計算自己有多長命.
這條數很難計算.
究竟長命好一點還是怎樣.
不過真正讓我們經歷到人生的轉變.
不是自然的改變.
也不是人生的規劃.
而是一些突如其來的經歷.
例如中了.
不論你是中了重病.
又或者是中了頭獎.
這些你掌握不到的事情.
它會將我們的人生一分為二.

$^{281}$以前沒有病沒有痛.
就吃吃喝喝.
大部分男士都不會去身體檢查.
誰知道檢查了什麼.
我以後怎樣生活.
我們是這樣的.
所以各位太太請你體諒.
我們一檢查就差不多玩完了.
人生就分割了.
以後想吃的全部都吃不到.
人生變得很悲哀.
我們是這樣的.
真正的改變不是那些順其自然.
而是我們掌握不到.
頂尖我想說的第一件事.
關於風景的轉變.
就是不論你喜歡不喜歡.
主動或者被動.
有心或者無意.
這些場景是不斷轉換的.
有時是不知不覺.
有時這些轉換是叫我們措手不及.
所以當我去想這個情況的時候.
人生的風景不斷轉變.
你需要問自己兩個問題.
第一 你究竟身在何處.
你周圍的環境是怎樣的.
你和我有沒有認真去看一看.
我們現身處身的環境.
不是幻想那種.
不是回憶那種.
眼前這種實況是怎樣的.
第二 你處身在這種環境裡面.
你還有什麼本領是用得著的.
有沒有一些東西已經過時了.
不適合用的.
大家還會不會用那些按鍵的電話.
電話沒有壞的 還可以按的.
但你會不會用的.
不會用的 為什麼.

$^{321}$我家裡的MacBook全新的.
我保存得很好.
但我都要賣掉它.
為什麼.
它追不到上網.
它每一次更新都搞了半個小時.
它不是有問題.
但大環境轉變了.
它不能夠再用.
這就是第一 我想和大家去想一想.
我們的風景.
我們處身的景況.
有些人與事.
是不是已經不同了.
而你經歷了不同的皇朝.
他經歷興衰.
他很清楚每個階段要配合上帝做什麼.
這是我們第一樣要發現.
看清楚周圍的環境.
究竟我們置身在什麼情況裡面.
如果你發現環境改變了.
接下來我要問第二個問題.
我們的心靈追不追得上.
或者是否能夠配合到.
這種不太心甘情願的環境.
這是重要的事情.
以往我們會用人生規劃.
人生上半場下半場去理解.
風景和心靈的變化.
不過當牧羊的日子內.
自己的年紀也成長.
我發現這個道理很難應用在真實的生命裡面.
我發現正如剛才所說.
真正令我轉變的不是我預計裡面的.
是完全出乎我意料的.
我發現當我再進入下半場的時候.
我已經不是踢同一個球.
我不是與同一班隊友.
對付同一批的對手.
甚至整個規矩都變了.

$^{361}$我不是可以輕易帶上半場的經驗.
去進入我的下半場.
是兩種的理解.
很多的改變不是輕易就這樣跨越得到.
問問前輩拿經驗.
是不可以的.
我的經驗是不適合下一代去用的.
我傳授不了很多東西給他去跟著做.
原來是不可以的.
人生未必是上下半場之分.
舉幾個例子.
假如你結了婚.
兩個人生活.
平時兩個人上班.
下班最多買盒兩sung 飯回去吃.
現在太太有嬰兒.
準備生多一個.
你會怎樣做?.
買盒三sung 飯?.
加多一個菜?.
加雙筷?.
我想不是吧.
整個人生都不同了.
前幾天我很早出門.
坐電梯碰見一位婆婆.
她回家.
在電梯口跟她聊兩句.
見她早上八點定著蒸飯回家.
她說兩句.
老公走了.
現在一盒飯三餐都吃那盒.
我猜少了一個人.
不是少了雙筷那麼簡單.
整個心靈都不同了.
就算這句說話用在職場.
都很難湊巧.
以往我們做下屬的時候.
最快樂就是吃午飯.
和一起說上司是非.
人生最滿足就是這些事情.

$^{401}$但當你做了上司的時候.
就沒有人和你吃飯.
你做了上司不是純粹加薪那麼簡單.
你將要承受的是你難以估計的事情.
你將要面對的就是沒有朋友.
你以往說人說得那麼興奮.
你不要升職了.
我保守你不要升職.
你會有報應的.
我想說是截然不同的兩回事.
不是我做下屬的經驗.
我跑前線懂得去找客.
就等於我可以應付很多複雜的人際關係.
特別我處身於中層的時候.
上面那個不長進.
下面那個不聽話.
這些是我以前做跑數的時候.
沒辦法用到的知識.
完全兩回事.
就算說自己身體也是.
年輕的時候你看醫生.
你期望醫生說什麼.
沒事了好起來了.
但到了某個年紀.
你去見醫生.
你最期望醫生說什麼.
沒有惡化.
沒有惡化得那麼快.
是兩種思考.
兩個人生的片段.
頂智文我想說.
我們的環境的轉變.
我們有沒有跟隨著這些客觀的事情.
去轉換我們的心靈呢.
我們的信仰.
我們的耶穌基督.
本來就是轉換心靈的高手.
聖經裡面常提醒我們.
舊的要變成新的.
你是新造的人.

$^{441}$耶穌經常將我們的人生一分為二.
我發現很多頂智妹或者朋友.
基本上卡在一個位置.
不進不退.
就是他不能夠接受眼前的環境改變之餘.
他也從來沒有想過調教自己的心靈.
這兩樣東西令到問題就出來了.
總覺得上帝欠了我們.
為什麼是這樣.
為什麼不給我以前的東西.
就好像出埃及那群以色列民.
去到曠野抱怨.
他幻想什麼.
我寧願像以前圍爐的時候.
吃回那些東西.
我心想你圍爐哪有那麼多東西吃.
全部都是幻想出來的.
你要抱怨上帝.
你可以塑造以前多美好都可以.
你講到天下無敵都可以.
你不喜歡現在的工作.
你可以幻想以前老闆對你好好的.
他一天都罵我三次.
現在罵四次.
你可以的.
幻想而已.
為什麼做不到.
神要耶利米向人民宣告.
請你睜開雙眼.
亡國的事實.
但不等於我丟棄你.
我希望每一個屬神的人.
我們都睜開雙眼.
望見眼前的景況.
這些是真的.
上帝要我們用新的心靈.
去面對這些事.
頂智梅是真的.
你上一次配眼鏡是什麼時候.
經過眼鏡店看到它八折.

$^{481}$很划算 先配兩副.
應該不是.
是當你發現視力有點模糊.
然後你要配合適的眼鏡.
讓你看得清.
視力模糊.
就算你閉上眼再實.
再努力.
你都是看不清.
你需要換一個客觀的配置.
讓你看得清.
就是這麼簡單.
不是好壞之分.
不是努力不努力的問題.
而是我們客觀的事實.
眼睛改變了.
變差了.
我們就要主動一點.
去配一副配合你現況的眼鏡.
去看周圍的事.
真的.
我們的心靈.
有沒有隨著我們的人生變化.
社會的大氣候.
去求問主.
我應該怎樣走下去.
我在過去的裡面.
學懂了什麼事情.
有什麼暫時我做不到.
我需要放出來.
你今天身在何處.
你有什麼東西仍然用得著.
你是需要放手.
是今天講到給我和你的提醒.
人生很多事情是轉變的.
不是簡單用上一個階段的經驗.
就可以帶入下半場.
我們趁今年未過.
還有一整天.
你嘗試定一定神下來.

$^{521}$去想一想.
今年你除了看了一場煙花之外.
你還看見過什麼事情.
我們有沒有睜開雙眼.
接受這些現實.
然後再問神.
在這樣的氣候.
這樣的環境.
這樣的情況裡.
我可以做什麼.
有什麼我需要暫時放下.
然後配合著你.
去走新一里路.
最後講一段經文.
對我自己的提醒和經歷.
回應小孩這個主題.
我有一對外甥.
我妹妹和妹夫生了一對外甥.
大概一個五年級.
一個中一.
上年十月的時候.
移民到英國.
其實過去這麼多年.
主要都是由我和太太.
去照料這對兒甥.
因為我妹妹和妹夫要輪班.
她住在我家附近.
所以過去十年.
都是我和太太去照顧她上學放學.
如果我時間許可.
不行就由外甥照顧.
所有功課都是由我負責.
十年就是這樣過.
當我知道她移民的消息.
對我來說是極沉重的事情.
因為很短時間我才接收到.
今年五月.
我和太太飛到英國探望這對兒甥.
住了兩個星期.
去英國沒有安排任何旅遊行程.

$^{561}$也沒有打算探望弟兄姊妹.
我只是在十四天裡.
每一天帶他們上學.
然後我和太太就坐車到攤位買菜.
回來接他們放學.
然後煮飯給他們吃.
這是我過去十年最快樂的事情.
大概十四天過了.
帶他們一家去曼城玩了三天.
臨到尾分手.
他們坐火車回到他們住的地方.
我和太太就坐火車去機場.
坐了半個小時.
我妹急叫我.
他們回到家沒多久.
她就說.
哥 你的外甥女哭到快崩潰.
她完全失控.
因為我上不了網.
網絡比較差.
不能夠視像.
她錄了一段音給我聽.
我太太到今天都不敢聽.
我外甥女五年級.
哭著這樣問.
她說舅父.
為什麼你們不移民過來.
她說其實在這裡做包裝都可以好好生活.
我知道我外甥女知道我有強迫症.
做包裝對我來說是很適合的.
我都說過.
我去到超級市場見到罐頭歪了.
我都會轉一轉.
我是這些人.
她哭著這樣問.
哭了很久.
我回答她留言.
她說舅母怕冷.
因為英國很冷.
我外甥女說.

$^{601}$我存錢買件外套給她.
為什麼你們不過來.
哭了三天.
外甥仔大一點 中學.
她不懂得哭.
但她媽媽知道她說不出聲.
因為外甥仔更加親我.
直至幾天之後.
她才大爆發.
我回到香港再跟她視像.
談了一段時間.
外甥女問我的問題.
為什麼我不移民過去.
這個問題我留在心裡很久.
我太太問我.
其實過去十幾年.
我和你在香港做過什麼.
究竟我們所愛又愛我們的人.
他問我一個問題.
我怎樣回答這個問題呢.
究竟我知不知道自己身在何處.
究竟我的心靈是否配合眼前的環境呢.
直至我看耶利米蘇預備這篇講章的時候.
神回答我這個問題.
環境是改變了.
他們兩個有離開香港的原因.
但你和太太也有留在這裡的理由.
我只可以跟外甥女說.
舅父不過來了.
舅父仍然在香港有一些很重要的事情要做.
你多點回來探望我吧.
頂至在這段經歷裡.
我自己也要學習放手.
對我來說.
這兩個小朋友是我人生過去十年快樂的支柱.
上帝讓他們離開一定有原因.
但對我來說.
我到今天這一刻還未適應得到.
我見我所有家人都適應得到.
但我不知道為什麼我適應不到.

$^{641}$我沒有一個支撐我人生最快樂的支柱.
上帝將它挪走了.
以前我覺得送上學接放學.
預備兩餐是理所當然.
現在都沒有了.
我要學會的是.
我要過一些不常有對方的生活.
這句話我不是寫給我的外甥女.
我是寫給我自己.
我要很努力去學一些不常有對方同在的生活.
不過我又得到什麼呢?.
我發現我現在再不需要為他們默書操心.
我不需要在假期裡幫他們想節目.
他們生日我不需要去買禮物.
我不需要為家人預備造節的飯.
我甚至不需要去老人院探望離世的爸爸.
我可以很專心侍奉耶和華.
這一年的侍奉是我這麼多年來從來沒有試過這麼專心.
上帝給了很多又大又難的事.
想到我的侍奉裡.
甚至有朋友開玩笑地問我.
為什麼你還沒有抑鬱?.
我說這段時間比較忙.
希望遲些有時間.
抑鬱也很奢侈的.
在今年裡.
上帝釋放了我很大量的自由度.
甚至可以專心去走另外一條路.
上帝很奇妙地讓我和John Poon Sir聊天.
可能他們都知道我沒事做會耍廢.
他安排了一些弟兄姊妹.
在過去一年裡給我機會和他們一起同行.
我和他們交換了很多流眼淚的故事.
我們亦交換了很多經歷神難處的事.
老實說我不能夠幫你們解決你們所面對的問題.
不過我自己這樣去評估.
我們能夠在一起的力量比單獨更大.
我在你們身上看到聖靈無聲的工作.
你們站得直立在身.
繼續走人生的路.

$^{681}$對我來說是最大的鼓勵.
與其說是我教導了什麼.
不如我們彼此同行.
我們發現原來仍然有路可行.
不需要卡住在那些你放不下的事跡.
眼前還有很多東西值得我和你去做.
弟兄姊妹是真的.
我們能夠放下的事一定是重要的事.
如果那些事常常令你猶豫不決不能夠放下.
那份量是很重要的.
我和你掙扎了很久不能夠釋懷.
甚至這些事情是構成你成為一個什麼人的事.
你不是希望國家興起.
但他也要放下.
神要懲罰.
他要面對.
如果那些事這麼輕鬆可以放下的話.
那些事早就不見了.
忘記了.
我們掙扎點就是不能夠放下.
不過我們要分清楚.
放下不是放棄.
或者是放在一邊.
神可能在某一天又要我拿起來.
又要重新開始.
當時候到了.
我們就可以再拿回這些曾經構成我們生命重要的事情的那部分.
在這個年頭裡面.
我發現原來我還有一些用法是我不知道的.
我是一個很早起床的人.
我四點多五點就在街上.
不是在家裡.
在街上.
你當我靈修也好.
神運也好.
拉筋也好.
總之我一定在街上.
通常在這個時間我就開始去工作.
我是一個絕對的神型人.
但原來我將自己交出來.

$^{721}$我不想我可以怎樣計劃.
讓上帝隨意使用.
原來神跟我說.
其實你只是一個夜奔芬.
晚上你也可以.
有些你自己的用法是你自己不知道的.
你將自己交出來給上帝.
上帝的用法可以很新奇.
是可以用得很奇妙.
總之就是.
藉著放下.
藉著這兩個我痛惜的外生.
神教曉我很多道理.
我希望我的決定都讓這兩個外生知道.
因為我走的路.
是有一位更值得我放下他們的神.
在我前面.
更值得我去找我自己舒適的生活.
在我前面.
因為這條路是上帝讓我和祂同行的路.
我希望祂望見的信仰.
是一個這樣的信仰.
我們所信的是亞伯拉罕的神.
以撒的神.
瓦國的神.
是一代接一代同樣凝信的神.
這是我很希望傳給我們的下一代.
你今天生在何處.
你還有什麼值得保留.
有沒有什麼值得放下呢.
用這樣去問.
如果你今天仍然有小朋友纏著你.
讓你失去很多meantime.
我聽說在人生最值得回憶的.
全部都是這些時間.
當他升到中學的時候.
那段日子你是不想記得的.
他還要扭要抱要問你.
要撒你的日子.
請你好好珍惜.

$^{761}$如果你的兒女又長大了.
不在身邊.
你又多了很多用不完的meantime.
多到要kill time.
不知道該怎麼辦.
你又好好珍惜這些自由.
你身邊還有很多朋友.
很多老友等著和你聊天.
這些時間你要用好.
如果你有工作壓力.
你仍然為上班而憂慮.
能夠賺錢養家.
是一件很微妙的事.
能夠有收入奉獻給教會.
讓教會繼續在不同的地方.
如登台一樣.
是值得榮耀的事.
不是每一個都有機會賺錢.
我都試過失業很一段時間.
那種徬徨我仍然記到今天.
都記得.
如果因不同的原因.
不需要上班.
很清閒.
你儘管去看看鳥語花香.
反正你上班的時候.
一直想下班.
現在長放.
你是不是很開心.
你都要享受你曾經很嚮往的日子.
不能夠反過來.
享受這些日子.
年輕力壯的時候.
禁意而行.
得罪人多.
得罪上帝更多.
恃著自己有無窮的體力.
以為自己不會病.
你去感恩.
你能夠匆匆忙忙的日子.

$^{801}$是上帝賜的.
但當你患病的時候.
你開始認真思索生命.
你開始去計算.
原來日子是可以很短.
你更認真去想.
什麼叫人生.
一道風景.
一種心靈.
今年要過去了.
你除了看了一場煙花.
吃過幾餐自助餐.
嘻嘻哈哈玩過幾輪派對之外.
你還有什麼剩下.
你學過什麼.
你有什麼本領.
可以帶入新一里路.
又或者我問.
你有沒有一些卡住你的東西.
今天仍然放不下.
如果我也學努力放下.
不如我們也一起.
反正前面還有新的風景.
等著我和你.
還有新的朋友在面前.
讓我們一起去走2024年新一里路.
我們一起祈禱.
神說前面的風景是陌生.
甚至令我們不安.
不過願你對耶利米所說的那句.
今天對我們每一個都說.
你不要懼怕.
因為我與你同在.
我要拯救你.
這是耶和華說的.
阿們.
\newpage

\allsectionsfont{\centering}

\setlength\parindent{0pt}
\setlength{\columnsep}{1.25em}
\setlength{\parfillskip}{0pt}
\setlength{\tabcolsep}{1em}
\raggedbottom

\pagenumbering{gobble}


\newfontfamily\leftfont[Path=../fonts/fell_french_canon/, Ligatures=TeX, ItalicFont=IMFeFCit29C.otf, BoldFont=AveriaLibre-Bold.ttf]{IMFeFCrm29C.otf}
\newfontfamily\leftcitationfont[Path=../fonts/frankruehl/]{FrankRuehlCLM-Medium.ttf}
\newfontfamily\centerfont[Path=../fonts/garamond/, Ligatures=TeX, ItalicFont=EBGaramond-SemiBoldItalic.ttf]{EBGaramond-SemiBold.ttf}
\newfontfamily\rightfont[Path=../fonts/averia/, Ligatures=TeX, ItalicFont=AveriaLibre-RegularItalic.ttf, BoldFont=AveriaLibre-Bold.ttf, BoldItalicFont=AveriaLibre-BoldItalic.ttf]{AveriaLibre-Light.ttf}
\newfontfamily\rightcitationfont[Path=../fonts/rashi/]{Mekorot-Rashi.ttf}
\definecolor{hcolor}{HTML}{D3230C}
\definecolor{rcolor}{HTML}{D36F0C}
\newcommand{\chfont}[1]{\centerfont{\huge\textcolor{hcolor}{#1}}}
\newcommand{\leftcitation}[1]{\leftcitationfont{\Large\textcolor{hcolor}{#1}}}
\newcommand{\rightcitation}[1]{\rightcitationfont{\normalsize\textcolor{rcolor}{#1}}}
\newfontfamily\flowerfont[Path=../fonts/fell_flowers/]{IMFeFlow2.otf}

\begin{sloppypar}

\chapter*{\chfont{編按結語}}

\columnratio{0.5,0.5}\begin{paracol}{2}

\fontsize{11}{13}\leftfont \Large \leftcitation{א} \leftfont 余少好文.宏志博覽群書而不忘.善存藏經籍文獻備後時之用。\leftcitation{ב} \leftfont 歸主年時.受友所薦.聞道網海.\switchcolumn\fontsize{11}{13}\rightfont \Large \leftcitation{ח} \rightfont 有見粵道之危.國之封講道千言亦將就至.急之何則為?\leftcitation{ט} \rightfont 嘗聞猶太者之傳承.在其力守口述之

\end{paracol}


\columnratio{0.32,0.32,0.32}\begin{paracol}{3}

\fontsize{11}{13}\leftfont \Large 尤以吳約翰遜者 \switchcolumn[2]\fontsize{11}{13}\rightfont \Large 統.以煉千載不

\end{paracol}

\columnratio{0.32,0.32,0.32}
\begin{paracol}{3}\fontsize{11}{13}\leftfont \Large 為重.其載上之粵語講道緩緩入耳.收之藏其音頻.善妥整存.反復而嚼.受益無窮。\leftcitation{ג} \leftfont 我城我國既限.歷一四一九之不測.肺疫延年.信徒靈長屢受圍創.神州燈臺數盡指日可待.粵道之求與日俱增。\leftcitation{ד} \leftfont 觀乎社、經、法、媒、言、信、網之地.愈趨受鋤.自翔不果.授受壓力.粵道聖言亦愈漸艱難。\leftcitation{ה} \leftfont 況崇基例乎.學苑講道屢逆權勢者.其言末強受壓.舊章盡刪以存其身。講道釋數失傳.徒嘆奈何。

\switchcolumn

\fontsize{11}{13}\centerfont 
\begin{tikzpicture}
    \node (0,0) [xshift=-0.10cm, yshift=-1.0cm, opacity=0.10]{\includegraphics[width=0.30\textwidth]{../ot_frontcover.png}} ;
    \node (0,0) [xshift=+0.20cm, yshift=+2.0cm, opacity=0.10]{\includegraphics[width=0.20\textwidth]{../christ_on_cross.png}} ;
\end{tikzpicture}
\Large 

\leftcitation{ס} \centerfont 詩百又廿七載:
\leftcitation{ע} \centerfont 非耶和華建屋宇.則匠人之經營徒.
\leftcitation{פ} \centerfont 非耶和華衛城邑.則守者之儆醒徒.
\leftcitation{צ} \centerfont 余獻是卷予華人社區.願為福音流通之器.願獻斯微材為祭榮耀上帝.
\leftcitation{ק} \centerfont 阿門

\switchcolumn

\fontsize{11}{13}\rightfont \Large 滅.時越次聖殿期及當今。\leftcitation{י} \rightfont 猶太者力廣納之.筆錄以卷軸.便以傳、閱、頌、攜、守、鎖、抄、譯、釋、編,得書塔木德、密示拿等經傳.家喻戶曉.傳流若芳。\leftcitation{כ} \rightfont 猶太者文以載道.傳其口述.今我輩粵道之傳應當作如是.遂力行粵音識辨之法.載言載道.以盡忠傳粵道以待興。\leftcitation{ל} \rightfont 蒙下賜恩惠.無畏海量字音文書.既馭上帝之道.今廣及粵語講道.重駛編程之技.匯導粵音遂字稿.重塑講道現場.以傚猶太卷軸之舉便以傳流。\leftcitation{מ} \rightfont 是卷乃粵音口述傳之屬.莫通華文白話之語.

\end{paracol}

\columnratio{0.5,0.5}
\begin{paracol}{2}\fontsize{11}{13}\leftfont \Large \leftcitation{ו} \leftfont 斯殺一違儆百逆.既禁壓之.我輩聞風無奈.在所難免。\leftcitation{ז} \leftfont 另有異人例乎.以版權之名.脅網絡頻道之舉.同授礙予粵道之存流。

\switchcolumn

\fontsize{11}{13}\rightfont \Large 惟待後繼來者之傚.以譯釋傳之於神州華文地。\leftcitation{נ} \rightfont 今能排程驅馭圖靈以編彙文檔,其碼長共數千千亦無逢大礙.全蒙上帝保守。

\end{paracol}



\columnratio{1}\begin{paracol}{1}

\fontsize{11}{13}\rightfont \Large
~~~~~~~~~~~~~~~~~~~~~~~~~~~~~~~~~~~~~~~~~~~~~~~~~~~~~~~~~~~~~~~~~~~~~~~~~~~~~~~\leftcitation{ר} \rightfont 二零二三年二月一日

~~~~~~~~~~~~~~~~~~~~~~~~~~~~~~~~~~~~~~~~~~~~~~~~~~~~~~~~~~~~~~~~~~~~~~~~~~~~~~~\leftcitation{ש} \rightfont 米迦勒

~~~~~~~~~~~~~~~~~~~~~~~~~~~~~~~~~~~~~~~~~~~~~~~~~~~~~~~~~~~~~~~~~~~~~~~~~~~~~~~\leftcitation{ת} \rightfont 書於香港

\end{paracol}

\end{sloppypar}
\end{document}
