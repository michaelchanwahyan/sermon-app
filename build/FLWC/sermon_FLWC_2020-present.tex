\documentclass{book}
%\usepackage[letterpaper, portrait, margin=1cm]{geometry}
%\usepackage[letterpaper, bindingoffset=0.2in, left=1in,right=1in,top=.5in,bottom=.5in,footskip=.25in,marginparwidth=5em]{geometry}
\usepackage[letterpaper, left=1in,right=1in,top=.5in,bottom=.5in,footskip=.25in,marginparwidth=1cm]{geometry}
% ---------------------
% mini-table-of-content
% ---------------------
\usepackage{minitoc}
\setcounter{minitocdepth}{1}
\setlength{\mtcindent}{24pt}
\setcounter{secnumdepth}{-2}
%\renewcommand{\mtcfont}{\small\rm}
%\renewcommand{\mtcSfont}{\small\bf}
%\usepackage{setspace}
%\usepackage{tocloft}
%\setlength\cftparskip{-1.2pt}
%\setlength\cftbeforesecskip{1.3pt}
%\setlength\cftaftertoctitleskip{2pt}
%\renewcommand{\cftsecafterpnum}{\hspace*{02.0em}}
%\renewcommand{\cftsubsecafterpnum}{\hspace*{02.0em}}

% ---------------------------
% Chinese Characters Packages
% ---------------------------
\usepackage{fontspec} 
\usepackage{xeCJK}
\setmainfont{Times}
\setCJKmainfont{BiauKai}
\newfontfamily\sblgoodhebrew{SBL BibLit}[Script=Hebrew,Contextuals=Alternate]
\newfontfamily\sblgoodgreek{SBL BibLit}[Script=Greek,Contextuals=Alternate]

\usepackage{ifpdf,cite,algorithmic,url,tikz}
\usepackage[cmex10]{amsmath}

% ---------------------------
% Hebrew Characters Packages
% ---------------------------
\usepackage{polyglossia}
\setmainfont{Times New Roman}

% -------
% General
% -------
\usepackage{multicol}
\usepackage{multirow}
\usepackage{color,colortbl}
\usepackage{xparse}
\usepackage{pbox}
\usepackage{stackengine}
\usepackage{titlesec}% http://ctan.org/pkg/titlesec
\usepackage{tabularx}
\usepackage{xltabular}
\usepackage{titlesec}
\usepackage{makecell}
\newcommand{\sectionbreak}{\clearpage}

\author{
  Editor, Michael Chan\\
  \texttt{michaelchan\_wahyan@yahoo.com.hk}
}
\usepackage{tocloft}

\usepackage{hyperref}
\hypersetup{
    colorlinks=true, % set true if you want colored links
    linktoc   =all , % set to all if you want both sections and subsections linked
    linkcolor =blue, % choose some color if you want links to stand out
}

% ----------
% Afterword
% ----------
\usepackage{marginnote}
\usepackage{sectsty}
\usepackage{ragged2e}
\usepackage{lineno}
\usepackage{xcolor}
\usepackage{paracol}

\begin{document}

\clearpage
%% temporary titles
% command to provide stretchy vertical space in proportion
\newcommand\nbvspace[1][3]{\vspace*{\stretch{#1}}}
% allow some slack to avoid under/overfull boxes
\newcommand\nbstretchyspace{\spaceskip0.5em plus 0.25em minus 0.25em}
% To improve spacing on titlepages
\newcommand{\nbtitlestretch}{\spaceskip0.6em}
\pagestyle{empty}
\begin{center}
\bfseries
\nbvspace[1]
\Huge
{%\nbtitlestretch
\Large
\textbf{粵語講道逐字稿 2020-present \\
       Youtube Channel: 流堂 Flow Church
       }}

\nbvspace[1]

{\large
Editor: Michael\\
\texttt{michaelchan\_wahyan@yahoo.com.hk}
}

\nbvspace[1]

{\large
Revision: \texttt{v1.1}\\
Last Update: \today
}


\vfill
\begin{tikzpicture}
    %remove comment for OT cover%\node (0,0) [opacity=0.03]{\includegraphics[width=15cm]{../bible_out/ot_frontcover.png}} ;
    %remove comment for NT cover%\node (0,0) [opacity=0.03]{\includegraphics[width=15cm]{../bible_out/christ_on_cross.png}} ;
    %remove comment for Bible cover%\node (0,0) [xshift=0.8cm, yshift=+2cm, opacity=0.03]{\includegraphics[width=10cm]{./christ_on_cross.png}} ;
    %remove comment for Bible cover%\node (0,0) [              yshift=-2cm, opacity=0.03]{\includegraphics[width=14cm]{./ot_frontcover.png}} ;
\end{tikzpicture}
\vfill

\end{center}

\newpage

\setcounter{tocdepth}{0}
\dominitoc
\begin{multicols}{3}
\addtocontents{toc}{\protect\hypertarget{toc}{}}
\tableofcontents
\end{multicols}

\large
%\twocolumn

% the color definition syntax is as follow:
% \definecolor{name}{system}{definition}
% example: a mono-channel color can be defined as
%          \definecolor{Gray}{gray}{0.9}
% example: an rgb-3-channel color can be defined as
%          \definecolor{LightCyan}{rgb}{0.88,1,1}
%          \definecolor{pink}{rgb}{0.68,0,0.68}

\definecolor{CUV1LightRed}{rgb}{1,0.75,0.75}     % for CUV1
\definecolor{LZZVLightGray}{rgb}{0.9,0.9,0.9}    % for LZZ
\definecolor{KJVVLightGreen}{rgb}{0.75,1,0.85}   % for KJV
\definecolor{CUV2LightYellow}{rgb}{1,1,0.75}     % for CUV2
\definecolor{CNVVLightBrown}{rgb}{1,0.85,0.7}    % for CNV
\definecolor{NRSVLightBlue}{rgb}{0.75,1,1}       % for NRSV
\definecolor{WENLLightPurple}{rgb}{0.95,0.85,0.9}% for WENL
\definecolor{TCV19PaleGreen}{rgb}{0.85,1,0.95}   % for TCV19
\definecolor{MSGVLightWhite}{rgb}{0.98,0.98,0.98}% for MSGV
\definecolor{NETSLightRed}{rgb}{1,0.75,0.75}     % for NETS
\definecolor{JPS1917LightYellow}{rgb}{1,1,0.75}  % for JPS1917
\definecolor{SBLGNTPaleRed}{rgb}{1,0.85,0.80}    % for SBLGNT

\section{目錄\small{(順時)}}
\label{sec:index_chronic}
{ \scriptsize


\begin{xltabular}{\textwidth}{|p{0.15\textwidth} p{0.6\textwidth}|p{0.07\textwidth} p{0.1\textwidth}|}
\hline
腓利門書 1:1-25 & \hyperref[sec:J3EQacUFDFI]{【網上崇拜】再見…再相見 | 腓利門書1\_1-25 | 20210206 [J3EQacUFDFI]} & 2021-02-06 & \href{https://youtube.com/watch?v=J3EQacUFDFI}{\texttt{ J3EQacUFDFI}} \\
    & \hyperref[sec:GmLFDCSkId4]{【這是最好的時代:給香港基督徒的神學八課】第1講:亂世中才明白甚麼是「基督徒」|20210522 [GmLFDCSkId4]} & 2021-05-22 & \href{https://youtube.com/watch?v=GmLFDCSkId4}{\texttt{ GmLFDCSkId4}} \\
    & \hyperref[sec:OTk7WEa_w50]{【這是最好的時代:給香港基督徒的神學八課】第2講:究竟好消息有多好?|20210619 [OTk7WEa-w50]} & 2021-06-19 & \href{https://youtube.com/watch?v=OTk7WEa-w50}{\texttt{ OTk7WEa-w50}} \\
    & \hyperref[sec:dNWjC8vnhS0]{【這是最好的時代:給香港基督徒的神學八課】第3課: Let’s flow|20210726 [dNWjC8vnhS0]} & 2021-07-26 & \href{https://youtube.com/watch?v=dNWjC8vnhS0}{\texttt{ dNWjC8vnhS0}} \\
    & \hyperref[sec:d_aSxcuQPus]{【這是最好的時代:給香港基督徒的神學八課】第4課:亂世的靈性修持|20210822 [d\_aSxcuQPus]} & 2021-08-22 & \href{https://youtube.com/watch?v=d_aSxcuQPus}{\texttt{ d\_aSxcuQPus}} \\
    & \hyperref[sec:akT8yKiTNTo]{【這是最好的時代:給香港基督徒的神學八課】第5課:一根刺的人|20210918 [akT8yKiTNTo]} & 2021-09-18 & \href{https://youtube.com/watch?v=akT8yKiTNTo}{\texttt{ akT8yKiTNTo}} \\
    & \hyperref[sec:K2_OK28IM68]{【這是最好的時代:給香港基督徒的神學八課】第6課: Passion|20211018 [K2\_OK28IM68]} & 2021-10-18 & \href{https://youtube.com/watch?v=K2_OK28IM68}{\texttt{ K2\_OK28IM68}} \\
    & \hyperref[sec:hq6PGyJ3aBs]{【這是最好的時代:給香港基督徒的神學八課】第7課:今日的「我」推翻昨日的「我」|20211121 [hq6PGyJ3aBs]} & 2021-11-21 & \href{https://youtube.com/watch?v=hq6PGyJ3aBs}{\texttt{ hq6PGyJ3aBs}} \\
    & \hyperref[sec:HS1KRCnzG5o]{【這是最好的時代:給香港基督徒的神學八課】第8課:跟隨耶穌的一百萬個可能|20211218 [HS1KRCnzG5o]} & 2021-12-18 & \href{https://youtube.com/watch?v=HS1KRCnzG5o}{\texttt{ HS1KRCnzG5o}} \\
以西結書 47:1-2-20220122 & \hyperref[sec:g_5XGfcpSSo]{【網上崇拜】聖殿水浸的異象|以西結書47\_1-2|20220122 [g\_5XGfcpSSo]} & 2022-01-22 & \href{https://youtube.com/watch?v=g_5XGfcpSSo}{\texttt{ g\_5XGfcpSSo}} \\
撒母耳記上 13:1-23-20220129 & \hyperref[sec:FLZJqJSyEdc]{【網上崇拜】召命人生|撒母耳記上13\_1-23|20220129 [FLZJqJSyEdc]} & 2022-01-29 & \href{https://youtube.com/watch?v=FLZJqJSyEdc}{\texttt{ FLZJqJSyEdc}} \\
    & \hyperref[sec:vVSfVrNtuKk]{【網上崇拜】虎年生肖道|20220205 [vVSfVrNtuKk]} & 2022-02-05 & \href{https://youtube.com/watch?v=vVSfVrNtuKk}{\texttt{ vVSfVrNtuKk}} \\
約翰福音 21:18-19-20220212 & \hyperref[sec:wntIcXZCGmo]{【網上崇拜】再有一次機會|約翰福音21\_18-19|20220212 [wntIcXZCGmo]} & 2022-02-12 & \href{https://youtube.com/watch?v=wntIcXZCGmo}{\texttt{ wntIcXZCGmo}} \\
彼得前書 1:17-21-20220219 & \hyperref[sec:GPCXXRzSHkw]{【網上聖餐崇拜】而家仲有乜好flow?!|彼得前書1\_17-21|20220219 [GPCXXRzSHkw]} & 2022-02-19 & \href{https://youtube.com/watch?v=GPCXXRzSHkw}{\texttt{ GPCXXRzSHkw}} \\
創世記 37:1-50:26-20220226 & \hyperref[sec:kIOQG9wgqRs]{【網上崇拜】相信一切是最好的安排|創世記37\_1-50\_26|20220226 [kIOQG9wgqRs]} & 2022-02-26 & \href{https://youtube.com/watch?v=kIOQG9wgqRs}{\texttt{ kIOQG9wgqRs}} \\
出埃及記 1:15-22-20220305 & \hyperref[sec:oFrw_raeCu8]{【網上崇拜】兩位懂得甚麼是「越位」的女生|出埃及記1\_15-22|20220305 [oFrw\_raeCu8]} & 2022-03-05 & \href{https://youtube.com/watch?v=oFrw_raeCu8}{\texttt{ oFrw\_raeCu8}} \\
彼得前書 2:18-25-20220312 & \hyperref[sec:YQgThbN8z0Q]{【網上崇拜】基督都愛越位|彼得前書2\_18-25|20220312 [YQgThbN8z0Q]} & 2022-03-12 & \href{https://youtube.com/watch?v=YQgThbN8z0Q}{\texttt{ YQgThbN8z0Q}} \\
羅馬書 15:14-21-20220319 & \hyperref[sec:3bu5V6aPJC0]{【網上聖餐崇拜】越位人生|羅馬書15\_14-21|20220319 [3bu5V6aPJC0]} & 2022-03-19 & \href{https://youtube.com/watch?v=3bu5V6aPJC0}{\texttt{ 3bu5V6aPJC0}} \\
創世記 1:1-5-20220326 & \hyperref[sec:0d9n3K2nnYY]{【網上崇拜】你最閃亮的一刻,是你踏進不安的一剎.|創世記1\_1-5|20220326 [0d9n3K2nnYY]} & 2022-03-26 & \href{https://youtube.com/watch?v=0d9n3K2nnYY}{\texttt{ 0d9n3K2nnYY}} \\
馬太福音 5:20-20220402 & \hyperref[sec:rbtYKzrN9IU]{【網上崇拜】Außerordentlichen 不要尋常|馬太福音5\_20|20220402 [rbtYKzrN9IU]} & 2022-04-02 & \href{https://youtube.com/watch?v=rbtYKzrN9IU}{\texttt{ rbtYKzrN9IU}} \\
使徒行傳 23:1-35-20220409 & \hyperref[sec:9oqAyDoUD6A]{【網上崇拜】害怕危險才是真正的危險|使徒行傳23\_1-35|20220409 [9oqAyDoUD6A]} & 2022-04-09 & \href{https://youtube.com/watch?v=9oqAyDoUD6A}{\texttt{ 9oqAyDoUD6A}} \\
    & \hyperref[sec:9wD1Nl5xbCs]{【網上受難節聚會】|反思疫症下的基督受難|20220415 [9wD1Nl5xbCs]} & 2022-04-15 & \href{https://youtube.com/watch?v=9wD1Nl5xbCs}{\texttt{ 9wD1Nl5xbCs}} \\
彼得前書 4:1-6-20220416 & \hyperref[sec:LSvYW2YLyv0]{【網上聖餐崇拜】得受苦一個選擇,但你仍有得揀|彼得前書4\_1-6|20220416 [LSvYW2YLyv0]} & 2022-04-16 & \href{https://youtube.com/watch?v=LSvYW2YLyv0}{\texttt{ LSvYW2YLyv0}} \\
哈巴谷書 1:5-20220423 & \hyperref[sec:UT74cFcV7PU]{【網上崇拜】上帝叫我厚多士(已FC)|哈巴谷書1\_5|20220423 [UT74cFcV7PU]} & 2022-04-23 & \href{https://youtube.com/watch?v=UT74cFcV7PU}{\texttt{ UT74cFcV7PU}} \\
  11:1-44-20220430 & \hyperref[sec:7_6qfdylGjE]{【網上崇拜】突破「越位」|約11\_1-44|20220430 [7\_6qfdylGjE]} & 2022-04-30 & \href{https://youtube.com/watch?v=7_6qfdylGjE}{\texttt{ 7\_6qfdylGjE}} \\
路加福音 15:11-31-20220507 & \hyperref[sec:b7gyPC12_AM]{【網上崇拜】首先學習如何被擁抱|路加福音15\_11-31|20220507 [b7gyPC12\_AM]} & 2022-05-07 & \href{https://youtube.com/watch?v=b7gyPC12_AM}{\texttt{ b7gyPC12\_AM}} \\
馬可福音 10:13-16-20220514 & \hyperref[sec:97SC38c6sqY]{【網上崇拜】最終的擁抱|馬可福音10\_13-16|20220514 [97SC38c6sqY]} & 2022-05-14 & \href{https://youtube.com/watch?v=97SC38c6sqY}{\texttt{ 97SC38c6sqY}} \\
彼得前書 4:7-11-20220521 & \hyperref[sec:2t83SY_sddQ]{【網上聖餐崇拜】擁抱的規矩…MM7|彼得前書4\_7-11|20220521 [2t83SY\_sddQ]} & 2022-05-21 & \href{https://youtube.com/watch?v=2t83SY_sddQ}{\texttt{ 2t83SY\_sddQ}} \\
羅馬書 14:1-15:13-20220528 & \hyperref[sec:OJx_AQpZJpY]{【網上崇拜】擁彼此接納──飯就一定要食,一齊食開心D ?|羅馬書14\_1-15\_13|20220528 [OJx-AQpZJpY]} & 2022-05-28 & \href{https://youtube.com/watch?v=OJx-AQpZJpY}{\texttt{ OJx-AQpZJpY}} \\
路加福音 15:11-31-20220604 & \hyperref[sec:NkL6a2IH8uY]{【網上崇拜】然後去學習擁抱一切|路加福音15\_11-31|20220604 [NkL6a2IH8uY]} & 2022-06-04 & \href{https://youtube.com/watch?v=NkL6a2IH8uY}{\texttt{ NkL6a2IH8uY}} \\
約翰福音 13:1-38-20220611 & \hyperref[sec:SLTU9ILLofg]{【網上崇拜】攬到底|約翰福音13\_1-38|20220611 [SLTU9ILLofg]} & 2022-06-11 & \href{https://youtube.com/watch?v=SLTU9ILLofg}{\texttt{ SLTU9ILLofg}} \\
詩篇 23:1-6-20220618 & \hyperref[sec:rvuxp0Bes90]{【網上聖餐崇拜】小心地滑…擁抱唔到|詩篇23\_1-6|20220618 [rvuxp0Bes90]} & 2022-06-18 & \href{https://youtube.com/watch?v=rvuxp0Bes90}{\texttt{ rvuxp0Bes90}} \\
路加福音 15:1-7-20220625 & \hyperref[sec:zMmzg_ext8I]{【網上崇拜】乜ye ye 問題一律建議抱緊處理|路加福音15\_1-7|20220625 [zMmzg\_ext8I]} & 2022-06-25 & \href{https://youtube.com/watch?v=zMmzg_ext8I}{\texttt{ zMmzg\_ext8I}} \\
提摩太後書 4:1-2-20220702 & \hyperref[sec:o_rYmsKLSrs]{【網上崇拜】仍未忘跟你約定假如沒有死|提摩太後書4\_1-2|20220702 [o\_rYmsKLSrs]} & 2022-07-02 & \href{https://youtube.com/watch?v=o_rYmsKLSrs}{\texttt{ o\_rYmsKLSrs}} \\
使徒行傳 20:17-38-20220709 & \hyperref[sec:XDQJvaySA3k]{【網上崇拜】就算會與你分離......|使徒行傳20\_17-38|20220709 [XDQJvaySA3k]} & 2022-07-09 & \href{https://youtube.com/watch?v=XDQJvaySA3k}{\texttt{ XDQJvaySA3k}} \\
路加福音 11:29-32-20220716 & \hyperref[sec:CzS_E_B5XMA]{【網上聖餐崇拜】誰可唱最後的信仰...|路加福音11\_29-32|20220716 [CzS\_E-B5XMA]} & 2022-07-16 & \href{https://youtube.com/watch?v=CzS_E-B5XMA}{\texttt{ CzS\_E-B5XMA}} \\
但以理書 3:1-30-20220723 & \hyperref[sec:RWAsMtjQmZ4]{【網上崇拜】難道我們未夠難?|但以理書3\_1-30|20220723 [RWAsMtjQmZ4]} & 2022-07-23 & \href{https://youtube.com/watch?v=RWAsMtjQmZ4}{\texttt{ RWAsMtjQmZ4}} \\
提摩太後書 4:6-8-20220730 & \hyperref[sec:muF9XhDEVwY]{【網上崇拜】兩鬢斑白都可認得你|提摩太後書4\_6-8|20220730 [muF9XhDEVwY]} & 2022-07-30 & \href{https://youtube.com/watch?v=muF9XhDEVwY}{\texttt{ muF9XhDEVwY}} \\
加拉太書 1:1-12-20220806 & \hyperref[sec:0mi_NsvpcRc]{【網上崇拜】自由是最美好的禮物|加拉太書1\_1-12|20220806 [0mi\_NsvpcRc]} & 2022-08-06 & \href{https://youtube.com/watch?v=0mi_NsvpcRc}{\texttt{ 0mi\_NsvpcRc}} \\
申命記尼希米記 30:1-4 & \hyperref[sec:hkGSf0_Eoow]{【網上崇拜】同是天涯流落人|申命記30\_1-4;尼希米記1\_8-9|20220813 [hkGSf0-Eoow]} & 2022-08-13 & \href{https://youtube.com/watch?v=hkGSf0-Eoow}{\texttt{ hkGSf0-Eoow}} \\
馬太福音 16:13-23-20220820 & \hyperref[sec:sQEhDyhKFnE]{【網上聖餐崇拜】跟最難約定的人講約定|馬太福音16\_13-23|20220820 [sQEhDyhKFnE]} & 2022-08-20 & \href{https://youtube.com/watch?v=sQEhDyhKFnE}{\texttt{ sQEhDyhKFnE}} \\
腓利門書 1:1-25-20220827 & \hyperref[sec:pF3HM3BllyQ]{【網上崇拜】我地|腓利門書1\_1-25|20220827 [pF3HM3BllyQ]} & 2022-08-27 & \href{https://youtube.com/watch?v=pF3HM3BllyQ}{\texttt{ pF3HM3BllyQ}} \\
傳道書 3:11-20220903 & \hyperref[sec:1uGH28VFFVk]{【網上崇拜】上帝的美學|傳道書3\_11|20220903 [1uGH28VFFVk]} & 2022-09-03 & \href{https://youtube.com/watch?v=1uGH28VFFVk}{\texttt{ 1uGH28VFFVk}} \\
希伯來書 11:8-22-20220910 & \hyperref[sec:oHgankweQ8E]{【網上崇拜】月是故鄉明|希伯來書11\_8-22|20220910 [oHgankweQ8E]} & 2022-09-10 & \href{https://youtube.com/watch?v=oHgankweQ8E}{\texttt{ oHgankweQ8E}} \\
約翰福音 6:1-15-20220917 & \hyperref[sec:Z8wgkxuhjIk]{【網上聖餐崇拜】仲諗D新野......得唔得呀|約翰福音6\_1-15|20220917 [Z8wgkxuhjIk]} & 2022-09-17 & \href{https://youtube.com/watch?v=Z8wgkxuhjIk}{\texttt{ Z8wgkxuhjIk}} \\
腓利門書 1:1-25-20220924 & \hyperref[sec:PqPiG_MpRK4]{【網上崇拜】我地新秩序|腓利門書1\_1-25|20220924 [PqPiG\_MpRK4]} & 2022-09-24 & \href{https://youtube.com/watch?v=PqPiG_MpRK4}{\texttt{ PqPiG\_MpRK4}} \\
以弗所書 1:15-23-20221001 & \hyperref[sec:1O4Wz5DFm4k]{【網上崇拜】教會的形狀|以弗所書1\_15-23|20221001 [1O4Wz5DFm4k]} & 2022-10-01 & \href{https://youtube.com/watch?v=1O4Wz5DFm4k}{\texttt{ 1O4Wz5DFm4k}} \\
以弗所書 4:17-24-20221008 & \hyperref[sec:Zv7Jalkm4FA]{【網上崇拜】新.型人|以弗所書4\_17-24|20221008 [Zv7Jalkm4FA]} & 2022-10-08 & \href{https://youtube.com/watch?v=Zv7Jalkm4FA}{\texttt{ Zv7Jalkm4FA}} \\
撒母耳記上 17:1-58-20221015 & \hyperref[sec:68fGAExhs0o]{【網上崇拜】想像力量同幻想會嚇你一跳|撒母耳記上17\_1-58|20221015 [68fGAExhs0o]} & 2022-10-15 & \href{https://youtube.com/watch?v=68fGAExhs0o}{\texttt{ 68fGAExhs0o}} \\
詩篇 121:1-8-20221022 & \hyperref[sec:l2LEDZopMVg]{【網上崇拜】惡人舞動|詩篇121\_1-8|20221022 [l2LEDZopMVg]} & 2022-10-22 & \href{https://youtube.com/watch?v=l2LEDZopMVg}{\texttt{ l2LEDZopMVg}} \\
列王記上 19:9-18-20221029 & \hyperref[sec:rw_6I9ppxNw]{【網上崇拜】做新事唔一定要型|列王記上19\_9-18|20221029 [rw-6I9ppxNw]} & 2022-10-29 & \href{https://youtube.com/watch?v=rw-6I9ppxNw}{\texttt{ rw-6I9ppxNw}} \\
路加福音 19:11-26-20221105 & \hyperref[sec:fVOAbAyMqQo]{【網上崇拜】未來的風險|路加福音19\_11-26|20221105 [fVOAbAyMqQo]} & 2022-11-05 & \href{https://youtube.com/watch?v=fVOAbAyMqQo}{\texttt{ fVOAbAyMqQo}} \\
腓立比書 3:13-21-20221112 & \hyperref[sec:giVKoZv8XXY]{【網上崇拜】未來的轉變|腓立比書3\_13-21|20221112 [giVKoZv8XXY]} & 2022-11-12 & \href{https://youtube.com/watch?v=giVKoZv8XXY}{\texttt{ giVKoZv8XXY}} \\
撒母耳記上 18:1-20:42-20221119 & \hyperref[sec:UHPz7So3h50]{【網上聖餐崇拜】致未來的牧者…友誼篇|撒母耳記上18\_1-20\_42|20221119 [UHPz7So3h50]} & 2022-11-19 & \href{https://youtube.com/watch?v=UHPz7So3h50}{\texttt{ UHPz7So3h50}} \\
馬可福音 8:27-9:8-20221126 & \hyperref[sec:ipiBLnwp8PI]{【網上崇拜】未來見|馬可福音8\_27-9\_8|20221126 [ipiBLnwp8PI]} & 2022-11-26 & \href{https://youtube.com/watch?v=ipiBLnwp8PI}{\texttt{ ipiBLnwp8PI}} \\
約翰福音 14:1-14-20221203 & \hyperref[sec:4yYwkRP32_4]{【網上崇拜】未來的耶穌|約翰福音14\_1-14|20221203 [4yYwkRP32\_4]} & 2022-12-03 & \href{https://youtube.com/watch?v=4yYwkRP32_4}{\texttt{ 4yYwkRP32\_4}} \\
約翰壹書 3:1-12-20221210 & \hyperref[sec:SVl4_oZWscg]{【網上崇拜】未來的愛心|約翰壹書3\_1-12|20221210 [SVl4\_oZWscg]} & 2022-12-10 & \href{https://youtube.com/watch?v=SVl4_oZWscg}{\texttt{ SVl4\_oZWscg}} \\
撒母耳記下 24:1-25-20221217 & \hyperref[sec:C1660tBb0Zk]{【網上聖餐崇拜】致未來「多餘」的牧者|撒母耳記下24\_1-25|20221217 [C1660tBb0Zk]} & 2022-12-17 & \href{https://youtube.com/watch?v=C1660tBb0Zk}{\texttt{ C1660tBb0Zk}} \\
路加福音 2:15-20-20221224 & \hyperref[sec:KRXO3Wxxfe0]{【網上崇拜】內向者的平安夜|路加福音2\_15-20|20221224 [KRXO3Wxxfe0]} & 2022-12-24 & \href{https://youtube.com/watch?v=KRXO3Wxxfe0}{\texttt{ KRXO3Wxxfe0}} \\
約翰福音 1:9-10-20221225 & \hyperref[sec:LRsUlh9Ini0]{【網上崇拜】外向者的聖誕節|約翰福音1\_9-10|20221225 [LRsUlh9Ini0]} & 2022-12-25 & \href{https://youtube.com/watch?v=LRsUlh9Ini0}{\texttt{ LRsUlh9Ini0}} \\
撒母耳記下 4:4-9:1-13-20221231 & \hyperref[sec:0FV84blFd3M]{【網上崇拜】感恩飯局-「邊個係你飯腳?」|撒母耳記下4\_4,9\_1-13|20221231 [0FV84blFd3M]} & 2022-12-31 & \href{https://youtube.com/watch?v=0FV84blFd3M}{\texttt{ 0FV84blFd3M}} \\
撒迦利亞書 2:6-8 & \hyperref[sec:rQtMXvUNeaE]{【網上崇拜】What is the apple of His eye?|撒迦利亞書2\_6-8;8\_4-6|20230107 [rQtMXvUNeaE]} & 2023-01-07 & \href{https://youtube.com/watch?v=rQtMXvUNeaE}{\texttt{ rQtMXvUNeaE}} \\
申命記 34:1-12-20230114 & \hyperref[sec:gptfSrlmqo8]{【網上崇拜】上帝視角|申命記34\_1-12|20230114 [gptfSrlmqo8]} & 2023-01-14 & \href{https://youtube.com/watch?v=gptfSrlmqo8}{\texttt{ gptfSrlmqo8}} \\
耶利米書 1:4-19-20230121 & \hyperref[sec:RPsjvP8W4CE]{【流堂崇拜】齋看見就夠曬|耶利米書1\_4-19|20230121 [RPsjvP8W4CE]} & 2023-01-21 & \href{https://youtube.com/watch?v=RPsjvP8W4CE}{\texttt{ RPsjvP8W4CE}} \\
    & \hyperref[sec:6BvsvxnAGsQ]{【流堂崇拜】兔年生肖道|20230128 [6BvsvxnAGsQ]} & 2023-01-28 & \href{https://youtube.com/watch?v=6BvsvxnAGsQ}{\texttt{ 6BvsvxnAGsQ}} \\
使徒行傳 3:1-10-20230204 & \hyperref[sec:GtMQusxSoOU]{【流堂崇拜】超越想像的視野|使徒行傳3\_1-10|20230204 [GtMQusxSoOU]} & 2023-02-04 & \href{https://youtube.com/watch?v=GtMQusxSoOU}{\texttt{ GtMQusxSoOU}} \\
哥林多前書 4:1-16-20230211 & \hyperref[sec:KAnMTZ32Dag]{【流堂崇拜】演員的自我修養|哥林多前書4\_1-16|20230211 [KAnMTZ32Dag]} & 2023-02-11 & \href{https://youtube.com/watch?v=KAnMTZ32Dag}{\texttt{ KAnMTZ32Dag}} \\
撒迦利亞書 2:1-5-20230218 & \hyperref[sec:4Dll86a7b18]{【流堂崇拜】講道者的自白和看見|撒迦利亞書2\_1-5|20230218 [4Dll86a7b18]} & 2023-02-18 & \href{https://youtube.com/watch?v=4Dll86a7b18}{\texttt{ 4Dll86a7b18}} \\
撒母耳記下 11:1-12:31-20230225 & \hyperref[sec:lsdGk_BkHa8]{【流堂崇拜】天光請開眼|撒母耳記下11\_1-12\_31|20230225 [lsdGk-BkHa8]} & 2023-02-25 & \href{https://youtube.com/watch?v=lsdGk-BkHa8}{\texttt{ lsdGk-BkHa8}} \\
    & \hyperref[sec:VfT5ldcLjqQ]{《致餘民及流散者:給香港基督徒的神學八課》第二季第1課|20230227 [VfT5ldcLjqQ]} & 2023-02-27 & \href{https://youtube.com/watch?v=VfT5ldcLjqQ}{\texttt{ VfT5ldcLjqQ}} \\
約伯記 42:1-6-20230304 & \hyperref[sec:dLJdySFiu9c]{【流堂崇拜】陳明|約伯記42\_1-6|20230304 [dLJdySFiu9c]} & 2023-03-04 & \href{https://youtube.com/watch?v=dLJdySFiu9c}{\texttt{ dLJdySFiu9c}} \\
撒母耳記上歷代志上 31:1-6 & \hyperref[sec:aPLQjM9J0JY]{【流堂崇拜】執迷不悔|撒母耳記上31\_1-6;歷代志上10\_13-14|20230311 [aPLQjM9J0JY]} & 2023-03-11 & \href{https://youtube.com/watch?v=aPLQjM9J0JY}{\texttt{ aPLQjM9J0JY}} \\
撒迦利亞書 5:1-11-20230318 & \hyperref[sec:y5NJfoAjRCI]{【流堂聖餐崇拜】《今際之國-飛行的書卷》|撒迦利亞書5\_1-11|20230318 [y5NJfoAjRCI]} & 2023-03-18 & \href{https://youtube.com/watch?v=y5NJfoAjRCI}{\texttt{ y5NJfoAjRCI}} \\
    & \hyperref[sec:7ZGXT0f30Z0]{《致餘民及流散者:給香港基督徒的神學八課》第二季第2課|20230325 [7ZGXT0f30Z0]} & 2023-03-25 & \href{https://youtube.com/watch?v=7ZGXT0f30Z0}{\texttt{ 7ZGXT0f30Z0}} \\
馬太福音 26:30-34-20230401 & \hyperref[sec:8KdYgVn_hzk]{【流堂崇拜】有關跌倒前的三件事|馬太福音26\_30-34|20230401 [8KdYgVn-hzk]} & 2023-04-01 & \href{https://youtube.com/watch?v=8KdYgVn-hzk}{\texttt{ 8KdYgVn-hzk}} \\
路加福音 23:32-43-20230408 & \hyperref[sec:v4hE6GM4QsI]{【流堂崇拜】我們都有錯|路加福音23\_32-43|20230408 [v4hE6GM4QsI]} & 2023-04-08 & \href{https://youtube.com/watch?v=v4hE6GM4QsI}{\texttt{ v4hE6GM4QsI}} \\
耶利米書 15:15-21-20230415 & \hyperref[sec:3PY1nwdp_0k]{【流堂崇拜】致歉-我們的九零後及千禧後|耶利米書15\_15-21|20230415 [3PY1nwdp\_0k]} & 2023-04-15 & \href{https://youtube.com/watch?v=3PY1nwdp_0k}{\texttt{ 3PY1nwdp\_0k}} \\
哈該書 1:1-15-20230422 & \hyperref[sec:S0X_1Lh_dHA]{【流堂崇拜】你先?我先?我地可能都會搞錯!|哈該書1\_1-15|20230422 [S0X-1Lh\_dHA]} & 2023-04-22 & \href{https://youtube.com/watch?v=S0X-1Lh_dHA}{\texttt{ S0X-1Lh\_dHA}} \\
以斯帖記 1:1-22-20230429 & \hyperref[sec:VNZbDAiXlG0]{【流堂崇拜】她和她最後的倔強|以斯帖記1\_1-22|20230429 [VNZbDAiXlG0]} & 2023-04-29 & \href{https://youtube.com/watch?v=VNZbDAiXlG0}{\texttt{ VNZbDAiXlG0}} \\
詩篇 98:1-9-20230506 & \hyperref[sec:D8sOzznkhGg]{【流堂崇拜】We sing everywhere|詩篇98\_1-9|20230506 [D8sOzznkhGg]} & 2023-05-06 & \href{https://youtube.com/watch?v=D8sOzznkhGg}{\texttt{ D8sOzznkhGg}} \\
    & \hyperref[sec:gRf39gjSNbM]{《致餘民及流散者:給香港基督徒的神學八課》第二季第3課|20230512 [gRf39gjSNbM]} & 2023-05-12 & \href{https://youtube.com/watch?v=gRf39gjSNbM}{\texttt{ gRf39gjSNbM}} \\
約翰一書 3:18-24-20230513 & \hyperref[sec:u6GL1Cm7cwU]{【網上崇拜】同心呼吸|約翰一書3\_18-24|20230513 [u6GL1Cm7cwU]} & 2023-05-13 & \href{https://youtube.com/watch?v=u6GL1Cm7cwU}{\texttt{ u6GL1Cm7cwU}} \\
猶大書 1:3-20230520 & \hyperref[sec:Aqi5hKVncec]{【網上聖餐崇拜】承傳arm beat的隱形遊樂場|猶大書1\_3|20230520 [Aqi5hKVncec]} & 2023-05-20 & \href{https://youtube.com/watch?v=Aqi5hKVncec}{\texttt{ Aqi5hKVncec}} \\
使徒行傳 2:1-13-20230527 & \hyperref[sec:OcD6qni0UQE]{【網上崇拜】霹靂一閃|使徒行傳2\_1-13|20230527 [OcD6qni0UQE]} & 2023-05-27 & \href{https://youtube.com/watch?v=OcD6qni0UQE}{\texttt{ OcD6qni0UQE}} \\
    & \hyperref[sec:RYCxV16hfwM]{《致餘民及流散者:給香港基督徒的神學八課》第二季第4課|20230528 [RYCxV16hfwM]} & 2023-05-28 & \href{https://youtube.com/watch?v=RYCxV16hfwM}{\texttt{ RYCxV16hfwM}} \\
馬太福音 25:1-13-20230603 & \hyperref[sec:40Zpw7rWZSQ]{【網上崇拜】【arm beat 慢歌版】一生守候|馬太福音25\_1-13|20230603 [40Zpw7rWZSQ]} & 2023-06-03 & \href{https://youtube.com/watch?v=40Zpw7rWZSQ}{\texttt{ 40Zpw7rWZSQ}} \\
使徒行傳 2:40-47-20230610 & \hyperref[sec:E2mqQjkFX2s]{【網上崇拜】同心用飯|使徒行傳2\_40-47|20230610 [E2mqQjkFX2s]} & 2023-06-10 & \href{https://youtube.com/watch?v=E2mqQjkFX2s}{\texttt{ E2mqQjkFX2s}} \\
約書亞記 2:1-22-24-20230617 & \hyperref[sec:CppPjcT08EA]{【網上崇拜】那些年,我沒有白過|約書亞記2\_1,22-24|20230617 [CppPjcT08EA]} & 2023-06-17 & \href{https://youtube.com/watch?v=CppPjcT08EA}{\texttt{ CppPjcT08EA}} \\
羅馬書 16:1-16-20230624 & \hyperref[sec:XixhhdfEXw8]{【網上崇拜】我地唔arm beat|羅馬書16\_1-16|20230624 [XixhhdfEXw8]} & 2023-06-24 & \href{https://youtube.com/watch?v=XixhhdfEXw8}{\texttt{ XixhhdfEXw8}} \\
    & \hyperref[sec:I6Z1WA7E0RA]{《致餘民及流散者:給香港基督徒的神學八課》第二季第5課|20230625 [I6Z1WA7E0RA]} & 2023-06-25 & \href{https://youtube.com/watch?v=I6Z1WA7E0RA}{\texttt{ I6Z1WA7E0RA}} \\
出埃及記 13:1-16-20230701 & \hyperref[sec:J_OpyaPLYIE]{【網上崇拜】奉獻第一|出埃及記13\_1-16|20230701 [J-OpyaPLYIE]} & 2023-07-01 & \href{https://youtube.com/watch?v=J-OpyaPLYIE}{\texttt{ J-OpyaPLYIE}} \\
以弗所書 5:1-2-20230708 & \hyperref[sec:vN0n_pNkXlA]{【網上崇拜】談犧牲|以弗所書5\_1-2|20230708 [vN0n-pNkXlA]} & 2023-07-08 & \href{https://youtube.com/watch?v=vN0n-pNkXlA}{\texttt{ vN0n-pNkXlA}} \\
瑪拉基書 3:6-12-20230722 & \hyperref[sec:k_DUSZ_Q45k]{【網上崇拜】係呀,講瑪拉基書|瑪拉基書3\_6-12|20230722 [k-DUSZ-Q45k]} & 2023-07-22 & \href{https://youtube.com/watch?v=k-DUSZ-Q45k}{\texttt{ k-DUSZ-Q45k}} \\
路加福音 14:1-35-20230729 & \hyperref[sec:nQfdSDvE_CU]{【網上崇拜】計唔掂|路加福音14\_1-35|20230729 [nQfdSDvE-CU]} & 2023-07-29 & \href{https://youtube.com/watch?v=nQfdSDvE-CU}{\texttt{ nQfdSDvE-CU}} \\
馬可福音 12:41-44-20230805 & \hyperref[sec:IJY_UvLqQqw]{【網上崇拜】窮寡婦的奉獻:你可能沒有想過的角度|馬可福音12\_41-44|20230805 [IJY\_UvLqQqw]} & 2023-08-05 & \href{https://youtube.com/watch?v=IJY_UvLqQqw}{\texttt{ IJY\_UvLqQqw}} \\
利未記 3:1-5-7:11-28-34-20230812 & \hyperref[sec:LgSzajW7sqo]{【網上崇拜】感謝祭|利未記3\_1-5,7\_11,28-34|20230812 [LgSzajW7sqo]} & 2023-08-12 & \href{https://youtube.com/watch?v=LgSzajW7sqo}{\texttt{ LgSzajW7sqo}} \\
哥林多前書 12:1-2-14-18-20230819 & \hyperref[sec:xHhMd2gjsxw]{【網上聖餐崇拜】顛覆教會恩賜的想像-奉獻為核心|哥林多前書12\_1-2,14-18|20230819 [xHhMd2gjsxw]} & 2023-08-19 & \href{https://youtube.com/watch?v=xHhMd2gjsxw}{\texttt{ xHhMd2gjsxw}} \\
馬可福音 14:3-9-20230826 & \hyperref[sec:br4_eJEULQ0]{【網上崇拜】真.拿.達|馬可福音14\_3-9|20230826 [br4-eJEULQ0]} & 2023-08-26 & \href{https://youtube.com/watch?v=br4-eJEULQ0}{\texttt{ br4-eJEULQ0}} \\
撒母耳記下 23:13-19-20230902 & \hyperref[sec:2n9NjI1RS9k]{【網上崇拜】大衛與三個沒有名字的勇士|撒母耳記下23\_13-19|20230902 [2n9NjI1RS9k]} & 2023-09-02 & \href{https://youtube.com/watch?v=2n9NjI1RS9k}{\texttt{ 2n9NjI1RS9k}} \\
創世記 11:1-9-20230909 & \hyperref[sec:n7jQq3kdpkc]{【網上崇拜】試o下唔好咁?|創世記11\_1-9|20230909 [n7jQq3kdpkc]} & 2023-09-09 & \href{https://youtube.com/watch?v=n7jQq3kdpkc}{\texttt{ n7jQq3kdpkc}} \\
路加福音 9:18-36-20230916 & \hyperref[sec:h7oDnhukFbo]{【網上聖餐崇拜】一齊做埋D 無聊ye 先至明白有乜意義|路加福音9\_18-36|20230916 [h7oDnhukFbo]} & 2023-09-16 & \href{https://youtube.com/watch?v=h7oDnhukFbo}{\texttt{ h7oDnhukFbo}} \\
使徒行傳 17:1-10-20230923 & \hyperref[sec:5EgvGimlwXk]{【網上崇拜】熱血福音戰士|使徒行傳17\_1-10|20230923 [5EgvGimlwXk]} & 2023-09-23 & \href{https://youtube.com/watch?v=5EgvGimlwXk}{\texttt{ 5EgvGimlwXk}} \\
    & \hyperref[sec:2QyWxsVtL8E]{《致餘民及流散者:給香港基督徒的神學八課》第二季第6課|20230924 [2QyWxsVtL8E]} & 2023-09-24 & \href{https://youtube.com/watch?v=2QyWxsVtL8E}{\texttt{ 2QyWxsVtL8E}} \\
民數記 27:1-11-20230930 & \hyperref[sec:JxHW7ujVbSI]{【網上崇拜】Shall We Talk?|民數記27\_1-11|20230930 [JxHW7ujVbSI]} & 2023-09-30 & \href{https://youtube.com/watch?v=JxHW7ujVbSI}{\texttt{ JxHW7ujVbSI}} \\
腓立比書 1:15-18-20231007 & \hyperref[sec:3_8UYgGhJ0E]{【網上崇拜】上莊的相反:結黨|腓立比書1\_15-18|20231007 [3\_8UYgGhJ0E]} & 2023-10-07 & \href{https://youtube.com/watch?v=3_8UYgGhJ0E}{\texttt{ 3\_8UYgGhJ0E}} \\
雅各書 2:1-9-12-17-20231014 & \hyperref[sec:CYdq0eOXFpM]{【網上崇拜】莊員的關係|雅各書2\_1-9,12-17|20231014 [CYdq0eOXFpM]} & 2023-10-14 & \href{https://youtube.com/watch?v=CYdq0eOXFpM}{\texttt{ CYdq0eOXFpM}} \\
馬可福音 9:1-13-20231021 & \hyperref[sec:3YrDRTxYY2U]{【網上崇拜】上莊猶如「勿說是推理」般|馬可福音9\_1-13|20231021 [3YrDRTxYY2U]} & 2023-10-21 & \href{https://youtube.com/watch?v=3YrDRTxYY2U}{\texttt{ 3YrDRTxYY2U}} \\
羅馬書 12:17-18-20231028 & \hyperref[sec:BDg16RM34JI]{【網上崇拜】正常發揮吧!|羅馬書12\_17-18|20231028 [BDg16RM34JI]} & 2023-10-28 & \href{https://youtube.com/watch?v=BDg16RM34JI}{\texttt{ BDg16RM34JI}} \\
    & \hyperref[sec:JKdFzjAsLZY]{《致餘民及流散者:給香港基督徒的神學八課》第二季第7課|20231101 [JKdFzjAsLZY]} & 2023-11-01 & \href{https://youtube.com/watch?v=JKdFzjAsLZY}{\texttt{ JKdFzjAsLZY}} \\
撒母耳記上 3:1-10-20231104 & \hyperref[sec:_cxLnHL_TWQ]{【網上崇拜】內心的小孩,仍活著嗎?|撒母耳記上3\_1-10|20231104 [\_cxLnHL-TWQ]} & 2023-11-04 & \href{https://youtube.com/watch?v=_cxLnHL-TWQ}{\texttt{ \_cxLnHL-TWQ}} \\
希伯來書 12:1-11-20231111 & \hyperref[sec:qJWlmXEzoSU]{【網上崇拜】孩子,我是這樣愛你的!|希伯來書12\_1-11|20231111 [qJWlmXEzoSU]} & 2023-11-11 & \href{https://youtube.com/watch?v=qJWlmXEzoSU}{\texttt{ qJWlmXEzoSU}} \\
耶利米書 50:29-30-20231118 & \hyperref[sec:2LJqqGa1zFo]{【網上崇拜】少年人必仆倒在地上|耶利米書50\_29-30|20231118 [2LJqqGa1zFo]} & 2023-11-18 & \href{https://youtube.com/watch?v=2LJqqGa1zFo}{\texttt{ 2LJqqGa1zFo}} \\
啟示錄 12:1-17-20231125 & \hyperref[sec:2LIl7VilU18]{【網上崇拜】細路仔唔識世界|啟示錄12\_1-17|20231125 [2LIl7VilU18]} & 2023-11-25 & \href{https://youtube.com/watch?v=2LIl7VilU18}{\texttt{ 2LIl7VilU18}} \\
    & \hyperref[sec:w1NzLUX2_GE]{《致餘民及流散者:給香港基督徒的神學八課》第二季第8課|20231126 [w1NzLUX2\_GE]} & 2023-11-26 & \href{https://youtube.com/watch?v=w1NzLUX2_GE}{\texttt{ w1NzLUX2\_GE}} \\
詩篇 90:1-17-20231202 & \hyperref[sec:lfg8MyM5M04]{【網上崇拜】大人者,不失其赤子之心者也|詩篇90\_1-17|20231202 [lfg8MyM5M04]} & 2023-12-02 & \href{https://youtube.com/watch?v=lfg8MyM5M04}{\texttt{ lfg8MyM5M04}} \\
路加福音 21:25-28-34-36-20231209 & \hyperref[sec:0oiGMpkgXB8]{【網上崇拜】超級耶穌基督,驚奇!|路加福音21\_25-28,34-36|20231209 [0oiGMpkgXB8]} & 2023-12-09 & \href{https://youtube.com/watch?v=0oiGMpkgXB8}{\texttt{ 0oiGMpkgXB8}} \\
馬太福音 8:1-4-20231216 & \hyperref[sec:sKBDQD8UIMg]{【網上聖餐崇拜】年少多好|馬太福音8\_1-4|20231216 [sKBDQD8UIMg]} & 2023-12-16 & \href{https://youtube.com/watch?v=sKBDQD8UIMg}{\texttt{ sKBDQD8UIMg}} \\
路加福音 2:10-12-49-20231223 & \hyperref[sec:dT3dN2jF8BQ]{【網上崇拜】天選的細路|路加福音2\_10-12,49|20231223 [dT3dN2jF8BQ]} & 2023-12-23 & \href{https://youtube.com/watch?v=dT3dN2jF8BQ}{\texttt{ dT3dN2jF8BQ}} \\
耶利米書 1:1-8-20231230 & \hyperref[sec:9ztySs_vnP4]{【網上崇拜】一道風景,一種心靈|耶利米書1\_1-8|20231230 [9ztySs-vnP4]} & 2023-12-30 & \href{https://youtube.com/watch?v=9ztySs-vnP4}{\texttt{ 9ztySs-vnP4}} \\
\end{xltabular}
}
\newpage

\section{目錄\small{(順卷)}}
\label{sec:index_scriptual}
{ \scriptsize


\begin{xltabular}{\textwidth}{|p{0.15\textwidth} p{0.6\textwidth}|p{0.07\textwidth} p{0.1\textwidth}|}
\hline
創世記 1:1-5-20220326 & \hyperref[sec:0d9n3K2nnYY]{【網上崇拜】你最閃亮的一刻,是你踏進不安的一剎.|創世記1\_1-5|20220326 [0d9n3K2nnYY]} & 2022-03-26 & \href{https://youtube.com/watch?v=0d9n3K2nnYY}{\texttt{ 0d9n3K2nnYY}} \\
創世記 11:1-9-20230909 & \hyperref[sec:n7jQq3kdpkc]{【網上崇拜】試o下唔好咁?|創世記11\_1-9|20230909 [n7jQq3kdpkc]} & 2023-09-09 & \href{https://youtube.com/watch?v=n7jQq3kdpkc}{\texttt{ n7jQq3kdpkc}} \\
創世記 37:1-50:26-20220226 & \hyperref[sec:kIOQG9wgqRs]{【網上崇拜】相信一切是最好的安排|創世記37\_1-50\_26|20220226 [kIOQG9wgqRs]} & 2022-02-26 & \href{https://youtube.com/watch?v=kIOQG9wgqRs}{\texttt{ kIOQG9wgqRs}} \\
出埃及記 1:15-22-20220305 & \hyperref[sec:oFrw_raeCu8]{【網上崇拜】兩位懂得甚麼是「越位」的女生|出埃及記1\_15-22|20220305 [oFrw\_raeCu8]} & 2022-03-05 & \href{https://youtube.com/watch?v=oFrw_raeCu8}{\texttt{ oFrw\_raeCu8}} \\
出埃及記 13:1-16-20230701 & \hyperref[sec:J_OpyaPLYIE]{【網上崇拜】奉獻第一|出埃及記13\_1-16|20230701 [J-OpyaPLYIE]} & 2023-07-01 & \href{https://youtube.com/watch?v=J-OpyaPLYIE}{\texttt{ J-OpyaPLYIE}} \\
利未記 3:1-5-7:11-28-34-20230812 & \hyperref[sec:LgSzajW7sqo]{【網上崇拜】感謝祭|利未記3\_1-5,7\_11,28-34|20230812 [LgSzajW7sqo]} & 2023-08-12 & \href{https://youtube.com/watch?v=LgSzajW7sqo}{\texttt{ LgSzajW7sqo}} \\
民數記 27:1-11-20230930 & \hyperref[sec:JxHW7ujVbSI]{【網上崇拜】Shall We Talk?|民數記27\_1-11|20230930 [JxHW7ujVbSI]} & 2023-09-30 & \href{https://youtube.com/watch?v=JxHW7ujVbSI}{\texttt{ JxHW7ujVbSI}} \\
申命記 34:1-12-20230114 & \hyperref[sec:gptfSrlmqo8]{【網上崇拜】上帝視角|申命記34\_1-12|20230114 [gptfSrlmqo8]} & 2023-01-14 & \href{https://youtube.com/watch?v=gptfSrlmqo8}{\texttt{ gptfSrlmqo8}} \\
約書亞記 2:1-22-24-20230617 & \hyperref[sec:CppPjcT08EA]{【網上崇拜】那些年,我沒有白過|約書亞記2\_1,22-24|20230617 [CppPjcT08EA]} & 2023-06-17 & \href{https://youtube.com/watch?v=CppPjcT08EA}{\texttt{ CppPjcT08EA}} \\
撒母耳記上 3:1-10-20231104 & \hyperref[sec:_cxLnHL_TWQ]{【網上崇拜】內心的小孩,仍活著嗎?|撒母耳記上3\_1-10|20231104 [\_cxLnHL-TWQ]} & 2023-11-04 & \href{https://youtube.com/watch?v=_cxLnHL-TWQ}{\texttt{ \_cxLnHL-TWQ}} \\
撒母耳記上 13:1-23-20220129 & \hyperref[sec:FLZJqJSyEdc]{【網上崇拜】召命人生|撒母耳記上13\_1-23|20220129 [FLZJqJSyEdc]} & 2022-01-29 & \href{https://youtube.com/watch?v=FLZJqJSyEdc}{\texttt{ FLZJqJSyEdc}} \\
撒母耳記上 17:1-58-20221015 & \hyperref[sec:68fGAExhs0o]{【網上崇拜】想像力量同幻想會嚇你一跳|撒母耳記上17\_1-58|20221015 [68fGAExhs0o]} & 2022-10-15 & \href{https://youtube.com/watch?v=68fGAExhs0o}{\texttt{ 68fGAExhs0o}} \\
撒母耳記上 18:1-20:42-20221119 & \hyperref[sec:UHPz7So3h50]{【網上聖餐崇拜】致未來的牧者…友誼篇|撒母耳記上18\_1-20\_42|20221119 [UHPz7So3h50]} & 2022-11-19 & \href{https://youtube.com/watch?v=UHPz7So3h50}{\texttt{ UHPz7So3h50}} \\
撒母耳記下 4:4-9:1-13-20221231 & \hyperref[sec:0FV84blFd3M]{【網上崇拜】感恩飯局-「邊個係你飯腳?」|撒母耳記下4\_4,9\_1-13|20221231 [0FV84blFd3M]} & 2022-12-31 & \href{https://youtube.com/watch?v=0FV84blFd3M}{\texttt{ 0FV84blFd3M}} \\
撒母耳記下 11:1-12:31-20230225 & \hyperref[sec:lsdGk_BkHa8]{【流堂崇拜】天光請開眼|撒母耳記下11\_1-12\_31|20230225 [lsdGk-BkHa8]} & 2023-02-25 & \href{https://youtube.com/watch?v=lsdGk-BkHa8}{\texttt{ lsdGk-BkHa8}} \\
撒母耳記下 23:13-19-20230902 & \hyperref[sec:2n9NjI1RS9k]{【網上崇拜】大衛與三個沒有名字的勇士|撒母耳記下23\_13-19|20230902 [2n9NjI1RS9k]} & 2023-09-02 & \href{https://youtube.com/watch?v=2n9NjI1RS9k}{\texttt{ 2n9NjI1RS9k}} \\
撒母耳記下 24:1-25-20221217 & \hyperref[sec:C1660tBb0Zk]{【網上聖餐崇拜】致未來「多餘」的牧者|撒母耳記下24\_1-25|20221217 [C1660tBb0Zk]} & 2022-12-17 & \href{https://youtube.com/watch?v=C1660tBb0Zk}{\texttt{ C1660tBb0Zk}} \\
列王記上 19:9-18-20221029 & \hyperref[sec:rw_6I9ppxNw]{【網上崇拜】做新事唔一定要型|列王記上19\_9-18|20221029 [rw-6I9ppxNw]} & 2022-10-29 & \href{https://youtube.com/watch?v=rw-6I9ppxNw}{\texttt{ rw-6I9ppxNw}} \\
耶利米書 1:1-8-20231230 & \hyperref[sec:9ztySs_vnP4]{【網上崇拜】一道風景,一種心靈|耶利米書1\_1-8|20231230 [9ztySs-vnP4]} & 2023-12-30 & \href{https://youtube.com/watch?v=9ztySs-vnP4}{\texttt{ 9ztySs-vnP4}} \\
耶利米書 1:4-19-20230121 & \hyperref[sec:RPsjvP8W4CE]{【流堂崇拜】齋看見就夠曬|耶利米書1\_4-19|20230121 [RPsjvP8W4CE]} & 2023-01-21 & \href{https://youtube.com/watch?v=RPsjvP8W4CE}{\texttt{ RPsjvP8W4CE}} \\
耶利米書 15:15-21-20230415 & \hyperref[sec:3PY1nwdp_0k]{【流堂崇拜】致歉-我們的九零後及千禧後|耶利米書15\_15-21|20230415 [3PY1nwdp\_0k]} & 2023-04-15 & \href{https://youtube.com/watch?v=3PY1nwdp_0k}{\texttt{ 3PY1nwdp\_0k}} \\
耶利米書 50:29-30-20231118 & \hyperref[sec:2LJqqGa1zFo]{【網上崇拜】少年人必仆倒在地上|耶利米書50\_29-30|20231118 [2LJqqGa1zFo]} & 2023-11-18 & \href{https://youtube.com/watch?v=2LJqqGa1zFo}{\texttt{ 2LJqqGa1zFo}} \\
以西結書 47:1-2-20220122 & \hyperref[sec:g_5XGfcpSSo]{【網上崇拜】聖殿水浸的異象|以西結書47\_1-2|20220122 [g\_5XGfcpSSo]} & 2022-01-22 & \href{https://youtube.com/watch?v=g_5XGfcpSSo}{\texttt{ g\_5XGfcpSSo}} \\
哈巴谷書 1:5-20220423 & \hyperref[sec:UT74cFcV7PU]{【網上崇拜】上帝叫我厚多士(已FC)|哈巴谷書1\_5|20220423 [UT74cFcV7PU]} & 2022-04-23 & \href{https://youtube.com/watch?v=UT74cFcV7PU}{\texttt{ UT74cFcV7PU}} \\
哈該書 1:1-15-20230422 & \hyperref[sec:S0X_1Lh_dHA]{【流堂崇拜】你先?我先?我地可能都會搞錯!|哈該書1\_1-15|20230422 [S0X-1Lh\_dHA]} & 2023-04-22 & \href{https://youtube.com/watch?v=S0X-1Lh_dHA}{\texttt{ S0X-1Lh\_dHA}} \\
撒迦利亞書 2:1-5-20230218 & \hyperref[sec:4Dll86a7b18]{【流堂崇拜】講道者的自白和看見|撒迦利亞書2\_1-5|20230218 [4Dll86a7b18]} & 2023-02-18 & \href{https://youtube.com/watch?v=4Dll86a7b18}{\texttt{ 4Dll86a7b18}} \\
撒迦利亞書 2:6-8 & \hyperref[sec:rQtMXvUNeaE]{【網上崇拜】What is the apple of His eye?|撒迦利亞書2\_6-8;8\_4-6|20230107 [rQtMXvUNeaE]} & 2023-01-07 & \href{https://youtube.com/watch?v=rQtMXvUNeaE}{\texttt{ rQtMXvUNeaE}} \\
撒迦利亞書 5:1-11-20230318 & \hyperref[sec:y5NJfoAjRCI]{【流堂聖餐崇拜】《今際之國-飛行的書卷》|撒迦利亞書5\_1-11|20230318 [y5NJfoAjRCI]} & 2023-03-18 & \href{https://youtube.com/watch?v=y5NJfoAjRCI}{\texttt{ y5NJfoAjRCI}} \\
瑪拉基書 3:6-12-20230722 & \hyperref[sec:k_DUSZ_Q45k]{【網上崇拜】係呀,講瑪拉基書|瑪拉基書3\_6-12|20230722 [k-DUSZ-Q45k]} & 2023-07-22 & \href{https://youtube.com/watch?v=k-DUSZ-Q45k}{\texttt{ k-DUSZ-Q45k}} \\
詩篇 23:1-6-20220618 & \hyperref[sec:rvuxp0Bes90]{【網上聖餐崇拜】小心地滑…擁抱唔到|詩篇23\_1-6|20220618 [rvuxp0Bes90]} & 2022-06-18 & \href{https://youtube.com/watch?v=rvuxp0Bes90}{\texttt{ rvuxp0Bes90}} \\
詩篇 90:1-17-20231202 & \hyperref[sec:lfg8MyM5M04]{【網上崇拜】大人者,不失其赤子之心者也|詩篇90\_1-17|20231202 [lfg8MyM5M04]} & 2023-12-02 & \href{https://youtube.com/watch?v=lfg8MyM5M04}{\texttt{ lfg8MyM5M04}} \\
詩篇 98:1-9-20230506 & \hyperref[sec:D8sOzznkhGg]{【流堂崇拜】We sing everywhere|詩篇98\_1-9|20230506 [D8sOzznkhGg]} & 2023-05-06 & \href{https://youtube.com/watch?v=D8sOzznkhGg}{\texttt{ D8sOzznkhGg}} \\
詩篇 121:1-8-20221022 & \hyperref[sec:l2LEDZopMVg]{【網上崇拜】惡人舞動|詩篇121\_1-8|20221022 [l2LEDZopMVg]} & 2022-10-22 & \href{https://youtube.com/watch?v=l2LEDZopMVg}{\texttt{ l2LEDZopMVg}} \\
約伯記 42:1-6-20230304 & \hyperref[sec:dLJdySFiu9c]{【流堂崇拜】陳明|約伯記42\_1-6|20230304 [dLJdySFiu9c]} & 2023-03-04 & \href{https://youtube.com/watch?v=dLJdySFiu9c}{\texttt{ dLJdySFiu9c}} \\
傳道書 3:11-20220903 & \hyperref[sec:1uGH28VFFVk]{【網上崇拜】上帝的美學|傳道書3\_11|20220903 [1uGH28VFFVk]} & 2022-09-03 & \href{https://youtube.com/watch?v=1uGH28VFFVk}{\texttt{ 1uGH28VFFVk}} \\
以斯帖記 1:1-22-20230429 & \hyperref[sec:VNZbDAiXlG0]{【流堂崇拜】她和她最後的倔強|以斯帖記1\_1-22|20230429 [VNZbDAiXlG0]} & 2023-04-29 & \href{https://youtube.com/watch?v=VNZbDAiXlG0}{\texttt{ VNZbDAiXlG0}} \\
但以理書 3:1-30-20220723 & \hyperref[sec:RWAsMtjQmZ4]{【網上崇拜】難道我們未夠難?|但以理書3\_1-30|20220723 [RWAsMtjQmZ4]} & 2022-07-23 & \href{https://youtube.com/watch?v=RWAsMtjQmZ4}{\texttt{ RWAsMtjQmZ4}} \\
馬太福音 5:20-20220402 & \hyperref[sec:rbtYKzrN9IU]{【網上崇拜】Außerordentlichen 不要尋常|馬太福音5\_20|20220402 [rbtYKzrN9IU]} & 2022-04-02 & \href{https://youtube.com/watch?v=rbtYKzrN9IU}{\texttt{ rbtYKzrN9IU}} \\
馬太福音 8:1-4-20231216 & \hyperref[sec:sKBDQD8UIMg]{【網上聖餐崇拜】年少多好|馬太福音8\_1-4|20231216 [sKBDQD8UIMg]} & 2023-12-16 & \href{https://youtube.com/watch?v=sKBDQD8UIMg}{\texttt{ sKBDQD8UIMg}} \\
馬太福音 16:13-23-20220820 & \hyperref[sec:sQEhDyhKFnE]{【網上聖餐崇拜】跟最難約定的人講約定|馬太福音16\_13-23|20220820 [sQEhDyhKFnE]} & 2022-08-20 & \href{https://youtube.com/watch?v=sQEhDyhKFnE}{\texttt{ sQEhDyhKFnE}} \\
馬太福音 25:1-13-20230603 & \hyperref[sec:40Zpw7rWZSQ]{【網上崇拜】【arm beat 慢歌版】一生守候|馬太福音25\_1-13|20230603 [40Zpw7rWZSQ]} & 2023-06-03 & \href{https://youtube.com/watch?v=40Zpw7rWZSQ}{\texttt{ 40Zpw7rWZSQ}} \\
馬太福音 26:30-34-20230401 & \hyperref[sec:8KdYgVn_hzk]{【流堂崇拜】有關跌倒前的三件事|馬太福音26\_30-34|20230401 [8KdYgVn-hzk]} & 2023-04-01 & \href{https://youtube.com/watch?v=8KdYgVn-hzk}{\texttt{ 8KdYgVn-hzk}} \\
馬可福音 8:27-9:8-20221126 & \hyperref[sec:ipiBLnwp8PI]{【網上崇拜】未來見|馬可福音8\_27-9\_8|20221126 [ipiBLnwp8PI]} & 2022-11-26 & \href{https://youtube.com/watch?v=ipiBLnwp8PI}{\texttt{ ipiBLnwp8PI}} \\
馬可福音 9:1-13-20231021 & \hyperref[sec:3YrDRTxYY2U]{【網上崇拜】上莊猶如「勿說是推理」般|馬可福音9\_1-13|20231021 [3YrDRTxYY2U]} & 2023-10-21 & \href{https://youtube.com/watch?v=3YrDRTxYY2U}{\texttt{ 3YrDRTxYY2U}} \\
馬可福音 10:13-16-20220514 & \hyperref[sec:97SC38c6sqY]{【網上崇拜】最終的擁抱|馬可福音10\_13-16|20220514 [97SC38c6sqY]} & 2022-05-14 & \href{https://youtube.com/watch?v=97SC38c6sqY}{\texttt{ 97SC38c6sqY}} \\
馬可福音 12:41-44-20230805 & \hyperref[sec:IJY_UvLqQqw]{【網上崇拜】窮寡婦的奉獻:你可能沒有想過的角度|馬可福音12\_41-44|20230805 [IJY\_UvLqQqw]} & 2023-08-05 & \href{https://youtube.com/watch?v=IJY_UvLqQqw}{\texttt{ IJY\_UvLqQqw}} \\
馬可福音 14:3-9-20230826 & \hyperref[sec:br4_eJEULQ0]{【網上崇拜】真.拿.達|馬可福音14\_3-9|20230826 [br4-eJEULQ0]} & 2023-08-26 & \href{https://youtube.com/watch?v=br4-eJEULQ0}{\texttt{ br4-eJEULQ0}} \\
路加福音 2:10-12-49-20231223 & \hyperref[sec:dT3dN2jF8BQ]{【網上崇拜】天選的細路|路加福音2\_10-12,49|20231223 [dT3dN2jF8BQ]} & 2023-12-23 & \href{https://youtube.com/watch?v=dT3dN2jF8BQ}{\texttt{ dT3dN2jF8BQ}} \\
路加福音 2:15-20-20221224 & \hyperref[sec:KRXO3Wxxfe0]{【網上崇拜】內向者的平安夜|路加福音2\_15-20|20221224 [KRXO3Wxxfe0]} & 2022-12-24 & \href{https://youtube.com/watch?v=KRXO3Wxxfe0}{\texttt{ KRXO3Wxxfe0}} \\
路加福音 9:18-36-20230916 & \hyperref[sec:h7oDnhukFbo]{【網上聖餐崇拜】一齊做埋D 無聊ye 先至明白有乜意義|路加福音9\_18-36|20230916 [h7oDnhukFbo]} & 2023-09-16 & \href{https://youtube.com/watch?v=h7oDnhukFbo}{\texttt{ h7oDnhukFbo}} \\
路加福音 11:29-32-20220716 & \hyperref[sec:CzS_E_B5XMA]{【網上聖餐崇拜】誰可唱最後的信仰...|路加福音11\_29-32|20220716 [CzS\_E-B5XMA]} & 2022-07-16 & \href{https://youtube.com/watch?v=CzS_E-B5XMA}{\texttt{ CzS\_E-B5XMA}} \\
路加福音 14:1-35-20230729 & \hyperref[sec:nQfdSDvE_CU]{【網上崇拜】計唔掂|路加福音14\_1-35|20230729 [nQfdSDvE-CU]} & 2023-07-29 & \href{https://youtube.com/watch?v=nQfdSDvE-CU}{\texttt{ nQfdSDvE-CU}} \\
路加福音 15:1-7-20220625 & \hyperref[sec:zMmzg_ext8I]{【網上崇拜】乜ye ye 問題一律建議抱緊處理|路加福音15\_1-7|20220625 [zMmzg\_ext8I]} & 2022-06-25 & \href{https://youtube.com/watch?v=zMmzg_ext8I}{\texttt{ zMmzg\_ext8I}} \\
路加福音 15:11-31-20220604 & \hyperref[sec:NkL6a2IH8uY]{【網上崇拜】然後去學習擁抱一切|路加福音15\_11-31|20220604 [NkL6a2IH8uY]} & 2022-06-04 & \href{https://youtube.com/watch?v=NkL6a2IH8uY}{\texttt{ NkL6a2IH8uY}} \\
路加福音 15:11-31-20220507 & \hyperref[sec:b7gyPC12_AM]{【網上崇拜】首先學習如何被擁抱|路加福音15\_11-31|20220507 [b7gyPC12\_AM]} & 2022-05-07 & \href{https://youtube.com/watch?v=b7gyPC12_AM}{\texttt{ b7gyPC12\_AM}} \\
路加福音 19:11-26-20221105 & \hyperref[sec:fVOAbAyMqQo]{【網上崇拜】未來的風險|路加福音19\_11-26|20221105 [fVOAbAyMqQo]} & 2022-11-05 & \href{https://youtube.com/watch?v=fVOAbAyMqQo}{\texttt{ fVOAbAyMqQo}} \\
路加福音 21:25-28-34-36-20231209 & \hyperref[sec:0oiGMpkgXB8]{【網上崇拜】超級耶穌基督,驚奇!|路加福音21\_25-28,34-36|20231209 [0oiGMpkgXB8]} & 2023-12-09 & \href{https://youtube.com/watch?v=0oiGMpkgXB8}{\texttt{ 0oiGMpkgXB8}} \\
路加福音 23:32-43-20230408 & \hyperref[sec:v4hE6GM4QsI]{【流堂崇拜】我們都有錯|路加福音23\_32-43|20230408 [v4hE6GM4QsI]} & 2023-04-08 & \href{https://youtube.com/watch?v=v4hE6GM4QsI}{\texttt{ v4hE6GM4QsI}} \\
約翰福音 1:9-10-20221225 & \hyperref[sec:LRsUlh9Ini0]{【網上崇拜】外向者的聖誕節|約翰福音1\_9-10|20221225 [LRsUlh9Ini0]} & 2022-12-25 & \href{https://youtube.com/watch?v=LRsUlh9Ini0}{\texttt{ LRsUlh9Ini0}} \\
約翰福音 6:1-15-20220917 & \hyperref[sec:Z8wgkxuhjIk]{【網上聖餐崇拜】仲諗D新野......得唔得呀|約翰福音6\_1-15|20220917 [Z8wgkxuhjIk]} & 2022-09-17 & \href{https://youtube.com/watch?v=Z8wgkxuhjIk}{\texttt{ Z8wgkxuhjIk}} \\
約翰福音 13:1-38-20220611 & \hyperref[sec:SLTU9ILLofg]{【網上崇拜】攬到底|約翰福音13\_1-38|20220611 [SLTU9ILLofg]} & 2022-06-11 & \href{https://youtube.com/watch?v=SLTU9ILLofg}{\texttt{ SLTU9ILLofg}} \\
約翰福音 14:1-14-20221203 & \hyperref[sec:4yYwkRP32_4]{【網上崇拜】未來的耶穌|約翰福音14\_1-14|20221203 [4yYwkRP32\_4]} & 2022-12-03 & \href{https://youtube.com/watch?v=4yYwkRP32_4}{\texttt{ 4yYwkRP32\_4}} \\
約翰福音 21:18-19-20220212 & \hyperref[sec:wntIcXZCGmo]{【網上崇拜】再有一次機會|約翰福音21\_18-19|20220212 [wntIcXZCGmo]} & 2022-02-12 & \href{https://youtube.com/watch?v=wntIcXZCGmo}{\texttt{ wntIcXZCGmo}} \\
使徒行傳 2:1-13-20230527 & \hyperref[sec:OcD6qni0UQE]{【網上崇拜】霹靂一閃|使徒行傳2\_1-13|20230527 [OcD6qni0UQE]} & 2023-05-27 & \href{https://youtube.com/watch?v=OcD6qni0UQE}{\texttt{ OcD6qni0UQE}} \\
使徒行傳 2:40-47-20230610 & \hyperref[sec:E2mqQjkFX2s]{【網上崇拜】同心用飯|使徒行傳2\_40-47|20230610 [E2mqQjkFX2s]} & 2023-06-10 & \href{https://youtube.com/watch?v=E2mqQjkFX2s}{\texttt{ E2mqQjkFX2s}} \\
使徒行傳 3:1-10-20230204 & \hyperref[sec:GtMQusxSoOU]{【流堂崇拜】超越想像的視野|使徒行傳3\_1-10|20230204 [GtMQusxSoOU]} & 2023-02-04 & \href{https://youtube.com/watch?v=GtMQusxSoOU}{\texttt{ GtMQusxSoOU}} \\
使徒行傳 17:1-10-20230923 & \hyperref[sec:5EgvGimlwXk]{【網上崇拜】熱血福音戰士|使徒行傳17\_1-10|20230923 [5EgvGimlwXk]} & 2023-09-23 & \href{https://youtube.com/watch?v=5EgvGimlwXk}{\texttt{ 5EgvGimlwXk}} \\
使徒行傳 20:17-38-20220709 & \hyperref[sec:XDQJvaySA3k]{【網上崇拜】就算會與你分離......|使徒行傳20\_17-38|20220709 [XDQJvaySA3k]} & 2022-07-09 & \href{https://youtube.com/watch?v=XDQJvaySA3k}{\texttt{ XDQJvaySA3k}} \\
使徒行傳 23:1-35-20220409 & \hyperref[sec:9oqAyDoUD6A]{【網上崇拜】害怕危險才是真正的危險|使徒行傳23\_1-35|20220409 [9oqAyDoUD6A]} & 2022-04-09 & \href{https://youtube.com/watch?v=9oqAyDoUD6A}{\texttt{ 9oqAyDoUD6A}} \\
羅馬書 12:17-18-20231028 & \hyperref[sec:BDg16RM34JI]{【網上崇拜】正常發揮吧!|羅馬書12\_17-18|20231028 [BDg16RM34JI]} & 2023-10-28 & \href{https://youtube.com/watch?v=BDg16RM34JI}{\texttt{ BDg16RM34JI}} \\
羅馬書 14:1-15:13-20220528 & \hyperref[sec:OJx_AQpZJpY]{【網上崇拜】擁彼此接納──飯就一定要食,一齊食開心D ?|羅馬書14\_1-15\_13|20220528 [OJx-AQpZJpY]} & 2022-05-28 & \href{https://youtube.com/watch?v=OJx-AQpZJpY}{\texttt{ OJx-AQpZJpY}} \\
羅馬書 15:14-21-20220319 & \hyperref[sec:3bu5V6aPJC0]{【網上聖餐崇拜】越位人生|羅馬書15\_14-21|20220319 [3bu5V6aPJC0]} & 2022-03-19 & \href{https://youtube.com/watch?v=3bu5V6aPJC0}{\texttt{ 3bu5V6aPJC0}} \\
羅馬書 16:1-16-20230624 & \hyperref[sec:XixhhdfEXw8]{【網上崇拜】我地唔arm beat|羅馬書16\_1-16|20230624 [XixhhdfEXw8]} & 2023-06-24 & \href{https://youtube.com/watch?v=XixhhdfEXw8}{\texttt{ XixhhdfEXw8}} \\
哥林多前書 4:1-16-20230211 & \hyperref[sec:KAnMTZ32Dag]{【流堂崇拜】演員的自我修養|哥林多前書4\_1-16|20230211 [KAnMTZ32Dag]} & 2023-02-11 & \href{https://youtube.com/watch?v=KAnMTZ32Dag}{\texttt{ KAnMTZ32Dag}} \\
哥林多前書 12:1-2-14-18-20230819 & \hyperref[sec:xHhMd2gjsxw]{【網上聖餐崇拜】顛覆教會恩賜的想像-奉獻為核心|哥林多前書12\_1-2,14-18|20230819 [xHhMd2gjsxw]} & 2023-08-19 & \href{https://youtube.com/watch?v=xHhMd2gjsxw}{\texttt{ xHhMd2gjsxw}} \\
加拉太書 1:1-12-20220806 & \hyperref[sec:0mi_NsvpcRc]{【網上崇拜】自由是最美好的禮物|加拉太書1\_1-12|20220806 [0mi\_NsvpcRc]} & 2022-08-06 & \href{https://youtube.com/watch?v=0mi_NsvpcRc}{\texttt{ 0mi\_NsvpcRc}} \\
以弗所書 1:15-23-20221001 & \hyperref[sec:1O4Wz5DFm4k]{【網上崇拜】教會的形狀|以弗所書1\_15-23|20221001 [1O4Wz5DFm4k]} & 2022-10-01 & \href{https://youtube.com/watch?v=1O4Wz5DFm4k}{\texttt{ 1O4Wz5DFm4k}} \\
以弗所書 4:17-24-20221008 & \hyperref[sec:Zv7Jalkm4FA]{【網上崇拜】新.型人|以弗所書4\_17-24|20221008 [Zv7Jalkm4FA]} & 2022-10-08 & \href{https://youtube.com/watch?v=Zv7Jalkm4FA}{\texttt{ Zv7Jalkm4FA}} \\
以弗所書 5:1-2-20230708 & \hyperref[sec:vN0n_pNkXlA]{【網上崇拜】談犧牲|以弗所書5\_1-2|20230708 [vN0n-pNkXlA]} & 2023-07-08 & \href{https://youtube.com/watch?v=vN0n-pNkXlA}{\texttt{ vN0n-pNkXlA}} \\
腓立比書 1:15-18-20231007 & \hyperref[sec:3_8UYgGhJ0E]{【網上崇拜】上莊的相反:結黨|腓立比書1\_15-18|20231007 [3\_8UYgGhJ0E]} & 2023-10-07 & \href{https://youtube.com/watch?v=3_8UYgGhJ0E}{\texttt{ 3\_8UYgGhJ0E}} \\
腓立比書 3:13-21-20221112 & \hyperref[sec:giVKoZv8XXY]{【網上崇拜】未來的轉變|腓立比書3\_13-21|20221112 [giVKoZv8XXY]} & 2022-11-12 & \href{https://youtube.com/watch?v=giVKoZv8XXY}{\texttt{ giVKoZv8XXY}} \\
提摩太後書 4:1-2-20220702 & \hyperref[sec:o_rYmsKLSrs]{【網上崇拜】仍未忘跟你約定假如沒有死|提摩太後書4\_1-2|20220702 [o\_rYmsKLSrs]} & 2022-07-02 & \href{https://youtube.com/watch?v=o_rYmsKLSrs}{\texttt{ o\_rYmsKLSrs}} \\
提摩太後書 4:6-8-20220730 & \hyperref[sec:muF9XhDEVwY]{【網上崇拜】兩鬢斑白都可認得你|提摩太後書4\_6-8|20220730 [muF9XhDEVwY]} & 2022-07-30 & \href{https://youtube.com/watch?v=muF9XhDEVwY}{\texttt{ muF9XhDEVwY}} \\
腓利門書 1:1-25 & \hyperref[sec:J3EQacUFDFI]{【網上崇拜】再見…再相見 | 腓利門書1\_1-25 | 20210206 [J3EQacUFDFI]} & 2021-02-06 & \href{https://youtube.com/watch?v=J3EQacUFDFI}{\texttt{ J3EQacUFDFI}} \\
腓利門書 1:1-25-20220924 & \hyperref[sec:PqPiG_MpRK4]{【網上崇拜】我地新秩序|腓利門書1\_1-25|20220924 [PqPiG\_MpRK4]} & 2022-09-24 & \href{https://youtube.com/watch?v=PqPiG_MpRK4}{\texttt{ PqPiG\_MpRK4}} \\
腓利門書 1:1-25-20220827 & \hyperref[sec:pF3HM3BllyQ]{【網上崇拜】我地|腓利門書1\_1-25|20220827 [pF3HM3BllyQ]} & 2022-08-27 & \href{https://youtube.com/watch?v=pF3HM3BllyQ}{\texttt{ pF3HM3BllyQ}} \\
希伯來書 11:8-22-20220910 & \hyperref[sec:oHgankweQ8E]{【網上崇拜】月是故鄉明|希伯來書11\_8-22|20220910 [oHgankweQ8E]} & 2022-09-10 & \href{https://youtube.com/watch?v=oHgankweQ8E}{\texttt{ oHgankweQ8E}} \\
希伯來書 12:1-11-20231111 & \hyperref[sec:qJWlmXEzoSU]{【網上崇拜】孩子,我是這樣愛你的!|希伯來書12\_1-11|20231111 [qJWlmXEzoSU]} & 2023-11-11 & \href{https://youtube.com/watch?v=qJWlmXEzoSU}{\texttt{ qJWlmXEzoSU}} \\
雅各書 2:1-9-12-17-20231014 & \hyperref[sec:CYdq0eOXFpM]{【網上崇拜】莊員的關係|雅各書2\_1-9,12-17|20231014 [CYdq0eOXFpM]} & 2023-10-14 & \href{https://youtube.com/watch?v=CYdq0eOXFpM}{\texttt{ CYdq0eOXFpM}} \\
彼得前書 1:17-21-20220219 & \hyperref[sec:GPCXXRzSHkw]{【網上聖餐崇拜】而家仲有乜好flow?!|彼得前書1\_17-21|20220219 [GPCXXRzSHkw]} & 2022-02-19 & \href{https://youtube.com/watch?v=GPCXXRzSHkw}{\texttt{ GPCXXRzSHkw}} \\
彼得前書 2:18-25-20220312 & \hyperref[sec:YQgThbN8z0Q]{【網上崇拜】基督都愛越位|彼得前書2\_18-25|20220312 [YQgThbN8z0Q]} & 2022-03-12 & \href{https://youtube.com/watch?v=YQgThbN8z0Q}{\texttt{ YQgThbN8z0Q}} \\
彼得前書 4:1-6-20220416 & \hyperref[sec:LSvYW2YLyv0]{【網上聖餐崇拜】得受苦一個選擇,但你仍有得揀|彼得前書4\_1-6|20220416 [LSvYW2YLyv0]} & 2022-04-16 & \href{https://youtube.com/watch?v=LSvYW2YLyv0}{\texttt{ LSvYW2YLyv0}} \\
彼得前書 4:7-11-20220521 & \hyperref[sec:2t83SY_sddQ]{【網上聖餐崇拜】擁抱的規矩…MM7|彼得前書4\_7-11|20220521 [2t83SY\_sddQ]} & 2022-05-21 & \href{https://youtube.com/watch?v=2t83SY_sddQ}{\texttt{ 2t83SY\_sddQ}} \\
約翰壹書 3:1-12-20221210 & \hyperref[sec:SVl4_oZWscg]{【網上崇拜】未來的愛心|約翰壹書3\_1-12|20221210 [SVl4\_oZWscg]} & 2022-12-10 & \href{https://youtube.com/watch?v=SVl4_oZWscg}{\texttt{ SVl4\_oZWscg}} \\
約翰一書 3:18-24-20230513 & \hyperref[sec:u6GL1Cm7cwU]{【網上崇拜】同心呼吸|約翰一書3\_18-24|20230513 [u6GL1Cm7cwU]} & 2023-05-13 & \href{https://youtube.com/watch?v=u6GL1Cm7cwU}{\texttt{ u6GL1Cm7cwU}} \\
猶大書 1:3-20230520 & \hyperref[sec:Aqi5hKVncec]{【網上聖餐崇拜】承傳arm beat的隱形遊樂場|猶大書1\_3|20230520 [Aqi5hKVncec]} & 2023-05-20 & \href{https://youtube.com/watch?v=Aqi5hKVncec}{\texttt{ Aqi5hKVncec}} \\
啟示錄 12:1-17-20231125 & \hyperref[sec:2LIl7VilU18]{【網上崇拜】細路仔唔識世界|啟示錄12\_1-17|20231125 [2LIl7VilU18]} & 2023-11-25 & \href{https://youtube.com/watch?v=2LIl7VilU18}{\texttt{ 2LIl7VilU18}} \\
    & \hyperref[sec:VfT5ldcLjqQ]{《致餘民及流散者:給香港基督徒的神學八課》第二季第1課|20230227 [VfT5ldcLjqQ]} & 2023-02-27 & \href{https://youtube.com/watch?v=VfT5ldcLjqQ}{\texttt{ VfT5ldcLjqQ}} \\
    & \hyperref[sec:7ZGXT0f30Z0]{《致餘民及流散者:給香港基督徒的神學八課》第二季第2課|20230325 [7ZGXT0f30Z0]} & 2023-03-25 & \href{https://youtube.com/watch?v=7ZGXT0f30Z0}{\texttt{ 7ZGXT0f30Z0}} \\
    & \hyperref[sec:gRf39gjSNbM]{《致餘民及流散者:給香港基督徒的神學八課》第二季第3課|20230512 [gRf39gjSNbM]} & 2023-05-12 & \href{https://youtube.com/watch?v=gRf39gjSNbM}{\texttt{ gRf39gjSNbM}} \\
    & \hyperref[sec:RYCxV16hfwM]{《致餘民及流散者:給香港基督徒的神學八課》第二季第4課|20230528 [RYCxV16hfwM]} & 2023-05-28 & \href{https://youtube.com/watch?v=RYCxV16hfwM}{\texttt{ RYCxV16hfwM}} \\
    & \hyperref[sec:I6Z1WA7E0RA]{《致餘民及流散者:給香港基督徒的神學八課》第二季第5課|20230625 [I6Z1WA7E0RA]} & 2023-06-25 & \href{https://youtube.com/watch?v=I6Z1WA7E0RA}{\texttt{ I6Z1WA7E0RA}} \\
    & \hyperref[sec:2QyWxsVtL8E]{《致餘民及流散者:給香港基督徒的神學八課》第二季第6課|20230924 [2QyWxsVtL8E]} & 2023-09-24 & \href{https://youtube.com/watch?v=2QyWxsVtL8E}{\texttt{ 2QyWxsVtL8E}} \\
    & \hyperref[sec:JKdFzjAsLZY]{《致餘民及流散者:給香港基督徒的神學八課》第二季第7課|20231101 [JKdFzjAsLZY]} & 2023-11-01 & \href{https://youtube.com/watch?v=JKdFzjAsLZY}{\texttt{ JKdFzjAsLZY}} \\
    & \hyperref[sec:w1NzLUX2_GE]{《致餘民及流散者:給香港基督徒的神學八課》第二季第8課|20231126 [w1NzLUX2\_GE]} & 2023-11-26 & \href{https://youtube.com/watch?v=w1NzLUX2_GE}{\texttt{ w1NzLUX2\_GE}} \\
    & \hyperref[sec:6BvsvxnAGsQ]{【流堂崇拜】兔年生肖道|20230128 [6BvsvxnAGsQ]} & 2023-01-28 & \href{https://youtube.com/watch?v=6BvsvxnAGsQ}{\texttt{ 6BvsvxnAGsQ}} \\
    & \hyperref[sec:9wD1Nl5xbCs]{【網上受難節聚會】|反思疫症下的基督受難|20220415 [9wD1Nl5xbCs]} & 2022-04-15 & \href{https://youtube.com/watch?v=9wD1Nl5xbCs}{\texttt{ 9wD1Nl5xbCs}} \\
    & \hyperref[sec:vVSfVrNtuKk]{【網上崇拜】虎年生肖道|20220205 [vVSfVrNtuKk]} & 2022-02-05 & \href{https://youtube.com/watch?v=vVSfVrNtuKk}{\texttt{ vVSfVrNtuKk}} \\
    & \hyperref[sec:GmLFDCSkId4]{【這是最好的時代:給香港基督徒的神學八課】第1講:亂世中才明白甚麼是「基督徒」|20210522 [GmLFDCSkId4]} & 2021-05-22 & \href{https://youtube.com/watch?v=GmLFDCSkId4}{\texttt{ GmLFDCSkId4}} \\
    & \hyperref[sec:OTk7WEa_w50]{【這是最好的時代:給香港基督徒的神學八課】第2講:究竟好消息有多好?|20210619 [OTk7WEa-w50]} & 2021-06-19 & \href{https://youtube.com/watch?v=OTk7WEa-w50}{\texttt{ OTk7WEa-w50}} \\
    & \hyperref[sec:dNWjC8vnhS0]{【這是最好的時代:給香港基督徒的神學八課】第3課: Let’s flow|20210726 [dNWjC8vnhS0]} & 2021-07-26 & \href{https://youtube.com/watch?v=dNWjC8vnhS0}{\texttt{ dNWjC8vnhS0}} \\
    & \hyperref[sec:d_aSxcuQPus]{【這是最好的時代:給香港基督徒的神學八課】第4課:亂世的靈性修持|20210822 [d\_aSxcuQPus]} & 2021-08-22 & \href{https://youtube.com/watch?v=d_aSxcuQPus}{\texttt{ d\_aSxcuQPus}} \\
    & \hyperref[sec:akT8yKiTNTo]{【這是最好的時代:給香港基督徒的神學八課】第5課:一根刺的人|20210918 [akT8yKiTNTo]} & 2021-09-18 & \href{https://youtube.com/watch?v=akT8yKiTNTo}{\texttt{ akT8yKiTNTo}} \\
    & \hyperref[sec:K2_OK28IM68]{【這是最好的時代:給香港基督徒的神學八課】第6課: Passion|20211018 [K2\_OK28IM68]} & 2021-10-18 & \href{https://youtube.com/watch?v=K2_OK28IM68}{\texttt{ K2\_OK28IM68}} \\
    & \hyperref[sec:hq6PGyJ3aBs]{【這是最好的時代:給香港基督徒的神學八課】第7課:今日的「我」推翻昨日的「我」|20211121 [hq6PGyJ3aBs]} & 2021-11-21 & \href{https://youtube.com/watch?v=hq6PGyJ3aBs}{\texttt{ hq6PGyJ3aBs}} \\
    & \hyperref[sec:HS1KRCnzG5o]{【這是最好的時代:給香港基督徒的神學八課】第8課:跟隨耶穌的一百萬個可能|20211218 [HS1KRCnzG5o]} & 2021-12-18 & \href{https://youtube.com/watch?v=HS1KRCnzG5o}{\texttt{ HS1KRCnzG5o}} \\
  11:1-44-20220430 & \hyperref[sec:7_6qfdylGjE]{【網上崇拜】突破「越位」|約11\_1-44|20220430 [7\_6qfdylGjE]} & 2022-04-30 & \href{https://youtube.com/watch?v=7_6qfdylGjE}{\texttt{ 7\_6qfdylGjE}} \\
申命記尼希米記 30:1-4 & \hyperref[sec:hkGSf0_Eoow]{【網上崇拜】同是天涯流落人|申命記30\_1-4;尼希米記1\_8-9|20220813 [hkGSf0-Eoow]} & 2022-08-13 & \href{https://youtube.com/watch?v=hkGSf0-Eoow}{\texttt{ hkGSf0-Eoow}} \\
撒母耳記上歷代志上 31:1-6 & \hyperref[sec:aPLQjM9J0JY]{【流堂崇拜】執迷不悔|撒母耳記上31\_1-6;歷代志上10\_13-14|20230311 [aPLQjM9J0JY]} & 2023-03-11 & \href{https://youtube.com/watch?v=aPLQjM9J0JY}{\texttt{ aPLQjM9J0JY}} \\
\end{xltabular}
}
\newpage



\section{腓利門書 1:1-25}
\label{sec:J3EQacUFDFI}
\textbf{【網上崇拜】再見…再相見 | 腓利門書1\_1-25 | 20210206 [J3EQacUFDFI]}
\newline
\newline
連結: \href{https://youtube.com/watch?v=J3EQacUFDFI}{\texttt{ https://youtube.com/watch?v=J3EQacUFDFI}} ~~~~ 語音日期: 2021-02-06 
\newline
\newline
\hyperref[sec:OfpcASuB51M]{\small{< < < PREV SERMON < < <}}
~
\hyperref[sec:index_chronic]{\small{[返順時目]}}
~
\hyperref[sec:index_scriptual]{\small{[返順卷目]}}
~
\hyperref[sec:yt29IwiNPNk]{\small{> > > NEXT SERMON > > >}}
\newline
\newline
腓利門書 1:1-25
\newline
\begin{longtable}{cl}
\hline
\hline
章節 & 經文 (和合本修訂版)\\
\hline
1:1 & \begin{tabularx}{0.7\textwidth}{X} 為基督耶穌被囚的保羅,同弟兄提摩太,寫信給我們所親愛的同工腓利門、 \end{tabularx} \\ \\ \relax
1:2 & \begin{tabularx}{0.7\textwidth}{X} 亞腓亞姊妹,和我們的戰友亞基布,以及在你家裡的教會。 \end{tabularx} \\ \\ \relax
1:3 & \begin{tabularx}{0.7\textwidth}{X} 願恩惠、平安從我們的父神和主耶穌基督歸給你們! \end{tabularx} \\ \\ \relax
1:4 & \begin{tabularx}{0.7\textwidth}{X} 我在禱告中記念你的時候,常為你感謝我的神, \end{tabularx} \\ \\ \relax
1:5 & \begin{tabularx}{0.7\textwidth}{X} 因聽說你對眾聖徒的愛心,和你對主耶穌的信心。 \end{tabularx} \\ \\ \relax
1:6 & \begin{tabularx}{0.7\textwidth}{X} 願你與人分享信心的時候,能產生功效,讓人知道我們所行的各樣善事都是為基督做的。 \end{tabularx} \\ \\ \relax
1:7 & \begin{tabularx}{0.7\textwidth}{X} 弟兄啊,由於你的愛心,我得到極大的快樂和安慰,因為眾聖徒的心從你得到舒暢。 \end{tabularx} \\ \\ \relax
1:8 & \begin{tabularx}{0.7\textwidth}{X} 雖然我靠著基督能放膽吩咐你做該做的事, \end{tabularx} \\ \\ \relax
1:9 & \begin{tabularx}{0.7\textwidth}{X} 可是像我這上了年紀的保羅,現在又是為基督耶穌被囚的,寧可憑著愛心求你, \end{tabularx} \\ \\ \relax
1:10 & \begin{tabularx}{0.7\textwidth}{X} 就是為我在捆鎖中所生的兒子阿尼西謀求你。 \end{tabularx} \\ \\ \relax
1:11 & \begin{tabularx}{0.7\textwidth}{X} 從前他與你沒有益處,但如今與你我都有益處。 \end{tabularx} \\ \\ \relax
1:12 & \begin{tabularx}{0.7\textwidth}{X} 我現在打發他回到你那裡去,他是我心肝。 \end{tabularx} \\ \\ \relax
1:13 & \begin{tabularx}{0.7\textwidth}{X} 我本來有意將他留下,在我為福音所受的捆鎖中替你伺候我。 \end{tabularx} \\ \\ \relax
1:14 & \begin{tabularx}{0.7\textwidth}{X} 但不知道你的意見,我不願意這樣做,好使你的善行不是出於勉強,而是出於自願。 \end{tabularx} \\ \\ \relax
1:15 & \begin{tabularx}{0.7\textwidth}{X} 他暫時離開你,也許是要讓你永遠得著他, \end{tabularx} \\ \\ \relax
1:16 & \begin{tabularx}{0.7\textwidth}{X} 不再是奴隸,而是高過奴隸,是親愛的弟兄;對我確實如此,何況對你呢!無論在肉身或在主裡更是如此。 \end{tabularx} \\ \\ \relax
1:17 & \begin{tabularx}{0.7\textwidth}{X} 所以,你若以我為同伴,就接納他,如同接納我一樣。 \end{tabularx} \\ \\ \relax
1:18 & \begin{tabularx}{0.7\textwidth}{X} 他若虧負你,或欠你甚麼,都算在我的賬上吧, \end{tabularx} \\ \\ \relax
1:19 & \begin{tabularx}{0.7\textwidth}{X} 我必償還。這是我—保羅親筆寫的。我並不用對你說,甚至你自己也虧欠我呢! \end{tabularx} \\ \\ \relax
1:20 & \begin{tabularx}{0.7\textwidth}{X} 弟兄啊,希望你使我在主裡因你得益處,讓我的心在基督裡得到舒暢。 \end{tabularx} \\ \\ \relax
1:21 & \begin{tabularx}{0.7\textwidth}{X} 我寫信給你,深信你必順服,知道你所要做的,必過於我所說的。 \end{tabularx} \\ \\ \relax
1:22 & \begin{tabularx}{0.7\textwidth}{X} 此外,還請給我預備住處,因為我盼望藉著你們的禱告,必蒙恩回到你們那裡去。 \end{tabularx} \\ \\ \relax
1:23 & \begin{tabularx}{0.7\textwidth}{X} 為基督耶穌與我一同坐監的以巴弗問候你。 \end{tabularx} \\ \\ \relax
1:24 & \begin{tabularx}{0.7\textwidth}{X} 我的同工馬可、亞里達古、底馬、路加也都問候你。 \end{tabularx} \\ \\ \relax
1:25 & \begin{tabularx}{0.7\textwidth}{X} 願主耶穌基督的恩與你們的靈同在。 \end{tabularx} \\ \\
[1ex]
\hline
\hline
\end{longtable}
$^{1}$各位姐妹平安.
不知道你們是否感受到現場的upbeat.
因為一連幾首快歌.
我就唱不完這麼多首快歌.
因為我沒有氣.
怕趕到的時候不夠氣.
我感受到勁敝隊帶給我們那種向前的感覺.
我今天的訊息都是希望有一個向前的動力.
或者在訊息當中和大家去想想.
新的一年新的開始的時候.
move on是甚麼回事呢.
今天我的訊息是說「肥你門」.
「肥你門」書裡面的內容是一段書信.
一會兒會和大家詳細了解內容.
其實今天選「肥你門」的訊息.
其實想呼應上個星期「做音目者」.
是說關於舊的事情做一個終結.
或者say goodbye.
或者當再要去面對舊.
面對曾經的時候.
其實下一步應該可以做些甚麼呢.
所以今天講題是「再見」.
一個是past tense.
但是如果再相見的時候.
是一個現在式或者future tense的時候.
對我們來說我們應該要預備些甚麼呢.
我就選了「肥你門」書.
「肥你門」書是一個比較短的書信.
對於你來說可能你很久之前看過.
因為很短.
一章裡面只有25節就完結了整件事.
很快別人問你讀了多少章書信.
你就讀完一卷.
很快就讀完一卷.
所以如果你沒看過不要緊.
一會兒有聲音導航.
用廣東話口語去讀一次「肥你門」書.
四分多鐘而已.
那把聲音都挺好聽的.
但不是我錄的.

$^{41}$希望在一直看著經文的時候.
經文是新日本的文字.
但用廣東話口語的方法去看這篇書信.
我們先聽聽讀經的內容.
「肥你門」書第一章.
為基督耶穌的緣故.
而成為囚犯的保羅.
和我們的弟兄提摩泰.
寫信給我們親愛的弟兄和同工肥你門.
還有姐妹阿肥亞和我們的戰友阿基布.
以及在你家裡追集的教會.
願恩典和平安.
從我們的父神和主耶穌基督臨到你們.
我在祈禱當中看到你的時候.
因為我聽說你對神的指紋是有愛心.
和對主耶穌是有信心.
願你能夠有效地和人分享信仰.
更深明白我們所做的一切美善的事.
都是為了基督.
弟兄啊.
所有的聖徒的心既然因為你而得到暢快.
我也因為你的愛心得到更大的喜樂和安慰.
我雖然靠著基督可以大膽地吩咐你.
去做那些你所應該做的事.
但是我情願憑著愛心來懇求你.
看在我這個上了年紀的.
現在又為了基督耶穌的緣故.
成為了囚犯的保羅的份上.
就是為了我在困所當中所生的兒子.
阿尼謀懇求你.
他過去對你來說是沒有什麼益處.
但是現在對你和對我來說.
都是有益處的.
我現在叫他自己.
即是我的心肝回到你那裡.
其實我本來是想將他留在身邊.
等他在我為福音被困所的時候.
代替你來服侍我.
不過,未得你的同意.
我並不願意這樣做.

$^{81}$叫你向我作出的任何幫助.
都不是出於勉強.
而是出於自願.
或者,他之前暫時離開你.
就只是為了讓你可以永遠地得到他.
不再是一個奴隸.
而是遠遠超過一個奴隸.
是作為一個親愛的兄弟.
他對我來說尤其是這樣.
更何況是對你呢?.
無論是在身體上.
還是在主裡面.
都是這樣.
所以,你如果還當我是你的拍檔.
就請你接納他.
好像你接納我一樣.
如果他有什麼得罪了你.
還是有什麼欠了你.
請你全部都算在我的身上.
我是一定會還給你的.
這句是我保羅親手寫的.
其實,不用我說你都知道.
連你自己都欠了我.
兄弟,我真的希望.
你使我在主裡面因你得益.
讓我的心在基督裡面得到安慰.
我寫信給你.
深信你一定會照著這樣做.
而你所要做的.
一定是會超過我所要求的.
另外還有一件事.
就是請你為我預備住的地方.
因為我盼望藉著你們的祈禱.
我可以蒙恩到你們那裡去.
為基督耶穌的緣故.
和我一起被囚禁的以巴忽問候你.
和我一起同宮的馬可.
阿彌達古.
底瑪.
路加.

$^{121}$也都問候你.
願我們主耶穌基督的恩典.
跟你的靈同在.
阿們.
(音樂結束).
不知道大家聽了廣東話的口語旁述.
是不是很親切呢.
當有時候讀經有些乏味.
覺得文字很生硬的時候.
你試一下讀經用我們本土語.
香港廣東話去讀.
其實感覺會來得親切.
我自己小小分享.
很多時候讀書信的時候.
都用對話的形式.
我想那種感覺你就會明白.
受書人或者寫書的人.
其實那種情懷是什麼.
有時候我們讀經.
可能會著重一些字面.
或者一些字意上的分析.
但其實你記住.
當初寫信的時候.
是將整個信仰的反省.
或者在信仰當中的那種得著.
和受書的人明白.
《肥理門》就是一本這樣的書信.
如果你看保羅的書信當中.
監獄書信有四卷.
是哥羅西書,爾忽所書.
《肥理門》書和《肥立比》書.
四卷當中《肥理門》書是最短的.
但我自己覺得這卷書.
能夠納入在新約的全書中.
有很濃厚的神學信息.
而當中能夠應用在我們生活中.
剛才我不知道你聽旁述的時候.
你感受到什麼.
我很喜歡這個表達方法.
因為它將那種情心告訴你.

$^{161}$譬如他用了心肝.
又或者在語氣的時候.
那種意境受書人會明白到.
兄弟啊,我知道這樣寫.
你也會照做的.
正正就是很熟悉受書人和他那種關係.
而將最底層的底牌告訴受書人.
整本書剛才你看下去的時候.
就是一個人做錯事.
離開了《肥理門》.
然後出走的一個奴隸阿尼西姆.
整件事就是去到保羅之後.
阿尼西姆他自己經歷了福音的改變.
輾轉就去到保羅的身邊.
成為協助保羅的其中一人.
詳細的情況這卷書上沒有提及.
但是在西書第四章裡面.
這個叫推基古和阿尼西姆的名字就出現.
因為保羅在獄中寫了的書信.
其中有兩卷書.
一卷是哥羅西書和《肥理門》書.
這兩卷書就透過這兩個人送達去.
推基古就帶著阿尼西姆.
去到哥羅西達地方.
因為當我們知道哥羅西書裡面的描述.
就出現了肥理門這個人.
知道他是來自哥羅西教會的人.
所以保羅就猜推基古和阿尼西姆.
就回到哥羅西達地方.
就將哥羅西書帶給哥羅西教會.
同樣都將阿尼西姆帶回去給肥理門.
就是肥理門書.
所以整件事肥理門就和阿尼西姆相遇.
接著阿尼西姆就停留在肥理門家.
推基古就再帶著哥羅西書去哥羅多教.
繼續他的行程.
第三卷書就帶去爾弗所教會.
《監獄書信》就是這樣完成了這三卷的帶動.
精妙的地方就是.
為什麼保羅要阿尼西姆.

$^{201}$要回去他原先離開的肥理門家.
整件事就是.
奴隸出走當然有他的原因.
奴隸出走有原因的時候.
現在奴隸出走了.
去了一個新地方.
有一個新生活.
但保羅為什麼要將阿尼西姆帶回去肥理門.
這個就是整卷書要上澄清的地方.
對於你來說.
既然都有一個新恩一葉.
為什麼還要回去舊的地方.
我們曾經在Flo Church開堂的時候.
應該是一百年.
當我們辦了兩個大型聚會的時候.
就是回魂夜和解慰之後.
我們中間都做了一些大型的團契.
分別有不同的人可以聚集.
一起去聊聊天.
去到開堂的時候.
都被人問一個問題.
當人們來到Flo Church之後.
你們有聚會.
到最後他們療完傷.
你們是不是會推回原本的教會.
如果這樣你會不會來Flo Church.
你應該會轉頭吧.
我剛剛才離開一個地方.
你現在就推回去那裡.
Flo Church是不是一個療傷的地方呢.
但我們都跟他說.
那件事不是那麼簡單.
從來離開一個地方有它的原因.
在整本書上是否沒有解釋.
當日亞歷西伯是因為什麼事離開菲利門.
沒有解釋.
很難揣測.
當然有些聖經書或者作者解釋.
其實可能保羅是想這件事.
去煽動奴隸制度.

$^{241}$但我不覺得這件事是主打.
反而就是沒有著墨的地方.
是不是不重要呢.
我不覺得不是重要.
之後保羅如何處理亞歷西伯和他的舊宿主.
即是菲利門之間的一種關係.
我今天重點就是.
前關係是沒有了.
但再見面的時候.
我們的心態是怎樣.
我今天想跟大家一起去想這個問題.
不代表一個訊息是可以完成解決的方法.
但我覺得我們很多時候要面對我們的曾經.
你舊有很熟悉的朋友.
你斷了關係.
你在街上見到他的時候.
你會點個頭問個安.
之後就走.
還是當看不到立即看電話.
還是怎樣呢.
還是你會很主動和他聊天呢.
如果那個是普通朋友可能可以.
很熟的朋友你會發現有些嫌隙離開.
如果那個是你的前度呢.
你是否活得比你好.
立即整理好一點.
整理好一點髮型.
是要讓你看到我現在多魅力.
你是用什麼心態去再相遇呢.
整件事對我們來說.
是很複雜的事.
亞歷西伯在文字裡面的描述不多.
不知道他做過什麼.
可能他真的偷了錢.
人們說他偷了錢.
因為保羅要給他錢.
或者交帳.
但是不是偷錢呢.
不知道.
但總之結果就是他出走了.

$^{281}$出走了一個地方的時候.
他就信了主.
然後就認識了被囚的保羅.
保羅就adopt了他.
成為他的helper.
然後就繼續去form工作.
就差他不和他form事工.
保羅剛才在經文裡面說.
原本想收回來自己用.
但因為我覺得不是我的.
我要將他交回給你.
所以亞歷西伯做了什麼.
是不知道.
他們中間的糾結地方是不知道.
另外一邊想看看肥李門.
肥李門是什麼人呢.
經文裡面開頭說的一件事就是.
其實他有愛心.
一開始經文就說他愛心很好.
很有名聲.
而他家裡就成為家庭教會的聚腳點.
他是有錢.
一個patron.
他可以促養奴隸.
他一定是有一定的財力.
所以他有錢又有愛心.
教會又在這裡參與的時候.
其實他的名聲很好.
那你可能就會說.
他名聲這麼好.
走得那個奴隸當然是頑皮.
就是不識貨.
很快就會順理成章.
他走就是他不對.
他一定是偷錢走的.
有時候我們聽別人的故事.
或者我們了解案情的時候.
我自己很多時候都說.
聽一邊那個一定是苦主.
就因為在淑齡的角度.

$^{321}$我們覺得人的雜性.
凡是那件事要靠近身體追討的時候.
一定會推卸.
就不覺得自己是自身是內.
一定是自身是外.
想排他那種追討.
如果這樣看.
阿彌西武是不OK的.
但是肥李門是不是因為這樣說就很OK呢.
我們都不知道.
因為不是實質的案情.
問題就是第三個.
保羅.
保羅有什麼事要做這個中介呢.
其實出走了就OK.
那他就給他一個新生活.
他離開了.
就算是《生命記》第23章都在說.
如果一個奴隸是離開了.
他的主人的時候出走了.
能夠接待他的人.
應該要給他一個安全屋住.
不應該送他回去.
舊約是這樣教的.
但是整件事保羅沒有接待他.
應該說保羅有接待他.
但是他沒有一直和保羅住.
保羅明白了之後.
反而送他回去.
那會不會推阿彌西武回去性死呢.
或者其實是令到阿彌西武很尷尬.
當我再見到我的宿主的時候.
我應該要什麼呢.
我可能帶著一些虧欠.
帶著一些對他來說的不適.
再一次面對他的時候.
是很尷尬的.
我現在和你隔著.
屏幕可能不太感受到.
但可能面對面的時候.

$^{361}$你就難搞一點.
你自己問一問自己.
你們很多都是適婚年齡.
現在疫情不能喝酒.
如果去喝酒的時候.
當然我不是叫你去X那裡喝酒.
我的point就是.
你的朋友會結婚的.
你們有common friend的.
你會問他去不去.
他去我不會去.
不要浪費我的位置.
我不做人情的.
整件事就是你不想再見到他.
這個就是point.
你一次可以不見.
但是突然而來在街上碰到的時候.
又不是回到剛才的東西.
回到剛才的東西就是扮作不到.
這個是一個很親密的關係.
如果相對沒有那麼親密.
但是有很多共事的關係.
就是舊教會的人.
舊教會的人.
你和他合作了很久.
又一起有很多服侍的空間.
大家相處日子都不少.
你再見回舊教會的人.
你是用什麼方式呢.
你有沒有重整過自己的心態.
以至再面對舊友的關係呢.
還是你覺得我們沒有交集的人生當中.
我們老死不相往來.
就當不認識這個人呢.
你是會用選擇這個方式.
我無意淡化當中你們受到的傷害.
你們面對的嚴規.
或者中間很糾纏的那種來來回回的對話.
這些完全是真實的.
我明白.

$^{401}$但是在過去Full Church開堂這兩年.
我聽到最近這些故事.
很多.
而Cold Call的時候.
每個星期我都會打電話給一些留名.
62745377那些.
即是我們的電話號碼.
我都和他們談的時候.
都聽一次他們的故事.
我說所有故事都是真實來經歷的.
但是故事的全部都是真實的.
但是故事之後.
今天要面對同樣都真實的.
你來到Full Church可能是療傷.
你來到Full Church可能是一個過渡.
但是如果你真的再面對舊有的教會的人.
你會用什麼心態.
又或者你用什麼形式呢.
這個就是我希望在《肥理門書》裡和大家去高呼.
其實保羅做了什麼呢.
保羅他要了解什麼呢.
以至他覺得他這麼大膽.
他居然推翻一個能夠出走的人.
回到他自己的宿主那裡.
其實他沒理由要推翻他.
你看看《肥理門》又不缺錢.
少一個奴隸什麼所謂.
其實不需要爭一個人.
反而有人在幫你.
豈不是更好.
他可以不用帶他.
所以帶他回去的時候.
冷嘲熱諷又不是.
自知不理又不是.
其實保羅在做什麼.
其實是好心做壞事.
但是從另一個角度來說.
我想問你.
你有沒有試過抱一個人.
不是抱那種抱.

$^{441}$是因為那個人你很認識他.
有些事他做錯.
他現在悔改了.
他想回頭你抱他.
你力戰他.
以至他不要緊我支持你.
我跟你一起去.
你有沒有試過抱一個人.
如果你試過.
你知道通常都是中招.
是不是.
出街都是很正常的.
中招是廣東話.
你會發覺一件事.
你要包底.
你要多熟悉這個人.
你有多熟悉這個人.
其實你又不是很熟悉.
但你又要抱他.
你憑什麼覺得你很肯定要抱他.
還有一個大前提.
可能他不是知道事實的全部.
你一直都聽著亞歷西姆說.
為什麼要出走的時候.
其實相反是怎樣.
其實保羅不是不認識菲利門.
你明白嗎.
剛才書信都說到菲利門是他認識的.
所以兩邊廂都知道.
保羅就做一樣東西.
就是將亞歷西姆帶回去給菲利門.
這個位置是有些問題的.
問題是還有很多細節位.
可能都要想一下.
如果真的要抱這個.
要推他回去.
或者帶他回去的時候.
是不是融合到呢.
你需要打底嗎.
所以保羅就寫一封信給亞歷西姆.

$^{481}$和推基股一起.
如果是亞歷西姆.
就去見菲利門.
你先看這封信.
你先不要罵我.
你先看這封信.
接著他讀完這封信的時候.
就等菲利門的反應.
整個過程就是這樣.
但我想同一樣說.
在剛才我開初的時候.
這封信帶去的時候.
其實保羅都同樣是帶哥羅西書和伊弗所書去.
哥羅西書裡面說了一些內容.
想和大家做一個平衡.
我們看看哥羅西書第一章的內容.
在螢幕上你會看到.
哥羅西書第一章第二十字.
既然藉著他在十字架上所流的血成就了和平.
便藉著他叫萬有.
無論是地上的天上的都叫自己和好.
你們從前與上帝隔絕.
因著惡行心裡與他為敵.
如今他藉著基督的肉身受死.
叫你們與自己和好.
都成了聖潔.
沒有瑕疵.
無可責備.
把你們引到自己面前.
保羅在哥羅西書第一章說的一件事就是.
我們以往是不可以的.
不可以的意思就是我們純粹因為罪的緣故.
我們背逆.
違背了上帝.
但耶穌基督十架寶血流出.
以致我們能夠透過這個中寶的角色.
成為一個橋樑.
可以與自己.
你不要再憎恨自己.
你可以與自己和好.

$^{521}$而可以在上帝面前有一個無可責備的開始.
同樣第二卷.
以弗所書第二章.
就是什麼?.
同樣是《加護書信》說到的.
第十六節.
「基督十字架滅了冤仇別藉著這十字架使兩下歸為一體.
與上帝和好了.
並且來傳和平的福音給遠處的人.
也給近處的人.
因為我們兩下藉著他被一個聖靈所感得而進到父面前」.
這段紅色的字說的是什麼?.
是十字架滅了冤仇.
不會因為我們大家的嚴拘.
大家所謂怨恨因為十字架.
就可以兩下.
即是成為一個橋樑.
大家能夠拼合一起.
而這個拼合是因為聖靈在當中做事.
保羅強調一件事.
不是靠仁義.
靠Lobbying.
大家兩邊做一個中介.
做一個協談.
那個不是沒用.
最重要的不是靠這件事.
最重要的是.
你知不知道自己要為自己負責.
十字架就掩飾了這件事.
第二件事就是.
你們兩個走在一起.
是有十字架在當中做一條和.
同樣聖靈在當中也有做事.
大家能不能感受在聖靈當中怨言虛偽.
聖靈的督責告訴你.
當初你為什麼說這句話.
這句話實在是傷害了對方.
而這句話你說的時候.
你沒有顧及對方的感受.
而你做的時候.

$^{561}$都沒有了解後果的嚴重度.
以致你說了你收不回.
之後你更加沒有跟進.
這完全都是雙方決裂.
又或者第一支箭未處理.
又再射第二支箭的問題.
聖靈就是不斷用這些方式去提醒.
彼此.
以致大家再見面的時候.
你願不願意做這個復和.
上帝既然已經跟我們開始復和.
而我們又願不願意當中去多走一步呢.
所以去到以弗所知第二章第十一節.
他說什麼.
所以你們當紀念.
你們從前按肉體示愛幫人.
也稱為未受割禮的.
這明願是那些憑人手在肉身上.
稱為受割禮的人所起的.
那時你們與基督無關.
在以色列民以外.
在所應許的諸約上是局外人.
並且活在世上沒有子望沒有上帝.
你們從前遠離上帝的人.
如今卻在基督裡靠著他的血.
已經得親近了.
這裡說的是什麼呢.
其實是耶穌基督做了個中介.
你們以前是沒有關係.
因為兩個都是不同的人.
但因為你們同領基督之後.
你跟基督有關.
你已經是同一個血緣.
你同一個血緣的時候.
你們就彼此親近.
你還記得剛才讀經的時候.
讀到一個點就是.
他已經不是一個奴隸.
他是你的弟兄.
大家都是同一個弟兄的時候.

$^{601}$大家就開始要走近一點.
要學習彼此接納.
剛才那封信就是這樣說的.
在這裡我們停一停.
剛才說的《玉加木書信》有三卷.
除了《肥立比》以外.
其實《哥羅西書》和《爾忽所書》中所說的訊息.
跟今天肥利門所說的內容是很相近的.
意思是什麼呢.
就是《哥羅西書》和《爾忽所書》.
我們都稱為是教會藍本的書卷.
是說一個教會的墓會.
那個墓養的神學立論.
或者那個教義上.
怎麼看基督的死.
基督的代屬.
基督的能力.
基督所謂.
不是所謂.
基督要我們去明白.
上帝要做中保的角色是什麼.
肥利門書正正就告訴我們.
那件事不是留在頭腦.
實際上生活我們都要經歷的.
實際上生活當中.
我們都學習到.
大家本身都是來自兩個個體.
但因著基督的緣故.
我們能夠彼此相近.
是基督的血覆蓋了我們的錯.
覆蓋了大家的錯.
跟著大家學習彼此相近.
是基督的血做了我們的中介.
以致我們能夠聯合.
這個就是肥利門正正經歷著這件事.
保羅不是不知道舊約的教導.
要給他一個更新的機會.
不要再回到舊的地方.
不是.
他的反應就是.

$^{641}$我們不再需要受前約的捆綁.
我們要面對新的關係建立.
我們要面對就是.
要看回自己的過去.
面對自己的將來.
我知道有些人舊的傷痛之大.
是很難一下子去平息的.
無論是身體上的傷痛.
心靈上的傷痛.
關係上的破裂.
這些很難有一個比較.
但是我想說的是.
那個已經是事實.
那個都是曾經.
聽過很多弟兄姊妹的分享.
很坦白說.
你見過輔導.
你見過教睦.
你見過你的閨密.
你的老闆.
說過很多次.
說完之後.
你是得到舒緩.
但是我都很希望.
要說多少次.
但是每次說的時候.
其實你都是不斷不開心.
正如我上一個月的訊息就是.
那你想怎樣重整呢?.
你想怎樣不再受核制呢?.
很多時候.
有弟兄姊妹找我聊天的時候.
或者是在.
他重述他離開教會的難處的時候.
我都說.
實際上我沒有什麼可以幫到你.
你願意告訴我.
我一定會聽.
但是聽完之後.
我都會和你輕輕分析一下.

$^{681}$或者有些情況.
我可能會建議一下.
但是我說最主要就是.
你怎樣再看這些關係.
你怎樣再看這些傷痛.
如果不是.
你下一次再說.
其實對你來說.
幫助在哪裡呢?.
你說當然有幫助.
起碼我知道.
有人會明白.
我說是呀.
我說不止你.
很多人都會有這樣的經歷.
換了同路人.
但是都要再一次面對自己困難的時候.
我就說.
其實主耶穌都知道.
但是主耶穌.
其實最想和你說.
其實我覆蓋了.
我覆蓋了.
就讓我覆蓋了的東西.
你就開了新一頁.
如果不是,你想一想.
上帝不是白祖嗎?.
還是你不相信上帝會覆蓋?.
還是你一直都不想覆蓋?.
我相信.
好像我上個月說的訊息就是.
其實靜下來的時候.
你想處理的.
你經常自己一個哭.
你不會開心的.
你和別人說的時候.
也不一定有出路的.
所以保羅.
既然他知道肥利門是一個什麼人.
是一個有愛心的人.

$^{721}$他知道他說.
其實我可以賜老賣老的.
但是他沒有這樣做.
他知道肥利門是一個什麼人.
他也知道亞歷西姆是一個什麼人.
而亞歷西姆現在已經信了主.
他改變了.
他能夠成為保羅的同工.
他就說.
不如我和你回去吧.
好不好?.
你也要面對這件事情的.
亞歷西姆就回去了.
因為他經歷了聖靈的更新.
親愛的兄弟姐妹.
你的傷痛仍然是.
傷疤仍然有.
但是傷疤成為你的核心.
還是什麼呢?.
其實要自己再認清這個關係.
和那個過去.
在經文裡面要提一個東西.
就是.
我們再看看下一頁.
就是第十一節經文.
他從前與你無益.
但如今與你我都有益處.
我現在打發他親自回里那去.
他是我心上的人.
就正如剛才我說.
他們雙方面.
保羅是雙方面都知道.
他現在就叫他們回去.
從前與你無益的意思是什麼呢?.
從前與你無益就是.
其實亞歷西姆的意思就是有益處.
就是有益處.
名字的意思就是有益處.
從前與你無益處.
現在有了.

$^{761}$可能以前的關係是不OK的.
但現在他已經成為一個新做的人.
他已經信了主.
他已經是一個好的福音的協助者.
既然你在家裡做教會的話.
他以前對你無益.
他現在對你有益了.
所以我就打發他回去幫你.
這個就是保羅想跟費利門說得清楚.
這個再下去就是.
以福所書第二章裡面的經文.
因他使我們和睦.
將兩下合而為一.
拆毀中間隔斷的牆.
而且以自己身體廢掉冤仇.
就是那記載法律法上的規條.
並為要將兩下直接自己做成一個新人.
如此便成就了和睦.
你看到嗎?.
在以福所書裡面說的情況就是費利門的情況.
因他使我們和睦.
其實原文的意思就是.
他是我們的和睦.
意思是基督就是我們的和睦.
如果沒有基督.
如果亞里斯多德沒有母信主.
故事並不是這樣的.
如果上帝沒有做事.
冤仇仍然存在.
所以正正是因為上帝做了事.
他就拆了中間的牆.
用自己身體廢掉冤仇.
就是記載法律上的規條.
從來有些東西是綁住了.
但因為基督的緣故就拆解了.
所以成就了和睦.
就是再一次面對舊的關係.
我去到這裡仍然很希望你明白.
我無意淡化或是小看你過去受的傷痛.
但你已經受夠了.

$^{801}$其實上帝已經掩飾了.
你說上帝沒有.
上帝沒有掩飾.
我不知道.
但上帝說.
你煩惱苦擔重擔.
你就去他那裡吧.
我不相信沒有事是上帝不能承擔的.
但重點是你有多少事交給上帝呢?.
還是我一定要生氣.
我一定要看死他.
還是怎樣?.
但你自己的心靈狀態.
精神狀態都不好.
又何苦呢?.
其實最不開心的.
應該是主耶穌.
其實我掩飾你.
你又不接受我掩飾.
你又不明白.
其實你可以有新的開始.
為何你不嘗試呢?.
我相信聖靈仍然會在當中提醒我們.
剛才說到.
總會聽到很多弟兄姊妹的難處和故事.
我都很坦白說.
有些事我沒有實際幫助.
但我通常和弟兄姊妹說.
我不是說你的故事很普通.
但我說你經歷的教會問題.
你自身的問題.
其實Flow Church有很多弟兄姊妹.
都有相近的經歷.
我的意思是.
你其實不乏同路人.
但他們為何會來Flow Church.
或者為何Flow Church會開展.
因為他們相信上帝已經掩飾了.
而他們可以選擇開啟新的頁面.
為何不兩者都一樣呢?.

$^{841}$就是接受上帝的掩飾.
然後開啟新的頁面.
正視舊有的關係.
我相信保羅提醒亞歷西姆.
面對舊有做的不是.
同樣就是面對舊有的關係.
我和你一起回去吧.
這樣就回去了.
所以你會看到.
其實保羅不是隨便的.
他雖然在那25節的經文中.
其實有些事是很認真的.
你會看到下面.
下面的時刻就是第15節.
保羅和他說.
「他暫時離開你」.
「或者叫你永遠得著他」.
保羅說的一個很簡短的濃縮版.
就是他曾經走過的.
就是他曾經走過.
但他現在回來了.
這個我們不知道為甚麼.
但上帝就給我們機會再次證明.
他就回來了.
我不知道你怎樣看一些.
舊有和好如初的關係.
和好如初.
我經常覺得這個字很迷.
是否真的沒有嚴規呢.
你再看回他.
你會有15,16猜想.
他會否像以前那樣對我呢.
你跟他說話是否還可以沒有隔膜呢.
有時我都不容易的.
我都被人騙過.
有時有人說潘Sir你很好.
可能很多人對你很好.
但我被人騙的時候.
難道我又要你吃苦閒罰嗎.
我不是比較.

$^{881}$但在過程當中.
有些事情你可能都會有些猜想.
有些猜測.
保羅說一個訊息.
或者我們在經文中聯想一個情況.
就是當再見到那個人的時候.
你可能有很多想法.
但正正那個想法對你來說.
一個提示就是.
不要再走舊有對大家的方法.
這個很重要.
就是吵架之後.
我相信總有吵過架.
吵架之後你會發覺有些說話.
是不可以在這個人面前再說的.
那你就會被記.
這個就是尊重.
這個就是我上一個月想認識的體面.
不夠體面的月花加級體面就是.
你懂得避重就輕.
最重要是你懂得欣賞人.
這個很重要.
從前他有益的時候.
現在就從事無益.
現在有益了.
他暫時離開你.
叫你永遠得著他.
其實都是說一個情況.
就是在那個時候學不到.
現在你學到了.
你之後就可以和他在一起.
整件事不容易去了解的.
但總會有些人經一事長一智.
學習怎樣和人相處.
在經文下去的時候.
你會看到.
就會是第十六節.
他現在不是再是奴僕了.
他高過奴僕是親愛的弟兄.
就剛才讀過那段經文.

$^{921}$因為他同樣是蒙基督的血所灑.
但大家都是因為基督的血.
能夠拆了那個圍牆.
大家能夠成為以合為一.
這個是上帝給他的恩典.
所以第十七節.
你若以我為同伴就收納他.
如同收納我一樣.
收納這個字又再出現了.
在保羅的信息裡面.
我不知道大家有沒有記得.
在保羅書上這個字.
其實出現了兩個段落.
一個段落就是.
在我講閃避球.
即是十月第二次講的信息裡面.
就是在講Nabano這個字.
就是接納.
接納什麼意思呢?.
就是你要學習.
懂得明白對方的難處.
以至你懂得接納.
保羅在羅馬書裡面.
十四章在講.
有人說吃肉可以.
有人說吃肉不行.
但吃肉可以吃肉不行.
對我們來說.
那個是自由的選取.
是信心的問題.
信心從來都有先後信心.
有人有信心先.
有人有信心後.
有人信心大.
有人信心小.
有人信心重.
有人信心輕.
從來都是一個.
即是場景上的比較.
沒有絕對的標準.

$^{961}$你就接納他有先後.
有能力的差異.
有大小的問題.
即是你接納.
這個字Nabano.
保羅這次仍然是用.
仍然是在講一個主動.
主動Pros Nabano.
好像羅馬書第十五章.
你們接納我.
如同基督接納我們一樣.
是主人蘇主動去接納我們每一個.
是主人蘇主動掩蓋我們所有的人.
保羅這次是在講.
如果你接納我是你的同工.
你就再主動接納他.
就是這樣.
保羅是要求或者是懇求.
讓你們明白到.
你就主動接納他.
他現在回來了.
你就主動接納他.
我看這裡的時候.
感受到保羅想起自己.
保羅在《少林傳》開初的時候.
他是一個很差的人.
沒有人會這樣和他一起.
見到他就走.
怕他被逼迫.
鬥死他.
當巴拿巴要抱著保羅.
去引戰使徒的時候.
人們說巴拿巴傻的.
但巴拿巴說不是的.
我領受了上帝.
但沒有人會理解.
但巴拿巴就抱著保羅.
所以今天保羅抱著阿彌西姆.
其實就是我曾經被人接納.
我也很希望這個回轉.

$^{1001}$同樣歸在基督名下的弟兄.
他也被你接納.
你走一步吧.
你願不願意再正視這個關係.
這是雙向的.
我開頭也說.
保羅是認識菲利門.
保羅也是和阿彌西姆同工.
兩個都認識.
保羅就做這個醜人.
做一個中介去做協調這件事.
我過去很坦白說.
有一些教務同工問我.
志剛,我有些會友去了你那裡.
我就說,哪一個是你的會友.
我也不認識.
我說真的.
我認識不少人.
但我真的不認識哪一個是.
因為事實上.
我們很坦白說.
每一個在Flow Church參與的弟兄姊妹.
我們都沒有問.
你的戶籍在哪裡.
你的組織在哪裡.
你從哪間跳過來.
我們從來都沒有這樣問.
我不知道現在留言打什麼.
我一整晚回去看都害怕.
但我仍然很坦白說.
其實.
有些是我知道的.
我也跟他說.
最大的問題不是在哪裡.
我們跟Flow Church的弟兄姊妹說.
這個不存在大家互通消息.
是互通消息的.
這個不存在台底交易.
不是這個意思.
但是.

$^{1041}$我也跟他說.
那個同工問我的原因.
不是想把你帶回去.
不是.
也知道不能把你帶回去.
但我認真跟大家說.
這個案件很敏感.
我也不會說名字.
但我想告訴你.
那個同工問我的原因.
他其實不是想你回去.
他是告訴我.
我見過他.
他現在比以前開心了很多.
我覺得.
真的很不同.
那種傷痛其實是相輔相成的.
是兩個都有不開心的.
我希望你不要覺得.
以片蓋全.
覺得每個人都是這樣.
我不是這個意思.
我又不無意說情.
告訴你一定是.
令到他的情緒好像很激昂.
我不是這個意思.
我只是告訴你.
那個是真實的.
有些教務同工是知道.
自己的同學有些限制.
有些原因 有些前因後果.
是很複雜.
以致你離開了.
但其實他知道你活得好.
其實他也很安心.
那又何苦你仍然帶著.
我說弟兄姊妹.
又何苦你仍然帶著舊的傷痛.
而一直都不復原呢?.
為什麼不做兩者呢?.

$^{1081}$就是你接受了.
那個已經是曾經.
當你再面對那種關係的時候.
上帝已經覆蓋了.
你開始了新的頁面.
保羅很想我們明白到.
聖靈一直在我們心裡動工.
開我們的眼睛.
看到有新的轉向.
同樣告訴我們.
耶穌基督已經做了那個中介.
我不會迷信一個信息.
就能夠改變太多.
但我希望你想想.
其實保羅寫哥羅西書.
以法索書給教會.
讓他們明白到基督的重要性.
肥利門書就告訴他們.
實務上我們怎樣運用這件事.
以致我們在關係上可以建立.
最後保羅很認真的.
保羅認真講什麼呢?.
第十八節.
他不是想著轉過去就算的.
如果他真的虧負了你呢.
他欠你什麼呢?.
我包底.
他真的知道亞歷西武有些事.
可能做得不對的.
所以他會跟他說.
他會負責任.
這個承擔的角色.
他一定會償還.
他會證明這是他的親筆寫.
所以最重要的就是.
那件事不是回到他手上.
說一句話就算了.
不是這個意思.
重點是.
保羅講得很清楚.

$^{1121}$仍然要對這件事處理好.
不要讓這件事重蹈覆轍.
也不要讓這件事.
笑笑口就過去.
從來都沒有看輕這件事.
一定要面對.
第二十節就是.
上帝會令我們有Rejoice喜樂.
這是耶穌基督的工作.
就是他的功能.
到最後有一節.
我寫給你.
他深信你會信服.
其實保羅很清楚.
大家雙方的為人.
大家處事的情況.
而他不是.
他不是就這樣.
做一個中介就算了.
你看到二十二節.
他說我會去的.
他會去家房的.
他會跟進那件事.
其實之後大家的關係是怎樣.
正正如在Folk Church.
我們所有的牧者可以說.
我們其實在小組.
花很大的氣勁.
去處理舊教會.
或者一些破碎了.
在其他層面的關係.
不要讓那些事情再核製你.
以致你得不到.
基督給我們豐盛的生命.
所以我們很多牧者.
都會做跟進的工作.
做個人的關心.
如果不準備就慢慢來.
但就不可以置之不理.
有時候大家太過迷信時間.

$^{1161}$以為放下時間慢慢去調探.
就會好.
但其實在時間當中不做事.
其實那件事是不會好的.
最後跟大家說再見.
在再見一種合作關係.
大家可能不會再合作.
真的可能不會再合作.
不會再合作.
就算了.
但如果再相見的時候.
我們仍然有一個弟兄姊妹的關係.
因為仍然是主內的弟兄姊妹.
我不是奢求大家要重修舊好.
一如以往那樣.
繼續可以無縫銜接.
做到事.
有時候這些太過天真.
那就停在合作關係上.
但在相見的時候仍然是主內的弟兄姊妹.
因為耶穌基督給我們一個聯合.
保持適當的社交距離.
大家在當中停在那裡.
但不要再被冤仇去克制你.
成為你每次見到的時候.
你又睡不著.
你又不開心.
你又怨自己.
都說不去.
又去.
所以整件事.
我希望大家明白.
緋天門面對亞里西姆是怎樣呢?.
我不知道.
但如果.
給大家一份功課試試做.
如果你回信給保羅.
你會怎樣呢?.
亞里西姆帶著這封信去見緋天門.
緋天門看完這封信的時候.

$^{1201}$亞里西姆在那裡.
接著保羅不在那裡.
如果你回信給保羅.
你會寫些什麼呢?.
有什麼選項呢?.
大家一起說說吧.
你也想想吧.
選項一.
保羅我不行.
你收回這個我不要.
選項二.
保羅我明白的.
但我接受不了.
你給我一點時間吧.
還是選項三.
保羅我真的明白到.
原來.
呼音真的會改變一個人的.
當初他出走.
有他的原因.
但呼音改變了他.
他能夠成為你的同工.
今天你帶他回來成為我的同工.
我感受到.
呼音本是上帝的大能.
是要救一切相信的.
我感受到上帝給我的能力.
我就接收他.
在呼音的工作.
繼續下去.
多謝你啊保羅.
三個選項.
你仍然可以選擇.
但你會選擇哪一個呢?.
願意在重整這個訊息裡.
你想想.
你選擇哪個選項來重整.
2021年已經去到二月了.
六分一了.
不要再等了.

$^{1241}$上帝與我們同在.
願主祝福你.
\newpage



\section{}
\label{sec:GmLFDCSkId4}
\textbf{【這是最好的時代:給香港基督徒的神學八課】第1講:亂世中才明白甚麼是「基督徒」|20210522 [GmLFDCSkId4]}
\newline
\newline
連結: \href{https://youtube.com/watch?v=GmLFDCSkId4}{\texttt{ https://youtube.com/watch?v=GmLFDCSkId4}} ~~~~ 語音日期: 2021-05-22 
\newline
\newline
\hyperref[sec:hx9eq2tkbx4]{\small{< < < PREV SERMON < < <}}
~
\hyperref[sec:index_chronic]{\small{[返順時目]}}
~
\hyperref[sec:index_scriptual]{\small{[返順卷目]}}
~
\hyperref[sec:k_0V9RXlAGE]{\small{> > > NEXT SERMON > > >}}
\newline
\newline
$^{1}$(這段影片由於拍攝時鏡頭不足,所以不得不重複).
(香港,香港,你永遠是塵夢鄉).
(香港,香港,你那些笑容).
(山頂看小島水裡頭,處處換上新裝).
(看看那海鷗飛過自由港).
(海邊看小島充滿裝,處處搖眼風光).
(這個市區的吸引沒法擋).
(日日聲香港,香港,太有望童年夢想).
(香港,香港,叫我不以為望).
(香港,我身心的抱憾,這裡讓我增長).
(有我喜歡的親友共陽光).
(路上人在跑過馬桿,看見我欣賞).
(這裡有許多好處沒法講).
(日日聲香港,香港,你永遠是塵夢鄉).
(香港,香港,你那些笑容).
(香港,太有望童年夢想).
(香港,香港,叫我不以為望).
(香港,香港,你永遠是塵夢鄉).
每個年代都有每個年代的神學.
作為土生土長的香港人.
我們似乎正在經歷一個最差的年代.
不過往往在最差的年代.
我們才能夠經歷福音信仰的最好.
就讓我們一起從聖經裡學習.
如何做這個年代裡的香港基督徒.
這是最好的時代給香港基督徒的神學百科.
(神學百科).
粉絲們晚安.
歡迎大家參加我們Flo Church的神學講座.
這是最好的時代給香港基督徒的神學百科.
這是一個什麼呢?.
我稱之為一個有道具有服裝的主學.
或者叫做一個錄了的神學講座.
這個也是我們Flo Church裡的一個新嘗試.
我們希望能夠有一個神學教導的百科講座.
但我們想認真一點.
大家有現場的頂字幕可以聽.
也有網上的頂字幕可以聽.
所以我覺得這個很突出.
平時都是用Zoom來講.

$^{41}$但這個有道具可以清除.
這個就是我們新嘗試.
有些頂字幕問.
這個是什麼呢?.
Flo Church現在變了.
Flo Church現在是否要做主學,門訓呢?.
是否我們要和其他教會一樣.
要做門徒訓練呢?.
我會說是Yes and No.
不是那些.
不是你一般認識的門訓.
不是你以前裝備過的門訓.
就像今天的題目.
我們不是要講門徒.
而是講基督徒.
所以最多叫做機訓.
我們要思考的是什麼呢?.
其實香港這十年有很多門訓.
不知道大家有沒有參加過門訓?.
有沒有試過被門訓過?.
都有吧?.
門訓不知為何成為了教會很流行的東西.
門徒成為了一個更加高層次的東西.
我聽過一些人說.
教會需要的不是參崇拜基督徒.
而是我們要更多門徒.
我們不是做基督徒.
而是做門徒.
門徒更加厲害.
我覺得不是的.
門徒只是其中一個名字.
所以今天我想講的是一個字.
叫做基督徒.
反而我要強調.
我們對於門徒的執著.
很明顯是來自於香港教會這幾年.
很多時候大家一起.
傳聞大練訓.
大家一起去訓練門徒.
好像一開始甚麼都不是.

$^{81}$就變成門徒才是.
我覺得這是教會以前很事工化的方式.
所以Folk Church不是做這些東西.
Folk Church不是純粹做那種門訓.
門訓是一種課程.
一種教會增長的方式.
一個事工.
門徒當然是很重要.
門徒是跟隨耶穌的人.
門徒是一群背十字架的人.
去學習耶穌基督的人.
不過我覺得門徒不可以純粹參加一個訓練.
或者聽完一些東西就算了.
反而我覺得更加要思考的是.
門徒本身是一個很特別的意思.
你知道潘Sir以前是教基教的.
基督教教育.
他說過一個很有意思的說話.
他說門訓是甚麼呢?.
門訓是一個身份價值的教育.
去告訴我們我們是甚麼.
我們是甚麼.
我們應該做甚麼.
這是一個很重要的學生.
不是一個套餐.
不是一個教會增長的訓練.
所以如果是這樣的話.
Folk Church不是做這種門訓.
不過如果作為一種身份價值的教導.
我覺得Folk Church的弟兄姊妹.
是很需要有這樣的認識和實踐.
我覺得如果有門訓的話.
門訓其實是一種永遠帶著時限的東西.
門訓是帶著一個時期.
在某個年代某個時代裡面.
去面對如何去做門徒.
即是說門徒是會過期的.
可能你以前學過的甚麼三寶.
或者是甚麼價值觀重整之甚麼.
那些是很重要.

$^{121}$但那些東西其實應該是按著不同的年代.
我們重新來去再學習和裝備過.
因為我們跟隨主耶穌.
跟隨主耶穌的時候.
我們就在不同的年代裡面.
在這個年代裡面.
即是說在一個後光光法的年代裡面.
我們學習如何去做一個門徒.
因此門徒不是一個永久有效的東西.
如果是的話.
它就不是一個很新的東西.
門徒是帶著時限.
舉個例子.
1934年潘福華.
他面對納粹德國的時候.
他說納粹主義將會改變.
即是帶來德國教會中建.
他就開始想到一個想法.
不如我用登山補訓來做一個門訓.
後來這個想法成為了甚麼書.
就是跟隨基督這本書.
所以每個年代每個時空的基督徒.
我們都重新來去思考.
我們應該如何來做門徒.
如何來做基督徒.
而不是純粹一套套裝.
不是一套裝出來.
大量生產一套套裝.
這是我們Full Church.
為何會想做一套門訓.
而是我們想面對國安法的時候.
我們如何能夠在這個年代.
重新來學習思考做基督徒.
所以基督徒的身份.
這個成為我們今天的主題.
但事實上初期教會都面對著這樣的主題.
初期的一班猶太人.
跟隨耶穌的人.
當耶穌升天之後.
這班人其實本身是甚麼人.

$^{161}$是一班猶太教徒.
一班猶太人.
所以他們出現了一個很特別的身份危機.
他們要分別出.
他們不是一班以前過往.
聖殿時期的猶太教.
而是一班一個嶄新的信仰.
一個群體.
一班相信基督徒的群體.
所以你問 Who am I?.
即是我是誰.
所以這個身份價值.
正成為初期教會裡面.
很重要的一個題目.
如果你們看《釋經》的話.
你會知道.
《釋經》裡面有很多不同的字眼.
關乎於一個門徒的自稱.
不知道大家是否認識.
第一個是甚麼.
第一個叫Adelphoi.
即是弟兄.
當然有姊妹.
這個是我們經常見到.
聖經裡面出現的一個門徒.
自我身份的一個名稱.
第二個是甚麼.
就是Pistero.
一個信徒.
強調一班相信的人.
一班相信耶穌基督.
復活.
是上帝神的兒子的一班人.
第三個就是Hagios.
即是一個聖徒.
強調一班分別為聖的人.
他們是不同於世界裡面的一班聖徒.
第四是Ecclesi.
大家都認識.
就是教會.

$^{201}$他們用這個字眼.
來稱呼自己的群體.
我們被招聚在一起去結集的一班人.
第五個當然是大家認識的.
Mathetis.
就是門徒.
這班基督徒經常出現.
是最基本.
最常見的一班人.
稱呼自己的字眼.
我們是學生.
我們是門徒.
最後我稱之為甚麼呢.
The Way.
即是道.
我稱之為道友.
即是這班人是學習真道的人.
相信真道的人.
無道友.
尋道友.
甚麼道友都好.
而今天要講的.
就是第七個字.
這個字在聖經裡面.
出現了三次.
一個很特別的字眼.
就是叫做基督徒.
Sorry.
不是.
不好意思.
這是別人這樣稱呼我們.
改了.
不會這樣稱呼.
耶L.
不是這個.
下一個.
是叫做基督徒.
基督徒不是一種.
今天所稱呼的佛教徒.
道教徒.

$^{241}$伊斯蘭教徒.
基督徒是一個Biblical的字眼.
即是這個字眼出現在你去辨別.
不同宗教.
信甚麼教.
基督教.
基督徒.
不是的.
我懷疑大家都誤會了意思.
基督徒的字眼在聖經裡面.
出現了三次.
雖然不多.
但是一個很特別的神學意義.
不是泛指任何相信基督教的人.
就是基督徒.
甚麼是基督徒呢.
沒事.
我想讀一讀.
大家一起讀.
現場觀眾一起讀.
就是叫做.
Christianos.
我一直讀成這樣.
Christianos.
基督徒.
我們是Christianos.
所以這個字在聖經裡面.
賦予我們第七個身份價值.
除了信徒,聖徒,門徒之外.
是一個很特別的身份價值.
今天我們會討論這個題目.
這個字是怎樣來的呢.
這個字其實是一個拉丁文化了的字.
即是它是將一個拉丁文grammar.
這樣去辨別了.
假設我們說有姜濤.
我不是搞爛笑話.
姜濤不是基督徒的朋友.
姜濤的追隨者叫甚麼名字.
叫姜粉.

$^{281}$(姜糖).
Sorry.
不好意思.
我不是.
叫姜糖.
我還未說不是.
姜糖.
如果按以前就叫姜濤濤.
因為是一群追隨.
屬於永死忠心不變的跟隨姜濤的人.
叫姜糖.
以前也是這樣.
這個字是一個辯字.
所以將這個字.
第一個字是希律.
根據馬克方的字.
叫做同希律一黨的人.
這個字其實是來自於一個.
不是那麼長的字眼.
就是Helodys.
Helodynus.
所以從希律這個字.
加上一個ion.
就變成了希律黨的人.
其實是一個比較短的字眼.
就是姜濤和姜糖.
所以用這個邏輯的話.
Christos.
就是基督這個字.
一群一樣.
你當他們是姜濤一樣.
即是誓死永不變的愛.
誓死跟從.
屬他的.
即是跟隨他的.
這樣一群人.
所以就叫基督徒.
Christianos.
所以基督徒的字就是這樣來的.
一個這樣的字眼.

$^{321}$這個字其實本身帶著一種政治意味.
因為只有凱撒和希律會這樣稱呼.
所以當人.
特別是一些不信基督的人才會這樣說.
即是一群外面的人才會叫他們做.
一個這樣的基督徒字眼.
有一個軍事意味.
一個政治意味.
即是希律,凱撒和自稱維尼塞亞的基督.
下一個故事我們今天就開始看.
我們看三段經文.
在聖地牢那三段有關出現過基督徒的字眼.
打段不熟一點.
就是小人傳第十一章第25,26節.
巴拿巴又往大掃羅.
掃出了就帶他到安提阿去.
他們足有一年的功夫和教會一同聚集.
教訓了許多人門徒稱為基督徒.
是從安提阿起手的.
這個就是第一段用到基督徒這個字的一段經文.
很明顯你見到這個群體是被外面的人來稱呼.
這個從來都不是他們自稱.
他們自稱做大英姐妹.
自稱做信徒或者門徒.
外面的人用政治術語.
用這個稱呼來稱呼這群人.
並且見到聖經這樣說.
門徒就稱之為基督徒.
所以我們問究竟門徒厲害一點還是基督徒厲害一點.
大家今天覺得門徒當然厲害一點.
門徒跟隨耶穌.
其實不是.
反而門徒是一個比較普通的字眼.
我覺得門徒基本上是.
最基本的東西.
不是什麼高峰層面.
不過當門徒變成基督徒的時候.
當門徒這群人升到基督徒的層次的時候.
其實一個很特別的東西就在當中不同了.
出現了第三者.

$^{361}$即是當一群人去見到他們的時候.
而他們覺得這群人這麼似他.
並且這麼願意在他面前見到他的時候.
這群安提安的信徒就被稱之為基督徒.
所以你見到這是一個不是基督徒的自稱.
而是別人永遠出現了第三身.
一個他者的自稱.
一個稱呼.
這群基督徒被別人掛著基督的名字.
從別人口中稱之為基督徒.
第二段就是《小人傳》第26章第24至29節.
保羅在審判中的一個自辯.
當保羅在自辯中的時候.
他講了一大輪很多的見證.
他自己在大馬士革中被耶穌的光弄得盲了.
被呼召並且受苦.
當保羅講了一大輪之後.
他開始說:保羅,你癲狂了嗎?.
你的學問太大反對你癲狂嗎?.
保羅說:非斯都大人,我不是癲狂.
我說的乃是真實明白話.
王也曉得這些事.
所以我向王放膽直言.
也深信這些事沒有一件向王隱藏的.
因都不是在背地裡做的.
接著我們聽到一個回應.
阿基伯王說:你信先知嗎?.
我知道你是信的.
當保羅這樣問他的時候.
阿基伯就跟保羅說:.
你想小微一勸.
你不要勸我了.
你省一點吧.
你想我做基督徒嗎?.
所以這是一句很戲語的說話.
都是來自一個不信主的人的說話.
基督徒在初期的時候.
永遠來自一個不信耶穌的人的口.
出現在保羅的見證裡.
當保羅不斷強調自己受苦.

$^{401}$在他面前見證基督耶穌的時候.
外面的人就被他的行動所回應.
就說:你想我做基督徒嗎?.
記住這個字不是我們今天說的基督徒.
而是帶著政黨的意味.
一種軍事的意味.
一種當時文化的背景.
你是否想我成為一個基督黨的黨員?.
所以當時保羅所做的.
其實是向一個人見證耶穌基督.
並且在一個政權底下.
讓其他人都見到耶穌基督.
在羅馬政權的一個官的底下.
見到耶穌基督這個王.
是這樣的一個事情.
所以基督徒永遠都是帶著一個見證.
並且是從他人口中的見證的結局.
到了第三段.
這段是彼得前書的經文.
第四章第十二至十六節.
到了後期的時間.
當外面的人不斷去稱呼這班人為基督黨員.
基督徒的時候.
教會裡的人都開始用到這個字眼.
不過有一個特別的角度.
他說:親愛的弟兄!.
有火煽煉到你,燃到你們.
不要以為奇怪,不要以為要歡喜.
因為你們是與基督一同受苦的.
你們是一起受苦的.
所以你們在榮耀顯現的時候.
可以歡喜快樂.
你們若為基督的名受辱罵便是有福的.
因為上帝榮耀的靈常駐在你們身上.
你們中間卻不可有人因為殺人,偷竊,作惡.
或管閒事而受苦.
若為了基督徒受苦.
卻不要羞恥.
都要因者明歸天上的上帝.
當基督徒第一次用到基督徒這個字去自稱的時候.

$^{441}$是帶著一種受苦的概念.
來稱呼基督徒這個字眼.
而受苦永遠都是來自於對外公共的見證.
在政權底下,這班人被稱為基督徒.
甚至乎在政權的高官底下.
指控他們為基督徒.
然後他們慢慢接受了這個字.
在受苦的情況下.
稱自己為基督徒.
所以基督徒在這三段經文中.
告訴我們一個很重要的意義.
與門徒不同.
門徒是一班學者.
一個學習耶穌的人物,一個學生.
但當你稱之為基督徒的時候.
你會有一個更重要的意味.
你在社會裡成為一個基督的見證.
並且願意隨時受到很多苦難.
我再看後面的文獻.
這是在聖經以後的教會文獻.
有一個文獻叫《十二兆遺訓》.
這也是早期的字眼.
我們搜尋一下所有有關基督徒的字眼.
第十二章有關接待客人的教導.
他用了兩個很接近的字眼.
情繪說,如果有人有客人的話.
你就去接待他.
大概是這個意思.
如果他是基督徒的話.
你就不要單單叫他去偷難.
因為基督徒是不會偷難的.
應該是更好的回報.
而不是純粹的自我開放.
但如果他是自我開放的話.
他就不是基督徒了.
他就變成了假托基督詐取私利的人.
中文看不明白是甚麼意思.
我們看下一頁.
下一頁就說更加重要的字眼.
一個就叫Christiano.

$^{481}$就是基督徒.
一群接待基督徒.
另一個就叫Christian Poros.
很相似的兩個字眼.
前者就叫基督徒.
或者基督黨羽.
或者基督的跟隨者.
一個就叫靠基督謀生的人.
這個字眼出現了很多次.
就是靠著基督耶穌的名字來謀生.
即是甚麼呢?.
我來去黐飲黐食.
或者用基督的名字.
來為自己利益的一群人.
所以稱之為基督玩家.
更加好一點的字眼.
很會玩的一群人.
即是掛著基督的名字.
來為自己的著數而生存.
我不知道你是否一個這樣的人.
如果基督徒不是單單是耶穌那麼簡單.
而是一種見證.
一種為主願意受苦的意思的時候.
我們掛著基督的名字的時候.
究竟我們是為了自己.
好會玩的樣子.
還是真真正正的去做一個基督徒呢?.
所以我想說的是.
基督徒這三個字真的不是開玩笑的.
不是純粹一個最普通的字眼.
反而是一個最不普通的字眼.
因為這個字正正是意味著甚麼呢?.
意味著我們基督徒本身的一個公共性.
當你被耶穌基督呼召為基督徒的時候.
這個名稱帶著一個很重要的社會公共性.
我們看看其他的文獻.
另一個就是約瑟夫.
可能大家都聽過.
一個猶太歷史學家.
他在公元95年96年的時間.

$^{521}$他就去到這樣的地方.
這群人全部都不信耶穌.
所以基督徒這字眼在教外文獻中比性更多.
這群基督徒.
基督黨語.
星之維.
這群所謂一直以基督為名的人.
這群不信的人會這樣理解.
政權就會這樣理解這群人.
一群以基督為名的人.
至今仍未消失.
當教會.
當這群門徒.
這群聖徒.
仍然在社會裡.
存在被不信的人看到的時候.
基督徒這字眼又再次出現.
當然.
更可怕的是.
基督徒這字眼成為了.
尼祿迫害教會裡的字眼.
當時羅馬的一個官員.
一個歷史學家.
用了Christianos這字眼.
來形容這群人.
是六四年羅馬大火的滋長者.
因為這群基督徒的緣故.
令羅馬起火.
因此從此之後.
基督徒就被迫害.
所以基督徒被迫害.
基督徒這字眼.
本來就是一個這樣的context.
這不是一個普遍的字眼.
而是一個政權.
來迫害基督徒的字眼.
基督徒在社會裡.
成為了公共層面意義的時候.
一個這樣的字眼.
亦成為了一個受苦的字眼.

$^{561}$後來整個羅馬帝國.
迫害基督徒三百多年.
其實大家有沒有想過.
基督徒是因為犯了什麼法?.
基督徒需要什麼來被捕?.
當然不是國安法.
是什麼?.
一個很簡單的法律.
叫做基督徒之名.
Norman Christian.
單單是因為你自稱為基督徒的時候.
單單這個事情.
就足以令到你犯法.
我就能夠來行政的權力.
來逮捕你.
迫害你.
甚至殺死你.
如果你再次強調.
如果你忘記了.
一個很普通的基督徒的意思的時候.
基督徒正正永遠都是承擔著這些的.
因為基督徒之名.
因為你在我面前作見證.
當時的社會就要逮捕這些基督徒.
因此我想說的就是.
基督徒永遠都是帶著一個.
很重要的社會性.
我們Full Church這套所謂的門訓.
為什麼會用這套課來開始呢?.
因為你看看我們很多時候市面上的材料.
什麼什麼價值觀.
或者是真金法.
基督徒什麼的.
全部都是用個人出發開始的.
你有沒有想過.
你今天如果上天堂的話.
你行不行?.
這是第一課第一句問題.
基督徒對於價值觀.
都是很自我開始的中心.

$^{601}$這是可以的.
我不是反對.
但是我想開始我們所謂的課程.
我們不如重新來思考.
我們不用我們個人的得救來開始.
因為基督徒這三個字.
不是從我們得不得救來開始.
而是從別人的眼中.
別人的口中來開始.
我們的存在.
正正是受苦的見證來開始.
所以我們這套所謂的門訓的起始點.
或者是我們整個課程的起始點.
永遠我們說.
不如我們寫一套.
不以個人做開始.
而是重新去認識我們的身份.
我們是門徒.
我們是信徒.
我們是教會.
我們是聖徒.
不過一個很重要的身份.
就是我們都是基督徒.
當你背負著這個名字的時候.
基督徒這個名字並不是說玩的.
古佛華在他被監禁臨死前十五個月.
寫了一首詩叫做《Christen und Heiden》.
翻譯為《與非基督徒》的一首詩.
一首獄中寫出來的詩.
這首詩我覺得寫得很有意思.
大家留意它吧.
因為它本身德文也有些押韻.
我讀出來.
.
當人來到上帝面前.
在他的困苦裡面.
這群人是任何人.
所有的人.
祈求的是甚麼?.
祈求都是一些很基本的東西.

$^{641}$麵包,幸福,快樂.
尋求的是甚麼?.
都是一些很基本的.
很普通人所尋求的東西.
在疾病裡面.
在聚債裡面.
在面對死亡裡面的拯救.
《與非基督徒》其實都是一樣的.
我們沒有甚麼特別.
我們都是有些甚麼事.
都是祈禱,尋求幫助.
我們人生都是尋求麵包,快樂,幸福.
這樣的東西.
不過第二節就給了我們一個很特別的分岔.
它說.
.
第二節裡面說到人去到上帝面前.
不過那個他字你見到嗎?.
那個他字令了另一個他.
不是一個普通人的那個受苦.
而是哪個他?.
是耶穌基督的那個他.
這群人是去到耶穌基督的苦難裡面.
發現耶穌基督在十字架裡面的貧窮羞辱.
同樣都是沒有麵包.
沒有一個棲息之地.
親眼目睹耶穌基督被凌辱.
在軟弱的裡面,在罪惡的裡面,在死亡的裡面.
唯有基督徒同行在這個苦難裡面.
通常說真正能夠去分別出基督徒和非基督徒的.
不是甚麼得救的名單.
並不是你缺了自媒.
並不是你是不是將來會上天堂.
並不是你是不是預定被揀選的人.
是否你得了救恩,是否得了拯救?.
不是.
真正去define基督徒和非基督徒的是甚麼?.
是基督徒願意和基督耶穌一起去經歷這個苦難.
親眼看見這個苦難並願意同行在這個苦難裡面.
第三節.

$^{681}$.
第三節,一個很大的逆轉.
上帝來到人的面前.
同樣這個人並不是分基督徒或非基督徒.
上帝仍然用餅來填滿我們心靈的滿足.
記住我們是所有的人.
為世上所有的人釘在十字架上.
基督徒,非基督徒.
同樣上帝都是去憐憫,愛惜.
所以你看見潘福華臨死前的十五個月裡.
一個很大的感觸.
我做了這麼久基督徒.
其實真正的基督徒.
不是因為我上了慕道班裡的第一課.
或德國聖經裡的第一課.
因為我能上天堂.
而是因為我願意去參與基督的苦難.
因為我在世界裡願意作基督的見證.
所以我想說的就是.
基督徒並不是一個名字這麼簡單.
基督徒是一個動詞.
「弟子妹,你今天基督徒了嗎?」.
「你今天下班後有沒有基督徒到?」.
不是說你是否基督徒.
而是你有沒有基督徒到.
你有沒有基督徒的父母?.
你有沒有基督徒的同事?.
你有沒有在香港社會裡.
做一個基督徒.
去成為一個基督徒?.
這是我們全聖教裡.
很想每一個弟子妹.
重新去聆聽的問題.
我們有沒有在香港社會.
在這個年代裡去做基督徒?.
去做基督徒?.
上帝呼召我們.
主要不是神學院的呼召.
不是更加高層次的呼召.
耶穌基督從來都是一次次地呼召我們.

$^{721}$就在你十多年前.
幾年前信主的時候的呼召.
耶穌基督呼召我們去成為基督徒.
不是單單說給你一份救恩.
一開始就搭上了.
或者說一開始就帶著這種召命.
基督徒這三個字.
已經是一個很高層次的呼召.
全聖教弟子妹我們去學習的.
這個開始說很簡單的第一課.
它是一個非常嚴肅的第一課.
耶穌基督的呼召.
是從你剛剛信主那一刻的開始.
呼召你去成為他的見證.
我自己的母會的牧師很有趣.
他有一個很特別的見證.
他說他在一個陪靈會裡信耶穌.
然後他在一個報道會裡納至去侍奉上帝.
剛好相反.
一般來說都是在報道會裡信耶穌.
陪靈會來去決止.
來去呼召成為傳道人.
他剛好相反.
在報道會裡被蒙召.
在陪靈會裡信主.
我也是這樣想的.
回想十幾年前的報道會的呼召.
那個決止.
其實同樣是耶穌基督呼召我們.
來去做他的見證.
所以這是我們從第一天接受的呼召.
我今天想和大家進行一個很特別的活動.
就是和大家重新來決止.
待會我們吃飯.
我講一句,你講一句.
在褲子裡說自己的名字就可以了.
我們一起來.
雖然是小玩,但不是玩的.
認真的.
這個決止的文是不同的.

$^{761}$不是純粹個人得教的決止.
因為決止是什麼呢?.
決止就是決定立志.
所以不是純粹得救.
而決止是一個立定志向.
我們作為Fold Church的弟兄姊妹.
願意在社會裡.
在這個年代裡去見證主耶穌.
開始祈禱.
我講一句,你講一句就可以了.
親愛的主耶穌.
我,陳偉安.
承認我有罪.
的罪釘在十字架上.
死而復活.
呼召我一生去跟隨你.
成為一個基督徒.
一生去見證你.
奉主耶穌基督的名求.
阿們.
恭喜你,你成為基督徒了.
喂喂喂.
你見到圖案了嗎?.
這麼久還沒到,講了九個字.
通常決而至都會有禮物的.
難得你這麼開心.
先喝一杯吧.
剛才講的都有點硬.
因為我們Fold Church是認真的.
今天很多頂姐妹再次重新思考.
以往覺得門徒是高階一點.
以往教會的門徒訓練是高階課程.
不是每個人都能讀到.
反而今天有一個新的觀念.
我覺得不是只做門徒.
門徒是基本的.
門徒是學校耶穌,跟隨耶穌.
但去到見證,去到社會.
是我們更加需要思考的東西.
在座頂姐妹有什麼反題可以問?.

$^{801}$或者有什麼不太清楚的字.
或者相關的內容都可以討論一下.
可以隨便發問.
主要是學來的.
可以問問題的,不用擔心.
不單單是一個訊息,或者是一個專題.
不清楚的,或者相關的內容大家可以討論一下.
有問題嗎?.
好,那邊.
給他咪高峰.
中間那個,謝謝.
(聽到之後想起《Imitation of Christ》這本書).
(或者這方面).
(我想看看老師有沒有補充).
你說什麼?一開始聽不到.
(《Imitation of Christ》).
(好像有一本這樣的書籍).
(所以《孝法基督》是一本中世紀的靈修書).
(我稱之為神秘主義的書).
(其實是一種強調我們和個人關係).
(因為中世紀那種靈修是很強調).
(我們怎樣能夠和耶穌有團契和契合).
(所以《Imitation of Christ》是一本).
(強調透過我們在生命中實踐).
(當然是一種有靈性和行為的實踐).
(來學習耶穌基督).
(所以這是很重要的).
(我們說跟隨耶穌).
(你覺得跟隨耶穌比較厲害還是學學耶穌比較厲害?).
(有沒有說哪一個比較厲害?).
(其實是兩個不同的說法).
(跟隨耶穌,還是去模仿).
(就像你所說的,他做什麼,你做什麼).
(你分心分心去模仿他).
(耶穌是一個模仿的字眼).
(這是我們靈性上的操練傳統).
(今天我們嘗試說的不是純粹靈命).
(雖然我們知道有一部會說靈命).
(但我們說我們整個神學思考是從第三者出發).
(不是從修身齊家自我平天下).

$^{841}$(首先從個人層面做好).
(靈性好一點,其他OK).
(慢慢做下去,不是這樣).
(而我們整個基督徒的行動).
(其實可以從這點開始出發).
(我覺得我們需要去打破).
(或者用一種新的思考去認識自己).
(不是先搞定自己,永遠都搞不定自己).
(永遠都做到不是最好).
(但我們全靈學).
(我們希望能夠從這樣的角度).
(開始我們的第一課).
(所以這個課程).
(可以說是給初信人).
(或者給我們從商俗人).
(很多年輕人都可以).
(最重要是我們從社會的角度).
(去思考我們的身份).
(其他呢?後面).
(不如我不在教會工作).
(我正在做自己的工作).
(我就不算是靠著基督去謀生的人).
(我何時才知道).
(牧者是靠著基督去謀生的人).
這個問題真是懂得問.
Bruce你回答吧.
我當然是靠基督去謀生.
我當然不斷宣揚基督和去見證基督.
我剛才覺得這個詞.
真的可以是一個專題.
繼續去討論一下.
因為其實John可以補充.
其實那班人.
你見到前面說的.
其實有些事情是不務正業的.
不應該做的事情他做了.
或者應該做的事情他不做.
即是叫做是狐假虎威.
我想牧者就不會是狐假虎威.
起碼Folk Church的牧者不是.

$^{881}$我想這個都是很認真去處理.
其實心態和行動的相符是重要的.
是的,這個我覺得是大家都值得一起去想.
前者基督徒是為主受苦.
但基督賺錢的人是用基督.
大家都carry著基督的名字.
其實大家都是見證.
因為基督徒都是被人看上基督.
那些賺錢的人都是被人看上基督.
但他們的做法是為了自己的利益去著想.
所以他可以是傳道人.
可以是現在問的傳道人.
或者已經是在做傳道人.
或者信徒.
我們今天用Bear這個名字的時候.
其實我們整件事是.
究竟你是想自己.
還是真的去為他作見證呢?.
所以我覺得這個不是純粹傳道人尊重的.
雖然傳道人很容易會犯這個錯.
因為他究竟是賺錢的.
但重點是我們是否去尋求益處.
靠著基督耶穌來尋求自己益處.
所以我覺得大家不妨在小組裡面討論一下這個terms.
自己是否一個很懂得玩基督玩家.
做了基督徒之後.
是否純粹用這個名字來為自己好處去做人呢?.
你好,我叫Yuki.
我剛才聽到的message就是.
基督徒是為基督受苦的人.
基督徒在仕途行傳的年代都會有一些受苦的經歷.
甚至乎是.
總之你自認自己是基督徒你就犯法.
剛剛信基督的人.
通常得到的message就會是.
信耶穌你就得永生.
你會有救恩.
你怎樣去解讀.
原來基督徒就已經是會受苦.
但其實又會說.

$^{921}$但又有一個救恩.
又會是恩聚德赦.
你怎樣解讀這兩個中間的落差?.
落差其實是同時.
我想說同時間一次過.
都是的.
因為聖經裡面有七個不同的稱呼.
我們是聖徒,是信徒,是門徒.
信徒和基督徒.
我想說這七個都是同時重要的.
我不是說其他不重要.
大公說當你從第一天信耶穌之後.
恭喜你你已經得救了.
這當然是真的.
我當然不會說是假的.
不過其實同時間這個呼召已經是開始了.
已經是叫我們去做基督徒.
這個見證.
基督徒當然不是拿苦來生.
我想說這個補充不是受苦的基督徒.
而是他願意為耶穌在社會裡面作見證.
從而肯定會受苦.
並且會承受這些迫害.
這個就是基督徒的定義.
如果問我我會說.
這些東西是同時間出現的.
我們是天父上帝的兒女.
這是我們同神關係裡面.
我們有一課會說這個話題.
我想說同時我們也領受了一個使命.
我們比較少去強調這個使命.
我覺得去到第十堂或是Extra才說.
不如你宣教 傳道人 傳福音.
但其實這個是從第一天開始.
我們的身份價值.
通常說我們的身份價值.
這是我們是什麼東西的問題.
多於說你做了這些就得救了.
或者已經成為教會裡面的頂姐妹.
才去想是不是見證耶穌.

$^{961}$我們嘗試去想.
不如我們從這一點開始去想.
我們Fold Church.
從這一點開始去想我們Fold Church是什麼呢.
我在網上也呼籲了弟兄姐妹問一些問題.
有幾個問題我覺得和見證有幾個關係.
其中一個也有很多人附和.
他們想知道多一點關於基督徒的公共性.
在現在的政治環境下.
怎樣可以實踐到 應用出來.
因為可能在現在這個時代.
很多事情都很無力.
怎樣可以實踐到 見證出來.
有沒有什麼可以再多說一點.
這是一個很好的問題.
我覺得這個也是大家一起去想.
當然我在神樂院是說得多一點.
因為是Offline.
當然在這個時勢裡面.
我們所做到的公共性有多少呢.
但我覺得如果看回聖經裡面的定義.
這個公共性其實不是一定要去到.
我們要去到公共領域層面 政策層面.
這些我們沒有了.
我們立法會也沒有了.
但反而我們能夠讓人去真的見到.
從聖經裡面所說.
在安提阿教會裡面.
安提阿裡面城市的人能夠見到我們.
甚至乎保羅在那個官面前能夠見到我們.
所以在我們的生活裡面.
我們生活裡面能夠讓人見到耶穌基督.
當然所說的不是純粹親戚父母那麼簡單.
而是我們在社會裡面仍然可以去說.
起碼在可以說的時候可以說.
所以我們封出一個崇拜的Online.
即使我們的宣講仍然是在說一些公共領域的東西.
所以這個我覺得我們FourShare都希望能夠做到.
而在底子母層面裡面.
起碼在我們的工作裡面.

$^{1001}$在我們的生活裡面能夠做到這件事.
所以重點是有第三者出現.
讓人見到我們在社會裡面仍然存在.
這個其實我覺得有很多想像.
起碼對我作為一個港道的人.
傳道人來說.
起碼我們FourChurch在一個網絡裡面.
仍然在回應一些社會的事.
這個我們能夠做到其中一個很微小的事.
但底子母所做的.
如果整個群體一起去做的話.
整件事是可以很寬闊.
所以雖然我覺得是世界艱難.
但我都覺得我們讓人見到的.
是讓人探望的.
讓人見到的.
是讓人認識到自己.
所以大家一起去做.
是一件能夠做到的事.
(MC) 這邊有問題.
(Feronica) 你好,我是Feronica.
剛剛看到一張slide很大感觸.
就是vocation呼召.
我看到vocation放假.
基督徒的價值.
就是時刻警醒.
天天背負自己十價.
有時回看現在社會發生的事.
看到有些基督徒.
這樣都可以做到基督徒.
其實怎樣面對自己.
真的沒有了一個vocation的基督徒.
時時刻刻都做到vocation.
我覺得是一個很大的難題.
(MC) 回答歡sir.
其實我剛才想用公共性去回應vocation.
因為其實在我們的信仰層面.
剛才John都說到.
基督徒是一個行動.
不是一個term.

$^{1041}$基督徒是行動的時候.
行動就是回應公共性一個很重要的表達.
譬如我們.
小組或者在我們信息裡.
我們會用我們的崇拜時間.
為一些社會有需要的群體.
或者一些意見去發聲去禱告.
做回我們基督徒.
我們禱告最大的權柄.
另外就是一些鼓勵.
可能用你的財幹.
那些基本上都會說.
但是可能更加具體公共性就是.
現在很多時候金錢的重新分配.
我們都會想我們的錢會放在哪裡.
支援什麼群體.
或者是什麼經濟圈.
或者是什麼對於你來講.
你可以重新去認定一些什麼價值.
這個都是我們公共性的參與.
所以你說那個vocation.
呼召未必真的要在一個崗位上要盡忠.
譬如好像.
我仍然是說.
我們fortune上面沒有內部事工.
一定要你involve的.
但我們常常強調就是.
我們的教會的出口是向著社會的.
所以我們仍然鼓勵.
在一個星期你聽完的訊息.
你在小組上面的支援.
或者是分享.
最重要就是我們回到社區裡面.
可以做到我們從教導上得到的那種價值觀.
的行動參與是什麼.
那個就是fortune上再強調的公共性.
所以如果真的好像今天聽到John講的那個訊息就是.
基督徒不是一個terms.
而是真的將那個行動走出去的話.
基本上你不會是一個vocation放假.

$^{1081}$而你每組的參與都是一個vocation.
這邊也有問題.
Hello.
我想問剛才說的那個教徒的terms.
是由第三者去灌給我們.
好像過往門徒那方面.
好像有很多指標告訴我們.
怎樣做到門徒.
我們怎樣知道自己是在實踐.
做一個基督徒呢.
如果這件事是別人賦予我們的話.
起碼有第三者.
我覺得是真的.
起碼你的生命裡面有一個.
能夠可以讓他看到耶穌基督的人.
這個好像有問.
究竟我自己在一個星期裡.
究竟我做人的時候有沒有一個第三者.
一個第三身的人.
可以看著我.
能夠看到耶穌基督呢.
我想這個很簡單的思維.
已經幫到我們去想.
我們怎樣能夠去做基督徒.
其實有時我們都不覺.
都忙著做一份工作.
或者很多的事.
都想不到這些事.
但當你一刻想到.
其實我是成為那個人眼中的人.
我在想怎樣能夠去.
將基督耶穌的探望.
那份救恩告訴他.
說的不是全副音這麼簡單.
而是能夠你所做的事.
在這個社會裡面.
社區也好 社會也好.
讓人看到.
這個意識是重要的.
永遠帶著一種有第三身的角度.

$^{1121}$來看著自己的行徑.
這個意識.
我們來到一個很特別的地方.
如果你說天父的兒女.
是一個看著天父上帝的靈性.
而我們的生活裡面.
是永遠帶著一個.
有一個人在我心裡展現的人.
你會有這個心上人.
然後你就可以慢慢地.
帶著基督徒的身份.
我可以說我小時候信耶穌之後.
但我其實是信主的見證.
當然我不是說家庭.
但我信耶穌之後.
兩三年內就帶著全家人信耶穌.
那時候我是中五.
我信耶穌之後.
我第一次帶我兩妹回教會.
那時候是我的生日.
他們說今年不收信.
就送給我吧.
不如今天回教會算了.
我說好啊.
然後就帶我回教會.
然後就信耶穌了.
我爸媽也是類似這樣帶著.
三兄妹一起上學.
上學後不如就做好一點.
回家就做好一點兒女.
會看到很不一樣.
然後我爸就信了耶穌.
所以說不是全方位技巧的問題.
而是我們在社會裡面.
我們不可以全方位地說.
我見證耶穌之後也好.
寫到神面也好.
說闊一點也好.
我們是帶著這樣的眼界.
去做人的時候.

$^{1161}$我覺得正正我們就是.
發揮基督徒應該要做的事情.
無論是社會也好.
或者是見證基督真理也好.
公義發聲也好.
我們是帶著這樣的思維去行動.
我回答一下.
剛才John說.
大家可能覺得有些笑話.
就好像他帶家人信主.
很簡單.
其實最主要是你看到改變.
別人會不會察覺你的改變.
第三者有沒有察覺你的改變.
是很重要的.
或者第三者覺得你行動的存在.
是否感受到基督的影響.
近來我身邊也有不少旁聽師.
其中有一位前同工是牧師.
他經常去旁聽.
為出庭的人祈禱.
他表明他是牧師身份.
你介不介意.
不知不覺有些人在當中感受到.
在他最不安的時候.
基督的平安臨到.
在過程當中他表明.
希望你結果如何.
也願意在當中.
上帝的平安.
給你一刻的觸動和平安.
我想這就是第三者.
去看我們在做甚麼.
而我們真的要主動去迎向第三者.
這正正就是.
仍然是那句.
基督徒就是一個行動.
你說不去做旁聽.
我們旁邊的弟兄姊妹.
也有擺放一些東西在身上.

$^{1201}$例如擺放口罩,飯券,現金券.
看到有需要的人就跟他們共享.
我們在紅土區也有社會參與.
有些弟兄姊妹也聽過.
你們基督徒真是很好.
這件事正如我常常說.
做了就是做了.
零和一的分別就是.
一做了就讓人感受到基督的行動.
有沒有網上的弟兄姊妹.
有問題嗎?.
哦.
好,我們八課當中的第一講就到這裡了.
我們看那邊.
網上的弟兄姊妹.
密切留意我們下一個月.
下一個月是什麼時候?.
是六月的第三個星期.
下個星期就沒有了.
一個月之後就沒有了.
好像崇拜一樣.
下個月見.
我們一個月一講.
先跟網上的弟兄姊妹說聲再見.
再見.
《香港》 歌手:陳靜茹 作詞:陳靜茹 作曲:陳靜茹.
香港我心真的國鄉.
這裡讓我生長.
有我喜歡的親友共陽光.
路上人在跑過他過.
幹勁令我欣賞.
這裡有許多好處沒發覺.
食一聲香港香港.
你永遠是曾夢鄉.
香港香港你哪識了哪我.
山頂看小島水雷塘.
處處換上新裝.
看看那海鷗飛過自由港.
海邊看小島處萬丈.
處處搖眼新光.

$^{1241}$這個市區的吸引沒發覺.
食一聲香港香港.
再有我童年夢想.
香港香港叫我不以為望.
香港我心真的國鄉.
這裡讓我生長.
有我喜歡的親友共陽光.
路上人在跑過他過.
幹勁令我欣賞.
這裡有許多好處沒發覺.
食一聲香港香港.
你永遠是曾夢鄉.
香港香港你哪識了哪我.
香港香港再有我童年夢想.
香港香港叫我不以為望.
香港香港你永遠是曾夢鄉.
香港香港你哪識了哪我.
Zither Harp.
\newpage



\section{}
\label{sec:OTk7WEa_w50}
\textbf{【這是最好的時代:給香港基督徒的神學八課】第2講:究竟好消息有多好?|20210619 [OTk7WEa-w50]}
\newline
\newline
連結: \href{https://youtube.com/watch?v=OTk7WEa-w50}{\texttt{ https://youtube.com/watch?v=OTk7WEa-w50}} ~~~~ 語音日期: 2021-06-19 
\newline
\newline
\hyperref[sec:adSMNfhKn24]{\small{< < < PREV SERMON < < <}}
~
\hyperref[sec:index_chronic]{\small{[返順時目]}}
~
\hyperref[sec:index_scriptual]{\small{[返順卷目]}}
~
\hyperref[sec:rjndxjSXpt8]{\small{> > > NEXT SERMON > > >}}
\newline
\newline
$^{1}$我只想知道.
你到底是什麼意思.
我只想知道.
你到底是什麼意思.
我只想知道.
你到底是什麼意思.
我只想知道.
你到底是什麼意思.
我只想知道.
你到底是什麼意思.
我只想知道.
你到底是什麼意思.
我只想知道.
你到底是什麼意思.
我只想知道.
你到底是什麼意思.
我只想知道.
你到底是什麼意思.
我只想知道.
你到底是什麼意思.
我只想知道.
你到底是什麼意思.
我只想知道.
你到底是什麼意思.
我只想知道.
你到底是什麼意思.
我只想知道.
你到底是什麼意思.
我只想知道.
你到底是什麼意思.
我只想知道.
你到底是什麼意思.
我只想知道.
你到底是什麼意思.
我只想知道.
你到底是什麼意思.
我只想知道.
你到底是什麼意思.
我只想知道.
你到底是什麼意思.

$^{41}$我只想知道.
你到底是什麼意思.
我只想知道.
你到底是什麼意思.
我只想知道.
你到底是什麼意思.
我只想知道.
你到底是什麼意思.
我只想知道.
你到底是什麼意思.
我只想知道.
你到底是什麼意思.
我只想知道.
你到底是什麼意思.
我只想知道.
你到底是什麼意思.
我只想知道.
你到底是什麼意思.
我只想知道.
你到底是什麼意思.
我只想知道.
你到底是什麼意思.
我只想知道.
你到底是什麼意思.
我只想知道.
你到底是什麼意思.
我只想知道.
你到底是什麼意思.
我只想知道.
你到底是什麼意思.
我只想知道.
你到底是什麼意思.
我只想知道.
你到底是什麼意思.
我只想知道.
你到底是什麼意思.
我只想知道.
你到底是什麼意思.
我只想知道.
你到底是什麼意思.

$^{81}$我只想知道.
你到底是什麼意思.
我只想知道.
你到底是什麼意思.
我只想知道.
你到底是什麼意思.
我只想知道.
你到底是什麼意思.
我只想知道.
你到底是什麼意思.
我只想知道.
你到底是什麼意思.
我只想知道.
你到底是什麼意思.
我只想知道.
你到底是什麼意思.
我只想知道.
你到底是什麼意思.
我只想知道.
你到底是什麼意思.
我只想知道.
你到底是什麼意思.
我只想知道.
你到底是什麼意思.
我只想知道.
你到底是什麼意思.
我只想知道.
你到底是什麼意思.
我只想知道.
你到底是什麼意思.
我只想知道.
你到底是什麼意思.
我只想知道.
你到底是什麼意思.
我只想知道.
你到底是什麼意思.
我只想知道.
你到底是什麼意思.
我只想知道.
你到底是什麼意思.

$^{121}$我只想知道.
你到底是什麼意思.
我只想知道.
你到底是什麼意思.
我只想知道.
你到底是什麼意思.
我只想知道.
你到底是什麼意思.
我只想知道.
你到底是什麼意思.
我只想知道.
你到底是什麼意思.
我只想知道.
你到底是什麼意思.
我只想知道.
你到底是什麼意思.
我只想知道.
你到底是什麼意思.
我只想知道.
你到底是什麼意思.
我只想知道.
你到底是什麼意思.
我只想知道.
你到底是什麼意思.
我只想知道.
你到底是什麼意思.
我只想知道.
你到底是什麼意思.
我只想知道.
你到底是什麼意思.
我只想知道.
你到底是什麼意思.
我只想知道.
你到底是什麼意思.
我只想知道.
你到底是什麼意思.
我只想知道.
你到底是什麼意思.
我只想知道.
你到底是什麼意思.

$^{161}$我只想知道.
你到底是什麼意思.
我只想知道.
你到底是什麼意思.
我只想知道.
你到底是什麼意思.
我只想知道.
你到底是什麼意思.
我只想知道.
你到底是什麼意思.
我只想知道.
你到底是什麼意思.
我只想知道.
你到底是什麼意思.
我只想知道.
你到底是什麼意思.
我只想知道.
你到底是什麼意思.
我只想知道.
你到底是什麼意思.
我只想知道.
你到底是什麼意思.
我只想知道.
你到底是什麼意思.
我只想知道.
你到底是什麼意思.
我只想知道.
你到底是什麼意思.
我只想知道.
你到底是什麼意思.
我只想知道.
你到底是什麼意思.
我只想知道.
你到底是什麼意思.
我只想知道.
你到底是什麼意思.
我只想知道.
你到底是什麼意思.
我只想知道.
你到底是什麼意思.

$^{201}$我只想知道.
你到底是什麼意思.
我只想知道.
你到底是什麼意思.
我只想知道.
你到底是什麼意思.
我只想知道.
你到底是什麼意思.
我只想知道.
你到底是什麼意思.
我只想知道.
你到底是什麼意思.
我只想知道.
你到底是什麼意思.
我只想知道.
你到底是什麼意思.
我只想知道.
你到底是什麼意思.
我只想知道.
你到底是什麼意思.
我只想知道.
你到底是什麼意思.
我只想知道.
你到底是什麼意思.
我只想知道.
你到底是什麼意思.
我只想知道.
你到底是什麼意思.
我只想知道.
你到底是什麼意思.
我只想知道.
你到底是什麼意思.
我只想知道.
你到底是什麼意思.
我只想知道.
你到底是什麼意思.
我只想知道.
你到底是什麼意思.
我只想知道.
你到底是什麼意思.

$^{241}$我只想知道.
你到底是什麼意思.
我只想知道.
你到底是什麼意思.
我只想知道.
你到底是什麼意思.
我只想知道.
你到底是什麼意思.
我只想知道.
你到底是什麼意思.
我只想知道.
你到底是什麼意思.
我只想知道.
你到底是什麼意思.
我只想知道.
你到底是什麼意思.
我只想知道.
你到底是什麼意思.
我只想知道.
你到底是什麼意思.
我只想知道.
你到底是什麼意思.
我只想知道.
你到底是什麼意思.
我只想知道.
你到底是什麼意思.
我只想知道.
你到底是什麼意思.
我只想知道.
你到底是什麼意思.
我只想知道.
你到底是什麼意思.
我只想知道.
你到底是什麼意思.
我只想知道.
你到底是什麼意思.
我只想知道.
你到底是什麼意思.
我只想知道.
你到底是什麼意思.

$^{281}$我只想知道.
你到底是什麼意思.
我只想知道.
你到底是什麼意思.
我只想知道.
你到底是什麼意思.
我只想知道.
你到底是什麼意思.
我只想知道.
你到底是什麼意思.
我只想知道.
你到底是什麼意思.
我只想知道.
你到底是什麼意思.
我只想知道.
你到底是什麼意思.
我只想知道.
你到底是什麼意思.
我只想知道.
你到底是什麼意思.
我只想知道.
你到底是什麼意思.
我只想知道.
你到底是什麼意思.
我只想知道.
你到底是什麼意思.
我只想知道.
你到底是什麼意思.
我只想知道.
你到底是什麼意思.
我只想知道.
你到底是什麼意思.
我只想知道.
你到底是什麼意思.
我只想知道.
你到底是什麼意思.
我只想知道.
你到底是什麼意思.
我只想知道.
你到底是什麼意思.

$^{321}$我只想知道.
你到底是什麼意思.
我只想知道.
你到底是什麼意思.
我只想知道.
你到底是什麼意思.
我只想知道.
你到底是什麼意思.
我只想知道.
你到底是什麼意思.
我只想知道.
你到底是什麼意思.
我只想知道.
你到底是什麼意思.
我只想知道.
你到底是什麼意思.
我只想知道.
你到底是什麼意思.
我只想知道.
你到底是什麼意思.
我只想知道.
你到底是什麼意思.
我只想知道.
你到底是什麼意思.
我只想知道.
你到底是什麼意思.
我只想知道.
你到底是什麼意思.
我只想知道.
你到底是什麼意思.
我只想知道.
你到底是什麼意思.
我只想知道.
你到底是什麼意思.
我只想知道.
你到底是什麼意思.
我只想知道.
你到底是什麼意思.
我只想知道.
你到底是什麼意思.

$^{361}$我只想知道.
你到底是什麼意思.
我只想知道.
你到底是什麼意思.
我只想知道.
你到底是什麼意思.
我只想知道.
你到底是什麼意思.
我只想知道.
你到底是什麼意思.
我只想知道.
你到底是什麼意思.
我只想知道.
你到底是什麼意思.
我只想知道.
你到底是什麼意思.
我只想知道.
你到底是什麼意思.
我只想知道.
你到底是什麼意思.
我只想知道.
你到底是什麼意思.
我只想知道.
你到底是什麼意思.
我只想知道.
你到底是什麼意思.
我只想知道.
你到底是什麼意思.
我只想知道.
你到底是什麼意思.
我只想知道.
你到底是什麼意思.
我只想知道.
你到底是什麼意思.
我只想知道.
你到底是什麼意思.
我只想知道.
你到底是什麼意思.
我只想知道.
你到底是什麼意思.

$^{401}$我只想知道.
你到底是什麼意思.
我只想知道.
你到底是什麼意思.
我只想知道.
你到底是什麼意思.
我只想知道.
你到底是什麼意思.
我只想知道.
你到底是什麼意思.
我只想知道.
你到底是什麼意思.
我只想知道.
你到底是什麼意思.
我只想知道.
你到底是什麼意思.
我只想知道.
你到底是什麼意思.
我只想知道.
你到底是什麼意思.
我只想知道.
你到底是什麼意思.
我只想知道.
你到底是什麼意思.
我只想知道.
你到底是什麼意思.
我只想知道.
你到底是什麼意思.
我只想知道.
你到底是什麼意思.
我只想知道.
你到底是什麼意思.
我只想知道.
你到底是什麼意思.
我只想知道.
你到底是什麼意思.
我只想知道.
你到底是什麼意思.
我只想知道.
你到底是什麼意思.

$^{441}$我只想知道.
你到底是什麼意思.
我只想知道.
你到底是什麼意思.
我只想知道.
你到底是什麼意思.
我只想知道.
你到底是什麼意思.
我只想知道.
你到底是什麼意思.
我只想知道.
你到底是什麼意思.
我只想知道.
你到底是什麼意思.
我只想知道.
你到底是什麼意思.
我只想知道.
你到底是什麼意思.
我只想知道.
你到底是什麼意思.
我只想知道.
你到底是什麼意思.
我只想知道.
你到底是什麼意思.
我只想知道.
你到底是什麼意思.
我只想知道.
你到底是什麼意思.
我只想知道.
你到底是什麼意思.
我只想知道.
你到底是什麼意思.
我只想知道.
你到底是什麼意思.
我只想知道.
你到底是什麼意思.
我只想知道.
你到底是什麼意思.
我只想知道.
你到底是什麼意思.

$^{481}$我只想知道.
你到底是什麼意思.
我只想知道.
你到底是什麼意思.
我只想知道.
你到底是什麼意思.
我只想知道.
你到底是什麼意思.
我只想知道.
你到底是什麼意思.
我只想知道.
你到底是什麼意思.
我只想知道.
你到底是什麼意思.
我只想知道.
你到底是什麼意思.
我只想知道.
你到底是什麼意思.
我只想知道.
你到底是什麼意思.
我只想知道.
你到底是什麼意思.
我只想知道.
你到底是什麼意思.
我只想知道.
你到底是什麼意思.
我只想知道.
你到底是什麼意思.
我只想知道.
你到底是什麼意思.
我只想知道.
你到底是什麼意思.
我只想知道.
你到底是什麼意思.
我只想知道.
你到底是什麼意思.
我只想知道.
你到底是什麼意思.
我只想知道.
你到底是什麼意思.

$^{521}$我只想知道.
你到底是什麼意思.
我只想知道.
你到底是什麼意思.
我只想知道.
你到底是什麼意思.
我只想知道.
你到底是什麼意思.
我只想知道.
你到底是什麼意思.
我只想知道.
你到底是什麼意思.
我只想知道.
你到底是什麼意思.
我只想知道.
你到底是什麼意思.
我只想知道.
你到底是什麼意思.
我只想知道.
你到底是什麼意思.
我只想知道.
你到底是什麼意思.
我只想知道.
你到底是什麼意思.
我只想知道.
你到底是什麼意思.
我只想知道.
你到底是什麼意思.
我只想知道.
你到底是什麼意思.
我只想知道.
你到底是什麼意思.
我只想知道.
你到底是什麼意思.
我只想知道.
你到底是什麼意思.
我只想知道.
你到底是什麼意思.
我只想知道.
你到底是什麼意思.

$^{561}$我只想知道.
你到底是什麼意思.
我只想知道.
你到底是什麼意思.
我只想知道.
你到底是什麼意思.
我只想知道.
你到底是什麼意思.
我只想知道.
你到底是什麼意思.
我只想知道.
你到底是什麼意思.
我只想知道.
你到底是什麼意思.
我只想知道.
你到底是什麼意思.
我只想知道.
你到底是什麼意思.
我只想知道.
你到底是什麼意思.
我只想知道.
你到底是什麼意思.
我只想知道.
你到底是什麼意思.
我只想知道.
你到底是什麼意思.
我只想知道.
你到底是什麼意思.
我只想知道.
你到底是什麼意思.
我只想知道.
你到底是什麼意思.
我只想知道.
你到底是什麼意思.
我只想知道.
你到底是什麼意思.
我只想知道.
你到底是什麼意思.
我只想知道.
你到底是什麼意思.

$^{601}$我只想知道.
你到底是什麼意思.
我只想知道.
你到底是什麼意思.
我只想知道.
你到底是什麼意思.
我只想知道.
你到底是什麼意思.
我只想知道.
你到底是什麼意思.
我只想知道.
你到底是什麼意思.
我只想知道.
你到底是什麼意思.
我只想知道.
你到底是什麼意思.
我只想知道.
你到底是什麼意思.
我只想知道.
你到底是什麼意思.
我只想知道.
你到底是什麼意思.
我只想知道.
你到底是什麼意思.
我只想知道.
你到底是什麼意思.
我只想知道.
你到底是什麼意思.
我只想知道.
你到底是什麼意思.
我只想知道.
你到底是什麼意思.
我只想知道.
你到底是什麼意思.
我只想知道.
你到底是什麼意思.
我只想知道.
你到底是什麼意思.
我只想知道.
你到底是什麼意思.

$^{641}$我只想知道.
你到底是什麼意思.
我只想知道.
你到底是什麼意思.
我只想知道.
你到底是什麼意思.
我只想知道.
你到底是什麼意思.
我只想知道.
你到底是什麼意思.
我只想知道.
你到底是什麼意思.
我只想知道.
你到底是什麼意思.
我只想知道.
你到底是什麼意思.
我只想知道.
你到底是什麼意思.
我只想知道.
你到底是什麼意思.
我只想知道.
你到底是什麼意思.
我只想知道.
你到底是什麼意思.
我只想知道.
你到底是什麼意思.
我只想知道.
你到底是什麼意思.
我只想知道.
你到底是什麼意思.
我只想知道.
你到底是什麼意思.
我只想知道.
你到底是什麼意思.
我只想知道.
你到底是什麼意思.
我只想知道.
你到底是什麼意思.
我只想知道.
你到底是什麼意思.

$^{681}$我只想知道.
你到底是什麼意思.
我只想知道.
你到底是什麼意思.
我只想知道.
你到底是什麼意思.
我只想知道.
你到底是什麼意思.
我只想知道.
你到底是什麼意思.
我只想知道.
你到底是什麼意思.
我只想知道.
你到底是什麼意思.
我只想知道.
你到底是什麼意思.
我只想知道.
你到底是什麼意思.
我只想知道.
你到底是什麼意思.
我只想知道.
你到底是什麼意思.
我只想知道.
你到底是什麼意思.
我只想知道.
你到底是什麼意思.
我只想知道.
你到底是什麼意思.
我只想知道.
你到底是什麼意思.
我只想知道.
你到底是什麼意思.
我只想知道.
你到底是什麼意思.
我只想知道.
你到底是什麼意思.
我只想知道.
你到底是什麼意思.
我只想知道.
你到底是什麼意思.

$^{721}$我只想知道.
你到底是什麼意思.
我只想知道.
你到底是什麼意思.
我只想知道.
你到底是什麼意思.
我只想知道.
你到底是什麼意思.
我只想知道.
你到底是什麼意思.
我只想知道.
你到底是什麼意思.
我只想知道.
你到底是什麼意思.
我只想知道.
你到底是什麼意思.
我只想知道.
你到底是什麼意思.
我只想知道.
你到底是什麼意思.
我只想知道.
你到底是什麼意思.
我只想知道.
你到底是什麼意思.
我只想知道.
你到底是什麼意思.
我只想知道.
你到底是什麼意思.
我只想知道.
你到底是什麼意思.
我只想知道.
你到底是什麼意思.
我只想知道.
你到底是什麼意思.
我只想知道.
你到底是什麼意思.
我只想知道.
你到底是什麼意思.
我只想知道.
你到底是什麼意思.

$^{761}$我只想知道.
你到底是什麼意思.
我只想知道.
你到底是什麼意思.
我只想知道.
你到底是什麼意思.
我只想知道.
你到底是什麼意思.
我只想知道.
你到底是什麼意思.
我只想知道.
你到底是什麼意思.
我只想知道.
你到底是什麼意思.
我只想知道.
你到底是什麼意思.
我只想知道.
你到底是什麼意思.
我只想知道.
你到底是什麼意思.
我只想知道.
你到底是什麼意思.
我只想知道.
你到底是什麼意思.
我只想知道.
你到底是什麼意思.
我只想知道.
你到底是什麼意思.
我只想知道.
你到底是什麼意思.
我只想知道.
你到底是什麼意思.
我只想知道.
你到底是什麼意思.
我只想知道.
你到底是什麼意思.
我只想知道.
你到底是什麼意思.
我只想知道.
你到底是什麼意思.

$^{801}$我只想知道.
你到底是什麼意思.
我只想知道.
你到底是什麼意思.
我只想知道.
你到底是什麼意思.
我只想知道.
你到底是什麼意思.
我只想知道.
你到底是什麼意思.
我只想知道.
你到底是什麼意思.
我只想知道.
你到底是什麼意思.
我只想知道.
你到底是什麼意思.
我只想知道.
你到底是什麼意思.
我只想知道.
你到底是什麼意思.
我只想知道.
你到底是什麼意思.
我只想知道.
你到底是什麼意思.
我只想知道.
你到底是什麼意思.
我只想知道.
你到底是什麼意思.
我只想知道.
你到底是什麼意思.
我只想知道.
你到底是什麼意思.
我只想知道.
你到底是什麼意思.
我只想知道.
你到底是什麼意思.
我只想知道.
你到底是什麼意思.
我只想知道.
你到底是什麼意思.

$^{841}$我只想知道.
你到底是什麼意思.
我只想知道.
你到底是什麼意思.
我只想知道.
你到底是什麼意思.
我只想知道.
你到底是什麼意思.
我只想知道.
你到底是什麼意思.
我只想知道.
你到底是什麼意思.
我只想知道.
你到底是什麼意思.
我只想知道.
你到底是什麼意思.
我只想知道.
你到底是什麼意思.
我只想知道.
你到底是什麼意思.
我只想知道.
你到底是什麼意思.
我只想知道.
你到底是什麼意思.
我只想知道.
你到底是什麼意思.
我只想知道.
你到底是什麼意思.
我只想知道.
你到底是什麼意思.
我只想知道.
你到底是什麼意思.
我只想知道.
你到底是什麼意思.
我只想知道.
你到底是什麼意思.
我只想知道.
你到底是什麼意思.
我只想知道.
你到底是什麼意思.

$^{881}$我只想知道.
你到底是什麼意思.
我只想知道.
你到底是什麼意思.
我只想知道.
你到底是什麼意思.
我只想知道.
你到底是什麼意思.
我只想知道.
你到底是什麼意思.
我只想知道.
你到底是什麼意思.
我只想知道.
你到底是什麼意思.
我只想知道.
你到底是什麼意思.
我只想知道.
你到底是什麼意思.
我只想知道.
你到底是什麼意思.
我只想知道.
你到底是什麼意思.
我只想知道.
你到底是什麼意思.
我只想知道.
你到底是什麼意思.
我只想知道.
你到底是什麼意思.
我只想知道.
你到底是什麼意思.
我只想知道.
你到底是什麼意思.
我只想知道.
你到底是什麼意思.
我只想知道.
你到底是什麼意思.
我只想知道.
你到底是什麼意思.
我只想知道.
你到底是什麼意思.

$^{921}$我只想知道.
你到底是什麼意思.
我只想知道.
你到底是什麼意思.
我只想知道.
你到底是什麼意思.
我只想知道.
你到底是什麼意思.
我只想知道.
你到底是什麼意思.
我只想知道.
你到底是什麼意思.
我只想知道.
你到底是什麼意思.
我只想知道.
你到底是什麼意思.
我只想知道.
你到底是什麼意思.
我只想知道.
你到底是什麼意思.
我只想知道.
你到底是什麼意思.
我只想知道.
你到底是什麼意思.
我只想知道.
你到底是什麼意思.
我只想知道.
你到底是什麼意思.
我只想知道.
你到底是什麼意思.
我只想知道.
你到底是什麼意思.
我只想知道.
你到底是什麼意思.
我只想知道.
你到底是什麼意思.
我只想知道.
你到底是什麼意思.
我只想知道.
你到底是什麼意思.

$^{961}$我只想知道.
你到底是什麼意思.
我只想知道.
你到底是什麼意思.
我只想知道.
你到底是什麼意思.
我只想知道.
你到底是什麼意思.
我只想知道.
你到底是什麼意思.
我只想知道.
你到底是什麼意思.
我只想知道.
你到底是什麼意思.
我只想知道.
你到底是什麼意思.
我只想知道.
你到底是什麼意思.
我只想知道.
你到底是什麼意思.
我只想知道.
你到底是什麼意思.
我只想知道.
你到底是什麼意思.
我只想知道.
你到底是什麼意思.
我只想知道.
你到底是什麼意思.
我只想知道.
你到底是什麼意思.
我只想知道.
你到底是什麼意思.
我只想知道.
你到底是什麼意思.
我只想知道.
你到底是什麼意思.
我只想知道.
你到底是什麼意思.
我只想知道.
你到底是什麼意思.

$^{1001}$我只想知道.
你到底是什麼意思.
我只想知道.
你到底是什麼意思.
我只想知道.
你到底是什麼意思.
我只想知道.
你到底是什麼意思.
我只想知道.
你到底是什麼意思.
我只想知道.
你到底是什麼意思.
我只想知道.
你到底是什麼意思.
我只想知道.
你到底是什麼意思.
我只想知道.
你到底是什麼意思.
我只想知道.
你到底是什麼意思.
我只想知道.
你到底是什麼意思.
我只想知道.
你到底是什麼意思.
我只想知道.
你到底是什麼意思.
我只想知道.
你到底是什麼意思.
我只想知道.
你到底是什麼意思.
我只想知道.
你到底是什麼意思.
我只想知道.
你到底是什麼意思.
我只想知道.
你到底是什麼意思.
我只想知道.
你到底是什麼意思.
我只想知道.
你到底是什麼意思.

$^{1041}$我只想知道.
你到底是什麼意思.
我只想知道.
你到底是什麼意思.
我只想知道.
你到底是什麼意思.
我只想知道.
你到底是什麼意思.
我只想知道.
你到底是什麼意思.
我只想知道.
你到底是什麼意思.
我只想知道.
你到底是什麼意思.
我只想知道.
你到底是什麼意思.
我只想知道.
你到底是什麼意思.
我只想知道.
你到底是什麼意思.
我只想知道.
你到底是什麼意思.
我只想知道.
你到底是什麼意思.
我只想知道.
你到底是什麼意思.
我只想知道.
你到底是什麼意思.
我只想知道.
你到底是什麼意思.
我只想知道.
你到底是什麼意思.
我只想知道.
你到底是什麼意思.
我只想知道.
你到底是什麼意思.
我只想知道.
你到底是什麼意思.
我只想知道.
你到底是什麼意思.

$^{1081}$我只想知道.
你到底是什麼意思.
我只想知道.
你到底是什麼意思.
我只想知道.
你到底是什麼意思.
我只想知道.
你到底是什麼意思.
我只想知道.
你到底是什麼意思.
我只想知道.
你到底是什麼意思.
我只想知道.
你到底是什麼意思.
我只想知道.
你到底是什麼意思.
我只想知道.
你到底是什麼意思.
我只想知道.
你到底是什麼意思.
我只想知道.
你到底是什麼意思.
我只想知道.
你到底是什麼意思.
我只想知道.
你到底是什麼意思.
我只想知道.
你到底是什麼意思.
我只想知道.
你到底是什麼意思.
我只想知道.
你到底是什麼意思.
我只想知道.
你到底是什麼意思.
我只想知道.
你到底是什麼意思.
我只想知道.
你到底是什麼意思.
我只想知道.
你到底是什麼意思.

$^{1121}$我只想知道.
你到底是什麼意思.
我只想知道.
你到底是什麼意思.
我只想知道.
你到底是什麼意思.
我只想知道.
你到底是什麼意思.
我只想知道.
你到底是什麼意思.
我只想知道.
你到底是什麼意思.
我只想知道.
你到底是什麼意思.
我只想知道.
你到底是什麼意思.
我只想知道.
你到底是什麼意思.
我只想知道.
你到底是什麼意思.
我只想知道.
你到底是什麼意思.
我只想知道.
你到底是什麼意思.
我只想知道.
你到底是什麼意思.
我只想知道.
你到底是什麼意思.
我只想知道.
你到底是什麼意思.
我只想知道.
你到底是什麼意思.
我只想知道.
你到底是什麼意思.
我只想知道.
你到底是什麼意思.
我只想知道.
你到底是什麼意思.
我只想知道.
你到底是什麼意思.

$^{1161}$我只想知道.
你到底是什麼意思.
我只想知道.
你到底是什麼意思.
我只想知道.
你到底是什麼意思.
我只想知道.
你到底是什麼意思.
我只想知道.
你到底是什麼意思.
我只想知道.
你到底是什麼意思.
我只想知道.
你到底是什麼意思.
我只想知道.
你到底是什麼意思.
我只想知道.
你到底是什麼意思.
我只想知道.
你到底是什麼意思.
我只想知道.
你到底是什麼意思.
我只想知道.
你到底是什麼意思.
我只想知道.
你到底是什麼意思.
我只想知道.
你到底是什麼意思.
我只想知道.
你到底是什麼意思.
我只想知道.
你到底是什麼意思.
我只想知道.
你到底是什麼意思.
我只想知道.
你到底是什麼意思.
我只想知道.
你到底是什麼意思.
我只想知道.
你到底是什麼意思.

$^{1201}$我只想知道.
你到底是什麼意思.
我只想知道.
你到底是什麼意思.
我只想知道.
你到底是什麼意思.
我只想知道.
你到底是什麼意思.
我只想知道.
你到底是什麼意思.
我只想知道.
你到底是什麼意思.
我只想知道.
你到底是什麼意思.
我只想知道.
你到底是什麼意思.
我只想知道.
你到底是什麼意思.
這就是我認識的福音.
不過我想說.
福音這個詞.
其實是一個.
在舊裡面出現過一次.
麻煩下一張.
舊裡面.
唯一一次出現.
Evangelion.
在哪裡呢.
其實是有的.
看回希利文版本的舊約.
就是七十字本的話.
你會發現.
有一句經文.
就是當時大衛.
對當人殺死掃挪王的時候.
那些說話.
他說.
從前有人報告我說.
掃挪死了.
他自以為報好消息.

$^{1241}$我就拿著他.
將他殺在西格拉.
這就作為他報好消息的賞詞.
所以舊裡面.
唯一一次出現了.
Good news這個字.
Evangelion這個字就在這裡.
其實沒有關係.
當時大衛就是說.
Good news.
就是將掃挪.
看為一個Good news.
就這樣看.
大衛就殺了他.
所以基本上.
Evangelion這個字.
基本上.
不關一個什麼福音的事.
這個字本身是一個很普通的字眼.
一個就是Good news的意思.
一個很平常的字眼.
所以你會發現.
如果你這樣看的時候.
我們就會知道.
其實這個字.
麻煩夏業.
我們就更加好理解.
不要當作福音的意思.
因為我們太過慣性.
就像Dragon.
大家重新理解.
這個Gospel.
這個福音.
其實就是一個好消息.
所以我們重新來思考.
不如我們重新來學習這個字.
其實Evangelion這個字.
其實就是一個好消息的意思.
一個好的消息.
什麼是好消息呢?.

$^{1281}$就是簡單.
假設你今天突然患了癌症.
醫生突然說你沒事了.
這是一個Good news.
就是這麼簡單.
所以當2000年前教會.
嘗試去講一個Gospel的時候.
這個不是什麼獨特的字眼.
他和世人說.
我有一個Good news給你們.
有一個好的消息給你們.
就是這麼簡單.
所以我們一起看下去.
所以你會發現.
其實這個字.
早在教會用這個字之前.
我們問.
究竟什麼時候開始用這個字呢?.
早在教會用這個字之前.
其實當時候的羅馬帝國.
就一早用這個字.
大家看到這個碑文嗎?.
這個是當時羅馬帝國的寫實.
一些歌頌皇帝的墓碑.
一些碑文.
你會發現有一個字.
不知道大家能不能看到.
我先把這個放大.
這樣放大了.
誒?.
不要不要.
請你放大鏡頭.
是了.
我放大了.
看到這個字嗎?.
是了.
看到這個字叫Evangelia嗎?.
是了.
不要動.
這個Evangelia這個字.

$^{1321}$是了.
不要動.
Evangelia這個字是福音的字.
所以這個字不要叫福音.
叫做Good news就對了.
我現在有一個Good news給你們聽.
凱撒的生日就是Good news.
當時的皇帝是很用這個字的.
我現在告訴你們.
我們有一個好消息.
我們現在可以打針了.
這個是Good news.
我們現在可以通關了.
這個也是Good news.
所以當時的Good news是一個縱數.
這個Gospel是加一個s字的.
所以當時羅馬帝國不斷地去宣傳很多的Good news.
News有一個s字.
其實是一個縱數.
也就是說.
其實這個是很平常的字.
有很多不同的Good news給當時的人.
在這樣的情況下.
當時的教會大概在公元30年左右.
耶穌過世真無多久之後.
教會就開始用這個字.
不過是用一個單數的字.
The only one gospel.
Evangelion.
不是一個縱數的字.
所以很明顯不同.
當時羅馬帝國是用一個縱數的Good news.
很多很多不同的信息.
很多很多很零碎的Good news給大家.
但當時的教會偏偏宣傳一個唯一的Good news.
所以整件事我們就說.
所謂的Gospel.
其實大家要忘記這個屬靈的字眼.
我們問究竟這個Good news是有多麼Good.
關乎於什麼.

$^{1361}$和我們有什麼關係.
在什麼的Context下被宣傳.
請下一張.
所以當時你發覺.
聖經裡面有兩個很不同的版本.
或者兩個不同的意味的Gospel.
第一個就是保羅的福音.
如果你不看福音書的時候.
你單單看保羅的書信.
保羅所說的Good news.
其實關於什麼呢.
大家看一看.
第一個就是.
甚至我從耶路撒冷轉到以來.
猜測直到傳基督的福音.
看到是一個基督的福音.
有關尼賽亞的Good news.
第二張.
然而我們沒有用過者權並.
倒凡事忍受免得基督的福音被阻隔.
要留意這個語言.
他用基督的語言.
就會有關基督的Good news的福音.
第三張.
這個更加重要.
因為上帝的義正在這福音上顯明出來.
這義事本於信而至於信.
如經所記.
義人必因信得新.
這個福音是關乎於叫人得救的.
叫人因著信的緣故.
能夠得到這個義.
所以這個福音是關乎於.
這個Good news是關乎於耶穌基督.
和我們的救恩.
能夠讓我們得到永生.
是一個十字架的福音.
然後這個更加重要的定義.
因為保祿說過.
在臨前十五章裡面.

$^{1401}$他說.
「你們我如今把先前所傳給你們的福音.
告訴你們知道.
這福音你們領受了.
有靠著.
暫立得住.
並且以你們若不是徒然相信.
能以持守我所傳給你們的.
就必因這福音得救」.
這是一個得救福音.
是什麼.
第一就是基督照聖經所說.
「為我們的罪死了.
而且埋葬了」.
又照聖經所說.
「第三天復活了」.
這是一個很典型的.
金羅安桑方派所講的福音.
因為它關乎於得救的福音.
就是耶穌基督為你得十字架.
然後復活.
你就能夠得到永生的福音.
最後一句.
這個福音是什麼.
甚至加上說.
保羅強調.
你不可以來到它搞錯.
這個是沒有其他別的福音.
那些其他錯的福音.
例如受國禮的.
國禮得救的.
都不是真正的福音.
唯有基督的福音.
我之前傳給你們的.
才是那個純正的福音.
這個就是保羅的福音.
我們今天基本上福音派.
或者基督派所講的福音.
其實是保羅的福音.
是一個經過保羅神學化.

$^{1441}$消化了之後.
因為基督耶穌受難.
十字架緣故.
他重新來到.
再一次的來到.
去理解這個福音的內容.
不過我們發現.
其實當你看福音書的時候.
耶穌也講福音.
耶穌也講福音這件事.
特別你看下一章.
《馬可》的時候.
第一章第一節.
有一個很特別的一句.
這是整本書的名稱.
其實這不是第一句.
而是整本書的書名也好.
或者整個的名稱也好.
他說什麼呢?.
順德的疑似.
耶穌基督福音的起頭.
如果你把滑鼠放在中文字上.
有東西可以看.
請看.
是.
看到這個嗎?.
這個是根據原文裡面的次序.
第一個就是阿肯.
就是從Eugenium.
就是福音的起頭.
整個書就是福音的起頭.
然後是耶穌基督.
就是耶穌基督.
然後是神的頤指.
所以整個名稱的原文.
如果直譯的話.
是什麼意思呢?.
就是福音的起頭.
耶穌基督.
神的頤指.

$^{1481}$如果我們相信.
馬可福音是第一本福音書的時候.
其實馬可正是在.
羅馬帝國的時間裡面.
在羅馬城市.
他嘗試去做一個新的genre.
一個新的體創.
當時沒有福音書這件事.
他是第一本福音書.
所以他嘗試去寫.
耶穌基督的生平和教導.
並且他要以福音來命名.
所以這是一個很突破的想法.
記住.
是沒有福音的這個Souling Dragon.
這個叫做好消息.
所以當馬可在羅馬帝國裡面的核心.
羅馬.
他嘗試去講一個.
整個耶穌的故事給人聽.
他就說.
這個就是福音的起頭.
就是說一個好消息的起頭.
就是耶穌基督.
所以你發覺.
當時他們嘗試做什麼呢.
他們嘗試去稱呼耶穌基督.
才是真正的good news.
而不是當時羅馬帝國的文宣.
當時羅馬帝國很多皇帝不斷在度假.
不喜歡出文宣來.
有什麼good news什麼的.
但是他說.
真正的good news.
其實只有一個.
就是耶穌基督自己.
而當你去看回福音書的時候.
你發覺福音書.
特別是婦女福音.
馬太.

$^{1521}$馬可路加的福音書.
所講的耶穌.
其實福音是怎麼樣的.
我們看一下.
其實耶穌也講福音的.
但是耶穌的福音從來都沒有講.
自己會得十字架.
不會這樣講.
耶穌講什麼福音.
我們看一下字眼.
耶穌就走遍加里里.
在國會堂里教訓人.
傳天國的福音.
你看到形容詞是不一樣的.
不是基督的福音.
而是天國的福音.
第二個.
你們去把所聽見所看見的事告述若寒.
就是乞子看見.
瓊子行走.
長大或風得竭症.
龍子聽見死人復活.
窮人有福音傳給他們.
這個很明顯是和當時的困境有關係.
耶穌所傳揚的好消息.
是關乎貧窮人.
關乎社會邊緣上的人.
一些蠻的,聾的,跛的,不方便的人.
耶穌所傳的好消息.
不是我要得十字架.
然後你信了就能得永生.
而是一個很具體的.
就是一個社會關懷的福音.
耶穌所傳的福音.
是告訴人現在就有好消息.
你不用等死去上天堂.
而是說你能夠得到真真正正的好消息.
第三段.
耶穌來到加里里.
宣傳神的福音.

$^{1561}$說日期滿了.
聖律的國近了.
你們當悔改信福音.
記住耶穌所講的福音.
不是叫你信耶穌.
而是叫你去知道天國近了.
上帝的國度將要來臨.
所以那些人可以得到釋放.
那些人能夠得到解放.
所以你們要悔改.
要信這個好消息.
就是一個關乎天國神的福音.
然後.
第一個重要.
就是耶穌在加伯倫會堂所講的福音.
當耶穌站在會堂里.
開始宣講.
要讀《以塞雅書特羅亞章》的經文.
主的靈在我身上.
因為祂用高高我.
叫我全福音級及貧窮的人.
被老的得釋放.
黑眼的得看見.
受壓制的得自由.
報告上帝月立人的欺凌.
這句話其實是一個非常解放的福音.
耶穌所講的就是.
尼塞雅將要來臨.
並且尼塞雅來臨之後.
那些貧窮人能夠得到這個好消息.
被老的人得到釋放.
黑眼的得看見.
受壓制的人得到自由.
然後你們記得.
在經文裡面是怎樣的.
耶穌被人打.
因為祂說什麼.
我就是那個福音的內容.
我的來到正正就帶來這個好消息.
所以你發覺耶穌所講的福音.

$^{1601}$是完全與保祿海嘯不同的.
保祿所講的是基督的福音.
耶穌所講的福音是一個關乎於天國.
是說受欺壓得到解放的福音.
最後請.
耶穌祖先們說.
我必須再辟承全上帝國的福音.
因為我奉差原是為此.
所以耶穌所講的福音是關乎於天國來臨.
地上的國度.
他們將要面對上帝國度來臨.
而我就是尼賽亞.
就是那位君王.
所以你發現.
整個福音基本上是根據.
以賽亞書.
請下一張.
就是那個經文.
請下一句.
整個以賽亞書裡面.
很重要的經文.
就是內章裡面.
主耶和華的靈在我身上.
因為耶和華用高我.
叫我傳好消息給貧窮的人.
差見我這致傷心的人.
報告我.
報告被勞的得釋放.
被囚的得出監牢.
報告耶和華的恩憐.
和我們上帝保守的日子.
我安慰一切悲傷的人.
所以整個福音是建基於舊日裡面所講.
一個以賽亞書內章所講的經文.
所以整件事是關乎於.
現世裡面.
一個很真實地尼賽亞.
去解放人民的福音.
所以我們嘗試做一個比較.
試一下,請.

$^{1641}$所以發覺是有兩個很不同的福音.
似乎很不一樣.
但兩個是不同的版本.
一個就是保羅嘗試去詮釋.
耶穌所講的Good News.
關乎於耶穌的十字架的意義.
十字架能夠叫罪.
能夠得赦免.
然後凡是信福音的.
就能夠罪得赦免.
並且能夠生復活.
這是我們所謂的.
傳統教會所講的福音.
而耶穌的福音是關乎於行動.
耶穌所講的是一個.
我們知道上帝的國度來臨.
我們知道現在的人.
受欺壓的人.
他們能夠得到釋放.
是一個傳給貧窮人的福音.
所以其實兩個都很重要.
你發覺有兩個不同的向度.
保羅的版本和方書的版本.
其實都是互相來捕捉著.
我們不能單單偏向其中一個.
單單看著保羅的版本.
你就會忘記方書所講的福音.
耶穌所講的福音.
其實是關乎於天國的來臨.
關乎於我們如何面向世界.
而耶穌的版本.
正正都是不能夠沒有保羅的版本.
因為保羅是很好的.
在整合整件事.
耶穌的來臨.
不是單單去幫助當事人.
而是來講整個世界.
如何終結人.
如何得到最後的釋放.
所以我想說的是.

$^{1681}$其實兩個福音都是要明白.
是互相捕捉的.
單單有Evangelium.
一個所謂保羅的版本的話.
就變成了一個離地的假想.
只是想著將來如何得救.
如何上天堂.
這個不是福音的全部.
而單單一個所謂關懷社會福音.
這不是整件事的全部.
我們知道上帝的請求.
不是純粹關乎於現世.
而是關乎於整個世界的終末.
和整個人類的未來.
所以就是說.
根據這幾年裡面的討論.
其實我們呼出的這些福音.
其實是一個.
我們不能忽略這兩邊.
我們所謂得救是重要的.
但卻不能忽略社會上的影響.
所以我覺得有一個很好的整合.
請下一張.
大家如果看Anti-Right的一本書.
其實是一本很好的書.
中文翻譯也有.
叫做Simply Good News.
Anti-Right是一個很簡單.
用了百多頁寫出來的福音書.
有福音內容的教導.
一個非常好的解釋.
究竟什麼是Good News.
他說耶穌所宣講的好消息.
和我們後來所講的好消息.
其實是一個相同的信息.
兩者其實不是矛盾.
是一個很重要的互相補充.
所以好消息是什麼呢.
就是獨一真實的上帝.
要透過耶穌和祂受死的復活.

$^{1721}$並在一切裡面.
即是他執掌權並統治世界.
即是耶穌的死和復活.
是要改變的.
是關乎於整個世界的事.
以色列人自古以來的盼望.
已經實現了.
可是那實現的方式.
是他們意想不到的.
神在拯救世界的計劃.
終於啟動了.
他按照他的應許.
以新的方式來管治大地.
解決世界的問題.
他並以他的榮譽和公義.
充滿世界.
但是他成就這一切的方法.
完全消乏所有先知的想象.
所以就說.
原來從前所說的尼賽亞.
正要來到了.
不過那方法.
不是我們所想象的那麼簡單.
聖約瑛他要去重新.
用新的方法來管治大地.
去解決整個世界的問題.
不是單單個人靈魂的問題.
福音所謂的Good News.
是關乎於整個世界的.
自古以來世界病了.
世上的人都病了.
其實大家都有病.
人個體自然病.
人的命運是病了.
但是世界都病了.
所以現在上帝要把疾病醫治好.
使世界和人都能重獲新生.
有生命力將江河一般流出來.
悉心不惜地注入世界.
這種生命力是一種新的力量.

$^{1761}$他就是愛的大能.
好消息是.
以上一切已經在耶穌的身上發生.
並且借著耶穌成就了.
所以耶穌來到是要改變的.
不單單是靈魂的得救.
而是整個世界.
有關國度上的改變和更新.
總不久將來.
整個受造的世界有同樣的變化.
整體人類.
我們每個人.
不論是誰.
都要得著更新改變.
這就是基督教福音.
所以將來整個世界將會改變.
整個人類的命運都要改變.
這就是好消息.
如果你補充好.
這個好消息.
當然是好消息.
但這些好消息不需要傳給別人.
不關我事.
如果日報贏了.
當然是好消息.
對我來說.
但可能不關你事.
唯一一個值得.
向全世界人說的好消息是什麼.
就是世界將會改變.
世界將會改變.
所以縮音的向度.
似乎是一個很重要的內容.
我們不單單關乎一個個人的得救.
而是關乎整個更加寬闊.
更加遠大的世界觀的改變.
下一張,謝謝.
所以你試著想想.
你反省一下.
我們認識福音之後.

$^{1801}$究竟這個比較寬闊的方觀.
對我們有什麼意義呢.
謝謝下一張.
以前我們很簡單.
福音就是一個很簡單的五分鐘內容.
我們都會說.
我們以前對著.
我們有罪.
然後就怎麼樣.
對不起,他應該是想創造世界.
然後我們有罪.
他都來了.
十字架裡面.
為我們釘死.
你只要認罪悔改.
就能得到福音.
然後大力決治.
這是一個很簡單的福音.
福音今天其實很簡單.
我們基本上可以說.
大亂引賊.
我們不斷地說.
信仰述德永生.
不斷地重復.
重復這個很公式化的福音.
不斷地教人家說.
不斷地說.
方成為了一個天堂的入場卷.
這張快速通過.
只要你按按鈕就能拿到快速通過.
方成為了一個得救的途徑.
一個很特別的方程式.
只要你能夠相信.
你就能夠上天堂.
這個福音是一個比較傾向簡化.
和沒有關係的東西.
你問面對今天的香港.
我和你學習的這個福音.
有什麼意義呢.
我面對這樣的世界的時候.

$^{1841}$你跟我說.
想快速通過可以死去上天堂.
是好的.
但這個福音似乎和我們沒有什麼關係.
都是一個比較遠的故事.
但我們說福音其實不是一個罐頭.
福音是一個關乎於整個世界.
關乎於上帝國度的改變.
關乎於我們的未來.
這個福音就是關乎於我們整個人.
你不能夠只是單單聽著故事來做人.
你要整個人去投入下去.
這個是關乎於我們.
你怎樣能夠全個人來真誠地貫徹這個故事.
你只能夠跟著整個故事走下去.
所以第一件事就是.
福音其實是一種世界觀.
當你去信福音的時候.
福音其實是說你怎樣去理解這個世界.
當你面對著今天的世界的時候.
你的世界觀其實是改變了.
就算你面對著今天的香港.
福音其實是一個關乎於一個Good News的時候.
這個Good News就是說.
上帝的國度已經開始離林.
世界已經是轉變了.
這個是關乎於你怎樣去看香港.
你怎樣去看香港的未來.
怎樣去理解現在的生活.
當你有這個所謂Good News的時候.
其實你是否真的有.
你是否真的帶著一種好消息的態度來看著這個世界.
這個世界和福音沒有什麼關係.
我們有一個拿著的故事去講.
耶穌帶我們去信耶穌.
然後十字架.
怎樣能夠得到永生.
但當我們看著香港的時候.
我們仍然覺得很悲傷.
仍然覺得很絕望.

$^{1881}$其實這不是一個對劇本的人生觀.
當你真正信福音的時候.
福音其實是一種完全真真正正的世界觀.
你怎樣去看著今天的世界和香港很重要的改變.
福音是什麼.
這個許詩就是說耶穌的徒將要和已經是更新的世界.
改變的不是我的靈魂.
不是我做得聖變那麼簡單.
而是整個的世界已經是改變.
所以我能夠理解今天發生了什麼事情.
因為我知道這個世界是有一個好消息.
所以今天經常說.
今天我們說國安法之後我們怎樣傳福音.
我想說的是.
其實今天我們更加明白這個好消息是多麼好.
因為這個好消息正正告訴你.
我們面對著這個世界.
我們知道的消息就是這裡.
你明白那個興奮的地方在哪裡嗎.
如果我們真的去明白這個好消息的時候.
今天我們更加能夠真正去明白這個好消息的真正深度.
所以這個好消息是讓我們能夠.
所謂叫做好消息地活下去.
我們能夠懷著這種好消息來做人.
帶著一個好消息的態度來做人.
因為我們真正是信福音的.
我們真正去明白這個世界將會是怎麼走的時候.
我們就拿著這個福音來好好做人.
福音再不是一個純粹簡單的演繹.
一本書,一個福音橋就完了.
而是和我們的生活有很大關係.
當時福音就是這樣.
當我們面對著很多困境的時間.
我們就知道這個好消息究竟有多好.
和我們這個世界有什麼關係.
不過下一章.
所以說了這麼久.
什麼叫傳福音呢.
當我們知道福音的內容之後.
我們的福音是怎麼傳的呢.

$^{1921}$我們有沒有一套東西去傳呢.
或者我們是怎麼去傳呢.
如果福音是關乎於我們去看世界的視野的時候.
我們的福音的傳揚.
首先就是我們要這樣做人.
按著福音的世界觀來做人.
你知道這個是好消息.
你就帶著一種很開心.
或者說是一個好消息的態度去活下去.
從而你也會和別人說.
其實你面對著香港.
今天面對著蘋果被人這樣搞的時候.
我們仍然要去.
正正在這個位置.
你也要和別人說.
我有一個好消息要告訴你.
就算蘋果被人這樣搞.
好消息是這樣的.
因為到了最後的時間.
黑暗的勢力將會滅亡.
世界將會被更新.
福音是能夠和你今天所見到的新聞.
是有點接點的.
每一個六月發生的事情.
你都可以用一個好消息來延續下去.
因為這些新聞.
這些事情都不是最後的完結.
而是能夠用這個好消息來和別人說.
雖然是這樣.
雖然那個人被判無罪.
但我們仍然告訴他.
我有一個好消息要和你說.
這個好消息正正和你有關係.
是真真正正和你有關係.
不是一個二千年前的故事.
而是關乎於你今天身處在香港裡面.
仍然可以去堅持下去的一個好消息.
所以這個福音.
我們首先不是一種很簡單的故事.
而是我們的生活態度.

$^{1961}$你只能夠懷著這種生活態度.
來延續這個福音的內容.
然後向你身邊的人.
來告訴你這個好消息.
所以下次你和別人說的時候.
你不需要拿著什麼東西來和別人說.
你和別人說.
就算面對著今天這樣一個新聞的時候.
我都可以和你說一個好消息.
因為耶穌的來臨.
世界最終可以這樣這樣這樣.
世界是會改變的.
縱然這樣的時候.
這個好消息正正就是二千年前.
信那幫門徒和別人說的好消息.
縱然羅馬帝國是這樣的時候.
我正正和你說過.
The only one gospel.
一個這樣的好消息給別人聽.
當然我們仍然會有一個很特別的迷戀.
我們有一個福音五色珠.
以前大家有沒有玩過.
大家都試過都訓練過.
我們都有一個這樣的教導.
簡單的五色珠.
講給別人聽.
來講福音.
這個福音是有些缺陷的.
因為他只看那些書.
只講到你的靈魂得救就完了.
大聖耶穌你就能夠缺陷.
所以我自己就重新來做一個新的.
叫做Full Church 五色珠.
(笑聲).
這個五色珠是五色珠.
麻煩你幫我按按.
我們試一下這樣做.
都有五個步驟和別人講.
假設你個出信者.
未信的人和你說.

$^{2001}$我就試一下講.
第一個.
我們都是綠色.
綠色很簡單.
差不多和舊的那個.
綠色你會想到什麼呢.
會想到就是.
可能是一些樹葉,高山.
甚至生命.
是的.
因為枯葉並不是綠色的.
神造人原意是讓人得到生命的.
並且享受一個和平,美善,豐盛的生命.
這個是上帝原創的旨意.
這個版本是一樣的.
不過加了和平的字.
一個很簡單的創造.
我和你講.
這個是我們FourTruck的五色珠.
好 第一個.
第二個就是罪.
很可惜.
黑色代表什麼.
當然是代表.
黑色的東西.
(笑聲).
看這個社會.
充滿黑暗和罪惡.
可以這樣和我們講.
這是現代版本的五色珠.
因為人有罪.
身處在一個不公義和欺壓的世界.
充滿著謊言.
你可以加上一些新聞例子.
黑警,裸二等等.
(笑聲).
這樣講吧.
黑色的東西.
人充滿著罪.
所以每個人都是罪人.

$^{2041}$當我們面對著黑暗的世代的時候.
我們是罪人.
都是被罪者.
每個人都是犯罪的.
都是被罪所困擾的人.
我們是被罪所影響,被欺壓.
同時也成為了罪人.
因此聖經說.
世人都犯了罪.
無論是個人,社會,政權.
都不怕負擔神的心意.
所以這個罪不單單是個人的問題.
而是整個社會,整個世界的問題.
當然包括很多不公義的政權.
所以世界充滿著黑暗,荒謬,罪惡.
所以這是我們第二個顏色.
就是罪.
這個罪是關乎於更加闊諧.
一個社會上的罪.
一個不公義和黑暗的世代.
然後就是紅色.
其實也是粉紅色.
耶穌是尼賽亞.
萬國的君王.
我們可以說.
耶穌就是各國的君王.
是萬國之君.
耶穌在世的時候.
宣告上帝國道的好消息.
記得耶穌的反部福音.
一個負力福音的反部福音.
事實上,祂就是人類好消息的起頭.
馬克思第一章第一節里所講的理由.
他來到世上.
代表著上帝國道正要降臨.
關乎於整個國道的理能.
耶穌在十字架上的死.
毀滅了世界上一切黑暗的權勢.
人類得著解放.
所以十字架不單單是來叫我們聚得聖明.

$^{2081}$而是來消滅黑暗的勢力.
死亡的毒勾.
黑暗的勢力都能夠被破壞.
所以不是純粹個人的罪.
而是整個魔鬼的角度.
一個邪惡勢力都被毀滅.
耶穌的復活正是新創造的開始.
復活不是單單暗示了我們能夠得到永生的復活.
而是整個世界將要被更新.
而這已經是開始了.
上帝國道將要圓滿的臨到.
世界上所有不義的政權將要終結.
世界被更新.
並且帶來終極的公義和和平.
這就是十字架的意義.
一個更加完整的意義.
十字架不單單流血為我們死.
更加能夠勝過世界的黑暗.
這就是正式所講的.
然後就是白色.
白色傳統來說就是叫你信.
然後就是缺志.
我們就把它稱之為盼望.
基本上是信望外三個很重要的元素.
你跟他說你信不信這個好消息.
當你去講這個世界觀的時候.
如果你是相信這個世界觀的時候.
你就是開始去信這個福音.
如果你相信的話.
你也是在黑暗時代里成為一個好消息的部分.
你更加要成為這個宗教的部分.
因為耶穌基督要呼召你.
你用生死的態度來活出這個世界觀.
並且成為一個基督徒去傳揚盼望.
這就是第一堂所講的.
我們第一天開始就是被呼召去傳揚這個福音.
所以當你來到去信這個世界觀.
這個世界的將來的時候.
你就要好好地用生死的態度來活得像這個好消息一樣.
帶著這份好消息來做人.

$^{2121}$在上帝的角度上.
在你完全離開之前.
用愛來幫助世上所有需要的人.
並且懷著盼望面對黑暗的時代.
這個正正就是我們所謂大人所謂缺志.
那個很重要的元素.
不是單單信.
而是你要帶著盼望.
當你有這份福音的時候.
你就有這份盼望來看著這個世界.
並且用愛來幫助.
好像耶穌一樣.
在在世里幫助貧窮的人.
這十幾億的人.
並且活得好像好消息一樣.
最後一句.
請稱之為光復.
你會看到最後都是黃色.
上帝的國度將要離臨.
創世之初的公義和平友愛國度.
終要被光復.
即從前被黑暗勢力所影響的世界.
將要去過去.
上帝公義和平國度.
被恢復出來.
到那一天.
人類將要和諧共處.
共同活在沒有死亡.
沒有謊言.
沒有欺壓.
真正自由的國度裡面.
耶穌的祝福在我們的王子里.
永遠永遠.
所以這就是我們的.
不識諸.
當然將來不會教大家做這些事.
但起碼我們知道.
如果我們用榜樣來說.
我們這樣來教我們福音.
福音是關乎於更加闊的東西.

$^{2161}$我們能夠帶著福音.
帶著盼望的做人.
來和別人宣講.
當你有些朋友.
被現在的香港.
很多的形勢打倒的時候.
我們會和他說一個好消息.
這個好消息.
是真真正正能夠幫助他.
面對今天的世界.
今天的社會.
所以最後.
我和大家分享一段經文.
就是保羅所說的.
他說我不以福音為恥.
福音本是神的大能.
就是.
保羅所說的一個好消息.
其實是一個很荒謬的好消息.
因為這個好消息.
其實是沒有什麼人覺得可信的.
猶太人不信這個福音.
面對著當時的政權.
當他說耶穌基督的國度.
是一個很難以想象的事情.
今天我們也是一樣.
今天我們也是用福音來做.
我們人生很重要的生活態度.
我們不以福音為恥.
雖然現在面對世界.
和我們的好消息.
好像還未能對得上.
但我們確實仍然以福音.
來做我們生活每一刻的嚮導.
因為整件事情.
是上帝的大能.
來到香港裡面.
這個大能.
正正就是福音的根源.
讓我們能夠經歷上帝.

$^{2201}$在二千年前已經開始工作的大難.
這個好消息正正已經來了.
讓我們能夠懷著這份盼望.
來學習,來過每一天.
我只能祈禱.
在你知道你是那位尼賽亞.
你昔日來到世界裡面.
宣講神國福音的時候.
要告訴你這個世界的歷史已經改變.
這個世界的國度將要毀滅.
因為你的國度將要離林.
我們相信這份好消息.
並且我們將要以這份好消息.
作為我們人生最重要的嚮導.
當我們面對每一天香港的新聞的時候.
我們知道這個好消息.
正正是能夠幫助我們.
來面對這些事情.
因為我們知道這個好消息.
正正就是這個世界每一天的轉變.
最終的結局.
讓我們一群付出的人.
能夠帶著這份好消息來過活.
更加成為一個傳福音的人.
一個傳揚你好消息的門徒.
能夠讓每個人因為好消息.
這個福音能夠得以得著盼望.
得著改變.
求主你幫助我們.
奉主命求.
阿們.
江仔.
你都講了一個小時了.
是嗎?.
但是你還沒有叫食物.
是嗎?.
今天有什麼吃的?.
今天有沒有五色炒麵?.
我以為你只喝一杯.
相計我就是好消息.

$^{2241}$是嗎?.
其實我聽到你說了這麼久.
那個福音很多東西都可以包底的.
很多東西都可以搭到嘴巴的.
其實應該是.
如果包不到底就不是福音.
我看見你說用福音台聊天.
那個世界觀.
其實很方便.
我們都可以融入到人群.
是的,可以試一下.
我做茶餐廳的.
我接觸很多客人.
我可不可以用這個位置去接觸客人.
你試一下想想.
你可以怎麼做呢?.
一來就問他.
我識過五色豬.
我跟你講一下.
我想要牛腩河.
加凍奶茶.
你試一下試一下.
怎樣能夠講到福音呢?.
世界觀可以怎樣改變呢?.
人是要吃東西的.
是.
但是有些東西不是吃東西可以解決的.
是不是這樣呢?.
然後呢?.
所以人有沒有東西不是靠吃東西解決.
而是要一些更厲害的東西可以解決到人以外的東西.
是不是這樣呢?.
可以嗎?.
就是這樣.
吃著吃著就發現那些東西不公義.
哦,我知道.
不公義之後就發現耶穌更新了世界.
就是東西被人弄丟.
甚至我經常被人拆牌.
是.

$^{2281}$我開什麼都被人拆.
是不是這個意思?.
然後就說不要緊.
因為有好消息可以告訴你.
因為將來就會得到更加好的公義的審判.
哦,有點難.
我開茶餐廳.
很多街坊有很多不同的環境.
大家覺得什麼是好消息呢?.
對他們來說,今天聽完之後.
或者你覺得好消息有什麼困難呢?.
是,請說.
教會有教會的包袱.
可以多說一點什麼意思嗎?.
上一輩的那種傳福音.
你這樣又對,你這樣又不對.
一想起就想起大教會.
我們中國人已經少了教會的包袱.
哎,站起來,不可以,不用站起來.
我怕被人….
先說一句.
我想傳統的….
自己長大也有接觸過這些教會.
自己是基督徒也會覺得反感.
不是基督徒的時候.
我瞭解為什麼別人會覺得反感.
變成了要有一個位置.
跳過他們心理的關口.
就像我來樓堂.
我之前對教會很反感.
要來就要迫他們很大很大的能量.
要過來.
你叫一個不是基督徒的人去聽.
你一說基督徒想說基督徒的話.
他就想走,走開,彈開.
明白,對.
剛才瞭解你舊有的包袱是什麼呢?.
例如會是比較性的包袱.
哦,就是你有罪,你一定要….
對,罪人,你不下….

$^{2321}$其實真的很….
你不信神,落眼地獄.
只有這個神是對的.
你信佛就迷信.
我雖然沒見過這個神,但我不是迷信.
那一種很傳統就是.
我告訴你這個就是神.
你下地獄,你明明可以選擇上天堂.
你還是要選擇地獄.
那就是罪人.
那種背景的話.
這個是不是剛才說的那個比較狹窄的觀念?.
以前的那種看法.
那是難傳的,在這個年代里.
即時間變成跟人說地獄.
我也不知道怎麼回事.
所以要闊一點.
講一下政局.
講一下新聞,講政治.
這件事是能夠接到福音的內容裡面.
因為這個福音就是關乎於.
那個世界的發展之後的東西.
就多過一個神話.
講一下天堂,地獄的話.
這個就好像不是跟現在的一些….
除非那個人很興趣.
但其實應該是關我們現在發生的事.
就好像你剛才說到耶穌.
平時的日常就是去不同地方.
去瞭解人間的問題.
其他人呢?.
後面有.
剛剛說到福音.
其中一件事就是天國的來到.
這個世界上將會被改變,被更新.
我這樣聽下去是一個對於將來的盼望.
假設是對於一些不是去教會.
或者不是信基督教的朋友來說.
就是怎麼會令他們去相信這件事.
可能是我們是信開的.

$^{2361}$或者是我們是去教會的.
我們會傾向比較相信這一套.
但如果是說一些將來會發生的改變.
是由未發生的時候.
而又可能每天看著社會的情況.
其實說起來的時候.
怎麼可以令他們容易相信我們這一套.
我覺得是相信的.
因為現在的人都沒有出路.
你看著現在這樣的社會.
個個都不知道怎麼辦.
但你跟人家說.
現在的政權會怎樣結局.
其實人們是會想聽的.
比我叫你不要理會.
不如你上天堂.
因為兩個是很不同的.
都是福音來的.
但我們所說的福音.
是真的在說現在的處境.
現在很多人.
我覺得福出應該全部都是福音.
因為很多人是前線的人.
他們是很絕望.
當然移民是一個方法.
但我們去面對這些的時候.
有什麼出路.
很多人當他沒有信仰的時候.
其實他沒有出路.
而我們的信仰不是叫他.
不如你死去.
之後上天堂.
而是能夠說一個.
關乎於這個世界的事情的時候.
當你帶著這個好事來過活.
其實是你知道如何面對這件事.
剛才說白色那段.
就是盼望那段.
我們可以懷著盼望.
去面對這個世界.

$^{2401}$所以白色那段其實不是信耶穌.
而是把盼望放在我們比較重要的位置.
不是信完就行.
而是我們能夠懷著盼望.
去看著這個福音.
或者從另一個角度來看.
好像是將來式的.
現在處理不了.
好像將來式.
但其實也是一個過去式.
因為其實在不同年代都會遇到一些困境.
或者在不同年代當中.
都會有沒有出路的情況.
但是在不同的年代當中.
福音或者好信息.
都是帶動那些年代的人.
去經過那些日子.
所以歷史是不斷地重演.
也是重現在那個年代發生的事.
好像過去這段日子.
有不同的聚會或者主題都在說.
在文革時期的基督徒是怎麼生活.
或者在一個大政權之下.
基督徒或者基督教信仰.
怎麼可以延續對人的盼望的好信息.
所以這件事就.
到現在好像是將來式.
但其實一直以來可能會是過去都發生的事.
成為一個借鏡.
我們可以去想這個.
回應剛才所說的盼望.
嗯.
如果我們都知道有些事是要盼望.
以前都是.
摩西死的時候都見不到那些人去加拿大.
我們是明白的.
我都會覺得可能我死之前.
都未必會見到政權倒下來.
但是對於那些未信的人來說.
他好像如果我們這樣說的時候.

$^{2441}$他會不會覺得很壓縮精神.
你只是想.
我事實上就見到.
不停這樣的事.
每一天都會比每一天都更加厲害.
你喜歡做什麼就做什麼.
可能我.
看一份報紙都有罪.
我扔報紙都會有罪.
我覺得怎麼去解決.
當我們去傳福音的時候.
那些人他這樣去.
說的時候我們怎麼去回應呢.
嗯.
當然我們不是這樣去完.
你見到我們很多今天.
很多被人抓了的都是基督徒.
或者是青少年.
他們其實.
都是懷著很深層次的信仰的力量.
那個就是福音的內容.
他們會知道這個世界怎麼完.
才會到今天做事.
所以我們不是說Q就沒事了.
但很多時候我們.
今天去面對.
去改變一些東西.
雖然被人抓了.
但他們心底里很重要的一份力量.
就是來自於這份所謂的壓縮精神.
其實不是.
是信仰那個Good News.
當你連Good News都沒有.
你根本就沒什麼可能會去.
不容易去.
今天來去行動.
所以我們.
當然有些人信了很正常.
不是每個人都信的.
這個就是.

$^{2481}$世界上總有些人不信這個Good News.
不過很多人都是不信那個Good News.
不信那個天堂版本的福音.
你叫我天堂地獄.
我不是很信.
我都不在乎這件事.
但你和他說這個世界.
其實這些政權或者今天.
這個社會公義怎麼能夠得到圓滿.
得到一個比較圓滿的解決.
這件事起碼能夠和他今天身處的有關係.
當然會有些人不信.
但是我們.
當我們有這份Good News的時候.
我們就能夠去做當下的行動.
很多的今日政治家都是信仰的人.
正是因為這樣的原因.
有些人會覺得很奇怪.
但這個也是動力.
是你摸不到將來的力量.
如果你覺得阿Q是一些很不實在.
或者是個人FF的東西.
我覺得反而現在這個Good News.
就正正不是阿Q.
因為我們現在眼見現實的環境.
那種不公義.
那種讀灰.
那種指責的文化.
其實我們打開馬太豐第十章的時候.
正正耶穌說.
你們出去傳福音的時候.
你就會遇到這樣的情況.
你就會遇到家人的摒棄.
你就會遇到很多人帶你去會堂去公審.
這件事就更加讓我明白到.
聖經說的話對於末世是真實的.
聖經說的話其實在二千年前.
已經是成書告訴我們.
這個世界的終局會是這樣的.
我現在叫做有幸.

$^{2521}$去見證到聖經說的話的真實.
切切就不是阿Q.
所以聖經在告訴我們.
那個就是Good News.
這個也是讓我們明白到.
其實有一本書一直告訴我們.
我們可以看著人與人之間的分離.
那種終局.
那種盼望的失去.
但是有耶穌當中能夠成就這件事.
這個就不是阿Q.
這個正正就是我們有機會.
有法可醫的時候.
我們更加要學習耶穌.
那種福音的闊點.
就是做好我們的日常.
可以做到周祭.
關心.
接待.
和瞭解人需要.
那個責任和本分.
對人來說就是Good News.
後面呢.
聽到大家問這些問題.
我想分享一下.
今天剛剛和我兩個未信的同事.
講起政治的問題.
我是基督徒.
我真的和他們說.
我說我們有信仰是好一點的.
因為相信上帝始終都會審判.
那些不公義的行為.
我心裡面都會有.
剛才大家提起的那種掙扎.
就是他們不信.
聽到的時候.
他們會有什麼想法呢.
但我說出來的時候.
我知道這個真的是我作為基督徒.
我的相信.

$^{2561}$我真的真的相信.
上帝始終會去審判.
人類歷史里一切不公義的事.
因為我的同事都是很.
很追求公義的.
因為發生了蘋果日報的事件.
我們從2014年講到2019年.
講到今年.
真的有很多感受.
去到大家都很灰心.
講到大家覺得根本改變不了.
社會的任何東西.
由我們有機會爭取到.
連爭取的機會都沒有了.
但去到最後.
我還是很掙扎地告訴他們.
我說我相信就算我們今時今日.
沒有任何事情可以做到.
就算我們改變不了.
香港現在的局面.
但我始終都相信.
他們現在在為所有人.
做過不公義的事.
他們一定會承受他們自己的後果.
我其中一個同事.
他們兩個都不是基督徒.
我其中一個同事都說.
是的,他們會受他們應有的報應.
我都跟他們說.
雖然有生之年我們可能看不到這些.
但是.
我就是將我心裡面.
相信的那件事講了出來.
我會覺得是.
就算他們不相信也好.
但至少我要知道我所信的上帝.
是最公義的審判者.
無論我們現在看到.
在法律界裡面.
有多少.

$^{2601}$有多少立場很偏頗的律師.
或者是裁判官.
或者是法官都好.
但是我們還有一個最公義的法官.
我當時也是這樣跟我兩個未信的同事說.
我不知道他們會覺得是.
假Q還是什麼的東西.
但是.
事實是不變的.
不變的就是我們的上帝.
就是不會變的,祂就是公義.
有時候我覺得.
在面對這樣的情況.
我再跟姐妹分享.
有時候面對這樣的情況.
當我們.
怎樣才叫做全福音呢?.
未必一定會跟她這樣說.
有時候面對這樣的情況.
有一種淡定的態度很重要.
因為我知道這個好消息的時候.
我是能夠知道怎樣面對這件事.
這種態度其實是很舒服的.
因為.
我想補充.
好消息除了將要做事.
其實除了做事.
2000年前十字架的耶穌舉動是做了事.
耶穌舉動已經是消滅了黑暗力量.
做了,不過是未完全地來到去.
去實現整個國度的理能.
所以我們那種拍摩不是單單等它做事.
而是它做了事了.
所以你看著今天的政權或者黑暗勢力.
你是淡定的.
因為你已經玩完了.
就好像不得之拳.
你已經死了.
沒看過.
你已經死了.

$^{2641}$你看著它已經死了.
它好像很囂張.
但它已經死了.
這種淡定是很重要的.
所以我們跟別人宣講福音.
就是這種心路態度.
我們福音的態度.
當你真的相信這個好消息的時候.
你雖然是投入.
但你仍然可以.
那種盼望就在那時候.
我們能夠知道世界是怎樣完.
以及上帝已經做了事.
那我們就能夠去安慰.
或者跟別人一起走前面的路.
雖然大家都很害怕.
雖然大家都不知道怎樣前面走.
但你又不是完全不知道怎樣走.
因為你知道上帝已經做了事.
以及將來會怎樣.
還有每個人都會有自己的信念.
我都很認同剛才梓梅說.
在審判的觀念上.
我自己都跟自己說.
如果真的沒有審判.
其實做好人和做壞人是沒有分別的.
還有你做什麼.
其實沒有一件事去制裁.
或者做一件事有後果.
其實我為什麼這麼辛苦呢.
今天最困難的一件事.
就是做好人的負代價更加多.
而做好人更加責任.
對自己的要求高.
所以我們背後就知道.
在終末總會有一件事要平反.
還有終末會有一件事.
會跟你去看你的清單.
其實在做什麼.
所以我們真的相信.

$^{2681}$上帝是會做事的.
在將來他會再回來的時候.
這是我們很相信的.
其實很多未信的人.
或者對信仰不深的.
未認識很深的人.
其實底層都相信.
有些事是現在未行.
但將來不代表沒有.
不過在他的系統里.
在他的世界觀里.
沒有所謂可以解釋到.
但我們在對話當中.
就可以告訴你.
我們的信仰告訴你.
好消息可以幫他明白到.
將來審判上帝會做些什麼.
因為上帝之前已經做了.
就是敗壞一個掌死權.
就是政治和人.
即壞人可以殺死耶穌的身體.
但耶穌可以復活.
所以有些事情是.
過去人以為這樣就解決到.
但上帝會用第二個方法去解決.
這就是我們的信仰.
而那樣東西就真的做了.
所以上帝會再做其他東西.
去整治這個混亂的世代.
或者是歪曲的世代.
甚至有罪惡的世代.
這個講得最多就是.
我們等待的那一天來.
這個就是我們的盼望.
所以有審判.
其實不同人都有不同的程度理解.
但我們對審判的理解更加深入.
剛才聽到Kim Sir講.
其實在這個時代.
我們做什麼是很重要的.

$^{2721}$我記得2019年.
有一晚很多基督徒.
在正宗那裡唱.
Sing Hallelujah to the Lord.
我也是一個連燈粉絲.
我記得在那一晚之前.
其實很多連燈都叫基督徒.
都有一個名字叫做耶L.
但我記得那一晚.
所有基督徒在正宗門口.
唱了那首Sing Hallelujah.
然後我們去了灣仔警察總部.
唱了Sing Hallelujah.
我記得那一晚連燈.
很多人都說.
不要再叫他們耶L了.
我們證明叫他們基督徒了.
我覺得是在這些.
我不知道誰會受審判.
因為我們不是神.
但我很相信上帝.
一定有公義的審判對每一個人.
那我們可以做些什麼呢.
我講一個我覺得挺有趣的.
原來很多普通的人.
不是信徒會看你基督徒是怎樣的.
我前幾天去了一個朋友的餐廳吃飯.
他知道我是基督徒.
他說我跟你講一個很有趣的事.
因為他的餐廳在附近.
他說你是否回附近的教會.
我說不是.
不過那裡有個地方.
他說會不會是你們教會的人.
我說我不知道.
他說他們幾個是唱詩的.
他們在那裡不停地說.
三個唱詩都不合音.
他就說最合音的是誰.
然後有一個人回答最合音的是琴.

$^{2761}$然後他就覺得.
你們的基督徒都挺有趣的.
他說其實我很期待.
他們再來我這家餐廳吃飯.
我很想再聽他們講的東西.
其實很多外面的人.
是不知道你基督教是怎樣的.
可能因為比以前很多.
弱定俗成或固定的印象去看.
然後他聽到你們基督徒.
會講這麼有趣的東西.
其實他們覺得很有趣.
我覺得如果是好消息.
就是我們告訴他們.
其實在這個境況里.
我們是怎樣的.
其實我想我們的好消息.
就是告訴他們我們有信仰.
我們做人可以是怎樣.
然後給他們看.
然後我相信如果他們看到你真的好.
我相信他們會自己走來.
去問你基督教是怎樣的.
所以我的朋友也會問.
為什麼他們不合音.
為什麼不齊.
為什麼是琴.
反而是琴最合音.
我想知道多一點.
他反過來問我.
我覺得我們現在的信徒.
是否可以傳出好消息.
就是我們做好自己.
然後我們將我們所知的.
或者他們有問題.
我們能夠解決的.
去帶給他們.
我覺得是否可以做這些.
我想說最後的總結和補充.
其實今下的世代里.

$^{2801}$每一次的政治話題.
都是所謂的福音契機.
這個福音契機.
這個詞是很章的.
福音契機.
就是講好消息的一個機會.
就是一個接到嘴的時候.
你想想.
如果我們每一次講政治的時候.
其實都能夠接到嘴.
能夠去講好消息給別人聽.
這個正是一個最厲害的時候.
我們福音教會.
或者我們香港教會里.
能夠講福音契機非常多.
因為太多政治事件.
因為每一個事件.
都是能夠我們去講好消息給別人聽.
這就很舒服.
所以我覺得就是這樣.
你不需要找一次報道會.
講誰是藝人有癌症.
又是怎樣怎樣.
或者是什麼.
每一次這些社會話題.
都成為了我們講好消息給別人聽的機會.
就是這麼自然.
還有沒有其他?.
有沒有好消息?.
在後面.
剛才大家都講到關於.
大家都講到關於政治那邊.
我就在想我們怎樣可以向弱勢傳福音.
譬如見到一個執子婢的婆婆.
八十多歲,跟著她住tong 房.
她有拿過那些三國金.
但是都不夠交租.
又或者是一些無家者.
我們見到她跟她講傳福音.
可能都探了很多次.

$^{2841}$但是她都會跟你說.
你不要再跟我講福音了.
因為她的處境令到她.
曾經有人跟她講過.
但是對於她來講這件事是很.
對她們來講好像很遙遠.
那我們可以怎樣將這個福音帶給她們呢?.
就是怎樣可以講到對於她們來講是.
真的可以入到她們的心裡.
或者我跟你分享我們做社官.
或者做輪捨服務的做法.
我們的重點就是.
那個不是上而下.
不是說她是弱勢我們去幫她.
也不是說她沒有什麼我們要供應給她.
反而是一個平等.
她是我們社區的一部分.
我們本身是輪捨.
在輪捨過程當中.
有什麼是可以共享呢?.
因為她也是我們整個社區的共同參與者.
只不過她在她能力擁有的空間.
她沒有我們那麼多資源.
所以我們對她的互動過程當中.
她有物資我們可以分享.
比如我們有相關的.
一些可以支援的東西.
我們就讓她知道她要不要.
簡單來講就是.
如果過去有參與過平等分享行動.
一個社區輪捨的互動.
你就會發覺很多參與平等分享行動的人.
就會打開袋子讓她選擇內容.
而不是我覺得你需要藥油我就給你藥油.
而不是我覺得你需要高布我就給你高布.
你打開之後.
我們很多時候都會發現.
當我們做一些輪捨互動的時候.
打開的時候.
你會感受到一件事.

$^{2881}$她不是貪心.
她不是什麼都要.
她只是按她自己所需要的東西.
她只是拿她自己的一部分.
這個共享的文化.
對她來說.
就是我們能夠和她接點的好消息.
我不是硬要提出好消息.
正正就是她有些東西.
你覺得她需要她未必一定需要.
但你能夠和她共享.
那就是一個好消息.
所以可能要調整一下.
參與過程當中.
不是我們比較優越.
去向下服事.
而是和她平衡.
所以我自己在過去的日子.
和頂智妹一起去做.
我不斷和頂智妹分享.
我們落區做相關的參與.
我們只不過是.
恢復回耶穌基督的日常.
耶穌基督的日常就是周遊四方.
行善善 醫治各樣的病症.
見到有需要的人就伸出援手.
因此這個字就是.
屬靈果之第五個特質.
Kindness.
就是有需要的時候.
伸出援手.
就好像幫一幫她.
幫一幫她.
知道她有需要的時候.
我們能不能夠補充支援.
這對她來說就是好消息.
但剛才你分享的情況就是.
很多時候過去教會.
做這些社會服務.
或者社會參與的時候.

$^{2921}$就好像從上而下這樣做.
最後還要放一張單張進去.
甚至有個五色豬.
給她做手繩.
這些很多時候就是用福音.
打包這件事.
其實就曲解了參與.
這個我覺得就是滑蛇添足.
還有很多時候做的過程當中.
就太多一次性.
其實福音就是關係.
或者好消息是跟人接觸的關係.
或者瞭解到人的接點空間是重要的.
所以那個婆婆覺得反感是正常的.
因為好像要做完這件事.
才可以受惠於你.
其實是一件很煩厭的事.
各位你好.
我覺得耶穌有很多盼望.
好像香港市道差.
很難找到谷.
我覺得都是透過不同的心裡面去探望的.
做義工能夠得到很大的盼望的.
謝謝.
剛才說到.
剛才的問題.
我想說一點.
所以就說不要福音罐頭.
方單章的問題就是將福音變成了一段文字.
剛才說的只是玩玩而已.
不是要將我們那套東西變成文字.
這些都會變成罐頭.
所以我們的福音如果純粹變成了一個信息.
那就真的沒意思.
純粹加多一張單章下去.
那福音就不重要了.
所以對於那個婆婆來說.
怎麼能夠真正是一個好消息呢.
這個只能夠是你整個人投入.
你能夠在她面前帶給她.

$^{2961}$多過說成為一個純粹的.
弱化了的公式或者單章.
所以福音罐頭不會這麼做.
我們肯定不會派單章.
我們就是單章.
我們的關係就是單章.
我們的行動就是福音.
我想問有時這些社關.
可能你一路一路做.
就會去到一個瓶頸位.
就是我用不用再和他說一次福音呢.
如果我完全不提及福音.
就好像只是社關.
好像又不知如何是好.
但是如果你和他再提及福音.
會不會又回到以前的位置.
我只是傳福音給你.
人家會想你只是想我相信福音.
有時會面對瓶頸位.
可能久不久去到某個位置.
就會想要不要再和他說一次福音.
如果面對這些情況.
應該可以怎麼做呢.
我先把經驗分享一下.
通常你慢慢和他接觸一兩次之後.
他會對你有一個研究.
就會問你.
其實你沒有工作嗎.
為什麼你這麼有空呢.
你會告訴他.
其實我不是沒有工作.
我是特意抽時間做.
他會問你很有錢嗎.
為什麼你有這麼多物資呢.
我們會告訴他.
其實我們想資源重新分配.
有些人有更多資源和渠道的時候.
我們就願意和你分享.
他會問你很多問題.
為什麼你會這樣做.

$^{3001}$我們就會問他們.
就好像有人問你們心中盼望的緣由.
你就常常準備以溫柔敬畏的心回答他.
這是比特前書里的信息.
你刻意和他分享.
他會覺得分享完之後.
我就拿你的東西.
越打越挨.
我拿你的東西就聽你的.
但你不和他分享.
他就會有興趣問.
為什麼你們會這樣做.
因為這個世界不是這樣的.
但你就告訴他.
我們做的不是這個世界要做的事.
我們真的反世界而行的事.
他就會覺得.
你們是什麼人.
當然有些人很清楚教會的定位.
或者我們的做法.
他就會持續地拿.
他拿完就走.
他就會和你.
我又不會美化這件事.
他真的會和你無瓜葛.
他想快點完成這件事.
但是他知道這是一個渠道.
到時他會問.
或者我們也會面對的情況就是.
當我們見到他.
有些情況我們會問.
我們可不可以為你祈禱.
或者有什麼你想分享.
我們可以聽你.
其實人的心.
在當時就會有個觸動.
最後那個slide裡面說.
這個福音是上帝的大能.
我相信在福音的契機當中.
適時就會有那個工作.

$^{3041}$這個不是很遠的事.
但是最重點就是.
不是一次性可以見到一件事.
是我們遲之後可能成為我們的日常.
成為我們的生活態度.
這個好的信息就慢慢.
讓人可以參與.
至於會不會成為一個樽頸.
我覺得未必.
但是他想問我們相關的內容的時候.
我們都會告訴他.
其實福音是什麼.
或者我們的信仰是什麼.
可能都要幫他重新瞭解.
到時他接受的程度.
或者理解的範圍會多一些.
這個很重要.
我覺得.
什麼叫一次福音.
一次福音其實是很奇怪的東西.
一次福音是頭尾整個故事講出來.
教是可以教的一次福音.
不是一個定義.
但是我就不是這麼相信.
我不是聽一次福音這麼相信.
可能我上教會的時候.
很溫暖.
看一段聖經.
一起祈禱.
一群人這樣生活.
那個東西是一個累積而成的東西.
通常聽一次福音已經相信了.
重新聽一次.
整個原理版本.
但首先他相信了.
想聽一次的意思.
對他來說.
福音其實是一個.
不斷地慢慢滲入.
那個見證.

$^{3081}$所以我都不需要.
有些人很反對.
做社會肯定不能說福音.
其實不需要那麼硬來.
說還是不說.
因為福音從來都不是一次半次.
講出來的東西.
就是你跟他說.
跟他祈禱.
跟他說耶穌.
這個人.
這個事情.
加起來就是你整個的見證.
所以我覺得不是一次福音.
還是兩次福音的問題.
福音不是一次兩次.
這樣講出來的故事.
不是單章.
所以我們所傳揚的.
正正就是一個你行動裡面.
不斷來到他.
累積出來的關係和見證.
所以當你想到一次福音時.
已經是.
一個故事形式的福音.
其實就不需要一定這樣.
想起上課是需要講一次給他聽的.
但這個已經不是.
信耶穌那件事了.
聖經裡面用.
撒種來處理那種.
多次性的情況.
撒種裡面的教導就說.
那個種子需要時間去.
醞釀發芽.
所以仍然是講福音.
是那種關係性.
如何讓他理解.
生命可以有另一個出口.
或者生命有另一個轉態.

$^{3121}$這個是關鍵.
有時候我不懂得如何去.
向我的朋友解釋.
我的朋友可能會問我.
如果這個神真的那麼.
外來那麼厲害的話.
為什麼他還會容許.
那麼不公義的事情發生呢.
為什麼好像連一些聲音.
都不容許呢.
我知道.
之前潘Sir可能會說過.
聖經其實是一個人生的菜單.
很多東西都可以在那裡找到.
是一個人生的指南.
但是不信的人.
又如何去聽我引經據典.
去講這些呢.
我如何貼地地.
去和他解釋這個好消息呢.
我都說了.
你要知道.
其實神是做了事情.
又會完全讓你見到.
祂做了的事情.
所以我們說.
怎麼說呢.
對於上帝.
如何去看待世界呢.
我經常覺得世界的開始和終結.
是一個很永恆的小息.
永恆到永恆中間.
一個很小的小息.
這就是世界歷史.
所以上帝不是不做事.
而是對人的短暫時間.
就是一個所謂的黑暗勢力.
不好的東西.
但是對上帝來說.
是一個很微不足道的事情.

$^{3161}$當然我們覺得很不容易.
也很難過.
因為時間很長.
但是對上帝來說.
這是一個很輕而易舉.
能夠解決的事情.
我們叫做不可能的可能性.
其實祂在神面前.
已經不存在.
都說祂已經死了.
但是對我們來說.
好像是很漫長.
所以上帝不是不做事.
其實上帝已經做了事.
對我們人的眼光來說.
仍然是一個很微小的角度.
當然是很不容易過渡的.
我這樣理解.
當然我們可以想象.
一個能夠即時破滅.
殺死所有的東西的神.
其實這是我們人所想象.
一個比較弱的超級英雄.
能夠做到的事情.
但是上帝是終極地解決問題.
所以我們有時要有眼光和軟度.
來看世界的黑暗.
我會這樣去說.
我曾經在《射傑美》的時候.
說過一篇信息.
都是說關於.
那些人很期望上帝.
即時做出審判或整治.
我當時其中一個解讀是.
其實我們人很短視和心急.
很想那些壞人在你面前撲倒.
但是你會發覺.
很多時候都沒有那麼.
天氣不似預期.
但是最後.

$^{3201}$是否因為不能滿足你的期望.
或者你的時間表.
你就覺得上帝不正.
或者不厲害呢.
我會回到剛才所說的.
上帝是在做事.
和上帝也在做事.
就如剛才John所說.
因為上帝在做事.
因為有審判.
上帝就在看這群人不斷發生什麼事.
上帝是在做事.
因為之前都會有歷史.
重復這些類近的情況.
但是當你回顧歷史的時候.
上帝其實已經對那些人.
有那時候的懲治.
只不過人在現在的進行當中.
看不到上帝工作的時刻.
這個情況.
所以我記得我和兩個兒子.
在家裡看電視的時候.
我的兒子就說.
餵,他們又說謊了.
然後我說是.
他說,傳導人你怎麼看這些事.
然後,那時候是七八月的時候.
19年.
每次四點鐘的時候都說謊.
然後我看著.
我說,我的答案就是.
我和他說.
你看多少,上帝也看多少.
在聖經的角度來說.
他們所做的.
是在上帝面前.
即使上帝對他們的憤怒.
這個是聖經說得很清楚.
所以對於上帝正在做事.
或者上帝將會做事.

$^{3241}$那個是將來式.
但仍然是現在進行.
但不是按我們的劇本.
或者按我們的時間.
這個在彼得前書也在說.
就是上帝不願意人沈淪.
願意萬人悔改.
這個在上帝計劃當中有時間表.
所以回應你朋友提問的時候.
我們會有些無奈.
因為上帝不是我們說.
要做什麼就做什麼.
但我們看回過去歷史和聖經的教導.
其實上帝是過去做事.
現在做事.
將來也會做事.
這個就是我們所相信.
那個有真有活的上帝.
我想問一個問題.
這個很宏觀.
關於世界的一個好消息.
其實和神愛世人.
這個概念有沒有些抵觸呢.
因為以往那個罐頭福音.
其實是很直接.
人犯罪.
然後信耶穌就上天堂.
神愛世人.
你信了就上天堂.
這個很直接.
所以比較受港豬歡迎.
但是這套宏觀的好消息.
其實是講世界觀.
去到最後.
第五隻豬.
其實是.
感覺像是死得人多.
其實如何可以.
與一些未信主的人聯繫.
去聽的時候.

$^{3281}$會覺得這個信仰.
是和自己有關係.
或者將來.
釘了之後有些關係.
可以如何去講.
去到最後的時候.
其實神愛世人的原文.
其實是什麼.
如果你看過.
就是什麼.
就是Cosmos.
就是神愛世界.
所以其實.
神愛世界是不足夠的.
在中文翻譯裡面.
其實神不單單愛世人.
更加不是愛基督徒.
而是愛整個世界所有的東西.
所以上帝.
在《藥王方經》中所講的愛.
其實是更加.
闊於我們所想的.
不是說我們有罪.
然後就能夠得救.
或者我們信仰得救.
上帝的愛是關乎於.
整個世界.
能夠得到改變.
所以基督徒環保也有關係.
因為我們所講的.
那個世界將會被更新.
所以這件事反而.
更加闊於我們所理解.
所以上帝愛這個世界的時候.
其實不是說熟悉這個世界.
而是很愛這個世界.
更新這個世界.
所以上帝的福音是說.
這個世界的歷史和改變.
多於純粹人的事情.

$^{3321}$所以上帝不單單愛.
一些信許人.
上帝愛所有的人.
所以這個所謂的福音.
是關乎於全人類.
雖然你可以不相信.
但不代表上帝不愛你.
上帝的愛的救贖.
也包括在一些.
所謂不相信這個計劃的人身上.
所以我覺得反而.
是闊於我們以前所理解的.
神話世人.
一個月才來一次.
你會點什麼吃呢?.
還沒點什麼吃.
過了幾個小時都沒來.
那就要等下一次才點.
我已經關門了.
好啊.
到下個月才見.
下次是什麼呢?.
我先想想.
下次好像是講教會.
那就是要叫多些不是教會的人來.
還是叫多些教會的人來呢?.
我真的要多點位置.
好啊.
叫多些人來可以試下一次.
那就下個月見了.
拜拜.
《香港》 作詞:陳汝佳 作曲:陳汝佳.
香港 我心中的國鄉.
這裡讓我生長.
有我喜歡的親友共陽光.
路上人在跑 過他港.
感驚為我欣賞.
這裡有許多好處沒發覺.
食一聲香港 香港.
你永遠是塵埃香.

$^{3361}$香港 香港.
你那色調那望.
山頂看小島水淚淌.
處處換上新裝.
看看那海鷗飛過自由港.
海邊看小島處萬丈.
處處搖曳新光.
這個市區的吸引沒法擋.
食一聲香港 香港.
再有我童年夢想.
香港 香港.
叫我不以為望.
香港 我心中的故鄉.
這裡讓我生長.
有我喜歡的親友共陽光.
路上人在跑 過他港.
感驚為我欣賞.
這裡有許多好處沒發覺.
食一聲香港 香港.
你永遠是塵埃香.
香港 香港.
你那色調那望.
香港 香港.
再有我童年夢想.
香港 香港.
QQ我吧.
\newpage



\section{}
\label{sec:dNWjC8vnhS0}
\textbf{【這是最好的時代:給香港基督徒的神學八課】第3課: Let’s flow|20210726 [dNWjC8vnhS0]}
\newline
\newline
連結: \href{https://youtube.com/watch?v=dNWjC8vnhS0}{\texttt{ https://youtube.com/watch?v=dNWjC8vnhS0}} ~~~~ 語音日期: 2021-07-26 
\newline
\newline
\hyperref[sec:Gv9ZCQJGqmE]{\small{< < < PREV SERMON < < <}}
~
\hyperref[sec:index_chronic]{\small{[返順時目]}}
~
\hyperref[sec:index_scriptual]{\small{[返順卷目]}}
~
\hyperref[sec:Gprv_Nw0Oi4]{\small{> > > NEXT SERMON > > >}}
\newline
\newline
$^{1}$我只想知道.
你到底是什麼意思.
我只想知道.
你到底是什麼意思.
我只想知道.
你到底是什麼意思.
我只想知道.
你到底是什麼意思.
我只想知道.
你到底是什麼意思.
我只想知道.
你到底是什麼意思.
我只想知道.
你到底是什麼意思.
我只想知道.
你到底是什麼意思.
我只想知道.
你到底是什麼意思.
我只想知道.
你到底是什麼意思.
我只想知道.
你到底是什麼意思.
我只想知道.
你到底是什麼意思.
我只想知道.
你到底是什麼意思.
我只想知道.
你到底是什麼意思.
我只想知道.
你到底是什麼意思.
我只想知道.
你到底是什麼意思.
我只想知道.
你到底是什麼意思.
我只想知道.
你到底是什麼意思.
我只想知道.
你到底是什麼意思.
我只想知道.
你到底是什麼意思.

$^{41}$我只想知道.
你到底是什麼意思.
我只想知道.
你到底是什麼意思.
我只想知道.
你到底是什麼意思.
我只想知道.
你到底是什麼意思.
我只想知道.
你到底是什麼意思.
我只想知道.
你到底是什麼意思.
我只想知道.
你到底是什麼意思.
我只想知道.
你到底是什麼意思.
我只想知道.
你到底是什麼意思.
我只想知道.
你到底是什麼意思.
我只想知道.
你到底是什麼意思.
我只想知道.
你到底是什麼意思.
我只想知道.
你到底是什麼意思.
我只想知道.
你到底是什麼意思.
我只想知道.
你到底是什麼意思.
我只想知道.
你到底是什麼意思.
我只想知道.
你到底是什麼意思.
我只想知道.
你到底是什麼意思.
我只想知道.
你到底是什麼意思.
我只想知道.
你到底是什麼意思.

$^{81}$我只想知道.
你到底是什麼意思.
我只想知道.
你到底是什麼意思.
我只想知道.
你到底是什麼意思.
我只想知道.
你到底是什麼意思.
我只想知道.
你到底是什麼意思.
我只想知道.
你到底是什麼意思.
我只想知道.
你到底是什麼意思.
我只想知道.
你到底是什麼意思.
我只想知道.
你到底是什麼意思.
我只想知道.
你到底是什麼意思.
我只想知道.
你到底是什麼意思.
我只想知道.
你到底是什麼意思.
我只想知道.
你到底是什麼意思.
我只想知道.
你到底是什麼意思.
我只想知道.
你到底是什麼意思.
我只想知道.
你到底是什麼意思.
我只想知道.
你到底是什麼意思.
我只想知道.
你到底是什麼意思.
我只想知道.
你到底是什麼意思.
我只想知道.
你到底是什麼意思.

$^{121}$我只想知道.
你到底是什麼意思.
我只想知道.
你到底是什麼意思.
我只想知道.
你到底是什麼意思.
我只想知道.
你到底是什麼意思.
我只想知道.
你到底是什麼意思.
我只想知道.
你到底是什麼意思.
我只想知道.
你到底是什麼意思.
我只想知道.
你到底是什麼意思.
我只想知道.
你到底是什麼意思.
我只想知道.
你到底是什麼意思.
我只想知道.
你到底是什麼意思.
我只想知道.
你到底是什麼意思.
我只想知道.
你到底是什麼意思.
我只想知道.
你到底是什麼意思.
我只想知道.
你到底是什麼意思.
我只想知道.
你到底是什麼意思.
我只想知道.
你到底是什麼意思.
我只想知道.
你到底是什麼意思.
我只想知道.
你到底是什麼意思.
我只想知道.
你到底是什麼意思.

$^{161}$我只想知道.
你到底是什麼意思.
我只想知道.
你到底是什麼意思.
我只想知道.
你到底是什麼意思.
我只想知道.
你到底是什麼意思.
我只想知道.
你到底是什麼意思.
我只想知道.
你到底是什麼意思.
我只想知道.
你到底是什麼意思.
我只想知道.
你到底是什麼意思.
我只想知道.
你到底是什麼意思.
我只想知道.
你到底是什麼意思.
我只想知道.
你到底是什麼意思.
我只想知道.
你到底是什麼意思.
我只想知道.
你到底是什麼意思.
我只想知道.
你到底是什麼意思.
我只想知道.
你到底是什麼意思.
我只想知道.
你到底是什麼意思.
我只想知道.
你到底是什麼意思.
我只想知道.
你到底是什麼意思.
我只想知道.
你到底是什麼意思.
我只想知道.
你到底是什麼意思.

$^{201}$我只想知道.
你到底是什麼意思.
我只想知道.
你到底是什麼意思.
我只想知道.
你到底是什麼意思.
我只想知道.
你到底是什麼意思.
我只想知道.
你到底是什麼意思.
我只想知道.
你到底是什麼意思.
我只想知道.
你到底是什麼意思.
我只想知道.
你到底是什麼意思.
我只想知道.
你到底是什麼意思.
我只想知道.
你到底是什麼意思.
我只想知道.
你到底是什麼意思.
我只想知道.
你到底是什麼意思.
我只想知道.
你到底是什麼意思.
我只想知道.
你到底是什麼意思.
我只想知道.
你到底是什麼意思.
我只想知道.
你到底是什麼意思.
我只想知道.
你到底是什麼意思.
我只想知道.
你到底是什麼意思.
我只想知道.
你到底是什麼意思.
我只想知道.
你到底是什麼意思.

$^{241}$我只想知道.
你到底是什麼意思.
我只想知道.
你到底是什麼意思.
我只想知道.
你到底是什麼意思.
我只想知道.
你到底是什麼意思.
我只想知道.
你到底是什麼意思.
我只想知道.
你到底是什麼意思.
我只想知道.
你到底是什麼意思.
我只想知道.
你到底是什麼意思.
我只想知道.
你到底是什麼意思.
我只想知道.
你到底是什麼意思.
我只想知道.
你到底是什麼意思.
我只想知道.
你到底是什麼意思.
我只想知道.
你到底是什麼意思.
我只想知道.
你到底是什麼意思.
我只想知道.
你到底是什麼意思.
我只想知道.
你到底是什麼意思.
我只想知道.
你到底是什麼意思.
我只想知道.
你到底是什麼意思.
我只想知道.
你到底是什麼意思.
我只想知道.
你到底是什麼意思.

$^{281}$我只想知道.
你到底是什麼意思.
我只想知道.
你到底是什麼意思.
我只想知道.
你到底是什麼意思.
我只想知道.
你到底是什麼意思.
我只想知道.
你到底是什麼意思.
我只想知道.
你到底是什麼意思.
我只想知道.
你到底是什麼意思.
我只想知道.
你到底是什麼意思.
我只想知道.
你到底是什麼意思.
我只想知道.
你到底是什麼意思.
我只想知道.
你到底是什麼意思.
我只想知道.
你到底是什麼意思.
我只想知道.
你到底是什麼意思.
我只想知道.
你到底是什麼意思.
我只想知道.
你到底是什麼意思.
我只想知道.
你到底是什麼意思.
我只想知道.
你到底是什麼意思.
我只想知道.
你到底是什麼意思.
我只想知道.
你到底是什麼意思.
我只想知道.
你到底是什麼意思.

$^{321}$我只想知道.
你到底是什麼意思.
我只想知道.
你到底是什麼意思.
我只想知道.
你到底是什麼意思.
我只想知道.
你到底是什麼意思.
我只想知道.
你到底是什麼意思.
我只想知道.
你到底是什麼意思.
我只想知道.
你到底是什麼意思.
我只想知道.
你到底是什麼意思.
我只想知道.
你到底是什麼意思.
我只想知道.
你到底是什麼意思.
我只想知道.
你到底是什麼意思.
我只想知道.
你到底是什麼意思.
我只想知道.
你到底是什麼意思.
我只想知道.
你到底是什麼意思.
我只想知道.
你到底是什麼意思.
我只想知道.
你到底是什麼意思.
我只想知道.
你到底是什麼意思.
我只想知道.
你到底是什麼意思.
我只想知道.
你到底是什麼意思.
我只想知道.
你到底是什麼意思.

$^{361}$我只想知道.
你到底是什麼意思.
我只想知道.
你到底是什麼意思.
我只想知道.
你到底是什麼意思.
我只想知道.
你到底是什麼意思.
我只想知道.
你到底是什麼意思.
我只想知道.
你到底是什麼意思.
我只想知道.
你到底是什麼意思.
我只想知道.
你到底是什麼意思.
我只想知道.
你到底是什麼意思.
我只想知道.
你到底是什麼意思.
我只想知道.
你到底是什麼意思.
我只想知道.
你到底是什麼意思.
我只想知道.
你到底是什麼意思.
我只想知道.
你到底是什麼意思.
我只想知道.
你到底是什麼意思.
我只想知道.
你到底是什麼意思.
我只想知道.
你到底是什麼意思.
我只想知道.
你到底是什麼意思.
我只想知道.
你到底是什麼意思.
我只想知道.
你到底是什麼意思.

$^{401}$我只想知道.
你到底是什麼意思.
我只想知道.
你到底是什麼意思.
我只想知道.
你到底是什麼意思.
我只想知道.
你到底是什麼意思.
我只想知道.
你到底是什麼意思.
我只想知道.
你到底是什麼意思.
我只想知道.
你到底是什麼意思.
我只想知道.
你到底是什麼意思.
我只想知道.
你到底是什麼意思.
我只想知道.
你到底是什麼意思.
我只想知道.
你到底是什麼意思.
我只想知道.
你到底是什麼意思.
我只想知道.
你到底是什麼意思.
我只想知道.
你到底是什麼意思.
我只想知道.
你到底是什麼意思.
我只想知道.
你到底是什麼意思.
我只想知道.
你到底是什麼意思.
我只想知道.
你到底是什麼意思.
我只想知道.
你到底是什麼意思.
我只想知道.
你到底是什麼意思.

$^{441}$我只想知道.
你到底是什麼意思.
我只想知道.
你到底是什麼意思.
我只想知道.
你到底是什麼意思.
我只想知道.
你到底是什麼意思.
我只想知道.
你到底是什麼意思.
我只想知道.
你到底是什麼意思.
我只想知道.
你到底是什麼意思.
我只想知道.
你到底是什麼意思.
我只想知道.
你到底是什麼意思.
我只想知道.
你到底是什麼意思.
我只想知道.
你到底是什麼意思.
我只想知道.
你到底是什麼意思.
我只想知道.
你到底是什麼意思.
我只想知道.
你到底是什麼意思.
我只想知道.
你到底是什麼意思.
我只想知道.
你到底是什麼意思.
我只想知道.
你到底是什麼意思.
我只想知道.
你到底是什麼意思.
我只想知道.
你到底是什麼意思.
我只想知道.
你到底是什麼意思.

$^{481}$我只想知道.
你到底是什麼意思.
我只想知道.
你到底是什麼意思.
我只想知道.
你到底是什麼意思.
我只想知道.
你到底是什麼意思.
我只想知道.
你到底是什麼意思.
我只想知道.
你到底是什麼意思.
我只想知道.
你到底是什麼意思.
我只想知道.
你到底是什麼意思.
我只想知道.
你到底是什麼意思.
我只想知道.
你到底是什麼意思.
我只想知道.
你到底是什麼意思.
我只想知道.
你到底是什麼意思.
我只想知道.
你到底是什麼意思.
我只想知道.
你到底是什麼意思.
我只想知道.
你到底是什麼意思.
我只想知道.
你到底是什麼意思.
我只想知道.
你到底是什麼意思.
我只想知道.
你到底是什麼意思.
我只想知道.
你到底是什麼意思.
我只想知道.
你到底是什麼意思.

$^{521}$我只想知道.
你到底是什麼意思.
我只想知道.
你到底是什麼意思.
我只想知道.
你到底是什麼意思.
我只想知道.
你到底是什麼意思.
我只想知道.
你到底是什麼意思.
我只想知道.
你到底是什麼意思.
我只想知道.
你到底是什麼意思.
我只想知道.
你到底是什麼意思.
我只想知道.
你到底是什麼意思.
我只想知道.
你到底是什麼意思.
我只想知道.
你到底是什麼意思.
我只想知道.
你到底是什麼意思.
我只想知道.
你到底是什麼意思.
我只想知道.
你到底是什麼意思.
我只想知道.
你到底是什麼意思.
我只想知道.
你到底是什麼意思.
我只想知道.
你到底是什麼意思.
我只想知道.
你到底是什麼意思.
我只想知道.
你到底是什麼意思.
我只想知道.
你到底是什麼意思.

$^{561}$我只想知道.
你到底是什麼意思.
我只想知道.
你到底是什麼意思.
我只想知道.
你到底是什麼意思.
我只想知道.
你到底是什麼意思.
我只想知道.
你到底是什麼意思.
我只想知道.
你到底是什麼意思.
我只想知道.
你到底是什麼意思.
我只想知道.
你到底是什麼意思.
我只想知道.
你到底是什麼意思.
我只想知道.
你到底是什麼意思.
我只想知道.
你到底是什麼意思.
我只想知道.
你到底是什麼意思.
我只想知道.
你到底是什麼意思.
我只想知道.
你到底是什麼意思.
我只想知道.
你到底是什麼意思.
我只想知道.
你到底是什麼意思.
我只想知道.
你到底是什麼意思.
我只想知道.
你到底是什麼意思.
我只想知道.
你到底是什麼意思.
我只想知道.
你到底是什麼意思.

$^{601}$我只想知道.
你到底是什麼意思.
我只想知道.
你到底是什麼意思.
我只想知道.
你到底是什麼意思.
我只想知道.
你到底是什麼意思.
我只想知道.
你到底是什麼意思.
我只想知道.
你到底是什麼意思.
我只想知道.
你到底是什麼意思.
我只想知道.
你到底是什麼意思.
我只想知道.
你到底是什麼意思.
我只想知道.
你到底是什麼意思.
我只想知道.
你到底是什麼意思.
我只想知道.
你到底是什麼意思.
我只想知道.
你到底是什麼意思.
我只想知道.
你到底是什麼意思.
我只想知道.
你到底是什麼意思.
我只想知道.
你到底是什麼意思.
我只想知道.
你到底是什麼意思.
我只想知道.
你到底是什麼意思.
我只想知道.
你到底是什麼意思.
我只想知道.
你到底是什麼意思.

$^{641}$我只想知道.
你到底是什麼意思.
我只想知道.
你到底是什麼意思.
我只想知道.
你到底是什麼意思.
我只想知道.
你到底是什麼意思.
我只想知道.
你到底是什麼意思.
我只想知道.
你到底是什麼意思.
我只想知道.
你到底是什麼意思.
我只想知道.
你到底是什麼意思.
我只想知道.
你到底是什麼意思.
我只想知道.
你到底是什麼意思.
我只想知道.
你到底是什麼意思.
我只想知道.
你到底是什麼意思.
我只想知道.
你到底是什麼意思.
我只想知道.
你到底是什麼意思.
我只想知道.
你到底是什麼意思.
我只想知道.
你到底是什麼意思.
我只想知道.
你到底是什麼意思.
我只想知道.
你到底是什麼意思.
我只想知道.
你到底是什麼意思.
我只想知道.
你到底是什麼意思.

$^{681}$我只想知道.
你到底是什麼意思.
我只想知道.
你到底是什麼意思.
我只想知道.
你到底是什麼意思.
我只想知道.
你到底是什麼意思.
我只想知道.
你到底是什麼意思.
我只想知道.
你到底是什麼意思.
我只想知道.
你到底是什麼意思.
我只想知道.
你到底是什麼意思.
我只想知道.
你到底是什麼意思.
我只想知道.
你到底是什麼意思.
我只想知道.
你到底是什麼意思.
我只想知道.
你到底是什麼意思.
我只想知道.
你到底是什麼意思.
我只想知道.
你到底是什麼意思.
我只想知道.
你到底是什麼意思.
我只想知道.
你到底是什麼意思.
我只想知道.
你到底是什麼意思.
我只想知道.
你到底是什麼意思.
我只想知道.
你到底是什麼意思.
我只想知道.
你到底是什麼意思.

$^{721}$我只想知道.
你到底是什麼意思.
我只想知道.
你到底是什麼意思.
我只想知道.
你到底是什麼意思.
我只想知道.
你到底是什麼意思.
我只想知道.
你到底是什麼意思.
我只想知道.
你到底是什麼意思.
我只想知道.
你到底是什麼意思.
我只想知道.
你到底是什麼意思.
我只想知道.
你到底是什麼意思.
我只想知道.
你到底是什麼意思.
我只想知道.
你到底是什麼意思.
我只想知道.
你到底是什麼意思.
我只想知道.
你到底是什麼意思.
我只想知道.
你到底是什麼意思.
我只想知道.
你到底是什麼意思.
我只想知道.
你到底是什麼意思.
我只想知道.
你到底是什麼意思.
我只想知道.
你到底是什麼意思.
我只想知道.
你到底是什麼意思.
我只想知道.
你到底是什麼意思.

$^{761}$我只想知道.
你到底是什麼意思.
我只想知道.
你到底是什麼意思.
我只想知道.
你到底是什麼意思.
我只想知道.
你到底是什麼意思.
我只想知道.
你到底是什麼意思.
我只想知道.
你到底是什麼意思.
我只想知道.
你到底是什麼意思.
我只想知道.
你到底是什麼意思.
我只想知道.
你到底是什麼意思.
我只想知道.
你到底是什麼意思.
我只想知道.
你到底是什麼意思.
我只想知道.
你到底是什麼意思.
我只想知道.
你到底是什麼意思.
我只想知道.
你到底是什麼意思.
我只想知道.
你到底是什麼意思.
我只想知道.
你到底是什麼意思.
我只想知道.
你到底是什麼意思.
我只想知道.
你到底是什麼意思.
我只想知道.
你到底是什麼意思.
我只想知道.
你到底是什麼意思.

$^{801}$我只想知道.
你到底是什麼意思.
我只想知道.
你到底是什麼意思.
我只想知道.
你到底是什麼意思.
我只想知道.
你到底是什麼意思.
我只想知道.
你到底是什麼意思.
我只想知道.
你到底是什麼意思.
我只想知道.
你到底是什麼意思.
我只想知道.
你到底是什麼意思.
我只想知道.
你到底是什麼意思.
我只想知道.
你到底是什麼意思.
我只想知道.
你到底是什麼意思.
我只想知道.
你到底是什麼意思.
我只想知道.
你到底是什麼意思.
我只想知道.
你到底是什麼意思.
我只想知道.
你到底是什麼意思.
我只想知道.
你到底是什麼意思.
我只想知道.
你到底是什麼意思.
我只想知道.
你到底是什麼意思.
我只想知道.
你到底是什麼意思.
我只想知道.
你到底是什麼意思.

$^{841}$我只想知道.
你到底是什麼意思.
我只想知道.
你到底是什麼意思.
我只想知道.
你到底是什麼意思.
我只想知道.
你到底是什麼意思.
我只想知道.
你到底是什麼意思.
我只想知道.
你到底是什麼意思.
我只想知道.
你到底是什麼意思.
我只想知道.
你到底是什麼意思.
我只想知道.
你到底是什麼意思.
我只想知道.
你到底是什麼意思.
我只想知道.
你到底是什麼意思.
我只想知道.
你到底是什麼意思.
我只想知道.
你到底是什麼意思.
我只想知道.
你到底是什麼意思.
我只想知道.
你到底是什麼意思.
我只想知道.
你到底是什麼意思.
我只想知道.
你到底是什麼意思.
我只想知道.
你到底是什麼意思.
我只想知道.
你到底是什麼意思.
我只想知道.
你到底是什麼意思.

$^{881}$我只想知道.
你到底是什麼意思.
我只想知道.
你到底是什麼意思.
我只想知道.
你到底是什麼意思.
我只想知道.
你到底是什麼意思.
我只想知道.
你到底是什麼意思.
我只想知道.
你到底是什麼意思.
我只想知道.
你到底是什麼意思.
我只想知道.
你到底是什麼意思.
我只想知道.
你到底是什麼意思.
我只想知道.
你到底是什麼意思.
我只想知道.
你到底是什麼意思.
我只想知道.
你到底是什麼意思.
我只想知道.
你到底是什麼意思.
我只想知道.
你到底是什麼意思.
我只想知道.
你到底是什麼意思.
我只想知道.
你到底是什麼意思.
我只想知道.
你到底是什麼意思.
我只想知道.
你到底是什麼意思.
我只想知道.
你到底是什麼意思.
我只想知道.
你到底是什麼意思.

$^{921}$我只想知道.
你到底是什麼意思.
我只想知道.
你到底是什麼意思.
我只想知道.
你到底是什麼意思.
我只想知道.
你到底是什麼意思.
我只想知道.
你到底是什麼意思.
我只想知道.
你到底是什麼意思.
我只想知道.
你到底是什麼意思.
我只想知道.
你到底是什麼意思.
我只想知道.
你到底是什麼意思.
我只想知道.
你到底是什麼意思.
我只想知道.
你到底是什麼意思.
我只想知道.
你到底是什麼意思.
我只想知道.
你到底是什麼意思.
我只想知道.
你到底是什麼意思.
我只想知道.
你到底是什麼意思.
我只想知道.
你到底是什麼意思.
我只想知道.
你到底是什麼意思.
我只想知道.
你到底是什麼意思.
我只想知道.
你到底是什麼意思.
我只想知道.
你到底是什麼意思.

$^{961}$我只想知道.
你到底是什麼意思.
我只想知道.
你到底是什麼意思.
我只想知道.
你到底是什麼意思.
我只想知道.
你到底是什麼意思.
我只想知道.
你到底是什麼意思.
我只想知道.
你到底是什麼意思.
我只想知道.
你到底是什麼意思.
我只想知道.
你到底是什麼意思.
我只想知道.
你到底是什麼意思.
我只想知道.
你到底是什麼意思.
我只想知道.
你到底是什麼意思.
我只想知道.
你到底是什麼意思.
我只想知道.
你到底是什麼意思.
我只想知道.
你到底是什麼意思.
我只想知道.
你到底是什麼意思.
我只想知道.
你到底是什麼意思.
我只想知道.
你到底是什麼意思.
我只想知道.
你到底是什麼意思.
我只想知道.
你到底是什麼意思.
我只想知道.
你到底是什麼意思.

$^{1001}$我只想知道.
你到底是什麼意思.
我只想知道.
你到底是什麼意思.
我只想知道.
你到底是什麼意思.
我只想知道.
你到底是什麼意思.
我只想知道.
你到底是什麼意思.
我只想知道.
你到底是什麼意思.
我只想知道.
你到底是什麼意思.
我只想知道.
你到底是什麼意思.
我只想知道.
你到底是什麼意思.
我只想知道.
你到底是什麼意思.
我只想知道.
你到底是什麼意思.
我只想知道.
你到底是什麼意思.
我只想知道.
你到底是什麼意思.
我只想知道.
你到底是什麼意思.
我只想知道.
你到底是什麼意思.
我只想知道.
你到底是什麼意思.
我只想知道.
你到底是什麼意思.
我只想知道.
你到底是什麼意思.
我只想知道.
你到底是什麼意思.
我只想知道.
你到底是什麼意思.

$^{1041}$我只想知道.
你到底是什麼意思.
我只想知道.
你到底是什麼意思.
我只想知道.
你到底是什麼意思.
我只想知道.
你到底是什麼意思.
我只想知道.
你到底是什麼意思.
我只想知道.
你到底是什麼意思.
我只想知道.
你到底是什麼意思.
我只想知道.
你到底是什麼意思.
我只想知道.
你到底是什麼意思.
我只想知道.
你到底是什麼意思.
我只想知道.
你到底是什麼意思.
我只想知道.
你到底是什麼意思.
我只想知道.
你到底是什麼意思.
我只想知道.
你到底是什麼意思.
我只想知道.
你到底是什麼意思.
我只想知道.
你到底是什麼意思.
我只想知道.
你到底是什麼意思.
我只想知道.
你到底是什麼意思.
我只想知道.
你到底是什麼意思.
我只想知道.
你到底是什麼意思.

$^{1081}$我只想知道.
你到底是什麼意思.
我只想知道.
你到底是什麼意思.
我只想知道.
你到底是什麼意思.
我只想知道.
你到底是什麼意思.
我只想知道.
你到底是什麼意思.
我只想知道.
你到底是什麼意思.
我只想知道.
你到底是什麼意思.
我只想知道.
你到底是什麼意思.
我只想知道.
你到底是什麼意思.
我只想知道.
你到底是什麼意思.
我只想知道.
你到底是什麼意思.
我只想知道.
你到底是什麼意思.
我只想知道.
你到底是什麼意思.
我只想知道.
你到底是什麼意思.
我只想知道.
你到底是什麼意思.
我只想知道.
你到底是什麼意思.
我只想知道.
你到底是什麼意思.
我只想知道.
你到底是什麼意思.
我只想知道.
你到底是什麼意思.
我只想知道.
你到底是什麼意思.

$^{1121}$我只想知道.
你到底是什麼意思.
我只想知道.
你到底是什麼意思.
我只想知道.
你到底是什麼意思.
我只想知道.
你到底是什麼意思.
我只想知道.
你到底是什麼意思.
我只想知道.
你到底是什麼意思.
我只想知道.
你到底是什麼意思.
我只想知道.
你到底是什麼意思.
我只想知道.
你到底是什麼意思.
我只想知道.
你到底是什麼意思.
我只想知道.
你到底是什麼意思.
我只想知道.
你到底是什麼意思.
我只想知道.
你到底是什麼意思.
我只想知道.
你到底是什麼意思.
我只想知道.
你到底是什麼意思.
我只想知道.
你到底是什麼意思.
我只想知道.
你到底是什麼意思.
我只想知道.
你到底是什麼意思.
我只想知道.
你到底是什麼意思.
我只想知道.
你到底是什麼意思.

$^{1161}$我只想知道.
你到底是什麼意思.
我只想知道.
你到底是什麼意思.
我只想知道.
你到底是什麼意思.
我只想知道.
你到底是什麼意思.
我只想知道.
你到底是什麼意思.
我只想知道.
你到底是什麼意思.
我只想知道.
你到底是什麼意思.
我只想知道.
你到底是什麼意思.
我只想知道.
你到底是什麼意思.
我只想知道.
你到底是什麼意思.
我只想知道.
你到底是什麼意思.
我只想知道.
你到底是什麼意思.
我只想知道.
你到底是什麼意思.
我只想知道.
你到底是什麼意思.
我只想知道.
你到底是什麼意思.
我只想知道.
你到底是什麼意思.
我只想知道.
你到底是什麼意思.
我只想知道.
你到底是什麼意思.
我只想知道.
你到底是什麼意思.
我只想知道.
你到底是什麼意思.

$^{1201}$我只想知道.
你到底是什麼意思.
我只想知道.
你到底是什麼意思.
我只想知道.
你到底是什麼意思.
我只想知道.
你到底是什麼意思.
我只想知道.
你到底是什麼意思.
我只想知道.
你到底是什麼意思.
我只想知道.
你到底是什麼意思.
我只想知道.
你到底是什麼意思.
我只想知道.
你到底是什麼意思.
我只想知道.
你到底是什麼意思.
我只想知道.
你到底是什麼意思.
我只想知道.
你到底是什麼意思.
我只想知道.
你到底是什麼意思.
我只想知道.
你到底是什麼意思.
我只想知道.
你到底是什麼意思.
我只想知道.
你到底是什麼意思.
我只想知道.
你到底是什麼意思.
我只想知道.
你到底是什麼意思.
我只想知道.
你到底是什麼意思.
我只想知道.
你到底是什麼意思.

$^{1241}$我只想知道.
你到底是什麼意思.
我只想知道.
你到底是什麼意思.
我只想知道.
你到底是什麼意思.
我只想知道.
你到底是什麼意思.
我只想知道.
你到底是什麼意思.
我只想知道.
你到底是什麼意思.
我只想知道.
你到底是什麼意思.
我只想知道.
你到底是什麼意思.
我只想知道.
你到底是什麼意思.
我只想知道.
你到底是什麼意思.
我只想知道.
你到底是什麼意思.
我只想知道.
你到底是什麼意思.
我只想知道.
你到底是什麼意思.
我只想知道.
你到底是什麼意思.
我只想知道.
你到底是什麼意思.
我只想知道.
你到底是什麼意思.
我只想知道.
你到底是什麼意思.
我只想知道.
你到底是什麼意思.
我只想知道.
你到底是什麼意思.
我只想知道.
你到底是什麼意思.

$^{1281}$我只想知道.
你到底是什麼意思.
我只想知道.
你到底是什麼意思.
我只想知道.
你到底是什麼意思.
我只想知道.
你到底是什麼意思.
我只想知道.
你到底是什麼意思.
我只想知道.
你到底是什麼意思.
我只想知道.
你到底是什麼意思.
我只想知道.
你到底是什麼意思.
我只想知道.
你到底是什麼意思.
我只想知道.
你到底是什麼意思.
我只想知道.
你到底是什麼意思.
我只想知道.
你到底是什麼意思.
我只想知道.
你到底是什麼意思.
我只想知道.
你到底是什麼意思.
我只想知道.
你到底是什麼意思.
我只想知道.
你到底是什麼意思.
我只想知道.
你到底是什麼意思.
我只想知道.
你到底是什麼意思.
我只想知道.
你到底是什麼意思.
我只想知道.
你到底是什麼意思.

$^{1321}$我只想知道.
你到底是什麼意思.
我只想知道.
你到底是什麼意思.
我只想知道.
你到底是什麼意思.
我只想知道.
你到底是什麼意思.
我只想知道.
你到底是什麼意思.
我只想知道.
你到底是什麼意思.
我只想知道.
你到底是什麼意思.
我只想知道.
你到底是什麼意思.
我只想知道.
你到底是什麼意思.
我只想知道.
你到底是什麼意思.
我只想知道.
你到底是什麼意思.
我只想知道.
你到底是什麼意思.
我只想知道.
你到底是什麼意思.
我只想知道.
你到底是什麼意思.
我只想知道.
你到底是什麼意思.
我只想知道.
你到底是什麼意思.
我只想知道.
你到底是什麼意思.
我只想知道.
你到底是什麼意思.
我只想知道.
你到底是什麼意思.
我只想知道.
你到底是什麼意思.

$^{1361}$我只想知道.
你到底是什麼意思.
我只想知道.
你到底是什麼意思.
我只想知道.
你到底是什麼意思.
我只想知道.
你到底是什麼意思.
我只想知道.
你到底是什麼意思.
我只想知道.
你到底是什麼意思.
我只想知道.
你到底是什麼意思.
我只想知道.
你到底是什麼意思.
我只想知道.
你到底是什麼意思.
我只想知道.
你到底是什麼意思.
我只想知道.
你到底是什麼意思.
我只想知道.
你到底是什麼意思.
我只想知道.
你到底是什麼意思.
我只想知道.
你到底是什麼意思.
我只想知道.
你到底是什麼意思.
我只想知道.
你到底是什麼意思.
我只想知道.
你到底是什麼意思.
我只想知道.
你到底是什麼意思.
我只想知道.
你到底是什麼意思.
我只想知道.
你到底是什麼意思.

$^{1401}$我只想知道.
你到底是什麼意思.
我只想知道.
你到底是什麼意思.
我只想知道.
你到底是什麼意思.
我只想知道.
你到底是什麼意思.
我只想知道.
你到底是什麼意思.
我只想知道.
你到底是什麼意思.
我只想知道.
你到底是什麼意思.
我只想知道.
你到底是什麼意思.
我只想知道.
你到底是什麼意思.
我只想知道.
你到底是什麼意思.
我只想知道.
你到底是什麼意思.
我只想知道.
你到底是什麼意思.
我只想知道.
你到底是什麼意思.
我只想知道.
你到底是什麼意思.
我只想知道.
你到底是什麼意思.
我只想知道.
你到底是什麼意思.
我只想知道.
你到底是什麼意思.
我只想知道.
你到底是什麼意思.
我只想知道.
你到底是什麼意思.
我只想知道.
你到底是什麼意思.

$^{1441}$我只想知道.
你到底是什麼意思.
我只想知道.
你到底是什麼意思.
我只想知道.
你到底是什麼意思.
我只想知道.
你到底是什麼意思.
我只想知道.
你到底是什麼意思.
我只想知道.
你到底是什麼意思.
我只想知道.
你到底是什麼意思.
我只想知道.
你到底是什麼意思.
我只想知道.
你到底是什麼意思.
我只想知道.
你到底是什麼意思.
我只想知道.
你到底是什麼意思.
我只想知道.
你到底是什麼意思.
我只想知道.
你到底是什麼意思.
我只想知道.
你到底是什麼意思.
我只想知道.
你到底是什麼意思.
我只想知道.
你到底是什麼意思.
我只想知道.
你到底是什麼意思.
我只想知道.
你到底是什麼意思.
我只想知道.
你到底是什麼意思.
我只想知道.
你到底是什麼意思.

$^{1481}$我只想知道.
你到底是什麼意思.
我只想知道.
你到底是什麼意思.
我只想知道.
你到底是什麼意思.
我只想知道.
你到底是什麼意思.
我只想知道.
你到底是什麼意思.
我只想知道.
你到底是什麼意思.
我只想知道.
你到底是什麼意思.
我只想知道.
你到底是什麼意思.
我只想知道.
你到底是什麼意思.
我只想知道.
你到底是什麼意思.
我只想知道.
你到底是什麼意思.
我只想知道.
你到底是什麼意思.
我只想知道.
你到底是什麼意思.
我只想知道.
你到底是什麼意思.
我只想知道.
你到底是什麼意思.
我只想知道.
你到底是什麼意思.
我只想知道.
你到底是什麼意思.
我只想知道.
你到底是什麼意思.
我只想知道.
你到底是什麼意思.
我只想知道.
你到底是什麼意思.

$^{1521}$我只想知道.
你到底是什麼意思.
我只想知道.
你到底是什麼意思.
我只想知道.
你到底是什麼意思.
我只想知道.
你到底是什麼意思.
我只想知道.
你到底是什麼意思.
我只想知道.
你到底是什麼意思.
我只想知道.
你到底是什麼意思.
我只想知道.
你到底是什麼意思.
我只想知道.
你到底是什麼意思.
我只想知道.
你到底是什麼意思.
我只想知道.
你到底是什麼意思.
我只想知道.
你到底是什麼意思.
我只想知道.
你到底是什麼意思.
我只想知道.
你到底是什麼意思.
我只想知道.
你到底是什麼意思.
我只想知道.
你到底是什麼意思.
我只想知道.
你到底是什麼意思.
我只想知道.
你到底是什麼意思.
我只想知道.
你到底是什麼意思.
我只想知道.
你到底是什麼意思.

$^{1561}$我只想知道.
你到底是什麼意思.
我只想知道.
你到底是什麼意思.
我只想知道.
你到底是什麼意思.
我只想知道.
你到底是什麼意思.
我只想知道.
你到底是什麼意思.
我只想知道.
你到底是什麼意思.
我只想知道.
你到底是什麼意思.
我只想知道.
你到底是什麼意思.
我只想知道.
你到底是什麼意思.
我只想知道.
你到底是什麼意思.
我只想知道.
你到底是什麼意思.
我只想知道.
你到底是什麼意思.
我只想知道.
你到底是什麼意思.
我只想知道.
你到底是什麼意思.
我只想知道.
你到底是什麼意思.
我只想知道.
你到底是什麼意思.
我只想知道.
你到底是什麼意思.
我只想知道.
你到底是什麼意思.
我只想知道.
你到底是什麼意思.
我只想知道.
你到底是什麼意思.

$^{1601}$我只想知道.
你到底是什麼意思.
我只想知道.
你到底是什麼意思.
我只想知道.
你到底是什麼意思.
我只想知道.
你到底是什麼意思.
我只想知道.
你到底是什麼意思.
我只想知道.
你到底是什麼意思.
我只想知道.
你到底是什麼意思.
我只想知道.
你到底是什麼意思.
我只想知道.
你到底是什麼意思.
我只想知道.
你到底是什麼意思.
我只想知道.
你到底是什麼意思.
我只想知道.
你到底是什麼意思.
我只想知道.
你到底是什麼意思.
我只想知道.
你到底是什麼意思.
我只想知道.
你到底是什麼意思.
我只想知道.
你到底是什麼意思.
我只想知道.
你到底是什麼意思.
我只想知道.
你到底是什麼意思.
我只想知道.
你到底是什麼意思.
我只想知道.
你到底是什麼意思.

$^{1641}$我只想知道.
你到底是什麼意思.
我只想知道.
你到底是什麼意思.
我只想知道.
你到底是什麼意思.
我只想知道.
你到底是什麼意思.
我只想知道.
你到底是什麼意思.
我只想知道.
你到底是什麼意思.
我只想知道.
你到底是什麼意思.
我只想知道.
你到底是什麼意思.
我只想知道.
你到底是什麼意思.
我只想知道.
你到底是什麼意思.
我只想知道.
你到底是什麼意思.
我只想知道.
你到底是什麼意思.
我只想知道.
你到底是什麼意思.
我只想知道.
你到底是什麼意思.
我只想知道.
你到底是什麼意思.
我只想知道.
你到底是什麼意思.
我只想知道.
你到底是什麼意思.
我只想知道.
你到底是什麼意思.
我只想知道.
你到底是什麼意思.
我只想知道.
你到底是什麼意思.

$^{1681}$我只想知道.
你到底是什麼意思.
一個無形教會.
其實是奧古斯丁發明的字.
在四世是形而上的.
這是Plato的字學.
一個最完美的燈.
是在天上的一個理型的燈.
如果讀過歷史的人.
真實的燈總是會爛.
還有很多不同的形狀.
總是會消逝.
總是會爛壞.
真正最完美的東西.
是在天上.
所以姜濤也不夠最帥.
是天上的一個姜濤的想法.
才是最帥的.
意思就是.
任何在地上的東西都不是完美的.
所以奧古斯丁就用了這些概念.
來解釋.
教會其實是地上的教會.
是髒髒的.
他這麼說是有背景的.
因為當時有些人叛教.
有些人以前被迫離開教會.
終於教會就不好.
奧古斯丁就嘗試去平衡這兩件事.
我們看著教會.
他們不行.
他們曾經不信耶穌.
他們曾經甚至離棄信仰.
他們很軟弱.
他們甚至說到很悶.
經常遲到.
總之.
奧古斯丁為瞭解釋地上教會的問題.
發明瞭一個字.
真正的教會.

$^{1721}$不要失望.
因為真正的教會是一個無法見證的教會.
上帝所預定的一群人.
是一個見不到的群體.
但他們是真正的教會.
地上的教會.
其實是什麼.
他們叫物子和敗子.
我經常忘記怎麼讀.
敗子.
所以將來審判的時候.
那些才是真正的基督徒.
這就是此類的看法.
所以奧古斯丁嘗試去解釋.
為什麼地上的教會那麼髒髒的.
其實不要緊.
是髒髒的.
因為我們說真正的教會.
真正上帝屬靈神聖的基督的身體.
其實是形而上的.
無法見證的.
這個說法好還是不好.
當然他能夠解釋.
能夠用無法見證的教會去解釋.
我們這群有形群體.
其實都是軟弱的.
都是有限的.
也不一定是真的.
最真最完美的.
其實在無法見證里.
所以大家是沒有對話的.
大家OK的.
因為教會是這樣的.
這個就是奧古斯丁的解法.
但其實當我們這樣做的時候.
其實是做什麼.
其實是不斷避免.
去解釋為什麼地上的教會髒髒的.
你只不過不斷推薦.
地上的教會其實不是真的.

$^{1761}$不是最真實的.
最真實的不是最真實的.
只不過是堂會.
不是教會.
很多這些說法.
但這個不是最好的方法.
去解釋現在的問題.
當然如果看回歷史裡面.
東正教天主教.
其實都嘗試去強調這個無法見證教會.
東正教會他們會嘗試.
當無法見證教會執行聖餐的時候.
聖餐就令到一間髒髒的教會.
重新成為一間教會.
這些不過好心不說了.
總之我們嘗試不斷地解釋用不同的神理論.
去將地上有形的髒髒的教會.
去合法化.
我們不要只是將完美放在天上.
這個是卡爾巴克稱之為教會的幻影論.
不知道你是不是這樣.
很多香港基督徒.
可能在網上對教會的批評者.
可能都是這樣.
他拿著教會一個完美的點法.
教會應該有愛心的程度.
教會的人應該彼此相愛.
教會應該不貪錢.
教會應該很關心我們每個人的需要.
教會應該民主.
很多這些不同的看法.
其實他們只是拿著理想的教會的圓形.
去對地上的問題比較.
這樣永遠會找到很多問題.
你拿著完美的教會點法.
去和見到的教會比較.
總會找到很多不同的問題.
這個我稱之為教會的幻影論.
幻影論這個字本身來自於耶穌的詮釋.
耶穌是一個幻影.

$^{1801}$認為上帝不會受到任何的問題.
所以他是一個幻影.
沒有特意來到地上.
這是一個異端的說法.
如果把這個看法去到教會的問題.
同樣也是.
教會是一個嘗試將地上的教會去否定它.
真正的教會是天上的.
最終的方法也是否定地上的教會.
從而去承認天上的教會才是唯一的教會.
這樣是不妥當的.
因為真正的教會就是我們現在看到的教會.
就是一群人.
怎麼可能去否定地上的教會呢.
所以今天我們嘗試去想.
究竟怎麼能夠解決這個問題.
可能這個就是大家以前的問題.
回到自己以前的誤會的時候.
很多的甩樓.
很多的問題.
很多的不理解.
很多令你覺得很激氣的地方.
我怎麼來理解地上的教會.
甚至Full Church也是.
Full Church也是一間地上的教會.
Full Church也不是一間完美的教會.
所以我怎麼能夠去平衡這件事呢.
這是我自己嘗試在這幾年里.
重新建構教會論.
我們稱之為教會的軟弱的本相.
我們嘗試不要將一些完美的東西.
作為一個default.
反過來我們嘗試將教會的軟弱.
成為我們的default.
我舉個例子.
問大家一個問題.
你也知道鋼琴裡面.
黑鍵和白鍵.
你覺得黑鍵還是白鍵是default的.
就是default的位置.

$^{1841}$你覺得黑鍵是降低了key的白鍵.
還是白鍵是升了key的黑鍵.
誰才是default的.
誰才是鋼琴原本default的按鈕.
後來按鈕是加進去的.
你明白我的意思嗎.
黑鍵還是白鍵.
你可以說是白鍵.
白鍵是default的.
後來人們發覺不夠音.
要sharp一點要fat一點.
就加上黑鍵.
其實是相對的.
黑鍵其實是default的.
白鍵只不過是黑鍵的sharp.
所以我們就用類似的思維.
就是說我們.
(問問觀眾).
所以我們嘗試用黑鍵來做default的狀態.
不是白鍵.
教會的default狀態.
其實不是白色而是黑色.
教會應該是凹凸不平的.
你不要把default看為教會是完美的.
黑鍵那些不好的.
是一些缺陷或者是意外.
我們嘗試反過來去想.
這個正正就是卡爾巴特在《羅馬書》裡面.
一個很好的一句話.
讀給大家聽.
今天看不到.
他說教會是雅各的教會.
只能發生在神跡裡面.
否則教會從來都是以素的教會.
卡爾巴特用了兩個概念來說教會.
一個叫以素的教會.
一個叫雅各的教會.
用教會語來比喻雅各和以素.
這裡麻煩大家.
這個在《羅馬書》時期.

$^{1881}$1922年裡面.
卡爾巴特嘗試去講教會論.
《羅馬書》是一本非常dialectic的書.
用一些很浪漫主義的手法來寫神學.
以素和雅各的教會.
就是說誰是奸誰不是奸.
很明顯吧.
以素是奸的.
以素是差一點.
雅各是好一點.
我們先當是這樣吧.
他說雅各的教會和以素的教會.
他說教會同時是雅各的教會.
同時是以素的教會.
他說教會是雅各的教會.
只能在神跡裡面.
否則教會從來都是以素的教會.
我們將黑厭看作是default.
教會就是一群人.
教會就是一群被呼召成為基督徒的人.
這個群體的行動就是教會.
這個人就是人.
就是一群人.
不過我們沒有否定過.
這個教會可以彼此相愛.
可以很溫暖.
可以做一些很正義的事情.
可以在正宗裡面.
轉化為法律.
不過這些事情.
是在神跡裡面的時候.
這些事情是一個.
唯有靠著恩典才能夠發生的事情.
如果沒有恩典的時候.
教會從來都是以素的教會.
卡邦尼的說話其實是重新把黑厭和白厭調轉.
不是說教會應該是很完美的.
為什麼突然間那個人不愛我.
為什麼那麼壞.
不要這麼想 反過來.

$^{1921}$這群人本來就很壞.
但竟然我們這群脾脾肋骨的人.
都能夠聚在一起.
做一些有意義的事情.
能夠在一起和諧共處.
都能夠有愛.
其實是神跡.
所以這是一個很大的教會觀.
我們沒有放棄過地上的教會是教會.
我們不會說這間是堂會.
我們不會否定這個教會只是一個.
visible church.
不是教會.
它仍然是真正的教會.
它仍然是上帝的教會.
不過它的好東西.
只能夠在禱告和恩典里.
竟然可以那麼好.
我們今天這群人聚在一起.
或者網上的人聚在一起.
我們都那麼有心來學神學.
是挺好的 大家都挺好的.
但這是神跡.
我經常說我在長洲裡面教神學.
是一個很棒的經驗.
我們山上有百多個神學山住在一起.
百多個好人.
全都人都不是好人.
有些是壞的.
起碼仇山是OK的.
仇山還是比較幼稚.
百多個好人住在一起.
是一件很難精靈的事.
所以是可以發生的.
我們說彼此相愛 法莫共用.
我們能夠一起有使命 有意象.
去感動 是可以的.
但這不是default.
這是因點.
所以我們嘗試這樣來理解教會是什麼.

$^{1961}$教會的本體就是從我們的軟弱開始.
之前的頁我先show剛才那一頁.
之前的頁.
所以我說教會的軟弱的本相.
我們用church的weakness來作為我們整個本體的開始點.
我們承認我們教會本身是軟弱的.
這個不需要驚訝.
這是我們會有的東西.
我們能夠大家不同意見都可以在一起.
這個反而是因點.
去到後面 去剛才那一頁.
剛才我們有個總結.
去到後面那一頁.
OK 下一頁.
所以我說教會的軟弱不是異數.
而是作為教會本體.
尤其當我們不嘗試為教會的有形無形作一個不必要的區分的時候.
教會的軟弱是一個非常重要的課題.
我們不再嘗試用有形無形來區分.
好的東西歸給無形.
不好的東西歸給有形.
而是我們見到真實地這間教會.
這個有形群體它就是教會.
只不過它們的好東西仍然是有的.
不過在因點裡面.
所以這個就是我們來思考教會的很重要的課題.
好 下一個.
所以 下一張 麻煩你.
所以教會的那個.
下一個就是教會的基督的離去和同在.
同樣都是.
基督的升天和基督的再來中間的一段時間.
這個正正就是教會時期.
當基督耶穌升天都住在了反向的時間.
這段居間期正正就是教會地上存在的時間.
這段時間是很特別的.
因為我經常都說過.
很多年前我教授都說過這篇道.
就是基督的離開.
我們經常都強調基督的同在.

$^{2001}$但其實基督同時的升天.
很重要的意義就是基督耶穌不在.
不是真實的在.
起碼就是.
所以我們要知道我們很多仍然是軟弱的.
我們只能夠靠住.
你看到耶穌不是一下子下來.
突然間幫你很多問題.
然後耶穌就說了.
我們只能夠在同在和離開的那種緊張裡面.
約翰福音14章裡面.
我們經常都在裡面.
祂在我們裡面.
但同時也是一個離別之言.
同時耶穌也是正正是祂的門徒.
耶穌將要離開.
所以耶穌的升天標誌著什麼.
就是祂的離開.
但祂的離開不是完全沒有做事.
是什麼呢.
祂會將祂在天上為我們代求.
所以我們只能夠在這樣的狀態裡面.
我們是基督的身體.
我們是基督的身父.
上帝的子民.
這些詞很威風.
不過我們確實也不是那麼行.
門徒.
一旦耶穌離開就軟到爆.
所以在這樣的狀態裡面.
我們只能夠得到一些很真實.
但又不是完全讓你理解到的東西.
基督耶穌會在天上為我們代求.
耶穌是在我們身邊.
不過是祂的靈在我們身邊.
所以在這樣的曖昧.
有些行有些不行.
這樣的狀態正正就是我們教會的狀態.
所以怎麼辦呢.
下一章請.

$^{2041}$怎麼辦呢.
所以說更新是一個必然的途徑.
既然我們說教會的軟弱.
正正就是我們的本體的時候.
更新就絕對不是一個純屬的偶發的事情.
教會的改變.
教會的更新.
成為教會的本體很重要的一環.
它不是偶發的.
它必定會間中轉變一遍.
所以我都說沒有東西是完美的.
唯有不斷更新和改變才是完美的.
所以教會也是一樣.
沒有一個完美的教會.
Full Church也不是.
但如果Full Church能夠不斷地更新.
不斷地改變.
不斷地按照我們的時代需要去改變.
我經常說.
今天我推翻昨天的我這件事不是不好的.
是很重要的.
你需要去打倒昨天的我.
因為你重新去聆聽上帝的聲音.
重新去明白我們應該怎麼做.
需要不斷地去改變和更新.
所以這是我們很重要的.
教會的更新是我們很重要的一環.
今天不說了.
可以再談談這個問題.
所以教會是不斷地需要更新和改變.
所以Full Church就這樣開始.
在三年前的時間里.
Full Church是一間教會.
是一間有形的教會.
是有運作的教會.
不過我們當然嘗試有一個新的方法運作.
一個所謂的Simple Church.
有時候也說重新推翻所有外在的東西.
抓緊那個essence.
針對那個essence來回應這個時代.

$^{2081}$所以我們仍然是做教會應該做的事.
不過不是用傳統的方法.
因為這些方法是可以變的.
但essence是我們不會變的.
所以Full Church的出現.
就是仍然是做一間.
知道是有軟弱的教會.
我們不會寄望這間教會沒有軟弱.
但是都很警惕軟弱的問題.
怎麼能夠在運作上.
重新減少軟弱的問題之類的.
這些會是上Info Group也聽過的.
所以Full Church的概念.
正正是從今天所講的教會觀來做開始.
下一張請.
我想多說一些教會的使命.
Full Church的mission.
所以教會Full Church在這個年代.
正正是做教會應該做的事.
從凝聚了一群人之後.
我們如何能夠見證基督呢.
這個群體如何能夠發生呢.
如何能夠藉著這群人做一些事來發生呢.
因為當我們做事的時候.
才出現教會這件事.
所以Flow字就是這個意思.
當我們這群人在行動的時候.
Flow Church才突然出現了.
當大家散會的時候就沒有了這教會.
我這樣理解的時候是會.
當我們聚集在一起.
當然敬拜就不可能了.
我們Full Church只要是在做事的時間.
包括見證基督耶穌.
在香港社會裡面.
我們是一個敬拜群體.
當我們彼此相愛的時候.
Full Church就形成了.
然後當我們小組結束了就散了.
Full Church就不出現了.

$^{2121}$我平時是這樣理解Full Church這件事.
所以教會是一聚一聚就在散.
這個的發生是很重要.
而我們今天Full Church正是想做這件事.
在香港裡面來秉持我們應有的使命.
就是去大堂上說話.
做一個見證基督的群體 基督徒.
這個是我們在香港裡面很重要的課題.
下一張.
所以回到實際問題.
究竟是否需要教會呢.
因為其實這個問題.
這一年裡面很多人都問這個問題.
特別是如果你是二十多歲的話.
你覺得教會.
我們這個無堂會.
很明顯Full Church不是一間無堂會的教會.
為什麼我不會做一間無堂會的教會呢.
因為無堂會是一個很好的理想.
但其實我們說Full Church.
教會本身作為地上的群體.
如果我們不否定它的話.
Full Church本身是一間教會.
它無可避免有這些東西.
它肯定是有運作上的東西.
有一些人的計劃的東西.
當然它可以是一個.
我們去短孫吧 不如我們走.
這個早點.
但其實我們如果想長遠運作的時候.
有堂會 其實這不是一件壞事.
問題是如何能夠有堂會.
能夠做到的事.
我們不想做了一間.
兩年之後就沒有了.
能否維持下去.
將使命 將教導.
所以每一方是一個很理想的想法.
但當我們說教會是地上的教會的時候.
我們就有一些很實際的東西要去想.

$^{2161}$所以這幾年裡面.
我從一個神學人.
慢慢變成一個教會人.
其實就是將一些神學的東西.
變成一些很實際的東西.
如何能夠顧及一些實際的東西.
這個是重要的.
所以我們不是覺得教會是不需要運作.
教會仍然是需要這些東西.
但我們想要減化掉.
或者變成一些有意義的東西.
而不是完全否定一些實際上的東西.
所以簡單來說.
對於無堂會法的問題.
我覺得這個是可以試試的.
但是我仍然覺得是有需要.
是有一定程度的運作的東西.
這個才能夠真正地.
解決一些實際的問題.
下一個 最後兩個問題.
所以這個時勢還需要教會嗎.
不知道大家的答案是什麼.
這個時勢其實是需要教會的.
應該有的 下一個 謝謝.
我給答案出來 是需要的.
為什麼需要.
因為其實教會的目的.
正正就是去私行宣揚盼望.
耶穌基督的福音的盼望.
所以這時候更加需要教會.
更加需要一個能夠存在香港的教會.
當然很多人都問.
究竟將來怎麼辦.
將來如果打倒賣尼的時候怎麼辦.
不過起碼我都說.
能夠存在就繼續存在.
能夠聚在一起就聚在一起.
只不過我們知道.
我們不是為了聚在一起而聚在一起.
帶著使命 能夠改變.

$^{2201}$帶著希望去做這件事.
其實現在很多人是需要耶穌的.
我們科學學院裡面.
也有發現有些不信耶穌的人.
來到教會裡面.
他們真的很需要耶穌基督的盼望.
來支持自己.
怎麼面對這個世界.
所以這個時勢更加需要教會.
更加需要我們這群.
更加明白什麼是教會的人.
第二條.
教會有什麼用呢.
有沒有用這個問題.
是一個很特別的問題.
有沒有用.
你覺得自己有沒有用.
我不知道今天有沒有用.
但我都覺得有沒有用這個問題.
其實這個問題.
都是問得很特別.
我忘了怎麼回答.
有沒有用.
應該說有用.
有用的.
因為我們是全是耶穌基督.
多說一句.
這個時候更加有需要.
當然我們不是問教會.
沒有需要 沒有用.
但我都覺得這些東西.
正正就是這樣.
你發覺很多東西你都做不到的時候.
我們說的這些耶穌的東西.
就變得更加特別有用.
以前你覺得都會法律援助.
或者出版一些什麼.
都是實際的東西.
我們做傳道人這行是最沒用的.
基本上.

$^{2241}$說的都是耶穌.
說的都是天國的東西.
但在這個年代裡面.
當你發覺做的那些東西.
都是沒什麼用的時候.
我們這些沒什麼用的東西.
都挺有用的.
我發覺都是會.
因為就只能夠說一些更加軟的東西.
就只能夠說一些.
就是生命的東西.
不是說一些那麼實際的東西.
但正正就是我們這些基督徒.
就是說這些東西.
宣揚一個比較軟的眼光.
宣揚一些看不到的東西.
這個有形的群體.
去說一些無形的東西.
正正就是我們教會.
香港教會.
對於香港這個年代裡面.
很重要的一個用處.
正正就是說一些沒用的東西.
好像沒用的東西.
但這些沒用的東西.
在什麼都沒用的時間裡面.
是最有用的.
所以大概就說到這裡.
我口渴了.
請給我一杯奶茶.
我經常看到你.
很多時候都說耶穌.
說教會.
其實這裡附近有很多教會.
我想問.
其實和我開茶餐廳.
或者開茶几.
有沒有分別.
深的問題.
就是什麼.

$^{2281}$就是有沒有分別.
我自己覺得.
開個檔.
剛才聽你.
其實不同人去不同教會.
其實不同人去不同茶几.
都差不多.
是不是.
當然是.
肯定是沒有一間茶几是獨市.
但有沒有一些.
每個人都要.
每一間開茶的人.
都覺得自己的奶茶最好喝.
對嗎.
我這間才是最正宗.
最好喝的那間.
這個自信我是有的.
OK.
但有沒有一些.
你每次來都是說不叫東西的.
有些人回教會.
就是只聽不做事.
是嗎.
會不會.
不如問一下他們.
對.
你們都回過很多教會.
剛才都說了.
你們過去可能回教會.
都有不同的經歷.
跟你今天聽他說完之後.
關於教會的感覺.
或者理解的.
其實是什麼.
大家可以說一下.
這間是第一間教會.
是不是初信.
或者選教會.
其實會選什麼.

$^{2321}$有沒有什麼指引.
我們一定會看到一個有型的教會.
很有型.
你這間教會是不是很有型.
要說一下.
要說一下這間教會是怎樣.
是嗎.
大家選教會.
或者是否像選茶餐廳.
試過覺得可以.
不是的.
是,後面有一個人問.
還有這個成分.
就是會去那間教會瞭解一下.
然後去知道多一點.
但是其實.
剛才和John說的.
有些地方都是相符的.
就是說.
一間教會是否真的做實事.
這件事是要有的.
因為很多坊間的教會.
都是會比較.
他們會集中在概念上.
但是實際的行動.
或者是可能.
例如落地.
你去做一些事情也好.
其實這件事是.
有些教會可能是欠縫.
我想有些討論一下.
什麼叫概念或者實事.
因為我們剛才說的.
就好像.
選茶餐廳.
他就說奶茶到位正.
就吸引到人來.
那會有什麼叫做.
是實事可以吸引到一群人來這間教會呢.
關心社會事實.

$^{2361}$其他人呢.
就是選教會有什麼.
考慮因素.
或者對你來說.
剛才聽完說.
教會其實是一件什麼事呢.
後面.
剛才在想John你提到的問題.
我們回教會究竟是找一個.
是同一群人來回教會.
還是回一個建築的教會.
我自己也有看過不同的教會.
也回過不少國家的教會.
也看過.
其實我自己也會想.
如果回到剛才John的問題.
究竟我去選一間教會.
或者我去找那間教會的時候.
究竟我是去找那間教會在做的事情多一些.
還是我找那間教會.
可能我這間大一點.
或者這間剛才是Hillsong之類的.
究竟我是應該如何去選擇教會.
第二個問題就是.
有些人會問.
如果教會是純粹是一群人.
而不是一個建築的時候.
如果我一大群人聚在一起的時候.
是不是代表我可以.
就這樣變成一個教會.
不需要回一間實體的教會.
我想我為什麼會想到handicap呢.
因為今天所說的話.
其實和選擇教會是有些.
當然是很實際的.
選擇教會是一個很實際的問題.
但是發覺剛才說了大量的東西.
其實那個縱深.
反而是大家能夠明白教會是什麼呢.
其實很多事情你應該是會做的.

$^{2401}$懂得如何去行動.
如果你知道原來教會是這樣的.
教會的重點是在於那個event.
那個行動.
其實那些東西.
你可以說是幫你選擇教會.
其實我覺得問題.
我就不會說今天付出是如何選擇教會.
但是問題是.
發覺因為問題有點.
真的不像茶几.
因為茶几是消費的.
因為餐廳是在消費.
你給錢買杯奶茶就要好喝.
所以這個問題.
我始終都不懂得這樣放在一起去談.
反而就是說.
如果我們是.
想起第一科.
我們是一個基督徒.
你被呼召.
基督徒來到去要做一些事的時候.
有點像你現在打手游.
你多些加入哪個工會.
你去做一些事.
我加入哪個工會去做哪些事.
我想選擇哪個工會來到去實踐使命.
多過我來到去哪裡好些.
所以是能夠可以問.
我想問哪個教會.
我想和哪些人一起來實踐使命.
我覺得這個會是.
我想金塘跟我的思路回答多些.
所以正正這個都是回到最基本.
今天確實是這個年代裡面.
你說上年那條法例或者疫情之類.
香港教會亂七彩.
現在重新再回頭問的時候.
其實應該也問這個問題.
我們應該是去.

$^{2441}$想哪個群體一起去行.
這個是重要的.
其他群體不代表行不通.
一起而已.
沒有說多少人一起.
沒有說什麼style一起.
所以我覺得是這樣去想.
你能夠明白這個問題之後.
你自然而然就會選擇.
我應該.
我選擇了群體.
選擇之後不是終極的重點.
你選擇之後要做些什麼.
所以.
不過我也覺得Full Trust是值得行的.
因為是大家好.
跟教會好不好和Facebook好不好一樣.
發現你的Facebook很悶.
為什麼.
因為你的朋友很悶.
就是這個原因.
所以跟教會好不好.
其實都是關乎教會的人.
他們是否有心實踐使命.
自然而然這群人就聚在一起.
這個也是我覺得.
為什麼Full Trust這兩年.
做的事有些什麼做對了.
其實都不是關乎這件事.
而是大家聚在一起.
正正就像我們說的.
那些水散了.
回到一起.
大家一樣.
大家都在實踐一樣東西.
這個正正就是我們Full Trust最美的地方.
就是大家.
大家能夠好的話.
這間教會不就好了.
其他人呢.

$^{2481}$以實際例子去回應John.
剛才說的一群人聚集.
就一起做一些他們覺得值得的事.
其實Full Trust在不同派對的參與.
其實都很多元.
有些弟兄姐妹群體.
一起聚集的時候.
會想一些平時覺得很習以為常的經文.
但當轉個向.
成為一個大家再深思.
就變成弟兄你錯了的一條片子的製作.
另外就是有一群弟兄姐妹.
很關心社區的論社.
他們就聚集.
就是服務紅土區的.
有低收入家庭.
有SEN小朋友的家庭.
就在這裡做親子媽媽的社區服務.
那個都是來自不同小組.
甚至是播放的弟兄姐妹.
就一起聚集.
每次都是召集活動去服侍.
另外有些弟兄姐妹就想.
不是用影像.
是用聲音.
所以他們每個星期有兩天回來錄Podcast.
去做不同的製作節目.
就是想用聲音去做宣揚.
用聲音做見證.
這個就是回應John說的.
Flow Church有不同群體的結聚.
就在做一個活動.
而那個活動是教會的行動.
也是我們的表徵.
大家有沒有其他關於.
過去教會的參與經歷.
或者是.
是的 後面.
我可能說得有點題外話.
不過我也分享一下.

$^{2521}$其實我覺得.
剛才John說到的.
怎麼說呢.
剛才我也很欣賞你們.
Flow Church的事工.
我也有留意過.
我覺得在堂會聚在一起.
其實就是給大家.
有個.
對我來說.
教會有個空間給你去.
怎麼說呢.
是沒什麼逃脫的.
只是說可以做到.
你平時的生活限制了你.
社會限制的價值.
那套主流價值.
限制了你去做的一些.
可以做的事.
甚至是對教會的定型.
去限制了.
例如說.
沒人想過你在.
你的主流堂會有人.
會有人拍片去.
想重演模擬當時聖經的場景.
那.
那就是.
而我在.
我曾經見過另一間.
就是關於John說的.
那個無堂會化的事.
雖然他的確是理想.
但其實在現在的情況.
我覺得不是不可行的.
整件事.
因為.
因為其實你們有沒有聽過.
新罪教會的YouTube頻道.
這個Riordan Church.

$^{2561}$其實它就是一間.
無堂會的教會.
為什麼這樣呢.
其實它只是一個YouTube頻道.
而它只是有.
兩至三位.
其實是不滿意.
一些堂會體制的人.
那些神職人員走出來.
就開了.
但是.
雖然.
他做的時候是.
去拍一些片.
去分享他們對一些.
基督信仰的新的反思.
理論上.
但其實.
他私底下也有一個.
TG group.
給我們.
給人去裡面討論.
那.
我當然有參加那個group.
其實是.
就是說去裡面.
真的可以和某些.
幾位弟兄姊妹.
真的有心交.
每晚都可以有些.
拋一些真心的.
拿很多.
以前在堂會沒有怎麼探討的信仰.
拿出來討論.
大家都是有得著.
有造就.
生命是有成長.
其實這些都算我們做了一切.
我想這個世代.
這個時代底下.

$^{2601}$其實.
其實真的.
用flow check.
flow的概念就是.
其實水.
不是一池的死水.
是可以.
是好像流動的水.
我們在香港這個地方.
是屬於海洋的人.
是屬於海洋的.
既然你這個地方.
沒辦法令.
實現你想的信仰.
你怎麼看的信仰.
其實你就可以.
其實你就不需要.
只留在原本的comfort zone.
你可以走出去.
用新的形式.
或者是去.
新的.
新的形式.
去新的地方去發展.
就是發展.
其實.
總之而言.
就是說.
我覺得不需要定一套很明確的.
就是標準去說.
教育監教會.
怎麼怎麼你才對.
總之.
這個是很personal的.
總之這間教會就是.
對你來說.
是真的對你的信仰.
是有成長.
那你也有那幫弟妹.
是可以form一個很近.

$^{2641}$你既然一次.
還可以留下最近一起去聯繫.
有一個很bonding.
那這間我相信.
是適合你可以去留下的地方.
那.
所以.
就是這樣.
多謝你分享.
其他呢.
所以我不反對這些教會.
我都說.
我都年紀不小.
所以我都覺得.
有些東西我想不到.
所以.
現在很多二十多歲的人.
想起我的東西.
我自己覺得.
我自己不會做的.
但其實有很多人是做.
其實是很重要的.
我都說不斷更新.
所以我.
我十幾歲的時候.
我都會去.
所以我.
我十幾年之後.
我都不是.
年紀都不小.
所以就慢慢fade out.
大家有沒有想過.
十幾年之後.
這幅畫怎麼辦.
如果還有的話.
所以其實很多時候.
反而重要的就是這個moment.
十幾年.
在這幾年裡面.
能夠做到的東西.

$^{2681}$就做這些東西.
有些新的東西.
新到我都覺得.
get不到的.
不重要的.
正正我現在的東西.
有些人都get不到的.
但都支持的.
有些前輩都會.
所以.
只要能夠work就行了.
就是work不work呢.
其實是能見到的.
能不能夠.
有一班人能夠支持到.
剛才說.
有TG調教心覺.
就work了.
就是能夠有些.
跟到耶穌的.
知識上的.
什麼都做到.
就work了.
所以.
當然.
能夠work多久.
是個問題.
或者work不work.
或者怎麼work.
是個問題.
但總之.
你試試.
試到work.
就真的work.
這個work不work.
就是很明顯.
有得prove的.
所以我覺得.
不反對的.
很多新類型.

$^{2721}$我沒想過的教會.
都可以出現.
我小小回應.
其實.
什麼是work.
因為.
我不是咬文嚼字.
因為.
應用神學.
或者實踐神學.
裡面有一個東西.
要去面對.
就是.
work out.
就是那樣東西.
做出來的時候.
John剛才說得對.
是要發生一些東西.
發生的時候.
你要看果效.
有兩個字.
常常掛在我自己的唇邊.
就是output和outcome.
是什麼.
output就是一些.
facts.
data.
statistic.
是可以量化.
可以評估的.
可能teach you group有多少人.
有多少人看我們現在live.
有多少人在這裡聚會.
這是一些.
實際上的東西.
但如果追數字.
如果還記得.
在flowchurch做登記崇拜的時候.
其實我們四分十幾秒就爆了崇拜.
其實可以直到永遠.

$^{2761}$但我們不是停在數字追求.
就是我們想.
其實有多少新朋友.
想來flowchurch.
是沒有機會來.
這是我們要正視的.
所以我們要upset原本的報名方法.
用第二個方式.
去令更多人可以帶新朋友來.
因為我們不是追output的數字.
我們是要處理outcome.
我們要change.
outcome是change.
要改變.
我們的信仰是.
更新而變化.
有很多人在teach group討論.
或者很多人在群體討論.
是好的.
是可行的.
但他要轉化什麼.
我想回應John最後的問題.
是在傳揚基督嗎.
教會的使命是在傳揚基督.
還是他圍內在討論教會呢.
這就不是實踐神學.
或者是想要帶到的信息.
是要轉化我們的生命.
以至我們可以見證基督教會存在的價值.
就是讓事情發生.
這是重要的.
你好.
我不知道接下來的主題會不會有些離題.
剛才提及了很多關於一些概念.
whatever.
想知道在牧養方面.
又有什麼概念可以分享.
因為都明白教會是希望一班人.
可以一起去做一些事情去宣揚福音.
所以大家都有很多不同的ideas.

$^{2801}$轉化一些action.
去社會的群體去做.
至於弟兄姊妹之間的牧養.
怎麼可以保持一定程度的屬靈狀態.
以至可以令福音的廣傳更加闊.
謝謝.
牧養嘛.
你先說主題.
我先說.
在infogroup里會帶來一個很重要的信息.
就是教會不是一個屬靈的家.
一個面向.
教會同樣是一個teaching institute.
一個教育學院.
或者當school.
不同階段有不同進程.
有不同進程當中有教學的內容或提升.
正正就是希望在不同階段的時候.
讓他們有轉變.
和知道自己的成長方向是什麼.
這個就是統稱叫做牧養.
在牧養過程當中.
每個小組里都會有一個牧者.
去埋身和瞭解全組的需要.
或者個別組員的需要.
就去里定.
或者去瞭解他實際上過去有什麼需要處理.
或者是開發他有興趣的地方.
過去處理的東西就包括一些舊有的堂會的經歷.
或者他對人事物當中可能有些難處.
就希望能夠梳理.
反而正面的就是開發他可以認識的渠道.
用學校的概念就是.
Flow Church是一所學校.
一所學校有不同的Club.
不同的Club就是他有興趣就去加入.
舉個例子.
有些弟兄姊妹都知道我們有Coffee Corner.
Coffee Corner就是我們將來會發展的Club.
有些弟兄姊妹正在受訓做一個Barista.

$^{2841}$可以將來做一些在這裡服務弟兄姊妹的工作.
有些就像現在在拍攝的那兩位姐妹.
都是一個Production Team.
他們都是小組組員.
他們參加Production Team的Club.
有些是Podcast 有些是拍片.
有些是做現場的收音.
其實都是不同弟兄姊妹的興趣.
或者他有興趣想發展成為他可以侍奉的空間.
所以牧羊在Flow Church的概念.
不僅僅是一些知識上的提升.
或者重溫一些比較重要的聖經教導.
同樣都是在你的生命有個行動.
令到那件事件在你生命當中出現.
這是重要的.
所以我先點題就是說回.
讓每一位願意加入Flow Church的弟兄姊妹.
入組之前要瞭解的內容.
我補充一下.
跟這位旁邊的女士對於牧羊的看法.
我想起其中一個想法.
教會的存在也好運作也好.
其實不可以只依賴牧者.
我不是說加入或者潘Sir的牧者存在不重要.
而是說信徒也要學會自發和組織性.
因為剛才潘Sir的說法.
Club這東西本身也是由人自發.
有興趣就去開創.
大家要有這樣的自發性和組織性的創業.
才可以令到教會本身都是百花齊放.
這樣的教會才可以不斷.
像水一樣生生不息.
萬一像阿John或者潘Sir的牧者.
突然不在你身邊.
或者他不幸地有事而不見了.
你的信徒如果不懂自發.
重新組織自己.
大家去組織維繫教會.
是不是就這樣就會被教會分散.
不是不應該是這樣.

$^{2881}$所以之後的教會應該是.
你要如何去繼續.
其實存在不重要.
在一個有形的物質世界來說.
就好像沒有大台.
但不可以沒有組織的一個.
一個這樣的屬靈的.
怎麼說呢?概念或者一個社區.
這一點我少少補充.
我認同弟兄的內容.
不過在Flow Church的做法是.
Flow Church的做法就是.
沒有人就沒有事工.
當然是需要人的.
所以我要讓你瞭解我們的做法.
我不是說你做的不對.
但我說Flow Church的做法就是.
沒有那種參與的人就不會有事工.
意思就是如果沒有人願意學咖啡.
我們就不會開一個咖啡事工.
重點就是他有興趣想用這個方法去服事.
或者讓他更加接觸到社區.
我們就開展這個事工的內容.
正正就是可能和其他教會經歷不同.
其他教會經歷就開了一個事工出來.
然後就邀請人去fill up那個事工.
如果那群弟兄姐妹發現.
他已經過了他想服事的時間.
那事工仍然存在.
每年都要邀請人去fill up那個事工.
就使事情不斷地要邀請人.
我們的做法就是.
那個事工是因為有那群弟兄姐妹做.
如果那群弟兄姐妹沒有了.
其實都證明那個事工是有效的日期.
在Flow Church的事工運作就是.
不會開了不能關的.
我們會知道有些事工是有效的日期.
做了繼續做.
就好像剛才回應John的做法.

$^{2921}$如果有些事件看到有果效.
又吸引到其他人的時候.
就會不斷地flow那些人來運作那個事工.
這個就看到當市場需要.
或者人弟兄姐妹需要的時候.
就能夠知道那件事就到位了.
這個是很直接去感受到.
這個也是Flow Church去扮.
如果你當在學校的club.
就算你放在booth.
那個club是否吸引人都沒有用.
會斷樁的.
但是這個就讓人明白到.
不是為了一個崗位去run那件事.
而是為了人去做那件事.
都是以人為本.
本身應該是可以尊重弟兄姐妹的需要.
或者他們有什麼專長.
有沒有其他呢?.
(觀眾:有).
我想問剛才提到教會有很多event.
其實本身事工的重要性.
是否和教會是同等的呢?.
教會....
譬如這部琴作一個比喻.
這部琴本身的存在是否也有價值呢?.
還是它要不斷地在彈很多歌.
那些歌才有價值呢?.
還是兩樣都重要呢?.
我想那個意思.
首先event和事工是兩回事.
event和事工不是同等的.
這個概念正正是一個行動driven的概念.
所以對我來說.
教會是一個行動driven的概念.
這個講得比較抽象.
比較conceptual.
教會是什麼呢?.
教會不是先有教會的being.
才有教會的不同行動.

$^{2961}$對我來說.
因為你做一些教會的東西.
這些東西就構成了教會.
對我來說是這樣.
當然教會是一個很複雜的概念.
因為信徒基督徒當然是一個人.
但教會就是一群基督徒.
一群信耶穌的人.
他們一起去做一些事.
這樣的事就構成了教會.
我先拋開教會的觀念.
教會不是一件東西.
是這群人行動了才構成教會.
我舉兩個例子.
我那時候在德國去過蘇黎世.
在那裡辭運了一個教會.
很大一間教會.
不知道是什麼姓姓母的教會.
那間教會只是一間景點.
已經沒有崇拜了.
那間是不是教會呢?.
當然不是教會.
那間教堂.
那次我看的時候.
旁邊有一群人在傳福音.
大聲的大叫福音.
後來有警察抓了他.
因為他很吵.
很大的對比.
不是因為教會一個人在那裡.
就做一些教會的事.
而是當你做一些教會應該做的事.
那個才叫教會.
教會是一個活動的意思.
活動是一個發生的事情.
而不是一個有本體的事情.
有時候是這樣說的.
所以不是教會什麼都不做.
那就是教會.
我經常說.

$^{3001}$開這個話題.
牧師也是.
按了牧就成牧師.
你星期天放假.
你也是牧師.
但牧師其實是建基於牧養的行動.
不是因為你做牧師才去牧養.
所以我覺得這是很新教的概念.
行動主要是有本體.
多於有本體才做出行動.
也是一樣.
我們是基督徒.
我們做一些基督徒的事.
才去構成我們的基督徒身份.
多於我已經得救了.
我才去想想怎麼做.
所以是一個很強的行動主要概念.
那所以.
不好意思想問一問.
有關於一個個人的信徒和教會的關係.
因為剛才聽到是以一個事件.
或者以一個行動的形式去參與一個教會.
那我舉一個例子.
如果一個信徒.
他在參與某一間教會.
他有一個很有passion的.
就假設是可能是中咖啡.
他又有另一個passion.
可能在另一方面.
他參加可能在另一間教會的.
一個這樣的event.
可能他也是在做一個.
一個教會性質的event在發生.
但他同時參與兩間堂會.
那其實這樣算不算是一個健康還是不健康.
當教會的觀念都傾向於一個事件發生化.
可能是沒有以前好像很傳統那種.
很固化或者是一個很堂會式的.
很恆常很規律.
或者是一個信徒通常都會在某一個堂會裡面.

$^{3041}$去完成他整個教會生活的那部分.
那現在如果有這樣的情況.
或者是你剛才提出的這個比較.
就是和平時教會的這個觀念比較flow一點.
就是闊一點或者碎片化一點的concept.
那會是一個健康還是不健康.
你怎麼看呢.
謝謝你的問題.
幫我補充我整個教會論的看法.
剛才我說行動driven.
但堂會.
我沒有機會去講堂會.
就是堂會正正是什麼呢.
堂會正正是去facilitate這些行動的一個組織.
你看馬保羅的書也這麼說.
堂會真的是一間公司.
這句話不是負面的.
堂會真的是一個purpose來製造很多行動的組織.
堂會就是這樣.
堂會搞崇巴.
堂會搞段宣.
堂會搞騎勞會.
都是去做這些事.
所以我覺得有堂會是重要的.
因為是需要有一個這樣的.
大也好小也好的組織.
來去人為地做這些事出來的.
所以我覺得我不贊成沒有堂會就是這個意思.
因為我覺得是有一些人為組織的東西.
不是無為的.
但現在這個世代其實是很多這樣的.
就是那個堂會A可能你也去過.
就是堂會A正在縱視.
但堂會B就有一個班友friend.
然後回復出崇拜之類的.
現在這個世代就是這樣.
這件事我覺得也沒有問題.
因為我都說了.
我回某間教會品牌的時工.
那個崇拜的教會.

$^{3081}$那個搞的騎勞會.
搞的段宣.
那些人.
但這樣的教會是不包那些東西的.
但我仍然是有一個.
就是所謂實踐基督徒要做的事.
我有敬拜.
有喜相愛.
有行動有使命.
不過是不同品牌的組織的人.
去搞的東西出來的.
這個似乎是那個趨勢.
在沒有疆界和一個後疫情時代.
現在是越來越有助於這樣.
而這件事其實又無損教會論.
因為剛才說過.
有行動有使命.
不過是不同品牌組織出來的活動.
這個我自己覺得是可以的.
似乎趨勢也是這樣.
所以我的教會論就會覺得.
是這樣的就可以了.
因為我們說行動為本.
正正不是唐會品牌為本.
所以我們有一個真正的行動.
這行動是有意思的.
不過是不同品牌出來的行動.
所以就是為什麼.
是更加有用的.
不過我仍然覺得是很多東西.
沒有機會講得那麼深.
我們有很多東西.
一個弟妹成長很多東西.
當然有崇拜.
一個很重要的就是一個mentor.
這個正正是牧者很重要的關係.
牧者的牧養最重要是.
能夠有人跟你一起去走.
需要很多選擇.
你可以去參加某個名道社的班.

$^{3121}$神流院的課程.
這些都是有的.
但是一個牧者牧養的關係.
這件事也是很重要的.
所以又不是純粹組織活動.
最能夠滿足的事情.
所以我也覺得.
為什麼我們仍然花很多資源.
去保護牧養.
雖然現在說一點.
雖然說Folkshot好像不是很簡單.
其實不是.
Folkshot仍然保護牧養的簡單.
這部分仍然是很簡單.
不過其他東西其實有很多不同的.
你現在在直播.
當然不簡單.
但是牧養仍然是最簡單的.
牧者也不需要碰那些東西.
去幫助牧養維持這種關係.
用Folkshot的例子.
或者是運作去回應.
Info Group也提過這個例子.
當Folkshot仍然在沒有限聚的情況下.
實體聚會大概有五百多人.
每個崇拜.
但一直看著數字.
整個會眾的比率.
或者成分是怎樣.
其實有三分一是有小組的.
另外三分一就是.
每個月大概來兩次崇拜.
另外有三分一是.
每個月來一次.
或者是新朋友.
其中有每個月來兩次的弟兄姊妹.
曾經跟我談.
他很認真地說.
潘Sir,其實我每個月來兩次崇拜.
因為我星期日.

$^{3161}$還要在堂會服事.
如果不用服事.
星期日我星期六就可以來.
如果星期日要服事.
我星期六就要錄音.
所以我就來不了.
其實Folkshot這樣聚會是否可行.
我問為什麼不行.
他說你不覺得我去兩次堂會不可以嗎.
我說我可以.
不過我不知道你教會是否可以.
他問為什麼Folkshot覺得可以.
因為我很認真地跟他說.
如果你覺得在這裡崇拜.
是可以讓你充滿.
而你在這一班人服事.
另一個群體而我碰不到.
你是一個教會的延伸.
本身是一個教會的一體形容.
這就是很重要的.
因為我能夠慰養你.
而你能夠慰養其他人.
這不是國安國職的一體化表徵嗎.
這也是剛才說的School的概念.
如果他自己堂會未能夠慰養.
或者他覺得要多點位置.
他在其他地方能夠吃得到.
或者吃得飽的話.
其實彼此配搭.
這也是需要的.
當然我仍然認同John所說的.
他一定要有一個教育.
否則他慢慢都會枯乾.
或者慢慢的那個慰養概念.
就變成了為了做而做的活動.
就沒有了那個.
慰養就是慰養.
所以慰者的角色仍然是很重要的.
而我們的教會觀不是闊.
是因為我們知道.

$^{3201}$應該不要用堂會去定義一個位置.
沒有了互為肢體配搭.
彼此協作的那種尾線.
十點了,我們收到報告了.
下次你就早點命令.
下次見,我們在食堂.
拜拜.
請不吝點贊 訂閱 轉發 打賞支持明鏡與點點欄目.
\newpage



\section{}
\label{sec:d_aSxcuQPus}
\textbf{【這是最好的時代:給香港基督徒的神學八課】第4課:亂世的靈性修持|20210822 [d\_aSxcuQPus]}
\newline
\newline
連結: \href{https://youtube.com/watch?v=d_aSxcuQPus}{\texttt{ https://youtube.com/watch?v=d\_aSxcuQPus}} ~~~~ 語音日期: 2021-08-22 
\newline
\newline
\hyperref[sec:gexfrTB3Ccc]{\small{< < < PREV SERMON < < <}}
~
\hyperref[sec:index_chronic]{\small{[返順時目]}}
~
\hyperref[sec:index_scriptual]{\small{[返順卷目]}}
~
\hyperref[sec:VhMBgPBkDi8]{\small{> > > NEXT SERMON > > >}}
\newline
\newline
$^{1}$我只想知道.
你到底是什麼意思.
我只想知道.
你到底是什麼意思.
我只想知道.
你到底是什麼意思.
我只想知道.
你到底是什麼意思.
我只想知道.
你到底是什麼意思.
我只想知道.
你到底是什麼意思.
我只想知道.
你到底是什麼意思.
我只想知道.
你到底是什麼意思.
我只想知道.
你到底是什麼意思.
我只想知道.
你到底是什麼意思.
我只想知道.
你到底是什麼意思.
我只想知道.
你到底是什麼意思.
我只想知道.
你到底是什麼意思.
我只想知道.
你到底是什麼意思.
我只想知道.
你到底是什麼意思.
我只想知道.
你到底是什麼意思.
我只想知道.
你到底是什麼意思.
我只想知道.
你到底是什麼意思.
我只想知道.
你到底是什麼意思.
我只想知道.
你到底是什麼意思.

$^{41}$我只想知道.
你到底是什麼意思.
我只想知道.
你到底是什麼意思.
我只想知道.
你到底是什麼意思.
我只想知道.
你到底是什麼意思.
我只想知道.
你到底是什麼意思.
我只想知道.
你到底是什麼意思.
我只想知道.
你到底是什麼意思.
我只想知道.
你到底是什麼意思.
我只想知道.
你到底是什麼意思.
我只想知道.
你到底是什麼意思.
我只想知道.
你到底是什麼意思.
我只想知道.
你到底是什麼意思.
我只想知道.
你到底是什麼意思.
我只想知道.
你到底是什麼意思.
我只想知道.
你到底是什麼意思.
我只想知道.
你到底是什麼意思.
我只想知道.
你到底是什麼意思.
我只想知道.
你到底是什麼意思.
我只想知道.
你到底是什麼意思.
我只想知道.
你到底是什麼意思.

$^{81}$我只想知道.
你到底是什麼意思.
我只想知道.
你到底是什麼意思.
我只想知道.
你到底是什麼意思.
我只想知道.
你到底是什麼意思.
我只想知道.
你到底是什麼意思.
我只想知道.
你到底是什麼意思.
我只想知道.
你到底是什麼意思.
我只想知道.
你到底是什麼意思.
我只想知道.
你到底是什麼意思.
我只想知道.
你到底是什麼意思.
我只想知道.
你到底是什麼意思.
我只想知道.
你到底是什麼意思.
我只想知道.
你到底是什麼意思.
我只想知道.
你到底是什麼意思.
我只想知道.
你到底是什麼意思.
我只想知道.
你到底是什麼意思.
我只想知道.
你到底是什麼意思.
我只想知道.
你到底是什麼意思.
我只想知道.
你到底是什麼意思.
我只想知道.
你到底是什麼意思.

$^{121}$我只想知道.
你到底是什麼意思.
我只想知道.
你到底是什麼意思.
我只想知道.
你到底是什麼意思.
我只想知道.
你到底是什麼意思.
我只想知道.
你到底是什麼意思.
我只想知道.
你到底是什麼意思.
我只想知道.
你到底是什麼意思.
我只想知道.
你到底是什麼意思.
我只想知道.
你到底是什麼意思.
我只想知道.
你到底是什麼意思.
我只想知道.
你到底是什麼意思.
我只想知道.
你到底是什麼意思.
我只想知道.
你到底是什麼意思.
我只想知道.
你到底是什麼意思.
我只想知道.
你到底是什麼意思.
我只想知道.
你到底是什麼意思.
我只想知道.
你到底是什麼意思.
我只想知道.
你到底是什麼意思.
我只想知道.
你到底是什麼意思.
我只想知道.
你到底是什麼意思.

$^{161}$我只想知道.
你到底是什麼意思.
我只想知道.
你到底是什麼意思.
我只想知道.
你到底是什麼意思.
我只想知道.
你到底是什麼意思.
我只想知道.
你到底是什麼意思.
我只想知道.
你到底是什麼意思.
我只想知道.
你到底是什麼意思.
我只想知道.
你到底是什麼意思.
我只想知道.
你到底是什麼意思.
我只想知道.
你到底是什麼意思.
我只想知道.
你到底是什麼意思.
我只想知道.
你到底是什麼意思.
我只想知道.
你到底是什麼意思.
我只想知道.
你到底是什麼意思.
我只想知道.
你到底是什麼意思.
我只想知道.
你到底是什麼意思.
我只想知道.
你到底是什麼意思.
我只想知道.
你到底是什麼意思.
我只想知道.
你到底是什麼意思.
我只想知道.
你到底是什麼意思.

$^{201}$我只想知道.
你到底是什麼意思.
我只想知道.
你到底是什麼意思.
我只想知道.
你到底是什麼意思.
我只想知道.
你到底是什麼意思.
我只想知道.
你到底是什麼意思.
我只想知道.
你到底是什麼意思.
我只想知道.
你到底是什麼意思.
我只想知道.
你到底是什麼意思.
我只想知道.
你到底是什麼意思.
我只想知道.
你到底是什麼意思.
我只想知道.
你到底是什麼意思.
我只想知道.
你到底是什麼意思.
我只想知道.
你到底是什麼意思.
我只想知道.
你到底是什麼意思.
我只想知道.
你到底是什麼意思.
我只想知道.
你到底是什麼意思.
我只想知道.
你到底是什麼意思.
我只想知道.
你到底是什麼意思.
我只想知道.
你到底是什麼意思.
我只想知道.
你到底是什麼意思.

$^{241}$我只想知道.
你到底是什麼意思.
我只想知道.
你到底是什麼意思.
我只想知道.
你到底是什麼意思.
我只想知道.
你到底是什麼意思.
我只想知道.
你到底是什麼意思.
我只想知道.
你到底是什麼意思.
我只想知道.
你到底是什麼意思.
我只想知道.
你到底是什麼意思.
我只想知道.
你到底是什麼意思.
我只想知道.
你到底是什麼意思.
我只想知道.
你到底是什麼意思.
我只想知道.
你到底是什麼意思.
我只想知道.
你到底是什麼意思.
我只想知道.
你到底是什麼意思.
我只想知道.
你到底是什麼意思.
我只想知道.
你到底是什麼意思.
我只想知道.
你到底是什麼意思.
我只想知道.
你到底是什麼意思.
我只想知道.
你到底是什麼意思.
我只想知道.
你到底是什麼意思.

$^{281}$我只想知道.
你到底是什麼意思.
我只想知道.
你到底是什麼意思.
我只想知道.
你到底是什麼意思.
我只想知道.
你到底是什麼意思.
我只想知道.
你到底是什麼意思.
我只想知道.
你到底是什麼意思.
我只想知道.
你到底是什麼意思.
我只想知道.
你到底是什麼意思.
我只想知道.
你到底是什麼意思.
我只想知道.
你到底是什麼意思.
我只想知道.
你到底是什麼意思.
我只想知道.
你到底是什麼意思.
我只想知道.
你到底是什麼意思.
我只想知道.
你到底是什麼意思.
我只想知道.
你到底是什麼意思.
我只想知道.
你到底是什麼意思.
我只想知道.
你到底是什麼意思.
我只想知道.
你到底是什麼意思.
我只想知道.
你到底是什麼意思.
我只想知道.
你到底是什麼意思.

$^{321}$我只想知道.
你到底是什麼意思.
我只想知道.
你到底是什麼意思.
我只想知道.
你到底是什麼意思.
我只想知道.
你到底是什麼意思.
我只想知道.
你到底是什麼意思.
我只想知道.
你到底是什麼意思.
我只想知道.
你到底是什麼意思.
我只想知道.
你到底是什麼意思.
我只想知道.
你到底是什麼意思.
我只想知道.
你到底是什麼意思.
我只想知道.
你到底是什麼意思.
我只想知道.
你到底是什麼意思.
我只想知道.
你到底是什麼意思.
我只想知道.
你到底是什麼意思.
我只想知道.
你到底是什麼意思.
我只想知道.
你到底是什麼意思.
我只想知道.
你到底是什麼意思.
我只想知道.
你到底是什麼意思.
我只想知道.
你到底是什麼意思.
我只想知道.
你到底是什麼意思.

$^{361}$我只想知道.
你到底是什麼意思.
我只想知道.
你到底是什麼意思.
我只想知道.
你到底是什麼意思.
我只想知道.
你到底是什麼意思.
我只想知道.
你到底是什麼意思.
我只想知道.
你到底是什麼意思.
我只想知道.
你到底是什麼意思.
我只想知道.
你到底是什麼意思.
我只想知道.
你到底是什麼意思.
我只想知道.
你到底是什麼意思.
我只想知道.
你到底是什麼意思.
我只想知道.
你到底是什麼意思.
我只想知道.
你到底是什麼意思.
我只想知道.
你到底是什麼意思.
我只想知道.
你到底是什麼意思.
我只想知道.
你到底是什麼意思.
我只想知道.
你到底是什麼意思.
我只想知道.
你到底是什麼意思.
我只想知道.
你到底是什麼意思.
我只想知道.
你到底是什麼意思.

$^{401}$我只想知道.
你到底是什麼意思.
我只想知道.
你到底是什麼意思.
我只想知道.
你到底是什麼意思.
我只想知道.
你到底是什麼意思.
我只想知道.
你到底是什麼意思.
我只想知道.
你到底是什麼意思.
我只想知道.
你到底是什麼意思.
我只想知道.
你到底是什麼意思.
我只想知道.
你到底是什麼意思.
我只想知道.
你到底是什麼意思.
我只想知道.
你到底是什麼意思.
我只想知道.
你到底是什麼意思.
我只想知道.
你到底是什麼意思.
我只想知道.
你到底是什麼意思.
我只想知道.
你到底是什麼意思.
我只想知道.
你到底是什麼意思.
我只想知道.
你到底是什麼意思.
我只想知道.
你到底是什麼意思.
我只想知道.
你到底是什麼意思.
我只想知道.
你到底是什麼意思.

$^{441}$我只想知道.
你到底是什麼意思.
我只想知道.
你到底是什麼意思.
我只想知道.
你到底是什麼意思.
我只想知道.
你到底是什麼意思.
我只想知道.
你到底是什麼意思.
我只想知道.
你到底是什麼意思.
我只想知道.
你到底是什麼意思.
我只想知道.
你到底是什麼意思.
我只想知道.
你到底是什麼意思.
我只想知道.
你到底是什麼意思.
我只想知道.
你到底是什麼意思.
我只想知道.
你到底是什麼意思.
我只想知道.
你到底是什麼意思.
我只想知道.
你到底是什麼意思.
我只想知道.
你到底是什麼意思.
我只想知道.
你到底是什麼意思.
我只想知道.
你到底是什麼意思.
我只想知道.
你到底是什麼意思.
我只想知道.
你到底是什麼意思.
我只想知道.
你到底是什麼意思.

$^{481}$我只想知道.
你到底是什麼意思.
我只想知道.
你到底是什麼意思.
我只想知道.
你到底是什麼意思.
我只想知道.
你到底是什麼意思.
我只想知道.
你到底是什麼意思.
我只想知道.
你到底是什麼意思.
我只想知道.
你到底是什麼意思.
我只想知道.
你到底是什麼意思.
我只想知道.
你到底是什麼意思.
我只想知道.
你到底是什麼意思.
我只想知道.
你到底是什麼意思.
我只想知道.
你到底是什麼意思.
我只想知道.
你到底是什麼意思.
我只想知道.
你到底是什麼意思.
我只想知道.
你到底是什麼意思.
我只想知道.
你到底是什麼意思.
我只想知道.
你到底是什麼意思.
我只想知道.
你到底是什麼意思.
我只想知道.
你到底是什麼意思.
我只想知道.
你到底是什麼意思.

$^{521}$我只想知道.
你到底是什麼意思.
我只想知道.
你到底是什麼意思.
我只想知道.
你到底是什麼意思.
我只想知道.
你到底是什麼意思.
我只想知道.
你到底是什麼意思.
我只想知道.
你到底是什麼意思.
我只想知道.
你到底是什麼意思.
我只想知道.
你到底是什麼意思.
我只想知道.
你到底是什麼意思.
我只想知道.
你到底是什麼意思.
我只想知道.
你到底是什麼意思.
我只想知道.
你到底是什麼意思.
我只想知道.
你到底是什麼意思.
我只想知道.
你到底是什麼意思.
我只想知道.
你到底是什麼意思.
我只想知道.
你到底是什麼意思.
我只想知道.
你到底是什麼意思.
我只想知道.
你到底是什麼意思.
我只想知道.
你到底是什麼意思.
我只想知道.
你到底是什麼意思.

$^{561}$我只想知道.
你到底是什麼意思.
我只想知道.
你到底是什麼意思.
我只想知道.
你到底是什麼意思.
我只想知道.
你到底是什麼意思.
我只想知道.
你到底是什麼意思.
我只想知道.
你到底是什麼意思.
我只想知道.
你到底是什麼意思.
我只想知道.
你到底是什麼意思.
我只想知道.
你到底是什麼意思.
我只想知道.
你到底是什麼意思.
我只想知道.
你到底是什麼意思.
我只想知道.
你到底是什麼意思.
我只想知道.
你到底是什麼意思.
我只想知道.
你到底是什麼意思.
我只想知道.
你到底是什麼意思.
我只想知道.
你到底是什麼意思.
我只想知道.
你到底是什麼意思.
我只想知道.
你到底是什麼意思.
我只想知道.
你到底是什麼意思.
我只想知道.
你到底是什麼意思.

$^{601}$我只想知道.
你到底是什麼意思.
我只想知道.
你到底是什麼意思.
我只想知道.
你到底是什麼意思.
我只想知道.
你到底是什麼意思.
我只想知道.
你到底是什麼意思.
我只想知道.
你到底是什麼意思.
我只想知道.
你到底是什麼意思.
我只想知道.
你到底是什麼意思.
我只想知道.
你到底是什麼意思.
我只想知道.
你到底是什麼意思.
我只想知道.
你到底是什麼意思.
我只想知道.
你到底是什麼意思.
我只想知道.
你到底是什麼意思.
我只想知道.
你到底是什麼意思.
我只想知道.
你到底是什麼意思.
我只想知道.
你到底是什麼意思.
我只想知道.
你到底是什麼意思.
我只想知道.
你到底是什麼意思.
我只想知道.
你到底是什麼意思.
我只想知道.
你到底是什麼意思.

$^{641}$我只想知道.
你到底是什麼意思.
我只想知道.
你到底是什麼意思.
我只想知道.
你到底是什麼意思.
我只想知道.
你到底是什麼意思.
我只想知道.
你到底是什麼意思.
我只想知道.
你到底是什麼意思.
我只想知道.
你到底是什麼意思.
我只想知道.
你到底是什麼意思.
我只想知道.
你到底是什麼意思.
我只想知道.
你到底是什麼意思.
我只想知道.
你到底是什麼意思.
我只想知道.
你到底是什麼意思.
我只想知道.
你到底是什麼意思.
我只想知道.
你到底是什麼意思.
我只想知道.
你到底是什麼意思.
我只想知道.
你到底是什麼意思.
我只想知道.
你到底是什麼意思.
我只想知道.
你到底是什麼意思.
我只想知道.
你到底是什麼意思.
我只想知道.
你到底是什麼意思.

$^{681}$我只想知道.
你到底是什麼意思.
我只想知道.
你到底是什麼意思.
我只想知道.
你到底是什麼意思.
我只想知道.
你到底是什麼意思.
我只想知道.
你到底是什麼意思.
我只想知道.
你到底是什麼意思.
我只想知道.
你到底是什麼意思.
我只想知道.
你到底是什麼意思.
我只想知道.
你到底是什麼意思.
我只想知道.
你到底是什麼意思.
我只想知道.
你到底是什麼意思.
我只想知道.
你到底是什麼意思.
我只想知道.
你到底是什麼意思.
我只想知道.
你到底是什麼意思.
我只想知道.
你到底是什麼意思.
我只想知道.
你到底是什麼意思.
我只想知道.
你到底是什麼意思.
我只想知道.
你到底是什麼意思.
我只想知道.
你到底是什麼意思.
我只想知道.
你到底是什麼意思.

$^{721}$我只想知道.
你到底是什麼意思.
我只想知道.
你到底是什麼意思.
我只想知道.
你到底是什麼意思.
我只想知道.
你到底是什麼意思.
我只想知道.
你到底是什麼意思.
我只想知道.
你到底是什麼意思.
我只想知道.
你到底是什麼意思.
我只想知道.
你到底是什麼意思.
我只想知道.
你到底是什麼意思.
我只想知道.
你到底是什麼意思.
我只想知道.
你到底是什麼意思.
我只想知道.
你到底是什麼意思.
我只想知道.
你到底是什麼意思.
我只想知道.
你到底是什麼意思.
我只想知道.
你到底是什麼意思.
我只想知道.
你到底是什麼意思.
我只想知道.
你到底是什麼意思.
香港 香港 天涯思念的夢.
香港 香港 再有我同你夢想.
香港 香港 叫我不已快樂.
香港 香港 你永遠是曾夢想.
每一個年代都有每一個年代的神學.
作為土生土長的香港人.

$^{761}$我們似乎正在經歷一個最差的年代.
不過往往在最差的年代.
我們才能夠經歷福音信仰的最好.
就是我們一起從聖經裡面學習.
如何做這個年代裡面的香港基督徒.
這是最好的時代.
給香港基督徒的神學百科.
各位弟兄姊妹晚安.
我們來到第四堂.
香港基督徒神學百科的第四堂.
開始我們會從福音基督徒教會之外.
我們會講一些比較應用性的東西.
我們會講一下靈修或者靈性.
題目叫亂世的靈性修持.
我自己喜歡這個名字.
因為修持這個名字在教會裡面比較少用.
修持是修得來又要持.
保持.
我挺喜歡這個.
這個名字似乎是更加盡心的題目.
比起所謂靈修.
靈修的字就是道士打坐.
修行或者修煉.
基督徒的靈性.
我自己回到香港之後這麼多年.
都不斷有寫或者有講.
如果你有看我以前的書.
講靈命靈修.
或者如果你有上神學院的課程.
我自己也有講到有關禱告或者靈性的題目.
所以今天我自己這一堂有很多東西要講.
關於我們整個基督徒.
如何理解我們的靈性靈命.
我自己那本書還沒寫完.
其中一件事就是講這件事.
這本書還沒寫完.
所以我很多東西可以和大家分享.
當然加上一個字就是亂世.
特別在這個年代裡面.
我們如何理解我們的亂世.

$^{801}$所以我們怎麼開始呢.
不如講一下我自己的靈情.
我自己是十八歲信耶穌.
可能都講過.
當時初信書的時候.
都是和大家一樣.
都是被教導我們要去靈修.
以前我們那時候年代是有那些.
Form of 金泉.
那些靈修書.
每個月都會派一本給我們去做.
有些很厚的靈修書.
我們會去做.
我初信書的時候很乖的.
都是一直跟著做.
都是一個很典型的.
很上進的初信者.
那時候就跟著靈修書做.
甚至乎給自己一些.
extra 的靈修功夫.
那時候我屬靈九果.
每天都會選一個果子來做.
今天是喜樂日.
就做一些喜樂的事情.
今天是忍耐就忍耐.
很多這些計劃給自己.
很快的我就讀神學.
讀神學的時候.
是我自己思考靈修.
變成一個很重要的階段.
那時候我住長洲宿舍.
那時候試過很多不同的方法.
來思考靈修這件事.
試過用很多不同的方法去靈修.
那時候試過看一本叫做.
揭露出來的能力的書.
是藤木思的.
那本書是一本很好的.
給自己屬靈反省的書.
那時候就寫到.

$^{841}$每天早上要跟著以前的屬靈的人.
五點鐘起床祈禱靈修.
那時候我跟著.
你不要幾點鐘睡.
起碼五點鐘起床靈修.
那時候剛學完文就用完文靈修.
試過用馮錦坤的羅馬書.
逐一靈修.
試過跑步靈修.
試過上天台裡面祈禱.
試過不靈修.
所以那時候正正就是一個.
不斷來思考靈修是一回事.
我自己以前是一個挺勤人.
以前.
現在是另一種方式.
以前是跟著靈修.
敬勤的方法來思考這件事.
後來我發覺我挺喜歡一個方法.
就是跑步靈修.
不知道大家有沒有試過.
跑步的時候一直祈禱.
甚至帶著這封.
聽聽廣道.
聽聽靈修分享.
跑步是一個跟上帝最親近的時間.
因為跑步的時候.
是我自己一直在呼吸.
完全是一個受眾物的狀態.
整個人心跳.
每分鐘跳130多下.
喘氣流汗.
那個狀態是一個最.
好像最貼近自己做一個.
本相就是一個被上帝所創造的人.
這樣的狀態.
後來我發覺.
原來靈修不是一定要在這裡.
靈修很多時候是一件事.
我發現靈修是我的生活裡面.

$^{881}$那時候我開始打算去外國讀書.
你知道我是研究巴特.
那時候我研究巴特的原因是這個主題.
他寫了一句很特別的說話.
祈禱和當基督徒是一樣的事情.
那時候就被這句說話吸引了.
為什麼呢.
什麼叫祈禱和當基督徒是一樣的事情呢.
開始發現.
祈禱不是純屬是閉上眼.
不是純屬靜下來.
或者做一個閱讀理解.
然後這樣的靜態活動.
而是我的生活.
我的整個人.
就是一種和上帝的關係.
然後我就去德國.
去德國的時候.
我每天四十五分鐘就去圖書館裡面工作.
看神學書.
看聖經.
預備講章.
那時候的生活是一至五.
跑步.
然後做運動.
去圖書館做論文.
回家.
一個很極度像作家的生活.
完全是很規律的生活.
那時候我開始慢慢衝破一個心理障礙.
這個大家先不要學.
要聽完我整個講話.
你才去反省這件事.
那時候我就開始開所謂的不靈修.
就是一般華人教會所定義的那種靈修.
我沒有做到.
這個是有點勇氣的.
因為你要突然衝破某些人叫你這樣做.
要這樣讀那本靈修書.
你也知道我現在是寫意道節的靈修學者.

$^{921}$我自己寫給別人看.
但我自己也不看自己的東西.
當然我會講更加詳細的進程.
那時候我就開始沒有這樣的靈修.
因為那時候我天天都在看聖經.
天天都有思想上帝.
天天都是來到.
整個生活都是一種.
我這樣理解.
整個生活都是和上帝的關係.
特別是卡爾巴特.
如果看卡爾巴特的話.
德文基本上沒有什麼靈修的字眼.
翻譯為Spirituality.
英文轉過來的德文字.
沒有一個具體的靈修的字眼.
我們會叫做敬虔性.
德文其實沒有一個靈性.
或者是這樣的字.
都是將英文的字變成德文.
所以開始就是跟著那個說法.
To pray and to be Christian.
是同一件事.
禱告或者做基督徒是一件事.
我當時就嘗試將靈修或者靈命.
融入我的教會.
在我的生活裡面.
我回來香港的2013年.
是懷著這樣的態度.
那時候我很恨那些靈修人.
心裡面不是恨.
我經常看別人的方法教靈修.
我會抗拒那些.
抗拒那些靈修大師.
通常靈修大師會說話很慢.
然後會很婦道形式地跟你說話.
我不是那種人.
但我自己讀的博士論文是做這些事的.
基本上都是一種禱告.
禱告基本上都是一種類似靈修的事.

$^{961}$我有一種反靈修的方法.
來理解靈修這件事.
那時候我懷著這種神學觀.
在建度裡面教書.
整個的生命就是靈修.
整個生活就是我的靈性的彰顯.
如果你看我的粉紅書.
其實都是大概會說這些東西的.
只不過我沒有說得那麼坦白.
不過我後來就去.
因為我在建度裡面教.
就教教會歷史.
教了大概.
基本上都教了六七年.
都會教教會歷史.
所以特別對於中世紀.
我稱之為神秘主義者.
我特別有興趣.
我就開始迷上神秘主義這兩樣東西.
神秘主義不是一些.
你今天好像很迷信的東西.
神秘主義反而是一些很知性的人.
他們覺得因為上帝是超越我們的.
所以他更加不能用我們的理性.
或者用我們的語言來理解他.
我發覺這幫中世紀的神秘主義.
特別是女性神秘主義.
一幫來自法國,德國,英國.
這些神秘主義的女性.
是一幫很前衛的人.
今天她們是用一些本土語言來寫書.
以前是拉丁文.
她們那時候就用一些法文,英文來寫東西.
以前是女性沒有什麼地位.
就用一些自己的語言.
來寫一些和上帝關係的文字出來.
我發覺原來這些東西是像我的東西.
因為你不知道大家有沒有聽過十格約翰.
就是十六世紀一個很出名的西班牙.
一個所謂的靈修的修道士.

$^{1001}$今天我們叫他靈修.
其實就是一個比較好聽的.
全部靈修的人都是神秘主義者.
今天你找一個在讀靈修博士的人.
他都是在研究那些人.
那些人正正就是神秘主義者.
我開始發覺原來我都喜歡這些東西.
因為你發覺.
剛才說十八藥王.
他就是用一些很.
你想想你要去說出和上帝關係的時候.
其實你用的語言是很不容易的.
不知道大家有沒有試過這樣.
聽到的時候.
你試過聽一些很精彩的讀法.
你抄下那些筆記.
你聽完之後發覺.
聽的時候會覺得很感動.
抄下的時候會發覺第一點.
愛主近身.
第二點多多祈禱.
第三 獻身基督.
發覺原來讀出來都沒有什麼特別.
但聽的時候發覺是很感動.
所以發覺和上帝的關係.
是一種只能夠用一些超越我們語言的方法.
才能夠記錄這種關係.
所以有一點點是私人或者神學的方式.
十八藥王只是一個西班牙私人.
如果你看回今天.
他在西班牙文學裡面是一個很重要的位置.
因為他用一些很特別的文學方式.
來形容這種和上帝的關係.
說到越南人.
除了和上帝的關係是一種生命的靈修之外.
我們更加要去找出一種和上帝的特別的時刻和關係.
而今天我們還教會靈修.
其實不夠神秘主義.
不夠靈修.
所以變成了一種閱讀理解.

$^{1041}$你發現靈修不過是閱讀理解.
甚至是詞語閱讀理解.
一堆文字叫你反省.
我經常覺得是很填鴨式的.
因為基本上我們去.
你想想什麼叫靈修.
就是你將某些靈修作者以前和上帝關係的東西.
人家講完你再討論一次.
少次人家將食物吃完再拿出來給你吃.
你將人家的東西來成為你的經歷.
這個可行的.
因為這個是上帝和你的關係.
但其實都是靠著人家的反省來成為你的反省.
所以今天其實很多東西要講.
因為可以講整個靈修的歷史.
神秘主義的一些重點.
如何奈若拉.
就是伊納爵.
如何將神秘主義規範化.
變成了今天所講的量產化的靈修之類.
但不講那麼遠.
今天我們想講的就是.
基本上我們想想.
如何重新來思考今天我們靈修這件事.
特別是全教會.
今天我們要講全教會.
如何理解我們的靈性靈修等等的主題.
所以我們首先要講一件事.
就是基本上我們的靈修或靈明.
這兩個是聖經沒有的字.
靈修和靈明.
基本上是聖經沒有的字.
這兩個字好像是過字.
可以嗎.
所以靈修和靈明是聖經沒有的.
不過我們經常會用到這兩個字.
我們和上帝的關係.
基本上是分兩部分.
如果我們用皇馬書來理解.
皇馬書頭十一章是什麼.

$^{1081}$就是上帝的工作.
上帝如何拆遣耶穌都不得到世上.
有聖靈來幫助我們.
如何拯救我們.
如何支持我們.
頭十一章就是上帝對我們的工作.
十二章開始是什麼.
我都講過很多次.
是人的倫理部分.
人如何去回應上帝.
所以我們和上帝的關係.
基本上是分開兩部分.
就是上帝如何在我們身上工作.
和我們如何去回應他.
所以基本上是雞和豉油的關係.
上帝出雞給我們所有東西.
而我們的靈性.
我們對上帝的回應.
就是這一點點的部分.
所以基本上我們的靈修.
是在因典里的關係.
所以今天首先要打明白.
靈修不是一種跑數.
基本上你能靈修的事.
已經暗示你在因典的關係里.
我們永遠都能夠懷著快樂.
懷著因典來回應上帝.
至於你有沒有回應.
如何回應.
這是一個極微小的部分.
上帝在基督里的工作.
已經是源源全全的.
主導整個的關係.
就算你今天沒有靈修.
這個因典仍然存在.
不過當然我們要問.
我們的回應是什麼意思.
我們如何回應上帝.
所以明顯有兩個不同的部分.
一個是客觀的部分.

$^{1121}$上帝已經做了.
你做不做也好.
上帝已經做了.
而我們所說的所謂的基督生活.
或者我們的靈性.
或者我們的靈修等等.
是一些我們回應上帝.
很微不足道的事物.
這是我們很重要的部分.
而對於我們對上帝的回應是怎麼樣.
基本上我們可以如何回應上帝.
如果將我們能夠回應上帝.
分為三大類的話.
基本上是分三大類.
這個在我們平時的講道里.
他將平時的講道分為三大類的話.
基本上離不開這三個東西.
是什麼呢.
我稱之為.
第一個就是遵從.
叫你做.
你能夠回應上帝.
就是跟從.
聽話的去遵從.
去做.
這個就是我們很大部分.
叫你傳福音.
叫你孝順父母.
叫你喜樂.
叫你敬拜.
這些全部都是遵從.
你能夠回應上帝.
可以做的事情.
有時有些東西不是只有做.
有時是一個信心的回應.
你信得過他.
你能夠用信心來回應他.
這些不關談為事.
所以我們基本上.
我們大部分華人教會.

$^{1161}$對於上帝的回應.
其實都是離不開這兩個.
遵從和信仰.
兩個部分.
你見到我還沒寫完第三個.
祈禱.
但我們發覺這三個東西.
其實都是一種.
互相重疊的關係.
即是說.
信仰等同於遵從.
遵從等同於祈禱.
祈禱等同於信仰.
三個是大家互為的一種關係.
所以我們通常都這麼理解.
你怎樣理解祈禱.
就是一種什麼.
就是一種遵從.
上帝叫你祈禱.
你便祈禱.
這個是對的.
因為你是聽從上帝的命令.
所以你便願意去回去祈禱會.
你願意去祈禱.
祈禱都是一種信心的表達.
我們願意相信祂.
所以你就會祈禱.
這個其實是卡巴神學.
但卡巴神學有一個很重要的補充.
其實反過來也是.
當我們願意去奉行上帝的時候.
當我們去嘗試去服從的時候.
其實這個本身也是一種祈禱.
我們願意去相信祂的時候.
其實這個也是一種祈禱.
什麼意思呢.
就是你願意去跟從耶穌.
去回應上帝的命令.
其實你帶著一種懇求的方式.
來去回應上帝.

$^{1201}$你願意去回教會.
不可停止聚會.
但你也帶著一種禱告的態度.
來去做這個命令.
因為發覺自己做不到.
當我們信的時候也是一樣.
我們願意信耶穌.
但我們只能夠說求主你幫助我.
因為我的信不足.
所以我們的信心是重要的.
但信心也成為了一種祈禱.
一種懇求的態度來去回應上帝.
所以你發覺.
任何事都可以是一種奉行.
任何事都可以是一種信心的表現.
任何事都可以成為一種禱告.
所以禱告可以這麼說.
正正是一種我們稱之為.
一切基督徒行動的底蘊.
一種任何的行動.
其實都是離不開禱告的本質.
包括你靈修.
意思不是叫你那一種祈禱.
不是叫靈修前祈禱.
或者靈修後祈禱那種祈禱.
而是你整個的生命.
你的靈命本身.
都是帶著一種祈禱的態度.
來去回應上帝.
主要是我願意去做.
但我做不到.
求你幫助我.
我願意相信你.
但我信心不夠.
求你幫助我.
基督徒其實最基本的一種行動.
就是這樣的祈禱.
不是一種聽話的祈禱.
或者命令的回應.
而是一種反過來.

$^{1241}$任何的奮鬥.
任何的信仰行動.
都是一種的禱告.
是當中的那樣.
所以我們的靈命.
是一種這樣的關係.
不知道你今天有沒有靈修.
不知道你這個星期有沒有靈修.
或者一個很軟弱的人.
他對於上帝那種的關係.
我覺得很多時候.
和上帝的關係.
是一種浪子回頭的關係.
或者是瑞利在聖殿裡面.
認罪的那種關係.
就是發覺你是all or nothing.
當你發覺自己無的時候.
突然間你可以去借著禱告.
回旋來親近他的時候.
你就是all.
你就是所有.
所以我們用這種理解來.
理解我們這種靈命態度.
因為我們華人教會.
常常都是會.
將靈命和靈修這個字扭曲了.
不知道大家怎麼看靈命.
你有沒有靈命.
你覺得靈命存不存在.
你說存在在哪裡.
靈命在哪裡.
靈命是不是在肚子里.
靈命是不是一些能吃的東西.
例如全對問你.
你這段時間靈命怎麼樣.
靈命是一些拿不出來看的東西.
但是我們發覺靈命是一種.
我用一種叫做唯命論的概念.
它不是真的有些東西叫靈命.
我們華人教會將靈命.

$^{1281}$看成一種.
有點大不慘的.
就像養鬼仔.
你養靈命在你生命裡面.
你要餵它.
你要養它.
你不餵它就會死.
靈命就會枯乾.
就像你養了貪婪歌詞.
你要養著它.
你靈命不好就靈修.
靈命就會加五.
它會培靈本就會加十.
整件事是百家組織分.
靈命是我們覺得可以存放的東西.
一樣東西叫靈命.
你要去養著它.
你要去培靈它.
要去修它.
所以發覺全對人的靈命好些.
因為它存放得很健康.
不知道等級一百多.
所以覺得靈命是積分.
我存放了很多.
所以都不存放.
其實這不是真正的靈命概念.
以前的人叫你不斷靈修.
就是為了好些靈命.
是對的.
靈修是跟你靈命有關係.
但不是積分的關係.
不是能夠存放到區塊鏈來換多些東西.
所以我們就想究竟我們所謂靈修.
我們平時所做的那些屬靈事情.
和我們靈命是什麼.
更加基本上是靈命.
如果靈命不是真實的鬼.
靈命其實是一種唯命論.
就等於熱氣.
熱氣是什麼.

$^{1321}$熱氣是在這裡嗎.
我熱氣.
熱氣是什麼.
就是一些氣.
也不是你肚子里的東西.
或者濕熱也是.
濕熱是不是肚子里黑色的東西.
濕熱只不過是我們給它一個名字.
把一些症狀給它一個名字.
所以靈命也一樣.
靈命不是一些我們來到.
覺得它真的存在一些實質的東西.
而是在量度你和上帝的關係.
所以基本上我們說.
靈命這個字是沒有的.
但其實它所說的東西是有的.
所謂靈命的靈字是什麼.
靈字其實就是性靈.
性靈在你生命裡面.
或者你的靈和上帝的靈之間的那種關係.
所以我很喜歡這兩段經文.
其實都是保羅在早期和後期.
早期在迦太叔.
後期在羅馬書所講的同樣的說話.
他說上帝就差他兒子的靈進入你們的心.
呼叫阿巴夫.
這是一個三一關係.
上帝的兒子差顯他們的靈進入你心裡.
同時你也會呼叫阿巴夫.
所以我們說呼叫是一種討告.
是一種呼喊.
我們成為上帝的兒女.
最基本的開始就是什麼.
就是你願意親口叫天父上帝做爸爸.
你就叫阿巴夫.
而阿巴夫是一種討告.
是一種生命的討告.
從此以後你任何事情.
整個生命.
你就願意用潛藏在生命裡面的阿巴夫的討告.

$^{1361}$去呼喊上帝.
它不是一個討告.
一個聽到聲音的討告.
而是你整個生命的態度.
從此以後就稱之為上帝的兒女.
所以這個是一個.
繼基督徒之後.
我們打堂講基督徒.
一個很重要的身份就是我們天父的兒女們.
我們是天父的兒女.
因為我們願意用我們的生命.
用一種討告來呼喊爸爸.
所以當我們願意呼喊他的時候.
我們的靈命就開始了.
因為上帝就差他兒子的靈進入你生命裡面.
你就出現了你有靈命這件事.
因為所謂靈命不是屬靈部分的生命.
而是一種你整個性靈在你生命裡面的生命.
所以所謂的靈命就是性靈在你生命裡面的一切.
如果你用這種看法來理解靈命的時候.
靈命就不是在你靈修那一刻.
靈命也不是在你上教會或者有沒有奉獻那一刻.
而是你整個的生命.
整個性靈在你生命裡面那種彰顯.
我以前也寫書寫過.
靈命是在你吸了那本靈修書之後開始.
你早上咬著三文治搭地鐵.
靈修之後.
你吸了那本聖經開始上班的時候.
那一刻正是你靈命的開始.
你怎麼做一個老闆.
你怎麼做一個員工.
你怎麼做一個女婿.
你怎麼做一個媳婦.
這些位置全部都是你靈命的彰顯.
所以這個就是我早期的靈命觀.
整個人的生命就是我的靈命.
我的靈命很壞.
想想一個很屬靈的人.
但是他很不環保.

$^{1401}$很奇怪.
他是一個很屬靈的人.
但是他發脾氣.
你覺得是不對勁的一件事.
因為所謂靈命好.
靈命好就是人品自然好.
你整個人.
你的人是好的應該會.
他不是經常祈禱過他.
但是他發脾氣.
他很玻璃心.
這樣就叫靈命不好.
因為不是一部分.
不是一個宗教的一部分.
而是你整個人的生命.
你怎麼去做一個好的下屬.
怎麼能夠盡忠職守在你生命里.
你見到前面那個人過閘.
你都不發脾氣罵他.
這些都是屬靈生命很重要的彰顯.
我經常說.
所謂屬靈九果.
怎麼能夠得到忍耐的果子.
方法不是靈修.
不是靈修這麼簡單.
而是真實地.
請你忍耐一下.
真實地有忍耐的生活和行動.
所以靈修或者你的宗教生活.
是離不開你整個的生命.
這是我回來香港之前的靈修觀.
整個人的生命就是靈修.
所以我開始沒有所謂的靈修.
其實不是不靈修.
而是將靈修融在我的生活里.
你發覺那些奴魂神父都是一樣的.
奴魂神父教的是什麼.
最後的究竟是什麼.
在修道院裡掃地都是靈修.
其實是一樣的道理.

$^{1441}$因為發覺靈修不是某種神聖時間.
而是你整個人生命的擴展.
如果你要靠你每天十五分鐘都沒有的靈修時間.
作為你靈修的量度單位.
你肯定靈命不好.
因為你最多是什麼意思.
24小時減15分鐘之後全部都沒有靈修.
就算我們這些傳道人.
有了聖經也好.
其實都是有限的時間.
你只能將你的靈命擴闊到整個生命里.
所以我就要說.
今天你聽不明白就算了.
這個活例用了很多年.
因為龍珠第三十三期裡面.
其中一個令我覺得很有啟發性的一幅圖畫.
如果你有看的話.
明白不明白就算了.
我不解釋那麼多了.
第一期就講到.
悟空和悟凡就是一個時間仿走出來的.
你會發現他們兩個人有什麼不同.
你會發現他們將超級殺人的狀態.
變成一個恆常的狀態.
如果你看龍珠就知道.
以前龍珠要爆炸就突然很厲害.
他說不要了.
我們將超級殺人的狀態變成一個恆常的狀態.
不是我要爆炸那一下才爆炸.
而是成為了生活的習慣.
所以他當時的想法.
他的比喻就類似這樣.
我們將上帝和上帝的關係.
不要局限在我們所謂的二道自建.
還是方法金船的靈修.
那十五分鐘都沒有的時間裡面.
而是你的靈明就是你整個的生活.
整個的生命的開始.
所以我們今天很想大家.
起碼這個我都是全職的教會來理解.

$^{1481}$我們如何理解我們靈明的其中一半部分.
這個一半.
因為後來我都有自己的反省.
我們的生命就是我們整個的宿命.
如何來在生活.
生命裡面來彰顯靈明.
才是關鍵.
不是那十五分鐘的靈修.
也不是那一個禮拜裡面.
一個小時回來的靈明.
這些是重要的.
但整個的靈明.
不單單在這些特定的時間裡面.
然後我們就說一點點.
深的那麼多的東西.
就是一種叫做關係性定論.
當我們去理解靈明的時候.
我們就要去理解性靈是什麼.
你要那麼多的深入的識別.
來認識什麼叫做性靈.
性靈是一些很不容易說的東西.
因為性靈是不能夠去摸到.
也不能夠那麼容易去論述它.
靈因派的好處就是將性靈的工作.
很具體化下去.
很容易明白.
性靈在你旁邊.
性靈感動我.
叫你怎樣怎樣怎樣.
什麼靈什麼區域邪靈.
怎樣搞到我們.
這就是將性靈的東西.
變成一些很具體化.
好處是很具體化.
但就很簡單.
太簡單.
我們不是這樣去看性靈.
性靈是一種我稱為關係的性靈.
性靈其實發覺.
性裡面每一段有關性靈.

$^{1521}$或上帝的靈經文.
其實都不是獨立出現的.
它總是會和性父或者性子.
或者和我們之間有關係.
因為性靈稱之為一個關係的建立者.
這裡我跳過了一些深入的東西.
不說了.
性靈可以是父和子之間的關係.
想想.
上帝是性靈感人的新耶穌.
耶穌洗禮的時候.
性靈就像甲子一樣降下.
性靈成為了.
上帝就拆遍性靈.
叫基督耶穌復活等等.
任何父和子的關係.
其實都是在性靈裡面的關係.
這種性靈關係正是很重要的.
而你發覺.
性靈裡面任何基督和我們的關係.
其實都在性靈裡面.
若不借助性靈.
就沒有人能夠稱耶穌為主.
所謂性靈其實就是基督耶穌的靈.
在我們的生命裡面等等.
性父也一樣.
天父上帝和世界的關係.
都是在靈裡面的關係.
而不是在靈裡面.
這裡我可以詳細地說.
上課時就不說這麼多.
總之性靈就是一種.
我們和天父.
或者父和子之間的關係.
他是任何關係的建立者.
包括人和人之間的關係都一樣.
我們同感一靈等等.
所以說來.
性靈就不是一些.
獨立於任何東西的一塊.

$^{1561}$我們嘗試不用靈因派的語言來說.
有時候都會問人.
你覺得性靈充滿更厲害.
還是跟隨基督更厲害.
誰說高級一點.
其實沒法想.
哪有比較.
因為性靈充滿就跟隨基督.
很難想象一個.
跟隨基督但不是性靈充滿.
很難想象他性靈充滿.
但不是跟隨耶穌.
所以我們嘗試將這些.
靈因派的很簡單的性靈語言.
變成一些能夠具體化的.
你不要問你自己有沒有性靈充滿.
這不是一種感覺那麼簡單.
問你有沒有跟隨耶穌.
我們是否效法基督.
還是說有一個所謂的成性靈.
其實是一樣的.
性靈叫你成性但同時也叫你效法耶穌.
一個客觀的毀身的行動.
正正是跟你主觀的.
那種屬靈經驗.
其實是分不開的.
就是說.
我再說一句.
你很難自我陶醉.
一個屬靈美滿的狀態.
而你很恨不恨.
你可以很愛主但你很恨不恨.
我不接受你這一套.
如果你是和神關係好的話.
我都說了.
靈命好人品自然好.
所以你一些很客觀的東西.
成為了我們能夠量度靈明的東西.
多於一些很簡單的屬靈觀念的語言.
所以我想說的是.

$^{1601}$今天我們去嘗試去說靈明的時候.
其實一些很具體的東西.
不是問你靈修多少次.
靈修多少次肯定會說什麼.
有什麼意義.
但靈修的意思並不是這樣能夠量度到.
可能你以前回教會的傳道.
問你這個禮拜有多少次靈修.
這個好像能夠將事情量化了.
但你究竟的生命行為是怎麼樣.
你個人的狀態是怎麼樣.
這個正是你的靈明的重要一點.
而更加重要的就是.
今天開始我人生後半段的靈明觀念.
我稱之為生命的靈.
這個在羅馬書第八章第二次裡面有.
本來說是生命的靈.
但生命其實不單單是一種教會神聖的靈.
它更加像我們的生命.
上帝的靈在我們的生命裡面.
我們就有氣息.
所以這個靈正是我們同一位的生命.
生命是叫我們有生命靈.
我們其中一本很重要的書.
我在美國那年翻譯的那本莫特曼的書.
就叫做生命的上帝.
聖經裡面有很多次數說到.
耶和華是永活的神.
Living God.
永活神不單單說他有永生或者永遠的生命.
而是一種生命的上帝.
因為上帝是生命的上帝.
而我們好像一個魚鹿.
學無溪水的奶奶.
一個飢渴的小鹿.
是學無上帝的生命靈.
所以我們的生命正是需要去追求這個Living God.
我們飢渴這位生命的上帝.
耶穌基督是叫人得生命.
並且得更豐盛.

$^{1641}$所以我們那種靈命.
其實不是計你有沒有躲起來靈修多少次.
而是你的生命.
如何來經歷很多不同生命裡面的事情.
我不知道你的靈命.
有些Milestone是如何去理解自己的靈命.
可能有些人的靈命.
可能是在一次車禍裡面出現.
然後轉變他的生命.
有些人在癌症裡面突然有很大的反省.
有些人寫過翻山書.
就有很多不同的東西.
有些人失戀.
然後突然發現上帝很需要等等.
我們的靈命永遠在我們的生命.
這些關口位裡面顯得重要.
因為我們的生命本身.
就是我們整個靈命一個很重要的場所.
所以我們嘗試去.
如果你說性靈是生命的靈的時候.
我們嘗試用一種.
另一種說法.
不是單單叫Spirituality.
而是叫Vitality.
就是我們的生命力.
你如何去面對你自己的人生.
面對現在移民的一個關口.
面對家裡人的疾病.
或者男女朋友之間的吵架.
或者夫妻之間的關係等等.
這些生命的那種量度和力度.
或者承受痛苦的力度.
正是我們屬靈生命裡面很重要的東西.
這些就很道地了.
這些就是非常具體的東西.
靈命不是一些你靈修次數.
或者你和神的關係是怎麼樣.
而是很具體地.
你和神的關係就在這個生命裡面.
如何面對這次的困難.

$^{1681}$這次的逆境.
這種生命的那種油韌力.
這種力量.
能夠抵抗很多傷口或者傷痛等等.
正是我們靈命的一種向度.
所以說亂世的靈性修持.
基本上我們呼出這大半年里.
其實都是在說類似的東西.
只不過是將這些的向度.
變成不同的主題裡面.
面對著這樣的時代.
我們的靈命是怎麼樣呢.
你以為你能夠靈修就沒事嗎.
你以為你靈得幾次修.
就可以面對今天的困難嗎.
或者當中上帝會和你說話.
但更重要的是.
你怎麼面對這個社會.
這種氣氛.
這種本身就是你的靈命.
生命靈.
正是你的性靈在你生命裡面.
怎麼能夠面對著你每一天.
每一刻裡面所面對的事情.
所以我們說.
當然如果幾年前我們在運動的期間裡面.
有很多不同的靈性操練.
其實都一樣.
就是你怎麼在生命裡面.
你在街上.
在中環.
在金鐘裡面.
怎麼能夠彰顯你的靈性.
到今天沒有了.
全部都沒有了.
但你的生命力仍然是面對著這個局勢.
只不過方法不同了.
可能會變成休閒.
可能變成減肥.
可能會變成減躁.

$^{1721}$等等不同的方面.
但其實都是一樣的.
就是你的生命.
怎麼在性靈裡面.
能夠面對這些一個一個的困難.
所以具體來說.
這種生命的靈性.
我們強調我們怎麼藉著性靈.
基督耶穌自己.
來面對著我們的生命.
耶穌也說過.
祂來叫人得生命.
豐盛生命.
這正正就是我們靈命的意思.
所以我們說.
我們嘗試去打破身體和靈魂的異原.
靈修不是只是靜.
閉上眼睛去大自然.
這些東西.
而是你整個的身體.
包括你自己的人是怎麼樣.
不是只是處理著屬靈的部分.
而是你整個人.
因為身體和靈魂是分不開的.
我們就不只是去陪一些屬靈部分的自己.
而是整個人.
你有沒有熬夜.
吃得好不好.
還有沒有胃痛.
這些全部都是屬靈生命的東西.
不要把它分開.
當然受苦主義是有意思的.
所以我說減肥.
這是另一種意思.
但其實是兩個不同的向度.
熱愛生命.
自己生命和別人的生命.
對於生命的熱愛.
對於人的熱愛.
正是我們這種熱愛.

$^{1761}$是一個很重要的素質.
另外就是.
精神學的感覺.
我們嘗試打開我們的五感.
來經歷這種屬靈部分.
也就是說.
不只是閉上眼睛.
我們的屬靈部分.
我們的禱告也不是只是閉上眼睛.
而是嘗試張開你的五感.
嘗試去看多一點.
聽多一點.
接受多一點.
思考多一點這個世界.
你這個靈性.
其實是和世界建立.
一種連結的方法.
所以靈修.
大家可能會來長洲.
因為長洲對你來說是一個新的地方.
但對我來說.
我不是在長洲靈修.
因為長洲是一個我自己住的地方.
可能我會去旺角靈修.
我們如何能夠連結這個世界.
如何能夠借助一種新的五感的衝擊.
嘗試重新去認識天賦的世界等等.
都是一些我們可以思考的課題.
這些全部都可以說很多東西的.
我今天很簡單的版本.
所以我們的生命.
我們的生活.
正正就是我們屬靈的場所.
Live energy.
可以翻譯做生命.
生命就是你生活.
所以你的靈性.
就是你如何生活.
你的生活.
你的壓力.

$^{1801}$你如何調節自己的生命節奏.
有沒有休閒等等.
這些全部都是我們靈性裡面.
可以經歷的課題.
最後我想說一點.
最後一些具體的修持部分.
我說了.
七年前我回來香港的時候.
我就說我的生活就是靈修.
但我發現神秘主義者.
我告訴他們.
我們在當中仍然是值得.
有一些特別的moment.
而這些moment正正是我們的洞見.
我們打破我們恆常裡面的生活.
突然你靜一靜下來.
去反思自己的生命.
這一下正正就是靈修的本意.
五百年前伊納爵.
他將一些很難跟的神秘主義的物觀.
變成一些很規範化的物觀主題.
譬如你寫一本書叫《神操》.
不是跑馬那些.
而是special exercise.
他說今天你就物想謙卑.
謙卑就是你今天物想的重心.
方法不重要.
誰寫的書都不重要.
而是你真的去想東西那一下才重要.
所以經常都覺得靈修是一種慰藏.
你有沒有能夠在你生活里抽出來.
能夠反省一下自己.
讓上帝的說話.
讓上帝那一下的安靜.
讓你能夠重新抽離一下.
從而去反省自己的生命.
這一下正正是靈修的重點.
所以不知道弟兄有沒有這樣的moment.
有時候有露台.
突然夜晚吃完飯.

$^{1841}$拿著一罐啤酒在露台裡邊.
吹吹風喝一口啤酒.
呃一下出來之後.
那一下的反省是很relaxing.
不是relaxing.
是將你那個生命抽出來.
那一下反省很重要.
所以這一下是重要的.
這一下才是靈修的本意.
相反那本看靈修書的方法.
不是不好.
因為總是有一些隱瞞方法.
不是每個人都能夠這麼容易反省自己.
看靈修書好像能夠幫助你反省.
但是過度依賴它的時候.
如果你能夠不靠它的話.
想東西是很重要的.
所以我想說.
找一個你自己合適的方法.
能夠帶給你自己有生命的refresh.
和那種rethink.
正正是我們靈修的重點.
如果你不是看書的人.
如果你不是理解很好的那些人.
不妨.
所以一定要跟那些.
要自建的那些.
你可以試一下.
用另一個方法.
能夠重新讓自己能夠articulate到生命.
為什麼人們這麼流行畫畫靈修.
還是書法靈修.
我自己也喜歡書法靈修.
我曾經一段時間也練書法.
就是覺得這一下是能夠讓你用一些不能.
因為都說了.
語言是一種很難articulate到上帝.
Art藝術是能夠幫我們articulate到.
自己和上帝的關係和感受.
這種的自省.

$^{1881}$可能是唱歌.
可能是畫畫.
可能是做手工等等.
可以成為你靈修的部分.
如果你是art的人.
如果你是運動人.
跑步.
游泳.
可能是一種方法.
只要你的生命能夠給你一種.
稱之為新.
一種新的衝擊的時候.
如果你發覺你靈修了二十年.
都沒有什麼起色.
不妨轉一種方法.
不用一定要跟那些靈修東西.
嘗試去找一種能夠和上帝親近.
一種你能夠articulate到和上帝說話的方法.
當然,就算是動有動的靈性.
靜有靜的靈性.
整個的重點就是你能夠給自己一個反省的時候.
今天香港人其實是很不懂得反省.
很不懂得能夠抽出來.
每一天去反省一下自己.
而我在中國大陸就太多反省了.
是過多.
我們起碼有一段時間能夠從.
當你是那種多層線感覺就會發覺太過反省.
想太多,想多了.
沒辦法想那麼多.
但你平時一個普通的頂智妹.
可能很忙的時候.
找一個給自己反省的moment.
這一下正正就是我們的生命開始.
重點不是你靈得多就自然好.
而是你那一下的refreshment.
那一下的更新正正就是我們每一次的意義.
所以不需要去問我這次有沒有得著.
還是有沒有學到什麼.
現在靈修不是學東西.

$^{1921}$而是純粹的去沈澱反省自己的生命.
想的未必都一定是你自己所謂的屬靈的topic.
純粹是你生命裡面遇見的topic.
不過在上帝裡面.
重新來重整自己的生命.
我們先講到這裡.
我們有很多問題可以和大家談.
特別是一些很具體的屬靈的情況.
可以和大家談談.
我們先談到這裡.
餵.
這杯酒今天很棒.
當然了.
剛才看你口渴.
剛才你說到有些地方.
和我做生意來說都很不同.
你說不要做那麼多.
但我們做餐廳最重要的是套餐.
單點人們覺得不值.
套餐會更值.
剛才你說不要做那麼多.
最重要是專一點適合自己.
這件事怎麼告訴別人.
不要做那麼多.
而單點會更加有得著.
不是不要做那麼多.
而是不要以為自己做得多.
就一定是好.
不是來得多就叫好.
而是吃得對才對.
自來也沒有什麼喜悅.
沒有什麼快樂.
不如找另一間.
我覺得這些都可以.
其實大家怎麼知道多或不多呢.
大家可以談談.
你做得多或少.
可能你的少對他來說也很多.
不過這麼說.
我很怕外面的人.

$^{1961}$陳茵安叫別人不要零收.
其實不關事.
因為大家很著重那些優先.
那就麻煩了.
你一年看不看完整本聖經.
夫子今周要多少次零收才對.
不要問這些.
重點不在這裡.
零收一個星期.
但每次看完就算了.
很貼合式的.
其實也沒有什麼意思.
大家怎麼看呢.
大家有什麼靈命上的奇難雜症.
可以和大家談談.
或者分享一下.
平時你怎麼零收.
或者你零收不到的原因.
這可能共鳴大點.
幫朋友問也可以.
因為這些事情大家都會有機會.
也會經歷.
其實零收可以很開放.
可以任何時候任何地方.
你做的事什麼都可以.
只不過是同神掛勾.
去關心神裡面.
或者教訓.
或者神的話.
甚至他喜不喜悅這件事.
他怎麼看.
其實是不是這樣呢.
我想回應一下.
廣義來說就是我回香港早期的看法.
沒錯.
你的生命就是你的靈命.
是一樣東西.
你怎麼做人.
怎麼做你的靈修.
或者你的靈性.

$^{2001}$是一樣東西.
你不能分開.
你最怕我零收就叫零元.
其他東西就好像分割.
應該融合在一起.
但是那時候又不代表.
都是這樣做.
那是不是真的不需要零收呢.
那個particular是重要的.
每一天.
為什麼人們不來長洲.
因為你特別特別.
我特意來這裡.
還有些東西.
所謂神聖之地.
或者神聖的時間.
這種東西是重要的.
因為它是在打破你的恆常.
從而令到你可以.
對你的恆常去作出一些反省.
所以兩樣東西加在一起.
就是我們所說的.
所謂的零收.
或者屬靈操練.
因為屬靈操練.
有一種我們基督教.
新教的.
其實是兩頭不到岸.
天主教最強.
就是因為我們說的那種.
沒有停過的.
那種contemplative.
500年前的一幫reformer.
他們重新delete了所有的東西.
而我們開始慢慢去發展.
一些清教徒式的零收.
就是一種discipline.
discipline是好的.
因為habit is power.
你的習慣是一種力量.

$^{2041}$所以你繼續做某些東西.
是有一定程度的力量.
但還教會就將這種.
habitual power.
這種習慣變成了一種很律法式的東西.
我們以為這樣跟著做.
就是了.
其實又不可以錯的.
你教一個小朋友.
當然叫他每天都溫書.
但溫書本身就不是.
你學習的東西.
是動見的.
是insight.
零收是整個生命.
但有時候我們那一刻的insight.
或者那種reflection.
那種反省.
是一種particular的時間.
我自己推翻自己.
七年前所講.
不是生命就是零收.
就不需要零收.
所謂零收或者那種particular的時間.
是有它的重要性.
以前的神秘主義者.
就是這樣.
親近上帝之類.
這個就是另一個topic.
因為講一些本體問題.
我們的靈魂.
如何跟上帝相遇.
這些很不同的東西.
基督教就不講這些.
我們是聚人.
所以就不講這些靈魂.
如何去上帝面前契合.
我們用聖靈來articulate這件事.
但不是純屬一種行為.
不是純屬清教徒這樣做.

$^{2081}$就算有了.
大概兩部分.
或者通常零收的困難是什麼呢?.
大家認為是什麼呢?.
沒有什麼特別困難.
沒有什麼特別零收.
後面.
因為剛才說anytime anywhere.
好像有不同形式都可以零收.
其實是不是不一定有讀經這件事在裡面.
就算我有默想到神.
因為傳統教會教的.
其實是零收.
讀經是一個foundation.
開始.
然後透過讀經去默想神的話語.
有些應用.
我覺得因為我們.
用聖經是最穩定的.
用聖經就不用說不會錯.
但我想說也不是一定.
昨天我說不要叫人用聖經零收.
不是這個意思.
我們大概是這樣.
上帝跟你說話.
當然有很多不同方式.
你以前有沒有試過通聖識.
揭開那一頁.
上帝和你看什麼.
為什麼不行.
你上帝和你看什麼.
為什麼不行.
你上帝和你看什麼.
為什麼不行.
當然可以.
我不反對這一套.
你用通聖識揭開.
今天上神和我說什麼.
你不看吧.
當然不好.

$^{2121}$上帝有很多不同方式和我們說話.
詩歌,反省,聖經.
甚至你和朋友聊天之後有些反省也是.
當然你是不是這樣就叫零收呢.
我都說反過來.
你的生命本來就是零收.
所以你不需要問那些是不是零收.
你本身整個人的生命就是上帝的零和你一起的方法.
問題是長遠來說你用什麼方法.
當然長遠來說是不是每次都沒有聖經.
當然這個不建議.
但是你說是不是一定要有.
沒有就不叫零收呢.
我就不是這樣看.
大家平時會怎麼零收.
有什麼想不通的地方.
最難在哪裡拍.
我接下去剛才John說的關於聖經的參與.
因為剛才有一個powerpoint是關於關係性的正靈的部分.
其實很多時候零收的時候我們都想尋求上帝的介入和指引.
但是很多時候都會和弟子妹妹去討論.
你有多敏感聖靈的提醒.
這就牽扯到你和聖靈的關係.
聖靈幫我們做分辨.
聖靈提醒我們.
很多時候過去接觸弟兄姐妹.
問如何明白上帝的旨意.
不外乎都是兩至三個處境.
一個處境就是我應不應該轉供.
第二個就是我應不應該和他開始拍拖.
第三個就是應不應該有比人生重要的決定.
那時候就尋求聖經的指引.
那時候就會遇到一個困難.
就是聖經說的一些東西不關於聖經.
但是在關係建立或者敏感度的時候.
這個才是你靈明要反映的.
對於我們來說.
平時聖經說什麼.
讓我們認識上帝是一個什麼上帝.
上帝想期望我們的人生.

$^{2161}$或者我們的信仰表達是什麼.
但是如果你平時沒有聖經的支持.
或者瞭解聖經對我們的提示的時候.
有什麼引領你去做分辨呢.
這個就是一個問號.
所以我覺得更加好就是.
把讀經習慣和靈修分開.
你覺得讀經是加起來的.
不是說靈修讀經.
讀經就是讀經.
讀經就是看多一點聖經.
看熟一點 想想.
但是靈修真的不是一種那麼量化的東西.
反而是有一種生命的反省.
和上帝反省自己和世界之間三者的關係.
這個比你學習一些東西還要重要.
還有問題嗎.
我想兩位可不可以分享一下.
香港人現在.
我自己在靈修上有困難.
每天都很忙 很趕.
其實靈修是一種靜態的東西.
需要靜下來.
我覺得是一個靜態的東西才能做到.
兩位有沒有分享有什麼方法.
讓香港人多一種途徑去靈修.
除了說.
唯一可以靜下來的就是.
每個星期的主日崇拜.
最長的時間靜下來.
其餘的時間.
其實都有個難度.
是嗎 比如說晚上臨睡前.
那些是祈禱時間.
不夠 而且你又會很累.
我覺得祈禱和靈修是不夠的.
祈禱時期都可以睡著.
說到底都是不夠時間.
太累 太忙.
整天困在這個狀態里.

$^{2201}$我覺得兩位是傳道人.
一個是神學院教授.
我覺得你們的時間是很足夠的.
說到底是時間.
時間這東西.
我經常落在這個光景里.
有時候很難.
比如說退休.
是很正的一件事.
起碼要兩三天.
你先要起來.
然後你要靜下來.
才能進入那個狀態.
但當你覺得時間足夠.
你又回到一個現實的世界里.
周而復始都是往返.
剛才John說到伊臘教的精神訓練.
數年初年.
我教基督教里的靈命培育課程.
都會用伊臘教的精神訓練.
其實當中最後的教學點.
其實都是和現代教學方法.
五感教育相近.
不說得太複雜.
其實人有不同的學習方式.
一個是視覺.
一個是聽.
第三個是做.
有些人是文字敏感.
看,預度理解會快一點.
有些人看不太快.
但聽人說的時候.
就馬上上腦.
就已經有圖案在腦子里.
有些人聽和看都不敏感.
你告訴他怎麼做.
他做完一次.
他就會變通.
其實要找到自己的方法接收.
這個是基本單元.

$^{2241}$所以剛才說的.
嘗試去尋找自己一個可以接收.
或者對你來說是互通的方法.
第二件事就是牽涉到時間的參與.
我從事教育.
就是你找到一個方法之後.
就開始要花時間.
每個訓練是要用練習.
所以你會明白.
一個叫做練習讓它完美的意思.
就是你不嘗試不斷去打磨自己的方法.
這樣的話很多時候你都不會靈活.
你牽涉到多少時間.
有些人用短時間就能夠做到.
有些人用長時間就能做到.
我自己的做法就是.
我自己視覺和聽都是敏感的.
因為有時看的時間就限制了.
剛才說要靜的時候才看得快和看得好.
我大部分時間都在街上.
或者要去不同的地方.
我會選擇走路的.
我每天走路的時間超過一個多小時.
有時我不坐車.
選擇走回家.
或者其他事情.
我走的時候就會聽我想聽的東西.
譬如聽podcast.
聽廣東話的聖經.
或者聽一些我覺得有質素的文章.
或者是一些講道的時候.
那就是我自己的靈修時間.
不一定一次過走一個半小時.
可能有時我回家要洗碗.
我都會在洗碗的15分鐘里聽一個podcast.
那個podcast對我來說是一些信息.
我有興趣想融入的.
就將一個regular或者particular的時間.
就斬件在不同的地方.
去融入我收集的或者我靈修的方式里.

$^{2281}$讓自己有調整的空間.
這個都是融會了.
就是選擇一個你自己接收的舒服和比較efficient的渠道.
再將一些時間散在不同的地方.
讓那件事成為你生活的一部分.
所以這個都是我和John相近的靈修方法.
還有我都覺得回到我們講第一部分.
就是你整個生命就是你的靈性.
這種觀念都是很重要的.
如果是那些趁你工作以外或者很忙以外.
就肯定是少的.
肯定少於傳道人.
少於教會的人.
但其實我們要的最基本的方法就是.
你不要覺得那些東西不是屬靈的東西.
我經常都寫不完那本書.
那本書的名字就想到了.
就是張開眼睛去禱告.
其實你打開眼睛的那個世界.
其實就是你禱告的世界.
不是你閉上眼睛那一刻才是天父那一刻的世界.
你面對現在.
面對你的工作.
那個辦公室.
那些人事.
或者很日常的那些excel那些東西.
那裡其實都是上帝在的.
所以我們需要練的就是練你那個生命的動策.
那裡都看到一些反省的位置.
都看到人與人之間有些不同.
那些未必只是看到屬靈的東西.
但那些其實都是屬靈的東西.
就是人與人之間的關係.
或者是一些不同的reflection.
所以重點都是.
你與其你squeeze時間.
不如你學習一下在這些位置裡面去洞察一下上帝.
當然這一刻是不容易的.
我們覺得總是帶著自然是容易一點發現上帝的.
但我們會正正就是在一些.

$^{2321}$外表不像有上帝的地方去找上帝.
這一刻是對城市人來說是重要的.
如果你不是你怎麼辦.
你這麼多年在城市裡面生活.
如果你去瑞士的話.
瑞士真的那些半山那些.
海底那些.
那些真的很好.
在那裡長大的人.
以前我在歐洲的時候.
每年去一次瑞士開會.
那裡真的那些藍天白雲.
有些很漂亮的草地.
那些蒼蠅都慢一點.
那些蒼蠅可以被你拍死.
那些人不習慣那些蒼蠅不會被人殺死.
那些地方很棒.
但我們就需要用另一個方法去找上帝.
上帝就不一定在那些地方裡面出現.
在你的小廚房裡.
我以前就覺得.
我很喜歡去信和那些地方.
信和也能找上帝.
你能夠在那裡.
很多人很稠密的地方去找上帝.
這一刻是我們去操練的地方.
網上也有些問題.
我們就抽了一條問.
就是Cliff Tom 問.
如何判斷在行動中的他正在靈修.
進入與神同在.
例如當他在跑步的時候.
聽到靈修經文或者祈隱.
其實他有時候會覺得他在跑步.
而祈隱是在走路.
在浪費自己.
令他自己沒有覺得那麼辛苦.
這個是牽涉到一個很大的範疇.
我們叫專屬.
專屬其實我也想找機會講道講.

$^{2361}$專屬是一個很重要的靈性或者做人.
所以跑步的時候專注在那裡.
反而我覺得是容易的.
跑步沒有東西做.
反而你繼續去繼續.
按照靈修歷史來說.
以前的沙漠教父都是這樣.
他們以前不住的禱告.
不住的禱告就是不斷重復一段的禱文.
就是這樣說.
所以有這些傳統.
所以跑步的時候可以不斷重復一些禱文.
其實也可以.
重點不是在禱文那裡.
而是你整個人的專注在某個上帝那裡.
我以前剛剛初信的時候.
那時候真的很勤奮.
平時都在想上帝.
開會都在想上帝.
那就不行了.
你會分心.
但意思不是這樣.
這個是要粗連的.
就是理解你日常的生活是一個天賦世界.
你去理解.
你幻想一下耶穌就在你前面.
當然這個是要練的.
但我們覺得就是說.
你生活裡面能夠變成你能夠專注的熟能生活.
你有沒有補充.
這個問題是很大的一個課題.
從教育來說.
米哈利有一個教育理論叫flow.
中文叫做心流.
在台灣翻譯是2020年2月初版.
很厚的一本書.
其實flow的過程是什麼.
不是flowchurch的flow.
是真的flow.
時間過得很快.

$^{2401}$他的感覺就是.
當你學習一件事.
專注過程當中.
時間過得很快.
而你不覺得那件事是受時間限制.
而你是投入在其中的.
就像我講到的時候.
講過進入多個zone.
你能夠掌握那個方式.
去讓自己更大的吸收.
我自己都享受跑步的時候.
聽討論或者聽東西的時候.
不知不覺就這樣跑了幾k.
因為是浪費了.
你覺得沒那麼辛苦.
但你就投入了在那個經文或者信息思考裡面.
因為你進入了那個心流的狀態.
所以重點是.
你能夠找到一件事可以專注.
你可能覺得不專注.
其實你慢慢習慣了那個方式的時候.
你會投入那個專注.
是不是還有網上的問題.
第一排都有一個.
第一排.
我是真的幫人問.
因為有個大兄想問.
他也在看直播.
他說想瞭解一下.
就靈修方面如何分辨出思考的想法.
並不是自我的想法.
而是主跟我說的對話.
他有個例子.
就是當我走路靈修的時候.
忽然想到一些東西.
思考到一些東西.
怎樣才知道這件事不是他自己的想法.
是上帝在跟我說呢.
所以為什麼我會說這個關係性靈論.
因為如果按照性靈論來看.

$^{2441}$性靈做的事情不是在我們之外.
其實性靈是雙方的.
它是建立在兩邊的.
我們說靈修和基督耶穌的關係.
就是在性靈裡面.
它是連結我們和耶穌基督自己.
所以我們就不像靈因派.
靈因派是什麼.
就是性靈要激怒我就要跌下來.
就好像我沒得選擇.
性靈其實是一些我們和上帝之間的關係的時候.
當你有那種想法的時候.
其實這件事本身不是單單出於你.
怎樣說呢.
性靈感動你的話.
你不會覺得很不滿.
你會成為你的想法.
明白嗎.
所以性靈不是完全是第三者.
它是我們的一個subjective god.
就是它讓我們能夠有這樣的感受和想法.
所以當然你敬畏地去想.
這個是不是上帝的想法呢.
這個是好的.
但其實上帝讓你這樣想.
這個不是偶然的事.
只要你懷著一種尋求討告的心.
來開放給上帝去求問.
這些所謂自己想法的事.
其實就是經過時間引證.
這些都是上帝的想法.
這些問題是不是讀神學.
是不是蒙照.
一樣的.
上帝讓你這樣想.
不要覺得這些事.
是突然從天而降.
聽到聲音對我說.
很多時候我們這樣想.
其實是必然的過程.

$^{2481}$因為上帝感動你.
你必然會想到一些事.
所以這些是很個別的.
我會再和他談談那些情況.
我會很多時候和他再討論.
那個是不是他自己想法的時候.
我會回到《約翰·加爾文》.
基督教義義卷一.
第一章第一段落.
裡面說的是.
那個想法或者那個瞭解.
有沒有認識上帝多些.
有沒有認識自己多些.
有沒有認識神和人之間的關係.
應該不會有衝突.
而那樣東西應該會豐富你.
對上帝或者對你.
剛才John說到.
在那個生活處境當中.
應該是參與在其中.
我仍然相信那句話.
就是聖靈一直都看著那件事.
不是偶發性的.
仍然不斷地發生在我們生命當中.
我都很少說那些.
我不是這樣說的.
聖靈神叫我這樣說的.
這些我最怕.
叫你付出.
我沒試過這樣說.
神就叫我怎樣怎樣.
因為其實中間我們去翻譯.
你這樣的想法.
其實帶你都有祈禱的感覺.
其實都是神讓我這樣想的.
很少說這個完全是屬於原始的他者.
我是完全不想的.
其實神叫我來的.
或者我不會這樣說.
網上是不是也有問題.

$^{2521}$來自Cheng Tech Team的留言.
各位有事目者都會問你的靈明有沒有成長.
想問靈明成長會不會和相信耶穌的連支有掛勾呢?.
例如你相信耶穌的時間是多久呢?.
但是上教會很少.
讀聖經很少.
又會不會和靈明成長有關係呢?.
或者上教會上得很長.
又會不會代表他的靈明比其他人高呢?.
所以有事目者問這個問題就覺得很大疑惑.
我就覺得不是的.
我覺得基本上一方面不是.
意思就是說他沒有關係.
不過人做了很久.
其實會有進步.
這樣也是.
很複雜的那件事.
這個很大題.
花了很多時間很多元素.
我都說不是做積分.
所以不是說做基督徒越久你的靈明就好了.
不過你做基督徒越久.
你又確實可以好了.
因為你發覺和基督的關係深厚了.
你又會有更多的經驗等等.
但是否必然呢?當然不是.
很多人越做越差.
我昨天才發了個Facebook post.
不是這樣的.
很多人初信是很好.
但是你做人做久了.
其實你會有很多人生經驗.
很多體會也會多了.
你學的東西也會多了.
但是還是那句.
看你人有沒有反省.
你個人是有反省的話.
那你就會突然浪子回頭.
那你那一刻就會突然好了.
但浪子回家後會不會比第二次好呢?.

$^{2561}$不一定.
所以我覺得沒有直接關係.
人生就是這樣.
人生裡面有很多這些.
你可以走錯路.
和上帝的關係一樣.
很多時候初信的時候.
多數是all nothing.
要麼就有在基督裡面.
要麼就不在基督裡面.
那些可能是你的聖經知識加深.
可能是你禱告多了經文.
花款更多.
或者是你熟悉語言多了.
或者是侍奉技巧好了.
或者人生的成苦心了.
但是是不是靈明好了呢?.
我覺得不一定是.
我嘗試用一個例子.
看看大家會不會應對到這個情況.
有健身生會籍這個問題.
有很多人有參與健身院的會籍.
是不是去多健身室就會有好的體態呢?.
其實你知道不是的.
有很多人經常去.
但是身體是沒有變的.
因為他在裡面沒有做事.
還有是不是懂得很多健身理論就會有的身體呢?.
都不是的.
因為他懂得但他不做.
相仿的情況就是.
回教會是否很靈明呢?.
都未必的.
是不是懂得很多查經書的資料.
或者讀經書代表很靈明呢?.
都未必的.
有時就是他做了一個口語.
所以我以往一段時間.
長期都在健身室.
都有很多靈修的提醒.

$^{2601}$有很多人都問我如何操馬甲線.
但我說你不要問我.
因為我沒有做過.
不如你找一個真的操馬甲線.
他告訴你.
如何熬過頭六個星期的熱身過程.
我就熬不了.
我操不了.
但你找一個操得到的人.
那個精神訓練同樣很重要.
不是時間坐得久就行.
在時間當中沒有工作是不會解決問題的.
所以不要迷信時間.
我們有一堂會講這個題目.
就是講成性的問題.
因為我們說.
Science of Certification.
就是成性的科學.
就是你借助靈修.
什麼靈性的增長.
就令你能夠越來越好.
所以叫成性的Science.
這個說法是真是假.
這個我們會再探討.
但我覺得這是一個成性的問題.
我們會如何越來越像耶穌呢.
這個問題是一個值得我們再花多一堂去思考的課題.
現場有沒有其他問題.
網上有沒有其他問題.
有一個來自Cliff Thomas的回應.
他說若然行動的目的是為了令人專注.
專注可以是看劇,工作,打球等等.
其實是否只要是專注的訓練.
而不是尋求神的同在呢.
還有他還有一個問題.
若然專注的方向是進入神同在.
除了鬆動狀況之外.
有沒有明顯的指標.
可以令他判斷自己是在靈修而不是專注呢.
其實剛才潘SIR說到流動.

$^{2641}$最容易流動的活動是什麼呢.
就是打遊戲.
打遊戲很容易流動.
因為視覺,聽覺都很專注在那一方面.
容易被吸引到那裡.
我覺得什麼叫專注上帝呢.
這個是很深的議題.
但我覺得當我們專注在某些東西的時候.
你都可以享受上帝某些東西.
當你理解整個生命是你的靈明的時候.
當你專注在上帝的世界的時候.
意思是上帝的世界不是俗世負面的世界.
這些東西都可以成為你去享受生命.
就是上帝給你生命的那種世界.
這就叫生命的靈性.
因為生命本身就是上帝賜予的.
所以這種方法其實都是一種去專注上帝的方法.
上帝,世界和我三個的關係.
而打遊戲是差一點的.
因為打遊戲和上帝的世界是比較遠一點的.
所以打遊戲你會發覺人是溫溫的.
然後又好像有點心虛.
但是當你專注在一些活動.
例如行山或者是世界上的工作.
你會感覺到上帝的多一點.
所以我覺得專注上帝可以是一些物觀.
純粹想上帝.
但是當你在生命裡面去注意的時候.
其實都是其中一個方式.
當然這不是唯一和全部.
但起碼你去跑步.
你去能夠在世界裡面來享受世界生命.
我覺得這是其中一種方法.
我不知道有沒有理解錯.
剛才聽回應問題的時候.
就會感覺到好像是怕在坊間的場景.
或者一些情況會好像是.
投入的時候就未必是很靈修的做法.
但是我自己的看法就未必是這樣的.
所以大前提就是.

$^{2681}$我相信這是天賦世界.
每一點都是上帝的管理範圍之內.
而我覺得從事基督教教育.
面對最大的困難就是.
過去教會的教導很多時候都有不同的規範.
例如一些經文的背誦.
或者一些形式的教導方式.
令到你可以進級.
但最大的問題就是.
在經文的背誦.
或者不同的課程進級的時候.
缺乏很多場景題.
場景題就是會友或者信徒.
遇到那個場景的時候.
他的反應,他的靈明.
如何可以支持他作出那個反應.
或者如何作出那個判斷.
甚至如何用聖經的原則去做好那件事.
其實是少了一些場景題去應對.
所以回到回應今天的內容.
我們希望靈明散居於不同的生活場景.
從而提升我們面對場景題的挑戰.
或者場景遇到困難的時候.
我們都會問.
如果我們有信仰的人.
有基督信仰的人.
我們如何應對那件事呢.
這個思考過程是重要的.
回應剛才John內容.
在一個亂世當中做一個反省.
就要想一想.
我自己是一個喜歡想場景題的人.
譬如見到有人撞車.
我都會去看看.
那裡有人吵架,我就會去聽.
有時過程當中.
如果我真的面對這樣的情況.
我應該怎麼辦呢.
信仰很多時候是很突發.
但如果能夠在當中有個先驗.

$^{2721}$或者先理解一個處境的時候.
其實會幫助你去應對自己的情節.
有沒有其他問題.
在後面.
我想問一下.
怎麼說呢.
我看過這個Re-Order Church.
他們經常說.
在靈修放空很危險.
又說很多靈因派的弟兄姊妹.
可能是遇到鬼.
我就開始有點害怕.
其實我是一個很容易放空的人.
因為我很喜歡發呆.
但是當你一放空.
一去想.
好像內心有些聲音.
我就會害怕.
咦,是什麼呢.
我會有個疑問.
就是.
我現在會看聖經.
也會祈禱.
但是.
一說要靜下來.
自己去安靜獨處.
去放空自己的時候.
我就開始有點害怕.
會不會有些邪靈入侵之類的.
因為有時候.
你會看到一些靈因派的弟兄姊妹.
很虔誠很敬虔.
但是你又會看到一些靈因派的朋友.
不只是靈因派.
有些人會看到一些很奇怪的行為出現.
我又分不清.
所以.
這件事我想問一下.
會不會有些方法可以.
令到自己很肯定.

$^{2761}$這個是聖靈的.
就是我.
在獨處想的時候.
那個交談的對象.
不是一些奇奇怪怪的東西呢.
其實是這樣的.
就是.
如果.
你是一個人.
你叫陶.
然後你想去.
你的心,你的靈.
你的靈魂,你的靈.
去尋求上帝.
你去做一些事.
那.
怎麼可能會遇見邪靈呢.
因為你的心是想尋求上帝.
如果.
我不知道怎麼理解放空.
我都有.
我沒記得聽YouTube的那些東西.
但是我覺得.
我自己那個就.
我自己對於這個.
不是他.
就是放空就會遇到邪靈.
這種說法.
什麼叫空.
因為我們是有聖靈在我們心.
在《加泰書第四章》那裡說.
聖靈就在你心靈裡面.
就成為基督徒.
你天父的兒女.
怎麼會空到連那個都沒有呢.
所以我們是天父的兒女.
怎麼空都不會空.
反而.
反而我都有一點像.
太久沒靜.

$^{2801}$就突然一靜就.
一靜就靜不下來.
突然就想很多東西.
所以我就不覺得會那麼容易.
空到連聖靈都沒有.
或者你連同神的關都沒有.
你是天父的兒女.
你在這裡.
你的土告生命裡面.
都在依靠著他,都去找他.
怎麼會撥錯線去到邪靈那裡呢.
聖靈上帝不會那麼差.
不會讓你找到.
所以我就不太相信.
靈恩派的說法.
怎麼邪靈就怎麼搞我們.
我不是不相信他們的存在.
但我更加相信上帝的掌權.
所以我覺得就不需要怕.
當你.
當然是怕的.
但我經常都覺得.
你相信有神.
更加相信.
多過那些東西.
所以就真的不需要擔心.
怕放空空到地步會被邪靈.
這個我們以前回教會.
都聽過這樣說.
什麼瑜伽空到什麼.
我不覺得我們空到.
因為我們不是空在我們生命里.
我簡短的.
我都不是很接受到.
那個放空.
是什麼一回事.
因為.
你不是真的放空.
因為你總有些事情是做的.
我自己具體的說.

$^{2841}$那段時間你會有經文.
那時候你會有一些.
所謂的靈修資料.
去填滿.
你說你法外的話.
就真的好像剛才John.
說的那樣.
你法外的時候.
最後你都會有聖靈在你心裡.
這個是上帝的.
肯定來的.
還有他是.
沒有離開過當中.
我真的很相信.
就是.
那些鬼.
那些靈.
或者邪靈.
看到我們.
應該是他們怕我們.
而不是我們怕他們.
不過我們可能都會怕.
因為我們沒有試過.
那樣東西那麼厲害.
所以怕鬼.
很多時候都是被.
看了鬼片.
怕了.
所以.
我最記得.
我兩個兒子的時候.
我就問.
我第一件事問他們.
小時候.
你信不信有鬼.
信啊電視有的.
我說信鬼不是因為電視告訴你有鬼.
是因為聖經裡面說到有鬼.
那有鬼怕不怕.
我不怕.

$^{2881}$因為有耶穌在裡面.
我小時候.
他小時候我已經告訴你.
你怕是很多人都會的.
但是你記住.
耶穌在裡面.
聖靈在裡面.
這個很重要.
讓他們明白.
從小就教他們.
聖靈內在他們心裡.
這個是一個很重要的憑據.
武器.
這個很重要.
好.
今天差不多了.
下個月有什麼.
這個月是青嶺特飲.
因為是夏天.
下個月就九月了.
九月就秋天了.
可能有.
下個月見.
拜拜.
鄭處女.
好吧.
現在我們就給您熱 packets.
好的.
那個.
冷卻一下.
好.
那好.
哪個.
老頓.
你.
是.
好了.
好.
吧.
好.

$^{2921}$你來吧.
好.
那我走了.
好.
那我走了.
好.
那我走了.
好.
那我走了.
好.
那我走了.
好.
那我走了.
好.
那我走了.
好.
那我走了.
好.
那我走了.
好.
那我走了.
好.
那我走了.
好.
那我走了.
好.
那我走了.
好.
那我走了.
好.
那我走了.
好.
那我走了.
好.
那我走了.
好.
那我走了.
好.
那我走了.
好.

$^{2961}$那我走了.
好.
那我走了.
好.
那我走了.
好.
那我走了.
好.
那我走了.
好.
那我走了.
好.
那我走了.
好.
那我走了.
好.
那我走了.
好.
那我走了.
好.
那我走了.
好.
\newpage



\section{}
\label{sec:akT8yKiTNTo}
\textbf{【這是最好的時代:給香港基督徒的神學八課】第5課:一根刺的人|20210918 [akT8yKiTNTo]}
\newline
\newline
連結: \href{https://youtube.com/watch?v=akT8yKiTNTo}{\texttt{ https://youtube.com/watch?v=akT8yKiTNTo}} ~~~~ 語音日期: 2021-09-18 
\newline
\newline
\hyperref[sec:Wv0tPAVEIA8]{\small{< < < PREV SERMON < < <}}
~
\hyperref[sec:index_chronic]{\small{[返順時目]}}
~
\hyperref[sec:index_scriptual]{\small{[返順卷目]}}
~
\hyperref[sec:TgQ5_ITPOW8]{\small{> > > NEXT SERMON > > >}}
\newline
\newline
$^{1}$(第1集).
(第2集).
(第3集).
(第1集).
(第2集).
(第3集).
(第4集).
(第5集).
(第6集).
(第7集).
(第8集).
(第9集).
(第10集).
(第11集).
(第12集).
(第13集).
(第14集).
(第16集).
(第17集).
(第18集).
(第19集).
(第20集).
(第21集).
(第22集).
(第23集).
(第24集).
(第25集).
(第26集).
(第27集).
(第28集).
(第29集).
(第30集).
(第31集).
(第32集).
(第33集).
(第34集).
(第35集).
(第36集).
(第38集).
(香港我心心的故鄉).

$^{41}$(這律讓我生長).
(有我喜歡的朋友共陽光).
(路上人在跑過 逃過).
(幹勁靜默欣賞).
(這律有許多可圈換法案).
(日日星光 香港 香港).
(你永遠是曾夢開).
(香港 香港 你那小娘娘).
(山頂看小島 水裡流浪).
(似是玩得很開心).
(看向那海鷗飛過自由港).
(海邊看小島 全萬種).
(處處搖眼星光).
(這個市區的吸引無法擋).
(日日星光 香港 香港).
(所有我憑念夢想).
(香港 香港 叫我不以為望).
(香港我真心的抱憾).
(這律讓我伸張).
(有我喜歡的親友共陽光).
(路上人在跑過 逃過).
(幹勁靜默欣賞).
(這律有許多可圈換法案).
(日日星光 香港 你永遠是曾夢開).
(香港 香港 你那小娘娘).
(香港 香港 再有我同你吻人).
(香港 香港 叫我不以為望).
(香港 香港 你永遠是曾夢開).
每個年代都有每個年代的神學.
作為土生土長的香港人.
我們似乎正在經歷一個最差的年代.
不過往往在最差的年代.
我們才能夠經歷福音信仰的最好.
就是我們一起從聖經裡面學習.
怎樣做這個年代裡面的香港基督徒.
這是最好的時代.
給香港基督徒的神學百科.
各位弟兄姊妹晚安.
無論是在現場的.
或者我們在YouTube裡面的.

$^{81}$來到我們神學百科的第幾課.
第五課.
這課是很特別.
如果之前我們看過我們那四課的時候.
我們都講過什麼是福音.
我們什麼是基督徒.
講到教會.
講到我們的靈性.
去到金堂.
我覺得我們特別是加進去的.
如果你看一般教會的門訓材料.
或者是一些初信的百科.
其實都沒有這個主題.
我覺得我們作為全聖教會.
我們在新的年代裡面.
我們做一些最基本的東西.
我很想加進去.
作為我們每一個全聖教會的弟兄姊妹.
都可以首先去思考的課題.
就是一根刺人.
我們今天會講一些比較具體的東西.
我們會講一下我們跟人的關係.
講一下我們跟教會的弟兄姊妹.
當我們去群體相處的時候.
我們會遭遇到的課題.
其實這個課題是很重要的.
基本上我們在教會裡面.
大家都是這樣的經驗.
作為全聖教會的弟兄姊妹.
可能都試過很多不同.
在教會群體裡面的相處.
當中是牽涉到很多東西.
一些是藝術.
都牽涉到一些基本的聖經神學的理解.
所以這個課題.
我自己很想去特意去講的一課.
如果我們說去延續.
我們第三課的時候所講的教會觀.
大家記不記得.
我們所講的教會觀.

$^{121}$是一個稱之為黑暗教會觀.
我們從鋼琴鍵裡面的黑字來做定義.
我們覺得教會群體裡面.
我們人和人之間.
我們做一個教會群體.
其實可以充滿著很多的盲點.
充滿著很多不理想的狀態.
但這個是我們人.
或者教會在地上一個最基本的狀態.
教會是一個美視群體.
都是我們的信仰.
這個永遠都在恩典裡面.
所以延續著這種教會觀.
我們知道我們的教會.
甚至地上的教會.
其實是它的本相.
永遠都是一班有問題的人.
我們這樣開始的時候.
我們就很具體想到.
我們在教會裡面.
作為基督徒.
怎樣能夠將這種軟弱.
這種人和人之間很多的不同的摩擦.
成為我們所謂的門訓裡面的其中一課.
這個是很重要的.
很多門訓.
很多不同的材料都會講那個理想出來.
應該要怎樣做.
但我覺得其實在那個應該.
那個理想和實際之間的落差.
其實我們更加要探討的就是那個落差.
所以這一課我們會一起去探討.
當我們成為基督徒之後.
當我們願意去決志成為一個見證基督的人.
我們去明白福音的意義.
明白教會.
明白我們的靈性之後.
我們回到我們一個很具體的生活的處境裡面.
怎樣和基督徒去相處呢.
所以今天我就會去用我們之前.

$^{161}$我之前都經常講的題目.
就是一根刺.
當然保羅在哥倫多後書裡所講.
他說你為這事我三次求過主.
叫這刺離開我.
他對我說我的恩典夠你用的.
因為我的能力是在人的軟弱上顯得完全.
所以我更喜歡誇自己的軟弱.
好叫基督的能力復辟我.
我為基督的緣故就以軟弱.
凌辱.
急難.
迫不.
困苦為何喜樂的.
因我什麼時候軟弱.
什麼時候就剛強了.
保羅在教會裡面所經歷的.
我之前講到都講過.
那根刺似乎更加的痛.
不是因為身體的軟弱.
而是人和人之間的關係.
就算是在哥倫多教會裡面.
當保羅去寫聖經書卷的時候.
人事的關係.
在教會裡面.
人事關係已經出現了很實際的狀況.
所以對保羅來說.
這根刺其實是一根很不容易拔掉的刺.
因為這根刺不單單是關於他自己.
而是關乎於他整個面對的教會群體.
當時教會出現了很多不同的人.
大家都是基督徒.
可能大家都有這樣的經驗.
最難處理的可能就是基督徒和基督徒之間的關係.
很多時候我發覺很特別.
明明大家都是好人.
為什麼會出現這麼多問題呢.
如果其中一方是壞人.
就很容易理解這件事.
因為他是壞人.

$^{201}$所以就有這個問題.
或者我是壞人.
自然就會出現問題.
但是當兩個都願意做好人的人.
或者願意覺得自己是好人的人.
當他們相處的時候.
竟然會出現很多很多問題的時候.
這就是我們今天會思考的問題.
所以當中我們會嘗試.
簡單地講一些最基本的神學原因.
今天不會講太多神學.
但基本上都會講一下.
我們對這個情況的一些理解.
當然我們都回到罪的題目.
我覺得我們在這個年代裡.
我們很多時候都會講罪.
但其實都不是很真正.
能夠好好地去講這個課題.
如果你要講一個罪觀.
這個都是神學課題.
罪論.
基本上可以很簡單地講.
由所謂的原罪.
或者是次組犯罪.
我們人人都有罪.
我們很多年前的決戰時期.
我都明白.
人人都犯了罪.
不過我們這個罪的題目.
其實不能單單純粹作為一個神學理論.
作為一個知識去理解.
我們真的嘗試將這個罪的課題.
放在我們的生活裡面.
特別是在我們教會和弟子妹相處的裡面.
所以我們今天會花十分鐘時間.
講一下少少罪論.
今天是有些厭煩的.
可能對一些做了很久的基督徒來說.
又會講罪.
但我覺得我們將這個罪的知識.

$^{241}$需要一個很正確的拿捏和定位.
我們很多年前講了太多的罪.
報道會,培靈會都叫你認罪悔改.
以致我們這十年八載都比較少提這個題目.
但我覺得這個題目其實永遠都是重要的.
特別是不知道你知不知道.
如果看回初期教會的時間.
看回整個罪觀的發展的時候.
你都知道天主教有一個懺悔.
有告解這件事.
天主教反而是比較容易處理.
當它有告解和懺悔.
因為很實際.
你犯罪你就告解.
這件事告解完之後就清了.
真的清了.
一件錯事就去神父那裡告解.
上帝藉著教會的權柄去免去你的罪.
我們新教有些搞事的.
新教是沒有告解禮.
基本上我們說耶穌基督是十字架.
一次過去免去我們一生的罪.
但這樣是令我們很強調罪之餘.
又很不強調罪.
變成我們的罪的問題.
好像永遠都不能夠有一個很具體的方案.
能夠去解決它.
特別是初期教會.
初期教會不知大家知不知道.
當未有告解保屬禮之前.
其實初期教會的信徒他們是很怕犯罪的.
因為他們以前還沒有一套很完整的神學.
去說基督徒犯罪之後究竟有什麼解決方案.
究竟有多少個階段.
以前覺得只有三次階段.
你想想理想主義只有三次犯罪機會.
犯罪之後就沒有了.
再犯罪就會落地獄.
所以初期教會他們對於罪的很審慎.
所以你看到頭兩三百年.

$^{281}$基督徒是非常堅貞他們的信仰.
他們都是非常委神.
因為其中一個原因是他們覺得不可以犯罪.
他們對於罪是非常小心和謹慎.
因為只有三次階段.
後來出現了保屬禮.
懺悔告解才成為一種解救.
所以以前你可能聽過.
君薩丁皇帝以前是一個隨從的洗禮神父.
因為他是臨死才洗禮的.
他要臨死才洗禮.
為什麼要這樣做.
因為他不想用高塔.
所以要拖到洗禮的客人臨死前.
馬上急救幫他洗完禮就死了.
所以這是一個這樣的方法.
所以對於罪這個課題.
其實我們基督徒是很曖昧的.
可能大家聽過.
馬丁路德我們新教的一個很重要的發起人.
一個很重要的神託.
對於我們基督徒來說.
同時是罪人.
同時是義人.
這個可能大家都聽過的神託.
這是一個很辯證的看法.
路德是一個很嚴謹的人.
是一個對自己非常高要求的人.
所以對於罪這個來說.
是一個很存在性的理解.
一個很追求完美的人.
但發現自己一點都不完美.
所以他才會很痛苦.
他發現自己很多時候仍然會得罪上帝.
得罪了人.
所以今天我們做基督徒的時候.
其實背著同時是罪人.
同時是義人這個神託.
其實是可能模糊的.
你和你說你是罪人.

$^{321}$你會覺得我是罪人.
大家都是罪人.
但我同時也是義人.
所以這個似乎是你不能夠很容易消化的.
天主教是很不同的.
是很均真的.
犯了罪就是罪人.
告完解沒有事就重新再來.
重新再來做好一點.
做錯了就真的要去懺悔.
所以對於罪這個課題.
我們其實還不夠真正的去認識.
特別是在去年.
當我寫《庚次》的時間裡.
我覺得對罪這個課題.
特別有很深的理解.
我覺得在社會上很多所謂的罪人.
甚至比基督徒更加好.
因為罪人往往是很複雜的.
今天我會說幾個不同的神學概念.
不想說得那麼複雜.
因為很多時候我們發覺有幾樣東西.
一個叫做罪者和被罪者.
這個是我們香港的神學人馮煒文先生提出來的概念.
所以這是很土產本土的概念.
被罪者.
他說我們每個人不單單是罪人.
更加是一個被罪者.
就是說我們同時間是罪人.
不過我們也不要忘記.
我們同時也是被罪所傷害的人.
所以今天我們在群體裡.
無論是在你家裡.
或者是社會.
或者是教會.
我們每個人同時也是罪人.
同時也是被罪傷害的人.
所以我很喜歡我自己寫的那一句.
我們無論是多麼的病態.
在這兩條十字架.

$^{361}$我們每個人都是或多或少都有一定程度的病態.
因為是被罪所影響.
可能是父母.
可能是成長.
可能是社會的壓抑.
令到我們都是被罪所影響的人.
從而成為一個罪人.
不要說自己說別人.
所以當我們發現在教育群體裡.
那個人是一個罪人的時候.
他同時也是被罪所捆綁.
被罪所深深傷害的人.
所以令到這個課題更加複雜.
當我們說罪人的時候.
我們不單單是說一個理論上.
阿當的後裔所以犯罪.
我們同時也是被罪傷害.
從而做出一些很錯的行為.
所以人和人之間的傷害其實是很複雜.
不是純粹一個壞人和一個好人那麼簡單.
而是往往更加深層次.
而另一個古著的概念就是Sin of Omission.
和Sin of Commission.
所謂Sin of Commission是甚麼意思.
原來那個罪不是單單是你做了.
我們不單單說犯罪.
而是當你沒有做到應該要做的事的時候.
這個也成為了一個罪.
最明顯就是.
這個故事很失望.
頭兩個人基本上是沒有做事.
其實這個冷漠.
這個不加以幫助.
或者是沒有行動.
這個也是一種罪.
所以更加複雜.
原來罪不單單是你做了甚麼.
而是在你應該要做的時候.
而你沒有做到的時候.
這個也在群體裡面構成了一個罪.

$^{401}$所以你會發覺.
很多時候我們不是有意的.
當我們在群體生活裡面的時候.
我們傷害別人的時間裡.
可能只是沒有做一個應該要做的事.
可能是沒有說一句話.
可能是沒有回一個訊息.
可能是沒有說早安.
可能是沒有做一個應該要做的安慰.
這個也可能構成一些傷害.
這個就成為了一種罪.
所以當我們重新在群體裡面.
作為一個神學課題去想到罪的時間.
罪不是一個單單的理論.
而是在一個很具體的生活裡面.
都是唯一.
唯有在群體生活裡面.
你才能夠去明白和認識罪.
所以這個很具體.
你離開了群體的時間.
你沒有辦法真正實踐出.
我不犯罪的操練.
一個人是減少犯罪機會的.
不言不說.
一個人在獨處的時間裡面.
確實犯罪機會是少很多的.
但你只能夠在群體裡面.
在教會裡面.
在社會裡面.
才能夠真正去具體.
讓自己去跟隨耶穌基督.
就是對抗罪.
所以這是第二個比較.
令我們今天很不容易的課題.
就是一個Sin of Omission的課題.
不為.
純粹的無為.
都是一些罪.
第三個更加複雜.
我稱之為無知.

$^{441}$其實這是一個頗有趣的課題.
我們下課會說Passion.
下課就會說Passion的課題.
天主教一個很重要的神奴大師.
Thomas Aquinas.
他對於罪是一個很有系統的分析.
大家有沒有聽過七宗罪.
七宗罪.
你可能看過那套電影.
原來我最近才知道.
七宗罪其實應該是七罪宗.
七宗罪的意思不是.
Seven Cases of Sin.
宗字不是解作Case.
不是七宗罪.
而是罪宗本身的一個term.
什麼意思呢.
原來他認為.
罪是有七個不同的源頭.
一切的罪都來自這七個不同的源頭.
什麼呢.
貪心.
驕傲.
還有說謊.
這些罪其實是一切罪惡的源頭.
還有影響.
所以七宗罪其實不是七宗Case的罪.
而是七個不同的源頭.
Thomas Aquinas說到.
他更深層次地說.
原來罪是有分內和外的原因.
今天不說外.
外來自於什麼.
撒旦.
人.
試探之類.
或者是我們的意志.
但是他說.
原來罪是有三個內在的原因.
這個值得我們去認識.

$^{481}$原來一個人犯罪是來自三個不同的原因.
今天說幾個.
其中一個就是passion.
因為你衝動.
情.
情就是解作你的情感.
或者你的情緒.
這個我們下課會再說.
因為衝動.
衰衝動.
所以就會犯罪.
所以這是內在原因.
這個就是惡意.
就是邪惡的意志.
想壞事.
這個很明顯.
原來除了衝動和惡意之外.
一個很重要的內在原因.
就是ignorant.
就是無知.
天主教翻譯為愚昧.
但我覺得用無知更加直接.
因為ignorant這個字.
很多時候我們犯罪.
不是我們.
我講個人的.
人犯罪.
其實是出於無知.
就是說我們對事情的真相.
是不足夠的理解.
這是一個很重要的因素.
很多時候不是因為你有惡意.
也不是因為你有衝動.
因為人的理性能夠可以壓衝動.
令你不去犯罪.
但其中一個很重要的原因.
就是出於ignorance.
對於事情的真相.
是沒有足夠的理解.
這些事情其實會導致人犯罪.

$^{521}$當然不是什麼都知道.
湯姆·桑瓦尼講得很清楚.
他說不是什麼都要知道.
但你應該要知道的.
而你不知道.
這下就是罪.
所以你不需要什麼都很厲害.
或者知道整盤棋子怎麼走.
或者整個的東西.
但是當我們有些基本的資訊.
或者對於事情沒有足夠的理解.
而導致的罪.
一些錯誤的事情.
這個是罪.
所以說起來.
很多的罪其實不是理想的.
因為你仍然以為自己有足夠的理性.
和足夠的善意.
不過是因為出冤.
我們沒有足夠的對於事實的認清.
所以我覺得基督徒和基督徒之間.
很多時候是因為這個問題.
我們對於一個別人的看法.
沒有足夠的理解.
對於事情的真相.
沒有足夠的認識.
對於一個客觀的事實.
沒有足夠的知識的時候.
這個ignorance.
在聖經裡面就是愚昧人.
這種的愚昧.
是令到我們犯罪的.
所以去到這樣的層次的時候.
一個人犯罪.
其實是更加多層面的.
很多時候人和人之間的傷害.
往往在這些位置是更加容易走出來.
更加傷害的.
我經常都覺得.
特別這幾年.

$^{561}$一個惡人.
一個很明顯的罪人.
更加容易更加好.
但很多時候基督徒.
因為出於ignorance的緣故.
所做出來的傷害.
是更加深.
所以這個令人很沮喪.
因為原來很多時候的傷害.
是因為這樣的緣故.
所以我曾經在Facebook上寫過.
明目張膽去做一個罪人.
可能會更加舒服.
當一個基督徒很多時候.
出於無知的緣故去傷害人.
這個反而是一個更大的傷害.
所以我們不要去低估.
在教會裡面.
人和人之間這種罪的傷害.
教會裡面有沒有壞人呢.
有沒有充滿惡意的人呢.
我覺得是有的.
但不是很多.
衝動的也不少.
但無知的更多.
對於事情的真相沒有足夠的理解.
懷著善意.
懷著理性.
但不足夠的知識.
會造成很多不同類型的罪和傷害.
這是我們教群體裡面.
很重要的具體處境.
所以我很喜歡.
我也在不同的場合提過.
一個美國神學家叫拉奎尼布.
他是一個非常真實的人.
一個基督教現實主義者.
所以他對罪從來都不會很吝嗇.
他考慮罪的因素在整個神學論述裡面.
在倫理裡面.

$^{601}$他寫很少太多理想的東西.
相反他寫了很多實際的東西.
來探討我們要如何考慮.
罪的元素在我們的社會裡面.
我們要如何做一個好行為出來.
所以他介紹了三本不同的書.
大家可以去讀.
一本叫做Moral Man and Immoral Society.
Study in Ethics and Politics.
這裡說到一個Moral.
道德的人與不道德的社會.
他說原來一群有道德的人.
聚在一起就變成了一個不道德的社會.
很奇怪的.
如果將這個課題壓縮到變成一個群體.
其實都是一樣的.
大家都是好人.
但聚在一起.
卻變成一個充滿著很多問題的群體.
所以這本書是非常精彩.
說到如何從個人去到社群裡面.
即是那個之間.
如何從大家都懷著好意.
同時導致很多惡的問題.
第二本就是一個很重要的書.
被譽為20本最好的書之一.
即是非洲書裡面.
The Nature and Descent of Man.
一本很簡單.
重提基督教的罪觀.
在這個世界裡面.
當時是1939年.
即是二戰的時間.
當飛機飛來飛去.
炸來炸去的時候.
尼泊爾就在英國裡面.
他給了一個很重要的Nature.
就是關於人的罪.
這個罪觀.
所謂很老土的論述.

$^{641}$在二戰當中是令人發生心醒的.
亦都說到一個問題.
即是去探討到人的罪.
究竟是一件甚麼事情.
第三本更加喜歡的.
就是The Truth of Light and the Truth of Darkness.
光明之子與黑暗之子.
他說.
用回耶穌的那個比喻.
即是你們的光明之子.
要去參考今世之治.
你們要學習今世之治這樣做人.
他說基督徒.
很多的光明之子.
不是壞的.
是蠢的.
即是你不夠那些黑暗之子那麼聰明.
所以他說.
我們一班光明之子.
更加需要去學習.
來明白到這個世界黑暗之子的玩法.
所以他從來都很強調.
這個世界和社會的黑暗.
不過這個不是悲觀的.
這些看法仍然是說.
我們怎樣能夠在這麼多盲點裡.
能夠活出我們應該有的倫理.
所以我希望大家都是.
在這個作為我們Flow Church的其中一個門訓課程裡.
大家去重視這個課題.
我們不單單去說理想群體應該怎樣做.
更加想我們實際上應該怎樣去面對赤裸裸.
活生生的一個群體.
怎樣去相處.
所以你問有什麼可以做呢.
其實都沒有什麼可以做.
唯有上主光照我們.
真的.
唯有怎樣去發現這些盲點呢.
我很喜歡彼得這段故事.

$^{681}$當耶穌去天神跡.
在海裡叫西門彼得.
打了很多雨回來的時候.
你猜都猜不到彼得突然爆了一句話出來.
完全不對劇本.
打完雨突然爆了一句話出來.
在耶穌的膝前說主啊離開我,我是個罪人.
這句話是完全沒有什麼context.
前面後面都沒有什麼直接關乎於說到罪.
還是什麼問題.
但是當耶穌去顯現他的榮耀的時間裡.
我喜歡這樣說.
當彼得去發現上帝的時間裡.
這是一個很重要的illumination,一個光照.
唯有耶穌能夠在我們的生命裡光照我們.
我們能夠發現自己是一個真正的罪人.
甚至乎可以說是一個PK.
是一個壞人,是一個賤人.
有時候我不知道大家怎樣理解自己.
當你做基督徒做了很久.
有沒有發現到偶爾望著鏡子.
我真的是一個PK.
或者你很熟悉的人.
你這一下真的很PK.
偶爾去想一下自己.
是否一個PK都很重要.
這種光照.
唯有上帝讓我們看到自己的錯誤和盲點.
是非常重要的.
所以我覺得這是我們可能很重要的一個方法.
來解救.
我之前也提過.
我覺得一間健康的教會.
很簡單.
基本上每一間教會都有這些問題.
就是人與人之間的關係出事.
就是唐主任和傳道人.
傳道人和鄧小姐.
鄧小姐與唐主任.
很多不同的組合.

$^{721}$總是會出事.
所以一間健康的教會.
是不能說不存在.
但一個動態過程.
你只能夠不斷地去減低苦頭.
減低自己對人的傷害.
第二就是強化玻璃心.
這是一個防洪工程.
加強我們的心.
我不是說大玻璃心.
而是減低一些受傷心靈的問題.
減少苦頭釋放出來的機會.
真人.
我們很多時候在教會裡面.
說一句話.
做一些事情.
一個舉動.
其實往往會傷害到人.
這個我自己也經常出事.
我今天活到41歲.
仍然在學.
不過確實這幾年是好一點.
苦頭是少釋放了.
所以我想大家也是值得去問.
如何能夠在人與人之間相處.
如何能夠減少苦頭釋放出來的機會.
不斷地去查看.
我今天有沒有不小心說了一些話.
傷害了人.
某個舉動會不會令人不開心.
有些比較說笑的話.
可不可以不需要說.
這是我們第一件事的工程.
在教會裡面要減少苦頭.
第二就是強化玻璃心.
不過大家不要誤會.
我不是批評玻璃心.
基本上任何心都是玻璃.
心與苦頭永遠不能比較.
一旦碰上去心就會爆裂.

$^{761}$強化玻璃心不是叫你不要傷害.
不要有任何感受.
而是當你被人劈到之後.
可以令到你不會在痛楚裡面太久.
甚至可以快點復原.
當然這是一個理論.
如何能夠做到很多不是在課堂上說到的事.
如何能夠讓心臟更加強大.
我們能夠面對弟兄姊妹的說話.
不會太過玻璃心受到傷害.
我都說過我跟你吃過火鍋.
我跟你說雞翼未熟你都不開心.
這樣就糟糕了.
別人不同意你你都不開心.
這其實很不容易.
所以如何能夠仍然.
其中一個位置就是相信弟兄姊妹的善意.
起碼第一次是.
起碼第一次的時候.
去相信弟兄姊妹是有善意的.
我很喜歡Serendi的一首歌.
這首歌叫做《屍》.
《我不知道你是否一個賤人》.
這首歌很有趣.
大家搜尋一下.
基本上我只是懷疑你是一個賤人.
我不會太快認定你是一個賤人.
我跟朋友談過.
他都說你可能是一個賤人.
這很重要.
不要太快去認定對方是奸惡著.
不是說不要.
我不是叫你永遠相信阿門.
因為他是好人.
起碼你第一次.
我都給一個信任.
你是好人來的.
我覺得你有善意的.
第二三次才慢慢去評價原來你是賤人.
這很重要.

$^{801}$不要太快去玻璃心.
心臟更加強大.
當然我這樣說是沒有責任的.
永遠是耶穌基督的恩典.
能夠幫助我們.
能夠面對這些問題.
所以我覺得客觀來說.
只有兩個方法.
弟兄姊妹之間相處.
減少苦頭的出現.
和去強化我們心臟.
是一個很重要.
能夠讓全教會健康.
兩個很重要的工程.
所以我覺得弟兄姊妹之間的相處.
都是兩個原則.
我自己都是激的人.
說話都不是很審慎的人.
不像潘Sir.
我很珍惜坦誠.
坦誠是我們基督徒很重要的美德.
在橫教會裡面.
基本上我希望Fortress能夠成為我們的DNA.
能夠去坦誠.
不需要裝模作樣.
不需要裝成基督徒.
不需要裝成自己很好的傳道人.
意思是真的算是好的.
不需要裝出來.
大家之間說話能夠說成實話.
這是我自己一向都覺得很重要.
無論是在我們侍奉裡面.
或者是在我們的群體裡面.
能夠將心裡面的說話坦誠地說.
不偽裝.
是很重要的事情.
不過我自己在35歲之後.
覺得這句說話其實是有下一句的.
坦誠是作為很重要的美德.
是一個第一步.

$^{841}$大家要去堅持的東西.
不過卻是要有第二步.
我最近很喜歡這句經文.
羅馬書的經文.
保羅大家都很認識的.
十二章裡面.
愛人不可虛假.
惡要厭惡.
善要親人.
愛弟兄要彼此親熱.
恭敬人要彼此推養.
我以前是很不喜歡後半句的.
要恭敬人要彼此推養.
就想起以前的橫教會的東西.
牧師牧師牧師牧師.
不好意思不好意思.
你那些.
大家你先你先你先.
我經常想起這些課題.
我都很怕橫教會的文化.
很有禮貌.
都是棉女針那些.
最恐怖.
但其實原文不是這個意思.
恭敬不是純粹基督教文化的禮貌.
愛人當然不可以虛假.
惡要厭惡.
善要親近.
這個是大家要坦誠.
我們要表裡如一去做.
但保羅後面那兩個很重要.
原文裡面就說.
兄弟的愛.
不是單單叫愛弟兄.
而是一個真正的兄弟的愛.
我們要用兄弟的愛來彼此對待.
大家真的在群體裡面.
要真是一個brotherly的關係.
彼此以尊榮.
在原文裡面就是說.

$^{881}$你要去honor.
給一個尊榮來看待對方.
所以這個是其實後面那兩個說話.
只是補充頭那兩句說話.
沒錯我們是要惡要厭惡.
善要親近.
但你仍然要怎樣.
仍然要真是視對方為一個兄弟.
family的對待.
並且是要honor他.
你要遷就他.
你要去愛著他.
這一下是重要的.
怎樣能夠可以去.
在我們的言語上.
仍然在每一刻裡面.
去honor對方.
去遷就對方.
去說出你不高興.
你認為錯的事情.
所以我覺得人長大了.
發覺原來我們弟子妹之間.
這些事情是很聚集的.
當然我們不會回到.
以前華人教會的那些位置.
但是在惡要厭惡.
善要親近.
和彼此去尊重和honor.
那個拿捏位置是很重要.
去拿那個智慧.
怎樣能夠不會被你的passion反抬.
不會因為ignorance.
去傷害對方.
我覺得大家都沒有惡意.
但怎樣仍然honor對方.
是一個很重要的學習.
很快我們說完下面那兩個.
所以這個我都說.
tolerance似乎是一個很重要的觀念.
我們比較少提的觀念.

$^{921}$tolerance不是解造容忍.
不是去忍受一些錯的事情.
而是一種寬容.
一種心胸裡面能夠容納更多的東西.
相信對方的美善.
你的心胸能夠擴大.
能夠嘗試去接納.
嘗試去包容對方.
嘗試釋出善意.
起碼在第一,二步時.
是一個很重要的東西.
第二就是復和.
我們怎樣能夠主動去復和.
這都是我們基督徒很重要的學習.
最後我想說的是.
潘復華在Nag Foger.
《聖經基督徒》這本書裡.
提出一個很重要的觀念.
他整本書裡有一部分是說登山補訓的詮釋.
他提到當耶穌教我們.
愛仇敵.
被人刮完一巴之後再刮多一巴.
這一下是怎樣解釋的.
他說是一個超乎自然的.
他說愛其實是超乎自然的.
愛本身或者耶穌所叫我們群體裡的愛.
是超乎自然的.
什麼意思呢.
你自然而然是不會這樣做的.
正正常常你是不會愛仇敵的.
你都不會去復和的.
你都不會去不計較的.
你都不會去不寬容的.
你都不會寬容那個人的.
因為是非對錯是人之常情.
但古佛說這種的愛是超乎自然.
是擺明不正常的.
是你正常的行為裡面你不會做的事.
所以這些行為是唯有靠著耶穌基督去幫助你.
去永遠的多你幾步.

$^{961}$的多你幾格.
逼你多走一兩步去嘗試.
是奇蹟地去做出來的行為.
所以頂尖末之間都是這樣.
我們自然而然有人得罪我.
我是不會喜歡他的.
但這種的愛正正在基督徒裡面.
是有可能去做到.
這個不是要大家說你不做就不是基督徒.
或者不做就是壞人.
而是永遠都是基督的愛激勵我們.
去逼我們用這份的愛去回應.
所以這份的愛是超乎我們能夠做到的事.
自然的愛我們就會喜歡一些我們喜歡的人.
群一些我們喜歡的人.
得罪我我就自然不會去任何的善待他.
但是愛受的這個命令告訴他什麼叫真正的愛.
一個真正耶穌基督所說的那種愛.
是超乎我們自然.
我們不可能做到的.
最後再說兩句.
有人打你右邊臉.
左邊臉也被人打了.
這句話怎麼解釋.
這句話我傾向這樣解釋.
這句話說出來不是叫你做不到的.
是嘗試叫你做的.
是可以讓你做的.
我會說為什麼.
其實這部分這句話很重要.
就是七章十二節.
所謂叫做Golden Rule.
就是你願意人怎樣對你.
你就怎樣去對人.
這句話跟孔子所說的完全不同.
跟說己所不欲勿施於人是完全不同道理.
你想人怎樣對你.
你就先去怎樣去對人.
我想潘Sir請我吃飯.
我會怎樣.

$^{1001}$我就嘻嘻.
不如我請你吃飯吧.
不要這樣.
怎樣能夠在別人的Facebook上加你.
就加他吧.
怎樣可以.
你太太當你好像皇帝一樣服侍.
你先當她是皇后.
所以這樣對你.
先用別人想對你的方法去對別人.
所以這是一個你主動的行為.
以眼還眼以牙還牙是被動的.
別人有仇敵對我.
別人以眼我就還眼.
以牙就還牙.
這是對的.
這是公義.
這是一個律法很重要的精神.
公平公正公義.
但耶穌跟他說.
當我們在群體相處的時間.
我們要超越自然的事情.
當別人對你有傷害的時間.
我們可以嘗試扭轉局面.
我嘗試給一個例子.
很多年前我去德國讀書的時候.
我就要買飛機票去德國.
當時我由香港飛去德國.
買了一張單程機票.
因為我不知道什麼時候回來.
我去德國這麼多年.
就買了一張單程機票.
後來我發現原來我回來香港都會回來.
我會結婚或探親.
放假我都會回來.
原來這些人是這樣的買法.
他們會在香港讀書的時候.
先買一張來回機票.
走過程就買一張沒有寫日期的來回機票.
拿著去程去德國.

$^{1041}$拿著回程機票就不用.
保留在這裡.
回來香港探親才用那張機票回來.
再買一張新的來回機票.
原來這樣買法便宜一點.
我怎樣可以買回這些便宜的機票.
這其實是一條智商題.
我怎樣可以買回這些便宜的機票.
當時我已經在德國.
我怎樣可以買回一張從香港出發的來回機票.
沒理由不認識這些.
Jet Soul的東西沒理由不認識.
我怎樣做.
當時我已經在德國.
我怎樣做.
怎樣才可以變回我在香港買一張去來回的機票.
德國香港的來回機票.
YouTube的朋友知道嗎.
怎樣.
如果你聰明的話.
我明知道是貴的.
明知道是虧本給國泰的.
我都買了一張貴的單程機票回來香港.
我就可以從此買回香港的來回機票.
我先買了一張貴的單程機票.
明知道是不划算的.
明知道是貴的.
不過我可以從此享受一張永遠是便宜的關係.
所以耶穌教的一樣.
被人打完右邊碟再打左邊碟不是白癡.
不是被虐狂.
而是我嘗試用愛來回應別人的恨.
明知道是傻子.
明知道是不公平的.
但我這樣可以去終結一段不理想的關係.
當你願意這樣去付出的時候.
你能夠終結一段不理想的來回關係.
重新開始一段以愛還愛的關係.
別人以眼你還眼.
這是一個被動的.

$^{1081}$永遠都是被動的.
你被人拉著鼻子走.
但你嘗試用愛去還人家的眼.
這就可以終止一段關係.
所以耶穌的意思不是叫你做傻子.
而是你有可能去這樣選擇.
去做主動地.
因為你想別人怎樣對你.
你先去怎樣去對人.
下星期一你嘗試上班.
你對著你很恨的老闆.
你嘗試去愛他.
試試吧.
試試對著那個客戶或弟兄姊妹.
你嘗試的去這樣做.
總括來說.
別人以眼你還眼.
別人以牙你還牙.
下次可以試試.
當別人再以眼的時候.
你想想.
我是上過神學百科的.
耶穌也說過.
我嘗試突然用愛去對待他.
他會突然說.
這個人是基督徒.
突然嘗試這樣做.
然後他就會這樣去愛.
大家就可以成為朋友.
你很快會問.
是不是可以的.
是不是這麼天真.
是不是安利.
可能是的.
可能這些想法是天真的.
不過我想最少在教會裡面.
是值得去試的.
外面我不知道.
外面很多人會利用我們的愛心.
但在教會頂姐妹裡面.

$^{1121}$我覺得值得這樣去試.
當別人對方踩你一腳的時候.
你嘗試用愛來回應.
嘗試扭轉這種恨的時間裡面.
就可能有上帝的聖靈幫助.
大家可以復和.
所以我想這是一個很具體的.
頂姐妹之間.
我們復和的寬容.
能夠在破碎的關系裡面.
勇敢地走出第一步.
這個的主動.
我希望Full Church能夠有這樣的DNA.
或者希望有這樣的理解.
嘗試能夠建立我們這個群體.
將這次成為我們成長裡面.
或者我們門訓裡面的一個很重要的課題.
用上帝的愛來衝破一些.
不可能我們做到的事.
這是一個冒險.
這是一個未必有回報的事.
但值得我們去嘗試.
用這份愛重新建立這個群體.
作為一個大家流散過.
回到Full Church裡面的頂姐妹.
我覺得我們都要學這個功課.
如何能夠用上帝賜給我們的愛.
來對待我們的頂姐妹.
這次好像有湯喝.
請問.
有,不過湯涼了.
我怕你喝完之後會肚痛.
我原本是打算給你喝湯.
但因為我最後聽到你說.
要用愛去對待別人的時候.
我其實還有些菜.
一會兒拿回來給你吃.
我都會愛你的,沒問題.
不過其實我剛才有一個很入心的內容.
就是主動去做一步.

$^{1161}$我們搞食肆.
其實食物經常給一些特價.
真的吸引多些人.
對我們的感官和關係都好一點.
因為我們基督徒應該主動去做一些事.
不是回應,回應是被人拉住鼻子走.
所以這是我們很重要的東西.
但我有一個問題.
我自己一個做,是很難的.
有時覺得自己經常吃虧.
有沒有什麼方法.
或者是如何令到多些人都覺得這件事是正的.
我覺得是教會裡面是可以的.
特別是Full Church裡面可以試一下.
公司真的難說.
甚至乎在網絡世界裡面.
真的有.
謝謝阿妹.
都說了,沒騙你.
我做大了.
我覺得教會是可以實踐這件事.
在外面職場就很難說.
或者在網絡世界裡面.
很多事情不是那麼簡單.
都有很多ignorance.
人家都還沒明白真相.
很多時候都化解不了.
但我們在教會裡面是值得這樣做.
大家都聽過這課.
大家都明白這件事的時候.
希望大家都可以嘗試被聖靈光照.
去嘗試做這件事.
我覺得Full Church是值得的.
大家有沒有什麼主動的經歷.
或者是你覺得在課堂裡面.
覺得不明白的地方.
因為我剛才聽到.
ignorance 無知都是很糟糕的東西.
大家有沒有什麼想問.
怕自己無知呢?.

$^{1201}$有,那裡.
後面.
其實我想問.
可不可以再說多一點點.
關於復和.
就是.
你知道.
其實可能傳統華人教會.
教人家的是假復和.
就是你要去包容.
去原諒.
去饒恕.
但可能你中間是沒有真相的.
是沒有彼此認罪的.
其實不是想像之中那麼容易的事.
如果你做主動嘗試去復和.
但是結果不是你想像之中那樣.
其實是更加受傷.
如果一個群體.
暫時都沒有復和得到.
那你怎樣去.
繼續去面對大家.
跟著去彼此坦誠地相處呢?.
第二個問題就是.
我想問怎樣去面對.
一些.
無知的人.
即是就算你.
可能你想跟他說一些東西.
或者說一些事實的某一部份.
他未知的畫面.
但其實你發現你告訴他.
其實他都好像.
聽完都沒有甚麼.
或者他甚至沒有興趣知道.
有時可能.
一些群體的無知.
其實都會繼續傷害到.
堂會裡面的其他弟兄姊妹.
其實都好像是一個困局.

$^{1241}$怎樣去.
去坦誠.
或者很真誠地去相處.
我覺得是很struggle.
還有很多傷害在裡面.
是的.
我們是真的.
不可以忽視那種傷害.
剛才說的絕對不是想.
純粹擺一個理論出來.
就叫大家做.
我都知道.
我自己都經歷了很多.
所以都知道.
這些事不是說完之後就行了.
剛才你說復和那裡.
我覺得.
復和是mutual.
同時間.
但我覺得復和是一個過程.
那個過程是一個.
一個.
很多很多.
想起很多東西.
可能是爸爸媽媽和自己之間.
或者是在社會裡面.
或者是在一個關係裡面.
很多不同種的復和.
那個復和從開始到完滿.
是一個很長的過程.
真的.
沒有了大家的認罪.
大家的真相.
其實是談不上復和.
談不上復和的意思是很vague的.
是因為談不上一個完滿的復和.
但我覺得我們剛才所說.
如果耶穌和我們說愛受敵.
或者是十字架裡面.
耶穌和旁邊那兩個被釘十字架的.

$^{1281}$赦免你們.
這種的.
我都稱之為復和.
是復和的開始.
我們在整個過程裡面.
很多時候.
譬如其中那本書.
Michael Wolfe.
寫了一本書叫做復和的書.
我稍後再說.
很精彩的.
就是這個問題.
怎樣能夠說復和的神學.
在南非裡面.
在一個政治裡面的復和.
真相 彼此認罪 道歉.
這些是一個很重要的復和.
我們都很明白.
沒有這些是談不上復和.
但我這個說話是值得再edit的.
是談不上完滿的復和.
過程是未完結的.
但我覺得我們一開.
即是耶穌教鼓勵我們.
先拋一個石頭出來.
這一步其實都是一個復和的開始.
我覺得會是.
當中是充滿著冒險的.
因為我們拋完出來.
可能是沒有回應.
可能更加多的傷害.
可能是別人覺得你很傻.
但我覺得.
我經常說.
全個地球裡面.
最有本錢去先拋一個石頭的人.
就是我們基督徒.
因為我們是先被上帝復和了的一群人.
我們有本錢去嘗試冒險去拋這個.
可能沒有回報的石頭出來.

$^{1321}$所以耶穌鼓勵我們做這一步.
這個鼓勵其實不是一種命令.
或者純粹是一種硬梆梆的條件.
做完就復和了.
不是這個意思.
但總是有第一步.
例如很多複雜的情況.
例如父子關係.
我們跟爸爸媽媽的關係.
可能耶穌給我們勇氣力量.
嘗試再拋多一步出來.
可能已經很多年了.
但我們又嘗試去說一聲早晨.
或者WhatsApp問他喝茶.
可能都是傷害.
可能都是沒有回應.
但聖靈給我們又一次勇氣去先做這個力量.
所以這個談不上復和的圓滿.
但這個都是復和的開始.
沒有這個就沒有了.
大家是停在這裡.
我們似乎是基督徒作為復和的使者.
能夠嘗試實踐.
我自己很喜歡和平這首歌.
和平是指這件事.
無論是獅子和羊.
或者人和人之間.
終極地就是耶穌的和平.
祂復和了這個世界.
祂才叫我們成就一切的復和.
所以我們可以嘗試做這件事.
所以我覺得是語言上的澄清.
這個復和不是圓滿復和.
但都是復和的一步.
亦都是重要的一步.
那Ignorance那裡.
反而認識了Ignorance就懂得放下.
因為我們知道這個世界裡.
唯有在天國裡才有無限長的時間能夠對話.
如果你看一些哈巴馬斯克的說話.

$^{1361}$叫做Discourse Ethic.
即是對話的倫理.
如果一個社會是理性的話.
有足夠的空間去對話的話.
是很重要的.
但我們沒有足夠的空間.
不夠時間去看別人的文章.
不夠時間去面對面對對談.
所以現在這個世代我們沒有足夠的空間.
真相去對話.
所以我們只能夠得到某些程度的真正滿足的對話.
但發覺社會裡面越忙.
我們越來越少這樣的空間對話.
可能夫妻都已經不同意.
大家真誠的對話.
所以我們確實對Ignorance是很嚴重的傷害.
所以我覺得世界裡面仍然充滿著這些問題.
因為Ignorance是解決不了的.
所以比較實際的.
就是你明白對方是Ignorance.
你就知道不需要那麼上心.
這就是我這幾年在Facebook的經驗.
所以也不需要太介懷別人的無知.
反而我覺得無知是對自己說.
自己可能是無知.
也不明白對方.
可能也會有錯的機會.
這樣去做人.
不知道潘Sir有沒有其他補充.
我覺得如果看復活.
剛才John說的那是一部分.
我覺得說復活之先應該是說饒恕.
就是饒恕是第一步.
我認識的饒恕或者我自己個人做的饒恕.
不需要期望對方有什麼回應.
是我自己主動去做的那部分.
意思就是我怎樣看待那件事.
他再做什麼對我來說.
已經不會再觸動我或者觸怒我.
舉一個例子.

$^{1401}$我那天請John去我家吃飯.
他去廁所.
走過我的走廊的時候.
這是例子不是真的.
不是真的.
走過我的走廊的時候.
心急.
手袖碰到花瓶.
掉在地上爛了.
他很急去完廁所出來.
跟我說不好意思.
打爛了你的花瓶.
那一刻他就跟我說.
不好意思打爛了你的花瓶.
但他不知道花瓶對我的意義是多麼重要.
事實上花瓶是爛了.
我不能再有那個花瓶.
他真的跟我說對不起.
我沒辦法了都爛了.
我就原諒了他.
但是每一次我回家的時候.
經過我的走廊.
原本我很喜歡的花瓶在那裡.
但已經沒有了.
但我每一次看到沒有花瓶的時候.
我就會想.
總之不要請他來吃飯.
不請他吃飯就不會爛了那個花瓶.
他一句對不起就弄了我的花瓶.
我沒有了.
很不開心.
但我可以做些什麼呢.
我可以做些什麼.
我唯有想想.
其實他也是不經意.
我最主要的就是.
主動去饒恕他那個無知.
或者不在意的情況下.
做了這件事.
這就是我做些什麼.

$^{1441}$而不是他做些什麼.
可能對你來說.
這也不是解脫.
但是我跟自己去處理這個情緒.
或者這個過程的時候.
每一次看到花瓶沒有了.
那個空的桌子的時候.
我都告訴自己.
那件事已經過去了.
不要再讓那件事再核製了.
是不開心的.
如果那個花瓶是很有意義的話.
你更加會對你的傷痛.
或者那個執著更加大.
但是那件事已經過去了.
你願不願意讓自己放下呢.
我自己在成長過程當中.
有幾次都要學習這個大課題.
讓自己明白.
有些事情你再堅持下去.
一來不會回頭.
二來又會令到那件事.
不斷地在自己的情緒勒索自己.
而對我再看回這個人的時候.
我就會更加對他有一種隔膜.
每次見到他的時候.
就想起沒有了那花瓶.
那我跟他的關係又怎樣可以繼續呢.
這個都是我們要正視的.
所以這個就會影響到那種復和.
或者那種關係可以繼續下去.
這個可以再有討論空間可以對話.
第二個就是在說無知那件事.
我想2019年開始之後.
我們周遭的環境.
很多人看的視覺都不同.
收的資訊都不同.
或者表達的立場.
或者是描述事件發生的經過都很不同.
我們都覺得為甚麼人們.

$^{1481}$不多點角度去看.
我自己從石劍未講到的時候.
都講過我接觸過的家庭.
在那半年.
很多都是兩代人之間的溝通關係.
或者視覺上的問題.
但我都跟那些年輕人說.
如果你信得過.
或者我的能力.
我都會跟你爸爸媽媽說.
我都讓他們.
我那時候講到都是.
我實際上都帶父母去現場.
看一下他們的視覺.
也是讓他們明白到.
有些事情他們在電視機看不到.
真實不是在電視機看到那個.
所以至於你說.
我可以回應.
無知其實是一個時間性的過程.
當我們在經驗學習過.
經驗經歷過.
或者在嘗試讓自己開放.
無知就是要開放.
開放自己.
給自己多點角度去看.
或者多點參與的時候.
就縮短那個無知的時間.
我用聖經的說話.
就是耶穌說.
「父啊赦免他們.
因為他們所做的他們不曉得」.
人是看不明白耶穌那時候.
為甚麼要上十字架.
但當過後的時候.
就明白.
其實無知是很多仕途.
當時或者門徒當時來說.
是不明白的.
如果在科幻書裡面.

$^{1521}$很多時候有兩個天.
科幻書裡面說.
有兩位身穿白衣的人.
跟他們說.
第一次出現.
就是在耶穌復活的第一天.
就是「加利利人啊.
你們為甚麼在死人中找活人呢.
他不在這裡.
當紀念他在加利利所說的話」.
就是那時候他們不明白.
耶穌必須受死埋葬.
第三天復活.
但現在你們明白了.
你們要看了.
第二次.
還有兩個身穿白衣的人.
跟那班門徒說.
就是《司徒行傳》第一章第八節.
對不起 第十二節說的訊息就是.
「加利利人啊.
你們為甚麼站著望天呢.
他怎樣上去.
亦會怎樣下來.
當紀念他在加利利對你們所說的話」.
其實很多時候.
人在無知的情況下.
看不到上帝的工作.
或看不到周邊的工作.
但等著時間過去.
或等著時間中慢慢去揭示.
看到整個群體的參與.
我們會慢慢.
我們求上帝讓我們看到.
一些我們要學的東西.
所以無知就是要經歷那個時間.
我們願不願意.
有沒有一個主動性.
讓自己不要單一視角.
或是要學習多點經歷.

$^{1561}$其他人嗎?.
網上的問題.
網上的Cliff Thomas問.
我想有很多離開了園堂會的朋友.
原本都是懷著這種寬容.
但這種寬容只能單方面終止一次性.
一次的聚群環.
卻無法終止.
其他人去一次又一次.
去開展一個新的聚群環.
聽起來都有點蒼白無力.
第一面對這種情況.
對人的信任.
或許比不信任的人更低.
信任.
對人的信任.
或許比不信主的人更低.
在這種單方面去走一步的狀況.
我們有沒有什麼出路.
還有第二.
有沒有一些可以增加大家的承諾的方法.
我覺得承諾是對的.
我覺得要補充.
剛才說承諾,復和這些字眼.
我正在想的反而是一個具體群體.
剛才的情況.
我們離開了群體或離開了那群人.
有時切割是重要的.
你活在那個群體或社會裡.
本身就是一個社會學的字眼.
一個現代的社會裡.
一個多文化的社會裡.
就需要有承諾.
所以這個很重要.
如何去說一句「合力」.
人有息相關.
大家如何學習.
可以有承諾.
這是一個很重要的精神.
在美國多文化裡就需要有承諾.

$^{1601}$很小的時候就學習.
大家有不同的文化.
因為他們在一個社會裡.
不會離得開.
承諾是說.
你在一個身處不可逃的環境裡.
你需要有承諾.
但有時我們也會學會.
我要離開.
我要割捨.
不要讓那個東西繼續在我心裡.
所以是兩個不同的東西.
你很難去承諾一些敢對你的人.
不是說你不應該赦免他.
而是具體的傷害.
所以我覺得是兩種不同的情況.
我們還未去到那個階段.
例如Future的弟姐妹.
在一個小組裡.
大家有不同的想法.
或者不同的取向.
這是要承諾.
所以大家要學習.
一起去相處.
但在另一方面.
我都覺得.
剛才你說一次又一次.
你說一次做到的.
但在一個序的循環裡.
我都覺得不是適應的問題.
而是另一個問題.
所以我都說.
今天說的話.
在一個全面的教會裡.
大家都學習去做.
是會容易些去做.
而且是可達的.
而在外面其實是很複雜的.
其實我不是第一次說這個話題.
但很多時候.

$^{1641}$有人問我在職場裡怎麼辦.
我都不斷利用我的愛心.
這是另一個課題.
但我覺得在教會裡.
大家都這樣去學習.
是容易做.
而且是值得去做.
所以在某些位置.
不要讓別人繼續傷害.
不要讓別人繼續在那些位置裡.
最少你要離開.
不要讓別人在那個循環裡這樣做.
我認同.
意思是.
有些位置.
我補充一句.
不是未試過的.
如果它是日復日.
或者是不斷地.
最循環地這樣做的時候.
你中間要說的話都說過.
你做的事都做過.
它都是用原原這個方法去對付你.
或者是對待你.
我覺得就要抽離.
因為不可以不斷地.
給這些情況去傷害自己.
但我認同.
既然你遇到這樣的經歷.
而你又找到一班相關的弟兄姊妹.
其實剛才說到Flow Church.
Flow Church有很多弟兄姊妹.
有相近的經歷.
我自己教Info Group的時候.
我在第三第四課都說過.
Flow Church的小組.
有些最大公因數.
雖然大家來自不同的堂會.
但是侍奉.
或者小組經歷.

$^{1681}$或者對人相處.
或者在教會上有些難處的時候.
都有相近的地方.
你一分享的時候.
就知道原來只有我這麼慘.
原來有些人都跟我差不多慘.
我的答案就是說.
我們盡可能不要將這些慘.
再在Flow Church的小組出現.
因為這個盡可能抽離.
不要再讓這件事再核製我們.
或者再重演出來.
我不是戴頭盔的.
我那時候也說.
但是我們要學習.
因為很多時候我們離開了群體的時候.
因為我們在那個環境太久了.
我們需要好不容易提醒自己.
剛才那件事.
你曾經被人這樣對待過的時候.
你就不要用那個方式再這樣對待人.
所以在小組當中.
應該用一個新的形態.
新的方式去建立一個.
比較好的小組的溝通方法.
小組的動態.
和表達方式的時候.
怎樣可以再貼近基督徒的本份.
所以我認同.
我們應該要產生一個新的群體的表達方式.
不要再重蹈舊的.
你曾經不開心的經歷.
我想說一些簡單的問題.
這個問題就是強化玻璃心.
想從個人層面出發.
譬如說.
很多人可能比較用心去做一件事.
很努力地在教會進行一些的侍奉.
或者他不能夠.
不是和大家想法比較相同.

$^{1721}$他很全心全意去維繫一個關係.
但是徒勞無功.
然後他就很容易傷害到自己的心.
可能其他人是不察覺的.
有些人可以說流童是一個.
很多受傷害的心靈.
在其他教會聚集在一起的人.
其實很多被受傷的心靈.
去到其他教會.
可能他已經有一個不好的經歷.
於是他已經不能夠像以前.
一開始很火熱地.
去拿出他的所有的熱誠去服侍主.
就好像去到一個點.
知道了和大家不太相同.
應該是去禱告主.
我要向主認罪.
我要去認同這裡.
否則我又要再離開.
莫非我也去到一個點.
我不能夠再受傷害.
我又再去另一個地方.
到最後我究竟是屬於哪個地方呢.
神啊 你要我去哪裡呢.
所以我很想從個人層面方面.
如何在哪些屬靈書籍.
或者哪一卷聖經.
或者哪一類型的參考.
可以強化自己的內在.
在我們可以重新出發.
重新在一個地方.
主說這個應許是我在這個教會.
我去全心全意地擺出我自己.
而我也可以真正地強化我的心.
即使我這次再受傷害.
我仍然能夠站立起來.
有沒有一些書籍可以給我們參考.
謝謝.
我覺得看書也沒有什麼用.
因為這個不是知識問題.

$^{1761}$當然看聖經.
但是反而我聽姐妹們的分享.
我想起剛才我說強化玻璃心.
意思是.
記住不是可以當那些東西沒事.
而是能夠快點沒事.
能夠不會在那裡.
那怎樣呢.
反而我覺得是需要.
曾經經歷那麼多傷害之後.
我覺得反而是何時可以強化.
就是找到一群很疼愛你的人.
那群人足夠在一段時間裡.
有群人來支持你.
還有一段時間後.
你才能夠重新侍奉.
或者做一些比較高危冒險的.
有機會破碎心靈的事.
我想起我女兒.
因為怎樣能夠叫到我女兒.
她的父母親育.
你去愛她愛得足夠.
她就會有自信.
她不會怕失敗.
反而你經常罵她.
她就會很怕失敗.
所以怎樣能夠強化一個人的心臟.
其實經歷很多的愛.
才能夠被強化.
所以我覺得不一定是看書.
而是真的可以在群體裡.
如果真的在群體裡.
我們經常說群體裡的弟妹.
不需要立即侍奉.
你在群體裡找到一群戰友的時候.
你才可以慢慢去想這些事.
有一定程度的同行的人.
才能夠幫你強化心.
因為其實你的心不會被強化.
你的心是受傷之後很快被復原.

$^{1801}$因為藉著很多弟妹的愛.
我去年年尾的經歷.
發現原來很多人是愛我的.
愛過的人比恨過的人多.
這件事是有用的.
是可以復原的.
所以我覺得強化身體的意志.
其實是一個愛的群體.
如果說書的話.
我突然想起一本書.
那本書是台灣校園出版社出版的.
叫做《玉子團契》.
那個玉子不是日本人吃的那種.
那是那裡的玉子酒吧的玉子.
那本書《玉子團契》.
是John Albert這位牧者寫的.
台灣翻譯成中文.
中文名叫做奧伯格.
如果作者.
《玉子團契》書裡面說什麼呢.
其實就是在說群體裡很多的不是.
就是不好的東西.
但是正正在很多不好的東西當中.
有不同的聖經教導我們.
要去接納和欣賞.
因為.
籠統一點說.
上帝接納我們.
這群不同類型的人.
走在一起的時候.
我們本身就是亂七八糟的.
我們亂七八糟的時候.
在一個群體當中.
總有很多需要協調.
需要很多調整.
需要很多磨合.
其中有一課裡面就是在說.
那些豪豬.
豪豬就是那些賤豬.
那些賤豬很想去接近人的時候.

$^{1841}$但是牠會刺到人.
人們經常跟牠保持距離的時候.
牠覺得很疏離.
牠怎麼可以在當中跟人相處呢.
又可以有那種關係呢.
就好像我們本身.
如果我們越是跟人接近的時候.
就是越是將自己不好的東西放大給人看.
那我們怎麼去跟人相處呢.
那本書就是在不同的題目.
就是在說.
在一個真誠的群體.
就是剛才那個課堂內容.
越是坦誠的時候.
越是要讓我們去懂得欣賞對方.
因為他願意將最真誠的一面去揭示給你.
就是我就是一個這樣的人.
我沒有裝假的.
而我過程當中.
有些東西是我不是你期望想的東西.
但是這個就是我.
我自己說到.
在Flow Church有一課是說十二體面.
就是說哥倫多前書十二章裡面.
有人體面的越化.
加急去體面的時候.
就是要懂得欣賞對方.
欣賞群體當中個別的特長.
或者是個別一些我沒有你有的東西.
很多時候我們人就看到.
一些你這麼厲害.
我不行.
但是很多時候就是.
看到別人很漂亮的東西.
就看不到一些.
可以欣賞一些其實你沒有.
但其實他也不是特別好.
這個過程當中其實.
是懂不懂得互為欣賞呢.
互為補足呢.

$^{1881}$保羅用很長的篇幅說.
不需要比較的.
但我們需要學習去欣賞對方.
我覺得.
我認同阿John說的.
心本身就是心.
肉心就是肉心.
就是你始終都會.
你會強化到他.
但復原那個resilient的power是重要的.
就是怎樣可以令到那件事.
回到本位呢.
就是一個群體.
就讓那件事可以快速回到本位.
就是可以再來過.
有一個網上問題.
來自風Sir的問題.
作為基督徒.
真的要為每一個傷害自己的人.
做寬恕.
例如曾經施虐自己的人.
問號.
我認同阿Poon Sir.
對於寬恕的定義.
我也是用這個看法.
有些東西不是說.
對方沒有道歉.
就不能夠寬恕.
我自己不是這樣理解寬恕的.
例如十字架裡面.
耶穌也不懂ignorance.
就是ignorance.
祂也不懂.
但我也故意寬恕他們.
我比較贊同耶穌.
當我們說寬恕.
其實寬恕不是一種權柄.
而是寬恕真正的受惠者.
不是寬恕那個人.
而是寬恕別人的人.

$^{1921}$所以寬恕就像寬恕說的.
是讓自己不會再受到傷害.
你已經被人傷害了.
而當中的.
耶穌寬恕別人.
是讓我們被這些問題釋放.
所以寬恕是能夠放下那件事.
所以世上只有一個人.
能夠有權柄去免去別人的罪.
這就是我們的上主.
我們寬恕別人不是免去別人的罪.
也沒有這樣的能力.
也不需要去免去別人的罪.
這與我們無關.
我們寬恕是讓我們.
不會再受到別人的加害.
很慘的.
你已經被別人傷害了.
還要在那個困境中.
走不出來.
所以寬恕是一種.
耶穌給我們一種恩典.
讓我們能夠離開這種.
雙重的影響.
所以寬恕最大的重要性是寬恕者.
他能夠不會被那個罪的問題.
來困鎖了他.
所以剛才的問題.
被別人侵犯寬恕.
所以寬恕永遠都不能叫別人寬恕別人.
因為寬恕是自己能夠從上恩典中.
可以離開這種罪債.
所以是一種恩典的出路.
讓我們不會被這個罪債所困擾.
因為基督耶穌已經能夠.
可以消除這個問題.
所以回到和平的那首歌.
十字架是一切復和的開始.
因為他滅掉一切的冤仇.
所以十字架讓我們能夠釋放.

$^{1961}$能夠被這個罪的問題.
仇恨所釋放.
不是你赦免他.
而是你真正被他的罪困鎖住.
所以寬恕讓我們能夠離開這個夢鴉.
我補充一點.
在談寬恕這個課題裡.
我剛才說過.
有些成長過程中都學習這個功課.
都看過很多不同的書.
其實真的有不同的.
你當作是作者或學派.
說有些不同的方式.
但我自己覺得.
真的你自己主動去做.
剛才我說的內容.
不要再讓事情核實自己.
所以剛才網上的朋友問到.
是否要寬恕所有的人.
我覺得具體一點.
你要知道那個步驟.
是不是一定John.
我主動去寬恕John.
是不是一定John有反應或回應.
或者是否去到John面前跟他說.
我多生你的氣.
是不是需要這樣.
不同人有不同的看法.
但我自己做木樣的過程中.
如果選擇一定要去當事人.
要對質 要他做.
你真的要為打爛我的花瓶道歉.
你知道花瓶對我的.
要他說到我很不開心.
要他明白 要他認同的話.
這個其實都不容易.
我為什麼這樣說呢.
你要自己跟自己去做一個評估.
當你講到很重要的東西.
如果對方的反應都不是你的期望.

$^{2001}$你會更傷心.
你明白我的意思嗎.
因為我曾經真的跟我的職稱說.
我真的要這樣跟他說.
我已經跟他預先說過.
如果他說原來我傷害了你嗎.
那我會說你更慘.
你會很生氣自己.
我生氣了這麼久你都不知道.
原來我自己生氣自己.
你更難過的日子.
你明白我的意思嗎.
有些不同的人用這些方法.
我就不是說對與錯.
我要期望一件事.
因為你要別人的回應.
如果別人的回應跟你的預期有落差.
那個都是第二個傷害.
我不是說一個方法.
但我在過程當中.
要說你自己能不能夠從你困所的環境釋放出來.
而你真的在當中讓自己明白到你曾經受傷害.
而那件事不會再觸碰你.
是需要時間的.
因為那件事越是複雜越是難的話.
你需要等待醫治的時間.
或者需要等待再看到不再令你情緒繃緊.
其實不是說我有什麼特異功能.
但真的要慢慢讓自己有空間.
有沒有其他問題?.
現場?.
沒有的話就來湯凍吧.
那就到下個月了.
下個月就可以令這碗湯熱起來.
下個月就說熱情.
那我長時間要拿個火鍋出來嗎?.
下個月差不多就可以弄個羊腩煲.
草樣?黑草樣?.
有機會的.
弟子妹希望大家都能夠經歷聖帝聖靈的愛.

$^{2041}$讓我們更加有愛去對待不同的人.
下個月見.
再見.
ball.
晚安.
早安.
\newpage



\section{}
\label{sec:K2_OK28IM68}
\textbf{【這是最好的時代:給香港基督徒的神學八課】第6課: Passion|20211018 [K2\_OK28IM68]}
\newline
\newline
連結: \href{https://youtube.com/watch?v=K2_OK28IM68}{\texttt{ https://youtube.com/watch?v=K2\_OK28IM68}} ~~~~ 語音日期: 2021-10-18 
\newline
\newline
\hyperref[sec:xbbmUvNItM8]{\small{< < < PREV SERMON < < <}}
~
\hyperref[sec:index_chronic]{\small{[返順時目]}}
~
\hyperref[sec:index_scriptual]{\small{[返順卷目]}}
~
\hyperref[sec:S_0UcgQqbRI]{\small{> > > NEXT SERMON > > >}}
\newline
\newline
$^{1}$(廣播中).
各位靚姐妹晚安.
我們來到神學八課的第六課.
很快我們經歷了六堂的時間.
接下來的七,八課就完結了整個系列.
香港神學,我們基督徒神學八課.
其實我們很想在這八課裡面去教神學.
讓我們能夠面對這個時代.
所以今天上課也是.
我們慢慢從一些宏觀的福音,基督徒,教會.
去到我們個人的裡面.
我們怎樣去面對自己的靈性.
我們怎樣去看我們的軟弱.
今堂是很特別的.
今堂其實不是平時你會聽過的題目.
叫做Passion.
為甚麼叫Passion呢.
我覺得這個字是很有意思的.
大家可以先問一下.
這個是甚麼呢.
我想問一下.
這個是甚麼呢.
我們香港人叫做熱情果.
原來熱情果Passion fruit.
這個字其實不是這樣的.
如果你看一些典故.
你會發現Passion fruit本身的熱情跟熱情沒有甚麼關係.
Passion fruit的原因.
簡單介紹一下.
它是來自於南美洲的巴西.
在17世紀傳到歐洲.
在新南海時代的時間.
有人拿著Passion fruit的花.
在西班牙的船駕時.
看到這個花就覺得很特別.
這個花其實有一個十字架的形具.
如果你詳細看的話.
它裡面有三個分裂出來的.
好像是三根釘.
整個花就像耶穌的荊棘觀念.

$^{41}$它加上五個黃色的花瓣.
就像耶穌的五個傷痕.
所以整件事的起點.
它本身先取了這個花.
就叫做Passion flower.
Passion flower本身應該解作甚麼呢.
就叫做壽難花.
原來整件事的Passion.
不是解作熱情.
而是壽難.
其實Passion有很多意思.
所以從Passion flower.
壽難花到壽難果.
來到香港就變成了熱情果.
就像熱情一樣.
其實完全不相關.
跟熱情沒有關係.
反而是壽難.
其實你也知道.
你去看很多不同的字的時候.
你會發現.
我們叫甚麼呢.
壽難曲裡面的英文叫.
Passion of the Christ.
原來英文裡面的Passion.
除了解作你的熱情之外.
它更加是解作壽難.
所以Mill Gibson拍了一套電影.
叫做壽難曲.
其實這個節奏很久了.
如果你有聽Classic Music的話.
J.S. Bach.
寫了兩個非常有名的.
就叫做St. Matthew Passion.
和St. John Passion.
兩個都是從.
這個黑西瑪利園開始講述.
整個耶穌受難的故事.
所以當我們.
特別是信仰來說.

$^{81}$那個Passion.
其實我們是.
首先去解作壽難這個字.
所以你會發現.
很特別.
當你去提到Passion的時候.
我們看回原本的字義.
你會發現其實.
Passion裡面.
有兩個很不同的希臘文.
一個是名詞.
就叫做Pathema這個字.
動詞就叫做Puzzle這個字.
兩個字其實是解作什麼意思呢.
看回字典裡就說.
第一個就是.
就是解作一個受苦.
就是一個受苦的意思.
然後它才解作.
所謂的Emotion.
一個很強烈的Inward Emotion.
一個內在的情感.
所以從來都是.
從聖經的字義.
到我們整個2000年裡面.
有關Passion的字.
其實它都是.
有兩個很不同的意思.
一個就是解作我們的熱忱.
我們的那份情.
那份熱忱.
另外它是解作受難.
所以今天我們就講.
當我們去想.
我們怎樣去形成.
我們一群Folk Church弟兄姊妹.
如果我們這八課裡面.
如果有一課是講我們.
我們Folk Church弟兄姊妹.
一個很獨有的DNA的話.

$^{121}$Passion成為我們一個.
很重要大家去思考和學習的課題.
都是的.
這十年裡面.
我們經常都開始慢慢聽人說.
做Growth.
首先做一個人.
我們開始就混淆.
究竟我們做基督徒和做人.
有什麼分別.
做一個有血有肉的人.
和做基督徒.
兩者之間究竟是一回什麼事.
我們經常說我們做一個人.
就是怕就怕.
我們面對著恐懼就恐懼.
當我們面對著悲痛就悲痛.
做一個有血有女人.
這是一個很真實的事情.
所以我們今天就探討問題.
究竟Passion這件事情.
對我們做基督徒有什麼意義.
所以今天我就會特別看一句字.
叫做Lydon Sharpe.
很有趣的.
德文裡面的Lydon Sharpe.
都是一個很有趣的字.
和Passion一樣.
如果你是一個很有熱忱.
去做一件事.
就叫做Lydon Sharple.
但Lydon本身是一個苦難解.
當你很喜歡砌模型.
就會叫做Lydon Sharple.
這樣做一些事.
所以當我很有熱忱去做一件事.
甚至和受苦有關係.
很奇怪.
這麼多年來.
受苦和熱忱都是同樣的字根出現的字眼.

$^{161}$今天我們就嘗試去認識一下.
究竟我們整個基督教裡面.
我們怎樣看Passion這件事.
回想到最早期.
大概在2000年前的希臘哲學的時候.
斯多瓦派很崇尚一種叫做.
Empathy.
即是無情.
你見到現在的A在前面.
即是一個沒有情感的意思.
古希臘哲學裡面的上帝.
更加超越更加高尚的上帝.
他們認為是一個沒有情感的上帝.
換言之.
情感Passion.
本身是一個負面的字.
即是我們稱之為凡夫俗子.
人世間的人才是這麼低等的.
才會有這種情感.
會受到一些Emotion影響.
所以2000年前的希臘哲學.
他們認為神明或上帝應該是沒有情感.
最高尚的最形而上的神.
是沒有情感.
所以我們基督教也是.
在初期的時候.
也很受到希臘哲學所影響.
我們所想的上帝.
都偏向一個沒有情感的上帝.
他是不被動搖.
不會被人動搖.
我們覺得很Moving.
很Touching.
無論是Touch或Move.
都好像被人搞到一樣.
所以上帝他們覺得.
不應該被人Move或Touch到.
上帝是高高在上.
不會被任何事影響到自己.
所以他們認為一個更加真正的上帝.

$^{201}$一個高階基督徒.
是一個沒有情感.
沒有任何情緒的人.
所以我們看回一些沙漠教父.
當時有一個John Cassian.
一個很出名的撫修主義者.
當時一些最早期的靈修傳統裡.
他們的撫修.
正正是嘗試透過這些撫修.
去磨練自己的靈魂.
不會被任何事影響.
所以這些人就是這樣.
在沙漠裡不斷地去撫修和操練.
他們認為要和靈魂裡的情感去搏鬥.
靈是最高上的.
魂.
即是我們的情緒.
或者是我們的魂魄.
是比較次等的.
所以他們認為.
大家可能聽過七宗罪.
最早期有八宗罪.
有八個非常不好的罪.
都是被一個人裡面的靈魂的情感所影響.
其中有貪吃,淫亂,虛榮,驕傲,憤怒,貪婪,懶惰.
其中一個是憂傷.
原來以前覺得憂傷是一個不好的事情.
我一個熟靈人的時候.
我是不會感到任何憂傷.
不會被任何事情影響我的情緒.
所以反過來.
快樂也不是最厲害的情感.
快樂這麼容易被人家迫起.
這麼開開心心也不是好事.
真正真正最高層次是怎樣的.
是無情.
是一個完整的情況.
所以我們摸一摸.
所以他們認為.
上帝也是一樣.

$^{241}$上帝是不能受苦.
也是沒有感情.
一個無法受苦和無情.
上帝不單止不會讓苦難去教導他.
更加不會被任何情緒去動搖.
他們初時後都認為.
究竟上帝是不是真正的十字架.
不過釘的是一個人子的耶穌.
還是神子的耶穌.
他們認為上帝是不會被十字架所影響.
釘的只不過是一個人的耶穌.
這是一個很早期的神學.
他們認為上帝是不會受苦.
言下之意也不會被受苦當中的情感影響.
所以這就是2000年前的傳統上帝.
不過我們所相信的耶穌並不是這樣.
耶穌是真的經歷十字架.
並且他是被情感所影響.
他會流淚.
他會被其他人去動搖.
這份我們遲些會講得很詳細.
愛的緣故是他甘願開放自己去動搖.
從而被情緒所影響.
所以我們發現.
通常最神聖的樣子就是這個.
不知道大家知不知道.
四月清史中最神聖的就是這個沙加.
因為他永遠都閉上眼.
閉上眼是一個完全不會被任何的外來事物.
看到最神聖的樣子就是這個.
閉上眼.
所以這正正就是我們一直以來所想像出來的.
上帝的模樣.
不被任何的情緒和事情影響.
所以你會發現這是我們初期教會的看法.
不過到了中世紀的時間.
多瑪斯·奎納就嘗試去想多一點.
所謂的passion是一回事.
阿夫娜給了一個非常全面和中性的理解.
我們就開始開展出來.

$^{281}$我們講一些神學.
有關passion的神學.
passion對於多瑪斯·奎納來說.
他寫得很詳細.
在《神大傳》1.2中的第22條到48條.
全部都講述有關passion這個課題.
對我們來說passion是什麼呢.
passion是我們靈魂裡必然會有的情感.
沒錯 靈是很厲害的.
我們有靈性.
不過我們作為一個人.
我們是不會沒有情感.
所以passion就像今天所說的emotion.
或者是我們的情緒.
我們的情緒 我們的情懷.
我們的感動 流落的情感.
正正就是passion這個字.
阿夫娜就問 究竟靈魂有沒有情呢.
或者作為一個人.
我們有沒有一個情感.
我們有沒有一個passion在當中呢.
究竟passion是好還是不好呢.
多瑪斯·奎納給了一個非常好的理解.
我覺得passion我們作為一個有靈魂的人.
流落的passion就像一條海一樣.
我們的河流是可以繁繁盛盛.
水能載舟 羊腹奏.
它可以成為一個非常可怕的江河.
一個氾濫的河流.
當我們爆發的時候.
當我們受到情感所影響的時候.
如果你記得上一課我講過.
上一課我講過犯罪.
犯罪其中一個外來因素是什麼呢.
就是passion.
所以我們作為一個人.
我們確實有時候會被我們的passion所影響.
甚至會令我們犯罪.
我們的情緒 我們的動怒.
我們的情慾.

$^{321}$所以passion就像一條河一樣.
它可以是很恐怖的.
它可以非常影響到我們整個人的生命.
不過作為一條河流.
它也可以有非常重要性.
它可以成為一個非常重要的.
譬如說水力發電.
它可以成為一個人的水塘.
也可以成為樹木的休養地方.
凡是河流流過的地方.
它都成為一個非常大的動力.
所以多半受訪的人認為.
passion其實它可以是好.
可以是不好的.
它基本上是忠誠的.
只要它能夠讓我們能夠藉著這些一道道的閘去閘住它.
好能夠適當地來控制我們的passion的時候.
那我就能夠去善用我們的passion.
所以我們作為人是不可以沒有passion的.
這是詩篇所說的.
詩篇第84篇第二次.
心腸的肉體向永生上帝歡呼.
My heart and my flesh sing for joy to the living God.
這裡說原來我們對上帝的歌頌.
或者我們的頌讚.
不是純粹我們的心靈.
而是我們的flesh.
所以我們作為一個人.
一個有靈魂有心的人.
我們對於上帝的那種流露.
我們不單單是屬靈的.
所以我們是屬靈之餘.
我們會牽涉到我們整個的身體和我們的所謂的情感.
所以我們是應該要去運用我們全人.
從我們的靈到我們的情感或者是魂.
去到我們的身體.
來做任何事情.
所以想想一個基督徒如果是沒有passion.
沒有情感的人.
超級屬靈.

$^{361}$是一個很奇怪的人.
說話沒有表情.
不被任何世事所影響到他.
這個不是一個人.
所以你會發現一個真正的基督徒.
應該是一個帶動從心靈到整個人到靈魂.
是整個人的全人的向度.
才是一個真正的人.
所以通常來說我們的passion是重要的.
問題是如何能夠好好擺放這個passion.
接著說一些比較深的東西.
Fernand說其實有兩種不同的passion.
一個叫做欲情.
一個叫做憤情.
一個叫做irrational.
什麼來的呢?有點複雜.
中文和英文都很難明白.
第一個我們先來解釋什麼是欲情.
欲情是什麼呢?.
就是我們一個sensible的那種情感.
那種passion.
是關乎於我們對於任何事情的那種感受.
分好和不好.
good and evil.
當我們去面對一些美善的事.
我們會怎樣呢?.
我們就會愛.
這個很特別.
Fernand給了一個非常複雜和詳細的哲學給我們聽.
當我們.
普遍來說我們叫愛.
愛一個手機.
愛一個人.
愛一個事情.
都是愛.
愛當我們還沒有得到的時候.
叫做什麼呢?.
就叫做desire.
當我們很想要的時候.
就叫做desire.

$^{401}$當我們得到的時候.
就叫做pleasure.
當我們已經得到的時候.
我們就能夠享受它.
並且得到快樂.
所以當我們愛.
我們愛任何的東西.
愛上帝.
愛任何的物件都是一樣的.
所以愛就分成了三種不同的passion.
愛和渴望和享受那種快樂.
這是我們人對於美善的事情.
三種不同的情感.
這是第一個.
分支.
第二個.
就是對一些不好的事情.
對一些evil.
對一些惡事.
那些事情是不能夠成全你的.
那些不是在你的perfection裡面有關係的.
所以你會怎樣呢?.
你會去珍惠它.
Generally你會珍惠它.
不過珍惠不是首先的.
我們首先有愛.
愛這件事.
我們反過來講就會珍惜某件事.
我們愛護真理.
就自然會珍惜那些虛假.
所以另一批的情感.
情緒.
Passion.
就是這三種.
一個是hatred.
一個是珍惠.
珍惠.
當我們看到一些.
當我們還沒有的時候.
就叫做Aversion.

$^{441}$就是一個名字.
就是討厭.
我想遠離它.
我想離開它.
任何的惡事.
譬如你嫁給一個很臭的.
或者是那些.
那個人你不喜歡他.
或者一些很討厭的國士.
都想遠離他.
例如香港那些人覺得是.
所以當我們發現一些不好的事情的時候.
我們會遠離它.
這是跟design剛好相反.
Design是當我們還沒有得到的時候.
就會想渴望它.
這個Aversion就是正確的.
我們想遠離它.
當我們去.
不能不去面對一些惡事.
跟它同在的時候會怎樣呢.
我就感到痛苦和悲傷.
所以悲傷和痛苦.
正正就是這批的情感.
這就跟Pleasure剛好相反.
當我們不是跟一些美善的事情在一起的時候.
當我們跟一些不好的事情在一起的時候.
我們就會出現到.
Sorrow.
就是痛苦的Passion.
這個就是第一批.
就是有關Sensible裡面的情感.
接著就是粉情.
希望大家能夠理解.
這些Passion是非常仔細的.
第二批就是什麼叫做粉情呢.
就是一些對於.
未能夠得到的東西.
一些將來的東西.
一些這樣的情緒.

$^{481}$或者是一些Passion.
叫做Irresistible.
就是粉情.
有些什麼呢.
它有分幾個.
一個就叫做Hope.
一個就叫Desperate.
Hope.
盼望.
中文翻譯成期盼.
你不一定要盼望一些很厲害的東西.
就是耶穌再回來.
耶穌再度過這些東西.
你盼望今晚不下雨.
都是一種盼望.
盼望今晚有宵夜吃.
都是一種盼望.
所以對於一些好的東西.
一些Goodness的東西.
你未擁有它.
但你仍然懷著一種盼望.
去想它.
這就叫做盼望.
這是一種人的情感.
相反.
就是我們的絕望.
我覺得明天一定會下雨.
你很悔.
這種Desperate.
正正就是一種這樣的情感.
這是一種Passion.
這是一種情感.
是一種情緒.
所以對於一些惡事.
我們對於將來那種絕望.
正正就是一種這樣的情感.
接著.
這些是一些好事的看法.
另外就是一些.
很惡的事.

$^{521}$很難的事.
第一個就叫什麼.
就叫做Dare.
這個就叫做大膽.
或者叫做勇敢.
你不怕那些東西.
對於一些非常大的困難.
你感覺到.
你是不會害怕.
你是有大膽去面對它.
當然這個不一定是好事.
你大膽到開大膽車.
大膽到你做一些惡事都可以.
所以這個不同於一個美善的勇敢.
這個純粹大膽才可以.
另外就是Fear.
就是恐懼.
恐懼也是.
可以恐懼上帝.
恐懼一些不好的事情.
都可以恐懼.
那你就不做了.
但它都可以出於恐懼.
純粹是一個害怕.
這些都是一些未及之事.
對於一些惡事.
對於將來的情緒表達.
最後一個就是Anger.
Anger就不是分好和壞.
因為好和壞都會Anger.
對一些正義的事.
你感覺到很憤怒.
你覺得那個人不對.
你就覺得很憤怒.
反過來.
你仍然對某些事都很憤怒.
所以這些情緒.
這些Passion成為了.
Thomas Aquanain非常重要的一個刻畫.
一個描寫.

$^{561}$所以簡單來說.
我有一個小總結.
Passion是分開兩個不同的情感.
一個叫做.
Conception.
一個叫Irresistible.
一個叫Sensible.
一個叫將來.
每個人都有一個好事和壞事.
這些好事包括什麼.
Love, Desire, and Pleasure.
接著一些惡事.
我會有一個Avoidance.
想避免它.
Hate, Pain, and Sorrow.
另外.
我們對於好事有一個Hope.
Desperate.
接著還有一個Evil.
有一個Dear和Fear.
最後就是一個中立的Anger.
這11個正面.
就是Thomas Aquanain所說的11個的Passion.
作為一個人.
作為一個靈魂.
必然會擁有的10個Passion.
它是有可以好和壞的.
我可以純粹Desire一些不好的東西.
Desire一些純粹物質的東西.
你可以去Hate一些不對的事情等等.
所以我想說這麼多是什麼呢.
這11個情緒.
是不是你這兩年裡.
正正面對著這個世界裡的情緒.
或者Passion.
所謂做一個有Passion的人.
其實正正都是這些東西.
當我們帶著Passion去做基督徒.
我們會有很多不同的愛.
渴望.

$^{601}$享受.
更加多的可能是痛苦.
或者是憎惡.
或者是討厭.
有時候我們會有盼望.
有時候會絕望.
有時候我們會勇敢.
大膽.
有時候我們會覺得恐懼.
有時候會覺得很憤怒.
這11個Passion.
作為一個有Passion的人.
是一個非常自然的事情.
是自然不會流露出來的事情.
不過它好還是不好呢.
記得,這要視乎什麼呢.
就是視乎你能不能夠用.
一個很好的東西來駕馭它.
所以剛才說了.
這11個不同的情.
Passion的意思.
愛,憎惡,怨望,逃避.
喜樂,哀愁.
期望,失望,畏懼,勇猛,憤怒.
這11個成為了我們很重要的Passion.
不過這11個Passion.
你能不能夠把它成為好事.
就在乎於什麼呢.
阿輝就告訴我.
就在乎這個Virtual.
Virtual正正是能夠.
好像一個提靶一樣.
來控制著.
或者有效地監管著你的Passion.
讓你的Passion不會太過厲害.
過度紅河氾濫.
成為一個災難.
得勝Virtual是一個非常重要的天主教觀念.
Virtual是一個上帝賜給我們的尾線.
信望,愛.

$^{641}$上帝給我們的聖靈.
正正可以令我們這些Passion.
成為一些非常重要的力量.
我們不可以沒有Passion.
因為沒有Passion.
我們就是一個沒有生氣的人.
我們懷著Passion去做人.
不過他要有這個得勝.
這個Virtual.
好好地去善用我們的Passion.
所以今天我們沒有機會詳細講Virtual.
因為這個是天主教的概念.
不過意思就是說.
我們仍然是需要上帝.
需要一些尾線的事.
來流露我們這11個Passion.
憤怒,哀愁.
這些不是差的東西.
只要我們為著一些正確的事.
只要我們懷著尾線的得幸去做.
就是一個美好的Passion.
令我們更加流露我們得幸的Passion.
就像這提法一樣.
我剛才講了.
講了這麼久.
講了很久.
好像不太關13世紀的神學.
關我們什麼事呢.
我想到四點和大家分享.
當我們去問.
我們基督徒要做一群有血有肉的人.
其實會是什麼事情呢.
第一就是我們.
一個叫做生命熱情.
既然Passion是我們一個很重要上帝.
給我們很自然的力量的時候.
我們本身對於生命.
是應該充滿著熱情的.
我們是懷著一種非常之.
享受和熱愛生命的激情.

$^{681}$來做人.
不知道大家有沒有看過這部電影.
《靈魂奇境》大家有看過嗎.
在這部電影裡面.
很特別.
整部電影是一個非常基督教的電影.
當中記得.
這部電影是說一個靈魂.
在這個世界裡面不斷去尋找.
基本上是兩個主角.
主角要去幫這個靈魂.
去投胎.
他也要去拿著貼紙.
那個Divine Spark.
這個靈魂BB.
一直都不能夠做人.
因為他找不到那個貼紙.
這個令到他.
是一個非常之有Passion.
做的事情.
他覺得有試過打球 唱歌.
什麼都試過.
但都激起不了他的Passion.
所以做不夠貼紙.
來做一個真正的人.
後來他成為了真正的人.
是什麼原因呢.
就是他能夠真正的來享受生命本身.
重點不是一定是說.
你能夠做到什麼目標.
記得那個主角.
這個Jazz Player.
他以為可以做一個出名的Jazz Player.
之後能夠成為一個最人生高峰.
但原來不是.
只是每天都做同樣的事情.
重點是什麼呢.
就是我們是享受生命本身.
不是純粹追求某個生命目標.
或者某一件事.

$^{721}$所以這種熱忱.
不是單單某種人生的高峰.
或者是我們生命中某種狀態.
而是單單回到我們生命本身.
純粹帶著一種Passion去享受.
和面對我們生命.
我們活著.
為你活著而活著.
我們去學習.
不是因為外面的環境.
或者我們的際遇.
才有熱忱或沒有熱忱.
而是單單的生命.
已經讓我們能夠充滿熱忱去做人.
這是第一點.
我們做Galto.
做一個人.
最基本最基本.
就是懷著一種Passion.
上帝給我們生命的Passion.
來面對我們的生命.
接著才面對我們人生裡.
香港這個年代很多不同的事情.
所以這是第一點.
我們找回我們生命本身的熱忱.
這不是一個.
被這個年代的時勢改變的東西.
我上個月和我的中學同學吃飯.
我們就說開了.
因為上學都是最後一次見面.
因為很快之後.
有一個去英國.
一個就回到美國.
一個就在中國上海工作很久.
其中一個同學就說.
我真的很想來到現在.
Dream House.
就是我60歲的時候.
在澳洲找一間很漂亮的海邊小屋.
白色的.

$^{761}$很漂亮的.
這就是我們的生活目標.
但其實.
我們如果將這個成為我們的生活目標.
這樣也很糟糕.
那就是說什麼呢.
就是說你現在這20年.
基本上是一個.
instrumentalized的生命.
純粹是為了達到目標而過的生活.
不是.
我們生命本身就是我們生命的熱心.
不是我們某個生命的高峰.
才是我們的熱心.
所以這是我們第一點.
我們懷著這種passion.
來做人.
第二.
剛剛相反.
就是我們去拒絕犬儒.
犬儒正正是.
一個passion相反.
他可以充滿智慧.
他可以充滿高見.
但他有一種對於事實.
對於reality.
其實是放棄你的態度.
因為他害怕牽涉在裡面.
所以他抽身.
單單在Facebook裡面.
觀看世事.
不敢.
也不願意將自己的生命.
投放在這個時代裡面.
就是一個犬儒主義者.
他可能是沒有被傷害的.
因為他害怕被傷害.
所以就不牽涉在當中.
冷眼旁觀.
去看這個世界.

$^{801}$這是一個犬儒主義的問題.
不知道大家喜不喜歡play safe.
近來我們很喜歡這個.
安心play safe.
play safe永遠是最好的.
play safe是一個最安全的做法.
不如大家都.
play safe大家就戴口罩.
不要出街.
play safe就不要做任何take risk的事.
教會不如我們怕有人感染.
怕被人搞.
我們就不要怎樣怎樣做.
我們可以做一個play safe的人.
做一個play safe的教會.
做一個play safe的生命.
但是play safe正正是一個非常安全的原理踏步.
當你的生命不去面對任何風險的時候.
這個好像很安全.
但其實這個非常.
開始已經是投降.
已經輸了.
所以這個我們不要去.
我覺得play safe是一個很好聽.
一個很好的教會說話.
但我們過分地play safe.
其實是一個不是太好的決定.
第二就是defeatism.
就是失敗主義.
因為我們過分害怕失敗.
所以我們就不牽涉下去.
我們就不嘗試去參與.
不嘗試去全情投入下去.
記住passion是同時解作甚麼.
解作受苦.
當你有熱忱去做人的時候.
是會經歷很多很多不同的.
被傷害的可能.
但play safe不是你的出路.
失敗主義也不是.

$^{841}$因為我過分害怕失敗.
所以我就寧願不去全情投入去參與.
這個是我們面對犬儒主義的問題.
第三既然passion是解作我們.
要將我們的靈魂身體完全獻上.
盡心盡性盡義.
正是我們一個非常非常重要的基督態度.
上帝呼召我們.
要去盡心盡性盡義.
來去敬拜去愛我們的神.
和好好的去扶人.
這種盡心盡性盡義.
全情投入獻上活祭.
完全的立志擺上.
正正就是passion的重要生命.
這個我只是想.
Fold Church的姐妹們認真思考的一點.
我覺得Fold Church如果是DNA的話.
其中一個很重要的DNA就是這樣.
我們是一個追求全情投入的群體.
你看到我們搞神學講座也可以搞到這樣.
敬拜隊,爆風球也可以出來出隊.
每一個位台前幕後都是完全到去盡心盡性盡義.
去追求最好.
這個就是我們passion.
passion不是沒有成本的.
當我們願意去做好一件事的時候.
是會擺上很多的事情.
但這個我覺得這個就是我們Fold Church很重要.
第六課要講的事情.
我們懷著passion來做基督徒.
來思考我們的教會.
一篇講道,一個powerpoint,一個拍攝.
都是我們盡心盡性盡義的獻上最好.
當中擺上我們是可以有的.
這個我覺得是很值得我們.
可以好好投入Fold Church這樣的教會.
所以我們懷著passion來做基督徒.
來獻上我們自己.
最後我要給大家一個德文.

$^{881}$第四就是Leidenbereit.
什麼意思呢?因為很難翻譯.
什麼叫Leidenbereit呢?.
Leiden怎麼解釋?.
就是suffering,Leiden.
Leidenbereit是什麼意思呢?.
就是一個…怎麼說呢?.
bereit就是預備的意思.
既然我們是做一個有passion的人.
我們就要Leidenbereit.
簡單來說就是這樣.
我們是準備受苦或者面臨傷痛.
這樣不代表我們會面對傷痛或者受苦.
但我們既然有passion的時候.
我們就自然地.
預備好我們隨時會有相對的傷痛或者受苦.
愛本身就是受苦.
當你愛一個人,當你闖開心去愛人的時候.
你就預約給別人.
一支箭來傷害你.
你可以不受苦的,你就收起自己.
你就不參與,你就不獻上.
但當我們既然是做一個有passion的人的時候.
我們就會預備好自己隨時會面對這個passion.
這個正正是基督徒受難的意思.
所以passion很明顯兩個意思就是這樣.
我們全程鋒利地去做人.
同時也預備好迎接生命裡可以出現的受苦.
但我們不怕,因為我們願意去做.
所以我們總結一下,四點.
就是生命熱情.
我們來對生命,最基本對生命充滿我們的passion.
不需要人生的高峰或者某個階段.
生命本身就是我們passion的對象.
第二就是拒絕犬語.
我們不願意play safe.
我們不願意純粹去害怕失敗.
第三,我們要盡心盡性盡義.
做好我們每一件事.
當中隨時去預備好我們要付出的.

$^{921}$第四,lighten the light.
我們預備好我們可以受苦的.
所以基督徒當然要做一個人.
但我們more than一個人.
我們會有血有肉之餘.
我們仍然不會說我們是一個人.
當然會這樣.
我們仍然會做一個基督徒.
嘗試來獻上最好.
這個就是我想和大家分享的第六課.
passion.
一點思想時間.
你想一下你自己是不是一個passionate的人.
是不是一個預備好自己.
有passion去做基督徒.
面對著這個年代.
我們的passion在哪裡.
我們怎樣能夠得著更多的熱忱.
面對這個世代.
我們一起祈禱好嗎.
祝福我們求你幫助我們.
當中我們每一個參與的頂尖妹.
讓我們能夠可以善用你給我們的passion.
我們的情.
我們對於你的愛.
對於很多惡事的憎恨.
我們的憤怒.
我們的憂傷.
我們的盼望等等.
求主你讓我們仍然有一個美好的virtue.
一個德行.
來好好管治我們這份情.
讓我們做一個有血有肉的人.
卻是成為一個能夠有美善的人.
將我們這份情懷.
這份passion.
好好來善用.
我們這個full church交託給你.
求主你幫我們這份教會.
成為一個有熱情的教會.

$^{961}$去獻上最好的給你.
夢主命求.
阿們.
沒想過你又來.
因為其實打了兩天風.
我就沒東西吃.
是嗎.
所以你看到我們只有水.
是這樣嗎.
其實你在賣什麼.
其實我們就.
公仔麵也有.
不是,我們是阿勒卡.
你想要什麼我們都可以做給你.
即是深夜吃糖果那些.
應該你來多幾次就知道了.
不過剛才我聽你說的內容.
我覺得我們做這間食店.
是有這個passion.
你有熱情果食.
我們沒有這些這麼西方的東西.
但我覺得我有這個passion.
就是因為我們很多時候都all in.
人家要什麼我們就大家去配合.
但其實對於普遍的街坊來說.
有沒有一些日常可以具體的例子.
可以用得著.
你覺得呢.
我自己覺得.
通常我和街坊都說.
你想吃什麼.
譬如我們擺生日飯.
就問他們有什麼想做.
看想吃飯.
我們就用他們的錢.
就幫他們做得好看.
但對於我們普遍街坊來說.
能力可能不是很多.
剛才你介紹了四點.
有些生命的熱情.

$^{1001}$就是說他自己的命.
或者能力可以做到什麼.
其實是怎樣可以用得著.
在日常生活裡.
我覺得這個反而是一個最基本的東西.
因為生命的熱情.
我覺得是人越大.
慢慢會明白這個東西.
純粹對於生命是一種的熱情.
不是生命裡面某些東西的熱情.
我們可以很喜歡沖咖啡.
對咖啡很有熱情.
你喜歡玩模型.
對模型很有熱情.
但你會發覺那些東西都會淡.
都會覺得我已經做了.
我已經做了某個位置.
但如果你對於生命沒有熱情的話.
你做什麼其實都是浪費的.
所以我覺得靈魂奇遇是很好看的.
因為不是在乎於你的際遇.
而是對於你活著這種的情.
那種的熱情.
那種的興奮.
那種的熱愛.
我覺得是首先要的.
基督徒我覺得是要去找回.
最根本對於生命的熱情.
這個我覺得才是最重要的東西.
我了解你來說.
我們盡量去探索一下.
然後就會找到一些我們覺得值得追的東西.
是不是這個意思.
可能就算是.
都不是.
我沒有看過《套戲制》.
我沒有.
《套戲制》說.
最後那個人就說原來我單單活著.
都是很好.

$^{1041}$原來我站起來.
原來我站起來看到外面的風景.
我能夠繼續做人.
都已經是值得我們去懷著熱情去做的東西.
所以你就可以很甘於我今天.
我未必一定要做什麼成功的大事.
而是我單單對於生命本身的熱愛.
這種正正是我覺得是一個很重要的第一點.
然後才去問.
我懷著這種熱情去做什麼呢.
不知道大家怎麼看.
大家有沒有看過那部電影.
每一天都是一個新的一天.
可以感受到那種熱情呢.
或者你現在每天有沒有熱情呢.
好像有點難吃.
或者你感受不到熱情.
有沒有想過什麼原因呢.
我自己覺得在受教育過程當中.
很少講感受.
很多時候都講自性.
講自性的發展.
講成績的表現.
講那件事能不能做到.
其實不是很講個人感受.
就好像你剛才所說的.
就是不起眼於食.
不容易被人觀察到你自己的情緒.
這樣就算是型.
不知道大家有沒有覺得.
在過去其實不是很懂得講自己的感受.
學校又沒有教.
而在過程當中很著重那種表現.
表現到你有什麼能力之處.
大家是怎樣成長出來的呢.
我再講一下第一個問題.
生命的熱情.
剛剛10月9日不是颳颱風嗎.
那天是我的生日.
其實我那天早上已經預計會回到Folk Church.

$^{1081}$總之就是出來吃飯.
中午的時候.
我想吃完飯就出去.
做完飯就打完.
就退下.
我覺得每年我都會寫一些東西.
今年我沒有寫.
因為我發覺.
開始人到了那個年紀.
原來颳颱風是趕走我開始的心境.
原來不需要有什麼.
不知道那天颳颱風是做什麼的.
由早上到晚上.
當時我忍到他出門.
發現他全天都在家.
不知道他悶不悶.
但原來你可以純粹在家裡.
當你燃起了這份對於生命的熱情的時候.
你不需要有內心.
不需要有那些東西.
我十幾歲的時候很喜歡去玩.
但你發現到了三個年紀.
你發覺我不需要去充實自己的生命.
而生命本身就是一個可以令你有熱情.
然後你才嘗試去做一些你心裡的東西.
不知道你有沒有這樣的想法.
我有啊.
你大我有吧.
我有啊.
為什麼我說我有呢.
我喜歡做運動.
但我老婆不喜歡做運動.
她現在沒有看應該不會的.
我老婆完全不喜歡運動.
連車都不會追.
我做運動的時候會流汗.
回家就會臭.
她問我為什麼這麼喜歡運動.
我說我覺得流汗是我能夠感受自己活著的感覺.
她說那你就當我是一個死人就行了.

$^{1121}$(笑聲).
她問我為什麼流汗是活著的感覺.
我說死的生物是不會流汗的.
然後她說那你就當我是死人就行了.
我覺得能夠有生命氣息.
能夠流汗就是一種活著的感覺.
我覺得這就是你所說的生命的熱情.
不是做什麼的.
流汗我已經覺得很開心.
我們要抓緊這件事才能面對這個世界.
十幾二十幾歲的時候我們是相反的.
還沒有自我的時候.
就開始去找一些東西去補充自己的生命.
去玩或者去目標.
讀大學.
但人是很空的.
還不知道生命是什麼.
當你發現上耶穌.
你知道上帝給你生命的時候.
我們用我們的心腸和肉體去歌頌我們永遠的上帝.
我們就發現我們正正就是可以單單活著.
來享受這個活著.
因為我自己也是十幾歲的時候.
我為什麼要上耶穌.
因為我覺得不知道為什麼要做人.
我覺得死了之後不知道要去哪裡.
然後就很想找一個人生目標.
什麼是我的人生目標.
我是一個很目標主義的人.
但是發現原來現在是不需要有這個目標.
正正一個很粗略的.
就是叫做「我們活著就是為了活著而活著」.
我們活著不是為了一些超級厲害的目標.
成為一個偉大音樂家.
或者成為一個怎樣怎樣.
買了幾層很大的房子.
而是單單當你有一個位置.
去到我純粹活著都能夠為了活著而感到有個熱誠的時候.
這樣你才能再做一個人.
才能面對今天的香港.

$^{1161}$或者面對今天面對其他的東西.
有些懸念.
但是我覺得這個很重要.
對於生命的熱忱.
如果你對生命沒有那種熱誠的話.
其他的熱誠是有的.
但是你會失望的.
你會發現做到了.
但你會發現原來都是這樣.
做不到你會覺得很灰心.
所以我覺得這點是很重要的.
對於生命的熱忱.
後面.
你好啊 我叫Amy.
其實我自己很有共鳴.
我猜大家不說話.
可能他們年紀不夠大.
因為人慢慢長大了.
我自己也少了定一些目標.
這個不知道是不是以前的教會氛圍.
每年有中,短,長期目標.
在整個教會設定.
都會很想去最後有些慶祝.
有些成果,收穫.
那就成為我們一個很有熱誠去追求的東西.
但是長大了之後.
慢慢覺得過了之後又怎樣呢.
我的人生還要繼續.
那是不是單純是我生命的那一點呢.
所以這是有共鳴的.
相信這裡在座的太年輕了.
但是有另一件事.
就是我想問你.
怎樣去感受自己的那種溫度呢.
你一個人很敷衍.
還有一個人很有熱誠.
那怎樣去分呢.
你說沒有目標.
也不可以完全沒有目標.
完全沒有那種向前的動力.

$^{1201}$我記得有一個年輕人.
跟我說他人生的座右銘就是.
世上沒有難事.
只要肯放棄.
(笑).
你可以去到每一天都不需要定目標.
去到可以很敷衍.
很沒有方向.
但是有時候教會或者.
以前的傳達教我們就是.
不可以太敷衍.
要有熱誠.
但是你剛才說的那種.
就是一種很自在.
可能去到某一個境況.
才可以去到這個狀態.
又不會介乎太敷衍信仰.
他也可以用這個態度去擺信仰.
不讀經 不領修 沒有目標.
一年讀完都是這樣.
那你自己怎樣去感受那種.
自己仍然有那種熱情.
怎樣去保持那種狀態.
大概是這樣.
其實你很有熱誠.
我覺得自己也不錯.
但我不太明白你怎樣去量度.
或者我怎樣知道.
或者弟兄姊妹我怎樣去教他們.
太多目標又好像太追求某些東西.
但沒有的話.
他真的跟你說躺平的態度.
那又怎樣呢.
我想這裡有兩個不同層面.
我覺得剛才第一個是生命的熱忱.
躺平是一點點.
躺平是不是好事.
我沒想過這個位置.
躺平是不是沒有熱忱呢.
我覺得可以有很多不同的躺平.

$^{1241}$我們.
躺平可以是一種失敗主義.
我投降了.
我其實已經犬喻了.
但也可以成為.
我仍然是享受.
或者我是熱忱地去活著.
所以我覺得有兩種不同的情況.
先說完.
這裡有兩種不同熱忱的定義.
我剛才聽完Emmy說的時候.
我也接觸過.
弟兄姊妹她也會回應.
可以吃的話不會動.
但對我來說我是挺闊的人.
我對弟兄姊妹的要求不算高.
有時別人覺得我很脾氣.
或者覺得包容度很大.
某程度上我覺得也是.
但我自己看科幻書.
我感受到耶穌不是一個很推動的人.
特別是在跟門徒相處的過程當中.
但如果真的要跟耶穌.
他就會有要求.
但要跟耶穌相處的時候.
耶穌大部分時間都是帶著門徒去看東西.
去經歷.
在過程中跟他們問問題.
做互動.
最後你看到耶穌也是用free end的方法.
讓他們想想.
你自己想清楚.
你也看到.
或者你也經歷過.
其實你在想些什麼呢.
所以剛才在信息裡.
有一個位置我覺得是很大的提醒.
當我們知道很多定義.
當我們知道很多別人想過的東西.
其實對於你來說是什麼回事呢.

$^{1281}$這是最重要的.
過去教會的教導很著重自成.
提醒經民.
但其實那些經民有多靠近呢.
其實從來都是自己去經歷.
和自己去用頭身那段經文的要求的時候.
你才會知道那件事對你來說有多大的passion.
你是不是受苦.
你有要求.
你就要去熬.
還是你自己去從那些經文裡.
或者教導裡找到自己的生命意義.
你一定要自己去嘗試.
不是別人告訴你.
也不是說要訂立一些條款.
做了就是了.
所以我自己覺得.
我大部分時間都是和弟兄姊妹一起去.
給他看見.
給他經歷.
或者給他問.
其實最大的.
你有興趣的地方是什麼.
所以無論對於你去想什麼叫做passion的話.
其實我仍然認同.
回到John的第一個本位.
就是其實你活著.
你每天一早起床.
你告訴自己我是怎樣生活.
這個是最根本的地方.
不然就很糟糕.
我覺得我和潘Sir談18課的內容.
其中一課初時叫做all in.
但他不是叫all in.
我想究竟full church有什麼DNA.
我們是DNA出來的.
我們是弟兄姊妹來說出來.
大家可以這樣去追求.
或者去想.
因為我們覺得基督徒是一個.

$^{1321}$八課裡面的一課想說的.
類似之前叫all in的名字.
後來我們就叫passion.
我覺得.
是也是的.
其實都幾all in.
不要犬儒.
盡心盡意.
又要承受苦難.
都幾是那些.
剛才Evan所說的那種.
但大前提就是第一點.
這件事不是要逼自己.
也不是一件很慘的事.
passion就是兩個意思.
就是熱忱和受苦.
那個熱忱是你很喜歡.
這樣去活出來的.
但同時間會遇到一些.
任何事情的一些事情.
因為這個字相反就是.
你去折迷.
就是犬儒.
我拒絕受傷.
我怕受傷怕失敗.
我就不去passion地做人.
我收起自己.
收起自己的情感.
不敢說話.
我覺得我們很想的就是.
當你真的看不到第一點的時候.
我們就自然而然去流露.
那二三四點出來.
當我們未找到最基本的.
對於上帝給我們的生命.
那種passion的時候.
其他的都是一種很律法的規條.
你一定要去盡.
一定要做到最好.
追求出月.

$^{1361}$其實不是這樣.
任何人.
就像銀紙比喻.
你多少錢的銀紙都能夠.
all in到的.
是法律的all in到的.
不是要強行逼著.
因為教會叫你釋放就做.
而是我帶著熱忱去參與.
這裡也是.
很多弟兄姊妹去參與.
福祉去釋放都是這樣.
我從來都沒有叫別人去做什麼.
但每個人都可以帶著熱忱去做.
是辛苦的.
但是快樂的.
因為我是享受的.
所以passion這次就是將享受.
和少許困難.
和我的生命.
混在一起的狀態.
就是我們做基督徒.
是一個很重要的特徵.
網上有問題嗎?.
Liz Mok問.
熱情是否需要一些載體.
才能展現出來呢?.
載體是否即是要一些行動的意思?.
我不明白.
應該就是了.
載體.
因為載體這個詞其實很空泛.
不過如果照我的理解來說.
熱情是需要有一個載體.
就是在場景上.
要有一個場景.
要有一個互動的時候.
才能感受到熱情.
不說其他,說自己吧.
有些弟兄姊妹經常問.

$^{1401}$我怎樣運用時間.
他們覺得我很忙.
我自己覺得我不是很忙.
我會和他們分享我的時間表.
或我的工作安排.
他們說我這樣也不算忙.
那我就是不知所措.
我說不用比較.
但我很享受.
我沒有一天很….
簡單來說.
他們看我的時間表.
很多時候都是見人的.
我說是呀,我很喜歡見人.
然後他們問我見人不累嗎?.
我說見人怎會不累呢?.
但重點不是我自己的感受.
重點是.
每當我聽弟兄姊妹的故事的時候.
我就感受到上帝在她生命當中的工作.
或者感受到上帝在她生命當中做著一些事.
所以你問我的熱情.
我的熱情是見到弟兄姊妹.
和她相處和見面的時候.
那個場景就是我的熱情的載體.
所以我從頭到尾都覺得自己是一個教會人.
我喜歡在教會的環境當中存在.
這就是我每天最期盼的熱情場景.
我用我自己的例子去了解熱情是載體的地方.
(字幕提供:Johnny).
(前面那裡,Jacky).
剛才聽到Johnny提到.
在熱情之前應該會有些熱愛.
剛才我聽到Denise的問題.
我在想是否需要熱愛才會開始有熱情在裡面.
假設我對熱情的愛可能消失了.
例如我突然玩了一個遊戲.
可能玩到厭惡了,不喜歡了.
我就不會再有熱情去玩這個遊戲.
或者去做某些事.

$^{1441}$如果是這樣的話.
我就好像沒有了載體.
我怎樣可以繼續有載體呢?.
這是一個很好的問題.
回到剛才討論過的問題.
我們….
就如Thomas Aquinas所說.
有十一個熱情.
這是我們人必然會有的東西.
其中一點我剛才沒有說.
任何熱情最根本都是愛.
他也這樣說.
因為Hatred也是愛的相反.
首先因為愛而衍生成Desire.
或者是Pleasure.
玩模型.
所以這件事.
當然你玩遊戲怎會玩到厭惡.
但我覺得可以的.
你不就玩到厭惡.
但我覺得最基本的就是.
對於生命的熱愛.
因為生命就是載體.
生命就是最基本.
如果我們連生命都沒有熱愛.
來開始玩模型的話.
明白嗎?.
模型可以玩厭惡.
但我覺得我做人做到很厭惡.
這就很糟糕.
這是我們很多人.
我自己也是.
原來我自己也不是很熱愛生命.
曾經那段時間也是.
所以我覺得.
這是我們最最基本的熱愛對象.
就是生命本身.
所以這是我們需要去燃起的第一點.
我們熱愛生命.
然後才去玩任何的東西.

$^{1481}$其他東西你可以有不同的轉變.
其實我最近很喜歡我女兒的舉動.
因為她做了一個很熱愛的人.
就是熱愛生命.
她整個人的喜怒哀樂都是很熱愛的.
她很開心就很瘋狂.
然後又會不開心,生氣.
其實我們也是.
我們發覺人長大了.
反而我們會收起這些熱愛.
但最基本的就是對於生命的熱愛.
如果你對於生命的熱愛.
你其實不太害怕其他人怎樣看你.
另一個部分我想說的是.
其實這些都是環教會的問題.
很多時候我們都很藏起來.
做教務就是要你循規蹈矩.
這是一個很無情的.
你開心就不要太瘋狂.
不開心就不要讓別人看.
但其實當我們熱愛生命的時候.
我們可以做出這樣的一個樣子.
我覺得這些東西是可以沒有的.
但當你捉到生命的熱愛.
其他東西其實都不是大大問題.
這是你活著很重要的東西.
我不做就做其他東西.
就像風琴那天.
在家裡也可以.
這是一個很重要的東西.
否則當你不是生命.
你做什麼都會覺得好像一個感覺不到的洞.
但其實你做了很多事情.
你覺得自己是一個很好的人.
因為你說其實罪的其中一個來源是熱愛.
其實我們壓低自己的熱情.
剛才你說教務的案件.
其實你不知道你投放熱情出去的時候.
你不知道是不是在投放斧頭.
在投放人.

$^{1521}$當你不知道你那個.
剛才你講了一句.
你沒有發展下去.
當我們自己不知道自己的斧頭是否可以.
當你嘗試盡心盡力去愛人.
或者怎樣都好的時候.
你發覺自己中招了.
犯了罪.
然後你救不回來.
沒得救.
當然你可以找神.
但無論如何.
你都是想儘量避免犯罪.
你避一避.
例如我們講回投放斧頭的案件.
你不想投放人.
最簡單的就是你見不到人就投放不了人.
(收起了).
收起了.
將所有的情感收起了.
你就不會無緣無故.
本來我打算很開放地跟你說話.
然後我很熱情地跟你說話.
原來那個人是接受不到那種熱情的.
他就覺得你在傷害他.
但其實你是很嘗試地跟你的朋友相處.
但其實你是不停地傷害他.
但你還是喜歡他.
你不是有心傷害他.
而是對他很熱情.
當你慢慢長大的時候.
你就會發覺.
有些事情是不可以太開放的.
你不開放反而是對人好.
你少投放一些斧頭.
對方就會自在一點.
慢慢就會覺得.
有很多事情是不應該做的.
可能小朋友有很多事情都可以自在地發表.
但你長大之後就沒有了.

$^{1561}$所以慢慢就會變成.
你不要騷擾我.
很被動.
但怎樣可以避免這個情況.
我明知我的熱情可能是罪的來源.
但我仍然可以繼續有熱情地做.
這是一個很好的問題.
其實正正就是.
為何很多人長大後會這樣.
就是因為有問題.
所以寧願收起.
所以那個方法.
正正不是收起.
而是增加那個熱情.
那個方法不是減低自己的熱情.
而是讓那個熱情更加有效地發揮出來.
令你不需要收起你的熱情.
就像河流.
重點不是不要留著河.
而是要更加控制住這條河.
那個控制.
不是控制自己.
這個名字好像很藏起來.
建立一些德行.
其實今天是一個很厲害的天主教概念.
一個.
最基本的信望愛.
Theological virtue 信望愛.
另外就是智慧.
這些東西正正可以幫我們.
將我們的熱情變得更加美麗.
譬如智慧.
我們說話是有智慧的.
不過我想說的是.
我們基督徒做人.
有血有淚的人.
我覺得一半對一半不對.
我們有血有淚是對的.
但有血有淚會傷害別人.
我只是說真話.

$^{1601}$這些是人話.
但人話也會傷害別人.
所以需要一個智慧.
需要有智慧去判斷.
我什麼時候說什麼時候不說.
我需要有愛心去面對這些東西.
所以重點是不要藏起自己的熱情.
而是問如何增加自己的德行.
這個有很多.
譬如是可以培養出來的.
Habitus.
有些德行可以培養出來.
信望愛是神給你的.
我們的重點是要增加.
基督徒應該有的智慧.
多於藏起這些熱情.
這個就是理論上的答案.
不需要藏起.
而是要好好增加自己的德行.
我做了40年.
慢慢發現自己以前也很藏起.
但發現當我建立多些Virtual.
這些東西我可以不需要藏起.
可以坦然無懼地將熱情流露出來.
也是好的.
不知道Person有沒有補充.
其實回應那個內容.
我回想起彼得後書第一章.
講關於生命和敬虔的教導.
第一章是講這個.
有了生命敬虔的表達.
就是有了信心就有德行.
有德行就有知識.
有知識就有節制.
節制加上忍耐.
基本上那幾節的經文裡.
一直慢慢提升.
到最後就是要有愛弟兄的心.
其實一個屬靈生命的追求過程中.
不斷地慢慢經歷完.

$^{1641}$上帝會加添.
而加添過程中.
上帝會幫你補滿.
你一直在過程中.
發掘了更豐盛的可能性.
到最後所有東西加起來.
就是令你能夠愛弟兄的心.
或者愛眾人的心.
這就是在行為上讓事情可以延展下去.
我想你過去可能也聽過.
或者早前也流行.
愛的反面不是恨.
恨也會花力氣.
就像剛才John所說.
Hatred其實是另一方面的愛表現.
他也會留意一舉一動.
才可以釘住他的字.
但是重點就是.
有些人現在愛的反面不是恨.
愛的反面是無情.
是不理會你.
對他來說.
這不是他最賞心的事.
反而我覺得這件事更難處理.
因為他已經沒有下一步.
對他來說那件事是什麼呢.
不是不重要.
不過你跟他說也無關痛癢.
不想再注視那件事情.
我覺得反而是難處理的.
另外我一點回應就是.
可以看一本書.
這本書比較舊.
是David Benner寫的.
《The Gift of Being Yourself》.
正主翻譯了那本書的中文名.
《天賦給我的禮物》.
是一本薄薄的書.
他是基督徒的心理學家.
也是一位牧者.

$^{1681}$他寫了這本書.
是一本很薄的書.
但這本書正正反映了一件事.
就是你的本相是一個怎樣的人.
其實看這本書是不斷去.
評價自己的本相.
評價自己作為一個人.
用我們今天的詞語.
作為一個人你的熱誠是什麼.
招牌沒有拆掉.
熱誠是什麼.
所以對我們來說.
《The Gift of Being Yourself》.
我中括了兩句話.
就成為這本書對我最大的提醒.
第一,Be Yourself.
做回你是一個怎樣的人.
上帝做你.
這個世界只有一個潘志剛.
同名同姓同性別都好.
但都不是同一個人.
所謂Be Yourself.
但如果你說Be Yourself.
我喜歡做什麼都可以.
亂來都可以.
David Benner 說.
Behave yourself.
就好像剛才提到.
彼得後書一章裡說.
有不同的東西加添.
當你Be Yourself的時候.
上帝會不斷加添.
補滿.
然後讓你最後可以有愛的展現出來.
這個就是我相信.
Passion 而有的生命的豐富.
謝謝.
我想回應剛才弟兄提出的.
我覺得如果用自己的熱情做事.
但傷害了別人.

$^{1721}$其實我們不是放棄.
而是從錯誤中學習.
因為我覺得沒有一樣東西.
或很少機會是你第一次做.
而你做得非常成功.
我覺得熱情的事.
是一樣值得你花時間去做.
如果你覺得愛人的時候.
你做的事是傷害了他.
即是你做的不是他需要的事.
你繼續去愛的時候.
你就會明白.
原來他要的不是這樣.
你就會一直不停地改善的時候.
你會做得更加好.
而你真的會做到你想做的事.
但如果因為錯了一次就放棄的話.
我覺得很難成功做到一件事.
是一個很好的回應.
因為愛從來都是經驗學習.
你會發覺我們與生俱來.
不是很懂得了解自己的取態.
很多時候你會看到小朋友.
都有一個表現.
我看我兩個兒子成長的時候.
我發覺只要是那個歲數.
就會有那個表現.
還沒有讀書.
意思是甚麼呢.
看他們過年的時候.
我就教會他們甚麼是「傳合」.
很多人是「糖果」.
所以他們主打供人「糖果庫」.
我都會看到小朋友是怎樣呢.
他口吃一件東西.
手拿著一件東西.
眼睛就看著那件東西.
全部都是給自己的.
人很多時候都是從自我出發.
去了解自己的需要.

$^{1761}$很想拿取一些東西.
但我們的信仰教愛人如己.
推己及人.
或者非以人乃於人.
是對外的時候.
其實這些全部都要學的.
不是我們知識上知道我們懂得做.
所以剛才姐妹回應一件事.
你有時傷害一個人.
你才知道說話有分寸.
你傷害一個人才知道.
懂得看別人那種反應的時候.
避重就輕.
這個對我來說.
都是一個比較有熱誠.
對生命有熱誠.
願意互相造就的過程.
如果回到剛才我說的.
如果他無情.
他就說「雖然理得你」.
「我已經說完我的事了」.
「完」.
那件事就會越搞越糟.
是不是網上有問題?.
菲姐問兩條問題.
第一條問題是.
她以前是一個很熱情愛主愛人.
現在變得很麻木.
甚至冷血.
怎樣才可以回復以前的狀態?.
第二條問題是.
以馬五師的門徒聽到耶穌講解聖經之後.
就心裡火熱.
為何現在的教會沒有這個效果?.
每間教會不同.
有些教會有的.
關掉冷氣.
我想第一個問題.
其實不是一時一刻就解決到.
我覺得是上次開學崇拜所說的.

$^{1801}$我都說臨彼祈禱.
求神給我們受傷的靈魂.
能夠回到初心.
其實初心就是那份熱誠.
我們小時候.
其實熱誠是每個人都有的.
重點不是問你怎樣有熱誠.
熱誠是自然的.
最自然就是有的.
不過我們因為受傷了.
或者我們長大了.
我們會去掩蓋它.
所以問題不是有或沒有熱誠.
而是你怎樣能夠在受傷的過程中.
能夠放膽將最初的那件事.
活出來.
從信仰到你怎樣看自己.
或者做人.
怎樣理解生命之類的東西.
這些全部都是我們本身有的.
上帝創造我們小時候.
小朋友就已經很有熱誠.
對於生命,對於世界,對於上帝.
只不過我們長大了,受傷了.
才會有很多這些加在上面.
所以我都覺得要慢慢來求上帝幫助我們.
去醫治我們.
我想這個有很多不同的情況.
不是一下子就沒事了.
所以我覺得這個重點不是行為.
不是我們靠做些什麼.
才能夠有熱誠.
而是問你本身都有.
怎樣能夠回復.
不妨聽聽我們在哪裡開學崇拜.
我這個都是求上帝.
讓我們重現這份熱誠和初心.
我的看法就是.
剛才都說到熱誠.
或者愛是一個經驗學習.

$^{1841}$所以有些事情不是一步到位.
或者很快就看到那個果效.
正如我開初都說.
我們的受教育過程當中.
常常都很著重表現.
就是有些東西出來讓人看到.
沒有的話你就會失望.
我們看結果多於看結果.
結果就是數據,數據,數據.
有東西可以讓你看到的.
有些東西是可以讓你看到的.
但結果是什麼?結果是改變.
改變是需要時間的.
時間就在過程當中.
改變很慢和不明顯.
要很久才能看到改變.
但我們對自己的要求都很高的時候.
就會很迷惘.
正如John所說.
他的篇幅就是.
他現在都覺得追不上.
他以前二十多歲的自己.
在這個過程當中.
總會有很多東西不斷想去.
重現原先的熱誠.
但我自己的看法就是.
我喜歡馬太福音第五章的結尾.
你們要完全將你們的天賦完全一樣.
我很喜歡完全這個字.
完全這個字希臘文是Teleos.
Teleos這個字根是T-E-L-E.
Tele是什麼?.
拍照就知道Tele.
鏡子是遠的,看遠的東西.
我們看遠的東西是什麼?.
就是看遠鏡.
就是Telescope.
Scope就是View.
我們用望遠鏡看遠的東西的時候.
那東西是否在那裡?.

$^{1881}$那東西是在那裡的.
在很遠的那裡.
在不在這裡?.
不在這裡.
在那裡,在很遠的那裡.
但那東西是在那裡的.
不過現在還未到.
我們朝著望遠鏡的方向走的時候.
最後我們會去到望遠鏡.
看到遠處那裡.
現在還未到.
其實我們的完全都是.
我們現在完全還未到?.
你們現在完全還未到?.
還未到.
你們已經完全到.
因為耶穌基督已經洗淨我們的罪.
但你們完全還未到?.
你們還未完全到.
因為你們還會犯罪.
但我們聖靈會提醒我們.
少犯罪.
我們會更新.
我們會提醒自己.
我們會慢慢越來越少.
但我們朝著一個更好的方向走.
這就是你們要完全像你們天父一樣.
那種進程過程.
Tenelos裡面的字是說.
我們的性情.
我們的熱情.
我們的做法.
會慢慢朝著那個方向走.
但我們現在不行.
所以我的回應就是.
我們不要看我們很多不行的地方.
我們就會frustrated.
那個情緒會有的.
但我們知道.
聖靈會提醒我們.

$^{1921}$聖靈會告訴我們.
你又犯錯了.
下次要再聰明一點.
下次不要欺騙聖靈的提醒.
心有責心的時候.
就停一停.
想一想.
不要太快去決定.
在過程中一直朝著那個方向走.
不要太快就斷纜.
不要太快就覺得沒有用.
我覺得這個進展過程當中.
就是我們成性.
或者我們在生命當中.
仍然感受到有再一次的機會.
我覺得passion是其中一樣東西.
讓我們感受到有再一次.
我感受到有新開始.
這個也是.
剛才聽回那個問題.
我感受那個問題.
弟兄或姐妹.
那個感覺就是沒有了.
我希望鼓勵他們.
其實有的.
現在可能你覺得.
不知道怎樣開始.
反而朝著那個方向走.
或者朝著想一開始.
你想追的過程.
其實用甚麼方法去追.
你再嘗試重新觸碰那個過程.
(記者:還有沒有回應?).
前面有一個.
我想嘗試一下.
是不是這樣說.
對熱情來說.
可能我們會定一些.
外在很高不可攀的目標.
其實是不是.

$^{1961}$回到單純的選擇.
而那個選擇就是我們本身.
本相其實是.
去投入到的.
就好像潘Sir說的.
去投入到的一種的場景.
選擇那個場景來投入.
來對生命的熱情.
在那裡去發揮.
還有我想其實那個管理方法.
其實我覺得會不會用普通一句.
譬如說.
當然很多時候我們都說.
隨遇而安.
有些事情改變了.
我們就會順著那個勢會改變.
但屬靈上.
其實就是說.
等候神帶領去下一個選擇.
或者遇到的下一個場景.
有時候不是我們.
我們意志上可以去.
定了目標就做得到.
或者我是這個場景.
我就一直在這裡發展.
其實也未必的.
是不是這樣就會.
放開一點呢.
我的意思是不會.
又會帶來自己很多失望.
同時也可以去.
找回原本那個.
自己投入的那個場景.
我先說.
是的.
意思就是.
有些事情是要.
保持著那個勢力去做.
你才會看到下一步有甚麼改變.
就好像我.

$^{2001}$上一個月說到.
《企硬講ye 》那篇.
裡面就是.
保羅其實他的馬其頓回應呼聲.
那個宣教旅程.
從來都不順利.
覺得上帝要開路.
上帝不是開這樣的路給他走.
但你看到他在過程當中.
他沒有記下他發願.
他看到原來那件事過了之後.
他看到上帝有些事情要他去做.
但他最重點就是.
他清楚他的宣教旅程為了甚麼.
就是讓外邦人得聞福音.
讓希臘文化有一個新的.
接受福音的向導.
這是很清楚他做了甚麼.
至於他自己的經歷.
被人打,坐牢,被人趕.
被人追殺.
這些過程對他來說.
他覺得不是最賞心或最需要處理.
反而是他在做他覺得自己應該做的事情.
我覺得對於我們做信徒來說.
問得最多的是.
我們現在做的事.
其實是不是上帝最喜歡.
或是最想我們要做的事.
做牧者最難回應的就是.
要和弟兄姊妹一起去經歷.
分享情況.
一起去談.
所以怎樣談呢?.
你會看到我們常常鼓勵.
你有一個屬靈群體.
你自己一個的時候.
你和牧者談.
可能只有一個一起去想.
但你看到其他弟兄姊妹有相訪的情況.

$^{2041}$你就知道上帝在這個群體當中正在做事.
用保羅回應.
在很多教會中提及一個很重要的訊息.
你的經歷會成為別人的祝福.
這個就看到整個群體.
會有上帝希望對這個群體當中的影響力.
所以我回應到這一點.
你的經歷會和其他人有出入.
但其他人的經歷會成為你的借鏡.
但大家一直在做一些以善的方向去做.
其實你會看到上帝在這個群體當中的教導是甚麼.
最後我講一下為甚麼我會用滑板作背景.
因為其實上年12月一更次之後.
我重拾生命的熱忱.
我就去了學滑板.
所以我覺得滑板是我打PASSION出現的圖畫.
所以我希望大家都可以重拾.
懷著生命的熱忱去開展你的生命.
無論是滑板還是任何事情.
都可以令你好好地去活.
我可以問問題嗎?.
那幅照片的主角是誰?.
不是我吧?.
我都想問.
網民問那幅照片是不是你.
希望大家喜歡今晚的課堂.
我們今晚到此為止.
下個月見.
再見.
(音樂播放).
\newpage



\section{}
\label{sec:hq6PGyJ3aBs}
\textbf{【這是最好的時代:給香港基督徒的神學八課】第7課:今日的「我」推翻昨日的「我」|20211121 [hq6PGyJ3aBs]}
\newline
\newline
連結: \href{https://youtube.com/watch?v=hq6PGyJ3aBs}{\texttt{ https://youtube.com/watch?v=hq6PGyJ3aBs}} ~~~~ 語音日期: 2021-11-21 
\newline
\newline
\hyperref[sec:BozY0a8wlNg]{\small{< < < PREV SERMON < < <}}
~
\hyperref[sec:index_chronic]{\small{[返順時目]}}
~
\hyperref[sec:index_scriptual]{\small{[返順卷目]}}
~
\hyperref[sec:gSBEvA3qrgQ]{\small{> > > NEXT SERMON > > >}}
\newline
\newline
$^{1}$再共你相.
請不吝點贊訂閱轉發打賞支持明鏡與點點欄目.
歡迎收看訂閱轉發打賞訂閱轉發打賞.
明鏡與點點欄目.
歡迎大家來到香港基督徒的神學八課第七課.
第七課快要完結了.
快完成了我非常完美的八課.
我們回看過去的六課.
其實我們都很複雜的心情.
從頭開始說基督徒,福音,教會,靈性,一根刺,熱情.
這幾課特別是中後段的時段.
其實很不容易去劃分.
我們八課中,到中後段的時間.
其實一直在想一個我們基督徒群體.
希望能夠一起塑造的模樣.
但這確實不能夠這麼容易地定下來.
所以想了很久,又有熱情.
希望這群人能夠看到自己的一根刺.
明白自己的罪.
這課跟之前的兩課都很相似.
我自己認為都是一些很個人的靈性.
或者我們如何做基督徒的議題.
大家都知道,我寫的書全部都放在基督徒生活裡.
基本上都是我對基督徒的生活.
如何做一個好的基督徒.
我自己是很有興趣去思考.
簡單來說,如果用神學的擺位.
金堂其實是一個叫「誠聖觀」的課題.
誠聖觀這個課題,好像很深,很悶.
誠聖觀是一個很簡單的課題.
如何做一個好的基督徒.
甚麼才算是一個好的基督徒.
整件事情是一個甚麼過程.
或者反過來,我們Fold Church的課題.
對我們Fold Church來說.
我們如何去定義一個好的基督徒.
如何去達成這個目標.
當然前面的全部都很重要.
如何在亂世操練一個這樣的時代的靈性.
我們懷著Passion,生命力去做人.

$^{41}$懷著一根刺,但仍然不斷地面對自己的罪等等.
金堂的名字是「今日的我推翻昨的我」.
很明顯,我想說的是一個有關更新的課題.
我們有兩個主題去討論.
一開始會先去認識一下「誠聖」這個課題.
「誠聖」這個課題,如果從神學的角度來說.
我們都一定程度要先認識它.
然後我會說,其實所謂的「誠聖」.
是一個甚麼的「誠聖」.
在21世紀,香港這樣的社會裡.
所謂的聖人模式是怎樣的.
我們如何做一個聖徒.
所以今天嘗試去說一個不容易去定義的課題.
我們先說「誠聖」.
「誠聖」其實是「誠聖」.
這個詞,如果我們用傳統的神學教義來說.
我們一個叫做「Order of Salvation」的裡面.
一個基督徒從他整個人的得救開始.
他有甚麼步驟會經過.
一開始我們會稱義.
我們被稱為義.
然後去成聖.
然後會繼續不斷地被差遣等等.
在一個基督徒成長或信主之後的過程.
往往很多人會將「誠聖」看為稱義之後的部分.
「誠聖」是一個如何能夠越來越像上帝.
越來越接近上帝.
甚至東方教會稱之為Deification.
即是更加神化.
越來越和上帝相似.
對我們新教徒來說.
這個課題是一點都不容易的.
反過來說,天主教是更加簡單.
我只是說簡單的意思,不是說它容易做.
起碼你懂得怎樣做.
如果今天你是天主教徒的話.
你的「誠聖」要做的事情很簡單.
我給你一個清單就能做到.
基本上你會接受水禮.
不是普通的水禮.

$^{81}$而是一個能夠洗去原罪的有功效的水禮.
然後你偶爾犯罪之後.
你可以去神父教堂告解.
告解完之後,你便洗完.
然後每個月你會領聖體.
聖體是一個幫助你靈明.
或是你的「誠聖」不斷更新的方法.
基本上就像吃藥一樣.
每個月一次,你就見效了.
或者有點像脫髮那些.
你一直做,就會收回頭髮.
所以有些事情是可以做到的.
而你一直做是能夠有效的.
甚至我們會說的事情是「德行」.
基本上是一個好基督徒.
做多一點好事,做少一點壞事.
減少犯罪,做多一點好事.
這就是基督徒或天主教徒可以做到的事情.
整件事雖然可以說是不容易.
但你都可以跟進.
你去大列的清單,你都會做.
你用心去做,你就做到.
因為天主教有一個觀念叫做「Habitus」.
一個習性的概念.
你做得多,你就自然會有好處.
你就不會有壞處.
所以你這樣做,你會覺得穩妥.
你覺得做基督徒,你誠信一步一步走.
你作為一個天主教徒,你就死了.
然後就有少許憐慮,上天堂了.
其實我越來越羨慕天主教的方法.
因為作為一個傳道人,我覺得容易教你們.
有什麼事就過來告解我.
有什麼事就來做,繼續做.
我就保證在我的教會裡面.
大家會一步一步走上誠信之路.
但基督教很不同.
五百年前,馬丁路德提出一個很矛盾.
又很真實的觀念.
這也是他自己人生裡面的掙扎.

$^{121}$同是事,二人,又是罪人.
這是一個很矛盾的觀念.
聖母Eustace在帕特卡拒絕了聖人模式.
馬丁路德是一個非常追求.
用心做好基督徒的人.
他覺得自己做不來.
然後他發覺無論如何.
他仍然是一個和上帝很遠的一個人.
不配領受恩典的一個人.
所以我們基督教,我們新教這樣開始.
我們作為基督徒.
我們面對誠信這件事就很矛盾.
究竟我們能否去誠信.
有沒有一些事是可以保證做得到.
來成為一個好的基督徒.
不知道是不是你的掙扎.
你可能做了很多年基督徒.
仍然覺得自己很浮沉.
或者很多事都做不到,做不好.
好像沒有一個方法可以保證做得到.
這似乎是我們面對的矛盾.
甚至我們會掛念很多的罪名.
或者過往做的事一直掛在心上.
假設一個姐妹離婚的時候.
離婚的一個人在教會裡永遠都是離婚.
起碼是離婚的一個地位.
掛念你一生.
再結婚也是再婚.
所以這些事情.
在基督教裡雖然叫做耶穌洗血的罪.
但這些內容似乎更難洗得走.
什麼罪是最難洗得走.
就是早已被赦免的罪.
天主教很簡單.
我真的去告解.
我就真的可以得到完全的赦罪.
我就懷著重新開始過.
但基督教很容易來說我們是一個義人.
但我們同時也是一個罪人.
所以這種矛盾.

$^{161}$希望大家能夠明白.
體會到這種複雜性.
我們基督徒去說誠信.
去說自己是一個好基督徒.
其實我覺得是一件很複雜的事情.
不知道大家有沒有唱過這首歌.
或者你也聽過這些說法.
誠信需要功夫.
其實我沒有怎麼唱過這首歌.
是一首很舊的詩歌.
但這句說話是很刻骨銘心的.
從小我就聽到這句說話.
誠信需要功夫.
我們華人教會會不斷說.
誠信需要功夫.
所以大家要用心追求誠信.
基本上不外乎都是那些.
都是靈修班教會的事情.
所以不知道大家是否覺得.
我們是否這樣就能成為一個好基督徒.
不斷很毀身地追求教會.
給我們設定的誠信功夫清單.
不過你會發現其實不是的.
誠信究竟需要功夫嗎?.
如果你看原文的字.
這個叫hagiazu.
hagiazu是一個動詞.
如果你找一個動詞.
大概就是這個字.
hagiazu就是hagia.
就是聖人.
使某人成為聖人.
是這個意思.
使某人成為聖.
所以這個字是一個吸脈動詞.
記得嗎?.
我們看過VT和VI.
甚麼是VT?就是吸脈動詞.
你需要一個object在後面.
我們有甚麼例子?.

$^{201}$在英文裡面.
I kill you.
I kill you是你是object.
你被我殺了.
所以是一個有subject有object的字眼.
誠信也是這樣.
其實不是我誠信.
我不是subject.
其實我是甚麼?.
我是object.
我是被誠信.
某某某誠信了我.
所以我們看約翰福音第十七章裡面.
耶穌說.
求你用真理使他們誠信.
你的道就是真理.
你怎樣猜我到世上.
我也猜他們到世上.
我為他們的緣故自己分別為聖.
叫他們也因真理誠信.
我們看見.
在哈爾亞訴這一節裡.
其實耶穌從來都說.
誠信的概念不是要你去誠信.
而是上天父上帝已經誠信了你.
所以你從來都不是誠信裡面的subject.
你不是主要的object.
我吃西瓜你是西瓜.
是subject.
所以耶穌說.
原來我們去探索誠信的時候.
其實從來都不是你的努力.
我這樣說.
這不是你的努力.
上帝是要使你們成為聖.
上帝已經在基督裡面.
使你們成為聖.
所以這是我們很重要的大前提.
誠信是在基督耶穌裡面客觀的基礎.
如果你以前是回宣道會.

$^{241}$我現在也是宣道會.
宣道會所說.
耶穌是誠信的主.
所以其實叫我們誠信的是上帝.
藉著耶穌基督叫我們誠信.
所以誠信需要用功夫這個字.
其實是我們要重新思考這個字的意思.
我不是說這個字是錯的.
而是我們要如何理解誠信需要用功夫.
容許我講一個聖誕樹的故事.
這是真的.
這是我在德國讀書的時候.
這棵聖誕樹旁邊的是我老婆.
聖誕節她穿著綠色的衣服.
這棵樹在德國的好處是什麼呢?.
你可以買一棵真的聖誕樹.
不單是真的聖誕樹.
更是一棵活生生的聖誕樹.
它還未死的.
還要用泥土養著它.
它不是被人砍下來的.
不是塑膠.
而是一棵樹.
所以我們買回來的時候很重.
下面的盆栽和泥土.
整棵都搬上來.
放在家裡很漂亮.
而且很香.
原來一棵未死的聖誕樹很香.
有很多特別的香味.
就放在家裡.
過了聖誕節25,26日.
過完新年.
發覺那棵聖誕樹很重.
搬它不容易.
那時候因為懶惰.
或者正在讀書.
沒什麼時間去做家裡的工作.
因為不是隨便放在門口.
要開車去某個地方.

$^{281}$所以那時候.
放在家裡的露台.
放到復活節.
整棵樹就一直放到復活節.
差不多放了半年在家裡.
有一天我坐在沙發上.
懶洋洋地看著那棵聖誕樹.
覺得不錯.
不如長期放在家裡.
下年搬出來.
下年就不用買一棵新的.
而且會變大.
說不定你坐著放十年八載.
好像海港城一樣大.
也不奇怪.
覺得不錯.
原來聖誕樹.
大家知道聖誕樹平日是做什麼的嗎?.
聖誕節以外的時間.
他們在那裡做什麼?.
他們仍然是做聖誕樹.
雖然他們已經是聖誕樹.
但他們仍然要成為聖誕樹.
你明白嗎?.
我想說這件事.
他們要成長.
他們要繼續變成大棵.
沒錯,他們是聖誕樹.
他們已經成為聖誕樹.
但他們仍然要繼續成為聖誕樹.
這個道理就是我們今天面對的情況.
你們已經成為聖.
在基督耶穌裡面.
在上帝十字架裡面.
你已經被稱義.
被成聖.
不過.
你要去做你應該有.
和已經有的事情.
基督徒永遠都是這樣.

$^{321}$你們去奮鬥的那件事.
不是你沒有的.
而是你一向都有的.
你是為著你有的事情去奮鬥.
無論是上帝的國度.
或是成聖的身份.
我們已經在基督裡面成為聖.
所以我們就要去成長.
我們就要慢慢去改變.
越來越大棵.
越來越漂亮.
越來越茂盛.
這就是我們今天.
起碼去談及成聖這個話題.
很重要的前提.
我們所謂要做的事.
不是你不做就不是聖人.
或者做了才是聖人.
不過你仍然要做.
做的意義不代表.
你做完之後有什麼.
而是你本身就是.
然後你去追求.
你去成長.
所以這是我們今天.
作為一個基督徒.
去思考成聖裡面.
一個很重要的課題.
我們開始思考成聖的科學.
Science of Sanification.
其實這就是我們所謂的靈性.
Define Spirituality.
就是Science of Sanification.
我們思考我們的靈性,靈命.
就是一個有關成聖的科學.
怎樣去走這條道路.
我們的成長.
我們從初信到活了的時候.
我們做基督徒是一個什麼路程.
是一條什麼路.

$^{361}$是不斷向上還是怎樣.
這就是我們嘗試去思考的事情.
怎樣能夠達至成長,成聖的道路.
有很多不同的理論.
今天很簡單,很籠統.
用圖片來做解釋.
很籠統的,不是有很多東西說的.
但很簡單地說出來.
我們看看路德的成聖觀.
基本上路德是沒有成聖觀的.
他說同時是聚人,同時是異人.
所以如果成聖是一條古事的線的話.
它是長期在一個低點.
你沒有所謂向上的可能.
為什麼呢?.
因為你同時是異人,同時是聚人.
你怎樣做都不斷掙扎在低點.
不要成為聚人.
不斷和聚搏鬥到你死的一刻為止.
這就是所謂的成聖觀.
基本上是沒有成聖的.
但最後仍然是一個極大的聚人.
這樣去掙扎著.
所以這是一些困難.
因為當時天主教最美麗的地方.
就是一些靈修傳統.
而路德是沒有什麼靈修可言.
因為你只能夠同序掙扎.
沒有什麼正面或積極的可能.
去談及做得更好的基督徒.
所以這就是路德的成聖觀.
不斷掙扎在低點的看法.
另一個我上次提及的極端.
就是楊惠思理的成聖觀.
當然楊惠思理和路德相差一段200年的時間.
不過楊惠思理對於恩信清義.
是更加強烈的經驗.
所以他會提及一個Christian Perfection的概念.
他覺得只要我們去追求的時候.
我們是有可能.

$^{401}$雖然不是每個人都可能.
但我們有可能去達至Christian Perfection的可能性.
你會跑到最後很高.
可以跑到很接近耶穌.
甚至是完美的perfection.
所以只要你追求.
只要你毀身.
只要你和上帝耶穌基督願意親近的時候.
你的線雖然可以是反覆的.
但是一個牛市.
是一個反覆向上的圖.
是會穩步上揚的.
雖然會有高低起跌.
但仍然在上升軌的裡面.
這就是他的比較樂觀的看法.
他認為基督徒是可以越來越好.
這是另一極的看法.
引申楊惠思理之後的相似經驗.
我特意帶來講.
就是十九世紀聖潔運動的經驗.
Holiness Movement的經驗.
他們是叫甚麼.
他們都是楊惠思理的發展.
他們認為有一個稱之為.
爆洩的情況.
突然間有一個叫second blessing.
一個第二次祝福.
因為聖靈的緣故.
你初初缺志的時候是一個起點.
但基督徒裡面會突然間經歷爆洩.
可能你聽過一些見證.
以前信主都不是很追求的.
突然間有一次車禍的時候.
他就重新再反省.
甚至動物神學中有很多人.
很多這些經驗.
稱之為第二次的祝福.
這是基督徒的爆洩.
我們先不要理會它的意思.
但起碼他認為仍然是一個向上的看法.

$^{441}$但需要聖靈一次的助力.
突然間聖靈的降臨和高末.
成為第二次爆洩的階段.
後來就形成了靈魂運動.
靈魂運動正正就是一個這樣的看法.
聖靈的秋雨降臨之類的.
有謊言.
所以我覺得那條路線比較不同.
但仍然是一個向上的看法.
第四個就是嘉義民的看法.
嘉義民的看法是非常不容易的.
簡單來說就像食神一樣.
寫一個「店」字.
就像基督一樣.
其實基督是畫不到的.
是畫不到那條線出來的.
總之就是你何時能夠去參與在基督的裡面.
你何時就是一個最好的狀態.
你何時不在基督耶穌的裡面.
你就沒有了.
所以稱之為 Participatio Christi.
Participation of Christ.
即是在基督裡面.
參與基督裡面.
只要你常常在基督裡面.
你就能夠是一個最好的狀態.
但當你離開了.
你就什麼都不是.
不過這樣說是對的.
但這樣說是什麼意思呢.
怎樣叫做在基督裡面.
怎樣叫做不在基督裡面.
怎樣叫做參與基督呢.
這個就似乎是比較抽象的.
所以你畫不到那條線出來.
你不知道你人生裡面何時叫做參與.
何時叫做不參與.
那條線是向上還是向下.
你不知道的.
不過他就說得很對的一件事.

$^{481}$就是在基督裡面.
所以這個就是.
總的來說.
我們是有四個比較去classify.
四種不同的成性觀.
去得出這四種的結論.
問題就是我們怎樣看呢.
或者我怎樣看呢.
我們Folk Church怎樣去理解.
這個成性的道理呢.
當然四個都有道理的.
我認為.
四個其實都是描繪著某些基督徒的狀態.
不過我想說的不是這些東西.
剛才那些成性觀的背景我們說完了.
我擔心的是另一些東西.
其實你也看到我們今天副題所說.
既然是成性.
倒不如更新自己.
為什麼這樣說呢.
因為其實你發現.
在教會裡面出現了這樣的狀況.
我稱之為丁蟹基督徒.
我寫過一篇文章.
這個題材我寫了一篇文章.
什麼叫丁蟹基督徒.
可能你看過大時代.
那你就知道誰叫丁蟹.
如果你沒看過就追溯一下.
丁蟹是一個什麼樣的角色.
丁蟹是一個很棒的角色.
丁蟹其實是一個好人.
他是正義的.
他覺得自己是正義的.
他是做好事的.
他其實是想做好事的.
他一直都在做一件他認為是對的事和好的事.
不過做錯了的意思.
他一直都以為.
藍潔英是喜歡他.

$^{521}$所以他才會這樣.
藍潔英死後.
你不喜歡我你早說.
打倒人家二十年都說了很多次.
他一直都不明白.
不知道.
但對丁蟹自己來說.
他一直都堅持著自己的良善來做人.
其實他真的不是詭詐.
他是單純地不知道自己做錯事.
所以.
所謂丁蟹基督徒是什麼意思.
他們是一群仍然追求誠信的基督徒.
仍然在自己的誠信裡.
去盡心甚至深化.
去追求耶穌和誠信的人.
不過客觀的原來是.
他有很多的限制和缺陷.
他們不知道.
所以我想說.
這個可能是更值得去提的一個課題.
我們在做好基督徒的時候.
我們更加會面對到很多的盲點.
有幾個案例.
第一個丁蟹基督徒的案例.
簡單來說就是缺乏知識.
如果你是很敬虔.
但缺乏知識的話.
你去追求其實都很糟糕.
如果你是很追求信仰.
但變成迷信的話.
當你缺乏知識的時候.
其實你的追求誠信都很有限制.
甚至你會變成很偏見的人.
你的神學沒有很好的聖經裝備.
沒有嘗試去擴闊自己的眼界.
去讀多點神學.
或者去看多點不同的信仰框架的時候.
你只是在你掛的那套進心.
當你缺乏這些知識的時候.

$^{561}$你可能會變成一個很虔誠.
但會看錯很多東西.
我在網上見過很多這種情況.
他們可能是很好的.
但他們未必有這樣的知識去嘗試.
讓他們好是更加好的.
所以我覺得呼出丁子妹.
很需要有這樣的知識.
敬虔愛主之愛.
追求之愛.
其實我們要有一定程度的知識.
去深入我們的靈明.
我自己讀神學的時候.
正正是這樣體會.
知識正正是幫助我們的靈性.
或者我們怎樣做基督徒.
因為當你知道.
你做得好之前.
你要知道要做些什麼.
你怎樣面對這個世界.
怎樣做好基督徒.
你知道之後.
或者你一定程度的深度之後.
你才能好好做好.
當然得知識沒有行為是錯的.
但單單有敬虔和追求和誠信.
而沒有知識.
仍然是一個很丁蟹的基督徒.
第二個情況就是缺乏包容.
有些基督徒其實是真的很好.
但當他缺乏一種愛心.
或者對於做不到的人.
是很嚴苛的時候.
這個就變成了另一種丁蟹基督徒.
他仍然是追求誠信.
仍然在這條路上.
是很熱切地對自己有很高的要求.
但他對其他做不到的人.
一旦嚴苛.
其實都會很麻煩.

$^{601}$變成了一些很不理想的人.
當然理性上是不會的.
當你有足夠的愛的時候.
其實你又會明白體諒其他人.
所以我覺得這個要自己引以為鑑.
當我們有屬靈的驕傲.
當我們做得好的時候.
我們缺乏包容的時候.
我們仍然是一個完全的.
第三就是過分主觀.
雖然這一點好像很濕性.
但我覺得是很真實的.
很多人是很主觀的.
他看事情太快.
去論定某些事情.
雖然他仍然是很追求.
和熱切地去執行信仰.
但有時他看事情太過.
覺得那些事情是看別人錯的.
當他的想法和價值不同的時候.
這就是錯.
我講得很抽象.
但具體我不想講.
因為有些人我認識.
很多人確實是很屬靈.
但其實是很主觀.
對於像我這樣的人.
就變得覺得不能客觀地看事情.
所以在這件事上.
遇到很多很好的基督徒.
但其實他過分主觀的話.
就不能理解很多其他事情.
所以在我缺乏慧子的情況下.
可能大家可以討論一下.
有什麼情況是有共鳴的.
丁蟹基督徒在你遇過的情況.
大家可以一會兒討論一下.
所以要避免成為一個很好的丁蟹基督徒.
要怎麼做呢?.
其實答案是很老套的.

$^{641}$就是更新.
所以我還是不容易講的.
後半段的話題.
因為其實答案是很簡單的.
我們就是要去更新自己.
我們要不斷地讓自己重新檢視自己.
我覺得這個更加重要.
相比起成性.
其實更新是更加讓我們能夠向前行一個必備的條件.
我們成性好像是一個很大課題.
但你對於自己的更新.
是一個似乎更具體的事情.
我們說教會的更新.
我們之前說教會的時候.
我們說教會不斷地改變.
不斷地改革.
當我們認同教會不斷地去改革的時候.
更加同樣道理.
教會裡面的人都不斷地需要改革.
所以大家想想.
大家有沒有去改變.
在這幾年裡.
最基本我們有沒有去更新我們的看法.
昨天的我和今天的我.
是不是不同了呢?.
今天我們會將這句話看成是一個負面的.
今天的我打倒了昨天的我.
你就會恥笑他.
但你想想.
如果你這幾年沒有改變.
其實這件事情反而更加詭異.
今天的你和十年前的你是一樣的.
一樣的想法.
一樣的體會.
如果一樣的話.
除了恭喜你之外.
其實會更加為你擔心.
所以我們問我們有沒有更新.
什麼更新了呢?.
特別是在《羅馬書》裡面.

$^{681}$當我們看到保羅說我們一個很重要的要求的時候.
我講過.
不要效法世界.
不要再困在世界的意識形態裡面.
只能夠透過聖靈所帶來的轉變.
這個更新心意變化是來自於聖靈的.
你不斷地來改變你自己.
不斷地改變你自己.
這個就是我們很重要的更新的意思.
當然另一個說話就是來自於.
這個歌頌前出第五七十節.
若有人作基督律.
他就是新造的人.
舊事已過都變成新的了.
這是我們之前在福音書的洗禮的時候.
我送給弟兄姊妹的經文.
但你想想你洗禮多久了?.
你洗禮可能已經有十幾二十年.
甚至乎三十年.
問題是你是否一個舊的新造的人.
你想想你回教會這麼久.
你對於新造的人有什麼感覺?.
理論上我叫新造的人.
但我已經回教會三十年了.
我是一個舊的新造的人.
所以我們所說的不是舊的新造的人.
我們仍然要歷久常新.
新造的人上帝是沒有舊的東西的.
所以我們所謂的新造人.
不是說你三十年前洗禮那麼簡單.
而是你不斷地每天都被更新.
在上帝裡面成為一個新的人.
新的想法.
新的關係.
所以你會發覺很多經文在聖經裡面.
每天早晨都是新的.
上帝是一個新的上帝.
新的字是一個很重要的神學字.
我們拉丁文叫做Novum.
上帝的屬性本來就是新的.

$^{721}$你會發覺有很多新的字.
新天新地新耶路撒冷.
新生命新的城新造的人.
所以上帝是沒有舊的東西的.
意思不是說他不斷地換東西.
而是他的本性就是新的時候.
我們只能不斷地有新的看法在裡面.
所以想想.
當我們如何能夠作為一個人.
能夠不斷地新.
你已經上了沙年.
你會發覺自己是舊的.
很多東西.
你唱一首詩.
我們聖餐那首《銅餅銅杯》.
都叫新歌.
都已經唱了兩年多.
但當你發覺.
當你唱了十年八載的時候.
這首已經變成舊歌了.
我們如何能夠在一些所謂舊的東西裡面.
仍然有新的東西呢.
仍然都是每個星期六.
星期六八點鐘崇拜.
我們開始發覺.
很多流出的新的東西都是舊的.
這個地方用著用著都是舊的.
當你聽了十年後.
發覺都是那些東西.
其實很難的.
一個人說了十年.
其實都明白他的笑話是從哪裡來的.
他的家底都聽過了.
所以重點是.
我們如何能夠在所謂.
舊的東西裡面.
能夠去更新.
這個更新是一個內在的更新.
如何能夠在自己的靈明.
在我們的生命裡面.

$^{761}$仍然保持一個新狀態.
我們第四課的時候說過.
叫我們找找你的靈性生活.
找一個新的靈修方法.
不知道你有沒有找過.
但你發覺總是要找找.
這個做得差不多了.
但你又要重新找新的.
這不奇怪.
所以發覺原來沒有一個完美的東西.
任何的活動任何的事情.
當你久了.
發覺總是會從新意分離.
慢慢變成一些你發覺很舊.
沒有常態的東西.
但這個時候你需要慢慢更新它.
我舉個例子.
我們今天常常用手機去說更新.
基本上更新是必須的.
你發覺沒有一個App.
沒有一個公司能夠寫出一個完美的軟件.
Microsoft Apple Google都是這樣說.
沒有一個完美的軟件.
真正的完美是什麼.
真正的完美就是不斷地去找出問題.
然後去整走它.
更新它.
不斷地這樣做.
所以與其說找一個誠信的基督徒.
不如說找一個願意更新的基督徒.
你在地上找不到一個聖人.
你只能找到一個願意不斷地去改變的基督徒.
我覺得這個很重要.
如果我們Full Church.
大家都能夠不斷地去更新自己的心態.
想法和靈性的時候.
我們沒有人完美.
我們都是一根刺的人.
但我們就是一個不斷地去更新的群體.
這個就是所謂上帝的身.

$^{801}$在聖靈裡不斷地去轉變我們自己.
這個就是我們所謂香港裡的聖徒.
我上次講道也講過.
這個年代是不需要英雄.
我們有很多的反英雄.
我想屬靈也是一樣.
我們這個年代不需要任何一個屬靈偉人.
一個完美的屬靈偉人.
全部東西都是完美的.
行事為人又端正.
有沒有不良嗜好.
其實都有的.
但總是有缺陷的.
我們不需要去建立一個完美的屬靈偉人.
任何屬靈的人其實都是不屬靈的.
不過他是願意去更新改變自己.
這個反而是更值得我們追求的地方.
所以你明白為什麼我們這個課題.
叫做與其成性都不如不斷地更新自己.
所謂捨己.
捨己可以說是一個很屬靈的情操.
不斷地去毀身.
不斷地去犧牲自己.
這個其實未必是英雄.
所說的不是sacrifice.
其實原文裡寫的是reject yourself.
這個人不是去犧牲自己.
而是不斷地去拒絕自己.
當然這個意思可以很多.
所謂打低自己的老我.
打低自己的舊我.
不過我覺得在我們這個context裡的意思就是說.
我們不斷地嘗試去拒絕自己.
去更新自己.
以前我這樣想可能是錯的.
以前我這樣去看可能需要被改變.
這個拒絕似乎是一種更新的意味.
所以我們Full Church裡的捨己.
可能就是這種態度.
不單單是為教會和上帝犧牲這麼簡單.

$^{841}$而是更加地不斷地去更新自己.
丁蟹基督徒也會捨己.
也會犧牲.
可能做了一些錯事.
但我們不斷地去反省自己.
說了這麼久.
我有幾點具體的說.
待會潘Sir來到我們會再多談一些具體的情況.
先說手.
為何你會選擇手來做這次課題的背景呢?.
我搜尋圖的時候.
搜尋這個Revolution.
當我搜尋Revolution的時候.
就給了我這幅圖.
這幅圖正正就是一個很好的象徵.
象徵著我們去革自己的命.
所謂更新正正是一種可能會血腥的.
血腥的是在你自己裡面.
有些東西你是這樣看.
有些東西你是這樣做人.
甚至乎這是你的性格.
不斷地去更新改變.
它是會痛苦的.
很多東西不斷去拒絕自己.
不斷地重新審視自己的想法和性格.
不是厭世.
不是不喜歡自己.
而是重新將自己不好的地方不斷更新.
怎樣做呢?.
我們有三點可以討論.
第一個就是嘗試開放.
我想第一點也是這樣.
嘗試去開放自己出來去被改變.
起碼assume自己可能是錯的.
就算你覺得自己對也好.
這是很矛盾的.
沒有人會做錯的事.
沒有人會做對的事.
但不斷去assume自己可能是錯的.
然後去尋求可能性.

$^{881}$學習更多的東西.
開放自己去學習.
一個人學習是真的.
你發覺自己成熟後.
學習是會慢了和差了.
我和洪禮方寫的書.
開始覺得自己越來越難改變.
因為我開始教人做人.
開始寫回應人做人.
當你成為老師的時候.
就越來越難成為學生.
但仍然讓我們去開放.
去準備自己去被改變.
這是一個很重要的態度.
第二就是反省.
都是很老套的.
發現自己的盲點.
這個不容易的.
可以回到小組裡.
或者今晚結婚和老公老婆聊天.
雖然你可能說了很多次.
我有什麼缺點.
問問你的好朋友.
你有什麼缺點可以改變.
這是很重要的.
不斷去看到自己看不到的東西.
所謂的盲點.
說是容易.
但我們仍然要去發現自己很多的盲點.
我會繼續聊.
很多這些課題.
有很多具體的情況可以跟大家聊.
發現自己更多的盲點是很重要的事情.
你的朋友.
你的組員.
你的另一半.
正正是幫你去發現盲點的一個很重要的位置.
第三就是對話.
我想特別是在這個年代裡.
我們真的要和一些和我們不同看法的人多些對話.

$^{921}$有句很好的說話.
就是說一個小小的真理的相反.
就是錯.
一個真理的相反就是錯.
但一個極大真理的相反是什麼.
是另一個極大的真理.
我覺得這是一個很受用的說話.
很多時候我們去看到盲點.
或者我們不能夠明白看法的時候.
往往就是我們缺乏和另一種體系的人去對話.
當我們嘗試去更新我們自己的時候.
嘗試去多些聽不同人不同看法的對話.
特別是我們可以說說這個話題.
有關我們的政治話題.
有關我們在教會裡.
不同立場和做事方法的問題.
如果上帝把一個和你完全相反性格的人放在你工作裡.
好幾年的時候.
這是一個很值得我們珍惜的機會.
你會爆炸.
但同時你會好好更新自己的方法.
一個physis, antithesis.
變成sympathies.
就會昇華.
你就可以學到更多東西.
這幾年我是這樣.
上帝把一個和我完全相反性格的人放在我身邊.
在我的生命裡.
當你克服了之後.
確實你能夠看得闊了很多.
能夠成長了很多.
記住我們是一棵聖誕樹.
聖誕樹是需要成長的.
雖然我們已經成為聖.
但我們往往需要不斷地去更新.
去走這條神聖的道路.
我們有一個討告時間.
無論你是在YouTube或在場.
我們一起討告.
求聖靈去臨到我們.

$^{961}$讓我們能夠提醒我們.
知道我們怎樣得到更新.
我們一起討告.
主我們求你去提醒我們.
讓我們能夠看見.
看見我們一些我們看不到的東西.
看見我們生命裡的盲點.
一些我們的缺點.
主我們每個人都有我們的缺點.
可能我們的性格.
可能我們的體見.
我們的立場.
求主你幫我們面對很多.
敵黨或相反的人的時候.
讓我們能夠有耐性.
開放自己去對話.
從而去更新.
主我們不願保留我們今天的狀態.
我們所想,所做的.
我們願意的不斷地去被改變.
我們願意今天的我.
打倒昨日的我.
願意成為一個更好的我.
求主你幫助我們.
願你去切割提醒我們保護弟妹.
生命裡可以被改變的地方.
具體的能夠去實踐這個成性的路.
馮春永求,阿門.
原來最後兩堂課了.
是啊.
開始有點掛念你了.
下期到你了.
OK.
今天你覺得這個課題怎麼樣?.
對我來說是很適合的.
因為我們做食肆.
我們經常去行間吃東西.
更新菜單.
還有自己想一些花樣出來.
吸引一些新客.

$^{1001}$所以更新對我來說是很重要的.
還有問問人.
東西行不行.
看看自己有沒有缺點.
所以後面那幾點對我來說.
開放一點.
自己除了一些手下小菜之外.
還要讓自己去嘗試多一些不同的元素.
食材.
是很棒的.
其實有些難實踐.
或者說個容易的.
大家都知道要更新的.
不過有沒有一些具體的經驗呢?.
我通常都是.
拿個話題出來.
跟別人去碰碰運氣.
我自己通常都喜歡一個方式.
不要太快就定了格.
還有問多一些可能性.
剛才我說主觀的人.
主觀的丁蟹.
正正是這個問題.
當你是很主觀的話.
你就很快在對話裡面.
就已經覺得我是對的.
他是錯的.
或者他就是這樣.
太快去主觀地看.
你就會變成沒有空間去開放.
還有你的對話都變成了.
已經定了角色.
他已經是反派.
我就是一個好人.
你是好人.
但其實別人可能是另一種看法的人.
這是我幾年來很多這樣的經驗.
我自己就經常都會有口頭禪.
當有一個問題的時候.
就不要太快覺得是一個問題.

$^{1041}$人們經常都想解決問題.
或者解決問題.
反而我覺得問題.
不如就質疑一下.
為什麼會有那樣東西出現.
或者質疑一下.
那樣東西對我來說.
其實觸動了我什麼呢?.
我仍然覺得留一條尾巴.
去繼續討論空間.
是比很快就定了格.
還有覺得應該是這樣的.
所以教會裡面吵架都是這樣.
教會吵架其實都是.
另一種大家都很追求的人.
你去吵架.
通常那個人不愛主不追求.
他就走了.
留下來和他吵架的人.
甚至乎吵到火紅火綠的人.
正正都是很追求的人.
都是願意成為聖的人.
很多時候問題就在這裡.
有些像你剛才說的丁夏.
其中一個重點就是包容性.
我覺得很多弟兄姊妹.
對誠性都有自己的觀念.
他未必很清楚.
我是屬於嘉義文.
還是衛斯理.
但他自己覺得在成長過程當中.
他已經對誠性有一個要求.
和一個他自己的看法.
就用那個看法主導了.
對其他人的標準.
這件事都是一個很普遍的現象.
是的.
所以不知道Future弟兄姊妹.
怎麼看呢?.
大家有什麼丁夏的feedback?.

$^{1081}$或者對誠性這個課題.
過去這幾年對自己的提醒.
有沒有什麼反省?.
(記者:你和教會說到火紅火綠).
(不代表我不想再更新).
(你是……).
聽得不太清楚,不好意思.
我說不再和教會的人.
罵到火紅火綠.
不等於我不想再更新.
只不過是我對我的組織.
我自己的根深柢固.
當然最健康的就是.
大家罵完之後.
繼續可以有一個關係在裡面.
但如果比如說社會撕裂.
你說教會裡面的人.
根本完全看不到年輕人的看法.
或者就算不認同也好.
至少和他們同行.
教會如果完全不想做這件事.
其實沒有意思再和教會說.
但這樣也不能夠說.
放棄再說.
就是一個人的靈明有沒有更新.
其實現在我覺得.
這個complexity可能大了很多.
為什麼?.
為什麼你覺得complexity大了很多?.
什麼意思?.
以前沒有這麼多問題發生的時候.
我覺得在我以前信主的途徑.
雖然可能有.
好像你剛才說.
要成性的那段時間很長.
大家有不同的看法.
怎樣才算成性.
但是沒有很多社會的動盪.
不會有很多這些問題.
去貼身地去明白這件事.

$^{1121}$當天下太平的時候.
我覺得大家的靈明.
最多是賺錢多一點.
賺錢少一點.
結婚離婚.
有工作沒工作.
進大學也進不了大學.
可能只是這些問題.
其實也算是一些舒服的掙扎.
但是當社會撕裂.
去到一些年輕人.
我不是年輕人.
但是有沒有希望在這個社會上.
又是一個再深一層的問題.
譬如以前教會不需要說.
以前的教會經常說.
我們很和平的.
信耶穌就很和平.
好像是新約裡的神才是真.
教會裡的神不是真.
這些話題完全不用說的時候.
走那條路很舒服.
變成我覺得可以面紅耳赤地爭辯.
也是一條健康的路.
但是當現在的撕裂這麼大的時候.
中間原來那個紅溝.
簡直是從這裡走到那裡.
那個空隙是完全沒有一個橋可以隔開的時候.
我看到我這邊的橋.
對方也不是在看到.
因為橋是大家一起面對.
去見的時候.
我是健康地去取得.
我覺得也是需要的.
同意.
所以我在想.
能不能夠在五年之後.
大家又再見面的時候.
大家又會走多了.
或者改變了.

$^{1161}$期望是一個過程.
我們說更新肯定是一個過程.
現在大家是一個很遠的距離.
如果大家都是一個更新中的基督徒.
五年之後原來大家都會改變了.
這麼多的改變可能是很多不同的改變.
但起碼會.
如果是嘗試去理論上.
其實可以拉近了.
我也是很渴想這個圖畫.
你說世代之爭也好.
或者是建制和年輕人之間也好.
我曾經想過會不會是年紀大的人難改變.
年輕人容易改變.
我不知道大家怎麼看.
不知道大家覺得.
所謂的OC food的人難改變.
還是年輕人容易改變.
不知道.
經常聽人說年輕人都很萌塞.
或者很多人都可以不斷被改變.
所以我想大家都是嘗試去.
好像一個途徑.
比較遙遠容易被上帝塑造.
來開放自己去做人.
這一刻是有差距的.
這一刻是很多不被接納的.
但自己是在遙遠的時候.
也是那一句.
上帝帶領我們更加明白.
明白別人為什麼會拒絕你.
我自己在這些日子感受到.
因為我不是接觸過很多外語教會.
大部分時間都是華語教會.
我就很感受那種愛的表達方式的差異.
特別很多時間都在香港的教會出入的時候.
聽過大家表達愛的方式.
那種從上而下.
我覺得這樣是適合他的.
於是他用那種方式主導了.

$^{1201}$我食鹽比你食米多.
用這種方式去教導他.
或者這樣做的時候.
打者愛也,愛者打多幾下.
那些管教的,但很主導性的時候.
我覺得不容易有溝通.
或者令事情現在來說複雜了.
就像今天的課題來說.
誠信其實是比較多向度.
不同年代大家追求的方式差異很大.
就像剛才說到第四堂關於靈修.
關於靈敏塑造的時候的方式.
以前只用文字.
用文本,用字,看讀經.
但現在可以聽,看影片.
有很多方式.
其實追求過程可以多向度一點.
因為方式也不同了.
所以回到現在爭論到面紅耳熱.
或者因為這個年代.
大家對道德標準.
或者對何謂善.
追求的步伐其實差異也很大.
就像剛才John說到年紀是一個問題.
我也見過一些年青人都很萌失.
我見過一些年長很開通.
一個地方就是歸根究底.
大家經驗過來就會感受到過程.
他經過年歲的時候就知道.
就像John說的會柔軟了.
這個也是生命歷練.
慢慢就會經歷到我現在這個階段.
不會很快就定格.
或者很快就覺得應該是這樣.
其實我覺得有點迷茫.
迷茫的地方是其實成性這件事.
剛才一開始提到.
是個人或是復興教會.
就是一個群體在成性或成長中究竟是怎樣.
我記得N年前的成性很簡單.

$^{1241}$就是分別為性.
我們是跟著世俗隨波逐流.
我們以聖經的教導為標準.
而我們是活出一個.
我們認為聖經應該叫我們這樣做.
我們就這樣做.
以前可以說是一個太平盛世的時候就這樣做.
我記得到了十多年前.
那段時間中東或者非洲很多地方很混亂.
而回教的狂熱有時會威脅基督徒的生命.
有時遊擊隊會去到一個地方.
「是基督徒嗎?」.
「是就殺,不是就放走」.
那時我會想.
為何現在文明社會還有這些事呢?.
基督徒有時會在突發情形面對.
那不是我們平時生活的獨特標準.
而是好像考驗你對基督教的信仰.
因為你在本能上.
在那一刻.
如果你是很害怕的.
你不承認是基督徒.
那就沒事了.
之後怎樣處理.
可能就像彼得那樣.
三次被認主之後.
然後就再懺悔.
我就在想.
其實到了現在.
好像是時代在不斷地變化.
因為我們面對的環境又變了.
我們可以面對不同的課題.
我就在想.
所以我有一個迷茫.
我們在追求怎樣的東西呢?.
所謂在信仰上的成長.
梁俊賢:所以就說不能夠用以前的方法.
因為你會發覺某個在七十年代.
追求成性的人.
會拒絕這一代的人.

$^{1281}$說他們不是基督徒.
那個情況就是這樣.
所以就說原來我的成性.
不是我做一馬嘴上壓.
不是我做得很多.
做得很好就能夠做到.
因為問題是你做得很闊.
原來我日讀經日靈修.
一定是好基督徒.
但原來這種方法.
你未能夠回應到很多.
其他時代的議題的時候.
都不是議題問題.
而是你要更新自己.
更新你的知識.
更新你的眼界.
不要太主觀.
看事物的時候是需要的.
我這樣說不是說你不.
剛才說的都跳過了.
仍然要做好基督徒.
那些事是大家在做的.
但這個之餘更加要更新.
不只是80年代那套觀念.
就叫做好基督徒.
其他就不是了.
不可行的.
我自己也試過很多這樣的經歷.
被很多前輩說我不是好基督徒.
但他們是很好的.
但他們有些東西好像不足夠.
反過來說.
我自己也更新了.
我33歲回來香港教書的時候.
正正是很多前輩覺得.
陳永安就是這個樣子.
33,34歲的時候.
寫的東西很激烈.
又怎樣怎樣說.
但其實我已經8年後了.

$^{1321}$很多東西已經改變了.
更新了.
但我印象就是在那個時候.
人是需要變.
也需要不是停留在那個裡面.
所以無論牧師或我.
誰都一樣.
大家都需要不斷更新.
開闊自己的眼界.
重新看事物.
不知道答不答到你的問題.
但我覺得迷惘的地方就是這樣不足夠.
所以更加需要更新.
大家對更新有什麼想法嗎?.
我想問一下更新.
即將到來的我們自己.
圍內小組也想面對2022年.
怎樣運作.
因為我們現在還在建立小組的模式.
我想問一下更新的意思.
代表以前回教會.
收集了其他組員的經歷.
通常有祈禱,崇拜,查經.
或是分享,見證.
更新代表這些元素都抹去嗎?.
或是基於這些元素.
轉變成另一種模式.
出現在現在FoldChurch的小組裡?.
我先簡單說.
更新不是代表抹去之前.
大家覺得是好的東西.
用食肆為例.
你會發覺食材不是雞,牛,羊的肉類.
其實也不會突然有什麼.
除非你說什麼分子料理.
其實是在吃羊而不是羊.
不是說這些.
反而是說基督徒成性.
讀經,祈禱,靈修的基本.
可以吸收的方法.

$^{1361}$其實不是不好的.
反而你說團體更新的時候.
我覺得.
我想回應更新和熱誠的連結.
你的小組接下來的運作.
有什麼可以令小組十多人裡.
都有熱誠.
回來小組享受.
這個才是重要.
因為我和學生說.
過去教會編週會.
大部分時間都用填充式編週會.
什麼叫填充式?.
第一週是查經週.
第二週是專題週.
第三週是祈禱會.
第四週是其他特別聚會.
很快就填充了.
因為你覺得屬靈的基本單元.
要全部填充.
但其實你沒有很大的熱誠.
想回來.
如果你不喜歡查經.
可能你不想.
第一週是查經週.
想填充就不回來.
你找不到熱誠回來.
反而我覺得你們十多人.
收了不同喜好的時候.
大家會否找到最大公因素.
或者覺得有什麼.
令你回來小組有熱誠.
這很重要.
對於我們來說.
好像Info Group.
我當中也和弟姐妹說.
你對小組有什麼期望.
有什麼令你想在小組中.
大家一起成長.
一起去過程.

$^{1401}$那個熱誠是很重要的.
所以對於你們來說.
小組裡面是說查經.
查經只是一個形式.
但關鍵是什麼呢?.
查什麼經呢?.
什麼主題呢?.
什麼內容呢?.
你們小組出來.
令大家有個向度.
有個熱誠回來.
這個才是最重要的.
不是說煮食的方法.
是煮什麼食物.
怎樣煮.
令你覺得.
很美味.
我想回來.
這個重新是重要的.
(包括有被辱罵的感覺).
(有辱罵的事是對大家的).
營養這個字也很浮誇.
因為營養有時是有品味的.
但重點是.
過去大家都試過.
你自己帶過那個周會.
或者準備那個周會的時候.
你自己最容易被那個周會得著.
那個過程當中.
不是你帶過那一刻.
是你準備那一刻.
已經在養你的人.
你自己已經在概念上開放了.
在內容上.
知識上更新了.
你帶的時候.
跟別人混的時候.
你就會多了一些技巧聆聽.
那個已經被牧養了.
不只是來自牧者牧養.

$^{1441}$你是透過教導.
透過彼此分享時.
就在牧養整個群體.
包括自己.
無論你是小組內容更新.
或者教會都一樣.
其實兩樣東西要執著.
哪些要 哪些不要.
哪些要拆 哪些不拆.
Flow Church本身就是教會改革.
我們說有些東西要改變.
當時拿著什麼來改變呢.
重新砍掉所有東西.
還是什麼呢.
就是說你很需要知道那個essence.
你知道那個essence.
你就知道其他東西可以拆掉.
我知道大家這個群體.
是需要實踐彼此相愛的.
我們是需要建立好關係.
我們要學習聖經.
這些是essence.
但怎樣做呢.
查經還是看影片.
還是什麼都好.
這些是方法.
抓住那個essence.
這是第一點.
拆掉.
有些人說.
我以前團體一定要搞美食烹飪周.
這些是form.
很迷戀這個烹飪周.
就是這個迷戀.
但烹飪周的目的就是.
大家彼此有興趣生活.
建立關係.
所以有些教會是這樣的.
有些教會20年都是做這些.
大家只是教會.

$^{1481}$一開始是ice breaking遊戲.
正正就是那個form變成了傳統.
不可砍掉的傳統.
大家30歲還玩ice breaking遊戲.
變成這樣的情況.
所以我們會抓住essence.
第一點.
第二就是你需要因應那個現狀.
一個小組裡面.
現在我的情況.
我們有什麼需要.
正如Poulson所說.
有什麼需要.
有什麼situation出現.
我們就是拼在一起.
essence和現狀需要去想新的東西.
更新就是這樣.
因為明年可能需要不同.
所以就重新去想.
所以確實沒有什麼是不可砍掉的.
烹飪周.
但其實essence就在這裡.
大家要分享的時間.
大家有彼此認識多一點的時間.
都有一段大家在靈明上進深的時間.
這個就是模樣.
所以我想essence拿著.
按著不同的時候需要去吻合.
這個就是我們不斷去更新.
要抓住兩個很重要的元素.
其他東西.
傳統一定要唱團呼呼團奮.
這個就不一定要.
你就可以砍掉它.
你拿著一些東西去決定.
所以你很知道essence的需要.
很多電影節目還不夠認識essence.
以為這個是必須的.
甚至Full Shot也是.
有些人說Full Shot好像改變了.

$^{1521}$以前沒有這些東西.
為什麼呢?.
因為那個情況不同了.
如果我說我做足十年都是這樣.
這個不就是傳統了.
我很但願Full Shot的傳統就是沒有傳統.
可以不斷地去更新自己.
按著不同的需要.
抓住教會essence去不斷地改變.
不斷地轉變.
這個就是我們很想做的事.
所以這個不是.
你今天的我打倒了昨天的我.
就是這樣.
就是需要更新改變.
我想問不斷更新的意思.
你也認同人是要不斷處於上升軌.
持續更新就是剛才說的一套.
要不斷上升.
更加越來越近耶穌.
另外也想問.
當我們擴闊視野和認識更多的時候.
很多時候都會接觸到很多不同的看法.
例如剛才了解到.
路德這樣看.
嘉義文這樣看.
好像每個人都有些道理.
反省自己.
我有什麼可以去學.
或者去吸收.
或者裡面有什麼要調教.
這個過程可否多說一些怎樣做.
因為有時越看得多.
又發覺自己不懂的原來有更多.
有時不太懂得怎樣建立自己的看法.
這方面有點深.
剛才那些圖表是很籠統的說法.
如果你問我.
我有點像嘉義文.
就是沒得說.

$^{1561}$沒得定.
因為你不一定只會在下面.
也不一定會向上面.
不過你問我當然想不想.
當然想向上面.
所以只不過是我想.
不是說我必定會保證.
所以我們當然是想向上.
更加好.
但那個方法就不是保證.
正正都不是說你保證怎樣做就怎樣好.
而是不斷更新.
不斷去更新.
因為人是會跌的.
我第一本書也說.
靈命就是一些東西.
你今天很好靈明.
第二天你有事.
突然崩潰就跌到地上.
但地上不代表你以後都衰逝.
你會好好的.
所以這個正正就是我們可以去跟隨.
去更新我們的情況.
所以有點像嘉義文.
就是你什麼時候在基督裡面.
什麼時候被更新改變.
你就可以走得高一點.
這個情況就是你沒有確定性.
但你只能不斷更新自己.
去尋找一個可以做基督徒的方法.
這個經歷我不知道怎樣分享.
潘Sir有沒有其他看法.
我自己.
這個問題很好.
還有很多東西可以說.
但我想先慢慢收窄.
我自己看自己的做法.
有點像John Wesley的專業.
我覺得會上下都要保持上下.
我覺得成性的觀念都是這樣.

$^{1601}$因為我覺得在聖經裡面.
人是有這樣的模式.
譬如彼得就是這樣.
他躲起來也好.
他都有上下.
但最後也有上下.
讓他更加知道.
他決定跟隨耶穌.
他不會回頭.
甚至像一個耶穌很喜愛的門徒約翰.
如果你看到路加福音第九章.
就是耶穌和門徒經過撒米爾村莊的時候.
那些撒米爾村民不接待耶穌.
約翰和雅各就跟耶穌說.
夫子我們需要求天火燒死那些人嗎.
這些是很差的東西.
那時候小主的註釋也說.
他是一個雷子.
性情很剛烈.
不可一世.
很威風.
他見過耶穌的真身.
你看到這個有多完美.
有多成性呢.
但你會看到這個被稱為這麼喜愛的門徒.
他見到自己的夫子釘上十字架之後.
他的性情就變了.
變了的地方就是.
最明顯看到的就是約翰一書三章十六節.
主為我們寫明.
我們從此就知道何為愛.
我們也當為弟兄寫明.
當初一個人說要燒死不接待耶穌的人.
但當見到夫子那種犧牲的愛的時候.
在他晚年的時候,他的性情就改變了.
他的願意犧牲,願意接納更多.
願意在當中有一些.
從年輕到老年的時候.
性情的改變.
看到他的專業是上下走.

$^{1641}$即使是上下走也好.
他也會有一個基督形象的表達方式.
你問我如何具體做呢.
我覺得上一課的passion.
可以在你自己的生命當中.
繼續提醒自己的方式.
或者讓自己繼續可以存活的心智.
我想早上年還是前年.
在網上很多時候都傳一句說話.
就是一個日本小說家.
叫做本間九雄說一句說話.
很多人三十歲就死亡.
八十歲才埋葬.
就是因為三十歲過後.
已經沒有人生生活的那種passion.
每天上班下班,做基本事.
但沒有什麼特別意義.
這一定不會是基督徒的成性觀.
或者基督徒的生活態度.
我們不是在意短暫時間的up and down track.
反而是我們在乎每天有沒有經歷.
或者在做信仰要表達的方式.
又或者信仰對我們自己的提醒.
我感受到有些東西是我覺得能力有限.
我需要別人提醒我.
或者我不懂的時候.
可以問我身邊的buddy.
所以在剛才還沒開始的時候.
跟John再談一下.
成性對我們來說一定不是個人性.
其實是一個群體.
和彼此的一個partnership去做這件事.
令這件事可以更加方便.
或者多些程度的不同層次.
(記者:大家對成性有沒有什麼困擾?).
(記者:還是覺得已經可以了?).
我想問是否純服性的帶領.
開始踏上成性的階段呢?.
因為很多聖賢都是純服性.
然後他們成為很多的牧者.

$^{1681}$或者是一些很好的傳道人.
我想問一下Flow Church有沒有異象呢?.
我不懂回答.
當然我們可以很簡單回答.
當然是聖靈帶領我們可以做基督徒.
就可以做好基督徒.
這樣就完結了.
但這是上帝自己的perspective去看.
但有時人的層面.
可能我們下第二季.
如果有的話我們會再說聖靈的topic.
因為聖靈是一個很abstract的topic.
什麼叫聖靈帶領你.
什麼叫有聖靈帶領你.
當然我們知道是有.
但問題是我們在一個人的層面去說.
其實不想停在這個句子就完結.
如果你說到聖靈帶領你.
大家就不知道拿著什麼回家.
所以我們仍然有些事情可以做.
都要去想.
所以為什麼會說和聖的東西.
當然我們說聖靈.
基督的靈.
作為一個成性的靈很重要.
但我們作為人都要有些事情去想和做.
所以如果單單說聖靈.
我是一個不負責任的教導者.
很重要.
但不單單說聖靈就完結.
有關這個topic大概是這個意思.
因為聖靈是工作.
聖靈是一個subjective god.
在我們生命裡.
所以我們要做什麼.
都是一個很重要的topic.
所以我們關心的.
我們具體在聖靈裡面怎樣做.
另一個話題.
很多時候大家都是.

$^{1721}$很多人在教會上吵架.
大家都是好基督徒.
難道你說他沒有聖靈.
這很主觀.
當我吵架的時候.
你沒有聖靈.
這樣很糟糕.
你說人家沒有聖靈.
我沒有聖靈.
所以這個很不容易說.
但我覺得是.
在一個時段來看.
大家要不斷被更新.
聖靈會帶領我們一起.
從一個不咬弦.
或者矛盾裡面.
慢慢去有合一的聖靈.
所以我覺得這個topic.
我大概想這樣說.
Full Church的vision很簡單.
就是我們在這個年代裡面.
為基督作耶穌的見證.
就是這麼簡單.
所以我們是一間.
去見證耶穌的教會.
特別是在這個年代裡面.
能夠用一種.
能夠容易到地明白的方式.
去宣揚基督耶穌的盼望.
我剛才在想的位置.
就是怎樣可以容易.
articulate那個層面.
通常我講聖靈.
我都會主要在門徒訓練的內容.
講那個身份教育.
因為剛才開初.
我講聖靈的帶領.
因為我們一路.
由第一天開始做基督徒.
聖靈就內在你的心裡.

$^{1761}$它一直都在你那裡.
所以我自己的教導層面.
常常都說.
聖靈叫保衛師.
又叫訓衛師.
就是它會圍著跟你說話.
它跟你說甚麼呢?.
其實就是提醒你一個.
上帝而來的身份.
你是一個基督徒.
你的身份而有的行為表現.
就是一個應該要有的東西.
用回聖經的說話.
就是一個新造的人有三個特質.
就是已分所書四章二十四節.
就是真理仁義聖潔.
在這個身份上.
你能不能夠去更加學悟.
和了解上帝的真理呢?.
在你的法則上.
你的判斷上.
有沒有上帝的公義呢?.
你的行為上有沒有一個.
聖潔的表現.
讓人感受到你是一個.
信主的人.
有那種應該要有的表達方式.
和行為原則呢?.
所以你說成聖的話.
或者聖靈的帶領過程當中.
就不斷重提你有那種身份.
而有最基本的三個向導的表現.
所以你剛才說到聖靈帶領的話.
我覺得先從你自己信徒身份.
或者基督徒身份當中.
去了解你的行為表現.
我覺得會比較具體一點.
不要只側重聖靈有沒有特別跟你說話.
或者一些方言.
或者其他比較抽象的.

$^{1801}$不容易去了解或看到的表現.
下一課就完結了.
下一課我們作為總結.
是關於跟隨主耶穌.
不重要了.
但我們這裡A類.
一台只有兩個人.
六點後沒有東西吃.
下次我們收店就吃一餐好一點.
下次我們收店再說吧.
這個很重要.
今天收店對我們來說.
第七課完結了.
下個月見了.
希望下個月見到大家.
好,拜拜.
\newpage



\section{}
\label{sec:HS1KRCnzG5o}
\textbf{【這是最好的時代:給香港基督徒的神學八課】第8課:跟隨耶穌的一百萬個可能|20211218 [HS1KRCnzG5o]}
\newline
\newline
連結: \href{https://youtube.com/watch?v=HS1KRCnzG5o}{\texttt{ https://youtube.com/watch?v=HS1KRCnzG5o}} ~~~~ 語音日期: 2021-12-18 
\newline
\newline
\hyperref[sec:cYPXvL44u1Y]{\small{< < < PREV SERMON < < <}}
~
\hyperref[sec:index_chronic]{\small{[返順時目]}}
~
\hyperref[sec:index_scriptual]{\small{[返順卷目]}}
~
\hyperref[sec:49X8yc0UC2g]{\small{> > > NEXT SERMON > > >}}
\newline
\newline
$^{1}$(廣播中).
《側徑》是蘇恩配以前寫的小說.
蘇恩配很久以前出生.
在七十年代時期突破很重要的人.
後來影響了蔡元雲,梁家麟等前輩.
蘇恩配姐妹寫這本小說《側徑》.
是講述他那個年代如何跟隨耶穌的方法.
後來這首詩歌叫《只有祝福》.
可能大家都聽過.
後來我在《Fourchurch》第四封家書也寫過.
大家有沒有記得這本書的內容.
它講述一個「子」字.
很有趣,如何用這個「子」字呢?.
這個「子」字是解作什麼呢?.
是解作「神明」的意思.
只有祝福,沒有奏作.
不是純粹解作「祝福Only」.
沒有任何奏作.
這個「子」字同時解作「上帝」的意思.
即是上帝仍然有祝福.
我想是一意相關.
基督徒一生中.
如果將時間線延長到永恆的時候.
確實是得到祝福的.
因為這是一個完完全全是上帝恩典的時間.
但基督徒一生也不只是寄望著那個終末.
而是此時此刻這個年代裡我們首要的關懷.
因此所謂祝福的意思是什麼呢?.
「子」字是解作「創造天地的上帝」.
祂是一個勝過世代的權勢的上帝.
雖然罪的勢力仍然存在.
死亡仍然存在.
不過我們仍然去認順.
世上只有祝福.
是上帝的恩典.
所以這個說話也成為了我們很大的提醒.
雙倍也說過.
與其就著黑暗不如燃燒自己.
這個說話在我們今天仍然是用得著.
在這個年代裡我們仍然走一條窄路.

$^{41}$這個窄路是昔日雙倍姐妹.
都嘗試去跟隨耶穌的方法.
可能方法不同.
但路是一樣的.
都是在黑暗裡嘗試跟隨.
嘗試去找一條可以走的道路.
所以受告裡面說.
只有祝福因我已踏上那則徑.
只有祝福因我已突破那罪的勢力.
只有祝福因我已勝過那死亡.
沒有就坐.
只有祝福.
這也成為我們全聖經頂尖輩嘗試去實踐的東西.
你問我今天說的也很虛偽.
沒有什麼特別的具體建議.
確實是的.
前面是沒有什麼具體的聰明智慧.
來去怎麼走.
但我想就是說.
我們確認承認.
我們有很多不同的走的方法.
可能有些人是不認同的.
可能有些人覺得我們是很另類的.
可能覺得我們是反傳統的.
但這個都是我們跟隨耶穌的方法.
很值得去尋求.
我們怎麼可以去跟隨耶穌.
最後這個部分我自己做了一下也挺感動的.
我嘗試去.
弄了雙倍之後就多加一些人出來.
全部都是死了的人.
嘗試加一些死了的人在這裡.
找回以前的屬靈前輩.
我們年輕的時候的樣子.
其實我們這個年紀.
三十多歲四十多歲的年紀的人.
每一代的我們今天稱之為.
我們很敬重的屬靈的前輩.
都做過年輕人.
都做過青年人.

$^{81}$他們都在那個年代裡.
嘗試去尋找他們那個年代裡.
跟隨耶穌的方法.
後來才成為了被定義為屬靈的傳統.
即是突破.
當時是一個很突破的雜誌.
今天也是.
這些新的事情.
很多不同年代的人.
他們都做過青年人.
嘗試去尋找一些跟隨耶穌的方法.
所以我特意把Full Church的標誌放在這裡.
我想如果你這樣看的時候.
你會發覺我們Full Church很快想到.
如果許可的話.
三十年後四十年後.
其實大家都老.
大家都成為了一些老人家.
或者一些退休人士.
但我們仍然是值得去問.
我們接下來的二十多三十年.
我們可以怎樣來到香港裡面.
繼續來生存下去.
跟隨耶穌下去.
所以我希望我們在這個年代裡.
懷著勇氣.
踏上未知的跟隨主的責勁.
責勁是什麼呢?.
就是一條很狹窄的道路.
狹窄有很多意思.
一來是艱苦.
二來就是很難找.
你需要去尋找出來.
第三就是很少人走.
所以這個正是我們基督徒.
我們去尋找耶穌裡面走的路徑.
如果這條路是這樣走.
有地圖.
有所有的起點和終點.
有中間的checkpoint的話.

$^{121}$未必是耶穌叫你走的道路.
是一條你前面都找不到.
我也不知道怎樣走的道路.
但這個成為我們今天.
大家一起來去共勉.
Full Church的目標仍然是那一句.
大家在這年代裡.
跟隨耶穌基督.
上師來尋找耶穌基督的足跡.
這樣走下去.
好,今天想和大家多聊天.
所以大家可以.
待會有聊天時間.
我們先祈禱.
好,我們先祈禱.
然後才有聊天時間.
祝我們去求你幫助我們.
因為我們在這一刻裡.
真的有很多未知之數.
我們Full Church的姐妹.
我們都不知道怎樣可以.
靠著你走過去.
如果我們能夠去效法你.
去跟隨你.
去尋找你的足跡.
我們都能夠在這個年代裡.
去找出我們繼續去跟隨你的方法.
無論是我們的目者.
我們一班的頂尖妹.
我們願意去有心的時候.
讓我們藉著不同的討論.
都能夠找得到我們去跟隨你的方法.
可能有些頂尖妹在外國裡.
有些頂尖妹在香港裡.
求主你成為我們這樣的一個主.
去告訴我們怎樣去跟隨你.
去有勇敢的去走一些新的事情.
讓我們能夠去找到一些.
可以知道你心意的方法.
藉著你的靈的帶領來幫助我們.

$^{161}$讓我們教會能夠成為一個更加全面的教會.
奉主命求,阿們.
喂,今天有什麼吃的?.
有花生,吃花生了.
有些花生米.
有杯飲料.
真的有7\%.
不用賣廣告都可以.
謝謝.
你最後一堂想說什麼?.
有什麼花生米?.
你覺得有什麼花生米?.
感動米.
我們的設定已經是很花生米.
我經常都說我們這個主學是有POP市的.
下一季的POP市會不會成雞?.
想一下,可能是花茶.
花茶可能是令女群看的.
我覺得感受上是很大的.
因為我覺得整個系列,八堂裡面最主要都是一些.
我自己覺得是一些氣質.
怎樣可以在身份上表達一些很實在的東西.
雖然你剛才最後的結束.
覺得好像不是一些很實在的東西.
其實我自己看就是.
其實呼應第一堂關於門徒訓練.
關於門徒這個身份其實是什麼一回事.
我覺得第一堂和第八堂.
最主要都是想帶出.
條命是什麼.
其實是你自己怎樣認受你自己是一個什麼身份的人.
今天這個訊息.
或者是最後這個PowerPoint裡面.
每一代的人其實都在尋找自己的身份.
可以做些什麼.
我覺得寄望Folk Church的弟兄姊妹都是.
不是一些很爆的東西.
不是一些很特別的與人不同的事情.
反而是在自己的崗位.
在自己的身份當中.

$^{201}$做到一些你可以做的東西.
這個是最重要的.
我自己在Folk Church三年.
認識很多不同的弟兄姊妹.
不過他們都很厲害.
很多弟兄姊妹其實都很厲害.
怎樣能夠可以在.
Folk Church不只是一個他們每個星期去敬拜的地方.
但是他們自己在外面是更加多可能.
我覺得是.
或者他們是一個很普通的牧者.
但是他們更加大的潛力.
我覺得Folk Church是一個很多不同的人聚在一起的地方.
但是他們自己都有不同的故事.
這個更加突破.
更加力量大的地方.
是呀.
我有時都跟一些組員或者弟兄姊妹說.
回到Folk Church有時會很尊重潘Sir,牧者.
但是我都常常強調.
在教會.
回到是一個彼此尊重的地方.
但是就不需要特別去調整自己的能力.
有時回到教會很謙卑.
很謙讓.
很多人說不行的.
但其實他在公司裡面是管理人員.
是有識之士.
回到教會好像就不敢說自己一些特別專長.
我覺得其實就不用.
回到大家就各行各職.
或者你有些什麼.
你又可以用到的地方.
大家彼此去欣賞.
彼此去配合.
這個是重要的.
講得俗一點.
你回到公司可能你IQ84.
你回到教會不用只有40.
有時就好像自貶身價.

$^{241}$或者有些東西是過份去謙讓.
其實我覺得不需要.
我和你都是在教會裡面的人.
我們那個群組都是這麼多.
最多都是外面多一點.
都是教會.
但其實等於每期更加闊一點.
我們暫時都是純粹讓教會裡面有些東西是能夠得到.
但更加大的場所就在教會外面.
所以為什麼FoodCenter都強調.
不要在教會裡面這麼多侍奉.
意思就是說不要只是在這裡.
我們說跟隨人數不要只是在教會裡面.
我們沒有一條路讓你跟隨人數走下去.
這樣是不好的.
可能會有更加大的需要.
特別是在這個年代裡.
教會其實不需要很多東西.
反而是外面更加多東西.
我常常都說我們提供一個平台給弟兄姊妹一起去分享.
這個位置就是.
我很感受保羅在《哥林多後書》說的.
每個人的經歷有前後.
我只不過是走你之前走過的路.
有個空間一起分享這個是重要的.
所以大家經歷了這麼多課.
有沒有什麼可以分享.
或者你期望在這些課堂內容.
你有什麼再想聽.
都可以大家談一下.
有,中間有.
我今天聽完十個八課之後.
我覺得這八課裡面的講道都比較詳盡和精彩.
我期望下一次的十個八課.
都可以加插一些不同的生活圖畫.
令故事裡面都很有趣.
都有開心的事情發生.
即是要製作一些場景劇.
即是說可以用故事都可以的.
不需要做劇.

$^{281}$其他呢.
有沒有什麼話題想將來談一下.
或者想知道多一點都可以.
網上有一個人問.
可能大家都談一下.
剛才我不是說到.
我們不是純粹.
所謂的根除.
成為了我們詮釋的問題.
很多不同的人對於耶穌有不同的詮釋.
所以我就說.
似乎不是純粹根據一個過去了.
或者這樣去做.
這個我想是對的.
我想大家沒有說誰錯.
可能有些教會.
他們的詮釋的耶穌.
會比較沒有那麼政治關懷.
或者沒有那麼多這些.
我們可能不是這樣看.
但我覺得這個正正就是很需要的.
因為大家有不同的詮釋.
之餘大家又不是反對對方.
我經常說我們Folk Church.
不是反對任何其他的傳統.
而是我們只不過是.
開放其他不同的可能.
所以我們正正就是要根除的.
是一個我們在前面的耶穌.
嘗試可以去尋求一些新的方式.
來根除耶穌.
所以這個我覺得不是純粹去拒絕.
過去或者其他的傳統.
而是嘗試去尋求一下.
我們在這個年代裡.
怎樣去做基督徒.
耶穌會怎樣做呢.
這個問題是值得我們用信心去尋求的.
在這一點我回應或者說一些看法.
我自己教基督教教育.

$^{321}$很多時候都被人問到.
課程設計怎樣可以令到弟兄姊妹.
在信徒的靈命培育上.
可以有一些準則或者進程.
但是我很多時候和一些主要學校校長.
或者一些信徒去領袖去談這些課題內容的時候.
我都先和他們說一件事.
就是某程度上現在的信徒培育.
都會受工業革命影響.
因為工業革命就影響到教育有一個模式.
怎樣可以有一個標準.
怎樣可以保證那件事是合格的.
但是我覺得信徒培育正正就未必一定要用這個方法.
因為我們傳統成長的學習階段.
都會有很多所謂門檻或者標準.
但是我常常都覺得.
基督教教育就沒有一個特定的標準一定要通過.
這個可能你可以不認同的.
但是我覺得聖經對於初信的.
對於年長的.
對於進心的其實有不同人.
有不同的領袖或者可以學習的內容.
所以不一定.
還有沒有限時間要多久讀完一本聖經.
其實就不一定要跑一個syllabus.
正正就是.
有時好像John.
我覺得最後那三句金句.
對他來說是他可以賴以過世的經文.
其實都可以的.
那個正正就是他每天提醒自己.
他的身份.
他的行為表現.
他對人對事的準則.
這個已經是很重要的.
反而不是說懂多少金句.
上了多少課程.
這個對於我們來說.
就不需要用課程內容.
或者上了多少課程套進去.

$^{361}$就說我是一個信徒領袖.
或者已經懂了這些東西.
你好.
我想問一個問題.
就是有很多不同的教會.
都有一些不同的側徑的路向.
有時候在這個側徑裡面.
本身一開始的一班人.
他們分了出來.
認為自己是一個新路.
然後但這班人走了一段時間的時候.
他們對聖靈的感動.
對印證在聖經上的明白.
和對可能大家一起交通過之後.
這個側徑明顯地有兩個東西的分別.
然後也有些教會.
可能本身在小的時候.
因為他可以是一種很自由的方式去明白主.
但因為這個側徑真的吸引到很多的.
或者我們的信徒.
去加入這個教會.
這樣就變大了.
變大了的時候.
我們就公式化.
我們就有一套準則的機制.
為了有效率去做事.
有時候這個側徑.
那個初衷已經是不同了.
或者那個路不知道怎麼去走.
有一些服務的弟兄姊妹.
或者參與的弟兄姊妹.
有時候很想保持那種合一.
保守那種合一.
但是他又在這裡.
他不能夠達到他們.
可能一個圈子裡面.
可能二十個人.
他們都有這個心.
他們做不到的時候.
他們退了下來.

$^{401}$我不服侍了.
我都會仍然留在教會.
甚至有一些就說.
我認為抓住一些主的說話.
我自己就再創另一條側徑.
但是其實經過了幾十年後.
有時候可能會發現.
原來所謂的分裂.
是一種彰顯主不同的榮美.
其實我們只是21世紀.
不知道還有多少個世紀.
我們才去到面見到神.
其實我很想知道.
當我們真的去到一個教會.
我很想保守合一.
我想繼續侍奉.
有一些就說.
有一人一票的這種情況.
可以去投票.
然後再去選擇我們前面的路向.
但當變成一個很大的教會的時候.
可能真的會有一些限制.
或者是當我們面對這個世代.
我們都很有抱負.
很有雄心.
認為我們會做得很好的.
這個側徑是會變大的.
現在是一條窄路.
將來我們怎樣可以.
繼續保持著這種初衷.
然後繼續去指望將來.
我們會有著主的榮美.
我們不會分裂.
我們永遠合一.
可否給我一些實際的例子.
我參考一下.
謝謝.
剛才你說的是一個很具體的故事.
很具體的.
這是你很真實的經驗.

$^{441}$或者是一些看過的東西.
我覺得重點不是.
不要將Foltrace 當成一條徑.
我經常說.
Foltrace 不是一條船.
不要想著Foltrace 要成為甚麼.
我經常跟Foltrace說.
Foltrace是一個大家聚在一起的地方.
但舞台在你們外面.
所以不需要幫Foltrace 想一些側徑或不側徑.
Foltrace 只是一個大家可以學習跟隨耶穌的方法.
但跟隨的實境和實踐不是在Foltrace裡面.
所以你說是否合一.
其實我覺得.
所以剛才那些是很虛弱的.
因為全部是沒有甚麼具體的方法.
如果是一些比較洗腦的方法.
不如叫你怎樣怎樣.
大家就做吧.
但其實是沒有的.
大家都是找自己去跟隨耶穌的方法.
而跟隨的方法不是去幫Foltrace.
去想之後的發展.
所以我唯一要說的.
大家一定要知道.
有很多不同的跟隨方法.
跟隨方法不是在Foltrace裡面.
而是在Foltrace外面.
這個才是真正能夠讓Foltrace去更新的地方.
因為Foltrace之前也說過.
Foltrace唯一一條不能改變的.
就是不斷要改變.
不斷更新就是我們Foltrace最要掌握的東西.
我今天是我 明天是我.
我們Foltrace本身也不是一個讓大家去發展的地方.
所以從來都沒有想過大家要去幫Foltrace發展.
而是成為一個平台.
讓大家在外面去跟隨耶穌.
而那個是側徑.
不是一個人走的就只有你.

$^{481}$不是大家一起走 走到很大.
而是大家有很多不同的方法.
所以這個明顯是不合一的.
大家都有不同的方法去實踐和去想和詮釋.
我覺得這個OK的.
東和西完全相反都OK的.
只是很多時候別人不OK我們.
但是我們是OK的.
我們是有不同的想法.
所以我覺得我們是需要這樣去做基督徒.
不知道潘Sir有沒有補充.
不是補充 我想大家相輔相成的地方就是.
我認同John剛才姐妹說的那些地方都是很具體的.
不過我覺得要有什麼例子.
我覺得真的不同人有不同的例子.
不是一個戴頭盔的說法.
因為耶穌和被追問的時候.
耶穌都沒有什麼具體例子.
人們問要怎麼做的時候.
耶穌就從他重述兩個大的綱要.
第一就是你要盡心盡意盡力去愛主你的神.
向上.
你知道你在敬拜的是誰.
其次有一雙方就是愛人如己.
其實耶穌都沒有說這些很特別的場景.
當然在馬太福音第五章至八章裡面.
有一些很日常的事情.
他會說一個律法.
一個有天國子民特質的氣質的人.
他應該在具體上怎麼做.
但是重點就是你碰到的人就是你要碰到的人.
你碰到的人不是我碰到.
所以我不可以告訴你.
你應該碰到的人怎麼做.
因為你最熟悉那些人.
瑞士理論社就是你最清楚誰是理論社.
我講得多麼學術也好.
多麼大包圍也好.
我不是跟那個人住.
是你才跟那個人住.

$^{521}$所以應該是你最清楚怎麼去對應那個人.
而不是我講所有的點.
我告訴你應該怎麼做.
因為教書在香港的教育方法就是.
什麼都講.
考不考的.
你去補習就會問這些考不考的.
不考就不要教了.
因為我會給錢的.
重點就是你常常都想拿一些精讀.
但是我覺得信仰就不是精讀.
信仰是一個通識.
通識的意思就是我應用神學就是.
你要懂得去探索場景是什麼.
第二件事是懂得分辨場景是什麼.
以至你做了神學反省.
你才知道這班人我用什麼方式.
這個處境我用什麼態度.
然後就是這件事有沒有違反我的聖經原則.
就是這樣.
所以很難是one size fits all.
或者是apply to all ages.
這個情況.
左邊這個.
不好意思.
想問兩個問題.
第一個問題可能被最後一幕觸動到我.
我也知道在過去的時候.
很多屬靈的前輩和長輩.
在當時都開創了一些新的路出來.
譬如黃明道或者黎作星.
在過去火紅的年代.
為中國的信徒帶來了一個屬靈的視野.
但是事後看回來.
似乎他們的屬靈派或不信派.
造成教會的基要主義.
或者教會的分裂.
回過頭來看.
是不是真的這麼對.
我們今天去走一條新的路的時候.

$^{561}$或者去探索不同的可能性的時候.
我們如何確保我們走上這條路.
或者開拓這條路.
會在現實的場景裡.
或者事後看回來.
不是走錯路.
這條路是真的對的路.
是上帝喜悅的路.
這是第一個問題.
反省我們應該如何去做.
這是第一件事.
第二件事我想回應.
你用蘇欣沛姐妹的小說《側徑》.
但我記得《側徑》裡有一幕是說.
當時的女主角選擇留在美國.
還是回台灣工作的時候.
她選擇了.
當時很多留學生不願意回到落後的東亞地區.
而願意留在美國繼續生活.
去服侍.
因為當時美國比較繁榮富庶.
台灣比較落後.
事實上最後蘇欣沛選擇了去台灣服侍.
當然後來她的癌症就是後話.
再回來香港發展突破.
這些全部都是後話.
但她說的側徑似乎不是純粹可能性的問題.
而是她看到.
這個抉擇是誰比較好走.
誰不太好走.
哪個代價的問題.
甚至哪個比較有好處.
坦白說.
似乎是那個面量會多一點.
我覺得你用她的側徑來詮釋.
會不會好像有一點不是那樣.
不好意思.
是的.
其實我喜歡這個字.
因為這個側徑的字.

$^{601}$我覺得和窄路的字比較優雅.
還有我喜歡這首歌.
剛才所說的這首歌.
但我自己去認識蘇欣沛的人生的時候.
我覺得有點鼓勵.
其實是不同的意思.
我都說側徑有三個意思.
一個是難走的.
一個是不容易找的.
一個是少人走的.
對我來說更加大意義.
譬如金塘那邊特別大意義.
就是要去尋找側徑.
不是那麼明顯.
還有一點另類.
所以我覺得在我們跟隨耶穌那邊.
當然不是為另類而另類.
但我自己這幾年裡.
感受就是多走一些另類的東西.
今天有人覺得Future還是有點奇怪.
所以我覺得還是有些另類.
但我們覺得.
我們是跟隨耶穌.
我們知道我們要.
只是不同的方式.
所以重點是.
我們正正是用不同的方式.
和一些不是既定的方法.
來跟隨耶穌.
這就是我自己對於側徑的意思.
對於這個問題.
我反而不覺得有點困難.
因為我都說任何東西都在對的地方.
它只不過是錯在時間裡面.
基要派是好東西.
不過是在什麼時間裡.
去練習基要信仰.
所以我們都可以成為基要派.
一百年之後.
任何信仰當你不去跟隨的時候.

$^{641}$就成為基要派.
所以我覺得重點不是內容對還是錯.
而是時間對還是錯.
所以為什麼說要跟隨.
我們要找的耶穌是前面的耶穌.
而不是純粹看回.
這個意思就在這裡.
如果純粹看回我們的傳統.
如果神學百科運用二十年.
我都會懷疑這些問題.
所以今天我們是好東西.
新的 去更新的.
但這套東西都不可能用二十年.
我想我們付出的正正就是.
一個法則.
就是要不斷反映和更新.
沒有什麼是不可以被改變的.
不過很深度的問題和分享.
我都想回應第一個.
怎樣才知道自己會不會錯.
或者怎樣去檢驗一下自己的想法.
對我來說.
信仰從來都是經驗學習的.
經一事象一智.
而在過程當中知道自己的錯漏.
或者知道自己的分辨機制.
或者能力在哪裡.
以至怎樣去不斷修正自己的路.
我覺得上帝給人理性最優美的地方.
就不是一個機械人.
就是可以在自由觀中不斷提醒自己.
這個都是我覺得在過去.
我自己認識教會比較少做的.
做思辨訓練或是一個場景題的情況.
因為很多時候教會教導的時候.
都是很著重經文的詮釋.
對生活的道德法則意義.
但其實我覺得要做多一件事.
就是要提出一個場景題.
就好像回應John今天說的.

$^{681}$如果現在在這個現況當中.
我們要面對這些做決定的時候.
我們的信仰和神學.
給我們一個反省的能力去到哪裡.
這個就在當中討論.
我也認同John剛才說的.
基要派對那個年代的人來說.
是一個頗重要的信仰教導.
可能可以保存當時的道德和信仰範疇.
是很實在的.
正如舊約的人.
其實律法本身是沒有錯的.
只不過是因為律法是讓人知道自己做錯事.
或者令人知罪.
律法本身是沒有錯.
本來在羅馬書講得很清楚.
只不過他要守律法成為別人的重擔.
那就是扭曲了那件事件.
律法本身的原意.
所以當一件事不斷地加添的時候.
就會偏離了原先的那件事.
這個正正就是我們Flow Church的彈性.
是好的.
就是不斷地去檢視我們在做的事.
在這個情況回應剛才的姊妹.
Flow Church很多事都可以改的.
我們那些團隊沒有很強烈的持續性.
我們會拆散重整.
不會說你做了很久.
然後就推倒重來.
又再叫一個團隊去做.
這個都是我們想讓那些弟兄姊妹.
不斷地有不同的更新.
或者對我們來說不要太過固定.
首先多謝John.
因為這八堂.
我自己的感覺就是.
其實John將在教會歷史裡面.
一些相對重要的信仰觀念.
一些人物的思想帶給我們看.

$^{721}$讓我們去看.
其實在過往那裡.
究竟在每個不同的時期.
他們對信仰是怎樣詮釋.
怎樣去實踐出來.
對我們是會有幫助的.
我一直回Flow Church.
或者聽八堂的時候.
我的感覺是.
我們應該從一些很大的宏觀的東西.
走回實際生活那裡.
去看我們人生裡面一些很細微的東西.
其實如果我們去真正去看.
我們會發現一些東西.
但對於我的感覺.
就好像很多年前去看.
一些叫做心靈雞湯的文章.
我覺得相對是比較主觀的.
主觀就是.
每個人去看同樣的事情之後.
大家可能有不同的看法.
有不同的領袖.
有些人會被鼓勵多一點.
有些人聽了會有一時的感動.
但未必會持續下去.
但方向是對的.
因為大家都是會向好的方向去做.
或者是向信仰去追尋去做.
但我的感覺是.
有時就好像沒辦法捉到.
如果用道家的說法.
有些東西不可以嚴存.
只可以意會.
我覺得在教導上有時會有困難.
例如我們經常說.
我們做一個人在最高基督徒.
我們想在我們的人生過後.
究竟在其他人心目中留下了多少回憶.
留下了多少影響.
我一定是對的.

$^{761}$但問題是.
這樣有多少影響呢?.
或者我們怎樣可以做得到呢?.
如果純粹是一心向著信仰去做好.
總之有人因為我的生存.
因為曾經跟我一起經歷過.
受到影響.
這樣就可以了.
這個說法也是對的.
但就好像沒辦法去量度.
或者沒辦法去令大家.
因為是一種教育的感覺.
還有在另一方面.
基督教很多時候我們除了強調.
大家彼此謙卑和諧.
去包容.
或者去明白對方.
大家去放長時間去看.
例如大家對信仰的傳息有不同的看法.
但可能過了幾十年之後.
大家看回的時候.
大家是有少許不同的做法.
但其實我們都是一樣.
會令到神的角度去擴展.
這是其中一樣.
但其實在基督教的信仰裡面.
有一條路就是我們有一個.
對真理的堅持.
或者是一個捍衛.
這就好像在《百科》裡面.
我覺得好像沒有說過這件事.
另外就是.
我想會不會其實在現在的年代.
我們除了去自身做好之外.
其實在基督教的信仰裡面.
我們會有一些情況.
會面對一些衝突.
我們要站出來.
有些事要說.
這就好像在《百科》裡面.

$^{801}$我們沒有特別提過.
我想我是這種語調.
我個人不是這麼衝突.
因為我覺得這個年代裡面.
當然可以學術研究一下這個問題.
真理是不是抗衡出來的問題.
這個可以討論一下.
我自己當然因為很多原因.
都不是這麼絕對.
我和神教授都是這樣.
說完之後.
大家都不會說這個就是答案.
都是一些大家想一下.
或者大家討論一下.
又不一定的.
會有什麼衝突呢.
都是這樣.
我自己都prefer.
因為都是我自己限制的.
我自己prefer在這個年代裡面的真理.
當然有些堅守.
堅守都不是靠一些defend.
或者我要這個才對.
這個就是對.
是容易的.
有時候都需要.
但有時候我們是.
需要的是反而是.
我都說了.
那幾堂課我都說過.
包容.
又說tolerance.
又說要更新.
這些字眼都不是很絕對的字眼.
確實是.
那個路徑確實是.
現在太多人覺得.
我們這裡才是正路.
所以我們要做的.
很多時候都是變成另類.

$^{841}$所以我又變了不是太強.
我不想說.
我這個才是正路.
其他就不是.
正路就比較差.
或者沒有那麼對.
我不會這樣說這些.
這樣.
所以我自己覺得.
我自己都覺得.
因為這個年代裡面.
太多這些太過絕對化了自己的東西.
特別都是.
當你讀了很多歷史.
和很多不同的觀點.
就發覺其實又不是那麼絕對.
我經常說一個.
小規模的true.
相反是false.
但一個極大的true.
相反是另一個true.
所以我覺得.
特別是耶穌基督裡面.
那個信仰的真理.
其實我覺得那個真理.
不是確定出來才有的東西.
而是大家來到去.
交流擁抱.
發現大家都有的東西.
是一些共享的東西.
不過是.
我要承擔.
因為我要抵抗.
因為你不這樣.
這些東西.
但我都會接受的.
所以潘Sir來到.
坐下來就把東西concrete一點.
變得大一點.
實際一點.

$^{881}$但我又覺得.
這些東西是需要.
有些wake 一點的.
因為實際上.
如果太不wake的話.
確實是變成.
大概就是這樣.
跟著來做.
很容易做.
但這個只不過是.
其中一種方法.
我都是要拉平衡.
但我都是嘗試去.
反映一些橫教會的傳統.
不是執行上那麼清楚的東西.
因為確實是.
規限了我們很多的方法.
我覺得這個題目挺好說的.
正正就是.
做教育來說.
什麼叫做標準.
或者什麼是要.
syllabus 或者要說的東西.
我想我.
get到其中一個point.
就是要怎樣提出那件事.
我們要再宣告.
或者再重申.
這個是我們的神學立場.
或者是我們的信仰起點.
我反而有一些疑問.
就是其實這些東西.
不用我們說.
你都知道基督信仰是什麼.
不用我們說.
其實有些東西你都知道.
但我們是否要再說一次呢.
是.
但是不是在課堂上說呢.
我覺得反而要想一下.

$^{921}$因為某程度上.
可能解讀錯誤.
不過我覺得有個情況.
我自己經歷過.
就是好像有些東西.
要有權威.
或者有title.
或者有能力的人說.
那件事才特別要加持.
其實如果是true.
一加一等於二.
這件事其實是小朋友說.
都是true.
不一定要是一個大學教授說.
才是true.
我們基督信仰有很多東西.
基本上是不會有妥協的.
譬如三位體是上帝.
基督的神人異性.
這些不用John說.
才覺得我們一定要說一次.
或者我們要confirm一些東西.
譬如同性戀.
或者其他一些.
可能很多所謂一個.
比較現代一些.
有爭議性的東西.
我覺得我們可以說的.
但又不一定要特別說得.
很嚴正其詞地去說一次.
反而有時有些難處理.
都會被弟兄姊妹問到一件事.
就是應該怎樣做.
好像現在網上都問.
應該怎樣做.
我自己經歷過的教導.
或者弟兄姊妹訓練的時候.
常常他們自己經歷一件事.
就是有得讓他選擇.
他都選錯.

$^{961}$我問你們.
你們不用回答我.
你們做了這麼多年人.
有幾樣東西你選對了.
有什麼不適合你選擇.
你都回答過.
不要說得很大是大非.
老婆老公不是那些.
說到人生生活.
有多少東西可以讓你選擇.
你會選擇對的.
其實你會發覺.
你很緊張選錯.
你不想選錯.
但我常常都覺得.
信仰就是經歷一個對與錯.
是與非的一個經驗學習.
連耶穌都沒有一個detail.
或者govern他回答的答案.
你看科林書裡面.
耶穌和他相遇的人.
耶穌基本上是沒有理會他的答案是什麼.
也不主導他回答什麼答案.
直接是你自己想.
你都不用告訴我最後是怎樣.
你看到尼哥底姆.
撒邁爾鄭棠婦人.
少年紀官.
凡是多敘述耶穌和他對話的人.
耶穌就是重點.
我呈現所有的fact給你知道.
我告訴你結果是怎樣.
但最後你怎樣想和決定.
你不要告訴我.
你就做吧.
你自己選擇吧.
這是耶穌和人的對話.
但反而你現在.
我不是在說什麼問題.
反而很多頂姐妹問.

$^{1001}$你告訴我應該怎樣做.
我很難告訴你怎樣做.
就如我剛才說.
我不是你熟悉那班你接點的人.
其實應該是你最熟悉.
你應該要想想.
你怎樣去articulate那件事.
在做的方法.
而不是我告訴你.
你三碗水煲兩碗喝了就行.
不是這樣.
做prescription是很難的.
因為很難有些東西.
好像吃藥一粒藥就搞定了.
特效藥就是對那件事有效.
其他東西沒有效.
我們都有一些suppose.
大家上完錄像或現場.
大家在小組裡再談.
談完再一起去想實踐.
這件事是整個環境多於答案.
我們就是想這樣.
實踐是一起實踐.
大家一起去想怎樣做.
但這不是說出來的答案.
答案都很簡單.
怎樣做都是做.
反而我覺得重要的是.
有些聖經基礎.
有些神學的background.
你知道這些東西.
你學了這些知識.
但實踐就不是單向講座.
能夠說出來.
很多不同教會的講座.
都是問怎樣做.
是很重要的.
因為是很具體的東西.
但又不是在這些場合.
能夠說得出來的東西.

$^{1041}$因為實踐不是講座.
能夠提供的東西.
是在群體裡.
大家一起去分享.
一起去走一些東西.
所以不是單聽講座.
大家一起去做一些材料.
在群體裡一起去做的東西.
前面.
開了嗎?.
開了.
喂喂喂.
OK OK.
其實我不是沒有問題.
不過我想回應一下.
我覺得Voltage去搞神學百科.
是很有心的.
在這個時代.
可能大家理解.
一般成長班.
我們可能看很多經卷.
或者很多茶經.
或者我想這個.
可能大家.
你們開這個平台.
就是想給大家.
在這個時代.
可能針對這個時代.
去有一些.
可能怎樣扣住.
我們這個年代.
和我們的神學.
其實有什麼關係.
其實可能想做一些這樣的事情.
我沒有上完百堂.
但是我發現你們在講道的時候.
有些閱題.
和你們這些主題.
是有些扣連的.
我覺得其實是挺好的.

$^{1081}$因為都有點深化了.
可能你說聽完之後.
無論你是追溯那個主題的東西.
或者是再解說的時候.
我都覺得是理解又多了.
又看得多了的面向.
我也同意.
有些時候.
有些題目未必可以講得這麼淋漓盡致.
或者講得有多深.
因為可能大家的理解全息.
可能講的時候.
那個情況.
大家都不是很一樣.
但是我想.
我會覺得.
可能你們剛才會問的.
未來下一季又會是怎樣.
我都會覺得是期待.
就是.
就好像你們剛剛說.
可能你們是經過很多的反省.
很多的反思.
然後再構思每一季的主題.
就好像今季.
我都覺得是很有心的.
因為其實我都.
因為這些主題.
這些內容.
我都很想去報.
我會覺得是.
正正可能像寬Sir剛才說的.
可能正正因為因應著我們那些場景.
而我們設立了這些主題.
這些題目.
但是.
在這些題目.
可能其實裡面說的神學.
都是我們最基本最核心的那些東西.
可能我們在教會學的都是這些東西.

$^{1121}$但是只不過是轉了一些場景.
其實大家再看看.
原來.
那些神學的訊息.
其實和我們今天是有關係.
其實可能讓我們更加知道.
究竟我們現在這個時候.
可能要怎樣走.
我.
也可能剛才聽到大家的分享.
就覺得.
我們其實都很容易落入一個位置.
就是我們聽了很多教授.
或者我們聽了很多講座.
其實我們經常都很想知道答案.
對於那種找不到答案.
又有些不安全感的狀態.
其實都明白.
我們其實經常都會很想.
很怕錯的情況.
但是都會覺得.
是呀.
好像剛剛這一篇.
可能這個分享裡面說的.
就是我們有些時候.
其實好像耶穌在福音書.
或者聖經裡面.
其實祂給我們的答案.
真的不會說是.
你要走一二三四五六.
有很多的規條.
或者是你一定有一個很準確的答案.
但是意思是一個很.
祂有一個原則.
可能或者祂已經告訴你.
答案是怎樣.
但是原來上帝真的會給我們.
有自由去走那些路徑.
而那些路徑其實是.
好像剛剛那樣說.

$^{1161}$有些心意或者有些上帝的話.
未必是一板一眼.
我們會很的確地知道.
但是其實.
你有這個心去追隨的時候.
或者是.
剛剛那樣在碟碰碰.
去試去撞的時候.
你知道有些路是走不通的.
有些路是可以走的.
那你繼續走下去.
看看是怎樣.
是呀.
多謝你.
後面有一個.
我再說一點.
多謝你看到我的心思.
還有回應剛才.
有些內容我忘記了說.
就是.
在教育上.
我們受教了很多年.
讀了很多年書.
有時候某程度上我們迷信了課程.
和迷信了教育.
為什麼這樣說呢.
但是對於一個人.
有時候有些東西不是學校教你的.
不是課程教你的.
如果是一個人的本意.
你會發覺最核心的東西就是良善.
守時.
認錯.
盡責.
這些待人處事的時候.
你會發覺都不是學校教的.
學校只不過提供一個場景給你.
很多這些東西.
在你三歲讀幼稚園之前.
你爸爸媽媽已經教你.

$^{1201}$錯了要認.
要守時.
或者要待人怎樣.
見到人會叫人的禮貌.
全部都是和你的生活場景.
和你最有密切關係的人.
去做那件事.
去教你那件事.
都不是學校教的.
學校只不過延伸那條場景.
延伸多些空間.
或者是調理化.
或者規範了那件事.
什麼叫做好.
這些都不是.
所以我們Flow Church.
去說這些課題的時候.
不是要redefine什麼叫做.
基督徒的syllabus.
也不是redefine什麼叫做.
good or bad.
反而是告訴你.
我們要用我們的能力.
去在我們的場景.
我們接觸的人當中.
去展現基督是什麼.
或者是展現基督信仰是什麼.
這個是重要的.
所以多謝你欣賞我們的課程.
後面是不是有一個.
終於聽完第一季的神學百課.
多謝John和潘Sir.
給我們上了總共八個月的神學百課.
我覺得這八堂非常有意思.
因為我第一次聽.
真的有人會將神學的東西.
基本的信仰確認.
那些神學的東西.
是普及到我們這麼多信徒的群體當中.
至於其實我見前面有幾位觀眾.

$^{1241}$都有提到的一些.
關於現在教會普遍的問題.
太過直接focus.
要去找答案那樣.
我會覺得對我來講.
問題會出於.
有時我們問的東西.
是連方向都錯了.
換言之是說.
有時我們太容易.
跳過了過程直接去到結論.
當然我會相信一些.
當這些東西是真的.
比如說其實很多信仰的概念.
基本的概念.
例如罪,救恩.
罪,救恩或者信心.
甚至是基督耶穌神真理本身.
無論在這八堂來說.
給我們知道一些.
關於這一方的概念.
其他前人神學家有什麼詮釋.
還是對我個人來說.
重新去看聖經.
其實發現是.
其實很多時候我們.
我們之所以這麼依賴去找答案.
是因為.
其實重點不是在答案本身.
因為答案是什麼其實不重要.
要走怎樣走其實不重要.
真正要知道怎樣走.
其實不是要靠這個.
而是靠的是.
你要找到你跟隨著什麼.
你為著什麼去做.
你清楚你跟隨著什麼.
在我們的情況下.
你跟隨著耶穌後繼是什麼樣的人.
又是什麼樣的神.

$^{1281}$你跟清楚.
自然就會.
你搞清楚整個的印象.
你就大概知道你應該怎樣去跟隨.
所以其實是.
所以我覺得.
潘Sir有幾個重點.
都是很反映我現在這一刻.
信仰教會的想法.
就是有時我們相處之間.
其實太多時間是專注在.
我覺得太少去做一些.
信仰的反映.
明明這是我們.
將我們賴以為生的一個必要的事.
但偏偏我們太多時間.
很快就跳過了這些是怎樣被正義.
然後直接就當它是一個.
好像當成了金波玉律.
但很少我們去.
重新想想去找找.
究竟這些是怎樣.
慢慢怎樣形成我們這些信念.
其實這個形成的過程.
是先是重要的.
而耶穌其實都說得出.
耶穌祂是當初所.
祂主要對人做的.
不是說答案.
要怎樣跟隨.
去過一個聖潔生活.
而是讓他知道這個世界的真相.
真理就是世界的真相.
這個無論放在什麼年代.
都是一樣不會變的.
所以其實重點在於.
怎樣過的時候.
就是要找回真相.
而這個求真的.
這個風氣的特質.

$^{1321}$在這個年代.
其中一個基督徒最需要重新培養.
就是群體裝備的特質.
所以這些百科其實.
定位都是一些基本的東西.
因為都是.
好像一些教會的基本百科.
得來其實又不是基本的東西.
因為都有一些.
重塑我們這麼多年想做基督徒.
因為大家都不是初信.
所以是讓大家去.
一些不是初信的人.
學一些初信的東西.
但那些東西其實都是.
正正是大家可以重新去學習的東西.
剛剛我才聊的時候.
我就想到我下一季叫什麼名字.
叫做神學百科之No Way Home.
你吃得很快.
其實都是預告.
我們下一季.
我們是想做一些.
關於海外基督徒.
和留德留基督徒.
兩批人不同的處境.
所以小初步在想.
下一季會分開兩個分支.
一班就是給海外的.
離開香港的基督徒.
還有怎樣可以去家裡.
怎樣可以.
那些課程都希望能夠可以去.
對於那個處境的幫助.
重塑一些所謂流散神學的東西.
對於留下來的基督徒.
繼續去延續一些.
再實踐多一些的東西.
所以如果這八課是基本的話.
下一個季度的東西.

$^{1361}$就是基本以上.
更加具體和更加.
要求的東西.
會去想一些.
剛才都有幾個關於.
怎樣做或者怎樣思考.
我自己都沒有想過.
我做信徒培訓.
或者課程設計的時候.
這些弟兄姊妹.
為什麼會有這樣的思路呢?.
我都認同.
楊醫他的神學反省.
環教會都很著重救贖神學.
很著重救贖進路.
得救或者是成聖.
或者在信徒成長過程當中的操練.
是否達標.
或者是否做到.
很著重那個訓練.
反而忽略了創造神學的平衡.
創造神學其中有幾個關鍵.
對於我覺得現在的基督徒來說.
或者作為一個結語的時候.
我覺得都可以給弟兄姊妹去想想.
創造神學其中一件事就是欣賞.
欣賞你現在可以做的空間.
或者你擁有的東西.
另外就是承擔.
即是有一樣東西在這件事當中交給你.
你能不能夠承擔到呢?.
第三件事就是護理.
你怎樣可以管理好你自己擁有的東西.
這都是創造神學會提醒我們.
我們現在無論你在海外或者在香港都好.
你做基督徒身份.
已經不是在說你是否得熟那件事.
是你現在在這個環境空間.
你怎樣去欣賞你可以擁有的空間.
你怎樣去承擔你可以做的東西.

$^{1401}$你怎樣去護理你可以擁有的資源.
這都是.
我覺得香港做神學反省的時候.
我自己常常都覺得.
不是對與錯的問題.
是有沒有做的問題.
我覺得都是希望在當中和大家一起去思考.
好啊.
下個場景.
過了八個月.
那些翻身位就差不多了.
多謝各位參與.
希望我們可以下季再見.
拜拜.
\newpage



\section{以西結書 47:1-2-20220122}
\label{sec:g_5XGfcpSSo}
\textbf{【網上崇拜】聖殿水浸的異象|以西結書47\_1-2|20220122 [g\_5XGfcpSSo]}
\newline
\newline
連結: \href{https://youtube.com/watch?v=g_5XGfcpSSo}{\texttt{ https://youtube.com/watch?v=g\_5XGfcpSSo}} ~~~~ 語音日期: 2022-01-22 
\newline
\newline
\hyperref[sec:FAOicP2MHq8]{\small{< < < PREV SERMON < < <}}
~
\hyperref[sec:index_chronic]{\small{[返順時目]}}
~
\hyperref[sec:index_scriptual]{\small{[返順卷目]}}
~
\hyperref[sec:FLZJqJSyEdc]{\small{> > > NEXT SERMON > > >}}
\newline
\newline
以西結書 47:1-2-20220122
\newline
\begin{longtable}{cl}
\hline
\hline
章節 & 經文 (和合本修訂版)\\
\hline
47:1 & \begin{tabularx}{0.7\textwidth}{X} 他帶我回到殿門,看哪,有水從殿的門檻下面往東流出,因為這殿是朝東的。水從殿的側面,就是右邊,從祭壇的南邊往下流。 \end{tabularx} \\ \\ \relax
47:2 & \begin{tabularx}{0.7\textwidth}{X} 他帶我出北門,又領我從外邊轉到朝東的外門,看哪,水從右邊流出。 \end{tabularx} \\ \\ \relax
47:3 & \begin{tabularx}{0.7\textwidth}{X} 他手拿繩子往東出去,量了一千肘,使我涉水而過,水到腳踝。 \end{tabularx} \\ \\ \relax
47:4 & \begin{tabularx}{0.7\textwidth}{X} 他又量了一千,使我涉水而過,水就到膝;再量了一千,使我過去,水就到腰; \end{tabularx} \\ \\ \relax
47:5 & \begin{tabularx}{0.7\textwidth}{X} 又量了一千,水已成河,無法過去;因為水勢高漲成河,只能游泳,無法走過。 \end{tabularx} \\ \\ \relax
47:6 & \begin{tabularx}{0.7\textwidth}{X} 他對我說:「人子啊,你看見了嗎?」他帶我回到河邊。 \end{tabularx} \\ \\ \relax
47:7 & \begin{tabularx}{0.7\textwidth}{X} 我回到河邊時,看哪,河這邊與那邊的岸上有極多的樹木。 \end{tabularx} \\ \\ \relax
47:8 & \begin{tabularx}{0.7\textwidth}{X} 他對我說:「這水往東方流,下到亞拉巴,直到海。所流出來的水,一入海就使水變淡。 \end{tabularx} \\ \\ \relax
47:9 & \begin{tabularx}{0.7\textwidth}{X} 這兩條河所到之處,凡滋生的動物都必存活;這水流到那裡,使那裡的水變淡,因此裡面有極多的魚。這河水所到之處,百物都必存活。 \end{tabularx} \\ \\ \relax
47:10 & \begin{tabularx}{0.7\textwidth}{X} 必有漁夫站在河邊,從隱‧基底直到隱‧以革蓮,全都成了曬網的場所。那裡的魚各從其類,好像大海的魚甚多。 \end{tabularx} \\ \\ \relax
47:11 & \begin{tabularx}{0.7\textwidth}{X} 但是沼澤與池塘的水無法變淡,只能作產鹽之用。 \end{tabularx} \\ \\ \relax
47:12 & \begin{tabularx}{0.7\textwidth}{X} 河這邊與那邊的岸上必生長各類樹木,可作食物;葉子不枯乾,果子不斷絕。每月必結新果子,因為這水是從聖所流出來的。樹上的果子必作食物,葉子可以治病。」 \end{tabularx} \\ \\ \relax
47:13 & \begin{tabularx}{0.7\textwidth}{X} 主耶和華如此說:「這是你們按以色列十二支派分地為業的地界,約瑟要得兩份。 \end{tabularx} \\ \\ \relax
47:14 & \begin{tabularx}{0.7\textwidth}{X} 你們承受這地為業,要彼此均分;我曾起誓應許將這地賜給你們的列祖,這地必歸你們為業。 \end{tabularx} \\ \\ \relax
47:15 & \begin{tabularx}{0.7\textwidth}{X} 「這地的疆界如下:北界從大海往希特倫,直到西達達口; \end{tabularx} \\ \\ \relax
47:16 & \begin{tabularx}{0.7\textwidth}{X} 又往哈馬、比羅他、西伯蓮(西伯蓮在大馬士革的邊界與哈馬的邊界中間),到浩蘭邊界的哈撒‧哈提干。 \end{tabularx} \\ \\ \relax
47:17 & \begin{tabularx}{0.7\textwidth}{X} 這樣,疆界是從大海往大馬士革地界上的哈薩‧以難,北邊以哈馬為界。這是北界。 \end{tabularx} \\ \\ \relax
47:18 & \begin{tabularx}{0.7\textwidth}{X} 「東界在浩蘭和大馬士革中間,基列和以色列地的中間,以約旦河為界。你們要量疆界直到東海。這是東界。 \end{tabularx} \\ \\ \relax
47:19 & \begin{tabularx}{0.7\textwidth}{X} 「南界是從他瑪到加低斯的米利巴水,經埃及溪谷,直到大海。這是南界。 \end{tabularx} \\ \\ \relax
47:20 & \begin{tabularx}{0.7\textwidth}{X} 「西界就是大海,從南界直到哈馬口對面。這是西界。 \end{tabularx} \\ \\ \relax
47:21 & \begin{tabularx}{0.7\textwidth}{X} 「你們要為自己按以色列的支派分這地。 \end{tabularx} \\ \\ \relax
47:22 & \begin{tabularx}{0.7\textwidth}{X} 要抽籤分這地為業,歸自己和那在你們中間寄居,生兒育女的外人。你們要看他們如本地出生的以色列人,他們要在以色列支派中與你們同得地業。 \end{tabularx} \\ \\ \relax
47:23 & \begin{tabularx}{0.7\textwidth}{X} 外人寄居在哪個支派,你們就在哪裡將地業分給他們。這是主耶和華說的。」 \end{tabularx} \\ \\
[1ex]
\hline
\hline
\end{longtable}
$^{1}$各位留堂的弟兄姊妹平安.
無論你是在香港 在家 在CityLab 在MIFC 在英國 加拿大.
我都歡迎你參加我們的留堂崇拜.
今天我們有一個很流暢的崇拜.
今天的講題叫做聖殿水浸的異象.
我給了英文名字叫做Flow Vision.
如果你看到我們在港台旁邊.
有非常好的建筑物隊安放了一盤金魚缸.
有三條黑色的魚.
應該看到三條.
今天我們會講一些很流暢的主題.
我們一起祈禱.
無論你在家或是現場.
我們一起同心去禱告.
祈禱.
求主你這樣去召喚我們的心靈.
讓我們恭敬的來到你面前.
我們來到去細聽你對我們的說話.
你對我們Full Church 每一個弟兄姊妹的說話.
讓我們來藉著你的聖經.
你親自的去教導我們.
怎樣去成為屬於你的教會.
幫助孩子不配.
但你的榮耀在我們當中.
引導我們去恭敬的去聆聽你自己的說話.
求主你這樣幫助我們.
我們會看一段以西傑書47章1到12節經文.
如果你有參加過Full Church崇拜.
你都知道我不是第一次講以西傑書.
甚至你都知道我其實很喜歡講以西傑書.
我們看過以西傑書第九章.
以西傑的行為藝術.
他長期飲水.
表達當時耶路撒冷的滅亡.
我們也看過以西傑書第37章.
以西傑在墳場的裡面.
叫做「骸骨復活」的經文.
今天我們會看一段以西傑書很後很後很後的經文.
是以西傑書的總結.
第47章的經文.

$^{41}$整卷以西傑書是有48章的.
所以以西傑書從第40章開始到48章.
是最後的段落部分.
晚年的以西傑先知再次見到耶和華上帝.
帶他見到的異象.
這是他人生裡面最後一個異象.
這個異象也是作為整個以西傑書的全卷書的拍檔.
我們可以看到這個段落的開始.
在第40章的開始.
以西傑書第40章第一節.
他這樣寫.
「第25年耶路撒冷被攻破後14年.
正在年初月之初十日.
耶和華的靈降在我身上.
祂把我帶到以茲列地」.
如果大家記得.
以西傑被上帝呼召.
是在他30歲的時候.
大概可以計算到.
現在以西傑是甚麼年紀.
以西傑是在他被擄第五年.
被上帝呼召.
當時他是30歲.
如果他剛好被擄了25年.
可以計算到.
大概以西傑是剛好在50歲的時間.
以西傑是一個國破家亡.
在異鄉外國生活了25年的.
突發地中海的中坑.
一個上世紀活下來的人.
人到中年做了幾十年人.
已經知道世界的運作方式.
不過對生活仍然帶著一些堅持.
對上帝的理想仍然帶著一些盼望.
仍然願意去見到.
在這個流落異鄉當中.
去嘗試去盼望一些可能性.
以西傑今年50歲.
你說30而立.
40而不劃.

$^{81}$50而知天命.
男人50是一個非常有意思的數字.
如果你翻查聖經的時候.
50歲其實是祭司退休的年紀.
這是在文數第8章25節所記載的.
以西傑今年50歲.
正正是面對祭司退休的年齡.
如果他當年又成為一個聖殿祭司的時候.
不過以西傑並沒有.
25年前當聖殿被毀.
耶路撒冷淪陷.
以色列人流亡的時候.
從此以西傑的生命就被擄到另一個方向.
以西傑的理想被擄去.
不能夠留在自己的家鄉耶路撒冷.
成為一個聖殿裡面的祭司.
他就被人擄到異鄉.
在異鄉裡面漂泊了25年的時間.
今天以西傑50歲.
或者這時候以西傑正在想.
如果一個What if的平衡宇宙.
一個耶路撒冷沒有滅亡的宇宙.
以西傑30歲的時候就成為聖殿裡面的祭司.
一直不休一直努力侍奉.
侍奉到50歲退休.
退休之前可能好好地為自己聖殿裡面.
即是祭司的工作去教教探.
接一接探.
然後就打算在耶路撒冷裡面找一個小屋.
然後就退休 種花 養魚.
但這些全部都是幻想.
以西傑突然想到.
他依然是在巴比倫裡面已經25年.
聖殿都被毀5月25年.
不過可能是天父上帝的安排.
亦都可能是上帝的一些幽默.
當以西傑50歲的年紀.
上帝就令以西傑竟然見到聖殿的異象.
聖經說.
在上帝的異象中帶他到以西的地.

$^{121}$安置在至高的山上.
在山上的南邊又彷彿一座城建立.
他帶我到那裡見到有人.
顏色如銅 手拿麻繩和涼薄的缸.
站在門口 但對我說人之也.
凡我所指示你的.
你都要用眼看 用耳聽 並要放在心上.
我帶你到這裡來 唯要指示你.
凡你所見的 你都要告訴以西列家.
上帝帶著以西傑到至高的山上.
到高宿六魂的山峰裡面.
他竟然見到新的聖殿.
不知道你能否明白這一刻以西傑的感受.
一個失去聖殿25年的人.
當他到50歲的時候.
他親眼再次見到聖殿的感受.
那一刻的感動.
那一刻的感動 我想大概就是一個25歲的香港青年人.
2019年的時候因為某些事而被捕.
然後被監禁了25年.
25年之後2044年.
這位年青人50歲的時候出來.
香港恢復他理想的模樣.
然後他在金鐘的煲底裡面見到一些東西.
這個大概就是以西傑當時見到聖殿的時候的感覺.
當然這個聖殿的意象是一個異象.
所謂異象就是一個vision.
一個上帝為以西傑去揭露的一個視野.
這個視野不是現在這一刻去發生.
但是它是一個將來會發生的事情.
這個就是異象的意思.
上帝要以西傑去見到他可以見到什麼.
或者說上帝要以西傑應該要他見到什麼.
所以在以西傑書第40到48章.
整個以西傑書的總結就在這8章裡面.
以西傑書正正是這個新的聖殿.
新的搖曬聖殿的事情.
簡單講講這8章的內容.
以西傑書第40到42章.
講的是整個聖殿新聖殿的佈局擺設.

$^{161}$聖殿裡面有多少擺設.
多少爪多少爪.
多少竿多少竿.
東門多高.
外院多闊.
內殿有多大.
這些的經文就在這裡出現了.
就好像昔日門見殿的時候的數據一樣.
以西傑書第43到46章.
講的是有關聖殿一些的律法.
及那個聖殿的規矩.
祭司的條例.
憲制規定等等.
那些節期的禮儀等等.
總而言之.
以西傑書第40到46章.
是一切有關聖殿裡面的描寫.
突然之間我們就去到今天的經文.
第47章的經文.
今天我們講道的經文.
我們先看看經文.
第47章第12節的經文.
基本上今天講道的聖經的譯本.
是一個和合本修訂版.
再加上我翻譯的地方.
改了一些字.
聽到出來.
以西傑書第47到12節.
將於2020年7月1日立法會會議席上處理的事項: 政務司司長.
簡單來說.
這就是我們今天要講的經文.
以西傑書第47章裡面.
聖殿出水的意象.
我想今天講的是叫什麼呢.
叫聖殿出水還是聖殿流水.
聖殿滲水.
想了很久.
我就想成聖殿水浸.
大概的概念就是這些.
所以今天講的經文是全中心最流行的經文.

$^{201}$流行出的經文.
以西傑書.
看到有水從聖殿裡面流出.
然後這些從聖殿裡面流出來的水.
這些流出來的水就越來越多.
越來越深.
越來越遠.
然後就流遍整個的全地.
這個就是整個經文裡面所講的概括內容.
不過我們要詳細看的.
先看第一節.
聖經說.
他帶我回到殿的門口看.
有水從殿的殿台下面往東流出.
因為這殿是朝東的.
水從殿的側面就是右邊.
從濟南的南邊往下流.
首先比較技術一點的一點.
就是如果你看和合本的翻譯.
舊和合本的翻譯.
翻譯並不是這樣的.
翻譯是怎樣寫的.
麻煩下一條.
他說以西傑被帶領.
不是 之前那裡.
沒有了.
有一個是兩個經文版本對照.
舊版本是說.
以西傑被帶領到聖殿的門.
沒有了 不要緊.
舊的和合本是寫著聖殿的門.
然後看到新聞就說.
有水從殿的門檻下面流出來.
不過參考不少學者看法的時候.
原文的字其實不是門檻.
不是舊版本的門檻.
而是電台的一個podium.
第一節都這樣說.
所謂電台podium是什麼意思.
就是一個寶座下面的.

$^{241}$在這裡.
不是 這是下一個.
不要了 我沒有預備到.
所謂的電台podium是什麼意思.
就是一個寶座下面的台階.
就好像我們現在的Fullshot裡面的台階.
但高了一層的台階.
這個台階是上面放寶座的.
想像一下一個寶座下面高出來的台階.
就像放著椅子的一個地方.
意思是說.
以色列就看到聖殿有水流出.
水的源頭其實不是和合本所說的門檻.
不是那個門口流出來.
而是那個寶座的台階流出來.
就是這麼技術性.
你會問 有分別嗎.
水從聖殿的門檻流出.
還是從台階流出.
有什麼分別.
要不要分得這麼細.
是有分別的.
我想說的就是.
幾年前我唱一首歌.
洪水泛濫之時 耶和華坐著為王.
這個經文.
經文是來自於詩篇第29篇第10節.
洪水泛濫之時 耶和華坐著為王.
這首歌唱什麼的意象.
就是耶和華的寶座之下.
縱然有洪水泛濫.
耶和華仍然在洪水之上.
就是坐著為王.
就是這樣的一幅圖畫.
所以以色列書第47章第1節.
講的是對應著這個圖畫.
洪水泛濫之時 耶和華仍然坐著為王.
同樣地 聖殿流出來的水.
生命的活水.
同樣都在聖殿的寶座之下流出來.

$^{281}$所以不是從門檻流出來.
而是從寶座的台階流出來.
在耶和華的寶座之下.
縱然我們會經歷洪水泛濫.
但在同一個寶座之下.
上帝更加允許有生命的活水流出來.
所以以色列就這樣說.
以色列就見到聖殿的寶座台階下面.
有水流出 並且向外流出.
以色列就沿著聖殿流出的水.
一直走一直走 走到很遠很遠.
第三節他就說.
他手拿繩子往東出去.
糧料一找 使我涉水而過.
水到腳果 他又糧料一遷.
然後使我涉水而過 水到膝.
再糧料一遷 使我過去 水就到腰.
又糧料一遷 水就已經成河 無法過去.
因為水水更像成河 只能游 永無法走過.
所以說從聖殿流出來的水.
越流越多 越流越深 越流越遠.
你會問一千爪有多遠.
一千爪用今天的單位來看.
就大概五百多米左右.
所以以色列每走五百米 水就再深一點.
再走五百米 水就再深一點.
初時到腳果 腳硬.
然後到膝蓋 然後到腰.
然後以色列就要游水.
用今天做例子.
可以想像一下在MFC突然發現台就滲水.
這個大概是台階的流水.
流出來.
然後水就流到我們的門口.
然後再流 流到下面大堂.
沿著水繼續走 走到五百米.
走到紅磡湖街附近.
水就大概走到腳硬位置.
然後再走 再走 走到紅磡火車站.
水就到膝蓋.

$^{321}$過閘時要在膝蓋過閘.
然後再走 再走五百米.
走到科學館附近.
整個差不多到腰的位置.
然後你就間中走.
走到喜伯倫堂.
你就要在安士丁游水.
大概是這樣的情況.
非常壯觀的情況.
由紅磡這裡流到喜堂.
大概是這樣的圖畫.
亭姐妹 正正在這個時候.
聖經卻問西傑一個非常重要的問題.
第六節.
人子啊 你看見了什麼?.
亭姐妹 你看見了什麼?.
如果有一天你回到MIFC.
真的看見MIFC水浸的話.
流到門口 流到紅磡湖街 流到紅磡火車站.
流到喜堂 回到喜堂.
你第一個想到的想法是什麼?.
回到這裡 問問Sam.
回到公司看到MIFC流水的話.
你想到什麼情景?.
分享一個我自己昨天寫講章的時候遇到的經歷.
我昨天在辦公室寫講章.
寫到大概是整個上午都在寫講章.
中午我回到家裡吃午餐.
你知道我家跟辦公室是很近的.
一分鐘的距離.
回到家裡 我女兒已經在家裡了.
因為我女兒基本上是全國施學停課.
要在家裡放大.
她在家裡玩 放學.
我回到家裡 打開廁所的門.
看到我女兒在廁所裡的洗手盆.
就看到這樣的情景.
她在做什麼呢?.
她在玩水浸芭比.
她自己在水裡把芭比浸在水裡.

$^{361}$她說這是一個水泳池.
我覺得這是順德給我的一個意象.
當我看到這幅圖畫的時候.
我明白水浸聖殿是一個什麼的事情.
真心說一句 水浸不是一件好玩的事情.
如果今天教會水浸.
你跟我問斗問題就是問什麼.
教會的家生怎麼辦.
教會有沒有買火險水險.
這個禮拜六 十八 怎麼辦.
同樣道理 聖殿水浸.
我也會不禁問.
聖殿裡面的神聖之物怎麼辦.
聖所的水怎麼辦.
至聖所又怎麼辦.
尤其是以色列的書.
用了六章的經文.
從四十章開始.
四十一 四十二 四十三 四十五 四十六章.
不斷去詳細描述聖殿裡面的內容.
聖殿裡面每一個擺設.
多少多少爪 多少多少竿.
東門多高多高 外院多闊多闊.
內殿多大多大多大.
聖殿的規則是怎樣.
祭司的條例是怎樣.
憲制的規定是怎樣.
很多很多這些細節.
然後去到第四十七章.
整個聖殿就浸了.
然後聖經就再沒有將重點.
放在聖殿裡面.
人子啊 你看見了什麼.
究竟整個意象.
要告訴我們什麼.
上帝要以色列去見到什麼.
而上帝要以色列看見的.
不是這個聖殿本身.
上帝要以色列見到的是.
以下那個經文所說的一些改變.

$^{401}$我們再看下面的經文.
第七節開始說起.
祂帶我回到河邊.
我回到河邊時.
河這邊與那邊的岸上有幾多的樹木.
祂對我說.
這水往東方流.
下到阿拉巴直到海.
所流出來的水一入海.
就使水變淡.
這兩條河流所到之處.
凡滋生的動物都必存活.
這水流到那裡.
使那裡的水變淡.
因此裡面有極多的魚.
這河水所到之處.
百物都必存活.
必有漁夫站在河邊.
從引基底直到引以隔連.
全部都成了曬網之場所.
所來的魚各從其類.
好像大海的魚甚多.
但是小宅與池塘的水無法變淡.
只能用作產鹽之用.
河這邊與那邊的岸上.
必生割量的樹木.
可作食物.
葉子不枯乾.
果子不斷絕.
每月必結生果子.
因為這水是從聖所流出來的.
樹上的果子必作食物.
葉子必可以治病.
當以色列回頭一看的時候.
以色列看到的.
聖殿流出來的水.
流過的經過的地方.
那個地方再一次.
再重新有生命力.
提九節說.

$^{441}$這兩條河所到之處.
凡滋生的動物都必存活.
這水流到那裡.
使那裡的水都變淡.
因此裡面有極多的魚.
如果你心水清的時候.
你會知道.
先說的是什麼.
先說的是死海.
死海再不是死.
這水流到那裡.
使那裡的水變淡.
大家知道以色列近東的死海.
有很高的鹽分.
所以沒有生物能夠在海裡生存.
任何生物一流到去的時候.
最後只會死亡.
所以死海其實是一個.
絕望死亡的地方.
任何的河流到死海.
全部都會死掉.
不過當聖殿流出來的水.
流到死海的時候.
死海就轉化了.
裡面有極多的魚.
就在河水所到之處.
百物都必存活.
甚至世別之說.
任何的地方.
樹木都變成食物.
葉子不枯乾果子不斷絕.
每月必有新的果子.
等等等等.
非常非常美好的圖畫.
如果你再深水清的時候.
你會發現.
在第九節經文裡面.
和本修正版裡面.
比舊版本又多了一個字.
請看下一個.

$^{481}$舊版本寫何水所到之處.
是河水所到之處.
新版本怎麼說.
是說這兩條河所到之處.
加了一條河在這裡.
以前只有一條.
新版本就多一條出來.
原來舊的和本.
是翻譯成一條河.
我認為原文裡面.
應該是說兩條.
所以你看到.
翁豐所說.
耶穌生來的時候說.
人若學了.
可以到我這裡來學.
信我的人就如經上所說.
從他腹中要流出.
活水的江河來.
裡面是一個眾數.
所以都是兩條河.
不是一條河.
然後下一個.
如果你看到這幅圖.
就更加明白情況.
這幅是古代近東的圖畫.
你看到什麼.
就是人拿著水瓶.
是怎麼流出來的.
是兩邊流出來的.
所以說這個情況.
從聖殿流出.
就像這個.
向東面流向西面.
意思是什麼.
就是整個世界都因此被充滿.
東到西.
全地都被這個生命的活水.
完完全全的充滿.
這個就是整個意象的意思.

$^{521}$從聖殿裡面.
那個寶藏下面流出來的生命活水.
這個生命水要向外湧流.
流到世界上每一個地方.
更新這個世界.
其實每一個地方都能夠充滿生命的盼望.
復興所有的萬民.
所以說這段經文是一段很flow出的經文.
一段很flow的經文.
你問那個聖殿水怎麼算.
怎麼搞好.
聖殿的水怎麼算.
對不起.
你不要這麼掃興.
這個從來都不是聖經關心的事情.
甚至有說聖殿的存在.
只不過是一個上帝復興萬物的方式.
聖殿所謂是神聖.
只不過他本身不是有多神聖.
門有多高 殿有多闊.
聖殿的神聖.
是因為他成為了耶和華上帝根深蒂固的源頭.
這個就是以色列的書所講的內容.
今天我很想用這段經文跟大家講一下成為我們flow church的經文.
我想每一個flow church的頂尖妹都很想.
都要思考這段經文.
想想我們flow church作為教會的意義.
在這個時實越來越緊張的年代.
大家都戴著口罩.
連聚會都很艱難的年代.
讓我們認真重新思考一個很簡單的問題.
為什麼教會 為什麼有教會.
為什麼要有教會.
我不是問為什麼要有聚會.
而是問為什麼要有教會.
為什麼香港要有教會.
為什麼你去到外國要有一間教會.
教會為了什麼而存在.
教會只不過是耶穌升天和在來之間.
一個卑微短暫帶著使命的存在.

$^{561}$教會是見證耶穌基督的群體.
教會是承載上帝榮耀福音的群體.
教會是上帝在地上全陽盼望的群體.
這個是我們存在的原因.
因此可以這樣說.
教會也不是為了牧羊而存在的.
牧羊很重要.
當然牧羊是我們作為牧者一個很重要的關懷.
但這不是教會存在的目的.
如果教會的存在是為了牧羊基督徒的話.
這是一個很嚴重的事情.
教會的使命不是牧羊基督徒.
因為這群被牧羊的人.
他們本身 這些基督徒.
都有世界上的使命.
教會不是為了自己人而存在.
教會也不是為了思念自己單單的生存.
當然現在有海量的信徒.
在外面很多需要牧羊.
香港有幾百間堂會面對著堂敲崇拜的問題.
但這也不是我們最大最終極的關懷.
這不是我們教會存在的原因.
剛才也提到 剛剛過去的星期三.
1月19日是Full Church三週年.
這是我當日自己拍的獨家照片.
當日崇拜的照片.
我突然想起.
今天是1月19日 Full Church三年了.
不如我們發個帖文吧.
吃個餃子慶祝吧.
三年前我們在石叫密神教會第一次崇拜.
焦聚一群離開教會的弟兄姊妹.
再一次聚在一起.
成為一間耶穌基督的教會.
但如果Full Church是一間耶穌基督的教會.
Full Church就不能純粹以焦聚基督徒而存在.
這只是教會的第一步.
教會也不可以純粹為了牧羊而存在.
因為牧羊只是教會的第二步.
教會終極的一步.

$^{601}$永遠都不是在教會的內部.
而是在教會的外面.
如果Full Church只是一間耶穌基督的真正教會.
Full Church就不能夠想太多自己.
(教育局局長).
剛才一班同工開始思考計劃B.
大半年很多的神學院機構教會都會思考計劃B,C,D.
如果香港有什麼東角豆腐.
我們可以怎樣?.
去到DEFCON5的時候教會可以怎樣?.
需要找路包嗎?.
需要一個離岸戶口嗎?.
需要預備一個海外基地嗎?.
最後我們的決定是.
我們沒有計劃B.
不刻意去思考計劃B.
不刻意去做某些事去保住Full Church.
我們不保.
與其刻意去做某些事.
不做某些事.
從而減低自己死亡的機會.
或者提高自己生還的可能.
與其刻意去延續自己的壽命.
倒不如無限復活.
死完又死.
結業之後又重聚.
這樣是更加有意思.
你保住它.
嘗試去減低它死亡的機會.
可能不是它.
這不是水特性嗎?.
你可以打破水瓶.
你可以截斷江河.
你可以凍結它.
你可以蒸發它.
但你不能完完全全地消滅水.
Full Church沒有了.
我們這班人仍然是水.
隨時隨地再結集.
再焦聚.

$^{641}$《耳塞傑書》告訴我們.
人子啊!你看見了嗎?.
頂曉門你們的眼光放在哪裡?.
時勢越艱難.
越是要將眼光放在外面.
越不能將眼光放在自己那裡.
當你只看著自己的生存.
想想自己如何生存崇拜的時候.
教會失去了我們自己.
這段時間我在學電單車.
電單車要考車路八字.
打U-Turn.
一轉就很容易撞車.
其實都有點難.
師傅教我們.
當轉急彎時.
首先要放鬆.
相對夾緊座位.
最重要的是.
眼不能看著自己前面的車.
眼要嘗試看遠一點.
看出一點.
看出彎後的地方.
然後給多一點油.
就能過了.
真的.
你要憑信心不要看著自己的車.
看一個和你的車沒有關係的地方.
這樣走過去.
同一道理.
面對著一個風高浪急的年代.
你要轉的彎越急.
你的眼睛就越不能看著自己.
你要憑信心看遠一點.
看闊一點.
看外面的需要.
當你只看著自己的需要.
看著自己教會能不能生存的時候.
你就要派車了.
如果Full Church的Full字有一個方向.

$^{681}$這個Flow必定是Outflow.
Flow出去.
讓我讀出三周年那段文字.
我們一起共勉.
三年前星期六.
我們懷著戰經的心情.
在實踐尾神智會的禮堂.
開始了第一次留堂崇拜.
從此以後.
每逢星期六晚.
我們帶著歡笑.
帶著眼淚一起結集.
從實踐尾到Occident.
從Facebook到YouTube.
從反修例運動到全球疫情.
從認識到再度別.
我們一直堅持著.
成為上帝的流動群體.
這三年.
生命一直在湧流.
頂智妹.
生命要繼續湧流.
我請你禱告.
再次求你建立我們教會.
讓我們一群整個Fold Church群體.
都懂得在這個時候.
我們要將眼光望得闊.
望得遠.
我們不求我們Fold Church的生存.
單單求我們盡上我們時代的使命.
讓我們在更加危急的時間裡.
我們要目標的不是.
怎樣去得到被牧養.
得到怎樣的能力去發展.
而是我們求尊你.
繼續使用我們.
讓我們成為這個世界的祝福.
讓我們成為在這個年代裡.
傳播盼望群體.
讓福音的好處.

$^{721}$讓生命的關懷和愛.
仍然藉著我們這個群體.
來去發放出去.
這是你呼召我們成為教會的使命.
讓我們一群頂智妹.
能夠可以記得.
怎樣在這個年代裡.
繼續成為你時代的見證.
奉尊命求.
謝謝 阿們.
\newpage



\section{撒母耳記上 13:1-23-20220129}
\label{sec:FLZJqJSyEdc}
\textbf{【網上崇拜】召命人生|撒母耳記上13\_1-23|20220129 [FLZJqJSyEdc]}
\newline
\newline
連結: \href{https://youtube.com/watch?v=FLZJqJSyEdc}{\texttt{ https://youtube.com/watch?v=FLZJqJSyEdc}} ~~~~ 語音日期: 2022-01-29 
\newline
\newline
\hyperref[sec:g_5XGfcpSSo]{\small{< < < PREV SERMON < < <}}
~
\hyperref[sec:index_chronic]{\small{[返順時目]}}
~
\hyperref[sec:index_scriptual]{\small{[返順卷目]}}
~
\hyperref[sec:vVSfVrNtuKk]{\small{> > > NEXT SERMON > > >}}
\newline
\newline
撒母耳記上 13:1-23-20220129
\newline
\begin{longtable}{cl}
\hline
\hline
章節 & 經文 (和合本修訂版)\\
\hline
13:1 & \begin{tabularx}{0.7\textwidth}{X} 掃羅登基的時候年三十歲,作以色列王二年。 \end{tabularx} \\ \\ \relax
13:2 & \begin{tabularx}{0.7\textwidth}{X} 掃羅從以色列中選出三千人:二千跟隨掃羅在密抹和伯特利山區,一千跟隨約拿單在便雅憫的基比亞。其餘的百姓,掃羅打發他們各自回自己的帳棚去了。 \end{tabularx} \\ \\ \relax
13:3 & \begin{tabularx}{0.7\textwidth}{X} 約拿單攻擊非利士人在迦巴的駐軍,非利士人聽見了這事。掃羅就在遍地吹角,說:「讓希伯來人都聽見。」 \end{tabularx} \\ \\ \relax
13:4 & \begin{tabularx}{0.7\textwidth}{X} 以色列眾人聽見掃羅攻擊非利士的駐軍,又聽見以色列為非利士人所憎惡,百姓就跟隨掃羅,在吉甲集合。 \end{tabularx} \\ \\ \relax
13:5 & \begin{tabularx}{0.7\textwidth}{X} 非利士人集合,要與以色列人作戰。他們有戰車三萬輛,騎兵六千,士兵像海邊的沙那樣多。他們上來,在伯‧亞文東邊的密抹安營。 \end{tabularx} \\ \\ \relax
13:6 & \begin{tabularx}{0.7\textwidth}{X} 以色列人見自己危急,軍隊被圍攻,百姓就藏在山洞、叢林、巖隙、地窖和深坑中。 \end{tabularx} \\ \\ \relax
13:7 & \begin{tabularx}{0.7\textwidth}{X} 有些希伯來人過了約旦河,逃到迦得和基列地。掃羅還在吉甲,所有的人都戰戰兢兢地跟隨他。 \end{tabularx} \\ \\ \relax
13:8 & \begin{tabularx}{0.7\textwidth}{X} 掃羅照著撒母耳所定的日期等了七日。但是,撒母耳還沒有來到吉甲,百姓就離開掃羅散去了。 \end{tabularx} \\ \\ \relax
13:9 & \begin{tabularx}{0.7\textwidth}{X} 於是掃羅說:「把燔祭和平安祭帶到我這裡來。」掃羅就獻上燔祭。 \end{tabularx} \\ \\ \relax
13:10 & \begin{tabularx}{0.7\textwidth}{X} 他剛獻完燔祭,看哪,撒母耳就到了。掃羅出去迎接他,向他問安。 \end{tabularx} \\ \\ \relax
13:11 & \begin{tabularx}{0.7\textwidth}{X} 撒母耳說:「你做了甚麼事啊?」掃羅說:「因為我見百姓離開我散去,你又不照所定的日期來到,而且非利士人已在密抹集合; \end{tabularx} \\ \\ \relax
13:12 & \begin{tabularx}{0.7\textwidth}{X} 我說:『現在非利士人已經下到吉甲來攻擊我,可是我還沒有向耶和華禱告。』所以我就勉強獻上燔祭。」 \end{tabularx} \\ \\ \relax
13:13 & \begin{tabularx}{0.7\textwidth}{X} 撒母耳對掃羅說:「你做了糊塗事了,沒有遵守耶和華-你神吩咐你的命令。不然,耶和華會在以色列中堅立你的國度,直到永遠。 \end{tabularx} \\ \\ \relax
13:14 & \begin{tabularx}{0.7\textwidth}{X} 現在你的國度必不長久。耶和華已經尋著一個合他心意的人,立他作百姓的君王,因為你沒有遵守耶和華所吩咐你的。」 \end{tabularx} \\ \\ \relax
13:15 & \begin{tabularx}{0.7\textwidth}{X} 撒母耳就起來,從吉甲上到便雅憫的基比亞。掃羅數點跟隨他的百姓,約有六百人。 \end{tabularx} \\ \\ \relax
13:16 & \begin{tabularx}{0.7\textwidth}{X} 掃羅和他兒子約拿單,以及跟隨他們的百姓,都住在便雅憫的迦巴,非利士人卻在密抹安營。 \end{tabularx} \\ \\ \relax
13:17 & \begin{tabularx}{0.7\textwidth}{X} 有突擊隊從非利士營中出來,分成三隊:一隊往俄弗拉到書亞地去, \end{tabularx} \\ \\ \relax
13:18 & \begin{tabularx}{0.7\textwidth}{X} 一隊往伯‧和崙去,一隊往邊界,下望朝著曠野的洗波音谷。 \end{tabularx} \\ \\ \relax
13:19 & \begin{tabularx}{0.7\textwidth}{X} 那時,以色列全地找不到一個鐵匠,因為非利士人說:「恐怕希伯來人製造刀槍。」 \end{tabularx} \\ \\ \relax
13:20 & \begin{tabularx}{0.7\textwidth}{X} 以色列眾人要磨鋤、犁、斧、鏟,就各自下到非利士人那裡去磨。 \end{tabularx} \\ \\ \relax
13:21 & \begin{tabularx}{0.7\textwidth}{X} 磨鋤或犁的價錢是三分之二舍客勒,磨斧或修整刺棒的價錢是三分之一舍客勒。 \end{tabularx} \\ \\ \relax
13:22 & \begin{tabularx}{0.7\textwidth}{X} 所以到了戰爭的日子,所有跟隨掃羅和約拿單的百姓找不到一個手裡有刀有槍的,惟掃羅和他兒子約拿單有。 \end{tabularx} \\ \\ \relax
13:23 & \begin{tabularx}{0.7\textwidth}{X} 非利士人的一隊駐軍出來,到密抹的隘口。 \end{tabularx} \\ \\
[1ex]
\hline
\hline
\end{longtable}
$^{1}$我們一起祈禱.
慈悲人愛的天父.
我們恭敬仰望.
因為承擔你的呼召.
從來都不是因為我們有多麼的把炮.
又或者有多強的韌力.
甚至有多大的才能.
承擔你的呼召.
是完全傳你莫大的恩典和鼓勵.
所以我們懷著一個恭敬的心.
仰望的心.
我們認定沒有了你的話語.
我們就好像一坨爛泥.
我們祈求你的話語成為我們腳前的燈路上的光.
明光指引在迷霧和黑暗當中.
我們看到亮光和方向.
求主幫助孩子講解你的話語清楚.
也求主讓我們有謙卑的心去聆聽.
奉耶穌基督的名祈禱.
阿們.
今天講到的經文.
是來自《撒姆爾記》上的十三章這段經文.
這段經文我今天改了一個題目.
叫做「照命人生」.
其實這個題目也挺老套的.
因為我想不到一些很前衛的東西.
我也覺得有時候上帝的話語也是很老套.
但也是歷久常新的一段話語.
當我們看下一章的powerpoint的時候.
你就會發現其實整個的撒姆爾記.
是坐落在一個非常大的上下文的結構當中.
老早在1943年.
有一位德國的學者很出名.
他的名字就叫做Martin Roth.
他這位學者就提出一個學說.
而這個學說到現在也成為一個非常經典的學說.
去到怎樣理解撒姆爾記.
這個學說published了一本書.
在1943年用德文出版.
英文的翻譯就在1981年翻譯了.

$^{41}$他的名字就叫做The Deuteronomistic History.
然後中文就叫做《新殿歷史》.
當然他這個著作的基調和理論是非常簡單的.
簡單的就是他相信聖經由生命記.
然後約書亞記.
然後就是事司記.
撒姆爾記上下.
還有列王記上下.
一個這麼長的歷史書.
就應該由一位我們叫做historian.
就是一位叫做歷史學家.
去到編寫而成.
而這位歷史學家當然他不是由第一筆.
在生命記那裡開始.
然後最後那筆在列王記下.
每一粒字他都自創.
他應該就不是這樣.
他是引用不同的來源.
編輯而成一卷的歷史書.
而這卷歷史書他有一種我們叫做編輯取向.
而這個編輯取向Martin Roth.
就認為這位作者的編輯取向.
就是根據生命記的神學.
去到編輯了整部這麼長的歷史書.
而編輯的年期.
其實就是秘魯的期間去到編輯而成.
什麼叫做生命記的神學呢.
生命記的神學我就知道.
原來生命記有一個很重要的神學.
就是在第六章的第四到第五節那裡.
而我們也知道當薩姆爾記坐落在一個.
我們叫做新典歷史.
為什麼叫做新典呢.
大家知道有個新字.
就知道這樣的編輯取向.
是和生命記有關的.
典就是典籍的意思.
這樣的生命記究竟有什麼神學重點呢.
我相信其中一個最重要的.
就是第六章的第四到第五節.

$^{81}$我就翻譯給大家看.
這段經文說.
聽亞以色列主我們神是獨一的主.
你要盡心盡性盡義愛主你的神.
我們非常熟悉這段經文.
這裡的獨一性.
其實關乎這個一字.
就不是一個存在性的一.
是一個關係性的一.
就好像我身邊不能和其他女性發生任何的關係.
只能和我老婆發生唯一的關係.
就不代表其他女性不存在.
其實她們都存在.
不過因為我和我老婆產生一個文約的關係.
就當其他女人都不存在.
類似是這樣.
其他女人不存在.
你就知道其實你們都存在.
所以存在不同於存在.
存在性的存在和關係性的存在.
是兩種不同的存在.
同樣地你如果當上帝是唯一的話.
不代表其他偶像不存在.
或者其他神明不存在.
其實他也存在.
不過他們和你沒有什麼關係.
你唯一只能和上帝發生愛的關係.
愛的定義在這裡非常具體.
因為愛不是解造一種感受.
也不是你要愛一些東西.
即是擁有一些東西這麼簡單.
其實愛是代表一個具體的行動.
就是遵守耶和華的誡命律例殿章.
所以Martin Lok就認為.
生命記這個神學取向.
是主導了整個新殿歷史.
特別是今天所提到的經文.
就是撒姆爾記上的十三章.
我們怎樣理解掃羅庇氣的經文.
我們不得不承認.

$^{121}$當我們今天改個題目.
就叫做「照命人生」的話.
反之我們就認為.
現在是一個沒有照命的世代.
雖然很多人只是尋找很多人生的意義.
但很少人找到一個終極的意義到底是怎樣.
近來有一個調查就說.
七成的青少年生活是非常之苦悶.
當中苦悶的主要原因就是無所事事.
五成人認為溫習又或者讀書是不感興趣.
由此可見他們感到苦悶.
是由於人生缺乏目標理想.
沒有奮鬥的動機所致.
而根據另一個調查顯示.
這個調查經常都會有.
就是世界上比較富強的地區的自殺率.
是遠比貧困的地區為高.
這就證明一點.
就是人想自殺不是因為他很窮.
而是因為他失去了人生的意義.
沒有照命和沒有盼望.
不想再生存下去.
我們或許不會因為這樣而至尋短見.
但這些調查報告精心.
反映了我們內心的心聲.
就是我們希望可以在人生充滿意義.
不甘心過一個平平無奇的生活.
偏偏因為我們生活中.
大大小小不同的責任和困難.
要我們不斷地逃避和思想.
活出人生的照命.
究竟我們怎樣去忠於照命.
活出豐盛基督徒的生命呢.
我們就很想藉著今天.
素羅這段經文.
給大家去到一個反思.
素羅這段經文其實有上下文.
撒姆爾記上的十三章的上文就是十二章.
十三章前面當然是十二章.
十二章前面就是第八章.

$^{161}$我們由第八章開始.
撒姆爾記上的第八章.
是描述以色列人要求撒姆爾為他們立王.
以至這個王可以好像列國一樣為他們征戰.
撒姆爾得到耶和華的允許之後.
他就在第九章去到高納素羅為王.
但是這位被耶和華所呼召的君王.
似乎一直都不想承擔作為君王的呼召.
又或者這麼說.
就是想拿了君王的光環.
但是不願意承擔君王的責任.
原來喜歡拿光環.
但是不喜歡這個責任.
其實這個就是素羅.
這位被耶和華所呼召做王的.
似乎一直不願意承擔.
無論他在撒姆爾記上的第十章二十二節.
描述到他藏在器皿當中.
不願意出來做王.
又或者撒姆爾記上的第九章二十一節.
就素羅強調自己屬於以色列之派當中.
最少的彼雅文之派.
當中最少的.
還有我們發現撒姆爾記上的十一章.
他只不過按照耶和華的吩咐.
只不過是攻打實力比較弱的亞門人.
而遲遲不肯按照上帝的吩咐.
去攻打實力比較強的非利士人.
種種的表現.
說明素羅不斷逃避耶和華對他君王的呼召.
所要應付面對的責任.
直到撒姆爾記的十二章當中進行一篇.
是撒姆爾一個非常重要的演說.
說明了君王 耶和華和先知.
在以色列王國的時代.
他們先後的角色.
他們的佈局應該怎樣.
和君王聽命的重要性之後.
就來到今天我們所讀的經文.
就是撒姆爾記上的十三章.

$^{201}$一到十五節的經文.
去到描述素羅怎樣面對上帝呼召.
首要做的任務.
而這個呼召首要做的任務.
就是攻打非利士人.
究竟我們可以在素羅私自獻祭.
又或者避棄的罪事當中.
得到什麼的借鏡呢.
我們首先看看第一點 下一章.
我們看看的.
就是第十三章的第一到第七節.
而這一點我改了一個題目.
就是勇於承擔剿命.
拒絕不斷逃避.
我們switch一switch看看經文.
看看經文 一到七節.
他就這樣說.
就說素羅登基的時候就年三十歲.
這個版本是三十歲.
我和合本的版本是四十歲.
原來原文沒有說到多少歲.
所以我們就猜測.
有可能是三十歲 有可能是四十歲.
我今年四十六歲.
應該我都做了皇很久.
在以色列作皇二年.
所以素羅登基年嘟嘟歲.
作以色列皇二年的時候.
就從以色列當中揀選了三千人.
二千跟隨素羅在麥沒和伯特利山.
一千跟從約拿丹在變雅問的基比亞.
其餘的人素羅就打發去國會國家去了.
約拿丹就走去攻打加巴的菲利士人的防營.
菲利士人就聽見了.
素羅就在遍地吹角.
意思就說我要令到希伯來人聽見.
以色列人眾人就聽見素羅攻擊菲利士人的防營.
又聽見以色列人為菲利士人所增護.
就跟隨素羅聚集在吉甲.
接著菲利士人就聚集和以色列人爭戰.

$^{241}$原來他們有車三萬兩 馬兵六千.
步兵海邊好像海邊的沙那麼多.
就上來去到伯雅文東邊的麥沒安營.
以色列的百姓見自己危急困逼.
就藏在山洞叢林石穴隱密處坑中.
有些希伯來人甚至過到約旦河非常害怕.
這句是我加的.
逃到迦特和基烈地.
素羅還在吉甲那裡.
百姓就更加害怕地跟隨他.
這是第一到第七節的經文.
如果你看這段經文.
你就會看到第二到第三節.
我們回到第二到第三節.
一開頭第二節經文就強調一千個字.
就說明約旦是有一千個士兵.
比起素羅兩千人的軍隊就少了一半.
理論上就應該是素羅的主力軍.
走去攻打菲利士人.
但三節卻叫人出乎意料.
竟然其實就是他的兒子約旦.
先去攻打菲利士人.
而另外在第三節當中強調素羅這個字.
就應該這樣翻譯.
第三節就說這個素羅.
就是這樣翻譯.
如果這樣翻譯的話.
你就知道他的語氣就差一點.
就說這個素羅只會遍地吹號.
不懂打仗.
我們就怎樣理解.
這樣跟素羅和約拿丹的對比呢.
當我們看回剛才那張powerpoint.
我們switch那張powerpoint.
我們看看薩姆爾記上第十章七到八節的經文.
其實第十章就是十三章之前.
薩姆爾對素羅的命令.
那命令我用紅色highlight.
那命令就是.
「這兆頭臨到你,就可以趁時移數.

$^{281}$因為神與你同在.
你當在我耳先下到吉甲.
我必指示你.
當見的繁祭和平安祭」.
特別就是紅色highlight那裡.
「趁時移數」四個字.
其實他的解釋.
就是薩姆爾期望素羅面對菲利士人的防兵的時候.
就可以做他手上「趁時移數」的事情.
就是作為一個君王.
以前君王原來是用來打仗的.
所以你做了君王一定要打仗的.
你成為一個君王就可以自由地去攻打菲利士人的這件事情.
但素羅我們回到剛才那段經文我們轉一轉.
你就會發現第十三章的二到三節.
他遲遲都沒有去攻打.
反而就是他的兒子首先去攻打菲利士人.
做了素羅本身應該要做的事情.
但是這個約拿登的攻打.
卻惹來菲利士人出動海邊的沙這麼多的軍隊.
來攻打以色列人.
聖經的作者其實是故意誇張描述以色列人的軍隊.
和菲利士人的軍隊的數目強烈的對比.
就帶出素羅面對菲利士人的軍隊的危機是非常之大.
無論是第五節說菲利士人軍隊的龐大.
第六節是說以色列人非常害怕.
非常害怕到藏在山洞石穴當中.
而素羅面對這個危機.
可謂進退兩難.
我解釋給大家聽.
一來他真的很不想去面對這個召命.
作為軍王的召命.
他想拿光環.
但是他不想承擔責任.
所以他不斷逃避上帝的呼召.
要做軍王應該要做的事情.
就是攻打菲利士人.
這是第一.
他不想的.
二來就是因為約拿丹這個敗家子.

$^{321}$他的行動就惹來菲利士人的攻打.
他就因此就不可以不面對這個危機.
可見耶和華就迫他去祥國.
就希望他不要放棄這個軍王的召命.
但是他一直走去逃避.
以至上帝迫使他去面對他生命的召命.
其實有時召命是要迫你去祥國去完成的.
我自己很怕綁佈道會.
我還記得我剛剛神學院畢業.
很久了十幾年前.
我在一間教會去做傳道.
當時我以前的同事.
亦即是我以前在學員傳道會這個機構當中做的同事.
就請我去到城市大學的學員傳道會.
有一場給大學生的報道會.
就請我做講員.
當時我就一口答應了.
答應了之後.
我就去到城市大學某一個lecture theater那裡.
就去到講這個報道會.
整個報道會都頗多籌備的.
特別為了這個報道會.
那班大學生就做了一件team shirt.
背後就是宣傳單張.
所以他遍地遊行的時候已經在宣傳.
就花了很多心機.
我開始講.
講的時候不斷地講.
我要介紹給大家聽.
其實我是印尼華僑.
你聽到我講廣東話其實不是很正.
因為我從小到大.
我媽媽是跟我講普通話和印尼話的.
所以我懂得聽印尼話.
所以印尼姐姐跟人講是非我是明白的.
但是我講廣東話不是很正.
我四歲才懂得講廣東話.
那時候我講報道會的時候.
有一粒字真的咬得不是很正.
一講出來就發音是一個粗口.

$^{361}$很尷尬.
然後全場大學生.
你知道大學生整個講道都不是很喜歡聽.
最喜歡聽那粒字.
一聽到那粒字就全場嘻哈大笑.
我真的很想找洞捐.
講完之後我就心想快點講完.
最後我就要呼叫.
總是一定有這個環節.
一呼叫就沒有人舉手.
沒有人舉手就快點走了.
我就想走去九龍塘地鐵站.
通過那又一城.
有一個大學生追出來.
追出來就說高全都.
我就停下來.
他就說要送一件T-shirt給我.
因為那T-shirt就是宣傳單張.
周圍走過的那個.
應該適合你的大碼.
然後他就說謝謝你今天來講道.
不用客氣.
他就跟我說.
你知不知道剛才有一個信主.
有一個信主.
他就跟我說不過那個人去到哪裡都信主.
然後我就說謝謝你.
那我先走了.
我就走了.
走了之後我整個人很崩潰.
崩潰在於你想清楚.
整個報道會那班大學生花了很多心機.
有單張有T-shirt.
有很多宣傳伎倆.
還有整個program有drama.
有話劇有詩歌.
花了很多心機.
我那時候就心想.
就是因為你那句粗口.
就搞砸了整個報道會.

$^{401}$然後那個大學生追出來就跟你說.
是帶了一個信主.
不過那個人是不是都會舉手.
那一刻我就自此發誓.
我以後不會再講報道會.
你叫我說培靈會讀經日讀經營.
甚至在建都神學院教書都沒有問題.
千萬不要叫我講報道會.
我去英國讀書回來.
回來的時候我在神學院教書.
都有人找我講報道會.
初頭我一兩次的邀請都拒絕.
不過第三第二次.
上帝就感動我.
就說你不是吧高明謙.
做傳道人不講報道會.
那也是沒理由不講.
所以我就接了一個報道會.
接了我又後悔.
那個報道會是一個幼稚園家長的報道會.
我就去講.
講完之後就只有一個家長缺字.
我都覺得Hallelujah.
自此之後那個幼稚園校長生氣了.
為什麼生氣呢.
他說我呼召的時候.
不呼召多兩次.
可能有多兩三個人信住.
他就這樣生氣了.
我就跟上帝說.
都說不要講了.
一講就出事.
接著又有一間教會找我講報道會.
接著上帝我都講不講好呢.
接著我又沒有一個理由去拒絕.
而我走去講.
講完之後那個傳道就走來跟我講了一句話.
那句話就說.
高博士下次都是請你講讀經日好了.
我自己心裡面是很不開心的.

$^{441}$很不開心.
因為我經常覺得我自己講得不好.
但是上帝就這樣一個電話來又一個電話來.
好像逼我埋牆腳.
叫我講.
最近做了牧師.
接著經常有個聲音跟我說.
牧師不傳福音你不是吧.
那種聲音那種鬱悶.
令到我都不知道怎樣去回應.
不過過往的2021年.
我相信大家都會感受到.
我們活在一個非常疲憊無力.
和沮喪和瘋癲的一年.
在這一年裡面.
坦白講我接了.
答應了六七個報道會這麼多.
每一個答應我都覺得會後悔.
但每一個答應我找不到理由去推.
我可以跟上帝講.
上帝我的照明都是解釋聖經.
報道會都不是解釋聖經.
都不是我的範疇.
但上帝都會逼你埋牆腳.
叫你去.
不過就算自己講得不是太好.
過往一年.
我都感受到人們容易信主了很多.
這個所謂容易信主不是時代容易.
大家明白嗎.
就是因為時代難.
人們才容易信.
時代難.
過往一年我見證了有150人信了主.
其實有一場報道會我很深刻.
一場報會是帶了44人信主.
在那裡我會感受到.
原來上帝是要用你.
不是在乎你真的講得多好.
是在乎你願不願意踏上.

$^{481}$跳進火坑那裡.
你真的承擔到那個照明.
而另外那些果效是上帝負責任的.
所以這個掃羅經歷.
很多時候都在反映我們的人生.
我們的人生會找一些理由去逃避.
因為那件事真的很難.
這個時代是很艱難.
面對前面好像摸著石頭過河.
因為我們眼前非常多非理士人.
你不要看旁邊那些不是非理士人.
你明白非理士人是叫人的.
非理士人.
而你的照明是要打他.
但是因為你害怕.
上帝就逼你去唱歌.
因為旁邊那個有若拿丹在.
他打了就逼你一定要打.
而整件事就是想我們知道.
原來上帝不是想作弄我們.
上帝是想讓我們能夠看到.
他想解決問題不是想創造問題.
讓你看到其實你可以的.
你能夠承擔到.
因為那個既然是來自上帝.
你要踏上那一步這個是很重要.
所以我們今年的年題叫outfall.
你要豁出去.
你要流出去.
而這個出去是那一步.
那一步因為你會看到這個時代.
真的很多人很無力.
很需要幫助.
而作為有信仰的我們.
我們每一個人.
有強而有力的耶穌基督的資源.
在我們內心裡應該都可以很強大地.
去幫助我們身邊那些需要幫助的人.
所以第一點我們就會留意.
就是拒絕不斷逃避.

$^{521}$我們去到下一張powerpoint.
就是13章第8到第12節的經文.
看完這張powerpoint.
我們就看回那段經文.
第8節就說蘇羅照著薩姆爾所定的日子.
等了七天.
然後薩姆爾還沒來到吉格.
百姓也離開了.
蘇羅走了.
蘇羅就說把梵濟和平安濟帶到我這裡來.
蘇羅就獻上梵濟.
剛獻完梵濟的時候.
薩姆爾就來了.
蘇羅就出去迎接他.
要問他好.
薩姆爾就說你做了什麼呢.
蘇羅就說因為我見百姓離開我散去.
你也不照著所定的日子來到.
而非利士人聚集在脈絡.
所以我心裡說.
恐怕我沒有禱告耶和華.
非利士人下到吉格攻擊我.
我就勉強獻上梵濟.
就是這一段的經文.
其實蘇羅也是很厲害.
他做事精明.
他很清楚需要做的事情.
他有他自己一套的計劃去面對非利士人.
他足足已經等了薩姆爾七天.
相信已經是他計劃當中一個的極限.
過了這個七天就立即去到私自獻祭.
當蘇羅完結了獻祭的那一刻.
薩姆爾就突然之間出現了.
整個描述用希伯來文看是很有趣的.
因為薩姆爾其實是用了兩個希伯來文字.
去問他你做了什麼.
而蘇羅就總共用了26個希伯來文字.
去解釋他的行動.
其實看牌面就好像帶出諸多解釋的意味.
起碼蘇羅有四大理由告訴他為什麼私自獻祭.

$^{561}$第一 他看見百姓離他而去.
第二 薩姆爾不在約定的時候來到.
第三 非利士人聚集在默默.
第四 他恐怕他沒有禱告耶和華.
四點 這個就足以看到蘇羅的侍奉.
其實是一種討好人的侍奉.
當百姓離開他.
自己本來等待薩姆爾的計劃就顯得不再有市場價值.
不受歡迎 但是為了面子的緣故.
為了自己的軍隊可以快一些有士氣.
就看到環境越來越惡劣.
就輕易地改變了本來的計劃.
去到受到百姓影響私自行事.
在這裡 蘇羅也同時放棄了自己一神的信仰.
就是他討好百姓 討好自己的面子.
就以這些成為他的必神.
我們看回那個power point.
那個power point 你會看到我們用紅色highlight的下半部分.
就是薩姆爾給他的命令.
這是第十章第八節.
他說 你當在我耳先下道吉格.
我也必下道那裡獻凡祭和平安祭.
首先我們看清楚.
第一點 你要等我七日.
這是第一點.
第二點 等我到了那裡 只是你當行的事情.
留意清楚 蘇羅有沒有等薩姆爾七日.
答案是有的 他真的等了薩姆爾七日.
第二點 蘇羅有沒有等到薩姆爾來呢.
是沒有的.
所以你就會發現第十章第八節.
他真的等了七天.
但是原來這個命令有一句叫做.
等我到了那裡 這一句蘇羅沒有做到.
因為他真的等了七天.
第二句就是等薩姆爾來 他是沒有做到.
如果是這樣 就是說薩姆爾對他的命令.
他就做一點就不做一點.
我們就叫這一種叫做selective obedience.
selective 即是選擇性.

$^{601}$其實這個蘇羅的選擇執法.
我們就會發現他在第十五章那裡也有.
他未滅盡亞馬利人的經文.
但是他滅一點不滅一點.
我們叫這個叫做選擇性執法.
選擇性執法是很邪惡的.
為什麼.
如果你完全不執法.
那你也來得坦蕩一點.
我摒棄耶和華的法例.
摒棄耶和華的命令.
我是一個大惡人.
我起碼都會condemn你是一個惡人.
我明白 完全.
但是我都會raise你.
是一個有integrity的人.
你承認你是惡人.
那ok 你抵落地獄了.
如果你要麼就完全執法.
完全執法就說.
我所有耶和華的命令我都執行.
那我就raise你.
真的很敬畏上帝.
很盡心盡性愛主你的神.
那你就上天堂.
那蘇羅是屬於哪個category.
蘇羅是屬於做一點不做一點.
然後就告訴別人他做了.
那就是告訴你這個人是叫做.
耶穌基督經常批判的那些叫做假冒為善.
內外不一.
那其實這種人都幾恐怖.
因為他會present到自己是很遵守耶和華的誡命.
但背地裡其實就是做一點不做一點.
那做一點不做一點其實都不是敬畏的.
都不是等於做.
其實這種人可能比那個完全不做更快落地獄都未定.
如果你又看得到這樣的判斷的話.
你就會想.
為什麼蘇羅又要做一點不做一點呢.

$^{641}$很有可能那個解釋.
就是他受到百姓的眼光去影響.
當百姓的眼光周遭的環境去影響到他本來的命令顯得不再有市場價值.
他那種市場導向的想法.
就驅使他成為他違反哪些指令是合乎市場價值.
又哪些指令不合乎市場價值.
他就selectively選取那件事去做.
我們就靠點擊率去為生.
我前陣子看了一本盧云的書.
大家知不知道誰是盧云.
盧云是天主教的神父.
很有名.
他有一段記載.
他提到心不由己的強制性的元素在我們的生命裡.
有一段說話他就這樣說.
他就說我是誰.
他這個說話是用我是誰做開始.
我是誰.
我是被讚賞被厭惡被痛恨被鄙視的一個.
無論我是鋼琴家還是生意人.
所在乎的就是人家怎麼看我.
如果忙碌是一件好事的話.
我就必須去到忙碌.
如果有錢是真自由的表現.
我就必須賺取更多的金錢.
如果郊遊廣闊是證明我的重要性.
這樣我就要交更多的朋友.
甚至是交一些重要的朋友.
來顯示我的魅力和重要性.
那種外間的觀感對我這個人的強制.
就潛藏在我們對內心裡的失敗的恐懼.
以至我們要避免失敗.
就不停驅拆自己.
囤積更多向外面對你的期望直奔.
所以你就要有更忙碌的工作.
更多的金錢.
更多的人際關係.
慢慢的那種強制.
就令你這個人失去自己.
你的人生就不停累積更多忙碌.

$^{681}$更多工作 更多朋友 更多又更多.
這就是心不由己的一種人生.
大家想不想這樣.
如果不想這樣.
我們知道上帝而來的召命.
是給我們釋放.
怎樣為之釋放.
既然耶穌基督已經釘在十字架上.
祂流出來的補血去洗淨我們的罪.
買熟了我們的生命.
而我們的生命的價值取決於有多少人.
有哪些人用多少錢買你.
就屬於你的價值.
如果耶穌基督用祂的補血.
重價買熟了你.
而這個是天下無價之寶.
我們又何苦自賣自己.
Art like 和那些心心.
我們因為那個寶貴的價值.
確立了我之後.
我就顯得自由.
自由於別人的期望.
自由於別人的看法.
我們的內心的糾結開始釋放.
因為我們知道我們的價值.
絕對不是靠這些東西去定義自己的價值.
所以一個忠於召命的人.
是確立自己我是誰.
這個問題.
是不容易受人影響的一個人.
專心討好上帝喜悅的人.
而不是自賣自己討好那些人的期望.
他是不會容易受人影響.
也不會人云亦云.
在工作和家庭諸多的困難當中.
以及自己的計劃當中.
你會仰望上帝而來的召命.
而那個召命其實不是想捆綁你.
而那個召命是想釋放你.
讓你活出一個自由的身份.

$^{721}$讓你不再沉在.
淹沒在別人的大洪流的期望當中.
而你慢慢做了一個倒模的人.
一個弱化的自己.
而這個自己沒有性格.
只是人云亦云的自己.
我們想這樣.
如果我們想這樣的時候.
我們就知道原來真正的人性.
不是來自這些.
真正的人性是來自上帝對你的確立.
這是第二點我們能夠看到.
第三點到下一章.
就是十三到第十五節.
我們轉一轉到經文那裡.
我們看看第十三節.
他在說什麼.
他就這樣說.
十三節.
「你做了糊塗事了.
沒有遵守耶和華理神所吩咐你的命令.
若遵守.
耶和華必在以色列當中建立你的王位直到永遠.
現在你的王位就不會長久.
耶和華已經尋著一個合他心意的人.
立他作百姓的君.
因為你沒有遵守耶和華所吩咐你的.
撒姆爾就起來.
從吉甲上到變雅敏的基比亞.
數羅數點跟隨他的人就只不過約有六百人」.
經文的十三到十五節是撒姆爾對數羅諸多解釋的回應和責備.
十三節一開始就責備數羅做了糊塗事.
愚民解造愚昧的事.
弟兄姊妹這個愚昧.
不是說你考試拿低分.
不是說你的IQ很低.
而是說你的道德有問題.
你IQ高 但道德有問題都屬於愚昧.
我們會看到這個愚昧事.
就跟遵守耶和華的命令那件智慧的事.

$^{761}$剛剛就是相反.
十四節當中就說明耶和華會為自己尋找一個合神心意的人.
合神心意 圓文直譯.
是緊貼神的心的人.
after his own heart.
這個人是跟神同心 有同一個目標.
跟愚昧事是一個相反的詞彙.
他是一種沒有自己計劃 只有上帝計劃的人.
不重視自己地位 是緊貼神心的人.
我們發現這段經文很有趣.
如果你重視君王的光環.
你正正就會失去君王的地位.
他就說這個王位要傳給另一個合神心意的人.
就是這個人.
我們看到這段經文 命令的字根出現四次.
命令本身是神由上而下的指令.
人只能做的 只不過是聽從.
而這個字的動詞其實是委任的意思.
十四節就說明耶和華已經委任一個合神心意的人.
正正就是這個字.
說明領袖和君王這個位置的確立.
是依賴由上而下的指令.
而不是個人如何控制人民的力量.
所以撒姆爾正正說明.
命令這個字的本身.
不再是死板的字句.
而是神確立信徒身份和召命的依據.
當一個人 包括素郎.
對他來說是很難忍的事 很難受的事.
因為人都是期望可以控制一切.
為自己的身份地位作很多的打算.
因為人總是想看到可見的地位.
而忽略了不能看見的內心.
你的安全感一定要建立在外在的因素.
而不是上帝的委任.
但神期望我們可以擁抱他的心.
因為 人是按外貌 耶和華是看內心.
我們想像一個情景.
你和我和其他人.
乘坐一架737的包機.

$^{801}$就要飛去另一個城市.
突然飛機引擎著火.
駕駛員衝出駕駛艙大叫.
我們要墮機了.
我們很快就要離開這裡.
這是命令.
幸好他知道降落傘放在哪裡.
因為沒有人知道 唯有他才知道.
將降落傘分派給所有的乘客.
稍微介紹降落傘的使用方法.
怎樣拉一拉 會彈出來.
怎樣穿 扣勁之類的方法之後.
每個人就排好隊.
打開緊急逃生門.
第一個乘客走到門邊.
對著大風大雨.
和駕駛員有一個請求.
乘客就問可不可以有一個請求.
駕駛員就說當然可以.
你有什麼請求.
我可不可以換一個粉紅色的降落傘.
駕駛員就不敢置信搖頭.
難道我給你一個降落傘還不夠嗎.
於是第一位乘客就跳下去了.
第二位乘客就走到門邊.
就和駕駛員說.
你有沒有辦法可以確保我跳下去的時候.
不會嘔吐呢.
駕駛員就說我沒有辦法.
但我只能夠保證每個人.
你們每一個都有降落傘可以跳下去.
然後一腳就伸了下去.
然後每個人都有每個人的請求.
但每個人的請求都滿足不了.
他們只能夠拿到降落傘.
有一個人就說機長拜託.
我有畏高症.
你可不可以除去我的恐懼.
你才讓我跳下去呢.
他就說我不能夠.

$^{841}$但我可以讓你降落傘.
另外一個人提出不同的策略.
他就說我們可不可以改變計劃呢.
讓我們和飛機一同墜毀.
或許我們有生存的機會.
駕駛員就笑著說.
你都不知道自己在說什麼.
然後一腳又伸了下去.
然後有一個乘客就說要有護眼鏡.
有一個人就說要有避震靴.
還有一個人就說要等到飛機稍微更貼近地面才跳下去.
全部的駕駛員都說你都不明白你們在做什麼.
然後一個一個全部就推他們下去.
每一個都有請求.
每一個的駕駛員都是重重複複.
都是說那句話.
我已經給你一個降落傘就夠了.
就你就可以跳下去了.
很艱難我們現在活在這個時代.
這個時代艱難的已經很複雜.
人與人之間又複雜.
工作又複雜.
教育又複雜.
什麼都複雜.
我們很想除去恐懼.
我們很想時勢會好一點.
我們面對日常生活已經夠艱難.
高牧師你還叫我說照命不是嗎.
但是這個時勢聖經已經說了.
我們沒有一個更好的時間.
面對末日這個時代是歪曲薄茂的時代.
時間那條的直線會一定向下走.
這個時代會越來越近末世.
你就知道會越來越亂世.
如果是這樣我們沒有可能.
沒有可能等到一個更好的日子.
沒有可能等到一個飛機更加貼近地面.
你才跳下去.
但是跳下去之前你只需要的.
就是一樣東西.

$^{881}$就是降落傘.
而你承擔照命只需要一樣東西.
就是根據撒姆爾的吩咐.
你那樣東西就是上帝而來的命令.
你抓緊那個命令.
掃羅就做到那個君王.
其實就做到那個君王.
只不過是selective.
如果是這樣我們就明白.
我們要放下那個按數區.
我們要勇於跳下去.
弟兄姊妹照命就是跳下去.
跳下去拔一拔就是這麼簡單.
那個降落傘assume一定work.
跳下去拔一拔.
弟兄姊妹是跳下去拔一拔.
無論那個飛機跌到多低都好.
跳下去拔一拔就可以了.
如果是這樣我們就會知道.
我們很多時候認為照命.
是要等到好時機才承擔.
原來在舊約裡面.
很多的照命都是最壞的時代而領受.
因為你要fulfill一個使命.
而使命一定要有一個問題的假設.
而那個問題越黑暗.
那個照命就越真實.
類似這樣我們就會看到.
這段經文給我們就是照命人生.
承擔照命的重要性.
所以我們去到最後一章的powerpoint.
我們總結三點和大家一起分享.
第一點.
很多時候上帝給你照命.
你要承擔那個責任.
他會逼你埋牆角.
但他不是想製造更多的問題.
他反而想解決問題.
讓你找到終極人生的意義.
這是第一點.

$^{921}$所以我們千萬不要再逃避.
第二點.
我們不要只為自己去行事.
那些百姓的眼光.
那些失去支持的市場價值.
而慢慢我們的人生就沒有了.
我是誰的重心和焦點.
而我是誰就是取決於上帝而來的照命.
去確立你自己.
這是第二點.
第三點.
就是我們不要做糊塗事.
因為我們知道我們所靠的.
就是唯一的降落傘.
無論那個降落傘是什麼顏色都好.
當然我不是說粉紅色是犯罪.
但最重要的是那個真的是降落傘就行了.
我們就會知道.
我們拒絕那個糊塗事.
單純地跳下去.
蚊一蚊就搞定了.
這是一個照命的人生.
我們祈求主的話語成為我們明光指引.
我們一起祈禱.
多謝天父.
讓你的話語繼續成為我們的鼓勵和提醒.
幫助我們虛心和溫柔地.
用信心回應主的道.
禱告是奉耶穌基督的名頭.
阿門.
\newpage



\section{}
\label{sec:vVSfVrNtuKk}
\textbf{【網上崇拜】虎年生肖道|20220205 [vVSfVrNtuKk]}
\newline
\newline
連結: \href{https://youtube.com/watch?v=vVSfVrNtuKk}{\texttt{ https://youtube.com/watch?v=vVSfVrNtuKk}} ~~~~ 語音日期: 2022-02-05 
\newline
\newline
\hyperref[sec:FLZJqJSyEdc]{\small{< < < PREV SERMON < < <}}
~
\hyperref[sec:index_chronic]{\small{[返順時目]}}
~
\hyperref[sec:index_scriptual]{\small{[返順卷目]}}
~
\hyperref[sec:wntIcXZCGmo]{\small{> > > NEXT SERMON > > >}}
\newline
\newline
$^{1}$鼎姐妹平安.
今天是大年初五.
恭祝留堂的鼎姐妹在苦靈裡身體健康.
心靈滿足.
願主的愛和保護常常與你們同在.
已經是不知不覺到了第三年時間.
Full Church的新昭禱已經是第三年了.
記得第一年是選年的大年初一.
記得當時把講道錄下來.
在Facebook直播.
之後我們面對一個漫長的疫情.
去年也是.
去年在牛年的時候.
也是要網上直播.
所以每逢到新昭禱都是網上直播.
可能這也是好事.
12年後我們可以重播.
不用再遇到次郎.
我自己是在2015年開始講新昭禱.
記得當時是羊年.
當時純粹是好玩.
試試寫新昭禱.
覺得反應還可以.
下年之後 後年再寫一次.
然後很快就雞 狗 豬 鼠 牛 虎.
寫到今年.
如果再寫四年的話.
再說套龍蛇馬.
我就集齊12隻.
有獎品拿的.
不過今年苦年是有難度的.
因為聖經是沒有老虎.
如果你在聖經上搜尋的話.
你會發現.
其實勉強都找到一些.
逼苦或者苦口的字.
下一張 麻煩你.
聖經上沒有老虎.
但如果你去看回這個字的時候.
是有逼苦和苦口的兩個字.

$^{41}$麻煩你給我展示一下.
聖經上沒有老虎.
為什麼呢.
因為原來在那邊附近.
是沒有老虎棲息的地區.
如果你看回全球老虎分布圖的時候.
下一張.
你會發現.
虎其實是一個很獨特的生物.
牠多數在印度.
在中國.
在東南亞地方.
是以牠的生活為主的.
不過現在也有不少老虎是絕種的.
剩下6款的物種.
幾乎是瀕臨絕種.
根據統計.
20世紀初的時候.
大概也有10萬隻老虎.
但現在只剩下3900隻老虎.
非常危險.
非常容易絕種.
老虎是一個非常強大的動物.
雖然小時候玩鬥獸棋.
獅子比老虎大.
不過老虎的肌肉比獅子多.
你會看到.
老虎的肌肉大約有58.8\%.
但老虎有72.6\%.
是比較強的.
你看到牠的手臂也比獅子強很多.
所以如果單對單的話.
老虎應該會贏得獅子.
所以小時候聽到的武松打虎.
其實應該不是真的.
根據學者的研究.
在歷史中真的有武松的人.
武松不是真的打死老虎.
而是曾經打死一個官二代.
叫做蔡虎的官二代.

$^{81}$所以叫做武松打虎.
不是打死老虎.
不過很有趣.
中國人是將老虎放在我們手中.
大家可能也知道.
如果你將手指的拇指和食指打開.
這個叫做虎口位.
大家可以試一下.
你可以試一下將手指打開.
這個插位就叫做虎口.
但大家知不知道為什麼虎口這個字.
我也不知道.
根據一些書.
虎口這個字其實是來自醫書.
千金要坊的醫書所記載的.
張眼請看.
原來根據千金要坊.
作者叫做孫仙妙.
孫仙妙是一個唐朝非常出名的神醫.
人稱為藥王.
根據幾個圖也是.
圖中很多都是和老虎有關係.
這個孫仙妙.
話說孫仙妙在河溪裡採藥.
在山下路過一個丫頭.
兩邊都是一個很窄的山路.
突然間看到一個非常大隻老虎.
張開大眼睛看著他.
即是說攔老虎就是這個意思.
你問孫仙妙怕不怕老虎.
她當然怕.
但你見到老虎突然間好像不舒服.
好像發燒.
好像會咳嗽.
不知道是不是中了燒.
她問老虎如果你要吃我.
就試試搖三下尾巴.
如果你要看病就試試點三下頭.
老虎就立即點三下頭.
孫仙妙知道老虎是要醫病的.

$^{121}$幫他打了一頓脈.
然後捉了他下體溫.
發覺真的有些病.
當她見到他口裡有些血漬的時候.
她發覺原來是.
應該是口裡有些事情.
然後她張開老虎的大口.
查一查發現原來有這麼長的金參在口裡.
原來他剛剛吃完一個女人.
金參插在他口裡.
孫仙妙就想著.
你竟然吃人.
我就立即不幫你醫病.
怎料老虎立即痛苦.
就立即抱著他不放.
孫仙妙就問他.
如果你不吃我就快點走.
就點三下頭.
如果你從此悔過的話就搖三尾巴.
老虎就搖三尾巴認罪悔過.
然後孫仙妙就立即幫他拔出一支金參.
給他吃一顆藥丸.
立即痊癒.
隔離濕透之後可以痊癒.
所以孫仙妙的圖畫經常有老虎在旁邊.
所以孫仙妙就在千金異方裡.
就把合谷穴稱為虎口.
大家可以試一下.
把合谷穴打開手指.
把手指放在這個位置.
試一下按一按.
如果把手合起來.
這個直線的位置就是合谷穴.
聽聞是一個非常好的穴位.
如果你Google的話.
Google告訴你.
只要按這個穴位.
就能夠治療頭痛 頭暈 心痛 牙痛.
眼睛痠痛 喉嚨痛 手背痛.
可以舒緩經痛 腹痛 扁桃腺炎.

$^{161}$失眠 導汗和悶熱.
可以通血路 降血壓 防中風.
舒緩過敏 鼻竇出血.
敏感 打噴嚏 流鼻水 耳鳴 耳痛.
便秘和防止痴呆.
非常好的一個穴位.
如果你試一下搜尋YouTube的話.
試過有個YouTube叫做「虎口良藥」.
就是講這個穴位的.
所以今天我們會講這個虎口.
這個開始講起.
不要給水降溫.
好 那我們開始講這個虎口.
在心理中出現過幾次虎口這個字.
其中一個很特別的意思.
就是這個二賽學書第四十章十二節.
我們稱之為一個經文.
我們一起讀一下下一章經文.
預備 1 2 3.
用聲斗勝大地的神父.
用聖情山領.
用意平平光明歸.
這個經文講什麼呢.
這個經文是第二二賽亞書的經文.
二賽亞書第四十章到五十五章.
我們稱之為第二二賽亞書.
和第一二賽亞書不同的是.
第二二賽亞書是一個非常相距150年的經文.
是一個以色列人被擄之後的經文信息.
所以這二賽亞書不再預言以色列人被擄的信息.
而是針對他們被擄之後.
以色列人被盼望的信息.
第二二賽亞書所講的是上帝對世界的主權.
以色列人的救贖.
錫安猶實蘭的根深.
所以二賽亞書第四十章作為整個第二二賽亞書的序言.
是講上帝對世界的主權.
可能你不知道你是多熟悉二賽亞書第四十章.
就算你背不背整章.
其實有不少經文不是陌生的.

$^{201}$我試下一章.
譬如說二賽亞書第四十章三到四節.
有人聲喊著說.
在曠野預備耶華的路.
在沙漠地修平我們上帝的道.
這是後來釋述的經文.
第一段就是二賽亞書第四十章七到八節.
草必枯乾花必凋謝.
因為耶和華的氣吹在旗上.
百姓成賢是草.
草必枯乾花必凋謝.
唯有我們上帝的話必永遠立定.
這是比得前書的經文.
最後一段大家更熟悉的就是.
疲乏的他賜能力.
愈弱的他加力量.
但等候耶和華的必重新得力.
他們必如英輾刺上藤.
他們奔跑卻不困倦.
行走卻不疲乏.
沒錯 原來這些經文都是來自二賽亞書第四十章的經文.
但今天我看過 可能大家不太熟悉.
就是第十二章 二節裡面.
他說誰曾用手心量諸水.
用虎口量蒼天.
用星斗盛大地的塵土.
用青山寧.
用天平平綱領.
就讓我們一起看看今天這段經文的意思.
很明顯你不需要任何聖經很深厚的根底.
大概明白這段經文的意思.
二賽亞書第四十二節.
講起一個五句的反問句的經文.
誰曾用手心量諸水.
用虎口量蒼天.
用星斗盛大地的塵土.
用青山寧.
用天平平綱領.
你不難理解這五句的反問句.
其實表達同一個的意思.

$^{241}$很明顯這五句的反問都有共同之處.
他用的是人類平時所用的器皿或者量度工具.
錯配在世界龐然的大自然裡面.
用手心量諸水是什麼意思.
我們平時用手心來災水洗面.
但如果你用你的手心來量度世上一切的水.
根本是沒有可能的.
用虎口量蒼天.
平時你用虎口量度單位.
虎口有多少.
桌子有多闊.
沙發有多長.
木塊有多高.
但如果你用自己的虎口量度蒼天的距離.
根本是沒有可能的事情.
同樣道理用星斗盛大地的塵土.
用青山寧.
用天平平綱領.
都是同樣的意思.
器具或者日常用品.
都是我們可以用來量度世界裡的事情.
但當你要量度整個大自然的時候.
你會發覺這些器皿立即顯得微不足道.
不知道你小時候去海灘玩有沒有這樣的想法.
當你去海灘玩的時候.
拿著水桶去舀水的時候.
有沒有想過如果夠勤力的話.
我可不可以舀完海灘裡面的水呢.
小時候有沒有想過.
有吧.
理論上是有可能的.
但實際上是沒有可能的.
是可以的但實際上是沒有可能的事情.
因此這就是經文的意思.
我們或者可以用手心來裝水.
但根本無法用手心來承載上面所有的海洋的水.
我們或者可以用我們的虎口來量度距離.
但你無法用你的虎口去量度蒼天的距離.
很明顯以塞亞書第四十章第十二節.
這幾句的反文句.

$^{281}$就是要表達人類的智慧是有限.
強調人類是渺小的.
正如後面經文所說.
萬民都像水桶的一滴.
就算如天平上的微塵.
它舉起眾海島好像極微之物.
塞亞書第四十章要強調的.
正正就是人類作為一個渺小的創造.
受造和他的有限.
我發現了一句很有意思的說話.
叫Little science take you away from God.
But more science take you to God.
一丁點的科學或許會叫你遠離上帝.
但只要你明白更多的科學的時候.
科學就會將你帶到上帝的面前.
我想這正正就是經文所說的意思.
不過我發覺人類是剛剛相反的.
這正正就是人類的通病.
人類以為懂得某程度上量度這個世界.
他就能量度世界的全部.
以為自己能夠認識大自然.
就能夠掌管大自然.
以為自己能夠用時針秒針量度時間.
就能夠控制操控時間.
以為自己能夠用顯微鏡發現病毒.
就可以完全控制病毒.
誰曾用手機追蹤病毒.
用體溫機控制疫情.
用兩個口罩保持健康.
用人道處理來杜絕病毒.
用風流強減來動態清零.
大家不要誤會我不是反智的覺得口罩沒用.
洗手沒用 謹慎不必要.
這都是人類要做的智慧和責任.
但我們要很嚴肅的問.
我們是不是好像這個所說.
當我們打算去清零的時候.
我們其實在用我們手中的虎口量度這個豬天.
去量度這個大自然.
可能你心裡面這樣想.

$^{321}$現在說到這個人是不是語意共存的論者.
是不是要和社會的政策對抗.
我回答不是.
我們要問的是一個很嚴肅的問題.
有關誰的問題.
《熱唱間書》第四章十二節.
這五句的反文句.
正正叫我們反思誰.
誰這個問題.
誰曾用手心量豬水.
用虎口量豬天 蒼天.
這個問題第一個答案是什麼.
Nobody 無人能及.
世上沒有人能夠用他的手心量度所有的水.
用他的虎口量度蒼天距離.
更重要的是這五個反文.
同時間告訴大家.
特別在這個熱唱間書讀給大家聽.
第二個答案.
唯有耶和華上帝是可以的.
唯有上帝能夠用他的手心量度豬水.
唯有耶和華上帝能夠用他的虎口量度蒼天.
世上真正配得有能力量度這個世界的.
只有創造這個世界的上主.
如果你再看後少少經文.
第40章22-23節的時候.
上帝坐在地球大圈之上.
地上的居民好像蝗蟲.
他鋪張窮倉如萬紙.
展開豬天如可望的帳棚.
他使君王歸於無有.
使地上的審判官成為虛空.
這五句的反文句告訴我們.
量度這一切的不是世上任何的人.
唯有耶和華上帝自己才是量度的主人.
很有趣的.
以創下書第42節所用的五個動詞.
量度,衡量,盛載,稱和評.
這五個有關量度的動詞.
其實在整本的舊裡.

$^{361}$都是用來形容耶和華上帝的作為.
「誰曾用手心量度」.
即是measure這個字.
用虎口去量蒼天.
衡量estimate.
用星斗去盛大地的塵土.
盛再解作擔當或扶持.
用稱去稱山嶺.
用天平去平綱嶺.
balance.
你發現這五個動詞.
在聖經裡不單止是形容耶和華上帝.
更加是上帝在人身上的動詞.
後面會發現.
全部都是有關上帝在人身上的作為.
下一章.
針研12節到第6節.
人看自己一切所行的都是清晰的.
耶和華卻衡量人心.
上帝是measure人的心.
前面是第五晚篇.
你要把你的重擔射給耶和華.
他必扶持你這個扶持.
正正就是contain擔當的意思.
前面是第五晚篇.
你們是心中作惡.
你們在地上稱出你們所行的強暴.
稱字都是這個字.
針研.
古者的天平為耶和華所珍護.
公平的法馬為他所起義.
公平的稱和天平都屬於耶和華.
這些全部都是有關上帝用他自己的量度.
去量度人的行為.
所以經文告訴我們一個很重要的訊息.
耶和華上帝才是世上一切的量度者.
上帝不單止用他的手心量諸水.
用他的虎口量蒼天.
更加用他的手量度人心.
稱出人所手足出來的行為.

$^{401}$珍護鬼詐的天平.
用公正的稱量度萬事.
上帝是一切的尺度.
大家想想上帝就像萬能的簡寫一樣.
你可以用他量度蒼天的距離.
量度過去的歷史.
量度這個社會的現象.
量度你和別人的距離.
量度你自己的行為.
所以各位弟兄姊妹.
這正正就是以下書經文所說的意思.
世上沒有人能夠完完全全量度這個世界.
唯有耶和華上帝才是一切的尺度.
我們不能搞錯.
雖然人類好像懂得用自己的方法量度世界.
但世人卻同時是耶和華量度的對象.
我們是量度者.
同時也是被量度者.
你不能搞錯.
你搞錯就很大件事.
可能你不知道.
在美國華盛頓.
有一個地方叫做National Institute of Standards of Technology.
這是什麼地方呢.
這個地方是專門收藏人類量度的單位.
巴黎有一個.
美國有一個.
究竟一米有多長.
一磅有多重.
一公升有多少.
這個地方很好地將人類的單位.
存放一個標準.
譬如說一米有多長.
就有一米的鐵牆.
去archive一米有多少.
這個叫做prototype.
一米的prototype就放在這裡.
是一個很有趣的地方.
你想想整個地方都放滿了人類量度單位.
一米一磅一厘米.

$^{441}$你問為什麼要這樣做.
我也不知道.
我想大概就是人類文明一毀滅的話.
大家就找回這個地方.
一米有多長.
請接會一個真正的尺度.
非常重要.
我覺得很多年前在長洲讀書的時候.
那時候我只有24歲.
那時候我都有跑步習慣.
每次跑步都會跑圈.
跑球場跑圈.
那時候沒有apple watch.
每次都跑一個距離.
跑12個圈.
我剛剛搬到長洲住.
第一次去到長洲的球場跑步的時候.
跑了12圈.
好像還是很足夠.
跑得很快.
長洲地靈人傑是不是特別好.
不累.
他們發現原來為什麼.
山頂道球場不夠400米.
只有300米.
所以長洲有很多東西是不符合標準.
建道的籃球場高度是不符合標準.
矮很多.
長洲泳池也沒有50米.
球場也不足500米.
麥當勞是沒有奶可以喝的.
全部都是不符合標準的地方.
所以一個正確的尺度是非常重要.
請接會你今天用什麼來量度這個世界.
是我們今天要問的問題.
或者當這個政權用自己的苦口量度這個世界.
你又怎麼重新量度你自己.
和這個世界和上帝之間的關係呢.
這個2022年的苦年.
我們怎麼能夠去過活.

$^{481}$我們怎麼能量度未來這一年呢.
記得除夕的時候.
我祝願呼出弟兄姊妹.
祝願大家能夠從谷底爬起.
能夠重新去過這個新的一年.
後來有人問.
你怎麼知道2022年已經是谷底.
怎麼知道去年是谷底.
可能低處不算低呢.
或者他是對的.
不過我們要去學習的就是.
我們不要用過往的經驗去量度2022年.
我們學習用上帝來量度這一年.
我們這班人仍然留在香港生活的人.
或者是一班明知山有虎.
偏向虎山行的人.
我這麼說是一班明知山有虎.
卻仍然戴著口罩在虎山生活的人.
弟兄姊妹.
2022年你怎麼去學習.
在苦口裡面的邊緣去生活.
在這個苦口裡面.
你親眼目睹這個社會越來越每況愈下.
不斷的清零.
強行檢疫.
停學 停市 人人自危.
大部分從恐懼病毒.
變成恐懼被人捉去檢疫.
求神拜佛.
叫自己不要和身邊的人.
和這個疫情政治牽上關係.
害怕死所處的大廈中招.
害怕被人封鎖.
害怕被人送去一個不願意去的地方.
請問在這個危險處處.
寒蟬效應的苦口裡面.
我們怎麼去量度這個世界.
你用什麼來量度這個世界.
當你這隻老虎認為.
他可以用他自己的苦口去控制這個世界.

$^{521}$去清扒這個世界的時候.
你在這個苦口裡面生活.
你怎麼可以生活去止損呢.
好好搞清楚.
你應該用上帝去量度這個世界.
當你搞清楚.
唯有耶和華上帝.
才配得去量度這個世界的時候.
你才能夠正確去判斷這個世界.
疫情已經達到第三年了.
你學到什麼.
如果這個疫情裡面.
你學到的大概都是.
買哪隻口罩比較好用.
哪裡買到哪隻口罩.
哪個區比較安全的時候.
我想你完全錯誤地量度這個疫情.
可能你浪費了這兩年時間.
就像兩年前.
一開始那邊.
疫情聖經歷史的神話反思那邊.
我之前也說過.
疫情是上帝給人類一個反省的時候.
疫情叫我們重新量度.
我們自己和自己.
自己和上帝.
自己和世界的關係.
今日誰曾用手機追蹤病毒.
用體溫器去控制疫情.
用兩個口罩保持健康.
用人道處理來杜絕病毒.
用封鎖和強檢來清零.
請學習用上帝量度你身邊的一切.
量度這次的疫情.
量度我們的生活.
量度我們的未來.
尤其是當我們身處在苦口裡面.
就更加要認清.
哪一個才是真正的苦口.
哪一個才是真正的大老虎.

$^{561}$哪一個只不過是一隻紙老虎.
當你搞清楚哪一個才是真正的苦口.
哪一個才能配得上苦口量度蒼天距離的時候.
你就能夠明白病毒的意義.
當你搞清楚哪一個才是真正的苦口.
你就會發現這個張牙舞爪的紙老虎.
其實都是微不足道的.
這是我自己近來很深刻的一句話.
只要你足夠的去敬畏上帝.
無論你做什麼你都不會走錯.
你就不會害怕.
你就不會恐懼.
我再強調.
我不是相信與疫共存論.
我相信的是與疫共存論背後的上帝和我們共存.
我們用祂的赤道 用祂的法則.
用祂的苦口去理解這個世界.
祝願福出頂之妹在苦年裡懷著敬畏去度日.
無畏無懼.
明知生老虎仍然懷著敬畏去侍奉葉和華.
我們一起祈禱.
我們一起在苦年裡懇求你.
讓我們在苦年裡.
首先能夠知道再次認清.
只要我們去投靠你的時候.
只要我們知道你是掌管這個世界的時候.
我們在苦口當中都無畏無懼.
我們能夠再次重新去思考.
你是怎樣來掌管世界.
讓我們能夠在這個疫情很猖狂的時候.
能夠知道應該做些什麼事情.
能夠知道這些疫情背後.
你對我們世人的心意.
你叫我們認罪悔改.
讓我們謙卑.
放下我們手上的工具.
再次去認清你是掌權的上帝.
讓我們能夠投靠你.
求你幫助我們.
奉主的名求 阿門.

\newpage



\section{約翰福音 21:18-19-20220212}
\label{sec:wntIcXZCGmo}
\textbf{【網上崇拜】再有一次機會|約翰福音21\_18-19|20220212 [wntIcXZCGmo]}
\newline
\newline
連結: \href{https://youtube.com/watch?v=wntIcXZCGmo}{\texttt{ https://youtube.com/watch?v=wntIcXZCGmo}} ~~~~ 語音日期: 2022-02-12 
\newline
\newline
\hyperref[sec:vVSfVrNtuKk]{\small{< < < PREV SERMON < < <}}
~
\hyperref[sec:index_chronic]{\small{[返順時目]}}
~
\hyperref[sec:index_scriptual]{\small{[返順卷目]}}
~
\hyperref[sec:GPCXXRzSHkw]{\small{> > > NEXT SERMON > > >}}
\newline
\newline
約翰福音 21:18-19-20220212
\newline
\begin{longtable}{cl}
\hline
\hline
章節 & 經文 (和合本修訂版)\\
\hline
21:18 & \begin{tabularx}{0.7\textwidth}{X} 我實實在在地告訴你,你年輕的時候,自己束上帶子,隨意往來;但年老的時候,你要伸出手來,別人要把你束上,帶你到不願意去的地方。」 \end{tabularx} \\ \\ \relax
21:19 & \begin{tabularx}{0.7\textwidth}{X} 耶穌說這話,是指彼得會怎樣死來榮耀神。說了這話,耶穌對他說:「你跟從我吧!」 \end{tabularx} \\ \\ \relax
21:20 & \begin{tabularx}{0.7\textwidth}{X} 彼得轉過身來,看見耶穌所愛的那門徒跟著,就是在晚餐時靠著耶穌胸膛說「主啊,出賣你的是誰」的那門徒。 \end{tabularx} \\ \\ \relax
21:21 & \begin{tabularx}{0.7\textwidth}{X} 彼得看見他,就問耶穌:「主啊,這個人怎樣呢?」 \end{tabularx} \\ \\ \relax
21:22 & \begin{tabularx}{0.7\textwidth}{X} 耶穌對他說:「假如我要他等到我來的時候還在,跟你有甚麼關係呢?你跟從我吧!」 \end{tabularx} \\ \\ \relax
21:23 & \begin{tabularx}{0.7\textwidth}{X} 於是這話在弟兄中間流傳,說那門徒不死。其實,耶穌不是說他不死,而是對彼得說:「假如我要他等到我來的時候還在,跟你有甚麼關係呢?」 \end{tabularx} \\ \\ \relax
21:24 & \begin{tabularx}{0.7\textwidth}{X} 這門徒就是為這些事作見證、並且記載這些事的,我們知道他的見證是真的。 \end{tabularx} \\ \\ \relax
21:25 & \begin{tabularx}{0.7\textwidth}{X} 耶穌所行的事還有許多,若是一一都寫出來,我想,就是全世界也容不下所要寫的書。 \end{tabularx} \\ \\
[1ex]
\hline
\hline
\end{longtable}
$^{1}$看到觀眾留言說很喜歡背景的紫紅色.
是的,紫紅色屬於GSO.
我相信今天對我們來說.
能夠一切敬拜的那種喜悅是很重要的.
我在想穿什麼衣服的時候.
我發覺原來我家沒有麻質的衣服.
其實今天也是一個披麻的日子.
因為在香港的崇拜.
教會的聚會被壓縮和扭曲到一個點.
其實是要披麻蒙灰去哀求上主.
今天最難過的情況就是.
沒有限聚或者沒有限人數的聚會.
只是喪禮才是沒有限人數.
其實我們每個禮拜都經歷過喪禮.
就是主耶穌的死和復活.
所以敬拜對我們來說是我們最珍而重之.
我們最需要結集的地方.
我自己預備開場白的時候.
其實是想扣連講題的.
就是再有一次機會.
但是沒想到今天的講題.
開場白真的很實際地經歷.
能夠再有一次崇拜的機會.
或者再有一次上網的機會.
弟兄姊妹和我們一起守候.
在討告當中可以做的事情真的不多.
就是討告之後突然間就出了畫面.
對我們來說一樣是很神奇的.
神很奇妙地接通了這件事情.
如果你覺得我們過去那三年做網上技術.
都有一定能力的話.
可想而知很多教會明天上午崇拜是很困苦的.
因為不是每個人都有這樣的能力.
或者經驗.
遇到解難的情況可以處理.
我看到的就是讓弟兄姊妹在當中可以鬆一口氣.
可以一起聽道.
其實也不是必然的.
交代一點心情的時候.
其實我自己寫這篇講章寫得很慢.

$^{41}$改過很多次.
因為定調和用什麼語態講這兩節經文.
其實都是我自己很不容易過的日子.
我很少講這麼短的經文.
我只選了《若人方圈》21章18和19節這兩節經文.
對我來說其實也是我自己很重要的.
這段日子的心路歷程.
因為剛才也有人問我這兩天這個星期很難過.
其實我不是這個星期很難過.
是這幾個月都很難過.
這兩節經文正正就是我不斷地跟自己說.
和再一次要經歷這兩段經文對我的提醒是什麼.
所以在這個月題Outflow的時候.
我就選了這段經文成為第二次講Outflow的內容.
對於再有一次機會也是我自己一個口頭禪.
因為我相信人生.
你的命沒有take two.
但是有很多機遇你可以take two.
錯了不要緊.
錯了有第二次再更新的機會.
或者再做的機會我們就盡力吧.
《若人方圈》第21章的開頭.
是來自這個地方就是提比利亞海.
提比利亞海邊其實也是主耶穌在過去.
被記載一些比較重要的與人相遇的地方.
特別對於門徒來說是他們被呼召的地方.
是他們經歷耶穌第一生的邀請的地方.
是他和耶穌有很多關於魚.
關於他們自己相遇過程當中的地方.
在這幾個月其實都見了很多將要離開的.
離開香港的弟兄姊妹.
其實再有一次機會這句話.
其實都是和相遇的弟兄姊妹聽得最多的.
能夠再有一次機會去當初拍拖的地方.
能夠再有一次機會去當初認識上帝的地方.
能夠再有一次機會回到當初決志的營地的地方.
其實提比利亞海對於門徒來說是很多回憶的地方.
很多和耶穌相遇和經歷的地方.
《約翰福音》有一章記載的內容就去到.
去到一個位置是什麼呢.

$^{81}$就是耶穌已經過去了.
耶穌已經不是在我們當中了.
我們回到打魚.
我們再回到我們以前的地方.
我們從粗故業.
對於這個地方來說是很深刻還是很難忘的.
在過去這幾個月就是弟弟姊妹.
能夠在香港再回到曾經相遇的地方.
其實很多回憶都是很難過的.
在當中彼得在第21章內的內容經歷了什麼呢.
你會看到他經歷的是.
有一章記載他再從粗故業去打魚.
但是在網魚之後.
他再一次經歷的就是.
上帝透過魚叫得人如魚.
這個應許再一次去呼召他.
他見到很多魚的時候.
他再想起很多關於打魚.
關於上帝叫得人如魚的那種應許和邀請.
另外在這個環境當中他會看到什麼呢.
他會看到上帝叫他再去預備早餐給他的時候.
見到五個餅兩條魚的時候.
他回想起很多上帝給他恩典的空間.
恩典的回憶.
他會經歷到很多.
他見到上帝做了很多amazing的工作的時候.
他知道他沒有選錯.
就好像安德烈和他哥哥說.
哥哥我們遇見尼塞亞.
他就馬上去找耶穌.
去到耶穌面前.
他給的是說得最快.
說得最準的答案就是.
主啊 你有永生之道.
我們還要歸從誰呢.
對於彼得尼說.
與和耶穌相遇的地方.
經歷耶穌神蹟的地方.
今天再回到提比利亞海的時候.
是很大的觸動.

$^{121}$去到第三件事就是.
你會看到那個過程就是.
去到一個火的時候.
見到那個火對於是什麼呢.
見到那個火把的時候.
三次不認上帝.
三次不認耶穌.
見到那個火爐邊的時候.
他再一次去懊悔.
其實為什麼我沒有急使出來認呢.
一個婦人把我推出來的時候.
為什麼我要選擇逃走呢.
再去到這個環境.
再去到這個處境的時候.
其實他有很多事情要想.
我想大家現在受疫情之下.
可能很多空間都沒有再去.
但是如果你真的要去的時候.
你走過很多地方.
你會有很多回想.
你有很多不同的經歷.
見到岸邊的火.
其實就是再一次提醒他.
在大祭宣完終的時候.
他否認上帝那種羞愧.
去到現在.
經文在十八節之前的時候.
就是主耶穌三次去問.
約翰的兒子.
你愛我嗎.
在過去你查經也好.
聽到也好.
你都不能去接收那個信息.
就是他當日三次不認耶穌的時候.
耶穌發出三次邀請去挽回他.
是看上去很工整的.
但是我自己覺得.
也許耶穌不是這樣做.
三對三去做一個拼湊.
可能重要的事情就說三次.

$^{161}$是很認真地問.
你看到彼得一而再.
再而三地去回答耶穌.
他知道耶穌知道.
但是他再一次去認定.
主啊你知道我是愛你的.
其實你沒有被人問過呢.
一而再再而三被人問.
我自己很多事情.
說了就是了.
因為我回答的答案.
或者我回應的過程當中.
我都會想過我才會說出來.
所以就算問我多少次都好.
我都會回答同一個答案.
不知道彼得經歷過這些日子.
這些回想的時候.
他再一次回答耶穌的時候.
你愛我嗎.
他經歷了這麼多.
耶穌再一次邀請他.
挽回他的時候.
他表達他愛耶穌.
其實什麼是愛耶穌呢.
弟兄姊妹在這些日子.
你有沒有想過什麼是愛耶穌呢.
在星期四的時候.
我們開了一個祈禱會.
在第六個禱告事項的時候.
就是為自己在疫情之下的禱告.
你自己在疫情之下的禱告.
疫情教會你什麼呢.
其實教會你更加愛上帝.
還是教會你更加不愛上帝.
還是不愛上帝.
楊福音第21章第18節這麼說.
我實實在在地告訴你.
你年少的時候自己束上帶子.
隨意往來.
但年老的時候.

$^{201}$你要伸出手來.
別人要把你束上帶你到不願意去的地方.
耶穌在說一個預言.
這個預言從彼得的過去展望到將來.
你過去可能真的可以做什麼都可以.
你過去可能真的可以.
去按你的心意去做你自己想做的事情.
但是你年老的時候就不是這樣了.
這段經文在我年輕的時候.
做侍奉人員.
或者想想自己可不可以侍奉上帝的時候.
我也覺得是一個呼召經文.
我初初夢照的時候.
也覺得自己放棄了一份比較穩定的工作.
收入和待遇都很好.
我自己的環境當中.
都是一個取捨之下去改變.
我小時候出來工作的時候.
我可以隨處按心意去做自己想喜悅的事情.
但是做了侍奉人員或者做了傳道人的時候.
就不是這麼容易了.
那時候也會有一點掙扎.
但是其實近這三年.
自從在flowchart一直開始服侍,託方.
和開始經歷不同的時候.
特別是香港這三年的轉變.
其實真的覺得很多事情是不情願的.
我自己寫講章寫到這個位置的時候.
就覺得有很多事情其實是我不想做的.
但是就在一個不情願的情況下要去做.
但是又要堅持一些應該要堅持的事情.
發覺有很多人會和你的看法不同.
怎樣去大家去接納.
怎樣可以去平衡.
而我在當中思考過程.
某程度上都在我不同的講章當中反映過.
特別我在講鮮鼻球的時候.
講Nabano接納這個字.
其實都是經歷過不同的模樣.
或者是教會運作的情況下.

$^{241}$不同的意見的平衡.
保羅在羅馬書16章.
關於信徒生活的時候.
我們要有一個氣質和表達.
大家真的是來自不同教會的背景.
我們就學習磋商,學習融合,學習彼此接納.
這個都是合一的操練方法.
而不願意是你自己本身有個意願.
但你選擇去融合,選擇大家互相去考量.
這個是你將你的主權放低.
今天有很多人你會發覺他不肯認錯.
你會發覺有些人他不肯妥協.
你會發覺有些人他不肯改.
一意孤行.
就令到很多事情越演越烈.
但是對於我自己或者是在香港教會.
其實有很多事情你是不情願的.
其實1月開始.
應該說1月8日開始.
香港教會的聚會環境就變了很多.
我們一些.
以我們flowcharts為例.
沒有自己的地方.
又或者可以改動的空間很窄.
有些事情都是受環境影響.
但我們都盡.
我是不情願的.
但我們都盡我們的能力.
仍然維持我們最卑微的要求.
就是每個星期都仍然可以有崇拜的機會.
無論是有沒有現場都好.
我們希望保持每個星期六晚.
可以跟弟兄姊妹一起去相遇.
不情願對我來說.
在這日子上都是很難過的日子.
其實我在寫的過程當中.
我又不想太賣弄.
好像賣弄悲情或者太過個人看法.
或者好像對一個純分享.
但是我們作為信徒.

$^{281}$我們在這日子當中有多少事情是你不情願的呢.
還是你覺得時間是這樣.
環境是這樣.
沒有辦法.
那你就接受了呢.
有時候那些事情.
細心去想的時候.
你就會發覺有些事情是真的不是這樣的.
是你可以真的改變的.
在這日子有一首歌對我來說.
是我自己很喜歡的.
那首不是詩歌.
或許你覺得我不是很熟悉.
那也沒有緊要.
這首歌是我自己常常提醒自己.
我現在走的是什麼路.
這首歌是欣宜的最難走的路.
對我來說是很寫實.
很回應我自己在這日子走的路.
可能知道我的會眾都知道我經常很喜歡走路.
那我走路的過程當中.
就是我自我修復的過程.
就是開心不開心.
緊張不緊張.
走路的時候就給自己一個緩衝.
有一個空間.
這首歌最觸動我的就是有幾段歌詞.
寧願.
寧願走一條不是容易的路.
走來走去都還沒到.
其實一直都會問一件事.
就是疫情什麼時候完.
開始的時候就說希望SARS.
就像SARS一樣.
一年到一年半就完成了.
當天氣熱的時候SARS就消聲了.
但現在一年兩年半到三年的時候.
其實疫情好像還是沒有頂.
而不同國家都有不同的政策.
又有不同反對的聲音的時候.

$^{321}$其實路好像越走越遠越遠都還沒到的時候.
其實是怎樣走下去呢.
有人說疫情可能要去到五年.
如果真的去到五年的話.
現在只有一半左右.
我就要想一想.
其實這條路要怎樣去堅持下去.
如果真的要五年的話.
現在去到一半.
我們怎樣繼續去生活呢.
真的要認真想一想.
又在走這條路的時候.
或者在這兩年在教會運作的時候.
這條路都不容易走.
而在這當中你會看到很多政策影響到教會的運作.
很多政策影響到教會對崇拜參與程度的質素和要求的時候.
其實有很多人的信念都很不同.
但你堅持的是什麼呢.
你堅持的是什麼崇拜的質素.
或者崇拜的要求.
或者你在崇拜當中會給到什麼弟兄姊妹呢.
不只是一個production.
我們常常都說不只是一個production.
是有質素 有慰養 到位.
其實是不容易的.
但是有些情況就是.
可能有聲音出 有畫面.
聲畫是否同步不要緊.
最重要是有就夠的話.
我就覺得有些事情不是這樣.
有些人覺得我們很清高.
但是我只是說.
我要找一些人一起共舞.
一起去相處 一起去同行.
在這裡我不得不要說.
就是我不孤單的.
我找到很多和我有共同理念的人.
但是在過去這兩年.
慢慢發覺大家在教會運作的空間.
或者情況都不是不盡相同.

$^{361}$或者在過程當中都不容易一起去同工.
或者 經常都說或者.
有些揣測 去到這個位置可能就得罪了一些人.
或者覺得好像發牢騷.
但是自從 我說過幾次.
自從19年下半年的時候.
梁芝與孫講在教會的講壇上.
已經是有很多不同的分門別類.
去到2020年聚會與分離.
又或者聚會與零命.
已經是影響很多教會.
怎樣去牧養弟兄姊妹.
去到2021年的時候.
就是分散與結連.
離開了香港的弟兄姊妹.
怎樣去再一次 或者牧養他.
還是留在這裡的時候我們才處理.
就是有很多這些不同的理念.
都不容易找到人一起共舞.
一起去共傷.
慢慢就將教會瓦解了.
慢慢就將教會去作出不同的分門別類.
我相信這些都不是真正牧養群體的人.
是想見到的.
但正正就是一個實況.
你不得不問有些什麼是要堅持的.
有些什麼是你覺得你走這條難的路的時候.
你最底牌要守住的呢.
過年不知道大家去哪裡.
我都有去拜年.
期間我跟兩個兒子看了一套舊戲.
那套戲是葉問2.
不知道大家有沒有看.
洪震天 洪師傅 洪金寶飾演.
如果你有印象.
就是他跟一個西洋的拳手打.
他哮喘發作 不敵.
但有一句話對我來說.
在這套戲再重新提醒.
在第一次看的時候我已經很記得這句話.

$^{401}$但跟兩個兒子一起看的時候.
我更加對我在這段日子.
一個很重要的提醒就是.
洪師傅說為生活我可以忍.
但為中國武術被侮辱我吞不下去.
其實在這日子教會有很多東西.
被限制被壓縮.
質素被扭曲.
其實我都明白到有些人覺得情有可原.
但我反而選擇為生活自限.
為生活有些地方不去.
為生活有些事我不做.
我可以忍.
但如果為崇拜質素上.
或者崇拜怎樣可以幫助被隔離的弟兄姊妹.
可以仍然跟上帝一起去敬拜.
仍然結連其他弟兄姊妹.
如果要妥協.
我吞不下去.
如果這是硬性子.
堅信撐到最後會回報.
我會繼續撐下去.
歌詞說到.
苦即使注定要熬.
你覺得上帝好像很久都未到的話.
但你要相信結局是最好的.
The best is yet to come.
最好的尚未來臨.
主耶穌一定會再回來.
這是我們信仰的英許.
福音書最重要提醒我們.
耶穌為我們預備地方.
我們等了很久.
等了很久為何很多事都未出現.
無論是公義.
無論是法治.
無論是我們可以發聲的空間.
你都會發覺很多事未到.
但我們仍然要知道底線.
我們要守住的是甚麼.

$^{441}$經文去到第十九節的時候.
你見到一個豁充的.
他說甚麼.
耶穌說者說.
是子在彼得.
要怎樣死.
榮耀神.
說了者說.
就對他們說.
你跟從我吧.
其實括著那句說話.
是後面修編的.
因為為甚麼呢.
約翰福音成書的時候.
彼得已經過世了.
已經殉道了.
對他來說.
是很後面的事來的.
彼得他真的死了.
為耶穌.
為信仰.
扶上自己的生命.
這句說話是重述.
他是怎樣用自己的信仰.
死在十字架上.
榮耀上帝.
其實彼得怎樣看跟從耶穌呢.
或者彼得怎樣看.
耶穌對他再一次呼籲.
你跟從我吧呢.
回想在約翰福音的信息.
在約翰福音的信息裡面.
你會見到一個.
十三章是雨節晚餐的經文.
雨節晚餐完結後.
那班門徒問.
特別是西門彼得問.
主啊 你去哪裡啊.
耶穌就回答他.
我去的地方.

$^{481}$你現在不能去.
但後來.
你會跟我去.
彼得就說.
主啊.
我們現在有甚麼不能跟你去啊.
我願意為你寫明.
我願意為你死啊.
但是耶穌的口吻又再出現了.
我實實在在告訴你.
雞叫已鮮.
你要三次不認我.
其實在這個對話當中.
也是彼得跟耶穌的對話.
就是你要去哪裡.
我為甚麼去不到.
耶穌說你現在去不到.
但你將來會去.
又是說一個將來式.
但你去做甚麼.
你去做甚麼.
你去.
你真的去.
你願不願意為我死.
我自己年輕的時候.
看一段經文.
就不是很深印象.
因為好像讀經一樣.
就好像讀書一樣讀了它.
那時候經常都要讀書.
就讀書讀了它.
但當我自己.
看第一本潘卓華的.
追隨基督的時候.
其中一句.
當然那時候不知道.
有其他翻譯.
那時候只會看中文.
但很深刻的印象.
就是基督呼召一個人.

$^{521}$呼召他說他死.
那時候對我來說很深刻.
原來做門徒真的會死嗎.
為甚麼當初我缺節訊紙的時候.
人們沒有這樣跟我說.
覺得好像有一點點困困.
但那時候都在.
初算栽培.
我被人初算栽培的時候.
都說到做門徒要付代價.
但問他有甚麼代價付.
他又說不出.
於是一直都給自己一個空間.
問一下自己.
如果耶穌問我要為他死的時候.
我死些甚麼呢.
我真的好像沒有甚麼要為他死.
還有那時候成長環境.
又沒有甚麼要求.
其實都不知道死甚麼.
死不打機.
死一些壞習慣.
死了些甚麼.
想不到甚麼代價要付.
因為信仰沒有考驗.
環境沒有甚麼要考慮.
在過去的日子.
我相信你聽過我講.
又聽過家sir講.
我們過去.
我們這40多歲的人.
在香港.
那30年的信仰其實很安恕.
沒有甚麼要付出.
沒有甚麼要考慮要堅持.
但去到這兩年.
運作教會的時候.
有很多東西不斷地拆.
你又不斷地推你的底線.
又不斷地要你妥協的時候.

$^{561}$其實你會發覺.
你死一次都不夠.
有些東西如果你不堅持.
人家就會被同化.
有些東西你不抓緊.
你就會沒有.
所以就回到剛才講.
生活我可以很多東西都不享受.
不做.
不跟.
不打.
但是為崇拜.
為信仰.
為教會可以運作的空間.
有些東西是吞不下去的.
你聽過房間你的大象這句話嗎.
現在房間的大象大到一個點.
你連站都不能站.
壓縮到一個點.
你連站都不能站.
你可能要屈.
甚至乎你坐著甚麼都不能做.
你還告訴人家.
這房間沒有大象.
這房間沒有大便.
仍然有聲有畫.
就可以繼續下去.
其實不是的.
認識我的人都知道.
我其中一個口頭禪就是.
不要到失去的時候.
才叫珍惜.
其實可以堅持可以做的事.
其實你一早都可以做.
為甚麼你選擇這麼快就妥協.
這麼快就不做呢.
所以.
你見我很多時候都呼籲.
有機會實體崇拜的弟兄姊妹.
實體崇拜.

$^{601}$有機會實時崇拜的弟兄姊妹.
一起實時崇拜.
因為你會知道有很多東西.
很容易阻隔了我們.
耶穌呼籲.
或者耶穌去跟彼得說.
他說.
我現在去的地方.
你現在不能去.
但你將來會去.
其實每一個認真跟從上帝的人.
都會有些東西你要死的.
你要跟隨.
你要割捨.
你要堅持.
你真的願意.
耶穌說.
我願意為你寫命.
耶穌做了一個問句.
你願意為我寫命嗎.
這個人呢.
經歷了耶穌給他的迴轉.
再一次有機會.
去給自己重新出發的時候.
他流到去什麼地方呢.
你會見到.
《士徒行傳》.
頭半部的主角就是彼得.
彼得做了很多事.
彼得做了這些事.
我選了的是林布蘭的.
《彼得在監獄》的一幅名畫.
現在放在以色列博物館.
這裡寫的內容.
都是彼得過去.
在《士徒行傳》的記載.
他流出去.
他讓他的生命流出去.
讓他的生命.
他被上帝再經歷.

$^{641}$他改變.
他願意去流向群體.
流向上帝的子民當中.
你會見到.
他很珍惜自己的身份.
他持守上帝給他有身份.
而有權柄.
他影響他周邊的人.
他行的神蹟.
為要讓更加多人相信.
他本著是上帝給他的大能.
他用他的身份去講道.
以致讓更加多人認識.
你們等待的尼塞亞.
就是我們殺死的耶穌.
現在耶穌基督已經復活了.
你們就悔改信福音.
他持守上帝給他一個.
基法盤石這個身份.
就把教會建立起來.
這個就是他很認清.
他再有機會流出去.
流進人群當中.
他要改變周遭的人.
他行的大能醫治.
都是讓人再感受到.
昔日主耶穌所做的什麼.
就是周遊四方行善事.
醫治各樣的病症.
讓人感受到.
生命再有機會延續.
就是去得聞福音.
死人是沒有福音可傳.
彼得所做的就是.
流進人群當中.
用他的身份影響更加多的人.
到最後他做的是什麼.
就是希望將耶路撒冷教會.
和安提阿吉教會.
做彼此的協調.

$^{681}$事實上我之前講到都講過.
耶路撒冷教會和安提阿吉教會.
因為禮儀.
因為一些應訊.
有不妙然的情況.
其實今天教會都有不同程度.
有不同.
有不少不容易協調的地方.
接下來香港教會的運作空間.
會有很大的變化.
我不是危言聳聽.
我不是要講很多隱喻.
我都不是這樣的人.
但我想講的是.
大面對不同環境的轉變.
不同政策.
對用詞上的解釋.
就影響到很多不同教會的政策.
和弟兄姊妹的理解.
對我們來說.
要考慮一下.
和要珍惜一下.
什麼要值得堅持.
有些是相對.
有些是絕對.
但這個是我們一定要認清.
我們要想的東西.
你覺得不需要想得那麼複雜.
不複雜不要緊.
不複雜就是有飯吃就吃飯.
有工作就上班.
不要想那麼多.
都是一個選擇.
但我會選擇.
上帝給我這個身份.
我怎樣繼續去做.
這條路是不易走的.
苦即使注定要熬.
我仍然覺得這個苦是值得的.
所以回看萬年.

$^{721}$彼得他生命將要結束的時候.
他講的溫馨提示是什麼呢.
彼得前書第二章.
在outflow的第一講.
我是講彼得前書第一章.
身份而有的行為表現.
到二月我呼應的內容.
我都是講彼得.
彼得前書第二章是這樣講的.
你們蒙召原是為此.
因基督也為你們受過苦.
給你們留下榜樣.
叫你們跟隨他的腳蹤.
你會看到嗎.
萬年的彼得.
他很清楚他有再一次被上帝呼召之後.
他願意跟從上帝.
他流進入群體.
將他的生命.
將他的能力.
去到周遭不同的人.
因為主耶穌.
何嘗又不是這樣呢.
所以他清楚他自己的呼召.
他和主耶穌一起去受苦.
因為他是在做耶穌給他的榜樣.
他只是在走耶穌後面.
跟著耶穌前面的路徑走.
去到第四章的時候.
他提醒.
「散居在小亞細亞 加帕多加」.
亞細亞 比推尼.
宗教會就說.
所以.
那照上帝旨意受苦的人.
要有一心為善.
將自己的靈魂交予那信實造化之主.
這句話不是主耶穌跟他說.
在約翰福音13章記載.
「你願意為我寫命嗎?」.

$^{761}$受苦是基督徒的印記.
受苦是主耶穌先行給我們的榜樣.
「你願意為主寫命嗎?」.
將自己的靈魂交給那信實造化的主.
是彼得去晚年的時候寫.
當彼得成書之後.
尼祿就開始大舉迫伯基督徒.
彼得就殉道.
這個就應驗了.
剛才約翰福音21章19節那段經文.
就成為了內容.
去到這裡的時候.
我自己那天靈修.
反覆去想這段經文.
我就說.
其實日子怎麼過呢.
現在才三年.
但是當我預備講章.
或者當我一直看這些經文的時候.
彼得由第一次他認識上帝.
到他離棄耶穌.
他好像剛才去到21章.
上帝再呼召彼得.
彼得願意再跟從耶穌的時候.
開展使徒行傳的技術.
到現在彼得成書.
彼得要結束生命.
大概是三十年.
他走了三十年不容易走的路.
他走了三十年.
是在羅馬政權逼迫.
是在不同教會裡面的糾紛.
在不同的人士張力上.
去募會去運作教會.
去跟弟子妹同行.
他走這條路走了三十年.
我們現在只是走了三年.
當我想到這一點的時候.
我就沒話說了.
如果苦即使注定要熬的話.

$^{801}$但你相信上帝從不遲疑.
祂一定會回來.
我們就咬緊牙筋.
走這條不斷被扭曲.
不斷被壓制.
但是我們仍然知道自己身份要走的路.
可能你再聽到再有一次機會.
去一些地方.
又或者再有一次經歷一些事.
現在很多地方都說限聚.
減少外出.
減少很多東西.
但我希望.
如果你真的困在家裡.
我希望你真的做一個練習.
就是想起當初你初信主的時候.
或者當初你最深經歷上帝的時刻是什麼.
我希望你再一次有機會想一下這件事.
你想的時候.
又想一下現在其實你是一個什麼人.
你經歷的事對你來說有什麼提醒呢.
我相信你再一次去回想.
跟耶穌相遇的日子.
耶穌會再做一件事.
耶穌做了什麼呢.
耶穌做的事我相信跟當日彼得說的話一樣.
就是.
你跟從我吧.
這個世界會過去的.
耶穌在當中提醒我們.
咬緊牙根走這條路.
你跟從我吧.
我就快回來了.
我希望在這段經文給我們的提醒.
有很多事是你不願意的.
而我不願意的.
但是我們不斷被不願意.
你都接受的時候.
但你要記住一件事.
有些事要堅持.

$^{841}$為生活.
有很多事你可以忍受.
但為信仰.
如果你不堅持的話.
就慢慢那個房間的大象.
就壓到你呼吸不到.
最後你就什麼都沒有了.
願意我們有靈巧的智慧.
但我們保存純良像甲子的心.
去跟隨上帝.
施哥活著一次提醒我們.
激勵我們.
願主與你們同在.
阿門.
\newpage



\section{彼得前書 1:17-21-20220219}
\label{sec:GPCXXRzSHkw}
\textbf{【網上聖餐崇拜】而家仲有乜好flow?!|彼得前書1\_17-21|20220219 [GPCXXRzSHkw]}
\newline
\newline
連結: \href{https://youtube.com/watch?v=GPCXXRzSHkw}{\texttt{ https://youtube.com/watch?v=GPCXXRzSHkw}} ~~~~ 語音日期: 2022-02-19 
\newline
\newline
\hyperref[sec:wntIcXZCGmo]{\small{< < < PREV SERMON < < <}}
~
\hyperref[sec:index_chronic]{\small{[返順時目]}}
~
\hyperref[sec:index_scriptual]{\small{[返順卷目]}}
~
\hyperref[sec:kIOQG9wgqRs]{\small{> > > NEXT SERMON > > >}}
\newline
\newline
彼得前書 1:17-21-20220219
\newline
\begin{longtable}{cl}
\hline
\hline
章節 & 經文 (和合本修訂版)\\
\hline
1:17 & \begin{tabularx}{0.7\textwidth}{X} 既然你們稱那不偏待人、按各人行為審判人的主為父,就當存敬畏的心,度你們在世寄居的日子。 \end{tabularx} \\ \\ \relax
1:18 & \begin{tabularx}{0.7\textwidth}{X} 你們知道,你們得以從你們祖先傳下來虛妄的行為中救贖出來,不是靠著會朽壞的金銀等物, \end{tabularx} \\ \\ \relax
1:19 & \begin{tabularx}{0.7\textwidth}{X} 而是憑著基督的寶血,如同無瑕疵、無玷污的羔羊的血。 \end{tabularx} \\ \\ \relax
1:20 & \begin{tabularx}{0.7\textwidth}{X} 基督是神在創世以前所預知,而在這末世才為你們顯現的。 \end{tabularx} \\ \\ \relax
1:21 & \begin{tabularx}{0.7\textwidth}{X} 你們也因著他而信那使他從死人中復活、又給他榮耀的神,好讓你們的信心和盼望都在於神。 \end{tabularx} \\ \\ \relax
1:22 & \begin{tabularx}{0.7\textwidth}{X} 既然你們因順從真理而潔淨了自己的心靈,能真誠愛弟兄,就該以清潔的心彼此切實相愛。 \end{tabularx} \\ \\ \relax
1:23 & \begin{tabularx}{0.7\textwidth}{X} 你們蒙了重生,不是由於會朽壞的種子,而是由於不會朽壞的種子,是藉著神永活常存的道。 \end{tabularx} \\ \\ \relax
1:24 & \begin{tabularx}{0.7\textwidth}{X} 因為「凡血肉之軀的盡都如草,他的一切榮美像草上的花;草必枯乾,花必凋謝, \end{tabularx} \\ \\ \relax
1:25 & \begin{tabularx}{0.7\textwidth}{X} 惟有主的道永遠常存。」這話就是傳給你們的福音。 \end{tabularx} \\ \\
[1ex]
\hline
\hline
\end{longtable}
$^{1}$各位丁姊妹晚安.
問候在世界各地不同的弟兄姊妹.
無論你是屬於地球任何一個地方.
如果你現在看Folk Church的聚會的時候.
願主的恩典夠你用.
和上帝的平安親自與你同在.
更深願在不同的地方,不同的境況.
上帝的慈愛緊緊跟隨著你們.
剛才在敬拜的時候看到下面留言.
有一位朋友叫Steve.
他中了COVID-19.
我相信他現在在家隔離中.
我很明白這個弟兄的情況.
因為我家人.
就是我妹妹的老公.
昨天就中了確診.
他整個家都很徬徨.
爸爸就在房間裡.
日幾已經不敢出門,不敢出房間.
在這個時候.
我想弟兄姊妹用一段很短的時間.
做一件事或兩件事.
特別是為在家隔離中的弟兄姊妹.
或香港人去禱告.
我更加想做多一件事.
請大家在禱告裡.
去紀念香港的醫護人員.
或者你發一個訊息給你認識的醫護人員.
跟他說.
身為香港人.
為了香港的醫護人員.
我心感自豪.
這個時候我們希望有一分鐘的時間.
你要簡短的祈禱.
或者你拿起你的手機.
發給一些在家隔離中的人.
或者你認識的醫護人員.
你告訴他們.
香港人仍然為香港的醫護人員.
心感自豪.

$^{41}$願意在這麼艱難的日子裡.
我們仍然與他們一起.
並肩作戰.
我們還有一點時間.
我想在這裡先說.
我們有些很實質的行動.
在這麼艱難的日子裡.
為香港人自己加油和打氣.
.
一會兒我們會是聖餐的時間.
請在螢光幕前的弟兄姊妹.
你準備好你的聖餐餅和杯.
更願意我們準備好我們的心靈.
一起領受上帝今天給我們.
豐富的筵席.
.
我相信在過幾個月裡.
我們變化得很快和很多.
在這幾個月裡.
我們變化到一個地步.
由明明還在上課,上學.
仍然可以一群人去吃飯.
到今天我們變化得很快.
變化到人心惶惶.
人心很恐懼.
就是害怕到一個地步.
我們覺得不知道該怎麼辦.
你看到地鐵站越來越少人.
街上很少人流離.
這個變化成為了我們這個時代裡.
大家一個共同的回憶.
我們仍然在問一個很重要的問題.
到底我們在害怕之餘.
做些什麼可以做.
或者在教會裡.
當要被停頓.
當所有的場所要被關閉起來的時候.
我們的問題仍然要問.
上帝的道是不是可以被攔阻.
上帝的心意是不是可以被.

$^{81}$完全因為人的緣故而有所破損.
今天我們仍然會想看一段我們.
上個月我自己的經文.
彼得前書的第一章.
我們再下一章來看一看.
其實頭兩章是說彼得前書的背景.
背景很簡單.
正如上次所說一樣.
彼得是寫給一些流散了的人.
流散的人是在說羅馬時期的時候.
有個皇帝叫做尼祿王.
我們上個月大約是這樣說.
因為大火的緣故.
他把罪名交給基督徒.
所以在羅馬城有很多的大巴.
在這個大巴裡面不同的人.
走到不同的地方裡面.
就去流散就去寄居.
這是整個彼得前書的背景.
所以基本上和我們現在香港的現況非常相似.
我們說什麼.
上次我們說寄居是在說被揀選.
被揀選是重要的.
寄居不是在說被人遺棄.
被人污名化.
甚或乎被人覺得他不行.
他貪生怕死.
彼得不用這些字眼形容流散寄居的人.
他用了揀選這個字眼.
我們再下一章.
我們說回今天我們想說的那段經文.
這段是彼得前書的17至21節.
一章的17至21節.
這一句是希臘文的一句句子.
你看到中文有很多句號.
但其實基本上不是的.
17至21節是一個很完整的句子.
我們試試讀一點.
這是我自己的翻譯.
他說如果你呼叫父親.

$^{121}$就是那位偏待人審判是按各人所作的功.
故你們應好好行在懼怕中.
在我們寄居的時期裡面.
你看這段經文的時候.
第一件事要想的是.
其實為何無端要呼求天父.
剛才我們的背景說什麼.
羅馬城大火.
尼祿王要屈基督徒.
所以要將基督徒大肆逼迫.
所以很多人來劉山.
走了去不同地方.
所以當劉山想起這件事的時候.
不期然會想到一個問題.
神父你是不是不公平.
神父你是不是明明是那些作惡的尼祿王.
在做很多邪惡的事情.
為什麼這一刻你仍然不讓他這個皇帝.
任期少一點短一點.
甚至立即死亡.
將所有的仇能夠報一下.
所以這句話是在說.
彼得勸慰那些劉山的人.
審判是按個人所做的.
仍然會是這樣的.
但問題是懼怕不是在說.
那個尼祿王懼不懼怕.
或者那些作惡的人懼不懼怕的問題.
他將問題放在哪裡.
他放在我們應該要懼怕的那裡.
他的重點是.
不是尼祿王你要看著.
審判快要來.
快要死了.
他不是這樣說這句話.
他說的一句很重要的話是.
那個懼怕不是在說尼祿王本身.
那些作惡的人本身.
作惡的人本身他懼不懼怕.
我們不知道.

$^{161}$但他說了一個很重要的事情是.
我們應該好好懼怕.
今天整個訊息我想強調的是.
到底我們為什麼要懼怕.
或者我們的問題是.
當彼得說其實神會按著人去審判的時候.
那這句話是否應該要跟尼祿王說.
他應該懼怕他做了這麼多邪惡的事情.
但彼得說神不偏待人.
他說審判是會按個人所行的功而做.
但他說我們應該要懼怕.
這個是很值得問的問題.
很值得思考的問題.
他說流散的人有什麼懼怕.
那些作惡的人應該要懼怕.
但他說我們應該好行在懼怕的裡面.
是什麼意思.
我們再看後面那幾節的聖經.
我們大體上會再掌握多一點.
為什麼我們要懼怕.
我們先看回經文.
他說真的要知道你們被贖回來.
基本上今天整個信息我會說被贖回來.
贖回這個字.
離開你們沒用的.
來自先祖的行為.
就算不能扭壞的銀子與金子.
我們先看這一句.
待會先說基督的補血.
像是沒有瑕疵這些.
待會聖餐的時候很適合我們用的一段聖經.
我們先看什麼叫被贖回來的時候.
要離開你們沒用的那些東西.
那些東西是什麼來自先祖的.
是不能扭壞的銀子與金子.
想說什麼.
其實猶太人的歷史裡面.
很重要的一樣東西是什麼.
就是他有很多被贖回的經歷.
無論是以色列.

$^{201}$在埃及過紅海的時候.
他們有被贖回的經歷.
或者他們被擄到巴比倫的時候.
他們期望被贖回.
回到耶路撒冷這個地方.
所以對猶太人來說.
被贖回其實從來都不是一件陌生的事情.
但彼得在這裡說我們要懼怕.
因為我們要知道我們會被贖回.
其實他想說的是一件什麼事情.
我們首先了解一下.
其實贖回的意思是什麼.
今天我們花最多的力氣時間就是說這件事.
我們下一張PowerPoint看一看.
在希臘文化裡面.
贖回這一個詞其實不陌生.
贖回其實通常都是用在奴隸身上的字眼.
奴隸一生要跟隨主人.
但當那個僕人儲夠一定的錢.
那筆錢可以很大.
不知道他為什麼會儲到這麼大的錢.
他就可以跟主人說.
主人我就可以回到神廟裡.
在神廟面前.
我把那筆錢給了主人.
從此以後我的身份就不再屬於主人的奴隸.
我反而是屬於神廟的僕人.
就是神廟其中一個僕人.
所以他把僕人的身份.
由一個人的身份轉化成為神廟裡面.
一個神僕人的身份.
所以對於贖回來說.
這個觀念其實是要用錢的.
就是要用錢買回來才可以.
對於一個僕人來說.
能夠主人放他走很簡單.
就是他用錢去買自己回來.
但對於這個觀念來說.
奇怪的地方是.
不是從此之後他自由的.

$^{241}$他仍然成為一個僕人的身份.
一個奴隸的身份.
但這個奴隸的身份不再屬於任何人的架構下.
他從此以後成為了上帝的僕人.
如果你這樣想清楚的話.
其實我們大體會明白.
彼得為什麼用這個字眼.
為什麼贖回這個觀念走出來.
他其實想說一件很重要的事情.
當猶太人在外邦羅馬這個地方裡面.
被逼迫被傷害被尼祿王攻擊的時候.
他們能夠逃脫.
逃去了不同流散的地方.
在不同流散的地方裡面.
他們生活的時候.
他們在看自己的生命.
好像被上帝贖回一樣.
有人甚至要殺害他們.
傷害他們.
破壞他們做的生意.
到他們可以自己逃脫.
去到不同地方裡面生活的時候.
這個贖回的感覺對於那些流散的人來說.
是一件很興奮的事情.
很雀躍的事情.
他從此以後就不需要再活在懼怕的裡面.
因為他懼怕尼祿王不知道會對他做些什麼.
但剛才你不是說.
彼得提醒那些流散的人.
其實他要害怕.
他要害怕什麼.
原來他要真正害怕的.
就是上帝他自己.
好像這個歷史背景一樣.
那個伯夷人從此以後.
沒錯.
他不需要再看他主人.
就是肉身的主人的眉頭眼額.
這個贖回的過程裡面不是贖回這麼簡單的.
從此以後你的身份是屬於那個神廟裡面的那個神.

$^{281}$同樣地彼得用這個觀念.
就在告訴那些流散的人.
沒錯你可以很憎恨尼祿王.
很不喜歡尼祿王.
但他說你們需要懼怕需要害怕.
因為你知道這個上帝贖回你.
去到一個不同的地方裡面寄居的時候.
你的身份從此以後.
不再要贖回自己.
你在那邊上帝仍然要與你同在.
你的身份從此以後就屬於上帝他自己.
這個觀念很實在地.
用一個希羅文化.
再一次演繹在流散者的心中.
但彼得不停在這裡.
彼得知道其實希羅文化裡面說贖回這個字眼.
其實有一定的限制.
反而在舊聖經裡面說贖回這個字講得更加多.
我們再講下一章.
再留下一章.
在七十事譯本聖經裡面.
在希列文的舊聖經裡面.
舊約是希伯來文寫.
但在兩約時期翻譯成希列文.
我們叫七十事譯本.
被贖回這個字在利美記有26次.
在詩篇有27次.
在初一及二有4次.
在二賽書中有12次.
在利美記裡面.
在彼得前書的一章16節裡面說.
我是聖傑你是聖傑.
這個觀念就是說.
不是說倫理道德教訓做得有多好的問題.
不是的,我是聖傑,你是聖傑這句說話.
無論在利美記或彼得前書的一章16節.
就是17節之前的那一節.
那個聖傑的觀念不是說.
上帝是做很多好事.
所以我要做很多好事.

$^{321}$因為他是聖傑,所以我是聖傑.
聖傑從來不是說一種倫理道德表現表達的問題.
聖傑其實真正的意思和贖回有關.
是因為我們的身份是屬於上主.
由你以前去到埃及的時候.
你會很confuse.
其實到底我是屬於埃及眾神裡面其中一位神.
但我的組織是屬於耶和華.
就這樣緣故.
還是要撕掉他以往所有的經歷和所有的看法.
從此以後我的生命就單單贖回我的主,上主.
就是那位自有永有的那一位.
所以如果這樣說的話.
贖回的意思就好像神廟裡面的那個僕人.
從此以後雖然離開了肉身的主人.
他還是要贖回一個神明底下的一個僕人的觀念.
如果這樣理解的話.
我們再看多一點一兩段經文.
看看贖回這一詞在舊聖經裡面其實是怎樣用的.
在生命記第七章八節他怎樣用.
他說:只因耶和華愛你們.
又因要守他向你們列祖所起的誓.
就用大能的手令你們出來.
從遺老之家救贖你們.
脫離埃及王法老的手.
這段經文清楚表明剛才他所說的話.
贖回這個觀念是要離開法老的手.
肉身他們的主人.
返回到歸回萬鈞耶和華那位自由永有的上主.
所以摩西帶領他們離開埃及.
進到曠野進迦南.
正正是這個觀念.
這串書說什麼.
他說:耶和華如此說.
你們是無價被賣的.
也必無銀被贖.
你剛才看回一章十七至二十一節的經文的時候.
你會明白為什麼他說.
不需要用金與銀我們被贖.
他很清楚地表明一句話.

$^{361}$以蔡亞書那邊說.
原來希臘文化裡的僕人被贖.
是需要有一個觀念的.
就是要錢買自己回來.
但屬於耶和華的人.
耶和華親自救贖他們有什麼不同.
就是不需要再用價錢.
金與銀.
去被贖回來.
這個顯明什麼.
不是僕人自己賺到錢.
幸運地將自己救贖.
整個猶太人救贖的觀念是什麼.
是上主自己親自無價將我們救贖回來.
親愛的姐妹.
或許你和我會有一個想法.
正如很多香港人我最近都聽這句話.
他說有錢的.
有學識的.
工作的性質在外國吃香的那些.
走的都走了.
可能誇張地說.
但你聽完這句話的時候.
好像在說一個我們走的觀念.
我們一個上帝大帝離開的觀念.
是因為我們自己做到的事.
但彼得在這裡再次強調什麼.
不是我們的能力我們的手.
賺了什麼可以上帝換來上帝救贖.
上帝用無價無銀將我們賣贖.
今天我們成了何等樣的人.
是上帝的恩典堆砌結連.
成就一個我們看起來好像很可以的自己.
但聖經在提醒我們.
被贖回這個觀念.
不是我們的能力.
如果我們有能力的話.
就好像出紅海之後所發生的第一件事.
他們用金與銀做了一隻金牛.
毒出來敬拜.

$^{401}$所以親愛的姐妹.
給我們深深的相信.
我們生命能夠被贖回.
不是因為我們手可以有能力.
我們手可以很剛強.
做很多很厲害的事.
他在提醒你和我.
贖回不需要金與銀.
是上帝親自的救贖.
所以詩篇34篇第22節這樣說.
耶和華救贖他僕人的靈魂.
凡透靠他的必不自定罪.
我們不仔細講詩篇34篇.
但我想表達一點很重要的是.
他在救贖的.
不是只是在救贖一個人的身體在哪裡.
今天在香港 今天在加拿大 今天在澳洲.
今天在哪一個城市.
他不是只是在講身體的救贖.
他在講的是一個僕人身心靈的救贖.
是否一個投靠的問題.
他仍然問一個很嚴肅的問題.
我們是否在投靠他.
如果這樣說的話.
我們再飛回之前17至21節的翻譯.
我們會開始明白這句話是說什麼.
他說真知道你們被贖回來.
離開你們沒有用的.
來自先祖的行為.
不是那些能扭壞的銀子與金子.
剛才講完那幾段舊約聖經的時候.
我們再重做一下他想說什麼.
他說一個人被救回 救贖.
不是他自己能力金與銀的錢.
是上帝自己親自的救贖.
是上帝用無價救贖.
而那個無價救贖不是在講身體從此以後就自由.
他說其實是不自由的.
用希羅的文化.
你更加是屬於那個神的僕人.

$^{441}$但不要忘記.
其中這一段聖經裡面那句是.
離開你們沒有用的.
來自先祖的行為這一句.
怎樣才能夠身心靈投靠給上主.
是因為我們可以撕掉.
我們以往的習慣和生活模式.
給得很清楚的是.
那些流散的人.
你不要以為你再留戀你先祖裡面的行為.
他說不是的.
你以往做過什麼在羅馬.
你整個家族在羅馬100年200年寄居的時候做過什麼.
你覺得很厲害很厲害.
他說你不要記得那些東西.
那些沒有用的.
你試一下在不同流散的地方.
忘記你先祖做過的一切.
你為上帝的緣故.
讓祂在你生命裡面去做新的事情.
丁子妹.
今日香港面對一個很複雜.
很混亂.
很多恐懼的時間.
這個時間是一個這樣的時間.
很多人有很多不同的消息出現.
有很多人有不同的放風的行徑出來.
有很多人在說很多不同的東西.
或者在策劃很多不同的事情.
沒錯這些都很真實.
很實在.
但是他說我們要懼怕的話.
就不是懼怕背後的人在做什麼.
彼得佐治提醒的是.
要怕的就是我們要怕.
我們怕什麼.
其實我們怕的是.
面對著這麼混亂這麼複雜.
這麼難堪的時候.
我們心裡面.

$^{481}$是不是還在想.
希望好像以往的日子.
盡快回來恢復就算.
我們是不是還在留戀.
我們以往對教會的觀念.
是一個什麼樣的教會觀念.
如果接下來24天之後.
教會已經成為了.
有些人不能夠進入的情況的時候.
我們不期願要問的是.
到底今天的教會觀是什麼.
我們是不是還想著.
以前是回教會的人就叫做教會生活.
但是如果下星期開始.
仍然有170多萬人.
不能夠進入教會的時候.
就算那些可能平時都不回.
但是偶爾想回的時候.
他都回不了.
其實教會還是什麼意義.
或者我們是不是仍然留戀.
以前是.
仍然要回到團體生活.
回小組 回教會裡面的房間.
坐在那張長椅裡面的一個位置.
那個就叫做教會生活.
如果今時今日.
有些人是因為這個緣故被lay off.
有些人因為這個緣故.
去不到買東西的時候.
我們仍然要問.
我們是不是還懷念.
以往的生活模式與習慣.
我們可不可以在前面的路裡面.
給多些人去建構和想像.
新的事情 新的事物出現.
令我最感興奮的其中一個廣告.
最近是什麼.
不是要賣廣告.
就是Hong Kong 什麼什麼那個.

$^{521}$就是他請很多物流的人.
他說他要請很多運輸員 開車的人.
我覺得那個令我很興奮.
不知道為什麼聽完那個廣告.
雖然我未必會做.
可能嘗試做part time.
你看到很多人在活一個新的模式.
新的架構.
因為真的有些人.
各樣各樣的緣故.
他不能夠再進入教會.
他不能夠再進入超級市場.
不能夠進入商場.
過去一個星期見到.
有人在大帽山剪頭髮.
過去一個星期見到.
很多家居確診的病人.
有很多中醫師.
遙距和他治病.
他看完病之後.
他煎完一些藥.
他叫LALA MOVE.
送一些藥.
在那些人的家樓下叫人去拿.
你突然間發現.
有很多需要的在面前.
我們活在恐懼裡邊.
是因為現實很恐懼.
但是還是像經文所說.
其實我們應該要恐懼的是.
我們仍然未變到.
我們仍然沒有離開那些沒用的.
離開那些先祖的行為.
仍然有一個很固定的意識形態.
在我們心裡面出現.
我們以為那些叫侍奉.
我們以為這些叫參與教會生活.
還是在這個世代裡邊.
我們應該害怕的是.
我們有沒有根生.

$^{561}$我們有沒有被贖回的感覺.
還是我們只是肉身上被贖回.
覺得自己沒有病沒有事.
很安全.
還是我們生命裡邊.
開始轉化,改變.
我們很明白.
我們現在很灰心.
我們都很明白.
我們覺得已經沒有想像的空間在前面.
你看見所有東西都好像似是而非.
古靈精怪.
你心裡邊會很悲傷.
很難過.
有些人離開了某個地方之後.
從此之後好像不見了一樣.
頂智媒大體上我們都明白.
我們活在一個很不容易的光景裡邊.
但如果彼得說要懼怕的話.
懼怕的不是說尼祿王的問題.
而是他要把大火的罪名掛在基督身上的問題.
真正要懼怕的是.
我們面對轉變的裡邊.
我們的新心靈態度上有沒有根生與改變.
還是我們讀的神學,讀的聖經.
以往聽教會講的東西.
都只適應在以往的教會生活.
我們裡邊有沒有心意根生而變化到.
在前面一個月,兩個月,三個月.
甚至一年,兩年的生活裡邊.
都會產生很多極大的變化的時候.
我們裡邊有沒有追得上這些變化.
重新定義什麼叫教會,什麼叫團契.
什麼叫彼此相愛.
親愛的弟兄姊妹.
如果你剛才在講道的時候.
你有發過一些SMS,WhatsApp,Message給那些醫護人員.
或者家居隔離的弟兄姊妹.
或者香港人的時候.
由以往一個城市裡邊,香港急速裡邊.

$^{601}$很多時候都是管自己的東西就夠了.
但開始在這個世代裡邊.
如果你問我自己的話.
我覺得沒什麼已經可以做得長久的時候了.
有什麼可以取替你牢牢的,形形異異的.
很多你覺得自己要建立的王國.
其實真的是一個電話,一個問候.
當人很害怕的時候.
上帝可以在他當中.
一食間盛餐.
耶穌基督設立餅和貝.
當他設立完死而復活之後.
他回到門徒當中.
那些門徒關上門很害怕.
耶穌的出現.
讓所有局面開始改變.
當時候的政權仍然殘酷,仍然卑鄙.
羅馬政權仍然無恥.
耶穌基督復活沒有將整個政權剷臨.
耶穌基督在人的恐懼裡邊.
讓他看到無限的可能.
讓今天這個餅和貝.
在螢光幕前面.
你守住這個餅和貝.
一食間你要去領受的時候.
你跟自己說.
我要將上帝這份平安.
這份更新的意念.
臨到在今天這個餅和貝.
跟留在香港的人說.
我們仍然相信上帝仍然可以在這個地上.
有更多的可能性.
多過我所想的.
求親愛的主耶穌.
祝福香港.
讓我們害怕的是.
我們變切,更生不夠.
讓我們看著前面的那個局.
有更多無限從上主而來的可能性.
在這個地方再一次出現.

$^{641}$好叫上帝從不偏待人.
上帝會按著各人所行而審判.
我們一起低頭祈禱.
上帝是你自己掌管一切.
你所掌管的一切.
從來都不是我們所思念,所思想.
由以往所經歷.
可以釀成一個什麼智慧可以了解很多.
上帝求你幫我們拿走這些.
我們以為現在這一刻.
我們看著前面的路.
仍然可以有很多方法,很多聰明的地方.
可以有很多智慧的技巧去做到.
上帝求你幫我們拿走這些.
那些沒用和先祖所留下來的東西.
上帝求你這個時候給我們.
在這個這麼複雜的世代裡面.
看到你的手你想做什麼.
在香港這個地圖裡面.
仍然有什麼在做.
求親愛的巴天父.
將你的心意揭示,顯露.
在我們這個世代當中.
藉著餅和杯再一次向我們顯示.
原來死了的尼塞亞.
才會真正帶來根生與改變.
求這個的驚訝.
成為我們今天領受這個餅和杯的時候.
都跟自己內心這樣說.
上帝永遠會做生事,其事那一位.
願我們這位親愛的主.
在香港這片地圖上.
做更多奇妙的事情.
Hallelujah.
我們這樣禱告.
奉耶穌基督彌補貴命求.
Amen.
\newpage



\section{創世記 37:1-50:26-20220226}
\label{sec:kIOQG9wgqRs}
\textbf{【網上崇拜】相信一切是最好的安排|創世記37\_1-50\_26|20220226 [kIOQG9wgqRs]}
\newline
\newline
連結: \href{https://youtube.com/watch?v=kIOQG9wgqRs}{\texttt{ https://youtube.com/watch?v=kIOQG9wgqRs}} ~~~~ 語音日期: 2022-02-26 
\newline
\newline
\hyperref[sec:GPCXXRzSHkw]{\small{< < < PREV SERMON < < <}}
~
\hyperref[sec:index_chronic]{\small{[返順時目]}}
~
\hyperref[sec:index_scriptual]{\small{[返順卷目]}}
~
\hyperref[sec:oFrw_raeCu8]{\small{> > > NEXT SERMON > > >}}
\newline
\newline
$^{1}$在開始聆聽上帝的話語之前.
我們一起禱告仰望主.
親愛的上主.
無論是在香港的我們.
或者是在遠方.
面對著戰爭烏克蘭的平民百姓.
我們都覺得很無奈 很乏力.
聖經這樣說.
他說有人靠車 有人靠馬.
但我們要提耶和華.
我們神的命.
他們都屈膝付賭.
我們卻站起來 堅立不移.
盼望詩人所的宣告.
都成為我們的宣告.
願主所憎惡的惡人.
願那些貪婪 充滿野心的極權.
能夠灰飛煙滅.
求主你也叫那些為你堅持到底的人.
將來可以在你面前.
能夠昂然地站立起來.
祈禱是奉耶穌基督的聖名字祈求.
阿們.
留堂的雙月題outflow.
去到最後一講.
很想以希伯來聖經中.
裡面一些流散的敘事.
去和大家宣講.
今天選了創世紀裡面.
虐室流散在外的經歷.
去和大家講.
不過值得留意的是.
虐室的流散和我們過往提及到的.
但爾利 以斯帖這些人有不同.
但爾利他們是因為國家面對外敵的入侵.
戰敗而被擄到其他地方.
他們面對著壓迫的管治.
但他們仍然運用智慧.
怎樣活出忠於耶和華的信仰.
但虐室的流散.

$^{41}$不是因為國家層面的原因.
而是因為他們家庭裡面.
有一些深層次的矛盾糾紛去造成.
以致令到他們同父異母的兄弟.
將他們賣到埃及.
透過創世紀37章至50章的敘事.
讓我們看到虐室在外地生活的種種經歷.
以及他們怎樣去踐行和持守.
他們自己固有的民族和信仰身份.
甚至在敘事中我們看到的是.
他最終成為了希伯來人的拯救.
我們看看當中對我們有什麼啟迪.
我們首先去看虐室被賣的原因和經過.
37章一開始讓我們看到的是.
虐室有一種性格就是為人很正直.
很剛直不阿.
第二節說他會將哥哥的惡行報給他的父親.
本來兄弟之間打小報告其實是一件很小的事.
不過前提是他爸爸偏愛虐室.
曾經在第三節告訴我們.
他曾經送了一件彩色的長袍給他.
這件長袍反映了什麼.
他將要在這個家庭做領導.
這兩件事加起來就足以引起哥哥對他的不滿和仇視.
再加上虐室有一種有那句說那句的性格.
他沒有留意到其實在男性的群體中間.
都可以有很多勾心鬥角的猜忌.
當他毫無保留地和他哥哥分享他所做的兩個夢的時候.
令到他們兄弟之間就產生了一個很嚴重的對立.
五節至到第十一節給我們看到的是.
這兩個夢可能從虐室來看.
其實只是上帝向他啟示.
他將來要攀升到一個高級.
他父母甚至兄弟一個領導的位置.
是上帝向虐室去啟示顯明他自己一個證據.
這個夢本身不是虐室自己的野心.
不過從兄弟的角度來看.
這些夢只是突顯了虐室既是雅各最寵愛的孩子.
並且不單止得到人的寵愛.
更加得到上帝的眷顧.

$^{81}$所以他們慢慢地不但恨.
下一章給我們看到的是.
經文告訴我們他們妒忌.
結果是怎樣呢.
在十九至二十節給我們看到的是.
虐室在他被他爸爸猜犯去找他哥哥的路上.
他哥哥於是合謀想伸手去殺了他.
原因說得很直白.
擔心虐室的夢境會成真.
他們怎樣稱呼他呢.
他們稱呼他是做夢的.
即是殺了他.
看看他將來的夢可以怎樣成就.
一開始是想謀害他.
但最後給我們看到的是.
他們將虐室賣給路過的商隊.
而商隊載著貨物經過迦南地.
將要前往埃及.
這些商人隨後就將虐室給了.
法老的官員.
就是護衛長波提弗.
無辜被賣到埃及.
即是做奴隸的虐室.
他可以很理直氣壯地怨天尤人.
又或者毒舌迦南地的人和事.
都是那裡不好.
又或者他選擇在埃及自暴自棄.
摒棄他自己的民族.
摒棄他的信仰身份.
不過虐室沒有這樣做.
虐室依然是在埃及.
持守他的信仰.
在37章的經文中.
給我們看到的是.
他在波提弗的家中.
無論眼前遇到的是信或是逆.
他都認定耶和華是他行事為人的唯一標準.
39章2到5節給我們看到的是.
虐室有耶和華和他同在.
以致讓他成為一個通達的人.

$^{121}$即是他做什麼在他手上.
經他手上的都很順利.
所以波提弗就將一切的事情交給他管理.
而且經他手上所做的一切都很順利.
虐室成為了波提弗家中的祝福.
虐室在埃及有上帝的同在.
亦都在人眼前蒙恩.
當創世紀的讀者看到這裡的時候.
就覺得一切都很好.
不過罪事者立即提醒我們.
即使有上帝同在.
又或者是那份同在那份恩典多到.
其他人都看到都好.
但是有上帝同在的人.
從來都不是對試探,對患難.
或者冤屈可以免疫.
相反那個忠於耶和華信仰的虐室.
很快就為自己帶來一個更大的苦難.
39章7到10節的經文給我們看到的就是.
很多都很順利的時候.
他開始面對著試探.
這個試探是來自波提弗的太太.
波提弗的太太起初就是用什麼.
用言語上去挑逗.
虐室起初就是採取一個三不的策略.
不聽從他,不跟他同寢,不跟他走在一起.
不過有一天波提弗的太太就趁家裡沒有人的時候.
怎樣呀?有進一步的行動.
11節給我們看到的就是他伸手拉著虐室.
虐室可以怎樣呀?他只能夠迴避.
36著走為上著.
但是結果換來的是什麼?.
結果換來的是這個婦人對虐室的誣衊.
很想跟大家看一下他這一番話有多麼歹毒.
我們看一下第14至17節他是怎樣稱呼虐室.
當然首先沒有提到他的名字.
但經文怎麼說?14節.
你看一下他就是在說波提弗.
帶了一個希伯來人到我們這裡戲弄我們.
第17節然後再說.

$^{161}$帶到我們這裡來的希伯來僕人.
將來要調戲我.
你看到這個女人用了一個什麼的言詞.
他在訴諸民族和階級的差異.
他在跟家人說那個外人.
那個希伯來人那個希伯來僕人來形容虐室.
首先就是開始分你和我.
那個是外人.
我現在投訴的是一個外人.
很容易當我們要分你和我的時候.
很容易敵我分明.
他在企圖令他的誣告更加有說服力.
然後他的指控是什麼?.
他說他是用戲弄.
戲弄這個字是一個很有機心的用字.
為什麼?這個字的意思就是說.
在說虐室在做什麼.
他簡單這裡兩個字.
但其實是在說虐室企圖想向他施暴.
幸好他大聲叫他停.
他才停手.
才倖免於難這樣的意思.
結果他的誣告是怎樣呢?.
當然我們再看下去的時候就知道.
他的誣告他的煽惑是有效的.
手握大權的波提弗很生氣.
於是抓了虐室進了監牢.
虐室在這一刻由僕人成為了階下囚.
在牢獄裡面39章21至23節.
經文再次給我們看到的是.
他在監獄.
在監獄這個很艱難的處境下.
耶和華仍然跟他一起.
獄長將所有的囚犯交給他管理.
經文特別用的字眼跟他在波提弗家裡其實很相近.
說的是都是很順利.
兩者都是一概不知不理.
完全放手交給虐室.
說的是虐室再次在一個很艱難的處境下.
能夠有神同在.

$^{201}$也在人眼前蒙恩.
不過我們再說.
儘管在獄中有上帝同在.
但虐室這次面對仍然是一個冤獄.
他在這裡沒有大張旗鼓地說要平反.
但不代表他就躺平.
算了 認命吧 不是.
虐室總是等候機會去為自己伸冤.
40章1至8節這裡一大段的經文.
是一個很長的敘事.
需要大家再仔細去看.
不跟大家讀出.
不過說的是虐室是時刻留意周遭的人.
看看究竟有什麼機會他可以反撲.
於是有一天機會來了.
他見到兩位官員的囚友面帶仇用.
於是他主動關心他們.
問他們是什麼原因.
他們說各自都做了一個夢.
當虐室知道他們做夢的時候.
他就知道機會來了.
第8節虐室主動跟他說.
可以自薦為他們解夢.
不過虐室在這裡沒有僭越上帝的榮耀.
他很清楚地跟這兩個官員說.
他自己只是一個中間人的角色.
解夢的仍然是上帝自己.
然後他就分別向他們兩個解夢.
第14-15節解完夢之後.
他跟酒精說了一番話.
他說你得福的時候請你記得我.
向我施慈愛.
在快佬面前提起我.
然後可以救我出監牢.
他如何形容自己呢.
第15節他說我實在是從希伯來人之地被拐來的.
我在這裡有沒有做過什麼.
好叫他們將我關在牢裡.
反映了什麼.
反映了他仍然記得自己的出身在哪裡.

$^{241}$是希伯來地.
強調自己過往被拐賣到埃及.
另外他再一次重申.
他是冤枉的.
他是無辜被人抓來監獄.
在艱難的環境裡.
他沒有什麼可以依靠.
他期望透過幫酒精解夢.
希望他可以轉達這個訊息給法老.
法老代表什麼.
法老就是代表他在外地最高權力的人.
他對法老有盼望.
希望法老可以將他從不公義的環境裡拯救出來.
當我們看到這裡以為弱失點.
正確地解夢.
應該可以立時解決他當下的困局.
不過如果我們這樣想的時候.
經文很快就提醒我們不是這樣的.
有誰會想到酒精會完全忘記了弱失呢.
23節的經文這樣說.
然而施酒長不記得弱失.
竟忘了他.
這裡的動詞特別用了一個形式.
就是一個重複強調的意思.
在說的是什麼.
在說酒精真的完全忘記了.
同樣在跟我們說.
到這一刻弱失沒有希望了.
大概是這個意思.
40節就是一個很悲傷的結尾.
然後41章一節.
我們再看下一章.
我們看中文的書本.
就是過了兩年.
可能就這樣讀了就覺得很快過了兩年.
但原文特意強調是完完整整的兩年過去了.
就像我們廣東話兩年咋和兩年啦.
其實是說後者過了兩年很長的時間.
過了兩年.
埃及最高權力的法老.

$^{281}$發了兩個很困擾他的怪夢.
而剛好酒精在這個時候.
終於想起昔日那個在監牢裡面.
幫他解夢的年輕希伯來人.
想到那個希伯來人有解夢的能力.
因為現實的處境見到就是.
法老的夢令到埃及境內所有最有能力的術士.
所有解夢的智慧人都沒有辦法去解決.
他終於想起藥室了.
法老就是因為酒精於是他召喚藥室.
藥室知道這個是一個翻身的機會.
經文給我們看到第十四節.
他就急忙剃頭刮臉.
換衣服進到法老面前.
好好地去整理好自己.
以示對法老的尊重.
期待法老可以將自己從不公義的牢獄生涯裡面去拯救出來.
藥室在這裡再次去講清楚.
解夢的能力是在於上帝.
祂只是一個中間人傳遞啟示.
特別留意.
我想大家留意.
祂這裡完全只是稱呼神.
祂沒有用耶和華這個字眼.
為何祂不用耶和華呢.
耶和華是上帝特別向祂的子民啟示一個聖若明字.
但是祂跟法老說.
祂沒有說其實是耶和華給我能力.
祂只是用一個很一般的稱呼.
「耶路謙神」這個字去稱呼神.
為甚麼呢.
在避免對本身已經有信仰的法老有冒犯.
我講神你可以認為是你那個神.
但其實我心裡知道解夢的能力是出於我所相信的耶和華給過我的.
在這裡讓我們看到.
在這個時候的約瑟.
比他年少時風黃不露向他哥哥複述兩個夢的時候.
他更加懂得人情世故.
更加懂得看時候看場地去說合宜的話.
他比以前有所成長有所進步.

$^{321}$讓我們看到在埃及的種種經歷.
令他變得更加謹慎.
更加小心自己的言行.
然後緊接著的就是約瑟很清楚地向法老講明.
為甚麼你會做了兩個夢呢.
是因為上帝有些事情將要成就.
並且要指示顯明給相關的人.
即是法老去知道.
重複的夢境就是表示上帝已經很肯定這件事一定會成就.
並且很快地去實現.
不過當約瑟跟法老講這句說話的時候.
就令人想起其實.
昔日約瑟都曾經做過兩個意思一樣的夢.
不過他的夢是等了很久很久才能夠成就.
解夢之後約瑟又給法老一個提議.
33節開始他就開始給法老建議.
再往後的一段經文讓我們看到.
當他解完夢然後給法老建議.
即是要找一個聰明有智慧的人派他去管理這個地.
等等的時候.
法老就立即接納了他的建議.
並且認定那個有智慧的人是誰.
就是約瑟自己.
讓我們看到法老去任用約瑟不是出於恐懼.
不是約瑟恐嚇他.
而是法老看到約瑟是有上帝的靈在他裡面.
法老沒有因為門戶之見或者種族的不同.
於是有所猶疑.
即是不用他.
要用自己人.
沒有.
讓我們看到的是.
這裡.
罪事者在這裡讓我們看到.
法老的全然接納.
我們再對比他自家的兄弟.
其實是一個很諷刺的對比.
法老對他全盤的接納.
但我們看看昔日他的兄弟是怎樣對他.
37章是說.

$^{361}$將要賣他的時候.
就是脫了他的衣服.
然後將他丟在一個坑裡面.
那個坑是沒有水的.
然後當米田的商人經過的時候.
就將他從坑裡拉上來.
拉上來不是為了救他.
而是為了要將他賣給二十萬里人.
然後最終輾轉去到埃及.
但是法老呢.
法老是將一件衣服給過他.
並且除了衣服之外.
還有金鏈等等的東西.
還有金鏈,車等等.
法老也是將他拉出來.
不過法老是將他從牢裡面拉上來.
很諷刺.
自家人對他的排拒.
但是去到埃及.
他是得到法老的全然接納.
法老除了給他很多衣服,金銀,車.
這些都是政治上權力的東西.
更加是為他安排了家室.
將祭司的女兒許配給他.
讓他可以在埃及地建立自己的事業和家庭.
這個家室最後可以很順利地融入埃及人的圈子.
在埃及發展他的事業.
讓我們看到的是.
留山去外地的人.
在他鄉可以生活得很精彩,很豐盛.
這個可能性是存在的.
不過當然了.
藥室的成功既是取決他個人.
如何在這兩種文化裡面.
他如何協調,如何協商這兩者的界線.
同時我們看到的是.
因為埃及對他的接納.
無論在政治,經濟上都是很開放和接納.
以致他有機會去融入和攀升.
當藥室運用上帝給他的技能,智慧.

$^{401}$以致他可以成功帶領埃及去渡過農業上的災難.
七年的封年也就是七年的方年.
在方年的期間.
他成為了埃及甚至是他自己家的拯救和祝福.
值得留意的是.
上帝沒有在以色列地興起一個以色列人.
然後提供以色列人在饑荒時的需要.
相反的是上帝讓藥室留山在外.
去到埃及工作.
最終他可以回饋提供以色列百姓的需要.
留山在外的藥室有上帝和他同在.
透過藥室.
上帝首先賜福給波提佛一家.
再回顧他的經歷.
藥室當然是因為有上帝的同在.
以致他可以在獄中為酒精.
或者在法老面前能夠正確地詮釋這個夢.
值得留意的是.
他在哪裡展示解夢的能力.
夢不只是在埃及出現過.
夢也曾在迦南地出現過.
他自己所發的夢.
但經文沒有說他有解釋過.
他只是敘述給他哥哥聽.
他的解夢能力原來是在埃及地.
為何他會在埃及可以解夢呢.
或者是因為他在埃及的身份.
一個極限的身份.
甚麼是極限的身份呢.
可能大家看完這個字也不知道為甚麼.
即是一種無寧兩可的狀態.
即是怎樣.
廣東話是那種不上不下的狀態.
在埃及的藥室正正就是一個這樣的處境.
他面對的是自己人排他去.
甚至我們再往後仔細看整個敘事的時候.
他哥哥去到埃及買糧的時候.
完全不認得他是希伯來人.
我們不要說他認不認得他是藥室.
他完全不會覺得他是希伯來人.

$^{441}$只是以為他是一個普通的埃及人.
但埃及的人是怎樣看他的.
埃及的人明明白白地稱呼他為希伯來人.
藥室他遠離迦南地.
令到他不只是局限於自己族群給他的束縛.
不過他離開迦南地.
又不代表他完全放棄或忘記自己希伯來人的身份.
為何這樣說.
我們今天沒有列出經文.
但他和他兩個兒子的改名.
他特別改了一個名字.
兩個都是一個希伯來的名字.
代表他仍然認定自己是希伯來人的身份.
他的兒子亦都是希伯來人的身份.
藥室這個外流到埃及地的希伯來人.
因為這個身份令他可以升任上天和人間的中間人.
以致他可以將上帝的啟示和不同人傳遞.
在埃及他被賦權.
然後他就帶著能力帶著power.
去拯救當時缺乏力量的上帝的子民.
上帝就是透過藥室.
在埃及人看起來是一個外人.
但他可以做到很多事.
他做到不單是傳遞上帝的訊息.
更加重要或更加有實際的價值是甚麼.
保存很多的性命.
創世紀去到最後.
當亞各都死了.
他的哥哥很擔心究竟藥室會不會趁他爸爸死了的時候.
要去報仇的時候.
50章19至21節.
藥室反而是安慰他的哥哥.
他跟他們說不用怕.
我怎可以代替上帝呢.
從前你們的意思是要害我.
但神的意思原是好的.
要使許多百姓得以存活.
成就今天的光景.
現在你們不要害怕.
我必養活你們和你們的孩子.

$^{481}$這裡說到是很多的百姓.
能夠得以存活.
好像彷彿暗暗回應著上帝昔日跟阿伯拉罕所納約的應許.
就是要祝福地上的萬族.
這裡讓我們看到的是.
上帝為藥室所安排的人生的歷練.
最終是惠及了埃及和他自己的家鄉.
藥室也為我們示範了一個忠於上帝召命的人.
一個好榜樣.
就是無論他在甚麼的境況.
甚麼的地方.
同時也讓我們看到.
人生的每個階段.
都有上帝為他所安排的一切.
很想跟大家說.
人生的旅途裡總是有很多不同的分岔路.
有很多人權衡過不同的考慮.
或者利害.
才做相應的決定和選擇.
無論是向外流散.
還是在荒謬的環境裡堅持留下.
我想說做這些決定.
第一是會令人覺得很震撼.
又或者是令到身邊的人很糾結.
前兩星期潘Sir提起過.
因而最難走的路.
走一條自己不情願的路.
似乎是我們人生裡必修的一課.
有些人在感情路上遭遇到很多挫折.
又或者有些人在追逐夢想的過程中失敗跌倒.
同樣或者是自身在香港的我們.
可能也是面對著一條又一條很艱難的路.
我想說這些都是不能避免的.
好幾年前.
我自己在侍奉路上都遇到不同的困境和不明白.
詳情可以留意Flow Church的Podcast.
有我的訪問.
不過我想說當我面對著困境和不解的時候.
當下是很難接受.
甚至乎是聲嘶力竭地質問上帝為什麼這樣.

$^{521}$不過這幾年下來.
我開始悟出了一個道理.
就是相信一切是上帝給我最好的安排.
過去的種種經歷或者是現在當下的艱難.
我們不知道為什麼.
甚或我們對未來都是充滿著恐懼.
我們在學習接受對過去或者未來的不理解.
不過我們作為信徒我們可以很肯定的是.
我們知道過去現在將來掌權的那一位是誰.
就是因為我們知道有上帝.
知道我們的上帝是那位全知全能.
以致我們這一班無力的人可以仰望他.
約瑟的經歷正正就是提醒我們.
在這個這麼破碎的世界上帝對我們禱告的回應不一定是迅速的.
約瑟的兄弟將他賣到埃及做奴隸.
他才只有17歲.
但是他最後被釋放的時候.
在第41章46節的經文告訴我們.
他已經30歲了.
讓我們看到困境屈辱的生活約瑟過了整整13年.
想要的結果沒有馬上出現.
不一定是上主討厭我們.
要將我們置諸死地.
或者這個可以是給我們的歷練.
讓我們在過程中可以更加認識他.
而不是懷疑他.
讓我們更懂得如何對這個世界作出更合宜的反應.
儘管我們對前路覺得很不知所措.
但是我相信我們身邊總會有一至兩個隊友吧.
又或者最重要的是有上帝時刻與我們同在.
永遠不會與我們割席.
上帝的同在給我們勇氣去實現很美麗的想像.
讓我們可以有力去撐下去.
讓我們都可以很快樂有動力地生活下去.
總括來說我想說上帝總是讓我在不同的處境.
有所得著和有所學習.
上帝的恩典足夠可以遮蓋我或者是環境的不足.
是呀我們都經歷過迷失,懷疑,挫敗.
但是我們都可以帶著上帝給我們的恩賜或者能力.
讓我們可以嘗試有勇氣地重整旗鼓地出發.

$^{561}$我希望我們可以越戰越勇.
昔日上帝與我們同在的記憶.
應該是有力量的.
假以時日這些我們對上帝的記憶.
就可以叫我們在艱難的道路上慢慢地復前行.
理想,公義或者夢想.
在我們看起來好像為什麼遲遲都還沒到.
但是我相信上帝為我們所安排的不會走數.
我相信即使我們置身在低谷裡.
或者我們處於一個有夢未能夠圓的日子.
上帝對我們有他的安排.
而我相信這個安排是對一切都是最好的.
就好像夢在整個虐室的敘事裡面.
其實是佔一個很重要的角色.
夢既是虐室受苦最起初的原因.
他不發那兩個夢.
他的兄弟就不會買他.
可能只是純粹妒忌他給爸爸的痛愛.
夢亦都曾經為他帶來獲釋放的契機.
他在牢獄裡面正確地為酒精解夢.
展示出他有解夢的能力.
展示出他如何靠著上帝在外地過活.
雖然他遇到一個很善忘的酒精.
等了兩年的日子.
但是法老在兩年之後又做了兩個令他不安的夢.
勾起了酒精的記憶.
最後將虐室帶到法老面前.
虐室的經歷讓我們看到一句說話.
It will be okay in the end.
If it is not okay, it is not the end.
有時候我們真的不用為一時的失利.
去自怨自艾,自暴自棄.
我盼望我們可以一起堅持下去.
我們繼續走在上帝為我們安排的道路上.
今天請敬拜隊為我們分享.
《相信一切是最好的安排》這首歌.
是因為我覺得歌詞好像側面那樣.
在說虐室的經歷.
在剛才歌者為我們所唱的.
我們不單單可以享受到聽覺的享受.

$^{601}$我們也從文字旋律中看到的是.
對將來的憧憬.
現實是黑暗的.
但是真的不用經常強調.
這是你和我早就知道.
並且我們其實處身在裡面.
或者在這個月題最後一講中.
我很希望Flow Church的弟兄姊妹可以陽光一點.
我不是叫大家無視現實種種的需要.
現實仍然是我們需要正視.
但是我們已經浸在負能量的環境裡已經很久了.
我們嘗試向上帝去支取力量.
讓我們的心態可不可以平衡一點呢.
今天特別很想去和那些.
剛剛去了外地.
還沒適應.
還沒融入他鄉生活的弟兄姊妹.
即使很艱難.
我希望你仍然有虐室那一份不認輸不妥協的精神.
咬緊牙關堅持下去.
在上帝為你所安排的處境裡好好地生活.
去敏感一下上帝如何一直與你們同在.
我相信當我們用不同的眼光去看世界上的人和事的時候.
我們所可以看到的風光和層次其實是可以更加廣闊的.
抓緊上帝的應許和盼望的我們.
這一生沒有保證天天平安.
但我相信如果我們抓緊上帝的話我們可以天天心安.
願意此心安處事吾香.
讓我們一同去低頭祈禱.
親愛的主.
現實的環境有很多大大小小的紅線.
令我們卻步.
我們可能已經不自覺地去自我審查.
求你給我們勇氣去反抗.
又求你給我們真理的靈.
叫我們不要把荒謬當作合理或是平常事.
求你也給我們堅信.
你為我們每個人已經度身訂造了不同的經驗和歷練.
我承認當我們在困難裡.
我們仍然真的覺得有不安.

$^{641}$但求主讓我們相信.
這是你為我們最好的安排.
縱然我們這一刻一點也不明白.
一點也不理解.
但主我求你讓我們在不明白不理解裡.
仍然持守對人的良善.
對人的良知.
求你就這樣幫助我們每一個.
祈禱是奉主名求.
阿門.
\newpage



\section{出埃及記 1:15-22-20220305}
\label{sec:oFrw_raeCu8}
\textbf{【網上崇拜】兩位懂得甚麼是「越位」的女生|出埃及記1\_15-22|20220305 [oFrw\_raeCu8]}
\newline
\newline
連結: \href{https://youtube.com/watch?v=oFrw_raeCu8}{\texttt{ https://youtube.com/watch?v=oFrw\_raeCu8}} ~~~~ 語音日期: 2022-03-05 
\newline
\newline
\hyperref[sec:kIOQG9wgqRs]{\small{< < < PREV SERMON < < <}}
~
\hyperref[sec:index_chronic]{\small{[返順時目]}}
~
\hyperref[sec:index_scriptual]{\small{[返順卷目]}}
~
\hyperref[sec:YQgThbN8z0Q]{\small{> > > NEXT SERMON > > >}}
\newline
\newline
出埃及記 1:15-22-20220305
\newline
\begin{longtable}{cl}
\hline
\hline
章節 & 經文 (和合本修訂版)\\
\hline
1:15 & \begin{tabularx}{0.7\textwidth}{X} 埃及王又對希伯來的接生婆,一個名叫施弗拉,另一個名叫普阿的說: \end{tabularx} \\ \\ \relax
1:16 & \begin{tabularx}{0.7\textwidth}{X} 「你們為希伯來婦人接生,臨盆的時候要注意,若是男的,就把他殺了,若是女的,就讓她活。」 \end{tabularx} \\ \\ \relax
1:17 & \begin{tabularx}{0.7\textwidth}{X} 但是接生婆敬畏神,不照埃及王的吩咐去做,卻讓男孩活著。 \end{tabularx} \\ \\ \relax
1:18 & \begin{tabularx}{0.7\textwidth}{X} 埃及王召了接生婆來,對她們說:「你們為甚麼做這事,讓男孩活著呢?」 \end{tabularx} \\ \\ \relax
1:19 & \begin{tabularx}{0.7\textwidth}{X} 接生婆對法老說:「因為希伯來婦人與埃及婦人不同;希伯來婦人健壯,接生婆還沒有到,她們已經生產了。」 \end{tabularx} \\ \\ \relax
1:20 & \begin{tabularx}{0.7\textwidth}{X} 神恩待接生婆;以色列人增多起來,極其強盛。 \end{tabularx} \\ \\ \relax
1:21 & \begin{tabularx}{0.7\textwidth}{X} 接生婆因為敬畏神,神就叫她們成立家室。 \end{tabularx} \\ \\ \relax
1:22 & \begin{tabularx}{0.7\textwidth}{X} 法老吩咐他的眾百姓說:「把所生的每一個男孩都丟到尼羅河裡去,讓所有的女孩存活。」 \end{tabularx} \\ \\
[1ex]
\hline
\hline
\end{longtable}
$^{1}$弟兄姊妹平安 願你平安 願你的身體平安.
知道在這星期 很多福音教的弟兄姊妹都身體患病.
無論你是在家裡 房間裡 還是在臥床裡 參加崇拜.
我們在基督耶穌裡面是在一起的 以同一個聖靈來被招聚.
在戰火瀰漫的世界裡面 在一個疫情滿佈的香港.
我們今天奉主耶穌基督的名來聚會.
因此無論剛才的敬拜到現在的講道.
我們在一起是靠著上帝來一同聚集.
我們在一起來祈禱.
教會的元首基督耶穌 你是我們教會的元首.
你是我們眾流磅肢體的頭.
我們一起來去呼求你.
你是那位拯救世界的救主 黑暗中的光輝.
今天我們一同來到你面前.
我們一起來去呼求禱告.
將我們的身心獻上.
將我們的全人獻上.
我們願意來聆聽在這個年頭 這個時刻.
你對我們每一個人親自的 個別的說話.
幫助孩子 孩子不配.
但是你今天藉著網絡 將你的說話宣告給我們聽.
讓我們能夠得著 知道他怎樣走.
求神你幫助我們.
奉基督的宿命至祈求 阿門.
今天我們會看《出一及記》的第一章.
《出一及記》第一章是整卷書的序言.
第一章的出一及記是沒有摩西的.
但是如果你都發覺整個的聖經已經開始來交代.
摩西時代的社會背景 政治因素.
埃及人和以色列人之間的張力.
當時候的種種文化 經濟 基建所帶來的一些地緣政治的問題.
這個背景是重要的.
因為大家都應該一定要知道.
因為這個是作為摩西興起的一些很重要的政治背景因素.
這個是作為一個引言裡面來交代的事情.
大家都知道《出一及記》的起點是創世紀古時的幾百年後.
聖經描述 埃及有新任的法老王上場.
這個新的法老大大 千秋百世.
新人士新作風 他要將猶太人看為一個問題.
並且作為一個問題來頒佈一些新的國家的政策.

$^{41}$所以你看到反猶這件事.
甚至乎是全國性的反猶.
是已經有幾千年歷史.
不是單單是德國希特勒時代才有.
所以這個法老的大大在他的任期期間.
一連串的實行了幾個不同的政治的政策.
去欺壓猶太人.
今天講到很簡單.
就是我們細心去留意一下.
這個法老的幾個不同的政治政策.
在發現 《出一及記》第一章.
他講了三個階段.
三個不同的反猶政策.
我們看到這三個政策裡面.
目的就是要去解決 去認為解決問題.
並且分成三個階段來實行.
我們先看第一章第八節開始.
我們先這樣講.
他說有不認識猶太人的弱勢的薪王起來.
智利 埃及對他的百姓說.
漢南 即使列民比我們還多.
又比我們強盛.
內巴 我們不如用巧計待他們.
恐怕他們多起來.
日後若遇什麼戰爭的事.
就聯合我們的仇敵攻擊我們.
離開這地去了.
於是埃及人就派督公去管制他們.
加重彈撫害他們.
他們為法老建造兩座積火城.
就是比東和塞塞.
於是越發撫害他們.
他們就越多起來 越發蔓延.
埃及人就因以色列人受煩.
埃及人嚴嚴地使以色列人做工.
使他們因做苦工覺得命苦.
無論是和泥是做磚.
是做各田間各樣的工.
做一切的工上都嚴嚴地待他們.
我見到這個是法老大大一個.

$^{81}$第一個的政策階段.
第一個的措施.
第一個措施是.
首先政府立即宣佈.
即是埃及人裡面的.
即是埃及地裡面的以色列人.
要被降格.
立即成為社會上的次等人口.
他們被剝削他們的經濟自由.
立例被強迫轉型.
成為一些的苦力.
當然這個是一個.
政策上一石二鳥的一個效果.
法老是除了為了要減低.
以色列人的社會上的影響之外.
更加是要去善用他們的勞動力.
成為推動社會經濟的力量.
國家搞好一些經濟.
一些的基建發展.
鋪橋踏路.
大型的工程.
大家可以想像到.
這個政策對以色列人的影響是極大的.
最少如果你是認真去想的時候.
你會想到這個政策.
是導致了不少的以色列民的中小企.
行業店舖老闆受影響.
全國的以色列人都立即要轉行.
所謂以色列人的非合肚就要關閉.
食肆要關門.
美容修教健身室.
全部都要關閉.
或者就算是以色列人的宗教處所.
從事宗教活動的神職人員.
都立即被人捉去做強檢.
即是地方裡面強迫的檢磚頭.
即是甚麼意思.
所以所有的以色列人都被迫從事.
這些大型基建工程.
這個是法老大大第一個.

$^{121}$全國頒布的人口政策.
不過基本上.
法老這個政策是失敗的.
聖經這樣描述.
只是越發苦害他們.
他們就越發多起來.
越發蔓延.
對於這個結果.
你都不是太難明白那個原因.
以色列人被強迫去勞動.
最少你想到.
政策是會令到每個人都大隻了.
健康了 體質好了.
所以從而間接幫助到.
以色列人口大幅增加.
所以法老的第一政策是宣告失敗.
由於法老第一個的政策失敗.
法老就立即修改法令.
埃及政府就頒布第二輪的措施.
聖經這樣說.
十六節.
你們為希伯來府人修身.
看他們臨盆的時候.
若是男孩就怕他殺了.
若是女孩就留他存活.
大家都讀過經文.
可能都沒有認真去想過.
經文背後發生什麼事情.
大家可能覺得.
這是一個很簡單的故事.
你會稱他為兩修身婆的故事.
兩修身婆拯救了以色列嬰兒的故事.
今天我們就嘗試設身處地認真思考.
法老這個法令究竟有什麼意義.
首先我們看看聖經描述.
第十五節.
黃崩說是兩個人稱之為.
以希伯來的修身婆.
首先我們不要再叫他修身婆.
修身婆這字便是他們的地位.

$^{161}$這不是很正確的說法.
首先不要叫他阿婆.
人家不一定是阿婆.
他們修身而已.
這是職業.
但不代表是阿婆.
修身婆不一定是阿婆.
大cum 者都不一定是大班.
大cum 者都是年輕人或男人.
都不奇怪.
所以應該給他多點尊重.
不要叫他修身婆.
我們叫他做什麼好呢.
應該叫他做醫護人員.
確實是.
他們是醫護人員.
以現代版來說.
如果在這個年代的話.
他們就是從事婦產科的醫護人員.
這是第一點.
另外中文聖經這樣稱呼.
這兩位婦產科的醫護.
做希伯來的婦人.
不知道你們對這個的詞語.
有沒有什麼疑惑.
這是不是很大問題.
為什麼這個法律是傻的呢.
是不是要用希伯來的修身婆.
來殺希伯來的嬰兒.
沒什麼可能的.
普京都不會找烏克蘭的人去攻打烏克蘭.
其實意思並不是解作希伯來人的修身婆.
而是為希伯來人修身的人.
他們不一定是希伯來人.
甚至應該不是希伯來人.
聖經特別說他們的名字.
他們的名字不是希伯來人的名字.
這兩個的護士.
一個名字叫斯弗拉.
一個名字叫普亞.

$^{201}$斯弗拉叫普亞.
Sephora 斯弗拉解作燕子的意思.
所以你姑且可以叫燕燕.
或者叫小燕.
普亞這個字大概是解作年輕少女的意思.
所以你姑且可以叫她妹妹.
所以你可以稱呼她做小燕.
燕燕或者叫她妹妹的意思.
聖經記載小燕和妹妹敬畏上帝.
不照埃及王的吩咐而行.
竟然存留男孩的性命.
他們因著敬畏上帝的緣故.
冒著危險.
作出一些公然抗命的行為.
違反掌權者的要求.
在這裡我們看到原來敬畏可以有一個很不同的表現.
敬畏成為了一個巨大的約束力.
叫人能夠勇敢地去堅決去不做一些事情.
甚至是違反掌權者的要求和法令.
不過我們很容易錯過了整件事的細節.
一個很關鍵的問題是.
為什麼這兩個女的醫護人員.
公然違反法律的命令之後.
可以是不用死的.
意思是法老王命令他們要去做事.
他們沒有做又不用死.
最少也要領著告票.
每人要罰一萬元或坐六個月監禁.
但法國是沒有的.
所以我們一起解釋一下.
先看看怎樣看這件事的描述.
法老的命令說法.
有希伯來的.
你們要為希伯來婦人收身.
看她們臨盆的時候.
若是男孩就把她殺了.
若是女孩就留她存活.
其實你會發覺和本是不夠準確的.
看她們臨盆的時候這個字.
原文是一個字.

$^{241}$就是看她們的那個"naim"這個字.
中文其實不是解作臨盆.
這是一個比較異譯.
不是解作臨盆的意思.
如果你看英文的時候.
發現英文是"see them upon their birth stool".
就是看她們在生產椅上怎樣.
其實原文裡面的"naim"字.
是解作"two stones"的意思.
不知道為什麼.
是解作"two stones".
兩個石頭.
其實小學者都不知道這個字是怎樣解.
以下我都嘗試去解釋一下.
有什麼學者的意見.
看看有什麼可能性.
第一個就是一個英文的意思.
就是"naim"的意思.
就是生產椅的意思.
這個就是古代人孕婦生小孩時坐的椅子.
坐著小孩就這樣下來.
所以這些人就發覺.
有沒有解作"two stones"的意思.
是不是解作生產的意思.
這個就是一些推斷.
第二個意思是什麼呢.
就是一些直白.
"two stones"解作什麼呢.
就是解作男生和嬰兒兩邊的東西.
所以發現當你看"two stones"的時候.
就看男生那邊.
如果是男生就殺出去.
大概是這個意思.
不過這兩個的翻譯.
其實都有一定程度的困難.
你會發覺這個方案有些困難.
究竟法律的計劃是什麼呢.
究竟法律要求接生的醫護.
是怎樣來下手去殺人呢.
你知道整個生小孩就是這樣.

$^{281}$首先你看到頭出來.
頭出來之後你慢慢去拿小孩出來.
你才看到他是男還是女.
所以你發覺.
如果你要verify是男是女.
你要將小孩全部拿出來.
才能夠下手.
所以我們問.
究竟整個殺人的process是怎樣.
難道那兩個接生的人.
看到男生就扔到街上嗎.
不可能的.
那些媽媽是不會放過他的.
他都不可以馬上捏死他.
其實是很難下手的.
所以我就去看看.
這個"Ablaim"的第三個解釋.
歷史上"Ablaim"的字.
在整個的舊裡面出現了兩次.
一次就在《刷牌記》.
一次就在《耶利米蘇》的十八章第三節.
就在這個有關搖像和圖例的比喻裡面.
他說 我就下到這個搖像的家裡.
正如他轉輪做器皿.
"Ablaim to stone".
就是這塊東西.
就是在玩圖例的那塊磚盤.
那塊東西.
和懷孕有什麼關係呢.
如果你看回古代埃及的文化研究.
你會發現.
原來古代埃及的神明.
他們認為.
你看看圖片就明白了.
他們認為什麼.
那些神明是用圖像裡面來做人.
人就好像一些圖像.
轉一轉就轉了一個人出來.
當你發現這些古代埃及文明的圖畫的時候.
你會發現.

$^{321}$Ablaim這個解作圖像的轉輪.
其實是和懷孕有很大關係.
你就明白.
原來叫你去看一下圖像的轉輪.
其實就是解作婦科檢查的意思.
原來就是說.
法老命令這兩個婦人.
當你去幫希伯來的太太去做婦科檢查的時候.
如果你發現是兒子的話.
你就和我放了她.
如果是女兒的話就放了她.
我就明白發生什麼事了.
原來法老第二個的政策是.
要求這班婦科醫護人員.
當為他們去為希伯來的孕婦做檢查的時候.
當孩子還沒出生的時候.
當這班婦科醫護知道是男的話.
就馬上和他殺了他.
簡單來說這是一個墮胎的政策.
原來法老是說.
當以色列的男嬰還沒出生的時候.
就暗暗地把他打了.
一旦希伯來的媽媽把孩子生出來.
這個政策就失效了.
政權也沒辦法去做到.
起碼比當時的政策多.
所以法老的第二個政策是.
簡單來說稱之為靜態清零.
什麼叫靜態清零.
清零很明顯.
法老就是要把所有以色列的男性清零.
不過這個清零是靜態的.
他要靜悄悄地做的.
如果是去到第二個政策為止.
法老是要殺了這些男嬰.
都要暗暗地串通醫護人員幫忙.
在內部出一個指令.
不可以公開出來.
雖然有這個想法.
但還是不能去到太盡.

$^{361}$所以我猜今天大家對這些事情都不陌生.
政權可能會暗暗地去做一些事情.
去改變一些事情.
不過無論如何.
整件事的關鍵仍然在於.
這兩個前線的醫護人員.
燕燕和妹妹.
他們決定不遵從法老的吩咐去做.
或者更加準確來說.
他們二人是很淑靈地巧妙地合法地.
避開了這個政權的惡法.
同時間他們又可以明目張膽地.
在法老面前解釋.
最後沒事.
這就是好事.
他們怎樣做呢.
他們給了什麼答案法老.
讓他們能夠安全地找到一個微小的客房.
在罵人當中生存.
剛才這樣寫到17節.
埃及王趙廖修琛陀來說.
你們為什麼做者是存留南下的性命呢.
修琛陀就這樣回答法老.
因為希伯來婦人和埃及婦人不同.
希伯來婦人本是健壯的.
我們說沒有道 他們已經生殘了.
可以說這兩個姐妹在法老面前說謊.
如果你搜尋YouTube.
找一條唐崇煥牧師的片子.
一個比較保守正氣傳統的華牧師.
會解釋為什麼這兩個姐妹會說謊的原因.
他們說這個說謊是淑靈的.
是可以的.
是神批准的說謊.
不過我認為他們兩個可以沒有說謊.
他們沒有必要說謊.
只要他們趕不及去希伯來婦人面前.
就已經可以了.
不需要說謊真的趕不及去.
他們有沒有違反法老命令呢.

$^{401}$沒有.
他們只不過說自己打算去.
不過對不起.
我遲了.
哎呀 關了.
那我沒有辦法.
下次可不可以快一點.
幫你幫你幫你.
我盡力了.
幫你幫你.
不過星期六休息.
我不看WhatsApp.
你快點叫他.
喂喂.
走了.
明明可以這樣敷衍.
這個就是敷衍.
敷衍 扮公 射波.
這是大公仔基本的本能.
教會曾經都會問一個問題.
基督徒可不可以敷衍.
你們說呢.
基督徒可不可以敷衍人.
基督徒做事要殷勤.
基督徒要盡心盡力.
做好事情不可以敷衍.
所以基督徒可不可以敷衍人呢.
我以前都覺得不可以.
任何一些負面的事情.
都不可以在基督徒身上發生.
不過直到幾年前.
我見到這幅圖就開始改觀.
這幅圖是幾年前.
我在網上見到的一幅圖.
就是教宗對著特朗普.
特朗普的時候黑面的圖.
我發覺原來這幅圖.
給我一個啟發.
原來教宗可以黑面.
對於一些你覺得不義的事情.

$^{441}$你覺得不對的事情.
你就可以黑面.
這叫做屬靈黑面.
我明白原來有些本質不好的事情.
都可以有屬靈的意義和價值.
我們今天才發現.
或者世上有一種美善.
叫做屬靈地敷衍.
當然特朗普時期敷衍是不對的.
對於一些壞事.
一些邪惡的事情.
違背自己良心的事情.
你敷衍了它.
這件事其實是屬靈的.
這兩位接生的女醫護.
他們沒辦法不去遵從法老的命令.
他們唯有選擇敷衍.
因為他們敬畏上帝.
他們選擇敷衍了法老.
幾年前我聽一個華人教會.
一個老前輩的故事.
話說幾十年前.
回歸前.
當時中共在北京召開了一些宗教會議.
秘密地邀請一班香港各大宗教的領袖.
北上去參加.
學習交流.
這位宗教領袖.
一位很多人都敬重的屬靈前輩.
都收到這個通知.
收到這個邀請.
要去坐飛機上去北京.
然後這位屬靈的前輩.
在起機的早上.
拿著行李坐電車.
然後在電車裡睡著了.
然後錯過了一班飛機.
上不到飛機就去不了.
對不起我睡著了.
我竟然在電車裡睡著了.

$^{481}$你明不明白.
原來有些事情我們是可以這樣去做.
你有沒有誠信.
你是不是這樣做人.
你有沒有正義.
但面對著今天的世代.
當這個世界越來越亂.
當世界的政權不斷越過他們的界線.
侵犯人類的生命人權.
侵犯人類的自由.
侵犯我們宗教信仰的自由的時候.
事情就變得越來越複雜.
我們要思考的地方就越來越多.
我們所做的回應.
也要好好地去思考.
這個月Full Church的題目的月題.
是叫做月位.
我今天不嘗試去解釋什麼是月位.
不過算了今天也解釋一下.
所謂月位.
就是進攻球員傳球的時候.
前面的進攻球員的身體.
被防守球員的身體.
是不可以前過他的.
如果前過的話就叫做月位.
我都盡力了希望你明白.
不明白就看你明白.
月位從來都是具爭議性的.
你可能也知道.
月位從來都是具爭議性的.
當然所謂月位是一種錯誤.
一種犯規.
一個不被容許的事情.
不過你發現一個球員是否月位.
永遠都是充滿著很多的爭拗.
就算現在有VAR.
你可以用慢鏡去重播.
用電腦去計算量度也好.
仍然是充滿著很多很多爭議性的.
你發現一個球是否月位.

$^{521}$是很視乎什麼.
很視乎怎樣畫一條線.
怎樣畫一條線.
這些全部今天我畫圖給大家看.
就是看看這條線怎樣畫.
是否守月位.
還是那個西樓高月位.
還是這樣.
所以很多人仍然是對.
怎樣界定那條線.
是有很多很多不同的看法.
不過更重要的是.
即使世界上有月位這回事也好.
但是每一場球賽裡面.
足球比賽對伍的雙方.
仍然是不斷不斷不斷不斷去嘗試.
去突破這個月位.
不斷在分秒的差號之間.
嘗試去走多一步.
走多一步.
走前多一步.
嘗試去突破這個月位陷阱.
今天我們在短片裡面.
見到兩個嘗試月位的女性.
她們敬畏上帝.
尊重生命.
在法老的掌權之下.
嘗試去走一條月位的路.
違反法老的政策.
從修生改為放生.
越過自己被人定義的界限.
當然這兩個醫護可以選擇.
Play safe.
即是安全安全穩穩陣陣.
遵守法老的政策.
反正我也是奉公守法.
即是甚麼.
他們可以這樣想.
但是他們沒有.
他們沒有Play safe.

$^{561}$他們沒有選擇安全地.
安於本份地被政權去定義.
他們覺得安全的位置.
你知道這樣做.
在當時.
這個決定反而是光明正大的.
是合法的.
順目掌權者.
不過知道這個其實.
也可以叫做平庸的惡.
當政權變邪惡.
你安安全全穩穩陣陣.
遵守政策.
奉公守法.
他叫你做甚麼就做甚麼.
他叫你不要做甚麼就不要做甚麼.
這個其實是一種平庸的惡.
當政權自己都越位的時候.
跟隨他的人.
就算是Play safe.
他都已經是越位了.
辛辛強調這兩個接生的女性.
敬畏上帝.
他們清楚的知道.
上主真正的界線在哪裡.
他知道.
上面很多牛鬼蛇神.
在那邊這樣那樣.
說你做錯了.
說你沒有跟足做.
說你不聽話.
但他們知道.
真正越位的界線在哪裡.
真正的越位線是上主來定的.
他們懷著敬畏的心.
緊緊的來執著上主的越位線.
勇敢地踏前一步.
突破政權定義的越位陷阱.
我們見過很多這樣的故事.
你只需要在書中聽到最後一章.

$^{601}$出入的第二章.
摩西的出生.
又是另一班越位的人做的事情.
幾十年之後.
摩西大以色列人出埃及.
又是一個越位的故事.
因此第一章作為整個出埃及記的序言.
是要告訴我們出埃及的精神.
出埃及的精神就是一個越位的精神.
不斷地跨越界線.
不斷地越過難關.
不斷地解決問題.
不斷地尋求上主的美善.
我們的救主耶穌基督也是一樣.
你以為馬祖很合法嗎.
你以為馬祖是羅馬政府官方的接班處所嗎.
你以為馬祖有足夠醫護人員嗎.
上帝差遣去獨身愛子耶穌基督降生.
本來就是一個越位的過程.
上帝越過自己的本位.
來到世上從黑暗去拯救我們.
他本有上帝的形象.
不以自己與神同等為強奪.
反倒虛己.
取老奴僕的形象成為人的樣式.
靜姐妹.
我們要越過這個平庸的惡.
在這個沒有和平.
沒有安寧.
似乎沒有將來的世代.
我們要越過這種平庸的惡.
積極勇敢敬畏地去面對未來的日子.
第一章最後一節.
聖經記載.
惡兆終於來臨.
法老吩咐他的眾民.
絕.
以涉人所生的難孩.
你們都要刁在河裡.
第三波的政策.

$^{641}$即將強制推行.
不再是經濟的奴役.
不再是暗地裡的墮胎政策.
而是真真正正的種族滅絕.
以色列人將要面臨第三波.
今天我們沒有時間去講第二章.
怎樣講以色列人去面對第三波.
不過一樣的.
以色列人依然的敬畏上帝.
呼求上帝.
在邪惡的世代裡.
越過自己的本位.
去拯救生命.
在不正常的時間.
離開自己的英分和本位.
去嘗試互相保衛.
和平不是從天降臨.
美善也不是亂世裡default的狀態.
你要不斷不斷去跨越.
甚至被人說你是越位.
去嘗試去建造這個美善.
或者我們有一天.
可能會面對相處的處境.
不過我們有很多先賢先聖.
成為我們這個越位的榜樣.
在二戰的時候.
當納粹政權去屠殺猶太人的時候.
很多人暗地裡拯救.
隱藏.
收埋這些猶太人.
他們教導我們.
怎樣可以跨越我們自己應有的界限.
過去的十年裡.
當中國大陸禁止基督教的時候.
內地家庭教會他們教導我們.
怎樣可以越過很多的界限.
依然的去聚會.
他們依然去懷著敬畏的心.
捉緊上帝 敬畏上帝的底線.
越過一次又一次的難阻.

$^{681}$前幾天在網上.
看到一幅很令我觸動的圖畫.
就是一個在烏克蘭基輔裡的神父.
帶著一班弟兄姊妹.
在基輔的防空洞裡.
帶領弟兄姊妹一起去主持彌撒.
在一個暗暗的燈光底下.
他們擺了他們能夠擺到的東西.
一個天花板都不夠高的一個大姐頭.
一班弟兄姊妹.
應該說幾個弟兄姊妹.
一起的來到去聚會.
他們奉主的名.
在戰火的當中.
依然做他們要做的事情.
他說他們的聚會一直都沒有停止過.
一直的來到去堅守.
沒有停止過去探訪.
沒有停止過去關懷一些難民.
每個星期仍然如常.
他們沒有針紙.
沒有戴口罩.
在處所裡都沒有安心出行.
他們依然的來到去聚會.
弟兄姊妹.
我們仍然可以在這裡聚會.
因此我們今天在這個空間裡.
作為Full Church.
我們一班的弟兄姊妹.
我們在今天裡.
我們一起來去聚集.
我們一起來去成為一個禱告的祭壇.
來為這個世界去代禱.
因此我們現在很立刻有一個特禱會的時候.
無論你在家裡.
你在臥床裡.
世界不同的地方裡.
我們成為一個禱告的祭壇.
去為我們世界去祈禱.
發揮我們教會的職能.

$^{721}$第一個事情.
很想的來為患病的弟兄姊妹和家人去祈禱.
知道這個星期裡收到很多很多不同的小組.
弟兄姊妹的患病.
小組的消息.
或者是他們家人.
一些老人家在醫院裡的消息.
我們都知道是很不容易的.
聽到這些消息越來越多的時候.
就好像打仗一樣.
一個倒下之後.
到另一個.
剩下來的人繼續向前.
我們一起祈禱好不好.
為到我們認識的弟兄姊妹.
如果你是小組的話.
為到你認識的弟兄姊妹祈禱.
如果你是患病的.
為到你祈禱.
求主醫治你.
求主讓你的病毒能夠清零.
能夠去健康的恢復.
如果你的家人有病的時候.
我們都為到他去祈禱.
我們一段祈禱的時間.
我們呼出教會的元首耶穌基督.
我們為到我們身體軟弱的弟兄姊妹.
獻上禱告.
他們都不幸的染上疫情.
他們喉嚨痛.
他們很不舒服.
他們很辛苦.
或者更加傷害的是他們的擔心.
擔心自己會傳染人.
傳染他們的親人.
求主你醫治他們.
讓他們在病床裡都能夠有很好的休息.
有很好的照顧.
讓他們每天在這段患病日子裡.
能夠去依靠你.

$^{761}$讓他們在有一個隔離的空間裡.
仍然可以有機會來祈禱你.
為到他們的家人我們更加去獻上禱告.
一些小朋友.
一些老人家.
我們都很擔心他們用染疫之後.
他們的身體狀況.
主我們求你特別去看顧他們.
特別是在他們患病送院之後.
很難有音訊.
很難有很好的跟進.
求主你幫助保守大靈.
唯有你在當中去連結我們眾志體.
將我們每一個人都能夠交託給你.
為到我們其他還沒有染疫的弟兄姊妹.
我們去求你保護.
保守大靈.
讓他們能夠免於這個瘟疫.
求主你保守.
奉主命求 阿門.
\newpage



\section{彼得前書 2:18-25-20220312}
\label{sec:YQgThbN8z0Q}
\textbf{【網上崇拜】基督都愛越位|彼得前書2\_18-25|20220312 [YQgThbN8z0Q]}
\newline
\newline
連結: \href{https://youtube.com/watch?v=YQgThbN8z0Q}{\texttt{ https://youtube.com/watch?v=YQgThbN8z0Q}} ~~~~ 語音日期: 2022-03-12 
\newline
\newline
\hyperref[sec:oFrw_raeCu8]{\small{< < < PREV SERMON < < <}}
~
\hyperref[sec:index_chronic]{\small{[返順時目]}}
~
\hyperref[sec:index_scriptual]{\small{[返順卷目]}}
~
\hyperref[sec:3bu5V6aPJC0]{\small{> > > NEXT SERMON > > >}}
\newline
\newline
彼得前書 2:18-25-20220312
\newline
\begin{longtable}{cl}
\hline
\hline
章節 & 經文 (和合本修訂版)\\
\hline
2:18 & \begin{tabularx}{0.7\textwidth}{X} 你們作奴僕的,凡事要存敬畏的心順服主人;不但順服善良溫和的,就是乖僻的也要順服。 \end{tabularx} \\ \\ \relax
2:19 & \begin{tabularx}{0.7\textwidth}{X} 倘若你們為使良心對得起神,忍受冤屈的痛苦,這是可讚許的。 \end{tabularx} \\ \\ \relax
2:20 & \begin{tabularx}{0.7\textwidth}{X} 你們若因犯罪受責打而忍耐,有甚麼可稱讚的呢?但你們若因行善受苦而忍耐,這在神看來是可讚許的。 \end{tabularx} \\ \\ \relax
2:21 & \begin{tabularx}{0.7\textwidth}{X} 你們蒙召就是為此,因為基督也為你們受過苦,給你們留下榜樣,為要使你們跟隨他的腳蹤。 \end{tabularx} \\ \\ \relax
2:22 & \begin{tabularx}{0.7\textwidth}{X} 「他並沒有犯罪,口裡也沒有詭詐。」 \end{tabularx} \\ \\ \relax
2:23 & \begin{tabularx}{0.7\textwidth}{X} 他被辱罵不還口,受害也不說威嚇的話,只將自己交託給公義的審判者。 \end{tabularx} \\ \\ \relax
2:24 & \begin{tabularx}{0.7\textwidth}{X} 他被掛在木頭上,親身擔當了我們的罪,使我們既然在罪上死,就得以在義上活。因他受的鞭傷,你們得了醫治。 \end{tabularx} \\ \\ \relax
2:25 & \begin{tabularx}{0.7\textwidth}{X} 你們從前好像迷路的羊,如今卻歸回你們靈魂的牧人和監督了。 \end{tabularx} \\ \\
[1ex]
\hline
\hline
\end{longtable}
$^{1}$在前幾天聽了一個故事.
有一個媽媽回家的時候很晚.
即是十一,二點.
她的子女們很想黏著她.
所以遲遲不肯睡覺.
等到媽媽回家的時候.
媽媽脫了口罩.
準備洗澡的時候.
那對子女跟她說了一句話.
她說:媽媽,為甚麼你的口罩印這麼深?.
怎麼不脫掉?.
媽媽心裡不知道怎樣回應,怎樣回答.
到之後那天早上.
媽媽一早就收拾行裝準備出發.
子女知道媽媽要出發的時候.
隨眼星鬆地跑到媽媽面前.
看著媽媽一眼.
她問媽媽一句話.
她說:媽媽,為甚麼昨天那些口罩印.
經過一晚都好像沒有脫掉?.
你可以想像這個媽媽是做我們很尊敬的.
香港醫護界其中一員.
所以希望在這個疫情的時候.
我們仍然尊敬,深深尊敬.
在不同崗位裡堅守著的每一個香港人.
香港的醫護仍然值得我們尊敬.
香港的醫護仍然值得很多人.
附上很多很多對他的敬愛.
無論疫情如何.
我相信我們每一個人都在面對.
在這兩三個星期裡很多複雜的事情.
無論是俄羅斯侵佔烏克蘭.
或者香港疫情突然間.
每天的數字過於慢計.
你身邊的朋友很多人都已經.
有很多不同的方法面對疫情.
我相信每一個人都在這個時候心情極度複雜.
做父母的你知道在這個星期開始.
子女開始沒有課.
又是另一個令人糟糕的事情發展中.

$^{41}$香港或許真的會面對很多很難的事情.
多謝光才敬拜隊給我們一個很安靜的時間.
讓我們的心靈能夠沉澱下來.
讓我們眼目再一次看回我們的主耶穌他自己.
心願在香港這個時候每一個香港人.
你心裡無論想什麼事情都好.
藉著這間教會的崇拜.
無論是海外還是姐妹.
你仍然為這個地方去禱告和紀念.
求上帝的愛在這個時候彰顯更多更多.
讓那個不一樣的經歷可以呈現在我們當中.
今天我們會看一段經文.
前書的二章十八至二十五節.
經文是這樣說的.
它說你們做僕人的要凡事敬畏主人.
它說不單是對著善良和溫和的.
就是乖僻的你都要信服他.
這句話其實是很難去理解和明白.
我們今天希望能夠解釋這句話想說什麼.
什麼叫做對著那些乖僻.
對著那些古靈精怪的主人.
你都要信服他.
其實他想說什麼事情.
他再讀下去的時候.
因為人若在神面前為良心的緣故.
忍受屈辱的苦楚是有福的.
你們若因為犯罪受責打而忍耐.
有什麼可誇.
但你們因行善而受苦.
能忍耐在神面前看是有福的.
我們再下一章.
然後經文說什麼.
為什麼你們要對著古怪的老闆的時候.
你要信服他.
他說你們為此蒙照.
因為他說了一個故事令我們很難理解.
基督也是這樣.
因為基督也為你們受過苦.
靠你們留下榜樣.
叫你們跟從他的宗行.

$^{81}$然後開始說以塞瓦舒的經文.
他說他從來沒有犯過罪.
口裡也沒有任何的詭詐.
他閉上嘴不還嘴.
受苦的時候不說恐嚇的話.
只怕自己交託過公義的審判者.
然後這些話全部都是來自唱書53章.
他牧牲上親身擔當了我們的罪.
所以我們既然不活在罪中.
就可以為義而活.
他說因他受的鞭傷我們得醫治.
他說你們從前好像迷路一樣.
但現在回到你們靈魂的牧人和監督那裡去.
這段經文難理解或難處理的地方是.
為什麼要對著一些古靈精怪的老闆.
或者我們職場裡面也遇到很多.
卑鄙無恥的老闆.
不近人情的人.
為什麼我們要忍耐.
為什麼要啞忍這些東西.
這段經文好像說.
因為耶穌基督第一受自駕都是這樣.
用以塞瓦書的體會和經驗去說.
受苦的牧人的經驗.
其實問題關鍵是什麼.
我們可以看看這段經文.
這段經文是否教我們啞忍.
我想問的問題是.
面對一些很不合情理的老闆.
或者那些做主人的人做得很差的時候.
我們所謂的行善受苦是什麼意思.
我們是否要學耶穌基督一樣.
在彼拉多治下受苦.
然後釘十字架死埋葬下陰間.
啞忍就可以了.
或者這段經文其實是在說啞忍的問題.
然後我們就靜靜地行善受苦.
無論受什麼苦都好.
繼續啞忍下去就OK了.
因為耶穌基督釘十字架的時候都是這樣.

$^{121}$今集我們希望花點氣力時間.
看看這段經文其實想說什麼.
尤其是對今天的香港社會來說.
或者對世界裡所發生的事情來說.
其實這段經文想和我們說一件什麼事.
或者我們首先要處理的是.
這段經文為什麼要用這麼多以賽亞書53章.
我們看一看下一章.
我們再看下一章.
它用的是以賽亞書53章的聖經.
以賽亞書53章這個「棟人之歌」.
其實很多時候都是在復活節用的.
就是說耶穌釘十字架死.
雖然我們很理解耶穌很默然承受一切人的罪.
將他自己擔當一切所有的苦難.
但在底下我們都理解和明白.
但我希望今天能夠有一個角度讓我們重新看一看.
其實以賽亞書裡面所說的「棟人之歌」.
其實為什麼要這樣說.
當然我們說的複雜一點的話.
在第二以賽亞.
就是40至55章以賽亞書裡面.
它有四首的「棟人之歌」.
那些「棟人之歌」放在第二以賽亞書上說什麼.
雖然我們沒什麼時間去理解這個這麼大體的圖畫.
但希望今天能夠藉著一點點讓我們看一下.
其實「棟人之歌」放在以賽亞書裡面想說什麼.
以致我們理解彼得用以賽亞書的時候.
說僕人和主人的關係的時候.
他為什麼要用耶穌「棟人之歌」這些比喻.
放在主人和僕人的關係上.
好幫我們去理解一下.
到底什麼叫「行善受苦」.
那個信服的意義到底是什麼.
我們再拿一張.
其實「棟人之歌」代表什麼.
其實應該是代表在猶太人秘魯到巴比倫的時候.
他去到不同的地方.
僕人受到很大的逼迫.
我們可以說更加講得更加真實的是.

$^{161}$其實當猶太人秘魯到巴比倫的時候.
他遇到很多宗教的領袖和很多先知.
被巴比倫打壓甚至囚禁甚至殺死.
這些故事不陌生.
所以基本上可能「倚創教書」沒興趣.
講很多事件出來.
譬如現在烏克蘭的戰爭一樣.
多少人在過程中受苦受難處.
基本上這些資訊在我們每天的裡面很多很多.
所以「倚創教」沒興趣將逐件逐件的微小的歷史事件鋪陳出來.
但我們可以相信的是.
在那個時代秘魯後.
很多猶太人正在經歷外邦人肆意的殺戮.
和對他們有很多的逼迫.
所以他說什麼叫「常經憂患」.
什麼叫「承認擔當一切的罪孽」.
其實就是在講以色列人先祖裡面所犯的一切的事.
秘魯後這班猶太人默然承受這一切的事情.
如果這樣說的話.
受苦獨人這個觀念.
其實對於以色列人來說從來都不陌生.
但奇怪的地方是.
我們今天想看一下這個受苦的經驗.
或者這個受苦的想法.
其實是什麼來的.
我們看一下.
其實這些受苦的經驗是來自《出埃及記》.
不過我們很輕輕地說一下《生命記》的事.
我們再留下一章.
這章太複雜了我們不討論了.
如果你看一下《生命記》和《出埃及記》的時候.
它說什麼呢.
如果你留意聽耶和華你的聲音.
行他眼中為證的事的時候.
則以聽他的誡命.
遵守他一切的律例.
我決不加在埃及人身上.
一切的疾病在你身上.
因為我是醫治你的耶和華.
《出埃及記》很奇怪.

$^{201}$它說在十災裡.
這個就是埃及裡.
法老和他全家所經歷的一切疾病.
你可以想像長子之死.
他說耶和華親自加在仇敵上.
使他有疾病.
所以在《生命記》裡他說什麼呢.
他說必是一切的病症離開你.
你所知埃及各種的惡疾.
決不加在你身上.
並加在一切恨你的人身上.
所以你可以想像.
在《出埃及記》的故事裡.
有疾病是仇敵有疾病.
埃及人有疾病.
法老全家有疾病.
一切敵黨神的外族人.
都會經歷上帝賜下的疾病給他們.
但剛才我們讀經文的時候.
很特別.
我們再回到下一章的時候.
你看回《以賽雅書》.
其實不是這樣說的.
他說原來那些僕人.
擔當了我們的病患.
背負了我們的痛苦.
我們以為他受責打被神.
擊打苦袋了.
這些擊打和苦袋好像什麼.
好像上帝在擊打埃及的法老.
不過這次不同.
《以賽雅書》不同的地方是.
這個擊打令患病的是誰.
是上主的僕人.
是他的僕人.
在《以賽雅二書》.
《第二智慧的修行》的僕人之歌裡.
每個人都被神擊打.
每個人都被神苦袋.
然後他說.

$^{241}$為了我們的罪孽被壓傷.
所以我們得平安的懲罰加在他身上.
因他受的鞭傷我們得醫治.
《以賽雅二書》呈現了另一幅圖畫.
原來上帝的僕人都會受到患難.
以前在《初一及記》裡.
我們常常想著的就是.
那些實在是那些埃及那些壞的人.
壞的人會受到災害.
但其實在《以賽雅二書》裡的時候.
那幅圖畫已經不再一樣.
不是那些在欺壓巴比倫人.
或者那些馬代波斯的人.
受上帝給他的疾病與痛苦.
倒過來.
那幅圖畫剛剛掉轉.
是上主的僕人竟然受一切疾病與痛苦.
受鞭傷受擊打受苦袋.
但奇怪的是.
原來上主的僕人受擊打受苦袋.
是使我們得醫治.
這個觀念其實我們很難理解.
或者這個觀念更加實在地說的是.
我們不容易掌握這是一個什麼樣的觀念.
在最近烏克蘭俄羅斯的入侵當中.
我們會明白到多一點.
其實《以賽雅二書》說的僕人之歌.
裡面的受苦是說什麼.
其實我們看到烏克蘭的裡面.
有很多地方被炸.
很多人妻離子散.
很多平民百姓受到死亡的威脅.
甚至犧牲了性命.
我們以為上帝都好像責罰他們.
苦待他們欺壓他們.
或者上帝好像沒有出手.
但如果你仔細看這十幾天.
即將到來的二十天.
你會發現人性仍然可以是美善的.
當很多人勸烏克蘭總統離開的時候.

$^{281}$都打不贏的時候.
他和他全家都仍然守在基輔.
他不是說不用生命賠償去做這個決定.
他知道俄羅斯很多特工會走來將他暗殺.
他知道這一切的事.
但有一個人可以堅信.
一場人家都看起來好像不會打贏的仗.
他仍然堅持.
你看他無論在英國國會講的說話.
或者昨天他那段的分享.
你發現原來人不只是為自己去講說話.
人不只是可以很自私地為自己的利益.
為自己的名聲.
為自己的土地可以再遼闊一些.
人其實可以放下自己更多.
去看上帝在做什麼事情.
你會看到陸陸續續有很多人支援.
甚至回到烏克蘭這個地方.
和那些人並肩一起作戰.
最近在網上流傳更加令人興奮的是.
連Airbnb都可以幫助烏克蘭的居民.
有很多人特意在基輔附近租Airbnb.
但他們不會去.
他們有經濟上的幫助.
很多很多這些事情.
讓我們覺得這些人的苦難.
這些人的被擊打苦待.
其實好像在我們心靈上得到一些安慰.
原來在整個聖經裡面.
除了壞人上帝親自懲罰洗滓餅的女生身上.
是彰顯上帝的榮耀和他的精彩的時候.
原來在《伊賽亞書》.
他在說另一幅圖畫.
同樣地見證著上帝的美善.
見證著上帝的作為仍然可以在我們當中.
原來就是他這人受苦.
就是他這人仍然願意.
不是為了自己的緣故去承擔不同的事情.
耶穌聽受自駕不是默然無聲的聽受自駕完.
其實耶穌知道他死了復活升天以後.

$^{321}$聖靈降臨下來的時候.
會有不一樣的圖畫出現.
原來默然的受苦不是啞忍.
原來默然的受苦是在說上帝會做一些更加奇妙的事情.
更加人想像不到的事情.
呈現在這個世代裡面.
讓這個世代的人看到不同的事情.
耶穌可以將所有疾病淋在他們那一代的人身上.
淋在羅馬人身上.
淋在猶太的領袖聖殿裡面.
高明釣魚杜猛岸然那些人身上.
耶穌不做這些事情.
耶穌做的事情是超越越過我們對苦難的想法和想像.
親愛的丙姐妹.
留在香港的我們這一班人.
不是說我們不知道我們有些什麼問題.
留在這裡的香港人.
不是不明白不理解我們默然受苦的是些什麼.
我們也會理解明白知道我們發生什麼事.
真正的我們留在這個地方的.
不是因為我們在啞忍現在我們忍受的一切.
我們突然間的手機會響緊急告示.
提醒我們QE是一個疫情的醫院.
我們對這些事情來說的時候.
我們心裡想的是.
我們忍了它接受了它.
這些叫信服.
但經文說的信服關於在以賽雅書關於耶穌身上的信服.
不是說對著那些乖僻的人說Amen.
信服的是上帝可以用人在苦難裡面所承受的一切.
超越我們的想法.
而上帝可以在超越的裡面.
創建很多更美善的事情走出來.
最近在娛樂界裡面.
其實我現在也不知道什麼叫娛樂界.
最近明哥入了榜.
Selina入了榜.
不是說ViuTV更加不是說CCTVB.
是說有一班音樂人.
將主流的音樂和非主流的音樂.

$^{361}$一起搞了一個選舉比賽.
聽聞搞的只是一班大專生.
那班大專生在他面對著這個世代很困難的時候.
他不是想著單單所謂的信服所謂的忍耐受苦.
他還在信在這個艱難的裡面.
有些事情可以再創造出來.
有些事情可以再不一樣的出來.
今天你和我都在面對著.
好像很難 很複雜.
很多困局的局面.
但我想鼓勵自己.
鼓勵大家的是.
耶穌基督能夠跨越.
耶穌基督能夠善勝.
因為人世間對苦難的定義他看不一樣.
他看苦難是默然忍受完之後.
苦難是可以成就更多更精彩的人生走出來的.
還是我們今天來到上帝的面前.
忍受一切的時候.
純粹是因為要忍耐下去.
純粹覺得沒有辦法.
而我們不想像更多.
可以超越我們現在這個空間裡面.
可以做更多的事情.
剛才說到樂壇.
除了明哥入選之外.
達哥最近竟然玩新興貨幣.
他將自己的歌在上面去賣.
那些虛擬貨幣可能對很多人來說都不知道是甚麼.
但很多人已經進入一個新的環境裡面.
繼續做他想做的事情.
今天如果耶穌基督默然死亡釘十字架死.
其實最終這個默然是要換來聖靈降臨福音遍傳的話.
我們要問自己的是.
這一刻我們來到上帝面前默然忍受.
為的是要見到甚麼.
那個默然忍耐忍受.
是幫我們去找上帝一個更新的創意.
可以在這個世代裡面做更多不同的事情.
我在時代論壇寫了一篇文章.

$^{401}$好像是上個星期.
我就說既然有突圓其來.
即是提早暑假投奔初夏的時候.
為甚麼基督教不能協作地做很多不同的事情出來.
在這六七個星期裡面.
為那些小朋友,為那些中學生,為那些家庭,為那些父母.
有另類的事情.
寫完這篇文章之後.
我在網上見到更多的是.
不知道是哪個師父.
他說他設計了一個星期的課程.
教那些小朋友煮菜.
不知道是哪個店舖.
又不知道教授甚麼銷售方法.
讓小朋友學習甚麼叫做金錢的運用.
突然之間見到在這個secular的世界裡面.
有很多人集體地為了六七個星期.
為香港人做了很多不同的事情.
今日作為殉道的你與我.
在一個默然忍受的時候.
我們是默默地忍受,然後沒有事做.
還是我們期待著這個默然忍受.
是要帶來上主更奇妙的作為.
讓我們這一代人見到.
如果今世之子.
用Airbnb來幫助烏克蘭.
很多補習天王,天后.
可以立即有六七個星期的課程.
讓不同的人立即上.
我們的問題要問.
上主的作為.
如何藉著他自己的子民們.
在這個世代裡面呈現.
這個島是我們值得問的問題.
如果耶穌越過一切人對苦難的詮釋的時候.
會帶來福音改變.
這二千年裡面.
凡信他釘十字架的人的生命就會扭轉改變.
這個榜樣成為你跟我一起學習的榜樣.
可不可以在這個世代裡面.

$^{441}$屬於上主的子民們.
讓人可以用這段聖經來講我們應該有的身份.
讓人見到我們擔當了一切的患難,痛苦.
好像被擊打,被苦待.
但眾人的生命得著意志.
苦難的階段見到甚麼.
我最近見到教會是.
一說教會要close down的時候.
很多教會裡面有不同的聲音出現.
誰沒有打針的同工要在三個月內打針.
否則不讓他上班.
我們突然之間想起很多.
好像要默然忍受下裡面要做很多事.
但我們沒有跳出應該要跳出的界線.
愉悅去看得更清楚.
在這些默然忍受的裡面.
上主的作為可以怎樣使用我們.
更加不一樣.
所以我在這裡呼籲的是.
希望更多教會,更多機構,更多事工.
可以摒棄很多門戶之見.
彼此協作.
可以做一番景象出來.
讓世人驚訝.
求天父親自憐憫香港的教會.
默然忍受不是永遠是一個passive mode.
默然忍受是說.
人看到我們堅持做的事.
看到尾線.
今天教會輸了甚麼.
就是讓人看不到.
在默然忍受的裡面.
教會有甚麼尾線走出來.
那個醫護的媽媽.
一年七日.
十幾個小時.
在醫院裡面工作.
雖然默然忍受.
她的子女們心心慶幸的是甚麼.
她媽媽所做的一切.

$^{481}$彰顯著人的尾線在當中.
心願教會在這些日子裡.
不是只問教會何時close down.
何時work out.
教會還可以在這個時代.
再做些甚麼出來.
讓這個課題成為我們.
繼續思考並賦予行動的課題.
我們一起祈禱.
天父你真是很奇妙.
你把耶穌基督釘十字架.
死下陰間復活升天.
你做這一切的事.
只是把你的愛之灼金釘十字架.
完了就算了.
你很知道.
這一切為著人的生命被拯救.
為著見證著甚麼叫做.
上主的尾線仍然存留於人間.
天父我求你.
雖然邪惡的事情我們不能逆轉.
或者我們不知道可以怎樣逆轉.
但天父我求的是.
在這些邪惡的彰顯和呈現的時候.
你給基督徒們.
和耶穌基督一樣.
我們受鞭傷.
人的生命得依次.
付出給我們的代價.
呈現出來的時候.
讓人看到基督那份美好.
在香港一個這麼困難的群體裡面.
求天父你繼續.
在前面好像很困難的時候.
在香港的教會堅持.
也為烏克蘭的教會禱告.
我相信上星期我們在敬拜的時候.
聽了一首詩歌.
那班弟兄姊妹正正已經被.
俄軍今日的消息全部圍困在城市.

$^{521}$天父我不知道這班基督徒.
現在生死如何.
但我真的求的是.
既然苦難邪惡離不開.
你使用基督徒.
超越他本能裡面可以做的事.
人性裡面覺得應該應該是怎樣.
但天父你提升他們的生命.
在這麼艱難的日子.
他們呈現了不一樣的圖畫.
讓烏克蘭的人民見證的是.
這班基督徒不只是口傳福音.
是用生命去見證.
甚麼是福音的大能.
願上主你自己的僕人.
繼續忍受一切.
卻換來你美好的旨意的呈現.
求你憐憫.
求你大大的工作.
將烏克蘭每一個人民再次恭敬放在你面前.
求你的靈親自與他們同在.
為到香港醫會界繼續禱告仰望.
求你保守他們脫離一切一切.
更多更多的勞苦.
讓他們可以盡快回復正常的工作狀態.
但我求的是縱然在困難的日子當中.
你額外的恩惠恩典.
尚與他們同在.
保守他們的出 保守他們的入.
在你的恩典的庇護下.
平平安安.
求主你聽我們在你面前的祈禱.
奉耶穌基督你寶貴的名字而求.
阿門.
\newpage



\section{羅馬書 15:14-21-20220319}
\label{sec:3bu5V6aPJC0}
\textbf{【網上聖餐崇拜】越位人生|羅馬書15\_14-21|20220319 [3bu5V6aPJC0]}
\newline
\newline
連結: \href{https://youtube.com/watch?v=3bu5V6aPJC0}{\texttt{ https://youtube.com/watch?v=3bu5V6aPJC0}} ~~~~ 語音日期: 2022-03-19 
\newline
\newline
\hyperref[sec:YQgThbN8z0Q]{\small{< < < PREV SERMON < < <}}
~
\hyperref[sec:index_chronic]{\small{[返順時目]}}
~
\hyperref[sec:index_scriptual]{\small{[返順卷目]}}
~
\hyperref[sec:0d9n3K2nnYY]{\small{> > > NEXT SERMON > > >}}
\newline
\newline
羅馬書 15:14-21-20220319
\newline
\begin{longtable}{cl}
\hline
\hline
章節 & 經文 (和合本修訂版)\\
\hline
15:14 & \begin{tabularx}{0.7\textwidth}{X} 我的弟兄們,我本人也深信你們自己充滿良善,有各種豐富的知識,也能彼此勸戒。 \end{tabularx} \\ \\ \relax
15:15 & \begin{tabularx}{0.7\textwidth}{X} 但我更大膽寫信給你們,是要在一些事上提醒你們,我因神所賜我的恩, \end{tabularx} \\ \\ \relax
15:16 & \begin{tabularx}{0.7\textwidth}{X} 使我為外邦人作基督耶穌的僕役,作神福音的祭司,使所獻上的外邦人因著聖靈成為聖潔,可蒙悅納。 \end{tabularx} \\ \\ \relax
15:17 & \begin{tabularx}{0.7\textwidth}{X} 所以,有關神面前的事奉,我在基督耶穌裡是有可誇的。 \end{tabularx} \\ \\ \relax
15:18 & \begin{tabularx}{0.7\textwidth}{X} 除了基督藉我做的那些事,我甚麼都不敢提,只提他藉我的言語作為,用神蹟奇事的能力,並神的靈的能力,使外邦人順服; \end{tabularx} \\ \\ \relax
15:19 & \begin{tabularx}{0.7\textwidth}{X} 甚至我從耶路撒冷,直轉到以利哩古,到處傳了基督的福音。 \end{tabularx} \\ \\ \relax
15:20 & \begin{tabularx}{0.7\textwidth}{X} 這樣,我立了志向,不在基督的名已經傳揚過的地方傳福音,免得建造在別人的根基上; \end{tabularx} \\ \\ \relax
15:21 & \begin{tabularx}{0.7\textwidth}{X} 卻如經上所記:「未曾傳給他們的,他們必看見;未曾聽見過的事,他們要明白。」 \end{tabularx} \\ \\ \relax
15:22 & \begin{tabularx}{0.7\textwidth}{X} 因此我多次被攔阻,不能到你們那裡去。 \end{tabularx} \\ \\ \relax
15:23 & \begin{tabularx}{0.7\textwidth}{X} 但如今,在這一帶再沒有可傳的地方,而且這許多年來,我迫切想去你們那裡, \end{tabularx} \\ \\ \relax
15:24 & \begin{tabularx}{0.7\textwidth}{X} 盼望到西班牙去的時候經過,得見你們,先與你們彼此交往,心裡稍得滿足,然後蒙你們為我送行。 \end{tabularx} \\ \\ \relax
15:25 & \begin{tabularx}{0.7\textwidth}{X} 但如今我要到耶路撒冷去,供應聖徒的需要。 \end{tabularx} \\ \\ \relax
15:26 & \begin{tabularx}{0.7\textwidth}{X} 因為馬其頓和亞該亞人樂意湊出一些捐款給耶路撒冷聖徒中的窮人。 \end{tabularx} \\ \\ \relax
15:27 & \begin{tabularx}{0.7\textwidth}{X} 這固然是他們樂意的,其實也算是所欠的債;因為外邦人既然分享了他們靈性上的好處,就當把肉體上的需用供給他們。 \end{tabularx} \\ \\ \relax
15:28 & \begin{tabularx}{0.7\textwidth}{X} 等我辦完了這事,把這筆捐款交付給他們,我就要路過你們那裡,到西班牙去。 \end{tabularx} \\ \\ \relax
15:29 & \begin{tabularx}{0.7\textwidth}{X} 我也知道去你們那裡的時候,我將帶著基督豐盛的恩典去。 \end{tabularx} \\ \\ \relax
15:30 & \begin{tabularx}{0.7\textwidth}{X} 弟兄們,我藉著我們的主耶穌基督,又藉著聖靈的愛,勸你們與我一同竭力為我祈求神, \end{tabularx} \\ \\ \relax
15:31 & \begin{tabularx}{0.7\textwidth}{X} 使我脫離在猶太不順從的人,也讓我在耶路撒冷的事奉可蒙聖徒悅納, \end{tabularx} \\ \\ \relax
15:32 & \begin{tabularx}{0.7\textwidth}{X} 並使我照著神的旨意歡歡喜喜地到你們那裡,與你們同得安息。 \end{tabularx} \\ \\ \relax
15:33 & \begin{tabularx}{0.7\textwidth}{X} 願賜平安的神與你們眾人同在。阿們! \end{tabularx} \\ \\
[1ex]
\hline
\hline
\end{longtable}
$^{1}$定治妹平安.
很開心在網上再跟大家一起崇拜.
今天的講題叫做月位人生.
我相信你也不難想像到.
特別是今天的講題內容是來自羅馬書.
應該是關於保羅的.
其實在1,2月的月題當中.
Outflow在我第二講的時候.
也用彼得這個人來跟大家總結一下.
我對於Outflow流出.
彼得的生命向外流動.
成為多人的祝福.
在第二個月題.
月位的月題當中.
講人物.
我很希望先用保羅.
他自己本身很多破格的行為.
和他的行動.
都改變了很多身邊的人.
今天羅馬書給大家一個新的看法.
也是關於大家的事情.
在羅馬書我發覺自己的講章中.
也有不少關於羅馬書的經文成為我的講章.
我自己也很喜歡羅馬書.
其中一個原因就是.
我覺得它很實用.
可能對於羅馬書來說很難.
很多教義或者一些terms很難明白.
或者好像很深.
要專人去解答.
當然總是可以很深入地去了解羅馬書.
但羅馬書對於受書的人.
要帶出的訊息.
或者保羅當時要讓受書的人明白的內容.
其實也可以從保羅的宣教旅程.
和他的心意當中去了解.
其實對於我們今天來說.
羅馬書是什麼呢.
會不會在當中因為我們過去的.
宗派或者教導上.

$^{41}$會主導了我們對羅馬書的看法.
但在新的一個連題.
或者新的一個訊息中.
我們再想想.
在過去你對羅馬書很快就想起.
這個人就是馬丁路德.
馬丁路德事實上在羅馬書上.
有很大的對他的啟發.
成為他在基督新教開展的時候.
一個很重要的leading sentence.
特別他在羅馬書的領授.
和在他自己的熟齡經驗中.
對他來說是一個很大的提醒.
其中羅馬書一章十六和十七節中.
好像投影片中你會看到.
對馬丁路德來說是一個很重要的轉捩點.
因為他透過這兩次經文.
他再一次成為他信仰的宣告.
第一章十六節是這樣說的.
「我不以福音為始.
這福音本是上帝的大能.
要叫一切相信的.
先是猶太人後是希臘人.
因為上帝的義正在這福音上顯明出來.
這義是本於信以至於信.
如經上所記 義人必因信得生.
But the righteous man shall live by faith」.
在當中這句話正正就是啟發了.
或者是光照了馬丁路德.
過去天主教他想要推動.
想要人去執行那種行善.
積德那種以換取上帝的恩寵.
這個想法.
馬丁路德說不是這樣.
他要超越了.
或者對於天主教來說.
這個是越位.
那條線過了位.
但馬丁路德從羅馬書.
他看到上帝給他的光照是什麼.

$^{81}$其實不是我們要做多少事情.
去換取上帝的恩典.
或者上帝的恩寵.
而是上帝以先破格離了的世界.
去展現了何謂恩典.
何謂人不能夠靠功德所做的工作.
來換取上帝的救贖.
而是單單去憑真誠.
去相信耶穌基督為我們所成就的一切.
以至我們稱義.
這個在保羅寫《爾弗所書》的時候.
都再一次表明.
我們是得救是本乎因也因著信.
這個馬丁路德他.
因信稱義要表達的信息更加會表明.
就是Justification by Grace through Faith.
意思是什麼.
意思就是藉著信.
我們得到恩典.
以至稱義.
是因為信的緣故.
而信對於我們來說是一個很破格的.
因為約翰·芬德說過.
從來都沒有人見過上帝只有父懷女獨生子.
將祂表明出來.
正正就是耶穌在地上的工作.
展現何謂天國.
何謂上帝.
何謂信上帝.
對於這件事是很不容易的.
但是路德再一次要高舉這件事情.
不是天主教所說的行善積德.
得蒙恩寵.
不是口傳那件事.
就是仕途統帥.
以至可以跟從.
不是人傳的.
而是聖經光照啟示我們.
再一次藉著信.
去得蒙稱義.

$^{121}$這個對於馬丁路德來說.
我們要破除舊有的所謂格局.
以至這件事我們要愉悅.
可以看到不一樣的人生.
七位弟兄姐妹也是.
路德當然是展現了給我們看到.
但羅馬書不只是說行為和信心的弔詭.
但是對於上帝來說是可以並存的事情.
羅馬書在篇章上.
保羅其實更加要表明.
信是需要的.
因為他在突破猶太人所說的格局.
所以在羅馬書的篇章中.
我之前在港獨也說過.
你會看到在第一章至十一章中.
他也是在說一些因信心而產生的果效.
以至福音對於人的改變.
但其實到了十二章和十五章.
有四章的內容.
整本羅馬書十六章.
但有四章的內容是說行為上.
如何與各個福音的意義相稱.
信心而有的行為.
其實行為是很重要的.
因為你的果子是從你的行為表現出來.
你的信心同樣也是從你的行為表現出來.
我喜歡羅馬書其中一個很重要的原因就是.
不是我們靠口說的.
我們口說的時候很容易就變成了唇舌.
就是說就無敵.
但做的時候你也會說.
無能為力.
但我們的行徑上.
正正就是表明我們信仰實在之處.
保羅是希望受輸的猶太人的基督徒的信徒.
在行為上要做多一點.
和做多一點.
做多一點的意思就是.
離開他本身的限制.
去超越了他覺得要有的本位.

$^{161}$而保羅是在書信中展現他自己的身份而有的.
到了今天.
經文的十五節.
你會看到就是結語.
保羅要交代之先.
在經卷當中他說最終章.
通常在結語的時候.
就不是在阿奧山場.
很快就飛了去.
但其實我們看電影也知道.
凡是尾聲結局的時候.
其實已經是綜合了全書.
或者整套戲最精髓的說話.
在今天要看的經文中.
你會看到第十五章第十四節.
我們讀出吧.
十五章第十四節.
弟兄們,我自己也深信你們是滿有良善.
充足了諸般的知識.
也能彼此勸誡.
但我稍微放膽寫信給你們.
是要提醒你們的記性.
特因上帝所給我的恩典.
使我成為外邦人.
作基督耶穌的僕役.
作上帝福音的祭司.
叫所獻上的外邦人.
因著聖靈成為聖傑.
可夢月立.
先看第十四和第十五節的經文.
裡面說什麼呢.
他開頭就是.
中文翻譯應該可以再親切一點.
用第一身去說.
屬於大家的共同身份.
而英文裡也說.
他做一個平衡.
I myself, you yourself.
我自己和你們.
要做一個平衡的意思就是.

$^{201}$其實過去你會看到.
保羅寫了五卷書信.
都是他建立的教會.
他寫給他們.
比如《鐵索羅家前後書》.
《迦太史》.
很多前後書.
五卷書信都是保羅建立的教會.
但羅馬的教會不是保羅建立的.
所以其實有不同人質疑.
保羅用什麼權柄.
要寫一封信給他.
我為什麼要聽你的話呢.
又或者我為什麼要聽你的話.
內容是真實呢.
其實有不同身份上的挑戰.
對於保羅來說.
但保羅不是沒有權柄.
對我們來說我們清楚.
但對於受輸的人來說.
他不是一下子接受.
但保羅就先跟他說.
我就不是拿我的頭牌.
或者我的身份.
我的位份出來和你搭理.
他是I myself, you yourself.
用一個同等的方法去說.
就是彼此勸戒.
才不會有一個大前提.
但這句話對於.
近來或者我們認識.
現在的環境來說.
就不是的.
但保羅就說.
就不是的.
有些事你會發覺.
現在說是事實的話.
但因為你的背景.
或者你的立場.
同一句說話就不同解讀了.

$^{241}$在於我們現在坊間的用語.
就是立場先行.
無論哪句說話就不同人說.
就不同看法.
或者不同結果.
其實這件事是不是今天才出現呢.
又未必的.
因為每次都會扯上一些爭議性的問題.
就好像以前.
我自己還是年輕的時候.
你總會穿一樣的衣服.
同一個人.
穿同一款衣服.
為什麼他穿得好看一點呢.
因為他帥一點.
你會發覺帥哥.
說話都好像特別醒神一點.
你的樣子平平凡凡.
穿衣服又沒有什麼特別標籤.
就好像比下去.
簡單來說.
帥哥穿件內衣都比你帥.
有時人就好容易有前設.
有很多事先入為主.
但這件事我們已經分開.
如果那是事實.
還是意見.
在過程當中我們可以做分辨.
回到保羅想說的.
他很希望我們有充足知識去做分辨.
而我們本於一個良善.
就是不要先入為主.
不要好像你不喜歡那個人.
或者那個人和你的立場不同的時候就定格.
但我們從一個良善的身份去欣賞對方.
大家平起平坐去處理的事情.
最重要就是我們可以做到彼此勸戒.
今天你會發覺有些人.
如果有批評.
或者有些意見不合的時候.

$^{281}$你都不難聽到.
選擇不聽.
選擇不看.
和選擇不理會.
但這件事很容易就將自己畫在荒島裡.
就定格了.
要他越位其實很難.
當初沒有人跟你說話.
其實不是當初.
其實是慢慢積累了.
保羅很希望猶太的基督徒.
在羅馬的基督徒.
他不要太快定格.
還記得我還在熙唐的時候.
兩篇道都是說關於耶路撒冷教會和安提拉教會.
令到濟密的事情.
就是那個異象.
就是雅各都想去調停一件事情.
就是不要太快定格.
去看看上帝的工作.
而不要太快定格.
其實上帝在過程當中不是在說吃不吃濟生的問題.
是能不能聆聽上帝在當中的工作.
我們在教會裡面總會有不同的合作.
我們在教會裡面總會有不同的意見.
但是在你的知識.
你的見識的範圍之外.
應該要加一個良善.
欣賞.
大家都有一起付出.
大家都一起去貢獻.
但是我們能不能夠彼此勸戒呢.
不是要搬弄一些排頭.
經驗.
慰訓.
能不能夠有一個一起去放膽呢.
所以第十五節.
我就稍微放膽寫信給你們.
不是要騙你.
我和你分享.

$^{321}$提醒什麼呢.
提醒你一件很重要的事.
其實上帝一直都在做事.
有時候我們就太快.
就覺得自以為是.
而看不到上帝的恩所在.
所以去到第十六節的時候.
也是他尾句要收.
他要說的這個訊息就是.
要提醒當初我們怎樣去接納人.
所以什麼是上帝恩典呢.
其實十五章之前.
我講閃避球那篇訊息的時候.
都保留在結尾之先.
其實要重提一個訊息.
就是怎樣去接納對方.
所以接納其實也不僅僅是我們自己接納.
是上帝先接納我.
我也提過幾次.
關於Nabhanu這個字就是接納.
但是後面說的如基督接納我.
Plus Nabhanu就是.
是上帝先接納我們.
以至我們得到一個恩典.
我們就在當中去學習接納人.
保羅在第八和第九節.
就是我們今天選取經文前面一點點.
就是在講那個真國禮是什麼回事.
真理使我們成性.
而前文的國禮已經過去.
這個也是羅馬書早前在講的內容.
是上帝連恤外邦人.
讓我們彼此去欣賞.
彼此去接納.
所以保羅很強調一件事就是.
信是很重要.
但是我們行動是多走一步去接納人.
親弟兄姊妹.
在我們遇到親弟兄姊妹.
或者同事或者我們周遭的人.

$^{361}$有些不咬弦或者意見不合的時候.
我們會不會主動多走一步呢.
會不會主動去想一些東西呢.
不容易的.
因為很多時候你會聽到一個對話就是.
喂 他碰到我的底線.
他踩了我的線.
但是那條線在當中對於你來說.
是你自己固有固步自封設定了.
我們有沒有與時去想一想.
在那條線當中有沒有其他水位可以了解呢.
我不是叫你退後.
我又不是叫你改.
但是在你做過程當中去接納.
有沒有一個主動去做反省呢.
最怕就是我選擇不聽.
我選擇不理.
我選擇不看.
這件事就什麼都不能改.
所以去到第十六節的時候.
保羅說.
使我為外邦人作基督耶穌的瀑易.
作上帝福音的祭司.
叫所獻上的外邦人.
因著聖靈成為聖傑.
可無越立.
這裡就真的說回.
其實我說這麼多的原因是什麼呢.
就是其實我有這個身份.
這個身份不是我自己來的.
是上帝呼召我來的.
他用的用詞對於當時受輸的人來說.
是很容易去掌握的.
作基督耶穌的瀑易.
Minister.
這個Minister的意思是什麼呢.
就是一個祭司.
對於猶太人來說.
祭司是什麼他們很清楚.
但是他很清楚.

$^{401}$是我做耶穌基督的祭司.
而這個作上帝福音的祭司.
其實就讓人去明白到.
是上帝先憐憫外邦人.
我只不過是做一個中介.
而這個中介是在做上帝福音的工作.
而聖靈就是令到他們可以成性清義.
可無越立.
其實保羅所做的事情.
所說的事情.
正正是希望受輸的弟兄姊妹.
其實我們在做.
特別我是在做上帝要我做祭司中介的角色.
去到下面的時候.
這件事是加插和大家去了解的.
《司徒幸傳》第21章裡面第27節是這樣說的.
那七日將完.
從亞細亞來的猶太人看見保羅在殿裡.
就慫恿了眾人下手拿他.
喊叫說:伊斯蘭人來幫助.
這就是在各處教訓眾人.
糟踐我們百姓和律法.
並且地方的.
現在又帶著希臘人進殿.
污穢了這聖地.
這裡說的是.
保羅他真的在做一個中介.
他真的帶一些希臘人去到聖殿.
他讓人明白到.
上帝是喜悅外邦人.
而他真的身體力行破格地越位.
外邦人是不能夠進入聖殿的.
但是他履行他做一個.
上帝福音的中介的祭司.
他帶領外邦人去到聖殿去敬拜.
但是猶太人就過不了這件事.
他看不到上帝的工作.
他反而看到他要嚴守口.
一路賴以為生或賴以為所依.
那種律法的精神的精義.

$^{441}$於是他拿手要抓保羅.
這件事是何時呢?.
《思量傳》有一章的記載.
大概是羅馬書成書一後一年.
保羅真的身體力行.
如他在書上所說的.
我不是口說的.
我真的要帶一些外邦人.
是上帝閱立的外邦人.
到上帝殿中去敬拜.
但不是寫了書就一定是每個人都接受的.
有些人仍然是故佈.
但保羅這個越位的人生.
他很清楚的呼召.
很清楚的要做.
就算你不喜歡.
但我要做上帝給我應該做的工作.
你可能說他硬頸.
但他清楚自己的線是什麼呢?.
就是要將猶太人和外邦人.
在上帝眼中的那條線拆掉.
我們要超越這件事.
結束的時候他這樣說.
所以論到上帝的事.
我在基督耶穌裡有可誇的.
除了基督藉我做的那些事.
我什麼都不敢提.
只提他藉我言語作為.
用神跡歧視的能力.
並聖靈的能力使外邦人信服.
甚至我從耶路撒冷直轉到.
以利利古到處傳耀基督的福音.
這裡說的是什麼呢?.
什麼是上帝的事呢?.
就是上帝的工作開展和廣傳.
一切都是宣揚耶穌就是基督.
就是這個世界的救世主.
基督真的藉著保羅在當中做不同的事情.
所以當中有三樣東西.
第一樣東西就是保羅的教導.

$^{481}$言語作為.
保羅的神跡歧視.
讓人明白到基督的大能.
第三樣東西就是聖靈的感動和催逼.
都是使外邦人信服.
在這三樣東西當中.
在保羅的書信和教導當中.
你都不難看到.
這就是上帝在他兩次到保羅準備第三次宣教旅程當中.
保羅在結束的時候讓他們明白.
其實上帝不斷都透過我的服事去做他的工作.
我們就在當中看看.
從耶路撒冷猜出他的工作.
能夠成就上帝的福音.
但是我們不得不想想.
為什麼那些人這麼固執.
或者這麼不想接受呢?.
都不是沒有原因的.
就像我們自己.
我們很難離開自己的本位去走出去.
或者那條線很緊.
緊到覺得自己很難跳過.
都不是沒有原因的.
回頭看回一點點的歷史.
在羅馬政權開展的時候.
我們很多時候看羅馬書都在看羅馬的政權.
對於猶太人的逼迫.
但是在早於羅馬政權開始.
猶太人的歷史就是.
主前586年亡國.
其實猶太很多貴族或者重要的人都被擄.
國家破滅.
在過程當中.
沒有了國家.
他們經歷過巴比倫.
經歷過波斯.
經歷過希臘.
去到現在的羅馬.
其實不斷被不同的大國去統治.
去瓦解.

$^{521}$沒有了自己的國家.
其實他們要反省一件事.
其實是不是我們沒有遵守上帝的律法.
以致我們來到這樣的地方呢?.
這個觀念都是很生命記的觀念.
很符合生命記的歷史觀.
所以對他們來說.
能夠再堅守自己.
就是所.
重複自傳那種.
就是律法.
對於我們來說.
是一個可以對上帝.
表示那種忠誠.
求上帝憐憫的一個很重要的憑據.
對於他們來說.
守律法是很重要.
是他們的身份的創造者.
令到他們自己可以堅守.
今天保羅跟他們說要拆解這件事的時候.
其實真的不容易.
就算我們自己也有我們自己的所信.
我們要做一個扭轉.
要做一個離開.
其實都要有很大的動力才能做到.
所以去到羅馬政權的時候.
要做一個挑戰.
是不斷地讓他們明白到.
其實福音的能力去到哪裡呢?.
保羅要帶出一個訊息就是.
其實不是再看地上.
或者是我們密守的那個律法.
其實耶穌常常都在我們的宣教旅程當中做新事.
我們就看看其實上帝當中發生了什麼.
所以保羅就在說一件事就是.
我立了志向.
不在基督的名被稱過的地方傳福音.
免得建造在別人的根基上.
就如經上所記.
未曾聞之他的信息的將要看見.

$^{561}$未曾聽過的將要明白.
什麼是我立了志向呢?.
對於保羅來說.
他立了什麼志呢?.
其實由他羅馬書的第一章一節已經說得清楚.
耶穌基督的僕人保羅奉召為使徒.
特派傳上帝的福音.
什麼是福音呢?.
其實福音這個字在當時來說.
不是基督徒的尊用.
我相信你之前聽我們Folk Church講道都講過福音這個字.
Eugeneion這個字.
Eug就是好意思.
Eugeneion就是信息.
所以福音就是好信息.
其實好信息在當時羅馬或者一些大國的宣傳口號來說.
其實是很普遍的.
好信息.
譬如一個新的皇帝上場.
他頒下一個豐年.
是一個好信息.
譬如一個皇帝可能有轉位.
有很多事情納了豐功偉績的時候.
就傳了一個和平的信息給人.
所以福音對於當時來說.
不是基督教獨有的.
但是保羅要立一個志向.
就是我蒙召要讓更加多人知道.
是上帝的好信息.
所以保羅是另一個穴位.
不是說我要基督徒去.
應該這樣說.
我不是單單要猶太裔基督徒和外邦人基督徒去做一個復和.
去做一個接納.
我更加要讓你明白到.
這是一個過程.
這個過程去見證就是.
上帝的福音會傳遍給還沒認識上帝的人.
這個就是上帝給世人的好信息.
是真的一個很重要的信息.

$^{601}$所以在羅馬書第十四章裡面.
論到那種飲食的過程當中.
保羅已經不厭其煩.
已經和他們說.
我不是和你說飲食的問題.
因為上帝的國不在乎吃喝.
是在乎公義和平並聖靈.
這三樣東西.
所以上帝的福音.
上帝的好信息要傳給其他人.
不是靠羅馬帝國給到好信息來.
是上帝會將公義和平和聖靈.
即是上帝的能力.
去告訴你這是好信息.
對於還沒認識上帝的人來說.
是一個很重要的表白.
所以保羅希望羅馬教會的信眾.
想遠一點.
不要只想自己.
想遠一點.
想到周遭.
羅馬帝國那麼大.
其實上帝的版圖更大.
想遠一點.
我們就應該要秉承.
將上帝的好信息.
要傳遍給更大的版圖的人.
想遠一點.
就是逾越你自己本身視覺的界線.
今天我們面對最大的問題.
好像我們沒有太多盼望.
今天我們面對最大的問題.
就是我們比較定格.
看到可能每天的過活都不容易.
我知道.
但是.
信仰常常都真的要提醒我們.
我們每一天都是新的.
我們的眼界.
是要擴展.

$^{641}$不是說我做得到.
我每天.
我自己每天.
你都聽我說過了.
有很多事情.
有很多事都不是即時一天解決到.
我真的每天起床照鏡的時候.
我都拍拍自己.
盡能力.
按自己可以做到的事.
做足.
做好每一天.
這個都是.
讓我更加看到.
上帝會透過很多時.
都讓我們有新的眼界.
新的穴位.
看到的事情.
保羅在結束的時候說了一句話.
就如經上所記.
未曾聞之他的信息將要看見.
未曾聽過的將要明白.
這句經文是什麼?.
這句經文就是.
二賽亞書的僕人之歌的經文.
在54章22章裡面都會提到.
在當中讓更加多人明白.
上帝的兒子.
他成為上帝真正的僕人.
而我們是他的跟隨者.
我們就將上帝僕人的好信息.
傳遞出去.
我們就去蒙照去做這個工作.
其實到我們今天.
我們怎樣去將信息傳遞呢?.
對我們來說.
每個人都不同.
但是我們真的希望有些事情.
慢慢一點一滴.
逐少逐少去做.

$^{681}$在3月.
4月是我們月位這個月題.
但是在3月當中.
我們分別都有不同月位的事情.
不是推拿月位.
即是突破了我們的東西.
因為我們當中想.
如果3月香港真的lockdown.
禁足出不了街.
怎樣搞呢?.
對我來說最大的難處就是.
我平時約吃飯的人約不到.
沒得做.
於是就想不如在家裡吃飯.
就約人定時吃飯.
同時也讓弟兄姊妹去知道.
You are not alone.
We are together.
我們用不同方式.
都可以繼續吃飯.
繼續去分享.
而那個分享不是單單.
我和你之間.
但其實我和你之間的生命故事.
都會成為別人的前後腳的祝福.
又或者可能原來.
其他人聽過.
原來你都是這樣.
我都是這樣的時候.
大家都知道.
We are not alone.
We are together.
這個是很重要的.
可能全日都很困苦.
但晚上想鬆一鬆的時候.
我們又有一些活動.
是希望大家彼此可以留個言.
彼此有個同分,同享.
星期一,三,五晚上23點59分開始.
就有《你沒有對著空氣》這個新節目.

$^{721}$讓你自己心中的說話.
可以記情.
可以和別人分享.
星期二,四有一些真誠的對話.
那個生命的觸動.
生命的見證.
都是彼此去做一個參考.
去到星期六.
我們由疫情開始.
我們知道教會很多時候都會停聚會.
停了崇拜以外.
小組又不是經常有.
我們就在崇拜之後.
有一個大型的團契.
不知不覺就有幾十到八十多.
甚至以前去過200個弟兄姊妹.
和我們一起實時在網上.
有一個大型團契.
其實這個都不是在實體可以做到.
我們幾何有一個團契有二百多人一起.
唱歌,分享.
但網上就讓我們離開我們的本位.
可以做多一點的事.
三月對我們來說.
我們呼應了月題.
就真是想突破我們既有的框架.
突破我們既有的事情.
我們多走一步.
離開固有的思考.
可以接觸更加多弟兄姊妹.
弟兄姊妹.
從來都不是停在腦袋.
有很多行動都可以參與.
你可以將一些內容轉載.
你可以留言給我們.
你可以參與我們的留言聽歌.
這個都是我們可以離開你自己本身的安書.
對於你來說.
你會不會畫了一個圈呢.
而你畫的圈會不會越畫越細呢.

$^{761}$我希望你不要不聽,不理,不看.
我希望你不要越畫越細.
但你願意走開你既定的位置.
或者去八卦一下我們發生的事情.
我們盡可能讓你明白到.
其實我們的信仰生命.
就好像保羅在羅馬書結語一樣.
他很希望猶太的基督徒在羅馬教會.
他們會看到更加多上帝的工作.
就是未聽過的將會聽到.
未見到的都會見到.
但聽過之後就會明白上帝的工作.
心願每一位弟兄姊妹.
我們在三,四月的日子.
透過不同的信息.
我們每人做一點點.
每人給自己多一點點空間.
其實你不知不覺就越過了.
可能過去已經很死的死線.
但上帝讓我們看見更多的作為.
我們一起祈禱.
天主上求你開我們的眼界.
看到我們要走的路.
在走的過程當中我們可能很膽怯.
或者覺得我們不行.
但求上帝讓我們有信心.
就好像當日你猜探保羅.
行步見步看到上帝很多的作為.
就算他不時要被下監的時候.
他仍然高聲歌唱讚美神.
因為他知道上帝與他同在.
求主你加添我們信心.
更加看到你要我們同在.
看到很多生命見證彼此的激勵.
看到很多弟兄姊妹的留言彼此分享.
就知道我們不孤單.
我們不甘在我們自己的區分.
我們願意走出.
享受上帝給我們越位的人生.
正如保羅在結束羅馬書的時候.

$^{801}$再一次提到.
僕人之歌的精義就是.
他會讓人未聽過能夠聽到.
未看過會看到.
聽了之後就會明白.
上帝的心意從而有不一樣的人生.
願主你繼續對我們說話.
我們感恩禱告.
奉耶穌的名銜.
阿門.
\newpage



\section{創世記 1:1-5-20220326}
\label{sec:0d9n3K2nnYY}
\textbf{【網上崇拜】你最閃亮的一刻,是你踏進不安的一剎.|創世記1\_1-5|20220326 [0d9n3K2nnYY]}
\newline
\newline
連結: \href{https://youtube.com/watch?v=0d9n3K2nnYY}{\texttt{ https://youtube.com/watch?v=0d9n3K2nnYY}} ~~~~ 語音日期: 2022-03-26 
\newline
\newline
\hyperref[sec:3bu5V6aPJC0]{\small{< < < PREV SERMON < < <}}
~
\hyperref[sec:index_chronic]{\small{[返順時目]}}
~
\hyperref[sec:index_scriptual]{\small{[返順卷目]}}
~
\hyperref[sec:rbtYKzrN9IU]{\small{> > > NEXT SERMON > > >}}
\newline
\newline
創世記 1:1-5-20220326
\newline
\begin{longtable}{cl}
\hline
\hline
章節 & 經文 (和合本修訂版)\\
\hline
1:1 & \begin{tabularx}{0.7\textwidth}{X} 起初,神創造天地。 \end{tabularx} \\ \\ \relax
1:2 & \begin{tabularx}{0.7\textwidth}{X} 地是空虛混沌,深淵上面一片黑暗;神的靈運行在水面上。 \end{tabularx} \\ \\ \relax
1:3 & \begin{tabularx}{0.7\textwidth}{X} 神說:「要有光」,就有了光。 \end{tabularx} \\ \\ \relax
1:4 & \begin{tabularx}{0.7\textwidth}{X} 神看光是好的,於是神就把光和暗分開。 \end{tabularx} \\ \\ \relax
1:5 & \begin{tabularx}{0.7\textwidth}{X} 神稱光為「晝」,稱暗為「夜」。有晚上,有早晨,這是第一日。 \end{tabularx} \\ \\ \relax
1:6 & \begin{tabularx}{0.7\textwidth}{X} 神說:「眾水之間要有穹蒼,把水和水分開。」 \end{tabularx} \\ \\ \relax
1:7 & \begin{tabularx}{0.7\textwidth}{X} 神就造了穹蒼,把穹蒼以下的水和穹蒼以上的水分開。事就這樣成了。 \end{tabularx} \\ \\ \relax
1:8 & \begin{tabularx}{0.7\textwidth}{X} 神稱穹蒼為「天」。有晚上,有早晨,這是第二日。 \end{tabularx} \\ \\ \relax
1:9 & \begin{tabularx}{0.7\textwidth}{X} 神說:「天下面的水要聚集在一處,使乾地露出來。」事就這樣成了。 \end{tabularx} \\ \\ \relax
1:10 & \begin{tabularx}{0.7\textwidth}{X} 神稱乾地為「地」,稱聚集在一起的水為「海」。神看為好的。 \end{tabularx} \\ \\ \relax
1:11 & \begin{tabularx}{0.7\textwidth}{X} 神說:「地要長出植物,就是含種子的五穀菜蔬,和會結果子、果子裡有種子的樹,在地上各從其類。」事就這樣成了。 \end{tabularx} \\ \\ \relax
1:12 & \begin{tabularx}{0.7\textwidth}{X} 於是地長出了植物:含種子的五穀菜蔬,各從其類;會結果子、果子裡有種子的樹,各從其類。神看為好的。 \end{tabularx} \\ \\ \relax
1:13 & \begin{tabularx}{0.7\textwidth}{X} 有晚上,有早晨,這是第三日。 \end{tabularx} \\ \\ \relax
1:14 & \begin{tabularx}{0.7\textwidth}{X} 神說:「天上要有光體來分晝夜,讓它們作記號,定季節、日子、年份, \end{tabularx} \\ \\ \relax
1:15 & \begin{tabularx}{0.7\textwidth}{X} 它們要在天空發光,照在地上。」事就這樣成了。 \end{tabularx} \\ \\ \relax
1:16 & \begin{tabularx}{0.7\textwidth}{X} 於是神造了兩個大光體,大的管晝,小的管夜,又造了星辰。 \end{tabularx} \\ \\ \relax
1:17 & \begin{tabularx}{0.7\textwidth}{X} 神把這些光體擺列在天空,照在地上, \end{tabularx} \\ \\ \relax
1:18 & \begin{tabularx}{0.7\textwidth}{X} 管理晝夜,分別光暗。神看為好的。 \end{tabularx} \\ \\ \relax
1:19 & \begin{tabularx}{0.7\textwidth}{X} 有晚上,有早晨,這是第四日。 \end{tabularx} \\ \\ \relax
1:20 & \begin{tabularx}{0.7\textwidth}{X} 神說:「水要滋生眾多有生命之物;要有鳥飛在地面以上,天空之中。」 \end{tabularx} \\ \\ \relax
1:21 & \begin{tabularx}{0.7\textwidth}{X} 神就創造了大魚和在水裡滋生的各樣活動的生物,各從其類,以及各樣有翅膀的鳥,各從其類。神看為好的。 \end{tabularx} \\ \\ \relax
1:22 & \begin{tabularx}{0.7\textwidth}{X} 神就賜福給這一切,說:「要繁殖增多,充滿在海的水裡;飛鳥也要在地上增多。」 \end{tabularx} \\ \\ \relax
1:23 & \begin{tabularx}{0.7\textwidth}{X} 有晚上,有早晨,這是第五日。 \end{tabularx} \\ \\ \relax
1:24 & \begin{tabularx}{0.7\textwidth}{X} 神說:「地要生出有生命之物,各從其類,就是牲畜、爬行動物、地上的走獸,各從其類。」事就這樣成了。 \end{tabularx} \\ \\ \relax
1:25 & \begin{tabularx}{0.7\textwidth}{X} 於是神造了地上的走獸,各從其類;牲畜,各從其類;和地上一切的爬行動物,各從其類。神看為好的。 \end{tabularx} \\ \\ \relax
1:26 & \begin{tabularx}{0.7\textwidth}{X} 神說:「我們要照著我們的形像,按著我們的樣式造人,使他們管理海裡的魚、天空的鳥、地上的牲畜和全地,以及地上爬的一切爬行動物。」 \end{tabularx} \\ \\ \relax
1:27 & \begin{tabularx}{0.7\textwidth}{X} 神就照著他的形像創造人,照著神的形像創造他們;他創造了他們,有男有女。 \end{tabularx} \\ \\ \relax
1:28 & \begin{tabularx}{0.7\textwidth}{X} 神賜福給他們,神對他們說:「要生養眾多,遍滿這地,治理它;要管理海裡的魚、天空的鳥和地上各樣活動的生物。」 \end{tabularx} \\ \\ \relax
1:29 & \begin{tabularx}{0.7\textwidth}{X} 神說:「看哪,我把全地一切含種子的五穀菜蔬和一切會結果子、果子裡有種子的樹,都賜給你們;這些都可作食物。 \end{tabularx} \\ \\ \relax
1:30 & \begin{tabularx}{0.7\textwidth}{X} 至於地上一切的走獸、天空一切的飛鳥,並一切在地上爬行的,有生命的動物,我把綠色植物賜給牠們作食物。」事就這樣成了。 \end{tabularx} \\ \\ \relax
1:31 & \begin{tabularx}{0.7\textwidth}{X} 神看一切所造的,看哪,都非常好。有晚上,有早晨,這是第六日。 \end{tabularx} \\ \\
[1ex]
\hline
\hline
\end{longtable}
$^{1}$讓我們再次轉目望向我們的上帝.
今天跟大家一起藉著一段.
應該大家很熟悉.
可能每年納次都會讀的經文.
《創世紀》.
我們重複讀了這段經文很多次.
這段經文究竟有什麼提醒給我們呢?.
《創世紀》在猶太人的聖經裡.
希伯來的聖經.
它不是叫做《創世紀》.
猶太人會用他們聖經原文的第一句.
頭兩個字作為書卷的名字.
所以在猶太人裡.
這本書叫做《起初》又或者《起源》.
其實相對於《創世》這個字.
《起源》,《起初》就更加適合這本書.
因為這本書講述了很多的起源.
世界的起源.
人類的起源.
罪惡的起源.
神拯救計劃的起源.
以色列民族的起源.
在《創世紀》的第一,二章裡.
我們看到神創造這個世界.
我們除了可以體會到神的偉大.
以及我們的渺小之外.
究竟這兩章的經文.
又有什麼我們值得去學呢?.
就好像剛剛在靈師之先.
帶領我們去思想的.
就是究竟神創造了一片什麼的世界.
這片世界跟我們又有什麼關係呢?.
丁子妹 今天我很想跟大家去思想.
我們面對一個又混亂又黑暗的世界.
我們要承認我們沒辦法扭轉這個局面.
不過.
我們仍然有一些的本領.
可以將上帝賜給你和我的每一天.
活得漂亮一點.
讓你身邊愛你的人.

$^{41}$你愛的人.
讓他們活得好一點.
今天整篇的講道.
我們會分三部分.
但我們仍然圍繞著一個訊息.
一個主題.
我很希望你聽完之後.
你仍然記得這一句話.
在你人生最閃亮的一刻.
不是你成功的那一刻.
而是你決定踏進不安的一剎.
你最閃亮的那一刻.
不是你名成利就的那一刻.
而是你踏進不安的那一剎.
讓我們先讀今天的經文.
《創世紀》的第一章.
我只是選了頭五節.
不過我們也會橫跨第一第二章.
創造的一些記載.
一章一節.
「起初 神創造天地.
地是空虛混沌 冤冤黑暗.
神的靈運行在水面上.
神說要有光就有了光.
神看光是好的.
就把光暗分開了.
神稱光為晝 稱暗為夜.
有晚上 有早晨.
這是頭一日」.
我們分兩段去看.
我們先看一二節.
「起初 神創造天地」.
這句話是整卷聖經的命題.
對於我們很有興趣.
神究竟有沒有創造過恐龍.
神有沒有做過一塊.
祂自己舉不起的石頭.
聖經一一都沒有回答.
祂對我們這些有興趣的提問.
完全沒有興趣地回答.

$^{81}$誰創造神 神又怎麼來.
聖經第一句就宣告.
這個世界是屬於上帝.
沒有其他答案給我們.
然後我們再看多揭兩頁聖經.
我們就會發現一件事.
就是很多人 靈界.
都不甘心服於這句聖經之下.
我們都不甘心服於.
世界就是屬於上帝之下.
所以我們由創世之初直到如今.
我們一直和上帝爭持擁有權.
當我們教會常常去說.
信靠 侍奉 敬拜 奉獻.
這些全部都是將我們的主權.
再交回給我們的上帝.
就是我們再伏在創世紀的一章一節.
「起初 神創造天地」.
然後我們繼續看第二句.
究竟上帝怎樣去創造天地.
祂所面對的是什麼呢.
聖經這樣去描述一章二節.
祂說 「地是空虛混沌 冤冤黑暗.
神的靈運行在水面上」.
空虛混沌是連在一起用的.
在整本聖經出現過三次.
空虛就不同傳道書的虛空.
空虛是代表像曠野一樣 渺無人煙.
而混沌除了什麼都沒有之外.
更加好像有一片混亂的感覺.
在耶利米斯去形容空虛混沌.
是形容在地震之後的景象.
不單止什麼都沒有 而且是一片混亂.
至於冤冤黑暗.
其實在猶太人的文化來說.
是一些邪惡的概念.
就像中國人看到的四字一樣.
我們都知道很多大台都不會慶祝三十四年.
是直接邁向三十五年.
是一些很忌諱 很邪的東西.

$^{121}$黑暗 冤冤都是猶太人在文化傳統裡面.
覺得一些不是好東西.
上帝所創造的是面對空虛混沌.
冤冤黑暗的世界.
然後那句話就是.
神的靈運行在水面上.
這句話引起很多神學的討論.
特別是說到神的靈.
究竟是聖靈還是直翼.
是一個很強而有力的風.
去撲向水面上.
但不同的理解都不能夠離開一個主調.
就是我們的神下定決心.
去治理眼前這一片空虛混沌.
冤冤黑暗.
我們的神起初就下定決心.
要徹底處理這個問題.
頂姐妹如果你聽到這裡.
我猜我們開始掌握到創世紀之後的記載.
整個創造不是純粹由無到有.
一個變氣法.
而是有目的處理空虛混沌.
冤冤黑暗.
所以我們看.
聖經的創造總共分了三部分.
第一至三日作為第一組.
第二組就是四至六日.
然後最後的一組是第七日.
我們看到第一日.
剛才所讀的聖經都有提到.
第一日創造了什麼.
上帝創造了光.
然後去到第二組的第一日.
就是第四日.
就是做回跟光相對的日月.
然後第二日創造了天和海.
相對的就是第五日.
在天和海裡面佈置了魚和鳥.
上帝處理的是由一片混亂變為有秩序.
由黑暗轉為光明.

$^{161}$由空虛什麼都沒有.
然後將它一一填滿.
第三日祂創造了植物和地.
相對的就是第六日.
祂創造了需要食物的人類和動物.
上帝預備好一切人之後出現.
不過整個物質世界不是創造的高峰.
創造的高峰是第七日.
第七日是什麼來的.
沒錯.
第七日是公眾假期.
上帝最偉大的創造.
是創造了一天公眾假期給我們.
剛才節目整個經文我們都開始掌握.
讓我做一個小小的總結.
然後我們再進入下一步的部分.
單是看上帝整個創造.
是很治癒的感覺.
早前有一個電視廣告都令我們會心微笑.
廣告的演員說他的夢想是上班.
我從來沒有這個夢想.
我不知道大家有沒有.
但如果在上帝的創造裡面.
上帝也沒有這個夢想.
我們試下放下神學或者《釋經》.
我們單單去代入人類.
阿當他人生的第一天怎樣過.
當阿當被創造的時候.
他可能睡在那裡.
聽到海浪聲.
聽到鳥叫.
慢慢驚醒了他.
他睜開眼睛就見到.
天空可以這麼藍.
水可以這麼清.
這裡是伊甸園還是馬爾代夫.
我猜他分不清.
然後上帝吩咐他要做些什麼呢.
你就是和小狗去跑步.
一會兒和獅子玩一會兒就好了.

$^{201}$園裡面所有的東西你都可以吃.
除了那棵東西你不要搞.
所有的你都吃.
阿當就按著神給他的一切美好的安排.
就嘗遍園裡面所有的果實.
然後累了他又睡了一覺.
當他再睜大眼睛的時候.
他竟然發現坐在他身邊.
是一個很美麗的生物.
這個美麗的生物吸引著他的目光.
睡完午覺見到這個生物.
他就和他談了一場戀愛.
然後晚上一起看星星.
然後等著他們兩個的.
是一個更美麗的日子.
公眾假期.
他們被創造之後.
是不需要急於工作.
上帝讓他們享受一切.
這個本來是上帝創造的一幅美麗圖畫.
丁子妹在聖經裡面我們理解差不多了.
今天想和大家一起去思想.
究竟上帝創造了些什麼出來呢.
就正如我們剛剛在敬拜之前.
領師的弟兄帶我們去思想.
在我們腦裡面.
究竟上帝創造了一幅什麼圖畫.
以致對今天的我有幫助呢.
我有一個問題我不知道大家有沒有想過.
究竟神用七天創造世界.
是快還是慢呢.
究竟用七天創造世界.
是很快還是太慢呢.
如果我們知道上帝是無所不能.
其實祂拍一隻手指.
所有世界的東西都創造了出來.
但我們的神腳是萬條屍理.
一天做一件事.
一天佈置一件事.
上帝不是急於要將整個世界.

$^{241}$由沒有變成有.
我們的神是處理混亂.
黑暗 空虛這些問題.
祂將一個節奏.
將一個規律.
放置在整個創造裡面.
上帝要處理的問題.
不純粹是由沒有變有.
而是祂將一個內在的規律.
今天都創造在你和我的心裡面.
我們都面對著一片空虛混亂的世界.
能夠抵禦這些空虛混亂.
是我們內裡面有一個更硬朗的規律.
以致我們活在這片混亂裡面.
仍然可以敵得過.
有時候我們可能看的電影太多.
煲劇煲得太厲害.
我們覺得地球毀滅的時候.
只要有個英雄出來.
大概兩個小時.
所有事情都可以扭轉.
哪管你失了一場戀.
大概去到第三,四集.
主角就會愛你.
又或者你找個更好的.
我們人生被催逼得很急.
很多事情很想斬一下眼就解決.
但似乎在聖經裡面讓我們看見.
就算上帝祂都是逐日.
將一些事情一滴一滴扭轉.
一滴一滴解決.
上帝的創造方法也很特別.
如果我們看.
黑暗和冤滅.
即水.
是邪惡的.
是不好的.
但神沒有將它們滅掉.
完全消失.
上帝對黑暗說.

$^{281}$從今天起.
你改過自新.
你不要再叫黑暗.
我稱你為夜.
變成一個浪漫的夜晚.
他也對心冤說.
你以後不要叫心冤.
我稱你為海.
你變成浪漫的海.
上帝將一些問題.
由邪惡扭轉到合上帝使用.
這是整個創造裡面讓我們看見的.
我想我們人生有時候會面對.
好像連上天也不幫你的日子.
很多事情一次過.
沒有問過我它就倒下來.
求助無援.
一片混亂.
就好像新一波的疫情一樣.
那片混亂比上一年更加可怕.
不單止疫情.
甚至整個政策.
都令我們震驚.
令我自己也去想.
上年我們經歷過一次.
去到今年.
究竟我有沒有進步到.
面對一片混亂的時候.
我有沒有一些東西學到.
以致我自己可以抵禦呢.
我記得上年教會需要暫停.
需要在家工作.
小朋友又不需要上學.
起初我也覺得頗享受.
也有一段時間沒有休息過.
但開始發現.
其實我自己的習慣是.
家裡沒有儲糧.
最多只有三四天的食物.
打算到街上去買很方便.

$^{321}$我從來都是這樣想的.
誰知在那段日子.
就發現什麼叫做空虛混沌.
淵面黑暗.
完全什麼都沒有.
那時候自己家裡口罩只有十個.
從來都沒有買到一支漂白水.
在整個過程中都感受到很大的張力.
又或者和弟兄姊妹一樣.
在上一年的時候.
在街上見到有人排隊.
自己也不理會是什麼.
總之排到隊尾才算.
然後問一問前面那個.
究竟排什麼的.
我通常得到的答案只有一個.
我怎麼知道.
就繼續排隊.
上年我是這樣過的.
我自己問.
究竟我經歷了一次之後.
今年我會不會不同了呢.
我發現經歷了一次之後.
自己也很多東西改變了.
第一我真的需要一個安息日.
我需要一個冷靜的腦袋.
去面對這一片混亂的事情.
我不能夠很急.
去馬上回應所有事情.
然後我開始學會.
將最壞的可能寫出來.
我發現我自己的腦袋.
只會越想越差.
世界是會滅亡的.
越想越不理想.
於是我自己拿筆.
我自己寫出來.
最多最多.
買東西吃是貴一倍.
可能要熬一段時間.

$^{361}$最擔心是小朋友沒東西吃.
對我來說.
所以由上一年開始.
我決定每晚都吃少一點.
是真的.
如果你看過我以前的造型.
應該是比較飽滿一點.
由上年開始.
我跟自己說.
我不吃不要緊.
小朋友有得吃就行了.
由那時候開始.
我建立一些新的習慣.
然後我跟自己說.
就算變得這麼差.
多加一句話.
那又怎樣.
最多是這樣.
不會再差下去.
所以今年我自己開始發現.
由上年開始建立的一些習慣.
去幫助到我.
今年面對的一片混亂.
我試過以往疫情之前.
跟太太兩個人晚上外出吃飯.
通常她一碟我一碟.
我一碟是不夠飽的.
我經常看著她吃完剩下的.
可以給我.
但今年我發現.
我跟太太兩個人吃一碟.
都已經太飽.
我發現自己完全失去了.
吃放題自助餐的恩慈.
上帝已經收回.
頂姐妹是真的.
我自己不是很喜歡混亂疫情.
但我真的喜歡混亂疫情的自己.
就是面對一片黑暗混沌.
我要想盡辦法.

$^{401}$去活出上帝給我的心意.
這是上帝喜悅的自己.
在聖經裡面說.
連神都要面對一片空虛混沌.
冤冤黑暗.
何況我們呢.
所以我歸納出有兩個可行的方法.
讓我們自己提醒.
第一就是.
一天一點將它改善.
不要太心急.
太多事急不來.
連上帝都是萬條屍.
用了七天的時間.
去將整個世界處理好.
我們又怎會比上帝厲害.
一天一點將它改善.
第二就像剛才所說.
有些事情就算你再努力.
你都解決不了.
無論你出盡氣力也好.
有些事是跟你一輩子.
你就讓它成為你人生的一部分.
等會有些傷口.
你是埋不了的.
就算你回頭.
讓你再遇見過這些問題.
你都是在做相同的決定.
有些事情是我們的限制.
唯有你跟它融成一體.
你接納它成為人生的一部分.
我們就可以應對這些問題.
神用了七天創造世界.
是不是太奢侈了.
上帝不是純粹創造一個物質的世界給我們.
上帝是教導我們.
怎樣面對一片空虛混沌.
冤冤黑暗.
上帝要將一個規律.
放在你和我的心裡.

$^{441}$以致我們有能力應對這些問題.
頂姐妹在創造裡.
神有什麼提醒給了你呢.
你有那幅圖畫.
仍然是落印在你心裡呢.
來到最後我想和大家分享.
關於劉唐月位這個主題.
大概在二月尾的時候.
收到這個主題.
我自己很感動.
甚至不是感動 是激動.
因為我在想這個問題.
也想了很一段時間.
在去年的十二月.
十二月八號.
回到長洲神學院裡.
去探一探神學生.
我正在思考.
究竟我自己人生.
還有沒有一些突破的位置呢.
每次想這些重要的問題.
都是回到神學院.
或者小時候長大的地方.
去繞一圈.
這次特意約了兩位神學生.
約他們吃早餐.
所以我坐早班車進去長洲.
見見他們 和他們聊聊天.
被他們兩位閃閃的眼光吸引.
兩位神學生很有抱負.
很多的想法.
他們還沒踏出工場一步.
但見到他們心裡的熱誠.
自己也被震撼.
那天回去除了想探望神學生之外.
也想探望其中一位師長.
我姑且用個代號給他.
叫沙膽陳.
那天去找他.
誰知道他剛好那天早上.

$^{481}$就出去了九龍.
不在長洲.
沒辦法 錯過了跟他見面的時間.
於是我吃完早餐.
就坐船離開長洲.
在船裡看著一片海.
自己也去問一個問題.
記得年輕的時候.
自己也很勇敢.
拿持也不止一次.
無論背著背囊去旅行也好.
創業搞到滿身血也好.
也不止單一次.
但今天竟然圍住一些.
自己很想突破 很想試的問題.
猶豫不決 缺而不幸.
我自己也問.
是不是年紀大了.
已經沒以前那麼勇敢呢.
我想了一遍又一遍.
我發現最大的分別就是.
現在已經跟少年的時候不同.
少年我只是為自己負責任.
我想做我自己喜歡做的事.
我不怕辛苦.
不怕一身血.
去做就去做.
但我知道我現在背著的是一堆責任.
我的決定不是純粹我自己.
我的決定是在影響很多人.
於是我在船裡笑自己.
我自己這樣想.
如果我經歷一些大的刺激.
我猜我勇敢一些.
可以突破眼前的那種限制.
心裡想了一句這樣的說話.
大概下午的時候.
回到屯門.
我在屯門侍奉 在屯門居住.
回到屯門的時候跟太太吃午飯.

$^{521}$我就說起今天的經歷.
我就有這樣的想法.
大概這個時候.
我電話響起.
醫院打給我.
他跟我說.
你爸爸不行.
我跟太太趕到醫院.
我仍然遲了兩分鐘.
見不到他最後一面.
我腦裡第一個問題.
我是不是騎錯了肚.
為什麼會這樣.
12月是教會最忙碌的日子.
12月是我狀態最差的日子.
我面對一片混亂.
甚至我排不到一個日期.
幫爸爸做一個正常的安息禮.
我每天早上睜開眼睛.
就跟自己說.
我今天只可以做好一件事.
今天我要寫講章.
我就寫講章.
今天要為教會做這些文件.
我就專心做這些文件.
今天我要預備帶茶經.
我就乖乖地預備帶茶經.
然後我將所有剩下的時間.
都陪家人.
在這個過程裡.
我深深體會兩件事.
這兩件事對我來說很大的提醒.
第一就是人生.
其實只需要一行東西就夠了.
我們可以在下一張的powerpoint裡.
可以看到.
在這段時間我幫爸爸處理遺物.
在家裡在護老院裡.
我發現他很多新的東西都沒用過.
原來在他人生跟到最後.

$^{561}$他只有一件舊的大褸.
幾件衣服.
他一個鬚刨.
一個電話.
一盤假牙.
還有一張家庭照.
對他來說.
或者對我的提醒都是一樣.
能夠陪你走過整個人生的東西.
其實不是那麼多.
頂尖我想說.
好好整理你自己的物品.
你純粹納至要為上帝做很多事.
你不如開始整理一下你的房間.
好好整理你擁有的物品.
有些東西上帝不喜悅的.
丟掉它.
不要再收起來.
有些東西是屬於別人.
你過多的.
捐掉它.
這個是上帝喜悅的.
當你有一天發現.
你對物品依賴越少.
你的心靈就越自由.
可以為上帝跑更遠.
這個是第一.
第二就是.
將一些事情慢慢去變好.
盡力去做.
但也不要勉強自己.
我們人生實在太多事想做.
但最後是一事無成.
頂姐妹你不要高估了.
自己在一年內能夠完成的事.
但同時你也不要低估了.
你五年之後所堅持出來做的事.
不要太過高估.
很短時間就得到很多人的認同.
但同樣你不要低估.

$^{601}$你堅持要做的事.
這些事情上帝有一天會用得著.
上帝用得著的不未必是你手裡的所作.
上帝用得著的是那個堅持的你.
頂姐妹真的不要強求.
自己能夠很快很容易到達終點.
但開始為你自己能夠離開你的起點.
而覺得自豪.
看清楚.
當阿伯拉罕聽到上帝呼召他的時候.
他也不知道自己去哪裡.
當摩西面向紅海.
一步一步走出去的時候.
他就再沒有回頭看追兵.
當彼得看到耶穌在海裡.
他踏出船的第一步.
他沒有帶水泡.
他沒有想後果.
當我們知道強敵要來到.
無論我們的國家多麼的微弱.
我們都上戰場.
去打一場不能夠贏.
但也不能輸的仗.
這些所有的圖畫.
都是震撼我們的心靈.
人生最閃亮的一刻.
不是你拿到成就的那一刻.
而是你決定去擁抱這一片不安.
你決定走出來的那一剎那.
那一剎那就是聖經裡所說的.
起初.
起初神創造天地.
今天我們都活在一片混亂的世界.
神父召我們行一條忠心的澤路.
一星期.
一個月.
一年.
讓我們堅持到多久就多久.
到有一天你回頭再看.
你會發現你自己滿身的傷痕.

$^{641}$都記載著你對上帝的忠誠.
你沒有放下過.
或者你已經看破紅塵.
對於立志這些玩意已經沒有興趣.
成功太少失敗太多.
我們已經很明白自己的限制.
但願今天的經文再次提醒你.
上帝下定決心.
要治你眼前的空虛混沌.
冤冤黑暗.
對抗混亂和黑暗.
讓我們生活先活得有規律.
就讓你今天開始.
將今天活得漂亮一些.
也讓你所愛的人活得更加好.
願主賜福給各位.
\newpage



\section{馬太福音 5:20-20220402}
\label{sec:rbtYKzrN9IU}
\textbf{【網上崇拜】Außerordentlichen 不要尋常|馬太福音5\_20|20220402 [rbtYKzrN9IU]}
\newline
\newline
連結: \href{https://youtube.com/watch?v=rbtYKzrN9IU}{\texttt{ https://youtube.com/watch?v=rbtYKzrN9IU}} ~~~~ 語音日期: 2022-04-02 
\newline
\newline
\hyperref[sec:0d9n3K2nnYY]{\small{< < < PREV SERMON < < <}}
~
\hyperref[sec:index_chronic]{\small{[返順時目]}}
~
\hyperref[sec:index_scriptual]{\small{[返順卷目]}}
~
\hyperref[sec:9oqAyDoUD6A]{\small{> > > NEXT SERMON > > >}}
\newline
\newline
馬太福音 5:20-20220402
\newline
\begin{longtable}{cl}
\hline
\hline
章節 & 經文 (和合本修訂版)\\
\hline
5:20 & \begin{tabularx}{0.7\textwidth}{X} 我告訴你們,你們的義若不勝過文士和法利賽人的義,絕不能進天國。」 \end{tabularx} \\ \\ \relax
5:21 & \begin{tabularx}{0.7\textwidth}{X} 「你們聽過有對古人說:『不可殺人』;『凡殺人的,必須受審判。』 \end{tabularx} \\ \\ \relax
5:22 & \begin{tabularx}{0.7\textwidth}{X} 但是我告訴你們:凡向弟兄動怒的,必須受審判;凡罵弟兄是廢物的,必須受議會的審判;凡罵弟兄是白痴的,必須遭受地獄的火。 \end{tabularx} \\ \\ \relax
5:23 & \begin{tabularx}{0.7\textwidth}{X} 所以,你在祭壇上獻祭物的時候,若想起有弟兄對你懷恨, \end{tabularx} \\ \\ \relax
5:24 & \begin{tabularx}{0.7\textwidth}{X} 就要把祭物留在壇前,先去跟弟兄和好,然後來獻祭物。 \end{tabularx} \\ \\ \relax
5:25 & \begin{tabularx}{0.7\textwidth}{X} 你同告你的冤家還在路上,就要趕快與他講和,免得他把你送交給法官,法官交給警衛,你就下在監裡了。 \end{tabularx} \\ \\ \relax
5:26 & \begin{tabularx}{0.7\textwidth}{X} 我實在告訴你,就是有一個大文錢還沒有還清,你也絕不能從那裡出來。」 \end{tabularx} \\ \\ \relax
5:27 & \begin{tabularx}{0.7\textwidth}{X} 「你們聽過有話說:『不可姦淫。』 \end{tabularx} \\ \\ \relax
5:28 & \begin{tabularx}{0.7\textwidth}{X} 但是我告訴你們:凡看見婦女就動淫念的,這人心裡已經與她犯姦淫了。 \end{tabularx} \\ \\ \relax
5:29 & \begin{tabularx}{0.7\textwidth}{X} 若是你的右眼使你跌倒,就把它挖出來,丟掉。寧可失去身體中的一部分,也不讓整個身體被扔進地獄。 \end{tabularx} \\ \\ \relax
5:30 & \begin{tabularx}{0.7\textwidth}{X} 若是你的右手使你跌倒,就把它砍下來,丟掉。寧可失去身體中的一部分,也不讓整個身體下地獄。」 \end{tabularx} \\ \\ \relax
5:31 & \begin{tabularx}{0.7\textwidth}{X} 「又有話說:『無論誰休妻,都要給她休書。』 \end{tabularx} \\ \\ \relax
5:32 & \begin{tabularx}{0.7\textwidth}{X} 但是我告訴你們:凡休妻的,除非是因不貞的緣故,否則就是使她犯姦淫了;人若娶被休的婦人,也是犯姦淫了。」 \end{tabularx} \\ \\ \relax
5:33 & \begin{tabularx}{0.7\textwidth}{X} 「你們又聽過有對古人說:『不可背誓,所起的誓總要向主謹守。』 \end{tabularx} \\ \\ \relax
5:34 & \begin{tabularx}{0.7\textwidth}{X} 但是我告訴你們:甚麼誓都不可起。不可指著天起誓,因為天是神的寶座。 \end{tabularx} \\ \\ \relax
5:35 & \begin{tabularx}{0.7\textwidth}{X} 不可指著地起誓,因為地是他的腳凳;也不可指著耶路撒冷起誓,因為耶路撒冷是大君王的京城。 \end{tabularx} \\ \\ \relax
5:36 & \begin{tabularx}{0.7\textwidth}{X} 又不可指著你的頭起誓,因為你不能使一根頭髮變黑變白。 \end{tabularx} \\ \\ \relax
5:37 & \begin{tabularx}{0.7\textwidth}{X} 你們的話,是,就說是;不是,就說不是。若再多說,就是出於那惡者。」 \end{tabularx} \\ \\ \relax
5:38 & \begin{tabularx}{0.7\textwidth}{X} 「你們聽過有話說:『以眼還眼,以牙還牙。』 \end{tabularx} \\ \\ \relax
5:39 & \begin{tabularx}{0.7\textwidth}{X} 但是我告訴你們:不要與惡人作對。有人打你的右臉,連另一邊也轉過去由他打。 \end{tabularx} \\ \\ \relax
5:40 & \begin{tabularx}{0.7\textwidth}{X} 有人想要告你,要拿你的裡衣,連外衣也由他拿去。 \end{tabularx} \\ \\ \relax
5:41 & \begin{tabularx}{0.7\textwidth}{X} 有人強迫你走一里路,你就跟他走二里。 \end{tabularx} \\ \\ \relax
5:42 & \begin{tabularx}{0.7\textwidth}{X} 有求你的,就給他;有向你借貸的,不可推辭。」 \end{tabularx} \\ \\ \relax
5:43 & \begin{tabularx}{0.7\textwidth}{X} 「你們聽過有話說:『要愛你的鄰舍,恨你的仇敵。』 \end{tabularx} \\ \\ \relax
5:44 & \begin{tabularx}{0.7\textwidth}{X} 但是我告訴你們:要愛你們的仇敵,為那迫害你們的禱告。 \end{tabularx} \\ \\ \relax
5:45 & \begin{tabularx}{0.7\textwidth}{X} 這樣,你們就可以作天父的兒女了。因為他叫太陽照好人,也照壞人;降雨給義人,也給不義的人。 \end{tabularx} \\ \\ \relax
5:46 & \begin{tabularx}{0.7\textwidth}{X} 你們若只愛那愛你們的人,有甚麼賞賜呢?就是稅吏不也是這樣做嗎? \end{tabularx} \\ \\ \relax
5:47 & \begin{tabularx}{0.7\textwidth}{X} 你們若只請你弟兄的安,有甚麼比別人強呢?就是外邦人不也是這樣做嗎? \end{tabularx} \\ \\ \relax
5:48 & \begin{tabularx}{0.7\textwidth}{X} 所以,你們要完全,如同你們的天父是完全的。」 \end{tabularx} \\ \\
[1ex]
\hline
\hline
\end{longtable}
$^{1}$弟兄姊妹平安.
在此祝願Full Church的弟兄姊妹.
無論你在香港.
或離開香港.
在基督耶穌裡面向大家問安.
願基督耶穌的恩惠.
常常伴隨著你們.
無論你遭遇任何事情.
願上帝的愛.
在基督裡面保守你們.
今日非常想念大家.
我自己很期待.
我們能夠很快的.
實體的來到聚會.
今日我們會繼續講月位這個議題.
這是我自己的第二講.
我自己越來越喜歡月位這個題目.
這個題目帶著一些反叛.
其實是不斷的跨越.
突破.
衝破我們的限制.
我自己也很期望這兩個月的訊息.
能夠真的幫助到Full Church的弟兄姊妹.
能夠帶來一些幫助.
如果大家還記得.
上一次我在講.
《Track up記》裡面.
兩個修身婆的故事.
兩個敬畏上帝的女人.
打破社會政權的規範.
活出敬畏上帝的信仰生命.
希望大家能夠思考我們.
面對社會時事的一篇講道.
今日我們會講一些.
熟齡一點的題目.
我們會講一下我們的熟齡生命.
可能我之前也提過.
我發覺自己這幾年.
從Full Church開始講道.
我自己是越來越少在外面接受講道.

$^{41}$為什麼呢?大概有幾個原因.
一來就是忙.
不過最重要的原因.
我發覺我自己是越來越.
不知道如何在其他教會裡面.
講一些一般的熟齡的講道.
一方面我不是有很大的動力.
不知道如何去講.
二來真的不知道.
在這幾年裡面風流火勢.
如何講出一般的熟齡的道.
不過今天我們仍然.
回到一個熟齡的題目.
這是很重要的.
仍然是思考我們的熟齡生命.
今天我們會講登山寶訓.
登山寶訓是一個大家非常熟悉的經文.
馬道福音第五章第六章第七章的經文.
尤其是第五章.
就算你不是對聖經研究的好.
你大概對馬道福音第五章的內容是很熟悉的.
很有印象的.
首先是白福.
虛心的人有福了,因為天國是他們的.
虛心的人有福了,因為他們必得見上帝.
然後就是作炎作光的經文.
你們是世上的炎,你們是世上的光.
你們的光也當是一種招數人前.
然後是第十七至第十二節.
這個有關律法的段落.
十七節,莫想我要來廢掉律法.
同時才知道,我來不是要廢掉,而是要成全.
然後從第十一節開始.
耶穌是連續講了六個對舊約律法的重新解釋.
你們聽見有話說,怎樣怎樣怎樣.
只是我告訴你們,其實是怎樣怎樣怎樣.
連續六次的句式.
論法路,論奸淫,論幽妻,論起誓,論報復,論安守的.
最後第五章的第十節,最後一節.
這樣說,所以你們要完全將你們的天賦完全一樣.

$^{81}$作為整個第五章的總結.
大家很熟悉的.
我懷疑第五章是你們唯一一章.
能夠將整個的段落背出來的經文.
每一節的經文,大家大概都聽到,聽過,每一節都查經,都查過的.
今天我們所說的,正正就是第五章中間的那節經文.
最最中間的經文.
就是一個,我認為是一個最重要的經文.
就是第五章第十節.
耶穌說,我告訴你們,你們的義若不勝於民事和法理上的義.
斷不能進天國.
一句,我們今天會一起來思考經文.
我們一起來祈禱.
祝你們將今天的崇拜獻上.
求主,你藉著這段講道的時間,你對我們說話.
藉著你昔日在登山寶訓裡面的教導.
透過聖靈再一次提醒我們.
叫你能夠,每一個頂尖妹,能夠活出你自己的話語.
看重你自己的要求和吩咐.
讓我們能夠明白,更加能夠被你改變.
求主你幫助我們.
讓我們能夠聽見你自己對我們說話.
通主名求,阿們.
馬福音第五章第十節.
耶穌說,我告訴你們,你們的義若不勝於民事和法理上的義.
斷不能進天國.
這個經文,我想,其實在這裡面是一個上下兩段經文裡面的轉折句.
經文的上面是十七到十九節.
耶穌提及的是有關一個舊有的律法的事情.
耶穌強調,他不是要廢掉律法.
不是要另立一套新的律法.
也不是成立一個新的宗教信仰.
一個倫理價值觀.
耶穌要做的是,在這個律法以上.
延伸開去,昇華.
從而更加好地去實踐耶和華上帝律法的本意.
簡單來說,耶穌不是要去整理舊有的系統.
而是要更新一套被猶太人玩爛了的系統.
這是第一個.
然後我們發覺,經文的下面開始.

$^{121}$從第一節開始,直到第四十七節.
耶穌一口氣就講了六個例子.
我稱之為六個反題.
去說明他怎樣去更新舊約的系統.
你們聽見有話說,怎樣怎樣怎樣.
只是我告訴你們,其實是這樣.
於是我們就來到很熟悉的教導.
不可發怒,不可動人念,不可憂妻,不可起誓.
被人打完你的右邊臉,被人打完左邊.
愛仇敵,這六個的教導.
其實都是源於第二十節的經文.
你們的義若不勝於民事和法理錯人的義.
斷不能進天國.
我們一起去看看經文的意思.
首先我們發現,這裡耶穌提到的就是.
原來在世界出現了兩種不同的義.
你們的義和民事法理錯人的義.
我們單單去停下來.
如果你去實心去想想.
不用多,單單停下來的時候.
單單這兩個字就足以令你感到驚訝.
你們的義和民事法理錯人的義.
原來世界上竟然出現了兩個不同的義.
兩個的正確,明明是不同的東西.
都可以被稱為義.
原來在耶穌吩咐門走出來的意外.
世界上有另一種的義存在.
基督徒這樣說不是一種相對主義.
不是說兩者都對,兩者都有道理.
大家都有義,大家都正確.
耶穌不是這樣說.
很明顯,耶穌是絕對不認同民事法理錯人的義.
耶穌的義並不是這樣.
這不是耶穌的義.
耶穌沒有去降低祂的要求.
承認,包容,接納民事法理錯人的義.
但這個訣竅在哪裡?.
雖然耶穌不認同.
但你要耶穌稱民事法理錯人的義.
這兩種東西是不義的時候.

$^{161}$耶穌又不是這樣說.
耶穌沒有去稱民事法理錯人是不義的.
沒有,耶穌不認同他們.
但你要耶穌妖魔化他們.
耶穌並沒有這樣做.
耶穌很公道.
耶穌並沒有妖魔化他們為一個窮安極惡的人.
也沒有妖魔化他們為渣男.
不是.
或者可以這樣說.
他們兩人,法理錯人和民事.
在他們的生命中大概都是一個反派.
一個負面的角色.
耶穌不認同他們的立場.
他們的立場不是耶穌的立場.
但你要耶穌將民事和法理錯人.
當成壞人.
耶穌又不是這樣看.
民事法理錯人都沒有去到最低惡極的地步.
一個妖魔,鬼怪,殺蛋的角色.
甚至民事法理錯人其實都是一個.
在信仰中教導人,肯義的人.
如果你將民事和法理錯人妖魔化為不義的時候.
事情是簡單很多的.
我們是耶穌的門徒.
我們是公義的.
他們是民事法理錯人.
他們是不義的.
這樣做是簡單很多的.
你會舒服很多的.
特別是在這個年頭.
如果你帶著妖魔化的頭盔做人的時候.
你的世界會豁然開朗的.
做人會心理平衡一點.
測度一個人的動機.
選擇用一種不良的解釋去詮釋一個人的行為.
這樣做是可以很舒服的.
然後就遺留.
這個人是不義的.
幫他講說話的都不義的.

$^{201}$這個人認同他都是不義的.
他帶著妖魔化的頭盔做人的時候.
是簡單很多的.
不用煩.
我再強調今天不是講一套相對主義.
不是說大家都覺得包容他是對的意思.
我講的是如果你沒有去選擇妖魔化這班人.
正如蘇修仰的時候.
雖然你不認同這班人的義.
沒有標籤他們是不義的時候.
在這麼偏的位置之間.
其實是很難頂的.
這就是我們基督徒的辛苦之處.
不知道你有沒有這樣的經驗.
在這幾年裡.
在香港在教會裡.
最難頂的是什麼.
就是這些最令你生氣的.
不是一些很錯的事.
錯就不用講了.
而是一些明明是好的事.
對的事.
要做的事.
做得不知所謂.
我不夠說他們是錯.
或者福音的緣故.
他們都是說教會好的.
都是防疫的.
可以是一些很多合理的解釋.
有些是根據的.
有道理的.
出現的可能就是一些官僚主義.
一些play safe.
我試過有一次去一個中派的婚禮.
疫情裡.
他的要求非常之嚴格.
到底會怎麼樣.
當然不用說全世界都戴口罩.
包括新郎新娘都要戴口罩.
然後他們準備怎樣.

$^{241}$他們不讓親的.
最後一刻是不讓他親.
不可以親.
成功婚禮不可以親.
因為要抗疫.
怕中招.
後來那些人就不理了.
就脫了口罩親了一口.
然後那個教會竟然會罰他們錢.
明不明白.
他們可以這樣做.
你說他們是有道理的.
他們要抗疫.
他們是怕傳染.
他們怕犯法.
但是你發覺.
很多時候那個所謂的義.
就變成了這樣的東西.
有很多的道理去支持.
但其實你發現.
耶穌並不是單單這樣做.
你們的義和文斯坦克蘇聯的義.
同時間存在.
正正是我們今天面對的困惑.
耶穌不拜這一套.
耶穌不承認他們是一個不義人.
不過.
耶穌並不是單單停留在他們的義當中.
面對著這兩個義.
耶穌要求門徒怎樣做.
耶穌直接要求門徒.
祂說你們的義若不勝於文士快長的義.
斷不能進天國.
這是一個非常非常嚴謹認真的命令.
所以你問門徒的義.
怎樣才能勝過文士快長的義.
其實首先.
原文並不是解作勝過的意思.
原文的意思是什麼?.
原文是叫paraiso的字.

$^{281}$它的意思不是要去勝過.
不是贏或輸的問題.
paraiso是什麼意思?.
大概可以解作多過,大過,好過,勁過,厲害過,超過.
都可以.
但總之就不是勝過.
你明白嗎?.
重點不是要他們贏他們.
重點不是勝負.
不是比較.
不是有高低之分.
當然可以解作質素上的改變.
或者數量上的改變.
不過這個字的本意並不是這個意思.
paraiso這個字是解作豐富.
豐盛.
abundant.
over.
超越這樣的意思.
說說你就明白了.
paraiso這個字其實出現在五餅二魚的經文裡.
當耶穌行完五餅二魚神蹟之後.
食物多到爆.
多到滿血.
十二個男子都裝不完.
所以就用了paraiso這個字去形容食物的狀況.
over 完.
超過完.
爆完.
多到爆.
滿到血.
這個就是paraiso這個字的意思.
頂智培.
耶穌要求我們的義.
正正是這樣.
耶穌要求門徒的義是一種超越.
可能頂智培都覺得.
丹佛教的教導很難遵守.
愛仇敵.
不可以罵弟兄.

$^{321}$被人打完右邊還被人打左邊是很難的.
但如果頂智培這樣想的時候.
可能你誤會了耶穌的意思.
如果你以為登山補訓只是一套難度更高的律法.
其實這個不是.
你不明白登山補訓的意思.
頂智培不是純粹要將道德倫理推到一個高得不能再高的位置.
耶穌不是純粹要做爛事.
他們說不可以犯奸淫.
我要你連想都不可以想.
他們要你愛人如己詐.
我要你愛埋仇敵.
他們叫你不可以殺人.
我要你連生氣都不准.
如果耶穌的意只不過是一套更高的要求.
更加難實踐更加嚴苛的倫理.
如果這樣的時候.
極其量耶穌只不過是一個更高要求的法律裁人.
耶穌想怎樣.
耶穌想你perizo.
頂智培耶穌的意和文字法上的意最大的差別.
不是純粹由level 5一下子跳到level 100.
你明不明.
重點不是level 100.
重點是你不要停留在level 5裡面.
重點是你不要以為所謂的尾線.
一去到某個程度就足夠就收工.
重點不是level幾.
而是你不要滿足在level幾.
不要自限尾線的可能.
不要局限正義的規範.
不要以為尾線足夠就足夠.
這正正是文字法上的問題.
文字法上的意不是不好.
它都是意來的.
都是好事來的.
不過這個意是有限制的.
只要你這樣做就足夠.
不用拖就足夠.
這正正是他們兩個意的問題.

$^{361}$他們為尾線去定下一個界線.
一個限制.
主啊我一個星期已經禁食兩次.
發得的都卷十分之一了.
頂智培.
今日我們生活的社會裡面.
很多很多這樣的意.
主啊我已經打了三針.
每天都有做檢測.
一條線.
每天都有做安心.
我都得針紙.
一定是一個二人.
我們教會跟足政府的要求.
我們全面配合.
我們沒有犯法.
不要為自己的尾線去下一個定義.
不要嘗試以為這樣就足夠.
這樣就叫做尾線.
如果你以為耶穌的義.
只不過是一套更加高的律法.
跟足就是神仙的時候.
其實你跟快速人員有什麼分別.
耶穌說凡向帶動動腦的男性受審判.
OK 那我不生氣.
那我暗算他.
耶穌說凡向見婦女動能念的就是犯奸淫.
那我不動能念就看AV.
耶穌說有人逼你走兩里路.
走一里路就走了兩里.
OK 我走了兩里.
你給我滾.
耶穌說要你為神禱告.
奉主命求 阿們.
OK 你糟糕了.
不是這樣的.
耶穌是叫你去perizo.
不斷地去超越.
不要停留.
不要問那個合乎尾線的臨界點在哪裡.

$^{401}$然後在這個臨界點裡更加得救.
耶穌是一個人你更加多.
更加多 更加多.
做多一點 做好一點.
去達到更加高的要求.
更加高的難度.
不斷地去超越它.
今天講到的講題是一個德文字.
這個德文字叫做Aus Authenticum.
一個來自《朱祥之篤》裡的字眼.
《朱祥之篤》裡的全式登山補訪第五章非常精彩.
對於第五章 布弗爾給了一個很有意思的題目.
叫做Aus Authenticum Discriminant Labels.
就是說基督徒生命的超平凡.
Aus Authenticum.
可以拆出來.
這個字眼大概是非比尋常的意思.
如果拆開這個字的話.
Aus 就是什麼意外的意思.
Authenticum 就是平凡自然的意思.
所以《朱祥之篤》這本書的中文版本.
我認為叫做超平凡 超平常.
布弗爾認為耶穌在登山補訪裡的教導.
是超過人的自然本能.
違反自然 是不正常的.
正常人是不會做的.
基督徒也不會例外的.
你看 被人打完右邊臉再扯到左邊臉.
愛一些憎恨自己的人.
為仇敵禱告 怎麼可能.
誰會做.
一個正常人.
只要是自決的 可以選擇的.
帶著理性的 都不會做的.
一個人正常發揮 是不會做的.
不過耶穌要求我們要超過正常發揮.
其實你知道的 做基督徒.
從來不是自然而然都能做到的事.
做基督徒是要一點力的.
你個人稍微放鬆一點.

$^{441}$體貼一點自己的時候.
你都做不到的.
聽之會不要尋常.
耶穌叫我們不要去嘗試去尋常.
不斷地來超越 打破自己的底線.
不斷去reach那個的司令.
實踐美善 實踐正義.
用這個態度來做人.
中文有句話很棒的.
叫做什麼 叫做好人做到底.
這句話是很真 很有智慧在背後.
從來都沒有人聽過 做壞人做到底.
做壞人是不需要做到底的.
做壞人怎麼做的.
做壞人做得夠就行了.
通常做壞人是很實用的.
做一件壞事 做到剛剛好就可以了.
做一件壞事 做到滿足你的需要就可以了.
我騙你一次 我不需要殺你全家.
多謝一句.
顧著的大腦 是不是.
做好人和做壞人不同.
做好人你只能夠做到底.
如果你不做到底.
如果你不盡力去做好人.
如果你去為你的美善去畫一個界限.
如果你把美善當成一個交差去做的時候.
整件事就不是美好的了.
所以我們今天試試改一改它.
不要那麼佛教 做下一句吧.
好人做到底 做基督徒就不要那麼垃圾.
No reserve, no retreat, no regret.
艇姐妹 是的 在這個年頭裡面.
做好人是很辛苦的.
當你不選擇用妖魔化來舒緩自己的壓力.
不想去自己的美善去自限.
堅持做一個公道的好人的時候.
是很辛苦 辛苦到爆的事情.
每天都看見社會裡面很多很荒謬的義.
如果叫你只要跟足這些規矩.

$^{481}$你的美善不要犯那些法律.
就是美善的時候.
這就是美善的話.
你受不了的.
不要問規矩是怎樣.
你要去超越那個規矩.
耶穌基督正正就是一個最大的榜樣.
祂既然愛世人 屬自己的人.
就愛他們到底.
耶穌的愛是不自然的.
十八極愛是不自然的.
耶穌的愛都是沒有界線的.
沒有為這個美善去下定義.
沒有玩任何的play safe.
這個不斷去reach那個limit的美善.
正正就是耶穌彰顯的那個意義.
其實心情是有些差的.
我懷疑自己每年一到.
都有一年一度的心情低落.
上年有一件事.
今年你問我因為什麼原因.
大概都是因為社會問題.
社會很差勁.
教會圈子很多困難 很多問題.
自己又代道的迷失.
自己發覺沒有了一個很堅定的信念.
去繼續支持自己下去.
找不到一個constant.
來繼續去面對.
發覺自己對於那個場景越來越心淡.
對很多人的看法都不同.
上帝好像越來越神秘一樣.
有時想很想去離開一下.
重新開始.
可能簡單來說對基督教很厭倦.
經常想著什麼時候可以再給上帝一次打開呢.
很想上帝有一個.
讓我自己重新有一個爆洩的機會.
一會兒我們會唱一首詩歌.
叫做《我願意給你》.

$^{521}$這首詩歌其實是我初信主的時候.
很喜歡的一首詩歌.
歌詞很有意思 雖然很簡單.
但如果你認真去唱.
當是真的唱的時候.
其實是一首很高難度的詩歌.
我願意給你一個整個的生命.
我願意跟你.
無論是苦楚快樂.
我仍願意跟你.
有時跟了自己跟了耶穌這麼多年.
突然回望自己的鏡子的時候.
發覺自己都不知道跟了什麼.
跟了一坨東西的樣子.
有時會想.
其實.
怎樣能夠跟隨耶穌.
什麼叫做神蹟.
近來我又打遊戲機.
又打《Elden Ring》.
《Elden Ring》可能你都聽過.
近來有很多熱門遊戲.
作為一個牧者 作為一個神蹟人員.
我打遊戲當然選擇做先知的角色來設定.
基本上我設定是選擇做先知.
做先知就是說.
我的能力值放在信仰裡面.
信仰是我的能力.
每一次我去禱告.
就能夠打怪.
禱告就可以令到我的神蹟出現.
那些神蹟就是我的大忌.
今天剛好打完遊戲機.
但實質上不是這樣的.
真正的神蹟其實並不是.
你按個按鈕就能夠出現的大忌.
你等級越高就能夠越勁.
你做得越勁 越高等級就越厲害.
真正的神蹟並不是這樣.
真正的神蹟是你明明只有等級1.

$^{561}$但你竟然可以做出等級100的事情.
偶然地.
我們做不到的.
但我們需要.
當我們願意這樣做的時候.
當我們願意去.
在等級1仍然嘗試去做的時候.
上帝就會出現.
最大的神蹟正正是這樣.
上帝並沒有改變了你.
上帝依然是用一個脫皮脫骨.
很多問題的你 一個等級1的你.
但你仍然可以偶爾去做到一些很厲害的事情.
嘗試去突破自己的困難.
去愛一些你愛不到的人.
超越一些自己做不到的事情.
那些是不自然的事情.
那些是很多問題的事情.
但只要你願意去做.
只要你做一次.
在當中去求天賦給你去做的時候.
這個其實就是最大的神蹟.
神蹟不是上帝讓你變成一個屬靈的偉人.
而是保持你是一個很軟弱的人.
但你仍然可以去做到這些這麼高要求的事情.
不要給自己一個的界限.
所以我們稍後會唱這首詩歌.
我願意給你.
希望頂尖會在你的生命裡面.
在這麼多荒謬的世代裡面.
堅持去做一個這樣的基督徒.
在很多困難裡面.
依然去嘗試和耶穌說.
我願意這樣去做.
我們去祈禱.
祈禱求你幫助我們.
讓我們一班頂尖會能夠實踐.
這個登山補訓的教導.
你叫我們去突破我們的界限.
你叫我們去嘗試.

$^{601}$去超越我們自己的軟弱.
求主你這樣去幫助我們.
很多的不足很多的不配.
但求主你這樣去成為我們的幫助.
願意你自己的聖言成為我們今天的力量.
洪主明求 阿門.
\newpage



\section{使徒行傳 23:1-35-20220409}
\label{sec:9oqAyDoUD6A}
\textbf{【網上崇拜】害怕危險才是真正的危險|使徒行傳23\_1-35|20220409 [9oqAyDoUD6A]}
\newline
\newline
連結: \href{https://youtube.com/watch?v=9oqAyDoUD6A}{\texttt{ https://youtube.com/watch?v=9oqAyDoUD6A}} ~~~~ 語音日期: 2022-04-09 
\newline
\newline
\hyperref[sec:rbtYKzrN9IU]{\small{< < < PREV SERMON < < <}}
~
\hyperref[sec:index_chronic]{\small{[返順時目]}}
~
\hyperref[sec:index_scriptual]{\small{[返順卷目]}}
~
\hyperref[sec:9wD1Nl5xbCs]{\small{> > > NEXT SERMON > > >}}
\newline
\newline
使徒行傳 23:1-35-20220409
\newline
\begin{longtable}{cl}
\hline
\hline
章節 & 經文 (和合本修訂版)\\
\hline
23:1 & \begin{tabularx}{0.7\textwidth}{X} 保羅定睛看著議會的人,說:「諸位弟兄,我在神面前,行事為人都是憑著清白的良心,直到今日。」 \end{tabularx} \\ \\ \relax
23:2 & \begin{tabularx}{0.7\textwidth}{X} 亞拿尼亞大祭司就吩咐旁邊站著的人打他的嘴。 \end{tabularx} \\ \\ \relax
23:3 & \begin{tabularx}{0.7\textwidth}{X} 這時,保羅對他說:「你這粉飾的牆,神要打你!你坐堂是要按律法審問我,你竟違背律法,命令人打我嗎?」 \end{tabularx} \\ \\ \relax
23:4 & \begin{tabularx}{0.7\textwidth}{X} 站在旁邊的人說:「你竟敢辱罵神的大祭司嗎?」 \end{tabularx} \\ \\ \relax
23:5 & \begin{tabularx}{0.7\textwidth}{X} 保羅說:「弟兄們,我不知道他是大祭司;因為經上記著:『不可毀謗你百姓的官長。』」 \end{tabularx} \\ \\ \relax
23:6 & \begin{tabularx}{0.7\textwidth}{X} 保羅看出他們一部分是撒都該人,一部分是法利賽人,就在議會中喊著:「諸位弟兄,我是法利賽人,也是法利賽人的子孫。我現在受審問是為有關死人復活的盼望。」 \end{tabularx} \\ \\ \relax
23:7 & \begin{tabularx}{0.7\textwidth}{X} 說了這話,法利賽人和撒都該人爭論起來,會眾分為兩派。 \end{tabularx} \\ \\ \relax
23:8 & \begin{tabularx}{0.7\textwidth}{X} 因為撒都該人一方面說沒有復活,另一方面沒有天使和鬼魂;法利賽人卻承認兩方面都有。 \end{tabularx} \\ \\ \relax
23:9 & \begin{tabularx}{0.7\textwidth}{X} 於是大大地爭吵起來;有幾個法利賽派的文士站起來爭辯說:「我們看不出這人有甚麼錯處;說不定有鬼魂或者天使對他說過話呢!」 \end{tabularx} \\ \\ \relax
23:10 & \begin{tabularx}{0.7\textwidth}{X} 那時爭辯越來越大,千夫長恐怕保羅被他們扯碎了,就命令士兵下去,把他從眾人當中搶出來,帶進營樓去。 \end{tabularx} \\ \\ \relax
23:11 & \begin{tabularx}{0.7\textwidth}{X} 當夜,主站在保羅旁邊,說:「放心吧!你怎樣在耶路撒冷為我作見證,也必怎樣在羅馬為我作見證。」 \end{tabularx} \\ \\ \relax
23:12 & \begin{tabularx}{0.7\textwidth}{X} 到了天亮,猶太人同謀起誓,說「若不先殺保羅就不吃不喝」。 \end{tabularx} \\ \\ \relax
23:13 & \begin{tabularx}{0.7\textwidth}{X} 參與這陰謀的有四十多人。 \end{tabularx} \\ \\ \relax
23:14 & \begin{tabularx}{0.7\textwidth}{X} 他們來見祭司長和長老,說:「我們已經發了重誓,若不先殺保羅就甚麼也不吃。 \end{tabularx} \\ \\ \relax
23:15 & \begin{tabularx}{0.7\textwidth}{X} 現在你們和議會要通知千夫長,叫他把保羅帶到你們這裡來,假裝要詳細調查他的事;我們已經預備好,在他來到這裡以前就殺掉他。」 \end{tabularx} \\ \\ \relax
23:16 & \begin{tabularx}{0.7\textwidth}{X} 保羅的外甥聽見他們設下埋伏,就來到營樓裡告訴保羅。 \end{tabularx} \\ \\ \relax
23:17 & \begin{tabularx}{0.7\textwidth}{X} 保羅請一個百夫長來,說:「你領這青年去見千夫長,他有事告訴他。」 \end{tabularx} \\ \\ \relax
23:18 & \begin{tabularx}{0.7\textwidth}{X} 於是百夫長把他領去見千夫長,說:「被囚的保羅請我到他那裡,求我領這青年來見你;他有事告訴你。」 \end{tabularx} \\ \\ \relax
23:19 & \begin{tabularx}{0.7\textwidth}{X} 千夫長就拉著他的手,走到一旁,私下問他:「你有甚麼事告訴我呢?」 \end{tabularx} \\ \\ \relax
23:20 & \begin{tabularx}{0.7\textwidth}{X} 他說:「猶太人已經約定,要求你明天把保羅帶到議會去,假裝要詳細查問他的事。 \end{tabularx} \\ \\ \relax
23:21 & \begin{tabularx}{0.7\textwidth}{X} 你切不要隨從他們,因為他們有四十多人埋伏,已經起誓,若不先殺掉保羅就不吃不喝。現在都預備好了,只等你的允准。」 \end{tabularx} \\ \\ \relax
23:22 & \begin{tabularx}{0.7\textwidth}{X} 於是千夫長打發那青年走,囑咐他:「不要告訴人,你已將這些事報告我了。」 \end{tabularx} \\ \\ \relax
23:23 & \begin{tabularx}{0.7\textwidth}{X} 於是,千夫長叫了兩個百夫長來,說:「預備步兵二百、騎兵七十、長槍手二百,今夜九點往凱撒利亞去; \end{tabularx} \\ \\ \relax
23:24 & \begin{tabularx}{0.7\textwidth}{X} 也要預備牲口讓保羅騎上,護送到腓力斯總督那裡去。」 \end{tabularx} \\ \\ \relax
23:25 & \begin{tabularx}{0.7\textwidth}{X} 千夫長又寫了公文,大略說: \end{tabularx} \\ \\ \relax
23:26 & \begin{tabularx}{0.7\textwidth}{X} 「克勞第‧呂西亞向腓力斯總督大人請安。 \end{tabularx} \\ \\ \relax
23:27 & \begin{tabularx}{0.7\textwidth}{X} 這個人被猶太人拿住,快被殺害時,我得知他是羅馬人,就帶士兵下去,把他救了出來。 \end{tabularx} \\ \\ \relax
23:28 & \begin{tabularx}{0.7\textwidth}{X} 因為我要知道他們告他的罪狀,就帶他下到他們的議會去。 \end{tabularx} \\ \\ \relax
23:29 & \begin{tabularx}{0.7\textwidth}{X} 我查知他被告發是因他們律法上的爭論,並沒有甚麼該死或該監禁的罪名。 \end{tabularx} \\ \\ \relax
23:30 & \begin{tabularx}{0.7\textwidth}{X} 後來有人把要害他的計謀告訴我,我立刻把他解到你那裡去,又命令告他的人在你面前告他。」 \end{tabularx} \\ \\ \relax
23:31 & \begin{tabularx}{0.7\textwidth}{X} 於是士兵照所命令他們的,連夜把保羅帶到安提帕底。 \end{tabularx} \\ \\ \relax
23:32 & \begin{tabularx}{0.7\textwidth}{X} 第二天,由騎兵護送保羅,他們就回營樓去。 \end{tabularx} \\ \\ \relax
23:33 & \begin{tabularx}{0.7\textwidth}{X} 騎兵來到凱撒利亞,把公文呈給總督,就叫保羅站在他面前。 \end{tabularx} \\ \\ \relax
23:34 & \begin{tabularx}{0.7\textwidth}{X} 總督讀了公文,問保羅是哪一省的人;一知道他是基利家人, \end{tabularx} \\ \\ \relax
23:35 & \begin{tabularx}{0.7\textwidth}{X} 就說:「等告你的人來到,我才詳細聽你。」於是他命令把保羅拘留在希律的衙門裡。 \end{tabularx} \\ \\
[1ex]
\hline
\hline
\end{longtable}
$^{1}$時勢真惡,但我相信在詩歌的內容和旋律療癒人心.
對於我們在這一刻更加會去感受.
能夠在一個群體當中敬拜,互相支持.
再已經不是地域阻隔,更加是在當中提醒我們.
我們不孤單,但我們更加要懂得依靠上帝.
在當中更加要去看歷史的發展過程.
對我們來說,其實是甚麼一回事.
是否現在特別差,還是以前也很差.
只不過我們沒有認真看聖經.
或者沒有認真去看過去本著基督信仰的人怎樣過活.
危險從來都是我們在生命當中常常要提醒自己.
但你有多害怕危險呢?.
而為了害怕危險時,而忽略了真正的危險.
在講題上可能有少少矛盾.
但盼望今天的經文和大家再一次了解.
我們的反合性真理,對我們來說正正是上帝要我們去明白.
上帝常常有很多似是而非,但對我們來說是很真實的.
今天選的經文是《士郎行傳》第23章.
整章的經文,有時你也會看到我講敘事的經文比較長.
因為聖經很多時候都不是一個斬鍵的片段.
而是一個生命的歷練和經歷.
比我們只看一個片段時,忽略了一個大圖畫.
其實《士郎行傳》第23章的內容的描述.
應該由我上一講,講保羅在羅馬書最後的終章結論時.
已經透露了少少.
在過去上一個月題Outflow,我就用了彼得講了兩個講題.
今天月位的月題,我用保羅也講了,現在是第二講.
在上一個月的講道當中,我曾經引用了這一段經文.
就是《士郎行傳》第21章第27節.
就是保羅帶了一群人去聖殿.
那群猶太人看不過眼.
以色列人拿來幫助,這就是在各處教訓眾人踐踏我們百姓和律法並且地方的.
他又帶著希臘人進殿,污穢了這聖地.
由此,也將開始有技術.
然後撒冷呢,騷動.
帶到一個點就是,連羅馬政府都覺得出事了.
於是要立即調兵來處理鎮壓,騷亂,看看發生了什麼事.
然後21章之後,22章都是在記述那群民眾在騷動.
而保羅在其中面對的,被人質問.
今天23章的經文就是,保羅被人提案,受審.

$^{41}$23章的經文,我們今天開始看的內容一轉就會是這個.
保羅定睛看著公會的人說,弟兄們我在上帝面前行事為人都是憑著良心,直到今天.
大祭司阿蘭妮婭就吩咐旁邊站著的人打他的嘴.
保羅對他說,你這粉色的牆上帝要打你?.
你坐堂為的是按律法審問我,你竟違背律法吩咐人打我嗎?.
站在旁邊的人說,你辱罵上帝的大祭司嗎?.
保羅說,弟兄們,我不曉得他是大祭司.
經世人記著說,不可違背你百姓的官長.
這裡有一點轉折,保羅被帶到公會審問的時候.
保羅第一個他自己的信仰告白就是.
其實我一直以來都憑著上帝給我的良心去做事.
直到今天,你看到馬上就被人打嘴.
其實為什麼會被人打嘴呢?.
因為對於當時要說這句話之後.
他們聽入耳就是,你基本上是在褻瀆上帝.
而打你是因為你的態度不好.
為什麼說良心說話都說是打呢?.
在大祭司他們的角度來說就是.
你帶了不應該帶的人去聖殿.
你毀壞了我們祖宗的教訓和常規.
而這件事是不容許的.
你還說這樣就是本著上帝的良心.
你基本上就是在侮辱上帝.
所以他就掌他的嘴.
但是保羅他馬上就反過來.
他又馬上說,你只是粉飾一場,上帝要打你.
意思就是你應該要審問我.
現在你反而吩咐人打我.
其實粉飾一場這個詞又不是沒有出現過.
如果你看到這封書.
耶穌基督在進入聖州的日子.
準備預戰晚山之先.
其實他在聖殿教訓眾人的時候.
都曾經在門徒面前說.
「凡他們教訓你們的,你們也要遵守.
因為他們坐在摩西的位上.
但不要效法他們,因為他們能說不能行」.
隨後就說那些文士和快才人.
他們是粉飾的墓.
其實這句話正正耶穌都說過.

$^{81}$今天保羅再說.
有些事情不是說就行的.
有些事情不是你以權就可以做的.
正正就是上帝是什麼一回事.
其實你自己知不知道呢.
對於整件事來說.
保羅仍然是憑著上帝給他的良心.
去做好他自己的本分.
當然有些人就看不起.
就覺得你不是.
第四節裡說的一句話就是.
你敢辱罵上帝的大祭司嗎.
如果按照你的說法.
保羅沒理由不認識大祭司.
起碼大祭司有件衣服可以認得出.
但因為那時候不是很趕忙的時間.
他們全部都走去工會就要審保羅.
或許他真的沒穿大祭司的袍.
保羅認不出.
又或許是因為保羅都離開了.
然後殺了十幾年.
他又不認得那幫人.
而且阿拉尼亞他不是做了很久.
對於你來說保羅不認得他是不奇怪的.
所以他說這番話就直戚其非.
就被人掌嘴.
但當有人說這是大祭司的時候.
保羅就馬上說得很清楚.
其實我不知道他是大祭司.
如果知道的話.
保羅其實都會遵守禮節.
或者尊卑之分.
因為這條經文都說得很清楚.
就是不可毀謗百姓的觀長.
保羅是知道的.
保羅不是他們眼中所謂.
指高棄揚.
又或者是自嘲傲默的人.
保羅不是.
保羅仍然清楚自己言語行事.

$^{121}$他事就說事.
不事就說不事.
這種信徒的品格.
是我們要學習的.
能夠在什麼環境都說得出.
他的言語行為是本著良心.
這是很重要的.
去到第六節的時候.
保羅看出大眾一半是薩都哥人.
一半是法利塞人.
就在公會中大聲說.
弟兄們.
我是法利塞人.
也是法利塞人的子孫.
我現在受審問.
是為盼望死人復活.
這個說出現在的本意是什麼.
第七節.
書了者說.
法利塞人和薩都哥人就爭論起來.
會眾分成兩黨.
因為薩都哥人說沒有復活.
也沒有天使和鬼魂.
法利塞人卻說兩樣都有.
於是大大地喧嚷起來.
有幾個法利塞黨的民事站起來爭辯說.
我們看不出這人有什麼惡處.
倘若有鬼魂或天使對他說過話.
怎麼樣呢.
這段經文的分段很特別.
保羅看到原來有不同的人在當中.
他就要說出一個重點.
就是死人復活.
我希望你不要看了保羅看到有兩群人.
於是挑撥爭端.
其實死人復活這個主題.
或者這個訊息.
保羅一直都有說.
或者在《士徒行傳》.
你會看到彼得有機會跟猶太人說的時候.

$^{161}$都是說你們釘死的耶穌.
就是我們等待的彌賽亞.
而他是受死埋葬第三天復活.
有機會要宣揚耶穌基督工作的時候.
司徒仍然是說耶穌復活的大能.
正正要讓人去明白到.
我們持守的律法已經不是能夠使我們.
使罪得赦的律法.
我們持守的律法.
也不能夠救我脫離死亡的律法.
而真正能夠令到我們有永生的.
仍然是耶穌基督並他復活的大能.
這是保羅想說得清楚.
所以不是看到兩群人於是挑撥離間.
從中獲取二人之利.
就可以自己脫身.
不是這個意思.
保羅仍然是本著要說得清楚.
因為他們兩群人都是守律法.
守到另外一個極端.
而排除了任何的可能性.
保羅說得清楚就是.
我們的盼望就是死人復活.
這是很重要的重點.
但是否能夠接受呢.
你看到就不能接受了.
於是就在爭拗.
就在圈養.
但是凡爾賽人有一個可愛的地方就是.
其實他們相信死人復活.
於是他們說.
其實你看看這個人有什麼惡處.
可能他真的收到一些訊息.
或者會不會再聽下去呢.
但是到了這個時候.
事實就已經很混亂了.
混亂的一點就是那時大起爭吵.
千夫長恐怕保羅被他們扯碎了.
就吩咐兵丁下去.
把他從眾人當中搶出來.

$^{201}$帶進刑樓去.
當夜主站在保羅旁邊說.
放心吧.
你怎樣在耶路撒冷為我作見證.
也必怎樣在羅馬為我作見證.
在這裡你會看到.
真正仍然是宣講上帝說話的人.
上帝要他同在.
你會看到有很多人仍然密守一些.
他們覺得持之以恆有效.
又或者他賴以為生的據點.
對於我們來說.
我們有多開放能夠接受不同的意見.
又或者有多開放放棄自己.
一直持守的意見呢.
這是歷代歷世都不容易的.
過去不同宗派的出現.
其實都是不同神學和解經的立場不同.
然後就演變了有不同的宗派出現.
過去都會為了神學.
為了一些聖經的信息.
有不同的爭辯.
有不同的解說.
但是慢慢去到近代的教會.
因為聖經和神學很多論述已經奠立了.
反而現在討論的就是教會運作.
現在反而討論得最多.
就是教會如何處事.
如何去分辨.
但這個我自己在之前彼得講道的內容裡.
都說過.
就是處理安提柯教會和耶路撒冷教會.
論到吃祭偶像之物的問題.
亦是有太多的爭拗.
而沒有任何那種妥協或者忍讓.
可以彼此欣賞看著上帝工作的寬容度.
今天其實教會仍然有很多大大小小的問題.
就好像疫情衍生出來的聚會安排.
其實很多教會都仍然討論到沒完沒了.
到接下來要恢復實體崇拜.

$^{241}$有現場的時候.
其實又是一大難題.
仍然知道有些是相對的情況.
但相對的情況下就牽涉到個人有不同的看法.
就令到問題有時不能夠三言兩語.
或者三兩次就解說得清楚.
保羅面對的問題.
他不是覺得自己可以解決.
但保羅的處事方法就是.
你們要知道的我盡力去說.
你肯聽的我盡力去解釋.
保羅要他們明白到.
有些絕對的我們要持守.
相對的我們要慢慢去欣賞和調和.
所以羅馬書上次我引用都在說.
上帝的國不是在乎吃喝.
是在乎公義和平與聖靈的喜樂.
今天在教會的不同意見表達.
教會不同的真詞.
有不同的運作方式的時候.
我們能不能夠仍然在一主一信一洗一聖靈當中.
感受到那種喜樂.
真的不容易.
但保羅仍然提醒我們.
有些事情先說了.
一定要繼續去執行.
但仍然憑著良心繼續下去.
第十一節就是.
放心吧.
你怎樣在耶路撒冷見證我.
你都會在羅馬怎樣去作見證.
上帝仍然清楚我們在做什麼.
上帝一直都看著我們.
在困難當中上帝都看著我們.
去到第十二節.
剛才說到半夜他們叫我們出來公會審問.
天亮了.
猶太人就同謀起誓說.
若不先殺保羅就不吃不喝.
很大決心.

$^{281}$第這樣同心起誓的有四十多人.
他們來見祭司長和長老說.
我們已經在這個大勢.
若不先殺保羅就不吃什麼.
現在你們和公會要知道千夫長.
叫他帶保羅到你們這裡.
假作詳細考察他的事.
我們已經預備好了.
不等他來到跟前就殺他.
保羅的外甥男聽見他們徹下埋伏.
就到邢裡告訴保羅.
保羅請一個白夫長來說.
你領著少年人去見千夫長.
他有事告訴他.
你會看到這個場景是什麼呢.
有些人義正詞嚴.
覺得自己真的在做大事.
做對的事.
這令我聯想起.
為什麼他們會這麼有勢力.
為什麼一定要置保羅於生死.
慢慢你會看到.
事件是不斷重複發生.
今天選的第23章經文.
你會看到的情況是.
去到公會審問.
而問的問題就是復活的問題.
而復活的問題解說不了之後.
有些人就已經想著私下.
下私刑要殺死他.
其實這段經文我特意引用.
就是去預備下個星期.
明天主日開始就是聖週.
明天就是耶穌基督.
騎驢進城.
預備進耶路撒冷.
然後上各個他山受苦的路.
這段經文希望大家做一點理解.
其實耶穌進耶路撒冷之前在做什麼呢.
他經過伯大利去伯伯大.

$^{321}$上艦蘭山.
然後進入喀倫溪.
然後進入耶路撒冷.
伯大利是什麼地方呢.
伯大利就是他當日.
令拉撒路復活的地方.
而伯大利這個地方名聲一響.
就是因為死人能夠復活.
但《約翰福音》第十一章記載.
耶穌行了神蹟的時候.
描述的最後一段話是什麼呢.
眾人就相意要殺害耶穌.
因為祂能夠使死人復活.
其實有人看到死人復活是開心.
但有人看到死人復活是不開心的.
因為這個人能力之大.
他說話行事都大有能力.
今天更加能夠使死人復活.
所有人都聽他的.
所以有些人看到死人復活神蹟.
是不開心的.
從那天就相意要殺害耶穌.
但耶穌已經預備上各個他山上的路.
你看到今天.
保羅去到耶路撒冷的時候.
保羅去到說死人復活是我們基督信仰最重要的核心.
也是能夠救我們脫離罪惡的困所.
最重要的福音的時候.
但是人是接受不了的.
他們更加要相意要殺害保羅.
其實人能不能夠在歷史當中面對教訓呢.
說就懂得說.
但在當中的時候.
有沒有一個聆聽的心.
有沒有一個信任.
有沒有一個可以理據表達清楚的過程呢.
真的不容易.
但保羅仍然盡他能力去說清楚我們要知道的事情.
第十八節.
「於是把他領去見千夫長說:.

$^{361}$「被囚的保羅請我到他那裡求我領這少年來見你,他有事告訴你.」.
千夫長就拉著他的手走到一旁私下問他:.
「你有什麼要告訴我?」.
他就說:.
「猶太人已經約定要求明天帶保羅到崩壞那裡去,假作他要詳細查問他的事,.
你切不要隨從他們,因為他們有四十多人埋伏,已經起誓說:.
若不先殺保羅,就不吃不喝,現在已經預備好了,只等你應運.」.
於是千夫長打發少年人走,祝福他說:.
「不要告訴人,你將這事報給我.」.
其實很大件事.
因為保羅自己本身都有羅馬公民的身份.
而這件事還沒有查夠清楚.
你們就自己私下用刑.
其實都解決不了問題.
去到第23節.
於是千夫長就做事了.
「千夫長便叫兩個白夫長來說:.
預備步兵二百,馬兵七十,長槍手二百,今夜開始往凱撒利亞去,.
也要預備牲口叫保羅騎上,護送到巡撫菲利斯那裡去.」.
千夫長又寫了文書大律說:.
「克勞蒂·呂西亞,請巡撫菲利斯大人案,.
這人被猶太人拿著,將要殺害,.
我得知他是羅馬人,就帶兵丁下去救他出來,.
因為知道他們告他的緣故,我就帶他下到他們的公會去,.
便查知他被告是因他們的立法辯論,.
並沒有什麼該死該綁的罪名.」.
你看到是不是很像?.
你看到在耶穌基督也是這樣,.
當日他們將耶穌拋來拋去,.
由公會再去到比拉多的時候,.
他們都是查不到有什麼罪名,.
都是主要處理耶穌罵他們,.
或者不滿耶穌所做的行徑,.
或者他們的能力的時候,.
但其實是沒有解決問題的,.
他們就要釘死耶穌..
其實很大陣仗,你見到人已經不少了,.
但已經說得很清楚,.
保羅其實也沒有致死的罪,.
只不過那班人是要殺害保羅,.

$^{401}$如同殺害耶穌一樣..
去到第30節,也是這個章節的尾段,.
「後來有人把要害他的計謀告訴我,.
我就立時解他到那裡去,.
又吩咐告他的人在你面前告他..
於是冰冰照所吩咐的他們的,.
將保羅夜裡帶到安提帕底,.
第二天讓馬兵護送,他們就會迎來..
馬兵來到凱撒利亞,.
把文書呈給巡撫,便叫保羅站在他面前..
巡撫看了文書,問保羅是那山的人,.
既曉得他是基里加,就說:.
「等告你的人來到,我要詳細聽你的事,.
便吩咐人把他看守在希臘的衙門裡.」.
你會見到轉接到另一個場景,.
轉接到另一個場景,就是第24章開始,.
其實每轉一個場景,.
保羅都再說一次他來耶路撒冷的目的,.
亦說上帝要他說的話,.
就是見證耶穌基督是生命的主,.
復活的主..
在今天讀的經文裡,.
有兩節經文抽了出來,.
第一節和第十一節..
你會見到在第一節的時候,.
「弟兄們,我在上帝面前行事,.
都是憑著良心,直到今天.」.
他的良心所做的事情就是見證上帝,.
而他見證上帝的時候,.
上帝也honor他,也confirm他,.
你放心說吧,.
你怎樣在耶路撒冷見證,在羅馬同樣都是,.
即是在預言保羅仍然有不同的見證要做..
所以到了第26章,.
即是在阿基帕王面前的時候,.
他仍然是在說他的生命見證,.
和基督福音核心的信息..
這個正正就印證了,.
《士徒行傳》,路加醫生要表達,.
一班門徒,他們的生命是什麼?.

$^{441}$就是第一章第八節..
「聖靈降臨在你們身上,你們就必得能力,.
並要做耶路撒冷,猶太傳遞,.
莫撒瑪利亞,直到地極作我見證.」.
《士徒行傳》從開初到末了,.
由彼得在頭半段做主線,.
到後半段保羅做主線,.
仍然是在見證耶穌基督,並祂復活的大能..
他們的心,他們的良心,.
就是在做他們當初所接受的事情,.
驅使他們繼續做下去..
親愛的姐妹,時勢真惡,.
我們在這麼患難的環境當中,.
有什麼驅使我們繼續堅守下去,.
和繼續做下去呢?.
其實常常都要問問自己,.
從而提醒自己..
我之前也說過,我每天早上起床,.
我都會看著鏡子,.
跟自己說,有什麼要做好,.
有什麼要想清楚..
這是我每天都在數算日子的時候,.
我都不斷地提醒自己..
你可能覺得不用這麼認真,.
或者不需要這麼慘,.
但我真的覺得,.
在有生命氣息的日子,.
很多時候都實事求是,.
我常常喜歡做了就是了..
在這兩年,我少做了很多運動,.
很多弟兄姐妹都知道我,.
有做運動的習慣,我少做了運動..
有些運動就沒有做,.
看了就當做了,.
看NBA就當打了球,.
雖然我常常看,.
看了就當打了,.
還有我喜歡看的運動,.
是我這輩子都不會做的,.
因為我沒有能力做,.

$^{481}$也沒有空間做..
我喜歡看不同大型的比賽,.
奧運會之類的,.
其中有一個運動,.
是我這輩子都不會做的,.
我年輕的時候玩過,.
始終那時候是高消費活動,.
我沒有玩過,現在長大了,.
年紀怕跌倒,不玩了..
不知道大家有沒有玩過這個運動,.
我給大家看一段影片..
大家有看過這段影片嗎?.
這段影片是這個人,.
給大家看一看圖畫,.
他被稱為冰上王子,.
與生結緣,.
這是他IG的照片,.
截圖下來,.
為什麼他被稱為冰上王子呢?.
他2014年和2018年.
都分別拿了兩屆.
冬季奧運會花式滑冰金牌..
在過去的冬季奧運會賽前預告,.
他如無意外,.
最大機會是三連霸..
於是很多日本人訪問他,.
剛才那段影片是2021年的預賽,.
他嘗試做四周半空中轉身,.
我不認識4A的運動,.
所以我選了這段影片給大家看,.
以慢鏡數到1,2,3,4,.
如果看原本的影片,.
你會覺得轉完了,.
差不多而已,.
但他真的認真做到,.
是高難度的動作,.
人們問他會否再拿到金牌的感覺如何,.
他反而說了這番話,.
他說:.
「我失去了四年前在屏昌那時.

$^{521}$那份必須贏金牌的感覺,.
比起收穫一塊金牌,.
我更加渴望能夠完成一次完美的4A旋轉.」.
訪問一出後,.
有不同的人更追問,.
其實你不拿到金牌,.
會否不合適,.
或不是太好,.
言下之意是,.
金牌除了屬於你,.
也是屬於國家的榮耀,.
還有,.
你有沒有考慮其他東西,.
會否過份執迷自己的挑戰,.
而忽略了其他人的考慮,.
但其實,.
他說得很清楚,.
對於他來說,.
金牌固然是一個肯定,.
但更加要追求自己突破的空間,.
他說自己年紀開始大了,.
如果如常發揮的話,.
其實很大機會會再拿到金牌,.
但正正因為自己年紀大,.
他更加不想,.
所謂我們香港的說話,.
就是吃老本,.
又或者叫穩定,.
就能夠做到自己的事情,.
他更加要提升自己,.
要追自己其實想做的事情,.
他真的很希望,.
能夠在一個大型的賽事當中,.
去做到他自己生命當中最大的挑戰,.
就是一個4A轉身,.
所以他不斷苦練,.
不斷提升,.
讓自己有更多的空間,.
他不想成熟的自己,.
不斷的因循,.

$^{561}$或者定格,.
而失去了冒險的精神,.
有些人說,.
這樣很危險的,.
萬一失手跌傷的時候,.
你連其他出賽的機會都沒有,.
但他的答案是,.
怕危險才是真正的危險,.
因為你已經沒有能力再有機會超越,.
或者追求自己最想追求的東西,.
當然有很多不同的意見,.
也成為他不同的壓力,.
但他仍然在剛剛的冬季奧運會,.
做他自己最大的挑戰,.
結果是什麼呢?.
結果就是失手,.
落地不穩,.
滑倒了,.
如果你有看新聞,.
我那晚有看新聞報告,.
也特意拍下這幾個鏡頭,.
我並不懂日文,.
於是就沒有看日文網站,.
但我懂中文,.
我立刻上台灣網站,.
台灣網站很厲害,.
立刻把日文訪問變成中文字幕,.
我就受惠了,.
我看看,.
有些人訪問他,.
如果你還記得,.
最後有很多人等PuPu出來,.
答謝他,.
他做了一件事,.
他仍然堅持他自己想跳出自己的安書,.
或者給自己更大的挑戰,.
就是做好自己想追求的4A轉身,.
最後的結果就是,.
他三甲不入,.
拿第四,.

$^{601}$但是冬季的委會,.
和滑冰的委員會,.
就承認了他能夠完成了4A,.
即是四周半的轉身,.
他真的能夠成就了他一心所追求,.
就是在大型的公開賽事當中,.
完成了他所定下的目標,.
縱使他失去了金牌這個殊榮,.
他失去了別人很緊張,.
或者很想有的三年半的清譽,.
很多人覺得這是一個危險,.
很多人覺得他任性,.
很多人覺得你罔顧了很多所謂公眾利益,.
但是他仍然知道他的初心是什麼,.
因為之後,.
有些人訪問他,.
他是這樣說的,.
他說,.
有什麼驅使你要突破這個危險呢?.
有什麼驅使你可以繼續去做這個挑戰呢?.
他就跟人分享,.
其實每次跌,.
受傷都是很辛苦的,.
但是他心中的九歲小朋友,.
常常提醒他,.
繼續跳,.
直到現在這一刻..
你心中的小朋友有驅使你嗎?.
你當初,.
是什麼驅使你做你喜悅的事情呢?.
雨生竟然告訴我們,.
他心中九歲的小朋友,.
不斷地催逼他,.
你繼續跳吧,.
你行的,.
你再突破自己吧,.
讓他在當中享受這個挑戰,.
享受這個追夢的過程,.
他今天得到,.
不是別人期望的得到,.

$^{641}$而是他真正自己覺得可以挑戰,.
是不完美的結局,.
但是在他來說,.
是一個肯定..
在我看他這個經歷的時候,.
我就想,.
一個坊間的人,.
都很清楚自己的良心,.
怎樣去追下去,.
去持守..
我們既然有基督信仰,.
我們能不能夠有這個,.
上帝給我們的良心,.
復甦了的良心,.
我們可以再堅持下去呢?.
我們有沒有見證上帝的能力,.
我們有沒有這個風骨,.
可以在不得時的時候,.
仍然做到我們要堅持的內容呢?.
從來都不容易的,.
第一個星期,.
有位牧師就做了,.
不是離我們很遠的,.
但是,.
不是容易不容易,.
是做不做..
我們每一個都嘗過主人的滋味,.
是經歷過重生,.
但是我們心中的小朋友,.
有沒有繼續提醒我們呢?.
聖靈,.
就是心中常常提醒我們的位置,.
我們有沒有聽聖靈的提醒和催迫呢?.
我不是說,.
每個人都像牧師一樣,.
不是,.
每個人有自己不同的崗位,.
每個人有自己不同要面對的事情,.
但是我們仍然可以做到,.
很少,.

$^{681}$但是仍然有,.
他人有份的崗位,.
在上一個月,.
我講每道微小的訊息的時候,.
我都說了,.
每個人做自己那一線的光芒,.
不知不覺你的光就照在人前,.
三月面對一個問題,.
就是有機會全城都要封鎖,.
留在家裡,.
我就想,.
在家裡有什麼可以做呢?.
於是就做了這個Delivered Poon,.
和你一起吃飯,.
在這個,.
Delivered Poon裡面的節目當中,.
其實四個星期,.
每逢二四都吃,.
中間就見了這八個不同的弟兄姊妹,.
如果你有看過的話,.
其實他們都有各自的難處,.
各自不容易為外人道的經歷,.
或許他們說得輕鬆,.
或許他們說得輕描淡寫,.
但是我邀請他們分享的時候,.
其實正正他們都經歷過很多難處,.
我覺得他們的生命經歷,.
都成為很多弟兄姊妹可以分享,.
我在最後一集Delivered Poon都說,.
這個節目不是要推廣很千奇百趣,.
或者很多特別,.
我只是希望大家在限制的空間之內,.
仍然可以越過限制,.
做更大的事,.
以前我們聽見證,.
很多時候都是大型聚會聽見證,.
現在哪有大型聚會,.
如果你要聽大型聚會才聽見證的話,.
其實很多事都不用說,.
但事實上很多弟兄姊妹在他們的生命當中,.

$^{721}$仍然有很多每天都經歷的難處,.
仍然有很多每天都經歷的,.
和實踐的,.
繼續信仰對他們的要求和提醒,.
不如試試想第二個方式吧,.
我坦白說,.
由第一個開始到最後,.
其實小則都有七百多個觀看,.
大則有一千八百個觀看,.
你就當每個節目真是有一成人認真看完,.
如果一千八百個觀看,.
有一百個人認真看,.
你就有一百個人認真聽見見證,.
不知不覺將信息繼續傳遞出去,.
七百個人看過,.
就可能有七十個人看,.
那認真就是讓我們可以離開本位,.
再傳遞出去,.
事實上不能讓我們有大型聚會,.
但我們仍然可以越過本位,.
將我們基督而有的見證,.
可以送到個人的手中,.
聽進個人的耳中,.
我們仍然可以由太傳地,.
撒瑪利亞直到地極,.
去宣揚上帝給我們的信息,.
如果單看標題,.
這是不邏輯的,.
害怕危險才是真正危險,.
知道害怕的時候才會懂得迴避,.
但我們的信仰常常有一個反合性的真理,.
就是似是而非,.
因為我們有很多處境,.
在當時一刻覺得不可行,.
但當我們經歷信仰的時候,.
上帝又成為可能,.
保羅在羅馬提醒過那麼多人,.
那些外邦人不是很明基督信仰,.
但他提醒,.
上帝為人所預備的是眼睛未曾看見,.

$^{761}$耳朵未曾聽見,.
連人心也未曾想見的事情,.
親弟姐妹不要讓自己所有的認知,.
限制了上帝在我們當中的作為,.
也不要讓我們所知的,.
來限制了上帝在我們當中所要發放的信息,.
或許有一刻是我們不明白,.
但我們開放,我們信任,.
有時候我們真的很努力做一些事情,.
最後還是不成功的,.
但如果我們怕了,.
我們就忽略了真正的危險,.
我們怕聚會,.
我們就忽略了很多人未聞福音,.
我們怕宣講,.
就很多人不知道真理,.
保羅正正就是不害怕,.
讓更多人聽得到,.
用他最後的能力,.
讓猶太人的茅室要開,.
最後,.
我希望大家在這日子要認真,.
也希望弟兄姐妹一起去參與,.
因為從來沒有人可以阻止我們感受和見證上帝的過程,.
上帝是真神,.
他是一個Living God,.
他和我們一起,.
他肯定保羅說,.
你放膽說吧,.
他與我們同在,.
今天仍與我們同在,.
我希望弟兄姐妹我們一起感受上帝與我們同在,.
無論得時不得時,.
我們每一個敬拜的真實,.
我們越開我們的本位,.
上帝與我們同在,.
是不得時的時候,.
但是上帝會加力給我們,.
You are not alone,.
We are together,.

$^{801}$我們一起祈禱,.
主人妹當我們打開你的說話,.
你讓我們看到的就是歷史是不斷地重演的,.
昔日主耶穌走過他山上的路,.
保羅他同樣地與你一樣,.
去見證你,.
去宣揚你,.
仍然被審,.
被殺害,.
今天這些日子可能會再出現,.
但是不得時的日子已經出現了,.
求主你再讓我們經歷你,.
而我們在當中再一次得到你的能力,.
我們感受你的同在,.
我們參與見證你的過程,.
是我們最光輝,.
最享受的日子,.
求主你閱立我們誠心誠意的敬拜,.
無論什麼地方都好,.
我們都是心相連,.
彼此支持,.
有願主你提醒我們,.
明天是你騎驢進城聖舟的開始,.
這個星期是主耶穌在地上最艱難的一個星期,.
有願意我們帶著一個反思,.
警醒的心智去迎接受難周,.
求主你的靈提醒我們,.
我們祈禱奉耶穌的名字,.
阿門..
\newpage



\section{}
\label{sec:9wD1Nl5xbCs}
\textbf{【網上受難節聚會】|反思疫症下的基督受難|20220415 [9wD1Nl5xbCs]}
\newline
\newline
連結: \href{https://youtube.com/watch?v=9wD1Nl5xbCs}{\texttt{ https://youtube.com/watch?v=9wD1Nl5xbCs}} ~~~~ 語音日期: 2022-04-15 
\newline
\newline
\hyperref[sec:9oqAyDoUD6A]{\small{< < < PREV SERMON < < <}}
~
\hyperref[sec:index_chronic]{\small{[返順時目]}}
~
\hyperref[sec:index_scriptual]{\small{[返順卷目]}}
~
\hyperref[sec:LSvYW2YLyv0]{\small{> > > NEXT SERMON > > >}}
\newline
\newline
$^{1}$頂姐妹平安.
真日.
在真日子裡面平安.
彷彿好像一個已經越來越變得陌生.
越來越不容易.
越來越值得珍重的事情.
頂姐妹你平安嗎.
在這幾個月裡面.
第五波的疫情當中.
從上年的十二月到現在.
教會的聚會被禁止.
停晚市堂食.
不能夠聚會.
學校停課.
隔離.
強檢.
禁足.
倒閉.
失業.
封鎖.
停飛.
頂姐妹你最近好嗎.
不知不覺之間.
疫情已經去到第三年.
這三年.
或者我們已經逐漸忘記.
弟兄姐妹口罩背後的面容.
一些新相識的朋友.
甚至我們都未見過他們的面容.
彷彿口罩好像成為我們面容的一部分.
我們的快樂.
深深抑壓在口罩的背後.
非常疏離.
就讓我們在這個不快樂.
世界充滿著苦難的日子.
在全球疫情大流行尚未完結的四月十五日.
我們一起來反思基督的受苦.
今天沒有新的事情要說.
沒有新的道理.
沒有新的洞見.

$^{41}$也沒有什麼嶄新的教導.
今天我們回到古舊十教的主題.
耶穌基督在十字架上被釘.
受苦流血.
承擔我們的罪.
為我們眾人死.
我們得著醫治.
得著寶貴的新生命.
永生的生命.
作為一個這麼簡單卻是重要的道理.
古舊的十教今天仍然的屹立.
縱然今天我們只能在網上參與.
在家用手機用電腦.
遠遠的去瞻仰十字架.
但是這個十教並沒有改變.
依然屹立.
今天我們焦點只有一個.
基督耶穌的十字架.
所以在這個第五波的疫情尚未退去的壽難節聚會.
我們一起去思考兩件事.
我們思考病毒和基督的十字架.
或者可能第一次會問這兩件事有什麼關係.
其實這三年裡我們每天都在病患和死亡的旁邊.
去到抽搐.
口罩,消毒劑,探熱器,測試棒,防疫錦衣等等.
我們生命中最接近病患和死亡的三年.
大概就是這三年.
第五波的疫情至今的死亡數字已經達到八千多人.
甚至全球死亡數字已經累積到六百多萬.
雖然你每天都重複去聽見這些數字.
每天都收到這些數字的更新.
但我們可能都忘記這些真真實實喪掉生命的人.
一堆在醫院被處理的屍體.
正是沉重地教導我們思考死亡和疾病.
所以在這個疫情裡.
就讓我們再一次思考.
基督耶穌在十字架裡受難和死亡.
我們看的一段經文是以賽亞書的經文.
在受難節經常被引用的經文.
以賽亞書第53章4到6節.

$^{81}$經文這樣說.
祂承然擔當我們的憂患.
背負我們的痛苦.
我們卻以為祂受責罰.
被上帝擊打腐敗了.
誰知祂為我們的過犯受害.
為我們的罪孽壓傷.
因祂受的刑罰.
我們得平安.
祂受的鞭傷.
我們得醫治.
我們都如羊走米.
各人偏行幾路.
也說使我們眾人的罪孽.
都歸在祂身上.
我們一起祈禱.
主耶穌十字架裡拯救我們的你.
我們今天在世界各地不同的地方裡.
面對一個從未遇見的疫情.
我們在這個疫情當中.
我們一同紀念你的苦難.
思考我們今天作為你的門徒.
怎樣來面對.
求主你教導我們.
十字架的大能光照我們.
讓我們每個人.
今天疫情當中.
雖然有口罩和很多隔離.
但和你沒有任何阻隔.
求主你對我們說話.
禱告奉基督耶穌名字祈求.
阿們.
今天的經文我們稱之為「獨人之歌」.
整段經文的分段.
其實是從以賽亞書第52章13節開始.
直到第53章第12節.
整段經文的主題叫做.
「受苦的僕人」.
Suffering Servant.
可能我們都已經習以為常.

$^{121}$但以賽亞書這段經文裡.
其實是用了病患或病這個字.
來作為一個隱喻.
一個metaphor.
聖經說他承言擔當我們的憂患.
背負我們的痛苦.
中文和本翻譯憂患這個字.
更加準確的意思.
不是一個人的困難或憂慮.
而是直接一個很簡單的字.
就是病.
所以你會發現一些英文的本.
直接將它翻譯為sickness.
所以英文一個很簡單直接的翻譯就是.
他承言擔當我們的病.
背負我們的痛.
救世主耶穌在十誡上.
去擔當我們的病痛.
歷史上聖經有不少的例子.
用疾病來隱喻人類的罪.
上帝的拯救就用醫治來做隱喻.
方書裡有無數這樣的例子.
講述耶穌醫病.
耶穌醫治癱子.
耶穌醫治大麻瘋病人.
耶穌醫治流血病的婦人.
這些醫病的背後.
其實都是指向福音裡一個非常重要的主題.
往往就是一個人的得救.
方書裡有很多病人痊癒的故事.
其實正正就是指向耶穌基督的救恩.
這個病和罪.
醫治和拯救的關係.
最明顯大概就在路加福音裡所講.
路加福音第五章31節.
耶穌說沒有病的人不需要醫生.
有病的人才需要.
之後耶穌就接著這樣說.
我來本不是召義人悔改.
乃是召罪人悔改.

$^{161}$所以病和罪.
有病有罪.
醫治拯救.
是一個非常明顯的隱喻.
今天就讓我們一起.
用病和痛的角度.
去理解這句話.
他承言擔當我們的憂患.
背負我們的痛苦.
現在這個世代.
病是越來越多的.
在城市裡每個人都帶著一些病.
不要誤會.
這樣說不是說我們的醫療退步.
相反正正因為醫療進步.
我們才發現人類社會對於病的定義.
是越來越提高.
所以世上是多了很多病.
以前你不稱呼為病的東西.
現在都會叫做病.
以前我們可能叫做小朋友不專心.
現在就叫做過動症.
以前我們就叫做物輸很低分.
現在就叫做讀寫障礙.
大家有沒有發現.
現在小朋友越來越多是戴眼鏡.
我自己的女兒四歲就開始戴眼鏡.
不是因為有什麼特別.
而是因為小朋友四歲就開始要眼睛檢查.
很多小朋友就需要戴眼鏡.
所以你問發鼻氣是不是病.
你可能說不是.
但經常發鼻氣可能就叫做狂躁症.
不開心是不是病.
現在可能就叫做憂鬱症.
當然這是一個醫學哲學的問題.
什麼叫做健康.
什麼叫做病.
少少糖尿是不是病.
衣高是不是病.

$^{201}$頭髮少是不是病.
晚上八點鐘睡覺是不是病.
凌晨三點都不睡是不是病.
當然我們可以引經據典.
為病作出一些定義.
或者為健康作出一些定義.
所謂健康就是身心靈正常發揮.
運作到的人.
然後所謂病就是一切導致我們身心靈都不正常運作的情況.
但我們仍然可以追問下去.
什麼叫做不正常運作.
什麼叫做不正常.
肥是不是不正常.
近視是不是不正常.
身體矮是不是不正常.
身體高是不是不正常.
因為我們發現在正常和不正常之間.
在正常和不正常之間.
其實是存在著一個非常廣闊的灰色地帶.
醫學就叫做subnormal.
就是輕輕低於正常的標準.
所以究竟用什麼標準.
哪個人叫做正常.
那個標準在哪裡.
哪個人才叫做病.
用疫情做例子.
一個有帶著病毒的人.
沒有病徵.
這樣是不是叫做病呢.
當然我們不是要處理醫學哲學的問題.
可以回到我們一個信仰問題.
如果聖經是用病作為罪的隱喻的時候.
那麼聖經裡面健康的標準在哪裡.
病的標準又在哪裡.
如果耶穌告訴我們.
沒有病的人不需要醫生.
有病的人才需要.
我本來不是要照罪人悔改.
但是照罪人悔改的時候.
耶穌也告訴我們.

$^{241}$世界上沒有一個健康的人.
連一個也沒有.
世界上沒有一個人是沒有病的.
沒有一個不是罪人的人.
可能大家都認識的.
希伯來文shalom這個字.
平安.
其實這個字就是解作well being.
一個完整的人的意思.
一個完全完整整齊的人.
你是不是一個well being呢.
聖經告訴我們.
世界上沒有一個人是完整的.
我在神學院教神學很多年了.
每次我講到罪論的時候.
都會經常提醒自己.
要很小心的跟同學說有關罪的問題.
聖經說世人都犯了罪.
規決了神的榮耀.
世人都犯了罪.
但不代表有罪是一件正常的事.
我經常跟自己說.
要避免正常人都會犯罪.
或者犯罪很正常.
不正常的.
因為正常不是這樣定義的.
人人都有罪.
人人都會犯罪.
但不代表這個叫正常.
罪永遠都不是正常.
雖然我在正統來說是一個與疫共存論者.
但是疫情之下很多人都感染了.
我們都不能稱這個為正常.
與病毒共存.
我們都不會稱這個病為正常.
不過鄧小平你我也知道.
我們早就活在一個被罪感染的世界.
遠遠在COVID出現之前.
我們早就活在一個被罪感染的世界.
幾千年之久.

$^{281}$如果罪是一個病毒.
如果這個病毒是罪的時候.
這個病毒傳染性極高.
你無法躲避.
戴多少口罩都沒有用.
洗手都沒有用.
封城都沒有用.
沒有任何的疫苗.
打多少針都是無濟於事.
按照聖經的描述.
早在你沒有福的時候.
你就已經被感染.
從遠古的歷史開始.
直到如今.
一直都無法清零.
這個病毒破壞力極大.
它沒有影響你的呼吸系統.
它影響你的生命系統.
影響著整個世界的社會系統.
更可怕的是.
這個病毒的死亡率是100\%.
在香港 在中國 在歐洲 在非洲.
每天都有數以萬計的人.
因為這個病毒而死亡.
每天有數以千萬計的人.
被這個罪的病毒傷害.
COVID是一場天災.
但是罪的病毒.
它成為人類歷史裡一次又一次的人禍.
我們見過一些病情惡化的人.
隨便攻擊別人的國家.
抓一些跟他意見不同的人.
無數的戰爭 無數的專橫.
濫殺無辜 迫害生命.
在這個疫情裡.
除了COVID.
我們見到的是更加可怕的病毒.
前陣子我有機會跟一位醫生朋友聊天.
在第五波疫情裡.
他在醫院裡做最前線的工作.

$^{321}$非常忙.
最近休閒一點.
找個假日可以來長洲玩.
他說了一句話令我非常深刻.
他說與其說不少病人是死於COVID-19.
倒不如說他們死於COVID-19.
原本可以好好照顧的人.
卻被迫搬來搬去.
行政混亂 安排失當 強迫清零.
結果沒有得到最及時最適當的照顧.
很多老人家死於COVID-19.
不是死於COVID-19.
疫情下我們見到很多的人禍.
更加多的是人禍.
恣意 無知 專環 自私 不信 恐懼 冷漠.
在這個真正致命的疫情之下.
我們必須要正視這個問題.
這個致死的病毒.
我們都是無一倖免.
在這個受難節的晚上.
正是我們一個很好的寶貴的機會.
在十字架面前.
去思想我們自己與這個致命病毒的關係.
這個罪的病毒.
如何影響我們的生命.
我們有沒有去感染別人.
有沒有不小心對別人造成身心靈的傷害.
當然 故事並未完結.
在這個被罪感染的世界.
而唱的書第53章.
正正給我們一道新的亮光.
我們再一次去聆聽上主的話.
祂承然擔當我們的憂患.
背負我們的痛苦.
我們卻以為祂受責罰.
被上帝擊打苦待了.
誰知祂為我們的過犯受害.
為我們的罪孽壓傷.
祂受的刑罰.
我們得平安.

$^{361}$因祂受的鞭傷.
我們得醫治.
在這裡中文聖經用了一個絕頂的用詞.
我認為這是一個非常非常美妙的翻譯.
承然和誰知.
承然 祂擔當我們的憂患.
誰知 祂為我們的過犯受害.
這兩個字正正是表達一種意外的發現.
一種極大的反差.
一種誤會 一種無知.
一種無知和誤會背後的愛.
承然 正表明了主耶穌基督十誡上的決心.
誰知 卻表明世人的無知.
因此 以唱阿書這段的經文.
正正是告訴我們三個事實.
第一 世人不折不扣地活在一個充滿罪惡的世界.
病患 苦痛 死亡 無可救藥.
這是我們面對的真相.
第二 同時諷刺的是.
世人卻對於自己被感染毫不知情.
他們以為自己是正常.
他們以為病人不是自己.
第三 誰知承然.
病毒的疫苗 真正的疫苗.
卻在二千年前靜悄悄地來到世界.
無人知曉.
在這幾個月裡 我徹徹底底感受到.
對於這個病 這一場疫情.
感到無能為力.
相信大家都沒有忘記前陣子.
大概三月初的時候.
我們到了一個一發不可收拾的局面.
感染的數字由幾個人到幾十人.
幾百 幾千 一萬 三萬 五萬.
提升上去.
感染人數不斷上升.
恆生指數不斷下跌.
感染人數遠遠超過恆生指數.
這是我們香港人無法想像的事情.
在那時候 全香港的市民.

$^{401}$包括你和我 無可奈何.
咬緊牙關 不能接受 還要接受.
硬著頭皮的面對.
然後我們聽見有朋友開始被感染.
自己的親人也開始感染.
最後自己也感染.
甚至我知道有弟兄姊妹的父母.
在這段時間裡離開世界.
我記得有一晚.
我在外面回家.
地鐵幾乎沒有人.
一個平時熱鬧的香港.
一片死寂.
回到家 脫下口罩 洗手 噴消毒.
雖然不是很累.
但心裡非常非常不善美意.
那時候我問.
究竟我們的福音.
我們所相信的十字架.
我們的救世主耶穌.
在這個疫情之下有什麼意義.
我的意思是.
疫情之下福音可以怎樣去說.
耶穌的福音是什麼.
對於我們一群戴著口罩的人.
什麼是我們所傳揚的信仰.
在那時候我想起一幅十字架的圖畫.
一個十字架的情景.
所以我們今天就有一個這樣的十字架.
十字架上 耶穌被感染.
一個完全健康的人.
真正健康的人.
唯一健康的人.
在十字架上感染了COVID.
或者總括來說.
全世界的病毒.
老人院 急症室 飯堂 食肆 黃大仙 屯門 將軍澳 日本 意大利 巴西 印度.
俄羅斯 烏克蘭 人類所有的病毒.
都一次過集中在耶穌基督的身上.
在這個時候.

$^{441}$耶穌基督的身體極度虛弱.
因為全球人類的病毒.
都一次過感染在祂身上.
耶穌的身體非常疲倦.
發高燒 呼吸短促.
流鼻水 味覺 嗅覺失靈.
喉嚨燒得像火一樣.
然後耶穌的肺功能開始失去運作.
心臟 腎臟 腸道 嚴重損害.
醫生斷定.
這是人類歷史裡最嚴重的重症.
不過最可怕的不是這樣.
最可怕的是十字架上被感染的耶穌.
被全世界的人隔離.
耶穌被封鎖在十字架上.
大聲的呼喊.
我的神 我的神 為什麼離棄我.
然後耶穌基督成為世界上.
最後一位感染COVID離開世界的病人.
不過隨著耶穌基督的死.
COVID在世界上從此消失.
你打開電視機 新聞報導正在這樣說.
這個世界恢復正常運作.
全世界的人脫下他們的口罩.
大家都見到大家的面容.
食肆重開 小學全日復課.
婚禮如常舉行.
飛機回復正常.
全球的關口重新開放.
香港人馬上去買機票去日本.
因著耶穌基督的緣故.
這個世界重新開始.
各位姐妹 這個不是一個簡單的事.
各位姐妹 這個正正是壽難節的意義.
在2000年前真真正正發生了的事.
我們都如釀酒迷.
各人偏行幾路.
而無話事我們眾人的罪孽都歸在他身上.
因他受的刑罰我們得平安.
因他受的鞭傷我們得醫治.

$^{481}$COVID定不可怕.
真正可怕的耶穌基督已經在十字架上為我們消滅.
想得到的只有一個問題.
正如福音書裡面.
耶穌曾經對一個病人來問.
《約翰福音》第五章第六節.
耶穌看見他躺著.
知道他病了許久.
就問他說.
你要痊癒嗎.
丁姐妹 你要痊癒嗎.
當然此時此刻在這個壽難節的聚會.
病毒仍未然夜消失.
我們仍然在網上.
不過我們都很肯定.
COVID終於有一日會消失.
疫情終有一日會過去.
甚至說不定很快就會過去.
我們都盼望明年的壽難節.
我們一班Full Church的弟兄姊妹.
我們希望不用再戴口罩.
一起紀念著耶穌.
不過作為基督徒.
作為一個已經被耶穌基督醫治的群體.
一個接受了真正疫苗的群體.
我們如何面對今天的處境.
正正是我們今天聚會中值得思考的問題.
我的意思是如果我們和一般人一樣.
戴口罩 打針 看新聞 再打針.
數安心碼 等待疫情的完結.
我們和未信的人有什麼分別呢.
在犯罪心理學的理論中.
有一個名為破窗效應.
Broken Windows Theory的理論.
學者發現一個被人破壞了的窗口.
如果沒有即時的修補.
社區附近將會引人更多去破壞其他窗口.
一輛汽車開到一輛滿街都是被人弄壞的車的社區.
很快就會成為下一輛被人破壞的汽車.
可能大家知道的經驗.

$^{521}$一個商場裡髒污的廁所.
只會越來越容易變得髒.
罪正正是一種感染.
反過來說 窗口被破壞了.
如果有人即時的去修補.
社區就不會容易的被人再破壞.
一輛汽車去到一個沒有壞車的社區.
這個汽車會更加的安全.
一個經常保持清潔的廁所.
更加容易去保持著清潔.
鄭青梅 雖然我們的城市.
我們的香港越來越烏煙瘴氣.
但我們仍然有責任 有能力.
去讓這個城市變得好一點.
今日 COVID 並不可怕.
真正叫人死亡的病毒.
不是 COVID 而是人的罪.
基督徒正正就是被呼召去對抗這個病毒.
在這個疫情裡傳播基督的疫苗.
在恐懼和不安裡傳播盼望.
正正是我們教會的使命.
What can we do?.
正正是今天我們受難節.
最後要思想的問題.
鄭青梅 不知道你是否已經打了針.
這個並不重要.
最重要的是.
既然主耶穌基督在十字架上成就了醫治和拯救.
讓我們活得像一個有疫苗的人.
活得像一個真真正正被上帝醫治拯救的人.
活得像一個真真正正得著健全健康的人.
這個正正是我們今天全教會弟兄姊妹的使命.
再一次再一次對抗這個可怕的疫情.
防止香港被感染繼續惡化下去.
我們一起祈禱.
你在十字架上承擔了我們眾人的罪孽.
主在你十字架上從此更新了整個世界.
你更加呼召了我們一群首先被醫治的人.
去成為別人的幫助.
讓一個烏煙瘴氣的香港可以慢慢地得著你的疫苗.

$^{561}$真真正正的疫苗.
讓人得著永遠生命的疫苗.
使用我們.
讓我們首先被醫治.
帶著健全的心靈去傳揚你的盼望.
面對一個很多人的罪的時候.
讓我們能夠可以對抗.
將疫苗去傳給他們.
讓這個社會的人得著終極永遠的醫治.
我們紀念你.
我們求你使用我們.
差遣我們.
在不久的將來我們很快實體的見面.
讓我們帶著康復的心靈.
再一次再一次回到你的心殿裡.
我們一起紀念你.
奉主命強.
阿門.
\newpage



\section{彼得前書 4:1-6-20220416}
\label{sec:LSvYW2YLyv0}
\textbf{【網上聖餐崇拜】得受苦一個選擇,但你仍有得揀|彼得前書4\_1-6|20220416 [LSvYW2YLyv0]}
\newline
\newline
連結: \href{https://youtube.com/watch?v=LSvYW2YLyv0}{\texttt{ https://youtube.com/watch?v=LSvYW2YLyv0}} ~~~~ 語音日期: 2022-04-16 
\newline
\newline
\hyperref[sec:9wD1Nl5xbCs]{\small{< < < PREV SERMON < < <}}
~
\hyperref[sec:index_chronic]{\small{[返順時目]}}
~
\hyperref[sec:index_scriptual]{\small{[返順卷目]}}
~
\hyperref[sec:UT74cFcV7PU]{\small{> > > NEXT SERMON > > >}}
\newline
\newline
彼得前書 4:1-6-20220416
\newline
\begin{longtable}{cl}
\hline
\hline
章節 & 經文 (和合本修訂版)\\
\hline
4:1 & \begin{tabularx}{0.7\textwidth}{X} 既然基督在肉身受苦,你們也該將這樣的心志作為兵器,因為在肉身受過苦的已經與罪斷絕了, \end{tabularx} \\ \\ \relax
4:2 & \begin{tabularx}{0.7\textwidth}{X} 使你們從今以後不再隨從人的情慾,只順從神的旨意,在世度餘下的光陰。 \end{tabularx} \\ \\ \relax
4:3 & \begin{tabularx}{0.7\textwidth}{X} 因為你們從前隨從外邦人的心意,生活在淫蕩、情慾、醉酒、荒宴、狂飲和可憎的偶像崇拜中,時候已經夠了。 \end{tabularx} \\ \\ \relax
4:4 & \begin{tabularx}{0.7\textwidth}{X} 在這些事上,他們見你們不與他們同奔放蕩無度的路就以為怪,毀謗你們。 \end{tabularx} \\ \\ \relax
4:5 & \begin{tabularx}{0.7\textwidth}{X} 他們必須在那位將要審判活人死人的主面前交賬。 \end{tabularx} \\ \\ \relax
4:6 & \begin{tabularx}{0.7\textwidth}{X} 為此,死人也曾有福音傳給他們,要使他們的肉體按著人受審判,他們的靈卻靠神活著。 \end{tabularx} \\ \\ \relax
4:7 & \begin{tabularx}{0.7\textwidth}{X} 萬物的結局近了。所以你們要謹慎自守,要警醒禱告。 \end{tabularx} \\ \\ \relax
4:8 & \begin{tabularx}{0.7\textwidth}{X} 最要緊的是彼此切實相愛,因為愛能遮掩許多的罪。 \end{tabularx} \\ \\ \relax
4:9 & \begin{tabularx}{0.7\textwidth}{X} 你們要互相款待,不發怨言。 \end{tabularx} \\ \\ \relax
4:10 & \begin{tabularx}{0.7\textwidth}{X} 人人要照自己所得的恩賜彼此服事,作神各種恩賜的好管家。 \end{tabularx} \\ \\ \relax
4:11 & \begin{tabularx}{0.7\textwidth}{X} 若有人講道,他要按著神的聖言講;若有人服事,他要按著神所賜的力量服事,好讓神在凡事上因耶穌基督得榮耀。願榮耀和權能都歸給他,直到永永遠遠。阿們! \end{tabularx} \\ \\ \relax
4:12 & \begin{tabularx}{0.7\textwidth}{X} 親愛的,有火一般的考驗臨到你們,不要奇怪,似乎是遭遇非常的事; \end{tabularx} \\ \\ \relax
4:13 & \begin{tabularx}{0.7\textwidth}{X} 倒要歡喜,因為你們是與基督一同受苦,使你們在他榮耀顯現的時候也可以歡喜快樂。 \end{tabularx} \\ \\ \relax
4:14 & \begin{tabularx}{0.7\textwidth}{X} 你們若為基督的名受辱罵是有福的,因為榮耀的靈,就是神的靈,在你們身上。 \end{tabularx} \\ \\ \relax
4:15 & \begin{tabularx}{0.7\textwidth}{X} 你們中間,不可有人因為殺人、偷竊、作惡、好管閒事而受苦。 \end{tabularx} \\ \\ \relax
4:16 & \begin{tabularx}{0.7\textwidth}{X} 若有人因是基督徒而受苦,不要引以為恥,倒要因這名而歸榮耀給神。 \end{tabularx} \\ \\ \relax
4:17 & \begin{tabularx}{0.7\textwidth}{X} 因為時候到了,審判要從神的家開始;若是先從我們開始,那麼,不信從神福音的人將有何等的結局呢? \end{tabularx} \\ \\ \relax
4:18 & \begin{tabularx}{0.7\textwidth}{X} 「若是義人還僅僅得救,不虔敬和犯罪的人將有何地可站呢?」 \end{tabularx} \\ \\ \relax
4:19 & \begin{tabularx}{0.7\textwidth}{X} 所以,照神旨意受苦的人要一心為善,將自己的靈魂交給那信實的造物主。 \end{tabularx} \\ \\
[1ex]
\hline
\hline
\end{longtable}
$^{1}$各位丁姊妹,晚安.
很開心大家能來到Fold Church的敬拜崇拜當中.
我相信有不少的海外丁姊妹.
可能是你離開以後.
你第一次復活節的崇拜.
心願復活大能的主.
在你不同的地方與你一起.
更願意復活大能的主.
的恩典和能力與香港.
那些仍然很忠心在醫護裡面.
服侍的護士醫生.
甚至各樣職能的人同在.
求恩典的主時常保護著他們.
今天我們會看比起前書的四章一至六節.
接下來的翻譯是我的翻譯.
如果你看一下第一節的話.
希望這幾節的聖經給我們一個想法和觀念.
應對我們現在面對的處境.
讓我們重新思考一下.
到底上帝的話語,祂自己的聲音.
如何跟今天,跟我們每一個人說話.
這六節聖經基本上如果你看和學本的話.
其實會引起很多的爭議.
又或者會引起很多不同的討論和想法.
但希望藉著這個翻譯.
給我們有多一點對這段經文的理解.
讓我們可以重新思考一下.
這段經文其實跟我們說的訊息.
是一個什麼訊息呢.
比起前書四章一至六節.
「基督在肉身受苦.
當準備好自己傳同一個意向.
因為那個決意在肉身上受苦的位.
就算決定向罪終止了.
或者止住了」.
其實這段經文這一節說得很特別.
他說耶穌基督釘死十字架受苦.
以至我們要準備好同一個意向.
同一個意向這個字其實我們可以解作.
同一個的意象或者同一個的方向.

$^{41}$或者同一個的目的.
意思就是說原來基督在肉身受苦.
我們屬於他的子民們.
都應該好像和他同一個的想法.
接著就問.
那個決意在肉身上受苦的基督耶穌.
其實他做了什麼.
或者他的受苦是為了什麼.
接著他就說.
其實他的受苦是要決定.
要向罪宣告一件事情.
這個罪要停止.
所以其實你和我打從我們信耶穌的時候.
已經理解和明白.
耶穌基督在肉身上受苦.
在十架上犧牲.
是要拯救我們.
將我們從罪惡裡面釋放出來.
所以原來耶穌基督這個受苦的經歷.
或者主動走上十架的經歷.
其實是在叫世人的罪.
因為耶穌基督流補血的緣故得以潔淨.
罪就可以停止了.
但是比得奇怪的是.
原來這一份的責任.
這一份的思想.
這一份的觀念.
這一份的價值觀.
他說不只是耶穌基督獨有的.
如果耶穌基督在肉身上受苦的時候.
他就叫你和我都要準備好.
我們原來要為主受苦的時候.
和別人產生一個很重要的方向.
就是叫罪停止.
叫罪呈現在這個世間.
叫人不再在罪裡面.
伏在罪的下面.
所以第二節他立刻怎樣說呢.
他說.
他希望我們餘下的人生裡面.

$^{81}$不要再活在人的慾望下面.
要活在神的旨意當中.
這個神的旨意要解釋一點點.
這個神的旨意不是.
我小時候不知道前路怎樣走的時候.
我們唱一首歌.
我望著前路是迷亂將往一片.
那個不是說神是怎樣帶領我的.
那不是.
其實神的旨意在第二節裡面.
回應第一節所說的.
其實如果我們人生裡面.
不再活在那些慾望底下的時候.
我們唯一的選擇是做些什麼.
就好像耶穌基督一樣.
肉身上受苦.
是要叫到罪停止.
是要叫罪能夠在黑暗裡面.
被光進入呈現出來.
這個是基督徒受苦的意義和目的.
如果耶穌基督釘十字架是這樣做的時候.
Nightwise.
我們屬於他的門徒們.
原來受苦是在說這件事情.
很多時候我們說基督徒受苦是在說什麼.
是我突然之間不幸運.
突然之間我有病.
突然之間我被老闆罵.
我們說的這些.
基督徒受苦不是.
比特沒有興趣處理這個課題.
比特處理的是.
耶穌基督是主動走上十架.
叫罪止住.
所以他說基督徒原來走入苦難的裡面.
不是你不幸運 不是遇到一些厄運.
所以你受苦 不是.
他說基督徒真正.
要學耶穌基督的是什麼.
是你知道罪惡在那裡.

$^{121}$你願意受苦.
叫罪惡停止.
叫那些邪惡的事情.
可以被彰顯.
不知道大家最近有沒有看《反起跑線聯盟》.
其實我們全家都看.
由一至五集沒有看之後.
就追回 追到第十五集.
其中有一幕.
就是小朋友要做一些模型.
其實結果你知道很多小朋友都做一些模型出來.
見到有個爸爸 Charles.
找人去做很多模型出來.
很漂亮的模型 不是小朋友做的.
拿出來.
去比賽.
很明顯結果都不用說.
最漂亮的那個都不是他做的.
的模型拿獎 我相信.
Charles在那個時候教小朋友講一句說話.
他說爸爸教你一個四字成語.
什麼叫做「惡立雞群」.
你想想 人在慾望裡面.
很想贏 很想拿第一.
很想在眾人面前成為最厲害的那個.
其實這些邪惡.
在我們生活的圈子裡面很多.
很多人經常都考第一.
還考第一很多年.
但考第一很多年的人.
最終都要被迫回家.
因為時間到了.
你可以想像的是.
這些事情在面前發生的時候.
他問一個問題是.
到底.
有沒有人在這些罪惡邪惡.
呈現的時候.
有人願意受苦.
將這些罪惡停止.

$^{161}$你看完第三節的時候你就會明白.
什麼叫做活在人的慾望底下更加多.
第三節你看這篇經文的時候.
你驟眼看.
在過去日子裡面.
我們已經足夠在我們幫人的慾望底下做事.
就是在豪飾私慾 或者喝醉酒 放煙.
或者喝很多.
跟那些律法所禁止的扮偶像當中.
你驟眼看的時候.
你覺得這些事情我都不太做.
為什麼我不喝酒.
我又不會做一些律法不讓我做的事.
我又不喜歡去吃東西.
但聽完之後不要誤會.
這段話所說的東西.
其實這些所做的每一件事情.
其實當時候.
在彼得社給那些流散的人去看的時候.
是在說每一個流散的人面對他新的環境的時候.
原來在希羅文化裡面.
這些事情是用來做朋友的.
你要跟他一起喝得很大.
你要跟他一起做一些很荒謬的事情.
你才可以跟他成為朋友.
很奇怪.
但那時候希羅文化就是這樣.
不單是這樣.
如果你想在那個時候.
新的地方去做生意的話.
你一定有些規矩.
規矩是什麼.
就是你要參加.
這些人所辦的一切聚會和.
很荒誕的事情.
你要跟他樂踏.
跟他扮朋友.
同一天做這些事情.
你才能夠有生意做.
所以彼得在說的.

$^{201}$是希羅文化裡面的事情.
你說他會問.
彼得想帶出些什麼.
如果我們有看一本很久的書.
那本書叫做《動物農莊》.
就是《動物農莊》.
George Orwell所寫的.
其中有一句說話是到今天.
或者在香港社會裡面仍然說得很貼切.
他一定會說.
All are equal.
But some are more equal than the others.
在《動物農莊》裡面.
寫了一個很大的規矩.
就是所有動物生來都是平等的.
不過到了後期的時候.
有些優越的人士.
有些喜歡設定規矩的人士出來說.
你要做到這些規矩.
你要做到這些事情.
才能夠成為大家所公認的領袖.
所以最後他加多一句說話.
Some are more equal than the others.
不是.
其實所有人都平等的.
但是有些是會比其他人平等的.
希臘文化裡面.
你要跟他做生意.
你要跟他做朋友.
你要在文化裡面能夠立足到.
生存到.
你一定要屈服.
跟那些人做朋友.
做這些喝醉酒.
很開心很荒誕的事情.
他說你這樣做才能夠成為我的朋友.
不是.
做朋友是.
我和你做朋友.
做朋友不是.

$^{241}$不是因為我要做些什麼.
才能夠做朋友嗎.
不是每個人都可以做生意嗎.
不是要參加了你這樣的情況下.
才能夠做生意嗎.
彼得正在針對.
在人罪惡裡面一個很嚴肅的課題.
人生下來好像每一個都平等.
但是在歷史裡面告訴我的是.
其實不是每一個人都平等的.
你要做了一些事情才叫平等.
你不做的話.
你連超級市場買罐汽水喝.
你都無能為力的.
在最近的日子.
大家知道下星期就能夠可以.
開放教會崇拜.
我聽聞.
很多教會在籌存中.
在思考中.
在門口設置多少部機器.
人流是要怎樣安排.
才能夠滿足責任.
尤其是一些相對大的教會.
當崇拜來到的時候.
他們在計算每一部機器.
一秒鐘可以做到多少人.
然後安排多少部機器在那裡.
要買些什麼製造器回來.
有些童工在問.
除了我們真的要做這些之外.
那些沒有得來的.
那些不算是.
有些人比其他人更加平等的人.
有這些位份的人.
在問可以怎樣處理.
很多我不應該說的話都不要在這裡說了.
我聽到之後心裡面.
都莫名的恐懼和驚訝.
其實真是.

$^{281}$當100周年,150周年的時候.
我們說教會一家.
是一班人.
是一個群體.
缺少了誰不行.
經常說的那麼多前書的那些.
沒有體面的人我們更加要給他體面.
但去到現在現實的環境裡面.
誰還比體面那些沒有體面的人.
我聽到這些說話.
不太方便在這裡說什麼.
令我很窄竊.
原來平時教會說的那些Dragon.
什麼家,什麼情,什麼主內一家弟兄姊妹.
去到這些時候.
突然間整個一家人的感覺全沒了.
這些傷心的事我們不要提那麼多.
我們提一些開心一點的事.
在這一個多星期裡面.
我聽到有些童工.
他親自說.
如果要四個星期崇拜的話.
將一些人exclude的話.
他說不如我只搞一次崇拜一個月.
另外那三次崇拜.
網上Zoom繼續算.
他想告訴那些.
some are less equal than the others的人說.
我仍然和你想在一起.
振奮一點聽到一位主教牧師.
他說如果這個小組.
這個團契裡面有人不能回教會的話.
這個牧師請求那個小組的頂姐妹.
一起不要回教會.
他請求那班頂姐妹.
一起去到那個人家裡面.
和他一起崇拜.
頂姐妹.
我們聽到很多.
耶穌基督受苦,復活.

$^{321}$聖天那些很浪漫的說話.
與感動的圖像等等的東西.
但是頂姐妹.
這個復活節.
其實並不是一個時候.
我們還停留在一個純粹紀念和浪漫.
耶穌受苦的經歷裡面.
如果比得上.
那些流散的人去到不同地方的時候.
受到很多人的不公平的對待.
很難的時候.
他說什麼.
可不可以那些一早在那裡的頂姐妹.
那些已經在20年前流散到那五個地方的頂姐妹.
你試試受苦,受人家逼迫.
磨爛,匪夷受人家各樣的難處裡面.
你為那些不願意跟他們一起.
好色,私慾,放膽的事做的人.
你可不可以受苦.
為他們的緣故做些事情.
最近在跟子女聊天的時候.
你知道那些小學生是要做了某些事情.
才能夠回到音樂課和體育課.
我跟兒子聊天的時候.
我說如果在班裡面有人說.
你不能回家了,你不要上音樂課,你不要上體育課.
我說兒子,你有沒有及時.
及時這個字是我正在紀念親愛的彭牧師.
或者你有沒有及時跟學校的人說.
我不去那些課,我留下來陪同學一起不上音樂課和體育課.
如果昔日.
彼得社那些剛流散的人.
他會面對很大這些難處的時候.
是不公平被對待,是一定要入鄉隨俗.
這是文化,這是必須要有的東西.
你一定要這樣做才能夠這樣做.
彼得勸勉甚麼.
你看看第3,4,5,6節.
你再看下去.
他說那些人會感到驚訝.

$^{361}$他們會羞辱你,這是文化.
這是我們入鄉隨俗的東西,你跟著做吧.
為甚麼不做呢.
彼得告訴他們.
如果你相信耶穌.
耶穌基督的受苦是要叫罪呈現.
叫邪惡能夠停止的話.
你就不走在他們的錯誤的道路上.
你會被人羞辱,你會被人驚訝.
為甚麼做這些愚蠢的事,不應該嗎.
今天在這個世代裡面.
你看到有人開始肆無忌憚.
將任何的導彈到處發射.
有人殘殺.
甚至有些人.
可以叫人幾天不吃東西.
但我們已經看不到這些新聞了.
在這個世代裡面.
有人要受苦.
目的是要叫罪惡停止.
叫罪惡停止在那個位置.
從來沒有好的下場.
那15000個俄羅斯的人.
反戰的人.
已經被抓,被鎖在監獄裡面.
受苦.
從來都是主動的.
受磨爛.
從來都是耶穌基督走上十界.
這個例子告訴我們.
要叫罪停止,不會因為他自己會停止.
罪真正會停止,是因為有人受苦.
罪才會停止.
你看著烏克蘭城的情況裡面的時候.
就是因為人主動地受苦.
在基輔附近的.
神學院的教務長已經死了.
唯有人主動走上一條不歸路的時候.
人驚訝,人羞辱他們的時候.
其他人看在眼內的時候.

$^{401}$才知道這些東西叫做錯.
人不再走在裡面.
今天.
有很多人想設立.
一些人比其他人更加平等的時候.
我想起耶穌.
耶穌在安息日.
醫病.
吃了.
他傻的.
約定俗成,安息日不可以醫病.
那些三十八年血流的女人.
摸一摸她,為甚麼要醫好她呢?.
這些血流的人.
是該看她的.
那些患麻風的人.
是不應該和耶穌一起的.
耶穌卻是.
主動在安息日去醫病.
和被鬼附的女人一起吃飯.
和那些碎尼走在一起.
醫好那些帶麻風的人.
是.
他換來甚麼下場?.
那些法理賽人問事.
那些人.
對他白眼.
覺得他不耐勞.
今天.
教會要做些甚麼?.
你與我要做些甚麼?.
在那些意識形態壓下來.
覺得這件事要做的時候.
受苦節的意義是.
我們可以為那些人.
一同受苦.
願意在這個苦節的時間裡.
親愛的兄弟姐妹.
真的不是再在網上.
看很多苦節的崇拜.

$^{441}$與文章.
也不是要重溫甚麼passion.
不是要再重溫甚麼.
很感動的文字.
甚麼大齋祈的土文等等.
與苦節的操練.
苦路十四站等等.
今年的復活節與壽苦節.
是這個時候問我們.
我們準備受甚麼苦.
讓人叫到我們所做的驚訝.
並會招來很多人對我們的恥辱.
但我們知道的是甚麼.
我們不走在錯的路上.
接下來當教會重開的時候.
那些因為各種原因不能回教會的人.
是成為異類被歧視.
還是我們真的將愛帶給他們.
但這個動作不是開會決定的.
是要進入苦難的裡邊.
招來苦難.
才讓人見到上主的美善.
在我們當中.
當世界開始變樣.
當世界無理的事成為了常理的時候.
你跟我要問.
復活節還是一個訊息.
還是你跟我生命活出來的東西.
有些人已經預備好挑戰.
預備好難關.
兩個多月已經不能去超市買東西.
去任何食肆吃任何東西.
接下來開戲院你不能看.
你想看怪獸與鄧不利多.
求天父憐問我們.
在這個復活節的時候.
復活的大能是因為信徒進入受苦的裡邊.
叫那些some are more equal than the others.
這些邏輯.
這些邪惡.

$^{481}$可以被徹底呈現在光明的下邊.
求主憐問求主幫助.
一起祈禱.
天父多謝你.
讓我們在這個世代裡邊.
不再pay save.
當一個世代的人.
可以說只有一個選擇.
叫做很多選擇的時候.
天父你給我們得受苦的選擇.
進入受苦人群裡邊.
和他一起經歷.
一起同行.
面對.
求天父你自己的愛.
淋到愛當中.
幫我們在這個很躁動的年代.
做不一樣的事情.
求天父你憐問和幫助.
讓你的愛.
藉著你子民們的信服於受苦.
帶來上帝的美善.
應許仍然臨在這片土地上.
多謝天父你.
聽我們正祈禱.
歡迎宿敬你寶貴名球.
我們下次再見吧!.
\newpage



\section{哈巴谷書 1:5-20220423}
\label{sec:UT74cFcV7PU}
\textbf{【網上崇拜】上帝叫我厚多士(已FC)|哈巴谷書1\_5|20220423 [UT74cFcV7PU]}
\newline
\newline
連結: \href{https://youtube.com/watch?v=UT74cFcV7PU}{\texttt{ https://youtube.com/watch?v=UT74cFcV7PU}} ~~~~ 語音日期: 2022-04-23 
\newline
\newline
\hyperref[sec:LSvYW2YLyv0]{\small{< < < PREV SERMON < < <}}
~
\hyperref[sec:index_chronic]{\small{[返順時目]}}
~
\hyperref[sec:index_scriptual]{\small{[返順卷目]}}
~
\hyperref[sec:7_6qfdylGjE]{\small{> > > NEXT SERMON > > >}}
\newline
\newline
哈巴谷書 1:5-20220423
\newline
\begin{longtable}{cl}
\hline
\hline
章節 & 經文 (和合本修訂版)\\
\hline
1:5 & \begin{tabularx}{0.7\textwidth}{X} 你們要向列國觀看,注意看,要驚奇,再驚奇!因為在你們的日子,有一件事發生,儘管有人說了,你們還是不信。 \end{tabularx} \\ \\ \relax
1:6 & \begin{tabularx}{0.7\textwidth}{X} 看哪,我必興起迦勒底人,就是那殘忍暴躁之民,通行遍地,霸佔不屬自己的住處。 \end{tabularx} \\ \\ \relax
1:7 & \begin{tabularx}{0.7\textwidth}{X} 他威武可畏,審判與威權都由他而出。 \end{tabularx} \\ \\ \relax
1:8 & \begin{tabularx}{0.7\textwidth}{X} 他的馬比豹更快,比晚上的野狼更猛。他的戰馬跳躍,他的戰馬從遠方而來;他們飛跑,如鷹急速抓食, \end{tabularx} \\ \\ \relax
1:9 & \begin{tabularx}{0.7\textwidth}{X} 都為施行殘暴而來,他們的臉面向東,聚集俘虜,多如塵沙。 \end{tabularx} \\ \\ \relax
1:10 & \begin{tabularx}{0.7\textwidth}{X} 他譏誚列王,嘲諷領袖,嗤笑一切堡壘,堆土攻取它。 \end{tabularx} \\ \\ \relax
1:11 & \begin{tabularx}{0.7\textwidth}{X} 那時,他如風猛然掃過,他背叛,顯為有罪;他以自己的力量為神明。 \end{tabularx} \\ \\ \relax
1:12 & \begin{tabularx}{0.7\textwidth}{X} 耶和華-我的神,我的聖者啊,你不是從亙古就有嗎?我們必不致死。耶和華啊,你派他為要行審判;磐石啊,你立他為要懲治人。 \end{tabularx} \\ \\ \relax
1:13 & \begin{tabularx}{0.7\textwidth}{X} 你的眼目清潔,不看邪惡,也不看奸惡,為何你卻看著人行詭詐呢?惡人吞滅比自己公義的人,為何你保持沉默呢? \end{tabularx} \\ \\ \relax
1:14 & \begin{tabularx}{0.7\textwidth}{X} 你為何使人如海中的魚,又如無人管轄的爬行動物呢? \end{tabularx} \\ \\ \relax
1:15 & \begin{tabularx}{0.7\textwidth}{X} 他用鉤子把他們全拉上來,用羅網捕獲他們,拉漁網聚集他們。因此,他歡喜快樂, \end{tabularx} \\ \\ \relax
1:16 & \begin{tabularx}{0.7\textwidth}{X} 向羅網獻祭,向漁網燒香;因為他藉此得豐盛的收穫與肥美的食物。 \end{tabularx} \\ \\ \relax
1:17 & \begin{tabularx}{0.7\textwidth}{X} 但他豈可因此屢屢倒空羅網,時常殺戮列國的人,毫不顧惜呢? \end{tabularx} \\ \\
[1ex]
\hline
\hline
\end{longtable}
$^{1}$歡迎大家繼續參加網上崇拜.
特別是實時參加的你.
在空中的這一秒.
無論你身處在哪一個時區.
你可以在即時通訊那裡.
參加以下這個觀眾互動簡潔局的環節.
我真的預備了兩份講章.
在左邊就是一個放煙花的好消息.
如果你想聽這個.
你可以留言告訴我們.
旁邊這個帶著眼淚的微笑.
是一個sad but true的壞消息.
如果你想先聽壞消息.
你可以留言告訴我們.
我很需要網上流淌目者.
Flow Church Pastor幫我看一看帖文.
頭十個留言的朋友選擇哪一個比較多.
我就先說哪一個消息.
如果平手的話.
就由monpost的同工投下神聖的一票.
有沒有結果?.
我看不到.
有結果了嗎?頭十個.
數得到嗎?.
sad but true 大家喜歡先苦後甜的.
沒有問題.
sad but true的消息就是.
我們沒有人知道這個疫症何時才會完結.
我們也不知道香港會不會.
第五波之後會不會有第六波.
我不知道.
但很想去紀念在疫苗通行證真正完結之前.
一些未能夠進入宗教場所.
和一些未能夠自由進入其他表列處所.
而帶來生活不便的一群.
也紀念香港教會的報道工作.
因為當我們正在煩惱.
誰會有未打疫苗不能回教會.
其實我們討論的都是自己人.
但與此同時.

$^{41}$堂會和一些未打疫苗.
或者不想使用疫苗通行證的福音對象.
距離比以往更加遠.
回到今天所說的經文.
經文其實是來自哈巴谷書.
哈巴谷書是小仙之書之一.
舊約希伯來聖經.
共有十二卷小仙之書.
少並不代表它不重要.
少只是它篇幅很少.
哈巴谷書只有三章.
我鼓勵你們.
短短地一次過看完三章.
以下哪一項是最符合你對舊約仙之的印象.
一,微卜先知.
即是預言神童.
教會今個月的月題是月位.
看球賽就最忌預測哪隊會贏.
可能會有燈神.
第二,神的代言人替上帝說話.
針對君王或百姓作出斥責.
第三,輔助君王或祭司.
開道以色列百姓.
大家可以繼續在即時通訊互動.
我事後會再看.
哈巴谷並非以上三類的先知.
先知身處的環境.
是被列強還治的以色列人.
以色列裡面.
在哈巴谷書的第一章和第二章.
記錄了先知和上帝的對話.
先知說出自己內心掙扎的過程.
先知告訴神.
當時我所身處的亂世中.
充斥了頹垣敗瓦.
我們很苦.
哈巴谷書的結構非常簡單.
一章二至四節.
哈巴谷第一次向上帝申訴.
五至十一節.

$^{81}$上帝第一次回答.
一章十二至十七節.
哈巴谷第二次向上帝申訴.
二章二至二十節.
是上帝第二次回應.
三章一至十九節.
是哈巴谷的禱告和意象.
而今日我們所看的經文.
是座落在第一次上帝對哈巴谷先知的回應.
哈巴谷身處的社會.
發生了甚麼事要向上帝申訴.
我們以後會透過閱讀或寫作文本劇本.
經常用到的五個W一個H六合法.
何事何地何人.
何事為何與何.
敘事者已經在哈巴谷書裡面好好記錄.
我們一起看下去.
一章第六節.
我必興起迦勒底人.
就是殘忍暴躁之民通行遍地.
霸佔不屬於自己的住處.
經文說到巴比倫的迦勒底人.
正興起和強大.
南國猶大國亡國前.
正面臨外敵巴比倫的威脅.
已經危在旦夕.
根據歷史我們知道.
哈巴谷是主前大約.
689至605年間開始做先知.
學者都相信成書的年份.
是在主前609至597這十幾年的期間.
當時一方面.
外國的勢力就越來越大.
與此同時這段時期.
南國最後那兩三個猶大王.
都是一些庸碌的君王.
已知道他們的國家充滿了.
強暴和罪惡.
哈巴谷書就是在這個.
外憂內患的背景底下寫成.

$^{121}$就解答了何時when這個問題.
假如我們真的去將哈巴谷書.
當做一部讀劇的劇本去看.
其實結構是非常簡單.
就是哈巴谷說 上帝說.
哈巴谷說 上帝說.
我們就很簡單解答了.
何人乎這個問題.
對話只涉及兩個人.
他們不是因為要遵守.
跨家庭二人限聚.
他們的訊息也不需要.
經過點對點加密.
第二章的第二節.
耶和華回答我說.
將這物事清楚地寫在看板上.
使人容易朗讀.
是上帝叫先知寫在看板上.
讓人看的.
就好像我們教會.
留堂製作室的場地.
初心的房間.
那裡是留堂發展史的縮影區.
記錄了過去不同年日的文宣和主題.
張貼出來是為了讓人容易看到.
看到的時候提醒我們.
留堂是為了什麼而立.
是為上主的福音.
搞動眾人的心.
相信上帝叫哈巴菊去整碧布.
也是這個意思.
上帝對先知的回答寫得出來.
就是絕對容許對話以外的任何人.
去查看和聆聽.
不單止和哈巴菊一個人說話.
第二章第一節.
我要站在我的瞭望台.
立在城樓上觀看.
看耶和華要對我說什麼.
發生這一段先知和上帝對話的地點.

$^{161}$就是在瞭望台.
城樓上的一個對話.
這裡解答了"Where".
哪裡的答案.
但這也是第二章後半段.
先知和上帝對話的地點.
但第一章一開始申訴的時候.
地點發生在哪裡.
是未知的.
在經文的上文下.
我們會看到.
當先知上了瞭望台之後.
上帝正在回答他的答案.
其實是在回應他一開始的問題.
所以我對於哈巴菊書.
作出了以下的分段.
一章二至四節.
哈巴菊就開始向上帝申訴.
五至十一節.
上帝其實是正在回答他.
但十二至十七節.
我會形容.
哈巴菊就好像打斷了上帝對他的答案.
他不滿意上帝對他的答案.
他繼續申訴.
我會形象化.
六四的部分.
哈巴菊就向上帝說.
你為什麼不答應我們.
為什麼看不到我們所面對的罪惡.
然後上帝就正在回答他.
他正在回答你.
但你又再搶白.
你又再插嘴.
第一段對話.
就是在這樣的情況下發生.
進入到經文的第一點.
想和大家分享的是.
身於亂世的先知有種責任.
是做申訴的專員.

$^{201}$先知哈巴菊的確是一個非常專業的申訴專員.
他在那封投訴信裡.
很清楚地說出.
南國猶大本土面對著什麼邪惡.
而第一章第二節申訴的第一句.
已經有一個自己很有共鳴的說話.
就是.
我呼求你不應運.
要到何時呢.
我有時也會氣稱.
第五波疫情好像回大.
回到2020年的新年.
疫情捲土重來.
又戴過口罩.
又來過網上送餐.
又要視像性餐.
維持了兩年多三年的疫情.
要到何時呢.
2020年開始就有很多航班停飛.
甚至有些國家是全國封關.
本來兩年前的今天.
我已經買了機票.
三月和四月.
我全家去宣教工場短宣兩個月.
但大家都看到.
事到如今.
究竟可以出發宣教了嗎.
我是很想疫情快點完結.
你呢.
心目中有沒有一些事情.
一直在問上帝.
要到何時呢.
我的長期病.
我的痛症何時才會康復.
我生活中所面對的難關.
憂骨 仇敵何時才會過去.
我失業的情況.
要到何時呢.
我的呼音對象何時才信主.
某某何時才會成功戒煙戒毒.

$^{241}$戒情緒勒索.
還有.
日常生活裡面.
在各間社會上所面對到的.
荒誕虛假詭詐.
要到何時呢.
自從上年.
6月25日開始.
我都聽過不少人說.
我不想再看新聞了.
因為他們很心灰.
他們不再等候.
先知哈巴菊呢.
哈巴菊他還在等候.
他還繼續擔當他身為先知的責任.
他繼續申訴.
是呀.
有種人是這樣的.
灰磯避世陷入迷失.
但哈巴菊他還是和上帝申訴.
從經文我們可以看到.
他記錄下來就是第一次的申訴.
但是我相信.
未被文字記錄的申訴.
就未必是第一次.
不會第一次就問.
要到何時呢.
等外賣都是問了幾次才問.
何時才到我.
我相信他是不斷不斷地去問.
不斷不斷地等待.
再繼續申訴.
先知向上帝的呼求.
他向上帝呼喊.
從他的角度.
上帝一次又一次不答允.
上帝未拯救.
所以他又再申訴.
在第一章三至四節.
先知問了兩個Why的question.

$^{281}$你為什麼要讓我看到這些罪孽.
你為什麼坐視姦惡.
再加上第一句.
要到何時呢.
其實這兩個字詞.
是經常出現在.
愛哥的典型格式.
哈巴菊先知他因為什麼事.
不斷向上帝哀求申訴.
當時的社會發生的亂象.
包括以下六方面.
罪孽,姦惡,毀滅,兇暴.
爭執,紛爭.
其中的罪孽和姦惡.
可以指向一些權力的濫用.
無論是濫用社會的權力.
宗教的權力.
還是國際上政治方面的權力.
在預備講章的時候.
俄羅斯入侵烏克蘭戰時已經達到第50天.
明天就是兩個月.
明明是入侵霸佔不屬自己的國家.
但卻將自己的血腥行為美化成為解放的行動.
這些是罪孽和姦惡.
另外一組的字眼.
毀滅和兇暴.
可以指涉及一些實際的身體暴力行為.
令到別人的身體受到傷害.
以及搶奪其他人的財物.
不是他的東西就拿走別人的.
就例如一個.
某個不能說的國.
疫情期間實施了不能說的封城.
他的民眾都很餓.
每一朝起床都要按掣搶菜.
還有很多的真相是搜尋不到.
被毀滅被消失.
不留下傳媒的痕跡.
兇暴和第二節所說的暴力.
原文是相同的字.

$^{321}$重複的字眼告訴我們.
這也是先知最介意的一種暴力.
最後一組字詞是.
爭執和紛爭.
屬於當時法庭的用語.
可能是先知年代.
當時有不少要涉及法律訴訟的案件.
出現了人與人之間的紛爭.
需要以法律訴訟的方式去處理.
而這個最後出現的詞組.
亦都很自然地引入到第四節的內容.
因此法律無效.
不是沒有律法.
是有律法也無效.
人證物證俱全.
都可以放生不起訴.
這是否正式的律政司.
在後國安法時代的香港.
有執業律師資格的立法會議員聲稱.
專家推出病毒共存的話就犯國安.
有MIRROR成員有份拍的劇集又說是犯國安.
令我也不禁問.
選員意見不是代表教會立場.
請問這是正式律用的律師嗎?.
先知說法律無效.
公理從未彰顯.
因為惡人為困異人.
所以公理遭受扭曲.
惡人的勢力太大.
以致法律發揮不到作用.
現代中文譯本有以下的翻譯.
正義永遠不得伸張.
正義被歪曲了.
你對這個翻譯有何看法?.
我很喜歡這個翻譯.
我覺得這個翻譯很神學士粉紅色的秋.
原文沒有永遠這個字.
但是這個翻譯.
很貼近先知感受的意義.
永遠是多遠?.

$^{361}$我問一些流散海外的朋友.
如果你來自的地方好起來.
你會不會想回來?.
他們回答我.
雖然我們沒有交還特區護照和身份證.
但我們是移民不是搬屋.
香港會病好的嗎?.
整體來說.
先知所身處的社會.
實在陷入相當惡劣的狀態中.
我指的是舊約哈巴谷先知所身處的社會.
換一句現代的說話形容.
哈巴谷就是在跟上帝說.
This city is dying.
教會閱題《傳奇》.
如果命運能選擇的講員.
他曾經說過.
在他迷失的時候.
他會記住以下這番說話.
我們常常向苦難問「Why」.
為什麼?.
但這永遠沒有答案.
我們應該問「How」.
怎樣做?.
這番說話解了他很多次的心結.
而上帝又怎樣回應「How」?.
他怎樣回應哈巴谷的「Why」?.
或者我們先想想.
在自己身處的國家.
各方面都如同步向死亡.
邪惡的事多到每天都像面對狂魔一樣.
先知會希望上帝怎樣回應它?.
你又希望上帝怎樣回應你心中的「Why」?.
大家可以繼續在即時互動留言.
我事後都會看回.
第二點我想和大家分享的是.
上帝叫我「好多事」.
面對先知的訴求.
上帝的回應是怎樣呢?.
上帝竟然對他說.

$^{401}$「你們要向列國觀看,注意看,要經期,再經期」.
就著哈巴谷先知所發問的問題.
上帝採用了先知相同問題的動詞來回應.
在第三節.
哈巴谷問「你為何要我看見罪孽?」.
上帝回應「你要看的不是罪孽,而是列國」.
哈巴谷問「神啊,你為何坐視姦惡?」.
在原文裡和上帝的回應是一樣的.
不單要向列國觀看.
而是要注意看.
要看到你大大吃驚.
雖然上帝沒有按照先知心中的標準答案去回應.
但上帝是在回應他每一個問題.
上帝告訴先知.
他要做一個令人大大驚奇的事.
是甚麼驚嚇或吃驚的事呢?.
「看啊,我必興起迦勒底人.
就是那個殘忍暴躁之民.
通行遍地,霸佔不屬自己的處處.
他是為了私行殘暴而來的」.
甚麼?我們本身已經受苦了.
你還要派出這個迦勒底人來入侵我們?.
豈不是落井下石,雪上加霜的情況?.
彷彿當先知明明跟你說.
「上帝啊,我們這個城市就是死定了」.
但上帝卻告訴他.
「是,我會興起更加惡的外敵來了結這個城市.
還不夠絕望,尚可更絕望」.
會不會有機會就是先知的心聲呢?.
問與答聽到這裡.
我認為上帝其實是在繼續回答先知.
但正正是因為上帝的回答是平復不了哈巴谷.
所以先知又不斷再申訴再申訴.
在《釋經》的結構上.
12至17節都是先知的插嘴.
他繼續一連串地問.
「為甚麼?為甚麼?為甚麼?」.
一直問這個問題.
他很希望上帝做些甚麼.
顯示神蹟給他看.

$^{441}$終止惡人的暴行暴力.
他繼續申訴.
但他有一個轉向.
在他繼續申訴的時候.
他的地點,他的人就移動了.
他去了瞭望台.
立在城樓上.
他真的去聽上帝的話.
這是哈巴谷的選擇.
上帝說你可以去看看烈國.
他就用一個上去瞭望台的方式.
用眼睛去看.
如果上帝都叫你去觀看烈國.
可能我們是看國際的新聞.
或者我們有甚麼方式去選擇去回應.
先知是否已經上去瞭望台的時候.
他是否已經準備好去聽上帝怎樣回答他.
你覺得呢?.
我就覺得未必.
但可以很肯定的是.
哈巴谷其實不知道上帝會再回答他甚麼.
但他已經準備好再申訴.
因為他已經說好了.
我要站在瞭望台.
我要聽上帝對我說甚麼.
下面那句就是最精彩.
我可以用甚麼話再向他訴冤呢?.
這裡的訴冤和第一章第二節的呼求和哭喊.
是不同的字眼,也是不同的程度.
如果我說第一章裡面的呼求和哭喊.
是屬於一百萬伏特那麼大的威力的話.
這裡第二章第一節的訴冤.
就是有二百萬零一伏特那麼大的威力.
他已經想好一定要再申訴.
他一定要再說他的訴求.
但上帝他一直都很貼心.
對準他的問題.
點對點,很重點地去回答哈巴谷每一個問題.
上帝在第二章第二節開始去回應他的時候.
就回答他問要到何時呢?.

$^{481}$這個答案.
上帝回答他.
這個是會有一定的日期.
眼前一切是會有終局.
是會結束,絕不落空.
雖然你覺得我是遲了.
但你要等候.
因為這個結局一定會臨到.
一定會結束.
不會再拖延.
哈巴谷的目光.
一直在注視著他眼下本土的暴力.
一次又一次申訴.
可以了嗎?夠了嗎?要到何時?.
為什麼?為什麼會這樣?.
Why? Why? Tell me why?.
但上帝對先知的回答是.
我要你觀看列國.
我要你多事.
我要你們厚多士.
將眼光拉闊放遠.
今天的經文.
第一章第五至上半節的.
你們向列國觀看.
正正就是上帝不是叫哈巴谷一個.
在上帝一開始去回應哈巴谷的時候.
已經是叫他.
你們,昔日是會閱讀到漢版內容的人.
今天就包括了聽著哈巴谷的訊息的我們.
曾經有一個競選第五屆特首的女士說.
上帝叫我參選.
我們是fact check不到的.
但上帝叫我們厚多士.
就是已經fact check了.
上帝很想我們向列國觀看.
上帝很想我們跨越我們的界限.
很想我們向外做一些超過自己本位的事.
經文部分就到這裡.
剛才先苦後甜.
聽完這個sad but true的消息.

$^{521}$我現在就送一個好消息給大家.
這個好消息就是一個很值得放煙花的消息.
就是我的二兒子今天是六歲生日.
我很想和各位家人.
特別是香港的家長.
如果你的家裡有K3的學生.
還有如果你是幼師.
你在教幼稚園.
疫情持續了這三年.
你們過得非比尋常的K1 K2 K3.
網課和停課的日數比面授的日數更多.
固然是人都瘋了.
但是兒子和各位家長和老師.
你們都很厲害.
是能人所不能.
所以今天歡呼的消息就是借位和兒子說個Happy Birthday.
又到了即時互動訊息的時間.
這個解答了.
上年母親節封的一個服侍照片.
我請你數一數途中有多少位七歲或以下的小朋友及BB.
在即時通訊那裡數一數.
是不是有人在回答呢?應該有的.
其實答案這幅圖中有四個.
其中有三個是我的.
有一個小朋友就在Amy目者的圖裡.
有些留堂達人就知道我家裡有三個小朋友.
但是很少人知道我這個家庭計劃背後的故事.
是和一位宣教的前輩有關.
人有時會被某一種身份去定型.
從而扮演不同的角色.
甚至越為扮演一些從未試過的角色.
多年以來.
他都是以牧師的身份從事牧養宣教和神學教育的侍奉.
然而在近年.
他的身份和角色上出現了一個越位.
一個突破.
就是由於香港有很多中社會都引起互動的案件.
他開始去法庭旁聽.
竟然得著旁聽師這個稱號.
有教務的同道去問他.

$^{561}$在今天的處境下.
行公義或者會帶來壓力和懼怕.
甚至被捕的風險.
牧師你有沒有心理準備.
他說.
醫院有遠目.
法庭有旁聽的牧師.
這裡有很多人.
很有情緒牧養的需要.
或者恐懼讓我們限制了自己.
以至無法為公義發聲.
不敢實踐愛.
但我們要認定一件事.
當你站在神的公義那一邊.
其實是不用怕的.
另外愛是沒有懼怕.
如果你的動機是從愛出發.
你是不用怕的.
萬一被捕.
我就會像仕途保羅和西拉一樣.
在獄中禱告唱詩讚美神.
開始我越位侍奉的2.0.
實踐我過去未能做到的新角色.
更新我的目光使命.
他還說.
可能我有坐牢的恩賜.
我心想.
牧師你不要這麼瘋狂.
這不是貶義.
在剛剛過去的提早暑假.
復活節的長假期.
我相信彭牧師一定在倉庫裡.
舉行復活節崇拜.
他會繼續講道.
他會繼續傳福音.
令到獄中充滿歌聲.
笑聲,祈禱聲.
過得像過夏令營一樣.
我先生在讀神學的時候.
受教於本身有四個子女的彭牧師.

$^{601}$當時我們只有大兒子.
老師知道我先生主修跨文化宣教.
有宣教的心智.
他鼓勵後輩要生養眾多.
要有更多的屬靈產業.
成為一條宣教的團隊.
可以出發宣教.
當日鼓勵人的是你.
聽書的是我老公.
但生的那個是我.
牧師你要平安無事.
出來負這個責任.
我直接認識一位後任宣教士.
除了像我一樣.
都是因為疫情要滯留在香港.
他的宣教路還有其他艱難.
就像這個交通標誌的路牌.
這個交通標誌的路牌.
就是我考博士的時候.
狂做練習的其中一題.
裡面的答案你們知不知道?.
猜吧.
不懂開車的就猜吧.
懂開車的就回答吧.
這個標誌的意思是.
前面道路兩邊收窄.
這個認識的後任宣教士.
他面臨一邊來自原生家庭的各種質疑.
香港有很多人需要你幫忙.
你為什麼要去一些落後的地方.
去幫一些不認識的人呢?.
那裡有那麼多人有需要.
你只能幫一個人.
幫不了第二個.
不需要你了.
另外一邊的窄路.
又來自一些教內人士的拆毀.
教務的圈子很小.
宣教士的圈子更加小.
全世界奉獻本身已經走一條窄路.

$^{641}$但另一個宣教的前輩.
他的名言帶給我們很大的鼓舞.
戴德新宣教士昔日越洋來到中國宣教.
面對重重的困難.
他曾經說過.
不信的人只見到困難.
但信的人會見到.
自己和困難之間是有上帝的.
他曾經說過.
不信的人只見到困難.
但信的人會見到.
自己和困難之間是有上帝的.
但信的人會見到.
自己和困難之間是有上帝的.
但信的人會見到.
自己和困難之間是有上帝的.
但信的人會見到.
自己和困難之間是有上帝的.
但信的人會見到.
自己和困難之間是有上帝的.
但信的人會見到.
自己和困難之間是有上帝的.
但信的人會見到.
自己和困難之間是有上帝的.
但信的人會見到.
自己和困難之間是有上帝的.
但信的人會見到.
自己和困難之間是有上帝的.
但信的人會見到.
自己和困難之間是有上帝的.
但信的人會見到.
我們流出去的人是不孤單的.
一路懷緬的時候.
讓我更加確定自己的初心.
很想將主耶穌基督.
十誡的救贖.
去拯救世界的福音.
傳給未認識祂的人.
特別是一些物質上很貧窮的人.
你們懂不懂唱這首流淌的詩歌.

$^{681}$它的名字叫流.
一般的教會很少問我們拿這個譜.
因為這是為流淌而寫成的一首詩歌.
這首流的歌.
其中這兩句.
都是我自己很喜歡的.
忘記身背後.
隨從盛風遠流.
同船托方宇宙.
全憑信心行走.
出去宣教是一個開放.
向列國觀看.
去看看這世界的一件事.
我想讓大家看看.
我曾經去.
而我又很想再去的宣教工場.
引述一下你們.
這些照片是我親手拍的.
這是一個森林.
如果你有帶攝影機.
可能會玩過一個叫做.
松鼠搬家的遊戲.
裡面有幾個口號.
松鼠搬家.
森林大火.
你就會去排隊.
你知不知道在真真正正的森林那裡.
在世界的中心呼叫.
森林大火是很棒的.
其實是很棒.
但其實不是太浪漫.
因為是很熱的.
差不多40度.
但照玩是很開心的一件事.
有人問我.
其實我準備好出發去宣教了嗎.
我想說在這個.
沒有人預計到.
忽然而來的全球大瘟疫.
或者種種我們身邊所發生的事.

$^{721}$我可以告訴大家.
我們不可能準備好所有的東西.
好像哈巴Book.
其實它照聽.
但它沒有準備好上帝會回答它什麼.
它只可以準備好自己怎樣回應.
擇路調頭是很難的.
但好像歌詞裡面這樣說.
隨著聖靈的聖風帶領.
隨著神的帶領.
上帝呼召我們.
上帝去叫我們做很多事.
我就憑信心去流出去.
東非是一個距離香港很遠的地方.
沒有直航到達的一個地方.
但我自己是很有信心.
可以在琉塘這裡.
組成到短宣隊去服侍.
因為我覺得大家都很夠流.
還有我們流人.
將來不只是香港這個地點出發去非洲.
你們還可以在歐洲下去.
在加拿大下去.
或者在澳洲上去.
有很多很多的地點可以出發.
弟兄姊妹.
我想和你們一起同行託方宇宙.
回應上帝.
回應祂.
我們一起去做很多事.
我們一起去上帝叫我們.
要去多事幹的地方.
Outflow不只是+44一個點.
我們可以隨著聖靈的帶領.
Outflow去上帝要我們去觀看的角落.
\newpage



\section{}
\label{sec:7_6qfdylGjE}
\textbf{【網上崇拜】突破「越位」|約11\_1-44|20220430 [7\_6qfdylGjE]}
\newline
\newline
連結: \href{https://youtube.com/watch?v=7_6qfdylGjE}{\texttt{ https://youtube.com/watch?v=7\_6qfdylGjE}} ~~~~ 語音日期: 2022-04-30 
\newline
\newline
\hyperref[sec:UT74cFcV7PU]{\small{< < < PREV SERMON < < <}}
~
\hyperref[sec:index_chronic]{\small{[返順時目]}}
~
\hyperref[sec:index_scriptual]{\small{[返順卷目]}}
~
\hyperref[sec:b7gyPC12_AM]{\small{> > > NEXT SERMON > > >}}
\newline
\newline
$^{1}$各位民安,大家好嗎?.
我希望你們各人都安好.
在這個星期的頭兩天.
有很多同學來到我家玩.
我們一起打牌,一起聊天.
其實我真的很久沒有試過這種.
同學相聚的感覺.
是很開心的.
我不知道這個星期大家又過得怎樣呢?.
但是在前兩天.
我看到一個香港資深的男演員.
87歲.
在酒店隔離的時候.
酒店職員因為防疫的政策.
而不敢開門.
有些說是等了衛生署的人來了才開門.
有些說可能酒店職員是等了15分鐘才開門.
但最後就已經發現他離開了世界.
而之前的一段時間.
就是當我在思考港島的時候.
我看到的同樣都是一些.
我從來沒有在香港見過的亂象.
我幾乎每一天都在網上看到一些影片和相片.
例如一個個的屍體.
竟然安放在有病人住的病房.
有人拿著一把很小的小剪刀.
去一個食肆的老闆打劫.
他明顯好像是很新手,初入行似的.
失敗了之後.
食肆老闆竟然說.
如果要上庭的話.
他會為打劫者求情.
老闆說他明白.
他明白是因為這個時勢而弄到他這樣.
我又看到一些影片.
在隔離營的人.
在大叫,在發洩.
其實這些都不好笑.
有很多很多的東西.
我心裡面是感覺.

$^{41}$我從來都沒有見過香港人好像瘋了.
我不想這樣說.
但我覺得大家都好像瘋了.
包括我自己.
大家都好像受著不同程度的苦.
上至長者,下至孩童.
生至活人,死至死屍.
都在受苦.
所以我今天選擇了約翰福音.
和大家一起思考一下.
因為約翰福音寫作的時間.
背景其實都是很瘋狂的.
他大概是在猶太戰爭失敗.
聖殿被毀之後.
甚至可能是去到羅馬皇帝多瑪田的時候.
去到屠殺大量的基督徒.
無論時間升到那裡.
約翰福音的作者.
都是在以色列人和基督徒.
受迫害的背景下.
去寫這第四本福音書.
很特別的是.
其實當時已經有三本福音書.
馬漢福音,馬太福音,勞教福音.
但正正是在他知道.
甚至可能已經讀過這三本福音書之後.
同時他面對著真實發生在.
以色列人和基督徒身上的事.
就迫他去思想.
究竟怎樣在他自身身處的環境當中.
再去看耶穌的教導.
耶穌所做過的事.
和耶穌所應許的事呢?.
所以我覺得約翰福音真的是一個很好的寶貝.
因為他是在那個時代的一個信仰反思.
他是書寫.
作者在他自身身處的環境裡面.
所重新思考的一個耶穌基督.
所以我想和大家一起去看看約翰福音.
作者面對著一個很混亂,很恐懼.

$^{81}$充滿極權和死亡的時間當中.
他怎樣用另一個角度.
一個有別於之前三本福音書的角度.
去看這個信仰.
去看我們作為基督徒要做的事.
所以今天希望大家和我一起去看看.
約翰福音第十一章.
拉薩路復活.
這段經文是約翰福音獨有的經文.
而選擇拉薩路復活.
也是很符合香港的情況.
又有提到生病和死亡.
我們一起去看看.
這個事跡大家應該不陌生.
如果大概說一下.
大家可以拿著一本聖經.
但在上面的事情是這樣的.
拉薩路是一個耶穌所熟悉的家庭當中的一個弟弟.
而他病了.
他的兩個姐姐瑪大馬利亞.
就派人傳訊息給耶穌.
請耶穌到他們那裡去醫治弟弟.
而結果耶穌沒有馬上出發.
他停留了兩天的時間.
直到耶穌說拉薩路已經死了.
他才說現在是時候要去叫醒拉薩路了.
但門徒他當時就很想勸退耶穌這個決定.
因為拉薩路住的地方很近耶路撒冷.
很容易會被有心抓耶穌的猶太領袖有機可乘.
但耶穌卻堅持要去.
而他終於去到拉薩路家附近的時候.
拉薩路已經死了四天.
已經是死透無力回天的狀態.
所以瑪大馬利亞跟耶穌說.
耶穌如果你早點來.
如果你早點來我弟弟就會康復.
但現在已經死了四天.
現在靈魂都進不去肉體裡.
已經無力回天了.
但耶穌說.

$^{121}$復活在我生命也在我.
信我的人雖然死了也必復活.
所以活著又信我的人.
必定永遠活著.
你信這話嗎.
然後他就復活了拉薩路.
因為整段經文有44節.
甚至如果我們更完整地看的話.
就要去到第12章.
所以今天我們不會這樣做.
現在我們集中看11章25-26節的經文.
這兩節經文裡面耶穌說.
我就是復活和生命.
信我的人雖然死了也要活著.
所以活著又信我的人.
必定永遠活著.
你信這話嗎.
25節我們一起看.
我就是復活.
我就是生命.
信我的人雖然死了也要活著.
這一句我們是明白的.
他是在說末日.
他在說末日上帝會復活.
所有忠於他的信道者.
我們看以賽亞書.
我們也看過但爾利書.
看過荷西亞書.
甚至我們看過兩約的文獻.
我們都知道.
以色列人一直都相信.
忠於上帝的信道者.
他只是睡了.
我們輕輕看一下經文.
以賽亞書第26章第19節.
他說.
屬你的死人要活過來.
他們的屍體要起來.
乃是住在塵土裡的必醒起.
並且歡呼.

$^{161}$因為你的金路像早晨的金路般臨到.
使地交出離世的人來.
我們再看多一個經文.
但爾利書第12章第2節.
他說.
必有許多睡在塵土中的人醒過來.
有的要得永生.
有的要受羞辱.
永遠被憎惡.
這些經文其實都是在說.
在末日審判的時候.
上帝是會復活信道者的身體.
正如馬大在第11章第24節的時候.
他也有這樣說.
我知道在末日復活的時候.
他必復活.
他說的是.
我知道在末日復活的時候.
我弟弟拉薩路都會復活.
只是在這裡.
耶穌說這句話.
是想同一時間揭示一件事.
就是.
他就是那位審判者.
所以25節的經文我們是容易明白的.
反而是26節.
他說.
所有活著又信我的人.
必定永遠活著.
這件事.
就這樣聽下去.
我們讀一讀出來.
我們有些難理解的.
好像不是很合理的.
好像有些不合邏輯的.
活著又信耶穌的人永遠不死.
那就是說.
所有基督徒都是不死身.
我們不要說現實不是這樣.
但就算在邏輯上.

$^{201}$也很難去明白.
怎樣去接上一句.
第25節所說的.
信我的人雖然死了.
這一句話.
你會不會活著.
相信的人永遠不死.
又怎會有信的人.
雖然死了呢.
我不知道大家跟不跟到這個邏輯.
所以其實26節的我一句.
活著又信我的人必定永遠不死.
其實不是真的不死.
其實他們的肉身都會死.
但在基督裡面的生命.
才是永遠不死.
而這個其實是一個很特別的呼召.
他說的是.
正正因為你在基督裡面的生命.
是不死.
是安全的.
所以你的肉身才可以放膽去死.
我上星期四的時候.
我幫忙主禮了一個安息人.
在一個安息禮才對.
我預備禮聚經文的時候.
其實我是在預備這一篇的信息.
我心裡面想的是.
這一句這麼常出現在安息禮的經文.
其實好像不是很用得.
我掙扎了很久.
但我又覺得他們的家人聽到這些福利經文.
應該會很安慰.
所以大家猜一下我最後是怎樣做呢.
我最後只給了第25節的經文.
因為第26節真的有點難.
我很難去跟他們說.
我們可以放膽去死.
這個真的很難.
但這26節的解釋.

$^{241}$我請大家弟兄姊妹暫時記住在腦子裡.
我們現在宏觀再看一看整個11章的經文.
我們一直以為.
一件事就是.
我們說拉薩路復活.
我們最重要的是什麼.
我們最重要當然是救回他的命.
但很特別的是.
作者用了幾乎整個11章.
去說如何復活拉薩路的同時.
在下一章第12章的時候.
竟然出現了很短的一句.
在第10節.
他說祭司長要殺了拉薩路.
第10節他說祭司長要殺了拉薩路.
這個是很特別的.
你想想耶穌施大忌.
他復活了一個人.
但最後在第2個章節.
就說他要死了.
所以最顛覆我的是什麼呢.
不是復活了拉薩路.
而是復活了之後.
他的後續故事.
究竟發生了什麼事呢.
為什麼拉薩路又會再死呢.
在第12章1至11節的經文裡.
耶穌去到拉薩路所住的地方.
那裡就是之前復活的地方.
當時有人為耶穌準備了一個筵席.
拉薩路就和耶穌一起吃飯.
同時有一大群猶太人.
知道耶穌就在那裡.
所以他就去找耶穌.
加上去找拉薩路.
第12章9節寫得很清楚.
他們來找拉薩路.
是因為想看看原本死了的人.
是不是真的復活了.
而他又是怎樣的呢.

$^{281}$這個吃飯的記載.
其實是在說什麼呢.
其實是在說一個見證.
耶穌和拉薩路一起坐直.
在復活發生的場地.
就是讓所有來的人.
都看到耶穌在拉薩路身上.
所成就的事情.
我們可以再看闊一點.
拉薩路這個事跡.
是在約翰福音第七個記號.
所謂的記號.
那個sign.
其實是說見證.
例如第五章.
耶穌一號坦了38年的人之後.
在五章十五節.
哪人就去告訴猶太人.
使得全宇的就是耶穌.
在第九章.
耶穌一號出生就盲眼的人.
而那個人就和他的鄰居.
和他經常見到的.
應該說和他經常見到的人.
和法利賽人.
去見證耶穌在他身上做的事.
經文甚至記載法利賽人和他爭辯.
然後可能因為說不過去.
所以就趕了這個已經復明的人.
去出會堂.
弟兄姊妹.
無論是個坦子.
還是個克子.
又或者是拉薩路.
其實他們都是遇上了耶穌基督.
去經驗了他的恩惠.
和他的能力.
然後就在眾人的面前.
包括在一些不喜歡耶穌的猶太人的面前.
去冒著被責罰的風險.

$^{321}$例如趕出會堂.
不能敬拜.
甚至是像拉薩路被祭司長鎖定要殺死那樣.
其實他們都是在見證基督.
這個讓我想起俄羅斯.
很多俄羅斯的反戰民眾上街示威.
去抗議.
但都被人抓了.
而早在一個月之前.
有一個電視台的記者.
同時據報說她也是兩個小孩的媽媽.
她衝入俄羅斯國家電視台的直播室.
她舉起了一個反戰的標語.
她事前拍了一個片.
去解釋為什麼她有這個決定.
她說近幾年來.
我本人就在俄羅斯第一頻道工作.
傳播克里姆林宮的謊言.
對此我感到十分恥辱.
我感到羞愧.
因為我令謊言得以傳播.
令俄羅斯人形同走獸.
一切從2014年開始.
但我們卻是保持沉默.
當克里姆林宮毒殺納瓦爾尼的時候.
我們沒有站出來抗議.
我們繼續對非人道的體制保持沉默.
今後十代人的努力.
都沒有辦法去洗刷這場兄弟戰爭.
所帶來的仇恨和恥辱.
弟兄姊妹.
難想像的是.
當我想到今天香港的時候.
比卡超上場之前和之後.
或許已經帶來很多意識形態的鬥爭和恐懼.
無論是對整個社會還是教會.
其實我最怕見到的是.
我們毫無掙扎地去接受所有的東西.
今天的疫苗通行證還是打兩針.
但到五月就要打第三針.

$^{361}$之後就要打第四第五第六針.
其實我越來越明白.
為什麼要越位.
因為當我們不思想.
不去掙扎如何去突破越位的時候.
我們一定做不到一個基督徒.
我們更加做不到一個人.
我也在問我自己.
我聽了《白堂道》.
究竟我如何去突破越位.
我一路思想.
我發現如果可以在一個這麼混亂的時候.
讓社會有上帝的聲音.
讓人有第二把聲音去聽一下.
去跟隨一下就好.
香港彷彿有很多把聲.
很混亂.
很多聲音.
但我發覺其實.
只有一種我們不想聽.
但迫著要聽的不良聲音.
例如我們聽到.
比卡超有七百多個精靈朋友.
我們聽到要打第四針.
我們聽到很多東西.
去限制大家的工作.
我們的學業.
甚至回教會.
甚至推動我們移民.
我身邊也有不少朋友已經離開了香港.
是完全很明白的.
但在這個社會裡面.
我發現,不知道大家有沒有同感.
但我竟然好像.
不是太聽得到教會的聲音.
在這個社會裡面.
我好像不是聽得很清楚.
上帝真理的聲音.
是如何回應.
讓我們所有香港人.

$^{401}$都可以有一個選擇去走哪一條路.
好像沒有了這一點.
或者聲音很小.
丙姐妹在第十一章第四節說.
這個病不至於死.
而是為了神的榮耀.
是令神的兒子因此得榮耀.
耶穌的榮耀說的是.
拉撒路因為遇上了基督.
連死亡都不怕去見證上帝.
耶穌的榮耀.
不只是復活了拉撒路.
不只是他復活的權柄.
他復活的能力.
不單止是這樣.
我想更多的是.
耶穌的榮耀.
是在於拉撒路生命往後的日子裡面.
他可以抵抗那個.
我們本身對死亡有天然的懼怕.
他可以抵抗這件事.
而去見證耶穌在他身上所做的事情.
而這才是基督的榮耀.
是我們今天在每一個基督徒身上.
所見到的基督的榮耀.
而這正是第二十六節所說的呼召.
因為你是在基督的裡面.
你的生命是不死的.
是安全的.
所以你的肉身可以放膽去死.
放膽去見證.
因為這次的閱題是閱位.
所以我想例子的時候.
我請教了我老公.
我看看有沒有足球的例子.
因為我完全不是足球世界的人.
他就跟我說曼聯.
我心想 什麼?曼聯?.
就算我不是足球世界的人.
我現在也知道他有點難說.

$^{441}$但他就跟我說.
以前曼聯很厲害.
他很自豪地說.
以前很多人都不喜歡曼聯.
為什麼?.
因為它厲害.
他說很多人在街上都穿著曼聯的球衣.
但現在曼聯就很廢.
我之前也曾經跟他說.
你不要再支持曼聯了.
你去支持利物浦.
你看看 像John一樣.
現在他多開心.
你支持曼聯你會很生氣.
但你支持利物浦.
你就會很開心.
我這樣說.
但他說他不會.
他會繼續堅持去支持.
他說他不會再勝利求迷.
他說如果要說他現在受什麼苦的話.
就是他經常都要捱野.
但要看屎球.
他本來很想去英國的奧脫福球場看球.
但他又很怕.
很怕這班人會踢一場垃圾的球出來.
會令他很生氣.
但他說他也不會轉會.
另外還有一個Flo Church的目者.
阿加Sir.
他說他的愛隊是愛華頓.
但可能要降班.
我早幾天和他慶祝生日的時候.
他說他會把他唯一的生日願望給了愛華頓.
他說只求他不降班.
其實我是很驚訝的.
我很驚訝這些球迷對球隊的精神.
即是球迷就算見到那一隊球隊怎樣也好.
自他們遇上那一隊球隊之後.
他們見證著這隊球隊的喜喜樂樂.

$^{481}$但他們繼續躺在身上.
身體力行.
不怕被人笑他們是曼聯的粉絲.
他們也不是只有嘴巴.
但他們仍然去看球去支持他們.
所以我很想鼓勵Flo Church的弟兄姊妹.
正正是前面環境好像很絕望.
其實是很絕望.
是覺得香港沒有了.
香港教會也會很快變天.
就是這個時候.
讓基督在你身上不死.
去努力思考.
怎樣把基督給我們生命的改變.
給我們的力量去見證出來.
因為是在這個時候.
就是這個最困苦的時候.
讓人見到基督徒的應對是不同的.
心境是不同的.
是會想辦法去突破的.
這樣才是見證基督.
這樣才是我們常常說的全福音.
我多說一點點.
怎樣去理解約翰福音作者.
在當時充滿迫害的時期.
還要寫拉薩路福這件事.
我想其中一個原因.
其中一個意思可能是這樣.
當耶穌復活了拉薩路的時候.
其實對馬大馬利亞.
甚至是很多猶太人來說.
其實是非常非常震驚.
因為耶穌是在當刻.
本身在末日審判的時候.
才會做的一個神蹟.
而這個神蹟只有上帝才做得到.
同時這個神蹟的出現.
也代表著以色列要復興.
所以我們會明白.
為什麼這麼多猶太人.

$^{521}$會專門來找耶穌和拉薩路.
去驗證一下.
是不是真的有這件事.
真的有復活這件事.
是不是的.
所以當我們去看復活的時候.
它不單只是關乎一個人.
或者一個家庭的事情.
對以色列人來說.
復活是代表一個民族的事情.
是他們對上帝工作的盼望.
但其實當時.
經文記載著.
真的有很多很多的猶太人.
甚至是外邦人.
都是因為復活了拉薩路這件事.
去相信了耶穌.
很多人都在夾道和他歡呼.
去期待耶穌復興以色列.
不過重點是.
我們記得我們在看約翰福音.
約翰福音當時的讀者群體.
他們是知道.
以色列是沒有復興.
他們甚至是在一場猶太戰爭裡面.
死了超過一百多萬個猶太人.
連聖殿都被奪去.
作者為什麼還要記載.
拉薩路復活的這件事呢.
我想其中一個原因是.
他很想告訴他們.
還有很多人和你們一樣.
因為見證基督而犧牲.
就好像拉薩路一樣.
就好像生來的核子一樣.
每一個之前因為在真理上不妥協.
在自己忠於上帝的身份上不妥協.
而受苦犧牲的人.
現在都成為我們.
承傳下去的一份力量.

$^{561}$如果我們記得的話.
其實本身約翰福音的作者.
寫這本書的時候.
的目的其實就是說.
是一個對耶穌基督的見證.
在21章第24節.
就是約翰福音最後九個經文.
作者說.
為這些事所見證.
並且記述這些事的.
就是這門徒.
我們知道他的見證是真實的.
約翰福音就成為了.
作者在身處我國的時候.
為耶穌所作的見證.
而我們每一個人.
都有一個為耶穌去到.
作見證自己最獨特的一份.
我嘗試去到進入足球的世界.
我真的去看了三條.
關於越位的youtube片.
去了解越位.
我發現原來越位.
是一個位置上的判定.
當進攻球員不能夠.
在不計龍門的最後一個防守球員後面.
更加近那條底線.
大家可以留言告訴我.
我理解有沒有錯.
所以他說球員要做的.
其實是要突破越位.
而不停的越位其實是一個嘗試.
他為的是最後要突破越位的我一下.
我老公說其實一場球賽.
不一定要搏過突破越位才會贏.
他說搏其實是一種心態.
他說是一個想爭取機會的心態.
他說如果你要說突破越位的話.
最出名的一個人就是.
這個我從來都不認識的人.

$^{601}$就是意大利的恩沙基.
他說有些人覺得他厲害.
有些人覺得他廢.
他會在某個位置上.
去等待某個時刻.
又有些人會突破越位是靠跑的.
當隊友傳球的時候.
他就努力跑到最後一個防守球員的後面.
然後控著球單對單的面對龍門.
然後就可以增加他入球的機會.
所以突破越位原來是一個心態.
是一個想爭取機會.
想再做多一步的心態.
最重要的是他不會怕有一條線.
他反而覺得這條線是可以突破的.
所以我們現在面對很多條線.
但最重要的是.
我們不要覺得這條線是突破不了的.
我們不能複製耶穌所做的事.
我們的越位也不是我們越到耶穌的位置.
去做耶穌所做的事.
耶穌所做的死人復活.
我們怎可以叫死人復活呢.
不過讓我們看看的是.
約翰福音第十一章第七節.
耶穌對門徒說.
我們再去猶太去吧.
然後門徒說.
拉比近來猶太人要用石頭打你.
你還要去哪裡去嗎.
耶穌示範了一件事.
他如何在一個困難重重的處境底下.
仍然去找拉撒路.
去幫他去救他.
其實他知道拉撒路所處的地方.
其實跟耶路撒冷很近.
他去就等於主動將自己置身在一個危機的裡面.
在十一章第十六節.
那稱為傷身者的多瑪.
對其他的門徒說.

$^{641}$我們也去跟他一同死吧.
門徒示範了他們如何在生死的掙扎裡.
仍然願意跟耶穌一起去.
我們還說的拉撒路.
甚至我們有更多的先賢先成.
無論耶穌門徒還是拉撒路.
他們都是在困難重重的處境底下.
仍然去堅持.
仍然不捨身不怕死地去見證上帝.
什麼是一個驅動.
我剛才在敬拜的時候.
我心裡很感動.
我再次回想起我跟耶穌的相遇.
是在一個公園.
就是在一個公園.
我一路走一路哭.
我問上帝.
我是在別人面前好.
我是在別人面前乖.
上帝是知道我的內心.
但為什麼仍然釘在十字架上.
你不做紅水就有大火.
耶穌你怎麼可以.
怎麼可以因為我們而受罪.
我怎麼配.
在公園裡.
我求上帝降服我.
我求上帝跟隨我.
我跟了好一段時間.
可能我也有很多做錯的.
很多事情我也不知道該怎麼做.
可能我也不是很突破.
但當耶穌跟我相遇的那一刻.
直到現在.
祂一直在我生命裡.
是祂推動我去做.
我應該要去做的事.
是祂給我堅持的心智.
是祂給我所有突破的力量.
否則我真的會很懶.

$^{681}$我真的走著走著看Netflix.
我就很開心.
我真的不用約人出街.
我很想透過這一段聖經.
鼓勵Flu Church的弟兄姊妹.
大家在自己的工作崗位裡.
其實都面對著很多的事情.
特別是在未來可能會有更多的立法.
對自由有更多的限制.
我們可以想像有更多的公權暴力.
但我相信Flu Church的弟兄姊妹.
都是深深地去經歷上帝.
有祂給我們一個不怕死的生命.
如果我們自覺沒有的話.
我們真的要祈禱.
我相信上帝會給我們有充分的恩典和力量.
只要我們有一個.
我們不怕去到突破那條線的心.
我們覺得是可以突破的.
前面會有更加多不能夠接受.
匪夷所思的事情會發生.
但正正是這樣.
今天才是我們毫無保留去見證上帝的時候.
讓我們一起去到.
想想自己可以在什麼位置見證到.
一個一直和我們一起的耶穌基督.
我們一起祈禱.
耶穌基督求你幫助我們.
去追隨我們很多的自以為的界線.
自以為很多不行不行的東西.
又或者是什麼都行行行行的東西.
我們裡面的生命.
祝我們此刻降伏在你的十格下面.
我求主你重塑我們的命.
我求主你讓我們真誠的面對自己.
讓我們再一次遇上你.
再一次被基督你激動我們.
在你面前這樣去禱告.
奉主耶穌基督得勝的明智而求.
阿門.

\newpage



\section{路加福音 15:11-31-20220507}
\label{sec:b7gyPC12_AM}
\textbf{【網上崇拜】首先學習如何被擁抱|路加福音15\_11-31|20220507 [b7gyPC12\_AM]}
\newline
\newline
連結: \href{https://youtube.com/watch?v=b7gyPC12_AM}{\texttt{ https://youtube.com/watch?v=b7gyPC12\_AM}} ~~~~ 語音日期: 2022-05-07 
\newline
\newline
\hyperref[sec:7_6qfdylGjE]{\small{< < < PREV SERMON < < <}}
~
\hyperref[sec:index_chronic]{\small{[返順時目]}}
~
\hyperref[sec:index_scriptual]{\small{[返順卷目]}}
~
\hyperref[sec:97SC38c6sqY]{\small{> > > NEXT SERMON > > >}}
\newline
\newline
路加福音 15:11-31-20220507
\newline
\begin{longtable}{cl}
\hline
\hline
章節 & 經文 (和合本修訂版)\\
\hline
15:11 & \begin{tabularx}{0.7\textwidth}{X} 耶穌又說:「一個人有兩個兒子。 \end{tabularx} \\ \\ \relax
15:12 & \begin{tabularx}{0.7\textwidth}{X} 小兒子對父親說:『父親,請你把我應得的家業分給我。』他父親就把財產分給他們。 \end{tabularx} \\ \\ \relax
15:13 & \begin{tabularx}{0.7\textwidth}{X} 過了不多幾天,小兒子把他一切所有的都收拾起來,往遠方去了。在那裡,他任意放蕩,浪費錢財。 \end{tabularx} \\ \\ \relax
15:14 & \begin{tabularx}{0.7\textwidth}{X} 他耗盡了一切所有的,又恰逢那地方有大饑荒,就窮困起來。 \end{tabularx} \\ \\ \relax
15:15 & \begin{tabularx}{0.7\textwidth}{X} 於是他去投靠當地的一個居民,那人打發他到田裡去放豬。 \end{tabularx} \\ \\ \relax
15:16 & \begin{tabularx}{0.7\textwidth}{X} 他恨不得拿豬所吃的豆莢充飢,也沒有人給他甚麼吃的。 \end{tabularx} \\ \\ \relax
15:17 & \begin{tabularx}{0.7\textwidth}{X} 他醒悟過來,就說:『我父親有多少雇工,糧食有餘,我倒在這裡餓死嗎? \end{tabularx} \\ \\ \relax
15:18 & \begin{tabularx}{0.7\textwidth}{X} 我要起來,到我父親那裡去,對他說:父親!我得罪了天,又得罪了你, \end{tabularx} \\ \\ \relax
15:19 & \begin{tabularx}{0.7\textwidth}{X} 從今以後,我不配稱為你的兒子,把我當作一個雇工吧。』 \end{tabularx} \\ \\ \relax
15:20 & \begin{tabularx}{0.7\textwidth}{X} 於是他起來,往他父親那裡去。相離還遠,他父親看見,就動了慈心,跑去擁抱著他,連連親他。 \end{tabularx} \\ \\ \relax
15:21 & \begin{tabularx}{0.7\textwidth}{X} 兒子對他說:『父親!我得罪了天,又得罪了你,從今以後,我不配稱為你的兒子。』 \end{tabularx} \\ \\ \relax
15:22 & \begin{tabularx}{0.7\textwidth}{X} 父親卻吩咐僕人:『快把那上好的袍子拿出來給他穿,把戒指戴在他指頭上,把鞋穿在他腳上, \end{tabularx} \\ \\ \relax
15:23 & \begin{tabularx}{0.7\textwidth}{X} 把那肥牛犢牽來宰了,我們來吃喝慶祝; \end{tabularx} \\ \\ \relax
15:24 & \begin{tabularx}{0.7\textwidth}{X} 因為我這個兒子是死而復活,失而復得的。』他們就開始慶祝。 \end{tabularx} \\ \\ \relax
15:25 & \begin{tabularx}{0.7\textwidth}{X} 「那時,大兒子正在田裡。他回來,離家不遠時,聽見奏樂跳舞的聲音, \end{tabularx} \\ \\ \relax
15:26 & \begin{tabularx}{0.7\textwidth}{X} 就叫一個僮僕來,問是甚麼事。 \end{tabularx} \\ \\ \relax
15:27 & \begin{tabularx}{0.7\textwidth}{X} 僮僕對他說:『你弟弟回來了,你父親因為他無災無病地回來,把肥牛犢宰了。』 \end{tabularx} \\ \\ \relax
15:28 & \begin{tabularx}{0.7\textwidth}{X} 大兒子就生氣,不肯進去,他父親出來勸他。 \end{tabularx} \\ \\ \relax
15:29 & \begin{tabularx}{0.7\textwidth}{X} 他對父親說:『你看,我服侍你這麼多年,從來沒有違背過你的命令,而你從來沒有給我一隻小山羊,叫我和朋友們一同快樂。 \end{tabularx} \\ \\ \relax
15:30 & \begin{tabularx}{0.7\textwidth}{X} 但你這個兒子和娼妓吃光了你的財產,他一回來,你倒為他宰了肥牛犢。』 \end{tabularx} \\ \\ \relax
15:31 & \begin{tabularx}{0.7\textwidth}{X} 父親對他說:『兒啊!你常和我同在,我所有的一切都是你的; \end{tabularx} \\ \\ \relax
15:32 & \begin{tabularx}{0.7\textwidth}{X} 可是你這個弟弟是死而復活,失而復得的,所以我們理當歡喜慶祝。』」 \end{tabularx} \\ \\
[1ex]
\hline
\hline
\end{longtable}
$^{1}$靜姐妹平安.
很開心 很開心 很開心.
我們一齊回來.
我們一齊獻上敬拜.
Hallelujah.
讓這個地方成為一個敬拜的地方.
我們一起的.
讓這個地方成為一個敬拜的地方.
今天令我想起我們留堂三年前第一次崇拜.
差不多這樣的情景.
靜姐妹你們平安嗎.
在疫情當中.
歡迎你回來.
網上靜姐妹也願你平安.
特別是倫敦靜姐妹.
今天是你們第一次的聚會.
在這裡和你送上耶穌基督的問候.
很開心我們能夠一齊去敬拜.
我求主讓我們今天有一個完整平安的敬拜.
大家能夠在當中去珍惜.
去聆聽上帝的話.
去一齊去擁抱.
這個月Full Church的月題是擁抱.
如果要為Full Church這個月題擁抱下一個定義的時候.
擁抱是一個身體的接觸.
不單是我們和人的關係.
更加是我們生命的熱度.
擁抱的對象不單單是人.
更加是一些信念和我們的想像.
不單單是贊同.
更加是用生命來投入.
分身來參與.
擁抱更加是天父上帝的愛.
我們首先被天父去擁抱.
然後再去擁抱別人.
以及天父的世界.
當然擁抱更加是行動.
我們實際地行出這個尾線.
或者我們可以從擁抱的希臘文說起.
以後你去希臘旅行.

$^{41}$當你想擁抱的時候.
你可以用這個字.
擁抱的希臘文是epipedo.
再說一次,epipedo.
以後你看到一個外國人.
epipedo.
大家一起問候.
epipedo這個字字面上的解釋.
解作fall on someone.
fall oneself on someone的意思.
整個人全躺在你身上.
所以當你去擁抱的時候.
當我擁抱你.
用百多磅的身體來fall on你的身體的時候.
這個就是擁抱.
這個字引申下來就是臨格.
fall upon的意思.
所以聖靈的臨格.
或者當然不幸事件的臨格.
都會用到epipedo這個字.
所以聖靈降臨.
也可以勉強地解作聖靈的擁抱.
或者大禍臨頭.
都可以解作不幸的擁抱.
聖經什麼時候出現epipedo這個字呢?.
其中一段就是今天所說的浪子的比喻.
大家都聽過很多很多次浪子的故事.
都聽過很多不同浪子比喻的詮釋.
無論你聽過的比喻有多不同.
最少都肯定的是浪子比喻的父親.
是比喻什麼呢?.
是比喻我們天上的父上帝.
我們的天父.
所以今天我們一起來思考.
用浪子比喻來思考我們天上的父神.
我們將要去學習.
我們將要去學習浪子比喻有關擁抱的課題.
我們一起來祈禱吧.
祝我們一起來學習擁抱.
讓我們能夠在當中去知道.

$^{81}$怎樣去認識自己的擁抱.
幫助孩子.
幫助我們今天聽到的每一個.
讓我們能夠被你的聖靈去根深.
用行動活出你的說話.
讓我們Full Church成為一個彼此相愛.
在一個這樣的年代裡.
彼此去擁抱的群體.
幫助我們.
奉主命求 阿們.
或者我們可以看看.
講一下比喻裡面擁抱的那一幕.
在第二十節的經文.
聖經說 小兒子於是起來往他父親那裡去.
他還在遠處的時候.
父親看見他 就動了廉穩.
跑過去擁抱他 親吻他.
這是我自己的翻譯.
比較重現代感的翻譯.
這就是父親擁抱兒子的那一幕.
就是小兒子回家的那一幕.
非常感人.
我們都記得.
這就是我們熟悉的浪子的故事.
小兒子敗了父親的家產.
用盡所有的金錢.
最後落泊在豬欄裡面.
跟豬同食.
用手去吃豬哨.
生活很潦倒很潦倒的情景.
然後聖經記載了一句這樣的說話.
他突然醒悟過來.
叮一聲.
就在這個時候小兒子就想通了.
聖經說小兒子醒悟過來就說.
我父親有多少的故宮.
口糧有餘.
我賭在這裡餓死嗎.
我要起來到我父親那裡去.
向他說.

$^{121}$父親我得罪了天.
又得罪了你.
從今以後我不配為你的兒子.
把我當作一個故宮吧.
這段是一段電視劇的心聲.
明明旁邊沒有人.
他就無端端在那裡.
有些心聲就說出來.
這個就是浪子回頭裡面.
最經典的一幕.
小兒子吃著吃著豬哨.
口水鼻涕都流了.
突然間背景音樂.
小兒子就頓悟.
小兒子悔改.
我們傾向覺得.
小兒子回傳悔改.
發生在豬欄這一幕.
我稱之為豬欄時刻.
所謂豬欄時刻就是一個人.
醒覺覺悟.
回歸上帝的一個轉捩點.
其實這幾年我講報道會.
很多時候都在用這段經文.
都會強調豬欄時刻.
我們人生裡面.
就是叮一聲.
我們發現我們需要上帝.
我就在報道會裡面去講.
縮音.
不過鄭敬之議員.
當我這次再去預備這段講道的時候.
當我再次去思想這段經文的時候.
我發現有些不是很對勁的地方.
讓我好好來解釋給大家聽.
如果你細心去研究這段經文.
你不難發現.
這段經文裡面.
其實是重複了小兒子這段頓悟的說話.
即是18節和第21節.

$^{161}$這兩節經文其實都是.
小兒子重複.
exactly重複同一句的說話.
18節我要起來到我父親去.
向他說.
父親我得罪了天也得罪了你.
從今以後我不配作你的兒子.
把我當作你的故宮辦.
21節.
小兒子就這樣說.
父親我得罪了天也得罪了你.
從今以後我不配作你的兒子.
即是以前我也有這樣的發現.
基本上兩節經文幾乎是一模一樣.
小兒子在豬欄裡面醒覺.
父親我得罪了天也得罪了你.
我不配作你的兒子.
於是小兒子就立心的回家.
當他遠遠見到他父親的時候.
就親口跟父親說.
父親我得罪了天也得罪了你.
從今以後我不配作你的兒子.
兩節經文是幾乎一模一樣.
分別在於前者是心聲.
小兒子在豬欄裡面心聲的想事情.
或者是小兒子回家的時候.
親口對父親說的一句話.
這是我以前也發現了的事.
不過兩節經文.
最近我有一個新的發現.
一個真的mind blowing的發現.
其實究竟小兒子在什麼時候悔改呢.
究竟小兒子真真正正悔改的那一刻.
是在哪裡呢.
是第十八節還是第二十一節呢.
為什麼會這樣問呢.
What if小兒子其實不是因為知錯.
而是為了生存.
What if小兒子在豬欄裡所謂的醒悟.
只不過是想到一條縮數生存的市橋.

$^{201}$What if小兒子回家的目的.
不是因為他悔改.
而是因為要賺錢呢.
我們再看經文.
七節小兒子醒悟過來就說.
我父親有多少的故宮口糧有餘.
我使餓死.
我老爸明明有很多錢.
我何必餓死呢.
十八節於是小兒子想到一條縮數市橋.
決定回家賺錢.
我要起來.
到我父親那裡去.
向他說.
父親我得罪了天.
我得罪了你.
從今以後我不配做你的兒子.
打我當作你的故宮吧.
原來對於小兒子來說.
這個醒悟.
這次回家.
只是一次補就業計劃.
最少他回家.
能夠成為一名故宮.
賺到錢.
有一定收入.
不用在豬欄那裡吃豬哨.
而且所謂的豬欄moment.
並不存在.
在豬欄裡的小兒子.
其實他沒有悔改.
只是在想怎樣生存.
回家只不過是手段.
悔改為了吃.
這就是我mind blowing的發現.
我真的發覺原來是這樣的.
不過我想說.
就算是這樣也好.
浪子的比喻是真的.
就算耶穌所說的浪子比喻也是真的.

$^{241}$浪子真的悔改到最後的時候.
浪子真的悔改.
只是不是發生在第十八節.
而是在第二十一節.
剛才不是這樣說嗎.
十八節和二十一節幾乎是一樣的.
幾乎.
只有一個很不同的地方.
十八節和第一節只有一個很不同的地方.
大家有沒有發現.
十八節和第一節有什麼不同.
沒錯.
就是有一句.
十八節是多了一句.
父親我得罪了天 我得罪了你.
從今以後我不配為你的兒子.
把我當作你的故宮吧.
有一節.
父親我得罪了天 得罪了你.
從今以後我不配為你的兒子.
完.
你發現兩句唯一的差別就是.
當小兒子見到父親之後.
他就刪掉了成為故宮的一句.
是不是.
當小兒子回到家的時候.
當他真真正正遇到爸爸的時候.
小兒子就再沒有向爸爸提出.
做故宮的那個要求.
小兒子就收口沒有說.
原本打算說的.
最後不說.
說不出口.
小兒子再沒有將回家.
成為一個賺錢的手段.
他都沒有去成為一個目的.
小兒子真真正正感受到.
爸爸的愛.
他悔改了.
父親我得罪了天.

$^{281}$又得罪了你.
這句話再不是語言藝術.
而是真真正正小兒子.
悔改頓悟流淚的說話.
他真真正正覺得自己不配.
成為這個兒子.
他真是悔改了.
所以這兩節經文表面上是一樣.
其實是完完全全不同的態度.
你問到底發生什麼事.
究竟小兒子的悔改.
在18至21節中間.
發生什麼事呢.
答案正正就是中間加了那句的第20節.
就是這段擁抱的經文.
我們再一次恭敬的來到聆聽經文.
第十節.
小兒子於是起來往他父親那裡去.
他還在遠處的時候.
父親看見他就動了憐憫.
跑過去擁抱他親吻他.
先記載當父親見到這個小兒子的時候.
就馬上跑過去擁抱他親吻他.
小兒子深深的被父親擁抱.
千言萬語都比不上這個擁抱.
在這個時候他就真真正正悔改了.
你問十分感恩.
有沒有這麼兒戲啊.
有沒有這麼簡單啊.
擁抱一下親一下就悔改了.
當然沒有這麼兒戲.
沒有這麼簡單.
因為父親這個擁抱並不是一個簡單的擁抱.
今天我們就一起來解釋一下這個擁抱是怎麼樣的.
我們一起來學習如何被擁抱.
要明白這個擁抱.
我們要回到整個浪子比喻的開頭.
第十二節.
整個浪子比喻.
起始一個非常簡單.

$^{321}$簡單得可以的幾句話.
耶穌說有一個人有兩個兒子.
小兒子對父親說.
父親請你把我應得的家業分給我.
他父親就把家業分給他們.
大家覺得很奇怪.
整個故事的節奏很奇怪.
小兒子對父親說.
爸爸你給我一些家產吧.
父親就把家產給了他們.
這個故事我覺得很密鬥.
從前有小朋友死了.
小兒子問父親要家產就給了.
爸爸我想要家產就給了.
整件事很快.
整件事發展到一個很不尋常的地步.
讀到這個故事不禁會去問.
究竟這個父親在想什麼.
他在做什麼.
既然這個父親這麼愛他的兒子.
怎麼會讓自己的兒子離開.
為什麼他不說不可以.
不准 不給.
總之他有很多很多方法.
可以叫小兒子留在他身邊.
我們問究竟這個父親在想什麼.
我以前不明白.
不過我現在身為人父了八年.
我慢慢開始明白.
也越來越體會.
浪子比喻這個父親究竟在做什麼.
我女兒今年八歲.
如果你看過我第一本書.
就是法輪功的書.
我有寫我的女兒.
那時候她只有一兩歲.
我寫我教她怎麼祈禱.
寫海南雞飯的靈修.
都提到我的女兒.
這樣就八年了.

$^{361}$還記得第一次去認識她的時候.
姑娘教我怎麼抱她.
頭放在這裡.
雙手繞著屁股.
緊緊地抱著.
自此之後抱女兒.
基本上是每天都做的事情.
抱著女兒上山.
抱著女兒坐船.
抱著女兒吃飯.
抱著女兒睡覺.
抱著女兒看聖經.
抱著女兒洗鋼章.
抱著女兒上崇拜.
抱著女兒去旅行.
有一段時間.
尤其是長洲很熱的天氣.
我曾經有一個感覺或幻覺.
覺得自己動來動去已經分不開.
汗水是疊在一起.
抱著.
時間過得很快.
就八年了.
那時候Amy目者.
還是剛入學的時間.
在神學院的一年級.
現在她已經是B級了.
時間過得很快.
這都是為人父母的經驗.
基本的東西.
不過.
你知不知道.
後來我發現.
有一個很有趣的事.
有一個很無奈的事.
原來當你和你的女兒.
相處日子越久.
當你和你的女兒.
關係越深.
當你越來越愛你的女兒的時候.

$^{401}$你發現.
有一件事是剛好.
正是相反的事情.
來到她的身上.
你發現.
你抱她的機會是越來越少.
真人.
即是她今年八歲.
相比起八年前.
手抱的時候.
我和她的相處日子.
和她的感情毫無疑問.
肯定是比八年前更加多.
更加深.
更加多的內容.
以前是一個單向的關係.
八年之後.
是一個雙向的歷史.
但你卻發現.
你能夠擁抱她的時間.
只會越來越少.
現在越來越懷念.
一隻熱騰騰的小樹熊.
抱起來的感覺.
所以那段時間.
即是半年前左右.
我在Facebook發了一段文字.
叫做「這是我最後一次抱你嗎?」.
「這是我最後一次抱你嗎?」.
「差不多一兩年沒有抱你」.
「你說你走得有點累」.
「兩個月巴巴肚你好嗎?」.
「我就抱起了你」.
「一個巨大的人兒」.
「兩個月 一歲 兩歲 八歲」.
「我都一直緊緊抱著你」.
「鬧市的商場 炎熱的山道」.
「沉靜的步伐 勉強的雙臂」.
「撐著 撐著」.
「如今 我抱起了你」.

$^{441}$「還抱著僥倖」.
「從海港城到尖沙咀碼頭」.
「對面的旁人送上懷疑的目光」.
「這麼大還要抱嗎?」.
(笑聲).
「不 一點也不大」.
「什麼時候 若是可以」.
「我都願意繼續抱下去」.
「但是 你已經到步了」.
「我默默地把你放下」.
「你靈巧地從背包拿出八達通」.
「入閘 收起 繼續向前行」.
「我在你身後 一直在你身後」.
「這是我最後一次抱你嗎?」.
作為父母 你也很明白.
放手是你擁抱你的兒女的一部分.
或者說 放手是擁抱的必經階段.
明明可以緊緊地擁抱.
卻以放手作為擁抱的方式.
去成就更加大的擁抱.
不知道你明白這句話的意思嗎?.
這句話不是分手的說話.
明明可以緊緊地擁抱.
卻以放手作為擁抱的方式.
來成就更大的擁抱.
不是分手.
說到父母.
擁抱有兩個階段.
Step 1.
每個擁抱 首先你要把雙手放開.
空空如也地張開雙手.
Step 2.
當你張開雙手後.
你便能夠緊緊地擁抱一個人.
你不張開雙手 永遠都擁抱不到一個人.
接著 你越想來一個迫撼.
就更加要把雙手放開 放得越多.
這是我們每個父母都學習的事情.
放手是你對兒女另一種方式的擁抱.
我都明白為何婚禮有一個環節.

$^{481}$就是爸爸要牽著女兒走進教堂的一幕.
我們說這一幕是必須的.
不可以沒有的.
因為每個爸爸都是在女兒結婚之前.
暫時來說全世界抱過新娘最多的男人.
所以爸爸牽著女兒走進教堂.
牽著女兒的手交給新郎.
是父親對女兒最大的擁抱.
放開.
我們開始明白浪子比喻中的父親做了甚麼.
浪子比喻中 爸爸的擁抱其實是一回甚麼事情.
我們明白這個擁抱其實真正發生在何時.
父親的擁抱原來是起始於第十二節那一刻.
完成在第二十節那一幕 相擁的一刻.
表面上擁抱出現在第二十節.
其實擁抱在第十二節就已經開始了.
表面上爸爸好像是放手.
實際上這是爸爸擁抱的現身.
表面上小兒子拿了家產就離開了.
實際上他依然從來一直都在爸爸的懷抱裡面.
沒有離開過.
如果你再看希利文的時候.
家產那個字 原文是Biox.
就是生命的意思.
爸爸將他生命的一半放手分給他的小兒子.
將他自己的生命完完全全放上去交託.
正正是父親的擁抱.
我們天父上帝的擁抱.
小兒子走得多遠.
這個天父上帝的擁抱就有多大.
小兒子離開多遠.
天父上帝的擁抱就有多廣闊.
小兒子離開多久.
天父上帝預備這個擁抱就有多久.
這個是我們天父上帝的擁抱.
今天我停在這裡.
今天就來做一些應用和總結.
原本這篇講章其實是打算整篇都講完.
先講小兒子的擁抱.
再講大兒子的擁抱.

$^{521}$不過今天太多了.
今天就容許我等下個月再講下集.
我就改講這篇講到下集.
講大兒子的部分.
我今天才知道明天是甚麼事.
明天是特首選舉.
我真的不說不知道.
我真的不知道.
你知道嗎.
我真的不知道.
(觀眾:你不說就不算了).
OK OK OK.
我想說從明天開始.
就有一隻比卡超不停地抱著你.
We and Us.
咳嗽.
我不明白為何有一個不遵守文法的人.
會叫人家守法.
明明都不遵守文法.
那怎會守法呢.
即是面對這幾年的香港.
即是這麼難受的疫情政治.
面對著一班這麼難受的人.
我們真的很需要擁抱.
所以今個月的月提一出.
當大姐們都知道今個月提是擁抱的時候.
很多的大姐們都已經讚好.
今個星期的海報.
我發現特別多讚好.
即是他們覺得.
看到這個字.
都已經讚好.
擁抱 真棒.
即是他們不用聽說甚麼.
單單說擁抱他們就覺得.
很棒 很安慰 很滿足.
我們都很想給天父去擁抱.
不過我不知道你怎樣去理解.
天父上帝的擁抱.
即是你的意思是.

$^{561}$你期望天父怎樣擁抱你.
即是突然間有一道白光.
整個人感到很溫暖.
然後深深感到天父上帝的愛.
在你身體裡面充滿你.
這個是真實的.
但這個不是經常都發生的.
事實上很多的情況.
都是挫折 無力.
沮喪巨多.
我預備這篇講道的時候.
在YouTube隨便搜尋一下.
天父擁抱這四個字.
看看有甚麼片出現.
得出的結果如我所料.
都是一大堆虛無縹緲的東西.
即是想想自己搜尋就知道.
都是一些不是很實在的東西.
不過公道一點說.
是無可避免的 確實也是.
天父的擁抱.
是未必如你想像中那麼直觀.
未必那麼容易去明白.
或者明顯感覺到.
或者問題重點不是我們要知道.
天父的擁抱.
而是我們要知道.
在我們沒有被發現的情況下.
我們仍然知道天父上帝擁抱我們.
這個是我們今天去思考的課題.
一些嘗試去感受的事情.
讓我們今天首先學習如何被擁抱.
認識天父上帝的擁抱.
這篇經文告訴我們.
天父的擁抱在你還沒察覺的時候.
其實一早就已經開始.
等你發現天父擁抱著你的時候.
其實天父已經是擁抱著你很久了.
這個就是這個道理.
再說我女兒多一點.

$^{601}$我女兒很喜歡在我的床上睡.
嚴格來說是她喜歡在我的床上睡著.
她知道每一晚都要回到自己的床上睡覺.
但她經常都求我們.
讓她在爸爸的床上睡著.
然後半夜爸爸就抱她回到床上.
她有這樣的癖好.
於是這麼多年我都是很多晚.
凌晨一兩點.
我都是靜悄悄的.
過過雪雪的抱著一個十幾十磅的小朋友.
睡到完全沒有知覺的小朋友.
抱她回到床上睡.
等於是一個完全沒有被發現的擁抱.
其實是真真正正的擁抱.
天父上帝的擁抱或者都是這樣.
在你還沒察覺的時候.
其實一早就已經開始.
等你發現他在擁抱你的時候.
其實他一早一直都在擁抱著你.
很久了.
雖然你不容易去感覺到這個擁抱.
但這個天父上帝的擁抱一直都在.
因此神姐妹讓我們好好去學習去擁抱別人.
讓我們好好學習去擁抱別人.
接下來我們將會有八篇有關擁抱的講道.
首先容許我先去賦予擁抱第一個的意義.
擁抱 真真實實的擁抱.
人與人之間的擁抱.
是天父擁抱我們的記號.
它像一個暖暖的reminder.
一個比溫馨更加溫馨.
一個溫暖的溫馨提示.
去提醒你 你仍然要記得.
天父正在擁抱著你.
天父仍然擁抱著這個世界.
是的 天父的擁抱是不容易察覺的.
但天父仍然在擁抱著這個世界.
這個尚未得熟的世界.
擁抱作為一個記號.

$^{641}$正正去提醒你自己.
也提醒你想提醒的人.
天父上帝仍然在擁抱著.
這是我們擁抱第一個的意義.
因此當你擁抱一個人的時候.
你正在提醒他 提醒自己.
天父正在擁抱我們.
相反地 如果你想去提醒自己.
提醒一個人 天父的愛的存在.
你嘗試用擁抱去提醒他.
求主幫助我們 求主擁抱我們.
求主讓我們成為.
他擁抱哀傷受苦者延伸出來的傍臂.
阿門.
\newpage



\section{馬可福音 10:13-16-20220514}
\label{sec:97SC38c6sqY}
\textbf{【網上崇拜】最終的擁抱|馬可福音10\_13-16|20220514 [97SC38c6sqY]}
\newline
\newline
連結: \href{https://youtube.com/watch?v=97SC38c6sqY}{\texttt{ https://youtube.com/watch?v=97SC38c6sqY}} ~~~~ 語音日期: 2022-05-14 
\newline
\newline
\hyperref[sec:b7gyPC12_AM]{\small{< < < PREV SERMON < < <}}
~
\hyperref[sec:index_chronic]{\small{[返順時目]}}
~
\hyperref[sec:index_scriptual]{\small{[返順卷目]}}
~
\hyperref[sec:2t83SY_sddQ]{\small{> > > NEXT SERMON > > >}}
\newline
\newline
馬可福音 10:13-16-20220514
\newline
\begin{longtable}{cl}
\hline
\hline
章節 & 經文 (和合本修訂版)\\
\hline
10:13 & \begin{tabularx}{0.7\textwidth}{X} 有人帶著小孩子來見耶穌,要他摸他們,門徒就責備那些人。 \end{tabularx} \\ \\ \relax
10:14 & \begin{tabularx}{0.7\textwidth}{X} 耶穌看見就很生氣,對門徒說:「讓小孩到我這裡來,不要阻止他們,因為在神國的正是這樣的人。 \end{tabularx} \\ \\ \relax
10:15 & \begin{tabularx}{0.7\textwidth}{X} 我實在告訴你們,凡要接受神國的,若不像小孩子,絕不能進去。」 \end{tabularx} \\ \\ \relax
10:16 & \begin{tabularx}{0.7\textwidth}{X} 於是他抱著小孩子,給他們按手,為他們祝福。 \end{tabularx} \\ \\ \relax
10:17 & \begin{tabularx}{0.7\textwidth}{X} 耶穌剛上路的時候,有一個人跑來,跪在他面前,問他:「善良的老師,我該做甚麼事才能承受永生?」 \end{tabularx} \\ \\ \relax
10:18 & \begin{tabularx}{0.7\textwidth}{X} 耶穌對他說:「你為甚麼稱我是善良的?除了神一位之外,再沒有善良的。 \end{tabularx} \\ \\ \relax
10:19 & \begin{tabularx}{0.7\textwidth}{X} 誡命你是知道的:『不可殺人;不可姦淫;不可偷盜;不可作假見證;不可虧負人;當孝敬父母。』」 \end{tabularx} \\ \\ \relax
10:20 & \begin{tabularx}{0.7\textwidth}{X} 他對耶穌說:「老師,這一切我從小都遵守了。」 \end{tabularx} \\ \\ \relax
10:21 & \begin{tabularx}{0.7\textwidth}{X} 耶穌看著他,就愛他,對他說:「你還缺少一件:去變賣你所有的,分給窮人,就必有財寶在天上;然後來跟從我。」 \end{tabularx} \\ \\ \relax
10:22 & \begin{tabularx}{0.7\textwidth}{X} 他聽見這話,臉就變了色,憂憂愁愁地走了,因為他的產業很多。 \end{tabularx} \\ \\ \relax
10:23 & \begin{tabularx}{0.7\textwidth}{X} 耶穌看了看周圍,對門徒說:「有錢財的人進神的國是何等的難哪!」 \end{tabularx} \\ \\ \relax
10:24 & \begin{tabularx}{0.7\textwidth}{X} 門徒對他的話非常驚奇。耶穌又對他們說:「孩子們,要進神的國是何等的難哪! \end{tabularx} \\ \\ \relax
10:25 & \begin{tabularx}{0.7\textwidth}{X} 駱駝穿過針眼比財主進神的國還容易呢!」 \end{tabularx} \\ \\ \relax
10:26 & \begin{tabularx}{0.7\textwidth}{X} 門徒就更為驚訝,彼此對問:「這樣,誰能得救呢?」 \end{tabularx} \\ \\ \relax
10:27 & \begin{tabularx}{0.7\textwidth}{X} 耶穌看著他們,說:「在人不能,在神卻不然,因為在神凡事都能。」 \end{tabularx} \\ \\ \relax
10:28 & \begin{tabularx}{0.7\textwidth}{X} 彼得就對他說:「看哪,我們已經撇下一切跟從你了。」 \end{tabularx} \\ \\ \relax
10:29 & \begin{tabularx}{0.7\textwidth}{X} 耶穌說:「我實在告訴你們,凡為我和福音撇下房屋,或是兄弟、姊妹、父親、母親、兒女、田地, \end{tabularx} \\ \\ \relax
10:30 & \begin{tabularx}{0.7\textwidth}{X} 沒有不在今世得百倍的,就是房屋、兄弟、姊妹、母親、兒女、田地,並且要受迫害,在來世得永生。 \end{tabularx} \\ \\ \relax
10:31 & \begin{tabularx}{0.7\textwidth}{X} 然而,有許多在前的,將要在後;在後的,將要在前。」 \end{tabularx} \\ \\ \relax
10:32 & \begin{tabularx}{0.7\textwidth}{X} 他們行路上耶路撒冷去。耶穌在前頭走,他們很驚訝,跟從的人也害怕。耶穌又叫十二使徒來,把自己將要遭遇的事告訴他們, \end{tabularx} \\ \\ \relax
10:33 & \begin{tabularx}{0.7\textwidth}{X} 說:「看哪,我們上耶路撒冷去,人子將被交給祭司長和文士;他們要定他死罪,又交給外邦人。 \end{tabularx} \\ \\ \relax
10:34 & \begin{tabularx}{0.7\textwidth}{X} 他們要戲弄他,向他吐唾沫,鞭打他,殺害他;三天後,他要復活。」 \end{tabularx} \\ \\ \relax
10:35 & \begin{tabularx}{0.7\textwidth}{X} 西庇太的兒子雅各和約翰進前來,對耶穌說:「老師,我們無論求你甚麼,願你為我們做。」 \end{tabularx} \\ \\ \relax
10:36 & \begin{tabularx}{0.7\textwidth}{X} 耶穌對他們說:「要我為你們做甚麼?」 \end{tabularx} \\ \\ \relax
10:37 & \begin{tabularx}{0.7\textwidth}{X} 他們對他說:「在你的榮耀裡,請賜我們一個坐在你右邊,一個坐在你左邊。」 \end{tabularx} \\ \\ \relax
10:38 & \begin{tabularx}{0.7\textwidth}{X} 耶穌對他們說:「你們不知道所求的是甚麼。我所喝的杯,你們能喝嗎?我所受的洗,你們能受嗎?」 \end{tabularx} \\ \\ \relax
10:39 & \begin{tabularx}{0.7\textwidth}{X} 他們對他說:「我們能。」耶穌對他們說:「我所喝的杯,你們要喝;我所受的洗,你們也要受。 \end{tabularx} \\ \\ \relax
10:40 & \begin{tabularx}{0.7\textwidth}{X} 可是坐在我的左右,不是我可以賜的,而是為誰預備就賜給誰。」 \end{tabularx} \\ \\ \relax
10:41 & \begin{tabularx}{0.7\textwidth}{X} 其餘十個門徒聽見,就對雅各和約翰很生氣。 \end{tabularx} \\ \\ \relax
10:42 & \begin{tabularx}{0.7\textwidth}{X} 耶穌叫了他們來,對他們說:「你們知道,外邦人有君王作主治理他們,有大臣操權管轄他們。 \end{tabularx} \\ \\ \relax
10:43 & \begin{tabularx}{0.7\textwidth}{X} 但是在你們中間,不可這樣。你們中間誰願為大,就要作你們的用人; \end{tabularx} \\ \\ \relax
10:44 & \begin{tabularx}{0.7\textwidth}{X} 在你們中間誰願為首,就要作眾人的僕人。 \end{tabularx} \\ \\ \relax
10:45 & \begin{tabularx}{0.7\textwidth}{X} 因為人子來,並不是要受人的服事,乃是要服事人,並且要捨命作多人的贖價。」 \end{tabularx} \\ \\ \relax
10:46 & \begin{tabularx}{0.7\textwidth}{X} 他們到了耶利哥。耶穌同門徒並許多人離開耶利哥的時候,有一個討飯的盲人,是底買的兒子巴底買,坐在路旁。 \end{tabularx} \\ \\ \relax
10:47 & \begin{tabularx}{0.7\textwidth}{X} 他聽見是拿撒勒的耶穌,就喊了起來,說:「大衛之子耶穌啊,可憐我吧!」 \end{tabularx} \\ \\ \relax
10:48 & \begin{tabularx}{0.7\textwidth}{X} 有許多人責備他,不許他作聲,他卻越發喊著:「大衛之子啊,可憐我吧!」 \end{tabularx} \\ \\ \relax
10:49 & \begin{tabularx}{0.7\textwidth}{X} 耶穌就站住,說:「叫他過來。」他們就叫那盲人,對他說:「放心,起來!他在叫你啦。」 \end{tabularx} \\ \\ \relax
10:50 & \begin{tabularx}{0.7\textwidth}{X} 盲人就丟下衣服,跳起來,走到耶穌那裡。 \end{tabularx} \\ \\ \relax
10:51 & \begin{tabularx}{0.7\textwidth}{X} 耶穌回答他說:「你要我為你做甚麼?」盲人對他說:「拉波尼,我要能看見。」 \end{tabularx} \\ \\ \relax
10:52 & \begin{tabularx}{0.7\textwidth}{X} 耶穌對他說:「你去吧!你的信救了你。」盲人立刻看得見,就在路上跟隨耶穌。 \end{tabularx} \\ \\
[1ex]
\hline
\hline
\end{longtable}
$^{1}$在擁抱這個月題當中.
其實我很快就想到第一講的經文內容.
因為圖畫在關於擁抱這個訊息的時候.
已經在我腦海中出現了.
整件事就讓弟兄姊妹去感受一下.
我就想透過馬學福音的記載跟大家去表達.
這段經文是馬學福音第十章的內容.
這段經文大家會看到.
大家很熟悉的 我們一起讀.
第十章第十三節.
有人帶著小孩子來見耶穌.
要耶穌摸他們.
門徒便責備那些人.
耶穌看見就惱怒對門徒說.
讓小孩子到我這裡來.
不要禁止他們.
因為在上帝國的正是這樣的人.
我實在告訴你們.
凡要承受上帝國的.
若不像小孩子.
斷不能進去.
於是抱著小孩子.
給他們按手.
為他們祝福.
當擁抱這個訊息敲定之後.
這個圖畫就出現了.
接著就想想.
選哪一卷方書去講呢.
因為除了比較後期的約翰福音.
沒有記載之外.
其他馬太 馬可 盧迦.
都有記載這段經文.
三段經文當中.
我選哪一段呢.
我選了馬可.
或者可以和大家平衡一下.
這段經文你會看到.
三個不同的福音書記載.
都有少少不同.
剛才讀了就是馬可福音第十章內容.

$^{41}$馬太福音內容都是大同小異.
你會看到.
有人帶著小孩子來到耶穌面前.
耶穌為他們按手禱告.
然後就擲避門徒.
講了耶穌要講的話.
然後耶穌和他們按手後.
他們就離開了.
盧迦福音都有這個記載.
這個記載內容有一個訊息.
就是其中一件事就是.
盧迦福音講得出.
除了小孩子外.
他帶了一些手抱的嬰孩.
對於你來說會是什麼一回事呢.
小孩是一個類別.
嬰孩是另外一個類別.
其實有不同的人帶著不同的人去到上帝面前.
而去到耶穌面前的時候.
就要想得到耶穌給他的祝福.
馬可,馬太,盧迦都記載了這件事.
而我選擇了馬可.
除了是一個比較順理成章.
完全記載了之後.
其實我選擇馬可都覺得.
經文裡面有一個脈絡.
特別是馬可福音記載的過程當中.
讓我們看到.
耶穌想幫門徒學什麼.
在接待人的過程當中.
又或者進入上帝國的人當中.
耶穌想我們學什麼.
所以我就和大家去走訪一下.
其實馬可福音想要表達的內容是什麼呢.
在馬可福音裡面.
第八至第十章.
其實是一連串.
馬可緝錄了耶穌關於他預言要受害的過程.
而在過程當中又要門徒去明白到.
他來的目的是什麼.

$^{81}$在馬可福音第八章裡面.
他帶了一個訊息就是.
人們問人說我是誰.
他就和門徒說你說我是誰.
於是乎耶穌就說了他是誰.
而彼得就是尼塞亞先知.
在過程當中.
最後耶穌要帶出第一個訊息就是.
他將會被害受死埋葬第三天復活.
這是他第一次預言的訊息.
到了第九章的開頭的時候.
他就帶了第二個訊息就是.
在登山的過程當中.
彼得亞各月漢看到耶穌的真相的時候.
他就覺得很棒.
這次沒有跟錯師父了.
因為師父偉大到連舊約的先知代表.
以利亞和摩西.
都是和他並列.
但是耶穌也在和他們說一個訊息.
就是這個不是你們想看的最重要的訊息.
但是到了山上.
有些事情發生了.
一會兒就會再說.
但是在第九章中段的時候.
耶穌也提到一樣東西.
就好像這裡看到第九章第三十節.
就是他離開那地方經過加利利.
耶穌不願意讓人知道.
於是再教訓就是.
其實不是你們想看的.
我們的目的是這個才是主調.
人子將要被交在人手裡.
他們要殺害他.
被殺之後過三天要復活.
那些門徒又不太聽得懂.
他們不明白什麼呢.
不是吧 你這麼厲害.
你在山上我們真的見過你真身.
這次沒有跟錯師父.

$^{121}$你這麼厲害 你告訴我們你將會死.
不是吧 不要開玩笑.
對他們來說.
未見過就不可以說.
見過就不應該這樣說.
而事實上那些門徒真的不明白.
所以又不敢問.
馬可要記載.
其實耶穌有心意去鋪路.
那些門徒真的不太知道.
不奇怪 因為跟的日子.
都是慢慢堆積到對耶穌的認識.
所以耶穌第二次說完.
他預言的東西.
那些門徒是不明白.
耶穌為什麼要這樣做.
耶穌就用了幾個場景.
或者馬可福音記載了幾個場景.
讓人明白到.
耶穌為什麼要這樣做.
下一段經文會跟大家說.
第一個耶穌要教的.
就是讓那些門徒去理解.
你看的東西和你知道的東西.
還有一段距離.
意思是什麼呢.
在第一段的經文.
剛才我回頭.
他們不明白又不敢問.
到了第三節的時候.
他們來到加伯隆的時候.
耶穌就在家裡問門徒.
你們在路上議論的是什麼.
其實發生了什麼事呢.
在第八章就問了我是誰.
你是基督 是永生神的兒子.
可以讓人說的.
耶穌就說你說得對.
於是就說我真正的目的.
就是要受死埋葬第三天復活.

$^{161}$這是第一次說預言.
在登山的時候.
登山變成了一個這麼大的場景.
而摩西和以利亞都和耶穌站台的時候.
這個真的是不得了.
於是下山的時候.
他們應該要遲遲去浸.
他們下山遲遲去浸.
那三個可能.
我推論將來有機會見到的時候.
他們會問一下.
不過重點就是.
我們會問他一件事.
就是剛才不得了.
我們在山上沒有辦法.
我們三個去.
我們三個見到什麼呢.
我們沒有跟錯老闆.
他厲害到原來摩西和以利亞都和他站邊.
還有聲音和我們說話.
然後就說怎樣怎樣.
怎樣排位.
於是就大家在議論.
當中其實見到的東西.
對於他們來說有多重要.
所以耶穌聽到他們遲遲浸的時候.
你們在路上議論的是什麼.
於是當然不出聲.
門徒不作聲.
因為他們在爭論.
如果排位彼得亞和約翰.
三個叫做.
不要排位.
三個一起並列.
其他一定要排第四.
所以在過程當中.
一定是在爭誰大.
於是怎麼樣.
三十五節耶穌坐下來的時候.
就跟那十二個門徒說.

$^{201}$有誰願意作首的.
作眾人的門後.
作眾人的傭人.
接著你會見到.
另一個小孩子站在他們中間.
抱起他對他們說.
OK.
凡為我名接待一個像小孩的.
就是接待我.
凡接待我的不是接待我.
乃是接待那差我來的.
耶穌做了一個很重要的表徵或表示.
讓這個小朋友來到的問題.
來到他當中.
讓人知道.
其實重點就是.
好像一個小朋友.
接待一個小朋友.
如果你接待我.
就好像你接待我.
你接待的不是我.
而是接待那差我來.
其實意思是說什麼呢.
其實就是說.
當時你過去可能聽到.
都會聽過解說還是經文.
小孩子你重溫一下.
當時來說是一個沒有權勢的人.
是一個受助的人.
是一個沒有身份的人.
但是對於.
耶穌來說.
其實你長大是為了什麼呢.
或者你排位是為了什麼呢.
又或者你見到一些很有能力的東西.
其實你最後想要什麼呢.
摸口飛.
在猶太人當時來說.
是受壓迫.
受克制.

$^{241}$受規管.
他們真的想用權力去做到.
達到他們的目的.
但是耶穌想讓他們.
去明白一件事.
不是要用這個方式.
反而就是用第二個方式.
去讓人去了解.
其實真正要接待上帝的.
是靠什麼.
我就帶了一個小朋友.
去到面前告訴你.
你怎樣接待一個小朋友.
你會見到在.
方書裡面.
真的 方書裡面都見到.
是有精的.
過程當中小朋友和婦人都不計算.
基本上在那個環境當中.
就不看他在眼.
但是上帝 耶穌也希望你明白到.
你看不在眼的東西.
上帝是看在眼的.
你覺得不需要看的東西.
上帝是看在眼的.
對於耶穌.
要表達更加重要的立場.
就是.
你願不願意接待這班人.
如果你願意接待這班人的時候.
不知不覺.
其實你接待了上帝.
這個訊息.
或者這個思路.
其實在馬太福音第25章.
主人回來的時候.
當說.
他接待了他的人.
那個人呆呆的.
我何時接待你.

$^{281}$你在他渴的時候給他水喝.
你在他餓的時候給他東西吃.
你在他冷的時候給他衣服穿.
在他坐牢的時候你探他坐牢.
不知不覺就做到.
上帝要接待人的人.
當那班門生.
是很著緊.
要說權利.
要說牌位.
要說可以掌控的過程當中.
耶穌和他們做了一個反題.
不如試試用別的方式看看.
而這個方式.
是希望他們明白到.
什麼是接待小朋友.
這個是第一個耶穌之後的教導.
馬太福音結束了第一個教論.
接著再下去仍然是第九章.
在經文的時候有第二個教導.
第二個教導是第38節.
約翰對耶穌說.
夫子我們看見一個人奉你的名講鬼.
我們就禁止他.
因為他不跟從我們.
耶穌就說.
不如禁止他.
反倒輕易誹謗我.
不敵擋我們的.
就是幫助我們的.
凡因你們是屬基督.
給你們一杯水喝.
我實在告訴你們.
他不能不得賞賜.
在經文中說明一個很明顯的訊息.
其實不是敵擋我們.
我們做的工作就是幫助我們.
重點就是.
不一定要是出名門的.
最主要知道.

$^{321}$做的是什麼是最重要.
不是說跟著名校出身.
那樣才是對的.
其他就一定不是那麼對.
不是.
其實做那樣是對是錯.
就好像一杯涼水.
有需要你能夠去.
那個叫恩慈.
有需要的時候.
你伸出援手.
這個是恩慈.
恩慈就是屬靈果之第五個氣質.
kindness.
最著重的是有需要的時候.
你能不能夠去參與.
就是有需要的時候.
那個人是真的無助.
又或者有缺陷.
又或者有些那一刻他真的沒有能力的時候.
你怎樣去能夠與他同行.
去參與其中.
這個就是涼水.
就是kindness 就是恩慈.
不是說一定要出名門.
不是說是哪一個.
是差派你做.
重點就是你知道那件事是上帝喜悅的.
就是在幫助上帝.
福音沒有專利.
福音不是說什麼叫名門正宗.
一定是耶穌基督教導我們.
特別是聖經在記載我們.
我們是不是在做聖經的教導.
不是什麼字頭.
也不是什麼宗派.
福音沒有專利.
那個是在幫助上帝的工作.
接著再下去的時候.
第三個教導是什麼呢.

$^{361}$又是馬可福音繼續下去.
第九章的教導就是.
比較長的.
你會看到你一讀.
你就回來了記憶.
你會看到凡使者信我的一個小節跌倒.
就是之前的連繫.
倒不如把大磨石山在丹利的人頸上.
泵在海裡.
倘若你一隻手叫你跌倒.
就怕它撞下來.
你缺了肢體進入永生.
強如兩隻手落在地獄.
入那不滅的火裡.
接著你看到一個相連排比的.
手又說了.
腳又說了.
眼又說了.
其實說到最後就是.
你拿著這麼多東西.
但你沒有做應該要做的事.
其實是沒有意思的.
你反而沒有了手手腳腳眼.
但你做到應該要做的事.
那個才是承受永生.
耶穌再一次讓門徒明白到一件事.
不是自恃你有什麼.
而忽略了需要做的事.
而不是你自恃擁有一些你覺得重要的事.
而忽略了或看不到.
甚至沒有參與在當中要接待的事情.
我已經.
怎麼說呢.
我兒子經常叫我染髮.
他覺得我染髮比較年輕.
我說我們爸爸也不是年輕.
你也不是很老.
我也不是很老.
因為我家裡放了他們小時候和他們拍的照片.
我那時候沒有那麼多白頭髮.

$^{401}$他說他叫我染髮的時候.
我說染了髮又怎樣.
他說染了髮又想要這個樣子.
其實這個樣子很簡單.
我遮了兩邊就沒有那麼白了.
我說是喔.
他覺得老是一件.
他不是很習慣老.
我說那也是.
老了職做了很多事.
有智慧.
或者你有很多財寶.
或者你有很多錢可以做很多事.
其實聊著聊著.
成年人真的.
你年紀大了會積聚很多東西.
積存很多東西.
包括你的知識.
你的能力.
你的人脈關係.
你的錢財.
你涉及的範圍.
其實你積聚越多的時候.
你越不願意放棄.
你不願意放棄之餘.
你更加會覺得.
你可以控制很多.
甚至你不願意有任何損失.
但是這段經文要帶出一個訊息.
如果那樣東西會令你半島.
如果那樣東西令到你不能夠.
做到上帝要你做的要求的時候.
那樣東西你就不要了.
那樣東西會令到你最後得不償失.
你會看到其實耶穌都想做一個一致性教導.
你們要爭位.
你們要排位.
你們要爭權的時候.
被你們拿到了.
但是最後你失去了.

$^{441}$進入永生.
有什麼用呢.
這個仍然是馬克福音的時候第三個教導.
去到馬克福音第四個教導會是什麼呢.
去到第四個教導就去到第十章.
又是一個比較長.
又是你們很熟悉的經文.
第十章的經文就是在說.
休妻.
那班快篩人就在挑選耶穌的底線.
或者對於律法上的那種看法.
因為這件事對於耶穌來說.
常常耶穌最大的困難就是.
人前人後都很多人監視他.
就是監視他.
因為要得著告他的把柄.
又或者正面衝突的時候.
就拿一些事情去刁難耶穌.
這個其中一個場景就是.
問到耶穌關於律法上取妻和休妻這個安排.
上段經文我不會讀.
不過大家會理解就是.
面對一個情況就是.
耶穌要重述當中婚姻這個盟約.
其實預表了上帝和人之間的契.
那種契就是神與人之間可以相和.
聯合那種信息.
但是因為人心硬.
他想背約或者想毀約的時候.
但是不是上帝想的.
不過既然你心硬.
上帝就容讓那件事情去發生.
其實一個婚姻怎麼會去到這個位置.
要離開要解約.
其實最後一定有很多原因.
特別用我們現在這個現代的觀感來看.
都會有很多原因.
但是總有前因是放不下自己.
還是當初沒有表達好自己.
或者在雙方的過程當中溝通.

$^{481}$是不是真的共識不到.
以至最後都要決裂.
我明天都會去講訓緬.
我神學院的學生明天就結婚了.
我是他關心小組的老師.
我明天就去講訓緬.
在過程當中其實都有很多.
應該這樣說.
每一次教婚前輔導.
或者做這些教育課程的時候.
其實很多經文或者信息.
是不斷地重提自己.
當我們去到婚禮當中.
聽到講誓詞的時候.
其實誓詞裡面講了很長的內容.
就是不同什麼處境.
什麼很實務上面面對的情況.
就是讓我們有一個先驗.
也是有一個共識.
不是說婚姻之後沒有問題.
但是總會有很多問題.
就是挑戰我們對婚姻這個文藥的底線.
但是你能不能忠貞自首呢.
這個就是我們要放下自己.
去守住這個承諾.
同樣都是跟隨上帝.
或者我們身為基督徒.
我們的信仰總會在時間當中.
有很多挑戰的地方.
也會有很多令我們覺得很難守.
但是我們看自己多一點.
看上帝的原則多一點.
我們看我們自己緊張的安全.
還是我們看我們身邊.
可以做到的事情.
這個也是一個不斷地不反問的地方.
所以耶穌就是說.
文藥對於以色列人來說是很重要.
但是上帝很看重對人的那種藥.
如果你看到創世紀.

$^{521}$要看到啟示錄的時候.
那個信仰語言就再出現.
就是我是祂的上帝.
祂是我的子民.
其實終極就是人要回到上帝的身邊.
再一次履行上帝原初創造這個世界.
和人之間要立這個藥.
但是人很多時候就沒有看重.
或者是沒有認真去看待.
以致他心眼就違背了這個藥.
去到第五個教導.
就是今天大家一起讀出的那段經文.
這段經文就是耶穌為小朋友祝福的經文.
在這段經文的時候你會看到.
其實由第二次耶穌預言他將會受害的時候.
那幫門徒不是很明白.
但是耶穌更加要指出.
不重要 但不要錯重點.
不明白可以再教.
但錯重點就是你的心很容易就會跟錯了.
跟了你自己的心而走.
你的心意而走.
所以用了五個教導.
去讓那幫門徒去明白.
而這五個教導是刻意馬可夫音編輯出來.
和那幫門徒或者和我們去分享.
馬可夫音被稱為上帝僕人的書卷.
這個上帝的僕人耶穌.
他要我們成為他們的僕人.
或者我們成為上帝僕人的時候.
我們要學習的歷程是什麼呢.
去到第五點的時候.
就和大家去看看.
其實相關有什麼內容.
我覺得今天關於擁抱這個課程.
或者這個課題可以多點思考.
第一個就是小朋友.
你會看到這個小朋友.
去到第五個教導的時候.
又再出現了.

$^{561}$第一個教導就是隨願為首的教導.
當問他你們的爭論是什麼.
誰最偉大的時候.
那個小朋友不知在哪裡出現.
突然間有個小朋友出現了.
然後就拿來做活動教材.
就教他們.
但這次這個小朋友又出現了.
當然你不要覺得是同一個小朋友.
不是那樣東西.
但你會覺得又會見到這個小朋友.
在出現帶著小朋友來到耶穌面前的時候.
其實耶穌是重提一個教導.
就是怎樣去接待.
面對需要承受上帝國特質的人的時候.
就是重提怎樣去接待這件事情.
所以有人帶著小朋友來見耶穌的時候.
要耶穌摸他.
滿肚子就擲避那些人.
擲避這個字就可圈可點.
因為擲避這個字.
這個字不只是馬克福音用這個.
連馬太和路加都用同一個原文的字.
這個字是在說什麼呢.
通常在科幻書裡面擲避.
都是在說一些比較關於罪.
關於很大件事就用這個字.
這個字如果用在耶穌的時候.
就直接用在一些說罪大的不是.
剛才說到第八章.
人說我是誰.
在該撒尼亞肥勒比的事件當中.
就是耶穌說完他自己將會遇害的時候.
彼得說夫子不好的時候.
耶穌就擲避彼得說.
撒旦退我後邊去吧.
就是這個語氣.
所以擲避那些人.
其實那些門徒是很義正詞嚴.
或者很誇張的.

$^{601}$就趕了他們.
其實這個不是很輕描淡寫.
你們不要覺得耶穌要休息.
應該不是這麼簡單.
只是分門別類.
只是排他.
只是爛了.
這件事其實應該都很嚴重.
所以那些看似無權無勢.
或者有需要的人.
去到耶穌面前的時候.
那些門徒就攔了他們.
擲避他們.
所以接著為什麼.
耶穌就生氣.
你們為什麼這麼大火氣.
為什麼要這樣做.
所以耶穌生什麼氣呢.
就是他生氣的一件事就是.
其實他們知不知道他們在做什麼.
我不知你.
應該這麼說.
我是不知道的.
不過我猜你應該生氣過人.
其實你知不知道生氣過人什麼呢.
或者你生氣的時候.
你能不能幫他review.
其實你知道我生氣過你什麼嗎.
在過程當中.
無論我自己教導也好.
教書也好.
或者什麼也好.
有時候都會和他去review.
那件事發生的成因和結果.
下次不要再重犯.
或者重蹈覆轍.
耶穌生氣什麼呢.
其實耶穌生氣.
從第八章的記載.
就是耶穌生氣的幾件事.

$^{641}$就是頭四個教導裡面.
首先第一.
你忘記要坐在最小的身上嗎.
為什麼你要擲避他們.
你忘記.
其實福音不是你們專利的嗎.
其實不只是你們才可以覺得那件事是對的.
其實那件事不是敵黨基督的話.
就要繼續去做.
就讓人去做.
餅這麼大不會做得完的.
就是大家按不同的能力繼續做下去.
第三件事要說的就是.
如果那件事是難阻到你接觸其他人.
或者難阻你承受.
難阻人去承受上帝國.
或者你自己承受上帝國的時候.
你就要剔除它.
不要被它影響.
但是會不會就是.
有些事你覺得已經是很好了.
你不想做了.
有些事你覺得要安全一點.
你不敢做了.
那件事反而令到你更加顧報子封.
最後就是.
你有多盡力去守住這個.
人可以和上帝相遇的機會.
又或者怎樣去讓人可以回到上帝當中的約呢.
耶穌生氣的原因就是.
為什麼你們這麼動氣.
為什麼你們要做這個工作.
耶穌生氣的原因就是.
其實教過你的.
老奴在說.
之前那幾個教導.
一路跟 一路教.
所以去到.
耶穌要說話了.
耶穌看見老奴對門徒說.

$^{681}$讓小孩子到我這裡來.
不要禁止他們.
因為上帝國的正是這樣的人.
不要禁止這一班人.
不要禁止這一班.
你們看似覺得無權.
沒有能力.
沒有慈憤的人.
他們想來上帝.
想來我面前.
就讓他們來.
不要禁止.
這個之前已經教了你們.
現在你們有沒有學到東西.
你的反應就是告訴我.
你沒有學到東西.
但讓你去明白到.
就是你要讓他們來.
因為他們才是真正要來上帝面前的人.
所以不要禁止他們.
因為上帝國正是需要這些人.
這個就令你聯想到.
其實看平行福音的時候都會見到.
有病的人用得著醫生.
沒有病的人用不得著.
我來不是召義人悔改.
乃是召罪人悔改.
耶穌同樣教導.
從來做人要有個一致.
不要你過了就不讓人過才行.
不要你入了就不讓人入才行.
耶穌要帶出一個不要禁止他們.
他們是屬於這班人的.
所以耶穌生氣什麼呢.
生氣的就是他很認真.
所以你看到福音書很多時候.
我實在告訴你們.
都是耶穌一些很重要的宣告.
就是凡要承受上帝國的若不像小孩子.
斷不能進去.

$^{721}$再一次告訴你們.
能夠遵守上帝國的人.
承受上帝國的人.
其實就應該像小孩一樣.
單純 認真 信任.
在過程當中學習倚靠.
我不知道你們怎麼看小朋友.
可能你周遭的小朋友.
已經很懂得人情世故.
或者很懂得游走在不同人的取態.
我覺得這些是以緣莫染.
是社會氛圍影響.
但我自己看著兩個兒子成長的時候.
我都感受到一個很重要的學習.
特別我自己在街上抱他.
如果說擁抱.
上次John就說他的女兒.
其實我當然有抱小孩.
我抱小孩的時候.
都遇到上次John說的情況.
特別是男孩子還被人說得快.
這麼大個孩子還要人抱.
但通常我都會在他面前走久一點.
特別去到門口.
進屋苑樓下的鐵閘的時候.
那個保安都會說.
「你還要爸爸抱啊」.
然後我說「是啊 他很累啊」.
就抱他.
因為男孩子你抱多久呢.
他會頂的日子都不多.
有些朋友很久沒見我大兒子.
我大兒子跟我一樣高.
所以我怎麼抱他.
我比他重.
重點不是說.
你抱不抱他的問題.
而是那個關係.
而在我看著兩個兒子成長.
我抱他們在街上的時候.

$^{761}$我感受到一樣東西.
你們可能都不難觀察到.
就是很多小朋友.
被爸爸媽媽或者家人抱著的時候.
他睡到.
你看到那個人快要跌倒.
你想幫他的時候.
那些小朋友都不醒的.
你發不發覺.
就算快要跌倒.
我們知道他跌倒.
但小朋友不會知道跌倒.
但父母都知道很危險.
但小朋友還沒睡到那麼熟.
其中一個原因是什麼.
就是他知道抱著的是誰.
他知道那個人總不會放手.
他知道他在他懷中是永遠安全.
這個就是信任.
一個你永遠信任的人.
他抱你了.
你有什麼害怕呢.
但如果我們要刻意將.
有機會被抱的人排除的話.
耶穌是會責備我們的.
因為之前教你的東西.
你沒有學過.
應該要做在少的事情你不做.
你反而覺得那件事你分類了.
第三件事就是.
在過程當中.
你擁有很多你以為能帶下去嗎.
但真正能做到上帝的事.
才能保存到永遠.
最後能夠守住上帝的永約.
才是最真實.
不是看自己心眼.
或是要做的決定.
承受上帝的國的人.
一定是像小朋友.

$^{801}$對上帝有絕對的倚靠.
虛心 信任.
這是耶穌很義正詞嚴地.
我實在告訴你們.
所以在馬可福音最終極的描述.
就是剛才兩卷.
馬太和盧迦都比較輕描淡寫.
但馬可福音說.
承受天國的人.
是得著上帝的擁抱.
是得著上帝的按手 祝福.
其實教會是承受上帝國的人.
教會是承受上帝國的人.
教會就應該要去擁抱一班.
看似沒能力.
看似沒資格.
而在過程當中有很多限制.
教會是否應該要承擔.
擁抱這個群體呢.
第二件事就是.
在過程當中.
一起去親近上帝.
在聚會當中.
上帝會祝福.
在過程當中.
又能夠感受到上帝那種親密.
教會存在.
是否應該這樣呢.
今天我們應該要思考.
其實教會是甚麼.
要再次思考.
坊間總會有很多不同的挑戰.
坊間總會有很多不同的規範掣肘.
但我們仍然要告訴自己.
我們是一班進入教會的人.
但我們更是一班要承受教會使命的人.
這是要一起去反省的.
為何我會選擇馬可.福音呢.
除了連續的教導.
帶出第五個小孩.

$^{841}$要受祝福.
人帶他來這個果盒的時候.
馬可.福音其實還有一個伏線.
你以為這樣就完結了嗎.
馬可.福音的記載還有一個伏線.
第十章第十七節的經文是甚麼呢.
又是大家很熟悉的經文.
就是少年的官.
盧家福音就將少年的官的身份說出來.
良善的夫子我要做些甚麼才能永生.
然後耶穌就說.
你為何叫我除了上帝之外沒有良善.
就叫他守戒命.
十戒裡面最主要的五戒.
他說這些我從小都守了.
但耶穌就跟他禮真說.
你就要變賣你的所有.
然後要跟從我.
要分得窮人.
你見到嗎.
但那個人聽了之後就悠悠手手走了.
因為他產業很多.
你會見到一個從小就進了門檻的人.
他自始有很多.
而他覺得要守的東西.
他覺得要守了.
但有些東西他帶不走.
就是錢.
你願不願意將錢分給窮人.
再轉變你的身份.
因為你見到其中一個分號就是.
你還要跟從我.
前面是分錢後面是跟從.
跟從耶穌是說你的身份要轉變.
但我不是說笑.
你跟耶穌.
你經常周遊四方.
你知道我做官的.
我坐訪的.
你叫我.

$^{881}$你經常跟罪人.
妓女一起.
你知道我的身份嗎.
我做官的.
你今晚去哪.
今晚睡街.
我不OK.
正正就好像我之前的教導.
你擁有得越多.
你是不是這麼容易跟從上帝.
所以經文下半段就在說.
你會見到一個很重要的訊息.
就是上帝國的人.
不是靠你依靠什麼進去.
反而是靠你捨棄什麼.
更加是你在需要的時候.
你會不會接待什麼.
這個是很重要的.
我也聽過.
有些人說上帝不喜歡錢.
不是.
萬物都是從主而來.
沒有東西不是上帝的.
我們有賺取金錢的能力.
我們有奉獻的能力.
都是上帝給我們的恩典.
當然我們有勞力.
是的.
沒有不勞而獲.
但重點就是.
當你越來越多東西的時候.
你會不會越來越自恃.
如果越來越自恃的時候.
就變成第三個教導.
你強顏有手有眼有能力.
但你最後落永生不滅的火.
這個都不是承受上帝國的人.
如果教會是這樣的話.
這個教會就是秘魯去巴比倫的教會.
就是最後走到滅亡的教會.

$^{921}$七弟姐妹.
我希望你明白到.
其實教會終極的目的是什麼呢.
教會終極的目的.
是讓人去到耶穌面前.
耶穌是將來要來的那位真正的救主.
耶穌會再來的.
而我們是教會的一員.
我們是在現在地上.
可以幫人認識耶穌的一個很重要的群體.
我們讓那班人能夠去到上帝面前進入會堂.
無阻隔地去敬拜上帝.
其實是讓他們可以最後得到一個終點的擁抱.
沒有人知道這條路要走多遠.
但是我們帶他們去到耶穌面前.
或者讓人去到上帝面前.
他終有一天都會去到終點和耶穌擁抱.
敬拜隊問我會不會唱Dear Jane的終點的擁抱.
當然我們這隊樂隊是一定可以的.
不過歌詞上不是很吻合今天的訊息.
有機會應該會和大家唱.
可能在其他地方.
但是那首歌裡面也有一個意象.
就是這條路要走多遠.
不知道.
但是這條路仍然很困難.
但是大家都朝著終點.
因為最後我們都會擁抱.
我很希望我們作為教會.
我們作為地上去帶人去到耶穌面前.
就讓我們無阻隔.
沒有分門類別地帶人去到耶穌面前.
得到一個終點的擁抱.
願意上帝繼續對我們說話.
願意我們都是秉恆帶人回教會.
這個最終目的.
就是下一次祈禱.
天上上帝每當我們打開你的話的時候.
昔日其實你不斷地去教導門徒.
但是門徒所作所示.

$^{961}$都是一個和我們不遑多讓.
還是一個需要學習.
需要更新.
需要上帝你真的責備我們的地方.
因為很多時候我們都仍然用了.
自以為是 自恃.
甚至乎看不到.
不認識上帝你真正的心意.
求主你繼續對我們說話.
讓我們真的好像一個小孩子一樣.
用一個單純倚靠的心.
用一個完全信任躺開的心.
去接受我們身邊的人.
也是教會更加成為一個出口.
讓更加多人在沒有任何的掣肘之下.
可以去到上帝面前.
是有不同的難處.
是有不同的挑戰.
但是求主你給教會有智慧.
在現在疫情經過了一個很長的時間.
人心疲憊更加是我們福音的契機.
願主你繼續幫助我們.
總會有很多挑戰.
總會有很多不同大家還沒拍齊的地方.
大家有不同的意見.
但是我們仍然知道.
「作在小子身上就是作在小子身上」.
求主你加添我們心力.
奉耶穌的名求.
阿門.
(影片完結).
\newpage



\section{彼得前書 4:7-11-20220521}
\label{sec:2t83SY_sddQ}
\textbf{【網上聖餐崇拜】擁抱的規矩…MM7|彼得前書4\_7-11|20220521 [2t83SY\_sddQ]}
\newline
\newline
連結: \href{https://youtube.com/watch?v=2t83SY_sddQ}{\texttt{ https://youtube.com/watch?v=2t83SY\_sddQ}} ~~~~ 語音日期: 2022-05-21 
\newline
\newline
\hyperref[sec:97SC38c6sqY]{\small{< < < PREV SERMON < < <}}
~
\hyperref[sec:index_chronic]{\small{[返順時目]}}
~
\hyperref[sec:index_scriptual]{\small{[返順卷目]}}
~
\hyperref[sec:OJx_AQpZJpY]{\small{> > > NEXT SERMON > > >}}
\newline
\newline
彼得前書 4:7-11-20220521
\newline
\begin{longtable}{cl}
\hline
\hline
章節 & 經文 (和合本修訂版)\\
\hline
4:7 & \begin{tabularx}{0.7\textwidth}{X} 萬物的結局近了。所以你們要謹慎自守,要警醒禱告。 \end{tabularx} \\ \\ \relax
4:8 & \begin{tabularx}{0.7\textwidth}{X} 最要緊的是彼此切實相愛,因為愛能遮掩許多的罪。 \end{tabularx} \\ \\ \relax
4:9 & \begin{tabularx}{0.7\textwidth}{X} 你們要互相款待,不發怨言。 \end{tabularx} \\ \\ \relax
4:10 & \begin{tabularx}{0.7\textwidth}{X} 人人要照自己所得的恩賜彼此服事,作神各種恩賜的好管家。 \end{tabularx} \\ \\ \relax
4:11 & \begin{tabularx}{0.7\textwidth}{X} 若有人講道,他要按著神的聖言講;若有人服事,他要按著神所賜的力量服事,好讓神在凡事上因耶穌基督得榮耀。願榮耀和權能都歸給他,直到永永遠遠。阿們! \end{tabularx} \\ \\ \relax
4:12 & \begin{tabularx}{0.7\textwidth}{X} 親愛的,有火一般的考驗臨到你們,不要奇怪,似乎是遭遇非常的事; \end{tabularx} \\ \\ \relax
4:13 & \begin{tabularx}{0.7\textwidth}{X} 倒要歡喜,因為你們是與基督一同受苦,使你們在他榮耀顯現的時候也可以歡喜快樂。 \end{tabularx} \\ \\ \relax
4:14 & \begin{tabularx}{0.7\textwidth}{X} 你們若為基督的名受辱罵是有福的,因為榮耀的靈,就是神的靈,在你們身上。 \end{tabularx} \\ \\ \relax
4:15 & \begin{tabularx}{0.7\textwidth}{X} 你們中間,不可有人因為殺人、偷竊、作惡、好管閒事而受苦。 \end{tabularx} \\ \\ \relax
4:16 & \begin{tabularx}{0.7\textwidth}{X} 若有人因是基督徒而受苦,不要引以為恥,倒要因這名而歸榮耀給神。 \end{tabularx} \\ \\ \relax
4:17 & \begin{tabularx}{0.7\textwidth}{X} 因為時候到了,審判要從神的家開始;若是先從我們開始,那麼,不信從神福音的人將有何等的結局呢? \end{tabularx} \\ \\ \relax
4:18 & \begin{tabularx}{0.7\textwidth}{X} 「若是義人還僅僅得救,不虔敬和犯罪的人將有何地可站呢?」 \end{tabularx} \\ \\ \relax
4:19 & \begin{tabularx}{0.7\textwidth}{X} 所以,照神旨意受苦的人要一心為善,將自己的靈魂交給那信實的造物主。 \end{tabularx} \\ \\
[1ex]
\hline
\hline
\end{longtable}
$^{1}$在網上的弟兄姊妹有些抱歉,剛才我沒有站好,沒有圍到網絡.
在崇拜前,在網上的弟兄姊妹有些抱歉,希望你今天都平安,能夠一起在上帝面前敬拜.
在這裡要說一個好笑的故事,昨天吃了一頓飯,有人找我吃飯.
其實是怎麼找的呢?是Millie,我們的行政,FortuneHunter行政大管家.
無端端把一些訊息傳給我,因為那是我二十多年前讀神學院的時候的同學.
因為我們讀不同的程式,所以我們不是很熟悉,基本上我相信已經有二十多年沒有見面了.
無端端有三個二十多年前一起讀過神學的弟兄姊妹見面.
我問見面要做什麼呢?其實我很緊張,我還問我們要準備什麼,要談什麼.
他說在這個世代裡,很多人都在平行時空地做很多平行時空的事.
我想一坐下來的時候,問明來意的時候,我想把這番話說給FoldChurch的弟兄姊妹聽.
可能代價會越來越大,可能會面對前面的路,不知道會怎樣走下去更多.
但起碼在這個時候,為自己能夠再堅持多一點下去.
這個心志,我希望FoldChurch的弟兄姊妹,我們一同去給一些.
但起碼給自己,給你們當中的每一位.
正如在網上留言的時候,有些人,無論你打針還是不打針.
我知道有很多不同的考量,但我仍然想跟弟兄姊妹說.
為你自己,可以再堅持多一點,再掙扎多一點.
有這個意念和想法,其實都值得給自己一杯Hagen-Dazs.
你沒有錢的話,不夠錢的話,你可以找John和潘Sir.
我相信他們很樂意請你吃一杯Hagen-Dazs.
真的,弟兄姊妹,他們三位沒有見過20年的同學.
其實不是只有他們,他說有很多人仍然默默地欣賞著.
在這個時候,仍然可以有掙扎和堅持的信仰群體.
其實我開場白是想說,愛華頓終於呼吸成功.
剛才有位目者跟我說,雖然他是利物浦的球迷.
但他也想愛華頓降班,但為了家Sir的緣故,他就不說了.
我們開始看聖經了.
今天會看彼得前書四章的七至十一節.
如果一至六節的背景,其實在上個月我負責訊息的時候.
我都已經解了四章一至六節.
今天我們希望能夠說七至十一.
這一段經文其實是一段很值得思考的經文.
我希望逐節逐節能夠說,以致我們大約明白這些經文是說什麼.
同樣地,我自己翻譯完之後會說得更好.
因為這些不是在日本或環球星會找到的版本.
無論中文還是英文.
第七節是這麼說的.
所有事情的近聊,需要藉著祈禱當謹慎和清醒.
這些說話好像很廢話.
其實不知道說什麼.

$^{41}$總之好像有些東西已經說了.
要祈禱,要警醒,要謹守,要清醒.
好像明白的.
但其實故事不是我們想像中那麼簡單.
如果上個月有留心一點的話.
其實四章一至六節是在說一個很複雜的環境.
如果還記得一點的話.
前書的背景是尼祿王的政治地位不穩.
所以他要出一場大火來冤枉基督徒.
把罪名交給基督徒之後.
整個羅馬城市的基督徒.
受到接下來大的逼迫.
其實不需要有逼迫.
只要說會捉拿基督徒,這些傳言一出.
很多人就會離開.
所以在主要六十幾年的時候.
有很多基督徒去了五個不同的地方.
那五個不同的地方.
在今天的小亞細亞,即是土耳其一帶.
在那裡,其實你知道羅馬人不是那麼容易.
能夠接納外面的人進去.
所以有很多基督徒新進去的時候.
連朋友都不能做,連生意都不能做.
如果你要做的話,你要適應很多東西.
尤其是對信仰價值有抵觸的東西.
所以彼得寫這本書的時候.
就是在說面對這些情況,面對這些情景的時候.
信徒要怎麼做.
所以這句話是這樣說的.
萬物的結果近了.
其實萬物的結果近了是什麼情況.
要說萬物的結果近了.
其實是在說傷害.
傷害這個字.
我相信我們不陌生.
這幾年間.
我們不只是失戀的傷害.
不是找不到工作的傷害.
等等,或者有人怎麼咒罵你的傷害.
這幾年說傷害的時候.

$^{81}$在說很多.
在我們生命裡對一些事情的價值觀取捨的傷害.
我們信一些東西,突然間那些東西會沒有了.
換了另一套說法.
這些傷害很大.
就好像在羅馬城裡的基督徒.
在彼得的任避下,教導下.
他可以做到耶穌基督吩咐福音的能力.
但是去到新的地方的時候.
突然之間所有東西都不work out的時候.
他們受到很大的傷害.
很多人攻擊他,將他置於死地.
其實背景不單是這樣.
其實大約還記得.
在二十多年前,主後的四十多年.
有另一次的逼白.
那次的逼白大體上.
都是有很多人流散.
你當香港的情況.
九七前走了一批,現在走了一批.
走了一批,九幾年走了一批.
就是.
他已經適應了.
已經不再掙扎任何東西.
最傷的是什麼.
和基督徒二十年前曾經見過面.
相處過的一班人.
他們都有份.
迫害生理的人.
這些事情不陌生.
因為你想像一下.
如果你做了一份工作二十多年.
你頭五六年,在阿姐的職場裡.
罵你罵得很厲害.
你做了二十多年之後,你會怎樣.
看著新的年輕人入職的時候,你會怎樣.
你都會再打他幾年.
很普遍的.
因為我入職前頭五年.
都是這樣被阿姐打過.

$^{121}$所以那時候.
這番話是不容易說的.
當那些新去的基督徒面對著.
很大的傷害的時候.
不單外面的羅馬人.
對他不喜歡之外.
連那些應該支持他和他一起的人.
都不知為何不喜歡他.
所以彼得在寫什麼.
萬物結局近了.
就是說在傷害裡.
我們要看結局.
不是看傷害本身.
找個最簡單的例子.
失戀.
失戀是一個傷害.
劈腿.
不要你.
你高大英明神武,很多頭髮.
但怎知他選一個又沒頭髮又肥.
分手不傷害.
他選一個比自己差的人.
你覺得很大傷害.
但他說萬物結局近了.
他說你想想.
失戀是什麼.
所以很痛苦.
不過失戀總是有結局的.
你會再拍拖.
除非你很差.
如果不是的話.
你失戀完之後.
三年十年.
你都會走出來.
都會拍拖.
但傷害本身是什麼.
他說如果你專注在傷害那裡.
你看結局近了.
其實失戀的人你叫他清醒很困難.
所以他這裡說.

$^{161}$不是只能夠直著祈禱.
才能夠清醒.
真的.
你面對很多困難的時候.
你被傷害的時候.
你有很多想法.
怎樣才能夠清醒過來.
離得開受傷害的情緒.
祈禱.
當我讀完這一節的時候.
我心裡很不忿.
這兩三年.
看不到結局.
很難看.
直著祈禱.
人才能夠清醒和謹慎一點.
不要被傷害的情緒.
整天在裡面出現.
第八節還說得很離譜.
你說叫我祈禱.
偶然清醒一下.
我說他會做一下.
第八節說話離譜到一個地步.
用殷勤的愛對大家.
因為愛能夠遮掩很多罪.
我們首先解釋愛能夠遮掩很多罪.
不是說.
我犯了很多罪.
然後我愛人多一點.
請人吃黑根大斯.
我就把我的罪都解除了.
取消了.
不要亂想.
意思不是這個.
這段聖經是來自.
《針研》第十章十二節的下半節.
《針研》第十章.
或者附近的幾章數.
其實是說.
人面對很多難處的時候.

$^{201}$有兩個不同的選擇.
你恨.
不過恨會挑更多的傷害出來.
所以經文下半節說.
唯有愛才能遮掩很多罪.
意思是想說什麼.
意思是說.
在那困難的裡面.
你受到很多傷害的時候.
通常有兩個選擇.
就是.
選擇一個又肥又沒頭髮的人.
又不帥.
你可以心裡懷不滿.
或者你怨毒他.
把你以前在IG所有一起的照片刪掉.
凡是他tag的story.
你就後面加一句「嬲豬養」.
就是.
但你知道這些只會惹更多不好的事.
所以聖經用《針研》的故事.
他想說的是.
他面對你被傷害的事.
他說那些傷害的事可以怎樣處理.
因為結局會近的時候.
或者那件事會完結的時候.
你用愛.
愛才是唯一的方法.
去對付那些傷害你的人.
我看到這裡的時候.
我又不是很明白.
我相信這裡不是說.
當你受傷害的時候.
你沒有情緒.
不是.
不是說這些.
你有不開心,難過,很憂悶的日子是真的.
但他的point是想說的是.
你看看那件事有個結局的.
有些事每一件事總會有一個完結的日子與時間.

$^{241}$所以我們不知道是何時.
但你看著那件事的時候.
清醒的時候你知道.
其實我們仍然可以有選擇的.
因為唯有清醒地,謹慎地,藉著祈禱地.
我們會選擇一些很奇怪的事.
那件事叫做愛.
當這個世界很紛亂.
很多被傷害的時候.
愛這件事才顯得高貴.
我想說的這份愛不是haggandize.
你希望你清醒的是.
那個愛是有代價的.
是你在一個受傷害的裡面.
是一個你被壓榨的裡面.
你被一個欺壓的裡面.
他說愛才會回應這些事.
其實有時候覺得這些說話很左膠.
你看看第九節.
那些說話很離譜.
他說.
你明白嗎.
那個選了那個.
對不起我得罪了那些沒頭髮和肥的人.
我說的是我自己.
你想像一下那個又沒頭髮又肥的人.
也甩掉了那個女生.
她打電話給你.
你有沒有空出來安慰我.
你想一下.
你收到這個電話的時候.
你怎麼會不發怨言.
你十六個字的髒話.
你可以說的你都會說.
你終於知道誰好你.
你終於知道誰真正對你好.
誰知他出來的時候.
他就說.
我不是想和你一起.
我是想借你的耳朵和肩膀叫一叫.

$^{281}$你那時候更生氣.
要接待人不生怨言.
這個說法不是說.
我騎個靚肚.
然後我很警醒緊守.
然後我用愛去對付他.
他不是說意識形態上.
意念上的東西.
用接待對待大家.
不生怨言是說一個實質的行動.
我這件事是怎麼頂的.
我搜尋我的資料庫.
人生裡遇到很多事的時候.
我想一下.
這些說話其實在我心裡是否有效.
我真的找不到.
你明白嗎.
你叫我得罪你的人.
你要寬恕他.
九成是廢話.
你搖搖搖他.
他不會知道你搖搖搖他.
還要他會再得罪你.
你什麼時候搖搖搖完他.
他突然跟你說聲對不起.
你明白嗎.
我不知道你是否基督徒.
你是否一個正常的人.
我們九成多在裡面面對別人傷害你的時候.
你從來拿不到一句對不起.
這個才是真實的我們人生.
你偶爾能夠拿到一個人.
突然跟你說聲對不起.
你明白嗎.
你覺得是上帝的恩典.
但九成多在我們生命的裡面.
傷害你的人.
第一.
你問他傷害你的人.
你告訴他你傷害我.

$^{321}$你不知道那些人通常會回答什麼.
是嗎 我有做過嗎.
我不記得.
你氣到不能氣他.
你那種痛苦有多難.
我輸出了Database的時候.
我想起一件事情.
那件事情是若干年前的.
十幾年前.
仔細的場景我不交代.
針對一個團隊.
我們有一個團隊.
專門針對團隊.
你知道有些事情不是很好.
我們基督徒很蠢的.
礙手礙腳地說.
復和吧.
復和除了用愛用祈禱警醒.
能夠復和之外.
就像經文所說要接待大家.
不生怨言.
之後我的同工很不長進.
想了一招很糟糕的計謀.
他說不如洗腳吧.
彼此.
結果抓了一群人一起洗腳.
那個經常針對團隊的人.
我以為有大食大.
有些胃大粒的又如何.
當然是找他.
洗他的腳.
誰知他平時夠兇.
那個團隊裡沒有人找他.
你知道我有時候也很左膠.
我說這麼難的事.
讓我來吧.
結果他沒有人.
如果沒有人.
環境會更惡劣.
我真的去幫他洗腳.

$^{361}$你知不知道.
要脫鞋脫襪.
熱水毛巾.
洗腳最高難度是擦腳趾縫.
你明白嗎.
才有誠意.
不然放在腳上拿起來擦乾.
不是洗腳.
你知道洗的時候.
可以擦的時候.
那個人.
有一句說話.
我一生也不會忘記.
你不要以為做這些把戲和伎倆.
就能贏回什麼.
嘩.
那一刻只能用粗言穢語.
你知道很斯文.
不說這些了.
我真的想拿一盆水淋在他身上.
你怎麼做.
你怎麼擦.
你怎麼幫他擦乾腳.
你不要用這些伎倆.
就能贏回什麼.
我只能說.
我人生中其中一個最大的成績.
是我平心靜氣幫他做完整個程序.
你看我這一刻.
我不會再做這些.
除非上帝屈下我的腳.
跪在這裡.
手不是我的.
他在幫我聖靈感動地擦.
我人生中不會再做這愚蠢和左膠的事.
耶穌才能洗腳.
人生中只有耶穌才能做這愚蠢的事.
由大到小.
你看完耶穌洗由大腳.
我們就很左膠.

$^{401}$你覺得你洗完由大腳.
由大就會多謝你.
不是的,這個世界.
比由大更恐怖的人多的是.
我看著自己能夠做完整個動作.
還跟他搭著肩膀站在一起.
我只能說這是第十節.
這個講恩賜.
這個恩賜不是講那些錢輸的恩賜.
他說上面那七八九節能夠做到.
祈禱土能夠清醒.
能夠在意念上用愛戰勝邪惡.
第九節能夠有行動地.
把你的愛實現出來.
他說得很清楚.
不是你做到的.
他說是分派不同恩賜的神.
給了不同恩賜我們.
才能夠這樣服侍人.
這幅圖畫很誇張.
為什麼我要選擇擁抱的規則.
離別的規則.
MM7就不關事.
是因為離別的規則.
Jazz的新歌.
如果你有留心看一下.
他最後那幾句歌詞.
他們分手.
失戀.
分手之後.
他說這個離別的規則是要怎樣.
是要多謝那個曾經傷害你的人.
這些是很左膠的說話.
因為Jazz的歌全部都是三部曲.
他現在是第一部曲.
我想聽他第二和第三首歌.
在這個系列裡面怎麼寫.
但他說的跟這段劇情說的很相似.
傷害你的人.
其實是一些逼著你成長的人.

$^{441}$你要多謝那些.
曾經在你生命裡傷害你的人.
我認識一些姐妹.
不要說幾多歲的.
有些姐妹.
有一次跟我說.
她很怕跟那些A0,A1的人拍拖.
跟A0人拍拖不好嗎.
你知道A0是Available.
但她是零次拍拖.
叫A0.
大家明白吧.
大家應該熟悉我.
她很怕跟那些A0,A1的人拍拖.
她說不是啊.
這個很清潔.
對愛很有幻想.
A0是這樣的.
A0就是那些經常放閃.
怕死你不知道的.
你明白嗎.
如果你A0過你會知道.
你曾經A0過.
每個人都見過A0.
A0才浪漫一點.
那些姐妹說.
跟A0的拍拖.
我浪費時間,付錢.
陪那些A0成長.
我寧可跟那些A5和A6.
因為A5和A6那些.
經歷了前五任的前女友.
給她成長的故事.
她應該會成熟一點.
我不知道這個道理對不對.
我又不明白那些年紀的姐妹的想法.
我不是女生.
但我覺得這個道理好像很對.
我猜這個小例子是說.
不是沒有道理的.

$^{481}$或者我們基督徒說的話是.
原來上帝給了恩賜我們.
去服侍那些.
傷害我們的人.
第十一節說得更加清楚.
如果有人說.
如果你想說一些話.
你不是說你受過什麼傷害.
不是.
因為萬物叫我近了.
所以你要說的就是神的話.
如果你真的能夠去服侍他.
去接待他的話.
他很明顯說.
這些能力不是來自你.
是來自神給你的.
然後他才說.
他說唯有你做不到的事情.
不懂得做的事情.
做不到不懂得做的時候.
你知道你所做的一切都是來自上帝給你做到的話.
這個叫耶穌基督.
凡事上藉著他.
叫神得到榮耀.
福音是什麼.
福音不是說的那種說的福音.
我們說耶穌基督釘十字架.
我們是死的.
然後每個人仰望他.
得救就蒙恩.
不是的.
不是只是說的.
凡事上藉著耶穌基督說的是.
是那些是做不到的.
那些很困難的.
但是上帝給你能力給你恩賜.
你能夠說上帝給你想說的話.
人就會驚訝.
是神自己得到榮耀.
我想再說一個最近的.

$^{521}$我自己在想的結論.
最近因為教會要面對.
有沒有人能回到教會.
這個問題.
引起很多教會界的討論.
我認識很多不同的教會.
有很多不同的討論.
有人很容易就做了那件事.
發了電郵.
明白我說什麼嗎.
發電郵很簡單.
OK的.
但有些人仍然不想發電郵.
那些不發電郵的人.
他不能夠在教會的場景.
我認識一個人.
他堅持多一點.
他去不到教會服事.
你知道去不到教會服事.
已經約好了.
有些墓者打來給他.
你為什麼不發電郵.
他就說想和那些仍然堅持的人同行.
他說你這樣做.
我沒有意見.
但你知不知道.
你不發電郵.
有很多頂子媒很想聽你.
但他沒法聽.
你覺得對不起那些頂子媒嗎.
那個同工怎麼回答.
這一著都很難回答.
他回答說.
耶穌說九十九隻都放下.
只找回那一隻.
他說這個好答案.
我想那些墓者不敢說什麼.
在這個時候.
在一個很多紛亂的時候.
我相信一個堅持不發電郵的人.

$^{561}$被不同的墓者都會和他說很多這些話.
他心裡會想什麼.
為什麼他做一件他覺得應該做的事.
他身旁所有的人都好像不會明白.
說那句話.
我就沒什麼意見你這樣做.
其實這句話已經很好.
有些人可能說得更加古靈精怪.
我想表達的是.
我們被人逼在一個.
鬥爭當中.
在一個內耗裡面.
在一個自己人很容易就打自己人的環境下.
我相信這個也是彼得寫信給.
那些剛剛移民去那五個地方的弟兄姊妹.
他面對著那些二十年前的基督徒.
對他們有意無意的傷害的時候.
彼得說.
你做到愛 做到接待他們.
不是因為我們本性上做到.
是因為我們本性上做不到.
我們只會覺得鬥講聖經.
誰講聖經講得對.
誰能夠串到對方.
以至對方能夠啞口無言.
誰贏不是.
他不是想說這件事.
用愛.
將那些不明白的人.
不理解的人.
尤其是那些人在背後.
藉著很多不同的事情.
能夠搶我們內耗的時候.
擁抱的意思是什麼.
John就說擁抱圓圓.
上兩星期的照片.
她現在受到威脅.
有個男孩.
如果你看她的Facebook就知道.
有個男孩在追求圓圓.

$^{601}$說笑的.
她不是真的.
很小的時候 怎麼追求.
John就說.
昨天就抓了圓圓去拍拖.
搶回地位.
你有看她的Facebook就知道她在說什麼.
她擁抱不是一種.
你愛一個人的擁抱.
擁抱其實是說.
擁抱一些跟你不同的東西.
你知道你做不到.
你擁抱不了.
你只是祈禱求神清醒.
你信的這個意念是愛.
能夠戰勝那些offend你的人.
你求神恩賜給你.
求神給你講要講的東西.
給你能力.
讓你能夠做到那些東西.
頂智媒今天什麼叫傳福音.
希望呼出頂智媒.
我們當中有很多很多的見證.
是說上帝在感動我們.
給我們能力.
在很多人會傷害我們的位置和環境當中.
堅持上帝神跡地救我們.
不用傷害彼此傷害.
Fortune出了一本家書.
不是在說我們跟別人很不一樣.
不是在說我們特別厲害.
不是不是.
Fortune出的家書.
是想說一件很真實的事.
當一些被邊緣化的人.
我們從不放棄.
縱然這個選擇.
是會招募很多人.
各樣各樣的說話和聲音.
但是跟著這段說話這段聖經.

$^{641}$給我們一個場景和一個教導.
在很多人誤會的時候.
很多人未理解的時候.
甚至很多人惡意地說.
甚至很多人惡意地傷害的時候.
甚至要置於死地的時候.
親愛的Fortune姐姐姐姐妹.
請你謹記.
不要在我當中產生任何的不信任.
不要在我們這個群體的內部.
這塊有很多外面像擁抱背後卻懂得的感覺.
Bull Church的群體.
仍然是一個以真理紮實並要進行的群體.
很叫福音地傳揚開去.
正如你所知道的.
昨天跟我吃飯的那三個人.
其實年紀都很大了.
你知道我二十多歲讀神學.
你明不明白.
那時候跟我做同學的人都比我大很多年.
不是一些年輕的人.
看到上帝的榮耀在我們當中.
是很多仍然對方信仰堅持的人.
在我們當中陪伴Bull Church的重行.
越困難.
不是因為困難的緣故我們各自一盤散沙.
困難不要叫Bull Church裡面成為一盤散沙.
在越困難的裡面求神讓我們說回祂說的話.
用祂的能力去服事我們服事不到的東西.
找回祂給我們不同的恩賜.
在困難裡面仍然服事.
當這個世代很多意識形態叫我們不斷分裂的時候.
求神讓我們繼續擁抱下去.
我一直在這裡禱告.
天父多謝你.
多謝你跟我們每一個人說話.
我求天父你將這段聖經成為我們生命裡面一個很重要的提醒.
尤其是一個很紛亂的世代.
一個很容易越來越混亂的時候.
我們不是說誰一定要對誰一定要錯.

$^{681}$其實對錯真的不是很重要.
那份愛那份堅持.
將很多事根本上不需要分對錯的時候.
可以讓我們看到一個新的局面.
我求你讓我們看到一個新的局面.
是你跟我們每一個Bull Church頂尖的說話.
特別是為世界各地的頂尖的姐妹去禱告.
我求天父的是.
你仍然祝福他們在那邊成就更多奇妙的事.
天父我求你.
叫他們都為我們Bull Church去禱告.
讓Bull Church是一個群體.
是來自世界各地不同的人用禱告承托住的群體.
為此我們可以有力量繼續走下去.
多謝天父你聽我們在你面前的祈禱.
歡迎送你寶貴名球.
多謝.
\newpage



\section{羅馬書 14:1-15:13-20220528}
\label{sec:OJx_AQpZJpY}
\textbf{【網上崇拜】擁彼此接納──飯就一定要食,一齊食開心D ?|羅馬書14\_1-15\_13|20220528 [OJx-AQpZJpY]}
\newline
\newline
連結: \href{https://youtube.com/watch?v=OJx-AQpZJpY}{\texttt{ https://youtube.com/watch?v=OJx-AQpZJpY}} ~~~~ 語音日期: 2022-05-28 
\newline
\newline
\hyperref[sec:2t83SY_sddQ]{\small{< < < PREV SERMON < < <}}
~
\hyperref[sec:index_chronic]{\small{[返順時目]}}
~
\hyperref[sec:index_scriptual]{\small{[返順卷目]}}
~
\hyperref[sec:NkL6a2IH8uY]{\small{> > > NEXT SERMON > > >}}
\newline
\newline
$^{1}$弟兄姊妹 各位朋友.
近處的 遠處的 都願你平安.
我們都知道平安不是必然的.
所以我們除了要繼續為香港祈禱之外.
也要繼續紀念在其他地方都失去了自由和平安的人.
願主憐憫.
烏克蘭被俄羅斯侵略了已經差不多三個月了.
本來想著強欲懸殊那種一面倒的情況都不似預期.
如果可以早些停止這場戰爭.
減少人命傷亡 減低財物損失.
那就好了.
不過都要看看結束戰爭的方式是什麼.
因為通常強勢那邊是會欺壓弱勢那邊的.
去被人妥協.
要人同意各樣土地權利資源.
人家拒絕的話.
就會批評他不合作 搞亂秩序.
這些人把事情反過來說.
稍為有常識的人除了激氣 仍是激氣.
一個人有情緒的話.
有時並不是因為他不理性.
而是因為正正有見識 有常識.
才會察覺到有問題 才會激氣.
所以從這個角度來看.
我們心裡面還有一團火的話.
是好事來的.
總比做一條鹹魚好.
有些有用心的人.
其實他明明不講道理.
還裝理性地回歸理性.
你就知道他是什麼樣子.
其實這些權力的角力.
在日常生活裡面.
在公司 在學校 在家裡.
甚至在教會裡面.
你都可能似曾相識.
誰強 誰弱.
有時並不是看他有沒有道理.
而是很視乎人數支持或反對那邊.
多人一點.

$^{41}$看哪邊的聲音大一點 惡一點.
至於誰有話事權.
都是一些關鍵.
通常沒有權就沒有 say.
少數服從多數.
就算多數服從少數也好.
都難免信得過情 失數易.
很難每個人都滿意.
如果想擁抱 其實談何容易.
即使大家關係很好.
見面的時候擁抱這些畫面.
在華人的文化都不常見.
但其實很多人都很想有.
包括我的小組組員都分享過.
不過他說覺得不好意思有點尷尬.
其實我也是的.
想 但又縮.
如果關係不太好.
有些事談不攏.
可能連眼神都想避開對方.
不想理會.
但就說人家不理會你.
沒風度沒禮貌.
跟你打招呼的話.
又說人家虛偽.
即使我贏你輸.
你都說了 很難搞的.
其實日光之下真的沒什麼新事.
羅馬教會.
在羅馬教會內部都有很大的分歧.
還炒得很激烈.
有人認為有些事觸碰到信仰的紅線.
不可以越位.
但有人就覺得是綠燈.
明明可以越位.
你不越位.
為什麼連人家想越位你都不讓人家越位.
其實大家關係很好的話.
一直都很好的話.
一去到這些位置就會有泥鳥.

$^{81}$就可以看到.
大家的感情是否經得起考驗.
究竟是同餅同杯合一教會.
還是飯有飯有都變成韭菜.
被人吃著或者想吃著人.
保羅在羅馬書12章裡面提到.
如果是可行的話.
總要盡力與眾人和睦.
而到今天所說的經文第14章裡面.
保羅就用了教會發生爭拗的例子.
去指出在繃緊緊張的狀態裡面.
實踐和睦共處是還有可能的.
那應該怎樣做呢.
保羅在這段經文裡面說了很多東西.
其實今天的經文都比較長.
有三十多節的經文.
但簡單來說.
他的重點就是要彼此接納.
在第14章一開始第一節就說到.
要彼此接納.
羅馬書到第15章七節.
到尾聲段落的時候.
保羅都重複再說彼此接納.
這個相信是重點.
其實這一類的屬靈原則.
好像彼此相愛.
其實不難明白.
但實踐起來有時真的不容易.
好像顧客服務.
Customer Service.
顧客至上.
這四個字我們每一個都認識.
但面對一些客人無理的要求.
是否甚麼都要Entertain就殺他們呢.
有些公司收到客人的投訴.
都未了解未Fact Check.
就會自動Default.
說職員不對立即跪下.
其實這些情況都頗普遍.
而顧客至上.

$^{121}$無極限到.
有條龍都出來了.
就是過散龍.
這條龍真的不是太好.
身為下屬.
你想駁嘴的話.
後果似乎.
很多時候都要硬食.
彼此接納大家都相信.
沒有人反對.
是很重要的.
但甚麼都要接納.
連一些很離譜的東西都要接受.
還引用聖經金句.
但又用得不對的話.
就好像剛才說的那條龍.
就Over了.
譬如聖經說甚麼呢.
跟你說聖經說不可以批評.
對著一些很受委屈的人來說.
對著他這樣說.
難免人家覺得.
我現在想表達不滿你都不給我.
這樣會造成傷害.
甚至是二次傷害.
相信在這裡.
或者在網上的你.
都可能以前經歷過.
很難過又很難堪.
接納是雙方面的.
總要看看是甚麼情況.
一會兒都會說這一點.
接納在消極那方面的意思.
就是不要批評.
不過保羅.
他不是自己都在批評教會嗎.
那是不是雙重標準呢.
其實認真討論.
談談哪個建議是利多於弊.
在討論過程中.

$^{161}$怎會沒有批評呢.
如果表達意見就說會引起仇恨.
這個社會怎會有進步呢.
有些解英家就說.
原來那些批評不是討論這件事那麼簡單.
而是夾雜著一些攻擊.
毀謗.
誇大 失實.
妖魔化人.
其實人性裡面的黑暗.
發生這些事都不出奇.
這些批評還會對人造成.
很嚴重的傷害.
保羅說會使人敗壞.
使人很憂愁.
有塊絆腳石放在他前面.
擊倒人.
還毀壞神的工作.
其實批評一下而已.
有沒有那麼嚴重呢.
都很玻璃心.
說你兩句就敗壞了你屬靈生命.
心理陰影比月球還大.
有沒有那麼誇張呢.
這些批評原來是很大件事.
保羅在解釋.
譬如絆腳石.
在羅馬書第九章都說過.
他在形容會令人跌倒.
會傷到一個地步.
是令人失去救恩.
其實為何那麼嚴重呢.
教會當時的分歧.
主要分了兩個立場.
一邊認為不可以吃肉.
絕對不可以吃.
酒都不能喝.
當然有些人覺得.
不讓我喝酒可能跟你拼命.
另外還要嚴格遵守宗教節期.

$^{201}$另一邊有些人相信.
這些東西對我們屬靈的生命.
沒有什麼影響.
甚至一點影響都沒有.
不准吃肉.
理你也傻.
聖經我會照讀.
不過肉照吃.
酒照喝.
其實我們都不肯定這個爭議.
是否因為肉和酒製造那麼偶像.
還是在羅馬這些愛邦人的地方.
那些肉類切割員.
他帥不帥就不知道了.
這裡沒有說.
有沒有六腳福基刀.
這些不是重點.
大家坐重點.
在他屠宰的過程裡.
那把刀可能切完豬肉.
或者牛.
所以有些人就很擔心.
會觸犯摩西的律法.
無論實際情況是怎樣.
可以肯定一點就是.
有一班信徒.
很擔心吃了肉會得罪上帝.
這些本身可能是.
猶太裔的基督徒.
可能還熱心到一個點.
在別人跟著他一樣.
都不可以吃肉.
這樣那麼霸道.
難免會引起一些反感.
也有解英家認為.
這個爭論的場景發生在甚麼地方.
就是發生在教會裡面.
他舉行聖餐.
而習慣上是會一起吃飯的.
本來一起吃飯開開心心.

$^{241}$誰知道就搞到勢成沮火.
在十四章十五節.
又說到.
如果你因為食物使弟兄憂愁.
就不是按愛心行事.
一說到憂愁.
就是不開心.
這幾年很多人都回不了鄉.
煙便宜到甚麼程度都沒用.
一說到都不開心.
但保羅所說的憂愁.
其實是厲害很多倍.
他的意思是說靈裡面的痛苦.
除了受不了弟兄姊妹的冷嘲熱諷.
感覺好像被一根刺刺到心.
入得很深.
更可能因為抵受不住群眾壓力.
看到很多人吃肉.
於是半信半疑.
半推半就.
吃就吃吧.
誰知吃了又怕得罪上帝.
在二十三節說.
有些人帶著疑惑地吃.
但吃完之後心裡很不安.
良心又受到責備.
於是又想離開教會.
去尋找尋求.
去加入一些認同自己是吃素的群體.
可能就是猶太教的群體.
以後就不用對著那班人.
保羅吩咐那些支持吃肉的人.
不要吃肉.
表面好像很和諧.
其實有原因的.
原來是事出有因.
在二十一節說到.
無論吃肉或飲酒.
或其他事都好.
如果令弟兄跌倒的話.

$^{281}$的確不要做.
何況在當中.
特別在十到十二節都提過.
我們都要站在神的審判台前.
要向神交代.
小心看路.
保羅說不要互相攻擊.
不要互相批評.
除了不想令人心灰意冷.
甚至失落了基督的信仰.
另外還有一個比較正面的原因.
就是在第三節.
神已經接納了他們.
弟子再繼續說.
神怎樣接納他們呢.
原來他們無論吃肉或不吃肉都好.
大家的心的出發點都是為了服侍神.
如果神不OK的話.
神是會干預的.
現在就是透過保羅去干預你的干預.
因為你的干預方式錯了.
換言之如果你繼續批評的話.
即是在批評神.
你不認同那些人.
神認為OK.
那暫時停火吧.
在第七至第九節.
保羅繼續說.
為主而活的重點.
我們若活是為主活.
是死是為主死.
其他人都和你一樣.
大家的堅持都是為了主.
換言之.
留意一下.
現在不是只有我或你最熟悉.
其他人都要多點欣賞.
立場一不同的話.
條件性神經反射.
設定default.

$^{321}$先否定別人.
是很傷心的.
很多人的成長環境.
習慣了很多批評.
讚賞很少.
肯定很少.
覺得不被接納.
我們很多人都是在這樣的環境中成長.
或許當中都不少受到影響.
甚至到現在都是這樣.
間中都是這樣對待人.
即使是這樣我們都要靠神的恩典.
去更新我們的生命.
不可以說我是這樣就這樣.
沒辦法的.
保羅在我們《耳熟能詳經文》中.
十二章一到二節.
他說弟兄姊妹們.
我以神的慈悲勸你們.
將身體顯向當作末制.
是聖潔的神所喜悅的.
你們如此侍奉大是理所當然.
我們有這種生活態度.
是因為要回應神的恩典.
我這個人可能還是三千八個.
神都接納.
神都這麼愛我.
神的愛融化我.
所以我願意立志擺上.
為新基督生命更新.
當然現在香港的教會所面對的爭論.
相信大部分都未嚴重到.
會令人想離開基督的信仰.
但是有人一旦離開教會.
即使不是改信其他的宗教信仰.
都很容易會失落並不理想.
求聖靈幫助我們.
幫助所有人都願意為主而活.
不過你可能也有一個疑問.
有一個問題.

$^{361}$我又不是保羅.
我怎知道神有沒有接納他的想法或做法呢.
如果不肯定別人這樣做是否為了主.
不是說我不想欣賞他.
而是我不想盲目欣賞或懶熟靈.
其實聽其言觀其行.
雖然不一定準確.
但憑著觀察認識.
總比靠估或感覺穩定.
例如堅持只吃素的人.
相信他們根據摩西的律法.
認為這些肉不能吃.
態度很認真.
而不是隨口說.
我說了不行就不行.
解釋又沒有邏輯.
可能是未想清楚.
或者任性喜歡怎樣就怎樣.
要人聽自己的話.
話說最近有一個晚上.
有人乘坐小巴紅VAN.
下車後才發現手機不見了.
走回頭.
希望可以找到.
走回小巴.
差不多到小巴站頭.
看看有沒有可能找到手機.
差不多到了.
遠處看到剛才的小巴司機.
已經站好.
還跟他揮手.
喂 一直在等你.
很開心地跟他說.
機主也很開心.
因為失而復得.
當場開心了.
當然要給點心給這位司機大哥.
如果我們對人沒有信任.
可能會覺得這位司機有可疑.
有謬論說他.

$^{401}$可能本身是想偷了手機.
但看到機主突然出現.
自己一時不小心.
很開心.
得意忘形地搖來搖去.
機主就出現了.
馬上轉口說.
我在等你回來.
其實懷疑如果沒有合理的基礎.
是可以無限loop.
永續懷疑.
但本身證明不了什麼.
但也否定不了什麼.
不過被人指控的就很慘.
如果你也試過被人誤會.
譬如你在店裡弄跌了一個玩偶.
散了.
有東西要你賠錢.
你會怎樣呢.
當然生氣.
給十元也嫌多.
請姐妹己所不欲.
願主幫助我們彼此接納.
純良 善良 良善如甲子.
醒目上.
大家可以填充.
你覺得誰是值得你欣賞和學習的.
無論是台下身邊的頂尖.
或是台上敬拜隊的成員.
你可以填充一下自己.
醒目是很重要的.
醒目我們也很喜歡.
Smart.
如果豬隊友為主做了一些.
很明顯不太聰明的事.
很含蓄.
不太聰明的事.
他本意是好的.
但好心做壞事.
也要出聲提醒他.

$^{441}$祝福他.
保羅經常說.
彼此包容 彼此接納.
不要緊 隨他吧.
其實保羅不是甚麼都無所謂.
有時也很不客氣.
站得很硬.
在哥林多前書第五章.
他直斥其非.
去丙教會.
他說他們自高自大.
你不應該覺得痛心.
因為有些人犯了嚴重的罪.
你也隨他吧 不出聲.
也不趕他走.
當然有些相對影響不大的事.
好像羅馬書這個案件.
保羅認為.
不論意見不同.
可以留一些空間給別人.
不一定要每次都堅持.
要全部跟足自己的意思.
有時70分 80分已經很好.
不用每次都要拿到盡.
去到120分.
這段經文其實有一個.
很明顯的糾結位.
我一直在預備經文的時候.
也覺得很卡.
卡了很久.
就是在十四節.
十四章十四節.
保羅說憑著主耶穌.
確知心信.
凡物本來是沒有不潔症.
即是說食肉是沒有問題.
那可以了.
即是說不准食肉的人是不對.
那就很簡單.
待會請潘Sir報告的時候.

$^{481}$加多一樣東西.
歡迎大家鎮留部.
一起吃日式串燒.
任食牛羊豬雞魷魚都可以.
還加上黑根特雪糕.
假設而已.
那我以後在教會食肉.
就不用被人眼望望.
以為自己做錯了什麼事.
以為自己好像那些在巴士上.
用手機看片.
開大喇叭的人.
不要這樣.
好了.
本來以為以後都可以堂堂正正.
食肉的時候.
誰知道晴天霹靂.
就像上星期一樣.
英超球隊阿士東維拉.
本來很驚喜地領先.
曼城2比0.
誰知道最後被曼城進了三球.
如果入少一球就好了.
打和.
利物浦都可以英超封王.
可惜創造四冠王的歷史美夢.
就此破滅.
不過三冠王都好.
今晚還要看歐聯.
想不到這個時候保羅就是這樣.
突然在這個時候再繼續說.
弟兄姊妹你還是不要食肉吧.
接著再補多一刀.
在十五章第一節.
我們這些堅強的人.
是應該分擔那些不堅強的人的軟弱.
明明食肉是對的.
為什麼現在又說不吃.
又說什麼什麼要分擔軟弱呢.
擺明是情緒勒索.

$^{521}$你都說了食素的人是軟弱的.
在十四章二節說到.
那些食蔬出來的是軟弱的.
即是說他們對這個信仰這個case的理解是錯了.
雖然你說我們堅強.
你說我們很堅強.
我跟你差不多level這麼厲害.
保羅多謝你這樣吹捧我.
但我知道你堅強的意思不是這樣.
但你說我堅強.
就是狠堅強狠堅強狠堅強.
你玩一個loop一個小時兩個小時都沒有用.
我不相信.
因為我知道你的意思.
剛才你都說我好像稱讚我靈性很好.
但其實你都知道我猛攻擊那些.
和恥笑那些和我的意見很不同的人.
只是剛好我對這件事的理解是正確.
老老實實.
我都覺得自己的靈性品格不算太差.
不過公道一點公道一點.
我都不是算很好.
但你叫我什麼堅強分擔軟弱.
是不是太看得起我呢.
Why me?.
Why always me?.
你都說了在十七節那裡說.
神的國不在乎飲食.
而在乎公義.
這樣做公平嗎.
公平嗎.
保羅我們看清楚一點.
其實保羅真的有處理過這件事裡的公義問題.
不是何諧無胡是非黑白.
起碼保羅有說清楚誰對誰錯.
保羅不是說食肉本身是沒有問題的.
而食素的人雖然精神可嘉.
可惜堅持錯了.
不過在神的國裡除了公義之外.
還有聖靈中的喜樂.

$^{561}$聖靈中的喜樂.
喜樂在這個段落裡.
最後的部分都有提到過.
在十五章十三節.
願賜盼望的神.
因你們的信.
將各樣的喜樂.
充滿你們的心.
由於時間或經文的蝙蝠關係.
今次未能詳細講解這段經文.
但簡單來說在第八至十二至十五章裡.
保羅引用了幾卷舊約聖經的經文.
說明基督已經按照神的應許作王.
而我們這些蒙恩有幸作神子民的愛邦人.
當然也包括猶太人在內.
大家都來自五湖四海.
現在可以帶著盼望一起敬拜.
侍奉基督.
Togetherness.
一起吃飯.
不要分開.
除非真的要流散.
在神的.
是一起吃飯不要分開.
是神的本意初心.
在十三節十五章.
就是承接上文.
去回顧喜樂.
亦都回顧.
要透過彼此的接納.
用這種方式來接待.
服侍耶穌.
因為我們相信.
基督耶穌是主.
所以信服主的心意.
去接納那些和我們意見很不同的人.
喜樂除了來自舊恩的喜樂.
盼望之外.
喜樂也來自和平.
融洽的人際關係.

$^{601}$喜樂是聖靈的果子.
裡面的一種特質.
而聖靈果子本身就是愛.
而聖靈裡面的喜樂.
就應該有愛的特質.
所以彼此相愛.
就會令人快樂喜樂.
正如聖經第十五章說到.
耶穌說遵守我的命令.
實踐彼此相愛.
就好讓喜樂.
存在你們的心.
在你們的心裡.
一起吃飯都可以很自在很開心.
不會吃飯前吃飯中.
吃飯後都會擔心被人說.
排斥.
修身不舒服.
彼此相愛彼此接納.
是很美麗的.
不過是要付代價的.
有時是要付代價.
問題是誰付這個代價.
雖然保羅勸人.
要分擔不堅強的人的軟弱.
相信不是說任何情況下.
都要絕對的來遵從.
當然要看情況.
在帖撒萊前書第三章.
保羅說得很白.
有些人是不肯工作.
明明有能力工作.
都不工作.
在這裡吃免費午餐.
保羅說這些人不可以給他免費吃飯.
要安分做工自食其力.
會不會在接受了萊加教會的人.
去處理這件事的時候.
之前就說不是啊.
羅馬書就看過你說過.

$^{641}$他說不要.
讓他一起在教會飯堂吃飯.
因為他有需要.
因為他只是一時軟弱.
他一時理解錯了聖經.
認為神供應人的需要.
現在就透過教會供應.
現在就透過我們.
透過教會供應給他們的需要.
他有能力工作也要供應他.
吃飯?.
傻丫頭不要玩了.
但相反在羅馬教會裡面.
那班支持吃素的弟兄姊妹.
就是出於信仰的堅持.
是情有可原的.
是認真的.
雖然他們比較固執.
不輕易去面對自己的軟弱.
也不願意認衰.
不輕易去面對自己的軟弱.
在教會聚會的時候.
就有人負責去公開宣讀.
保羅這封書信.
當有些信徒聽到的時候.
可能是堅持吃素的那班弟兄姊妹.
可能就跟旁邊的朋友聊聊天.
說保羅是仕途徑.
地位崇高學識遠播.
但不是說你說什麼我都要聽.
誰適合我我都要聽.
現在還說我軟弱.
是的.
獨立思考批判能力的確是很重要的.
不會輕易被人弄掂.
但有時候那份自信不肯聽.
完全不聽人說.
甚至說誰適合我就聽誰.
其實可能意思實際上只是說.
跟我立場一樣的就聽吧.

$^{681}$我自己也是過來人.
有人提醒自己的時候.
當刻是很難接受的.
有時候真的要過了很久很久很久.
久到要以年計.
好幾年再加上一些跌跌碰碰的經歷.
才會看到自己的盲點.
是的.
當日他這樣說是對的.
說得對的.
是為我好的.
覺今是而昨非.
不過就後悔了.
可能關係已經不能再好好.
回不了頭.
不過現在後悔也沒用.
很糾結也沒什麼用.
不如放棄吧.
It's time to let go.
提醒自己不要再犯這些錯.
Don't be afraid.
頂枝妹你想不想let go.
有些事.
除了看《聖經》之外.
也會考慮去看.
最近有一套電影是說戰機師的故事.
電影也很棒.
那套電影除了說攻防戰.
也很有真實感之外.
其實很多情感位置也很棒.
當然下星期也有一套電影會上映.
明就明吧.
後定義號之類.
也要留意.
保羅勸人.
十五章第一節.
特別對吃肉的弟兄姊妹.
很硬意.
其實是叫他們硬吃.
分擔不堅強的要軟弱.

$^{721}$雖然我們剛才也說了.
不是任何處境都要絕對的順從.
這些吩咐.
但這裡有一個原則也要留意.
就是要為人設想.
有沒有為人設想.
現在保羅說了誰誰是不對的.
但他聽也要時間消化.
請你為他設想忍耐一下.
我也明白的.
吃火鍋怎麼可以不吃肥牛呢.
連喝啤酒也不行.
是挺慘的.
不過每個星期只有一次聖餐和聚餐.
一個星期只有這天不好吃.
不是叫你天天都不能吃.
希望你能夠諒解.
因為他們真的有機會.
離開基督的信仰.
希望他們能夠早日明白你們一片的負心.
彼此接納.
弟兄姊妹願不願意這樣做.
保羅這樣寫了這封書信.
相信保羅寫的時候也沒有被這個把握.
他沒有的.
可能有一點點.
反而覺得可能會得罪所有人.
被人罵爆.
這個可能性更大.
一時又說人家軟弱.
一時又說遷就那些和自己有牙齒印的人.
吃力不討好.
說得什麼都玩命.
不過保羅為了基督去勸人去作出犧牲.
他自己也身先士卒.
做好榜樣.
就好像保羅自己所說的.
第十五章第三節.
不求自己的喜悅.
只求基督.

$^{761}$正如基督不求自己的喜悅.
為了其他人的益處.
促成彼此接納.
就可以經歷去體會他有天下特色的喜樂.
為人設想OK.
不過他值得我為他設想嗎.
他值不值得 他配不配.
不容易.
撒瑪利亞人的比喻裡我們都熟悉.
律法師去問耶穌.
誰是我的鄰舍呢.
換言之.
這個律法師在想誰是我的鄰舍呢.
我就會幫他.
如果不是我就不會幫他.
耶穌的回應說問題並不在於他是誰.
而是要看他有沒有需要.
有沒有這個需要.
要不要幫.
要不要擁抱.
這就是愛人如己 愛倫如己的精義所在.
這一刻 弟兄姊妹.
你想擁抱誰呢.
你覺得誰最需要你的擁抱呢.
那些你覺得跟你意見不合不同的人.
他是不是都需要擁抱呢.
現在沒有打針的人.
我所認識的都有充分的理由不打.
即使是想拿豁免.
但可能都比起咳有線更加困難.
無論是如常.
沒辦法繼續如常參加崇拜.
但相信香港的教會很多都是不想拒絕人去進入教會.
除非抗疫抗到一個點.
就是沒有心力.
甚至充滿了腦毛.
即是不知道從哪裡來的腦毛.
於是不問前由 很機械式的配合.
完全沒有事.
有人是想去 但去不到的苦況.

$^{801}$香港教會所面對的處境並不一樣.
各自都有不同的資源.
有不同的限制 有不同的難處.
相信很多都有經過祈禱.
尋求如何在種種的限制裡面.
找出一個最合宜的方式.
來回應這些突如其來的改變和挑戰.
但很可惜都找不到.
都不太輕易找到一個位置.
可以令每個人都清心滿意.
怎樣都可以有些期望.
有些人是照顧不到.
不是說分開兩個腦打邊爐.
這個不吃辣的就可以吃一個不辣的湯底.
一個就辣的湯底.
因為現實裡面很多東西.
是多過兩個選項的.
大家的意見很平常.
但願意我們的出發點.
都是願意為人設想.
即使未能夠完滿解決問題.
但起碼相信可以緩解氣氛和關係.
亦都解開多一些的糾結.
假如教會有同工未能夠進入教會.
繼續工作侍奉.
可否有所考慮.
其實可以考慮在家工作.
或者在其他地方辦公.
或者有些工作是真的要在教會裡面執行.
執行不到 做不到.
可否按比例 按工作量或性質來扣除.
然後怎樣調整.
按比例去發薪水.
大家都為大家想想.
尋求共識.
不是想都不想就分手.
這是不理想的.
教會按個個情況來考慮可能的方案.
應對目前的改變需要.
Full Trust都不例外.

$^{841}$很想接納有限制的人.
受到限制的人.
同時都很想重視.
那些限制少一些的人的權利.
作出取捨.
實在不容易.
空間有限 位置不多 杯水車薪.
但相信我們所做的都是按照主的心意去做.
去接納 去包容.
可能有人覺得這些安排值100分.
也都角度不同.
可能有人覺得低於80分.
大家看法有差異可以理解.
不過願意我們都繼續持開放的心.
Full Trust向來強調.
不是門開了就不能關.
隨時有心理準備作出改變.
不會一成不變.
不變的應該是我們效法基督的心.
效法基督.
即使面對有100個issue.
100個issue.
有1萬個意見.
願意我們有100萬個接納.
但有一個心已經很足夠.
有一個效法基督的心.
有公義 也都有喜樂.
願聖靈幫助我們.
好像戰鬥機的機師一樣.
無論是急轉彎.
急速的轉彎.
加速 減速 上升 下降.
反轉再反轉.
身體都能夠承受到那些.
令人很難受的G-force的壓力.
進入這些Danger Zone.
這些危險的地帶.
超越了人體的極限.
即使面容扭曲.
仍然可以保持頭腦清醒.

$^{881}$去保持初心不變.
完成上帝的旨意.
願榮耀歸於我們.
諸耶穌基督的父神.
在15章13節裡面.
有一個祝福 祝禱.
我也很願意在結束的時候.
用這節的經文.
跟大家一起彼此的祝福.
願賜盼望的神.
因你們的信.
將各樣的喜樂 平安.
充滿你們的心.
使你們藉著聖靈的能力.
大有盼望.
\newpage



\section{路加福音 15:11-31-20220604}
\label{sec:NkL6a2IH8uY}
\textbf{【網上崇拜】然後去學習擁抱一切|路加福音15\_11-31|20220604 [NkL6a2IH8uY]}
\newline
\newline
連結: \href{https://youtube.com/watch?v=NkL6a2IH8uY}{\texttt{ https://youtube.com/watch?v=NkL6a2IH8uY}} ~~~~ 語音日期: 2022-06-04 
\newline
\newline
\hyperref[sec:OJx_AQpZJpY]{\small{< < < PREV SERMON < < <}}
~
\hyperref[sec:index_chronic]{\small{[返順時目]}}
~
\hyperref[sec:index_scriptual]{\small{[返順卷目]}}
~
\hyperref[sec:SLTU9ILLofg]{\small{> > > NEXT SERMON > > >}}
\newline
\newline
路加福音 15:11-31-20220604
\newline
\begin{longtable}{cl}
\hline
\hline
章節 & 經文 (和合本修訂版)\\
\hline
15:11 & \begin{tabularx}{0.7\textwidth}{X} 耶穌又說:「一個人有兩個兒子。 \end{tabularx} \\ \\ \relax
15:12 & \begin{tabularx}{0.7\textwidth}{X} 小兒子對父親說:『父親,請你把我應得的家業分給我。』他父親就把財產分給他們。 \end{tabularx} \\ \\ \relax
15:13 & \begin{tabularx}{0.7\textwidth}{X} 過了不多幾天,小兒子把他一切所有的都收拾起來,往遠方去了。在那裡,他任意放蕩,浪費錢財。 \end{tabularx} \\ \\ \relax
15:14 & \begin{tabularx}{0.7\textwidth}{X} 他耗盡了一切所有的,又恰逢那地方有大饑荒,就窮困起來。 \end{tabularx} \\ \\ \relax
15:15 & \begin{tabularx}{0.7\textwidth}{X} 於是他去投靠當地的一個居民,那人打發他到田裡去放豬。 \end{tabularx} \\ \\ \relax
15:16 & \begin{tabularx}{0.7\textwidth}{X} 他恨不得拿豬所吃的豆莢充飢,也沒有人給他甚麼吃的。 \end{tabularx} \\ \\ \relax
15:17 & \begin{tabularx}{0.7\textwidth}{X} 他醒悟過來,就說:『我父親有多少雇工,糧食有餘,我倒在這裡餓死嗎? \end{tabularx} \\ \\ \relax
15:18 & \begin{tabularx}{0.7\textwidth}{X} 我要起來,到我父親那裡去,對他說:父親!我得罪了天,又得罪了你, \end{tabularx} \\ \\ \relax
15:19 & \begin{tabularx}{0.7\textwidth}{X} 從今以後,我不配稱為你的兒子,把我當作一個雇工吧。』 \end{tabularx} \\ \\ \relax
15:20 & \begin{tabularx}{0.7\textwidth}{X} 於是他起來,往他父親那裡去。相離還遠,他父親看見,就動了慈心,跑去擁抱著他,連連親他。 \end{tabularx} \\ \\ \relax
15:21 & \begin{tabularx}{0.7\textwidth}{X} 兒子對他說:『父親!我得罪了天,又得罪了你,從今以後,我不配稱為你的兒子。』 \end{tabularx} \\ \\ \relax
15:22 & \begin{tabularx}{0.7\textwidth}{X} 父親卻吩咐僕人:『快把那上好的袍子拿出來給他穿,把戒指戴在他指頭上,把鞋穿在他腳上, \end{tabularx} \\ \\ \relax
15:23 & \begin{tabularx}{0.7\textwidth}{X} 把那肥牛犢牽來宰了,我們來吃喝慶祝; \end{tabularx} \\ \\ \relax
15:24 & \begin{tabularx}{0.7\textwidth}{X} 因為我這個兒子是死而復活,失而復得的。』他們就開始慶祝。 \end{tabularx} \\ \\ \relax
15:25 & \begin{tabularx}{0.7\textwidth}{X} 「那時,大兒子正在田裡。他回來,離家不遠時,聽見奏樂跳舞的聲音, \end{tabularx} \\ \\ \relax
15:26 & \begin{tabularx}{0.7\textwidth}{X} 就叫一個僮僕來,問是甚麼事。 \end{tabularx} \\ \\ \relax
15:27 & \begin{tabularx}{0.7\textwidth}{X} 僮僕對他說:『你弟弟回來了,你父親因為他無災無病地回來,把肥牛犢宰了。』 \end{tabularx} \\ \\ \relax
15:28 & \begin{tabularx}{0.7\textwidth}{X} 大兒子就生氣,不肯進去,他父親出來勸他。 \end{tabularx} \\ \\ \relax
15:29 & \begin{tabularx}{0.7\textwidth}{X} 他對父親說:『你看,我服侍你這麼多年,從來沒有違背過你的命令,而你從來沒有給我一隻小山羊,叫我和朋友們一同快樂。 \end{tabularx} \\ \\ \relax
15:30 & \begin{tabularx}{0.7\textwidth}{X} 但你這個兒子和娼妓吃光了你的財產,他一回來,你倒為他宰了肥牛犢。』 \end{tabularx} \\ \\ \relax
15:31 & \begin{tabularx}{0.7\textwidth}{X} 父親對他說:『兒啊!你常和我同在,我所有的一切都是你的; \end{tabularx} \\ \\ \relax
15:32 & \begin{tabularx}{0.7\textwidth}{X} 可是你這個弟弟是死而復活,失而復得的,所以我們理當歡喜慶祝。』」 \end{tabularx} \\ \\
[1ex]
\hline
\hline
\end{longtable}
$^{1}$頂姐妹平安.
在世界各地我們聚集的頂姐妹平安.
今天我們懷著歡喜快樂.
卻是悲喜交集的心情.
我們一起來聚集.
我們用八福來作為我們敬拜的宗旨.
我們一起來聆聽上帝的話.
我們一起來祈禱.
主系我們獻上我們的敬拜.
在這個特別的日子裡.
我們再一次尋求你的面.
我們在世界不同地方的香港的頂姐妹.
我們一起來藉著敬拜.
我們一起來禱告你.
我們去讚美你.
在這個尚未得熟的世界裡.
我們知道我們仍然要歡喜快樂的來到你面前.
我們卻是紀念到很多受困苦逼迫的人.
求主你藉著這個主日對我們說話.
幫助孩子.
孩子不配戴你自己的說話.
遠遠勝過我的能力.
願意你的靈在當中成為我們全家流淌的幫助.
讓我們生命被更新.
讓我們有能力面對這個世界.
求主你上帝我們來祈禱.
我們會說浪子比喻的下集.
上個月我們說了一篇有關擁抱的講道.
一篇有關路加福音浪子比喻的經文.
講題是首先學習如何被擁抱.
順序說,篇道還沒說完.
今天我們會說完它,說完下集.
下集的題目叫做然後就學習擁抱一切.
所以今天的篇道和上個月的篇道是相關的.
是延續上一次講道的訊息.
如果你沒聽第一篇的話,不要緊,千萬不要關機.
也不要離場,你之後聽回就可以了.
你可以在我們Folk Church的YouTube或Spotify.
找回2022年5月7日的崇拜講道,你聽回.
簡單地說一下我們的上回題.

$^{41}$在上篇講道中,我們重新思考浪子比喻的故事.
我們重新思考小兒子回轉的時間.
小兒子回轉的悔改那一刻可能不是在豬欄.
而是在和他爸爸擁抱的那一刻.
然後我們就去思考我們天父上帝的擁抱.
大家還記得那個金句嗎?.
「明明可以緊緊擁抱,卻以放手作為擁抱的方式來成就更大的擁抱」.
我們發現,特別是父母更加明白.
原來放手也可以是擁抱的方法.
所以最後我們就重新思考天父對我們的擁抱.
天父對我們的擁抱,往往在你未察覺的時候.
天父上帝就已經擁抱了你.
這就是我們上次講道的題目.
上次我們停在第十節.
父親擁抱小兒子.
今天我們會繼續講下去.
簡單地說,我們見到父親擁抱小兒子之後.
聖經記載了一個短暫的大團圓結局.
第二節,父親就吩咐僕人說.
「把那上好的袍子快拿出來給他穿,把戒指戴在他頭上,把鞋穿在他腳上,把那肥牛毒來宰了,我們可以吃喝,富有快樂」.
「因為我這個兒子是死而復活,失而又得的」.
他們就快樂起來了.
一個典型的快樂結局.
原來這麼多年來,大家都怪錯了TVB.
原來BBQ不是TVB發明的.
在2000年前,聖經早已記載BBQ大結局.
我發現路加福音特別強調快樂的結局.
小兒子悔改,父親很快樂.
甚至乎大家宰肥牛,開一個和牛BBQ派對.
一起來吃喝快樂.
聖經記載第十四節,他們就快樂.
他們很快樂.
這個快樂的延續,正正成為浪子比喻兩個段落的轉折.
這個快樂的BBQ,可以說是小兒子故事的結尾.
同時也是大兒子痛苦的開始.
原來大兒子故事正正是起始於這個快樂的延續.
這個快樂的延續背後,正正是大兒子心靈風暴的開始.
聖經說,第25節.
「那時大兒子正在田裡,他回來,離家不遠,聽見作樂跳舞的聲音,便叫一個穀人來,問是甚麼事?穀人說:你兄弟來了,你父親因他無災無病地回來,把牛毒宰了.」.
28節,大兒子卻非常憤怒.

$^{81}$這個正正是我們今天開始下集的故事開展.
經文第28節這樣說.
「大兒子很憤怒,不肯進去,他父親就出來勸他,他父親說:我服侍你這麼多年,從來沒有違背過你的命,你便沒有給我一隻山羊羔,叫我和朋友一同快樂.
但你這個兒子和娼妓,吞進了你的產業,他來了,你倒圍他,宰了肥牛毒.」.
聖經記載,大兒子非常憤怒.
大兒子完全體會不到父親和小兒子一起BBQ的快樂.
甚至當大兒子知道他們兩個在燒牛扒,在茶館的時候,大兒子完全崩潰.
大兒子說:你們可否不要再TVBBQ了?.
大兒子很憎無視.
「我不想再看你們愛回家了,我不要開心速遞.」.
大兒子非常憤怒.
我很明白大兒子的感受,尤其是我這種輕度社交焦慮的人來說.
當一大群人很開心自嗨的時候,當你又不是很分享他們那份開心的時候.
你會感覺到那份憤怒,那份寂寞和難受.
但你不知道如何自恕.
舉個例子,網上有人為浪子比喻這段宰肥牛的經文,作了一首詩歌.
詩歌不是說浪子回頭,而是說父親和小兒子BBQ,宰肥牛唱得很開心的情景.
詩歌的歌名叫做「肥牛毒fat cow party」.
你可以在YouTube搜尋一下,是一首很另類的詩歌.
「肥牛毒fat cow party」這首歌曲很奇怪.
我每次看這首歌,我真的投入不到這首詩歌.
對不起,我無法享受那種敬拜.
肥牛毒fat cow party,我體會不到父親和小兒子那種發落感受.
我反而很能夠明白大兒子那種難受的地方.
當我看到一群人在自嗨,慶祝的時候.
我很明白大兒子那種被淘汰的感覺.
「場景越快樂,大兒子就越痛苦」.
原來,弟子們,一個快樂的延籍,正正是另一個浪子故事的開始.
真的,弟子們,原來耶穌所說的浪子比喻其實有兩個,不是一個.
小兒子的故事固然是浪子的故事.
但在浪子故事的背後.
他更加埋藏了一個沒有人知道,更加複雜,更加棘手.
甚至連經埋藏到腐化發臭的浪子的故事.
如果說浪子比喻是一個離家出走的故事.
遠遠早在小兒子分家產離家出走之前.
大兒子早就已經離家出走了.
大兒子只是將離家出走這件事從家中帶走.
他是一個從家中帶走的浪子,是可以的.
表面上大兒子仍然在家中,但他早就已經不在家中.
我們看看大兒子的說話,29節.

$^{121}$大兒子對父親說,我服侍你多年,從來沒有違背過你的命.
你並沒有給我一隻山羊羔和我和朋友一同快樂.
但你這個兒子和昌記吞盡你的產業.
他來你倒要為他宰了肥牛.
經文裡面最令人關注的是.
大兒子在裡面說了兩次「沒有」這兩個字.
我從來沒有違背過你的命.
你並沒有給我一隻山羊羔.
中文的分析可能不太明顯.
經文裡面,聖經的字不是一個普通的否定字.
而是一個很強的字,「Never」這個字.
「Never, never, double never」.
我們看看這兩個「never」的意思.
第一句「never」,大兒子說從來沒有違背過你的命.
I've never negated your command.
雖然大兒子每天都在家裡面作為父親的長子.
但他沒有一刻覺得自己是父親的兒子.
他覺得自己是甚麼?.
他覺得自己是一個執行命令的奴僕.
大兒子說「我服侍你這麼多年」.
「服侍」這個字是「duneo」.
就是奴隸的意思,做孤離的意思.
他很淒涼,原來大兒子一直都以為自己是一個奴隸.
一個執行命令的奴隸,不是父親的兒子.
大兒子覺得自己的人生只不過是一份工作.
在家裡面打卡,一個work from home的奴隸.
第二個「never」,大兒子說你從來沒有給我一隻山羊羔.
You never give me a goat.
第二句說話,我告訴他.
大兒子覺得自己很貧窮.
甚至比小兒子更加貧窮.
雖然大兒子每天都在極度富有的家裡.
但他覺得身邊沒有一件東西屬於他自己.
大兒子覺得自己一無所有,極度貧窮.
我問問大家,你覺得小兒子比較窮,還是大兒子比較窮?.
我們試想想,小兒子耗盡父親的家產.
最後變得一無所有.
大兒子呢?大兒子連一毫子都沒有花過,就已經一無所有.
小兒子飢餓到在豬欄裡面吃豬哨.
大兒子好像吃了很多東西,但只記得自己連一隻羊都沒有吃.

$^{161}$小兒子醒悟過來,要成為父親的奴僕.
大兒子一直都覺得自己就是家裡的奴僕.
而終於大兒子比小兒子更加貧窮.
因為他連坦誠地說出要離家出走的勇氣都沒有.
世界上最遙遠的距離.
就是每天都在天父的旁邊,他感覺自己一無所有.
所以大兒子沒有辦法去擁抱他的弟弟,不行.
試問一個離家出走的浪子,如何擁抱一個剛回家的人?.
試問一個更加破碎的人,如何擁抱一個破碎被復原的人?.
大兒子比小兒子更加痛苦.
你如何能夠苛求他去擁抱自己的弟弟呢?.
我不知道這是否你生命的處境.
明明知道天父在,卻感覺自己沒有任何力量.
明明知道自己是基督徒,但感覺自己比任何人更加脆弱.
明明知道自己每天都在教會,卻感覺不到有人在理會自己.
然後他要面對著這個荒謬的社會,面對著這個世界.
我知道每個星期,流動崇拜有幾千個頂尖會參與.
包括你,或者陸播中的你.
我知道這幾千人當中,不是全部人都回復教會.
雖然現在我也不知道什麼叫回復教會.
你們是來自西各地,不同城市,不同地方,不同教會的頂尖姐妹.
我知道大部分頂尖姐妹,可能這輩子都沒有機會認識你.
都可能沒有機會見到你.
我想說的是,我只有一個願望.
無論你是什麼原因,什麼背景,什麼動機,你參與流動崇拜.
我只有一個願望.
我希望你能夠參與流動崇拜之後,你不再感覺天賦遙遠.
你回流動崇拜之後,你能夠知道作為基督徒,你如何去面對這個世界.
無論你有回教會還是沒有回教會,回什麼教會.
都是這個願望.
我們一起看看《浪子比喻》中的父親如何回應大兒子的難題.
我們一起看看經文,看看經文中的父親如何回應.
在《老風》第十五章中的最後兩節經文.
31至32節.
31節,父親對他說:兒啊,你常和我同在.
我一切所有的都是你的.
只是你這個弟兄兄弟是死而復活,失而復得的.
所以我們你當歡喜快樂.
總的來說,父親是鼓勵他哥哥去擁抱弟弟.
即是父親給了一個理由去鼓勵這個做哥哥的恩賢快樂來擁抱自己的弟弟.

$^{201}$基本上,父親對大兒子說的這番話沒有什麼特別.
沒有,裡面基本上沒有什麼新的事是說的.
沒有新的資訊,也沒有什麼鮮為人知的秘密.
父親對大兒子說的是他知道的東西.
父親說了一些,俗一點說,誰不知道爸爸是男人的東西.
兒啊,你常和我同在,我一切所有的都是你的.
父親稱呼大兒子為兒啊,在這個原文中的technology是比較親密的意思.
不過我估計大兒子也不會因為這樣而感到驚喜.
父親說你常與我同在,大兒子說我知道,我這麼多年都會奉孝.
父親說我一切所有的都是你的.
大兒子大概會說,是吧.
我不知道,如果你是大兒子,你會否接受父親的理由.
這幾個星期我不斷思考父親這句話.
兒啊,你常和我同在,我一切所有的都是你的.
我一直都stuck在這句經文裡面,不斷想,不斷想.
有空就想,想了好幾個星期.
作為一個目者,對這句話我是很困擾的.
我問,究竟這句話對大兄姐妹來說是有什麼意義.
怎麼說呢,這句話的道理是對的,是正確的,是屬靈的.
每句話都是真的,我們找不到什麼反駁的理由.
但我問,對於大兒子來說,他聽到這句話會有什麼反應.
我意思是,大兒子的問題就是沒有反應.
他就是感受不到自己是父親的兒子.
感受不到自己和父親有份,他不覺得.
所以浪子比喻最後這句話,我很困擾.
不過無論如何,故事就停在這裡.
耶穌的比喻就停在這裡.
耶穌並沒有交代大兒子最後的反應.
但大兒子的結局是怎樣,我們不知道.
他有沒有接受父親的勸告.
他願不願意再一次去擁抱,承認,接受他的弟弟.
我們不知道.
大兒子有沒有真正明白父親這句話背後的意思.
我們不知道.
不過,丁子妹,我們有責任去知道.
聖經裡面,父親這句話背後是什麼意思.
「兒啊,你尚和我同在,我一切所有的,都是你的」.
這句話不是普普通通一句跟大兒子解釋.
你看看你多好,這麼多年不在家.
你看看你多幸福,有冷氣可以上網,可以吃東西.

$^{241}$你經常和我在一起,我的就是你的.
你看看你弟弟多慘,你多好.
你接受他吧,不是的.
如果你這樣去理解父親這句話,你錯了.
如果你有聽我們上一篇講道的時候.
你知道有關天父上帝的擁抱.
重點從來都不是那個位置,是距離.
我記得無論小兒子有多遠.
他仍然在天父的懷抱裡.
所以無論有多近,有多遠.
這不是問題,不關事的.
天父都可以擁抱他.
所以「兒啊,你尚和我同在,我一切所有的,都是你的」.
重點不是大兒子有多近,就在這裡.
你看看你每天都對著我,多幸福,不是這樣的.
這句話有一個更深層次的意義.
意思就是當你面對這個世界.
你只能夠在天父的懷抱裡面去面對這個世界.
「兒啊,你尚和我同在,我一切所有的,都是你的」.
這句話其實是教導大兒子和教導我們.
如何去面對這個世界,如何去理解我們的生命.
正如我上次所說,這篇道是上一篇道的延續.
兩篇的信息是緊緊關聯的.
首先學習如何被擁抱.
然後就學習擁抱一切.
你要先去學習如何處於天父的擁抱裡.
然後就用這個擁抱去學習擁抱任何事情.
作為基督徒,你要擁抱一個人.
擁抱一件事,從來都不是你去擁抱.
而是你在天父上帝的擁抱裡去擁抱他.
不知道大家有沒有發現這個模式.
聖經裡面有很多這樣的模式.
「我們愛,因為神先愛我們」.
「你們要去,因為我已經為你們預備了地方」.
「你們要背十字架,因為基督耶穌釘十字架」.
「我們信,因為耶穌基督的信」.
「我一切所有的,都是你的」.
其實大家可能誤會了父親的意思.
其實父親不是要大兒子擁抱自己的弟弟.
更加不是命令他擁抱自己的弟弟.

$^{281}$父親是要邀請大兒子在父親的懷抱裡去擁抱他的弟弟.
同樣,父親也不是要大兒子自己要歡喜快樂.
命令他一定要快樂.
父親是邀請大兒子和他一起快樂.
參與父親的快樂.
所以父親說「我們」.
「我們,你當歡喜快樂」.
這個快樂是參與天父上帝的快樂.
和快樂的上帝一同快樂.
和擁抱世界的天父一同擁抱世界.
正正是我們快樂的理由.
正正是我們擁抱的理由.
「我一切所有的,都是你的」.
這個就是我們今天想講的擁抱的主題.
回到擁抱的主題.
正如我在這台上定義擁抱的意思.
擁抱是身體的接觸.
它不單只是我們和人的關係.
更加是我們生命的熱度.
擁抱的對象不單是人.
更加是想法和信念.
我們的生命,我們的命運.
它不單只是贊同.
更加是生命的投入.
主動奮身的參與.
擁抱是天父上帝的愛.
我們首先被天父擁抱.
再去擁抱別人.
以及天父的世界.
所以我們學習如何被擁抱.
今天我們學習如何去擁抱.
我們的生命.
我們身處的世界.
一個不太理想的世界.
尤其是今天.
6月4日.
一個特別的日子.
我們沉默,但沒有忘記的日子.
面對著一個不理想的世界.
一個尚未得熟的世界.

$^{321}$我們如何面對這個世界.
可能大家都聽過尼采.
這個哲學家.
最近我在神學院教哲學.
有比較多機會去思考尼采這個人.
雖然他是一個無神論者.
甚至是一個反基督教的人.
不過尼采是一個很有意思的人.
尼采是一個虛無主義者.
尼采認為生命沒有意義.
沒有上帝,沒有尾線.
沒有價值,沒有真理.
正正一個看世界為虛無的人.
尼采提出一個很有意思的說法.
叫做熱愛命運.
Amor Fati.
尼采在《法律的科學》這本書裡.
這樣寫.
熱愛生命.
Amor Fati.
讓這句說話從此成為我的所愛.
我不會向醜陋宣戰.
我不會指控醜陋.
我亦不會指控指控他的人.
無視他們是我唯一的否定.
總而言之我要成為生命的肯定者.
尼采這句說話是什麼意思呢.
既然活著是虛無的.
生命是沒有意義的.
那你的生命唯一能夠擁有的是什麼.
就是你的生命.
所以尼采說.
Amor Fati.
熱愛生命.
擁抱命運.
無論你所遭遇的是多麼的醜惡.
痛苦.
黑暗絕望都好.
你通通都成為生命的擁抱對象.
簡單來說.

$^{361}$尼采的Amor Fati是一種貪吃蛇的概念.
如果你不知道什麼是貪吃蛇就算了.
就是生命的意志.
生存的意志.
活著的意志是你唯一的意志.
世界的一切.
無論多麼的糾纏.
都成為你生命擁抱的養分.
所以尼采另一句很有名的名句.
凡殺不死我的.
必使我更強大.
Was mich nicht umbringt.
Macht mich stärker.
是一個很有霸氣的人.
我想說的是.
當我讀尼采的時候.
一個主張虛無主義的尼采.
都可以擁抱命運.
我們作為基督徒.
深知天父擁抱這個世界.
我們豈能夠不可以去強力擁抱這個世界嗎.
去擁抱一些我們遭遇的命運嗎.
真的.
面對這個世界.
擁抱是一種霸氣.
首先學習如何被擁抱.
然後去學習擁抱一切.
不單單是接受.
更加懷著強大力量.
快樂地去面對.
容納它.
吃掉它.
說到這種霸氣.
我就想起周星馳電影裡.
功夫的結尾.
周星馳大戰火雲邪神.
當周星馳在半空出現了這個妖神掌.
壓倒性地打贏了火雲邪神之後.
他最後說了一句話.
一句很淡然的一句話.

$^{401}$你想學嗎.
我教你吧.
快樂地擁抱是一種極大的霸氣.
廣東話有一句話叫做.
我擁抱你.
你是不是擁抱他.
我擁抱他.
一句很有型的說話.
一句很有份量的說話.
不單只是擁抱.
更加是毫無保留.
全情投入.
奮心地去擁護他.
正正因為天父上帝擁抱.
我們可以去擁抱這個不太理想的世界.
和裡面的人.
沒錯.
擁抱是危險的.
擁抱是極之危險的.
可能我是一個身處80年代的人.
作為一個看著80,90年代的港產片長大的我來說.
當你看到擁抱這個字.
我覺得是一件很危險的事.
不知道同齡人明白我說什麼嗎.
因為80,90年代的港產片.
當一個人擁抱一個人的時候.
十次又五次.
都是有不好的結果的.
想想周潤發.
穿著一件大袍.
去擁抱一個人的時候.
接著會怎樣呢.
吳宇森那些人.
在肚子裡射了幾槍.
或者一把刀.
就是那些樣子.
十次又五次擁抱都是不好的.
擁抱是一件危險的事情.
因為你躺在身上.
或者你抱著一個你不陌生的人.

$^{441}$不過我們作為基督徒.
一群知道天父上帝擁抱的人.
豈不是全世界裡最有本錢.
去擁抱一切的人呢.
因為天父擁抱世界.
我們可以擁抱天父所擁抱的世界.
我們能夠擁抱我們面對的命運.
不知道大家認不認識這位包山王.
就是黎智偉.
黎智偉是一名香港攀山運動員.
有亞洲攀山王之稱.
曾經參加攀石巡迴賽.
全球排名第八.
以前在長洲搶包山裡面.
獲冠軍 稱為長洲包山王.
2011年有一天.
他在屯門公路開電單車.
突然後面有一輛超速的私家車撞到他.
撞到他整個人躺下.
後來發現是醉駕.
不幸的是同時間.
當他躺在公路的時候.
有另一輛超速的私家車.
剪到他身上.
壓傷了他的脊椎.
這次的交通意外.
令黎智偉從此半身癱瘓.
終身要成為一個殘疾人士.
我不知道你面對這個厄運.
你現在面對的世界.
你的命運是不是和黎智偉差不多.
2011年的時候黎智偉只有29歲.
他從來沒有想過.
全球排名第八的攀山好手.
這個身份只能夠維持到29歲.
29歲之後他要終日面對著一部輪椅.
成為一個半身癱瘓的人.
一輩子要面對著這個冷冰冰的輪椅.
流過無數的眼淚.
經歷很多很多沒有人能夠感受的痛苦.

$^{481}$最後黎智偉決定擁抱他的命運.
有一幅照片很震撼我.
幾年之後黎智偉決定擁抱他自己坐的輪椅.
抱著他去爬山.
抱著一個好像注定他一生要怎麼走的輪椅.
去爬上他喜愛的攀石的山.
擁抱輪椅爬上去.
2016年他坐在輪椅上爬上獅子山頂.
所以一套戲叫《獅子山上》.
是講他的故事.
2021年他挑戰攀登新界最高的大廈.
如心廣場.
仍然抱著輪椅爬上去.
黎智偉活著是一份恩典.
生命是一份恩典.
因為天父上帝擁抱著這個世界.
我們就可以去擁抱這個世界裡面的任何事情.
我不知道你現在面對什麼事情.
可能是你家人患病.
可能你自己很辛苦.
可能面對著很困難的關係.
可能是面對著香港這個困局.
不用怕,不用縮,不用哭,不要避,不要怯.
勇敢快樂,霸氣地去擁抱它.
我們沒有力量去擁抱我們面前的厄運.
但是天父擁抱這個世界.
擁抱這個世界同時去擁抱我們.
都叫我們有力量去擁抱他擁抱的這個世界.
我一切所有的都是你的.
上帝愛世界.
如果你有看我前幾天要致見的靈修.
我說上帝愛世界.
神愛世人的一句話.
上帝愛世界.
相信任何的事情都不能逃避這句話.
無論是罪或惡.
雖然這些根源和上帝沒有關係也好.
但是沒有東西能夠避免上帝愛世界這個事實.
沒有東西能夠避開上帝愛世界這個事實.
十字架的上帝是一個快樂的上帝.

$^{521}$因為祂戰勝了黑暗.
快樂戰勝了哀號.
笑容戰勝了哭號.
擁抱世界的上帝戰勝了世間的苦痛.
戰勝了六月四日的黑暗.
正正因為快樂的上帝掌權.
我們能夠快樂地去面對世間很多的憂愁.
不是莫記憂與愁要快樂上歌唱.
念記憂與愁我們卻能夠快樂上歌唱.
擁抱這個世界成為一種我們基督徒生存裡面的見證.
我一直祈禱.
在你慰勞我們當中軟弱的頂尖裡面我們祈禱過你.
求主你擁抱我們.
讓我們有力量能夠面對很多我們不能夠面對的憂暗.
讓我們能夠有力量去擁抱我們所遭遇的厄運.
我們記得你擁抱我們.
你更加擁抱這個世界.
讓我們能夠勇敢快樂地去擁抱他們.
我們不能夠.
但在你的懷抱裡面我們仍然可以做到.
求主你紀念我們Fold Church裡面的群體.
無論在不同地方我們一起去藉著今天我們再一次在你裡面得力.
因為你擁抱我們.
多謝您的收聽,我們下次見..
\newpage



\section{約翰福音 13:1-38-20220611}
\label{sec:SLTU9ILLofg}
\textbf{【網上崇拜】攬到底|約翰福音13\_1-38|20220611 [SLTU9ILLofg]}
\newline
\newline
連結: \href{https://youtube.com/watch?v=SLTU9ILLofg}{\texttt{ https://youtube.com/watch?v=SLTU9ILLofg}} ~~~~ 語音日期: 2022-06-11 
\newline
\newline
\hyperref[sec:NkL6a2IH8uY]{\small{< < < PREV SERMON < < <}}
~
\hyperref[sec:index_chronic]{\small{[返順時目]}}
~
\hyperref[sec:index_scriptual]{\small{[返順卷目]}}
~
\hyperref[sec:rvuxp0Bes90]{\small{> > > NEXT SERMON > > >}}
\newline
\newline
約翰福音 13:1-38-20220611
\newline
\begin{longtable}{cl}
\hline
\hline
章節 & 經文 (和合本修訂版)\\
\hline
13:1 & \begin{tabularx}{0.7\textwidth}{X} 逾越節以前,耶穌知道自己離世歸父的時候到了。他一向愛世間屬自己的人,就愛他們到底。 \end{tabularx} \\ \\ \relax
13:2 & \begin{tabularx}{0.7\textwidth}{X} 晚餐的時候,魔鬼已把出賣耶穌的意思放在加略人西門的兒子猶大心裡。 \end{tabularx} \\ \\ \relax
13:3 & \begin{tabularx}{0.7\textwidth}{X} 耶穌知道父已把萬有交在他手裡,且知道自己是從神出來的,又要回到神那裡去, \end{tabularx} \\ \\ \relax
13:4 & \begin{tabularx}{0.7\textwidth}{X} 就離席站起來,脫了衣服,拿一條手巾束腰, \end{tabularx} \\ \\ \relax
13:5 & \begin{tabularx}{0.7\textwidth}{X} 隨後把水倒在盆裡,開始洗門徒的腳,並用束腰的手巾擦乾。 \end{tabularx} \\ \\ \relax
13:6 & \begin{tabularx}{0.7\textwidth}{X} 到了西門‧彼得跟前,彼得對他說:「主啊,你洗我的腳嗎?」 \end{tabularx} \\ \\ \relax
13:7 & \begin{tabularx}{0.7\textwidth}{X} 耶穌回答他說:「我所做的,你現在不知道,但以後會明白。」 \end{tabularx} \\ \\ \relax
13:8 & \begin{tabularx}{0.7\textwidth}{X} 彼得對他說:「你絕對不可以洗我的腳!」耶穌回答他:「我若不洗你,你就與我無份了。」 \end{tabularx} \\ \\ \relax
13:9 & \begin{tabularx}{0.7\textwidth}{X} 西門‧彼得對他說:「主啊,不僅是我的腳,連手和頭也要洗!」 \end{tabularx} \\ \\ \relax
13:10 & \begin{tabularx}{0.7\textwidth}{X} 耶穌對他說:「凡洗過澡的人不需要再洗,只要把腳一洗,全身就乾淨了。你們是乾淨的,然而不都是乾淨的。」 \end{tabularx} \\ \\ \relax
13:11 & \begin{tabularx}{0.7\textwidth}{X} 耶穌已知道要出賣他的是誰,因此說「你們不都是乾淨的」。 \end{tabularx} \\ \\ \relax
13:12 & \begin{tabularx}{0.7\textwidth}{X} 耶穌洗完了他們的腳,就穿上衣服,又坐下,對他們說:「我為你們所做的,你們明白嗎? \end{tabularx} \\ \\ \relax
13:13 & \begin{tabularx}{0.7\textwidth}{X} 你們稱呼我老師,稱呼我主,你們說的不錯,我本來就是。 \end{tabularx} \\ \\ \relax
13:14 & \begin{tabularx}{0.7\textwidth}{X} 我是你們的主,你們的老師,尚且洗你們的腳,你們也應當彼此洗腳。 \end{tabularx} \\ \\ \relax
13:15 & \begin{tabularx}{0.7\textwidth}{X} 我給你們作了榜樣,為要你們照著我為你們所做的去做。 \end{tabularx} \\ \\ \relax
13:16 & \begin{tabularx}{0.7\textwidth}{X} 我實實在在地告訴你們,僕人不大於主人;奉差的人也不大於差他的人。 \end{tabularx} \\ \\ \relax
13:17 & \begin{tabularx}{0.7\textwidth}{X} 你們既知道這些事,若是去實行就有福了。 \end{tabularx} \\ \\ \relax
13:18 & \begin{tabularx}{0.7\textwidth}{X} 我不是指著你們眾人說的,我知道我所揀選的是誰;但是要應驗經上的話:『吃我飯的人用腳踢我。』 \end{tabularx} \\ \\ \relax
13:19 & \begin{tabularx}{0.7\textwidth}{X} 事情還沒有發生,我現在先告訴你們,讓你們到事情發生的時候好信我就是那位。 \end{tabularx} \\ \\ \relax
13:20 & \begin{tabularx}{0.7\textwidth}{X} 我實實在在地告訴你們,接納我所差遣的就是接納我;接納我的就是接納差遣我的那位。」 \end{tabularx} \\ \\ \relax
13:21 & \begin{tabularx}{0.7\textwidth}{X} 耶穌說了這些話,心裡憂愁,於是明確地說:「我實實在在地告訴你們,你們中間有一個人要出賣我。」 \end{tabularx} \\ \\ \relax
13:22 & \begin{tabularx}{0.7\textwidth}{X} 門徒彼此相看,猜不出他說的是誰。 \end{tabularx} \\ \\ \relax
13:23 & \begin{tabularx}{0.7\textwidth}{X} 門徒中有一個人,是耶穌所愛的,側身挨近耶穌的胸懷。 \end{tabularx} \\ \\ \relax
13:24 & \begin{tabularx}{0.7\textwidth}{X} 西門‧彼得就對這個人示意,要問耶穌是指著誰說的。 \end{tabularx} \\ \\ \relax
13:25 & \begin{tabularx}{0.7\textwidth}{X} 於是那人緊靠著耶穌的胸膛,問他:「主啊,是誰呢?」 \end{tabularx} \\ \\ \relax
13:26 & \begin{tabularx}{0.7\textwidth}{X} 耶穌回答:「我蘸一點餅給誰,就是誰。」耶穌就蘸了一點餅,遞給加略人西門的兒子猶大。 \end{tabularx} \\ \\ \relax
13:27 & \begin{tabularx}{0.7\textwidth}{X} 他接了那餅以後,撒但就進入他的心。於是耶穌對他說:「你要做的,快做吧!」 \end{tabularx} \\ \\ \relax
13:28 & \begin{tabularx}{0.7\textwidth}{X} 同席的人沒有一個知道耶穌為甚麼對他說這話。 \end{tabularx} \\ \\ \relax
13:29 & \begin{tabularx}{0.7\textwidth}{X} 有人因猶大管錢囊,以為耶穌是對他說「你去買我們過節所需要的東西」,或是叫他拿些甚麼給窮人。 \end{tabularx} \\ \\ \relax
13:30 & \begin{tabularx}{0.7\textwidth}{X} 猶大受了那點餅以後立刻出去。那時候是夜間了。 \end{tabularx} \\ \\ \relax
13:31 & \begin{tabularx}{0.7\textwidth}{X} 猶大出去後,耶穌說:「如今人子得了榮耀,神在人子身上也得了榮耀。 \end{tabularx} \\ \\ \relax
13:32 & \begin{tabularx}{0.7\textwidth}{X} 如果神因人子得了榮耀,神也要因自己榮耀人子,並且要立刻榮耀他。 \end{tabularx} \\ \\ \relax
13:33 & \begin{tabularx}{0.7\textwidth}{X} 孩子們!我與你們同在的時候不多了;你們會找我,但我所去的地方,你們不能去。這話我曾對猶太人說過,現在也照樣對你們說。 \end{tabularx} \\ \\ \relax
13:34 & \begin{tabularx}{0.7\textwidth}{X} 我賜給你們一條新命令,乃是叫你們彼此相愛;我怎樣愛你們,你們也要怎樣彼此相愛。 \end{tabularx} \\ \\ \relax
13:35 & \begin{tabularx}{0.7\textwidth}{X} 你們若彼此相愛,眾人因此就認出你們是我的門徒了。」 \end{tabularx} \\ \\ \relax
13:36 & \begin{tabularx}{0.7\textwidth}{X} 西門‧彼得問耶穌:「主啊,你去哪裡?」耶穌回答:「我所去的地方,你現在不能跟我去,以後卻要跟我去。」 \end{tabularx} \\ \\ \relax
13:37 & \begin{tabularx}{0.7\textwidth}{X} 彼得對他說:「主啊,為甚麼我現在不能跟你去?我願意為你捨命。」 \end{tabularx} \\ \\ \relax
13:38 & \begin{tabularx}{0.7\textwidth}{X} 耶穌回答:「你願意為我捨命嗎?我實實在在地告訴你,雞叫以前,你要三次不認我。」 \end{tabularx} \\ \\
[1ex]
\hline
\hline
\end{longtable}
$^{1}$頂姐妹平安.
願在不同地區的頂姐妹在崇拜當中經歷上帝.
在剛才的詩歌當中.
我相信你也感受到一種療癒.
特別是恢復我自己一首很喜歡的詩歌.
特別是在孤單懼怕的時候.
還有在過程中有不絕的嘆息.
感受到那種安慰.
今天選擇的經文是.
約翰福音13章.
關於耶穌在雨戰晚餐所作的事發.
選擇經文的過程中.
其實會不斷回想起.
童恭之前說的擁抱的訊息.
打火鍋,大吃大餐,出腳趾.
不知道你之前有沒有聽過擁抱的訊息.
上星期又說抱抱.
霸氣地抱.
在我自己想第二篇講章的時候.
在未配合或未聽到其他童恭說的訊息的時候.
這段經文是我一直都想講的.
正正就是這一節的經文.
就是13章第一節的經文.
我自己18歲才信耶穌.
我不知道你的信主的歲數是多久.
或者幾多歲的信主.
我18歲才信耶穌.
我之前對基督教信仰或基督信仰是零的.
對我來說沒有回過教會.
也沒有人帶我回過教會.
但中間也會有一兩次參加教會的活動.
每年參加教會活動一次.
那個叫青年主日.
可能對電影節目也不陌生.
青年主日怎麼會回去呢.
其實青年主日我是參加崇拜.
我不是很投入.
那天也不知道要做什麼.
我在等崇拜報告的時候就上台拿獎.
拿什麼獎呢.

$^{41}$不是我一個人拿的.
是整個團隊人員拿的.
因為我中四的時候幫一間教會打籃球隊.
那時候有教會盃.
簡單來說我是90年代初的康體士工的福音報道的種子.
可能對你們來說已經很遠了.
因為現在也沒有康體士工.
因為很多教會覺得沒有什麼果效就接了這個士工.
因為湊了很多球隊最後也沒有什麼人信主.
又很消遣.
最後不如就接了他.
但我自己在康體士工運作幾年的球隊.
我每年都有一次回教會就是青少年主日.
下午就是打球.
跟教會的人打球.
在當中又叫他們認識一下.
但去到18歲我自己有個經歷之後就回了教會.
我開始看課本書.
人家說你開始看聖經了.
回教會.
我看書沒什麼難度.
因為那時候也是在讀書.
所以就照樣當作是試書.
我從馬太福音開始看.
看馬可.
然後就是約翰.
然後就是路加.
因為人家說約翰遲一點再看.
你剛剛初信是看不明白的.
只是字而已有多難明白.
我心想閱讀理解這些東西難不倒我.
那時候很囂張.
現在也囂張了.
最後看到約翰福音的時候.
看到這一節經文我就停了下來.
我就覺得.
預節以前耶穌知道自己離世歸附的時候到了.
祂既然愛世間屬自己人.
就愛他們到底.
這句話我停了很久.

$^{81}$剛剛初信的時候看.
為什麼這麼怪呢.
狀態其實應該都.
為什麼會這樣做呢.
知道自己快不行的話.
就應該做自己喜歡做的事.
是不是.
如果我覺得自己快不行的話.
我還會做這些事.
不如我想自己做想做的事.
但我就覺得.
如果真的想代入耶穌.
或者我自己覺得要做自己想做的事.
耶穌反而選擇做一個.
就是愛自己喜歡的人.
愛到底.
我覺得這個真的很厲害.
常常都說耶穌是很愛人.
耶穌的愛是很大愛的話.
這段經文是我第一次看的時候覺得.
如果知道自己在世日子不多.
祂就盡祂的日子去愛祂喜歡的人.
愛到底.
我覺得.
我那一刻就第一次感受到.
原來耶穌的愛.
是在祂有限的時間當中.
愛到盡.
這句經文對我來說.
我做基督徒來說.
是一個指標性的教導.
不知道大家看這段經文對你來說.
有什麼啟發.
或者對你有什麼影響.
但對我來說.
這個愛是一個.
很不容易.
很遠.
很遙不可及.
但這個是對我來說.

$^{121}$是耶穌愛.
一個很重要的具體的經文表達.
但我慢慢理解的時候.
約翰福音其實.
約翰福音的編排是一個很有.
很有深度慢慢進程的.
去讓我們明白.
約翰福音第一章一開頭就是.
「到成了肉身住在我們中間」.
「充充滿滿的有恩典有真理」.
「我們也見過他的榮光」.
「正是負獨生子的榮光」.
我們.
當別人說上帝愛有多厲害.
他就說這個愛厲害到.
他來到我們中間.
而這個愛不是傳的.
是自己獨生子.
在當中去展現給你看.
約翰福音帶出的訊息就是.
那個愛不是遙不可及.
那個愛也不是你聯想.
或者是推敲出來.
那個愛就是住在我們中間.
而在當中你會感受有恩典.
有真理.
約翰正正就是一個見證人.
將耶穌的見證寫出來.
我們看約翰福音.
不知道大家的經歷是怎樣.
約翰福音的記載很多時候都是獨特的.
他是一個主觀視覺.
靠近身體過程當中.
將我認識的主耶穌是怎樣.
我就呈現在你面前.
因為他真的住在我們中間.
充充滿面的有恩典有真理.
約翰福音第十三章是一個特別驚悚的開始.
其實約翰福音第一至十二章.
其實都是耶穌很重要的七個神蹟的表現出來.

$^{161}$由他在加拿大婚宴將水變酒.
去到第十一章.
使拉撒復活.
就說了一個很重要的宣告.
就是「復活在我,生命也在我,.
信我的人雖然死了也必復活」.
一個能夠變質的上帝.
和一個生命在掌握當中的上帝.
這七個神蹟就展現給世人看到.
就是記載在頭十二章的經文.
去到第十三章的經文的時候.
約翰的記錄轉了.
他就將耶穌怎樣身體力行.
將他的愛具體讓受眾去明白.
特別是他所愛的那群門生去明白.
所以去到第二節的時候.
經文是說一件事.
就是很大一段字.
雖然你不用怕.
你沒有腦花.
這些字是小的.
因為很熟悉.
我也不需要特別去跟大家讀.
我只不過是將一個段落跟大家說.
在第二節到第十一節這個段落的時候.
就是大家很熟悉的洗腳的經文.
在洗腳的經文的時候.
細節大家都記得.
但我有一節經文跟大家去處理.
彼得就跟主耶穌說.
「主啊!你真的洗我的腳?」.
耶穌就回答.
「我所做的你如今不知道.
後來必明白」.
不知道大家對於科幻書裡面記載.
特別是約翰福音.
耶穌經常說「我時候未到」.
如果我是一群門生.
你經常說「未到」.
那什麼時候才到呢?.

$^{201}$但你做的事我又不明白.
但你又做.
其實很難過的日子.
我不知道你在平時跟別人相處.
或者工作的時候.
你不明白的時候.
你怎樣處理.
我教兩個兒子.
不明白就問吧.
你可以自己試試做.
但你不懂的時候問別人.
但耶穌的答案就是.
「我所做的你如今不明白.
將來才明白」.
將來是什麼時候呢?.
對於我們現在這個年代.
很多時候要即時有答案.
我不知道大家找答案的方法是怎樣.
通常我認識的.
不懂就馬上問Google大神.
你不會問百度.
你當然問Google.
如果你不喜歡Google.
你當然問Siri.
都是問的.
耶穌說「我現在做的事.
你如今不知道.
你將來才知道」.
我就問將來是什麼時候呢?.
如果那件事對我重要的話.
如果那件事對我有幫助的話.
為什麼你現在不坦白.
清清楚楚說清楚.
為什麼要將來才知道呢?.
我當時看經文就覺得.
耶穌很喜歡賣弄關子.
或者很喜歡Print Buy.
以前小時候讀經.
就好像追連續劇.
追小說.

$^{241}$看他什麼時候會說.
然後不斷追.
看他最後都沒有說.
沒有說的時候你會覺得.
耶穌騙人.
其實將來什麼時候才會明白呢?.
但是你大了.
了解多了.
聖經就不是這樣.
用對答方式寫.
但是耶穌說你將來知道.
你不明白.
但是他會示範給你看.
對於你來說.
我們信仰有多少東西.
是要身體力行呢?.
這個都是常常我提醒自己的.
不要做現今的化彌賽仁.
一個Lip Service.
你的口是讚美神.
你的口是稱頌神.
你的口是愛主.
但是你的心就沒有做.
你的行為又看不到.
信仰和行為的相符.
是上帝給我們一個很重要的提示.
對我們來說.
耶穌不是單單說.
他告訴你這件事是重要的.
將來你會明白.
但是他做一次給你看.
在經文下去的時候.
他做了.
他做了一件事.
就是他洗完腳之後.
他穿回衣服.
接著他就解釋了.
他解釋了.
洗完腳穿回衣服.
坐下就跟他們說.

$^{281}$做完這個服侍的動作.
我向你們所做的.
你們現在明白嗎.
你們說我.
你稱呼我為夫子.
稱呼我為主.
你們說的是沒錯.
我本來就是.
我是你們的主.
是你們的夫子.
都尚且洗你們的腳.
你們也當給主洗腳.
我給你們作了榜樣.
是叫你們照我.
我向你們所做的去做.
耶穌去示範了一個謙卑.
服侍他人.
不計較自己身份的愛的行徑.
對於當時的門徒來說.
是不容易接受的.
如果你認識新約的背景.
或者認識當時一個蓄勞的制度.
就是可以養奴隸的制度.
其實都不一定要洗腳的.
你會看到在加利利.
一些記載.
或者在科目書裡面記載.
是在門口裡面有幾缸水.
是人們進屋之前自己洗一洗.
就進去了.
都不是有一個專人跟你洗腳的.
但是耶穌.
他很認真跟他說.
我現在告訴你.
如果你覺得.
不是,如果你稱呼我做夫子.
是老師的時候.
我都願意放下這個身段.
去服侍你的時候.
其實你們都應該要學習.

$^{321}$放下身段去做這件事.
當我們說愛的時候.
很多時候都有計較.
當我們說愛的時候.
很多時候都有些條件.
我為什麼要這樣做?.
這句話.
很多時候在處理爭執的時候.
就出現了.
他這樣做.
我為什麼要這樣做?.
我為什麼要這樣做.
要有一個答案.
有一個原因.
因為中間有爭執.
耶穌提醒了一個東西.
就是你願意服侍.
你願意去愛人的時候.
你不是計較你的身份.
也不是計較你的地位.
有時候就是你願不願意去做.
你覺得要做在那個人身上.
這個其實是很難.
我相信你自己經歷過.
或者你自己都見過一些.
不可愛的人.
你不願意去做.
但是對我們來說.
耶穌知道當中有一個人是賣祂.
祂都仍然照做.
這段經文裡也說.
中間有人賣祂.
但是耶穌沒有說洗到那個.
不好意思我拿了東西.
然後回頭又忘記了.
不是.
耶穌都會做.
在過程當中.
有很多不同的計算.
有很多不同的考慮.

$^{361}$在愛當中有很多前設的條件.
我今年有兩個婚禮講.
講了一個.
接下來九月都會講第二個.
兩對都是神學院的學生.
找我做婚前輔導.
我也跟他們提醒一件事.
就是進入一個新階段的時候.
就是一個新愛的表達的開始.
因為已經是成為夫婦了.
愛要昇華.
而在當中也是挑戰對愛行動的指標.
當我們讀誓詞.
或者你參與過婚禮的時候.
這個月我認識很多人都結婚了.
你都不陌生的聽誓詞.
或者你在哪一個堂會聽誓詞.
都有類近的.
無論什麼什麼.
其實你留心一下.
講的誓詞大同小異的狀況.
但前提是無論什麼都好.
我都會All in.
我都會愛你.
即是他講越多條件都好.
講得多少條件都好.
他都說那些不是條件.
我都會All in.
即是講得越多condition.
那些都不是condition.
因為我給你是一個unconditional love.
你明白我的意思嗎.
即是當我們說要很多條件的時候.
我為什麼要這樣做.
就是條件的時候.
但耶穌告訴我們.
你的愛可不可以無條件呢.
不容易的.
真的不容易.
但耶穌提醒他們.

$^{401}$我做了一次給你看.
你說我有這樣的身份.
我本來就有這樣的身份.
但我將這個身份都可以不要.
愛你你原顧我.
我All in.
這個就是他既然愛世間屬自己的人.
就愛他們到底.
Unconditional.
你越講得多條件都好.
我都說那個都不是我的條件.
他照做.
但他希望什麼.
我給你們作榜樣.
叫你們照著我.
向你們所做的去做.
這個是很難.
連這個難.
耶穌已經告訴你.
你現在做不到.
將來你會明白.
耶穌不是強我們所難.
他知道我們軟弱.
他知道我們現在做不到.
但你將來會明白.
但你要學習去照做.
接著到下一段經文.
下一段經文背景是.
加里蘭猶大去賣耶穌.
去到21節.
耶穌講了這個話.
很憂愁.
告訴他們.
我實實在在.
中間有一個人要賣我.
這一段的描述是.
猶大去賣耶穌.
他在講中間的事情.
那些門生就在中間穿插.
問我是不是我.

$^{441}$於是就搞了一輪.
當我自己看這段經文的時候.
我覺得其實耶穌都知道.
為什麼還要安裝他.
為什麼不中間指使二五仔.
因為看很多電影都是這樣.
大佬的時候說到齊了.
一講到齊的時候.
大佬到齊了.
這樣就齊了.
不是黑社會.
可能太多.
接著就說.
接著就想找二五仔出來.
但耶穌又不是用這個方式.
去處理這件事情.
對我來說.
其實耶穌在等什麼呢.
耶穌不是不知道.
但耶穌在想什麼呢.
對你來說.
耶穌是不是傻子呢.
耶穌是不是被人擺了之後.
他都覺得懵然不知呢.
我自己在過去這二十年.
跟頂姐妹相處.
穆妹的時候.
我自己很感受一件事.
就是處理一些感情.
或者小組之間的關係.
總會有很多角力.
會有不同的約見.
聽不同的話.
聽不同的故事.
對我來說.
當聽單方面的時候.
那個一定是副主.
但聽多方的時候.
就會看到圖畫的立體.
慢慢來久而久之.

$^{481}$會發現有些人會借你.
去傳說話出去.
即是借你過橋.
但當你處理得多的時候.
你其實是知道他在借你過橋.
當你有人生閱歷.
當你處理得多這些糾紛的時候.
你知道有些人會借你過橋.
我都被學生問過.
或者被弟兄姊妹問過.
潘Sir你覺得他真的借你過橋嗎?.
我說我沒有那麼厲害.
我未必即時知道.
但就算我知道.
我都會在當中協調.
你不需要認同我的做法.
但我自己的做法就是.
不是我不知道他借我過橋.
是他不知道他自己在做什麼.
你明白我的意思嗎?.
有時就是有很多人.
以為自己很聰明.
或者自己有方法.
可以有不同的搬弄.
但我的存在就是想平息.
那個紛爭.
彌補之間的差異.
如果我在當中仍然可以做這個調配的話.
我仍然會做一件事.
哪怕你覺得好像被人搬弄了.
或者借我過橋的時候.
就好像主耶穌不是不知道.
猶大在做什麼.
只不過祂不知道.
「父啊,赦免他們,因為他們所作的,他們不曉得」.
上帝仍然留一線空間.
讓他做迴轉.
上帝仍然留一線空間讓他明白到.
其實你不需要走這條路.
其實這群人有另一條路可以走.

$^{521}$為什麼你要選擇走那條路呢?.
對於猶大來說.
他是自己選擇走那條路.
不知道你怎麼看擁抱.
上星期John講到的時候.
他那個年代其實跟我差不多.
他看完那個年代的電影.
就是抱著也沒有好結果.
他還很型型色色地.
大家還是有印象.
在抱這個過程當中.
路加福音描述得很好.
路加福音25章描述猶大出賣耶穌的時候.
帶著一群小孩去出賣耶穌的時候.
他抱著耶穌親嘴.
耶穌跟他說.
「猶大,你是要用親嘴來出賣我嗎?」.
他是抱著耶穌.
親他,出賣他.
你以為耶穌不知道誰在抱他嗎?.
但耶穌的回應是.
「你照你的做吧」.
我上星期報告的時候都說.
其實真的要抱一個人是很不容易的.
當你知道你抱的那個是他會出賣你的.
你就明白到無論什麼條件都好.
其實我都想愛你的.
只不過你不知道你是被愛.
而你仍然選擇.
走你自己覺得不正確的路.
親兄弟姐妹很多時候我們在面對弟兄姐妹之間的爭拗.
面對小組,面對同工當中有很多不同的意見.
是不容易處理的.
但我希望你試下多想一步.
其實很多時候是意見上的分歧.
大家試下再調解一下.
大家再試下想一下對方在想什麼.
不要有太多,太快排他就覺得沒可能.
就覺得這樣就要分開.
去到經文的尾段的時候.

$^{561}$就去到第31節.
這個經文很有趣.
他是這樣的.
第31節.
他寄出去,他是誰?就是猶大.
他寄出去,耶穌就說.
如今人子得了榮耀,上帝在人子身上也得了榮耀.
上帝要因自己的榮耀人子.
並且要快快地榮耀他.
這個就是猶大去做他要做的事.
上帝的榮耀就是要把耶穌帶到他山上.
承擔眾人的罪.
但是耶穌說真正的話.
真正的話是什麼?就是這一句.
「我賜給你們一條生命令乃是叫你們彼此相愛」.
「我怎樣愛你們,你們也要怎樣相愛」.
「你們也要彼此相愛的心」.
「因此就認出你們是我的門徒」.
剛才的經文就是他寄出去.
猶大仍然選擇按他自己的路走.
他是選擇離開耶穌.
走了.
耶穌就說了一個很重要的話給他所愛的人聽.
就是「我給你一條生命令」.
「這條命令就是叫你們彼此相愛」.
對我自己來說是一個常常提醒自己的話.
無論頂姐妹有多少嚴規也好.
無論頂姐妹有多少意見平衡不了也好.
我都希望大家學習停一停.
將那些事件可能攤分一點慢慢談.
總會有處理的空間的.
但原則就是不要反臉不成仁.
讓大家再沒有相見.
因為我們跟外面坊間的人不同.
我們仍然是主內的弟兄姊妹.
仍然覺得「零零頭」是不容易做的.
有些事情不是說我三言兩語你就會明白到.
有些困難.
但上帝的聖子耶穌在地上最後的說話告訴我們.
「這條生命令就是叫我們彼此相愛」.

$^{601}$我們能夠有彼此相愛的心.
人們就認得出我們的愛是一個無條件的愛.
我們的愛是為弟兄姊妹彼此相容.
我們學習彼此一起去理行耶穌基督.
給我們身體去行那種愛.
當我們要擁抱一個人的時候.
其實就有很多不同的考慮.
但主耶穌提醒我們.
在擁抱的過程當中.
你知道他是你的弟兄,他是你的姊妹.
我們是希望大家能夠互相去平衡.
互相去共融.
這是最重要的.
通常去到一些比較新的命令.
New Order的時候.
科幻書中有一個人常常跑出來.
一馬當先的那個就叫彼得.
彼得說了一句話.
主你去哪裡我都會跟你去.
耶穌就跟他說.
我現在去的地方你不能去.
但後來會去.
常常都想吊著彼得尾.
不要那麼衝動.
先等一下.
上帝常常要我們熟能功課要學習等待.
不要太快.
後來下一本書你就會明白到.
快快的聽慢慢的說.
那種需要沉澱需要思考.
彼得就說.
主啊我為什麼現在不可以跟你去呢.
我願意為你寫命.
耶穌很多時候都會跟他禮贈.
你真的會為我寫命.
寫命是什麼.
寫命是無條件的.
寫命是All in.
你是不是真的會為我All in.
我相信今天我們經歷了這幾年.

$^{641}$All in其實越來越難.
人生當中有很多抉擇.
有很多要考慮的東西.
對你來說真的要All in的是什麼呢.
我寫這篇講章的時候.
或者我構思今個月第二篇講章的時候.
其實都有很多經文在當中穿插.
穿插期間要思考一下.
想不到就要回覆message.
其中一封message有些東西要跟同工交代的時候.
我有同工這期很少說話.
通常只會發sticker給我.
通常是ok的sticker.
不是ok的.
是一個人的.
是人形貼紙的sticker.
因為他喜歡那部電影.
所以他發的那部電影的sticker.
都是由男主角給我的.
他說潘先生你一定要看這部電影.
我說我會看電影的.
我剛剛這個星期二晚就教完我今年的晚間課程.
我星期二晚的課程叫交友與戀愛.
我教完了.
交友與戀愛的課程功課.
人們都很奇密.
這些要做功課的嗎?.
哦是的.
我的功課就要寫一些個案分析.
我有三套電影可以讓他們選擇.
一套是80後的電影.
一套是90後的電影.
一套是00後的電影.
要顧及不同年齡層的同學.
他們選擇電影的scenario當中寫個案分析.
我都不可以太搜究.
我都是與時並進.
我的同工不斷推介他的sticker給我.
我也要了解這個sticker來自何方.
這個sticker是來自這部電影.

$^{681}$哇 馬上有共鳴.
不知道你們有沒有看過.
這部電影拿了兩個大獎.
最佳女主角.
已經開始咬耳朵.
大家都在討論劇情.
這部電影對我來說.
我都覺得好看.
當中可能有些人不喜歡結局.
今天不會說這件事.
都不會劇透.
但我看這部電影的時候.
我有很大的感受.
雖然這部是愛情電影.
沒有吻的.
很少吻的.
所以不用看這些.
但這部電影我感受到.
常常有很多這些情景.
這些情景是什麼呢.
就是.
我去不到.
可不可以幫我下.
那就.
幫我按一下.
你再按一下.
OK.
常常都有很多這些情景.
高科技都是高風險.
你會看到常常有很多這些場景.
這些是擁抱的場景.
在這套劇目裡面.
我常常感受到.
很不容易走的路.
大家都有理說不清.
大家都有自己既定立場.
大家都有自己的限制.
但去到一個欲哭無淚的時候.
大家再相遇.
擁抱就解決了一切.

$^{721}$當你看到.
在剛才片的左邊的時候.
一對朋友.
他有很多嫌隙.
有很多誤解.
但最後就是.
就是.
澄清了.
解決了.
那就相擁而哭.
第二個你會看到就是.
媽媽和女兒之間.
有太多誤解了.
有些立場在她小時候是不明白.
但媽媽自己也有很多限制.
但當冰釋了之後.
兩母女又相擁而哭.
去到第三個就是.
一個長距離的關係.
大家都是珍愛.
值得等待.
在過程當中又相擁而哭.
去到第四個的時候就是.
你會發覺有很多問題.
其實不是即時解決到.
但在大家互信之下.
等待過程當中.
最後都會相擁.
當然可能對你們來說.
最不容易接受的就是.
下面那一連環圖.
就是男女主角.
在最後一個情深的.
分離的擁抱.
我還沒揭穿吧.
我不相信你們知道就不看了.
我想對於你們來說.
為什麼這麼難接受呢.
或者對於你們來說.
為什麼這件事是一個這麼.

$^{761}$為什麼對你們來說.
是一個這麼刻骨銘心的事情呢.
正正就是其實.
有很多事情對於你們來說.
感受是很重要的.
感受都要先走.
但是當我們感受.
我們很多時候都會擁抱了.
感受這個反應.
但其實感受過後.
我們都會有理性去分析.
還有一個執念.
去堅持你繼續走下去.
當我們常常都說.
要擁抱什麼的時候.
有什麼堅持呢.
耶穌告訴我們.
是主耶穌的命令.
和主耶穌身體力行的教導.
祂做給我們看.
告訴我們.
就算那是condition.
都不會再是condition.
那個意念.
那個教導.
就成為我們繼續all in.
去愛對方的行徑.
親弟姐妹.
在這部戲裡面.
我很感受到.
就是每個人去愛對方.
都有很多不容易解說的空間.
每個人去愛對方.
都有很多不同的限制.
但當大家都明白對方的時候.
擁抱就是最sweet的moment.
擁抱就是大家一起去明白到.
大家都是為了對方彼此相愛的緣故.
去走到這一步.
我知道.

$^{801}$很多弟兄姐妹對教會.
有很多不同的期望.
有很多不同的傷痛.
我在Flow Church都聽過很多弟兄姐妹的故事.
是很不容易的.
當我說這篇經文的時候.
我都知道.
是會勾起你很多東西.
有些人那些人是不值得愛.
那些故事是不值得看.
但既然有新的一頁的話.
就學習新的群體當中.
去彼此相愛.
不會美化Flow Church.
也不會美化Flow Church的小組.
但總會學習去執行.
上帝給我們的生命.
那條生命令就是彼此相愛.
剛才說到.
要到什麼時候才明白呢?.
我選的經文和大家穿插的.
就是約翰一書第三章的經文.
約翰一書第三章的經文.
就是約翰晚年寫的.
差不多到了第一世紀的時間.
約翰一書第三章的經文.
是晚年約翰寫給當時很多異端出來的環境教會.
如何去認清我們才是耶穌基督的門徒?.
愛的表徵是最重要的.
而如何履行一個愛的表徵.
就是你願意為弟兄可以做的事情是什麼.
異端很多時候都是用自己的出發.
按自己的condition去做事.
但是約翰告訴當時受異端影響迷惑的教會.
我們真的彼此相愛.
為弟兄所做的事情.
就是能夠將那些異端排他的閥門.
就是這條新的命令.
你們應當彼此相愛.
我選了幾個字跟大家說的.

$^{841}$經文就是.
你們應當彼此相愛.
這就是你們從起初所聽見的命令.
就是剛才約翰福音第十三章.
次級的新命令.
第十六節.
三章十六節.
約翰福音大家都認識.
是上帝愛世人甚至將他獨生子賜給.
這個就是無條件的愛.
但是到了約翰一書三章十六節的時候.
那個經文是.
主為我們捨命.
我們從此就知道何為愛.
我們也當為弟兄捨命.
這個說話出自約翰的手筆.
在我看約翰一書的時候.
是最震撼和最能夠co-relate.
約翰他自己本人的經歷.
當如戰晚餐之後.
赫西瑪聯的土文.
猶大就是用親嘴去賣耶穌.
耶穌就上了各個他山上的路.
門徒就流散.
唯獨是約翰是看著他自己的夫子.
他自己的主是上十字架的.
他看著自己的夫子為眾人捨命.
他就寫下這句說話.
告訴人們.
主為我們捨命.
我們從此就知道何為愛.
這個愛是unconditional.
這個愛是為他所愛的人捨命.
我們能夠和異端分辨.
我們能夠和其他人分辨.
就是能不能夠為弟兄捨命.
我們相愛不只付言語舌頭.
總要在行為和誠實上.
或許你真的不能夠和他講愛的話.
但我們可以有其他行徑保持關係.

$^{881}$談不攏.
現在談不攏.
其實不代表將來談不攏.
現在不咬弦.
不代表將來不會復合.
對於我們來說.
擁抱這個訊息在最後.
我今天講題叫做「擁到底」.
在擁到底的時候.
對我們來說就是一個不容易做.
而耶穌真的擁他所愛的人擁到底.
我們也學耶穌去擁到底.
當我們說擁抱就是擁的時候.
就是說擁抱就是擁的時候.
我們可以做些什麼.
其實是不斷地挑戰我們寫記.
你願不願意寫記到底.
願不願意不要看自己那麼大.
或許看自己的意見那麼大.
願不願意在當中繼續去學習.
人越大經驗值多了.
你可能覺得自己看得獨到一點.
但是看別人比自己強又是什麼回事呢?.
當Ian在講羅馬書15章的訊息的時候.
其實在上一年2020年的時候.
我講閃避球的時候也講過那段經文.
正正在不同意見分歧上.
我們要兼顧那些未兼顧的人.
在過程當中都是有先後.
有時間上的差異.
當我們要擁抱.
說要擁抱的時候.
我們同樣都是試著學習寫記.
學習去履行上帝給我們的新命令.
在這個年代是很難的.
因為是在挑戰我們的底線.
也是在挑戰我們做的事情.
但是我覺得環境到了這一刻.
能夠見到所愛的人就擁抱一下他吧.
你不知道見到什麼時候.

$^{921}$這一刻你可能還能見到.
下一個月你可能見不到.
會不會還是因為當初的小小嚴規.
就成為陌路人呢?.
如果外面那個是坊間的人.
可能是朋友.
但是那個如果是主內的弟兄姊妹.
我希望我們試著多走一步.
耶穌愛我們到底.
我們也學習愛對方到底.
我們一起祈禱吧.
主啊,這條路是不容易走的.
風光的日子也不容易.
我們不知道什麼時候能見到.
但是我們相信我們未來會見到.
當我們在未來的時候.
我們見到一個曾經因為一些緣故.
分開的弟兄姊妹.
我們如果再見到一些曾經有爭執的弟兄姊妹.
求主你幫助我們.
我們彼此擁抱.
一抱吻恩仇.
主耶穌你在世餘下的日子.
仍然爭取時間身體力行去示範何謂愛.
而那個愛是無條件.
去為到身邊的人去走出來的時候.
求主你幫助我們.
在這個大時代大難關當中.
去身體力行去遵行這個新命令.
未來仍然有很多困難.
但是我們在未來會相見.
因為主耶穌會再回來.
我們就在不同環境當中彼此去支持.
繼續走下去.
求主你幫助我們.
未來見.
奉主耶穌的名求.
阿門.
\newpage



\section{詩篇 23:1-6-20220618}
\label{sec:rvuxp0Bes90}
\textbf{【網上聖餐崇拜】小心地滑…擁抱唔到|詩篇23\_1-6|20220618 [rvuxp0Bes90]}
\newline
\newline
連結: \href{https://youtube.com/watch?v=rvuxp0Bes90}{\texttt{ https://youtube.com/watch?v=rvuxp0Bes90}} ~~~~ 語音日期: 2022-06-18 
\newline
\newline
\hyperref[sec:SLTU9ILLofg]{\small{< < < PREV SERMON < < <}}
~
\hyperref[sec:index_chronic]{\small{[返順時目]}}
~
\hyperref[sec:index_scriptual]{\small{[返順卷目]}}
~
\hyperref[sec:zMmzg_ext8I]{\small{> > > NEXT SERMON > > >}}
\newline
\newline
詩篇 23:1-6-20220618
\newline
\begin{longtable}{cl}
\hline
\hline
章節 & 經文 (和合本修訂版)\\
\hline
23:1 & \begin{tabularx}{0.7\textwidth}{X} 耶和華是我的牧者,我必不致缺乏。 \end{tabularx} \\ \\ \relax
23:2 & \begin{tabularx}{0.7\textwidth}{X} 他使我躺臥在青草地上,領我在可安歇的水邊。 \end{tabularx} \\ \\ \relax
23:3 & \begin{tabularx}{0.7\textwidth}{X} 他使我的靈魂甦醒,為自己的名引導我走義路。 \end{tabularx} \\ \\ \relax
23:4 & \begin{tabularx}{0.7\textwidth}{X} 我雖然行過死蔭的幽谷,也不怕遭害,因為你與我同在;你的杖、你的竿,都安慰我。 \end{tabularx} \\ \\ \relax
23:5 & \begin{tabularx}{0.7\textwidth}{X} 在我敵人面前,你為我擺設筵席;你用油膏了我的頭,使我的福杯滿溢。 \end{tabularx} \\ \\ \relax
23:6 & \begin{tabularx}{0.7\textwidth}{X} 我一生一世必有恩惠慈愛隨著我;我且要住在耶和華的殿中,直到永遠。 \end{tabularx} \\ \\
[1ex]
\hline
\hline
\end{longtable}
$^{1}$各位丁姐妹,早安晚安,很高興可以在這裡再一次跟大家分享.
我相信最近,自《小心地話》這首歌出來之後.
跟基督教有些東西相似.
基督教在這兩年很喜歡說咒助詩.
咒助詩彷彿成為了一個大家由以往很禮貌,很客氣,很天真,很慢爛,很仁慈,很愛.
突然間這兩年我們說多了咒助詩.
詩篇裡面其實充滿著不少關於咒助的詩篇.
敵人要死,敵人的小朋友要死.
咒助到一個地步好像跟我們平時基督教所論述的不一樣.
《小心地話》這首歌我聽最主要是因為這首歌很特別.
我最近在家唱卡拉OK.
不是,沒有,你知不知道現在有沒有卡拉OK?.
有的,都有的,你不去唱,現在有很多人有些卡拉OK神器.
他唱,原來唱不到這首歌.
怎樣唱都唱不到這首歌.
因為假音不知怎樣去.
看他唱,現場也很…現在你可以上網聽聽,不要介意.
唱得挺好的,不是很完美,但很難得唱成這樣.
這首歌除了很難唱之外,其實Violence寫的歌詞很精彩的地方是.
全世界都有地板,你明白吧?.
大家現在可以Google一下這首歌.
他說《小心地話》,惡人會不小心,又小心也好.
一滴水就會跌死他.
或者他最壞的那句是,吃飯的時候,閻羅王就在旁邊等著你.
又祝你好平安,不過病足幾千天.
祝你好平安地病足幾千天.
這些說話其實是在這個時代裡,大家會共鳴的東西.
教會其實很不會說這些說話.
說完之後就會有人跟你說我們要愛心彼此包容.
你一定會有些L的朋友跟你說.
所以就算在這兩年多人認識了也好.
但其實我們在教會裡很少真的會去應用.
坦白說,就算我們Folk Church的弟兄姊妹也好.
我們也未必準備好,只是在門口見面.
我們就他閻羅王,很好吃飯.
不過閻羅王在旁邊等著你.
說完之後你也好像不太愛主.
這些表現很差勁.
所以我想嘗試多做一件事.
除了說奏坐詩之外.

$^{41}$嘗試將奏坐詩的context或者奏坐詩的處境.
放在一篇我們很熟悉的詩篇下去看.
以後跟你讀詩篇23篇其實是在奏坐它.
你明白我的意思嗎?.
既然是奏坐詩,你不要說哪篇,你查一下就知道.
奏坐人很慘的,基本上跟Wyman那首詞一樣.
但我想將奏坐詩的context.
放在一篇我們很常用的詩篇身上.
即是詩篇23篇.
我今天想做這件事.
我希望不要將一個傳統.
即是在一個人很苦難很困苦的裡面.
你叫他突然間忘記一切.
起落,肥立比書說事實起落.
我再說你們要起落.
你看看,整個肥立比書都叫你起落.
所以你就要起落.
這些論述亂說的,不過很普遍的.
總之就是要逼你起落.
但我想嘗試將今天用詩篇23篇.
一篇很熟悉的詩篇.
我們嘗試用另一個角度.
看如何將奏坐詩某些精髓.
放在我們日常裡面去用.
我不想說奏坐詩純粹是用來.
譬如唱完《小森地化》.
其實心很涼.
涼的意思是覺得.
真是很好了,有人可以說出.
自己心裡面很想說的話.
你對著一些很壞的人.
很惡的人的時候.
你期望他真的讓他好好吃飯.
但其實嚴老王經常在等他.
你叫他在醫院住千多天.
讓他平平安安地住下去.
即是病但不要死.
不過很痛苦.
你明白嗎?.
那些說話是這個意思.

$^{81}$你嘗試住千劫醫院.
但就很平安.
其實這些說話都不知道是甚麼.
我想嘗試除了這些想法之外.
我想嘗試甚麼令我們在.
面對很多困難.
坦白說,六月對我們來說.
是很多回憶的月份.
無論在任何媒體裡面.
六月永遠都是我們.
有很多美好過的日子的月份.
在這月份的標誌下.
其實有很多東西.
在這些日子裡面.
我們不知道為甚麼沒有了.
又或者我們要說得更加真實的是.
其實面對這樣的情況的時候.
我們怎樣再生活下去.
即是如果你上半個月.
即是六月九日,六月十六日.
這日子.
即是如果你還去看看.
幾年前的新聞的時候.
其實不同人都有不同的反應.
我們有些人會覺得不敢再看.
我們有些人會覺得很想去謾罵.
現在連謾罵的機會.
連想去做任何表達的機會.
我們都好像完全失去.
如果要繼續生活下去的人.
其實要想甚麼.
當我們每年都會經歷的時候.
其實我們會想些甚麼.
我希望今天能夠.
起碼這個詩篇的想法.
自己和自己在六月份裡面的一個想法.
詩篇.
23篇我們不是不熟悉的詩篇.
我們很熟悉的.
我們看一下經文.

$^{121}$我們今天看經文夠了.
其他東西不說了.
「要牧養我,不缺乏」.
我要說「牧養我」這三個字.
它是parable.
即是一個分詞.
其實這個分詞的意思是.
它可以不單純當作一個名詞的用法.
也可以當作它是一個.
它不斷牧養你的用法.
即是它是繼續牧養你.
繼續體會你.
明白你這種處境.
「我不缺乏」是難解的.
因為如果你想像一下.
耶和華牧養你.
即是它看著你.
它看顧著你.
其實它看顧著你的時候.
你缺乏些甚麼呢.
或者甚麼缺乏之下.
耶和華要牧養你.
這個說法念口旺.
我們明白的.
其實耶和華說「少了牧者,必要之缺乏」.
如果還活該為何要罰.
但如果耶和華牧養你的時候.
「我不缺乏」.
為甚麼你缺乏了甚麼.
即是問題是你有甚麼缺乏了.
需要耶和華現在來牧養你.
沒有答案.
至少這一節這一句沒有答案.
但希望後面我能找到答案.
但我們要問的問題是.
甚麼情況下.
我們覺得很缺乏.
缺乏了一些東西.
而耶和華要刻意在我們當中牧養我們.
這是第一個我們要問的問題.

$^{161}$即是缺乏了甚麼.
安全感?.
缺乏了沒有人陪伴?.
還是缺乏了甚麼.
這是我們要問的問題.
第二節是我們最喜歡的金句.
雖然我沒有改動過任何翻譯.
因為這句太喜歡了.
「他使我躺在青草地上,你我在可安歇的水邊」.
這句是甚麼?.
是不缺乏.
不缺乏的意思是甚麼?.
或者這叫初期的不缺乏症狀.
不知道自己在說甚麼.
試想像一下初期的信主.
初期對耶和華還有很浪漫的時期.
我想形容為一個.
等於以前我最多聽見證.
我們中國生團的見證.
最多聽見的是甚麼?.
就是等巴士,巴士走了.
我祈禱,不知為何第二架巴士會很快在後面出現.
這就是青草地和溪水旁.
所有事情都是浪漫的.
前面的巴士很遲到.
你明白嗎?只是那麼簡.
沒有甚麼神蹟.
前面的巴士遲了.
你會發現,哎呀,走了.
主下這架巴士要15分鐘.
但為何這麼快來一架呢?.
這些見證我們聽完後會很Amen.
主真的很厲害.
把這麼大的巴士搬到我面前.
Hallelujah.
這些叫初期的浪漫徵狀.
信主後我們喜歡這些.
例如你很愛主,祈禱.
神真的為你將你最愛的女孩放在你面前.
你就覺得,我的人生就美滿了.

$^{201}$這還不是青草地,溪水旁.
這些是初期浪漫徵狀.
基本上每個基督徒都會經歷.
不缺乏的是甚麼呢?.
當我們想要一些東西的時候.
我們信仰的初期.
好像,我不想得罪那些.
拍攝那些見證.
你明白我說甚麼嗎?.
你不明白我說甚麼嗎?.
我懶得說名字.
我想說,我尊重他們.
拍攝那些見證永遠都是有Cancer.
祈禱,神醫治那些.
或者我離婚,神我上了主.
突然間就不離婚.
神又將我老婆給我.
然後就因愛下去.
這些叫初期浪漫徵狀.
就好像所有東西突然間變了青草地,溪水旁.
這個觀念都不錯.
因為如果耶和華真的沒有養我們.
我們有需要的時候,祂給我們東西.
其實浪漫的,溫馨的.
我不是那些很傻的.
我們喜歡受苦的.
不是,神你給我多點錢,Amen.
神給我十間樓住.
我可以收租,不用打工.
這個,Hallelujah,讚美主.
這些青草地,溪水旁.
如果你是騎的話,可能有的.
不過可能不關上帝事.
只不過我是貪心.
但這個是初期浪漫徵狀.
我們喜歡這套畫.
青草地裡面,溪水旁裡面.
甚麼都不做.
耍廢,廢廢地.
不想工作了,不想努力了的感覺.

$^{241}$躺平的感覺.
我們喜歡的,誰想打工呢?.
這個人生裡面.
受苦,做事,根本是一件很拖累的事情.
但去到第三節的時候.
這個初期浪漫徵狀就斷了.
因為他使你的魂蘇醒.
你會發現,信仰不是一個純粹青草地溪水旁.
有巴士到,你喜歡了十年的女孩.
突然在你面前跟你說我愛你.
沒有了,沒有了.
女神形象原來是去撩鼻屎的.
他是不洗澡的.
你開始蘇醒,大約是這樣.
或者在信仰歷程裡面.
你可以發現,不是每次祈禱巴士都在.
只有一次.
往後99次裡面,巴士都不會到.
死了小心,原來祈禱不是很聽的.
這就是你靈魂蘇醒的時候.
你的魂被啟發.
由你以往看甚麼叫缺乏.
看甚麼叫耶和木養你.
你開始蘇醒,不是單純用一個角度.
或者一個觀念去想那件事.
不再是.
所以那句是為自己的命引導我走二路.
你開始走一條叫做上帝為你揀選的路.
那個缺乏不同了.
那個缺乏不再是說我有需要.
而是說答應我.
然後整個見證拍出來.
Hallelujah就是美主,不是不是.
你開始由99次巴士不到的時候.
你發覺原來到不到不重要.
最重要的是我和上帝的關係.
你開始.
Taste信仰的本質.
不是皇大先識的信仰.
不是那些求神拜佛.

$^{281}$初十五裝個香.
聽完禱告有求必應.
不是.
你開始,每個人開始問.
我在上帝裡面走一條路.
是走一條甚麼路.
如果要說一點哲學.
就是虛無主義.
正正在打這件事.
第三節是.
當基督教成為Christiandom.
是掌管所有哲學歷史.
我講中世紀的時候.
去到存在主義興起.
科學,工業革命等等不同之後.
虛無主義等等這些論述.
講的是.
其實有沒有一條叫做二路呢.
還是基督教要賦予人所有東西都有意義.
你明不明.
我們Purpose Driven Life.
你明不明.
總之這本書賣得到.
最近Mick Rowan都退休了.
終於換了一個新的人.
Mick Rowan都病得很慘.
兒子又過世,他自己又很不舒服.
Purpose Driven Life成為了.
基督教好像必然的東西.
我們人生有照命.
我們人生是有意義的.
上帝做你是有方向的.
我們每個人都好像找一個照命.
上帝你一定要告訴我.
我要做些什麼.
去神學院的獻身營.
就是在尋求.
到底我在上帝面前要做些什麼.
我要宣教,我要去傳福音.
還是我要在哪裡為上帝做事.

$^{321}$大體上第三節是.
你開始Awakening之後.
你發覺這個人生有意義.
走一條路.
但正如剛才所說.
虛無主義的出現不是沒有意思的.
存在主義裡面的虛無主義.
開始在第四節裡面的呈現.
他說我行過死任的幽谷.
原來你突然發現.
明明第三節是說.
為自己的名人道我走二路.
我以為那條The Path of Righteousness.
那個二路是什麼.
為主打爛美好的仗.
好像保羅那樣.
我守的都守住了.
很浪漫,從此以後有公義的冠冕為我傳流.
我們被這些東西太浪漫化.
覺得二路是一條這樣的路.
但怎知第四節告訴我們.
走那條二路.
原來等同於死任的幽谷.
原來那個魂甦醒.
在上帝裡面.
好像必然要經歷死任的幽谷.
坦白說.
在教會三十多年.
初期那些熱血的弟兄姊妹.
我總覺得千萬不要攔阻他們的熱血.
他們永遠都傳福音跑先過人.
他們在教會服侍,寧死熱心.
但在教會十年之後.
教會所有的遊戲規矩,你都知道怎麼玩.
你玩玩就ok.
不要太認真.
我十年花在教會裡.
令團契搞到幾百人也好.
教會招搖撥撥也好.
但十年後.

$^{361}$人與事改變.
有些事不似預期.
你發現你十年的功夫,十年的努力.
好像突然之間白費.
我們覺得信仰只是我們人生的一小部份.
我們開始會覺得.
死任的幽谷是說.
信仰落在一個.
不再是初期浪漫的時期.
清早地溪水旁的觀念.
好像完全沒有了.
死任的幽谷的出現.
只不過令我恐懼.
雖然它說日不怕灶害.
因為你與我同在.
你的丈,你的肝都安慰我.
我們說它在經歷甚麼.
我們在經歷信仰.
好像不再是以往那麼高峰.
那麼火熱的時候.
我們突然之間.
害怕在信仰裡面再遇到任何事.
死任的幽谷不是說.
有很多淒涼的事發生.
我最近有一個同學.
他所經歷的就是.
他的哥哥打完針.
所有事情都沒有關係.
不過他去世了.
他爸爸一個星期前又去世了.
打完針之後又去世了.
當然沒有對比的.
所有事情都要這樣說.
人生突然之間.
你很害怕會遇到.
有些事情是很不似預期.
難得死任的幽谷過了.
走過了.
我們以為應該好像.
《鵪鶉見證》的影片所拍的一樣.

$^{401}$從此以後公子和王子.
不是公主,是甚麼?.
公主和王子就快快活活生活下去.
不是,只要不是.
你開始害怕.
不至於死,但你很害怕.
在信仰裡面.
不知何時有一波出現.
在信仰裡面.
你覺得你不想再跟這些事情一起.
以前在教會裡面.
你為教會做很多事.
你一往而退.
當你經歷了一些事之後.
你很怕再為教會做任何事.
你知道你已經不同了.
雖然你療傷了.
你見了輔導.
你見了靈性導師.
去思維症院.
七日六夜.
你發現你的象都在安慰你.
是真的.
但你心裡面是在害怕.
你覺得.
我可以再有力量去面對嗎?.
當我再認真去侍奉的時候.
只不過是有其他事情會走出來.
你心裡面在害怕.
其實侍奉主是一條點你的路.
明明說走易路,變成死刃的幽谷.
明玩野的說話.
你要跟上帝說.
上帝,在你裡面甚麼都好.
你給我甚麼我都受.
你所給我的試探,無非人能忍受得過.
所以你給我都能忍受得過.
是,過了,但我裡面害怕.
最恐怖的是第五節.
第五節的「攻擊我的人」.

$^{441}$我不想翻譯,但不知如何翻譯會好些.
「攻擊我」這字跟「牧養我」是一樣的.
一樣的詞,是一個分詞.
所以我想譯作「再攻擊我的」.
但「人」字不太好.
我不想用「人」字,應該是「再攻擊我的」.
意思是那些敵人不斷攻擊在你面前.
試想想,這篇詩篇叫「大衛的詩」.
大衛在想甚麼.
原來在他信仰走到過了第一個才是幽谷的時候.
你當作被小顱追殺,我們找些情境來講,容易明白一點.
他人生正在準備的是甚麼.
往後的日子裡,很多人會攻擊他.
甚至到後期,是沙龍的兒子將他趕走到皇宮裡.
甚至將他父親和所有人置於死地.
在第四節裡,我們已經玩完.
在我們信仰歷程裡,過完第四節後,我們已經傷了安慰源.
但我們還會忍耐很久,因為我們仍然想忘的是甚麼.
「師父」其實是青草地溪水旁,是有東西看的.
是上帝知道我做完甚麼之後,會祝福我所做的事.
我們快要知道,原來我們所經歷的都是死在幽谷.
Full Church 頂梓梅,我相信在這幾年裡.
你來Full Church,是因為我們說是或多或少.
在不同以往的經歷裡,在教會生活當中.
都是遇到很多這些事情.
你一定會害怕,我也會害怕.
第五節最可怕的是,害怕完之後.
我們要在攻擊我們那些.
如果夜話木仰著我們,同時發生甚麼事.
是那些敵人都在攻擊你.
原來夜話木仰著我不缺乏,那不缺乏是甚麼.
是不是很恐怖,如果這樣說下去.
我們不缺乏敵人.
如果夜話木仰著我不只缺乏.
如果夜話木仰著你,繼續跟你走下去的話.
原來不只缺乏不是說甚麼.
是說那些在攻擊你的人.
如果《小心地獄》是一首歌,在吃飯的時候.
在旁邊等著他,這句說話挺好聽的.
但你看第五節.

$^{481}$他說在不斷攻擊你的人面前.
人家都在對著你.
即是暗箭也好,明箭也好,都在射你.
他不用《小心地獄》的歪文歌詞.
所以他說在那裡木仰著你,是為你擺設筵席.
他用油膏撩我的頭,說我可以喝那杯.
7月1日又會轉一些事情.
你看到一些人的名單出現.
你會覺得不知道怎樣反應.
或者很多人已經不想反應.
第五節他好像在用《小心地》的奏助詩的觀念.
但他的方向不是說敵人會怎樣.
奏助詩的重點是說.
見到敵人會怎樣,我們心安.
你覺得我二十多三十歲比某些人長命.
我就鬥長命,看你怎樣.
什麼結局.
現在有些人的結局只有一個人.
他沒有團隊,只有一個.
你可以看一下結局.
奏助詩很想用對方來令自己內心愉快.
但第五節這裡,他不是說敵人怎樣.
第五節跟奏助詩有很大的分別.
或者他想說另一個面相的奏助詩是什麼.
是我可以在敵人攻擊的時候.
上帝木仰到我那個地步,為我擺設了筵席.
我可以很安然地在這裡工作,在這裡喝,在這裡吃.
享受我的生活和人生.
六月除了很多事情發生之外.
六月有很多人生日.
你知不知道?.
Dajer,Marv和最近的Winker.
都是六月生日.
榮耀問我為什麼知道.
因為我在六月四日,Day生日那天.
我被一個人邀請,你不要猜他是誰.
就是很年輕的那個.
我怕他聽到,我懶得說他名字.
我被一個很年輕的人抓了我去尖沙咀拍照.
因為Day那天生日,他在青春紙對出那裡有個燈箱.

$^{521}$我和這個很親的人.
你知不知道?中學生.
很親的人去看看Day的燈箱在哪裡.
他很開心地拿著手機.
看到之後就拍了一張照片.
拍完之後他就說.
爸爸,不是,不是.
(笑聲).
對不起,女兒,我錯了.
拍完之後他就走了.
我問他在哪裡.
他說在旺角,Day有三張大的Band.
掛在那裡,你知不知道?.
你知道的,你是COLOR的粉絲.
他就走去旺角,坐巴士去尖沙咀的燈箱.
拍完之後,他怎麼拍的呢?.
就當他沒有看到.
他拿著手機,隨便拍一些高一點的,漂亮一點的.
側一點的,像這樣拍的.
隨便拍一張就算了.
我說我來到拍一張就行了.
他說是的,走吧.
我人生有什麼意義?.
(笑聲).
他拍完尖沙咀再去旺角拍完之後.
就再去基尼地城.
基尼地城是什麼呢?.
是有手搖飲品.
那天Day生日就有福袋送.
就排隊,等買杯飲品就有福袋.
我就排隊.
我以為真的有福袋.
排到我說,不好意思,今天都排了.
他說不要緊,我們都有幾張英文什麼.
那些大那些,我不懂說什麼.
然後我們去喝杯飲品.
然後他拿著那三張東西就很開心地走了.
然後我完了那晚,就回去問他一個問題.
我說,其實今天我們做過什麼?.
我說你要拍照的話.

$^{561}$大把網上那些已經有人拍得很漂亮的照片都放在那裡.
你手機那張有什麼好認?.
他說有什麼意義?.
我知道他不回答之後,我就覺得我問多了.
我就悔改了.
不要問,不要問.
只要信是有意義的.
你知道我突然之間沒有三個小時.
我人生其實有很多事情做.
你知道,想都想不到.
完了照片,女兒,不是不是不是.
那個很年輕的那個.
睡醒了,趴在我的床頭抱著我.
說了一句話.
她說,爸爸,不是不是.
真的不知道怎麼說.
我很開心你昨天陪我去.
嘩,我終於意義在哪裡了.
接著我開始問那些二十多三十歲的傳道童工.
就問他們,其實這件事是什麼來的?.
我不明白,我真的不明白.
他們一群人都在罵我.
他說,他肯叫你去,你已經撿到了.
你還說什麼意義?.
我突然一驚醒的感覺是.
原來這些就是在敵人面前擺設筵席.
你明白我想說什麼嗎?.
原來有一個生活.
原來有一個仍然活著的感覺.
你能夠經歷一些你沒有經歷過的事.
其實那就是上帝的杯滿日.
上帝的恩典和再次見證.
丁子梅,如果奏坐詩所說的是敵人的結局的話.
詩篇二十三篇所說的是一個很難很不容易.
你明知道前面的敵人只會繼續攻擊你.
繼續傷害你.
你的心靈越來越脆弱.
你不敢再做任何事的時候.
原來在那一刻不要再問有什麼意義.
試試學學虛無.

$^{601}$試試讓自己覺得.
我正在做一些活著的事.
正正面對在這個世界裡.
很難再找到意義的時候.
我們是開心的.
如果剛才說.
在一百年前存在主義裡的虛無主義.
是要針對基督教所有的東西.
都彷彿有意義的時候.
所有東西都是有意義的生活的時候.
原來有些時間.
尤其是在一次世界大戰和第二次世界大戰期間.
發生了很多匪夷所思.
人的生命隨時會沒有.
炸彈隨時會下來.
其實到今天一樣.
虛無主義正正要反應的.
我們今天在人生裡.
信仰裡.
面對好像不是很容易做到.
我們覺得我們生命裡應該要做的事的時候.
突然間我們要轉變.
特別想和海外的弟兄姊妹說.
當傳道人要做一個初武員的時候.
當牧師要被迫提早退休.
在那邊游手好閒.
沒有事要做的時候.
虛無是一個答案.
虛無是說.
當所有東西都不似預期.
好好活著.
好好找一些你覺得沒有意義的事去經歷一下.
那個是你在虛無的裡面.
在一個大漩渦的裡面.
你覺得自己活著的價值.
今天在這個時間.
海外的弟兄姊妹.
如果你彷彿放棄了很多東西離開了.
我特別想你好好享受.
這兩三年做虛無的感覺.

$^{641}$和留在這裡的弟兄姊妹說.
我們好像已經很難像以往一樣.
為上帝創功立業.
揚名立萬.
好像這些日子已經過去的時候.
信仰剩下來的是什麼.
或許存在主義裡面的虛無是說.
當你被很多好像覺得有意義的.
在引導我們想法的同時.
有很多不為人所愛到.
其實有意義的東西.
都值得我們生命去突破.
就像一個傳道人在加拿大做一個倉務員.
不是價值突然之間貶低下來.
在一個虛無的裡面.
仍然上帝可以在那時候和我們說話.
我們那時候才知道原來.
吃一餐飯是不需要趕的.
原來可以人與人之間好地生活.
好地相處.
讓自己被愛一下.
給自己空間愛一下.
好像我今天仍然記得.
6月4日的時候.
6月5日的早上.
女兒衝進來抱著你.
跟你說一些有意義的東西.
我竟然大逆不道地說.
女兒其實你覺得昨天的事真的有意義嗎.
她可能不是很開心就走了.
香港留下來的頂梓梅.
我們已經不再是那個時候.
很多東西都要曉遇.
很有上帝的心意.
虛無是在說我們活著.
我們吃一餐飯.
喝一杯.
Obi經常引領牧者喝酒.
不要說他了.
你知道我們牧者很喜歡喝酒.

$^{681}$他經常都引領我喝酒.
說他享受一些.
不是我們這一刻覺得很有意義的東西.
那個酒原來是在建立你跟一個人.
一個很特殊.
很以前你沒有擁有過的經驗.
裘天賦在這個時候.
用虛無去安慰我們的心靈.
當你仍然很渴望很多東西會改變的時候.
而那些東西只會越來越差的同時.
讓我們好好生活.
去對抗這些.
好像永遠不會改變的狀況.
讓我們看到上帝.
可以在有意義的同在.
上帝也可以在很沒有意義的裡面.
跟我們在一起.
如果你要問上帝無無聊.
我覺得上帝跟約伯這個遊戲是最無聊的.
原來無聊.
承受了信仰.
一個很重要的考驗.
裘天賦祝福香港的信徒.
在無聊的裡面.
我們更加享受弟兄姊妹之間的關係.
享受更多一起經歷過.
在敵人面前.
我們每喝的一杯酒.
吃的每一餐飯.
我們一起祈禱.
天賦多謝你讓我們今天有這個時間和空間.
我特別想禱告的是.
為了海外的弟兄姊妹我們首先去禱告.
其實真正擁抱.
不是真的擁抱很多很有意義.
覺得很值得擁抱的東西.
其實真正擁抱的是.
連敵人在你面前都要擁抱.
如果擁抱是在說.
縱然如此.

$^{721}$上帝都讓我們仍然有這個空間.
天賦我求你祝福海外的弟兄姊妹.
當他們開始找意義.
安定下來之後.
他們知道自己在你面前.
每一天所活的是怎樣.
你仍然與他們同在.
仍然與他們經歷.
亦為我們留下的弟兄姊妹祈禱.
讓他們學會信仰的基本步.
和你一起.
不放手地抓住你.
你又抓住我們.
你又將我們帶到敵人面前.
我們去吃飯.
去喝一杯.
求天賦讓我們擁抱這些東西.
讓我們面對敵人的時候.
面對邪惡的時候.
我們怎樣好好照顧自己.
擁抱自己心靈的需要.
求天賦憐憫我們.
幫助我們.
我們恭敬將我們每一個弟兄姊妹的生命.
交在你手當中.
他們親自與我們同在.
奉耶穌基督的寶貴命求.
阿門.
\newpage



\section{路加福音 15:1-7-20220625}
\label{sec:zMmzg_ext8I}
\textbf{【網上崇拜】乜ye ye 問題一律建議抱緊處理|路加福音15\_1-7|20220625 [zMmzg\_ext8I]}
\newline
\newline
連結: \href{https://youtube.com/watch?v=zMmzg_ext8I}{\texttt{ https://youtube.com/watch?v=zMmzg\_ext8I}} ~~~~ 語音日期: 2022-06-25 
\newline
\newline
\hyperref[sec:rvuxp0Bes90]{\small{< < < PREV SERMON < < <}}
~
\hyperref[sec:index_chronic]{\small{[返順時目]}}
~
\hyperref[sec:index_scriptual]{\small{[返順卷目]}}
~
\hyperref[sec:o_rYmsKLSrs]{\small{> > > NEXT SERMON > > >}}
\newline
\newline
路加福音 15:1-7-20220625
\newline
\begin{longtable}{cl}
\hline
\hline
章節 & 經文 (和合本修訂版)\\
\hline
15:1 & \begin{tabularx}{0.7\textwidth}{X} 許多稅吏和罪人都挨近耶穌,要聽他講道。 \end{tabularx} \\ \\ \relax
15:2 & \begin{tabularx}{0.7\textwidth}{X} 法利賽人和文士私下議論說:「這個人接納罪人,又同他們吃飯。」 \end{tabularx} \\ \\ \relax
15:3 & \begin{tabularx}{0.7\textwidth}{X} 耶穌就用比喻對他們說: \end{tabularx} \\ \\ \relax
15:4 & \begin{tabularx}{0.7\textwidth}{X} 「你們中間誰有一百隻羊,失去其中的一隻,不把這九十九隻留在曠野,去找那失去的羊,直到找著呢? \end{tabularx} \\ \\ \relax
15:5 & \begin{tabularx}{0.7\textwidth}{X} 找到了,他就歡歡喜喜地把羊扛在肩上。 \end{tabularx} \\ \\ \relax
15:6 & \begin{tabularx}{0.7\textwidth}{X} 他回到家裡,請朋友和鄰舍來,對他們說:『你們和我一同歡喜吧,我失去的羊已經找到了!』 \end{tabularx} \\ \\ \relax
15:7 & \begin{tabularx}{0.7\textwidth}{X} 我告訴你們,一個罪人悔改,在天上也要這樣為他歡喜,比為九十九個不用悔改的義人歡喜還大呢!」 \end{tabularx} \\ \\ \relax
15:8 & \begin{tabularx}{0.7\textwidth}{X} 「同樣,哪一個婦人有十塊錢,若失落一塊,不點上燈,打掃屋子,細細地找,直到找著呢? \end{tabularx} \\ \\ \relax
15:9 & \begin{tabularx}{0.7\textwidth}{X} 找到了,她就請朋友和鄰舍來,對她們說:『你們和我一同歡喜吧,我失落的那塊錢已經找到了!』 \end{tabularx} \\ \\ \relax
15:10 & \begin{tabularx}{0.7\textwidth}{X} 我告訴你們,一個罪人悔改,神的使者也是這樣為他歡喜。」 \end{tabularx} \\ \\ \relax
15:11 & \begin{tabularx}{0.7\textwidth}{X} 耶穌又說:「一個人有兩個兒子。 \end{tabularx} \\ \\ \relax
15:12 & \begin{tabularx}{0.7\textwidth}{X} 小兒子對父親說:『父親,請你把我應得的家業分給我。』他父親就把財產分給他們。 \end{tabularx} \\ \\ \relax
15:13 & \begin{tabularx}{0.7\textwidth}{X} 過了不多幾天,小兒子把他一切所有的都收拾起來,往遠方去了。在那裡,他任意放蕩,浪費錢財。 \end{tabularx} \\ \\ \relax
15:14 & \begin{tabularx}{0.7\textwidth}{X} 他耗盡了一切所有的,又恰逢那地方有大饑荒,就窮困起來。 \end{tabularx} \\ \\ \relax
15:15 & \begin{tabularx}{0.7\textwidth}{X} 於是他去投靠當地的一個居民,那人打發他到田裡去放豬。 \end{tabularx} \\ \\ \relax
15:16 & \begin{tabularx}{0.7\textwidth}{X} 他恨不得拿豬所吃的豆莢充飢,也沒有人給他甚麼吃的。 \end{tabularx} \\ \\ \relax
15:17 & \begin{tabularx}{0.7\textwidth}{X} 他醒悟過來,就說:『我父親有多少雇工,糧食有餘,我倒在這裡餓死嗎? \end{tabularx} \\ \\ \relax
15:18 & \begin{tabularx}{0.7\textwidth}{X} 我要起來,到我父親那裡去,對他說:父親!我得罪了天,又得罪了你, \end{tabularx} \\ \\ \relax
15:19 & \begin{tabularx}{0.7\textwidth}{X} 從今以後,我不配稱為你的兒子,把我當作一個雇工吧。』 \end{tabularx} \\ \\ \relax
15:20 & \begin{tabularx}{0.7\textwidth}{X} 於是他起來,往他父親那裡去。相離還遠,他父親看見,就動了慈心,跑去擁抱著他,連連親他。 \end{tabularx} \\ \\ \relax
15:21 & \begin{tabularx}{0.7\textwidth}{X} 兒子對他說:『父親!我得罪了天,又得罪了你,從今以後,我不配稱為你的兒子。』 \end{tabularx} \\ \\ \relax
15:22 & \begin{tabularx}{0.7\textwidth}{X} 父親卻吩咐僕人:『快把那上好的袍子拿出來給他穿,把戒指戴在他指頭上,把鞋穿在他腳上, \end{tabularx} \\ \\ \relax
15:23 & \begin{tabularx}{0.7\textwidth}{X} 把那肥牛犢牽來宰了,我們來吃喝慶祝; \end{tabularx} \\ \\ \relax
15:24 & \begin{tabularx}{0.7\textwidth}{X} 因為我這個兒子是死而復活,失而復得的。』他們就開始慶祝。 \end{tabularx} \\ \\ \relax
15:25 & \begin{tabularx}{0.7\textwidth}{X} 「那時,大兒子正在田裡。他回來,離家不遠時,聽見奏樂跳舞的聲音, \end{tabularx} \\ \\ \relax
15:26 & \begin{tabularx}{0.7\textwidth}{X} 就叫一個僮僕來,問是甚麼事。 \end{tabularx} \\ \\ \relax
15:27 & \begin{tabularx}{0.7\textwidth}{X} 僮僕對他說:『你弟弟回來了,你父親因為他無災無病地回來,把肥牛犢宰了。』 \end{tabularx} \\ \\ \relax
15:28 & \begin{tabularx}{0.7\textwidth}{X} 大兒子就生氣,不肯進去,他父親出來勸他。 \end{tabularx} \\ \\ \relax
15:29 & \begin{tabularx}{0.7\textwidth}{X} 他對父親說:『你看,我服侍你這麼多年,從來沒有違背過你的命令,而你從來沒有給我一隻小山羊,叫我和朋友們一同快樂。 \end{tabularx} \\ \\ \relax
15:30 & \begin{tabularx}{0.7\textwidth}{X} 但你這個兒子和娼妓吃光了你的財產,他一回來,你倒為他宰了肥牛犢。』 \end{tabularx} \\ \\ \relax
15:31 & \begin{tabularx}{0.7\textwidth}{X} 父親對他說:『兒啊!你常和我同在,我所有的一切都是你的; \end{tabularx} \\ \\ \relax
15:32 & \begin{tabularx}{0.7\textwidth}{X} 可是你這個弟弟是死而復活,失而復得的,所以我們理當歡喜慶祝。』」 \end{tabularx} \\ \\
[1ex]
\hline
\hline
\end{longtable}
$^{1}$22年的上半年.
隨著國際間防疫政策的放寬.
我家申請宣教的步伐終於有了眉目.
這段時間有不少的流人都表達.
很開心聽到有Flow Church的牧者即將去非洲宣教.
但就不知道他是誰.
他就在這裡.
那怎樣認他呢?.
我可以說是全教會的牧者之中.
皮膚最黑的一個.
不信的話轉個圈給你看.
就是這樣認了.
雖然我經常被誤認是東南亞地區的朋友.
其實我是來自香港的.
我是香港人.
這個是舊版立法會網頁.
上面顯示這封信.
你可以掃描這個QR Code.
會看到我沒有戴口罩的樣子.
下一頁.
這一頁就是舊版立法會的網頁.
裡面是立法會歷史的頁面截圖.
那裡說到香港自從1841年1月26日起.
至1997年6月30日為止.
是英國的殖民地.
我家裡即將宣教的工場在東非馬達加斯加.
這個國家在1960年6月獨立.
62年之前它曾經是法國的殖民地.
它的殖民地身份是法國和馬達加斯加的新舊官方網頁.
和教科書都一致承認的歷史.
所以我去到馬達加斯加我要學法文.
在未有流利的語言溝通之前.
都相信要很多指手畫腳去表達.
除了一些翻譯軟件之外.
我都可以透過圖像一些Emoji去溝通.
所以以下就陪我一起模擬一下圖像如何說到.
知不知道這幅圖片代表著流唐哪一個月替.
猜兩個字的.
下一個這幅圖猜四個字的.
大聲一點.

$^{41}$擁抱世界.
沒錯了就是我的心智.
繼續下一頁.
這幅圖繼續猜四個字的.
是一百隻羊.
下一個.
這個呢.
法理菜人.
下一個.
這是猜一個人的.
這是別人考我.
我沒有答案.
他給了我提示第三個字是顏色.
因為我沒有答案所以你們認識的就在即時通訊留言.
下一個.
這次是猜兩個字的.
是有些味道的.
是文士.
問號.
這次最長也是最後一個.
這是甚麼.
九十九隻羊.
沒錯今晚的訊息圍繞著路加福音十五章一至七節.
是關於比喻裡的失陽.
在十五章當中.
關於父親大小兒子的篇幅佔了全章的三分之二.
聖經本來沒有一個明顯的標題或分段.
現在我們所看到的第十五章.
大多數都被人命名為失陽的比喻.
失淺的比喻浪子的比喻.
其實這個分段是有些誤導的.
因為耶穌不是說三個比喻.
第三節耶穌用比喻對他們說.
這裡說的比喻是單數詞.
是耶穌說的一個比喻.
不過是分了三個部分說.
就像連載的電影或漫畫.
是同一套東西.
同一個故事同一個比喻.
浪子比喻是我們聽得比較多.

$^{81}$我們今晚會一起看首部曲.
Episode 1 失陽.
Episode 2 失淺.
都會說一點點的.
如果你想順序聽完三部曲的上集和下集.
你可以搜尋5月7日和6月4日的《流唐講道重溫》.
首部曲裡面有100隻羊的木人.
Episode 2 有10塊銀錢的婦人.
Episode 3 有兩個兒子的爸爸.
都是在比喻天父.
這三部曲裡面共通的地方.
大家都失去了自己擁有的東西.
他們都花了很多的心機.
很多的時間.
很多的心血去尋找和去等.
共通的地方還包括.
他們最後是找到的.
遇到自己失去的東西.
我一路看這個比喻的時候.
都想起自己不見的經驗.
有試過找得到.
有試過找不到.
正所謂掉了東西就試試不要撿了.
你越不找到.
有些東西就不知道何時突然間會重新出現.
是嗎?.
好像南海的那隻船一樣.
看看失去了什麼.
如果那樣東西對你來說是很重要的.
其實你是一定會去找的.
你會親自去找.
你重囚都要去找得到.
好像走失了的寵物.
掉了的錢包.
沒有備份的電話.
失蹤的家人和手足.
我覺得不見了東西.
最後找得到其實就可以完結了.
就可以停在那裡.
就是說整個第15章.

$^{121}$其實可以去到第5節.
甚至只是停在第5節的上半節.
都可以完結了.
其實都不用去到32節那麼長.
是嗎?.
不是的.
因為這個比喻絕對需要用到三部曲去表達.
經文的重點放在找到了之後是怎樣呢?.
1,2,3部曲的三組片段的共通點就是.
當那個擁有者.
他找到他失去的東西.
他開心之餘.
他還會邀請人和他一起歡喜快樂.
請他的朋友和鄰舍.
家人,親戚.
甚至是傭人姐姐.
他都會歡迎他一起來家裡吃飯高興一下.
在前一章.
《路加福音》第14章1至24節當中的經文.
他提到耶穌去法利塞人的領袖家裡吃飯.
法利塞人就是剛剛那個emoji.
他們是什麼人呢?.
可以用我昨晚在開場聽過的一首流行曲形容法利塞人.
兩個字.
隔離.
為什麼這樣說呢?.
他們是沒有肺炎,沒有後痘.
都過著隔離的生活.
因為他們要遠離一切不潔的人和事.
就是法利塞人的宗旨.
他們由律法師和民事組成.
堅持遵守一些成文或者口傳的律法.
如果別人不fit in 他的那套的話.
他就會定義別人犯了罪.
上星期嘉Sir在講道裡也提及.
舊約詩篇23篇.
多道說耶和華神是他指紋的目者.
神要為他指紋擺設筵席.
這裡新約路加福音15章.
耶穌也在用木人的形象和擺設筵席.

$^{161}$去描述天父對罪人的接納和拯救.
木人,筵席這樣的形容.
很容易讓當時聖經年代的人進入這個比喻.
因為正正就是他們日常的生活.
在他們的傳統裡是很容易理解的東西.
耶穌講比喻的目的一直都是明就明.
有耳聽的就應當聽.
只要明白的人明白.
耶穌用這三部曲的排列也很有意思.
先是講了牧羊人和羊.
接著就講婦人和錢.
接著就講爸爸和兒子.
來到今天的香港教會.
我們很少有人養羊.
我想像耶穌要向我們講比喻的時候.
他的排序可能首先是講虧錢.
因為我留意到香港教會對錢.
對一些奉獻收入也很緊張.
本身香港人對錢也很著緊.
耶穌可能會放到最後才講失讓.
因為我們香港官方某些部門的人.
找到動物之後會將牠人道毀滅.
這個比喻的首部曲.
在十五章七節裡.
簡直有解答.
我們一起讀第七節.
讀解答 讀答案出來.
第七節我們一起讀.
「訴你們,一個罪人悔改,在天上也要這樣罰」.
我們可以從一些相同的字眼得知.
第七節裡說的九十九個不用悔改的義人.
其實在比喻裡代表著九十九隻羊.
而在第七節這個答案裡說的一個罪人.
就是在比喻中失去的那一隻羊.
在第七節裡說「你們」.
這個字眼跟比喻裡的字眼一模一樣.
「你們」就是耶穌說的比喻的對象.
祂要說給誰聽呢?.
就是要說給法利塞人和文士聽.
我二次創作了關於這個首部曲.

$^{201}$比喻中的首部曲.
「你們中間有誰的團契本身是有一百人的」.
「流失了其中一個會友」.
牧者就放下那九十九個回家很穩定.
又沒有離開的會友.
去找那個沒有再回來教會群體的人.
一直很忍耐地找.
自己找的.
還沒找到的時候.
可能這個離開團契的人還沒準備好被人找到.
經常耳濡目染.
封鎖了牧者.
牧者一直等,一直堅持.
一直繼續聯絡牧者.
聯絡了不知多少次.
終於都約到了.
很神奇.
連牧者自己也覺得很不相信.
我找到了.
找到的時候就很開心.
牧者尋人找到了.
在去蒲頭那天.
牧者就聽他分享了很多.
當氣回腸.
在他泥濘長怒之中碰撞的故事.
牧者就托他上肩上騎駁馬.
是不夠力的.
所以牧者就輕輕地.
又大點夾硬地給了他一個肋.
用愛將他拉近.
拉回進來信仰的群體.
用愛將他抱緊.
抱緊他所經歷過的傷口.
心裡面不想他再流失.
不想他再流浪.
故事發展下去會怎樣呢?.
我移創到這裡.
第六節裡面.
經文說.
對他們說的他們.

$^{241}$和你們和我一同歡喜的你們.
是指哪一個呢?.
第十五章第六節的回到家裡.
和第十五章第七節的在天上.
雖然字眼不是完全一樣.
但是在第七節裡面說.
在天上的使者會為罪人悔改而歡喜.
正正解答.
耶穌其實是在比喻.
天上的使者就是那些一起來慶祝失去的羊.
回到家裡一起慶祝的朋友和鄰舍.
迷讓回家.
罪人悔改.
原來不是每個人都和你一起那麼開心.
肯一起來吃一餐.
肯一起來開心的.
就固然是耶穌的朋友和鄰舍.
你不肯認約.
約你你不認記.
不一起慶祝.
還說回耶穌.
不單止你沒有幫忙找.
你還私下議論耶穌的做法.
這一場慶功宴就沒有你的份了.
你就成為了法利塞人和民事.
我相信我們當中.
沒有人立志想成為法利塞人和民事.
但是當某一天.
我們真的成為了法利塞人和民事.
過程往往是不知不覺的.
不要以為我這篇導讀是教會人辯認.
誰這樣做就是法利塞人和民事.
不是的.
坦白說.
有一點私心.
也有的.
聽到可以自用.
也可以分享給那種宗教領袖.
實情其實這篇講章.
也是對自己的一個告誡.

$^{281}$以免自己成為了這一種宗教領袖.
實不相瞞.
教內的確有一批人.
在宗教圈裡扮演員搵食和拿菜.
嘴上說支持耶穌愛罪人.
接納向基層報道.
贊成我們向小眾和少數族裔群體宣教.
到頭來只是一些連題口號.
用來說說而已.
沒有實質做的.
甚至自己不去做.
去做的人他們也爛了.
以下是一個有鬼故feel的真人真事.
我聽過的說法.
教會本身的事工也搞不定.
如何再放更多資源去宣教呢.
漸漸在這樣的氛圍下.
大家都淪為宗教演員.
我們也彼此提醒.
不要成為宗教演員.
哪怕只是宗教carnifer都不要.
路加福音裡面.
耶穌和法利塞人不是第一次交手.
1至14章中間發生過很多次火藥味濃的對答.
較為吸引我的是以下這兩個片段.
5章29及30節.
有一個瑞麗是利微.
她在自己家裡為耶穌大設筵席.
其他瑞麗一起來吃飯.
耶穌參與那一餐的時候.
法利塞人和民事就向耶穌的門徒發怨言.
你們為何會和瑞麗和罪人一同吃喝呢.
原來來到第15章.
法利塞人真的很一致.
他在第15章的時候.
說你們為何要和瑞麗和罪人一起吃飯.
到第15章他還在問這個問題.
他們認為耶穌和罪人吃飯是很不妥的事.
而這個問題早就問過了.
第二個吸引我的片段.

$^{321}$又是他們的交手.
7章37及38節.
說到城裡有一個女人是罪人.
知道耶穌在法利塞人家裡吃飯.
她就拿著盛滿香膏的玉瓶.
站在耶穌的背後.
靠著耶穌的腳哭.
這個片段讀到女人哭.
我也有點哭了.
因為這次的飯局明知設宴的是法利塞人的家.
這個女人是罪人.
她明知法利塞人不歡迎她.
但單單因為耶穌在那裡.
明知別人不歡迎.
我也去.
是因為她好想親近耶穌.
原來這個女人所做的事.
就反照法利塞人.
她當然不會去吃那頓飯.
如果耶穌和罪人一起吃飯.
但她自己不去.
別人去她也不開心.
耶穌和罪人一起吃飯.
其實也很歡迎法利塞人一起吃飯.
一起歡喜快樂.
只是他們不斷問為什麼.
為什麼.
但他們也不參加.
Timothy Keller在《The Prodigal God》這本書裡.
他形容到擁抱罪人的上帝.
是一個揮霍的上帝.
其實他放下了99隻羊去找那一隻.
他保證不到一定能找到.
也不知道耗時多久.
或者多少天.
但他已經撇下了99隻.
其實是一件很薄的事.
是一個很高風險的投資.
是法利塞人和民事難以理解的.
我小時候聽這個比喻的時候.

$^{361}$我其實不太知道.
一隻羊和一塊錢的價值是多少.
我一直以前也在想.
他放一頓慶功宴的花費.
用的錢會不會比他失去的東西.
本身的價值更高.
所謂的要「倒貼」.
在這裡我有一點想替法利塞人和民事平反.
因為我想他們和我一樣.
其實也不是很壞.
可能和我一樣在想.
上帝你真的很輝煌.
你的花費很大.
才能找到一個.
因為預備講99隻羊.
我重看了《逆風跑的99隻羊》這本書.
其中一篇的名字叫做《教會的衰人》.
可否請作者John親口讀一段給我們聽.
教會裡有衰人嗎?.
當然有.
多嗎?.
這算不算是你自己的問題?.
視乎你是否一個容易受傷的人.
不過要說像電影般那種徹頭徹尾.
傳言敗壞魔鬼化身般的大壞蛋.
我想在教會卻幾乎沒有或只佔極少數.
大部分情況下.
很諷刺的是教會裡的衝突.
往往都是因兩個外著的人而起.
因此所謂教會的衰人.
大部分情況都是外著的衰人.
謝謝.
這比簽書更加好.
對著這樣的人.
即是外著的衰人.
其實大家都是宗教領袖的話.
我會想你真的搞砸了整個行業.
其實真相是甚麼?.
有法理財人和民事對於衰人的不接納.
涉涉私意的抗拒.

$^{401}$其實大大增加了報道的難度.
對著一些教外的人.
人們其實是怎樣看我們?.
有一天我看到網上有一個對話.
貼文的版主說.
耶穌如果今天開始.
你可以令我的小狗大小二便都會屈膝.
而不是屈膝地.
三個月後我就會開始回教會.
我看到一個這樣的貼文.
網民一說教會是最假的地方.
網民二說.
信就好了,回甚麼教會?.
然後這個版主回覆.
回而已,不代表要理會教會裡面的人.
網民三.
信不要緊,回教會才糟糕.
教會一大群罪人.
而且最糟糕的是.
他們真的沒有想過要改.
這個星期我收到一個朋友對我宣教的祝福.
他是一間超過90年歷史以上的教會工作.
他跟我分享.
Joanne,我會好好為你一家祈禱.
希望你們一家平平安安踏上宣教路.
教會其實是很可怕的地方.
比外面的世界更壞.
竟然是這樣.
不想這樣可以怎樣?.
正所謂三歲定八十.
我們留堂這間三歲半的新教會.
肯不肯用主耶穌所展示的方式去擁抱罪人?.
即是我們自己暫時未做到.
我們肯不肯容讓其他人去做而不發怨言?.
不私下議論,不攻擊.
我們和牧者一同歡喜吧.
牧者失去的羊已經找到了.
可不可以阿們?可不可以?.
我在留堂開始的時候.
我是年紀最輕的童工.

$^{441}$今天已經不是了,有很多後輩了.
但當時很深刻.
剛剛參加的時候.
我們正在製作一個Beta版的留人記錄系統.
這個系統記錄了認識了誰.
我們沒有記錄你來自哪間教會.
你不要跟我說.
有什麼原因離開教會等等.
這個記錄系統原先的構思很深刻.
記得這樣做.
我們所認識的留人是一朵花.
如果你有打電話給他.
和他聊天.
就像倒水一樣.
心目中想像的.
IT朋友應該做到的.
那朵花會長大.
如果跟他外出吃飯.
那朵花會再大一點.
我們這個Beta版就沒有延後了.
有這樣的構思.
我起初不太明白.
自從小朋友在疫情期間.
我們忍不住開始買遊戲機.
我就明白了.
這個概念很像遊戲中的育成遊戲.
你餵食物給他.
帶他外出玩.
他就會長大.
你喜不喜歡.
如果有高人可以幫我們繼續發展這個Beta版.
是一個很有心思的同行.
不知道為什麼有江湖傳聞.
跟我們說你們是不是每次約到一個會友吃飯.
才會有薪金.
我們真的有一個開會中.
有同行的傳達人問我們.
我說沒有.
但我想告訴你.
你現在回教會這個地方.

$^{481}$很想用心栽種每一個.
我們很想抱緊每一個受傷過.
或任何原因迷失的人.
雖然剛才很誇張.
有一個教會的童工對我的宣教祝福.
竟然說教會有多可怕.
但剛好是同一天.
我收到有留堂的弟兄姊妹跟我分享.
很感恩在新教會遇到你.
打破了我對弟兄姊妹舊有不好的想法.
大家對這段經文都很熟悉.
所以我不是想說什麼新的大道理.
我是很想和大家一起實踐.
耶穌尋找人的方法.
很想我們一起實踐.
耶穌擁抱罪人的方法.
所以回到今天的講題.
什麼問題呢?.
一直建議抱緊處理.
我會說教會的問題.
一直建議抱緊處理.
有一種處事方法是可以撇除行政.
撇除侍奉的崗位.
撇除角色身份.
單純地擁抱.
靜看著對方無言語.
仍然視覺安慰是可以的.
各位的花朵.
好像我們這個月題裡面.
彩繪畫的花朵.
花朵的串法和花朵一樣.
也和我們這個流人紀錄系統的心目中的花一樣.
很想和你們一起同行.
很想和你們一起做到.
一起有難題,有人際問題,有人事上受過傷害.
我們一起抱緊處理.
我們一起同心祈禱.
主耶穌,我們透過您所說的比喻.
我們也透過法利賽人和民事的反應.
我們知道主耶穌.

$^{521}$你很想在擁抱罪人的時候.
也很想擁抱法利賽人和民事.
如果這一刻上帝來光照我們.
我們發現我們的法利賽人指數已經窄界了.
我們的民事指數已經超標了.
求主您軟化此心使我寬裕.
不再只是抱怨.
不再只說目者.
回想自己曾經也是被主擁抱的時刻.
好像回應詩裡所說.
單純地和主耶穌同行.
很願意我們曾經被您擁抱.
我們也不難阻止上帝.
讓您藉著不同的人,不同的目者.
去擁抱更多迷失的羊.
我們同心的禱告.
求主您光照.
求主您進入我們的心.
教導我們.
祈禱奉耶穌基督的名義而來.
阿們.
\newpage



\section{提摩太後書 4:1-2-20220702}
\label{sec:o_rYmsKLSrs}
\textbf{【網上崇拜】仍未忘跟你約定假如沒有死|提摩太後書4\_1-2|20220702 [o\_rYmsKLSrs]}
\newline
\newline
連結: \href{https://youtube.com/watch?v=o_rYmsKLSrs}{\texttt{ https://youtube.com/watch?v=o\_rYmsKLSrs}} ~~~~ 語音日期: 2022-07-02 
\newline
\newline
\hyperref[sec:zMmzg_ext8I]{\small{< < < PREV SERMON < < <}}
~
\hyperref[sec:index_chronic]{\small{[返順時目]}}
~
\hyperref[sec:index_scriptual]{\small{[返順卷目]}}
~
\hyperref[sec:Lm8jyw8dJf8]{\small{> > > NEXT SERMON > > >}}
\newline
\newline
提摩太後書 4:1-2-20220702
\newline
\begin{longtable}{cl}
\hline
\hline
章節 & 經文 (和合本修訂版)\\
\hline
4:1 & \begin{tabularx}{0.7\textwidth}{X} 我在神面前,並在將來審判活人死人的基督耶穌面前,憑著他的顯現和他的國度鄭重地勸戒你: \end{tabularx} \\ \\ \relax
4:2 & \begin{tabularx}{0.7\textwidth}{X} 務要傳道;無論得時不得時總要專心,並以百般的忍耐和各樣的教導責備人,警戒人,勸勉人。 \end{tabularx} \\ \\ \relax
4:3 & \begin{tabularx}{0.7\textwidth}{X} 因為時候將到,那時人會厭煩健全的教導,耳朵發癢,就隨心所欲地增添好些教師, \end{tabularx} \\ \\ \relax
4:4 & \begin{tabularx}{0.7\textwidth}{X} 並且掩耳不聽真理,偏向無稽的傳說。 \end{tabularx} \\ \\ \relax
4:5 & \begin{tabularx}{0.7\textwidth}{X} 至於你,凡事要謹慎,忍受苦難,做傳福音的工作,盡你的職分。 \end{tabularx} \\ \\ \relax
4:6 & \begin{tabularx}{0.7\textwidth}{X} 至於我,我已經被澆獻,離世的時候到了。 \end{tabularx} \\ \\ \relax
4:7 & \begin{tabularx}{0.7\textwidth}{X} 那美好的仗我已經打過了,當跑的路我已經跑盡了,該信的道我已經守住了。 \end{tabularx} \\ \\ \relax
4:8 & \begin{tabularx}{0.7\textwidth}{X} 從此以後,有公義的冠冕為我存留,就是按著公義審判的主到了那日要賜給我的;不但賜給我,也賜給凡愛慕他顯現的人。 \end{tabularx} \\ \\ \relax
4:9 & \begin{tabularx}{0.7\textwidth}{X} 你要趕緊到我這裡來。 \end{tabularx} \\ \\ \relax
4:10 & \begin{tabularx}{0.7\textwidth}{X} 因為底馬貪愛現今的世界,已經離棄我,往帖撒羅尼迦去了;革勒士往加拉太去;提多往撻馬太去; \end{tabularx} \\ \\ \relax
4:11 & \begin{tabularx}{0.7\textwidth}{X} 只有路加在我這裡。你來的時候把馬可帶來,因為他在服事上於我有益。 \end{tabularx} \\ \\ \relax
4:12 & \begin{tabularx}{0.7\textwidth}{X} 我已經打發推基古往以弗所去。 \end{tabularx} \\ \\ \relax
4:13 & \begin{tabularx}{0.7\textwidth}{X} 我在特羅亞留給加布的那件外衣,你來的時候要帶來,那些書也帶來,特別是那幾卷羊皮的書。 \end{tabularx} \\ \\ \relax
4:14 & \begin{tabularx}{0.7\textwidth}{X} 銅匠亞歷山大多方害我;主必照他所行的報應他。 \end{tabularx} \\ \\ \relax
4:15 & \begin{tabularx}{0.7\textwidth}{X} 你也要防備他,因為他極力抗拒我們的話。 \end{tabularx} \\ \\ \relax
4:16 & \begin{tabularx}{0.7\textwidth}{X} 我初次上訴時,沒有人前來幫助,竟都離棄了我,但願這罪不歸在他們身上。 \end{tabularx} \\ \\ \relax
4:17 & \begin{tabularx}{0.7\textwidth}{X} 惟有主站在我身邊,加給我力量,使我能把福音完整地傳開,讓所有的外邦人都聽見;我也從獅子口裡被救出來。 \end{tabularx} \\ \\ \relax
4:18 & \begin{tabularx}{0.7\textwidth}{X} 主必救我脫離一切的兇惡,也必救我進他的天國。願榮耀歸給他,直到永永遠遠。阿們! \end{tabularx} \\ \\ \relax
4:19 & \begin{tabularx}{0.7\textwidth}{X} 請向百基拉、亞居拉和阿尼色弗一家的人問安。 \end{tabularx} \\ \\ \relax
4:20 & \begin{tabularx}{0.7\textwidth}{X} 以拉都在哥林多住下了。特羅非摩病了,我把他留在米利都。 \end{tabularx} \\ \\ \relax
4:21 & \begin{tabularx}{0.7\textwidth}{X} 你要趕緊在冬天以前到我這裡來。友布羅、布田、利奴、革老底亞和眾弟兄都向你問安。 \end{tabularx} \\ \\ \relax
4:22 & \begin{tabularx}{0.7\textwidth}{X} 願主與你的靈同在!願恩惠與你們同在! \end{tabularx} \\ \\
[1ex]
\hline
\hline
\end{longtable}
$^{1}$弟子妹平安.
剛剛過去兩天.
我們用了最隆重其事的方式.
用了一個八號風球.
來慶祝我們香港特別行政區回歸十五週年.
我也覺得非常高興.
所以今天我們在現場裡.
有敬拜隊.
潘Sir和我.
來向各位Full Church的弟兄姊妹問安.
無論你是在香港.
在英國.
在北美.
在澳洲.
在台灣.
世界各地.
願耶穌基督的恩惠常與你同在.
流唐新一月的樂題是約定.
如果你剛才有留意的話.
剛才鋼琴伴奏Shirley彈的那首歌就是約定這首歌.
如果你認識的話.
證明你的年紀跟我差不多.
也有差不多成熟.
今天的講題也是來自於約定的歌詞.
仍未忘跟你約定假如沒有死.
約定這首歌是林夕為王妃寫的一首歌詞.
這首歌是在香港回歸前.
1997年2月出版的一首歌.
我記得那首EP叫做玩具.
我有買的.
那時候我讀中五.
坐車到旺角信和買了幾張CD.
那張CD的EP每首歌都很好聽.
第一首是暗湧.
第二首是約定.
第三首是敷衍.
然後是玩具.
最後是我信.
每首歌都很好聽.
懷疑我們今天Full Church的兩個字的樂題.

$^{41}$都是來自CD的概念大碟.
你會問為什麼Full Church七八月的樂題叫做約定.
我也不知道怎麼解釋.
我總覺得Full Church每個月的樂題的誕生.
是一件很神秘的事情.
你當作是上帝的安排.
不過要為約定下一個信仰的意義.
其實也不困難.
約定這兩個字其實都是聖經裡面的關鍵字.
約是上帝和人納的約.
是上帝信實行動的憑證.
定是上帝的預定.
上帝的安排.
所以無論是約還是定.
都提升著我們信仰永恆不變的價值.
上帝的應許.
上帝安排的命定.
特別在這個風起雲湧的年代.
所以在2022年7月1號.
第一個崇拜.
讓我們思想上主的約定.
我們一起來祈禱.
祝我們將崇拜Full Church每個弟兄姊妹的心交託給你.
讓我們能夠安靜.
讓我們能夠專注.
讓我們能夠聆聽你的說話.
讓你昔日透過保羅的說話提醒我們.
如何面對我們這個時代.
求主你幫助孩子能夠專心.
奉主命求 阿們.
今天我們說的是提摩太后書.
提摩太后書是一封監獄書信.
不過這不是今天想說的重點.
我們說的是提摩太后書是一封保羅晚年的書信.
我所說的晚年不是一般的晚年.
而是保羅臨死前的晚年.
雖然這都是學者的估算.
提摩太后書的寫作年份大概是68,69年左右.
最早最少也有64年時間寫的.
可能你不知道提摩太前書和提摩太后書.

$^{81}$雖然是兩卷前後腳的書卷.
但兩卷書其實不是連續的書卷.
兩卷是不同年代寫成的書信.
提摩太前書是保羅第一次在羅馬坐牢寫的.
大概在公元61年左右寫成.
最遲也不超過63年.
而提摩太后書是在68,69年寫的.
甚至乎是在70年代寫的.
所以兩封書信其實是隔了大概七至八年的時間.
提摩太前書和提摩太后書之間相隔了64年.
如果你對初來九代歷史有認識的話.
公元64年,64是一個極之重要的年份.
64年羅馬大火.
從此以後羅馬帝國就開始迫害基督徒.
從此出現了一個極差的年代.
正式宣告彌臨.
所以提摩太後書和提摩太前書是截然不同的書信.
雖然都是教學書信.
但提摩太前書保羅仍然會寫很多有關教會的實務.
一些向提摩太的事公的交代.
一些真理教義的教導.
提摩太後書卻是很不同的.
提摩太後書是一封很個人的書信.
提摩太後書記載了很多保羅自己很個人的留言.
不單止是純粹的教導.
因為保羅要教的都教了.
所以保羅寫給提摩太純粹是個人感情上的叮囑.
是很豐富的鼓勵.
每一句都是意中心詳.
每一句都是生命的剖白.
每一句都是生命最後的一些流露.
因為提摩太後書是保羅在聖經裡面最後一卷的書信.
今天我們就去看這段經文.
提摩太後書第四章.
正是最後一卷書信的最後一章.
保羅書信的最後.
保羅對提摩太叮囑的最後.
保羅生命的最後.
我們看一下經文.
提摩太後書第四章到二節.

$^{121}$保羅說我在神面前.
並在將來審判活人死人的基督耶穌面前.
憑著祂的顯現和祂的國度祝福地.
務要傳道無論得時不得時.
總要專心並用百般的忍耐各樣的教訓.
策避人警戒人勸勉人.
這段大家都很熟悉的經文.
特別是那句金句.
務要傳道無論得時不得時.
總要專心.
我記得以前我上教會.
傳道人經常來到很喜歡曲這句金句.
務要傳道無論得時不得時.
總之他們曲這句金句的意思是什麼.
就是什麼時候都要傳福音.
傳到什麼時候都要傳.
傳不到的時候都要傳.
不斷地傳.
吃飯睡覺上班都要傳.
不斷地傳就對了.
不知道大家有沒有這種感覺.
我小時候覺得.
整個金句就像保險公司的標語.
總之不斷地傳就可以了.
當然道理是真的.
務要傳道無論得時不得時.
總要專心.
但是道理是真實的.
但是這句經文要講的時候.
我們很容易忽略.
保羅在整段說話背後.
背後有一個很巨大的背景.
一個你不能忽視的背景.
這個背景就是提摩太后書.
其實是保羅臨死之前寫的一封書信.
保羅知道自己將要離開世界.
所以向提摩太后說出這番話.
你不需要看後面的經文.
你看後面一點點就能看到.
在第四章第六節.

$^{161}$保羅這樣說.
我現在被囂顛.
我離世的時候到了.
那美好的帳我已經打過了.
當跑的路我已經跑盡了.
所信的道我已經守住了.
從此以後有公義的官免為我存留.
保羅寫這段說話.
因為他知道自己將要面臨死亡.
弟子們如果你細心去思想.
一個知道自己將要死亡.
這件事其實是一件很難以形容.
一件很不可思議的事情.
你不單止要死.
你更加知道將要死.
一件我們從來沒有遇見過.
也不容易去體會.
就算遇見也只能遇見一次這麼多的事情.
不知道那一點.
如果你知道自己將要死.
你會怎樣理解你自己這一刻的生命.
我最近看YouTube.
看了一段有關安樂死的短片.
是一個日本人.
一個大概五十多歲的女士.
患上了末期癌症.
很痛苦很痛苦.
癌細胞擴散到全身.
於是她不想去纏累別人.
纏累自己的親人.
減輕自己的痛楚.
她就選擇安樂死.
不過安樂死在日本是犯法的.
於是她和她的朋友一起遠赴到瑞士.
坐飛機訂酒店.
聯絡當時安樂死的醫院.
花了好幾十萬.
一群人到了安樂死的醫院.
住在病房裡.
當地的醫生就和她有三天的冷靜期.

$^{201}$叫她再三考慮.
是否真的選擇安樂死.
最後這位女士仍然決定這樣做.
然後到了安樂死那天.
家裡的親人和女士的朋友.
一起在病房裡.
醫生就簡單介紹整個安樂死的程序.
醫生就將安樂死的注射液.
即是毒藥.
掛在床邊倒吊的藥物袋裡.
把按鈕交給女士.
就像床頭開關一樣的按鈕.
醫生就跟她說.
只要輕輕一推.
安樂死的毒液就會注射到血液裡.
不夠幾分鐘.
就會告別這個世界.
整個YouTube就記錄了整個死亡的過程.
這位女士就將按鈕一推.
幾分鐘的時間.
她跟她的朋友道別.
道謝.
她的朋友哭笑為她高興.
依依不捨地告訴她.
很快就可以離開這個世界.
再見.
但今天不是討論安樂死對與錯問題.
而是說當這位日本女士推開按鈕後.
那三分鐘的震撼.
那三分鐘房間裡的氣氛.
是奇怪到一個地步.
笑容,眼淚,死亡,生命,道別.
什麼都有.
一個將要死亡的人.
一個知道自己將要死亡的人.
並且正正面對自己死亡的人.
這個人似乎明白了一些東西.
潘Sir之前介紹過Swing樂隊的一首歌.
叫《那邊見》.
黃偉文填詞.

$^{241}$歌詞正正是講死亡的事情.
一首歌開始這樣寫.
有些人太早了斷.
有些人去得太突然.
有些人看到了光線踏前.
更多人看不見.
我們每個人都面對著死亡.
每個人都逃不過天然的力量.
每個人都面對著死亡.
只能謙卑.
但卻不是很多人放死亡在自己面前.
今天我們這個社會很輕視死亡.
雖然我們看到很多死.
電影也有,打機也有.
一個炸彈砰一聲.
死了一半人.
打機打三國無雙的國草遊戲.
很多人死,打魔鬼兄弟,死了幾萬次.
死亡彷似是一件很輕易的事情.
好像很熟悉的事情.
但我們卻完全感受不到死亡應該有的重量.
如果我們明白死亡的重量.
如果我們真的明白保羅在監獄裡將要面對死亡.
並且知道自己將要面對死亡.
我們再讀保羅這段話.
或許可以體會更多的東西.
更加明白這段經文不是純粹.
叫你去傳福音,像保險一樣去推銷東西那麼簡單.
保羅說,我在神面前.
並在將來審判活人死人的基督耶穌面前.
憑著他的顯現和他的角度祝福你.
務要傳道,無論得時不得時,總要專心.
你會發現保羅花了一整節的經文.
去將他自己要說的那句話儲起來.
你會發現保羅兩節經文.
正正正正要說的內容其實在第二節.
保羅的第一節其實是一段很長很長的儲氣的說話.
我在神面前.
並在將來審判活人死人的基督耶穌面前.
憑著他的顯現和他的角度去祝福你.

$^{281}$保羅隆重其事.
奉上帝的名.
奉基督耶穌的名.
憑著耶穌基督的顯現.
憑著他的角度去祝福提摩提.
等於我們要記住.
當保羅,每次保羅在聖經裡面隆重其事.
去奉主的名去說話,你要聽.
因為這不是普通的事情.
是一個極度嚴肅的事情.
我們不會隨隨便便奉主的名去說話.
你不會看到我無緣無故奉主的名叫你做事.
我不會奉主的名.
我告訴你,待會新彭福壽居在第二號房.
我不會奉主的名.
我們呼出的號碼是62745377.
我不會奉主的名叫你靈修祈禱或者返崇拜.
雖然都重要,但這些都不會奉主的名去說.
但保羅是在生命的最後.
儲了一節聖經的戲.
去奉主的名去叮囑提摩提.
保羅不單單奉基督耶穌的名.
他強調是奉將來審判活人死人基督耶穌面前.
憑著他的顯現,他的國度去叮囑提摩提.
保羅說話的重點是要指向一切的最後.
不單只是他生命的最後.
更加是世界的最後.
保羅邀請提摩提.
邀請他和一起在這個將要審判活人死人的主面前.
在他的顯現面前,在他的國度面前去說這番話.
今天我想說的重點就是這個.
學習用結局去理解我們當下的生命.
學習從結局中去理解我們作為基督徒.
作為教會繼續存在的理由.
我自己侍奉的神學院.
今年有三位老師退休.
分別是梁家倫院長,馮耀榮牧師,張美眉牧師.
梁家倫院長和我的關係.
從我入職開始到今天Full Church成立.
一直都在支持我這個後輩.

$^{321}$馮牧師是我自己很欣賞的牧師.
我甚少會追一個人的講道.
但我肯定會追馮耀榮牧師的講道.
一定會聽.
張美眉牧師更加是我母會的牧師.
我打拼了之後在教會做方案幹事.
第一個工作的老闆就是他.
這個月我參加了好幾次學院裡搞的退休聚會.
聽他們分享.
聽他們回望自己幾十年的侍奉.
很多很多的體會,反省.
最大的領悟是甚麼呢?.
就是覺得退休真的好.
大家不要誤會.
我說退休真的好不是因為不用工作.
而是我發現當一個人退休的時候.
當一個人去到退休那一刻的時候.
他能夠站在終點那裡去看回自己.
他終於都能夠明白自己一生裡的侍奉.
究竟是一回甚麼的事情.
我聽他們的分享.
回望自己的侍奉.
哪年哪年經歷過甚麼.
最難忘的是甚麼.
三位老師都清一色.
有一種很一樣的態度.
那種很悠然自得.
輕鬆容易.
easy peasy.
感恩的那種從容.
即是說那種從容不是裝扮出來的.
也不是純粹客氣,抱起不抱憂.
而是如果你能夠在終點裡去回望自己走過的那條路.
你就會發現無論自己過去的經歷是甚麼.
遭遇過甚麼.
有多麼傷心.
有多麼憤慨.
你都會發現更辛苦的都不是那麼辛苦.
你會發現你終於可以說.
一切都是天父上帝的安排.

$^{361}$一切都是上主的約定.
可能我在其他的場合都說過.
我自己是1999年被上帝呼召成為傳道人.
當時在讀大學一年班.
我都說過.
我發覺自己蒙召於太平的年代.
裝備於太平的年代.
他是侍奉於亂世.
在我蒙召的時候.
我沒有想過自己在今天的香港裡去侍奉.
不過雖然我不知道.
但是上帝知道.
早在1999年上帝呼召我的時候.
上主一早就知道.
他知道2019年香港發生了甚麼事.
上帝知道他要呼召這個人.
終於在2022年香港裡侍奉他.
甚至乎1999年的時候上帝一早就知道.
當時仍然是特首的董伯伯.
有一天會無故腳痛.
接他班的人無故坐牢.
然後接下來的那些人.
689,777,比卡超,Hello Kitty,Kirby那些.
他都知道.
在這樣的情況下上帝是呼召我.
上帝知道他要呼召這個人.
在2019年的時候.
膽大膽小地開了一間教會叫做Full Church.
2019年的時候我總以為自己會去宣教.
真的.
我那時候做恩賜侍奉表.
我是有宣教和殉道恩賜的.
沒有想過自己會讀神學博士.
沒有想過去德國.
沒有想過回神學院教書.
沒有想過會開教會.
各位將來審判活人死人的基督耶穌.
在時間的最後.
知道這個世界的發展和忠告.
在這個點上.

$^{401}$上帝真的在這個點上呼召我們.
作為傳道人.
可能你也一樣.
特別是對於今天在聽海外的頂尖妹來說.
你離開香港不是偶然的.
我知道離開香港是一個迫於無奈.
突如其來不由自處的決定.
但在上主的眼中.
祂站在時間的最後.
當上主創造你的時候.
當你小時候去信耶穌.
在中學缺志的時候.
上主就知道你下半生其實是怎樣.
頂尖妹,這是很真實的.
如果我們將我們的生命倒轉來看.
我們一生的年日都是上主和我們的約定.
雖然我們站在這一刻.
我們甚麼都看不清.
但從終點去回望.
從將來審判活人死人的舊主的角度去回望.
憑著祂的角度和顯現去回望的時候.
一切一切都不是偶然的.
一切都是上主的約定.
頂尖妹,我想說甚麼呢.
如果我們從終結終局回望現在的時候.
我們就明白保羅所說的.
無論得時不得時是甚麼意思.
大家都知道時機希臘文是kairos.
不時不時的希臘文就是eukairos和akairos.
eukairos就是好時機.
good season, good timing.
akairos就是壞時機,甚至是無時機.
out of season.
我猜你也同意.
timing, season是很重要的東西.
good season, bad season, good timing, bad timing.
農作物的收成,乳膚,捕魚,打獵,股票.
全部都是講求season和timing的東西.
考車才下雨,崇拜才打風.
這些是bad timing.

$^{441}$按著timing去做事,你就會事半功倍.
你不按timing去做事,你就會事倍功半.
雖然如此,如果從終點的角度來看.
從這個世界的終點來看.
這個世界的過程中所謂的good timing, bad timing, season, bad season.
其實並不重要.
無論你在過程中遇到的時機有多好,有多差.
如果從你的生命的終結,從世界的終結去看這個年代.
其實也沒有所謂的good和bad,對吧?.
我記得一年前我們也圍著電視機看張家朗.
男子花劍決賽.
最後張家朗15比11,勇奪奧運金牌.
其實在比賽的過程中,張家朗試過連續落後4分.
毅毅負責.
但如果你在冠軍獎台回望連續落後的4分.
這4分其實也不算什麼.
如果你是聖誕節球迷,你也記得2019年聖誕節球迷贏了歐聯.
四強作客第一回合,我們慘敗了巴塞隆拿3比0.
但同樣道理,當你看到利物浦贏了歐聯冠軍的時候.
四強作客輸了3比0,也不算什麼.
聽哲培,學習從終點去理解自己的人生.
你就懂得如何去面對好的時代,以及壞的時代.
那個將要在審判活人死人基督耶穌的面前的主.
那個將要顯現的國度必然會引領你的生命.
那位前前前任天主教教宗,仰望23世.
臨死前說了一句很美麗的話.
Any day is a good day to be born, and any day is a good day to die.
每一天都是出世的好日子,每一天都是死亡的好日子.
無論得時不得時,good season, bad season, good timing, bad timing.
從上主約定好的終點來看,去理解.
你發現good season, bad season,都是一樣的去面對.
說了這麼久,保羅其實是要說一個的重點,一個的行動我們要去做.
來到最後的最後,保羅叮囑提摩太,務要全道.
無論得時不得時,總要專心.
這個不是保險公司的slogan,而是我們面對時代的態度.
這是我們教會的使命.
和盤的翻譯是很奇怪的.
和盤,專心這個字,希臘文不是解作專心.
Apistethy這個字是解作stand near,stand by,站在旁邊的意思.
所以和專心是沒有什麼關係的.

$^{481}$保羅勸勉提摩太,既然面對這位將來審判沒人死人的基督耶穌.
無論是good timing, bad timing,作為教會,你就要站在這裡.
你就要站在這裡,總之你靠近他,站在這裡,站近他.
這種是否都站近他,靠近他的態度.
無論得時不得時,所謂的stand by,get ready的意思.
不過可能大家對get ready,stand by有些誤會.
我常常都以為stand by是那些我為你祈禱了十多年的friend.
什麼叫我為你祈禱了十多年?.
常常有人說,我為你祈禱了十多年.
你是不是十多年了?.
是不是可以什麼都不做,只祈禱十多年?.
不是的,你十多年來祈禱過一次,偶爾就說一下.
如果有個弟兄跟他說,我等你十多年,你不要相信他.
他不是十多年.
保羅所說的get ready,stand by不是那些friend.
所謂stand by只有兩個可能性.
stand by只有做和沒得做.
好的時機就做,壞的時機其實都做.
只是有人不讓做的意思.
所以無論是good or bad時機,你都要站近,靠近.
教會都要存在.
good timing當然做,bad timing不是不做,只不過是做不到.
講一個通俗一點的例子.
就好像這咕嚕咕新年財路德華說.
好牌爛牌都要打.
爛牌有爛牌的打法.
爛牌越爛的牌就越要用心打.
好牌當然可以吃糊.
爛牌其實都要爭取吃糊.
只不過時間不對,不到你說了算.
不過不要緊,反正最後你都知道.
去到最後當你踩足十八圈之後.
將來會贏的那個仍然是你.
因為基督耶穌已經是贏了.
所以無論是好牌爛牌,作為教會都一樣要照打.
頂智慧,我否認今天的香港,今天的教會是一個極差的年代.
副牌爛到是沒得再爛.
政局混亂,奸人當道,疫情政治,經濟低迷,移民.
我們連最後的信仰都沒有了.
很多頂智慧溫馨提醒我.

$^{521}$很多的說話現在不是能夠隨便說.
不是時候bad timing.
但無論如何,full church的存在不是純粹的圍爐取暖.
full church要繼續存在,作耶穌基督的見證.
因此各位留堂的頂智慧.
我願full church成為一個坐到磨爛脊.
坐到麻將館關門的教會.
由第一圈的東打到最後一圈的北.
無論是好牌爛牌,繼續打下去.
繼續告訴世上的香港人聽,耶穌基督是主.
有得說繼續說,沒得說不讓說,想辦法說.
轉個彎去說,等下個moment爆出來說.
用百般的忍耐,各樣的教訓.
輕鬆的,搞笑的,嚴肅的,責備人,警戒人,勸勉人.
各位full church的頂智慧,我們要好好的stand by.
就算坐在別人的下家,你都要繼續打下去.
別人誅死你,你都要繼續這樣打下去.
無論在香港,無論在海外,繼續為主耶穌基督打下去.
上帝仍然有說話要跟這個世代說.
我們仍然有可以傳揚的真道.
有叫人生命得著盼望的福音.
今日我只是在思想full church存在的理由.
我們每個禮拜存在的理由.
今日的場地沒有人.
下星期帶一個未信主的人回來.
帶一個很久沒有回教會的人回來.
這個是我們作為教會在香港的使命.
如果我們只是為了崇拜,為了自己聽道的時候.
我們沒有做到我們作為教會的本分.
真的,不知道你有多久沒有試過帶人信主.
不知道有多久沒有試過帶人回教會.
帶他回來,我有信心我們full church能夠承載他們.
站在這個時代的旁邊,隨時stand by.
隨時好好地傳揚基督耶穌.
主,我求你幫助我們,為我們教會祈求.
為我們教會在不同地方的弟兄姊妹去祈求.
叫我們不單單自己去聚會.
無論是多麼差的年代.
求主你讓我們教會成為一個介紹人.
去認識基督耶穌盼望的教會.

$^{561}$讓full church不是純粹一間敬拜得很好的教會.
而是一個真的可以叫香港人得著基督盼望的教會.
我們不單單是療傷,我們更加分享耶穌基督的美好.
幫助我們,讓我們在這麼差的年代仍然打下去.
仍然打這場美好的仗.
奉主命求,阿們.
\newpage



\section{使徒行傳 20:17-38-20220709}
\label{sec:XDQJvaySA3k}
\textbf{【網上崇拜】就算會與你分離......|使徒行傳20\_17-38|20220709 [XDQJvaySA3k]}
\newline
\newline
連結: \href{https://youtube.com/watch?v=XDQJvaySA3k}{\texttt{ https://youtube.com/watch?v=XDQJvaySA3k}} ~~~~ 語音日期: 2022-07-09 
\newline
\newline
\hyperref[sec:Lm8jyw8dJf8]{\small{< < < PREV SERMON < < <}}
~
\hyperref[sec:index_chronic]{\small{[返順時目]}}
~
\hyperref[sec:index_scriptual]{\small{[返順卷目]}}
~
\hyperref[sec:CzS_E_B5XMA]{\small{> > > NEXT SERMON > > >}}
\newline
\newline
使徒行傳 20:17-38-20220709
\newline
\begin{longtable}{cl}
\hline
\hline
章節 & 經文 (和合本修訂版)\\
\hline
20:17 & \begin{tabularx}{0.7\textwidth}{X} 保羅從米利都打發人往以弗所去,請教會的長老來。 \end{tabularx} \\ \\ \relax
20:18 & \begin{tabularx}{0.7\textwidth}{X} 他們來了,保羅對他們說:「你們自己知道,自從我到亞細亞的第一天,我怎樣跟你們相處, \end{tabularx} \\ \\ \relax
20:19 & \begin{tabularx}{0.7\textwidth}{X} 怎樣凡事謙卑,以眼淚服侍主,又因猶太人的謀害經歷試煉。 \end{tabularx} \\ \\ \relax
20:20 & \begin{tabularx}{0.7\textwidth}{X} 你們也知道,凡對你們有益的,我沒有一樣隱瞞不說的,或在公眾面前,或在每一個人的家裡,我都教導你們, \end{tabularx} \\ \\ \relax
20:21 & \begin{tabularx}{0.7\textwidth}{X} 不論猶太人和希臘人,我都已證明他們當在神面前悔改,信靠我們的主耶穌。 \end{tabularx} \\ \\ \relax
20:22 & \begin{tabularx}{0.7\textwidth}{X} 現在我被聖靈催迫要往耶路撒冷去,雖然不知道在那裡會遭遇甚麼事, \end{tabularx} \\ \\ \relax
20:23 & \begin{tabularx}{0.7\textwidth}{X} 但知道聖靈在各城裡向我指證,說有捆鎖與患難等著我。 \end{tabularx} \\ \\ \relax
20:24 & \begin{tabularx}{0.7\textwidth}{X} 我卻不以性命為念,只要走完我的路程,完成我從主耶穌所領受的職分,為神恩典的福音作見證。 \end{tabularx} \\ \\ \relax
20:25 & \begin{tabularx}{0.7\textwidth}{X} 「我素常在你們中間到處傳講神的國;現在我知道,你們眾人以後不會再見到我的面了。 \end{tabularx} \\ \\ \relax
20:26 & \begin{tabularx}{0.7\textwidth}{X} 所以我今日向你們作證,你們中間無論何人死亡,罪不在我。 \end{tabularx} \\ \\ \relax
20:27 & \begin{tabularx}{0.7\textwidth}{X} 因為神一切的旨意,我並沒有退縮不傳給你們的。 \end{tabularx} \\ \\ \relax
20:28 & \begin{tabularx}{0.7\textwidth}{X} 聖靈立你們作全群的監督,你們就當為自己謹慎,也為全群謹慎,牧養神的教會,就是他用自己血所買來的。 \end{tabularx} \\ \\ \relax
20:29 & \begin{tabularx}{0.7\textwidth}{X} 我知道,在我離開以後必有兇暴的豺狼進入你們中間,不顧惜羊群。 \end{tabularx} \\ \\ \relax
20:30 & \begin{tabularx}{0.7\textwidth}{X} 就是你們中間也必有人起來,說悖謬的話,要引誘門徒跟從他們。 \end{tabularx} \\ \\ \relax
20:31 & \begin{tabularx}{0.7\textwidth}{X} 所以你們要警醒,記念我三年之久,晝夜不斷地流淚勸戒你們各人。 \end{tabularx} \\ \\ \relax
20:32 & \begin{tabularx}{0.7\textwidth}{X} 現在我把你們交託給神和他恩惠的道;這道能建立你們,使你們和一切成聖的人同得基業。 \end{tabularx} \\ \\ \relax
20:33 & \begin{tabularx}{0.7\textwidth}{X} 我未曾貪圖一個人的金、銀或衣服。 \end{tabularx} \\ \\ \relax
20:34 & \begin{tabularx}{0.7\textwidth}{X} 你們自己知道,我靠兩隻手工作來供給我和同工的需用。 \end{tabularx} \\ \\ \relax
20:35 & \begin{tabularx}{0.7\textwidth}{X} 我凡事給你們作榜樣,叫你們知道應當這樣勞苦,扶助軟弱的人,又當記念主耶穌的話,說:『施比受更為有福。』」 \end{tabularx} \\ \\ \relax
20:36 & \begin{tabularx}{0.7\textwidth}{X} 保羅說完了這些話,就和大家跪下來禱告。 \end{tabularx} \\ \\ \relax
20:37 & \begin{tabularx}{0.7\textwidth}{X} 眾人痛哭,抱著保羅的頸項跟他親吻。 \end{tabularx} \\ \\ \relax
20:38 & \begin{tabularx}{0.7\textwidth}{X} 叫他們最傷心的,就是他說「以後不會再見到我的面」那句話。於是他們送他上船去了。 \end{tabularx} \\ \\
[1ex]
\hline
\hline
\end{longtable}
$^{1}$(主持人:各位姐妹平安).
(主持人:今天是約定這個月題的第二講).
(主持人:如果你有留意到這個講題).
(主持人:承接了上一個禮拜的講題).
(主持人:都是來自黃飛約定這首歌).
(主持人:今天如果你留意到經文的內容).
(主持人:其實都很貼切的).
(主持人:因為今天的經文內容是來自).
(主持人:《時代行傳》後面的篇章).
(主持人:對於如果你熟悉《時代行傳》的訊息的時候).
(主持人:你會看到).
(主持人:保羅的一幕一幕的行程的篇章).
(主持人:已經開始慢慢終結了).
(主持人:話別的說話開始說).
(主持人:對於六七月來說).
(主持人:都是頗為普遍的).
(主持人:因為之前在祈禱會也好).
(主持人:或者在不同的場合).
(主持人:大家都會感受到).
(主持人:有很多分離的場合).
(主持人:當舅師說約定這個月題的時候).
(主持人:這段經文對我來說是一個很大的提醒).
(主持人:其實面對人的分離).
(主持人:面對人的不同的處境).
(主持人:其實怎樣可以保持一段關係).
(主持人:或者怎樣可以繼續).
(主持人:信守一個承諾).
(主持人:其實一點都不容易).
(主持人:何況未來會怎樣).
(主持人:會不會再見到呢).
(主持人:更加說不定).
(主持人:但大家都會本著一個盼望).
(主持人:一個相信去繼續下去).
(主持人:這段經文是記載在).
(主持人:《四十行傳》第20章的後半段).
(主持人:經文的第一節是這樣說的).
(主持人:第17節是這樣說的).
(主持人:本來從米利都打發人往爾忽所去).
(主持人:請教會的長老來).
(主持人:這節是第17節經文).

$^{41}$(主持人:其實對於如果不是很熟悉).
(主持人:聖經的地理).
(主持人:或者是《四十行傳》的脈絡過程當中).
(主持人:其實很多人名都不熟悉的).
(主持人:就正如外國人來到香港).
(主持人:告訴你我們會經過大埔).
(主持人:再下一站就是那裡).
(主持人:其實大家都不會明白的).
(主持人:因為不會去過一個地方).
(主持人:就不會熟悉的地點).
(主持人:我以前教《四十行傳》的朱一學的時候).
(主持人:就和一班學生).
(主持人:就一路看著地圖).
(主持人:就一路和他講解《四十行傳》的篇章的時候).
(主持人:對於他們完了整季朱一學).
(主持人:他們很多經文都不記得).
(主持人:但就記得保羅的行程和脈絡在哪裡).
(主持人:對於我來說其實都是很重要).
(主持人:所以今天在講這麼大的篇幅的過程當中).
(主持人:又希望和大家重溫一下).
(主持人:其實保羅的行蹤或者宣教的經歷).
(主持人:對於我們今天有什麼提醒呢?).
(主持人:在開始的時候和大家看一看一幅截圖).
(主持人:其實保羅第三次的宣教旅程).
(主持人:是很…還確的範圍都很闊的).
(主持人:你光看地點).
(主持人:你會看到從安提拉開始出發到雅典).
(主持人:其實一點都不容易).
(主持人:不要說以前).
(主持人:就算是現在叫你在敘利亞這個地方).
(主持人:到雅典徒步的方法).
(主持人:其實都需要很多經歷).
(主持人:何況那時候沒有那麼多地圖).
(主持人:而交通工具又不算像現在那麼多選擇).
(主持人:更加沒有Google Map).
(主持人:何況現在有時候有人用Google Map都會看錯方向).
(主持人:但保羅正正去不同的方式的時候).
(主持人:都是用這個徒步).
(主持人:隨走隨存的方式去將耶穌就是基督).
(主持人:福音就是上帝的大能).

$^{81}$(主持人:去叫一切相信的).
(主持人:就透過不同的路程去經歷).
(主持人:甚至乎是經過很多的患難).
(主持人:所以花點時間去看看).
(主持人:其實保羅對於我們來說).
(主持人:今天我們面對全福音).
(主持人:面對很多困境).
(主持人:面對教會的運作很多難關的時候).
(主持人:其實保羅的生命正正就是幫我們有很多借鏡).
(主持人:甚至對我們來說).
(主持人:以前其實是靠什麼經歷).
(主持人:今天我們可不可以再重溫).
(主持人:甚至乎重現呢?).
(主持人:用一個圖去跟大家說).
(主持人:保羅在安提拉出發).
(主持人:去經歷宣教旅程的時候).
(主持人:首先他第一站會經歷的).
(主持人:就是迦拉提佛女教和彝伐所).
(主持人:大概是兩年的時間).
(主持人:在當中去建立教會).
(主持人:和去鞏固一班弟兄姊妹).
(主持人:在這段期間的時候).
(主持人:其實他不是沒有做其他事情).
(主持人:他很多時間不是講學).
(主持人:他都會寫信去處理).
(主持人:不同教會面對的問題).
(主持人:在當中你會看到).
(主持人:根據考古).
(主持人:他應該是寫過幾封信給哥倫多教會).
(主持人:面對哥倫多這間教會).
(主持人:這麼多是非和爭拗的地方).
(主持人:應該在我們收到的哥倫多前書的時候).
(主持人:或者我們現在看的哥倫多前書).
(主持人:在當中你會看到).
(主持人:論到之前寫給你們的).
(主持人:其實可能是哥倫多第一封信).
(主持人:現在保羅重述一個重要的事情).
(主持人:再講解清楚).
(主持人:可能他們不明白).
(主持人:又或者是曲解了的事實).

$^{121}$(主持人:這是我們看的哥倫多前書的內容).
(主持人:但其實在考古的過程中).
(主持人:都了解到一件事).
(主持人:可能中間其實保羅).
(主持人:試過去哥倫多教會處理的問題).
(主持人:但他們教會有紛爭).
(主持人:保羅就回到議會所當中).
(主持人:再要處理又寫信去).
(主持人:可能就是哥倫多第三封信的內容).
(主持人:但在這個過程中).
(主持人:保羅不只是停留在議會所).
(主持人:因為他聽到馬其頓的聲音).
(主持人:他繼續完成上帝給他的呼召).
(主持人:所以他繼續去西邊走的時候).
(主持人:下一站就會去到馬其頓第一個接點).
(主持人:就是要去希臘的方向).
(主持人:他就會經過一個港口).
(主持人:所以去到的過程中).
(主持人:他又遇到不同的困難).
(主持人:之前我在Full Church講到都講過).
(主持人:如果你記得).
(主持人:保羅去經歷過程中).
(主持人:其實遇到很多困難).
(主持人:但又被一個被誣陷的女孩).
(主持人:要得利搖錢樹).
(主持人:但保羅將她趕走之後).
(主持人:其實又出了其他問題).
(主持人:中間又出了很多).
(主持人:有些人不喜歡保羅).
(主持人:於是就去誣陷保羅).
(主持人:保羅就傳羅馬人不能夠接受那種福音).
(主持人:於是就叫多些人來迫迫保羅).
(主持人:保羅就連夜坐船走).
(主持人:經過暗飛玻璃).
(主持人:現在你見到是馬其頓其中一個港口).
(主持人:輾轉就到菲律賓這地方).
(主持人:保羅就在菲律賓寫哥林多後書).
(主持人:因為他主要想去哥林多處理問題).
(主持人:過程當中有些人未到).
(主持人:書信先到).

$^{161}$(主持人:以至希望可以調整一下).
(主持人:或者讓他們明白到你的原意是什麼).
(主持人:在耳筆鎖就寫哥林多後書).
(主持人:去到再落).
(主持人:其實保羅沒有停他的腳步).
(主持人:面對很多困難).
(主持人:很多大問題的時候).
(主持人:他繼續向南走).
(主持人:下一個階段就會去到不同的景地).
(主持人:最後就落到哥林多之前的雅典).
(主持人:下一步就會去到哥林多的地方).
(主持人:去到雅典大家都很熟悉).
(主持人:在阿里巴巴古山上和一些哲士辯論).
(主持人:那個微式之神).
(主持人:今天去到哥林多的時候).
(主持人:處理教會的問題).
(主持人:但同樣寫他的羅馬書).
(主持人:羅馬書我在Float Church也說過).
(主持人:幾次講到都是講羅馬書的內容).
(主持人:羅馬書不僅僅是大家一直以來).
(主持人:認識關於教義的叢述).
(主持人:但其實羅馬書是一個宣教的書卷).
(主持人:本來正正在哥林多這個混雜的地方的時候).
(主持人:是自羅馬軍權的統治之下).
(主持人:教會的設立).
(主持人:其實本意就是要宣揚上帝國的事情).
(主持人:所以羅馬書其實是一本宣教的書卷).
(主持人:因為第一章已經講得出).
(主持人:一個很重要的就是).
(主持人:福音本是上帝的大能).
(主持人:要叫一切相信的).
(主持人:整本書就是告訴你).
(主持人:人犯罪的習慣和它的機制是怎樣).
(主持人:然後就是福音是怎樣令到人).
(主持人:可以得聞一個赦罪的福音).
(主持人:以致到最後其實用了大概三分一的篇章).
(主持人:是講當你的新心靈都被更新的時候).
(主持人:你怎樣奉獻給上帝成為一個活祭).
(主持人:然後有一個新的生活形態).
(主持人:就是第十二章之後).

$^{201}$(主持人:十三至十五章的內容).
(主持人:然後十六章就是一個禱告).
(主持人:其實就是預表了一個很重要的訊息).
(主持人:就是他將會回去耶路撒冷).
(主持人:再開展他的宣教旅程).
(主持人:所以在那麼多教會).
(主持人:他寫完羅馬書之後).
(主持人:你會看到).
(主持人:保羅就沿路折返回頭).
(主持人:你會看到藍色的就是沿路折返回頭).
(主持人:又經過又去到耳忽所這個路程).
(主持人:去到耳忽所的路程的時候).
(主持人:你會看到今天出了).
(主持人:就是米利都這個地方).
(主持人:其實米利都就是一個中轉站).
(主持人:就是保羅走了一半).
(主持人:他在耳忽所那裡停一停).
(主持人:就是叫耳忽所的長老來).
(主持人:他將會要回耶路撒冷開大會).
(主持人:他在說希望在五旬節之前).
(主持人:能夠去到耶路撒冷和一班長輩).
(主持人:去講解他將會去另一個宣教旅程).
(主持人:為他完成上帝給他的使命).
(主持人:去到這個位置的時候).
(主持人:就會是今天不會覆蓋的讀經範圍).
(主持人:接著你會看到).
(主持人:教會知道他將要回耶路撒冷).
(主持人:面對很大的危險的時候).
(主持人:教會就很不開心).
(主持人:教會不想他去這個地方).
(主持人:不想保羅走這條路).
(主持人:但是保羅當然就不會聽).
(主持人:因為他知道他定義要做的流程).
(主持人:或者次序是什麼的時候).
(主持人:所以他繼續行程).
(主持人:最後你會看到).
(主持人:就會回到耶路撒冷).
(主持人:就是之後你看的21章的內容).
(主持人:很快跟大家講解).
(主持人:重溫了保羅在這些時間的經歷).

$^{241}$(主持人:當然有這個背景的時候).
(主持人:你看之後保羅的講論的時候).
(主持人:你會更加靠近).
(主持人:用了一點時間跟大家補底).
(主持人:重溫一下這些訊息的時候).
(主持人:希望你能夠再感受到).
(主持人:保羅和一班以弗所長老重溫).
(主持人:其實有些細節的事情我不多講).
(主持人:但是我講出來的時候你們是知道的).
(主持人:那段經文是什麼呢?).
(主持人:就是現在要讀出來的經文).
(主持人:他們來了保羅就說).
(主持人:你們知道自從我到亞細亞的日子以來).
(主持人:在你們中間始終為人如何).
(主持人:服侍主凡事謙卑眼中流淚).
(主持人:又因尤大人的謀害經歷事年).
(主持人:你們也知道凡與你們有益的).
(主持人:我沒有一樣被遺不說的).
(主持人:或在眾人面前或在各人家裡).
(主持人:我都教導你們).
(主持人:又對尤大人和希臘人證明).
(主持人:當向上帝悔改信靠我主耶穌基督).
(主持人:在經文中).
(主持人:保羅第一段要介紹給長老知道).
(主持人:其實我之所以來到這一刻).
(主持人:其實你們也知道).
(主持人:所以經文其中一些重點).
(主持人:就是他很希望那些已忽所的長老).
(主持人:或與他與會的弟兄姊妹明白一件事).
(主持人:其實你一直看著我做事).
(主持人:其實你也知道).
(主持人:保羅是如何做事).
(主持人:他是一個為人).
(主持人:那些人如何對待他).
(主持人:其實他面對不同的試煉).
(主持人:你也知道).
(主持人:但保羅整場都是咬緊牙筋去做).
(主持人:他要堅持上帝給他呼召要做的事).
(主持人:但凡是對教會和對人有益的事).
(主持人:他每件事都說出來).

$^{281}$(主持人:以及去教訓他們).
(主持人:他要證明一件事).
(主持人:就是上帝的能力會奉給他).
(主持人:他信的不是他自己).
(主持人:而是信差距離的主).
(主持人:對於這件事來說).
(主持人:你的經歷如何呢).
(主持人:在過去的日子).
(主持人:在運作教會).
(主持人:特別這兩年多).
(主持人:在香港運作教會的過程).
(主持人:面對社會的生態的改變).
(主持人:民生的改變 政策的改變).
(主持人:面對疫情的限制).
(主持人:不同運作的空間).
(主持人:面對人心的轉向).
(主持人:人的流動).
(主持人:其實面對很多很多的困難).
(主持人:有些事情不是你離開了就不會知道).
(主持人:其實有很多不同的途徑都會知道).
(主持人:今天信息很通達).
(主持人:只差你會不會繼續在一個群組).
(主持人:或者是離開群組).
(主持人:對於你來說).
(主持人:其實要你知道一些事情).
(主持人:其實不難的).
(主持人:但是你認不認同呢).
(主持人:或者你想不想知道呢).
(主持人:今天保羅回頭路回去耶路撒冷).
(主持人:他仍然秉承著一個福音的使命).
(主持人:希望繼續走下去).
(主持人:他知道有一件事是他仍然要完成).
(主持人:做愛邦使徒的債).
(主持人:他繼續走下去).
(主持人:哪怕期間有很多的不適).
(主持人:有很多困難).
(主持人:如果你想知道期間有多大的困難).
(主持人:剛才說到在整個行程當中).
(主持人:其實保羅寫過).
(主持人:最少我們知道).

$^{321}$(主持人:有七六的都有三封).
(主持人:一封是哥倫多前書).
(主持人:一封是後書).
(主持人:另外一個就是羅馬書).
(主持人:羅馬書比較少講他的患難).
(主持人:因為重點是講福音的大能).
(主持人:和上帝普世的宣教的工作).
(主持人:但是哥倫多前後書).
(主持人:記錄了很多保羅的辛酸).
(主持人:其中哥倫多後書).
(主持人:你會看到一個).
(主持人:納至侍奉上帝的人).
(主持人:他的路從來都不是容易走的).
(主持人:而不容易走的過程當中).
(主持人:其實他為什麼還要走下去呢?).
(主持人:和大家看兩段).
(主持人:哥倫多後書的經文).
(主持人:你會看到在哥倫多後書第十一章).
(主持人:他在講一個).
(主持人:我和大家編輯了幾節的經文).
(主持人:你會看到他有很多受苦).
(主持人:很多坐牢 很多受鞭傷的過程).
(主持人:而過程當中不是容易的).
(主持人:是打到他半死的).
(主持人:而路程又不是順利的).
(主持人:你看到船壞了).
(主持人:而在過程當中又被人打劫).
(主持人:又在過程當中又被人誣害).
(主持人:而在當中亦會遇人不淑).
(主持人:有假弟兄 漢子幫你其實是騙你).
(主持人:而過程當中亦會見到).
(主持人:在身心靈俱疲的時候).
(主持人:他仍然知道一件事).
(主持人:上帝給他馬其頓的這個呼召不是假的).
(主持人:因為他期間都見到很多).
(主持人:得聞上帝福音的人).
(主持人:得著解放 得著醫治).
(主持人:得著他有福音而有的喜樂).
(主持人:他就知道走這條路 縱然是困苦).
(主持人:但仍然看到果效).

$^{361}$(主持人:所以你會見到在後書第十二章).
(主持人:講了很多不同的陳述的時候).
(主持人:但十二章十二節講的是很重要的訊息).
(主持人:我在你們中間用百般的忍耐).
(主持人:藉著神跡 歧視 異能 顯出仕途的憑據來).
(主持人:因為那些人是在挑戰保羅的身份).
(主持人:你裝甚麼呢?).
(主持人:你不是耶穌安納的仕途).
(主持人:你這麼大支話 講這麼多幹甚麼?).
(主持人:你有甚麼能力?).
(主持人:你又不是入室的).
(主持人:其實你像甚麼?).
(主持人:但保羅不跟他們爭拗).
(主持人:也不會拿牌頭出來).
(主持人:他仍然知道 你問我 我會答).
(主持人:而答我是答得出 我是為甚麼緣故).
(主持人:但過程當中 他不是要跟人爭拗).
(主持人:他仍然按照上帝給他的時間表 一步一步地做).
(主持人:從來都不容易的).
(主持人:正如之前在講道也提及過).
(主持人:或者在不同的過程也提及過).
(主持人:今天現在在香港).
(主持人:甚至你認真做一個有成則 有原則的人).
(主持人:從來都不簡單).
(主持人:有很多人會挑戰你 你像甚麼).
(主持人:有很多人會問你 你為甚麼要這樣做).
(主持人:有很多人會質疑 你有多大的誠信).
(主持人:如果你要解釋完).
(主持人:就算你解釋完 他也可以不相信你).
(主持人:是否解釋完 你就會不做呢?).
(主持人:又或者他不相信你 你會不會動搖呢?).
(主持人:這是否我們作為一個認真的人).
(主持人:要不斷地跟自己說呢?).
(主持人:在過去這幾年).
(主持人:不斷地問自己為甚麼要繼續走這條路).
(主持人:不斷地問自己).
(主持人:每一個抉擇是否可以讓你選擇 都是選錯的).
(主持人:而過程當中有很多看似是 但其實不是的事).
(主持人:但你會否經一事長一智呢?).
(主持人:而在過程當中你會發覺).

$^{401}$(主持人:有很多你當初是同路的 慢慢就變成了陌路的).
(主持人:你當初是跟你同心的 最後又不跟你同心的).
(主持人:你如何面對這些情緒 你如何面對這些抉擇呢?).
(主持人:其實你再看保羅 你就會感受良多).
(主持人:保羅當初是一組人去宣教).
(主持人:最後因為意見不合就變成兩組人).
(主持人:但最後一群人想走那條路).
(主持人:但上帝又不開那條路 就走另一條路).
(主持人:但那條路就像剛才所說的 困難到不得了).
(主持人:伊弗所教會這次再遇到保羅).
(主持人:保羅跟他重溫).
(主持人:其實你要問 你都不用問).
(主持人:我經歷的事你們都知道).
(主持人:你跟我一起經歷了兩年).
(主持人:這兩年我遇到的困難 我跟那天不多教會的癡纏).
(主持人:你不是不知道的).
(主持人:我再說一次 都是那些事).
(主持人:所以經文下去的時候是甚麼).
(主持人:經文再下去的時候你會見到).
(主持人:他講了以前 他現在講現在).
(主持人:現在我往日又撒冷去 心神迫切).
(主持人:不知道在哪裡要遇見甚麼事).
(主持人:但知道聖靈在各城裡向我指正).
(主持人:雖有困所與患難等待我).
(主持人:我卻不以性命為念 也不看為寶貴).
(主持人:只要行完我的路程 成就我從主耶穌所領受的職事).
(主持人:證明上帝因為的福音 我素嘗在你們中間來往).
(主持人:傳講上帝國的道 如今我曉得).
(主持人:你們以後都不得再見我的面料).
(主持人:所以我今日向你們證明).
(主持人:你們中間無論何人死亡 罪不在我身上).
(主持人:因為上帝的旨意 我並沒有一樣 避諱不傳給你們的).
(主持人:你會見到在第二段 跟那些長老要說的話就是).
(主持人:接下來我知道要做些什麼 我知道這件事並不容易).
(主持人:因為最重要的是 耶穌跟我們說過的).
(主持人:就是祂離開 聖靈就是那個保衛師).
(主持人:祂會告訴我們很重要的訊息).
(主持人:所以你會見到 聖靈提醒什麼呢?).
(主持人:聖靈提醒一個很重要的知識就是).
(主持人:有困鎖在前面等著 但是保羅清楚).

$^{441}$(主持人:這條命是上帝救回我的).
(主持人:我不會指著這個是上帝給我的保障).
(主持人:我不走這條路 我仍然知道我會繼續走這條路).
(主持人:雖然你們知道你們不會再見到我的樣子).
(主持人:但是你們現在再見的時候 我希望你們明白到).
(主持人:我會將重要的事情繼續告訴你們).
(主持人:上帝旨意是什麼呢?).
(主持人:上帝旨意就是凡是對教會有重要的事情).
(主持人:祂都講得清楚明白).
(主持人:這就是保羅建立教會的原意).
(主持人:我希望當我們再看保羅的時候).
(主持人:我們常常都覺得保羅很了不起 我們沒有保羅那麼厲害).
(主持人:但是在過程當中我們不是要作出比較).
(主持人:但是你會看到無論多難都好 上帝的靈會讓我們有體見).
(主持人:上帝的靈也是讓我們有不同的職家力).
(主持人:在過去這三年 Flow Church 2018年8月開始暑期活動).
(主持人:第一炮就是回魂夜 其實6月底就開始籌備了).
(主持人:所以2018年的6月開始就是Flow Church的起步).
(主持人:但真正面對一個很大的挑戰其實是2019年6月開始的Flow Church).
(主持人:整個社會氣氛不同了 聚會的氣氛都不同了).
(主持人:我們講道甚至我們運作教會的方式都不同了).
(主持人:多了很多挑戰 多了很多不同的困難).
(主持人:2019年6月開始的Flow Church到2022年).
(主持人:這三年過程當中其實我們都面對很多困難 很多挑戰).
(主持人:有一條路是覺得再走下去是越走越窄的).
(主持人:但在過程當中我們看到上帝給我們一種肩負).
(主持人:上帝給我們一種能力 我們繼續去分辨 繼續去走下去).
(主持人:這個從來都不是一件易事).
(主持人:但我們知道我們對信徒 對教會要做的堅持).
(主持人:我們沒有一件事是避諱不說的).
(主持人:或許你之前沒有參加過Flow Church).
(主持人:參加過Flow Church崇拜 或者都不會知道).
(主持人:之後想加入Flow Church小組的弟兄姊妹).
(主持人:上Info Group的時候 第一課我就會講得很清楚).
(主持人:Flow Church是一間教會 現在教會有些事我們一定會堅持).
(主持人:而最重要就是 凡是動搖教會根基柱石的事情).
(主持人:我們一定會去堅守 凡是有機會打差的時候 我們一定會執政).
(主持人:無論是聖餐 無論是聚會的空間 怎樣可以堅持每個星期崇拜).
(主持人:這都是我們很重要的事情).
(主持人:但我們要讓每一位加入Flow Church的弟兄姊妹 要讓他們明白到).

$^{481}$(主持人:香港有很多間教會 為什麼你要來).
(主持人:而來的時候 我希望你選擇的時候 你清楚要選擇明白).
(主持人:但凡是重要的事情 我們沒有一件事是忌諱不說的).
(主持人:因為我們要證明那是上帝福音的所在).
(主持人:這都是我們希望 保羅昔日如是 今天我們建立教會如是).
(主持人:這是上帝給我們的約 我們怎樣可以不持守呢).
(主持人:這是上帝給我們的命定 我們怎樣可以不堅持呢).
(主持人:這就是保羅昔日傳講在教會當中 今天在Flow Church仍然繼續去持守).
(主持人:所以到了第三個段落經文的時候 你會看到聖靈出現).
(主持人:聖靈立你們作全群的監督 你們就當為自己謹慎).
(主持人:也當為全群謹慎 目仰上帝的教會就是他用自己血所買來的).
(主持人:我知道 我去後必有兇暴的豺狼進入你們中間 不外釋羊群).
(主持人:就算你們中間也必有人起來說撥謬的話 要引誘門徒跟從他們).
(主持人:所以你們應當警醒 紀念我三年之久 晝夜不住地流淚 勸誡你們各人).
(主持人:我自己看到這段經文的時候 其實就很感受到歷史再一次重現).
(主持人:怎樣再重現呢?).
(主持人:你會看到經文的keyword就是聖靈的出現).
(主持人:聖靈做了什麼呢?).
(主持人:聖靈提醒我們要謹守 而我們就是一群目仰上帝教會的人).
(主持人:聖靈的出現是什麼時候?).
(主持人:你會看到在約翰福音 我自己也說過三次).
(主持人:關於聖靈在黑西瑪利亞之先 耶穌在如字晚餐).
(主持人:就是約翰福音第十三十四十五章這三章裡).
(主持人:聖靈的出現就是上帝 特別是耶穌特別跟那群門徒說).
(主持人:保衛師的出現就是鞏固我們的信心).
(主持人:聖靈(Para Kratos) Para就是圍著Cleo說話).
(主持人:聖靈就是那個訓衛師 他會圍著對我們訓練).
(主持人:圍著去安慰我們 對我們說話的位置).
(主持人:所以他提醒我們有些事要謹慎).
(主持人:而過程當中是幫助我們去做分辨的能力).
(主持人:而你會看到保羅提醒).
(主持人:你們中間一些人說的一些東西是不對版的).
(主持人:他會誣蔑你們 會唆擺你們 會引誘你們).
(主持人:其實這些說話早在耶穌年代也有).
(主持人:但是在保羅自己 剛才我跟大家重溫的時候).
(主持人:你會看到他也有在賈弟兄當中去陷害保羅).
(主持人:所以將來也有 現在也有).
(主持人:賈弟兄從來也在教會出現 不是以前有現在沒有).
(主持人:末世就是有賈弟兄當中 末世就是要我們分辨).
(主持人:當然我不是叫你發神經).

$^{521}$(主持人:就是讓你四處找找 看看是誰).
(主持人:不是傻強那種 那個是不是臥底 那個就是臥底).
(主持人:不是叫你發神經不斷問別人).
(主持人:我們有多敏感 給我們分辨的能力).
(主持人:又或者我們有多熟悉 上帝透過他之前的用人).
(主持人:怎樣謹慎自守 這個是做好).
(主持人:所以先做好自己的本份 而不是叫你質疑對方).
(主持人:我們先做好自己的本份 叫我們明白到).
(主持人:其實上帝給我們的位份 我們在整個團隊).
(主持人:整個配搭當中的合作是什麼).
(主持人:求主你幫助我們 在這幾年當中).
(主持人:我們面對很大很大的挑戰).
(主持人:我們面對很多人事的變動).
(主持人:我們怎樣去分辨 怎樣去欣賞 怎樣去合作).
(主持人:本來這三年 其實他走訪不同的教會).
(主持人:但同樣都是面對很大流淚禱告和勸勉各方各地的人).
(主持人:剛才我說歷史是重演的 為什麼呢).
(主持人:因為你會見到在預節晚餐的時候).
(主持人:其實耶穌是不斷地跟門徒說).
(主持人:當時猶大就出去了 跟那十一個門徒說).
(主持人:詳細你可以看《約翰福音》第十五章).
(主持人:耶穌那裡有三節經文提醒那班門徒).
(主持人:你不會再見到我 你不會再見到我 你不會再見到我).
(主持人:但耶穌就告訴你 聖靈會幫你 聖靈會幫你 聖靈會幫你).
(主持人:而在過程當中就告訴你 總會有很多困難的).
(主持人:總會有很多難處的).
(主持人:其中一段經文就是).
(主持人:在世上你們有苦難 但你們可以放心 因為我勝過世界).
(主持人:這是《約翰福音》第十五章第三十三節下的經文).
(主持人:但《約翰福音》第三十三節上的經文就是).
(主持人:我留下平安給你們 在世上你們有苦難 但你們可以放心).
(主持人:耶穌是先留下平安 那種平安就是).
(主持人:聖靈會是一個勸慰師 會保衛我們).
(主持人:以致我們去面對苦難).
(主持人:所以耶穌在預言當中已經告訴我們).
(主持人:我們在末世做信徒不會是舒服的).
(主持人:不會是沒有困難的 不會是沒有凶險的).
(主持人:不會沒有人騷擾你的).
(主持人:但你真的要認真 聖靈會在我們當中).
(主持人:我們要敏感 我們要謹慎).

$^{561}$(主持人:上帝會為我們一班同行者).
(主持人:保羅正經歷了 我們現在正在經歷).
(主持人:這三年來香港太多事了).
(主持人:這三年來香港太多限制 太多變化).
(主持人:但在flowchurch的過程當中).
(主持人:每一次教infogroup 現在是教第十次了).
(主持人:每一次都是在重述過去).
(主持人:教會和香港社會如何同步 如何不同步).
(主持人:flowchurch如何在這當中掙扎 如何決定).
(主持人:如何繼續走下去).
(主持人:從來都不是我們有先知的能力).
(主持人:微卜先知那種).
(主持人:但我們仍然有一個先知 就是看到前科如何做).
(主持人:我們如何做).
(主持人:前科就是試圖保羅如何咬緊牙根 排除萬難).
(主持人:我們仍然要有一個先知去做).
(主持人:因為上帝讓我們建立教會這個約).
(主持人:而定義我們要走這段路的時候).
(主持人:這個約定豈不是我們今天在香港教會要繼續做下去嗎).
(主持人:我們只不過是將保羅昔日 一個受苦的先知).
(主持人:一個宣教的先知 繼續去下去).
(主持人:今天我們就是重現這個歷史在香港).
(主持人:其實不僅僅是香港).
(主持人:其實散居了在不同地方的弟兄姊妹).
(主持人:包括在網上直播的你).
(主持人:你每個星期都堅持和我們一起看崇拜).
(主持人:特別是在英國的弟兄姊妹).
(主持人:你實時和我們結集了一班弟兄姊妹).
(主持人:一起去小組 一起去團契 一起去崇拜).
(主持人:為的就是你知道你不孤單).
(主持人:每個星期約在一起).
(主持人:豈不是一個很重要 你特別要排過的時間嗎).
(主持人:你們在英國是下午黃金檔期).
(主持人:是約很多朋友玩的).
(主持人:但你們豈不是一點鐘就站在一起).
(主持人:應該說12點45分就站在一起).
(主持人:一點鐘就開始一起崇拜).
(主持人:然後你們就小組).
(主持人:我知道加拿大弟兄姊妹要更早起床).
(主持人:因為你們是早上).

$^{601}$(主持人:其實你們散居在不同地方).
(主持人:大家都站在一起 繼續約定時間一起崇拜).
(主持人:是否你們珍惜崇拜大家走放一起呢).
(主持人:不容易的 但這個堅持是上帝的喜悅).
(主持人:去到經文下一段的時候).
(主持人:你會看到保羅說最後的說話).
(主持人:如今我把你們交託上帝和他恩惠的道).
(主持人:這道能建立你們 叫你們和一切誠誠的人同得基業).
(主持人:我未曾貪圖一個人金銀衣服).
(主持人:我這兩隻手 上宮給我和同人的需用).
(主持人:這是你們自己知道的).
(主持人:我凡事給你們作榜樣).
(主持人:叫你們知道應當這樣勞苦 輔助軟弱的人).
(主持人:又當紀念主耶穌的話說).
(主持人:施比壽更為有福).
這段說話是保羅和長老最後的一句說話.
我希望你不要錯過重點.
就是保羅叫人奉獻 準備他去耶路撒冷.
不是.
因為我們很多時候將施比壽更為有福.
就讀入了教會呼籲人奉獻.
保羅不是這個意思.
在這段經文裡.
保羅要說的是.
去到最後要離開這班人.
最後要交托的是什麼呢.
他們很想保羅在這裡.
或者很想再見到保羅.
但是保羅告訴一個很重要的訊息.
從來都不是人的問題.
從來就是人安插在哪裡.
就是上帝在其中.
所以在這裡要帶出一個訊息就是.
保羅現在將二福所教會.
全眾去托付在上帝的道裡.
只有上帝的道才能夠令我們成聖.
我們一起去領受上帝的基業.
這是上帝給教會的命定.
上帝給教會的福分.
而我們怎樣做.

$^{641}$正如我剛才所說.
先做好自己.
保羅就是用自己的生命.
他的新教會告訴你.
他應該怎樣做.
不是貪圖物質上的供應或享用.
反而他要做的就是.
他自己親力親為.
這也是我自己侍奉多年的心志.
無論我是做信徒領袖也好.
到我做傳道人也好.
我都是不斷告訴自己.
我做到的我都盡力去做.
我能夠參與的我都盡力去參與.
不是說其他弟兄姊妹不能幫忙.
不是說其他弟兄姊妹做不到這件事.
不是.
其實能夠一起共融一起參與.
其實是一個很美善的.
肢體本身就是各按各職.
在保羅的侍奉生涯當中.
他從來都沒有拿他的title出來.
從來都是大家彼此各按各職.
彼此一起協作.
做好那件事.
保羅常常都用他自己.
在一個怎樣的榜樣當中.
希望其他弟兄姊妹都成為一個新教的榜樣.
其實這個借鏡.
豈不是就是我上一個月提到.
「攬到底」.
就是耶穌他自己.
在臨離開的時候.
就是和他們洗腳嗎.
洗腳就是一個新教的表現.
洗腳就是一個愛的表現.
耶穌是先做一個愛的表現.
做一個服飾的表現.
跟著施班波一條新的命令.
就叫你們彼此相愛.

$^{681}$今天保羅同樣都是在做一個僕人領袖.
就是他自己怎樣新教.
怎樣不受俸祿.
怎樣去堅持他自己所做的事情.
為的事就讓其他弟兄姊妹一起去學習.
不是要拍齊的.
但是就按你自己的職份和能力.
做好你自己的崗位.
這個豈不是各按各職的尾線呢.
保羅提醒.
這個重點就是.
你能夠施禦.
就是福氣.
所以施比壽就是你能夠施禦.
在參與過程當中.
就是大家一起有的福氣.
不是說你奉獻了.
我給了錢就算了.
不是說你貢獻了物資就算了.
能夠一起參與.
去能夠經歷上帝的恩典.
就是福氣.
施比壽近來的福的時候.
其實是說一起參與施禦的過程當中.
是感受上帝的福氣.
這個就是上帝命定教會存在的所有.
七弟兄姊妹.
去到這裡.
你就會看到保羅三段的信息.
聽到這裡的時候.
那群信徒怎樣呢.
那群信徒就會看到那段經文.
就是保羅說完.
他們就跪下一起祈禱.
眾人痛哭抱著保羅的頸鶴和他親嘴.
叫他們最傷心的.
就是他們說以後不能再見我的面.
那句話於是就送他上樹.
就是剛才說到.
就是下去耶路撒冷的路程.

$^{721}$我不知道你聽到保羅的心聲的時候.
你自己會不會眼濕濕.
但我自己在這三年多經歷教會的.
不斷地變身.
不斷地有不同的限制.
不斷地有不同的挑戰.
無論是地方.
無論是聚會的條例.
無論是小組分聚會的空間.
甚至有一個製作間的時候.
又會沒有冷氣.
有很多東西是不斷地.
剛才才和弟兄姊妹一起分享.
我們很多時候都不是處理完一件事.
才有第二件事.
是還沒處理完一件事的時候.
又有第二件事.
但我經常說.
我認命的.
正正就是有得做就繼續做.
正正就是在過程當中.
知道這件事能夠接通弟兄姊妹的時候.
我相信仍然是上帝祝福我們的所在.
能不能見到面.
當然是想.
見不到面很不開心.
我兩個兒子.
之前看《洩露人傳》的時候.
看到這一段經文的時候.
我小兒子就說.
看到這一段的時候.
他們真的很慘.
不能夠再見到保羅.
真的很傷心.
對於香港人來說.
能夠感受到的.
他的同學都移民了.
或者周邊的人都會離開讀書.
上學雖然現在有上實體.
但上完學年之後.

$^{761}$他的同學都會離開.
不同年紀的人都會面對分離.
特別是這兩年.
是很靠近的.
不想說再見.
雖然轉了地方都可以FaceTime.
但很不容易的.
所以很多人都想說.
約定.
將來再見.
對於你來說.
你怎樣看這兩個字.
約定或者再見.
我自己都會說的.
因為我知道有些行蹤的話.
我會主動去約人.
但我其實在.
特別這年多.
很多人約我吃飯.
知道告訴我要離開.
又或者知道什麼時候離開的時候.
我最後都會跟他說.
未必會再見到你.
但我希望仍然是.
你找到一間合適的教會.
仍然有教會的生活.
上帝的約仍然在教會當中施行.
上帝的約仍然在你生命當中施行.
上帝的定義就是.
叫我們得生命.
並且得得更豐盛.
真正的約定.
不是約定見什麼人.
而是約定在上帝的生命當中.
這個約定才是最重要的.
主耶穌在赫西瑪利園.
為眾仕徒祈求.
我不是為世人祈求.
我是為你們祈求.
你們就是上帝交付給耶穌基督.

$^{801}$這個約定是最重要的.
保羅已經將上帝最重要的東西.
毫不忌諱告訴眾教會.
就是希望將眾教會的弟兄姊妹.
放在上帝恩惠的福音當中.
去屏行這個約定.
而不是沒有了保羅就不能作神魔.
所以保羅不在意再見不到那班弟兄姊妹.
是在意他們有沒有再在上帝的約當中.
親弟弟姊妹.
我們散居在不同地方是不要緊的.
但我們繼續去物色一個.
同信仰的群體.
同行的群體.
在這個約定當中繼續有教會生活.
在這個約定當中.
繼續秉承上帝給我們.
基督徒這個身份去見證祂.
這個是昔日保羅的心願.
是今天上帝給我們在教會的心願.
願意約定這個信息再一次去帶動你.
每個星期都約定和上帝見面.
每個星期都約定一起崇拜.
願上帝的恩惠的福音常永續同在.
貼近上帝的心意.
貼近上帝給我們的提醒.
貼近聖靈的勸誡.
貼近上帝的心.
\newpage



\section{路加福音 11:29-32-20220716}
\label{sec:CzS_E_B5XMA}
\textbf{【網上聖餐崇拜】誰可唱最後的信仰...|路加福音11\_29-32|20220716 [CzS\_E-B5XMA]}
\newline
\newline
連結: \href{https://youtube.com/watch?v=CzS_E-B5XMA}{\texttt{ https://youtube.com/watch?v=CzS\_E-B5XMA}} ~~~~ 語音日期: 2022-07-16 
\newline
\newline
\hyperref[sec:XDQJvaySA3k]{\small{< < < PREV SERMON < < <}}
~
\hyperref[sec:index_chronic]{\small{[返順時目]}}
~
\hyperref[sec:index_scriptual]{\small{[返順卷目]}}
~
\hyperref[sec:RWAsMtjQmZ4]{\small{> > > NEXT SERMON > > >}}
\newline
\newline
路加福音 11:29-32-20220716
\newline
\begin{longtable}{cl}
\hline
\hline
章節 & 經文 (和合本修訂版)\\
\hline
11:29 & \begin{tabularx}{0.7\textwidth}{X} 當眾人越來越擁擠的時候,耶穌說:「這世代是一個邪惡的世代。他們求看神蹟,除了約拿的神蹟以外,再沒有神蹟給他們看了。 \end{tabularx} \\ \\ \relax
11:30 & \begin{tabularx}{0.7\textwidth}{X} 約拿怎樣為尼尼微人成了神蹟,人子也要照樣為這世代的人成為神蹟。 \end{tabularx} \\ \\ \relax
11:31 & \begin{tabularx}{0.7\textwidth}{X} 在審判的時候,南方的女王要起來定這世代的人的罪,因為她從地極而來,要聽所羅門智慧的話。看哪,比所羅門更大的在這裡! \end{tabularx} \\ \\ \relax
11:32 & \begin{tabularx}{0.7\textwidth}{X} 在審判的時候,尼尼微人要起來定這世代的罪,因為尼尼微人聽了約拿所傳的就悔改了。看哪,比約拿更大的在這裡!」 \end{tabularx} \\ \\ \relax
11:33 & \begin{tabularx}{0.7\textwidth}{X} 「沒有人點燈放在地窖裡,或是斗底下,總是放在燈臺上,讓進來的人看見亮光。 \end{tabularx} \\ \\ \relax
11:34 & \begin{tabularx}{0.7\textwidth}{X} 你的眼睛就是身體的燈。當你的眼睛明亮,全身就光明,當眼睛昏花,全身就黑暗。 \end{tabularx} \\ \\ \relax
11:35 & \begin{tabularx}{0.7\textwidth}{X} 所以,你要注意,免得你裡面的光暗了。 \end{tabularx} \\ \\ \relax
11:36 & \begin{tabularx}{0.7\textwidth}{X} 若是你全身光明,毫無黑暗,就必全然光明,如同燈的明光照亮你。」 \end{tabularx} \\ \\ \relax
11:37 & \begin{tabularx}{0.7\textwidth}{X} 耶穌正說話的時候,有一個法利賽人請他吃飯,耶穌就進去坐席。 \end{tabularx} \\ \\ \relax
11:38 & \begin{tabularx}{0.7\textwidth}{X} 這法利賽人看見耶穌飯前不先洗手就很詫異。 \end{tabularx} \\ \\ \relax
11:39 & \begin{tabularx}{0.7\textwidth}{X} 主對他說:「如今你們法利賽人洗淨杯盤的外面,你們裡面卻滿了貪婪和邪惡。 \end{tabularx} \\ \\ \relax
11:40 & \begin{tabularx}{0.7\textwidth}{X} 無知的人哪!造外面的,不也造了裡面嗎? \end{tabularx} \\ \\ \relax
11:41 & \begin{tabularx}{0.7\textwidth}{X} 只要把杯盤裡面的施捨給人,對你們來說一切就都潔淨了。 \end{tabularx} \\ \\ \relax
11:42 & \begin{tabularx}{0.7\textwidth}{X} 「但是你們法利賽人有禍了!因為你們將薄荷、芸香,和各樣蔬菜獻上十分之一,疏忽了公義和愛神的事;這原是你們該做的—至於其他也不可忽略。 \end{tabularx} \\ \\ \relax
11:43 & \begin{tabularx}{0.7\textwidth}{X} 你們法利賽人有禍了!因為你們喜愛會堂裡的高位,又喜歡人們在街市上向你們問安。 \end{tabularx} \\ \\ \relax
11:44 & \begin{tabularx}{0.7\textwidth}{X} 你們有禍了!因為你們如同不顯露的墳墓,走在上面的人並不知道。」 \end{tabularx} \\ \\ \relax
11:45 & \begin{tabularx}{0.7\textwidth}{X} 律法師中有一個回答耶穌,說:「老師,你這樣說也把我們侮辱了。」 \end{tabularx} \\ \\ \relax
11:46 & \begin{tabularx}{0.7\textwidth}{X} 耶穌說:「你們律法師也有禍了!因為你們把難挑的擔子放在別人身上,自己卻不肯動一個指頭去減輕這些擔子。 \end{tabularx} \\ \\ \relax
11:47 & \begin{tabularx}{0.7\textwidth}{X} 你們有禍了!因為你們建造先知的墳墓,那些先知正是你們的祖宗所殺的。 \end{tabularx} \\ \\ \relax
11:48 & \begin{tabularx}{0.7\textwidth}{X} 可見你們祖宗所做的事,你們是證人,你們也贊同,因為他們殺了先知,你們建造先知的墳墓。 \end{tabularx} \\ \\ \relax
11:49 & \begin{tabularx}{0.7\textwidth}{X} 所以,神的智慧也曾說:『我要差遣先知和使徒到他們那裡去,有的他們要殘殺,有的他們要迫害』, \end{tabularx} \\ \\ \relax
11:50 & \begin{tabularx}{0.7\textwidth}{X} 為使創世以來所流眾先知的血的罪都歸在這世代的人身上, \end{tabularx} \\ \\ \relax
11:51 & \begin{tabularx}{0.7\textwidth}{X} 就是從亞伯的血起,直到被殺在祭壇和聖所中間的撒迦利亞的血為止。是的,我告訴你們,這都要向這世代的人追討。 \end{tabularx} \\ \\ \relax
11:52 & \begin{tabularx}{0.7\textwidth}{X} 你們律法師有禍了!因為你們把知識的鑰匙奪了去,自己不進去,要進去的人,你們也阻擋他們。」 \end{tabularx} \\ \\ \relax
11:53 & \begin{tabularx}{0.7\textwidth}{X} 耶穌從那裡出來,文士和法利賽人就開始極力地催逼他,盤問他許多事, \end{tabularx} \\ \\ \relax
11:54 & \begin{tabularx}{0.7\textwidth}{X} 伺機要抓他的話柄。 \end{tabularx} \\ \\
[1ex]
\hline
\hline
\end{longtable}
$^{1}$各位頂梓梅萬安.
剛才在YouTube的時候.
看到大家的留言.
有哪一家科的頂梓梅.
有英國的 有不同地方的.
香港的頂梓梅挺可愛的.
他說他們來自流浮山.
這是挺好笑的一個留言.
願神祝福無論在哪個地方的頂梓梅.
求天父的愛繼續常與你們同在.
我相信Fold Church不是一個死咕咕的組織.
我相信Fold Church會在不同的地方.
繼續流動 繼續流散.
讓每一個人可以在不同的地方體會上帝的愛.
我們今天的主題是約定.
約定這個題目其實是一個挺.
我去思考的時候覺得是一個挺特別的題目.
通常約定的時候我們都是跟別人說約定.
譬如我今天剛剛吃完一餐離別的飯.
所謂流散飯.
一個家庭認識了他們二十多年.
我們會臨走 臨離開要回來Fold Church過去前.
我會跟他說一句有機會在加拿大再見.
所以約定基本上我們會用來放在.
跟別人的約定.
很多時候我們都會覺得情侶上來說.
我們會覺得到老的時候.
我都約定跟你一起走下去人生的道路.
約定的感覺或者意思.
好像跟某一個人或某一個群體.
或者將來有一天 我在想的是.
香港會不會有一天可以在流散的人.
舉辦一個流散的回歸的聚會.
所有流散的人能夠回到香港.
這個是約定.
或者我們會很想.
離開的人會說其實他也很想念香港.
還沒離開已經很想念.
期望約定有一天能夠回來.
看一個不一樣的香港.

$^{41}$如果我們只說在跟別人一起約定的話.
其實我們會在焦點上會模糊了.
其實你和我都知道.
那些很浪漫的約定.
或者我們認為很重要的約定.
那些人一定會附會的.
但其實現實上我們知道.
很多時候約定不一定會成就.
正如今天的題目一樣.
我們不要以為唱《最後的信仰》的人.
一定會有最後的信仰一樣.
不用等很久.
這首歌唱了沒多久之後.
唱的人會告訴你.
是你們read多了.
是你們說多了.
詞不是我寫的.
我只是唱的.
所以約定 如果你再想深一層的話.
其實約定不是說你和人之間的約定那麼簡單.
其實約定真正在說的.
應該是在說自己和自己的約定.
我希望今天能夠看一段經文.
給我們一種感覺.
什麼叫做和自己的約定.
我希望我們能夠有些空間.
再思考一下.
到底在約定這個句子.
不是句子 是詞語裡面.
什麼才是關鍵.
是別人爽約.
別人不在是關鍵.
還是怎麼樣.
我們再說多一點.
我們再進入經文.
如果以色列人被擄之後要70年歸回.
你可以想像.
70年前約定了的人.
大體上都沒有回去.
70年來應該已經死了很多.

$^{81}$所以約定不一定是說.
一群人走在一起.
聖經說約定或者要回歸回去.
其實想說什麼.
我希望今天能夠看一段.
熟悉一點的經文.
我們能夠看一看.
我們下一節.
下一張powerpoint.
我們今天看的是路加福音第十一章.
29至32節.
這四節聖經是這麼說的.
第一句這麼說.
當眾人持續聚集的時候.
耶穌開口說.
這個世代是一個邪惡的世代.
他這個字其實是求記號.
他應該指世代的意思.
除了約拿這個記號之外.
再沒有記號給你.
如果你翻開和合本聖經的時候.
這個記號通常會譯作神跡.
但其實四尾韓的字不是一定解神跡.
more or less他解記號這個字.
我們會譯得好一點.
所以第29節他說.
有很多人聚集的時候.
耶穌說了一句話.
他說這個世代很邪惡.
這個世代不知道為什麼.
想見到一些很厲害很威風的東西.
好像一個記號那麼厲害的東西.
但耶穌很特別.
他說除了約拿的記號之外.
再沒有記號給你們.
這個世代.
我們再看第30節.
約拿說.
如何為你們未人成了記號.
人只要照樣為這個世代的人成了記號.

$^{121}$原文沒有記號.
所以我括了那個字.
意思是說耶穌基督.
都會成為這個世代的記號.
留心的第30節.
耶穌說除了約拿的記號之外.
已經沒有記號給你了.
30節他說什麼.
什麼叫除了約拿的記號.
沒有其他記號給你.
30節他怎麼說.
他說約拿怎樣成為了你們未人的記號.
這個說話我要再解釋多一點.
什麼叫記號.
為什麼耶穌說其實沒有記號給你看了.
除了約拿的記號.
到底約拿的記號是什麼.
第30節他說.
約拿成為了你們未人的記號.
如果一個邪惡的世代裡.
我們期望見到一些很大的事情.
很威嚴的事情.
很神蹟的事情.
我們期望的是什麼.
好像某些大型報道會.
在大球場.
一萬多人.
突然之間有一個講完呼嘲.
有一萬人走下去.
這些好像覺得很厲害的東西.
我們覺得這是一個記號.
又或者我們會覺得記號是什麼.
就是突然之間.
明明那個射火風球離我們只有五公里.
突然之間它轉頭.
180度轉向離開了.
我們覺得這些叫記號.
但你看看第30節這一句.
約拿成為了你們未人的記號.
如果你熟習一點約拿書的話.

$^{161}$其實整個約拿書裡面.
最沒有神蹟性.
最沒有奇形怪狀的事情出現.
就是約拿在你們面前說了一句話.
30天這個城就傾覆了.
這個說話沒有神蹟.
沒有歧視.
沒有任何很厲害的東西出現.
你說約拿書厲害的是什麼.
就是它坐在船上.
突然之間狂風大作.
但它掉下去之後.
突然之間風平靜了.
又或者第二章約拿書.
那條魚吞了約拿竟然.
三天之後它沒有吃東西.
吐了出來.
這些叫記號.
這些叫厲害的東西.
或者第四章.
那棵編麻樹突然之間枯萎.
這些起碼是一個.
我們會想成為記號的東西.
我們很少會覺得.
約拿說了一句話.
做了一件事情.
這個竟然成為了.
美麗美人的記號.
如果是這樣說的話.
弟兄姊妹.
什麼是記號.
而這個說的是.
人只要成為這個世代的記號.
其實我們想說的記號.
是說什麼的記號.
是想說耶穌基督五餅二魚.
我們想說耶穌基督的記號是.
醫治好男人成寡婦的兒子.
葉老的女兒.
或者拉薩路的復活.

$^{201}$這些是我們心裡想的記號.
但如果和這一節經文是平衡的話.
約拿做的事情.
就是一句話.
足以成為利利美的記號.
耶穌基督的記號.
並不是我們心中想的那些.
很誇張.
很死人復活.
大浪大海突然翻騰的時候平靜.
或者耶穌基督在水面上行走.
這些成為了我們心裡想的記號.
什麼是真正耶穌基督的記號.
什麼是約拿真正的記號.
我希望今天能夠多花點氣力時間.
再說多一點.
我再加一張31-32節.
你看看31-32節.
基本上兩節的經文是沒有平衡的.
當審判的時候.
南方的女王要起來聽這個世代的罪.
因為她從地極而來.
要聽所羅門的智慧看.
這裡有一個人比所羅門都大.
很明顯這個人是說人子.
就是說耶穌基督.
32節他說什麼.
一樣的.
32節.
當審判的時候.
利利美人要站起來.
要聽這個世代的罪.
還要約拿所傳的悔改.
這裡有一個人比約拿都大.
這兩節奇怪和特別的地方是.
南方的女王和利利美人.
會起來聽這個世代的罪.
她想說什麼.
我們很快去講解一下.
這兩節經文想說什麼.

$^{241}$其實這兩節經文想說的是.
無論是南方的女王和利利美人.
看到的記號.
那些記號可能是對於.
所羅門來說.
她想聽她智慧的言語.
對於南方的女王來說.
那個記號是.
所羅門從耶和華而來.
滿有智慧的說話.
所以對於南方的女王來說.
真正的記號也不是那些很奇怪的東西.
很誇張的東西.
正如約拿書所說的一樣.
雖然第一章第二章第四章.
好像有很多神蹟.
很多古怪的東西.
但真正這段經文要表達的第三章.
約拿輕輕說了一句話.
那句話彷彿好像.
來自耶和華.
去到所羅門給她智慧的言語一樣.
所以這四節的聖經.
主要的重點.
不是在說.
很多我們心中在想的神蹟奇事的東西.
不是不是.
她在說一些很平凡的東西.
她在說一些很簡單的東西.
正如聖經也沒有記載.
對於所羅門的智慧有什麼這麼厲害.
可能她判斷過兩個媽媽爭兒子的事情.
或者某些書卷.
可能是針研亞哥還是什麼.
可能是她所寫的意思.
她有智慧的解法.
但實際的智慧.
我們不是很掌握是怎樣的.
如果是這樣的時候.
我們不期而要問.

$^{281}$到底這兩個這麼平衡的結構.
不是在說很偉大的事情.
只不過是在說人口中說的話的時候.
這個記號是什麼意思.
我們再回到下一張powerpoint.
正如我開頭第一張powerpoint也是這樣說.
耶穌說這個世代裡已經沒有神蹟記號.
祂說只有一個.
那個記號叫約拿的記號.
約拿的記號其實我們叫做.
sign of Jonah 即是約拿的記號.
這個of Jonah 對於希臘文來說.
是一個genitive 即是所有格.
這個所有格的用法可以不同.
你可以把它當成一個possessive 擁有的.
即是約拿所擁有的記號.
但其實這個所有格genitive.
可以把它當做apposition.
意思是說.
這個記號其實等同於約拿本身.
意思是說除了約拿.
就沒有其他記號可以給你看.
所以更加證明了一件事.
在整個四字聖經裡.
祂不是要教人去做一些很奇妙的事.
是很神蹟的事.
即是突然之間有些事情就變化了.
沒有幻想 沒有期待.
如果你聽某些電台節目.
youtuber說的東西多的話.
評論香港的事情的話.
希望什麼事情發生就好了.
就會有什麼改變.
無論烏克蘭 俄羅斯.
如果你聽某些評論的話.
突然之間會發生什麼.
什麼事情就會好了.
我們這個世代和耶穌的世代一樣.
面對很多艱難的事情.
面對很多複雜的事情.

$^{321}$面對很多你控制不了的事情的時候.
我們期望好像Avengers那樣.
一隻手指一半的人死了.
我們期望有些事情突然之間.
彈指之間會改變.
我們覺得那些叫做浪費記號.
但聽到經文多次強調一件事情.
真正的浪費記號.
是那個人做了什麼很厲害的事情.
不是梭羅門打贏了別國的國家.
上帝用300人打敗了梭羅門附近的敵人.
不同的事件 打仗.
沒有這些記載.
我們以為第二章約拿書.
那麼多神蹟奇事.
到第三章 到利利美的時候.
應該有更多更厲害的神蹟奇事.
以至利利美人突然之間悔改才合理.
但最不合理的是.
整個第三章沒有神蹟和沒有奇事.
走三天的路程.
他走一天說一句話 收工.
真正的記號.
不是說那個人做了什麼很偉大的事情.
真正的浪費記號是什麼.
是那個人自己.
我們再看那張PowerPoint.
我們再發展下去.
我們下一章.
所以這段經文你見我再翻譯的時候.
除了約拿等同於記號.
再也沒有記號給他.
如果是英文的話.
簡直是Anette's Decide.
就是Jonah himself.
就是他自己.
我們再看一張PowerPoint.
這是最後一張.
在這段經文上.
上一段的段落.

$^{361}$14至26節.
記載了一個很出色的故事.
這個故事其實我想了很多年.
就是耶穌趕走阿爸鬼.
耶穌正在驅散阿爸鬼.
鬼魔一出去.
阿爸鬼就說話.
阿爸就說話.
不是阿爸鬼.
是阿爸的人說話.
做人都這麼驚奇.
但其中有些人就說.
他是教鬼魔的王別西卜驅趕鬼魔.
就是說耶穌你為什麼這麼有能力.
趕走那隻阿爸鬼.
因為你只不過是靠鬼王別西卜趕鬼.
接著那句16節說.
另外有些人試探耶穌.
向他要求從天上來的神蹟.
其實神蹟的字是Semeon.
就是記號.
耶穌為什麼要說29至32節.
除了約拿的神蹟.
這個記號.
是因為你看14至16節.
耶穌明明做了神蹟的事.
你試一下今天.
假設今天當中.
Worship tune裡有人有鬼附.
突然之間滾到地上.
亂說話.
John走出來.
加上潘Sir走出來.
阿爸鬼走.
你可以想像.
鏡頭不停照播著.
明天的view數一定超過3萬.
你明不明白.
一定超過3萬.
人家覺得這些叫好看.

$^{401}$那個人突然起來說.
我沒有那隻鬼了.
我們looking forward的是那些東西.
但你想想.
就算3萬view也好.
當中一定有些人會說.
John原來是靠鬼王別西卜趕鬼的.
剛才那些趕鬼動作是假的.
演一場戲出來的.
就算成功的話.
他也是鬼附身才會有這麼多能力.
你明不明白.
就算有神蹟有奇事也好.
這個世代也不一定會相信.
就算今天充滿了.
死人復活耶穌基督.
在社福金裡面.
所有的神蹟.
Fold Church也有能力做完.
你可以想像.
你覺得會惹來更多人羨慕Fold Church.
還是會惹來更多人攻擊Fold Church.
對於一個人子.
滿有能力.
做一件對的事.
一個啞巴鬼被鬼搞.
他說不了話.
耶穌因此的手在他身上.
趕到鬼.
那隻鬼能夠離開.
這個世代的人.
仍然求更多不一樣的神蹟.
丁子妹.
耶穌沒有proud of他.
趕鬼離海.
五餅而魚.
死死人復活.
沒有.
他做那些事.
不會因為他這樣的緣故.

$^{441}$他的肚子不會餓.
他不需要坐船.
他不會釘十字架的時候.
他不痛.
沒有這些故事發生.
沒有.
耶穌基督在說的是.
不是他在地上做了很多事.
有人跟從他.
是他口中所說天國的福音.
讓下一代人聽得明白.
好像南方的女王覺得.
所羅門是耶和華智慧的化身.
她聽智慧的言語.
好像莉莉美人一樣.
若拿由上主而來的智慧.
說一句話.
下面的人就改變.
其實我們自己本身最大的問題.
最難約定的是什麼.
是我們自己無法監聽自己的生命.
往後走下去的時候.
我們走成怎樣.
在一個艱難的世代.
在一個不容易世代的裡面.
我們看著很多人.
唱唱《最後的信仰》.
我們說我們不再聽這個版本.
我們要聽林家謙的版本.
這個版本我不會再聽了.
我不會再給他熱門評價.
林家謙那個我會再聽這首《最後的信仰》.
我們在這個世代的裡面.
最令我們更加沮喪和難過的是.
唱《最後的信仰》的人.
突然之間不再唱下去.
他說你只不過是自作多情.
你讀錯了我.
要說約定.
是因為世局艱難.

$^{481}$如果世局不艱難的時候.
說約定的意義不大.
正如什麼時候說約定.
彩虹下的約定.
就是一個醫生要離開的時候.
很艱難的時候我們才會唱.
到那邊彩虹的時候我們會再相遇.
約定是最艱難的時候唱.
是因為在艱難的裡面.
我們無法保證自己走下去的時候.
走得對.
什麼叫做約拿等同於神蹟本身.
約拿被紀念.
不是單純因為他經歷過三夜一條魚的經歷.
不是因為他在第一章的裡面.
掉下河之後突然風浪平靜沒事.
約拿被紀念.
不是因為他在生命裡面.
經歷了很多我們覺得很誇張的事情.
很奇怪的事情.
約拿這個記號.
是說約拿這個人本身他自己.
讓人一記得他就記得一件什麼事情.
就是他把上帝的話語.
在《喜利美城》說了.
正如所羅門王.
能夠繼續在整個《馬大福音》的啟述裡面.
所羅門這個名字能夠出現.
不是因為他的成就比大衛厲害.
不是他建了聖殿很威風.
只不過因為什麼.
他求智慧.
以至到南方地極的人.
都走來看望那個給智慧的上主.
萬鈞的耶和華.
丁子妹.
我不知道.
你心裡面對著香港的變局的時候.
你會想到什麼.
最近都得知.

$^{521}$神學院今年收生又很差.
有時候我會想不是的.
我不敢確定.
但聽到一點消息.
在這個時代裡.
讀三年神學還有什麼出路.
我們會想的是.
看好一點.
我們會在艱難的裡邊.
會想一些較為容易的東西走下去.
如果真的要約定要煲底見的話.
從來都不是因為.
我們放下手和腳什麼都不做.
等突然間一個大的神蹟下來.
突然間所有的人不見了.
好像Avengers那樣插手指.
看哪一半人沒有了 死了.
終於天下太平.
約定最難的是.
那個人是否繼續堅持.
上帝在他心裡做的事.
我講完這個例子.
就結束了 希望幾分鐘.
上星期四又來了一次驚恐症的來臨.
上星期四來的那次更加恐怖和難過.
如果去年你有聽的話.
上年五月開始他來一次.
第一次在人生裡.
這次更加難處理 難搞.
我記得上個月Full Church 港獨的時候.
離開MFC的時候.
有對弟兄姐妹.
我們一起坐電梯下去聊天.
姐妹問了我一句話.
家Sir你的驚恐症.
Panic attack 還會不會來.
我說我希望能夠捱到七月頭.
我就覺得OK.
誰不知七月第一個星期真的來.
所以基本上這個星期我不敢出街.

$^{561}$出街的話我要拖著老婆的衣服尾.
扯著我女兒的袋子.
我怕我隨時會不行.
今早準備完Full Church 編導的時候.
整個人就快暈倒了.
狀態很差.
要睡很久.
我想說什麼.
這一年裡的驚恐症.
這個星期驚恐症很恐怖.
這個星期很嚴重.
除了去了Full Church 退休營.
我們有吃有玩.
不知為何就醫好了一點.
但Other than that 前前後後都很差.
其實想著退休營都去不了.
我覺得去西貢太遙遠.
我覺得這一年裡.
有一樣東西我學會了.
我學會放下一些東西.
我學會放下一些.
我看起來好像很重要的東西.
當你走不了 動不了.
當你的狀態很差的時候.
你已經不像以前那樣.
有些事情你會做到.
這一年學會的是.
慢慢放下一些.
我眼裡覺得好像很重要很重要的東西.
你放下它.
開始問自己的是.
其實什麼才是這一刻.
上帝給你生命裡.
成為記號的那樣東西.
不要搞錯.
或者不要混亂.
我們要達成某些標誌.
某些人認同的角色.
身份地位.
我們要有一個名號名分.

$^{601}$別人就覺得你是something.
沒錯 是的.
在那些法利賽人罵耶穌是靠鬼王別西部趕鬼.
那些人的心裡是在想這些東西.
大部分的弟兄姊妹.
大部分的人.
都是用這個角度去看.
什麼叫做到事情的人.
但是你看一下約拿.
你看一下所羅門.
真正在新約聖經裡.
提到約拿的沒有其他聖經.
正正是在這裡.
或者瑪麗福音同一段經文.
說的是.
不是那個人曾經輝煌過什麼.
是他慢慢找到他生命裡.
上帝給他做的那件事是什麼.
你做的那件事.
不是因為你有名分有title.
你就會做到.
是你什麼都沒有的時候.
被一條魚套出來的時候.
在外邦的人面前.
其他人可以敬拜上帝.
立刻在第一章.
他在修獄主的名.
他一下去立刻風平浪靜.
是在說沒有的時候.
就好像耶穌基督.
被人冤枉他靠鬼王別西部趕鬼.
明明有credit拿的.
我趕完那個阿巴鬼之後.
我應該有三萬view四萬view.
有人會看的.
是沒有的時候.
唯有沒有的時候.
或者你願意放下那些.
好像我看起來很緊張的那些東西的時候.
你的心才會清明多一點.

$^{641}$你要在上帝面前成為一個什麼的記號.
不是一個牧師的銜頭.
不是一個某個職位的銜頭.
不是職場裡面有沒有人看得起你.
不是ice cream.
大家很喜歡看那個和牙樂.
要拿完十八年的紀錄.
就威在別人面前.
真正的約定是.
上帝創造你.
上帝有一個很獨特的氣味.
他讓你身旁的人聞到你是誰.
真正成為記號不是因為你有身份有地位.
有角色有金錢有權力.
真正成為記號.
是因為你將上帝在你生命裡要演繹的東西.
可以在沒有的裡面你能夠演繹出來.
讓所有人看到.
這個人不是因為很多人抬他翹.
捧他上來.
所以他被記得這個人.
是當這個人被人說是鬼王別西北趕鬼.
若拿是被自己不覺的人嘲笑.
為什麼要去尼尼未成做這麼蠢的事.
在沒有的時候.
讓人知道上帝在他生命裡.
他應該流露著一個什麼氣味.
下年五十歲.
我覺得五十歲是什麼時候.
過去一年的Panic attack.
告訴我.
當你在自己生命裡.
覺得自己拿了很多東西的時候.
放下.
有一次跟那群傻瓜分享.
我曾經有一個月在挑戰自己.
如果沒有牧師這個銜頭.
沒有博士這個銜頭.
別人認識你是誰.
你還敢說自己是上帝的兒女嗎.

$^{681}$還是別人認識你是因為你的銜頭.
別人賦予你機會.
賦予你欣賞的眼光.
當這些東西沒有的時候.
我怎樣成為記號.
今天在一個艱難的時代.
真正約定的.
不是自己為自己建立很多.
豐功偉績的事情.
如果約拿到今天為所傳頌的.
只不過是說一句話.
丁子妹.
你要在這個混亂的世代裡.
為上帝說一句什麼話.
耶穌基督所誇耀的.
不是祂有很多五餅二魚.
吃剩的餅和魚.
祂所誇耀的是.
這間你要準備的餅和杯.
放下自己.
甚至將生命都可以放下.
這個本身才是真正的記號.
求天父親自憐憫我們.
幫助我們.
在一個很困難的時代.
在一個你做得對的事情.
別人都說你靠鬼王別西卜趕鬼的世代.
丁子妹.
我相信將來的日子裡.
更多人會說這些話.
那個階段不是真的為了.
我還要在別人面前取回名分和記號.
那些title.
我們開始出賣自己.
開始和很多東西compromise.
真正成為記號的是.
雖然這麼困難.
別人取走了你很多東西.
你可以跟自己說.
其實我也在丟下很多東西.

$^{721}$他拿不到我任何一些東西.
心願基督教界.
當有很多人開始出賣的時候.
開始靠邊站的時候.
我們和自己再一次有一個約定.
讓自己成為一個沒有東西的記號.
上帝會用這個沒有東西.
成為記號.
讓這個信仰群體.
能夠再堅持下去多一步.
我們一起祈禱.
天父望著前面的路.
沒有人能在你面前自誇任何東西.
看著歷史.
很多好像很剛強的人.
很多說話很漂亮的人.
開始為了要有很多東西.
或者不想損失一些東西.
而去做了很多妥協.
甚至把信仰某些位份.
都妥協起來.
天父我們不是覺得自己不會.
只是因為我們心裡都害怕自己會成為那些.
所以我們在你面前去求.
讓我們學習放下.
不是我做了多少件神蹟奇事.
是我這個人能夠繼續走在信仰裡邊.
這個才是真正的記號.
阿爸天父.
我祈禱的是.
Fold Church的領子妹.
學習放下.
學習失去.
慰藥的是.
在前面很艱難的道路裡邊.
仍然可以忠心走下去.
求天父你憐憫我們.
祝福在不同地區.
有很多艱難的弟兄姊妹一樣.
他什麼都沒有.

$^{761}$祝福你親自憐憫保守.
讓你成為我們的唯一.
他憐憫我們幫助我們.
藉著一間餅和杯.
再一次幫助我們宣告.
放下我們覺得好像很重要的侍奉.
可以揚名立萬的心態與狂妄.
再一次用你的餅和杯.
成為我們的一切.
多謝天父你聽我們這次的祈禱.
奉耶穌你寶貴名堂.
Amen.
\newpage



\section{但以理書 3:1-30-20220723}
\label{sec:RWAsMtjQmZ4}
\textbf{【網上崇拜】難道我們未夠難?|但以理書3\_1-30|20220723 [RWAsMtjQmZ4]}
\newline
\newline
連結: \href{https://youtube.com/watch?v=RWAsMtjQmZ4}{\texttt{ https://youtube.com/watch?v=RWAsMtjQmZ4}} ~~~~ 語音日期: 2022-07-23 
\newline
\newline
\hyperref[sec:CzS_E_B5XMA]{\small{< < < PREV SERMON < < <}}
~
\hyperref[sec:index_chronic]{\small{[返順時目]}}
~
\hyperref[sec:index_scriptual]{\small{[返順卷目]}}
~
\hyperref[sec:AzkFq5FHvK0]{\small{> > > NEXT SERMON > > >}}
\newline
\newline
但以理書 3:1-30-20220723
\newline
\begin{longtable}{cl}
\hline
\hline
章節 & 經文 (和合本修訂版)\\
\hline
3:1 & \begin{tabularx}{0.7\textwidth}{X} 尼布甲尼撒王造了一個金像,高六十肘,寬六肘,立在巴比倫省的杜拉平原。 \end{tabularx} \\ \\ \relax
3:2 & \begin{tabularx}{0.7\textwidth}{X} 尼布甲尼撒王差人將總督、欽差、省長、參謀、財務、法官、地方官和各省的官員都召了來,為尼布甲尼撒王所立的像行開光禮。 \end{tabularx} \\ \\ \relax
3:3 & \begin{tabularx}{0.7\textwidth}{X} 於是總督、欽差、省長、參謀、財務、法官、地方官和各省的官員都聚集,站在尼布甲尼撒所立的像前,要為尼布甲尼撒王所立的像行開光禮。 \end{tabularx} \\ \\ \relax
3:4 & \begin{tabularx}{0.7\textwidth}{X} 那時傳令的大聲呼叫說:「各方、各國、各族的人哪,有命令傳給你們: \end{tabularx} \\ \\ \relax
3:5 & \begin{tabularx}{0.7\textwidth}{X} 你們一聽見角、號、琴、瑟、三角琴、鼓和各樣樂器的聲音,就當俯伏,拜尼布甲尼撒王所立的金像。 \end{tabularx} \\ \\ \relax
3:6 & \begin{tabularx}{0.7\textwidth}{X} 凡不俯伏下拜的,必立刻扔在烈火的窯中。」 \end{tabularx} \\ \\ \relax
3:7 & \begin{tabularx}{0.7\textwidth}{X} 因此百姓一聽見角、號、琴、瑟、三角琴和各樣樂器的聲音,各方、各國、各族的人就都俯伏,拜尼布甲尼撒王所立的金像。 \end{tabularx} \\ \\ \relax
3:8 & \begin{tabularx}{0.7\textwidth}{X} 在那時,有幾個迦勒底人進前來控告猶大人。 \end{tabularx} \\ \\ \relax
3:9 & \begin{tabularx}{0.7\textwidth}{X} 他們對尼布甲尼撒王說:「願王萬歲! \end{tabularx} \\ \\ \relax
3:10 & \begin{tabularx}{0.7\textwidth}{X} 你,王啊,你曾降旨,凡聽見角、號、琴、瑟、三角琴、鼓和各樣樂器聲音的,都當俯伏拜這金像。 \end{tabularx} \\ \\ \relax
3:11 & \begin{tabularx}{0.7\textwidth}{X} 凡不俯伏下拜的,必扔在烈火的窯中。 \end{tabularx} \\ \\ \relax
3:12 & \begin{tabularx}{0.7\textwidth}{X} 現在有幾個猶大人,就是王所派管理巴比倫省事務的沙得拉、米煞、亞伯尼歌;王啊,這些人不理你的諭旨,不事奉你的神明,也不拜你所立的金像。」 \end{tabularx} \\ \\ \relax
3:13 & \begin{tabularx}{0.7\textwidth}{X} 當時,尼布甲尼撒大發烈怒,命令把沙得拉、米煞、亞伯尼歌帶過來;他們就把這幾個人帶到王面前。 \end{tabularx} \\ \\ \relax
3:14 & \begin{tabularx}{0.7\textwidth}{X} 尼布甲尼撒問他們說:「沙得拉、米煞、亞伯尼歌,你們不事奉我的神明,不拜我所立的金像,是真的嗎? \end{tabularx} \\ \\ \relax
3:15 & \begin{tabularx}{0.7\textwidth}{X} 現在,你們若準備好,一聽見角、號、琴、瑟、三角琴、鼓和各樣樂器的聲音,就俯伏拜我所造的像;若不下拜,必立刻扔在烈火的窯中,有哪一個神明能救你們脫離我的手呢?」 \end{tabularx} \\ \\ \relax
3:16 & \begin{tabularx}{0.7\textwidth}{X} 沙得拉、米煞、亞伯尼歌對王說:「尼布甲尼撒啊,這件事我們不必回答你, \end{tabularx} \\ \\ \relax
3:17 & \begin{tabularx}{0.7\textwidth}{X} 即便如此,我們所事奉的神能將我們從烈火的窯中救出來。王啊,他必救我們脫離你的手; \end{tabularx} \\ \\ \relax
3:18 & \begin{tabularx}{0.7\textwidth}{X} 即或不然,王啊,你當知道,我們絕不事奉你的神明,也不拜你所立的金像。」 \end{tabularx} \\ \\ \relax
3:19 & \begin{tabularx}{0.7\textwidth}{X} 當時,尼布甲尼撒怒氣填胸,向沙得拉、米煞、亞伯尼歌變了臉色,命令把窯燒熱,比平常熱七倍; \end{tabularx} \\ \\ \relax
3:20 & \begin{tabularx}{0.7\textwidth}{X} 又命令他軍中的幾個壯士,把沙得拉、米煞、亞伯尼歌捆起來,扔在烈火的窯中。 \end{tabularx} \\ \\ \relax
3:21 & \begin{tabularx}{0.7\textwidth}{X} 這三人穿著內袍、外衣、頭巾和其他的衣服,被捆起來扔在烈火的窯中。 \end{tabularx} \\ \\ \relax
3:22 & \begin{tabularx}{0.7\textwidth}{X} 因為王的命令緊急,窯又非常熱,那抬沙得拉、米煞、亞伯尼歌的人都被火焰燒死。 \end{tabularx} \\ \\ \relax
3:23 & \begin{tabularx}{0.7\textwidth}{X} 但是這三個人,沙得拉、米煞、亞伯尼歌被捆綁著,掉進烈火的窯中。 \end{tabularx} \\ \\ \relax
3:24 & \begin{tabularx}{0.7\textwidth}{X} 那時,尼布甲尼撒王驚奇,急忙站起來,對謀士說:「我們捆起來扔在火裡的不是三個人嗎?」他們回答王說:「王啊,是的。」 \end{tabularx} \\ \\ \relax
3:25 & \begin{tabularx}{0.7\textwidth}{X} 王說:「看哪,我看見有四個人,並沒有捆綁,在火中行走,也沒有受傷;那第四個的相貌好像神明的兒子。」 \end{tabularx} \\ \\ \relax
3:26 & \begin{tabularx}{0.7\textwidth}{X} 於是尼布甲尼撒靠近烈火窯門,說:「至高神的僕人沙得拉、米煞、亞伯尼歌,出來,來吧!」沙得拉、米煞、亞伯尼歌就從火中出來。 \end{tabularx} \\ \\ \relax
3:27 & \begin{tabularx}{0.7\textwidth}{X} 那些總督、欽差、省長和王的謀士一同聚集來看這三個人,見火不能傷他們的身體,頭髮沒有燒焦,衣裳也沒有變色,都沒有火燒過的氣味。 \end{tabularx} \\ \\ \relax
3:28 & \begin{tabularx}{0.7\textwidth}{X} 尼布甲尼撒說:「沙得拉、米煞、亞伯尼歌的神是應當稱頌的!他差遣使者救護倚靠他的僕人,他們不遵王的命令,甚至捨身,在他們神以外不肯事奉敬拜別神。 \end{tabularx} \\ \\ \relax
3:29 & \begin{tabularx}{0.7\textwidth}{X} 現在我降旨,無論何方、何國、何族,凡有人毀謗沙得拉、米煞、亞伯尼歌的神,他必被凌遲,他的房屋必成糞堆,因為沒有別神能像這樣施行拯救。」 \end{tabularx} \\ \\ \relax
3:30 & \begin{tabularx}{0.7\textwidth}{X} 那時王在巴比倫省使沙得拉、米煞、亞伯尼歌高升。 \end{tabularx} \\ \\
[1ex]
\hline
\hline
\end{longtable}
$^{1}$今日ge 訊息ge 經文係《但義理書》三章.
有三十節ge 經文.
一個好激勵人心ge 神蹟故事.
讓我地先聽一次呢三十節ge 經文.
《但義理書》三章.
一節.
利布格尼沙王造了一個金像.
高六十爪 闊六爪.
立在巴比倫省的杜拉平原.
利布格尼沙王.
差人將總督 鑫差 省長 參謀.
財務 法官 地方官和各省官員都召了來.
為利布格尼沙王所立的像行開光禮.
於是總督 鑫差 省長 參謀 財務 法官.
地方官和各省官員都聚集.
站在利布格尼沙王所立的像面前.
要為利布格尼沙王所立的像行開光禮.
那時全靈的大聲呼叫說.
各方各國各族的人啊.
有命令傳給你們.
你們一聽各號琴 三角琴.
鼓和各樣樂器的聲音就當俯伏.
拜利布格尼沙王所立的金像.
凡不俯伏下拜的.
必立刻 永在 烈火的搖盅.
因此百姓一聽各號琴 三角琴.
和各樣樂器的聲音.
各方各國各族的人就都俯伏.
拜利布格尼沙王所立的金像.
第八節.
在那時有幾個迦勒底人進到來.
控告猶大人.
他們對利布格尼沙王說.
願王萬歲.
你 王啊.
你曾降旨凡聽見各號琴 三角琴.
鼓和各樣樂器聲音的.
都當俯伏拜祭金像.
凡不俯伏下拜的.
必永在 烈火的搖盅.

$^{41}$現在有幾個猶大人.
就是王所派管理巴比倫省.
事務的沙德拉 米薩 阿伯尼哥 王啊.
這些人不理你的御旨.
不侍奉你的神明.
也不拜你所立的金像.
當時利布格尼沙大發烈怒.
命令把沙德拉 米薩 阿伯尼哥帶過來.
他們就把這幾個人帶到王面前.
利布格尼沙王問他們說.
沙德拉 米薩 阿伯尼哥.
你們不侍奉我的神明.
不拜我所立的金像 是真的嗎.
現在你們若準備好一聽見各號琴 三角琴.
鼓和各樣樂器的聲音.
就俯伏拜我所做的像.
若不下拜.
必立刻永在 烈火的搖盅.
有哪一個神明能救你們脫離我的手呢.
沙德拉 米薩 阿伯尼哥對王說.
利布格尼沙 這些事我們不必回答你.
即便如此.
我們所侍奉的神能將我們從烈火的搖盅救出來.
王啊 他必救我們脫離你的手.
即或不然 王啊 你當知道.
我們絕不侍奉你的神.
也不拜你所立的金像.
十九節.
當時利布格尼沙怒氣填空.
向沙德拉 米薩 阿伯尼哥變了臉色.
命令將瑤燒熱比正常熱七倍.
又命令他軍中的幾個壯士把沙德拉 米薩 阿伯尼哥捆起來.
永在烈火的搖盅.
這三個人穿著內袍 外衣 頭巾和其他的衣服.
被捆起來永在烈火的搖盅.
因為王的命令緊急 瑤又非常熱.
亞台 沙德拉 米薩 阿伯尼哥的人都被火焰燒死.
但是這三個人 沙德拉 米薩 阿伯尼哥被捆綁著.
掉進烈火的搖盅.
二十四節.

$^{81}$那時利布格尼沙王驚奇急忙站起來.
對某士說 我們捆起來永在火裡的不是三個人嗎.
他們回答王說 王啊 是的.
王說 看啊 我看見有四個人.
並沒有捆綁 在火中行走也沒有受傷.
那第四個相貌好像神明的兒子.
二十六節.
於是利布格尼沙靠近烈火搖門說.
至高神的僕人 沙德拉 米薩 阿伯尼哥出來來吧.
沙德拉 米薩 阿伯尼哥就從火中出來.
那些總督 音差 省長和王的謀士一同聚集來看這三個人.
見火不能傷他們的身體 頭髮沒有燒焦 衣裳也沒有變色.
都沒有火燒過的氣味.
利布格尼沙說 沙德拉 米薩 阿伯尼哥的神是應當稱頌的.
他察顯使者救護依靠他的僕人.
他們不遵王的命令 甚至捨身.
在他們神以外 不肯侍奉敬拜別神.
現在我講旨 無論何方何國何族.
凡有人誹謗沙德拉 米薩 阿伯尼哥的神.
他必被凌遲 他的房屋必成糞堆.
因為沒有別神能像這樣施行拯救.
那時 王在巴比倫省使沙德拉 米薩 阿伯尼哥告成.
各位頂姐妹平安.
多謝主令為我們讀出剛才三章一至三十節的經文.
今天很想藉著《但義理書》第三章.
我們看看但義理那三位的朋友.
哈拿尼亞 米沙尼 阿薩尼亞.
他們如何在接二連三的人禍中.
仍然持有他們對上帝的信靠和盼望.
在進入經文之前.
很想稍稍講一下猶太人對禍福的觀念.
猶太人認為禍和福並不是最終的結果.
福不一定是好.
禍患亦不一定是代表不好.
視乎人面對禍和福的處境.
他的態度是怎樣.
而不是那件事的本身.
所以禍和福不能一概而論.
就像我們中國人有一句說話.
叫做禍是福所倚 福是禍所伏.

$^{121}$說的是禍是福有時不太看表面.
可能禍患出現時.
福氣緊接著會臨到等等.
猶太人亦相信禍福這一切都是來自上帝.
不過總的來說禍和福只是人間的標準.
這不代表上帝對人所看事物的看法.
另外但義理書這本書最明顯處理.
兩個令當時猶太人覺得頭痛的課題.
第一就是亡國.
第二就是殉道.
亡國可以用先知的傳統去解答.
上帝賜福給遵守律法的人.
懲罰那些活逆他的人.
如果按這個教導去看.
以色列民他們犯罪敬拜偶像.
亡國自然是一個很合理的懲罰.
不過殉道卻不可以用先知的傳統去解釋.
不可以說他們犯罪以至受罰.
所以因而他們要面對殉道.
就好像剛才所讀的但義理書經文.
但義理和那班朋友其實沒有犯罪.
相反他們為何會受到刑罰.
是因為他們仍然忠於上主.
他們忠於上主是沒有得福.
相反是招致殺身之禍.
所以這裡告訴我們.
上主不一定即時賞善罰惡.
有時是需要忍耐等候死後的審判.
對於講求即時,講求效率的我們來說.
這可能是一個艱難的信仰操練.
但義理書三章之前是說到尼布甲尼撒.
他被自己的夢所困擾.
但他當很想有人為他解夢的時候.
他卻不願意自己先說出夢的內容.
反而是要求解夢者可以先告訴他.
夢的情況是怎樣.
否則他就要殺光巴比倫國內所有術士等等.
結果第二章告訴我們.
但義理和他三個朋友化解了這場危機.
但義理如何解夢.

$^{161}$夢境預表甚麼呢.
在這裡不詳細說明.
但義理曾經這樣跟尼布甲尼撒說.
他說你就是夢中的金像的金頭.
可能是這番說話激發起尼布甲尼撒.
決定在杜拉平原要建立一個金像.
一個高約27米,闊約2.7米.
一個高高瘦瘦的人形金像.
讓人可以遠遠地看到這個金像.
他建立了這個金像之後.
他下令國民必須按照惡星的提示去敬拜這個金像.
尼布甲尼撒為何要設立這個金像呢.
可能是想炫耀他的國勢有多強.
炫耀他的豐功偉績.
但同時亦希望透過全國強制性去敬拜金像.
希望可以從外在達到一種政教合一.
希望可以達至萬眾一心.
可以震懾全國的政策.
你說拜金像這個命令難不難遵守呢.
對巴比倫人來說其實不難.
他們一直都有多神的文化.
拜多一個金像難不難呢.
不難的.
拜多一個好像順便一樣.
又或者是要遵守很多的規例.
防疫的規例難不難呢.
可能對某些地方的人來說不難.
你叫人去掃區.
他們隨時就掃.
可能馬子轉了顏色.
不能乘車,拿錢.
其實他們都很接受得到.
最近在網上看過一段片.
其實是有一個人隨便在閘口坐下.
路過的人是怎樣.
路過的人就拿電話給他看.
他們其實好像至少被馴服.
你叫他做甚麼就做甚麼.
他們不會想為甚麼.
所以就好像這個情況.

$^{201}$你要巴比倫去拜多一個金像難不難呢.
不難的.
他們從小就被教導.
你叫我做甚麼.
你叫我拜神我就拜.
再說拜金像是代表甚麼呢.
在巴比倫國內.
其實王就代表著.
好像一個巴比倫的主神.
瑪道在人間統治.
所以你不拜金像是代表甚麼.
你拒絕侍奉巴比倫王.
再加上剛才我們聽到的經文告訴我們是.
你不願意拜金像換來的是死罪.
第六節告訴我們是怎樣的.
必立刻,泯災,烈火的堯宗.
所以聽到這條命令.
但凡有理性的人.
都不會用自己的性命來開環扇的.
所以我們就要去做些甚麼呢.
我們就要去做些甚麼呢.
但凡有理性的人.
都不會用自己的性命來開環扇的.
但是當噩聲驟起.
竟然有三個人不跪.
因為對於秘魯的以色列人來說.
拜金像其實是直接衝擊他們的信仰.
不可以敬拜別神.
不可以跪拜.
不可以侍奉偶像.
不可以侮辱神的教導.
而樂器的名稱亦都重覆了四次.
分別是.
國,號,琴,室,三角琴,鼓.
以及各種樂器.
我不知道剛才我們敬拜隊有沒有七種樂器那麼多.
但是這一堆樂器代表什麼呢?.
代表了一些鋪天蓋地,游說人要效忠王的聲音.
形成了一個好像眾星圈娃的場面.
這些樂器一奏起.

$^{241}$人要立刻放下自己手上所做的一切.
立刻要跪.
要去跪拜這個金像.
違抗命令的人就會葬身火曜.
當樂聲一奏起.
信服王明,不信服王明.
就即時分別出來.
對於信服上帝的以色列人來說.
每一次樂聲奏起.
不單單是一個信仰的挑戰.
更加是一個生死有關的抉擇.
這幾位敬虔的年青人.
經歷過第一章的被培訓,被改名,被飲食.
第二章他們面對著一個橫蠻無理,猜忌深重的尼布甲尼薩.
他肆意要屠殺全國智慧人的危機.
這次哈拿利亞,米沙利和亞薩利亞.
在第三章再次面對著一個信仰的挑戰.
他們要面對的是什麼?.
要不要在這個金像面前去屈膝呢?.
這次是針對他們的生死抉擇.
他們可能心想.
又來?難道我們面對的艱難還不夠嗎?.
可不可以讓我們休息一下?.
但以理書的罪事者一而再再而三地提醒我們.
是的,那些挑戰,那些困難是會不斷地出現的.
今天我們可能都面對著大小不一的信仰的抉擇.
可能是各項政策衝擊著我們信仰的底線.
我們願不願意因為捍衛我們的信仰.
會跟這些政策去周旋到底呢?.
還是我們面對著從上而來的命令.
我們不再掙扎了.
我們就決定信服了.
我們會不會軟弱到連想怎樣對抗都不想呢?.
我們會不會未戰先降呢?.
我們會不會太容易妥協和被同化呢?.
我想說哈拿利亞,米沙利亞和亞薩利亞這三個人.
他們那種可貴可敬之處.
是他們在這裡沒有用「人在江湖,身不由己」作為一個藉口.
沒有,他們沒有想怎樣走正面,怎樣妥協.
他們只是單單選擇怎樣忠於上主.

$^{281}$而且他們那份忠心不是廉價的.
是要負上性命的代價.
頂姐妹我們有沒有做好準備.
當我們面對著信仰抉擇的時候.
我們想都不想就站在上帝的那一邊呢?.
其實我想說三友的處境其實跟我們都很相似.
他們處身在一個甚麼都被政治化的處境.
三章十二節加拿大人特別強調他們是甚麼?.
特別強調他們是猶太人.
政治上他們被視為猶太人代表甚麼?.
代表他們是被露的人.
就好像我們現在的處境.
做甚麼都被人無限地上綱上線.
說甚麼,穿甚麼顏色的衣服.
總有人會將我們歸邊.
做甚麼都好像要跟你的立場扯上關係.
繼而這幾個加拿大人是怎樣?.
再提到他們的官位.
他們在做甚麼?.
我們仔細去看整卷但義理書的時候.
我們有理由相信但義理和他三個朋友.
其實他們在受訓的時候的表現已經很出色.
他們受訓完然後進入職場的時候.
可能比一般人升得更快.
所以他們面對著的是很多政敵對他們的質度.
這班政敵特別在這裡提醒尼布甲尼薩.
說他們其實在管理巴比倫省的事務.
即是說他們的官位很高.
但卻是不遵守皇命.
背後的潛台詞就是說這班人在忘恩負義.
領受了巴比倫這個皇位這麼大的恩惠.
你還這麼忘本?.
為甚麼你還掛著昔日耶和華信仰的好?.
為甚麼你仍然忠於昔日耶和華信仰?.
是在說這件事.
但罪事者是讓我們看到這幾位敬虔的猶太青年.
即是說他們身處巴比倫是一個很高的位置.
加官進爵卻沒有改變他們的信仰.
對他們來說.
謹守上帝的誡命比遵從皇命重要.

$^{321}$這班政敵自然不會放過現在這個剷除異己的機會.
他們在皇面前刻意煽惑皇.
他們說:皇啊!這班人不理你的御旨.
不侍奉你的神明.
也不拜你所立的金像.
這三筆指控顯然是有效的.
利布格尼薩聽到有人敢於跟他拒絕的時候.
公然挑戰皇權.
拒絕效忠.
於是他就大發裂怒.
但我們有沒有想過.
這班政敵對他們的指控的確引起人間權貴的憤怒.
但反過來看.
這卻是他們堅守信仰.
忠於上帝一個最高的讚譽.
另一邊廂.
當利布格尼薩聽到有人這樣指控他們的時候.
他很生氣.
但他沒有像他們那樣.
他頒布法令.
他一開始說要立時, 永在火曜.
但他沒有這樣做.
他立刻召見三個人.
接著對話反映了什麼.
反映了他想先確認這個控告是否屬實.
他問這三個人.
你們不拜我的金像是否真的.
有人問會否反映黃都很看重這三個朋友.
側面反映他們都是忠於巴比倫的臣子.
於是黃不會輕信謠言指控.
想再給他一次機會.
但我們也反過頭來想.
其實會否反映黃的變態.
有時他立條法例要你跪.
他不是想你不遵從.
不是想你不守, 弄死你.
他只是想看人怎樣跪他.
想看人怎樣舔他.
現在說效忠.
不只是說.

$^{361}$效忠更要表現激動.
要感動流涕, 哭喊才合格.
利布格尼撒願意給三友多一次機會.
他跟他說當樂器奏起時.
你們三個人如果俯伏拜我.
就可以保命.
利布格尼撒再次挑戰他們.
他說你拜就沒事.
黃再次將生死的抉擇.
明明白白地放在三人面前.
告訴他們一個簡單的事實.
順黃者生, 逆黃者亡.
黃再次說有哪個神明能夠脫離他們的手呢.
黃的提問, 其實我們都知道這不是問題.
其實他在表達甚麼呢.
首先是表達他的自信.
他自信覺得沒有一個神明可以拯救他們脫離他們的手.
又或者他的自信覺得.
他自己的威權可以令所有有信念的人都會跪下.
面對著黃的揭問和挑戰.
這三個人是怎樣回應呢.
十六節這樣說, 他說利布格尼撒.
這件事我們不必回答你.
直呼今上的名.
不打算為自己進行抗辯.
清楚地對黃說, 其實我現在根本不需要跟你解釋.
接著他們的回答表明了.
他們其實已經預備面對最壞的結果.
他們寧死不屈.
他們昔日樂器奏響的時候不拜金像.
現在和將來都不會屈服.
我想說這個聽起來好像很大不敬的回答.
有沒有想過其實當他說的時候背後是需要多大的勇氣和堅持.
他們對上主的忠心在昔日不跪的時候已經彰顯了出來.
但是當利布格尼撒再給他一次機會的時候.
他們這一份的忠心更加顯得更加真誠, 更加準確無誤.
寧願捨棄最後保命的機會.
仍然決定忠於上主而死.
但是都不願意背信偷生.
17節這一段經文經常成為我們去說的.

$^{401}$其實是很擇地有聲地表達了他們的心智.
他說即便如此我們所侍奉的神能夠將我們從烈火的搖鐘救出來.
他說王啊他必救我們脫離你的手.
即或不言.
王啊你當知道我們絕不侍奉你的神也不拜你所立的金像.
我想說其實他們這個述字既是表達了他們的忠心.
但也反映了他們的神觀是很正確的.
一是看到他們對上主堅定不移的相信.
他們信什麼呢.
他們說如果主願意的時候.
他們所信的神是有能力的.
是那個大能的神絕對可以將他們從火搖救出來.
表明了他們對上主的信心是沒有動搖過的.
但與此同時他們也認定神有自己的主權.
神不一定按人的時間方式去拯救.
他們面對著神的主權是怎樣的.
他們是全然信服的.
他們說即或不言.
即使神不救我們.
我們也不會侍奉你的神明也不會拜你的金像.
即是他們在不確定上帝下一步會怎樣做的時候.
就算要葬身火搖.
他們也絲毫不會動搖不會質疑.
仍然是忠於上主.
上帝怎樣做不是他們最高的考慮.
這也不是他們的考慮.
他們唯一掌握到的考慮就是.
他們忠於上帝.
能否逃出死亡的確不是他們的考慮之列.
他們寧可殉道卻決意不拜那個神像.
我們有時會說他們是否為自己打圓場呢.
即或不言.
即使他們不教我們祈禱.
神願你醫治就算你不醫治怎樣怎樣.
我們常常覺得這些好像是打圓場的信仰台詞.
但我想說.
其實這不是打圓場的屬靈言語.
剛剛相反.
他們其實是在敵人面前宣告他們的信仰.
無論或生或死都不會跪.

$^{441}$因為他們在表明他們所信靠所效忠的只有一位.
弟兄姊妹.
即便如此.
即或不言.
是否我們的信念呢.
當事情不是如我們預期.
又或我們對上主的旨意不是很拿捏得到的時候.
我們還是不是高舉神的主權呢.
我們是否定義信服地信靠他呢.
我想說三友這番宣言背後反映的是.
他們對上主的大能和慈愛的確信.
他們仍然堅守他們作為以色列人和上主所立的約.
哈拿尼亞,米沙利和亞薩尼亞.
他們那份宣告不是只有「講」字.
他們其實是用自己的性命去宣告他們的信仰.
無論什麼境況都忠於上主至死不渝.
第二章的夢境.
是人間的政權會每況愈下地走.
看來不能遙幸的巴比倫王朝.
是會有倒台的一天.
如果我們有這樣的相信和認定的時候.
上帝的子民在巴比倫王朝應該如何自處呢.
19至23節是三友的見證.
他提醒上帝的子民.
即是說你在受壓迫的處境下.
你其實可以怎樣做.
他們最終是怎樣呢.
剛才我們聽的時候.
他們最後決定不跪.
於是王將他們送去火窯.
整個敘事告訴我們.
忠於信仰是不會豁免火窯的苦難.
但在苦難中有上帝與他們同在.
24至25節讓我們看到的是.
當他們看到他們在苦難中有神同在的時候.
那個暴君對耶和華有更深的認識.
他親眼看到原來真的有神.
可以將他們脫離火窯和他的手.
以至他最終沒有辦法不承認耶和華才是最高的神.
弟兄姊妹.

$^{481}$我想說我覺得在往後的日子.
信仰的挑戰只會是越來越多.
有時我在想.
究竟做傳道童宮要不要登記註冊呢.
政權會不會要我去說一些合乎他意識形態的道呢.
要不要我為他去做一場樣板戲呢.
要不要我去為他去粉飾太平.
要不要我在一些很重要的會議去說.
我們仍然很有宗教自由.
仍然可以很隨意地去傳道呢.
要不要我去為國安服務呢.
有時候我都問自己.
我跪還是不跪呢.
我是死硬地堅持留守在這裡.
還是怎樣呢.
這個月的月題是約定.
所以最後我都想說關於上帝對人的命定.
有時候我想說當我接觸不同人的時候.
經常都聽到究竟這個是不是上帝為我預備的呢.
那些情況可能是.
究竟我要不要接受某個人做我的男朋友或女朋友呢.
我要不要獻身讀神學呢.
我要不要這個時候結婚呢等等.
總是為了這些事情要好好祈禱.
要聽清楚這是不是上帝對我的命定.
然後我才做決定.
總是面對這些情況.
我覺得如果能夠等.
我們就等久一點才落實去做.
是的,剛才所說的不是不重要.
有些事情,剛才所說的不是我們隨隨便便輕率的決定.
不過我都想說.
其實我們決志信主.
願意立志去追隨基督.
其實都是一個很重要的命定.
弟兄姊妹,如果我們一旦認定了.
我們追隨耶穌基督.
是上帝對我們的呼召的時候.
我想說,我希望我們都有長期信守這個承諾的決心和預備.
不要輕易給自己有推倒重來的機會.

$^{521}$我想說,去回應上主給我們的呼召.
這條路一定有很多的艱難,有很多的挑戰.
我們會不會可以說到我們願意堅持下去,不退縮呢?.
選擇了,我們就不要單單視為只是我個人的抉擇.
我們更加視之為是上帝對我們的命定.
讓我們守在祂為我們所預備的崗位.
有時候面對著困難.
我們都學習不要過份重視自己的感受,喜惡.
可不可以為上帝堅持多一會呢?.
堅持就有機會去成就無論是生命或者是侍奉的傳奇.
我想說,我們的特首說自己不是一個口號式的長官.
我想說,我都不是一個口號式的牧者.
不過有時候,口號卻是可以成為我們的提醒.
有時候想起威廉·波頓,一個宣教士.
他曾經說過這三句話.
不保留,不退縮,也不後悔.
他自己在他的宣教路上就是這樣去答出.
有時候,當我預備這篇道的時候.
我都想起我自己夢召時的初心.
我經常都在想,什麼是初心?.
我怎樣去辭去對上帝的仰望?.
給大家看一幅圖.
有時候,為什麼人會忘記了他的初心呢?.
有時候我們會想,是不是因為在過程中有太多事情發生了.
於是分了心呢?.
不過,當我在侍奉的路上.
看到很多人向不同的立場去靠攏.
有時候,忘記了初心.
未必是因為中間發生了太多事情.
其實很簡單,有時候是因為被利欲分心.
只是為了滿足自己的慾望.
於是忘記了為什麼當初要這樣做.
走在與初心相反的路上.
有時候,有些所謂識時務的人.
還可以很振振有詞地說.
他只是靈巧像蛇般,帶著智慧走順羊的路.
走他傳道的路.
最近,我們Flow Church一班同工退休.
當中有很多的分享.
同工之間,彼此之間好像多了一些認識.

$^{561}$有同工這樣說我.
他說,覺得我是一個很簡單的人.
我很同意,我真的是一個很簡單,很直接的人.
我不懂得怎樣轉彎抹角.
我真心做不到怎樣可以表面效忠人.
但內心卻是效忠神.
最後,補充多一點有關《但義理書》的一些成書背景.
作為整個講道的總結.
《但義理書》一至六章的序事.
設定在公元前六世紀.
即是當巴比倫統治的時期.
有一班猶太人被擄到巴比倫.
但很多學者認為.
巴比倫的成書時期.
其實是公元前二世紀.
西流古王朝.
安提阿古統治的時候.
簡單一點就是借古預金.
說說安提阿古是一個怎樣的人.
安提阿古四世設立了法令.
禁止猶太人一切的宗教活動.
擁有陀拉,即是我們現在擁有聖經的.
搜到你家裡有經書.
是會被處死的.
如果他找到書籍的抄本.
亦都會銷毀.
不讓猶太人去守安息日.
不讓他們行國禮是犯法的.
如果有媽媽要和小朋友行國禮.
是會被處死的.
猶太教種種的祭祀都被禁止.
甚至安提阿古要求.
他們要將希臘主神宙斯的神像.
放在聖殿的祭壇上.
以色列人需要為希臘的神明去搭建祭壇.
並且逼他們要獻畢業的動物.
有時甚至是強迫猶太人要吃豬肉.
另一方面就是.
有些願意為他們執行祭祀的祭師.
他們可以名成利就.

$^{601}$安提阿古四世就是這樣.
在信仰上去逼迫猶太人.
試圖去威逼,離遊.
讓他們放棄自己的信仰.
去歸順政權.
弟兄姊妹我不知道你剛才聽到那些歷史的時候.
你有什麼想法.
不過我自己在想.
其實公元前六世紀的巴比倫.
又或者是公元前二世紀的西遼古王朝.
其實和今天某個政權的做法都很相似.
哈拿尼亞,米沙尼和阿薩尼亞.
在迫害時期的猶太人成為了一個樣板.
告訴他們什麼.
告訴那些當時被壓迫的猶太人.
叫他們不要放棄抗爭.
要奮力對抗安提阿古的邪惡政權.
不過現實是怎樣.
但我們看《伊利書》的時候.
看到是一個快樂的結局.
不過如果我們看回歷史.
或者看回兩約之間的典外文獻.
《瑪加比一》《伊利書》的時候.
安提阿古在位的時候.
那些忠於信仰的猶太人.
最終的下場是怎樣.
他們很多被殉道.
並且在《瑪加比一》《伊利書》.
我們看到他們甚至用一些很不人道的方式去殉道.
無論是聖經或者歷史都告訴我們.
有時候不一定可以即時逃離苦難.
不過《但以利書》的中心告訴我們.
我們可不可以在不同的境況中.
仍然可以持守那份信仰.
支撐我們背後的力量是甚麼呢.
我想就是上帝與人所納的約.
我們有沒有表現出那份忠心.
信心和勇氣.
以致令到人為了我們的勇敢而被震懾呢.
我們可不可以讓惡人看到.

$^{641}$其實我們是有上帝與我們同在呢.
我們有沒有為自己可以站立得穩.
為自己可以堅持而做好準備呢.
至死忠心會不會是我們的立志呢.
讓我們一同低頭祈禱.
親愛的主.
香港轉換了新的管治團隊已經差不多一個月了.
在這段時候我們總是聽到很多流言蜚語.
有很多未知的情況.
不知道掌權的人五時三刻又會加甚麼政策.
留在這裡的我們可能會覺得很憂慮.
很多不安.
或許我們留在這裡看到很多事情都變得面目全非.
有時候都不敢想像我們還剩下甚麼.
有時都會問自己究竟信仰的底線我們最終守不守得住.
上主我們有很多不肯定.
不過你自己的話語是告訴我們.
你始終是掌權的那位.
你是掌管朝代興衰.
你是掌權者更一的那位.
你將我們命定在此時此地的香港.
我相信有你對我們的命定.
求你加添力量給我們每一個.
加添勇氣給我們.
讓我們可以做我們應該做.
讓我們可以做我們能夠做的事情.
求你使用我們每一個.
讓我們在這個時代為你去做.
你要我們做我們應該做.
我們有能力做的事情.
求你使用我們每一個.
祈禱在奉耶穌基督的聖名時祈求.
阿們.
\newpage



\section{提摩太後書 4:6-8-20220730}
\label{sec:muF9XhDEVwY}
\textbf{【網上崇拜】兩鬢斑白都可認得你|提摩太後書4\_6-8|20220730 [muF9XhDEVwY]}
\newline
\newline
連結: \href{https://youtube.com/watch?v=muF9XhDEVwY}{\texttt{ https://youtube.com/watch?v=muF9XhDEVwY}} ~~~~ 語音日期: 2022-07-30 
\newline
\newline
\hyperref[sec:AzkFq5FHvK0]{\small{< < < PREV SERMON < < <}}
~
\hyperref[sec:index_chronic]{\small{[返順時目]}}
~
\hyperref[sec:index_scriptual]{\small{[返順卷目]}}
~
\hyperref[sec:0mi_NsvpcRc]{\small{> > > NEXT SERMON > > >}}
\newline
\newline
提摩太後書 4:6-8-20220730
\newline
\begin{longtable}{cl}
\hline
\hline
章節 & 經文 (和合本修訂版)\\
\hline
4:6 & \begin{tabularx}{0.7\textwidth}{X} 至於我,我已經被澆獻,離世的時候到了。 \end{tabularx} \\ \\ \relax
4:7 & \begin{tabularx}{0.7\textwidth}{X} 那美好的仗我已經打過了,當跑的路我已經跑盡了,該信的道我已經守住了。 \end{tabularx} \\ \\ \relax
4:8 & \begin{tabularx}{0.7\textwidth}{X} 從此以後,有公義的冠冕為我存留,就是按著公義審判的主到了那日要賜給我的;不但賜給我,也賜給凡愛慕他顯現的人。 \end{tabularx} \\ \\ \relax
4:9 & \begin{tabularx}{0.7\textwidth}{X} 你要趕緊到我這裡來。 \end{tabularx} \\ \\ \relax
4:10 & \begin{tabularx}{0.7\textwidth}{X} 因為底馬貪愛現今的世界,已經離棄我,往帖撒羅尼迦去了;革勒士往加拉太去;提多往撻馬太去; \end{tabularx} \\ \\ \relax
4:11 & \begin{tabularx}{0.7\textwidth}{X} 只有路加在我這裡。你來的時候把馬可帶來,因為他在服事上於我有益。 \end{tabularx} \\ \\ \relax
4:12 & \begin{tabularx}{0.7\textwidth}{X} 我已經打發推基古往以弗所去。 \end{tabularx} \\ \\ \relax
4:13 & \begin{tabularx}{0.7\textwidth}{X} 我在特羅亞留給加布的那件外衣,你來的時候要帶來,那些書也帶來,特別是那幾卷羊皮的書。 \end{tabularx} \\ \\ \relax
4:14 & \begin{tabularx}{0.7\textwidth}{X} 銅匠亞歷山大多方害我;主必照他所行的報應他。 \end{tabularx} \\ \\ \relax
4:15 & \begin{tabularx}{0.7\textwidth}{X} 你也要防備他,因為他極力抗拒我們的話。 \end{tabularx} \\ \\ \relax
4:16 & \begin{tabularx}{0.7\textwidth}{X} 我初次上訴時,沒有人前來幫助,竟都離棄了我,但願這罪不歸在他們身上。 \end{tabularx} \\ \\ \relax
4:17 & \begin{tabularx}{0.7\textwidth}{X} 惟有主站在我身邊,加給我力量,使我能把福音完整地傳開,讓所有的外邦人都聽見;我也從獅子口裡被救出來。 \end{tabularx} \\ \\ \relax
4:18 & \begin{tabularx}{0.7\textwidth}{X} 主必救我脫離一切的兇惡,也必救我進他的天國。願榮耀歸給他,直到永永遠遠。阿們! \end{tabularx} \\ \\ \relax
4:19 & \begin{tabularx}{0.7\textwidth}{X} 請向百基拉、亞居拉和阿尼色弗一家的人問安。 \end{tabularx} \\ \\ \relax
4:20 & \begin{tabularx}{0.7\textwidth}{X} 以拉都在哥林多住下了。特羅非摩病了,我把他留在米利都。 \end{tabularx} \\ \\ \relax
4:21 & \begin{tabularx}{0.7\textwidth}{X} 你要趕緊在冬天以前到我這裡來。友布羅、布田、利奴、革老底亞和眾弟兄都向你問安。 \end{tabularx} \\ \\ \relax
4:22 & \begin{tabularx}{0.7\textwidth}{X} 願主與你的靈同在!願恩惠與你們同在! \end{tabularx} \\ \\
[1ex]
\hline
\hline
\end{longtable}
$^{1}$各位姐妹平安.
今日的講道會延續上一篇在7月2日的講道.
我們仍然會看提摩太后書第四章的經文.
一段保羅生命最後的留言.
所以你會見到今日的講題仍然是王妃約定這首歌的歌詞.
如果你的年紀足夠成熟讓你知道的話.
是兩斌斑伯都可認得你.
延續上一次我們對於提摩太后書第四章一至二的經文.
今日我們會繼續看下去.
上一次我們講到保羅面對著自己的死亡.
從生命的終點回望自己的生命.
亦從審判活人死人的耶穌基督.
他的顯現和角度去思考自己的當下.
去勸勉提摩太.
今日我們會延續這個課題.
不單單是提摩太的生命.
更加是保羅對於自己生命的自白.
如果你心水清的話.
你會發現這兩篇道其實都是回到一個很重要的主題.
就是思考我們的死亡.
我們從我們一生的終點來思考我們的生命.
並且認識我們和上主之間的約定.
如果我們上一次的月題擁抱.
浪子比喻所講的叫我們擁抱生命的話.
今個月的月題約定是有關面對我們的死亡.
所以我們會看一段保羅在提摩太後書最後的自白.
一段他自己生命的總結.
經文是在提摩太後書第四章落到第八節.
經文這樣說.
我現在被囂顛 我離世的時候到了.
那美好的帳我已經打過了.
當跑的路我已經跑盡了.
初心的道我已經守住了.
從此以後有公義的冠冕為我存留.
就是按著公義審判的主.
到了那日要賜給我的.
不但賜給我 也賜給凡愛慕他顯然的人.
我們一起去將我們的禱告 去恭敬的人來禱告上帝.
天父 最速幾刀 親愛的聖靈.
求全你與我們同在.

$^{41}$叫到全世界每一個閣樓裡面.
全聖體的弟兄姊妹.
我們此刻一齊的來敬拜你.
將我們的心 將我們的全人.
專注在你的聖言裡面.
你自己親自對我們說話.
你自己的說話親自的來對應我們此刻的生命.
我們當下每個人獨特的處境.
求主你這樣與我們同在.
求主你去眷顧著整個聖言的宣講.
讓你的靈在當中引領我們的生命.
提醒我們的生命.
願你自己的說話長存.
奉主明求 阿們.
提摩太后書第四章第六至八節.
一段我自己很喜歡的經文.
保祿說 我的現在被囂墊.
我離世的時候到了.
那美好的象我打過了.
當跑的路我跑盡了.
所想的道我守住了.
從此以後有公義的冠冕為我傳流.
我記得我讀大學夢召的時候.
有一首詩歌叫做獻上生命到最後.
詩歌是用這段經文寫的.
這首詩歌跟獻生命制是同一系列的詩歌.
是我自己在大學裡最火熱的時候.
願意為主殉道.
很火熱被蒙召.
一個很熱血的時代.
所以真的很喜歡這段經文.
因為這段經文裡正正是這種熱血.
我很喜歡這段經文.
甚至喜歡到一個地步.
我甚至不太敢用這段經文來講道.
為什麼這麼說呢.
因為我覺得這段經文實在太過神聖了.
神聖到一個地步.
我覺得自己有些是高攀不起.
不敢用網誌來講這段經文.

$^{81}$不敢去教這段經文.
不是嗎?.
那美好的象我打過了.
當跑的路我跑盡了.
所想的道我守住了.
試問世界上哪一個人能夠說出這番話.
哪一個人能夠聲稱自己的生命.
能夠完成.
能夠滿額地完成.
甚至就算是目者也是一樣.
試問哪一個人能夠傳達這番話.
或者能夠實踐這段經文.
然後走上港台裡面去教人呢?.
說自己要打的仗都打過.
當跑的路都跑盡了.
所想的道都守住了.
然後就在港台裡面教人做呢?.
沒有的.
我覺得是不行的.
對不起,就算頂智慕也一樣.
就算頂智慕也一樣.
這段經文也沒有什麼大關係.
你也沒有什麼需要用這段經文來做應用.
因為還沒有到時候.
你還沒有臨死之前.
似乎還沒有到那個位置.
來聽這段經文.
不過話雖如此.
這番話仍然是上帝的話.
我們也需要去聆聽.
或者我們用一種方式去思考.
用另一種角度去理解這段經文.
正如上次所說.
今天我們嘗試用一個終點的角度.
來思考這段經文.
保羅說:那美好的仗我已經打過.
當跑的路我已經跑盡了.
所信的道我都守住了.
這句是保羅自己一生中最後的寫照.
不過特別要提的是.

$^{121}$我們發現保羅在這段話後面繼續這樣說.
他說:從此以後有公義的冠冕為我存留.
保羅在經文的最後.
在他臨死之前.
輕輕的去沾染自己死後的光景.
去幻想一下自己死了之後.
在上帝面前所得到的東西.
保羅這樣描寫.
從此以後有公義的冠冕為我存留.
我們問:究竟什麼叫做公義的冠冕.
公義的冠冕這五個字聽起來好像很容易明白.
說出來好像很順口.
但是老實說.
我們不知道大家有沒有想過.
什麼叫做公義的冠冕.
大家不妨在YouTube留言.
說說你之前對這五個字的印象.
什麼叫做公義的冠冕呢.
很有趣的.
如果你去搜尋聖經的話.
你會發現.
其實聖經裡面有不少是記載了冠冕這個字.
生命的冠冕.
年歲的冠冕.
榮耀的冠冕.
不能壞的冠冕.
永不需殘的冠冕.
和這個公義的冠冕.
你問:將來在天堂是不是也要帶著這些冠冕出街呢.
將來天堂會不會像打遊戲一樣.
有很多款式的冠冕給你裝備去拆呢.
我不知道.
天堂有沒有冠冕.
是不是有冠冕的問題.
我不知道.
不過我知道的是.
其實在聖經裡面.
冠冕這個字.
其實是要去表達一種神聖.
貴重賞賜的意思.

$^{161}$所以要明白這些冠冕.
我們重點不需要放在冠冕上.
而是放在冠冕之前的形容詞上.
例如.
生命的冠冕其實就是強調生命.
榮耀的冠冕就是強調榮耀.
不能壞的冠冕就是強調不能弄壞的狀況.
因此雖然我們未必很明白什麼是公義的冠冕.
但是你最少可以知道.
公義的冠冕其實就是要表達.
保羅離世之後.
他將要得到的一種公義的狀態.
保羅臨死之前.
他說從此之後.
他將要在經歷一個天堂裡面.
一個完完全全公義的狀態.
所以我們要問的不是.
問什麼叫做公義的冠冕.
而是問.
為什麼保羅要在這段經文裡面.
特別要提到他將要離開世界之後.
特別要很在意這種公義的狀態.
為什麼在這段經文的臨終留言裡面.
特別要強調公義這個課題.
這正是我們今天要思考的問題.
坦白說.
這幾年我們在講壇裡面.
我們講得最多的其中一個話題.
就是公義這個字.
或者這個字的相反.
不公義.
有人叫我在講壇裡面.
就少講這些東西.
我就少講這些東西.
不過.
或者這幾年我們講得太多.
反而間接令我們未必完全明白.
聖經裡面所講公義這個字.
最基本最基本最基本的意思.
其實公義是一個很有意思的字.

$^{201}$你會看到.
很多時候聖經會翻譯成很多不同的字眼.
有時候叫做公義.
有時候叫做正義.
或者是義公平.
所以公義可以解作.
在社會層面一個公平正確的做法.
也可以解作一個個人道德行為裡面.
正直正確正義的行為.
也可以解作一個決定審判裡面的公平公正合理的結果.
所以所謂公義最基本最基本最基本的意思.
這個意象.
就是一種井井有條的秩序.
真人.
所謂公義是一個正確的擺放.
Everything in its right place.
如果一個人的行為在上帝眼中是正直的擺放.
這個就是義.
一個正確的好行為.
如果是人和人之間的關係恩怨的時候.
對待的時候.
這是一個正確的擺放的話.
這個就是一個社會的公義.
如果一個審判能夠正確的擺放.
這個就叫做公平.
所以公義就是一種絲毫不偏離.
完全完美無瑕正確準繩的擺放.
Perfect.
Perfecto.
大家有沒有看過網上YouTube有關強迫症治癒系的影片.
那些腳目都剛剛好.
塌塌陷.
工廠Bad出來的倒模是完美無瑕的倒模出來.
DVD機的Screen Saver Logo剛好在螢幕左上角的位置就彈了出來.
畫出一條完美無瑕的直線.
這些影片就叫做Satisfying Video.
看完之後是剛剛好.
塌塌陷.
你心裡是非常治癒和愉快的.
這種絲毫不偏離.

$^{241}$完美無瑕正確準繩的擺放.
這種Perfect.
正正就是一種公義最原始的狀態.
所以你可以想到所謂天國.
公義冠冕天國就是這樣的狀態.
世間一切的事情.
一切的事物.
一切的關係.
都能夠重新置放在一個完美正確有秩序.
井井有條永恆黃金比例的水平線上.
一切的不正常.
不對等.
不公平.
不正確.
不真實的事情.
都重新擺放在一個正常.
對等.
公平.
正確.
真實的狀態.
所以公義是一件很治癒系的事情.
相反.
所謂不公義.
大概你可以找一些強迫症朋友不要看的影片.
你找找看.
那些叫做Unsatisfying Video.
就是籃球射不進去.
USB線插不進去.
切蛋糕切得很醜的那些.
我預備了一條這樣的影片.
叫做強迫症人士不要看的影片.
大家看看.
頂得順眼.
啊.
哇.
你.
你翻了.
對.
什麼啊.
啊.

$^{281}$喔.
啊.
轉轉轉.
啊.
轉轉轉.
喔.
啊.
背背背背.
啊.
背.
啊.
什麼啊.
啊.
轉轉.
啊.
這個.
哇.
這個.
啊.
順序錯了.
啊.
圓圓的.
啊.
為什麼要射得那麼輕.
怎麼射得那麼輕.
啊.
真的.
啊.
這個.
這個就是不公義了.
這個就是一切不公義最原始的狀態.
就是一個不合擺位.
就是不正確不正常的狀態.
這種就是朱古力豆擺錯.
USB線插進去也插不進去.
切蛋糕也很醜.
哇 線擺得什麼都不清.
就是不公義的最原始的原型.
所以無論是人內心的不義.
或者是社會的不公平.

$^{321}$或者是人與人之間的不對等.
通通都是一種不完整的狀況.
所以回到最初的問題.
為什麼保羅在《經文》裡面要提及公義的冠冕呢.
原因很簡單.
就是因為保羅在離世前要得到公義的冠冕.
為什麼呢.
因為他在一生裡都處於一個不公平.
不公義和不義的世界裡.
這個豈不是保羅一生的寫照嗎.
保羅被監禁 被打 被捕 受辱.
他被人誣告 被攻擊 被誤會.
一件事.
他處於暴政的羅馬帝國之下.
他目睹很多人被不公平的對待.
很多很多不圓滿 不正常的事情發生.
甚至他看到自己作為一個罪人.
都是一個不公義的寫照.
一個不義的人.
這些一切都是不義的.
都是不公義的.
都是擺放錯了的朱古力豆.
這個保羅在世傳活的寫照.
正正都是我們今天的寫照.
明明要十五分鐘之內到的卻不見人.
明明可以進來去崇拜的卻無端端被禁止.
明明殺了人卻沒有付任何代價.
明明已經裝得很好的卻無端端掉下來.
世界上很多的委屈.
很多的誤會.
一些不信任 一輩子的誤解.
一輩子的偏見.
一輩子的仇恨.
我們都活在一個不正常的世界裡面.
或者說一個不完美不完整的過程裡面.
不正常 不理想 不整齊.
事與願違.
正正這個不義和不公的背景.
保羅寫在一生裡的總結.
那美好的仗我已經打過了.

$^{361}$當跑的路我已經跑盡了.
所信的道我都守住了.
正正在這個不公不義的狀態之下.
保羅要去寫下他生命最後的留言.
今天我們嘗試從這個角度.
從一個錯放擺位不公義的角度.
來理解保羅這三句話.
這三個他一生裡面的原則.
美好的仗我打過了.
當跑的路我已經跑盡了.
所信的道我都守住了.
或者說我所信的我都守住了.
保羅以上這三個意境.
這三個隱喻.
這三個行動 這三個原則.
正正是面對這種公義冠冕獻決不在場的狀態之下去寫成的.
所以我們下半段就會去一個三點式講道.
講這三點出來 很簡單地說.
美好的仗.
A good fight to fight.
第二個是當跑的路.
A race to finish.
第三就是要守住的信.
The faith to keep.
我們先說第一個.
美好的仗.
A good fight to fight.
這是最基本的聖經觀察.
在這三個隱喻裡面.
唯一一個有形容詞的隱喻就是這個.
他說有一個A good fight to fight.
保羅特別是用一個形容詞加在這個fight裡面.
叫做The good fight.
不過其實good fight是一個很奇怪的字眼.
明明是fight.
明明是一種很不好的東西.
為甚麼又叫做good呢.
明明是一件流汗,辛苦,掙扎.
甚至乎是要流血的事情.
跟人打架.

$^{401}$為甚麼又會是good呢.
我懷疑將good字和fight字黏在一起用的.
大概就是二千年前的保羅.
他是第一個人首先將這兩個字.
強行拉在一起說的.
面對一個不公義,不完整的世界.
所以確實人生是一場競鬥.
一場打架.
一個fighting club.
如果你看過Rocky第六集.
Rocky Boba的劇話.
有一段寫得很漂亮的劇本.
一段非常漂亮的說話.
用fighting來比喻人生.
他說The world ain't all sunshine and rainbows.
It's a very mean and nasty place.
And I don't care how tough you are.
I'm gonna hit you to your knees.
And keep you there permanently if you let it.
You, me or nobody is gonna hit as hard as life.
用廣東話的翻譯就是.
這個世界不會只有陽光和彩虹.
這個世界很mean,很骯髒.
無論你多tough也好.
這個世界都會打到你跪下.
要你永遠跪下.
如果讓他這樣做的話.
世上沒有人打你.
打得比你的命還要金.
所以人生是一個很tough的過程.
就是一個fighting.
為什麼保羅會稱這個叫做good fight呢.
保羅仍然稱它叫做good fight.
因為保羅知道自己在fight什麼.
保羅享受這個fight的過程.
保羅知道最後fight完之後的結果.
保羅有一個good fight to fight.
因為這場流汗,流血,流淚的打架.
基本上,本質上,總的來說,總括而言.
總而言之.

$^{441}$after all, over all, above all.
是美好的.
可能大家都知道我最近學電單車,學二鹿.
發覺二鹿發燒友其實是一個很特別的動物.
尤其是最近35度的天氣.
發覺熱到怕街.
當全世界在塞車的時候.
前面有個大車在你後面.
在你前面的時候,真是滴汗.
開到一個地方,找個馬路或骨位來泊.
泊不到,基本上是泊到濕透.
所以之前我幻想過自己將來可以穿西裝.
可以去踩單車,廣道.
基本上這件事是不可能的.
濕到爆是會是.
不過對於那些駕二鹿的人來說.
整件事after all, over all, above all.
仍然是正的.
仍然是good的.
當你開到一段直路.
當你一上公路的時候.
風吹過來,仍然是美好的事情.
聽姐妹,保羅將生命比喻做good fight.
就是要好好去描述我們和我們人生之間的那種關係.
面對著這個很多不公不義的世界.
很多狗屎的世界.
你仍然要去努力奮鬥.
用力出拳,給予掙扎.
流汗,流淚,流血.
但是當你靜下來的時候.
你仍然可以去笑著說.
我的人生仍然是一場good fight.
人對你不公平嗎?.
這個世界,這個社會不公義嗎?.
面對的現實不對樣子嗎?.
這個世界的不理想.
仍然我們是帶著一種這樣的笑容去掙扎.
純粹嘗試去享受基督和你同行的過程.
不要懷著太多的憤怒.
不要懷著太多的惡毒和怨恨.

$^{481}$懷著微笑去面對著打這場不公義,不合理,不正常的仗.
仍然去笑著覺得這場彈仔交是good的.
這是第一個.
保羅第二個隱喻是當跑的路.
保羅臨死前說自己完成了他要當跑的路.
不過要跑這條當跑的路.
最大最大的難度.
就是你不知道什麼叫當跑的路.
一句聽起來好像很廢的說話.
永遠只有跑完的人.
才知道自己應該跑哪條路.
你說是不是?.
是永遠跑完整個路程.
你才知道自己跑過什麼路.
我現在大概可以跟別人說.
上帝真的呼召我做傳道人.
因為我基本上都做了十五年.
十六年都做了十年.
雖然我不肯定自己是不是忠心的傳道人.
因為我還沒有到最後一刻.
但我幾乎可以知道我是蒙召的.
因為我都做了這麼久.
你移民到一個新的地方.
你問這是不是上帝呼召你去呢?.
你說肯定是.
但你是不是pretty sure肯定是呢?.
當你子女長大了.
當你有孫子的時候.
成口英文的時候.
你到時就可以說.
上帝真的呼召我來這裡.
真的.
永遠都是你走到最後的時候.
你才能夠確定這是上主的命定.
或者公道一點說.
如果人生是一場跑步的話.
我們大概都可以看得見前面.
十五秒至二十秒之間.
可以做些什麼.
有什麼路可以跑.

$^{521}$前面有什麼石可以避.
但如果是更加遠更加遠的路程.
你就不知道跑去哪.
什麼時候停.
怎樣跑這些東西.
沒有人可以肯定自己要跑哪條路.
每個人都要懷著信心地跑下去.
並且在一切完成之前.
都只能夠漸漸,進步一步一步地.
發現自己人生裡面每一個階段的路程.
這正正就是當跑的路最難的地方.
你不知道怎樣跑.
不知道跑去哪.
都不知道什麼時候跑完.
不過有一件事你是知道的.
最少.
就是你可以繼續跑下去.
雙腳不停地揮動.
雙手不停地擺放.
一隻腳離你同時.
一隻腳用力地抓地.
心臟保持著地膨脹收縮.
讓自己的肺部有規律地將氧氣上體內.
不斷地重複下去.
這樣就是一個不斷跑步的過程.
面對著這個不公不義的世界.
繼續跑下去.
跑到上帝叫你停為止.
或者.
跑著跑著發現自己越走越不對路.
要quit.
要離開.
要轉行.
要休息.
要移民.
但你繼續只能夠跑下去.
繼續keep moving forward.
這是保羅第二個生命的原則.
正如Rocky這樣說.
But it ain't about how hard you hit.

$^{561}$It's about how hard you can get hit.
and keep moving forward.
How much you can take and keep moving forward.
不是在乎你有多大力的勢力.
而是無論你吃了多大力.
你都繼續地去下去.
不要放棄仍然的堅持.
這是保羅第二個很重要的原則.
所以保羅頭兩個隱喻就是這樣.
認定這個人生的象是美好的.
並且無論如何.
keep住繼續地奔跑.
最後最後.
我們見到保羅最後第三個隱喻.
其實這個不是隱喻.
這是保羅直接地說出來.
保羅說要好好地去守住你的信.
keep your faith.
羅本說所信的道都守住了.
但原來裡面有一個很簡單的一句話.
就是信我已經守住了.
這裡所說的信可以解作信仰.
可以解作你的基督教信仰.
但它的意思絕對不單單是你的基督教信仰.
或者你的信耶穌這麼簡單.
因為faith是雙方的.
一個一體的兩面.
它是你對上帝的信心.
更加是上帝對你的忠信.
信心和忠信連結在一起.
正正就是一個約定.
天父上帝和你的約定.
你和天父上帝的約定.
我們說我們都知道上主命定了你的一生.
全部都是天父的安排.
無論你今天離開香港.
今天你要離開一個地方.
一個團體.
今天你要轉行.
你要轉型.

$^{601}$這些全部都是天父上帝的保守帶領.
但你都有你的partner去做.
堅決地相信他.
堅持忠信於他.
這個約定將要陪伴你一生的路程.
陪伴著你要打那個美好的奮鬥.
直到你兩鬢單白的時候.
直到你離開世界的時候.
我們懷著笑容認定這個fight是good的.
我們不斷地不放棄.
繼續走下去.
無論這個世界有多不對稱.
多麼的什麼對稱都好.
但我們不斷地做這兩件事.
最後的是好好守住這份上主的約定.
上主的約定是永恆的.
你當下正是要盡力地去保持上帝這份相信.
謹守著上帝這份承諾.
接下來我們會唱一首詩歌.
就是Come Let Us Flow.
原來Come Let Us Flow這首詩歌的一段歌詞.
正正就是提摩太后書第四章的經文.
她說在這黑暗荒誕的時代.
痛苦掙扎像無盡等待.
承載恩典跑到終點得冠冕.
我要飛向天際的曙光.
我主應許是無盡探望.
以信心以忠信豁出去.
繼續行這條未知的旅程.
各位福出的弟兄姊妹.
無論你這一刻身處在哪裡.
我奉耶穌的名.
邀請你和天父上帝做一個約定.
想想在你面前當下有什麼心事.
你要去解決.
想想你當下困擾你的心事.
面對著這份心事.
你嘗試和天父上帝做一個約定.
我們的父上帝是信實的.
是忠信的.

$^{641}$而我們要憑著信心.
持守著這份忠信去回應他.
好好為這件事打一個約.
邀請你和我一起停下來.
去禱告.
去和天父上帝為你當下生命的事情.
做一個約定.
這個約定是你的禱告.
也是天父上帝和你的恩典之間的關係.
邀請你如果可以的話.
也可以將這份約定寫在你的留言區裡面.
我們一起放上去和天父上帝一起納約.
我真的祈禱.
讓你知道你是那位信實的神.
你在我們生命的起初.
你在我們生命的終點裡面.
在永恆裡面當下看著我們此刻的人生.
我們一起去求你將我們心頭裡面的大石.
將我們的心事來禱告你.
交託給你.
我們一起和你納了一個這樣的約定.
我們願意去跟隨.
願意去回應你.
你同時也帶領著我們.
主人我們知道我們生命裡面每一個階段.
無論是大小的困難.
都是有你自己的命定和旨意.
因此我求主一齊的來到.
將我們此刻的困難交託給你.
我們願意忠信於你.
我們願意持守我們的生命.
在這個不公義,不如意,不理想的處境裡面.
我們仍然的向前行.
仍然的認定這件是美善的大靈.
仍然去知道這是你自己為我們預備的事.
我們不放棄.
我們繼續相信.
求主你這樣來引導我們.
奉主命求.
阿門.

\newpage



\section{加拉太書 1:1-12-20220806}
\label{sec:0mi_NsvpcRc}
\textbf{【網上崇拜】自由是最美好的禮物|加拉太書1\_1-12|20220806 [0mi\_NsvpcRc]}
\newline
\newline
連結: \href{https://youtube.com/watch?v=0mi_NsvpcRc}{\texttt{ https://youtube.com/watch?v=0mi\_NsvpcRc}} ~~~~ 語音日期: 2022-08-06 
\newline
\newline
\hyperref[sec:muF9XhDEVwY]{\small{< < < PREV SERMON < < <}}
~
\hyperref[sec:index_chronic]{\small{[返順時目]}}
~
\hyperref[sec:index_scriptual]{\small{[返順卷目]}}
~
\hyperref[sec:hkGSf0_Eoow]{\small{> > > NEXT SERMON > > >}}
\newline
\newline
加拉太書 1:1-12-20220806
\newline
\begin{longtable}{cl}
\hline
\hline
章節 & 經文 (和合本修訂版)\\
\hline
1:1 & \begin{tabularx}{0.7\textwidth}{X} 我使徒保羅和所有跟我一起的弟兄,寫信給加拉太的眾教會。我作使徒不是由於人,也不是藉著人,而是藉著耶穌基督與使他從死人中復活的父神。 \end{tabularx} \\ \\ \relax
1:2 & \begin{tabularx}{0.7\textwidth}{X} 見上節 \end{tabularx} \\ \\ \relax
1:3 & \begin{tabularx}{0.7\textwidth}{X} 願恩惠、平安從我們的父神和主耶穌基督歸給你們! \end{tabularx} \\ \\ \relax
1:4 & \begin{tabularx}{0.7\textwidth}{X} 基督照我們父神的旨意,為我們的罪捨己,要救我們脫離現今這罪惡的世代。 \end{tabularx} \\ \\ \relax
1:5 & \begin{tabularx}{0.7\textwidth}{X} 願榮耀歸給神,直到永永遠遠。阿們! \end{tabularx} \\ \\ \relax
1:6 & \begin{tabularx}{0.7\textwidth}{X} 我很驚訝你們這麼快就離開那位藉著基督之恩呼召你們的神,而去隨從別的福音; \end{tabularx} \\ \\ \relax
1:7 & \begin{tabularx}{0.7\textwidth}{X} 其實並沒有另一個福音,不過有些人騷擾你們,要把基督的福音更改了。 \end{tabularx} \\ \\ \relax
1:8 & \begin{tabularx}{0.7\textwidth}{X} 但無論是我們或是天上來的使者,若傳福音給你們,與我們所傳給你們的不同,他該受詛咒! \end{tabularx} \\ \\ \relax
1:9 & \begin{tabularx}{0.7\textwidth}{X} 我們已經說了,現在我再說,若有人傳福音給你們,與你們以往所領受的不同,他該受詛咒! \end{tabularx} \\ \\ \relax
1:10 & \begin{tabularx}{0.7\textwidth}{X} 我現在是要得人的心,還是要得神的心呢?難道我在討人的喜歡嗎?我若仍舊想討人的喜歡,我就不是基督的僕人了。 \end{tabularx} \\ \\ \relax
1:11 & \begin{tabularx}{0.7\textwidth}{X} 弟兄們,我要你們知道,我所傳的福音不是按照人的意思; \end{tabularx} \\ \\ \relax
1:12 & \begin{tabularx}{0.7\textwidth}{X} 因為我不是從人領受的,也不是人教導我的,而是藉著耶穌基督的啟示而來。 \end{tabularx} \\ \\ \relax
1:13 & \begin{tabularx}{0.7\textwidth}{X} 你們聽說過從前我在猶太教中的行徑,我怎樣竭力壓迫殘害神的教會。 \end{tabularx} \\ \\ \relax
1:14 & \begin{tabularx}{0.7\textwidth}{X} 在猶太教中,我比本國許多同輩的人更激進,為我祖宗的傳統更熱心。 \end{tabularx} \\ \\ \relax
1:15 & \begin{tabularx}{0.7\textwidth}{X} 然而,那位把我從母腹裡分別出來、又施恩呼召我的神,既然樂意 \end{tabularx} \\ \\ \relax
1:16 & \begin{tabularx}{0.7\textwidth}{X} 把他兒子啟示在我心裡,讓我在外邦人中傳揚他,我就沒有跟有血有肉的人商量, \end{tabularx} \\ \\ \relax
1:17 & \begin{tabularx}{0.7\textwidth}{X} 也沒有上耶路撒冷去見那些比我先作使徒的,惟獨到阿拉伯去,後來又回到大馬士革。 \end{tabularx} \\ \\ \relax
1:18 & \begin{tabularx}{0.7\textwidth}{X} 過了三年,我才上耶路撒冷去見磯法,和他同住了十五天。 \end{tabularx} \\ \\ \relax
1:19 & \begin{tabularx}{0.7\textwidth}{X} 至於別的使徒,除了主的兄弟雅各,我都沒有見過。 \end{tabularx} \\ \\ \relax
1:20 & \begin{tabularx}{0.7\textwidth}{X} 我現在寫給你們的是在神面前說的,不說謊話。 \end{tabularx} \\ \\ \relax
1:21 & \begin{tabularx}{0.7\textwidth}{X} 以後我到了敘利亞和基利家一帶; \end{tabularx} \\ \\ \relax
1:22 & \begin{tabularx}{0.7\textwidth}{X} 那時,在基督裡的猶太各教會都沒有見過我的面。 \end{tabularx} \\ \\ \relax
1:23 & \begin{tabularx}{0.7\textwidth}{X} 不過他們聽說「那從前壓迫我們的,現在竟傳揚他原先所殘害的信仰」。 \end{tabularx} \\ \\ \relax
1:24 & \begin{tabularx}{0.7\textwidth}{X} 他們就為我的緣故歸榮耀給神。 \end{tabularx} \\ \\
[1ex]
\hline
\hline
\end{longtable}
$^{1}$頂姐妹平安.
剛才看到留言.
看到有加拿大的頂姐妹.
你們真的很早.
希望你們不是睡不著.
因為你們還沒天亮.
但是都.
很感受到.
無論在哪個地方.
在什麼時間.
都能夠一起敬拜.
很大的恩典.
所以.
我希望你們.
很大的恩典.
今天我們.
轉了場地.
真的對我們來說.
哪個地方都能一起.
敬拜.
雖然氣氛好像.
跟我們之前的氣氛不同.
但是能夠有頂姐妹一起聚集.
就是一個敬拜的地方.
多謝敬拜隊選的歌.
我自己很享受.
剛才那種.
在那個很甜靜的氛圍.
裡面去敬拜.
自己得到.
一種很重要的.
安書.
但今天選的經文.
其實就不是那麼輕鬆.
或者不是那麼.
容易去.
消化.
因為今天選的經文是一段身份認同的經文.
在約定.
這個主題裡面.

$^{41}$很希望有些很重要的訊息.
在當中跟頂姐妹重溫.
可能對於你來說.
是你過去.
信主的時候的101課程.
但我相信經過.
這三年香港的經歷的時候.
特別是在.
教會的生態.
裡面的經歷的時候.
這些經文對你來說.
可能是重新去認識.
又或者是重新你自己.
在靈理上的那種認知.
我選的是.
迦太書的第一章.
第二章的經文.
因為講的是自由是最好的禮物.
如果你有看過迦太書.
你都知道.
是保羅他很.
慾縱心腸去警醒一班.
他曾經帶過的弟兄姊妹.
他很稀奇.
他們很快離開了當初.
傳給他們的訊息.
迦太書你說.
是很容易懂.
也是的.
因為很多東西都很平鋪直敘.
很白的.
但你說是很容易掌握.
又不是的.
因為裡面牽涉到很多核心價值.
就是一個信徒要怎樣去表達自己的信仰.
又在當中.
很多人會指指點點的時候.
你怎樣去明白.
我就不是因為你指指點點才做.
我是因為你沒有指指點點.

$^{81}$我都會做.
而你指指點點.
是不會令我害怕的.
這個是一個.
很重要的訊息.
迦太書是一個.
保羅很早期寫的書信.
貼上來迦前後書.
和他同期.
都是保羅在第一二次宣教旅程.
當中經歷迦太地區的時候.
他建立一班弟兄姊妹.
面對他們.
走錯了的路.
他寫了一封信.
讓他們明白到重要的信仰基礎.
是什麼.
我們試試重溫一下.
這十二字經文.
以致我們一次過看到.
保羅書寫的脈絡.
我讀出來.
弟兄姊妹.
在螢光幕旁邊都可以一起讀.
簡介是迦太書第一章.
第一節起.
作仕途的保羅不是由於人.
也不是藉著人.
乃是藉著耶穌基督.
與叫他從死裡復活的.
父上帝.
和一切與我同在的眾弟兄.
寫信給迦太的各教會.
願因平安.
從父上帝.
與我們的主耶穌基督.
歸於你們.
基督照我們父上帝的旨意.
為我們的罪寫記.
要救我們脫離.

$^{121}$這罪惡的世代.
但願榮耀歸於上帝.
直到永永遠遠.
阿們.
我希期你們.
這麼快離開那.
是因為我.
是藉著基督之恩.
照你們的.
去從別的福音.
那並不是福音.
不過有些人.
搞擾你們.
要把基督的福音更改了.
但無論是我們.
是天上來的使者.
若傳福音給你們.
與我們所傳給你們的不同.
他就應當被就坐.
我們已經說了.
與我們所領受的不同.
他就應當被就坐.
我現在是要得人的心呢?.
還是要得上帝的心呢?.
我豈是討人的喜歡嗎?.
若營救討人的喜歡.
我就不是基督的僕人了.
弟兄們.
我告訴你們.
我素來所傳的福音.
不是出於人的意思.
因為我不是從人領受的.
也不是人所教導的.
我不是從人領受的.
也不是人所教導的.
乃是從基督耶穌.
啟示來的.
我第一次祈禱.
天父上帝.
每當我們打開.

$^{161}$你的說話的時候.
你仍然對我們說話.
保羅當日語重心長.
寫了一封信給加泰教會.
求主你繼續對我們說話.
以致我們認清我們的身份.
又認清福音的工作.
又認清耶穌基督.
代屬的功能.
求主你恩代.
指責提醒.
奉耶穌的名求.
如果你看過保羅書信的時候.
加泰書的開局.
一開首的時候.
跟其他保羅書信有幾明顯不同.
我們從第一二節開始.
你會看到作詞和保羅.
裡面有一個描述.
不是由於人也不是直著人.
乃是直著基督耶穌.
叫他從死裡復活的父上帝.
第一個.
不是由於人是一個眾數.
其實意味著不是一群人.
告訴我.
或者我位於那群人裡面.
在那個群組裡面的人.
假設當時是一群.
十二門徒當中.
也不是直著人.
那個是單數的人.
所以意味著.
不是透過巴拿巴這個人引誡.
而整件事.
不是直著一個群體.
不是直著哪一個人.
是直著耶穌基督.
他自己親自的參與.
這個也是保羅.

$^{201}$被人質疑他私底下身份的起因.
對於保羅.
他要告訴加泰教會的人.
他憑著什麼權柄.
什麼身份.
去說這番話.
不是一個群體.
也不是來自哪一個人.
是直著上帝他自己親自的呼召.
他寫給.
加泰的各教會.
加泰是一個地區.
這個地區根據保羅的.
宣教旅程的過程當中.
可以分南面和北面.
有不同的教會.
散居在當中.
保羅在當中傳遞.
福音的核心信仰給他們.
這群弟兄姊妹.
他面對的困難就是.
很多時候會說.
他帶著什麼身份.
說這番話.
這對於我們也是.
很多時候我們在一個群體當中.
或者在一個工作環境當中.
我們都會收到卡片.
有時候卡片就知道他代表什麼.
原來是一個普通的.
Project Manager.
每個人都叫Project Manager.
就是跟班.
我沒有offense.
先不要.
我先戴上頭盔.
因為卡片是你自己印的.
還是公司呢.
很多就算.
你說Director也好.

$^{241}$有些大公司很多都叫Director.
有時候就是.
大家都會看個Title.
輪支排輩就成為.
你是帶著什麼.
原因或者帶著什麼.
崗位身份來.
是的,很多時候都會被質疑.
我自己都身歷其境.
就是.
有時候就是.
你開會的時候你帶著什麼身份.
去的時候.
別人就會猜度你其實.
你後面的後台是什麼.
保羅很強調.
就是不是一班人.
不是耶路撒冷教會.
也不是哪一個.
帶他信主.
是每一個都是上帝親自呼召的人.
每一個帶著.
科音信息都是來自耶穌基督的.
救贖工作.
他寫給宗教會讓人.
明白到如果那個.
環境仍然是輪支排輩.
仍然是看Title的時候.
其實這個不是最.
主要的福音信仰.
在經文下面.
跟大家看看的時候.
就是看在第十二節.
在這段經文裡面.
的最後那句.
就是你會看到.
保羅在結束這段.
說話裡面的時候.
他也重申我不是從人領受.
不是哪一個教我.

$^{281}$是耶穌基督.
他親自review.
啟示讓我們明白.
啟示可能對很多弟兄姊妹.
是一個很神學的名詞.
但啟示其實就是.
上帝說話的方式.
上帝用人理解的方式去啟示.
他自己本身.
簡單在.
堂會主一學的時候我也說啟示.
啟示上帝對人說話.
啟示的過程當中.
就是不說你不知道.
不說真的不知道為什麼.
耶穌基督有這個方法.
但說你也知道不完.
因為上帝會按人的需要.
或者理解當中.
逐步告訴人.
但一定會說多一點.
知道多一點.
重要的事情上帝一定會透過不同的人.
告訴我們.
剛才在敬拜的詩歌.
你也會感受到.
唱歌的內容.
歌詞描述有些東西.
我們猜測不到.
但上帝就畢竟成就了啟示.
保羅在.
《哥倫多書信》也說過.
特別二章那裡.
就是上帝為人所.
預備的是什麼呢.
是眼睛未曾看見.
耳朵未曾聽見.
人心也未曾想過的事情.
其實有些事情我們沒有想過.
但上帝竟然用這些.

$^{321}$這樣也行嗎.
何必這麼麻煩.
這麼蠢.
這就是保羅說的.
在人看為愚蠢.
但在上帝卻是救贖的大能.
上帝很多時候.
就是不用人的方法去想.
或者不用人覺得重要.
或者理性合理的方法.
但上帝就按人的心意.
去讓人明白.
上帝工作的原因.
所以很多時候我們.
看聖經都是一開始不太明白.
但有些人生歷練的時候.
或者經歷的時候.
你會明白多一點.
聖經有個很奇妙的地方.
就是一開始的時候.
你自己用自己的角度去看.
帶茶經或者弟兄姊妹一起去思考的時候.
又會多一點不同的角度.
保羅在第一節.
和第十二節帶出的訊息.
其實領受著什麼呢.
如果不是從人的.
不是從群體.
是上帝有什麼重要的東西要告訴我們呢.
就在這十幾節經文裡面.
跟大家慢慢去看下去.
下一章的時候你會看到.
第三至第五節的經文.
是很典型的保羅問安的經文.
剛才說到.
加泰書.
貼壽樓加錢書.
是早期寫的書信.
你會看到後面寫的書信.
包括是.

$^{361}$菲勒比書.
或者是前後書.
羅馬書的問安是很直接的.
就是願因為平安.
從上帝歸於你們.
但在對白裡面.
第三節和第五節.
就是願因為平安和願榮耀.
都說了.
但你會看到.
在問安的過程當中.
夾雜了一句話.
就是下一章經文.
第四節.
在兩個問安當中夾雜了.
基督召我們.
負上帝的旨意.
為我們的罪寫記.
要救我們脫離這罪惡的時代.
其實有些怪的.
問安就問安.
為什麼中間夾雜了一句話呢.
這句話是什麼意思呢.
如果我們看完.
加泰書後面的經文.
你就會明白.
為什麼保羅會說這句話.
因為寫信的那個人.
很心急.
很想告訴你.
這封信的主旨是什麼.
這封信的主旨就是告訴你.
是耶穌基督.
為我們所做的工作.
為我們做的工作是什麼呢.
就是照顧上帝的旨意.
來到我們中間.
為我們的罪寫記.
而令我們脫離這個罪惡的時代.
今天對於大家來說.

$^{401}$可能這是101課程.
你已經知道.
我們全部都是重新了幾個圖.
但在當時來說.
他們要脫離一個.
猶太人的信仰.
猶太人的信仰就是守律法.
脫罪.
守律法去獻祭.
為自己的罪.
去贖罪.
對他們來說.
袋贖是一個很新的觀念.
對他們來說.
是不容易消化的觀念.
保羅寫一封信.
給迦太教會.
讓他們明白到.
其實真正的福音就是這個.
人不能夠離開.
罪的捆綁.
所以要透過一個途徑.
而那個途徑.
不是過去猶太人一直持守.
那個守律法.
稱義守律法.
是可以袋贖的觀念.
而真正是來自.
耶穌基督承擔了.
我們這個真正的.
贖罪的羔羊.
對他們來說.
是一個一定要重新去確立的地方.
因為在當時.
我們理解.
就是.
猶太裔.
或者是一些猶太派的基督徒.
他們本身是猶太人.
他們信了耶穌.

$^{441}$他們真的信耶穌.
但他們沒有離開了.
守猶太教裡面的.
守節.
或者是守律法的要求.
他們將這個信息帶到.
迦太教會當中.
是叫那些已經確信了.
他們自己基督信仰.
重新的基督徒.
要歸化了猶太人的文化.
要守國禮.
要行禮如儀.
做回要守律法的要求.
這個正正就是保羅.
要義正.
慈艷的告訴你.
不是的,不是要做這件事.
福音已經補滿了.
耶穌基督已經完成了.
救贖的功能.
我們信的是死而復活的耶穌.
他已經完成了.
上帝對罪的要求.
不需要再守律法.
不再需要重蹈.
猶太人守的.
律法的要求.
但對於當時的人來說.
後面的經文.
要講一個很重要的信息.
就是.
他們做了很多細微的工作.
下一章你會看到.
在迦太書裡面.
第3,4,5,6,每一章裡面.
都講一些很重要的信息.
保羅其實在後面的篇章.
逐步和他們拆解.
他們和你說什麼.

$^{481}$其實你們不要這麼無知.
我已經和你說過.
現在已經完成了救贖.
不要再被人迷惑.
要外添一些工作.
第二,他們很熱心的待你.
其實不是好意.
他們想離間了.
當初你所信的內容.
其實你.
其實你也很好.
其實有誰攔阻你呢?.
叫你們不信服原先傳給你的真理.
第6章裡面提到.
凡稀圖外貌體面的人.
都勉強你們行國禮.
無非是怕.
自己為基督十字架受逼迫.
因為其實這裡說的是.
保羅都曾經被逼迫過基督徒.
但是他自己悔改之後.
他重新清晰.
基督信仰的核心所在是什麼.
對於這些猶太派基督徒.
他們沒有離開.
守律法的要求.
因為在於我們來說.
其實都不是很重要.
在於我們來說是什麼呢?.
其實都不難理解.
因為在基督信仰.
最大的挑戰.
叫人接受就是.
有沒有這麼便宜.
就是信就行.
不需要附帶什麼條件.
我不知道.
你有沒有看漫畫.
如果你有看漫畫.
你會看那套.

$^{521}$常常都說等價交換.
這個世界很多天.
都沒有免費午餐.
每樣東西都有價值.
只差你交換.
是否交換了.
或者什麼形態都好.
總是有些條款有些條件.
總不是說講信就算了.
難就難在.
就是要做些事.
還有很多人覺得做了些事.
是心安理得.
保羅其實在加拿大書.
下面的經文裡面.
都拆解一個很重要的信息.
就是律法對人的提醒.
律法是叫人明白自己.
有大罪之身.
你無能力自救.
你不是守律法就可以清義.
但要靠信耶穌而清義.
保羅不是反對律法.
對人的規範和要求.
律法是好的.
沒有規範人很容易.
就會放縱.
律法的本意是叫人明白.
到自限.
保羅很清楚告訴你.
是良心提醒.
但不要讓.
律法成為我們的核制.
如果律法成為核制.
成為門檻成為條件.
有些人就用這個條件來核制你.
這也是耶穌在地上的時候.
跟那些門生講過.
就是逛街.
見到那些法利賽人.

$^{561}$就是跟那些門生講.
凡他們教訓你們.
你們要遵守因為他們坐在摩西的位上.
但不要效法他們.
因為他們能說不能行.
耶穌其實明白到.
這些行律法的人是有心無力的.
但他將那不能負的.
押加諸別人身上.
今天這件事再重現在加泰教會裡面.
就是那些猶太派的基督徒.
他們沒有掉到律法上.
反而為了加泰教會的信徒.
他們就說你們要歸化了猶太教.
來補滿.
他們的行為工作.
保羅就說這個不是福音.
真正的福音就是.
基督按上帝的旨意.
受死.
為我們的罪補滿.
死而埋葬.
第三天復活.
這就是基督信仰的核心.
我們是信這個.
不是靠別人的行為.
正正就好像.
大家熟悉的金句.
就是二忽所書第二章經文.
我們得救是本福音.
不是出於行為.
是藉著恩典.
保羅很希望你們剛開始上力的時候.
不要忘記了.
因為很容易入屋.
做些事心安理得.
如果你有看過我們神學八課的時候.
我們其中有一個是講靈修的.
就是靈命修持.
今天有興趣可以重溫我們youtube頻道的神學八課.

$^{601}$關於現代人的靈命修持課堂內容.
就是很多時候用很多benchmark去評估我們自己的靈命.
包括用聚會的形式.
用讀經.
又或者用其他背金句.
又或者是一些粗練.
我又沒有做.
首先我強調.
如果你看到這裡的時候.
我不是反對聚會.
也不是反對讀經.
也不是反對你熟練粗練的那種習慣.
不是.
但不是做多少.
做多與少來討上帝的喜悅.
那怎麼解釋.
有時候弟兄姊妹.
過去都說潘Sir我覺得自己靈命很差.
然後我說為什麼你覺得靈命差.
我很久沒有回教會.
我很久沒有祈禱.
我覺得自己沒有了這些東西.
覺得與上帝很疏遠.
覺得自己很差.
我說那你回到上帝的身邊.
但是我很怕.
但是我很認真跟他說.
就是其實你當初是否缺志的時候.
你不是很認真.
你不是很相信耶穌基督為你的罪而死.
而你是真心承認耶穌基督成為你個人的救主.
他說他相信.
你曾經也熱心回教會.
因為很多其他緣故離開了教會.
他說是.
但我說.
你真誠缺志.
相信上帝那一刻.
就是一個很重要的缺志禱告.
上帝已經是寫下了.

$^{641}$上帝已經赦免了你.
那是什麼意思.
我說基督的補血已經浮避了你.
已經洗淨了你的罪.
你是因耶穌基督的血而重新的基督徒.
意味著.
這個例子可能有點敏感.
但事實上是真實的.
你如果是我的兒子.
你要跟我脫離關係的話.
不是登報就能解決問題.
你登報也解決不了問題.
因為你是我的兒子.
你是有我的DNA的.
在法律上你可能不是不認我兒子.
但你仍然有我的血緣關係.
你離不開我和你父子的關係.
意味著你真心相信耶穌基督.
上帝的血洗淨了我們的罪.
我們已經是上帝的兒女.
《約翰福音》第一章第十二節.
「凡接待他的名,就是信他名的人,他就似他們權柄,作上帝的兒女.」.
你已經是再一次能夠成為上帝的兒女.
你沒有離開過上帝和你之間的舊屬關係.
那你現在怎麼解釋.
你只不過很久沒有和他聊天.
你很久沒有親近他.
你沒有了一個很親密的溝通關係.
我希望你明白.
不是所有宗教情操的benchmark.
是可以主導你和上帝的關係.
他是輔助你.
是幫助你.
但那個是不會扭曲你和上帝之間的關係.
真正和上帝之間的關係.
是耶穌基督的工作.
這就是保羅切切在說.
保羅在第四節說得很清楚.
「願因平安和榮耀中間夾雜了基督耶穌的工作.
才能夠補滿了上帝在我們當中的恩典.」.

$^{681}$不是我們做什麼.
不是出於人為.
也不是出於自己.
猶太派的基督徒就說要附加律法.
附加其他東西去補助.
以致討上帝的喜悅.
保羅說什麼?.
保羅在下一節經文說第六節.
這是別的福音.
所以保羅在第六節說.
很稀奇為什麼你這麼快離開這個信仰.
不穩定.
是喔,真的要做點事.
做點事安心一點.
我一直有讀經有祈禱.
但我也強調.
你有讀經有祈禱.
那是令你更加明白上帝的話.
更加能夠保持和上帝的溝通關係.
這是好的.
但那不是令你自己安心.
不是令你覺得自己好像和上帝更加緊密.
對於我們來說.
那不是福音.
福音是耶穌基督單方面為我們做的工作.
而我們是接受上帝的恩典.
所以在第六節到第九節裡.
在下一章的經文.
我跟大家排一排的時候.
你就會發現一件事.
就是保羅要很清楚.
紫色的字說的就是.
真正叫我們得贖的是基督的CALL.
是基督呼召我們.
感召我們.
讓我們能夠接受祂.
令我們接受祂的時候.
這就是福音真正的工作.
所以基督的恩典是說.
Carries of Christ.

$^{721}$是說上帝的禮物.
上帝的禮物就是給了我們.
我們願不願意接受呢.
上帝.
有些人.
說真的.
現在有很多禮物.
你可以拒絕的.
正如上帝向很多人發出邀請.
讓他們去接受福音.
很多人都會拒絕.
你都可能曾經拒絕過很多次.
但是總有一天.
上帝的CALL來到的時候.
你接受了.
這就是上帝給我們的福氣.
不需要任何的.
保羅很嚴正地告訴大家.
凡是在耶穌基督的工作以外.
加添任何條件的.
都是搗亂了福音.
都是被人要咒詛.
因為他們覺得.
耶穌的工作不夠.
所以要靠外添的東西加進去.
加泰書裡面說得很清楚.
如果你八月還沒開始讀經的經卷的時候.
加泰書有六章.
你一個月慢慢讀這六章的時候.
你就會看到.
保羅為什麼這麼心急.
因為他不想當初.
懷著一個很純真心的人.
去相信福音.
去接受的時候.
反而讓其他人去沾染了.
你們要加這些東西.
你們要做這些東西.
你們要受國禮.
你們要守律法.

$^{761}$這樣才能夠補滿基督的工作.
保羅說不是.
不是這樣的意思.
真正的意思是.
耶穌基督已經完成了這件事情.
對於你來說可能很差.
但是我希望你明白到.
今天我們面對的身份就是.
有很多東西就有比較.
我自己在Folks Church講道的時候都講過.
疫情其實在挑戰我們對信仰的規範.
和對聚會的形態是什麼.
有些姐妹覺得.
網上直播沒有禮儀感.
一定要發現場崇拜才能夠集中.
這個是她的感受.
但有些覺得我其實都不需要.
我看了就行了.
所以我看完就算了.
就算是睡著看也好.
什麼都好.
每個人都有自己的習性.
但我仍然強調就是.
你要調整你自己.
其實你帶著什麼心態去崇拜.
這個很重要.
第二件事就是.
疫情沒有了一些常規的聚會.
你可以很.
同教會很疏離.
你可以很多.
以前別人提醒你要回聚會.
沒有人提了.
因為沒有聚會的時候.
你真的可以不回聚會.
你可以走在鐘擺的另一邊.
但我也見過很多弟兄姐妹.
在鐘擺的另一邊.
就因為沒有了常規聚會.
又不需要服事.

$^{801}$他一個星期聽三四堂道.
是不同教會的道.
又上不同的無牆教室.
聽不同的課程.
他靈明和聖經的知識又多而又多.
反而現在可以回到實體的時候.
他就更加問我.
下一步想讀神學.
我應該要尋求什麼呢?.
在疫情之下.
信仰真的有很多不同的擺位.
但仍然是問.
你自己要想想真正核心信仰是什麼.
無論什麼擺位都好.
不要作出一個.
有很多不必要的比較.
但最重要的就是知道.
耶穌仍然在我們當中.
我們什麼隨時隨地都可以一起去敬拜.
所以.
保羅要澄清一個很重要.
有很多人覺得保羅.
用一些東西來說人.
用什麼身份來說人.
或者他帶著一些討好人的心態.
去說我們那些搞亂人的不是.
第十節說了.
保羅很認真.
其實如果你看那個語氣.
寫封信一開始.
寫到這些語氣的時候.
其實你可能不想看後面.
因為好像在罵人.
但保羅很認真說.
第十節.
我現在要得人的心.
還是要得上帝的心呢?.
如果我真的要討人喜歡.
其實他就不是基督的僕人.
下一章我特意分開和大家看看.

$^{841}$其實第一個問題.
我現在要得人的心嗎?.
保羅的答案不是.
基本上他很清楚.
要籠絡什麼人.
我不是來自耶路撒冷.
使徒那種傳遞的訊息.
他要做外邦人.
他不是討人喜歡.
他不是.
他清楚自己是一個外邦的使徒.
他所做的職份是什麼.
我是不是要得上帝的心呢?.
這兩個答案都說得通.
第一.
是.
保羅很清楚.
他就是要得上帝的心.
因為他就要清楚他的呼召.
他就要履行他欠福音的債.
他就做上帝要他的工作.
他是討上帝的歡喜.
但又可以不是.
這個不是的答案是什麼呢?.
其實.
我不需要討上帝的心.
因為基督已經覆蓋我.
奉給我.
不是我做什麼可以討上帝的心.
而是基督已經被上帝接納.
是基督接納我.
以至我被上帝接納.
不是我做什麼可以討上帝.
討不到的.
但基督已經可以蓋過我.
已經可以補滿上帝的心意.
所以在討上帝的心意裡.
又是又不是.
但無論如何.
都不是人可以做什麼.

$^{881}$就是耶穌基督做工作.
所以.
他跟著說.
我喜事討人喜歡嗎?.
我營救討人喜歡.
我就不是基督的僕人了.
僕人這個字就出現了.
僕人是什麼意思?.
你說服人就是做事.
服人就是僕人就是做事.
但在保羅的書信裡.
保羅要提醒.
特別在這十二節經文.
當中我抽下面第一節和第十節出來給大家看.
第一節開頭就是.
作師徒的保羅不是猶豫人也不是直著人.
而是直著基督和死而復活的上帝.
而第十節裡面說.
基督的僕人就是.
你看到師徒保羅和基督的僕人.
這兩個字在保羅的身份出現.
就是.
他做這個師徒.
看上去是很高級的身份.
但是在保羅的眼中.
他是一個僕人.
我不知道你本身對僕人的觀念是什麼.
或許你家裡也有外傭.
我家裡也有.
這個例子可能在香港不是很完全.
因為你看不懂的群組.
有些外傭很厲害.
可能很能說話.
但用僕人的觀念是什麼呢.
僕人的觀念就是.
僕人很清楚一個很重要.
就是管家的職份.
做得僕人就很清楚他不是主人.
就算主人不在家裡.
他也是僕人.

$^{921}$很清楚自己身份的角色.
第二件事.
僕人很清楚什麼呢.
他的工作範圍是做什麼.
沒有僕人是不知道自己的工作範圍.
因為做得你的僕人就是你的工作範圍.
第三.
第一個就是很清楚.
就算主人不在家裡.
他也是僕人.
身份.
第二件事是工作範圍.
第三件事就是時限.
僕人很清楚每件事要做多久.
還有主人什麼時候回來.
回來的時候要交給主人.
做得清清楚楚.
第四.
僕人很清楚他的工作能力.
因為主人一定要知道他的工作能力.
委派他的工作能力.
做好他的工作.
第五.
僕人不會很清楚.
他做不到這件事的時候.
他的結果是什麼.
所以無論是身份.
工作範圍.
能力和時限.
這五件事.
做一個僕人.
是一定很清楚.
所以保羅.
你看起來一個仕途的職份很超越.
但是他更加清楚.
我做一個僕人.
這個觀念其實是引用自.
《生命記》第十五章十六十七節說的.
一個曾經是被賣做了奴隸的.
他會有一個限期.

$^{961}$他會釋放.
釋放了他.
但是他不想走.
因為主人很好.
於是主人很好的收納他.
就成為一個畜奴.
畜奴其實將來.
就會演變成為他家裡的管家.
他其實可以有自由.
可以自由地離開.
但是他因為這裡很好.
主人很好.
他想回頭去服侍主人.
自己甘願做一個僕人.
保羅要帶出一個信息就是.
其實我們已經很自由了.
因為信耶穌的時候.
我們什麼都好做.
我們不再需要受罪的克制.
也不再需要受旁人的指指點點.
我們可以是一個自由人.
但是保羅就清楚知道.
我以往很密守律法.
為的事就迴避了上帝的憤怒.
今天耶穌基督的工作已經奉庇了我.
我再不需要很硬手的律法.
而我成為一個自由人.
我是因著耶穌基督的工作.
我願意成為耶穌的僕人.
去服侍耶穌.
這就是保羅要帶出的信息.
其實他可以很自由.
但我反而放棄我原本可以自自由的東西.
去成為耶穌的僕人.
你拿著這個鑰匙去看保羅的書信的時候.
你就知道為什麼保羅面對這麼多.
千辛萬苦他都要完成上帝給他的工作.
因為他很清楚自己的身份.
他很清楚自己的能力.
很清楚他的服侍範圍.

$^{1001}$很清楚每樣東西都有時限.
這就是保滿主人要他做的工作.
親愛的弟妹.
今天我們要有一個身份認同.
你留在香港也好.
或者你散居在不同地方也好.
你仍然是耶穌基督的兒女.
你仍然是耶穌基督的僕人.
我們在哪一個地方也好.
我們仍然可以自由地去做我們所做的工作.
不需要有太多額外的比較.
也不需要有什麼交換性質.
但我們就可以做好我們自己僕人身份的工作.
所以去到最後第十一二節的內容裡面說什麼呢?.
就是弟兄們.
我告訴你們.
我素來所傳的福音不是出於人的而是.
因為我不是從人領受的.
也不是從人教導我的.
乃是從基督耶穌啟示來的.
你就會明白一件事.
他真正不是要服務人.
不是要聽人指指點點.
他真正要服務的是耶穌.
是耶穌令他得自由.
是耶穌親自釋放他.
不需要密守律法的要求.
自由本身是一個很重要.
上帝給我們的禮物.
保羅無懼了一個恐懼的自由.
不怕自己做漏事.
不怕自己沒做事.
因為耶穌基督已經保滿了.
真正的福音是上帝令我們有釋放.
可以自由地去做我們心所願的事.
在疫情當下.
現在已經三年了.
我經常看YouTube.
當然會分享我們的YouTube頻道給弟兄姊妹.
但我看YouTube的時候發現了.

$^{1041}$其實我自己很喜歡看不同的人物的專訪.
有很多頻道我都特別常常留意.
其中一個頻道.
是講述700萬香港人的生活.
可能你有看過也會留意到我在講什麼.
我每次看的時候都很感動.
因為真的反映了很多香港人的生活.
不需要有很特定的身份.
不需要有一個什麼機構支持.
不需要有什麼人贊助.
他們自己打工就做另一件事.
下班以後或是下班以外的時間.
全力去做好自己的生活型態.
我每次都很欣賞那些香港人.
很欣賞那些年輕人.
大家都是咬緊牙根在香港繼續工作.
但每次看完的時候.
我心裡都欣賞之外.
其實我都不開心.
因為我覺得今天基督徒的見證很乏力.
基督徒的見證可能都仍然留在教會怎麼做.
Flow Church的教會官.
在Info Group裡第一堂已經講得很清楚.
我們是期望和希望和弟兄姊妹一起.
將基督信仰改變我們的生命的能力.
去散開在我們接觸的生活環境裡.
Flow Church沒有一些內部架構.
去叫弟兄姊妹有什麼事工.
你做什麼.
不是這樣的形式.
我們反而是想重整弟兄姊妹.
你的身份而有那種自由的能力.
你就在你接觸的群體當中.
去用你這個生命特質去影響其他人.
如果外面坊間的人沒有基督信仰.
他自發.
他看到社會需要.
都有這個打一份工以外.
用自己的時間影響他人.
我們有基督而有那種自由的禮物.

$^{1081}$我們豈不是可以做另外一個感染力嗎?.
我希望在這些空間不斷地思考.
過去教會的生活環境當中.
已經主導了很多.
用時工模式去主導弟兄姊妹可以服侍的空間.
但是我希望.
如果你一直都是在Flow Church參與.
或者你認同Flow Church的教會觀的時候.
我們很希望.
我們沒有實體聚會的性質.
但我們有散居的性質.
我們散了在我們周遭接觸的群體當中.
就讓我們能夠感受上帝給我們得著自由.
這個生活形態.
可以繼續去影響身邊的人.
一會兒回應詩是唱得著自由.
是我自己一首很喜歡的詩歌.
因為每一次聽首詩歌的時候.
我再一次去認定.
是主耶穌基督的工作釋放了我們.
以致我們不用再怕.
做不夠.
守不夠.
而有那種恐懼.
耶穌基督恢復了我們和上帝之間.
父與子的兒女關係.
願意我們仍然持守這個等待的盼望.
在不同的環境當中.
繼續過我們的基督徒生活.
求主幫助.
\newpage



\section{申命記尼希米記 30:1-4}
\label{sec:hkGSf0_Eoow}
\textbf{【網上崇拜】同是天涯流落人|申命記30\_1-4;尼希米記1\_8-9|20220813 [hkGSf0-Eoow]}
\newline
\newline
連結: \href{https://youtube.com/watch?v=hkGSf0-Eoow}{\texttt{ https://youtube.com/watch?v=hkGSf0-Eoow}} ~~~~ 語音日期: 2022-08-13 
\newline
\newline
\hyperref[sec:0mi_NsvpcRc]{\small{< < < PREV SERMON < < <}}
~
\hyperref[sec:index_chronic]{\small{[返順時目]}}
~
\hyperref[sec:index_scriptual]{\small{[返順卷目]}}
~
\hyperref[sec:sQEhDyhKFnE]{\small{> > > NEXT SERMON > > >}}
\newline
\newline
$^{1}$頂枝妹平安.
很開心見到大家.
真的很感動.
剛才在應拜的時候.
大家來到這裡.
其實已經是一件很奇妙的事情.
先來個問安.
倫敦的頂枝妹大家好.
祝你們平安.
願你們平安.
香港的頂枝妹大家好.
很開心今天在倫敦.
我們Folk Church今天在倫敦直播.
這是我們第一次這樣去做.
所以頂枝妹我們背後有四五百人在後面.
我們在網上.
但今天這篇講道.
是會對海外的弟兄姊妹說.
所以今天的講題.
今天的角度和訊息.
都是針對海外的Folk Church弟兄姊妹.
如果按以前教會的說法.
今天就是outfold主日.
今天的主題是海外弟兄姊妹.
不過不要緊.
我們香港頂枝妹先說說.
今天早上我聽到潘Sir說.
我們紅磡那邊的場地.
冷氣已經全部安裝了.
我們下個星期.
就可以回到紅磡那邊.
參與現場崇拜直播.
所以我明白平時在海外頂枝妹.
看我們outfold.
看我們YouTube的時候.
大家都是很好.
大家都是在海外.
但都繼續參與崇拜.
但都明白有時候.
我們都是會說香港的事情.

$^{41}$或者我們針對香港的弟兄姊妹去說.
今天我們特別來到這裡.
特別去探望大家.
說一下為什麼我會來英國.
其實我是路經的.
我不是坐飛機過來港島.
我不是走的.
這次是我自己在神學院.
來到公幹.
一個人來的working trip.
所以來探訪一下.
我看到一些校友和同學.
也探訪了很多不同的英國教會.
所以今天星期六路經倫敦.
遮掩耳朵也好.
什麼也好.
當然來到Full Church裡面.
我昨天五點半才下飛機.
所以今天早上六點起床.
很厲害.
可以說是睡過夜.
聲音大聲一點.
跟大家見面.
我由倫敦留到星期一.
之後我會開車去不同的城市.
下個星期六.
我應該會和萬城的Full Church頂支部一起崇拜.
我們在萬城有另一個聚會點.
昨天我們Full Church崇拜rehearsal的時候.
有位姐妹問我.
John你來到倫敦Full Church有什麼感覺.
當時我就不回答她.
問她有什麼感覺.
我感覺就好像自己去了另一個multiverse裡面的Full Church.
就是不同的宇宙裡面.
同一群弟兄姊妹.
你看到外面有同一張627257的海報.
但那個是英國電話.
其實是同一樣差不多的東西.
但細節是不同的.

$^{81}$所以同一個Full Church崇拜.
同一群弟兄姊妹.
但完全不同的感覺.
不同的地方.
但都是我們Full Church的群體.
我從來都沒有想過自己會在這樣的教堂.
樓底那麼高.
穿著T恤的講道.
是很奇妙的事情.
他們說會有耶穌光在後面.
Full Church三年.
我們一開始的時間就是在石劍尾神朝會.
有沒有人參加過的?.
石劍尾年代都有.
然後我們就去希伯倫堂.
想不到在倫敦Shepherds Bush的St. Simon Church.
我們都有一個流唐崇拜.
所以我們今天很開心.
能夠和大家見面是一件很開心的事情.
我們一起來祈禱.
一起來聆聽上帝的說話.
因為你的恩惠你的慈愛.
繼續跟隨我們.
我們無論漂泊到任何一個地方.
我們都仍然知道.
你的恩守你的慈神愛塑.
在我們後面在我們前面.
在我們附近.
求主讓我們刺下你今天的說話.
藉著聖經藉著你自己的聖言.
去鼓勵在海外的弟兄姊妹.
讓他們仍然捉緊你自己的說話.
繼續去成為一個基督徒.
成為一個你所喜悅.
你的命定安排的一個羊.
求主你這樣牧養他們.
求主藉著今天孩子不配.
但你仍然藉著今天的訊息幫助我們.
你自己在靈體當中.
親自回應我們生命中的各種困難.

$^{121}$無論在倫敦.
無論在英國的不同城市.
悉尼澳洲加拿大美國.
很多不同的弟兄姊妹.
縱然我們流散到不同的角落.
但我們仍然在崇拜中聚在一起.
求主你親自對我們說話.
逢主命求.
今天我們會說《生命記》第30章.
《生命記》第30章記載了摩西帶領以色列民.
在40年曠野生活中的一個總結.
40年.
年老的摩西帶領著以色列民.
在曠野行走了40年的時間.
由昔日耶路華和以色列民納約的西乃山.
從西乃山開始.
40年的時間.
他們終於走到摩亞平原的一個地方.
大家在摩亞平原裡.
就像大家這樣坐在這裡.
遙望著對面上帝所應許的艱難美地.
在這個時候.
《生命記》第30章記載了摩西對以色列民的一段說話.
請聽我讀出.
摩西在摩亞平原.
40年後對著這群以色列民說.
我所沉鳴在你面前的一切就坐都臨到你身上.
你在耶和華.
你的神追趕你到的萬國中必心裡追念祝福的話.
你和你的子孫若盡心盡性歸向耶和華.
你的上帝照著我今天一切所吩咐的.
聽從他的話.
那時耶和華你的神必連續你.
救回你這被擄的子民.
耶和華你的上帝要回傳過來.
從分散到你的萬國萬民中.
將你焦醉回來.
你必被趕散的人.
就是在天涯的.
耶和華你的神也必從那裡將你焦醉回來.

$^{161}$今天我們花點時間去理解這段經文背後的時空脈絡.
其實這段摩西在摩亞平原對著以色列民所說的話.
是處於一個很複雜的時空裡面.
為什麼這麼說呢.
正如剛才所說.
摩西講述的是整個生命記裡面的尾聲.
整個曠野流散的尾聲.
大家如果記得的話.
我們都知道整卷的生命記.
幾乎都是摩西在摩亞平原裡面的講論.
摩西站在一大片的摩亞平原上.
重新去講述上帝的律法典章.
講過了.
再來到他重申一次.
不過很有趣的是.
去到生命記的尾聲.
生命記的編者.
再次來到去強調.
摩西這番說話的地理位置.
29章第一節說.
「這事而說在摩亞地.
吩咐摩西和以色列民納約的話.
就是在他們在河列山上納約的說話」.
生命記的編者再次強調.
昔日的河列山.
40年前的河列山.
和今天摩亞平原的對比.
昔日的河列山.
今天的摩亞平原.
足足40年.
以色列人原來已經離開了埃及40年.
40年是一段什麼時間呢.
很簡單.
你想想40年前你做了什麼.
是不是.
這裡暗笑.
40年前我都還沒出生.
我也是僅僅出生.
我也是剛剛出生.
沒錯.

$^{201}$40年是一段很長的時間.
40年是一條巨大的.
鴻溝時代的分水嶺.
世代之爭.
想想40年之後.
2062年.
如果你繼續留在英國的話.
你大概是什麼情況.
是不是.
想想40年之後.
香港大概是什麼情景.
你香港的親人又會怎樣.
香港教會又會怎樣.
Flo Church又會怎樣.
如果有的話.
40年是一段很長的時間.
捉星仔.
一群坐在摩訶平原聽摩西講論的以色列民.
其實大部分都是捉星仔.
不知道你介不介意將來的子女.
會被稱呼為捉星仔捉星女.
說廣東話不是很正.
中文不是很厲害.
但滿口流利英文.
對香港不是很熟悉.
這樣的一群人.
這群以色民大部分都是捉星仔捉星女.
他們大部分都不是埃及長大的.
他們跟著自己的父母漂流流散.
他們成為一群流落人.
雖然他們有些人是在埃及出生.
但他們從來都沒有經歷過父母以前的情景.
即父母與法老抗爭時的情景.
他們還未出生 還是很小的時候.
他們不知道父母為何要離開.
他們當時面對著什麼困難.
面對著什麼政局.
他們只是跟著父母離開埃及.
離開後對自己出生的地方沒有印象.
這群捉星仔捉星女.

$^{241}$他們每天都是吃馬拿長大的.
吃足四十年馬拿的人.
即奶茶 他們不懂得喝.
菠蘿包那些不是很懂得吃.
他們要承接著四十年前摩西在西乃山的教導.
重新去斟酌.
將西乃山以四十年前的上帝的誡命.
要帶入迦南.
然後他們要去落地生根.
所以摩西要再次重申.
以後話上帝的律法.
教導這群記者.
這大概就是生命記的主題.
即摩西重申一次.
以前四十年前的教導.
好讓他們可以落地生根.
這個我想大家都聽過.
這些大家都可能都知道.
基本的聖經知識.
不過以下我所說的可能你沒聽過.
其實不是兩個時空.
而是三個時空.
聖經所描述的經正是.
處於一個過去 現在 和將來.
三個很不同的時空裡面.
為什麼這樣說呢.
如果你去留心摩西這段.
在摩亞平原的講述的時候.
你會發現一個很奇怪的東西.
我再一次給大家聽聽.
有什麼特別的東西.
摩西說.
大家有沒有看到.
有沒有什麼東西.
很特別想問為什麼.
在經文裡面.
如果你細心留意經文的時候.
你會發覺.
你問吧.
其實是很奇怪的.

$^{281}$為什麼是秘魯的子民.
為什麼是回傳.
什麼是分散 什麼是天涯.
什麼是朝聚回來.
說什麼.
這群的色靈文.
這群畜生仔.
不是好好的坐在摩亞平原裡面嗎.
他們不是跟爸爸媽媽流散到.
出來這裡.
將要去到一個新的地方嗎.
一個應去之地嗎.
說什麼秘魯.
說什麼回傳.
什麼分散.
什麼天涯.
什麼朝聚回來.
你會發現.
這段說話裡面.
其實是耶和華上帝.
藉著摩西的口.
對著未來人.
所說的話.
或者說.
耶和華上帝是藉著摩西.
對著這群畜生仔.
畜生女.
再對著這群未來人.
說話.
所以經文不單單是回憶.
40年前.
西乃山的教訓.
更加是指向著.
900年之後.
以色列人亡國.
秘魯再次流散異鄉群體.
摩西對著這群畜生仔說話.
透過對他們的說話.
也是上帝對著將來.
900年之後的人.

$^{321}$來對他們說話.
所以摩西地區出現了.
三個平行時空.
40年前的西乃山.
現在的摩西平原.
將來.
900年之後.
秘魯到巴比倫.
的以色列文.
所以今天那張海報的經文.
我們加多一段經文.
就是來自於未來的經文.
尼希米記第一章.
八到第九節.
摩亞平原.
900年後.
以色列文.
當他們在迦南地再次流落異鄉.
國破家亡之後.
900年之後.
尼希米對著當時在異地的以色列文.
這樣說.
求你紀念所吩咐你的僕人摩西的話.
說你們若犯罪.
我就把你們分散到萬民中.
但你們若歸向我.
遵守遵行我的誡命.
你們被趕散的人.
雖在天涯.
也必從那裡將他們焦聚回來.
大到我所選擇.
立為我名的居所.
你會發現經文幾乎一模一樣.
生命記和尼希米記.
幾乎差不多一段相似的經文.
生命記和尼希米記.
雖然幾乎一模一樣.
但你會發覺兩段經文.
處於兩個不同的背景.
不同的時空.

$^{361}$不同處境的書卷.
兩段書卷相差幾乎一百年.
一千年時間.
生命記說的是.
以色列文從埃及流散到迦南地.
尼希米記說的是一千年之後.
這班人的子子孫孫.
選擇從巴比倫回流到自己的地方.
如果你眼尖的時候.
你會發現兩段經文都很特別重視一個.
今天我們所說的重點主題.
一個特別的字眼.
經文是重覆了一句這樣的說話.
它說你被趕散的人.
就是在天涯的.
而說神也必從那裡將你焦聚回來.
這是生命記.
尼米基也差不多.
你們被趕散的人.
須在天涯.
我也必從那裡將他們焦聚回來.
兩句是一模一樣.
今天我們就用天涯.
來說我們今天說到的一個主題.
《本聖經》很特別.
在經文裡面特別翻做天涯這個字.
天涯原文希伯來文就是.
Besekh Hashemayim.
可以說是The End of Heaven.
這個意思.
Besekh Hashemayim.
就是The End of Heaven.
這個字在其他經文也有.
生命記 詩篇 以創亞書也有.
但在和本就翻成了天邊這個字.
只是在這段經文裡就翻成了天涯這個字.
我很喜歡這個翻譯.
天涯比原文更有詩意的字眼.
天涯是什麼意思呢.
顧名思義天涯就是一個很遙遠的地方.

$^{401}$對於當時聖經年代的人來說.
所謂天涯就是古代人.
看到最遠最遠最遠的地方.
所謂天和地相連的地方.
今天所說的地平線.
Horizon.
就是人類用肉眼目測最遠的地方.
就叫做天涯.
天和地連接的地方就叫天涯.
是我們人類認知最遠的地方.
以前的人.
這個人類肉眼目測最遠的地方有多遠呢.
其實不太遠.
如果你在倫敦橋那裡看.
最遠的地方在哪裡呢.
也是倫敦吧.
倫敦也很大.
你看到也是倫敦.
我猜吧,我不太熟悉.
我小時候住在元朗的公屋.
我們在騎樓那裡看.
看到最遠最遠的地方是哪裡呢.
就是深圳.
深圳的樓就看到了.
這就是我看到的天涯.
元朗最不忘不忘的地方.
就是深圳的樓.
所以對於摩西和那一代人來說.
他們在摩亞平原上.
他們能夠想像的天涯.
大概就是這個曠野的盡頭.
或者他們能夠遙遠看到那個未知的迦南地.
很模糊的迦南地.
對於尼希米和他們那一代人來說.
他們所說的天涯有點不同.
對他們來說.
天涯是一個很遠很遠的地方.
就是他們自己身處的巴比倫.
他們就在天涯那裡.
很遠很遠,距離耶路撒冷很遠.

$^{441}$所以以前的以色列人祈禱.
當他們離開耶路撒冷.
他們仍然會望向東方.
就是東邊的耶路撒冷.
望向那個方向來祈禱.
這就是這樣的方法.
所以大家以後也可以這樣做.
望向東邊,望向香港.
圍著香港祈禱.
這樣會比較有效.
這個就是天涯的意思.
望著最遠的地平線.
雖然都在倫敦.
但你望向方向,幻想一下天涯.
就是這樣.
這是古代人的看法.
對於我們日世紀一般的現代人來說.
所謂天涯,所謂遠.
又是什麼意思呢.
大家知道蘇格蘭有個地方叫天涯海角.
叫做John Road.
其實都不太遠.
特別是現在大家在倫敦住.
都不太遠.
你開車去的話要開多久.
十個小時.
差不多就能上去.
就可以去到這個天涯海角的地方.
老實說,如今對於這個世界來說.
所謂沒有什麼地方叫遠.
凡在地球都不能叫遠.
你可以坐飛機上網.
同步直播Zoom.
每個星期我們留堂崇拜.
雖然我們直播的地方在紅磡.
但你們在不同的角落.
打開YouTube.
就是這裡.
相反對於我們今天.
我們現代人來說.

$^{481}$所謂天涯.
所謂遠或者近.
其實是成為了一個很主觀的事情.
在感受上多一點.
是感受多一點的.
摩西時代的人來說.
覺得自己身處於天涯.
因為他們在摩亞平原上.
覺得自己和四十年前的埃及很遙遠.
尼西米基時代的人來說.
他們在天涯裡.
因為他們覺得自己距離自己的家鄉很遠.
事實上我們回顧猶太人歷史的時候.
我們發現整個舊約的歷史.
都是一段流散的歷史.
從阿伯拉罕的吾爾.
流散到約瑟年代的埃及.
由摩西年代的埃及.
流散到約書亞的曠野.
由曠野流散到迦南地.
由迦南地避露到亞述巴比倫.
你問究竟以色列人的家在哪裡.
這群人所謂原本的地方在哪裡.
什麼叫做近 什麼叫做遠.
其實沒有一個絕對的答案.
蘇格蘭有個天涯海角.
海南島也有一個.
倫敦的倫敦海島.
深圳也有一個.
什麼叫天涯 什麼叫海角.
都是一個相對的問題.
是一個很感受上的問題.
所以鄧小平.
這幾年我們在香港所經歷的.
我們經歷了的.
叫我們下定決心.
做一個決定.
找一個足夠遙遠的地方.
一個足夠遙遠的距離.
走到一個天涯.

$^{521}$去到一個足夠遠的地方.
好讓我們能夠離開.
那個我不能夠說的東西.
跟它保持一定的距離.
當然當你離開了.
那個我不能夠說的東西.
你跟它保持距離.
同時無可避免地.
跟你所愛的人.
愛你的人.
你所愛的地方.
都遠離了.
原來所謂移民.
就是去到某個距離甚遠的天涯.
讓自己跟一些不想要的東西.
能夠走遠一點.
跟你自己的理想.
跟你自己的生活.
可以走近一點.
倫敦是一個遙遠的地方.
不是距離問題.
而是這個倫敦.
是跟你某些東西很近.
跟你的理想比較接近.
跟那個政權比較遠.
剛剛過去一個星期.
我結婚18週年.
看不出來吧.
18週年.
上個星期我跟太太去了大澳.
Staycation.
其實大澳也算是一個天涯.
大澳位於大嶼山最西邊的地方.
也是香港最西邊的地方.
我們星期日出發.
初時打算在屯門坐船去大澳.
誰知十點半船很早就爆炸了.
坐不了船.
於是改坐車去東涌.
第一次坐屯赤隧道.

$^{561}$經過隧道去到東涌.
看到360.
發現自己也沒有坐360.
不如試試坐遊客.
坐纜車去到岸平.
再轉巴士去到大澳.
小時候暑假也會跟同學去大澳玩.
因為我同學家裡有一個漁棚在大澳.
過幾晚很好玩.
游泳.
我20年沒去過大澳了.
中午去到大澳.
立即試試大澳的蝦醬炒飯.
不知道大家還記不記得.
整個大澳是蝦醬味.
到處走走坐坐船.
大澳的船家說.
坐船可以看到中華白海豚.
當然看不到.
付了錢就什麼都沒有.
看到很多大嶼山的海.
很美很美.
到黃昏的時候.
我和太太訂了大澳文物酒店.
是全大澳最高檔的餐廳.
我們從小鴻森走過去.
沿著海邊一路走.
經過蝦醬廠買手信.
其實是很浪漫的.
由大澳走到酒店的長廊.
你會發覺黃昏時.
整條街都是貓.
是很棒的.
整條街都是貓.
一條街最少有六七隻貓.
就睡在街上.
這就是很美的大澳.
好像不覺得是香港.
整條街都是貓.
如果你想找貓拍照.

$^{601}$我們拍到你傻.
一百多張照片都拍不完.
很多隻貓在那裡.
一路走一路走.
我突然和太太說.
將來退休如果想移民.
不如移民來大澳.
太太已經習慣了我發瘋的說話.
都不理我.
我說真的.
移民不如來大澳.
最少回香港探親.
搭橋回到香港.
很快.
快很多.
她說大澳很藍的.
我說無所謂.
到時候藍和黃我都不在乎.
我只想過一個平靜.
退休農村的生活.
在大嶼山拿禁區制裁.
只要搭電單車.
過著簡樸的生活.
國安法影響不到我的.
當然不是叫大家移民去大澳.
這不是認真的想法.
不過這個想法背後的道理.
希望大家能夠體會.
遠或近其實很在乎於我們內心.
你要遠離某些不能夠的東西.
你要遠離那個政權.
你可以去到很遠很遠.
你都可以很近很近.
因為遠或近其實是一件很主觀的事情.
相反你真正要去計較遠近的.
其實不是那個.
而是在天上的那一位.
真正我們要去care.
自己是遠還是近.
是天上的那一位.

$^{641}$先提醒我們.
你和你的子孫若盡心盡性歸向耶和華的神.
照著我今日一切所吩咐的聽從.
他們說那時耶和華的神必連續來救回你這被擄的子民.
耶和華的神要回傳過來.
從分散你到萬民中將你焦聚回來.
你們若歸向我.
遵守我的誡命.
你們被趕殺的人.
雖在天涯.
我又必從那裡將你們焦聚回來.
很近.
帶到我所選立為我命的居所.
今天你走遠遠離了政權.
但你和你的上帝是否很近呢.
你千山萬水來到這個地方.
究竟你是和上帝遠了.
還是距離上帝近了呢.
等批Visa 買樓 辭職 搬屋 搬三次屋 再搬多七次屋.
你和你的上帝是近了還是遠了呢.
世上最遙遠的距離.
就是我站在你面前.
你卻不知道我愛你.
這是張小嫻小說裡的一句金句.
從此這句話成為網絡中經常被引用的說話.
世上最遙遠的距離.
頂智慕對你來說.
什麼叫做遠.
什麼叫做近.
遙遠對你來說.
究竟是什麼叫做真正的遙遠.
或者世上最遙遠的距離.
就是你千山萬水長度不斷拔起.
漂流遠方.
最後你發現你仍然和上帝很遠.
頂智慕奉耶穌的名.
去問大家心裡的那句話.
你遠離了那個政權.
但你有沒有走近你所信的神.
我們再一次去聆聽聖經的說話.

$^{681}$你們若歸向我.
謹守遵行我的命令.
你們被廣散的人衰在天涯.
我也必從那裡將你們焦聚回來.
帶我到你為我命所立的居所.
上帝必然會焦聚我們.
無論我們在哪裡.
上帝上帝必然會焦聚我們.
無論是摩西的年代.
還是尼西米的年代.
縱然大家身處於不同的天涯.
但我們被上帝要求我們做同一件事情.
你們要盡心盡性歸向耶和華.
遵守遵行上帝的誡命.
只要他們和子孫盡心盡性歸向耶和華的時候.
無論他們有多遠.
無論他們在天涯.
他們仍然可以和上帝很近.
這正正就是舊約的流散神學.
重點不是物質的距離.
而是你和上帝的距離.
今天我們同樣領受同一個的命令.
只要我們盡心盡性歸向耶和華.
遵守遵行耶和華的誡命.
無論你在天涯海角.
你在倫敦.
曼徹斯特.
伯明翰.
溫哥華.
多倫多.
悉尼台灣.
香港大澳.
你仍然和耶和華上帝可以很近.
這是上帝昔日的應許和命令.
也是今天我們一群留堂的弟兄姊妹.
領受的命令.
不知道你現在的光景是怎樣.
找到工作了沒有.
是不是又要搬.
塞到好沒有.

$^{721}$你的教會群體是不是又找到了.
將來和你同行的弟兄姊妹.
在這個地方又找到了沒有.
無論在哪裡都好.
只要我們願意一心一意地.
再一次去朝聚一切.
上帝必然地成為我們的主.
這令我想起以前在德國的情景.
歐洲的華人教會.
以前我在德國讀書的時候.
我第一間幫忙建立的教會不是留堂.
而是柏林宣導會.
當時好像現在這樣.
一個三十多人的教會.
一個很小的群體.
但是一群可以稱得上為教會的群體.
因為我們每個星期都能夠真心彼此相愛.
真正的關係.
真正的去見證耶穌基督的群體.
我經常說.
如果倫敦全教會是一個平衡宇宙的全教會.
這間教會某些東西好像很全教會.
但有些東西是有一點點不同的.
我覺得這間教會可以成為一個.
比Folk Church香港更加熟絡.
因為我們這裡更加可以守望.
這個我覺得是香港Folk Church都不容易做到的事情.
因為太大.
但是這裡在不同的角落的Folk Church群體.
真正可以實踐的.
正正是這件事.
只要你仍然做基督徒.
仍然成為一個見證的群體的時候.
無論在哪裡.
我們都一樣可以開始我們的生活.
好了,我停在這裡.
一會兒我會照平時的流塘崇拜的模樣.
好像以前在希伯倫堂.
站在門口送客,說再見.
是的,會再見我們.

$^{761}$說了再見就再見.
我現在就像頭七回來的那些.
飄飄來,說身後事就走.
下星期你見不到我了.
但這個屬靈群體仍然在這裡.
你在海外,在天涯.
這個屬靈的家仍然在這裡.
都請自為家不是一些教會的口號.
而是真真正正讓你在海外.
在流散的地方.
找到大家可以一起走的弟兄姊妹.
一群一起見證基督的群體.
基督耶穌的故事仍然開展.
哪怕政權如何.
弟兄姊妹,真人.
無論你在海外不同的地方.
但願你能夠找到.
自己在那裡的屬靈群體.
我們一起的,繼續去生命永留.
無論在哪一個地方.
我們一起祈禱.
因為你親自帶領我們.
每個人走到不同的地方.
剛好我們叫flow church.
我們叫流堂.
你讓我們流散到不同的地方.
我們知道.
我們只要去遵行你的命令.
招聚在一起.
我們不孤單.
我們仍然可以走一條你所引領的道路.
無論是二十年後,四十年後.
我們仍然繼續去跟隨你.
鼓勵我們,建立我們,激勵我們.
讓我們的生命.
縱然今天離開了那個地方.
但我們是越走越近你自己.
求主你親自感動我們.
讓我們熱切地彼此去抓緊.
去找一個我們彼此相愛的群體.

$^{801}$能夠成為一個新的教會群體.
求主你帶領我們.
奉主命構,阿門.
\newpage



\section{馬太福音 16:13-23-20220820}
\label{sec:sQEhDyhKFnE}
\textbf{【網上聖餐崇拜】跟最難約定的人講約定|馬太福音16\_13-23|20220820 [sQEhDyhKFnE]}
\newline
\newline
連結: \href{https://youtube.com/watch?v=sQEhDyhKFnE}{\texttt{ https://youtube.com/watch?v=sQEhDyhKFnE}} ~~~~ 語音日期: 2022-08-20 
\newline
\newline
\hyperref[sec:hkGSf0_Eoow]{\small{< < < PREV SERMON < < <}}
~
\hyperref[sec:index_chronic]{\small{[返順時目]}}
~
\hyperref[sec:index_scriptual]{\small{[返順卷目]}}
~
\hyperref[sec:pF3HM3BllyQ]{\small{> > > NEXT SERMON > > >}}
\newline
\newline
馬太福音 16:13-23-20220820
\newline
\begin{longtable}{cl}
\hline
\hline
章節 & 經文 (和合本修訂版)\\
\hline
16:13 & \begin{tabularx}{0.7\textwidth}{X} 耶穌到了凱撒利亞.腓立比的境內,就問門徒:「人們說人子是誰?」 \end{tabularx} \\ \\ \relax
16:14 & \begin{tabularx}{0.7\textwidth}{X} 他們說:「有人說是施洗的約翰;有人說是以利亞;又有人說是耶利米或是先知中的一位。」 \end{tabularx} \\ \\ \relax
16:15 & \begin{tabularx}{0.7\textwidth}{X} 耶穌問他們:「你們說我是誰?」 \end{tabularx} \\ \\ \relax
16:16 & \begin{tabularx}{0.7\textwidth}{X} 西門‧彼得回答說:「你是基督,是永生神的兒子。」 \end{tabularx} \\ \\ \relax
16:17 & \begin{tabularx}{0.7\textwidth}{X} 耶穌回答他說:「約拿的兒子西門,你是有福的!因為這不是屬血肉的啟示你的,而是我在天上的父啟示的。 \end{tabularx} \\ \\ \relax
16:18 & \begin{tabularx}{0.7\textwidth}{X} 我還告訴你,你是彼得,我要把我的教會建造在這磐石上,陰間的權柄不能勝過它。 \end{tabularx} \\ \\ \relax
16:19 & \begin{tabularx}{0.7\textwidth}{X} 我要把天國的鑰匙給你,凡你在地上所捆綁的,在天上也要捆綁;凡你在地上所釋放的,在天上也要釋放。」 \end{tabularx} \\ \\ \relax
16:20 & \begin{tabularx}{0.7\textwidth}{X} 當時,耶穌囑咐門徒不可對任何人說他是基督。 \end{tabularx} \\ \\ \relax
16:21 & \begin{tabularx}{0.7\textwidth}{X} 從那時起,耶穌才向門徒明說,他必須上耶路撒冷去,受長老、祭司長和文士許多的苦,並且被殺,第三天復活。 \end{tabularx} \\ \\ \relax
16:22 & \begin{tabularx}{0.7\textwidth}{X} 彼得就拉著他,責備他說:「主啊,千萬不可如此!這事絕不可臨到你身上。」 \end{tabularx} \\ \\ \relax
16:23 & \begin{tabularx}{0.7\textwidth}{X} 耶穌轉過來,對彼得說:「撒但,退到我後邊去!你是我的絆腳石,因為你不體會神的心意,而是體會人的意思。」 \end{tabularx} \\ \\ \relax
16:24 & \begin{tabularx}{0.7\textwidth}{X} 於是耶穌對門徒說:「若有人要跟從我,就當捨己,背起自己的十字架來跟從我。 \end{tabularx} \\ \\ \relax
16:25 & \begin{tabularx}{0.7\textwidth}{X} 因為凡要救自己生命的,要喪失生命;凡為我喪失生命的,要得著生命。 \end{tabularx} \\ \\ \relax
16:26 & \begin{tabularx}{0.7\textwidth}{X} 人若賺得全世界,賠上自己的生命,有甚麼益處呢?人還能拿甚麼換生命呢? \end{tabularx} \\ \\ \relax
16:27 & \begin{tabularx}{0.7\textwidth}{X} 人子要在他父的榮耀裡與他的眾使者一起來臨,那時候,他要照各人的行為報應各人。 \end{tabularx} \\ \\ \relax
16:28 & \begin{tabularx}{0.7\textwidth}{X} 我實在告訴你們,站在這裡的,有人在沒經歷死亡以前,必定看見人子來到他的國裡。」 \end{tabularx} \\ \\
[1ex]
\hline
\hline
\end{longtable}
$^{1}$剛才敬拜完之後 心裡面頗掛念的是.
最近看了一個新聞 關於台灣一個女孩在東南亞所發生的事情.
這件事牽涉到很多問題.
香港也有十多個人 台灣也有不少的人.
在我們唱敬拜讚美她的時候 很多人的生命都落在很危難的當中.
最近有些片段在TG Group裡面的情況出現.
不單止是東南亞發生 如果大家再追溯東南亞發生的事情.
背後也有很多複雜的事情 牽涉到很多問題.
就算現在烏克蘭裡面有很多逃難了的人 去了世界各地不同地方.
其實販賣人口的集團在不同地方都在做很多類似的工作.
希望我們記掛著這個世界上的需要.
今天我們特別想說的是馬太福音十六章.
我們看看下一頁的經文.
今天我們會看十六章的23至33節.
我想說一下這段經文 其實是一段很特別的經文.
經文是這樣說的 耶穌來到凱撒尼亞菲勒比.
問門徒人子是誰.
有人說是伊利亞 有人說是耶利米 其中才是一個.
耶穌問他們 你們覺得我是誰.
亞瑟文彼得回答 他說你是基督永活神的兒子.
耶穌對他說約拿的兒子西門你有福了.
他啟示你不是血氣的人 而是我們天上的父.
我還要告訴你 你是彼得.
我要將這個盆石建立在我的教會.
死亡的權勢都不能勝過他 再下一頁.
我要把天國約時給你 你在地上捆綁 在天上也捆綁.
你在地上釋放 在天上也會釋放.
耶穌隨即叫門徒來 不要對他說任何人 他是基督.
21至23節 在這個時候耶穌向門徒指出.
他必須到耶路撒冷去.
在眾長 路祭司長 同經學家手下受很多苦並且被殺.
第三天復活 彼得就將他拉到一邊.
掌摑耶穌說千萬不要 這件事不會發生在身上.
耶穌轉過來對彼得說 撒旦退到後面去.
這是害人的陷阱 因為你不關心神的事 只關心人的事.
這段經文是很特別的經文 我們看下一頁的powerpoint.
其實不知道大家有沒有看到地圖.
你會看到藍色的地方叫加利利湖.
耶穌的加利利湖在加利利湖.
你看一下 該撒尼亞腓拉比在紅色的黑門山下面.

$^{41}$如果看地圖的話.
你會發現上面要經過下鎖 很多地方.
才能到達該撒尼亞腓拉比.
如果去過以色列的話.
你知道那個地方基本上是一個叫做.
Banias 彈的地方 和別斯巴島彈.
最北的地方叫彈.
所以基本上該撒尼亞腓拉比是一個屬於很北部的地方.
我那時候去以色列的時候.
在加利利湖附近有很多resort酒店.
你坐車上去 繞一繞兩小時才能上去.
去到彈 或者去到現在所謂的該撒尼亞腓拉比這個地方.
所以聖經裡面記載 無論是馬太記載 馬可記載 和路加記載也好.
耶穌在記載這件事之前 是記載耶穌五餅二魚的時候.
可能在加利利芬餅完結後 四千人之後發生的事情.
如果你看地圖的話.
加利利在下面 你明白嗎.
海的地方就是加利利湖附近.
發生五餅二魚的事件.
無端端經文提了一件事.
耶穌走了去 經過了該撒尼亞腓拉比.
基本上是不驚訝的.
沒有人會上去北邊的.
就算我們坐車去的時候.
經過高欄高地 也要兩個多小時.
起碼兩個多小時的車程.
才能去到那個地方.
所以如果不車 人走的話.
基本上不是你走一天就能走到.
所以問題關鍵是.
為什麼要記載該撒尼亞腓拉比這個小的地方.
這個角色.
其實聖經特別記載耶穌去了那個地方.
馬可和馬太有記載該撒尼亞腓拉比.
但路加福音發現不太合理.
所以基本上路加記載這件事的時候.
其實沒有提到該撒尼亞腓拉比.
因為不像無端端要走上去一天以上的路程.
再回到加利利.
所以為什麼要在該撒尼亞腓拉比.

$^{81}$發生整件事情呢.
這是很值得去問和討論的一件事情.
我們再按一下PowerPoint.
該撒尼亞腓拉比是什麼呢.
該撒尼亞這個地方代表海撒.
海撒代表Caesar.
Caesar就是羅馬君王的名字.
基本上腓拉比這個地方.
不要想是腓拉比輸了就沒有關係.
其實為什麼這個地方叫該撒尼亞腓拉比呢.
是因為在耶穌出生前的時候.
大希律為了要擦鞋.
大希律就是追殺耶穌的大希律.
要擦鞋.
就把其中一個海撒的王.
叫做所謂的Augustus.
就是奧古士督這個王.
就是羅馬的大君王.
其中一個很厲害的人.
死了之後人們敬拜他.
稱他為神.
稱他為和平的福音.
全部關於耶穌的title都放在海撒尼亞.
這個奧古士督身上.
所以大希律為了擦鞋的緣故.
就特別在這個地方做了一點事.
就是建立了一個廟給他.
建立了一個廟.
專供奉Caesar Augustus.
所以去到這個時候.
海撒尼亞腓拉比這個地方.
本來不是叫這個名字.
是叫另一個名字.
但是大希律做完這件事之後.
他的兒子.
他的兒子大希律死了.
他的兒子就叫做肥尼.
肥立.
肥尼或肥立都好.
就是Philip.

$^{121}$這個兒子就索性.
既然爸爸也會擦鞋.
他就直接將地方的名字改了.
就叫做海撒尼亞肥拉比這個地方.
所以這個名字是這樣的.
但是這個地方有什麼特別呢.
就是這四樣東西.
原來這個地方其實一向都拜假神.
由很早期拜巴力神.
到今時今日你去到這個地方.
是不見得的.
其中拜巴力神.
其實你看到那個地方有很多洞.
其中一個最大的洞.
就是巴力神在那裡.
就是代表他能夠通往陰間的權柄.
所以為什麼經文要說.
彼得陰間的權柄.
就是不能勝過你呢.
是因為那個地方充滿著巴力的故事.
其中有一個很大的洞.
那個洞是什麼意思呢.
就是唯有巴力才能夠通往死任陰間的地方.
所以那個地方基本上是拜開假神.
然後就拜開潘神.
潘神就不關潘Sir的事.
你可以叫他潘神.
你叫我潘Sir 叫他潘神也可以.
不過這個名字不是很老例.
潘神是希臘牧羊的神.
他們都在供奉他.
他是半生羊半生人的類似物體的神明.
他是專門搞很多女人.
所以他老婆在那裡拜他老婆.
總之那裡拜很多東西.
其中最出名的就是潘神的東西.
今時今日你都找到很多關於拜潘神的東西.
到今時今日我想說.
你去到那個地方的時候.
另外就要拜宙斯.

$^{161}$因為潘神屬於希臘神話裡的低層次的神.
最大的神就是宙斯.
所以他也拜宙斯.
他還拜了什麼呢.
就是大希律剛才所說的.
就是拜了奧古斯都.
就是連一個死了的羅馬君王.
大家都供奉他.
所以基本上海撒利亞肥亞比這個地方.
是用來擦鞋的地方.
你知道現在中學都需要讀文件.
學校那些.
很多文件要讀.
文件要寄去學校.
有些人要讀他的東西.
是一樣的.
基本上就是用來.
你想拜哪個神.
擦哪個人的鞋.
你就把文件給不同的地方.
雖然類似的.
基本上是這樣.
但這個地方特別的地方是.
集眾多的神在當中.
很多的神在當中.
你看到這些神.
有真是神 有真是人.
有真是獸放在一起.
所以海撒利亞肥亞比是.
很多神明參集的地方在當中.
如果這樣說的話.
問題關鍵是.
為什麼耶穌要走去那個地方.
然後突然之間.
有人去問他.
你說人子是誰.
這句話是什麼意思.
要明白這件事的話.
我們需要多花一點點力氣.
我們再回到PowerPoint.

$^{201}$我們會容易明白一點.
在經文中有些特別的東西.
是其他經文不會有的.
第一就是海撒利亞肥亞比.
這個地方是偶像和君王出現的地方.
剛才你看到那四個神明.
不止四個 先說這四個.
突然之間有很多神明在那裡出現.
因為那個地方本身在賦予神明出現.
在神明出現的時候.
有一件事就是.
天上的父突然之間給彼得啟示.
他之前都不知道耶穌是基督.
他覺得耶穌是否尼塞亞.
你是不是基督.
他都沒有認到.
但去到那裡.
特別是去到該撒利亞肥亞比的時候.
才突然之間有人跟他說.
天上的父啟示你.
這段經文其實沒有稱讚彼得.
彼得說你是永生永活神的兒子.
你是基督這些.
其實耶穌馬上指著他說.
這些不是血氣能夠明白的.
是天上的父啟示你才明白的.
所以問題是為什麼要在偶像面前.
在很多宋神明面前.
突然之間有一幕戲要演出來.
是天上的父親給彼得啟示.
讓彼得懂得說.
耶穌是基督是永生的兒子.
最後就是石頭.
石頭的故事我們今天不說太多.
希臘文也有兩個字.
Levoise 還是 Petra 兩個字.
我們不說那些複雜的威石頭.
總之這個故事裡面很奇怪.
耶穌說要把盤石彼得建立在他身上.
這個就成為了天主教很多背後故事.

$^{241}$彼得就是第一個主教.
第一個天主教裡面最有權威的人物就是彼得.
因為他就是那個建立盤石.
但是我要強調的是.
在經文裡面出現了三樣東西是這麼特別的.
在眾王偶像面前.
接著發生什麼事.
上帝的靈或天父親自啟示給人知道.
耶穌是基督.
最後就是這個基督下來發展是什麼.
就是他好像一個盤石一樣.
有人會建立在他裡面.
這三樣東西不是偶然出來的.
問題是為什麼突然之間有這三樣東西要這樣說出來.
為什麼要這樣說.
這三樣東西本身有什麼特別.
尤其是頭兩樣.
在眾偶像神明面前.
突然之間你很清醒.
你明白一些東西.
有什麼特別.
要走這麼遠的路.
要走路的話.
我相信坐車兩三個小時.
你走路要走十幾個小時.
有些正經研究學者就說沒有去到那裡.
不過假裝去過那個地方.
用那個地方來說一些東西.
但不要管耶穌有沒有去到.
我們也不會知道.
但在經文表達的很奇怪的是.
為什麼在眾神明面前神啟示給你.
這個不是七師傅的問題.
你有沒有看七師傅.
不是說很多神明面前突然有個啟示.
我知道你怎樣.
我們今天會想這些東西.
我們會想這些東西.
這樣就厲害一點.
但我想今天我們花點氣力.

$^{281}$花一點點氣力.
我們一起來玩一個遊戲.
你猜在舊聖經裡面.
有什麼場景.
是在眾神明偶像面前.
屬於上帝的人有啟示.
你試想一下.
接著又會關石頭事.
你猜在舊聖經裡面有沒有.
有什麼場景想到.
這個應該是我們Full Church不會搞的聖經問答比賽.
我們很少搞這些東西.
有什麼在眾偶像面前.
神會啟示讓人明一些東西.
明完之後就有石頭會出現.
你有沒有想過這些情景.
在舊聖經裡面.
這個故事你要想一想.
第一句話是什麼.
人子是誰.
人子就是英文叫son of man.
你猜son of man這個字在舊有哪個出現過.
如果大家知道關於天啟文學就會知道.
人子是屬於什麼書.
有個蛋字那個.
大耳利書.
其實整個故事是屬於大耳利書第二章.
我們看多一個powerpoint.
我們試一下看第二章.
其實整個第二章大耳利書是說這件事.
說什麼呢.
王召了法術師 術師 巫師 迦納底人.
要將一個夢告訴他們.
你知道大耳利書發生什麼故事.
就是王發了夢他忘記了.
然後召了所有人來問.
你告訴我我發了什麼夢.
沒有人敢告訴你.
你發了夢你忘記了.
你告訴我你夢我幫你解夢.

$^{321}$比虐室更離譜的事情.
虐室都知道夢是什麼.
然後他會解夢.
但王就誇張.
他發了夢我忘記了.
但應該很嚴重.
請各位所有偶像.
所有會法術的人都出來面前.
故事發生什麼.
之後就沒有人敢告訴王發生什麼事.
因為沒有人會說出夢是什麼.
又解了夢.
所以基本上這班人全部會死.
但大耳利你知道有主角光環.
有主角光環就一定出來.
哈哈哈哈我搞定了.
所以基本上去到第二章的時候.
夜間異象裡的奧秘就向大耳利顯現.
然後他又稱頌神.
因為終於救了所有人.
所以基本上這個故事.
奇怪的地方是什麼.
在所有偶像勢力底下.
萬鈞耶和華親自啟示給他的子民大耳利聽.
以至神的子民大耳利能夠救到所有人.
基本上就贏了.
所以如果這樣說的話.
這個故事跟剛才我們在蓋瑟利亞菲勒比的故事類似.
就是在很多神明面前.
耶穌問到底人子是誰.
基本上每個人都不會答對.
因為你看著耶穌的時候.
耶穌是誰呢.
耶穌是一個木匠的兒子.
是一個出身於韓美的人.
他怎麼能夠成為一個真正的尼塞亞君王呢.
所以基本上沒有人相信.
但是這個啟示臨到大耳利這個啟示.
同樣地這次臨到彼得身上.
是林丹氏的.

$^{361}$石頭為什麼要說石頭呢.
因為這個夢發完之後.
發完之後就發現.
原來這個夢裡面是在說很多歷代的君王的興衰.
說完之後就說.
最後有一個很厲害的角度會出現.
這個角度是什麼呢.
就是一個非人手造成的石頭.
就將銅鐵銅泥銀和金全部弄碎.
那些什麼金銀銅鐵石那些.
就代表羅馬波斯馬代.
希臘羅馬.
大約是這樣子.
所以有一個非人手造的石頭會出來.
打贏這些以往世代的君王和偶像.
所以如果這樣說的話.
對於耶穌走去蓋撒尼亞肥勒比去說這件事.
就不是無端端發生出來的.
對於耶穌來說.
祂走去那裡.
祂要將大耳利書第二章的故事.
重新演繹一次在蓋撒尼亞肥勒比.
告訴他的門徒.
大耳利書裡面所說的話.
這次再一次應驗在他們當中.
如果金銀銅鐵泥這些.
代表著什麼巴比倫波斯馬代到希臘羅馬的話.
今時今日這個石頭.
就是非人手造出來的石頭.
能夠打贏一切的角度.
所以對於彼得來說.
如果他能夠明白大耳利書第二章的時候.
其實他會興奮的.
他彷彿他的身份等同於大耳利.
其實他也不明白耶穌基督是誰.
有人說是以利亞.
有人說是耶利米.
有人說是死去的先知其中一個.
但彼得能夠受到啟示.
能夠明白聰明的地方是什麼.

$^{401}$他會說到這是基督.
這是永活神的兒子.
是不一樣的.
但事實上當我們這樣理解和明白的時候.
我們發現一件很重要的事情是什麼.
原來耶穌親自來到.
該撒尼亞腓亞比.
在眾神明和在奧古斯督面前.
在宙斯面前.
在潘神面前.
在其他神明的面前.
祂要告訴人聽.
祂才是那個基督.
祂就是那個非人手.
而作出來的石頭.
能夠打敗一切的仇敵.
其實你聽到這裡.
你也會覺得.
我也知道耶穌是萬王之王萬主之主.
天上地下所有權柄都給了祂.
所以祂最厲害.
贏了命.
我們語謂是這樣的.
我們唱Hallelujah讚美主神.
很厲害.
但奇怪的地方是什麼.
再看多一張拍子.
看看有沒有.
你看看完了大耶穌第二章.
祂說完預言.
然後夢解了之後.
發生什麼事.
發生的是.
尼泊加尼撒王就面服於地.
向但義利下拜.
並吩咐人給但義利奉上貢物和香品.
於是王晉升但義利賜給他很多珍貴的禮物.
並委派他管轄巴比倫全省.
還納他管理巴比倫所有的治事的總監.
但義利書最後的四節.

$^{441}$記載了一件很美好的事情.
這些像大台劇BBQ的情景.
基本上是贏了.
然後作升但義利.
一家團圓很開心.
這是但義利書的結局.
但你猜不到的就是.
馬太十六章21至23節的結局.
和這個結局不一樣.
之前他都跟足了.
在眾神明面前.
有啟示和石頭的形象.
這些馬太和但義利書都平行的.
但到了馬太十六章21至23節結尾的時候.
剛才已經讀了.
多謝.
這麼快就開估了.
繼續按吧.
按對了.
結局不一樣.
但義利被作升了.
有榮耀有尊貴.
但你看這一段.
最奇怪的是.
魔鬼你退我後面去吧.
簡直突然間.
耶穌說了一句話.
其實是無厘頭的.
你說你贏.
你是非人所作.
出來石頭能夠打敗一切仇敵的時候.
你明白這是多大的榮耀.
今天你是一人之上萬人之下.
你明白嗎.
你期望我去上去.
有一個一人之下萬人之上的人.
已經是最強的.
他這麼強.
已經贏了.
這個非石頭所作出來的金銀銅鑾石都贏了的時候.

$^{481}$如果按馬利福音的論述.
陰間的權勢不能勝過你.
連陰間的權勢.
霸凌的權勢都給你了.
你不會突然間有一個很大的相反.
想像突然間說什麼.
耶穌說.
我三天後就會被殺.
受很多苦.
不過三天後復活.
這個是對於彼得來說.
最接受不到的東西.
你之前說你多厲害.
天花亂墜都好.
你說大義利書各樣的東西的時候.
在做偶像在神明面前.
你是唯一最厲害最厲害的那個.
偶像和任何的權勢都大.
突然間你說你這麼厲害的時候.
你為什麼突然間要死.
雖然彼得拉住他.
千萬不可.
這件事不會發生在你身上的耶穌.
你帶我離去.
走十幾二十個小時.
去凱撒利亞菲亞比的高山那裡.
看著那些神廟的時候.
我認你耶穌基督是永生神.
而只是基督這麼厲害.
你突然間出來跟我說你會死.
你明不明白.
你試一下林家謙演唱會.
你試一下林家謙突然跟你說.
所有的歌都不是我作的.
你明不明白.
其實那些歌不是我作的.
什麼一人之境那些.
我都是抄的.
其實有一個什麼.
其實有一個咪嘴的.

$^{521}$全部都不是我唱的.
你明不明白.
當別人捧到你很紅的時候.
你又演唱會去唱.
去容管的時候.
突然間紅到最後的時候跟你說.
一人之境不是我寫的.
所有的歌都是假的.
不好意思我要退出.
你明不明白.
所有人都會被一樣.
此外不可這件事會發生在身上.
你一定說不相信.
林家謙明明這麼帥又買墨茶廣告.
你怎麼會他不是這樣.
你接受不了.
所有的歌都不是他寫的.
什麼老朋友.
什麼一葉秋.
全部都是作出來的.
別人的歌詞.
別人的曲.
全部都不是我寫的.
甚至唱都不是我唱的.
這個這麼大的對比.
接受不了.
所以耶穌都很離譜的.
耶穌都不近人情.
我覺得彼得很正常.
這件事黏在你身上.
發生什麼問題.
很正常的反應.
我喜歡林家謙這麼久.
突然間.
萬萬演唱會我都去聽.
今晚我都去.
我怎麼能夠相信.
突然間他所有的東西都是假的.
耶穌說魔鬼撒旦退我後面去.
丁子妹.

$^{561}$和神約定.
或者耶穌和彼得的約定.
是什麼約定.
是一個帶你回到舊約的世界觀.
好像彈耳朵.
好像在尼泊爾王面前.
每個人都很差.
獅子坑.
全部都輸了.
但他們全贏了.
他帶我們去到.
整個第二章至第一章的故事裡面.
去看回那件事.
但突然間的對比.
回到過去.
其實我會死的.
大家以為耶穌根本就是那個石頭.
就是大理書所描述的那件事.
為什麼突然間他要死.
其實坦白說.
約定這件事.
我們守不到不是因為我們差.
其實是正常.
我們好像一樣.
正常地守不到.
因為耶穌的約定太怪了.
你帶我去那個地方天花亂墜.
你突然間要把我從高處掉下來.
我要接受你會死.
我應該看著你登基做王.
成就更大奇妙的事情.
突然間要接受耶穌基督要釘十字架.
要死還要被人受害.
發生什麼事.
原來在整個信仰裡面的約定.
起碼在基督教信仰裡面最基礎的約定.
都是荒謬的.
極荒謬的.
我們守不到約定很正常.
基本上這麼荒謬的事情沒有人會接受到.

$^{601}$你看彼得的生平.
就算他聽完這些經歷了這些都好.
去到耶穌真的要去到耶路撒冷要死的時候.
他只是跟著.
不承認.
你是不是跟耶穌一黨的.
雞蹄子前三次不承認.
就算你給他一個手指探他的肋旁.
刺了你之後.
要用福音第20章.
他願意去哪裡就去打魚.
這些約定對彼得來說.
守不到.
不知道發生什麼事.
很正常的.
就算我全都知道這些都好.
我知道三次福報都好.
我探了他的手指頭.
肋旁各樣的都好.
我最終的選擇都是去打魚.
是他無端端誣陷我.
你愛我比這隻蛇更深.
問完三次之後.
我才回答他.
最後你知道.
就完了.
做什麼都沒有人知道.
彼得都不知道發生什麼事.
當到第二章的時候.
聖靈降臨在那個人身上.
他突然間說了別的方言.
突然間看一看.
說完話.
三千人信住.
說完話.
五千人信住.
他那一刻才明白.
發生什麼事.
我們面對香港的轉變.
坦白說.

$^{641}$這六七八月.
這幾個月.
走的人很多.
在我這個年紀.
大白天.
我這個年紀.
走的人很多.
吃飯已經吃到一個地步.
已經很沒感覺.
我不覺得我要安慰他.
我覺得他們要安慰我.
你明不明.
我已經去到一個地步.
我還說你不要不捨得香港.
廢話.
你去到那邊.
那麼多人在那邊.
你明不明.
我為什麼要安慰你.
你知道嗎.
我一個人在這裡.
你應該安慰我.
但從來沒有.
走的人都不會安慰我.
你怎會走去那裡.
留在這裡的人不會.
根本上現在這個世界.
已經被丟了.
但在這麼混亂的裡邊的時候.
我們面對一些情況是很困難的.
最近有沒有陪審團的問題.
我們都會覺得很複雜.
我們都不知道要說什麼.
或者可以做的事情很少.
什麼是約定.
我們很想問的問題是.
耶穌在這個時代裡邊.
什麼是約定我們可以守住.
守住什麼.
我想跟進的.

$^{681}$其實在這麼混亂和不知道的裡邊.
我們沒有什麼可以約定的.
約定是很正常的.
我們不知道怎麼走.
我們不知道怎麼做.
看著前面的東西.
你覺得很混亂很複雜很困難的時候.
留下來的人.
可能下一年有更多人會走的時候.
你還在守什麼.
什麼是你很實際的內容.
要告訴自己我正在守住.
那個守住是什麼.
我想說我們要承認.
我們不知道和我們不認識.
Likewise 給的一樣的.
我們以為有個約定很厲害.
然後突然間.
美滿結局一定是BBQ大結局.
我們經常以為要這些.
但你問我彼得心無力情.
被耶穌罵他是什麼.
撒旦你退我後面去.
因為你不是關心神的事.
我今天留在香港.
還要這麼傻還在這裡.
我還要守在這個地方的時候.
我不是關心神的事.
我不留在這裡.
還說我不是關心神的事.
關心人的事.
最近有個青少年同工和我說了一番話.
令我反省了這幾個月.
你知道很多教會都沒有青少年.
Young Adult.
你知道很多人都很想為青少年做些事.
青少年同工聽完之後.
回去和青少年說.
外面有很多人都想為青少年做些事.
青少年同工問完之後.

$^{721}$下面的青少年說什麼你知道嗎.
拜託你們.
說了幾十年.
你不要再為我們青少年做任何事.
你試想一下這句說話.
你想清楚這句說話.
拜託你們不要再為青少年做任何事.
這句說話.
異曲同工是什麼.
你說香港打壓默.
知道前面做什麼.
承認我們不知道.
承認我們堅持的東西.
只不過是堅持在摸鬼撒旦的東西.
好過我以為我們自己真的知道.
上帝在這個世代在做什麼.
我做一個很無聊的故事完結.
我最近家裡有個家人.
一天我回家的時候.
他拿一本簿給我.
你拿簿給我做什麼.
他說我想做一個鑰匙扣.
想做一個明信片.
我說你為什麼做鑰匙扣和明信片.
他說因為Ivy So生日.
因為8月3日是Ivy So.
他說你做來做什麼.
我想算一下錢要給多少.
我想在我的IG買Ivy So的生日紀念品.
我第一個反應是.
我放完工回去.
看到他拿一本簿子.
說好價錢在哪裡買.
我就覺得很驚訝.
我是不是在做夢.
中國生會突然間會搞這麼多東西嗎.
他還要幫Ivy So慶祝生日.
然後就買鑰匙扣和明信片.
他要設計好.
然後再幫她剪輯.

$^{761}$他只問我拿一件東西.
你知道拿什麼嗎.
沒錯就是Payme帳號.
我很興奮.
我說阿女.
對不起.
又輸了.
我說你終於賺錢了.
我心裡想.
我說你算一下錢怎樣賺錢.
你買200份以上就可以賺錢了.
結果Ivy So生日就結束了.
你喜歡8月頭生日吧.
結局就不告訴大家了.
但我奇妙的是.
這件事我從來沒想過.
一個裝學生會自己去找印刷商.
去找啤膠鑰匙扣.
跟他談完就做出來.
我想說驚訝.
是我們這個年代需要的東西.
而這個驚訝的東西.
我們可以放下.
我們以為我們做的那些東西.
我們才能在信仰裡有多些約定.
我們不期望突然有很多人.
懂得為這個世代把脈.
基督教教要怎樣走.
但在基督教裡.
容許更多的人.
做更多特別的事情出現.
只要有空間.
你聽得懂那些青年說什麼嗎.
請你不要再為我們做任何事.
因為裝學生懂得自己做.
他覺得應該要做的事.
頂尖我們在一個不一樣的世代裡.
前面的路沒有水晶球.
看到任何東西.
但相信只要我們有足夠的空間和承載力.

$^{801}$很多的生命.
很多精彩的東西.
會令到這個世代.
雖然前面的路不知道怎樣走.
但每一個spark.
每一個小的火花和驚訝.
足夠令到我們.
看著前面不知道路要怎樣走的裡邊.
可以再走多一點.
耶穌從來沒有告訴我們發生什麼事.
坦白說就算告訴我們也只有這句.
我們今天這個世代當中.
你需要什麼.
需要一個土壤.
一個空間.
讓信仰在不同的空間和群體裡邊.
做了很多令人覺得驚訝的事.
有種像火花一樣.
燒了一下.
雖然很快過去.
但看到的人看著火花告訴自己.
原來上帝還可以在這個世代裡邊.
有不同的火花出現.
到了某一個moment.
好像五旬節聖年降臨的時候.
上帝將一條路指導給.
這一代世代的人.
應該怎樣走下去.
約定不是說一些很堅定的東西.
我們跟著走.
約定就是這樣.
說不到一些很堅定的東西怎樣走下去.
但上帝就告訴你.
可以走到那裡.
那天祝福留下的香港人.
當很多人已經離開的時候.
留下的香港人.
讓自己的生命.
離開很多框架.
多一份信仰的精彩和一樣.

$^{841}$在自己的生命裡邊能夠研發出來.
離開的香港人.
我們都期盼的是.
在世界各地不同的地方.
有一個在拉加科.
你知道住在拉加科是多漂亮的.
我去過就知道了.
拉加科是一個多漂亮的地方.
那個地方全部都是樂園.
你住在哪裡都是樂園.
剛才看到留言.
有人問候Folk Church.
他在拉加科.
拉加科即是甚麼.
尼加拉瓜大瀑布.
去到這麼漂亮的地方.
我希望你在那裡.
都找到上帝給你的啟發.
找到上帝在你生命裡邊離開了之後.
離開香港之後.
你可以在那裡看到更多不一樣的東西.
告訴香港人.
離開的人.
繼續都能夠承載上帝的恩典.
讓我們彼此同心.
看這個恩典多而又多.
雖然前面的路沒有人知道怎樣走.
但你和我可以彼此鼓勵.
走一條不知道的路.
我一起祈禱.
天父多謝你讓我們今天空間時間.
一齊來到你面前.
天父你知道我們人生很渺小.
日子很短小.
我們知道上帝你總是會有很多驚訝.
在我們心裡出現.
就好像彼得所經歷的一樣.
他以為是一個高峰.
是一個很精彩的人生.
從之後會榮華富貴跟著他.

$^{881}$但怎知他面對一個.
他要跟的人是一個死亡的人.
是一個復活的人.
是一個升天的人.
是一個不再存在的人.
怎樣走他人生每一步.
走下去都不知道怎樣走的時候.
你卻讓他可以走到.
雖然跌跌碰碰起跌跌.
但天父我求的是.
你藉著今天的餅和杯聖餐.
讓我們重溫這個故事和經歷的時候.
讓我們都好像靠著你的恩典一樣.
學習彼得這樣走下去.
求恩主的手.
你施恩在我們身上.
憐憫祝福無論在香港或世界各地的弟兄姊妹.
在面對前面的日子好像很艱難.
我不知道怎樣走下去的時候.
面對很多人的生命被克制被剝削.
落入一個困難裡邊的時候.
我們一起活出.
我們應該在上帝的面前活出那份火花.
讓生命繼續精彩.
讓生命繼續彼此點燃.
讓生命來到你面前的時候成為不一樣.
求主你憐憫我們幫助我們.
多天父你看我們面前有什麼禱告.
奉耶穌基督你寶貴的名字而求.
阿們.
謝謝大家.
\newpage



\section{腓利門書 1:1-25-20220827}
\label{sec:pF3HM3BllyQ}
\textbf{【網上崇拜】我地|腓利門書1\_1-25|20220827 [pF3HM3BllyQ]}
\newline
\newline
連結: \href{https://youtube.com/watch?v=pF3HM3BllyQ}{\texttt{ https://youtube.com/watch?v=pF3HM3BllyQ}} ~~~~ 語音日期: 2022-08-27 
\newline
\newline
\hyperref[sec:sQEhDyhKFnE]{\small{< < < PREV SERMON < < <}}
~
\hyperref[sec:index_chronic]{\small{[返順時目]}}
~
\hyperref[sec:index_scriptual]{\small{[返順卷目]}}
~
\hyperref[sec:1uGH28VFFVk]{\small{> > > NEXT SERMON > > >}}
\newline
\newline
腓利門書 1:1-25-20220827
\newline
\begin{longtable}{cl}
\hline
\hline
章節 & 經文 (和合本修訂版)\\
\hline
1:1 & \begin{tabularx}{0.7\textwidth}{X} 為基督耶穌被囚的保羅,同弟兄提摩太,寫信給我們所親愛的同工腓利門、 \end{tabularx} \\ \\ \relax
1:2 & \begin{tabularx}{0.7\textwidth}{X} 亞腓亞姊妹,和我們的戰友亞基布,以及在你家裡的教會。 \end{tabularx} \\ \\ \relax
1:3 & \begin{tabularx}{0.7\textwidth}{X} 願恩惠、平安從我們的父神和主耶穌基督歸給你們! \end{tabularx} \\ \\ \relax
1:4 & \begin{tabularx}{0.7\textwidth}{X} 我在禱告中記念你的時候,常為你感謝我的神, \end{tabularx} \\ \\ \relax
1:5 & \begin{tabularx}{0.7\textwidth}{X} 因聽說你對眾聖徒的愛心,和你對主耶穌的信心。 \end{tabularx} \\ \\ \relax
1:6 & \begin{tabularx}{0.7\textwidth}{X} 願你與人分享信心的時候,能產生功效,讓人知道我們所行的各樣善事都是為基督做的。 \end{tabularx} \\ \\ \relax
1:7 & \begin{tabularx}{0.7\textwidth}{X} 弟兄啊,由於你的愛心,我得到極大的快樂和安慰,因為眾聖徒的心從你得到舒暢。 \end{tabularx} \\ \\ \relax
1:8 & \begin{tabularx}{0.7\textwidth}{X} 雖然我靠著基督能放膽吩咐你做該做的事, \end{tabularx} \\ \\ \relax
1:9 & \begin{tabularx}{0.7\textwidth}{X} 可是像我這上了年紀的保羅,現在又是為基督耶穌被囚的,寧可憑著愛心求你, \end{tabularx} \\ \\ \relax
1:10 & \begin{tabularx}{0.7\textwidth}{X} 就是為我在捆鎖中所生的兒子阿尼西謀求你。 \end{tabularx} \\ \\ \relax
1:11 & \begin{tabularx}{0.7\textwidth}{X} 從前他與你沒有益處,但如今與你我都有益處。 \end{tabularx} \\ \\ \relax
1:12 & \begin{tabularx}{0.7\textwidth}{X} 我現在打發他回到你那裡去,他是我心肝。 \end{tabularx} \\ \\ \relax
1:13 & \begin{tabularx}{0.7\textwidth}{X} 我本來有意將他留下,在我為福音所受的捆鎖中替你伺候我。 \end{tabularx} \\ \\ \relax
1:14 & \begin{tabularx}{0.7\textwidth}{X} 但不知道你的意見,我不願意這樣做,好使你的善行不是出於勉強,而是出於自願。 \end{tabularx} \\ \\ \relax
1:15 & \begin{tabularx}{0.7\textwidth}{X} 他暫時離開你,也許是要讓你永遠得著他, \end{tabularx} \\ \\ \relax
1:16 & \begin{tabularx}{0.7\textwidth}{X} 不再是奴隸,而是高過奴隸,是親愛的弟兄;對我確實如此,何況對你呢!無論在肉身或在主裡更是如此。 \end{tabularx} \\ \\ \relax
1:17 & \begin{tabularx}{0.7\textwidth}{X} 所以,你若以我為同伴,就接納他,如同接納我一樣。 \end{tabularx} \\ \\ \relax
1:18 & \begin{tabularx}{0.7\textwidth}{X} 他若虧負你,或欠你甚麼,都算在我的賬上吧, \end{tabularx} \\ \\ \relax
1:19 & \begin{tabularx}{0.7\textwidth}{X} 我必償還。這是我—保羅親筆寫的。我並不用對你說,甚至你自己也虧欠我呢! \end{tabularx} \\ \\ \relax
1:20 & \begin{tabularx}{0.7\textwidth}{X} 弟兄啊,希望你使我在主裡因你得益處,讓我的心在基督裡得到舒暢。 \end{tabularx} \\ \\ \relax
1:21 & \begin{tabularx}{0.7\textwidth}{X} 我寫信給你,深信你必順服,知道你所要做的,必過於我所說的。 \end{tabularx} \\ \\ \relax
1:22 & \begin{tabularx}{0.7\textwidth}{X} 此外,還請給我預備住處,因為我盼望藉著你們的禱告,必蒙恩回到你們那裡去。 \end{tabularx} \\ \\ \relax
1:23 & \begin{tabularx}{0.7\textwidth}{X} 為基督耶穌與我一同坐監的以巴弗問候你。 \end{tabularx} \\ \\ \relax
1:24 & \begin{tabularx}{0.7\textwidth}{X} 我的同工馬可、亞里達古、底馬、路加也都問候你。 \end{tabularx} \\ \\ \relax
1:25 & \begin{tabularx}{0.7\textwidth}{X} 願主耶穌基督的恩與你們的靈同在。 \end{tabularx} \\ \\
[1ex]
\hline
\hline
\end{longtable}
$^{1}$好,大英姐妹平安.
今天很開心可以在這裡跟大家分享神的說話.
剛才的敬拜很感動.
不知道Matrix是否知道我的心想.
好像說了我的偏導出來.
所以現在一起祈禱.
希望上靈親自是我們每一個人的說話.
我們一起禱告.
因為是你的愛招敬我們在一起.
以致我們今天能夠在不同的地方.
我們都能夠一起去敬拜你.
求聖靈你現在在我們每一個人的心裡.
去教導我們明白你自己的說話.
也好像給我們你有足夠的愛.
以致我們能夠走在你的話語當中.
以致我們這一班大英姐妹.
能夠真的成為你自己的見證.
我們這樣祈禱奉耶穌的聖名祈求.
阿們.
2019年6月1日是歐聯總決賽.
是利物浦對熱刺的大日子.
當年我對足球仍然毫無興趣.
我丈夫是利物浦球迷.
當時還可以出夜街.
沒有人知道什麼是口罩.
仍然沒有限聚令的香港.
是會一大班不認識的人.
一起聚集,一起看球賽.
一起叫囂,一起打氣.
叫加油的香港.
當年我跟我的丈夫.
一起去了荃灣的熊貓酒店.
和一班利物浦的球迷.
即是利迷去看球賽.
整場球賽.
一班利迷多數是男人.
多數是男士.
剛陽味極重.
熱血沸騰.
陣臂高呼.

$^{41}$不停唱會歌.
請問這裡有沒有利迷.
Hello 我們今晚一定要贏.
不停唱會歌打氣.
我其實只是覺得很好笑.
我不停大聲笑.
很滑稽.
當時我仍然是我.
我還未跟利迷成為我們.
整場球賽我只是一個旁觀者.
見證著他們跟著球賽.
大呼小叫.
大約三分鐘或兩分鐘.
對方有個手球.
我們得到一個十二碼.
然後全班男人.
Sorry 太大聲.
就是這麼大聲.
其實我還未知道什麼事.
入球時很興奮.
差點被人入球時.
就大呼小叫.
還有贏了冠軍時.
那種激動.
那班男人.
差不多哭了.
有些還流了眼淚.
我坐在這裡.
被這班男人的激情.
和熱血吸引.
那一天.
我立志要成為利迷.
或者大部分.
頂姐妹.
大部分姊妹.
不太明白足球的熱情.
和激情的感染力.
一個月前.
給了我一張MIRROR的演唱會門票.
很羨慕.

$^{81}$因為我很疼愛我的組員.
謝謝你.
讓我可以看到.
第一場MIRROR的演唱會.
賣門票時.
我連網址也沒有.
抱著佛系的心態.
拿到一張門票.
我心存感恩.
我自問其實不是鏡粉.
在12個裡面.
其實我只喜歡Ian.
我最多叫自己.
Hello.
所以我整場都只顧著.
Ian何時出來.
Ian很帥.
然後.
一直這樣看.
當然我和其他鏡粉一起大叫.
很開心.
大約一秒說一次.
很帥.
類似這樣.
一起跳舞.
我聽到有些人說.
神經病.
的確不神經病.
的確在紅館裡.
我們這些鏡粉.
我們這些蝦佬.
雀屎什麼都好.
看到他們12個.
發出那種像姨母的慈愛笑容.
看到自己的偶像.
發出超過90分貝的聲浪.
我們一群鏡粉.
可以說是同呼同吸.
同叫同笑.
雖然只是Hello.

$^{121}$但我全程都非常投入.
非常開心.
第一場之後.
鏡粉看到很多脫勾和危險.
甚至聯署要求正視.
改善.
之後發生什麼事.
相信不需要我講.
一群鏡粉.
很緊張很期待.
自己入場一起大叫.
一下子變得很傷心很難過.
由同叫同笑.
變成同憂同哭.
群體的力量很大.
擁有難以想像的影響力.
今天是約定的最後一講.
講了很多我們和上帝的約定.
今天我很想和大家去想.
其實我們.
信了耶穌之後.
我們和身邊的弟兄姊妹.
我們.
這群信耶穌的群體.
和利迷和鏡粉.
或者其他群體有什麼分別.
我們夠不夠熱血.
夠不夠激情去感染人.
去加入我們.
我們的力量夠不夠大.
可以去發揮我們應有的影響力.
今天我會和大家看.
肥理門書.
肥理門書在我心中是一本很特別的書卷.
是保羅寫的.
他沒有講一些很難明的教義.
沒有講一些很難明的神學.
也沒有講一些基督徒生活.
做老公要怎樣怎樣.
做老婆要怎樣怎樣的勸勉.

$^{161}$完全沒有.
保羅寫肥理門書是很個人化.
因為他是在講他自己.
還有肥理門.
和肥理門的奴隸亞尼西姆.
三個男人的關係.
這封書可以說是保羅想用他們三個男人的關係.
去講我們.
一群信耶穌的人的關係.
我們看第一節.
第一節.
為基督耶穌被囚的保羅和兄弟提摩太.
這裡告訴我們作者是保羅.
而當時保羅.
是為了傳福音而坐牢.
然後寫信給我所親愛的同工.
肥理門和我妹阿菲亞.
還有和我當兵的阿基布.
以及在理家的教會.
第二節讓我們知道.
他寫給肥理門.
還有他妹阿菲亞和阿基布.
還有理家的教會.
我想說肥理門是他們家的主人.
他的家應該很大.
大到可以被教會聚會.
而肥理門家的教會.
很大可能是.
哥羅西教會.
因為肥理門書的結束.
和哥羅西書的結束.
那個問案提到的人名是一樣.
我們顯示是同一個群體.
所以相信是哥羅西教會.
哥羅西教會雖然不是保羅親自建立.
但都有可能是保羅第三次宣教旅程的時候.
有哥羅西的人聽到.
然後回去哥羅西建立教會.
所以保羅即使不是.
哥羅西教會的所謂創辦人.

$^{201}$建陀牧師.
但保羅對哥羅西教會來說.
都是一個德高望重的牧者.
一個福音愛邦人的使徒.
第十節保羅說.
他寫這封信是為我在困所中.
所生的兒子亞歷西姆.
或者亞歷西謀求你.
保羅說明這封信.
是為了這個隱私.
而寫的.
保羅說明這封信是為了這個隱私.
保羅形容亞歷西謀是他的兒子.
指的是保羅帶了他去信耶穌.
保羅為亞歷西謀求什麼呢?.
如果你有看過.
《肥利門書》.
或者你都會聽過一些說法.
傳統說法.
我們會基於第十八節.
就說他如果虧虧你什麼.
或者欠了你什麼.
就估計亞歷西謀這個奴隸.
應該偷了他主人肥利門的錢.
然後再基於第十五節.
第十五節就說也許.
他和你暫時分離.
是要讓你永遠得著他.
而假設亞歷西謀偷了錢.
就逃跑.
輾轉之下不知怎樣.
去了羅馬的監獄.
碰到保羅.
保羅在監獄裡帶了肥利門信耶穌.
保羅希望肥利門.
抱歉.
保羅帶了亞歷西謀信耶穌.
然後希望肥利門主人.
可以原諒亞歷西謀.
肥利門書.

$^{241}$不過這個解釋有些奇怪有些難通.
第一亞歷西謀是奴隸.
即是他偷了錢.
然後逃跑.
然後在哥羅西去到羅馬.
一大段的路程.
其實很難的因為路途很遙遠.
但最大的不可能是.
亞歷西謀是奴隸.
他不可能和保羅.
一個羅馬的公民.
在同一個監獄.
即使他們無理無故.
遇到羅馬也好.
也沒有理由在監獄裡.
碰到.
所以這個可能性比較低.
所以我較為認同近來有另一個說法.
就是肥利門其實是派.
他的奴隸亞歷西謀.
去監獄服侍保羅.
因為當時的監獄不像今天的監獄.
有懲教署的人看著你.
給你吃飯叫你吃橙沒有的.
羅馬的監獄是沒有飯吃的.
是要囚犯自己負責的.
所以囚犯是需要有人去照顧他.
菲勒比書曾經說過.
爾巴忽提這個人.
就是菲勒比教會.
派他去照顧當時坐監的保羅.
所以有可能肥利門也是這樣做.
哥羅西教會就派了亞歷西謀.
進到監獄去照顧保羅.
這個解釋.
可以解釋到為何奴隸可以去到那麼遠.
去到羅馬.
又解釋為何亞歷西謀會進到監獄.
見到保羅.
亦解釋到為何第十二節.

$^{281}$保羅會把亞歷西謀.
拆回到肥利門.
那裡.
沒可能的如果他真是一個逃犯.
他不可以到處走.
他是一個逃犯的話.
他會被監禁的.
所以逃犯是重罪.
被囚的亞歷西謀.
沒可能保羅讓他放就放.
即使他是一個德高望重的牧師.
以上這兩個說法其實都沒有辦法證實.
但如果亞歷西謀.
不是偷了肥利門的錢.
這封信的目的.
就不是求情.
就不是幫他求原諒.
求復和.
所以保羅寫這封信就有另外的目的.
第十二節.
我現在打發他.
親自回到你那裡.
他是我心上的人.
我本來想把他留下.
為我福音受菌所供.
替你侍候我.
但我不知道你怎樣想.
我就不願意這樣行.
這裡提到保羅叫亞歷西謀回去.
有可能表示這封信.
肥利門書就是由亞歷西謀.
親自帶回肥利門的家.
然後肥利門看完覺得.
我就送亞歷西謀回到保羅身邊.
然後第十六節.
更加說.
保羅期望亞歷西謀.
回到他的時候.
肥利門當他是甚麼.
不再是奴僕.

$^{321}$乃是高的奴僕,是親愛的兄弟.
第十七節.
他和他的同伴收納他.
收納即是接納.
不是收拾東西的收納.
保羅希望肥利門.
不要再將亞歷西謀.
看為奴隸.
而是像保羅一樣.
以親愛的兄弟.
去看亞歷西謀.
保羅寫這封信.
不是幫亞歷西謀求原諒.
而是想肥利門知道一件事.
就是亞歷西謀信了耶穌.
希望肥利門.
用親愛的兄弟的眼光.
去對待亞歷西謀.
可能你會覺得有甚麼特別.
但在當時來說就很特別.
保羅這個期望.
其實對整個高羅西教會.
對肥利門也好.
都是一件匪夷所思的事.
不知道應該怎樣做.
因為保羅這個說法.
是很顛覆當時羅馬社會.
人與人相處的社會階級制度.
當時的羅馬社會.
是由很多不同的.
家庭組成.
大家庭是分很多不同階級.
當時的家庭很大.
家庭成員除了老婆子女.
還會包括那些.
僱工工人奴隸.
這些毫無血緣的人.
他們都是家庭的一份子.
肥利門是這個家庭的一家之主.
是這個大家庭裡面.

$^{361}$最高權力的一個.
亞歷西謀是他的奴隸.
是這個大家庭裡面最低層的一個.
是完全沒有權力的那一層.
當時的奴隸只是一部生產機器.
是工具.
好像一顆螺絲.
只是資產.
不過他有生命.
奴隸不屬於任何人的兒子.
也不屬於任何人的父親.
即使他生了兒子.
他的兒子都是屬於他的主人.
主人與奴婆的關係就是這樣.
主人就是要保養這件工具.
免得它壞了.
以致大家庭的運作.
運作不到.
即使奴隸被主人信任.
提攜也好.
奴隸依然是附屬於主人.
高低階級分得非常清楚.
所以保羅在這裡竟然說.
亞歷西謀做了我屬靈的兒子.
還叫肥利門不要當他是奴隸.
相反.
要當他是親愛的弟兄.
其實很震驚.
好像今天突然間說.
你當這顆螺絲是人.
不止這樣.
你當這顆螺絲是像你一樣重要的人.
聽起來很荒謬.
一來這個世界不是這樣運作.
二來這個觀念太新.
對肥利門和整個教會來說.
根本不知道該怎麼辦.
保羅知道是很困難.
所以他寫這封信.
第三節.

$^{401}$願因平安從神我們的父.
和主耶穌基督歸於你們.
這句很普通的問候說話.
我們在保羅書上經常看到.
其實非常顛覆性.
也同時調教了我們的思維方向.
當時羅馬帝國.
剛才說過是大家庭.
主人是最高級.
那麼多個大家庭裡面.
誰是最高級的?.
就是羅馬皇帝.
羅馬皇帝又叫家父.
他擁有絕對的主權.
可以叫你死就死 叫你生就生.
主人或家父.
是整個家庭.
整個羅馬皇帝的恩主和救主.
他私人給你飯吃.
多謝多謝多謝.
他給你屋住.
多謝多謝多謝.
他不讓你活 你根本無法活.
當時的人就是這樣.
羅馬皇帝私人給你.
平民百姓要怎樣做?.
你就要擁戴他 要歌頌他.
你要唱好他 你要絕對聽從他.
因為如果沒有皇帝.
根本我們的偽民.
和偽都不如.
當時的觀念是很根深蒂固.
根本不覺得有問題.
但保羅一開始就提醒我們.
我們這班信耶穌的群體.
我們不是用社會上.
大家庭的社會制度.
和人相處.
我們是用神的家.
那種方式和人建立關係.

$^{441}$第三節這一句就告訴我們.
我們應該怎樣看.
這句問候語其實提到兩個銜頭.
第一個是神.
我們的父.
第二個就是主耶穌基督.
這兩個銜頭我們今天經常說.
但他就告訴我們.
身處的大家庭.
誰才是最有權力的.
保羅說師恩為平安.
不是從別人的父羅馬王而來.
而是從我們的.
父神而來.
第二個銜頭是主耶穌基督.
當時是說誰可以做主.
不是今天有工人姐姐.
那些就叫做主人.
很少大家庭 做主人的.
是特別崗位.
就像特首 不是每個人都做到.
所以主人好像今天真的長官.
首長 主席.
是特別的稱號.
但保羅說誰是首長.
不是肥利門.
是主耶穌基督.
在神的家這個架構.
最高權力的是我們的父.
和我們主耶穌基督.
習慣了在這麼多階層.
這麼多階級生活的肥利門.
和哥羅西教會.
其實還有我們.
可能我們都會好奇地問一句.
誰是主耶穌.
第二把交椅是誰.
他提起誰會上位.
我們會這樣想.
公司知道老闆快要退休.

$^{481}$我們就想誰會上位.
誰上位會影響公司的架構.
是不是.
假如或萬一.
我們老闆被人抓了.
走開走近不在 誰說了算.
我們會這樣想.
保羅也應該猜到我們會這樣問.
所以他用自己身份.
做一把尺.
告訴大家.
神的家的架構是怎樣.
保羅在肥利門書.
如何形容自己和自己的境況.
第一節.
他說他是被囚的.
為基督耶穌被囚的保羅.
第九節他又說為基督耶穌被囚.
第十節他又說為我在困所中.
第十三節再說多一次.
是為福音所受的困所中.
肥利門書只有25節.
說了四次.
他是怎樣形容自己.
他不是形容自己是外國人的使徒.
他不是形容自己是福音的使者.
他不是形容自己是你的顧問牧師.
他形容自己是囚犯.
囚犯代表什麼.
代表沒有自由.
代表被核制.
代表最卑微.
代表奴隸.
代表最低那一層.
我們的神是我們的家父.
我們的耶穌基督是主.
他是最高.
而保羅說自己和奴隸一樣.
是最低.
原來保羅不停說自己是被囚的.

$^{521}$甚至用這個身份來自誇.
是想讓我們知道.
在神的家的架構很簡單.
只有兩個階級.
最高是主耶穌.
最低是保羅.
我們是什麼等級.
我們不可能和神同等.
最低的已經是自居奴隸的保羅.
我們不會高過保羅.
但又不能低過保羅.
即是說.
原來我們竟然可以和外國人的使徒保羅同等.
同一個等級.
但同時我們也和奴隸亞歷西姆同等.
保羅在這裡呈現的一幅圖畫.
是一個更完善的制度.
是和當時羅馬社會的大家庭很不同.
是一個神的家.
神的大家庭的體制.
大家庭的家父是羅馬皇帝.
羅馬皇帝有很多不同階級.
但當我們相信了耶穌.
我們就進入了神的家的體制.
在神的家的體制下.
耶穌基督才是我們的父.
在父神之下.
我們每一個都是弟兄姊妹.
我們每一個的關係就是弟兄姊妹.
世界的階級身份.
不再成為我們和別人相處的框架.
在神的家.
我們社會上的身份即使沒有改.
但我們的關係也只有弟兄姊妹.
無分高低.
無分上下.
無分輕重.
誰也不比誰重要.
誰也不比誰高尚.
或者反過來說.

$^{561}$人人都重要.
人人都高尚.
可能在我們個人的原生家庭裡.
我們和自己的兄弟姊妹有不同的關係.
可能有些人很親密.
有些人很疏離.
可能有些人和自己的兄弟姊妹互相疏離.
有些人很親密.
可能有些人和自己的兄弟姊妹互相傷害.
有些人互相扶持.
或者我們在神的家裡.
弟兄姊妹之間的關係都是類似.
有些比較熟悉.
有些好像不認識.
有些互相支持.
有些竟然認知.
但保羅說亞彌西某不再是奴隸.
而是弟兄的時候.
我們除了知道我們是同一個層次.
我們更加需要知道.
我們要怎樣看待別人.
我們更加需要更新我們的觀念.
就是我們怎樣看待別人.
我們怎樣衡量別人的價值.
以前我們看亞彌西某只是工具.
只是螺絲.
但弟兄姊妹關係提醒我們.
亞彌西某是一個人.
是一個尊貴,有價值.
和獨特的人.
即使我們和其他弟兄姊妹的關係有不同.
但我們需要更新的是.
我們有沒有當弟兄姊妹是一個人.
是一個尊貴,有價值和獨特的人.
當我們看每個弟兄姊妹.
都是一個和自己一樣.
獨特又有價值的人.
我們才會記得我們一樣重要.
當人是人其實容易嗎.
被人當是人在香港常見嗎.

$^{601}$今天雖然是一個很追求平等的社會.
亦沒有以往羅馬的大家庭階級.
但我們可能會覺得我們只是公司裡的一粒螺絲.
社會的奴隸.
甚至我們在群體裡只是一個工具人.
可能只是一串數字.
人可以被當成一粒棋.
被人操控,玩弄.
去達成某些目的.
人可以被當成一堆數字.
要不要清理零.
或者看你的出席率有沒有80\%.
但從來看不到你的生命.
人可以被當成工具.
時工要人的時候我們會叫人見面.
但沒有時工的時候我們就甚麼都沒有.
原來當人是人不是我們想像中那麼容易,那麼自然.
我們很容易受這個世界的影響.
我們用了你做了多少.
懂多少,賺多少錢,有多少資產.
生活有多少之多彩,有多少人找自己工作.
有多少張專業證書,有多少學位.
只用你有得走和走不到.
或者不走去衡量自己和別人的價值.
所以當我們需要刻意記住.
我們身處在神的家的體制下.
我們要用神的家的體制去看待人.
人不是工具.
而是一個尊貴又價值.
是一個獨特被神所愛的人.
唯有這樣我們才會記得.
在弟兄姊妹這個層次裡.
無分高低,無分上下,無分輕重.
我的意見重要,你要聽.
但你的意見也重要,我也要聽.
我有東西想分享.
弟兄姊妹也有東西想分享.
我覺得這樣做才對.
但可能弟兄姊妹有其他想法,也對的.
你們願意人怎樣待你,你們也要怎樣待人.

$^{641}$神的家的體制提醒我們.
在弟兄姊妹是一個獨特的人之外.
神的家的體制其實同時挑戰當時的君主制度.
剛才我提及過羅馬帝國下有很多階層.
羅馬皇帝是最高.
然後每個大家庭都按著階級去運作.
要保障羅馬帝國的國家安全.
就是要維持這些大家庭穩定.
所謂大家庭好,羅馬帝國就好了.
羅馬帝國好的話,大家庭就好了.
但我們想想神的家的體制.
其實正正危害羅馬帝國的國家安全.
因為他顛覆了這個階級.
顛覆了這個大家庭.
在父神之下,奴隸不再是奴隸,而是弟兄姊妹.
當時哪有人會當奴隸為弟兄.
是不可能的.
但菲利門書保羅告訴我們.
原來信了耶穌在福音的能力下.
奴隸可以成為我們的弟兄.
我們這個群體的獨特性.
我們這個群體的影響力就在這裡.
原來我們的關係可以打破階級.
我們的關係可以改變常規.
我們的關係甚至可以顛覆世界.
我們的影響力其實就應該是這麼厲害.
原來我們有多影響到我們.
我們這個身處的地方.
在乎我們這群人.
有多做到神的家體制下的關係.
今天的講題叫做「我地」.
咦,沒有了海報.
不要緊.
今天的講題叫做「我地」.
謝謝.
設計部呢.
我們Folks Church很厲害,有設計部.
都是弟兄姊妹,很好.
設計部設計海報的時候.
很細心地問我.

$^{681}$你是「我地」還是「我地」.
如果你很執著於廣東話政治的話.
「我地」是有個口字的.
口字有個地.
如果你沒有口字的話.
就是地方的地.
我回答他,是「我地」,沒有口字的.
一來是想食字.
二來是想說.
我們的關係是可以影響到我們身處的地方.
我們的關係是可以很影響到我們身處的地方.
你身處哪裡?.
你身處哪裡的地?.
你掛住哪裡的地?.
你嚮往一片怎樣的地?.
我們大家未必還在獅子山下.
但這個地可以不分界限.
可以跨越空間.
因為有你有我的地方.
就有我地.
我們就可以建立出一片屬於「我地」的地.
我們的地是根據神的家的體制運作.
以至我們可以打破階級.
奴隸可以成為弟兄關係的地方.
根據神的家的體制運作.
我們的地可以改變常規.
人不再被當作工具.
而是被看為神所愛和獨特的人.
如果我們可以活得出這種多元又合一的關係.
我們甚至可以顛覆整片的土地.
甚至顛覆全個世界.
在今天不同的國家.
想擴展或肯定自己的土地.
自己地界的方法.
有些會靠打仗.
有些會講出來.
但我們這群信耶穌的群體.
我們要擴展屬於神的國.
神家的地界.
原來是靠我們這個群體.

$^{721}$活出真正的「我地」.
你想我們這個地方是一片怎樣的地.
有多大?有多遠?.
是荒涼還是豐富?.
原來我的地是怎樣.
不是靠打仗.
不是靠口.
是靠我們這群信耶穌的群體.
靠著福音的能力.
好好地活出來.
我們的地原來一直是靠我們.
我們一起祈禱.
我們的父神和主耶穌基督.
求你幫助我們.
當我們今天看《肥理門書》的時候.
你再一次提醒.
我們弟兄姊妹的關係是同等的.
我們是被你所愛的.
我們是一個獨特的人.
你更加告訴我們.
當我們可以這樣相處的時候.
我們這種關係可以顛覆世界.
可以打破階級.
可以突破常規.
求主你賜給我們從你而來的愛.
以致我們一群信了耶穌的群體.
我們可以改變我們的土地.
擴展我們的土地.
求神你幫助我們.
我們這樣祈禱.
奉耶穌的聖名祈求.
阿門.
\newpage



\section{傳道書 3:11-20220903}
\label{sec:1uGH28VFFVk}
\textbf{【網上崇拜】上帝的美學|傳道書3\_11|20220903 [1uGH28VFFVk]}
\newline
\newline
連結: \href{https://youtube.com/watch?v=1uGH28VFFVk}{\texttt{ https://youtube.com/watch?v=1uGH28VFFVk}} ~~~~ 語音日期: 2022-09-03 
\newline
\newline
\hyperref[sec:pF3HM3BllyQ]{\small{< < < PREV SERMON < < <}}
~
\hyperref[sec:index_chronic]{\small{[返順時目]}}
~
\hyperref[sec:index_scriptual]{\small{[返順卷目]}}
~
\hyperref[sec:oHgankweQ8E]{\small{> > > NEXT SERMON > > >}}
\newline
\newline
傳道書 3:11-20220903
\newline
\begin{longtable}{cl}
\hline
\hline
章節 & 經文 (和合本修訂版)\\
\hline
3:11 & \begin{tabularx}{0.7\textwidth}{X} 神造萬物,各按其時成為美好,又將永恆安放在世人心裡;然而神從始至終的作為,人不能測透。 \end{tabularx} \\ \\ \relax
3:12 & \begin{tabularx}{0.7\textwidth}{X} 我知道,人除了終身喜樂納福,沒有一件幸福的事。 \end{tabularx} \\ \\ \relax
3:13 & \begin{tabularx}{0.7\textwidth}{X} 並且人人吃喝,在他的一切勞碌中享福,這也是神的賞賜。 \end{tabularx} \\ \\ \relax
3:14 & \begin{tabularx}{0.7\textwidth}{X} 我知道神所做的都必存到永遠;無所增添,無所減少。神這樣做,是要人在他面前存敬畏的心。 \end{tabularx} \\ \\ \relax
3:15 & \begin{tabularx}{0.7\textwidth}{X} 現今的事以前就有了,將來的事也早已有了,並且神使已過的事重新再來。 \end{tabularx} \\ \\ \relax
3:16 & \begin{tabularx}{0.7\textwidth}{X} 我又見日光之下,應有公平之處有奸惡,應有公義之處也有奸惡。 \end{tabularx} \\ \\ \relax
3:17 & \begin{tabularx}{0.7\textwidth}{X} 我心裡說:「神必審判義人和惡人,因為在那裡,各樣事務,一切工作,都有定時。」 \end{tabularx} \\ \\ \relax
3:18 & \begin{tabularx}{0.7\textwidth}{X} 我心裡說:「為世人的緣故,神考驗他們,讓他們看見自己不過像走獸一樣。」 \end{tabularx} \\ \\ \relax
3:19 & \begin{tabularx}{0.7\textwidth}{X} 因為世人遭遇的,走獸也遭遇,所遭遇的都一樣:這個怎樣死,那個也怎樣死,他們都有一樣的氣息。人不能強於走獸,全是虛空; \end{tabularx} \\ \\ \relax
3:20 & \begin{tabularx}{0.7\textwidth}{X} 都歸一處,都是出於塵土,也都歸於塵土。 \end{tabularx} \\ \\ \relax
3:21 & \begin{tabularx}{0.7\textwidth}{X} 誰知道人的氣息是往上升,走獸的氣息是下入地呢? \end{tabularx} \\ \\ \relax
3:22 & \begin{tabularx}{0.7\textwidth}{X} 總而言之,人能夠在他經營的事上喜樂,是最好不過了,因為這是他應得的報償。他身後的事誰能領他回來看呢? \end{tabularx} \\ \\
[1ex]
\hline
\hline
\end{longtable}
$^{1}$Hello 大兄姐妹 晚安 平安.
很久不見了 由去年.
由去年沒有見面的時間 由八月初到現在九月.
四個星期 但回來的時候感覺上真的相隔了很久.
由平行宇宙回到宇宙裡面.
真的 回來的時候冷氣就很棒.
牆壁就拆掉了 桌子也有一個.
好像有一個新的形式一樣.
我自己還是不斷地在消化.
在英國那兩個星期回來香港之後.
很多的感覺 很多的經歷 一些的反省.
反省我們全聖教在香港這個地方裡面.
九月 十月和這兩個月的月題我們就叫做迎新.
本身其實是一個很簡單的意思 迎接新事.
如果你們記得的話 我們上年的月題是甚麼.
上年九月十月的月題是甚麼.
開學嘛 大家都很記得吧.
其實都想做一些類似的事情 迎接一些新的事情.
後來不如就用這個形式來形容.
我經常覺得我們全聖教或者全聖教很重要的精神.
就是我們 都是我經常開始的時候都跟人說.
我們是在捉著教會的essence.
然後執著這個essence來不斷地隨著不同的時間來改變我們的form.
所以這個我自己很大的信念在這幾年裡面.
就是在捉著我們的本質.
然後我們隨著世代的轉變而我們可以不斷地更新我們的形態.
所以今天我們 或者這個月我們會多說一些迎這個字.
form這個字.
可能有些同工會說新這個字.
今天我會想說迎這個字 form這個字.
而當我們去說form這個字的時候.
有一個很重要的字就是美.
我們會思考美這個課題.
還有今天說的有一點神學的.
待會你要留意一下.
偶爾你可能要回家再看一次.
所以今天我們一起選了一段經文.
就是傳道書.
傳道書第三章的經文.
讓我讀吧.

$^{41}$請聽聖經的說話.
傳道書第三章第一節.
凡事都有定期 天下萬物都有定時.
生有時 死有時 災中有時 不出所災中的也有時.
殺戮有時 醫治有時 拆毀有時 建造有時.
哭有時 笑有時 哀動有時 跳舞有時.
拋掟石頭有時 堆砌石頭有時 懷抱有時 不懷抱有時.
尋找有時 失落有時 保守有時 捨棄有時.
撕裂有時 縫補有時 靜默有時 言語有時.
喜愛有時 恆惡有時 真戰有時 和好有時.
第九節.
這人看來 做事的人在他的勞碌上有什麼益處呢?.
我見神叫世人勞苦 使他們在其中受經煉.
上帝造萬物 各按其事成為美麗.
又將永生按自在世人心裡.
然而上帝從始至終都作為人不能插頭.
我們一起祈禱.
天父我們將我們講道的時間交託給你.
我們僑星獻上我們的敬拜.
讓我們全世界不同各地的弟兄姊妹我們一起.
來聆聽你自己的說話.
你藉著你自己的聖言去教導我們.
怎樣去認識你自己的榮美.
讓我們知道美對我們的宿命生命.
對我們在這個時代的重要.
幫助孩子不配.
但你自己對我們說話.
透過網絡 透過你的靈在當中.
來拉緊我們.
讓我們的生活能夠緊貼你.
讓我們的生活能夠可以看見你.
求主你這樣幫助.
逢主命求 阿們.
傳道書的經文.
我想大家不陌生.
凡事都有定期.
生有時 死有時.
殺戮有時 已治有時.
我想整個傳道書.
大家最熟悉的都是這段經文.

$^{81}$可能這麼多章裡.
你能夠說得出來的.
就是這段有時有事的經文.
今天我們會嘗試從一個.
完完全全另一個角度.
來思考這段經文.
一個你平時沒怎麼想過的角度.
如果你細心去讀這段聖經的時候.
你會發現.
這些有時有事有時有事.
經文的背後 經文的下面.
其實緊接著一段很有趣的聖經.
好像只有我們來思考這段經文.
11 字 神造萬物 覺按其時成為美好.
又將永生安置於人心裡.
然而上帝從始至終的作為人不能插透.
以前我也覺得.
大概是一至八字裡面的.
有時有事的總結.
所以 某事有時有時.
凡事都有定期.
然後 神造萬物 覺按其時成為美好.
神造萬物 覺按其時成為美好.
所以我們以前覺得.
這段經文是一段很佛系的經文.
是一種很佛系的想法.
總之就是In His Time.
算到橋頭自然直.
善有善報 時辰未報 時辰未到.
分手的時候就是分手的時候.
復合的時候就是和開的時候.
去圖書館的時候就是靜默的時候.
Overall的時候就是言語有時.
跌了東西就是尋找有時.
搬屋就是寫有時.
總之以往的屬靈結論就是.
一切都是In His Time.
所謂都是上帝的時間.
大家很熟悉詩歌 In His Time.
In His Time, In His Time.

$^{121}$He makes all things beautiful.
In His Time.
這首詩歌其實正正就是來自這段經文.
神做萬物 各安其事 成為美好.
上帝有他的時間 一切都是In His Time.
只要是In His Time 一切事物都會好.
不過其實你開始發現有些bug的存在.
神做萬物 各安其事 成為美好.
He makes all things beautiful.
In His Time.
如果你是很認真去執著的話.
你把經文慢慢讀出來.
你會發現中英文版本其實是不配合的.
你會發現因為聖經所說的翻譯是有點不同的.
神做萬物 各安其事 成為美好.
和He makes all things beautiful.
其實是不同的翻譯.
因為美好和美麗其實是兩回事.
我們不能夠拉在一起去說.
我們看看希伯來文.
原文是Yepet.
Yepet不是top.
top就是美好.
Yepet就是美麗.
這就是希伯來文.
大家可以懂得的知識.
創世紀第一章.
神創造天地 神看一切為什麼.
為好的.
這個就是top的字.
就是美好.
但傳道書第三章所說的不是美好.
所說的不是好.
而是單純純全的美麗.
Beautiful.
如果你去搜尋希伯來文聖經的時候.
搜尋Yepet這個字.
十次有九次都是翻譯成美麗.
或者容貌盡美.
唯一一次就是傳道書這段.

$^{161}$所以其實應該跟大隊.
應該是美麗就比較對.
阿伯拉罕的妻子撒萊容貌盡美.
拉潔容貌盡美.
約瑟容貌盡美.
大鱷容貌盡美.
阿比蓋容貌盡美.
阿撒龍容貌盡美.
婦女美貌而無見識.
如同金環帶在珠臂上.
都是Yepet這個字.
都是漂亮beautiful的意思.
所以全部Yepet都是解作漂亮美麗的意思.
所以經文是這樣的.
神做萬物按著祂的時間.
使萬物變得美麗.
這個就是我們今天一起來思考的經文.
我們信仰裡面和美麗之間的關係.
要明白我們要先簡單說說有關美這回事.
所以就有問題了.
之後的五分鐘是有哲學味道的.
美是什麼呢.
美是一件有關form和appearance的事.
它是一個形的問題.
一個外在的事情.
所謂美不是事物的本質.
而是事物的形式和顯露.
所以美和好是不同的.
好是一些比較本質的東西.
美是一些比較外在的事情.
問大家一下.
如果讓你選擇的話.
你會選好還是美.
我問得比較具體一點.
郭帝卿.
將來娶一個老婆.
是一個好老婆還是一個漂亮老婆.
或者姐妹.
你將來嫁給一個好男人還是一個帥哥.
是不是.

$^{201}$我兩樣都要.
不算了.
不過想想.
純粹是技術上來說.
我家裡有經驗.
你當然要選一個老婆.
當然選一個好老婆.
我老婆真的很好.
我今天要對著全部.
七千多view的youtube.
說這句話.
我老婆真的很好.
因為她很擔心別人開電單車.
但她仍然讓我開電單車.
真的很好.
所以好老婆是一件相對比較永恆的事情.
一些本質比較long lasting的事情.
漂亮老婆是有時限的.
早晚又不同.
十年前和十年後又會不同.
所以美麗是短暫的.
美麗是一剎那之間的顯露.
大家可以想想十年後.
潘Sir和我五十多歲的老人家.
脫髮 地中海一定有.
再過十年.
勁拜和Alex都是一樣.
都是會脫髮 地中海M字頭.
所以就不漂亮.
就不是那麼帥.
所以就發現聖經對於外貌.
其實這些外在的恥乎.
是可以面倒地去反對.
人是看外貌.
耶和華是看內心.
神不以外貌取人.
我們從今以後不憑著外貌認人了.
這都是很多聖經所說的經文.
不過我們雖然不憑外貌去認人.
但不代表我們忽視了美麗這個課題.

$^{241}$美麗是教會一直忽略的一個課題.
除了Full Choice之外.
我都很著重美.
特別是基督教新教.
基督教新教從Martin Luther開始.
就開始將美學和信仰分開.
即是路德是反對天主教的形式和外貌.
來表露上帝本體的一個過程.
他們不覺得用一些外在的東西.
一些形式的東西來表達上帝的本體.
講一些技術性的東西.
1518年路德在海德堡反對榮耀神學.
強調上帝和人之間的認知性的可能.
其實是講其中一種美學關係.
我們不能直接去認識上帝的外貌.
用一些很形式的東西去表露上帝.
所以同時路德一方面是反對榮耀神學.
同時拆毀了整個神學美學的可能.
但其實上帝是一切美麗的本源.
如果你去看聖經去搜尋「耶丕」這個字的時候.
詩篇4.1篇第二節.
耶和華上帝比世人更美.
這個「耶丕」都是在形容我們的上帝.
一切的美都是從上帝而來.
所謂美其實就是上帝本體的展現.
剎那之間的外露.
美不是上帝本身.
而是上帝的彰顯.
所以上帝在世界的彰顯.
是無可避免用一種形式去展現出來.
上帝某程度上是用一種形式去展現出來.
所以這種展現正正就是上帝的美.
以前我們是用榮耀這個字去形容.
上帝的榮耀Doxa就是上帝在世上的展露.
稱之為榮光.
但榮光的意思是什麼呢?.
就是榮美.
所以榮美的美正正就是一樣的東西.
所以上帝的榮耀和上帝的美是相似的事情.
好了,這些比較深入的東西就結束了.

$^{281}$不過,就像我剛才所說的.
美和好是不同的.
這個世界是好的.
上帝也是好的.
但這個世界的美麗.
上帝流出來的美.
就好像一個煙花一樣.
是短暫的.
是脆弱的.
是瞬間就消失.
當邪惡一來臨.
這些美麗的事情就會立即消失.
以前我講到也有用過一首歌.
Malitue Airport的《美麗新香港》.
這首歌詩歌.
我發現很有趣.
這首歌詩歌雖然叫《美麗新香港》.
但歌詞裡面一句美麗都沒有.
全部都是在說現在已經逝去的東西.
為什麼?.
因為歌詞所說的美麗已經逝去了.
已經不在了.
講了這麼久.
究竟美跟這段經文有什麼關係呢?.
回到經文最基本的問題.
為什麼傳道書第三章會講美這個課題?.
生有時,死有時,災中有時,不出災中也有時.
實錄有時,醫治有時,撐位有時,建造有時.
哭有時,笑有時,哀動有時,跳舞有時.
最後結論是.
上帝做萬物,按著他的時間使萬物變得美麗.
究竟是什麼意思呢?.
傳道書第三章所描述的.
其實從來都不是100\%正向樂觀的說法.
不是傳道書所說的.
當你讀經文的時候.
我們都可以用以前的總結.
in his time,一切都是他的時間.
所以我們經常聽到頂智梅的見證.
所謂的in his time的見證.

$^{321}$生活突然出現某個困難.
在徬徨無助之際.
不知如何是好的時候.
突然之間不知道是誰搞頂智梅的親戚朋友.
剛好巧合.
無端端無端無情情在那裡工作.
突然間平日放假,今天又是調教.
突然間出現幫到你.
剛好可以聽full charge.
後來又回想,十年前,我還在立志.
這些見證都是.
神真的奇妙了.
很多in his time的見證.
剛剛好有些奇妙的時間出現.
你間中都會聽到in his time的見證.
當然是對的,這真是上帝的時間.
不過坦白說.
在上帝的時間的同時.
大部分的時間是什麼意思.
就是不是in his time.
其他這些奇妙的in his time之後.
其實都不是上帝的時間.
時不對,仍要等待.
尚未來臨,不是時候會有發生.
請知傳道書其實不單單要表達一種佛系的訊息.
不是純粹正面樂觀的,剛好巧合的東西.
而是很公道地說.
其實這個世界的現狀.
一點都不好.
從來都不是一些樂觀L的傳道書.
不是一面倒地說出一些很樂觀正面的訊息.
生有時,死有時,萬物都有定時.
其實都不是說樂觀的東西.
甚至乎你發覺第九,十三節經文裡.
你會發現.
經文裡所說的日光之下的人生其實不是太好.
看看有沒有第九節這樣說.
「這樣看來,做事的人在他的勞碌上有什麼益處呢?.
我見神叫世人勞苦,使他們在其中受經煉.
神將永生安置在世人心裡.

$^{361}$然而上帝從此之中的作為人不能參透.
我知道世人莫強於終生喜樂行善.
並且人人吃喝在他的勞苦中享福.
這神的恩賜」.
傳道書作者不是一個樂觀L.
當然也不是一個完全絕望L.
而是很公道地點出.
在日光之下的世界真相.
大概有八成?八成半?.
在日光之下所遭遇的大概有八成的事情.
都不是太好的.
大概有兩成至一成半是有好事發生的.
最少法輪功經文是這樣說的.
你發覺在日光之下.
世界是一個沒有利益,沒有証據,沒有好事發生的世界.
為什麼這樣說?.
第一,沒有利益.
第二,這看來做事的人在太陽爐上有什麼益處呢?.
什麼利益在那裡?.
這是一句從經濟利益的角度出發的觀點.
作者說,在日光之下,你沒有什麼利益,你沒有賺錢,沒有贏.
這也是我們這幾年來的經驗.
這個世界是有人贏的,但很少有人贏.
大部分人都是沒有贏,或者賺一點點就輸光了.
這是第一個,沒有利益.
第二個是沒有証據,實不至.
又將永生安置在世人心裡.
然而神從始至終都作為人,不能參透.
傳說說明明知道上帝有永恆的概念去看世界.
明明知道上帝理論上會出手.
明明知道這是天父世界.
天父是擁抱.
不過對於這個世界,對於上帝作為人,你是沒有証據的,不是很悉透的.
你不能夠參透,你捉到也不會脫國.
又說沒有証據.
就是一個不太好的狀態.
第三個是不太好.
十二節,他說,我知道世人莫強於終生喜樂行善.
這句話中文是難明白的,英文會比較明白.
I know that there is no good in them, but for a man to rejoice and to do good in his life.

$^{401}$很有趣的語法,除了這些之外.
然後知道什麼?這個世界好嗎?這個世界是不是好?.
There is no good 除了那些微不足道,小恩小惠,小事之外.
就是那些B-grade的good.
偶爾你可以在IG上發文,笑笑.
偶爾有好事發生,但這些都是B-grade的東西.
你要有超級好事發生,sorry,沒有.
所以等於十一節傳道書第三章所說.
這個世界其實不是太美好的世界.
沒什麼益處,你不能夠擦透不是太美好的世界.
No profit, no clue, not very good.
所以你說神創造萬物,各按其時成為美好.
Not really,不是太美好的.
這個就是傳道書很公道的評價.
不過,傳道書很肯定說,它仍然可以有美麗.
雖然如此,傳道書仍然很公道地點出.
神造萬物,按著它的時間.
世界上的事情,偶然會有美麗的東西出現.
雖然沒什麼益處,沒得稱讚,但仍然可以美麗.
雖然一桶霧水,前路茫茫,你不知道在做什麼.
但仍然可以見到一些美麗的東西.
雖然只是B-grade級數的好事,但仍然可以是美.
這份美,正正就是上帝在世界裡的一些詩詩的痕跡.
我們要去思考這些詩詩,很細小的美.
是怎樣的.
我們活在一個醜陋的世界.
人性醜惡,人心醜陋,手段安裝.
美麗新香港,令我們非常陌生.
昔日我們發覺很美的香港,似乎已經離開了我們.
因為這正正就是傳道書所說的世界.
美麗不是永遠的,醜惡一旦來臨.
美麗的事一旦被醜惡摧毀,就會瞬間即逝.
不過縱然美麗真的好像煙火一樣.
美麗卻短暫而脆弱,美麗隨時間改變而消失.
但美麗仍然可以隨著時間,又會瞬間的閃一閃出現.
這幾年,我們在2019年到現在.
其中一件我覺得是好事,就是這件事.
偶爾你會發覺一些天空洗板的Facebook,IG帖文.
你會發現是突然之間毫無先兆的.
當你下班的時候,抬頭望天.

$^{441}$突然間天色變得美到你不相信.
尤其是我住長洲,望著海,哇,美到不得了.
然後在一小時後的IG和Facebook上.
你會發覺你朋友在土瓜灣,觀塘,尖沙咀,沙田.
不斷在那裡post自己拍下來的天空照片.
造成一個網絡上的小型洗板.
我覺得這件事是很有意思的事情.
你問,粉紅色的天空和香港前途有什麼影響呢?.
粉紅色的天空能夠持續多久呢?.
粉紅色的天空能夠改變世界什麼呢?.
是的,突然間出現一片這麼美麗的天空.
其實都是一些小事的小事.
新聞不會報導的,之後就會消失.
但一個多小時的振奮.
幾個小時的網絡洗板.
一剎那之間的美麗.
正正都是從上帝而來的美麗的事情.
值得我們去remind.
上帝在世界裡面存在的痕跡.
因為美麗正正是上帝一個很重要顯露的痕跡.
不知道大家有沒有看過這幅畫.
應該看過這幅畫吧.
是梵高的,梵高是一個很有名的傑作.
叫做《星夜》.
現在在紐約博物館展覽中的一幅畫.
不知道大家知不知道這幅畫是何時畫的.
梵高在1889年.
差不多在他晚年的時候.
情緒崩潰.
人生最悲慘的時候.
在精神病院的窗口.
從精神病院的窗口看著外面的夜空.
去畫出來的一幅畫.
幾個月前.
他才剛砍了自己的耳朵.
原來梵高是一個頗為敬虔的人.
梵高的父親是牧師.
梵高對於信仰是很有那種passion.
他27歲才開始學畫.
他24歲的時候是讀神學的.

$^{481}$不過神學院就不收他.
嫌他太過骯髒.
口齒不靈利.
怕會影響信主.
這樣說吧.
怕他會令人不信耶穌.
所以就不找他做傳道的工作.
這幅畫是他生命中最悲慘的時候.
畫出來的一幅畫.
我想說什麼呢.
與其說我們很膚淺地說.
他畫得很漂亮.
其實不是這樣說.
梵高能夠看到世界很美.
一個藝術家不是畫得很美.
而是他首先看到這個世界很美.
他將他看到的美麗畫出來.
梵高這樣說.
一個勞動者的形象.
一個耕在田上的泥溝.
一片沙灘.
廣闊的海洋和天空.
都是美的.
終身從事這個表現隱藏在當中的美麗.
確實是值得的.
他被困在精神病院.
看著窗口的天空的時候.
他看到的正正是上帝的美.
天父世界的美.
我們都是一樣.
嘗試去學習去捕捉這個世界的美麗.
這些美麗的事情.
正正是上帝在世界裡走過的痕跡.
環境極差.
但是美麗的事物依然存在.
因為它們是上帝流動過的痕跡.
所謂美麗就是你去發現上帝蹤影那一刻.
哇,好美啊.
世界上每一片美麗都是從上帝而來的.
上帝不單單創造這個世界.

$^{521}$更加是創造一個美麗的世界.
一切的美麗都是上帝痕跡的顯露.
如果你覺得要相信這天父世界太沉重.
太難以相信.
如果你覺得生命很可憐.
嘗試去找一些美麗的東西.
不一定是大自然.
任何美麗的事情都是上帝顯露的瞬間.
我們去聆聽另一段傳道書的說話.
經文這裡沒有,我讀出來.
另外有一段是記載了美麗的經文.
傳道書第五章十七節.
我所看見的美就是人在神賜他一生的日子吃喝.
享受日光之下勞碌得來的好處.
因為這是他的份.
總之告訴我們原來那個美.
正正都是一些生活基本的事情.
我們吃喝生活.
享受我們勞碌得來的一點點好處.
be great的好處.
這個就是美麗.
默默地將永恆繼續存在你的心裡面.
時勢醜惡人心醜陋.
但依然堅持在世上去捕捉美麗的那一剎那.
前幾天有一對日本宣教士去長洲探望我.
他們在日本領養了一個有唐氏綜合症的嬰兒.
一個日本嬰兒.
帶著嬰兒來到長洲.
嬰兒兩歲多,很可愛.
他最喜歡吃飯.
他不可以加任何東西,只能加白飯.
因為日本不是烏冬就是白飯,只能白色.
所以他只喜歡吃白色的東西.
加醬油,加什麼他都不吃.
但他一看到吃飯,那個開心度很可愛.
很可愛.
他就和我女兒一起在長洲海灘玩.
就是一個這麼美麗的嬰兒.
和我很美麗的女兒.
嘗試在我們的生命裡面.

$^{561}$去記得去留意這些美麗的東西.
無論是嬰兒還是姜濤嬰兒.
一樣的.
Mirror 定 Color.
明年一月有Blackpink演唱會.
一隻貓.
一杯很美的咖啡.
7-11要換的小忌廉的傘.
一片漂亮的天空.
在生活裡面我們學習去捕捉這些美麗.
去追求這些美麗.
去創造這些美麗.
在生活裡面藉著這些美麗的事情.
我們學習去讚美上帝.
我們讚美上帝.
正正在我們生活裡面去讚美上帝.
讓我們一起活下去.
我們祈禱.
祖尼求你讓我們繼續去學武美麗.
這些美麗雖然是一些很短暫.
很小規模的東西.
但我們在當中被提醒.
我們知道你仍然在世界裡面.
雖然時勢不好.
雖然環境不好.
但我們仍然可以去學習去捕捉這些美麗.
讓我們Full Church 弟妹妹懂得欣賞美麗.
懂得重視美麗.
因為這些都是從你而來.
這些源頭都是你.
我們能夠從這些美麗裡面找到你.
求主你這樣去教導我們.
讓我們一起成為一個美麗群體.
同尊名求.
\newpage



\section{希伯來書 11:8-22-20220910}
\label{sec:oHgankweQ8E}
\textbf{【網上崇拜】月是故鄉明|希伯來書11\_8-22|20220910 [oHgankweQ8E]}
\newline
\newline
連結: \href{https://youtube.com/watch?v=oHgankweQ8E}{\texttt{ https://youtube.com/watch?v=oHgankweQ8E}} ~~~~ 語音日期: 2022-09-10 
\newline
\newline
\hyperref[sec:1uGH28VFFVk]{\small{< < < PREV SERMON < < <}}
~
\hyperref[sec:index_chronic]{\small{[返順時目]}}
~
\hyperref[sec:index_scriptual]{\small{[返順卷目]}}
~
\hyperref[sec:Z8wgkxuhjIk]{\small{> > > NEXT SERMON > > >}}
\newline
\newline
希伯來書 11:8-22-20220910
\newline
\begin{longtable}{cl}
\hline
\hline
章節 & 經文 (和合本修訂版)\\
\hline
11:8 & \begin{tabularx}{0.7\textwidth}{X} 因著信,亞伯拉罕蒙召的時候就遵命出去,往將來要承受為基業的地方去;他出去的時候還不知往哪裡去。 \end{tabularx} \\ \\ \relax
11:9 & \begin{tabularx}{0.7\textwidth}{X} 因著信,他就在所應許之地作客,好像在異鄉,居住在帳棚裡,與蒙同一個應許的以撒和雅各一樣。 \end{tabularx} \\ \\ \relax
11:10 & \begin{tabularx}{0.7\textwidth}{X} 因為他等候著那座有根基的城,就是神所設計和建造的。 \end{tabularx} \\ \\ \relax
11:11 & \begin{tabularx}{0.7\textwidth}{X} 因著信,撒拉自己已過了生育的年齡還能懷孕,因為她認為應許她的那位是可信的; \end{tabularx} \\ \\ \relax
11:12 & \begin{tabularx}{0.7\textwidth}{X} 所以,從一個彷彿已死的人竟生出子孫,如同天上的星那樣眾多,海邊的沙那樣無數。 \end{tabularx} \\ \\ \relax
11:13 & \begin{tabularx}{0.7\textwidth}{X} 這些人都是存著信心死的,並沒有得著所應許的,卻從遠處觀望,且歡喜迎接。他們承認自己在地上是客旅,是寄居的。 \end{tabularx} \\ \\ \relax
11:14 & \begin{tabularx}{0.7\textwidth}{X} 說這樣話的人是表明自己要尋找一個家鄉。 \end{tabularx} \\ \\ \relax
11:15 & \begin{tabularx}{0.7\textwidth}{X} 他們若想念所離開的家鄉,還有回去的機會。 \end{tabularx} \\ \\ \relax
11:16 & \begin{tabularx}{0.7\textwidth}{X} 其實他們所羨慕的是一個更美的,就是在天上的家鄉。所以,神並不因他們稱他為神而覺得羞恥,因為他已經為他們預備了一座城。 \end{tabularx} \\ \\ \relax
11:17 & \begin{tabularx}{0.7\textwidth}{X} 因著信,亞伯拉罕被考驗的時候把以撒獻上,這就是那領受了應許的人甘心把自己獨生的兒子獻上。 \end{tabularx} \\ \\ \relax
11:18 & \begin{tabularx}{0.7\textwidth}{X} 論到這兒子,神曾說:「從以撒生的才要稱為你的後裔。」 \end{tabularx} \\ \\ \relax
11:19 & \begin{tabularx}{0.7\textwidth}{X} 他認為神甚至能使人從死人中復活,意味著他得回了他的兒子。 \end{tabularx} \\ \\ \relax
11:20 & \begin{tabularx}{0.7\textwidth}{X} 因著信,以撒指著將來的事給雅各、以掃祝福。 \end{tabularx} \\ \\ \relax
11:21 & \begin{tabularx}{0.7\textwidth}{X} 因著信,雅各臨死的時候給約瑟的兩個兒子個別祝福,扶著枴杖敬拜神。 \end{tabularx} \\ \\ \relax
11:22 & \begin{tabularx}{0.7\textwidth}{X} 因著信,約瑟臨終的時候提到以色列人將來要出埃及,並為自己的骸骨留下遺言。 \end{tabularx} \\ \\ \relax
11:23 & \begin{tabularx}{0.7\textwidth}{X} 因著信,摩西生下來,他的父母見他是個俊美的孩子,把他藏了三個月,並不怕王的命令。 \end{tabularx} \\ \\ \relax
11:24 & \begin{tabularx}{0.7\textwidth}{X} 因著信,摩西長大了不肯稱為法老女兒之子。 \end{tabularx} \\ \\ \relax
11:25 & \begin{tabularx}{0.7\textwidth}{X} 他寧可和神的百姓一同受苦,也不願在罪中享受片刻的歡樂。 \end{tabularx} \\ \\ \relax
11:26 & \begin{tabularx}{0.7\textwidth}{X} 他把為彌賽亞受凌辱看得比埃及的財物更寶貴,因為他想望所要得的賞賜。 \end{tabularx} \\ \\ \relax
11:27 & \begin{tabularx}{0.7\textwidth}{X} 因著信,他離開埃及,不怕王的憤怒,因為他恆心忍耐,如同看見那不能看見的神。 \end{tabularx} \\ \\ \relax
11:28 & \begin{tabularx}{0.7\textwidth}{X} 因著信,他設立逾越節,在門上灑血,免得那毀滅者加害以色列人的長子。 \end{tabularx} \\ \\ \relax
11:29 & \begin{tabularx}{0.7\textwidth}{X} 因著信,他們過紅海如行乾地;埃及人試著要過去就被淹沒了。 \end{tabularx} \\ \\ \relax
11:30 & \begin{tabularx}{0.7\textwidth}{X} 因著信,以色列人圍繞耶利哥城七日,城牆就倒塌了。 \end{tabularx} \\ \\ \relax
11:31 & \begin{tabularx}{0.7\textwidth}{X} 因著信,妓女喇合曾友善地接待探子,就沒有跟那些不順從的人一同滅亡。 \end{tabularx} \\ \\ \relax
11:32 & \begin{tabularx}{0.7\textwidth}{X} 我還要說甚麼呢?若要一一細說基甸、巴拉、參孫、耶弗他、大衛、撒母耳和眾先知的事,時間就不夠了。 \end{tabularx} \\ \\ \relax
11:33 & \begin{tabularx}{0.7\textwidth}{X} 他們藉著信,制伏了敵國,行了公義,得了應許,堵住了獅子的口, \end{tabularx} \\ \\ \relax
11:34 & \begin{tabularx}{0.7\textwidth}{X} 滅了烈火的威力,在鋒利的刀劍下逃生,從軟弱變為剛強,爭戰中顯出勇猛,打退外邦的全軍。 \end{tabularx} \\ \\ \relax
11:35 & \begin{tabularx}{0.7\textwidth}{X} 有些婦人得回從死人中復活的親人。又有人忍受嚴刑,拒絕被釋放,為要得著更美好的復活。 \end{tabularx} \\ \\ \relax
11:36 & \begin{tabularx}{0.7\textwidth}{X} 又有人忍受戲弄、鞭打、捆鎖、監禁、各等的磨煉; \end{tabularx} \\ \\ \relax
11:37 & \begin{tabularx}{0.7\textwidth}{X} 他們被石頭打死,被鋸鋸死,被刀殺,披著綿羊山羊的皮各處奔跑,受貧窮、患難、虐待。 \end{tabularx} \\ \\ \relax
11:38 & \begin{tabularx}{0.7\textwidth}{X} 這世界配不上他們,他們在曠野、山嶺、山洞、地穴,飄流無定。 \end{tabularx} \\ \\ \relax
11:39 & \begin{tabularx}{0.7\textwidth}{X} 這些人都是因信獲得了讚許,卻仍未得著所應許的, \end{tabularx} \\ \\ \relax
11:40 & \begin{tabularx}{0.7\textwidth}{X} 因為神給我們預備了更美好的事,若沒有我們,他們就不能達到完全。 \end{tabularx} \\ \\
[1ex]
\hline
\hline
\end{longtable}
$^{1}$弟兄姊妹平安.
剛才在網上看到有弟兄姊妹在中秋文安報平安.
我自己也很喜歡過中秋節和聖誕節這兩個節日.
我也很喜歡慶祝.
我很喜歡一群人圍著做同一件事.
一起吃月餅也好,燒燈籠也好.
我小時候很頑皮,住的地方很空曠.
網上的弟兄姊妹是不對的.
不應該鼓勵這件事,是禁止的.
中秋節是崇拜,團圓並不必然.
也不容易就得.
在中秋節晚上和弟兄姊妹崇拜是一件美好的事.
特別在現在這麼患難的日子.
每次唱《如雨》這首歌也有很多聯想.
很多影像出來.
縱使風浪在前或在中的時候.
《如雨》這首歌再次提醒我們.
作著為王,上帝看著的事.
縱使我們現在還沒看到出路.
但路上仍然是上帝帶領著.
作了一本很美的圖畫.
這本圖畫是《月事故鄉名》.
我本星期所說的題目.
對我來說,這圖畫的色溫很中二.
真的好像在看月.
我到現在還沒見過今晚的月光.
稍後會出去.
我還沒做節日.
明天和後天連續兩天都會做節日.
今天和弟兄姊妹一起做節日.
《月事故鄉名》這句說話.
每週都會說為什麼會選擇這個講題.
今年特別是認識很多人.
都離開了香港.
或因不同的事不能夠相聚.
特別是受隔離令.
或其他原因.
不能夠有很多思念和不同的感受.
為什麼在這個月的月題.
關於迎新的月題中說這個訊息.

$^{41}$因為今天是中秋節.
在當中做一個串連.
我也想說一個中秋節的訊息.
當我們想起月亮.
會想起上帝的創造.
當我們想起月亮的時候.
會想起上帝在當中的管理.
上星期John和大家說了.
欣賞周遭.
欣賞上帝的創造.
欣賞周遭的微小事物.
哪怕出現小事.
但上帝的創造仍然是.
讓我們看到祂在其中.
我選擇的經文是.
在希伯來書第十一章.
大家說關於信心的經文.
大家可能都很熟悉的金句.
我不會說整個十一章.
我會集中說幾節.
但整個十一章關於信的描述.
當中是很具體.
因為他用了很多不同人的串連.
點題和大家重溫一下.
關於希伯來的訊息.
希伯來書十一章的訊息.
第一節大家已經很熟悉.
信則是所望之事之實體.
是未見之時的確具.
古人在這信上得了美好的證據.
一開始第一節.
很多時候都有很多不同的困惑.
什麼叫做所望之事的實體呢?.
什麼叫做未見之時的確具呢?.
在過程當中.
可能有很多不同的沉迷.
但作者其實在後面就告訴我們.
其實實體就是很多人會經歷.
要行出信仰最底的那一層要什麼.
而未見之時的確具就是.

$^{81}$在下面的經文描述的人.
從來都未見到事件完成.
但他們看到將來要發生的事.
他們就拿著那件事.
就是上帝會包底.
上帝會做.
這就是他們可以憑據的地方.
所以他說.
古人在這信上得了美好的證據.
第三節.
我們因著信就知道.
諸世界是藉上帝的話造成的.
這樣就看見並不是從顯然之物造出來的.
作者要表達一個很素景的訊息.
就是我們要觀看旁邊發生的事.
觀看上帝所做的萬物.
因為如果你只限看自己能夠承擔.
或者自己知識所達的地方.
很多時候都被自身限制.
反而不能夠欣賞或者觀察到.
其實你能夠離開自己的自限.
你會發覺上帝在旁邊都做了很多事.
你就知道人算什麼.
在上個星期的訊息中.
會提醒弟兄姊妹.
當我們每件事都很緊張.
是否在他時間的時候.
我們都會用我們自己.
可能自己覺得那件事是上帝的時間.
因為你覺得有什麼事就有什麼事.
就像John上星期用的例子.
但是當你用你自己的時間.
表達上帝做事的時候.
或者你沒有用你自己的時間.
表達上帝做事的時候.
他仍然是He makes all things beautiful.
重點就是上帝不會因應我們所做什麼.
又或因應我們會不會做什麼的時候.
而忘記了他要做什麼.
上帝仍然是愛我們.

$^{121}$仍然守他對我們的約.
以致向我們不斷的施慈愛.
這是上帝給我們一個很重要的訊息.
但是很多時候我們就忘記了.
忘記了只是看我們現在很辛苦.
我們現在很困苦.
很困惑.
不知道怎麼選擇.
我們現在不知道怎麼去做一個裁定.
去還是不去.
留還是不留.
做還是不做.
這些就是這一刻看到的事情.
我相信大家知道要有信心.
但是你不知道怎麼去實踐這個信心.
或者怎麼去讓自己看得遠一點.
但是作者跟我們說.
你看周邊發生的事情.
你看前人不同的經歷.
對我們來說是一個提醒.
第四節開始.
其實他就用了不同的內容.
恩著信是希伯來書十一章裡.
很重要的關鍵字.
恩著信就說不同的人的出現.
恩著信就說不同人發生的事情.
你會看到這裡有十六個.
在第十一章描述的聖經的人.
有些人覺得聖經裡的偉人.
但有些事情不是很偉大.
發生的事情.
你會看到這裡聖經描述的人.
不是每個人都很有信心.
你會看到這十六個人當中.
無論他的經歷如何.
但是上帝仍然會透過他的生命.
去做不同的觸發.
上帝也會透過他們不同的生命.
讓後面的人看到.
即或不然.

$^{161}$上帝會令到國安其事成為美好.
人始終在時限當中.
看不遠也看不透很多事情.
但是你會發覺.
當我們靜下來.
或者停下來的時候.
看看周邊發生的事情.
看看和你同行的人發生的事情.
看看你前人發生的事情.
你會發現一件事就是.
信靠.
當我們有共同的信仰的時候.
你會看到很多不同的前人.
但是我們又失誤地去說一個.
很重要的信息就是.
我們不是第一次聽見證.
你回教會那麼久.
無論你回了多少間教會都會有見證.
你自己也會說見證.
說見證最大的難處是什麼呢.
最大的難處就是.
他的見證不是我.
是不是.
他又娶到一個漂亮老婆.
又厲害的老婆.
又好的老婆.
他又嫁到一個有型的男生.
又好的男生.
我不是.
上次John用這兩個例子的時候.
如果這次沒有聽上星期的時候.
就聽一下吧.
但是你會發覺有很多.
見證也好.
或者他經歷也好.
其實不容易重複的.
而且也不是很關你事.
但是你留心一下.
我自己是挺喜歡聽見證的.
在參與神樂園的過程.

$^{201}$每年我學院的時候.
每年都有50個新生.
大概有50個畢業生.
如果聽見證的時候.
大概一年會聽到100個見證.
50個新生見證.
50個畢業生見證.
加起來100個的時候.
有些是你認識的.
你看到他那三四年在學院的改變的時候.
而那些學生畢業的時候.
他又看回自己入學新生見證的時候.
你會發覺上帝就在那兩三年.
或者三四年當中.
去磨練他 拆毀他 塗造他 培育他.
那個其實也不關你事.
那個也不關其他同學事.
是關他自己事.
你也學不到他.
你也做不到他.
那我們聽見證做什麼呢.
但是另生情也不是不關你事.
就正如這16個所謂信心的偉人.
聖經人物裡面.
也不關你事.
他們也不同.
但是當我們看他的生命歷練的時候.
你會看到有很大的程度.
就是有些common factor出現.
他的生命經歷.
他的信心動搖的地方.
他有些東西他自恃.
他開始學習要重新去理解的時候.
總有些不同的生命片段.
總有些不同的生命歷練.
跟你有相似的地方.
有common factor.
這個就是我們聽見證.
需要去學習.
需要去聆聽的地方.

$^{241}$他不是完全跟你無關的.
他跟我 你跟我.
都是有血有肉的人.
總會有不同掙扎.
有不同面向.
而我們最大的common factor.
就是認識上帝.
而我們最大的common factor.
都是要學習去倚靠上帝.
他講的見證告訴你.
他怎樣倚靠上帝.
跟他怎樣不倚靠上帝.
同樣跟我們一樣.
所以我們聽見證的過程當中.
其實就找出那個highest common factor.
最大公因數.
看聖經也是.
聽見證也是.
在團契 在少祖分享裡面也是.
保羅提醒就是.
不知不覺你的經歷.
就會成為別人的祝福.
你的經歷就會成為別人的借鏡.
不知不覺都是提醒自己.
所以弟兄姊妹.
不要小看你的生命歷練.
也不要忘記要看看.
旁邊事情的發生.
當你看到旁邊事情發生的時候.
你會發覺.
上帝會透過一個.
騙語.
一個人的對話.
一個生命的小篇章.
就提醒.
你其實仍然在他的愛當中.
只不過你忘記了.
在這麼多的16個裡面.
第11章的信心的人物當中.
中間有幾節經文包在中間.

$^{281}$那30幾40節經文裡面.
包了幾節經文.
今天就是我想和大家看這幾節經文.
這幾節經文是第13至16節.
就成為希伯來書信心裡面.
一個很重要的教導.
他的最大公因數是什麼呢.
這些人都是存著信心死的.
並沒有得著所應許的.
卻從遠處望見.
且歡喜迎接.
又承認自己在世上是客裡.
是寄居的.
說者願的話的人.
是表明自己要找一個家鄉.
他們若想念所離開的家鄉.
還有可以回去的機會.
他們卻羨慕一個更美的家鄉.
就是在天上的.
所以上帝被稱為他們的上帝.
並不以為恥.
因為他已經給他們預備了一座城.
這四節的經文就包含了.
在那16個信心的人物當中.
也是他們最大的公因數.
Common Factor.
我們看一下裡面的內容.
第13節.
第13節裡面說的是什麼呢.
有幾個說話可以分開來說.
首先裡面有些內容就是這樣.
你會看到.
這群人就是那16個聖經列舉出來.
經歷了上帝的生命拆毀建造和執政.
而這群人是存著信心死的.
他們是沒有得著所應許.
意味著大家可能看過希伯來書.
或者看過舊約的教導裡面.
那16個人物當中.
就以阿伯拉罕為例.

$^{321}$大家都認識的.
而當中前後都在說阿伯拉罕的描述.
都是在說上帝應許阿伯拉罕.
就是你的子孫.
好像天上的星星海女莎那麼多的時候.
其實他只有一個兒子.
而你會發覺他在說.
他的國成為大國成為萬族的時候.
其實他都沒有自己的住處.
很多看似上帝開了空頭支票.
但是他仍然相信上帝會結帳.
而在過程當中你會看到.
那群信聖經的人物.
他在拿著上帝的應許.
不會落空的時候.
所以你會看到.
從遠處望見.
且歡喜迎接.
又承認自己.
這三個片語.
在中文來說就不是很看得到.
但是在另外一些語言.
譬如英文又或者是在原文.
希臘文裡面當中.
你會看到是一個主動.
他們雖然看不到將來要成就的事.
但是他深信會成就.
所以他們是很主動.
我們會看到這件事.
很深信會看到這件事.
而我是主動去迎向朝著那個方向走.
因為上帝就叫我這樣做.
而是很清楚承認自己.
是一個什麼光景的人.
即是說不然是常常在這些信心懷疑人當中.
沒有被動搖的警方.
有很多人會疏擺他.
有很多人去動搖他.
有很多人去質疑他.
你的神是不是這麼厲害.

$^{361}$你的神是不是這麼威風.
你的神是不是真的做到的時候.
就正如我們在過去的日子的時候.
總會有很多很難決定的.
有很多事情很難去做一個分辨的.
但是你所持守的.
你所堅定的是什麼呢.
真的每個人都不同.
我相信你的抉擇和其他弟兄姊妹的抉擇不同.
但是我們最大的公因素就是.
我們仍然相信上帝是看著.
上帝是在管理著我們.
因為上帝說的是什麼呢.
我們承認的是什麼呢.
我們本身就是一個將會過境的.
但是我們還沒有過境.
我們不會置之不理.
我們仍然會做好一個.
但我們會朝著一個更美的家鄉走下去.
我不知道怎樣看這句話.
但是對於你來說.
有很多人是過境的話.
就不會這麼留心.
或者不會這麼認真.
反正都不是自己的事.
隨便隨便.
但是我自己就常常提醒自己.
每一個階段都有一個學習的功課.
每一個階段都有上帝提醒的功課.
有些東西就是你學了之後.
你下一階段就不需要重新學.
在過去這一年半.
就是18個月左右.
Flowtry就不同弟兄姊妹去了外地.
有些就剛好有Oval Mission.
就是參與的地方他們去了Oval Mission.
有些就沒有的.
有些過程當中他去了之後回來.
有些就去了之後.
他主要去解決一些事情.

$^{401}$最後都會回來.
在過程當中他都提醒.
或者跟我分享.
他回來香港的時候.
都跟一些弟兄姊妹說.
如果你擔心有什麼.
就可以跟他分享.
又或者他都分享一個很重要的地方.
就是他當初以為自己去到一個地方.
就可以的.
以為自己可以的.
但是發現原來跟他想像中都差很遠.
在過程當中.
他就不斷去問自己.
是不是選錯了.
又或者在這個階段的時候.
是不是應該要多一點準備呢.
接著他跟我說一件事.
感恩他還有第二次機會.
就是他回到香港的時候.
他重新去equip自己.
去確保一些東西.
接著他預備跟別人分享.
每個人都很怕選錯.
但是上帝很多時候給我們機會.
去重新去學習.
又有不同的人在當中提醒去幫助.
其實不一定要離開香港才學到的.
其實我們在香港仍然要學到的.
我不知道.
Folks Church很多弟兄姊妹都是在職的.
現在面對很多空間可以轉工.
我認識的弟兄姊妹都是在這一季轉工.
在轉工的時候都很怕選錯.
覺得沒什麼信心.
問一下怎樣可以讓自己突破安書區去轉工.
但很多時候有機會他問我的時候.
我都跟他分享.
其實你很怕錯嗎.
還有你轉工的時候.

$^{441}$其實你都是想一些東西你想學的.
或者你想跳出安書區.
想讓自己更加增值.
擴闊自己的眼界.
其實這個是成長必經的階段.
就把握機會去試.
這是信心的跳躍.
當然我不是口渴就推他出去就這樣做.
但我跟他分析認識他的過程當中.
其實他真的有能力.
他只是欠一個肯定.
真的.
你都是有能力的.
其實你都知道的.
但有時候你欠一個肯定.
周邊的人會給你一個肯定.
周邊的人會跟你分析那件事情.
你會知道上帝會給我們很多不同的人提醒幫助.
這幾個片語提醒我們有主動性.
看不到不要緊.
我們可以主動去尋求.
未必是可以的時候.
或者不知道是否可以的時候.
其實我們都可以主動去認證那件事情.
這個經文提醒我們認定自己的身份.
其實可以很享受過程當中.
下面那幾個經文是什麼呢.
既然我們會主動的時候.
有些人去到第十四節.
說者說的人士表明自己要走過家鄉.
他們若想念所離開的家.
可以回去的機會.
其實不行就回頭.
不行就回頭.
其實有些事情你覺得不行就回頭.
不在乎一個很愚蠢的.
搞錯我出去了就不回頭.
很馬不停蹄回頭草.
又或者這些覺得回到家的時候.
我不是一個失敗者.

$^{481}$可能很多事情會被打敗.
又或者你基本上預備了不行就回頭.
在過程當中.
經文都提到你找一個家鄉.
但你若想享受的時候.
你可以回頭.
但其實這是人之常情.
但那十六個信心的人物.
他經歷他們的過程當中.
其實他們都會咬緊牙根繼續走下去.
念鄉或者鄉愁對於念舊.
在這些日子其實大家都不陌生.
上個月嘉Sir就講了一篇道.
他說應該要安慰一下他.
他才是最慘的那個.
因為他只留下了過去.
其實我也吃了很多這些飯.
我是開心吃的.
因為他找我吃的關係.
他臨離開想大家吃飯.
我是開心吃那頓飯.
雖然我自己是有點不捨得.
但我是開心吃那些飯.
但吃那些飯.
有一個最大的功因素.
就是想當年.
說了很多話.
你知不知道那時候我和你一起.
我知道.
但我自己的人設.
不是很喜歡想當年.
當然我是念舊的.
但我不會經常亂舊.
我經常想將來可以做什麼.
遲些可以做什麼.
我經常想將來的事早點實現.
所以認識我的人都知道.
我經常想退休.
不要笑.
我已經想了怎樣退休.

$^{521}$我預備了20年準備退休的過程.
我40歲開始想退休.
我連在州會帶小組組員的時候.
我都說和他們想怎樣提早退休.
其實是想將來可以做什麼.
是想前景 想將來.
因為我相信.
上帝給我們生命氣息的時候.
人有很多發展的空間.
人生不只是打工.
人生不只是一個階段.
人生不只是你周邊的人.
你還有很多可以帶領的東西.
那是信心.
那是上帝給你可以拓展的空間.
那個眼界.
那個境地.
但吃這些飯.
很多時候我們以前的時候.
還去柴灣灣燒燈籠.
我們以前煲臘很開心.
我們每個人都拿著月餅罐出來.
每個人都拿著幾包臘.
放到整個大堂.
這些是你們不知道的.
你們聽過新聞片.
呼籲大家不要煲臘.
又會有二級和三級的燒傷.
但這些是我們不理的.
小時候的時候.
網上看的節目應該是不對的.
真的認真.
不應該的.
臘粥只可以用來吹.
不可以拿來煲.
待會兒小朋友要收起臘粥.
我們有生日會.
無論如何.
重點其實就是.
那些飯常常都想當年.

$^{561}$你回頭.
你回頭就不要走.
你這麼緊張這麼留戀.
就不要走.
有很多原因.
你要跳出去.
有很多原因你要走出去.
你要開展新的境地.
這樣就拿著信心.
走出去.
形態轉了.
轉一個新的境地.
轉一個新環境.
面對新的挑戰.
最近有很多人.
有很多.
不同的思鄉的.
情感.
在我自己的Facebook的朋友.
就有很多不同的Post up.
中秋節.
我們現在今晚是中秋節.
London現在是下午.
還沒有看到月光.
不知道今晚我的朋友會不會.
Post一些月光照.
外國月光是不是特別圓呢.
當然不是.
認不認識他.
我不認識.
越夜亦赦地.
赦地.
是不是私人特別感受多一點呢.
當然不是.
很多唐代的私人.
或者宋代的.
以前是文人.
其實都經歷了很多生命上的.
起起跌跌.
在患難當中.

$^{601}$戰禍當中.
有很多不同的別離.
那首詩是這樣的.
可能大家都讀過.
不知道是不是.
《書古斷人行》.
邊秋一顏勝.
露從今夜白.
月事故鄉明.
友弟皆分散.
無家問死生.
記書長不達.
放乃未休兵.
意思是什麼呢.
書頭上的古聲.
隔絕了人的來往.
在邊菜的秋天中.
一隻孤雁在叫.
從今晚開始.
就進入了白露節.
就是農曆的八月.
月亮仍然是故鄉最明亮的.
有兄弟.
但我們要分散.
沒有家.
都問不到他是死是生.
寄去洛陽的家書.
常常都送不到.
何況戰亂頻繁.
沒有停止呢.
其實在今天.
這個場景.
都不遑多讓.
疫情其實分隔了很多人.
戰亂在烏克蘭.
也分隔了很多人.
判刑也分隔了很多人.
李寄風信.
也不知道他收不收到.
在過程當中.

$^{641}$有很多分離.
有很多親情.
有很多不同的.
情緒.
是很難去承擔.
不一定要去到外國.
才有鄉愁.
不一定要去到不同的地方.
才有那種掛念之情.
我們面對最大的困難.
就是很多時候.
不懂得.
面對自己的情感.
但我相信.
無論在我們網上的崇拜.
或是在實體的聚會當中.
我們慢慢就找到.
很多同路人.
一起分享.
所以我一開始就說.
你嘗試看看周遭的事情.
周邊的人.
其實不知不覺你會發覺.
原來你也是這樣面對分離.
你也是這樣面對掙扎.
你也是這樣面對牆外牆外的問題.
你也是這樣面對在香港和外國的問題.
不知不覺就找到一群同路人.
去彼此支持.
加添你的信心.
要走出去不容易.
但不要回頭.
在新的形態.
新的形式當中.
上帝給我們新的挑戰.
但會更加確實.
上帝給我們的英雄.
第十六節.
那群信心的聖經人物.
他們帶著什麼呢.

$^{681}$他們卻羨慕一個更美的家鄉.
就是在天上的.
所以上帝被稱為.
他們的上帝.
並不以為此.
因為他已經給他們.
預備了一座城.
這句話.
上面有一個字.
叫做更美.
什麼是更美呢.
上星期John的信息中.
介紹了兩個希伯來文.
一個叫Yaveth 一個叫Tor.
一個叫美.
一個叫好.
更美是什麼意思呢.
如果在.
希臘文中.
特別是當中.
更美要表達的.
那個字叫Craton.
其實是在說.
一個Better.
更加好的.
但這個字在.
聖經中很少用.
用得最多.
更美的字.
就是在希伯來書中.
其後那兩次.
都是在彼得前書.
和在.
哥倫多前書.
都不是在說一個很重要的描述.
但是更美.
在希伯來書中說得很清楚.
跟大家輯錄在.
希伯來書中更美的內容.
你會看到什麼呢.

$^{721}$就是希伯來書第七章開始的時候.
他說.
上帝.
即是耶穌為我們預備的.
是更美的盼望.
是更美的約.
是一個更美的應許.
耶穌就是更美的中寶.
耶穌也是更美的祭物.
耶穌為我們.
所灑的血所成就的.
是一個更美的產業.
而我們就要.
朝著一個更美的家鄉.
因為我們經歷的.
是更美的復活.
而這個是成就了.
上帝更美的事.
去到下一章.
也是希伯來書說到.
終極的時候.
就是.
第十二章.
二十四節就是.
新約中寶耶穌.
以及所灑的血.
這個血是比亞伯拉罕的血.
所說的更美.
希伯來書的作者.
想讓讀者.
去明白一件事.
上帝透過耶穌.
不斷讓人去明白.
上帝為人所預備的.
是眼睛未曾看見.
耳朵未曾聽見.
而人心也未曾想過.
更美的事.
那些全部都不是.
我們用頭腦知識.

$^{761}$看到周圍的事.
可以揣摩到的東西.
是一些用信心才看到的事情.
但是上帝就會為我們.
預備了一個更美的事.
「Creator」這個字.
在新約來說.
只是希伯來書.
用得最多.
而是用了十次.
讓我們欣賞上帝的榮美.
但是我們自己.
是否欣賞上帝給我們.
生命當中的榮美呢.
或者上帝現在教導我們.
而我們是否看到.
我們身邊或我們旁邊.
很多發生更美的事.
還是我們仍然看到.
只是聲風血雨.
只是看到很多崩壞的事情.
而看不到我們可以迎接.
去迎戰一個新的開始.
新的挑戰呢.
親愛的弟兄姊妹.
月提是迎新.
我們不能回頭的.
我們不要再想以前的事情.
你回頭是可以的.
就好像回舊的家鄉.
做回舊的生活.
你可以去回到.
以前的生活.
你可以去做回舊的生活.
你可以去做回.
可以不聞不問.
但是對於我們來說.
其實還有很多可為.
很多空間可以發展.
我們可以有很多事情可以繼續嘗試.

$^{801}$有很多事情可以繼續做.
但我們要轉一個新的形態.
新的形式.
新的方向.
但是那件事在做的過程當中.
憑著信心走上去.
我們會看到上帝的新事.
也看到更美的事.
會在當中發生.
我仍然很喜歡那句說話.
In His time.
He makes all things beautiful.
所以在第十一章裡面.
那十六個信心的聖經人物裡面.
包含了這幾節的經文.
這幾節的經文.
就是成為信心裡面.
最大的公因數.
我今天希望在這個訊息.
在中秋節的時候.
我們團圓.
我們一起集合.
一起思想.
上帝給我們更美的事.
是甚麼呢?.
我們要承認我們有個身份.
無論你是怎樣看這個寄居.
我們都要知道上帝和我們走這條路.
我們承認我們的自限.
但我們承認有同行者.
我們承認我們的不足.
但上帝會看著我們.
而在當中我們更加去羨慕.
一個更美的家鄉.
就是這一刻.
未看到.
但方向是正確.
因為我們認識我們上帝.
這是上帝在我們整個.
屬靈群體當中.

$^{841}$一起帶領我們.
我們羨慕一個更美的家鄉.
而最大的公因數.
你見到.
我們有16個聖經人物.
他們到他們.
生命結束的時候.
都未得著所應許.
但他們不以.
他們所信的耶和華上帝為恥.
我們都以.
我們的信仰為榮.
不以福音為恥.
這福音本是上帝的大能.
兄弟姐妹.
有一句說話.
在這段日子我常常.
看到.
亦提醒自己.
我來是叫人得生命.
並且得得更豐盛.
這是耶穌說的話.
但今天我們很多人.
得著了耶穌.
但我們得著不到耶穌豐盛的生命.
因為很多弟兄姊妹.
都不是.
活得很好.
我希望無論你在.
哪一個地方.
找到同行的弟兄姊妹.
不要自己一個.
在不同地方.
或離開香港.
找教會.
找同行的弟兄姊妹.
不要只看網上的.
找一個群體.
彼此的認識.
一起去同行.

$^{881}$為什麼呢?.
因為你看到的.
就是看到自己的東西.
同行的時候.
你會看到上帝.
透過弟兄姊妹一起.
共建更美的新形態.
新的階段.
新的出發.
很盼望迎新這個主題.
就在中秋節.
當我們有很多思愁.
很多享受.
想起以前的時候.
但同樣想到.
今天一起團圓.
今天一起彼此認識.
新的開始.
希望網上.
如果你在外地的弟兄姊妹.
在那邊有新生活.
就在那邊.
認識新的信仰群體.
開始你的新生活.
不要再停在那裡.
而在香港的弟兄姊妹.
就回到所屬的群體當中.
重新開始.
已經快進入.
快進入.
2022年的第四季.
是時候為自己.
做計劃.
不要等到2023年才做.
很希望.
喜伯來書講信心.
同樣都是講近美的事.
讓上帝在我們生命當中.
有近美的事發生.
我們一起禱告.

$^{921}$天主上每當我們.
打開你的話的時候.
你仍舊對我們說話.
你仍舊告訴我們.
每一位願意.
跟隨你的人.
你在他生命當中.
會預備了.
對他最好的事情.
哪怕是.
他看不到.
但你透過周邊的人.
讓他明白到.
上帝在其中.
這個對我們來說.
我們很小信.
我們看不到.
但求主你為我們預備.
亦求主你常常提醒我們.
我們亦是持守.
上帝給予我們信心.
我們走回家的路.
我們走上帝.
給予喜悅的路.
哪怕環境.
多麼困難.
我們不以福音為恥.
不以耶穌為恥.
求主你加添信心.
奉主名求.
\newpage



\section{約翰福音 6:1-15-20220917}
\label{sec:Z8wgkxuhjIk}
\textbf{【網上聖餐崇拜】仲諗D新野......得唔得呀|約翰福音6\_1-15|20220917 [Z8wgkxuhjIk]}
\newline
\newline
連結: \href{https://youtube.com/watch?v=Z8wgkxuhjIk}{\texttt{ https://youtube.com/watch?v=Z8wgkxuhjIk}} ~~~~ 語音日期: 2022-09-17 
\newline
\newline
\hyperref[sec:oHgankweQ8E]{\small{< < < PREV SERMON < < <}}
~
\hyperref[sec:index_chronic]{\small{[返順時目]}}
~
\hyperref[sec:index_scriptual]{\small{[返順卷目]}}
~
\hyperref[sec:PqPiG_MpRK4]{\small{> > > NEXT SERMON > > >}}
\newline
\newline
約翰福音 6:1-15-20220917
\newline
\begin{longtable}{cl}
\hline
\hline
章節 & 經文 (和合本修訂版)\\
\hline
6:1 & \begin{tabularx}{0.7\textwidth}{X} 這些事以後,耶穌渡過加利利海,就是提比哩亞海。 \end{tabularx} \\ \\ \relax
6:2 & \begin{tabularx}{0.7\textwidth}{X} 有一大群人因為看見他在病人身上所行的神蹟,就跟隨他。 \end{tabularx} \\ \\ \relax
6:3 & \begin{tabularx}{0.7\textwidth}{X} 耶穌上了山,和門徒一同坐在那裡。 \end{tabularx} \\ \\ \relax
6:4 & \begin{tabularx}{0.7\textwidth}{X} 那時猶太人的逾越節近了。 \end{tabularx} \\ \\ \relax
6:5 & \begin{tabularx}{0.7\textwidth}{X} 耶穌舉目看見一大群人來,就對腓力說:「我們到哪裡去買餅給這些人吃呢?」 \end{tabularx} \\ \\ \relax
6:6 & \begin{tabularx}{0.7\textwidth}{X} 他說這話是要考驗腓力,他自己原知道要怎樣做。 \end{tabularx} \\ \\ \relax
6:7 & \begin{tabularx}{0.7\textwidth}{X} 腓力回答他:「就是兩百個銀幣的餅也不夠給他們每人吃一點點。」 \end{tabularx} \\ \\ \relax
6:8 & \begin{tabularx}{0.7\textwidth}{X} 有一個門徒,就是西門‧彼得的弟弟安得烈,對耶穌說: \end{tabularx} \\ \\ \relax
6:9 & \begin{tabularx}{0.7\textwidth}{X} 「這裡有一個孩子,帶著五個大麥餅和兩條魚,但是分給這麼多人還算甚麼呢?」 \end{tabularx} \\ \\ \relax
6:10 & \begin{tabularx}{0.7\textwidth}{X} 耶穌說:「你們叫大家坐下。」那地方的草多,人們就坐下,男人的數目約有五千。 \end{tabularx} \\ \\ \relax
6:11 & \begin{tabularx}{0.7\textwidth}{X} 耶穌拿起餅來,祝謝了,就分給坐著的人,也同樣分了魚,都照他們所要的來分。 \end{tabularx} \\ \\ \relax
6:12 & \begin{tabularx}{0.7\textwidth}{X} 他們吃飽後,耶穌對門徒說:「把剩下的碎屑收拾起來,免得糟蹋了。」 \end{tabularx} \\ \\ \relax
6:13 & \begin{tabularx}{0.7\textwidth}{X} 他們就把那五個大麥餅的碎屑,就是大家吃剩的,收拾起來,裝滿了十二個籃子。 \end{tabularx} \\ \\ \relax
6:14 & \begin{tabularx}{0.7\textwidth}{X} 人們看見耶穌所行的神蹟,就說:「這真是那要到世上來的先知!」 \end{tabularx} \\ \\ \relax
6:15 & \begin{tabularx}{0.7\textwidth}{X} 耶穌知道他們要來強迫他作王,就獨自又退到山上去了。 \end{tabularx} \\ \\ \relax
6:16 & \begin{tabularx}{0.7\textwidth}{X} 到了晚上,他的門徒下到海邊, \end{tabularx} \\ \\ \relax
6:17 & \begin{tabularx}{0.7\textwidth}{X} 上了船,要過海往迦百農去。天已經黑了,耶穌還沒有來到他們那裡。 \end{tabularx} \\ \\ \relax
6:18 & \begin{tabularx}{0.7\textwidth}{X} 忽然狂風大作,海浪翻騰。 \end{tabularx} \\ \\ \relax
6:19 & \begin{tabularx}{0.7\textwidth}{X} 門徒搖櫓,約行了十里多,看見耶穌在海面上走,漸漸靠近了船,他們就害怕。 \end{tabularx} \\ \\ \relax
6:20 & \begin{tabularx}{0.7\textwidth}{X} 耶穌對他們說:「是我,不要怕!」 \end{tabularx} \\ \\ \relax
6:21 & \begin{tabularx}{0.7\textwidth}{X} 門徒就欣然接他上船,船立刻到了他們所要去的地方。 \end{tabularx} \\ \\ \relax
6:22 & \begin{tabularx}{0.7\textwidth}{X} 第二天,留在海的對岸的眾人發覺那裡原來只有一條小船,而且耶穌沒有同他的門徒上船,是門徒自己去的。 \end{tabularx} \\ \\ \relax
6:23 & \begin{tabularx}{0.7\textwidth}{X} 另外有幾條從提比哩亞來的小船,卻停靠在主祝謝後給他們吃餅的地方附近。 \end{tabularx} \\ \\ \relax
6:24 & \begin{tabularx}{0.7\textwidth}{X} 這時眾人見耶穌和門徒都不在那裡,就上了船,往迦百農去找耶穌。 \end{tabularx} \\ \\ \relax
6:25 & \begin{tabularx}{0.7\textwidth}{X} 他們在海的對岸找到他後,對他說:「拉比,你幾時到這裡來的?」 \end{tabularx} \\ \\ \relax
6:26 & \begin{tabularx}{0.7\textwidth}{X} 耶穌回答他們說:「我實實在在地告訴你們,你們找我,並不是因見了神蹟,而是因吃餅吃飽了。 \end{tabularx} \\ \\ \relax
6:27 & \begin{tabularx}{0.7\textwidth}{X} 不要為那會壞的食物操勞,而要為那存到永生的食物操勞。這食物是人子要賜給你們的,因為父神已印證了。」 \end{tabularx} \\ \\ \relax
6:28 & \begin{tabularx}{0.7\textwidth}{X} 於是他們問他:「我們該做甚麼才算是做神的工作呢?」 \end{tabularx} \\ \\ \relax
6:29 & \begin{tabularx}{0.7\textwidth}{X} 耶穌回答,對他們說:「信神所差來的,這就是神的工作。」 \end{tabularx} \\ \\ \relax
6:30 & \begin{tabularx}{0.7\textwidth}{X} 於是他們對他說:「你行甚麼神蹟,好讓我們看見而信你呢?你到底要做甚麼呢? \end{tabularx} \\ \\ \relax
6:31 & \begin{tabularx}{0.7\textwidth}{X} 我們的祖宗在曠野吃過嗎哪,如經上寫著:『他從天上賜下糧食來給他們吃。』」 \end{tabularx} \\ \\ \relax
6:32 & \begin{tabularx}{0.7\textwidth}{X} 於是耶穌對他們說:「我實實在在地告訴你們,那從天上來的糧不是摩西賜給你們的,那從天上來的真糧是我父賜給你們的。 \end{tabularx} \\ \\ \relax
6:33 & \begin{tabularx}{0.7\textwidth}{X} 因為神的糧就是那位從天上降下來,並且賜生命給世界的。」 \end{tabularx} \\ \\ \relax
6:34 & \begin{tabularx}{0.7\textwidth}{X} 於是他們對他說:「主啊,請常常把這糧賜給我們!」 \end{tabularx} \\ \\ \relax
6:35 & \begin{tabularx}{0.7\textwidth}{X} 耶穌對他們說:「我就是生命的糧。到我這裡來的,絕不飢餓;信我的,永不乾渴。 \end{tabularx} \\ \\ \relax
6:36 & \begin{tabularx}{0.7\textwidth}{X} 可是,我告訴過你們,你們已經看見我,還是不信。 \end{tabularx} \\ \\ \relax
6:37 & \begin{tabularx}{0.7\textwidth}{X} 凡父所賜給我的人,必到我這裡來;到我這裡來的,我總不丟棄他。 \end{tabularx} \\ \\ \relax
6:38 & \begin{tabularx}{0.7\textwidth}{X} 因為我從天上降下來,不是要按自己的意願行,而是要遵行差我來那位的旨意。 \end{tabularx} \\ \\ \relax
6:39 & \begin{tabularx}{0.7\textwidth}{X} 差我來那位的旨意就是:他所賜給我的,要我一個也不失落,並且在末日使他復活。 \end{tabularx} \\ \\ \relax
6:40 & \begin{tabularx}{0.7\textwidth}{X} 因為我父的旨意是要使每一個見了子而信的人得永生,並且在末日我要使他復活。」 \end{tabularx} \\ \\ \relax
6:41 & \begin{tabularx}{0.7\textwidth}{X} 猶太人因為耶穌說「我是從天上降下來的糧」,就私下議論他, \end{tabularx} \\ \\ \relax
6:42 & \begin{tabularx}{0.7\textwidth}{X} 說:「這不是約瑟的兒子耶穌嗎?我們豈不認得他的父母嗎?現在他怎麼說『我是從天上降下來的』呢?」 \end{tabularx} \\ \\ \relax
6:43 & \begin{tabularx}{0.7\textwidth}{X} 耶穌回答,對他們說:「你們不要彼此私下議論。 \end{tabularx} \\ \\ \relax
6:44 & \begin{tabularx}{0.7\textwidth}{X} 若不是差我來的父吸引人,就沒有人能到我這裡來;到我這裡來的,在末日我要使他復活。 \end{tabularx} \\ \\ \relax
6:45 & \begin{tabularx}{0.7\textwidth}{X} 在先知書上寫著:『他們都要蒙神教導。』凡聽了父的教導而學習的,都到我這裡來。 \end{tabularx} \\ \\ \relax
6:46 & \begin{tabularx}{0.7\textwidth}{X} 這不是說有人看見過父,惟獨從神來的,他才看見過父。 \end{tabularx} \\ \\ \relax
6:47 & \begin{tabularx}{0.7\textwidth}{X} 我實實在在地告訴你們,信的人有永生。 \end{tabularx} \\ \\ \relax
6:48 & \begin{tabularx}{0.7\textwidth}{X} 我就是生命的糧。 \end{tabularx} \\ \\ \relax
6:49 & \begin{tabularx}{0.7\textwidth}{X} 你們的祖宗在曠野吃過嗎哪,還是死了。 \end{tabularx} \\ \\ \relax
6:50 & \begin{tabularx}{0.7\textwidth}{X} 這是從天上降下來的糧,使人吃了就不死。 \end{tabularx} \\ \\ \relax
6:51 & \begin{tabularx}{0.7\textwidth}{X} 我就是從天上降下來生命的糧;人若吃這糧,必永遠活著。我為世人的生命所賜下的糧就是我的肉。」 \end{tabularx} \\ \\ \relax
6:52 & \begin{tabularx}{0.7\textwidth}{X} 因此,猶太人彼此爭論說:「這個人怎能把他的肉給我們吃呢?」 \end{tabularx} \\ \\ \relax
6:53 & \begin{tabularx}{0.7\textwidth}{X} 耶穌對他們說:「我實實在在地告訴你們,你們若不吃人子的肉,不喝人子的血,在你們裡面就沒有生命。 \end{tabularx} \\ \\ \relax
6:54 & \begin{tabularx}{0.7\textwidth}{X} 吃我肉、喝我血的人就有永生,並且在末日我要使他復活。 \end{tabularx} \\ \\ \relax
6:55 & \begin{tabularx}{0.7\textwidth}{X} 我的肉是真正可吃的;我的血是真正可喝的。 \end{tabularx} \\ \\ \relax
6:56 & \begin{tabularx}{0.7\textwidth}{X} 吃我肉、喝我血的人常在我裡面,我也常在他裡面。 \end{tabularx} \\ \\ \relax
6:57 & \begin{tabularx}{0.7\textwidth}{X} 永生的父怎樣差我來,我又怎樣因父活著,照樣,吃我肉的人也要因我活著。 \end{tabularx} \\ \\ \relax
6:58 & \begin{tabularx}{0.7\textwidth}{X} 這是從天上降下來的糧,不像你們的祖宗吃過嗎哪還是死了;吃這糧的人將永遠活著。」 \end{tabularx} \\ \\ \relax
6:59 & \begin{tabularx}{0.7\textwidth}{X} 這些話是耶穌在迦百農會堂裡教導人的時候說的。 \end{tabularx} \\ \\ \relax
6:60 & \begin{tabularx}{0.7\textwidth}{X} 他的門徒中有好些人聽見了,就說:「這話很難,誰聽得進呢?」 \end{tabularx} \\ \\ \relax
6:61 & \begin{tabularx}{0.7\textwidth}{X} 耶穌心裡知道門徒為這話私下議論,就對他們說:「這話成了你們的絆腳石嗎? \end{tabularx} \\ \\ \relax
6:62 & \begin{tabularx}{0.7\textwidth}{X} 如果你們看見人子升到他原來所在之處,會怎麼樣呢? \end{tabularx} \\ \\ \relax
6:63 & \begin{tabularx}{0.7\textwidth}{X} 聖靈賜人生命,肉體毫無用處。我對你們所說的話就是靈,就是生命。 \end{tabularx} \\ \\ \relax
6:64 & \begin{tabularx}{0.7\textwidth}{X} 可是你們中間有些人不信。」耶穌起初就知道哪些人不信他,哪一個要出賣他。 \end{tabularx} \\ \\ \relax
6:65 & \begin{tabularx}{0.7\textwidth}{X} 於是耶穌說:「所以,我對你們說過,若不是蒙我父的恩賜,沒有人能到我這裡來。」 \end{tabularx} \\ \\ \relax
6:66 & \begin{tabularx}{0.7\textwidth}{X} 從此,他門徒中有很多退卻了,不再和他同行。 \end{tabularx} \\ \\ \relax
6:67 & \begin{tabularx}{0.7\textwidth}{X} 耶穌就對那十二使徒說:「你們也要離開嗎?」 \end{tabularx} \\ \\ \relax
6:68 & \begin{tabularx}{0.7\textwidth}{X} 西門‧彼得回答他:「主啊,你有永生之道,我們還跟從誰呢? \end{tabularx} \\ \\ \relax
6:69 & \begin{tabularx}{0.7\textwidth}{X} 我們已經信了,又知道你是神的聖者。」 \end{tabularx} \\ \\ \relax
6:70 & \begin{tabularx}{0.7\textwidth}{X} 耶穌回答他們:「我不是揀選了你們十二個嗎?但你們中間有一個是魔鬼。」 \end{tabularx} \\ \\ \relax
6:71 & \begin{tabularx}{0.7\textwidth}{X} 耶穌這話是指著要出賣他的加略人西門的兒子猶大說的;他本是十二使徒裡的一個。 \end{tabularx} \\ \\
[1ex]
\hline
\hline
\end{longtable}
$^{1}$好,各位丁姊妹,可能我們不是晚安,或者早晨,或者其他午安.
向世界各地的丁姊妹問安,求天賦一恩典在你當中.
為什麼要向世界各地的人問安呢?.
因為特別是他們在過去的一個星期,有一件事他們沒有做到的.
或者他們做不到的,就是他們沒有看到小薯茄的六周年的數.
想介紹給你們看,如果有機會再出的話,是很值得看小薯茄的六周年的數.
我以為我去的時候是很多很年輕的小學生,或者中學生去的.
誰知道跟我年紀差不多的都大有人在.
你知道我不是被迫的被迫的去.
整場兩個小時九個字的show是挺好笑的.
是你沒什麼機會睡覺的.
洪嘉豪又很帥,唱歌好聽.
不過小薯茄在最後一幕,他刻意加了一幕在最後.
那一幕是令我很感動和震撼的.
我相信每一個人在這個時代裡.
所做的事情都不得不將我們周遭所面對的事與人與事.
都會有些反應.
做娛樂的人不只是想做娛樂.
娛樂的人都想將身旁發生的事情.
以一幕很噁心的樣子去呈現出來.
我祈禱的是我們基督徒在自己崗位上.
或者在不同的地方裡都繼續有這樣的情況出現.
我相信在香港裡仍然有很多人堅持.
在很難的事情裡仍然想做一些不同的事情.
所以這個月的月題是講新的.
所以大體上,我還沒按這個powerpoint會不會出到我那裡.
所以我想了一個題目,還在想新的事情是否可以.
今天我們會思考一些東西.
我們一起思考一個課題.
尤其是小說劇可以六年的裡邊.
基本上我的子女開電視是沒有興趣看的.
只有我的老人家才會看.
好像阿兵被人取笑他會做汪阿姐.
20年後也會受歡迎.
所以對小朋友來說,他們看YouTube看其他東西.
是比看電視多.
小說劇這六年基本上將一些文化一些觀念.
植入在我們當中.
不再用電視機的文化去主導娛樂.
起碼他們成功.

$^{41}$譬如青年荼毒室.
哲學可以入屋.
哲學可以很簡單地將他們在想的東西.
在這個時代裡邊能夠說出來.
我相信在這個時代裡邊.
很多人會做一些新的東西.
而他們能夠成功.
所謂成功就是入到屋.
又或者很多人都努力.
但未必成功入到屋.
但問題是教育在這個時代裡邊.
很多人都覺得很悶.
我都說得很厭.
基本上教會裡邊很難再看到年輕人.
或者青少年在當中.
基本上走的離開的很多.
不是純粹移民問題.
我們仍然沒有一種新的模式.
或者新的東西可以令到下一代的人.
覺得可以容易在裡面福音能夠成長.
但我知道現在很多人在做這些事情.
很多人不同的弟兄姊妹.
同工都在努力這些事情.
但我想今天能夠從一個我們很熟悉的故事.
喂,我在按東西.
對不起,對不起,我正在按東西.
我希望今天從五餅如魚那裡我們能夠看一點東西.
這個故事其實我們很熟悉.
不過我今天選了一個樣式的版本.
不是馬太馬可路架.
這個神蹟是四卷福音書都記載的.
我試試看一看.
他說後來耶穌到加利利去.
就是提比利亞海.
有一群人看到了他在病人身上所行的神蹟就跟從了他.
耶穌上了山之後和門徒坐在那裡.
那時候猶太人的逾節都快到了.
耶穌舉目觀看,看到一大群人向他走來.
就對弗利說,我們在哪裡可以拿一些餅來給他吃呢?.
他說者說是要試驗弗利.

$^{81}$其實他早知道自己應該要怎麼做.
弗利回答他就說.
就算用200銀兩的餅.
200銀兩大約是六至八個月的人工.
如果你人工四萬塊的話大約是三十二萬左右.
也不能夠每人拿一小塊吃.
當然了,你三十二萬在自助餐店.
很難餵飽五千人.
一個門徒就是西門.
彼得的兄弟安德烈對耶穌說.
這個小孩子拿著五個大餅兩條魚.
那時候那麼多人.
我們應該算不得什麼了.
門徒就將這五個大餅給大家吃.
耶穌就吩咐他們坐下.
原來那個地方有很多草.
於是他們就坐下.
南的有五千.
耶穌就拿了餅來.
感恩就分給他們吃.
分魚也是這樣.
他們多少就給多少.
他們吃飽之後.
耶穌對門徒說將剩下的零碎收拾起來.
免得浪費.
最後的三節.
五個大餅.
大家吃剩的零碎收拾起來.
裝滿了十二個男子.
人們看見耶穌所行的神蹟.
就說這個真的要來到世界內才知道.
耶穌知道群眾來.
是要強迫他做王.
就獨自退到山上去了.
五匹魚我們都熟.
我們最喜歡唱的.
像《小男孩的餅和魚》.
或者《我是個普通的人》.
大致上我們是這樣理解.
但我今天選了一個約翰福音版本.

$^{121}$其實你看見特別的地方是甚麼呢.
你見到綠色.
剛才所有綠色的東西都是要highlight的.
在今天的講道裡.
但我們未必講得完.
強迫他做王.
他獨自退到山上去.
這個是在福音書其他裡面沒有記載的東西.
問題是為甚麼要強迫他做王.
然後故事是他獨自退到山上去.
其他福音書裡面不講這件事.
只有約翰福音講的.
我們要立即問的問題是.
我們很難想像耶穌行了五匹魚之後.
餵飽了很多人之後.
突然之間人們就叫他做王.
但他不做.
然後他退到山上去.
如果我們問一下經文的話.
耶穌都很笑的.
那些門徒就到了加伯隆那邊.
在提比利亞那邊就到達加伯隆那邊.
去到那裡的時候.
那些人就看著船上有誰在.
結果整晚發現耶穌都沒有上過船.
只是有些門徒過去.
然後那些門徒就過了加伯隆.
加伯隆就叫耶穌.
耶穌見到之後.
那些人就問他.
耶穌你昨晚是怎樣到海的.
我見不到你坐船的.
結果原來耶穌.
接著的故事是耶穌離海.
耶穌沒有坐船.
他在水上漂.
師兄.
興宮水上漂的朋友.
他漂過去.
他走過去.

$^{161}$其他科目書都有這樣記載這件事.
所以他去到那裡的時候.
那些人就問他.
你做不做王.
大體上故事就是這樣.
這些記載其實在其他科目書裡面.
講少少.
但如果在科目書裡面.
再讀整個第六章的話.
你會發現其實有很多.
很特別的記載.
是其他科目書沒有記載的.
要處理這個問題.
要處理這件事的話.
你知道.
你聽得多.
你知道我想講甚麼.
我一定問的.
要強迫他做王.
他就退了到山上.
在聖經裡面.
你會不會找到這兩個主題同時出現.
你明白嗎.
通常新約寫東西是看舊約的.
七八成都是這樣.
問題是這件事其實有沒有發生過.
新約作者他將約翰.
他要將整個五平原的故事.
他要重新再詮釋他寫的時候.
我們要問的是.
他憑甚麼這樣寫.
當然他一定憑舊聖經.
如果憑舊聖經的時候.
我們就問.
你立即搜尋一下你的資料庫.
或者搜尋一下你的手機.
你可不可以找到.
有甚麼舊的聖經裡面的記載.
是王是要被孤立的.
人家想他做王帝.

$^{201}$但是他逃了出去.
是不是很難找呢.
我想講是很難找的.
基本上是找不到的.
找不到的.
不過其實是很難找到的.
那個故事是甚麼呢.
我們開估一下.
那個是三二記上的九二十章.
我們今天沒有時間講三二記九二十章那麼多.
你容許我沒可能做這件事.
其實三二記上的九二十章.
就是在講薩姆爾孤立蘇羅的故事.
用兩張聖經.
整個兩張聖經就講.
薩姆爾如何孤立蘇羅.
這個故事我們不講他背後的仔細的東西.
今天沒有時間講.
但我們起碼看到的是甚麼呢.
當薩姆爾孤立蘇羅之後.
薩姆爾就在眾人中抽籤.
逐個支派抽.
每個支派抽完之後再抽一個家族.
再抽蘇羅的父親的名字.
再抽蘇羅出來.
孤立完他之後.
蘇羅做了一件事是很好笑的.
蘇羅發生了甚麼事呢.
蘇羅竟然是躲在器皿裡.
躲在器皿裡.
大家說抽到你了蘇羅.
大家就看蘇羅在哪裡.
蘇羅在哪裡.
怎會看不到你呢.
當人找不到他的時候.
誰出聲呢.
你猜不到的.
是耶和華.
耶和華好像是和大家說的.
應該和薩姆爾說的.

$^{241}$他在器皿裡.
結果大家走到器皿裡.
就捉了蘇羅出來.
這個故事.
整個故事第九十十章.
有很多很荒謬的事情發生.
我們今天再講一次.
我沒有時間講九十章.
但起碼這個傳統.
九十章的傳統.
薩姆爾找蘇羅這個傳統.
蘇羅被孤立完之後.
他自己躲起來.
要解釋的.
不過我們今天不解釋.
要解釋很長氣要講很多東西.
但我們不長氣講九十章.
這個傳統竟然放在約翰福音裡面.
我們待會再看多一會.
不單止是這個傳統.
剛才你記不記得我們讀聖經的時候.
五丕如的情景在哪裡.
他說如月節將近.
快到了.
第四節.
那時猶太人的如節快到了.
這個如月節的交代.
基本上在馬太,馬克和路加裡面.
從來沒有交代過如節.
唯獨是在約翰福音交代.
如節快到了.
而在薩姆爾記九至十章裡面.
講了很多關於如節.
一些獻祭的東西.
你先相信我.
你回去慢慢看.
我要證明給你看.
九十章.
薩姆爾記上一章和約翰福音有關係.
我今天要證明給你看.

$^{281}$不過我知道證明不完.
另外有一個故事很好笑.
這句是約翰福音獨有的.
馬太,馬克和路加五丕如沒有.
這句是世上來的才知道.
無端端耶穌分五丕如之後.
接著說吃剩了.
有個人說這是世上來的才知道.
其實我不知道說什麼.
我以為是才知道.
這句說話在十四節裡就完結了.
但其實很奇怪.
在薩姆爾記上.
十章的五至十三節.
同樣地說一件事.
蘇羅無端端受感做了先知.
有人說原來蘇羅也成為先知.
你怎知道.
為什麼會這樣說.
再說一次.
九十章我不說的.
我們不說.
總之他有這些東西.
我只是想讓你知道.
有這些東西.
為什麼在安納一個王的時候.
蘇羅要成為先知.
要受感說話.
整大劈的故事.
九十章有很多內容.
不說了.
我只是想讓你知道.
另外.
將那小男孩的餅和魚.
讓我獻給主.
我們唱吧.
有五丕如.
其實你可以想像.
你以為大家沒有錢嗎.
其實大家也有錢.

$^{321}$有錢就去買吧.
不過是否存夠.
兩銀子不要緊.
起碼五丕如.
一個銀兩也不值.
五丕如.
一個銀兩也不值.
明明大家可以用錢去買東西.
買多少就多少.
大家一起吃一點.
但不是的.
是要用小朋友的五丕如.
你明白嗎.
我們沒有問.
為什麼小朋友的五丕如這麼重要.
哈哈哈哈.
其實也是來自三毫二基.
三毫二基在掃羅被勾勒的過程中.
他不見了一隻驢.
他爸爸不見了幾隻驢.
他就和一個在希伯來文聖經中.
形容他的僕人一起找一隻驢.
結果不知為什麼.
他去找了先知三毫二.
三毫二就勾勒他.
故事很奇怪.
我不會說為什麼.
不要問這些問題.
總之事情就是這樣發生.
但在七十事譯本中.
我們在中文聖經希伯來文翻譯中.
翻譯成僕人.
但原來在七十事譯本的聖經.
希列文的舊聖經.
三毫二上第九章中.
僕人的字就不是價僕人.
希列文就是解孩子.
小朋友.
而小朋友發生什麼事呢.
三毫二上第九章.

$^{361}$他不懂驢.
迷了驢.
因為他要找一隻驢仔.
迷了驢.
然後有個小朋友就說.
叫僕人在舊聖經.
但在七十事譯本的舊聖經.
小朋友就說.
不如我們拿四分一的赤鴨肋出來.
去問問人.
這條路怎麼走.
所以整個故事很奇妙.
是個小朋友無端端提出這個建議之後.
就引導了蘇羅去找一條路.
然後就找到薩姆爾.
其實最終是找薩姆爾.
薩姆爾在第十章的一至三節.
就高立了蘇羅.
所以有個小朋友出來獻一些東西.
是造王之前的必經的東西.
我是不是這樣寫.
沒錯.
這是另一件事.
所以你發現原來.
上拉小男孩的餅和魚不是.
不是因為小朋友帥.
或者小朋友很無私.
不餅和魚不是.
因為在薩姆爾的上的裡邊.
那是一個上帝安排一個很小的人物.
在裡邊讓做王的時候.
能夠成立到他自己能夠做到王.
其實是蘇羅找到薩姆爾的故事.
所以那些人要耶穌做王.
就要有個小朋友獻了五餅魚.
就好像薩姆爾上的第九章第十章一樣.
還有其實在薩姆爾上的第十章.
大概一至十幾節裡邊.
他出現了一個字.
希臘文來的.

$^{401}$就是所謂四面王.
所謂sign這個字.
即是記號這個字.
四面王.
所以這一個字在第十章裡邊.
說蘇羅做王的時候.
都強調有很多奇妙的.
所謂的記號.
或者中文成經譯為兆頭.
這個字出現.
就好像在約翰豐裡邊.
有七個所謂的四面王.
即是所謂記號.
第一個就是水面酒等等.
五餅魚是其中一個很重要的記號.
所以記號是代表這個王立之前.
會有發生的事情.
所以為何對於猶太人來講.
他見到很多.
薩姆爾上的九章十章發生的事情.
所以對於約翰來講.
其實這些事情全部指向他做王這件事.
即是薩姆爾上納蘇羅做王.
蘇羅被選之後就躲起來.
眾人都找不到他.
但你會發覺很奇怪的是甚麼呢.
很奇怪的是.
在蘇羅的故事裡邊.
最終他做王.
耶和華說他在那個戲名裡.
在戲名裡就抓了他出來.
然後眾人就高罵他做王.
然後他成為了以色列裡邊的王.
但你可以想像的是.
在約翰豐裡邊.
耶穌沒有做王.
耶穌不單止沒有做王.
他還說了很多難聽的話.
其實約翰豐的六章.
我們不會讀下去.

$^{441}$即是26至59節.
一大段對話.
跟他的粉絲講的對話.
你明白嗎.
如果你喜歡.
如果你有粉絲的話.
例如Edan.
我女兒做了Edan的粉絲.
她最近做的.
我老婆剛入了.
看完媽媽做了第二個之後.
我老婆就做了姜濤的粉絲.
她有燈牌送了來.
我不明白我家裡發生甚麼事.
我不知道為甚麼會變成前夫.
我沒有入那個組.
如果你有粉絲的話.
有人在Telegram群組裡.
看你的新聞.
你會講些好話.
Edan你怎樣.
或者姜濤你怎樣.
會做這些事.
耶穌不是.
耶穌跟三位主人.
最不同的地方是甚麼.
蘇羅做王.
但耶穌沒有做王.
不單止他沒有做王.
那幾樣他.
他在提比尼亞湖的北邊.
他要過去加帛隆的時候.
那些人盯著耶穌.
因為他要做王.
所以就像粉絲一樣追著他.
死追著他.
如果你知道姜濤在銅鑼灣住.
他經常在銅鑼灣打球.
他十二點下去.
凌晨十二點下去.

$^{481}$可能你在修頓見到他打球.
所以那些粉絲經常盯著他.
在修頓他會來打球.
所以他有沒有坐船.
過到加帛隆那邊.
他知道.
那些跟隨他五千多人都知道.
但奇怪的是.
明知道那麼多粉絲跟著你.
要你做王.
耶穌在二十六節.
他講了很多難聽的話.
那些人說.
耶穌啊.
你能夠餵飽五千人.
好像天降馬來一樣.
我喜歡.
我給你讚好.
看你的影片多點.
多點觀看數量.
多點讚好.
那好吧.
你就收.
安撫一下那些人.
就算了.
耶穌沒有做這些.
耶穌奇怪的是.
祂說你們期望吃馬來嗎.
祂說你們知道.
馬來那一代人吃完之後.
怎麼樣.
全部都死了.
你明白嗎.
在抗疫那四十年.
從二十歲以上的人.
都死了抗疫.
如果你明白五經歷史的話.
所以耶穌用這個例子.
那些人吃完.
你們期望我會繼續.

$^{521}$會做那個奇蹟.
就是給你吃馬來.
那些人就會死了.
所以你不要那麼期望.
耶穌說.
耶穌期望什麼.
我才是天上來的馬來.
祂說你們要吃我的肉.
喝我的血.
就是一間聖餐.
你明白嗎.
一間聖餐.
就是聖餐的觀念.
祂說你們要喝我的血.
吃我的肉.
那些人聽完之後.
都瘋了.
我來靠你們.
如果你是Edan和姜濤的粉絲.
我期望見到你們.
總之你們做什麼.
我都會哇.
你明白嗎.
就夠了.
你不需要說任何話.
總之你們去哪裡.
戴上漁夫帽.
冷氣營.
遮住半個口.
遮住口.
喝什麼.
讓那些人哇.
就像姜濤打球.
跟別人打架.
哇.
這樣就行了.
其實很簡單.
你讓你的粉絲滿足就夠了.
耶穌正在做另一件事.
耶穌叫所有的粉絲.

$^{561}$怎麼樣.
這句你不要看了.
這裡就是.
你見不見格納.
薩納那個地方.
和加伯隆.
那裡就是坐船過去.
這些地圖不需要.
不重要的.
他的門徒有很多聽見第六十節.
就說這話實在難以接受.
誰能聽得進去呢.
這句很慘.
這句話很.
我不知道什麼時候會說出來.
我想如果有人喜歡聽我說聖經.
假設.
你跟那些人說.
什麼.
你說什麼聖經可以說到別人.
會不喜歡你.
還有這句.
第六十節最離譜.
他說他的門徒中很多人退卻離去了.
不再與耶穌同行.
Walk with Jesus這個字.
不說了.
其實同行的字不是團契同行.
那個字應該是相反的.
老師會與學生同行.
Walk with 學生.
那時候的觀念.
如果老師能與學生一起走.
Walk with them.
就證明這個老師很愛.
從來沒有學生不想與老師一起走.
以前的歷史.
以前的文化.
通常有老師能與學生走.
學生求之不得.

$^{601}$這段經文奇怪的地方是.
那些人退卻.
每個人都Dislike 耶穌的Patreon.
突然Dislike.
下面留言寫了很差的說話.
如果你有看小說劇.
就是怎樣.
那個.
那個叫Finally.
就是.
他將所有Haters.
寫給他的說話.
作一首歌.
他將所有Haters留言的東西.
寫出來告訴別人.
那些人多討厭我不喜歡我.
耶穌明明那些人造王.
捧著他.
你明不明白.
突然間他說了一番說話.
26至59節.
那些人聽完之後受不了他.
覺得我跟錯人了.
對不起.
浪費時間浪費青春.
那些燈牌其實沒用的.
我這麼辛苦付了會費.
其實我在MIRROR演唱會也沒得出現.
等得一樣.
我們就退去.
我在這裡問一個問題.
我們在這個年代裡.
什麼東西會令我們這樣.
大體上.
Full Church.
不想是這樣.
對不起.
其實應該想這樣還是不想這樣.
你明不明白.
不想這樣.

$^{641}$其實是好的.
其實後面是好的.
想這樣.
又好像不是很好.
過去是.
星期四是我神學院那位老師.
我上年在這個時候.
特別改編導.
是紀念這位老師.
楊石昌醫生.
我神學院的老師楊醫.
星期四就是他過世兩週年.
我到了1號2號.
我忘了是1號還是2號的凌晨.
我做夢.
我很少做夢.
我應該有做夢.
我很少記得我做的夢.
我少的.
我忘記多我做夢.
那個夢很斷斷續續.
那個夢說什麼呢.
楊醫跟我去了美國.
他說他在美國的那層樓.
沒有的.
做夢是假的.
不是真的.
楊醫說他在美國的那層樓要賣了.
他說沒興趣買那層樓.
你知道多無聊.
做過什麼.
其實那個故事不是買不買那層樓的問題.
在一個過程裡.
我很想和他聊天.
我想和他說話.
那個夢基本上.
醒來的時候我不想結束.
我發覺好像.
我有很多東西很想和他說.
我還沒和他說完.

$^{681}$很特別的.
我不知道為什麼.
發了這樣的夢.
那個夢是.
我這半年裡.
我都沒有什麼夢可以記得.
這個夢到今天還很清楚.
在草皮上.
想捉住他坐著聊天.
去到家的沙發.
最珍惜的圖畫就是.
他坐著,我在這裡.
我有些東西想和他說.
他聽得到.
到這次四距兩週年.
很多感覺很難.
很難形容.
但那個夢好像和我說的是.
其實.
有很多東西在這兩年裡.
很想和他聊的時候.
已經沒有機會.
楊醫.
為什麼要說楊醫.
我記得他.
去世前兩三年.
有一次.
他喜歡做一些夢想小組.
其實那個夢想小組就是.
傳道人拿港島經文和內容的聚會.
你知道傳道人其實沒有什麼經文可以說.
不知道說什麼.
所以楊醫很好.
捉了一班傳道人.
圍了一圈.
夢想完之後.
他解聖經給你聽.
所以我經常懷疑.
那個月有很多人都是說那篇經文和內容.
真的真的.

$^{721}$不是說笑,是真的.
我看那些一起參加的人.
大體上.
都錄了音.
我知道他應該這個星期六日.
港島就是說那篇.
大家都很期望.
在那些聚會裡.
楊醫有一次很笑.
他拿了一個手稿出來.
你知道他很大年紀.
寫書,寫詩篇.
他寫了50篇詩篇.
寫了出來.
解經,怎麼解,希伯來文.
寫好了.
我問他,他手上有很多手稿.
他拿出來給我看.
他說你不出書嗎?.
他說那句話,人們聽不明白.
我立刻說.
人們聽不明白不要緊.
你先給他複製.
他完了之後.
我們在這個話題裡兜轉了很久.
我們問.
為什麼這個人都說了很多年詩篇.
為什麼他寫詩篇.
沒有人明白呢?.
他再說.
其實出去講座.
人們經常找他出去講.
下面的人,大體上的人.
都未必明白他講話.
我們其實很多年前.
經常找他出名的港島.
十幾年前已經說了.
我不再港島了.
他說他不想再港島了.
如果他最近.

$^{761}$他還跟明道社合作.
那些團契.
茶經團契.
他說他想跟別人交流.
過程裡.
他說其實我出去講話.
人們都不明白我講話.
這個話我想了很久.
你講了這麼多年.
為什麼你講的人都不明白.
我心裡想,謙虛而已.
不是的,他覺得很認真.
他說,下面的人都不明白我說什麼.
什麼創造神學.
什麼物理指令.
什麼功能指令.
他寫了書.
其實他一本書都寫過.
他出版的書全部都是後人幫他筆錄的.
從來不是他自己筆寫的.
他都不寫的.
他覺得那些東西沒有人明白.
我開始明白一點.
聽完這篇經文之後.
這篇經文是為他而想的.
我開始明白.
原來有些新的東西.
對聖經有些知識的東西.
不再是以往的.
假設.
你很開心.
找到知識.
或者你會想繼續.
讓下面的人認識多一點.
但坦白說下面的人其實不認識的多.
正如Fold出來有些新的東西做.
有些新的東西做.
其實是不是很多人不明白.
Fold出來做什麼.
我們拍的影片水準很好.

$^{801}$看完影片應該明白.
看完影片都未必會明白.
未必會買入.
其實做一些新的東西的時候.
耶穌期望什麼.
耶穌為什麼說.
不要再吃麻辣.
那代人死光了.
不如吃生命的糧.
耶穌說他才是生命的糧.
你吃我吧,喝我的血吧.
說完這些話.
其實很恐怖,那些人都走了.
不再與耶穌同行.
是學生不再跟耶穌.
其實是羞辱你的.
只有老師不跟你同行.
在那時候的文化.
那班人不跟你同行.
羞辱老師.
其實老師沒了道.
在香港這個年代.
面對這個問題.
有些問題不想再說.
說了N萬年.
青少年問題說了十幾二十年.
都沒有青少年.
傳承的問題說了二十幾三十年.
教會都沒有傳承.
這些問題不值得再討論下去.
你明白嗎.
最近有人跟我說.
聽了我的聚會就說了一些話.
我真的很不耐煩地.
反駁了他一句.
他說的那些話,理論上一定是對的.
媽媽是女人,怎會不知道.
你只不過說媽媽是女人的事.
怎會不明白.
但問題是.

$^{841}$耶穌正在執著什麼.
耶穌正在執著的關鍵是什麼.
整件事.
不單止是他在做新的事.
他要說清楚他想做什麼的時候.
他要讓下面的人真正明白.
讓下面的人真正明白.
耶穌在做什麼.
所以彼得很厲害.
有一次他很厲害.
彼得說,你有永生之道.
還要投靠什麼呢.
我們已經相信了,你是聖者.
但耶穌說完之後.
他還要很活該地說一句.
我們明明有十二個人.
但你其中一個是魔鬼.
那五千人都走了,不重要.
連十二個人裡面也有一個是魔鬼.
什麼來的.
做新的事不是.
不是在意識形態裡面.
我們做新的事,然後一群人湧進來.
哇,新的事,很興奮.
起碼耶穌在這裡說.
什麼叫做新的事.
因為做新的事.
所以很多人很想做新的事.
只不過他在意識形態裡面.
認同做新的事.
但到了他要做新的事的過程.
和經驗裡面的時候.
很多人仍然給舊的事.
總之你給我吃馬騮就夠了.
每天有馬騮吃就夠了.
這些舊的事還在他心裡.
看不清楚他要的新的事.
到底是什麼新的事.
我不知道說得清不清楚.
我最後時間到了.

$^{881}$我再說兩件事就完結了.
很快很快.
我最近去了一個教會廣場.
那個廚師是行主任.
他跟我說一件事.
他說他正在做.
三萬多個IG的.
追蹤者的帳號.
他在那裡募了很多.
微信主的.
開青少組的時候,新朋友來是從晚上六點.
到凌晨一點.
他說那些即稱很喜歡.
在家裡吃東西聊天.
吃宵夜,到凌晨一點.
那些.
他說他正在來商家很崇拜.
我看到報告事項,你邊說他.
我邊說他,他想搞新的崇拜.
他跟我說完之後.
他想搞新的崇拜.
他說我想說到十分鐘.
他說他想在二十分鐘的時間.
做什麼?做瑜伽.
我想讓那些即稱認識一下自己的身體.
跟自己的身體.
對話.
弟兄姊妹.
我這邊最後第二件事想說的是.
我期望崇拜可以不同.
崇拜的模式,形式可以不同.
不一定聽別人說話.
崇拜可不可以.
有幾十個booth.
有做.
陶瓷的.
即是ghost那些.
不要想那些.
做陶瓷的.
做插花的.

$^{921}$做跳舞的.
做畫畫的.
做詩歌創作的.
等等,可以有很多不同的人.
他用他.
裡面其中一個.
gift因此.
他敬拜上帝.
他服侍上帝.
而那些booth很多的裡面.
可能十八區.
有一區是這樣做的時候.
所有崇拜的人.
星期日就不是要回教會四面牆.
他去到那裡.
走進去.
每個booth裡面的體會.
怎樣去敬拜上帝.
人怎樣不同地敬拜上帝.
而你怎樣可以參與.
在不同人的恩賜下.
一起敬拜上帝.
如果那個.
姐妹能夠想到.
崇拜對於即清來說.
不是光榮他聽很多東西.
他想那班人.
在忙碌的生命裡面.
能夠和自己的身體對話.
你明白嗎.
我學了三十多年都不懂.
手不要做那麼多.
心不要想那麼多.
腦停止.
完了.
我到今天都不懂.
我很想參加崇拜.
我為什麼還在聽.
研究你說的話.
然後我要做什麼.

$^{961}$新的東西.
不是純粹說一個意識形態.
大家喜歡新的東西.
喜歡新的東西.
是我們明白.
我們為什麼要有新的東西.
耶穌說.
真正天降下的馬拿不重要.
是天降下.
生命之量耶穌基督.
這件事才夠新.
不要抱著.
以往的馬拿的恩典.
舊有的恩典.
耶穌解給他聽的時候.
多人退卻了.
什麼是新的東西.
是一定經歷很多考驗.
難處拒絕.
但上帝仍然可以在這裡.
當耶穌基督走上.
赫西瑪利園.
禱告.
掙扎完之後再走上十家.
十家.
期望Fold Church群體.
在未來的日子裡.
不是因為Fold Church夠新.
不是Fold Church的燈光夠.
是Fold Church的意識形態裡.
每一個弟兄姊妹.
明瞭Fold Church的路.
走什麼.
欣賞潘Sir一定會搞Info Group.
他要講清楚.
Fold Church在做什麼.
坦白說.
講完之後,人們不一定會回來.
也有很多人會退卻.
你面對組長要埋心目養你的時候.

$^{1001}$你會選擇遲到.
不到.
今天.
我們在做這個餅和杯的時候.
用第六章的Context.
是多人會退去的.
但問題關鍵是.
面對新的事物的時候.
我們是否真的清楚.
我們在想什麼.
在做什麼.
耶穌不是要.
大家拍手掌.
喜歡的東西.
Fold Church更加不是.
有些人想了.
有些事要怎麼做.
Fold Church期望的是.
每一個弟兄姊妹.
你都有在你裡面.
很獨特的想法.
你可以在當中.
在這個群體裡面.
能夠被承傳,被操奏.
被演繹出來.
基督的身體從來.
豐富的.
基督的身體從來不單元的.
不是單一的.
不是幾個,不是數十個.
希望在這個難關裡面.
求主憐憫和幫助.
正如在這幾個月裡.
有一個目者.
在法庭裡面.
他在做一些新事.
在說一些.
在這個世代裡面.
應該要聽的東西.
他說港台不再在.

$^{1041}$教會裡面.
他說港台在.
教會裡面.
他說港台在.
一個很另類的地方.
今時今日.
不同的人.
在做不同的事情.
心願我們不是那些.
只喜歡新事的人.
心願.
當耶穌說了一些.
很難的事.
不容易的事之後.
我們經得起這個考驗.
走過去.
是用生命迎接.
一些新的事.
我一直低頭禱告.
天父多謝你讓我們今天有空間和時間.
來到你面前去.
去看.
《約翰福音》的經文.
一張很豐富的經文.
一張令到我們.
一想起新事的時候好像很興奮.
看完經文.
一想起新事就有很多人退去.
我求天父你幫我們.
原來做新事.
要經得起考驗.
經得起堅持.
沒有人喜歡.
沒有人歡喜.
但有一小數的人.
堅持走到那件事出來.
天父如果在這個世代裡面.
你會興起小樹茄.
你會興起其他的人.
我求天父你在基督教群體裡面.

$^{1081}$你會興起更多的人.
做一些不一樣的事情.
可叫你由榮美得到彰顯和呈現.
多謝主耶穌你今天親自和我們每個人說的話.
我們祈禱奉耶穌基督保衛你.
明知已求.
\newpage



\section{腓利門書 1:1-25-20220924}
\label{sec:PqPiG_MpRK4}
\textbf{【網上崇拜】我地新秩序|腓利門書1\_1-25|20220924 [PqPiG\_MpRK4]}
\newline
\newline
連結: \href{https://youtube.com/watch?v=PqPiG_MpRK4}{\texttt{ https://youtube.com/watch?v=PqPiG\_MpRK4}} ~~~~ 語音日期: 2022-09-24 
\newline
\newline
\hyperref[sec:Z8wgkxuhjIk]{\small{< < < PREV SERMON < < <}}
~
\hyperref[sec:index_chronic]{\small{[返順時目]}}
~
\hyperref[sec:index_scriptual]{\small{[返順卷目]}}
~
\hyperref[sec:1O4Wz5DFm4k]{\small{> > > NEXT SERMON > > >}}
\newline
\newline
腓利門書 1:1-25-20220924
\newline
\begin{longtable}{cl}
\hline
\hline
章節 & 經文 (和合本修訂版)\\
\hline
1:1 & \begin{tabularx}{0.7\textwidth}{X} 為基督耶穌被囚的保羅,同弟兄提摩太,寫信給我們所親愛的同工腓利門、 \end{tabularx} \\ \\ \relax
1:2 & \begin{tabularx}{0.7\textwidth}{X} 亞腓亞姊妹,和我們的戰友亞基布,以及在你家裡的教會。 \end{tabularx} \\ \\ \relax
1:3 & \begin{tabularx}{0.7\textwidth}{X} 願恩惠、平安從我們的父神和主耶穌基督歸給你們! \end{tabularx} \\ \\ \relax
1:4 & \begin{tabularx}{0.7\textwidth}{X} 我在禱告中記念你的時候,常為你感謝我的神, \end{tabularx} \\ \\ \relax
1:5 & \begin{tabularx}{0.7\textwidth}{X} 因聽說你對眾聖徒的愛心,和你對主耶穌的信心。 \end{tabularx} \\ \\ \relax
1:6 & \begin{tabularx}{0.7\textwidth}{X} 願你與人分享信心的時候,能產生功效,讓人知道我們所行的各樣善事都是為基督做的。 \end{tabularx} \\ \\ \relax
1:7 & \begin{tabularx}{0.7\textwidth}{X} 弟兄啊,由於你的愛心,我得到極大的快樂和安慰,因為眾聖徒的心從你得到舒暢。 \end{tabularx} \\ \\ \relax
1:8 & \begin{tabularx}{0.7\textwidth}{X} 雖然我靠著基督能放膽吩咐你做該做的事, \end{tabularx} \\ \\ \relax
1:9 & \begin{tabularx}{0.7\textwidth}{X} 可是像我這上了年紀的保羅,現在又是為基督耶穌被囚的,寧可憑著愛心求你, \end{tabularx} \\ \\ \relax
1:10 & \begin{tabularx}{0.7\textwidth}{X} 就是為我在捆鎖中所生的兒子阿尼西謀求你。 \end{tabularx} \\ \\ \relax
1:11 & \begin{tabularx}{0.7\textwidth}{X} 從前他與你沒有益處,但如今與你我都有益處。 \end{tabularx} \\ \\ \relax
1:12 & \begin{tabularx}{0.7\textwidth}{X} 我現在打發他回到你那裡去,他是我心肝。 \end{tabularx} \\ \\ \relax
1:13 & \begin{tabularx}{0.7\textwidth}{X} 我本來有意將他留下,在我為福音所受的捆鎖中替你伺候我。 \end{tabularx} \\ \\ \relax
1:14 & \begin{tabularx}{0.7\textwidth}{X} 但不知道你的意見,我不願意這樣做,好使你的善行不是出於勉強,而是出於自願。 \end{tabularx} \\ \\ \relax
1:15 & \begin{tabularx}{0.7\textwidth}{X} 他暫時離開你,也許是要讓你永遠得著他, \end{tabularx} \\ \\ \relax
1:16 & \begin{tabularx}{0.7\textwidth}{X} 不再是奴隸,而是高過奴隸,是親愛的弟兄;對我確實如此,何況對你呢!無論在肉身或在主裡更是如此。 \end{tabularx} \\ \\ \relax
1:17 & \begin{tabularx}{0.7\textwidth}{X} 所以,你若以我為同伴,就接納他,如同接納我一樣。 \end{tabularx} \\ \\ \relax
1:18 & \begin{tabularx}{0.7\textwidth}{X} 他若虧負你,或欠你甚麼,都算在我的賬上吧, \end{tabularx} \\ \\ \relax
1:19 & \begin{tabularx}{0.7\textwidth}{X} 我必償還。這是我—保羅親筆寫的。我並不用對你說,甚至你自己也虧欠我呢! \end{tabularx} \\ \\ \relax
1:20 & \begin{tabularx}{0.7\textwidth}{X} 弟兄啊,希望你使我在主裡因你得益處,讓我的心在基督裡得到舒暢。 \end{tabularx} \\ \\ \relax
1:21 & \begin{tabularx}{0.7\textwidth}{X} 我寫信給你,深信你必順服,知道你所要做的,必過於我所說的。 \end{tabularx} \\ \\ \relax
1:22 & \begin{tabularx}{0.7\textwidth}{X} 此外,還請給我預備住處,因為我盼望藉著你們的禱告,必蒙恩回到你們那裡去。 \end{tabularx} \\ \\ \relax
1:23 & \begin{tabularx}{0.7\textwidth}{X} 為基督耶穌與我一同坐監的以巴弗問候你。 \end{tabularx} \\ \\ \relax
1:24 & \begin{tabularx}{0.7\textwidth}{X} 我的同工馬可、亞里達古、底馬、路加也都問候你。 \end{tabularx} \\ \\ \relax
1:25 & \begin{tabularx}{0.7\textwidth}{X} 願主耶穌基督的恩與你們的靈同在。 \end{tabularx} \\ \\
[1ex]
\hline
\hline
\end{longtable}
$^{1}$大英姐妹平安.
剛才的敬拜很感動.
不知道大家有沒有留意歌詞.
看到上帝真的要開我們的眼.
讓我們真的認識祂的時候.
我們會看到我們真的是一群蒙愛的人.
我們在聽講道之前一起有個禱告.
我們的主,我們的上帝.
一起向你敬拜.
一起懇求你繼續充滿我們每一個.
叫到我們現在去聽你話語的時候.
聖靈你去教導我們,引導我們.
賜力量給我們.
以致我們真的走在神的說話裡.
讓我們能夠實踐你給我們愛的使命.
也讓我們能夠繼續認識你更多.
主啊,你賜福我們每一個.
我們這樣祈禱交託.
奉耶穌的使命祈求,阿們.
今天我的講題叫做.
我們新秩序.
咦?.
怎樣?.
我們有個PowerPoint.
我們新秩序.
我跟我丈夫說.
我已經想好了我的講題了.
叫做我們新秩序.
然後他的回應第一句是.
咦?你說古惑仔嗎?.
其實我沒有想過古惑仔.
我無法連結的.
他這樣說我就想.
咦?那些人是不是真的會想到古惑仔呢?.
那我當然就馬上去Google大神問一問.
他真的給了我兩套戲.
那就是剛才那個海報.
第一套,我不知道大家認不認識這些古惑仔.
我那個年代就不是這幾個古惑仔.
你們是哪個年代的?.

$^{41}$我是陳浩.
他們都好像叫陳浩南.
但就不是這些人做的.
我沒有看過的.
其實Google給了我三套戲.
有一套是外國片.
直譯沒有問題.
另外兩套就是這套.
一套就是左手邊2016年叫做.
古惑仔江湖新秩序.
我沒有看過.
但大約就是說他們轉了大佬.
所以江湖有一番腥風血雨.
所以有新的大佬.
所以有新的秩序.
另一套就叫做新秩序.
是2022年的.
大家知不知道原來今年有套這樣的戲.
可能大家都不知道.
因為在大陸上映的.
資料顯示,很有趣.
資料顯示這套片的主打是什麼呢?.
就是暴力犯罪爽片.
你看到別人暴力和犯罪很爽.
都很奇怪.
主角張家輝為了報仇.
然後捲入黑白兩道的勢力.
又是有一番腥風血雨.
很有趣.
我們以為黑社會.
古惑仔這些群體應該無法無天.
應該做什麼都可以.
但其實不是.
他們都有自己的群體秩序.
由他們自己的群體建立.
屬於他們群體的人.
就守住他們的秩序.
他們其實是看他們的秩序.
比其他的秩序,其他的法律更大.
更有權威,更有效.

$^{81}$所以他們才可以漠視其他的秩序.
做他們想做的事.
都沒有什麼可怕.
也沒有人管他們.
其實Google除了給了我這幾套戲.
還給了我一本書.
不知道大家知不知道有這本書.
這本書叫做《香港新秩序》.
我作為,我覺得我們.
特意望著鏡頭說.
我覺得我們作為今天香港人.
是一定要買這本書的.
這本書叫做《香港新秩序》.
由橙新聞出版社出版的.
大家知不知道橙新聞.
很多新聞出版的,橙新聞.
好了,書本簡介聽一聽.
留心.
在國安法下,香港的憲制更加完整和完善.
擁有新的秩序,塑造了新香港.
這個是書本簡介.
這本《香港新秩序》這本書.
是告訴我們.
本身我們都有一個秩序.
不過失效了.
一般般.
所以來一個新的.
其實差不多,都是換了一個大佬.
所以再建立一個有效.
適合用,更完善的秩序.
讓我們香港可以有一個新的秩序.
重回正軌.
原來新秩序會想到這些.
新秩序就是本身有的.
不過不適合用.
或者那個秩序不適合我們那個群體用.
我們不是跟那個大佬.
所以我們不用.
我們一班信耶穌的群體.
其實我們在跟一個什麼秩序呢?.

$^{121}$我們的秩序是誰定出來給我們的呢?.
又或者我們秩序的內容.
我們清不清楚呢?.
我們看我們那個秩序的權威.
夠不夠勁,夠不夠大呢?.
以致我們要做的事.
是不害怕的呢?.
今天我會和大家看《肥理門書》.
如果我們細心去看《肥理門書》的時候.
我們會發覺保羅.
他說的,他做的.
其實都跟當時一個羅馬帝國.
很不同的新秩序.
其實我上一個月都是說《肥理門書》.
所以有關詳細背景.
他們的立場.
我今天不重複.
大家有興趣可以訂閱我們流堂的YouTube channel.
可以翻看.
我的堂是8月20日.
忘記了,總之是第四週.
大家有興趣可以翻看我所說的話.
沒有也不要緊,繼續聽我說就行了.
《肥理門書》是保羅坐牢的時候.
為什麼他坐牢呢?.
因為傳福音,所以他坐牢.
在監獄裡面.
在教會的主人肥理門.
派了他的奴隸亞尼西姆.
去照顧保羅.
保羅在監獄裡面帶領亞尼西姆信耶穌.
然後寫了一封信給他的主人肥理門.
和肥理門的家人.
即是哥羅西教會.
上個月我說過.
保羅一開始在第三節.
這一句的問候語.
他說「願因為平安.
從神我們的父和主耶穌基督歸於你們」.
這一句問候語提醒我們.

$^{161}$我們一群信耶穌的群體.
手的秩序.
就是由我們的父神和主耶穌基督而來的.
因為他是我們這個群體最強的那個.
其餘我們全部都是頂子妹.
我們平起平坐.
無分高低.
無分上下.
無分輕重.
所以第十六節是肥理門書裡面保羅的期望.
他期望就是肥理門.
當亞尼西姆不要再是奴隸了.
是親愛的弟兄的關係.
相信了耶穌.
我們擁有新的關係和身份.
我們是親愛的弟兄姊妹的關係.
在我們今天來說其實是很理所當然的.
有一點廢的我們覺得.
是這樣的.
但當時的羅馬帝國不是這樣想的.
保羅叫肥理門做弟兄.
是很衝擊的.
是一個新秩序.
今時今日的香港.
我們未必體會到保羅這個新秩序的高難度.
因為我們今天是習慣了弟兄姊妹這個稱呼.
如果今天有人問那個是誰.
哦 都是弟兄來的.
其實代表什麼.
代表他也相信耶穌.
也是代表他也相信耶穌.
剛剛說那個是誰.
教會的姊妹.
代表什麼.
大家分頭去一間教會的意思.
未必有什麼特別關係.
純粹是.
或者那個關係其實很遠.
我們習慣了叫別人做兄弟.
我們也想像韓國的惡霸是自己的惡霸.

$^{201}$我們很習慣這些兄弟姊妹的關係.
所以我們不覺得保羅說.
我們當亞歷山河是弟兄.
是一件這麼難的事.
我們很不明白羅馬帝國的階級觀念和奴隸制度.
其實是比我們想像中.
更大影響力.
更複雜.
保羅在這裡叫.
你當他是弟兄.
不是像我們今天說的.
認識人比認識字的那種關係好.
不是說我認識某某老闆.
這是我的朋友.
關照一下吧.
然後可能就讓他升職.
對他好一點等等.
完全不是這個想法.
要在羅馬帝國的奴隸制度下.
要活出保羅說的這個新秩序.
要用弟兄姊妹.
要親外弟兄姊妹關係相稱.
可以說是一件很瘋狂的事.
是很難以想像的.
亞歷山盟是奴隸.
奴隸或者整個奴隸制度.
在羅馬帝國是一個怎樣的存在呢.
我不知道大家知不知道.
羅馬帝國有資料說.
有六千萬人口.
知不知道有多少個是奴隸.
有三分一人口是奴隸.
即是說有二千萬人是奴隸.
我計算得對吧.
六千萬人口.
三分一人口是奴隸.
有二千萬人是奴隸.
數目非常龐大.
擁有大量的奴隸.
你想不想到對國家有什麼好處.

$^{241}$就是擁有大量.
和又便宜的生產力.
當時的奴隸不是只是做我們想像中的厭惡性的工作.
或者低層次工作.
有些奴隸是可以厲害到.
可以管理主人家裡的大小事務.
包括管理他的財富.
管理他的家裡都可以.
有些主人甚至會讓他們的奴隸讀書.
以至他們可以代替主人出門遠行.
做信差 做生意都可以.
甚至成為自己兒子的老師.
都可以是奴隸做的.
奴隸的工作是廣泛到我們難以想像.
所以某程度羅馬帝國當時為什麼經濟這麼蓬勃這麼發達呢.
就是因為他們擁有大量可以說是不用錢的生產力.
其實整個羅馬制度的目的.
都是維護著羅馬世界的利益.
所以你要管理這二千萬人的奴隸.
羅馬帝國明白到不可以只是用殘暴和威嚇的方法對待他們.
所以羅馬帝國是可以建立了一套很複雜又很仔細的法律.
所謂去管理這班奴隸.
羅馬有兩套的法例.
一套是管人 一套是管物品.
奴隸是放在物品的法律裡.
所以羅馬的法例是會保障奴隸.
但不是因為他們的人權.
也不是因為人性的角度去保障他們.
而是全部出於經濟的角度.
你可以想像到的.
你沒有了一個奴隸.
或者你的奴隸做了一些事.
令到主人在經濟上有損失的時候.
其實他虧蝕了.
所以是有一套的法律去保障主人的財產.
有一點像今天你買保險.
這些東西很貴吧?.
不知道有沒有買保險.
有聽過嗎?.
娛樂公司會幫歌手買一雙腳的保險.

$^{281}$以免他雙腳花了.
做秀不漂亮.
買一雙保險.
羅馬的法律也是這樣.
幫奴隸當作物品去做.
有法律去操控或者管理奴隸都不夠.
羅馬帝國的文化或者價值觀.
更加營造到整個制度是很理所當然的.
同時奴隸都覺得自己賣命為主人為國家去做事是對的.
可能大家都聽過.
如果在羅馬帝國裡面.
你在街上逛街.
透過你的穿衣外表.
你就已經分到他人的階層.
知道他是羅馬公民.
知道他是奴隸.
知道他是主人.
不同階級的人會穿不同顏色和不同布料的衣服.
所以如果有些大型活動.
例如去到兜售場.
最前的那些.
現在多少錢的演唱會票.
1980元的那些票.
就是穿著紫色衣服.
穿著很漂亮的帝王級官員.
就是坐在那裡.
有一排紫色的.
高一點的就是白色衣服的羅馬公民.
坐在山頂的就是雜色什麼顏色都有.
粗衣麻布的奴隸.
你想不想像到.
當時社會的文化是將人分階級.
是普遍到.
你用眼睛已經看到你是什麼人.
用眼睛已經會分你是什麼階級.
用眼睛已經在想.
我要尊重你.
還是我當作看不到你.
再加上當時有一個叫做恩庇制度.
是更加鞏固奴隸制度.

$^{321}$這種階級的文化.
恩庇制度代表.
上面的人要照顧下面的人.
上面的人給你屋住.
給你工作.
給你飯吃.
下面的人本身什麼都沒有.
你就是靠著這班主人給你.
所以你下面的人就要擁護他.
你就要效忠他.
是很好的這個制度.
在整個羅馬帝國.
對人有階級之分.
是正常到不得了的.
我一生出來就是奴隸.
有主人給我吃東西.
有屋給我住.
我已經很感恩.
管我是對的.
有些人一生出來是帝王之家.
特別一點 能力大一點.
我就要去管人.
這種有權去管人.
或者壓榨人.
或者我們今天所說的剝削人.
對於當時來說是理所當然到不得了.
最重要是這些人仍然覺得.
這個階級 這個秩序.
是沒有問題之餘.
是一個好秩序.
因為奴隸可以安居樂業.
就是所謂.
在家裡有地方住有飯吃.
大家庭穩固.
國家又好.
又安全又穩定又繁榮.
今天我們去看的時候.
這種操控.
這種將人分階級.
甚至將人工具化的好秩序.

$^{361}$是深深影響整個羅馬帝國的人.
可以說是入了血.
他們的生活習慣對人的態度或者掛席觀.
是完全受到這件事影響.
奴役奴隸.
或者視奴隸為工具.
是合情合理.
兼且合法.
這種合情合理合法的價值觀.
是每一個人都是這樣做的.
堂堂羅馬帝國的奴隸制度.
法律又嚴謹.
輿論工程又做得這麼好.
我們這班信耶穌的群體.
說要當奴隸是親愛的弟兄.
這種新秩序.
就好像我們要打大佬一樣.
怎麼做啊?.
有什麼可能會做到?.
你做到別人都當你傻子.
我們面對的就是.
好像一個不能撼動的帝國和制度.
保羅叫我們憑什麼去抗衡?.
或者我們擁有什麼.
比這種權勢更強大更有力量?.
保羅以身作則.
在《葉肥蒙書》示範給我們看.
他憑什麼去抗衡.
這個看來好像不能搖動的帝國或者制度.
第八第九節他這樣說.
我雖然靠基督能夠放膽吩咐你合而的事.
然而像我自己有年紀的保羅.
現在又是為基督耶穌被囚的.
寧可憑著愛心求你.
放膽吩咐這四個字.
廣東話的直譯是.
我大可以直接命令你.
叫你做事.
命令你 要求你.
做應該要做的事.

$^{401}$保羅為什麼可以有這個權?.
就是因為他的地位 他的身份.
他雖然不是親手創立這個哥羅西教會.
但是他是外邦人的使徒.
相傳正經都是這樣說的.
他存荒的時候有哥羅西的人信了耶穌.
然後回到哥羅西建立教會.
所以對哥羅西教會或者對肥理門來說.
保羅是一個什麼人呢?.
是獨高望重的牧師.
加上和本這裡譯作吩咐.
吩咐這個字原本就是一個軍事用途的字.
就是This is an order.
不可以say no.
是要絕對服從的.
保羅說我大可以就有這個權.
去吩咐你這樣做.
保羅很明白.
在羅馬世界上一層的人叫下一層的人做事.
理所當然 無可拒絕.
吩咐得最多的人是誰呢?.
就是主人 就是肥理門.
肥理門他作為主人他都很明白.
他應該就是最常命令人的那個.
他要吩咐的時候沒有人可以反對他.
而他也同時明白.
地位高過他的人吩咐他做事.
他都無法反對.
所以保羅如果想肥理門視亞尼西姆為弟兄的時候.
其實保羅order他就可以了.
你叫他這樣做他就要做了.
用權力落order威嚇人.
是最有效最快捷的方法.
但保羅說我雖然可以這樣做.
我雖然有權order你.
但我寧願用愛心求你.
他有權不用 他說他用愛.
今天我們說愛可能有人覺得太天真.
愛這個字我們今天說得太簡單太輕易了.
情歌會出現 看劇會出現.

$^{441}$沙冷希我們經常說.
派心我們有一千種方法.
我們連message send stickers.
都有很多個心心跟你說I love you.
都可以這樣做.
愛我們究竟想到什麼.
愛我們會想到自我中心.
自己喜好.
我喜歡就愛 不喜歡就不愛.
或者我們會想到我們什麼都愛.
不理的我們包容接納的那種愛.
又或者我們今天說愛我們會生氣.
今時今日你還說什麼愛.
大把正經事要做.
公義你都未到你跟我說愛.
或者我們又再想到就是.
管理香港的人今天說要愛國愛黨.
是不是他們的行動就叫做愛.
第八第九節保羅說我要憑著愛心求你.
這個愛字用到的字是叫agape.
我想可能有人聽過這個原文的字.
不用理他 總之是這個字.
總之是這個字.
新的用這個字形容愛.
我們今天可能有些人不懂原文都聽過.
agape 是不是.
要不要一起讀一次.
不用了 想啊.
對了 用這個字.
當時不常用的這個字.
當時表達愛通常用另外兩個字.
上面就是保羅用的字.
下面就是另外兩個字.
懂不懂看呢.
不懂的話待會回去問John或者問潘Sir.
總之兩個字用英文讀就是.
Eros或者Filio.
隨便啦 其實都是在說愛的意思.
當時羅馬帝國的人通常說愛.
喜歡 情慾 人與人之間的愛.

$^{481}$就用下面兩個字.
就是這個意思.
agape 有些人說形容犧牲的愛.
無論怎樣都好 什麼字都不解釋.
最重要就是知道.
當時的人不常用的.
agape 這個字.
保羅是特意用這個字.
或者《新約聖經》都是特意用這個字.
去形容神對人的愛.
當時的人聽到agape的時候.
會停一停想.
有些東西.
不是平時說的那種愛.
不是我們想的那種.
保羅是特意用這個字去突出.
告訴我們.
我們的愛和世界的愛是不同的.
這份愛不是大愛.
不是自我中心的愛.
不是愛國愛黨的愛.
不是動不動就生氣的那種愛.
保羅所說的愛.
和世界所說的愛是不同的.
或者甚至.
我們想《聖經》的愛都可能有些不同.
保羅在《斐理門書》裡面.
很短的《斐理門書》.
他沒有詳細去說.
究竟他心目中.
或者《聖經》裡面.
他覺得耶穌基督的愛是什麼.
但是他已經assume了.
斐理門是知道的.
所以他不需要再解釋.
同時也assume我們這班人是知道的.
他都不需要解釋.
他已經明白到.
如果我們重新再看保羅書信裡面.
說耶穌基督的愛的時候.

$^{521}$我們會知道原來愛.
比我們想像中更豐富.
比我們的理解更大.
保羅所說的這個愛.
我們可以想像到.
這個愛是當我們還作罪人的時候.
神就無條件拯救我們那種愛.
是和我們復和的愛.
愛是我們得救的源頭.
同時愛也是神給我們的召命.
因為神和我們復和.
因為我們與神復和.
我們得到從神而來的愛.
我們有新的身份和使命.
我們的使命就是要世界.
重回神的正軌.
令到人與人復和.
人和世界都復和.
保羅這裡說除了用愛心去求利.
其實他有行動的.
他做出來給肥利們知道.
這封書信其實正正就是保羅.
向肥利們展示他怎樣用生命.
去活出這個愛的召命.
保羅是一個德高望重的牧師.
一個為福音受苦.
一個去很多不同地方傳福音的外邦人使徒.
我們可以想像保羅是非常忙碌.
兼且坐牢.
很多事情要兼顧.
他現在親手寫一封信給肥利們.
如果你是肥利們.
你現在還未打開這封信.
你可能有幾分緊張.
在想堂堂德高望重的牧師寫一封信給你.
你覺得他會說什麼?.
你會想他是不是叫你提醒你教會應該怎樣做.
有什麼事情你要留意呢?.
但是你打開後發現.
這封信是為了一個奴隸.

$^{561}$一個在全世界眼中微不足道.
不當是人.
只是工具的奴隸去寫的一封信.
你留意.
當年寫信不像我們今天傳訊息那樣容易.
輕率地傳完後覺得寫錯了.
都會刪除.
不是這樣的.
保羅當年要執筆寫信.
是代表他用行動告訴肥利們.
他就是看重阿尼西姆.
他就是看阿尼西姆是人.
他看阿尼西姆就是他親愛的弟兄.
肥利們拿著這封信.
他就真的體會到.
什麼叫做視阿尼西姆為人.
保羅示範給他知道.
就好像我們.
從來牧師傳導人約你吃飯.
都應該叫你做事.
近來開了新的試訪.
有沒有興趣參與?.
又或者.
又或者第二個原因.
你做錯事了.
上台說話好像說錯了.
照你廢吧.
牧師傳導人約吃飯.
有一點點.
突然有一天John或潘Sir約你吃飯.
心裡會害怕.
如果他們約我吃飯.
其實我就不害怕.
就是這樣說.
可能害怕.
糟了,是不是做錯事了?.
是不是想解僱我?.
是不是以後不讓我講道?.
就是這樣想.
誰知道原來他關心你.

$^{601}$跟你聊天.
你祈禱.
你能想像到嗎?.
窩心的.
傳導人約你吃飯.
沒有目的.
只是想關心你.
只是看重你.
不是叫你做事.
費利門拿了這封信就是這種感覺.
這麼忙的保羅.
竟然為了眾人眼中的工具人.
寫了一封信.
費利門很驚訝.
同時也體會到.
保羅不是只會說話.
保羅會用行動表達.
他真的把阿彌西佈當作是弟兄.
第十四節.
他說.
但我不知道你的意思.
我就不這樣做.
叫你的善行不是出於勉強.
而是出於甘心.
保羅說.
我有權自己自作主張.
就把阿彌西佈留下來.
讓你來服侍他.
留下來服侍我吧.
但他不敢做.
目的是什麼?.
目的是他叫你的善行不是出於勉強.
而是出於甘心.
保羅的意思是.
如果我要留下來.
費利門就沒有話可說.
可能他不想.
但他不能拒絕.
所以保羅說.
我不勉強你.

$^{641}$我希望你的善行是出於甘心.
甘心這個字.
可以譯作自願.
我們聽自願可能會想.
自願即不是自願.
只是表面自願.
實際上是沒得選擇.
當時不是這樣的.
自願在當時是什麼狀態呢?.
就是不被奴役的狀態.
即你不是奴隸.
他叫你的善行不是出於勉強.
而是出於甘心.
即你不是奴隸.
他的意思.
但我們去想想.
當時階級觀念這麼重的時代.
什麼人可以有自願去選擇?.
誰可以不被奴役?.
其實是沒有的.
除非你是羅馬皇帝.
坐在最高位的那個.
其餘身在羅馬帝國的人.
嚴格來說.
都沒有自願可言.
但保羅在這裡跟肥美們說.
我不勉強你.
我不用權壓制你.
你可以按著你自己的意願去選擇.
這個是因為保羅對肥美們的愛.
愛給人有自由.
保羅用行動給肥美們去體會到.
這種愛的新秩序.
愛的秩序令人有自由.
你有自由去選擇.
即使有權力.
但我們可以自願不用.
即使我位於高位.
但我可以選擇苦就卑微.
看一個奴隸為我親愛的弟兄.

$^{681}$這種秩序跟羅馬世界的秩序.
是完全相反的.
羅馬的階級觀念是用權壓下來.
逼人就範.
只是會跟自己地位高的人拉關係.
而不會苦就卑微.
跟奴隸同等.
保羅用什麼方法去抗衡羅馬帝國?.
是愛.
保羅很明白.
上帝體重的不是我們的行動.
他要的不是我們只有外表.
表面聽話的機器.
所以他不用世界的方法.
他不用權力去帶領我們.
上帝體重的是我們的存在.
我們真的被上帝的秩序.
上帝的愛去更新改變.
以致我們甘心樂意自願.
去活出愛的照明.
比權勢更強大.
更有力量的是愛.
權勢可以改變外在的我們.
它可以動我們的行為.
但改變不了內在的我們.
和我們的信念.
但愛可以.
當我們活出愛的照明.
我們即使看起來好像動不了羅馬帝國.
這個文化路不可破.
合情合理合法的秩序.
我們動不了.
但我們也可以靠上帝給我們的愛.
走我們要走的新秩序.
我選了《逃詔》這首詩歌.
作為回應詩歌.
詩歌一開始的歌詞是這樣說的.
我讀出來.
「竹林深處」.
還是叫Alex現在唱出來?.

$^{721}$(笑聲).
「竹林深處」.
「我視著隱於荒野」.
(笑聲).
他們真的在唱.
「山河冰川」.
「我更是渺小」.
「星環不抖」.
「哪會是我可擦透」.
「生還可以便已足夠」.
剛才我讀得好像很有趣很開心.
但你再細心看這幾句歌詞.
其實很無奈的.
很無力的.
香港近幾年的氣氛都是這樣.
做什麼都沒用.
我們試過激進.
我們試過什麼都不做.
都看不到任何改變.
世界很大.
我們很小.
政局時勢,疫情.
什麼都測不透.
我們累了.
我們的心很累.
慢慢我們覺得.
好好活著.
其實就當贏了.
香港這幾年的氣氛.
由一邊的鐘擺.
一擺擺到另一邊.
我們好像只看見世界的改變.
越來越差,越來越壞.
但似乎我們不知道怎樣回應.
甚至我們都沒力氣去回應.
但我們一班信耶穌的群體.
我們力量真的這麼弱嗎?.
我經常都想.
我們是一班信耶穌的群體.
我們力量真的這麼弱嗎?.

$^{761}$我不甘心.
耶穌明明說我們是光.
黑暗不能勝過我們.
耶穌明明說我們是炎.
我們要成為社會的良心.
耶穌明明說我們是祂的兒女.
我們擁有聖靈的大能力.
我們難道真的什麼都做不到嗎?.
《斐理門書》提醒我們.
原來即使大國有多大.
原來即使制度有多爛.
環境有多爛.
世界有多黑.
我們都要做我們要做的事.
我們都要活出愛這個召命.
當我們每一個人都回應這個召命的時候.
羅馬帝國就算再強大.
奴隸制度即使有多穩固.
都會因為愛而逐步瓦解.
保羅沒有叫我們要推翻奴隸制度.
保羅也沒有要起什麼革命去挑戰羅馬帝國.
保羅的著眼點根本不是要對付.
或者不是要改變這個奴隸制度.
整本書保羅沒有一個字眼是關於.
要釋放,自主等等.
這些和解放奴隸有關的字眼.
或者這麼說.
保羅最關注的從來都不是有沒有奴隸制度.
保羅不著眼於世界的秩序.
因為他知道這個秩序對他來說.
沒有任何的權威性.
相反保羅只著眼於我們有沒有堅守.
我們要走的秩序.
我們有沒有好好去活這個愛的召命.
以前我們活出愛的召命的時候.
我們是被人知道.
我們不屈服於他們的價值.
原來我們這班人要改變.
我們所踏之地.
我們身處的地.

$^{801}$不是要看環境.
不是要看制度.
而是我們要看好自己.
我們有沒有當我們的秩序.
比其他的秩序更大權威.
我們有沒有堅持在黑暗的世代.
仍然要走我們自己的秩序.
當我們能夠走出愛的召命.
我們身處的土地就會改變.
我們相不相信愛有這種力量.
我們相不相信我們擁有這種.
可以改變我們.
我們這個地方的力量.
我們相不相信愛有力量.
弟兄姊妹.
我們相不相信我們擁有這種.
可以改變我們身處土地的力量.
如果你相信的話.
我們要一起做.
然後我們一起放長雙眼.
可以看看我們這個土地.
有什麼改變.
我們一起禱告.
愛我們的主.
愛全世界的主.
我們向你祈求.
求你幫助我們.
可能我們軟弱.
可能我們無力.
可能我們疲倦.
但主你說我們是你的兒女.
我們擁有聖靈的大能力.
我們也擁有從你而來的愛.
當我們今天說.
我們相信愛有力量的時候.
求你同樣給我們活出.
你給我們愛的召命.
以致我們這個群體.
視你的秩序比其他的秩序更有權威性.
更有強大的時候.

$^{841}$我們每一個人都能夠愛.
能夠去經歷你的愛.
也同時能夠去愛身邊的人.
愛這個土地.
愛這個世界.
求你幫助我們.
給我們力量.
看見我們自己有沒有堅持.
去行這個愛的召命.
求你賜福我們.
賜福我們的土地.
我們這樣祈禱.
奉耶穌聖名祈求.
阿門.
\newpage



\section{以弗所書 1:15-23-20221001}
\label{sec:1O4Wz5DFm4k}
\textbf{【網上崇拜】教會的形狀|以弗所書1\_15-23|20221001 [1O4Wz5DFm4k]}
\newline
\newline
連結: \href{https://youtube.com/watch?v=1O4Wz5DFm4k}{\texttt{ https://youtube.com/watch?v=1O4Wz5DFm4k}} ~~~~ 語音日期: 2022-10-01 
\newline
\newline
\hyperref[sec:PqPiG_MpRK4]{\small{< < < PREV SERMON < < <}}
~
\hyperref[sec:index_chronic]{\small{[返順時目]}}
~
\hyperref[sec:index_scriptual]{\small{[返順卷目]}}
~
\hyperref[sec:Zv7Jalkm4FA]{\small{> > > NEXT SERMON > > >}}
\newline
\newline
以弗所書 1:15-23-20221001
\newline
\begin{longtable}{cl}
\hline
\hline
章節 & 經文 (和合本修訂版)\\
\hline
1:15 & \begin{tabularx}{0.7\textwidth}{X} 因此,我既然聽見你們對主耶穌有信心,對眾聖徒有愛心, \end{tabularx} \\ \\ \relax
1:16 & \begin{tabularx}{0.7\textwidth}{X} 就不住地為你們感謝神,禱告的時候常常提到你們, \end{tabularx} \\ \\ \relax
1:17 & \begin{tabularx}{0.7\textwidth}{X} 求我們主耶穌基督的神,榮耀的父,把那賜人智慧和啟示的靈賜給你們,使你們真正認識他, \end{tabularx} \\ \\ \relax
1:18 & \begin{tabularx}{0.7\textwidth}{X} 照亮你們心中的眼睛,使你們知道他呼召你們來得的指望是甚麼,他在聖徒中所得榮耀的基業是何等豐盛, \end{tabularx} \\ \\ \relax
1:19 & \begin{tabularx}{0.7\textwidth}{X} 並知道他向我們這些信的人所顯的能力是何等浩大,這是照他的大能大力運行的。 \end{tabularx} \\ \\ \relax
1:20 & \begin{tabularx}{0.7\textwidth}{X} 這大能曾運行在基督身上,使他從死人中復活,又使他在天上坐在自己的右邊, \end{tabularx} \\ \\ \relax
1:21 & \begin{tabularx}{0.7\textwidth}{X} 遠超越一切執政的、掌權的、有權能的、統治的和一切有名號的;不但是今世的,連來世的也都超越了。 \end{tabularx} \\ \\ \relax
1:22 & \begin{tabularx}{0.7\textwidth}{X} 神使萬有服在他的腳下,又使他為了教會作萬有之首; \end{tabularx} \\ \\ \relax
1:23 & \begin{tabularx}{0.7\textwidth}{X} 教會是他的身體,是那充滿萬有者所充滿的。 \end{tabularx} \\ \\
[1ex]
\hline
\hline
\end{longtable}
$^{1}$頂姐妹晚安.
在世界各地不同地方的頂姐妹晚安.
很開心今天能夠見到頂姐妹.
我們一起去敬拜.
很欣賞敬拜隊剛才的燈光.
我們很熱愛.
我們很用心去準備.
我們現在去聽上帝的說話.
轉眼間我們來到十月.
秋天了.
大家穿了衣服.
十月我們仍然會用迎新的主題.
來思想我們一些信仰的課題.
特別是這次和大家去想的.
就是有關對教會的意義.
上次我們說的是上帝的美學.
探討到美,外貌和型之間的關係.
我看到其他講員都講得多的是新.
所以這次我繼續講型.
繼續型多一點.
我們一起去思考型這個課題.
上星期我做了一個夢.
有點奇怪.
但是一個很深刻的夢.
我做夢我自己開小巴.
我沒有開過小巴.
我從來沒有想過自己做夢會開小巴.
開一輛綠色的小巴.
一輛棍波車的小巴.
後來我醒來後Google了一下.
原來小巴真的是棍波車.
是真的.
我懂得開棍波車.
我在德國的時候開車也是開棍波車.
但不知為何那晚我做夢.
我把小巴開到路口的時候.
突然停了.
然後就死火了.
於是我就重新開車.
把油門調到最低.

$^{41}$就想拉棍波的波箱.
不知為何小巴的棍波箱是很複雜的.
複雜到很難用.
我不懂得用.
我弄來弄去都不懂得怎樣弄棍波.
於是小巴就停在路中心.
很尷尬.
我感覺到全世界馬路的人都看著我.
路過的旁人和司機.
後面的車都看著我.
怎樣把車拖著.
當然小巴裡面的人.
每個人都看著我.
他們就等我.
等了幾分鐘.
不知道是多久.
開始不耐煩.
人們就開始下斜.
不過就突然有一群人.
在小巴裡面.
好像不趕時間.
就坐在那裡.
開始聊天.
聊著聊著就開始查經.
就這樣.
我就坐在那邊一起查.
一看窗口就知道到了.
就是今天的夢.
有人可以幫我解解夢嗎.
今天的講道.
或多或少都和夢有關.
雖然我也不知道有什麼關係.
今天我們會講一篇教會的道.
今天的主題叫做教會的形狀.
一篇教導性比較強.
不是靈性的培靈.
也不是軟弱安慰.
今天我們不是講個人的需要.
我們一起來思考.
我們群體.

$^{81}$我們作為教會.
作為全教會.
我們的使命和責任.
我們一起去祈禱.
主你幫助我們.
讓我們今天一起聚集.
我們在教會裡實體.
我們一起參與.
這個直播崇拜的頂展會.
或者在不同地方.
我們在網上一起敬拜的頂展會.
我們一起去聆聽你的話.
求主你的意象.
你的使命.
你自己的心意.
能夠可以藉著你自己的聖道當中.
親自個別的向我們顯明.
太子仍然是軟弱不賠.
但你幫助.
讓我們同感一靈.
一起去聽你的話.
一起去思考你的話.
奉主命求.
阿們.
好,我們開始了.
今天要講的是保羅.
在《以弗所書》的一段經文.
第一章,一段保羅的禱告.
今天我們只不過是講整段禱告的最後一句.
不過我們也讀了整段禱文.
《以弗所書》第一章,15到23節的經文.
讓我來讀.
保羅這樣來祈禱.
因此,我既聽見你們信從主耶穌.
禱告的時候,常提到你們.
求我們主耶穌基督的上帝.
榮耀的父.
將那賜人智慧和啟示的靈賜賞給你們.
使你們真知道他.
並且照明你們心中的眼睛.

$^{121}$使你們知道他的恩兆有何等指望.
他在聖徒中的基業有何等豐盛的榮耀.
並且知道他向我們所信的人.
所獻的能力是何等浩大.
就使照他在基督身上所運能的大能大力.
使他從死裡復活.
叫他在天上在在自己的右邊.
遠超過一切執政的,掌權的,有能的,主治的和一切有名的.
不但是今世的,年來世的也超過了.
也將萬日伏在他的腳下.
使他為教會作萬有之首.
教會是他的身體.
是那充滿萬有者所充滿的.
這個看似是一個平平凡凡為弟兄姊妹的祈禱.
你發覺越奇就越深奧.
一段非常之深奧難明.
很神學的一個禱告.
我們就一段時間比較技術性的去做一些釋經.
簡單來說,保羅為弟兄姊妹祈禱.
祈禱弟兄姊妹能夠真知道耶穌基督的父神.
知道他的恩兆,知道他的豐盛.
知道上帝的能力是何等的浩大.
然後保羅在禱告的後半段.
就要向弟兄姊妹解釋.
這位耶穌基督究竟有多大的浩大.
所以在20節,保羅說.
上帝召他在基督身上所運能的大能大力.
叫他從死裡復活,叫他在天上坐在自己的右邊.
遠超過一切執政的,掌權的,有能的,主持的和一切有名的.
不但是今世的,連來世的也超過了.
不知道為什麼喘氣讀這裡.
很厲害,這段經文.
簡單來說,耶穌基督很厲害.
耶穌世一就是這樣.
這就是保羅禱告,要說的事情.
不過最古怪的是土文最後那兩句.
保羅的土文最後突然間話題一轉向.
就去到教會這個主題.
223字,保羅說.
有將萬有伏在他的腳下,使他為教會作萬有之首.

$^{161}$教會是他的身體,充滿萬有之所,完全充滿的.
不知道大家在節目有沒有同感.
和我一樣,問同樣的問題.
為什麼整個禱告最後兩節.
會無端端無端無情情提到教會這兩個字.
如果你留心去讀聖經的話.
你會發現整段的土文基本上是沒有提到教會.
甚至整個第一章都沒有提到教會.
為什麼保羅會突然間要轉到教會的主題.
為什麼?.
去到最後兩節,保羅突然間好像一個突入式廣告一樣.
強行拉教會,拉到一起來說.
完全是有些強行來的.
好像你看林家謙演唱會一樣.
看得很開心,最後兩首歌突然間出現家謙姐.
跳兩首舞,就完場了.
不是不行的,但你會問為什麼.
你會問有什麼關係,林家謙和家謙姐.
是沒有什麼關係的.
所以同樣道理,耶穌基督很厲害很厲害.
世界萬有都在他的腳下.
和教會這個主題有什麼關係呢.
我們看看第22節.
保羅說:又將萬有伏在他的腳下,使他為教會作萬有之首.
將萬有伏在他的腳下,我明白的.
基督耶穌任何東西都在他的腳下.
基督是萬有之首,我都明白的.
總之,經文就說耶穌很厲害.
天上地下一切的都在他的腳下.
他是一切的元首,耶穌很厲害,世一.
我們不明白的是為什麼這句話要加插一個「為教會」這三個字.
當你讀的時候,你未必很明白.
什麼叫基督為教會作萬有之首.
原文就更加模糊.
如果你是壽學生,或者你懂得原文的話.
這是一個dative,插進去的字.
你可以解作for the church,或者to the church.
其實你不太明白什麼意思.
為什麼基督為教會作萬有之首呢.
就是一個很奇怪的情況.

$^{201}$23節都是一樣.
保羅說:教會是他的身體,是那充滿萬有者,完全充滿的.
教會是他的身體,我們明白的,我們聽過的.
耶穌說充滿萬有,我們都明白的.
但是教會是那充滿萬有者所充滿的呢.
我們又不明白.
其實你不明白的,你應該不明白的.
這麼多年聽了這麼多經文,你都不明白這段經文是什麼意思.
所以要明白這段經文,我們又要做technical一點.
要去集中去明白經文裡面所說的充滿.
就是philoma這個字的意思.
以下真的很technical.
首先,這句經文裡面出現了兩次充滿這個字.
一次是名詞,就是philoma.
一次是分詞,就是palisable.
一個是philomenon這個字.
Anyway,整句是什麼意思呢.
就是叫做the church is fullness of Christ, that filling everything.
就是the church is fullness of Christ, that filling everything.
所以教會是基督的充滿,充滿一切.
大概是什麼意思呢.
整件事很古怪的地方就是充滿這個字.
可以是一個active的意思,也可以是一個passive的意思.
就是說the church is fullness of Christ.
可以解作基督充滿教會.
也可以解作基督被教會充滿.
OK,就是被教會圓滿的意思.
所以你可以選擇基督充滿著教會,fullness of Christ.
也可以是基督被教會所圓滿的意思.
兩個的方法都是文化上OK的.
第二個很值得注意的地方是後面那句說話.
feeling everything.
這句說話又可以指向著教會來說.
又可以指向著基督來說.
就是說基督充滿一切.
也可以是教會充滿一切.
總而言之,結合這兩個variable的話.
得出四個可能性.
我們看看下一個.
四個可能性.

$^{241}$這句經文23節有四個可以這樣解釋的.
第一是基督充滿教會,基督也充滿萬有.
這個大概是和本的意思.
基督充滿一切,也充滿著教會.
第二就是基督充滿著教會,從而叫教會充滿萬有.
第三就是教會圓滿了基督,complete了基督的工作.
很叫基督能夠充滿萬有.
第四,教會圓滿了基督,而且充滿著萬有.
我知道大家現在是很混亂.
我們不是要處理這些問題.
我們今天也處理不了這些問題.
我要說的是,如果你知道這段經文的複雜性.
你就會明白為什麼保羅在這個討論裡面.
說到基督很厲害的時候.
為什麼突然間要黏著教會一起說.
基督,耶穌和世界的關係是離不開教會和世界的關係.
上帝將萬有伏在基督的腳下.
使祂為教會作萬有之首這件事.
是你不能夠離開教會和萬有.
無論你哪個的解讀也好.
整個的解釋都是說教會和世界上的一切關係的萬有是不能夠分割的.
正因為基督是萬有之首,基督是充滿萬有.
教會也無可避免地充滿著萬有之中.
舉一個單純的例子,可能不是很準確,但也起碼單純一點.
就像吃火鍋一樣,你一放入芫茜,整鍋火鍋都是芫茜味.
一放入芫茜之後,整鍋火鍋都是充滿著芫茜味.
你分不清的,你可以不吃羊,我就不放羊.
你不吃響鈴,就不要放響鈴.
但一放入芫茜,你吃和不吃都是充滿著芫茜味.
大概這樣的比喻,其實是不對的比喻.
基督不是芫茜,教會也不是芫茜,我想說大概這樣的意思.
我明白教會是基督的身體是什麼意思.
更加明白,教會是基督的身體是什麼意思.
基督是頭,基督是頭,教會是身體.
所以基督是最高的,他是領導者,我們以前是這樣理解這句話.
教會是基督的身體,這句話是說什麼.
基督在哪裡,教會就在哪裡.
頭在哪裡,身體就自然分不清,就在那裡.
如果頭在這裡,身體不在那裡,就恐怖了.
一個沒有頭的身體是什麼,是基督的屍體.

$^{281}$在河邊棄置的屍體,無頭的躺在那裡.
所以基督在哪裡,教會作為基督的身體,都在萬有的裡面.
所以基督在哪裡,基督在萬有之中.
基督是萬有的元首,基督是一切都是復合的腳下.
所以教會也要留在整個當中.
這就是我們今天嘗試去解經的部分.
說了這麼久,整段經文對我們有什麼意見呢.
我們今天講題叫做教會的形狀.
我們問究竟教會的形狀是什麼.
前陣子神樂院有一個宣傳,我負責課程宣傳.
找一些人設計了一張桌布.
這個課程是關於教會更新和重整之類的.
所以桌布的設計需要教會一點.
貼文來了第一稿,我覺得不行,你弄一棵樹給我.
然後我就找人說,可不可以教會一點.
然後就說不行,沒戲了,不好意思.
然後貼文第二天就傳回來了.
他就加了這張給我,下一張,麻煩你.
令整件事就教會一點.
有沒有啊?.
應該有吧,就是這樣.
貼文就加了這張給我,就教會了整件事.
幸好不是Fourshot這樣搞,不然我們會不滿.
所以就是這樣,我明白設計師的用意.
想教會一點,就弄幾個標誌給你.
教會就教會一點.
但問題是,什麼時候見過香港會這樣.
香港教會不是這樣吧.
香港教會從來都沒有一間屋和十字架在上面.
所以我們問究竟怎樣才叫做教會.
教會怎樣才叫做持教會呢.
經常聽人說教會不像樣.
究竟教會怎樣才叫像樣.
怎樣才叫持教會.
教會怎樣才叫持教會.
教會的外表是怎樣.
教會要是什麼樣.
教會要怎樣才叫教會.
教會是不是一定要有程序表.
是不是要星期三晚上在祈禱會.

$^{321}$是不是要站起來敬拜.
團契是不是一定要背團歌團訓.
是不是一定要玩一下ice breaking game.
崇拜後是不是一定要有一個儀式牧師人祝福的東西.
真的,有些人是很重視祝福的.
沒有祝福就不安樂.
想著多忙都要回來教會.
崇拜唱歌,講道都要來祝福.
被祝福就完事了.
所以這樣才叫做持樣.
所以什麼叫做持樣.
什麼叫做持教會的模樣.
正正是我們來問的問題.
我不知道教會怎樣才叫持什麼.
我們知道教會是什麼.
教會是基督的身輔.
教會是基督的身體.
教會是萬有者的充滿.
所以與其問教會持什麼.
我們更加要在乎的就是教會是什麼.
我們用潘Sir的金句.
持就不是,是就不持.
我們不需要持教會.
我們需要是一間教會.
我不需要持牧者.
我要確定我是一個上帝呼召的牧者.
你也是一樣,你不需要持基督徒.
但你要100\%肯定.
你是一個跟隨耶穌基督的基督徒.
見證著基督的基督徒.
教會不被定型.
教會沒有任何固定的形狀.
教會是任何形狀.
教會是萬有之中.
用任何形狀去見證耶穌基督.
因為基督是萬有之首.
一切都是祂的腳下.
教會作為基督的身體.
就用這個方式來存在世界裡.
想想嚴世.

$^{361}$教會散落在世界裡.
沒有任何固定的模樣.
充滿萬有之中.
二十年前,香港教會很流行去講無牆教會.
言下之意就是一個沒有牆,沒有隔膜.
那邊的人可以來到當中的一個概念.
不過,聖經的想法似乎比這個更加激進.
更加緊貼教會所示.
教會根本連任何固定形狀都沒有.
教會可以這樣,教會可以那樣.
教會充滿萬有之中.
教會要不斷成熟.
到一個地步,它能擺脫任何外表宗教的形狀.
撇去任何宗教的約束.
這個不是教會被約化,世俗化.
而是被基督的真道昇華成為世上的炎.
世上的光散落在世間裡.
這幾年我作為香港教會前線運作教會的人.
一個很深的體會.
教會越不像教會.
你就正正是在做真正的教會.
用回毛柬道,傻強那句話.
當你看見一個人在做一件事.
很不專心地看著,他就是警察.
我懷疑幾年後將來.
我們都越來越經歷這樣的道理.
樣子越不像教會的地方.
裡面就更加多的基督徒.
一群說得很少屬靈的人.
他們就越能夠真心真正地對人說話.
一個地方越不像宗教處所.
那裡就是人遇見上帝的地方.
你問我,在YouTube那裡.
幾百萬個頻道裡面搞一個網上崇拜.
整件事是不是很大不道呢.
我會說,基督耶穌在萬有之中.
教會作為基督的身體.
教會就是這些雜雜的地方.
需要存在下去.
潘福華在《六宗書鑒》裡面提到一個概念.

$^{401}$叫做非宗教的基督教.
潘福華說,用語言來解說宗教.
無論是從神學或純粹出乎見前的觀點來說.
這個時代都過去了.
內心和良心的時代.
就是宗教的時代都過去了.
我們正正走向一個完全沒有宗教的時代.
很有意思的.
人世間的宗教只能夠寄居在宗教處所當中.
教會卻充滿在萬有的裡面.
人世間的力量可以摧毀有形的宗教.
卻不能消滅沒有固定形狀的教會.
哪裡有基督哪裡就有教會.
Ubis Christus, EBClesium.
很重要的神學說話.
哪裡有基督,那裡就是教會的存在.
教會要化為灰塵.
灰塵散播在世界每一個角落.
在絕望之處.
教會自身都感到暗淡之處.
在微不足道的塵埃中.
教會發現,正在見證基督在萬有的裡面.
之後這一部分我是沒有寫下去的.
我特意不寫得太詳細.
因為想跟大家真心說一個應用的問題.
就是留堂應該如何存在.
是我這幾乎每天都在想.
尋問上帝的問題.
所以幾個很零碎的片段跟大家分享.
我覺得這一年在留堂裡是很不容易的.
是很複雜的一年.
疫情,第五波,又針指,又轉場.
基本上教會除了少數之外.
基本上每個星期要做的.
只是維持每個星期的崇拜.
你們是很好的,去把關.
我心裡的血,去呈現最好的.
讓我能夠維持去敬拜上帝.
每個星期,全世界很多弟兄姊妹.
參與我們的崇拜.

$^{441}$很多人能夠在敬拜當中去遇見神.
這是我非常之值得驕傲的事情.
不過縱然留堂每個星期都是獻上最好的敬拜.
我們都能夠去參與這個敬拜.
但我不想留堂只是每個星期維持這樣的崇拜.
留堂的存在,留堂作為教會的存在.
不單單只是做崇拜.
也不只是做牧養.
如果我們每個星期只是不斷崇拜.
一個好的崇拜.
教會就不應該有教會的功能.
教會需要充滿萬有的見證.
耶穌基督是救主.
這是我們作為香港教會很重要的課題.
上個星期我和一群香港教會的有心人.
一起去聚會.
一群仍然在香港的牧者.
不打算離開的牧者.
大家聚在一起.
彼此在分享這段時間在做什麼.
大家有一個很大的共鳴.
就是我不灰心.
我不會覺得很灰心.
面對香港教會壓倒性的走下坡.
我們很不容易.
我們在想如何繼續生存.
如何做一些有意思的事情.
相反我在暑假的時候去英國.
英國是另一個圖畫.
稱之為教會開箱.
很多新的教會.
丙伯林有一間.
有些牧者是專做開教會的工作.
開完就走.
十個黃色執事聚在一起開執事會.
然後就開始教會.
有地方有程序表.
執事會就開始崇拜.
我們可以做很多很像教會的事情.
但要問我們究竟怎樣才算真正的教會.

$^{481}$昨天,今天片段很零碎.
昨天和一班神學生一起去上教會歷史.
去教會歷史.
我們一起唱一首詩歌.
叫做《教導的根基》.
講到早期教會被迫拜的情況.
這班神學生是很特別的.
基本上這三位都是2020年之後讀神學的人.
講法之後他們仍然選擇入長洲讀神學的人.
已經是第三屆.
這班人是天然地樂觀.
不可以說天然地樂觀.
起碼他們是真的知道發生什麼事.
不離開香港.
他們選擇在教會裡開始他們的侍奉.
這班人正是令我很大的激勵.
因為他們將來正是開始延續教會應該做的事情.
我想說什麼呢?.
今天我們讀完保羅這段經文.
我們很體會到教會的神聖.
教會是一些很神聖的東西.
天上地下一切的權柄都在基督耶穌的腳下.
這件事是教會都在一起.
因此成為一個很重要的上帝的事情.
所以如果你真正明白什麼是教會的時候.
你就不應該那麼悔.
教會可以成為任何的形狀出現.
這2000年教會有很多不同的形狀出現過.
在羅馬帝國被迫北的時間.
在中國教會隱藏的時間.
在輝煌的時間.
很多不同形式的教會存在.
很多不同款式的教會存在.
背後仍然是一個上帝所充滿的教會.
充滿在世界裡.
流唐是什麼形狀呢?.
流唐是一個很特別的形狀.
流唐本尊坐了一些人.
我們在網上在世界各地有幾千人.
我們小組有幾百人.

$^{521}$我們網上加起來就成為了流唐的形狀.
我們都說我們拍不到一些教會堂慶的細菌照片.
我們一輩子都沒有.
一輩子都拍不到這樣的照片.
流唐的形狀正正是在服侍這個世代的形狀.
所以我自己很但願流唐有三件事能夠實踐.
能夠見到這樣的存在.
流唐可以這樣的存在方式.
三個很特別的方針.
第一流唐應該是懷著盼望快樂的存在.
希望弟子妹能夠作為流唐.
弟子妹作為基督徒.
能夠快樂有盼望的存在.
第二流唐希望弟子妹能夠在崇拜裡遇見上帝.
能夠每個星期真真正正讓人遇見上帝的地方.
頭兩個我覺得我們都在做的.
第三個我覺得我們應該做多一點.
流唐應該滲進香港社會的角落裡.
充滿萬有之眾.
讓不同的人能夠見證基督耶穌.
我們應該滲入多一點這個世界裡.
讓大學生能夠信耶穌.
讓一班比我們更加年輕的年輕人.
能夠知道基督耶穌的盼望.
帶他們信主.
讓紅土區讓香港一些弱勢的社群.
真正能夠得到幫助.
這個是我們流唐應該要做的事情.
讓一些五歲以上的小朋友.
回不了教會的小朋友.
能夠回到教會.
這個都是我們可以做的事情.
我們有很多可以做.
我們要滲入香港這個地方裡.
讓不同的人真真正正見證到耶穌基督.
這個是教會的功能.
這個都是我們流唐應該有的形狀.
請問你們願不願意一起去參與呢.
你願不願意一起去跟我們成為不同的形狀.
在香港裡面.

$^{561}$在世界裡面.
讓人可以這樣得到祝福.
我們一起祈禱吧.
有段時間我們一起尋求上帝.
因為你讓我們這間教會有一個很特別的身份.
你讓我們在不同的地方裡面.
都能夠連成一群.
我們在倫敦.
我們在曼城.
在悉尼.
在溫哥華.
有不同的群體我們聚集.
我們求主你這樣來兼顧我們.
在網上更加有一群弟兄姊妹.
他們未方便來到實體當中.
但卻仍然來敬拜你.
主你為我們流唐.
我們去獻上我們的禱告.
求主你讓我們能夠知道你的心意.
成為一個有盼望的群體.
一個在崇拜裡面遇見你的群體.
一個在萬有當中.
去滲入去成為灰塵.
成為流動的水一樣.
我們不會被切斷.
因為我們沒有任何固定的形狀.
求神你這樣來幫助我們.
讓我們更加明白.
你讓我們走下去的心意.
讓我們知道我們怎樣來成為一個.
在這個年代裡面.
一間真正的香港教會.
求主你這樣來幫助我們.
逢尊名求.
阿門.
\newpage



\section{以弗所書 4:17-24-20221008}
\label{sec:Zv7Jalkm4FA}
\textbf{【網上崇拜】新.型人|以弗所書4\_17-24|20221008 [Zv7Jalkm4FA]}
\newline
\newline
連結: \href{https://youtube.com/watch?v=Zv7Jalkm4FA}{\texttt{ https://youtube.com/watch?v=Zv7Jalkm4FA}} ~~~~ 語音日期: 2022-10-08 
\newline
\newline
\hyperref[sec:1O4Wz5DFm4k]{\small{< < < PREV SERMON < < <}}
~
\hyperref[sec:index_chronic]{\small{[返順時目]}}
~
\hyperref[sec:index_scriptual]{\small{[返順卷目]}}
~
\hyperref[sec:68fGAExhs0o]{\small{> > > NEXT SERMON > > >}}
\newline
\newline
以弗所書 4:17-24-20221008
\newline
\begin{longtable}{cl}
\hline
\hline
章節 & 經文 (和合本修訂版)\\
\hline
4:17 & \begin{tabularx}{0.7\textwidth}{X} 所以我這樣說,且在主裡鄭重地說,你們行事為人,不要再像外邦人存虛妄的心而活。 \end{tabularx} \\ \\ \relax
4:18 & \begin{tabularx}{0.7\textwidth}{X} 他們心地昏昧,因自己無知,心裡剛硬而與神所賜的生命隔絕了。 \end{tabularx} \\ \\ \relax
4:19 & \begin{tabularx}{0.7\textwidth}{X} 既然他們已經麻木,就放縱情慾,貪婪地行種種污穢的事。 \end{tabularx} \\ \\ \relax
4:20 & \begin{tabularx}{0.7\textwidth}{X} 但你們從基督學的不是這樣。 \end{tabularx} \\ \\ \relax
4:21 & \begin{tabularx}{0.7\textwidth}{X} 如果你們聽過他的道,領了他的教,因為真理就在耶穌裡, \end{tabularx} \\ \\ \relax
4:22 & \begin{tabularx}{0.7\textwidth}{X} 你們要脫去從前的行為,脫去舊我;這舊我是因私慾的迷惑而漸漸敗壞的。 \end{tabularx} \\ \\ \relax
4:23 & \begin{tabularx}{0.7\textwidth}{X} 你們要把自己的心志更新, \end{tabularx} \\ \\ \relax
4:24 & \begin{tabularx}{0.7\textwidth}{X} 並且穿上新我;這新我是照著神的形像造的,有從真理來的公義和聖潔。 \end{tabularx} \\ \\ \relax
4:25 & \begin{tabularx}{0.7\textwidth}{X} 所以,你們要棄絕謊言,每個人要與鄰舍說誠實話,因為我們是互為肢體。 \end{tabularx} \\ \\ \relax
4:26 & \begin{tabularx}{0.7\textwidth}{X} 即使生氣也不要犯罪;不可含怒到日落, \end{tabularx} \\ \\ \relax
4:27 & \begin{tabularx}{0.7\textwidth}{X} 不可給魔鬼留地步。 \end{tabularx} \\ \\ \relax
4:28 & \begin{tabularx}{0.7\textwidth}{X} 偷竊的,不要再偷;總要勤勞,親手做正當的事,這樣才可以把自己有的,分給有缺乏的人。 \end{tabularx} \\ \\ \relax
4:29 & \begin{tabularx}{0.7\textwidth}{X} 一句壞話也不可出口,只要隨著需要說造就人的好話,讓聽見的人得益處。 \end{tabularx} \\ \\ \relax
4:30 & \begin{tabularx}{0.7\textwidth}{X} 不要使神的聖靈擔憂,你們原是受了他的印記,等候得救贖的日子來到。 \end{tabularx} \\ \\ \relax
4:31 & \begin{tabularx}{0.7\textwidth}{X} 一切苦毒、憤怒、惱恨、嚷鬧、毀謗,和一切的惡毒都要從你們中間除掉。 \end{tabularx} \\ \\ \relax
4:32 & \begin{tabularx}{0.7\textwidth}{X} 要仁慈相待,存憐憫的心,彼此饒恕,正如神在基督裡饒恕了你們一樣。 \end{tabularx} \\ \\
[1ex]
\hline
\hline
\end{longtable}
$^{1}$頂姐妹平安.
在今天預備講章的時候.
想了一段時間.
就是如何在這個題目中.
讓頂姐妹有一個感受.
其實都想了一個星期.
講題是我自己定的.
但覺得通常有點捉蟲.
定完之後.
新通常都能夠有一個客觀感受.
都不難理解或看到.
但型這個字就真是可圈可點.
特別型如果用廣東話.
就看用什麼語氣說.
你好型喔.
這些就感受到是什麼.
但這次你是型.
這句說話的感受就有點不同.
型這個字就真是.
如果你有認識其他語文的人.
我記得我第一次參與台灣短宣的時候.
我們香港人當然說你好型.
或者讚一個人的時候.
台灣人聽不懂什麼叫你好型.
因為他沒有這個形容詞去形容別人型.
所以型在現在的環境.
特別你可能都理解過.
當那件事你不尷尬.
就是別人尷尬的環境下.
都可以很型.
這樣就很難說了.
作為一個講員.
或者作為一個要在與你表達過程當中.
有些東西沒有一個標準.
或者有些東西沒有一個參照的時候.
你很難有一個共享的空間.
讓別人去理解的時候.
就很容易做到遲不達意.
但當我們有不同的閱歷.
當我們有不同的一起經歷的時候.

$^{41}$你會發覺有些東西慢慢變大了.
不一定要以往的方法.
就是看外觀,看衣著.
看一些所謂的周邊的文化去定義.
反而型可能是這些東西全部都拿走了.
可能是一些說話.
可能是一些行動.
可能是一些堅持.
可能是一些你不容易去具體.
去用一些所謂實物或者是一些形態上面去劃分.
但是你從中感受到那件事是很型的.
講完都好像不知道講什麼.
但我想你慢慢會去感受到.
其實型是你人生閱歷當中.
你會看到那件事你覺得值得欣賞的地方.
這個我相信你就會明白我說什麼.
在我預備這個講題的時候.
這段經文我上個月都想好了.
但上個月我不用這段經文.
因為上個月到我講的那一周就是中秋節.
跟大家在網上和現場一起過節.
我上個月的閱題講的經文就是講希伯來書.
就是講一個中秋節.
講月事故鄉的那個訊息.
我就留下這個月才講這段經文.
為什麼這個月才講呢?.
原因也是剛剛過了星期二.
是香港的重陽節公眾假期.
我不知道你們去了哪裡.
但是我們很多flowchurch的小組和弟兄姊妹.
就去了麻灣參加了flowchurch的第三屆水禮.
那天是水禮的日子.
就是一班弟兄姊妹.
他願意在這個環境當中接受水禮.
去開展他基督徒新的一頁.
立志跟從上帝.
這件事就型了.
對於我來說.
我希望在這個訊息當中.
在洗禮幾天的弟兄姊妹當中.

$^{81}$讓他們感受到一個新的迎人.
上帝的說話怎樣去幫助我們.
當日的敬拜隊就由小組的弟兄姊妹去組成.
就成為那班水禮者的一份禮物.
敬拜隊的主令都選了.
當中關於一的經文.
一信一信一洗一上帝那段經文.
而John上個星期說關於教會的形狀的時候.
都是用已忽所書的經文.
無獨有偶.
我今天說的都是已忽所書的經文.
我希望大家都感受到.
在過去你查經或者理解經文的時候.
已忽所書被稱為教會的藍圖.
正正就是關於教會裡面的旅外的事情.
都是在已忽所書當中很詳細地表明.
我今天集中在已忽所書第四章裡面.
一個新類的迎人.
在當中怎樣去被上帝的說話再一次提醒和更新.
這個就是今天想說的內容.
在經文當中第四章第十七節.
第四章裡面的全章.
就是剛才我說過一信一主一上帝那段經文之外.
其實中間就說到大家很熟悉的.
就是他所賜的有仕途有先知有牧師和教師.
這個都是國安職份的教會內部的描述.
但去到當職份定了的時候.
保羅想提醒已忽所教會.
或者提醒我們作為信徒.
在教會裡面的內部.
其實我們這班新類型人.
我們應該怎樣去重拾上帝原先做人的美善呢?.
保羅很希望我們被上帝更新之後.
我們有再展現的地方是什麼?.
這本書第四章第十七節我選的經文.
跟大家一起讀出.
第十七節.
「所以我說,且在主裡確實地說,.
你們行事不要再障礙幫人傳虛妄的心行事,.
他們的心地昏昧,與上帝所賜的生命隔絕了,.

$^{121}$都因自己無知心裡剛硬,.
良心既然喪盡,就放盡私慾,.
貪行種種的污穢..
你們學了基督,卻不是這樣,.
如果你們聽過他的道,領了他的教,.
學了他的真理,.
就要脫去你們從前行為上的舊人,.
這舊人是因私慾的迷惑漸漸變壞的,.
又要將你的心智改換一新,.
並且穿上新人,.
這新人是照著上帝的形象做的,.
有真理的人義和聖潔..
抱歉,可能說話有少少slur,.
因為這兩天舌頭不太聽話,.
不過我相信都清楚的..
這段經文很大段,.
這段經文頭十七至十九節,.
講舊有你以往未信主前的行為和表達方式,.
今天我想講的是洗禮的弟兄姊妹,.
或者你已經洗禮了一段很長的時間,.
其實對你的生命的改變是什麼呢?.
我們回到教會這麼久,對我們來說是什麼一回事呢?.
所以我集中跟大家看幾節,.
由二十節開始看,.
所以下一張的投影,.
你會看到第二十節就是,.
想講你們,保羅說你們學了基督,.
就不應該是這樣,.
他不是這樣,就不應該是這樣..
第二十節,如果,.
這個字,如果這個字,.
if indeed,英文的翻譯,.
其實聖經很少用這樣說,.
其實聖經很少說如果,.
聖經通常都很直率,.
很肯定說一些上帝要我們知道描述性的內容,.
但是保羅說如果的時候,.
就是如果你們聽過他的道,.
領過他的教,.
學了他的真理,.

$^{161}$有三樣東西就是,.
你們信了耶穌這麼久,.
又洗了禮,.
你聽過他很多的教導,.
但是你聽了之後,.
你又沒有接受這個教導呢?.
聽了是不是一定接受?.
你接受了這個教導,.
你學不學到東西呢?.
保羅是用一個提問方式,.
回教了這麼久,.
坐了崇拜這麼多次,.
你是不是真的坐在那裡?.
我經常說笑,.
又是有一點點的耶語,.
就是為什麼你想不到東西,.
是不是坐著了?.
我,我,我,我,我是囂張的人,.
那當然我是笑笑口,這樣,.
但是你坐崇拜坐了這麼多次,.
你聽過這麼多堂道,.
你有沒有領受那堂道對你的punchline,.
對你的要求,.
而學到東西,對生命有提升呢?.
這個就是保羅提醒的,.
如果,你既然坐了這麼多次,.
所以對於我自己教實踐神學的時候,.
實踐神學,我過程當中有一個叫學習的進程,.
叫知情義行,.
你知道那件事,.
那件事能不能觸動你的情感,.
以至你有意願去執行呢?.
不是知道就會懂得做,.
也不是知道就馬上會做,.
也不是知道就會想去做,.
你知道那件事要觸動你的情感,.
覺得,是的,無論如何都要去試一次,.
你有意願去執行,.
不是每個人都喜歡的,.
我教書的時候就喜歡用少年官的例子,.

$^{201}$少年官問耶穌良善的夫子,.
應該要做些什麼才能得到永生?.
耶穌就跟他數律法的要求,.
這些我從小就守住了,.
耶穌就跟他再禮真,再多要求,.
你就變賣你的所有,然後還要跟從我,.
但是那個少年人又怎樣?.
喂,變賣所有?.
不是吧,還要跟從你?.
你今晚住哪裡你都不知道,.
你經常跟罪人法律,罪人紀律一起,.
我做官的,.
你會發覺那個少年的官知道很多東西,.
但是他的情感,.
上帝是催逼耶穌去催逼他,.
你情不情願?.
但是那個少年的官不情願,.
他最後結果是什麼?.
他就悠悠愁愁走了,因為他錢很多,.
他沒有意願去執行跟隨耶穌,.
其實今天我們洗禮了,.
回教會這麼久了,.
對於你們來說,.
你們聽了這麼多堂道,.
有多少事情你們領了教訓,.
而學到東西呢?.
保羅就提醒我們,.
你們學了基督就不應該是這樣,.
所以要有些事做了,.
做什麼呢?.
就好像耶穌說的,.
與大相關的行動,.
科幻書裡面,.
你會看到耶穌的對話內容,.
很多時候都是一個說了就有要求,.
我們叫做speech act,.
耶穌每個speech都是要求跟從者有行動去回應的,.
我們說是基督徒跟隨耶穌的,.
其實上帝仍然是希望我們好像當日跟隨耶穌的人,.
聽了我的話就要這樣行罷,.

$^{241}$但是上帝沒有逼我們,.
上帝仍然等我們有個回應,.
直到今天都是在等的,.
所以第22節就是,.
如果你們聽過的話,.
就要脫去你們從前行為上的舊人,.
有些舊,.
什麼是舊呢?.
就是過去你還沒認識耶穌之前的生活,.
反過來問一問,.
其實你信了主和沒信主,.
你有什麼改變呢?.
哦,我改變就大了,.
於是就說很多話了,.
就好像我們剛剛剛水禮的弟兄姊妹,.
都要寫個洗禮見證的,.
哇,寫得很長,.
幾千字,.
當中要幫她修訂一下,.
其實具體一點,.
不需要由信主開始,.
但是當她要想的時候,.
她要想一下,.
其實上帝讓我信主都是很奇妙的事,.
我都要寫出來,.
真的,.
未信主前和信主後,.
你生命當中有什麼改變,.
那個對你來說,.
是應該要歷久常新的提醒,.
但同樣都是說一個現在,.
你聽了這麼多堂道,.
你的生命的改變是什麼呢?.
所以脫去從前上的舊人的時候,.
脫去這件事,.
即是要剝了,.
其實原本是穿著的,.
你穿了些什麼呢?.
知不知道自己一直穿著些什麼呢?.
保羅說要脫去從前舊人穿著的東西,.

$^{281}$其實你未信主之前,.
你穿著的是什麼呢?.
跟大家重溫一段,.
未有這段教導之先,.
穿著的是什麼呢?.
天起了涼風,.
耶華在園中行走,.
那人聽見耶華的聲音,.
就害怕躲藏起來,.
於是無花果樹的葉子編作裙子,.
遮蓋自己的身體,.
耶華說,.
哪個告訴你赤身裸體的?.
當中的聲音害怕,.
耶華說,.
我聽你的聲音害怕,.
就躲藏起來,.
遮蓋自己的身體,.
哪個告訴你赤身裸體的?.
然後事情就公開了,.
當初那個人以為吃了果子之後,.
眼睛的明亮,.
發覺自己原來看的東西不同了,.
人就用自己的方法去遮蓋自己的罪,.
人就用自己的方法去遮蓋自己覺得不是的東西,.
穿著舊的東西,.
從來都是掩蓋你自己的事情,.
信了主,.
你上帝更新之後,.
那個舊有的遮蓋方式就要除去了,.
今天你仍然是否用你過去的方式去生活呢?.
今天你仍然是否用過去的方式去遮蓋你自己的行事呢?.
今天你仍然是否用一個你覺得賴以為生,.
舒服的方法去經營你的生活呢?.
如果仍然是一樣的話,.
信主前和信主後,.
你要脫去的是什麼?.
你不用告訴我,.
我也不會全部知道,.
但最清楚的應該是你自己,.

$^{321}$脫去是指舊人,.
舊人是什麼呢?.
舊人是因私慾的迷惑,.
就是那人看見很漂亮,.
可以吃的,.
吃了很像上帝,.
因私慾的迷惑,.
他漸漸變壞了,.
對於我們來說,.
你有很多前因,.
你有很多過去,.
但在過程當中,.
你能不能因信主之後,.
一件一件地脫去,.
讓上帝幫我們更新呢?.
脫去舊的東西,.
是你意識到自己的開始,.
放下所有從來都不容易,.
因為你放下就赤裸了,.
放下就沒有你以往儲存了那麼久的東西,.
當我們經常聽到放下主權給上帝,.
放下一切跟隨上帝的時候,.
其實不容易,.
但對於大家來說,.
脫去是我們接受新的開始,.
你不除舊的,.
你裝不下新的,.
所以這段經文說的是,.
將舊的脫去,.
離開阿當對我們的影響,.
離開你過去一心跟隨了那件事的重點,.
所以去到下一節的時候,.
本來要提及的,.
又要怎樣呢?.
你脫去之後,.
那要穿上什麼?.
不是讓你赤裸裸的到處走,.
上帝不是要這樣羞辱我們,.
也不是要我們,.
正正就是上帝原先給我們的骨架,.

$^{361}$可以承載上帝給我們的豐足的恩典,.
所以你要怎樣?.
你要將你的心智改換一生,.
其實真正難處出錯不是身體,.
上帝原先做我們這個身體,.
這個身形是很好的,.
你發覺我們不會每個人都一樣,.
同一個高度也不同身形,.
我早前收拾了衣服給我大兒子,.
我覺得他應該穿得到,.
我不是穿舊衣服買新衣服,.
我覺得我已經過了穿那些衣服的時代,.
因為我已經不再像他那麼青春,.
我就說我收拾了一些T恤給他,.
他說很漂亮,.
我說是,我保存得很好,.
我說你應該穿得到,.
他說不是,行不行?.
我說T恤不要緊,.
你年輕時穿的感覺跟我現在中年穿的不同,.
我說行,他不只可以,.
我說行的,因為我大兒子跟我一樣高,.
他說穿上去就不行,.
我雖然有高度,.
但沒有你那麼橫,.
這是時間問題,.
而且我說現在這個年代不需要這些身形,.
你這些魔形,韓國男孩的身形就對了,.
我想他明白到一件事,.
有很多事物有時限性,.
有很多事物不需要太多比較,.
上帝原先做我們的模型是很好的,.
是欣賞的,.
但只不過是外添的東西令你覺得很沉重,.
外添的東西覺得你承載不了,.
要附和別人對那件事的要求,.
這不是模型問題,.
是心態問題,.
所以要求的不是要拆毀這個模型,.
上帝要求的就是這個模型是我做的,.

$^{401}$怎會不好呢?.
但是心態破壞了,.
漸霧了,.
冷餓者就動搖你的心態,.
吃吧,很漂亮的,.
吃了好像上帝一樣,.
他動搖了心態,.
所以要除去的就是舊有的習慣,.
不同的事,.
但要改的一定是你的心智,.
你的心態,.
所以心智要改換一生的時候,.
這個字心智改換一生,.
renew in the spirit of your mind,.
親弟姐妹,.
我們個人的模型是上帝獨特做給我們的,.
就是你的body,.
但我們的mindset,.
信主前和信主後,.
你的外觀不會改變的,.
但你的心智要改變,.
信主之後,.
我們要不斷提醒,.
我們怎樣renew,.
怎樣renew our mindset,.
所以改換這個字,.
enemyo這個字是要說一個訊息,.
它是動詞,.
但是被動式的,.
因為根本是被動,.
不會你自己改的,.
你自己改不了的,.
一定要透過外力去幫你,.
因為以弗所書第二章第一節,.
我們死在罪惡過犯當中,.
他叫我們活過來,.
一個死人不會自己叫自己死,.
一定有人叫醒你,.
正如有些事是被動,.
你要人幫的,.

$^{441}$以弗所書要強調一件事,.
如果上帝不是親自呼召我們離開罪惡,.
人是無能力自救,.
離開他過去一直堅持的東西,.
所以同樣在心智renew的時候,.
他仍然是用被動的方法,.
你今天接受了耶穌,.
你被根生,.
雙德的靈就住在我們,.
這就是以弗所書第一章第十五節,.
上星期John讀那段經文之前,.
聖靈就住在我們心裡,.
所以聖靈會提醒我們要改變我們的mindset,.
聖靈會提醒我們,.
你已經不再需要用舊的方式,.
但你可能還能拿住,.
所以保羅提醒我們,.
你慢慢要脫離,.
有些東西不能再用舊的方式,.
所以第二十四節,.
並且穿上新人,.
這新人是照著上帝的形象做的,.
有真理的人意和聖潔..
這段經文單單在二十四節,.
可以多講一篇,.
什麼叫做真理的人意和聖潔呢?.
但今天我想講的是,.
我們是一個新人,.
上帝已經更新了我們,.
其實對我們來說,.
你的更新不只是停留了你洗禮的那一刻,.
我跟同工開同工會的時候,.
Folk Church經歷了三年的水禮,.
由2020年開始到現在,.
我們已經為….
第一年有16個,.
第二年有20個,.
第三年有10個,.
加起來已經有46個弟兄姊妹接受了水禮,.
我們仍然是跟同工說,.

$^{481}$洗禮的弟兄姊妹要更貼近,.
因為有很大的挑戰就是,.
教會的弟兄姊妹洗完禮之後就被提了,.
我真的苦笑了,.
為什麼洗完禮之後不見了那個人呢?.
為什麼洗完禮之後不見了他沒有回來呢?.
他真的相信洗禮就完成了他了足的心事,.
是不是這樣呢?.
不會的,對嗎?.
對你來說,洗禮是什麼呢?.
其實當初你要求洗禮,.
或者你想洗禮是什麼,.
就正如洗禮班第四堂是我上的,.
頭三堂是小組目者上的,.
第四堂上的時候,.
我都希望聽到他們,.
是你告訴我為什麼要洗禮,.
洗了見證的時候,.
都是不斷重提自己,.
其實我為什麼要洗禮,.
是不是因為我回家這麼久,.
人們說你也要洗禮,.
想想洗禮,.
哦,是,我也回家這麼久,.
應該洗禮的,.
那就變成行禮如儀,.
其實就不是自己,.
如果你不是很理解的話,.
我叫你除去,.
你都不知道要除什麼,.
我有穿過嗎?.
我不覺得有,.
因為你都沒有意識到自己的限制和困難在哪裡,.
所以在這個新人的過程當中,.
你可能已經洗禮很久,.
但我想跟你說,.
你仍然是一個新人,.
因為上帝會每時每刻都更新我們,.
只要你哪一個時刻願意放下你舊有的習慣,.
放下你舊有的想法,.

$^{521}$放下你舊有覺得,.
你持守覺得很重要,.
那是你安全或是安恕的東西的時候,.
你願意去學習耶穌,.
或者是回應上帝的教訓要求的時候,.
仍然是一個新人,.
Flowchart去到三年了,.
我們小組都突破三十,.
對我們來說,.
我們聽過很多弟兄姊妹的故事,.
亦聽過很多不同人在信仰生命當中的高高低低,.
但我們仍然希望弟兄姊妹在當中會有一些東西放下,.
有一些東西改變,.
不是代表那件事不重要,.
但覺得在學習過程當中,.
讓上帝的說話再一次提醒我們,.
其實上帝要一個新人是什麼呢?.
你的Mindset應該要再想想,.
其實什麼為之真理的仁義?.
仁義是什麼?.
仁義是一個道德標準,.
仁義是一個判斷的方法,.
對於你來說,.
什麼為之不仁,什麼為之不義?.
你心中的八尺,是社會告訴你,還是聖經告訴你?.
你所謂合法,.
但不代表合上帝的真理,.
所謂這個世界覺得對的事,.
不代表上帝喜悅的事,.
所以仁義這個標準,.
是要不斷問自己,.
是不是這樣?.
這樣做對嗎?.
我自己在Folk Church的講道也說,.
《羅馬書》15章曾經提到,.
眾人以為善的事,.
你要留心做,.
其實留心的字同樣是要分辨去做,.
齊齊做的時候,.
沒有事就一起做,.

$^{561}$這樣又不好,.
一波風跟隨,.
你會跟扯太貼,.
所以Folk Church對我們來說,.
在網絡文化當中已經很普遍,.
但對於我們要判辯的時候,.
仁義和聖潔往往走在一起,.
你判辯那件事,.
你做那個行徑的時候,.
你有沒有一個聖潔的標準?.
聖潔的標準就是上帝讓我們分辨為聖,.
和上帝讓我們看到,.
有什麼是上帝喜悅,.
有什麼是上帝憎惡,.
聖潔在這個環境中不容易,.
一刻三言兩語去說,.
但你何時分辨到那件事對不對呢?.
很多時候都是事後,.
但我可以相信一件事,.
就是聖靈會在當中對著我們的心說話,.
曾經講道提過,.
聖靈是耶穌在臨去喀西馬利云之先,.
跟那班門徒說,.
真理的靈會臨到,.
令到你們想起我說的話,.
真理的靈是什麼?.
又稱保衛師,.
和合本旁邊幾個小字叫做分衛師,.
教訓的分,安慰的衛,.
它會教訓我們,.
也會安慰我們,.
因為聖靈的原文叫做Para Kratos,.
Para就是圍著的意思,.
Kratos是speak,.
它是圍著我們說話,.
所以如果你去求告上帝,.
上帝我願意行使你給我要判辯仁義的力量,.
以至那件事是不是聖潔的表達呢?.
或者是不是上帝喜悅的表達的時候,.
真理的靈住在我們心裡,.

$^{601}$它不會不跟我們說話,.
這就是土主喜悅,.
所以快快的聽,.
慢慢的說,.
就是我們去學習聽上帝的聲音,.
要學習留心從來不是一時三刻的事,.
要學習記性更加不是,.
記性一定要學,.
記性的時候要練,.
經文對於我們的提醒純純都是,.
你有多盡心在當中學習呢?.
就好像剛才從第二十節開始讀,.
你坐過很多次崇拜,.
你還是真的坐完就當做完了那件事?.
今天的難處就是,.
你足不出戶都可以參加崇拜,.
因為你覺得,.
我是在參與的,.
不過我受肉身限制,.
我不可以出街,.
我就要坐在這裡,.
看網誦,.
而你是投入到的,.
但是崇拜從來都不是說你接收,.
崇拜從來都說一個群體,.
參與,.
互動,.
那個synergy,.
對我們來說,.
很多事我們在頭腦上知道,.
但是走出來其實是缺乏操練,.
最後那個投影是,.
月提,.
迎新,.
當我們說到想要改變,.
想有些事要去更新,.
想有些事新的迎向,.
首先,.
我再重申,.
我們上帝給我們這個新造的人,.

$^{641}$我們的figure是上帝的創造,.
每個人都是獨特的,.
那個沒有對錯,.
反而是我們的mindset,.
我們的思念,.
我們的想念,.
有沒有想上帝為先,.
有沒有願意放下過去覺得很持守的東西,.
所以當你想說,.
我接下來這一年要change,.
有些事改變了,.
要改動的時候,.
我盼望在現在10月,.
11月,.
12月,.
我很慘的,.
我希望你不要到2,3年才想,.
就還有這三個月,.
兩個半月的時候想想,.
如果要改的時候,.
我希望你那個change那個字,.
那個G,.
可能你覺得你仍然很懷念那些good old days,.
不要要了,.
那個可能不是你,.
你可能有其他更好的日子,.
你可以有轉變,.
你可能覺得那個G是很不濟的東西,.
你要放下,.
你要換,.
換什麼?.
我希望你換個C下去,.
那個C是什麼?.
仍然是,.
記住,.
由你當初信主到你洗禮,.
是上帝call我們的,.
是上帝呼召的,.
如果不是聖靈感動你,.
你不會接受洗禮,.

$^{681}$如果不是聖靈感動你,.
呼召你,.
你不會願意跟隨耶穌,.
所以記住,.
上帝每時每刻都是call我們,.
特別在現在這麼困難的時期,.
很多時候你要做仁義,.
做聖潔的決定,.
都不容易,.
但是你要想這一刻,.
你為什麼要想?.
因為你知道那件事是上帝喜悅,.
其實上帝都call你,.
因為第二個C就是,.
你願意過一個Christ-centered life,.
以基督為中心的生命,.
這個就是保羅提醒我們,.
所以頂尖夢很希望,.
新的一年,.
我們將那個G,.
你要轉,.
變成做chance,.
新人仍然有很多機會,.
新人仍然有很多挑戰,.
但是那個機會比我們更加相遇,.
更加多同路人,.
讓我們看到上帝在我們生命當中,.
仍然會作新事,.
所以迎新這個主題,.
是Flowchurch很希望,.
弟兄姊妹,.
去看見將來,.
看遠像,.
哪怕你什麼光景都好,.
無論香港或者其他景地都好,.
你願意以基督中心,.
去回應上帝的呼召,.
你仍然是一個新人,.
仍然看到上帝在我們生命當中的新事,.
記念到在Flowchurch,.

$^{721}$接受水禮的46位弟兄姊妹,.
特別是剛剛過了幾天的弟兄姊妹,.
或者你已經洗禮很久都好,.
你接受了水禮,.
就是上帝召喚我們,.
你會看到上帝給我們很多chance,.
我們一起祈禱,.
天上上帝,.
從來都是聽了很多話,.
但是知易行難,.
但求主你幫助我們,.
不要抹殺聖靈給我們的提醒和幫助,.
不要抹殺聖靈給我們的催逼,.
因為祂常常提醒督責我們,.
又願意我們知道那件事是困難的時候,.
可能我們即時做不到,.
但求主你給我們多一次機會,.
我們願意跟從你,.
上帝你一直對我們很多寬容,.
你沒有逼我們,.
你仍然等我們,.
但我希望主我們由離開舊有,.
我們覺得舒適,.
或者覺得順理成章的生活,.
我們願意放下,.
我們願意學習,.
脫去舊有一切,.
我們願意在當中學習新的一頁,.
求主你幫助,.
你願意祝福我們的群體彼此接納,.
彼此提醒,彼此扶持,.
在當中我們繼續開新的一頁,.
求主你因領奉耶穌的名求,.
阿門..
\newpage



\section{撒母耳記上 17:1-58-20221015}
\label{sec:68fGAExhs0o}
\textbf{【網上崇拜】想像力量同幻想會嚇你一跳|撒母耳記上17\_1-58|20221015 [68fGAExhs0o]}
\newline
\newline
連結: \href{https://youtube.com/watch?v=68fGAExhs0o}{\texttt{ https://youtube.com/watch?v=68fGAExhs0o}} ~~~~ 語音日期: 2022-10-15 
\newline
\newline
\hyperref[sec:Zv7Jalkm4FA]{\small{< < < PREV SERMON < < <}}
~
\hyperref[sec:index_chronic]{\small{[返順時目]}}
~
\hyperref[sec:index_scriptual]{\small{[返順卷目]}}
~
\hyperref[sec:l2LEDZopMVg]{\small{> > > NEXT SERMON > > >}}
\newline
\newline
撒母耳記上 17:1-58-20221015
\newline
\begin{longtable}{cl}
\hline
\hline
章節 & 經文 (和合本修訂版)\\
\hline
17:1 & \begin{tabularx}{0.7\textwidth}{X} 非利士人召集他們的軍隊來爭戰。他們聚集在猶大的梭哥,在梭哥和亞西加中間的以弗‧大憫安營。 \end{tabularx} \\ \\ \relax
17:2 & \begin{tabularx}{0.7\textwidth}{X} 掃羅和以色列人也聚集,在以拉谷安營,擺陣迎戰,要與非利士人打仗。 \end{tabularx} \\ \\ \relax
17:3 & \begin{tabularx}{0.7\textwidth}{X} 非利士人站在這邊的山上,以色列人站在那邊的山上,當中有谷。 \end{tabularx} \\ \\ \relax
17:4 & \begin{tabularx}{0.7\textwidth}{X} 從非利士營中出來一個挑戰的人,名叫歌利亞,是迦特人,身高六肘一虎口。 \end{tabularx} \\ \\ \relax
17:5 & \begin{tabularx}{0.7\textwidth}{X} 他頭戴銅盔,身穿鎧甲,甲重五千舍客勒銅。 \end{tabularx} \\ \\ \relax
17:6 & \begin{tabularx}{0.7\textwidth}{X} 他腿上有銅護膝,兩肩之中背負銅矛。 \end{tabularx} \\ \\ \relax
17:7 & \begin{tabularx}{0.7\textwidth}{X} 他的槍桿粗如織布機的軸,槍頭的鐵重六百舍客勒。有一個拿盾牌的人走在他前面。 \end{tabularx} \\ \\ \relax
17:8 & \begin{tabularx}{0.7\textwidth}{X} 歌利亞站著,對以色列的軍隊喊叫,對他們說:「你們出來擺陣作戰是為了甚麼呢?我不是非利士人嗎?你們不是掃羅的僕人嗎?你們選一個人出來,叫他下來到我這裡吧。 \end{tabularx} \\ \\ \relax
17:9 & \begin{tabularx}{0.7\textwidth}{X} 他若能與我決鬥,把我殺死,我們就作你們的奴隸;我若勝了他,把他殺死,你們就作我們的奴隸,服事我們。」 \end{tabularx} \\ \\ \relax
17:10 & \begin{tabularx}{0.7\textwidth}{X} 那非利士人又說:「我今日向以色列的軍隊罵陣。你們叫一個人出來,跟我決鬥吧。」 \end{tabularx} \\ \\ \relax
17:11 & \begin{tabularx}{0.7\textwidth}{X} 掃羅和以色列眾人聽見非利士人這些話就驚惶,非常害怕。 \end{tabularx} \\ \\ \relax
17:12 & \begin{tabularx}{0.7\textwidth}{X} 大衛是猶大伯利恆的以法他人耶西的兒子,耶西有八個兒子。在掃羅的時候,這人年老,在眾人中受敬重。 \end{tabularx} \\ \\ \relax
17:13 & \begin{tabularx}{0.7\textwidth}{X} 耶西最大的三個兒子跟隨掃羅出征。出征的三個兒子名字是:長子以利押,次子亞比拿達,三子沙瑪。 \end{tabularx} \\ \\ \relax
17:14 & \begin{tabularx}{0.7\textwidth}{X} 大衛是最小的,最大的三個兒子跟隨掃羅。 \end{tabularx} \\ \\ \relax
17:15 & \begin{tabularx}{0.7\textwidth}{X} 大衛有時離開掃羅,回伯利恆為他父親放羊。 \end{tabularx} \\ \\ \relax
17:16 & \begin{tabularx}{0.7\textwidth}{X} 那非利士人早晚都出來站著,共四十日。 \end{tabularx} \\ \\ \relax
17:17 & \begin{tabularx}{0.7\textwidth}{X} 耶西對他兒子大衛說:「你拿一伊法烘了的穗子和十個餅,跑到營裡去,交給你的哥哥, \end{tabularx} \\ \\ \relax
17:18 & \begin{tabularx}{0.7\textwidth}{X} 再拿這十塊奶餅,送給他們的千夫長,並要問你哥哥好,向他們要個憑據回來。」 \end{tabularx} \\ \\ \relax
17:19 & \begin{tabularx}{0.7\textwidth}{X} 掃羅和大衛的三個哥哥,以及以色列眾人,都在以拉谷與非利士人打仗。 \end{tabularx} \\ \\ \relax
17:20 & \begin{tabularx}{0.7\textwidth}{X} 大衛早晨起來,把羊交託一個看守的人,照耶西所吩咐的帶著食物去了。到了軍營,軍隊剛出到戰場,吶喊叫陣。 \end{tabularx} \\ \\ \relax
17:21 & \begin{tabularx}{0.7\textwidth}{X} 以色列人和非利士人都擺列陣勢,彼此相對。 \end{tabularx} \\ \\ \relax
17:22 & \begin{tabularx}{0.7\textwidth}{X} 大衛把東西留在看守物件的人手中,跑到戰場,問他哥哥好。 \end{tabularx} \\ \\ \relax
17:23 & \begin{tabularx}{0.7\textwidth}{X} 他與他們說話的時候,看哪,那挑戰的人,就是迦特的非利士人歌利亞,從非利士隊伍中上來,說了同樣的話,大衛聽見了。 \end{tabularx} \\ \\ \relax
17:24 & \begin{tabularx}{0.7\textwidth}{X} 以色列眾人看見那人就非常害怕,從他面前逃跑。 \end{tabularx} \\ \\ \relax
17:25 & \begin{tabularx}{0.7\textwidth}{X} 以色列人說:「這上來的人你看見了嗎?他上來是要向以色列人罵陣。若有人能殺他,王必賞賜他大財,將自己的女兒嫁給他,並在以色列人中免除他父家納糧服役。」 \end{tabularx} \\ \\ \relax
17:26 & \begin{tabularx}{0.7\textwidth}{X} 大衛對站在旁邊的人說:「若有人殺這非利士人,除掉以色列人的羞辱,他會怎樣呢?這未受割禮的非利士人是誰,竟敢向永生神的軍隊罵陣!」 \end{tabularx} \\ \\ \relax
17:27 & \begin{tabularx}{0.7\textwidth}{X} 百姓照同樣的話對他說:「若有人殺了那人,必這樣待他。」 \end{tabularx} \\ \\ \relax
17:28 & \begin{tabularx}{0.7\textwidth}{X} 大衛的長兄以利押聽見大衛與他們所說的話,就向他發怒,說:「你下來做甚麼呢?在曠野的那幾隻羊,你交託誰了呢?我知道你的驕傲和你心裡的惡意,你下來只是為了看戰爭!」 \end{tabularx} \\ \\ \relax
17:29 & \begin{tabularx}{0.7\textwidth}{X} 大衛說:「我現在做了甚麼呢?只是問一句話也不可以嗎?」 \end{tabularx} \\ \\ \relax
17:30 & \begin{tabularx}{0.7\textwidth}{X} 大衛離開他轉向別人,問了同樣的事,百姓也照先前的話回答他。 \end{tabularx} \\ \\ \relax
17:31 & \begin{tabularx}{0.7\textwidth}{X} 有人聽見大衛所說的話,就在掃羅面前報告;掃羅就派人叫他來。 \end{tabularx} \\ \\ \relax
17:32 & \begin{tabularx}{0.7\textwidth}{X} 大衛對掃羅說:「人不必因那非利士人灰心。你的僕人要去與他決鬥。」 \end{tabularx} \\ \\ \relax
17:33 & \begin{tabularx}{0.7\textwidth}{X} 掃羅對大衛說:「你不能去與那非利士人決鬥,因為你年紀太輕,他從小就是戰士。」 \end{tabularx} \\ \\ \relax
17:34 & \begin{tabularx}{0.7\textwidth}{X} 大衛對掃羅說:「你僕人為父親放羊,有時獅子來了,有時熊來了,從群中抓走一隻羔羊。 \end{tabularx} \\ \\ \relax
17:35 & \begin{tabularx}{0.7\textwidth}{X} 我就追趕牠,擊打牠,把羔羊從牠口中救出來。牠起來攻擊我,我就揪牠的鬍子,打死牠。 \end{tabularx} \\ \\ \relax
17:36 & \begin{tabularx}{0.7\textwidth}{X} 你僕人曾打死獅子和熊,這未受割禮的非利士人必像獅子和熊一樣,因為他向永生神的軍隊罵陣。」 \end{tabularx} \\ \\ \relax
17:37 & \begin{tabularx}{0.7\textwidth}{X} 大衛又說:「耶和華救我脫離獅子和熊的爪,他必救我脫離這非利士人的手。」掃羅對大衛說:「你去吧!耶和華必與你同在。」 \end{tabularx} \\ \\ \relax
17:38 & \begin{tabularx}{0.7\textwidth}{X} 掃羅把自己的戰衣給大衛穿上,將銅盔戴在他頭上,又給他穿上鎧甲。 \end{tabularx} \\ \\ \relax
17:39 & \begin{tabularx}{0.7\textwidth}{X} 大衛佩刀在戰衣上,試著走走看。因大衛沒有試過,就對掃羅說:「我穿戴這些不能走路,因為我沒有試過。」於是他脫下身上的這些軍裝。 \end{tabularx} \\ \\ \relax
17:40 & \begin{tabularx}{0.7\textwidth}{X} 他手中拿杖,又在溪中挑選了五塊光滑的石子,放在袋裡,就是牧人帶的囊裡,手裡拿著甩石的機弦,迎向那非利士人。 \end{tabularx} \\ \\ \relax
17:41 & \begin{tabularx}{0.7\textwidth}{X} 那非利士人漸漸走近大衛,拿盾牌的人在他前面。 \end{tabularx} \\ \\ \relax
17:42 & \begin{tabularx}{0.7\textwidth}{X} 非利士人觀看,見了大衛,就藐視他,因為他年輕,面色紅潤,容貌俊美。 \end{tabularx} \\ \\ \relax
17:43 & \begin{tabularx}{0.7\textwidth}{X} 非利士人對大衛說:「你拿著杖到我這裡來,我豈是狗嗎?」非利士人就指著自己的神明詛咒大衛。 \end{tabularx} \\ \\ \relax
17:44 & \begin{tabularx}{0.7\textwidth}{X} 非利士人又對大衛說:「來吧!我要把你的肉給空中的飛鳥和田野的走獸。」 \end{tabularx} \\ \\ \relax
17:45 & \begin{tabularx}{0.7\textwidth}{X} 大衛對非利士人說:「你來攻擊我,是靠著刀槍和銅矛,但我來攻擊你,是靠著萬軍之耶和華的名,就是你所辱罵、帶領以色列軍隊的神。 \end{tabularx} \\ \\ \relax
17:46 & \begin{tabularx}{0.7\textwidth}{X} 今日耶和華必將你交在我手裡。我必殺你,砍下你的頭,今日我要把非利士軍兵的屍體給空中的飛鳥和地上的野獸,使全地的人都知道以色列中有神, \end{tabularx} \\ \\ \relax
17:47 & \begin{tabularx}{0.7\textwidth}{X} 又使這裡的全會眾知道,耶和華使人得勝,不是用刀用槍,因為戰爭全在乎耶和華。他必將你們交在我們手裡。」 \end{tabularx} \\ \\ \relax
17:48 & \begin{tabularx}{0.7\textwidth}{X} 那非利士人起來,迎向大衛,走近前來。大衛急忙往戰場,迎向非利士人跑去。 \end{tabularx} \\ \\ \relax
17:49 & \begin{tabularx}{0.7\textwidth}{X} 大衛伸手入囊中,從裡面掏出一塊石子來,用機弦甩去,擊中非利士人的前額,石子進入額內,他就仆倒,面伏於地。 \end{tabularx} \\ \\ \relax
17:50 & \begin{tabularx}{0.7\textwidth}{X} 這樣,大衛用機弦和石子勝了那非利士人,擊中了他,把他殺死;大衛手中沒有刀。 \end{tabularx} \\ \\ \relax
17:51 & \begin{tabularx}{0.7\textwidth}{X} 大衛跑去,站在那非利士人身旁,把他的刀從鞘中拔出來,殺死他,用刀割下他的頭。非利士眾人看見他們的勇士死了,就都逃跑。 \end{tabularx} \\ \\ \relax
17:52 & \begin{tabularx}{0.7\textwidth}{X} 以色列人和猶大人就起來吶喊,追趕非利士人,直到該和以革倫的城門。被殺的非利士人倒在路上,從沙拉音直到迦特和以革倫。 \end{tabularx} \\ \\ \relax
17:53 & \begin{tabularx}{0.7\textwidth}{X} 以色列人追趕非利士人回來,搶奪了他們的軍營。 \end{tabularx} \\ \\ \relax
17:54 & \begin{tabularx}{0.7\textwidth}{X} 大衛拿著那非利士人的頭帶到耶路撒冷,卻把那非利士人的軍裝放在自己的帳棚裡。 \end{tabularx} \\ \\ \relax
17:55 & \begin{tabularx}{0.7\textwidth}{X} 掃羅看見大衛去迎戰非利士人,就問押尼珥元帥說:「押尼珥,那年輕人是誰的兒子?」押尼珥說:「王啊,我在你面前起誓,我不知道。」 \end{tabularx} \\ \\ \relax
17:56 & \begin{tabularx}{0.7\textwidth}{X} 王說:「你可以問問那孩子是誰的兒子。」 \end{tabularx} \\ \\ \relax
17:57 & \begin{tabularx}{0.7\textwidth}{X} 大衛打死那非利士人回來,押尼珥領他到掃羅面前,大衛手中拿著非利士人的頭。 \end{tabularx} \\ \\ \relax
17:58 & \begin{tabularx}{0.7\textwidth}{X} 掃羅問他說:「年輕人,你是誰的兒子?」大衛說:「我是你僕人伯利恆人耶西的兒子。」 \end{tabularx} \\ \\
[1ex]
\hline
\hline
\end{longtable}
$^{1}$好 等一等姊妹晚安.
上個月我在Vulture篇上就在說《掃羅》.
雖然篇上主要是在說五餅二魚.
但牽涉到一個課題就是關於掃羅造王的問題.
其實今天很想 希望能夠輕輕點題一下.
到底有一個問題困擾了我很多年.
其實在三位意見上的時候.
發現為什麼掃羅會被嫌棄.
大衛會沒事.
坦白說大衛不是壞.
聽人說 免得說了.
他不是那個位置 其實也是.
其實他最壞是把別人的老公放在戰場.
最危險的地方先弄死.
其實他還壞很多東西.
他最後數點人數.
瘟疫來到了.
他在田裡祈禱.
瘟疫止住了 死的人很多.
但他其實沒事.
我的問題是為什麼大衛會被上帝蒙拯救.
他怎麼錯都好像沒事.
掃羅壞的東西.
可能在憲制上等不到撒滿爾來.
然後他就壞了.
之後就不是很好.
我的問題是為什麼掃羅沒有機會.
大衛壞了那麼多東西都有機會.
這個問題困擾了很多年.
從小時候讀聖經開始.
除了上帝選大衛之外.
不選掃羅.
好像選亞各不選爾掃羅.
但我今天希望能夠花點氣力時間.
很少的.
希望能夠講到大衛和掃羅之間的分別.
當然我們今天不會直接講到大衛和掃羅很多不同的東西.
因為牽涉到基文很多.
但希望今天能夠藉著打哥利亞這個故事.
我們兒童專學學開始講起的故事.

$^{41}$我想聽聽看.
到底發生什麼事.
我們是不是要拍一個Post.
有的.
在哪裡.
沒錯.
我們下一張就行了.
這個純粹搞笑.
我們下一張就行了.
再按一張就行了.
我們先看看哥利亞實力.
先看看小經文.
再下一張.
他在17章裡說.
他從菲利士人裡面有個土戰的人叫哥利亞.
是加特人身高六爪零一虎口.
六爪零一虎口我們翻譯出來是七尺高.
他應該是挺高大.
頭戴著銅盔.
身穿海甲.
甲重五千四十六.
我不翻譯現在的東西.
腿上有銅護膝.
兩腋中有背負著銅戈.
我們再按下去.
他們真的很厲害.
第八節怎樣說.
哥利亞對著以色列的軍隊站著呼叫說.
你們出來擺陣.
我不是菲利士人嗎.
你們不是數羅的僕人嗎.
你們可以從中間選一個人來.
使他可以從我那裡來.
九十十一節.
再看完三節.
如果他能夠和我戰鬥的話.
將我殺死.
我就作他的僕人.
我約聖了他將他殺死.
你們就作我的僕人.

$^{81}$服侍我們.
菲利士人又說.
以色列人的軍隊罵陣.
你們叫一個人出來與我戰鬥.
最後一節.
以色列和數羅的族人聽見.
菲利士人的這番話.
就驚恐害怕.
基本上其實.
大衛打哥利亞這個故事裡面.
其實他初時花了很多氣力說.
哥利亞有多厲害.
我講一些近代的.
我現在經歷的異名.
我一月就搬了來.
在市區住.
我的女兒和兒子.
以前在離島村校.
讀書.
離島村校的小朋友.
個個都很厲害.
不同方面都很厲害.
但一出到市區.
這九個月十個月.
我發現.
市區的小朋友.
無所不能.
神乎其技.
跟著是.
他什麼都可以.
市區的.
村校的都很厲害.
我想說.
但一去到市區.
我發覺.
什麼都懂的人.
那些小朋友.
我覺得我們.
都有教他們.
有額外教的東西.

$^{121}$但怎樣教都好.
怎樣追都追不到.
那些神乎其技.
又懂fencing 又懂saxophone.
又游水.
每次游三次.
兩小時飯都不吃.
四點才回家吃飯的小朋友.
那些很奇妙的小朋友.
我發覺.
我好掙扎.
原來.
我很不濟.
為什麼人家.
很厲害.
基本上這九十個月.
我經常看著那些巨人.
我四十幾歲.
就是大衛.
那些小朋友每個在我面前都是巨人.
哥利亞feel.
你知道這些環境下.
你沒有.
一技之長.
是很難生存的.
基本上.
都幾不容易.
所以基本上.
巨人.
和哥利亞這些故事.
其實今天都有.
起碼這九十個月裡.
我經常覺得.
怎樣跑才跑到.
一個人七呎這麼高.
你叫他做多些tabata.
或者打多些籃球.
他不是每天都七呎嗎.
是不是每天塞滿了他的中文英文數學.
請人來教他.

$^{161}$然後找專人來教他.
學奧數.
然後又不知怎樣.
其實真的.
我們在村校的生活太美好.
一出來生活就很焦急.
焦急到一個地步.
起碼我都失戀了.
到底什麼叫教育呢.
這些很真實的事件.
所以其實.
大衛和哥利亞的事件是很common的.
不單止這樣.
譬如你看到烏克蘭.
和俄羅斯.
你又見到.
哥利亞和大衛的故事.
每個人都梳他們的實力.
每個人都不避忌.
避諱地將他們最厲害的東西.
傳出來.
所以經文都不避諱.
將哥利亞有多厲害的東西.
都講出來.
我們再按下去.
我想說.
哥利亞實力叫我們歸回現實.
歸回現實.
現實即是說.
我們有一套想法.
但現實環境裡逼使我們.
有些想法.
不知道要怎麼妥協.
這個很殘忍.
你再看下去.
再看下去.
哥哥的興士.
哥利亞故事裡.
除了講哥利亞外.
還見到大衛的長兄叫伊利亞.

$^{201}$伊利亞很打得.
基本上伊利亞應該是.
大武爾看著他的時候.
就以為他是神所高立的.
所以他應該可以.
但他都在索羅軍中打仗.
但他不喜歡.
見到大衛的時候.
他就發怒.
你抗疫那幾隻羊.
你交託給誰.
我知道你的心裡有很多惡意.
你下來就要看這個征戰.
但我做什麼.
我不是沒有緣故嗎.
其實在整個現實裡.
是很現實的.
那個巨人很現實.
當你要面對巨人的時候.
現實到那個地步的時候.
你想想有沒有機會不敢做.
打贏他.
不要退縮.
你看那些哥哥們已經不like你了.
他馬上不follow你的page.
你再按下去.
我意思是想說.
哥哥的興士.
叫我們歸回現實.
艱難同心似鳳.
其實你發現.
有時候.
我們想挑戰一些巨人的時候.
一些高牆的時候.
雞蛋丟過去的時候.
你發現過去那幾年裡.
很多家人都不和睦.
你要挑戰一些高牆的時候.
你家裡的人總是有不同聲音.
不同意見.

$^{241}$正如我的子女要讀書.
要上什麼班.
一定有爭拗.
幫她這間.
不要這間.
教那間不要教那間.
給她時間玩玩.
讓她繼續努力.
其實很難.
現實是逼使我們.
在一個.
真實的環境下.
我們好像沒有選擇.
我們再看多一個.
蘇羅的戰衣.
蘇羅的戰衣是說什麼.
我們再按下去.
看那幾次經文.
那兩次是怎麼說的.
蘇羅將自己的戰衣給戴在村上.
將那個銅灰.
給他戴上.
給他穿上海格.
但他就將刀誇在戰衣外.
試試能不能走.
因為蘇羅沒有穿慣.
對蘇羅說.
我穿戴著也不能走.
因為蘇羅沒有穿慣就脫了.
我想說的是.
我們在現實裡被迫.
要跟著別人的節奏走.
我們被迫要跟著別人的步伐走.
這是現實裡不單止.
令我們覺得恐懼.
或者不知道怎麼處理的時候.
家人跟你想法不一樣之餘.
更難度高的是.
你被迫好像要穿回.
蘇羅的戰衣.

$^{281}$但他總要有劇情.
就是說蘇羅就讓人知道.
他穿了.
大家都期望他穿蘇羅的戰衣.
因為你年輕時去打仗.
你代表誰?代表蘇羅去打仗.
所以要穿蘇羅的戰衣去打仗.
你打贏的話.
你就行了.
因為代表了蘇羅.
其實在現實環境裡.
到今時今日.
我們跟著很多人的步伐走.
跟著很現實的環境裡.
我們要走下去.
步伐好像.
不能走不能不走.
這個是整個.
《三昧紀相》.
第17章裡說哥利亞的故事的時候.
一個很真實的寫照.
純粹不是.
拿五個石頭.
丟在哥利亞的頭上.
然後躺下.
他描繪了很多.
面對著高牆.
面對著雞蛋與高牆的時候.
我們走.
我們跟著.
人家設定的行程和節奏.
走下去.
我們看看.
最近還沒說《野人老師》.
大家有沒有看?Stanley的演技.
是不是有些進步?.
Frankie就.
好像比較目獨.
不過他的角色可能也是這麼目獨.
這10晚.

$^{321}$下星期又有多10晚.
昨晚就開始說.
Frankie要打柔道.
之前學校.
是真實的.
不過不是在平澳洲.
我Google平澳洲的時候.
原來很多傻的人和我一樣.
都一起Google過平澳洲.
你現在Google一下.
那些人說.
平澳洲在哪裡?.
平澳洲在哪裡?是Google得多的.
是馬上出來的.
但其實香港人沒有平澳洲.
只有平洲和大澳.
所以.
《野人老師》說學校在平澳洲.
其實就騙了很多人.
連我傻傻地.
我跟老婆說.
老婆,平澳洲在哪裡?.
你都不知道?就在那裡.
我馬上.
怒怒地Google一下.
原來真的沒有平澳洲.
原來根據人的步伐走.
是很普遍的.
原來這間學校.
在何福堂旁邊.
在屯門.
15週年了.
我最近買了一本書.
是他們15週年的時候.
說這間自由學校.
由開始到末了.
現在有很多老師.
現在有十多個老師.
在這間自由學校裡教書.
我認識一個.

$^{361}$很好的朋友.
他的兒子是從小到大.
讀書很挫敗的.
很挫敗.
他轉了一兩間學校.
結果都不行.
結果就去了自然學校.
他真的在那裡讀書.
他讀書的時候.
我知道,我們一起吃飯的時候.
我說過,他們是1至6年班.
一起上課.
他們不是一些.
很規矩的學校.
不是一年班,二年班,三年班.
那六年班.
很多時候一起上課.
一起自主學習,很開心.
我這個朋友的兒子.
他的夢想是做足球員.
結果他真的.
接下來他應該.
這個月會去.
代表香港青年軍.
去打龍門.
去中亞一個國家.
代表香港比賽.
很多人在.
跟隨人節奏的裡面.
不甘心的.
現實上是.
我們很多要妥協的地方.
現實上我們不是說.
我們人生突然之間.
開始很跳脫,離開所有人.
設定一個議程,我們做一些.
我們覺得很威猛.
很勇猛的事情.
完全是脫離框架的事情.
不是,現實上我們也有.

$^{401}$很多人跟著框架做事.
但是.
你不可以否認的是.
在這個世界裡面.
有很多人.
仍然不甘心在框架當中.
自然學校.
是一間這樣的學校.
如果你記得鏗鏘集的時候.
也訪問了一個.
小朋友,這個小朋友是.
很厲害的小朋友.
他所有科目都考第一.
所有獎都拿了.
學校所有的比賽.
他都參加就會贏.
真實的鏗鏘集的報告.
報告完之後,接著發覺.
讀了幾年書之後,他不行了.
他精神壓力太大.
他開始患有不同的精神問題出現.
結果他就跑去自然學校.
去讀書.
他基本上是不適應的.
那些人完全不懂事.
他們不懂有多少顆.
可能他們也懂.
他會問一些他覺得很厲害的事.
別人不懂的事.
結果這個小朋友花了一年的時間.
去學如何跟別人相處.
由不懂得跟別人相處.
不享受跟別人相處.
到那一年裡,他慢慢的.
心靈改變了一樣.
就好像昨天還是前天那一集.
Stan這個兒子,軒軒.
一去到那間學校就只顧著打機.
結果幾天之後.
他就說我不需要手機了.

$^{441}$然後我老婆就說.
我也要去那間自然學校.
就是.
我怎樣也能讓他委任我.
就這樣下一仗.
所以其實.
是真實的.
人是總是會想在高牆裡.
在現實裡.
不跟著節奏走.
所以那些事是跳脫的.
但其實自然學校是.
一間這樣的學校.
如果你知道最感動的.
那個位置是鄭丹瑞.
抱著一個自閉症的小朋友.
我相信無論是IG.
或者其他的社交媒體.
都有瘋傳這一段的短片出來.
如果你跟開ViuTV的話.
所以你知道的是.
在現實裡.
我們好像無能為力地.
主要走一個套路.
但散會議記上17章想說的是.
我們不一定這樣走.
我們再看一個PowerPoint.
我這個想法是來自.
Eugene Peterson的一本書.
今天.
Peterson他說這個是.
大衛對上帝的想像力.
即是他的想像力.
我們再看一下.
他的想像力是什麼.
他說什麼.
以色列人彼此說.
這個上來的人你看到什麼.
他上來是要向以色列人罵陣.
如果能夠殺了他的話.

$^{481}$王必想他大財.
將自己素奴的女兒給他為妻.
讓以色列人免他傅家立的當差.
但大衛不是這樣想的.
他說什麼.
他站在旁邊的人問他.
除掉以色列人的羞辱.
你怎樣待他.
這個未受國禮的非利士人是誰.
竟然向永生神的軍隊罵陣.
這個25,26節其實很奇怪.
25,26節有人問.
如果你贏了的話.
有錢的了 有老婆的了.
你明白嗎.
當時來說你有名譽有地位.
有一切 你打贏的話就行.
但26節很奇怪.
26節的焦點.
大衛不是要女人和錢.
26節的焦點.
竟然是誰敢向永生神萬軍耶和華罵陣.
這句話是大衛開始跳脫.
開始有想像力去想一些不同的東西.
我講想像力.
我要先confess.
十年前我去佛羅里達州.
奧蘭多的迪士尼世界.
你知道迪士尼世界是很厲害的.
在奧蘭多.
其中有四個theme park.
其中一個theme park是專放世界的spectacular.
世界各地很特別的東西.
日本館 什麼館.
去墨西哥看.
每個國家很特別的東西.
你知道晚上有個show.
其中一個theme park就是香港放煙花.
在castle.
那個不是.

$^{521}$這個theme park是在中間的湖.
晚上有燈光.
你知幻彩要香江的朋友.
當然漂亮很多的要香江.
最初有邪惡 魔鬼.
邪惡的樣子出來.
很黑暗 很邪.
突然之間.
那個show的最高潮是什麼.
就是mickey mouse上來.
你知道mickey mouse是怎樣的嗎.
mickey mouse上來之後.
我們放我們的imagination.
一講完imagination之後.
燈光就這樣.
所有黑暗勢力就突然消失.
對我過了十年前來說.
我覺得真的超無聊.
你知道多貴一張票嗎.
在奧蘭多.
你讓故事作得好一點可以嗎.
我有imagination.
就沒有了.
什麼來的 無聊到這樣.
這十年間我開始悔改.
原來Walt Disney.
他小時候畫的mickey mouse和mini是怎樣的呢.
他放完工之後.
他坐在那裡看著子女.
玩那些see-saw 玩那些slide.
那些很.
不知道玩了幾十年都要玩那些.
你明不明白.
幾十年都是see-saw.
很開心.
有時候千秋或什麼滑梯那些.
那些slide和那些東西.
所有公園都是這樣.
他坐在bench上看著子女玩的時候.
他很不開心.

$^{561}$他說這個世界小朋友就是玩這些嗎.
玩完這些之後小朋友會變聰明嗎.
有沒有東西可以讓小朋友玩完之後.
他想的東西會不同了.
他的世界觀會改變了.
因為這個緣故他辭職了.
就開始創作mickey mouse和mini.
到今天.
我明白這個故事之後.
我明白為什麼mickey mouse要升上來.
要有imagination power.
原來imagination 想像力.
是足以打敗很多東西的.
是足以打敗我們在routine裡.
有很多規矩 有很多框架.
有很多集而為常 有很多高牆的東西.
我們覺得沒有可能性的東西.
我們乖乖地compromise 留在這裡.
留在這裡就算了.
但是imagination是將人離開了.
我們現實環境裡所有的框架.
進到不同的地方告訴自己.
我不一定要在這裡.
所以Walt Disney說什麼呢.
我們再下一個.
再下一個 下一個經文.
32,37字說什麼呢.
大衛對蘇羅說.
人都不必因菲利士人膽怯.
這個人傻傻的.
這個人要去跟菲利士人戰鬥.
蘇羅對大衛說.
你不能去跟菲利士人戰鬥.
因為你年紀太輕.
他自幼就做戰士.
大衛對蘇羅說.
這人為父親放羊.
有時來了獅子 有時來了紅人.
群中就射了一隻羊羔.
35字說.

$^{601}$我就追趕他 擊打他.
將羊羔從他口中救出來.
他起來要害他.
他就捉住他的鬍鬚將他打死.
大衛傻的.
紅人和獅子.
他是誰.
他說你僕人打死獅子和紅人.
這個未受國利的菲利士人.
向滿身神的軍隊罵陣.
也好像我所要對付的獅子和他一樣.
大衛又說.
耶和華救他脫離獅子的紅和爪.
也必救他脫離菲利士人的手.
蘇羅對大衛說.
你可以去 但必要同在.
這個說法很壓前.
你去吧 死吧.
不過說得偉大一點.
下一章.
在這個點.
我想說的是.
對萬軍耶和華的想像.
是從小培養出來的.
我覺得這點是重要的.
我覺得這點非常重要.
原來大衛能夠夠膽打.
哥尼亞不是因為他厲害.
不是因為他計算好石頭的角度和力度.
怎樣能夠做.
你知道《鳳凰聖經》有很多人做實驗.
有人真的做過實驗.
找些石頭 找些阿杉.
怎樣能夠打一個七呎的巨人.
打一個阿圖 把他躺下死掉.
有人還研究力度.
怎樣能夠打死一個七呎高的巨人.
我們不要說那麼科學的證據.
我純粹想說的是.
大衛能夠這樣打.

$^{641}$覺得贏哥尼亞.
要有信心打贏他.
是因為他從小對上帝有想像力.
這是我們這兩個月的主題.
型和新是很需要有的東西.
型和新不是有人很有型.
有些新的東西跟隨著我們跟隨.
不是的.
有型的東西 有新的東西出來.
不是一班人跟隨著很有型很新的東西.
那個叫做想像力.
有些很有型的東西 有新的東西出來的時候.
其實是每一個人在問.
在我生命裡面有些什麼東西是很有型的.
但有些新的想像力走出來的.
新的東西走出來.
不是一班人跟隨新的東西.
有新的東西走出來.
是每一個人都在問.
我在我生命裡面有很多框架.
和很多被人限制 面對高牆 面對壓力的時候.
我可以怎樣跳脫離開這些部份.
走一些上帝想我走的路.
而大衛那一刻能夠贏到.
是因為他從小開始.
他在培養對上帝的想像力.
我的上帝是可以的.
所以我們再看後面那兩張.
所以你看看再看.
所以你看到他最後用什麼武器.
他不穿索羅林軍衣.
他就拿著杖 放在袋裡面.
在木人的籠裡面.
然後拿著機緣迎接菲利士.
下一張.
你看到他有五樣東西.
你看看你再拿一張.
這個是菲利士人剛好有的東西.
哥利亞.
他用五樣很無聊的東西對付別人.

$^{681}$五樣很厲害的東西.
我最近我女兒考試.
我得罪了我女兒.
她數學拿到高分.
很好的成績.
我又不長進.
我是很開心的.
其實很好.
我拿一份卷來欣賞.
然後我吃飯的時候放在桌面上.
吃完飯的時候.
其他家人就不知道為什麼.
把送頭送渣放在桌面上.
你知道所有責任都歸於我身上.
真的很慘.
他很珍惜.
為什麼這麼珍惜.
因為去年他開始中學的時候.
他第一個醒來的時候.
我沒有理會他.
你知道不理會很重要.
放手很重要.
放手就算了.
出來的時候就不太好.
到第二個醒來的時候.
我看到有些危機感.
開始英雄的爸爸要出場.
幫他數學幫他IS.
結果他數學考得很好.
所以他今年開學的時候.
我都沒有幫他.
從今年開始數學和IS都沒有幫他.
有空就做一下.
我都是打嘴炮.
我都是不想幫他.
看他自己來.
所以他高分的試卷.
是他很開心的.
我應該很自豪.
一個讀Maths Major的爸爸.

$^{721}$永久小女兒.
在數學的深淵.
他能夠自發圖強.
現在能夠過幾個月都能夠好成績.
應該很興奮.
但這幾個月.
有一件事是揮之不去地的.
我不能忘記的.
不可以忘記的.
是我女兒八月的時候.
她說要做團契的那件事.
我女兒八月的時候.
無端端來跟我說.
爸爸神跟我說話.
又神跟你說話.
又是性靈感人的朋友.
真的很煩.
經常跟我說.
說什麼呢.
是不是讀好數學和IS.
我說如果是就好了.
女孩能夠讀好IS和數學.
是不是一件很美好的事.
你明不明白.
不要那麼Stereotype.
女孩不懂數學才行.
不是的.
你們可以的.
我以為神跟他說這些.
他不是的.
他說神感動我.
要在團契做小助手.
小助手.
做完小助手.
有兩三天的退休營.
在她學校辦的.
專門給服務團契的人.
去退休.
她放了那兩天的退休.
寫的內容在桌面上.

$^{761}$我走進房間.
沒辦法.
那兩三天她還放在那裡.
我就說不好意思.
這樣就是邀請我看.
(笑聲).
是不是.
她兩三天還放在那裡.
都不收起來.
我看完.
我真的很感動.
她開始學什麼是侍奉.
學什麼是服侍上帝.
她安靜的時候.
她寫下服侍上帝是怎樣怎樣.
我不說內容了.
怎樣怎樣怎樣.
我很感動.
我最感動的是.
她學校的團契.
我做傳道運的時候.
2000年我服侍這個團契.
服侍到2012年.
12年.
所以她的老師.
我很熟很熟.
我從來沒想過是什麼.
是我女兒可以在這個團契服侍.
這兩件事合在一起是數學.
和她的團契服侍.
她現在逢星期五去服侍團契.
每次團契服侍完回來.
我都很八卦地問她.
怎樣啊.
不可以問太多.
她只回答OK.
OK就是很好.
她不會說內容.
她只說OK.
我聽到她說OK我就很開心.

$^{801}$我問自己.
她OK開心一點.
還是數學高分開心一點.
坦白說.
兩個都那麼開心.
但是.
但你問我裡面震撼的是什麼.
她從小.
她聽了神跟她說要做的事.
她現在去學師姐教她服侍.
用琴服侍.
她學Classic那些.
她說師姐教她按chord.
她學按chord去彈歌.
她說她要做小助手的時候.
她要怎樣幫手做Powerpoint.
怎樣去買東西.
我說神啊.
這個是我做不到的事.
這是我不會做得到的事.
她說不要彈絲琴.
又要撓女兒.
很難不會做這些事.
但一個人能夠從小開始.
她聽到她知道要做什麼.
她衷心去做.
成為了她人生裡往後要打更多難關的仗的時候.
她知道那個上帝還在.
那個從小跟她說話的上帝.
那個從小到大幫她的上帝.
到她人生遇到很多難關的時候.
那個上帝沒有走過.
仍然還在.
今天要說imagination新的東西的時候.
其實挑戰你跟我一件很真實的事情是什麼.
我們有多相信上帝可以在生命裡做些新的事.
祂可以超脫離我們現實的事.
我不想說得很超脫.
我女兒仍然學普通話.
我仍然逼我女兒去跑步.

$^{841}$做田徑.
有些事是很現實地她要去的.
但在現實裡總是有些空間.
上帝可以在那裡跟她做一些事.
令她們覺得上帝是真的.
昨天發生多件事.
小聲說.
我兒子不懂得串字.
他測驗的物書他不懂得串字.
他突然祈禱神幫他串字.
他祈禱後突然有人家的試卷飛到他面前.
傳上來.
他看到答案.
他發覺自己錯了.
神跟他說成神.
他把錯的答案交給老師.
每一個很重要的時刻.
遇到上帝的時候.
他選擇上帝.
那不是一個失誤.
那是一個很難得的經驗.
那個經驗叫你跟我對上帝的想像.
不是所有人都要滿分.
不是所有人都要身懷絕技.
現在在香港這個現況.
明天一件很重要的事情.
發生.
星期日唇拜.
很重要 每個星期都有.
有很多人會為自己建立一個王國.
很多人會為自己建立一個屬於他的王國.
如果你要問蘇羅和大衛.
回答我最初的問題的時候.
不是那兩個人的ethical problem有問題.
都有 兩個都不好.
但蘇羅和大衛最大的分別是什麼.
是同樣地做到那件事.
表面上做到那件事.
我們做了很多侍奉的果效.
但我們享受的是.

$^{881}$因為我讀數學.
我幫到女兒做到那件事.
我為她做到那件事.
我興奮.
還是同樣地做到那個果效.
但那件事不關我什麼事.
我只不過相信上帝會變出一些東西.
他真的變了.
同樣地侍奉有果效.
人們眼中看下去.
這件事都work的.
大家看起來都沒有分別.
但蘇羅和大衛最大的不同是什麼.
大衛永遠對上帝有imagination.
他不是做一個侍奉完成.
他不是有些事做完他fulfill了.
不是的.
人們跟他說有老婆有錢.
他說他在辱罵萬軍耶和華.
所以我不like這件事.
人們穿盔甲打仗.
我不穿.
我拿五個小石頭就甩他.
今天要說新的話.
要說型的話.
尤其是在這個時代.
明天過去之後.
我相信局面可能更加.
不知道會發生什麼情況.
但正正因為這樣.
就令凡信萬軍耶和華的子民們.
他有新的想法.
他可以對上帝有新的想像.
你可以見證的是.
我們這個群體裡面.
有很多人會做很多特別的事情.
是上帝使用他.
而叫我們其他人都啞口無言.
而我們正在證明和見證的是.
萬軍耶和華的榮耀真的在我們當中.

$^{921}$我再說一次.
大衛不是一個figure.
引千千萬萬人跟從他.
那些千千萬萬人跟從他是什麼.
是跟從他做到那件事.
但真正的在體.
真正大衛得到的地方是什麼.
是他想每一個人都有.
他對上帝的imagination.
上帝的想像 上帝的力量.
可以在我們想像不到的地方出現.
你記不記得《詩篇》第八篇.
我做完這句就總結.
《詩篇》第八篇說什麼.
我最厲害的那首是什麼.
「便說人算什麼」.
「你竟顧念他 他算什麼」.
你記得歌詞嗎.
這首詩歌裡面沒有一句.
他作這首詩歌的時候.
他有一句經文沒有了.
「從鷹牙嘴中得到能力」.
你想想做一個王.
大衛做一個王.
他想說什麼.
「便說人算什麼」.
我做一個王.
我算什麼.
為什麼.
因為萬鈞爺說.
他在鷹牙嘴中得到能力.
就算一個小朋友.
一個沒能力的.
在拍片的小朋友.
上帝可以使用他也贏.
這是imagination的問題.
今天.
看著明天過去之後.
現實是很真實的.
但這個時候同樣是一個什麼時候.

$^{961}$是一個我們可以體會.
上帝可以在這個世界裡.
他有千千萬萬的想法.
會work out那些東西.
今天不是人去跟從新的東西.
是我們每一個人生命的本質裡.
都越來越操練對上帝.
在我生命裡面.
應該要有的東西.
我一齊在這裡禱告.
天父多謝你給我們今天這個空間時間.
我們不明白為什麼.
他們keep out上來的時候.
imagination能夠打贏.
就好像.
我們不明白為什麼大衛可以拿著五塊石頭就打贏.
我們不明白為什麼.
我們不明白為什麼大衛可以拿著五塊石頭就打贏.
哥利亞一樣.
一樣都好像很無聊.
一樣都好像.
跟我們的想像.
跟我們的想法.
完全不一樣的東西出現.
天父我求的是你.
讓我們看著這些故事的時候.
不是覺得很無聊.
很白癡.
不是人做到.
我們裡面已經好像那些.
其他人一樣.
都承認是膽怯.
但天父我求你在這個世代裡面.
讓我們對你.
在我們生命裡面所做的事.
多一份對你的想像.
是你會有你的想像在我生命的面前.
是你會在我們前面鋪一條路.
是超乎我們所想所求的一樣.
天父我求天父你親自與我們同在.

$^{1001}$你親自帶領我們每一個弟兄姊妹.
是我們每一個人都經歷.
每一個人都越來越相信.
有這件事在我們生命裡面會出現.
昨晚君爺說你親自憐憫.
在這個世代裡面.
很多人已經為你將生命擺上.
望著烏克蘭的土地.
過去的星期.
大片的土地被人炸到.
有多少人傷亡.
天父我仍然在這個絕望的裡面.
你讓我們對你有想像.
求你親自成就在我們當中.
讓我們見證大會和哥利亞的故事.
今日都會在我們裡面出現.
求你親自憐憫和保守帶領.
多謝天父你向我們面前的祈禱.
奉求耶穌保衛您的名字而求.
阿們.
\newpage



\section{詩篇 121:1-8-20221022}
\label{sec:l2LEDZopMVg}
\textbf{【網上崇拜】惡人舞動|詩篇121\_1-8|20221022 [l2LEDZopMVg]}
\newline
\newline
連結: \href{https://youtube.com/watch?v=l2LEDZopMVg}{\texttt{ https://youtube.com/watch?v=l2LEDZopMVg}} ~~~~ 語音日期: 2022-10-22 
\newline
\newline
\hyperref[sec:68fGAExhs0o]{\small{< < < PREV SERMON < < <}}
~
\hyperref[sec:index_chronic]{\small{[返順時目]}}
~
\hyperref[sec:index_scriptual]{\small{[返順卷目]}}
~
\hyperref[sec:rw_6I9ppxNw]{\small{> > > NEXT SERMON > > >}}
\newline
\newline
詩篇 121:1-8-20221022
\newline
\begin{longtable}{cl}
\hline
\hline
章節 & 經文 (和合本修訂版)\\
\hline
121:1 & \begin{tabularx}{0.7\textwidth}{X} 我要向山舉目,我的幫助從何而來? \end{tabularx} \\ \\ \relax
121:2 & \begin{tabularx}{0.7\textwidth}{X} 我的幫助從造天地的耶和華而來。 \end{tabularx} \\ \\ \relax
121:3 & \begin{tabularx}{0.7\textwidth}{X} 他不叫你的腳搖動,保護你的必不打盹! \end{tabularx} \\ \\ \relax
121:4 & \begin{tabularx}{0.7\textwidth}{X} 保護以色列的必不打盹,也不睡覺。 \end{tabularx} \\ \\ \relax
121:5 & \begin{tabularx}{0.7\textwidth}{X} 保護你的是耶和華,耶和華在你右邊蔭庇你。 \end{tabularx} \\ \\ \relax
121:6 & \begin{tabularx}{0.7\textwidth}{X} 白日,太陽必不傷你;夜間,月亮也不害你。 \end{tabularx} \\ \\ \relax
121:7 & \begin{tabularx}{0.7\textwidth}{X} 耶和華要保護你,免受一切的災害,他要保護你的性命。 \end{tabularx} \\ \\ \relax
121:8 & \begin{tabularx}{0.7\textwidth}{X} 你出你入,耶和華要保護你,從今時直到永遠。 \end{tabularx} \\ \\
[1ex]
\hline
\hline
\end{longtable}
$^{1}$各位弟兄姊妹平安.
We are the child of God.
我剛才一起敬拜的時候.
覺得很感動.
我相信大家也是在一個敬拜的裡面.
我今天想跟大家說的是.
詩篇121篇的經文.
加上上個星期說.
他想了很久很久.
他想不到為何上帝是選大衛.
而不是選數羅.
其實同樣的這篇詩篇.
我也是掙扎了很久.
我的掙扎是甚麼呢.
就是為了上帝不睡覺.
這樣保護我們的訊息.
他說你出你入.
從現在到永遠.
是保護著我們的訊息.
在現在我們是怎樣理解.
和我們是怎樣相信.
在這兩個月.
在我身邊的朋友.
組員很多都應該聽過我的掙扎.
我上網上.
在座的聽過我的掙扎的.
突然之間我覺得辛苦你們了.
因為我應該說了很多很多.
而事實上.
我在預備的過程當中.
我是一次又一次.
很想放棄這段經文.
因為我只是在理性的層面上.
我理解上帝保護我們.
我只是可以在歷史的向導下.
去明白上帝一直以來.
都有不同的形式去保護著我們.
就好像我們回顧我們的一生.
我們一定會說多說少.
我們都會很感恩上帝的保護.

$^{41}$這些都是真的.
不然我們今天都不會坐在這裡.
但我的掙扎的位置是.
人在當下面對困苦的時候.
我很難用詩篇121篇去說.
上帝無時無刻去保護我們.
可以去回應得到.
我在想會不會是.
詩人你是不是說得太大了.
你講說我會保護你的性命.
我會幫你脫離一切的邪惡.
一切的困難.
一切的災禍.
會不會說得太大了.
我們拿著保險.
有時候我們都會買保險.
保險都不會這樣包你.
但我們現在這八節的經文.
就好像包你一輩子.
包你全部一樣.
但我們現在好像身在一個.
很不容易的環境裡面.
所以今天第一個迎新.
我想跟大家一起對這個.
121篇的經文的迎.
有一個新的想法.
而第二個迎新的是.
我們基於我們對這段經文的想法.
我們可以在這個不容易環境當中.
可以知道我們可以怎樣應對.
我們一起看看.
120篇的經文.
我今天會嘗試用一個.
古近東神明的角度.
跟大家一起看這篇詩篇.
雖然我們不能確定.
這121篇是在秘魯期間.
還是回歸之後的作品.
但無論是哪一個階段.
以色列人其實都已經吸收了.

$^{81}$很大量有關古近東神明.
又或者是巴比倫神話的文化.
所以我們一起戴著這副眼鏡.
這個古近東神明文化的眼鏡.
我們一起看一看整首121篇.
我們第一個powerpoint.
第一節.
他說「我舉目向著縱山」.
這是我的翻譯.
大家有沒有想過「縱山」是甚麼?.
這裡其實可以有很多的解釋.
但在當時的文化背景來說.
如果說「縱山」在這裡說是外邦的神明.
應該會比較貼切.
為甚麼這樣說呢?.
原來當時的人會在山上.
建很多神明的神壇和神寺.
在山上也是這些神明活躍活動的地方.
而對當時的人來說.
這些神明就是他們的保護者.
但對於以色列人當然不是.
所以當先人說他建築縱山的時候.
然後說「我的幫助從何而來?」.
他說的是.
其實我看著前面有那麼多的邪魔外道.
怎麼辦?.
這就是縱山和神明之間的關聯.
而第三,第四節.
我們下一個powerpoint.
他說「夜戶說不睡覺不打頓」.
私人在這裡為甚麼要說睡覺打頓?.
我以前會以為是私人的私儀.
私人有自己的獨特的私儀.
我們是不能夠參透的.
我都會想.
是否隨時隨刻24小時都保護著我們呢?.
你現在看不到.
但其實我說這句的時候.
我的嘴角有上揚.
因為我覺得整件事好像很浪漫.

$^{121}$即是有一個人24小時都保護著你.
但如果我們從神明的角度去看的話.
其實就是兩件事.
因為古近東的神明是需要睡覺的.
其實聖經裡面曾經提過.
不知道大家有沒有印象.
例如在《獵王記》上第18章的時候.
那裡是說.
「以利亞串了幾百個巴力先知」.
第18章27節說甚麼呢?.
「到了中午,以利亞嘲笑他們說:.
大聲呼求吧!因為他是神,也許他正在默想,或事無繁忙,或正在旅行,或正在睡覺,你們要把他叫醒.」.
我們以往看「以利亞串了幾百個巴力先知」的時候.
我們以為只是隨便串一下而已.
但原來他所說的都是言之有物的.
他說:你的神是不是睡了覺?.
你就把他叫醒吧!.
所以原來他們的神明是真的要睡覺.
這裡讓我想起.
我們未有去旅行之前.
我們經常去宅度假.
我們宅度假去酒店的時候.
我們也會掛一個牌在門口.
我們也會寫著「Do not disturb」.
原來神明也是要「Do not disturb」的.
如果打擾了神明睡覺的話.
我在文獻裡看到.
原來是很嚴重的.
我們可以看看下一張的slide.
古巴比倫的文獻裡.
其實他記載了一個洪水滅世的事.
有一段是這樣說的.
整段我不讀出來.
但原來洪水滅世的原因.
是因為人類的噪音.
已經吵到神明睡不著覺.
他說:I am losing sleep over their loud doings.
所以就滅了人類.
其實還有很多說是因為吵著神明睡覺.
後果很嚴重的事情.

$^{161}$但我不可以在這裡再說下去.
因為我最初寫這個講章的時候.
我寫得很生氣.
我不停地寫.
我老公看到螢幕後.
他問我:你整篇的主題是不是睡覺?.
你寫了這麼多關於睡覺的東西.
我立刻醒覺.
我立刻醒覺停了筆.
所以我們回歸主線的是.
原來斯人在第三至第四節.
他說:耶和華不睡覺.
不打頓.
其實是有一個稻草人.
他在中山那裡看到的那堆神明.
我們繼續下去.
去到第六節.
大老子說:日間太陽不傷你.
夜間月亮也是.
大家應該會循著這個思路去想到.
太陽月亮代表甚麼呢?.
記得我們今天是拿著神明的角度去看.
其實我們在舊約的聖經裡面.
就會發現到.
原來日月是我們以色列人拜祭的其中兩個偶像.
我們看看《聯王紀》第23章第5節.
他說:他廢除了從前猶大烈王所立.
在猶大各城的休壇和耶路撒冷的周圍.
焚香拜偶像的祭祀.
又廢除向巴黎日月星辰和天上萬象焚香的人.
《約伯記》31章26至28節其實也有說.
他說:我若見太陽照耀或明月行在空中.
以致心中暗暗地受到迷惑.
用自己的嘴親手.
那麼這也就是該受審判的罪孽.
因為我欺控了高高在上的神.
除了聖經之外.
我們應該都可以想像得到.
在巴比倫的時候.
都有太陽神和月神.

$^{201}$我們打遊戲機的時候.
其實在座有沒有人.
打遊戲機的時候.
有些名字其實就是太陽神和月神的名字.
有些.
嘩 很厲害.
他們是不是要出場了.
我先叫一下他們的名字.
太陽神叫沙瑪.
月神叫新.
太陽神代表的是法律.
正義.
還有他還可以驅魔.
所以有病的人.
原來他們會相信.
如果我向沙瑪十祈禱的時候.
是可以幫我們驅魔.
可以擺脫那個苦境.
我們下載下一張powerpoint.
我們可以看一下太陽神.
因為我上網發現.
原來羅浮宮有一個浮雕.
是黑了太陽神.
左邊那個.
坐在那裡的那個就是.
而月神.
我們以為應該是女人.
但發覺不是.
他是一個男人.
還是一個很老的男人.
而且他更厲害.
他有很大量的稱號.
例如是眾神之父.
眾神之主.
其中一個很特別的是.
他說月神是一個萬物的創造者.
所以我們看到第六節.
關於太陽和月亮.
和神明之間的關係.
就是在這裡.

$^{241}$最後我們再看第七節.
第七節耶和華說.
祂會保護你脫離一切的災禍.
大家有沒有想過.
一切的災禍包什麼.
你的保險說.
包你一切的災禍.
但其實下面也有很多不同的細點.
一切災禍是什麼.
原來我們看原文的時候.
災禍是介Evil.
是一切的邪惡.
而這一切的邪惡.
指的不單是第六節所說的太陽神和月神.
更加是包含所有的邪惡.
所以我們這樣看整篇.
下一張slide是.
我們整篇詩篇其實.
都充斥著很多神明偶像.
充斥著很多邪魔愛道.
而我想說的是.
詩人就是面對一個這樣的環境.
所以當我們看回第一節.
詩人說我舉目向著眾生.
那裡來我的幫助的時候.
其實詩人是描寫著自己看著這個世界.
看著這麼多的邪魔愛道.
去呼喊說.
怎麼辦.
誰可以幫到我.
我們不完全肯定.
詩人是什麼時候寫這首詩歌.
所謂上行之詩.
那15首其實都是收集回來.
而形成一本的詩集.
但是其實從詩篇的本身.
我們可以想像.
詩人所面對的生活是很不容易的.
我們很多時候看這篇詩篇的時候.
我們都不會覺得原來詩人是身處一個.

$^{281}$這麼困難的環境.
可能我們會以為.
詩篇121篇的詩人.
是叫我們現在舉頭望向天.
我們舉頭望向天.
我們就會有對上帝的信心.
但應該不是這樣.
反而是詩人和我們一樣.
她是看著現實世界.
然後覺得很壓迫和很無力.
然後去到第二節的時候.
她就說我的幫助從耶和華而來.
創造天地定.
在這裡出現了一個.
我們應該會懂得的修辭手法.
頂真.
我們懂得的,對吧?.
我們還記不記得.
我們小時候上中文課的時候.
我有點害怕.
頂真就是說.
上一句的結尾.
和下一句的開頭.
是用相同的字的.
詩人第一節是用我的幫助去結尾的.
第二節是用我的幫助去開頭.
這是頂真,是一種修辭手法.
其實這種手法.
我們讀下去的時候.
我們就是感受到.
原來詩人真的很需要這份幫助.
她的初衷是什麼?.
她寫這篇詩篇的初衷就是.
我想尋求幫助.
我想我們都可以很跨時空地.
和詩人共情這件事.
因為我們這三年多的時間.
其實我們和詩人一樣.
都被很多的邪魔外道包圍著.
他們在我們眼前生崩活跳.

$^{321}$不亦樂乎.
而我們同樣在問.
究竟從哪裡來我的幫助?.
然後詩人就說.
我的幫助從耶和華而來.
創造天地的.
她這樣說.
平時我們中文會說.
我的幫助從造天地的耶和華而來.
但原來原文的結構.
「創造天地的」這五個字.
在後面額外加上去去形容耶和華.
我舉個例子.
就好像說.
家Sir專爆子女避身的.
又例如說.
清心專上講家Sir做勵志的.
這些就是額外加上去.
中文翻譯本身.
我們做的或講的都沒有錯.
我的幫助從造天地的耶和華而來.
其實都是在形容的.
但可能因為我們平時都是這樣說.
造天地的耶和華.
我們說得很順口.
所以我們可能會錯過了.
創造天地的這五個字.
其實是詩人刻意加上去.
如果是這樣的話.
我們之後會問一個問題.
我們會問.
為什麼是創造天地的.
為什麼是這五個字.
它可以說很多其他形容.
萬主之主,萬王之王.
都可以.
為什麼是創造天地的.
當我去看古晉都文獻的時候.
我發現了.
原來在巴比倫.

$^{361}$甚至在巴比倫以前的神明.
他們的創造之神.
是在創造之後.
去休息的.
而這個休息的舉動.
是有一個特別的含義.
是代表著這個創造之神.
去轉移.
去離開他所創造的地方.
又或者是說.
這個創造之神已經對這個地方.
沒有了興趣.
我們這樣會不會點擊到.
我們應該會點擊到.
詩人想說的是.
耶和華天地的創造者.
不是這樣.
所以詩人用了.
接下來的三至八節.
去解釋耶和華.
不像古晉都神明那樣.
去解釋耶和華.
跟他所創造的世界.
很貼近.
跟他所創造的人很貼近.
這個就是第三至八節.
我們可以看的角度.
我們一起看看.
例如第三至四節.
詩人不停地強調.
我們唱歌的時候.
會發現到.
上帝沒有休息.
上帝沒有睡覺.
詩人想說的是.
其實這些邪神會睡覺.
但耶和華不會.
在古巴比倫流行一個祈禱.
這個祈禱是對著他們的.
Star God.

$^{401}$不是Star Lord 是Star God.
是他們的星星神.
他向星星神祈禱的是什麼呢.
他說因為Great God.
是大神.
他需要睡覺.
夜晚的職位星星神.
你可不可以做了.
因為是頂上的.
所以不能做足.
只能維持有限度的服務.
當他們祈禱的時候.
可能有一半的安全感.
但另一半.
覺得不夠保護.
他們都覺得很害怕.
詩人在第三至四節.
正正就是說.
耶和華作為創造主.
祂就不會讓你害怕.
因為祂不打盹.
因為祂不睡覺.
我們在任何時候.
都不需要因為上帝不在而覺得害怕.
詩人就說.
祂是這樣貼近.
祂所創造的人.
例如第五至第六節.
祂強調上帝會看著我們.
無論日夜.
都不會被太陽神和月神.
傷害我們.
很特別的是第五節.
祂形容創造主會在你的右手邊.
成為你的隱蔽.
如果我不是.
要說到.
我也不會留意到.
右手邊就是右手邊.
隱蔽就是隱蔽.

$^{441}$為什麼要說你的右手邊.
你的隱蔽.
這是看一些學者的分享.
我才發現才知道.
原來是在形容.
打仗的時候.
一般的右手會拿著攻擊性武器.
而左手就拿著.
防護性武器.
例如盾.
所以通常敵人.
會攻擊你哪一邊.
右手邊的舉手.
左手邊的舉手.
竟然有左手邊.
為什麼你能舉手呢.
你有盾啊.
為什麼呢.
OK.
因為一定是攻擊你右手邊.
你不要想著.
你違法不破.
拿著劍.
敵人攻擊不到.
沒有啦.
所以現實是什麼.
在戰場上一些重要的人物.
會被配一個兵在右手邊.
幫他拿盾.
所以當詩人說創造主.
會在你右手邊.
用你的陰秘的時候.
其實是在說敵人想攻擊你的弱點的時候.
創造主會像.
拿著盾那樣.
去保護你.
所以我們就明白.
為什麼詩人會形容陰秘的時候.
要加個「你」字.
說是你的陰秘,你的右手邊.

$^{481}$說了這麼久.
詩人用三節六節.
用不睡覺你的陰秘,你的右手邊.
去形容耶和華作為創造主.
他怎樣沒有退出.
創造的世界.
去形容他怎樣貼近創造的人.
其實還有.
很豐富.
有一個很特別的地方.
詩人原來很強調耶和華這個名字.
耶和華這個名字.
當然我們可以譯作上主.
中文三個字兩個字其實都可以.
重點是.
詩人很強調是哪一個.
第五節他說.
耶和華是保護你者.
你的陰秘,在你右手邊.
第七節到第八節.
詩人更加.
用了很多「他」這個字.
在中英文.
未必會看到全部.
但我們看下PowerPoint.
就會看到第七節到第八節.
他說耶和華.
他會保護你脫離一切的.
邪惡,他會保護.
你的生命,第八節他說.
耶和華他會保護你.
出你入,從今直到永遠.
詩人很強調.
是耶和華.
是耶和華他自己.
不好意思,我又說一說.
加Sir,我的感覺就是.
加Sir會教你希臘文.
加Sir會在你.
考希臘文試的時候.

$^{521}$在你身邊教導你.
加Sir他會保證.
一定讓你一輩子都記得希臘文.
他會讓希臘文.
成為你的生命.
加Sir他會在任何.
時間任何地點.
教導你希臘文.
從今天直到你.
人生完結的時候.
什麼感覺呢.
你們說說什麼感覺.
當然是很開心的感覺.
一定很開心.
沒有其他的.
只有開心.
所以整件事是什麼.
整件事是詩篇.
121篇,五至八節.
其實是和我們細細道來.
這個創造主.
耶和華他自己.
怎樣親自向.
這些被造物走近.
說了這麼久.
當我預備的時候.
我想是這裡的時候.
其實我.
在問詩人一個問題.
我開始在想.
詩人你說了這麼多.
詩人你.
得到幫助了嗎.
你最初的初衷.
你得到回應了嗎.
然後.
我又不見詩人有寫.
所以我就困在這裡.
所以我就很想放棄.
這個詩篇.

$^{561}$我說你說了這麼多.
但你都沒有寫.
實際的結局是怎樣.
然後我就想,其實他不寫也好.
因為寫了可能.
有一點成功見證的感覺.
你成功不代表我成功.
但是.
你什麼都沒有寫.
詩人我又很不明白.
你憑什麼去懷疑.
說創造主很積極地參與.
這個世界.
難道只是說,詩人你很有信心.
但我又很難.
在現在.
和自己.
和身邊.
和我的組員.
和我的所有人.
說,是我們就是.
我要很有信心.
所以我想了很久.
很久.
這篇詩篇,我們到現在.
究竟是怎樣去理解.
然後突然一刻.
我發現.
121篇的重點.
或許不是.
詩人有沒有見到葉和華.
幫助到她的某一件事.
所以是哪一件事.
或許都不重要.
反而是.
詩人感覺她在一生的路途上.
如果沒有了上帝.
她的腳就會跌倒.
在每一天像戰場一樣.
生活的裡面,她就會被攻擊至死.

$^{601}$所以是.
詩人知道.
只有上帝可以幫到她自己.
只有上帝可以幫到她.
而同一時間.
我們一起看看.
詩人雖然在整篇的詩裡面.
寫了很多群魔亂舞.
而且每件事都迫在眉睫.
你看到她從一節到第八節.
不斷地說,我感覺到迫力.
但當她說.
上帝保護的時候.
她所強調的.
她一直都是.
上帝和她之間的事.
我們看看經文.
除了之前說的.
很多關於她之外.
其實還有很多個.
「你」出現了.
第三節,她說.
她不讓你的腳搖動.
她不打斷地保護你.
第五節,耶和華是保護你者.
耶和華成為你的.
隱蔽,在你右手邊.
第六節,日間太陽不傷你.
第七節,耶和華.
她會保護你脫離一切邪惡.
她會保護你的生命.
第八節,耶和華.
她會保護你的出入.
從今直到永遠.
八節當中有五節的內容.
都是說耶和華和你之間.
整篇詩篇只有八節.
但很奇妙地.
產生了一種感覺.
就好像說.

$^{641}$很多很多的邪神.
在人生的旅途裡.
一直都在攻擊著.
很影響我們.
但很奇怪的是.
耶和華又沒有說.
沒有理會他們.
耶和華連眼尾都沒有.
看過這些邪惡的一眼.
詩人沒有說.
耶和華會打敗他們.
這一點很特別.
我們一說耶和華.
就是以色列人.
應該說.
為以色列人打敗敵人.
如果我們說耶和華的時候.
例如出埃及也好.
進加拿大也好.
耶和華都會為我們打倒外族.
打倒那些.
放在我們面前的仇敵.
我們再說一點.
有些學者其實有提到.
所謂出埃及降十災.
其實每一個災.
就是對應著一個神明.
同樣地加拿大.
當然也有他們的神明.
而耶和華也打敗了.
但這次詩篇121篇.
詩人不是說上帝也會打敗.
這些神明.
不是,詩人形容的.
最多是在於他和你之間.
他和你,耶和華和你.
詩人強調的是.
耶和華和我們之間的一切.
耶和華所有的動作.
都不是對著神明做的.

$^{681}$他所有的動作.
都是對著我們做的.
詩人在面對這個現實世界的時候.
我想他是醒覺了一件事.
邪惡包圍著我們.
是現實.
就像我們每一天.
都和詩人一樣.
在經歷.
面對家人的禍患.
面對窒息的生活和社會的時候.
我們都可以.
和詩人共感.
我們同樣地在呼喊上帝.
只是他醒覺.
在面對現實處境的時候.
無論他身處在.
哪一個外族統治之下.
究竟他是巴比倫也好,是波斯也好.
無論是什麼也好.
耶和華和他自己之間.
才是一個應對這群.
群魔亂舞的關鍵.
才發現自己.
有多需要這位耶和華上帝.
我們當然都知道.
我們需要上帝.
但或者不同的是.
當我們長時間.
看不到他保護的時候.
有時候我們.
應該說.
或者我們.
有時候會有點想放棄.
我相信我們.
都掙扎過.
但我們開始有點不知道該怎麼做.
慢慢地在我們.
生活的裡面.
上帝和你之間的部分.

$^{721}$越來越少.
而少了之後我們又慢慢地.
覺得在生活上.
好像沒有太大的分別.
我們沒有了.
上帝都是這樣過.
我曾經聽到有人說.
他說他以前的書櫃.
放滿了神學書.
他不是神學生.
但他買了很多這些神學書.
他用了很多時間.
去看.
但他跟我說.
現在好像有一種感覺.
這些跟他的生活.
沒有太大的關係.
我最近有聽到一個朋友說.
以前有一個和他一起拍檔.
做青少年團契導師的人.
每個暑假都準備.
非常好的營地.
讓青少年團友.
去認識上帝.
他教得很好.
但現在.
他的拍檔回來教會.
只是因為他要帶女兒.
上主一行.
當我去思想迎新的時候.
我感覺一下.
我身邊的人同事.
我感覺到的其中一件事.
是大家都很努力.
但有時候.
心裡面都會有些放棄.
不是放棄上帝.
可能是.
放棄了.
上帝和我們之間的.

$^{761}$此時此刻.
其他人.
未必可以察覺到.
但我們心裡面.
其實我們自己都知道.
所以很想鼓勵大家.
當先人面對.
生活中現實的惡的時候.
他無力呼喊之後.
他從心靈裡面去醒覺.
一切一切.
都在於他和上帝之間.
耶和華和他之間.
才是一個應對.
人世間每一個當下的關鍵.
我想的東西是.
當我們.
每一個人.
認真地.
每一個人.
認清自己有多需要上帝的時候.
他就沒辦法因為見不到上帝.
而放棄.
所以.
我們不要將我們的信仰放在將來.
我們不要將我們的信仰放在將來的盼望那裡.
不要放棄我們自己和上帝的每一個當下.
而當說到上帝和我們之間的時候.
我想到以前的一個.
看卡通片的時候.
我們可以下載一張powerpoint.
那些主角.
面對怪物的攻擊的時候.
他會變身.
我很認真的.
我是想說.
我想解釋上帝和我們之間.
他會變身.
我是想解釋上帝和我們之間.
但我真的想到這些卡通片.

$^{801}$就是當很多怪物在攻擊.
他變身的時候.
然後有一陣光在圍著主角.
但通常都會圍很久.
幾十秒都在看著主角在轉圈.
我自己就會問.
為什麼怪物不在這個時候攻擊呢.
按理說這個時候主角應該是最弱的.
但每一次.
只要一升起這陣光的時候.
基本上.
你們都會看到怪物站在光的外面.
他們不會動的.
就這樣站在那裡.
也不會進來.
不知道為什麼他進不來.
而那陣光就會和主角纏繞在一起.
然後就變身.
好像穿那些器具.
Cindy Moon穿的.
手套那些.
但我自己就覺得.
這不是什麼.
總之就是.
我的光不會直接攻擊.
而是和光纏繞了主角.
變身之後.
就打倒了敵人.
還有一樣東西不知道大家有沒有發現.
打倒敵人.
是一招之間的事.
但整個卡通.
最高潮的.
往往就是光包著主角的那一下.
是光和主角.
之間的纏繞.
而更加特別的是.
這個高潮.
是卡通片.
播放都不會悶的.

$^{841}$每一集都會有這陣光出現.
只要主角要打怪.
每一集主角都會和這陣光.
糾纏在一起.
大兄姐妹.
讓我也有詩意一點.
因為我們現在說詩篇.
我不知道夠不夠詩.
帶了這麼大的東西.
我想讓這陣光.
都集集地出現在.
我們每個人生的.
每一個定格裡面.
集集地出現.
在我們人生.
好像放電影.
我們在演出的時候.
出現在我們每一個定格裡面.
我想詩篇是一個.
很需要用生命去感受.
和去做的.
體會的文字.
對於處於2022年的我來說.
我在詩人的詩歌裡面.
拿到一份共情.
和一份提醒.
讓我們可以在面對.
面前的一切的時候.
由心靈深處去覺醒.
只要.
是我們和上帝之間.
我的神魔鬼怪.
都是一個背景.
經常都在.
但是.
上帝的保護就是.
在他和你的前搖裡面.
讓我們一起祈禱.
天父上帝.
你最知道我們.

$^{881}$每一個人的心境.
我們現在究竟.
是一個什麼的狀態.
天父上帝說.
你是一個沒有離開.
你所創造的.
創造主.
我求你讓我們每一個人.
心的裡面.
如果有分隔.
我求主你幫助我們.
去連結在你的裡面.
如果我們有.
任何的.
問號.
任何的.
想法.
我們求主你.
在這個.
前搖的當中.
讓我們有.
生命的體會.
主你求你.
讓我們去到.
經驗去到體驗.
發覺我們其實.
當我們沒有和上帝.
和上帝在一起的時候.
其實我們不是想像.
之中那麼OK.
主你求你幫助我們.
沒有人可以.
幫助我們只有上帝可以幫助我們.
求你讓我們.
去到醒覺和體驗.
我們在人生的旅途上.
有多需要你這位上帝.
是很需要.
很需要你這位上帝.
以致我們不能放棄.

$^{921}$以致我們不能.
只是在這個信仰上面.
去等候一個所謂的.
末日的審判.
以致我們放棄了我們人生旅途上.
每一個和上帝.
可以一起經驗的.
每一個瞬間每一個的當下.
主你就求你在我們.
生命的裡面.
去到與我們一起.
去到面對前面所有的挑戰.
我們禱告奉主耶穌基督.
得勝的明智祈求.
阿母.
\newpage



\section{列王記上 19:9-18-20221029}
\label{sec:rw_6I9ppxNw}
\textbf{【網上崇拜】做新事唔一定要型|列王記上19\_9-18|20221029 [rw-6I9ppxNw]}
\newline
\newline
連結: \href{https://youtube.com/watch?v=rw-6I9ppxNw}{\texttt{ https://youtube.com/watch?v=rw-6I9ppxNw}} ~~~~ 語音日期: 2022-10-29 
\newline
\newline
\hyperref[sec:l2LEDZopMVg]{\small{< < < PREV SERMON < < <}}
~
\hyperref[sec:index_chronic]{\small{[返順時目]}}
~
\hyperref[sec:index_scriptual]{\small{[返順卷目]}}
~
\hyperref[sec:fVOAbAyMqQo]{\small{> > > NEXT SERMON > > >}}
\newline
\newline
列王記上 19:9-18-20221029
\newline
\begin{longtable}{cl}
\hline
\hline
章節 & 經文 (和合本修訂版)\\
\hline
19:9 & \begin{tabularx}{0.7\textwidth}{X} 他在那裡進了一個洞,在洞中過夜。看哪,耶和華的話臨到他,說:「以利亞,你在這裡做甚麼?」 \end{tabularx} \\ \\ \relax
19:10 & \begin{tabularx}{0.7\textwidth}{X} 他說:「我為耶和華-萬軍之神大發熱心,因為以色列人背棄了你的約,毀壞了你的壇,用刀殺了你的先知,只剩下我一人,他們還要追殺我。」 \end{tabularx} \\ \\ \relax
19:11 & \begin{tabularx}{0.7\textwidth}{X} 耶和華說:「你出來站在山上,在耶和華面前。」看哪,耶和華從那裡經過。在耶和華面前有烈風大作,山崩石裂,耶和華卻不在風中;風後有地震,耶和華也不在其中; \end{tabularx} \\ \\ \relax
19:12 & \begin{tabularx}{0.7\textwidth}{X} 地震後有火,耶和華也不在火中;火以後,有輕微細小的聲音。 \end{tabularx} \\ \\ \relax
19:13 & \begin{tabularx}{0.7\textwidth}{X} 以利亞聽見,就用外衣蒙臉,出來站在洞口。聽啊,有聲音向他說:「以利亞,你在這裡做甚麼?」 \end{tabularx} \\ \\ \relax
19:14 & \begin{tabularx}{0.7\textwidth}{X} 他說:「我為耶和華-萬軍之神大發熱心,因為以色列人背棄了你的約,毀壞了你的壇,用刀殺了你的先知,只剩下我一人,他們還要追殺我。」 \end{tabularx} \\ \\ \relax
19:15 & \begin{tabularx}{0.7\textwidth}{X} 耶和華對他說:「去吧,從原路回去,往大馬士革的曠野去。到了那裡,你要膏哈薛作亞蘭王, \end{tabularx} \\ \\ \relax
19:16 & \begin{tabularx}{0.7\textwidth}{X} 又膏寧示的孫子耶戶作以色列王,並膏亞伯‧米何拉人沙法的兒子以利沙作先知接續你。 \end{tabularx} \\ \\ \relax
19:17 & \begin{tabularx}{0.7\textwidth}{X} 將來逃過哈薛之刀的,必被耶戶所殺;逃過耶戶之刀的,必被以利沙所殺。 \end{tabularx} \\ \\ \relax
19:18 & \begin{tabularx}{0.7\textwidth}{X} 但我在以色列中留下七千人,是未曾向巴力屈膝,未曾親吻巴力的。」 \end{tabularx} \\ \\ \relax
19:19 & \begin{tabularx}{0.7\textwidth}{X} 於是,以利亞離開那裡走了,遇見沙法的兒子以利沙;他正在耕田,在他前頭有十二對牛,自己趕著第十二對。以利亞經過他,把自己的外衣搭在他身上。 \end{tabularx} \\ \\ \relax
19:20 & \begin{tabularx}{0.7\textwidth}{X} 以利沙就離開牛,跑到以利亞那裡,說:「請你讓我先與父母吻別,然後我就跟隨你。」以利亞對他說:「因我對你所做的事,你去吧,然後回來。」 \end{tabularx} \\ \\ \relax
19:21 & \begin{tabularx}{0.7\textwidth}{X} 以利沙離開他回去,宰了一對牛,用套牛的器具煮肉給百姓吃,隨後就起身跟隨以利亞,服事他。 \end{tabularx} \\ \\
[1ex]
\hline
\hline
\end{longtable}
$^{1}$Floor Church 弟兄姊妹平安.
參加了Outflow Mission.
已經開始了一起聚會的.
在倫敦,曼城,溫哥華和悉尼的弟兄姊妹都與你平安.
參加了Outflow Mission.
正在等候目者一起起步.
去建立一個同行群體的.
在Melbourne,在多倫多,在Birmingham的弟兄姊妹.
你們都平安.
很開心可以與大家一起在這裡敬拜,讚美神.
一起思考神的話語.
時間過得很快.
不知不覺又到了這次迎新月題的最後一講.
我今天在這裡預備了一篇人物的講道.
與弟兄姊妹分享.
這次分享的人物裡.
總共有七千人.
選擇他們的原因不是因為他們人多勢眾.
也不是因為他們有什麼豐功偉績.
更加不是因為他們有什麼型男索女.
他們都沒有行過什麼神跡騎士.
相反,我選擇他們的原因.
正正因為他們和我們一樣平凡.
不過,在我密上他們有關的經文裡.
我發覺這群平凡的人有一個很特別的地方.
原來這群人和我們相隔二,三千年的人.
他們在當時的社會環境,生活經歷,生命挑戰和掙扎.
都和我們這群平凡的人很相似.
所以我很想透過他們的經歷.
他們走過的路.
借骨鑒金.
希望可以在我們這個充滿無力感的社會裡.
世代裡找到一絲方向和出路.
所以在分享之前.
我很想邀請弟兄姊妹和我一起讀出.
今天的講道經文.
今天的經文是在聶王記上的十九章九至十八節.
麻煩大家PowerPoint.
好,我們一起看.
他在那裡進了一個洞,就住在洞中.

$^{41}$耶和華的華臨到他說.
伊利亞,你在這裡作什麼?.
他說:我為耶和華盲君之臣大發熱心.
因為以色列人背棄了你的約,毀壞了你的壇.
用刀殺了你的先知,只剩下我一個人.
他們還要尋索我的命.
耶和華說:你出來站在山上.
在我面前,那裡.
耶和華從那裡經過.
在他面前有裂風大作.
崩山碎石.
耶和華卻不在風中.
風後地震.
耶和華卻不在其中.
地震後有火.
耶和華也不在火中.
火後有微小的聲音.
伊利亞聽見.
就用外焉蒙上蓮.
出來站在洞口.
有聲音向他說.
伊利亞,你在這裡作什麼?.
他說:我為耶和華盲君之臣大發熱心.
因為以色列人背棄了你的約,毀壞了你的壇.
用刀殺了你的先知,只剩下我一個人.
他們還要尋索我的命.
耶和華對他說.
你回去,從旺野往大馬色去.
到遙高哈舍作阿南王.
由高寧氏的孫子耶和作以色列王.
並告阿伯米荷拉人沙發的兒子伊利沙作先知接續你.
本來躲避哈舍之都的必被耶和所殺.
躲避耶和之都的必被伊利沙所殺.
但我在以色列人中自己留下七千人.
自未曾向巴力屈膝.
未曾與巴力親聚的.
我們一起禱告.
因為你的愛吸引我們.
因為你的愛招聚我們.
讓我們這群分隔在世界各地的弟兄姊妹.

$^{81}$可以因著你自己的愛我們在這裡聚集.
聚集在這個時空中一起敬拜去讚美你.
求你這一刻都與我們同在.
幫助我們,幫助孩子可以說得清楚.
幫助弟兄姊妹可以聽得明白.
幫助我們聽完之後.
可以有能力去實踐出來.
去榮耀去見證你自己的名.
祈禱奉主的名,阿們.
全本聖經只有兩個地方.
下一張,麻煩你.
是直接提及這七千人的故事.
他們分別記載在舊約.
剛才所讀的獵王記上十九章十八節.
另一個是記載在新約的羅馬書十一章四節.
但因為羅馬書十一章四節很明顯是引用.
來自獵王記上十九章十八節.
所以嚴格來說.
全本聖經只用了獵王記上十九章十八節.
一節經文,總共十二個希伯來字.
去記錄這七千人的故事.
因此如果我們要更全面地了解.
這七千人的生命故事.
我們可以更全面地去起他們的底.
我們就必須要借助其他的經文.
而這些經文分別分佈在獵王記上十六章至二十二章.
不過因為時間關係.
我們只會涉獵十六至十九章這四章經文.
所以今天我們可能會遊走很多不同的經文.
希望弟兄姊妹可以在當中和我們一起去思考這些經文.
首先讓我們先看看獵王記上十九章十八節經文所處的位置.
如果熟悉伊利亞先知求死故事的弟兄姊妹.
相信都會知道.
伊利亞先知在加密山上戰勝了四百五十位巴黎先知之後.
他滿心歡喜地以為.
整個阿哈和耶駒會悔改歸向神.
可惜現實卻不似預期.
阿哈和耶駒不單止沒有悔改.
反而變本加厲找人去追殺伊利亞.
因為伊利亞在極度失望和驚慌之下.

$^{121}$他急急忙忙地用了幾十天的時間.
輾轉地到河獵山的山洞.
在山洞裡.
神跟伊利亞說了五句話.
而這五句話就記載在弟兄姊妹剛才所讀的十節經文裡.
在這十節經文裡.
他先後重複了兩次問伊利亞你在這裡做什麼.
他的目的就是要伊利亞冷靜下來.
思考一下自己當下的情況.
然後就吩咐他去高納哈舌和耶駒和伊利沙.
他為了回應伊利亞在第十和第十四節裡的那兩句牢騷.
那兩句一模一樣的牢騷.
牢騷是說什麼呢?.
我為耶和華盲君之臣大發熱心.
因為以色列人背棄了你的約,毀壞了你的壇.
用刀殺了你的先知,只剩下我一個人.
他們說要尋索我的命.
為了要回應伊利亞這一句話.
他在伊利亞臨下山之前.
臨出山洞的時候.
他就加上一句.
但我在以色列中為自己留下七千人.
是未曾向巴力屈膝.
未曾向巴力親嘴.
於是這七千人的故事.
就在這樣的情況下被記錄在聖經裡.
提到這七千人.
神提到這七千人的其中一個目的.
是要告訴伊利亞.
當時在以色列人當中.
仍然專一信靠神.
除了伊利亞你之外.
還有七千人.
但同時間神又好好在這經文當中.
留下了有關七千人的兩方面的資料.
第一方面就是.
這七千人全部都是以色列人.
第二個資料就是.
這七千人是和伊利亞同時期的以色列人.
既然他們是屬於伊利亞同時期的以色列人.

$^{161}$所以我們就可以推斷.
他們應該和伊利亞一樣.
是生活在公元前九世紀.
根據聖經的記載.
這個時期可算是以色列人最叛逆.
最不信和宗教方面最混亂的時期.
當時以色列已經從大衛.
即是所羅門時期的聯合王朝.
一分為二.
分裂成為以色列國和猶太國.
兩個分裂王朝時期.
經文在第十八節說.
但我在以色列人中為自己留下七千人.
當中的以色列.
相信是指分裂後的北角以色列國.
而這七千人相信.
很大可能是生活在北角的以色列國.
那些星斗市民和善子.
熟悉這段歷史的弟兄姊妹.
相信都會知道.
位於北面以色列國.
在分裂初期.
在耶羅波安的帶領下.
開始偏離神.
去拜偶像.
在歷時七八十年後.
到了這七千人的年代.
在阿蝦的統治下.
更加變本加厲.
整個群體已經叛逆到極點.
因為根據獵王記.
上十六章三十三節的經文.
我們下一章可以看到.
他這樣形容.
阿蝦行而胡說眼中看為惡的事.
比他以前的獵王更甚.
他犯了李伯.
即是兒子耶羅波安所犯的罪.
又娶了西頓王傑巴力的女兒.
夜洗別為妻.

$^{201}$去侍奉巴力.
在瑟瑪利亞建造巴力廟.
在廟內為巴力祝壇.
阿蝦又作亞瑟拉所行的野.
惹禍說以色列神的怒氣.
比以前的豬王更甚.
這裡說他比以前的豬王更甚.
比以前的豬王更壞.
經文形容阿蝦.
是叛逆和不信道的極點.
如果再參考以利亞.
剛才我所說的.
十九章十節和十四節的路數.
以色列人背棄了神的約.
毀為神的壇.
用刀殺先知.
我們可以看到.
阿蝦王的叛逆.
除了自己去敬拜.
侍奉巴力之外.
他還在全國各地.
大力建造巴力神廟.
進一步引誘人民.
轉向敬拜巴力之外.
他還積極引導.
縱容以色列人.
去拆毀耶和華的祭壇.
去迫不及待.
甚至殺害信奉耶和華神的人.
這個情況就像早幾年.
ISIS那樣.
又或者像這幾年.
某些極權國家一樣.
毫不講理.
很粗暴地去拆十字架.
拆教堂.
甚至乎在政治迫害信徒.
關押信徒.
殺害信徒.
我相信這些圖像.

$^{241}$相信大家都歷歷在目.
不過以色列在阿蝦.
軟硬建師之後.
一方面建造巴力神廟.
一方面全力鼓吹全民拜巴力.
另一方面.
他又用全面管治.
不是,是全面打壓的形式.
去迫不及待.
去殺害不肯信服.
不肯信從他去拜巴力的先知和人民.
以致在當時整個以色列國.
當時應該有百多二百萬人口當中.
變節的變節.
改教的改教.
跪低的跪低.
殺害的殺害.
最後只剩下七千零一人.
是未向巴力屈膝.
未曾與巴力親嘴.
而在社會方面.
我們在十九章一至二節裡.
我們下一章的部分.
我們會看到這七千人.
生活在暴政濫權.
沒有法治的社會當中.
因為在這兩節經文裡.
我們看到當時阿哈將.
以利亞在加密山上所行的事.
都告訴耶誓別人.
耶誓別人居然可以越權.
在未經審判之下.
未經判罪之下.
他就決定找人去召集他.
去找人與他如約處決.
就像和以利亞說.
明日約在這個時候.
我若不使你姓名.
像那些人的姓名一樣.
願神明重重降罰我.

$^{281}$這七千人所處的社會狀況.
其實就像我們現在某些城市.
大家很清楚的城市一樣.
為了排除異己.
政權可以未審先囚.
未審先判.
這樣去迫害人民.
我們再看看他生活的經歷方面.
這七千人在以利亞的.
既然說這七千人是以利亞同期的人.
所以他所經歷的.
必定都和以利亞很相似.
尤其是所經歷的三年零六個月的.
乾旱時期.
這六個月裡的食物清零.
食水清零的乾旱日子.
因為根據獵王紀上.
十七至十八章的記載.
以色列人在阿哈的帶領下.
轉向敬拜和侍奉巴力.
因為他們相信.
巴力可以控制風雨和雲的大神.
可以降雨滋潤他們的大地.
可以使他們的土產.
出產豐富的農作物.
為了讓這班半日既以色列人看到.
耶和華才是掌管萬有.
掌管風和雨的獨一真神.
耶和華要使天閉塞三年零六個月.
不下雨不降雨.
這七千人因為掌權者.
和一班無知人的罪.
他們經歷了長達三年多的.
缺水缺糧.
食物清零.
食水清零的日子.
情況可能都很像我們現在的境地.
我們都是因為某些人的罪.
某些人的錯誤.
某些政權的錯失.

$^{321}$我們經歷了很多年靜態清零.
動態清零的艱難日子.
所以沒有像某些地區.
被迫強制隔離到食物清零.
食水清零.
甚至地震都要繼續被迫清零.
在這種荒謬的遭遇.
但是被強封.
被強檢.
被強制隔離.
被強迫打針.
那些日子真的很無奈.
很難受.
我在這個月頭都經歷了.
這種被強迫隔離的苦況.
在這裡我也學家Sir.
爆一下子女的鼻聲.
話說在上個月尾.
我大兒子中了招.
我們一家就馬上變成了密切接觸者.
根據法例.
我們最少要隔離七天.
但是過了三天.
我們一家大小.
除了小兒子之外.
陸陸續續由密切接觸者.
變成了確診者.
於是又要從頭數過七天.
在第六第七天裡.
要兩條線才可以出關.
我不知道自己有什麼特別的地方.
我全家都可以在第六第七天.
由二線變成一線.
我總是在二線.
足足等了四天.
我前前後後在家裡呆了十四天.
那種想死.
沒得出街去崇拜.
沒得出街去開祖.
沒得出街去踢球.

$^{361}$沒得出街去跑步.
沒得出街去跟弟兄姊妹吃飯.
那種人真的想死.
不過呢.
剛才以上的事情.
都不算是最衝擊這七千人的生命.
對於他們的生命最大衝擊的.
可能是以利亞在山洞裡.
剛才說的那兩句話的其中幾句.
因為以色列人背棄你的約.
毀壞你的壇.
用刀殺了你的先知.
當中用刀殺了你先知.
只剩下我一個人.
這裡可能在勾畫上.
應用上.
以利亞有一點誇大了.
因為不只剩下他一個人.
但他所提供的資料是很寶貴.
我們看到以色列在亞哈永續作王的二十年裡.
他對敬拜侍奉耶和華的人.
和先知的迫害.
那些先知可能拍拍手掌.
可能提醒官員.
你跌了東西.
你跌了良心.
都可能受迫害.
他們輕就囚禁.
重就殺無赦.
相信這幾千人在這個時期裡.
他們每天每個星期.
都會聽到很多先知.
很多敢言人士.
很多正直有良心的人士.
有很多願意為公義出聲.
為真理發聲的人.
一個一個被未審先囚.
未判先殺.
相信他內心的憤怒.
他生命的掙扎.

$^{401}$那種無力感.
可想而知是多重.
生活在一個不信不義.
無力感的世代裡.
這七千人又做什麼呢.
神在第十八節告訴我們.
他們是未曾向巴黎屈臣.
未曾與巴黎親嘴.
當中的巴黎.
剛才也說過.
是外邦人的偶像.
他被譽為掌管風雨.
並且可以使土地肥沃.
出產洋農作物.
豐收高位臣.
另一方面.
在那個時期.
巴黎也是阿哈王和耶洗別.
極力鼓吹廣泛推行的敬拜對象.
向巴黎屈膝.
與巴黎親嘴的動作.
一方面是當時巴黎的敬拜禮儀.
但神說這七千人.
未向巴黎屈膝.
未曾與巴黎親嘴.
除了表示這七千人.
一直忠心於耶和華.
信靠耶和華之外.
暗示他們一直沒有向阿哈.
向耶洗別.
這個邪惡的集團.
低頭妥協.
雖然神在這節經文中.
只記載了這七千人.
沒有做過什麼.
他們沒有妥協.
但他沒有在這裡交代.
究竟這群人做了什麼.
在這個無力感人的世代.
時期.

$^{441}$他們做了什麼.
他們是怎樣生活的.
不過聖經在其他地方.
最少有兩節經文.
都讓我們看到.
這七千人當中的部分人士.
他們當時是怎樣生活的.
第一節是在十九章十九節.
我們看看下面.
這節經文提到.
我們很熟悉的一個人物.
就是以利沙.
如果以利沙也是神口中的七千人.
其中一位.
我相信大家應該沒有人會反對.
經文告訴我們.
以利沙在接觸以利亞作王之前.
其實是一個很普通的農民.
雖然他相對是有錢一點.
因為他有十二隻牛.
但他每天所做的都是.
去烘牛去耕田.
在那個黑暗的時期.
他都是做一個普通人要做的事情.
他努力地工作.
去維持自己的生活.
去供應家人日常的需要.
他沒有實行過什麼很厲害的神蹟騎士.
他是當時.
另外一節經文是記在十八章第四節.
這節經文說.
耶誓別人殺耶和華種仙子的時候.
俄巴底將一百位仙子藏了.
每五十人藏在一個山洞裡.
拿餅和水去供養他們.
當中的一百位仙子.
我相信都是神口中所提到的.
七千人裡面的其中一百人.
經文說他們為了逃避阿孝和耶誓別人.
追殺逼迫.

$^{481}$他們接受了阿孝和俄巴底的幫助.
躲藏在山洞裡.
在那個黑暗的時期裡.
我們大家都猜到.
他們每天可以做的事情.
就是靜靜地在山洞裡.
好好地喝水.
好好地吃餅.
好好地保護自己的身體.
好好地保存自己的生命.
好好地和一群同路人相處.
不要聊著聊著就出去吃餿飯.
他們在當中互相支持.
互相圍爐取暖.
他們並沒有像以利亞那樣.
行那麼多又新又型又勁的神蹟騎士.
相信他們當中絕大部分的人.
試過和阿孝和耶誓別人正面對壘.
他們絕大多數的人.
相信都沒有像以利亞那樣.
叫死人復活.
降火燒盡飯祭.
禱告後降大雨.
停止三年零六個月的旱災.
他們沒有走過那麼型那麼勁的神蹟騎士.
他們甚至乎連去做旁聽.
去送車去送機去接機.
去用Patreon去支持被欺壓的人.
去堅持光顧良心企業.
去透過Airbnb去支持烏克蘭的人.
這些被人認為很小的事情.
很小的事情都沒有機會沒有能力去做.
更加沒有說要去領事館示威.
更加不會被人拉進去打.
他們在黑暗時期.
他們充滿無力感的時期.
他們唯一可以做的.
就是剛才我所說的.
躲在山洞裡好好照顧自己.
好好保存自己的身體.

$^{521}$好好和一群人和路人相處.
默默地去持守自己的信仰.
保持自己的初心.
不向巴力屈膝.
不向巴力親嘴.
不向邪惡的政權妥協.
保持對不公義事情的一種憤慨.
不習慣.
不盲目.
所以這七千人做的.
可能是一些很不起眼.
很小的事情.
但是神卻在第十八節那裡.
特地親自去紀念他們.
親口地在以利亞這位大先知面前.
這位走過神跡奇事的超型大先知面前.
去肯定他們.
去誇耀他們.
弟兄姊妹.
原來在神的心目中.
為神做身事.
為神去行公義好憐憫.
不一定要做好型的事情.
是不是?.
親愛的弟兄姊妹.
今天我們所處的世代.
可能比那七千人所處的世代.
更加不順.
更加黑暗.
尤其是經過上個星期之後.
我們看到那個高牆.
比以前更加高.
更加厚.
更加堅硬.
以致我們那種無力感.
更加嚴重.
我們可能正在懷疑.
我們過去所做過的事情.
所堅守的信念.
是否還有意義.

$^{561}$我們可能正在懷疑.
現在我們所做的每一件小事.
那種果後有多大作用.
我們可能正在猶豫.
單憑我們自己一人之力.
將來還可以做什麼身事.
最近在網上.
我看到一篇叫做.
無力感心理學.
人如何可以自處的文章.
這篇文章是折錄自.
恐懼與希望.
寫在亂世心理學這本書.
這篇文章對無力感有多好的剖析.
當中有一句說話.
我覺得他講得很好.
他說.
大家可以小聲跟我一起讀.
他說.
在亂世之中.
或許要與人喜歡.
有明確結果的天性抗衡.
摒棄後果主義的思考模式.
因為一來在亂世之中.
沒有人能夠準確地估算後果.
二來就算徒勞無功.
行動本身就能夠對抗無力感.
也令我們作為無負於人的生命.
這段說話提供了兩個對付無力感的方法.
第一個就是摒棄後果主義的思考模式.
第二個就是行動.
因為當摒棄後果主義.
放下了做生事一定要有型.
一定要勁.
一定要有用.
一定要有後果的心態.
我們就會看到.
其實還有很多做生事的行動空間.
我們還有很多創意可以發揮.
文章之後又舉了一個很好的例子.

$^{601}$他說沒有人能夠憑一己之力去阻止全球亂化.
是不是.
但我們卻可以.
這個我有一點點將他改變.
我們可以從關心環境.
少用膠袋.
節約用電.
這些小事上開始做起.
從一些我們可以控制的事情上去做起.
他的想法就和他在波斯尼亞的一個求生.
我今天找不到曼聯的例子.
因為想回應John經常用利物浦的例子.
我是曼聯的.
這位求生他怎麼說呢.
他說我沒有能力去決定自己和哪個國家隊去比賽.
但我有能力去決定去不去參加這個比賽.
一個很喜歡踢球的人.
他都願意放下自己去踢球的時間.
去將自己最喜愛的東西放下.
為敵.
就是要.
他怎麼說呢.
我永遠只會為和平而戰.
為公義而戰.
在這裡我又學一下潘Sir玩一下劇透.
不知道大家有沒有.
當中有多少弟兄姊妹呢.
看了《阿媽有了第二個》這套戲.
看了的舉手.
還沒看的那些想去看的舉手.
兩個.
這麼久都不去看.
我看過了八月上畫的.
這麼久都不去看.
那我劇透都不會得罪你了.
在YouTube上我就沒眼睛看了.
你再看我也不理了.
不過我也有良心的.
我沒有跌倒的.
《阿媽有了第二個》的詳細內容.

$^{641}$我真的不方便劇透了.
但是我只不過分享其中一幕.
其實我覺得自己很深刻印象的一幕.
就是關於小N的.
下一幅.
那個女孩的嘴巴很大.
我其實.
叫什麼鄧麗英.
老母啊我真的中了.
有老母證明我真的中了.
她在《方程大雄大子》那天.
成為天王巨星那天.
她說了一句.
那句話我又是有老母的.
我不是很清楚.
我大約記得她的意思是.
我沒想到自己可以在.
做天王巨星的工程裡有份.
她很興奮的叫了出來.
其實小N在這套戲裡.
只不過是一個很不起眼的角色.
她在這個過程裡.
她是負責去執頭執尾.
做一下文書.
送一下文件.
這些不起眼的工作.
主力捧紅方程的.
當然是無信軍所演繹的.
美鳳姐.
但最後呢.
小N也很驚訝自己.
竟然可以有份於造星大事當中.
沒錯.
我們未必每個人都像美鳳姐.
那麼有魄力.
有能力.
可以扭轉乾坤.
去化腐朽的神奇.
我不是說姜濤是腐朽.
他是神奇.

$^{681}$我沒有得罪你們.
千萬不要找我算帳.
我還想放心地走到樓下.
但我們可以像小N那樣去堅持.
堅持.
其實我這裡有一個.
但我們可以像小N那樣去堅持.
堅持在小事上盡心盡力去擺盤.
我用了什麼修辭修法呢.
清心.
我有學的.
當小N堅持默默地在小事上擺盤.
在不起眼的事情上.
不計果效地全力擺盤的時候.
他卻成為了造星計劃中一塊很重要的拼圖.
就像那七千人一樣.
雖然他沒有走過.
以利亞神跡歧視的好營的事情.
但他們卻單單在微小事情上.
在生活上擺盤.
他最終也有份於.
神在審判阿蝦耶洗別的計劃中.
成為重要的拼圖.
最終也得到神親自的紀念.
將它記錄在聖經裡.
我怕我會有一點成功神學.
但聖經真的這麼說.
親愛的弟兄姊妹.
不向巴力屈膝.
不向巴力親嘴.
在我們現代的意義是什麼呢.
我相信除了代表不拜黃大仙.
不拜觀音不拜關公之外.
也可能還包含了.
持守和實踐聖經在公義念文上的教導.
包含了不向邪惡政權妥協.
不向不義的團伙靠攏.
保守自己不要變成十一個.
保持自己對不公義的事情的憤慨和不習慣.
不合作和掙扎.

$^{721}$而我們現在可以做什麼小事呢.
其實我看過.
我們可以像那一百位先知.
在山洞裡生活作起點.
我剛才說了很多次.
他們在當中好好地喝水.
好好地吃餅.
好好地照顧自己的身體.
好好地保存自己的生命.
到之後好好地和一群同路人相處.
好好地和一群弟兄姊妹在Flow Church裡.
去同行.
彼此去建立.
建立一個同行群體.
也就正如在Outflow Mission當中.
我看到很感動.
有弟兄姊妹不惜坐一個多兩個小時的火車.
去倫敦.
去當地的教會.
我們Flow Church的教會當中.
一起去敬拜.
一起去開祖.
我們可以從這個起點開始做起.
然後我們一起在當中等候神的意象.
等候神的帶領.
等候神的感動.
等候神的差遣.
好不好啊弟兄姊妹.
最後.
在第十八節經文裡有一個字.
是重複了兩次.
下一章.
但是和合本聖經是沒有把它翻譯出來.
如果用原文直譯.
在第十八節應該是這樣翻譯的.
我在以色列中還有七千個人.
他們全部都未曾向巴力屈膝.
他們全部都未曾與巴力親嘴.
經文在這裡強調他們是整體和全部.
不單止指出他們.

$^{761}$這七千人全部未向巴力屈膝.
未和巴力親嘴.
但同時也帶出.
告訴我們那七千人在山洞裡.
或者在他們生活裡所擺上的每一分每一毫.
無論是做的新事.
還是做的舊事.
無論是做的大事.
或者是小事.
就算那些小事還沒小到.
小到是不見人影.
沒有人看得到.
但是神全部都會看到.
神全部都會紀念.
神全部都會閱立.
盼望我們當中每一位弟兄姊妹.
無論在香港.
無論在海外.
我們都在神的面前將來.
夢紀念.
夢閱立.
願主幫助我們.
\newpage



\section{路加福音 19:11-26-20221105}
\label{sec:fVOAbAyMqQo}
\textbf{【網上崇拜】未來的風險|路加福音19\_11-26|20221105 [fVOAbAyMqQo]}
\newline
\newline
連結: \href{https://youtube.com/watch?v=fVOAbAyMqQo}{\texttt{ https://youtube.com/watch?v=fVOAbAyMqQo}} ~~~~ 語音日期: 2022-11-05 
\newline
\newline
\hyperref[sec:rw_6I9ppxNw]{\small{< < < PREV SERMON < < <}}
~
\hyperref[sec:index_chronic]{\small{[返順時目]}}
~
\hyperref[sec:index_scriptual]{\small{[返順卷目]}}
~
\hyperref[sec:giVKoZv8XXY]{\small{> > > NEXT SERMON > > >}}
\newline
\newline
路加福音 19:11-26-20221105
\newline
\begin{longtable}{cl}
\hline
\hline
章節 & 經文 (和合本修訂版)\\
\hline
19:11 & \begin{tabularx}{0.7\textwidth}{X} 眾人正聽見這些話的時候,耶穌因為將近耶路撒冷,又因他們以為神的國快要顯現,就接著講了一個比喻, \end{tabularx} \\ \\ \relax
19:12 & \begin{tabularx}{0.7\textwidth}{X} 說:「有一個貴族往遠方去,為要取得王位,然後回來。 \end{tabularx} \\ \\ \relax
19:13 & \begin{tabularx}{0.7\textwidth}{X} 他叫了自己的十個僕人來,交給他們十錠銀子,說:『你們去做生意,直到我回來。』 \end{tabularx} \\ \\ \relax
19:14 & \begin{tabularx}{0.7\textwidth}{X} 他本國的百姓卻恨他,打發使者隨後去,說:『我們不願意這個人作我們的王。』 \end{tabularx} \\ \\ \relax
19:15 & \begin{tabularx}{0.7\textwidth}{X} 他得了王位回來,就吩咐叫那領了銀子的僕人來,要知道他們做生意賺了多少。 \end{tabularx} \\ \\ \relax
19:16 & \begin{tabularx}{0.7\textwidth}{X} 頭一個上來,說:『主啊,你的一錠銀子已經賺了十錠。』 \end{tabularx} \\ \\ \relax
19:17 & \begin{tabularx}{0.7\textwidth}{X} 主人對他說:『好,我善良的僕人,你既在最小的事上忠心,你有權柄管十座城。』 \end{tabularx} \\ \\ \relax
19:18 & \begin{tabularx}{0.7\textwidth}{X} 第二個來,說:『主啊,你的一錠銀子已經賺了五錠。』 \end{tabularx} \\ \\ \relax
19:19 & \begin{tabularx}{0.7\textwidth}{X} 主人也對這個說:『你管五座城。』 \end{tabularx} \\ \\ \relax
19:20 & \begin{tabularx}{0.7\textwidth}{X} 又有一個來說:『主啊!看哪,你的一錠銀子在這裡,我把它包在手巾裡存著。 \end{tabularx} \\ \\ \relax
19:21 & \begin{tabularx}{0.7\textwidth}{X} 我向來怕你,因為你是嚴厲的人:沒有放的,也要去拿;沒有種的,也要去收。』 \end{tabularx} \\ \\ \relax
19:22 & \begin{tabularx}{0.7\textwidth}{X} 主人對他說:『你這惡僕,我要憑你的話定你的罪。你既知道我是嚴厲的人,沒有放的也去拿,沒有種的也去收, \end{tabularx} \\ \\ \relax
19:23 & \begin{tabularx}{0.7\textwidth}{X} 為甚麼不把我的銀子存在銀行,等我來的時候,連本帶利都取回來呢?』 \end{tabularx} \\ \\ \relax
19:24 & \begin{tabularx}{0.7\textwidth}{X} 於是他對那些站在旁邊的人說:『把他這一錠奪過來,給那有十錠的。』 \end{tabularx} \\ \\ \relax
19:25 & \begin{tabularx}{0.7\textwidth}{X} 他們對他說:『主啊,他已經有十錠了。』 \end{tabularx} \\ \\ \relax
19:26 & \begin{tabularx}{0.7\textwidth}{X} 主人說:『我告訴你們,凡有的,還要給他;沒有的,連他所有的也要奪過來。 \end{tabularx} \\ \\ \relax
19:27 & \begin{tabularx}{0.7\textwidth}{X} 至於我那些仇敵,不要我作他們王的,把他們拉來,在我面前殺了!』」 \end{tabularx} \\ \\ \relax
19:28 & \begin{tabularx}{0.7\textwidth}{X} 耶穌說完了這些話,就走在前面,上耶路撒冷去。 \end{tabularx} \\ \\ \relax
19:29 & \begin{tabularx}{0.7\textwidth}{X} 快到伯法其和伯大尼,在名叫橄欖山的地方,他打發兩個門徒, \end{tabularx} \\ \\ \relax
19:30 & \begin{tabularx}{0.7\textwidth}{X} 說:「你們往對面村子裡去,進去的時候會看見一匹驢駒拴在那裡,是從來沒有人騎過的,把牠解開,牽來。 \end{tabularx} \\ \\ \relax
19:31 & \begin{tabularx}{0.7\textwidth}{X} 若有人問為甚麼解開牠,你們就這樣說:『主要用牠。』」 \end{tabularx} \\ \\ \relax
19:32 & \begin{tabularx}{0.7\textwidth}{X} 被打發的人去了,所遇見的正如耶穌對他們所說的。 \end{tabularx} \\ \\ \relax
19:33 & \begin{tabularx}{0.7\textwidth}{X} 他們解開驢駒的時候,主人問他們:「為甚麼解開驢駒?」 \end{tabularx} \\ \\ \relax
19:34 & \begin{tabularx}{0.7\textwidth}{X} 他們說:「主要用牠。」 \end{tabularx} \\ \\ \relax
19:35 & \begin{tabularx}{0.7\textwidth}{X} 他們把驢駒牽到耶穌那裡,把自己的衣服搭在上面,扶耶穌騎上。 \end{tabularx} \\ \\ \relax
19:36 & \begin{tabularx}{0.7\textwidth}{X} 他前進的時候,眾人把衣服鋪在路上。 \end{tabularx} \\ \\ \relax
19:37 & \begin{tabularx}{0.7\textwidth}{X} 他將近耶路撒冷,正下橄欖山的時候,一大群門徒因所見過的一切異能,都歡呼起來,大聲讚美神, \end{tabularx} \\ \\ \relax
19:38 & \begin{tabularx}{0.7\textwidth}{X} 說:「奉主名來的王是應當稱頌的!在天上有和平;在至高之處有榮光。」 \end{tabularx} \\ \\ \relax
19:39 & \begin{tabularx}{0.7\textwidth}{X} 人群中有幾個法利賽人對耶穌說:「老師,責備你的門徒吧!」 \end{tabularx} \\ \\ \relax
19:40 & \begin{tabularx}{0.7\textwidth}{X} 耶穌回答:「我告訴你們,若是這些人閉口不說,石頭也要呼叫起來。」 \end{tabularx} \\ \\ \relax
19:41 & \begin{tabularx}{0.7\textwidth}{X} 耶穌快到耶路撒冷,看見那城,就為它哀哭, \end{tabularx} \\ \\ \relax
19:42 & \begin{tabularx}{0.7\textwidth}{X} 說:「但願你在這日子知道有關你平安的事,不過這事現在是隱藏的,你的眼睛看不出來。 \end{tabularx} \\ \\ \relax
19:43 & \begin{tabularx}{0.7\textwidth}{X} 因為日子將到,你的仇敵要築起土壘包圍你,四面困住你, \end{tabularx} \\ \\ \relax
19:44 & \begin{tabularx}{0.7\textwidth}{X} 並要消滅你和你裡頭的兒女,連一塊石頭也不留在另一塊石頭上,因為你不知道你蒙眷顧的時候。」 \end{tabularx} \\ \\ \relax
19:45 & \begin{tabularx}{0.7\textwidth}{X} 耶穌一進聖殿就趕出在裡面做買賣的人, \end{tabularx} \\ \\ \relax
19:46 & \begin{tabularx}{0.7\textwidth}{X} 對他們說:「經上說:『我的殿是禱告的殿,你們倒使它成為賊窩了。』」 \end{tabularx} \\ \\ \relax
19:47 & \begin{tabularx}{0.7\textwidth}{X} 耶穌天天在聖殿裡教導人。祭司長、文士和百姓的領袖都想殺他, \end{tabularx} \\ \\ \relax
19:48 & \begin{tabularx}{0.7\textwidth}{X} 但找不出方法來,因為百姓都側耳聽他。 \end{tabularx} \\ \\
[1ex]
\hline
\hline
\end{longtable}
$^{1}$丙姐妹平安.
我看陳茂波和習近平都不帶,那我也不帶了.
(笑聲).
留堂11月12月的主題是未來.
我經常覺得未來這個字是很futuristic的.
我自己是上世紀人,我自己在上世紀活了20年.
所以我整個少年時期都生活在XX年那些.
一直都很期盼著2000年那些,不知道大家是否這樣.
大家有沒有經歷過1999年的倒數?.
還記得1999年的倒數取值?.
在哪裡?如果你經歷過1999年的倒數的話.
可能你會記得1999年踏入2000年的複雜的心情.
首先你會知道什麼是千年蟲,你不知道的就算了.
你明白的,千年蟲,1999年的倒數是很複雜的.
既興奮又恐懼,又期待又有一點擔心.
再加上你是一個基督徒,1999年你心裡暗地裡有一個想法.
但你不敢跟別人說,因為你一說出來.
你會覺得好像很傻,但你又會相信.
你也想過一下,2000年是否就是聖經裡的千禧年?.
然後,千禧年是否會再來?是否會世界末日?.
有的,一定有這樣想過.
媽媽也想過,當你倒數的時候,會不會突然有什麼事情發生?.
所以記得1999年我在教會倒數.
5,4,3,2,1,Happy New Year,Happy 2000.
然後你就會看一看手錶,看手錶有沒有問題.
看一看窗戶,看手錶有沒有問題.
還沒有,沒有問題,Happy New Year.
我認為2022年是一個很復古的年頭.
我還沒有消化到原來自己在2022年的年份裡.
不知道大家有沒有看過《回到未來》?.
有嗎?有,是經典.
如果你看過的話,就知道2015年10月21日是什麼日子.
知不知道?2015年10月25日.
其實就是《回到未來》的主角在1987年坐著時光機去到未來.
主角坐著一架跑車的時光機.
當時速達到88英里,大概是八四公里.
就能夠穿梭到未來的裡面.
去到2015年10月21日.
我記得當時看電影的興奮感覺.
主角一下車,2015年是一個非常復古的年份.

$^{41}$大家看過的話,空中滑板.
Hoverboard,一按按鈕,Likey鞋會突然啜動.
就變成一隻腳的新科技,那些釘就變成很大塊Pizza的東西.
當然還有視像會議,Zoom很厲害.
這就是當時的2015年的感覺.
所以今年2022年,當我置身在一個比未來更加未來的未來.
有一種很特別的感覺.
原來小時候的未來已經過去了.
我沒辦法想像未來之後的未來會是怎樣.
是一個很奇怪的感覺.
小時候經常想將來會是怎樣.
但當你置身在將來的將來的時候.
你開始不太敢想,也不太想想我們的將來.
最少發現未來沒有你想像的那麼復古.
沒有你想像的那麼帥.
如果有一個人坐過時光機,由1989年來到2022年.
你會覺得很尷尬.
你不知道要帶什麼去介紹給他看.
原來未來人打扮這麼復古,每個人都戴著口罩,那麼帥.
我覺得會很懦弱.
但未來人很高科技,一進餐廳就會帶著高科技的東西.
會再掃一掃,再掃一掃,double verification.
整件事很有未來感.
原來未來人那麼著重健康.
小朋友到老人家都要檢查一次.
很快就檢查到.
唯一不變就是故事.
仍然維持在一個萬幾點的水平.
很親切,很安慰,很懷念.
所以對於未來,我們似乎沒有了小朋友時期的期盼.
不再是那些哆啦A夢式的未來.
也不是back to the future的未來.
你發現原來未來可以是很天啟式的.
即是瘋狂物思,即是Mad Max那種天格式的未來.
或者是整個世界都是Zombie的未來.
我也不知道何時開始發覺未來是越來越不像樣.
開始沒有了那份期盼,沒有了那份憧憬.
甚至是恐懼,甚至是絕望.
因此我們在Full Church這兩個月.
我們一起思想未來這個課題.

$^{81}$一起重新思考我們的信仰.
可以怎樣幫助我們和未來好好相處.
我們一起祈禱.
祈求你帶領我們,讓今天的港島.
你對我們說話.
你告訴我們怎樣面對我們每一個人的未來.
幫助孩子不配,但你親自的聖靈.
對我們每一個艇姐妹說話.
在網上的艇姐妹,在不同地方的艇姐妹.
我們一起求聽,求你自己的說話.
求你這樣來同在,逢春命求,阿們.
今天我們看一段經文是路加福音十九章十二到二十三節.
一段和本的標題叫做十定銀子的比喻.
經文有一段是平衡經文.
記載在馬拉福音的二十五章裡.
大家很熟悉的,就是這個財降的比喻.
兩段經文是很相似的.
無論是故事的橋段,故事的結構.
最後的結局都是很相似的兩個平衡的比喻.
兩段經文都說一個主人離開了他的地方.
分別把錢給他的僕人,叫他們去投資.
當他回來之後,第一個僕人賺了很多錢.
第二個僕人也賺了很多錢.
最後第三個僕人就把錢收起來.
收了主人的錢,最後被這個僕人去責備.
一個又惡又懶的僕人.
耶穌稱讚這些在生意獲利的僕人.
既良善又忠心的僕人.
兩個比喻都是同樣的套路.
不過當你細心去留意經文的時候.
你會發現經文之間有不同的出入.
一會兒我們會說出入.
所以一起去看,看今天的經文.
特別是今天的比喻裡面,我們選了路加的版本.
我們一起讀吧,讀一下經文.
第十二節開始,1,2,3.
好,謝謝.
基本上大家都很熟悉經文.
所以今天就不花時間說裡面的細節.
我可以說的是,如果我們去比較路加福音這個版本.

$^{121}$和馬太福音那個版本的時候.
當你去比較這兩個比喻.
你會發覺這兩個比喻裡面.
那些僕人用錢的方式和他們所賺取的回報是很有趣的.
你會發現有一個很特別的發現.
就是路加福音的版本所說的回報率.
是比馬太福音的回報率是瘋狂地高的.
大家記得嗎?.
馬太福音說的是五千銀紙,二千銀紙,一千銀紙.
相比之下,其實就是大概一倍回報.
五千那個賺到五千,二千那個賺到二千.
一千那個一千.
相比之下,路加福音的回報率是非常高的.
一錠銀紙賺多少?.
一錠銀紙賺到十錠,一倍十,比概念.
一萬元可以賺到一百萬.
一個非常瘋狂的回報.
(十萬).
對不起,十萬,我瘋了.
一萬元賺到十萬,也很厲害吧.
十萬賺到一百萬.
就算是第二個撥銀,也有五倍.
一個銀紙賺到五個銀紙,五倍的回報率.
哪裡找得到呢?.
所以馬太福音的比喻和路加是兩個完全不同的檔次.
我們問,究竟怎樣才叫做有十倍的利潤?.
你也知道,你要有十倍利潤,就要有十倍的風險.
利潤越高,風險就越大.
這是天然的定律.
股票,期權,債券,大小,鞋,桌上遊戲,都是一樣.
全部都是這個定律.
回報越高,你要賭博的風險產生越大.
你想想,這幾年你也知道NPF.
曾經你提到進取基金的話.
這幾年NPF也會跌得很進取.
所以路加福音說賺了十倍,五倍的捕銀.
其實他們同時也在承受十倍和五倍的風險.
我們問,甚麼叫做十倍風險?.
是甚麼感覺?.
為了呈現這個感覺.

$^{161}$我嘗試翻查賭博的網站.
一倍十是一個甚麼形勢呢?.
今晚有一場德甲.
哈化柏林對拜恩姆尼克.
主場哈化柏林對拜恩姆尼克.
賠率是一倍十.
主場哈化柏林贏拜恩是一倍十.
我做了柏林六年.
我曾經親自參與柏林主場對拜恩姆尼克.
那時我買了一件哈化柏林的頸巾進去捧場.
六比零,被人炒六比零.
所以你明白有多困難.
今晚我們試試賭一下.
想想誰會買柏林的東西.
一倍十.
今晚你不熟悉德甲也好.
你想想世界盃.
首爾奧克利亞世界盃.
英格蘭第一場對哪一隊.
對伊朗.
同樣賭球網的資料.
伊朗贏英格蘭的賠率是一倍十.
OK的,英格蘭有時也會踢一些球出來.
但你說伊朗不是很厲害,當然不厲害.
你要去英格蘭是沒甚麼可能的事.
一倍十就是.
這就是風險.
不是沒可能,就是難度高.
非常非常難度高.
這就是十倍風險的感覺.
原來,星馬哥告訴你.
耶穌所稱讚的忠心有良善的僕人.
這班賺到飛起的十個銀子的僕人.
同樣是承受著十倍風險的一班人.
真人,真人.
沒有一個上主所交託我們的責任.
是不需要受風險的.
沒有一件美善的事是不需要付代價的.
任何我們向上主交代負責的人.
不怕,即是基督徒,教會也好.

$^{201}$我們都要為一些上主美善的事.
去承擔一些風險.
尤其是今天的社會.
任何一件事,真是的.
真是每一件事,一件美善的事.
你都要冒上風險去做.
我們看到大把例子.
教會做一些應該要做的事的時候.
就要負上一些風險去做.
耶穌稱讚這些承擔風險的人.
做一個忠心的僕人.
相反,耶穌去譴責那些自以為忠心.
他們以為是守住家業.
即是很保險,很play safe的人.
耶穌是譴責他們.
所以當你細心一下去留意這個比喻.
聖經裡面怎樣描述這個被斥責的僕人.
聖經描述當主人去稱讚這兩個僕人之後.
第三個僕人就開始去做一件事.
珍貴.
主啊你看看,你看看這個銀子.
我保存在手帕裡,My precious.
這是一件真銀.
一公斤的真銀.
完了,戲完了.
真的,聲裡面用一個.
用手帕包著這一句的形容.
是非常之可圈可點.
想想用一條mongchichi手帕.
包著那些銀.
既安全又穩陣.
感覺非常之良好,不怕弄濕.
只要不太濕,不怕污糟,只要不太污糟.
手帕包著就無有怕.
你問這個僕人做事細不細心.
非常細心.
你問這個僕人做事穩不穩陣.
非常穩陣,這個僕人做事可不可靠.
非常可靠.
但這個並不是天國的態度.

$^{241}$甚至耶穌稱呼這個僕人為.
惡僕,a weak servant.
你可能會問,為何這個.
這麼穩陣,play safe的僕人.
有什麼地方這麼惡呢.
你會有些懷疑.
為什麼用手帕包著錢有多惡.
你何時見過惡人用mongchichi手帕.
沒有的.
這個僕人是這麼小心.
用mongchichi手帕包著錢.
還不是一個忠心的管家嗎.
這個僕人有多caring, 多謹慎.
多愛惜主人的錢財.
究竟這個僕人的惡在哪裡呢.
答案非常簡單.
僕人沒有回應主人的要求.
這是耶穌基督的要求.
你可以說,to be fair.
這個僕人主人的要求是很高的.
甚至這個主人的要求是harsh.
我僕人說什麼呢,他說,我原是怕你.
因為你是一個嚴厲的人.
所謂嚴厲就是你這個人很harsh.
連主人都承認他很harsh.
他的要求是很harsh.
主人的要求有多harsh呢.
主人要求他的僕人沒有放下的還要去拿.
沒有中下的還要去收,是什麼意思呢.
明明沒有中的還要去收取.
明明沒有放下的還要去拿.
他要無本生利.
他要追求極高的回報.
槓桿的利潤.
然後他要求他的僕人去承受一個或多或少大的風險.
主人說你明知我是harsh的.
你還不去做.
所以我們發現非常之鄭重的道理.
對於基督徒來說.
承受風險從來都不是一個option.

$^{281}$也不是一個可能性.
它是耶穌對我們每個人的命令.
也是基督對教會的命令.
先不要說初來教會為主殉道.
冒著風險殉道傳福音的門徒.
也不要說坐船來到中國的風險.
作為基督徒你或多或少都要學習用信心.
承受一定程度的風險.
去迎臨未來或多或少的uncertainty.
在這個邪惡的世代裡面.
好好地去承擔一些風險.
承受耶穌要我們去行的行為.
教會應該要做的責任.
在2019年開始.
面對這幾年發生的事.
疫情,政治,法律條例等等.
我想我們每一個香港教會.
每一個基督徒,每一個機構.
都無可避免地去參與一個有關風險的討論.
基本上我們每天都在討論同樣的事.
如何去跟風險周旋.
妖魔是選擇非常保守,非常謹慎.
極度play safe地去做人,去做教會.
同時去妥協,放棄教會應該要做的事情.
我曾經說過不讓石的教會.
罰石的教會.
我看過一個廣告.
牧師游靈堂的非常好的教會.
大家記得去他的分禮.
非常願意去承擔風險.
但他卻做了教會可以做的事情.
選擇去堅持耶穌基督要我們作為教會應付的責任.
跟隨他的腳蹤.
做更加好的事.
承受隨之而來必定會有的風險.
這是我們基督徒群體要問的問題.
to be or not to be.
take risk or play safe.
很多人以為面對風險.
第一個反應是如何避免風險.

$^{321}$他們以為面對風險只有一件事要做.
就是要避免.
你做了安心了嗎.
這是我多年來聽得最難聽的溫馨提示.
我不知道為什麼會將安心這個名詞配上做這個動詞.
做安心我覺得很難聽.
什麼叫做安心.
做功課做運動我聽過.
但我沒聽過做安心.
我不明白為什麼要做安心.
總之今天的社會要求我們.
作為教會的.
不要出事.
不要有公關災難.
不要惹禍上身.
一切要去play safe.
直到主在來.
阿們.
不過真人耶穌在這個銀子的比喻裡.
告訴我們.
風險是必定會存在的.
問題的關鍵不是你如何去避風險.
而是你為了哪一個去承擔什麼風險.
我再說一次.
問題不是你要避免風險.
而是問你應該為了哪一個去承擔什麼風險.
應盡的風險.
即是面對未來.
從你今晚到明天到明年到十年後.
風險必定會存在.
沒有一個未來是沒有風險的.
做事有風險.
不做事也有風險.
進取有風險.
play safe也有風險.
問題是你為了哪一個去承擔哪一種風險.
如果我們去詳細探討風險這回事的話.
風險可以分為三類.
第一類是已知風險.
風險的出現率是既定的.

$^{361}$你也知道大概幾\%.
就像大小一樣50/50.
一個既定的風險.
第二類是未知風險.
風險的出現率是不確定的.
你不容易去估計也不能去評估.
很多人很關注這兩個風險.
已知風險和未知風險.
他們以為面對風險.
只要play safe就能夠完全避免.
多做事多風險.
少做事少風險.
不做事沒有風險.
他們以為是這樣.
追求最好太危險了.
我寧願去追求一個穩陣的次好.
不過我們很多時候忽略第三類的風險.
就是隱藏的風險.
隱藏風險是隱藏的.
是長遠的.
是慢性的.
是未來的.
隱藏風險是依附在今天的平庸和無為的背後.
當教會以為保持穩陣.
先不要亂來.
少做少錯的時候.
隱藏風險其實是不斷地累積.
當教會沒有做一件應該做的事.
當教會沒有做好.
我強調是沒有做好要做的事的時候.
當教會選擇安全.
輕易地做.
選擇次好.
或者不做的時候.
風險其實一直累積在將來的裡面.
這幾年很多教會可能都是這樣.
今天都知道很多教會面臨很多的困難.
不是一兩年一個兩個錯誤的決定.
那個失敗可能是過去的十年二十年.
很多很多個很微小的很僵化.

$^{401}$很保守很安全的決定.
累積成為一個慢性的衰亡.
教會為了減少眼前的承受風險.
去醞釀出一個長期毀滅性的風險.
惡瀑就是這樣.
以為用一塊小手帕包著錢.
這就是一個最穩陣.
當下裡面最安全的做法.
對的.
當下裡面最穩陣是這麼做的.
但當他用小手帕包著錢的時候.
他就開始去為自己的將來慢性地建立一個隱藏的死亡和風險.
大家不要誤會我這樣說.
不是因為我出於冒險性格的人才這樣說.
我們仍然需要去謹慎去考慮.
我們應該去承擔什麼風險.
好好去管理我們的風險.
但我們要認清我們面對的是怎樣的風險.
為著誰去承擔什麼風險.
其實很多風險專家認為.
全世界只有一種風險是我們一定要避免的.
就是滅頂之災風險.
就是一舖清袋.
不能翻身的風險.
就是你玩股票棋拳.
輸一輸就破產.
就沒得玩了.
就是這樣的概念.
滅頂的風險是要避免的.
因為當這些風險發生的時候.
你就再沒有翻身的機會.
所以很多專家告訴我們.
一定要去設法去避免滅頂之災一舖清袋風險.
話說回來.
對於我們基督徒來說.
全世界唯一一樣不會出現滅頂之災風險的是什麼.
豈不是就是我們上帝的國.
上帝的國是不會讓你搞砸的.
上帝的國不會一舖清袋的.
任何的事都可以玩完.

$^{441}$Twitter 匯豐 騰訊.
上帝的國是例外的.
上帝的國不會因為某人或某班人錯誤的決定而被搞垮.
一個人可以搞死一個教會.
一個宗派.
但上帝的國度是不會因為這樣的錯誤去承擔這個風險.
所以時常說全教會是可以倒閉的.
這個我是不介意的.
但天國是不會倒閉的.
而我們要做什麼呢.
我們要去努力去為更加遠大的天國獻上我們的最好.
選擇天國的最好不是全教會的最好.
全教會可以因為風險的緣故而去承受很大的問題.
但我們要去照顧的是上帝的國度.
我希望你有我同樣的想法.
不要為了去保住一些眼前的東西.
而去嘗試去退縮.
我們要為上帝的國度冒上一定程度的風險.
究竟我們是為了全教會的好處還是為了天國的好處呢.
這個我們在流唐底之門一定要好好搞清楚問題.
假如我們的眼光只是短視地看到我們現在這樣.
保住我們每個禮拜很安全地有崇拜.
保住我們現在某些東西的時候.
其實這是對我們全教會來說是不好的決定.
我都說過我在2022年這年裡.
其實我是很不安的.
每個星期保持著崇拜.
很穩定地有幾千人參與我們崇拜.
其實我是有點不安的.
我覺得如果我們Folk Church十年都是這樣做.
只做這件事的話.
一點都不是好事.
我擔心Folk Church會變成一間普通的堂會.
我怕Folk Church純粹為了自己的好處和存在而生存.
然後保持著每個禮拜每個禮拜這樣流下去就算了.
流唐要做一個忠心良善的僕人.
我們一起就要去面對去承擔一些美善的風險.
天子們天國的風險.
我們值得去為主去承受一定程度天國的風險.
大家不要忘記路加福音這個風險比喻有一個很特別的位置.

$^{481}$一個和馬太很不同的故事橋段.
路福音用錢的比喻裡面.
其實隔壁有一條支線是很不自然的說出來的.
主人是要去接受王位.
他的國度正要來臨.
這是路加獨家的馬太沒有這樣說.
路福音特別強調.
耶穌就說了一個比喻.
因為他快到耶路撒冷了.
而且他們以為神的國立即要顯現.
於是耶穌就說了.
有一個權貴往遠方去接受王位.
然後回來.
你發現馬太沒有這段話.
路加福音裡面特別強調.
無論這班僕人拿著錢怎樣花.
怎樣投資.
怎樣虧蝕.
怎樣賺都好.
主人得著王位.
主人的國度將要來臨.
是完全沒有關係的.
這班人是虧蝕是賺.
主人仍然會坐著王位回來找他.
無論這班僕人承受多大的風險.
無論這班僕人是賺還是虧蝕.
主人的國度必定會來臨.
總之是贏.
所以聖經說主人是很虛榮的.
主人真是很虛榮的.
因為主人就是要贏.
主人就是會贏.
我們的主耶穌基督.
就是那位德性的主.
我們是面對著這個德性的君王.
去冒我們眼前的風險.
所以主人吩咐這班人.
一賠十.
英格蘭對伊朗.
買伊朗.

$^{521}$哈佛柏林對拜仁奧利亨.
買哈佛柏林.
為我的緣故.
不妨去冒一些風險.
反正我的國度是必成的.
你一個錯誤的決定.
是不會搞垮我的天國的.
你要做的一件事情.
忠心的.
良善的.
好好去運用我們所有的.
在這個黑暗的世代裡面.
做出一些上帝要求我們做的事情.
基督徒要做的事情.
教會要做的事情.
不單只是做.
是要做到最好.
不要害怕.
不要膽怯.
應該要做什麼就做什麼.
承擔天父命令我們.
因為美善的緣故.
要承擔的風險.
頂姐妹.
你今天怎樣和未來去共處呢.
你怎樣和未來去共處呢.
有些人很快會回到英國.
有些人可能下年計劃離開香港.
有些人下年又會在香港裡面.
計劃去做一些事情.
有些結婚.
有些轉工.
有些有新的階段.
你怎樣和你的未來共處呢.
面對著未來.
香港的未來.
教會的未來.
你自己的未來.
學習忠心.
勇敢.

$^{561}$善良的去承擔一些風險.
這個和你的性格沒有關係.
你可以仍然是一個很謹慎的人.
但作為基督徒.
作為教會.
你好好的要向耶穌基督去交徵.
承受他所交託給你的風險.
去為主獻上最好的.
這是我們全教會很想去做的事情.
主在來.
這三個字很有意思.
主在來.
廣東話叫做主在來.
主未來.
但祂正要來臨.
耶穌基督在將來那裡.
在將來那裡.
等待著我們.
祂在將來那裡說.
未來見.
好好地去承擔一定的風險.
在這個變化莫測的世代裡.
在一個沒有人知道明天會發生什麼事的世界裡.
好好地去冒著風險.
去跟隨我們的主.
憑著信心.
去踏出這一步.
阿們.
\newpage



\section{腓立比書 3:13-21-20221112}
\label{sec:giVKoZv8XXY}
\textbf{【網上崇拜】未來的轉變|腓立比書3\_13-21|20221112 [giVKoZv8XXY]}
\newline
\newline
連結: \href{https://youtube.com/watch?v=giVKoZv8XXY}{\texttt{ https://youtube.com/watch?v=giVKoZv8XXY}} ~~~~ 語音日期: 2022-11-12 
\newline
\newline
\hyperref[sec:fVOAbAyMqQo]{\small{< < < PREV SERMON < < <}}
~
\hyperref[sec:index_chronic]{\small{[返順時目]}}
~
\hyperref[sec:index_scriptual]{\small{[返順卷目]}}
~
\hyperref[sec:UHPz7So3h50]{\small{> > > NEXT SERMON > > >}}
\newline
\newline
腓立比書 3:13-21-20221112
\newline
\begin{longtable}{cl}
\hline
\hline
章節 & 經文 (和合本修訂版)\\
\hline
3:13 & \begin{tabularx}{0.7\textwidth}{X} 弟兄們,我不是以為自己已經得著了;我只有一件事,就是忘記背後,努力面前的, \end{tabularx} \\ \\ \relax
3:14 & \begin{tabularx}{0.7\textwidth}{X} 向著標竿直跑,要得神在基督耶穌裡從上面召我來得的獎賞。 \end{tabularx} \\ \\ \relax
3:15 & \begin{tabularx}{0.7\textwidth}{X} 所以,我們中間凡是成熟的人,總要存這樣的心;若在甚麼事上存別樣的心,神也會把這些事指示你們。 \end{tabularx} \\ \\ \relax
3:16 & \begin{tabularx}{0.7\textwidth}{X} 然而,我們達到甚麼地步,就當照這個地步行。 \end{tabularx} \\ \\ \relax
3:17 & \begin{tabularx}{0.7\textwidth}{X} 弟兄們,你們要一同效法我,也當留意看那些效法我們榜樣的人。 \end{tabularx} \\ \\ \relax
3:18 & \begin{tabularx}{0.7\textwidth}{X} 因為,我屢次告訴你們,現在又流淚告訴你們:許多人行事是基督十字架的仇敵。 \end{tabularx} \\ \\ \relax
3:19 & \begin{tabularx}{0.7\textwidth}{X} 他們的結局就是滅亡。他們的神明是自己的肚腹;他們以自己的羞辱為光榮,專以地上的事為念。 \end{tabularx} \\ \\ \relax
3:20 & \begin{tabularx}{0.7\textwidth}{X} 我們卻是天上的國民,並且等候救主,就是主耶穌基督從天上降臨。 \end{tabularx} \\ \\ \relax
3:21 & \begin{tabularx}{0.7\textwidth}{X} 他要按著那能使萬有歸服自己的大能,把我們這卑賤的身體改變形狀,和他自己榮耀的身體相似。 \end{tabularx} \\ \\
[1ex]
\hline
\hline
\end{longtable}
$^{1}$弟兄姊妹平安.
剛才敬拜隊的詩歌.
我相信很多弟兄姊妹.
無論歌詞或是那種律動.
都會幫我們做一個很重要的串連.
我自己也很享受其中.
特別在不斷宣告.
上帝是昔在今在以後永在.
那位見路者 同行者的觸動.
其實是很感動的.
我相信詩歌每一次去唱的時候.
不僅僅是唱那種音樂感.
或是那種節奏.
是唱出你的心情.
而那種心情正正就是.
有時你說不出來.
但詩歌就不斷在當中帶動.
我們將心中的說話.
透過詩歌再次呈現.
每一次唱詩歌的時候.
都很感受到你不孤單.
有很多弟兄姊妹一起去經歷.
是不是?.
說到未來的時候.
上星期說到未來的危險.
今天是一個相連的訊息.
就是未來在當中要經歷危險的時候.
你要轉變的是什麼?.
未來的轉變就是今天的題目.
在經文上我選取的是.
菲勒比書的經文.
菲勒比書的經文當中.
是給我們做一個提醒.
我自己很喜歡菲勒比書.
因為當在菲勒比書當中的內容.
再次提醒我們.
菲勒比教會是一間在羅馬的駐防城的教會.
他面對羅馬的政權和羅馬的文化當中.
對我們今天的提醒有什麼幫助?.
過去這兩年.

$^{41}$我相信你在不同的聚會都聽過.
說菲勒比書的.
但是今天我看菲勒比書的時候.
我希望看一下.
保羅希望菲勒比教會在將來.
其實他們要有什麼準備?.
或者在面對轉變的過程當中.
怎樣讓自己的身份再得到認同的時候.
去迎向未來的改變?.
我選的經文是第三章.
第十三節開始.
等下姐妹可以一起讀.
OK?.
可以出嗎?.
謝謝.
我們一起讀第十三節開始.
我們一路讀到二十一節.
請.
謝謝各位姐妹.
我們一起禱告.
將我們每當打開你的說話的時候.
仍然歷歷在目.
因為你是色在今在.
以後永在的柱.
你為你的子民在每個年代當中.
都有我們要聽的說話.
以致上帝你仍然帶著我們朝向你的方向.
繼續走屬主你的路.
願主你今天坐直.
享受你當得的榮耀.
用聖靈感動我們.
在聽你的說話的時候.
就當遵他而行.
祈禱奉耶穌的名求.
阿們.
在讀這段經文的時候.
我相信你都不難去理解.
保羅也是現在說一個很重要的訊息.
讓我們明白到當下我們的處境.
是不容易的.

$^{81}$但又讓我們明白.
當下有些人是很威風的.
但是.
保羅也是用一個忠告的角色告訴我們.
他們的結局是會沉淪的.
而我們在這個屬天子民的身份當中.
我們怎樣自處.
以致我們怎樣繼續走下去.
是保羅說一個現在.
又是說將來的事情.
當我們看保羅書信的時候.
我們又不難理解.
就是很多時候寫書信去.
就是保羅還沒去到那裡.
又或者是過了之後.
他有些事情想跟弟兄姊妹再重提的時候.
他寫.
預先寫信去.
讓他們去明白到.
保羅想處理他們的問題.
我自己很多時候在福音書講道.
都喜歡在書信當中做一個.
就是.
講道的經文.
因為我常常都很想從.
保羅的生命當中去看到一個.
很重要的訊息就是.
其實一個牧者的心腸.
他怎樣去顧及他的兒女呢.
就是肅靈的兒女.
在過程當中你會感受到.
保羅正在活現.
就是我活著就是基督.
那個僕人心智的工作.
今天我們沒有讀到.
肥立彼書第二章.
肥立彼書第二章就是.
保羅去寫耶穌基督.
道成肉身那種虛己.
那種書尊降貴.

$^{121}$那種僕人的心智.
甚至死在十字架上.
成就上帝的工作.
那種僕人.
那種致死忠心.
完成使命.
那個新教.
而保羅正正就是朝主的方向.
去走他.
就算是那一刻.
甚至乎他預示將來.
要做的工作.
這個就是保羅想讓更多人明白到.
他為什麼這樣做.
肥立彼書大家都知道.
是保羅後期寫的.
在獄中的時候寫.
但是可能傳統.
大家一直以來.
去認識肥立彼書的時候.
都會記住就是講喜樂.
但是你慢慢當你.
教會經歷久了.
人生越來越多了.
看聖經多了的時候.
你又很難去理解.
就是一個坐牢的人.
為什麼會講喜樂呢.
一個懷才不遇.
甚至乎是不公平待遇.
但是又想做的事情.
不讓他做的過程當中.
其實有很多冤屈的.
這個人如何談得出喜樂呢.
當我們真的要知道.
其實他背後做事.
或者是對.
他要傳遞什麼訊息給弟兄姊妹的時候.
對我們今天其實來說.
是很大的提醒.

$^{161}$所以今天花點時間.
和大家從第十三節到二十一節.
這幾節經文當中.
其實他對三張保羅大的訊息是什麼.
慢慢來看看.
第十三節.
弟兄們博不士以為自己已經得著了.
我只有一件事.
就是忘記背後努力面前的.
經文有很多弟兄姊妹會背的.
還有很多時候都會寫在.
送給別人的卡上.
不知道是考得不好的時候寫.
還是.
不是不是不是.
應該考得好的時候都會寫.
叫你再上一層樓.
有時入學的神學生.
想入學蒙召的時候.
我都聽過入學見證是說這段經文.
其實.
我已經得著了.
Nabano這個字.
在我講道之前都講過.
保羅很喜歡用這個字.
得著.
抓住.
捉緊.
Nabano這個字其實是.
保羅常常都用主動的方式去講.
其實你自己要捉緊一點.
你自己主動一點.
因為在過程當中.
我們要抓住什麼信念.
抓住什麼原則.
去過你的生活.
其實是看清楚你自己是什麼人.
我常常都講.
今天做人認真是很辛苦的.
不認真的人其實真的沒有原則.

$^{201}$是不是呢.
有點爛是不是.
是不是.
你都要想一下是不是呢.
是不是呢.
通常你講這些好像無關痛癢.
沒有什麼絕對標準的話.
就很多haters可能會出現.
就是一些toxic的東西就出來了.
但是.
我們可能很難界定那個人有沒有原則.
那個人很難界定那個人有沒有抓住一些東西.
但你自己知道你自己沒有抓住東西.
今天別人問你的決定.
你的立場或者你的表達方法.
或者你用什麼判斷那件事.
其實你都在問你自己在得著什麼.
很困難.
但是保羅提醒弟兄們.
我不是說自己已經是抓住了.
但他不是自己抓住了.
但他又事實上是抓住了.
他抓住了一件事就是.
忘記背後面前不是代表他忘記了他以前身份上的東西.
在第三章之前他都數自己的身家.
他自己的身份.
但那件事以往那麼自恃對他來說.
今天不是再拿那件事出來去曬.
或者拿那件事出來告訴別人他多厲害.
他反而是記住一件事就是.
向著標竿直跑.
就是一定有個方向性.
他抓住一件事.
而那件事是有方向性.
要跑下去的時候.
他知道為什麼事而跑.
今天我相信你旁邊如果.
你談得來的人.
又或者過去這兩三年你覺得.
你認識的人當中.

$^{241}$你做了很多很多的決定.
就是大家在談話當中.
表達過程當中.
其實大家就表達到不同的意願.
意見.
方向.
去怎麼走下去.
不是對與錯的問題.
是大家能不能夠相和.
大家能不能夠繼續有一路同行.
那種方向性.
本來他提醒一件事.
他第一件事是一個勸勉.
弟兄們.
不是說我自己拿著什麼.
但我事實上又拿著一些東西.
我拿著一件事就是.
不是因為羅馬的文化.
是很著重那個身份.
很著重他是什麼背景.
他有什麼權利.
但是保羅要拿的他都有.
但是重點他不是那件事.
他要得的是什麼呢.
是耶穌基督給我獎賞.
你可能會問.
那耶穌基督給了什麼獎賞你呢.
或者耶穌基督給了什麼獎賞保羅呢.
我相信你看保羅的夢照.
或者他的經歷當中.
你不難找到的.
但反而保羅去跟弟兄們說.
我們今天拿著的是什麼呢.
正如我之前很多時候都會說.
今天做人認真就很辛苦.
做一個認真的基督徒都很辛苦.
因為總會有很多人問你的看法.
問你的意見.
問你的立場.
聖經怎麼看.

$^{281}$基督徒應該有什麼身份.
很多東西都要.
你以往可能不是這麼認真想.
但是現在你很難不認真想.
所以在整個過程當中.
要想清楚其實.
聖經對我們今天的挑戰.
或者聖經對我們來說.
我們怎樣在生命當中去演繹出來呢.
補一點.
大家可能一直都聽過.
菲律賓是一個什麼地方呢.
是一個羅馬管轄的駐防城.
在四大派傳.
你會知道他們有全副軍裝的士兵.
出出入入的.
而他們在菲律賓住的人.
很著重自己羅馬公民的身份.
因為羅馬公民的身份是很超越的.
他們是很喜歡將人分門別類.
分等級的.
而過程當中.
有些人是在一個叫做.
被分出來有不同級別的人的時候.
他們其實享有的特權是很誇張.
你會聽過保羅Sir都說.
他們會放縱私慾.
他們會在當中很喜歡斂財.
很喜歡貪婪.
這個都是.
我想你看一些西方羅馬的宮廷戲.
你都不難理解.
我自己以前有一套戲.
看過很多次.
叫做Gladiator.
帝國驕雄.
看電影節目沒什麼反應.
應該不是那個年代.
他是遲到回到未來的.
回到未來是老很多的.

$^{321}$我為什麼看過很多次那套戲呢.
其中原因就是那時候.
我結了婚.
結婚之後都有再看.
結婚之前更加看.
就是我喜歡玩音響.
那時候那套戲.
是最好試音響的其中一套戲.
因為羅馬近義場.
有很多馬車走來走去.
你會看到馬車在滾來滾去.
從那邊橫梅滾過來.
來吧來吧.
同一部戲同一個部分看很多次.
感受橫梅立體聲.
家裡都能夠享受.
這是一個告訴大家.
為什麼看那麼多戲.
看了那麼多次.
但其實那套戲.
我每次都看完整套.
就感受到在羅馬.
分等級的人.
他們上面坐不同位分的人就不同.
最高位分的坐包廂.
坐包廂的人.
另外就會是沒有包廂的.
最下層就是奴隸.
出來和野獸搏鬥.
是愚賓的.
整件事就是在當時的環境.
大家不難理解.
菲律賓的環境當中.
就是在很多羅馬的文化當中.
分等級有不同的人.
因為有特別身份有特權階級.
他會享受一些特別福利.
其實今天都一樣.
今天仍然有特權階級.
今天仍然有些人會有特別福利.

$^{361}$有些人會特別安排.
有些人有特別通道.
不難理解.
然後他再問一問.
你知道我是誰嗎.
重點就是.
很多事情會令到人.
想拿到特權階級的時候.
他會增進.
他會告訴別人.
我認識了誰.
保羅想提醒弟兄姊妹.
如果過去是想追求這些.
今天其實應該要想一下.
其實我們是否還在追求這些.
當然仍然是覺得要持守這些.
這些東西可以令到你在社會上的身份得到保障的話.
其實今天你是否還要看這些呢.
你會看到這段經文.
今天沒有讀.
但是保羅在肥勒秘書說.
我已經看萬事都變作奮土.
為要得著基督.
其實重點就是.
當認識耶穌之後.
你的視覺有沒有改變過.
這是很真實的.
常常都問弟兄姊妹.
你信主前和信主後.
你判斷是非的原則有沒有改變過.
你信主前和信主後.
你對事件的看法和價值觀有沒有改變過.
在判斷的過程當中.
你是用什麼原則.
用什麼方法.
用什麼價值去看待那件事呢.
能不能像保羅所說.
我看耶穌基督為至寶呢.
但是耶穌在當時的羅馬政權.
在肥勒比城這些地方.

$^{401}$不是一些很亮麗的公職.
你會發覺羅馬的政權.
要將一個人褫奪他的收入.
是用一個naked的方法.
就是脫光他的衣服.
第二個方法就是要高舉去示眾.
這兩樣東西在耶穌基督身上都出現了.
但是今天這個人能夠被脫光衣服.
還是高舉去示眾的時候.
你居然說這個人.
你認信這個人是至寶.
你會不會太過分了.
但是保羅就是說.
我不要拿我的身份.
我的羅馬公民權的身份.
我仍然高舉我主耶穌基督.
並他釘十字架.
我已經為他凋棄萬事.
看作糞土.
這個轉念對於保羅來說.
就是他要得著的東西.
在過去19年到現在.
我們Folk Church的講道都不難聽過.
就是我們實在在香港做基督徒.
活得舒服了起碼有60年了.
我們真的不用受到什麼逼迫.
要求甚至為自己的名義.
基督徒的名義.
而做一些捍衛性的工作.
但是大家去過宣教工場.
無論是北上也好.
或者南下其他地方.
你會發覺有很多地方.
不容易去講福音信仰.
不容易將聖經在一個陌生人面前.
攤開和我一起講.
要做很多轉化.
因為在當地來說是禁止的.
但是你會不會覺得.
當地是禁止.

$^{441}$回來香港可以讀.
你又覺得很開心呢.
又不是很多人是這樣.
你問我.
我有個笑話.
但事實上是不好笑的.
就是你家裡有多少本聖經.
很多人都不止一本.
但都放在櫃子上.
你未必要這樣供奉.
但你不會拿出來.
很多人手機上都有聖經.
但是又未必要拿出來看.
不論什麼都好.
對我們來說.
方便的東西.
又或者是隨手可得的東西.
對你來說是否捍衛自保.
是否在幫助你.
反而是你自己的執念是什麼.
你持守的是什麼.
這是很重要的.
所以.
本來要提醒一件事.
就是第十五節.
所以我們中間凡是完全人.
總要存著者的心.
若在什麼事上存別人的心.
神也必以此指示你們.
這裡說什麼呢.
讀完第十六節.
然而你們到了什麼地步.
就當接著什麼地步行.
保羅也有目者心腸.
想讓我們更加理解.
其實有些東西可以按步驟班.
不是一步就這麼大步.
上帝也不是要求我們一步到位.
所以慢慢調整自己的心態.
所以這個字出現了.

$^{481}$凡是完全人.
總要存著人的心.
完全這個字.
保羅在菲勒比書用過兩次.
這個字對於什麼呢.
就是提尼沃斯這個字.
是在說一個遠像.
就好像耶穌在登山寶訓.
說一個小結的時候.
你們要完全像你們天父一樣.
那要怎樣了解提尼沃斯這個字呢.
通常我們就叫.
這個字根叫提尼.
提尼就是遠的意思.
就像現在在拍照.
我前面這個就是廣角鏡頭.
但這邊就是提尼.
遠照它.
這邊也是.
兩邊也是.
可以zoom遠的.
提尼就是一個遠照它的東西.
遠照它.
另外這個字根.
比大家更加熟悉的意思.
望遠的那個叫什麼.
英文叫telescope.
scope就是view.
tele就是遠.
你能夠.
telescope就是望遠鏡.
能夠看到遠那個照它的位置.
所以telescope就是看遠的東西.
望遠鏡.
望遠鏡能看到的東西.
它在不在這裡.
它在這裡.
在不在這裡.
不在這裡.
在哪裡.

$^{521}$在那裡.
那裡在哪裡.
遠那裡.
那是什麼.
那東西是不是.
是的.
不過是不在這裡.
不在這裡.
但你走著那個方向.
你就會去到那裡.
所以我們要完全像你們天父一樣.
你完全沒有.
我看到這樣.
你完全的了.
因為耶穌基督已經救贖了你.
你所有罪已經買清了.
你完全.
但你完全沒有.
你未完全.
因為你還會犯罪.
是不是.
什麼時候才會不犯罪.
耶穌回來的時候.
你們就不會再犯.
我們就不會再犯罪.
OK.
講得清楚一點.
是的.
所以重點就是.
耶穌會回來嗎.
會.
回來嗎.
還沒回來.
所以Telaius的詞就是這樣.
完全就是這一刻.
我們基督已經救贖我們.
我們完全.
但完全沒有.
未完全.
但你會看到那個方向.

$^{561}$明白嗎.
所以.
本來要提醒一件事.
我們中間.
凡是完全的.
都要有個心.
就是.
大家的心是怎樣.
這個心上帝會指示我們.
每個人做的力度不同.
每個人做的步伐都不同.
不用拍齊的.
但最重要要做.
所以說將來就是.
不要再停在這裡.
我們要走.
為未來朝著這個方向走.
望遠鏡的地方.
告訴你.
那裡遠處有要去的地方.
它在不在.
它在.
它在就不在這裡.
你要去那裡就要怎樣.
要走去.
或者坐車去.
或者有人帶你去.
重點就是要去.
那個是一個將來式.
但同樣都是現在式.
你明白我的意思嗎.
所以.
要然而我們到了什麼地步.
就當照著什麼地步走.
重點就是要做.
保羅提醒.
肥辣比較會.
弟兄姊妹一個很重要的訊息.
日子不會好過.
周圍的環境不會因為我們做了什麼會變.

$^{601}$變不是外面.
變是我們自己.
日子不會好過.
周圍的環境不會因為我們做了之後會變.
但是.
重點就是我們要繼續做下去.
這個很重要.
所以保羅去到第17節說什麼.
弟兄們.
你們要同校法.
我也當留意那些照我們榜樣行的人.
所以保羅就說.
我們正在做.
你不懂做.
或者不知道怎樣做.
或者是.
想跟呢.
你跟我們吧.
也當留意那些照我們榜樣行的人.
這個榜樣是誰.
是不是很虛的.
不是.
在第二章提了兩個都是junior.
是跟著保羅走的.
那個是誰.
一個是提摩泰.
第二個是爾巴弗提.
在第二章下半段就有了.
保羅提醒一件事就是.
不是要一個人做.
是要一班人做.
不是一個人叻.
要是一班人一起做.
這個很重要.
今天我們的心態就是覺得.
有人做就行.
自己不用做.
有人帶著就行.
我們跟吧.
跟可能是別世做的.

$^{641}$但是跟著都要做.
因為總要有人接手的.
所以.
保羅提醒我們一個很重要的訊息就是.
是一個colonial.
一個團契.
團契對於保羅來說就是一個群體.
保羅常常是希望我們建立一個屬靈的群體.
不要孤單.
也不要覺得自己很孤單.
上帝會為我們預備一班人.
兩個禮拜前啟威就說過七千人.
我不知道我們是不是那七千人當中.
帶我們學習成為一個群體.
所以你看到我常常都說.
希望弟子妹一起參與.
一起去有一個synergy.
就成為一個群體.
我記得在MIFC剛剛來到的時候.
我說過一篇道.
那時候也是在做直播.
就是在說誕生的那個月題.
十二月.
我其中一個訊息就是在說一個.
重點就是.
如果你覺得你自己的才幹.
是只有一千.
回應上個禮拜的訊息.
五千 一千 二千.
這些 或者十定銀子.
那些.
如果你覺得你自己是只有一千.
你是二千.
或者五千.
或者你覺得不好了.
我沒有一千 我只有一元.
只有一元.
我什麼都不是.
但我想說.
如果教會要一萬元的話.

$^{681}$我們教會.
不是要一個五千元.
兩個五千元.
湊夠一萬元.
教會其實應該要一萬個一元.
去成就那件事.
整件事.
從來都不是說.
一兩個人做.
幾個人做 是一群人.
你願意執意去.
跟隨耶穌.
願意執意去.
上帝讓我們.
靈明根深的那個價值觀.
看這個世界的時候.
那個遠像在那裡.
不是在這裡.
我們朝著那個遠像去走.
那個就是上帝.
要我們走的地方.
你不知道怎麼做不要緊.
上帝說什麼 經文說什麼.
一個很重要的訊息.
你什麼地步 上帝就告訴你.
走到什麼地步.
關於我們自己的時候.
保羅開始轉轉筆風.
就提一提.
因為很多人做事.
跟基督的十字架作對.
我屢次告訴你們.
現在又流淚告訴你們.
通常這些位置都是保羅曾經.
經歷過一些冤屈.
或者不對等的事情.
看《少爺行傳》就看到了.
但保羅用了四句.
很重要的騙語.
去說這群人的不是.

$^{721}$我排一排出來給大家看.
第一句就是.
他們的結局就是沉淪.
他們的神就是自己的.
土福.
以自己的羞辱為榮耀.
專以地上的事為念.
這個字又出現了.
終結局.
TALOS.
就是剛才那個字根.
TALY那個字根.
保羅已經在說將來式.
這群人現在是威是勢又如何.
他將來的結局.
就是沉淪.
而重點.
告訴你.
他們的神就是.
自己的土福.
因為現在這群人在他們的環境.
在他們的特權階級是吃喝快樂的.
是中辱的.
是將不同的人壓榨下去.
抬高自己的.
滿足自己的.
土福.
他們以自己的羞辱為榮耀.
對於我們來說.
剛才我也在說.
就是.
他們的價值觀和我們不同.
他們想用.
脫光衣服.
高舉來羞辱耶穌.
但是對於我們來說.
這是耶穌承擔我們.
邊傷醫治我們的痛苦.
他們會專以地上的事為念.
就是想他們現在眼前所見的東西.

$^{761}$他們想他們能夠觸手可見的東西.
這個字呢.
我希望多說一點字.
因為保羅常常用的字語帶相關.
以地上的事為念.
念Pronier這個字呢.
其實.
在之前.
在第二章的時候.
就是以基督為心的心.
這些人就是.
以地上的事為他的意念.
但是我們.
因著基督的緣故被根深.
我們就以基督的事.
為基督的心.
所以仍然是回到那件事.
保羅希望我們.
信主之後要轉念.
轉念之後要有行動.
在過程當中.
不只是看那些人.
眼前的事.
我們要看將來的事.
現在的事.
是會辛苦的.
但是現在的事.
會過去的.
將來的事會來到.
保羅一直都是.
從現在到將來.
用將來提醒.
現在的現在.
同樣提醒我們.
今天我們一直都在經歷一件事.
就是上帝你什麼時候呢.
上帝我們如何呢.
常常我們都會問這件事.
但是保羅提醒我們.
上帝沒有離棄我們.

$^{801}$上帝仍然在當中.
所以最後.
那兩節經文.
保羅提醒我們身份是什麼呢.
我們的身份就是.
我們卻是天上的國民.
並且等候救主.
就是主耶穌基督.
從天上降臨.
剛才說過.
我們在這裡.
耶穌會回來.
耶穌在這裡.
耶穌已經完成了工作.
但是耶穌說我去完事.
為你預備地方.
在過程當中我們就在等待.
雖然盼望.
我們知道耶穌一定會回來.
但是等待的過程是不容易的.
但是.
我們所相信的就是.
等待的過程當中不是只有等待.
我們一群人一起等.
一起經歷那個信心.
和那個工作.
第二節.
「我萬有歸服自己的大能.
將我們這卑賤的身體.
改變形狀.
和他自己榮耀的身體相似」.
你會看到.
保羅仍然強調一件事.
就是世界的人.
不認識耶穌.
就覺得他是卑賤的.
他是被唾棄的.
他是一個羞辱的形狀.
但是我們.
看這件事.

$^{841}$卻是上帝的大能.
你拿著原則.
在保羅寫書的時候.
和你回想你過去聽的經文.
上帝為我們.
所作的是眼睛未曾看見.
耳朵未曾聽見.
連心都未曾想過的事情.
但是上帝.
用一個人看為愚絕的方法.
成就上帝的大能.
今天我們身體.
是什麼形狀.
保羅也在說.
很多人將不同的人定為不同的階級.
無論他穿什麼衣服.
出身什麼身份.
奴隸生的仍然是奴隸.
自由人生的是自由人.
所以身份.
因為這樣而定了人.
但是對我們來說.
上帝不是看這件事.
上帝會將那件事情.
轉化成為一個榮耀的身體.
可能對於大家來說.
要消化的東西很多.
因為保羅的層次.
說得很豐富.
很多時候在.
第二三四這三章裡面.
常常都在做穿插.
但是保羅要帶出的訊息就是.
我們現在經歷的事情.
是將來得到榮耀的過程.
我們現在經歷的事情.
上帝在將來.
告訴我們現在的人.
其實這是上帝知道的事情.
所以.

$^{881}$保羅提醒我們.
我們未來是要改變的.
是要轉變的.
但是轉變的過程當中.
是不容易的.
剛剛就.
上個禮拜三.
就是這個年度的.
Info Group最後一課.
今天最後一屆.
開始第一課.
通常都會和大家節目做很多.
過去三年的.
教會發生的事情.
其中.
有一個月.
是大家很難過的.
其實很多月都很難過.
不過那個月難過的地方是.
是.
收到很多弟兄姊妹.
在那些月份.
是不懂祈禱.
和不知道怎麼祈禱.
每次祈禱的時候就哭.
那是2019年9月.
然後.
2010年我們每個月都有月提.
那時候不是雙月提.
是每個月一個月提.
我們10月的時候就做了一個月提.
叫做祈禱.
那個祈禱.
設計師的版面是.
一雙手在石牆上.
一雙手的祈禱.
那個月就是講祈禱的主題.
John就講了一個月提.
第一講就是講.
灰土.

$^{921}$灰心的灰.
我就和他唱了一個不是反調的.
是朋友來的.
那個月提.
我講到叫做.
不灰土.
我用的經文就是菲律賓書第一章.
保羅為當時的人.
菲律賓教會.
祈禱是祈什麼.
那個月出了一首歌.
是最熱的.
那首歌叫做什麼.
聽到聲音.
叫做.
禱告的距離.
很多弟兄姊妹透過那首歌.
去得著釋放.
得著鼓勵.
那些日子是真的.
不知道說什麼好.
師哥就成為我們的幫助.
師哥就成為我們的提醒.
師哥也告訴我們我們不孤單.
上帝仍然為我們.
在不同困境當中.
預備一班弟兄姊妹.
可以同行.
可以分享.
可以承擔.
不代表現在不重要.
只是看將來.
不是 現在都重要.
因為我們要繼續走去將來.
所以走那條路.
是辛苦.
但是一班人一起走.
那時我講不悔討的經文.
拿了幾章跟大家分享.
其實保羅期望.

$^{961}$期望那個群體有什麼轉變.
保羅是這樣為他們祈禱的.
保羅祈禱什麼.
第一章第九節.
我所禱告的就是.
要你們的愛心在知識和各樣見識上不斷增長.
使你們能分辨是非.
在基督的日子.
作真誠無可指責的人.
依靠著耶穌基督.
結滿人兒的果子.
歸榮耀稱讚給神.
第一章裡面.
保羅很多時候寫書信.
第一章就將他最有感覺.
最想講的東西在當中講出來.
其實他知道.
我們的環境.
或者當時的環境是不容易的.
但是他為那班弟兄姊妹.
禱告.
愛心.
是一定要有的.
上一個星期John的訊息.
是一個末世的訊息.
那段經文之前.
是講末世的時候的情況.
是怎樣的.
馬太方第24章講得很清楚.
要攻打民國.
要攻打國.
其中我原本想講那段經文的一節.
不法的事真多.
人的愛心.
漸漸冷淡.
這件事.
我相信在這段日子.
你不難找到.
因為不法的事多.
你就沒有.

$^{1001}$心想做好事.
因為你覺得.
你的心很傷.
你的心很亂.
你都不知道怎樣做.
還有你怕很多問題.
你想了很多.
你自己不知道怎樣自處.
所以保羅提醒.
末世的日子其實怎樣.
未來的日子怎樣.
我們要有更.
更加有愛心.
或者我們的愛心當中.
要增長.
所以你們的愛心.
在知識和見識上.
多了來做什麼.
就叫我們分辨.
親愛的弟妹.
今天不容易分辨.
我們站在台前.
不代表一定比你們厲害.
但是在過程當中.
我們分享.
彼此提醒.
彼此一起.
去協作.
做好應該要做的事情.
這個就是在.
我們需要一起去學習的地方.
所以第一章結束的時候.
保羅在討論結束的時候.
他是這樣說的.
他說.
最重要的是什麼.
你們行事為人要與基督的福音相稱.
這樣無論我來見你們.
或者不在你們那裡.
都可以聽到你們的警方.

$^{1041}$知道你們同有一個心智.
站立得穩.
為福音的信仰齊心努力.
絲毫不怕敵人的威嚇.
以此證明.
他們不會沉淪.
你們不會得救.
這是出於實.
你會看到.
剛才我讀的第三章經文.
在第一章已經出現了.
清清讓我們去明白.
保羅早已將將來式的事情.
在現在的時候.
跟信徒說得清清楚楚.
其實我們正在走那個終局.
而那個終局是美好的.
只不過現在.
是一定難捱的.
我不知道大家喜不喜歡.
看那些超級英雄的電影.
其中有一個.
在《Endgame》的時候.
其實未來就是未知的.
所以.
每一個可能性.
是看這一刻的決定.
才能處理將來的可能性.
所以他的假道面容扭曲.
就是計算了很多可能性.
我不知道你們有沒有看.
不過重點就是.
仍然是在說.
當下我們每走那一步.
我們可能都未必做得很好.
但剛才說.
你怎麼走,上帝會啟發我們.
重點是我們繼續朝著那個方向走.
才是很重要.
所以.

$^{1081}$最後那兩節經文保羅是這樣說的.
因為我們是什麼呢?.
我們是一群蒙恩的人.
我們是一群信服基督的人.
連基督都會受苦.
我們就和他一起受苦.
而基督都要去征戰.
而基督帶領我們去征戰.
因為你的征戰.
就與你們曾在我身上見過.
現在所聽到的事一樣.
保羅能夠坐牢都這麼開心.
因為他知道.
只是有時限.
保羅能夠坐牢.
也有時限.
保羅能夠坐牢.
也有時限.
保羅能夠經歷苦楚.
因為他執意知道.
耶穌是這樣做.
我學習.
耶穌也是這樣做.
我活著就是基督.
是保羅上上提醒的.
所以未來的轉變.
我不是有什麼板斧.
我只不過再重現.
保羅他要迎向未來.
的心智的轉變.
就是什麼?.
就是我們的信仰.
是因為基督的緣故.
我們的生活就和基督的福音相稱.
因為他能夠改變人的大能.
而我們有兩件事要記住.
就是不要讓世界同化.
這個世界的終局.
只是滅亡.
這個世界是不會有盼望的.

$^{1121}$但是我們的盼望.
是因為我們已經是天國的子民.
就像剛才所說.
你看到的東西.
那個地方是不是在這裡?.
在這裡.
是不是在這裡?.
不是在這裡.
但是我們將會去那個地方.
這個關鍵是我們要拿著.
所以拿著什麼呢?.
就是用上帝.
耶穌基督的心為心.
成為我們走每一步路.
走向未來轉變的開始.
今天花了比較長的時間.
和大家看《斐拉比書》這段經文.
我希望在年終的時候.
大家有空間靜一靜.
想一想.
你今年持守的是什麼?.
你要開拓二,三年的執意是什麼?.
過去在香港.
或者你去了海外的地方.
你的信仰的紮根的轉變是什麼?.
以致你面向將來.
上帝要走我們走的那段路.
要走的路的力量是什麼?.
弟兄姊妹.
每個人都會有一個判斷的時間.
我相信你不會是那些.
你不想那個問題.
那個問題就不存在的人.
你會發覺.
你不想那個問題都會來.
你要記住.
這個世界不是你不找它.
它就會離開你.
這個世界就是你不想它.
它都會找你的.

$^{1161}$因為它想將你同化.
如果不是.
怎麼會像咆哮的獅子.
到處尋找很吞噬的人呢?.
不要想著可以獨善其身.
反而要回到出來.
和這個信仰群體.
持守基督的心的執意.
走未來的路.
這個才是上帝喜悅過的生活.
Amen.
我們一起祈禱.
天爺上帝.
每當我們真的打開聖經的時候.
很多事情在過去提醒.
到今天仍然生效.
每一次聽到書信上的提示.
仍然對今天我們.
要執意行上帝的道路.
仍然是一個很重要的提示.
求主你幫助我們.
聖保羅勸勉信徒.
要走的路不容易.
但是仍然有很多前行的人.
做了腳蹤.
叫我們可以學習.
今天我們要走的路都不容易.
但是我們仍然看著腳蹤.
我們一群人.
持守這個信.
就行上帝給我們要行的宗旨.
求主你加添我們的信心.
加添我們的心力.
團結我們.
奉耶穌的名求.
Amen.
\newpage



\section{撒母耳記上 18:1-20:42-20221119}
\label{sec:UHPz7So3h50}
\textbf{【網上聖餐崇拜】致未來的牧者…友誼篇|撒母耳記上18\_1-20\_42|20221119 [UHPz7So3h50]}
\newline
\newline
連結: \href{https://youtube.com/watch?v=UHPz7So3h50}{\texttt{ https://youtube.com/watch?v=UHPz7So3h50}} ~~~~ 語音日期: 2022-11-19 
\newline
\newline
\hyperref[sec:giVKoZv8XXY]{\small{< < < PREV SERMON < < <}}
~
\hyperref[sec:index_chronic]{\small{[返順時目]}}
~
\hyperref[sec:index_scriptual]{\small{[返順卷目]}}
~
\hyperref[sec:ipiBLnwp8PI]{\small{> > > NEXT SERMON > > >}}
\newline
\newline
$^{1}$好,各位頂智媒晚安,海外的頂智媒晚安.
這個月題其實是一個很難說的月題.
說未來,因為我們沒有水晶球.
所以其實說未來其實是一個很不容易的課題.
坦白說,這篇的經文,待會說的那段其實我改過來的.
基本上兩三個星期前我定了另一段經文說的.
不過,這個星期,即是過去的六天的時候就轉了一段經文.
這個課題對我來說,想了很久不知道該說什麼.
我們先看看PowerPoint.
我們看看PowerPoint.
我想說一下我神學畢業的時候,1999年.
那時候剛出道,那時候說未來教會是說什麼.
我還記得那時候立志一生,那時候很敬虔.
今天也是.
就是想用聖經改變,聖經不要解得很奇怪.
希望能夠用聖經解得好一點,能夠幫助教會.
那時候有這樣的想法.
或者對教會是有期盼的.
不怕困難,或者覺得沒所謂的困難.
其實那時候我們老師經常都跟我們說一句.
那句話經常都在騙我們,今天也在騙我們.
他說最重要就是鬥長命.
總之你教會裡面長命過那些人,你就贏了.
仇遠老師也是這樣說的.
你不要放棄,經常叫我們不要放棄.
鬥長命,總之他們覺得我們會長命一點.
就是類似這樣,那時候,99年的時候是很覺得.
未來對香港教會,譬如97之後.
教會的生態是怎麼樣,或者面對一個競爭之後.
我們可以怎樣做一些事情,能夠建立得更加好.
大體上99年是這樣的.
或者往後的日子裡面.
我們畢業的時候,同學畢業的時候的未來.
是有一些憧憬和盼望的,不是那些.
很……不知道怎樣了,沒有的.
但是下一張.
其實現在,譬如說未來的目者.
譬如說這幾年的目者,或者接下來這幾年的目者.
其實當他們說教會要出來做的時候.
其實他們,我猜,不是想到很多東西.

$^{41}$有的是很多unfinished business,那些回憶.
有沒有意外,一個月後,12月大約第二個星期.
彭牧師會離開他被囚禁的地方,我猜測兩三個星期他會出來.
我們有很多unfinished business.
譬如好像過去那兩天,睡不著覺,不知道為什麼.
很慘,有兩天的時候,兩三點起床.
然後睡不著覺,不知道為什麼,狀態很差.
我記得那兩天有些時間祈禱,在睡不著覺的時候.
那個祈禱是回到三年前在紅磚屋.
就是我進去紅磚屋的時候的經歷.
就是在凌晨兩三點的時候.
就是連續兩晚驚醒了,睡不著覺.
祈禱的是三年前發生的事情.
unfinished business好像成為了我們裡面有很多東西.
你看到前面的時候,你會發覺你後面有很多東西卡住你.
有很多東西好像不知道怎樣,令你很為難.
不知道怎樣走下去的感覺.
我不知道大家怎樣,但起碼對我來說.
其實我有個powerpoint寫下六七件事,但我不敢打出來.
其實我講一句就寫完了.
我想講那六七件事,但有些事不是很好講.
現在講話很煩.
所以我只能夠講的就是.
你會有很多不同的東西堆積在你裡面.
成為你往前路的時候,好像後面有很多東西還沒解決.
和我99年的時候,仇遠畢業的時候.
你會覺得,哇,耶,出來吧.
你會改變一些東西.
但現在這一代的人出來的時候,面對場景.
或者留下的人在香港這個土地裡面.
你看著前面的時候,我覺得和他都一樣.
總是有很多回憶還沒sell down.
我記得那兩天祈禱的時候.
進去紅磚屋的時候,看到那些情景.
不詳細講了.
你可以想像那些東西還在你裡面纏繞著.
好像告訴你,其實你三年的事情都還沒過去的感覺一樣.
所以我想今天能夠分享一點.
特別這大半年,有很多傳道同工和宣教士.
尤其是這大半年.

$^{81}$我碰到的時候,很特別的感覺.
基本上這大半年裡面,都不少.
我猜沒有20個,都有17,18個.
這一類的同工會走來和我說話.
碰到,不知在哪裡見到面的時候.
他們會說,其實我們都是聽Fold Church講道的.
你明不明白,其實他們都沒有在會.
他們都有他們可以做的,聽的東西.
什麼叫做我們在Fold Church聽你講道.
其實是什麼意思.
有人和我說十幾次之後.
John,Poon Sir有些例外.
我講一下自己,我不覺得我特別什麼.
為什麼要聽呢.
他們可以聽,可能他和我說話就是講他們,不是講我.
我想,什麼叫做他聽Fold Church講道是什麼.
不少地在教務同工講這番話.
我懷疑,我懷疑而已,沒有什麼證實.
我猜Fold Church是在講一些大家的collective memory.
Fold Church仍然在我們面對很多以往的經歷裡面.
我們將那些經歷還在我們前面,放在前面.
還沒放下,我們不是說不要什麼,要向前望.
我們沒有這麼容易.
但起碼很多人和我們講一番話.
當有些回憶出來的時候.
前面的路是讓我們消化這些回憶.
前面的路不是只要望向前面,哇很bright,很漂亮,很怎樣.
過去的事情還沒settle,還在unfinished business的時候.
我們怎樣去digest這件事在我們前面的裡面.
我相信這件事成為了.
人們會覺得要聽Fold Church講道的地方.
上個月我和John在後面聊天.
我和他說,下年,後年我不想.
這些話就不想公開講,真是很煩.
不是,就是想講什麼呢.
總之很簡單講,不要講這麼複雜.
這些錄影很煩,出了家.
總之我們就說.
其實現在出去講道不知道講什麼好.
Fold Church是一個地方我們可以.

$^{121}$講道我們覺得想講的東西.
或者我們覺得可以是我們一些.
面對前面的時候,我們有很多未完成的記憶回憶的時候.
這些是在shape我們前面走的每一步.
而不是我們突然間覺得切割了.
什麼都不記得了.
我們可以前面就叫做未來.
以前可以這樣叫未來.
但我想望向前面的日子一樣.
烏克蘭的頂尖一樣.
我所知道波蘭有很多烏克蘭的難民.
其實他們一樣很熱誠.
他們敬虔,他們很愛主的態度.
在波蘭的頂尖會猶甚.
如果大家知道這些歷史的話.
你聽這一兩年的話.
他們在裡面熱心愛主.
不是因為他們比波蘭的人特別熱心愛主.
是因為經歷,是因為那些未完成的事情.
令他們望向前面的路的時候.
那些東西成為他們的推動力.
所以現在講未來的時候.
我們要講什麼.
這是我很苦惱這幾個星期的掙扎.
坦白講,講未來.
面對回憶的時候.
那些回憶出來其實是不好受的.
那些記憶的片段很不容易.
我們寧可做一個分割的人,分裂的人.
繼續生活,繼續忙碌.
但我相信我們的信仰不會容許我們這樣走下去.
我還有幾點,不要這麼說了,我們講聖經.
我們下去.
今天的題目是講未來的目者.
純粹捉一些噱頭,因為我的噱頭很差.
但目者,你知道我的意思不是全職這麼簡單.
我們當中的姐妹都是一個目者.
in all means.
好,我們講一下正經的事情.
我們今天講約翰和大衛.

$^{161}$我今天想講友誼.
Friendship.
第一,我們講約翰和大衛的故事.
其實大體上是在中學生團契.
或者小學生團契的時候講的.
我們很少在成人團契裡面查約翰和大衛的.
因為我們講約翰和大衛.
有約翰為大衛附上一切.
這些只能騙中學生和小朋友.
你明白嗎?在整個世界裡面.
你什麼時候見到你生命裡面出現一個約翰.
除了見鬼.
(笑聲).
所以在成人世界裡面.
我們很少講約翰和大衛.
你突然抓一個人出來說你是約翰和大衛.
即是怎樣?我要死在你刀下嗎?.
你明白嗎?這個論述是很困難的.
另外要想一想的是.
其實聖經裡面不講友誼.
你不要計針研.
不要說約伯記的三個朋友.
另外一個朋友的主朋狗友.
不是講那些.
真的講友誼.
其實聖經是不觸碰的.
你什麼時候見到聖經講友誼.
沒有的.
你見到戈恩和阿伯是你殺到我殺你.
一個殺一個.
阿伯沒有殺人.
你見其他東西都是你對我我對你.
友誼在聖經裡面不受歡迎.
基本上是不講的.
你很少聽到一些很浪漫溫馨的友誼.
我不敢直言說友誼在這個世界裡很穩定.
但起碼我們要問的問題是.
為什麼聖經這麼少講友誼.
或者反過來講.
其實約拿丹和大衛的友誼是什麼.

$^{201}$我們要問約拿丹和大衛的友誼.
是不是中學生或私學團契.
在你的心靈裡最重要有個好朋友.
好朋友跟你一起面對.
有很多難處.
你明白嗎.
是不是這個關於約拿丹和大衛要講的東西.
這個是我要問的問題.
除了很現實很世俗化地覺得.
我們新唐人裡面不再有很多約拿丹之外.
每個人都想有大衛.
沒有人想有約拿丹.
你選朋友想做誰.
我當然想有大衛.
有約拿丹肯為我付出一切.
多好啊.
除了這些之外.
其實整個18至20章的《十味記》上.
講了關於大衛和約拿丹的東西.
其實講了一些很老土的東西.
多謝多謝18章一節.
其實這些很老土.
什麼心心相契合.
因為我不翻譯.
通常我都想翻譯自己的版本.
不過因為太過.
我不想說什麼.
太過.
給我一個字眼.
什麼心心相契合.
什麼約拿丹心與大衛的心相契合.
你的閨密不會講這些東西的.
你明白嗎.
我做個蛋糕給你.
你做個蛋糕給我算了.
你很少說我和你的心心相契合.
還要說約拿丹愛大衛.
如同愛自己的姓名.
如果要講一點翻譯的話.
其實那個心.

$^{241}$其實和姓名的字一樣.
姓名的字其實就是.
心的字.
沒理由說約拿丹的姓名.
和大衛的姓名心心相契合.
更加不知道說什麼.
原文是這樣的.
今天不討論這個問題.
18至20張很多東西要搞.
不搞那麼多時間.
很老套的.
約拿丹愛大衛如同愛自己的姓名與他納約.
我們沒有閨密是納約的.
你明白嗎.
戴手指印來吸引.
我和你心心相契合那些.
或者用手指公.
碰你額頭一下.
印一印你.
耶和你納約.
少了這麼低B的東西.
即是.
但很奇怪.
即是.
很feminine.
突然間寫作.
約拿丹和大衛.
是這麼feminine.
兩個男人.
先不要扯歧視.
兩個牛高馬大的男人.
一個打敗哥利亞.
一個是索羅的長子.
應該是英明神武的感覺.
突然間他寫了一些很肉麻的東西.
不止一次.
你看第19章也是.
又是這句.
索羅的兒子約拿丹卻喜愛大衛.
其實他不是喜愛.

$^{281}$那個字.
其實那個字是.
原文就是.
我喜歡你所做的事.
不是愛的意思.
只不過是喜歡他所做的事.
我們再看20張.
又是突然間.
很老套的東西.
約拿丹因為愛大衛.
如同愛自己的姓名.
就叫他再起誓.
整個第18.
19,20張講的東西.
都是關於大衛要逃避索羅王的追殺.
所以整個約拿丹.
如何在這件事裡說話.
有些很肉麻的東西.
我不方便翻譯.
如果你再.
聖經是有些所謂.
很粗俗的東西.
很粗俗的.
就在第18,19,20張裡.
你找到的.
公開場合我們不說了.
不要說很粗俗的東西.
索羅罵約拿丹.
罵得很粗俗.
只不過中文翻譯.
已經翻譯得很溫文易亞.
你沒試過希伯來文.
那種粗俗你可以研究一下.
不過不說那些.
但整個三章聖經裡.
刻意說.
這段感情.
花了很多脈絡去說.
這是我想問的問題.
我想問為什麼.

$^{321}$花了三章聖經.
追殺他就追不到了.
就走了.
大衛一定有主角光環.
他一定帶光環.
你追他都不會死.
你可以想像是這樣.
就像龍珠.
龍珠的劇集.
人要和怪獸打.
打的時候也不知佔劇情.
只有20\%.
80\%也不知說什麼.
就像足球小將一樣.
球永遠都不會進去.
90\%是說大智慧以前是怎樣.
太耐是怎樣.
你明白嗎.
為什麼要搞這樣的劇情.
聖經為什麼要刻意搞這些劇情出來.
或者這些劇情出來.
想表達什麼.
我們看看後面那句.
我想說的話.
其實約留丹對大衛.
衛生的情懷是表明了一些東西.
是表明.
約留丹願意.
放下他的王權的意思.
我們說.
其實寫這三章聖經的時候.
其實他想說的是.
約留丹是自己願意放下.
這個王權.
我講得再清楚一點.
我的意思是.
其實這三章聖經裡說約留丹和大衛的感情.
我們可以聽到感情.
我不想講這些題目.
不要講這些東西.

$^{361}$有些人會講某些東西.
我們不講那些東西.
但是論述者或者敘事的人.
要將約留丹成為一個很女性化的角色.
他表達愛情.
其實不只是約留丹表達.
不過這三章都是約留丹表達.
全部都是約留丹主動.
所有動詞都是約留丹做的.
但其實在三位姨姬下的第一章.
當約留丹死了之後.
大衛反過來講.
大衛其實講的是.
約留丹對他的愛情勝過任何婦女.
這些話不知道是什麼.
你試一下跟你老婆說.
我和某某的好兄弟的感情.
勝過任何的婦女包括你.
當然我們今天.
我們不應該用今時今日的眼光講這些東西.
其實不公平.
但是你可以想像.
大衛最後寫了一首詩.
紀念約留丹的時候.
那兩個男人是很特別的.
但十八至十章裡面表達的東西.
是想講的是.
論述者特別要將約留丹的感情.
或者他對大衛的愛情.
放在某一個位置裡面.
這樣寫的時候.
是想講.
大衛其實不是在褫奪.
蘇羅的王位.
我再給你一點例子.
你記不記得大衛在《引機》底洞裡面.
蘇羅在小便.
你記不記得.
蘇羅在小便.
大衛就在聞.

$^{401}$他都真的在聞.
他拿刀割他的衣服.
然後出去的時候.
我不殺你.
你記得嗎.
另一次就在什麼曠野.
西伐好像是那個地方.
蘇羅又在睡覺.
約留丹就拿了他旁邊的槍.
其實可以一刀刺死他.
他又不刺死他.
所以大衛經常講的東西是.
在過程中.
這個是耶和華的仇高者.
所以我不殺他.
最後蘇羅死的時候.
有個拿蘇羅兵器的人.
去跟大衛說.
大衛可以作王.
他會做事.
就去找大衛.
大衛我殺了蘇羅.
其實是沒有殺到.
如果看審美記上殺一張.
誰知大衛立即說.
你承認了你殺他.
OK Fine Good.
你殺了神的仇高者.
所以你都要死.
所以整個脈絡裡邊.
由18 19 20.
你殺到後面的時候.
你發現神的仇高者.
是不可以隨便動手的.
你明白我的觀念嗎.
所以大衛最後能夠.
拿到蘇羅的王權.
蘇羅死了之後.
當然約留丹也死了.
同場死了.

$^{441}$被亞馬利亞人殺死.
但其實那個王權繼承者是誰.
是約留丹.
所以18 19 20 章已經講明.
不理他約留丹和蘇羅一起死也好.
約留丹是自己願意放下.
寫他這麼feminine.
或者寫他這麼.
好像很愛大衛的緣故.
是想讓所有人都知道.
事實是甚麼.
大衛其實是耶和華的仇高者.
而你們以為是蘇羅的後人.
但蘇羅的後人.
約留丹告訴你其實不是.
所以約留丹說了一句話.
他說願意你的仇敵快快死去.
其實仇敵是他爸爸.
你明白嗎.
他有這樣的說話.
跟他納約的時候.
約留丹親自說的.
他說我希望將來做你的宰相.
這樣說的.
但宰相的翻譯不一定準確.
我們不教太仔細的.
所以如果你這樣理解和明白的時候.
約留丹和大衛之間.
其實是在說.
誰願意放下他應該有的東西.
去成就對方有的東西.
我再給一個小例子就夠了.
我再下一個power point.
Shakar這個字.
希伯來文的這個字.
Shakar這個字.
你見到經文裡.
我看我的power point好一點.
他怎樣說.
他在十八章裡面.

$^{481}$第五節.
十四十五節.
和三十節裡面.
都用了更精明.
更精明這個字Shakar這個字.
好奇怪出現很多次.
這麼多次表明大衛做事能幹.
有能力.
很smart.
是因為想表達什麼.
想表達大衛是承認將來做王.
如果這樣爭論下去的話.
整個罪事者寫這個故事的時候.
是想將整個大衛的能力.
他的智慧提升.
約翰丹沒有什麼可以做.
好像一個太子.
但他好像一個沒什麼用的太子一樣.
只說我愛你.
我愛你.
我愛你的太子.
你明白嗎.
只是說我和你納藥.
我幫你走路.
我被爸爸罵.
沒所謂的.
這些對比.
更加呈現一幅.
很真實的圖畫是什麼.
原來友情.
在這裡來說.
不是說我們想的那些閨密.
我們不是說那些.
我這樣做因為你會維護我.
而你這樣做因為你也會維護我.
我們不是在職場裡說的那些夥伴.
也不是在教會裡.
你出去行走江湖的時候.
你拿著卡片的時候.
派給別人的時候.

$^{521}$我在圖畫裡做什麼.
那不是這類型的交往叫友情.
起碼不是.
起碼罪事者想表達的是.
這個友情.
這個相愛.
這兩個人的契合.
是在說對我有好處.
即是看回約旦.
不是我應該承接皇位嗎.
什麼時候輪到大衛你承接.
我是蘇羅的兒子.
是在說友情.
不是在說我和你有好處.
有好處不是.
這個友情是說明明我有好處.
我放下那些好處給你有好處.
這個define友情和友誼.
和我們一般理解的友誼不一樣吧.
我…謝謝大家.
你記不記得上個月說的篇.
你記不記得我說大衛打哥利亞的時候.
我想說大衛是對上帝有imagination.
你還記得嗎.
迪士尼可能還記得.
起碼你記得這個termine很好.
如果要跟17章的脈絡說的話.
17章完就有19-20章.
我今天想表達的是.
約拿丹願意犧牲給大衛.
對於敘述者寫的17-18章.
他所說的是.
約拿丹願意犧牲愛情.
不是純粹我和他做朋友.
是他人生中看到什麼.
看到大衛對上帝有imagination.
你明白嗎.
一個信主群體裡.
不是所有人都對上帝有imagination.
大部分的肢體門.

$^{561}$你說什麼我就做.
不錯啊 去吧.
約拿丹能夠commit在大衛身上.
應該是因為他看到.
其實剛才18節的心心劇合那麼老套.
是連著17章最後幾句.
他打完哥利亞之後寫的說話.
對於約拿丹.
是因為他看到這個人大衛.
對上帝有真正的imagination.
所以約拿丹在commit的是.
這個人對上帝有imagination.
身上.
今天說有情有義.
是想說什麼.
不是兩個人之間彼此有好處.
有alliance.
我越多朋友.
越多電影節目.
越多人和我friend熟.
看得起我.
like我的東西.
follow我的東西.
那些叫friendship.
你明白嗎.
你facebook一千個friend如何.
你最近研究.
我們人類最多只能有六七十個friend就夠了.
我們不知為何有千幾個friend覺得很開心.
你是不是傻.
你什麼時候和千幾個friend吃飯喝茶灌水.
你不會的.
那些不是的.
那些叫扮friend.
有情是什麼.
你看聖經define的是.
他commit是一個人.
他那個人對上帝有imagination.
我講現代的例子.
我可以講近代的例子.

$^{601}$講我自己的例子.
但這些東西不方便說.
將來再說.
十年後回憶錄的時候應該說一下.
我講一下我所認識的人.
我認識一個同學.
神學院的同學.
他很蛇歸.
我們宿舍搬出去住的時候.
我們要殺老鼠.
老鼠進入我們宿舍.
走來走去.
很麻煩.
難道與鼠同眠.
我們就找一些人一齊殺老鼠.
殺老鼠.
我這個同學是牧師.
牧師同學.
他現在做牧師.
他很斯文地走過.
哈哈哈哈老鼠.
為什麼大驚小怪.
很可恥.
大家在殺老鼠.
他在說.
為什麼大驚小怪.
幫忙都不幫忙.
我說.
身為將來的牧師.
你如此這般.
你哪裡來牧養到頂子妹.
你明不明白.
真是豈有此理.
我會憎恨他.
最後那年我同房.
哈哈哈哈.
我會講多一點.
我最近不敢這樣說.
他在教祖.
做了二十多年.

$^{641}$十年前左右.
他很可愛地.
安納了做牧師.
做了牧師.
我們就一齊.
但他教的傳統理念很好笑.
他們不安納女的.
只安納男的.
他本身有個女.
傳道成為主任傳道.
即是堂主任.
他做了牧師之後.
就談.
是不是你牧師大人.
你應該做堂主任.
他就跟我聊天.
我做什麼.
我笑他.
我記得在坐天星碼頭.
他打給我.
我好心你.
你問下你自己是誰.
你哪裡會做領導.
哈哈哈哈.
很直接.
我認識他.
是我的師姐.
帶我兩屆.
幾有能力.
幾powerful的女士們.
她的樣子一看.
你真的害怕.
一定是主任牧師的材料.
我好心你不要做.
你安納來幹嘛.
現在執事叫你上.
你不就該死.
他就預知.
他安完納就做.
多壞.

$^{681}$之後那一兩年.
每年逢是十二月就來了.
他叫我家仔.
你明白.
現在才升級做家Sir.
以前是家仔.
他說家仔.
我們教會要想年替.
你幫我想些辦法.
教會應該怎樣做.
接著我又跟他說.
哈哈哈哈.
年替而已.
幹嘛大驚小怪.
哈哈哈哈.
那一刻是我人生最.
最得人最過癮的一句.
報我老鼠之仇.
豈有此理.
我怎樣做門徒訓練.
每一個年替應該怎樣安排.
我說你不懂就不要做.
說完他就不明白.
你以為我告訴他他懂.
其實他不懂.
結果他熬了三四年.
終於去堂會.
終於通過了安納女傳道.
變了女牧師.
一做之後.
我第一時間就找他.
是時候推你下來.
不要搞來搞去.
結果他跟執事會說.
執事會不相信他.
執事會覺得.
你是不是謙卑得太多.
你明白嗎.
你執事會有問題嗎.
結果不是.

$^{721}$結果這個同學.
我這個同學牧師仔.
真的結果.
半年後.
執事會相信他.
說真的.
就將那個女牧師.
做回堂主任.
他現在這七八年.
就樂得清閒.
不需要想命題.
全年做什麼.
我最近請了這個同學.
因為他有奇特的經驗.
他人生奇特經驗不止一個.
還有一個.
但暫時不說.
他這個奇特經驗.
我抓了他們.
在我的哲學院找同學談.
他講完這個事件之後.
下面的同學.
一個comment.
大部分同學.
齊聲的一個comment.
他說為什麼我們在我們鄉教會裡.
不常見.
大部分下面的人.
聽完之後.
我馬上回一句.
是的.
永遠都存在我們鄉教的特色.
真理歸真理.
真理要實踐到.
在我們裡面.
尤其是自己身上的時候.
就變成一個很普通的說法.
我體會什麼.
一個人對上帝的imagination.
是值得人用所有東西.

$^{761}$去承傳那個人對上帝有的imagination.
我們可以沒有上帝的imagination.
每個人都有.
這件事可能很奇怪.
但如果上帝呼召人的時候.
那個人對上帝特別有感覺.
他知道那些東西是怎樣的時候.
真正講未來目者的友誼.
不再是我派張卡片給你.
你請我講道.
我請你講道.
你明不明白.
我去幫你.
你幫我.
不是我教會裡面要請一個神學院的老師.
來我教會裡面講道.
這句說話都不應該講.
將來沒有生意了.
不是要貼面.
貼上多點金的問題.
以前是可能是.
幸知有效.
但看著未來的時候.
我們選什麼人做朋友.
哪些才是我的朋友.
大衛講得很精彩的是約翰達成就了他.
人對上帝有想像力.
他可以在上帝裡面想很多東西.
為上帝發很多夢.
有很多人身旁.
看到這些東西.
願意付出他要付出的東西.
犧牲他要犧牲的東西.
去成就上帝的國實踐.
今天我們要問的目者們.
我們要取悅上師的喜悅.
取悅我們主教牧師的開心.
可以做的其實都不差.
我想說這些.
但什麼才是我們真正在目者之間的關係與友誼.

$^{801}$多謝這張powerpoint.
要結束了.
其實你不要以為大衛有好處.
你不要以為大衛打完哥利亞之後.
凡事精明之後有好處.
沒有的.
由21章到30章.
大衛在曠野裡十幾年.
很坎坷的.
對上帝想像力的人.
不是說他要抬著轎給他.
他就說想啊.
真正對上帝想像力的人.
是經歷人生很多起起跌跌.
他還能堅持下去的這班人.
那些經常在高位置的人.
從來沒經歷過很多磨爛的人.
請小心.
你看著大衛.
我們不只是抬轎抬那些人.
要把基督教普京的明星拿出來.
但那些人沒經歷過.
十幾章聖經.
要在阿吉和面前裝瘋扮癲.
把口水都流到鬍鬚上.
被阿吉和看著大衛.
這個癲的大衛.
那裡來他殺死萬萬.
那些英明神往哪去了.
今天我們要問的是.
我們把我們的眼目.
面對著未來.
有很多回憶的時候.
放在什麼身上.
還是我們互相有利益.
有好處.
那些叫做friendship.
還是我們的眼光.
看著聖經十八至二十章裡說的.
真正的friendship.

$^{841}$不是把人idolize.
捧幾個明星上場.
是我們看著那些人的生命質素裡.
他在經歷什麼.
他在做什麼.
不只是在說什麼.
真正佩服的人.
是你和他相處的時候.
他仍然是說話的他.
有很多人真正和他相處.
我們最近的小組裡.
我們小組有很多教會的幹事.
有四位.
那四位教會的幹事.
你們知道同一個結論是什麼嗎.
不知道能不能上街.
你明白吧.
不想上街.
你唯有相處才能看到真實的一面.
是什麼一面.
今天有很多人物無聲.
在做很多對上帝的角度有意義的事情.
找那些人做朋友.
成為了未來裡.
上帝在我們不知道的前路當中.
那些人會為上帝做到事的.
求天父連同香港的教會.
離開我們那種.
不得罪人.
那些東西.
專心看著那些.
上帝在這個世代裡會用的人.
求主幫助我們祈禱.
天父多謝你讓我們今天有這個空間時間.
讓我們生命當中去問一問自己.
我們的友誼在課什麼.
天父說許我們會想做大衛或深惡復害修煉的雅蘭丹.
但天父求你在我們生命當中.
賜下幾個很真誠的朋友.
是可以跟他說要說的話.

$^{881}$是可以彼此為著上帝各的緣故.
放下自己有的想法和想像.
投入上帝你自己角度裡的想像.
看到上帝在這個時候.
為我們應該要做的事.
求萬鈞耶和華.
你自己是明.
在香港這片土地裡.
仍然有更多這樣的人.
為你擺上了他們的一切.
求天父你憐憫我們.
幫助我們.
讓每個人來到你面前的時候.
都看到你的真實.
都看到你的榮耀.
以致我們的友誼.
在這個方向裡邊.
可以再走多半步.
天父你聽我們祈禱.
歡迎你寶貴名堂.
阿門.
\newpage



\section{馬可福音 8:27-9:8-20221126}
\label{sec:ipiBLnwp8PI}
\textbf{【網上崇拜】未來見|馬可福音8\_27-9\_8|20221126 [ipiBLnwp8PI]}
\newline
\newline
連結: \href{https://youtube.com/watch?v=ipiBLnwp8PI}{\texttt{ https://youtube.com/watch?v=ipiBLnwp8PI}} ~~~~ 語音日期: 2022-11-26 
\newline
\newline
\hyperref[sec:UHPz7So3h50]{\small{< < < PREV SERMON < < <}}
~
\hyperref[sec:index_chronic]{\small{[返順時目]}}
~
\hyperref[sec:index_scriptual]{\small{[返順卷目]}}
~
\hyperref[sec:4yYwkRP32_4]{\small{> > > NEXT SERMON > > >}}
\newline
\newline
$^{1}$各位同學,早安.
關於未來這個月題,我自己想了很久.
在我們今天身處的社會裡面.
我想變化的速度都很快.
快到我們要追趕上的速度都好像很吃力.
或者我們都好像沒什麼空間去想未來.
未來好像慢慢想的時候又好像變得很奢侈.
所以我想來想去都是覺得與其說未來.
不如我們今天就說一下究竟耶穌昔日怎樣呼召他的門徒去跟從他.
因為我相信無論未來怎樣變都好.
耶穌這個呼召都是很真實的.
都在我們每一個人的生命當中去邀請我們跟從他.
我們開始的時候有個禱告.
先祈禱.
今天的經文就選了在馬可福音八章二十七節到九章八節.
請聽我讀出今天的經文.
聖經是這樣說的.
八章二十七節.
耶穌就從門徒出去.
往該撒尼亞肥納比的村莊去.
路上就問門徒.
他就說人說我是誰呢.
他們就說有人說是施洗約翰.
有人說是以利亞.
有人又說是先知中的一位.
於是耶穌繼續問門徒.
你們說我是誰呢.
彼得就說你是基督.
於是耶穌就勸戒他們.
就不要跟其他人說.
於是就開始教訓他們.
人子要受許多的苦.
被長老祭司獎經學家棄絕.
並且被殺.
第三天就復活.
耶穌很坦白地跟門徒說這些話.
但彼得就拉住他.
就罵他.
於是耶穌轉身對著門徒說.
就罵彼得.

$^{41}$就說撒旦亞退我後面.
因為你不思念神的事.
你只思念人的事.
於是就叫眾人和門徒都來.
就跟他們說.
若有人要跟從我.
就當捨己.
背棄他的十字架來跟從我.
因為凡要救自己生命的.
必喪掉生命.
凡為我和福音緣故喪掉生命的.
必救了生命.
人就是賺得全世界.
賠上自己的生命有什麼益處呢.
人還能拿什麼來換生命呢.
在這個淫亂罪惡的世代.
將我和我道當作可恥的.
人在他父的榮耀裡.
聖天使一起降臨的時候.
都要將這個人當作可恥.
耶穌又跟他們說.
我實在告訴你們.
站在這類有人未嘗過死未而前.
必要看見神的國大有能力降臨.
過了六天.
耶穌就帶著彼得 雅各 約翰.
上了高山.
就在他們面前轉變形象.
衣服放光極其潔白.
甚至地上的布都不能漂得那麼白.
忽然有伊利亞摩西.
顯現跟耶穌說話.
於是彼得就跟耶穌說.
夫子 我在這裡真好.
我們可以搭三座棚.
一座為你 一座為摩西 一座為伊利亞.
彼得不知道該說什麼.
因為他們很害怕.
有一個雲就來遮蓋他們.
有聲音從雲裡出來.

$^{81}$就說這是我的愛子.
你們要聽從他.
於是門徒就四處看.
看不到其他人.
只看到耶穌跟他們單獨在一起.
相信我剛才讀的這兩段片段.
大家都不會陌生.
如果我們要搞一個基督教十大的.
popular經文選舉的話.
我相信這兩段都會榜上有名.
但即使是那麼熟悉的經文.
我們都有一些問題需要去思考.
或者正正是那麼熟悉的經文.
可能有一些問題我們會miss了.
或者看漏了沒有問到.
為什麼耶穌呼召他的門徒的時候.
第一個的命令會是寫記呢.
為什麼會是寫記呢.
很有趣的.
如果我們弄一個填充題的話.
若有人要跟從我.
其實你放任何的屬靈字眼進去.
都是work的.
就是好像都很make sense的.
譬如說若有人要跟從我.
可以就當為新.
就當火熱.
就當愛主.
就當悔改.
其實你發現你放任何詞語進去.
都好像是那件事.
所以今天的問題就是問.
為什麼耶穌第一個.
對於跟從他人的命令.
會是寫記呢.
寫記有什麼特別呢.
要耶穌煞有介事地去做出這個命令呢.
而我們知道這個字在.
《馬可福音》裡面是第一次出現的.
所以我們就開始去看看今天的經文.

$^{121}$首先我們發現.
無論《馬可》《馬太盧家》三卷福音書.
都有耶穌呼召門徒.
和登山變象的故事.
而很明顯.
他們都是間距了六天.
作者好像對六天中間發生過的事情.
是沒有什麼興趣.
很直接就將.
耶穌呼召門徒.
之後就馬上去到登山變象的畫面.
作者好像不想.
side track我們去看其他的東西.
因為他覺得.
耶穌呼召門徒就一定要去到登山變象.
所以我們就要想一下.
究竟登山變象有什麼特別呢.
為什麼作者要選擇這個片段去寫下呢.
我還是脫下口罩.
(笑).
為什麼要寫下呢.
如果我們細心去看登山變象這個故事.
由九章二節到九章八節.
我們會發現.
它中間很多詞語.
很多景象.
其實是和《出埃及記》裡面的.
摩西上西奈山的故事是很相似的.
譬如powerpoint已經展示了.
譬如有六日這個字.
然後又有三個同伴.
然後又會改變形象.
又會有雲彩覆蓋.
大家都會很害怕.
powerpoint就展示了五樣東西.
但其實有很多的.
所以我們會發現.
你看登山變象.
必然會想起摩西上西奈山這個故事.
所以我們繼續問下去.

$^{161}$想起這個故事又代表什麼呢.
所以我們就去到《出埃及記》.
去看看摩西上西奈山究竟是一件什麼事情.
如果我們打開《出埃及記》.
我們就馬上發現.
摩西上西奈山其實是一個.
橫跨二十多章經文的故事.
它不是一次的上落.
它是又上又落.
所以很簡單就將摩西上西奈山的故事.
分成三部曲.
很簡單的跟大家說一下.
第一部曲就發生在《出埃及記》第十九章.
經文就開始說.
以色列人來到西奈山的山腳.
安寧了.
於是上帝就說.
現在你看到我怎樣拯救你們.
你想不想和我納藥.
於是以色列人就說.
凡耶和華所吩咐的.
我們都必遵行.
於是摩西就代表以色列人.
上了這座山.
去聽上帝有什麼吩咐.
二十二十一二十二十三章.
就是十誡.
一些律例典章.
於是我們很快就來到第二部曲.
摩西聽完上帝的吩咐之後.
第二十四章.
他就下山.
就口頭和百姓說.
上帝有什麼吩咐.
十誡諸如此類的東西.
於是百姓就說.
凡耶和華所吩咐的.
我們都必遵行.
很開心.
於是摩西就弄了一個祭壇.

$^{201}$又弄了一些血.
又把律法寫在書上.
讀給以色列人聽.
以色列人聽完又繼續說.
凡耶和華所吩咐的.
我們都必遵行.
我們留意到.
以色列人是不斷說yes.
摩西就很開心.
但輾轉之後.
就再上山.
摩西就想看看上帝.
還有什麼其他的吩咐.
大經文這次就說.
摩西上了去四十天.
都一段頗長的時間.
由二十五章開始.
到三十一章.
就是我們經常讀經計劃.
會出不到埃及的地方.
不知道大家有沒有試過讀經計劃.
經常都死在埃及.
應該都是這幾章.
這幾章是說什麼呢.
這幾章是說會幕怎麼建.
有多大有多寬.
用什麼布料.
東西怎麼擺的.
祭典是怎麼進行的.
都是挺悶的.
所以摩西就聽了四十天.
於是我們就去到第三部曲.
就是三十二章.
這個故事突然之間出現了一個反高潮.
三十二章一開始.
聖經是這樣說的.
他說那些人見摩西遲遲不下山.
於是就圍著亞倫.
亞倫不如我們弄個神像出來.
帶領我們向前走.

$^{241}$因為那個摩西.
我們不知道他發生什麼事.
於是亞倫就收集了一些金器.
就弄了一隻牛杖出來.
就指著那隻牛杖.
就說這個就是耶和華.
就是帶領我們出埃及的那個.
我們一起拜祂.
於是百姓就弄了一個祭典.
獻祭.
就一起吃喝快樂.
就很開心.
摩西還在那座山上.
於是上帝就跟摩西說.
你快點下山.
因為你的百姓已經敗壞了.
我們知道故事最後.
摩西下山了.
很生氣.
於是就打破了那個像.
就輾轉搞了很久.
又跟上帝求情.
最後又上山.
又去聽上帝的吩咐.
如果你就這樣聽我說這三部曲.
可能你都會馬上.
有一個問題出現.
就是為什麼這個故事會突然之間.
有一個反高潮呢.
明明頭兩部曲.
以色列人都是很乖的.
不斷說梵耶和華所吩咐的.
我們都遵行.
但為什麼去到第32章.
就會突然之間.
去拜金牛獨呢.
這件事很神奇.
轉變在哪裡呢.
我們可以說以色列人的問題.
是因為他們覺得.

$^{281}$其實拜上帝.
跟拜其他神差不多.
在他們的經驗.
在他們的認知裡面.
因為他們在埃及做過奴隸.
埃及我們知道有很多的神.
有很多的神像.
五花八門的神都有.
雖然神多.
但拜的方法是差不多的.
大家都是有一個神像.
然後又是要獻祭的.
然後又是有祭典.
所以在以色列人等摩西的過程裡.
可能有一天他們突然之間覺得.
其實都不用等摩西.
這件事不複雜的.
我們都可以搞定.
拜神有多複雜呢.
就是那些東西.
又是要弄一個神像.
又是要獻祭.
又是要有一個祭典.
與其我們無了其等摩西.
不如我們先弄了.
他下來我們就可以馬上走.
或者就算我們拜完.
摩西不下來.
我們都可以照樣走.
我們拜這個神.
都是無非想祂祝福我們.
令我們可以獲取那個迦南地.
我們現在弄了.
就可以去.
以色列人的問題.
就是他們想到.
要這樣去拜.
然後他們就走了去拜.
他們沒有想過的是.
原來上帝第一個要他們做的事.

$^{321}$不是做什麼律例.
典章.
不是怎樣獻祭.
上帝第一件事要以色列人做的.
是要學習等待.
是要等上帝.
或者等摩西.
不過以色列人就錯過了這個位置.
就弄了一隻金牛毒出來.
如果我們去到新約.
我們就會發現.
其實門徒的問題.
都很類近.
如果我們記得.
我剛才有讀過的經文.
耶穌就說.
我是誰?問門徒.
彼得就說你是基督.
什麼意思呢?.
對於彼得來說.
耶穌就是那個受高者.
就是上帝在那個年代裡.
揀選了的人.
再簡單一點來說.
彼得覺得耶穌是一個人民領袖.
是一個會帶領他們光復以色列.
恢復大衛王朝.
打敗羅馬人的一個這樣的領袖.
所以當耶穌聽到門徒這樣說的時候.
馬上要他們閉嘴.
和要教導他們.
不是的.
人子是要受苦的.
是要受害的.
還要被氣絕.
還要被殺的.
所以彼得聽到耶穌這樣說的時候.
第一個反應不是問耶穌為什麼.
也不是關心耶穌.
彼得第一個反應是罵耶穌.

$^{361}$我想說的其實有時候我們誤會了.
門徒是不想受苦.
或者你誤會了門徒是貪圖安逸.
其實不是的.
門徒是可以為耶穌上刀山下油鍋的.
門徒也可以為耶穌去承受很多不同的事情.
甚至可以為他去打打殺殺.
去犧牲都在所不惜.
但前提是什麼.
前提是耶穌一定要是那個領袖.
所以當耶穌說他要死的時候.
彼得是很大反應的.
因為你死了我們還可以跟誰.
我們還可以指引誰去復國.
所以如果我們這樣去看.
無論新約的門徒.
或者舊約裡面的以色列人.
都是在分享差不多的問題.
就是他們都是想了一些東西.
他們認定了一些東西.
他們覺得一定是這樣.
一定是對的.
然後他們就去做.
如果我們用這個角度去說.
我們可能會稍微明白一點.
為什麼耶穌在呼召門徒跟從他的時候.
第一個給門徒的命令是否定自己.
中文的翻譯就是寫記.
英文的翻譯就是deny yourself.
就是否定自己.
什麼意思呢.
否定自己不是說我們跟了神之後.
我們就沒有了自己.
我們只可以有上帝.
不是的.
我們是永遠都有自己的.
耶穌說的否定自己是說.
我們在日常生活的大大小小的選擇裡面.
我們去學習去否定自己.
去質疑自己.

$^{401}$去問自己不同的問題.
以致我們生命裡面有些空間.
是容讓上帝去介入和帶領的.
不過大兄姐妹.
我想今天我們的問題.
不是我們不知道耶穌的吩咐.
我們都知道.
也不是我們不懂得做.
我們都懂得做.
我們都會做.
我想我們今天最大的問題.
是我們太過有自信.
在跟從耶穌這件事上.
我們經常都會覺得自己跟得不錯.
跟得多好.
跟得多貼.
跟得多足.
甚至我們的自信會大到一個位.
我們覺得就算沒有否定自己這一部分.
我們都可以繼續跟耶穌.
我們都可以說後面的背十字架.
付代價.
我們可以說後面的.
怎樣為主寫命.
怎樣為主放下金錢.
因為我們覺得.
當我們信了耶穌一段日子的時候.
我們覺得跟從耶穌是一件很自然.
很自然的事情.
很多時候我們越信得久.
我們就會越自然.
我們做很多的決定.
都不再會經過這個否定自己的過程.
我們很快就會決定了.
跟著就做了.
然後可能有時候我們會有些疑惑.
於是我們決定完之後.
我們就會反過來問.
究竟上帝的心意是什麼呢.
我們甚至可能自信到一個位.

$^{441}$是覺得我們無論怎樣選擇都好.
我們做什麼都好.
我們怎樣決定我們的前路都好.
我們都可以跟從耶穌.
就好像什麼呢.
就好像舊約裡面的以色列人.
以色列人的自信.
是覺得就算我弄一隻金牛像出來.
我都可以跟從上帝.
明明之前摩西說過不可以做像.
他們弄了個像出來.
依然覺得可以跟從上帝.
因為我們只要指著這隻像.
說祂是上帝.
我們一起拜祂.
我們就跟著祂.
又或者好像門徒那樣.
只要耶穌不死的話.
無論祂說什麼都好.
我們一起擁戴祂.
其實祂最後都可以帶領我們.
推翻羅馬政府.
所以我覺得今天我們的問題.
是需要問一下自己.
就是究竟我們心裡面.
真的有多少的空間.
多少的位置.
是真的可以讓上帝介入.
讓上帝帶領的呢.
我們又有多少的決定.
是我們等待上帝的帶領.
而不是我們決定好了.
之後才反過來.
去問上帝的心意呢.
又或者我們心裡面.
剩下多少的想法.
是容讓上帝去挑戰.
容讓祂去改變我們的呢.
所以耶穌就說完這個社稷之後.
祂就說第二樣東西了.

$^{481}$祂接著下去說就是.
背起祂的十字架.
來跟從祂.
耶穌用背十字架這個圖像.
去形容否定自己的狀態.
就是當我們願意去實踐.
這個否定自己.
我們去質疑了.
我們去凡事大小.
我們都去問的時候.
我們會發現自己進入了一個.
懷疑人生的狀態.
因為當所有的東西都不確定的時候.
當我們沒有一樣東西.
是可以很肯定地說.
祂是上帝的心意.
我們跟著走就行了.
我們當沒有一樣東西是這樣的時候.
其實這個狀態是很辛苦的.
這個狀態是很累的.
所以耶穌用了背十字架這個圖像.
去形容這個狀態.
雖然是辛苦.
但是耶穌說你一定要背著祂.
你才能跟從.
於是耶穌又說第三句了.
祂說凡要救自己生命的.
必喪掉生命.
凡為我和福音的緣故喪掉生命的.
必要得著生命.
耶穌是說.
當我們願意學習去否定自己的時候.
我們慢慢就會發現.
有另一個真實存在.
就是我們的想法.
我們的計劃.
不一定對的.
不一定只有一條路做到.
原來上帝的方法.
上帝的工作也可以是可以的.

$^{521}$也可以是好的.
我就在流唐服侍了兩年.
最近就有一個經歷.
我就為了一個小組帶了一個攝影機.
這個攝影機就叫作.
我是作為傳道童工.
第一次準備就帶.
所以我就在腦裡做了很多.
模擬的.
就在想這個攝影機會是怎樣.
因為我也很想這個攝影機是有些用的.
或者是做到一些東西.
於是我就從我的經驗.
我就去想究竟怎樣帶好.
究竟我的組員的反應會是怎樣.
活動會怎樣進行.
有多少人會願意分享.
就想了很大堆東西.
於是去到攝影機的時候.
我就發現一切都不是我想的那樣.
無論他們進攝影機的狀態.
大家上班後累到想死.
一回來就想睡覺.
然後大家圍在一起.
想著分享的時候就拿些薯片出來吃.
明明我是想著應該都很嚴肅.
都是說一些很深入的東西.
大家就開著一罐薯片.
就在那裡吃.
好像吃花生那樣.
我那一刻看著他們的時候.
活動就開始了.
就在那裡聊天.
我就在那裡想一件事.
糟了不是我的計劃.
能不能做到呢.
還是我是否應該要去強迫大家.
去到我想的那樣東西.
要他們收起薯片.
要他們精神一點.

$^{561}$會不會是這樣呢.
但是同一時間我在想.
是不是真的照我這樣想.
攝影機進行完.
那件事就能成功呢.
是不是真的按照我的計劃.
100\%做完所有我想的東西.
大家分享的模式是這樣.
是不是就代表大家的關係會變好.
或者會代表大家的分享會變得更深入.
於是我就在第一晚安靜了一會.
他們就在那裡聊天.
我就在糾結這個問題.
究竟怎樣好呢.
於是我最後就決定.
放手.
因為我在想.
我沒有辦法肯定.
我沒有辦法在上帝面前寫包單.
就是你只要跟著我的計劃做.
那件事一定會成功.
所以你幫我吧.
你讓他們收起薯片吧.
不是這樣的.
所以我就開始嘗試.
用一個第三者的角度去觀察一下.
再聽一下他們說話.
然後再嘗試去帶動一些活動.
我就慢慢在這個營裡面.
過了之後.
我再回望.
我就發現一件事.
原來上帝是真的很神奇的.
那個營裡面.
所有的東西都不是你的計劃裡面.
但是它是有效的.
而且是有效得比你想像的更加好.
無論是分享的人數.
無論是那個深度.
或者大家的互動.

$^{601}$都是比我想像中好很多.
我就在這一段時間.
這個星期.
就跟朋友吃飯.
就說起踢球.
我老婆是不看球的.
我們就說起阿根廷對沙地阿拉伯.
我老婆就很快有反應.
她就說.
阿根廷應該贏了.
因為一個不看球的人都知道.
阿根廷是傳統強隊.
沙地阿拉伯.
沒有聽過.
所以應該都沒有懸念.
但是大家知道.
最後就是.
這場就是世界盃的其中一場.
最冷的賽事.
就是沙地阿拉伯贏了.
或者我這樣.
我這個星期都有留意足球.
雖然準備講章.
但是我自己都有點喜歡看球的.
就想看德國對日本那一場.
但是我就因為在做其他東西.
就錯過了.
於是我就半場的時候.
就上網查一查比數.
就看到寫著.
德國正在領先.
一球這樣.
我就想.
應該都沒有懸念.
德國是贏了.
但是當然大家知道最後.
就是日本贏了.
我想講什麼呢.
我想講的是.
在足球這件事上.

$^{641}$你都會看到.
就算憑你自己的經驗.
你的認知.
其實有很多東西都不是絕對的.
你覺得他會贏.
他未必會贏的.
更何況我們人生裡面.
其實有很多決定.
有很多我們的想法.
或者我們的執著.
其實可能都是我們自己想的.
不一定是真的.
所以耶穌再講下去的時候.
他就舉了一個例子.
他就說人若賺得全世界.
他賠償自己的生命.
有什麼益處呢.
人還可以拿什麼來換他自己的生命呢.
相信在座每一位.
或者在網路上的你.
或者甚至是不信耶穌的人.
聽到耶穌講這句話.
我想他都會明白的.
他都會明白的.
因為很明顯的.
就是我們每個人都知道.
其實金錢物質是換不到生命的.
是不是.
但是如果讓我們選擇.
可能我們都會選錯.
我們可能都會選擇了金錢.
選擇了物質.
而我們不自知.
所以耶穌才一開始要講.
你需要否定你自己.
因為我們的自己.
其實沒有我們想像中那麼可信.
所以最後耶穌最後那句話.
就是說.
凡為我和我的道.

$^{681}$當作可恥.
人子在父的榮耀和聖天使降臨的時候.
都要把那個人當作可恥.
耶穌的意思是.
在跟從他的這條路上.
你只可以相信他.
沒有一個方法.
既賺得全世界.
也都可以得著生命.
沒有一個方法是不需要否定自己.
而當我們願意去否定自己.
甚至是那個充滿自信.
覺得自己跟得很足.
跟得很貼.
跟得很好的那個自己.
當你願意去否定他的時候.
你就會發現.
其實我和你的信心.
都不是我們想像中那麼大.
我們最後那個相信.
不是相信自己可以跟到耶穌.
不是相信自己做到多少事.
不是相信自己可以怎樣.
去否定到自己.
以致我們跟到耶穌.
我們的相信.
是相信當我們每一次願意.
去否定自己的時候.
耶穌的聲音.
他的帶領是會出現.
當我們願意否定自己.
到一個我們很迷惑.
我們很無所適從的狀態的時候.
耶穌他是會出來.
去帶領我們.
這個是我們將我們的信心.
放回到上帝那裡.
所以去到最後呢.
就要說一下.
耶穌呼召門徒.

$^{721}$這段說話裡的第一句.
耶穌是說.
若果有人要跟從我.
耶穌不是在這裡設定一個標準.
或者一個標準.
是要所有的門徒.
都要達到的.
就是你不做到.
你就不是我的門徒.
不是.
耶穌是在說.
他的這個呼召.
其實是一個邀請.
他邀請門徒去經歷他自己經歷的事.
耶穌就是那個會否定自己的人.
他甚至去到喀西馬利園的時候.
他都向上帝禱告.
是願上帝的旨意成就.
然後他最後真的走在十字架這條路上.
耶穌這個經歷告訴我們知道.
原來釘十字架.
都可以是一件好事來的.
都可以成就上帝的計劃.
成就上帝的心意.
在人的角度看下來.
釘十字架是達成不了任何事情的.
他釘十字架是趕走不了羅馬人的.
也都沒有辦法帶來民族復興.
甚至他都激勵不了他的門徒.
他釘完門徒是四散的.
是一個人看下去.
什麼都達成不了的狀態.
但是他依然可以是一件好事.
他依然可以成就上帝的工作和計劃.
所以耶穌邀請每一個跟從他的人.
去經歷這件事.
就是經歷怎樣在我們每一個人的生命裡面.
發現原來上帝的計劃都可以work.
都可以好好.
但是我們想的事情才是好的.

$^{761}$我們一樣可以經歷到上帝的美好.
和他的神奇.
就是上帝怎樣可以將一件很差的事情.
帶到最後他都能用到.
就是一個化腐朽為神奇的狀態.
今天我的講題是未來見.
我期望我們每一個人都是走在這條路上.
即使未來我們不知道.
可能我們都是會散居在世界各地.
或者散居在這個城市的不同地方.
但是我相信只要我們願意學習去否定自己.
我們願意真的這樣去跟從耶穌的時候.
我們會在一個遙遠的未來裡面再一次相見.
然後我們真的可以分享上帝在我們生命裡面的奇妙.
就是他怎樣令到我們看到.
原來上帝的計劃真的可以work到.
真的可以好好.
然後我們真的可以成為他一個美麗的見證.
我們一個祈禱.
因為我們知道不是我們做到什麼.
不是我們很有自信地說我們能夠跟從你.
我們跟得很好.
或者不是我們怎樣可以去否定到自己.
否定到我們能夠跟從你.
而是因為我們知道你永遠都在.
你在我們的身邊.
以致當我們願意這樣去學習.
這樣去質疑否定自己的時候.
我們相信到你一定會帶領我們.
你一定會讓我們看到另一條路.
是一條可能人看起來是沒什麼可能.
不是很work的道路.
但是只要我們願意去信服.
我們願意去擺上的時候.
我們一樣可以經歷到你自己的美好.
甚至比我們想的好上千倍百倍.
上帝求你去帶領我們每一個人.
讓我們生命裡重新經歷你的美好.
讓我們找到原來我們自己的計劃不是真的很好.
你的計劃才是最好的.

$^{801}$求你去幫助我們.
我們這樣禱告.
奉耶穌基督的名求.
阿們.
\newpage



\section{約翰福音 14:1-14-20221203}
\label{sec:4yYwkRP32_4}
\textbf{【網上崇拜】未來的耶穌|約翰福音14\_1-14|20221203 [4yYwkRP32\_4]}
\newline
\newline
連結: \href{https://youtube.com/watch?v=4yYwkRP32_4}{\texttt{ https://youtube.com/watch?v=4yYwkRP32\_4}} ~~~~ 語音日期: 2022-12-03 
\newline
\newline
\hyperref[sec:ipiBLnwp8PI]{\small{< < < PREV SERMON < < <}}
~
\hyperref[sec:index_chronic]{\small{[返順時目]}}
~
\hyperref[sec:index_scriptual]{\small{[返順卷目]}}
~
\hyperref[sec:SVl4_oZWscg]{\small{> > > NEXT SERMON > > >}}
\newline
\newline
約翰福音 14:1-14-20221203
\newline
\begin{longtable}{cl}
\hline
\hline
章節 & 經文 (和合本修訂版)\\
\hline
14:1 & \begin{tabularx}{0.7\textwidth}{X} 「你們心裡不要憂愁;你們信神,也當信我。 \end{tabularx} \\ \\ \relax
14:2 & \begin{tabularx}{0.7\textwidth}{X} 在我父的家裡有許多住處;若是沒有,我就早已告訴你們了。我去原是為你們預備地方去。 \end{tabularx} \\ \\ \relax
14:3 & \begin{tabularx}{0.7\textwidth}{X} 我若去為你們預備了地方,就必再來接你們到我那裡去,我在哪裡,叫你們也在哪裡。 \end{tabularx} \\ \\ \relax
14:4 & \begin{tabularx}{0.7\textwidth}{X} 我往哪裡去,你們知道那條路。」 \end{tabularx} \\ \\ \relax
14:5 & \begin{tabularx}{0.7\textwidth}{X} 多馬對他說:「主啊,我們不知道你去哪裡,怎麼能知道那條路呢?」 \end{tabularx} \\ \\ \relax
14:6 & \begin{tabularx}{0.7\textwidth}{X} 耶穌對他說:「我就是道路、真理、生命;若不藉著我,沒有人能到父那裡去。 \end{tabularx} \\ \\ \relax
14:7 & \begin{tabularx}{0.7\textwidth}{X} 既然你們認識了我,也會認識我的父。從今以後,你們就認識他,並且已經看見他了。」 \end{tabularx} \\ \\ \relax
14:8 & \begin{tabularx}{0.7\textwidth}{X} 腓力對他說:「主啊,將父顯給我們看,我們就知足了。」 \end{tabularx} \\ \\ \relax
14:9 & \begin{tabularx}{0.7\textwidth}{X} 耶穌對他說:「腓力,我與你們在一起這麼久了,你還不認識我嗎?看見我的就是看見了父,你怎麼還說『將父顯給我們看』呢? \end{tabularx} \\ \\ \relax
14:10 & \begin{tabularx}{0.7\textwidth}{X} 我在父裡面,父在我裡面,你不信嗎?我對你們所說的話不是憑著自己說的,而是住在我裡面的父在做他的工作。 \end{tabularx} \\ \\ \relax
14:11 & \begin{tabularx}{0.7\textwidth}{X} 你們要信我,我在父裡面,父在我裡面;即使不信,也要因我所做的工作信我。 \end{tabularx} \\ \\ \relax
14:12 & \begin{tabularx}{0.7\textwidth}{X} 我實實在在地告訴你們,我所做的工作,信我的人也要做,並且要做得比這些更大,因為我到父那裡去。 \end{tabularx} \\ \\ \relax
14:13 & \begin{tabularx}{0.7\textwidth}{X} 你們奉我的名無論求甚麼,我必成全,為了使父因兒子得榮耀。 \end{tabularx} \\ \\ \relax
14:14 & \begin{tabularx}{0.7\textwidth}{X} 你們若奉我的名向我求甚麼,我必成全。」 \end{tabularx} \\ \\ \relax
14:15 & \begin{tabularx}{0.7\textwidth}{X} 「你們若愛我,就會遵守我的命令。 \end{tabularx} \\ \\ \relax
14:16 & \begin{tabularx}{0.7\textwidth}{X} 我要求父,父就賜給你們另外一位保惠師,使他永遠與你們同在。 \end{tabularx} \\ \\ \relax
14:17 & \begin{tabularx}{0.7\textwidth}{X} 他就是真理的靈,是世人不能接受的。因為他們既看不見他,也不認識他;你們卻認識他,因他常與你們同在,也要在你們裡面。 \end{tabularx} \\ \\ \relax
14:18 & \begin{tabularx}{0.7\textwidth}{X} 我不會撇下你們為孤兒,我必到你們這裡來。 \end{tabularx} \\ \\ \relax
14:19 & \begin{tabularx}{0.7\textwidth}{X} 再過不久,世人不再看見我,你們卻會看見我,因為我活著,你們也要活著。 \end{tabularx} \\ \\ \relax
14:20 & \begin{tabularx}{0.7\textwidth}{X} 到那日,你們就會知道我在父裡面,你們在我裡面,我也在你們裡面。 \end{tabularx} \\ \\ \relax
14:21 & \begin{tabularx}{0.7\textwidth}{X} 有了我的命令而又遵守的人,就是愛我的;愛我的人,我父要愛他,我也要愛他,並且要親自向他顯現。」 \end{tabularx} \\ \\ \relax
14:22 & \begin{tabularx}{0.7\textwidth}{X} 猶大(不是加略人猶大)問耶穌:「主啊,為甚麼親自向我們顯現,而不向世人顯現呢?」 \end{tabularx} \\ \\ \relax
14:23 & \begin{tabularx}{0.7\textwidth}{X} 耶穌回答他說:「凡愛我的人就會遵守我的道,我父也會愛他,並且我們要到他那裡去,與他同住。 \end{tabularx} \\ \\ \relax
14:24 & \begin{tabularx}{0.7\textwidth}{X} 不愛我的人就不遵守我的道。你們所聽見的道不是我的,而是差我來之父的。 \end{tabularx} \\ \\ \relax
14:25 & \begin{tabularx}{0.7\textwidth}{X} 「我還與你們在一起的時候,已對你們說了這些事。 \end{tabularx} \\ \\ \relax
14:26 & \begin{tabularx}{0.7\textwidth}{X} 但保惠師,就是父因我的名所要差來的聖靈,他要把一切的事教導你們,並且要使你們想起我對你們所說的一切話。 \end{tabularx} \\ \\ \relax
14:27 & \begin{tabularx}{0.7\textwidth}{X} 我留下平安給你們,我把我的平安賜給你們。我所賜給你們的,不像世人所賜的。你們心裡不要憂愁,也不要膽怯。 \end{tabularx} \\ \\ \relax
14:28 & \begin{tabularx}{0.7\textwidth}{X} 你們聽見我對你們說過,我去了還要回到你們這裡來。你們若愛我,就會因我到父那裡去而喜樂,因為父比我大。 \end{tabularx} \\ \\ \relax
14:29 & \begin{tabularx}{0.7\textwidth}{X} 現在事情還沒有發生,我預先告訴你們,使你們在事情發生的時候會信。 \end{tabularx} \\ \\ \relax
14:30 & \begin{tabularx}{0.7\textwidth}{X} 我不再和你們多說了,因為這世界的統治者將到,他在我身上一無所能。 \end{tabularx} \\ \\ \relax
14:31 & \begin{tabularx}{0.7\textwidth}{X} 我這麼做是照著父命令我的,為了讓世人知道我愛父。起來,我們走吧!」 \end{tabularx} \\ \\
[1ex]
\hline
\hline
\end{longtable}
$^{1}$好 等待節目平安.
今日的講題是未來的耶穌.
正如剛才Neshus也說.
Full Church今個月的未來月題一直在吃字.
未來和未來.
未來更是我這個月很想講的主題.
未來除了是講我們的將來之外.
我們也想思考一下耶穌基督的再來.
特別是今天是張臨期的第二個主日.
所以我們也期待耶穌基督的來臨.
Advent這個字 張臨這個字.
本身這個字的Latin文Vent來解做Come.
雖然我們Full Church不是傳統意義上的禮儀教會.
但我們也在這個月裡 十幾月.
思考耶穌基督的再來.
我們一直相信一個未來的主.
一個將要來臨的主.
我們藉著今天的題目.
思考耶穌基督的未來和我們面對的信仰.
我藉著祈禱.
再求你親自對我們說話.
你自己是我們那位的救主.
我們教會的元首.
我們作為你的群體.
我們恭恭敬敬的來到在你面前.
我們再次去仰望 去懇求你親自對我們說話.
太子不算什麼.
但你自己使用今天的崇拜.
你自己對著我們說話.
在一個腐敗的氣命裡.
我們仍然能夠完整地聽到你自己的話.
求你的說話幫助我們弟子妹.
在海外 在不同地方裡.
我們一起去崇拜弟子妹.
我們一起來懇求你自己的說話.
讓我們能夠再次思想.
你昔日對門徒的說話的時候.
我們生命裡得到方向和力量.
我們知道如何面對我們的未來.
求你這樣幫助我們 奉主命求 阿門.

$^{41}$今天的經文是《讓福音》十四章一號四節.
雖然是這段經文.
但今天我們會說很長的經文.
其實整個《讓福音》十三章到十七章.
都是我們今天要說的段落.
請聽我讀出這段經文.
這段耶穌基督在《讓福音》的說話.
你們心裡不要憂愁.
你們信上帝也當信我.
在我父的家裡有許多住處.
若是沒有 我就早已告訴你們了.
我去完事為你們預備地方去.
我若去為你們預備了地方.
就必再來接你們到我那裡去.
我去那裡 叫你們也在那裡.
我往那裡去 你們知道.
那條路 你們也知道.
所以今天我們不單看這段經文.
我們看整段十三章到十七章的經文.
《讓福音》十三章到十七章.
是教會傳統裡稱之為離別之言.
英文叫做farewell discourse.
一段耶穌基督在被賣的晚上.
將要離別他的門徒.
在那個晚上大概三四個小時左右.
耶穌將要被捉拿.
門徒將要四散.
牧人將要被擊打.
羊群從此四散失敗.
就是在告別和分離之前.
耶穌和他的門徒道別.
這段是整個十三到十七章的內容.
你會發現很有趣.
整個《讓福音》有21章經文.
足足佔了五章經文.
來形容這一餐飯.
有一章經文裡有五章.
基本上是四分之一的段落.
單單是講這一餐飯的內容.
十三章是講耶穌對門徒洗腳.

$^{81}$設立聖餐.
十四章是講猜險聖靈.
十五章是葡萄樹與之子.
十六章是講耶穌勝過世界.
十七章是大祭司的禱告.
然後就是臨別之言的內容.
耶穌將要離別.
耶穌將要不起道.
這是整段段落的重要中心思想.
第一句是這樣說.
十三章第一節.
耶穌說:與之以前.
耶穌知道自己離世歸附的時候到了.
然後三十三節.
耶穌說:小子們,我還不多的時候.
願你們同在,後來你們要找我.
但是我去的地方你們不能到.
十六章二十八節.
我從父出來到了世界.
我又離開世界往父那裡去.
整段的離別之言.
不斷不斷不斷去重複耶穌的離別.
耶穌將要出發.
耶穌不再在這個世界裡.
我們問為什麼用方法作者.
會花了四分之一的段落去說這頓飯.
這段farewell的飯呢.
原因是離別之言.
不單單是在說門徒的問題.
而是在說歷世歷代的教會.
將來過去我們今天的教會.
所面臨的處境.
離別之言裡.
幾乎每句話都是將來式的字眼.
一段未來的字.
耶穌不單單是向他們說未來的事.
更加是向門徒說.
耶穌未來,耶穌未回來的時候.
你們可以怎樣做.
門徒向門徒去描述.

$^{121}$去鼓勵他們將來.
當耶穌不在的時候.
教會可以怎樣做.
這二千年將來的時間.
我們可以怎樣做.
他說整個的經文段落.
告訴我們耶穌將要離開世界.
並且宣告在耶穌升天和主在來之間.
那段時間裡.
未來的世界.
耶穌還沒回來的世界裡.
我們應該可以怎樣做.
今天我們正正處於這樣的處境.
耶穌已經離開了這個世界.
曾經和門徒同在.
但已經不是這樣了.
祂將要回來,但祂還沒來.
這個未來的世界.
一個未來的耶穌的世界.
今天我就用三句話.
跟大家說.
怎樣去面對耶穌還沒回來.
耶穌未來的情況.
第一句話很簡單.
耶穌還沒回來.
你知道吧.
耶穌還沒回來.
耶穌不在,耶穌還沒回來.
這是我們比較少去思考的課題.
我們很少說耶穌不在.
耶穌還沒回來.
我們比較相反.
我們很強調耶穌同在.
耶穌在.
我所說的耶穌同在.
其實是聖靈同在.
耶穌的靈今天在.
耶穌差去的靈當中在我們這裡.
這是耶穌再來的意義.
無論你怎樣強調耶穌同在.

$^{161}$耶穌是還沒回來.
如果不是的話,主再來就沒意思了.
耶穌是還沒完全回來.
所以其實升天節是這個意思.
升天節我們比較少去慶祝這件事情.
升天節是說什麼?.
就是祂已經升天了.
祂不再在地上.
祂會再回來.
但是我們作為教會.
面臨耶穌不是完全回來的年代.
不過坦白說.
耶穌在,耶穌同在是很安慰的.
是很容易的.
耶穌愛你,耶穌超超超超超愛你.
耶穌看顧你.
很多這些話我們經常都會說.
和你身邊的教會人士說.
不是這樣說,我意思是這樣說.
我們經常都叫別人說.
我們很多的師父都是這樣.
伴我一生,與你同行.
親愛的主牽我的手.
沿途有你,這些話.
我們都在崇拜不斷地去唱.
耶穌是同在,耶穌在這裡.
耶穌在這裡成為了我們很多很多信仰問題.
或者很多疑難的解救.
拍拖失戀,不用怕,耶穌在這裡.
耶穌擁抱,不用擔心,耶穌幫助.
冥冥不好,不要忘記,耶穌在這裡.
不用怕,耶穌在大廳.
這些話.
最大的困難是什麼?.
過去那幾年.
我們發覺教會每個禮拜說的支票.
似乎我們找不到兌現的情況.
很記得幾年前,警察這樣說.
叫你叫耶穌來見我.
就說了這件事情.

$^{201}$我們相信耶穌在這裡.
但我們不能夠去知道.
或者不能去證明他在這裡.
耶穌未完全全的在來.
我們教會面臨的處境.
這幾年我們就是這樣的情況.
我們流淚,我們失望,我們絕望.
正正就是面對一個.
耶穌基督為什麼你不在這裡.
為什麼會這樣的情況.
我們祈禱,我們很多次的來去呼求耶穌.
未來的耶穌.
正正是我們教會這幾年.
面臨這樣的情況.
雖然耶穌得勝.
但他並沒有完全全.
令世界的黑暗完全消去.
聖經這樣說.
真光已經照耀,黑暗漸漸過去.
這是一句很矛盾的經文.
這不是我們的經驗.
真光已經照耀.
平時我們晚上回家.
一開黑燈燈就光了.
但是真光已經照耀.
黑暗是很慢很慢的過去.
這正是主在來之前.
主升天之後.
這段居間期我們所面臨的情況.
我們作為教會.
我們面對著很多這些很掙扎的情況.
同時我們有很多生命的懸殊.
我們每個人面對著這樣的情況.
但事實上,我想告訴你.
當離別之言這段經文完了之後.
當耶穌離別之後.
門徒立即完完全全的失敗和跌倒.
18章,《永康律法》18章都開始這樣說.
隨著耶穌基督往他路出發的時候.
門徒就失敗了一敗塗地.

$^{241}$耶穌被捉拿,門徒就恐懼驚慌.
三年的信心,他不成棍.
門徒就四散.
他被捉拿著一把刀去切別人的耳朵.
完全不像一個門徒的模樣.
門徒在第18章的表現.
正正是跟第13到17章的教導.
是一個極大的反差.
實在是很大很大的對比.
門徒的失敗正正是反映我們歷世歷代教會的困惑.
我們教會有很多軟弱.
大家作為Full Church的頂尖媒體都知道.
很多教會裡不理想的情況.
我們的生命都是一樣的.
不要說別人,說回自己.
我們的生命都是一樣的.
耶穌基督不在,我們的生命都是很多很多的脆弱.
這十幾年的牧羊告訴我.
十個頂尖媒體裡有九個都是軟弱的.
剩下一個去讀神學,但都是軟弱的.
真人,我自己在按木禮之後.
說說自己的按木之後的感受.
按木之後我發覺人是精神了.
人又健康,又醒神了,又是背了.
好像突然有些力量.
因為按納為牧師是一個比我更理想的身份.
我是背著一個比我更好的人去做.
所以或多或少都會被鼓勵,做更好的自己.
明明自己都沒有那麼好,但都嘗試去做得更理想.
這是很好的一件事.
所以我牧之後發覺這兩個星期.
我個人好像精神了很多.
不過我在實驗院跟同事聊天.
他們兩個都是按木的牧師.
三個中學生聊天.
「你按木,怎樣?你好像精神了一點」.
我說「是嗎?我沒有,我現在差不多」.
他這樣說,如果我差不多就沒有特別了.
當然我也知道,我很怕,也應該會.
這些可能是短暫的鼓勵.

$^{281}$我不覺得是神奇力量,突然升級.
我仍然是一個普通人.
仍然是一個很普通,軟弱的人.
我不覺得按木之後會突然有額外的力量出現.
有人問我是否開始找我祈禱.
好像祈禱比較靈,開始找我多祈禱.
我覺得不是,應該大家都是一樣.
其實很多弟兄姊妹的生命,就算是牧者也好.
其實都有很多軟弱和問題.
問題只不過是你知道不知道.
我自己發現原來自己是很被傷害.
當我被某些很正派的人拒絕的時候.
這是我這幾年最大的一根刺.
被一些很正路的人排在門外.
所以我們都很想耶穌再回來,或者耶穌在.
如果耶穌在,我們的生命就可以更加有力.
耶穌回來,我們就不用講道,等耶穌來講.
耶穌回來,我們就找祂來跟政府說話.
不過我們知道耶穌還沒有回來.
這是我們第一句要知道的事情.
耶穌還沒有回來.
我們問耶穌還沒有來,究竟耶穌去了哪裡呢?.
究竟最後晚餐的道別告訴我們一個什麼事實呢?.
耶穌這樣說,我去完事為你們預備地方去.
第二節,我若去為你們預備地方,就必再來.
接你們到我那裡去.
我在那裡,叫你們也在那裡.
耶穌說我離開是為了為你們預備地方.
我將來會再回來,去接你們到我那裡去.
我在那裡,叫你們也在那裡.
然後十三章十二節就講了一個很震撼的話.
耶穌說,我實實在在地告訴你們.
我所做的事,信徒的人也要做.
並且做得更加大的事.
因為我往乎那裡去.
耶穌說了他去哪裡.
在臨別之言當中,當問耶穌去哪裡的時候.
耶穌給了一個很特別的答案.
我稱之為一個本體論的答案.
耶穌說我要往乎那裡去.

$^{321}$原文是prostron patera,即是to the father.
耶穌要去的地方,不是一個地方.
耶穌不是去某個地方.
而是要prostron patera,即是to the father.
去乎上帝那裡.
原來臨別之言所說的,耶穌要去的地方.
不是一個地方,不是一個目標.
也不是一個階段,一個理想.
而是一切地方,一切目標,一切理想背後的蹤跡.
就是天父上帝那裡.
其實去天父上帝那裡,這五個字都未必能夠最好的表達.
因為原文來說,都沒有所謂天父那裡.
天父都不是那裡.
天父就不是一個地方.
他是任何地方的更加根本.
就是天父上帝.
實際上沒有一個比天父更加基本的地方.
中間都不是.
所以,原文來說,不是去天父那裡.
而是去天父,to the father.
往天父那裡去.
這是我們第二句要記得的話.
我們正在去天父上帝那裡.
to the father.
這點對我們今天作為教會群體有很大的反省.
往天父那裡去.
成為了我們面對這個世界.
我們自己的人生,我們面對這個世界歷史的發展.
一個最終極的方向.
我初上書的時候,很早就有機會看潘復華的《獄中書簡》.
有一句話我很記得.
20年前到今天都仍然記得.
我便簡單了,不斷地都會曲這段話出來.
我在家書也寫過.
潘復華在坐牢的時候.
他正在面對自己可能要面臨死刑.
納粹德國還未解決問題.
很混亂的情況下.
潘復華說.
無論任何情況.

$^{361}$我們總能找到一條通往天父上帝的道路.
無論什麼情況.
我們總能找到一條通往天父上帝的道路.
很有意思.
潘復華不是在說問題如何解決.
不是在說德國有什麼出路.
不是在說自己身邊有什麼出路.
他給了一個很神學的答案.
我們今天所面對的情況.
我們找到一條去天父那裡的出路.
我們可能找不到一條出路可以幫助香港.
但我們找到一條出路.
總有一條出路能夠去天父那裡.
去天父那裡成為我們今天最終極的出路.
這句話成為了我後來很重要.
半句座右銘的一句話.
無論我人生裡面面對任何事情.
或大或小.
任何情況總有出路.
那條出路正正可以去到天父那裡.
我只要找到一條出路去到天父那裡就足夠了.
真人在我們人生裡.
我們作為基督徒.
在我們人生裡面.
我們面對著很多不同的事情.
任何的事情都離不開往天父上帝那裡去.
這個最終極的終點.
無論是六四事件.
無論是疫情.
不僅是明年要去移民.
Bad Plain 演唱會.
日本贏了西班牙.
任何事情.
都是指向往天父那裡去.
那個終極的終點.
在我們人生的階段裡.
可能你面對著自己快要移民.
我明年要走.
但我知道我最後.
我肯定知道.

$^{401}$我要去最終極的地方.
就是天父上帝那裡.
所以下次你上別人的航班.
你可以這樣去祝他順風.
你說.
我祝你要去天父那裡.
加油.
真人.
無論你去英國.
你去北美.
在香港也好.
我們都是向天父那裡.
就像一個很大型的郵輪.
這個世界就像一個郵輪一樣.
裡面有很多不同的事情發生.
有些人在畫畫.
有些人在聊天.
有些人在跳舞.
在吃東西.
在吃自助餐.
在睡覺.
船裡面有很多不同的活動.
但這艘船只有一個終極的方向.
就是往天父上帝那裡去.
耶穌基督正正帶領著我們這個世界.
往天父上帝那裡去.
或者裡面有很多人.
很多的詭計.
很多的問題.
我解決不了.
我不懂得解決.
但我知道這艘船.
最後還是往天父上帝那裡去.
在我們人生的旅程裡.
正是這樣.
所以我們面對我們的生命.
面對我們的社會的時候.
我們很清楚的第二句話.
我們是去天父上帝那裡.
最後我們看到經文裡.

$^{441}$一個最後更加重要的一點.
經文裡這樣說.
第十四章二到五節.
在乎的家裡有很多住處.
若是沒有我就早已告訴你們.
去完之後你們要給地方去.
然後突然多瑪就走出來.
他就說.
主啊我們不知道你往哪裡去.
怎麼知道那條路呢.
一個很重要的問題.
帶出一個很重要的答案.
多瑪去問耶穌.
主啊我們不知道你往哪裡去.
怎麼找到那條路的時候.
耶穌說了什麼.
耶穌對多瑪說.
我就是道路.
真理 生命.
耶穌基督就是那條道路.
耶穌就是那條通往父上帝那裡去的道路.
我們能夠面對未來的那條道路.
耶穌的道路不單單是我們福音橋的神人關係那條橋.
耶穌的道路不是純粹一個得救的方法.
一個拯救的竅門秘訣.
耶穌基督是道路.
因為祂是我們人生一步一步向前的依據.
耶穌更加是你面對未來的那個依靠.
面對著社會年代變遷的時候.
我們可以去尋求那個腳蹤.
我想說的是今天我想重新去思考這句話.
這句話不是純粹一句屬靈靈修的說話.
而是一個更加闊的社會歷史意義層面的一句話.
因為耶穌基督是道路.
因為耶穌基督是未來我們能夠做的.
我們就是要去跟隨這個道路.
跟隨耶穌.
我大概說出一個圖畫.
這幾年讀神學裡我對主在來這句圖畫.
我不知道你怎麼想主在來這句圖畫怎麼畫.

$^{481}$主在來可能有個雲.
然後有耶穌就這樣回來.
我就不是.
對我來說主在來不是從上而下.
在IFC跳下來.
而是耶穌基督在將來.
在未來那裡.
在時間線那裡.
從未來跑過來我們現在那裡.
這幾年裡我一直想像理解主在來的圖畫.
耶穌基督在未來.
迎向著我們直奔跑過來.
我想過不說這個例子.
但我還是說吧.
因為很多人都不明白.
如果跟我差不多年紀的話就應該明白.
大家有沒有.
我不是看你.
如果你有看龍珠第17期的話.
那時候悟空已經死了.
他在蛇島上正在回來.
不知道大家記不記得.
頭好像跟我差不多年紀大.
悟空對比達之前那場.
你不明白就算了.
你跳過去吧.
耶穌將要在蛇島上正在跑回來那一刻.
我小時候覺得很難忘.
悟空正在回來.
悟空不斷在飛回來救他們.
主耶穌都一樣.
主耶穌在未來那裡.
正在未來那裡直奔回來.
耶穌基督是未來的主.
祂在未來那裡回來.
而我們今天站在現在這個點.
我們要做的是面對著這個未來.
去迎向祂.
去跟隨祂.
去踏進這個走向未來的腳踵.

$^{521}$大概今天我想說的道理就是這幅圖畫.
耶穌正在未來那裡飛回來.
而我們今天面對著很多困難.
很多的困惑.
耶穌還沒回來.
但我們能夠做的.
我們知道天父上帝在最後面.
我們今天就是去跟隨主耶穌.
去迎向我們的未來.
耶穌未來.
耶穌基督是我們的道路.
暗示了一個呼召和跟從的關係.
耶穌基督在未來那裡呼召我們.
叫我們去一步一步地踏前一步.
踏進耶穌基督的未來.
耶穌基督在我們面前.
在我們前面.
在未來那裡呼召我們去跟隨祂.
很有趣的.
《約翰福音》是一個很有趣的地方.
如果你去比較這四本方書的時候.
你會發現約翰福音是唯一一本方書.
是耶穌沒有呼召過彼得的.
其他三本是有呼召彼得的.
在祂很早的時間裡.
約翰福音在什麼時候呼召彼得呢?.
耶穌什麼時候呼召彼得呢?.
就是在第21章的時候.
當祂快要離別彼得的時候.
當耶穌將要離開的時候.
耶穌就呼召彼得.
叫祂跟隨祂.
正正是耶穌將要出發.
正正是要在出發之前.
耶穌就呼召彼得說.
來呀跟隨我.
你問為什麼耶穌要在祂離開世界的時候.
才呼召彼得跟隨祂呢?.
因為.
我這樣說.

$^{561}$因為耶穌將要離去.
並且尚未來臨.
面對著這位未來的救主.
耶穌對我們每個人說.
你要跟從我.
跟從我不是純粹是做好一些.
靈修道德意義上的跟隨耶穌.
你要踏進這個前面未知的未來.
去跟隨這位在我們前面的救主.
我們要和我們的明天去交手.
縱然明天是一隻惡魔也好.
我們都需要和明天去交手.
因為耶穌在我們的明天等著我們.
呼召我們.
頂智妹.
我們這樣去理解我們的未來.
所以第三句話.
就是主呀我來.
三句話.
面對著我們今天的未來.
耶穌還沒有回來.
我們正在前往天父上帝那裡.
還有主呀我來.
耶穌當然和我們同在.
但這個同在不是一個很穩定的觀念.
耶穌的同在和耶穌的離去.
我曾經說過這個比喻.
我再說一次.
其實是正正暗示了我們的跟隨.
我們只能夠不斷向前走.
去踏進一個未知的未來.
去找我們那位救主在未來的腳踏.
我們才能夠跟得上他.
我也說過這個竹印的比喻.
很多不同地方裡.
沙灘中的一雙竹印.
一雙還是兩雙呢.
是一雙自己一個人孤獨地走.
還是耶穌背著你走.
答案是仍然是一雙竹印.

$^{601}$因為我們要跟隨這位在我們前面的耶穌.
他在未來那裡.
我們一步一步地去跟隨他.
去經歷一些他已經知道的未來.
我所說的是我們的社會.
我們正在面對我們的香港.
我們正在面對我們自己的生命.
我們正在面對未知的世界的時候.
我們的方法是什麼呢.
我們的方法仍然是很老套的方法.
就是我們要跟隨我們的耶穌.
這幾年我自己有一點點體會.
三年前 從2019年到今天.
最近我有機會去評論自己之前的事情.
前幾天跟一對宣教士吃飯.
是西人來的.
跟他用英文介紹Fold Church.
就說我在2019年做了什麼.
2020年做了什麼.
2021年做了什麼.
這幾年裡我們面對很多香港社會的事情.
但同時間這些故事.
其實都是我自己跟隨耶穌的故事.
你自己嘗試去想起來.
這幾年裡面你面對社會的事情的時候.
其實都是同時間一個你跟隨耶穌的故事.
你問我們香港怎麼算呢.
我不知道怎麼算.
但我知道的就是.
你只要去跟隨著耶穌基督這樣走的時候.
我們就是這樣去面對今天的香港.
或者我們不再相信明天會更好.
不過就算我們不能確定明天會更好.
有一件事我們可以肯定.
我們可以肯定明天會更加接近我們的主耶穌.
只要你今天好好地去跟隨主耶穌.
明天就比昨天更加接近祂.
你就能夠繼續往天父那裡去.
怎樣來面對未來呢.
真正確定祂還沒回來.

$^{641}$我們不會依賴那些同在.
耶穌抱著你的安慰那麼簡單.
我們知道我們無論怎樣都好.
整個世界都是向著天父上帝那裡.
我們唯有跟隨主耶穌.
來繼續面對我們未知的未來.
這個可能充滿惡魔的未來.
待會我們會唱一首詩歌.
是我特意叫勁拜隊去唱的一首詩歌.
這首詩歌的歌名其實是很傳統的歌.
叫做《I Come to Thee》.
中文叫做《耶穌我來就你》.
一首很古老的詩歌.
這首詩歌正正是一種我們對於上帝耶穌的回應.
主啊 我來 我來緊.
你在未來等著我們.
讓我一步一步跟隨你去踏進未知的未來.
讓我願意的去跟隨你 我來就你.
我們一起祈禱.
主啊 我們知道你是那位未來的救主.
你雖然還沒回來.
但是你在未來那裡一直呼喚我們去跟隨你.
救主你賜給我們力量去面對今天.
讓我們能夠有信心去踏出 跨過 踏進未知的未來.
我們不知道這個疫情的政策何時完結.
我們不知道我們教會能夠在香港這個地方裡.
可以仍然有多久.
我們不知道我們自己的生命裡.
明年離開香港之後有什麼打算.
我們不知道當我們今天在一個新地方裡的時候.
可以做什麼.
我們不知道我們留在香港.
你會怎樣使用我們.
大主啊 我來就你.
我們願意來奔向你.
勇敢的去跟隨你的腳踏進一個未知的未來.
求你的聖靈在當中引導我們.
你曾經在裡面跟我們說.
你賜禱告給我們.
讓我們能夠去呼求你.

$^{681}$你賜聖靈給我們.
讓我們能夠感應到你.
你賜彼此相愛的群體給我們.
讓我們可以不是孤單地面對未來.
求主你幫我們留堂的頂尖妹.
無論在香港 在不同的地方裡.
我們一起勇敢地面向未來.
因為未來有你的時候.
我們知道我們有盼望.
\newpage



\section{約翰壹書 3:1-12-20221210}
\label{sec:SVl4_oZWscg}
\textbf{【網上崇拜】未來的愛心|約翰壹書3\_1-12|20221210 [SVl4\_oZWscg]}
\newline
\newline
連結: \href{https://youtube.com/watch?v=SVl4_oZWscg}{\texttt{ https://youtube.com/watch?v=SVl4\_oZWscg}} ~~~~ 語音日期: 2022-12-10 
\newline
\newline
\hyperref[sec:4yYwkRP32_4]{\small{< < < PREV SERMON < < <}}
~
\hyperref[sec:index_chronic]{\small{[返順時目]}}
~
\hyperref[sec:index_scriptual]{\small{[返順卷目]}}
~
\hyperref[sec:C1660tBb0Zk]{\small{> > > NEXT SERMON > > >}}
\newline
\newline
約翰壹書 3:1-12-20221210
\newline
\begin{longtable}{cl}
\hline
\hline
章節 & 經文 (和合本修訂版)\\
\hline
3:1 & \begin{tabularx}{0.7\textwidth}{X} 你們看父賜給我們的是何等的慈愛,讓我們得以稱為神的兒女;我們也真是他的兒女。世人不認識我們的理由,是因他們未曾認識父。 \end{tabularx} \\ \\ \relax
3:2 & \begin{tabularx}{0.7\textwidth}{X} 親愛的,我們現在是神的兒女,將來如何還未顯明。我們所知道的是:基督顯現的時候,我們會像他,因為我們將見到他的本相。 \end{tabularx} \\ \\ \relax
3:3 & \begin{tabularx}{0.7\textwidth}{X} 凡對他有這指望的,就潔淨自己,像他是潔淨的一樣。 \end{tabularx} \\ \\ \relax
3:4 & \begin{tabularx}{0.7\textwidth}{X} 凡犯罪的,就是做違背律法的事;違背律法就是罪。 \end{tabularx} \\ \\ \relax
3:5 & \begin{tabularx}{0.7\textwidth}{X} 你們知道,基督曾顯現是要除掉罪;在他並沒有罪。 \end{tabularx} \\ \\ \relax
3:6 & \begin{tabularx}{0.7\textwidth}{X} 凡住在他裡面的,不犯罪;凡犯罪的,未曾看見他,也未曾認識他。 \end{tabularx} \\ \\ \relax
3:7 & \begin{tabularx}{0.7\textwidth}{X} 孩子們哪,不要讓人迷惑了你們;行義的才是義人,正如基督是義的。 \end{tabularx} \\ \\ \relax
3:8 & \begin{tabularx}{0.7\textwidth}{X} 犯罪的是出於魔鬼,因為魔鬼從起初就犯罪。神的兒子顯現出來,是為了要毀滅魔鬼的作為。 \end{tabularx} \\ \\ \relax
3:9 & \begin{tabularx}{0.7\textwidth}{X} 凡從神生的,不犯罪,因神的道存在他裡面,他也不能犯罪,因為他是由神所生的。 \end{tabularx} \\ \\ \relax
3:10 & \begin{tabularx}{0.7\textwidth}{X} 這就顯明誰是神的兒女,誰是魔鬼的兒女了。凡不行義的,不是出於神,不愛他弟兄的,也是如此。 \end{tabularx} \\ \\ \relax
3:11 & \begin{tabularx}{0.7\textwidth}{X} 我們要彼此相愛。這就是你們從起初所聽到的信息。 \end{tabularx} \\ \\ \relax
3:12 & \begin{tabularx}{0.7\textwidth}{X} 不要像該隱;他是屬那邪惡者,殺了自己的弟弟。為甚麼殺了他呢?因為自己的行為是邪惡的,而弟弟的行為是正直的。 \end{tabularx} \\ \\ \relax
3:13 & \begin{tabularx}{0.7\textwidth}{X} 弟兄們,世人若恨你們,不要驚訝。 \end{tabularx} \\ \\ \relax
3:14 & \begin{tabularx}{0.7\textwidth}{X} 我們知道,我們已經出死入生了,因為我們愛弟兄。沒有愛心的,仍住在死中。 \end{tabularx} \\ \\ \relax
3:15 & \begin{tabularx}{0.7\textwidth}{X} 凡恨自己弟兄的,就是殺人的;你們知道,凡殺人的,沒有永生住在他裡面。 \end{tabularx} \\ \\ \relax
3:16 & \begin{tabularx}{0.7\textwidth}{X} 基督為我們捨命,我們從此就知道何為愛;我們也當為弟兄捨命。 \end{tabularx} \\ \\ \relax
3:17 & \begin{tabularx}{0.7\textwidth}{X} 凡有世上財物的,看見弟兄缺乏,卻關閉了惻隱的心,神的愛怎能住在他裡面呢? \end{tabularx} \\ \\ \relax
3:18 & \begin{tabularx}{0.7\textwidth}{X} 孩子們哪,我們相愛,不要只在言語或舌頭上,總要以行為和真誠表現出來。 \end{tabularx} \\ \\ \relax
3:19 & \begin{tabularx}{0.7\textwidth}{X} 從這一點,我們會知道,我們是出於真理的,並且我們在神面前可以安心, \end{tabularx} \\ \\ \relax
3:20 & \begin{tabularx}{0.7\textwidth}{X} 即使我們的心責備自己,神比我們的心大,他知道一切。 \end{tabularx} \\ \\ \relax
3:21 & \begin{tabularx}{0.7\textwidth}{X} 親愛的,我們的心若不責備我們,在神面前就可以坦然無懼了。 \end{tabularx} \\ \\ \relax
3:22 & \begin{tabularx}{0.7\textwidth}{X} 我們一切所求的,就從他得著,因為我們遵守他的命令,行他所喜悅的事。 \end{tabularx} \\ \\ \relax
3:23 & \begin{tabularx}{0.7\textwidth}{X} 神的命令就是:我們要信他兒子耶穌基督的名,並且照他所賜給我們的命令彼此相愛。 \end{tabularx} \\ \\ \relax
3:24 & \begin{tabularx}{0.7\textwidth}{X} 遵守神命令的,住在神裡面,而神也住在他裡面。從這一點,我們知道神住在我們裡面,這是由於他所賜給我們的聖靈。 \end{tabularx} \\ \\
[1ex]
\hline
\hline
\end{longtable}
$^{1}$弟兄姊妹平安.
先紀念到我最近這個星期.
身邊很多人確診.
回不來 見不到.
另外一些不是他自己確診.
是家人確診.
都要被隔離.
紀念到仍然受疫情困擾的弟兄姊妹.
不過近這兩個星期.
有些人都未必見到.
很多人去飲酒.
不知道你身邊的朋友多不多.
可能這些日子.
日期都會見面.
尤其是少見面.
不過說到末世訊息.
今天都是說末日的日子.
首先可以肯定.
大家都不是被提.
雖然我們仍然在.
今天說的內容.
關於未來我們的愛心.
這個時刻說愛心.
其實都很困難.
因為大家對愛的理解.
投入或表達的空間都很不同.
但我都覺得對於這個世代.
我們基督信仰的愛.
我們仍然需要很認真.
和具體地和弟兄姊妹.
去了解一下.
在預備這個訊息的時候.
想藉著第二次說.
關於未來這個訊息的時候.
和其他同工說的訊息有些穿插.
上星期John就說到.
未來的耶穌正在跑回來.
過程當中讓我們去預備.
和去感受一下.
當然耶穌會再回來.

$^{41}$但耶穌第一次在人群當中的時候.
當時的門徒都問過耶穌.
那時候大概是什麼狀態.
所以經文大概都是在.
馬太封第24章的時候.
其實那班門徒問過.
耶穌都說過.
請告訴什麼時候這件事.
降臨和世界的末了.
有什麼預兆.
耶穌就回答了.
這個大家都很熟悉的經文.
我就不再重讀.
但點題你會看到.
經文裡面提到.
有很多人冒耶穌的名字.
或者扭曲了基督信仰迷惑.
還有很多人.
而在過程當中你會發覺.
國家對國家打仗.
和有很多饑荒.
世界是不穩定.
而在很多時候.
你打仗的時候的環境.
當然是很惡劣.
這些都是災難的起頭.
但重點我想說.
災難起頭之後的過程當中.
那時候很困難.
人要將我們落在患難裡.
有些殺害.
重點你會發覺.
彼此陷害.
仇恨增加.
才會迷惑人.
第十二節帶出.
我覺得很重要的訊息.
因為不法的事越來越多.
人的愛心會越來越冷淡.
我自己很感受到.

$^{81}$不只是今年的事.
這些日子.
其實你會發覺.
人的愛心的表現也不容易.
你會發覺.
好心好像做了壞事.
又或者別人挪用你的可愛心.
來拿錢.
差不多十年前.
內地有一個.
很流行的題目.
叫做.
中國式的參附.
意思是什麼呢.
因為很多人在內地.
假裝撞到.
又或者假裝跌倒.
身邊有人扶他的時候.
你為什麼撞我.
所以中國式的參附是什麼呢.
如果真的見到.
有人跌倒的話.
首先第一件事就是.
我見到有個人跌倒.
我跌倒了.
現在問題是.
跌倒的時候.
要下去.
我現在準備要扶這個人.
要扶這個人.
這個叫做.
中國式的參附.
就是其實我不是不想.
幫他 只不過我幫他的時候.
他反而被人.
挪用了愛心.
來冤枉我錢.
正正是因為有很多社會議題.
就是怕被人冤枉錢.
所以有些人失救而死.

$^{121}$正正是因為.
我不是不想幫他.
所以我自己上了身.
就要做了.
很多額外的工作.
所以因為不法的事.
增多了很多人的愛心.
漸漸冷淡.
但對於我們.
基督信仰常常說.
愛倫如己.
愛倫社要有一個.
行為上的參與過程.
當中是不容易的.
不是一刀切或者一筆劃線.
告訴我們不可以這樣.
不要太多計算.
但事實上.
要真的執行的時候.
我們真的要有一些事情要想一想.
才能夠保障.
保護對方.
所以.
不法的事增多了很多人的愛心.
漸漸冷淡的時候.
在過程當中.
要怎樣呢 有一個等待.
很多時候我們著眼點.
不是十二節.
而是十四節的位置.
很多教會都著重福音派的教會.
都希望天國的福音.
能夠全面萬民作見證.
將重點放在.
第十四節那裡.
就是傳福音 令到更多人信主.
從來都不是對與錯.
不過就是在過程當中.
我們信仰.
怎樣去.

$^{161}$每樣東西都可以.
慢慢去調整一下.
或者不要偏頗 著重一兩個.
重點.
這個是耶穌在世的時候.
那群門生問耶穌.
你來的過程.
或者終極你來的時候.
會遇到什麼情況.
耶穌就正面跟他們說了.
這些可能性.
但今天我們看的是約翰一書.
約翰一書是什麼時候成書呢.
大概是主要一世紀.
差不多耶穌降生.
就.
開始第一世紀.
第一世紀的話.
假設可能是耶穌升天之後.
五六十年之後.
約翰一書就成書.
耶穌升天之後.
五六十年之後.
約翰見過耶穌.
自己的真身.
在他晚年的時候.
他怎樣看耶穌的教導.
或者怎樣執行耶穌.
給我們的命令.
在這個禮拜John的訊息當中.
都在說一條新的命令就是彼此相愛.
那個訊息是在說.
我們怎樣去.
等待耶穌.
而我們的行徑當中怎樣有一個承接.
所以在約翰一書.
第二章.
第二章的結尾是這樣說的.
小子們.
你們要往主裡去.

$^{201}$你們要住在主裡面.
這樣他若顯現.
就可以坦然無懼.
當他來的時候.
在他面前也不致慚愧.
你們若知道.
他是公義的就知道.
凡行公義之人.
都是他所生的.
年老的約翰就在說一個.
很重要的訊息就是.
我們住在主裡面.
當他再顯現的時候.
再回來的時候.
我們可以無愧站在他面前.
這句話.
不知道大家過去教會.
或者是你讀經的時候.
聯想一些教導.
有什麼看法.
可能你會記得一件事.
我們要加.
剛才好像在說傳福音.
我們要將大使命快點.
做好.
讓更多人可以信主.
你帶了多少人信主.
你有沒有跟人決了志.
很緊張的就是自己.
我們可能唱歌的時候.
有首歌.
你們年輕的時候.
也不關事.
看你回到教會.
唱的曲風.
有一首歌叫.
喜可空手回天家.
我不懂播.
其實你們不懂播.
只是不想發聲.

$^{241}$有一首歌叫喜可空手回天家.
無論是生命聖詩也好.
或者普天頌讚也好.
或者頌主新歌也好.
這些你們也可能聽過.
也有這首歌.
這首歌裡面說.
要加快作主公.
為上帝贏取更多的靈魂.
總之要save a life.
就是這樣.
當你聽這首歌.
或者聽了一些教導的時候.
你心中就暗轉.
我其實帶了多少個缺志.
那個也關我事.
是不是.
好像不是.
我跟他祈禱.
我跟他做過三福.
我跟他說過方橋.
我怎麼算呢.
我想想就好像不是我.
不斷的籌算.
其實我見上帝的時候.
我拿什麼給上帝呢.
今天我可能也問問你.
去上帝面前如果說個全封音的話.
你見上帝見耶穌的時候.
要有些東西.
給祂的.
那你可能很痰痽.
然後想想其實我也不是做得很好.
我是不是真的可以坦然無懼.
甚至不至於羞愧.
去上帝面前.
所以有些人其實說.
耶穌快點回來.
耶穌不要這麼快回來.
耶穌這麼快回來.

$^{281}$我趕不到出口.
但你自己生活很困難的時候.
你是想耶穌.
耶穌快點回來.
如果是耶穌的話我聽你祈禱.
你順境的時候就想我快點回來.
你不順境的時候就想我不要這麼快回來.
或者你有個.
有額度的時候.
我就快點回來.
其實我們的信仰是不是這樣呢.
如果這樣說耶穌回來的話.
就好像問你們我什麼時候回來好.
一定不是這樣的.
是不是.
這樣是不是很矛盾呢.
或者其實我們要帶著什麼心態.
或者預備去等耶穌回來.
晚年的約翰.
和那班受輸的弟兄姊妹說.
我們怎樣去見上帝呢.
我們要拿著什麼.
去見上帝呢.
是不是拿著你多少個缺志的quota.
帶了多少人.
是什麼呢.
今天就看我們第三章的經文.
第三章的經文.
約翰一書第三章的經文.
我們一起讀.
我們讀第一至第十二節.
我們請.
我們一起禱告.
上帝我們打開你的說話的時候.
你仍舊對我們說話.
願你的說話肩撞我們.
亦提醒我們的身份.
以致我們不要被世代的人所扭曲.
我們再一次去認定.
我們是你的兒女.

$^{321}$你對我們的提醒.
對我們的肯定.
以致我們有見主面的時候的憑據.
求主你的大力我們.
奉耶穌的名求.
你會看到.
剛才約翰他用了.
十二節的經文提醒那班受輸的弟兄姊妹.
因為當時的環境.
教會發展了三四十年的時候.
就有很多人有不同的聲音.
去將真道.
將基督的名.
將一些教道.
去挪用了或扭曲了.
約翰是一個.
很認真去處理.
上帝是甚麼本身的使徒.
因為他自己見證過.
所以你會看到.
由約翰福音到約翰一二三書.
都在說這個上帝.
是哪一個上帝.
從來沒有人見過上帝.
只有父懷裡的獨生子將祂表明出來.
我們也見過他的榮光.
正是父獨生子的榮光.
所以約翰是很認真說.
我是真的見過的.
所以在約翰一書的時候都在說.
這個是我們親手摸過.
親眼見過.
而相處過的.
所以當有人扭曲基督信仰.
扭曲基督的本意的時候.
約翰在他最後的時候.
他就嚴明正申.
去告訴別人.
我們其實是甚麼身份.
受書.

$^{361}$受書的書信裡面.
都有條理讓我們去理解.
其實寫信的人.
不是亂七八糟.
當然有他的思路.
在剛才的經文裡面你會看到.
其實說了很多上帝兒女的.
表達.
以及關於罪的影響.
或者是對於身份上的那種.
我們要認清的地方.
所以輕輕和大家去排一排的時候.
你會看到.
在剛才讀的.
二章最後那兩節的經文.
去到第十節的經文.
即是三章十節經文的時候.
你會看到剛才讀得.
好像很重複的字.
其實是一個很條理的表達空間.
讓你去理解.
其實我們的身份.
有些甚麼做.
有些不應該做.
有些是有些不是的過程當中.
其實你會來來去去.
就是在說公義和罪.
第二章二十九節下.
行公義的人就是從上帝生的.
犯罪的人其實他違背了律法.
凡是住在上帝裡面的就不犯罪.
凡繼續犯罪的.
就是未曾看見他.
第七節.
凡行義的就是義人.
犯罪的就是屬魔鬼.
所以第九節去到最後.
從上帝生的就不繼續犯罪.
凡不行義的就不是神的兒女.
其實講得很清楚.

$^{401}$就是在過程當中.
去看看你的行為上.
我們基督信仰常常提醒.
在經文上你都會很多時候聽到.
耶穌很強調就是你的信仰和行為當中.
那種互動和參與.
信仰從來都不是用口的.
因為耶穌都是罵那班法利賽人.
你們在做唇舌.
就好像以前那樣.
你口是親近我.
你的心就遠離我.
因為心是主導你的行動上.
你的表達手法.
一看去提醒受書的信徒.
去明白一件事就是.
其實你的行徑.
是讓你再一次去明白到.
你是一個什麼樣的人.
這個都是呼應當日.
科姆書記載耶穌.
或者是要讓那班門生知道.
好樹就結好果實.
壞樹就結壞果子.
你的果子就看到你是什麼樹.
這個都是我們傳統教導的.
我們做了什麼.
或者我們沒有做什麼.
其實你自己知道的.
我經常在不同的信息當中.
或者在報告的時候都說.
又到年尾了.
你有沒有空間去檢視一下.
你今年你自己的人生呢.
你過得怎樣呢.
或者你想有轉變.
但是轉變了沒有呢.
轉變了多少.
有什麼計劃.
你都要想一想.

$^{441}$你不是百無聊賴.
每天都要睜大眼睛.
等著閉上眼睛.
你不是這樣的.
過程當中你都要想一想.
你自己的行徑.
其中我有一句說話.
都說過好幾次.
就是我說今天做人不容易.
而最辛苦的地方.
就是做一個有要求的人.
還有認真的人.
都是今天很辛苦.
如果你做一個有要求的人.
又認真的基督徒.
你更加辛苦.
因為你會發覺有很多東西.
跟過去的信仰教導.
跟現在去落實用的時候.
你會發覺那個逼力更加大.
就好像我剛才輕輕說的.
愛心的問題.
或者我們能不能夠推己及人.
或者在過程當中.
怎樣去評估那個狀況.
一點都不容易.
由19年到現在.
19年很明顯.
Folk Church的信息.
很多時候都讓頂尖的去了解.
教會的良心.
或者教會的表達空間.
教會跟社會中間的互動.
都是在說信仰的逼力.
就是你們的心跟行徑.
是不是相符.
所以回到約翰.
都想再重申一件事.
就是你行公義還是想犯罪.
如果你犯罪過程當中.

$^{481}$你就不是屬於上帝.
但犯罪這件事.
在約翰一書剛才讀的經文.
可能聯想到過去大家.
我真的.
我真的仍然會犯錯.
我仍然不是聖潔的人.
我見上帝的時候.
上帝會不會覺得.
我不是上帝的兒女.
會不會因為我仍然做得不好的時候.
上帝見我的時候.
或者我見上帝的時候.
上帝說我從來都不認識你.
你會很害怕.
這個都是弟兄姊妹.
曾經問過我的問題.
所以在編輯這個小小分段的時候.
有些字跟大家.
小小重溝一下.
就是我在那個用詞上面.
你或者是那個了解的時候.
我就選一個給大家去看一下.
等我一會.
回到一張.
對了.
對不起.
好像第九.
你會發覺第九節.
凡從上帝心的就不犯罪.
因上帝的道存在他心內.
他也不能犯罪.
因他是由上帝生的.
這個說話可能對一些弟兄姊妹來說.
是有些棘手的.
我還在犯錯.
或者一些事情我覺得都不OK的.
我是不是不是由上帝生的呢.
或者我是不是上帝不起悅的地方呢.
對於這個第九節的時候.

$^{521}$你會發覺我排的時候.
你會看到第九節上的時候.
凡從上帝生的就不繼續犯罪.
為什麼我不能犯罪.
就不繼續犯罪.
那裡會有一點點出入呢.
這個字.
我想跟大家再了解一下.
因為翻譯的問題.
在過程當中.
其實他有一個時態.
英文叫做Tense.
就是不能犯罪和不繼續犯罪的意思是什麼呢.
他不是說信了上帝的人就不能犯罪.
意思是什麼呢.
信了上帝就一定不會犯罪.
犯了罪就一定不是上帝生的人.
不是這個意思.
也不應該是這個意思.
因為你看看.
就算像保羅寫羅馬書的時候.
他也說過.
心靈是願意.
但肉體是軟弱的.
而在過程當中.
我們事實上都知道.
在過程中.
我們每一次犯錯的時候.
其實你的心是知道.
只是你還是很不情願之下.
或者是狀態真的做了.
所以要了解這個過程當中.
讓你明白到.
你是沒有以前犯錯的模式.
但不代表你不犯錯.
在過程當中.
你只會慢慢越來越少犯錯.
而過程當中是慢慢離開你以前的犯錯.
又轉個方向來說.
你以往做錯事.

$^{561}$或者做一些不好的事的時候.
你是不會經腦直接去做的.
但你信了上帝之後.
你的心會提醒你.
想一想好像不是那麼好.
你會停一停想一想.
那一刻.
你的精靈就提醒你.
那件事不是上帝的喜悅.
那件事不是你信徒應該要做的事.
你以往不會.
但你相信自己會.
所以你那一刻停下來.
你抽身不做那一次.
其實已經是不會持續犯罪的行為.
所以約翰提醒一件事.
不要用那些律法式去再被捆綁.
也不要用那些說話去克制人.
那件事不是令你不成為上帝兒女.
你要了解一件事.
你自己都要抗衡.
不要繼續犯罪.
你自己都會抗衡.
而在過程當中慢慢疏離.
慢慢越來越疏遠.
越犯就越少.
我不知道過去.
可能對於罪.
我不認識大家.
都不可能全部問.
但你會發覺在信主久了之後.
以往有些事.
可能是壞習慣.
或者壞事.
信主久了.
慢慢會抽離.
那件事已經再不會是你自己的事.
你其實正在經歷一件事.
就是你不會持續犯罪.
你不會不斷犯罪.

$^{601}$約翰想提醒.
你知道慢慢離開了.
過去犯錯犯罪的習性的時候.
其實已經認證了.
你是從上帝而生.
因為你信服聖靈提醒你.
抗衡的習慣.
所以聖靈又稱為訓衛師.
會教訓我們.
會安慰我們.
他會對我們的心說話.
會提醒我們.
所以我們不要掩蓋著.
我們的心的提醒.
就是羅馬書所說的良知會提醒我們.
所以頂智媒對我們來說.
不只是單單看那件事有沒有做.
我們也在看那件事做的頻率.
不是做一次就推倒你過去的事.
而你發覺你犯錯.
或者你做不安的事.
你的頻率會越少.
間距會越縮.
這個就是不持續犯罪的意思.
約翰很希望那班信徒去明白一件事.
就是上帝是什麼呢?.
上帝的本意是愛我們.
上帝是想我們.
藉著耶穌基督再一次取回兒女的身份.
所以你拿著這個原則的時候.
你看回第三章第一至第三節的時候.
你就會明白到.
有三件事是約翰很希望.
受輸的頂智媒去看得通的.
那三件事是什麼呢?.
第一件事就是.
賦給我們的是什麼呢?.
那種愛是讓我們再一次能夠成為上帝的兒女.
這個是約翰福音第一章講得很清楚.
「凡接待他的就是信他明的人.

$^{641}$他就賜他們權柄作上帝的兒女」.
所以我們因著信耶穌為我們的罪待贖死了的時候.
我們的身份就會改變.
我們改變的身份是因為上帝的愛給了我們.
我們才能夠成為上帝的兒女.
我不知道大家能夠感受到上帝的兒女有多重要.
但是這件事正正就是我們能夠見耶穌的時候.
我們可以坦然無懼.
最大的憑據.
耶穌不是認你做多少事.
耶穌是認你有沒有認祂.
這個是最重要的.
第二件事就是約翰很希望去明白一件事.
主約顯現我們必要像他.
像他是什麼?.
耶穌是什麼?.
耶穌就是上帝的兒子.
我們像耶穌就是我們都被納入成為上帝的兒子.
所以我們能夠進天國.
能夠再一次有面去見上帝.
或者我們不怕我們有沒有做了什麼.
有沒有交帳所有的東西.
其實重點就是我們是拿著上帝的兒女身份進入天國.
你拿著登機證件上機.
因為登機證件是你的名字.
這個很簡單.
但是你真正成為一個地方的人.
是那個市民.
是那個真正認同你是一個什麼人.
是接受了你是什麼人.
今年我們認定的是什麼?.
耶穌已經給了我們通行證.
但是真正去到耶穌面前要見耶穌的時候.
那個重點是什麼?.
那個重點就是我們是上帝的兒女.
這個是沒有人可以褫奪.
沒有人可以改變.
第三件事約翰很希望我們明白到.
我們是會見到耶穌的真身.
當上個星期John說.

$^{681}$我們要跑到去和未來跑回來的耶穌相遇的時候.
你期望見到什麼呢?.
我們會見到耶穌的真身.
你見過耶穌真身沒有?.
先問我 先問清楚.
我就沒有的.
我沒有.
我參與教會這麼久.
有時候聽過一些弟兄姊妹和我說.
我聽到上帝和我說話.
我真的很認真.
上帝是怎樣和你說話的?.
我真的很虛心地問.
我問上帝怎樣和你說話?.
上帝的聲音是怎樣的?.
又或者上帝用什麼方式和你說話?.
我不是玩的.
我真的想問.
有一次我去代課.
不用想到很複雜的代課.
我去代課是因為有個主教學老師.
我們不是這間教會.
我以前是在木會.
有間主教學老師他生病了.
我就去代課.
我是傳道人有時候會頂更.
我就問今天教什麼?.
今天教五餅二魚.
我心想糟糕了.
這些老套的神蹟怎樣教?.
我問你那一班?.
他說那一班.
他說了名字那一班.
我心想要先避一課.
因為那一班是唐主任的兒子.
那一班是主教學校長的女兒.
其他就是執事子女.
資深導師.
因為那一班是.
教五餅二魚.

$^{721}$我心想都要先避一避課.
避一避課就因為不是我.
我不是怕他父母.
一定不是.
重點是我覺得.
我怕意思是什麼?.
我怕我進去的時候.
見到我的時候.
我今天教五餅魚.
他說可以了.
不如讓我們玩吧.
你代課而已.
我試過是這樣.
真的跟他爭論.
一個小時主教學.
我給你說35分鐘.
你給15分鐘我打機.
我去家裡.
我希望你不要見過.
或者你不要遇到.
我真的遇過.
爭論.
但我真的不會.
我教教學法.
我當然不會讓他覺得.
我教東西很悶.
我去到的時候.
真的每人派一張紙.
我穿了這個教法.
不過我的學生以前聽過.
我每人派一張紙給他.
A4紙.
我告訴他.
我今天教一個神蹟.
然後他們說.
又教神蹟?.
因為要循環課程.
無論如何當然教.
我說今天教五餅二魚.
然後他說.

$^{761}$他們全部.
有六個人看著我.
他說這個課程我們很熟.
然後我說.
你看到紙裡面有四格.
你就用你自己覺得.
五餅二魚是什麼.
你用四格漫畫畫給我看.
然後他說不畫可以嗎?.
不畫你就用這個方式去表達.
無論如何有些很乖.
有些真的要畫.
在過程當中.
其中一個很有趣.
第一格就是.
很值得.
第一格就是很多人.
然後很多人.
第二格就是有個小朋友走出來.
拿著東西走出來.
第三格先不說.
第四格就有很多人.
然後有十二個男子.
你猜第三格是什麼?.
第三格是一個很重點.
第三格就是.
他畫了耶穌.
然後就畫了一個盤.
然後.
然後就寫了BOOM.
B O M BOOM.
我就問他.
我說這是什麼?.
你是不懂英文.
還是你覺得我畫得不好.
然後我說這是什麼.
你想表達什麼.
他說耶穌就是.
變了五餅二魚.
餵飽了這麼多人.

$^{801}$我問耶穌在你心目中是什麼?.
耶穌在我心目中.
Jesus is a magician.
你知道他大一點英文.
我說耶穌是一個魔術師.
可以變了這麼多東西.
吃給這麼多人.
我說耶穌是一個魔術師.
可以這樣.
魔術師是耶穌嗎?.
當然不行.
耶穌是有魔術的能力.
但有魔術的能力也不可以是耶穌.
那耶穌是什麼?.
我在過程中跟他們對話.
因為我想讓他們.
很牢固的概念.
或者覺得聽這些神蹟.
是很dramatic的東西.
我希望讓他們明白.
聖經我們常常說.
我們看五餅二魚的故事.
那個是故事.
如果那個是一個已故的事.
OK.
但如果那個是故事.
這個不OK.
你明白我的意思嗎?.
因為故事如果當圖畫故事的話.
是你有不同的narrative.
可以去說那件事.
但我告訴你那個是神蹟.
那個是miracle.
不是故事.
我就這樣跟他們說.
我想讓他們明白到.
你的概念是什麼?.
你對耶穌的想法是什麼?.
就是我們怎樣跟祂對話.
讓祂了解.

$^{841}$最後其實有六個人.
有一個人一直都不出聲.
在畫畫畫.
然後下課.
然後他就說.
隨便他吧.
我說不是啊.
你畫得很多啊.
我問他你在畫什麼?.
他說我在畫耶穌.
然後我說你在畫耶穌?.
耶穌是怎樣的樣子?.
一會兒你就知道了.
(笑聲).
然後我就很期望他畫的耶穌是什麼.
其實耶穌的樣子就是.
長長的頭髮.
褸子是軟的.
踢著sandals.
十六世紀的意大利人.
小朋友.
就是在說.
年輕人.
他有一個概念.
我想問對你來說.
耶穌的真相是什麼?.
如果真的未來的耶穌.
跑回來找你的時候.
你懂得認耶穌嗎?.
你認得出祂是耶穌嗎?.
你憑什麼那個是耶穌?.
剛才約翰在說.
有很多假基督在外面.
有很多敵先知在當中.
或者敵基督在當中.
你憑什麼去認耶穌?.
你用什麼去認耶穌?.
你憑什麼去認耶穌?.
你用什麼去.
分辨那個是耶穌?.

$^{881}$約翰告訴你.
我們會見到祂的真相.
在未來我們會見到祂的真相.
我們會見到祂的真相.
你用什麼去認真相?.
我就用.
我是信耶穌基督.
為我的罪死.
釘起十字架.
第三天復活.
祂就是承擔我重罪的那個.
那個是誰?.
他是拿撒冷耶穌.
他是上帝的兒子.
所以這三句說話.
讓我們再一次去明白到.
我們最.
要自恃的.
不是我們帶了多少人信主.
又沒有面子去見耶穌.
不是自恃我們有多少好行為.
幫了多少人去.
有面子去見耶穌.
我們要恃著帶著那樣東西.
就是耶穌.
祂是愛我 為我捨己.
我恩著祂的緣故.
我才能夠再一次.
能夠claim 回我是上帝的兒女.
我.
好像祂一樣.
就好像祂是.
守身的.
是不是? 耶穌是守身的.
我是跟著祂那個.
我也是上帝的兒子.
你是上帝的女子.
所以我們會見到.
真主是上帝給我們的.
confirmation 肯定來的.

$^{921}$所以.
老年的約翰.
很希望很多.
眾說紛紜的時候.
我一定要出聲.
而我出聲我希望肯定.
我是肯定受輸的弟弟姐妹.
這件事才是最重要.
所以去到.
這裡的位置的時候.
第十一節.
所以.
我們應當彼此相愛.
這就是你們從.
起初聽見的命令.
起初聽見的命令就是.
上個禮拜John在說.
約翰福音.
第十三至十六章的經文.
特別十四十五章裡面.
說得很清楚.
我賜給你們的生命令乃是你們彼此相愛.
所以你們要當彼此相愛.
從起初.
所聽見的命令.
不可像該人他是那屬惡者.
殺了他的兄弟.
為什麼殺了他呢?.
因為自己行為是惡.
兄弟的行為是善的.
在這裡停一停.
先說一說.
其實該人和.
阿伯是兩兄弟.
他們都是一個家庭的人.
為什麼約翰會在這裡說該人和阿伯呢?.
就正是.
教會已經運作了三四十年.
其實是一家人.
在教內互相指責.

$^{961}$在教內互相有不同的聲音.
去判斷或是.
平頭稟囑.
或是扭曲了一些不應該的東西.
其實就好像兩兄弟自相殘殺.
這個不是耶穌想看到的事情.
所以約翰希望讓我們明白到.
我們如何彼此相愛.
不要在教內做這些事情.
這個說話對於.
我想今天說的未來的愛心是什麼呢?.
我們很多時候都覺得外面如何做.
我們盡力做.
有時候盡力做都做不到什麼大改變.
但我們就按我們來做.
但其實外面那個相對易搞的.
反而教內是難搞的.
在這幾年你會看到事實上就是這樣.
有很多事情就是不做就不錯.
越做就越錯.
有時候很複雜.
但我覺得每個堂會都有自己的難處.
每個堂會都有自己的限制.
不要緊的.
但有些很大是大非.
是教會根本性的東西.
就不能相對.
就好像.
有什麼人可以回教會?.
什麼人都可以回教會.
就是這樣.
這件事就一定要.
用最大的氣力.
去做好這件事.
那個不是相對.
那個是絕對.
如果不是就會令到很多事情.
被排擠.
就失去了教會在本身.
地上的意義.

$^{1001}$但很多時候都會有不同的聲音.
有不同的限制.
就是.
所以你會回到.
好像之前都在說的play safe的問題.
對我們來說.
就有很多不同的.
界線.
或者有很多不同的看法.
所以我們怎樣去做到這件事呢?.
或者我們怎樣去了解這件事呢?.
我希望在.
這一章裡面.
第三章裡面.
引用了兩節的經文和大家說.
都是同一章.
第三章的經文.
約翰是這樣說的.
親愛的兄弟.
我們的心若不責備我們.
就可以向上帝坦言無懼了.
並且我們一切所求的.
就從他得著.
因為我們謹受他的命令.
行他所喜悅的事.
約翰在說的很清楚.
就是我們的心.
如果不責怪我們.
其實上帝是可以的.
因為剛才說過.
一路以來上帝的心.
正靈會對著我們的良知說話.
上帝提醒我們要抗衡.
正靈提醒我們要信服.
提醒我們要回想.
這也是耶穌在說.
真理的靈會令我想起我們的話.
所以重點就是.
心怎麼提醒你.
你順著去.

$^{1041}$你的心不罵你.
不抵擋你.
上帝OK.
真的弟兄姊妹.
我們守住教會內部.
也是做好教會的外部.
這個是約翰很希望.
當受輸的弟兄姊妹.
其實對我們來說也是.
我們守住他的誡命.
做上帝喜悅的事.
一會兒回應的歌.
我就沒有選詩歌.
我就選了一首時代曲.
原因就是因為那首時代曲.
對我來說也是.
我自己喜歡的歌.
我也拿了這首歌.
給我在18年的時候.
和一班畢業生.
下山的時候的糞便.
那班的糞便.
這首歌就是.
Gin Lee的抱夢而活.
不是一個很Hot Pick的歌.
那時候應該不是很大推.
但這首歌我喜歡的地方就是.
裡面有兩段歌詞.
是我想和大家分享的.
就是當你有夢方可堅強.
清楚所追跟所想.
多麼瘋狂但仍然是你.
靠歲月來策略.
不再作夢的很清楚.
其實沒資格埋沒你.
伸出雙手再創.
遠路總是漫長.
始終會尋獲應走的方向.
由19年到現在.
其實你可以Keep Minimal.

$^{1081}$俗稱躺平.
躺平也不是不好的.
你可以調養新息.
但你長期躺平的話.
我就會問你What's next?.
真的.
我不是那類人.
但我又不是不懂評估.
不過重點是.
你會發覺你想做事的時候.
有很多人和你唱反調.
或者做很多反題.
說不要這樣不行.
很快就會告訴你.
不要想不要想.
不要想不讓想.
但我問其實我們活在.
無論你在哪個地方.
無論在香港還是外地.
其實你應該還有些東西.
上帝給我們生命氣息.
我們仍然可以做.
每個人都不同.
我不是你 你不是我.
你不用做我那些.
我做不到你的那些.
但每個人都有自己那些.
但你找回你自己那些.
這是很重要的.
所以有很多人會告訴你.
不行的 不行的 不行的.
但其實他什麼都不行.
所以沒有作夢的人很清楚.
其實他沒資格說你.
不過他喜歡說你.
他喜歡說你.
你做到 他做不到.
即是你做到 他做不到.
他就沒有面子.
但你做的時候.

$^{1121}$他告訴你做不到.
即是兩個都做不到.
OK 打和.
是不是.
但我正正就覺得.
這件事是不OK.
真的不OK.
所以我常常都說.
做就一 不做就零.
但在零和一當中.
哪怕是錯總算是過.
很重要.
另外一段歌詞就是.
微笑罷曾經多麼的失意淒喪.
不一定順境.
但仍可以靠著熱誠去寄望.
誰有夢抱悶而活不會枯乾.
找個有緣人分享所有愛共理想.
我只想給你打氣.
無論事情讓身心多傷.
請向未來看.
對我來說.
我很多時候都會見人.
很多時候都會分享我近來要做的事.
或者我接下來會做的事.
關於我自己的事.
我通常想接下來的三天左右.
或者最多三個星期.
但關於教會的事.
可能會想三個月.
但關於長遠發展我會想三年.
我仍然相信有很多生命氣息的人.
可以一起投入.
仍然有很多人在未來當中.
可以加入整個團隊.
加入整個發展的空間.
Flowchart只是一個可以大家沉夢.
剛剛上期時壇訪問的時候.
我都說.
香港教會很難請人.

$^{1161}$但Flowchart都保持有目者來.
但每一個進來Flowchart的目者.
我都會跟他說.
目的和教師一體兩面的職份.
是本身的核心.
你的呼召.
但你來到不只是帶組.
我說你有什麼想做.
你可以跟我分享.
如果你分享到那件事.
我就告訴你.
我會盡力去做到.
這個就是可以.
在教會可以想像想做的事情.
當你看收錢池說.
人如果沒有夢想和癌症有分別的時候.
為什麼教會不可以這樣做呢.
教會本身以往有很多創新.
但可能現在不是.
最後希望聽回應歌的時候.
大家可以搭訕歌詞.
最後用這張相片.
想跟大家分享最後一句說話.
這張相片是我拍的.
是用手機拍的.
拍這張相片的時候.
是分秒必爭的時間.
不是因為黃昏所以分秒必爭.
不是這些.
我沒有這麼浪漫.
重點是什麼呢.
大家都記得有一個台風叫山竹.
這個不是山竹.
這個是玉兔.
玉兔和山竹都是超級台風.
威力一樣.
那個是2018年11月的時候.
還是很熱還在打風.
這張相片是我.
有一天我就快要掛風球.

$^{1201}$快點走.
我在工作我不知道.
我又沒有開.
我有一個同工樹木同工敲門說.
潘Sir你還不走快點出去.
你沒有宿舍在.
你要回家了快點走.
我說為什麼要走還沒有時間.
你快點相信我.
如果你現在不走.
有je 喱到碼頭的時候.
你應該全身濕了.
現在還沒有下雨.
你相信我快點走.
他很情詞迫切地告訴我.
叫我快點走.
因為我很相信長洲人.
他看雨是很準的.
他這麼情詞迫切地叫我.
這麼用愛心叫我走.
我雖然還沒有做完.
但我還是走了.
我關門出去.
一下去排坊的時候.
看到天空這麼漂亮.
就自然地拿出手機.
拍了照.
一拍的時候.
其實我.
為什麼要拍照.
因為我覺得很漂亮.
其實這個背後是很黑的天.
無論我們的環境多麼黑.
多麼不妥當.
我們望著遠方.
仍然會有天色上藍的日子.
好像剛才唱第三首歌一樣.
環境是影響我們.
但是上帝在前頭.
我們會在未來見.

$^{1241}$我拍完照之後.
那位同工說你還不走.
其實我十幾年坐船的經歷.
我很清楚自己的步速.
是幾點鐘下去.
而不會離開船上.
但是我真的下去了.
早了.
但我真的很厲害.
我還沒上船就下雨了.
如果聽到的話就知道了.
我沒有濕身.
全身而退.
就上了船.
很大雨.
真的.
找一個欣賞你的人.
找一個跟你同行的人.
走我們去見耶穌的路.
我們拿著我們的夢想.
想做的事情.
找一些有緣人去分享.
大家一起走下去.
那條路雖然不容易.
但是上帝在其中.
我們抱夢而活去見上帝.
求主加力.
\newpage



\section{撒母耳記下 24:1-25-20221217}
\label{sec:C1660tBb0Zk}
\textbf{【網上聖餐崇拜】致未來「多餘」的牧者|撒母耳記下24\_1-25|20221217 [C1660tBb0Zk]}
\newline
\newline
連結: \href{https://youtube.com/watch?v=C1660tBb0Zk}{\texttt{ https://youtube.com/watch?v=C1660tBb0Zk}} ~~~~ 語音日期: 2022-12-17 
\newline
\newline
\hyperref[sec:SVl4_oZWscg]{\small{< < < PREV SERMON < < <}}
~
\hyperref[sec:index_chronic]{\small{[返順時目]}}
~
\hyperref[sec:index_scriptual]{\small{[返順卷目]}}
~
\hyperref[sec:KRXO3Wxxfe0]{\small{> > > NEXT SERMON > > >}}
\newline
\newline
撒母耳記下 24:1-25-20221217
\newline
\begin{longtable}{cl}
\hline
\hline
章節 & 經文 (和合本修訂版)\\
\hline
24:1 & \begin{tabularx}{0.7\textwidth}{X} 耶和華的怒氣又向以色列發作,激起大衛來對付他們,說:「去,數點以色列人和猶大人。」 \end{tabularx} \\ \\ \relax
24:2 & \begin{tabularx}{0.7\textwidth}{X} 大衛對跟隨他的約押元帥說:「你來回走遍以色列眾支派,從但直到別是巴,數點百姓,我好知道百姓的數目。」 \end{tabularx} \\ \\ \relax
24:3 & \begin{tabularx}{0.7\textwidth}{X} 約押對王說:「願耶和華-你的神使百姓的數目增加百倍,使我主我王親眼得見。我主我王何必要做這事呢?」 \end{tabularx} \\ \\ \relax
24:4 & \begin{tabularx}{0.7\textwidth}{X} 但王堅持他對約押和眾軍官的命令。約押和眾軍官就從王面前出去,數點以色列的百姓。 \end{tabularx} \\ \\ \relax
24:5 & \begin{tabularx}{0.7\textwidth}{X} 他們過約旦河,在迦得谷中、城的右邊亞羅珥安營,與雅謝相對。 \end{tabularx} \\ \\ \relax
24:6 & \begin{tabularx}{0.7\textwidth}{X} 他們來到基列,到了他停‧合示地,又來到但‧雅安,繞到西頓。 \end{tabularx} \\ \\ \relax
24:7 & \begin{tabularx}{0.7\textwidth}{X} 他們來到推羅的堡壘,以及希未人和迦南人的各城,又出來,到猶大尼革夫的別是巴。 \end{tabularx} \\ \\ \relax
24:8 & \begin{tabularx}{0.7\textwidth}{X} 他們來回走遍全地,過了九個月又二十天,就回到耶路撒冷。 \end{tabularx} \\ \\ \relax
24:9 & \begin{tabularx}{0.7\textwidth}{X} 約押向王報告百姓的總數:以色列拿刀的勇士有八十萬;猶大有五十萬人。 \end{tabularx} \\ \\ \relax
24:10 & \begin{tabularx}{0.7\textwidth}{X} 大衛數點百姓以後,心中自責。大衛向耶和華說:「我做這事大大有罪了。耶和華啊,現在求你除掉僕人的罪孽,因我所做的非常愚昧。」 \end{tabularx} \\ \\ \relax
24:11 & \begin{tabularx}{0.7\textwidth}{X} 大衛早晨起來,耶和華的話臨到迦得先知,就是大衛的先見,說: \end{tabularx} \\ \\ \relax
24:12 & \begin{tabularx}{0.7\textwidth}{X} 「你去告訴大衛:『耶和華如此說:我向你提出三樣,隨你選擇一樣,我好降給你。』」 \end{tabularx} \\ \\ \relax
24:13 & \begin{tabularx}{0.7\textwidth}{X} 於是迦得來到大衛那裡告訴他,問他:「你要國中有七年的饑荒呢?或是你在敵人面前逃跑,被追趕三個月呢?或是在你國中有三日的瘟疫呢?現在你要考慮思量,我怎樣去回覆那差我來的。」 \end{tabularx} \\ \\ \relax
24:14 & \begin{tabularx}{0.7\textwidth}{X} 大衛對迦得說:「我很為難。我們寧願落在耶和華的手裡,因為他有豐盛的憐憫;我不願落在人的手裡。」 \end{tabularx} \\ \\ \relax
24:15 & \begin{tabularx}{0.7\textwidth}{X} 於是,耶和華降瘟疫給以色列。自早晨到所定的時候,從但直到別是巴,百姓中死了七萬人。 \end{tabularx} \\ \\ \relax
24:16 & \begin{tabularx}{0.7\textwidth}{X} 天使向耶路撒冷伸手要毀滅這城的時候,耶和華改變心意,不降那災難,就對那在百姓中施行毀滅的天使說:「夠了!住手吧!」耶和華的使者正在耶布斯人亞勞拿的禾場那裡。 \end{tabularx} \\ \\ \relax
24:17 & \begin{tabularx}{0.7\textwidth}{X} 大衛看見那在百姓中施行毀滅的天使,就向耶和華說:「看哪,我犯了罪,行了惡,但這群羊做了甚麼呢?願你的手攻擊我和我的父家。」 \end{tabularx} \\ \\ \relax
24:18 & \begin{tabularx}{0.7\textwidth}{X} 當日,迦得來到大衛那裡,對他說:「你上去,在耶布斯人亞勞拿的禾場上為耶和華立一座壇。」 \end{tabularx} \\ \\ \relax
24:19 & \begin{tabularx}{0.7\textwidth}{X} 大衛就照著迦得的話,照著耶和華所吩咐的上去了。 \end{tabularx} \\ \\ \relax
24:20 & \begin{tabularx}{0.7\textwidth}{X} 亞勞拿觀看,看見王和臣僕向他走過來。亞勞拿就出去,臉伏於地,向王下拜。 \end{tabularx} \\ \\ \relax
24:21 & \begin{tabularx}{0.7\textwidth}{X} 亞勞拿說:「我主我王為何來到僕人這裡呢?」大衛說:「我要買你這禾場,為耶和華築一座壇,使瘟疫在百姓中停止。」 \end{tabularx} \\ \\ \relax
24:22 & \begin{tabularx}{0.7\textwidth}{X} 亞勞拿對大衛說:「我主我王,你眼中看為好,就拿去獻祭。看,這裡有牛可以作燔祭,有打糧的器具和套牛的軛可以當作柴。 \end{tabularx} \\ \\ \relax
24:23 & \begin{tabularx}{0.7\textwidth}{X} 王啊,這一切,亞勞拿都獻給王。」亞勞拿又對王說:「願耶和華-你的神悅納你。」 \end{tabularx} \\ \\ \relax
24:24 & \begin{tabularx}{0.7\textwidth}{X} 王對亞勞拿說:「不,我一定要按價錢向你買;我不能用白白得來的東西作燔祭獻給耶和華-我的神。」大衛就用五十舍客勒銀子買了那禾場與牛。 \end{tabularx} \\ \\ \relax
24:25 & \begin{tabularx}{0.7\textwidth}{X} 大衛在那裡為耶和華築了一座壇,獻燔祭和平安祭。耶和華垂聽了為那地的祈求,瘟疫就在以色列中停止了。 \end{tabularx} \\ \\
[1ex]
\hline
\hline
\end{longtable}
$^{1}$各位丁姐妹晚安.
最近經常想起一件事.
就是聖誕節.
木組會二十多年.
由畢業到現在.
說聖誕節其實是很慣常.
但這幾年都挺抗拒說聖誕節.
因為智叔說他不習慣.
其實在這個年代要說訊息要怎麼說.
都成為了很多牧者的難題.
尤其是聖誕節的時候.
是否還是一個很溫馨很愉快.
純粹是一個慶祝的算數.
還是耶穌基督降生在這個時代.
香港面對很多事情要不習慣的時候.
我們應該要怎麼重新詮釋聖誕節.
這些都是我們要去想的課題.
例如Full Church在23月4日和25日有一個絕望研究所.
我想正正在裡面我們重新思考一下.
到底聖誕節是什麼.
聖誕節當我們看燈飾.
他們有沒有燈飾看.
看一些裝飾漂亮一點的東西之外.
其實聖誕節今天對於我們意義是什麼.
這個成為了希望我們在不習慣的裡面的時候.
讓自己生命裡面有些事情不習慣.
就是要不習慣的時候.
例如很多匪夷所思的事情發生.
很多古怪的事情.
新聞經常出現.
在這些日子裡面的時候.
自己經常都有一些習慣不習慣.
才能夠幫自己在這些事情裡面不習慣.
除了那些事情不要習慣之外.
自己平常生活裡面有沒有什麼.
讓自己不要習慣常做的.
我想這個是在幫我們.
在日子裡面要面對很多事情不要習慣的時候.
有少少的幫助和提醒.
今天我們會說十二記下的24章.

$^{41}$基本上上兩個月.
即是從十月我沒有記錯.
十月十一月我都說十二記上.
今次我們會說十二記下的24章.
我們再說經文那段.
再按一張.
原來很小.
希望你們看到.
看到就可以了.
我看到就可以了.
我先看看手機.
原來這麼小.
這一段經文其實是我.
花多氣力研究三個月記上下的.
其實很多事情不應該說.
不過讓我表達一下很辛苦.
我花一年時間看了三個月記上下.
我應該半年左右就看完三個月記下.
其實我經常想問一個問題是什麼.
除了三個月記上下很多故事之外.
很多關於大衛 索羅 薩姆爾的故事之外.
我經常想問這些故事說什麼.
這個是我今天總結.
我下年應該不說三個月記.
我想問的問題.
很奇怪的三個月記下的24章.
還有一章記載了一件事.
叫做數點人數.
大家在網上看節目.
你快點讀.
不要看我的樣子.
現場的觀眾請你看著經文.
你快點看一次.
我先說一下為什麼三個月記下這麼奇怪.
數點人數基本上不應該放在最後.
如果你按照歷代至上的記載.
其實數點人數.
不應該放在大衛想為耶和華建立聖殿的時候.
大約的時間.
所以大衛想的聖殿是在三個月記下很早期.

$^{81}$所以基本上不應該在數點人數.
放在最後一章聖經.
就是三個月記上下的事件.
所以這個問題搞了我很多年.
我想弄清楚一下.
三個月記下24章想說什麼.
為什麼將一件時序不對的事.
再說一次.
刻意要把它放在最後.
用數點人數成為最後的事情.
來總結整個三個月記上下.
當然我已經看了幾千頁.
但我不會有時間說這麼多.
但這是我今天的重點.
希望你離開前.
你其他笑話的事都不記得.
今天不說笑話了.
我希望你記得三個月記下24章為什麼是最後.
你可以思考一下.
今天的重點是這個.
我們再下一章.
當你看完已經不讀了.
三個月記下第四章.
最好四章都不正常.
其實不只是第24章不正常.
三年饑荒 欺騙人 被殺那件事.
如果你回去讀一下.
三個月記下21章.
其實那件事都很古怪.
古怪到不能古怪.
因為沒有時間說太多.
但接下來22章23章.
說了兩件事.
第一是大衛勇士的描述.
無端端說大衛很多勇士.
其實那些勇士的描述.
其實來自三個月記上17章.
你還記得十月份我說大衛對哥利亞的時候.
大衛勇士的描述大體上類似大衛打哥利亞的故事.
然後大衛說一些事.

$^{121}$作一些詩篇.
大衛經常說詩.
他說兩篇.
所以三和四是兩篇詩篇.
第五是大衛的勇士再一次說.
大衛有很多勇士.
再說他的勇士.
最後是數點人數.
我問的問題是.
為何要在數點人數之前說大衛的勇士.
和大衛的詩歌.
當然這是我八卦想問的問題.
但今天只說數點人數.
我們再下一章.
我們今天只說數點人數.
數點人數有什麼問題.
你知道最終死了七萬人.
大衛數點人數其實整個故事很奇怪.
你試一下犯罪得罪神.
結果所有Full Church的人死.
你會很慘的.
大衛得罪上帝.
他數點人數.
但最終死與大衛無關.
意識到地裡有七萬人死了.
你想像一下.
假設潘Sir做這種很壞的事.
假設而已.
不要說了.
我們全部Full Church的人死了.
其實整個故事不知道是什麼.
所以整個故事無端端死了七萬人.
大衛就用壇裝法.
在聖殿.
往後就是聖殿的地方.
是一個聖殿.
就裝個香.
然後天使就不殺了.
所以整個數點人數的故事是很奇怪.
但在這些之前.

$^{161}$我們解釋一個問題.
先說完問題.
不過今天只說一點.
為什麼數點人數.
算不算錯.
我們再看一張.
其實文數記是可以數的.
你也知道文數記叫做Numbers.
所以by default一定是數人數.
一至四張數很多人數.
二十六張再數一次.
所以文數記其實是很奇怪的.
數了兩次人數.
你馬上會問.
為什麼摩西的時候可以數人數.
但數完又要瘟疫.
又死七萬人.
為什麼.
你看看第一節就有答案.
其實是在曠野.
夜禍.
我看見夜禍的字.
麥克風遮住了.
夜禍在西賴曠野.
回望中巧遇摩西船.
所以下一張好像我highlight了.
有些很奇怪的東西.
你按一張下一張.
沒錯.
是綠色了.
所以數人數不是不行.
不過數人數其實是要夜禍說要做才可以做.
原來數人數是要夜禍說去做才可以做.
不是隨便可以做.
我們再下一張看一看.
什麼叫做夜禍說要做才做.
所有人數是什麼觀念.
我們要解釋一點.
花點氣力.
誰創造了那件事.

$^{201}$那個人就能夠數人數.
你試想想若白記或詩篇.
有很多數點星獸.
有很多誰做了什麼.
誰創造了什麼.
誰想了這些東西出來.
其實這些所有能夠.
你有資格數是因為什麼.
是因為你創造了它.
你有資格在它上面.
所以你可以數那些東西.
譬如說今天.
以色列人今時今日.
去到猶太地方.
譬如那些off-dog shoes.
正統的.
正統不知道怎樣翻譯.
那些off-dog shoes戴上黑帽和長袍.
在猶太某地區住的那班人.
如果你在以色列說.
我今天舉行人口普查.
我想說那班人一定不理會你.
他說所有人數是不能數的.
所以政府就算做人口普查也好.
以色列一定沒有人會呼應你.
不單是off-dog shoes.
很多人都不理會你.
就是因為這件事.
而這件事背後是什麼.
他說的是.
原來耶和華救了以色列人.
離開埃及地.
到抗野的時候.
是上帝親自請救這班人.
是上帝親自重新去.
由埃及離開.
要進入迦南地.
要成為上帝子民的時候.
耶和華才能夠.
在那個時候去數點人數.

$^{241}$代表著這班人.
是屬於他的.
所以一至四張數點人數.
其實數完之後是做什麼的呢.
就是圍著Terminal Echo的會幕.
走來走去去獻祭.
所以數點人數的意義是代表著什麼.
原來是你請救了我.
是你創造了我.
所以我可以讓你用.
所以數點人數不是一個很.
隨便的事情.
亂來的事情.
所以我們說回聖誕節.
你記不記得聖誕節是什麼.
其實是一個Sensus.
是人口普查.
所以大衛的子孫.
約瑟先帶瑪利亞.
回到伯利恆.
如果不是的話.
其實他應該在納薩勒或其他地方.
所以都是一個有權柄的人.
才能夠做到什麼.
數點人數這件事情.
數點人數是指.
我在說的事.
我是一個Kingship.
Kingship的Power在我手上.
所以我們這樣說.
是一個叫Counter Creator.
是為了耶和華才能夠.
正確地數點人數.
因為他創造了以色列民眾.
他數人數是對的.
但是為烏龜王八蛋的地方是什麼.
是因為他扮上帝.
他以為他數完人.
他就是那班人說了算的那個.
所以如果這樣說的話.

$^{281}$整個《撒滿爾記》下24章.
要把數人數放在這裡.
是在說什麼.
原來真正要表達的是.
大衛是一個王.
但他不是一個Kingship.
我再說一次.
大衛雖然是一個王.
但他沒有Kingship.
真正的Kingship是萬軍耶和華.
說得地道一點.
我們教會裡有很多領袖.
一個領袖,兩個領袖,三個領袖,四個領袖.
那些領袖是領袖.
但是那些領袖沒有Kingship.
唯一Kingship是萬軍耶和華.
誰的領袖突然之間.
烏龜王八蛋起來說.
我其實有Kingship.
那群體就玩完了.
就好像大衛或《撒滿爾記》下24章所說的一樣.
好,我們再下一個.
我把這個點解釋了.
這個我們不說了.
我們再下一個.
我再說一個剛才佩洛西說的.
大衛和法老.
我們再看下一段經文.
這段經文經常是性難題的問答.
《撒滿爾記》上有一章同樣地說蘇聯人數.
你看看第一節就夠了.
撒旦起來攻擊以色列人.
激動大衛蘇聯人數.
你覺得subject是誰?是撒旦是嗎?.
剛才我叫你讀三個字的下24章.
下一章.
耶和華向以色列人發怒.
就激動大衛.
這個性難題是什麼?.
到底是撒旦激動大衛蘇聯人數.

$^{321}$還是耶和華激動大衛蘇聯人數?.
在希伯文中激動是incite.
是挑去裡面的desire.
到底是魔鬼挑大衛蘇聯人數的desire.
這個慾望代表自己有權柄.
還是耶和華挑動?.
兩段經文有兩個不同的subject出現.
做頂音姊妹最開心.
魔鬼經常引誘我犯罪.
incite我裡面犯罪.
但原來經文教我.
是耶和華incite我.
即刻滿足和快樂.
當然這樣解釋就糟糕了.
你知道一定不是這樣.
經文的問題是什麼.
又是撒旦激動大衛.
什麼又是耶和華激動大衛.
這個我們要問的問題.
下一章我們會清楚多一點.
在摩西五經裡尤其是《初一及記》.
你記不記得法老這個故事.
法老經過十災.
最近我兒子考十災.
他在聖經文大比賽的學校考十災.
我這個父親不長進.
我教他背完十災之後.
我就說其實十災是什麼.
不是什麼嘩災失災.
其實十災是背後埃及的神明.
其實每一個失災.
尼羅河的水變血.
嘩災 黃災 黑暗之災.
其實每一個災背後.
都是對付背後的埃及神明.
不是亂十災來的.
在這些十災之後.
無端端通常要記住一件事.
法老剛起上來.
當然是硬心.

$^{361}$但通常聖經會加一句.
耶和華洗法老的心剛起上來.
這個問題跟剛才的.
《道德實務》二記下的24章.
耶和華引起大會.
還是歷代至上的21章.
撒旦挑動大會的問題一樣.
這些說法不是說誰主動的問題.
到底是耶和華搞還是魔鬼搞的問題.
犯罪很明顯是從創世紀開始.
知道一定引起了撒旦.
所以歷代至上要寫.
其實是撒旦做的.
但這些撒旦做之餘.
如何能夠表達耶和華的王國.
就是要寫耶和華洗大會.
要寫耶和華使法老的心剛起上來.
明明是我剛起上來.
明明是我想數的人數.
是我衰的.
但這個動作本身.
或者你所引起的後果.
耶和華洗你做這件事.
耶和華洗你的心剛起上來.
是說你做這件事做出來的所有東西.
耶和華在這些事中有傾泄.
所以耶和華洗大會激動他能夠數人數.
不是說耶和華整他.
他像機械人按了按鈕.
大會立即數人數.
不是說耶和華真的洗他的意思.
是大會他自己做這件事.
法老自己做這些壞事.
但這些壞事整個事件會發生什麼情況出現.
萬骨耶和華才是真正有傾泄的那個.
不是大會不是法老.
所以撒滿爾記下.
剛才讀了十字聖經.
你見到約翰蘭得了一件好事.
約翰蘭通常是一個頑皮的元帥.

$^{401}$他說老闆你知道數人數.
耶和華不會讚好.
你不要做.
你不明白嗎.
然後聖經說什麼.
他說皇的命令勝於一切.
勝過他們.
所以那裡說什麼.
他們以為皇帝屬於大位.
但經文告訴你.
真正的皇帝不是大位.
即使他是一個皇帝.
但他沒有皇帝.
好像有一個名字叫做皇帝的人.
但他沒有皇帝.
真正的皇帝是誰.
有萬骨耶和華.
如果我們解釋得明一點.
我們再看下一張powerpoint.
問誰才是真正的king.
是誰.
我們再下去.
這裡我們不說了.
我們cut了它.
我們再下去.
這個就是最初的powerpoint.
你看完之後.
我想再解釋大位勇士和這個point.
解釋一句就夠了.
拿觀念.
我們再下一張.
應該是很長的字.
你看到大位勇士描述那裡.
有一句說法.
你看到嗎.
菲利士人與以色列人打仗.
我再看回我這個powerpoint.
我都太小了.
眼睛下次要打大一點.
他說菲利士人與以色列人打仗.

$^{441}$大位就帶領僕人下去.
與菲利士人交戰.
大位就什麼.
被忽了.
接著他說什麼.
韋仁一個兒子.
以實比諾要殺大位.
他的銅槍重三百四十克.
這句說法與哥利亞的說法一樣.
有佩著新刀.
但洗腦亞的兒子亞比西幫助大位.
攻打菲利士人.
將他殺死.
當日跟隨大位的人向大位起誓說什麼.
以後你不可以再和我一起出善.
恐怕會熄滅以色列的燈.
為什麼要記大位的勇士.
這成為了最關鍵的說法.
大位能夠成為king.
是因為他在十七章《撒姆爾記》上.
打贏了哥利亞.
或者他在撒姆爾記的二十一章至三十一章.
他打了很多勝仗.
例如他帶了四百人追回自己的子女回來.
他打了很多勝仗 打贏了很多次.
但終有一日.
這個王是被罰的.
不單是王被罰.
你以為王被罰玩完就完嗎.
不是 《聖經》記載了很多次.
有很多個大位勇士.
原來大位不是只有他一個勇士.
我們看著一個king 一個偉大的人.
覺得他做了很多事很偉大很聰明.
很厲害 像大位一樣.
我們就以為上帝沒有為我們預備.
更多一個像大位一樣的人.
我們就看著那個人.
就是他才行 就是他才行.
他就是天下無敵.

$^{481}$說大位的勇士其實要打敗什麼.
你以為大位做了七年在伯倫.
三十三年在耶路撒冷.
做王就是四十年之久.
所有事永遠都屬於他嗎.
那個現實是沒有的.
這個世界上沒有永遠都存在的東西.
沒有一個英雄有他的英雄的年代.
他貼了一個標題叫做Expiry date.
他的源不單止是打不到仗.
這個王源是因為什麼.
是因為他沒有king ship.
他想拿king ship 但拿不到.
十二記下二十四章 宋元素的時候.
就說一個王不應該拿king ship.
他拿了 他扮上帝.
他扮覺得自己最威.
他扮那些百姓跟著自己四十年.
還不是屬於我的百姓.
他想數一下有多少個.
他想數多少個 數了一百三十萬.
八十萬屬於以色列.
五十萬屬於猶大拿刀的士兵.
幾震撼的一件事.
到今時今日有百多萬人口打仗.
你問全世界什麼國家是這樣.
三十多幾下要完成一件事.
一個有title叫做king的人.
他不可以拿king ship.
一個有title叫做king的人.
他不能夠拿上帝的king ship.
我們再下去 最後一章.
我想表達的是.
《士司記》是在找士司.
你會發現那些士司個個都很奇怪.
基顛 參孫 底波拉.
沒有正常 姐妹是正常的.
但你記住那時候沒有人會相信女士司.
那女士司是跟著打仗的男人.
仔細說 但那些士司很奇怪.

$^{521}$那些人沒有title king這個名號.
所以那個人古靈精怪是沒有問題的.
這只不過是上帝拾起一個士司.
又好色又變態的參孫.
等他頭髮長出來後.
突然把柱子推開 人就死了.
你覺得沒有理由的.
你不會找一個大勁敗隊的人.
又嫖妓又什麼的.
最後一次大勁敗就給我吧.
你不會的 你過不了潘Sir的.
你覺得參孫是這樣.
那你先把頭髮長出來吧.
不是 問題不是那個人的問題.
那個人不行 但那個king是誰.
他能夠捱完 頭髮長出來.
推開那兩條柱子就搞定那些人.
人不是看著那個人 不是看著參孫.
看著那個king是誰.
萬鈞爺說他自己做.
拉懷到三明紀上的時候.
那些人就說.
三母耳 我可以像外國的王一樣.
有個王來管理我們嗎.
三母耳就殺了他.
三明紀那邊說不可以有王.
我們當中 你不想死嗎.
然後那些百姓多囂張.
三母耳 看看你們兩個兒子多囂張.
這個我加上去.
多不可以 你還有什麼人能管我們.
當三母耳記上的時候要說有個王的時候.
整個三母耳記就跟你說.
你可以有個王.
但這個王不可以有kingship.
真正kingship只有上面.
這個訊息是我想做一些退休的訊息.
上星期五 沒記錯.
有一個泰國牧師的追思禮拜在香港出現.
我不知道大家認不認識他.

$^{561}$我不說名稱了.
有一個泰國牧師.
大約兩三個星期前他在泰國.
他們打獵 因為他是山區牧師.
這個山區牧師在整個泰國.
即東北近緬甸的地方.
他建立了六十多間教會.
他在緬甸也建立了三間所謂的兒童之家.
和很多教會.
不過軍政府之後就將所有教會都燒了.
在緬甸這個地方.
這個牧師就在山區成長大.
兩三個星期前他打獵.
他有很多小朋友和青年人跟著他.
他專做青少年士工.
由十多歲做到五十多歲.
今年五十多歲去世.
他的槍有三發子彈.
他開了三發子彈打獵物.
然後他將槍交給了一個九歲的小朋友.
九歲的小朋友.
如果他空了子彈.
因為他和九歲的小朋友是朋友.
就給了他一把沒有子彈的槍.
誰知哪裡來有發子彈.
九歲的小朋友開了.
打中牧師.
牧師經過山區.
直升機飛過來.
等等運回泰國Bangkok.
時間趕不及.
結果在直升機上.
他的兒子陪著他一起去世.
我聽過「追思禮拜」的時候.
他的兒子說話.
他的兒子應該多說一點背景.
他的父親牧師.
即是過世的牧師.
其實是一個牧師的家庭成長.
他的父親是牧師.

$^{601}$不過他父親做牧師的時候.
他的兒子一心想做整個山區的牧師頭.
因為他父親是.
這個牧師頭的.
他的父親是牧師.
他父親做了多年牧師.
結果有宣教士來到.
來到.
不說仔細了.
帶了他父親去耶穌.
做兒子.
現在死了的牧師做兒子.
很奇怪很憎恨他的父親.
明明我可以繼承你父業.
要做一個全山區最偉大的牧師.
為什麼你相信耶穌.
他的經歷跟保羅一樣.
他很憎恨的時候.
耶穌跟他說話.
你為什麼這麼不喜歡我.
他聽完這句話之後.
他就跌在地上相信了耶穌.
這個牧師在那幾十年裡.
帶了很多青年人起來.
到現在這個年紀都是一樣.
他的兒子回溯的時候.
他說在爸爸去世前的一個星期.
這個兒子娶了一個香港人.
所以他講廣東話.
他不是講泰文.
他不是十八字級 學權媽級那些.
他講廣東話.
他的老婆是香港人.
在四十八月他說.
在他死前的一個星期.
神跟他說了好幾次.
他說他死了就有好處.
保羅的一句話.
我活著就是基督.
我死了就有好處.

$^{641}$他不明白這句話是說什麼.
他說在他死前的一晚.
他爸爸捐了很多.
所有的奉獻都給了很多人.
所以他到處做事工.
在山區裡做.
他說很少一家人聚在一起.
他說之前那一晚.
難得一家人聚在一起.
拍了很多照片.
吃了一頓很好的飯.
他兒子經過了一個星期後.
原來神早就跟他說了.
他爸爸要離開.
所以他很感恩.
這個爸爸離開.
神沒有跟他說話.
雖然這是很慘的事.
但他見證了他爸爸.
這四十多年做了這麼多事.
他三十多歲.
三十五歲.
這麼多年他做了這麼多事的時候.
他可以做完他手上的事.
回到天父那裡.
他覺得很開心.
我聽完這些說話.
我裡面有些不安.
我說.
先不要說潘Sir.
先說John.
John有一天他死了.
我說很開心.
我說這裡說什麼呢.
我只是這樣比喻.
他沒有死.
他還在這裡.
你明白我意思嗎.
我突然間在那幾天.
上星期六,日,日思考的時候.

$^{681}$我腦海裡出了幾個字.
我覺得洋洋灑灑.
一個服事主的人是一個什麼人.
在他做完他的事情之後.
可以洋洋灑灑.
離開.
就是最好的.
今天的題目叫多餘的牧者.
牧者經常覺得自己很重要.
Eugene Peterson有一本書叫.
Unnecessary Pastor.
即是不被需要的牧者.
那本書對我牧會是當頭棒學的.
我們以為有很多被需要.
但我看著那個人.
那個牧師去世.
整個安息禮拜.
所有人說的話.
我說是一個人做完他的事情.
做完.
大家很開心地看著他所做的事情.
大家很開心地慶祝他.
曾經做過的事情.
那一刻我突然間有一種羨慕.
我說主啊.
我死的時候能夠.
去到這個地步.
或者不要死.
千萬不要死.
到你Expiry date的標籤貼完.
到那一天你完了.
你Say goodbye 走了.
你覺得上帝用了你50歲.
用了你60歲.
你覺得用完就不需要再祈求上帝.
讓你70歲到80歲.
搞壞自己的名聲.
也搞壞上帝的國度.
面對香港前面很複雜,很難.
我們已經很難覺得可以安身立命.

$^{721}$就因為這個不能夠再容易地安身立命.
更加讓牧者.
或者一個侍奉主的人.
可不可以只Focus在.
上帝讓他做的事情.
他做完之後.
他做完就完.
更重要的是.
這個牧者過世.
這個泰國牧師過世.
下面有多少青少年.
你知道他在泰國.
在安息禮拜搞了四天.
那個九歲的小朋友.
在那裡四天不肯走.
但是在過程中.
有多少人在那裡.
跟師母和不同的人說.
牧師所做的事情.
在感染我這一生.
我願意跟隨牧師所做的事情.
樣樣沙沙地說.
不是說你做完就完.
真正的樣樣沙沙.
除了你做完.
覺得時間到的時候停止之外.
你還發現.
下面有很多人.
會繼續你所做的事情做下去.
今天如果要說教會的改革.
如果說教會.
要不一樣.
每一個人在某一個崗位裡.
有一個延期的時間.
他完結了.
然後他的侍奉.
他的各樣東西.
是被上帝用的.
以致下面有些人有眼望.
他不需要像大衛一樣.

$^{761}$數碟人數搞壞自己的袋子.
他之後完結.
下面的人知道.
原來這個人一生做這件事.
是被上帝做完.
其他人看著他.
一個讚嘆是.
我都希望可以像他這樣.
上帝給他的東西做完就夠了.
面對很難安身立命.
面對很難有些事情.
可以沒有一個限期.
我們發覺很多事情.
突然有一個限期.
有一個限期.
你身邊的人越來越多.
你不想走.
留下來的人.
你突然問自己.
是否要成為一個快要走的人.
我們在這些崗位和情況裡.
很難在一種很穩妥性.
一個很安全感的時候.
我想建議.
或者我想提醒.
問上帝給你的侍奉是甚麼.
上帝給你的侍奉崗位是甚麼.
你是否做完這個崗位之後.
你可以洋洋灑灑地離開.
而這個離開.
你可以感染到很多很多的人.
一起走上你離開的崗位.
我們很習慣的是甚麼.
基督教群體裡.
很多人永遠都存在.
這些就是我們憎恨的事情.
如果全都是教會的弟兄姊妹.
沒有這個思維.
沒有這個想法.
我們和一般的教會其實沒有分別.

$^{801}$永遠都存在.
和洋洋灑灑地離開.
在動盪不安的日子裡.
我們需要操練自己一個很重要的心智.
待會聖餐.
是在說聖誕節的時候耶穌降生.
這個降生不是說祂很舒服浪漫.
以馬來從來不是說很溫馨很開心的事情.
以馬來是在說甚麼.
在白雷恆裡面兩歲以下所有小朋友都死了.
等到耶穌長大.
經過三年多釘十字架.
洋洋灑灑在十字架裡面說成了.
一式間的餅和杯.
讓香港人在很不習慣的裡面.
試試在給自己生命的裡面.
不習慣再多一點東西.
不要再拿著那些好像很安全的安全感.
不要覺得自己永遠都要存在價值與意義.
讓自己的延期日完結.
不只是你完結.
你衷心去侍奉你做的事.
總會吸引到下面的人.
和你成為另一個讓上帝可以使用洋洋灑灑的人.
一式間的餅和杯.
請你紀念的是.
耶穌就是一個這樣的人.
走上十架.
就這樣說成了.
我們一起聽下禱告.
天父多謝你給我們今天有空間和時間.
我一起來到你的面前.
當這個世界很動盪的時候.
甚至很混亂.
很多東西沒有公理.
沒有公義.
龍門任搬的時候.
我們見到很多人受傷.
我們見到很多人在一個悲慘不容易的裡面.
但你卻見證很多落在艱難裡面的人.

$^{841}$他們心靈裡面有何等的剛強.
彷彿這班人在學洋洋灑灑.
可以放下他們以往堅持的東西.
在一個不容易的光景裡面.
仍然可以靠你活下去.
我求親愛的天父.
你幫助我們每一個人.
讓大衛成為我們的境界.
成為我們心裡面要警惕自己的事情.
我更願意是撒滿異忌上下的信息.
給我們很多做領袖的人.
能說話的人.
好像有權柄的人.
去承認King's Shirt不在我們身上.
我們不偷King's Shirt.
我們不裝扮沒有King's Shirt.
其實我們在拿很多.
我們真的願意回到你們面前誠實地.
把King's Shirt還給你們.
讓自己貼上一個標誌.
有個Expiry date.
求主耶穌你自己.
三年多在地上所做的一切.
成為我們的榜樣.
你說成了.
你就升天.
回到天父的右邊.
求你的話語繼續跟我們心裡面說話.
讓我們改變我們人生的態度.
方向.
讓在艱難的歲月裡面.
我們更加懂得善用剛音.
善用時間.
善用僅有的空間.
求天父你幫助我們.
多謝天父你聽我們面前的祈禱.
奉耶穌基督保貴榮智而求.
阿門.
\newpage



\section{路加福音 2:15-20-20221224}
\label{sec:KRXO3Wxxfe0}
\textbf{【網上崇拜】內向者的平安夜|路加福音2\_15-20|20221224 [KRXO3Wxxfe0]}
\newline
\newline
連結: \href{https://youtube.com/watch?v=KRXO3Wxxfe0}{\texttt{ https://youtube.com/watch?v=KRXO3Wxxfe0}} ~~~~ 語音日期: 2022-12-24 
\newline
\newline
\hyperref[sec:C1660tBb0Zk]{\small{< < < PREV SERMON < < <}}
~
\hyperref[sec:index_chronic]{\small{[返順時目]}}
~
\hyperref[sec:index_scriptual]{\small{[返順卷目]}}
~
\hyperref[sec:LRsUlh9Ini0]{\small{> > > NEXT SERMON > > >}}
\newline
\newline
路加福音 2:15-20-20221224
\newline
\begin{longtable}{cl}
\hline
\hline
章節 & 經文 (和合本修訂版)\\
\hline
2:15 & \begin{tabularx}{0.7\textwidth}{X} 眾天使離開他們,升天去了。牧羊人彼此說:「我們往伯利恆去,看看所成的事,就是主所告訴我們的。」 \end{tabularx} \\ \\ \relax
2:16 & \begin{tabularx}{0.7\textwidth}{X} 他們急忙去了,找到馬利亞和約瑟,還有那嬰孩臥在馬槽裡。 \end{tabularx} \\ \\ \relax
2:17 & \begin{tabularx}{0.7\textwidth}{X} 他們看見,就把天使論這孩子的話傳開了。 \end{tabularx} \\ \\ \relax
2:18 & \begin{tabularx}{0.7\textwidth}{X} 聽見的人都詫異牧羊人對他們所說的話。 \end{tabularx} \\ \\ \relax
2:19 & \begin{tabularx}{0.7\textwidth}{X} 馬利亞卻把這一切的事存在心裡,反覆思考。 \end{tabularx} \\ \\ \relax
2:20 & \begin{tabularx}{0.7\textwidth}{X} 牧羊人回去了,因所聽見所看見的一切事,正如天使向他們所說的,就歸榮耀於神,讚美他。 \end{tabularx} \\ \\ \relax
2:21 & \begin{tabularx}{0.7\textwidth}{X} 滿了八天,他們就給孩子行割禮,又給他起名叫耶穌;這是他還沒有在母腹裡成胎以前天使所起的名。 \end{tabularx} \\ \\ \relax
2:22 & \begin{tabularx}{0.7\textwidth}{X} 按摩西律法滿了潔淨的日子,他們就帶著孩子上耶路撒冷去,要把他獻給主。 \end{tabularx} \\ \\ \relax
2:23 & \begin{tabularx}{0.7\textwidth}{X} 正如主的律法上所記:「凡頭生的男子必歸主為聖」; \end{tabularx} \\ \\ \relax
2:24 & \begin{tabularx}{0.7\textwidth}{X} 又要照主的律法上所說,用一對斑鳩,或用兩隻雛鴿獻祭。 \end{tabularx} \\ \\ \relax
2:25 & \begin{tabularx}{0.7\textwidth}{X} 那時,在耶路撒冷有一個人,名叫西面;這人又公義又虔誠,素常盼望以色列的安慰者來到,又有聖靈在他身上。 \end{tabularx} \\ \\ \relax
2:26 & \begin{tabularx}{0.7\textwidth}{X} 他得了聖靈的啟示,知道自己未死以前必看見主所立的基督。 \end{tabularx} \\ \\ \relax
2:27 & \begin{tabularx}{0.7\textwidth}{X} 他受了聖靈的感動,進入聖殿,正遇見耶穌的父母抱著孩子進來,要照律法的規矩而行。 \end{tabularx} \\ \\ \relax
2:28 & \begin{tabularx}{0.7\textwidth}{X} 西面就把他抱過來,稱頌神說: \end{tabularx} \\ \\ \relax
2:29 & \begin{tabularx}{0.7\textwidth}{X} 「主啊,如今可以照你的話,容你的僕人安然去世; \end{tabularx} \\ \\ \relax
2:30 & \begin{tabularx}{0.7\textwidth}{X} 因為我的眼睛已經看見你的救恩, \end{tabularx} \\ \\ \relax
2:31 & \begin{tabularx}{0.7\textwidth}{X} 就是你在萬民面前所預備的: \end{tabularx} \\ \\ \relax
2:32 & \begin{tabularx}{0.7\textwidth}{X} 是啟示外邦人的光,是你民以色列的榮耀。」 \end{tabularx} \\ \\ \relax
2:33 & \begin{tabularx}{0.7\textwidth}{X} 孩子的父母因論耶穌的這些話就驚訝。 \end{tabularx} \\ \\ \relax
2:34 & \begin{tabularx}{0.7\textwidth}{X} 西面給他們祝福,又對孩子的母親馬利亞說:「這孩子被立,是要叫以色列中許多人跌倒,許多人興起;又要成為毀謗的對象, \end{tabularx} \\ \\ \relax
2:35 & \begin{tabularx}{0.7\textwidth}{X} 叫許多人心裡的意念顯露出來;你自己的心也要被劍刺透。」 \end{tabularx} \\ \\ \relax
2:36 & \begin{tabularx}{0.7\textwidth}{X} 又有位女先知,名叫亞拿,是亞設支派法內力的女兒,年紀已經老邁,從童女出嫁,同丈夫住了七年, \end{tabularx} \\ \\ \relax
2:37 & \begin{tabularx}{0.7\textwidth}{X} 就寡居了,現在已經八十四歲。她不離開聖殿,禁食祈求,晝夜事奉神。 \end{tabularx} \\ \\ \relax
2:38 & \begin{tabularx}{0.7\textwidth}{X} 正當那時,她進前來感謝神,對一切盼望耶路撒冷得救贖的人講論這孩子的事。 \end{tabularx} \\ \\ \relax
2:39 & \begin{tabularx}{0.7\textwidth}{X} 約瑟和馬利亞照主的律法辦完了一切的事,就回加利利,到自己的城拿撒勒去了。 \end{tabularx} \\ \\ \relax
2:40 & \begin{tabularx}{0.7\textwidth}{X} 孩子漸漸長大,強健起來,充滿智慧,又有神的恩典在他身上。 \end{tabularx} \\ \\ \relax
2:41 & \begin{tabularx}{0.7\textwidth}{X} 每年逾越節,他父母都上耶路撒冷去。 \end{tabularx} \\ \\ \relax
2:42 & \begin{tabularx}{0.7\textwidth}{X} 當他十二歲的時候,他們按著過節的規矩上去。 \end{tabularx} \\ \\ \relax
2:43 & \begin{tabularx}{0.7\textwidth}{X} 守滿了節期,他們回去,孩童耶穌仍舊在耶路撒冷。他的父母並不知道, \end{tabularx} \\ \\ \relax
2:44 & \begin{tabularx}{0.7\textwidth}{X} 以為他在同行的人中間,走了一天的路程才在親屬和熟悉的人中找他, \end{tabularx} \\ \\ \relax
2:45 & \begin{tabularx}{0.7\textwidth}{X} 既找不著,就回耶路撒冷去找他。 \end{tabularx} \\ \\ \relax
2:46 & \begin{tabularx}{0.7\textwidth}{X} 過了三天,他們發現他在聖殿裡,坐在教師中間,一面聽,一面問。 \end{tabularx} \\ \\ \relax
2:47 & \begin{tabularx}{0.7\textwidth}{X} 凡聽見他的人都對他的聰明和應對感到驚奇。 \end{tabularx} \\ \\ \relax
2:48 & \begin{tabularx}{0.7\textwidth}{X} 他父母看見就很驚奇。他母親對他說:「我兒啊,為甚麼對我們這樣做呢?看哪,你父親和我很焦急,到處找你!」 \end{tabularx} \\ \\ \relax
2:49 & \begin{tabularx}{0.7\textwidth}{X} 耶穌對他們說:「為甚麼找我呢?難道你們不知道我應當在我父的家裡嗎?」 \end{tabularx} \\ \\ \relax
2:50 & \begin{tabularx}{0.7\textwidth}{X} 他所說的這話,他們不明白。 \end{tabularx} \\ \\ \relax
2:51 & \begin{tabularx}{0.7\textwidth}{X} 他就同他們下去,回到拿撒勒,並且順從他們。他母親把這一切的事都存在心裡。 \end{tabularx} \\ \\ \relax
2:52 & \begin{tabularx}{0.7\textwidth}{X} 耶穌的智慧和身量,並神和人喜愛他的心,都一齊增長。 \end{tabularx} \\ \\
[1ex]
\hline
\hline
\end{longtable}
$^{1}$Hello,Hello,Hello,.
這是德國的最新發明,.
逆晶體原離子藍光麻糬智能太陽眼鏡,.
這個型號叫做內向者2000,Introvert 2000 Pro,.
只要戴了眼鏡就不怕在社交上跟別人聊天,.
因為你連人都看不到,.
各位姐妹,平安夜快樂,.
歡迎大家來到這個流堂的平安夜崇拜,.
不如我們自拍一張,好不好?.
大家有沒有戴眼鏡的?.
有的話就戴,好嗎?.
我們先自拍一張,.
預備,好嗎?.
我現在看不到大家,.
不知道大家在哪裡,.
這裡好嗎?.
可以了嗎?.
好,來吧,.
1,2,3,.
這邊先拍一張,.
3,我不夠高,.
來吧,1,2,3,.
低一點,.
好,1,2,3,.
真的看不到,.
頂智慧平安,.
很開心,我們很久沒試過一起,.
實體很多人來參與我們這次崇拜,.
上一次這麼多人是在疫情前,.
是2020年初的時間,.
在希伯倫堂,600多人,.
很久都沒試過這麼多人一起崇拜,.
大家好嗎?.
平安嗎?.
好,先說完這裡,.
三位朋友,你們好嗎?.
順便也挺好反應的,.
今天的平安夜講題叫做.
「內向者的平安夜」,.
很明顯你都知道,.

$^{41}$是我說的,.
我是一個內向的人,.
做基督徒不容易,.
因為教會裡的很多活動.
都是為外向者而設的,.
例如以前我們上教會的時候,.
崇拜差不多結束,.
主席就會叫大家走出裡面.
跟別人握手,.
整件事我作為一個內向者,.
是要被迫成為一個外向者,.
你要離開你的安署區,.
憑信心走出裡面,.
跟別人握手,.
這時候背景的授詞,.
大家就唱,.
我們都是一家人,.
我想一家人裡,.
全家都是這麼外向,.
是一件很恐怖的事情,.
例如平時我們講完就會說,.
跟你哥哥說,.
這時候做一個內向者,.
都被迫成為一個外向的人,.
你要被外向,.
明明不想說話,.
都要跟隔壁的不認識的說話,.
說一些不是你想說的話,.
說一些別人要你說的話,.
縱使有時候是廢話,.
這也不是最難,.
最難就是你要回隔壁的跟你說話,.
你給他反應,.
有時候很多活動都是為外向者而設,.
唯一一個很適合我們內向者而設的活動是什麼呢?.
就是Secret Angel,.
小天使行動,.
大家玩過吧,.
Secret Angel,.
這個活動超棒的,.

$^{81}$完全是為我們內向者而設的活動,.
玩Secret Angel是不需要說話的,.
不需要見人的,.
一個人靜悄悄不說話,.
關心別人,.
我覺得這個活動很好,.
小時候玩完之後,.
我有這些反省,.
我覺得天使之上的天使,.
應該都是內向的,.
所以作為一個內向者,.
在教育裡面是很不容易的處境,.
你要學習寫記,.
你要背十字架,.
你要將你的腦袋釘死,.
你要離開你的安書區,.
你要憑著信心,.
靠上帝恩典,.
根深而變化地去做一個外向者,.
去回教會,.
內向的基督徒不容易做,.
內向的牧師更加不容易做,.
所以當我很多年前夢照的時候,.
當我身邊的弟姐妹知道我要做傳道人,.
她立刻說,.
「啊?」.
我就很尷尬,.
打完場說,.
「是的,上帝的呼召很奇妙的,.
人不能擦透的一些東西,.
我覺得我讀神學的時候,.
最害怕的是什麼呢?.
作為一個內向人,.
最害怕的就是要開始學講道,.
「糟了,講道怎麼算好呢?.
我這麼內向,這麼不喜歡說話,.
我要說三十分鐘的,.
怎麼算好呢?」.
所以我這樣想,.
不過後來這樣想,.

$^{121}$發覺原來不是的,.
OK的,.
完全沒有問題的,.
因為內向人都可以說得很好的那一套,.
為什麼呢?.
因為我們可以寫講章,.
我發現我只要好好地預備在那裡講章的時候,.
我將我要說的話寫好,.
我就能夠按著我預備好的東西去說,.
只要這樣說,.
我就能夠完整地寫好好的講道,.
我覺得講道比聊天更加容易,.
二十倍我覺得是會,.
我可以將我之前所預備的東西說出來,.
我覺得這是一件非常自由踏實的事情,.
我只要預備好所有的東西,.
我就能夠說出來,.
我可以去預備講道裡面每一個的時刻,.
平時你看到我在這裡,.
在講台詞的時候,.
其實我是在這裡預備好想的東西,.
我不是即興想的東西,.
明白嗎?.
我只是在這裡假裝想東西,.
你看過我的講章,.
寫著沉默五秒,.
你會寫這個字,.
就是沉默五秒,.
所以鄧姐妹,.
真的,.
在講道裡面是很需要沉默五秒的,.
所以我開始接受我自己內向的性格,.
甚至我更加喜歡自己內向的性格,.
我不覺得內向是一個性格上的缺陷,.
很多人都覺得有些誤解對於內向,.
他們以為內向人是不喜歡笑的,.
不太開心的樣子,.
內向人也可以很開心地笑,.
也可以很大聲地笑,.
我們不是抑鬱,.

$^{161}$抑鬱是完全兩回事的,.
抑鬱和狂躁和焦慮一樣,.
是一種病,.
病是可以治療的,.
但是內向是不能治療的,.
內向是會跟你一輩子,.
作為Youtuber,.
跟你一生,甚至會伴隨你到永生的裡面,.
所以不要以為上天堂的時候,.
你就會變成一個外向人,.
不會的,.
你仍然在天堂裡面是一個很內向的人,.
有一天,當你在天堂裡面的時候,.
突然間有一個天使飛過來,.
然後就,.
Hi,.
然後在天堂裡面就開始Dead Air,.
Hallelujah,.
沉默五秒,.
本身我很欣賞我自己內向的性格,.
因為內向是好東西,.
如果你要我重新去選擇的話,.
就好像打擊一些人設,.
我仍然會選擇做一個內向的人,.
因為內向不是一種缺陷,.
內向的人仍然有愛,.
內向的人說話很好笑,.
可以說是,.
我們只不過是喜歡事物的深層次的部分,.
內向的人通常比較冷靜,.
比較有自信能力,.
外向的人比較善於思考,.
比較有深度,.
所以我懷疑耶穌也是內向的,.
OK,.
明天就會說耶穌是外向的,.
就是,.
耶穌雖然是走進人群裡面,.
但往往會怎樣呢?.
會離開人群裡面,.

$^{201}$去獨處,.
去得靈,.
所以我相信耶穌是內向的,.
當然我想說的內向,.
是一種往內向,.
是一種靈性的意思,.
那我就說一點,.
一個靈性,.
內向的人是怎樣呢?.
我一直祈禱,.
因為你很愛我們,.
讓我們能夠在裡面,.
去經歷這次的崇拜,.
求主你這樣幫助我們,.
讓我們能夠在今天,.
在平安夜裡面,.
好好的去默想,.
讓你的說話,.
可以成為我們的力量,.
求主你幫助孩子,.
孩子內向,.
孩子不配,.
但你這幾句話,.
可以成為我們的幫助,.
逢轉命求,.
阿們..
好,.
我們先看看經文,.
今天經文是一段聖誕節的經文,.
《路遭灰翁》第二章,.
15到20字經文,.
他說,.
眾天使離開他們,.
升上天去了,.
木人的人彼此說,.
我們往伯利行去,.
看看所成的事,.
就是主所指示我們的,.
他們急忙去了,.
就尋見瑪利亞和約生,.

$^{241}$又有那鷹孩餓在馬槽裡,.
機緣看見,.
就把天使論這孩子的話傳開了,.
凡聽見的,.
就詫異木人對他們的話,.
瑪利亞卻把這件事,.
傳在心裡,.
反覆思想..
我記得這是一段平安夜的經文,.
不過我們以前都比較著重於前半部分,.
比較高潮的部分,.
我們比較少留意,.
就是經文後面,.
一個反高潮的部分,.
我們先花點時間,.
弄清楚那個次序,.
我們回顧一下整個平安夜當晚的流程,.
第一,.
約生和瑪利亞去到伯尼衡,.
找不到雷電,.
他們就決定住在馬槽裡過夜,.
第二,.
當天晚上,.
瑪利亞就開始肚子痛,.
生了,.
然後,.
伯尼亞就在當晚生了他們的鷹孩,.
這邊就是馬槽的情況,.
這邊相同時間,.
我們記得在伯尼衡的野隊裡,.
天使報信,.
伯尼衡的野隊裡,.
有一群牧羊人,.
主的使者,.
他們是比較內向的天使,.
就站在他們旁邊,.
主的榮光就是照著他們,.
天使就說,.
不要懼怕,.
我給你們大力信息,.

$^{281}$是關乎萬民的,.
因為今天在大衛的城裡面,.
你們身要救主,.
就是主基督,.
你們要看見一個鷹孩,.
包著布,.
臥在馬槽裡,.
那就是記號了,.
第四,.
是敬拜的時間,.
聖經說忽然間有一隊天使的敬拜隊出現,.
想想Alex,.
高高,.
天使翼,.
然後就開始大聲地敬拜,.
再次說,.
在至高之處,.
榮耀歸於上帝,.
在地上平安歸於他所戲謔的人,.
然後就是一場天使的敬拜,.
Hallelujah,.
Gloria, Gloria, Gloria,.
榮耀降下,.
最後,.
聖經再提五,.
眾天使離開他們,.
升天去了,.
今天我們說的經文,.
正正就是眾天使離開他們之後發生的事情,.
沉默五秒,.
第六,.
牧羊人聽見天使報信之後,.
就連忙去找約瑟瑪利亞,.
發現這個鷹孩臥在馬槽裡,.
他們見到約瑟瑪利亞,.
就開始將天使有關鷹孩所說的話,.
轉述給他們兩個聽,.
你知不知道,.
聖經所記載的這個平安夜,.
其實是一個熱鬧過後的平安夜,.

$^{321}$一個眾天使離開之後的聖誕節,.
一個高潮過後,.
反高潮的一個段落,.
可能我真的翻譯了很久,.
通常我們敬拜之後,.
當天使敬拜完散了出來,.
是什麼畫面來的?.
有什麼畫面?.
台上有什麼畫面?.
可能你不知道,.
問敬拜都知道了,.
是什麼畫面?.
沒錯,就是收禮物,.
基本上都是清場,.
收拾音響的事情,.
所以當眾天使天君喜悅,.
敬拜完之後,.
整個高潮完了之後,.
其實是一個散場,.
走人,.
收拾東西,.
清場的畫面,.
聽之會真正的平安夜,.
其實是一個這樣的氣氛,.
聽之會真日,.
都知道平安夜,.
英文不是peaceful night,.
是叫什麼?.
是silent night,.
嚴格來說,.
silent night,.
重點不是平安,.
而是silent,.
寂靜,.
不出聲,.
沒話說,.
Tag air,.
如果你不喜歡tag air,.
我們用一個比較好的字,.
我們叫做sacred air,.

$^{361}$一個神聖的空氣,.
就在這個神聖的空氣當中的時候,.
天使離開了,.
每個人都回去的時候,.
散水了,.
good night了的時候,.
不再熱鬧的時候,.
真正的平安夜,.
就正式開始,.
就在這個夜難人靜的平安夜,.
馬槽裡面的瑪利亞,.
一個剛剛用正常的程序,.
誕下她的baby耶穌的母親,.
才這樣記住這個moment,.
19節,.
瑪利亞卻把這一切事存在心裡,.
反覆思想,.
今晚我們就去思考這節經文,.
不知道你明不明白瑪利亞的心情,.
弟子妹,.
你明不明白?.
不明白吧?.
可能社會研究完了,.
你比較明白迪瑪利亞的心情,.
是不是?.
迪瑪利亞把一球控在心裡,.
反覆盤旋,.
12碼就打到了,.
是不是?.
我們不是說迪瑪利亞,.
我們是說瑪利亞,.
我們思考經文,.
瑪利亞卻把一切的事情存在心裡,.
反覆思想,.
簡單來說,.
經文裡面有兩個動詞,.
我們就研究一下這兩個動詞,.
首先第一個動詞,.
存在心裡,.
瑪利亞把一切的事情都存在心裡,.

$^{401}$其實,.
原文裡面的字是叫做.
Suntereo,.
一個很少出現的,.
一個新約的希臘文字眼,.
Suntereo是什麼意思呢?.
就是解作存放,.
保存,.
保護這樣的意思,.
知不知道出現了三次在聖經裡面,.
三次都是在科幻書裡面出現的,.
第一次就是在耶穌的生瓶,.
生酒的比喻裡面,.
耶穌說,.
每有人將生酒裝在舊瓶裡面,.
有時一樣,.
皮袋就裂開,.
酒漏出來,.
連皮袋也壞了,.
唯獨把生酒裝在生瓶裡面,.
兩樣都保存了,.
這個保存,.
就是Suntereo這個字眼,.
另一段就是在馬福音裡面,.
有一段有關希律王保存約翰的性命的經文,.
他說因為希律王知道約翰是異人,是聖人,.
所以敬畏他,保護他,.
聽他講論,.
照他所行的,.
樂意聽他的,.
總之Suntereo是什麼意思呢?.
就是捍衛寶貴,.
珍而重之,.
存留下來的意思,.
好的酒,.
你要好的去保存它,.
視為珍寶的去保存它,.
性命也是一樣,.
你要去保護它,.
將它視為珍寶的去保存下來,.

$^{441}$這個就是Suntereo的意思,.
經文裡面的英文版很好,.
就是Merry treasure up all these things,.
捍衛寶貴的來到它存留下來,.
好好地去保護它,.
珍惜它,就是這個意思,.
Treasure up,.
不過,.
究竟瑪麗亞Treasure up什麼呢?.
她視什麼為寶貴呢?.
保存什麼呢?.
這個經文大概就是天使報的起訊,.
就是這個關乎萬民的起訊,.
不過強調,.
可能大家都忽略了一個細節,.
其實瑪麗亞從來都沒有聽過天使報的信,.
雖然她是天使報上其中一個觀眾,.
因為是萬民,.
但瑪麗亞從來都沒有聽過當晚天使的信,.
你記得她整晚都在生小孩,.
實際上是什麼呢?.
瑪麗亞當晚一直都在夜晚,.
在馬場那邊,.
突然之間聽到一群夜晚開工,.
突然間進來的一群男人,.
揣述這個消息,.
試試進入瑪麗亞的情境,.
你剛剛生完小孩,.
我明白很多人都沒有生過小孩,.
剛剛生完小孩,.
連電影都還沒回,.
可能連磁鐵都還沒剪,.
血滴滴的整個馬場,.
突然間一群Food Panda,.
一群送宵夜的男人,.
一群二輪車手,.
一個個停在你家門口,.
隨頭龜跟你說,.
恭喜你啊,.
然後抱你一臂,.

$^{481}$然後就在那裡搞很多事情,.
你剛剛生完小孩,.
你完全躺在床上,.
一群人突進,.
你完全消化不了發生什麼事,.
我想如果瑪麗亞聽的是天使的消息,.
會好一點,.
效果會好一點,.
起碼會有一點榮光,.
有敬拜讚美,天使天君,.
但瑪麗亞面對的是一群南亞車手,.
這樣感覺的人,.
夜班工作的人,.
所以瑪麗亞還沒好好地消化當晚發生的事情,.
實際上瑪麗亞自從懷孕之前,.
天使吉伯烈來到她那裡,.
跟她說你要生小孩,.
她是神的兒子,.
她是大衛的子孫,救世主,.
到現在撥一聲生出來了,.
她依然還沒弄清楚整件事的來龍去脈,.
OK,現在兒子生了,.
What next?.
瑪麗亞其實是完全不知道的,.
不過瑪麗亞是好好地將這件事情視為珍貴地保存,.
保護她,放在心裡,.
瑪麗亞不是純粹記住,.
不是純粹要她記得,.
甚至她尚未明白,.
沒有了解,.
沒有吸收成為知識,.
沒有轉化成為經驗,.
不過平安夜當晚,.
瑪麗亞卻選擇了將尚未了解的事情,.
放在心裡好好地保護,.
珍而重之,.
捍衛寶貴,.
Treasure up!.
不知道你有沒有這樣的經歷,.
不明白,.

$^{521}$但依然捍衛是寶,.
有一位同工,.
她跟我們分享家裡生活的艱難,.
她老公最近說要換畫,.
我剛才所說的換畫,.
是Literally換畫,.
要將她家的沙發結婚照換掉,.
要換她老公訂的那幅藝術畫,.
一幅價值六位數字的畫,.
問她同工,.
問她怎樣做好,.
她都不作聲,.
後來這幅畫真的到了家裡,.
掛在客廳上,.
她就發了這幅畫出來,.
我心想,.
真的是一幅經典的藝術品,.
完全不明白是什麼,.
一堆黑色的東西,.
然後就一堆東西,.
就變成這樣的藝術畫,.
接著我發訊息給她,.
她說,漂亮!.
真的,.
我覺得同工是學習這個Santireo的教導,.
尚未明白,.
沒有了解,.
未能消化,.
但仍然要捍衛寶貴,.
珍而重之,.
保存下來,Treasure up,.
放在客廳裡最重要的位置,.
因為你相信這幅畫將會成為你家的棒棒,.
這就是Santireo的意思,.
第一個字是Santireo,.
捍衛珍貴的,.
保存下來,.
第二個動詞,.
馬利亞所做的第二件事是什麼?.
就是反覆思考,.

$^{561}$馬利亞把這一切的事都存在心裡,.
反覆思考,.
這是和合本的翻譯,.
反覆思考,.
但原文的字更有深層次的意思,.
其實要翻譯這句字不容易,.
因為字和經文不太合,.
是無法解釋的,.
原文是Sombano,.
意思是「遇見」,「相遇」,.
但很多時候這字是解釋為「負面的遇見」,.
「負面的相遇」,.
例如經文,老套方第十四章,.
兩隊軍隊打仗的時候的相遇,.
就是Sombano這個字,.
對敵的狀態,.
舊約裡面也出現了Sombano這個字,.
是用來挑撥的意思,.
所以馬利亞不斷地反覆思考,.
不是偉大的新時代特色社會主義,.
學習綱領的那些溫習,.
一天想三次,.
不會想三個小時的思想練習,.
在馬利亞的內心裡面,.
每時每刻都遇見一個.
她不能理解,無法疏離,.
無法明白,不能解的結,.
儘管馬利亞不斷嘗試去釐清,.
去解決,消除這個難解的結,.
嘗試在內心裡面去找一個平衡,.
但她找不到這樣的平衡,.
她知道她彈下來的鷹鞋,.
是人類的盼望,她知道的,.
但她還不知道這個鷹鞋,.
怎樣可以成為人類的盼望,.
馬利亞知道問題,.
馬利亞知道問題的答案,.
但她不知道這個問題的答案,.
能夠怎樣解決這個問題,.
你明白嗎?.

$^{601}$她知道這個密碼,.
但她不知道怎樣登入,.
馬利亞繼續去視這個答案為寶貴,.
因為這是人類大起的訊息,.
馬利亞嘗試去珍惜,.
容許她自己的內心裡面,.
一直處於一個不能理解,.
無法明白的狀態,.
仍然去堅持這些東西是寶貴的事情,.
留在她的生命裡面,.
所以馬利亞卻將這些字,.
存在心裡,.
反覆思想這個字,.
可以這樣翻譯,.
馬利亞不太明白,.
不能理解,.
卻將它當為寶一樣的好好保存,.
在心裡嘗試去找一個平衡點,.
她一直找不到,.
不過仍然在內心裡,.
好好地去保存它,.
這個正正就是平安夜,.
馬利亞的心靈狀態,.
弟姐妹,.
作為基督徒,.
我們這幾年裡,.
我們大概要學習這種心靈狀態,.
去度過我們每一個的聖誕節,.
不知道你這幾年,.
聖誕節怎麼過,.
有到聖誕,.
香港四散,.
我們知道聖誕節是慶祝救主的降生,.
我們知道平安夜是萬民大喜的信息,.
我們知道有一英孩為我們而生,.
但我們不知道這個救主,.
怎麼去幫助我們去解決我們這片土地的問題,.
我們不知道我們下年的聖誕節會發生什麼事,.
甚至我們不知道,.
將來的香港聖誕節還有沒有假房,.

$^{641}$我們大概一直處於這種狀態,.
去度過未來一個又一個的聖誕節,.
不明白,不理解,無法梳理,.
但依然去捍衛寶貴,.
好好記住它,.
好好存放在你心裡,.
Treasure,.
地上的基督徒就是一群知道故事的開始,.
知道故事的結尾,.
卻是搞不通中間的細節的人,.
不過沒關係,.
Treasure it up,.
每一年懷著盼望,憑著信心,.
度過一個又一個的聖誕節,.
瑪利亞在這一切事上存在心裡反覆思想,.
而且瑪利亞一直保持這種心理內心狀態33年,.
瑪利亞一直看著這個BB成為小朋友,.
一直看到他長大,.
心裡也解不通這個小朋友將來會怎樣成為人類的救世主,.
直至到《約翰福音》第19章,.
瑪利亞看見自己的孩子被釘在十字架上,.
在這個時候,十架裡面的耶穌對他母親說,.
母親看你的兒子,.
原來我的兒子死在十字架上,.
在這個百里行裡面的嬰孩將來要這樣去改變這個扭曲的世界,.
瑪利亞堅持用這樣的心態,.
不理解,但是捍衛寶貴33年,.
瑪利亞是一個內向的人,.
因為他不只是看事情的表面,.
他往他的內心去堅持,.
往他的內心出發,.
瑪利亞的內心非常強大,.
我說他非常強大,不是因為他堅強的性格,.
或者是有驚人的意志力,.
瑪利亞的內心很強大,.
因為他堅持要珍惜,.
將救主的訊息珍惜,.
我們記得專屬眾,.
瑪利亞立志,我的心要尊主為大,.
我的靈要以救主為樂,.

$^{681}$無論任何事情,.
聽姐妹,今天我們的講題叫做.
內向者的平安夜,.
所謂內向者,.
就是一個看事情不會只是看表面的人,.
堅持注重自己內心的人,.
聽姐妹,在今年,今個,今天,.
這個平安夜,讓我們回到自己的內心,.
聽姐妹,今天你的內心是怎樣?.
是不是表面堅強,.
內裡脆弱不堪?.
你是否外表上上都是哈哈大笑,.
內裡一直都隱藏著痛苦?.
你一直面面俱圓,.
內裡卻是很多埋怨苦澀?.
平安夜,平安夜,.
求主教導我們,回到我們的內心,.
寂靜,強大的內心..
我自己近年越來越重視自己的內心狀態,.
可能人大了,.
發現內心比一切都重要,.
今天星期二發生了一件事,.
昨天我在灣景酒店過夜,.
全家,忘記帶髮膠,.
中午去了旺角買髮型產品,.
買了一瓶頭泥,一支鹽水,.
付錢時,那個售貨員就在推銷我,.
他說,先生你有沒有用過洗髮水?.
我說,普通的..
他說,先生,你頭髮不能騙人,.
你試試用這個防止脫髮洗髮水吧,.
我心想,真的,有沒有搞錯?.
我頭髮少,你不要這樣,.
我不是過分地說,.
我騙他,.
騙人是罪,我沒有騙你,.
我剛在星期三,.
在神學院街踢球,.
七時坐船出門,.
參加神學院盃,.

$^{721}$我參加比賽已經十多年了,.
基本上十年,.
看到校友都和我踢過球,.
不過由我二十四歲讀神學到現在,.
四十二歲,.
發覺人真的老了,.
身體狀況開始下滑,.
爆發力完全沒有了,.
基本上想做動作都做不到,.
打到第四場突然還要扭傷,.
就提早下班,.
後來我們看到排名,.
好像是尾三,.
以前讀書時排名是冠軍,.
我們看到是冠軍球隊,.
又輸球又受傷,.
拐著腳很累,.
就走去換衣服,.
那一刻我是很氣餒的,.
我發覺自己在做很矛盾的事情,.
我做了一件什麼事?.
我做了一件越做越做得不好的事,.
明不明白?.
你越做越做得不好,.
我發覺沒什麼是這樣的,.
你越做越做得不好,.
我問你還可以踢多久的球?.
然後我就開車去No Mission No Life,.
喝杯奶,安慰自己,獎勵自己,.
坐下來靜下來,.
突然想起一段聖經,.
詩篇的經文,.
我的肉體和我的心腸雖殘,.
但神是我心裡的力量,.
使我能夠福分直到永遠..
真日,就算我的身體衰殘,.
只要我的心靈沒有減退,.
我竟然可以繼續踢下去..
所以我明白維園的老伯是怎麼回事,.
一群仍然有理想,.

$^{761}$仍然有心力的一群男人,.
在維園繼續存在..
真日,世界可以變,身體可以衰退,.
你卻要保守你的心,.
你的內心,學名叫做靈魂,.
一種永恆的特徵,.
它不會被這個世界,.
它可以不會被這個世界影響,.
就算你的身體改變,.
就算這個世界改變,.
你的內心依然可以非常非常健壯..
那麼請問2014年世界盃的美斯,.
和2022年世界盃的美斯有什麼不同?.
我認為最大的差別,.
就是美斯的內心,.
明明是年紀大了,.
明明是跑動慢了,.
最大的差別是,.
美斯的內心強大了很多..
最新的,人越大,.
你越會發現,.
最重要的不是越做越多,.
或者越做越快,.
而是你裡面的人,.
是一個怎樣的人..
面對著這幾年的香港,.
你的心臟要強大,.
似乎是一件必須的事情..
學習,敏感你自己,.
好好去保守你自己的愛心..
重要的不是這些決定,.
或者是這些行動,.
而是你裡面的人,.
是怎樣,.
處於什麼狀態裡面..
正所謂,.
每一個基督徒,.
每一個spiritual的基督徒,.
都需要是一個內向的人,.
不要看世界的表面,.

$^{801}$不要那麼膚淺,.
一直往你的內心出發..
無論這個世界發生什麼事,.
Treasure up!.
請支配我們一段時間,.
尋找我們的內心..
平安夜,Silent Night,.
重點是寂靜,.
因為唯有寂靜,.
才能讓我們回到我們的內心..
神學家Carl Reiner,.
寫過一個很有趣的文章,.
叫做The Theology of Christmas,.
他說,.
最好預備慶祝聖誕節的方式,.
就是寂靜..
寂靜,.
讓我們回到我們的內心..
平安夜,聖誕節,.
我們發現上帝的救恩..
其實,.
壽難節和聖誕節,.
是兩個不可分割的節日,.
兩個不可分割的事件..
上帝的拯救,.
不單單是標誌著十誡上的死亡,.
更加是耶穌基督應懈的降生..
因此,.
從耶穌基督的生,.
到耶穌基督的死,.
從聖誕節,.
到壽難節,.
整個是上帝救贖的旅程..
馬槽,.
是寂靜的,.
牧羊人都回去了..
這才是我們平安夜的本相,.
讓我們回到我們的內心,.
將一切上帝的救恩,.
Treasure up,.

$^{841}$不斷的來到他反覆思想..
平安夜,.
我們要回到我們的內心,.
最少我們要回到時間,.
去寂靜,.
去獨處,.
感受一下這種寂靜..
二晚來,.
上帝的救恩,.
讓我們祝你聖誕快樂..
\newpage



\section{約翰福音 1:9-10-20221225}
\label{sec:LRsUlh9Ini0}
\textbf{【網上崇拜】外向者的聖誕節|約翰福音1\_9-10|20221225 [LRsUlh9Ini0]}
\newline
\newline
連結: \href{https://youtube.com/watch?v=LRsUlh9Ini0}{\texttt{ https://youtube.com/watch?v=LRsUlh9Ini0}} ~~~~ 語音日期: 2022-12-25 
\newline
\newline
\hyperref[sec:KRXO3Wxxfe0]{\small{< < < PREV SERMON < < <}}
~
\hyperref[sec:index_chronic]{\small{[返順時目]}}
~
\hyperref[sec:index_scriptual]{\small{[返順卷目]}}
~
\hyperref[sec:0FV84blFd3M]{\small{> > > NEXT SERMON > > >}}
\newline
\newline
約翰福音 1:9-10-20221225
\newline
\begin{longtable}{cl}
\hline
\hline
章節 & 經文 (和合本修訂版)\\
\hline
1:9 & \begin{tabularx}{0.7\textwidth}{X} 那光是真光,來到世上,照亮所有的人。 \end{tabularx} \\ \\ \relax
1:10 & \begin{tabularx}{0.7\textwidth}{X} 他在世界,世界是藉著他造的,世界卻不認識他。 \end{tabularx} \\ \\ \relax
1:11 & \begin{tabularx}{0.7\textwidth}{X} 他來到自己的地方,自己的人並不接納他。 \end{tabularx} \\ \\ \relax
1:12 & \begin{tabularx}{0.7\textwidth}{X} 凡接納他的,就是信他名的人,他就賜他們權柄作神的兒女。 \end{tabularx} \\ \\ \relax
1:13 & \begin{tabularx}{0.7\textwidth}{X} 這些人不是從血生的,不是從情慾生的,也不是從人的意願生的,而是從神生的。 \end{tabularx} \\ \\ \relax
1:14 & \begin{tabularx}{0.7\textwidth}{X} 道成了肉身,住在我們中間,充充滿滿地有恩典有真理,我們也見過他的榮光,正是父獨一兒子的榮光。 \end{tabularx} \\ \\ \relax
1:15 & \begin{tabularx}{0.7\textwidth}{X} 約翰為他作見證,喊著說:「這就是我曾說:『那在我以後來的先於我,因為在我以前,他已經存在。』」 \end{tabularx} \\ \\ \relax
1:16 & \begin{tabularx}{0.7\textwidth}{X} 從他的豐富裡,我們都領受了恩典,而且恩上加恩。 \end{tabularx} \\ \\ \relax
1:17 & \begin{tabularx}{0.7\textwidth}{X} 律法是藉著摩西頒佈的;恩典和真理卻是由耶穌基督來的。 \end{tabularx} \\ \\ \relax
1:18 & \begin{tabularx}{0.7\textwidth}{X} 從來沒有人見過神,只有在父懷裡獨一的兒子將他表明出來。 \end{tabularx} \\ \\ \relax
1:19 & \begin{tabularx}{0.7\textwidth}{X} 這是約翰的見證:猶太人從耶路撒冷差祭司和利未人到約翰那裡去問他:「你是誰?」 \end{tabularx} \\ \\ \relax
1:20 & \begin{tabularx}{0.7\textwidth}{X} 他就承認,並不隱瞞,承認說:「我不是基督。」 \end{tabularx} \\ \\ \relax
1:21 & \begin{tabularx}{0.7\textwidth}{X} 他們又問他:「那麼,你是誰?是以利亞嗎?」他說:「我不是。」「是那位先知嗎?」他回答:「不是。」 \end{tabularx} \\ \\ \relax
1:22 & \begin{tabularx}{0.7\textwidth}{X} 於是他們對他說:「你到底是誰,好讓我們回覆差我們來的人。你說,你自己是誰?」 \end{tabularx} \\ \\ \relax
1:23 & \begin{tabularx}{0.7\textwidth}{X} 他說:「我就是那在曠野呼喊的聲音:修直主的道。」正如以賽亞先知所說的。 \end{tabularx} \\ \\ \relax
1:24 & \begin{tabularx}{0.7\textwidth}{X} 那些人是法利賽人差來的。 \end{tabularx} \\ \\ \relax
1:25 & \begin{tabularx}{0.7\textwidth}{X} 他們就問他:「你既不是基督,不是以利亞,也不是那位先知,那麼,你為甚麼施洗呢?」 \end{tabularx} \\ \\ \relax
1:26 & \begin{tabularx}{0.7\textwidth}{X} 約翰回答:「我是用水施洗,但有一位站在你們中間,是你們不認識的, \end{tabularx} \\ \\ \relax
1:27 & \begin{tabularx}{0.7\textwidth}{X} 就是那在我以後來的,我給他解鞋帶也不配。」 \end{tabularx} \\ \\ \relax
1:28 & \begin{tabularx}{0.7\textwidth}{X} 這些事發生在約旦河東邊的伯大尼,約翰施洗的地方。 \end{tabularx} \\ \\ \relax
1:29 & \begin{tabularx}{0.7\textwidth}{X} 第二天,約翰看見耶穌來到他那裡,就說:「看哪,神的羔羊,除去世人的罪的! \end{tabularx} \\ \\ \relax
1:30 & \begin{tabularx}{0.7\textwidth}{X} 這就是我曾說『那在我以後來的先於我,因為在我以前,他已經存在』的那一位。 \end{tabularx} \\ \\ \relax
1:31 & \begin{tabularx}{0.7\textwidth}{X} 我先前不認識他,如今我來用水施洗,為要使他顯明給以色列人。」 \end{tabularx} \\ \\ \relax
1:32 & \begin{tabularx}{0.7\textwidth}{X} 約翰又作見證說:「我曾看見聖靈彷彿鴿子從天降下,停留在他的身上。 \end{tabularx} \\ \\ \relax
1:33 & \begin{tabularx}{0.7\textwidth}{X} 我先前不認識他,可是那差我來用水施洗的對我說:『你看見聖靈降下來,停留在誰的身上,誰就是用聖靈施洗的。』 \end{tabularx} \\ \\ \relax
1:34 & \begin{tabularx}{0.7\textwidth}{X} 我看見了,所以作證:這一位是神的兒子。」 \end{tabularx} \\ \\ \relax
1:35 & \begin{tabularx}{0.7\textwidth}{X} 又過了一天,約翰同兩個門徒站在那裡。 \end{tabularx} \\ \\ \relax
1:36 & \begin{tabularx}{0.7\textwidth}{X} 他見耶穌走過,就說:「看哪,神的羔羊!」 \end{tabularx} \\ \\ \relax
1:37 & \begin{tabularx}{0.7\textwidth}{X} 兩個門徒聽見他的話,就跟從了耶穌。 \end{tabularx} \\ \\ \relax
1:38 & \begin{tabularx}{0.7\textwidth}{X} 耶穌轉過身來,看見他們跟著,就問他們說:「你們要甚麼?」他們對他說:「拉比,你在哪裡住?」(「拉比」翻出來就是老師。) \end{tabularx} \\ \\ \relax
1:39 & \begin{tabularx}{0.7\textwidth}{X} 耶穌說:「你們來看。」他們就去看他在哪裡住。這一天他們就跟他同住;那時大約是下午四點鐘。 \end{tabularx} \\ \\ \relax
1:40 & \begin{tabularx}{0.7\textwidth}{X} 聽了約翰的話而跟從耶穌的那兩個人,其中一個是西門‧彼得的弟弟安得烈。 \end{tabularx} \\ \\ \relax
1:41 & \begin{tabularx}{0.7\textwidth}{X} 他先找到自己的哥哥西門,對他說:「我們遇見彌賽亞了。」(「彌賽亞」翻出來就是基督。) \end{tabularx} \\ \\ \relax
1:42 & \begin{tabularx}{0.7\textwidth}{X} 於是安得烈領西門去見耶穌。耶穌看著他,說:「你是約翰的兒子西門,你要稱為磯法。」(「磯法」翻出來就是彼得。) \end{tabularx} \\ \\ \relax
1:43 & \begin{tabularx}{0.7\textwidth}{X} 又過了一天,耶穌想要往加利利去。他找到腓力,就對他說:「來跟從我!」 \end{tabularx} \\ \\ \relax
1:44 & \begin{tabularx}{0.7\textwidth}{X} 這腓力是伯賽大人,是安得烈和彼得的同鄉。 \end{tabularx} \\ \\ \relax
1:45 & \begin{tabularx}{0.7\textwidth}{X} 腓力找到拿但業,對他說:「摩西在律法書上所寫的,和眾先知所記的那一位,我們遇見了,就是約瑟的兒子拿撒勒人耶穌。」 \end{tabularx} \\ \\ \relax
1:46 & \begin{tabularx}{0.7\textwidth}{X} 拿但業對他說:「拿撒勒還能出甚麼好的嗎?」腓力說:「你來看。」 \end{tabularx} \\ \\ \relax
1:47 & \begin{tabularx}{0.7\textwidth}{X} 耶穌看見拿但業向他走來,就論到他說:「看哪,這真是個以色列人!他心裡是沒有詭詐的。」 \end{tabularx} \\ \\ \relax
1:48 & \begin{tabularx}{0.7\textwidth}{X} 拿但業對耶穌說:「你從哪裡認識我的?」耶穌回答他說:「腓力還沒有呼喚你,你在無花果樹底下,我就看見你了。」 \end{tabularx} \\ \\ \relax
1:49 & \begin{tabularx}{0.7\textwidth}{X} 拿但業回答他:「拉比!你是神的兒子,你是以色列的王。」 \end{tabularx} \\ \\ \relax
1:50 & \begin{tabularx}{0.7\textwidth}{X} 耶穌回答他說:「因為我說在無花果樹底下看見你,你就信嗎?你將看見比這些更大的事呢!」 \end{tabularx} \\ \\ \relax
1:51 & \begin{tabularx}{0.7\textwidth}{X} 他又說:「我實實在在地告訴你們,你們將要看見天開了,神的使者在人子身上,上去下來。」 \end{tabularx} \\ \\
[1ex]
\hline
\hline
\end{longtable}
$^{1}$各位姐妹平安.
很開心在這裡見到大家.
昨晚崇拜時,我在報告時也跟弟兄姐妹分享.
很難得場地可以有100\%讓弟兄姐妹一起聚會.
我們能夠享受得來不易的空間.
也跟網上的弟兄姐妹打招呼.
剛才我在網上看也有百多個弟兄姐妹跟我們一起.
特別是多倫多的朋友.
現在是凌晨三點.
溫哥華的朋友也是半夜.
還有倫敦的朋友.
現在你們早上八點.
一起看直播.
謝謝你們跟我們一起崇拜.
能夠透過科技和不同的方式.
彼此聯合是一個很大的恩典.
今天來到會場的時候應該有不少東西要拿.
進來拿了貼紙了嗎?.
拿了甚麼顏色?紅色的?.
紅色的,可以出個答案.
紅色的是甚麼?.
有誰是紅色的?.
今天是紅色主題.
愛向者的聖誕節.
綠色呢?有拿綠色嗎?.
綠色的,OK.
不只是昨晚才拿綠色,今天也有綠色的.
喜歡不同顏色可以拿不同的貼紙.
通常小朋友會馬上貼在身上.
正在甩,很開心.
除了兩個貼紙之外還有沒有要拿?.
拿甚麼?.
這個小朋友已經在照人了.
有兩個電筒.
懂得用電筒嗎?.
應該懂的,應該OK的.
有些已經開了.
先保留著.
今天我們來到第二個節日崇拜.
很難得Fellow Church很少在10月25日舉行.

$^{41}$因為我們不是特定這個日子去崇拜.
但今天連續兩個聚會.
一起是一個很大的恩典.
無論場地也好,剛好是星期六,日.
我們可以兩天一起有弟兄姊妹去崇拜.
今天講的訊息.
在開始的時候.
進場的時候,弟兄姊妹聽了兩個MV.
MV是一個動畫.
動畫版最初是2019年的時候.
我們12月的月題是講詩語.
這是我們月題的詩歌.
當中也讓弟兄姊妹一起透過12月.
去養一個行動和信仰反省.
最記得2019年12月.
平安夜我們也去做一個社區的參與.
我們不是報戒任.
我們去分發物資.
和一些街坊和地方的朋友分發物資.
但那次的活動其實很困難.
因為大家也記得2019年12月不是一個很平淡的日子.
我們也被驅散,不可以集結.
很快地,沒有什麼聊天的空間.
就完成了那次分發物資.
去到另外一個.
剛才看的MV是一個真人版的MV.
就是2020年我們做了一個再重唱.
讓我們弟兄姊妹再一次去聯想.
其實在我們人與人接觸是最緊密的.
但2020年正正是疫情開始的一年.
不斷地分隔我們.
不斷地讓我們不能和別人去結連.
但聖誕節正正就是告訴我們一個很重要的訊息.
就是上帝差派他自己的兒子來到世上.
走進人群.
我們是他的兒女.
我們是不是更加應該走進人群呢?.
所以疫情是不斷地將我們分隔.
將我們分成不同的類別的時候.
但我們作為上帝的兒女.

$^{81}$我們有沒有空間去多走一步呢?.
今天你想一個外向者,一個內向者.
其實你用什麼去了解?.
用心理測驗,用一些行為模式.
但據我認識和我知道的.
其實外向者,內向者.
其實是一個你充滿能量的方式.
一個內向者他如何充滿能量呢?.
就是他自己安靜的時候.
他會建立能量.
他自己可以得到力量.
一個外向者他與人相處與人結連的時候.
透過氣氛氛圍.
讓他感受到被充滿能量.
其實如果你了解這個狀況的時候.
你就知道.
其實一個內向者也可以很開朗.
有好的社交,好的溝通方式.
一個外向者.
其實也有他自己很無聊的時間.
他自己有不安的時間.
無論你是從什麼途徑知道.
你是一個外向者還是內向者.
我們的喜怒,哀驚,我們的情緒.
都是上帝做我們的一部分.
不要被一些你知道的事情困住了.
其實上帝讓我們享受我們的生命.
讓我們享受上帝所賜予的.
正正就是我們從上帝而有的那種喜樂.
可以與人相近.
在詩歌裡提醒一個信息就是.
聖誕歌響起了.
我們能不能夠重現基督的關懷.
以及在這個階段中去思雨呢?.
我今早去了一間堂會講報道會.
都是講聖誕的信息.
聖誕信息裡面也不是僅僅講傳統的經文.
我希望與一班堂會的弟兄姊妹去想一下.
當周遭環境的氛圍.
都在講聖誕節要買禮物,要交換.

$^{121}$其實耶穌基督的出現是一個什麼情景呢?.
當我們要聖誕的歌響起.
提醒我們思雨的時候.
我們可以送禮物給別人.
但我們還有什麼可以送給別人呢?.
當然我們很快就會跳到去.
我們送耶穌給別人.
我們將聖誕信息真義告訴別人.
但其實講聖誕真義是很難的.
難到一點就是.
我當初在中學階段的時候.
也有回過教會.
回教會的原因是因為.
別人說那裡好吃好玩.
當然我們沒有什麼地方去.
就叫著跟著去.
真的好吃好玩.
由早上十點鐘到下午四點鐘.
都是吃和玩的.
還有禮物拿.
這些地方在哪裡找呢?.
不用給錢.
但當我慢慢透過途徑認識了教會.
參與了.
其實當我認識了耶穌的工作的時候.
我就覺得聖誕節其實不是這麼好的事情.
為什麼不是這麼好的事情呢?.
不是值得這麼開心的.
當你知道耶穌降世是要死的話.
其實如果你知道祂出世就是為了死.
其實有多開心呢?.
是吧?.
你要介紹人認識聖誕節的原意就是.
祂出世就是為了死.
祂出世就是為了人.
我們世人的話.
其實很難是一個很開心的.
只可以慶祝聖誕節.
你慶祝一個人都會死.
這不是搞嘟嘟笑.

$^{161}$當我們認真看這個信息的內涵的時候.
其實真的很困難告訴別人.
是一個歡喜快樂的日子.
但我們也不需要扭曲基督信仰.
反而我們認真去看.
既然耶穌是為了我們人沒有能力自救.
或者人沒有能力面對這個世界崩壞的時候.
基督信仰對我們就是基督要出現.
那光是真光照亮一切身在世上的人.
這就是我今天講聖誕節的信息.
我選的經文都不是來自馬太馬賀.
路加這些傳統聖經的.
講滅術聖誕節的經文.
我反而選了一卷就是《約翰福音》.
沒有記載耶穌降生的經卷.
耶穌來這個世界是什麼?.
祂就是世人的光.
所以經文裡面提及一個很重要的信息.
那個光是真光照亮一切身在世上的人.
祂來到這個世界就要讓人明白到.
這個世界因罪的緣故被玷污.
被黑暗蒙蔽.
但是上帝確實要把祂自己的兒子.
道成肉身.
住在我們中間.
讓人明白到這個世界會有真光.
而真光就是藉著我的兒子.
照亮身在世上的人.
福帖書照亮這個字.
它想要表達的信息就是.
這個世界不容易看清.
因為很多東西在蒙蔽著你.
這個世界不容易看清楚.
因為很多地方在陰暗.
但是這個真光會令到你看得清.
這個真光會令到你看得見.
以致你有得選擇.
你要懂得選擇.
但是怎樣.
它在這個世界.

$^{201}$這個世界都是它做的.
但是世界的人就不認識它.
對的.
在香港其實要知道福音信仰其實就不難的.
因為香港如果以中學為例.
有三分一的中學都是教會.
即是中派辦學的.
所以很多中學的.
很多中學生有三分一都一定聽過.
宗教課或者福音營.
或者福音樂.
他們不情願也好.
他們都會經歷過這日子.
所以耶穌基督的信仰對於.
起碼有三分一基督教辦學的群體當中.
是沒有說沒有接觸過.
但是接觸到是否認識呢.
當然不是.
但是對於我們這群信仰群體的時候.
今天無論是新朋友也好.
或者是你回到教會很久也好.
其實要認識基督信仰.
從來都不是一些很容易的事.
要帶出一個信息.
要讓人明白到.
就像剛才所說.
其實真的要了解聖誕節.
其實聖誕節不是一些很開心的日子.
正如剛才所說.
但是正正因為這個世界有很多困難.
上帝就做了一件新事.
讓祂自己的兒子來到這個世界.
讓別人有一個新的途徑.
可以回到天父那裡.
這就是道路.
所以這是要認識.
要認識這條路的過程中.
從來都需要時間.
但是我們會不會花時間呢.
一個輔佐經文跟大家去看的就是.

$^{241}$保羅在對教會一個很重要的書信.
大家都認識的就是《依乏所書》.
《依乏所書》第一章十八節和三章九節.
有段經文是說.
這個輔提訴(光照).
上帝要照明我們認識什麼呢.
就是.
並且要照明你心中的眼睛.
使你們知道祂的恩照有何等的指望.
祂在聖徒中得的基業有何等豐盛的榮耀.
其實要告訴我們這群信了主.
無論多久都好.
你要看清楚.
看清楚我們心中的眼睛要看清楚.
讓我們知道.
上帝給我們這個呼召.
其實有遠景的.
有一個遠景.
這個遠景就是我們將會得到一個很豐盛的人生.
和我們看到將來的榮耀.
其實都是在說將來式.
雖然現在不行.
但是將來會行.
我之前在流唐也說到.
說未來的信息的時候也在說.
現在很困難.
但是我們要展望將來.
不只是看當下這個短視.
看長一點.
另外三章九節提到一件事就是.
「有使眾人都明白.
這歷代以來隱藏在創造萬物之神女的奧秘是如何安排的」.
這裡沒有照明.
誰照明呢?.
其實照明這個字就等同明白.
為什麼明白呢?.
第九節裡面說的這個信息就是.
有些東西隱藏.
遮住了.
蒙蔽了看不清.

$^{281}$又或者黑漆漆.
看不到的時候.
如何可以讓人明白到呢?.
就是你照.
看到.
譬如有些東西.
這個地方這個場景很黑.
黑的過程當中我們當然看不到.
但是當我們拿個電筒.
我現在收上電筒.
就拿個電筒照的時候就知道哪一個路有.
那個東西本身就在這裡.
透過一光來到的時候就看到.
假設我們現在去探險.
這個地方是一個寶藏.
我告訴你裡面有很多寶藏.
但是黑的你看不到.
但是當你拿著光.
去到不同的地方照一照.
這個是聖杯.
照一照這裡是一個寶劍.
這個是寶石.
你去到不同地方.
你照明.
你就會明白到.
你照到去哪裡.
你就明白到.
這裡真是一個寶藏.
親愛的弟兄姐妹.
當上帝的說話.
當我們更加去認識到基督信仰.
被上帝的說話光照的時候.
我們更加明白到.
耶穌為什麼要來這裡.
我們更加會明白到.
我們為什麼需要耶穌.
我們更加會明白到.
上帝給我們的恩典.
其實很豐盛.
聖達芝除了我們一直以來聽了.

$^{321}$不同科幻書說聖嬰的出現.
和天使的和唱.
和瑪利亞和約瑟的心智的時候.
我們更加會看到.
其實上帝預示我們.
祂降生之後.
我們怎麼接觸.
讓我們被上帝光照之下.
我們怎麼看上帝的安排.
所以到了經文裡面.
你會看到一個.
就是下一個我想跟大家說的.
我們要過活是怎麼過活呢.
也是來自《以下福音》.
大家很熟悉的經文.
就是第《以下福音》第十章第十節.
我們通常熟悉的就是.
十節下.
就是我來是叫人得生命.
並且得得更豐盛.
但是我想跟大家看整個第十節.
就是到賊來.
無非是要偷竊,殺害,毀壞.
惡者從開初就破壞上帝的計劃.
沒有這麼簡單的.
他更加去煽惑亞當.
去吃分別煽惡術的果子.
但是上帝看在眼裡.
人是需要經過這個歷練.
讓他做選擇.
信服還是選擇.
信從自己的心.
這個是上帝給人一個測試.
一個試驗.
但是到我們成為.
亞當的後裔的時候.
我們都經歷過很多困難.
很多事情.
我們被蒙蔽.
或者我們自己的人生當中有很多.

$^{361}$不同上面的難處.
我們很多時候除了自己的心.
也除了濫惡者的編排.
或者是煽動.
我們就走了一條我們不想走的路.
或者做了不想做的決定.
所以盜賊來到的時候就是想破壞.
想讓我們抽離.
不再回到上帝當中.
又或者不想聽聖經的教導.
不想明白.
簡單的OK了.
但是耶穌的出現是什麼?.
耶穌的出現就是我來是叫人得生命.
上帝給我們的生命氣息.
這條命是怎麼過?.
我們當然有選擇.
但是你選擇躺平地過.
還是選擇去參與.
去不同方式.
去做到你想做的事情呢?.
做得到.
很多人都可以的.
但是我口頭禪就是做得到.
又做得好.
「好」和「好」差一個字的時候.
就是你怎麼去轉化.
或者怎麼去更加認識那件事.
做一個基督徒難不難呢?.
其實也不算很難的.
你考慮誠認心理相信.
上了教會.
上了舞蹈班.
可能洗了禮.
就完成了所謂基本的事情.
所謂基本的事情.
但是那是做得到基督徒.
但是做得好基督徒呢.
就多很多事情了.
一個星期有兩天上教會.

$^{401}$現在已經很厲害了.
除了崇拜之外.
還可能上小組一個星期.
上兩天已經很好了.
但是你會聽過很多次.
我們的信仰表達.
不只是那兩天信徒.
七天都是天父的世界.
七天都是你的生命.
我們是不是一個戴著面具的基督徒.
還是我們真的是呢?.
所以我們像一個基督徒.
還是像一個基督徒呢?.
像就是不是.
是就不需要像.
而是我經常口頭禪.
A課就是A課.
真課就是A課.
真課就是真課.
在重點.
和大家去了解.
大家都跟到我的思路.
OK.
所以重點就是.
耶穌就說.
我來到是要叫你們得著生命.
而這個生命是很豐盛的.
這個字對我來說是最難說的.
特別這幾年.
什麼是豐盛生命呢?.
Living well.
很難.
Being well很難.
因為不容易.
但這個字.
豐盛這個字是.
Abundant of grace.
即是上帝的恩典會在當中出現.
無論我們什麼環境也好.
上帝的恩典都會在當中出現.

$^{441}$我們能不能感受會看到呢?.
我們能不能夠知道或明白呢?.
就是我們看不看到上帝的光.
我們看不看到上帝的公認.
你明白嗎?.
這是一個信心很重要的提醒.
對我們來說.
很多時候我們認真讀經.
你會看到經文上.
在不同年代的經文.
即是科文書之後.
其實有不同的書信.
所以每個不同年代的經文.
就是很多歷世歷代信徒.
每一次過聖誕節.
每一次去重覆紀念耶穌基督的降生.
每時每刻經歷我們這個信仰群體.
面對的困難的時候.
這個信仰提醒我們什麼.
以至上帝的說話光照我們的時候.
我們會看到平時被蒙蔽的東西.
我們會明白平時隱藏的東西.
但我們彼此提醒彼此幫助.
真正認識聖經.
我們更加會知道.
上帝來這個世界.
就是為了我們的緣故.
所以另外一個輔助經文.
更加會看到這個光照是什麼呢?.
就是希伯來書.
說的一個很重要的信息.
希伯來書第六章裡面.
會提到一個身份上.
你奉承生命的身份有什麼?.
論到那些已蒙了光照.
Fortissio這個字又再出現了.
就是我們這群被光照.
成為上帝兒女的人.
有幾個特質.
就是我們嘗過天恩的滋味.

$^{481}$你感受到上帝在我們生命當中的參與.
而我們也感受到聖靈的提醒.
甚至獨澤和催促.
而我們嘗過上帝的說話滋潤我們.
而我們會如是將來.
上帝會再回來.
希伯來書再一次提醒就是.
上帝藉著祂兒子來到這個世界.
我們現在慶祝的聖誕節.
是更加重要的.
就是上帝參與在其中.
讓我們更加明白到.
我們這個群體要認識我們自己的身份.
看到上帝的工作.
看到我們自己身份更加寶貴.
更加多的可能性.
我花很長的篇幅跟大家說.
聖誕節的經文以外.
正正都是提醒我們.
其實聖誕節是在提醒我們的身份.
是耶穌基督來世上提醒我們.
有兒女的身份.
可以成就上帝福分的身份.
是否容易呢?.
不容易的.
連耶穌在地上的三年多的日子都不容易.
我們是祂們的跟從者.
我們怎會容易呢?.
所以希伯來書在說信心的篇章第十章的時候.
它在說預示將來.
就好像我們現在.
希伯來書已經是二千年前成書.
但是它今天在說關於信心的篇章.
和我們現在面對的困難.
不是差很遠.
讓我一起讀出.
我讀給大家聽.
第32節在說.
第十章第32節開始.
你們要追念往日蒙了光照.

$^{521}$Fortissio又一個字出現了.
就是上帝要我們看見.
蒙了光照以後所忍受大真正的各樣苦難.
一面被毀榜遭患難成了氣境.
叫眾人觀看.
一面陪伴那些受這樣苦難的人.
希伯來書在說.
很多人受苦.
而這段經文之前就是那些因信而受苦的偉人.
他們經歷的事情歷歷在目.
在每個年代.
每個時段都有為上帝受苦的人.
無一倖免.
而去到後續.
我們受書看聖經的人.
其實都會經歷這件事.
所以一定有人陪你的.
第34節.
因為你們看信了那些被困所的人.
並且你們的家業被人搶去.
也甘心忍受.
知道自己有更美長存的家業.
剛才說的永恆的事情.
大家都看著將來.
耶穌基督會再回來.
這是我們心信的.
上帝已經為我們預備這個封城.
第35節.
所以你們不可丟棄勇敢的心.
全這樣的心必得大賞賜.
珍惜啊.
珍惜我們這種信心.
也珍惜我們這個身份.
我們的信仰群體.
每年不可以慶祝聖誕節.
但我們更加要每時每刻.
去珍惜我們的身份.
更加每時每刻珍惜我們的信仰群體.
彼此提醒.
不要自己走差.

$^{561}$不要自己落單.
大家走在一起.
第36節.
所以我們必須忍耐.
使你們行了神的旨意.
就可以得著所應許的.
上帝有使命給我們.
我們有任務的.
不是國家給的.
OK.
我們本身有任務.
是上帝給的.
我們的任務做好我們自己的身份.
我們的任務做好我們自己接觸的群體.
做好我們自己崗位而有的工作.
這是上帝給我們的.
不要小看.
我近幾年有時.
都看到很多弟兄姊妹很洩氣.
又或者是很多情緒不穩定.
是理解的.
不容易的.
就好像剛才在台後.
跟攝製隊一起聊天的時候.
大家不知不覺就在說這兩三年.
我們怎樣過.
Flow Church是怎樣過.
Flow Church就是每天都在想.
不同的方法.
怎樣讓弟兄姊妹可以見到面.
因為疫情限制.
我們又沒有自己的地方.
又怎樣可以運作到一個好的崇拜.
讓不同的弟兄姊妹都可以每個星期一起見面.
網上見也好.
網絡見也好.
即是打字見也好.
都是一個我們可以結連的空間.
我們做詩歌可以讓弟兄姊妹有不同的聯想.
有不同的使用方法.

$^{601}$我們扮演不同的可能性.
就是讓更加多人去.
透過信仰彼此結連.
真的.
你看回一個很實際的地方.
我們實體做直播的時候可能有200人.
但我們網絡.
每個崇拜大概有6,000-7,000人.
我們教會有多少人.
我真的不知道.
這個也是John經常說.
我們真的不知道Flow Church有多少人.
但我們不需要Flow Church有多少人.
我們需要每個星期.
去結集更加多弟兄姊妹和他們打氣.
去做一次珍惜身份.
當我們.
頭髮很不濟的時候.
整理好.
睡醒又是新的一天.
又來過.
提醒自己.
我們有自己的身份.
現在我們可能被人搶了很多東西.
現在我們可能沒有了很多東西.
現在我們可能沒有了.
好像剛才說的沒有了家當.
我們仍然在場外場內有不同患難的日子.
但對我們來說.
終極的.
不是等耶穌回來解決.
我們每時每刻都會解決我們的問題.
實體參與的弟兄姊妹.
你走出來參與每個聚會.
就是對這個群體的支持.
不僅是Flow Church.
是這個信仰群體的支持.
節期是最容易凝聚弟兄姊妹的.
復活節,聖誕節.
總會有很多人會出來的.

$^{641}$但我們不是節期信徒.
我們是基督徒.
每時每刻都是基督徒.
有人說我是一個愛向者.
其實我不是的.
有人知道我說一個愛向者的崇拜.
是的.
二選一.
是不是.
但我自己真的想說愛向者.
因為如果你們普遍人覺得愛向者是要接觸多一點群體的話.
我是一個開朗的內向者.
我不是狡辯的.
事實上是真的.
我喜歡的運動或我喜歡的東西.
我都是喜歡群體活動.
無論我起家是打排球的.
然後轉打籃球.
都是群體運動.
我喜歡的聚會當中.
都是喜歡和大班人一起參與.
但我自己charge up energy.
我自己通常都是很窄的.
看書,聽歌.
我老婆經常說我.
她應該在看我.
她說我喜歡的東西都不用外出.
是的.
我每個星期都會做什麼.
我每個星期都會在星期六和頂姐妹和敬拜隊一隊去崇拜.
但我是無法投入崇拜的.
因為我有很多事情都緊張.
看頂姐妹,流程,網上有沒有脫線.
很緊張,我無法投入崇拜.
我什麼時候崇拜呢?.
我每次都是完事.
十二時多回家洗澡.
我一點多的時候開始我的一人崇拜.
重看我們的崇拜一次.
我是這樣的人.

$^{681}$對我來說.
每一個崇拜都是上帝讓我們charge up.
無論內向還是外向.
崇拜就是讓我們提醒我們的身份.
也是我們可以接觸到弟兄姊妹.
有一首歌我很喜歡.
在我很困難的時候.
我常常都提醒我自己.
那首歌裡面的內容.
那隊band叫Weber Band.
不知道你什麼時候開始喜歡它.
我都很久之前開始喜歡它.
這首歌是2009年出的.
每一次我自己覺得不太upbeat的時候.
我都會聽這首歌.
這首歌提醒我.
我是很喜歡那樣東西.
這首歌叫做阿波羅.
這是我其中一段的歌詞.
不過現在打阿波羅.
就不會去雪糕了.
我的old school就是雪糕.
現在打阿波羅應該去了Jeremy那首歌.
但是完全是兩個世界.
Jeremy的阿波羅.
我沒有offense.
不過Weber Band的阿波羅.
是我每次聽的時候都很upbeat.
令我自己可以很提醒.
Yes!我就是這樣了.
那首歌聽一點給大家聽.
好.
(歌詞).
(如果我們還想戀近乎熱透).
(如雷要落下 待我瞬間走火).
(如能在光速間 蒸發絕望).
(幾千世事讓破 這溫暖不要打破).
(世界沒有夢 火花如雜荒).
(毫無掌心的心).
(我們就忘記過去).

$^{721}$(盡量證明給人 無畏的守護).
(無畏的守護).
特別的地方就是讓很多弟兄姊妹一起參與.
對我來說.
我可以提醒自己.
I always shine for you.
你可以做些什麼呢?.
這個是昨晚看到的圖畫.
弟兄姊妹點起它的燈.
讓它知道我在這裡.
你手上的電筒.
你懂得用嗎?.
平時很簡單的.
我轉一下轉一下就是了.
你轉不轉得到?.
我看到有人轉得到.
說的是你手上的電筒.
不是你手機的電筒.
你不要拿你自己手機的電筒.
你「似」即是「不是」.
你要拿我們手上派給你的電筒.
你懂得用嗎?.
有時有些東西.
你以為你懂得用.
但其實你用的時候你是不懂得用的.
你平時的信仰也是.
你以為你自己懂得信仰.
但問清楚.
其實你是不明白的信仰.
如果你懂得開.
你開吧.
讓你更加明白.
我們的身份而有的盼望.
那種能力.
是要用出來的.
弟兄姊妹.
I will always shine for you.
但你又可不可以光照你身邊的人呢?.
讓它點起我們的光.
無論我們在哪裡.

$^{761}$我們的光.
就照在人前.
讓人見到你.
得榮耀.
在天上的符.
我第一次祈禱.
天上的上帝.
我們就拿著你.
給我們這裡每度微小的光.
我們繼續.
去照亮身邊的人.
我們這盞燈.
不會熄的.
是上帝你給我們.
存在心裡的光.
是耶穌基督.
降世的光.
那光是真光.
照亮一切身在世上的人.
今天我們.
是上帝的兒女.
我們是上帝.
在地上光明的使者.
有願意我們的光.
繼續.
去光照身邊的人.
不容易的日子.
仍然在這裡.
但歷世歷代的信徒.
仍然持守這個光.
我們是上帝.
仍然持守這個光.
直等到我主.
榮耀中在來.
祈禱.
奉耶穌基督的名求.
阿門.
\newpage



\section{撒母耳記下 4:4-9:1-13-20221231}
\label{sec:0FV84blFd3M}
\textbf{【網上崇拜】感恩飯局-「邊個係你飯腳?」|撒母耳記下4\_4,9\_1-13|20221231 [0FV84blFd3M]}
\newline
\newline
連結: \href{https://youtube.com/watch?v=0FV84blFd3M}{\texttt{ https://youtube.com/watch?v=0FV84blFd3M}} ~~~~ 語音日期: 2022-12-31 
\newline
\newline
\hyperref[sec:LRsUlh9Ini0]{\small{< < < PREV SERMON < < <}}
~
\hyperref[sec:index_chronic]{\small{[返順時目]}}
~
\hyperref[sec:index_scriptual]{\small{[返順卷目]}}
~
\hyperref[sec:rQtMXvUNeaE]{\small{> > > NEXT SERMON > > >}}
\newline
\newline
$^{1}$今天是2022年的第365日.
很開心可以和大家一起度過.
終於到我這個外向型的目者跟大家講道.
Flowchart實在有很多內向型的目者.
有些人說歡呼.
呼呼給上帝.
因為創造實在太奇妙.
又內又外.
很開心和大家度過.
其實對於我來說哪一天和大家敬拜都這麼開心.
因為我最喜歡人多.
今天很特別.
在社交媒體大家都會看到.
大家不斷在回顧.
好像人生快要終結一樣.
做了一個小總結.
我發現人家的人生這麼豐富.
有十幾點.
我怎麼想都不太想到.
因為我通常想的都只能夠在那一個月.
回顧過去一年.
不知道在你腦海裡剩下什麼.
剩下的通常都是比較重要的人事物.
你又猜猜我的腦充滿什麼.
可能我的組員會比較知道.
我的兒子.
我的兒子也適合一半.
一半是兒子.
一半都是上帝的創造.
都是人類.
一個又一個和人一起的飯局.
有些飯局我們會很想去.
但有些就相反.
例如公司某些.
慶幸大部分我今年出席的飯局.
都是開開心心地去.
過節.
farewell farewell再farewell.
去喝酒.
小組.

$^{41}$舊同學.
街坊聚會等等.
途中這個是什麼飯局呢.
大家應該知道吧.
回到Flo Church應該點到他們的相.
無論我怎麼遮掩他們.
就是我的同事聚餐.
我也有同事的.
同事聚餐.
我們除了一起吃好東西之外.
這次聚餐還讓我知道.
某人家裡多了一幅畫.
今天就不說這幅畫.
如果有興趣知道我們對話內容.
可以看回上星期24號的平安夜崇拜.
對我來說飯局吃很重要.
吃之外大家交流都非常重要.
一個飯局可以連繫很多不同人的關係.
如果讓你去聖經的時代.
和聖經時代的人連繫起來.
你會想去哪個飯局呢.
大家想不想起聖經有什麼飯局.
可能最容易想到的就是主餐.
很震撼吧.
或者以斯帖請哈曼那一餐.
但今天大家看到我穿成這樣.
家裡衣服不多.
這樣是去喝的裝束.
不好意思.
看得出是比較樸素.
我穿成這樣去喝.
是想帶大家去見識一下.
我們一起去一個皇室的飯局.
和皇室成員迎接一位重要的貴賓.
去之前我們先來起底.
我們一起認識一下這位貴賓.
這位貴賓是怎樣的呢.
.
這個飯局的貴賓就是米菲波切.
可能大家會有些陌生.

$^{81}$因為相對於她的父親約拿丹.
爺爺素羅.
她在聖經裡的出鏡率很少.
但有她出場的情節都會非常緊湊.
生於皇室的米菲波切.
本來應該是萬千寵愛在一身.
但竟然一個消息傳來.
她要和兩位至親永別.
還要跌斷雙腿變成落難的傷殘兒童.
更可怕的是.
跌斷雙腿之後她仍要擔驚受怕.
政權更替原本她皇孫的身份.
反而為她帶來危機.
她每天都要擔心被人挖出來.
或者被人滅口.
她苟且偷生.
去到大概二十歲左右.
米菲波切叩拜說.
你的僕人算什麼.
不過如死狗一般.
景蒙你這樣眷顧.
又是短短一個出場.
一個動作.
一句說話.
一位極度卑微的年輕人就出現在我們眼前.
在他開口之前.
其實攝影師的他已經不理痛楚.
臉伏在地叩拜.
非常非常卑微.
前陣子我在一個街坊聚餐上.
有一位家長跟我分享一件事.
到現在我都不是很消化到.
在疫情期間.
他一位至親因病入院.
住院的時候因為感染COVID-19.
病逝.
在他收到這個消息的時候.
他馬上趕去見他最後一面.
但在匆忙間他仍然在家裡.
拿了一套很體面的衣服.

$^{121}$打算去醫院幫他換上.
但是這個請求當時就被醫院拒絕.
因為當時兵荒馬亂.
他們說感染COVID-19的死者要特別處理.
那位家長聲淚俱下.
他看著穿得很單薄的這位至親.
他苦苦哀求.
他這樣形容.
他說那一刻他真的沒有辦法.
哭了.
他覺得自己很低微也好.
他也不斷求.
但仍然得不到這個機會.
他覺得這就是他一生的遺憾.
不能讓他的至親可以穿得暖.
好好看著離開.
這位家長感到卑微.
因為他在那一刻.
他沒有了一個權力.
他沒有權力做一件認為很簡單的事.
而米飛波切的卑微.
其實同樣來自沒有權力.
他就連拒絕參加一個飯局的權力也沒有.
你們會不會有一些飯局不敢拒絕.
可能是老闆那些.
可能是父母那些.
可能是某一些你覺得影響力很大的人.
但究竟誰令到米飛波切這麼卑微.
甚至形容自己是死狗一樣呢.
我們一起看看薩姆爾記下九章一至五節.
這次邀請姐妹一起讀黑色的字.
請弟兄幫我讀紅色的字.
紅色的字大家都可以留意.
是一些行動.
薩姆爾記下九章一至五節.
大衛說.
「訴羅家 還有剩下的人沒有.
我要因若拿丹的緣故向他示恩」.
訴羅家有一個僕人名叫洗巴.
有人叫他來到大衛那裡.

$^{161}$王對他說「你是洗巴嗎?」.
他說「僕人是」.
王說「訴羅家 還有沒有剩下的人.
我要照神的慈愛恩待他」.
洗巴對王說.
「還有若拿丹的一個兒子.
雙腿是缺的」.
王對他說「他在哪裡?」.
洗巴對王說.
「可能他在羅底巴阿米利的兒子馬吉嘉裡」.
於是大衛和派人去.
從羅底巴阿米利的兒子.
馬吉嘉裡照料他來.
大家剛才都看到紅色字.
弟兄讀的就是.
一而再再而三.
有一個人尋問米菲波切的下落.
這個就是以色列王大衛.
米菲波切顯得這麼卑微.
是因為面對著以色列王.
他只是一個地位低下.
沒有權沒有勢 錢都不多的人.
大衛王其實一個命令就可以將他滅口.
蘇羅加和大衛家爭鬥很多年.
如今大衛作王.
其實新王朝要滅絕舊王朝的人.
是非常常見的事.
就算大衛不出手.
我們在聖經都看到有些小卒.
有時會自作主張.
去殺了他們的後人.
米菲波切的叔叔就是因為這樣而死了.
如果你是米菲波切.
外面到處都說大衛找你啊.
要召你入宮啊.
你未顧回過神就聽到門外.
「葛國」.
你有什麼心情呢.
給你 你開不開門啊.
一路拍攝 你開不開門啊.

$^{201}$可能你臉都青了.
心馬上離一離.
不過大衛一眼就看穿了米菲波切的不安.
馬上派了一粒定心丸給他.
他叫他.
米菲波切 你不用怕.
我必因為你父親若拿丹的緣故向你示恩.
這句你不要懼怕 你不用怕.
我們最近可能都經常聽.
在聖誕節的佳音裡經常聽到.
就好像天使向著驚慌的瑪利亞所說.
瑪利亞 你不用怕.
你在神面前已經蒙恩.
就是一句安慰.
一個恩典降臨的宣佈.
就扭轉了整個緊張的局面.
馬上可以鬆一口氣.
大衛的承諾好像天使報信一樣.
他去了米菲波切那一刻極度的恐懼.
唉 去到這裡我們都可以鬆一口氣.
如果留堂再搞劇場.
可能大家都還沒累完.
再搞劇場我覺得可以用這個故事.
因為米菲波切這個故事非常有戲劇性.
我們再看大衛對他的兩個承諾.
會更加覺得超乎想像.
第一個承諾.
大衛承諾米菲波切可以重新得到.
數羅家的產業.
大衛就向米菲波切說.
把你祖父數羅的一切田地都歸還你.
原本寄人籬下的米菲波切.
可以再次擁有屬於自己的家.
包括埋僕人的照顧.
第二個承諾.
就連作者都跟米菲波切一樣驚訝.
所以他重覆又重覆 重覆又重覆的記述.
大衛說.
你也可以常與我同席吃飯.
你主人的兒子米菲波切.

$^{241}$卻要常與我同席吃飯.
原本進皇宮參與大衛的飯局.
應該是米菲波切一生人裡面.
最不想去吃的那一餐.
但最後米菲波切吃完一餐又一餐.
還當上大衛的飯腳.
於是米菲波切與王同席吃飯.
如王的兒子一樣.
米菲波切住在耶路撒冷.
常與王同席吃飯.
飯腳我們今天可能有很多.
旁邊那個是不是你的腳呢.
我的童宮都是我的腳.
我們在童宮群組一說到吃飯就最精神.
當然其他時候我們都是各按本份.
飯腳在後面.
甚至我們可以沒有什麼原因.
我們就同人同桌吃飯.
在教會也是.
有沒有位置.
我吃個飯就送拜.
飲茶搭桌.
或者打工仔午餐.
我一段時間在荔枝角住.
那個午餐的場景實在太震撼了.
可能觀塘也是.
或者金鐘那邊也是.
去到茶餐廳坐.
很多時候人們以為他們是情侶.
但原來是搭桌的.
可以坐得非常親密.
那些卡位是這樣的.
兩個大男人就要坐在一起.
但這個在古晉東的時候.
是完全想像不到的.
古晉東時期邀請人參與飯局.
有信息的.
坐在一起代表什麼?.
我們是朋友.
代表我們有關係.

$^{281}$甚至宣示主人邀請你來.
我會保護你.
會保護那個客人.
大衛邀請米妃波切參與皇室的飯局.
其實就是要向人表明.
沒有人可以動她一條頭髮.
甚至還有一些大衛掃羅家的敵人.
沒有人可以陷害她.
大衛向米妃波切私恩.
是因為她剛才提到.
曾經和若拿丹納藥.
其實那個藥原本只是要求.
大衛讓他的後人可以生存下來.
很卑微的一個要求.
生存就可以了.
好像我們今天很多人一樣.
生存完就可以了.
大衛做到這樣.
其實超額完成.
甚至開了整個飯局.
他還要放下一些東西.
放下他對掃羅的怨恨.
掃羅曾經迫害他.
甚至和掃羅家長久的爭戰.
那些一切的回憶都要先放下.
他以神的慈愛去愛仇敵.
絕對是主日學教材級.
美好榜樣經典的榜首.
學習做一個私恩的人.
就學大衛吧.
更加有解經家的形容.
在他看來.
這一段的記載.
在整個舊約裡面.
就好像是因典最偉大的勵政.
但是我們不禁會想.
或者有些組員有時都會.
查經的時候會跳機.
我有大衛這麼有才有勢.
和你開一百圍盤賽宴都可以.

$^{321}$不如我們換個角度.
看看米飛波切.
在米飛波切的經歷裡面看.
一天還沒死.
一天都有人有機會打救我.
我在米飛波切大概這個年紀的時候.
二十多歲.
不是很久.
時日怎麼計算就唯有我知道.
在那個年紀.
我家突然間有些經濟困難.
但又剛剛入大學.
在大學第一個城市.
我要非常非常節儉才能生存.
那時候七仔一個叫龍鳳茶樓.
有沒有人知道是什麼.
有人點頭.
我就會吃兩餐.
一隻鳳爪午餐.
另一隻就晚上.
很多大學的聯義飯局.
很想去但都不能去.
想入數連入會費都交不到.
當時我是初信主.
剛剛進去聽福音.
信了一會.
但為什麼我不會像米飛波切那樣.
有一個大恩人將我的生命.
或者將我的生活完全扭轉.
有一個大恩人去幫我呢.
如果我們看聖經.
只是關注在人物的經歷.
他的一些好行為.
我們很容易會對上帝失望.
因為我們的生命.
不會完全跟聖經所出現的劇本一樣.
甚至有時候會覺得很相反.
而大衛行在米飛波切身上的事.
其實有一個重要的焦點.
大衛說.

$^{361}$我要以神的慈愛去待他.
就這樣聽可能我們經常都會聽到.
但這句話是有重量的.
這個焦點在神的慈愛.
那個重量就是.
他承載著大衛很多很多的過去.
他的高高低低.
他的一切的經歷.
我們都一起跟大衛回顧一下.
大衛在面對迫害的時候.
面對那位素羅.
同時那位素羅又是上帝所高納的素羅.
他很矛盾.
他有這樣的反應.
然後大衛也起來.
從洞裡出去呼喚素羅說.
我主 我王.
素羅回頭觀看.
大衛就屈身.
便伏於地下拜.
以色列王出來要尋找誰呢?.
你要追趕誰呢?.
不過是一條死狗.
一隻跳蚤而已.
原來大衛曾經都覺得自己好像一條死狗.
上帝的說話.
透過先知傳達去到大衛的耳邊.
萬君子耶和華如此說.
我從羊圈中將你召來.
叫你不再目放羊群.
立你作我百姓以色列的君王.
主耶和華 我是誰?.
大衛更加向上帝很真誠地回應.
主耶和華 我是誰?.
我的家算什麼?.
你竟帶領我到這地步呢?.
主耶和華 這在你眼中還看為少.
你又說道你僕人的家將來的情況.
主耶和華 這是人的常理嗎?.
原來尊貴的以色列王大衛.

$^{401}$都有著米菲波切的影子.
他曾經都非常非常卑微.
形容自己好像一條死狗.
當大衛與米菲波切相遇的一瞬間.
一幕一幕的回憶湧上大衛的心頭.
大衛就好像在看自己的過去.
看到自己的影子.
不知道你生命中有沒有遇過這樣的人呢?.
看到一些別人的經歷.
會看到自己的影子.
過去曾經都是如此度過.
大衛曾經都向上帝高立的訴羅王.
苦服跪拜.
他亦自貶為僕人.
自稱死狗.
面對超乎想像的恩典的時候.
他同樣感到驚訝.
從前約拿丹向大衛示恩.
如今大衛向米菲波切示恩.
因為他先經歷到上帝的恩典.
然後學習像上帝那樣.
施予恩典給人.
縱然貴為以色列王.
他仍然很清楚.
他所擁有的.
他今天坐的位置不是屬於他.
主權在上帝手裡.
就算是他請米菲波切吃的這一餐.
原本他是沒有機會吃.
從前訴羅王設宴.
大衛不能夠出席.
為什麼呢?.
因為訴羅想殺他.
他是得到上帝的拯救.
才有機會在一個王室裡面吃飯.
還可以宴請別人.
邀請米菲波切.
不過當大衛不再至今卑微的時候.
都是他失敗的時候.
嘉Sir上次提到大衛的至高.

$^{441}$即使大衛是上帝所高納的王.
但他沒有真正的王權.
而他企圖竊取上帝的權力.
當他不再卑微.
就是他失敗的時候.
大衛和米菲波切的飯局.
我們不能真實地參與.
但大衛的舉動就像一個預演.
預演著耶穌邀請我們的筵席.
一開始提到.
今年出席了很多飯局.
大部分都很開心.
外向的我都很喜歡去飯局.
通常我老公都說又有嗎?.
又去哪裡?.
這個誰?.
這個三姑還是六婆?.
我經常去飯局.
是不累的.
一點都不累.
但有一餐我完全不想去.
是沒有一種期待.
甚至心裡有點抗拒.
那一餐是一個火葬禮後的安慰飯.
我自己有一位年齡相若的傳道人.
都是朋友.
比我更早畢業回天家.
我還在回想的時候.
記得最後和他吃的一餐是在元朗吃雞煲.
當時還期待著下一次.
但就沒有下一次.
我一想起以後不能再一起聚餐.
在那個時刻我就覺得很遺憾.
還有心裡有一種說不出來的傷感.
但上帝給我一個盼望.
就是在未來我和他會再次在神國的筵席裡面共聚.
耶穌常常用筵席的比喻去講解神國的來臨.
至今卑微的才有份參與神國的筵席.
就算他是一個被社會遺棄的人.
就算他是劫腿的人.

$^{481}$不單止如此.
耶穌更加經常參與飯局.
成為罪人的飯腳.
他用同桌吃飯這個行動.
告訴那些所謂宗教領袖.
他會擁抱每一個和他同桌吃飯的人.
每一個願意悔改的罪人.
都可以和他一起同席.
向主卑微的人會得到接納.
得到恩典.
並有能力向人施恩.
甚至向被棄者擺設筵席.
米菲波切.
心存感恩.
是因為卑微的她遇見大衛.
大衛懂得施恩.
是因為卑微的她遇見上帝.
無論我們每一個到年尾.
我們去數算恩典.
去感恩.
心挖苦.
望著前面.
2023年.
我很想再用自己的生命服侍人.
向人施與恩典.
我們只有一個原因.
就是因為卑微的我們遇見上帝.
我們才可以感恩.
我們才有能力施恩.
讓我們一同去禱告.
慈愛手約的天父.
2022年可能是很多人一生裡.
所經歷最複雜的一年.
實在太多悲喜交集的場面.
我們很需要天父你的慈愛.
繼續和我們同在.
我們甚或有一種矛盾的心情.
不想和這一年說再見.
因為在這一年裡.
有一些人真真實實離開了我們.

$^{521}$甚或是我們.
舊日的香港.
都有一些.
面貌已經不再一樣.
但同時我們又在困境裡.
見到上帝給我們的新方向.
給我們有能力去面對.
給我們仍然可以有一個群體.
共聚彼此支持.
向前.
無論是開始.
無論是終結.
我們只願繼續跟隨上帝.
求你去帶領.
神啊我們知道無論我們在哪個崗位.
去到上帝面前是同樣.
求你給我們這個時間.
可以好好去思想.
好好面對上帝.
我們無法誇自己的成就.
軟弱.
能力.
無法誇任何我們得到的東西.
因為一切都是屬上帝手.
這段時間我們透過一首詩歌.
我們一起去聆聽.
成為我們的禱告.
仍未懂得低微.
就繼續謙卑禱告.
仍未懂得低微.
願能放下自身.
崛強.
我們也有一點安靜的時間.
我們回顧一下過去一年.
你想一下上帝怎樣帶領你.
上帝怎樣在你的掙扎裡跟你同在.
有什麼是上帝賜予恩典給你.
有什麼是當你失去的時候.
上帝幫助你.
帶領你經過.

$^{561}$讓我們一一去數算.
上帝給我們的恩典.
英丁就是帶領我們那一位.
明天就是2023年的開始.
我們也要禱告.
祈求上帝幫助我們.
在未來一切的決定裡.
我們也認定主權屬於上帝.
讓我們一同去為自己為流淌.
我們能夠成為謙卑跟從主的門徒.
謙卑跟從主的教會去祈禱.
我們一起去禱告.
香港人的香港,香港人的香港..
\newpage



\section{撒迦利亞書 2:6-8}
\label{sec:rQtMXvUNeaE}
\textbf{【網上崇拜】What is the apple of His eye?|撒迦利亞書2\_6-8;8\_4-6|20230107 [rQtMXvUNeaE]}
\newline
\newline
連結: \href{https://youtube.com/watch?v=rQtMXvUNeaE}{\texttt{ https://youtube.com/watch?v=rQtMXvUNeaE}} ~~~~ 語音日期: 2023-01-07 
\newline
\newline
\hyperref[sec:0FV84blFd3M]{\small{< < < PREV SERMON < < <}}
~
\hyperref[sec:index_chronic]{\small{[返順時目]}}
~
\hyperref[sec:index_scriptual]{\small{[返順卷目]}}
~
\hyperref[sec:gptfSrlmqo8]{\small{> > > NEXT SERMON > > >}}
\newline
\newline
撒迦利亞書 2:6-8
\newline
\begin{longtable}{cl}
\hline
\hline
章節 & 經文 (和合本修訂版)\\
\hline
2:6 & \begin{tabularx}{0.7\textwidth}{X} 耶和華說:「來,來!你們要從北方之地逃回;因我曾把你們分散到天的四方。這是耶和華說的。」 \end{tabularx} \\ \\ \relax
2:7 & \begin{tabularx}{0.7\textwidth}{X} 來!住巴比倫的錫安百姓啊,逃吧! \end{tabularx} \\ \\ \relax
2:8 & \begin{tabularx}{0.7\textwidth}{X} 萬軍之耶和華在顯出榮耀之後,差遣我到擄掠你們的列國那裡,他如此說:「碰你們的就是碰他自己眼中的瞳人。 \end{tabularx} \\ \\
[1ex]
\hline
\hline
\end{longtable}
$^{1}$鄧姐妹平安.
這是我們劉棠2023年第一個崇拜.
祝大家新年快樂.
2023年我自己是充滿期盼和樂觀的.
祝我幾天祈禱.
盼望劉棠的鄧姊妹.
都能夠在2023年裡面.
經歷一個滿有盼望,重新出發,平安,快樂的一年.
這些不是客套的說話.
過去幾年我們都覺得不容易.
2019年我們都經歷了人生裡面最難忘的一年.
然後2020年是痛苦和流淚的延續.
2021年都沒有什麼特別的感覺.
想快點skip.
2022年都沒有什麼特別的氣息.
不過踏入2023年.
我都似乎感覺到世界,社會來到2023年的時候.
情況好像出現了一點點的改變.
不是客觀環境的突破.
這個社會依然都是差不多.
而世界的人似乎我們改變了一些東西.
有些人離開了一些困局.
有些人領悟到一些道理.
有些人見到一些事情.
能夠成長的成長.
能夠找到一些出路的.
找到一些出路.
雖然這些出路未必是一片光明的東西.
但我們最少每個人都嘗試踏入.
走進我們2023年.
所以2023年我是樂觀的.
因為能夠改變的不是這個世界而是我們.
當我們成長了,改變了,強大了的時候.
同時間這個世界都不能不說是被改變.
所以這個原因我們面對2023年.
Full Church的第一個主題是漢建.
漢建Vision.
不知道大家有沒有聽過類似的說話.
我們的視野決定我們的高度.
不知道大家明不明白這句說話的意思.

$^{41}$我們的視野決定我們的高度.
當然可能聽過這句說話的相反.
高度又決定視野.
所謂高度決定視野其實都是一些物理原則.
你走高一點就望遠一點.
是玉瓊千里目的朋友.
視野決定高度是一個人生的哲理.
在我們的人生裡面.
我們作為基督徒.
我們作為天父上帝的兒女.
我們的眼光正在定義我們的高度.
你看得有多遠.
我們的生命就可以有多高.
你看得有多闊.
我們的生命就可以有多廣闊.
漢建似乎是一切事物的第一步.
所以面對2023年.
我們懂得如何去看世界.
在一切行動和動機以先.
我們懂得如何去看世界.
我們思考我們的信仰和視野的問題.
所以我們講到的題目是.
What is the apple of his eye?.
正如這個海報給了你一些提示.
這篇道的起源就是《絕望研究所》.
其中一個海報或是一個蘋果的袋.
其實《絕望研究所》這套劇中.
其中一個很重要的開始.
其實就是馬丁路德一句名言.
我都講過在劇中.
如果明天是世界末日.
我都會選擇將蘋果樹種下來.
不知道為什麼路德會講蘋果樹.
但總之就是蘋果樹.
於是劇的原故.
我們從一棵蘋果樹開始.
然後就有He-Man那首《蘋果不見了》這首歌.
然後出現了I am the apple of his eye這個袋.
當我們He-Man去找這句口號.
他打算去印的時候.

$^{81}$我就再一次去接觸這句經文.
就是上帝看顧他如同保護他眼中的同人.
這句經文.
我發現原來我信主公國人都沒有認真去研究過.
眼中的同人這個經文.
所以就選擇在今天崇拜的過程中.
我們就一起來思考一下.
在年之初的過程中.
我們去認識一下這句經文.
這個face對我們信仰的意義.
我們一起祈禱.
祝你和你去幫助我們.
讓我們在新一年裡.
我們去尋求看見你.
看見你所看見的東西.
求你幫助我們.
雖然我們現在是閉上雙眼.
但我們卻是能夠看得見.
我們需要看見的事情.
求你去眷顧著我們一會兒講道的時間.
你自己的說話.
你的聖靈在當中.
在我們每一個流唐的頂節目當中.
無論他在現場紅磡的地方.
或者是在不同世界的角落裡.
我們知道聖靈你包圍著這個世界.
求你這樣的力度去一起.
讓我們這個群體能夠同樣的力度去看見.
奉傳明求 阿們.
What is the apple of his eye?.
我們就先講一下這個apple of his eye的典故.
apple of his eye這個典故.
我本的中文其實就叫眼中的瞳人.
如果你翻舊翻了很久的話.
你對聖經有些認識的話.
你大概都聽過.
聖經裡面眼中的瞳人這句話.
上帝看顧他如同保護眼中的瞳人.
這句經文.
所以你會看到.

$^{121}$其實經文裡面這個face.
apple of his eye.
是出現過五次這麼多.
第一段就在生命記 沙爾章裡面.
然後說 遇見他在曠野 荒涼野獸.
恐叫之地.
就環繞他 看顧他 保護他.
如同保護眼中的瞳人.
第二段是詩篇十七篇第八節.
是一個禱告.
詩篇是這樣的.
來到祈禱.
求你保護我 如同保護眼中的瞳人.
將我隱藏在翅膀的陰下.
第三段是針研的一段說話.
針研第七章.
遵守我的命令就得存活.
保守我的法則就好像保守眼中的瞳人.
第四段.
耶利米亞哀歌一段的哀求.
哀歌這樣說.
錫安的城牆啊 願你流淚如何.
晝夜不息 願你眼中的瞳人流淚不止.
很悲傷的一個經文.
最後一段就是我們今天看的經文.
撒加利亞書第二章第八節.
萬軍之尼和華說.
在你顯出榮耀之後.
猜險我去懲罰那攏略你們的列國.
摸你們的就是摸我眼中的瞳人.
在這五段經文裡面.
英文其實都是翻譯成.
Apple of his eye.
一個這樣的字眼.
不過不知道大家知不知道.
Apple of his eye為何叫Apple呢.
為何不叫Orange.
為何不叫Banana.
這麼好笑.
Apple of his eye是一個英文的理語.

$^{161}$是一句成語.
意思是眼裡面一個極度喜愛.
很愛慕的人.
眼中比任何人更加寶貴的一位.
就是叫Apple of his eye.
這個英文的理語.
最早文獻出現在第九世紀.
一個很古典的英文裡面.
一個基督教的經典裡面.
出現一個Face.
後來十七世紀的莎士比亞.
這個仲夏夜之夢.
都用了Apple of his eye在故事裡面.
後來英文聖經.
King James Version.
就會將所有剛才的經文.
翻譯成Apple of his eye.
所以是一個英文獨有的現象.
中文沒有Apple這個字.
中文有的,但聖經沒有.
所以我們不知道.
不知道為何會叫Apple.
總之就是叫Apple.
不過我們知道.
原來在舊約的希伯來文聖經裡.
原文其實不是Apple這個字.
原文裡面的字是解作瞳孔的意思.
就是我們眼球.
大家記得以前的Form 2那些嗎.
那些Science.
我讀文科,我只讀那些.
眼球,然後就是中間那個叫做.
叫做Iris,然後就是中間那個叫做.
叫做Pupil,中間那個洞就叫做瞳孔.
不過原來希伯來文.
或者本國希臘文裡面.
那個字其實直譯就解作女兒.
那個字,Doctor女兒.
那個字,或者是小孕兒.
這樣的意思.

$^{201}$所以其實不知道有沒有關係.
應該是有關係的.
即是瞳孔的意思就是Pupil.
所以Pupil解作學生的意思.
所以這個瞳孔和學生和人兒.
應該是有些關係的.
所以這個跟聖經裡面的字是相似的.
所以你就明白.
無論是蘋果也好.
叫做瞳孔也好.
叫做學生也好.
或者是女兒也好.
其實都是這樣的東西.
不過和本就很特別.
中文聖經就翻譯了一個很美麗的字眼.
就叫做瞳人這個字.
一個非常有意思的字眼.
即是中文沒有西方英文那個Apple.
他就把它翻譯成眼中的瞳人.
一個非常美麗的字眼.
文言文聖經.
即是一百年前的版本聖經.
就翻譯成眼中的瞳子.
或者是《論中本》翻譯成眼中之女.
很有型.
眼中之女.
所以有這麼多的詞彙.
原來古代中國早就有瞳人這個字.
即是瞳人就是瞳孔.
在唐朝的道家唐兒歌裡面寫.
骨重神寒天妙氣.
一雙瞳人剪秋水.
宋代的《贈女觀唱》的詩裡面也這樣說.
瞳人剪水腰如簇.
一幅烏紗果寒玉.
所以中國原來就只有十八瞳人.
瞳孔是有的.
一直有瞳人這個字眼.
所以無論是聖經或是中國的唐詩也好.
古代人一直都將瞳孔.

$^{241}$即是眼球裡面的洞洞.
解作一個人或蘋果的意思.
這個就是Apple of the eye的一些資料.
一些短故.
明白Apple of the eye的短故之後.
今天我們就開始去說.
聖經裡面其中一節的蘋果.
即是撒加利亞書裡面的Apple of the eye.
先說說撒加利亞書的一些簡單背景.
撒加利亞書是一本比較後期的先知書.
一本是紀錄了以色列人被腦流亡之後.
嘗試去歸回一個小先知書.
先知撒加利亞對流亡海外的以色列人說話.
亦對住一些存留在耶路撒冷鄉下的漁民.
這些剩下的人去說話.
公元5,6年之後.
耶路撒冷滅亡.
以色列人被腦.
其實分了幾批不同的人.
一批是流亡海外的以色列人.
被腦的人.
一些是遷移到國外移民的以色列人.
但仍然有一些是存留在耶路撒冷裡面.
仍然在耶路撒冷被毀的廢墟裡面.
仍然生存下去的以色列人.
情況今天香港差不多.
有些留下的 有些是離開了.
因此撒加利亞書和哈蓋書.
他們都有同樣的目的.
就是要去代表著耶和華上帝.
去吩咐一班身處異地的以色列人.
和一班殘留在耶路撒冷廢墟的以色列人.
來說話.
吩咐他們要重返耶路撒冷.
並且重建耶路撒冷城.
恢復他們的聖殿.
可能大家不知道.
原來哈蓋書和撒加利亞書.
一到八章是一個系列的書卷.
哈蓋書和撒加利亞書一至八章.

$^{281}$是一個系列.
撒加利亞書是一個強調漢見.
視野的少先之書.
撒加利亞先知強調耶和華上帝的漢見.
藉著撒加利亞先知的漢見.
去表達出耶和華上帝漢見耶路撒冷的願景.
所以撒加利亞書頭八章.
有很多很多的倚仗.
一些視野 一些漢見.
去表達上帝自己的漢見.
漢見就是倚仗.
vision.
是一樣意思的字.
但我想說的是.
其實所謂的漢見.
或者vision 或者倚仗.
其實是一些不容易消化的東西.
一些難以置信的東西.
一些尚未發生的東西.
撒加利亞書說到流亡海外的以色列人.
先知跟他們說.
去說服 去遊說 去窺會.
其實是純粹將將來的漢見.
一個尚未發生 還未出現的事情.
去說出來.
所以現實的景象.
和所謂的漢見.
兩者的差距其實是甚大的.
說和真實那一套是相差甚遠.
就等於加Sir無端端跑出來.
因為平時說到我講得太多.
先回到原位.
加Sir無端端回到原位.
在youtube呼籲大家.
在這兩年 特別是這兩年.
移民海外的頂梓梅.
在這裡說.
遊說他們回來.
回香港.
大聲說香港現在沒事了.

$^{321}$香港現在一切正常.
恢復我們的自由.
珍寶村又怎樣的 說出來.
大家回來回流吧 重建香港吧.
甚至作激一點 玩什麼.
大陸都ok的.
大家回深圳玩 沒問題的.
現在大陸很安全 過得很健康.
沒事了 回來吧.
海外頂梓梅不是傻的.
每天都看到新聞.
他知道香港現在發生什麼事.
知道大陸發生什麼事.
所以就算加Sir真是一個.
耶和華的先知也好.
他代表著耶和華.
他講話也好.
他講的每句話都是屬實也好.
他所講的話其實都難以置信.
難以說服別人.
加Sir怎麼說都好.
都不容易拉得動.
在海外的那班人.
一班嘗試已經開始落地生根的人.
回到香港裡面.
試問你在英國裡面.
很不容易找到地方住.
找到工作.
開始穩定的時候.
現在還可以過個星期.
看英超的時候.
他回香港吧.
很難的.
加拿大頂梓梅也一樣.
開始慢慢落地.
突然要回來.
其實一點都不容易的事情.
我講的不是具體的移民.
不是那邊好一點.
現在這裡不好的意思.

$^{361}$我想講的是.
撒加尼亞書大概就是面對著這樣的情景.
耶路撒冷其實是一個.
頹垣敗瓦 烏煙瘴氣.
一個廢墟的景象.
撒加尼亞就是嘗試去說服一班流亡海外.
已經慢慢落地生根的人.
追到老婆生兒子的人.
回流回去耶路撒冷裡面.
所以在這個現實景象和異象漢建.
那個相差甚遠的前提之下.
上帝就補充了一句這樣的說話.
希望能夠呼籲流亡海外的少年們能夠安心.
不用怕 回來重建 沒問題.
這個說話 今天想講這個說話.
請聽我讀出 撒加尼亞書第二章1-6節.
我從前分散你們在天的四方.
現在你們要從北方之地逃回.
這是耶和華說的.
與北北鄰人同住的錫安文娜.
應當逃脫.
萬君子耶和華說.
在顯出榮耀之後.
差遣我去懲罰那擄掠你們的列國.
摸你們的就是摸我眼中的同人.
這裡上帝是要求流亡海外的這些人歸回.
叫他們不用怕.
因為耶和華上帝會保護他們.
因為他們就是耶和華眼中的同人.
Apple of his eye.
甚至撒加尼亞書給了一個非常生動有趣的描述.
耶和華說摸你們的就是摸我眼中的同人.
就是這句說話.
聖經文字的意思是什麼.
你們回來吧 你們不用怕.
你們可以回流.
因為我會幫你懲罰那些惡人.
總之誰摸你們 誰搞你們.
就是摸我的眼.
一句有點奇怪 但很有說服力的說話.

$^{401}$誰搞你們 就是在搞我的眼.
不是髒話.
誰搞你們 就是在搞我的眼.
不知道大家記不記得這句說話的氣場.
誰搞你們 就是在搞萬軍之耶和華的眼.
上帝說誰搞你們 就是在搞我的眼.
因為你們就是上帝眼中的將來.
任何人要傷害你 就是在傷害我的眼.
我的purple.
聖體書正是要表達出你們不需要害怕將來.
雖然你們看不到將來.
但你們就是我眼中的同人.
誰搞你們 就是在搞我的瞳孔.
這是一個切膚之痛的愛護.
上星期日 1月1日.
我和女兒去了屯門友愛村玩滑板 踩滑板.
因為聖誕節 福壽很忙 我沒有時間陪她.
上星期日完課 兩父女拍拖.
去友愛村玩X-game場.
不知道大家有沒有去過.
屯門有一個非常潮的滑板X-game場.
在卡牌上 有三層.
一層是踩滑板 一層是玩BMX.
去到那裡 場上沒有人.
我和我女兒就包了整個場地玩.
很開心地玩了一個小時.
差不多離開的時候.
我女兒就說想玩多十分鐘.
就玩了 怎知道很後悔.
一玩就出事了 樂極生悲.
差不多離開的時候.
滑板失腳 失平衡.
滑板就飛走了.
衝去我女兒的方向.
就令她失平衡 跌倒在地上.
我馬上跑過去.
我見她神色嚴重 就知道大件事了.
她沒有哭 但她一脫下口罩.
我發現她的口罩是她兩隻門牙的三分之一.
在口罩上.

$^{441}$兩隻門牙的前端沒有了.
跌倒了 整個口都是血.
那一刻我真的非常非常後悔.
很貪心 很亂.
還要是痕跡.
我就馬上走了.
在柳萼村買了一瓶冰水給她清洗.
在她的車裡 我叫自己冷靜.
馬上上網Google打電話 看怎樣做.
因為是公假 牙醫沒有開.
幸好以前有個大同學是做牙醫的.
馬上WhatsApp他 問他怎樣辦.
他說都不嚴重 總之星期二看就OK.
不用看急症.
最怕是那晚吃飯 她已經Settle down了.
還說很痛 吃飯後突然脫牙.
不過原來不是那隻 是她的乳豉.
嚇死我 突然脫了乳豉 三條牙都有事.
嚇到我 後來等到星期二.
我就帶她去看牙醫 帶我去同學那裡看.
極兼波之演 終於補了一隻牙.
回復她美麗的面貌.
這張相片是我在牙醫門口等的時候幫她拍.
拍給她媽媽看 沒事 沒事.
我想說當我 很美麗 是不是.
當我見到她的笑容的時候.
我get到 我知道 我發現.
她就是我眼中的瞳孕.
She is the apple of my eye.
我的眼中之女.
我明白為什麼舊約會把眼中的瞳孕.
直譯叫做The daughter of the eye.
我眼中的女兒.
那種心肝寶貴如珠如寶 萬千寵愛的感覺.
我的女兒就是我眼中的瞳孕.
當我 當你 或者說.
當你眼中的瞳孕受到傷害的時候.
你寧願受傷的是你自己.
我想你作為父母的 你也感受到這個感覺.
如果你子女崩牙的話.

$^{481}$你寧願自己是崩牙的那個.
你也不想他對門牙有事.
所以這個例子我跟同工談過一輪.
我也是說 你知道最近姜濤.
最近沒有以前「造星」的時候那麼瘦.
網上也有不少人很無聊.
有人嘗試攻擊他 去燒他.
我很明白這種的.
我稍微明白了 明白這種的感受.
因為我也試過胖過.
很多時候需要關心他.
不知道喜歡姜濤的人會不會這樣祈禱.
主啊 求你不要叫姜濤胖.
他已經有很多壓力了.
我寧願胖的是我自己.
我認真的 不是說笑.
有沒有試過這樣祈禱.
因為有很多壓力 很多輿論.
很多姜濤粉絲寧願自己胖.
也不想這樣.
我想說 如果姜濤是你眼中真正的同人的話.
你會這樣祈禱.
主啊 我寧願自己胖.
也不想姜濤胖.
他已經很辛苦了 求你去幫助他.
如果你真的這樣祈禱的話.
你真的很愛姜濤.
你寧願自己 可能是他媽媽.
或者是他親人都會這樣去祈禱.
不想他有事.
寧願自己承擔這麼多的壓力.
這麼多的傷害 這麼多的重擔.
所以說 如果我們作為一個普普通通的人.
都會有眼中的同人的時候.
何況我們天上的父上帝.
他看待我們的兒女.
他看待他的兒女.
這更加不是一個心肝寶貝 如珠如寶.
萬千寵愛的事情.
誰在騷擾他 其實是在騷擾我的眼睛.

$^{521}$希望大家能夠明白.
眼中同人這種寶貴.
無論是你身邊愛護的人.
或者是你很喜歡的偶像也好.
嘗試知道這樣的一種愛.
更加知道在2023年裡面.
上帝正正都是待我們一樣.
如同一個眼中的同人.
我們被傷害 等同於上帝自己被傷害一樣.
不過今天我們的篇道不是停留在這一點.
當然我們的篇道可以停留在這一點.
成為一篇很有安慰的訊息.
告訴大家我們就是上帝眼中的同人.
大家非常感到安慰地面對2023年.
但我並不單單想說到這一點.
2023年我們是認定我們是上帝眼中的同人.
不過我們更加要知道的是.
同一雙上帝的眼睛.
撒加利亞書所說的上帝的視野.
不單單停留在這一點.
聽我讀另一段經文.
同樣是撒加利亞書的經文.
這一段記載了耶和華雙眼的經文.
撒加利亞書第八章四到六節.
萬貫之言為此說.
將來必有年老的男女坐在耶和華雙眼上.
因為年紀老邁就手拿拐杖.
城中街上必滿有男孩女孩玩耍.
萬貫之言為此說.
到那日這是在如勝的眼中看為稀奇.
在我眼中也看為稀奇嗎.
這是萬貫之言說的.
而我想看見年老的男女坐在撒加利亞.
在耶路撒冷街上.
看見街中滿是男孩女孩玩耍.
這是耶路撒冷將來和平的景象.
這是上帝從廢墟中看見和平的景象.
所謂的視野 所謂的意象 所謂的看見.
正正就是這樣的視野.
一個尚未呈現 尚未發生.

$^{561}$在任何一個行動以先 首先出現的必要因素.
不過值得留意的是第六節.
對於這個視野有關眼睛的問題.
耶和華提出一個值得我們去深思的問題.
上帝去反問一班耶路撒冷剩下來的人.
他說 這是在愚聖的民中看為稀奇.
在我眼中也看為稀奇嗎.
這句在我眼中看為稀奇嗎.
實在是可圈可點的說話.
這句話的翻譯意思是甚麼呢.
這些事情你沒見過嗎.
你覺得很陌生嗎 你覺得很出奇嗎.
你以為我和你一樣會一般見識.
從來都沒見過.
你覺得很陌生很出奇嗎.
你不明白嗎.
這句話正正道出了.
以色列民和耶和華上帝兩者眼光的差異.
以色列人的原文見到的.
只不過是一個摧毀了的廢墟.
但是上帝在這個頹垣敗瓦的地方.
見到一個和平的國度.
一件很諷刺的事是.
以色列人作為耶和華上帝眼中的同人.
自己都見不到耶和華上帝所看見的.
更諷刺的是.
這群人其實不是看不到的.
他們不是看不到東西.
他們不是盲的.
這群人不是盲人.
他們是有眼睛.
可以見到很多東西的人.
不過他們見到很多很多很多東西.
他們見不到他們應該要見到的東西.
經文告訴我們.
以色列人不是看不見而是看不到.
當我預備經文的時候.
我認識一個冊分書.
見到一個早期的教父.
一個很陌生的教父.

$^{601}$叫做萬人滴滴莫斯.
即是Didimous de Blin.
一個盲的基督徒.
他是盲的.
但是他寫了很多書.
其中一本就寫了撒加利亞書的Commentary.
一個色經.
他說.
一個盲的人去思想撒加利亞書的時候.
他說撒加利亞其實是一個.
心靈的眼睛裡面被Enlighten了的人.
他能夠見到很多很多他們應該見到的東西.
一個失明的人.
一個盲的人.
可能比我們雙目見全的人.
見得更加好更加多.
不知道你有沒有同感.
在這個年頭我們可以看到很多很多東西.
但是我們看得越多東西.
其實越看得越少東西.
什麼意思呢.
這個年代是越來越.
不知道大家開始看Facebook和IG.
發覺這個世界越來越TikTok化.
現在很討人厭.
YouTube Facebook全部都是很短的片.
是垃圾.
很多都是很無聊的.
那你就三個小時了.
在床上滾滾滾.
我們有很多東西可以看到.
但是很多東西都看不到.
我們越多東西可以看到.
就越多重要的東西看不到.
怎樣叫一個人看不見.
你可以蒙掉他的雙眼.
你可以挖他的眼出來.
可以關到哪裡.
但是最簡單的方法是什麼.
就是讓他看得更加多的東西.

$^{641}$看到他看不到他應該要看的東西.
漢見生反詞不是漢不見.
漢見生反詞是忽略.
明明在你面前.
你卻有眼無珠.
眼大看個籠.
視而不見.
所謂視而不見.
就是明明篤口篤鼻的存在.
都變得不存在.
今次等於面對2023年.
我們嘗試學習.
用上帝的眼光去看事情.
保持這個vision.
保持這個意象.
看一些尚未出現的東西.
很多人只是看一些已經出現的東西.
一些過去的東西.
一些已經成為事實的東西.
我們作為基督徒.
特別作為future基督徒.
我們要看一些尚未出現的事情.
今天這裡的崇拜應該是爆滿的.
現場崇拜是爆滿的.
我們沒有試過坐到.
現在是不是廣播都有人.
沒有吧.
總之是三間房都爆滿.
我們從披堂回來這裡的時候.
其實我們一早vision.
看見很想這個場坐到今天這樣.
希望可以坐到爆滿.
在我四年前.
今天是一月.
四年前一月十九日.
就是我們留堂第一次崇拜.
四年.
四年前當我創立留堂的時候.
其實在堂上看到大家的景象.
四年前已經一早出現了.

$^{681}$就因為這個圖畫.
我在台上見到台下的人.
這個景象.
我就開始去想怎樣開始留堂.
vision.
正正就是一些尚未出現.
你嘗試去看見一些尚未出現的事情.
你只是去看一些出現了的東西.
一些廢的東西.
看很多都沒有用.
作為天上父神眼中的同仁.
看見是一個倫理問題.
我們應該看些什麼.
我們應該要看些什麼.
不是看多少.
看多快.
而是應該看些什麼.
嘗試不要看一些我們不應該看的東西.
偏見.
成見.
短見.
阻礙我們生命裡面去看這個世界的東西.
2023年.
留堂將會有很多很多新的事情發生.
一些暫時尚未出現的事情.
一些暫時看不見的事情.
我們卻要用上述的眼光.
去看這些東西.
這就是我們所謂的意象.
vision.
我們將會.
我們要向一些未認識主的年青人.
去全陽盼望.
所以計劃.
我們今年裡面會做一些大專心的事工.
一些初職的事工.
去帶一些未信主的人.
能夠認識耶穌去盼望.
特別是一些年輕人.
我們要去建立一個小朋友群體.

$^{721}$讓大家等姐妹的小朋友都能夠開始.
能夠回到教會.
我們要更加關心很多社會上的人.
看見他們的需要.
他們都是上帝眼中的同人.
我們首先要看見他們.
從而去幫助他們.
讓我打開一盤牌.
其實看見這個題目.
主要是想講社慣.
我們很希望留堂等姐妹.
更加參與我們的社會關懷工作.
如果看過我們《留膠片》的話.
社關是我們很重要的一部分.
我們希望我們今年真的做多一些.
我們社關的工作.
看見一些人的需要.
所以發現這個月裡面.
考業裡面Amy會經常上台.
不斷去講很多from的事情.
目的只有一個.
就是讓你看見社區的需要.
讓我們全家人都知道我們可以做些什麼.
等姐妹為自己祈禱.
2023年不單單是叫人對你零眼相看.
刮目相看.
你更加要對這個世界零眼相看.
刮目相看.
看見天父上帝要你看見.
我們祈禱.
祖力強尼幫助我們.
讓我們看見我們應該看見的事情.
因為你一直都看顧著我們.
我們深信我們是你眼中的同人.
我們更加深信更加多的人.
被忽略的人是你眼中的同人.
願意你幫助我們.
我們能夠看見他們的需要.
看見一些尚未發生的新事.
我們2023年裡面.

$^{761}$這個群體我們留堂教會是強大的.
因為我們看見有信心.
去嘗試達到一些尚未達到的東西.
求主你幫助我們.
我們今天的群體.
今年繼續要成長.
更加憑著信心.
去看見更多我們自己的作為.
求主你這樣建立我們.
奉主命求.
阿門.
\newpage



\section{申命記 34:1-12-20230114}
\label{sec:gptfSrlmqo8}
\textbf{【網上崇拜】上帝視角|申命記34\_1-12|20230114 [gptfSrlmqo8]}
\newline
\newline
連結: \href{https://youtube.com/watch?v=gptfSrlmqo8}{\texttt{ https://youtube.com/watch?v=gptfSrlmqo8}} ~~~~ 語音日期: 2023-01-14 
\newline
\newline
\hyperref[sec:rQtMXvUNeaE]{\small{< < < PREV SERMON < < <}}
~
\hyperref[sec:index_chronic]{\small{[返順時目]}}
~
\hyperref[sec:index_scriptual]{\small{[返順卷目]}}
~
\hyperref[sec:RPsjvP8W4CE]{\small{> > > NEXT SERMON > > >}}
\newline
\newline
申命記 34:1-12-20230114
\newline
\begin{longtable}{cl}
\hline
\hline
章節 & 經文 (和合本修訂版)\\
\hline
34:1 & \begin{tabularx}{0.7\textwidth}{X} 摩西從摩押平原登上尼波山,到了耶利哥對面的毗斯迦山頂。耶和華把全地指給他看:從基列到但, \end{tabularx} \\ \\ \relax
34:2 & \begin{tabularx}{0.7\textwidth}{X} 拿弗他利全地,以法蓮、瑪拿西的地,猶大全地直到西邊的海, \end{tabularx} \\ \\ \relax
34:3 & \begin{tabularx}{0.7\textwidth}{X} 尼革夫,從棕樹城耶利哥的平原到瑣珥。 \end{tabularx} \\ \\ \relax
34:4 & \begin{tabularx}{0.7\textwidth}{X} 耶和華對他說:「這就是我向亞伯拉罕、以撒、雅各起誓應許之地,說:『我必將這地賜給你的後裔。』現在我使你親眼看見了,你卻不得過到那裡去。」 \end{tabularx} \\ \\ \relax
34:5 & \begin{tabularx}{0.7\textwidth}{X} 於是耶和華的僕人摩西死在摩押地那裡,正如耶和華所說的。 \end{tabularx} \\ \\ \relax
34:6 & \begin{tabularx}{0.7\textwidth}{X} 耶和華將他葬在摩押地,伯‧毗珥對面的谷中,只是到今日,沒有人知道他的墳墓。 \end{tabularx} \\ \\ \relax
34:7 & \begin{tabularx}{0.7\textwidth}{X} 摩西死的時候一百二十歲,眼目沒有昏花,力量沒有衰退。 \end{tabularx} \\ \\ \relax
34:8 & \begin{tabularx}{0.7\textwidth}{X} 以色列人在摩押平原為摩西哀哭了三十天,為摩西哀哭居喪的日期才結束。 \end{tabularx} \\ \\ \relax
34:9 & \begin{tabularx}{0.7\textwidth}{X} 嫩的兒子約書亞,因為摩西曾為他按手,他就被智慧的靈充滿。以色列人聽從他,照著耶和華所吩咐摩西的去做。 \end{tabularx} \\ \\ \relax
34:10 & \begin{tabularx}{0.7\textwidth}{X} 以後,以色列中再沒有興起一位先知像摩西的,他是耶和華面對面所認識的。 \end{tabularx} \\ \\ \relax
34:11 & \begin{tabularx}{0.7\textwidth}{X} 耶和華差派他在埃及地,向法老和他的一切臣僕,以及他的全地,行了各樣神蹟奇事, \end{tabularx} \\ \\ \relax
34:12 & \begin{tabularx}{0.7\textwidth}{X} 又在以色列眾人眼前顯出大能的手,行了一切大而可畏的事。 \end{tabularx} \\ \\
[1ex]
\hline
\hline
\end{longtable}
$^{1}$多謝每一位在現場參加崇拜的電影姊妹.
今天不同了,今天沒有在門口點名.
所以進來的時候才知道有認識的人來了.
特別是我看到一整組人一起來.
可能剛剛吃完飯,要習慣在附近找吃的.
那邊貴很多.
那邊的價錢還可以.
一個星期吃一餐還可以.
可能要適應一下.
新朋友可能要適應一下.
都是佐敦站近尖沙咀站.
雖然地址是尖沙咀.
在奧斯丁度.
但應該是佐敦恆豐中心出口最快到這裡.
不用介紹了.
因為我上星期報告完之後.
如果看不到Google map就幫不到他.
我就被人問.
潘Sir不懂得看Google map,什麼方法最快去.
我真的告訴他,是佐敦站最近.
很開心大家一起來崇拜.
今天的講題是上帝視覺.
不知道你對這個字或用詞有什麼聯想.
如果玩遊戲的話,有上帝視覺就無往而不利.
因為整盤數都看了,知道要怎麼玩.
不知道你會不會打麻將.
不打麻將不要緊.
上帝視覺對於你來說是什麼呢.
有些人你了解是一個預視的能力.
或者是中觀全局的能力.
但對於我來說.
上帝視覺是第三身去看那件事.
很多時候你都會牽涉在當中.
或者成為其中一部分的人.
但有時候你抽一抽出來.
成為第三身去看這件事的時候.
我相信你的感受或你的看法.
或者將來下一步怎麼做.
會有一個比較不同的觀感.
那段經文如果你有看過或預習過.

$^{41}$就是講生命的第34章.
生命的第34章是生命的最後一章的經文.
比較特別的經文是我自己在這段日子.
特別是漢建這個主題當中.
我自己比較深的感受.
所以在漢建這個閱題當中.
第一講就選生命的第34章.
跟大家重溫一下這段經文.
這段經文不是很長.
我們一起讀有三章.
投影片一起來.
所以就第34章的第一節.
請.
摩西松.
我第一次讀.
挑上帝打開你的說話.
求上帝光照我們明白你的心意.
有願說話再一次成為我們今年的提醒.
以致我們在當中認識上帝.
更加認識我們的光景.
求主你教導我們.
奉耶穌的名求.
阿們.
請你重溫這段經文對你來說有什麼看法.
或者你去過聖地的時候.
你真的上過尼泊爾山的感覺如何.
我沒有去過聖地.
我剛剛看到有個同工就去了聖地.
對於你來說.
看這個尼泊爾山是什麼日子.
我這個禮拜預備講章的時候.
我就上Google Earth.
因為去不到.
Google一下.
上Google Earth.
看看.
和我以前上Google Earth很不同.
現在立體很多.
還有很多旁邊的資訊.
你可以看到很多不同的圖畫.

$^{81}$我截了個圖.
和大家分享一下.
上山看的畫面就很漂亮.
你行山當然會打卡會拍照.
但如果你是摩西.
你上到這個山的時候.
不知道你的心情如何.
因為其實現在我們讀34章的時候.
就知道他死期到了.
其實之前已經講了.
已經講了上這座山的時候.
就是你的日子到的時候.
但是.
上帝就叫摩西上去.
那你是帶著什麼心情上山呢.
但你上到看到什麼呢.
這個就真的.
回到個人自己的感受和反應.
那你是很不情願上去.
還是很期盼上去.
你不情願的意思就是.
你知道上到去就是你的死期.
就是完了.
還是你很期盼.
我終於可以下班了.
還是你很期盼.
你上去說.
我就是想看看上帝賜給我們.
這群以色列人的境地是什麼.
那你看看你上這個山帶著什麼.
你是覺得.
我這麼辛苦.
沒有工作也有勞.
我真的帶著這群頑民.
去到一個境地.
我現在一見到就去不到.
有沒有搞錯.
我想你代入聖經裡面.
其實摩西在想什麼呢.
或者摩西的場景.

$^{121}$到我們真的要面對這麼大的張力的時候.
是什麼呢.
如果你看一段經文.
你過去的直到現在.
特別這兩三年有很多經歷.
你會為一些事抱不平.
你會為一些事.
覺得為什麼上帝不工作.
你會為一些事.
覺得這麼快就忘記了.
有沒有搞錯.
你當時在哪裡.
你會有很多不同的聯想.
但是對於我們去了解這段經文的時候.
其實摩西為什麼不能去.
摩西為什麼不能去.
其實摩西為什麼不能去.
我們有不同的.
分教會的弟兄姊妹.
讀了多少經的弟兄姊妹.
對於摩西不能進去.
應許地.
或者不能過約旦河.
的原因.
其實就是.
有人犯錯.
或者以色列群體當中.
有什麼原故.
其實不讓那一代人進入.
應許地.
整件事對於.
今天就不會很詳細地說.
但是重點就是.
其實在開初.
新命紀如果你回去.
如果今年還沒開始讀經的話.
或者今天開始.
就開始讀經.
就試試讀新命紀的時候.
尤其是第一章開始已經在說.

$^{161}$其實上帝已經告訴.
探子聽.
你回去告訴以色列民.
這個地方是什麼地方.
但是那群人聽完之後.
他們很害怕.
他們覺得上帝恨他們.
上帝為什麼帶我們去這個地方.
那些人又高大又威嚇.
其實看不到上帝.
應許.
雖然十二個探子當中.
有兩個是告訴你.
不是啊,這個地方很好.
如上帝說的風產.
如上帝說的美麗的地方.
不只是看到一些負面的情況.
但是因為.
不信的緣故.
然後上帝就說.
你這代人不能夠進入.
我應許你的地方.
其實問題就是.
我們會有苦澀.
或者我們覺得有不公允的情況.
我們很多時候都會質疑上帝.
其實上帝.
你有沒有看.
其實很多事情不是這樣的.
我們很多時候都著重.
我們做了什麼.
或者做了什麼和上帝做一個協議.
但是我們認識的上帝.
是不是真的這麼不公平呢.
又或者我們認識的上帝.
是不是因為.
看一件事就這麼生氣.
這麼小器呢.
但其實當初.
是不是上帝不讓他們進去.

$^{201}$不是,是他們覺得上帝恨他們.
是覺得上帝立心不好.
但他們在這裡禁地方.
對於我們.
可能不是最靠近的人.
我們也不是以色列人.
不是當時的人.
可能不容易了解他們的困境.
但是我們在過去的日子.
我們的信仰很多時候都被衝擊.
由19年開始.
到2020年.
我們Flowchart有很多弟兄姊妹.
都不是第一次回教會.
他們對於教會生活.
或者教會文化.
都有很多不同的經歷.
每次教Info Group的時候.
都會和弟兄姊妹一起去說.
其實.
你們是鼓起很大的勇氣.
去再進入一個新的教會群體.
你們是鼓起很大的勇氣.
去想想再投身一個群體的時候.
我需要適應的地方.
因為你事實上經歷了很多.
你和上帝的協議.
今天我希望在新的一年.
和大家一起去思考.
其實摩西是怎樣去看待這件事呢?.
你會發覺摩西.
他不是和那些以色列人.
繼續去爭持.
他仍然去做上帝要他做的工作.
就是你只管去吧.
繼續帶著這個群體.
去完成我叫你走的路程.
他就繼續去做.
在這段經文中.
有些比較難處理的地方.

$^{241}$就是.
在這段經文中.
有些比較難處理的地方.
就是摩西做了很多事情.
上帝怎樣去看待摩西.
做了很多事情.
而摩西的終局對於摩西來說.
是否公平呢?.
是否他應該得到的東西呢?.
這段經文在文字上來說.
為何會變成.
變成詩歌呢?.
我的powerpoint.
謝謝.
這段經文.
如果你細心看.
這段經文.
如果你細心看的時候.
你會看到摩西上山上.
他看到全地.
看到不同的地界.
耶和華巴基烈全地直到旦.
拿忽他利全地伊法連.
其實對於摩西來說.
是.
當時他還沒有看到.
因為你細心想想.
何時才分地呢?.
就是約書亞打勝仗.
然後就把不同地.
分在所屬的支派上.
你怎樣看這段經文呢?.
就是.
當然你了解的.
可能是後製的人去編紗.
這段經文.
然後就成為《新聞記》的終結.
但對於我們來說.
上帝確確實實.
是預備了一個疆界.

$^{281}$讓人明白到.
上帝已經.
把伊特蒂賜給.
他所屬的群體.
對於我們來說.
上帝仍然是.
履行他所應許的地方.
所以你會看到.
耶和華亞伯拉罕的約.
再一次在這段經文上出現.
就是第四節.
耶和華對他說.
這就是我向.
亞伯拉罕以撒瓦國.
起誓應許之地.
說我必將這地賜給你的求約.
現在我使我所求.
我必將這地賜給你的求約.
現在我使我所求.
你的眼睛看見了.
你卻不得過到那裡去.
摩西是第一代.
出埃及的人.
第一代出埃及的人.
不能夠盡應許地.
所以他就停在那裡.
上帝履行他的一班步.
但上帝仍然履行他.
對以色列人守的約.
你這一代不能過.
但下一代可以去.
你的工作就停在那裡.
你會看見我仍然.
履行我對亞伯拉罕所起的誓.
這段經文中.
其中一個字.
與大相關.
就是「所以」這個地方.
如果你記得.
有一次在舊約.

$^{321}$說登山見遠離的時候.
你會回到創世紀的時候.
創世紀就是.
亞伯拉罕與羅德.
登上高地的時候.
望你們選擇地方.
羅德看見地方很平.
亞伯拉罕就讓他先選擇.
於是他選擇了.
羅德選擇的其中一個地方.
就是「所以」.
對於亞伯拉罕.
當時他讓了給他的侄子.
他自己就登上山地居住.
但是.
耶和華答應亞伯拉罕.
你會去住那個地方.
今天那件事就成就了.
亞伯拉罕的後裔.
是能夠在當初.
就算他讓了給羅德也好.
羅德也不是在那裡長住.
真正能夠承擔.
或者承受耶和華的福的.
就是亞伯拉罕之約.
你會看見這件事.
無論人做什麼都好.
只要上帝應許了你的事.
你不會落空的.
兜兜轉轉也好.
可能兜了很大的彎也好.
上帝仍然會做他對我們守的事.
所以守著思慈愛的耶和華.
或者守著思慈愛的上帝.
他對昔日的人如是.
今天對我們的人如是.
在之前的祈禱會.
和這個月的祈禱會.
都是在說.
這兩個月都是大時大節.

$^{361}$但在這一個多禮拜.
你會發覺.
之前由19年開始的案件.
就排到現在去審.
和去判刑.
事實上是不容易的.
特別是過時過節.
有些人不能夠和家人過節.
我們當中有不同人.
有不開心的地方.
或者記掛的地方.
有些人是認識的.
最主要是什麼呢?.
最主要是仍然彼此提醒.
彼此提醒一件事.
就是日子不容易過.
不過我們仍然相信.
一些事情就是.
我們本著良心去做的事.
上帝是監察人心的主.
縱然現在的問題還未解決.
但我們仍然相信.
是上帝叫我們活著的.
這些說話是很容易說的.
做是很難做的.
但我們仍然等待這一刻.
因為好像當日.
以色列民經歷了很多苦難.
但苦難來自他們不信所犯的惡果.
今天我們見到.
我們認識的人.
他正在經歷苦難.
但苦難不是來自他的惡果的時候.
上帝會紀念他所受的苦.
你明白我的意思嗎?.
所以同樣是苦難.
但苦難的源頭不同.
但上帝是懂得分的.
上帝是知道所作何事的.
所以這就是我們憑著信心去等待.

$^{401}$和憑著信心去公義的主.
在榮耀中再來的時候.
在寶座上.
他就作出判決.
這是我們一直以來持守.
相信上帝是不會不知情的.
所以難說的話就是.
要安慰.
但仍然要經歷被加害的日子.
回到這段經文再下去的時候.
你會見到第五至第八節.
「於是耶和華的僕人摩西死在摩亞地.
正如耶和華所說」.
這段經文去到這個時候.
就作一個中轉.
摩西就完結了.
摩西就死了.
他不是死在摩亞地.
起碼他不是死在摩亞平原.
是在尼泊爾山上.
對於生命記來說.
是摩西的死.
但文素記的33章.
就是他哥哥.
但他三年的哥哥阿倫死了.
阿倫是123歲過世.
摩西的過世和阿倫的過世.
是很相似的.
都是耶和華說.
吩咐他們上山.
他們兩個都上山.
上山之後.
就死在山上.
兩代領袖在山上完成了他們的生命.
對於這句話.
是什麼呢?.
第六節就是.
耶和華將他們埋葬在摩亞地.
沒有人找到墓.
這個也不難理解.

$^{441}$因為很多人喜歡將東西抬高.
然後聖化了.
於是成為朝聖.
說上帝不想後人做這個功績.
然後就找不到這個東西.
不知道大家會不會有去聖地的心態.
我其實也想去.
有沒有人去過?.
看到.
但普遍是沒有去過.
我不知道你們有沒有去過.
可能覺得.
我現在有這麼多地方去.
我當然不去聖地.
當然去另外一些聖地.
是吧?.
會的.
但起碼基督教不是要你一年去一次聖地朝聖.
你去聖地.
我們其實說聖地也不應該這樣說.
起碼我們去看回耶穌當日在地上生活的日子.
走一些路.
或者看看加利海原來可以這樣.
或者可能了解一下.
但對於我們來說.
我們很清楚我們不是要供奉什麼聖人.
但一個人的離去.
事實上對我們來說是一個很重要的提醒.
剛才說到.
阿倫和摩西都在山上悄悄然離開.
耶和華吩咐他們上山.
他們就安然上山.
沒有大鼓大鼓.
沒有人跟從他們.
就是安然上山.
聽完上帝的話.
看完他們要看的東西.
然後就睡著.
但經文很巧妙.
又是很特意說了一件事.

$^{481}$摩西死的時候120歲.
眼目沒有昏花.
精神沒有衰敗.
短短三個phrase說了一年歲.
說了他們的精神狀態.
說了他們的身體狀況.
我覺得比我好.
我都在想一件事.
我是不是要換眼鏡呢.
其實現在沒有老花.
我看這些字是清楚的.
眼目沒有昏花.
精神沒有衰敗.
其實就是在說.
其實他們還可以做的.
他們不是做不到.
所以上帝說.
你已經畫了牙.
你已經過了有效日期.
你就step down.
他又不是.
其實他離世的時候.
其實狀態是ok的.
狀態當然ok.
就是82歲還可以上山.
有時我覺得這些經文.
真的很難教.
我自己都會行下山.
有些人叫我不要自己行山.
很危險.
但我那些很簡單.
我不是那些五星星那些山.
普普通通讓自己放空.
看看一點點高地.
但都碰到人.
我碰到一個認識的人.
他說潘Sir你又行山.
我想休息一下.
跟你行一段吧.
好吧.

$^{521}$下次見到我就可以行一段.
沒問題的.
就行一段吧.
其實以前沒有什麼.
不知道是不是長生觀.
越行就覺得自己的氣不順.
我說不如不要說話.
就行一會.
聊聊天.
不是說客.
我想說.
其實82歲還可以上山.
為什麼要這麼大來大故還要上山呢.
其實真的.
上山上.
由上帝真的讓摩西看看.
其實你看看.
我說的話是真的會成就.
我的應許不會落空.
同時我也想你看看.
叫做了一個心願也好.
又或者看到.
這班以色列人的將來是什麼.
不一定是你可以經歷完.
但你會看到的將來.
我不知道你的心態會不會怎樣.
這些日子對我來說.
我常常都想預視多些將來的可能性.
無論是堂妹也好.
團隊也好.
家人也好.
就是想想前瞻望遠一點.
也是讓你看遠一點.
上星期John說.
你的眼睛的高度.
你的視界.
你的眼睛的高度.
去到哪裡就是讓你看到遼闊的空間.
我兩個兒子.
有時候也成為我說的內容.

$^{561}$特別我小時候.
他當然不夠高.
他說為什麼看不到.
你們在說什麼.
於是我就從他那裡抱高.
你看他的眼睛開始放光.
高一點.
最高了.
抱到這個位置.
都不夠你高.
我再高一點.
行了,剛剛好,不要動.
特別是小時候還有花市.
經常都要抱起他.
是的,你會發覺.
人能夠站在高位.
看到遠景的時候.
他的心有很多想像.
有很多可能性.
覺得自己可以去那裡.
又或者看到將來.
對於上帝.
讓摩西去到山上看到.
是什麼呢.
我相信除了上帝應許也好.
還有看到這個群體的將來.
在這個地方.
他們去領受上帝的恩典.
第八節就是.
以色列人在摩亞平原為摩西哀哭了三十天.
為摩西哀哭的日子就滿了.
這句話就是他們慣常會做的事.
但是我想集中說一節經文.
就是「於是耶華的僕人摩西死在摩亞地.
正如耶華所說的」.
這句話.
我再了解裡面的內容.
其實我覺得有另類的看法.
當我們覺得摩西上了山上.
我看到很好.

$^{601}$但是我都不能去.
又或者我已經完成了.
可能很情緒,很感受.
我已經完成了我的工作.
這樣就完了.
但是我想讓你看多一個解釋.
就是正如耶華所說的這個字是什麼.
這個字原本的文字是三個希伯來字.
這樣放在一起.
其中一個正如就是.
如果用英文就是according to.
就是在.
但是再下去的時候.
就是耶華是一個字.
什麼叫所說的呢?.
其實所說的這個字就是口.
口.
又可以選擇說,說話.
其實是一個口.
如果直譯來說.
正如耶華所說的.
就是可以譯在.
在耶華的文中.
意思是.
如果上個星期.
你聽的訊息裡.
我保護你如同保護我眼中的銅人.
你搞我眼的時候.
耶華今天對摩西的死去.
或者摩西的離去的時候.
耶華對這個僕人所作的.
最後所作的就是.
耶華的吻別.
有什麼你覺得是很.
安心或者覺得是很安詳呢?.
用施布真牧師Charles Berger.
他用拉比的解說來說.
就是.
就好像母親在小朋友睡前.
吻他一下.

$^{641}$然後就慢慢放他到床上.
就好像耶華上帝.
捨了摩西的靈魂.
接走了他.
就把他肉身埋葬在墳墓裡.
用這個方式去送別他的僕人.
對我來說我覺得是最好的.
不需要很多人跟上山.
就是耶華上帝叫你上山啊摩西.
走吧.
就走吧.
我們看啊.
看遠一點.
那個就是我.
歷行當日阿伯拉罕的約.
你看到嗎?.
看完就.
啪.
絕腳.
他就不離開了.
我覺得對於一個僕人來說.
是一個很好的安捱.
你所做的事.
我知道.
我就要我問.
給你.
所以當.
跟你分享這句說話的時候.
就讓你明白到.
其實耶華一切都看在眼裡.
耶華知道摩西所作的事.
也知道摩西的難處.
摩西仍然是.
做什麼呢?.
聖經描述他是一個很謙和的領袖.
他慢慢按耶華的心意做好.
他要做的工作.
120歲去到這個環境當中.
其實他的能力仍然有.
但是他的工作已經做完.

$^{681}$然後上帝就讓他安然離去.
經文再下去就是.
第二個領袖出來.
就是亂的兒子約書亞.
約書亞不是今天我想說的.
約書亞的出現其實是散佈在.
新冥紀的內容裡.
由他做探子到他真的成為摩西的副手.
一直都有榮譽在當中.
摩西一直都是教他.
為他祈禱.
也看到他出去征戰的時候.
他如何去表現對耶華的那種尊重.
成為新一代領袖的一步一步去學習.
所以約書亞不是一些突然出來的事情.
是看著摩西怎樣做.
他跟著怎樣做.
對於這句說話來說.
反而是後面的十至十二節.
就是這章經文最後的三節.
我自己覺得摩西一生服侍的.
就是毛志明.
又或者用我們.
現在我們很少納悲.
應該不用這個方式.
如果用我們做安息禮拜來說.
就是他的術士.
他的術士說.
這個人在以色列當中.
再沒有興起一個先知.
好像摩西.
他是耶華面對面所認識的.
所以第一件事就是.
真正能夠確認是耶華面對面的認識.
是誰呢.
摩西是.
而他能夠做到的.
就是在埃及和法老傳遞當中.
做了歧視.
而在帶領以色列人當中.

$^{721}$他有大量的手做管治的工作.
所以摩西是誰呢.
摩西被稱為耶華的僕人.
這句說話可能覺得好像很簡單.
又有一個僕人.
又有很多僕人.
對呀.
但是認真說的.
或者了解的.
在聖經裡面被稱為耶華的僕人.
後面那個是摩西.
耶華的僕人摩西.
或者摩西耶華的僕人.
這是他的專稱.
你說大衛也是耶華的僕人.
這是一個統稱.
另外聖經裡面真的說.
耶華的僕人.
出現了一次.
那是約書亞.
但約書亞能夠稱為耶華的僕人.
都是因為他接觸摩西的工作.
所以真正能夠稱為耶華的僕人.
那個就是摩西.
這就是聖經對我們.
要告訴我們.
一切上帝都知道.
而他真的呼召他出來.
要帶領一個屬於他的群體.
這個人是耶華所認識的.
上帝呼召他的人.
做上帝的工作.
所以上帝一定會記住他.
這個是稱為耶華的僕人.
所以摩西的經歷.
我希望今天和大家一起去看看.
我們的閱題是漢見.
我希望大家漢見一個偉人.
從他呼召到開始.
現在就看他最終章.

$^{761}$對於你們來說.
什麼是漢見呢.
有一句說話.
可能是眼看為實.
這句說話對於你們來說.
用廣東話來說.
有圖有真相.
可能會更加近.
見到相才為真.
或者真是看實物才是真.
其實當我們說眼看為實的時候.
見到才是值得相信的.
我們信仰其實不容易見到.
起碼你沒見過上帝.
我是謙卑地看著.
你說見過.
我就想之後和你談談.
很多東西我們不是見過.
但我們所信的是什麼呢.
反而當我們問眼見為實的時候.
我希望你們不要忘記.
我們的信念很多時候取決於.
我們的覺醒.
上星期說到最後的三筆.
不知道你們記不記得.
有時不是我們看多少.
是我們忽視了什麼.
對於我們來說.
有時不是我們看多少.
是我們看完之後有什麼覺醒.
今天對我們來說.
我們不是第一次聽見證.
我們也不是第一次在教會傳講見證.
但事實上見證對我們來說很難重覆.
因為他的經歷很難重覆.
但見證最重要的目的.
不是叫你重覆當事人所做的事情.
就是從中看到其實有什麼元素.
在信仰當中要我們覺醒.
他的見證不能夠重覆.

$^{801}$但上帝是那個共同元素.
上帝能夠透過他的見證.
表現他對上帝的跟隨.
倚靠和忍耐.
其實那個覺醒對我們見證來說是重要的.
那個是需要我們一起學習的.
就如今天我們看摩西.
我當然不是摩西.
摩西百二歲,我哪有百二歲.
不是單向的指標.
但摩西的侍奉,他的能力.
如何可以用在他帶領以色人當中.
這是我們可以借鏡.
但摩西對上帝的忠誠.
更加是我們要覺醒的地方.
新的一年,真的.
風水學不是說很多崗位.
或者沒有什麼一定要大家.
每年一定要侍奉這些.
侍奉心智表的教會.
但是你今年的忠誠在哪裡呢?.
你看見什麼你願意去付上時間呢?.
你看見什麼願意在當中參與呢?.
你看見什麼願意在當中學習.
想了解更多,以致你想投身呢?.
這是需要你去看的.
你連看都不看.
其實忽視了很多.
上帝邀請你參與的可能性.
和邀請你參與彼此的關心和提醒.
這是很重要的.
《國王的新衣》是我小時候.
教兩個兒子看過的圖畫書.
但是你長大了.
你會知道《國王的新衣》不是給小朋友看的.
《國王的新衣》是給什麼人看的呢?.
是給講書的人看的.
正正就是一頁一頁的內容.
就是告訴講書的人.
其實你看到你有沒有出聲.

$^{841}$其實你看到你有沒有去跟進.
其實你看到你有沒有立心去面對.
這也是每個講《國王的新衣》故事的時候.
要真正處理的地方.
所以在之前講道也說過.
沉默就是邪惡的幫兇.
所以是時候看見就要有改變.
這就是覺醒.
2020年1月11日.
第二講是2022年1月2日的月題.
就是曠野.
我當時講這個月題.
是講生命的第八章.
就是講40年的曠野.
摩西點和那班以色列人說.
你們生生代要記住.
先祖在40年曠野的生活.
免得你再重蹈覆轍.
摩西已經跟他說過.
那時候的原因是什麼呢?.
那時候的原因是我們來了希塘.
記住是2020年1月的時候.
我們2019年11月30日就完成了石劍美的日子.
12月開始就來希塘.
四個星期都很多人.
紅心勃勃.
紅心不是說發財.
不是這些紅心勃勃.
紅心勃勃就是我們地方大了.
容納多一些弟兄姊妹.
因為那時候在石劍美要站在樓梯.
希塘是大了很多.
讓更多上教會.
或者認識我們的弟兄姊妹可以參與.
但是2020年1月11日.
過了之後我就再沒有在希塘講了.
到新春度John在網上講之後.
就已經是另一回事.
應該這樣說.
到2月的時候我們就開始網上送.

$^{881}$所以我想看那時候我穿什麼衣服.
可不可以穿那套衣服上來呢?.
原來是我們沒有.
因為那時候都沒有Facebook.
那時候不用直播.
都沒有網上送.
為什麼我沒有呢?.
在哪個硬盤呢?.
可不可以想一想.
那時候在希塘沒有直播.
所以不知道穿什麼衣服.
這樣.
曠野是什麼呢?.
事實上2020年1月之後.
進入2月的日子.
我們就是網上崇拜.
這個就是我們的曠野.
由你見到最左上角的.
就是在新蒲崗的時候做直播.
接著我們就.
那個租約就沒有了.
不能在新蒲崗的姐妹要退租了.
我們就很開心.
喜伯倫堂就借了隔壁的副堂給我們.
我們就在喜伯倫堂做直播.
偶爾我們疫情放寬了.
在50\%的時候.
我們都能夠回來的.
一回來我們就立刻搞一些特色聚會.
就搞了一個開學崇拜.
但是有好景不常.
有得來的時候.
都會變成一個人的崇拜.
就是打風.
所以你見到.
很大的反差就是.
我站在一個沒有會眾的環境下.
就做一個崇拜.
我那時候就想.
如果不要打風了.

$^{921}$我都有祈禱.
希望避一避.
但是沒有了.
其實最大部分的時間.
就是回到MIFC那裡崇拜.
Folk Church其實在這兩年經歷了.
曠野真的漂流的日子.
當我每次說我們都漂流了很多地方的時候.
頂尖的節目就說.
當然啦我們flow嘛.
就是很開心的.
是啊是啊我們flow的.
我們很flow的.
但是上過infogroup的頂尖姊妹.
我都說.
其實每一次能夠聚會.
或者能夠一個群體一起.
參與崇拜是得來不易的.
還有就是.
每一次的重點.
都是希望頂尖姊妹明白到.
為什麼我們要這麼用心.
和這麼盡力去經營一個崇拜.
因為敬拜完事教會.
在地上讓人認識.
上帝一個很開放的途徑.
任何人都可以一起來參與.
今天我們能夠回來喜拜輪堂.
其實是很困難的.
當然大家在外面排隊都排好了.
今天很早.
我問同工.
7點3已經開始.
還沒到7點3已經開始有人排隊了.
但是我們回來的預備都是.
每一次開始設置機器.
到試音到做直播開始的時候.
其實我們只有100分鐘.
100分鐘由這個場地沒有.
到現在可以和你網上.

$^{961}$和大家一起有全組的敬拜.
我們有100分鐘時間去預備.
我們不想遲.
也不想你站太久.
但我們仍然覺得是值得的.
每個星期一起去看見上帝的工作.
很希望在結束的時候.
如果大家覺得.
沒了那個爆棚.
我會唱歌的.
在最後上帝的視覺的時候.
我希望大家.
如果你過去受疫情限制.
你的信仰生活不太穩定.
可能網上有機會參與一下.
新的一年試一下.
讓上帝讓你看見更遠的地方.
看到一個屬靈群體開始焦聚.
看到更多在這個群體的可能性.
我希望大家一起參與.
讓我們看見所看見的東西.
意思是什麼.
不是我們一班一直侍奉的人.
看到需要和願著.
我希望你參與敬拜的時候.
和我們一起看到那樣東西.
你不是一個觀眾.
你是一個參與者.
看到上帝要我們看見的東西.
上帝的視覺希望我們一起參與.
你有你的呼召之處.
我們有我們的呼召之處.
每一個上帝的呼召.
都是給我們一個回應.
呼召不一定是全職侍奉的呼召.
呼召不一定是一些很困難的時候.
才要挺身而出做superman.
不是.
呼召是上帝給我們一個回應.
摩西被上帝呼召.

$^{1001}$讓上帝和我們一起在當中看見.
上帝給我們的視覺.
讓我們去回應上帝給我們的呼召.
在新的一年.
新的開始.
新的經歷.
我相信苦難仍然會在我們當中.
但是若果苦難是來自惡延伸出來.
我們就會悔改認罪.
若果苦難是來自上帝給我們的考驗.
又或者叫我們去經歷上帝的信實的時候.
我們仍然會回應上帝給我們的呼召.
我們一起看看.
看看上帝給我們的路是怎樣.
\newpage



\section{耶利米書 1:4-19-20230121}
\label{sec:RPsjvP8W4CE}
\textbf{【流堂崇拜】齋看見就夠曬|耶利米書1\_4-19|20230121 [RPsjvP8W4CE]}
\newline
\newline
連結: \href{https://youtube.com/watch?v=RPsjvP8W4CE}{\texttt{ https://youtube.com/watch?v=RPsjvP8W4CE}} ~~~~ 語音日期: 2023-01-21 
\newline
\newline
\hyperref[sec:gptfSrlmqo8]{\small{< < < PREV SERMON < < <}}
~
\hyperref[sec:index_chronic]{\small{[返順時目]}}
~
\hyperref[sec:index_scriptual]{\small{[返順卷目]}}
~
\hyperref[sec:6BvsvxnAGsQ]{\small{> > > NEXT SERMON > > >}}
\newline
\newline
耶利米書 1:4-19-20230121
\newline
\begin{longtable}{cl}
\hline
\hline
章節 & 經文 (和合本修訂版)\\
\hline
1:4 & \begin{tabularx}{0.7\textwidth}{X} 耶利米說,耶和華的話臨到我,說: \end{tabularx} \\ \\ \relax
1:5 & \begin{tabularx}{0.7\textwidth}{X} 「我尚未將你造在母腹中,就已認識你;你未出母胎,我已將你分別為聖,派你作列國的先知。」 \end{tabularx} \\ \\ \relax
1:6 & \begin{tabularx}{0.7\textwidth}{X} 我就說:「唉!主耶和華,看哪,我不知道怎麼說,因為我年輕。」 \end{tabularx} \\ \\ \relax
1:7 & \begin{tabularx}{0.7\textwidth}{X} 耶和華對我說:「不要說『我年輕』,因為我差遣你到誰那裡去,你都要去;我吩咐你說甚麼話,你都要說。 \end{tabularx} \\ \\ \relax
1:8 & \begin{tabularx}{0.7\textwidth}{X} 你不要怕他們,因為我與你同在,要拯救你。這是耶和華說的。」 \end{tabularx} \\ \\ \relax
1:9 & \begin{tabularx}{0.7\textwidth}{X} 於是耶和華伸手按住我的口,對我說:「看哪,我已將我的話放在你口中。 \end{tabularx} \\ \\ \relax
1:10 & \begin{tabularx}{0.7\textwidth}{X} 我今日立你在列邦列國之上,為要拔出,拆毀,毀壞,傾覆,又要建立,栽植。」 \end{tabularx} \\ \\ \relax
1:11 & \begin{tabularx}{0.7\textwidth}{X} 耶和華的話臨到我,說:「耶利米,你看見甚麼?」我說:「我看見一根杏樹枝。」 \end{tabularx} \\ \\ \relax
1:12 & \begin{tabularx}{0.7\textwidth}{X} 耶和華對我說:「你看得不錯;因為我要看守我的話,使它實現。」 \end{tabularx} \\ \\ \relax
1:13 & \begin{tabularx}{0.7\textwidth}{X} 耶和華的話第二次臨到我,說:「你看見甚麼?」我說:「我看見一個水燒開的鍋,從北而傾。」 \end{tabularx} \\ \\ \relax
1:14 & \begin{tabularx}{0.7\textwidth}{X} 耶和華對我說:「必有災禍從北方發出,臨到這地所有的居民。 \end{tabularx} \\ \\ \relax
1:15 & \begin{tabularx}{0.7\textwidth}{X} 看哪,我要召北方列國的萬族。這是耶和華說的。他們要來,各安寶座在耶路撒冷的城門口,周圍攻擊城牆,又要攻擊猶大的一切城鎮。 \end{tabularx} \\ \\ \relax
1:16 & \begin{tabularx}{0.7\textwidth}{X} 這民離棄我,向別神燒香,跪拜自己手所造的,我要針對這一切惡行,向他們宣讀我的判決。 \end{tabularx} \\ \\ \relax
1:17 & \begin{tabularx}{0.7\textwidth}{X} 所以你當束腰,起來,將我所吩咐你的一切話都告訴他們;不要因他們驚惶,免得我使你在他們面前驚惶。 \end{tabularx} \\ \\ \relax
1:18 & \begin{tabularx}{0.7\textwidth}{X} 看哪,我今日使你成為堅城、鐵柱、銅牆,對抗全地和猶大的君王、官長、祭司,並這地的百姓。 \end{tabularx} \\ \\ \relax
1:19 & \begin{tabularx}{0.7\textwidth}{X} 他們要攻擊你,卻不能勝過你,因為我與你同在,要拯救你。這是耶和華說的。」 \end{tabularx} \\ \\
[1ex]
\hline
\hline
\end{longtable}
$^{1}$特別跟自己說一句.
新年其實都很開心.
因為要來港島.
所以我家人5時多吃飯.
吃完就趕過來.
所以多謝家人的體諒.
亦都有點不開心.
因為其實新年開始意識到一件事.
就是身邊的朋友好像都不在了.
不在的意思不是過世.
是不能夠一起見面吃飯.
你想團拜,拜年.
你會發覺很多已經不在了.
譬如大學的團契.
或者我神學院的同學.
其實很多都已經離開了.
所以感覺是差的.
我相信其實.
尤其是跟離散了的弟兄姊妹說.
在網上.
可能你移民了.
離開了不同的地方.
其實你也有家人在香港.
或者今年新年.
你也很想一家人在一起的時候.
我們做不到.
你覺得好像丟下了父母親在香港.
或者一些家人.
你所掛念的家人在香港.
你會難過.
正如Folk Church弟兄姊妹一樣.
你說她正在經歷.
種種大大小小各式各樣的離別.
我們都要適應.
所以我猜想的.
在這個新年裡.
或者接下來的新年當中.
很不一樣.
譬如我家裡有一個小朋友.
他知道他將會有三個好朋友.

$^{41}$三個朋友都離開.
我對他很不開心.
我也不知道怎樣跟他祈禱.
我跟他祈禱的時候.
他就離開了.
其實很多這些感覺.
都在我們當中.
我希望.
劉先生弟兄姊妹.
你有年糕吃.
加州已經將年初一成為公假.
恭喜加州的弟兄姊妹.
希望年初一的時候.
我們有更多人一起聚集.
這些不開心的事很多.
今天會說一個訊息.
我希望說得快.
我想大家回家.
(笑聲).
我希望說得快一點.
雖然有些事情不清楚.
不是我想的.
我盡量性餐.
但我想今天說.
新年我不想說一個開心的訊息.
或者聽完之後興奮的訊息.
我今天沒有準備說這些.
我今天想說一個很難說的訊息.
我今天想說一個很困難的訊息.
說完不開心的訊息.
這是我想了一段日子.
才決定說這個訊息.
希望今天開心的度假.
下星期John會說套年.
不是運程.
差點說錯話.
他說套年度.
即是套一陽眉.
他明年交給John.
John今天彈結他.

$^{81}$和侍奉的弟兄姊妹.
今天能夠在當中侍奉.
還有今天出席的弟兄姊妹.
所有在年三十侍奉的.
很少會這樣做.
除了全聖餐禮外.
今天我們會說.
我們按一按耶利米斯一章四至十九節.
我要先說一點.
其實說呼召經文.
我們做哲學院的老師.
很少說耶利米斯一章.
基本上不說.
耶利米斯一章.
不是一些很用來呼召經文.
或者用來容易呼召經文.
其實它沒有這個功能.
耶利米斯是令所有人都走開.
和不想侍奉的人.
就看耶利米斯一章.
所以今天很不幸地.
我們選了這段經文看.
希望我們能夠經歷一下這些.
很快,耶華說「臨到我說了」.
「我使你在母腹中成形而先」.
「就認識你,你未出母胎」.
「我已經使你分別為性」.
「立你做列國的先知」.
「我就說,主耶華,我還年輕」.
「不知道該怎樣說話」.
我還是用這個背景.
即是PowerPoint的背景.
我用了很多年.
我打算今年更改.
真是不好意思.
我們先回到第一節.
回到上一段PowerPoint.
人生都不想改變.
你看看這段經文.
其實有多嚴重.

$^{121}$這個「趙命」不是普通的趙命.
這個趙命不是隨便有一個趙命.
這個趙命很特別.
它使你在母腹中成形.
而先就認識你.
所以這個趙命是在媽媽的肚子裡.
耶華已經呼召你做過一件事.
分別你為性,成為列國的先知.
所以呼召這件事.
其實不是在說崗位的侍奉.
而是在說一個人上帝做你的時候.
上帝做你的時候.
放了甚麼特質,元素在你的生命當中.
以至你可以滿足上帝給你的趙命.
這是你從小到大的事情.
不是突然間轉換工作.
或者轉換侍奉崗位.
或者由這間教會出去另一間教會做.
或者從這個崗位去另一個崗位做.
不是純粹這樣蒙召.
蒙召很明顯是上帝在你生命當中.
他正在做那個他.
他正在榮耀上帝給他創造時的特質.
待會再說特質.
耶利美德馬上就說.
他說「糟糕了,主爺說我還年輕」.
「不知道該怎樣說話」.
因為他知道要做列國先知.
我不說列國先知了.
我們快了.
他馬上說「我年輕,不知道該怎樣說話」.
再下一章.
耶利美德馬上就說.
「你不可以說我還年輕」.
「因為我差遣你到哪裡去,你都要去」.
「無論我吩咐你說甚麼,你都要說」.
「不要怕他們,因為我與你同在」.
「要拯救你」.
耶利美德馬上這樣說.
於是耶利美德馬上伸手摸他的嘴對我說.

$^{161}$「要知道我已經將我的話放在你口中」.
這個召命其實很隱蔽.
你聽到這個召命,基本上不會喜歡.
所以我很少聽到仇遠老師.
用這個經文來呼召人,服侍.
說明了甚麼?.
即是我要吩咐你說的東西.
你不要怕他們,因為我要你同在.
要來拯救你.
你一聽到就會覺得.
誰有人去呼召人的時候說.
「你很糟糕的,我叫你做甚麼你就做甚麼」.
「你做的時候,很多人會加害於你,得罪你」.
「不過你不要怕他們,因為我要你同在」.
「我一定拯救你」.
你不會用這些說話來恐嚇別人.
你通常會說「是,是逢真好」.
「是逢沒有喜樂,沒有恩典」.
「嘻嘻哈哈,很開心」.
總之你歡迎召命.
你很少聽到召命是說.
「你很糟糕的,我要你說甚麼你就要說甚麼」.
「但不要緊,我會拯救你」.
意思是你大難臨頭的時候.
他會來救你.
如果你看耶利米生平,是很糟糕的.
他在井裡被人捉了.
然後被埃及人關起來.
基本上他被王追殺.
神也拯救他.
所以沒有人喜歡一個召命.
是被追殺,被困住.
還有被困在井底.
你試試今天頂智梅蘭.
今天年三十,明天年初一.
「蒙召」.
「蒙召」是甚麼?.
由年初一至年初十.
就困在哪裡.
你問「這是甚麼?不過不要緊」.

$^{201}$「神拯救你,神與你同在」.
「Amen, Amen」.
就是呼召.
我相信沒有人會喜歡這樣的召命.
這個召命其實是一個很恐怖的召命.
就算教會做傳道人.
你記得傳道人哄你做侍奉的時候會怎樣說.
一定是「滿有恩典」.
「上帝的慈愛很豐富,你經歷它吧」.
很少說「我會拯救你」.
如果他說「你去做這個侍奉吧」.
「但我會拯救你」.
你想想.
「這個甚麼侍奉需要人拯救的呢?」.
所以這些基本上不受歡迎.
亦不太多人說.
最後這一句才是最恐怖的.
「今天我要立你在列邦列國之上」.
「我要拔出拆毀毀滅傾覆」.
「又要建立雨災眾」.
我當然要花多點時間說這句.
我不明白這句話是說甚麼.
其實我想了很久.
「我要拔出拆毀毀毀毀傾覆」.
你試試去InfoGoogle找潘Sir.
跟潘Sir說.
潘Sir問你「你為甚麼來Fold Church」.
「我要拔出Fold Church」.
「拆毀Fold Church」.
「毀壞和傾覆Fold Church」.
你馬上就有目者收留你.
但這句話是沒有的.
你說「不是,這是神對我呼召」.
「耶利米斯一張這樣說的」.
這些經文是有點瘋狂的.
不說也不明白是說甚麼.
沒有人會突然說.
「我的角色就是拔出毀壞」.
「還要傾覆和拆毀」.
如果原文不解釋.

$^{241}$你當然沒時間說是甚麼.
甚麼都快.
所以他說「又要建立雨災眾」.
你說「我要建立災眾」.
我明白.
你跟潘Sir說.
「我們Fold Church」.
「是要建立雨災眾」.
是可以的,你會點讚.
你不會說頭四個字詞.
我想我們想想.
這些說話是甚麼.
我想花點氣力想想.
「拔出拆毀毀壞傾覆」是甚麼.
對我們來說是甚麼意義.
最近我認識一個青年人.
認識他十多年.
最近教會的團契.
有人拍拖,年輕的.
十多歲的.
他拍拖.
我認識十多年的青年人.
他走來跟我說.
他說…差點露餡.
他說「啊」.
他跟我說「啊」.
他應該會稱呼我.
不過不說了.
他稱呼我「啊」.
他說「我也很想拍拖和結婚」.
嘩.
你明白我多震撼嗎.
他無緣無故見到.
他的團契裡有個大他兩年的人.
他在拍拖.
他走來跟我說.
「爸爸」.
對不起,女兒.
你這封利是今年大封一點.
真是的.

$^{281}$我們可以剪掉.
剪掉.
他就說「我也很想拍拖和結婚」.
我第一反應就是.
他團契裡有甚麼男生我認識.
我馬上反應.
他團契裡有甚麼男生的樣子.
是可以的.
我很緊張.
在他進入這個階段之前.
已經有好幾個壞人.
現在已經傳到別人身上.
是女生,三十歲附近的.
很討厭的,她經常跟我說.
「你那個青年人」.
「他拍拖的時候,你一定不知道」.
「你不用知道就知道」.
我馬上大罵他.
「我做青少年,工作起家」.
「他的樣子動來動去」.
「他化妝時,或剪頭髮時」.
「怎樣弄都知道他想拍拖」.
「我都不知道,你放心」.
「他一定不會讓你知道」.
在那裡衝擊之後.
我上台問他喜歡甚麼人.
想試探一下他.
但我知道我說完這句後.
他一定會失手.
他一定會閉嘴不說話.
我很聰明地說.
「這些事很普遍」.
嘩,拿著心說.
「真的很普遍,沒所謂」.
說完這句後,他很後悔說這句話.
之後他又說了一大堆話.
終於我失手了.
「那你現在有甚麼人嗎?」.
他馬上就走了.
我知道我失敗了.

$^{321}$我已經後悔了.
我下次應該會再機靈一點.
為了套他的料.
我真的要用豬一般的智慧才能解決.
我之後在想甚麼呢?.
我之後在想.
如果他真的這樣做.
我馬上想,他會不會懂得希伯和Greed?.
應該不會吧?.
那我怎麼做?.
我真的不行.
我心想,如果他真的有甚麼.
最近我們「木者」群組.
又在討論這些課題.
這兩天這麼刺激我.
即是那些坊間的人.
即使是基督徒信義代.
都很喜歡做那些甚麼的.
不要說了,新年樓樓.
然後找這些人來刺激我.
我心想,如果他真的這樣做.
我會否接受?.
我問我行不行?.
九成九不行.
問題不是你行不行.
而是有沒有問題.
如果有,那又如何?.
難道你已經行不行?.
我那一刻開始想.
拔出拆毀毀壞傾覆.
我給他戴隱形眼鏡五,六年.
希望他變得漂亮一點.
你明白嗎?.
希望他的牙齒變得漂亮一點.
將來會好看一點.
原來都是拆毀,拔出,毀壞和傾覆.
我心想,怎麼辦?.
原來我沒有分享.
是被別人分享了.
我心想,怎麼辦?.

$^{361}$我終於想到了一件事.
我突然問自己.
我小時候是怎樣的?.
哈哈哈哈.
是的,我小時候也是這樣.
不是,我小時候是很好的.
我小時候也是有這些事情.
直到今天.
我突然想起.
原來在我的世界觀裡.
凡是拆毀,毀壞和拔出的事情.
好像在我心中要刪除了沒有是好的.
我們不太容許.
有些事情正在拆毀,拔出.
有些事情不知道什麼時候結局.
我們不習慣.
我們習慣的是很盼望.
很有「耶」.
假設,如果你的一段關係在去年失去了.
親人,男朋友,女朋友失戀.
當作是遺禮.
你覺得沒有.
是什麼?.
你可以想起一件事.
在拆毀,拔出,傾覆的時候.
那些事情對你有什麼意義?.
純粹是快點結束拔出,傾覆,拆除的事情.
有新的開始,我都忘記了.
還是我們正在拔出,拆毀,傾覆的事情.
我們正在跟它一起走.
這件事是在幫助我.
2023年對著香港教會前的路的時候.
我覺得起碼對我有意義的事情.
原來拔出,拆毀,毀滅,傾覆.
在上帝的手中不是不好的.
我們的盼望不要太賴意.
覺得建立建立,再重再重.
就像過去30年教會一樣.
分堂分堂,直堂直堂.
封陰封陰,傳陽傳陽.

$^{401}$哇哇哇,人數多多多.
我們覺得人數少不好.
我們覺得沒有建立,沒有再重的不好.
我們看著拆毀,毀壞的事情.
我們覺得差勁.
我為什麼不容許一個正常的青年人.
有正常青年人的經歷.
那些經歷就算在我眼中是.
拔出,拆毀,毀壞.
又有何干呢?.
那不是我隨便他,不理他.
不是那個問題.
有些事情是他要走過的時候.
我正在點滴這個走過的過程.
很想閉上眼,快點轉身.
哦,明天有女朋友,你要yeah.
還是拔出,毀壞,傾覆的過程.
其實是有意義的.
我們再看漢建,因為要講漢建.
我們看下一張powerpoint.
這裡有兩個意象.
這兩個意象無聊到爆.
這兩個意象很無聊.
他說「爺爺,你別說」.
「爺爺,你看到什麼」.
「我看到這個扁桃樹的枝子」.
「你對我說,你看得不錯」.
「因為我一直留意我的畫」.
「使它成就,完全九不搭八」.
這兩個意象其實你不知道在做什麼.
你看著那東西,看著枝子就說你看得不錯.
「因為我一直留意我的畫」.
「使它成就,都無關」.
唯一關係的是原文.
「扁桃樹」和「留意」這兩個字.
在原文上,三個字母是一樣的.
我不說複雜的,總之類似一樣的東西.
但意思不同.
他想表白的是.
你留意著扁桃樹.

$^{441}$就好像留意著上帝話語一樣.
意思是,這個意象是什麼意象.
是一個意象是說.
你不需要做事的意象.
我人生最期盼的意象是這個意象.
如果上帝召喚我,我不用做.
我只要留意著他的畫,成就就行了.
即是你「Watch over」.
英文就是「Watch over」.
你看著那棵樹的枝子.
就等同於看著上帝的畫.
因為他會成就.
意思是什麼.
就是看著一些毀壞的東西.
繼續毀壞下去.
看著那些已經傾覆了的東西.
它不斷地傾覆下去.
很熟悉吧.
和我們的處境一樣.
和我們面對的一樣.
拆毀了的東西,毀壞了的東西.
你現在只看著每天都被拆毀下去.
那些傾覆了,拔出了的東西.
到現在這一刻仍然在拔出來.
沒有新意.
這個意象,這個召命是說.
你看著那些東西,毀壞到最後.
我心裡想,這些是什麼呢?.
我在想了很久.
其實這個意象是什麼?.
這個漢建是什麼?.
上帝的話語會成就.
所以那些假先知在耶利米蘇52章裡.
不是經常說巴比倫不會傾覆嗎?.
而耶利米蘇就要玩狂顏.
耶和華說會傾覆.
他們就看著耶路撒冷會傾覆.
然後那些假先知又如何?.
全部不成功.
他們不需要做任何事.

$^{481}$因為他們沒有事做.
被抓,被抓,被打.
他們寫了一個要這樣寫的東西.
寫了出來要讀給大家聽.
然後被黃竹抓.
那些書卷被燒.
但他們的責任就是守住.
上帝給他們的說話.
他們一直說下去.
我們再看下一個意象.
耶和華說要臨到我.
這是第二個意象.
他說你叫什麼?.
我按到一個沸騰的鍋.
鍋口從北面向南傾側.
耶和華對我說.
所以我從北方發出.
臨到這裡所有的居民.
耶和華宣告說.
我要照北方列國的宗族.
他們要來在耶路撒冷城門口.
國安補助.
攻擊周邊的城牆.
又攻擊猶大的眾城邑.
要向他們宣告審判.
因為他們的一切惡行.
就是離棄我.
向別臣銷獻祭物.
向自己手上的偶像下拜.
這個信息基本上很難說.
對不起,我今天也不懂得說.
等於如果今天我說.
香港教會玩完了.
你明白嗎?.
他的意思是這樣.
他想說的是.
玩完了,不要再有期盼.
北方而來的事情是怎樣.
他們列國會走過來.
然後神會向你們這些人審判.

$^{521}$其實就完了.
所以最終耶路撒冷被毀.
城牆倒塌.
就是結局.
雖然耶利美哀歌出現.
就是因為要哀哀.
耶利美被毀.
漢建的整個結局被毀.
是耶利美要做的事情.
剛才第一個意象漢建的.
也較為抽象地說.
但第二個意象漢建的是甚麼.
說得很清楚漢建的是.
就是毀壞到最後.
我在想甚麼是.
看著毀壞到最後的靈性是甚麼.
我會生出甚麼靈性.
或需要甚麼靈性才能面對這件事.
我們經常想說.
面對負面的事情.
我們就說要正面.
是嗎?.
我們需要有正面的心理學.
正向教育.
現在很多學校都在說正向教育.
畫畫之類的.
不知甚麼畫.
經常畫的.
然後就說正向.
我不是想進入討論.
但我想表達的是.
我們對負面的事情.
對拆毀的事情.
我們不喜歡.
我們不想見到.
你想想熱戀的時段.
還是想想甚麼.
你分手的時候的感覺.
你一定要快點理解分手的感覺.
回到熱戀中.

$^{561}$我被人愛開心.
沒有人喜歡一直站在.
被拆毀的過程中.
如果你說2023年要說一件事.
是我這一刻最想說一句話是甚麼.
我最想說的是.
我很怕,很怕或恐懼.
香港教會回復正常.
我很怕香港教會.
沒有甚麼令之後.
很多信徒,領袖,姐妹會覺得如常.
這是我2023年對香港教會最擔心.
不知如何說出口的一番說話.
我們的問題很多.
香港教會的問題很多.
過去十年,二十年浮現出來.
多到一個地步不能再多.
疫情過後.
我們奉獻少了.
人數少了.
現在沒有這些令之後.
大家期望只是奉獻多一點.
人數多一點.
滿足就夠了.
而在這幾年裡所呈現的問題.
突然好像完全忘記了.
已經夠慘了.
人數跌了三成.
奉獻跌了三成.
已經不知如何是好.
只要一指洩.
感覺一下就好了.
但我寧可香港教會繼續被拆毀.
我寧可香港教會繼續經歷拆毀時.
我們覺得痛苦.
我們覺得需要上主的恩典.
我們不知道拆毀到何時.
不知道拆毀到何時才足夠.
在那種感覺中.
我們開始找上帝.

$^{601}$上帝說「你何時會來?」.
我們開始會想變化.
我們開始會想一些.
做一些以往不常做的事.
做一些新的事.
在上帝新的時代.
在拆毀的裡面.
不如先是上帝在拆毀裡面.
祂會做的事.
恐怕我們很快會窩蜂走回以前.
用以前的方法.
用以前的伎倆.
用以前覺得能夠安心立命.
賴以為生的生存技巧.
以示用技術.
就能夠回復一切.
我怕是.
既然開始醞釀土地的時候.
人開始想一些新的方法模式.
做一些事的時候.
突然間這些東西被打壓.
突然間這些東西的苗頭生長不了.
人心開始轉向一些.
以往他們熟悉的東西.
那些叫建立和栽種的東西.
留山的頂姐妹.
你在外國.
在不同的地方.
我知道很多地方在搞年宵.
在吃角仔.
在吃年糕.
但我期望的頂姐妹.
去到新的地方裡面.
嘗試一下甚麼拔出了.
甚麼毀壞了.
從而看到上帝在那裡建立了甚麼.
我更期望.
全面開放教會的頂姐妹.
我們不是一間教會轉去另一間教會那麼簡單.
我們期望問自己.

$^{641}$問教會的是.
傾覆到何時.
拔出到何時.
在拔出和傾覆的時候.
甚麼才是上帝在這個時候會做的事.
還是我們看到大陸的氣氛都可以的時候.
我們的思維就回到最正常的.
用以往的想法去想教會上當然有的事.
那種被上帝逼我們改變的氣候沒有了的時候.
我們開始沒力.
去看上帝在這個新的時代裡面.
他做了甚麼新的事情.
最後.
零五分,下一張.
「因此你會蓄腰」.
我那句,他說.
第三行,「而我今天使你堅誠」.
「鐵柱和牆可以抵抗全地」.
「抵抗猶大的君王,首領,祭司和國民」.
其實這個使你成為就等同於.
第一版的《Powerpoint》裡面.
他在武福裡面使你成胎而先的那個使你.
他做你的照明,他還做了你甚麼.
他還做了你甚麼.
你是堅誠,鐵柱,銅牆.
原來要在一個新時代裡面.
看著一些事情傾覆毀壞.
你裡面要多堅,多鐵,多銅.
最後說這個例子,我完了.
我跟她說,這年多有「Panic attack」.
由我很不想她來.
由我到她來,我控制不了.
由她來,我說要跟她做朋友.
我最近有一個很變態的心態.
當她每一次快要來的時候.
一次很搞笑,她現在來的形式很可愛.
她三,四點鐘便醒來.
然後她便來.
一來的時候,我下意識很緊張,很害怕,很惶恐.
抱著老婆,老婆每一次完結後都很溫馨地跟我說.

$^{681}$「你昨晚真的很好,三,四點鐘抱著我睡」.
我跟她已經告訴了,其實與她無關.
其實我很無奈.
不,順便說.
我最近開始覺得她來得太快,太多.
我開始有種傻傻的心態.
「你來吧,歡迎到來,你想怎樣便怎樣」.
我懷疑面對傾覆,毀壞,拆毀的時候.
尤其是屬於這個時代的事情.
歡迎她來是一種靈性.
是一種純粹的操練.
當你歡迎她來的時候.
你已經準備好.
面對一些事情來的時候.
你用甚麼心態去處理.
唯有這樣才能抵禦.
我們以為我們很快會回到正常.
當這個社會,這個時代.
某些人作為主旋律出來說.
一切如常,開心,滔氣揚眉的時候.
上來還做著一班人.
在這個世代裡面.
學習甚麼叫做「傾覆」.
如果潘卓華在這個年代當中.
面對著國家教會,全政投入,納粹主義.
潘卓華怎樣看傾覆?.
他自己開了一個小團契,叫做團契生活.
在那裡學習.
看著一些事情在傾覆.
看著一些事情在突出.
而去學習那個屬靈的操練.
避免自己走入別人那些邪惡的主旋律當中.
Flo Church領事妹.
元爾塞利安這個聖餐.
我們看看耶穌基督釘十字架死.
看看他這個釘十字架的傾覆.
讓我們學習.
他一生在地上的日子.
就是面對著自己要走上十架.
原來不是要走上十架.

$^{721}$被拔出,被傾覆.
才是等候救恩.
真正的兆頭.
求主使我們在這個年代裡面能夠清醒.
使Flo Church領事妹清醒.
我們不是來到這裡當中.
做一些很像以往做的事情.
在傾覆,拆毀的時代裡面.
學習一個什麼的靈性.
以致面對前面新的世代.
上帝會做新的事情.
我一齊低下肚子.
天父多謝你讓我們今天有這個空間時間.
我們一齊來到上帝的面前.
讀你自己這段話語.
堅誠銅牆鐵柱.
一聽就害怕.
就算你說你同在,你保護,你拯救也好.
心靈中是膽怯.
但求天父你.
請我們承認我們不行.
我們不習慣,不喜歡這些.
所以我們一齊來到你面前去呼求.
你幫我們可以在新時代裡面.
做新的事情.
求天父你自己在當中親自與我們同在.
帶領我們每一個人.
多謝天父你聽我們面前的禱告.
奉你出於保衛民教.
阿門.
\newpage



\section{}
\label{sec:6BvsvxnAGsQ}
\textbf{【流堂崇拜】兔年生肖道|20230128 [6BvsvxnAGsQ]}
\newline
\newline
連結: \href{https://youtube.com/watch?v=6BvsvxnAGsQ}{\texttt{ https://youtube.com/watch?v=6BvsvxnAGsQ}} ~~~~ 語音日期: 2023-01-28 
\newline
\newline
\hyperref[sec:RPsjvP8W4CE]{\small{< < < PREV SERMON < < <}}
~
\hyperref[sec:index_chronic]{\small{[返順時目]}}
~
\hyperref[sec:index_scriptual]{\small{[返順卷目]}}
~
\hyperref[sec:GtMQusxSoOU]{\small{> > > NEXT SERMON > > >}}
\newline
\newline
$^{1}$弟兄姊妹平安 新年快樂 新春大吉.
在大年初七裡 恭祝留堂弟兄姊妹.
靈性進步 身體健康 主恩滿日.
如果你最近才回流堂的話.
你也知道我有一個很倔強的嗜好.
每年都會講一篇新潮道.
我自己是在羊年的時候開始講的.
所以也是在2015年的時候開始講.
羊 猴 雞 狗 豬.
2019年就開始在Fall short講.
就開始講鼠 牛 虎 今年就是兔年.
所以算起來我還差龍蛇馬三隻還沒講.
應該到2026年我就會講完12個.
不過我發覺今年是第一次在Fall short現場講新潮道.
你有沒有留意到.
在選年2020年的時候.
那時候我在美國 大年初一.
所以拍了片回來.
之後就COVID了.
很多事情發生.
一年牛年在MFC那裡都是沒有人的場.
都是一個清的場.
去年虎年也是只有十幾人.
小組就一起看.
今年是第一次有現場的第一節目.
我們一起去參與.
來到第四年的時候.
因為大家都沒有去旅行.
去不了旅行.
不過如果你有機會去的話.
你去日本的話.
假設你去一個地方.
去一個廣島.
廣島有一個無人島.
叫做大九野島.
有沒有聽過.
大家有沒有去過.
這個大九野島是一個兔子島.
以前因為日本是一個毒氣島.
二戰之後就做了一個毒氣工廠.

$^{41}$後來就重建了.
人們就帶了八隻兔子進去.
開始重建了博物館.
誰知道這八隻兔子就開始繁殖.
不斷地生產一千隻兔子.
所以整個島都是兔子.
大家有機會就可以去.
所以預備今年的兔年新少島的時候.
我開始留意一下.
在地球其他國內的人.
如何預備兔年新少島.
我就找到這個.
就是一篇這樣的港島大綱.
大家有沒有看到.
他用了三個point的商文.
就是together.
和we live together.
今年我們就不會說這些.
不會說食字.
會說兔子.
希望我們都嘗試去說一些.
兔子對我們一些的屬靈反思.
和一些意義.
聖經裡面有沒有兔子呢.
聖經裡面有兔子.
今年總比去年好.
去年苦練的話.
記得是沒有老虎.
我們唯有用了虎口去說.
今年主要有兔子.
哪裡會說到兔子呢.
只有兩篇文章有兔子.
這篇文章就在利美記第十一章.
和生命的十四章經文.
其實兩篇都是差不多.
都是說同樣的內容.
我們一起讀.
一段就是利美記第十一章經文.
我們一起讀.
預備一二三.

$^{81}$(念經).
但道長或分派之中,得我赤誠大師樂德旺,.
因為道長不分派,就被你們蒙特捲頸;.
但因為道長不分派,就被你們蒙特捲頸;.
導致因為道長不分派,就被你們蒙特捲頸;.
至於因為道長不分派,就被你們蒙特捲頸;.
.
就是這樣,一段在摩西五經的經文,.
耶和華上帝透過摩西向以色列人講解.
潔淨不潔淨食物的經文,.
哪些動物是潔淨,哪些是不潔淨,哪些可以吃,哪些不可以吃,.
這是耶和華上帝對以色列人的律法的頒佈,.
以色列人的信仰,以色列人的身份和生活方式..
兔子,駱駝和沙蕃,沙蕃就是石泥,.
都被列為不潔淨的動物,.
兔子被聖經列為不潔淨,其實是nothing personal,.
不是什麼邪惡的地方,也不是上帝針對,不喜歡他..
事實上兔子被視為不潔淨,純粹是生活上或衛生上的理由,.
不是因為宗教的理由..
為什麼有些律法說不可以吃兔子?.
這裡對於食物的規條有一定程度的規定,有兩個條件,.
第一是分題,不是吃那些,.
就是腳不要分開兩邊,這樣分題..
第二就是倒著,或者叫反措,.
凡是可以吃的動物,都必須符合這兩個條件,才可以吃,.
要同時間符合這兩個條件才可以吃..
所以分題是什麼?.
如果分題不是反措的話,譬如豬,.
腳分開,但胃不是反措的話,就不可以吃..
倒著,但不是分題的話,譬如兔子,.
也不可以吃..
什麼叫倒著?什麼叫反措?.
就是這些動物把食物吞下去,.
到胃部開始消化輪,再吐出來,再咀嚼,再吞下去..
所以這是一個雙重的消化過程,.
稱之為倒著,或者叫反措..
牛羊也是反措的動物,.
他們的胃有四個部份,.
把草吃完之後,不斷消化,再吐出來,再咀嚼,再消化,這樣就叫反措..
至於兔子的倒著,雖然兔子沒有四個胃,.

$^{121}$但他的食物都是一些草和植物,.
或者紅蘭花,跟牛羊也是相似的,.
也是類似反措的動物,.
不過其實不是反措,.
當他吃的時候,他就會含著一包腮,.
一邊跑,一邊含著,.
就在這裡跑,這樣,這樣,這樣..
這是一個過程,.
所以當他跑回洞裡的時候,.
才吐出來再吃,.
所以這個不是倒著,.
不過以前是一個倒著過程,.
所以叫做類似倒著的動物..
不過無論如何,雖然兔子是反措的動物,.
但因為兔子的腳不是分體,.
所以也不可以吃,.
所以被列為法律上不可吃的動物..
所以以色列人不吃兔子不是因為可愛,.
我知道你們在等我講這句,.
兔兔這麼可愛,怎麼可以吃兔兔?.
OK,我講了..
不過以前的以色列人不吃兔兔不是因為可愛,.
因為他是一個不是分體的動物..
不過可能大家不知道,.
兔子是全球僅次於雞,鴨和豬,.
是全世界被吃得最多的動物..
全球每年有11億隻兔子被宰殺成為食物..
所以想想天堂裡的兔子,.
如果有兔子的話,.
每年有11億隻兔子..
香港人比較少吃兔子,.
我們不是那麼習慣吃兔子,.
平時在餐廳裡不會吃到兔子,.
薑蔥兔子飯沒有,.
沙嗲兔子麵也沒有,.
海南兔子也沒有..
不過我要confess,.
我吃過一次,.
當年在讀書的時候,.
大概12年前,.

$^{161}$聖誕節的時候,.
我和老婆去了捷克旅行,.
去布拉格玩,.
到了歐洲逢放假就去玩,.
我們預訂了一個修道院的酒店,.
以前是修道院,.
變成了一間酒店,.
一個很古式的古堡餐廳,.
不是很高級的餐廳,.
是很典型的歐洲式的老牌餐廳,.
餐牌裡面有rabbit這個字,.
我覺得立刻要試試看,.
喝完湯之後,.
服務員就帶上一碟兔子的尾巴,.
雖然沒有寫明是兔子的尾巴,.
但一看就是尾巴,.
尾巴的形狀剛剛好,.
應該是左尾巴..
我要補充一件事,.
當年是2011年,.
無線播放《天與地》,.
林寶怡,陳豪的《天與地》,.
當我吃的時候,.
我覺得自己在吃家明一樣,.
我口裡一咬,.
我感覺到那肚子的肌肉彈得很厲害,.
那種力度是很強的,.
那肌肉的力度,那種能量,.
一咬下去,.
整個人感覺到有兔子在肚子裡,.
是一個很難忘的經歷,.
一個非常咀嚼的口感,.
那晚我就做夢了,.
在酒店裡面,.
我想說的是,.
兔子的腳是很強勁,.
而且是一個跑得很快的動物,.
這就是我們今天講道的重點,.
大家都聽過龜老賽跑的故事,.
兔子和烏龜賽跑,.

$^{201}$因為兔子驕傲,.
跑到下午睡覺,最後輸了比賽,.
所以故事說,.
做人不可以驕傲,.
因為如果你驕傲,你就會輸,.
我覺得這故事是很誤導小朋友的,.
我覺得問題不是驕傲,.
誰說驕傲會輸呢?.
誰說會跑得慢呢?.
不是的,驕傲跑得很快,.
你看到奧運冠軍選手,.
全部都很驕傲,.
戴著黑超金鏈,.
跑完之後,.
戴著一副站著的帽子,.
有自拍,.
所有人都很驕傲,.
所以其實驕傲是可以的,.
驕傲也可以很快的,.
所以兔子的問題不是因為驕傲,.
我認為,.
我卑微的意見,.
in my opinion,.
我錯了,.
兔子輸的原因不是因為驕傲,.
而是因為眼睡,.
驕傲歸驕傲,.
眼睡歸眼睡,.
是兩回事,.
那套材料告訴他,.
你跑步,驕傲是可以的,.
但你不要跑去睡,.
這些事很離譜,.
跑一跑,他就睡了,.
很過分的,.
你何時見過驕傲人會睡覺,.
我夜晚,就睡了,.
不會的,.
兔子跑一跑,.
睡覺不是因為驕傲,.

$^{241}$而是因為眼睡,.
還不是普通的眼睡,.
是病態的眼睡,.
跑一跑突然間睡了,.
是一件很嚴肅的問題,.
上面寫著,.
發作性昏睡病,.
是真的,.
大家有沒有看過,.
1991年有一套電影叫,.
My Own Private Adderhall,.
叫北極的天空,.
沒有看過,.
但很舊的,.
是在奇爾維斯和Villa Phoenix,.
製作的一套電影,.
戲中Villa Phoenix,.
就是這樣,.
有發作性昏睡病,.
跑一跑,突然間睡了,.
所以,.
這個故事裡面,.
兔子是一個病態的病人,.
烏龜只贏了一個嚴重性,.
有發作性昏睡病的兔子,.
勝之不武,.
正常的兔子,.
是不會輸給烏龜的,.
Q毛也不會輸給烏龜,.
如果你不相信,.
我可以show video給你看,.
兔子非常快的video,.
讓你看看,.
厲害吧,.
轉播一下,.
突然一下子,.
影片就轉過了,.
兔子真的很快的,.
所以大家聽過一句話,.
正如處子動若脫兔,.

$^{281}$什麼意思?.
什麼叫做正如處子動若脫兔?.
在行動中,.
行之前就像一個未出嫁的女孩一樣,.
一樣沉靜,.
一旦行動起來,.
就像一隻逃脫的兔子一樣,.
非常快速穩健,.
正如處子動若脫兔,.
脫兔這個字是很可圈可點的,.
脫兔成為了兔子最強的狀態,.
當一個形容詞加上動物,.
成為一個名詞,.
不是隨便的事,.
全部都是值得我們留意的,.
例如肥豬,.
猛虎,巨龍,脫兔,.
全部都很貼切,.
還有和牛,.
脫兔是非常值得我們留意的狀態,.
一個逃脫中的兔子,.
是100\%潛在的狀態的兔子,.
所以我越來越欣賞兔子,.
兔子未必是全世界最快動物,.
但仍然是全世界最擅長逃脫的動物,.
逃脫所講求的不單是快,.
更加要有頭腦,機警和靈活,.
兔子的眼睛是睜得很開的,.
所以視野很廣闊,.
甚至乎他能看到後面,.
他擅於奔跑,.
他的鼻孔很大,.
充分有足夠的氧氣,.
讓他能夠不斷跑步,.
他在急速轉彎時,.
馬上追不到那些捕獵者,.
所以如果養兔子,.
兔子很少眨眼,.
因為他睡覺時也會睜開眼睛,.
讓他能夠知道四周的環境,.

$^{321}$如何應付突如其來的危險,.
甚至動物學家發現,.
兔子經常有一個逃生路徑,.
當他被捕獵時,.
馬上可以找到路線逃跑,.
從來沒有被迫入牆角,.
所以兔子是天生專業的逃跑好手,.
讓我們在兔年中學習,.
做一隻動如脫兔的基督徒,.
做一間動如脫兔的教會..
當我預備兔子的資料時,.
我看維基,.
維基有一句說話很震撼我,.
他寫了什麼呢?.
Rabbits are prey animals,.
什麼叫做prey animals?.
兔子是一個被捕獵的動物,.
稱之為獵物,.
我覺得這句說話很殘忍,.
原來在世上有兩大類動物,.
一類叫捕獵者,.
一類叫獵物,.
妖魔就是捕獵一些動物的動物,.
妖魔就是被捕獵的動物,.
所以兔子是一個最典型的prey animal,.
是被人捕獵的,.
你看到澳門的跑狗,.
有兔子在追牠,.
我們今天身處的社會,.
我們似乎都被界定成為別人的獵物,.
我們身為教徒,.
我們都不選擇做捕獵者,.
不過原來這個時代,.
這個地方告訴我們,.
當你不選擇去做捕獵者的時候,.
你就會成為捕獵者的獵物,.
這幾年我們經常被別人成為別人的對象,.
我們成為別人眼中的target,.
被統治的對象,.
被和諧的對象,.

$^{361}$國家安全穩定的對象,.
兔子行動機敏,身型靈巧,.
能用極快的速度越過前行路中的重重障礙,.
逃避危險和災難,.
春節以至,.
我們戰勝了剛剛過去的苦年裡種種風光路急,.
完成了艱難繁重的任務,.
迎面而來的是全新的機遇,.
當然也可能有風險挑戰,.
面對機遇和挑戰,.
踏上新的征程,.
我們要動如脫逃地奔跑,.
這段是我們國家偉大共產黨年初的賀詞,.
你看我們說得多好,.
你什麼時候叫人玩捉迷藏呢?.
捉那個叫你跑快一點的,.
我就會捉你,跑快一點,.
沒有的,是吧?.
不過雖然兔子被歸類為獵物的動物,.
但牠是天生professional,.
善於躲避暴獵者的動物,.
牠的速度,靈敏,聰明機警,.
再加上牠的逃脫意識,成為了一隻脫逃,.
不過不要誤會,脫逃不是叫你去移民,.
不一定是移民,.
就算你今天離開香港頂尖都好,.
你都仍然要開啟你的脫逃模式,.
因為我們所面對的不單單是地上的惡者,.
更加是天空中撒旦魔鬼的暴獵,.
逃避雖可恥,但有用,.
何況在上帝眼中,逃避絕對不是可恥的事情,.
所以裡面有很多很多逃避的經文在當中,.
我們可以看一看,.
(主席發言).
所以裡面有很多很多有關逃避,逃跑的經文,.
特別是最後一句,.
蓋巴塔的那句,.
所謂少年的恕辱不單單是情慾上的事情,.
更加是我們人心裡的苦毒,.
我們的紛爭,驕傲,妒忌等等,.

$^{401}$面對一些邪惡的力量,.
面對自己心中的惡念,.
地上邪惡的暴獵者,.
或者天空掌權的試探等等,.
原來聖經教導我們,.
你面對的方法就是逃離,.
聖經從來都沒有教導我們擊敗它,.
它沒有教導我們擊敗邪惡,.
或者要完全勝過那個試探,.
聖經叫我們快點離開,.
為什麼呢?.
因為第一,耶穌在十教裡面已經戰勝黑暗,.
不是你贏,是耶穌贏,.
第二,你根本沒有辦法贏,.
聖經教導我們一個最重要的動詞,.
一個Biblical的方法,.
就是逃離,馬上離開,.
不知道你面對著什麼事情,.
面對著自己心裡很多的罪,.
或者面對著很多的試探,.
上帝叫我們快點遠離它,快點走,.
剛才唱歌很好,.
我們要有Goodness of God,.
我們要追隨,.
被Goodness of God吸引,.
同時間不斷地逃避,.
逃跑一些不好的事情,.
黑暗的事情,.
敗壞的事情,.
所以做這件事很簡單,.
你要麼就逃避,.
要麼就被上帝吸引,.
真的,面對自己很多問題的時候,.
你今天有沒有好好地逃避它,.
逃避是一個Active的逃避,.
如果你開車,我們叫Defensive Driving,.
即是你要積極地逃避一些可能有意外的事情,.
面對著地上的惡人,.
面對著自己心裡很多的試探,.
面對著魔鬼殺蛋對我們的傷害攻擊,.

$^{441}$你要很認真地逃避它,.
好像一隻脫套一樣,.
你要很有意識地逃避它,.
實際上基督徒是天生的逃避高手,.
2000年前我們的初教會裡面的先人先聖都是一樣,.
他們是靠著逃避,靠著警醒,靠著躲起來,.
來起家的,.
不單單是政權,更加是魔鬼殺蛋的攻擊,.
我很少說魔鬼殺蛋,.
但是這個從來都存在,.
無論是很多很多的試探,.
很多罪的困難,.
特別是弟兄,.
我們很好的逃避這些試探,.
23年做一隻醒醒目目,.
醒醒定定,聰明機警,行動敏捷的脫套,.
面對著很多不同的攻擊,.
很多不同的暴力,.
我們能夠逃避它,.
在這個年頭裡面,.
作為香港人,作為基督徒,.
我們一起打開這個脫套模式,.
好好地過這一年,.
學習做一隻這樣的兔子,.
很多人不知道,其實Full Church的logo,.
那個F是故意細階的,.
大家有沒有留意?.
我們從來都沒有大階的F,.
為什麼?.
因為我覺得這是我自己的意願,.
因為細階的F比較輕巧,.
我覺得比較flow,.
就是這樣,.
細階的F比大家更加輕巧,更加flow,.
就像兔子一樣,.
我們不需要留,不需要大大份,.
我們不是成為一個洪水猛獸,.
我們要成為像兔子一樣,.
一隻這樣的,.
一隻很靈活流動,.

$^{481}$很容易轉身的脫套,.
這個就是我們教會裡面的模式,.
最後一本書跟大家分享,.
一本書叫做The Rabbit and the Elephant,.
Why small is the new big for today's church?.
一個很有意思的一本書,.
說到為什麼小能成為真正的大,.
對於今天的教會來說,.
一個很好的比喻書裡面,.
他說如果你放兩隻大象在房間裡面,.
關上門,.
22個月之後,.
當你打開門,.
你發現什麼?.
你會看到一隻新的大象BB在裡面,.
就BB了,小朋友了,.
不過如果你將兩隻兔子放在房間裡面,.
關上門,.
22個月之後,.
當你打開門,.
會怎樣?.
你會看到一隻大象的大便,.
英文有句諺語叫做Brit Light Rabbit,.
兔子是一個極度發展性力強的人,.
一個動物,.
六個月就可以生小兔,.
懷孕期只有30天,.
一生就生8隻,.
想一下一年可以生多少隻,.
所以兔子是一個非常非常生命力強的一種動物,.
雖然小小,.
保持我們教會一樣,.
我們裡面的小組也是一樣,.
不知道大家有沒有回小組,.
很希望每個小組都像兔子一樣,.
他們都能夠有生命力,.
雖然小小,.
但極其旺盛,.
能夠有生命力,.
能夠很機警,.

$^{521}$能夠很靈活地處於今天的香港裡面,.
不要成為魔術師抽出來的那隻兔子,.
也不要成為純粹被人養的兔子,.
要做一隻自由奔放,.
機警,.
靈活,.
快速,.
善良的兔子,.
這個是我們2023年一起來追求的事情,.
逃過試探,.
逃過很多政權魔鬼撒旦的威脅,.
保持心裡面的善良,.
擁有生命力的,.
雖然是小,.
雖然不明顯,.
但我們可以做到這樣的教會,.
今年我們會有很多這樣的事情會發生,.
我們有很多不同的,.
outreach的方法,.
希望能夠像蘇仔一樣,.
可以帶到一些未信主的人,.
帶到一些離開教會的人,.
剛剛有個調查,.
原來全港大概有70萬基督徒,.
在這個調查裡面,.
比較準確的數字,.
返教會的人是30萬都沒有,.
所以其實想想有40萬人是沒有返教會的基督徒,.
我們可以這樣做,.
可以叫他們回來,.
保持我們教會的那種靈活,.
那種生命力,.
同時間在生命裡面,.
能夠離開很多的問題和試探,.
我們一起祈禱,.
祈求主席幫助我們,.
保守我們流唐教會,.
我們能夠在套聯裡面,.
像套車一樣,.
能夠靈活,.

$^{561}$能夠可以機動,.
能夠可以聰明地,.
自由奔放地走這條道路,.
求主幫助我們,.
叫我們脫離這些試探,.
脫離這些暴烈,.
主我們知道我們當中有很多頂枝末芯,.
裡面有很多困難,.
可能他們心中仍然受到很多暴烈者的攻擊,.
無論是殺彈的一些傷害,.
或者試探,.
或者是一些人的暴烈,.
求主你教導我們,.
能夠可以像套友一樣,.
能夠避開這些試探和傷害,.
能夠擁有生命力的,.
在各社群裡面,.
在群體裡面,.
我們一起來去反擊,.
來去見證你自己,.
求主去保守我們流唐教會,.
不要被人捉,.
不要被人暴烈,.
能夠可以為你奔跑,.
逢春永久,.
阿門..
\newpage



\section{使徒行傳 3:1-10-20230204}
\label{sec:GtMQusxSoOU}
\textbf{【流堂崇拜】超越想像的視野|使徒行傳3\_1-10|20230204 [GtMQusxSoOU]}
\newline
\newline
連結: \href{https://youtube.com/watch?v=GtMQusxSoOU}{\texttt{ https://youtube.com/watch?v=GtMQusxSoOU}} ~~~~ 語音日期: 2023-02-04 
\newline
\newline
\hyperref[sec:6BvsvxnAGsQ]{\small{< < < PREV SERMON < < <}}
~
\hyperref[sec:index_chronic]{\small{[返順時目]}}
~
\hyperref[sec:index_scriptual]{\small{[返順卷目]}}
~
\hyperref[sec:KAnMTZ32Dag]{\small{> > > NEXT SERMON > > >}}
\newline
\newline
使徒行傳 3:1-10-20230204
\newline
\begin{longtable}{cl}
\hline
\hline
章節 & 經文 (和合本修訂版)\\
\hline
3:1 & \begin{tabularx}{0.7\textwidth}{X} 下午三點鐘禱告的時候,彼得和約翰上聖殿去。 \end{tabularx} \\ \\ \relax
3:2 & \begin{tabularx}{0.7\textwidth}{X} 一個從母腹裡就是瘸腿的人正被人抬來,他們天天把他放在聖殿的一個叫美門的門口,求進聖殿的人施捨。 \end{tabularx} \\ \\ \relax
3:3 & \begin{tabularx}{0.7\textwidth}{X} 他看見彼得、約翰將要進聖殿,就求他們施捨。 \end{tabularx} \\ \\ \relax
3:4 & \begin{tabularx}{0.7\textwidth}{X} 彼得和約翰定睛看他,彼得說:「看著我們!」 \end{tabularx} \\ \\ \relax
3:5 & \begin{tabularx}{0.7\textwidth}{X} 那人就注目看他們,指望從他們得著甚麼。 \end{tabularx} \\ \\ \relax
3:6 & \begin{tabularx}{0.7\textwidth}{X} 彼得卻說:「金銀我都沒有,但我把我有的給你:奉拿撒勒人耶穌基督的名起來行走!」 \end{tabularx} \\ \\ \relax
3:7 & \begin{tabularx}{0.7\textwidth}{X} 於是彼得拉著他的右手,扶他起來;他的腳和踝骨立刻健壯了, \end{tabularx} \\ \\ \relax
3:8 & \begin{tabularx}{0.7\textwidth}{X} 就跳起來,站著,又開始行走。他跟他們進了聖殿,邊走邊跳,讚美神。 \end{tabularx} \\ \\ \relax
3:9 & \begin{tabularx}{0.7\textwidth}{X} 百姓都看見他又行走,又讚美神, \end{tabularx} \\ \\ \relax
3:10 & \begin{tabularx}{0.7\textwidth}{X} 認得他是那素常坐在聖殿的美門口求人施捨的,就因他所遇到的事滿心驚訝詫異。 \end{tabularx} \\ \\ \relax
3:11 & \begin{tabularx}{0.7\textwidth}{X} 那人正在稱為所羅門的廊下,拉住彼得和約翰,大家都覺得很驚訝,一齊跑到他們那裡。 \end{tabularx} \\ \\ \relax
3:12 & \begin{tabularx}{0.7\textwidth}{X} 彼得看見,就對百姓說:「以色列人哪,為甚麼因這事而驚訝呢?為甚麼定睛看我們,以為我們憑自己的能力和虔誠使這人行走呢? \end{tabularx} \\ \\ \relax
3:13 & \begin{tabularx}{0.7\textwidth}{X} 亞伯拉罕的神、以撒的神、雅各的神,就是我們列祖的神,已經榮耀了他的僕人耶穌,這耶穌就是你們交付官府的那位,彼拉多決定要釋放他時,你們卻在彼拉多面前棄絕了他。 \end{tabularx} \\ \\ \relax
3:14 & \begin{tabularx}{0.7\textwidth}{X} 你們棄絕了那聖潔公義者,反而要求釋放一個兇手給你們。 \end{tabularx} \\ \\ \relax
3:15 & \begin{tabularx}{0.7\textwidth}{X} 你們殺了那生命的創始者,神卻叫他從死人中復活;我們都是這事的見證人。 \end{tabularx} \\ \\ \relax
3:16 & \begin{tabularx}{0.7\textwidth}{X} 因信他的名,他的名使你們所看見所認識的這人健壯了;正是他所賜的信心使這人在你們眾人面前完全好了。 \end{tabularx} \\ \\ \relax
3:17 & \begin{tabularx}{0.7\textwidth}{X} 「如今,弟兄們,我知道你們做這事是出於無知,你們的官長也是如此。 \end{tabularx} \\ \\ \relax
3:18 & \begin{tabularx}{0.7\textwidth}{X} 但神藉著眾先知的口預先宣告過基督將要受害的事,就這樣應驗了。 \end{tabularx} \\ \\ \relax
3:19 & \begin{tabularx}{0.7\textwidth}{X} 所以,你們當悔改歸正,使你們的罪得以塗去, \end{tabularx} \\ \\ \relax
3:20 & \begin{tabularx}{0.7\textwidth}{X} 這樣,那安舒的日子就必從主面前來到;主也必差遣所預定給你們的基督耶穌來臨。 \end{tabularx} \\ \\ \relax
3:21 & \begin{tabularx}{0.7\textwidth}{X} 他必須留在天上,直到萬物復興的時候,就是神自古藉著聖先知的口所說的。 \end{tabularx} \\ \\ \relax
3:22 & \begin{tabularx}{0.7\textwidth}{X} 摩西曾說:『主—你們的神要從你們弟兄中給你們興起一位先知像我,凡他向你們所說的一切,你們都要聽從。 \end{tabularx} \\ \\ \relax
3:23 & \begin{tabularx}{0.7\textwidth}{X} 凡不聽從那先知的,必將從民中滅絕。』 \end{tabularx} \\ \\ \relax
3:24 & \begin{tabularx}{0.7\textwidth}{X} 從撒母耳以來和後繼的眾先知,凡說預言的,也都曾宣告這些日子。 \end{tabularx} \\ \\ \relax
3:25 & \begin{tabularx}{0.7\textwidth}{X} 你們是先知的子孫,也是神與你們祖宗所立之約的子孫,就是對亞伯拉罕說:『地上萬族都將因你的後裔得福。』 \end{tabularx} \\ \\ \relax
3:26 & \begin{tabularx}{0.7\textwidth}{X} 神既興起他的僕人,就先差他到你們這裡來,賜福給你們,使各人回轉,離開你們的邪惡。」 \end{tabularx} \\ \\
[1ex]
\hline
\hline
\end{longtable}
$^{1}$大英姐妹晚安,我是阿迪.
我加入了Flow Church有三個月的時間.
我是1988年入建道神學院的.
一直住在長洲.
在長洲侍奉了大概有十年的時間.
當時是服侍中學生.
到2011年的時候.
我出來了荔枝角一棟大廈的教會去侍奉.
做大專和….
笑甚麼?.
因為我想有很多人都在這裡相認了.
所以就….
但當時我仍然在荔枝角服侍.
我仍然住在長洲一段時間.
每一次回長洲的時候都會經過中環.
在中環的時候有時會見到一些推紙皮的長者.
有一次我大概要趕十點船.
見到一個婆婆在推紙皮.
她推在電車路附近.
一直走著電車路.
大家都知道電車路旁邊其實只有單線行車.
單線行車的時候.
我發覺有些巴士或車在她後面飄過.
然後就停在她的位置前面.
我發覺不知是哪來的….
好,看到了.
不是很對咪.
所以當時不知是從哪來的感動.
我就去幫她一起去推.
推了一會的時候.
我發覺為何她推得那麼出呢?.
因為她幾乎霸佔了整條單線行車.
所以車就要繞過去了.
我就問婆婆你這樣推很危險.
她就說你不明白的.
她很簡單的回覆我.
我都沒有追問她太多.
然後我就站在她右邊.
我就嘗試借力去推她進去.
她又跟我鬥力去推出來.

$^{41}$我又嘗試借力去推她進去.
推著推著就發覺推不動車.
為甚麼?.
因為下了洞.
那一刻我就發現.
從我的眼光視野去看她是很危險.
但是原來從她的經驗和日常裡面.
她告訴我原來她推得出.
就是避開了那些坑渠窿.
大家意識到吧?.
雙黃線那裡有一些坑渠窿.
所以她就要推出.
當時我發覺其實我不是很理解她的狀況.
當我和她一路行這段路的時候.
我就發覺理解到原來她的狀況.
是因為她知道自己不夠力去推.
她是一個不夠五呎的婆婆.
七十多歲.
那街上的人就一直都知道.
其實她每晚都會路經那裡.
所以店舖九點多關門.
當我們路經的時候.
他們是拿盒子皮搭在車上.
當我一路和她行由中環.
一路去到上環的時候.
我發覺她開始看不到.
她就是靠那條雙黃線去辨別自己的位置.
在那一刻我突然間發覺.
在很多她的處境.
用我的眼光去看.
是危險的.
是怎樣怎樣.
但原來我不是在她的世界.
或者不知道原來她的限制是這樣.
有時我們的視野.
我們怎樣看一件事.
怎樣看一個人.
或者怎樣看那件事發生.
是被我們自己的視野框架去框住了.
我們去看別人的時候.

$^{81}$未必知道他正面對的一些難處.
或者這樣說.
我們未必看到他真正的需要.
今天我們看《仕途行傳》第三章.
大家是一個很熟悉的故事.
講到彼得約翰去見到一個人的需要的時候.
他怎樣即時去作出回應.
他們的行動不單止幫到這個人.
而且扭轉了這個人和其他旁觀者的生命.
我們看到PowerPoint的時候.
我們一起讀《仕途行傳》第三章第一至十節.
好,預備起.
新初禱告的時候.
彼得約翰上聖殿去.
有一個人生來是瘸腿的.
天天被人抬來放在殿的一個門口.
那門名叫美門.
要求進殿的人周濟.
他看見彼得約翰將要進殿.
就求他們周濟.
彼得約翰定睛看他.
彼得說:你看我們.
那人就留意看他們.
只望得著什麼.
彼得說:金銀我都沒有.
只把我所有的給你.
我奉拿撒勒人耶穌的名起來行走.
於是拉著他的右手扶他起來.
他的腳和裸子骨立刻見撞了.
就跳起來站著又行走.
和他們進了殿走著跳著讚美上帝.
百姓看見他們行走.
讚美上帝.
認得他是那素常坐在殿的門口.
求周濟的.
就因他所遇著的事滿心唏奇驚訝.
在《使徒行傳》第三章一節這樣說.
新初禱告的時候.
彼得約翰上聖殿去.
當時敬虔的猶太人.

$^{121}$每天三次就會去聖殿祈禱.
新初就是下午三點.
是人流最多的時候.
在這個時候很多人經過的時候.
就可以給錢周濟他.
整件事的過程其實很簡單.
使徒看到有需要的人求助.
他行了一個神蹟.
令到這個素常沒有走過路的人.
都能夠又跑又跳.
受助者是一個天生行不到路的人.
在《使徒行傳》第四章二十二節.
說到他已經有四十多歲.
就是說他從小到大都是這樣生活的.
他行不到路他沒有事情可以做.
就是他的朋友將他每一天抬去那個聖殿.
就求人給錢他.
大家都很熟他.
每天都見到他.
一直這樣生活了一段很長的時間.
第三節他看見彼得.
約翰要進殿就求他們周濟.
新譯本周濟這個字是施捨的意思.
他見到彼得和約翰.
他一如以往的看到什麼呢.
就是要見到他們就問他們去拿錢.
乞求他們給錢.
但是《使徒行傳》第二章剛剛說到.
就是那幫門徒行了很多神蹟.
醫治了很多人.
而且講不同的方言.
令到很多人信主.
當時整幫以色列人猶太人.
他們都聽到耶穌和他的使徒.
用了行很多的神蹟.
他聽過很多的事蹟.
神怎樣醫好人.
神怎樣去令他只可以走路的這些神蹟.
但我現在在想為什麼.
他那一刻不是.

$^{161}$作者不是描述他去求神醫治他.
而是他慣常的去求他們給錢.
只是求周濟.
第三至五節裡面.
我們看到他從看.
那個過程裡面.
看到他有一些的轉變.
下一張powerpoint.
第三至第五節我們看到有四個體字.
這四個體字有三個層次.
有三種的強度.
分別是漢見,定睛漢,漢和流義漢.
下一張.
再下一張.
可以再下一張.
這裡就是背景解釋.
當時那些人.
他們會路經美門進入庫房.
他就在美門外面.
繼續下去就行了.
我們就要看下一張.
他們就在美門門口.
那些以色列人進去奉獻的地方就要經過美門.
所以他就在外面等他.
繼續下去我們就要看下一張.
這個製作是比較複雜了.
不是我製作的.
繼續下去就可以了.
下到下一張經文就對了.
在這個段落裡面.
講到有四個漢見漢這個字.
情節上其實也很簡單.
那個攝製隊的人.
他看到那些仕途.
於是他就叫他們.
門徒也看到他.
然後他就跟他說.
那個門徒跟那個攝製隊的人說.
你看看我們.
那個人就注目在他們身上.

$^{201}$在這個過程裡面.
出現了一種微妙的變化.
原文第三節和第四節.
那個漢見和定睛漢.
只是一個普通的分詞.
即是一個普通的動詞.
描述他看到.
我看到了.
單純的就像你看到我.
我看到你這樣.
但是跟著定睛漢可能是.
我只是單純的提到.
不過我留意到你.
留意到你.
看到你多一秒兩秒的意思.
而這個段落的重點.
其實是在第四節.
第四節的漢字是一個命令式.
彼得就跟那個人說.
你看看我們.
跟著那個人就留意漢.
留意漢是另一種動詞.
就是直述的動詞.
意思是在層次上.
比漢見多一點.
但又未去到命令式的重要性.
第五節留意漢這個字.
在ESV英文譯本是.
Fix his attention.
即是注視著新漢譯本.
是一個比較重要的描述.
當他從一個不同的角度.
去看彼得的時候.
他產生了一些的領略.
或者一些的變化.
這個窮途的人.
在他生命可以被扭轉的那一刻.
他沒有求神去醫治他.
聖經描述他仍然是求.
那個使徒去周祭給他錢.

$^{241}$那一刻他不是看到神.
也不是看到神過往在那個地方.
很多的作為.
而是他仍然在看.
他自己一個很實際.
現實的需要.
有沒有錯呢.
沒有錯的.
因為他需要生活.
因為他仍然需要.
好像過往的日子.
每天去那裡去乞錢.
他才可以維持他的生活.
沒有錯的.
但是他可能在那一刻.
他miss了一個很大好.
去經歷神的一個生活.
或者他有個前設.
因為大家都查過經知道.
在舊約裡面.
那些人認為.
人之所以身體有事.
是什麼原因.
是因為有罪.
又或者他爸爸有罪.
他的上一代有罪.
所以他就面對著身體的殘缺.
而當時在這一班的人.
身體有殘缺的人.
他們不可以進入聖殿.
他是自覺是一些劣等的人.
他對生活沒有一種的盼望.
所以當他面對一個.
可以扭轉他人生的機會的時候.
他仍然是在求最普通的生活.
去接受現實.
甘願接受自己是一個.
過那種繼續問人拿錢的生活.
他的視野看不到神可以醫治他.
所以他就沒有這樣去求.

$^{281}$甚至有時候我們都是.
強制關機.
明知這樣求都沒有用.
又或者事情都不會改變.
我們就選擇不去這樣去求這個期望.
就算有都不是留給自己的機會.
今日我們都是這樣.
這種生活的態度.
就或者限制了神在他生命裡面的工作.
不過好在經文不是停在這裡.
當彼得和約翰見到這個人的時候.
最後他們看到的時候.
他們是看內.
他們不只是看到他在這裡.
他們是再注視多他一會兒.
之後就產生劇情上的變化.
一會兒再說多一點.
不過我想問大家.
就是我們在我們的生活裡面.
都會遇到很多的期望.
都會看到很多的東西.
我們會用第三第四節.
那種很外在很表面的去看.
無論是你的生活的需要.
無論是你認識上帝和神的關係.
怎樣經歷神的事上.
還是我們可以在聖經裡面.
在原文看這個字Blackpool.
這個意思其實是很豐富的.
即是他看到.
不單止他肉眼看到.
而且他看到的時候.
他心裡面是有一份的感動.
而因著這個感動.
他能夠有領略.
而且去回應上帝.
彼得和約翰很想他有這種看見.
不單止是看到我們給錢你.
不單止是看到我可以醫治你.
而是看到你的人生.

$^{321}$怎樣可以因著神.
而心裡面生命上產生不同的轉變.
這個就是聖經第四節裡面.
看這個視野.
Blackpool這個字.
我今天去形容為.
超出我們日常視野的一種看見.
這樣看的話.
如果我們的屬靈生活.
或屬靈生命.
我們日常生活.
我們都有這種視野.
你會看到很多背後神.
要你怎樣去經歷他心意的一些歷程.
你會發現更加多.
你會經歷更加多.
靈姐妹我們日常所看見的.
不應該只是轉變不到的社會現象.
我們都會自身其中.
有時好像這個瓊斯一樣.
不能變的.
情況就是這樣.
我們都會看到我們自身.
在社會上那種轉變不到的環境.
和灰暗的前景.
但是我們要看見環境裡面.
自己裡面的需要.
我們要看到環境裡面.
我們怎樣和神建立一個關係.
甚至這個關係是可以.
祝福到我們身邊的人.
當我們有這種視野看見的時候.
你就會發現原來神給了我們很多資源.
我和我的組員.
上星期就做了性格測驗.
恩賜侍奉問卷.
原來我們那組很多人都有.
連問連述的恩賜.
神給了我們這些資源.
究竟我們是放下了.

$^{361}$還是我們有在生活上活現展現出來.
我覺得這個是今天作為基督徒.
在暗淡的環境裡面.
我們更加應該展現基督徒.
在這種環境上所看見.
而繼續去走.
繼續去影響我們身邊的人.
我們來到這裡.
我想大家都抱著一種重新開始.
教會生活.
你自己人生這個階段的生活.
我們在三個月裡面.
短短三個月我聽了很多組員的故事.
在我們人生某一個點.
你都看到一個要離開.
重新開始的一個需要.
無論是新的日常.
新的教會生活.
新的小組.
我們都要去適應.
但我自己去思考的時候.
你看不看到神帶你來這裡.
當然有時候我們都會聽到.
有各種千百樣的原因.
我離開舊的地方.
但不單止神要帶你來這裡逃避.
避一避是需要的.
沉澱一下.
休息一下是需要的.
我想問的是.
你意識到神帶你來這一個小組.
這一個教會.
神想你做些什麼.
過往二十年我在教會服侍九十後的年青人.
我再進修的時候.
我的論文都是寫九十後的屬靈的模樣.
我來了Flow Church.
潘Sir看得起我.
讓我帶一組跟我差不多.
而且比我年紀大.

$^{401}$有一組我以前不是很多經驗的.
三十多歲四十多歲的小組.
我嘗試大膽地跟潘Sir說.
可不可以多帶一些年輕的.
死不斷氣還想帶年輕的.
我感恩的.
因為我都覺得這仍然是我的專長.
我仍然還有力量.
但我一直都在問神.
為什麼你帶我來這裡.
現在的狀況.
我是不是我接下來應該這樣去做.
你看我IG都是照明為年青人開路.
這是以前不變的.
絕望研究所.
很多人都有去.
昨晚有分享會.
第二幕.
我旁邊的男人就哭了.
我就給他紙巾.
我也想拿紙巾給自己.
但我看到他哭我就忍住.
那一幕很深刻.
大家有去過的日子.
應該會勾起你少少回憶.
那一幕開始的時候.
大家都舉起一本聖經.
這個是我.
那一刻已經開始很震撼.
沒多久之後他們就坐在地上.
再沒多久之後.
有些人他們起身.
沒有拿那本聖經.
大概知道想表達什麼.
或者我自己看見一個圖片.
就是有很多人.
他們繼續前行他們的人生.
他們沒有再繼續持守.
或者回教會.
再沒多久.

$^{441}$就是有一班人.
他們慢慢一本一本聖經這樣撿起.
那一刻我彷彿我看到.
有這種好像Blackpool的體驗.
我看到神要我在這裡認真的.
去對待每一個很想重拾教會生活的弟兄姊妹.
80年齡.
只要有聖經在地.
那一刻我心裡很感動.
就是我就要去撿.
我猜我們人生裡面有很多不同的階段.
負擔很多不同的看見.
所以很想藉著這一點去問大家.
在這一點你來到這裡.
有什麼你很想看到.
而有什麼你很想在你的信仰生活上.
繼續前行的時候.
你看到神在你生命的旨意和作為.
第二點就是除了超出日常的視野之外.
他們做了一個超出日常的宣告.
在第一節說到他們原本那班士徒是做什麼的.
是去祈禱的.
他們出到門口的時候.
就聽到有人去到門口附近.
就聽到有人要周祭.
於是他們就第四節定睛看他.
新漢語的日本語叫做注視著.
就是剛才所說的.
他們多看幾眼.
有些好的東西或是好的東西.
可能都會多看幾眼.
所以他們就不單純的看一看.
他們就把他們那一刻原本趕去祈禱的腳步.
他們停一停.
注視著那個人.
英文是Directed his gaze at him.
Gaze的意思是看著長時間.
少少這樣去凝望著.
在這個停一停.
看一看的行動.

$^{481}$彼得接著就說了.
金和銀我都沒有.
只把我所有的給你.
下一章.
這裡是對的.
我奉那西勒人耶穌的名.
叫你起來行走.
我們就心想.
我們行走當然是起來.
但這是兩個字.
也是重要的動詞並令式.
起來的意思是.
肅成.
醒覺的意思.
興起的意思.
行走就是Literally.
起來走.
當那班門徒看到這個人的需要的時候.
他叫他興起.
叫他的心興起.
叫他繼續前行.
這個人.
門徒看到這個人.
他那一刻的需要.
不再是他所乞求的錢這麼簡單.
而是他的心需要被醫治.
需要被安慰.
需要繼續前行.
或者在我們這個人生的階段.
我們都在這種需要被神再興起.
繼續前行的階段.
你看不看到.
你當然看到才會來這裡.
我相信或我希望.
大家都是這樣.
能夠去經歷神.
這個人.
他都差不多放棄了.
四十年這樣天天被人抬過來.
就是黑錢.

$^{521}$他接受了這種現況.
我相信我們有一種從神而來的視野.
我們是可以不接受現狀.
即使你面對的是不如意.
即使你面對的是灰暗.
即使你正在走的路.
你不知道可以怎樣走都好.
仍然需要去學習.
怎樣為主而興起.
怎樣為主而繼續行走.
而這個過程我相信神會有工作在你的生命裡面.
使徒看到.
這個人需要脫離他的舊我.
開展他新的生活.
我想我們身邊都有很多很有需要的人.
可能你每天都會見到.
但是我們的工作繁忙.
我們趕緊上班趕緊放工趕緊約人.
我們就習慣了.
香港人都是.
有時會視而不見.
有時我們會見到.
但我們會有千百個原因.
我們是不去多走一步.
去幫助當時那一刻.
即使你認為他是很有需要的人.
我們在旺角E出口那裡經常聽人busking.
那裡都有一些黑錢的人.
有一次我和身邊的人討論.
究竟應不應該給錢他.
他就說不是.
他們應該是那些集團式的詐騙.
你給了錢他其實是幫不到他的.
說了一串這些東西.
但無論如何我最後都給了.
因為那一刻我覺得自己的心被感動.
而心生了一種連續出來.
我估計我們要去學習.
看見而且看得見之後.
去好像之前第四節那樣.

$^{561}$那種Black hole的看見.
是需要有一種行動上的回應.
在生活裡面我們都試下停一停.
注視著一些有需要的人.
我相信當時那班門徒.
他們被聖靈充滿.
不單止是行神跡.
講謊言這些.
他們被聖靈充滿.
是他們能夠看見有需要的人.
真正裡面的需要.
是這樣的.
當彼得和約翰看到這個人的需要的時候.
在第七節說到.
他就拉他的右手扶他起來.
當我們見到身邊的人有需要.
無論是心靈的身體的各樣.
我們其實作為基督徒.
我們都可以有事去做到.
過去可能有一段日子.
我們都會覺得我們的靈明差.
我們的社會環境不是很好.
工作很忙諸如此類.
給多少少空間自己去看.
神要你看到的東西.
而且你試下用行動去回應.
你就會經歷神.
回應我一開始說到的經歷.
有時我就開始習慣了.
去幫一些推紙皮的婆婆.
尤其是婆婆.
比較少公公.
在以前我教會的樓下.
有一條大南西街.
主要的幹道.
不只是交通的幹道.
而是推紙皮人的幹道.
在近山邊的瓊林街.
有一個廢物回收的檔口.
有時我就和.

$^{601}$未到的.
推他們一起.
因為你會發現.
如果你真的停一停.
有時他們不是推一架車.
有時他們是右手推一架車.
左手拉一架車.
一個六十幾七十歲的人.
他這樣的時候.
我通常都會衝過去幫他.
不理會一會有什麼聚會.
當我和他們走到那裡的時候.
我會心情很激動.
一整車的紙皮賣到多少錢?.
我們那裡也和他們兌換.
他有一些鐵的.
他的鐵也收.
四十元.
跌到這麼高.
那些回收的人.
有一次就罵那個婆婆.
為什麼?.
通常你留意就知道.
他們會弄濕那些紙皮.
為什麼?.
重一點.
那些回收的人都知道.
會少算他們的錢.
這種景象在我眼前是多麼扭曲.
你有沒有見過一些很不公.
很不合理的事?.
不合理的是.
一個七十幾歲的婆婆公公.
他們仍然要在街上日曬雨淋.
去做這些推紙皮,執紙皮的工作.
很不容易.
你會發覺大部分推紙皮的婆婆.
是在駝背的.
她腰痠骨痛.
我會和他們聊天.

$^{641}$有時候會送一些藥貼給她.
我做的有限.
我比較少付錢.
也有的.
有時候給幾十元或一兩百元.
去喝茶.
熟悉的那些.
有幾個在那裡已經是霸主.
每區都應該有霸主.
熟悉的區就知道誰打誰.
你會發覺.
原來夏天放一些退熱貼在背包.
冬天放一些暖貼在背包.
你見到她就可以派給她.
你改變不了社會現狀.
你不可以變成一車紙皮是三百元.
當你改變不了現狀的時候.
你會感受到無奈,沮喪,不開心.
但我想說的是.
作為基督徒.
當你看到的時候.
我們都可以去回應這些現況.
做一些事你覺得.
是不是真的有用?.
未必.
可以幫到她很多.
但在那一刻.
我相信其實是幫到她的.
只要你仍然憑著這種看見.
憑著這種相信.
一定可以祝福到他們.
推推紙皮.
我有時跟蹤他們.
有些人推紙皮.
會在垃圾桶撿東西.
撿完東西去哪裡?.
我在深水Po 區長大.
所以跟著就知道.
他們就去鴨寮街買.
然後就可以去下張.

$^{681}$你見到這些地攤.
鴨寮街圍著鴨寮街.
例如黃金商場對岸那邊的鴨寮街.
你會發覺整個區域都是這些婆婆公公.
他們收集一些你家裡都不要的東西.
就去賣.
去得多的時候.
我有時會拿家裡自己不要的東西.
跟他們一起坐一會.
為何我要坐一會?.
就是他可以將一對.
我告訴我老婆.
一對不要的五六百元的鞋.
他可以賣二十元.
他不懂的.
有牌子的東西他都不懂.
我就跟他說.
那些人問價二十元.
我說不行.
起碼要八十元.
我就認識了這個婆婆.
七十四歲.
她有公屋住.
但她每天都去開檔.
在七十一門口.
如果大家經過的話.
可以去.
很多你可以看一下.
生活是不是艱難.
貧富懸殊是不是很厲害.
是.
深水Po 區一定是.
這裡其實紅磡區.
土瓜灣也是.
不過我不熟悉這一帶.
同工很好.
下星期帶我們去看一下.
因為我們其實.
我這樣說.
說了這麼多.

$^{721}$都是配合這兩個月的主題.
是不是.
都是想大家出去.
服侍一下我們附近的街坊.
我就真的自認不熟.
所以我都要在這一區學一下.
你有來過區的.
是不是.
都有些有需要的人.
所以我鼓勵大家.
你都要去學一下.
看一下.
可能你的負擔.
或者你的愛心.
未必是用在這個.
這麼具體的群體.
不過我相信.
神會感動你去看一些.
你可能幫到的群體.
幫到的人.
同事.
家人.
都有.
這一次之後.
我又發覺.
我可以有更加多的機會.
去接觸和幫助人.
在三年之前.
下一張.
當時是剛剛第一波.
我在二月.
韓國.
封關之前.
我找到一個朋友.
寄了七千個口罩給我.
當時你有幾十個口罩在身.
都怕人老了.
是不是.
有一段時間很害怕.
因為人人都只剩十幾個.

$^{761}$排隊.
我很多.
和弟兄姊妹和兒子.
排隊.
但是.
我覺得不要只派一張.
於是有一天.
我決定.
一路走.
開始的時候.
由荔枝角.
以前的教會.
開始走.
見到很多.
都是撿紙皮的婆婆.
就派給她.
下一張.
到北河街.
見到一些.
下一張.
見到一些清潔工.
又是賣廣告.
我們教會都服侍清潔工.
見到清潔工.
膽粗粗地問他們.
你們夠不夠口罩.
當時人人都說不夠.
我當時都有八百個口罩在身.
有點害怕.
一路去派.
可以看那段片.
一路嘗試用另一個角度.
很想看到.
有些人有需要.
過去跟他們聊天.
問候一下他們.
一路由深水Po 區.
走去油麻地.
走去尖沙咀.
走去紅磡區.

$^{801}$有這些露宿者.
實況者.
推紙皮.
任何人.
見到有些需要.
都過去聊天.
然後就可以.
我記得那天.
除了以前在金鐘之外.
隨意跟人聊天.
最多的一次經歷.
連結到那些人.
我相信有時我們要走出去.
我們的日常都是.
沒什麼.
以前參加教會.
一兩次的清真牛肉餅.
最後你看到那些.
那些街道繞來繞去.
那天我看到很多東西.
我相信在我們日常的日常以外.
上班.
行山.
爬艇.
除了浴室之外.
我們都可以找一些時間空間.
去經歷.
讓行程自己去看到.
原來神在社區.
或者你的生命裡面.
可以做些什麼.
所以其實是很特別的.
對我來說.
是一個很大的經歷.
神藉著我.
讓我.
讓我看到我可以.
怎樣去擴闊我自己的視野.
的確在香港上班.
會令到你的人都懷疑人生.

$^{841}$令到你的視野都窄了.
鼓勵大家.
和你的組員也好.
自己也好.
找一些空間.
去看看社區.
別人.
心裡面的需要.
生活上面的需要.
我們不是每個人都有記憶.
但是只要我們有一些微小的.
能夠獻出.
在那一刻.
你就能夠.
祝福到人.
是不是時間到了?.
我們都會看到黑暗.
我們都會看到沮喪.
我自己都曾經有這種的.
想法或者感覺.
但是當我們.
一路去和人經歷的時候.
我們很希望.
我們身邊的人.
不單止看到黑暗和沮喪.
更加在不同程度裡面.
我們能夠看見.
神的恩典.
神的作為.
我十年前.
最後了.
我十年前開始有夢.
一看眼科醫生.
他說我青光眼.
右眼只剩下一成半視力.
我就說.
要不要像海賊王那樣.
戴眼罩?.
他說不用.
雖然你只剩下一成視力.

$^{881}$但是其實能夠幫助的.
你看到七成的闊度.
我發覺的時候.
就是因為我帶足球隊.
我們那隊進了球賽.
我都看不到.
不知道.
我看歡呼聲才知道.
我面對了人生的限制.
發覺都很不開心.
我很喜歡看東西.
看球賽,看劇集.
又看字幕.
不懂日文,韓文.
你會發覺在你人生的某些點.
你會有些限制出現.
到18年我就要戴.
漸進鏡.
透露了我的年齡.
19年我就戴普通眼鏡.
因為19年的時候要逃跑和跳欄.
你知道戴漸進鏡是很容易跌倒的.
那時候跌倒是很大件事.
轉一下圖.
很大件事.
所以在我人生的不同階段.
現在我要看電腦,看遠足.
還有現在我有四副眼鏡.
在不同的人生階段.
或者你看不同的東西.
你用不同的角度去看.
鼓勵大家.
有時候我們真的專注在眼前的生活.
鼓勵大家用神的眼光.
而你看到的.
我甚願神用你所看到的.
你所得著的去回應.
而在回應裡面.
我相信,我知道.
你是能夠大大的傾力上帝.

$^{921}$結束的時候.
想大家有少少安靜的時間.
我不知道你在你人生裡面.
有沒有經歷過.
神曾經給你看到的一些東西.
是你可以做到的.
你有負擔的.
又或者曾經調轉.
用另外一個角度去看.
你是躺在那裡的強者.
你是大大的經歷過神的恩典.
今天我們說看見.
不知道你最近看見過甚麼.
我想大家安靜.
合上眼睛.
有時候我們的眼目是含糊的.
看不清楚怎樣走.
有時候我們會覺得很迷惘.
很累.
但很想大家注目在上帝和你的關係.
祂的愛.
祂的拯救.
所以今天.
當各位弟兄姊妹.
你仍然坐在這裡的時候.
你是謀求或追求一種.
新的同神的關係和宿命向度.
你很想成長,很想看見.
很想經歷上帝.
有甚麼是你能看到的.
我想大家去禱告,去回應.
如果你發覺好像沒有甚麼看到.
但我相信你去求,你去問.
神一定會讓你看到.
\newpage



\section{哥林多前書 4:1-16-20230211}
\label{sec:KAnMTZ32Dag}
\textbf{【流堂崇拜】演員的自我修養|哥林多前書4\_1-16|20230211 [KAnMTZ32Dag]}
\newline
\newline
連結: \href{https://youtube.com/watch?v=KAnMTZ32Dag}{\texttt{ https://youtube.com/watch?v=KAnMTZ32Dag}} ~~~~ 語音日期: 2023-02-11 
\newline
\newline
\hyperref[sec:GtMQusxSoOU]{\small{< < < PREV SERMON < < <}}
~
\hyperref[sec:index_chronic]{\small{[返順時目]}}
~
\hyperref[sec:index_scriptual]{\small{[返順卷目]}}
~
\hyperref[sec:4Dll86a7b18]{\small{> > > NEXT SERMON > > >}}
\newline
\newline
哥林多前書 4:1-16-20230211
\newline
\begin{longtable}{cl}
\hline
\hline
章節 & 經文 (和合本修訂版)\\
\hline
4:1 & \begin{tabularx}{0.7\textwidth}{X} 人應該把我們看為基督的執事,為神的奧祕的管家。 \end{tabularx} \\ \\ \relax
4:2 & \begin{tabularx}{0.7\textwidth}{X} 所求於管家的,是要他忠心。 \end{tabularx} \\ \\ \relax
4:3 & \begin{tabularx}{0.7\textwidth}{X} 我被你們評斷,或被別人評斷,我都以為是極小的事;連我自己也不評斷自己。 \end{tabularx} \\ \\ \relax
4:4 & \begin{tabularx}{0.7\textwidth}{X} 雖然我不覺得自己有錯,卻也不能因此判為無罪;審斷我的是主。 \end{tabularx} \\ \\ \relax
4:5 & \begin{tabularx}{0.7\textwidth}{X} 所以,時候未到,在主來以前甚麼都不要評斷,他要照出暗中的隱情,揭發人的動機。那時,各人要從神那裡得著稱讚。 \end{tabularx} \\ \\ \relax
4:6 & \begin{tabularx}{0.7\textwidth}{X} 弟兄們,為你們的緣故,我拿這些事應用到我自己和亞波羅身上,讓你們從我們學到「不可過於聖經所記」這話的意思,免得你們自高自大,看重這個,看輕那個。 \end{tabularx} \\ \\ \relax
4:7 & \begin{tabularx}{0.7\textwidth}{X} 使你與人不同的是誰呢?你所有的有哪一個不是領受的呢?若是領受的,為何自誇,彷彿不是領受的呢? \end{tabularx} \\ \\ \relax
4:8 & \begin{tabularx}{0.7\textwidth}{X} 你們已經飽足了,已經富足了,用不著我們,自己就作王了。我願意你們果真作王,讓我們也可以與你們一同作王! \end{tabularx} \\ \\ \relax
4:9 & \begin{tabularx}{0.7\textwidth}{X} 我想,神把我們作使徒的明顯地列在末後,好像定死罪的囚犯,因為我們成了一臺戲,給世界、天使和眾人觀看。 \end{tabularx} \\ \\ \relax
4:10 & \begin{tabularx}{0.7\textwidth}{X} 我們為基督的緣故成為愚拙的;你們在基督裡倒是聰明的。我們軟弱,你們倒強壯;你們有榮耀,我們倒被藐視。 \end{tabularx} \\ \\ \relax
4:11 & \begin{tabularx}{0.7\textwidth}{X} 直到如今,我們還是又飢又渴,又赤身露體,又挨打,又到處漂泊, \end{tabularx} \\ \\ \relax
4:12 & \begin{tabularx}{0.7\textwidth}{X} 並且勞碌,親手做工;被人咒罵,我們就祝福;被人迫害,我們就忍受; \end{tabularx} \\ \\ \relax
4:13 & \begin{tabularx}{0.7\textwidth}{X} 被人毀謗,我們就勸導。直到如今,人還把我們看作世上的污穢,萬物中的渣滓。 \end{tabularx} \\ \\ \relax
4:14 & \begin{tabularx}{0.7\textwidth}{X} 我寫這些話,不是要使你們羞愧,而是要警戒你們,好像我所愛的兒女一樣。 \end{tabularx} \\ \\ \relax
4:15 & \begin{tabularx}{0.7\textwidth}{X} 雖然你們在基督裡有無數的導師,卻沒有許多父親,因我是在基督耶穌裡用福音生了你們。 \end{tabularx} \\ \\ \relax
4:16 & \begin{tabularx}{0.7\textwidth}{X} 所以,我求你們要效法我。 \end{tabularx} \\ \\ \relax
4:17 & \begin{tabularx}{0.7\textwidth}{X} 因此,我已差提摩太到你們那裡去。他在主裡面是我親愛和忠心的兒子;他要提醒你們,我在基督耶穌裡怎樣行事為人,在各處各教會中怎樣教導人。 \end{tabularx} \\ \\ \relax
4:18 & \begin{tabularx}{0.7\textwidth}{X} 有些人以為我不到你們那裡去而自高自大。 \end{tabularx} \\ \\ \relax
4:19 & \begin{tabularx}{0.7\textwidth}{X} 但是,主若准許,我會很快到你們那裡去;我所要知道的,不是那些自高自大者的言語,而是他們的權能。 \end{tabularx} \\ \\ \relax
4:20 & \begin{tabularx}{0.7\textwidth}{X} 因為神的國不在乎言語,而在乎權能。 \end{tabularx} \\ \\ \relax
4:21 & \begin{tabularx}{0.7\textwidth}{X} 你們願意怎麼樣呢?要我帶著棍子到你們那裡去呢,還是帶著慈愛溫柔的心呢? \end{tabularx} \\ \\
[1ex]
\hline
\hline
\end{longtable}
$^{1}$頂姐妹平安.
網上頂姐妹平安.
我見剛才有些人進來的時候已經買了巧克力.
應該是送給別人的.
剛才很感動.
特別唱第二首避難所的時候.
聽到很多頂姐妹的聲音.
這也是我最享受現場崇拜的時候.
和頂姐妹一起開口唱歌.
每一次聽到的時候都會目光動.
今天的講題叫做演員的自我修養.
不知道大家想到什麼.
今天沒有周星馳的戲.
不是講戲裡面.
今天我選擇的經文是講江南多前書的經文.
我之前在Evil Church講道也有講過.
特別我自己很喜歡保羅的侍奉經歷.
作為一個目者或信徒.
保羅的經歷是我們如實看到侍奉生命的轉變和倚靠.
我今天選擇的經文是江南多前書第三章.
如果大家看過江南多前書的話.
你都知道江南多教會是一間很繁的教會.
繁的教會問題很多.
保羅寫信去跟他們申辯.
又或者梳理他們的問題.
我們先聽聽這段經文說什麼.
《江南多前書》第四章.
所以,人應該將我們視為基督的僕人.
是神奧庇的事的管家.
對管家的要求就是要他有誠信.
對我來說,無論是被你們論斷.
還是被其他人論斷.
其實都是很小的事.
就連我自己也不去論斷自己.
我就算自己問心無愧.
也不能就此以為意.
因為判斷我的乃是主.
所以,事後未到.
什麼都不要去論斷.
直至主來到.

$^{41}$祂要照出黑暗裡的隱情.
顯明人心裡的動機.
到時,每個人要從神那裡得到稱讚.
弟兄啊,我為你們的緣故.
將這些事套用在我自己和阿波羅的身上.
叫你們在我們身上學到的.
不會超過聖經所記的.
免得你們中間有人自高自大.
看重這個,看輕那個.
究竟是誰使你與別不同呢?.
你有什麼不是領受的呢?.
既然都是領受的.
那為什麼還要自誇.
好像那些你所誇口的不是領受回來的呢?.
你們已經滿足了.
已經富足了.
不再需要我們.
自己就可以做王了.
我也不知道多麼想你們真的做到王.
這樣我們就可以和你們一起做王了.
我想,神將我們這些做使徒的推出來示眾.
排在最後面.
好像那些定了死罪的囚犯.
因為我們已經成為了一個壯觀的場面.
給世界來觀看.
就是給天使和世人來觀看.
我們為基督的緣故成為了愚拙的.
而你們在基督裡面反而成為了聰明的.
我們軟弱.
而你們反而剛強.
你們就受人尊重.
而我們反而被人藐視.
直到現在.
我們還是又飢又渴.
衣衫藍柳.
遭受毒打.
居無定所.
並且親手勞苦做工.
被人辱罵.
我們就祝福.

$^{81}$被人逼迫.
我們就忍受.
被人毀謗.
我們就好言相勸.
直到現在.
人仍然將我們看為世界上的垃圾.
萬物之中的渣滓.
都還是這樣.
我寫這些話.
並不是要叫你們羞愧.
而是要警戒你們.
好像勸戒我所親愛的子女一樣.
你們在基督裡面.
即使有上萬個監護人.
但是做你們屬靈父親的.
卻是只有一個.
因為是我在基督耶穌裡面.
藉著福音生了你們.
所以我勸你們要效法我.
我們一起禱告.
天使你每當我們打開你的話.
你仍舊對我們說話.
願主你藉著你的聖言.
再一次在這個時空當中.
讓我們得著教導.
也聖靈讓我們明白到.
不同的處境.
不同的說話.
對今天我們的壽終.
上帝你仍然喜悅我們進行你的聖悔.
我們祈禱奉耶穌的名求.
阿們.
剛才也說到.
哥倫多教會是一個很不容易運作的教會.
書信裡面如果你看過.
其實哥倫多教會是保羅建立了三年左右.
的教會.
中間出現一些問題.
保羅就寫信回去處理他們的問題.
現在我們看的哥倫多前頭書.

$^{121}$就是編輯了其中一些書信裡面的內容.
如果看書的時候.
看書信的時候.
你會發覺.
輪到之前寫信給你們的時候.
其實就可能不見了那封信.
但其實保羅來來回回.
其實處理他們很多爭拗.
哥倫多教會其實我自己也覺得很賞心.
因為哥倫多的環境很像香港.
一個鑽口貿易的地方.
一個多神的地方.
一個很多不同的意見都可以成為一個學說.
很多人去追隨.
很香港.
我記得那時候我在堂會教哥倫多前書的時候.
就說一個很靠近的例子.
就是我在長洲教書.
有一次我下山.
如果你去過見道的話.
見道在山上.
然後就想著下山去吃東西的時候.
我們學院有一個牌坊.
如果你去過的話.
經過的時候.
有些人來長洲旅行的時候.
就看到這裡是一間神學院.
是一間學校.
你會看到這個就是見道神.
(笑聲).
我想說這個很令人頭痛.
這個就是見道神.
然後我聽到說.
不是不是.
這個是基督教的神學院.
不是見道神.
因為見道神學院.
其實長洲是一個很奇妙的地方.
有天后廟.
有不帝廟.

$^{161}$有很多.
所以覺得多過見道神.
有多奇怪.
就是這樣.
好像很難理解.
但是我聽到就自可.
我聽不到就不知道說了什麼.
就是.
保羅的心態都是.
在一個外邦人的地方.
他皈信基督信仰.
有很多他們潛移默化.
或者他們從小開始聽大的東西.
要糾正他們的信仰上.
我們要持守的東西.
其實很困難.
所以你剛才聽到那些經文裡面.
他們在爭論.
我厲害了.
我懂一些東西了.
三年了.
我處理到一些東西了.
他就以為自己是得到.
保羅就告訴你.
其實不是這樣想.
還有一些事情.
我真的希望你得到.
你知道.
保羅的口吻.
剛才用廣東話聽的時候.
其實有些反諷的意味在這裡.
所以這段經文其實不是很難處理.
但是有些重點.
特別在新的一年.
我們再能夠有機會回教會.
有實體聚會的時候.
正正是面對一些.
新的群體.
又再一次.
又和弟妹相處的時候.

$^{201}$希望讓我們看見一些關係.
看見一些靈覺.
看見一些可以大家互相再學習的空間.
你會看到.
由第一節開始.
到第五節裡面.
有些比較重要的字.
跟大家說.
保羅再一次宣告一件事.
就是基督的執事.
剛才你聽廣東話的讀經就是「瀑人」.
他說的是一個管家.
有些事情你要管理.
我之前也說過.
大概管家是什麼身份.
或者他有什麼功能.
既然是管家.
就一定不是主人.
主人不在家的時候.
你還是管家.
你沒有主人不在家的時候.
你狐假虎威就覺得自己.
主人不在你最大.
身份不清楚的話就會出事.
所以主人不在家的時候.
你仍然是一個管家.
所以管家應該很清楚自己身份.
第二件事就是.
作為管家.
他一定有他的能力.
不然請他回來幹什麼呢.
所以他應該清楚自己的能力是什麼.
既然管家有些事情要管.
一定清楚他的權力範圍在哪裡.
所以能力應該在範圍當中清楚.
如果不是.
那個是我的能力以外.
或者範圍以外.
他就不是管.
有些地方不能去.

$^{241}$所以管家清楚自己不是主人.
他清楚自己請回來的能力.
也清楚他管理的範圍的時候.
就會有一個交帳.
有些事情是我要承擔.
和我要告訴我的主人.
我能做到和做不到的清單.
第三件事.
最後一件事就是.
他一定會清楚.
主人安排了他的工作.
一定有一個時限.
多久完成.
什麼時候要交進度.
一定會很清楚.
所以.
布洛提醒我們既然要管理.
或者我們是一個基督執事.
一個僕人的時候.
其實你不用說別人.
你不要說別人.
你自己都不要說自己.
其實真正可以管理的.
不是和你旁邊的人交帳.
也不是和別人說你不是.
你就是不是.
其實你應該清楚你和上帝之間的關係.
和你和上帝之間的契通是什麼.
這個很重要.
今天我們不難找到.
旁邊總有人說三道四.
總會拿到一件事就發到很大.
或者拿到一個點就說.
你死了.
你還不讓我看到你的東西.
但是事實上是不是事實的全部.
大家做了.
這裡已經沒有中學生了.
是吧.
OK.

$^{281}$你教中學就不是.
重點就是你會發覺.
你做了一段時間的人.
你總會發覺不可以這麼快就立論.
因為什麼.
時候還沒到.
上帝一定是看完整件事的.
這個是我們相信的.
所以你有做沒有做.
你做到做不到.
或者你做成怎樣.
總會有不同的尺子在不同的人的嘴裡.
我和弟姐姐妹都分享.
其實那把尺子不是放在別人的嘴巴裡.
是放在自己這裡.
如果那把尺子放在別人的嘴裡.
就會把你剁死.
他放在伸縮尺.
是不是.
真的會把你剁死.
你會看到.
我上次在說訴羅和大衛的時候.
訴羅就是不清楚自己.
當婦人唱歌.
訴羅殺死千千.
大衛殺死萬萬的時候.
他很緊張婦人那把口.
那把尺子的時候.
就把他綁死了.
從那天起.
他就很憎恨大衛.
其實如果清楚他自己是一個君王的身份.
他很清楚他自己的權責和範圍當中.
為什麼要和大衛比較呢.
其實論斷當中難免有比較.
論斷當中難免就是.
其實哪一個標準才是對的.
但是他不是要和你交漲.
他不是要等你作主.
其實你會發現關係又不是那麼緊密.

$^{321}$又不是一定要和你去有一個很.
很條理要交代的地方.
其實真正我們要知道的就是.
其實事情的始末.
上帝仍然看到的.
只有時候還沒有到.
所以主來.
他要照出暗中的隱情.
總有很多事情.
不是事實的全部.
總有很多事情是說都說不完.
但是過程當中.
我們要知道.
我們是對得起主.
個人要從上帝那裡得到稱讚就是.
保羅是一個很認真看待他要承擔責任的人.
所以對於保羅來說.
他要去行宣教旅程.
他要做宣教工作的過程當中.
他每一站盡力去做.
縱使別人看下去覺得.
不是吧,這樣也可以?.
好像你看《紹行傳》第十六章.
十七章的時候.
十五,十六章的時候.
保羅往馬其頓方向回應呼召的時候.
他中途被下載監獄.
被人毒打.
整個過程當中.
是很無奈的.
但是你會看到.
保羅仍然會沉默.
唱詩讚美神.
對我們來說.
我覺得是很難去接受的.
我也是為你做事.
其實可不可以路簡單一點呢?.
又或者路不要那麼辛苦呢?.
但是保羅在過去的宣教.
他自己的經歷當中.

$^{361}$他知道上帝看著他.
是很難的.
特別是過去我們這幾年.
有很多事情.
我自己也說到.
好像不斷提醒自己.
只要認真做事就煩了.
每件事都好像要交代給全世界.
才可以做得到.
但是有些事情.
你要清楚自己是否正在做這件事.
接著再下去的時候.
第六節到第八節的時候.
保羅就說清楚了.
弟兄.
當然現在會包括姐妹.
這些你也知道的.
他就說我們正在做的事.
不只是口傳.
要建基於聖經有沒有說.
口傳就真的會有教主的來龍去脈.
但是否有聖經的意義.
或者是否有聖經的立場和做法呢?.
這是我們緊張的.
就像我每一次.
跟弟兄姐妹說.
Info Group的第一堂的時候.
和第二堂的時候.
第一堂是說What is the full church?.
全港那麼多間教會.
為甚麼多一間教會?.
第二堂就是說.
Full church是怎樣運作的?.
我特別第一堂的時候.
很認真地跟弟兄姐妹說.
是因為你們想了解Full church.
我有責任要說得清楚一點.
Full church這間教會的聖經立場.
和它的運作方式.
它堅持的是甚麼?.

$^{401}$既然是教會.
有些事情是不可以退的.
既然是教會.
有些事情一定要做的.
既然是教會.
有些事情不堅持下去.
它就不是教會.
我正正是第一堂的時候.
一定很認真地跟弟兄姐妹說得清清楚楚.
其實有得讓你選擇.
你不一定選擇Full church.
我的同工說.
這些是趕客的說話.
但事實上.
我們是執行.
仍然是愛所信聖經的教導.
教會的堅持是甚麼?.
這裡我不會重述Info Group的事.
如果你聽過你記得.
特別是有些問題在過去兩年.
運作教會難到一個點.
但我們覺得仍然要做就要做.
仍然要守就要守.
那個難處對我們來說.
其實可以鬆一鬆手就簡單很多.
就不需要搞得那麼複雜.
但對我們來說.
我們覺得有些事情是不可以變的.
所以弟兄們說.
效法人是需要的.
但最重要的內容是建基於聖經.
弟兄姐妹今天我們如果想.
不要說論斷.
給一些評論的時候.
又或者建議的時候.
或者關心的過程當中.
我們有沒有一個.
聖經給我們一個提示.
又或者一個觸動呢?.
說到這些terms好像很熟悉.

$^{441}$其實問根本就是.
你本於什麼動機原因去說那句話.
有時你自己不太覺.
我常常提醒自己.
不要過口快過腦.
有時過口快過腦就是什麼?.
就是叫什麼?.
跟到我?.
就是不經腦.
是不是?.
就是過口快過腦就是不經腦.
對不起說錯了.
說錯了就是說錯了.
說錯了之後怎樣?.
我教兩個兒子.
今天我兩個兒子要見家長.
所以像正一點.
我跟兩個兒子說.
說錯了就是說錯了.
之後要做什麼?.
你們不是我的兒子.
但我想你都應該猜一下.
會怎樣?.
就是.
你說錯了就要說對不起.
收不回來.
是呀,你說了就收不回來.
但不代表你不用道歉.
所以你說之前.
你想想你說那句話.
其實你想達到什麼目的?.
你說的時候.
你想期望對方有什麼反應?.
你說的時候.
你在想什麼?.
其實這個是要make sense.
今天你發覺make sense是很難的.
因為common sense is not common.
是不是?.
但重點是什麼?.

$^{481}$重點就是聖經提醒我們.
快快的聽,慢慢的說.
一個大家很熟悉的.
《雅各書》第一章的提醒.
信徒有的行為.
信心而有的行為.
第一件事就是說你的wordings.
你用什麼詞彙.
用什麼方式去表達是重要的.
因為雅各是什麼?.
雅各是當時處理耶路撒冷教會.
和安提亞教會的衝突.
就是大家的wordings上的用詞.
太尖酸.
要求太多.
令到教會第一個會議當中.
其實是失敗的.
聖經有沒有說呢?.
我之前都說到,都說過.
雅各最後就是一錘定音.
就是本於上帝將福音.
推向萬民的原意.
我們去接納外邦人.
歸順基督.
他不需要守猶太人的律法.
本於聖經的教導.
這個是很重要的.
直至今天都是.
我希望我們這個信仰的群體.
一起聚集.
守著我們賴以為生的聖經.
這間教會的運作.
三年了.
就像說那麼多教會.
我們仍然不是因為一個人.
不是一群人.
而是因著我們仍然在捍衛.
和運作聖經對教會的要求.
所以原則就是.
有什麼不是領受的呢?.

$^{521}$全部都是來自我們被耶穌救贖.
我們認同教會應該這樣走下去.
反而你反過來.
做得久了.
你覺得自己威風.
做得久了.
覺得自己有些渣拿的時候.
就自己作黃鳥.
保羅就是提醒那群那麼多教會的人.
其實有什麼是給予的呢?.
全部都是給你.
上帝給你的.
所以你會看到.
他違背了上一章的投影.
他做了一個僕人.
當主人還沒回來的時候.
他以為自己是主人.
所以他就可以說三道四.
他就可以論斷人.
他就說人不是.
親弟姐妹對我們來說也是.
我們周邊可能都見不少.
或者你的團隊.
當然我不要開口重了.
或者你見過的人的時候.
但是對你來說.
你可能排他不到.
離開不到.
或者你轉不了工.
但是你記住.
或者我自己常常提醒自己.
正正有人板在這裡.
我就不要做你這樣.
這個很重要.
如果不是這個世界又多了一個人.
又少了一個認真的人.
再下去的時候.
第九節.
是一句我自己覺得今天很特別.
要跟大家說的話.

$^{561}$「我想上帝把我們仕途明明列在幕後.
好像死罪的囚犯.
因為我們成了台戲.
給世人和天使觀看」.
這節經文在第四章裡面來說.
第四章有21節的經文.
第21節是一個問句.
但是在第一至二十節當中.
中間那一節就是這節.
第九節的經文.
本來在第九節經文裡面帶出一個訊息.
或者這一章中間的經文帶出一個訊息.
是一台戲.
在這個檔期大家看了很多戲.
不用問.
我想大家如果看了戲的話.
你有份幫他捧到第一名.
破了票房.
大家看戲的時候帶著什麼心情呢?.
有些人是認真.
有些人是假期想找一部戲消遣一下.
但是你會發覺你看過很多戲.
你有自己的想法.
有你自己的喜好.
你有沒有想過你自己也是一台戲呢?.
胡應詩不是唱人生如戲那些.
不是那些吧.
上演到人生盡….
不是唱那些.
胡應詩不是唱那些.
不用擔心.
都不用擔心的.
不是.
但你有沒有想過你自己如果真的是一台戲的話.
你會怎樣去做這台戲呢?.
你有沒有想過呢?.
我有想過.
有些電影節目比熟悉一點.
我曾經聽過說過.
我為我自己預備了安息禮拜.

$^{601}$不用….
你應該沒有問.
因為我自己覺得.
有些歌我喜歡.
我想在安息禮拜裡面有人唱.
還有我覺得.
我去過不少安息禮拜.
我覺得安息禮拜那些訊息不行.
我會自己說的.
我認真的.
我曾經跟我老婆說過.
她不知道有沒有看直播.
不過她應該會看的.
我曾經跟她說過.
我自己說.
我老婆說.
你知道什麼時候死就說.
那又是.
如果突然死亡.
心臟病說不出.
那怎麼辦呢?.
要不要錄下來呢?.
錄下來的時候可能要改.
不然罐頭就沒有意思了.
所以我一直在想.
安息禮拜這部電影做完了.
借幕的時候我要說什麼呢?.
保羅在他的生命歷程當中.
很多時候在每個階段都有一個評價.
就是這一幕劇完了有什麼呢?.
就好像我們那個轉播研究所的時候.
我們有不同的一幕幕劇.
但你再深思.
每一幕劇都有一個訊息.
讓你在不同階段的信仰反映出什麼.
今天你去到這個歲數.
或者去到這個階段.
你那一幕劇是什麼呢?.
那一幕劇有什麼呢?.
那一幕劇有人看的.

$^{641}$就是你周遭環境的人.
跟你相處的人在看著你.
而讓你又給什麼人看見.
這就是你刻意經營的.
在《神學院》裡面.
我自己教了一科是個人成長的科目.
其中一本書是學生一定要看的.
很薄的書,八頁而已.
不難看的,那些字.
但要真的去答真的書.
就相當花時間.
那本書是David Benner寫的.
就是《The Gift of Being Yourself》.
中文是科幻正主翻譯的.
叫做《天賦給我的禮物》.
曾經斷版的.
後來因為不斷開課都叫人訂.
就再版了.
不關我的事.
那本書我覺得比較靠近.
值得可以做評論的書.
其中David Benner說的一個訊息就是.
在環境當中.
我們做一個信主的人.
很多時候都是發展出一個真實的自己.
讓別人去認識.
FALSE的真實的自己.
那個假意思不是你裝假.
就是你覺得那樣讓人感覺到你更舒服.
那個真實的自己.
但那個不是最真實的你.
你的真實的自己很多時候都是被保護的.
因為你不夠膽.
用你的真實的自己去讓別人看.
不是說FALSE的自己是錯的.
但是這部電影讓人看到的是甚麼.
很多時候真是一個FALSE的自己.
但這個世人.
這個世人在看的可能都是一個FALSE的自己.
但是天使會看的.

$^{681}$恐龍很特別.
他講了天使.
天使是甚麼?.
天使是一個中面.
天使往往出現.
你看科林書或舊約.
天使往往出現就是上帝的代表.
天使的出現就代表上帝的臨在.
所以你會看到這部電影不是你旁邊的世人在看.
連上帝的代表都在看.
即是上帝都在看.
天使是Angelus.
希臘文.
其實意思是Messenger.
他傳達訊息,傳遞者.
所以這部電影你正在做的時候.
天使是在看.
然後訊息又帶回給上帝.
你的FALSE SELF是讓人看見.
但你的FALSE SELF都是在被上帝看.
我們是怎樣做這部電影呢?.
我在轉網研究所第一場謝票的時候.
上台的時候.
因為將最真實的東西呈現出來.
我到這一刻仍然相信.
只要那件事是真實的.
就會感動人.
但那真實不是我告訴你真實.
而是你自己知道那件事是否真實.
我不會知道那是真實的還是FALSE SELF.
因為我看到的仍然是你讓我看的那件事.
但你自己知道那件事是否真實.
而天使都是在看我們在看的那件事.
所以在這段經文裡.
很特別的就是.
這部電影有幻面的就是世人都在看的.
你旁邊的人都在看的.
弟兄姊妹在看的.
家人在看的.
朋友在看的.

$^{721}$但上帝的中面都在看我們在做這部電影.
但你看不看到你自己怎樣演這部電影呢?.
這就是弟兄姊妹你要靜下來去做這個評價.
你用甚麼方式.
你用甚麼力度去演這部電影呢?.
是真的.
對我來說一定是辛苦.
對你來說也是辛苦.
但我經常都說這個世界假的事太多了.
其實不需要你再做多樣假的事.
會不會大家做真事.
但做真事對你來說最挑戰就是怕受傷.
是不容易的.
但是主人是看在我們做真事情.
放在祂的眼裡.
上帝是看著我們怎樣去演這部電影.
在這兩年.
我看見大型的書店.
或者是下載量.
都很多人下載John Orwell的《1984》.
我不知道大家有沒有看過這部電影.
或者你都不難聽過中間的人不同的引用.
其中一句說話.
《1984》裡面John Orwell說的說話.
誰能控制過去就能控制將來.
而誰能控制現在就能控制過去.
這些哲學家說話真是.
吊來吊去就要你思考一段時間.
但要深入一點說.
他要指出一件事.
就是誰有能力詮釋歷史就是在控制.
再深入一點想.
你會發覺教科書上的修編.
將一百年前的東西修改.
過了一代人之後就忘記了一些東西.
誰有能力詮釋歷史.
讓舞台的劇本改了.
其實就會有一個控制.
今天對我們來說.
我們看見的東西.

$^{761}$我們有什麼一定要抓緊呢?.
今天我們看見的東西.
有什麼我們要說出來指正呢?.
是很多但又很難.
當你看電影的時候.
覺得最後的劇本很重要.
所有事情都是錯的時候.
你有沒有能力告訴你錯在哪裡呢?.
我一看的時候就覺得.
演戲吧.
你看很多評論都說悲哀但事實.
但對我們來說.
我們的信仰是否在這一刻.
你要堅守的什麼呢?.
就像上第一堂AV Group的時候.
都告訴你.
有些東西是很難的.
簡單一點就不用那麼複雜.
但我們要將一些東西.
要認真看待的時候就複雜了很多.
減慢了很多.
但在信仰上.
有些東西一定要堅實做好.
其實因為我們不是在做.
讓弟子妹看到.
是讓上帝都在看教會.
怎樣運作教會.
John Arroyo剛才說的控制時間.
因為歷史裡面說的.
歷史書很多時候都是反映.
當前掌控權力的主事人.
他可以控制的空間.
和他的力度在哪裡.
所以歷史書通常在當代寫的時候.
都是歌功頌德的.
過了之後就會拿出來改.
唯有司馬遷寫史書的時候.
他寫得很真實的時候.
最後他死在行刑.
公刑事後.

$^{801}$我自己很喜歡一個時事評論員.
他過世了.
他叫李怡先生.
我自己很喜歡他.
我喜歡看他的書.
他很有智慧對於社會.
對於歷史.
對於人心.
很有智慧.
很多時候都很喜歡看他講話.
其中有一樣東西.
他講關於歷史.
關於人怎樣去鋪排的時候.
他講這番說話就是.
歷史除了年代和人名是真之外.
其他都是假的.
小說除了年代和名字是假的.
它的內容是真的.
這句說話我初時不是覺得很震撼.
但是自從2011年,2014年.
2017,2019,2021這些時候.
你會發覺很多東西.
原來現實都會出現.
調轉的時候.
就可以好看過電影.
是很無奈的.
是不容易的.
但是上帝就看看.
我們自己的電影怎樣去掩護你這個角色.
我做不到你的電影.
你做不到我的電影.
但每個人都在做電影的時候.
旁邊的人都在看我們.
但上帝都在看我們.
看我們怎樣去做角色.
你的修養,你的表達空間是什麼呢?.
最後你會看到.
就是保羅說.
我寫這話就是要我們學基督.
基督在地上三年多的日子.

$^{841}$其實是不舒服,不順利.
也不是眾人所要有的.
期望角色的扮演和效果.
但他成就的就是福音.
因為福音就是上帝的大能.
要叫一切相信.
你信不信得過耶穌?.
你信不信得過耶穌的能力改變?.
這個都是常常在問自己.
當我兒子問我.
電視新聞不是其他報道一樣的時候.
或者他見到有人說謊的時候.
我記得家裡一起看新聞的時候.
我兒子就問潘全道.
他說謊.
我的答案就是.
你都知道他說謊.
我也知道他說謊.
你猜他知不知道自己說謊?.
你看見的是什麼?.
如果你知道說謊.
我知道說謊.
他不知道自己有沒有說謊的時候.
不重要.
但上帝也在看這件事.
只不過是他在上帝面前.
積蓄上帝對他的憤怒.
如果這個世界沒有審判的話.
做總統和飯桶是沒有分別的.
但你要記住這個世界有審判.
而地獄的火是真實的.
如果地獄的火不是真實的話.
耶穌是不需要為我們死的.
你明白嗎?.
所以認真的信仰.
是認真去想現在的人怎樣做這台戲.
而我們怎樣做我們這台戲.
你不要理人.
所以保羅說你不要論斷人.
我自己也不論斷自己.

$^{881}$但我做什麼上帝是知道.
因為我會和我的主交著.
祂會顯出暗會的事情.
你明白嗎?.
我希望今天的訊息你再看.
聖經是不會過時的.
聖經是會看我們現在面對的情景當中.
在保羅在操縱多倫多教會三年的時候面對的情況.
其實人是沒有變過.
我們有本書去跟的時候.
我們也不認真去跟.
我正正告訴你.
我們所有的事情都是本於聖經的教導.
這個就是我們怎樣面對上帝給我們的劇本.
你明白嗎?.
最後結束的時候.
我自己選擇了希伯來書的教導.
是信心篇的教導.
是一段圖文.
我自己覺得.
這段圖文就是.
我按不到.
是,OK,謝謝.
希伯來書第十章32至36節是這樣說的.
「你們要追念往日蒙了光照以後所忍受大真賤的各樣苦難」.
「一面被毀謗遭患難」.
「成了氣境叫眾人觀看」.
「一面陪伴那些受這樣苦難的人」.
「因為你們體恤了那些被捆綁的人」.
「並且你們的家業被人搶去也甘心忍受」.
「知道自己有更美長存的家業」.
「所以你們不可丟棄勇敢的心」.
「全這樣的心必得大賞賜」.
「你們必須忍耐使你們行願了神的旨意」.
「就可以得著所應許的」.
這是希伯來書說信心篇裡面一個我自己很喜歡的提醒.
上帝知道我們難處.
因為他自己也經歷過難處.
上帝知道很多事情不是即時看到果效.
因為他三年多也沒有即時的果效.

$^{921}$人們也不太知道他在做甚麼.
但他仍然默默去做他在地上的那台戲.
就是走上各個他山成就救恩.
讓福音轉化人的那台戲.
不容易,但每一個跟隨他的人都要學習.
我稱之為甚麼呢?.
我稱之為我們每一個演一台戲的時候.
我稱之為演員的自我修養.
不是史坦路,拉夫斯基那個.
是我們每一個在做自己的獨腳戲.
或者跟別人合作的戲當中.
我們怎樣可以演好這句話.
青頂姐妹你跟我們一起崇拜.
跟我們一起小組.
跟我們一起在這三年多的日子.
仍然希望可以走更遠的路.
仍然希望在地上可以讓更多人得聞福音.
讓更多可能很久沒有上課的頂姐妹可以出來.
看了網上很久,覺得看完就完成了.
我們希望她跟我們一起實體聚會.
這是上帝給我們的福分.
也是我們可以一起參與.
讓世人觀看,讓天使觀看.
我們這麼大的敬拜隊.
一起享受上帝給我們的這個現職.
一會兒回應詩的時候.
是唱全信靠上帝.
和剛才大家唱第三首的詩歌.
結尾的說話就是我自己很重要的心聲.
無法可講述.
願唱歌稱讚.
讓我見證上帝救恩.
將我導引.
我的劇本上帝仍然會帶著我寫.
但上帝仍然帶著你寫你的劇本.
看看你自己的劇本.
看看身邊的人.
天使在觀看.
因為上帝在觀看.
我們一起祈禱.

$^{961}$姐妹當我們打開說話.
是你的說話.
你讓我們看見.
上帝作事直到如今.
你的說話在今天仍然對我們受用.
又讓我們在經文當中.
不只是看到經文上的勸諱.
乃是身體力行當中.
做到上帝你給我們的那種風骨.
一個基督徒有的持守的風骨.
一個教會能夠一起去.
立命在這個世上.
讓人得聞福音.
改變大人的教會.
求主你幫助我們.
此路難行.
但主耶穌已經讓我們去經歷.
亦求主你繼續帶領我們.
在我們人生的舞台上.
盡力去演.
將真實的一面讓人們看見.
感動多人參與主你的教會.
歸心於你.
祈禱奉耶穌的名求.
阿門.
拜拜!.
\newpage



\section{撒迦利亞書 2:1-5-20230218}
\label{sec:4Dll86a7b18}
\textbf{【流堂崇拜】講道者的自白和看見|撒迦利亞書2\_1-5|20230218 [4Dll86a7b18]}
\newline
\newline
連結: \href{https://youtube.com/watch?v=4Dll86a7b18}{\texttt{ https://youtube.com/watch?v=4Dll86a7b18}} ~~~~ 語音日期: 2023-02-18 
\newline
\newline
\hyperref[sec:KAnMTZ32Dag]{\small{< < < PREV SERMON < < <}}
~
\hyperref[sec:index_chronic]{\small{[返順時目]}}
~
\hyperref[sec:index_scriptual]{\small{[返順卷目]}}
~
\hyperref[sec:lsdGk_BkHa8]{\small{> > > NEXT SERMON > > >}}
\newline
\newline
撒迦利亞書 2:1-5-20230218
\newline
\begin{longtable}{cl}
\hline
\hline
章節 & 經文 (和合本修訂版)\\
\hline
2:1 & \begin{tabularx}{0.7\textwidth}{X} 我舉目觀看,看哪,有一人手拿丈量的繩。 \end{tabularx} \\ \\ \relax
2:2 & \begin{tabularx}{0.7\textwidth}{X} 我問:「你到哪裡去?」他對我說:「要去丈量耶路撒冷,看有多寬多長。」 \end{tabularx} \\ \\ \relax
2:3 & \begin{tabularx}{0.7\textwidth}{X} 看哪,與我說話的天使出去,另有一位天使迎著他來, \end{tabularx} \\ \\ \relax
2:4 & \begin{tabularx}{0.7\textwidth}{X} 對他說:「你跑去告訴這個年輕人說,耶路撒冷必有人居住,如同無城牆的鄉村,因為其中的人和牲畜很多。 \end{tabularx} \\ \\ \relax
2:5 & \begin{tabularx}{0.7\textwidth}{X} 耶和華說:『我要作耶路撒冷四圍火的城牆,並要作城中的榮耀。』」 \end{tabularx} \\ \\ \relax
2:6 & \begin{tabularx}{0.7\textwidth}{X} 耶和華說:「來,來!你們要從北方之地逃回;因我曾把你們分散到天的四方。這是耶和華說的。」 \end{tabularx} \\ \\ \relax
2:7 & \begin{tabularx}{0.7\textwidth}{X} 來!住巴比倫的錫安百姓啊,逃吧! \end{tabularx} \\ \\ \relax
2:8 & \begin{tabularx}{0.7\textwidth}{X} 萬軍之耶和華在顯出榮耀之後,差遣我到擄掠你們的列國那裡,他如此說:「碰你們的就是碰他自己眼中的瞳人。 \end{tabularx} \\ \\ \relax
2:9 & \begin{tabularx}{0.7\textwidth}{X} 看哪,我要揮手攻擊他們,他們就必作自己奴僕的擄物。」你們就知道萬軍之耶和華差遣了我。 \end{tabularx} \\ \\ \relax
2:10 & \begin{tabularx}{0.7\textwidth}{X} 耶和華說:「錫安哪,應當歡樂歌唱,因為,看哪,我要來,要住在你中間。 \end{tabularx} \\ \\ \relax
2:11 & \begin{tabularx}{0.7\textwidth}{X} 在那日,必有許多國家歸附耶和華,作我的子民。我要住在你中間。」你就知道萬軍之耶和華差遣我到你那裡去。 \end{tabularx} \\ \\ \relax
2:12 & \begin{tabularx}{0.7\textwidth}{X} 耶和華必收回猶大,作為他聖地的產業,他必再度揀選耶路撒冷。 \end{tabularx} \\ \\ \relax
2:13 & \begin{tabularx}{0.7\textwidth}{X} 凡血肉之軀都當在耶和華面前靜默無聲,因為他從他的聖所奮起了。 \end{tabularx} \\ \\
[1ex]
\hline
\hline
\end{longtable}
$^{1}$做一個教務其實有福氣的.
因為可以跟新人證婚.
見到很多對新人的結合是很開心的.
做一個教務都很開心的.
因為可以有很多安息禮拜做.
開心的不是因為見到人過身.
開心的是因為你可以在很多人的傷痛裡.
你有這個福氣可以陪人走過.
尤其是為很多人難過的時候.
你可以跟他們一起經歷.
十多年前跟我一個家人去籌備婚禮.
做教務有更大的福氣是.
自己的家人結婚的時候.
在教堂裡面我做主禮和訓練.
那時候我覺得可以為自己的家人.
在他們結婚的裡面能夠駐禮和訓練.
是上帝給我的很大的福氣.
想不到的是在過去的星期裡面.
其中一個家人的另一半突然離世.
突然離開.
在這幾天籌備喪事的時候.
有更想像不到的是.
你可以在喪禮裡面.
你負責主禮,你做慰眠.
我估計要接受突然離去的失去.
其實從來都不是一件容易的事.
開始明白這幾天.
可以說是哭到眼淚都不想再哭.
這幾天你拿著紙巾碰眼睛的時候.
你覺得眼睛會痛.
那時候你就會跟自己說.
不要再這麼容易哭了.
這番說話我已經練了很多次.
昨晚和今早我已經練了很多次.
希望我能夠冷靜一點說.
John突然走過來坐在我旁邊.
他用很慈性溫柔的聲音跟你說.
你怎樣?.
多謝你.
我估計面對離開失去不是一件容易的事.

$^{41}$今天早上,不,是早上完結後.
下午去了我家人的家.
他們只剩下一個七歲多的孩子.
我拿了很多他爸爸小時候的照片.
我們看他沒見過的照片.
因為這幾天收集照片.
看很多他爸爸小時候的樣子.
告訴他爸爸小時候是怎樣.
然後拿一本繪本.
說一條鯨魚失去了爸爸.
會經歷一些什麼情緒.
繪本看到四分三的時候.
孩子已經衝入房間哭了.
哭了一個多兩個小時.
我估計這個時間是很不容易的.
但到今天這一刻.
見他剛才一起吃飯的時候.
我不想說這個時候有什麼上帝的恩典憐憫.
這些好像很罕見的說法.
但能夠可以跟他一起經歷.
一起面對.
我相信這是在生的人可以做的福氣.
香港面對很多的難處.
有些人的配偶在流離的旅邊.
但他仍然被監禁.
不能與自己親愛的人一起共度最後的時刻.
新年的時候有很多去了英國加拿大的人.
有些回來了.
也有很多在這次拜年的旅邊.
基本上我們在農曆新年的拜年.
很多時候都要Zoom meeting.
在一個不容易的情況下.
人與人之間的關係.
因著各樣的緣故被分離.
我相信這些現況和感覺.
是香港在這個地方的旅邊.
我們很常見的事情.
或是在日子的旅邊.
我們也會更常見到這些事情.
心願面對這些難過.

$^{81}$面對這些不容易.
面對自己覺得眼淚.
叫他不要再流的一刻.
我們不是要求眼淚止住.
而是要馬上開心.
求的是自己可以在這些旅邊.
與更多的人一起經歷.
在這個年代的旅邊各樣的傷痛.
如果大家在這兩個月的旅邊.
經歷了這些離別的時候.
心願這些離別.
與自己的心靈說一番重要的話.
生命無常.
活在當下.
好好珍惜眼前見到的每一個人與事.
其實早前提到一個題目.
叫做「一個港島者的自白」.
這個題目其實是我想說的.
這個題目是因為一些事情.
我們先看看我們今天讀的經文.
我們讀了很短的.
讀了五節,我們不讀完.
十三節,是塞加利亞書的二章.
二章的五節是這麼說的.
他說他觀看有一個人.
手裡拿著量度羽絨的繩子.
於是他就問:你要去哪裡?.
他跟我說:我要量度耶路撒冷.
看看它有多寬有多長.
那與我說話的天使正要出來的時候.
有另一位天使迎著他來.
對他說:你快點跑去告訴那個年輕人.
耶路撒冷終於有人居住.
好像沒有城牆的地區.
因為其中有很多人羽生畜.
我想今天不是說很長.
我今天想你看著紅色和那些不知道叫什麼顏色的字.
可能讀藝術的人會給我一種顏色.
但我不想叫他橙色那麼傻.
讓你取笑我.

$^{121}$那種顏色的字.
通常我們在塞加利亞書.
我們會看到二章.
塞加利亞書裡面有八個二章之多.
基本上第一個二章和第五個二章類似.
第二個二章和第六個二章相類近.
第三個二章和第七個二章類似.
而第四個二章和第八個二章有些關聯.
這個是第三個二章.
所以它應該和第七個二章有些關聯.
但我今天不會說那麼多二章的東西.
我想偏頗一件事.
我想問的是在先知書裡面.
我們看到很多二章.
或者我們在小先知書裡面.
塞加利亞書是一本小先知書.
我想問的問題是.
這些二章是什麼?.
看到有一個天使.
有一個人,其實應該是天使.
不解釋那些東西.
量著一些東西量耶路撒冷.
無聊至極.
對我們今天意義都不大.
譬如有一個天使來.
告訴年輕人.
耶路撒冷將會有人居住.
因為那裡好像沒有城牆地區.
所以有很多人與生畜.
但我們讀利希米記的時候.
不是建成城牆的嗎?.
塞加利亞書的二章裡面.
它說為什麼沒有城牆地區.
有很多人和生畜呢?.
所以問題是問.
那些二章是什麼意思?.
我們要問的問題.
二章想代表什麼?.
二章其實在小先知書裡面.
它在表達,在說的是一件什麼事情?.

$^{161}$我們用幾張文章.
麻煩文俊.
我們用幾張文章.
我想先到以色列的書.
以色列的書的文章.
你幫我按下去.
應該可以按下去.
按下去就應該有.
多謝.
你看到以色列的書.
四十章它是這樣說的.
我們很快地讀.
在神的二章裡面.
它帶我到以色列.
安置在一個很高的山.
在山的南面.
好像很像城的建築物.
它把我帶到那裡.
看到一個人.
好像是銅手.
裡面拿著麻繩和測量器.
站在門口.
然後我又看到電外的四面牆.
那個人的手又是亮著綠爪.
每爪都在打.
你可以數下去.
或者你看著你手上的聖經.
即是手機.
你發現其實四十章至四十八章.
都是在說耶路撒冷的故事.
問題是為何以色列的書說得那麼長.
四十章至四十八章.
而撒加利亞的書.
只有一句說話說.
這是我第一要問的問題.
二章原來是重複的.
撒加利亞的書說的二章.
和以色列的書說的二章.
以色列的書用了九章聖經去說.
又要量度南門,東門,北門,西門.

$^{201}$如果你望一望.
四十一,四十二,四十三章的.
以色列的書就是在說這些東西.
我的問題是.
為何撒加利亞的書用一句說量完.
它沒有量完.
找人量一下.
很短的.
以色列的書用了很多章聖經去量.
我再按多一個powerpoint.
是以塞亞書的.
以塞亞書65,66章說什麼呢.
它很快就要創造新天新地.
先前的事沒有人再記住.
也不會再浮現在腦海裡.
也不會再有那些活數日的鷹孩.
意思是不會夭折.
不會有天年不遂的老人.
即是說要活到很多年的老人家.
不會因為他很年輕而死.
大約是這個意思.
一百歲死都是年輕.
活不到一百歲就是被奏助.
然後下面一句.
「豺狼與小羊一起吃草」.
「獅子與牛一起吃草」.
「蛇以犁土為食物」.
然後就不再有傷害的說法.
你看著豺狼和小羊.
獅子與牛.
如果你今天放在一起.
就一定有一隻會死.
你明白嗎?.
所以這幅圖畫是上面第十幾個.
十七節的說話.
是Sunday Sunday.
意思是Sunday Sunday裡面.
在二創書中的Sunday Sunday.
說的是甚麼呢?.
是說將來的Sunday Sunday.

$^{241}$所有東西都要重新創世紀.
亞當夏娃的時代.
所有的飛禽走獸都會很友好地在一起.
就是這樣.
剛才的說法.
如果你還記得在《撒加利亞書》.
我們去到那二章.
剛才的顏色那裡.
他說因為其中有很多人與生畜.
這句說話就等同於.
以塞瓦書65章和66章.
說的幕後的Sunday Sunday的故事.
如果以西傑書9章聖經說.
梁,聖殿,關於聖殿的事.
這裡只用紅色的說完.
而以塞瓦書65和66章.
很多篇幅說的幕後的Sunday Sunday.
發生了什麼事的話.
而這裡因為在其中有很多人與生畜.
就是正正在說.
以塞瓦書65和66章那些異象.
好了,到今天真的要說的話.
介紹完這些之後.
我想說什麼.
以塞瓦,以西傑.
大約主前六百多年.
先當是這樣吧.
這些Dating我不正確了.
《撒加利亞書》什麼時候.
你當是五百年,五百零幾年.
最久的或最接近的是五百二十年.
相差大約一百年的時間.
我當是以塞瓦和以西傑的圖畫.
比對《撒加利亞書》.
就是回歸之後.
尼西米爾,以斯拉那些故事.
完結之後才說《撒加利亞書》.
所以相差一百年.
我想表達的問題是.
異象這東西.

$^{281}$不是說我今年有個異象.
我下年有另一個異象.
異象不是說.
我突然想起今天神叫我做這件事.
我做,然後到明天就突然做另一件事.
異象對於我們來說.
有時候會覺得是什麼.
是那些很神怪的東西.
或者很奇怪的東西.
好像很一下就要做的東西.
但如果你看清楚《小仙子書》裡面說的異象.
證明其餘《撒加利亞書》裡面的八個異象中的其餘七個.
其實其餘七個異象都和《大小仙子書》裡面的異象差不多.
所以我今天選擇最容易說的異象.
它只不過重覆在以西傑書.
和以塞俄書裡面所表達的異象.
所以你明白它為何寫得這麼短小.
因為這個異象不再需要重覆說.
因為其他書卷已經說了.
如果是這樣說的話.
《撒加利亞書》這兩個異象只不過重覆上帝給以前的先知裡面留下來的異象.
而這些異象對於撒加利亞這個年代.
過了一百年之後的那班人.
仍然守緊著那個異象而不放棄.
期盼新的耶路撒冷出現.
不要像哈格諸所說的天花板漏水的聖殿那麼up 心.
說的耶路撒冷不是說很荒涼的耶路撒冷.
所以為何《利美記》裡面.
以色列劇裡面要記載回歸的人.
因為回歸的人應該很多.
但回歸的人其實只有很少.
而它紀念回歸的人肯回歸下來.
耶路撒冷荒土裡面.
被廢了七十年的荒土裡面.
誰都願意回來就行了.
所以《上行之詩》其中有一篇詩篇是很無聊的.
是說耶路撒冷生孩子特別興旺.
不是因為耶路撒冷風水好.
不是因為耶路撒冷甚麼.
是因為那裡的人不夠多.

$^{321}$所以請你回到耶路撒冷生多些孩子.
一起來建立那個荒廢的耶路撒冷.
如果大約這樣理解和明白的時候.
我們會表達甚麼.
原來異象對於我們來說.
起碼在這裡.
是一百年的追求.
不是我這一代人完.
我下一代人就有另一個異象.
是我們這一代人完成之後.
好像完成不了.
而下一代人繼續接著.
而他們說的時候不需要再回應.
以西傑書四十節碩畢這麼長.
不需要再回應.
以亞細亞六十五節碩畢這麼長.
就是紅色和那種不知甚麼顏色的.
一句就牽動到.
這個異象上帝繼續要堅持下去.
所以如果你今天離開禮堂的時候.
請你記得.
小仙之旅書裡的異象.
從不高深.
從不莫名的難搞.
不是突然說一些我不明白的東西.
不知說甚麼.
今天你不會走出來量希堂.
你不會量全教會的歡迎回家的牌.
有多高有多闊.
有多窄的顏色.
你不會做這些事.
異象是想表達和說.
是一代傳承給下一代.
而這些天啟的出現是在說甚麼.
是因為沒有人再承傳.
才唯有用天啟.
我們叫它做apocalyptic literature.
用所謂天啟文學的手法去表達這些事情.
為何我要說這段.
是因為上個月在Fold Church講完.

$^{361}$一組人.
他叫我不要爆他們.
不過不好意思.
你都不知道是哪一組.
一組Fold Church的頂尖姊妹.
就無緣無故送了張卡給我.
那張卡很大張.
裡面每個人都寫了一句說話.
大約是這樣.
我懷疑我這個年紀.
我20多30歲在幕會做青少年的時候.
在中國生日的時候.
他會送一張卡給我.
寫了這麼大張卡為家具.
還有sticker和小貼紙貼緊.
我收到之後.
驚訝的程度是.
好像回到二十多年前.
做中國生團契的時候.
你生日還是甚麼日子的時候.
送一張這麼大的卡給你.
收到之後.
我有點害羞和不好意思.
為何攝影王送這麼大的卡給我.
我問為何呢.
頂尖姊妹是他說的.
他說家Sir多謝你這幾年的講道.
那一刻我基本上不夠時間反應.
其實我真的很開心.
開心到一個地步.
去到地鐵的時候.
我馬上打開.
看每一張紙說甚麼.
對於我來說.
基本上我很少很少.
有人會.
一早有人說多謝你講道.
那張卡我就放在我家.
我女兒的琴的上邊.
放在那裡大約三四個星期.

$^{401}$然後我再運回神學院.
然後整個球.
一次過把那張卡放在中間.
貼在那裡.
你可以想像我多麼珍惜那張卡.
這麼多年來都不再收到這些卡.
如果我再收到也很奇怪.
但很久沒有收到.
總之是多謝你.
你知道在地鐵上我打開每一張的時候.
你知道我們.
甚麼呢?.
不知道要說甚麼.
你知道你要分那些.
很客氣跟你說的話.
你明白我嗎?.
多謝你啊.
那些.
多好啊.
你知道坐了一段時間.
你都會感受到那些人寫的東西.
是隨便寫還是怎樣.
他真的每一句寫的說法.
他都是在說你講道的關連.
其中有一個是一兩個.
我是很感動的.
他說家Sir.
初時聽你講道不知道你在說甚麼.
他說我們現在終於明白了.
這句說話引致我今天要說道的所以言.
對一個講道者來說.
或者對一個講道的人來說.
我講我自己.
我也想講一些大家開心聽得容易的道.
我也想講一些.
寫一些東西出來.
或者講一些東西出來.
令人開心的東西.
或者世界上有很多人.
發生很多事情.

$^{441}$不同的東西出現.
你都可以講一些東西.
或者寫一些東西出來.
回應現在發生的事.
但對於我自己來說.
講道不是在說.
一個反應的過程.
沒錯我們會回應這個時事.
或者我們在回應現在的事.
但一個講道者對於我來說.
講話或者講道.
真正的意義是在說.
你看著聖經在說甚麼.
你只能夠如實地去說聖經想說的東西.
或者你花了很長時間去準備.
這個聖經在說甚麼.
其實我問了很久的問題.
就是小仙主義上在說甚麼.
找到小仙主義上在說甚麼.
可能是花了十年二十年的時間.
突然一下子你驚醒了自己.
原來小仙主義上大約是這樣.
從來不將經文方便自己化.
從來不容許經文為自己所說的東西去服務.
起碼基本上對於一個講道者.
我自己來說.
到今天仍然堅持.
而我堅持的是.
我不想很容易地在一個星期內.
去回應很多東西.
可能別人可以.
我不是說別人.
我是說我自己.
我沒有這個能力.
沒有這個能力的原因.
不是因為我不可以寫一些東西出來.
是因為那些東西的問題.
是我回應純粹因為要反應.
去做這個動機去做這件事情.
還是我真的有些東西.

$^{481}$是關於這個課題.
我想過我想說的.
還是在一個潮流.
要說這些東西.
所以要說這些東西.
對我自己準備的一篇講道來說.
這些年,二十多年.
每一篇道,準備完.
我都花了一兩個星期的時間.
去問自己這篇道對我的意義是什麼.
我要足夠有兩個星期的時間.
這篇道跟我說的話.
令我有一個悔改的心.
這篇道第一時間悔改的那個.
不是頂智妹,是我自己.
所以你明白.
我的講道理.
我從來不會說宣教士見證.
我從來都不會說很多那些.
一百年前,二百年前.
那些術明偉人的歷史.
那篇道的內容.
是因為我這兩個星期.
要來的時候,要說的時候.
那篇道跟我說什麼.
它提醒我什麼.
要我悔改什麼.
講道不是娛樂.
講道不是一個.
可以令下面開心的東西.
但我要跟你說的是.
在這麼多年間的裡面.
我永遠聽著的是什麼.
我想快點說.
我想快點寫.
我想快點完.
我想有些東西讓人知道你在這裡.
你想說一些話告訴別人.
你在這些議題裡面.
你有一個發言權.

$^{521}$你想說一些話.
你想說一些容易的東西.
讓下面的人容易明白理解.
以致你會受歡迎一點.
他們說這十幾年的裡面.
充滿著這些誘惑.
尤其是當頂梓梅跟你說.
聽你講道不明白的時候.
我跟自己說的是.
如果一個群體的裡面.
要講道的話.
其實你需要十年的時間.
才能夠.
牧養頂梓梅對聽道的質素提升.
聽道不是聽開心的東西.
不是聽一些容易入口的東西.
沒錯,我也要說一些簡單的.
說得容易一點.
讓Job Look的姐妹們寫得快一點.
我經常有這些掙扎是真的.
但如果講完一篇道.
一個訊息.
不是在經歷難處.
就像那些異象一樣.
一百年前講完.
一百年後的掙扎還在.
所以我貼完那個東西.
在我家的鋼琴上.
我出入看著它的時候.
我問我自己在看什麼呢?.
我知道我自己想要什麼.
我也想有人會跟你講這些說法.
我想多些人寫張卡來跟你講這些.
但不用做了.
我教你什麼.
我只是說說而已.
我上來的時候.
我看著.
我知道我放在辦公桌的中間的時候.
我明白了.

$^{561}$好像最近有些議題.
我講完這個就夠了.
什麼愛和聆聽和尊重.
我不否認在這個世界裡.
很多愛是亂愛的.
我不否認這個世界裡.
很多愛都是很隨便的.
有些人說愛其實是沒有愛的.
我明白理解的.
我完全知道.
但我更加想表達的是什麼.
如果你真的嘗試愛一個人.
嘗試用心去愛一個人的時候.
你什麼時候會從他願不願意聆聽.
和他願不願意被尊重的角度.
去表達你的愛呢?.
我不是說愛一表達的時候.
你一定要受.
一定要什麼.
我想表達的是.
愛和尊重.
其實雙方永遠都受傷的.
打從我們帶小朋友.
你很愛他的小朋友.
你想他做一些事.
你不想他做一些事.
他一定會跟你說.
你不尊重他.
他受傷了.
你也受傷了.
我只怕在我們這個年代裡.
一個快速的社會裡的時候.
我們一窩蜂地.
很反應的.
看著一些事 看著一些事.
抽一些水 說一些話.
如果我們沒有去到問題核心裡.
耶穌基督釘十字架死問過你嗎?.
我不是說他這樣做可以.
我們也可以.

$^{601}$我不是說我們所有愛.
都不需要理會別人的感受.
我不是的.
你不要給我意見.
但我想說.
真正你愛一個人的時候.
你會有什麼感覺呢?.
當你愛一個人的時候.
當你被一個人愛的時候.
這永遠是一個受傷的歷程.
你知道這個險一定是沒有下去的.
在愛與被愛裡.
是否充滿著很多對不起.
充滿著很多不懂得做.
很多不知表達的嗎?.
待會兒吃飯.
聖餐在說什麼?.
聖餐想和你和我說.
這份愛.
是可以很多人不接受的.
聖餐的設立.
去紀念耶穌基督.
是因為這個世界上.
紀念他的人很少.
面對著撒加利亞書這一段說法.
是代表著.
我們在這個世代裡.
有些事情要堅持.
堅持不用很快,很即時的東西.
去表達上帝的話語.
給自己更多的耐性,思考.
作為一個傳道者,一個講道者.
是面對著很多誘惑.
面對著很多人.
你不知道為什麼他的專頁會有這麼多讚好.
你不會明白他的貼文.
會有這麼多人給他心.
他在說什麼?.
你面對著很多這些掙扎和誘惑的時候.
你很想寫些東西出來.

$^{641}$或者說些東西出來.
讓更多人喜歡和理解.
但我只是期盼自己.
永遠都有一兩句話提醒自己.
那句話對我是什麼意義?.
那句話對我來說是在說什麼?.
更重要的是.
我不是因為要反應.
而宣講上帝自己的話語.
是他的話語才會對我說.
他怎麼說,我怎麼做.
這也是我一生情願的期盼.
希望一生人都可以這樣走下去.
雖然各樣各樣的誘惑仍然存在.
但我心想弟兄姊妹.
你和我都是這樣經歷和面對.
失去了至親的人的另一半.
當日.
那個七八歲的小朋友.
跟我一起回家.
我和他默書.
一起做功課.
讓我的家人可以自己一個.
可以好好休息.
好好愛上.
很好笑.
他說.
我們在地鐵站玩遊戲.
我說玩龍蝦.
不吃,你吃.
不知道你會不會玩.
玩完龍蝦之後.
他就問.
玩完了,有什麼玩?.
我說在地鐵上還有什麼好玩?.
他說,你港島吧.
我沒理由在車廂裡港島.
不可能.
你知道我做了什麼事嗎?.
我播了上個月.

$^{681}$播出了我的港島給他聽.
你猜猜.
我播了上個月.
他也猜不到.
二十多分鐘的時間裡.
他專心聽.
他帶著電梯.
我帶著電梯聽.
我聽回自己說話.
上個星期.
不是,上個月我說.
我們要經歷拆毀和拔出.
人生不斷經歷很多拆毀和拔出.
所以我決意.
不說什麼上帝有很多恩典.
在眼淚裡.
或者想說.
上帝做了什麼歧視神蹟.
要淡化那些拔出和哀傷.
比起拔出的時候.
就專心給上帝拔出.
好經歷內心.
各樣的心目歷程.
在這個世代要說清楚一件事.
讓人理解明白.
從不容易.
心願.
Fold Church.
或者聽Fold Church的港島的弟妹們.
你們不是在聽好笑的事.
開心的事離開.
我沒有仔細說.
關於撒加利亞書的掙扎和挑戰.
你知道堅持一件事一百年.
殊不容易.
心願我們在香港這個年代裡.
都可能經歷很多年的哀傷和難過.
讓我們繼續堅持.
我們一起聽土地公.
天父多謝你.

$^{721}$讓我們今天有這個時間.
我求天父的是.
你幫我們堅持我們心靈裡.
你所給我們所堅持的事.
雖然很多時候會遇到挑戰難處.
沮喪失望.
但值得堅持的事.
就是因為正正在最艱難的裡邊.
我們可以堅持.
在最艱難的裡邊.
才顯得.
矜貴.
求天父祝福Fold Church的弟兄姊妹.
我們一起彼此同心學習.
同心經歷.
讓上帝在這個世代裡邊.
做他要做的奇妙的事情.
多謝天父你聽我們前面的祈禱.
奉耶穌保貴命求.
\newpage



\section{撒母耳記下 11:1-12:31-20230225}
\label{sec:lsdGk_BkHa8}
\textbf{【流堂崇拜】天光請開眼|撒母耳記下11\_1-12\_31|20230225 [lsdGk-BkHa8]}
\newline
\newline
連結: \href{https://youtube.com/watch?v=lsdGk-BkHa8}{\texttt{ https://youtube.com/watch?v=lsdGk-BkHa8}} ~~~~ 語音日期: 2023-02-25 
\newline
\newline
\hyperref[sec:4Dll86a7b18]{\small{< < < PREV SERMON < < <}}
~
\hyperref[sec:index_chronic]{\small{[返順時目]}}
~
\hyperref[sec:index_scriptual]{\small{[返順卷目]}}
~
\hyperref[sec:VfT5ldcLjqQ]{\small{> > > NEXT SERMON > > >}}
\newline
\newline
$^{1}$(麥美娟: 請問你對於「同工」的看法是怎樣的?).
剛才我自己也很投入.
因為勁霸隊實在太肉緊了.
我自己也是一個經常很肉緊的人.
所以我的同工經常都很喜歡看我說話.
說話很肉緊 做事很肉緊.
我玩遊戲也很肉緊.
有些同工在搖頭.
我看到這個講題.
我想起一次跟牧者玩過一個類似狼人殺的桌遊.
有沒有人是未玩過的?.
狼人殺.
未玩過的 旁邊的人教一下他.
又未玩過.
一會兒找人教一下他.
類似是那些.
分好人和壞人的陣型.
有些人做壞人 有些人做好人.
在最緊張的時候.
那一次最緊張的時候有個警示出現了.
這個警示是從一位牧者的錶而來的.
他的錶響起.
就說「現在是九十分貝了」.
因為現場實在大家太肉緊了.
大家不斷在叫「是九十分貝」.
這九十分貝都主要來自他旁邊的那位牧者.
那位牧者就很肉緊地叫.
「他?一定是壞人啊」.
「你剛才看見他把尾巴挾起來嗎?」.
「喂!我們才是一隊的」.
「一隊啊」.
「你現在想揭穿我?不是吧?」.
「我剛才做過甚麼 你不是看不見我的吧?」.
「我是好人來的」.
「你相信我嗎?」.
「相信啊 有些人相信」.
但可惜那位叫到九十分貝的牧者.
也有人不相信他.
其實是…不要揭穿他了.
因為無論現實或遊戲.

$^{41}$很多時候都真假難辨.
今天我就帶大家參與一場宮廷版的狼人殺.
待會我們閉上眼.
不用怕的 閉上眼.
不用怕的.
閉上眼一起感受一下光明和黑暗的角力.
魔法.
上去吧.
請不要看著.
我們一起閉上眼.
天黑請閉眼.
這一刻我們來到一個宮廷.
天黑了.
有一隻獵物就在狼人的眼前.
狼人很想殺牠.
但很奇怪.
這個狼人對獵物的態度非常友善.
非常關心.
當獵物退下.
狼人就開始行動.
他寫了一封信.
他寫這封信交給他的下屬.
他指示他這樣做.
「你和我派他去戰場裡面最危險的地方」.
「然後你們就撤退」.
「讓他被敵人自然地殺死」.
狼人的下屬收到這封指令.
他心想.
「大家都撤退」.
「只剩下他一個」.
「這不就很明顯了嗎」.
「所有人都知道他們是狼人」.
於是他就將獵物和周圍的士兵.
都派到最危險的戰場上.
天光請開眼.
大家可以打開眼.
這次死的是烏莉亞和幾個士兵.
我們一起讀《薩姆二記》下十一章十六至十七節.
有交叉的部分可以暫時跳過.
好 預備開始.

$^{81}$「約阿偵測城的時候」.
「知道敵人那裡有勇士」.
「就派烏莉亞到那地方」.
「城裡的人出來和約阿打仗」.
「僕人中有幾個士兵被殺」.
「嚇人烏莉亞也死了」.
大家都很熟悉狼人殺.
知不知道致命傷是甚麼.
通常死是怎樣死.
(死是死).
是 有些人聽到了.
「信錯隊友」.
「你以為隔壁那個人幫你」.
「他可能真的幫你」.
「但可能是隔壁那個人在害你」.
「認錯隊友的烏莉亞」.
「她望著曾經和他出生入死的隊友」.
「誰知到最後」.
「是派她去戰場最危險的地方」.
「想她死的那個」.
「但我們好心知肚明」.
「背後還有指使她的那位主腦」.
「誰交信給約翰」.
「誰就是主腦」.
我們再一起讀.
《撒謬義記》下十一章14至15節.
「早晨,大衛寫信給約翰」.
「交烏莉亞親手帶去」.
「他在信裡寫著說」.
「要派烏莉亞到戰爭激烈的前線去」.
「然後你們撤退離開她」.
「使她被擊殺而死」.
原來主腦就是當時已經成為了軍王.
以色列軍王的大衛.
還要很特別.
你看不看到.
那封信交給了誰.
原來這封死亡密令.
是先交給烏莉亞.
她從來沒有想過她手裡握著的.

$^{121}$是一封殺害自己的死亡密令.
不過就算她揭破又如何.
可不可以改變呢.
要殺她的是以色列的掛事人.
烏莉亞暗地裡被殺害.
其實都是源於一件暗地裡發生的事.
請聽我讀出《撒謬義記》下十一章.
原來之前發生了一件這樣的事.
在有一天黃昏的時候.
大衛就起床.
在皇宮上的平頂散步.
他在平頂上看見一個婦人沐浴.
這個婦人非常漂亮.
大衛就派人打聽她是誰.
有人就說認識的.
她是米爾年的女兒.
赫人烏莉亞的妻子拔士巴.
大衛就派使者將婦人接回來.
來到大衛那裡.
這個婦人月經剛剛結症.
大衛跟她同寢.
他就回家.
那個婦人之後懷孕了.
派人告訴大衛.
我懷孕了 有了嬰兒.
雖然這件事我們看到.
黃昏的時候即是入黑.
在晚上的時候.
將軍的老婆被人召入宮.
都是大件事.
在經文沒有提及過.
她被人包著頭 蒙著頭.
很鬼鬼祟祟的.
沒有掩飾.
但是相信除了大衛的婦人.
都有其他人看見.
為什麼烏莉亞被人搶了老婆.
都懵然不知.
都不走出來說一句話.
這是因為當時烏莉亞在哪裡.

$^{161}$她被人派去打仗.
就是被大衛派去前線打仗.
而一向和以色列人出入沙場的大衛.
這次沒有去到 一起征戰.
大衛遠離了刀光劍影的戰場.
但是他自己留在宮裡.
都上演了一個內心的小劇場.
一個天使與魔鬼的小劇場.
去計算著自己如何保住光明的形象.
遮掩了他犯奸淫的罪行.
為求目的達到.
大衛就急召了烏莉亞回來.
急召她回來之後.
他還用了一個招數.
發動了一個招數.
就是灌醉她.
灌醉她.
讓她以酒亂性.
她就會回家與老婆親近.
那肚子就沒有人知道是誰.
不過烏莉亞有很多理由.
其中一個理由就是.
大家都在打仗.
她如何可以回去與老婆親近.
堅決回去.
這就令大衛想下一個計謀.
就是借刀殺人.
我們拉遠一點 看整個局勢.
剛才說大衛是狼人的陣營.
他的對手是誰.
是否在戰場上的那一班人.
大衛曾經強調過.
以色列軍是永生上主的軍隊.
現在他將自己軍隊的勇士除去.
其實就是轉投了黑暗的陣營.
整個局勢是怎樣的.
他現在與上主對抗.
站在上主的對面.
形勢有這麼大的改變.
我們很多時候看這段經文都會說.

$^{201}$一定是色字頭上一把刀.
在座的男士有沒有人用這段經文來引誡你們.
小心一點 可能漢奸就出事了.
不要看那麼多東西.
這個又不能全錯.
很複雜 你們問那些藍屋姐.
這個色字頭上一把刀.
是不是就是這段經文的總結呢.
這把刀確實是.
確實是變節的開始.
但是令到他陰謀越發越大.
背後有更重要的原因.
一個關鍵的因素.
這個原因就是一個字.
「派」字.
這個「派」是猜拳.
是與一個人連繫了.
可能我派你去幫我做一些事.
你的上司會派你去出差等等.
在整段經文裡面.
大衛不斷做的就是猜牌.
大衛猜牌約壓.
派約壓率領神僕和以色列的士兵去打仗.
去對前線打仗.
他派人打聽婦人是誰.
派人去接那個婦人回來.
之後他的計謀想開始的時候.
他又派人叫約壓命令布里亞回來.
派布里亞去送死.
派人將拔士巴接入宮裡面.
一連串猜牌的行動.
大衛不斷行使他君王的權力.
猜牌不同的人達成自己的陰謀.
有權力的人才可以猜牌.
而大衛做每一個惡行之前.
都是先猜牌.
他為了掩飾而猜牌.
為了姦淫而猜牌.
他為了殺人而猜牌.
他的惡行是因為他濫用猜牌這個權力.

$^{241}$並且忘記當初猜牌他管理國家的是誰.
當初猜牌他管理以色列國的是上主.
有一次我和一位街坊叔叔.
看著街上聊天.
聊開一些社會議題.
他很教我.
來看看這裡誰擺什麼勢力.
講到最後講很多社會議題.
最後他就咬牙切齒地說.
權力越大貪污一定越大.
阿妹你知不知道.
我就咬牙切齒地說.
權力死人腐敗我們常常聽.
我們基督徒是否迴避了它.
我不要有權就行了.
我形容很簡單.
不要那麼多權力就行了.
如果是這樣基督徒就很難搞了.
因為你不能升職不能做老闆不能請工人.
因為你都有機會派人去工作.
生孩子都很大試探.
今天有小朋友在.
一不小心有些父母可能會用權力操控子女.
做他想做的事走他想走的路.
其實從聖經時代直到今天.
權力都存在在不同的關係裡面.
就算是剛才我們看不斷猜派人的是大衛.
我們看經文其實約翰和拔士巴都曾經有猜派.
曾經都運用他們的力量權力去猜派.
就算當時體為很弱勢的拔士巴.
他都派人去通知大衛.
而在生活裡面我們有時會收到不同的角色牌.
我們在不同的場合就是不同角色.
我們都要思考怎樣去運用權力.
先合乎上主的心意.
我自己曾經收過一個角色牌.
就是什麼呢.
很厲害的.
可以影響人的成長.
在還沒讀神學之前.

$^{281}$我是一位幼稚園老師.
其中有一位相熟的家長和他聚會的時候.
他經常都說當年一件事.
就是當年我面試他的兒子的時候那件事.
他就跟我說.
黃老師你知不知道.
那時候我多麼徬徨.
剛剛搬屋來到這一區.
到處去找學校.
誰知道每間學校都說我的兒子.
話又不懂說.
多句.
很多東西都不懂.
英文又不擅長.
很難追到進度的.
每個人都說回家等通知.
沒有一間學校肯收他.
我來見你.
已經打好書數了.
我自己都說我的兒子很多東西都不懂.
誰知道你跟我說.
你說慢慢教吧.
教小朋友是老師的責任.
你放心吧.
我每次聽完他說.
我都被當時青春的自己所感動.
原來我年少這麼有抱負.
感動啊.
你都感動了.
因為上主當時給我有些微的影響力.
我可以決定收不收他.
當時我就收了一位.
被很多學校放棄.
很多準則裡面都不會選他的小朋友.
但我們知道上主的眼睛不是這樣看.
信徒要思考的是我們怎樣.
有時限制自己的權力.
以免做了上帝恨惡的事.
有時我們又要好好運用我們的權力.
可以去做上主喜悅的事.

$^{321}$特別是我們上教會的人.
很特別的.
教會這個場景.
這裡有些不同.
可能我們以前上教會.
很多權力.
你做組長有權力.
有導師的權力.
招待都有權力.
我們怎樣運用權力.
都是上主看重的事.
有一次我就帶一位無家者.
去到某教會參與主日崇拜.
很好.
期了一段時間肚子餓.
他又肯跟我去.
我們約了就去了.
怎料那一次就吃閉門羹.
這個閉門羹嚴格來說.
只是給那位無家者吃.
他沒有拒絕我進去.
理由他們給我的.
就是我們這個教會沒有這個事工.
那裡崇拜不適合他.
甚至乎他好像有些味道.
可能會影響到會友.
試想像如果當日站在門口的你.
你有權接待或者不接待他.
你又會怎樣回應呢.
你的回應可能只有你知道.
但是大衛他面對事情的反應.
大家都看到.
宮裡的人看到.
他的親信都看到.
其實都是有些重要的蛛絲馬跡.
被人看穿了他的底牌.
其實在過往他有幾次運用權力的情況下.
我們做一些比對.
就發現他整個人都變了.
他的態度.

$^{361}$他的反應有很大很大的變化.
好像換了一個人.
我們一起整理一下.
其中一件事.
上次港獨都有分享過.
大衛對著米菲波切的時候.
米菲波切是腳都震了.
他不知道大衛怎樣對待他.
但是大衛第一句跟他說的是什麼.
你不用怕.
一個君王跟他說你不用怕.
之後還派人照顧他.
常常跟他吃飯.
非常之關顧.
第二個事件我們可以看到.
昔日的大衛是怎樣呢.
這件事發生是因為大衛差派一些神僕去慰問別國.
但是人家不領情.
不領情之下他們做了什麼呢.
他們就將神僕的鬍鬚剃了一半.
又割斷他們下面的袍子.
令他們露出下體.
然後就放他們走.
當這班神僕被人用一個好像收獄戰犯的方式對待的時候.
大衛即時的反應是派人迎接他們.
派人迎接這班受獄的士兵.
讓他們有安身之所.
還可以讓他們慢慢來.
等你們的鬍鬚長了.
有體面.
你才回來面對大家.
這兩件事大衛都有行使他軍權的力量去幫助弱勢的人.
而從他之後的行動和反應.
更加看到他怎樣去看待人.
他對待米菲波切的時候.
當日的他.
有上主同在的他.
細心安慰.
願意去照顧.
不分你我.

$^{401}$不會覺得自己是君王.
就不會同桌吃飯.
當他對著那班受獄的神僕.
有上主同在的他.
體恤照顧他們.
維護他們.
尊嚴.
非常之體貼.
但是當他轉頭了黑暗的陣營的他.
對烏尼亞和那班士兵.
他說了什麼.
他說了一句.
刀劍有時會吞滅這個人.
有時會吞滅那個人.
即是說白一點.
打得仗有時會死這個.
有時會死那個.
他竟然對人命無動於衷.
還是他熟悉的戰士.
假仁假義的他.
還叫人不要那麼傷心.
你認為他身邊的人.
會不會察覺到他這個變化呢.
權力不單止令到大衛對罪埋沒.
對上主視而不見.
更加令他看東西不同了.
看什麼不同了.
他看人不同了.
他覺得自己高人一等.
自己的命比別人優越.
他輕視其他的生命.
這種看人的角度.
其實都出現在我們的社區.
有時我和清潔工聊天的時候.
都會聽到他們很多在職場被欺壓的狀況.
在他們的角度.
他們說沒辦法.
上級有權.
甚至一個懂投訴的路人都有權.
他們經常被路人的權力所影響.

$^{441}$有時就算是因為別人攬權.
而令到他受到無理的對待或要求.
他們都沒能力或不懂得怎樣拒絕.
甚或乎是因為害怕.
要承受一些後果.
所以他都不可以為自己發聲.
有個清潔伯伯就說.
我們都是無權無勢的.
他們被人輕看.
有時在他們的言談之間.
他們覺得是必然.
有一次新年那段時間.
我帶兒子和二公去探望一位清潔伯伯.
我們在公廁門外聊天.
聊得很開心.
聊天的時候.
有位工友都認識.
走過來想捉住兒子的手.
玩玩.
那位伯伯就很緊張.
嚇住他.
喂!不要搞.
不行的.
接著我就說.
捉住兒子的手.
開心玩玩.
他就說.
不好.
我們這些人.
不方便的.
我們這些人.
我聽完之後.
非常不開心.
原來有時社會將他們輕看.
自己習慣被輕看.
自己都會輕看自己.
當然我們和他們的接觸.
就是告訴他們.
什麼是正常的關係.
沒有誰比誰更高尚.

$^{481}$輕看人命的大衛.
在這一場宮廷版的狼人殺入面.
其實沿路都有很多很多指示給他看到.
其實很多指示都可以告訴他.
你可以由黑暗那邊走出來.
走回上主的陣營裡面.
我們看看那些指示是什麼.
第一個指示就是約季.
當烏尼亞拒絕大衛的時候.
他劈頭就說起約季.
約季是代表神的同在.
是神和以色列人納約的象徵.
大衛本身都非常重視約季.
非常重視神與人同在.
這番話是出自一個外邦人的口.
一個歸順了以色列國的外邦人的口.
其實這個指示已經足夠.
讓大衛感到羞愧和醒覺.
第二個指示就是烏尼亞的名字.
雖然他是外邦人.
但他有希伯來的名字.
他的名字的意思是上主是光.
所以大衛每一次與烏尼亞的對話.
看到烏尼亞.
他應該都會想起上主就是光.
給了機會和時間.
但大衛都無視.
不過上主一直都看到.
而且他看到大衛是以色列人的領袖.
如果他轉投了黑暗的陣營.
後面那班以色列民會怎樣.
可能全部都會一起跟過去.
所以上主為了挽回他和那班子民.
他就出了一張牌.
這張牌是甚麼牌呢.
很厲害.
這張牌叫做牌先知.
他就派了先知出去.
他的功能是甚麼.
先知的功能就是揭開大衛藏在籠底的惡行.

$^{521}$令他見光死.
不能再扮君子.
先知更宣佈.
上主會在日光之下宣佈報應大衛.
讓他承受惡果.
認清主權在哪裡.
主權在上主那裡.
上主這一回合很成功.
當大衛的底牌被揭開的時候.
他都立刻醒一醒.
轉投回上主的陣營.
願意悔改離開黑暗.
今日我們生命的主權是屬於誰呢.
我們生命的主權是屬於耶穌.
是屬於誰呢.
屬於耶穌的門徒.
主的光一直會指引我們.
我們一起讀約翰福音三章十九至二十一節.
預備開始.
光來到世上.
世人因自己的行為示惡的.
不愛光.
倒愛黑暗.
這就定了他們的罪.
凡作惡的人都恨護光.
不來接近光.
恐怕他的行為被暴露.
但實行真理的人就來接近光.
為要顯明他的行為是靠神而行的.
耶穌進入黑暗的世界.
是要揭示黑暗陣營的底牌.
為那些被封鎖在黑暗裡面的人.
提供逃生的出路.
提供力量去面對黑暗.
我們接受光就可以離開黑暗.
最近耶穌的光都照著我.
去年有一個很多年沒見的幼教同學.
看到我兒子的照片就說.
探望你兒子吧.
很可愛.

$^{561}$但又有防疫措施.
總之約都約不成.
有一天剛才的同學A.
有一個同學B就說.
今天見到同學A.
他見到他發生的事之後.
就令我很想快點約同學A.
他說什麼呢.
他說今天收到同學A的卡片.
原來他做了校長.
我又看一看那間學校.
我們很多同學都做校長.
那間學校是我心儀兒子想讀的幼稚園.
同一個系列的.
當刻我自然就這樣想.
這次可以了.
找他去踏路.
一定容易很多.
其實真的.
很現實.
校長說一句不收你嗎.
沒理由吧.
可以有特權或者加分.
之後就無限FF.
我的兒子前途一片光明.
想得很遠.
在我想約他的時候.
我就想起當年我怎樣面試人.
當年那個滿腔熱誠.
剛才仍然在感動我的自己.
我可以運用權力去幫助弱勢的學生.
但今天當有機會來到的時候.
我竟然想借助別人的權力.
讓其他應該公平競爭的學生成為弱勢.
真的很真實.
可能我們每一天都面對一些生命的抉擇.
有時權力未必在我們手中.
但我們都受到引誘.
想透過穿針引線.
透過別人的權力得到好處.

$^{601}$或者遮蓋我們一些惡行.
當我們去擔當不同角色的時候.
我們有不同的看見.
我們可能擁有不同的權力.
又或者我們今天被別人的權力牽扯著.
令我們動彈不得.
但很願意上主的光成為我們生命唯一的指引.
成為我們的力量.
又願我們Flow Church.
我們的弟兄姊妹.
我們看見弱勢的時候.
可能弱勢就存在在你身邊.
有時可能我們看不見.
或者選擇看不見.
當他們受到不公平對待的時候.
我們能夠提供實際的幫助.
並且指引他們.
見到上主的光.
一會兒我都會有一些時間.
邀請大家祈禱.
你為自己祈禱.
你為身邊見到的弱勢群體去祈禱.
你為需要迴轉的人去祈禱.
我們一起去禱告.
耶穌你就是真光.
能夠認識到你.
能夠接受你.
我們的生命就可以離開黑暗.
這個世界有太多的誘惑.
罪惡充滿在我們周圍的時候.
每一刻我們都需要去選擇.
我們選擇就近光.
接近光.
還是離開上主的指引.
求上主去指引我們.
這一刻.
你都圍到我們的城市去禱告.
會不會有一些群體.
是你特別有感動.
祈求主讓他們在.

$^{641}$可能沒有那麼多權力.
真的很客觀的限制.
令到他們成為弱勢的時候.
他們怎樣可以看見上主的光.
他們怎樣可以得到實際的幫助.
你為一個有感動的群體去祈禱.
我們又為自己去禱告.
今天你的角色是什麼.
你的狀態是什麼.
你已經認識了耶穌一段時間.
你可能離開了祂一段時間.
又或者你還不是很認識祂.
還沒有相信祂 還沒有接受祂.
無論你在什麼階段.
你都可以禱告.
求上主讓你看到光.
讓你看到現在身處的黑暗.
讓你看到那些指示.
祈求上主讓你有力量.
可以走出來跟隨祂.
或許有一些自己不為人知的話.
說不出任何人聽.
很需要去認罪.
這一刻你都跟上主說.
我們為自己的生命去祈禱.
親愛的天父.
我們懇切祈求你.
幫助我們在黑暗當中.
我們更多看到上主給我們的指示.
更多看到你的光去引領我們.
幫助我們.
更多讓我們經歷出黑暗入光明.
我們很多的試探.
很多的試煉.
求主都給我們力量.
讓我們看見.
上主你想我們看見的.
愛你所愛 恨你所恨.
求主憐憫我們.
我們的祈禱交託.

$^{681}$是奉主耶穌的聖名致以求.
Amen.
\newpage



\section{}
\label{sec:VfT5ldcLjqQ}
\textbf{《致餘民及流散者:給香港基督徒的神學八課》第二季第1課|20230227 [VfT5ldcLjqQ]}
\newline
\newline
連結: \href{https://youtube.com/watch?v=VfT5ldcLjqQ}{\texttt{ https://youtube.com/watch?v=VfT5ldcLjqQ}} ~~~~ 語音日期: 2023-02-27 
\newline
\newline
\hyperref[sec:lsdGk_BkHa8]{\small{< < < PREV SERMON < < <}}
~
\hyperref[sec:index_chronic]{\small{[返順時目]}}
~
\hyperref[sec:index_scriptual]{\small{[返順卷目]}}
~
\hyperref[sec:dLJdySFiu9c]{\small{> > > NEXT SERMON > > >}}
\newline
\newline
$^{1}$(拍攝中).
(廣播聲).
(攝影機搭載音樂).
我們是時代之子.
是被上帝揀選.
在這個年代見證耶穌的基督徒.
我們身於這個年代.
被這個年代分散.
有人流散到海外尋覓理想.
有人繼續在本土奮鬥下去.
但又如何?.
基督徒仍然需要作基督徒.
香港人仍然是香港人.
究竟上帝的旨意如何?.
我們應該如何生活?.
我們應該如何為主而活?.
無論你身處何處.
只要你是香港人.
邀請你和我們一起思想這個流散年代的信仰.
致愚民與流散者.
給香港基督徒的神學百科.
(音樂結束).
各位弟兄姊妹晚安.
很高興今天和大家一起上香港基督徒的神學百科第二季.
我小時候看《四十名穿梭機》.
有一首歌叫做.
我也能做到.
所以今天.
這一季我會做地勤.
第二季我們會延續第一季.
我們課程系列是全聖教的基因.
希望能塑造流淌群體的屬靈信仰.
最基本的原則.
第一季是基本.
第二季我們會延續到一些處境.
特別是在我們這幾年流散的情況.
所以第二季我們會在流散的情況中.
一起思考我們的信仰.
今季我們也很特別.
每一季的應用都會分開兩邊.

$^{41}$一個是愚民.
香港的基督徒如何面對我們的課題.
如何能夠在香港繼續盡心信仰.
另一方面是我們在海外流散的弟兄姊妹.
所以今天我們每一課.
同一課題都會同樣地.
讓兩邊弟兄姊妹去思考信仰.
所以一個同閉.
兩邊的應用.
第一課我們叫做.
從此我們分離愚民與流散者.
我想這是一個引言.
去討論一下流散的課題.
流散的課題.
可能大家都面對著處境.
如果我們回看我們的字.
可能大家都見過的字.
Diaspora.
這個字是解作流散.
或是外地移居移鄉的人.
這個字在學術上有很多層面應用.
無論是人類學,社會學,歷史研究.
信仰都是一樣.
所以Diaspora這個字是指本地外國人的情況.
無論是香港的南亞裔.
或是美國的華僑等等.
都是一個很重要的研究.
所以當我們回看聖經的時候.
Diaspora這個字是基督教神學中.
一個很重要的字眼.
實際上以色列文的歷史.
從來都是流散歷史.
所以當我們研究Diaspora這個字眼.
不單單是我們香港人的情況.
我們香港人的民族.
我們去不同的地方.
飄流到海外.
但這個字本身是我們信仰中一個很重要的課題.
所以我們就一起來看看.
我們一班香港人.

$^{81}$我們去思考我們的處境.
同時也去認識聖經中.
我們有關流散的信仰.
這就是我們今天的目的.
要去研究這個字.
我們發現它本身可以解作兩個字眼.
這個字是一個來自希臘文的字眼.
叫Diaspora.
但如果你看舊約的時候.
舊約的希臘文的本.
就是74頁本.
正正就是用Diaspora這個字去翻譯那個秘魯的字.
如果你明白那個基礎.
74頁本來就是一班流散了猶太人的年代寫成的.
當時猶太人已經分散到不同地方.
主前大國是幾百年時間.
他們重新用希臘文.
重新去認識和詮釋他們古道的信仰.
所以想一想.
一班用英文的人.
去明白香港一樣.
所以是一個第二代第三代的人.
他們用希臘文.
去翻譯希伯來文的字眼.
所以Diaspora本身是一定程度的流散處境.
以色列人是一班離開了他們的本族父家的地方.
在整個帝國裡面.
從而用希臘文去明白摩西五境.
或者以前的阿伯拉罕信仰.
所以這個字眼是直接用希伯來文裡面.
秘魯的Lut這個字眼.
所以這個字眼可以解作.
秘魯或者散居或流散的字眼.
其實可以有很多研究.
如果你去探討Diaspora這個字眼.
我們可以知道.
本身猶太人.
為什麼說Diaspora是一個很重要的信仰根基呢.
因為本身猶太人的歷史就是流散歷史.
無論是從阿伯拉罕開始.

$^{121}$阿伯拉罕要離開他的本族父家.
漂流到海外.
一個這樣的處境.
到後來秘魯的斯里文.
都是Diaspora的處境.
然後在耶穌的年代裡.
他們散居在整個的帝國裡面.
然後整個的中世紀.
就是中世紀歐洲.
他們都仍然散居在整個的不同的國內.
到二戰的時候都是這樣.
所以猶太人的歷史本身就是Diaspora的歷史.
一本書叫Encyclopedia of Jewish Diaspora.
是千頁厚的.
只是說猶太人的Diaspora的輸入.
都足夠成為一本百科全書.
所以是一個非常流散的.
猶太人是第一名.
第二名可能是印度和中國人.
所以我們看聖經裡的時候.
Diaspora本身是一個非常重要的舊學課題.
當然大家讀聖經的時候.
如果你上了教育課.
都大概知道一些基本的知識.
就是猶太人和以色列人.
他們是曾經被擄.
他們首先是在北國.
在公元前721年被亞述國擄去.
然後南國猶太.
他們是在巴比倫.
公元前586年來到去被擄.
所以被擄的歷史本身就是一個流散的事情.
記住Diaspora是解作.
被擄也可以這樣解釋.
當時他們這班人離開他們自己的地方.
有很多不同的原因.
首先就是被擄.
就是被人把他們抓去監獄.
也可以是一些生活上的原因.
有些人沒有被抓去.

$^{161}$純粹因為要工作.
就要去到不同地方居住.
這也是當時猶太人的情況.
有些人是發覺自己的地方.
經濟或城市被毀.
就要離開家園.
所以以前的被擄.
也可以是一些流散的成份.
他們不是純粹被人抓去這麼簡單.
可能也是因為不同的原因.
就離開了自己的家.
所以如果我們去留意一些舊約裡面.
希伯來文裡面.
一些被擄相關的字眼是很多的.
就是Diaspora是一個希臘文.
一個泛指很多時候流散被擄的希臘文.
但我們回看希伯來文的時候.
其實有很多不同的字眼.
都是說差不多的意思.
可能大家最熟悉的就是Galut.
我們有一個IG專頁叫Galut.
就是流散者們.
Galut這個字是解釋什麼呢.
原來這個字不是一個技術性的字眼.
這個字不是純粹技術性的.
是在說一個以色列人被擄的情況.
而是一個因著移動導致的視線改變.
這個很特別.
就是說一個因為被擄走了.
所以就看不到.
這樣很簡單的意思.
所以它可以解釋為一個位置上的轉移.
就是被人Galut這樣做.
就是被擄走了看不到.
所以它也解釋為一個消失或者失去.
總之你看不到他.
這個就是Galut的意思.
所以它最基本的意思不是一個純粹的秘魯.
不是巴比倫或者亞述國的秘魯.
而是一個很簡單的因為不在所以看不到的情況.

$^{201}$接著第二個字就是解作Shabah.
Shabah這個字是解作戰爭下的俘虜.
古境東裡面的戰爭是很流行的.
總之我打完一場仗的時候.
我就自然會俘虜對方的士兵.
所以這是一個很慣常的近東的做法.
但是當面對一些很大型的戰爭的時候.
一群人就被擄走了.
那就不是解作戰俘那麼簡單.
如果香港輸了.
不是那些士兵被人俘虜.
而是整個香港人都被人抓了的話.
那就叫做Deportation.
一個很大型的俘虜的意思.
所以Shabah這個字是解作一個這樣的意思.
就是在戰爭下的移居的人被人抓了.
所以是一種大規模的遷徙.
因為戰爭的緣故.
因為戰敗了.
所以就要離開家園被人抓了.
這也是和某些機密裡面和我們的自由有關係.
就是一個救贖或者賣贖的原因.
一個經濟上的賤額.
是這樣的意思.
所以另一個字就是這個.
就是Shabah這個字.
是一個戰爭下的俘虜的意思.
第三個就叫做Nada.
Nada這個字是講述物理上的移動.
所以看到很多的聲音裡面都有很簡單的字.
譬如拋.
Wing.
《九龍生命記》第十個章第五節裡面.
拋來拋去.
或者是Wing走了.
這樣都是一種物理上的移動.
Nada.
有一個譬如C篇第六二篇第五節.
就是拉下來.
都是一種動作上的移動.

$^{241}$另外就是有撞開的字.
所以原來當我們發現聖經裡面.
有Nada的字都是有時候被用來做一個流散.
或者是被擄相關的字眼.
意思就是人被人抓走了.
撞開了.
或者是拋來拋去.
所以這個就是第三個.
跟我們流散或者被擄相關的希伯來文.
然後就是分散.
Pasul這個字.
這個字其實是有一個水字在裡面.
其實就是散水.
所以當水散開了.
正正跟我們flow出一樣.
所以它的水是散開了.
無論是在一個真實上的水散開.
或者是在一種隱喻上都一樣.
例如泛藍都一樣.
洪水泛藍.
或者是水散開了.
都是一種跟流散有關係的字眼.
另外一個字眼就是戰爭蔓延.
戰火蔓延的蔓延.
正正是慢慢擴散開去.
都是這個意思.
所以你會發現獵王記上面.
有一個字眼叫入門四散.
就是用這個字眼.
就是這個Pasul這個字.
就是人們流散了.
另外當然有殺種的字.
就是散開了種子.
所以這個就是第四個.
有關流散裡面.
希伯來文裡面的字眼.
第五個很有趣的.
就是那個.
大色毒藥的字是什麼.
就是筲箕.

$^{281}$就是一個筲箕.
它就是一個筲箕的動作.
筲箕就是篩走東西.
篩走了那些人.
所以這個本身是堪披.
聖經裡面有堪披的字.
就散走了它.
最重要的經文就是這個.
神要埃及成為最荒涼.
最荒蕪.
最荒廢.
而且也要被擄.
散在列國.
那個原文裡面的散字.
其實就是這個筲箕的字.
筲箕的字.
想想伊斯利文.
是一個很生動的字眼.
上帝就用筲箕.
散開了.
篩走了伊斯利文.
所以這個就是第五個.
有關流散裡面的字眼.
非常生動的字眼.
筲箕.
動詞.
最後就是這個菩薩.
就是這個物件上的四散.
比如耶利米茲書所說.
伊斯利是打散的羊.
是被獅子趕出來的.
首先是亞述王將牠吞滅.
末後是巴比倫王來殺王.
將牠的骨頭截斷.
所以這個被人打散的羊.
就用了這個菩薩這個字.
來形容伊斯利文.
被亞述.
被巴比倫趕走的意思.
所以發現原來在舊約裡面.

$^{321}$不只是秘魯.
不只是Galut這個字.
也不只是Diaspora.
而是有很多不同的經文.
都是說相似的字眼.
就是筲箕.
分散.
或者是蔓延.
這些字眼其實都是嘗試.
來形容一個很重要的情況.
就是伊斯利文.
他們是分散了.
他們是住在不同的地方.
可能是因為秘魯的緣故.
可能是因為當時候.
秘魯的緣故相關的很多經濟活動.
迫著要在那裡做奴隸.
或者是要甘願離開自己的地方.
來生存.
這個就是當時候一些舊約.
我們看到的字眼.
為什麼要認識這麼多的英文呢.
我們發現原來.
所謂的秘魯.
或者流散.
我們問是不是一樣的字眼呢.
今天香港人流散到海外.
我們能不能夠用秘魯來形容自己的處境呢.
今天我們去將信仰.
去理解我們今天的處境的時候.
我們怎樣去理解自己呢.
我們是秘魯了的一個群族.
還是一個因為什麼原因而分散呢.
這個就是今天想和大家去談的課題.
如果我們去總結整本舊約的時候.
有一本書今天不拿出來說了.
是非常厲害的書.
非常之.
剛才那些字眼研究的結論.
就是這樣.

$^{361}$八百多頁的一本德文書.
他就去研究究竟.
Diespora這個字.
和Glut.
或者很多相關的字眼裡面.
究竟是一個什麼的理解.
舊約的人在經文裡面.
怎樣去理解這些處境.
作者就這樣有七個不同的階段.
或者是一個stage.
他將整本的舊約.
分成七個category.
來嘗試歸納出七個不同的狀況.
或者七個不同.
在秘魯流散的處境裡面.
他怎樣去理解自己.
甚至乎是一種沉迷.
有點像我們今天那些.
人遇見了.
哀傷的那五個階段那些.
大家看我們一起去吧.
第一個.
首先是秘魯作為一種上帝的審判.
這個大家都很熟悉的.
以色文秘魯或者流散.
因為他們做錯事.
他們因為被上帝懲罰.
因為他們是潑逆.
所以上帝就要他們秘魯.
這個是其中一個.
但不是唯一一個.
是一個很重要的category.
即是聖經裡面有很多經文.
都是嘗試來將秘魯流散.
看為一種上帝的審判.
因為他們做錯事.
所以他們就要流散.
所以就要被人去俘虜.
第二個.
秘魯作為一種無可避免的現實.

$^{401}$他們就發現這是一個.
不可以避免的情況.
無論是耶利米書的教導.
你肯定會流散.
肯定會被捕.
這是肯定會出現的事情.
即是發現原來他們開始要接受.
去接受這樣的處境.
這是一個上帝既定的事情.
你需要去面對.
才會預言.
大家必定會被俘虜.
這是第二個情況.
是一種第二個階段.
對於流散和秘魯的理解.
第三個.
甚至乎在科學經文裡面.
後來出現了一種盼望.
這本經文是盼望他們能夠歸回.
甚至乎是重視將要歸回的那種英雄.
都有經文的.
今天我們不展示經文.
因為太多了.
到了第三個階段.
這本經文開始會在秘魯裡面出現盼望.
知道上帝和華相會叫他們歸回.
我們能夠回到耶路撒冷裡.
安山等等的情況.
這是第三種層面.
開始不是純粹的審判.
而審判之後上帝會歸回.
會拯救.
仍然會幫助.
第四.
是一種流散者綜活的團聚.
這個叫綜活.
發覺很多先知書裡面的預言.
以色列文在綜活的時間.
能夠回到自己的家鄉.
能夠大家歸回上帝.

$^{441}$這個不單是一種能夠盼望的事.
更加是一種綜活的事實.
這就是第四種情況.
一種綜活性的團聚.
第五.
開始接受流散的處境.
最出名的經文是什麼.
就是那句.
你們為者城求平安.
你們開始要為自己在異鄉裡.
為那個城市求平安.
你們會住那裡很久.
所以你們要開始接受流散的處境.
買樓的買樓.
大家在那裡慢慢找工作.
慢慢來生活.
接受這個流散的處境.
這是第五種現實.
可以接受了.
第六.
甚至強調上帝在流散和秘魯中的同在.
上帝不單在耶路撒冷.
不僅僅在聖殿裡.
更加在秘魯的國度裡.
最明顯的就是以色列的書.
以色列的書在異鄉河裡.
看到上帝的榮耀.
是一個外邦的異象.
耶和聖帝在外邦的天空裡出現的異象.
所以強調上帝也會同在.
在秘魯當中.
最後.
更加強調流散和萬民救贖的關聯.
以色列成為了列邦萬國的祝福.
因為他們流散出去.
所以他們更加成為了上帝使用的工具.
來祝福萬國.
這是七個類別.
如果我們把整本舊約.
把所有的經文分類.

$^{481}$把剛才所說的字眼分類.
這樣就有七個不同種類的經文.
甚至是有一個層底.
最開頭的時間裡.
是純粹審判.
因為我們做錯事.
所以流散 秘魯.
慢慢地發覺 接受這個事實.
慢慢地有盼望 有應許.
慢慢地強調能夠綜合性的團聚.
然後開始要在那裡生活.
深刻強調上帝的同在.
最後 甚至覺得流散是一種祝福.
對其他人是祝福.
所以我們很籠統 很快地.
把整本舊約.
有關流散的資訊.
知道 所以發覺.
秘魯和流散.
不是絕對傳言是負面的.
不只是因為我們做錯了.
被人罰.
而是有很多很多不同的意思在當中.
這是第一個.
第二個就是我們看到的.
一個很重要的字眼.
叫做Shakina.
如果你在家的話.
可以一起讀.
Shakina 這個字眼.
這是一個很重要的猶太教的字眼.
中文叫做赦金納.
不過南部人讀Shakina算了.
Shakina 為什麼呢.
就是解作一個上帝的同住.
上帝住在那裡.
或者上帝休息在那裡.
所以上帝就是在世界上的同在.
我們覺得上帝同在不是很特別.
都是同在.

$^{521}$但其實我們今天將這個同住看得太藍.
因為上帝無處不在.
反正哪裡都在.
但是耶和華說上帝的同住.
是一個很不簡單的事情.
上帝住在哪裡.
上帝你可能聽過經文.
是在天庭裡.
或者在聖殿裡.
或者天上的天都不容上帝居住.
這些經文.
所以當以色列人強調.
超越耶和華上帝住在當中的時候.
其實是一個很不簡單的事情.
或者我們用聖靈來解釋.
聖靈的同在.
但是當以前的猶太人覺得.
耶和華的同住是一個很重要的字.
不是隨便來的同在.
上帝是否同住.
是一個不容易解釋的便宜的事情.
上帝在哪裡呢.
譬如經文有一個叫做.
在初學者的經文裡.
我要住在以色列人中間.
作他們的上帝.
他們必知道我是耶和華他們的神.
是將他們從埃及地領出來的.
我要住在他們中間.
我是耶和華他們的神.
這個就是上帝的應許.
當一群埃及人.
以色列人被人追趕的時候.
上帝是住在他們中間.
這個字其實是一個很重要的字眼.
因為上帝竟然跟著他們走路.
經過紅海再在曠野裡.
不斷跟著他們走.
所以這是一個很特別的字.
所以上帝住在這裡.

$^{561}$當然我們發覺早期裡.
我們用這個來梳理.
就是這個聖神巷.
當所羅門去見聖殿之後.
上帝就住在聖殿裡.
上帝在聖殿當中.
所羅門說.
耶和華神說我必住在幽暗之處.
我已經建造殿與在你的居所.
為你永遠的住處.
所以所羅門說這是你的殿.
這是你永遠的住處.
你就住在這裡.
所以問上帝在哪裡.
上帝就在聖殿裡.
這是一個很重要的Temple Theology.
強調上帝住在聖殿裡.
這個圖就是古代裡的理解.
上帝是藉著聖殿與世界去關聯.
當然你發覺後來所羅門也這樣說.
天上的天都不足以居所.
你住在天上的天.
你不是住在聖殿裡.
所以是很弔詭的.
沒錯聖殿是上帝的居所.
但後來也強調聖殿不足以成為他的居所.
這幅圖是這麼說的.
我們任何人都經過聖殿去找到上帝.
以色列人是怎樣.
以前被擄了.
仍然往回猶太地方去朝見上帝.
溫哥華是第一眼.
要往回香港的地方去祈禱.
所以發現原來這樣的聖神學.
上帝就住在聖殿裡.
我們不詳細說聖神學.
後面就更加精彩.
但當以色列人被擄後怎麼辦.
他們不是在猶太.
聖殿被毀了.

$^{601}$那怎麼辦呢.
上帝怎會住呢.
所以出現了這個Shakina.
雖然聖殿被毀.
但上帝的同在就轉移到一群人身上.
上帝不是住在神聖的殿裡.
而是住在以色列人的中間.
這是一個很重要的概念.
因為上帝不是跟著死物或聖物.
而是跟著人.
人流散到哪裡.
上帝就跟著去哪裡.
因為這樣的上帝.
所以我就住在他們的中間直到永遠.
上帝不再是天上的居住的居所.
而是在以色列人的中間的位置.
就是一個Shakina的概念.
所以在以色列的書中.
人子也說這是我寶座之地.
是我腳掌所踏之地.
我要住在這裡.
在以色列人的中間直到永遠.
上帝說我要住在你們當中.
直到永遠.
直到你們歸回回去的時候.
所以你會發現.
如果拉到很遠的時候.
這個是猶太教信仰.
當然斯列文猶太人.
他們在中世紀的時間裡.
仍然相信.
雖然他們不在猶太人居住.
在歐洲不同的地方.
但上帝就跟他們同在.
所以有這樣的流散的信仰.
所以發覺對猶太人來說.
聖殿神學是重要的.
不過更加重要的是流散神學.
因為他們在世上經歷了幾千年.
都是流散的歷史比較多.

$^{641}$他們住在外國.
但上帝就跟他們同在.
是這樣的信仰.
接著我們看到.
到了新約時間也是一樣.
原來耶穌的說話.
因為無論在哪裡.
有兩三個人奉安的名聚會.
我們就在他們中間.
這是耶穌所說的.
這個經文其實源自於.
猶太教的一句名言.
一句密西那格言.
有兩個人用托拉一起聚會.
我們就有席琴那在他們中間.
所以經文其實是來自猶太的席琴那概念.
耶穌就是那個席琴那.
兩個人一起聚會.
今天在英國,多倫多.
或者在不同地方.
只要一起聚會.
就有上帝的席琴那.
就有上帝的同住.
他就住在當中.
所以這個就是到了新約時間.
原來是借用了這樣的概念.
去創造席琴那的神學.
當然我們知道.
最終極的席琴那是什麼.
就是耶穌基督.
他就在我們中間.
這個住也是這個字.
都是同住.
上帝的兒子就住在世人當中.
有恩典有真理.
所以我們是對的.
我們強調耶穌.
強調聖靈.
我們就有上帝同住.
哪裡有聚會.

$^{681}$在希伯倫堂也好.
在紅磡也好.
這個就是我們上帝的同住.
所以你會發覺.
這個就是我們在教裡面.
很重要的概念.
從聖殿到流散.
到席琴那.
上帝的同住.
去新約.
新約裡面其實有很多的經文.
都是在說一些這樣的情況.
都仍然在用despair這個字眼.
新約裡面其實不是特別多.
despair這個字眼.
但是當他在說despair的時候.
其實有三種不同的情況.
第一是什麼呢.
就是當時候.
真的散居在羅馬帝國猶太人.
當時候猶太人已經亡國了.
就散在不同的羅馬帝國地方.
保羅就是其中一個.
保羅就是其中一個猶太人.
散居在羅馬地方的公民.
所以第一個意思.
很明顯就是散居了的意思.
一群散居了的猶太人.
第二就是流散基督徒.
因為被迫的緣故.
大家記不記得.
當時是反被迫害的時候.
然後因為被迫害.
就要散居在不同地方.
所以despair就是這樣的情況.
就是一群因為迫害的緣故.
而迫著要散居分散基督徒.
不過我們發現.
當我們看《釋經史》的時間.
despair不單單是一種處境性.

$^{721}$或者是地理上的意思.
更加是一種神學上的流散概念.
就是說流散不單單是說.
當時猶太人或者基督徒的處境.
更加是基督徒作為基督徒.
教做為教會應該有的質素.
所以我們就會說一下這一點.
為什麼會成為一個神學上的字眼.
despair.
第一個就是亞國書.
亞國書第一章這樣說.
作神和追溯基督僕人的雅各.
請散處十二支派人的安.
這個散居了的十二支派是什麼意思呢.
是解作真的散居了十二支派的人.
還是一種symbolic meaning.
亞國書說的是.
這群十二支派是不是真的literally.
十二支派.
還是一種比喻上猶太人的十二支派.
是一種symbolic meaning.
還是真正是十二支派的猶太人呢.
所以如果你覺得是後者的時候.
如果是一種純粹象徵意義的話.
這種散居其實不一定是那種.
真的散在不同地方.
而是一種信仰上很重要的意味在當中.
另外就是希伯來書.
更加強調.
亞伯拉罕在信夢照的時候就遵命出去.
往將來要得的基業地方去.
出去的時候還不知往哪裡去.
因為信他就住在應許之地作客.
好在在異地居住帳棚.
與那夢和夢一個應許的二十二個一樣.
他們就是在異地裡居住.
這些人都存信心死的.
並沒有得著所應許的.
卻從遠處按見.
且歡喜迎接.

$^{761}$有時仍在世上是客裡是寄居的.
所以這個很重要的身份就是說.
基督徒在地上的那種客裡神學.
我們在地上是客裡.
我們經常上一首歌.
我們是客裡我們是寄居的.
這個寄居不是真實上的地理意味.
而是一種信仰的意味.
我們就是留在地上的人.
無論你住在香港也好.
在外國也好 元朗也好.
都是一種客裡的身份.
都是一種散居的身份.
所以看到希伯來書所說的散居.
就不是真的分散在不同地方.
是一種基督徒的本質.
我們就是一種客裡的身份.
我們就是在地上的一個人.
我們的家鄉在哪裡.
我們的家鄉在天上 天家的概念.
所以我們學某一個近美的家鄉.
因為在天上.
如果我們的家鄉在天上的時候.
我們今天在地上就是什麼.
就是一個迪斯珀拉 就是這個意思.
所以發現新約是慢慢將迪斯珀拉.
變成一個神學概念.
沒錯他們是被迫不留散的.
或者是散居了.
但是更重要的是什麼.
他們真正的家鄉其實就在天上.
所以每個人在基督徒.
都是一個這樣流散的人.
最後更重要的經文就是這個.
就是這個彼得前書.
這個非常重要.
都是打一句.
一個有名的文案.
耶穌基督的使徒彼得.
寫信給那分散在本都.

$^{801}$加拉太 加帕多加 亞細亞.
被推來寄居的.
就是招父臣的先見被揀選.
我刻了三個字.
其實原文裡面是這三個publicable.
去形容這班人.
三個形容詞.
首先就是他們是分散了.
他們是異鄉客 一個寄居的人.
他們被揀選.
所以彼得就用了這三個形容詞.
去形容基督徒.
他們是被上帝揀選的.
一個異鄉客 一個分散了的人.
所以彼得前書更加將流散這個字眼.
不是一種純粹地理上的問題.
而是一種基督徒身份問題.
他們是上帝揀選的子民.
上帝是他們的constant.
雖然他們是分散.
但他們仍然是被上帝揀選的一班基督徒.
他們是一班異鄉的人.
他們的方式和生活習慣是跟其他人不同的.
這就是異鄉的意思.
所以今天在香港裡面也是一樣.
仍然是有一個異鄉的身份.
基督徒跟其他人是不同的.
這就是異鄉的意思.
第三就是散居.
dyspora 他們是分散的.
所以如果我們去理解這個經文的時候.
散居不是說你去加拿大或英國那麼簡單.
而每個基督徒都需要去領受這三個形詞.
他們是被上帝所分散的.
這是上帝的命令.
我們甘願去被分散.
不是因為客觀的元素.
不是因為某個政府的強權.
而是因為我們基督徒就會被分散.
我們被上帝揀選.

$^{841}$我們是一個異鄉的角色.
所以你會發覺.
從舊弱到新弱.
你會發現原來被擄或流散.
對我們來說不是純粹一個客觀的政治現實.
不是純粹因為某一年出現某條法例.
而是一個我們基督徒應該要有的質素.
所以你會發覺.
我會嘗試去講這一點.
這三個形詞.
這個我不講了.
就是擄散作為我們教會的使命.
原來擄散如果不是客觀的政治現實.
而是我們基督徒作為教會的重要的確具.
我們作為教會的觀念.
擄散是我們的教會使命.
更好的形詞.
我們稱之為「踩險」.
大使明這麼說.
你們要去.
我們小時候就說你們要去.
這個去本身就是一種什麼.
就是一種擄散.
大家要分散去不同地方.
我們實踐了大使明.
所以你會發覺整個的擄散.
可以成為我們很重要的觀念.
剛好我們叫flow church.
所以我們flow church本身就成為一個這樣的觀念.
flow church和擄散這幾年我們都用了這些keywords.
be water我們都用了這些keywords.
無論是be water也好.
flow也好.
擄散也好.
其實都是一個我們核心的價值.
所以我上次去說去形容.
怎樣去理解我們作為flow church.
怎樣理解擄散這件事情.
所以擄散是我們教會的使命.
是我們一件帶著使命去做的事情.

$^{881}$一種我們教會的本質.
我們理解教會的概念.
這樣去理解這個擄散.
第一就是擄散和教會的差遣.
這個差遣這個字.
其實正正就是一種叫做mission.
一種叫做使命的字眼.
其實我經常說mission day.
或者叫做missionary.
好像叫做宣教或者叫做使命.
其實mission本來就是差遣.
上帝自己來差遣一個人.
這是最基本的意思.
所以不單單是宣教或者某些使命.
而是上帝去差遣.
上帝差遣他的獨生兒子來到世界上.
這是第一層的差遣.
然後在永福特色七章裡面說什麼.
耶穌是差他的門徒去到世界上.
這是上帝耶穌基督差遣門徒的差遣.
然後我們教會會差遣宣教士和差遣人.
所以是這樣的理解.
所以教會本身正正就是一個這樣的群體.
我們教會有幾個很重要的步驟.
我們教會是會招聚.
Ecclesia就是教會的聚集.
留堂在四年前是這樣.
重新來招聚一群無教會基督徒.
我們重新來聚集.
然後我們重新來建立我們的群體成長等等.
但是第三就是差遣.
我們不單單是聚的.
也不單單是彼此有個團契或者模樣的.
而是我們有使命的.
上帝差我們去世界裡面.
所以是關乎我們教會和世界的關係.
我們怎樣和世界建立關係.
我們怎樣來見證上帝.
就是差遣的元素.
所以教會本身的分散是帶著這樣的差遣.

$^{921}$每個禮拜崇拜裡面.
我們都有一個這樣的差遣的意味.
我們今天聚完聽完道.
唱散會詩.
大家就走出去差遣.
帶著使命的去見證上帝.
所以流散不單單是給流散者.
如果你是在看YouTube的.
你在海外的時候.
當然這個流散是與你無關的.
因為本身你是一個流散的處境.
大家看我們在這裡像在座的都一樣.
今天我們在香港.
我們所謂的漁民.
就是留下來的人.
我們仍然帶著這樣的使命.
去做同樣的事.
所以我們作為流塘.
一個流散的教會.
一個流散群體.
這是我們每個人都應該思考的問題.
怎樣能夠在我們今天的生活裡.
在香港 在海外.
怎樣能夠來見證上帝呢.
這是我們第一個很重要要問的問題.
第二就是這個.
那時候教會是甚麼呢.
我就稱之為教會是任何形狀.
你聽過一次我這樣說過.
如果我們說流散是我們一個教會觀的時候.
一種帶著使命的教會觀的時候.
教會的形狀是甚麼呢.
教會就不再是一個建築物或一群人.
而是一個我稱之為liquid church的概念.
其實有一本書叫做liquid church.
就是將教會看成一種liquid.
可以滲透在不同的地方裡.
我們不是一件東西.
我們是流塘.
我覺得這些全都是keyword.

$^{961}$流塘正正是這樣的群體.
一個液化教會.
它是一個流散在世界不同角落裡.
滲透在世界裡的群體.
所以這是比無牆教會更加重要.
以前我們說甚麼無牆教會.
無牆教會只不過是一個無牆教會.
人們進來不需要牆就能進來.
但我們更加進一步.
我們不是無牆.
更加是liquid church.
我們的教會是沒有任何形狀的.
你無法定義流塘的群體是怎樣.
因為你會發覺.
剛剛做了數據.
有七成是香港人在流塘崇拜.
有三成是海外的.
所以發覺原來流塘的群體.
不是純粹那麼容易畫出來的.
它是一個liquid.
它可以任何形狀.
它可以任何的形態.
它可以是一個floam的石光群體.
一個坐在一起學習的群體.
所以這是我們很重要的流塘的概念.
既然我們是流散的時候.
我們自然而然很強調我們是流散出去.
但我們會有很多不同的方式去連結.
藉著我們崇拜去聚集.
但這不是永遠的事情.
我們是會分散出去.
成為一個很重要的使命.
在不同的角落滲透.
這也是關我們事的.
關原民事的.
當你在香港住的時候也是一樣.
我們作為基督徒.
在你的工作裡面.
在你的公司裡面.
作為一個這樣的水滴.

$^{1001}$一個這樣的liquid.
去應驗.
在教會延伸.
所以今天我們來到.
現在教會的時候特別流塘.
我們是一個比其他地方更不像舊的東西.
我們沒有舊的東西.
很強調就是大家的那種使命.
在不同的地方成為水滴.
來見證耶穌.
所以這是我們第二個很重要的形容.
無論你是在海外還是在香港.
我們成為教會的一部分.
我們未必一定有一個很concrete的形狀.
畫出來一個十字架.
一個三角形屋頂的教會.
但我們是一個liquid church.
一個很flexible.
很高流動性.
能夠滲透在世界裡面.
見證上帝的一個群體.
第三.
我想這是一個很重要的一點.
就是我們所說.
就是一個.
我們的流散.
今天我們分離.
我想大家都沒有人猜到.
當我們在流塘開始的時候.
沒有人猜到我們會離開香港.
我想說.
這個分離.
其實走和留在這裡都是很痛苦的.
我們被告別.
我們告別人.
這種分離令到我們流塘被分散.
有些人去了英國.
有些人去了加拿大.
有些人回到香港.
我們好像很傷.

$^{1041}$好像被人激散了.
好像被人篩走了.
但如果我們今天回頭看.
今天想說什麼.
這個流散.
這個despair.
絕對不是因為一個客觀因素.
而是我們作為基督徒.
一個很重要的使命.
我們正正是一個被揀選群體.
一個異鄉客群體.
一個流散群體.
這個不是一個客觀因素弄走我們.
而是我們知道上帝帶著命令去猜險我們.
上帝帶著命令.
要我們帶著使命.
這樣去面對我們的處境.
所以如果你是在海外.
頂姐妹.
不要覺得因為哪個國家令到你要走.
而是因為上帝早在你出生的時候.
祂知道今天你在海外.
我們記得孖孖順是揀選這群人.
順便強調一件事.
我們是被流散.
是異鄉.
但被揀選.
我們每個人的生命裡面.
你住在哪裡.
決定住在哪裡.
這個是上帝的命定.
所以我們嘗試去思考上帝的命定和召命.
來想想我們自己的情況.
基督徒只能夠帶著召命去做人.
我們流散是意味著一種召命.
我們怎樣能夠帶著上帝在我們人生裡面.
要我們做的事.
去到不同地方裡面.
來延續下去.
今天你可能離開海外.

$^{1081}$很多基督徒.
傳道人.
你都知道我經常去海外.
這兩年裡.
見到很多基督徒.
可能暫時不知道做什麼.
甚至傳道人都不知道做什麼.
但一定要找回自己的召命.
上帝要我來這裡.
我不是因為落難.
不是因為要逃避某些事.
要被迫出走.
而是我是正面帶著召命.
來面對我的處境.
上帝知道的.
上帝揀選.
我們流散是一種使命.
是差遣.
同樣香港人都一樣.
香港人都一樣.
我們都仍然是一個流散群體.
我們走來這裡.
但仍然是一個被流散這個詞形容的群體.
我們今天在我們的工作裡.
仍然在香港的時候.
都是帶著召命.
知道今天我們怎樣可以來到香港.
繼續延續上帝的召命.
這是我們上一季的第一堂所說.
基督徒的意思.
不是純粹宗教徒的身份.
而是我們怎樣見證上帝基督徒的意思.
或者見證基督群體.
所以第二季我們加上一個處境.
我們雖然這幾年裡面.
面對這個情況.
但我們是帶著這樣的一個身份.
去面對我們的處境.
不知道在場的弟妹妹.
怎樣去想.

$^{1121}$你怎樣去理解這幾年發生的事情.
怎樣去想你的未來.
你是否帶著一種召命去想自己的未來.
你是否將「留山」這個字.
看為一種使命.
我想這個就是我們留堂裡面.
一個很重要的價值.
我們是沒有形狀的教會.
它是一個很堅定帶著使命的教會.
我們一起祈禱吧.
然後我們請Captain Poon上來.
因為你讓我們留堂.
帶著「留」這個字.
你在早在四年前.
你都知道我們作為一個留堂的時候.
一個Flow Church的時候.
你叫我們去實踐這個「留山」的召命.
我們留堂不是被人擊打分散.
而是帶著一種更加強烈的使命.
來理解我們的處境.
無論我們在英國,在加拿大,在澳洲.
我們仍然來思想我們留堂.
作為一個這樣的教會.
一個很奇怪.
有時在YouTube,有時在Facebook.
有時在實體裡面.
好像分散它能夠不同方式來招聚的教會.
求主你這樣來幫助我們.
將我們每個人作為留堂的分子.
都能夠作為一個真正的基督徒.
來思考我們的召命.
我們求主你幫助我們.
讓我們能夠在金堂裡面.
更加認清聖經裡面很多很多的基礎.
我們知道這個流散,被擄.
不是純粹一個客觀政治現實.
而是一個對於萬國萬民的心意.
更加是你對我們的呼召.
求主你這樣幫助我們.
奉主命求,阿門.

$^{1161}$喂.
這個人次真的不一樣.
哇,真是.
很忙啊.
你坐下來吧.
你剛剛從多倫多回來嗎?.
你工作辛苦了.
辦公時間不要喝東西.
但現在不是辦公時間.
謝謝.
今天聽了一個內容.
可能對弟兄姊妹來說都有點新.
但其實剛才你所說的過程當中.
其實都有很多歷史源流.
過去教會很少特別提及.
或者是很認真去說.
在場很久沒見過機場有這麼多人.
不知道大家.
很多棒隊都在等你出來.
應該在等你工作.
應該可能都有些問題可以相關大家一起討論.
特別這個.
我想過去教會比較少討論的主題.
有沒有等待登機的乘客.
有些問題想參與一下呢?.
不用登記也可以問的.
我想問一下.
你怎樣想自己的未來?.
你怎樣想這幾年後的未來?.
你覺得有什麼做?.
可以分享一下.
剛才我看網上的時候.
看到有些弟兄姊妹問.
不要只說加拿大,英國.
其實真的有很多地方都涵蓋.
因為我相信周邊有很多弟兄姊妹.
總有不同朋友去的地方.
從來都沒聽過.
但事實上去了.
帶著不同的情感去.

$^{1201}$或者原因去.
對於今天在內容裡再一次提醒.
可能有些事情我們這一刻還不明白.
但其實都帶著一些使命.
可能未必覺得.
沒想過用使命這個原因去.
不過對於大家來說.
今天可能參與現場的你.
你在想什麼呢?.
有問題嗎?.
有些神學的問題想問一下.
第一點是.
在地老之前是聖殿神學.
就是上帝住在聖殿裡.
地老之後就流散在以色列人中間.
你覺得上帝有沒有搬過房子呢?.
我的意思是.
祂本身住在聖殿裡.
然後我躲在山樓裡搬了房子.
第二個問題也和這個有關.
因為本身地老流放是一個懲罰.
我想知道這個轉變是人的心態改變.
還是一開始由懲罰成為祝福.
這個變化究竟是一件怎樣的事呢?.
因為如果最後變成祝福.
就不是罰了.
沒有罰.
反而我反過來想.
你們這麼壞.
你們準備好可以出去了.
就放你們出去.
好像很奇怪.
我問你.
我想其實不是大家互相反對.
不是大家有矛盾.
我想是同一件事上.
祂肯定是懲罰.
因為很清楚寫明是上帝審判.
因為他們犯罪博弈.
但同時原來這件事有很多向度去理解.

$^{1241}$在審判之餘.
原來上帝有憐憫.
有盼望.
甚至會和他們同在.
這種事其實不會有矛盾.
沒錯他們是很慘的.
但同時也可以成為人的祝福.
所以舊約裡面.
我一本書強調.
這七個類別裡面.
是七個不同的角度去理解同一件事.
早期覺得沒有得罰.
很慘.
流淚就走了.
但當他們注入十八個愛.
原來也不是純粹是一種傳言的審判.
也有加上是上帝的盼望.
也會歸回.
也會有上帝同在.
所以我們今天面對的情況是很慘的.
但同時也有上帝的使命.
也有上帝的臨界.
所以剛才說的Shakina.
所以這是一種很重要的.
聖經也是.
我也不是強調聖經在聖殿裡面.
聖經是一個通達聖帝的地方.
所以聖經也是在天上的天裡面.
所以聖經只是一個中心點.
來將我們敬拜.
將我們的禱告.
如《伊武信》般升上去.
所以聖經是一個我們朝向的方向.
這是一個這樣的理解.
就算在那個時期裡面.
聖經也不是一個真正的上帝同住的地方.
而是一個象徵性的上帝名字的地方.
所以當後來聖經被毀的時候.
更加一樣.
原來上帝是可以不單單是天上的天.

$^{1281}$更加會跟著人們落難走.
這是一個很強大的愛.
因為耶和華上帝不是生神.
用教育的概念來講.
不是在山上.
不是巴黎.
但是他甘願跟著這班被人迫害的猶太人.
去到不同地方跟他一起走.
所以很強.
我們說得很低俗.
上帝同在.
上帝總是同在.
他經常說.
但是對於當時來說.
同在是一個很重要的愛.
因為他願意跟著一些落魄的人.
去到不同地方.
跟著他們在後面走.
所以這是一個更加深層的意義.
還有一個更加長遠的歷史.
聖經是說幾百年.
但是整個猶太歷史幾千年.
都是流散.
所以更加重要的是.
他是流散的神學.
一個上帝願意跟著這些人.
想想二戰的時候的猶太人.
在集中營裡.
上帝跟著他們在集中營裡.
所以上帝就是這樣.
所以我們作為基督徒.
更加要知道.
我們流散是一件什麼事情.
我們今天的處境.
沒有否定過.
也不是阿Q覺得不是一種災難.
但是這個之餘.
上帝仍然在當中跟他們同在.
還有帶著使命跟他們走.
用應用神學的人設來回應.

$^{1321}$就好像大家熟悉的約瑟.
約瑟當初被他的兄弟賣的時候.
原是惡的.
但是當他回想的時候.
他就看到上帝的美意.
而用同一個切合角度.
就是在約瑟當中的Sakina.
就是他要解夢的時候.
他要去執政的時候.
他就感受到耶和華神與他同在.
這個你就會明白到.
那個過程當中的相近.
有些事情想請教.
剛才提到有七個.
我用一個點來解釋.
因為我始終不太掌握到.
那個七個點是七個不同的階段.
還是七個不同的認知的觀點.
如果是七個不同的階段的時候.
它是有什麼刺激呢.
是由一個階段轉去另一個階段呢.
如果是七個不同的認知.
或者一個世界觀的時候.
是不是在當中的時候.
是同一時間在流散群體裡.
其實是有不同的群體.
有不同的認知.
但是如果是有的時候.
是什麼原因引致不同的群體有不同的認知.
我最想掌握的就是.
它那七點.
我看到其實好像是.
對於我們現在的基督徒來說.
第七點好像是.
較為安慰的訊息.
或者是一個盼望的訊息.
開頭那幾個點.
可能都是較為困難的認知.
如果用階段認知也好.
都是較為困難.

$^{1361}$怎樣在當中的時候.
作者去提.
在頭幾個階段認知也好.
怎樣去承載他們繼續走下去.
我覺得這是我自己.
對於當下的時候.
其實是較為關心和想掌握到.
謝謝.
這個其實是那本八百多本書.
最後的章節的結論.
所以那本書裡面就是將所有的經文字研究.
做歸納的研究結果.
所以這是一個很客觀的分析.
總之教育裡面有七個category.
將經文全部放在一起.
分七組.
七組不同意味的經文.
所以本身是一個很客觀的step.
原來教育裡面有七個不同的category.
有沒有一些階段性呢.
其實是一種詮釋.
有沒有呢? 我覺得是有的.
因為看到它們很明顯是從一個過得被罰.
慢慢慢慢接受.
好像一些階段性.
所以我覺得這七個.
其實是純粹經文裡面有記載.
以適當去理解自己的秘魯的回應.
我覺得是有階段性的.
但是怎樣到下一個階段.
我應該就沒有那麼清楚.
純粹我們知道教育裡面有不同的情況.
可能後期一點就慢慢發現.
原來上帝會同在.
所以我覺得又不是最後只有七個才是樂觀.
基本上是第一個審判之外.
那六個其實全部都是一些.
在困難裡面.
來發現一些好事的經文.
接受了宗教的團契.

$^{1401}$或者是上帝同在.
所以其實是沒有一種很容易的理解.
法國聖經裡面是這樣.
七個不同的經文.
講同一件事情.
所以我覺得很難給一個很簡單的答案.
原來只要這樣做就下一個階段了.
又不是這樣.
所以我覺得這件事就是.
教育裡面講秘魯.
是一個很明顯的痛苦現實.
但是會有這樣的分類.
所以我覺得我們又不容易很簡單.
將它簡化成為一個七個流程.
但這是一個事實.
聖經裡面有七個不同的.
描述講秘魯的經文.
(記者:劉先生,如果我們留下來).
(記者:我們在香港的教育).
(記者:劉先生,如果我們).
(記者:基督徒的一個使命).
(記者:我們應該視他為一個使命).
(記者:但如果).
(記者:好像很清楚我們在亂流中).
(記者:應該做些什麼).
(記者:但如果我不知道).
(記者:我知道神給我一個照明).
(記者:但我不知道應該).
(記者:擺自己在什麼位置或方向).
(記者:我們可以怎樣做?).
(記者:我正在看網上).
(記者:他問怎樣做).
(記者:怎樣找到自己的照明).
(記者:如果有人知道的話).
(記者:我看照明這個詞).
(記者:其實不是很高的東西).
(記者:我常常覺得上帝).
(記者:他會給你一個照明).
(記者:但我看照明).
(記者:其實不是很高的東西).

$^{1441}$(記者:我常常覺得上帝).
(記者:他會給你一個照明).
(記者:但我看照明).
(記者:他會給你一個照明).
(記者:我常常覺得上帝).
(記者:做我們出來).
(記者:有他自己給我們的).
(記者:那種心意).
(記者:意思是).
(記者:假設你).
(記者:當你真的做一個泥公仔).
(記者:那你自己的比例).
(記者:你做的姿勢是什麼).
(記者:其實你自己也有一個想法).
(記者:然後你想成品是什麼).
(記者:其實照明就是).
(記者:我自己看).
(記者:上帝給我的照明就是).
(記者:上帝應該有一個).
(記者:事情要我做).
(記者:對我來說).
(記者:我常常說的人設).
(記者:你的gift和你的talent).
(記者:你的經歷).
(記者:還有你自己的心意).
(記者:有沒有一些群體).
(記者:有沒有一些特別的場景).
(記者:你喜歡).
(記者:對於我來說).
(記者:你在成長過程當中).
(記者:在信仰生命當中).
(記者:不斷去了解).
(記者:自己的過程).
(記者:發生了什麼事).
(記者:具體來說).
(記者:過去在神學院).
(記者:會接觸到很多).
(記者:想進來讀神學的弟兄姊妹).
(記者:他們告訴我).
(記者:他們感受到上帝的呼召).

$^{1481}$(記者:我想了解一下).
(記者:他們的呼召具體是什麼).
(記者:就像我教應用神學).
(記者:就是有沒有特別的群體).
(記者:他們特別上心).
(記者:他們的技能).
(記者:或者他們的進路).
(記者:是用什麼方式).
(記者:然後大概怎樣用到).
(記者:他們將來出來服侍的方法).
(記者:所以你問我).
(記者:怎樣可以了解).
(記者:留下或出去的弟兄姊妹).
(記者:他們的呼召或使命).
(記者:其實就).
(記者:我就會朝著這個方向去想).
(記者:這樣讓自己知道).
(記者:我不是排他).
(記者:是他了上帝史無變有的能力).
(記者:就是突然間沒有).
(記者:他突然給你一次).
(記者:我不是這樣).
(記者:但大多上帝都會用).
(記者:一直以來的方式).
(記者:就好像).
(記者:你看很多聖經人物).
(記者:我經常都說).
(記者:那個例子就好像).
(記者:大衛能夠打死哥利亞).
(記者:不是因為他單單靠上帝的能力).
(記者:因為他自己都說).
(記者:當時蘇羅給了他一把劍).
(記者:他說他用不到的).
(記者:這件衣服我穿是動不了的).
(記者:但我一直都是).
(記者:對於野獸對於熊的方式).
(記者:就是這樣).
(記者:那就是上帝一直給他).
(記者:還有大衛是一個詩人).
(記者:他是唱歌).

$^{1521}$(記者:他放羊他悶就做這些事).
(記者:我想未必是一個).
(記者:他突然爆出來的事).
(記者:就是這樣).
(記者:我們跟我老婆聊天).
(記者:就說如果我們).
(記者:十幾二十年前).
(記者:沒有回應夫子要讀神學).
(記者:會怎樣呢).
(記者:我講到這個位置).
(記者:我應該和大家一樣).
(記者:沒有回教會).
(記者:沒有回到外國).
(記者:不知道去哪裡).
(記者:我老婆說).
(記者:因為他都是傳能).
(記者:如果沒有回應召命).
(記者:突然爆出來).
(記者:上帝應該給我召命).
(記者:如果不回應A).
(記者:上帝會給他召命).
(記者:這樣也可以嗎).
其實很有趣.
我不知道什麼召命.
但一個有召命的人.
我知道如果沒有召命的人會怎樣.
我覺得自己有召命已經很足夠.
明不明白.
我不知道是什麼.
但我知道自己有召命.
這個信念很重要.
如果沒有召命去活.
我只會去英國.
然後看看怎樣做.
雖然我不知道是什麼.
但我知道我有.
但找不到.
但我不知道是什麼.
都是在家裡找工作.
但我知道我有召命的話.

$^{1561}$是一個完全不同的態度去活.
所以你問我我召命什麼.
我可能是傳達人.
但總之我知道我有.
所以很重要.
無論在香港還是海外.
我們是被揀選.
很強調這個字.
我們是異鄉的客女.
是被揀選的.
這個字是一個很重要的字眼.
因為我們知道雖然我們散了.
收了又流散了.
但我們知道我們有上帝的命.
在當中的活.
是不能分開三個字.
我們是流散.
不過我們被揀選.
所以我們流散之餘.
我們知道自己.
是有一個事要去做.
雖然我不知道是什麼.
是可以的.
但我知道我是有多過無.
最怕是當一個有召命的人.
去到那裡之後.
突然沒有了一個召命.
這樣去活.
我不是為了做安全導人.
而是他沒有召命.
他沒有這樣做人.
是很痛苦的.
純粹在那裡.
因為這個緣故我就走了.
但你沒有召命.
他繼續下去.
你離開是可以的.
但你離開之後.
你真的繼續去找.
深信我是有召命的.

$^{1601}$你面對可能做一個家庭主婦.
在加拿大做某個工作.
但你知道是有的.
是重要的.
其實是揀選.
所以這個心態是重要的.
可能第一次會卡的一個位置.
我不知道什麼叫召命.
可能會卡在一個位置.
傳導人或傳職侍奉.
這個比較普遍容易掛單.
我打份工.
或者我在一個崗位上.
我有什麼召命.
這些位置會卡的.
我以前用過一本書.
跟弟兄姊妹講解一下.
你的角色是什麼.
那本書是一個退休.
呼蘭神學院的老師.
他叫做Robert J. Clinton.
他寫的書叫做.
The Making of a Leader.
裡面中立了很多聖經裡面.
一些領袖上帝怎樣去教導他.
或者培育他.
其中要帶出的訊息就是.
當初他未做一個領袖之前.
其實就是他在不同單位的職位.
怎樣做好他崗位上的職份.
慢慢他做到一樣.
又再加一點給他.
慢慢加多一點的時候.
不知不覺就能夠完成上帝給他做領袖的職位.
所以過程是一個慢慢的.
你說做不到那怎麼辦.
Clinton的說話就是.
學習過程失敗了.
那就要重做.
即是要再做.

$^{1641}$不過用另一個方式.
或者他不回應的時候會怎樣呢.
上帝用另一個方法.
或者用另一個途徑.
讓他再試那個方式.
好像若拿一樣.
即是上帝擺在每一個人生命當中.
都有一個使命.
我們人生的過程當中.
就是不斷去認證上帝給我們的使命.
不一定是全職.
是在你的職位裡面怎樣領導.
以致你可以成為一個領導職位.
所以整件事就是.
這件事就沒有地位限制.
留在香港也好.
或者散居其他地方.
你只不過是換了場景.
上帝仍然在你的生命當中有他的心意.
以致你在不同崗位當中.
去lead out 流出.
這就是你的照明.
剛才聽到你提及七個stage.
一個是接受流散的處境.
舊約的人被人擄走.
之後那個國家又變得非常強大.
擄走那些國家.
他們在那裡怎樣面對一個矛盾.
就是他們受害者.
即是仇人.
亞歷山河因為他身處在那裡.
他也要祝福一下.
因為那裡不好.
自己也不好.
以前的人怎樣面對這個.
舊約的人怎樣面對這個矛盾.
技術上我們不是說技術上.
但你會發覺流散.
或者被擄其實不是相當差.
亞述國是好很多.

$^{1681}$亞述國被擄是會好一點.
亞述國被擄是會.
一擄就會.
那時候好像很.
一擄是一家人一起擄的.
這樣是好的.
即是一起移民去那裡.
巴布倫是一個很壞的擄國.
他們會只擄男人走.
很凄利子散.
所以其實他們和.
都是會有很多不同的情況.
有時是會是.
擄工人一起走.
有時是一擄的時候.
所謂的一擄是一個deputation.
一個遷移.
所以就不打算只掛著監牢.
不一定這樣.
是一種遷移的情況.
所以其實亞述國是會將.
這些人和人搬來搬去.
令他們分散.
因為他們不能團結.
分散了.
所以不是一種純粹是.
太差的.
雖然都是差的.
所以不是說.
生活.
今天這樣.
今天都很差.
但都是生活.
所以所謂接受.
算是一種生活.
去找工作.
去生活.
去有信仰.
所以又未必是一個破家亡的情況.
是有的.

$^{1721}$但又不是全部都是這樣.
平民其實都是很普通.
這樣就被人遷移.
所以有這樣的情況.
這個是很technical.
technical的意思是.
在神目上怎樣運作.
或者怎樣去感受.
但我們之後的課堂都會講.
因為我自己覺得.
回應剛才的提問.
其實我們華人教會.
或者華人群體.
很少去學習什麼叫做分離.
我不知道大家成長過程當中.
有沒有教一個.
情緒教很少教的.
其實分離這件事.
在你的成長當中.
可能都沒有怎樣接觸.
所以要怎樣面對分離.
還有家這個觀念.
在我們華人.
特別是華人教會來說.
是很濃烈的.
你有什麼情況會離開這個屬靈的家.
離不開這三個原因.
如果你去讀神學.
你就帶著祝福離開這個屬靈的家.
但如果你說.
你結婚也是帶著祝福的.
不如就他離.
就不要你過去.
都還可以的.
但你說你搬遠了.
轉教會.
想轉這個屬靈的家.
其實香港有多大.
都是一個多小時車程.
但你搬到天水圍回柴灣.

$^{1761}$都很麻煩.
這些分離的感覺.
很不容易解決.
特別是.
我只是說.
雖然剛才聖經說了很多方式.
令人去分離.
或者整個群體分離.
但起碼我們香港.
或者過去成長這麼多年.
很少說這些話題.
或者沒有告訴你.
有一個離開的計劃.
所以我們要認真了解.
這個聖經的教導.
在現在的實務上.
怎樣去理解和執行.
後來也知道.
先知後繼先知.
要人們回到耶路撒冷.
人們不肯離開.
不肯回到自己的家鄉.
因為那邊太過安穩.
不敢回到耶路撒冷.
重建城牆之類.
所以發覺.
在外邦生活也不差.
他們的經濟.
或者文化.
適應了下一代.
就像我們今天的處境.
很多人都有不同的情況.
反而是.
問題是怎樣生活.
我們香港也一樣.
怎樣能夠在這樣的環境.
繼續去接受.
這樣的處境.
不是同意,但仍然有這樣的生活.
來開展.

$^{1801}$這是我們下一步要說的.
怎樣有新生活,新生命.
怎樣在這樣的環境裡面有新的開展.
甚麼是新呢?.
想問一個.
跟趙明有關的問題.
你剛才說用了藥室做例子.
你說可能被擄或留散.
可能跟趙明是相關的.
是同意的.
但我想問一個問題.
是先後次序的問題.
有時候可能你是被擄.
或留散了.
之後回頭看才發現.
原來趙明是神安排的一部分.
又或者另一個時候.
神給你趙明.
你去離開.
到底是哪一樣東西.
會比較多.
或者兩者都有.
還是怎樣去分呢?.
主要是想問.
關於這兩者之間的關係.
和先後的事.
我會看到.
可能是我們這樣.
或者是.
我不知道可不可以用你們兩個做例子.
例如你們有趙明建立一個.
領類.
Flow Church的教會.
還是其實Flow Church建立完.
之後慢慢發現.
原來上帝一開始叫我們這樣建立.
這間教會是有這樣的一個趙明呢?.
謝謝.
我先回答.
我回答的問題.

$^{1841}$先不是賣關子的.
不過又不要覺得是搞笑.
就是.
你怎樣認識女孩子的?.
你怎樣選擇女朋友?.
你是.
看完這麼多女孩子的標準.
或者你本身已經設定好標準.
我有五個標準.
中了這五個.
這個就可以做我女朋友.
還是你覺得.
這個女孩子不錯.
然後認識.
認識之後發現都不是我的杯茶.
我想說的意思就是.
怎樣才算是一個趙明呢?.
有時候有些東西.
有一個很具體的圖畫.
然後完全贊同下去.
但有些東西.
就是你做著做著.
你會對自己的趙明清晰.
我.
我自己比較喜歡.
去.
覺得信仰是一個.
談感情的過程.
信仰就是.
當然可以很神學.
Faith seeking understanding.
你帶著信心去求認識都可以.
但我們上帝.
我自己一直理解.
都是很幽默的.
在一個過程當中.
一個情投意合.
我靠近他多一點.
他又給我認識多一點.
我抽離的時候.

$^{1881}$他又不會不理我.
這個沉迷過程是重要.
我不是要淡化這麼嚴肅的問題.
不過你背後問的東西.
其實是想有一個方程式.
但我們正正就很難有一個方程式.
我和John都不會同意.
但我們兩個.
更加未必適合你用.
但我們會將一些可能性.
和你分享.
信仰的多面性.
或者豐富.
正正就是.
不說你不知道.
說你都知道不完.
但說多一點就知道多一點.
這個就是我.
在啟示上.
上帝啟示都是這樣.
不斷地上帝啟示.
讓子民知道.
到耶穌基督.
來到的時候.
約翰用得很好.
從來沒有人見過父上帝.
只有父懷裡的獨生子.
將祂表明出來.
約翰一二三書的時候.
他在晚年講得清楚.
這是我們親手摸過.
親眼見過和親眼聽過.
他就將所有東西.
第一身去表達.
過程當中.
有圖畫.
但又可以讓你互動.
沉覓一下.
希望你明白我的意思.
十居其九.

$^{1921}$都是不知道.
你會去了.
才知道.
靈鵪派就知道了.
收到一些東西.
阿伯拉罕.
都是憑信就走.
所以十居其九.
都是這樣的模式.
坦白說.
剛才我說要做明日進.
因為知道有東西.
有的時候.
有這個.
有這個就去.
有些人會拿著東西.
會拿著東西去.
但那東西是否真的.
你都不會知道.
但都可以拿著東西去.
神叫我這樣做.
但最後回頭看.
更加重要.
因為那是事實.
是上帝的作為.
像呼吸.
我只是摸摸.
現在就這樣做.
所以很多時候.
不妨試試拿著東西.
但又知道.
最後回頭看.
才會更加實際.
拿著東西的原因.
是你確信有.
神會帶領.
有一個召命.
就這樣做.
所以我經常都拿著東西.
你說神叫你去.

$^{1961}$我會說對.
但是否我們不需要.
多數都不知道.
不確定.
不妨說是.
反正大家容易去想.
但又不要太絕對.
留一些空間給上帝.
去改.
以前我在德國讀書時.
有幾個同學來過德國.
然後又回到香港.
讀不完或讀不成.
都是這樣的情況.
但儘管來.
都是這樣看神怎樣開路.
回頭看原來神不是帶我去德國讀書.
就是這樣的情況.
(記者:第一班機的乘客還有沒有其他登機前要問的問題?).
好像是茶會.
(記者:現在沒有茶會了).
現在不玩團了.
(記者:準備上機了).
麻煩你了.
我下班了.
我們下個月再見.
拜拜.
多謝大家收看.
\newpage



\section{約伯記 42:1-6-20230304}
\label{sec:dLJdySFiu9c}
\textbf{【流堂崇拜】陳明|約伯記42\_1-6|20230304 [dLJdySFiu9c]}
\newline
\newline
連結: \href{https://youtube.com/watch?v=dLJdySFiu9c}{\texttt{ https://youtube.com/watch?v=dLJdySFiu9c}} ~~~~ 語音日期: 2023-03-04 
\newline
\newline
\hyperref[sec:VfT5ldcLjqQ]{\small{< < < PREV SERMON < < <}}
~
\hyperref[sec:index_chronic]{\small{[返順時目]}}
~
\hyperref[sec:index_scriptual]{\small{[返順卷目]}}
~
\hyperref[sec:aPLQjM9J0JY]{\small{> > > NEXT SERMON > > >}}
\newline
\newline
約伯記 42:1-6-20230304
\newline
\begin{longtable}{cl}
\hline
\hline
章節 & 經文 (和合本修訂版)\\
\hline
42:1 & \begin{tabularx}{0.7\textwidth}{X} 約伯回答耶和華說: \end{tabularx} \\ \\ \relax
42:2 & \begin{tabularx}{0.7\textwidth}{X} 「我知道,你萬事都能做;你的計劃不能攔阻。 \end{tabularx} \\ \\ \relax
42:3 & \begin{tabularx}{0.7\textwidth}{X} 誰無知使你的旨意隱藏呢?因此我說的,我不明白;這些事太奇妙,是我不知道的。 \end{tabularx} \\ \\ \relax
42:4 & \begin{tabularx}{0.7\textwidth}{X} 求你聽我,我要說話;我問你,求你讓我知道。 \end{tabularx} \\ \\ \relax
42:5 & \begin{tabularx}{0.7\textwidth}{X} 我從前風聞有你,現在親眼看見你。 \end{tabularx} \\ \\ \relax
42:6 & \begin{tabularx}{0.7\textwidth}{X} 因此我撤回,在塵土和爐灰中懊悔。」 \end{tabularx} \\ \\ \relax
42:7 & \begin{tabularx}{0.7\textwidth}{X} 耶和華對約伯說話以後,耶和華就對提幔人以利法說:「我的怒氣向你和你兩個朋友發作,因為你們議論我,不如我的僕人約伯說的正確。 \end{tabularx} \\ \\ \relax
42:8 & \begin{tabularx}{0.7\textwidth}{X} 現在你們要為自己取七頭公牛、七隻公羊,到我的僕人約伯那裡去,為自己獻上燔祭,我的僕人約伯就為你們祈禱。我必悅納他,不按你們的愚妄處置你們。你們議論我,不如我的僕人約伯說的正確。」 \end{tabularx} \\ \\ \relax
42:9 & \begin{tabularx}{0.7\textwidth}{X} 於是提幔人以利法、書亞人比勒達、拿瑪人瑣法遵照耶和華所吩咐的去做,耶和華就悅納約伯。 \end{tabularx} \\ \\ \relax
42:10 & \begin{tabularx}{0.7\textwidth}{X} 約伯為他的朋友祈禱。耶和華就使約伯從苦境中轉回,並且耶和華賜給他的比他從前所有的加倍。 \end{tabularx} \\ \\ \relax
42:11 & \begin{tabularx}{0.7\textwidth}{X} 約伯的兄弟、姊妹,和以前所認識的人都來到他那裡,在他家裡跟他一同吃飯。他們因耶和華所降於他的一切災禍,都為他悲傷,安慰他。每人送他一塊可錫塔和一個金環。 \end{tabularx} \\ \\ \relax
42:12 & \begin{tabularx}{0.7\textwidth}{X} 這樣,耶和華後來賜福給約伯比先前更多。他有一萬四千隻羊,六千匹駱駝,一千對牛,一千匹母驢。 \end{tabularx} \\ \\ \relax
42:13 & \begin{tabularx}{0.7\textwidth}{X} 他也有七個兒子,三個女兒。 \end{tabularx} \\ \\ \relax
42:14 & \begin{tabularx}{0.7\textwidth}{X} 他給長女起名叫耶米瑪,次女叫基洗亞,三女叫基連哈樸。 \end{tabularx} \\ \\ \relax
42:15 & \begin{tabularx}{0.7\textwidth}{X} 在全地的婦女中找不著像約伯的女兒那樣美貌的。她們的父親使她們在兄弟中得產業。 \end{tabularx} \\ \\ \relax
42:16 & \begin{tabularx}{0.7\textwidth}{X} 此後,約伯又活了一百四十年,得見他的四代兒孫。 \end{tabularx} \\ \\ \relax
42:17 & \begin{tabularx}{0.7\textwidth}{X} 這樣,約伯年紀老邁,日子滿足而死。 \end{tabularx} \\ \\
[1ex]
\hline
\hline
\end{longtable}
$^{1}$兩姐妹平安.
很開心,先看看大家的樣子.
我們教會開了四年,三年被教會隔絕.
我每次握手送客的時候,都很想看下半截.
很開心見到大家的樣子.
期待越來越認到大家的樣子.
坦白說,真的認不到.
這幾年真的很考全道人.
尤其是在門口見到,拜拜的.
不容易認到大家的樣子.
很開心見到大家,一起回來崇拜.
很歡迎在不同地方和我們一起敬拜的弟姐妹.
這個月的月題是Sorry.
上星期當潘Sir公佈這個月的月題的時候.
我特別留意一下現場的弟兄姐妹有什麼反應.
我發現是沒有什麼反應.
每次這個月題的時候,都想知道弟姐妹對月題有什麼看法.
未來,很有趣.
漢建,Sorry,很有趣.
不知道大家對月題有什麼想法.
不知道大家有什麼感覺.
我自己建議這個月題的時候.
其實很想全道人的弟兄姐妹去思考一下罪.
我發覺我們開了4年,都比較少談及罪的題目.
所以我之前定的題目是認罪,懺悔,自首.
不過後來還是不要太嚴肅.
比較隨意一點比較好,不要太嚴肅.
後來就用了比較原性的Sorry來做月題.
不過我發覺香港人都很講Sorry.
我懷疑是我們全球除了日本人之外,講得最多Sorry的民族.
這樣令我們覺得Sorry這個字越來越失去應有的份量.
我懷疑是真的.
Sorry是我們每天講得最多的字之一.
它同時是我們最沒有去留意的一個字.
即使你講了也不覺得,也不記得,也不是很在意的字眼.
特別我發現我通常坐地鐵或逛街.
不小心撞到別人,是別人先跟我說Sorry.
大家有沒有發覺?.
明明是我撞到他,他也先跟我說Sorry.
所以我覺得這個字越來越失去應有的份量.

$^{41}$不好意思,對不起,Sorry,抱歉.
講得太多,好像也講得太少.
我們還發明了Sorry,Law這三個字.
很厲害的字.
實際上沒有一個語言是明明講了出來,突然又收回.
Sorry,Law,講了又收回.
所以我覺得很有趣.
希望這個月能夠一起思考一下.
道歉,悔罪,懺悔這個話題.
認認真真地對上帝,別人,自己說Sorry.
今天的講題叫做「陳明」.
我自己很喜歡「陳明」這個字.
一來就像一個人名.
確實是一個人名,應該有人會叫陳明.
陳明的意思是陳述心明,述說清楚的意思.
即是你一五一十地告訴別人.
將你心底裡的東西陳滅出來.
《醒》裡有很多陳明這個字眼.
令我一段時間懷疑這個字眼不是一個熟人字眼.
不過原來不是一個中文很普通的字眼.
我們一起來去教祖比神,一起去祈禱.
為我們Full Church去禱告.
為我們每一個頂尖會.
無論在網上與我們一起參與的不同地方的頂尖會.
我們在香港來看崇拜的頂尖會.
在現場的頂尖會.
我們每一個Full Church的頂尖會.
我們的生命交託給你.
我們能夠可以審查自己的罪.
能夠認真去認清我們一些的過犯.
讓我們的生命再一次因為這樣的緣故.
更加能夠親近你.
將我們的問題,將我們隱形未見的問題.
我們都求主你向我們陳明.
讓我們能夠同樣地向你陳明.
求主你幫助我們.
孩子是一個罪人.
但是你自己親自去使用.
讓你的聖言,讓你的說話成為我們心裡改變的力量.
能夠改變我們的生命.

$^{81}$能夠讓我們看清我們心裡的盲點.
求主你幫助我們.
奉主命求,阿門.
今天我們會看《約伯記經文》.
《約伯記經文》最後一章,第四十二章.
在《約伯記經文》裡面.
即是約伯經歷了四十一章的苦難.
來到最後一章的時候.
星座藉著《約伯記經文》的口去作出一個整卷書的總結.
不過要明白第四十二章.
我們不能夠離開三十八到四十一章的經文.
事實上三十八開始是一個明顯的段落.
即是耶和華上帝顯現.
然後就和約伯來對話.
三十八到四十一章是耶和華上帝的提問.
你可以打開聖經看看.
三十八章開始是耶和華上帝一連串的提問.
然後四十二章一到六節就是約伯對這些提問的回應.
讓我大膽推測一下大家讀《約伯記經文》的習慣.
我想大家都是大概看頭看尾.
你都是看他頭兩章,然後看最後一章.
這樣就看完了.
整卷書大家都是記得約伯記很慘.
全家死了,山麻風.
然後約伯的朋友很刻薄.
然後就看最後一章.
慘完,上帝就出現了.
就完,我從前風雲有你.
我現在親眼看見你,就完.
這個大概都是我們對約伯記的認識.
我們會再多說一點,最後的部分.
我們會說說三十八到四十一章的段落.
我們發現從約伯記三十八章開始.
耶和華在全風向約伯顯現.
然後耶和華上帝就好像發了瘋一樣.
一口氣向約伯連續問了五十三個問題.
我數過的,沒錯,是真的有五十三個問題.
我不知道你連續被人問五十三個問題是什麼感覺.
如果我平時教書,人們突然間一兩句說五十三個問題.
我就趕他們走.

$^{121}$大家都沒有試過一連串被人問五十三個問題.
而且是一些很難回答,很奇怪的問題.
而且是耶和華上帝問你.
一大堆奇奇怪怪,天文地理,生物常識,哲學道理的問題.
例如說三十八章十九節.
光明的居所從何而來?黑暗的本位在何處?.
三十九章第一節.
嚴艱的野山羊何時生產?你知道嗎?.
沒有下毒的之期,你能察覺嗎?.
三十八章第十六節.
你曾進到海原或深淵的隱密處行走嗎?.
四十章第一節.
你能用魚鉤鉤上海怪嗎?.
你能用繩子壓它的舌頭嗎?.
十三節.
誰能剝開它的外衣?誰能進它上下牙骨之間?.
很多很多國際性的問題.
然後問約伯.
你知不知道這五十三個問題的答案是什麼?.
當然約伯不知道.
然後對於耶和華上帝這五十三個問題.
第四十二章.
約伯終於開口回應.
星星話第一節.
約伯回答耶和華說.
我知道你萬事都能做.
你的旨意不能難阻.
誰用無知的言語使你的旨意隱藏呢?.
我所說的是我不明白的.
這些事太奇妙,是我不知道的.
求你聽我,我要說話.
我問你,求你指示我.
很有趣的是.
明顯這段回應裡.
對於上帝這五十三個你知不知道的問題.
先祿用了三次希伯來文.
「也」「知道」這個字.
來回應.
第二節約伯說我知道.
「也」「你萬事都能做」.

$^{161}$第三節約伯說.
我所說的是我不知道的.
第四節約伯懇求耶和華指示我.
指示也是「也」「知道」這個字.
甚至第五節整個約伯記最出名的金句.
我從前封聞有你.
現在我親眼看見你.
所說的都是一些知識.
明白上帝意思的字句.
所以我們今天講道.
去處理一個問題.
這個有關「知道」的問題.
約伯記第四十二章第六節.
約伯在塵土和爐灰中懊悔.
這樣去和知識,知道之間的關係.
我們問究竟約伯知道什麼.
他明白了什麼.
他頓悟了什麼.
尤其是經歷了這四十二章的苦難之後.
這個「知道」究竟有什麼意義.
我嘗試用四個點來總結.
有點複雜但也很簡單.
因為我已經點了.
第一個點.
約伯知道自己不知道.
即是想說約伯深深體會耶和華上帝的奧秘.
不能察覺.
深深知道自己完全不知道.
面對這五十三條的問題.
他只能說自己不知道.
這是第一點.
約伯知道自己不知道.
第二.
約伯某程度上知道耶和華的一些知識.
聖經說.
約伯知道上帝萬事都能做.
約伯的子不能難阻.
所以這些有關上帝的奧秘.
約伯不知道這些奧秘.
但他知道自己不知道這些奧秘.

$^{201}$這是一個很重要的知識.
意思是約伯不是知道奧秘的內容.
但他知道這些是上帝的奧秘.
這是第二點.
即是他知道上帝的一些奧秘.
大概是這樣.
第三.
承上題.
約伯清楚自己不知道耶和華.
約伯知道自己不認識耶和華.
他所知道的事.
就是知道上帝的事很奇妙.
上帝的旨意是隱藏的.
他很清楚知道這一點.
第四.
約伯祈求耶和華.
叫祂知道更多.
縱然很多人不知道.
但祂祈求上帝指示他.
叫他能夠明白.
叫他能夠知道.
這就是四點.
沒有重點.
大家聽我說.
更加入心入意.
即是約伯知道自己不知道.
約伯知道耶和華.
即是某程度上知道耶和華.
第三.
約伯知道他不知道耶和華.
第四.
約伯祈求知道更加多的上帝.
縱然之.
約伯知道上帝.
他同時間知道自己更加多.
不知道上帝.
我稱之為.
這是我們作為一個基督徒.
一個很重要的起點和基礎.
我稱之為一個摸底的過程.

$^{241}$很喜歡伊夫塞書.
保羅的一句經文.
與眾聖徒一同明白.
基督的愛是何等的長闊高深.
並知道自己的愛是過於人所能察覺的.
很有趣.
知道愛是過於人所能察覺的.
即是他過於我所能夠明白的.
是一句很矛盾的說話.
知道他超過自己所能夠知道的東西.
上帝的奧秘.
上帝的愛何等的長闊高深.
是無盡的.
但我用盡一切方法去探索他的內容.
好像你伸手摸底的失動一樣.
你越摸得越深.
就越發覺自己原來.
更加是這麼多都未明白.
即是你越摸得越深.
才發覺原來是這麼深.
但仍然是相差甚遠的感覺.
不斷的知道.
不斷發現自己不知道.
這個大概就是我們認識上帝的方法.
也是我們做人很重要的一個道理.
真的,頂智妹.
知道自己不知道.
知道自己是無知.
知道自己有些東西是看不清楚.
知道自己會做錯.
是一件很重要的道理.
甚至是一個很重要的知識.
知道自己不完全在真理的裡面.
甚至和真理是相差甚遠.
是我們人生一個很重要的定誤.
正如 Plato 所說.
不知道自己無知是雙倍的無知.
很有意思的說話.
不知道自己無知是雙倍的無知.
所以知道自己不知道.

$^{281}$和不知道自己不知道是四倍.
是相差很遠.
不知道怎麼算.
當你認識上帝越多的時候.
當你走這條基督徒的道路越來越久.
你就會發現上帝的奧秘.
自己和這個奧秘是越來越遠.
然後你會發現自己越來越渺小.
你越認識上帝越多.
你就會發現自己的無知.
不是謙虛的無知,而是真正的無知.
發現兩者的差距.
這種對於自己無知的發現.
是一個非常寶貴的得著.
是一種知識.
透過發現上帝的奧秘.
發現上帝的愛.
發現上帝的恩典.
那個不可能插頭的長闊高深.
不斷摸底,不斷發現自己的過程.
還要發現自己的軟弱.
本相和那個無能.
這是一個極度寶貴的基督徒的知識.
重點不是要去掌握那個奧秘.
而是你開始去發現上帝這個奧秘的開端.
所以認識而無患.
是智慧的開端.
我們未必能夠掌握.
但我們卻知道那個距離有多遠.
這樣就很足夠.
就像我們看日色一樣.
我們不能夠直接觀看.
但透過一個間接的方式.
來明白大概那個差距.
簡單來說,說得比較道理一點.
知道自己有什麼不妥是很重要的.
這是人生中很重要的道理.
作為一個人,你必須有時間跟自己說.
我真的很壞.
是這樣的.

$^{321}$有時間會有這樣的頓悟.
一定有這樣的想法.
否則你真的很不妥.
如果沒有這個想法的話.
有時跟朋友說.
你這樣做真的不太行.
你說不是我行.
不是你決定的,是吧.
而是其他所有人告訴你行不行.
是不是OK.
你不能說不.
你這樣做不行.
不是,我覺得很行.
是嗎?對不起,幸好我問你.
不會這樣,是不是.
你怎樣才是一個正確的回應.
如果有人跟你說你很不妥的時候.
你會說,哎呀,什麼事.
我做錯了什麼,快點跟我說.
有人跟你說你有問題的時候.
你這樣去反應才對.
認真去聽別人怎麼跟你說你的問題.
就像有人說你臉上有點髒.
你說,不是我髒.
你不會這樣說,是不是.
你會說,那你這邊擦掉吧.
所以這是很重要的.
做人的道理.
我講一下我自己.
我自己被人叫基督教KOL.
雖然我很不喜歡這個字.
現在在新學院裡被侍奉了十年.
接著你打圍棋找到我的名字.
不知道為什麼,誰說的.
這些東西.
好像很好.
但是我越大年紀.
就越發現自己那個很不行.
我真的很,偶爾.
他會說,阿直中不行,我真的不行.

$^{361}$我不是沒有自信,你明白嗎.
我四十歲,我不會沒有自信的.
而是真的發覺自己很多東西都很不行.
但是覺得是幸運的.
我發覺自己很不行.
同時間是一種祝福.
知道自己有多壞.
知道自己有什麼不是很擅長.
哪些做得不好.
我自己是知道壞的.
這件事是很重要的.
特別是在新學院裡.
或者現在的環境裡.
越來越少人跟你說一些直接抽擊你的說話.
人們都會遷就著你說.
如果你去做到某個層次.
人們不是太坦白.
他不是騙你,但也不會直接抽擊你.
都會留一點尊敬.
或者留一點力度跟你說話.
所以你很難發現自己有什麼問題.
如果你是升到這麼上下.
身邊都是一些比較正面的說話.
因為人們少跟你說一些不太合聽的說話.
但是做一個人,你犯錯的機會是一樣的.
不是你做得好.
而是人們跟你說的少了.
這些說話.
所以真心發現自己不是很行.
其實是一個很重要的知識.
不是謙虛,而是真正的認知.
最怕你覺得自己很沒有缺點.
很沒有瑕疵.
覺得任何人說你都是不對.
一個真正認識上帝.
或者真正去探索上帝的奧秘的人.
一個基督徒是真真正正發現自己的渺小和很多的缺漏.
這是一個很寶貴的知識.
所以你會發覺經文裡面最後一節.
經文裡面最後一節,第六節,約伯說.

$^{401}$他說:我從前風聞有你,現在親眼看見你.
因此我厭惡自己,在塵土和道灰中懊悔.
約伯終於明白,約伯終於知道,約伯終於頓悟.
經歷了一連串的苦難之後.
當他約伯承認自己是不知道,是不行的時候.
他竟然能夠沾釀到一點點上帝的中型.
一個人稱為偉大的上主的一點點中型.
更加去發現他自己的本性.
我從前風聞有你,現在親眼看見你.
因此我厭惡自己,在塵土和道灰中懊悔.
我們先說經文的上半部分.
約伯說我從前風聞有你,現在親眼看見你.
這句話的意思本來是說我從前用耳朵聽見你.
現在我用眼睛看見你.
當然重點不是媒體的問題.
不是純粹從耳朵變成眼睛,從聽歌變成視覺的問題.
他說的是真正的經歷.
一個真正的自己發現的親身經歷.
不是耳朵聽回來一句名題,一句說話.
而是親身去歷歷在目,看見,體驗到的真相.
不是頭腦裡面的一句聲音說話.
不是別人說給他聽.
而是他自己面對面去見到那種動見.
發現自己是一個罪人.
發現上帝,發現自己是一個罪人.
這就是陳明的意思.
約伯一五一十完完全全發現自己不知道,無知,有限.
卻竟然同時間見到上帝.
約伯說我親眼看見你.
唯有約伯發現自己無知的時候.
他才能真真正正去見到上帝.
他承認自己無知才能見到上帝.
從而去見到自己的本相.
這個向上帝一五一十的陳明.
同樣毫無保留地,上帝也一五一十地向他陳明.
詩篇三十二篇第五節.
我向你陳明我的罪,不隱瞞我的惡.
我說我要向耶和華承認我的過犯.
你就赦免我的罪惡.
將我們全部一五一十向上帝陳明.

$^{441}$把所有牌都攤出來.
發現自己一張牌都沒有.
竟然可以掉過地去見到那張黃牌.
大概就是這樣的情況.
不知道李信珠這麼久.
跟上帝的左隔是什麼.
不知道李信珠這麼久.
你怎樣去理解自己.
基督徒最淒涼的是連自己也騙了.
見不到一些應該要見到的問題.
不知怎樣說.
承認自己的問題.
嘗試將一些見不到的問題都陳明出來.
才能發現真正的上帝.
發現真正的自己.
約伯之後說因此我厭惡自己.
在塵土和爐灰中懊悔.
聖經裡面厭惡的字其實沒有自己的字.
所以厭惡這個字可以解作否定.
也可以解作retract.
即退後的意思.
約伯發現自己的本相.
將自己放在上帝面前一個適當的位置.
約伯見到自己的無知.
見到自己的渺小.
見到自己的罪.
他見到自己塵土的那個基本上.
知道自己有限.
知道自己在上帝面前那個應有的位置.
知道自己應該是怎樣.
這個就是約伯retract到一個塵土懊悔的狀態.
因此約伯懊悔.
約伯切切地跟上帝說對不起.
因為懺悔其實是一種知識.
作為一個人在上帝面前.
一個最基本,最根本,最重要的一個知識.
任何一次真心的懺悔.
都是一個你回到自己最default的狀態裡面.
重新再來過.
很重要.

$^{481}$原來我是一個壞人.
原來我有討人厭的地方.
原來我仍然很介意.
仍然我都是很得意,驕傲的.
我想以前我們看TVB劇的那些對白.
當某個角色大覺大悟的時候.
就會哭哭啼啼地說什麼.
媽媽,我知道錯了.
很重要這個知錯這一個字.
知和錯是分不開的.
錯誤永遠都牽涉一個認知的問題.
或者我們的罪永遠都牽涉一個認知的問題.
所以記得彼得在耶穌面前.
發現自己當他見到耶穌基督上帝的大能的時候.
他就說我是一個罪人.
請你離開我.
你見到上帝,你就見到自己的本相.
相反,你承認自己的無知.
你才能夠真正發現那位的上帝.
回到我們最基本的狀態裡面.
一個塵埃的本相.
但對我們新教徒來說.
認罪懺悔從來都不是一件難事.
因為人是有罪.
人都是犯了罪.
是一個很頭腦上的教義.
這是我們新教徒一些很麻煩的問題.
我們同時是義人,同時是罪人.
成為了一個你頭腦上的知識.
一個教義.
這個頭腦上的教義.
更加令我們很難去審查自己的問題.
反正世人都犯了罪.
當你知道一個這樣的知識的時候.
你覺得自己有問題.
是一件沒什麼特別的事情.
你頭腦裡面會認罪.
你都懂得.
理論上你要向耶穌赦免你的罪.
因為你十年前,二十年前.

$^{521}$你都認過一次的.
但約伯的懺悔並不是這樣.
他是真正的見到自己的本相.
一個真正的發現和頓悟.
原來我真的是一個壞人.
或者對某些人來說.
我是一個壞人都不是全錯.
所以我是一個罪人.
其實是一個認順.
你要憑信心來相信.
我們以前初信署的時候.
和全方位的個人都會說.
你有沒有說過謊話.
你有沒有驕傲.
你有沒有去憎恨人.
有啊,那你就有罪了.
你會得點頭的.
你會認同的.
我都覺得自己做錯了一些事情.
當然這些是.
那個罪是比較深層次.
因為我是一個罪人.
不單單是你一個發現.
而是一些你要去相信.
即是說你現在覺得你不是有罪.
其實你要相信自己是一個.
你還沒有發現,還沒有認清.
不只是一些關乎於你自己發現的罪.
那些可能只是在冰山裡面變了一層.
是更加多的問題.
更加對人的傷害.
或者對上帝對自己的傷害.
這些罪我們看不到.
所以常常都發現罪是一個認順.
你要相信.
就算你自己不覺的時候也好.
你要相信憑信心知道.
我仍然有些問題在.
即是在你沒有犯錯的時候.
你仍然要相信自己仍然是一個罪人.

$^{561}$仍然是一顆塵土.
上兩星期是聖夫日.
聖夫日是我們基督教大齋期裡的開始.
傳統人們會將一些中女.
燒成灰.
然後就塗在頂紙本的額頭上.
表示一個悔改的象徵.
當然這是一個象徵.
但很提醒你一個很重要的認順.
你其實還是有些東西你沒有發現.
你是一個塵土.
思念自己本是塵土.
如今也是塵土.
將來仍然歸於塵土.
知道自己的本相.
我希望底尖會.
這兩個月裡思考的課題.
我們這幾年沒有講過這個課題.
思考自己的問題.
有幾點小小的應用.
第一 剛才所講.
相信自己仍然有些問題.
這是一個很重要的一個假設.
或者不妨去假設的道理.
我知道大家以前的教會很多事情.
大概都是那些藍圖.
很多問題.
很多其他教會經過的傷害.
我不是批評受傷害.
不是這樣做.
但嘗試在當中.
從一個塵土的角度去理解.
嘗試想想自己有沒有一些部分.
仍然要向上帝去陳明.
很值得我們去想一想.
任何事情都想一想自己有沒有錯誤的部分.
值得不妨想一想.
不是說一定有.
不過你既然相信.
常常都犯了罪的時候.

$^{601}$很值得我們更加去講這個無知.
不知道 嘗試去求問上帝.
自己當中有些什麼.
當中有什麼圖畫.
第二 對自己去寬容一點.
不要逼得太緊.
知道自己是塵土.
就落於自己在這個本上.
真正見到自己是塵土.
你就不會自視過高.
也不會自卑過度.
接受自己是塵土.
體諒別人都只不過是塵土.
所以當你這樣看的時候.
你就發現原來自己.
就不會對自己太過高要求.
或者逼得自己太盡.
因為你是塵土.
但仍然是在上帝恩典之下的塵土.
在恩典當中我們仍然落於塵土.
發現自己本上的時候.
就懂得放鬆一點.
去理解自己的身份.
第三點我想講的就是寬容.
正正因為你明白自己只不過是塵土.
對別人寬容.
正正是一件最自然不過的事情.
不要太快去認定一個人是壞人.
也不要太快去認定對方是全然的邪惡.
不要太快覺得別人有錯.
也不要太快覺得自己擁有真相.
覺得自己是公義.
自己就是公義那一邊.
所以寬容是一種我們常常要去參透的智慧.
當然我們不容易見到自己是錯的.
所以是一種認順.
嘗試去尋求下去.
留一個空白位置問自己是不是有錯的地方.
所以寬容不是去純粹擁抱社會上的多元性.
不是相對主義.

$^{641}$更加不是埋沒公義和真理.
或者是去批評受害者.
相反寬容是建基於公義社會之上.
它是公義社會的第二個層次.
公義者要成熟地明白.
這個世界往往比我們所想的複雜.
真相往往是值得我們繼續討論下去.
發現下去.
見到自己是塵土.
就叫你好好的去認識其他人.
明白對方也只不過是塵土.
這個叫我們能夠帶著一種寧靜和勇氣.
去面對一些不同的事物.
一些和不同的共存.
所以寬容是一種很值得我們去思考的態度.
是人和人之間很重要的相處態度.
甚至可以嘗試去描述一個人.
今天我給你一個理由去考慮去描述一個人.
以前我們覺得描述一個人很視乎對方.
對方有沒有道歉.
對方有沒有悔改.
對方有沒有回應.
這些都是對的.
但更簡單的其中一個理由.
不是計較對方是怎樣.
也不是計較功能性上有甚麼好處.
一個更本體的理由.
就是你只不過是一個塵土.
所以就要描述.
因為塵土和塵土之間的恩怨.
都只不過是塵土.
對吧?.
免我們的債.
與同我們免了的債.
一個很簡單的描述理由.
因為只不過是塵土.
輕輕地抹掉它.
就好像窗邊的微塵一樣.
這個當然不是迫不是監.
但我們嘗試給一個理由去思考.

$^{681}$原來如果我們只不過是塵土的時候.
這些恩怨.
可以抹掉它.
對於這段若白的說話.
就完結了.
上帝似乎沒有怎樣給回應.
經文第八節.
一個很輕描淡寫的技術.
對上帝說話以後.
就完結了.
對於若白的懺悔.
上帝好像沒有怎樣說話.
但這個沉默.
是非常重要.
千言萬語.
這個沉默.
正正就把這些很複雜的問題.
放在這個沉默的裡面.
可能大家都知道.
我的女兒叫做婉婉.
她的名字其實是單字一個「願」字.
「願」字其實就是主統文裡面的「願」.
就是願你的名為聖.
願你的光覺能.
願你成就這個「願」字.
其實這個不是我第一個改的名字.
我第一個改的名字.
都是單字一個.
因為我德國讀我的論文是做祈禱的.
所以我打算改名叫做「陳土」.
真的.
叫做「陳土」.
我認真地說.
後來我改到真的想改的時候.
突然間我做老師的頂支部就說.
你不可以叫做「陳土」.
將來他回學校被人取笑.
我罰不了那些人.
我罰不了取笑他的那些人.
所以後來我就改名叫做「陳園」.

$^{721}$不過其實我是不斷地想.
如果我生了三個孩子.
第一個就叫做「陳土」.
第二個就叫做「陳園」.
第三個就叫做「陳明」.
如果是男孩子就叫做「陳明」.
因為這三個字其實都講述一個很重要的狀態.
我們人生裡面.
很多恩恩怨怨.
很多你對我好你錯我對的說話.
都只不過是向上帝的呼求.
都是一個陳土.
讓我們一起去審查一下自己的生命.
剛才那些說話可能是.
不可能針對到自己的情況.
不過都不妨嘗試去在神面前去默想,祈禱.
我們有一段默想的時間.
將來在你面前我們去沉默.
我們去審查自己.
我們仍然不單只是去發現.
更加要很確信我是一個罪人.
我是一個犯錯的人.
很多事情我都看不到.
主要我將我整個人的生命.
陳明在你面前.
將我的過犯.
在過去很多的事情裡面.
人家的錯誤,人家的傷害.
我自己的回應,我的錯誤.
主要都是將在你陳明.
主要我們知道我還看不到自己很多的問題.
如果我想承認我一定會承認.
但我看不到.
求主讓我們能夠.
讓你的聖靈去光光照我們.
看見我們的過犯.
尋明我們的問題.
讓我們在這個最原本的位置裡面去重新開始.
讓我們能夠在當中懂得寬容.
甚至乎有力量去饒恕.

$^{761}$不是因為對方做了甚麼.
對方仍然可能是一樣的.
但我思念我們只是在塵土的時候.
很多事情我都可以單單在恩典之下去抹去.
求主你幫助我們.
求主你繼續對我們說話.
繼續在這幾個月裡面帶領我們.
看見自己的一些過犯.
在你面前承認我們的罪.
求主你赦免我們的罪.
因為我們只是一個卑微的塵土.
你卻愛我們.
逢尊命求.
阿門.
(影片完結).
\newpage



\section{撒母耳記上歷代志上 31:1-6}
\label{sec:aPLQjM9J0JY}
\textbf{【流堂崇拜】執迷不悔|撒母耳記上31\_1-6;歷代志上10\_13-14|20230311 [aPLQjM9J0JY]}
\newline
\newline
連結: \href{https://youtube.com/watch?v=aPLQjM9J0JY}{\texttt{ https://youtube.com/watch?v=aPLQjM9J0JY}} ~~~~ 語音日期: 2023-03-11 
\newline
\newline
\hyperref[sec:dLJdySFiu9c]{\small{< < < PREV SERMON < < <}}
~
\hyperref[sec:index_chronic]{\small{[返順時目]}}
~
\hyperref[sec:index_scriptual]{\small{[返順卷目]}}
~
\hyperref[sec:y5NJfoAjRCI]{\small{> > > NEXT SERMON > > >}}
\newline
\newline
$^{1}$今天晚上我們會透過一個大家熟悉的聖經人物.
掃羅 神第一位呼召的君王.
他的失敗的例子去思想對不起這件事.
今天晚上我們會分三部分去完成這個訊息.
第一部分我們會看經文.
我們透過經文去了解究竟掃羅因什麼原因而死.
究竟他死在什麼原因裡面.
第二部分我們會透過掃羅的一生.
去嘗試在掃羅的錯裡面給我們正面的提醒.
然後最後一部分就是自己一個個人的體會.
今天我會用一個重點去貫穿這三部分.
有些錯不是單一的事情.
不是對某件事判斷的錯.
甚至不是道德上的問題.
有些錯是根本的錯.
就好像我們拿著一張錯的地圖.
就算你多努力去走你人生的路.
最後都是錯.
就如掃羅一樣.
他一生執迷不悔地去相信他自己那套是對的.
就算他多努力演繹他的人生.
最後都不是走上帝的路.
不過反過來.
如果我們認認真真去認錯.
我們不再執迷過去那些錯誤的理解.
我們就會有機會去發現新的路徑.
新的可能.
有些事情就會慢慢出現在你和我的眼前.
所以今天我希望透過這個訊息.
我希望你散會的時候仍然都記得.
我們認錯不是叫我們沒面子.
而是讓我們可以見到原來還有新的可能.
首先我們會看經文.
今天我們會透過三段的經文.
去想一想究竟掃羅是因什麼而死.
第一段經文是在《撒姆爾記》上的最後一章.
31章1至6節.
經文是不複雜的.
如果你看到的話可以看著經文.
讓我簡單去解釋.

$^{41}$以色列人再次和他的老對手菲利士人爭戰.
這次以色列人不夠打.
只有逃亡.
對手菲利士人窮追不捨.
先殺了幾個王子.
最後軍王掃羅都身受重傷.
他不想像他的先祖參孫落在菲利士人手上.
被人凌辱.
於是他就叫他旁邊拿兵器的人.
將自己了斷.
拿兵器的人不敢下手.
於是掃羅自己了斷了生命.
然後聖經記載拿兵器的人.
他都忠於他的主人.
死了自殺.
經文最後很簡單描述.
掃羅和他的忠義和領軍的都戰死在沙場裡面.
這段就是薩姆爾記上最後的一個段落.
成王敗寇.
不夠人打.
死在敵人手裡面.
有什麼出奇.
有什麼特別.
有什麼需要解釋呢.
反而在歷代誌上補充了.
有兩句經文很值得去想.
歷代誌上他這樣去描述.
這是薩姆爾記沒有的.
歷代誌這樣說.
他說掃羅的死是因為干犯了耶和華.
是沒有遵守耶和華的命.
又因為他求問交軌的婦人.
沒有求問耶和華.
所以神耶和華就使他被殺.
將他的國歸給耶西的兒子大衛.
如果我們將兩段經文一起看.
我想問一個問題.
掃羅的死因何在.
他究竟是什麼原因而死.
是因為死在不夠人打.

$^{81}$還是因為他沒有遵守神的命令呢.
薩姆爾記是沒有提及這兩句.
是歷代誌補充了這個原因.
我就開始去想.
究竟他為什麼不遵守神的命令.
這是其中一個原因.
經文有一個提示給我們去找.
我們就查一下薩姆爾記的十五章.
當時發生什麼事.
就是他不遵守神的命令是什麼事呢.
當時他又跟外敵打仗.
對手是另外一批外族人.
阿瑪力人.
神吩咐他要滅盡所有.
不要連累男女孩童吃奶的全部都要殺盡.
掃羅最後怎麼做.
掃羅很聰明.
在第九節裡面說.
他說掃羅和百姓卻連戳阿鴿.
就是對方的皇帝.
和連戳上好的牛羊牛獨.
所有好的東西都留下不肯滅.
就是不聽神的命令.
如果我們將三段經文加起來去想.
你覺得問題在哪裡呢.
你想想兩軍打仗.
我抓了對方的將領.
抓了對方的君王.
我將所有戰利品留下.
對將來打仗是很有用的.
掃羅所做的不是一些笨的行為.
是所有英明的君王將領都會做的事.
我俘虜了對方的敵人.
下一次再打我勝算大很多.
三段經文真正讓我們找到的.
出問題的不是掃羅.
是耶和華的要求.
神給了一個不合常理的要求給掃羅.
叫他跟著做.
掃羅考慮了.

$^{121}$他知道這不是他的想法.
於是經文記載.
如果大家有興趣可以回去看.
他選擇了沒有按這個屬靈方法去做.
他選擇了世上所有君王都會用的方法.
英明的方法去走他的路.
所以最後他選擇俗世那條路.
往後他的一生.
都是不停地選擇世界上認為最英明.
靠車靠馬靠戰利品那條路.
所以三段經文加在一起.
我就開始明白《撒慕爾記》上最後一段.
為什麼他這樣記載掃羅.
沒有解釋任何他死亡的屬靈原因.
我發現掃羅已經一早踏上一條俗世的路.
他選擇了一個成王敗寇.
弱肉強食.
優勝劣敗.
所有的俗世觀念的遊戲規則.
所以去到尾他不夠人打.
死在沙場裡面是正常不過的.
《撒慕爾記》甚至不需要提他有什麼屬靈的原因.
他這樣戰死.
因為他跟上帝所給他的路已經隔走隔路.
完全沒有關係.
反倒是《撒慕爾記》記載掃羅最後一章的對上那段經文才是重要.
對上那段經文大家猜猜是記什麼的.
是記他的對手大衛同樣打外邦人.
他的對手大衛被外邦人入侵.
但大衛卻是過了一個非常不合理的爭戰結果.
明明對手比他強.
擄掠了他的家人.
但大衛卻是可以以小勝多.
以弱勝強.
甚至在過程裡面有些自己人沒有跟他打仗.
但他打贏了之後.
甚至將戰利品分給那些沒有去打仗的人.
他以善勝惡.
《撒慕爾記》是想將兩個君王.
兩種打仗的方式陳述在我們面前.

$^{161}$而最有趣的是.
你猜大衛跟什麼敵人打呢.
亞馬利人.
就是當年掃羅放過的亞馬利人.
頂姐妹如果我們歸納了這三段經文.
我嘗試比較兩位君王.
他對Sorry的理解.
大衛的人生跟掃羅的人生截然不同.
兩位明明都是上帝呼召的人.
他呼召的過程都是很相似.
兩位都有人生裡面一些陷阱.
一些錯事.
但我們對比大衛和掃羅.
他對於對不起這個看法.
是截然不同的.
大衛在往後的日子.
他都做過一些壞透的事.
我們都知道的.
但當他知罪之後.
他是流淚悔改禁食.
他是認認真真去面對自己的邪惡.
甚至他寫下很多詩篇.
去表達他很想歸回在神的身上.
他要走回那條熟悉的路.
但如果你看掃羅.
他就剛剛相反.
聖經有記載掃羅說對不起.
不過他不只是說對不起.
他還加了一個字.
Sorry, Law.
問題不在對不起.
問題在哪裡.
問題在語氣上.
你試一下得罪太太.
你跟她說Sorry和Sorry Law.
你猜後果有什麼不同.
我猜你都知道問題在哪裡.
掃羅很想很快滾過那個問題.
他不是打算認真反思.
他打算棄掉就算了.

$^{201}$我說了對不起.
解決了它好不好.
大家不要再爭辯了.
這個才是說對不起的問題.
我不覺得自己全錯.
只不過事情搞大了.
我就解決了它.
繼續走就行了.
問題不是有沒有說對不起.
問題是有沒有認真知道自己做錯了.
神給了很多機會掃羅.
但掃羅一次一次又滾過去.
當沒事發生.
他一生決定走一條跟世俗君王一樣的路.
所以撒姆爾記上最後記載那場戰役.
他就像世俗君王一樣.
不夠人打死在沙場.
就是這麼多.
多說一句都不需要.
頂姐妹第一部分我們看完掃羅的一生.
執迷不悔.
他認為自己對的事情就堅持到底.
哪管上帝出聲.
我當沒事發生.
滾過就行了.
有什麼所謂.
反正全世界都是這樣做.
我就照做.
然後我們進入第二部分.
我們在這個執迷不悔的生命裡.
有什麼值得我們去學習呢.
我嘗試去想.
有兩方面在掃羅的人生裡.
給我自己一些提醒.
第一我們真的需要在靈性裡保持敏銳.
我相信生活在香港是一個資源很有限的地方.
我們現在以往疫情的時候.
廁紙不是買的.
廁紙是什麼呢.
搶的.

$^{241}$我下去搶廁紙.
我跟太太說.
人才不是培育的.
人才是什麼呢.
小時候媽媽教我不要搶東西.
但我不知道為什麼.
我生活在一個需要搶的地方.
早前我教會一對弟兄姊妹.
準備分娩.
她很擔心通關之後買不到她想買的牌子的奶粉.
於是太太就吩咐丈夫.
你嘗試去找有沒有我們想買的牌子的奶粉.
丈夫很好.
去一間沒有.
兩間沒有.
五間.
八間.
十間.
都沒有.
他都打算放棄.
隨便吧 都買其他牌子先頂住.
最後他走到第十三四間.
他找到三罐.
給你.
你會怎樣做.
掃羅曾經有兩次落在大衛的手上.
大衛的下屬跟他說.
機會來了.
你一刀解決他.
以後問題就沒有了.
給你.
你會怎樣做.
全世界都懂得做的事.
上帝卻是出聲.
你不要這樣做.
頂尖木有時很多俗世的觀念不是錯的.
買盡三罐.
見到有三罐奶粉.
但偏偏神在這個弟兄裡面說了一句話.
留下一罐.

$^{281}$或者某一個家庭.
某一個祈禱.
讓這位弟兄有這個感動呢.
是不是.
買盡人人都懂得.
但這位弟兄敏銳聖靈對他的提醒.
不一定按著世俗的方法去過我們的人生.
有時上帝會出聲.
他想我們用另外一種方法去表達你的人生.
勤力工作是正常的.
多讀兩個學位.
多儲錢去旁身.
去找你自己喜歡的人.
去一個你認為適合生活的地方.
繼續過人生.
只要不違反聖經的原則.
上帝都喜悅的.
不過上帝有時刻意提出一個不太合理的要求.
甚至是有些笨的要求.
這個叫做屬靈的要求.
請你不要按世俗的玩法去過你的人生.
你又願不願意呢.
你有沒有再敏銳到上帝對你說話.
不是按你的人生經驗去判斷所有事.
而是上帝刻意跟你說.
人人都可以做.
但你這一次不要這樣做.
頂枝梅蘇羅不是沒有經歷過神.
神亦都多次提醒他.
但他已經慢慢不想再敏銳上帝.
他要走自己那一套.
他自己想到一套人生的方法.
拿著一張他自己畫的地圖.
去完成他的一生.
這才是令我們警醒的地方.
所以第一點我想我們彼此提醒.
保持對上帝敏銳的心.
保持對上帝的靈.
仍然敏銳.
人人都做,人人都對.

$^{321}$這些不是大是大非的錯.
買了三罐有什麼問題.
有什麼犯罪.
但上帝說你可以不是這樣.
你是不同.
這是第一.
保持對靈的敏銳.
第二.
就是保持對信仰的掙扎.
我猜大家完了流唐崇拜之後.
最多你們都是去分組吃糖水.
聊聊天.
有沒有人約了頂枝梅去麻將館打四圈.
真的有嗎.
看你笑得這麼開心.
今晚找到水魚了.
我們都不會過份到.
將這些黑白的界突然本末倒置.
我們都會知道有條界線在這裡.
但我們發現在疫情過後.
特別我自己在牧羊裡.
我都發現這條界越來越模糊.
有時候我們回崇拜.
可能因著工作的緣故.
有些特別的事情.
不能夠穩定地出席.
這個我明白的.
有時候約了朋友.
又或者前一晚玩了.
我教會禮拜日崇拜.
有時候早上來不及起身.
還沒準備好回來敬拜.
回來烏味喝水.
我明白的.
偶然一次出現這些情況.
上帝沒有擊殺我們.
是吧.
起碼我們在這裡.
上帝紀念我們.
沒有投訴.

$^{361}$但是當我們將偶然一次.
變為理所當然.
那條界線不見了.
問題就出來了.
當我們偶然一次回不了崇拜.
我們都有些許怯懼.
回不了了.
下次要好一點.
準備得好一點.
交代得好一點.
一次.
到第二次.
都是有些許怯懼.
第三呢.
第四呢.
整個疫情過後.
不回崇拜已經變得理所當然.
我估計這是我親眼看到的事.
有什麼所謂.
回不回都是這樣.
反正政府隨時停止我們都可以.
我自己都停止自己.
當政府說不准弟兄姊妹進教會的時候.
我近乎是抓狂.
我教會在平台上.
我將來跟太太說.
我要坐在教會門口.
有人回來.
我都會跟他說一篇道.
跟他私性餐.
我真的這樣做.
我是抓狂對這件事.
結果你估計有沒有人回來.
當然沒有.
我自己浪漫一下.
那天還是最冷那天又下雨.
弟兄姊妹是真的.
我有條界在這裡.
有些事情我無所謂.
大部分都無所謂.

$^{401}$但對於淑齡的界.
你跟我還有沒有所謂.
這條界會不會漸漸自己都模糊了.
沒什麼所謂.
自由就行了.
上帝最大的懲罰.
就是讓我們任意而行.
你想怎樣就怎樣.
掃羅就是這樣.
我不理你.
你自己喜歡走自己的路.
你就走飽了.
反倒你都不跟我.
最大的懲罰就是你任意而行.
而過得很幸福.
連你迴轉的提醒都不給你.
你不覺得很可怕嗎.
弟兄姊妹是真的.
保持對信仰一種張力.
這個功能在團契裡必須要出現.
我們在小組的生活裡.
除了有很溫馨的時間.
回來一起說上司是非之外.
我們彼此要保持一種對信仰的掙扎感.
我眼前這群弟兄姊妹.
仍然傻傻惡惡地堅守著那些笨的道理.
眼前這群人跟我一樣.
相信不要效法這個世界.
這群人跟我一樣.
會先求神的國和神的義.
就算這個世界追求要贏得所有.
但我還有一群弟兄姊妹.
他們認為貧窮是有福的.
團結的生活是堅固著我這一部分.
我仍然要在裡面保持和世界的一種張力.
因此上帝容許我們經歷一場大病.
一種關係的破裂.
一種人生的傷痛.
他想讓我們當頭捧腔.
他要你跟我說.

$^{441}$朋友,小心點.
走到這一步你要停一停.
想一想.
你再這樣下去.
我救不了你.
你再執迷在你自己認為對的地圖裡.
過你的人生.
我再大聲說,你都不會回來.
頂姐妹保持信仰的張力.
蘇羅就是沒有了這個群體.
她獨行獨斷.
連她的兒子都不理她.
頂姐妹你有沒有一個讓你保持張力的群體.
有沒有一些人還可以彼此提醒.
讓我們相信不靠車不靠馬是可能的.
還有沒有這些人在.
我們自己是不是這些人成為別人的提醒呢.
頂姐妹平心而論.
信主和未信主的.
我們都在工作,家庭,健康,前途裡掙扎.
信主的我們就求耶和華.
不信主的求誰呢.
王大仙.
一樣這樣求.
我們都是求家宅平安.
順順利利.
升職加薪.
不要加工作.
我們都是求這些.
頂姐妹我們求王大仙的最大不同之處.
就是就算上帝沒有給我這些.
我也不會拜別神.
就算上帝不允許我所有的祈求.
我仍然會在那些爛的人生劇情裡唱歌.
我仍然會在困乏裡跟別人分享.
我仍然會在病患裡傳感恩的心.
說感恩的話.
頂姐妹信耶穌如果萬事如意的話.
我們傳的福音是很容易的.
很吸引的.

$^{481}$但就算我們信主不是萬事如意.
我們仍然堅信這個信仰.
我們的福音不是吸引.
是震撼.
我們所傳的福音是震撼的福音.
不是求什麼就有什麼.
我就信.
你當我是什麼.
這麼容易說了算.
而是我決定了這位神是真真實實.
帶領過我走過很多很多關.
就算將來我所求的未必如願以上.
但你不會一兩次就放下跟隨這位主吧.
你當我是什麼.
當我拜別神嗎.
我是信這位神帶領我走完這條人生路.
大衛就是這樣選擇.
你以為大衛的刀劍少嗎.
大衛一生他都要回到神那裡.
哪管他被自己的親生兒子追殺.
哪管更可怕的事情出現在他家庭裡.
詩篇所寫的全部都是真真實實.
流過眼淚.
悲痛過.
認真認錯過.
這才是神選定大衛作為君王的原因.
大衛不是道德上比素羅更好.
而是他堅信這位上帝是真真實實.
這才是被選定的原因.
頂姐妹我希望透過今天的經文.
我們敏銳聖靈的說話.
也敏銳有頂姐妹跟我們一起.
保持著一份信仰和俗世的張力.
是辛苦的.
是有拉力的.
在公司裡要做好人是艱難的.
不過怎樣.
我是繼續做好人.
我什麼好.
我是吃虧的.

$^{521}$不過上帝喜歡.
我就做.
就是這麼多.
然後到第三部分.
是一個個人的經歷.
小小的題外話.
我是一個很早起床的人.
大概五點多我就已經在街上晃來晃去.
通常晃到大概七點鐘.
我的掙扎就出來了.
七點鐘有很多食肆就開了.
我就開始掙扎.
究竟回家吃早餐.
跟太太一起吃好.
還是自己在街上吃好.
雖然有點誇張.
但我每一天都掙扎.
每一天經過餐廳都掙扎.
我知道回家吃就省錢一點.
健康一點.
也可以陪太太一起吃.
但我又知道在餐廳吃.
又好像自癒一點.
甚至近乎儀式感.
有一次在餐廳掙扎完之後.
決定在餐廳吃.
去到餐廳我又掙扎一輪.
A餐還是B餐.
你知道A餐是什麼嗎.
火腿通嘛.
健康一點.
但B餐又好像自癒一點.
沙嗲牛肉麵.
通常都是.
在我掙扎的時候.
後面來了一位婆婆.
你知不知道她怎麼叫.
我都大開眼界.
她應該是熟客.
她跟老闆娘說.

$^{561}$老闆娘我要一個沙嗲牛肉拼五香肉丁丁麵.
加奶醬豬熱奶茶多奶.
我忍不住看一看.
一位很瘦小很小的婆婆.
嘩 嘆為觀止.
原來可以這樣叫的嗎.
我之後跟幾位弟兄說.
他們真的去試.
真的可以的.
四位弟兄吃了這餐四份.
當那餐經過我送到那位婆婆身邊.
我瞄一瞄.
一大早七點.
像盤菜一樣.
最後的早餐.
頂尖的我想說什麼呢.
我有時候很想越過這條健康的死線.
我很想投奔俗世.
我很想相信今晚可以玩野一點.
我很想一廂情願覺得.
藥箱是可以變成首飾箱.
我甚至最喜歡的茶餐廳的金句.
就是吃飽一點才有力減肥.
頂姐妹是的.
正當我寫這段經文的這一部分的時候.
我腦海正盤算.
我今晚一定要放盡一點.
開放我食放題的功能.
正是我寫這一句的時候.
我電話收到一個WhatsApp.
是我在網上跟頂姐妹讀經.
一個弟兄去分享他那天讀經的體會.
我看完之後眼都白了.
我照讀給你聽.
我那天早上收到這個WhatsApp.
他這樣寫.
他說年輕的時候我真的狂吃不胖.
腰圍一直維持在32.
血壓是120 80.
但是我不知道在哪天開始.

$^{601}$一切都走下坡.
我看見這個短訊.
我深呼吸了多少下.
相信活的神是很麻煩的.
如果我相信一個偶像.
我跟他的來往只是逐單計.
我去拜他.
他靈驗.
我便還神.
但相信活的上帝.
他是常常打擾我的人生.
他常常提醒我不要走這條逐世痛快的路.
我知道我可以越界.
我不是做犯罪的事.
不過上帝說話.
上帝有話對我說.
你可以怎樣做.
人大了我知道吃喝不敢隨意.
就算失敗很多次也好.
我仍然會不斷努力.
多菜少肉.
少糖少鹽.
每日八杯水每晚早點睡.
現在我過了十點鐘的時候.
鹹蛋超人的警號便會響.
好像快不行了.
所以我很快便說完.
頂尖妹真的說.
我們對飲食有要求.
聖經不是為身材好一點.
而是生存的模式.
我知道我不可以任意而行.
聖經這樣說.
操練身體益處還少.
如果我們仍然著緊身體的變化.
聖經提這些都不是最重要.
就正如我們的靈性.
不是用來點綴我的信仰生活.
不是讀多點聖經給隔壁的.
乖一點好一點他不錯.

$^{641}$我讀少一點也沒問題是差一點.
不是對不起不是.
我們的靈性是我們在曠野裡的指南針.
它是告訴我們走的路是否正確.
我們的靈性是在敏銳上帝隨時的發聲.
因為我和你都沒有走過這條人生的路.
不懂得怎樣走.
靈性是帶領我們跟著上帝一步一步走.
去完成你的一生.
靈性不是點綴是用來救命.
頂尖我們在素羅和大衛的經歷裡.
充充分分知道這件事.
最後在剛才那段經文裡有一個彩蛋.
一個叫拿兵器的人.
不知道你們有沒有留意到這個人物.
聖經提了很多次.
拿兵器的人他是誰.
你知不知道.
撒慕爾記上十六章記載.
大衛曾經是素羅拿兵器的人.
你可以試幻想.
如果當年素羅可以放下自己的身份.
容許一個像羅賓.
我小時候看過《蝙蝠俠羅賓》.
羅賓的角色是用來做什麼的.
就是每次被人抓去.
然後蝙蝠俠去救他.
他的作用就是被人救.
完全沒有意義.
每次都是這樣.
但如果當時素羅容得下上帝.
用這個配角一樣的小哥大衛.
成為君王拿兵器的人.
你想想那段經文會變成怎樣.
素羅把大衛拍住.
加上他很能打的兒子.
再加上言無虛發的撒慕爾.
你想想非利士人哪是他的對手.
等於如果當年素羅容得下上帝呼召另一個人.
成為他的大靈.

$^{681}$他容得下這個故事發生.
整本聖經有很多不同的變化出現了.
不過可惜我們見不到素羅認認真真去悔改.
我們見不到這個拿兵器的人成為新一代的領袖.
領子妹是說.
或者你今天都遇上一個難的處境.
有時候我們都會問.
人人都有為什麼只有我沒有.
又或者反過來.
人人都沒有.
但偏偏是你有.
為什麼會發生這些事.
有時候我們不是想跟上帝說對不起.
而是反過來.
我想上帝跟我說對不起.
為什麼你搞到我的人生是這樣.
領子妹我不知道你有沒有偷偷想過這些話.
上帝你是不是對不起我.
你是不是搞到我這樣.
不過在素羅的人生裡面.
在大衛的人生裡面.
我們自己要學懂.
上帝有時候容許一些難處.
臨到你臨到我身上.
他很期望我和你用另一種演繹的方式.
去表達一個有信仰的人生.
當我們能夠放下俗世很多很多的觀念.
當我們不再執迷我們過去認為對與錯的解釋.
重新跟上帝結連.
我和你可能會發現一條新的路徑.
一種新的可能.
在眼前出現.
讓我們一起祈禱.
人生走到一個迷陣.
一個彎位.
我們好像不懂得怎樣走下去.
神你透過今天的經文.
透過敬拜的詩歌.
讓你的靈喚醒我們的心.
讓我們走出這個禮堂的時候.

$^{721}$我們抬頭望上天.
知道上帝你仍然對我們微笑.
你仍然容許我們去走新的路徑.
有新的可能.
願你引領賜福給每一位.
祈禱奉主名求.
阿們.
\newpage



\section{撒迦利亞書 5:1-11-20230318}
\label{sec:y5NJfoAjRCI}
\textbf{【流堂聖餐崇拜】《今際之國-飛行的書卷》|撒迦利亞書5\_1-11|20230318 [y5NJfoAjRCI]}
\newline
\newline
連結: \href{https://youtube.com/watch?v=y5NJfoAjRCI}{\texttt{ https://youtube.com/watch?v=y5NJfoAjRCI}} ~~~~ 語音日期: 2023-03-18 
\newline
\newline
\hyperref[sec:aPLQjM9J0JY]{\small{< < < PREV SERMON < < <}}
~
\hyperref[sec:index_chronic]{\small{[返順時目]}}
~
\hyperref[sec:index_scriptual]{\small{[返順卷目]}}
~
\hyperref[sec:7ZGXT0f30Z0]{\small{> > > NEXT SERMON > > >}}
\newline
\newline
撒迦利亞書 5:1-11-20230318
\newline
\begin{longtable}{cl}
\hline
\hline
章節 & 經文 (和合本修訂版)\\
\hline
5:1 & \begin{tabularx}{0.7\textwidth}{X} 我又舉目觀看,看哪,有一飛行的書卷。 \end{tabularx} \\ \\ \relax
5:2 & \begin{tabularx}{0.7\textwidth}{X} 他問我:「你看見甚麼?」我回答:「我看見一飛行的書卷,長二十肘,寬十肘。」 \end{tabularx} \\ \\ \relax
5:3 & \begin{tabularx}{0.7\textwidth}{X} 他對我說:「這就是向全地面發出的詛咒。凡偷竊的必按書卷這面的話除滅,凡起假誓的必按書卷那面的話除滅。 \end{tabularx} \\ \\ \relax
5:4 & \begin{tabularx}{0.7\textwidth}{X} 萬軍之耶和華說:我要把這書卷送出去,進入偷竊者的家和指著我名起假誓者的家,停留在他家裡,連房屋帶木頭和石頭都毀滅了。」 \end{tabularx} \\ \\ \relax
5:5 & \begin{tabularx}{0.7\textwidth}{X} 與我說話的天使前來,對我說:「你要舉目觀看,看那出現的是甚麼。」 \end{tabularx} \\ \\ \relax
5:6 & \begin{tabularx}{0.7\textwidth}{X} 我問:「這是甚麼呢?」他說:「這出現的是量器。」又說:「是他們的眼目,遍行全地。」 \end{tabularx} \\ \\ \relax
5:7 & \begin{tabularx}{0.7\textwidth}{X} 看哪,圓形的鉛蓋被抬起來,有一個婦人坐在量器中。 \end{tabularx} \\ \\ \relax
5:8 & \begin{tabularx}{0.7\textwidth}{X} 天使說:「這是罪惡。」他就把婦人推進量器裡,把鉛蓋壓在量器的口上。 \end{tabularx} \\ \\ \relax
5:9 & \begin{tabularx}{0.7\textwidth}{X} 於是我舉目觀看,看哪,有兩個婦人前來,她們的翅膀中有風,翅膀如同鸛鳥的翅膀。她們把量器抬起來,懸在天地之間。 \end{tabularx} \\ \\ \relax
5:10 & \begin{tabularx}{0.7\textwidth}{X} 我問那與我說話的天使:「她們要把量器抬到哪裡去呢?」 \end{tabularx} \\ \\ \relax
5:11 & \begin{tabularx}{0.7\textwidth}{X} 他對我說:「要抬到示拿地去,為它建造房屋;等預備妥當,就把它安放在自己的臺座上。」 \end{tabularx} \\ \\
[1ex]
\hline
\hline
\end{longtable}
$^{1}$各位頂展姐妹,晚安.
大家都不覺得有107級吧.
這個地方是我以前在幕會十多年的時候.
一年都會來兩三次的地方.
我中派通常都會在這裡聚會.
大時大節,尤其是復活節的時候.
所以這個地方都很陌生.
通常我們都懶的,坐車上來.
坐車上來,就不走上來.
回到這裡,感覺就像回到Future最早期的時候.
在石硤尾的年代.
外面都坐滿頂展姐妹.
又不要看到最後面的頂展姐妹樣子.
好,先問一句.
《金祭之國》大家有沒有看過?.
沒有,那就不要說那麼多.
《黑暗榮耀》和《金祭之國》我選了日劇.
韓劇我不是很感激.
所以我都是選日劇看.
希望十八歲以上才看《金祭之國》.
《山下至九》是很特別的演出.
不要說太多了.
知道的人就知道說什麼了.
主題是說Sorry.
這兩個月主題是說Sorry.
其實在說訊息之前要先說一點.
在教會裡,我自己覺得.
最大的情緒勒索就是.
要說Sorry和寬恕.
其實我們在現實環境裡.
很多人都得罪了自己.
你不知不覺間得罪了很多人.
你都不知道.
但其實有時候我們要說寬恕這個課題的時候.
其實有時候是很勉強的.
勉強的意思是.
其實他還沒寬恕到.
我們就會很快弄幾句金句出來.
如果你不寬恕,得罪了別人.
天父也不寬恕你了.

$^{41}$你就覺得很糟糕.
但我想表達的是.
一個還沒寬恕到別人的人.
其實心靈裡要對一些人的仇恨.
其實他心裡一點都不好過.
其實他都很想將.
他仇恨的人快點放下.
可以開一個新的一頁.
但我們很快聽到一些說話.
你快點寬恕吧.
因為你不寬恕的話.
天父也不寬恕你.
你會覺得原來上了天堂.
那快點寬恕吧.
這些說法是很不人性化的.
我估計在寬恕這個課題裡.
或者Sorry這個題目裡的時候.
其實第一樣要說的是.
不要太容易讓自己.
在不知道的情況下.
無端端去寬恕一些.
你還沒寬恕到的事情.
寬恕是一個過程.
寬恕是很真實的.
因為你受傷了.
受傷的過程裡.
你要自己療好傷.
能夠將那個人放下.
或者將事情放下.
那個從來不是一時三刻.
變魔術有一個Magic Wand.
就可以變身走得了.
坦白說Full Church頂梓梅.
或者不要說Full Church.
很多沒有回到四共祥教會的頂梓梅.
在外面街外的頂梓梅.
其實不少受了四共祥教會裡.
一些傷害.
我所接觸的頂梓梅.
或者我所聽到的.

$^{81}$或者我所經歷的.
傷害是真實的.
傷害不是一時三刻.
改變到的東西.
所以如果要說這個課題的時候.
我第一個前言.
在說聖經之前.
不要太輕易地.
一句金句來到.
我就可以馬上做得到.
金句做不到是很正常的.
沒有人一定有金句來到.
你就完全submissive.
然後我投降.
我可以的.
但那個不是一下子搖旋完之後.
就突然間所有東西都忘記了.
認真面對這個傷害的過程.
得著醫治.
得著釋放.
從來過程才是最珍貴的地方.
我們今天要說《撒加利亞書》.
因為我今天看不到powerpoint.
我要這樣看.
我們下一章.
我們今天看《撒加利亞書》第五章.
一至十一節.
這段經文有《非人的書卷》.
對不對?.
什麼?.
下一章是.
對嗎?.
我不是很抵到.
我要很辛苦.
我想《金字之國》是因為《非人的書卷》.
你知道《金字之國》第二季.
season 2的時候.
就有葵扇king.
紅心queen那些.
有看《金字之國》的人就知道我說什麼.

$^{121}$沒有就不說了.
他這裡也說《非人的書卷》.
你看看經文.
是這麼說的.
他說觀看漢建一個《非人的書卷》.
他問我看到什麼.
我回答漢建一個《非人的書卷》.
長二十爪.
闊十爪.
二十爪如果一個帝卿高.
六尺的話.
大約五個帝卿.
大約二十爪就是三十尺.
如果一個帝卿是六尺的話.
五個這麼高的六尺的帝卿這麼高.
闊十爪等於二十尺.
二十尺等於六尺的話.
大約三個多的六尺的帝卿.
你的想像是這樣.
不要說我了.
我五尺多而已.
他對我說.
這個就是全地面發出的咒詛.
他說凡偷竊的要按這個書卷除滅.
凡在假勢的要按這個書卷除滅.
下一個powerpoint.
麻煩你.
萬鈞儀說我要將這個書卷送去.
進入那個偷竊者指著我的名字.
在假勢的人的家裡.
停留在那裡.
連房屋大石頭和木頭和石頭都毀滅了.
我們看看下一個.
我在網上找到飛行的書卷.
大約是這樣.
他沒有攜線.
King和紅心Queen.
沒有金魚之鶴.
但最奇怪的是.
這個飛行書卷有些特色的東西.

$^{161}$第一個書卷為什麼要飛行.
這是整個撒加利亞書中難解的地方.
撒加利亞書本身很難搞.
沒有人會說撒加利亞書除了.
非才能乃靠我的靈方能成事.
第四章除了金句之外.
其餘的東西我們都不知道撒加利亞書說什麼.
可能十多章說騎驢仔進入聖城.
好像耶穌一樣.
只有這兩段我們熟悉.
但飛行書卷是想說什麼呢.
為什麼書卷要飛行呢.
這個意象其實很奇怪.
就像上個月所說的.
撒加利亞書有八個意象.
上課的訊息和這個意象是有關連的.
不過今天我們都不說了.
今天只說飛行.
飛行這個特別的地方是.
我們很多時候的信仰.
是在思夢場的教會.
就好像以色列人.
由大衛的時代開始.
或者摩西的時代開始.
進入了加南地之後.
大衛和很多王朝之後.
書卷裡面表達的律法.
只在加南地.
而奇怪的地方.
撒加利亞書是什麼時候的呢.
是秘魯後的書卷.
秘魯後的書卷.
大約是530年或500年附近.
即是主前.
大約主前586年秘魯.
或者605年秘魯.
看有多少次的秘魯.
所以由以前.
律法只在耶路撒冷或猶太傳遞.
到撒加利亞書的年代.

$^{201}$這個書卷講的律法.
即耶和華的華語.
不再只是停留在耶路撒冷或猶太這個地方.
因為那時候被巴比倫滅了.
滅完之後.
為何撒加利亞書會出現呢.
是因為巴比倫滅了.
西帝教王朝之後.
這個巴比倫不久後.
都被波斯王和古列王滅了.
而這個滅了的故事.
即巴比倫終於可以.
即波斯王可以借得低島.
巴比倫王.
這個故事成為猶太人覺得.
萬鈞耶和華的律法.
不再停留在猶太地或耶路撒冷.
所以這個飛行書卷.
是在全面上飛行.
即所有地面上都飛行.
最簡單的意思.
飛行書卷想講甚麼.
就是律法不再地位在一個迦南地上.
從此以後猶太人有一個不同的想法.
原來我被巴比倫王滅了之後.
聖經預言七十年後.
我會回到耶路撒冷.
為何會回到.
因為這個飛行書卷帶來審判.
帶來上帝的話語威力.
連巴比倫王好像大尼書.
那隻手指.
我不看那些經文.
意思是你今天國就滅了.
滅了之後.
明天就波斯王來到.
取締了整個巴比倫的國度.
所以飛行書卷的意思是.
讓猶太人對萬鈞耶和華的律法.
不要再停留在猶太地才能夠成功.

$^{241}$飛行書卷是在講.
原來萬鈞耶和華的律法.
在遍地裡游來游去.
以至到列國都要服在萬鈞耶和華的律法下.
這是飛行書卷一個很重要的意思.
如果最簡單講的話.
我已經看不到.
在寫什麼呢?.
我真的看不到.
原來講到這張了.
PowerPoint哥哥靠你們了.
我都不知道自己在講什麼.
我真的看不到.
靠你們了.
你撕到哪一張就撕哪一張.
所以整個飛行的律法.
這個書卷的觀念是什麼呢?.
讓我們離開了.
我們原有對萬鈞耶和華的想法.
到今時今日.
例如Fold Church的弟兄姊妹.
無論世界各地.
在不同的弟兄姊妹.
流散了不同地方的弟兄姊妹.
今天John應該在英國.
應該和Fold Church在英國倫敦的弟兄姊妹相遇.
他們應該在這裡.
原來教會的觀念不再只是我們以前.
有一個宗派式的觀念.
起碼Fold Church突破了一個地方.
我們沒有宗派的包袱.
沒有宗派的觀念.
甚至去到英國.
去到澳洲.
去到北美不同的地方.
開始了一個我們從來沒有想過的觀念.
這個成為了我們在一個新的時代.
或者一個迫於無奈的新時代.
我們有些不同的東西出現.
就好像猶太人被迫回歸耶路撒冷的時候.

$^{281}$他們問的問題.
原來萬鈞耶和的律法不僅僅是在猶太地.
原來萬鈞耶和的律法是在世界各地.
就算巴比倫的滅亡.
都屬於萬鈞耶和的律法審判下.
以至到波斯王興起.
然後波斯王容許以色列人.
回到耶路撒冷重建聖殿.
撒加利亞書要表明一件很重要的事情.
經歷了苦難,困難,被亡國,被擄之後.
原來萬鈞耶和說有新的東西在做.
而這個做法是開啟了猶太人對整個律法.
可以在世界各地有一個新的觀念.
如果這樣說的話.
我的問題要馬上問的是.
Fold Church的弟兄姊妹有什麼是新的?.
我馬上要問的是.
除了我們經常到不同的地方.
因為我們經常租不到地方.
不是穩定地租到一個地方.
所以今天我們來深崇.
除了我們會很Fold之外.
有什麼是我們新的東西?.
我試試這樣去描繪這幅圖畫.
我們經歷了一些事情.
所以我們會去到一個新的角度.
大家明白嗎?.
我們遇到一些事情.
所以我們會去到一個新的角度.
這個新的角度好像Fold Church.
就是這個群體.
情況像什麼呢?.
像金祭之國.
我們來看看金祭之國.
有一張Powerpoint是說它的.
大家認不認識?.
如果你有看的話.
那是岑珈其在扮的那個.
再看下一張.
我沒有騙大家.

$^{321}$下一張應該是葵線King.
葵線King是傻子.
不停地射殺那個.
大家不要看了.
很殘忍的.
每個人都射爆頭.
好,我們再看下一張.
你看看就算是當真.
最近他拍的遊戲節目.
全部抄金祭之國.
所以岑珈其,彤彤那些.
最近還有陳柏宇.
再加上Marv去拍.
他玩的遊戲全部都抄金祭之國的遊戲.
為什麼我要說一下金祭之國呢?.
我不會劇透的.
雖然我經常都會.
但是不會.
金祭之國是什麼呢?.
是一個很無聊的自閉.
不是自閉.
就是沒有什麼事情做的男生.
在現實世界裡沒有什麼事情做.
經常被弟弟,爸爸小看.
突然間到了一個新的角度.
他成為了一個很英雄式的人物.
女生就是這樣.
不過你不要問為什麼突然去到新的角度.
這是劇透的.
不要看了.
第二季才知道.
最後才知道.
但是很奇怪的是什麼呢?.
整套金祭之國想表達的東西就是.
他們很討厭在日本.
懷疑是日本的文化.
其實有很多很虛假的東西.
在金祭之國裡玩遊戲.
你做好人的話.
其他人就會懷疑你是一個偽善的人.

$^{361}$偽善這個字在金祭之國用得最多.
在我的不文明的統計下.
經常聽到你做好事.
大家就會說你是一個很偽善的人.
他想抱怨什麼呢?.
原來在我們現實社會裡.
我們都知道要道歉.
要知道做好事.
要知道什麼.
但其實你和我哪一骨子都知道.
猜對方是否真的這樣想.
我們說對不起.
其實對方是否真的對不起.
不用說.
小朋友一要他道歉.
他一定會說「對不起」.
一道歉.
另一個人就會聽到.
什麼叫「對不起」.
你是不是真心對不起.
我們現實世界裡有很多傷害.
給予傷害的東西.
金祭之國就是在說.
現實是這樣的.
讓你突然在一個新的角度裡.
玩遊戲.
玩完遊戲之後.
就試到你是否一個真誠的人.
是否一個真心真意的人.
還是一個純粹為自己利益的人.
金祭之國就好像在玩一個遊戲.
最有名的遊戲就是.
童童的遊戲.
如果你知道小薯茄的童童.
有份去玩那個遊戲.
和沈家琪一起玩遊戲.
你要箍著鏡箍後面.
有一個攜線紅心梅花街轉.
就是一群人一起玩.
你就要靠別人告訴你.

$^{401}$你後面那個是什麼花.
攜線紅心梅花街轉.
你看不到的.
大家就告訴你.
你相信他的話.
你就去他的房間.
你說攜線.
如果你真的攜線.
就沒事出來玩.
如果別人騙你.
你相信.
其實你是紅心的.
你不去攜線.
就好像無緣無故的死光槍.
就死了.
如果你看這個遊戲的話.
你會發覺很殘忍.
和你同盟的人.
信得多認真都好.
去到後來.
都是用鬼咋地對你.
金祭之國有一個叫海濱的地方.
海濱的地方是什麼.
他們玩遊戲之前.
大約有三天.
玩完遊戲通過了三天的.
不用再玩的.
那三天他們做什麼.
就是天天派對.
你以為在一個新的角度裡.
那些東西洗過牌再來過.
我們人性上會好嗎.
金祭之國就告訴你.
換了一個新的角度裡.
是不是所有人就好像突然之間.
將你裡面應該要有的東西.
出來還是.
將裡面邪惡的東西.
不斷地出來呢.
撒加利亞書說的東西是一樣的.

$^{441}$撒加利亞書說的.
你以為巴比倫被滅.
波斯來到.
我們能夠回到新的角度裡.
從此好嗎.
你看下一個經文吧.
很無聊的.
再看下一張.
下一張.
我們不要看這個了.
下一張.
看第五至十一節吧.
我看不到的.
我讀一讀.
他第五至十一節說什麼.
飛行的意象的孫子說了.
他說天使說這是罪惡.
他將婦人推進涼氣裡.
把圓蓋壓在涼氣的口上.
於是他叫木棺漢漢.
有兩個婦人前來.
他們的翅膀裡有風.
如同官鳥的翅膀.
他們將涼氣抬起來.
沿在天空之間.
意思是什麼呢.
很簡單.
有一個女人是有罪的.
是邪惡的.
他將這個女人.
即是邪惡.
放在涼氣裡.
把她整個撐起.
飛到天地之間.
跟飛行書卷一樣.
不過這次換成一個邪惡的女人.
坐在涼氣上.
飛來飛去.
看完下一個.
第十十一節.

$^{481}$我問那個跟我說的天使.
你要將這個涼氣.
即是這個邪惡的女人.
抬在哪裡.
他說要抬到士拿地去.
為她建造房屋.
等她預備妥當.
就將她安置在自己的桌上.
你看這段經文.
士拿這個地方是什麼地方.
士拿這個地方是巴比倫.
意思是巴比倫滅了猶太人.
然後他.
巴比倫國.
當時你也知道.
巴比倫將很多偶像.
放在聖殿的供奉敬拜.
整個以色列地理充滿著.
巴比倫偶像文化的敬拜.
所以在那個時候.
當耶路撒冷再次有人回歸的時候.
那裡也充滿著很多巴比倫偶像的敬拜.
所以要建聖殿要建城牆的時候要怎樣.
就將那些偶像的東西.
就好像一個女人代表邪惡.
先不要說歧視的問題.
你罵聖經歧視我也不知道怎樣回答.
總之找一個女人代表邪惡.
他說拜託巴比倫的邪惡.
現在就要有一個圖畫.
就是他要離開耶路撒冷.
回到士拿那裡.
意思是請耶路撒冷裡面.
當人們回歸的時候.
這是一個新的角度.
不要將巴比倫人那幾十年的邪惡和罪惡.
留在這個地方.
好像一個涼氣一樣回到原本的地方.
剛才說金祭之國.
你以為有一個新的角度.

$^{521}$你以為有一個新的上帝容許的事情發生.
就是古列王 波斯王容許人回來.
打敗了巴比倫 巴比倫終於毀滅.
你以為這個角度之後.
人性就好了嗎.
很簡單說.
哈蓋書 撒加利亞書 馬拉記書.
這三卷是秘魯回歸後的書卷.
這三卷書卷你看看.
很短的 最長的撒加利亞書.
罵以色列人罵到不能罵到.
你可想像.
好像跟以塞亞 耶利米和以色列的罵的一樣.
雖然這裡有一個象徵性的形象.
就是離開了.
但其實邪惡和罪惡仍然在那裡.
我們說它也會受過很多傷害.
我們說它會人生很多不如意的事.
我們會經歷很多覺得很淒涼很難過的事.
很不容易的經歷.
有時候我們很不容易地離開了那些感覺和情緒.
去到一個新的地方.
我們希望在新的地方會有不同.
但其實這全部都像是我們想像的一幅圖畫.
現實並非如此.
上個月說我妹夫突然去世.
這個月做完國安式禮拜.
這個月對於我們家裡來說.
最大的不同是什麼?.
是我們整家人.
兩個小朋友和我和我太太.
說了很多生死教育的課題.
死去是怎樣的?.
死後剩下的人會怎樣?.
他會過什麼樣的感覺?.
沒有一個人在身邊.
這個家會變成怎樣?.
傷痕什麼時候會完結?.
不捨得什麼時候會停下來?.
記得國安式禮拜那天.

$^{561}$我兒子本來不去的.
他說他很害怕.
他很害怕不去.
我說我不去.
我怎樣都不去.
我談了這麼久還不去.
很難過.
不知為何神感動他.
星期五國安式禮拜.
星期四他突然說好,我去.
我說好,不過大念的時候.
我要負責大念.
我說你不要走過來.
大念就是屍體.
我說你不要走過來.
你去外面.
大念的時候.
他不知從後面走到我前面.
看著我.
看了很久.
我做完大念立即捉住他.
不要再看了,走了.
對於他來說是一個新的體見.
我這個小朋友.
他一去到那個地方的時候.
看著我每個兒子的時候.
他不可以走過來哭.
哭了很久.
我說你哭什麼?.
他說我看到他的表弟.
沒有了爸爸,他很想哭.
我們人生會遇到一些事情.
令我們有重新定位.
會有些事情不同了.
經歷了苦難,離別,傷害之後.
我們人生會突然有些事情.
重新定位了.
起碼我們兩個小朋友.
會願意花更多時間陪表弟玩.
我們也會花時間陪表弟一起經歷很多事情.

$^{601}$我發覺我最近不懂得跟表弟說話.
你跟他說你好嗎?.
是廢話.
你過得怎樣?.
也是廢話.
你最近好嗎?.
也是廢話.
你今天睡得好嗎?.
也是廢話.
你發現連說一句話的能力都沒有.
你連說一句能夠安慰的話都找不到.
重新定位在心裡.
我們開始有些事情會覺得想不一樣.
想期望有些事情改變.
這是很真實的.
就算你不願意也好.
你也被迫進入心裡.
心裡有些事情正在重新定位改變.
我最近發現.
就算是這樣也好.
其實我們內裡也有很恐怖的事情.
前一星期六.
我家裡的女兒突然走來.
早上哭了20分鐘.
她哭什麼呢?.
經歷了那些事情後.
她問我爸爸.
你會不會突然在家裡睡一覺.
走出來就死了?.
她哭著問我.
那一刻你猜我回答什麼呢?.
不會.
最慘的是那星期.
我經常都要晚上起床.
心裡很痛.
你可以想像那些心理病的東西都走出來.
我心想死了我會不會呢?.
接著我想回答.
回答她更神奇的是.
其實生命在上帝手裡.

$^{641}$又不可以這樣回答.
你說可以的.
如果可以的話.
上帝真的很慘.
我心裡想可以的.
她哭也好.
我也沒有回答她.
我回答不了.
我才做完.
那星期就有盼望.
天家會再見.
我們會說這些.
我們曾歸曾,土歸土.
天家會再重遇.
你要給一個盼望的訊息.
那一刻我跟女兒說.
爸爸也會走的.
曾歸曾,土歸土.
我自己做了火葬禮的時刻.
天家再見了,女兒.
這還不是最難的.
最難的是之前晚上.
老婆跟我睡覺的時候.
她跟我說.
明天女兒要看醫生.
我問她看什麼.
她說要看醫生.
已經說了兩三個星期.
因為她背上長了一顆東西.
凸了出來.
我的朋友的醫生跟我說.
突然催我老婆.
你還不看?.
快點來看吧.
我老婆說.
我們這個朋友醫生.
從來都沒什麼事.
就不會叫我們來看她.
她叫來一定是大事.
剛好之前的一二月.

$^{681}$我們認識了一對夫婦.
有個兒子.
比我女兒大一歲.
他背上又剛好長了一顆東西.
凸了出來.
結果正在做化療.
那一晚.
有了這些背景之後.
那一晚我老婆就很不長進.
她說看你怎麼辦.
你最疼愛的那個.
有事你就不知道怎麼辦.
我心裡想豈有此理.
不是你最疼愛的那個嗎.
豈有此理.
我最疼愛的那個.
那天是星期五還是六.
她就問我什麼時候死.
我就看著她背上的那顆東西.
她就哭著說.
爸爸如果你死了我怎麼辦.
我就不懂回答她.
其實陳居正也很肚餓.
你可以想像.
就算有重新轉變.
我們也不容易走回應該要走的路.
就算有.
全聖餐有什麼特色.
你明白全聖餐有什麼特色嗎.
回到最根本的問題.
全聖餐會不會回到以前一樣.
我意思是以前.
我們以前的東西一樣.
什麼東西在定義全聖餐.
如果我們曾經有傷害經歷的人.
在教會的環境裡.
對人不是很多信任.
對人頂智的愛.
那些.
你會懷疑很多.

$^{721}$想療傷的過程.
在呼出什麼不同感.
這是我們很值得問的問題.
就像我們會想像.
我們以往的東西.
在這個地方新開始.
應該以前的東西.
突然間在這裡消失.
我要說金祭之國也好.
或是說《撒加利亞書》也好.
我們不需要這個幻想.
是不是倒灰塵.
我們不要有這個幻想.
反而我們接受.
我們裡面很多不行的東西.
金祭之國在新的角度裡.
有很多人照舊.
但有些人誠實的是什麼.
他覺得他可以重新開始.
就像主角一樣.
他只是一個玩遊戲機的男孩.
在現實世界裡.
但誰知道到了另一個角度.
他好像是一個不一樣的人.
有些人給多一次空間,機遇.
他遇到不同的事情的時候.
他可以堅持有新的東西走出來.
這是define and fold church.
沒錯,我們不會將很完美,很美善.
突然性格消失.
突然變得溫文爾雅.
整個主耶穌出現.
我們不會幻想這些東西.
我們仍然帶著很多很奇怪的東西出現.
但define and fold church好一點的是什麼.
我們不一定要去海濱那裡天天派對.
或者我們不需要用以往傷害過我的東西.
拿來試試可不可以傷害別人.
你問我.
在被傷害過程裡最大的挑戰是什麼.

$^{761}$如果像那些人那樣傷害別人也挺爽快的.
好像做一些自己傷害了很久的事.
屈居很久的事.
能夠有空間在這裡發達出來就好了.
其實也可以的.
我猜猜不知道可不可以.
這些牧者是否能夠相信這個問題.
但除了這些之外.
我們可不可以有多一點東西是我們不同的.
好像那個女人的邪惡.
離開了我們.
離開我們很象徵性.
所謂不一定全部離開.
但起碼有一點東西在我心中.
拿走了.
將多一點點的美善放在Fold Church群體裡.
如果要說對不起的話.
其實在基督教圈子裡.
我們很少聽到別人說對不起.
我們好像是一個經常要說寬恕.
因為耶穌基督寬恕我們.
所以我們經常要寬恕別人.
我們很少是一個認錯的群體.
我們很少是一個.
好吧,我放下自己.
我先說對不起.
Fold Church五年了.
頭幾年很興奮很開心.
你參加小組,那個人很新鮮.
那個人跟你不認識,你想認識一下他.
認識了三四年之後你發覺.
不是和以前的頂尖一樣嗎?.
(笑).
有分別嗎?.
我想有分別.
我想多分別的是.
沒錯,我其實也沒有分別.
但在新的角度裡.
多給自己一點點不同了的東西.
我們可以用傷害對待別人.

$^{801}$去海濱叫Party開心.
Yeah.
但會不會在這個時候.
我們有些新的東西.
可以在這個群體裡出現.
是不同的呢?.
待會聖餐的時候.
當我們說起教會裡最重要的.
群體相處的時候.
告訴我們這裡的群體.
我可以比以往的東西.
有一點點不一樣.
而那些一點點不一樣的東西.
就是在Define Fold Church不同的地方.
我們一起低頭祈禱.
Tiff,謝謝你給我們今天空間和時間.
我們來到你面前的時候.
我們求天賦你自己.
親自用你每一個同在.
因為你知道我們的軟弱.
我們的艱難.
我們不是要很快.
要有些不同.
但天賦這個不同.
在我們心靈裡也很重要.
我求天賦你讓我們拿捏到.
這兩者之間的掙扎和矛盾.
讓我們學會靠著你的恩典.
在這些位置裡能夠走下去.
主聖靈,求你親自為Fold Church.
Define這個群體.
是聖靈親自Define這個群體.
是如何走下去的.
求你親自帶領我們.
幫助我們.
天賦你聽我們在你面前的祈禱.
逢日蘇基督也保貴命球.
Amen.
\newpage



\section{}
\label{sec:7ZGXT0f30Z0}
\textbf{《致餘民及流散者:給香港基督徒的神學八課》第二季第2課|20230325 [7ZGXT0f30Z0]}
\newline
\newline
連結: \href{https://youtube.com/watch?v=7ZGXT0f30Z0}{\texttt{ https://youtube.com/watch?v=7ZGXT0f30Z0}} ~~~~ 語音日期: 2023-03-25 
\newline
\newline
\hyperref[sec:y5NJfoAjRCI]{\small{< < < PREV SERMON < < <}}
~
\hyperref[sec:index_chronic]{\small{[返順時目]}}
~
\hyperref[sec:index_scriptual]{\small{[返順卷目]}}
~
\hyperref[sec:8KdYgVn_hzk]{\small{> > > NEXT SERMON > > >}}
\newline
\newline
$^{1}$我只想知道.
你到底是什麼意思.
我只想知道.
你到底是什麼意思.
我只想知道.
你到底是什麼意思.
我只想知道.
你到底是什麼意思.
我只想知道.
你到底是什麼意思.
我只想知道.
你到底是什麼意思.
我只想知道.
你到底是什麼意思.
我只想知道.
你到底是什麼意思.
我只想知道.
你到底是什麼意思.
我只想知道.
你到底是什麼意思.
我只想知道.
你到底是什麼意思.
我只想知道.
你到底是什麼意思.
我只想知道.
你到底是什麼意思.
我只想知道.
你到底是什麼意思.
我只想知道.
你到底是什麼意思.
我只想知道.
你到底是什麼意思.
我只想知道.
你到底是什麼意思.
我只想知道.
你到底是什麼意思.
我只想知道.
你到底是什麼意思.
我只想知道.
你到底是什麼意思.

$^{41}$我只想知道.
你到底是什麼意思.
我只想知道.
你到底是什麼意思.
我只想知道.
你到底是什麼意思.
我只想知道.
你到底是什麼意思.
我只想知道.
你到底是什麼意思.
我只想知道.
你到底是什麼意思.
我只想知道.
你到底是什麼意思.
我只想知道.
你到底是什麼意思.
我只想知道.
你到底是什麼意思.
我只想知道.
你到底是什麼意思.
我只想知道.
你到底是什麼意思.
我只想知道.
你到底是什麼意思.
我只想知道.
你到底是什麼意思.
我只想知道.
你到底是什麼意思.
我只想知道.
你到底是什麼意思.
我只想知道.
你到底是什麼意思.
我只想知道.
你到底是什麼意思.
我只想知道.
你到底是什麼意思.
我只想知道.
你到底是什麼意思.
我只想知道.
你到底是什麼意思.

$^{81}$我只想知道.
你到底是什麼意思.
我只想知道.
你到底是什麼意思.
我只想知道.
你到底是什麼意思.
我只想知道.
你到底是什麼意思.
我只想知道.
你到底是什麼意思.
我只想知道.
你到底是什麼意思.
我只想知道.
你到底是什麼意思.
我只想知道.
你到底是什麼意思.
我只想知道.
你到底是什麼意思.
我只想知道.
你到底是什麼意思.
我只想知道.
你到底是什麼意思.
我只想知道.
你到底是什麼意思.
我只想知道.
你到底是什麼意思.
我只想知道.
你到底是什麼意思.
我只想知道.
你到底是什麼意思.
我只想知道.
你到底是什麼意思.
我只想知道.
你到底是什麼意思.
我只想知道.
你到底是什麼意思.
我只想知道.
你到底是什麼意思.
我只想知道.
你到底是什麼意思.

$^{121}$我只想知道.
你到底是什麼意思.
我只想知道.
你到底是什麼意思.
我只想知道.
你到底是什麼意思.
我只想知道.
你到底是什麼意思.
我只想知道.
你到底是什麼意思.
我只想知道.
你到底是什麼意思.
我只想知道.
你到底是什麼意思.
我只想知道.
你到底是什麼意思.
我只想知道.
你到底是什麼意思.
我只想知道.
你到底是什麼意思.
我只想知道.
你到底是什麼意思.
我只想知道.
你到底是什麼意思.
我只想知道.
你到底是什麼意思.
我只想知道.
你到底是什麼意思.
我只想知道.
你到底是什麼意思.
我只想知道.
你到底是什麼意思.
我只想知道.
你到底是什麼意思.
我只想知道.
你到底是什麼意思.
我只想知道.
你到底是什麼意思.
我只想知道.
你到底是什麼意思.

$^{161}$我只想知道.
你到底是什麼意思.
我只想知道.
你到底是什麼意思.
我只想知道.
你到底是什麼意思.
我只想知道.
你到底是什麼意思.
我只想知道.
你到底是什麼意思.
我只想知道.
你到底是什麼意思.
我只想知道.
你到底是什麼意思.
我只想知道.
你到底是什麼意思.
我只想知道.
你到底是什麼意思.
我只想知道.
你到底是什麼意思.
我只想知道.
你到底是什麼意思.
我只想知道.
你到底是什麼意思.
我只想知道.
你到底是什麼意思.
我只想知道.
你到底是什麼意思.
我只想知道.
你到底是什麼意思.
我只想知道.
你到底是什麼意思.
我只想知道.
你到底是什麼意思.
我只想知道.
你到底是什麼意思.
我只想知道.
你到底是什麼意思.
我只想知道.
你到底是什麼意思.

$^{201}$我只想知道.
你到底是什麼意思.
我只想知道.
你到底是什麼意思.
我只想知道.
你到底是什麼意思.
我只想知道.
你到底是什麼意思.
我只想知道.
你到底是什麼意思.
我只想知道.
你到底是什麼意思.
我只想知道.
你到底是什麼意思.
我只想知道.
你到底是什麼意思.
我只想知道.
你到底是什麼意思.
我只想知道.
你到底是什麼意思.
我只想知道.
你到底是什麼意思.
我只想知道.
你到底是什麼意思.
我只想知道.
你到底是什麼意思.
我只想知道.
你到底是什麼意思.
我只想知道.
你到底是什麼意思.
我只想知道.
你到底是什麼意思.
我只想知道.
你到底是什麼意思.
我只想知道.
你到底是什麼意思.
我只想知道.
你到底是什麼意思.
我只想知道.
你到底是什麼意思.

$^{241}$我只想知道.
你到底是什麼意思.
我只想知道.
你到底是什麼意思.
我只想知道.
你到底是什麼意思.
我只想知道.
你到底是什麼意思.
我只想知道.
你到底是什麼意思.
我只想知道.
你到底是什麼意思.
我只想知道.
你到底是什麼意思.
我只想知道.
你到底是什麼意思.
我只想知道.
你到底是什麼意思.
我只想知道.
你到底是什麼意思.
我只想知道.
今天我們嘗試去做一個舊的題目.
但是我們用一個新的角度去理解.
特別是能夠回應我們時代的一個向度裡面.
所以我們今天就會去探討.
講一些我自己信主的時候的一些見證.
那時候我是十八歲信耶穌.
那時候我很記得我信耶穌之前.
我經常都有一個習慣.
就是我想重新再開始.
我是從五年級開始信耶穌.
那時候很明顯很記得這句話.
有人做基督徒.
他就是新來的人.
舊事已過都變成新的了.
這是每一個信主的人.
信主的時候很重要的經文.
周培的人跟你說.
你現在是信耶穌了.
基督裡面.

$^{281}$你有一個新的生命.
你有新的開始.
那時候我有一個習慣.
就是每次我很不開心的時候.
我就會去洗澡.
不知道為什麼.
你是不是很習慣.
洗完澡之後.
我重新很清新的感覺.
我好像重新開始一樣.
就是一個很舒服.
洗完頭洗完身都乾透了.
有一個很新的開始的感覺.
但發現原來是不行的.
那時候很大的感覺就是.
當我去信耶穌之後.
我就是一個重新的開始.
我是一個罪人.
我以前是一個很不開心的人.
一個失喪的人.
我現在在基督裡面.
又重新開始.
這是我自己在很多年前.
信主的時候一個很大的震撼.
就是我知道在基督裡面.
我重新再開始.
但不知道這個是不是你的感覺.
當你信主這麼久的時候.
你是不是一個新做的人呢.
可能你信主十幾年.
信主二十年.
二十年前剛剛決志的你.
和今天的你.
其實是不是都這麼新呢.
如果二十年前的新做的人.
今天是不是都已經舊了呢.
這個其實我今天很有興趣問這個問題.
什麼叫做新做的人.
什麼叫做一個重新開始的人.
當我們想主這麼久.

$^{321}$當我們面對著自己的信仰.
這麼多年在教會裡面經驗的時候.
我們怎麼來保持.
或者去認識我們在聖經裡面的一個英雄.
就是那個《後書聯經文》所說的.
一個新做的人.
所以今天我們就會思考一個.
很簡單的課題.
都是一個大家聽過的課題.
就是有關新做的人這個字.
如果我們去看回聖經的時候.
你發現聖經裡面這個新做的人.
其實你發現希臘文裡面.
就是Kanei,就是Christie.
這個字就是解作新的創造.
所以如果你是認真看經文的時候.
那個原文裡面其實就不是.
正確地和本所說的新做的人.
我發覺某一段的中文聖經.
是比較貼切的.
因為基本上是沒有人字在裡面.
有個新字.
和這個Christie這個字是解作創造.
就是Creation.
所以如果你是嚴格去認識這個創造.
新做的人的話.
其實更好的翻譯是新的創造.
它叫做New Creation.
所以NIV的版本其實翻譯得挺好的.
所以如果我們去校正方向.
用New Creation來理解經文.
或者我們的身份的時候.
今天我們就會發覺一個.
看得很多的東西.
更加闊的概念.
來理解我們這個新做的人的狀況.
我們就會這樣.
說的是一方面面對一個新環境.
我們面對的香港.
或者是海外裡面移民之後.

$^{361}$面對一個新生活.
這個也是和新字有關係的.
所以我們所謂的新生活.
就是用一個新做人.
New Creation來重新思考這個課題.
如果你去看保羅的神學的時候.
特別是這個學者.
Hans Dezsibis.
這個我很喜歡的一個.
美國籍的德裔的聖經學者.
他很簡單地概括了保羅.
在特別是加拿大書裡面.
或者是整個神裡面.
一個很重要的拯救的四個步驟.
很明顯第一個.
Death and Resurrection of Christ.
基督耶穌的死和復活.
這個是整件事情的第一個步驟.
第二.
第二就是.
Putting on of and dying and rising with Christ in baptism.
這裡所說的是披帶基督的意思.
就是剛才所說的.
我們和基督同死同復活.
這個是使用洗禮.
來重新去carry住基督耶穌的生命.
死過再活過來.
是這樣的開始.
然後第三.
就是有關聖靈的尼林.
The gift of the spirit of God.
就是聖靈的尼林.
第四個步驟.
就是一個更加長遠的.
就是Living in a new creation.
這個也是大家和耶穌的關係裡面.
或者有關德教的秩序.
我們如何去理解.
我們在保羅的世界裡面所說的.
德教的秩序.

$^{401}$有什麼步驟會高呼過他呢.
在整個德教的事件上.
很明顯第四個.
一個是非常長遠的.
從你決志之後.
洗禮之後.
都是這樣的生活.
就是Living in a new creation.
你可以叫做上帝的言論的新生命.
所以那本書是很對的.
這個新生命.
New life.
新的生命.
新生命.
不僅是重生.
不僅是在基督裡面.
有一個重生的New creation.
更加是你開展一個新的生命.
生活.
所以Live的字是解作生活.
生活也可以.
所以不僅是擁有一個新的生命.
更加是要按著這個新的生命.
重新來生活.
這個就是我想大概.
在那本綠色的書裡面.
那八個來源.
就是要去.
決志.
奉獻.
有團契.
敬拜之類的.
今天我們.
當我們新屬二十年之後.
我們很熟悉那些東西.
初順.
成人百貨.
新生命新生活那些東西.
我們都很熟悉.
我們如何去理解.

$^{441}$當下我們的新生命.
新生活呢.
特別是面對著我們.
很contextualize的處境.
今天我們香港.
或者你離開了香港之後.
一個新的環境裡面.
其實是有關係的.
所以當我們去.
看得出.
這樣來理解這個新生命的時候.
你就會發覺.
在我們的信仰當中.
是一個這樣的角色.
有關這個新族的人.
或者這個新生命經文.
其實有兩段.
達比蘇奇這段.
是否說不出後書的第五章經文.
有人在基督里.
他就是新造的人.
舊事已過都變成新的了.
一切都是出於上帝.
他借著基督使我們與他和好.
又將勸人與他和好的積分.
遲及我們.
這就是上帝在基督里.
叫世人與自己和好.
不將他們的過犯歸到他們身上.
並且將自己和好的道理托付了我們.
這些大家很熟悉的經文.
在基督裡面的人.
就是新造的人.
舊事已過了.
都變成新的了.
這句話可能在大家.
在《說人書》之前.
是一個非常大的震撼.
很多以前不好的事情.
未信主之前舊的事情.

$^{481}$遺憾事.
很多犯罪的事.
都能夠移過一個新的事情.
奈何當你信主之後.
發覺很多新的事.
都變成一些舊的事.
你曾經.
在教會很多很雀躍的地方.
一個很新的體驗.
靈命.
信仰.
崇拜.
很多都是一些新的東西.
但發覺回到20年.
10年之後.
很多新的東西.
就再也不是新的.
我們會探討這個課題.
究竟這個新字是什麼意思呢.
這是大家比較熟悉的經文.
其實有一段的.
大家知不知道.
另外一段有關新造的.
有一段經文.
就是在這個.
加泰書的經文.
加泰書第六章.
最後一章的經文.
但我不斷以別的誇口.
只誇我們耶穌基督的十字架.
因這十字架就我而論.
世界已經釘在十字架上.
就世界而論.
我已經釘在十字架上.
受過禮不受過禮都無關要緊.
要的事就是作新造的人.
可能大家不是很記得這些經文.
只記得之前的一些.
世界被釘在十字架上.
這些經文.

$^{521}$後面保羅做了一個總結.
一個application.
God like不God like不重要.
重要的是你要做一個新造的人.
你需要是一個.
God和不God無關.
But new creation.
這個就是保羅在初期裡面.
提出new creation這個概念.
一個這樣的經文.
所以我今天就會去.
詳細地去思考.
什麼叫做new creation.
new creation對我們今天.
面對著自己的屬靈生命.
或者作為full church群體.
我們面對著今天香港.
或者我們海外.
開展新生活.
我們有什麼意義呢.
好了,我們就來看看.
不過如果你去.
認真地去想.
new creation.
其實就不是那個新造的人.
不是完全一樣.
new creation是一個更加.
廣闊一點的神的概念.
他在說的是上帝新的創造.
當你說到創造的時候.
他就不是純粹在說.
你重新做人.
所以新造的人不是.
純粹告訴你.
你重新做人.
這個對的.
這個不單止是這樣.
上帝的救贖工作.
其實是一個更加宏觀的工作.
他不單單是全新的你.

$^{561}$重新來過.
這樣開始做人.
而是說整個創造的重新開始.
所以這個字你會想起什麼呢.
就想起創世紀.
創世紀裡面有一個creation.
創造.
六天的創造.
而去到新約的時候.
在基督裡面.
有一個new creation.
所以你會發覺.
是這樣的對比.
所以新造的人.
不是重新做人.
沒錯.
你二十年前重新做人.
你出了字.
做基督徒想做人.
但是所牽涉的.
不單單是你重新再做人.
再來過.
會講這麼簡單.
不是一個純粹個人層面的事情.
而是一個上帝對於這個世界.
的一個救贖工作.
所以我們今天就去看看.
new creation.
究竟我們用什麼的.
當詞去理解new creation呢.
其中一個就是這樣.
你發現.
就是這個.
它關乎於猶太的啟示文學.
讓我先弄弄這個.
好.
OK,不要弄它.
在這裡.
其實保羅.
當他在他的.

$^{601}$出版本裡面.
講new creation的時候.
其實是關於一些猶太.
猶太教.
在兩天之間.
或者在教育裡面.
所講的一些啟示文學的根據.
其實猶太教一直都有講new creation的概念.
最少在以前的書裡面都提過.
這是比較在正典裡面出現的經文.
我們來看一下.
他說看啊,我做生天生地.
從前的事不再被紀念.
也不再追想.
你們當因我所做的永遠歡喜快樂.
因為我做夜宵蘭為人所喜.
做其餘的居民為人所樂.
大家都聽過.
我當初都講過幾次.
新夜老撒冷.
新出埃及.
我要做一件新事.
開新的江河.
其實裡面都很多這些經文.
上帝要做一個新天新地.
一個new creation.
不再是舊約.
創世紀以前的creation.
而是.
耶穌說上帝會重新來更新這個世界.
重新來做一個新天新地.
所以是一些中末的味道.
有一些天啟文學的味道.
如果你再看多一些時候.
看多一些猶太教的小小的.
其他的經卷的時候.
你會發現.
譬如在這個以色列的十六章.
雖然在聖經裡面沒有.
但都是一些猶太教經典.

$^{641}$這裡就沒有中文翻譯.
他說.
We are done after death.
As soon as everyone of us.
yield up his soul.
We shall be kept in rest.
Until those times come.
When we renew the creation.
所以耶和臥上是會重新來更新這個創造.
或者在八六書裡面.
都說.
And a new world is coming.
所以當保羅說到新造人的時候.
其實不是說我們怎麼重新再來過這麼簡單.
而是說整個世界.
從舊約開始.
耶和臥上在以色列文所應許的.
他將要更新這個世界.
重新來創造一次.
所以這個我今天不說太深.
究竟是一個restoration.
還是一個重新再來過.
還是一個什麼的創造.
但發現當保羅說到new creation的時候.
其實不是純屬一個個人層面的事情.
是一個從舊約開始.
一直都有這樣的背景的事情.
所以當保羅說到new creation的時候.
其實羅馬書也是這樣說的.
當羅馬書說.
我想現在的苦楚.
要比起將來要顯得我們的榮耀.
就不足為價而了.
受造之物.
切忘等候上帝眾子的顯出來.
因為受造之物伏在胸之下.
得向上帝榮耀.
這個受造之物就是那個creation.
一切的creation.
將要等到上帝將來的臨臨.

$^{681}$那個榮耀的時候.
就會被更新.
所以無論是在哥林多後書第五章.
還是在加拉太書第六章.
還是在羅馬書第八章也好.
這三個問題告訴我們.
保羅所說的new creation.
一個更加宏觀.
一個更加有猶太教背景.
一個綜活裡面這樣的一種視野.
所以說了這麼久.
究竟這個有什麼關係呢.
我們經常說.
十字架不是一個個人得救的方法.
當基督在十字架上.
祂是叫整個世界得以改變.
這是一個更加完整的福音.
一個更加準確的美術福音.
得救的不是你自己.
以前怎麼不開心.
然後怎麼可以被拯救.
而是整個世界都借著基督十字架.
去被更新改變.
譬如這句經文.
叫我們知道他知的奧秘.
要照他所安排的.
在日期滿足的時候.
在天上地上.
一切所有的都在基督裡面同歸於一.
一切全部的都在基督耶穌裡面.
重新來到再被拯救.
如果你有聽我們播出那首和平.
Evangelist的詩歌.
他講的一樣東西.
十字架全部的和平.
都在基督裡面同歸於一.
所以當我們去理解的時候.
其實基督耶穌的十字架.
基督女.
你成為新生命的人.

$^{721}$New Creation.
不是說你重新再開始.
做基督徒.
一個浪子的比喻.
這樣的方法.
而是整個世界的改變.
所以我們應該是這樣看的.
我們用一個世界的水平.
來去明晰這個叫做New Creation.
這個神話概念.
所以你都知道.
基督耶穌的復活.
基督耶穌的死.
然後復活.
這個復活是叫整個世界.
成為一個New Creation的開始.
他是一個初熟的果子.
一個投生的果子.
他是第一個復活的.
而一切基督裡面都成為復活裡面的開始.
所以復活節應該是開心的.
因為他講的是一個.
全世界能夠重新被更新被創造.
所以發現復活節是在春天出現的.
他特意強調這個世界得以一個.
重新的開始.
一個New Creation.
任何人在基督裡面都能夠不再被罪.
和被這個世界的權並和死亡來克制.
一個重新的創造.
所以我們拿著這種視野.
我們就明白當我們去理解新造的人的時候.
其實你應該從一個世界來做對比.
保羅也是.
當他每一次講New Creation的時候.
其實他是在講一個世界和New Creation之間的張力和改變.
所以為什麼會說當你今天面對香港的時候.
你是應該或者可以用New Creation來理解今天的香港.
或者理解今天你面對一個新的移民國家.
你的新生活.

$^{761}$Anyway都是世界.
如果說在基督裡面是一個新的創造的時候.
不在基督裡面是什麼.
就是一個舊的創造.
一個Old Creation.
Not in Christ.
In the Old Creation.
世界不是邪惡的.
你都知道楊方所講的.
上帝愛世界.
所以世界本身是美好的.
所以世界是有秩序的.
是美麗的.
不過在保羅的文獻裡面.
世界被形容為一個罪惡和死亡的一個展示.
所以保羅的文獻裡面.
世界是一些負面的字詞.
本來是好的.
但它是充滿著罪惡.
和被死亡.
承載著死亡和罪惡.
所以這樣說.
無論在《光陰厚書》也好.
《網絡學書》也好.
世界都是一些不好的東西.
但有證據的.
大家一起看看.
如果我們去看經文的時候.
我們很多時候都是忽視了.
基督裡面就是新造的人.
一個New Creation.
其實後面的話.
上帝在基督里.
我特意改了翻譯.
和本叫世人與自己和好.
記住原文裡面不是世人.
是Cosmos.
就是世界.
所以和本是很搞事的.
全部把人都出來了.

$^{801}$其實不是世人是世界.
所以保羅說.
我們是一個New Creation.
我們作為New Creation.
在基督裡面叫世界與自己和好.
所以保羅一直在比較什麼.
在比較世界和New Creation.
兩個都是非常龐大的事物.
Old Creation 世界 Cosmos.
和New Creation.
所以你看到.
我特意要把它作為一個比較.
先不說這個.
就是一個Cosmos.
一個紅色的有罪的世界.
還有保羅說.
在基督裡面一個新的創造.
一個新的世界.
兩個這樣對比.
這個也是.
無論是在哥倫多後書的第五章.
還是在卡塔遜的第六章.
都是一樣.
都是世界.
保羅說什麼.
我是一個在基督裡面.
十字架的緣故.
對我來說.
世界已經被釘在十字架上.
對世界來說.
我已經被釘在十字架上.
所以保羅當他在談論這個世界的時候.
世界是一個被釘在十字架上的物體.
然後說我們是一個新的創造.
一個New Creation.
所以明白.
每一次保羅說New Creation的時候.
都是在比較什麼.
就是那個Cosmos.
就是那個世界.

$^{841}$OK.
我們要明白這一點.
因為你沒有那麼宏觀.
你就不明白後面的那些應用是說香港的.
所以我們明白到保羅的New Creation的視野.
是需要和Cosmos的視野不相比.
那你就明白.
究竟是什麼意思.
一本書就是這樣說.
一本書就是將保羅和政治和New Creation.
去分析出來.
羅馬帝國.
當時保羅身處的羅馬帝國.
正正就是Cosmos的代表.
這個世界是邪惡的.
羅馬帝國也是邪惡.
我不是說羅馬帝國是邪惡.
因為它在世界的一部分裡面.
而羅馬帝國是世界上最象徵意義的代表.
所以保羅究竟是怎麼去看羅馬帝國呢.
這個很多人都在討論.
保羅對於羅馬帝國的討論.
他不是要去推翻它.
也不是去信服它.
明不明白.
保羅不是打算推翻它.
但不打算信服它.
因為羅馬帝國正正是屬於世界裡面.
一個很重要的部分的時候.
其實它也是被罪和死亡所克制住.
這個就是保羅對於羅馬帝國的立場.
羅馬帝國是一個最典型.
最象徵意義的世界的一個政權.
一個這樣的物體.
所以保羅不是個人而是很憎恨羅馬帝國.
但保羅說的是一個更加宏觀的問題.
就是一個有關cosmos的問題.
所以另外也是.
上帝的角度也是一樣.
上帝的角度從來都不是要去和羅馬帝國對抗.

$^{881}$上帝是沒法比的.
上帝的角度真正要對抗的是這個世界.
anyway 世界都是上帝所愛的.
所以世界都是需要被renew.
需要被更新.
所以明不明白整個的視野和比較.
上帝的角度是.
到很大程度是完全可以更新這個世界.
這個世界本身是好的.
所以要在十字架裡面洗去它的罪和死亡.
而羅馬帝國保羅所面對的政權或者社會.
其實在這個很渺小的地位.
是屬於世界的一部分.
或者是一個世界的代表而已.
所以保羅要去.
他所說的renew.
一個new creation的時候.
其實都是衝著羅馬帝國而講.
不過其實感動的時候就發覺是很渺小的.
這個政權雖然有很多不好的東西.
但是他說其實已經搞定了.
因為基督裡面的十字架.
已經可以叫做有一個new creation.
完全是超過了.
推翻了整個世界的問題.
這個是能夠讓我們.
藉著一個好像不太關事的.
新做的人和香港有什麼關係.
其實是有關係的.
所以如果2000年前.
保羅這樣來理解羅馬帝國的時候.
他用new creation來面對當時的政權的時候.
因為每次在講世界的時候都講new creation的時候.
就明白我們怎麼來理解.
這個我之前講過.
我那篇國安法的那一部我都提過.
我們不要效法這個世界.
不要被這個世界困住.
你不要去confirm它.
你不要聽它說.

$^{921}$你不要fix在那裡.
停不停地讓它.
你不要跟它.
你不要抄它.
你不要學它.
但是你要它更新你的心意.
在政治裡面.
所以同樣一個差不多的道理.
面對著這個世界的時候.
你不要被它改變你.
你不要跟它.
但是你要尋求一個transformation.
你要去做一些新的事情.
因為你是一個new creation.
我們最大的問題是什麼.
其實我們今天是new creation.
不過我們是一個活在舊世界裡面的new creation.
到哪一天.
一個完全中的來臨的時候.
全世界會被更新.
但是作為基督徒.
我們是new creation.
但是我們在一個舊世界裡面.
仍然出現一個這樣的處境.
所以我們不要跟它.
當你在這個世界裡面.
如果你沒有去發現自己的身的時候.
你就會被這個世界跟隨著.
就會被同化了.
好了,我不說了.
大家聽回哪裡.
這個講回哪裡.
有關這個不要confirm這個世界.
和一個更新的問題.
所以我們如果去這樣講的時候.
有幾點我們可以值得來一個總結.
第一.
就是我們可以重新來定義.
什麼叫做新生活.
什麼叫舊,什麼叫新.

$^{961}$我們用一個更加高的層次去理解今天的新香港.
什麼叫做新,什麼叫舊.
你會發現真正的新.
其實不是在這個世界裡面出現.
這個世界是一個羅馬帝國的old creation.
這個世界有多少新鮮滾熱辣的新常態都好.
其實這些都是舊的東西.
在神眼中.
在神眼中真正的新.
不是這些所謂新鮮的事情.
不是一些新的轉變.
或者新的政權移交.
或者你去了新的地方裡面生活.
而是那個新字在你的生命裡面.
多於在生命的外面生活.
這個很重要.
這個雖然是把definition這麼說.
你其實都感覺不到的.
但是你要知道這個definition.
就是真正的新不是在這個我們面對的新環境.
新處境.
而是那個新.
在神的眼中.
在我們基督裡面的新創造.
那些是舊的東西.
正如我所說.
所謂日光之下無新事.
這個也是傳說中所說.
我在日光之下.
沒有什麼是新的事情.
政權又是一個政權.
古代很多的政權的問題.
今天仍然存在.
很多事情來來去去都是這樣.
所以你發現.
這是一些新的改變.
但發現那些都是一些舊的東西.
那些是日光之下的東西.
沒有什麼特別新的現象出現.
所以我們可以prepare to be surprised.

$^{1001}$我們是surprised的.
我都知道.
當我要用VPN我都會surprised.
不知為什麼.
沒有那麼多香港人都用VPN.
但這些事情又發覺.
不是太令你那麼驚訝.
我們知道.
這些都是一些舊的東西.
舊的板斧.
都是為了操控.
都是為了某些.
以前出現過的某些問題.
所以縱然是新發現.
但都是一些舊有的歷史.
重復又重復.
所以這些是我們要的基本盤.
我們去重新定義.
什麼叫新什麼叫舊.
我們知道今天所發現的新的東西.
都不是一些新的事情.
反而.
什麼叫新.
新是一個很獨有上帝專有的詞語.
新的事情.
是上帝就是有新的事情.
因為上帝是一個新的上帝.
不是重新來一個吸出來新的意思.
這個新是不會被時間沖淡或流逝.
回答一個問題.
20年前你做了新壽的人.
今天你是否舊了.
想想你自己是否舊了.
對於以前的信仰.
今天是否已經.
起了壽.
已經發覺很多的.
已經滑了牙.
或者已經不同了.
很多以前新的驚喜.

$^{1041}$很多新的事情.
發生了.
唱了新的詩歌都算舊了.
以前我第一次唱贊給傳承人.
你也開心.
今天你不會唱這些歌.
所以不關事.
上帝的新不是被時間的新出現.
而是上帝裡面是歷久常新.
因為新本身就是上帝的屬性.
所以上帝的新.
只能夠在上帝裡面.
不關你一個的時間問題.
所以今天你仍然是一個新的創造.
不是因為你對了決志.
而是因為你仍然是一個新的創造.
為什麼.
新的創造.
新是一個上帝的專有名詞.
新出埃及.
新耶路撒冷.
我要做一件新事.
譬如說唱新歌.
全部都是上帝專有的詞語.
所以將來在天堂裡面.
你不會沈悶.
因為全部都是新的.
新不是因為你沒玩過.
而是因為一個這樣的狀態.
無論如何我也不知道是怎樣.
是一個很新的狀態.
所以今天我們要問的問題就是.
你面對著這個世界.
你面對著今天的香港.
或者你面對著將要移民的新地方.
你是一個新基督徒.
面對著一個舊世界.
還是說一個新世界的舊基督徒.
有很多反省的地方.
究竟你面對著這個世界的時候.

$^{1081}$哪一個是新一點的.
一個不許你去新香港.
但是是一個舊的你去面對這個新香港.
或者說我是一個新的創造.
面對著一些舊的世界.
Cosmos 和新的創造.
很有趣的.
究竟我們面對著這個世界.
是怎樣去理解它呢.
這些所謂的新事.
其實都是舊的東西.
重要的是我們本身一個新的創造.
都是一樣.
就算是移民也好.
面對著一個新的環境.
其實重要的不是這一點.
而是你自己是一個新的創造.
所以就要形容我們.
我們一些比較具體的思考.
我這次的思考會分開兩邊.
一邊就是給我們來香港的人.
一邊就是留下來的人.
因為發覺都是面對著一些新的環境.
同樣一個課題.
兩邊的人可以怎樣面對呢.
第一個是我們面對著新常態.
留下來的人面對著新常態.
我們說這些所謂新的常態.
其實這些不是新的事情.
就算你是面對著一個新生活也好.
你去移民.
你是一個新的環境.
其實這些環境新是對的.
你當然要適應.
但從一個上來說.
我不是教你怎樣去適應新生活.
而是你要記得我們真正新的事情.
不單單是這些新的環境.
而是我們本身一個新的生命.
當你懷著這個新的生命.

$^{1121}$去面對這些新的事情的時候.
你才能夠應對.
所以重點不要放在眼前的環境.
無論是這個香港.
或者是你面對的新國家.
重點是你自己內心的新是怎樣.
所以我們都會說.
我們香港這邊就說不要習慣.
意思就是我們不要被它確定.
一方面我們提醒自己.
不要習慣這個新的香港.
不要覺得這些事情成為了我們日常的事情.
不要成為新的常態.
我們就慢慢習慣.
同時我們在海外的時候.
我們都說要適應環境.
我們要適應新的生活.
要接受當地的社會.
重點是一樣.
重點不是你習慣還是不習慣.
或者是適不適應的問題.
而是你如何來到裡面去面對這個世界.
當然你要習慣一下這些事情.
但這些不是上課題.
我們要習慣的說.
這些不是我最擅長的說話.
但作為一個聖經的人.
我們要知道的就是.
面對這個環境.
你習慣還是不習慣.
重點都不是這些生活問題.
而是你自己內心的生命.
是否有一個新的力量去面對它.
所以重點在這裡.
無論這個世界如何轉變.
那個持續性其實是你的信仰.
那個持續性是你的信仰.
這個在你內心的新生命.
是一個最重要的持續性.
當我們有這個新的眼界的時候.

$^{1161}$我們就能夠去應對很多不同的變化.
因為那個真正的新來自於我們的生命裡面.
所以說得很像.
我今天只有一個命題.
就是我們所謂的新生活.
我們說的基督徒新生活.
不是說你如何去重新適應今天的香港.
這麼簡單.
也不是說你移民之後.
如何去適應新生活.
而是你內心的生命.
是否一個新的創造.
那是什麼呢?這個新創造.
有什麼可以簡單地去描述一下呢?.
這個也是我翻譯的那本書.
默德曼的《豐盛的上帝》那本書.
有一句我講得很好的.
有一個字叫做.
Incipit Vita Nova.
一個叫做開戰新生命.
默德曼說.
什麼叫自由.
有一個很有趣的定義.
自由.
他說自由不是我們能夠做到多少事情.
也不是我們能夠有多少主權.
有多少可能性.
而他的自由是在於.
是你能不能夠引發一些新事物出來.
當你能夠有能力去引發一些新事物的時候.
這個就是真正的自由.
這是一個很特別的定義.
我覺得這個很有啟發性.
當我們去面對著香港.
或者是面對著新的生活的時候.
你的身心的創造就是什麼呢?.
就是上帝給你一種仍然可以引發新事物的力量.
今天的香港好像很多都沒怎麼變化.
或者很多時候都已經是一個緊局.
重點不是你有多少自由可以去做些什麼.

$^{1201}$起碼這個不是定義.
定義就是你有多少能力去做一些新的事情.
這種創造性.
這種在沒有的有的力量.
才是那種自由.
他說每個人都是被誕生者.
每個都是一個被創造.
一個新生的人.
這些未誕生的人都是一個新生命的起始者.
一個新生命的開始者.
每個人的生命都是獨特的.
無論怎麼被基因影響.
無論這個社會怎麼限制.
生命的開端.
卻讓生命有別於任何前來與之後的生命.
生命仍然可以去做一件新事.
在香港裡面.
當然在海外也一樣.
不是說你適不適應.
還是能不能確定.
而是你能不能開始一件新事.
充滿著創意地去開始一件新事.
我想這個可以很實際地去量度.
今天我們是不是一個新創造.
在很多很局限制的事情裡面.
我們可能能夠做新事.
因為我覺得翻牆也是一件新事.
我們可以突破今天的限制.
比較有創意地去做新事.
想不到今天我們會講翻牆這個話題.
所以新生命就是這樣.
我們仍然有一種力量.
去無視外界的限制.
去創造一件新的事情.
最後一個總結.
所謂基督徒的新生活.
不是在乎世界環境是怎樣.
因為這個世界是一些舊的東西.
所以這個世界是怎樣的新常態也好.
或者新環境.

$^{1241}$新的移民國家也好.
這個不是重點.
這個影響不了你的新生活.
我們的新生活.
也不是20年前的那個缺席.
而是什麼呢.
是基督裡面的那種力量.
因為你每天都有一個新的生命.
它能夠讓你對這個世界保持一個新的觀點.
一個新的眼光.
保持一個新的可能.
這個是我們今天在基督裡面.
新生命裡面.
去面對世界一個很重要的思考.
很喜歡這句話.
愛歌的經文.
就是每天精神都是新的.
這個是上帝給我們一個很大的應許.
在基督裡面一個新的生命力量.
我們就能夠有新的觀點.
新的眼光.
新的可能.
我們就祈禱吧.
因為你給我們Full Church弟姐妹.
在這裡今天去思考.
你給我們一個新的生命.
我知道我們這個新的生命.
不是在乎於我們信主連日有多少.
而是我們在你裡面.
得到這份生命力量.
所以我們此刻很多可能還沒有去.
尋索思考和發問的時候.
但是你求你的靈當中去跟隨我們.
讓我們能夠對於今天的環境.
無論是香港還是國外.
我們面對的生活.
我們自己的生命.
讓我們有一個新的開始.
讓我們能夠可以.
有一個力量去想.

$^{1281}$和去開始一件新事.
讓我們能夠不會被時間.
或者是社會的規限所限制.
在這裡永遠發現一個新的事情.
來展現我們在這裡的新生命.
幫助我們Full Church弟姐妹.
都能夠有這個做新事的力量.
奉主命求.
阿們.
Captain Poon.
阿莫基.
我今天聽了很多個新字.
你也是個新屋.
是新邊.
我不知道正在等待上機的人.
聽了很多新字.
有沒有什麼問題.
或者是覺得有些東西棘棘的.
大家可以隨時問問題.
我們看到了.
今天講了很多在舊的世界.
等一下.
好 謝謝.
今天也講了很多在舊的世界.
有什麼新的觀點.
新的眼光.
但是其實我還沒有很明白.
是真的應用.
我相信新的東西不是說.
我們在舊的世界找新的餐廳.
看新的書.
現在還沒有去到那裡.
香港的環境不允許教會的團聚.
用一些新的方法去有團體生活.
其實今天的香港.
或者是一些流散到外地的人.
那些新的東西.
其實在應用上.
有沒有一些實質的例子.
剛才我說了.

$^{1321}$比如面對著香港.
或者說一個比較遠的例子.
這個我曾經說過.
柏林被人分開東德和西德.
東柏林和西柏林.
當時柏林的人民.
其實為了自由.
因為當時蘇聯封了整個東柏林.
那些人就去.
我記得那時候去過柏林.
博物館就說.
這件事是叫人激發創意的一個.
最大的方法.
那時候博物館就展示很多.
人們怎麼從東柏林走到西柏林的方法.
有些人是在車後面車尾.
就運過去西柏林.
有些人就坐中興氣球.
有些人就在對面那裡.
用床單綁一條繩子就滑過去.
有些人就扮警察走過去.
那博物館是很有趣的.
他說當人失去自由的時候.
是最能激發起創意的方法.
所以就說面對著這個Old World.
這個羅馬帝國的時候.
重點不是一些新的常態.
或者一些新的形勢.
新的法律.
而是我們能夠可以去用我們的.
上帝給我們的創意.
新事來做出新事.
重點不是服從還是反抗.
明不明白.
兩個都不是.
是新的事情.
你要麼就要服從這個舊的世界.
或者你走一次反過去.
其實都不是.
而是你做一件新的事.

$^{1361}$一條第三道路.
一個third way去面對它.
所以這個就是精神.
就是我們嘗試能夠去突破框框去做事.
不是服從還是反抗.
而是突破.
所以這是一個例子.
所以我們可以說.
面對著這個世界的時候.
就是說根據信仰上.
將社會和新的創造.
新的創造來連在一起.
所以新的創造.
就是讓我們能夠有這個.
發揮做新事的一種能力.
大家可以試著想想.
什麼可以做出來的事.
無論是教會.
你面對的生活處境.
或者是怎樣都好.
可以有很多的例子.
還有問題嗎.
那邊有.
有問題嗎.
你好.
剛才看到關於這個新創造.
然後你又引用了以蔡阿書.
然後我看到新業路剎那.
所以不止想起創世紀.
想起啟示錄.
然後我就有一個問題.
因為你一直說.
新的創造.
似乎是耶穌基督的死和復活之後.
就已經出現了.
那我理解為.
以蔡阿書說的新天新地.
好像已經出現了.
就是我們已經在新的創造裡面.
那是不是在事實裡面.

$^{1401}$所說的新天新地.
其實都已經出現了.
我就想著.
我就一頭問號.
所以就在想.
我這樣理解是不是有問題.
沒問題.
其實都是最經典的.
已經出現了但還沒出現的那些.
明白嗎.
就是所謂新創造不是新世界.
已經是彰顯.
但又未完全彰顯的狀態.
所以我們有這樣的緊張.
我們的新創造.
是可以有能力.
可以做新事的.
但它在一個古代的世界裡面.
這個就是那個緊張.
又未完全是.
新生地所說的新創造.
或者是所說的天堂的環境.
但又是一個能夠去彰顯.
這一詞最難的.
是彰顯還是呈現.
還是活出.
我們是和新創造有關係的.
我們有可能去.
生活在新創造的生活里.
有一個已然未然的狀態.
就是這樣.
所以就是這種.
這種中活性的緊張.
這樣說的時候.
或者你可以用一個.
道地點.
或者可以用一個.
另一個方向去想的方法.
就是.
我們很多時候比較到.

$^{1441}$我們會上天堂.
是上天堂.
但當我們悼念主土文的時候.
會提到一件事.
願禰的國降臨的時候.
我們這個是主耶穌教導我們的土文.
是上帝的國降臨的時候.
上帝的國.
剛才The Kingdom of God.
也說了一個信息.
就是上帝的國會臨到.
在我們當中去彰顯的時候.
其實不是我們上去.
是上帝的國來到我們當中.
去將祂的能力彰顯.
所以用剛才羅馬書.
帶出的經文的時候.
你會看到.
他叫我們不要確定在這個世界里.
而有的行為.
所以第十四章.
第十五章的羅馬書里.
保羅基本上用了.
羅馬書差不多五分一的篇章.
說一個信徒.
有這個身份而有的生活.
應該怎麼做.
他用了不同的例子去說.
既然你這群不是猶太裔的基督徒.
在一個希羅文化.
羅馬政權之下.
要活出一個新的形態.
的生活形式當中.
我們怎麼可以做到一些事.
所以回應剛才John說的信息.
其實不能夠拿一個方程式.
告訴你.
在哪個地方就做些什麼.
反而就是在你的處境當中.
你應該會有一個新的向度.

$^{1481}$新的mindset去過你的新的生活.
這個就是他在回應上帝.
給我的新的創造.
不知道你能否理解這個向度.
我想問.
我自己覺得香港是一個.
大家很要走同一條路的地方.
如果你有些不同.
很難走.
我想問基督徒的新生活.
如果以個人去做一些事.
或者回應來說.
好像力量很薄弱.
我想問.
在個人和群體的新生活中.
可以如何理解去做到.
對,本來是說群體.
他說群體能夠扶持.
去做一些新的事情.
或者群體裡面.
和這個羅馬帝國有些不同的做法.
有個不同的方式去活.
譬如我們當時Fold Church.
我們整個群體一起.
做一些新的方法.
面對著疫情的時候.
就是這些例子.
所以是一個群體的事情.
大家一起去.
又不是去confirm他.
但又不是去推翻他.
而是嘗試一些新的方法.
去走第三條路.
我們不會跟.
但又不是不跟.
跟一些不跟一些.
這些是一個很好的例子.
所以群體是說大家去做這些事.
所以不是說你自己.
要和社會做些什麼.

$^{1521}$這群人是一起來活一個社會.
教會就是一個新的群體.
和這個世界價值是不同的.
這才是力量.
所以不是你自己去做一些新的行動.
一個人做叫做action.
一個行為.
一群人做就是一個社會.
一個群體.
這樣才能有互動.
更加容易行出來.
一個人做就是一個社會.
有些奇怪的人在地球上.
不知道在做什麼.
但一群人做就是一個社會.
就可以不需要在社會裡面.
不是,是一個小社會.
當時羅伯特說.
教會就是一個小的社群.
他們一群人是能夠生活的.
所以有些小圈子.
叫做教派主義的看法.
但這種教會群體.
正正是大家能夠work到.
生活到.
前面這裡.
其實也有點像.
我不知道是我見識少還是什麼.
我覺得香港有一個特別的情況.
就是香港人的工作.
應該是世界問鼎的其中一位.
我很明白那種.
如何讓自己的心智.
仍然保持著新.
不要磨滅.
但我也很深信.
可能有很多弟兄姊妹.
甚至包括自己也是.
有些位置就是.
你的日常生活已經被佔據得太多.

$^{1561}$以至你仍然能夠.
和某些弟兄姊妹圍爐.
再分享,再希望那團火不要熄滅.
但好像實質上.
真的沒有這樣的空間.
讓你讓自己的生命.
不自達盡於世界當中.
分別出來.
我就想.
是不是以前.
如果說以前在教育里.
那種是他們的.
呼求的方式.
是我們差一點理解.
我們應該怎麼去渴求.
還是我們還沒得到.
那種所謂的new creation.
我們還差什麼.
還有我們可以怎麼在.
我們現在這些信徒群體里.
不只是互相支持.
是真的有一些東西.
在香港這個情況做出來.
我說的情況.
雖然可能局勢是一樣.
但我覺得另一樣就是.
那種工作對我們的生命的捆綁.
在香港是特別一點.
我覺得.
我先回答.
我就稱之為身份教育.
就是你多麼認受.
這是你的身份.
和你多麼認受.
是你的身份而有那種行為表現.
舉個例子.
我怎麼教兩個兒子.
去做好一個學生的身份.
小時候和他們去旅行的時候.
他說爸爸你為什麼還不快點看.

$^{1601}$我快要下機了.
這部戲快要完了.
快點看吧.
因為我在填紙.
下機的時候要填新報表.
我就說這些你自己填吧.
然後我兒子就問我.
occupation在這裡寫什麼.
我說你平時做什麼多.
你平時最主要的時間做什麼.
學生.
你就寫學生.
在這個過程中.
我就教他一件事.
其實你的身份是學生.
所以做功課是你要投入的地方.
在這個過程中.
讓他明白身份而有的行為表現.
剛才你說到環境很差.
或者環境很不就的過程中.
其實我們要做這件事.
是很不容易的.
反而我就會用身份的角色去看.
上帝把我們放在香港.
把我們放在其他地方.
那也是你的身份.
你去到不同地方.
雖然環境改變了.
但你的身份沒有改變.
你應該在你的環境中.
活回你的身份而有的表現.
所以你要有時間和職能.
在當中想辦法去做出來.
對於我來說.
應該要融入在你的日常生活中.
反而如果日常生活中沒有改變.
其實你做哪裡都好.
有沒有改變都好.
其實你未必做到基督徒的本份.
所以我其中一句口頭禪就是.

$^{1641}$「似」即是「不是」.
你似一個基督徒還是不是一個基督徒.
就看你平時有沒有做基督徒應該有的事.
不只是取決於環境.
也不是取決於你的信仰在哪裡.
反而是你自己怎麼去持守.
因為耶穌的緣故.
你成為一個新做的人.
因為耶穌的緣故.
你再一次重拾上帝兒女的身份.
那種職能.
你好,如果有一個基督徒.
他本身很壓抑.
經常都很希望自己去愛神.
但神就告訴他.
你要放鬆自己.
你要做一個新的人.
你要心思跟身而變化.
那他就成為一個很輕鬆很愉快的人.
因為耶穌就提醒他.
凡事都可行.
大不都有益處.
但他就真的很希望在2023年.
決定做自己很有天分.
很喜歡做的事.
但其實這件事.
是世上的小學.
其實都是老舊世界出來的某些東西.
其實作為一個比較熟悉的基督徒.
嚴格來說.
可能有些人就覺得他不是跟身而變化.
而是你去愛你自己的舊我.
去到這裡.
其實他真的很想做一個跟身而變化.
隔身的自己.
其實他已經放下了.
他不再服侍主.
他很開心.
他也很少看聖經.
但也會禱告.

$^{1681}$也感受到神的同在.
他說其實這個真的神給我在2023年.
成為一個隔身的自己.
但其他人不認同他.
反而又為他禱告.
很擔心他的情況出現.
要繼續牧養這個迷途的基督徒.
我想請問.
他其實內心真的覺得神讓他跟身而變化.
但身邊的人這樣說的時候.
他很迷茫.
你會不會給一個建議.
這個迷茫的基督徒.
完全認同.
因為.
我這樣說.
我是很明白這些情況.
剛才這樣.
跟身不代表完美.
我剛才認為不是完美的基督徒.
很多時候仍然是很多的軟弱.
所以這個就是所謂.
你要renew一個new creation的意思.
因為new creation不是從你一個以前的罪人.
變成完美的人.
new creation正正就是一個這樣的狀態.
你需要不斷的去被更新和改變.
而你更新完的那一刻.
其實是一個beta版.
0.幾個version來的.
是新的.
但又不是final的.
所以每個人都是這樣.
更新不代表好了.
是好一點.
或者是另一個perspective去看這件事.
可能都是差的.
也不出奇.
所以更新不代表好了很多.
我們人生裡面就是不斷的被transform.

$^{1721}$最怕的是你已經停在那個位置.
以為我已經完美了.
就是這樣.
所以其實我想.
很多fortune 的人的掙扎.
不只是剛才所說的人的心路歷程.
每個人都是這樣面對自己的生命.
不斷問自己我是否一個新做的人.
好像還是很不行.
其實很多舊的東西.
剛才說的比較多的是社會層面.
但說到屬靈層面多一點也是一樣.
我們仍然有很多問題.
所以我們這個新的創造的意思.
正正不是一個perfect的創造.
所以是基督裡面不斷的被更新.
所以很體會剛才所說的情況.
我的看法是.
姐妹剛才提到的尋求幫助的人.
我通常覺得要多聽幾次之余.
還有不同的人聽聽他的表達方式.
因為很多時候我的反省是.
自己說的都是自己的苦主.
其他人一起聽的時候會給他更多角度.
重點說的是.
如果自己是基督徒是很孤單.
是很不容易的.
不是說一定要圍圍圍大群人一起.
才是一群好的團體.
反而是你將你的難處和其他人分擔.
和其他人分享.
其他人給你的看法.
這個在當中就是聖經提到的.
兩三人聚集的時候上帝就在當中.
就好像上次我們第一堂裡面說的.
上帝與人同在Shakina的意思.
那種會幫我們去分辨.
我們這個群體團結一起的時候.
有什麼向度要我們一起走.
或者是我們跌倒的時候.

$^{1761}$怎麼可以提醒.
所以我覺得剛才的提問是有種苦澀.
或者是不安的.
但是我相信在慢慢一直去接觸.
和不同的人一起去守望或者聽這件事.
其實慢慢會看到有一些共通性.
是大家都可以做的.
說得好像很凶狠.
但是流動的場景是有這些群體慢慢去結集.
然後透過不同的活動或者聚會的時候.
就讓訴說這些狀況的弟妹感受到不孤單.
其實就會看到一個群體慢慢被凝聚.
就是這樣.
另外我想說的是新的視野.
剛才華文說新的視野.
新的眼界去看這個世界.
也是看自己.
所以永遠保持一個新的眼界.
去看自己和信仰是很重要的.
如果多些應用的話.
希望Footswap的人都能夠多些去接受其他人的看法.
從而大家可以更加更新自己的盲點和看法.
這個也是重要的.
對於我們Footswap群體來說.
其實很需要去不斷去.
不要覺得自己是一個最後的產品.
這個就是一個新創作很重要的意思.
不然就是一個舊的完美的產品.
其實都是一些舊的東西.
所以更新的眼光和視野是很重要的.
有沒有其他問題?.
這邊.
我想問一下新創作在.
如果個人生命的應用層面.
其實是否只單單在乎於.
一個跟從基督的道德規範生活的人.
和一些不跟從基督道德規範生活的人.
這麼簡單.
還是其實規範不只是在這方面.
是否跟隨這方面呢?.

$^{1801}$這個問題可以有很多不同的答案.
當然是.
沒理由不跟隨耶穌.
當然是跟隨耶穌.
但什麼叫跟隨耶穌呢?.
我今天所說的就是比較強調將來的可能性.
不單單是回望.
看回方文秀所說的耶穌所做的事.
這當然是.
但很多時候我們所面對的世界.
或者所說的耶穌.
其實不單單是以前.
2000年前在加利利海一帶的耶穌.
而是將來做生事的耶穌.
所以當我們這樣說的時候.
我們今天強調的就是新的可能性.
這都是一些跟隨.
上一集也說過.
跟隨耶穌的方法.
不是回望耶穌.
而是將來的耶穌.
跟隨他也是一種方法.
所以做生事也是跟隨耶穌的方法之一.
所以緊守耶穌的教導當然是.
同時也會因為這樣的緣故.
去做一些生事.
所以這兩個不是懷疑我的事情.
而是兩個都有.
總的來說當然是跟隨耶穌.
或者是一些法則,規則.
但有時候規則不是在聖經裡面寫出來的.
你需要在聖經裡面去跟隨耶穌的話.
就不單單是聖經寫過的東西那麼簡單.
那些是聖經的原則.
但很多事情都是更加在世界裡面.
你需要去面對的處境.
這樣去面對.
而不是在很律法的方法裡面.
就算數了.
或者可以從.

$^{1841}$我們看哥羅西書裡面.
關於基督論的一些內容.
因為哥羅西書裡面帶出一個信息.
就是說我們不分什麼類型的人.
無論是希利里人,維魯人,自主的人.
但接著保羅就提到.
不同身份的人.
我們都會有一些不同的.
所謂的創作.
所以說到要傳揚基督的方法.
就是用詩歌,頌慈靈歌.
不同的方法去做一些方式去傳頌.
這個就不僅僅是一個道德規範的嚴謹.
或者是一個單一向度.
反而就是我們這個身份而有.
我們如何傳揚它的時候.
有很多創作的空間.
或者新的生活形式.
讓人去明白到.
生生活而有的多面性.
有沒有其他問題?.
其實潘Sir每一科都有做教材.
如果你們是在小學就知道.
無論是在網上參加也好.
或者是想有更多具體應用上的問題.
想討論或者談.
其實在那些材料都有.
教材或者今天講的比較理論神學多一些.
但如果很多想討論的應用.
其實在教材裡面.
是有些問題.
讓大家去思考大家的生活和生命.
和分享.
所以都建議大家.
都可以去參加或者用這些資料.
差不多登機了.
預告一下.
下次是四月.
都是四月最後一星期.
下一次就是我們的航班.

$^{1881}$就是FC4.
FC4是多少?.
FC428.
FC428航班.
邀請大家登機時間是八點鐘.
今天的題目就是.
是他也是你和我.
基督徒的身份認同.
我們下個星期下個月再見.
拜拜.
字幕志願者:劉文英.
主持人:劉文英.
(字幕志願者:劉文英).
\newpage



\section{馬太福音 26:30-34-20230401}
\label{sec:8KdYgVn_hzk}
\textbf{【流堂崇拜】有關跌倒前的三件事|馬太福音26\_30-34|20230401 [8KdYgVn-hzk]}
\newline
\newline
連結: \href{https://youtube.com/watch?v=8KdYgVn-hzk}{\texttt{ https://youtube.com/watch?v=8KdYgVn-hzk}} ~~~~ 語音日期: 2023-04-01 
\newline
\newline
\hyperref[sec:7ZGXT0f30Z0]{\small{< < < PREV SERMON < < <}}
~
\hyperref[sec:index_chronic]{\small{[返順時目]}}
~
\hyperref[sec:index_scriptual]{\small{[返順卷目]}}
~
\hyperref[sec:v4hE6GM4QsI]{\small{> > > NEXT SERMON > > >}}
\newline
\newline
馬太福音 26:30-34-20230401
\newline
\begin{longtable}{cl}
\hline
\hline
章節 & 經文 (和合本修訂版)\\
\hline
26:30 & \begin{tabularx}{0.7\textwidth}{X} 他們唱了詩,就出來往橄欖山去。 \end{tabularx} \\ \\ \relax
26:31 & \begin{tabularx}{0.7\textwidth}{X} 那時,耶穌對他們說:「今夜,你們為我的緣故都要跌倒。因為經上記著:『我要擊打牧人,羊就分散了。』 \end{tabularx} \\ \\ \relax
26:32 & \begin{tabularx}{0.7\textwidth}{X} 但我復活以後,要在你們之前往加利利去。」 \end{tabularx} \\ \\ \relax
26:33 & \begin{tabularx}{0.7\textwidth}{X} 彼得回答他說:「即使眾人為你的緣故跌倒,我也絕不跌倒。」 \end{tabularx} \\ \\ \relax
26:34 & \begin{tabularx}{0.7\textwidth}{X} 耶穌說:「我實在告訴你,今夜雞叫以前,你要三次不認我。」 \end{tabularx} \\ \\ \relax
26:35 & \begin{tabularx}{0.7\textwidth}{X} 彼得說:「我就是必須和你同死,也絕不會不認你。」所有的門徒都是這樣說。 \end{tabularx} \\ \\ \relax
26:36 & \begin{tabularx}{0.7\textwidth}{X} 耶穌和門徒來到一個地方,名叫客西馬尼。他對他們說:「你們坐在這裡,我到那邊去禱告。」 \end{tabularx} \\ \\ \relax
26:37 & \begin{tabularx}{0.7\textwidth}{X} 於是他帶著彼得和西庇太的兩個兒子同去。他憂愁起來,極其難過, \end{tabularx} \\ \\ \relax
26:38 & \begin{tabularx}{0.7\textwidth}{X} 就對他們說:「我心裡非常憂傷,幾乎要死;你們留在這裡,和我一同警醒。」 \end{tabularx} \\ \\ \relax
26:39 & \begin{tabularx}{0.7\textwidth}{X} 他就稍往前走,俯伏在地,禱告說:「我父啊,如果可能,求你使這杯離開我。然而,不是照我所願的,而是照你所願的。」 \end{tabularx} \\ \\ \relax
26:40 & \begin{tabularx}{0.7\textwidth}{X} 他回到門徒那裡,見他們睡著了,就對彼得說:「怎麼樣?你們不能同我警醒一小時嗎? \end{tabularx} \\ \\ \relax
26:41 & \begin{tabularx}{0.7\textwidth}{X} 總要警醒禱告,免得陷入試探。你們心靈固然願意,肉體卻軟弱了。」 \end{tabularx} \\ \\ \relax
26:42 & \begin{tabularx}{0.7\textwidth}{X} 他第二次又去禱告說:「我父啊,這杯若不能離開我,必須我喝,就願你的旨意成全。」 \end{tabularx} \\ \\ \relax
26:43 & \begin{tabularx}{0.7\textwidth}{X} 他又來,見他們睡著了,因為他們的眼睛困倦。 \end{tabularx} \\ \\ \relax
26:44 & \begin{tabularx}{0.7\textwidth}{X} 耶穌又離開他們,第三次去禱告,說的話跟先前一樣。 \end{tabularx} \\ \\ \relax
26:45 & \begin{tabularx}{0.7\textwidth}{X} 然後他來到門徒那裡,對他們說:「現在你們仍在睡覺安歇嗎?看哪,時候到了,人子被出賣在罪人手裡了。 \end{tabularx} \\ \\ \relax
26:46 & \begin{tabularx}{0.7\textwidth}{X} 起來,我們走吧!看哪,那出賣我的人快來了。」 \end{tabularx} \\ \\ \relax
26:47 & \begin{tabularx}{0.7\textwidth}{X} 耶穌還在說話的時候,十二使徒之一的猶大來了,還有一大群人帶著刀棒,從祭司長和百姓的長老那裡跟他同來。 \end{tabularx} \\ \\ \relax
26:48 & \begin{tabularx}{0.7\textwidth}{X} 那出賣耶穌的給了他們一個暗號,說:「我親誰,誰就是。你們把他抓住。」 \end{tabularx} \\ \\ \relax
26:49 & \begin{tabularx}{0.7\textwidth}{X} 猶大立刻進前來對耶穌說:「拉比,你好!」就跟他親吻。 \end{tabularx} \\ \\ \relax
26:50 & \begin{tabularx}{0.7\textwidth}{X} 耶穌對他說:「朋友,你來要做的事,就做吧。」於是那些人上前,下手抓住耶穌。 \end{tabularx} \\ \\ \relax
26:51 & \begin{tabularx}{0.7\textwidth}{X} 忽然,有一個和耶穌一起的人伸手拔出刀來,把大祭司的僕人砍了一刀,削掉了他一隻耳朵。 \end{tabularx} \\ \\ \relax
26:52 & \begin{tabularx}{0.7\textwidth}{X} 耶穌對他說:「收刀入鞘吧!凡動刀的,必死在刀下。 \end{tabularx} \\ \\ \relax
26:53 & \begin{tabularx}{0.7\textwidth}{X} 你想我不能求我父,現在為我差遣比十二營還多的天使來嗎? \end{tabularx} \\ \\ \relax
26:54 & \begin{tabularx}{0.7\textwidth}{X} 若是這樣,經上所說事情必須如此發生的話怎麼應驗呢?」 \end{tabularx} \\ \\ \relax
26:55 & \begin{tabularx}{0.7\textwidth}{X} 就在那時,耶穌對眾人說:「你們帶著刀棒出來抓我,如同拿強盜嗎?我天天坐在聖殿裡教導人,你們並沒有抓我。 \end{tabularx} \\ \\ \relax
26:56 & \begin{tabularx}{0.7\textwidth}{X} 但這整件事的發生,是要應驗先知書上的話。」那時,門徒都離開他,逃走了。 \end{tabularx} \\ \\ \relax
26:57 & \begin{tabularx}{0.7\textwidth}{X} 抓耶穌的人把他帶到大祭司該亞法那裡去,文士和長老已經在那裡聚集。 \end{tabularx} \\ \\ \relax
26:58 & \begin{tabularx}{0.7\textwidth}{X} 彼得遠遠地跟著耶穌,直到大祭司的院子,進到裡面,就和警衛同坐,要看結局怎樣。 \end{tabularx} \\ \\ \relax
26:59 & \begin{tabularx}{0.7\textwidth}{X} 祭司長和全議會尋找假見證控告耶穌,要處死他。 \end{tabularx} \\ \\ \relax
26:60 & \begin{tabularx}{0.7\textwidth}{X} 雖然有好些人來作假見證,總找不到實據。最後有兩個人前來, \end{tabularx} \\ \\ \relax
26:61 & \begin{tabularx}{0.7\textwidth}{X} 說:「這個人曾說:『我能拆毀神的殿,三日內又建造起來。』」 \end{tabularx} \\ \\ \relax
26:62 & \begin{tabularx}{0.7\textwidth}{X} 大祭司就站起來,對耶穌說:「這些人作證告你的事,你甚麼都不回答嗎?」 \end{tabularx} \\ \\ \relax
26:63 & \begin{tabularx}{0.7\textwidth}{X} 耶穌卻不言語。大祭司對他說:「我指著永生神命令你起誓告訴我們,你是不是基督—神的兒子?」 \end{tabularx} \\ \\ \relax
26:64 & \begin{tabularx}{0.7\textwidth}{X} 耶穌對他說:「你自己說了。然而,我告訴你們,此後你們要看見人子坐在權能者的右邊,駕著天上的雲來臨。」 \end{tabularx} \\ \\ \relax
26:65 & \begin{tabularx}{0.7\textwidth}{X} 大祭司就撕裂衣服,說:「他說了褻瀆的話,我們何必再要證人呢?現在你們已經聽見他這褻瀆的話了。 \end{tabularx} \\ \\ \relax
26:66 & \begin{tabularx}{0.7\textwidth}{X} 你們的意見如何?」他們回答:「他該處死。」 \end{tabularx} \\ \\ \relax
26:67 & \begin{tabularx}{0.7\textwidth}{X} 他們就吐唾沫在他臉上,用拳頭打他,也有打他耳光的, \end{tabularx} \\ \\ \relax
26:68 & \begin{tabularx}{0.7\textwidth}{X} 說:「基督啊,向我們說預言吧!打你的是誰?」 \end{tabularx} \\ \\ \relax
26:69 & \begin{tabularx}{0.7\textwidth}{X} 彼得在外面院子裡坐著,有一個使女前來,說:「你素來也是同那加利利人耶穌一起的。」 \end{tabularx} \\ \\ \relax
26:70 & \begin{tabularx}{0.7\textwidth}{X} 彼得在眾人面前卻不承認,說:「我不知道你說的是甚麼!」 \end{tabularx} \\ \\ \relax
26:71 & \begin{tabularx}{0.7\textwidth}{X} 他出去,到了門口,又有一個使女看見他,就對那裡的人說:「這個人是同拿撒勒人耶穌一起的。」 \end{tabularx} \\ \\ \relax
26:72 & \begin{tabularx}{0.7\textwidth}{X} 彼得又不承認,起誓說:「我不認得那個人。」 \end{tabularx} \\ \\ \relax
26:73 & \begin{tabularx}{0.7\textwidth}{X} 過了不久,旁邊站著的人進前來,對彼得說:「你的確是他們一夥的,你的口音把你顯露出來了。」 \end{tabularx} \\ \\ \relax
26:74 & \begin{tabularx}{0.7\textwidth}{X} 彼得就賭咒發誓說:「我不認得那個人。」立刻雞就叫了。 \end{tabularx} \\ \\ \relax
26:75 & \begin{tabularx}{0.7\textwidth}{X} 彼得想起耶穌所說的話:「雞叫以前,你要三次不認我。」他就出去痛哭。 \end{tabularx} \\ \\
[1ex]
\hline
\hline
\end{longtable}
$^{1}$頂姐妹平安.
在網上的頂姐妹平安.
這篇道是我以前曾經說過的道.
如果在很多年前.
大概在六七年前 七八年前.
我曾經作為外來港元.
去過理教會講道的話.
可能我曾經在這裡說過這篇道.
當然這篇道我重新再寫過一次.
是一個Full Church的Remake版本.
重新說這篇道的原因.
是因為這星期是受難節前的一個星期.
今天的講道經文正正就是.
在耶穌受難前的一段故事.
一段有關彼得三次不認主.
發生之前的一段故事.
一段耶穌和彼得之間的對話.
所以這篇道很適合在受難節前的一個星期.
來說的篇道.
這篇道當然是有關Sorry這個題目.
正如上星期梁國權老師所說.
Sorry可以分為兩大類.
一個是跟人說的Sorry.
一個是跟上帝說的Sorry.
今次我們會說有關跟上帝說的Sorry.
特別是跟耶穌說的Sorry.
不知道大家有沒有試過跟耶穌說Sorry.
可能比較少.
跟耶穌說Sorry好像有點奇怪.
一般來說我們跟神說Sorry.
我們稱之為認罪.
認罪祈禱大家都應該試過.
跟上帝說對不起.
或者認罪對我們新教徒來說.
其實是一件很抽象的事.
我們都知道耶穌已經赦免了我們的罪.
所以我們每一次禱告認罪的時候.
似乎都是一件不太功能的事情.
天主教會覺得這件事是懺悔告解.
真的有用的.

$^{41}$懺悔完告解之後.
你的罪能夠得到赦免.
會比較荒謬.
我們基督教新教認罪祈禱.
就不是那麼功能性.
我覺得有時候是我們的功能性比較好.
起碼我們覺得懺悔告解之後.
你去潘相面前告解.
有人去寬恕你.
你覺得會比較實在.
所以我們會一起去思想.
跟耶穌說Sorry.
我們跟神的關係.
透過彼得和耶穌說的一段說話.
一段真論.
來反省我們跟耶穌的關係.
我們一起去看一段經文.
就是《馬利福音》第26章30-34節經文.
我們一起去聽我讀.
他們唱了詩就出來往橄欖山去.
那時耶穌對他們說.
今夜你們為我的緣故都要跌倒.
因為經上說我要擊打木人.
揚州分散鳥.
但我復活以後要在你們耳先往加利利去.
彼得說眾人雖然為你的緣故跌倒.
我卻永不跌倒.
耶穌說我實在告訴你.
今夜雞叫耳先.
你要三次不認我.
彼得說我就是必須和你同死.
也縱不能不認你.
眾門徒都是這樣說.
我們一起祈禱.
祝福我們一起來敬拜你.
剛才我們全聚的弟子妹在網上.
我們在實體敬拜裡面.
我們都獻上我們真誠的頌讚.
求主你切剋對我們每一個心靈說話.
讓我們每一個全聚弟子妹的生命.

$^{81}$再一次來得以醒察.
我們在受難前的一個星期裡.
我們預備好自己來迎接你受苦的日子.
好讓我們的生命能夠行事為人.
配這個蒙召恩相請.
能夠做一個合理心意的兒女.
求你幫助我們.
幫助孩子.
孩子不配.
求你使用.
蒙主命,求.
阿們.
今天的經文是發生在.
下西馬尼園和最後晚餐.
兩個段落中間的一段事情.
在耶穌被埋的那一夜.
耶穌和他的門徒在耶路撒冷裡.
過了最後一個的雨節.
耶穌和門徒一同坐著.
要說的臨別說話都說了.
晚飯都吃了.
要唱的詩歌都唱了.
因此耶穌和門徒就離開他們晚飯的地方.
一同去下西馬尼園那邊.
正正就在最後晚餐和下西馬尼園中間.
就發生了今天我們看的經文.
就在這個時候.
整個最後晚餐的最後.
耶穌就和他的門徒.
在最後晚餐裡.
說了最後晚餐裡的最後一段話.
這段跟整段的最後晚餐的氣氛.
是很不配合的一段話.
耶穌就和門徒說.
今夜你們為我的緣故都要跌倒.
是一句不容易消化的說話.
聲裡面跌倒這個字.
其實在風書裡面是一個經常用的字眼.
我們都知道你們要跌倒就不是真的跌倒.
跌倒是一個比喻.

$^{121}$跌倒這個字英文就叫做stumble.
就是失去平衡.
幾乎要跌倒的狀態.
不過其實跌倒這個字的希臘文.
其實在風書裡面是一個很強烈的字眼.
給個例子大家.
馬科第五章裡面說過.
耶穌說若是你的右眼叫你跌倒.
就挖出來丟掉.
寧可失去白體中的一體.
不叫全身丟在地獄裡.
十八章第六節.
耶穌也這麼說.
凡使者信我的一個小字跌倒的.
倒不如把大磨石.
撿在人的頸上.
沉在深海裡.
基本上跌倒是一件落地獄的事情.
甚至是推下海這麼大這麼嚴重的事情.
所以在風書裡面.
每一段經文提到跌倒這個字.
其實不是簡單說你跌倒.
或者是說你不小心做錯事.
而是一個失去救恩.
或者是一個非常嚴重的落地獄的程度.
跟耶穌不單止是關係變差了.
而是沒有關係.
是斷絕關係的意思.
所以今天我們聽這篇道之前.
我們要調整好這個字.
跌倒是一個比較淑女的童話.
但跌倒這個字不是純粹說輕輕跌倒.
做錯一點點事.
所以再做好一點.
而是在你生命裡面的一個屬靈危機.
一個很重要很艱難的地步.
所以門徒將要面對這樣的情況.
在今夜.
大概在最後晚餐三四個小時之後.
三年跟耶穌同行建立的信心的關係.

$^{161}$一夜之間就沒有了.
門徒將要翻天覆地推翻他們的信仰.
所以耶穌說今夜你們每一個緣故都要跌倒.
是一句很嚴重的話.
如果你細心留意的話.
耶穌引用了舊約.
撒該利亞書第十三章第七節經文.
來講這句話.
耶穌說我要擊打木人.
羊就分散了.
這裡耶穌將原本撒該利亞書經文.
萬君之和華.
改成「我」這個字.
上帝自己親口的命令.
我要粉碎我的木人.
然後他的羊就分散了.
今晚將要發生的是.
就好像羊沒有了木人一樣.
耶穌首先被擊碎.
然後跟隨他的門徒流離.
跌倒離棄他們的信仰.
就好像一個沒有木人的羊一樣.
我們不是很明白沒有木人的羊是什麼情況.
我們在香港也不太見到羊.
我們經常看到羊在煮飯.
但是羊就很少見.
想想像一下.
在商場裡一個嬰兒推著推著.
突然父母不在.
嬰兒車就在商場三樓停在那裡.
這是一個非常嚴重危機的事情.
所以耶穌說你們每個人今晚將要跌倒.
你們將要失去你們的信仰.
就好像沒有了木人的羊一樣.
耶穌說你們將要為我的緣故跌倒.
是一句不是很適合當時的氣氛.
最少不是很適合當時的farewell.
最後晚餐的說話.
這兩年大家也吃了不少farewell飯.
想想林彪之前你跟鄧兄姊妹的朋友.

$^{201}$在移民之前你們就吃飯.
很開心大家聊天吃飯.
說以前的往事.
吃飯拍照自拍.
林彪突然間的朋友說.
其實我走了之後你也不會再找我了.
這是一段很不適合farewell的說話.
所以耶穌是這樣說.
耶穌在最後晚餐的最後.
說了一句非常不適合最後晚餐的說話.
不是一些珍重道別或是勸勉.
而是一句你們將要離開我.
信仰跌倒的說話.
為什麼耶穌在最後會說一句這麼令人詫異的說話呢.
所以這時候彼得很忍不住.
開口就跟耶穌說了幾句.
開始研究一下耶穌和彼得之間的那段說話.
No, No, No, No, 不會的.
這些事情絕對不會發生的.
眾人雖然為你懸固跌倒.
我卻永不跌倒.
彼得似乎很清楚自己的性格.
也對的,世上最明白自己的就是自己.
我絕對不是這樣的人.
絕對不會得罪上帝.
不會成為渣男水人的那種人.
彼得這樣說.
其實彼得沒有說謊.
最少在那個時候是沒有的.
你問彼得他會不會想這樣做呢.
不認主,出賣主,這些都是不會的.
不會的,我不會跌倒的.
我不會願意跌倒的.
不過不願意是一件事.
但是不是真的跌倒也是另一件事.
所以從這個對話裡面.
我們看到今天我們所說的第一件事.
一個跌倒的人.
在跌倒之前發生的第一件事.
就是不覺得自己會跌倒.

$^{241}$不過也很合理.
一個跌倒的人在跌倒之前.
應該不會覺得自己會跌倒.
PK總是意外.
你想想,如果跌倒之前也跌倒過.
就不會跌倒了.
這是很正常的事情.
不過彼得知道自己會跌倒.
耶穌警告了他.
不過他也跌倒了.
至少彼得不願意這樣做.
我想我們也一樣.
我想沒有哪個人願意做一個非常糟糕的人.
事實上每一個很糟糕的情況.
不是一朝一夕發生的事情.
如果有一條線叫做上帝的標準的話.
我們跟這個上帝標準.
不是一夜之間就突然滑落.
而是我們每一天慢慢地變化.
這樣去遠離這個標準.
離開神,犯罪,得罪上帝.
或者成為一個上帝眼中不合格的人.
得罪我們身邊的人.
我們沒有人願意變成這樣.
至少這不是我們的理想.
世上沒有人立志成為一個討人厭的人.
在我們人生中.
我們往往跟朋友.
跟家人,父母的關係.
往往不知不覺地去到某個境地.
我們成為別人眼中的壞人.
一個不喜歡的人.
得罪了上帝,得罪了人.
為什麼公司旁邊的部門的人.
會當我做仇人.
為什麼連打招呼都變得這麼困難.
為什麼我跟父母的關係.
已經回到一個只回去吃飯的關係.
這是一個長年累月的問題.
甚至我們自己知道.

$^{281}$很多自己生命中的一些問題.
一些的罪.
開始的時候我們都有一點點的罪疚感.
慢慢就開始變得沒有感覺.
然後開始理解為一件正常的事情.
繼續這樣去順.
這都不是我們一開始想的那個計劃.
這些不是我們想要的.
但我們是做了的.
所以有人這樣說.
有人做人的原則這樣說.
其實對得起別人.
對得起自己.
其實單單憑這句說話.
作為我們人生的座右銘.
其實是不足夠的.
每個人都覺得自己對得起自己.
很少人明知道那件事是不對的.
都是這樣做下去.
通常人們明知道自己是錯的.
就不會做.
很少人明知道錯都會做.
所以每個人都對得起自己.
歧視別人的人.
永遠都覺得自己是沒有問題的.
他歧視別人是有問題的.
自私人都是一樣.
覺得自私總是有原因和理由的.
有問題的人.
總覺得是別人的問題.
一句傷害別人的話.
永遠都是說出來之後.
才發現原來這麼大破壞力.
我們每個人都是這樣.
所以經文裡面還有一句很有趣.
他說眾門徒都是這樣算.
每個門徒都是這樣.
我們每個人都是這樣.
雖然所有的門徒都是這樣說.
但彼得的說話.

$^{321}$有另一個令我們反思的地方.
他說眾人雖然為你的緣故跌倒.
我卻永不跌倒.
眾人都會跌倒.
但彼得是一個例外.
在彼得和耶穌的爭論裡.
彼得開始選擇退而求其次.
彼得作出讓步.
他同意耶穌的命題.
但他否定了命題.
他身上的有效性.
即是說.
你對,但不關我的事.
你說的會發生.
但我沒有這樣做.
彼得同意跌倒這件事.
但他不同意這件事在他身上會發生.
他覺得自己是一個例外.
人都是這樣.
總覺得自己是一個例外.
有人說我們眼睛是長在自己頭上.
你看什麼都好.
你都是中間.
所以你覺得任何事.
都似乎是自己是一個例外.
你看不到自己.
你看不到世界.
所以斯人徐志摩寫了一段很有趣的說話.
他說誰都以為自己是例外.
在後悔之外.
誰都以為擁有的感情也是例外.
在變淡之外.
誰都以為戀愛的對象剛好也是例外.
在改變之外.
然而最終發現除了變化.
無一例外.
我們都很想成為那個例外.
但這件事的想法.
是我們在想像中很危險的想法.
一方面你高估了自己的能力.

$^{361}$覺得自己永遠站在有道理的那一邊.
覺得自己比其他人更加清醒.
覺得自己是在問題中單方面的受害者.
覺得自己對得起天父上帝.
這是一個很危險的想法.
這是一個很真實的屬靈道理.
當一個人覺得自己處於屬靈高峰的時候.
這就是屬靈低潮的開始.
或者你不是這樣看.
或者你說不是的.
我永遠都是屬靈低谷的人.
快點關心我吧,牧師.
大家都謙卑的.
大家都不覺得自己有多厲害.
可能你沒有高估你的能力.
不過你可能低估了這個世界的環境.
但變成都是這樣.
當耶穌真的被捉的時候.
當其他門徒都躲起來的時候.
他才發現原來一個人在大祭司家裡.
在大喊的時候.
每個人都在審耶穌的時候.
那一刻他真的會害怕.
那一刻他才知道自己是這樣的.
事實上在我們這個世界裡.
很多時候我們都處於這樣的環境裡.
任何事情都可以叫我們成為犯罪的理由.
能夠叫你犯罪的事情.
不知不覺之間就臨到.
大家知道大家出來工作這麼多年.
外面有很多的試探.
讓我這些做教會的人更加明白.
任何事情都可以成為一個試探.
任何組合都可以.
兩個人去旅行可以成為一個試探.
一個人出差可以成為一個試探.
一部電腦可以成為一個試探.
侍奉都可以成為一個試探.
所以說一個故事.
一個五歲的小朋友.

$^{401}$在一個糖果店外面走來走去.
眼睛不停地看著糖果店的糖果.
好像忍不住就想拿.
然後老闆看著這個小朋友.
這麼久都看著.
就問他想偷我的糖果嗎.
然後那個小朋友怎麼說.
他說我不是想偷你的糖果.
我是想忍住不偷你的糖果.
這是一個很真實的事情.
不要笑.
我想說這幾年裡我們最大的試探是什麼.
可能大家沒有想過.
這幾年我發覺是一個很大的試探.
就是在我們面前有一個很大的惡.
在我們面前有一個很大的惡.
唯願公義如滔滔江河.
唯願公平如大水滾滾.
灰飛煙滅.
這些東西.
當一個極大的惡在我們面前的時候.
當我們這幾年不斷嘗試去回應和去反抗.
逃避這個極大的惡的時候.
一個極大的惡在你面前.
你就看不到自己的惡.
真的.
為什麼我特意在這兩個月要說「對不起」這個話題.
我很想全面宣傳.
不過問題的重點都不是彼得是否準確地評價自己.
或者去評估這個世界.
整個問題.
最大的問題基本上不是在跌倒這件事裡.
跌倒是一個問題.
但這不是耶穌和彼得說話的重點.
其實我們一直被彼得的說話誤導了.
我們不小心被彼得的說話牽著鼻子走.
我們沒有聽清楚耶穌要和彼得說的話.
大家留心看耶穌說了什麼.
耶穌說什麼.
「今夜你們為我的緣故都要跌倒.

$^{441}$因為經上記著說.
我要擊打木人,揚就分散了.
但是我復活以後.
要在你們以先往加里尼去.
耶穌想和彼得說的是.
重點不是你將會跌倒.
而是你跌倒之後.
我會在哪裡等你.
我會在加里尼等你.
但彼得沒有將這句話聽進耳朵裡.
彼得只是將重點放在自己那裡.
我會跌倒,我不會跌倒.
彼得要辯論.
彼得和耶穌爭辯自己會否跌倒.
眾人雖然為了緣故跌倒.
但他永不跌倒.
然後耶穌看到他這麼極力反駁.
他就說出了那句話.
我實在告訴你們.
今夜皆要以先你要三次不認.
但彼得仍然不服氣.
仍然要爭辯.
他反駁說我是不會跌倒.
我必須要和死種不能不認你.
我是不會跌倒的.
所以聖經留下了一個很特別.
一個很深深的沉默.
耶穌沒有再和彼得爭辯下去.
耶穌只是留下一個沉默.
耶穌沉默.
因為整個討論已經是一個錯誤的方向.
整個討論只留於彼得會否跌倒這個題目.
但這不是耶穌要講的重點.
再爭辯下去.
彼得會否跌倒.
這就是耶穌的重點.
因為耶穌說.
如果我跌倒的話.
我會跌倒.
這就是耶穌要講的重點.

$^{481}$再爭辯下去.
再討論下去.
只會讓討論越來越錯.
耶穌不是要和彼得討論.
他能否跌倒.
而是你能否做到.
所以耶穌沒有出聲.
因為耶穌一開始就不想和彼得討論.
彼得會否跌倒這個問題.
他會跌倒還是不會跌倒.
這不是重點.
耶穌要和彼得討論什麼.
耶穌和彼得說.
當你真的下一次跌倒的時候.
你要知道我會在哪裡等你.
這才是耶穌和我們要說話的重點.
這才是耶穌整段說話的意思.
是最後晚餐最後一句說話的重點.
當你跌倒之後.
我會永遠在那個地方等你.
就是在你認識我的那個地方等你.
我記得很多年前.
小時候.
媽媽帶我去香港那些.
拉屁股的公園玩.
如果你和我差不多年紀就知道.
以前的公園是真的.
那些鋼架那些.
爬的那些.
你沒玩過那些嗎.
那些叫爬鋼架.
全部都是鐵造的.
不是膠來的.
是鐵的.
那時候香港的鐵架是那些.
那麼高一層.
像一個球那樣的.
鐵架是這樣爬.
很高的.
我記得那時候很高.

$^{521}$五六歲的時候都很高.
要爬的那些鋼架.
那時候媽媽帶我去公園玩.
我一去到就爬.
因為都挺高的.
就爬.
媽媽就坐在公園椅子上.
看著我.
每次我爬到一格的時候.
我就立刻扶著.
回頭看著我媽媽.
試一試爬到.
然後媽媽就這樣.
然後我就繼續爬.
爬到一格之後.
她就再看著我.
然後媽媽就這樣.
媽媽就微笑.
媽媽就點頭.
現在我做人爸爸.
我就明白這個點頭.
這刻微笑是什麼意思.
媽媽點頭.
媽媽微笑其實.
她不太關心我爬到多高.
她關心什麼.
她關心的就是.
如果你跌倒的話.
你要知道媽媽在這裡.
媽媽就在你下面.
記住.
我就來幫你.
弟子們.
今天的敬拜到今天這裡.
或者整個兩個月的月替.
都很想和你們說.
我們知道.
每個基督徒.
總是有很多軟弱的時候.
總是有犯罪的時候.

$^{561}$總是有很多生命的問題的時候.
全家都是弟子們.
都是一樣.
我們要記得.
耶穌基督在這裡等我們.
我們跌倒.
或者這一刻沒有跌倒.
這個不是重點.
反正沒有人可以說.
我永不跌倒.
我會.
你的目者會.
潘Sir都會.
每個人都會.
重點不是在你跌不跌倒的問題.
而是當你下一次.
真的遇到一個很嚴重的時候.
當你遇到一個.
聖經所說跌倒.
這樣的情況的時候.
或者你之前跌倒了.
今天你在網上看Full Church.
或者現在開始重建.
重拾的時候.
我們要記得.
耶穌就在你下一個跌倒的地方.
等著我們.
最後晚餐.
最後一句話.
耶穌不是要預言什麼.
耶穌不是要預言.
讓你三次不認主.
而是跟彼得說.
你跌倒之後.
你可以怎樣做.
你記得.
在那個地方找我.
基督徒或者我們Full Church.
最危險的地方是什麼.
不是跌倒.

$^{601}$我們知道我們有很多.
不同的背景弟子們在這裡.
最危險是我們跌倒之後.
我們總是有很多不同的原因.
很漂亮的.
當那件事沒有發生.
沒有人知道.
只有你自己知道.
問題是當你跌倒之後.
什麼時候.
怎樣.
在哪裡.
重新去找回我們的主耶穌基督.
真的想想自己.
順著這麼多年.
從你自己的誤會.
到你今天這個教會.
很多的傷害.
自己的問題.
我們同神好像很近.
又好像很遠.
回到來.
我不知道你現在的情況怎麼樣.
當我預備.
道歉的時候.
我發現.
道歉是什麼意思.
跟神說對不起是什麼意思.
道歉其實是一個關係的延續.
任何一句真而眾知的道歉.
其實都是我們期待著.
這段關係能夠繼續下去.
你才會說對不起.
剛好都是神給我的經歷.
上星期我和我女兒吵架.
一件很少發生的事情.
因為我女兒只有九歲.
我很少和我女兒吵架.
事緣都是一件很無聊的事情.
每晚我和我太太會和我女兒一起看靈修書.

$^{641}$那本靈修書叫《我與天父chat chat》.
是一本中文幾口的書.
一方面靈修一方面看中文.
家長就是這樣.
一方面讀一下中文.
又可以看靈修.
每晚睡前她都和我一起讀.
差不多有一晚十點多.
差不多睡覺.
她就過來說要靈修.
我說好吧.
那時候我在用Steam Deck的彈機.
那時候我立刻收.
也不拖她.
我先save遊戲.
讓我走走.
當我save遊戲的時候.
她突然在玩.
她在戳我的畫面.
你知道Steam最雜的是什麼.
就是按錯按鈕.
她在戳我的畫面.
就save錯了遊戲.
我立刻罵她.
不要這樣.
罵了一下.
她就很不爽.
被人罵了一下.
然後她就不理我了.
我和她一起靈修.
她就不理我了.
然後她就說自己靈修了.
但是她會隨便靈修.
隨便看.
然後我就和她越來越僵.
一個在床頭一個在床尾.
兩個就在那裡很憤怒.
弄了很久.
然後她媽就走進來.
她說不要再弄了你們.

$^{681}$明天再說.
因為你女兒已經受不了.
因為她睏了.
睏到沒什麼理性.
她已經硬睡了.
然後她就回到自己的房間.
關燈睡覺.
然後我和師母聊了幾句.
我忍不住回到房間.
在床邊說對不起.
早點睡.
當我想有明天早上的時候.
我就想今晚就說對不起算了.
當你願意說對不起是什麼意思.
你願意和那個人有將來.
能夠有第二天.
能夠延續這個關係下去.
所以耶穌和彼得這個故事裡面.
沒有說過任何對不起.
不過其實你知道.
耶穌在彼得跌倒之前.
甚至知道自己跌倒之前.
就一早已經預備了這樣的關係.
彼得在不認罪之前.
甚至知道自己有問題之前.
耶穌一早就已經為我們.
預備了這樣的關係.
耶穌和彼得說.
當我復活以後.
要在你們的門前先往加利利去.
加利利是什麼地方.
如果你去加利利的話.
加利利是一個海邊的地方.
一個非常大的海.
海邊有漁船.
一個風和日麗.
氣候很溫和的地方.
不過這個不是重點.
加利利是彼得和耶穌認識的地方.
加利利是彼得認識主耶穌的地方.

$^{721}$加利利是彼得起初的信仰建立的地方.
加利利是昔日彼得很快樂.
很有熱誠地跟隨著的地方.
耶穌在加利利等著我們.
還記不記得你剛剛坐上主的時候的樣子.
在中學生團契那裡.
傻乎乎地在做團長.
在團契那裡跟團友吃到地的那種.
很有熱誠地在侍奉那個樣子.
耶穌在加利利等著我們.
從各個他的十字架.
復活的空墳墓.
再回到昔日的加利利.
耶穌邀請我們回到昔日的地方.
那個你起初很有熱誠.
跟隨著耶穌基督的地方.
在那裡,復活的大能.
賜人生命的力量等著我們.
我們發現耶穌等待彼得並沒有確實的時間.
經文裡面好像沒有這樣寫.
你說這樣也有的,這樣等人也有的.
什麼時候?.
耶穌什麼時候約了彼得?.
耶穌約了彼得的時間.
就是彼得願意回到加利利的時間.
彼得願意什麼時候回去.
耶穌什麼時候就在那裡等他.
所以你記得是不是.
《約翰福音》第二章.
彼得回到加利利的時候的經文.
大家記得在漁船裡面.
然後在岸邊,耶穌就在那裡煮早餐.
大家試想一下一個完美的星期日早上.
當你起床的時候.
你床上突然聽到廚房那些「叉叉叉」的聲音.
一陣煎香腸的味道.
對不起,還有一些奄列的味道.
當你走出你的床,走到飯桌的時候.
發現奄列旁邊還有一些剛煮熟的麵包仔.
旁邊還有一些切好的青瓜和番茄仔.

$^{761}$奄列切出來的蛋汁流出來.
很漂亮,很漂亮,很漂亮.
還有一些火腿絲在裡面.
當然還有一個非常完美的咖啡.
然後你看到幫你煮早餐的不是你媽媽.
也不是你高人姐姐.
煮早餐給你的是主耶穌.
主耶穌在等著你.
等你回到你昔日的模樣.
就在這個完美的星期日早上.
你抹乾你的眼淚.
再一次回到主耶穌的身邊.
對不起.
我們回到這間教堂正正就是期盼這個再一次.
我們可以有再一次.
(音樂播放).
\newpage



\section{路加福音 23:32-43-20230408}
\label{sec:v4hE6GM4QsI}
\textbf{【流堂崇拜】我們都有錯|路加福音23\_32-43|20230408 [v4hE6GM4QsI]}
\newline
\newline
連結: \href{https://youtube.com/watch?v=v4hE6GM4QsI}{\texttt{ https://youtube.com/watch?v=v4hE6GM4QsI}} ~~~~ 語音日期: 2023-04-08 
\newline
\newline
\hyperref[sec:8KdYgVn_hzk]{\small{< < < PREV SERMON < < <}}
~
\hyperref[sec:index_chronic]{\small{[返順時目]}}
~
\hyperref[sec:index_scriptual]{\small{[返順卷目]}}
~
\hyperref[sec:3PY1nwdp_0k]{\small{> > > NEXT SERMON > > >}}
\newline
\newline
路加福音 23:32-43-20230408
\newline
\begin{longtable}{cl}
\hline
\hline
章節 & 經文 (和合本修訂版)\\
\hline
23:32 & \begin{tabularx}{0.7\textwidth}{X} 另外有兩個犯人也被帶來和耶穌一同處死。 \end{tabularx} \\ \\ \relax
23:33 & \begin{tabularx}{0.7\textwidth}{X} 到了一個地方,名叫髑髏地,他們就在那裡把耶穌釘在十字架上,又釘了兩個犯人:一個在右邊,一個在左邊。 \end{tabularx} \\ \\ \relax
23:34 & \begin{tabularx}{0.7\textwidth}{X} 〔 這時,耶穌說:「父啊!赦免他們,因為他們所做的,他們不知道。」〕士兵就抽籤分他的衣服。 \end{tabularx} \\ \\ \relax
23:35 & \begin{tabularx}{0.7\textwidth}{X} 百姓站在那裡觀看。官長也嘲笑他,說:「他救了別人,他若是基督,是神所揀選的,救救他自己吧!」 \end{tabularx} \\ \\ \relax
23:36 & \begin{tabularx}{0.7\textwidth}{X} 士兵也戲弄他,上前拿醋送給他喝, \end{tabularx} \\ \\ \relax
23:37 & \begin{tabularx}{0.7\textwidth}{X} 說:「你若是猶太人的王,救救你自己吧!」 \end{tabularx} \\ \\ \relax
23:38 & \begin{tabularx}{0.7\textwidth}{X} 在耶穌上方有一個牌子寫著:「這是猶太人的王。」 \end{tabularx} \\ \\ \relax
23:39 & \begin{tabularx}{0.7\textwidth}{X} 同釘的犯人中有一個譏笑他,說:「你不是基督嗎?救救你自己和我們吧!」 \end{tabularx} \\ \\ \relax
23:40 & \begin{tabularx}{0.7\textwidth}{X} 另一個就應聲責備他,說:「你是一樣受刑的,還不怕神嗎? \end{tabularx} \\ \\ \relax
23:41 & \begin{tabularx}{0.7\textwidth}{X} 我們是應得的,因為我們是自作自受,但這個人沒有做過一件不對的事。」 \end{tabularx} \\ \\ \relax
23:42 & \begin{tabularx}{0.7\textwidth}{X} 他對耶穌說:「耶穌啊,你進入你國的時候,求你記念我。」 \end{tabularx} \\ \\ \relax
23:43 & \begin{tabularx}{0.7\textwidth}{X} 耶穌對他說:「我實在告訴你,今日你要同我在樂園裡了。」 \end{tabularx} \\ \\ \relax
23:44 & \begin{tabularx}{0.7\textwidth}{X} 那時大約是正午,全地都黑暗了,直到下午三點鐘, \end{tabularx} \\ \\ \relax
23:45 & \begin{tabularx}{0.7\textwidth}{X} 太陽變黑了,殿的幔子從當中裂為兩半。 \end{tabularx} \\ \\ \relax
23:46 & \begin{tabularx}{0.7\textwidth}{X} 耶穌大聲喊著說:「父啊,我將我的靈交在你手裡!」他說了這話,氣就斷了。 \end{tabularx} \\ \\ \relax
23:47 & \begin{tabularx}{0.7\textwidth}{X} 百夫長看見所發生的事,就歸榮耀給神,說:「這人真是個義人!」 \end{tabularx} \\ \\ \relax
23:48 & \begin{tabularx}{0.7\textwidth}{X} 聚集觀看這事的眾人,見了所發生的事,都捶著胸回去了。 \end{tabularx} \\ \\ \relax
23:49 & \begin{tabularx}{0.7\textwidth}{X} 所有與耶穌熟悉的人,和從加利利跟著他來的婦女們,都遠遠地站著,看這些事。 \end{tabularx} \\ \\ \relax
23:50 & \begin{tabularx}{0.7\textwidth}{X} 有一個人名叫約瑟,是個議員,為人善良正直, \end{tabularx} \\ \\ \relax
23:51 & \begin{tabularx}{0.7\textwidth}{X} 卻沒有附從別人的所謀所為。他是猶太的亞利馬太城人,素常盼望著神的國。 \end{tabularx} \\ \\ \relax
23:52 & \begin{tabularx}{0.7\textwidth}{X} 這人去見彼拉多,請求要耶穌的身體。 \end{tabularx} \\ \\ \relax
23:53 & \begin{tabularx}{0.7\textwidth}{X} 他把耶穌的身體取下來,用細麻布裹好,安放在鑿巖而成的墳墓裡;那墳墓從來沒有葬過人。 \end{tabularx} \\ \\ \relax
23:54 & \begin{tabularx}{0.7\textwidth}{X} 那日是預備日,安息日快到了。 \end{tabularx} \\ \\ \relax
23:55 & \begin{tabularx}{0.7\textwidth}{X} 那些從加利利和耶穌同來的婦女跟在後面,看見了墳墓和他的身體怎樣安放。 \end{tabularx} \\ \\ \relax
23:56 & \begin{tabularx}{0.7\textwidth}{X} 她們就回去,預備了香料香膏。在安息日,她們遵照誡命安息了。 \end{tabularx} \\ \\
[1ex]
\hline
\hline
\end{longtable}
$^{1}$弟子妹平安.
歡迎參加Flo Church在受難與復活之間的崇拜.
我們堂會是星期六聚會.
所以每年在受難的日子當中.
星期六是我們崇拜的時間.
今天仍然是選擇講十字架的訊息.
其實也掙扎會講復活的訊息.
但是回應和配合堂會Sorry這個月提.
我覺得要再次重提十字架訊息對我們來說也是很重要.
很高興今天看到很多弟兄姊妹在現場和我們一起崇拜.
不知道你們這兩天在街上的感覺如何.
我覺得清閒了很多.
我今天應該是看到這兩天最多人的地方.
希望你不要覺得你崇拜來這裡好像不能去旅行.
但我相信你是擺了對的位置來這裡.
特別紀念身體有缺陷.
周邊也有很多人流感.
所以戴口罩是必須的.
剛才詩歌投入過程中我自己很受感動.
因為其實我今天三點鐘已經開始聽這些歌.
但是在這麼多弟兄姊妹一起唱的時候.
特別是後面的聲音很澎湃.
不單是這邊後面的.
這邊後面也很澎湃.
是很被觸動.
但我相信基督教不是動之以情.
同樣是說之以理.
今天選的經文是路加福音的十字架經文.
我們一起重讀上帝的說話.
我們在路加福音第23章第32節開始.
又有兩個犯人和耶穌一同帶來處死.
到了一個地方名叫築留地.
就在那裡耶穌釘在十字架上.
又釘了兩個犯人.
一個在左邊一個在右邊.
當下耶穌說.
父啊赦免他們.
因為他們所做的他們不曉得.
丁丁就尖叩翻他的衣服.
百姓站在那裡觀看.

$^{41}$官府也恥笑他說.
他救了別人.
他若是基督上帝所揀選的.
可以救自己吧.
丁丁也戲弄他.
上前拿槍在他後說.
你若是猶太人的王.
可以救自己吧.
在耶穌耳上有一個牌子寫著.
這是猶太人的王.
那和丁丁的兩個犯人.
有一個譏笑他說.
你不是基督嗎.
可以救自己和我們吧.
那一個就應聲責備他說.
你既是一樣受刑的.
還不怕上帝嗎.
我們是應該的.
因我們所受的.
與我們所做的相稱.
贊這人沒有做過一件不好的事.
就說耶穌啊.
你得過降臨的時候.
求你紀念我.
耶穌對他說.
我實在告訴你.
今天你要和我在樂園裡.
我們一起禱告.
天上帝每當我們打開你的說話.
再重溫當日的情景的時候.
求主你今天營救我們.
對我們的說話.
以至我們再一次明白到.
你掛在木頭上.
死在十字架的緣由.
是為了什麼.
而我們每一個面對十字架的時候.
同樣要做出抉擇.
同樣要面對當中要決定的事情.
求主你幫助.

$^{81}$開我們的心.
開我們的耳.
以至我們明白你的心意.
我們祈禱奉耶穌的名求.
阿們.
十字架我相信大家不陌生的.
有很多電影節目可能現場.
你戴的飾物都是有十字架.
或者是你周遭環境當中都不難見到.
今天不是說十字架的飾物的symbolic meaning.
是實際上當日的十字架要面對的什麼.
你看到各個他山上有三個十字架.
最少你看到有三個十字架在當中.
第一個十字架就是耶穌基督已經被掛在當中.
另外就有兩個銅釘在十字架的人.
你會看到很多壁畫.
或者很多基督教關於復活節的畫.
當中都有三個十字架.
三個十字架有三個不同的意思.
三個不同的對答.
所以從文字上要處理的時候.
下一張請.
你會看到在經文抽一些對話當中.
你就不難理解其實說話帶出的力量是什麼.
當我們要跟別人聊天的時候.
其實你都會想了解他說什麼.
英文有一句說話.
You read my mind? Do you read me?.
在read的過程當中其實是在讀你的字.
其實是在讀你的說話.
很多時候你會發覺為什麼跟一些人不容易溝通.
因為他每個字你都聽得到.
但你不太聽得懂他整句意思是什麼.
因為是不是沒有邏輯.
還是有些用詞你不太明白.
其實都有點複雜.
但今天你會看到在選了這段經文當中.
有不同的對話其實要表達一個比較重要的訊息.
我們看下一張powerpoint的時候.
你會看到耶穌開頭的時候說了一句話.

$^{121}$就是父啊 聖啊 他們 因為他們所做的他們不曉得.
其實今天我們自己做人做了一段時間.
有些很小的.
但我相信大部分都是年輕人.
或者已經有一段年齡的弟兄姊妹.
其實你是不是很清楚自己每一個決定在做什麼.
其實你是不是很清楚你自己每一個決定的前因和後果.
而你還沒有承擔計算過之後承擔的後果.
真的 我相信這幾年在香港我們一起經歷的過程當中.
你會發覺你每一天出街.
你每一個網上的comment.
你每一個你要做的表達.
其實你都有心思熟慮過.
你都會想一下值不值得做.
還有是不是這樣做.
還有做完之後其實那件事是不是你可以控制到呢.
從來你會發覺可能過去你本身是一個很好的環境.
但時態轉了 身份轉了.
還有你的心境轉了的時候.
你會發覺很不容易.
但都是當下要做決定.
十字架的環境今天要面對的情況就是.
耶穌掛在木頭上.
祂望著下面的人.
祂第一句說話是說.
父啊赦免他們.
因為他們所做的他們不孝得.
這個是很重要.
其實很多人不知道自己為什麼要這樣做.
因為跟風這樣做.
因為我不做我有問題.
我怕有問題我這樣做.
我知道這樣做.
但我選擇不抗拒我照做.
其實我相信你過去這些日子.
你不難見到這樣的狀況.
我不知道你這兩天看了什麼貼文.
你都會見到一些貼文.
就是當初要做了什麼.
現在要承擔結果.

$^{161}$可能要倒閉.
可能要重新洗底.
我想問一件事.
其實當初要做的時候.
他自己知不知道呢.
我這幾年常常都問自己.
和跟自己說.
每一個做的決定.
每一句說話.
和要表達是什麼.
讓人知道的訊息.
我是要想清楚.
但我不是Play Safe的做法.
我是清楚別人問我的時候.
我是告訴當時的人聽.
我為什麼做這個決定.
這個是很不容易的.
我在當中都曾經和大家說過.
在過去日子特別是很困難.
大家香港很困難的日子的時候.
每天早上起來看著鏡子拍一拍自己.
就是今天要醒目的一個人.
我相信在這些日子當中.
我今天下午和一個教務員傾談的時候.
都說當坊間說復常的時候.
教會是不是都是復常呢.
是復常在之前.
很笑聲滿載溫馨的日子.
還是真的重新面對新的環境.
新的香港環境之下.
如何默會呢.
這個是很真實的事情.
耶穌說.
「父啊,世面貪門,因為貪門所執,貪門不曉得.」.
但我們今天就是問我們做事.
我們知不知道.
你會發覺有些人是這樣的.
官府就笑他.
他救了自己?.
他救了別人?.

$^{201}$如果你是基督的話.
你是上帝揀選的.
你救自己吧.
丁丁就笑他.
如果你真的是猶太人的王.
你救自己吧.
你會看到他們是忽視無視上帝的能力.
很多人問為什麼要相信耶穌呢.
你覺得耶穌很厲害.
其實是不是覺得.
或者知道耶穌很厲害的人就會相信耶穌呢.
未必的.
是不是知道基督教是很好的人.
人們就會相信呢.
未必的.
但真正你問問自己.
你是因什麼原因相信耶穌.
你是因什麼原因覺得基督教是可信的.
你願意花時間.
花你的生命在當中追隨.
真的要問這個問題.
其實在這段經文.
你會看到那些笑的人.
其實他不是未見過.
如果他說得出你是基督.
你是上帝揀選.
你是猶太人的王.
這些對話當中.
其實他一直都在見耶穌之前做的事.
我沒有讀《約翰福音》.
但《約翰福音》在第十二章第一節的時候.
是在說.
雨戰前六天耶穌再去伯大利.
伯大利是什麼地方呢.
即是上星期六.
如果說日子.
上星期六的時候.
耶穌再去伯大利.
然後他就進聖城.
然後撒冷準備進城.

$^{241}$就是這個星期的日子.
耶穌再去伯大利的意思是什麼呢.
就是去見一次瑪大,瑪利亞和拉撒路.
這三個弟弟.
在《約翰福音》第十一章是在說什麼呢.
就是拉撒路他死而復活的經文.
耶穌在臨進城之前.
再去過瑪大利那家人的時候.
再見他們一家三口.
然後就用真拿大香膏去告耶穌.
但《約翰福音》第十一章.
說拉撒路的事件的時候.
你會見到他們死了的弟弟拉撒路.
他們很傷心.
但耶穌說他們是睡著了.
但他們都不是很相信.
他說你相信復活吧.
耶穌說了一句話.
「復活在我,生命也在我,信我的人,雖然死了也必復活.」.
但瑪大的答案是什麼呢.
瑪大的答案是.
「主啊,我信在末後的日子是會復活的.」.
是信了,末後嘛.
但耶穌不是這樣.
耶穌沒有和他們爭辯.
耶穌說拉撒路在哪裡.
你帶我去吧.
他們就帶拉撒路去.
不要去,已經是四天了.
已經可以發臭了,不要去.
他們照樣帶耶穌去.
耶穌就救了拉撒路.
拉撒路就復活了.
很多人看到的事件.
但是不是很多人看到復活的神蹟.
是不是很多人看到耶穌說得出做得到的事情的時候.
那些人就會相信耶穌.
經文第56節說一句話.
「從那日起,他們就相已要殺害耶穌.」.
不是每個人看到神蹟.

$^{281}$看到復活這麼厲害的神蹟.
就會相信耶穌.
不是每個人看到神蹟的騎士.
大家就會覺得那件事是屬於我.
我會贊成下去.
不是的.
你會看到有些人是這樣更加不相信.
因為他令到他地位上.
能力上受到威脅.
他都是在想自己.
所以你會看到.
今天我們看到耶穌釘在十字架上.
他們的說話要表達.
如果你真的這麼厲害.
意思就是你是基督.
基督是猶太人期盼等待的彌賽亞.
要救他們的位置.
如果你真的基督是上帝的子孫.
你救自己吧.
你展示給我看.
你跳下來吧.
你將你的能力彰顯吧.
你再行一個神蹟出來吧.
冰釘也說.
如果你真的是猶太人的王.
你救自己吧.
在希臘的文化.
在羅馬的政權之下.
每一個王都為自己的王權爭戰.
你如果是猶太人的王.
你就救自己吧.
你會發覺每一個人去面對耶穌的時候.
耶穌釘在十字架上.
他沒有跟祂抗爭.
沒有跟祂做任何的申辯.
耶穌說了一句話.
父啊.
赦免他們.
其實他們自己在做什麼.
他們自己不知道.

$^{321}$親弟姐妹.
如果你面對耶穌的時候.
你跟耶穌說了什麼話.
相信在座很多人都做了基督徒一段時間.
如果你身邊都遇到一些不相信耶穌的人.
或者是你跟他分享了福音很久.
他都沒有回應的時候.
其實他當中不相信什麼呢.
當中他不明白的是什麼呢.
其實可以慢慢繼續去討論.
每個人面對耶穌.
他覺得基督教都是很感性的.
我就不是很感性.
我是很理性的.
其實我覺得不是分感性理性.
上帝做人就有左右腦.
右腦是一個情感.
左腦是一個邏輯.
我們每個人都可以有這個.
其實不是分感性理性.
其實就是他不認清耶穌是什麼.
或者他不接受的耶穌是什麼.
這個就是大家可以繼續對話下去.
耶穌沒有跟他們去抗爭,抗辯.
耶穌選擇沉默.
下一章你會看到彼得怎樣看待這個情景呢.
彼得前書第二章21節是這樣說的.
你們夢照原是為此.
因為基督也為你們受過苦.
給你們留下榜樣.
叫你們跟隨他的腳蹤行.
他並沒有犯罪.
口裡也沒有鬼.
他被罵不還口.
受害不說威嚇的話.
只將自己交託.
拿按公義審判人的主.
其實耶穌很清楚.
他自己掛在木頭上的恩由和目的是什麼.
彼得怯怯面散居在不同地方的基督徒.

$^{361}$你們也會遇到這些逼迫.
這些苦待.
這些挑戰.
這些不解.
但是彼得提醒你們夢照原是為此.
因為耶穌基督掛在木頭上那一天.
用什麼反應.
他留下榜樣給我們看到.
就是他沒有錯.
他沒有騙人.
你罵我.
你會看到受苦的經文裡.
要帶出耶穌對話當中.
很多耶穌都沒有抗辯.
但當彼得問你是猶太人的皇媽.
你說的是.
身份上耶穌一定會確認.
有工作上耶穌一定會表達得清楚.
但是其他東西.
耶穌不多談.
因為上帝天父看著他所做的事情.
身體的前面我們也是.
我們這幾年生命生活都改變很多.
但如果你真的每天都知道自己在做什麼的時候.
上帝都知道你在做什麼.
不需要被論斷.
你不需要論斷自己.
有關論斷這個課題.
我在上一個月有兩講.
我都說過.
包括上帝的視覺和演員的道德修養.
裡面都說過論斷這個內容.
今天很多人會留言.
在你的媒體.
在你的工作上有平頭品足.
但是你作為一個跟隨上帝的人.
你清楚自己在做什麼.
就繼續做吧.
天父都看到.
下一段經文我們看的是.

$^{401}$你會看到那些犯人開始一句一句地說.
但是旁邊那兩個十字架的人強盜.
他們的對話就成為第二段的焦點.
我們再下一張.
你會看到兩個犯人的對話.
他們自己說.
第一個犯人說.
你不是基督嗎.
你可以救自己和我們吧.
這次就跟耶穌說.
如果你真的救了主.
你就救我們兩個吧.
我們意味著不如三個一起跳下去吧.
就離開吧.
整件事是希望.
如果真的這麼厲害.
不如走一個神蹟就便宜我們兩個吧.
但是那個囚犯被另一個囚犯的說話罵他.
他就說.
你何時一樣受刑還不怕上帝嗎.
這是第一句說話.
這句說話我不知道你感受如何.
但是我聽下去和你會看到文字上.
你都一起受刑.
你都不怕上帝.
我不知道你怕不怕上帝.
現在沒有什麼對話空間.
你問問自己你怕不怕上帝.
有些人怕有些人不怕.
因為你會問.
什麼是怕.
怕我就不敢回教會.
什麼是怕上帝.
你要問我怕不怕上帝.
如果我怕上帝我應該不敢犯罪.
我怕上帝會擊殺我.
說個笑話.
我們每個星期都會把台上的椅子搬到下面.
那你每張椅子下面都有本什麼.
後面的勁敗隊在笑.

$^{441}$勁敗隊搬的椅子很珍而重之.
因為每張椅子裡面都有本聖經.
一怕掉了本聖經就說上帝會擊殺你.
你糟蹋了上帝的話在地上.
這個笑話在勁敗隊裡面每個星期都會笑一笑.
如果你這麼看重上帝的話你又不靈修.
我舉個例子.
這個氣氛應該是嚴肅一點.
如果你不怕上帝的話.
如果你這麼怕上帝的話.
你應該每個星期都很準時崇拜.
但其實生活中你又不是這麼怕上帝.
上帝很寬容我的.
上帝很有大愛的.
你還不怕上帝媽這個說話其實可圈可點.
你問你自己有多怕上帝.
如果你這麼怕上帝的話.
你很多事情都很謹慎.
如果你不怕上帝的話.
你每天就擺上帝在哪裡呢.
上帝我去打機你先在這裡休息.
外面很危險的.
還是上帝我要喝一杯東西.
外面的人說很多髒話你不要聽.
你回家等我.
是不是這樣呢.
我希望好像有一點點嬉笑.
但我希望你明白到.
其實我們是不是真的這麼怕上帝呢.
可能不是的.
還是你心中覺得有問題就找上帝.
沒有問題就不要打擾上帝.
上帝隨傳隨到的.
但真的怕上帝的人.
其實都不是一個生命比較穩定的環境.
可能大家知道我以前在醫院工作.
其實上下病房的時候.
久不久都會撞到軟木.
我通常撞到軟木都不會打擾軟木.
因為軟木上了病房.

$^{481}$都是有些事情要做的.
都會聊聊天.
不一定每次都要搶救.
靈魂啊靈魂.
都會聊聊天.
但你會看到.
其實軟木很多時候跟病人聊天的時候.
你會看到不同病人真的有不同的反應.
有些人仍然覺得.
我聽過很多次了.
多謝你.
不如找下一個吧.
但有些人就會在把握那一刻.
就跟上帝說.
他要缺志.
你不是那個階段.
你不知道自己的生命去到哪個位置.
但人真的去到生命一個位置的時候.
他無力.
他無助的時候.
他面對自己生命的時候.
生命主的出現.
他會不會把握.
那時候他什麼叫做害怕.
你明白我的意思嗎.
你真的明白嗎.
可能你都還沒有明白.
因為我見過一些弟兄姊妹.
真的死過活生生的時候.
之後很愛主的.
我不是要大家死過活生生才愛主.
但不要等到你很害怕上帝的時候才會轉.
其實一直有機會有恩典.
不是必然的.
我自己在2006年的時候.
去過外地一個地方上課程.
應該是2004,2005年那段時間.
去過外地上一個課程.
那時候因為還沒有正式全時間服侍的時候.
很輕鬆.

$^{521}$但我見到有弟兄姊妹.
她什麼筆記都抄.
上課的時候抄了筆記.
我見了一天兩天.
我說其實是講普通話的.
我問她為什麼你經常要抄那麼多筆記.
她說我要記住要講的內容.
我說其實不用抄的.
如果你想上這堂錄音可以訂的.
不用抄的.
她又給你一張光碟.
她說謝謝.
第三天我見她還在抄的時候.
我心想她是聽不懂我的普通話.
還是她看不到別人說外面可以錄.
其實熟了三天.
她見到這個人那麼煩.
她回答我.
我是幾條村供她一個人去馬來西亞上課.
幾條村教會供她一個人去馬來西亞上課.
我自己一個人就自己上課.
我才明白到原來她沒有錢買那些CD.
她是抄了所有的.
變成了逐字藤稿.
寫下所有老師說的話.
附帶手帕.
回去她要教我幾條村的東西.
那種敬畏.
我就馬上俯伏在她面前.
能夠見到上帝的時候.
或者能夠見到那件事是真而眾知的時候.
不是每個人都懂得做這個俯伏.
幾條村的人供她一個.
她就回去教那班弟兄姊妹.
人生要去到一個位置的時候.
你怕不怕上帝呢?.
這個囚犯他怕上帝.
我們活該的.
因為我們受的和我們所做的是相稱的.
但這個人連一件錯事都沒有做過.

$^{561}$他很清楚一件事就是耶穌不應該在這裡.
但是他在這裡.
他很清楚一件事就是.
他們所受的刑罰是應當的.
因為我用了我的選擇方法.
過過生活.
做過那件事.
我現在受的刑罰是叫做履行他罪有應得的結果.
所以他很清楚一件事就是.
每一個人要帶出去上色的人.
每一個人去到上帝面前.
他都要拓出他所有的事情.
我相信大家信主的時候.
都會說過一件事叫做.
耶穌會再來的.
耶穌再來的時候會有審判的.
基督教其實很清楚要讓每一個信主和不信主的人都知道.
會有審判的.
不是代表我們不用.
上主不用.
我們都會.
我們都會在白色大寶座前面將我們所做的事情.
但我們所做的錯事.
耶穌基督已經覆蓋了.
不是到那時候才.
但不代表我們所做的事.
我們不用跟上帝說.
剛才在敬拜的時候都在說一件事.
就是相投.
我們做過不同的事.
信了主之後我們都會犯不同的錯.
信了主之後我們都會有錯誤的決定.
我們都會求上帝的憐憫.
我們求上帝的寬恕.
這是很重要的事情.
你能夠將你所說的事情說給別人聽.
很難說的.
有時候你不想讓我知道.
但你又會不會選擇將你的錯跟上帝說.
但你又不想跟上帝說.

$^{601}$因為不想上帝認罪.
好像很複雜.
上星期John說的訊息的時候.
我們不像天主教.
天主教會有些告解.
但其實基督教在禮儀的教會裡都有.
例如禮賢會 崇禎會 信義會 聖公會.
他們在禮敘裡都會有一個赦罪的環節.
禮敘當中.
就是主席會跟頂子妹一起有認罪討問.
當頂子妹一起跟主席有認罪討問的時候.
主席就會宣洩.
你們若說自己無罪.
便是自欺真理不在你們心裡.
你們若認自己的罪.
上帝是信實的 是公義的.
必要赦免你的罪.
洗清你一切的不義.
這是主席的宣洩.
接著他會說主與你們同在.
會中說也與你們同在.
主的平安尚與你們同在.
上帝將他的宣洩賦予給每一個願意認罪的頂子妹.
上帝的平安臨到當中.
主席就會下台跟頂子妹握手.
我以前做主禮的時候很喜歡這個環節.
因為很感受到握手的過程當中的親密.
不過現在可能不要握手了.
主禮完結了這個問安環節的時候.
才進入聖道禮聽上帝的話.
面對上帝的時候.
他很清楚我該死.
這是我做的事我應得的.
但是我去到你的面前.
因為你是沒有做那件事.
耶穌被掛在木頭上.
是承擔了眾人的罪.
不同人的罪.
而每一個去到上帝面前的我們.
你有沒有和上帝和盤托出你所犯的罪.

$^{641}$你求上帝饒恕.
很多時候我們都會和上帝爭辯.
這些小事上帝不介意應該不用站在這裡.
有些事情大的時候不知道怎樣和上帝說.
所以最後最後沒有說.
小的又覺得不用說.
大的又不知道怎樣說.
於是就沒有說.
但是我希望大家面對你自己的十字架.
你會不會求上帝饒恕.
所以他最後說一句.
你得國降臨的時候求你紀念我.
我不知道大家怎樣看他的說話.
但是我看他的說話的時候.
我覺得這個強度其實是超強.
他有一個透視.
有一個遠像看到.
他真的相信上帝的國.
是在耶穌傳道當中展現出來.
耶穌在地上周遊四方.
除了行善事.
醫治各樣的病症.
就是傳講天國的福音.
上帝的國在其中.
這個強度可能在中間聽過.
但是他最後面對耶穌那一刻.
他說了最後的說話.
耶穌啊 你得國降臨的時候.
求你紀念我.
他相信耶穌的國會展現.
這是信心.
我們都相信上帝的國在當中.
我們相信上帝會.
耶穌會再次回來.
但是當日第一個跟耶穌說的.
就是這個強度.
很希望我們再重看這段經文的時候.
你會更加明白到.
其實這個強度.
他有三樣事情他做了.

$^{681}$第一樣事情就是.
他知道他生命不在他掌握當中.
他面對難處.
最後關頭的時候.
他擺一機會.
他面對上帝.
說回第二樣事情.
說回他自己所做的錯事.
他願意悔改.
第三樣.
他求寬恕之外.
他求上帝的國讓他進入.
這個就是那個強度.
他背著他的十字架.
被釘過程當中.
他面對這三個決定.
同樣都是我們.
耶穌給了他最後的應許.
我實在告訴你.
今天你要跟我再落遠了.
我實在告訴你.
這句話我相信大家都不陌生.
在科幻書裡面這句話通常是.
我鄭重地告訴你.
或者我很謹慎地告訴你.
今天你就跟我再落遠了.
不知道你看這句話也好.
我經常都問.
看了經文大家的反應是什麼.
哇 那就正了.
這是第一個被上帝肯定入落遠.
這個落遠.
你知道這個落遠不是那邊.
大嶼山的落遠.
你會看到那個落遠就是.
上帝肯定他一定會.
verbal這樣confirm的.
是不是.
你會看到.
哇 真的好得無比.

$^{721}$你也想好成這樣.
是不是.
但有些人看這句話的時候.
就一般般.
就是說.
哇 搞錯.
這麼兒戲.
這句話.
或者這麼不公平.
前兩天跟我老婆逛街.
經過一間餃子店.
我老婆說想吃.
我就陪她進去吃.
她在選餃子的時候.
選完就坐下來吃.
選完就在等.
我老婆說這麼多姐姐在包餃子.
四個就很多工資.
就算一下算一下人手.
我就聽到那四個包餃子的姨姨在說話.
就在說這段經文.
哈哈 o下.
o下 這樣.
不是基督教開的.
不是 她說了一段經文.
其實我是聽不到她說什麼.
但當我要預備講章的時候.
這些經文都上路了.
突然間她說了一句.
哪有這麼兒戲和這麼不公平.
你有沒有聽過.
那個囚犯最後在耶穌旁邊說了一句.
耶穌就跟他說.
今天就在天堂.
哇 多麼沒天理.
這些經文都上路了.
她說了一句.
我當然是質疑而聽.
她就說多麼沒天理.
我聽不到其他三個姨姨有回她.

$^{761}$她就繼續說.
你說是不是.
如果那個囚犯做到最後的時候.
十惡不赦突然間一個說.
我今天就跟你去天堂.
有沒有搞錯.
起碼都要罰回去.
這樣不公平.
是不是.
為什麼不是.
當然是.
罪有應得這件事一定.
從阿姐的角度來說是合理的.
哪有這麼大隻甲拿.
現在我們不說甲拿.
哪有這麼便宜.
按他們的常理來說.
因為要判刑.
講一句就可以超脫.
這樣就很划算.
你會看到這是人慣常的思考模式.
就是要論功行賞.
要按功德.
但是你會看到今天這個是救恩.
什麼叫救恩.
救就是挽回.
他已經死了.
我就挽回他.
什麼叫恩.
恩典是.
是恩典.
就是不配得.
配得那些叫功架.
不配得那些叫恩典.
救恩就是要挽回他.
令他得著不配得的事情.
你看到這個.
他知道自己罪有應得.
他看到這個.
他看到上帝的國會在這裡.

$^{801}$他看到這個.
他有能力.
但他選擇不用這個能力救自己.
他為其他人的罪釘在十字架.
耶穌的時候.
這個是信心.
所以耶穌看得出.
我鄭重告訴你.
你今天就和我在樂園裡.
祂不是隨便的.
弟兄姊妹我們能否做到這三件事.
你懼怕上帝嗎.
你做錯事有沒有跟上帝說對不起.
你看到上帝的國.
你願意上帝的國在其中.
你有沒有參與上帝的國.
那個囚犯做了.
耶穌給他肯定.
他肯定了耶穌的國.
耶穌的國也肯定他.
今天你就和我在樂園.
親弟兄姊妹.
十字架是什麼地方呢.
十字架是一個沒有神蹟的地方.
但最大的神蹟就是上帝的兒子.
降世為人釘在十字架上.
十字架是一個沒有光的地方.
是一個很幽暗的地方.
但是那光是真光.
照亮一切身在世上的人.
那種那光就在十字架當中出現.
十字架是什麼地方.
十字架是一個沒有自己辯護的地方.
耶穌是避麻不還口.
但是十字架就彰顯.
最大上帝的能力的地方.
就是救恩的出現.
我盼望大家.
順著主的好來.
你要從述和重溫.

$^{841}$十字架不是一個普通的地方.
十字架是我們救恩的出處.
十字架也是我們上帝的大能.
所以保祿很清楚.
我不傳別的.
我只誇耶穌基督並他釘十字架.
這是保祿傳道最核心的訊息.
很希望每年我們能夠有經歷復活節的日子.
除了我們享受最長的假期之外.
十字架也是我們思想上帝的說話最核心所在.
十字架從來都是一個很大張力的地方.
我希望今天的訊息不僅僅是動之以情.
我希望你更加明白上帝的說話是說之易理.
每一個人面對十字架的時候.
你選擇是一個強度去挑戰耶穌.
還是選擇另一個強度去歸向耶穌.
而耶穌在其中就彰顯十架的大能.
救恩的開始.
我們一起祈禱.
主耶穌 多謝你愛我們.
多謝你給我們機會認識你.
多謝你呼召我們成為你的門徒.
多謝你揀選我們能夠有福分.
在我們有生命氣息的時候去接受這個救恩.
今天我們打開你的說話.
你營救對我們說話.
因為這個古老的十架仍然發揮效力.
雖然歷史已經成就.
但今天營救對我們是很重要的提醒.
我們眾弟姊妹都是罪人.
但耶穌因為你愛我們的緣故.
你死在十字架上.
這也是你告訴我們.
你透過十字架向眾人發出的邀請.
願意我們當中有未認識上帝的.
又或者我們在網絡上轉發.
讓更多人知道耶穌基督的心意.
這也是我們希望他們能夠在有生的日子.
能夠接受耶穌基督.
多謝主你讓我們一起崇拜.

$^{881}$願主你坐直.
享受你當德的榮耀.
我們祈禱感恩.
奉耶穌基督的名求.
阿們.
在今天崇拜開始的時候.
用了啟示錄的宣召經文.
在結束今天信息的時候.
我同樣用啟示錄的信息.
跟大家說一句結束.
啟示錄第三章20節是這樣說的.
看那我站在門內叩門.
若有聽見我聲音就開門的.
我要進他那裡去.
我與他 他與我一同坐直.
讓我們一起踏進光明之向.
\newpage



\section{耶利米書 15:15-21-20230415}
\label{sec:3PY1nwdp_0k}
\textbf{【流堂崇拜】致歉-我們的九零後及千禧後|耶利米書15\_15-21|20230415 [3PY1nwdp\_0k]}
\newline
\newline
連結: \href{https://youtube.com/watch?v=3PY1nwdp_0k}{\texttt{ https://youtube.com/watch?v=3PY1nwdp\_0k}} ~~~~ 語音日期: 2023-04-15 
\newline
\newline
\hyperref[sec:v4hE6GM4QsI]{\small{< < < PREV SERMON < < <}}
~
\hyperref[sec:index_chronic]{\small{[返順時目]}}
~
\hyperref[sec:index_scriptual]{\small{[返順卷目]}}
~
\hyperref[sec:S0X_1Lh_dHA]{\small{> > > NEXT SERMON > > >}}
\newline
\newline
耶利米書 15:15-21-20230415
\newline
\begin{longtable}{cl}
\hline
\hline
章節 & 經文 (和合本修訂版)\\
\hline
15:15 & \begin{tabularx}{0.7\textwidth}{X} 耶和華啊,你是知道的;求你記念我,眷顧我,向迫害我的人為我報仇;不要把我取去,因你不輕易發怒,要知道我為你的緣故受了凌辱。 \end{tabularx} \\ \\ \relax
15:16 & \begin{tabularx}{0.7\textwidth}{X} 耶和華-萬軍之神啊,我得著你的話就把它們吃了,你的話是我心中的歡喜快樂;因我是稱為你名下的人。 \end{tabularx} \\ \\ \relax
15:17 & \begin{tabularx}{0.7\textwidth}{X} 我並未坐在享樂人的會中歡樂;因你的手,我就獨自靜坐,你使我滿心憤慨。 \end{tabularx} \\ \\ \relax
15:18 & \begin{tabularx}{0.7\textwidth}{X} 我的痛苦為何長久不止呢?我的傷痕為何無法可醫,不能痊癒呢?難道你以詭詐待我,像流乾的河道嗎? \end{tabularx} \\ \\ \relax
15:19 & \begin{tabularx}{0.7\textwidth}{X} 所以耶和華如此說:「你若回轉,我就使你歸回,站在我面前。你若能將寶物和無用之物分別出來,你就可以當作我的口。他們必歸向你,你卻不可歸向他們。 \end{tabularx} \\ \\ \relax
15:20 & \begin{tabularx}{0.7\textwidth}{X} 我必使你向這百姓成為堅固的銅牆。他們必攻擊你,卻不能勝過你;因我與你同在,要拯救你,搭救你。這是耶和華說的。 \end{tabularx} \\ \\ \relax
15:21 & \begin{tabularx}{0.7\textwidth}{X} 我必搭救你脫離惡人的手,救贖你脫離殘暴之人的手。」 \end{tabularx} \\ \\
[1ex]
\hline
\hline
\end{longtable}
$^{1}$好,頂智慧晚安.
昨天五月天來了.
石頭說了一句話.
五月天幾年沒來了.
他說香港變了很多.
他說有些東西沒有變.
昨天阿凡和方圓.
有很多東西值得我們去思考.
起碼仍然覺得.
香港人仍然為香港人說話.
無論你認不認同.
在制度當中你可以做到多少.
但你仍然相信制度當中.
只不過有人將很多螞蟻搞來搞去.
我們仍然告訴自己.
可以彼此相愛.
彼此堅持.
尤其是今天鋪天蓋地.
意識形態的高舉.
到底我們今天應該怎樣走下去.
所以今年我自己選擇的經文是.
《耶利米書》.
在一月份的時候.
我記得說了一段經文是《耶利米的呼召》.
講《耶利米的呼召》的時候.
其實想帶出一樣東西.
我們面臨在一個拆毀的年代.
我們正在建立很多年來的事物.
我們不期然地面對一個拆毀的時間.
不容易經歷.
彷彿《耶利米的呼召》跟我們說話一樣.
我們試試看.
今天我選了另一段《耶利米的經文》.
我第一次做Canva.
就是.
Canva is the king.
這樣.
他十五節是這樣說的.
他說:你是知道的耶和華.
紀念我和顧念我.

$^{41}$從逼迫我的人中為我報仇.
不要寬容他們當他們取我的命.
你是知道的.
我為你的緣故受了凌辱.
《耶利米》講這一段的說話.
其實是在講一個很艱難的時代.
在講一個他要宣告這個地方會被拆毀.
這個地方會被拔出.
這個地方會不容易地生活下去.
當要講這些的時候.
其實大家知道的話.
其實很多人不覺得這些說話合聽.
他們要說好耶路撒冷的故事.
無論是祭司.
無論是大祭司.
無論是宗教的領袖.
他們都決意要將好的故事繼續說下去.
所以那些講不好故事的疑人.
成為了眾人針對的對象.
《耶利米》經歷了很多之後.
他發出了這個祈禱.
基本上十五節,十六節,十七節,十八節.
是他祈禱的一個很重要的內容.
異象加附在耶利米身上的時候.
不是在講他做了很多偉大的事情.
你看這一節單單的聖經.
他在講救命的禱告.
異象不會帶來他風光的日子.
他蒙召成為上帝使用的僕人.
不是帶來一個可以在很高的位份之中.
叱吒風雲.
他沒有這些野心屈服.
他連以利亞可以在加密山上.
能夠稱霸,打敗先知.
這些僅有的威嚴和尊榮他都沒有.
耶利米從來沒有神跡歧視異象伴隨著他.
有的,只不過在異象裡面.
他受盡凌辱.
在十六節他繼續再講一些東西.
他說:我沉著你的話語,就吃下他們.

$^{81}$你的話語於我成為歡喜和我心中的喜悅.
因你的名字於我被呼喚.
萬鈞耶和華.
這句話跟十五節很不同.
十五節明明在講他受盡凌辱.
很慘,很艱難.
但十六節你看他有多少個喜悅.
即是歡喜.
喜悅這兩個字眼出現了.
他突然說.
吃了他的話語之後.
突然間他裡面歡喜和快樂.
吃了他的話語是在講什麼呢?.
應該是在講一月的時候.
我講關於耶利米一章的故事.
耶和華要將他的話.
要怎麼講放在耶利米口中.
所以彷彿他吃了耶和華.
給他的說話之後.
他就從此以後歡喜快樂.
但我要表達的是.
十五,十六章其實是一個很變態的兩節聖經.
十五節其實是在講他很不開心.
受盡凌辱,很淒涼.
十六節突然間他吃了上帝的位置之後.
他就很開心.
到底他開心還是不開心?.
到底他歡喜還是不歡喜?.
到底他被人攻擊.
甚至被打,被成刑.
被凌辱得他很淒涼.
到底哪一件事才是真?.
十五,十六節這兩節聖經.
其實是一個很強大的張力.
張力是意象.
當意象到一個人身上的時候.
其實不是在講很開心,很偉大的事.
對耶利米是一個很凌辱的事.
是一個很差勁的事.
他不知道為什麼要做這麼傻和傻的事.

$^{121}$但他總之做了.
做完之後沒有好處.
經常被人打,被人欺負.
但轉眼之後.
十六節就說一些奇奇怪怪的事.
不是的,我吃了上帝給我的東西之後.
說完之後我也有力量.
有了力量之後就充滿了.
然後就很開心快樂.
然後他說.
因你的名字於我被呼喚.
其實這句話是因為耶和華的名字.
這個名字就代表耶利米屬於耶和華.
所以他決意屬於耶和華.
所以他歡喜快樂.
這兩節聖經怎麼理解?.
他就是一個很精神錯亂的人.
你會發現所有服事主的人.
只要有上帝異象的人.
他生命裡都充滿著這兩個張力.
你服事主的時候.
你很淒涼,很慘.
但當你決意說我不做了.
我做其他事.
不知為什麼就轉過頭來.
突然之間上帝就說沒辦法.
我都要跟著你走.
然後就堅持.
堅持了多久又被人打到抽筋.
然後很不開心又不想做.
不做之後又突然之間.
沒辦法了,都要繼續走.
大體上.
15,16只是說一個跟從上帝服事的人.
一個很必然的心態.
而這個張力其實基本上.
是一個正常服事主的人必然會經歷的事情.
尤其是當那個異象是在說.
拆毀的時候.
出現的時候和毀壞的時候.

$^{161}$17,18字我們再看一看.
他說我沒有坐在那些燕落的會中.
都沒有歡樂.
他說因你的手我獨自靜坐.
他說因你.
又說是你使我充滿著怨恨.
他說為什麼我的痛苦變得這麼長久.
他說我傷痕無法醫治.
拒絕痊癒.
難道你當我如鬼一樣的河水.
不可信賴.
那些希伯怎麼翻譯呢?.
我不說了,不詳細說那些.
我想表白的是.
17,18只是說.
當一個侍奉主人經歷了多年.
當你寫耶利米書的時候.
他侍奉了40年,大約40年.
他寫了40年的歷程裡面.
其中15,16節.
說了張力圓之後.
他17,18節想表達什麼呢?.
原來那些張力圓之後會有一個後遺症.
那個後遺症是什麼呢?.
是他的痛苦長久.
是他的傷痕治癒不了.
他覺得彷彿上帝騙他.
他說他待他如鬼一樣的河水不可信賴.
上帝,你這麼枯燥嗎?.
你叫我做這些事嗎?.
做這些事怎麼會帶來這麼多傷痕呢?.
幾個星期前.
我認識了一個二十多年的牧師.
他去年邀請我去教會講道.
他叫我星期六日去兩邊道.
星期六日去講道.
他講道的時候好像是四點崇拜.
結果他叫我早點來.
三點前來.
我問他來幹什麼呢?.

$^{201}$他說聊聊天.
我答應了.
大約兩點到,聊天.
他說喝東西了.
我以為他真的喝東西了.
喝東西了,我就去喝.
喝東西了.
他不是去喝東西.
那裡是一些很….
不要說太多,仔細一點.
不是喝東西.
他隨便去士多買點東西.
買東西和我喝東西.
我以為起碼茶星人那些.
我以為茶星人也有.
原來沒有.
我隨便買一杯解茶.
就當喝東西了.
誰知道他就說.
喝啤酒,喝喝喝.
他拿了兩罐啤酒.
拿起來.
坐下來的時候.
他找了附近的自修室.
已經空了.
空了自修室.
已經沒有人租了.
他說進來吧.
他說兩罐啤酒放在這裡.
喝吧.
我心裡想.
你不用講道,你喝吧.
因為我講道.
你灌醉我,你真是大件事.
我面紅著說道.
你不是很嚴重嗎.
結果我沒有中計,我只喝了解茶.
他就不停地喝兩罐啤酒.
他喝完之後就說話了.
我想說他做了二十多年牧師.

$^{241}$是很老的朋友.
認識了很多年,二十多年.
他喝醉了,臉都紅了.
那些年輕人,二十多三十歲的.
相信不過我.
我心裡想,搞什麼鬼呢.
因為我認識他是二十多年前到今天.
基本上他帶大了很多年輕的童工.
青少年.
基本上他在公書教學.
如果他讀不成書.
他說有一個我還認識的童工.
他現在離開了,去了英國.
他說哪裡有童工做我們教會.
又做福音幹事那些,做傳道人.
讀完九張shot這麼多.
公書教學幫他提升各樣的技能.
九張shot,九個degree.
然後他喝完之後.
現在怎樣了,去哪了.
都走了.
他想表達什麼呢.
就是那些十幾二十年的年輕人.
十幾歲到現在三十多歲.
樓也賣了,車也有,子女也有.
怎樣了,有了就去哪了.
就走了.
他說現在做什麼好.
金陵,金陵的電影節目最值得做.
又有錢,又穩定.
茶几他一定在.
你問那些二十多三十歲的弟兄姊妹.
有空又不知道去了哪.
他怎會整天在那.
他有個傳道人,不過不要說太多資料了.
免得你知道.
他那個傳道人很厲害,是他的拍檔.
又能彈,又能唱,又能講道.
什麼都行,是傳道人.
我領教過他.

$^{281}$他手擒,又能彈,又能唱,又能講道.
我都見過.
什麼都行,基本上.
那幕詩他說什麼.
今晚在那打火鍋.
打火鍋,你以為我不知道他想做什麼.
他想跟我說七月的時候.
他跟他,誰,不要說誰.
就飛了,又不在那了.
我就明白他為什麼要喝兩罐啤酒.
(笑聲).
我不知道你認不認同.
其實剛才十七十八字的經文.
傷口不治癒.
耶和華待他如鬼咋的江河.
不可信賴.
他的痛苦變得長久.
其實想深一層的是.
其實一個侍奉者的生涯.
在一個拆毀的年代.
在一個拔出的年代裡.
其實變得很不正常.
耶利米亞是這樣的.
大體上這番話.
我們很少敢跟上帝說.
上帝,我的痛苦變得長久.
你想醫治我嗎?.
我吹什麼?我不讓你醫.
你不敢這樣說.
你明白嗎?很難的.
耶和華待他如鬼咋的河水.
最奸狡的是你.
你專騙我.
你不可信.
希伯真的這樣翻譯.
我不是騙你的.
你很少.
你證明這一句.
經歷了十五十六節的張力之後.
你以為那些張力很容易過嗎?.

$^{321}$你為上帝做事.
做完很辛苦.
你不想做.
不想做完上帝突然跟你說.
要你吃完做甘甜做開心快樂.
吃完之後又再堅持.
又再傷害過又再慘過.
這些張力如是者一年兩年三年.
到了耶利米亞四十年之後.
他的想法是什麼?.
四十年之後的想法就是.
他裡面傷到一個地步.
不知道怎麼搞下去.
我不是只想說我出賣牧師.
過了這個月.
就是我入伍.
所以我不是很想過四月.
你知道.
剔隔就要剔.
你接受不到.
四字以下剔是很開心的.
因為你不會.
到五就五十以上.
五十以上是包括很多年紀.
所以其實剔五是很慘的.
不過過了這個月.
我就不行了.
就要剔那一格.
所以這篇道基本上.
我想是跟自己說的一篇道.
就是四十九歲.
前的最後一篇.
跟自己要說的道.
其實坦白說.
侍奉了二十多年的裡面.
十五十六節的東西是有的.
你堅持二丈不得好死.
堅持二丈完你傷債.
你傷完之後你不要想做.
不想做完到一個地步.

$^{361}$上帝會跟你再說話.
你再走下去.
但十七,十八節.
那個是很真實的寫照.
我記得.
我去年十二月的時候.
我有一個很嚴肅的祈禱.
跟上帝祈禱.
我說:主啊,入五了.
還有十年侍奉.
我說:減吧.
我問祂有什麼應該做.
有什麼不做.
再做.
好像很有意義的東西.
可不可以不要有那麼多精力.
再做那些好像很無謂的東西.
可不可以告訴我.
有些東西是在十年來.
我需要做的東西.
所以大體上上年年尾那幾個月.
我不斷在想這件事情.
有五個小時的心情.
在十二月的時候.
那個祈禱是我沒有試過那麼認真.
很久沒有那麼嚴肅的祈禱.
我跟上帝在爭論.
我問上帝.
其實已經試了二十多年.
其實你問我.
我是否很盡力.
我可以跟祂說我也很盡力.
我也很用心.
這十多年的裡邊.
我很少偷懶.
這十多年裡邊.
要做的事情我都有做.
我跟上帝說.
我可不可以已經做了二十多年.
可不可以在這十年裡邊輕鬆一點.

$^{401}$主.
我問祂.
遇到的事情可不可以遇得簡單一點.
不要那麼複雜.
帶著這個祈禱.
我覺得很對.
我帶著這個祈禱.
二,三年頭那幾個月.
我覺得很對.
我開始的心態是.
有複雜的事情就不要碰.
弟兄姊妹有什麼事.
看一看,轉個頭就忘記了.
有些事情你應該要說的.
就跟上帝說.
上帝說都不需要我說.
其他人說就可以了,走.
我開始體會.
十七十八字的.
裡面的傷痕.
拒絕被醫治.
我感覺到十七十八字裡面說.
上帝就算你告訴我.
接下來十年要做什麼.
你待我都像魚的鬼渣.
江河一樣.
你都是不可信的.
坦白說,主啊.
過去二十多年裡邊,有時候我都相信你.
相信完之後,很老套的.
你跪下祈禱.
主啊,這次你幫我吧.
他每次都不幫你.
不幫完你之後,死在地上.
再轉個頭說,不要緊,萬事都是.
互相效力的.
那我困自己,又進去繼續.
我困了很多次了,我還困.
是不是傻的.
傻的嗎.

$^{441}$我帶著十五十六十七十八字的觀念.
我想看看耶和華怎麼回答.
你看看.
去到PowerPoint.
按下去.
按下去,是否要再按下去.
十九先,不好意思.
十九節耶和華回答了.
他說,你若回轉.
我就將你回轉.
我聽了這句,我真的想說.
什麼呢.
這句很不老套.
不合聽.
這句很不合聽.
這句我很想踩了它.
他說,在我面前你可站立.
站立的意思是serve.
意思是你在我面前,你都可以繼續侍奉.
接著他最難明白.
他說,你若從.
無用中帶出貴眾來.
你就可以.
繼續做我的口.
即是說話.
他說,他們回轉會歸向你.
你卻不可以.
回轉歸向他們.
不複雜.
只是解釋那句.
我最不明白那句.
什麼叫做.
你若從無用中帶出貴眾.
出來.
這句說法其實是.
很多解釋.
我要表達.
我查過不同的東西,都有些解釋.
第一個最.
跟上文和下文.

$^{481}$可以連繫到的解釋.
對於我來說是合理的.
原來.
十五,十六,十七,十八那些經歷.
是我們眼中看為.
無用的經歷.
你明白我說什麼嗎?.
十五,十六,十七,十八.
那些傷痕,你不能醫治.
你覺得.
耶和華在困你.
你覺得.
你為何經常要受那些不應該的對待.
我們眼中看這些為無用的東西.
就好像耶利米.
不斷宣講耶路撒冷倒地.
那些宣講耶路撒冷的故事.
一定不會這樣說話.
你才倒地,什麼時候耶路撒冷倒地.
誰真正倒地,就是你倒地.
說好耶路撒冷的故事.
一起來,送貼紙給你.
我今天收到很多張.
你可以想像的是.
那些人才覺得你無用.
那些人覺得你無用之餘.
自己也覺得你捐錢的那些.
都是無用的.
這些是無用的.
而實質的耶利米.
說完那些東西,被人打.
被人欺負,被人罵,被人吵,被人封鎖.
被人關.
他說的話被人燒掉.
不要再記下來.
人們都想將他說的話都變成無用的東西.
但是,你看這段說話.
沒有說其他安慰的說法.
他只是說.
那些無用的東西.

$^{521}$是會帶出貴重的東西.
就算解釋到這裡也好.
我和上帝第一個馬上問的問題.
我說,扯.
扯是有禮貌了.
我說,無用.
你哪會帶出貴重的東西.
是什麼.
我明白你想說什麼.
我在讀經的時候.
我只是想說什麼.
那些好像是無用的東西.
但是在你眼中也很有用.
你也是這麼老套的.
你從來都是這麼老套.
你還在做什麼.
你可不可以讓我們覺得.
那些東西是貴重的.
可不可以不要.
別人不覺得貴重.
自己也不覺得貴重.
你覺得貴重.
我問,你怎麼看的.
全世界都說不貴重.
但是你說貴重.
你可不可以有些時候.
給點面子.
有些東西做出來是有用的.
是貴重的.
基本上.
這段經文的翻譯.
我兩個星期前做的.
通常我準備兩個星期.
希望能夠明白這段經文.
之後這個星期.
我又想.
怎麼說好,還有什麼應用.
你明白嗎?.
我準備說成這樣.
我這裡想不到.

$^{561}$我想了很久.
怎麼貴重.
又騙你們,又騙我自己.
不要緊的.
當上帝繼續堅持就行了.
這些說法我又說不出口.
這個星期發生一件事.
一個我以前.
牧養了很久的.
我想他從中幾年.
牧養的一個青年人.
現在不青年.
現在年紀大了.
生了兩個小朋友.
他叫我吃飯.
因為他知道我妹妹的事情.
他等下來安慰我.
我沒想到他來到.
吃完飯之後.
他和他老公一起來.
跟我聊聊天.
最後我們一起祈禱.
他們的祈禱.
不知為何.
我後來覺得很久沒有祈禱.
不是很久沒有祈禱.
很久沒有一個正常人為我祈禱.
那個祈禱突然說.
主啊,我很想要這個祈禱.
祈什麼都可以.
他說祈禱.
我立即很勁乾式.
你明白嗎?.
你祈禱也是這樣.
那一刻不是.
那一刻是很勁乾式.
我說主啊,你呼他說話.
說什麼都可以.
不中聽都可以.
祈禱之後.

$^{601}$我都沒有消化那種祈禱的內容.
和對我的震撼.
離開的時候.
那對夫婦.
叫Edson.
我的英文名字.
不用知道.
她說可不可以擁抱你.
我心想,很久沒有人擁抱我了.
她的老公.
很禮貌地擁抱.
其實沒有什麼擁抱.
老公就說.
我也要.
(笑聲).
我覺得上帝好像在回答我的問題.
原來貴重不是在說.
你做的事.
人看.
或者你自己看.
幾個價值.
貴重不是在說.
這個貴重.
貴重是你覺得很沒用的時候.
有些人跟你說.
繼續走這條沒用的路.
我要在這邊跟自己說.
我特別想說的是.
我希望你明白.
一些快50歲以上的人的心態.
其實修成期也好.
不是修成期也好.
不想再辛苦下去是真的.
不想再辛苦下去.
假裝有意義.
不辛苦下去.
假裝一下功是心態.
有些事.
不需要像以前那麼認真.
說了些事.

$^{641}$聽完就算了.
有些事.
不要再堅持下去.
多一點點.
收回一下.
開始.
我最近看一個MEMES.
當你覺得.
世界上所有事都不關你的時候.
你就真正開始活出你的人生.
我看完之後.
我說很感動.
突然之間你就覺得.
神經病,很感動.
你真的想所有事都不邊界你的生命.
沒有東西再邊界你了.
是不是.
為什麼.
《殭屍死兒44條屍》是涼的.
Edan在小說裡說什麼.
因為心靜自然涼.
你多無聊.
不是這麼無聊.
我要表達的是.
由我以前20多30歲服侍到40多歲.
都還想堅持認真一下.
我開始明白.
50歲之後.
不是很想再做這些事.
想扮一下做一下就夠了.
你們不會知道我扮一下做一下.
我有20多30年.
告訴你們我正在做這些事.
你不會突然50歲覺得.
這個人也不是這樣.
我可以騙你們.
其實我還是在做那些事.
只要我再肉緊一點就像是.
我真的發現.
我會明白.

$^{681}$為什麼年輕的那一輩.
不覺得信仰精彩.
你知道我的題目是想跟90後和00後道歉.
我明白開始.
為什麼在我小時候.
我看不到很多模型.
告訴我.
他年紀大老邁的時候.
繼續堅持被人打.
被人欺負.
他突然出來說.
我都可以傷痕累累.
只不過他跟我說.
沒用的.
你被人拿出鬼種來.
我好像聽不到我的前輩.
說這些說法.
我只怕我這代50歲以後的人.
開始墮落以先.
我們都沒有任何榜樣.
告訴下面那一代人.
其實信仰到50 60歲之後.
都可以堅持下去.
最近在網上有一個前輩.
我猜他70多歲.
他不知道.
對某些時機.
教育的事評論.
下面的人就問他.
你夠膽死.
夠膽說這些.
接著那個前輩.
說一番說法.
他說我到這個年紀.
都不說這些.
誰說的.
那時候很震撼.
很eye catching.
我看著那個留言.
久久不能平復.

$^{721}$我就是那些.
到差幾歲都不會再說話的人.
到差幾歲的時候.
可以獨善其身.
開始覺得.
自己已經升上神台位.
不再有任何想法.
總之大家平靜安穩.
我不得罪你,你不得罪我.
你好我好,基督徒好相愛.
賣掉這些無聊價值.
不就夠了.
起碼這次這段經文.
跟耶利米說的是.
叫他繼續.
其實那兩三節聖經.
對他的答案.
都是重覆.
你的召問根本就是.
十五,十六,十七,十八節的經歷.
各位二三十歲的弟兄姊妹.
如果昔日你沒有什麼榜樣.
很多屬靈的前輩.
好像你現在的想法.
這兩天和解之後.
青年圖圖室.
那個朱文和阿全的說話.
如果你有看的話.
就是在說.
當好像有些異象在走的時候.
他墮落的時候.
他有千百萬個理由.
支持自己的墮落.
還要說得很大聲.
正如調解中心的.
阿祖.
那個老闆一樣.
當一些異象要墮落的時候.
我們都有千百萬個理由.
讓自己堅持自己繼續墮落下去.

$^{761}$讓人都覺得.
自己在堅持異象.
在香港很艱難的日子裡.
我祈禱自己和我們這一輩以後的人.
看著十七,十八節之後.
罵完上帝,投訴完.
都聽到十九至二十一節的經文.
說那些雖然沒用.
但那些沒用的東西.
就只是寶貴的東西.
我希望將來的日子裡.
二,三十歲的弟兄姊妹們.
你堅持你現在堅持的東西.
拒絕很容易妥協.
在制度裡的敗壞.
無論你在制度裡的堅持.
或者跳出制度裡.
做你想做的事情.
你裡面的初心從來不改變.
尤其是一個崩壞的世代裡.
怎樣走下去.
都是你和我的問題.
我們這一代五十歲以後的人.
正如馬雲所說.
不關聖經的事.
馬雲知道他說什麼嗎?.
他說五十歲以後的人只做一件事.
就是幫我們二,三十歲的人.
好好讓我們下一代的人.
可以起來承擔.
天國上帝榮耀的職份.
心願在未來的日子裡.
我們不只是說.
怎樣幫助二,三十歲的人.
心願看到更多年紀大的人.
不再為著自己的名份.
沽名釣譽.
是認真用心力.
將福音傳承給我們下一代的人.
就像我這個牧師朋友一樣.

$^{801}$經歷很多艱難.
發脾氣.
喝兩罐啤酒.
我想跟你說.
他明天,星期日我見到他.
又是一條好漢走出來.
繼續做他覺得應該要做的事情.
求主憐憫我們這個世代.
求你體諒一下那些五十歲以上.
很想裝模作樣.
告訴你繼續虔誠的人.
其實他心裡軟弱非常.
但我心願我們這一輩的人.
認真幫助下面二,三十歲的人.
接好信仰的棒.
不再為自己的名聲,利益的緣故.
考量任何的事情.
縱然這樣想完.
不得好死.
但這個仍然可以跟上帝說.
你告訴我這是無用的.
裡邊的貴重.
求主幫助我們這一班年紀漸老的人.
祝願下一代的人.
不要單純看著前輩的人做了多少.
雖然我們這一代的人可能會令你很失望.
但你仍然有你那份在上帝裡.
堅持要做的事情.
我一切都有禱告.
天父多謝你讓我們今天有這個空間和時間.
我們來到你面前.
當這個世代在說世代之爭.
大家不明白對方.
大家不願意去理解的時候.
求天父你自己跟我們說的是.
仍然上帝預備了一些人.
想跟下一代去談.
而下一代仍然可以有些人.
他可以看著.
他怎樣走他信仰的路.

$^{841}$天父這是一個沉重的時代.
是一個艱難的時代.
亦是一個很多欺騙很多詭詐的時代.
我只求主照亮我們的眼睛.
看到當看到的東西.
看到你的光明在哪裡.
我們就可以跟從下去.
願主你的恩典就這樣和我們在一起.
祝福Full Church的弟兄姊妹.
每一位來到你面前的弟兄姊妹.
生命都繼續得著你親自的光照.
心裡明亮.
看得清主的真理在哪裡.
祈禱奉耶穌你寶貴的名球.
\newpage



\section{哈該書 1:1-15-20230422}
\label{sec:S0X_1Lh_dHA}
\textbf{【流堂崇拜】你先?我先?我地可能都會搞錯!|哈該書1\_1-15|20230422 [S0X-1Lh\_dHA]}
\newline
\newline
連結: \href{https://youtube.com/watch?v=S0X-1Lh_dHA}{\texttt{ https://youtube.com/watch?v=S0X-1Lh\_dHA}} ~~~~ 語音日期: 2023-04-22 
\newline
\newline
\hyperref[sec:3PY1nwdp_0k]{\small{< < < PREV SERMON < < <}}
~
\hyperref[sec:index_chronic]{\small{[返順時目]}}
~
\hyperref[sec:index_scriptual]{\small{[返順卷目]}}
~
\hyperref[sec:VNZbDAiXlG0]{\small{> > > NEXT SERMON > > >}}
\newline
\newline
哈該書 1:1-15-20230422
\newline
\begin{longtable}{cl}
\hline
\hline
章節 & 經文 (和合本修訂版)\\
\hline
1:1 & \begin{tabularx}{0.7\textwidth}{X} 大流士王第二年六月初一,耶和華的話藉哈該先知向撒拉鐵的兒子猶大省長所羅巴伯和約撒答的兒子約書亞大祭司傳講,說: \end{tabularx} \\ \\ \relax
1:2 & \begin{tabularx}{0.7\textwidth}{X} 「萬軍之耶和華如此說,這百姓說,建造耶和華殿的時候還沒有到。」 \end{tabularx} \\ \\ \relax
1:3 & \begin{tabularx}{0.7\textwidth}{X} 耶和華的話藉哈該先知傳講,說: \end{tabularx} \\ \\ \relax
1:4 & \begin{tabularx}{0.7\textwidth}{X} 「這殿荒涼,你們自己還住天花板的房屋嗎? \end{tabularx} \\ \\ \relax
1:5 & \begin{tabularx}{0.7\textwidth}{X} 現在,萬軍之耶和華如此說,你們要省察自己的行為。 \end{tabularx} \\ \\ \relax
1:6 & \begin{tabularx}{0.7\textwidth}{X} 你們撒的種多,收的卻少;你們吃,卻不得飽;喝,卻不得足;穿衣服,卻不得暖;領工錢的,領了工錢卻裝入有破洞的袋中。 \end{tabularx} \\ \\ \relax
1:7 & \begin{tabularx}{0.7\textwidth}{X} 「萬軍之耶和華如此說,你們要省察自己的行為。 \end{tabularx} \\ \\ \relax
1:8 & \begin{tabularx}{0.7\textwidth}{X} 你們要上山取木料,建造這殿,我就因此喜樂,且得榮耀。這是耶和華說的。 \end{tabularx} \\ \\ \relax
1:9 & \begin{tabularx}{0.7\textwidth}{X} 你們盼望多得,看哪,所得的卻少;你們收到家中,我就吹去。這是為甚麼呢?因為我的殿荒涼,你們各人卻只為自己的房屋奔走。這是萬軍之耶和華說的。 \end{tabularx} \\ \\ \relax
1:10 & \begin{tabularx}{0.7\textwidth}{X} 所以,因你們的緣故,天不降甘露,地也不出土產。 \end{tabularx} \\ \\ \relax
1:11 & \begin{tabularx}{0.7\textwidth}{X} 我命令乾旱臨到土地、山岡、五穀、新酒、新油和地上的出產,也臨到人和牲畜,以及一切人手勞碌得來的。」 \end{tabularx} \\ \\ \relax
1:12 & \begin{tabularx}{0.7\textwidth}{X} 那時,撒拉鐵的兒子所羅巴伯、約撒答的兒子約書亞大祭司,和所有倖存的百姓都聽從耶和華-他們神的話,就是哈該先知奉耶和華他們神差遣所說的話;百姓在耶和華面前存敬畏的心。 \end{tabularx} \\ \\ \relax
1:13 & \begin{tabularx}{0.7\textwidth}{X} 耶和華的使者哈該奉耶和華差遣對百姓說:「我與你們同在。這是耶和華說的。」 \end{tabularx} \\ \\ \relax
1:14 & \begin{tabularx}{0.7\textwidth}{X} 耶和華激發撒拉鐵的兒子猶大省長所羅巴伯、約撒答的兒子約書亞大祭司,和所有倖存百姓的心,他們就來為萬軍之耶和華-他們神的殿做工。 \end{tabularx} \\ \\ \relax
1:15 & \begin{tabularx}{0.7\textwidth}{X} 這是在大流士王第二年六月二十四日。 \end{tabularx} \\ \\
[1ex]
\hline
\hline
\end{longtable}
$^{1}$頂姐妹平安,在網上送拜的頂姐妹平安.
剛才謙明說得很對,可以來到神日殿一起敬拜祂不是必然的事.
另一樣我領受的是可以在神日殿中侍奉祂更加不是必然的事.
因為在這個星期一,我起床的時候,我發覺我喉嚨很痛,聲沙咳出痰很濃.
接著不久,我小兒子就告訴我,爸爸我兩條線.
我馬上就進去廁所,看看我多少條線,幸好那兩條線是黏在一起的.
但是今天都是一條線,難保明天不是,最怕是星期五才來兩條線怎麼辦呢.
我馬上就WhatsApp給潘Sir,潘Sir潘Sir,有沒有什麼指引呢?.
因為萬一都要穩定,後補後備,我們有很多同工在場.
除了清心之外,清心是失聲的.
潘Sir就說,現在這些只是風土病,只要你沒有失聲就行了.
但是我心寒寒,當時我聲沙咳,我不敢告訴他.
我跟他說潘Sir,幫我祈禱吧.
所以這幾天我一直在喘氣,跟老婆都聊笑了,跟兒子都說笑了.
昨天去踢球都走了半個小時,怕真的聲音不夠.
所以大家幫我祈禱吧,我希望今天可以有足夠的氣,講完整個行動.
不知道大家還記不記得上次我站在台上和大家分享的那個行動.
在去年十月,我記得當時和大家用七千人這個故事去思考.
我們在暴政下所產生的那種無力感,我們應該怎樣面對.
這次我就很想帶大家去看看,性機裡面另一班更加大群的人.
他們每次都會作一點點的內在反思,思考一下我們和神之間的關係.
思考一下我們生命中某些優先次序.
這些既細微又重要,但是又很容易搞錯的事情.
所以今天我很想和大家分享,我跟潘Sir說.
不行,就算我中了我也要回來,如果萬一失聲你可不可以幫我讀.
幸好今天沒有找他,所以開始分享之前我就不讀了.
邀請大家幫我讀這十五字經文.
上次我用七千人去和大家作思考.
今天我打算用五萬人和大家一起去思考剛才所說的優先次序.
或者大家聽完之後會覺得,剛才我讀的裡面,怎樣讀都找不到五字或五萬字.
但如果熟悉猶太人回歸歷史的弟兄姊妹都會知道.
除了我們在哈蓋書第一章有記載他們的事情之外.
我們在以色列記和利希米記都有相關的記載.
而我剛才所說的五萬人群體就出於這兩本書.
而五萬這個數字就在以色列記第二章和利希米記第七章裡面計算出來.
怎樣計算呢?我們一起看看吧.
根據以色列記第一章的記載.
耶和華親自去激動波斯王古列的心.
叫他頒下禹旨讓猶太人回到耶路撒冷.
去重建家園,重建聖殿.

$^{41}$在所羅巴巴和耶穌大人的帶領之下.
聖經在以色列記第二章64-65節裡面記載.
當時願意回歸的人有回眾四萬二千三百六十人.
木被有七千多人,又有唱歌的男女.
所以將他們加起來就大約有四萬九千八百多人.
我將他們四寫五入之後就簡稱為五萬人.
正如以色列記第一章一節所提供的資料.
五萬人生活在波斯王古列年代.
當時大概是主前538年.
距離猶太人第一次被擄.
即是主前608年來說已經足足有70年.
如果以第三次被擄的時間去計算的話.
即是在主前586年來計算的話.
這班人當時已經有50年之久.
所以我們有理由相信這五萬人當中.
大部分的人都是在被擄期間在巴比倫出生.
他們可算是巴比倫土生土長的猶太人.
另一方面我們一般人對這個群體都有個錯覺.
因為這班猶太人就像以色列人一樣.
是很苦的.
他們流浪異鄉的二等公民一定是很苦.
生活一定像以色列人一樣.
被奴役,被迫,被苦待,被歧視.
但其實實際的情況並不是我們所想像中那麼差.
因為如果我們對比其他經文.
我們會看到這班猶太人在當時的生活.
無論在經濟,宗教和社會層面上.
都是得到一定程度的自由和尊重.
我們回看社會方面.
我們可以在耶利米書29章一節中.
看到耶利米先知寄信給被擄猶太人.
經文中看到巴比倫政府容讓被擄到巴比倫的猶太人.
和留在耶路撒冷的猶太人之間.
可以有書信的自由來往.
所以我們可以看到他們之間的通訊自由.
並不會被人說是勾結外國勢力.
更加不會把他們當作外國代理人去看待.
在宗教方面我們可以看到.
根據《撒爾利書》六章十字記載.
當時的猶太人某程度上還可以享有自己的獨經,禱告和敬拜生活.

$^{81}$他們的宗教自由似乎都沒有被巴比倫政府有很大的限制.
第三方面在經濟方面.
雖然他們在經文中沒有直接看到.
巴比倫政府給猶太人享有什麼程度的耕田和經商的自由.
但我們可以肯定的是.
這五萬人在回歸耶路撒冷的時候.
他們的經濟能力是有一定的能力的.
因為我們在以色列的二章六十四至六十七節看到.
這班回歸的猶太人帶著七千個腹皮.
當時四萬多人帶著七千個腹皮.
五個人有一個腹皮.
將雷,馬,駱駝加起來有八千多隻.
雖然當中有部分是當地人捐獻給他們.
但我們可以肯定.
他們不是窮到窿的那種.
所以以上三方面我們可以看到.
這五萬人一定不是像埃及的那班以色列人.
因為在巴比倫生活得很苦.
被迫去巴比魯拍攝.
沒有一口好食.
所以被迫要回耶路撒冷.
相反他們可能在巴比倫已經有很好的適應.
甚至發展得不錯.
他們可能有自己的房屋.
有自己的田地.
甚至有自己的事業.
有自己的群體.
更加有相對的社會地位.
其實這五萬人大可以繼續留在巴比倫.
馬照跑,舞照跳,鼓照炒.
賺多點錢.
但他們願意回歸.
讓我們看到他們不是那種等次的人.
他們不是只顧自己的利益.
馬照跑,舞照跳,鼓照炒.
賺多點錢.
而將神的話語置之不理.
他們不是那種只要賺多點錢.
就可以將公義,憐憫拋諸腦後.
他們不是那種只要賺多點錢.

$^{121}$就可以不再追求公義,真理,民主,自由的普世價值.
我相信他們不是那種只要賺多點錢.
就可以不理會社會上被壓迫,被迫迫,被無理囚禁的人的需要.
我相信他們不是那種只要賺多點錢.
就可以泯滅良心,出賣人性的人.
弟兄姊妹,有沒有想過.
這一班人他們被譽為成長在一個背叛,混亂,墮落,不信的巴比倫國度裡.
他們為什麼可以出於污泥而不染呢?.
我相信是他們父母很有關係.
我相信他們父母在他們成長當中.
不斷地將神的話語,聖經教導,人性的道德底線.
將猶太人的歷史鉅細無遺地教導他們.
以致他們有正確的價值觀.
以致他們有正確的人生的優先詞綴.
以致他們可以出於污泥而不染.
所以上星期六嘉Sir提醒我們這班50+的人.
要幫助90後,00後的年青人.
今天我在這裡都很想鼓勵我們當中的90後,00後的年青人.
尤其是在當中以為人父母的或者將會成為人父母的弟兄姊妹.
我們要好好地幫助我們下一代.
好好地教導我們的孩子.
幫助他們建立一個正確的價值觀.
教導他們有一個正確的生命優先詞綴.
將神的話語,聖經教導,人性應有的道德底線.
將香港的歷史,將香港所發生的事情.
鉅細無遺地去教導他們.
讓他們正確地認識真理,認識歷史.
鉅細無遺地去教導他們.
以致他們不會沾染上這些馬照跑,舞照跳,鼓照炒.
賺多點錢的世俗思想.
親愛的爸爸媽媽們,親愛的準父母們.
在小朋友成長過程的時期裡.
家庭和學校是對他們影響最大的.
亦都是他們最信任的地方.
亦都是他們最願意開放自己去接受被塑造的地方.
但是當我們回看我們現在的教育制度.
我想我們是時候站起來.
擔起更多的責任.
在家庭當中作教導的工作.
以致他們有能力去辨別是非.

$^{161}$有能力去抗衡那些謊言.
有能力去分辨那些荒謬的事情.
親愛的爸爸媽媽們.
我們所處的這個世界.
那種背叛,那種混亂,那種不信,那種墮落.
真是不下於巴比倫,是嗎?.
我們孩子真是需要你們.
將教導他們的時間.
配伴他們成長的時間.
去擺在優先的位置上.
你們願意嗎?.
另一方面,這五萬人響應回歸.
回去耶路撒冷重建聖殿.
除了要扶上剛才所說的.
可能房屋,田地,社交圈子,社會江湖地位.
還要面對很多困難和挑戰.
例如回歸屠城的風險.
回去耶路撒冷面對那些頹垣敗瓦.
面對人生路不熟.
要面對重新來過的挑戰.
但是他們心智依然堅定.
可見他們當時是將神的利益,神的心意.
擺在自己的優先位置上.
所以,親愛的弟兄姊妹們.
我們今天回到教會當中.
我們在跟隨主的路途上.
我們將誰的心意擺在最前面呢?.
我們記得3月18日.
大家有沒有印象?.
潘Sir在台上凝需.
他說自己信心少.
大家還記得他凝需的樣子是怎樣嗎?.
今天潘Sir不在,大家可以放膽說.
哎,不行,立仔幫幫忙.
童工說我信心少.
我承認.
童工說我信心少.
不要趁潘Sir不在,我們作大反.
一次就夠了.
大家看到這段片.

$^{201}$你看看凝需的樣子.
凝需的樣子.
潘Sir不好意思.
凝需的樣子.
他多滿足,多開心.
我從未見過有人凝需得這麼開心.
但我相信.
潘Sir的樣子是很滿足.
為什麼呢?.
因為他看到我們Flow Church這個群體.
絕大部分的弟兄姊妹.
都是在跟隨主的路途上.
願意去付上代價.
絕大多數的弟兄姊妹.
當時沒有被這107級的樓梯所嚇倒.
更加沒有被這些夾攝.
不方便,不懂路,不懂去的念頭去阻礙.
而不去崇拜.
我想可能弟兄姊妹會說.
你會不會說得太大?107級而已.
有什麼大事?.
其實我覺得我和潘Sir都看到了一種起點.
107級是一種起點.
我們看到原來我們身邊有這麼好的弟兄姊妹.
都是願意去學習.
去操練自己向跟隨主的路途上.
去學習付上代價.
學習以神為優先.
以神的心意為優先.
就好像那五萬人一樣.
是不是?.
去到以色列的第三章.
尤其在這五萬人.
回到耶路撒冷之後沒多久.
他們就很齊心地聚集在耶路撒冷.
以色列的第三章一節.
是這麼說的.
如果.
不好意思,我聽到.
他們是如同一人這樣聚集在耶路撒冷.

$^{241}$他們如同一人這樣聚集在耶路撒冷做什麼呢?.
經文在三章二至五節進一步帶出.
這群猶太人聚集在一起.
要為神去祝壇.
向神去獻祭.
同時間他們又出錢.
又出力地開始動工去建造聖殿.
因此他們在回歸短短一年之內.
完成了聖殿的奠基工程.
可見他們對神的敬畏.
對信仰的認真.
雖然耶路撒冷是他們的故鄉.
但當中大部分的人都是巴比倫土生土長的人.
他們可以算是一群初度貴境.
除了要適應新環境之外.
他們應該還有很多生活上的事情要處理.
例如他們要找地方安頓自己的家人.
預備建屋,買建材.
甚至乎要替別人拿回被霸佔的農地,地方.
很多事情需要放上心思.
但他們依然堅持將神的事情放在優先位置上.
為神的事情放上心思,負少,時間和金錢.
我記得在2021年年尾.
我們Flow Church推出了一個Outflow Mission.
是鼓勵留下在外地的教務同工和弟兄姊妹.
一起齊心在當地去共建一個敬拜上帝的群體.
去共建一個信仰上同行的群體.
我記得當時只是短短十天.
就有超過一百位弟兄姊妹和教務回應.
他們大多數都是初度貴境.
他們都是人生路不熟.
有些剛剛到達還沒有租屋.
還是住在Airbnb.
有些就算已經租到屋.
都還要為找工作,考車牌去傷腦筋.
但他們依然很熱心地,很齊心地回應Outflow Mission.
跟他們傾談的時候.
我很感受到他們對神的敬畏.
對重建同行群體的渴望.
對自己的敬拜,心靈的重視.

$^{281}$雖然當中有些弟兄姊妹因為某些原因.
譬如交通或者其他特別原因.
最終都未能夠在Outflow Mission裡面參與.
但我知道他們大部分都很積極地.
在自己的家附近找Local Church.
去建立自己的信仰上同行群體.
有些住得較遠的.
連Local Church都找不到的.
所以他們現在都迫在網上跟我們一起崇拜.
我相信神都一定可以看到你們的擺相.
你們的優先次序.
並且閱立了你們的敬拜.
我們再看聖經.
我們看到好景不常.
在第四章當五萬人慶高彩烈為奠基儀式慶賀的時候.
聖經記載他們遇上難阻,遇上很多挑戰.
經文說有些住在當地的外族居民來難阻他們.
他們千方百計去擾亂他們,恐嚇他們,阻攔他們.
甚至有人用錢不斷收買當地的官員來難阻他們.
消磨他們建建的心智.
在這些惡劣的環境下.
在這些外來的壓力,威脅下.
甚至有些政權的威脅下.
這五萬人就開始退縮.
他們的手就放軟下來.
結果他們在回歸第二年.
大約是主前535年建電工程就完全停頓下來.
一停就停了15年.
時間一斬過了15年之後.
我們就看到經文的下蓋書.
於是神就差派下蓋先知去到當中傳講神的話語.
叫他們重新啟動重建聖殿.
在這裡.
在這裡,《華本聖經》在五章一節中.
描述神叫下蓋先知去到他們去跟他們講勸勉的話.
而呂鎮忠亦本亦說鼓勵他們.
不過如果我們參考原文.
《聖經》記載.
我們會看到新譯本的記載.
會貼近多些.

$^{321}$因為原文真的沒有勸勉,鼓勵這些字眼.
純粹就好像新譯本這樣翻譯.
就說下蓋先知去到跟猶太人傳講信息而言.
另一方面,如果我們再看回剛才我們所讀的.
《哈該書》一至四節中.
我會發覺神差派下蓋先知去跟這五萬人講.
不單是沒有這種鼓勵,勸勉的語氣.
相反,他是帶著那種雜備.
因為神在《哈該書》第二節中.
稱呼這五萬人為「這百姓」.
他不再用那種親切的角度.
「我的百姓」,這些稱呼他們.
如果我們比對當時神在《以賽亞書》中.
稱呼那些勃逆的以色列人.
是稱呼他們「這百姓」的時候.
我們就會看到神當時的心情.
當時真的去雜備這五萬人.
神要下蓋先知去雜備這五萬人.
雜備他們什麼呢?.
經文在第四節中展示神對他們不滿的地方.
經文說:「這殿盈盈荒涼,你們自己還住天花板的房屋嗎?」.
當中的天花板房屋原文是指有蓋的意思.
他指向的不單是有屋頂的房子這麼簡單.
更是指向他們用很高貴的木材去遮蓋房屋.
或者去裝飾牆壁.
來建造一個非常華麗的房子.
就好像新年代本中所翻譯的一樣.
「我的聖殿盈盈荒涼,你們卻在建築華麗的房子裡」.
下蓋先知批評他們不是批評他們住屋過份華麗.
下蓋先知責備他們的地方是.
他們既然有時間有金錢去裝潢自己的家.
為什麼他們任憑聖殿去荒涼去荒廢呢?.
他們這一刻到底是將神放在什麼位置上呢?.
神叫下蓋先知指出這五萬人在這十幾年裡.
因為被外來的社會環境,被政治,政權的威脅.
被經濟的影響.
不知不覺地將生命的優先次序放錯了.
放亂了.
不再像回歸初期那樣.
將神的事情,心意放在優先位置上.

$^{361}$取而代之的是將自己的喜好,享受,事情放在神的事情,心意之先.
就像經文第九節所說.
「我的殿荒涼,你們各人卻顧自己的房屋」.
弟兄姊妹,這幾年我們都經歷了很多社會的變遷.
經歷了漫長的政治,疫情,經濟上的衝擊.
我們有沒有像這五萬人一樣.
不知不覺地放錯了我們的生命的優先次序呢?.
最近我在網上看到一個有關香港基督徒在疫情期間參與網上崇拜的調查報告.
報告當中發現原來有大約4\%的人參加超過五個教會的網上崇拜.
參加兩至五間的有17\%.
參與一間教會的網上崇拜有53\%.
完全沒有參與網上崇拜的有大約25\%.
根據這個比例,到了疫情的中期,即大約2021年底.
香港每四位基督徒當中有一個信徒沒有參與網上崇拜.
究竟有多少人因為環境和疫情衝擊而改變了他們的優先次序.
不再看重敬拜的生活?.
我相信一定不會全部.
因為我知道有些長者不懂得用這些數碼.
但我相信當中也有不少.
因為相信大家都遇過聽過身邊的人.
尤其是在修例期間.
很多信徒看到很多不公義的事情.
他們懷疑上帝的公義.
以至有些人生氣上帝.
有些懷疑上帝有沒有存在.
之後就沒有再去崇拜,去教會.
可能連網上崇拜也不去.
求主憐憫,求主憐憫這班弟兄姊妹.
幫助他們去重建他們的敬拜生命.
回到神的懷抱當中,重新投入教會生活.
另一方面,可能弟兄姊妹也會問.
在復常的情況下,我們留在家裡網上崇拜又如何看呢?.
在這方面,我們也可以用優先次序去審視.
首先,崇拜沒有一個最好的形式.
也沒有一個最好的地點,地方,處所.
就好像耶穌基督和井旁婦人所說.
你們拜父也不在這山,也不在耶路撒冷.
真正的敬拜並不在於任何的地點和形式.
真正的敬拜是在於我們內心的態度和焦點.
在你們拜父這四個字裡.

$^{401}$我們可以看到兩方面和崇拜有關的教導.
首先,拜父這兩個字帶出我們敬拜的中心是上帝.
所以我們在崇拜當中,每一個環節,每一個人.
都應該圍繞著神,以神為優先.
以神得榮耀為優先,以神為中心.
而你們這兩個字是一個眾數.
某程度上,看到是多於一個人的意思.
某程度上帶出崇拜的群體性.
而事實上,我們在很多聖經其他地方.
我們都看到神是閱立我們集體的敬拜和教導.
所以我很同意有目者在網上說.
崇拜是神指引集體朝見神的約會.
一群得救重生的信徒聚集在一起朝見神.
靠著聖靈的能力和引導,將神當得的榮耀歸給神.
因此,我相信如果我們一群人聚集在一個地方.
在一個家中,用網上崇拜去敬拜神.
去崇拜,神都必定閱立.
不過如果我們因為身體不適,要照顧家人.
可能有些特別的原因,需要個別留在家中.
去網上崇拜,我相信神都會明白,祂都會閱立.
但如果我們考量純粹以自己的方便和舒適.
可以放棄雙腳去唱詩歌,去喝杯咖啡去崇拜.
我們可能要想一想,我們那次的崇拜.
我們是否以神為中心,以神為優先.
最近,我在Outflow Mission中.
認識一位姐妹,和她聊電話.
我知道她是幾年前剛剛移居英國.
到Po 後,她很努力尋找教會去崇拜.
她加入教會侍奉,融入教會.
不過,最近因為教會有教務.
禁止她選擇一首詩歌去崇拜,敬拜詩歌.
她仍毅然離開服侍多年的教會.
之後她嘗試在自己附近,去找其他教會崇拜.
但都發覺,沒有一間可以容納到她對公義念文的渴求.
但很感恩,她在朋友的推介介紹之下.
她知道流唐有網上崇拜,於是在Outflow Mission回應.
很可惜,我們在該區還沒有聚會點.
所以她現在每個星期都迫於在網上以流唐的YouTube來崇拜.
話說回頭,大家可能會覺得.
選一首詩歌不讓,都要離開?.

$^{441}$大家知不知道這首詩歌叫什麼名字?.
Alex聽著了.
是榮譽降下.
我說,這首詩歌只是為香港祈禱,這樣都不行?.
兄弟姐妹,我們看到原來那個人在教會裡.
回到教會當中,可能站在侍奉崗位上.
所以他們的心可能都不是將神放在優先的.
他們在做審查,做自己喜好的事情.
他將自己的喜好,將自己的政見放在敬拜上帝的優先上.
其實在我們成長過程裡.
我們都可能試過不知不覺地搞錯這幾方面的次序.
我們可能曾經,不知道大家有沒有試過.
又或者大家有沒有聽過身邊的人這樣說.
就是我看看誰先講完講到,才決定回不回教會.
我有聽過,有些姐妹我親耳聽過.
今天在裡面崇拜,我一首歌都不唱.
我問他為什麼?.
我沒有一首喜歡唱的.
我們在敬拜當中,都很可能像他們這樣.
不知不覺地以自己為中心.
以滿足自己為優先.
不單止在敬拜,在崇拜當中.
就算在侍奉,在奉獻,我們都可能會這樣.
我不知道在這段時間,可能那種社會對我們的經濟都很大衝擊.
不知道會不會聽到有些人會說.
或者心裡想,讓我夠用才奉獻吧.
讓我有時間才侍奉吧.
我不知道我們大家有沒有這種掙扎.
我們這些掙扎會不會在這個環境當中.
在疫情當中,在社會當中被放大了呢?.
甚至被扭曲了呢?.
或者又被打亂了呢?.
盼望我們這次趁在這個服償日子裡.
我們一起檢視一下我們自己在這方面的態度.
校正我們的優先次序.
以致我們每個星期都可以獻上輕香的祭給上帝.
我們每一天都可以將自己獻上作為活祭.
去榮耀神,去尊崇神.
我們一起禱告.
親愛的天父上帝,你是各位尊貴榮耀的獨一真神.

$^{481}$你是配得我們全心全意的敬拜.
你是配得我們一切至高無上的重讚.
願你自己在我們當中得著應得的榮耀.
得著當得的讚美.
祈禱奉耶穌得勝名球.
\newpage



\section{以斯帖記 1:1-22-20230429}
\label{sec:VNZbDAiXlG0}
\textbf{【流堂崇拜】她和她最後的倔強|以斯帖記1\_1-22|20230429 [VNZbDAiXlG0]}
\newline
\newline
連結: \href{https://youtube.com/watch?v=VNZbDAiXlG0}{\texttt{ https://youtube.com/watch?v=VNZbDAiXlG0}} ~~~~ 語音日期: 2023-04-29 
\newline
\newline
\hyperref[sec:S0X_1Lh_dHA]{\small{< < < PREV SERMON < < <}}
~
\hyperref[sec:index_chronic]{\small{[返順時目]}}
~
\hyperref[sec:index_scriptual]{\small{[返順卷目]}}
~
\hyperref[sec:D8sOzznkhGg]{\small{> > > NEXT SERMON > > >}}
\newline
\newline
以斯帖記 1:1-22-20230429
\newline
\begin{longtable}{cl}
\hline
\hline
章節 & 經文 (和合本修訂版)\\
\hline
1:1 & \begin{tabularx}{0.7\textwidth}{X} 這事發生在亞哈隨魯的時代,亞哈隨魯從印度直到古實統治一百二十七個省, \end{tabularx} \\ \\ \relax
1:2 & \begin{tabularx}{0.7\textwidth}{X} 就是亞哈隨魯王在書珊城堡中坐國度王位的那些日子。 \end{tabularx} \\ \\ \relax
1:3 & \begin{tabularx}{0.7\textwidth}{X} 他在位第三年,為所有官員和臣僕擺設宴席,有波斯和瑪代的權貴,各省的貴族與領袖在他面前。 \end{tabularx} \\ \\ \relax
1:4 & \begin{tabularx}{0.7\textwidth}{X} 他把他榮耀國度的豐富和他偉大威嚴的尊貴給他們看了許多日子,共一百八十天。 \end{tabularx} \\ \\ \relax
1:5 & \begin{tabularx}{0.7\textwidth}{X} 這些日子滿了,王又為所有住書珊城堡的百姓,無論大小,在御花園的院子裡擺設宴席七日。 \end{tabularx} \\ \\ \relax
1:6 & \begin{tabularx}{0.7\textwidth}{X} 院子裡有白色棉和藍色線,用細麻繩、紫色繩繫在白玉石柱的銀環上,又有金銀的床榻擺在紅、白、黃、黑大理石鑲嵌的地上。 \end{tabularx} \\ \\ \relax
1:7 & \begin{tabularx}{0.7\textwidth}{X} 用金器皿盛酒,有很多不同的器皿,照王的厚意提供豐富的御酒。 \end{tabularx} \\ \\ \relax
1:8 & \begin{tabularx}{0.7\textwidth}{X} 飲酒有規定,不准勉強人,因為王吩咐宮裡所有的臣宰,讓人各隨己意。 \end{tabularx} \\ \\ \relax
1:9 & \begin{tabularx}{0.7\textwidth}{X} 瓦實提王后在亞哈隨魯王的宮內也為婦女擺設宴席。 \end{tabularx} \\ \\ \relax
1:10 & \begin{tabularx}{0.7\textwidth}{X} 第七日,亞哈隨魯王飲酒,心中快樂,就吩咐在他面前侍立的七個太監米戶幔、比斯他、哈波拿、比革他、亞拔他、西達、甲迦, \end{tabularx} \\ \\ \relax
1:11 & \begin{tabularx}{0.7\textwidth}{X} 請瓦實提王后頭戴王后的冠冕到王面前,讓各民族和官員觀看她的美貌,因為她容貌美麗。 \end{tabularx} \\ \\ \relax
1:12 & \begin{tabularx}{0.7\textwidth}{X} 瓦實提王后卻不肯遵照太監所傳的王命前來,所以王非常憤怒,怒火中燒。 \end{tabularx} \\ \\ \relax
1:13 & \begin{tabularx}{0.7\textwidth}{X} 按王的常規,辦事必先詢問知例明法的人。那時,王詢問通達時務的智慧人, \end{tabularx} \\ \\ \relax
1:14 & \begin{tabularx}{0.7\textwidth}{X} 就是在王左右常見王面、在國中坐高位的波斯和瑪代的七個大臣,甲示拿、示達、押瑪他、他施斯、米力、瑪西拿、米慕干: \end{tabularx} \\ \\ \relax
1:15 & \begin{tabularx}{0.7\textwidth}{X} 「瓦實提王后不遵照太監所傳的王命,照例應當怎樣辦理呢?」 \end{tabularx} \\ \\ \relax
1:16 & \begin{tabularx}{0.7\textwidth}{X} 米慕干在王和眾官長面前回答說:「瓦實提王后這事,不但得罪王,並且有害於亞哈隨魯王各省的臣民。 \end{tabularx} \\ \\ \relax
1:17 & \begin{tabularx}{0.7\textwidth}{X} 因為王后這事必傳到眾婦人那裡,她們就會藐視自己的丈夫,說:『亞哈隨魯王吩咐瓦實提王后到王面前,她卻不來。』 \end{tabularx} \\ \\ \relax
1:18 & \begin{tabularx}{0.7\textwidth}{X} 今日波斯和瑪代的眾夫人聽見王后這事,必向王所有的官長照樣說,如此必造成無數的藐視和憤怒。 \end{tabularx} \\ \\ \relax
1:19 & \begin{tabularx}{0.7\textwidth}{X} 王若以為好,請降諭旨,寫在波斯和瑪代人的條例中,永不更改,不准瓦實提再到亞哈隨魯王面前,把她王后的位分賜給比她更好的妃子。 \end{tabularx} \\ \\ \relax
1:20 & \begin{tabularx}{0.7\textwidth}{X} 王的諭旨一傳遍全國,國土縱然遼闊,凡作妻子的,無論丈夫是尊貴或卑賤,都必尊敬他。」 \end{tabularx} \\ \\ \relax
1:21 & \begin{tabularx}{0.7\textwidth}{X} 王和眾官長都以這話為美,王就照米慕干的建議去做。 \end{tabularx} \\ \\ \relax
1:22 & \begin{tabularx}{0.7\textwidth}{X} 王下詔書,用各省的文字、各族的語言通知各省,使凡作丈夫的在家中作主,各說本地的語言。 \end{tabularx} \\ \\
[1ex]
\hline
\hline
\end{longtable}
$^{1}$《一字二十二字》 詞:陳曦 曲:陳曦.
阿哈徐老作王.
從印度直到古實.
統管一百二十七省.
阿哈徐老王在書山城的宮登基.
在位第三年.
為他一切首領臣服設擺筵席.
有波斯王馬袋的權貴.
就是各省的貴就與首領在他面前.
他把他榮耀之國的豐富.
和他美好威嚴的尊貴.
給他們看了許多日.
就是一百八十日.
這日子滿了.
又為所有住書山城的大小人民.
在御園的院子.
屢次擺筵席七日.
有白色,綠色,藍色的杖子.
用細麻繩,紫色繩.
從銀環內繫在白玉石柱上.
有金銀的床塔.
擺在紅,白,黃,黑玉石的鋪石地上.
用金器皿賜酒.
器皿各有不同.
御酒神多.
足顯王的厚意.
喝酒有禮.
不准勉強人.
因王吩咐宮內的一切神材.
讓人覺取己意.
皇后雅實堤在阿哈徐老王的宮內.
也為婦女設擺筵席.
第七日.
阿哈徐老王喝酒.
心中快樂.
就吩咐在他面前.
侍納的七個太監.
米胡曼,比斯他,哈波娜,比格他.
阿拔他,西達,甲家.
請皇后雅實堤.

$^{41}$投帶皇后的冠冕到王面前.
使各等臣民看她的美貌.
因為她容貌神美.
皇后雅實堤卻不肯.
遵太監所傳的皇命而來.
所以皇臣發怒.
心如火燒.
那時.
在王左右上常見王面.
國中坐高位的.
有波斯皇馬代的七個大臣.
就是甲斯娜,士達,阿瑪他.
他斯斯,米力,瑪西娜,米姆干.
都是達時務的名節人.
按皇的常規.
辦事先順民之禮明法的人.
皇問他們說.
皇后雅實堤不遵太監所傳的皇命.
照例應當怎樣辦理呢.
米姆干在王和眾首領面前回答說.
皇后雅實堤這事.
不但得罪王.
並且有害於王國省的臣民.
因為皇后這事必傳到眾婦人的耳中.
說哈徐老王吩咐皇后雅實堤到王面前.
她卻不來.
她們就藐視自己的丈夫.
今日波斯王馬代的眾婦人聽見皇后這事.
必向王的大臣照樣行.
從此必大開藐視和憤怒之端.
皇若以為美.
就降旨寫在波斯王馬代的禮中.
永不更改.
不准雅實堤再到王面前.
將她皇后的位分賜給她還好的人.
所降的旨意全片通過.
所有的婦人.
無論丈夫貴賤都必尊敬她.
王和眾首領都以米姆干的話為尾.
王就照這話去行.

$^{81}$發詔書用各省的文字.
各族的謊言通知各省.
使為丈夫的在家中作主.
各說本地的謊言.
上個月我知道粵題是Sorry的時候.
當我構思講章的時候.
我和一個英文不太好的小朋友.
溫習的時候.
他就見到一個這樣的題目.
請PowerPoint的弟兄幫我按一下.
看到了嗎?.
按吧.
看到一個英文題目.
就是「Sorry for your loss」.
那個小朋友就立刻回答.
「Say! Say sorry」.
然後我就跟他說.
Sorry不一定是say.
可以是feel.
可以是work to be.
今天是粵題Sorry的最後一講.
我想到的是.
一個人無論是和上帝說Sorry.
和別人或自己說Sorry都好.
是需要經過一番自我的反省.
才可以很油衝地說一句Sorry.
所以今天就藉著《耳私貼記》第一章.
我們去想想個人內省的課題.
我不知道當提起《耳私貼記》的時候.
大家第一印象是什麼?.
當然很容易想到《耳私貼記》.
因為書卷就是以它命名.
還有很萬能的經文.
總之有什麼情況就跟你說.
「焉知你得了皇后的位分」.
「不是為了現今的機會嗎?」.
就叫你去做.
除此之外.
大家對《耳私貼記》還想起什麼呢?.
我不知道大家.

$^{121}$高高為我們讀《耳私貼記》第一章的時候.
我不知道過往有沒有很用心地去看過這章經文.
大家有沒有留意到當中提及的皇后雅實題呢?.
皇后雅實題只是很短暫出現在《耳私貼記》第一章.
她因為拒絕阿哈徐魯的召喚.
拒絕出席王的筵席.
被王和一班王室顧問褫奪了她皇后的位分.
勒令她永遠不可以再見王.
很快就被罪事者寫出整個《耳私貼記》的罪事.
當然如果我們有稍稍讀過《耳私貼記》的話.
我們可能會被書卷裡面的哈曼.
即是一個惡人.
或是一個英雄的耳私貼去吸引.
無疑他們兩個的形象是很典型.
自然會令人留下深刻的印象.
但雅實題呢?.
我想說雅實題對於我們基督教或基督徒來說.
是一個很有趣的課題.
為什麼這樣說呢?.
傳統教會通常去解讀我們剛才所讀的經文的時候.
會認為雅實題拒絕亞哈徐魯的要求.
他們第一個反應是甚麼呢?.
他們不聽話.
是一個不服從的表現.
他們是應該被廢的.
所以很多時候教會對這個敘述有一個很片面的看待.
又或者我們再看《耳私貼記》的時候.
很多時候她都是一個很順從.
不說話,默默做事的一個女性.
不過當我再看《耳私貼記》第一章的時候.
發現這個幾乎被人遺忘的角色.
其實有很多讓我們反省和思考的地方.
今天就透過雅實題看看對我們有甚麼啟發.
首先介紹一下雅實題.
根據猶太拉比對聖經的傳息.
雅實題是一個甚麼身份呢?.
雅實題是巴比倫王室的後裔.
是尼布格尼薩爾薩的元孫.
白沙薩的孫女.
年紀輕輕的雅實題就在但爾利牆上的文字.

$^{161}$即是那晚預言白沙薩被殺的那一晚.
事件就記載在但爾利書第五章.
她就被當時的馬代人放過了.
即是沒有留下活口,沒有殺了她.
但劉氏王,即是但爾利書第六章.
那個扔但爾利入獅子坑的那一位.
就利用雅實題作為鞏固自己政治的一個資產.
利用它做甚麼?.
將她嫁給自己一個很忠實的臣子.
一個前途無可限量的年輕軍人.
瓦哈徐魯.
為甚麼要安插一個人在瓦哈徐魯身邊?.
就好像放個善人在他身邊.
以斯帖記一章一節這樣說.
「這是發生在瓦哈徐魯的時代.
瓦哈徐魯從印度直到古實統治127個省.
這裡沒有稱呼他為王.
所以猶太拉比估計.
當時瓦哈徐魯還未正式成為王.
他的確是大樓市一個很驍勇善戰的一個將帥.
不過給我們看到的是.
他的行軍策略已經促成了波斯帝國的一個初型.
已經有127個省的規模.
而在這次大規模擴張之後.
帝國裡面出現了一場爭奪權力的鬥爭.
當然既然剛才我所說.
瓦哈徐魯是一個很能打的將帥.
他的軍事成就自然就是成為了.
統治波斯帝國的一個合理候選人.
又例如就成如我之前所說.
瓦哈徐魯和瓦什提其實只是一宗政治婚姻.
當瓦哈徐魯的勢力未穩固的時候.
他只可以容讓瓦什提在他身邊.
但當他慢慢手握軍政權的時候.
他就想剷除前朝軍王安插在他身邊的人.
他很希望可以從皇后手中取回所有的控制權.
希望可以集中所有的事務在他手中.
一章三節說他在位第三年就擺設賢直.
這個賢直其實是想將皇后剷除其中一個計劃付諸實行.
經文講到就是剛才我們讀經文的時候.

$^{201}$就看到第一個賢直為期長達六個月.
講的是他宴請帝國裡面一些很重要的人物.
包括波斯的軍官貴族地區長官等等.
在裡面向他們展示自己的財富和軍政權力等等.
然後緊接著第一個賢直之後就有第二個賢直.
就是向書山城大概是首都首府的所有人.
去開放自己宮殿的花園.
第八節講的是讓他們可以肆無忌憚地去喝酒.
為甚麼呢?.
是要確保所有人包括阿哈徐老自己都喝夠酒.
然後可以方便將之後發生的事情歸咎於醉酒的影響.
而不是單純一個精心策劃了很多日的宮廷政變.
一章九節都講給我們聽.
就是在王擺設賢直的時候.
其實皇后雅實緹都在宮裡面招待一班婦女.
她當時其實是有事要負責的.
不過我們再看的時候.
醉使者將婦女的賢直和王的賢直平衡地放在一起.
但是我想我們剛才聽的時候.
發覺醉使者花了很多筆墨去描述王所擺的賢直.
講了很多的佈置等等.
但是皇后的賢直卻只是很簡單一句地交代.
這裡給我們看到一個很強烈的對比.
我們看到這裡的時候會問.
為甚麼波斯男女要分開喝酒,吃飯呢?.
不是單純因為男女之隔等等.
而是曾經有學者對當時的波斯文化有一個這樣的研究.
他說當時代的波斯人有幾個特點.
第一,他們會喝酒.
第二,他們很喜歡搶東西.
第三,波斯男人通常都是好色之徒.
這樣的安排,為甚麼男女要分開喝酒,吃飯呢?.
是要令女性不會在賢直期間.
被那些喝酒的男人去借醉行兇,被侵犯.
這就是男女分開喝酒,吃飯的其中一個原因.
到了第十至十一節就說.
喝酒喝到第七天.
王就很興奮.
阿哈徐魯就將瓦實緹召喚到男性的賢直裡面.
《經文》說美其名是想向眾人炫耀皇后的美貌.

$^{241}$不過如果我們再聽一些背景的時候.
是否純粹想炫耀皇后的美貌呢?.
或者說得白一點,其實他想皇后來是甚麼呢?.
是要她去招呼一班被大量酒精影響的男人.
至於是怎樣的招呼呢?.
這是值得我們去思考的.
不過我們再看《經文》,究竟是想她怎樣招待呢?.
《經文》說是請七個太監.
然後請瓦實緹投帶皇后的冠冕到王面前.
瓦實緹聽到的是投帶皇后的冠冕.
猶太拉比的詮釋是.
要求皇后只是帶皇后的冠冕來到這班男性的賢直.
簡單來說就是叫她帶著後冠.
然後在王和所有醉酒的賓客面前赤身露體.
我想說這個要求不單是要她將自尊放低.
更加是對一個女性一個很大的侮辱.
我們再看《經文》,七這個數字其實在第一章三次重複出現.
首先第一次是第七天.
然後王吩咐七個太監傳召皇后瓦實緹到來.
然後在皇后要想怎樣懲處瓦實緹的時候.
她也是向身邊七個智慧人諮詢.
而瓦實緹這個名字在這章也七次被提及.
七這個數字對猶太人來說是一個很特別的數字.
但對瓦實緹來說也對她的命運起了一個很關鍵的作用.
醉使者特別提到王的心因酒而歡喜.
除了因為酒,我也相信是因為賓客面對著王的富有和慷慨.
對他有很多的讚美,阿諛奉承等等.
王在賓客面前不斷的吹捧下.
阿哈徐老慢慢的自我膨脹到一個地步.
要求七個太監領帶著皇冠的皇后瓦實緹到王面前.
向民眾和領袖展示她的美貌.
希望用這個作為筵席的壓軸戲.
我想說阿哈徐老這個做法其實就是將瓦實緹當作一件戰利品.
除了炫耀自己在軍政權的成就之外.
還想藉著皇后贏得一班大臣對她的效忠,支持,擁戴.
我想說這個做法就像將瓦實緹當作其中一個財產.
她在物化瓦實緹.
將她和她的身體當作娛樂醉酒賓客的玩物.
我想說當下瓦實緹可以有兩個選擇.
第一就是失去尊嚴.

$^{281}$她可以去,但她在皇后的日子可以繼續做皇后.
第二就是不去.
她會失去皇后的位份,一切的享受.
但她可以保持著作為人的尊嚴.
那一刻瓦實緹心裡究竟在想甚麼呢?.
如果她被虛榮心去充分投老.
或者她習慣了享受福縣廣大的國家所帶給她的榮華富貴.
她可以將這種侮辱視作等閒.
她可以跟自己說.
其實她只是展示自己有多漂亮.
她可以這樣說,經文也是這樣說.
她可以跟自己說,其實一群人都醉了.
延籍過後發生了甚麼事.
大家都可能不記得了.
或者瓦實緹也可以跟自己說.
那個時代就是鼓勵女人不加思索地去順從男人,聽男人的話.
當時很多女性都是默默無聲地去遵從這樣的要求.
有時候很多人覺得順服是一個讓人得到提升,攀升的機會.
有些人因而選擇背棄了自己的信念.
瓦實緹可以給自己很多很多的原因去游說自己.
去吧.
不過,瓦實緹面對著眼前一個絕對的權力.
我想她經過一番反省和思考.
她最後選擇了甚麼.
她沒有屈服,她選擇了抗爭.
她拒絕迎合黃,不合理的要求.
她拒絕成為男人的性對象和玩物.
她沒有將她自己本身作為皇后的既得利益放在一個不能動搖的位置.
她寧可面對因為反抗而帶來的失去或危機.
她也致力維護她作為人的尊嚴.
她選擇不做一個順民的皇后.
而是用行動去反抗這個無理的要求.
她跟黃說不,她say no.
我想說她的拒絕是顛覆了男人對女人物化的願望.
她也行使了她作出選擇的自由.
瓦實緹給我們看到的是.
她不只是外表漂亮.
而是她能夠權衡外面複雜的情況.
並且能夠作出決定,是一個很自主的女性.
我想強調的是.

$^{321}$瓦實緹那種堅持或抗爭並不是廉價的.
女性的反抗在當時是沒有人能猜到的.
並且當你反抗的時候.
男人會視為一個很大的威脅.
所以經文給我們看到的是.
一群有權有勢,但醉酒不理智的男人要怎樣?.
他決定要將皇后瓦實緹從她的位置拉下來.
她的反抗令她失去皇后的地位.
但她卻願意承擔這個後果.
願意犧牲自己皇后的位份.
也不願意輕易拋下自己作為人.
一個很重要的核心價值.
儘管她放棄的是什麼?.
放棄的是一個從印度到埃塞俄比亞.
遍及127個省的王國.
她的自尊和榮譽感比整個王國都要高.
都要重要.
雖然瓦實緹從此在《以斯帖記》中不再被提及.
缺席了.
不過,她的缺席我們可以看為是一種留白.
她仍然在發聲.
她用行動去表明.
她沒有屈服在不合理的強權下.
19世紀英國有位著名的貴官詩人這樣說.
他說一個人的自我反省.
自我認識.
自我控制.
這三者可以將一個人的生命.
引向一個很大的力量.
我想說這個力量的大.
可以令瓦實緹用行動.
向當時握有權力的男性.
去表達她的立場.
她拒絕為醉酒的丈夫.
放下她作為女性的尊嚴.
她就是因為她敢於表達立場.
而被記錄下來.
我們想想.
如果瓦實緹就像當時其他女人一樣.
隨波逐流.

$^{361}$男人叫她做什麼就做.
不想想是否合理.
不反抗.
只是做一個順命的皇后.
或許歷史很快就忘記她.
因為她只是一個為利益隨波逐流的女人.
不過瓦實緹沒有盲目地服從不合理的當權者.
或許她那一刻已經做好了.
隨時做好準備死的準備.
的確我們看到.
她打算死是一個很合理的推測.
因為她所認識的瓦剋隨路是什麼.
她的反抗激怒了她.
原文形容她怎樣生氣.
氣得好像被火燒一樣.
帝國裡也沒有人站在她那邊.
如果我們再看的時候.
在整個敘事.
有一個皇室貴就向她伸出援手.
大概就是皇的意志已經凌駕了全國人的常識.
縱然大家都知道這是一個完全不合理的命令.
不過她在現實裡看到的是.
有七個皇室顧問.
她是趨炎附勢.
他們把握機會向皇提議.
為瓦剋提議度身訂做一個法例.
怎樣呢?.
做低她.
趕走她.
過往去讀《以斯帖記》的時候.
我們都會視《以斯帖記》,《末底改》.
是普爾哲的英雄.
因為的確是他們兩個合力拯救猶太人.
免遭滅族之難.
不過在我看來.
我覺得瓦剋提也是一個英雄.
她拒絕在皇和一群醉酒的人面前貶低自己.
她選擇重視自己固有的信念.
不願意屈服在丈夫無理的要求.
不願意接受那些示意對她的踐踏.

$^{401}$我會形容瓦剋提為個性很倔強的人.
她不是利用自己的美貌.
女性的身份去為自己謀利益.
不是利用這些去找著數.
相反她很清楚她是一個有原則.
有所為有所不為的女人.
瓦剋提的反抗令我想起五月天的《倔強》.
我想大家看看下一張PowerPoint.
《倔強》這首歌其實已經推出了很久.
在五月天,當她在滾石唱片年代.
這首歌的歌名亦叫做《Stubborn》.
不過當她轉到上一張圖片.
即是她轉到《相信音樂》即是現在的公司.
歌名亦叫做《Persistence》.
頂姐妹,我不知道你怎樣看「固執」這個字.
固執看起來是一個很負面的詞彙.
不過中國人有句說話叫做「擇善而固執之」.
重要的是我們為了甚麼去固執呢?.
其中有幾句歌詞很觸動我.
我不會唱歌,放心.
它是這樣說的.
「當我和世界不一樣,那就讓我不一樣」.
「堅強對我來說,就是以剛克剛」.
「如果我們對自己妥協,如果對自己說謊」.
「即使別人原諒,我也不能原諒」.
還有一句.
「逆風的方向更適合飛翔」.
「我不怕千萬人阻擋,只怕自己投降」.
我想說這堆歌詞.
很有種「道之所在,雖千萬人吾往矣」的氣魄.
即是我堅持的是真理的話.
縱然有千萬人阻擋我.
但我都一樣繼續去.
瓦薩提的倔強,瓦薩提的堅持.
是因為他重視他心裡一個很重要的價值觀.
他寧可冒著失去地位的風險.
都要維護作為女性為人的尊嚴.
拒絕王,不公義,不文明的要求.
我想說瓦薩提的拒絕行動.
被他當時代的人指責是一種傲慢,不明智.

$^{441}$又或者是現在大男人主義的教會.
都用相似的力度去抨擊他.
不過我想說通過他反抗屈服在男性手下.
他其實是在為一些無法為自己發聲的女性.
做了一個榜樣.
即使這樣做要付上很高昂的代價.
但因為他有自己所重視的信念.
這個信念勝過世界主流的價值觀.
他就是身體力行地實踐對女性的解放.
縱然革命尚未成功.
不過他至少都走了第一步.
所謂的希望不是確信事情一定會有好的結果.
而是無論結局都肯定自己所做的每一度微小.
都有他自己的價值.
瓦薩提就是用他一人之力去抵抗整個波斯的體制.
看起來好像徒勞無功.
不過我想說他的行動.
卻是為這個荒謬的世代賦予多一點意義.
頂姐妹,今天上帝將我們擺在此時此地.
或者我們都面對著很多荒謬的要求.
我們可能會面對著一些無理的指責和批評.
當外人對我們不明白,不解,鄙視.
或者是迫不得已的時候.
我們能不能像瓦薩提一樣去堅持自己的信念呢?.
堅持上帝置放在我們心中的良心和真理呢?.
作為記憶徒,我們能否在艱難的處境.
仍然堅持進行上帝頒布給我們的誡命呢?.
我們願不願意去踐行上帝給我們的良心呢?.
在瓦薩提的敘述中,不知道大家看到些甚麼呢?.
有沒有看到你自己的影子?.
我們是否處於一個窒礙我們體驗神完全榮耀的境況呢?.
昔日我們可能對一些不好的環境.
或者對我們靈性有破壞的處境.
我們可能會捉緊就命.
我們可能會給自己很多理由.
一人在江湖,身不由己,不行的.
我們可能曾經默許過這些事情加諸在我們身上.
但今天我們會否因為上主的緣故.
而對這些對我們具破壞性的環境說NO呢?.
我覺得如果我們曾經順命,認命,捉緊就命.

$^{481}$我們仍然可以.
所以今天要祂說NO,永遠都不會太遲.
我想說,作為上主的門徒.
縱然處身在艱難的現實.
但我深信亦無改上帝賜給我們.
靈性,正直,公義的特質.
祂就是藉著我們,使用我們在此時此地.
在這個時代,在這個世代去彰顯.
我相信,如果我們選擇堅持上帝的真理的時候.
我相信總有一天,會證明我們是站在上帝那一邊.
如果我們選擇為上帝而做的時候.
我相信總有一天,上帝會為我們所作真誠的行動,抉擇.
對我們作出肯定.
總有一天,我們會被上帝所認同.
求主幫助我們,讓我們一同低頭祈禱.
親愛的上主,我們有時也要向你承認.
在現在的世代,在現在的處境.
要堅持對的信念,持守真理,真的很不容易.
特別是在現在荒謬,顛倒是非,邪惡的世代.
大主你將真理,靈性,良心賜給我們.
讓我們知道何謂善,何謂美.
主呀,求你加能賜力給我們.
讓我們在這裡仍然能夠散發人性美好的光輝.
也求你讓我們對你仍然有盼望.
我們深信上主你公義的審判.
有一天會臨到這個地上.
惡人會遭報,義人會被主你所肯定和掌上.
求主就是這樣幫助我們.
在這條路上繼續走下去.
願意我們走得好,終結得好.
求你就是這樣幫助我們.
多謝你聽我們祈禱.
奉耶穌基督的聖名自祈求.
阿們.
主呀,求你加能賜力給我們.
願意我們走得好,終結得好.
阿們.
主呀,求你加能賜力給我們.
願意我們走得好,終結得好.
阿們.

$^{521}$多謝您收睇時局新聞,再會!.
\newpage



\section{詩篇 98:1-9-20230506}
\label{sec:D8sOzznkhGg}
\textbf{【流堂崇拜】We sing everywhere|詩篇98\_1-9|20230506 [D8sOzznkhGg]}
\newline
\newline
連結: \href{https://youtube.com/watch?v=D8sOzznkhGg}{\texttt{ https://youtube.com/watch?v=D8sOzznkhGg}} ~~~~ 語音日期: 2023-05-06 
\newline
\newline
\hyperref[sec:VNZbDAiXlG0]{\small{< < < PREV SERMON < < <}}
~
\hyperref[sec:index_chronic]{\small{[返順時目]}}
~
\hyperref[sec:index_scriptual]{\small{[返順卷目]}}
~
\hyperref[sec:WqLko1gsXRE]{\small{> > > NEXT SERMON > > >}}
\newline
\newline
詩篇 98:1-9-20230506
\newline
\begin{longtable}{cl}
\hline
\hline
章節 & 經文 (和合本修訂版)\\
\hline
98:1 & \begin{tabularx}{0.7\textwidth}{X} 你們要向耶和華唱新歌!因為他行過奇妙的事,他的右手和聖臂施行救恩。 \end{tabularx} \\ \\ \relax
98:2 & \begin{tabularx}{0.7\textwidth}{X} 耶和華顯明了他的救恩,在列國眼前顯出公義; \end{tabularx} \\ \\ \relax
98:3 & \begin{tabularx}{0.7\textwidth}{X} 記念他對以色列家的慈愛和信實。地的四極都看見我們神的救恩。 \end{tabularx} \\ \\ \relax
98:4 & \begin{tabularx}{0.7\textwidth}{X} 全地都要向耶和華歡呼,要揚聲,歡唱,歌頌! \end{tabularx} \\ \\ \relax
98:5 & \begin{tabularx}{0.7\textwidth}{X} 用琴歌頌耶和華,用琴和詩歌的聲音歌頌他! \end{tabularx} \\ \\ \relax
98:6 & \begin{tabularx}{0.7\textwidth}{X} 用號筒和角聲,在大君王耶和華面前歡呼! \end{tabularx} \\ \\ \relax
98:7 & \begin{tabularx}{0.7\textwidth}{X} 願海和其中所充滿的澎湃,願世界和住在其間的發聲。 \end{tabularx} \\ \\ \relax
98:8 & \begin{tabularx}{0.7\textwidth}{X} 願大水拍掌,願諸山在耶和華面前一同歡呼; \end{tabularx} \\ \\ \relax
98:9 & \begin{tabularx}{0.7\textwidth}{X} 因為他來要審判全地。他要按公義審判世界,按公正審判萬民。 \end{tabularx} \\ \\
[1ex]
\hline
\hline
\end{longtable}
$^{1}$今天我們在中環碼頭去結集.
我們一起敬拜我們的上帝.
我們一起大聲開口.
去讚美我們的上帝.
我為了今天這個聚會.
非常感恩.
也非常自豪.
我們做了一件很有意義的事情.
甚至我覺得是一個壯舉.
今天我們在香港這個公共空間裡面.
幾百人.
我們這裡好像有大概有.
我的名字叫陳偉安.
我們是一群基督徒.
我們是一間香港的教會.
我們的教會的名字叫做.
Flow Church 流堂.
今天講道的講題.
叫做We Sing Everywhere.
其實是特意的.
今天我特意選了一個英文的題目.
We Sing Everywhere.
因為我知道如果我的題目是中文的話.
好像會有些不太禮貌.
我們到處唱.
我們到處唱我們的神.
我們好像不是很有厚度.
所以我們用英文比較好一點.
今天我們會講的經文是C篇.
那十八篇.
一段我會去講一下我們今天去唱歌的原因.
We Sing Everywhere的原因.
我們到處唱的原因.
我們一起去做另一套創舉好不好.
我們一起來去中讀上帝的話.
在這個地方中去中讀上帝的話.
我們大聲地讀出這段經文好不好.
C篇第九十八篇經文.
如果你是看不到的話.
你可以打開你的手機出來.

$^{41}$C篇第九十八篇的經文.
我們向這個世界宣讀上帝的話.
預備起.
.
是你拯救我們的好消息.
求主你讓我們今天可以在你這裡.
知道我們怎樣來做你的門徒.
開口敬拜你.
將我們的生命成為一個敬拜的聲音來獻上.
求主你這樣去焦聚我們流唐群體.
為我們在網上,在中環,在各地的電視節目,在電影裡面.
去將我們的生命教堂.
願我們流唐成為一把聲音去呈獻給你.
奉主命求,阿門.
詩篇98篇是一篇有關敬拜的詩篇.
這首詩篇的主題是聲音.
是上面一切的聲音.
這個聲音更加是有方向的.
這個聲音是有主題的.
聲音是歸向耶和華上帝的聲音.
是讓我們一起去看這段詩篇.
詩篇98篇的第一節.
詩人是去發出一個公開的邀請.
甚至是向全人類發出一個公開的邀請.
你們要向耶和華唱新歌.
為什麼我們基督徒經常都要唱歌.
大家有沒有想過,為什麼要唱歌.
為什麼要唱歌,為什麼不是跳舞可以嗎.
沖咖啡可以嗎,打關斗可以嗎.
你們要向耶和華打關斗都可以的.
你知道一件很真實的事.
不是每個基督徒唱歌都像高高或Alice那樣.
在教會裡面真的有不少不一不全的基督徒.
都大有人.
特別敬佩唱歌不是最好聽的人在教會裡面唱歌.
唱歌好聽是很自然的,你應該唱歌.
所以我覺得這很合理.
但如果一個人唱歌不是最好聽.
但仍然能夠投入開口去敬拜的時候.
這才是我最佩服的地方.

$^{81}$所以每次我見到頂尖會很用力去讚美這件事.
是非常敬佩,非常美麗的事情.
事實上詩篇98篇是一篇這樣的邀請.
是一個向全人類發出歌唱的邀請.
為什麼基督徒要唱歌.
我們可以見到詩篇告訴你一個很簡單直接的原因.
第二次這樣說.
因為耶和華行過奇妙的事.
他的右手和聖臂為他施行救恩.
詩篇98篇告訴我們.
我們唱歌是因為上帝已經施行了拯救.
上帝拯救了我們,拯救了這個世界.
這個就是我們唱歌的原因.
然後詩篇98篇告訴我們.
上帝不單止拯救了我們.
他更加顯明了他的拯救.
叫我們能夠見到他的救恩.
第二次這樣說,耶和華顯明了他的救恩.
在你眼前顯出公義.
紀念他對耶穌的慈愛和信實.
地的四極都看見我們上帝的救恩.
所以你是不能夠不唱的.
無論你唱成怎樣.
懂不懂和音,震音,假音都好.
你不能夠不唱.
我們的救主已經拯救了這個世界.
他已經做了,已經通知了我們.
所以你不能夠說我不知道.
我看不見,收不到通知.
因為上帝不單止是去拯救.
更加顯明了他對世界的拯救.
因為上帝拯救了我們.
所以我們就唱歌.
就是一個這樣的原因.
或者你未必很懂得去唱歌.
但我們每個人都懂得能夠開口去歌頌.
這個是我們對於上帝作為的回應.
我們有意識的去做.
隨心自然地去做.
這個就是我們今天唱歌的原因.

$^{121}$不過更重要的是.
詩篇那十八篇其實不單單是一個邀請你去敬拜.
詩篇那十八篇更加是邀請整個世界的敬拜.
整個世界一起去發出歌聲.
整個世界去回應上帝.
這樣的一個傳遞救恩的邀請.
所以經文這樣強調.
第三節說.
地的四極都看見我們上帝的拯救.
因為上帝拯救了整個世界.
所以全地都看見祂的救恩.
所以全地都成為一股敬拜的聲音.
所以發覺從第四節到第八節那裡.
詩篇那十八篇就這樣的戲謔說.
整個的全地一切所有萬物都來去敬拜.
正話全地都要向耶和華歡呼.
要揚聲歡呼歌頌.
用琴歌頌耶和華.
用琴來跟詩歌的聲音去歌頌祂.
用號角的聲音在大公王耶和華面前歡呼.
願海和其中充滿的澎拜.
願世界和著其中的發聲.
願大水拍掌.
願珠珊在耶和華面前歡呼.
這個是整個世界發出的聲響.
因為整個世界都知道.
耶和華上帝的拯救.
去回應耶和華上帝的拯救.
這次就明白.
為什麼我們基督徒詩歌耶.
就是星之維.
基督徒那麼多填詞人.
寫了那麼多詩歌.
詩歌不斷都是講什麼.
都是那些大自然.
花花草草.
海洋.
石頭.
浪花.
風雨電.

$^{161}$敬拜讚美的詩歌.
剛才我們唱了不少是不是.
就是天地讚美.
從日落之處到日落那方.
還有無言的讚頌.
要吶喊要頌揚.
這是天賦世界.
還有人類群星閃耀時.
是的.
就是first take很棒.
我覺得這首是詩歌.
基本上是講世界上上帝的創造.
盤石唱讚美歌聲.
風雨電也歌頌.
雲霧同眾吹頌.
全地奏響聲勢無比.
從日落之處到日落那方.
全地高聲重讚.
還石都開口.
深海在唱歌.
榮耀至聖上帝.
這是天賦世界.
我們質議靜聽.
宇宙唱歌.
四周響應.
星辰作樂同聲.
從太陽出來之地.
直到日落那邊.
處的明.
當讚美的.
你發現無論是聖經.
這都是我們很熟悉的詩歌.
充滿著都是那些花花草草.
森林小鳥.
海洋大地.
這樣的一些讚美的聲音.
你發現無論是這些詩歌.
都可以聽到一個很重要的訊息.
就是天地.
風海.

$^{201}$全部這些都一起去讚美上帝.
你問為甚麼.
為甚麼會這樣.
基督教小說是否真的喜歡玩這些東西.
喜歡一些寫景.
是不是要講到盤石唱讚美詩歌.
激起浪花那個.
已經會高一點.
所以我們更加要寫一些山水樹木在那裡.
是不是加了兩隻白斑會更加high一點.
不是.
我們這樣唱這些天地讚美.
不是純粹因為二人法.
不是純粹加一些山水樹木.
還是甚麼.
而是地的四極都知道.
都看見上帝的拯救.
所以地的四極.
都是發聲敬拜我們的上帝.
這個是世界存在的基本東西.
事實上這個世界一直都是敬拜我們的上帝.
昨日如是.
今天如是.
將來也如是.
這個世界很乖.
比我們人類更加乖.
這個世界按著上主的命定.
去到開口發聲去敬拜我們的上帝.
從耿古到今日.
這個世界沒有一刻是停止過去發聲去敬拜我們的上主.
大海的聲音.
空氣流動的聲音.
雷電雨點的聲音.
大地的聲音.
沒有一刻是停止敬拜我們的上帝.
這段時間剛剛是初夏.
大家都很熱.
但大家不要只看著很熱.
最近其實花開得很漂亮.
大家不要發覺.

$^{241}$花開得非常燦爛.
如果你們有機會來長洲我家的話.
或者我去office的話.
我養了很多鮮花在花園.
這個時候正是施肥的時候.
因為是非常漂亮.
非常璀璨的時候.
所以有機會你買盆花回家.
看看這朵花如何來敬拜上帝.
每一個季節.
每一個角落.
每一個生命的存在.
都在敬拜我們的上主.
這個就是我們世界的本相.
這個就是天父的世界.
聽子妹.
We sing everywhere.
因為everywhere都在敬拜我們的上主.
我們來到這裡不是因為我們在流唐敬拜.
而是我們參與這裡一直敬拜的敬拜.
我們參與維多利亞港.
每天每分每刻的敬拜.
我們來到中環這個土地裡面.
在香港這個天空之下.
我們和這個世界一起去敬拜我們的上主.
我們去敬拜創造我們的主.
拯救我們的主.
這個就是我們今天來到這裡的目的.
We sing everywhere.
不過.
世上仍然有些人.
並不是這樣.
你知道.
世上人類.
總是這個世界的例外.
雖然全地都在敬拜.
地的世界都看到上帝的救恩.
但是唯有人類並不是每個人都知道.
唯有人類需要被提醒.
唯有人類並沒有開口去敬拜我們的主.

$^{281}$雖然上主已經顯明他的救恩.
但是世界上仍然有不少的人.
他們不知道.
也沒有去獻上自己的敬拜.
人類知道很多東西.
但唯有不知道的原來就是.
他需要去敬拜他的主.
親愛的弟兄姊妹.
各位流氓弟妹.
這個就是我們今天來到這裡去敬拜.
我們上主的另一個很重要的目的.
我們的敬拜.
我們的聲音.
剛才大家每一個很用力發出的聲音.
正正是去告訴這個世界.
上帝的拯救.
上帝的公義.
上帝的審判.
因此.
《詩篇教百篇》最後一次這樣說.
因為他來要審判全地.
他要按公義審判世界.
按公正審判萬民.
這個是世界敬拜的聲音.
指向世界終末的公義審判.
這個世界敬拜的聲音.
叫我們在這個世界裡面.
知道上帝的公義將要立到.
我們知道世界的救主的拯救.
弟妹.
正正是因為這個原因.
我們要去開口去唱歌.
因為我們知道.
我們作為基督徒.
我們是一群基督徒.
我們知道上帝的救恩.
我們更加盼望上主公義的來臨和審判.
因此.
留堂頂子妹.
我們的敬拜不是一群自嗨.

$^{321}$不理世事的敬拜.
世上有很多這樣的敬拜.
但我們不是.
最少我們不是.
我們剛好相反.
因為我們關懷我們的地方.
關懷這個世界.
所以我們更加要獻上我們的敬拜.
有人問.
這個世界已經如斯田地.
我們還可以高歌去敬拜嗎?.
他們問.
你們的基督徒敬拜.
能夠令這個地方變得更加好嗎?.
你們的敬拜有用嗎?.
沒錯,我們的敬拜未必能夠見上而將要離臨.
這個是我們今天在很多人面前敬拜的目的.
有人曾經這樣說過.
引用瑞士哲學家卡爾巴特的一句話.
雖然我作為研究卡爾巴特的人.
我不知道這句話是何時來的.
但他這樣說.
笑聲是最接近上帝恩典的東西.
我們的笑聲是世界上最接近上帝恩典的東西.
我覺得更加準確的說法是.
我們敬拜的聲音.
是世上最接近上帝公義的東西.
我們的聲音乃是上帝公義不滅的明證.
世上沒有人能夠來去蔑析敬拜的聲音.
沒有人可以.
因此,親愛的姐妹們.
讓我們將我們的敬拜放在我們今天這個年代的生命裡.
讓我們走到世界的每一個角落.
成為一道敬拜的聲音.
讓我們的哭泣化為公義.
期待公義來臨的歌聲.
讓我們可以在這片土地裡.
能夠見到上帝的恩典和拯救.
Everywhere, Everytime, everyone.
丁姐妹們,就快日落了.

$^{361}$你可以看一下我們的天空.
已經越來越暗.
黑夜將要來臨.
周杰倫演唱會也將要來臨.
明天將會是新的一天.
明天將會是我們再一次向耶和華上帝唱新歌的一天.
你們要向耶和華唱新歌.
那個新歌不是說你每次唱都要叫Alex唱新歌給你聽.
你們要向耶和華唱新歌.
因為每一次的敬拜都是我們的重新開始.
每一次我們去歌唱敬拜上帝.
都是我們一個重新的開始.
去認識世界的本相.
生命是一場敬拜.
再一場敬拜.
不斷下去.
頂多讓我們以敬拜來渡過我們的年月.
來迎向尚處的未來.
所以今天講道的總結很簡單.
只是幾句.
只是幾句沒什麼特別.
請你們繼續.
請你們繼續去敬拜.
請你們繼續你們如常的敬拜.
繼續你的歌聲.
用你的生命的聲音去證明.
這個世界是有上帝.
我們這個地方是有上帝.
因為祂拯救這個世界.
更加直至到祂公義再降臨.
讓我們去繼續我們敬拜.
\newpage



\section{}
\label{sec:gRf39gjSNbM}
\textbf{《致餘民及流散者:給香港基督徒的神學八課》第二季第3課|20230512 [gRf39gjSNbM]}
\newline
\newline
連結: \href{https://youtube.com/watch?v=gRf39gjSNbM}{\texttt{ https://youtube.com/watch?v=gRf39gjSNbM}} ~~~~ 語音日期: 2023-05-12 
\newline
\newline
\hyperref[sec:WqLko1gsXRE]{\small{< < < PREV SERMON < < <}}
~
\hyperref[sec:index_chronic]{\small{[返順時目]}}
~
\hyperref[sec:index_scriptual]{\small{[返順卷目]}}
~
\hyperref[sec:u6GL1Cm7cwU]{\small{> > > NEXT SERMON > > >}}
\newline
\newline
$^{1}$(第四章 基督徒).
我們是時代之子.
是被上帝揀選.
在這個年代見證耶穌的基督徒.
我們身於這個年代.
被這個年代分散.
有人流散到海外尋覓理想.
有人繼續在本土奮鬥下去.
但又如何.
基督徒仍然需要作基督徒.
香港人仍然是香港人.
究竟上帝的旨意如何.
我們應該如何生活.
我們應該如何為主而活.
無論你身處何處.
只要你是香港人.
邀請你跟我們一起思想.
這個流散年代的信仰.
致願民與流散者.
給香港基督徒的神學百科.
.
不過有一點關係.
勉強也說得通.
今天我們會討論一個主題.
就是身份認同.
就是身份認同這個詞語.
當我們想到這個詞語時.
便想到上帝和我們之間的身份.
坦白說,我跟潘老師預備的時候.
這不是我自己提出的主題.
因為我不懂得說.
這個詞不是我常用的詞語.
很少提到身份認同這個詞語.
這個詞也不算是神學詞語.
應該是一些心理學,社會學的詞語.
所以我很少用這個詞語說話.
平時很少提到身份認同的問題.
當我預備這課的時候.
我發覺對自己也有很大幫助.
雖然我很少用這個詞語說話.

$^{41}$但我有這些問題.
我有身份認同的問題.
我覺得自己也面對這些問題.
所以今天也是….
可能潘老師稍後問答時.
我們可以多說一點.
怎樣看我們身份認同這個詞語.
我覺得這個詞語.
中文翻譯出來也很特別.
身份,身份證.
用香港人的身份.
即是我們香港人對於我們一個群體.
一個很獨特,很獨特的身份.
就是我們香港人.
當這個社會不能滿足.
某個群體的身份認同的時候.
簡單來說,當香港人的身份價值.
還未在社會中被重視的時候.
因為多元,對吧?.
特別是香港社會是多元的.
當多元的緣故不被重視的時候.
他們就會強烈渴求,渴望,有尊嚴.
並且被重視.
重視自己的身份.
如何理解我們一群香港人.
所以這成為了一個政治訴求.
這就是政治最基本的原則.
但今天的課是神學課.
除了我們內裡的心理和成長之外.
除了我們說的社會政治之外.
我們更加在想.
我們作為教會群體.
如何理解我們的身份.
特別是留堂.
今天我也會嘗試去定義一下.
留堂的身份.
如果要說的時候.
這也是我們很少去想的問題.
所以今天我們會說一些.
我不太熟悉的題目.

$^{81}$因為這不算是很神學的題目.
起碼上一部劇是這樣.
甚麼是身份呢?.
最早提出身份認同的人.
大家可能都知道.
就是艾瑞爾·艾森.
一個很有名的心理學家.
他用了「身份」這個字來提出理論.
來說我們人的成長.
我想這是大家熟悉的東西.
甚麼是intimacy.
或是alignment.
因為人在不同的成長期裡.
都有不同的階段去達成.
這也是一個人累積的經驗.
或是他的信念,價值和回憶.
去訴出他如何看待自己.
所以我們人的心理成長.
或是人的成長.
都是很重要的.
我們如何看待自己.
我們在這個世界裡存在的時候.
在不同的人生階段裡.
我們如何理解自己.
他更說.
如果我們實踐不到.
就會出現成長的問題.
這些都是很普遍的心理學的看法.
另一個就是社會學上.
我們稱為collective identity.
原來身份認同不單單是個人層面.
社會學家更覺得.
這是一種群體裡的身份認同.
這個認同不單單是一個人.
而是一群人.
他們如何互動.
如何各自理解自己.
形成一個群體身份.
所以這個身份.
這個群體的身份.

$^{121}$會令我們收到這幾年的政治.
這是一位很有名的學者.
Francis Fukuyama.
是一位很有名的社會政治學家.
這本書很新的.
2018年才出版.
他提出了一個身份政治的問題.
我覺得他能夠用這個理論.
去回顧我們過去幾年做了什麼.
他說原來我們一個群體的身份.
其實我們一群人.
是有一個很強烈的需求.
先用香港人的身份.
我們香港人對於我們一個群體.
有一個很獨特,很獨特的身份.
就是我們香港人.
當這個社會不能滿足.
某個群體的身份認同的時候.
簡單來說,當香港人的身份價值.
還沒有在社會中被重視的時候.
因為多元,特別是在香港社會是多元的.
當因為多元的緣故.
而不被重視的時候.
他們就有一個很強烈的學求.
來去渴望,有尊嚴.
來去被重視.
重視自己的身份.
如何去理解我們一群香港人.
所以這就成為了一個政治訴求.
這就是所謂的政治最基本的原則.
就是因為一個多元的緣故.
這群人很想更加被尊重.
他們本身的存在.
所以這件事同時也被近代的全球化.
以及社交媒體的興起.
更加促進了這件事.
我們在社交媒體中.
更加容易去視察到我們的身份.
這件事未必是靠Facebook,IG,社交媒體.
來看到我們香港人是怎樣的.

$^{161}$我們的身份價值很容易在當中交流.
以及被認同.
所以當社會制度.
不能夠滿足群體的尊嚴的時候.
就成為了一個很強烈的民主運動.
當然香港人的成果是失敗了.
但福克斯媽媽所說的.
如果成功的話.
其實又會回到另一個極端.
就是會成為一種民族主義.
或者是成為一個很強大的政治力量.
這可以是一個很正面或負面的事情.
所以我們要嘗試去平衡.
以及我們作為一個人和一個群體.
如何平衡個人的權利和自由.
以及集體身份這件事.
我們不單單是個人.
我們更加是一個集體的身份.
同時我們也會尊重自己個人的認同.
所以這就是我們.
很簡單,我也不是這方面的人.
但也簡單地說一下.
身份在心理學,社會學和政治裡面.
有幾個我們都會觸碰到的字眼.
當然我們要說的.
我自己特別有很大的感受.
就是身份危機.
這個字我聽過很多次.
但我們發現原來這個字眼.
其實也不是….
原來是發生的.
我自己也說過,我有中年危機.
我35歲的時候.
是一個完全合法地….
本來35歲至65歲是一個中年危機的開始.
我剛好35歲的時候就有中年危機.
是甚麼呢?原來就是….
一個人到了那個階段的時候.
你會發覺你開始不知道自己在做甚麼.
那時候會覺得很迷惘,很迷失.

$^{201}$或者是有點虛空.
原來這樣的感受.
就是身份認同的危機問題.
原來「identity」這字眼不是那麼遙遠.
我們怎樣看自己.
我們怎樣去反映自己的現狀.
還有這個世界裡面.
你的生活或社會的世界.
和你自己的身份之間的差距.
正成為了身份認同危機的問題.
所以就算你未過35歲也好.
你仍然有不同的階段裡面.
可能會面對的問題.
特別是在這幾年裡.
當我們面對著….
可能你是移民的,留在香港的.
這幾年裡發生的事情.
戴口罩或轉教會.
這個也是我們….
今天我們會說的問題.
就是這個世界的轉變.
令我們對於自己的身份被動搖.
或者是不被滿足.
或者需要尋求更新的時間.
我自己看了一些資料.
我覺得有一個不錯的分類.
是這樣的.
其中一個身份認同危機叫「displacement」.
稱為與世界和過去中斷.
你會發現這個世界突然轉變了.
不再是你以前認識的世界.
這個就是「displacement」最簡單的情形.
你發現你所熟悉的過去.
已經不能有一個「continuity」.
你發現過去的香港,過去的社會.
不再是這樣的時候.
你就出現了這樣的「displacement」.
你覺得自己活在一個….
不是在繼續活在自己的世界裡.
所以很多人有不同的做法.

$^{241}$你會發現自己沒有根.
你會想尋根,卻尋不到.
你無法追溯一些你能夠認識的日子和世界.
所以很多人開始想懷舊.
或者回想以前.
或者去看某些共同回憶.
有很多不同的方法.
老人家說舊時.
這也是一種身份認同危機.
他發現現在的社會發展得太快.
不能夠適應.
或者我們不認識香港.
我們就開始想如何算數.
所以任何情況.
當我們發現我們所認識的過去.
和現在有一個差別的時候.
這就是出現了我們其中一個.
身份認同危機的問題.
我們不知道如何回應.
這是第一個我們所說的身份認同現象.
不知道你們有沒有這樣的情況.
無論在你的教會裡.
或者在香港裡.
或者你在外國裡.
你都可能有這樣的情況出現.
發覺無法支撐.
你發覺你的過去好像突然被剝奪了.
第二個稱為「建構」.
就是本身的字解作「退縮」.
或者是困囚.
我們發覺無論是身體被困住.
或者是身心靈被困住.
你發覺無法實踐自己.
這個環境不能夠活出真正的你.
或者你以前覺得可以.
突然覺得真正的不是你.
所以當你不能夠活出.
你想活出的自己的時候.
你就會發覺有這樣的問題.
其實真的有很多都會發生.

$^{281}$我35歲的時候就是這樣.
我覺得自己不想愁遠教書.
教一輩子,就這樣重複下去.
有些事情我是想做的.
但我做不到,就覺得很迷茫.
我自己的夢境也是這樣.
我21歲的夢境也是這樣的情況.
發覺我浪費了大半年時間.
原來我內心有些事情想做.
但我做不到.
或者我未能實踐我想做的事情.
所以無論是你內心的想法.
或者是性別.
或者是你對自己的渴望.
你發覺你很想去做.
但你被很多環境….
我聽過很多時候.
一些媽媽結婚多年.
她發現自己被困在家裡.
很多這種情況都是這樣.
所以當我們的生活.
越來越成為一種束縛的時候.
你就很想活出你真正的自己.
你發覺你每天起來的時候.
發現越來越不像自己.
做一份工作,可能大家都是這樣.
轉工的原因.
或者是你自己甚至離開香港的原因.
都是這樣.
就是因為發覺現在的生活.
不再能夠成為自己.
「我不屬於這裡」.
「我要去別的地方」.
「我要成為別人」.
不知道你們有沒有試過.
想成為另一個人.
認真地想成為另一個人.
可能不再是這樣的生活方法.
很想去一個完全不同的方法生活.
這是我很多次在行屍裡.

$^{321}$經常想著很想去過另一種生活.
所以我們總是發覺這種轉變.
因為這個環境不能讓你活出自己的時候.
就會出現這樣的身分認同問題.
雖然字好像很複雜.
但很簡單,就是我們迷失或迷惘.
第三個.
剛才那兩個基本上有甚麼方法解決.
問題在哪裡?.
無論是當你被人發現世界的擾亂.
當你迷失的時候.
或者當你發覺你被囚禁在生活裡的時候.
我們都很自然地會尋求自由.
很想打破這個僵局.
尋找你過去熟悉的世界或生活.
或者去實踐你真正的自我.
這種對於自由的渴望.
這也是我們性中很少提及的重要課題.
上帝給我們追求自由的重要性.
就是我們作為一個人.
我們很想去實踐自我.
因為我們的選擇.
上帝給我們生命.
我們來實踐我們的生命.
是一個很重要的元素.
所以當我們無論是面對迷失或擾亂.
我們都很想憑著一些方法去尋找自由.
但當我們面對尋找突破的時候.
我們會出現第三種問題.
就是「壓抑自由」.
當我們一下去尋求自由的時候.
不知道你們有沒有試過這樣.
當你一下去突然離開舊的階段.
但那一刻你突然間太自由.
你還未能定義自己的下一個步驟的時候.
那一刻也是另一種危機.
你會迷失自己.
你不知道如何定義自己是誰.
太多的自由,太多的可能性.
令我們不知道如何去成為自己.

$^{361}$所以這是我們尋求身分認同時.
其中一個反動力.
當我們尋求自由的時候.
我們反而會面對另一種迷失的情況.
我們不知道如何去做.
可能你會很自由到某個地步.
你會失去很多東西.
人會不知道如何去站立.
所以這很簡單.
我們可以說是用這三種情況去嘗試.
嘗試去想想我們不同的狀況.
無論是我們所說的「迷失」.
「壓抑自由」或「壓抑自由」.
嘗試去想想我們有沒有遇到這種情況.
第一個,我們說說移民大姐妹.
當我們離開香港的時候.
其實可能也是面對這種情況.
我們用這套理論去理解自己的情況.
可能我們發覺香港不再是香港.
香港的社會價值已經不再是以前那樣.
所以我們就要離開.
離開本身就是一種尋求過去的東西.
所以發覺很多移民國家都是這樣.
唐人街是在那個年代的.
80年代香港在哪裡?就是加拿大.
張學友或周國榮的歌曲仍然在播放.
教會的詩歌也是一樣.
仍然在播放那個年代的詩歌.
所以我們在樓下也是一樣.
如果你突然在樓下.
過去的話也會在榮耀廈播放.
所以都是這種情況.
我們會嘗試尋找我們的過去.
所以這是一個很自然的現狀.
但當我們移民的時候.
我們同時面對著一種.
overwhelming freedom的情況.
我們有太多的可能性.
我不知道自己應該怎樣去定義自己.
我去哪個城市都可以.

$^{401}$我做什麼工作都可以.
但我接下來應該怎樣做人呢?.
我怎樣去定義自己呢?.
反而是有一種迷失在這裡.
所以對於流散的鄧小平來說.
他們很明顯地尋求自由.
嘗試不要被一些東西去困擾.
去尋求這種新的方向.
但我們仍然面對著很多認同問題.
香港人是一個很好的collective identity.
我們有一種固定的共同體.
但我們仍然面對著這種情況.
我們未必能夠在一剎那裡.
找到我們的身分認同.
另外,我們一群留在香港的姐妹.
我們發現香港越來越不熟悉的時候.
我們過去的香港不再熟悉.
我們該怎麼辦呢?.
這很值得大家回到小時候去討論.
我們現在是在甚麼階段?.
我們在發現以前的香港不是這個時候.
我們沒有一個比較可行的方向.
去理解自己.
我們只是一個過客嗎?.
我們純粹對這個世界,這個社會.
沒有任何積極的期盼嗎?.
我們只想賺錢,生活好一點就算了.
我們怎樣去理解自己在香港?.
我們同時也要活出我們的真我.
我們如何在香港的社會裡.
去做回我們自己.
我們是否有些東西被…….
無論在法律上或恐懼中.
我們不能真正地活出來.
當然,我再說一次.
那個collective identity是重要的.
那個群體是重要的.
我們仍然很需要那個群體.
來承載我們這些有價值觀念的東西.
所以當我們發現世界不同了.

$^{441}$我們更加需要一個collective identity.
保留在香港.
我們會否…….
又是另一種overwhelming freedom.
我們會否覺得現在反正沒有甚麼可以留戀.
我們反而過分地去找一些我們想做的事.
但我們找不到呢?.
所以問題是…….
我再回應一句.
我們有沒有身份的台階呢?.
以前的時間.
你真正的自己.
有沒有被這幾年裡打敗了呢?.
第三個我想特別說的就是教會.
我們作為留堂人.
我們其實也面對這三種問題.
我特意加了一個隱許度.
因為我很少說「留堂人」這個詞.
很少稱呼大家為「留堂人」.
我一向都很強制自己不要說這些詞.
因為這些詞我都很…….
該怎麼說呢?我都未有認同危機.
我不是很認同我們這個群體是怎樣的.
我不是叫「Full Church人」.
但我覺得好像很奇怪.
我不想叫大家「Full Church人」.
甚麼「留堂人」之類.
因為我不想特意要給大家一個名字.
所以今天我們討論的問題.
我們這些留堂人.
其實我們也有這種問題出現.
我們有沒有好好地去處理呢?.
我們跟過去的信仰生活.
是否已經中斷了?.
它可以是好,也可以是壞.
我們怎樣跟以前的教會生活.
是否有一種分離?.
可能是因為我們離開教會.
這裡有很多離開教會的人.
你離開教會的時候.

$^{481}$其實你是否發現.
你以前信主的情況是沒有了.
有一個很大的分離.
你嘗試尋求一個找回以前教會.
或者信主的美好來留堂這裡.
還是你嘗試去實踐自己的信仰.
但以前教會實踐不到.
所以我嘗試去打破心靈的囚禁.
去尋回真正的自己.
同時我們又會怎樣?.
有超越自由.
反正我們星期天不用去崇拜.
我們有太多的自由.
令我們突然離開前教會.
好像很多事情都已經完全不同了.
所以我覺得這是一個頗有信心的問題.
對我們這群人來說.
因為我們似乎都是面對著這樣的心靈問題.
因為留堂一直都是一個很奇怪的體系.
留堂無法定義它是多少人的教會.
它不是普通的教會.
因為YouTube裡的人很多.
很多人都看留堂.
但我們又好像是一個很模糊的群體.
我們似乎不是定義自己的群體.
我們一起想想.
我們怎樣定義自己的群體.
在一個真正的社群的時候.
我們怎樣去看待自己.
我們會否稱自己為留堂人.
還是其他名字.
我們怎樣去理解自己的群體.
可能你的組別會比較好.
比較容易看到.
但對整個全群教會來說.
我們好像很不容易定義自己.
這些都是一些問題.
我覺得值得我們去思考.
無論你是離開香港.
或者你在香港生活.

$^{521}$或者你回到FourShot.
我們都值得用這個角度.
去輕輕想想.
我們面對著的每一個問題.
接著就說一些我比較熟悉的事情.
一些比較聖經上的事情.
我們在聖經裡怎樣去說身份認同.
或者從我們的信仰裡.
我們怎樣去好好地回應.
剛才所說的很多問題.
其中一個我覺得很簡單.
我們基督徒有甚麼身份認同.
我們有甚麼身份.
其實不難說的.
聖經裡有很多身份.
你是甚麼.
上帝所愛的兒女.
你是門徒.
你是神的僕人.
你是弟兄姊妹.
你是天國的子民.
你是神的形象.
有很多這些.
但我想說的是.
這些說出來其實很簡單.
但你知道這些代表甚麼.
你知道你是神的兒女.
是否等於沒有甚麼問題.
好像似乎不是這樣.
雖然我們有很多.
在聖經裡給予我們的身份.
我們有很多不同的角色和崗位.
你是門徒,你是跟隨耶穌的人.
有很多不同的身份.
身份是不缺的,是我們的名字.
但我想我們怎樣好好地.
將這些身份.
聖經裡的身份更有說服力地.
去幫助我們面對這些生命危機的事情.
跟大家看一首詩.

$^{561}$一首潘福華的詩.
這首詩叫作「我是誰」.
「我是誰」.
這首詩是潘福華在監獄裡寫的.
當時他在柏林的車叫監獄裡.
坐牢坐了好幾年.
他開始有一個很明顯的身份認同危機.
他開始對我這個身份.
對自己整個的生命迷失.
或者是一種真正的危機.
所以他寫了這首詩.
我們可以一起唸這首詩.
他說「我是誰」.
「他們常對我說」.
「當我步出牢房」.
「從容,愉快,堅定的」.
「彷似一位從城堡出來的國族一般」.
這也是潘福華在外表上.
很多人都看到他仍然是一個很有氣質的人.
仍然是一個很信靠神的人.
在監獄裡生活著.
「我是誰」.
「他們常對我說」.
「當我和肉觀說話」.
「輕鬆,友善,清醒的」.
「彷如我正在發洩好另一半」.
仍然可以對人有很多不同.
有禮貌和非常淡定的舉止.
「我是誰」.
「他們常對我說」.
「當我在不幸的日子」.
「平靜,微笑,高傲」.
「彷如一個慣常勝利的人」.
仍然是一樣.
面對著這麼艱難的監獄日子.
仍然可以保持鎮定,微笑.
面對著一切.
「我真是他們所說的那樣嗎」.
「還是只有我自己心裡明白」.
「不安,饑渴,病態」.

$^{601}$「彷如一隻籠中鳥」.
「為呼吸而掙扎」.
「彷如被人捏住喉嚨一半」.
這就是他們很深刻地體會.
自己內裡的情況.
這也是一個建構.
他們是被人困住.
不能活出自己真正想做的人生.
無論在心靈或身體裡都是這樣.
「我想念色彩,花朵,鳥語」.
「我奢望安慰的話語和同門的愛戀」.
「我痛恨獨裁和深空狹窄」.
「我而不定,期待大事降臨」.
「思念千里之隔的朋友」.
「卻只能無力地顛簸」.
「我的禱告,思想和舉動都令人煩厭」.
「都是虛空」.
「我虛弱地想隨時告別一切」.
這很明顯是對於過去的扭曲.
他似乎對過往的日子很掛念.
他仍然覺得現在的生活.
他的狀態完全割裂了.
所以他找不到自己的狀態.
我想說,坐牢坐得好並不容易.
坐牢是覺得自己已經坐牢.
其實是一個不容易的想法.
「闖明」中有很多不同的.
對於外面的世界,對於自己的實踐.
對於過去有很多不同的思考.
特別是有很多時間.
「我是誰?這個還是那個?」.
究竟是外表的那個,還是內在的那個呢?.
「我是誰?今天是這樣的人」.
「明天是這個人嗎?」.
「兩個又是我嗎?」.
「人前的偽君子,在自己面前的可憐軟弱者」.
「還是在我裡內是一個被打敗的軍隊」.
「在已經贏得勝的情況下,無序地後退」.
他就問「我是誰?」.
他似乎不是很確定.

$^{641}$我想說這是一個掙扎.
我不是說他的外表不是真的.
他仍然是一個很有禮貌的人.
仍然是一個微笑,有信念的人.
這不是假的.
但他似乎不知道應該怎樣定位自己.
面對著這種情況.
他不知道怎樣活出自己.
他怎樣跟現狀有甚麼關聯.
他怎樣在現在的情況下做人.
他完全不知道.
以這個證據來看.
他放大到香港的時候也是一樣.
如果以證據來說.
香港正面對著監牢.
我不是說監牢,而是監牢.
在這種情況下.
我都遇到一個問題.
我們是否真正能夠活出真我呢?.
我們是否知道怎樣跟現時生活的世界相處呢?.
這是我們今天可以問的問題.
他最後開始有一點曙光.
他說我是否在贏了的情況下無序地後退呢?.
當然這是很明顯的.
說的是基督耶穌得勝.
即是耶穌已經贏了.
所以就算他坐在監牢裡.
他也是在贏了的情況下.
像是打輸了的軍隊一樣.
所以這是一個很奇怪的風事.
但也很明白的情況.
雖然他知道耶穌已經贏了.
但他就像輸了的人一樣.
他正在面對自己.
所以有一個問題.
他究竟是誰?.
他究竟用了甚麼.
用了哪把尺子來開始想他自己呢?.
明白嗎?.
我想我們的問題是.

$^{681}$我們不是不知道自己做了甚麼.
問題是我們不知道誰懂得.
我們應該開始理解自己的起點.
我們是用甚麼尺度.
或是用甚麼量度單位來看自己.
才是對的.
所以兩者都是真的.
不是外表是假的,內在才是真.
而是我們應該怎樣去看自己呢?.
最後很重要的,一句很簡單的結束.
他說我是誰.
這個孤獨的問題嘲笑著我.
無論我是誰,你認識我.
我屬於你,上帝.
這正是整首詩最後的總結.
對於「我是誰」的問題.
這個問題不斷嘲笑著自己.
但最後的討告.
也是一個很重要的形象.
無論我誰也好.
他還未弄清楚.
他的身分認同危機的問題.
他不是知道自己的身分.
才會離開監獄或有出路.
不妨在還未掌握自己是誰的狀態下.
來結束這首詩.
所以我們面對著身分認同危機時.
我們的答案不一定是我們知道自己是誰.
我們是一種生命問題.
可能到了某個階段,我們會有點迷失.
但答案不一定是我們知道自己是誰.
我知道自己是誰,就做人.
我們永遠都有一種距離.
我和自己永遠都有一個差距.
你嘗試拉近這個差距是對的.
但你不一定要完全消滅這個差距.
我仍然有一個問題是留白的.
我未必知道這一刻我完全是誰.
或我的身分是怎樣.
但我知道甚麼?.

$^{721}$無論是誰,只要你認識我.
我屬你,上帝.
如果看德文,這首詩叫「Wer bin ich」.
「Wer」即是「Who」.
「Bin ich」即是「Who am I」.
但最後的回應是「Dein bin ich」.
這字跟「Wer bin ich」的字很相似.
不是「Wer」.
但「Dein bin ich」在語法上有點特別.
它不說我是誰,而是說我是你.
「你」在語法上是指屬於你的意思.
德文有個字叫「受格」.
「你」不是「You」.
而是屬於你的「You」.
所以就說「Dein bin ich」.
所以這字回應是「我屬於你」.
但最後的回應是「Wer bin ich」.
即是我誰,我就是甚麼.
我就是我所不知的誰.
但我屬於你.
所以這很大的回應是.
我們有時未必知道自己是誰.
但我知道我屬於誰.
所以在這個比較闊的亮光之下.
我們可以不斷尋找和實踐自己是誰.
但我們未必有一個很具體的.
這一刻我們是誰.
這就是我對這首詩的看法.
彭佛華用了「Dein bin ich」.
作為整個生物形態的總結.
即是我屬於上帝.
所以我們發覺有趣.
經文中有時會有這樣的經文.
它說「我現在活著的不再是我」.
似乎基督教中很少叫你不要做自己.
你不是自己,你是基督,耶穌等等.
但意思不是叫你不要做自己.
而是基督在你內活著.
所以經文中不是叫你不做自己.
也不是不實踐自己.

$^{761}$而是你需要在基督內好好實踐自己.
所以更清楚地辨認.
你們是歸入基督的,是披萊基督的.
不分猶太人,希臘人,自主的,遺奴的.
男的,女的.
這些都是認同.
你們有不同的身分.
無論是性別或是社會身分.
你可以有很多不同的身分.
但你們是屬乎基督的.
這就能夠確定.
你可以有很多不同的身分.
但總的來說,你是屬於基督,耶穌的.
這就是我們最重要的答案.
雖然不是身分.
這不是身分.
但是比身分更闊的範疇.
只要我們屬於基督.
我們就可以在這種狀態下.
不斷尋找更相似的身分和價值.
我們有些總結.
第一句話是「Dine bin ich」.
今天學得慢了,不好意思.
「Dine bin ich」.
意思是「我屬於你」.
「你」是「dein」.
「Dine bin ich」.
我們一起記得.
「Dine bin ich」,我是屬於你的.
這是一種禱告,我是屬於你的.
這是我們很重要的神學答案.
聖經沒有給我們唯一的身分認同.
有很多不同的生活狀態和狀況.
但只要我們屬於上帝.
我們可以有很多不同的空間.
尋找我們的身分認同.
所以我們人生中不需要有完美的和諧.
沒有一刻完全完美.
找到自己的目標.
這是我的理想和目標.

$^{801}$我們未必只是這樣的狀態.
我們多數是在尋找的狀態.
尋找,仍然有點迷惘.
我們會嘗試減輕迷惘.
但我們不可能完美地沒有迷惘.
這是我們很重要的想法.
沒有完全的正確性在過去.
所以我們不斷尋找,不斷拉近.
但我們在「Dine bin ich」的情況下.
尋找我們的身分.
第二,我們的教會群體.
這是我們很重要的幫助.
當我們在整個群體中.
我們就更容易地有身分認同.
流淌是一個甚麼樣的群體呢?.
我們經常說流淌是一個不被定型群體.
流動不被定型群體.
我們很不容易去定義自己的群體.
不過我們正正是一群人.
我們不斷地尋求我們對上帝的實踐.
我們就是一群這樣的人.
在這種情況下,我們反而會尋找自己.
我們就是一群這樣的人.
我們分明不是正常教會,大家都知道.
但我們是基督徒,我們屬於基督耶穌.
這是我們很重要的肯定.
所以大家做基督徒.
特別是今天做香港人也好.
或者面對著很多離開教會的朋友也好.
教會群體很重要.
我們不能過於自由.
不要太自由.
任何事都像基督徒一樣,沒所謂.
反正我們都可以不回教會.
或者做基督徒就算了.
不是這樣的.
我們仍然是在一個群體中做基督徒.
其中一個我覺得很重要的就是聖餐.
我們也說了一些聖餐的事.
聖餐是我們整個教會.

$^{841}$來定義我們自己的時刻.
我們是BYOB,即是Bel C.
我們帶了一杯回來.
這種狀態,行徑正正是定義我們整個教會.
我們有很多不同的人,不同類型的杯.
但我們回到教會,我們一起成為一體.
所以我們的聖餐是很好的體現我們的身分的機會.
聖餐更加是一個基督徒的參與.
所以我們覺得聖餐從來都不是受洗.
因為任何基督徒,只要他信耶穌.
他就是我們一起的一個群體.
所以我們就可以一起藉著聖餐.
去成為我們的教會群體.
所以Full Church可以說是有很多不同的形態.
可以有小組,有不同的MFC.
但聖餐這一刻令我們流唐.
成為一個很顯而易見的群體.
大家是多元,不同的故事.
但一起去領聖餐.
所以這是我們整個教會很重要的時刻.
那次我們不需要領洗禮.
因為我們知道聖餐是給所有基督徒的.
所以是一個傳統的習慣.
但對我來說,任何缺志基督徒都能領聖餐.
所以今天的總結就是.
上帝,我屬你.
無論我是任何人,還是未成為任何人.
我都是屬於耶穌基督的.
無論我是誰,我如何,我未能如何.
我就是你的兒女.
所以既然我們知道我們是屬於上帝的時候.
我們的身分可以浮動.
可以沉覓中,可以有些迷惘.
但我們知道無論如何.
我們仍然是屬於上帝自己的.
這就是我今天簡單的分享.
我們一起討穀笑,好嗎?.
接著我們有Q and A時間.
我們都知道我們是屬於你的.
我們崇賢迷失.

$^{881}$我們對於自己的身分.
對於我們身處的社會.
身處的狀態.
可能仍然有很多不確定的事.
主佑大愛求主你,來幫助我們.
面對我們不同的生活狀況.
無論是工作,教會群體或生活.
我們求主你幫我們在當中找到你.
當我們知道是屬於你的時候.
我們就可以做很多事情.
來沉覓你給我們的生活.
在當中讓我們看到我們應有的身分.
求主你幫助我們.
逢尊命求,阿門.
你檢查好身分上機了嗎?.
差不多完成了.
檢查身分很重要.
不適合身分上不了機.
大家都有登機證,對吧?.
可以吧?.
有沒有帶護照?.
只是檢查身分而已.
今天這個課題很重要.
身分,大家本身有身分危機嗎?.
開始有點嘴巴彎彎的.
應該有的,所以說時也會點頭.
是嗎?.
不是他所說的中年危機吧?.
大家應該不是這個想法.
青春期危機吧?.
容貌危機,大家覺得是.
這個課題大家思考過程中.
今天除了學到一句德文外.
還有沒有身分問題.
大家覺得有點卡嗎?.
你說就行了,後面大家會一起參與.
沒有身分問題,沒有身分危機.
有的,可能太深入的問題.
太深入? 太深入.
大家在香港,網上等於不會參與.

$^{921}$我說一個真實的個案.
以前的組員去了英國.
他已經過了半年.
當初他也不太急於找工作.
但去了英國後.
他覺得沒有工作.
他不太清楚自己在做甚麼.
聽到這裡,你不知道有甚麼想法.
但我在過程中.
他出發時,我跟他說.
「你放下香港的專業去到」.
「你也說不用做你的專業」.
「你說倒垃圾也可以」.
「或者找其他人做倉務也可以」.
「我說你不行的」.
「你覺得可以放下」.
「以前在香港的專業」.
「去到你也不太緊張薪金」.
「你隨便找工作也可以」.
「我說你不行的」.
然後他說「先試試吧」.
「我說你很難的」.
為何我看得通呢?.
其實是在工作上找不到自己.
你聽得懂嗎?.
不知道你在香港會否這樣.
你的工作能否找到自己.
或者你做基督徒的期間.
能否找到自己.
這也是我們要考慮的.
(記者: 請問教會是否有教會的弟兄姐妹?).
其實現在更深入了.
(記者: 請問教會是否有教會的弟兄姐妹?).
前面那位弟兄.
第二行.
(記者: 剛才沒有承接你那句).
不要緊.
(記者: 因為很多教會).
(強調自己在教會裡的身份認同).
(當有教會的弟兄姐妹離開時).

$^{961}$(他們就會說自己離開教會).
(那種感覺是).
(他們把普世教會和地方教會).
(很混亂地).
(給人一種感覺是).
(隔離教會的).
(都是表兄弟姐妹這件事).
(該如何處理呢?).
其實所謂的堂會.
或者是社群.
是一個真實的群體.
雖然我們說有普世無形教會.
但我們需要有一個.
具體,有形的群體.
所以身份應該是那個身份.
不是說我屬於普世教會.
這是對的.
但在現實來說.
我們需要有一個具體的群體.
多於一個「反正全世界都是教會」.
所以我會這樣看.
我們不難.
你想主宰的就是大教會.
但在我們的生活中.
我們需要有一個具體的群體.
一個實體教會.
或者一個堂會.
一個群體的堂會.
所以我從來不建議人說.
「這個普世教會就是我的教會」.
這是對的.
但我們需要有一個實體的群體.
來成為你的身份象徵.
和身份認同.
所以我覺得兩者都重要.
我可以問你一個問題嗎?.
我不是很明白表兄弟姐妹的意思.
問得好,但其實我也不明白.
當他說到你離開某一個堂會時.
你就是離開了教會群體基督的本身.

$^{1001}$有另一間教會的兄弟姐妹是不同的.
他們是他們,我們是我們.
我明白每個堂會有自己獨特的地方.
但當過分強調一件事時.
很容易產生疏離感.
「別人的教會是別人的教會」.
當然他會再補充一句.
「我們不要單顧自己的事」.
「Bilibili」,會有另一堆事.
但這很奇怪.
我覺得會有很強烈的排他性.
因為這樣推遠一點.
整件事都會很異端.
不要說推遠一點,放大一點.
其他宗派可能都用不下.
宗派中神學和聖餐的方式.
禮儀方式不同,所以有不同的宗派出現.
但如果用親屬關係來說.
即是跟一個爸爸.
但有不同兒子的面向的宗派表達.
如果這樣說,其他宗派的做法未必可以.
不心夠他的表兄弟的想法如何.
但我想這與身分無關.
我覺得這樣會太過專制.
令那件事好像只有我對.
其他人都不太對,這樣就不太可以.
回到身分方面,大家覺得有甚麼困難.
對於你們來說.
今天的課題是關於身分認同這件事.
(無聲).
還是覺得沒有甚麼大難處?.
那裡有點口,是否有問題?.
(有,在那裡) 在後面.
其實我覺得我不是有很大問題.
可能你不用特別回答.
但我純粹好奇的是….
因為在說身分的流動性.
特別在流動教會裡.
我覺得大家好像不太想定義.
例如我們是甚麼群體.

$^{1041}$我們也很模糊,是一條界線.
所以永遠在尋找的時候.
我覺得比較難去看.
我們在個人裡面.
究竟我們是否有一套….
或者有很多…因為太專注.
所以在議題上.
我也不知道如何去看自己與身分的關係.
特別是…我不提及具體的議題.
也不是特別問你們.
你們有甚麼看法?也不是.
我純粹有時候在想.
如果別人問我.
我作為「Folk Show」大型節目.
我是不能夠為他回答的.
因為始終如果在議題上.
我們有一個很光譜的看法.
未必是立場.
但有一種聖經的呈現.
我也不能替他回答.
不能替教會回答的時候.
我有時候會在想.
這種流動性,其實是否….
我也不太能拿捏.
這種永遠在流變的對議題看法.
或者對教會,宏觀的事情的看法.
應該是可以如何應對的.
所以我會用自己對社會的理解去回答.
當然不是代表「Flow Church」.
但有時候也很想看看.
有沒有方式可以從聖經角度.
了解更多「Flow」是怎樣的.
所以這是我們….
說得很準確地說出我對「Flow Church」的看法.
剛才說流淌人.
我自己也很害怕這樣說.
或者覺得不太有關.
其實也不是不行的.
但我不想像以前我們中學時期.
穿著一件寫生衣.

$^{1081}$然後有很多不同的卡,身分證.
我也想過,先做一張「Flow Church」證.
大家有一個分析分析.
掛在這裡,也挺有型的.
我們可以有很多這些東西.
但我都說這些不重要.
但重要的是甚麼?.
重要的是我們的信念.
所以我們「18課」就是這樣解釋.
「18課」就是我們希望….
雖然大家可以有很多不同款式.
和不同樣子的「Flow Church」人.
但我們也不需要有「Flow Church」家.
或「Flow Church」衣服.
或甚麼來定義我們自己.
但信念是一樣的.
所以我們在首兩季,下一季,三季.
對某些議題,我們的想法是一樣的.
我們是重視這些想法的.
這就是我們「Flow Church」的信念.
所以縱然我們未必有很硬件.
或物件上的身分認同共同點.
但想法,相信的態度是一樣的.
但我們覺得是不同的.
「Flow Church」的人的想法.
或我們的神學,信仰.
是和「出面教會」有一定的不同.
不是說是大公正的東西.
而是我們的質地是不同的.
所以「18課」想我們可以展示出來.
我們可以有不同的東西.
我們說靈修.
我們說我們怎樣看基督徒.
怎樣看跟隨耶穌.
怎樣看流散.
這就是我們「Flow Church」的頂尖姊妹.
獨特之處.
這就是我們的身分認同.
(陳健波)我用教會運作的做法回應.
「Flow Church」到了第四年.

$^{1121}$由石硤美年代開始參與崇拜的頂尖姊妹.
加入小組.
或是很穩定和我們相聚的頂尖姊妹.
我也被不同的人問過.
「Flow Church」沒有甚麼活動.
我說我們有很多活動.
他們說你們有活動.
但不是像部門般參與.
因為他們覺得.
除了疫情令我們很多聚會封鎖.
其實過去在教會.
如何有一個所謂的「Church Member」身分.
就是你屬於甚麼部.
你屬於栽培部,傳道部.
新朋友部,會友部.
那些部就是他的身分.
他覺得那裡認同了這間教會的參與.
但「Flow Church」不是用這個運作方式.
他一直不太容易擺明自己的位置.
所以他覺得.
正如你剛才提到的.
如何介紹或如何在「Flow Church」中.
表達一些東西.
不容易具體地帶出來.
所以這也是我們的運作狀況.
但我們不是要有部讓你掛單.
讓你覺得有安處.
沒有世界混亂的情況.
反而我們覺得有些題目.
是大家一起要合作,要帶出來的.
這正正是說甚麼是基督徒.
早期教會被稱為.
「他們就是基督徒」.
為何安提拉那班人會被稱為基督徒.
基督徒不是要承認自己是基督徒.
基督徒是要宣揚基督.
所以他被稱為基督徒.
這句說話很重要.
我們要承認自己是基督徒.
但你沒有宣揚基督.

$^{1161}$這是重要的.
因為那班人從早期一直宣揚基督.
所以他被稱為基督徒.
但我們沒有宣揚基督.
但你承認自己是基督徒.
我覺得你名不符實.
這就是你沒有自己的基督徒身分.
所以我們一個不被定型的群體.
去演化或延續出去.
其實我們在哪裡都能夠宣揚基督.
我們才能夠運作出自己的身分.
這就是流動性.
你明白我的意思.
所以這是很重要的.
不是我們要在教會裡運作出什麼.
去認識自己是基督徒.
去服侍教會的身分.
但教會從來不應該是向內的.
教會是向外的.
所以我們被猜出去宣揚基督.
我們就是把基督徒的身分向外展現.
這就是我們應該要有.
從信仰而有的身分.
這就是重點.
(記者:請問你會在教會做什麼?).
我會做一些教會的事.
例如教會的教會.
我會做一些教會的事.
例如教會的教會.
我會做一些教會的事.
例如教會的教會.
我會做一些教會的事.
例如教會的教會.
我會做一些教會的事.
例如教會的教會.
我會做一些教會的事.
例如教會的教會.
我會做一些教會的事.
例如教會的教會.
我會做一些教會的事.

$^{1201}$例如教會的教會.
其實很開心的.
轉眼間發現看到很多不同的東西.
有一次在一些工作活動當中.
會與不同的群體交際.
我遇到一些我以前工作的那類人.
看到他們在做自己的事.
那一刻我出現了一種.
「我在做什麼呢?」.
我很嚮往以前那種.
如果我還在那個地方.
我就是在做那些事的人.
那一刻是很傷心的.
真的有一個危機.
直到跟同事分享.
自己再想.
真的能夠知道.
原來那個移動.
那個工作崗位的轉變.
仍然我看得出是上帝帶我離開.
所以那個時刻.
不是因為我不是以前崗位的職位.
而是因為我仍然是上帝的兒子.
他帶我去這個崗位.
甚至是將來我是.
應該說我頗確定.
這個未必是我永久的崗位.
但無論我再轉換,再沉覓也好.
知道我們的根源是與上帝連繫的時刻.
那就是我真正的身分.
因為浮世間的職位.
定形不了我們的身分.
那個東西永遠都是短暫.
但最不會變的就是你是上帝的人.
那是最重要的,所以很能夠反映.
想分享一下自己經歷過這件事.
謝謝.
因為我經常提醒自己.
形式是一件事.
形式有很多不同的形式.

$^{1241}$但你的本質是甚麼是最重要的.
我以前經常說笑.
穿醫生袍的人不一定是醫生.
精神病人穿醫生袍也會扮醫生.
除了醫生袍,你仍然是醫生.
因為你的本身的色彩就是醫生的能力.
所以不是著重你的外觀.
或是你的所謂的「Title」.
其實你的能力是表現出你的功能.
以及你可以有身分.
即是表達出來的工作,這是很重要的.
所以仍然是那句話.
你「Proclaim Christ」.
你才是一個基督徒.
所以無論你探病也好,探監也好.
探訪也好,你探甚麼也好.
你帶著你的基督信仰去祝福他人.
就是有身分而有工作.
(陳零九)我也加一句.
在這個時代,越來越多這樣的場合.
正如你剛才所說.
你明明是醫生,但不是以醫生的身分做醫生.
對吧?現在穿警察服的.
未必是警察的那種.
特別是傳道人也是.
傳道人很多時候更不是傳道人.
方式就是傳道人.
所以很多時候,世界叫你.
未必用一個典型的消防員,警察,醫生的身分做人.
有時候很浮動,是否像slash一樣.
基本上是slash的.
沒有一個很強烈的身分.
但最重要的是你知道你在神面前.
你是一個怎樣的人,做甚麼事.
那些外加的狀態是一些很外表的東西.
也不是最重要的東西.
最重要的是你是一個怎樣的人,做甚麼事.
多於外面的人怎樣看你.
我們要習慣了這樣做人.
(陳零九)還有身分.

$^{1281}$裡面有一個分類是自我身分.
你自己是否意識到自己身分而有的能力.
以及表達方式是很重要的.
剛才說到像不像.
我自嘲說一件事.
就是我初出來做傳道人的時候.
被年輕人說.
「潘先生,你也不像傳道人」.
我說「我不像嗎?我哪一部分不像?」.
很緊張.
然後他想了想.
其實傳道人是不遮蓋頭的.
然後我心想.
遮蓋了頭就不是傳道人了.
我說甚麼不….
(陳零九)那些是用髮蠟的.
對,我說唐主任也是遮蓋了頭.
不過他用髮蠟,我用髮泥.
其實也是遮蓋了頭的.
然後他說「但你真的不像」.
我自己的口頭禪是.
「像」即不是.
我是一個傳道人,我不用像.
這件事讓他明白到.
其實有些人會帶著舊有或社會的觀念.
去看待身分.
但你如何展現自己的自我身分.
其實你一定要清楚.
不要因為社會的規則.
而主導了你的看法.
這是我們….
特別是我們在上一年的神學百科裡.
也想說的八個核心.
我們現在在這個環境當中.
做基督徒時.
我們如何展現八個特質也是重要的.
所以有一個這樣的模式.
穿著這件衣服,這樣返教會化.
就是基督徒.
但我們不是這樣.

$^{1321}$我們是說你們可以有不同的基督徒樣貌.
但只要你們是屬於上帝就可以.
所以羅唐就不是有這樣的版本.
你跟隨這個版本,就是基督徒.
剛才阿寬Sir說的類似.
前面有一個,剛才….
我想問,作為基督徒.
剛才說我們對自己的身分認定.
例如,如果用醫生的身分.
他有否通過實驗,是否真的醫生.
我們在心裡有否認清.
我們順序去上帝的時候的過程.
但我們做也是.
就是「To have to be」和「To do」.
我們「Be」了.
我覺得自己像剛才那位師一樣.
屬於神.
但我們也有「To do」.
以前教會有很多說法.
你們做基督徒要這樣做….
這些是你們要做的做法.
有些人會說.
可能你平日要拿著「Four Steps to Christ」.
到處跟別人說.
「你信神嗎?不信神就去地獄」.
這樣說的.
那些是其中一個「To do」.
你去宣教,其中一樣就是你的「To do」.
有些人會說.
「不是這樣的,你平日愛人」.
「誠實地做事」.
「這些都可以讓人看見」.
「你如何分開自己」.
「你是一個基督徒」.
「讓人看見你在神裡的特質」.
我經常想.
如果別人要看我的特質.
而他相信有神.
他們應該覺得沒有神的存在.
我覺得這也是很困難的.

$^{1361}$例如,問回身分.
如果對我個人來說.
我是否只是讀經.
我相信神是愛我的.
那我就是基督徒.
如果在Flo Church.
不太著重我們去侍奉.
那我們的「做」在哪裡?.
我不知道這是否很中國人的事.
或是很香港人的事.
我經常都要做.
才能證明我是那樣的人.
梁國雄:當然要是.
我經常說行動是重要的.
特別是新教.
新教是用行動來定義自己.
我經常說.
牧師不是因為牧師的身分.
才做牧師的事.
你是做牧師的事.
才被人確定你是牧師的身分.
基督徒也一樣.
基督徒.
如果想回到第一課的第一課.
基督徒的意思是甚麼?.
基督徒的意思是.
我們去傳揚基督的人.
多於是一個身分.
所以基督徒.
每一個聖經中的名詞.
本身都是一個動詞.
基督徒是傳揚基督的人.
門徒是跟隨耶穌的人.
頂姐妹或神二女.
就是一個天父,爸爸的人.
所以每一個身分.
都來自於一個行動.
只是行動不是那些教會的行動.
不是叫你去回收甚麼.
報甚麼,侍奉甚麼.

$^{1401}$而是你做基督徒的事.
有很多的.
我們會說敬拜.
中文碼頭會說敬拜.
我們會叫你去回一個實體的群體.
在社會上彰顯上帝的公義.
每個碼頭都會說的.
這類行動.
不是教會的活動.
一定是那些才定義為基督徒.
不是用教會的東西來定義自己.
而是做基督徒.
在社會中做基督徒.
這就是我們做的行動.
這也是我對於流唐.
為何沒有那麼多侍奉.
因為你將你的時間和工作.
可以放在外面的社會中.
不是叫你去學放假去玩.
而是去見證基督耶穌.
這就是我們為何少侍奉的原因.
不是叫你去玩.
而是叫你將時間去服侍外面的人.
(梁繼昌)用John剛才說的.
Erikson的身分理論.
每一個身分轉變的時候.
都有一個責任.
他能否做到的事情.
以至身分中的確立.
基督徒也有相似的情況.
有些時候是你有身分.
而沒有相關的能力去幫助.
舉個例子.
你有老師的身分.
但沒有盡老師的身分.
而去想一些你可以做到的事情.
例如我本身是老師.
我用我的方式來指.
讓我要教的東西.
是通達,清楚,能夠幫助別人.

$^{1441}$這就是我要為我的教學對象做好這部分.
這就是我要想的事情.
正如我們之前在Info Group.
或在不同訊息中也說過.
特別是我們《Voice for Church》的泥膠片中.
提到第三部分,社會參與.
我們基本上是散去在香港不同地方.
居住和工作的群體.
但你碰到的人,我碰不到.
但你熟悉的人,我不熟悉.
但你就想方式讓他們接觸到基督信仰.
這就是你的職業任務.
就是你要做的工作.
不一定是定讀經,祈禱.
或其他教會崗位的事情.
這一定要透過你去思考,去做.
這才是正確的.
正確的意思是能夠對應呼應需要.
這其實是保羅所說的.
我在哪裡就做什麼人.
我面對什麼群體就用什麼方式對待它.
無論是希利利人,其他人,法外人.
我想這就是我們需要以身份對應的工作表現.
(梁繼昌)網上的.
不過我剛才回應John所說的.
Flow Church,流塘人.
我也認同是很好的,要包圍它.
因為之前John按摩的時候也說過.
其實Flow Church沒有細菌相.
很多人聚集的.
因為Flow Church這個群體是很廣泛的.
有Zone A的就是實體現場崇拜加小組.
Zone B的就是武俠小組,但有現場崇拜.
Zone C的就是網絡上.
很多人偶爾也會參與Flow Church的網上活動.
所以要定義流塘人是一件不容易.
或者是不需要定義的事情.
不過對於我們來說.
就是如何可以一起參與.
在教會群體當中建立一個相聚的形態.

$^{1481}$我覺得這個身份對我們來說很重要.
(陳克勤)我們下課就會多說這一點.
我們下課的主題是「家」.
所以就說教會作為一個家的一種看法.
所以我們下課就叫做Homeless.
就是回家這個字.
所以我們都會探討.
無論是流散還是….
他們對教會有甚麼重要性.
我們作為流塘人.
如何看待所謂的教會「家」.
如何理解.
我覺得是一個挺好玩的主題.
教會和家的關係.
(陳克勤)明天我們會有戶外崇拜.
我覺得流塘人.
我們想延伸這個沒有界限.
去接觸更多外界的人.
我覺得這個延伸是我們可以做到的.
還有希望做得更加遠.
讓教會不只是四面牆.
或者是一個普通的聚會時段.
我們真的在不同社區.
或者是機會可以接觸更多.
可以認識教會群體的人.
好,最後有沒有其他問題?.
好,可以登機了,準備.
我們下次是這個月底,對嗎?.
對,這個月底是第四課.
OK,遲些見.
再見.
謝謝大家.
\newpage



\section{約翰一書 3:18-24-20230513}
\label{sec:u6GL1Cm7cwU}
\textbf{【網上崇拜】同心呼吸|約翰一書3\_18-24|20230513 [u6GL1Cm7cwU]}
\newline
\newline
連結: \href{https://youtube.com/watch?v=u6GL1Cm7cwU}{\texttt{ https://youtube.com/watch?v=u6GL1Cm7cwU}} ~~~~ 語音日期: 2023-05-13 
\newline
\newline
\hyperref[sec:gRf39gjSNbM]{\small{< < < PREV SERMON < < <}}
~
\hyperref[sec:index_chronic]{\small{[返順時目]}}
~
\hyperref[sec:index_scriptual]{\small{[返順卷目]}}
~
\hyperref[sec:Aqi5hKVncec]{\small{> > > NEXT SERMON > > >}}
\newline
\newline
約翰一書 3:18-24-20230513
\newline
\begin{longtable}{cl}
\hline
\hline
章節 & 經文 (和合本修訂版)\\
\hline
3:18 & \begin{tabularx}{0.7\textwidth}{X} 孩子們哪,我們相愛,不要只在言語或舌頭上,總要以行為和真誠表現出來。 \end{tabularx} \\ \\ \relax
3:19 & \begin{tabularx}{0.7\textwidth}{X} 從這一點,我們會知道,我們是出於真理的,並且我們在神面前可以安心, \end{tabularx} \\ \\ \relax
3:20 & \begin{tabularx}{0.7\textwidth}{X} 即使我們的心責備自己,神比我們的心大,他知道一切。 \end{tabularx} \\ \\ \relax
3:21 & \begin{tabularx}{0.7\textwidth}{X} 親愛的,我們的心若不責備我們,在神面前就可以坦然無懼了。 \end{tabularx} \\ \\ \relax
3:22 & \begin{tabularx}{0.7\textwidth}{X} 我們一切所求的,就從他得著,因為我們遵守他的命令,行他所喜悅的事。 \end{tabularx} \\ \\ \relax
3:23 & \begin{tabularx}{0.7\textwidth}{X} 神的命令就是:我們要信他兒子耶穌基督的名,並且照他所賜給我們的命令彼此相愛。 \end{tabularx} \\ \\ \relax
3:24 & \begin{tabularx}{0.7\textwidth}{X} 遵守神命令的,住在神裡面,而神也住在他裡面。從這一點,我們知道神住在我們裡面,這是由於他所賜給我們的聖靈。 \end{tabularx} \\ \\
[1ex]
\hline
\hline
\end{longtable}
$^{1}$各位姐妹平安,歡迎回來喜帕靈堂崇拜.
氣氛很熱烈,我這個星期的腦袋仍然在看上星期的影片.
所以我變得很興奮.
在我今天預備的訊息當中,選了楊逸書的經文.
跟大家在Armbeat這個月題當中一起思想.
在我頭頁當中,第一個頁面是這個圖畫.
裡面有兩個人,頭顱有兩個不同的意象.
如果你看過楊逸書的時候.
你會看到蓮露的約翰帶著一個心情.
一個眷惠的心情和教會的年輕人.
特別是一群在當中傳承和聚會的弟兄姊妹的說話.
因為在楊逸書描述的背景當中.
有一個比較困難的情況就是.
教會運作了一段時間.
有很多不同的意見.
有很多不同的想法.
當中在教會有很多爭辯.
約翰在當中跟他們分享.
特別是想釐清其實什麼才是最重要的呢?.
我想過去對於你來說.
你可能回過不只一間教會.
或者在過程當中你都不只參與過一個群體.
都是認信基督的群體.
但總有大大小小當中不同意見的分歧.
或者表達方式上的差異.
以致你對於群體當中失去信心.
這件事不是現在才有發生.
是早於公元一世紀教會開展的時候.
知道有人當中都有不同意見的表達.
但怎樣接受和表達或釐清呢?.
很多時候有些人是用他的理性.
或者合理的理據去認為那件事應該是這樣的.
其中司徒約翰要處理的就是.
怎樣才算是耶穌基督是怎麼回事呢?.
其中一個是當時約翰很認真地說清楚.
不是你這樣想的方法.
因為當時有一些人會覺得.
耶穌是什麼呢?.
耶穌切切實實是一個人.
他是一個很道地的人.

$^{41}$他是一個很善良.
很忠心.
很有行善.
很有愛心的力量的人.
但當他洗禮的時候.
聖靈就住在他那裡.
他繼續行很多神蹟騎士.
但當他要上十字架的時候.
聖靈就離開了他.
然後就死在十字架上.
就完結了他的工作.
但約翰要告訴當時的人.
不是這樣的理據.
也不是這樣的理解.
耶穌基督切切就是.
同貞女瑪利亞所生成為人.
要表明一件事就是.
上帝用一個人不能想到的方法.
竟然猜測他的獨生兒子來到世界上.
成就這個救恩.
但人不能夠消失.
上帝怎麼可能這樣做呢?.
上帝怎麼用這樣的方法呢?.
上帝怎麼可以死在十字架上呢?.
接受一樣東西就是.
一個受苦的僕人.
一個受死上帝的兒子.
也不能夠接受一個被釘的上帝.
一個被定罪的上帝.
但約翰告訴我們.
上帝愛我們的緣故.
就是願意用自己的兒子的命.
來換我們大罪之身的命.
這就是上帝用一個人看為愚蠢.
但卻能夠反映上帝大能的地方.
約翰要讓人明白.
不一定是合我們的理性.
但上帝就是一個超然的愛.
讓人明白.
所以約翰要讓人明白.

$^{81}$我們在爭論的東西.
或是我們不能平衡.
或是不能調整的是什麼呢?.
是合理的理.
還是上帝會用另一個方式.
去接受一個大愛呢?.
所以當時如果你有了解教會歷史的時候.
他們會覺得耶穌只是一個幻影.
不是很真實的.
但約翰說不是.
上帝的兒子真真實實地在我們中間.
所以在《約翰一書》第一章開始的時候.
這是什麼呢?.
這是我們親眼見過.
親手摸過.
和我們親身見證過的上帝的兒子.
約翰用他的一生.
用他的心意去說.
但約翰也不是單單告訴他們.
理性上不能過.
不代表不真實.
但同樣告訴他們.
其實上帝最重要告訴他們.
一個重要的訊息.
不僅僅是上帝的愛.
而我們跟隨他的人.
我們都能夠展現上帝的愛.
就是在說的訊息.
我們的愛如何能夠延續下去.
讓更多人因為我們的愛.
彼此相愛的心.
眾人因此認出.
我們是跟隨上帝的門徒.
約翰很認真地說.
所以說的道理不是靠我們的理性.
是靠我們有愛心而有的行為.
這是約翰在一二三書裡說得很清楚.
用很多術式讓我們明白.
我們說自己有愛心.
但我們會不會用能力和行徑去表明出來.

$^{121}$如果你不能夠愛漢得見的弟兄.
你很難愛漢不見的神.
但你能夠具體愛側邊的弟兄.
你就能夠更加感受到上帝.
藉著我們的愛.
讓更加多人認識上帝.
所以我們的愛心不是在言語.
還要有行為.
到了開場白快要完結的時候.
你會發覺.
其實大條道理不難說的.
但真的行出那個道理.
才是最大的困難.
當我們感受到上帝的愛.
我們回到教會.
但教會或我們成長過程中.
都教導我們.
我們要信同行一致的時候.
才是最困難的.
你會聽過很多次.
能夠愛那不可愛的思維之外.
你會覺得那個人不行.
這麼差.
又或者他對我這麼差的時候.
我很難愛他.
於是你要物件處而發.
不要看別的地方.
可能他還有好處.
發覺自己可能變成一個考古學家.
在一些不能發掘的地方.
要發掘一些要愛他的東西.
我也理解有些地方是難的.
但重點就是.
在過程當中.
你更加明白到.
真的用愛去實行出來.
或者要體驗到一個無條件的愛.
其實是很困難的事.
但約翰告訴我們.
在過程當中.

$^{161}$你會經歷上帝.
當你愛一個不容易愛的人當中.
你更加會經歷上帝.
如何去塗造我們.
或者加靈給我們.
以致我們看到.
在我們越想辦法.
如何可以參與其中的時候.
上帝的那種真實.
所以去到經文裡.
第三章第十八節開始說.
小子們啊!我們相愛不要只在言語和舌頭上.
要在行為和誠實上.
從此就知道我們是屬真理的.
並且我們的心在上帝面前可以安穩.
這個就是我剛才說的開場白的時候.
經文上要文字上要表達.
約翰用的一個信息就是.
我們相愛不要只是言語和行事.
在行為和誠實上.
誠實這個字其實跟下面第十九節說的真理是一樣的.
意思就是我們的行為和我們所相信的真理.
是要在行動上表明出來.
你做這個行徑是建基於聖經給我們的提醒和教導.
我們做這個行為的原因.
我們做到在人身上的時候.
因為我們建基於聖經的原則的時候.
別人就會知道我們做這件事不是從我們而來.
而是從聖經的教導而來.
別人就會對你所受的教導有興趣.
剛剛過了上星期五的神學百科第三講的時候.
在最後解答問題的時候.
弟兄姊妹問的時候.
我也用了一個身份的例子跟大家說.
就是說今天你在基督徒.
你不是說自己是基督徒.
你是claim(宣揚)自己是基督徒.
其實我們不是claim(宣揚)自己是基督徒.
基督徒的出現是因為那些安提安教會的人宣揚基督.
以致那些人經常宣揚耶穌是基督.

$^{201}$祂為世人待宿.
能夠有一個挽回.
可以離開罪的身軀的終局.
可以有新生命.
可以有永生.
是宣揚基督的工作.
以致他們那班人被稱為基督徒.
所以我們不是claim(宣揚)自己是基督徒.
是我們proclaim(宣揚)基督.
以致我們成為一個基督徒.
你聽得明白嗎?.
重點就是我們claim(宣揚)自己是基督徒.
是你…就是反過來不要說基督徒.
你說你專業.
你說你自己的專業.
但專業從來都不是你自己說自己專業.
是別人說你專業.
一樣的.
所以你自己說自己專業.
我說我做大板燒.
大板燒.
我說我做大板燒專業的.
你信不信?.
你轉頭就對了.
因為你沒見過我做.
我自己都沒做過.
是不是?.
我說一件事.
說我自己是KOL.
你都不會信的.
從來我都沒自己做過KOL.
重點從來都不是你claim(宣揚)你自己.
你claim(宣揚)你自己是基督徒有什麼厲害?.
是別人因為見你的好行為.
歸榮要給你所信的上帝.
那個是proclaim(宣揚)基督而有.
你是to be honored(榮譽)成為一個基督徒.
你明白我的意思嗎?.
這個是很重要的.
所以我們被教導成為基督徒.

$^{241}$我常常都說被教導是需要的.
但重點就是我們每天有沒有live up(活出)我們的信仰.
活出我們的信仰.
以致別人因為我們的信仰.
看到你是因為信耶穌.
你進行聖經的教導.
你是因這個改變而受.
別人發現你的信仰真的很特別.
你平時在做什麼?.
當別人有問題有興趣的時候.
你就是有機會可以說.
所以在行為和真理上.
從此就知道我們是屬真理的.
並且我們心在上帝面前可以安穩.
安穩的意思就是心安理得.
我知道我為什麼事做.
我知道我為什麼要做.
而我做的過程當中是有困難.
但耶穌基督是知道我的困苦.
所以我們的心如果不責備我們.
上帝會比我們的心大.
約翰老稱一件事就是.
其實你的心是一個很重要的指標.
今天這些日子我們不斷經歷的事.
其實你本著良心做.
你拿著自己的心做.
你想很多決定都是很難的.
但你會發覺越是難的時候.
就是要你做決定.
在決定過程當中我們在想什麼呢?.
所以在這件事我們要看下去的時候.
我們會在想其實這個心是在說什麼呢?.
所以我們的心若不責備我們.
上帝的心比我們大.
在我們看的書信中.
保羅也用了不同的方式.
在說卡地亞的心的教導.
我相信我們大部分的弟兄姊妹.
在做決定的時候都會懂得.
第一個功能就是是非之心.

$^{281}$那樣東西是對與錯,善惡.
特別是這幾年大家都看得很清楚.
第二樣東西是關於心.
那樣東西是否上帝喜悅.
用聖經的說話就是.
看上帝為義的那樣東西.
或者看為惡的.
那樣東西都是本著.
你能不能夠去順應.
就好像徐牧師那時候說的那一套.
掃羅其實是可以分的.
但他掩著自己選擇不做上帝喜悅的事情.
其實我們也是的.
我們做了這麼多年人.
很多時候過程當中.
你知道那樣東西是不是上帝喜歡的.
不過你知道那樣東西是上帝喜歡與否.
有時候你會選擇.
做完這一次吧.
就是這樣.
你會覺得我相信上帝會原諒我的.
我的肉體很軟弱.
我的心靈是願意.
但我的肉體很軟弱.
於是做完這一次.
所以保羅也在說.
其實每一次做錯決定或做錯事的時候.
其實在做那一刻之前.
你的心其實提醒了你.
不過你是掩著,遮著這樣.
第三件事就是.
心其實對我們是一個管治.
剛才說提醒你.
提醒你.
好像很辛苦.
其實你從另一個角度看.
其實這個心為什麼會縮起來.
你想想你還沒信主.
或者你還沒認真信仰的時候.
其實有些東西你不是覺得沒什麼.

$^{321}$以前都是這樣.
每個人都是這樣.
不用錢就下載.
漫畫書是用來買的嗎.
基本上很多人分享給你.
你就開心分享.
但你現在發覺不是.
好像別人有版權.
這樣就是同流.
你會一刻覺得同流.
你不想合污.
有這個觸動.
你的心對你有管理.
現在比較少.
我認識很多弟兄姊妹都有知識產權.
那些歌是會訂閱的.
會買的.
逐手逐手買的.
或者下載一個每月的訂閱.
就可以無限聽.
但以往不是.
以往是一隻手指位.
剛剛睡醒就抄給你.
打機也是.
我住了11年在深水Po .
我每天都出入深水Po D出口.
深水Po D出口是什麼出口.
醒目了.
真的是黃金出口.
那真的是黃金出口.
很多錢從那個出口進入去.
黃金出口最多是什麼呢.
就是手指抄.
抄game.
是吧.
你會發覺.
我跟那些年輕人說.
你知不知道抄game.
其實就是偷東西.
這麼大件事嗎.

$^{361}$我說你沒有給過錢.
我的手指也是錢.
我說你抄完一隻.
你會洗了裝第二隻.
以前他不是很覺.
但我真的告訴他.
其實你沒有給過錢.
是別人破解之後.
你還好心不停地看那裡有破解版.
於是就下載.
其實心提醒這件事.
就是他開始在管理你.
以往你不覺得很大件事.
以往你不覺得是一些很需要交代.
或者是一些很需要釐清的事.
但心開始管理你的時候.
你願不願意被他管理呢.
如果你不願意被他管理的時候.
其實你仍然是在用你過去未信主的方法.
去行事為人.
但你願意被他管理的時候.
其實你每一天都在經歷上帝在當中提醒你.
就是耶穌在臨去各個他山上的時候.
跟那群門徒說.
聖靈就會令你想起我說的話.
聖靈就是給我們最重要的保衛師提醒我們.
保衛這個字就像戰隊.
其實我仍然喜歡和合本旁邊的細字翻譯.
就是那個訓衛師.
他是教訓和安慰.
Paracletus就是圍著你和你說話.
就是你又做了.
這個要小心一點不要指著別人.
你又做了.
或者旁邊說不要這樣做.
這樣做不好的.
你一會又不開心.
他會說你但又會提醒你.
在過程當中.
聖靈就是上上座的工作.

$^{401}$其實在管理.
不知不覺你就會發覺.
你以往沒有那麼敏感.
但你信主內.
你願意跟從上帝的時候.
你的心其實在管束你.
每一次管束.
你轉個角度說.
其實聖靈提醒你.
你應該開心.
上帝在我這裡.
很重要.
很多弟兄姊妹都感覺不到上帝.
感覺不到聖靈提醒.
感覺不到聖靈在你心裡.
我說不是的.
如果你對罪敏感.
你知道那件事不應該做.
那一下觸動.
上帝就與我們一起.
這個很重要.
所以約翰就說.
親愛的弟兄.
我們的心如果不責備我們.
就可以向上帝坦然無懼.
並且我們一切所求.
就從祂得著.
因為我們遵守祂的命令.
行祂所喜悅的事.
所以不會一下子.
缺完智之後就突然.
超人變身.
不會.
但在過程當中.
慢慢你會感受到.
你多走上帝的一步.
走近上帝的一步.
或者遵行聖經的教訓.
嘗試的時候.
慢慢你的看法會改變.

$^{441}$你的心會狠更多.
你的心又會敏感多些.
聖經告訴我們一個重要的事.
你越求的時候.
你就會感受到上帝的真實.
但困難就來了.
大家其實試過的.
我相信.
通常初信主的時候.
當然是雄心勃勃.
然後就覺得.
我每天都要靈修.
然後我每個星期六.
要上小祖.
星期日就上崇拜.
阿門.
每個星期.
一開始是OK的.
接著說.
我今天靈修不到.
我明天補回.
你真的明天想補回.
但一天讀兩章.
受不了.
我今天先讀一章.
然後又假日.
又累.
然後又晚上睡覺祈禱.
早起才阿門.
你會發覺.
日復日越答越多的時候.
你會發覺.
很不容易.
你的良心就責備了.
所以有時.
良心最大的挑戰.
就是你對良心的感覺.
你怎樣感受罪惡.
那種罪疚感.
你一直都做不到那種罪疚感.

$^{481}$OK.
大家都好像有點觸動了.
罪疚感就是你很緊張.
那件事是否上帝會定我們的罪.
基督教很多時候都說罪.
很多時候就說外面的人.
你是罪人.
你快點悔改.
你死了之後去哪.
這件事就很catchy.
但我常常都說.
外面的人我們當然要緊張.
但我心想你缺了智之後.
我還緊張.
因為你不要覺得.
進了門檻就沒事.
重點就是你缺了智之後.
其實要做跟進.
有些事要幫你成長.
以致不要覺得拿了boarding pass就上機.
所以第一件事就是對罪的感覺.
你的罪疚感.
你怎樣提醒自己.
第二件事就是.
良心提醒的一件事.
有些什麼常常都塞住了我們.
剛才說到.
弟兄你的心如果不責備你.
上帝就可以坦然不向上帝面前.
有什麼塞住你.
令你覺得自己裝不到上帝.
有時都跟弟姐妹傾下這件事.
有時一定不會開名.
但如有雷同.
告訴你很多人都是這樣.
其中一件事就是.
我真的很喜歡那件事.
我真的放不下.
接著就是時間分配.
你其實知道有些聚會你想參加.

$^{521}$但人家叫你去聚會的時候.
你就會選擇去.
又或者你覺得.
你知道每天都要找時間出來.
跟上帝祈禱.
但你一直都沒有下定決心去做這件事.
在過程當中你會發覺.
有些事其實你會知道.
霸佔了你很多精神時間.
但你又不捨得割舍.
其中一件事就是.
我在小組或跟弟姐妹傾下的時候.
我常常都鼓勵弟姐妹多點祈禱.
她說我有啊.
我睡前都會祈禱.
不像早上阿門那種.
我都說都好.
現在香港有三成四成人都是失眠.
你能夠祈禱的時候睡得著.
其實是一件好事.
但我說你自己說過什麼你都不記得.
你自己說過什麼你都不記得.
祈禱不是說完之後.
阿拉丁等上帝做事.
祈禱就是一個自我觀察.
你檢視一下自己祈禱的事.
自己有沒有跟進.
你清理自己的優先.
祈禱是一個自我觀察的過程.
那我就說.
會不會找一些時間認真分別出來跟上帝祈禱呢.
所以我常常都會建議弟姐妹.
你會上班的.
應該會的.
你上班的時候你會吃午飯.
就在吃午飯完結.
就是在下午工之前.
拿十至十五分鐘給自己整理一下.
就是祈禱.
吃飯當然是吃到最後一分鐘.

$^{561}$我就說OK.
你覺得吃飯吃到最後一分鐘是你的權利.
我就說退而求其次.
你覺得你什麼精神狀態最好的時候.
就在你清醒的時候.
拿十分鐘給上帝祈禱.
有些弟姐妹是做到的.
不過做了一陣子就不記得了.
因為很散.
於是我就再用第二招.
就是教鐘.
就是教鬧鐘.
就在手機上.
如果你每天都找一段時間出來教鬧鐘.
跟上帝祈禱.
真的.
有些事情我們很愛很喜歡的東西就不記得了.
不知不覺就走差.
也做少了.
其實很多事情你都知道是重要.
對你好的.
是對你好的.
但你就選擇不做.
因為做了其他事.
說到身體.
我想問做運動對身體好的認同的請舉手.
那你們有沒有做運動.
有了有了.
有了.
有了就問你做了多少.
你會發覺你問任何一個做運動對身體的好.
你認不認同.
很多都舉手.
但為什麼你不做運動呢.
因為你把時間攤開了做其他事.
其實一樣的.
其實心提醒你的時候.
所以你的心不責備你.
其實你自己知道.
有些事情是你取捨了.

$^{601}$第三件事就是.
有時心怪怪的.
好像說不清楚.
不知是什麼回事.
其實有時有些事情就是做事的方法.
剛才說到約翰他面對教會的衝擊.
就是很多人在會內吵架.
或者有爭議.
但其實早於我之前都說到的時候.
說到《時代行傳》都說過.
其實雅各都處理過.
就是安提拉教會和.
還有撒冷教會.
令到問題上.
譬如手折的問題.
吃東西的問題.
一些行事方法.
外邦人不認識的情況下.
就有很多衝突.
有時心覺得那件事怪怪的.
你不得不問一件事.
那是我的習慣不同.
還是別人認識不同呢.
有時在那個過程當中.
你有沒有去坦然一點.
或者很坦誠地和對方去談呢.
這件事有時會覺得.
在教會當中.
不容易去慢慢疏理或拆解一些問題.
我們兩個星期前就做了一個.
《你沒有對著空氣》.
Sorry特別版.
我聽回弟兄姊妹留言的時候.
其實很多位弟兄姊妹.
很勇敢地表達對對方真誠的道歉.
和表達自己內心的傷痛.
是很不容易的.
但回想過程當中.
有件事就是.
你和對方道歉了.

$^{641}$你和對方表達了你的不安和不事的時候.
當然我們都期望對方有反饋.
或者在現實當中去處理.
但有時未必可以.
可能對方已經事過境遷.
不在你旁邊的環境.
又或者在過程當中已經分隔了.
那件事是怎樣呢.
可能就覺得懸空了.
但是.
約翰提醒我們.
就是上帝的心比我們大.
其實我們向上帝坦承認了.
當我們做得不足,不好的事情的時候.
上帝仍然會有一個宣洩給我們.
有一個讓我們調整.
不要讓這件事繼續克制你.
以致你容易跌入冷惡者的圈套.
就是不斷地在當中自裂和自傷.
約翰提醒一件事就是.
如果你常常有些十功都補不到一過的心態的時候.
有些事情是很難讓你離開或走出的.
我再強調就是我們錯的時候.
要主動去承認.
我們錯的時候要主動尋求對方的寬恕.
和尋求對方的理解.
當對方仍然需要時間去消化.
對方仍然需要時間去沉澱.
以致他在過程當中有些事情還要讓我們去認識.
但我們仍然在等待一個時間.
但就說了不要再沉下去.
沉下去的時候就會令到那件事很多時候都萬劫不復.
另外關於心裡面再說最後一點就是.
有時候上帝告訴我們一些重點.
就是不再定的時候.
但是你會不會接受呢?.
舉一個例子.
在福音書裡面.
那個行人被拿的婦人.
被人帶到耶穌面前.

$^{681}$我不再重述那段經文.
但最後的時候.
耶穌說沒有人定你的罪.
我也不定你的罪.
只是你以後不可再犯.
耶穌清楚知道他是做得不對的.
但是耶穌就停在這裡.
先不判刑.
但叫他不要再重犯.
上帝的心比我們大.
比我們有多一個出口.
比我們有再一次的機會.
但是我們就要接受上帝給我們的赦免.
這個很重要.
如果不是的話.
你會發覺你拿不住這個應許.
和上帝給我們赦罪這個應許的時候.
很多時候我們就不相信上帝.
因為過程中.
哪有這麼有好處.
沒那麼容易.
但是上帝給我們的那種愛.
就是給我們有再一次根深蒂固的力量.
有時候我們知道自己錯.
我相信你過去成長環境當中.
你會發覺你經歷過恩典的時候.
你會更加珍惜再有做一次的機會.
再有經歷的機會.
有時候我們很介意.
我們希望做到一些事情去彌補.
但有時候你會發覺人生不是這麼容易.
你做到一些事情去彌補.
其實你做一些事情去彌補的時候.
某程度上你都是相信等價交換.
我希望做一件事.
能夠取消我之前做的事情.
但是有時候就不行.
我們要懇求上帝.
或者相信上帝明白到.
我已經盡我的心力.

$^{721}$做好我自己的本分.
求主赦免.
我提醒自己不會再重蹈覆轍.
免得又會有下一次的問題出現.
經文提醒我們.
如果我們的心不說我們.
其實上帝都希望每一次赦罪的時候.
都提醒我們.
不要再落入自憐自責的心態.
因為有一段經文想和大家去了解.
在啟示錄第十二章的經文.
十至十一節的經文是這樣的.
我聽見在天上有大聲音說.
我上帝的救恩能力.
國道並他基督的權柄.
現在都來到了.
因為那在我們上帝面前.
晝夜控告我們弟兄的已經被摔下去了.
弟兄勝過他是因高揚的血和自己所見證的道.
他們雖至於死也不礙色成名.
這裡說的他就是那惡者.
被稱為古蛇.
就是大龍.
就是撒旦.
他不斷地捉住我們一些.
不相信上帝赦免的那種患得患失的感覺.
在當中核制我們.
你配什麼敬拜.
你自己都不是一個好人.
你配什麼服侍.
你很好嗎.
你站台做什麼.
你配什麼幫人.
你不就是幫人幫到這樣.
你會發覺這些自憐的心態.
就在旁邊突然彈出來.
然後就在誣蔑你.
以致你的自我形象.
以致你的心智.
以致你的能力.

$^{761}$你覺得是給下去.
但我們能不能夠在當中再一次告訴你.
是,我是一個無力的罪人.
我是一個無用的.
我在當中做得不好.
但是上帝的愛已經遮蓋了我.
上帝的能力奉給我.
這個就是保羅再提醒.
那根刺仍然在你生命當中觸發他的.
但是我的能力.
是在我軟弱的時候變為剛強.
在啟示裡面.
仍然是約翰的手筆告訴我們.
其實冷惡者是垂死用方法去觸發.
我們不要再靠近上帝的身邊.
是觸發我們不要再.
或者拉開我們.
不要再覺得自己很厲害.
是,我們覺得自己不厲害.
但是上帝幫助我們.
所以如果下次你想參與聚會.
想服侍,想做見證的時候.
你當中有把聲音告訴你.
突然說你配嗎?.
你這樣做,你行嗎?.
但如果你仍然知道你行在上帝.
你是在幫人認識上帝的時候.
我很希望你再一次確立.
你所做的是上帝的喜悅.
就好像耶穌和彼得說.
撒旦,退我後面去吧.
因為那件事不推脫上帝的心意.
第二段經文用的是約翰一書的經文.
約翰一書的經文這裡是說第一章.
其實我自己很喜歡的.
我自己在禮儀教會成長過程當中.
每一個主日主席都會在認罪祈禱後.
做宣洩的經文.
我們若在光明中行.
如同上帝在光明中就彼此相交.

$^{801}$他兒子耶穌的血也洗淨我們一切的罪.
我們若說自己無罪便是自欺.
真理不在我們心裡了.
我們若認自己的罪.
上帝是信實的,是公義的.
必要赦免我們的罪,洗淨我們一切的不義.
約翰很認真讓每一個明白上帝道的人就是.
我們每一個都是帶罪的身體去到上帝的面前.
而聖經再一次告訴我們.
每一次你帶罪去上帝面前.
上帝會赦免我們的罪,洗清我們的不義.
不要再翻舊帳.
不要再用以前你的不是.
來合制你現在和將來要做的事.
這是上帝給我們的肯定.
所以在很多人,特別是那時候.
很多人都覺得要做一些事情去取代自己的事情.
但約翰說不是,上帝已經做完了.
所以最後那段經文他在說.
就是上帝的命令就是叫我們信他兒子耶穌基督的命.
且照他所賜給我們的命令彼此相愛.
遵守上帝命令的就是在上帝裡面.
上帝也住在他裡面.
我們所知道上帝住在我們裡面.
是因他賜給我們的聖靈.
這段經文好像很重複和剛才之前說的東西.
但事實上是第三章的總結.
我希望我們如果像約翰當年面對很多教會的紛爭.
不同意見的表達.
是合你的理不代表合我的理的時候.
約翰仍然提醒我們.
重點是什麼呢?.
重點不是那種理據.
重點就是耶穌基督的工作.
真正要蘊含裡面帶出的訊息就是.
你信的是誰?.
就是耶穌基督的命.
上帝兒子的命.
上帝兒子就是在我們生命當中.
讓我們感受到什麼是上帝要我們彼此相愛.

$^{841}$你能夠和上帝或我們履行上帝的彼此相愛.
我們就在上帝裡面.
也在基督裡面.
聖靈賜給我們幫助我們.
今天說的內容其實不是很複雜.
第一個內容就是.
上帝仍然喚醒我們的心.
我們會知道什麼是對的,什麼是不對的.
什麼是上帝喜歡的,什麼是上帝不喜歡的.
提醒我們的時候就是.
叫我們下一次或這次不要這樣做.
提醒我們的時候.
也感受到你被提醒.
其實上帝就在我們生命當中.
聖靈就幫助我們.
這是很真實的.
你會感覺到上帝在我們身邊.
那種感覺是很好的.
而聖靈是什麼呢?.
聖靈就是耶穌在他復活的時候.
在那十一個門徒當中顯現的時候.
他就吹一口氣.
你們受聖靈.
我們也是.
我們缺志那一刻.
聖靈就在我們身邊.
從來都沒有離開過.
直到耶穌再回來.
我希望大家一起去同夫同級.
去經歷聖靈.
上個星期我們去了戶外崇拜.
四百多個弟兄姊妹一起.
我初時也不知道有這麼多人.
我就走來走去.
沒有上台的時候走來走去.
看看環境.
怕有些突發事要處理.
在核心的外圍走.
核心的外圍.
原來很涼.

$^{881}$但當要進入核心的內圍的時候.
就覺得很熱.
但很感受到大家的呼吸一致.
當Ice在上場的時候.
就說今天的師班.
今天的敬拜.
不只是台上的我們.
同樣是我們一班台下的弟兄姊妹.
一起去敬拜.
我們同心去做這個呼吸.
同心去做這個敬拜.
這是給我們一致.
但對我們來說.
我們會不會在經歷這件事呢.
我希望同一個訊息.
今天也來到.
我們來崇拜弟兄姊妹.
不只是一個星期.
星期六的時間.
Flo Church很希望.
每一個參與崇拜弟兄姊妹.
是帶著聖靈給我們這個能力.
我們走進社區.
走回工作間.
親友間.
居間.
我們彼此相愛的行為表現.
難一定會難.
但做一點的時候.
就會感受到.
是喔,不同了.
不同了那一刻.
你就會知道上帝的真實.
有時你不試,你不知道.
就好像做運動.
可能跑五分鐘就喘氣.
就受不了.
那你就試下一次.
保持跑五分鐘.
再跑幾次,五分鐘.

$^{921}$你就會去到六分鐘,七分鐘,八分鐘.
一樣.
如果你說跑步很悶.
你就玩玩WinFit.
起碼讓自己有動力.
最大前提.
那我要買WinFit.
那還是不玩了,先問人借.
你連借的動力都沒有.
你什麼都不做的時候.
那也很難幫到你.
起碼要有動力開始做.
我希望弟兄姊妹.
我們一起去同心.
我們的心就是聖靈提醒我們.
我們與基督同心.
上帝與我們一起.
我們在呼吸.
就是聖靈給我們一個同呼同吸.
同感一靈的那種觸動.
大家一起去經歷.
這個才適合上帝給我們生命的別.
在這個相遇題當中.
很希望我們在上半年.
我們就開始幫助.
我們自己的朋友.
我們自己的家人.
在上半年我們就開始.
拾取到一些節奏.
以致我們在.
二三年的下半年我們還有很多東西.
可以大家一起同步.
Full Church不只是台上.
或者是其他工作人員.
Full Church是一個群體.
一起去Workout.
一起在香港.
在我們身處的環境當中.
特別是海外弟兄姊妹也好.
可以參與在其中.

$^{961}$我真是祈禱.
天上帝.
一項年老的時候.
他如重心長地面對一群年輕人.
縱使面對很多.
就是堂會上.
或者是不同.
立論當中的爭議.
但是要釐清的.
仍然不是他們的理性.
是他們的.
信心和行為.
希望在當中能夠做好.
每一個本份.
就是能夠讓愛去感動人.
讓愛能夠改變人.
求主你幫助我們.
我們不僅僅只是.
聽上帝的話.
我們更加是做上帝.
給我們的話而有的那種.
行動力.
Full Church是一個流動的群體.
我們在不同的地區當中.
都可以焦聚不同的人.
去幫助我們.
聖靈提醒我們的心.
我們管束自己的心.
每一次被提醒的時候.
我們就知道上帝在當中.
每一次被管束的時候.
就知道上帝愛我們.
每一次被執拾的時候.
赦免的時候.
我們相信上帝赦免我們.
我們不要再被.
自己.
或者冷惡者.
繼續去克制.
我們不再走下去.

$^{1001}$求主你教導我們.
拿著這個英雄.
繼續過每一天.
多謝主讓我們一起崇拜.
多謝你讓我們一生跟隨你.
祈禱奉耶穌得名及.
阿門.
謝謝大家.
\newpage



\section{猶大書 1:3-20230520}
\label{sec:Aqi5hKVncec}
\textbf{【網上聖餐崇拜】承傳arm beat的隱形遊樂場|猶大書1\_3|20230520 [Aqi5hKVncec]}
\newline
\newline
連結: \href{https://youtube.com/watch?v=Aqi5hKVncec}{\texttt{ https://youtube.com/watch?v=Aqi5hKVncec}} ~~~~ 語音日期: 2023-05-20 
\newline
\newline
\hyperref[sec:u6GL1Cm7cwU]{\small{< < < PREV SERMON < < <}}
~
\hyperref[sec:index_chronic]{\small{[返順時目]}}
~
\hyperref[sec:index_scriptual]{\small{[返順卷目]}}
~
\hyperref[sec:OcD6qni0UQE]{\small{> > > NEXT SERMON > > >}}
\newline
\newline
猶大書 1:3-20230520
\newline
\begin{longtable}{cl}
\hline
\hline
章節 & 經文 (和合本修訂版)\\
\hline
1:3 & \begin{tabularx}{0.7\textwidth}{X} 親愛的,我一直很迫切地想要寫信給你們,論到我們同享的救恩,但我覺得有必要現在就寫信勸你們,要為從前一次交付給聖徒的真道竭力奮鬥。 \end{tabularx} \\ \\ \relax
1:4 & \begin{tabularx}{0.7\textwidth}{X} 因為有些人偷偷地進來,就是早就被判定受懲罰的不虔誠的人,他們把我們神的恩典變為放縱情慾的機會,並且不認獨一的主宰—我們的主耶穌基督。 \end{tabularx} \\ \\ \relax
1:5 & \begin{tabularx}{0.7\textwidth}{X} 這一切的事,你們雖然知道,我卻仍要提醒你們:從前主只一次就救了他的百姓出埃及地,後來卻把那些不信的滅絕了。 \end{tabularx} \\ \\ \relax
1:6 & \begin{tabularx}{0.7\textwidth}{X} 至於那些不守本位、離開自己住處的天使,主用鎖鏈把他們永遠拘留在黑暗裡,等候大日子的審判。 \end{tabularx} \\ \\ \relax
1:7 & \begin{tabularx}{0.7\textwidth}{X} 同樣,所多瑪、蛾摩拉和周圍城鎮的人也跟著他們一樣犯淫亂,隨從逆性的情慾,以致遭受永不熄滅之火的懲罰,作為眾人的鑒戒。 \end{tabularx} \\ \\ \relax
1:8 & \begin{tabularx}{0.7\textwidth}{X} 照樣,這些做夢的人也污穢身體,輕慢掌權者,毀謗眾尊榮者。 \end{tabularx} \\ \\ \relax
1:9 & \begin{tabularx}{0.7\textwidth}{X} 天使長米迦勒為摩西的屍首與魔鬼爭辯的時候,尚且不敢用毀謗的話譴責他,只說:「主責備你吧!」 \end{tabularx} \\ \\ \relax
1:10 & \begin{tabularx}{0.7\textwidth}{X} 但這些人毀謗他們所不知道的。他們與那些沒有理性的牲畜一樣,只做本性所知道的事,敗壞了自己。 \end{tabularx} \\ \\ \relax
1:11 & \begin{tabularx}{0.7\textwidth}{X} 他們有禍了!因為他們走該隱的道路,又為財利往巴蘭的錯謬裡直奔,並在可拉的背叛中滅亡了。 \end{tabularx} \\ \\ \relax
1:12 & \begin{tabularx}{0.7\textwidth}{X} 這樣的人是你們愛筵上的污點;他們無所懼怕地同你們宴樂,彷彿牧人只顧餵飽自己。他們是無雨的浮雲,被風飄蕩;是秋天沒有果子的樹,死而又死,連根被拔出來; \end{tabularx} \\ \\ \relax
1:13 & \begin{tabularx}{0.7\textwidth}{X} 是海裡的狂浪,湧出自己可恥的沫子來;是流蕩的星,有漆黑的幽暗永遠為他們保留著。 \end{tabularx} \\ \\ \relax
1:14 & \begin{tabularx}{0.7\textwidth}{X} 亞當的七世孫以諾曾預言這些人說:「看哪,主帶著他的千萬聖者來臨, \end{tabularx} \\ \\ \relax
1:15 & \begin{tabularx}{0.7\textwidth}{X} 要審判眾人,證實一切不敬虔的人所妄行一切不敬虔的事,又證實不敬虔的罪人所說頂撞他的剛愎的話。」 \end{tabularx} \\ \\ \relax
1:16 & \begin{tabularx}{0.7\textwidth}{X} 這些人喜出怨言,責怪他人,隨從自己的情慾而行,口說誇大的話,為自己的利益諂媚人。 \end{tabularx} \\ \\ \relax
1:17 & \begin{tabularx}{0.7\textwidth}{X} 親愛的,至於你們,要記得我們主耶穌基督的使徒從前所說的話。 \end{tabularx} \\ \\ \relax
1:18 & \begin{tabularx}{0.7\textwidth}{X} 他們曾對你們說過,末世必有好嘲弄的人隨從自己不敬虔的私慾而行。 \end{tabularx} \\ \\ \relax
1:19 & \begin{tabularx}{0.7\textwidth}{X} 這就是那些好結黨分派、屬乎血氣、沒有聖靈的人。 \end{tabularx} \\ \\ \relax
1:20 & \begin{tabularx}{0.7\textwidth}{X} 親愛的,至於你們,要在至聖的真道上造就自己,藉著聖靈禱告, \end{tabularx} \\ \\ \relax
1:21 & \begin{tabularx}{0.7\textwidth}{X} 保守自己常在神的愛中,仰望我們主耶穌基督的憐憫,進入永生。 \end{tabularx} \\ \\ \relax
1:22 & \begin{tabularx}{0.7\textwidth}{X} 有些人心中猶疑,你們要憐憫他們; \end{tabularx} \\ \\ \relax
1:23 & \begin{tabularx}{0.7\textwidth}{X} 有些人你們要從火中搶出來,搭救他們;有些人你們要存懼怕的心憐憫他們,連那被情慾污染的衣服也要厭惡。 \end{tabularx} \\ \\ \relax
1:24 & \begin{tabularx}{0.7\textwidth}{X} 願那能保守你們不失腳,使你們無瑕無疵、歡歡喜喜站在他榮耀之前的、 \end{tabularx} \\ \\ \relax
1:25 & \begin{tabularx}{0.7\textwidth}{X} 我們的救主獨一的神,藉著我們的主耶穌基督,得享榮耀、威嚴、能力、權柄,從萬古以前,到現今,直到永永遠遠。阿們! \end{tabularx} \\ \\
[1ex]
\hline
\hline
\end{longtable}
$^{1}$今天穿衣服打領帶.
不是穿衣服,不過打領帶是因為.
這條領帶代表一個球隊.
如果你看到這個標誌的話.
我怕今晚不打的話.
我以後就不打它了.
所以今晚的場球很重要.
我講完就走了.
所以今天搞得我有點緊張.
第二個緊張的原因是什麼呢.
今天講完之後,我想要八月才再見大家.
六月和七月,六月整個月多一點.
我會去了以色列.
有兩團以色列的隊.
終於回去了,我四年沒有回.
這是我最想回去的地方.
接著一個月去英國.
探望一些很久沒見的朋友.
如果在英國的電影展會.
你在街頭碰到一個傻子走來走去.
打聲招呼,希望我們彼此再認識.
所以這個夏天完結之後.
有兩個多月的時間會不在香港.
所以想念大家.
希望八月的時候回來正常,完好無缺.
今天會講一個訊息.
因為軒工創作了一首歌.
Wyman寫的詞.
所以我們希望講一下這首歌.
這首歌出來的時候.
我聽到也很感動.
因為我第一次聽到.
Wyman上鏡.
所表達的東西是很特別的事情.
我希望我們講一下類似的東西.
我今天會想講尤大書第三節.
尤大書只有一章,其實不長.
總共有25節經文.
但今天講第三節.
我發現自己是讀新約的.

$^{41}$我很少講新約.
我比較多講舊約,有些不務正業.
所以今天想做一些正常的事情.
尤大書第三節是很特別的書卷.
第三節講的東西是.
整本書為什麼要這樣寫.
他怎麼寫呢?.
他說「親愛的,雖然我十分願意寫給你關於我們肉身的拯救.
但我覺得需要寫的是關於你們如何為信仰爭戰.
信仰是什麼呢?.
就是一次傳給昔日的信徒的信仰」.
這是基本的翻譯.
這是我的翻譯.
我想逐個節目來看.
這一節分三部分來看.
3A,3B,3C.
3A講什麼呢?.
如果看上面的部分.
他最初想寫給他關於肉身的拯救.
或者用Greed來講.
其實是Common Salvation.
這個很困難.
什麼叫做「我很想寫關於我們的Common Salvation」呢?.
你叫我翻譯中文是肉身的拯救.
原因是因為Common Salvation不是我們平時想的那些.
Common Salvation就是救恩.
他不是想講救恩.
可以啦.
但他真正的意思是肉身的拯救.
其實背景是什麼呢?.
如果由大主寫作大約62至66年.
其實是一個很政治動盪,經濟動盪環境.
好像今天這樣.
今天加息加得很厲害.
你可以想像銀行未必能撐得住.
可能經濟衰退.
很多人會討論這些問題.
或者政治上很動盪.
今天G7開會.
剛剛出了一些聲明.

$^{81}$你可以想像這個聲明代表背後很多很多的事情.
或者這個世界接下來會發生很多事情.
無論是俄羅斯侵佔烏克蘭.
之後的發展等等.
你可以想像暑假不會是一個容易的暑假.
其實情況好像由大主寫作的時候一樣.
在以色列正面對政治和經濟動盪的時候.
在經濟動盪和政治動盪的時候.
基督徒成為了某部分被逼迫.
被人誣衊的對象.
只請我們交代.
所以在那個時代基督徒被逼迫.
有很多難處的時候.
其中有些神蹟奇事.
神蹟奇事是什麼呢?.
就是有些被迫迫的基督徒.
會被上帝親自拯救.
這些故事是在傳流的.
雖然我們未必很熟悉.
但是如果我們看《少數行傳》.
有這些例子的.
譬如《少數行傳》第十二章.
關於彼得被人捉拿之後.
然後天使怎樣拯救他.
這些我們稱為肉身的拯救.
或者叫做common salvation.
即是上帝藉著他的子民受傷害.
被人捉拿的緣故.
上帝親自用很多神蹟奇事.
拯救他們脫離危險人的手.
所以第三節A.
其實是想寫一些關於.
最近發生的愛當中.
有很多基督徒雖然被迫害.
被人捉拿,被迫迫.
但是上帝親自做了很多奇妙的事情.
拯救了他們.
所以我想告訴你.
其實在一個這麼艱難的世代裡.
你被人捉拿,被迫迫,被人審害.

$^{121}$其實上帝仍然有恩典實際會拯救你.
所以猶大想寫關於這個common salvation.
是在說那時候的二人.
一個很實質的需要.
這個實質需要多到一個地步.
是很難,很不容易.
在很多很苦難的環境裡.
基督徒不知道該怎麼辦的時候.
上帝真的有神蹟地將他們拯救.
這個時代也有.
不過不說太多例子.
所以如果這樣說的話.
第三A是猶大起初寫這本書的時候.
想說那時候的人很關心的議題.
到底在接下來會被迫害.
接下來會被人捉拿,被人拘捕的時候.
到底上帝會不會很神蹟地拯救我們.
大體上猶大想寫關於這個題目的東西.
但奇怪的是去到第三B.
不是這樣寫的.
第三B下面的東西不是看得很好.
其實聖經在下面.
但我覺得要寫的是關於你們如何為信仰爭戰.
他突然不寫肉身上帝如何拯救我們脫離危難.
他想寫我們如何為信仰fight for的東西.
fight for我們信仰的東西.
fight for這個字.
希臘文這個字.
其實在兩頁文獻裡面.
尤其是在死海古卷裡面.
是經常用的字眼.
這個字眼表達什麼.
或者想說什麼.
是在說死海古卷的群體.
他們躲藏在困難.
即將來的暑假我去困難.
我沒試過住五星級酒店.
我這次住兩晚.
Excellent.
當然困難的年代不是有五星級酒店.

$^{161}$其實是沒有的.
他們住在一個很曠野的地方.
一個很艱難的地方.
代表著他們持守信仰的純正.
持守信仰的本質是什麼.
他們覺得就是要在曠野裡持守.
所以fight for這個字眼不是一個簡單的字眼.
fight for這個字眼其實是說.
信仰當快要被人侵吞.
被人在信仰裡加上了很多雜質的時候.
他們就要離開那些雜質.
所以他們離開了希臘王的手.
無論在法利昌還是薩德哥人的手下.
他們走了去曠野的地方.
他們自成了一個群體.
所以叫做困難群體.
就是要fight for信仰的純存.
信仰傳下來的真實是什麼.
本質是什麼.
所以用fight for這個字眼不是一個很隨便的字眼.
去形容他們要為信仰爭戰.
不是的.
這個字眼是說信仰本質開始慢慢被侵蝕.
被危害的時候.
我們如何在被侵蝕危害的信仰本質的同時.
我們能夠fight for信仰真實的原貌是什麼.
所以下一句.
下一句是.
對不起.
很久沒有說greet了.
所以說一下給我聽.
希利文.
這個字第三C是什麼呢.
就是一次全級昔日信徒的信仰.
其實.
指一下會不會好一點.
這個是一個對的字.
看到嗎.
這個是一個對的字.
這個字是解信仰.

$^{201}$是吧.
假裝是可以就說是.
給點反應就行了.
不需要說你真的相信我就行了.
那邊都指一下.
這個字是解article對.
下面這個是解.
多謝這邊.
其實是解favor.
這一句.
我不知道網上看不看到.
自求多福.
這一句是什麼呢.
這一句是.
一次全給信徒的.
就是那一次全給信徒的.
黃色那句希利文.
是在中間.
如果你說是講the favor.
那這個favor的字.
應該放在這裡.
是可以的.
是沒錯.
你明白嗎.
突然之間要favor的字離整句那麼遠.
是不是很奇怪.
是沒錯.
就是這樣.
這樣的寫法是什麼呢.
黃色那句我們叫它是punishable cause.
不要理了.
總之.
你真的已經是天才了.
這樣都會看.
其實在猶大書裡面.
有很多次這樣寫法.
意思是什麼呢.
他明明想用中間那句punishable cause.
形容信仰的字.
他凡是這樣做.

$^{241}$他就會把信仰那兩個字.
不是一隻字.
不過article和favor的字.
拉遠它.
將要形容信仰中間那句說話.
插在中間.
(哦~).
這個就是代表.
好的希臘文寫法.
(哦~).
沒錯.
講完了.
最難的講完了.
大家有沒有反應.
猜不到.
插在中間那句punishable cause.
要形容的那個字.
是代表他很想emphasize的東西.
是想代表.
他想表達的那件事.
是什麼事.
所以他在講fight for信仰的時候.
VB不是在講fight for信仰嗎.
信仰被人蠶食中.
所以我們要fight for信仰.
但他說這個信仰是什麼.
他說你在fight for什麼信仰.
他說你在fight for信仰是一個.
昔日傳給使徒的信仰.
是你要fight for.
即是耶穌基督.
傳給使徒的信仰.
你要fight for.
如果耶穌基督傳給使徒們的信仰.
大約是主後30年左右的話.
去到大書寫作.
你當60至66年.
或者當80年吧.
無所謂的.
總之是後期的.

$^{281}$什麼才是猶大最關懷的.
是那時候的經濟政治.
令到信徒被逼迫被捆綁.
被捉拿被傷害.
他們肉身蒙拯救這件事情.
比較重要.
那時候是重要的.
所以他想寫.
但他想了想.
他說沒錯那些都很重要.
不過我最怕的是什麼.
信仰的本質.
由耶穌傳給使徒.
信仰本質.
我不再強調不再說的時候.
你們只顧著信仰.
有沒有拯救到你肉身的問題.
信仰比你為了信仰而受迫迫.
有沒有被拯救重要.
沒錯信仰裡我們也會強調.
我們受迫迫的時候上帝也會拯救.
我們沒有工作的時候上帝也會拯救.
沒錯這些都重要.
我也想寫這些給你.
但我怕我寫完這些給你之後.
你忘記了信仰的本質是什麼.
信仰有沒有傳承下去.
更重要.
還是我們自己本身肉身更重要.
雖然我不想把兩者分得太開.
說得我們分得開一點.
但對於猶大來說.
什麼才是關鍵.
所以他用了一個這樣的政治課程.
形容信仰是什麼.
他說信仰蠶食我們揮霍的時候.
揮霍什麼.
揮霍的是我們的信仰有沒有傳承下去.
由耶穌基督傳給使徒的信仰.
到三十年後的今天.

$^{321}$然後你們當中還合理一點.
我講到背景會容易明白一點.
如果那時候有一班人叫做契洛.
契洛是憤慮黨.
憤慮黨說他們要起革命.
66年至70年代打仗的時候.
契洛起兵.
希望打贏羅馬人.
打贏羅馬人之後.
他們希望能夠做到覺得應該要做的事.
他們覺得打贏羅馬人.
代表著耶穌來塞瓦會再來.
就會回來.
所以很多信徒軍售很嚮往的是什麼.
如果我們能夠打贏仗的話.
主耶穌就會回來.
沒錯,這些好像很有意思的東西.
聽起來,如果我們真的信耶穌.
或者渴望主耶穌回來的話.
不如我們揮霍羅馬人.
這些都是很應該說.
或者都是.
但只怕如果不是.
沒錯,事實上不是.
主後73年馬薩大最後失守的時候.
整個契洛運動完結.
去到主後的一百三十幾年.
有另一個反抗羅馬.
就是巴奎巴的反抗.
完結之後.
所有耶路撒冷的耶穌.
從此以後要被趕離開耶路撒冷.
耶穌也不能回來.
歷史一幕證明這件事.
所以猶大對的是.
沒錯,你可以很投入在那些東西裡.
但那並不代表信仰的全部.
信仰就算被羅馬人侵佔完.
打敗完.
到2012年再趕走所有耶路撒冷的信徒.

$^{361}$不是,不是,是猶太人.
離開整個以色列地.
要等到多少年.
主後的1948年5月14日.
再一次獨立戰爭之後.
以色列人才回歸到現在的耶路撒冷.
對猶大來說.
他不知道那場仗贏不贏,輸不輸.
但那場仗不能定義信仰是甚麼.
那場仗不是因為他贏了還是輸了.
而定義我們的信仰.
要承諾在那裡,你猜對,真的會回來.
不是,不是,沒有這些.
是我們想到的東西多於信仰的本質.
最近俄羅斯入侵烏克蘭一年多.
最近說大反攻.
基本上這個月我每天都看新聞.
你知道香港沒甚麼新聞看.
那你就看其他新聞.
要看不同國家的新聞才能看到新聞.
不然就沒甚麼新聞看.
你可以想像,現在也沒甚麼好看.
我就追追新聞,評論那些.
這個月我問何時打仗?.
5月打了沒有?.
反攻了沒有?.
我很想見到正義彰顯.
尤其是人生開始越來越覺得公義會彰顯的時候.
有其他地方公義彰顯都開心.
泰國選舉,一個40歲的年輕人能贏了一百萬一席.
你心裡會覺得,嘩!.
你會覺得,嘩!.
我們沒有,人家有,嘩!.
(笑聲).
多謝反應,多謝.
所以過去的一個月,我潘英蓋地地.
很想知道,到底習近平那班人何時會打.
攻到沒有?克里米亞能拿回嗎?.
拿回一張.
終於你心靈的不平衡,不變態的心態.

$^{401}$能夠有一絲絲安慰回頭,是嗎?.
但我想想,就像那段經文所說.
我們好像,起碼我.
為了這幾年有很多東西.
很想知道,很想關心,很想明白.
都很重要,我想說.
但有時候,那些東西原來.
有與無,得與失.
有沒有轉變,有沒有改變.
其實跟我們的信仰.
要不要傳承下去.
是不能掛勾的.
所以最近,我自己一個反省是甚麼?.
對我自己一個最大的反省是.
為何不花點氣力時間.
做一些信仰傳承的事.
先看看說甚麼.
這個說了,不說了.
這三幅圖是甚麼來的呢?.
大家右手邊那幅,這幅女孩的那幅.
是烏克蘭上年被打仗的時候.
小朋友被背在山洞裡.
很慘的時候.
那個做小提琴手的人.
沒有因為打仗而放棄,不拉小提琴.
他繼續在地牢裡拉小提琴.
讓小朋友能夠好好睡覺.
雖然外面起哄哄,有很多飛彈.
我找到另外兩幅,這個男士.
這個是在波斯尼亞和黑山打仗的時候.
大約三十年前的打仗.
其中一幅圖畫最震撼的是甚麼呢?.
他表演的演唱會廳.
被人炸毀了,我希望大家看清楚.
就是這樣倒下,頹垣敗瓦.
Kenneth 幫我放進去,我都不懂得放.
總之放在這裡吧,很慘.
這個音樂家在頹垣敗瓦裡.
他繼續拉小提琴.
沒有因為打仗.

$^{441}$其實那時候他死了二十二人.
他死了二十二人.
他看到有二十二人死了.
不詳細說,他就拉了二十二人.
這些圖畫突然讓我想起.
我從事哪一行?.
我從事甚麼行業?.
我可以花很多時間去關心這些事.
我不是說不重要.
都重要的.
我YouTube的訂閱.
電台,評論節目.
經常都很想聽.
蕭先生說「準一次可以嗎?」.
期望他準,有時候.
又會有燈.
「先生,恭喜你」.
「祝福你身體健康」.
沒甚麼的,愛你.
但是.
我們在信仰承傳上.
尤其是這麼艱難的時候.
是否因為我們看著周遭的環境.
很混亂,很複雜.
突然將信仰傳承下去.
好像放軟了手腳.
這起碼在這幾年裡.
對我來說.
是最大的反省.
幾年了,過了這麼久.
到底應該要做甚麼?.
還可以做甚麼?.
還是我心裡都沒有想過.
世界這麼混亂.
不如躺平吧.
不如享受生活吧.
不如吃東西,開心一下.
不如去宅度假.
水上樂園下面的水層新開.
不如二千多元睡一晚.

$^{481}$開心一下吧.
沒錯,我們可以做這些.
沒問題的.
但當躺平的時候.
是否更加需要.
我們在說想像如何放下去.
我在想一些東西.
這幾個月我不是離開大家.
我想做一些事.
我最近去了一個較佳的市集.
市集有很多.
有一個台灣回來的香港人.
他做朱古力的.
不知道大家認不認識他.
好像有一個店在尖沙咀.
他說他在台灣拿了很多獎.
十多個獎.
他說朱古力是自己做的.
你可以做自己喜歡吃的朱古力.
然後我就問他.
為甚麼你可以做朱古力做二十年呢?.
他很堅定的眼神跟我說.
他說因為我知道.
這是我想做的事.
那一刻我覺得.
比起聽一篇道.
比起我講的那些.
震撼很多.
一個人無論發生甚麼事.
他說他這幾年特意回來的.
他在台灣已經撈到風生水起.
他說他這幾年想回來.
一個弟兄.
那天還很少有小朋友.
我跟他談了很久.
我說你可不可以.
有些想法.
我說你可不可以說一下.
信仰跟你做朱古力的關係.
我猜奧比會喜歡的.

$^{521}$他說是呀,我可以說的.
我說我可以說.
朱古力要配哪種酒喝是開心的.
對不起,奧比不喝酒的.
說笑的.
他想交代的是.
他可以在這個過程裡.
跟我一起分享.
上帝那種獨特.
那種獨特是你才有的.
是你才會想到.
你才會做到.
而配甚麼東西.
是你自己生命裡的負責.
我說這個很信仰.
所以我跟他談了很久.
我說可不可以.
有些事是將來找他幫忙.
所以跟他談了很久.
留了電話在攤檔裡搞了很久.
最搞笑的是甚麼.
我走去另一檔的時候.
有些年輕人.
很多年輕人擺攤檔.
每個年輕人.
尤其是女生.
不要這樣說,男生女生都有的.
每個人來問我拿電話.
我很少被人抄牌.
很多年沒試過.
重點是.
說笑的.
抄牌不是說笑.
你知道我感動的點是甚麼嗎.
其實你可以想像.
很多人在等待.
有人給他一個平台.
他也做了很多事.
他也用信仰去演繹他的福音是甚麼.
所以他相信的.

$^{561}$不同攤檔的人.
他也在做不同的事.
我見他們很有心機.
我就留下電話.
我將來聽聽.
有沒有機會聯絡合作.
可以有些事做一下.
我想最後說一件很小的事.
我期望將來的日子.
我們可以關心現實的很多問題.
我已訂閱的頻道不會刪除.
我都會關心,我都會看和聽.
但我要跟自己說.
起碼入伍後要跟自己說.
信仰可否讓我們看到.
是一群人一起努力演繹他們自己.
在這個世代.
他們應該演繹的信仰是甚麼.
坦白說.
我們說要一個無牆的教會.
說了很多年.
我在教會做的時候已經說.
在改革中年代要說甚麼.
說信徒皆濟私.
坦白說到五百年後.
今天我們哪裡信徒皆濟私.
我們說信仰不是為了聚集.
是為了出去.
說了就有了.
我發現很多神學的理念.
信仰的價值觀.
在我們裡面曾經出現過.
出現過的話.
我們沒有練習過.
你問我自己最大的激動是甚麼.
每個星期看著.
去到不同的教會講道的時候.
除了Fold Church.
剛才去的教會講道.
明天去的教會講道.

$^{601}$你可以想像.
只有兩三個.
二十多三十信在那裡.
最多的了.
他們還肯回來.
那些去了哪裡.
我們可不可以.
不要成為一間.
北美或者歐洲的教會.
開始只有老人家回來.
在英國很多的聖公會.
其他教會.
那些很漂亮的.
變成了星巴克.
變成了博物館.
我很怕香港教會.
淪落成這樣.
上幾個月去了一間.
我十多年前去過的教會.
一間座堂式的教會.
不過不要說太多細節.
我十多年沒有去.
我去了一去講道.
我呆了.
這間座堂式大的教會.
只剩下一堂神拜.
全部都白髮蒼蒼.
丁子梅可不可以.
發一個夢想.
如果崇拜模式.
除了聚集在這裡.
敬拜聽道之外.
如果崇拜真正的意義.
不是在講信仰的傳承.
只是他們聚集開心.
一齊來享受.
這個應該不是.
上帝想見的崇拜.
我希望在未來的日子裡.
有一個夢想.

$^{641}$在一個戶外的地方.
有很多信仰有歷練的.
二十多三十歲的人.
他們在自己的攤位裡.
和別人分享.
他們為何要做這些事.
他們的信仰表達的是甚麼.
有一個平台.
可以讓一班人.
告訴別人.
他們在傳授甚麼.
甚麼是他們的部分.
如果全民造星五那班人.
那些評判說.
要給年輕人一個機會.
有一個平台的話.
為何基督教要做這些事.
我們從來都不會容許.
我們下一代的人.
有一個平台.
讓他們做應該做的事.
如果那些很懶散的.
躺平了的人.
去到那裡的時候.
有些和他們差不多年紀的人.
為他們信仰.
如果他們正在努力.
我希望那裡成為了.
一班人在集聚結.
去思考和發酵.
在這麼敗壞的年代.
在這麼複雜的年代裡.
每一個人可以怎樣.
傳適信仰.
而告訴別人.
我的信仰是這樣傳適的.
如果信仰這個群體.
真正要向外.
outreach的話.
我相信如果有這樣的.

$^{681}$聚集崇拜出現的話.
很多未信的人走進來.
看著一班一班的人.
在告訴他們.
什麼叫做福音信仰.
而福音信仰不再單一.
我差點想說.
得罪機構.
我想說.
不要說了.
我想說是一種信仰模式的表達.
是那些仍然未相信.
或未知信仰是什麼的人.
他看到那班群體裡.
可以傳適很多不同的信仰.
而你可以被歡迎.
一起去參與信仰的話.
我相信這是信仰本質的傳遞.
今時今日.
我問我自己.
看完第三節的時候.
什麼是我在揮霍的信仰.
為我自己寫了一本書出名.
為我在哲學院界耕耘.
你明白嗎.
爭上位.
搏下做院長.
沒有沒有.
一定沒有.
你明白嗎.
這段cut了.
麻煩你post出來.
已經說大了.
你明白我說什麼.
post不是那裡.
你明白我想說什麼.
糟糕了.
我這星期.
我真的得罪人了.
李榮光不是.

$^{721}$表面上假裝要說什麼.
讓大家喜歡聽聽.
實際是為了自己的興趣.
或自己的明星前途.
為自己要有的東奔西跑.
可不可以.
信仰再單純一點.
我們這些已經入伍的人.
已經沒什麼好做了.
可不可以一起.
如果我們當中有入伍的.
特別呼籲那些.
幫幫忙.
我不知道這件事.
怎麼成就.
我沒辦法.
我祈禱求神在這個年代裡.
真的讓一群有心的人.
越來越堅定.
知道他們在信什麼.
他們在用什麼去詮釋.
表達信仰給別人聽.
如果全民造星五的年輕人.
popping那個.
記得那個popping嗎.
有沒有看過.
就是跳得多好.
多少人.
有個去busking的.
在澳洲busking的.
樣子不漂亮的那個.
彈結他很漂亮.
彈得很好.
唱得很開心的那些.
你知道這些人有信仰的.
他們的信仰比我們的信仰.
更加多.
今天為了下一代.
頂尖永間聖餐的時候.
除了我們聚集.

$^{761}$一起分享基督的餅和杯之外.
我們祈禱的是.
可不可以這個餅和杯.
是真的在祝福我們下一代.
而他們可以很有信心.
和自豪地跟別人說.
什麼是他們信仰的詮釋.
他們用什麼去表達.
他們真的在信仰.
不是用嘴巴去表達.
裘天賦在香港.
這麼複雜混亂的時代.
看著很多複雜的事.
令人沮喪,難過.
甚至將你心思意念.
全部都奪去.
我去這個可能是.
尤大叔說的共同寫作.
他不再寫關於這些事.
他寫的是什麼.
我希望他寫的東西是.
這是我想做的事.
海外頂智媒跟你說一句.
如果隱形的遊樂場.
是你現在這一刻.
只可以想像的話.
剛才我分享香港的一幅圖畫.
如果可以成就的話.
我鼓勵海外的頂智媒們.
你都可以在你身處不同地方的裡邊.
由一個隱形的遊樂場.
變成一個實體的遊樂場.
不要將香港教會的信仰模式.
照舊倒模地放在.
你現在所處的環境裡邊.
試試在這個時代裡邊.
做一些在這個時代裡邊.
應該要全息的信仰.
而不是墨守成規.
抱殘守缺地.

$^{801}$做一些以往曾經做過的事.
讓主在這個時代裡邊.
做一件新事.
青年樂隊可以上場.
我踢一首歌你們可以上來.
我很少選歌.
因為上星期大B在這裡大敬拜.
大B是我很小的時候認識的.
我知道大B已經生了子女.
所以我選了一首War Harmony的歌.
《得我住的城山》.
我最喜歡的就是.
有頭的那句話的說法.
有一班人.
是逢主永不倒退.
我祈禱的是.
在香港這個環境裡邊.
有一班人因為信仰的緣故.
不再倒退.
認真建立在這個世代裡邊.
我們應該要建立的東西.
我們一起要一個簡單的禱告.
天父我求你保守帶領著我們.
你讓我們在這個世代裡邊.
見到你給我們的意象與方向.
當上個月要和一班.
九十後千里後的人道歉的時候.
我們不是真的只想說道歉.
我祈禱的是在這個世代裡邊.
有更多新的事會做.
有更多不一樣的人.
去詮釋更多精彩多元的信仰.
讓信仰在這一代群體裡邊.
繼續茁壯成長.
哪怕時代很壞.
時代很差.
人仍然可以在廢墟裡邊拉小提琴.
人可以在地窖裡邊繼續拉小提琴.
今日屬於你的子民們.
仍然可以為著下一代的緣故.

$^{841}$繼續詮釋他們的信仰.
讓信仰可以傳承下去.
主也求你保守帶領.
讓你在這個世代裡邊做新事.
做奇妙的事.
都是主的所領.
聽我們這次的祈禱.
福音所領保貴民心.
弟兄姊妹.
不同的政權.
不同的政治組織.
都有他們的路線圖.
他們有他們的信念.
而我們.
我們都有我們的路線圖.
我們的信念.
一條登上我們主的聖山的路線.
而我話如此說.
有一場若是風處不到對.
心中見過.
我已有這照明.
此生親歷熱愛處不減退.
站在這天門下.
就舉把.
發耳發見.
主你是何等可畏.
獻上這段言語.
撒緊這假裡.
眼穿處把心中灑力都奪去.
白晝所有的阻隔.
讓我真實面對你.
白晝所有的阻隔.
讓你面對你.
白晝所有的阻隔.
讓我真實面對你.
白晝所有的阻隔.
讓你面對你.
自我清潔的手.
等我主的聖山.
自我清潔的手.

$^{881}$等我主的聖山.
自我清潔的手.
等我主的聖山.
自我清潔的手.
等我主的聖山.
(全力光影到我們).
(全球我 榮耀的軍火).
(將要進來 將要進來).
(抬頭我 榮耀的軍火).
(將要進來 將要進來).
(何等美麗 何等可愛).
(如此珍貴 不可取代).
(何等美麗 何等可愛).
(如此珍貴 不可取代).
(不可取代).
自我清潔的手.
等我主的聖山.
自我清潔的手.
等我主的聖山.
自我清潔的手.
等我主的聖山.
自我清潔的手.
等我主的聖山.
\newpage



\section{使徒行傳 2:1-13-20230527}
\label{sec:OcD6qni0UQE}
\textbf{【網上崇拜】霹靂一閃|使徒行傳2\_1-13|20230527 [OcD6qni0UQE]}
\newline
\newline
連結: \href{https://youtube.com/watch?v=OcD6qni0UQE}{\texttt{ https://youtube.com/watch?v=OcD6qni0UQE}} ~~~~ 語音日期: 2023-05-27 
\newline
\newline
\hyperref[sec:Aqi5hKVncec]{\small{< < < PREV SERMON < < <}}
~
\hyperref[sec:index_chronic]{\small{[返順時目]}}
~
\hyperref[sec:index_scriptual]{\small{[返順卷目]}}
~
\hyperref[sec:RYCxV16hfwM]{\small{> > > NEXT SERMON > > >}}
\newline
\newline
使徒行傳 2:1-13-20230527
\newline
\begin{longtable}{cl}
\hline
\hline
章節 & 經文 (和合本修訂版)\\
\hline
2:1 & \begin{tabularx}{0.7\textwidth}{X} 五旬節那日到了,他們全都聚集在一起。 \end{tabularx} \\ \\ \relax
2:2 & \begin{tabularx}{0.7\textwidth}{X} 忽然,有響聲從天上下來,好像一陣大風吹過,充滿了他們所坐的整座屋子; \end{tabularx} \\ \\ \relax
2:3 & \begin{tabularx}{0.7\textwidth}{X} 又有舌頭如火焰向他們顯現,分開落在他們每個人身上。 \end{tabularx} \\ \\ \relax
2:4 & \begin{tabularx}{0.7\textwidth}{X} 他們都被聖靈充滿,就按著聖靈所賜的口才說起別國的話來。 \end{tabularx} \\ \\ \relax
2:5 & \begin{tabularx}{0.7\textwidth}{X} 那時,有從天下各國來的虔誠的猶太人,住在耶路撒冷。 \end{tabularx} \\ \\ \relax
2:6 & \begin{tabularx}{0.7\textwidth}{X} 這聲音一響,許多人都來聚集,各人因為聽見門徒用他們各自的鄉談說話,就甚納悶, \end{tabularx} \\ \\ \relax
2:7 & \begin{tabularx}{0.7\textwidth}{X} 都詫異驚奇說:「看哪,這些說話的不都是加利利人嗎? \end{tabularx} \\ \\ \relax
2:8 & \begin{tabularx}{0.7\textwidth}{X} 我們每個人怎麼聽見他們說我們生來所用的鄉談呢? \end{tabularx} \\ \\ \relax
2:9 & \begin{tabularx}{0.7\textwidth}{X} 我們帕提亞人、瑪代人、以攔人,和住在美索不達米亞、猶太、加帕多家、本都、亞細亞、 \end{tabularx} \\ \\ \relax
2:10 & \begin{tabularx}{0.7\textwidth}{X} 弗呂家、旁非利亞、埃及的人,並靠近古利奈的利比亞一帶地方的人,僑居的羅馬人, \end{tabularx} \\ \\ \relax
2:11 & \begin{tabularx}{0.7\textwidth}{X} 包括猶太人和皈依猶太教的人, 克里特人和阿拉伯人,都聽見他們用我們的鄉談講論神的大作為。」 \end{tabularx} \\ \\ \relax
2:12 & \begin{tabularx}{0.7\textwidth}{X} 眾人就都驚奇困惑,彼此說:「這是甚麼意思呢?」 \end{tabularx} \\ \\ \relax
2:13 & \begin{tabularx}{0.7\textwidth}{X} 還有人譏誚,說:「他們是灌滿了新酒吧!」 \end{tabularx} \\ \\ \relax
2:14 & \begin{tabularx}{0.7\textwidth}{X} 彼得和十一個使徒站起來,他就高聲向眾人說:「猶太人和所有住在耶路撒冷的人哪,這件事你們要知道,要側耳聽我的話。 \end{tabularx} \\ \\ \relax
2:15 & \begin{tabularx}{0.7\textwidth}{X} 這些人並不像你們所想的喝醉了,因為現在才早晨九點鐘。 \end{tabularx} \\ \\ \relax
2:16 & \begin{tabularx}{0.7\textwidth}{X} 這正是藉著先知約珥所說的: \end{tabularx} \\ \\ \relax
2:17 & \begin{tabularx}{0.7\textwidth}{X} 『神說:在末後的日子,我要將我的靈澆灌凡血肉之軀的。你們的兒女要說預言;你們的少年要見異象;你們的老人要做異夢。 \end{tabularx} \\ \\ \relax
2:18 & \begin{tabularx}{0.7\textwidth}{X} 在那些日子,我要把我的靈澆灌,甚至給我的僕人和婢女,他們要說預言。 \end{tabularx} \\ \\ \relax
2:19 & \begin{tabularx}{0.7\textwidth}{X} 在天上,我要顯出奇事,在地下,我要顯出神蹟,有血,有火,有煙霧。 \end{tabularx} \\ \\ \relax
2:20 & \begin{tabularx}{0.7\textwidth}{X} 太陽要變為黑暗,月亮要變為血,這都在主大而光榮的日子未到以前。 \end{tabularx} \\ \\ \relax
2:21 & \begin{tabularx}{0.7\textwidth}{X} 那時,凡求告主名的都必得救。』 \end{tabularx} \\ \\ \relax
2:22 & \begin{tabularx}{0.7\textwidth}{X} 「以色列人哪,你們要聽我這些話:拿撒勒人耶穌就是神以異能、奇事、神蹟向你們證明出來的人,這些事是神藉著他在你們中間施行,正如你們自己知道的。 \end{tabularx} \\ \\ \relax
2:23 & \begin{tabularx}{0.7\textwidth}{X} 他既按著神確定的旨意和預知被交與人,你們就藉著不法之人的手把他釘在十字架上,殺了。 \end{tabularx} \\ \\ \relax
2:24 & \begin{tabularx}{0.7\textwidth}{X} 神卻將死的痛苦解除,使他復活了,因為他原不能被死拘禁。 \end{tabularx} \\ \\ \relax
2:25 & \begin{tabularx}{0.7\textwidth}{X} 大衛指著他說:『我看見主常在我眼前,他在我右邊,使我不至於動搖。 \end{tabularx} \\ \\ \relax
2:26 & \begin{tabularx}{0.7\textwidth}{X} 所以我心裡歡喜,我的舌頭快樂,而且我的肉身要安居在指望中。 \end{tabularx} \\ \\ \relax
2:27 & \begin{tabularx}{0.7\textwidth}{X} 因你必不將我的靈魂撇在陰間,也不讓你的聖者見朽壞。 \end{tabularx} \\ \\ \relax
2:28 & \begin{tabularx}{0.7\textwidth}{X} 你已將生命的道路指示我,必使我在你面前充滿快樂。』 \end{tabularx} \\ \\ \relax
2:29 & \begin{tabularx}{0.7\textwidth}{X} 「諸位弟兄,先祖大衛的事,我可以坦然地對你們說:他死了,也埋葬了,而且他的墳墓直到今日還在我們這裡。 \end{tabularx} \\ \\ \relax
2:30 & \begin{tabularx}{0.7\textwidth}{X} 既然大衛是先知,他知道神曾向他起誓,要從他的後裔中立一位坐在他的寶座上。 \end{tabularx} \\ \\ \relax
2:31 & \begin{tabularx}{0.7\textwidth}{X} 他預先看見了,就講論基督的復活,說:『他不被撇在陰間;他的肉身也不見朽壞。』 \end{tabularx} \\ \\ \relax
2:32 & \begin{tabularx}{0.7\textwidth}{X} 這耶穌,神已經使他復活了,我們都是這事的見證人。 \end{tabularx} \\ \\ \relax
2:33 & \begin{tabularx}{0.7\textwidth}{X} 他既被高舉在神的右邊,又從父受了所應許的聖靈,就把你們所看見所聽見的,澆灌下來。 \end{tabularx} \\ \\ \relax
2:34 & \begin{tabularx}{0.7\textwidth}{X} 大衛並沒有升到天上,但他自己說:『主對我主說:你坐在我的右邊, \end{tabularx} \\ \\ \relax
2:35 & \begin{tabularx}{0.7\textwidth}{X} 等我使你的仇敵作你的腳凳。』 \end{tabularx} \\ \\ \relax
2:36 & \begin{tabularx}{0.7\textwidth}{X} 故此,以色列全家當確實知道,你們釘在十字架上的這位耶穌,神已經立他為主,為基督了。」 \end{tabularx} \\ \\ \relax
2:37 & \begin{tabularx}{0.7\textwidth}{X} 眾人聽見這話,覺得扎心,就對彼得和其餘的使徒說:「諸位弟兄,我們該怎樣做呢?」 \end{tabularx} \\ \\ \relax
2:38 & \begin{tabularx}{0.7\textwidth}{X} 彼得對他們說:「你們各人要悔改,奉耶穌基督的名受洗,使你們的罪得赦免,就會領受所賜的聖靈。 \end{tabularx} \\ \\ \relax
2:39 & \begin{tabularx}{0.7\textwidth}{X} 因為這應許是給你們和你們的兒女,並一切在遠方的人,就是給所有主—我們的神所召來的人。」 \end{tabularx} \\ \\ \relax
2:40 & \begin{tabularx}{0.7\textwidth}{X} 彼得還用更多別的話作見證,勸勉他們說:「你們當救自己脫離這彎曲的世代。」 \end{tabularx} \\ \\ \relax
2:41 & \begin{tabularx}{0.7\textwidth}{X} 於是領受他話的人,都受了洗;那一天,門徒約添了三千人。 \end{tabularx} \\ \\ \relax
2:42 & \begin{tabularx}{0.7\textwidth}{X} 他們都專注於使徒的教導和彼此的團契,擘餅和祈禱。 \end{tabularx} \\ \\ \relax
2:43 & \begin{tabularx}{0.7\textwidth}{X} 眾人都心存敬畏;使徒們又行了許多奇事神蹟。 \end{tabularx} \\ \\ \relax
2:44 & \begin{tabularx}{0.7\textwidth}{X} 信的人都聚在一處,凡物公用, \end{tabularx} \\ \\ \relax
2:45 & \begin{tabularx}{0.7\textwidth}{X} 又賣了田產和家業,照每一個人所需要的分給他們。 \end{tabularx} \\ \\ \relax
2:46 & \begin{tabularx}{0.7\textwidth}{X} 他們天天同心合意恆切地在聖殿裡敬拜,且在家中擘餅,存著歡喜坦誠的心用飯, \end{tabularx} \\ \\ \relax
2:47 & \begin{tabularx}{0.7\textwidth}{X} 讚美神,得全體百姓的喜愛。主將得救的人天天加給他們。 \end{tabularx} \\ \\
[1ex]
\hline
\hline
\end{longtable}
$^{1}$頂姐妹晚安.
對我作為牧者來說.
我今天最大的關懷是.
基督徒和教會.
在未來非常近的宗教限制下.
我們會做什麼準備.
我不知道大家有沒有聽說過.
其實下一步就差不多是宗教的範疇了.
使徒行主時代出現的事情.
很可能在非常近的未來.
我們就要去面對.
我可以想像的是.
在這段時間甚至是這星期.
我聽到一些非常荒謬的事情.
好像機關槍一樣在我們身上.
很可能這些事情會在宗教的領域出現.
但我不會舉例子.
因為我怕他們很懶.
抄襲我在這裡說的話.
我不想讓他們知道我重視什麼.
所以我不會在這裡說什麼.
而在這個月裡.
我單獨約談弟兄姊妹.
我的組員的時候.
我有兩次問他們.
其實你沒有什麼準備.
然後我兩個組員不約而同.
都想了一會.
然後就問有什麼可以準備.
然後我就說.
對呀有什麼準備呢.
我們再停了一會.
然後就說不如我們一起.
在這個節目裡討論一下.
我們可以怎樣去準備.
然後我的組員又回覆我.
他說我想即使大家.
去討論這個話題.
我去問大家也不會得出什麼.
大家都不會想到什麼.

$^{41}$所以今天我向上帝祈禱的是.
透過《史萊恆傳》2章1至13節的經文.
讓我和所有的弟兄姊妹.
可以Being Empowered.
去嘗試去準備一下.
作為留在香港的.
教徒和教會的未來.
我們這段經文一起進入去.
大家應該都很熟悉.
我輕輕說一下.
有一天使徒和門徒.
如常像大家一樣聚集在一起的時候.
突然之間有一股很大很大的風.
從天空吹下來.
然後還有一個很奇怪的情景.
就是竟然有很多條舌頭.
好像火一樣的出現.
而且分別降臨在.
每一個使徒和門徒身上.
然後聖經去形容.
這個是聖靈充滿.
而且因為聖靈.
他們可以懂得不同地方的語言.
弟兄姊妹.
大家知不知道.
這個禮拜正正就是.
聖靈降臨的主人.
如果一會兒.
這裡有什麼事情發生.
我們千萬不要害怕.
我們有聖經的.
我們如果聽到隔壁有古怪的聲音.
你舉手示意.
我們後面會有人招待他.
我們先說回經文.
第五至十三節.
經文形容了有不同地方來的人.
他聽到整個屋的人都在說.
不同地方他們各自的鄉談.
他們各自的鄉談.

$^{81}$他們覺得非常詫異.
第七節.
他們詫異驚奇地說.
這些說話的不是都是加利利人嗎.
我們每個人為什麼聽到.
他們說我們生來所用的鄉談呢.
作者路加在這裡.
我們數了一二三四五六七八九十.
這裡有十五個.
十五個不同款式的鄉談.
他們很詫異這班加利利人.
在他們眼中.
加利利人只可能是鄉下人.
為什麼他們厲害得像.
很多國風會的外交人員.
為什麼他們可以這麼厲害.
他們不明白.
所以有些人說他們是不是喝大了.
還要是喝劣質的酒.
喝了很多才喝到這麼大.
這段經文大概是這樣.
而我今天只是想捉一個地方.
這段經文的全副音到普天下去.
這段經文的意思都是毫無疑問.
大家一直以來都是這樣聽的.
但是今天我想試一下.
由彼得如何向眾人解釋.
去重新看一次.
聖靈降臨帶給我們的力量.
是些什麼.
《史特恆傳》二章十六至十八節.
是彼得引用.
《舊約先知書》《約爾書》.
去解釋剛剛發生的事.
我們一起看.
二章十六至十八節.
「這正是藉著先知約爾所說的.
神說在末後的日子.
我要將我的靈囂冠.
繁血欲滋驅的.

$^{121}$你們的兒女要說預言.
你們的少年要見義將.
你們的老人要做義盟.
在那些日子.
我要把我的靈囂冠.
甚至給我的僕人和婢女.
他們要說預言」.
彼得在這裡告訴我們.
大家看到一個很震撼的場面.
是因為上帝應驗了.
先知約爾書的我的應許.
在這一刻.
就是上帝的靈囂冠.
祂的僕人和婢女.
而聖靈充滿了他們之後.
他們就會像先知一樣.
這樣工作.
彼得就是這樣.
帶大家去解釋這件事情.
我們再深入一點看一看.
彼得如何引用約爾書.
和彼得.
如何引用約爾書.
彼得不是直接抄寫的.
他是改過一些東西.
不知道大家看這段經文的時候.
有沒有發現過這一點.
而他改過的東西.
就很重要.
就是他特別想要高光的東西.
其中一樣是在約爾書裡面.
他講到.
「在那些日子.
我要把我的靈囂冠.
甚至給我的僕人和婢女的時候.
其實已經完結了.
沒有再多說了.
但是當彼得引用的時候.
在第十八節.
他額外再加多一句.

$^{161}$我在PowerPoint上有用橙色的字.
去highlight.
他們要說預言.
為什麼呢?.
為什麼會突然加上一句呢?.
上一句其實已經有說了.
你們的兒女要說預言.
已經說了要說預言.
為什麼彼得覺得不夠.
還要補多一句呢?.
彼得為什麼要這麼強調先知的工作呢?.
特別是講到當中的.
言說的部分呢?.
為什麼他要把聖靈降臨.
很close 很緊緊地.
要和先知的工作放在一起呢?.
其實我看到這裡的時候.
我問上帝.
我說.
上帝要我講一篇怎樣的道.
我們有聖靈.
是 我們是有聖靈的.
但是我想我們很多人都不會覺得.
自己和先知有什麼關係.
現在聽到什麼.
時代先知.
我們都會覺得.
是不是有什麼神秘的能力.
其實是什麼來的.
而當我看回.
舊約先知書的時候.
從他們的經歷.
我都覺得很糟糕.
上個星期嘉Sir說他不務正業.
他說他講太多舊約.
但我多謝他.
我本身都知道.
先知很慘.
但他說得更慘.
我不知道大家記不記得.

$^{201}$我自己有個很深印象.
是他的堂主任.
還是牧師朋友請他喝東西.
然後就不停地在那裡.
我忘記了他是喝兩罐啤酒.
還是只是有瓶裝的東西.
他就說.
栽培下一代都沒有用.
栽培完都要走了.
但他講完之後.
堂主任不停地在那裡.
不斷地在那裡工作.
先知就是這樣.
我記得好幾年前.
有個教會的傳道人生了一個兒子.
那時候我還在做神學實習.
他跟我介紹他兒子的名字.
他就很有趣.
他就說他叫Jeremiah.
他叫耶利米.
他是刻意去改這個名字.
他很希望他兒子在這個時代.
做一些怎樣怎樣的事.
你們會知道是怎樣怎樣.
但我就忘記了.
因為當時我想的.
是很慘.
耶利米一輩子都這麼慘.
作為一個爸爸.
你竟然還想你兒子成為一個這麼慘的人.
這樣好嗎.
這是我心裡的想法.
我沒有跟他說.
但我記得.
我當然沒有跟他說.
但聖靈降臨這件事.
能夠把聖靈充滿.
和先知的能力.
工作緊緊扣在一起.
我就會問一個問題.

$^{241}$在《使徒行傳》.
甚至是作者路加.
是怎樣看聖靈充滿.
和先知能力這件事.
所以我們今天除了《使徒行傳》之外.
我們還會看多一點路加福音.
大家都知道.
路加福音和《使徒行傳》.
是同一個作者.
所以我們一起帶著這個問題.
去看路加福音.
這個問題我們再說一次.
在路加裡面.
是怎樣看聖靈充滿.
和先知能力的這件事.
我們看看.
耶穌是何時被聖靈充滿.
《詩章》一節.
耶穌被聖靈充滿.
從約旦河回來.
聖靈將祂引到曠野.
那時發生了什麼事呢.
就是耶穌被試探.
但最後耶穌是贏的.
然後去到《詩章》十四節.
在耶穌正式傳道之先.
路加形容.
耶穌沒有聖靈的能力.
然後回到加利利.
我們又看看.
被聖靈充滿的耶穌.
和先知的能力有什麼關係.
在同一章第四章十六至十七節.
在耶穌正式傳道的時候.
在會堂裡打開了一卷.
聖經.
在會堂裡打開了一卷.
先知書.
二賽亞書.
耶穌打開二賽亞書.

$^{281}$自己找到第六十一章.
主的靈在我身上.
因為祂用高高我叫我傳福音.
給貧窮的人.
大家記得這段經文嗎.
聽展會耶穌說的.
他的工作.
就是要讓先知所說的.
應驗在眾人當中.
我們看看.
二十一節他怎麼說.
耶穌對他們說.
今天這經應驗在你們耳中了.
我再說一個例子.
被聖靈充滿的耶穌.
和先知的能力工作有什麼關係.
耶穌讀完二賽亞書之後.
他又寫了兩個先知工作的例子.
我們看看下一個powerpoint.
一個是以利亞.
一個是以利沙.
他當中提到.
西頓實立法的一個寡婦.
和敘利亞國的賴孟.
大家知道.
他們是外邦人.
在外邦地方.
所以耶穌為什麼要誣衊他們.
耶穌誣衊他們.
是想告訴我們.
昔日的先知.
奉上帝的靈.
已經被差派到外邦人當中.
耶穌還說.
當時以色列當中.
其實也有很多寡婦.
很多大麻瘋的病人.
奉上帝的靈.
就是沒有差派先知到以色列當中.
反而去了外邦寡婦.

$^{321}$和外邦將軍那裡.
這些說話.
是否好聽.
傷害同胞的感情.
當時的人一定予以嚴重的譴責.
當時會堂的人.
聽到這番說話.
他們很生氣.
要一起推耶穌出城.
推耶穌下山.
耶穌是厲害的.
路加形容耶穌在他們之間.
直行直過.
所以我們回歸一開始的問題.
作者路加.
是怎樣看聖靈和先知的能力.
原來是耶穌自己.
先被聖靈充滿.
然後他的行為.
就好像先知一樣.
要將福音傳給外邦人.
他照傳.
他要做什麼作為.
他要做什麼見證.
說什麼話.
他都會照說.
而大卿子母我們看《史圖行傳》.
都是這樣.
《史圖》和《門徒》.
都是這樣做.
他們都是先被聖靈充滿.
然後就像昔日的先知和耶穌一樣.
將上帝的工作和道.
說出來.
阿彌陀佛對我來說.
是這件事.
我舉兩個《史圖行傳》的經文例子.
就在今天的經文二章十一節.
經文裡面說.
他們被聖靈充滿之後.

$^{361}$不同的人都聽見他們用暗默的香談.
講論上帝的大作為.
重點是講論上帝的大作為.
不是只是我十五個地方的語言.
我十五個地方的語言講完之後.
其實他都是在講上帝的大作為.
在第四章.
當彼得和約翰.
在那些大祭司.
亞該法約翰和亞歷山大.
和那些大祭司的親屬.
包圍著他們的情況下.
去查問他們.
你們憑什麼能力.
奉誰的名去做這件事的時候.
經文說.
那時彼得被聖靈充滿.
對他們說.
現在你們所見到的病人都醫治.
是因為你們所釘在十字架.
神使他從死人中復活的.
那撒勒人耶穌基督的名.
等等等等.
然後第十三節.
經文說.
他們看見彼得和約翰的膽量.
也知道這兩個人是沒有學問的平民.
他們就很驚奇.
重點依然是彼得被聖靈充滿之後.
他可以在眾人的面前.
講耶穌基督的作為.
每一次的聖靈充滿.
在使徒行傳裡面.
那群使徒門徒就可以在眾人的面前.
去講出上帝的作為.
而有一樣東西.
很想大家特別去知道的是.
因為使徒行傳的經文是額外提到的.
他說連對手都驚訝他們的膽量.
也很驚奇他們為什麼可以這麼會說話.

$^{401}$在路加福音.
弟兄姊妹你們看的時候.
不知道你們有沒有看到.
我就沒有看到.
他沒有怎樣形容耶穌.
因為聖靈而有膽量.
但在使徒行傳.
經常都在講.
大家記不記得彼得三次不認主.
他不認主的時候.
那些人就說.
嘩!我認得出.
這個人曾經和耶穌在一起.
我們看看這段經文.
他說他們見彼得有他們的膽量.
又看出他們原是沒有學問的平民.
就很驚訝.
他認出他們曾經和耶穌在一起.
都是這一句說話.
他同樣都認得.
嘩!.
但現在的彼得竟然可以在這麼多人圍著他的情況下.
他有膽去找回信仰去伸變.
去找回這群圍著他的人去悔改.
路加很清晰地表達.
這份膽量是從上帝.
是從聖靈來的.
就好像他們被釋放回去之後.
他們將他們遇見的事情.
告訴門徒.
通常在這個時候.
我們就害怕.
我們就開始說.
我們不要再講了.
別人都叫我們不要再講了.
但他們不是.
經文記載他們早上一起祈禱.
我們留意他們是怎樣祈禱.
他說29節.
主也要求你監察.

$^{441}$他們的威嚇.
使你的僕人放膽講你的道.
伸出你的手讓伊至神即其事.
藉著你的聖僕耶穌的名行出來.
他們祈禱之後.
聚會的地方震動.
他們被聖靈充滿.
然後他們可以放膽傳講神的道.
所以在這裡我想破一破.
大家可能有的迷思.
聖靈充滿不是一些很自high的東西.
也不是一些很神秘的經驗.
聖靈充滿是在我們這個時代.
讓我們可以不要縮.
讓我們可以很有智慧地.
在眾人面前宣講.
上帝作為的一個力量.
聖靈充滿是讓我們有這一份力量.
前幾天有個組員.
應該是看了最近的新聞.
他也有發訊息.
他說我知道.
他說我們的決定.
不只是出於眼前的利益.
也要考慮道德和上帝的想法.
他說現在的教會.
現在的香港教會.
會憑福音.
會憑福音做到這個empowerment.
他說他覺得有點難.
收到這個訊息.
如果你們收到.
你們會怎樣反應呢.
我覺得這個感受很真.
這件事很難.
非常難.
但我可以說什麼呢.
我自己是否確保自己都可以做得到呢.
我也不敢說.
我只可以說.

$^{481}$耶穌叫我們等聖靈.
在《士多人傳》一章八節.
他說聖靈降臨在你們身上.
在一章第四至第五節的時候.
他叫那些使徒和巫徒等.
留在耶路撒冷.
不要離開.
因為要等聖靈.
耶穌也叫我們等聖靈.
讓聖靈充滿我們.
但我怕我做不到.
但我看到.
我做不到.
但我求神讓我們.
不知為何做好.
不知為何做到.
都可以的.
其實耶穌自己早就說過.
他說人帶你們去會堂.
去到官府或有權柄的人面前.
不要思慮怎樣分數.
不要說什麼話.
因為只是在這個時候.
聖靈要指教你們.
當說的話.
耶穌所說的話.
在《士多人傳》中已經應驗了.
甚至在耶穌自己身上.
也是這樣.
他也是被抓和釘.
在歷史的長河裡.
我們也看到.
但到今天的香港.
我更加覺得非常近.
弟兄姊妹.
我擔心的是.
聖靈還沒有機會.
為我們做見證.
我們就要一起退縮.
我們不敢說聖經.

$^{521}$不敢說上帝的真理.
平時也沒有這麼合一.
但現在就合一了.
一起縮了.
我擔心的是.
當宗教自由被限制的時候.
我們只覺得無可奈何.
但我們沒有靠著聖靈.
勇敢地走出來.
去見證.
上帝是不容限制的.
然後全香港的人.
就會看到.
回教徒是這樣的.
你知道嗎?.
幾年前.
我竟然有點羨慕回教徒.
當水炮車射到清真寺的時候.
回教徒是馬上反對的.
然後我第一次.
見到他們道歉.
還要拿著一塊布.
去清潔.
我怕我們會退縮.
就像過去的疫苗通行證.
連崇拜都可以拒絕.
未打針或未打夠針的人.
但聖靈充滿之下的.
信徒群體是怎樣的呢?.
他們是在.
眾官員甚至是聖殿人員.
圍著他們的時候.
什麼聖殿人員?.
就是現在說的教會那群人.
圍著我們的時候.
整個勢是要他們閉嘴.
他們閉嘴的時候.
我們要站出來.
堅定地表達耶穌的復活.
你們應該悔改.

$^{561}$他們知道自己在做什麼.
然後他們說.
我們下一個關鍵字.
寫了19字.
聽從你們不聽從神.
在神面前合理不合理.
你們自己判斷.
我們所看見所聽見的.
我們不能夠不說.
弟兄姊妹讓我在這裡.
很大膽地說.
我們面對大爆發的時候.
的回應或準備.
可能正正就是.
不可能對不起.
應該是正正就是.
繼續放膽去說上帝的道.
教徒不能夠被滅聲.
上帝的道不可以在香港消失.
我們看看使徒行傳.
那些人他們一直被打壓.
但他們一直.
整本書的重點是什麼?.
我們看下一個關鍵字.
使徒行傳第六章第七節.
神的道興旺起來.
耶路撒冷的數目就增加很多.
八章二十五節.
使徒結束了見證.
並宣講了主的道周回耶路撒冷.
十二章二十四節.
神的道日見興旺.
十三章四十九節.
於是主的道傳遍了我一帶的地方.
下一章.
十五章三十五至三十六節.
保羅和巴拿巴留在安提沃.
與許多人一起教導.
並傳揚主的道.
過了這一節保羅和巴拿巴說.

$^{601}$我們回去從前宣揚主道的覺醒.
十九章十節.
二十節.
這裡有兩田之九市一起主宰亞洲人.
無論猶太人還是希臘人都聽見主的道.
這樣主的道就大大興旺.
我記得自己剛剛來Flo Church的時候.
有一次開祖.
我問.
大家的信仰有什麼可以妥協.
有什麼是不可以的.
弟兄姊妹.
雖然我們還不知道中間限制會有多少.
但我想我們第一件很需要知道的事.
我們的信仰是什麼.
我們的信仰是什麼呢.
上帝的道.
你會說什麼呢.
我們真的看回彼得.
亞視提反.
保羅和其他門徒.
在《史同行傳》的每一篇講道.
我們聽聽他們怎麼說.
他們很長.
有一章保羅說三章.
他們都是在說他們的信仰.
說他們一直見證.
用生命跟隨著的信仰.
我們是什麼呢.
如果我們自己都不清晰的知道.
很容易被割到體無完膚.
第二件事.
我很想大家知道的是.
見證是不能等下一次.
見證就是這樣.
見證就是有這個屬性.
不能說今天先不見證.
等下次.
我們現在不是在說感恩分享的見證.
我不敢了.

$^{641}$下次再說吧.
沒所謂.
今天我們很可能面對的情況是.
就好像六國論裡的一句.
今日割五成.
明日割十成.
然後大家看下一段.
有人說.
我們的忍耐是為了能夠傳福音.
那你傳的是什麼福音呢.
有人說.
我們的忍耐是為了能夠繼續牧羊.
那你牧羊弟兄姊妹的是什麼呢.
有人說.
我們要有智慧.
要靈巧像蛇 純良如甲子.
但史徒行傳六章十節去記載.
史提番正正就是用智慧.
和聖靈去說話.
他面對那些大祭司.
百姓 長老 文士.
當中他講了一篇很長的見證.
他為基督信仰去見證.
(深呼吸).
弟兄姊妹其實.
我一邊看著這篇聖經.
一邊見著香港.
我是害怕的.
我不知道大家會不會呢.
大家有沒有看過一套動漫.
當中有一個角色叫我妻善逸.
金色頭髮這個.
有看的大家大概都知道.
他好像來搞笑.
每次見到他出場的時候.
都是一副害怕的樣子.
但他卻是我最喜愛的角色排行榜中的第一名.
是最受大家喜歡的角色.
我想或許是.
因為這個角色夠人性.

$^{681}$簡單來說就是.
當我們面對很強大的威脅的時候.
我們都會害怕.
但他最棒的地方是什麼呢.
雷之呼吸 一之型 霹靂一閃.
當善逸睡著了的時候.
他竟然可以變得很無謂.
和很勇猛.
大家知道嗎.
睡著了的那個也是他.
沒睡覺的那個也是他.
人是可以.
又膽小 但又很大膽.
好像很衝突一樣.
但會有的.
我就是這樣.
我害怕 害怕 其實我編導都差不多說完了.
都說到尾聲了.
我知道是聖靈不停地給我勇氣和話語.
我求聖靈幫我們.
給我們膽量.
給我們勇氣.
讓我們知道我們應該怎樣見證.
怎樣見證上帝.
基督的十教福音.
是不容妥協的.
越是見證 福音就越是傳開.
越是見證 就越證明信仰的真實.
是人願意去竭力去維護的.
聖靈充滿不是一個重點.
不是什麼聖靈激倒.
說一些我們聽不懂的東西.
我們可以很安全的.
因為耶穌也被聖靈充滿.
我們得到上帝的這個英許.
在約爾書的這個英許.
正正就是讓我們可以面對這一刻.
請我們一起祈禱.
你讓我們知道.
當我們膽小的時候.

$^{721}$聖靈你仍然在激勵很多其他的人.
讓我們見得到.
希望還能見到浸大學生會的會長.
還在堅持.
真的還有人這樣做.
顛覆我不能想像的是.
香港這個地方.
我們基督徒.
教會會很被動的接受扭曲信仰的一切.
就像過去兩年.
因為疫苗通行證而不能回教會一樣.
我們向你祈禱.
我聖靈我求你.
給我們膽量去做一些.
跟之前不一樣的決定.
主我求你.
聖靈請求你這一刻去充滿我們.
建立我們.
給我們膽量.
讓退縮和讓步.
不會成為我們第一步.
讓步不會成為我們第一個選項.
奉主耶穌基督得勝的名字祈求.
阿門.
\newpage



\section{}
\label{sec:RYCxV16hfwM}
\textbf{《致餘民及流散者:給香港基督徒的神學八課》第二季第4課|20230528 [RYCxV16hfwM]}
\newline
\newline
連結: \href{https://youtube.com/watch?v=RYCxV16hfwM}{\texttt{ https://youtube.com/watch?v=RYCxV16hfwM}} ~~~~ 語音日期: 2023-05-28 
\newline
\newline
\hyperref[sec:OcD6qni0UQE]{\small{< < < PREV SERMON < < <}}
~
\hyperref[sec:index_chronic]{\small{[返順時目]}}
~
\hyperref[sec:index_scriptual]{\small{[返順卷目]}}
~
\hyperref[sec:40Zpw7rWZSQ]{\small{> > > NEXT SERMON > > >}}
\newline
\newline
$^{1}$香港人仍然是香港人.
究竟上帝的旨意如何?.
我們應該如何生活?.
我們應該如何為主而活?.
無論你身處何處,只要你是香港人,.
邀請你和我們一起思想這個流散年代的信仰..
致愚民與流散者,給向我基督徒的神學百科..
各位頂尖會員,大家好!.
歡迎大家來到我們神學百科第二季的第四課,.
Homeless 回家的人生..
我覺得這課是很好玩的..
首先,這課是很切合我們這個年代的需要..
一方面是我們很多頂尖會員移民到海外重新找教會..
教會成為了一個很重要的家..
這也是我以前在德國讀書的經驗..
我在德國柏林生活了六年..
德國柏林教會是我的家..
我的家常常被人迫使我睡覺..
很多頂尖會員來我家過夜,睡沙發..
我們會煮飯給他們吃..
每逢星期五晚上,我會煮十多碗飯給他們吃..
我懂得煮飯,煮飯給他們吃..
所以是一個很小的群體,三十多人的群體..
送機,接機,去家裡玩..
很多這些事情,特別在外國也是這樣..
中秋節,新年這些,在外國是找不到人跟你一起過的..
教會的頂尖會員就跟你一起過..
以前你在家裡,新年,拜年,中秋節,.
教會成為了一個很重要的功能..
所以我覺得海外的教會,海外的頂尖會員,.
他們在英國,加拿大,在不同的地方,.
他們重新來重拾他們的上群體..
家裡這個概念,很值得我們花一點時間,.
用聖經,用神學來思考..
這方面就算在香港也是,.
香港裡面,我們橫教會,.
我們很流行用家字來指稱,.
以前沒有會,你自己在教會叫什麼家,.
不知道是不是這樣稱呼..
大家會穿同一件T恤,.

$^{41}$然後回到家裡,這是家事報告,.
要有家書之類..
我們橫教會也很喜歡用家字來成為教會一個很重要的歸屬感的地方..
當然我們知道,Full Church不是很流行,.
我挺在意不去說這些東西..
我一向都不太強調Full Church人,.
Full Church家,還是大家回到這裡..
不過很有趣的是,Full Church的Banner,.
那個Elijah寫什麼呢?.
就是歡迎回家,.
但實際上又是突然叫大家回家..
所以今天我們會說一個很特別的題目,.
也跟我們Full Church很重要的DNA有關,.
我們怎麼看教會作為家這件事呢?.
事實上我們知道這幾年來,.
很多人對於教會家的觀念是有些批判的..
我在網上看過一些文章,.
很強烈地說不應該把教會當成家..
你知道以前的神學,牧師,.
用什麼聖經來背後他們的說法,.
為什麼說教會是家?.
其中一個很重要的就是,.
教會是神的家,.
這些是神的家事,.
諸如此類..
所以今天我們會花一點時間去看聖經,.
究竟我們如何理解神的家這個概念呢?.
有些人反對,覺得那個經文不是這樣解釋,.
諸如此類..
所以今天我們會去探討這個課題,.
很多人都說我在柏林,.
這裡是柏林的我,.
我覺得今天是一個很開心的日子,.
今天我仍然跟那些頂次妹非常親密,.
因為我們是共同生活了幾年的弟兄姊妹..
所以當時海外的教會往往都成為了一個同鄉會的功能,.
這個沒有什麼貶義的,.
當然是有一種功能的,.
海外裡,你在外國裡,.
一群香港人,一群華人聚在一起,.

$^{81}$這是一個很重要的概念..
所以今天如果你在看我們百貨裡面的海外頂次妹,.
我想你也很同意吧,.
我們的海外教會,.
我們Full Mission,.
我們Full Church這個群體,.
其實都是一個成為了你很有歸屬感的地方..
所以今天我們可以來思考這個課題,.
其實我們華人教會對於家的觀念似乎是很重要的,.
這個是關乎我們對於堂會忠誠的文化,.
往往,如果很負面去理解的話,.
教會用了家這個觀念來鎖住你,不讓你走,.
有些人這樣說過,.
你不要離家出走,.
你離開家裡是很嚴重的事情,.
他們覺得你不應該去其他地方去謀生,.
你應該回到自己家裡,.
所以一旦堂會稱自己為家,.
你應該對這個地方,對這個堂會忠誠,.
你要像一家人一樣,.
你要有責任,.
你要有一個承諾在家裡,.
你要給家用,.
你要幫忙去建設家,.
但似乎我們流唐並不是這樣,.
我一開始也說過,流唐似乎不是用這個方式去理解,.
但我們如何去理解我們四教會的教會,.
如何去看待家,.
究竟我們是贊成還是反對呢?.
我會去想一想,.
大家知道這首歌嗎?.
回家,Eternity Girls,.
我稱之為基督教第一隊女團,.
這首歌是大家很熟悉的詩歌,.
回家,.
我們往往會將我們的信仰和回家連上關係,.
我們試試研究歌詞,.
報佛是否感覺疲倦了,.
得得轉轉,沒了沒完,.
所以當我們信耶穌的時候,.

$^{121}$其實就是回家,.
其實有根據的,.
浪子比喻本身就是這樣的比喻,.
浪子回頭,回到家裡,.
所以我們不能否定,.
在我們的信仰裡,.
回家似乎是一個很重要的觀念,.
你回轉歸向上帝,.
正正是回到家裡,.
而當中我們信主時候的教會,.
我們就會自然地看為一個家,.
所以大家想想,.
以前的教會,.
或者所謂的母會,.
你稱它為母會,.
就是有媽媽的成分,.
所以你會發覺,.
我們信仰裡往往會很強調回家,.
都是一個家的字,.
當然我們華人教會特別,.
華人文化裡面,.
特別重視家,.
這也是我們根深蒂固的一件事情,.
我們對於家庭的觀念,.
似乎比西方文化更加重要,.
所以我們就去理解,.
究竟我們教會是不是神的家呢?.
我們怎樣去理解神的家呢?.
如果你看回經文的時候,.
我們這個神的家,.
是叫House of God,.
House of God,.
Echo Theou,.
就是一個上帝的家的意思,.
是什麼經文呢?.
最基本來說,.
我們知道提摩太前書第三章十五節,.
教會是永生神的家,.
這個大家都熟悉的經文,.
保羅這樣寫,.

$^{161}$倘若我擔言日久,.
你也可知道在神的家當中怎樣行,.
這家就是永生上帝的教會,.
真理的基石和注釋和根基,.
所以我們很多時候,.
我們環教會就用這段經文來強調,.
教會就是神的家,.
實際上,.
如果我們真的去明白這段經文,.
或者保羅很多連串裡面,.
很多這些神的家的經文,.
你會發現,.
Echo那一字,.
其實就是解作家,.
不過你會發現,.
這字可以有兩個不同的解釋,.
一個是解作House,.
就是解作Home的意思,.
一個是概念上的家,.
可以解作Building,.
一個是House的意思,.
所以這字其實有兩個不同的意思,.
一方面是Home,.
就是House的意思,.
一個是概念上的家,.
一個是建築物上的屋,.
如果是這樣的話,.
經文裡面有兩個可以不同的解法,.
教會是永生神的屋,.
還是神的家呢?.
這樣就可以有兩個很明顯的意思,.
如果是一個上帝的屋,.
是一個殿的概念,.
一個Temple of God,.
一個House of God,.
如果是Home的話,.
如果是一個家庭的意思的話,.
是另一個解法,.
所以我們可以發覺,.
可以有很多不同的思考,.

$^{201}$但如果我們看回經文裡面,.
看回提摩太前書第三章的話,.
看回上下文的時候,.
第三章裡面似乎有些暗示,.
譬如第四到五節這樣說,.
好好管理自己的家,.
這個家字也是個House,.
可以解作屋,.
或者是個Home,.
好好管理這個House,.
使兒女凡事端莊順服,.
人若不知道管理自己的家,.
焉能照顧上帝的教會呢?.
似乎這段經文裡面暗示了什麼呢?.
似乎是有關子女,兒女,家庭的事務,.
這樣說的話,.
這段經文裡面似乎是說,.
教會真的是一個Family,.
一個家的觀念,.
有兒女,有父親管理之類的,.
這樣說的話,.
似乎是神的家,.
神的Home,.
是這樣的意思,.
不過,.
再看15節之後的話,.
你會發現,.
那個字眼裡面說到有Pillar,.
有個基石和根基的話,.
又似乎是一個建築的概念,.
因為說有根基,.
有基石的話,.
這個家是說Building多一點,.
所以其實是不容易去肯定,.
經文裡面是說一個Family,.
Home的概念,.
還是說一個Building的概念,.
這是第一個,.
提摩太前書裡面所說的,.
似乎是可以有兩種不同的解法,.

$^{241}$可以是一種Family,.
Home的概念,.
家的感覺,.
也可以解作一個純粹的Building的意思,.
所以究竟教會是神的Building,.
還是純粹一個Family呢?.
我們還是未知道,.
我們再看另一段經文,.
就是希伯來書第三章和第十章的經文,.
有一次這樣出現,.
希伯來書裡面曾經有這樣的理論去強調,.
他比摩西算是更配多的榮耀,.
好像建造房屋的比房屋更加的尊榮,.
因為房屋都是必有人建造的,.
但建造的都是上帝,.
摩西為僕人,.
在上帝的全家承言尊重,.
為要證明將來必傳說的事,.
但基督為兒子,.
治理神的國家,.
若將可誇的盼望和膽量堅持到底,.
便是他的家了..
這個價值仍然可以兩不解法,.
Home Family又可以,.
Building又可以,.
所以究竟經文裡面是說一個Family,.
Home,.
還是Building呢?.
似乎我們都不容易理解,.
這裡說的是一個建造房,.
似乎是Building的比喻,.
不過又說到一個僕人,.
在一個全家進宗,.
又好像是一種Family的意思..
我們看回另一段經文,.
帖本流出第十章,.
又有大祭司治理神的家,.
被我們心中天良的虧欠已經灑去,.
身體用清水洗淨,.
就當存誠心和忠心的信心,.

$^{281}$來到神面前..
似乎在這裡,.
治理神的家裡面,.
我們都不太能夠肯定,.
究竟是Building還是House的意思..
第三段,.
就是前書第四章的經文,.
因為時候到了,.
審判要從神的家起手,.
又是我們不容易理解的..
這個經文似乎所說的是什麼意思呢?.
神的家起手審判的時候,.
起碼我們知道,.
這個審判所說的神的家是什麼意思,.
似乎未必,.
就算我們說教會也好,.
未必是說堂會的意思..
他所說的神的家,.
不是純粹說某一間的堂會,.
而是整體上帝的子民,.
審判是由整體的上帝子民開始說起..
所以我們可以知道,.
就算經文所說,.
可以說是教會作為神的家也好,.
是Building也好,.
他最少所說的不是一個堂會,.
是一個家的意思..
他所說的是一種整體基督徒群體宏觀的教會,.
是一個神的家的意思..
你不能不否定,.
我們稱之為弟兄姊妹,.
稱之為天父爸爸做爸爸,.
是一個很Family的概念..
聖經裡面說到弟兄們,.
Brother and Sister,.
天父上帝,.
耶穌是我們的兄長,.
我們是弟兄姊妹,.
Children of God,.
我們是上帝的兒女,.

$^{321}$因為我們是上帝的兒女,.
所以我們可以說是一個家的意思..
我們可以說是一個家的意思..
教會是整體教會的縮影..
當我們說我們是一個普世教會的時候,.
我們不能忽視我們實際是一個見得到的教會,.
一個很具體的群體成為我們對教會的投射..
所以沒錯,我們整體來說是宏觀的一間教會,.
但我們是有一個堂會概念,.
起碼有一個群體概念..
這個群體是你實踐教會理想的地方..
任何教會的實踐理想都應該放在這個具體的群體裡面來實踐..
所以將堂會理解為一個家,其實並不是完全錯誤..
但我們要如何理解這個家的概念?.
是不是那種要你完全忠誠,.
只是這樣的概念呢?我覺得並不是..
所以我們想說,我們作為全聖教群體,.
我們對於家的觀念並不是反對,.
但我們要重新思考究竟我們應該如何理解作為神的家的意思..
事實上,我們回看中國華人的教會,.
為什麼我們會強調這個家字呢?.
你會發現其實有很多宣教的原因在這裡..
十九世紀裡面出現了一些稱為基督教的天家小說..
事實上,因為中國人很強調家,.
一個是孔子的文化,一個是儒家的思想..
所以當西教士來到中國傳播的時候,.
他們是用了這種重視家庭的觀念來傳教..
所以他們是用了很多西方的小說來作為傳教的第一步..
你會發現,這些可能大家不知道懂不懂,.
我們華人很強調稱天堂為天家,.
雖然西人也有一個叫做Heavenly Home的字,.
但他們沒有我們強調,我們很少叫天堂,.
我們會叫天家多一點..
西人也會說Heaven,.
但比較少用Heavenly Home這個字..
所以我們對於死去的地方,我們稱之為翻天家..
我們也很強調,我們用的都是家字..
所以我們的信仰似乎一開始就把家庭的觀念吃進去了..
無論是天羅律程,無論是後面所說的忍家,當道,或安樂家,.
這些是西教士來到傳播的時候,他們翻譯的小說..

$^{361}$這些小說成為了很好的報導工具,.
他們把西方基督教的文學翻譯成中文..
這些小說全部都說一件事,.
就是原來信仰是一個回家的過程..
大家有沒有看過天羅律程?.
可能很舊了,如果你是比較資深的就看過..
我們的信仰是一個天路,回天家的天路,.
一個朝聖之旅..
所以他理解為我們在塵世裡慢慢回到天家的朝聖之旅..
所以天羅律程是一個這樣的概念,.
我們在地上生活是一種朝向朝聖之旅,.
我們在塵世裡是一個寄居的生活,.
我們最後會回到天家裡..
這就是天羅律程最重要的故事框架..
另外一本就叫做忍家當道,.
他也很強調家庭背景的故事..
故事說什麼都不重要,.
但最後還是要回到家庭觀念作為一個結束..
他覺得家庭是一個很重要的家庭,.
家庭裡大家如何實踐信仰就是一個這樣的故事..
最後一本叫做安樂家的書,.
叫做Home Sweet Home,.
也是用一個這樣的比喻,.
他將家比喻為天堂,.
將天堂比喻為家,.
用了哥羅西書的經文,.
天堂就是天家的基業,.
我們將來有一個榮耀的城,.
一個安樂家,.
就是我們所學求的地方..
所以這些在晚清時的翻譯,.
其實是慢慢地進入我們華人教會對信仰的理解,.
我們信仰裡似乎有一個很根深蒂固的,.
將家字放在信仰裡,.
我們知道天堂是我們的天家,.
我們在信仰裡是一個塵世隔離的人生..
所以我們來到這裡理解的時候,.
確實是有一些經文的意思在裡面,.
譬如用第十四章裡面,.
耶穌說什麼呢?.

$^{401}$「在我父的家裡有很多的住處,.
若是沒有,我早已告訴你們,.
我去元兆位你們預備地方.」.
所以耶穌也這樣說,.
祂強調將來是一個父的家,.
耶穌為我們預備的就是父的家,.
所以你知道你的家就是將來一個天家裡面的生活,.
我們正正是一個邁向回天家的旅程..
所以我們這班人,.
我稱之為我們那班流散的人,.
我們是一班被上帝揀選的人,.
我們漂泊在塵世裡面的人..
家這個字,.
其實也是一個很值得我們重視的觀念,.
不過那個家不是真正的家,.
不是我們的堂妹,.
不是我們以前回過的家,.
而真正的家是什麼?.
就是我們流散的旅途裡面,.
最終極的天家..
我們知道我們人生裡面是一個邁向回家的旅程..
所以我們這種神學,.
我稱之為客女神學,.
我們很強調,.
在塵世裡面,.
世界裡面,.
只不過是一個客女,.
我們在地上沒有一個真正的家,.
我們是一個邁向天家的過程..
這個神學其實我們不是陌生人,.
我們在傳統環境裡面也很強調,.
這種客女人生,.
大家不知道會不會唱這首歌,.
大家都知道這本是什麼,.
是青年聖歌,.
我們很傳統的一些聖詩..
這首歌叫做什麼?.
叫做《這世界非我家》,.
This world is not my home,.
這首歌本身是一首20世紀初的時候,.

$^{441}$一首英文的美國的詩歌,.
This world is not my home,.
這個世界不是我家..
但是我們在幾十年前的環教會來說,.
我們就很強調,.
這種世界不是我家,.
甚至我們對世界有一種反感,.
我覺得世界是一些貪愛世界,.
世界是和上帝敵黨的地方,.
所以我們不要碰太多這個世界,.
因為我們家就在我們的天堂裡面..
我們看回我們的中文翻譯詩歌,.
這世界非我家,.
我停留如客女,.
我積財寶在天,.
時刻仰望我主,.
天門為我大開,.
天使呼召迎也,.
故我不再貪愛,.
這世界為我家..
可能大家會唱這首歌..
如果我們看回這首歌的英文原文,.
你會發覺有一個很特別的地方,.
原文裡面是說.
I can't feel at home in this world anymore,.
就是說我對於這個世界再感覺不到at home的感覺,.
這是原文的歌詞,.
I can't feel at home in this world anymore,.
不過中文的翻譯似乎比英文更加強烈,.
不單覺得這個世界沒有家的感覺,.
更加是覺得不再貪愛這個世界,.
不再覺得這個世界是我的家..
有些差異在這裡,.
我不覺得這個世界是我的家,.
不代表我是反對這個世界..
所以發覺中文在我們橫教會裡面,.
對於這種客女神學,.
我們加強了一種對世界反感的.
一些很古舊的思想,.
以前的基督徒都是這樣,.

$^{481}$不要看電影,.
不要唱卡拉OK,.
這些世界的東西全部都是不好的,.
這就是客女神學..
我們在家裡,世界裡,只是客女,.
所以不要關心社會,.
不要理會這個世界,.
因為我們是客女..
我們全家人不是這樣,.
所以我們重構我們全家人,.
怎樣去理解我們怎樣看客女神學..
當我們看回這首詩歌的後來翻譯,.
後來有很多不同的人都嘗試重新翻譯這首歌,.
譬如你看到,.
不再寫這句話,.
叫我不再貪愛世界,.
即是世界為我駕馭,.
後來的翻譯者都嘗試重新翻譯這首歌詞,.
你會發覺這首歌詞全部都避免了.
這麼反對世界的神學,.
即是神是愛伴我,陪伴我,.
走擇路天交往,.
明白世界已經不再是我心所盼,.
容我滿有望盼,.
不感到孤單,.
全部都不敢將客女神學翻譯出來,.
原本是說,.
我再也不想再感到孤單,.
但這群人都不想保留對世界不是我家的感覺,.
所以你會發覺很有趣,.
我們很強調,.
我們都知道客女神學是一個很重要的,.
不能否定神學,.
我們是天家,.
耶穌是我們預備的地方,.
我們真正的家是天家,.
這個世界是流散了的世界,.
我們是一個漂泊的人生,.
我們對於這個世界是一個暫時的過程,.
但我們全聖教的客神學是甚麼呢?.

$^{521}$沒錯,我們強調是客女,.
我們是流動群體,流散群體,.
但我們不是對這個世界有任何反感,.
我們很重要,.
其中一個神學家叫做Johannes Baptiste Metz,.
他是世界的神學,.
即是說原來世界對於上帝來說並不是一件負面的事情,.
我經常說,.
用第三章十六節,.
上帝愛甚麼?.
愛世界,.
原文上帝愛世界,.
心得獨生子,.
且恃有貪玩,.
所以上帝所愛的不是世人,.
而是整個世界,.
當耶穌降世的時候,.
當祂來到世界的時候,.
祂就將世界包在祂身上,.
所以作為基督徒,.
我們再也不覺得這個世界是一種反感的感覺,.
因為天父擁抱這個世界,.
我們在客女人生中,.
我們仍然會擁抱這個世界,.
我們不覺得像以前的橫教會一樣,.
客女等同於對抗這個世界,.
不參與這個世界,.
我們是一個流散群體,.
但我們是擁抱這個世界,.
我們流散是為了在世界裡作我們的見證,.
所以我想說甚麼呢?.
今天我稱之為,.
說了大堆好像不太容易消化的事情,.
我想說的是我們對於回家這個字的看法,.
我們作為一個班基督徒,.
流淌的頂尖妹,.
我們怎樣看流淌,.
我們怎樣理解我們在地上的流散過程,.
所以今天我們會說的字叫做空明,.
空明這個字其實在一年前左右,.

$^{561}$我去討論神學的時候,.
我發覺這個字是很有意思的字,.
因為我們在今天香港裡面,.
或者你在外國裡面,.
其實我們是一個空明的過程,.
甚麼是空明?.
正正是一個漫長的回家的過程,.
就好像那些季候鳥一樣,.
我們是一個很長途的回家的過程,.
所以我們是一個朝向著一個家的旅程,.
我們知道地上沒有真正的家,.
我們不想將任何一個的堂會,.
看為我們終極的奮鬥目標,.
聽說很多這樣的故事,.
特別是台灣教會,.
台灣教會其實很強調,.
你是不可以轉教會的,.
大家記得嗎?.
之前有一個台灣的報紙才說這件事,.
在批評一些轉教會的人,.
發覺很多時候我們環教會都是這樣,.
我覺得這個堂會就是你的家,.
所以你不要離開,.
你的終身事業,.
你的侍奉,.
你的遺失都在堂會裡面,.
這個堂會的家就是你最終極的目標,.
Full Church並不是這樣,.
我們知道我們不想大家.
只是為一個堂會去賣命,.
但我們也強調,.
我們其實不是反對家這件事,.
我們覺得我們是一個回家的旅程,.
我們是一個朝向天家的過程,.
但在我們這個流散的過程裡面,.
家這個概念仍然是重要的,.
因為我們仍然有一個落腳的地方,.
我們仍然有一段很深厚的關係,.
我們仍然有弟兄姊妹,.
我們仍然有一個大家彼此好像家庭的關係,.

$^{601}$所以我們Full Church並不是反對家庭,.
我們很強調,.
我們是一班彼此相愛的弟兄姊妹,.
我們只不過不想大家.
單單為一個堂會獻上所有,.
這個就是你的終身事業的目標,.
所以我們是一個回家的過程,.
這個home字是一個動詞,.
這個home字不是一個名詞,.
因為我們強調這個home不是地上某一個地方,.
而是天上的地方,.
但我們home名是很重要的,.
我們每個星期寫著歡迎大家回家,.
是一個真心的說法,.
希望你能夠在回來崇拜裡面,.
真的能夠有一個關係,.
這個家不是你賣命的堂會,.
但是一個真實的關係,.
希望大家能夠回來教會裡面,.
回來小組裡面,.
都能夠有一個回到家的感覺,.
這個是你一個季後鳥裡面,.
一個短暫中途停下的地方,.
能夠不斷來到樂極的地方,.
我自己很喜歡武明裡面的一個角色叫斯力奇,.
斯力奇是一個很有趣的故事,.
他是偶爾會回來的,.
他會飾演一個旅行者的角色,.
吹著一支笛子,.
突然很有型地來到,.
他就像我一樣,.
不太會說話,.
然後回來就站在那裡,.
每次他回到武明谷都是一個回家的過程,.
雖然他一年回來一次,.
不是叫大家一年回來一次,.
但起碼每次回來都有一個真正的家的感覺,.
雖然是一個漂泊的旅程,.
但他每次回到武明谷都是和一班人很好的關係,.
大家一起很有詩意地看月亮,.

$^{641}$然後一起聊天,.
然後第二天他就走了,.
所以這種感覺,.
這種圖畫,.
就是我們所說的,.
我們在一個空明的過程,.
天家是我們真正的家,.
但每個星期你們回來教會裡面,.
戰戰兢兢,.
就像斯力奇回來一樣,.
大家聚在一起,.
那份關係是重要的,.
你不需要將堂會成為你終日毀身,.
不要走,.
鎖死的地方,.
但每次你回來,.
每個星期你回來,.
都是一個這樣的概念,.
所以我們不可以忽略家這個字,.
很有趣的,.
我們有天國這個字,.
我們會將天國稱為天家,.
但我們中國人更加將國,.
country,.
或是state,.
加上了家字,.
明明就國就國,.
為何叫國家呢?.
所以我們發覺,.
我們真的很難抽起那個家字,.
雖然我知道,.
今天我們呼出群體,.
家有很多必然的,.
家是一個回來喝湯的地方,.
未必有一個完整的家庭,.
或是你的家人生活,.
未必是一般的,.
但重點不是模樣或模式,.
而是真正的關係,.
我們希望能夠和弟兄姊妹,.

$^{681}$和你目者,.
有一份真正的家庭關係,.
這是我們Folk Church很想大家擁有的地方,.
甚至乎你是需要去承諾下去,.
我想說,.
我們不是像那些堂會一樣,.
將Folk Church稱為Folk Church家,.
然後你就是這樣貢獻自己進去Folk Church的發展,.
但又相反地說,.
當我們知道Folk Church是一個地上落腳的家,.
我們都可以去毀身貢獻這個家,.
只不過不是那種那麼病態地貢獻,.
所以如果你是海外的弟兄姊妹,.
我們有一首歌叫《默念》,.
是關於一個在外國生活的瑣碎事,.
寫一封家書給他的家人,.
所以這種關係是我們流堂很想擁有的事情,.
無論你在海外還是在香港,.
我們需要在海外建立一份這樣的家庭關係,.
真的像一個家庭一樣的關係..
所以家作為一個動詞,.
教會家,我們是家,.
只不過我們不會忘卻我們流散的身份和使命,.
我們不會將教會家,.
我們唯一的目標,.
而是我們知道我們在世界裡,.
要作見證,.
我們有我們的身份在世界裡,.
所以我覺得,.
今天我和Poon Sir談了很久,.
我們想大家做些什麼呢?.
我想大家其實都是一樣,.
我們Full Church是不是一個家呢?.
我會說Full Church是,.
也可以說不是,.
它是一個真正的家,.
因為你是一個有關係的地方,.
它是一個你和弟兄姊妹有關係的地方,.
它都要求你彼此相愛,.
去為生的地方,.

$^{721}$但我們Full Church不是將這個家,.
視為一個你信仰唯一的目標,.
它是一個像史力奇一樣,.
每次回來都會去貢獻,.
你都會去奉獻,.
你都會去得著力量的地方,.
但不是唯一的地方,.
所以我覺得很想大家,.
將來在組裡面,.
來談談的事情,.
不妨去談談,.
你在Full Church裡面,.
你回到這個地步,.
你覺得你自己是不是,.
覺得在Full Church成為你的家呢?.
我們不會想你去,.
鎖在家裡不讓你走,.
但你是不是一個,.
其中一個的家園,.
來發揮一個弟兄姊妹的身份,.
來為其他人,.
來貢獻自己呢?.
所以我們說,.
大家都很,.
今日這個課題我覺得很重要,.
因為我們Full Church經常都是,.
說真一句,.
如果我們要大家去,.
賣命的話,.
我過兩天都會說的,.
大家都應該很想,.
commit(承諾)下去,.
但我經常都不說的,.
經常都不說,.
大家要怎樣侍奉,.
大家要怎樣做,.
我不說,.
但不代表這個流堂,.
不是我們地上,.
一個這樣的群體,.

$^{761}$我們是需要,.
來思考,.
我們怎樣來將你的生命,.
放一些時間在這裡,.
去怎樣在你小組裡面,.
彼此去建立家呢?.
起碼你的小組是你的家,.
因為這是一個真正的教會,.
真正的群體就在這裡,.
所以想想你怎樣回小組,.
你怎樣來在這個群體裡面,.
成為一個很重要的一員,.
來貢獻自己,.
今日有些不容易結束的地方,.
但希望大家可以在現場節目裡面,.
可以多些來談談,.
你怎樣看流堂作為你的家,.
怎樣看你以前的教會,.
怎樣看我們流散的使命,.
怎樣看我們空明這個概念呢?.
潘Sir應該差不多回來了,.
我先叫他過來,.
他剛剛下機,.
何時吃你煮菜的技巧?.
我煮菜?哦,我好像沒有吃過,.
我以前是煮甚麼,.
我忘記了煮甚麼,.
你煮番茄,.
番茄蛋,.
今日講家這個觀念,.
應該大家都有不少,.
應該可以把空間給多些弟兄姊妹分享,.
因為當時講流堂的弟兄姊妹,.
流堂不是第一個你回教會的,.
所以大家應該有相關的經驗,.
大家可以說一下,.
他收到麥克風了,.
當中聽的內容有沒有甚麼內容,.
覺得跟你以前接收的很不同?.
網上有沒有人問呢?.

$^{801}$有,好像有,.
如果教會是一個家的概念,.
大家每次都是回家,.
但沒有一個固定的教會,.
是否代表每個人都可以多一個家呢?.
這個很好,.
我們可以談談這個問題,.
我們可以有幾個住家,.
他說現在作為家中的小三,.
又應該如何處理這個家的衛生問題,.
躲在衣櫃裡面,.
有人回來,快躲起來,.
我自己在infogroup的時候,.
都說過弟兄姊妹如何回教會,.
其中有一個真實個案,.
不是回答過一次,.
是兩個不同的人都問相關的問題,.
因為他說,.
潘Sir,我一個月回教會兩次,.
我說這麼有趣,.
如果我星期日不需要服事,.
星期六我就會回來flowchurch,.
因為星期日他要服事,.
星期六他要rehearsal,.
所以他就要回教會預習,.
如果他不用服事,.
星期六他就會回來,.
所以一個月他有兩次回來flowchurch,.
有兩次就不回來,.
接著他就問,.
潘Sir,這樣是不是吃兩家茶禮,.
不是這麼好,.
flowchurch是如何看待的?.
我說flowchurch沒有問題,.
你不覺得這樣好像有兩個家不是這麼好嗎?.
接著我說我不覺得,.
我說如果你覺得flowchurch的聚會能夠charge up你,.
或者你能夠在當中敬拜,.
你是能夠在當中得著的話,.
你就繼續可以服事你另外的群體,.

$^{841}$which is我們是服事不到的,.
因為flowchurch的教會觀念是one church的觀念,.
你服事的那群是我們服事不到,.
而你在flowchurch可以得著,.
可以繼續服事其他人的時候,.
這個彼此協作,.
豈不是一個不要單顧自己的事,.
也要顧別人的事的教會觀念做法嗎?.
接著他就明白了,.
他說這樣也不錯,.
但你也要問一問你服事那群堂會的目者,.
如何看待這個做法,.
因為我們是沒有所謂的,.
但有些觀念大家未必是協同的話,.
大家一起去了解會好一點,.
所以好像剛才網上問的問題,.
是不是多一個家呢?.
我們是OK的,.
對我們來說沒有衝突的..
我經常這樣說,.
這個home不是一個固定的概念,.
不如我們把它變成一個動詞,.
不如我們叫做homeing,.
為什麼呢?.
因為我覺得這是流堂一個很特別的地方,.
我們不會叫你不要再回你自己的舞會,.
或者怎樣脫手,.
但就是回到這裡,.
你侍奉在這裡,.
全部都在這裡,.
我們不會像以前的教會一樣,.
這個就是你的家,.
所以你不要離開,.
只把你的信仰實踐就算了..
我們很強調,.
家是一種流動的概念,.
我們現代人都是這樣,.
我們現在還有一個現代人,.
年輕人,都市人,.
我們的家是怎樣呢?.

$^{881}$很多時候都不是這麼簡單,.
早餐吃一個有雞蛋,香腸,很經典的早餐,.
然後上學,上班,回來,爸爸媽媽都在,.
一個這麼經典的家,.
但家是一個我們能夠信任的地方,.
一個可以承載自己的地方,.
所以這件事我覺得是有的,.
應該有的,.
流堂這個地方,.
我們的小組,.
特別強調這個小組,.
就是希望能夠建立到一種關係的地方,.
它可以是很多不同的外揚,.
即是形式,.
但當中的東西是一樣的,.
有一次我喜歡一首歌,.
不知道大家有沒有聽過,.
就是陳以貞的家這首歌,.
如果你聽過的話,.
大家回去搜尋一下,.
有一首歌叫家,.
歌詞寫得很好,.
輕輕牽著你的手,.
慢慢長路一直走,.
哪裡都是我們的家,.
我覺得只要牽著那個人的手,.
我們一起走,.
哪裡都是我們的家,.
所以我想這件事就是這樣,.
我們流堂的那個家,.
不是一個很簡單的堂會機構,.
你不斷去發大財來搞,.
而是這裡的人,.
可以有很多不同的關係,.
可能在網上,.
可能是不同的身份,.
重點不是有多少個家,.
不是有多少個住家,.
而是你在這裡能夠有一個.
很重要的家庭關係,.

$^{921}$有真正家庭的人,.
和你一起走這個地方,.
你可能好像史力奇一樣,.
偶爾你就離開這個地方一會兒,.
但這種關係是重要的,.
當然更加理想,.
可以去承諾這份關係下去,.
不只是拿走,.
所以這就是我們嘗試去強調,.
我們不想鎖死你,.
但這也是你要去承諾的地方,.
因為這正是一個真實的彼此相愛的群體,.
一個家的關係,.
所以我想大概是這樣,.
我們重點不是有多少個住家,.
而是當你來到這裡的時候,.
你不要只是拿東西走,.
而是真實關係的,.
一個彼此建立的身份..
有沒有問題?.
在後面..
剛才聽了也很好,.
兩位牧者都很大方,.
覺得全聖教的人可以很自由地,.
不需要完全全聖教的侍奉,.
或者必須要留下來,.
但全聖教也有家務,.
回看其他堂會,.
他們也會困著大家,.
我們一起來聚會,.
星期三我們就祈禱會,.
星期四就崇拜,.
練師,.
有很多東西玩,.
很多東西搞,.
你們又會怎樣看這些衡量?.
我們想你自動自覺,.
家務就是這樣,.
大家都知道小時候爸爸媽媽都叫做家務,.
要麼就哭哭啥,.

$^{961}$要麼就用很多不同的方法,.
威逼理由,.
但其實一個家庭就應該自動自覺,.
大家很舊的電影,.
以前那套電影,.
陀奧斯汀下海,.
陀奧斯汀在劇集,.
大家雖然不是這樣的一家人,.
但大家的自動自覺,.
去埋伏,.
去貢獻,.
就構成了一個家,.
我要去勒索你,.
你回來一定要怎樣做,.
不然你就不是家人,.
是沒有意思的,.
只是很多破碎的家庭就是這樣,.
但真正的家就是不用出聲,.
潘Sir也說我們預備課程,.
很多時候很多掙扎,.
你問就對了,.
我們不會主動叫你做家務,.
但家務當然要做,.
我們想你自動自覺做,.
不是想你不做就不能回來,.
不做就不是家人,.
最美麗的家庭就是大家自動自覺回來,.
不成功就埋伏,.
就抹掉,.
這才是一個理想的家,.
因為我們成長的家就是這樣,.
是一個底子,.
主動扶持的地方,.
我和弟兄姊妹回應,.
提問這些過程當中,.
其實家這個觀念在過去的堂會是很好用的,.
譬如家用,家規,.
大家有功能性去做到那件事,.
是好用的,.
我剛才說的轉堂會,.

$^{1001}$剛才John說的台灣或香港都很不容易,.
通常你轉教會離不開那兩至三個原因,.
第一是你嫁了或娶了,.
跟了人去,.
你帶著祝福他,.
然後你去讀神學,.
他都期望你會回來,.
然後你搬了轉教會,.
香港有多大?.
只坐一個小時車,.
不是很遠,.
你離開了,.
沒有做家務,.
家務原本你做,.
現在要找人去頂,.
這些地方是不容易協調的,.
或者會令到像現在這樣說的話題,.
會有些爭端,.
但我們都希望讓弟兄姊妹說,.
疫情下其實正正就是顯了很多東西,.
這三年香港教會在疫情下,.
其實沒有了很多聚會,.
很多功能侍奉組別都停工了,.
教會會不會死呢?.
其實教會真的不會死,.
教會只要維持崇拜,.
有一個牧養的過程當中,.
大家每個星期在崇拜當中有一個synergy,.
其實教會是繼續運作到,.
但突顯了一件事,.
就是你沒有了那些服侍,.
或者沒有了那些崗位之後,.
你發覺可能跟教會沒有什麼network,.
就是沒有什麼聯繫,.
突顯了一件事,.
教會過去可能著重很多functional approach,.
或者functional domain的東西,.
而忽略了relational domain,.
關係,聯繫那件事,.
我不知道你過去跟弟兄姊妹,.

$^{1041}$或者牧者之間的關係是怎樣,.
很多時候你看回check history,.
很多都是功能性,.
交代你什麼時候回來,什麼時候合作,.
其實靠近的東西不是很多,.
交心的東西也不是很多,.
但疫情突顯了,.
我們再重新去想一下,.
其實教會take out了這些功能性的東西,.
其實剩下什麼呢?.
空名最重要,帶出一個訊息,.
就是那種關係最純粹的,.
所以fortune就真的,.
fortune是不會沒有事情做的,.
教會也沒有事情做,.
但重點就是那些事情是否必要做,.
但如果大家一起見到需要的話,.
一起做就是最賞心,.
我想大家都有做家務,.
很難說,.
但家務這件事就是,.
你做不覺,不做就很覺,.
不做就很骯髒,.
但每人做一點點就是做了那件事,.
所以好像說得很遠,.
不過最重要是大家投心,.
這樣就ok了..
後面那邊有一個,前面有一個,.
後面先可以,.
Hello各位,.
我是在海外年紀的,.
我是昨天才回來的,.
我回來之前就很開心,.
收到訊息有今天的課,.
覺得很合聽,.
我可以分享一下,.
我在海外的年紀,.
我覺得我在英國的時候,.
我不覺得那裡是教,.
在周邊都是很多外國人的樣子,.

$^{1081}$很少人說廣東話,.
會覺得我究竟去了哪裡,.
很飄泊的感覺是有的,.
還有跟香港的時差,.
基本上你想找朋友,.
或者等於知道有些困難,.
就算我留個訊息給他,.
可能他第二天才回覆,.
我覺得在這裡,.
個人很不穩定,很不紮緊,.
剛去到又不是認識了很多朋友,.
初初我有回教會,.
後來覺得教會不適合,.
去了兩個月就沒有回,.
一直沒有再玩教會,.
所以覺得那裡人很飄泊,.
剛才我聽的時候,.
其中有一個powerpoint是說,.
homing是guiding to a destination,.
那一刻給我一種安慰,.
無論我在海外,.
或者現在我回到香港,.
其實我回來的時候,.
我都有想,.
究竟我回來香港是旅行,.
還是回家呢?.
我都有這樣的疑問,.
Facebook有很多人留言給我,.
說你是回家的,.
我調整了心態,.
對,我也可能是回家的,.
但剛才那句guiding to a destination,.
開始給我一個啟發,.
就是我不需要著重,.
我究竟在哪個地方生活,.
我最終的目標就是天價,.
我覺得這樣想對我來說會舒服些,.
因為我覺得我去年的時候,.
我經常都很想在英國找一個家的感覺,.
我想除了我沒有回教會,.

$^{1121}$認識的人不多,.
剛才John說,.
回教會是跟別人有一個關係,.
我覺得我在去年沒有跟別人有關係,.
所以我不覺得那個地方是一個家,.
雖然我那裡有工作,.
有我自己同事,.
但你只可以說那些是朋友,.
不像以前在flowchurch少了的時候,.
你可以分享自己的事,.
而對方的feedback,.
就算回教會崇拜,.
大家也會約會,.
或者之前會一起吃東西,.
你會覺得跟這些人有關係,.
那家的感覺會重些,.
我這次回來一兩天,.
除了約朋友,.
也很感恩flowchurch,.
幫我聯繫一下弟兄姊妹,.
他們也很熱烈,.
也會出來跟我吃飯,.
我就開始覺得,.
好像找回家人一樣,.
我今天再進來這個地方,.
我覺得感覺很熟悉,.
以前在這裡開小組,.
可能我分享得不是很好,.
不過我覺得很明白,.
你要跟別人有一個關係,.
也有一個家的感覺,.
多謝你分享..
現在,尤其在教會,.
對於多數講家的觀念很重,.
通常有些門檻可能很小,.
可能要求班足一年穩定的聚會,.
才可以讓你洗禮,.
進來做一些事夢,.
去擔任一些崗位,.
要洗禮後才可以做,.

$^{1161}$其實以前一直流傳,.
這些傳統究竟應該怎樣面對呢?.
剛才也說了,.
我們沒有一個固定的教會,.
其他是怎樣呢?.
我們在流動的過程中,.
我想你的問題,.
我會分開來說,.
他回家多久,.
跟他工作和洗禮,.
我就不會找唯一的,.
例如你剛才說洗禮後才可以擔任崗位,.
以flowchurch來說,.
就不要說其他,.
flowchurch也不會一加入flowchurch,.
回了flowchurch就立即工作,.
對我們來說,.
以洗禮為例,.
要加入小組才會跟他進行洗禮,.
原理就是,.
我們覺得洗禮不是一個魔法,.
洗禮是一種關係的建立,.
讓他有牧羊根進,.
跟他開展熟齡新的一頁,.
跟侍奉是不掛勾的,.
洗完禮他也不一定要一人一侍奉,.
所以我想回應你,.
工作不是因為洗禮的門檻,.
工作也不是因為洗禮後續要有什麼委身,.
我想釐清這個沒有連帶關係,.
至於關於多久可以工作,.
我們沒有時限,.
對我們來說,.
他是否看到需要,.
是否願意委身才一起參與,.
不是你剛才的思路或想法,.
好像要有門檻做了才參與,.
你明白我的意思嗎?.
(何君堯:如果我們回想起什麼角度看呢?).
這不是Flow Church的想法,.

$^{1201}$不是侍奉和洗禮掛勾,.
你不洗禮就有多厲害也不做,.
不是這些思路想法,.
(何君堯:剛才有舉手).
我來Flow Church沒多久,.
其實這些主題,.
剛才說的客女或回家,.
都是很傳統到現代也有的討論主題,.
不過時間關係,.
只分享少少,.
關於教會或Flow Church,.
剛好過去一年或幾個月,.
經歷了很多更新,.
以前我比較傳統,.
不喜歡弟兄姊妹不每個星期回來,.
以前我是這種,.
覺得很辛苦,.
為何人們當是Staycation,.
不理解,.
後來我自己也受不了,.
我完全不回來,.
因為太多事情處理不了,.
然後放下了,.
最後我覺得神令我明白,.
祂想我因為祂的我而回來,.
不要因為被迫要回來,.
不要覺得我不回來就很頑皮,.
而是一種我想回來,.
和其他人一起敬拜神,.
想和大家一起那樣東西而回來,.
所以我覺得家是,.
如果我想回來就有家的感覺,.
如果被迫就沒有,.
另外一件事是,.
有時可能靈唱的弟兄姊妹,.
或是講道,.
當她說一些信念我很認同時,.
其實也會有家的感覺,.
因為有時人數多的church,.
很難認識這麼多人,.

$^{1241}$或是很難和大家很熟,.
但如果台上的人說的信念很認同,.
其實也會有一種連繫感,.
就是這樣..
我想大家很值得看一套日劇,.
同一承下,.
我想我們劉唐想做的家是這樣的,.
不是傳統中國家春秋那些,.
家庭成為一種捆索,負擔,政治,.
一種包袱,責任,.
而是好像同一承下一樣,.
大家自願自負成為一個家庭,.
是一種自由,.
但在自由之下又願意付出,.
這個我們正想營造,.
不容易的,.
因為需要大家一起參與付出,.
才成為同一承下,.
早餐法子加起來,.
大家一起付出,.
去關心,.
才成為這樣的家,.
而不是一種傳統教人吃不消的家庭..
就是這樣..
有沒有其他對於家的想法,觀念,.
或者大家對於投入家的程度,.
有不同的分享?.
如果沒有的話,.
如果你收看我們YouTube的話,.
歡迎你,.
不是歡迎,.
我們都是在小組裡面,.
繼續分享,.
這個也是一個很重要的題目,.
希望你們都能夠分享,.
你怎樣理解自己的教會生活,.
自己對於群體的貢獻,.
還有看同一承下,.
這是一個很好看的日劇..
我們就來吧,.

$^{1281}$下個月再見吧..
今天YouTube也有很多人留意..
是啊..
好吧,.
我們就回家吧..
好吧,.
再見..
謝謝大家.
\newpage



\section{馬太福音 25:1-13-20230603}
\label{sec:40Zpw7rWZSQ}
\textbf{【網上崇拜】【arm beat 慢歌版】一生守候|馬太福音25\_1-13|20230603 [40Zpw7rWZSQ]}
\newline
\newline
連結: \href{https://youtube.com/watch?v=40Zpw7rWZSQ}{\texttt{ https://youtube.com/watch?v=40Zpw7rWZSQ}} ~~~~ 語音日期: 2023-06-03 
\newline
\newline
\hyperref[sec:RYCxV16hfwM]{\small{< < < PREV SERMON < < <}}
~
\hyperref[sec:index_chronic]{\small{[返順時目]}}
~
\hyperref[sec:index_scriptual]{\small{[返順卷目]}}
~
\hyperref[sec:E2mqQjkFX2s]{\small{> > > NEXT SERMON > > >}}
\newline
\newline
馬太福音 25:1-13-20230603
\newline
\begin{longtable}{cl}
\hline
\hline
章節 & 經文 (和合本修訂版)\\
\hline
25:1 & \begin{tabularx}{0.7\textwidth}{X} 「那時,天國好比十個童女拿著燈出去迎接新郎。 \end{tabularx} \\ \\ \relax
25:2 & \begin{tabularx}{0.7\textwidth}{X} 其中有五個是愚拙的,五個是聰明的。 \end{tabularx} \\ \\ \relax
25:3 & \begin{tabularx}{0.7\textwidth}{X} 愚拙的拿著燈,卻沒有帶油; \end{tabularx} \\ \\ \relax
25:4 & \begin{tabularx}{0.7\textwidth}{X} 聰明的拿著燈,又盛了油在器皿裡。 \end{tabularx} \\ \\ \relax
25:5 & \begin{tabularx}{0.7\textwidth}{X} 新郎遲延的時候,她們都打盹,睡著了。 \end{tabularx} \\ \\ \relax
25:6 & \begin{tabularx}{0.7\textwidth}{X} 半夜有人喊:『看,新郎來了,你們出來迎接他。』 \end{tabularx} \\ \\ \relax
25:7 & \begin{tabularx}{0.7\textwidth}{X} 那些童女就都起來挑亮她們的燈。 \end{tabularx} \\ \\ \relax
25:8 & \begin{tabularx}{0.7\textwidth}{X} 愚拙的對聰明的說:『請分點油給我們,因為我們的燈要滅了。』 \end{tabularx} \\ \\ \relax
25:9 & \begin{tabularx}{0.7\textwidth}{X} 聰明的回答:『恐怕不夠你我用的;你們還是自己到賣油的那裡去買吧。』 \end{tabularx} \\ \\ \relax
25:10 & \begin{tabularx}{0.7\textwidth}{X} 她們去買的時候,新郎到了。那預備好了的,與他進去共赴婚宴,門就關了。 \end{tabularx} \\ \\ \relax
25:11 & \begin{tabularx}{0.7\textwidth}{X} 其餘的童女隨後也來了,說:『主啊,主啊,給我們開門!』 \end{tabularx} \\ \\ \relax
25:12 & \begin{tabularx}{0.7\textwidth}{X} 他卻回答:『我實在告訴你們,我不認識你們。』 \end{tabularx} \\ \\ \relax
25:13 & \begin{tabularx}{0.7\textwidth}{X} 所以,你們要警醒,因為那日子,那時辰,你們不知道。」 \end{tabularx} \\ \\ \relax
25:14 & \begin{tabularx}{0.7\textwidth}{X} 「天國又好比一個人要出外遠行,就叫了僕人來,把他的家業交給他們。 \end{tabularx} \\ \\ \relax
25:15 & \begin{tabularx}{0.7\textwidth}{X} 他按著各人的才幹,給他們銀子:一個給了五千,一個給了二千,一個給了一千,就出外遠行去了。 \end{tabularx} \\ \\ \relax
25:16 & \begin{tabularx}{0.7\textwidth}{X} 那領五千的立刻拿去做買賣,另外賺了五千。 \end{tabularx} \\ \\ \relax
25:17 & \begin{tabularx}{0.7\textwidth}{X} 那領二千的也照樣另賺了二千。 \end{tabularx} \\ \\ \relax
25:18 & \begin{tabularx}{0.7\textwidth}{X} 但那領一千的去掘開地,把主人的銀子埋藏了。 \end{tabularx} \\ \\ \relax
25:19 & \begin{tabularx}{0.7\textwidth}{X} 過了許久,那些僕人的主人來了,和他們算賬。 \end{tabularx} \\ \\ \relax
25:20 & \begin{tabularx}{0.7\textwidth}{X} 那領五千的又帶著另外的五千來,說:『主啊,你交給我五千。請看,我又賺了五千。』 \end{tabularx} \\ \\ \relax
25:21 & \begin{tabularx}{0.7\textwidth}{X} 主人說:『好,你這又善良又忠心的僕人,你在少許的事上忠心,我要派你管理許多的事,進來享受你主人的快樂吧!』 \end{tabularx} \\ \\ \relax
25:22 & \begin{tabularx}{0.7\textwidth}{X} 那領二千的也進前來,說:『主啊,你交給我二千。請看,我又賺了二千。』 \end{tabularx} \\ \\ \relax
25:23 & \begin{tabularx}{0.7\textwidth}{X} 主人說:『好,你這又善良又忠心的僕人,你在少許的事上忠心,我要派你管理許多的事,進來享受你主人的快樂吧!』 \end{tabularx} \\ \\ \relax
25:24 & \begin{tabularx}{0.7\textwidth}{X} 那領一千的也進前來,說:『主啊,我知道你,你是個嚴厲的人:沒有種的地方也要收割,沒有播的地方也要收穫, \end{tabularx} \\ \\ \relax
25:25 & \begin{tabularx}{0.7\textwidth}{X} 我就害怕,去把你的一千銀子埋藏在地裡。請看,你的銀子在這裡。』 \end{tabularx} \\ \\ \relax
25:26 & \begin{tabularx}{0.7\textwidth}{X} 他的主人回答他說:『你這又惡又懶的僕人,你既知道我沒有種的地方也要收割,沒有播的地方也要收穫, \end{tabularx} \\ \\ \relax
25:27 & \begin{tabularx}{0.7\textwidth}{X} 就該把我的銀子放給兌換銀錢的人,到我來的時候可以連本帶利收回。 \end{tabularx} \\ \\ \relax
25:28 & \begin{tabularx}{0.7\textwidth}{X} 把他這一千奪過來,給那有一萬的。 \end{tabularx} \\ \\ \relax
25:29 & \begin{tabularx}{0.7\textwidth}{X} 因為凡有的,還要加給他,叫他有餘;沒有的,連他所有的也要奪過來。 \end{tabularx} \\ \\ \relax
25:30 & \begin{tabularx}{0.7\textwidth}{X} 把這無用的僕人丟在外面黑暗裡,在那裡他要哀哭切齒了。』」 \end{tabularx} \\ \\ \relax
25:31 & \begin{tabularx}{0.7\textwidth}{X} 「當人子在他榮耀裡,同著眾天使來臨的時候,要坐在他榮耀的寶座上。 \end{tabularx} \\ \\ \relax
25:32 & \begin{tabularx}{0.7\textwidth}{X} 萬民都要聚集在他面前。他要把他們分別出來,好像牧人分別綿羊、山羊一般, \end{tabularx} \\ \\ \relax
25:33 & \begin{tabularx}{0.7\textwidth}{X} 把綿羊安置在右邊,山羊在左邊。 \end{tabularx} \\ \\ \relax
25:34 & \begin{tabularx}{0.7\textwidth}{X} 於是王要向他右邊的說:『你們這蒙我父賜福的,可來承受那創世以來為你們所預備的國。 \end{tabularx} \\ \\ \relax
25:35 & \begin{tabularx}{0.7\textwidth}{X} 因為我餓了,你們給我吃;渴了,你們給我喝;我流浪在外,你們留我住; \end{tabularx} \\ \\ \relax
25:36 & \begin{tabularx}{0.7\textwidth}{X} 我赤身露體,你們給我穿;我病了,你們看顧我;我在監獄裡,你們來看我。』 \end{tabularx} \\ \\ \relax
25:37 & \begin{tabularx}{0.7\textwidth}{X} 義人就回答:『主啊,我們甚麼時候見你餓了,給你吃;渴了,給你喝? \end{tabularx} \\ \\ \relax
25:38 & \begin{tabularx}{0.7\textwidth}{X} 甚麼時候見你流浪在外,留你住;或是赤身露體,給你穿? \end{tabularx} \\ \\ \relax
25:39 & \begin{tabularx}{0.7\textwidth}{X} 又甚麼時候見你病了,或是在監獄裡,來看你呢?』 \end{tabularx} \\ \\ \relax
25:40 & \begin{tabularx}{0.7\textwidth}{X} 王回答他們說:『我實在告訴你們,這些事你們做在我弟兄中一個最小的身上,就是做在我身上了。』 \end{tabularx} \\ \\ \relax
25:41 & \begin{tabularx}{0.7\textwidth}{X} 「王又要向那左邊的說:『你們這被詛咒的人,離開我!進入那為魔鬼和他的使者所預備的永火裡去! \end{tabularx} \\ \\ \relax
25:42 & \begin{tabularx}{0.7\textwidth}{X} 因為我餓了,你們沒有給我吃;渴了,你們沒有給我喝; \end{tabularx} \\ \\ \relax
25:43 & \begin{tabularx}{0.7\textwidth}{X} 我流浪在外,你們沒有留我住;我赤身露體,你們沒有給我穿;我病了,我在監獄裡,你們沒有來看顧我。』 \end{tabularx} \\ \\ \relax
25:44 & \begin{tabularx}{0.7\textwidth}{X} 他們也要回答:『主啊,我們甚麼時候見你餓了,或渴了,或流浪在外,或赤身露體,或病了,或在監獄裡,沒有伺候你呢?』 \end{tabularx} \\ \\ \relax
25:45 & \begin{tabularx}{0.7\textwidth}{X} 王要回答:『我實在告訴你們,這些事你們沒有做在任何一個最小的弟兄身上,就是沒有做在我身上了。』 \end{tabularx} \\ \\ \relax
25:46 & \begin{tabularx}{0.7\textwidth}{X} 這些人要往永刑裡去;那些義人要往永生裡去。」 \end{tabularx} \\ \\
[1ex]
\hline
\hline
\end{longtable}
$^{1}$各位兄弟姐妹晚安.
今天我們很慢地說話.
一個月前我們在中環碼頭崇拜.
在Armbeat裡面的第一個.
不要太低了.
一個月前我們在中環碼頭崇拜.
Armbeat崇拜.
所以我說第一篇到.
我們會唱Everywhere.
其實我很想說這篇.
因為我覺得Armbeat應該是慢歌.
我今天穿同一件衣服.
同一句說話.
我覺得Armbeat不一定要很快.
Armbeat應該是慢才行.
如果你想想你要去Armbeat.
應該怎樣做.
Armbeat你要做甚麼.
基本上你要做幾個步驟.
第一步驟.
當你發現自己錯了節拍.
你就停下來.
無論是隔樂器還是音樂.
第一步驟.
停下來自己錯了節拍.
第二步驟.
你就去聆聽對節拍的聲音.
第三步驟是甚麼.
一個很重要的第三步驟.
就是你要等一等.
等到某個位置去進入.
然後我就彈奏.
其實等待我想說Armbeat的課題.
當我們很想去arm 上帝的別.
我們這八個星期裡說Armbeat.
很重要的一定要說的課題.
就是我們要去等上帝的別.
然後去同步他的別.
所以如果世上有一樣東西.
叫做上帝的別的時候.

$^{41}$我們一定要去學習.
等待這件事情.
我們會讀一段經文.
我們會看一段經文.
我們是不是有的.
《馬達福音》二十五章.
一到十三節.
我們一起拿起手機.
打開你的手機.
我們一起讀聖經.
《馬達福音》第二十五章.
一到十三節的經文.
我們一起讀.
預備一二三.
那時天國好比十個童女拿著燈.
出去迎接新郎.
其中有五個是愚拙的.
五個是聰明的.
愚拙的拿著燈卻不預備油.
聰明的拿著燈又預備油再氣命裡.
新郎言辭的時候.
他們都打盹睡著了.
半夜有人喊著說.
新郎來了你們出來迎接他.
那些童女就都起來收拾燈.
愚拙的對聰明的說.
請分級油給我們.
因為我們的燈要滅了.
聰明的回答說.
恐怕不夠你我用的.
不如你們自己到賣油的那裡去買吧.
他們去買的時候.
新郎到了.
那預備好了的.
和他進去坐直.
忙在關了.
其餘的童女隨後也來了.
說主啊主啊給我們開門.
他卻回答說.
我實在告訴你們.

$^{81}$我不認識你們.
所以你們要警醒.
因為那日子那時辰你們不知道.
我們一起祈禱.
天父我們求你幫助我們.
我們等候你的說話.
等候你的聖靈在當中.
是子不配.
但是成為你.
不過傳遞你訊息的器皿.
求主你在當中.
你的靈在當中.
和我們現場的人.
在網絡裡的人.
在YouTube裡的人.
我們一起去領受你的說話.
我們靜默等候.
求主你今晚和我們同在.
防止命中.
我們會說這段.
十個童女的經文.
我想大家都很熟悉經文.
十個童女拿著燈.
去等候新郎的來臨.
裡面有五個是愚蠢的.
五個是聰明的.
當他們拿著燈在門外等的時候.
等到新郎來之前.
我這樣想.
應該是五個蠢的一邊.
五個聰明的一邊.
我這樣想.
五個蠢的一邊.
五個聰明的一邊.
誰知新郎到半夜還沒到.
蠢的沒有預備多餘的油.
聰明的有.
因此當新郎來的時候.
餘下的沒有油.
所以就無法進入現寂的裡面.

$^{121}$在外面哀哭自恥.
這就是我們一向所認識的.
十個童女的故事.
我們以前的教導是.
我們要學習做聰明的.
不要做蠢的.
我們要去預備自己油.
我們的燈需要油.
求諸相似下.
我不知道為什麼他會這樣說.
什麼求諸相似下.
今天才知道原來是次下.
諸事似來.
我會說一個十個童女的解讀.
一個不太傳統的十個童女的解讀.
當然今天的解讀不是唯一的.
不需要覺得我對的都錯了.
不過是很值得大家細心思考的解讀.
上星期我和我小祖查經就查了這段經文.
我們發現這個比喻原來有很多鬼的地方.
很多疑點很多破綻.
很多不尋常的地方我們平常是沒有發現的.
不信我們一起研究一下.
我們一起打開大家的聖經看看.
首先這個比喻裡面有五個蠢的童女.
五個聰明的童女.
就很不尋常.
你很少可以看到五個蠢的人站在一起.
哪有這樣的.
五個蠢的五個聰明的.
這裡都沒有.
這裡全部都是聰明的.
五個蠢人不會坐在一起.
我不覺得會坐在一起.
我不知道五個蠢是什麼樣子.
五個蠢的五個聰明的一起共事.
我想起一套戲.
正義回廊.
當我想起聰明的童女和蠢的童女的時候.
我就想起正義回廊裡面的阿倫和肥仔.

$^{161}$大家想一下.
想像一下阿倫和肥仔古裝版的女人版的畫面.
你給我一點油吧.
我沒有問題的.
就是這樣的劇段.
蠢的童女就是這樣.
所以我們會說一個正義回廊.
十個童女版的故事.
其實我不知道你們有沒有發現.
這個比喻裡面的聰明和蠢是沒有關係的.
我不知道為什麼要安排這個前設.
首先我們不會代表說人蠢.
因為這件事情是很沒禮貌的.
他們的差別其實都不在乎於聰明和蠢.
他們是一個童女.
童女是有預備有和沒有預備有.
你可以說是有預備和沒有預備.
或者是很勤力和很懶.
或者是很細心或者是不細心.
其實這個比喻裡面不是很關於聰明和蠢的事情.
所以我問究竟這五個童女蠢在哪個地方呢.
另外那五個童女聰明在什麼地方呢.
我想了很久後來我想到了.
就在這一節裡面.
在第六節裡面.
我一直在看第六節.
是這樣記載的.
第六節半夜有人在哭著說.
新郎半夜就回來了.
新郎來了你們出來迎接他.
那些童女都起來收拾燈.
如說的對聰明的說.
你給我一點油吧.
我沒有問題的.
就是這樣.
然後我們就看一下那些聰明的童女.
怎樣來回答.
阿倫給你油.
我怕我那些油都不夠用.
不如你們自己去賣油的地方買吧.

$^{201}$那我去樓下買吧.
這樣就下去買了.
大家有沒有發現.
半夜三更有人去樓下買油.
你說聰明都沒有聰明.
大家都知道五更保濕有幾點.
五更保濕七點.
你叫那些蠢的半夜三更出去買油.
你說他們是不是很聰明.
那五個是不是很蠢.
新郎要來了.
這個時候就點了五個蠢的去買油.
半夜三更很明顯是想陰他們.
後來新郎就來了.
新郎來的時候蠢的不見人.
聰明的就可以進去.
蠢的就沒有辦法進去燃燒的營.
所以我想說.
如果你留意時間的話.
其實蠢的童女不能夠進燃燒營.
其實不關點燈的事.
他們出去買東西.
你會發現新郎從來都沒有責怪他們的燈沒有油.
他們只不過是新郎來的時候他們不在.
所以這五個聰明的童女真的很聰明.
所以你看多TVB那些.
延禧攻略那些.
公心計就明白這個比喻多一點.
整個十個童女B是一個宮女之間的鬥爭.
大家在陰來陰去.
當然這是我自己的解讀.
不一定是對的.
不過以下我要說的.
就真的是鐵證如山.
證據各作.
有聖經經文支持的東西.
我說的是其實這十個童女.
在故事裡面十個都睡著了.
十個都睡著了.
這十個童女無論是聰明的蠢的.

$^{241}$當新郎來之前全部都睡著了.
你以為聰明的沒有睡覺嗎.
你以為聰明的最好蠢的就不好嗎.
不是的.
無論是聰明的還是蠢的.
他們都沒有好好在門外等他們的主人睡著了.
所以這五個聰明的童女其實沒有你想像中那麼好的.
這件事他們兩個都有份的.
法官大人.
我們太過習慣.
我們習慣歌頌這五個聰明的童女.
我們要做聰明的童女.
不要做愚蠢的童女.
因為聰明的童女可以進入賢直.
甚至賭果為因.
因為這五個聰明的童女都睡著了.
所以反過來就說.
其實睡覺是可以的.
因為聰明的童女都睡著了.
他們會這樣想的.
但睡覺是不可以的.
十個童女比喻裡面.
耶穌教什麼.
耶穌有什麼教導.
不用猜耶穌有講的.
最後一節就是十三節.
耶穌清清楚楚的講出這個比喻的解話.
是要教什麼的.
所以你們要警醒.
因為那日子那時辰你們不知道.
什麼叫做警醒.
Awake.
不要睡覺.
所以耶穌沒有叫你去歸邊做哪一種的童女.
不是教你入天國的秘技.
耶穌只是告訴你.
天國這回事.
你是需要警醒等候.
不要睡覺.
其實在眾多的耶穌的教導裡面.

$^{281}$甚至在整本新約裡面.
睡覺都不是一件好事.
在路加的另一個版本裡面.
路加第十二章三十五節.
如果大家有興趣看的話.
類似十個童女的比喻.
就是僕人要求主人.
在婚宴的筵席回來的時候.
就站在門外等他.
耶穌說什麼.
你們腰要束上帶.
燈也要點著.
自己好像僕人.
等大號主人從婚宴筵席回來.
他來到叩門就立刻叩開門.
主人來了看見僕人警醒.
那僕人就服了.
這是另一個平衡版本.
類似這樣的教導.
都是叫你們不要睡覺.
赫西瑪利園也是一樣.
門徒當他在赫西瑪利園睡著了.
耶穌怎麼說.
耶穌怎麼教導他們.
耶穌跟他們說不要睡覺.
你們要警醒禱告.
免得入了迷幻.
基本上聖經每一次出現警醒這個字的時候.
都是跟睡覺有關係的.
赫西瑪利園前出的五章第六節.
我們不要睡覺像別人一樣.
總要警醒緊守.
所以十個童女的比喻裡面.
十個童女都沒有好好地站在門外等候主人.
雖然聰明的童女很聰明.
有預備燈所需要的油.
但這個比喻不是純粹留於.
教你如何進入她的現實裡面.
這個比喻不是純粹告訴你.
哪些人可以進去.

$^{321}$哪些人不能進去.
你怎麼進去.
這是個人得救觀念的看法.
如何進入她的天國裡面.
不是這麼簡單.
或者這個比喻告訴你.
天國其實是一件什麼事情.
上帝的工作其實是一件什麼事情.
你要用什麼節奏去跟隨天國的節奏.
所以今天我們會說等候這個題目.
十個童女的比喻是一個有關等候的比喻.
等候是一件痛苦的事情.
沒有人願意等.
我自己也不是一個喜歡等待的人.
認識的人都知道.
我是一個比較行動進取的人.
我自己的呼召.
去德國讀書.
我開Foodchurch.
基本上都沒有很多猶豫的地方.
基本上都知道就去.
都沒有什麼等.
快和慢我選擇快.
所以我開電單車是很開心的.
因為我可以走紅隧.
我由灣仔走到紅隧.
不斷地走很多輛車就可以走.
無論你開Tesla都可以讓我走.
不過人長大了.
做人做久了.
你開始發現等候的重要.
我不是說等有多好.
其實等一點都不好.
人生有很多無數慘痛的經歷.
告訴你不等有多不好.
不等你就需要付上不等待的代價.
我都說我是一個很進取的人.
所以我在Foodchurch有很多新的想法.
很多新的事情想做.
潘Sir是一個將事情放慢的人.

$^{361}$他不是不做.
他不是等波中.
他真的將事情慢慢落實去做.
他跟我說先通通氣.
在WhatsApp群組先通通氣.
人事疏通了就去吧.
等待不是一件好事.
正所謂早買早享受.
錢買平幾舊.
但上帝的事從來都不能讓你早買早享受.
用股票的術語來說.
你過早入市.
你會輸得很慘.
摺飛刀.
股市不斷跌.
心急撈底就輸死你.
不過我想說.
等待上帝最難的地方不是這個.
問題是我們怎樣等.
我們很多等節目都知道.
我們今天要等上帝.
其實你沒有去等上帝.
你知道等上帝.
其實你未必有真正去正確地去等上帝.
今天我想說這件事.
我們太多廉價的等待.
因為我們等人是不需要等人的.
世界上有很多發明.
有很多科技.
有很多產品.
去幫你代替等人的時間.
等一個人.
你從來都不需要等一個人.
我說什麼呢.
現在的時代不同了.
基本上你從來去等人.
你是不需要真的等.
30年前我怎樣約人等人.
30年前還沒有手機.
連代表我上網都怎樣等人.

$^{401}$我們說明天晚上7點半.
尖沙咀碼頭.
五字旗桿椅.
很準確的.
明天晚上7點半.
尖沙咀碼頭.
五字旗桿椅.
時間地點很清楚.
因為以前沒有手機.
你一定要在那裡.
正確地說就等.
你只能夠等.
去等的時候你都沒有手機拿出來玩.
你最多拿本書出來看.
現在怎樣等人.
明天晚上尖沙咀.
完.
尖沙咀你住完嗎.
以前我們是這樣說的.
尖沙咀你住完嗎.
尖沙咀在哪裡.
幾點.
我們不會說.
去到差不多了.
晚上差不多六點多五點.
才開始跟進那件事.
尖沙咀在哪裡.
幾點.
即使你今天去到真的等.
你都不會真的等.
你會拿手機出來.
你會去看IG.
你會去打機.
你會去福泊室.
你會去訂機票.
去occupy你的時間.
你只是在做你自己的事.
順便去等那個人.
所以我稱之為廉價的等待.
今天我們等待.

$^{441}$其實我們沒有好好地去等待.
你沒有等待一個人.
所以我都說.
姐妹有個女兒說.
我會等你十年.
這句是什麼說話.
是不是站在那裡等十年.
你會做很多事.
所以.
《聖經》裡面說.
世上最有效.
最biblical的等待方法.
其實是什麼.
就是睡覺.
世上最好去度過時間的方法.
就是睡覺.
你坐飛機.
你去英國機場最好.
凌晨一點鐘機.
你去到吃完晚餐.
喝杯紅酒就睡.
然後睡醒吃早餐就到了.
基本上是一個最好的時間.
你睡醒就會到英國.
所以《聖經》這麼說.
眾門徒都睡著了.
然後耶穌就說.
你們為什麼不能和我警醒片刻呢.
你只需要開著你的sleep mode的時候.
你就沒有任何的等待的成本.
就算是科幻片也是.
只要你等待一個.
開啟你的sleep mode.
幾千萬光年距離.
就這樣睡過去.
我想說的是.
十個童女的比喻裡面.
其實正正是說一個.
屬靈的sleep mode.
十個童女的比喻告訴他.

$^{481}$耶穌任何一個的.
有關比喻告訴他.
等待其實不是一件廉價的事.
當你說等待上帝的時候.
其實不是放下他.
然後就繼續做其他的事.
耶穌說你們要去警醒.
你們要去這樣的去等待.
你今天是不是開了一個.
屬靈的sleep mode.
肉身雖然在這裡.
是一個基督徒.
但你可能已經不知不覺之間.
開了一個sleep mode.
尤其是網上頂智媒.
你是不是處於一個.
屬靈sleep mode的狀態.
跟自己說我今天有回網上送禮.
但其實我的屬靈狀態.
已經去到一個能源保護模式.
睡了.
便然有去.
不過你是流力.
你的信仰是不是已經沉睡了.
我明白的.
因為我自己也試過.
我剛剛十八課才說.
我又很迷惘.
有一個身份認同危機.
人生去到某一個階段.
突然間檢視自己.
一直很努力的事.
突然發覺不知道自己在做什麼.
當我見到這個世界的運作方式很荒謬.
教會的方式運作得很荒謬的時候.
灰心到一個點.
不知不覺之間.
我突然不想做這些事.
最可怕的是其他人不發現.
我繼續在運作一些外面的事.

$^{521}$外面有很多人仍然覺得很好.
依然跟著我的時間表去侍奉.
但我發現內裡有些事物是不同了.
我發現我不想純粹去做一些宗教服務.
就算呼出很多人.
提很多人回來也好.
我發覺不是很多人真的跟隨耶穌的時候.
我只不過是在做一個很受歡迎的宗教服務.
我開始在想我究竟在做什麼.
我突然間有一段時間很迷失.
不知道我應該做些什麼.
有什麼意義.
後來我去了做foodpanda.
我發覺我大學畢業後.
我從未試過做外面的工作.
我一開始就做教會的工作.
我是第一次做外面的工作.
我發覺很開心.
我試過送一個榴槤給一個中文老婆婆.
我看到她拿著榴槤.
我覺得我做得很好.
我真的可以把榴槤送給她.
我從未覺得這樣可以幫到別人.
多過我純粹去做一些宗教的事.
當這些宗教的事只不過是滿足別人的宗教渴望.
而不是真正去做一些有意義跟隨耶穌的時候.
所以我發覺我內心有一個沉睡的意識.
甚至我發現我內心的心算.
我開始有些懷疑.
上帝在不在.
上帝是否存在.
就算這裡很受歡迎也好.
我開始在想我在做什麼.
這個狀態維持了幾個星期.
然後突然有一刻.
我在深水Po 地鐵站.
上帝在深水Po 地鐵站跟我說.
我不是不存在的.
只是你要等我.
等候上帝正正是一個.

$^{561}$當你發現上帝不在.
唯一一條很絲絲點點滴滴的一條線去連繫著.
當你覺得上帝好像不在的時候.
等待祂成為你唯一仍然連繫著祂.
一個很重要的一條線.
你知道每十個童女的比喻.
我告訴你.
無論你的屬靈狀態是怎樣.
你的燈是否有油.
你的燈可能沒有油.
你仍然需要做一件事.
你要一生的去等候這位上帝.
天國就是一回這樣的事情.
這是我和你面對天國的態度.
不管你今天如何面對.
或者你正在等待很多事情.
從你等待上帝為你預備的另一半.
等待上帝公義的來臨.
等待上帝很真實地跟你說話.
等待上帝醫治你的家人.
一個極度難捱的日子.
等待上帝沒有做事.
等待上帝不在.
所以在這段時間裡.
你只能做一件事.
你都只能做一件事.
雖然是做一件事.
這不是廉價的等待.
不是純粹將事情放下.
你需要在你心靈裡.
內心的宇宙裡.
不斷地用很多流星.
不斷地尋找上帝.
去跟祂說心裡的狀態.
每天將上帝放在你面前.
祂是沒有行動的.
是一個極度深的基督徒的功課.
你表面上一點都沒有做到.
但你裡面其實是熾熱的.
裡面不斷地尋求祂.

$^{601}$去禱告祂.
去尋求祂.
所以外面就像一個沒有行動的人.
裡面其實你不斷地尋求祂.
去學習祂.
你的意識,你的心思,你的意念.
不斷地去尋找祂.
這就是警醒.
在二戰的時候.
有一個美國神學家.
叫做Richard Lieber.
不是你聽過的李寶儀.
是這個李寶儀的弟弟.
Richard Lieber.
二戰的時候.
當這個世界不斷地大亂邪惡.
在掌管這個世界的時候.
Richard Lieber寫了一篇文章.
叫做The Grace of Doing Nothing.
以前我不贊同他的說法.
特別是幾年前.
難道不做事嗎.
但我發覺The Grace of Doing Nothing.
這個意思不是不做事.
而是表面上是沒有做事.
但裡面其實是做很多很多的事.
裡面不斷地尋求上帝.
你的心思,意念都在尋求上帝.
積極地去記住這件事.
積極地去等候他.
這個是關於盼望.
這個是關於信心.
這個是關於禱告.
這個任由你怎麼說.
但這個就是警醒.
裡面我們仍然記得.
很著緊,很關懷.
但我們好像表面上什麼都沒有做到.
明天是5月35日.
我們很快只能做Nothing.

$^{641}$我們不能做任何事.
但我們等候上帝.
等候上帝不是等運到.
我們心思,意念裡面.
我們很著緊,很記得.
我們仍然要這樣去尋找他.
接下來的十學百課.
那課叫做不在場的上帝.
Absent of God.
如果你覺得這個時候.
在你生命裡面.
這個世界,這個社會.
上帝好像不在的時候.
是不在的.
所以你要等他.
等他成為了你和他.
唯一仍然能夠牽著的那份關係.
不知道大家有沒有看過.
空中飛人的教學YouTube.
沒有的.
但之前有一本書教人如何做空中飛人.
一本小小的搜查書.
書中教人如何做空中飛人.
即是表演的空中飛人.
空中飛人分為兩部分.
一個叫做接的人.
Catcher 負責接的人.
一個叫做Flyer 飛的人.
如果在空中飛人的表演裡.
接和Flyer 兩者是一個非常微妙的關係.
每一個學習做空中飛人的人.
都開始教他一個很嚴守的定論.
什麼定論呢?.
就是Flyer的人負責Flyer.
接的人負責接.
聽起來好像很簡單.
但一個非常重要的.
第一個你要學做空中飛人的.
第一個的規則.
Flyer的人負責Flyer.

$^{681}$接的人負責接.
這是整個表演裡最重要的精髓.
當飛人飛到半空的時候.
即是放手的時候.
當他在半空的時候.
他要將自己的身體保持放鬆的狀態.
他記住他的工作就是保持飛行的狀態.
並且他等待接他的人.
在適當的時候去接他.
飛人絕對不可以打破這個定律.
去嘗試去接.
即是接你那個人.
Flyer的人絕對不可以去接.
飛人千萬不可以去接.
接你那個人.
飛人的工作只是完全.
順便來等待.
然後接的人就負責接他.
他只需要在半空中的狀態裡.
默默地去等待.
那一刻.
我們今天都處於這樣的狀態.
在空中的感覺是不好受的.
即是那種忐忑不安.
Identity crisis.
不知道該做什麼.
沒有另兩種感覺.
在人生裡突然間去到這樣的位置.
或者我們唯一的方法就是.
我們只能夠警醒.
好好地去記得.
接的人將會接著你.
忍耐,警醒,等候.
然後.
突然間有一刻.
上帝的手緊緊地抓住你的手.
然後整個空中飛人的表演就大功告成.
全場的人熱烈地拍手掌.
歡呼讚好.
因為你的耐心.

$^{721}$因為你的勇氣.
因為你願意去等待.
上帝的美好旨意.
就在你身上彰顯出來.
你看到不在場的上帝.
再次出現在你生命裡.
一會兒我們會請敬拜隊去唱一首歌.
那首歌的歌名叫《一生守候》.
這本是一首情歌.
「你」字我們理解為上帝的「你」.
就是當日幫我在深水Po 地鐵站的時候.
我聽著這首歌的時候.
上帝對我說.
他說「我在這裡,你要等我」.
我們一起留意歌詞.
\newpage



\section{使徒行傳 2:40-47-20230610}
\label{sec:E2mqQjkFX2s}
\textbf{【網上崇拜】同心用飯|使徒行傳2\_40-47|20230610 [E2mqQjkFX2s]}
\newline
\newline
連結: \href{https://youtube.com/watch?v=E2mqQjkFX2s}{\texttt{ https://youtube.com/watch?v=E2mqQjkFX2s}} ~~~~ 語音日期: 2023-06-10 
\newline
\newline
\hyperref[sec:40Zpw7rWZSQ]{\small{< < < PREV SERMON < < <}}
~
\hyperref[sec:index_chronic]{\small{[返順時目]}}
~
\hyperref[sec:index_scriptual]{\small{[返順卷目]}}
~
\hyperref[sec:CppPjcT08EA]{\small{> > > NEXT SERMON > > >}}
\newline
\newline
使徒行傳 2:40-47-20230610
\newline
\begin{longtable}{cl}
\hline
\hline
章節 & 經文 (和合本修訂版)\\
\hline
2:40 & \begin{tabularx}{0.7\textwidth}{X} 彼得還用更多別的話作見證,勸勉他們說:「你們當救自己脫離這彎曲的世代。」 \end{tabularx} \\ \\ \relax
2:41 & \begin{tabularx}{0.7\textwidth}{X} 於是領受他話的人,都受了洗;那一天,門徒約添了三千人。 \end{tabularx} \\ \\ \relax
2:42 & \begin{tabularx}{0.7\textwidth}{X} 他們都專注於使徒的教導和彼此的團契,擘餅和祈禱。 \end{tabularx} \\ \\ \relax
2:43 & \begin{tabularx}{0.7\textwidth}{X} 眾人都心存敬畏;使徒們又行了許多奇事神蹟。 \end{tabularx} \\ \\ \relax
2:44 & \begin{tabularx}{0.7\textwidth}{X} 信的人都聚在一處,凡物公用, \end{tabularx} \\ \\ \relax
2:45 & \begin{tabularx}{0.7\textwidth}{X} 又賣了田產和家業,照每一個人所需要的分給他們。 \end{tabularx} \\ \\ \relax
2:46 & \begin{tabularx}{0.7\textwidth}{X} 他們天天同心合意恆切地在聖殿裡敬拜,且在家中擘餅,存著歡喜坦誠的心用飯, \end{tabularx} \\ \\ \relax
2:47 & \begin{tabularx}{0.7\textwidth}{X} 讚美神,得全體百姓的喜愛。主將得救的人天天加給他們。 \end{tabularx} \\ \\
[1ex]
\hline
\hline
\end{longtable}
$^{1}$頂姐妹平安.
網上的頂姐妹平安.
你看到的崇拜是Flow Church.
這次在崇珍會心水報堂和頂姐妹一起直播崇拜.
今天除了燈我主教山之外.
還要是冒著雨擔著傘.
這也是對頂姐妹的挑戰.
很開心無阻大家一起去敬拜.
我相信在呼吸的過程中.
都感受到一起參與同呼同吸的感動.
對於崇拜來說.
Flow Church每次都希望讓頂姐妹一起去感受到訊息.
無論是講道.
特別是詩的訊息都成為我們整個屬靈生命的提醒.
今天選的經文是承接上次清心講道.
關於二章的經文.
她在二章頭的經文.
我就選擇二章尾的經文.
上次講道是霹靂一閃.
所以很害怕剛才在等雨的時候.
很害怕.
因為兩個星期前.
我在目者群組裡才傳出美國的教會被雷劈中.
是一場大雨.
然後我們另一個群組說.
在下雨了快點.
因為我真的很害怕.
我上次在二樓說信心少.
同工就截圖了.
這次不要說信心了.
這次選的經文是二章尾段的經文.
四十至四十七節是講整個群體新建立的時候遇到的場景.
我們先重溫下一段經文.
大家一起讀.
《史祿行傳》第二章第四十節.
起,請.
最後一段經文對於我們信主的年日.
吾脈生.
大日我們今天再重讀上帝你的說話的時候.
求上帝你的靈在當日賜予給眾聖徒的教會.

$^{41}$今天同樣賜過給我們.
讓我們同感一靈.
去領受上帝的話語.
遵主的吩咐.
成就上帝在給教會的使命.
求主你使用我們.
幫助我們.
祈禱奉耶穌的名求.
阿們.
大家很熟悉的經文.
這段經文當中有幾樣東西.
想跟大家一起在這段時間再拿出來看.
首先第一段經文會說到.
彼得要見證和勸勉當時的人.
因為彼得在他的講論當中.
就在說一班猶太人.
他們所信的上帝.
其實耶穌基督已經展現了.
真正拜上帝和耶穌基督的工作.
而你看到這句話就是說.
你們當救自己脫離這彎曲的世代.
我相信如果今天你看到世代是很邪惡.
或者覺得世代是很本末倒置的時候.
不只是這個年代.
以往也有.
我相信福音對我們來說.
不僅僅是一個分辨的能力.
更加是一個提醒.
就是我們怎樣可以在不同世代當中.
去面對彎曲,薄貌的環境.
我們仍然可以分辨之餘.
還有有據理力爭.
可以讓人明白到.
其實你可以選擇的.
其實你可以轉的.
其實你可以不需要同流合污.
更加可以不需要靠邊站的.
你仍然有你的起點.
仍然可以宣講.
當然有人會認同.

$^{81}$有人會不認同.
但認同的人就是.
領受他的話就受了洗.
在過程中讓那些人有機會選擇.
可以讓你選擇的時候.
你想走舊路還是走新路.
你走新路的時候.
你就會有一個新的轉向.
你願意轉向的過程中.
就成為你新的一頁.
剛剛上個星期日是幾號?.
是6月4號.
我報告的時候.
6月4號是一個很特別的日子.
原因是什麼?.
原因是因為那天是第四屆流淌的水泥.
上個星期日的水泥當中.
其實Victor.
就是我們的傳導人.
的訊息和一班水泥者.
要再一次讓他明白到.
是一個很勸勉的.
一個很警世的話.
就是你在今天.
這一刻要接受水泥的意義是什麼?.
或者原因是什麼?.
正正就是你可以選擇.
你可以選擇不洗泥.
甚至不是現在洗泥.
但是我們也很認真.
因為你想開展新的一頁.
你想在屬靈生命上刷新.
讓上帝在你屬靈生命當中加把勁.
讓你感受到你生活有力.
而我們在大公的群體當中.
成為一個團隊.
可以繼續去見證.
這就是一個很重要的訊息.
彼得也是呼籲他的親屬.
他的同胞.

$^{121}$不再需要守舊約的條文.
耶穌基督已經是新約的中寶.
彼得說得很認真.
就是其實舊的東西幫不了你.
既然有新的東西.
你願不願意轉變?.
同樣也是在問你.
舊的東西幫不了你.
新的東西你願不願意轉變?.
不要說信不信主.
不要說是不是我們的基督信仰.
就說在香港生活了這麼多年.
有些東西你會發覺是以往幫不了你的.
如果有新的方法你願意試?.
這句話是很危險的.
因為要看你試什麼.
但是我們信的主.
或者我們心中的聖靈.
會提醒我們分辨.
所以我不說外面的生活環境.
不說你的仕途.
我只說我們的信仰.
其實過去的三年.
很多東西反映了我們信仰的實在.
和我們的堅持是什麼.
當有機會再自由返教會.
再可以公開參與聚會.
無分左隔的時候.
其實你最想要看重.
和你最珍惜的是什麼?.
彼得提醒灣曲撥米鴿世代.
最重點是什麼?.
洗禮是一個很重要.
讓我們提醒自己.
當然你知道我不是叫你再洗.
在我上洗禮班的時候.
說洗禮是一次.
只要你清楚奉父子聖靈的名洗禮.
無論是什麼教會都好.
我們都歸入了大公教會.

$^{161}$我說洗禮是一次.
對你來說.
但是我們可以重新去認順.
讓我們再歸納上帝的名下.
所以第42節說的重要的事情.
你既然領受了.
我們就恆生遵守使徒的教訓.
彼此交接.
交接這個字好像比較怪.
但是如果用其他頁本.
就是彼此團契.
剛才在過程當中.
敬拜開頭的時候.
一人的避難所.
二人的避難所.
我們的避難所.
我們在一個群體當中.
可以一起去聚集.
一起可以彼此親近.
一起彼此去承擔.
常常都提醒大家.
You are not alone.
並且We are not alone.
不只是香港.
我們在網上有很多群體.
我們可以同呼同吸.
這個就是我們彼此團契.
可以享受.
Mark Bank是什麼呢?.
團契的過程當中.
不僅僅只是圍爐一班人.
我們再一次重述Mark Bank的意義.
當我們每一次.
領受聖體和補血的時候.
再一次去感受到.
我們破碎的身體.
因若基督的緣故.
我們可以彼此聯合.
下個星期是我們的聖餐.
聖餐再一次提醒我們彼此聯合.

$^{201}$更加是要追溯.
能夠讓我們彼此聯合.
是因為基督先撕開了他的身體.
為我們捨棄.
道理我相信你們都懂.
但是願意一起來彼此聯合.
這是我們要走的那一步.
是真的.
過去這幾個月.
見過不同的教會.
有時我主日還會去不同的堂會講道的時候.
都會問一件事.
或大家關心的一件事.
就是有多少會眾會回來實體聚會.
有很多原因有出席.
有很多原因沒有出席.
但仍然是說一句.
實體的聚會不是只聽一篇訊息.
實體聚會是讓我們感受到.
我們聯合過程當中.
一起去紀念.
一起去參與.
這是很重要的.
麥餅做了一次告訴我們.
在當中去述說耶穌基督給我們的恩典.
給我們的教導.
而最後提醒一件事就是祈禱.
不知道你自己的祈禱習性是怎樣.
我之前都講過.
我常常鼓勵弟妹有個祈禱的習慣.
長短不要緊.
最重要是有做.
就好像你吃東西.
你一六百都好都會吃東西.
你可能二四零.
二四零那些不是叫斷食.
應該說是絕食.
二四零那些叫絕食.
但你無論什麼比例都好都會吃東西.
和祈禱一樣.

$^{241}$你祈長或短都好.
你有祈禱都是和上帝交心.
但我希望在祈禱當中.
讓我們再一次將一些時間.
一個時段分別出來.
和上帝建立關係.
許頂姐妹就覺得.
我好久了.
上帝可能不聽我祈禱.
我仍然和她分享一樣東西.
你是耶穌基督的血買回來的.
我們是因著耶穌基督的血有生的生命.
我們和耶穌建立一個親屬的關係.
祂是掌興.
我們是一個親屬的關係.
是沒有東西可以拿走的.
祂不會不回應我們.
只不過你的溝通關係差了.
就正如你要登報和家人斷絕關係.
舉個例子.
你只是法律上證明你斷絕了關係.
但你的DNA仍然是持有你的原生家庭.
我們也是.
耶穌基督的血買熟了我們.
我們建立的是父與子之間的關係.
你沒有和祂聊天.
只不過是沒有溝通關係.
祈禱就是再重新保存我們本身的溝通關係.
我們是再來的.
本身已經存在我們.
我們存在我們.
我們去拿出來.
所以彼得要提醒當時的弟兄姊妹.
三件事.
團契讓我們一起聚集.
我們不孤單.
我們去面向新的一頁.
麥餅是從中之中.
就是叫我們去知道.
這個群體是因著耶穌基督而建立的.

$^{281}$我們是宣揚耶穌基督.
以至我們是基督徒.
我們是宣揚基督.
我們才是基督徒.
最後就是我們一起.
一個信仰是祈禱的群體.
很希望這個對於我們疫情後.
我們能夠可以如常有聚會生活的群體.
我們試一下做回這三件事.
這個是刻意放在我們每個星期的時間表.
彼此團契.
無論你有沒有小組也好.
崇拜就是我們彼此聯繫的地方.
聖餐就是我們彼此聯繫的地方.
祈禱也是我們彼此聯繫的地方.
我說的話沒有什麼新的.
我作為一個傳道者.
你問我我講一百篇道.
我都是講這些的.
因為這個就是教會存在最核心的意義.
但對我來說.
又不是很多弟兄姊妹願意做.
正如剛才說道理你都懂.
但是要行出來的時候.
就要放在每個星期的時間表裡面.
這個是要你刻意放在這裡.
下一段經文是什麼呢?.
就是我們看到眾人都懼怕.
試圖行了許多的神蹟.
我們看這一節.
43節這裡裡面.
為什麼人們這麼害怕呢?.
害怕什麼呢?.
如果你還記得當日.
耶穌的屍首不見了的時候.
其實科恩書要記載的就是.
很多人怕耶穌真的復活.
而這班試圖傳揚的就是.
你們殺的耶穌已經復活了.
而且在我們這個群體當中顯現.

$^{321}$有40日之久.
而這段經文的背景就是.
耶穌升天之後的十日.
就是五十日.
五場節的時候.
祂升天降臨.
就是聖靈降臨在眾人身上.
就好像上次清心說的.
他們有聖靈的能力賦予給你們.
他們很害怕.
他們害怕什麼呢?.
其實都害怕被人抓的.
因為那時候凡是和耶穌一夥的.
都會有不同的人想埋他們身.
無論是發起賽人或者是其他文士.
他們都質疑這個信仰.
甚至羅馬的政權都不喜歡.
因為耶穌死了之後.
還有很多後遺的東西.
會群起.
令到當時的環境很大張力.
所以害怕理解的.
但他們又很驚奇.
這班仕途為什麼會做這麼多奇妙的事呢?.
今天我們香港的環境.
都不是最需要.
因著信仰的緣故有太多懼怕.
但是在過去日子.
信仰有沒有給你一個驚喜呢?.
或者有沒有看到.
其實在疫情當中.
上帝仍然給我們看到.
其實有很多可能性呢?.
如果你一直都是參與我們福出崇拜的時候.
我相信你會看到的.
其實上帝仍然有透過很多群體.
透過我們.
仍然做了很多奇妙的事.
最奇妙的就是.
仍然是.

$^{361}$我相信講到主在內.
特別是在我們群體當中.
就是5月6日的戶外崇拜.
我仍然覺得這是一個超自然的神蹟.
是很不容易.
但是我們能夠在一個公開的環境當中.
可以有聚集.
可以有崇拜.
可以讓更加多人知道.
其實我們是可以.
We sing everywhere.
不是去一個禮堂.
是我們真的在哪裡都好.
我們有敬拜群體.
可以一起走出去.
不是說眾志成城就可以的.
但是我們仍然相信.
只要我們有心去傳揚基督的時候.
上帝的sign.
神蹟.
會讓我們感受到祂的miracle.
你如果參與在其中.
你就會知道.
你有一切經歷的時候就會知道.
有時很多時候你都會聽我說.
盡量呼籲弟兄姊妹一起來實體崇拜.
當然我知道有很多弟兄姊妹不容易出現.
特別是海外的弟兄姊妹.
他們不出現.
應該說出現不了.
但是最大的安慰就是.
很多時候很多弟兄姊妹在海外回來香港.
無論探親也好.
或者辦事也好.
他們都會星期六回來和我們崇拜.
但是星期六通常都是和家人吃飯的黃金時間.
但是他們都會找一個星期回來一起崇拜.
就好像我知道今天也是.
他們今晚上飛機.
明天回來香港.

$^{401}$但是他們已經說明不讓我聽.
下星期他們會回來吃一頓崇拜.
因為很希望和一個群體一起參與.
最主要就是他們感受的就是.
可以參與的機會不多.
有機會參與.
參與一次就一次.
這個是重要的.
我講這些話.
我都希望你不要轉了一邊.
我並不是要情緒勒索大家.
說有機會讓你回來.
你不回來就該死.
我不是這些.
但是我想讓你明白到.
其實沒有失去過的時候.
有時候你不懂得珍惜.
有時候都說下次會來的.
或者下次有戶外崇拜.
這次戶外崇拜來不了.
下次有戶外崇拜.
我會參與的.
但是我很難告訴你.
我們有沒有下一次戶外崇拜.
因為現在這個年代.
或者這個環境當中.
沒有約定俗成.
這次可以.
下次也可以.
因為每一次都是一個恩典.
一個神蹟.
所以我自己做事也好.
或者參與也好.
我常常都說.
每件事都有下日期.
做得一次就一次.
每次都當最後一次這樣做.
我相信這個都是我們.
正在享受和正在經歷.
怕什麼?.

$^{441}$不知道.
很多人都有很多事情怕.
但是我仍然看到很多奇妙的事.
正在發生.
如果你一直和我們一起.
我相信你都感受到.
我們很多奇妙處處.
不是迪士尼那些奇妙處處.
這些詞被人用了就很難再用了.
我們不要再說了.
再下去第二段的時候.
信基人再一次繁物公用.
並且賣了田產家業.
照個人所需用的分級.
這句經文大家應該都很熟悉.
我不會說.
問你們有沒有十一奉獻.
我不是說這些.
繁物公用這個詞.
在環教會可能比較敏感.
或者說每個人都要拿自己的份拿出來.
當然這個重點可以成為一篇講章.
但是我想今天集中一件事.
其實重點不是放在.
賣掉自己的東西.
要共產共同富裕.
不是這個想法.
其實大家重點要放在.
集在一起的過程當中.
是將所需用的分級各人.
是一個共享文化.
什麼是共享文化?.
就是見到有需要的時候.
你願意伸出援手.
見到有需要的時候.
你會動了持心.
好撒瑪利人的教導.
一個很重要的就是.
不是說什麼身份問題.
最重要就是.

$^{481}$經過的人.
看不看到那個人的需要.
而看到那個人的需要的時候.
他會不會願意參與呢?.
你猜利美人和那幫經過的宗教領袖.
他看不到嗎?.
他看到,但是他不願意參與.
但是撒瑪利人他看見.
他動了持心.
他願意參與.
所以將所需用的分級各人.
是一個身體力行.
心意轉向的改動.
今天我們都會見到很多東西.
但是你會不會行動呢?.
凡物公用語講的一個訊息就是.
將我們所得的.
重新去分配.
讓你看到的時候.
你將你管理的事情.
重新分配,分一邊出來.
放進去.
所以你的財寶在哪裡.
你的心都在哪裡.
你願意支持.
你願意訂閱.
你願意點都好.
你願意課金都好.
你就是將你自己的心放在那裡.
疫情下.
又或者是在這幾年的大環境當中.
你很多東西都要管理.
你很多東西都需要想一下.
怎樣去重新去定位.
值不值得去支持.
對於我們都是.
Folk Church將所有的資材.
放在兩部分.
一部分就是我們每週的崇拜.
裡面我們所用的器材.

$^{521}$我們的物資.
還有其他的配搭上.
我們希望弟兄姊妹.
能夠有一個高質素的崇拜.
你聽的音響.
你看見設置.
因為我們希望.
將最好的在崇拜當中呈現.
因為教會的崇拜是值得的.
我們將最好的給上帝.
其實過去教會開展的時候如是.
你去到歐洲.
你看到很多教會.
最好的建築.
最好的音樂.
最好的藝術.
最好的設置.
都是在教會.
將所有東西呈現給上帝.
為什麼今天教會不是這樣做.
我們成為一個敬拜群體當中.
我們將弟兄姊妹奉獻給我們的時候.
我們都將最好的東西放在當中.
跟弟兄姊妹一起去參與.
這個是很重要的.
第二個就是牧羊群體.
Folso 將很重要的資材.
都放在一群牧者大組裡面.
讓弟兄姊妹感受到.
近身的牧羊和貼心的那種關顧.
在過程當中.
每一個小組裡面.
是來自很多不同成長背景的弟兄姊妹.
是不容易的.
大家都要去碰一下.
大家都要交心.
彼此去協調一下.
才會有一個比較穩定溝通的信仰群體.
是不容易的.
但我相信我們的牧者.

$^{561}$都是盡力去協調.
和讓你感受到被關顧.
被牧養.
這個是我們剛才說過.
彼此交接.
擦餅.
禱告的群體.
這個都是Folso Church.
最重要的核心.
這個也是我們將資源放在最多的地方.
我們的牧者.
在Infogroup的時候我也說過.
我們是因材去侍奉.
不一定要分餅仔.
牧者要分不同的事工.
我們重新定義牧者的工作.
這個也是Folso Church看重的地方.
我們就將這件事重新去定位.
讓弟兄姊妹感受到.
教會其實有另外一個向度.
或許你對過去教會有不同的感受或者經歷.
但是Folso Church是一間希望.
去讓弟兄姊妹感受到.
其實教會最重要的essence是什麼.
所以信的人都要習一處凡物公用.
公用的重點是.
將各人所需要的分給各人.
不多取.
同一段經文在《平行》就會去到第四章.
我今天不是說那裡.
但是第四章都會出現這件事.
大家每人都要拿回自己的一份.
不多取.
以致浪費了.
弟兄姊妹.
其實每一個禮拜回來Folso Church.
崇拜的弟兄姊妹.
都不是每個人都入了小組.
有些都是每個禮拜和我們一起崇拜.
Folso Church就是一個群體.

$^{601}$有些弟兄姊妹會入小組.
會和我們一起崇拜.
有些只是和我們一起崇拜.
沒有入小組.
但是對我們來說.
這也是一個穩定的信仰群體和我們一起.
因為我認識的弟兄姊妹.
她在這裡崇拜.
感受到這個信仰群體.
她charge up了之後.
她都有外面的服侍.
這就是我們感受到.
「將所需用的分給各人」.
接著和大家看看這張圖畫.
在它還沒有出現的時候.
我已經做了這個powerpoint.
所以.
今天好像只剩下一張.
Valentin Hoffman.
這個荷蘭的設計師.
這張照片是十年前的.
十年前拍的.
現在不同了.
現在是左邊.
如果你昨天看到新聞報告.
這次他帶了兩隻來.
現在爆了一隻.
Double Dutch其實是有意象的.
譬如喜字是兩個喜字.
Double Happiness.
朋友兩個月.
你會看到有十八個港鐵站.
和有電車在港島區的時候都會行走.
由10號到23號都是一個類似的movement.
但我今天不是說這件事.
但我在預備這張圖的時候.
我想起一個故事.
這個故事不是他的故事.
這個故事是關於這位老人家的.
可能很多人都不認識他.

$^{641}$他的故事我什麼時候聽呢.
我第一次聽的時候是2016年.
我和兩個兒子.
他們還是很小的時候.
去香港歷史博物館.
那時候還沒有重整.
那時候香港歷史博物館有一個香港館.
香港館裡面剛剛是說香港的發展史.
其中就說香港的輕工業.
就是塑膠.
他有一個video訪問他.
我就站在那裡看.
看完之後.
我是不其然的.
眼睛有很多.
未至於要哭的.
但我覺得這個願景真的很感動.
和我很渴求的.
林亮先生他被稱為中國的變形金剛之父.
變形金剛的原因是什麼呢.
因為他早期是幫孩之寶做代工.
孩之寶可能大家都不知道是什麼.
總之他就是一個玩具生產商.
外資.
但他不只是做美國孩之寶.
他也做Sanrio的Hello Kitty.
有些他也幫忙做.
他是做塑膠.
但他的出現不是做塑膠.
他做銷售.
然後他轉行做了塑膠.
但我今天想說的不是唱好香港的故事.
不是.
也不是說香港輕工業的發展.
現在可以再轉裝.
不是.
我只是說他那番話.
他那番話就是.
他那時候去做玩具銷售.
以前的玩具就是.

$^{681}$用那些玩的鐵車仔.
可能你當Hot Wheels那些鐵車仔.
或者你現在叫做新潮的Tomica那些.
鐵車仔.
他那時候的玩具全部都是鐵車仔和鐵甲人.
全部都是用鐵.
他們買玩具定價.
因為鐵需要秤.
秤用的重量原材料.
計算一下定價多少錢.
這樣來計算.
但他們的小鴨.
這些塑膠玩具這麼輕.
怎麼定價.
人們就會問.
多少錢呀.
買得多少錢呀.
只是啤出來的.
成本很低的.
你怎麼定價.
哪有人喜歡.
所以他們在香港不受歡迎.
於是他和一班同道.
一起做塑膠行業的同道.
於是就自資去了德國.
參加歐洲的玩具展.
做了第一代的黃色小鴨.
接著呢.
發了.
這個黃色小鴨去到歐洲之後.
很多小朋友因為香港是倒水.
用花灑.
花灑很時髦.
通常都是倒水.
人家外國不是的.
有個浴缸的.
是不是.
你夾夾夾夾夾夾.
這樣.
當然是玩這些的.

$^{721}$人家當然是大賣.
香港不能賣.
發了.
但是.
他說了一句話對我來說是很受用.
他說那時候沒有人看得起這些設計.
沒有人覺得這東西是可以的.
但是他就.
那時候他們什麼.
他們一班人一起做這個玩具島模.
想一下什麼模.
怎麼可以令塑膠的彈性好一點.
沒那麼容易變形.
味道又好一點.
沒那麼臭.
還有顏色又鮮艷一點.
令到人家感受到那種常新的感覺.
不會有老化.
玩具老化的感覺.
他們不斷地想.
想什麼方法可以做得更加好.
他們說.
我們那時候.
是一班人一起.
想什麼.
想怎樣做好這個輕工業.
或者做好這些塑膠玩具.
我們一班人一起找一碗飯吃.
而不是自己找自己一碗飯吃.
這個話我聽下去.
我覺得.
是我很想要有的東西.
我們是一班人一起找一碗飯吃.
而不是自己找自己一碗飯吃.
那時候我想起這句話.
因為我在搞第五屆青少年工峰會.
我每年都在搞青少年工峰會的時間.
我跟一些教務同工說.
今天青少年在教會很荒涼.
因為很多人都不回教會.

$^{761}$我說.
如果你只是還在想辦法.
做自己的青少年.
你一定是呆單的.
你只有十幾個怎麼做.
但我說不是.
是我們一起去做青少年.
一起做青少年的時候.
我們才能夠做起青少年.
不要只想自己的堂會.
找自己的一碗飯吃.
是我們一起找同一碗飯吃.
這件事最重要.
我自己的第二年畢業.
寫的論文是寫香港青少年的.
香港以中學為例.
有三分一都是教會辦學的.
所以香港那時候有十幾萬的會考.
有三分一其實都在教會裡面.
其實你能夠接觸的年青人不少.
今天去到這個環境.
Flo Church他想做什麼呢.
Flo Church不是想找自己的一碗飯吃.
Flo Church是希望動搖更加多弟兄姊妹.
更加多教會.
一起去動香港教會的生生態.
這個很重要.
整個信仰群體不可以獨善其身.
只是Flo Church work.
其他不work.
香港教會都不work.
香港教會會被瓦解.
其實Unity是說.
整個群體同心合意一起去參與.
不只是相反.
看細一點.
不只是你去崇拜.
是你身邊的人都去崇拜.
你認識的弟兄姊妹都去崇拜.
小弟都叫得認得人.

$^{801}$有些認識我的弟兄姊妹都知道.
有時我都會叫他的名字.
他說為什麼你記得我的名字.
我說我都盡力去記.
但有時有些弟兄姊妹認得到.
我叫到他的名字的時候.
我就知道.
他有時走的時候.
散會.
他就會和我說.
因為他指著那個.
那個是他一直都想帶回來的.
舊團友.
舊朋友.
舊同事.
舊同學.
他很久沒有回教會.
他一直都想邀請很久.
他帶到他回來.
他這樣.
我就.
是喔.
真是.
不只是你一個.
是你身邊的人都可以感染到.
不只是flowchurch想人多.
是全教會一起去面向.
香港還有很多基督徒.
說自己是基督徒.
有七十多萬.
說自己是基督徒.
但只有不夠二十萬常歸回教會.
所以你一個可以帶三個.
不一定回來這裡.
但帶他回來這個信仰群體很重要.
林亮先生的話對我來說是很大的提醒.
我們是一班人找一碗飯吃.
而不是一個人找自己一碗飯吃.
所以經文載下去的時候.
我希望和大家.

$^{841}$信的人在一起的時候.
凡物公用不是僅僅是錢.
不是僅僅是資材.
他們最後是什麼.
他們是同心合意去興旺福音.
他們且在家中抹餅.
存著歡喜的心誠實用飯.
一起吃一碗飯.
當時的環境.
其中一個難處就是.
有人是被捉拿.
被問話.
因為他.
他自己是基督徒.
他是跟隨耶穌的人.
被人捉拿之後.
誰安家.
教會就將他們帶到他們家中.
成為他們一起用飯的群體.
有人幫他們做接濟.
有人成為他們的後盾.
有人成為他們的支援.
拿著信仰的緣故.
離開了這個群體.
但是沒有人幫助你.
不是.
是一起去找同一碗飯吃.
一起去參與.
誠實的心用飯就是.
每人不拿多.
大家按大家的需要去分給人.
這是共享的文化.
所以共享是一個人情味.
將有需要的時候.
分給各人.
最後第47字經文是.
信的人都聚集一處.
一起共享.
你會看到最後的圖畫是什麼.
大家一起齊心去讚美.

$^{881}$你會看到.
每一次去讚美.
一起感受那種unity.
我們一起有一個目標.
一起去參與.
一起將這個訊息傳遞出去的時候.
讚美上帝.
我們得益喜悅.
上帝會將得救的人.
賜給這群能夠寄託的群體.
人多不是因為.
扮得好.
扮得好是基本.
是必要的.
人多是因為.
大家能夠在當中見到上帝.
在這裡投入可以敬拜.
一起可以參與.
感受上帝在其中.
你就會帶人回來.
真的.
我們不僅僅顧及現場的弟兄姊妹.
將最好的音響.
因為有一次.
就是講5月6日戶外崇拜那次.
有弟兄姊妹帶了家人來.
家人不知道他們回什麼教會.
他帶他們來看戶外崇拜.
他當初以為教會戶外當然是做busking.
打木箱鼓.
來到發覺原來音響這麼厲害.
是啊.
我們音響很厲害.
真的.
是以十萬計在戶外崇拜當中.
用在這裡.
只是那對喇叭.
我們租回來.
租回來.
也二十萬.

$^{921}$不是租二十萬金.
不是.
不是租金二十萬.
那對喇叭值二十萬.
我講得清楚一點.
哪個攝影機要剪一下.
我們只是租的.
我們沒有二十萬的喇叭.
教會是值得有這樣的牌頭.
是不是.
因為我們敬拜上帝的聲音.
是不可以小於周杰倫的.
這是很重要的.
是不是.
所以我們讚美上帝.
得眾民的喜悅.
上帝就會將人帶到在我們當中.
這個就是教會存在.
We sing everywhere.
我希望弟兄姊妹.
你帶弟兄姊妹來崇拜的時候.
你是不以福音為恥的.
你帶她來的時候.
她會感受到.
We sing everywhere.
We sing together.
這個很重要.
我們聚集一處.
共享我們所有的詩情.
我們一起讚美上帝.
這個是給得.
希望眾人一起看到的地方.
同心共憤.
仍然是我自己今天的講題.
因為我們是一起找.
同一碗飯吃.
這碗飯是上帝賜給我們的糧食.
我們一起供應.
我們肉身的需要.
但同樣供應我們靈裡的需要.

$^{961}$黑色間的那首詩歌.
我相信比我們有很多upbeat.
而upbeat的樂題.
我用這個訊息.
就是因為真正做到上帝的beat.
就是我們在一起.
但我們同樣都猜出去.
讓更加多人在一起.
這個就是上帝給我們時代的beat.
我再一次祈禱.
天上帝我相信.
你給我們每個人都有一個使命.
可能不同階段有不同.
但我們在香港.
我們這個群體當中.
我們一起去建立.
為的事就是.
我們在這個教會.
讓更加多人在困境當中.
可以得聞福音.
在這個不安的時候.
我們播下平安的訊息.
在他需要支援的時候.
我們成為隨時的幫助.
亦讓他明白到.
主你是他的避難所.
而他不孤單.
因為他實體都能夠感受到.
我們這個群體.
求主你帶領我們.
我們亦將這個流動的群體.
恭敬交給主.
求主繼續使用我們.
我們一起去感受到.
上帝你擺在我們身上.
在哪裡都好.
都是上帝你使用的器皿.
都是上帝你的回眸.
回聲.
Everywhere.

$^{1001}$我們祈禱感恩.
奉耶穌的名求.
阿們.
阿門.
\newpage



\section{約書亞記 2:1-22-24-20230617}
\label{sec:CppPjcT08EA}
\textbf{【網上崇拜】那些年,我沒有白過|約書亞記2\_1,22-24|20230617 [CppPjcT08EA]}
\newline
\newline
連結: \href{https://youtube.com/watch?v=CppPjcT08EA}{\texttt{ https://youtube.com/watch?v=CppPjcT08EA}} ~~~~ 語音日期: 2023-06-17 
\newline
\newline
\hyperref[sec:E2mqQjkFX2s]{\small{< < < PREV SERMON < < <}}
~
\hyperref[sec:index_chronic]{\small{[返順時目]}}
~
\hyperref[sec:index_scriptual]{\small{[返順卷目]}}
~
\hyperref[sec:XixhhdfEXw8]{\small{> > > NEXT SERMON > > >}}
\newline
\newline
約書亞記 2:1-22-24-20230617
\newline
\begin{longtable}{cl}
\hline
\hline
章節 & 經文 (和合本修訂版)\\
\hline
2:1 & \begin{tabularx}{0.7\textwidth}{X} 嫩的兒子約書亞從什亭暗中派兩個人作探子,說:「你們去窺探那地和耶利哥。」於是二人去了,來到一個名叫喇合的妓女家裡,在那裡睡覺。 \end{tabularx} \\ \\ \relax
2:2 & \begin{tabularx}{0.7\textwidth}{X} 有人告訴耶利哥王說:「看哪,今夜有以色列人到這裡來窺探此地。」 \end{tabularx} \\ \\ \relax
2:3 & \begin{tabularx}{0.7\textwidth}{X} 耶利哥王派人到喇合那裡, 說:「你要交出那來到你這裡、進了你家的人,因為他們來是要窺探全地。」 \end{tabularx} \\ \\ \relax
2:4 & \begin{tabularx}{0.7\textwidth}{X} 但女人已把二人藏起來,卻說:「那兩個人確實到我這裡來過,他們從哪裡來,我卻不知道。 \end{tabularx} \\ \\ \relax
2:5 & \begin{tabularx}{0.7\textwidth}{X} 天黑、要關城門的時候,他們就出去了。他們往哪裡去我也不知道。你們趕快去追他們,就必追上。」 \end{tabularx} \\ \\ \relax
2:6 & \begin{tabularx}{0.7\textwidth}{X} 其實,這女人已經領二人上了屋頂,把他們藏在她擺列在屋頂的亞麻梗中。 \end{tabularx} \\ \\ \relax
2:7 & \begin{tabularx}{0.7\textwidth}{X} 那些人就往約旦河的路上追趕他們,直到渡口。追趕他們的人一出去,城門就關了。 \end{tabularx} \\ \\ \relax
2:8 & \begin{tabularx}{0.7\textwidth}{X} 二人還沒有睡之前,女人就上屋頂,到他們那裡, \end{tabularx} \\ \\ \relax
2:9 & \begin{tabularx}{0.7\textwidth}{X} 對他們說:「我知道耶和華已經把這地賜給你們了,並且我們也都懼怕你們。這地所有的居民在你們面前都融化了。 \end{tabularx} \\ \\ \relax
2:10 & \begin{tabularx}{0.7\textwidth}{X} 因為我們聽見你們出埃及的時候,耶和華怎樣在你們前面使紅海的水乾了,並且你們怎樣處置約旦河東的兩個亞摩利王,西宏和噩,把他們完全消滅。 \end{tabularx} \\ \\ \relax
2:11 & \begin{tabularx}{0.7\textwidth}{X} 我們一聽見就膽戰心驚,人人因你們的緣故勇氣全失。耶和華-你們的神是天上地下的神。 \end{tabularx} \\ \\ \relax
2:12 & \begin{tabularx}{0.7\textwidth}{X} 現在我既然恩待你們,求你們指著耶和華向我起誓,你們也要恩待我的父家。請你們給我一個確實的憑據, \end{tabularx} \\ \\ \relax
2:13 & \begin{tabularx}{0.7\textwidth}{X} 要救活我的父母、兄弟、姊妹,和所有屬他們的,拯救我們的性命脫離死亡。」 \end{tabularx} \\ \\ \relax
2:14 & \begin{tabularx}{0.7\textwidth}{X} 那二人對她說:「我們願意以性命來替你們死。你們若不洩漏我們這件事,當耶和華將這地賜給我們的時候,我們必以慈愛和誠信待你。」 \end{tabularx} \\ \\ \relax
2:15 & \begin{tabularx}{0.7\textwidth}{X} 於是女人用繩子把二人從窗戶縋下去,因為她的屋子是在城牆邊上,她也住在城牆上。 \end{tabularx} \\ \\ \relax
2:16 & \begin{tabularx}{0.7\textwidth}{X} 她對他們說:「你們暫且往山上去,免得追趕的人遇見你們。要在那裡躲藏三天,等追趕的人回來,你們才可以走自己的路。」 \end{tabularx} \\ \\ \relax
2:17 & \begin{tabularx}{0.7\textwidth}{X} 二人對她說:「你叫我們所起的誓與我們無關, \end{tabularx} \\ \\ \relax
2:18 & \begin{tabularx}{0.7\textwidth}{X} 除非,看哪,當我們來到這地的時候,你把這條朱紅線繩子繫在縋我們下去的窗戶上,並要叫你的父母、兄弟和你父的全家都聚集在你家中。 \end{tabularx} \\ \\ \relax
2:19 & \begin{tabularx}{0.7\textwidth}{X} 凡離開你家門往街上去的,他的血必歸到自己頭上,與我們無關;凡在你家裡的,若有人下手害他,他的血就歸到我們頭上。 \end{tabularx} \\ \\ \relax
2:20 & \begin{tabularx}{0.7\textwidth}{X} 你若洩漏我們這件事,你叫我們所起的誓就與我們無關了。」 \end{tabularx} \\ \\ \relax
2:21 & \begin{tabularx}{0.7\textwidth}{X} 女人說:「就照你們的話吧!」於是她送他們走了,就把朱紅繩子繫在窗戶上。 \end{tabularx} \\ \\ \relax
2:22 & \begin{tabularx}{0.7\textwidth}{X} 二人離開,到山上去,在那裡停留三天,直等到追趕的人回去。追趕的人一路尋找,卻找不著。 \end{tabularx} \\ \\ \relax
2:23 & \begin{tabularx}{0.7\textwidth}{X} 二人回來,下了山,過了河,來到嫩的兒子約書亞那裡,向他報告他們所遭遇的一切事。 \end{tabularx} \\ \\ \relax
2:24 & \begin{tabularx}{0.7\textwidth}{X} 他們對約書亞說:「耶和華果然將那全地交在我們手中了,並且那地所有的居民在我們面前都融化了。」 \end{tabularx} \\ \\
[1ex]
\hline
\hline
\end{longtable}
$^{1}$今天我們會看兩段經文.
若果你對聖經熟悉,就記得有十二探子的故事.
摩西當年差派十二探子,打算進攻迦南地.
結果十二探子有十個回來回報.
他說那裡不能打,那裡是危險的地方.
結果上帝因著他們回來報惡信,懲罰他們.
結果在曠野繞了四十年.
摩西五經的最後一章生命記記載.
摩西已經站在迦南地的邊緣.
他望著耶利哥,但最後他不能進入英許地.
約書亞記接續了這個故事.
新一代的領袖約書亞準備帶領新一批的以色列民進攻耶利哥.
四十年可以練出很多每日抱怨的以色列民.
四十年的兜圈,他們有足夠條件天天向上帝抱怨.
同樣的四十年可以練出一個約書亞的人物.
今天我們會對比四十年前摩西派探子的經歷.
和四十年後身為領袖的約書亞,他又如何去派探子.
我們試過透過這個對比去認識這位新的領袖.
今天我想大家記得一個訊息.
我希望你們離開的時候仍然記得.
我們那些好像打圈,悶鬧,挫敗的日子.
其實我們並沒有白過.
在這些我們連自己都不知道發生什麼事的日子裡.
有沒有練出一個沉著應戰的你呢?.
約書亞做得到,我和你又會變成一個什麼樣的人呢?.
我們先看經文,經文記載在約書亞記的第二章.
我選取了其中的四節,中間探子去到耶利哥城的故事省略了.
我只選取了約書亞如何去猜探子,和探子回來如何回報.
如果你有印象的話,都可以回憶一下.
讓我簡單讀出這段經文.
當下聯的兒子約書亞從十庭暗暗打發兩個人作探子.
吩咐說你們去窺探那地和耶利哥.
於是二人去了,來到一個妓女名叫拉合的家裡.
就在那裡躺臥.
然後經歷過在城裡的探險,結尾22-24節這樣說.
這兩個人到山上在那裡住了三天.
等著追趕的人回去了.
追趕的人就一路找他們卻找不著.
二人就下山回來,過了河.
到亂的兒子約書亞那裡,向他陳述所遭遇的一切事.

$^{41}$又對約書亞說:耶和華果然將那全地交在我們手裡.
那地的一切居民在我面前心都消化了.
在文述記載的十三章裡記載.
四十年前摩西差派十二探子去迦納.
當年的年輕約書亞也是十二探子之一.
正如我剛才所說,當他們窺探全地回來的時候.
有十個跟眾人說那裡不能打的.
那裡是巨人,我們是蚱蜢.
結果神因著他們的不信.
於是懲罰他們漂流了四十年.
四十年後,以前的探子成為領袖.
他今天又怎樣去猜探子呢?.
我們嘗試從三個角度去分析一下.
約書亞跟摩西差派探子的不同之處.
第一,是誰決定差派探子的?.
四十年前是神親自跟摩西說.
你要差派探子去窺探這地.
但四十年後,誰決定差派探子的?.
是約書亞自己,神沒有吩咐.
約書亞自己決定的,記住第一點.
第二,四十年前所差派的探子是什麼人?.
全部是出名的領袖,十二支派的領袖.
約書亞是當年其中一個支派的領袖.
但這次約書亞差派了什麼人去?.
名字也沒有,而且是暗暗差派的.
四十年前摩西差派十二探子是非常高調.
所有民眾都知道.
回來的時候,這十二個領袖是向所有人匯報的.
但這次約書亞所做的是暗暗差派.
差派兩個沒有人認識的人去窺探這地.
回來的時候,只是向約書亞一個人匯報.
記住第二個分別.
第一個分別是約書亞自己決定.
第二個分別是暗暗的,整件事沒有人知道.
第三,四十年前那十二探子是用了四十天窺探全地的.
第四十年後,約書亞讓這兩個探子只集中在耶利哥三天就夠了.
就這三個分別,有人這樣表達.
他說約書亞不是很信神.
因為神母吩咐,他竟然自己決定派探子.
再了解一下當地的情況.

$^{81}$而且行事鬼祟,暗暗地做,不讓人知道.
甚至為什麼他決定窺探三天這麼少.
因為約書亞有個經驗.
當年探子探了四十天,結果回來不敢打仗.
神就懲罰他們在曠野漂流四十年.
現在派三天,最多三年.
三年後再來一次,怎樣都有賺.
所以有人覺得,為什麼約書亞常說要剛強壯膽.
就是他覺得約書亞可能有點膽怯.
可能不夠膽去攻打,所以做出一連串的動作.
我們轉一轉,另一個角度去想這位領袖.
第一,為什麼約書亞不窺探全地,只是窺探三天呢?.
在我來說,四十年前,約書亞自己已經看遍全地.
在四十年來,他每一天都在想怎樣攻打迦南地.
他很明白,要進迦南一定要征服耶利哥.
所以他只集中要耶利哥的資料就足夠.
其他不需要跟我說,我知道哪裡是核心.
三天就足夠,速戰速決.
第二,為什麼暗暗?.
如果你有印象,當年推翻不打迦南地的.
就是公開投票,就是有十個知名的人.
影響全族人決定不打.
約書亞說,今天我不需要投票.
我知道我們必須攻打這個城.
我只需要資料,你告訴我怎樣的情況.
我就準備攻打,我不需要民意.
因為上帝已經交了這個地方,我已經等了四十年.
第三,約書亞和摩西經歷神是很不同的事情.
摩西經歷神是經歷很多神跡歧視.
摩西跟上帝的交往是充滿神跡的.
約書亞跟上帝的交往是怎樣的?.
他一直站在戰場最前線.
他是靠實戰經歷上帝的帶領.
他沒有跟上帝有很多直接的溝通.
如果你看約書亞記的話,你都會發現.
他是站在最前線的戰士.
他只能夠運用他自己的本領,就是打仗的技巧.
我就派探子去窺探,做足準備.
然後準備配合上帝怎樣去打這場仗.
約書亞只能夠運用他自己的本領.

$^{121}$用他經歷上帝的方法.
他不能夠重複摩西的經歷.
他用自己的方法,配合上帝準備打這場仗.
與其我們覺得約書亞是一個膽怯的人.
但我們綜觀整個歷史的記載.
約書亞從來在戰場最前線.
他沒有一次退縮過.
這個領袖他已經等了40年.
當年派探子的挫敗一直記在他心裡.
如果有一天我要再派探子.
我就不會再聽任何人反面的意見.
因為我知道上帝的心意是這樣.
當我們看這段經文的時候.
我不知道比利有沒有一個很強的感受.
40年來,約書亞和摩西在曠野裡好像在繞圈子一樣.
但聖經記載,他們在曠野裡戰無不勝.
沒有打過一場失敗的仗.
現在他將要面對的是迦南裡最強的盾,耶利哥城.
耶利哥城從來沒有失手.
最強的弓,你當是雷神之鎚.
打在美國隊長的盾上.
你覺得是不是準備看一場大戰呢?.
不過故事不是這樣說的.
故事一轉,探子回報說了一個很驚人的消息.
那個妓女跟兩個探子說.
其實當年,40年前你們出埃及的時候.
我們的心已經消化了.
原來那群以為那群巨人很可怕.
很怕他們,視他們為蚱蜢.
但原來反過來,那群耶利哥巨人很怕那群出埃及的人.
因為他們在曠野裡戰無不勝.
甚至連最強的埃及都怕了他們.
原來真正懼怕的不是那群在曠野游來的人.
而是準備被攻打的巨人.
他們才是懼怕.
這種是什麼感受呢?.
我嘗試舉一個假的例子.
假的,不是真的,是假的.
你要記住是假的,跟旁邊的人說是假的.
某一天我收到一個消息.

$^{161}$原來我中學的其中一位同學已經離世.
病患離世.
一群舊的朋友,舊的同學.
去到安息禮裡.
看到這張女同學的照片已經老了.
心裡有很多回憶.
這就是自己當年暗戀,不敢表達的一位.
我太太正在看直播.
是假的.
你先讓我投入.
原來當年我一直不敢去表白.
這位女同學原來這樣就離開了.
在散籍之後,幾個舊同學一起吃東西,聊聊天.
其中有個熟悉這位女同學的同學說.
她說她這幾年晚年的時候過得不太好.
婚姻也不是很愉快.
然後她隨隨說一些往事.
她說這位女同學當年是暗戀你的.
是假的.
我不知道如果你是當中的主角.
你聽見的時候,你心裡有什麼感受.
你有什麼感受?.
不是假的,我問你有什麼感受.
我猜你若書亞聽見這個回報回來.
他感受良多.
原來當年他的堅持是真的.
他看到的那個城是上帝給他們的.
為什麼那十個人看不到這個情面.
原來上帝是預備了一份禮物給他們.
不是預備一個難關去刁難他們.
他會想起剛剛死去的師父摩西.
如果我們當年能夠拗得服眾人.
我們就不用兜圈四十年,白白這樣過.
原來我那四十年就是因為當年的失敗.
以至於我在這個日子裡打了個圈.
領子妹,我想這個故事給我很深的體會.
好像兜了個圈,好像浪費了很多日子.
不過正正是這些日子.
他卻是練出了一個沉著應戰的若書亞.
他練出了新一代的領袖.

$^{201}$他跟摩西不同,他經歷的上帝也不同.
上帝用得著當年的年輕人.
讓他帶領新一代的人進入新的階段.
這些兜圈的日子有沒有練出一個沉著應戰的你呢?.
頂次門聖經說完了,我們又轉一個話題.
我們嘗試從若書亞這個領袖.
去想想人生的成與敗.
有些成功未必是好的成功.
成功也有壞的成功.
但有些失敗也未必是一件壞事.
失敗裡仍然有一些好的元素.
有一次在銀行排隊,好像要做一些手續.
聽見前面兩位太太在分享買股票的情況.
其中聽見一位太太說.
幸好走得快,輸了幾千元.
我聽的時候也為她感恩.
輸了幾千元也這麼開心,這麼豁達.
我覺得這樣的投資心態也不錯.
事前才如糞土.
我一開始也不知道,原來她有後話.
陳太太就慘了,輸了幾萬元.
哈哈哈.
原來她的開心是建立在別人的痛苦裡.
我不知道這種成功的快樂是否一件好的東西.
我們一開始出來工作,以為賺二萬元已經很成功.
但當我們有了穩定的收入.
見到跟我們差不多工作量的人,他三萬.
當我們去到三萬,又見到五萬.
見到五萬,又見到更多.
我們不斷追求在這些比較裡.
就算不是要贏,我們也不要輸.
我們慢慢跌入這種世俗的成功裡.
頂尖的耶穌提過很重要的事.
祂叫我們不要玩贏得全世界卻賠償生命的遊戲.
道理是明白的.
但是我們不是太接受.
我們有時候在信仰,在人生裡活得疲累.
是因為我們一邊追求上帝的應許,同在的同時.
我們又不甘心在世界裡輸得太慘.
頂姐妹,如果我們想套用世俗的成功.

$^{241}$放在我們的人生裡.
我們就好像拿著一張錯的地圖.
不斷努力奔跑一樣.
就算你多努力奔跑,最後也找不到你要的東西.
因為我們的地圖本身就是錯的.
用了錯的價值觀,繼續努力去做你認為對的事.
最後也得出來是一個徒然的結果.
不是所有成功都是好的.
有些是壞的成功.
反過來,有些失敗可能是好事.
我們遇上一次失敗,可能令我們謹慎.
甚至叫我們退縮.
但是在若舒雅的經歷裡.
40年前她的堅持被人推翻.
甚至差點被人打死.
她卻是練出更深的智慧.
她更知道等候的力量.
自古有人說,失敗乃成功之母.
不過這句話並不完全對.
以色列人在40年曠野,他還不夠失敗嗎?.
他失敗還不夠多嗎?.
不是純粹一個失敗就練出成功.
唯有我們在失敗裡沉澱反思.
認真看待,了解為何失敗.
我們才練出點滴的智慧.
人長大了,我們就有勇氣回頭看.
為何當年會遇上這些挫敗.
若舒雅當年的挫敗,最主要的原因是什麼?.
是有一群豬隊友.
兩個人碰不著十個人.
碰不著十個民族.
若舒雅不是能力的問題.
不是她觀點錯誤.
就算當年有上帝,有摩西.
有十界,有雲柱,火柱.
都不能扭轉當年的局面.
若舒雅的失敗不是她個人的因素.
而是整個大環境不容許她在那一刻扭轉這件事.
我想說的是,有些失敗與你無關.
你不需要拼命攬上身.

$^{281}$就算有齊上述條件.
時機還未到嗎?.
就是還未到.
上帝仍然有空間,有心意.
在這些看似繞圈的日子裡.
塑造著我與你.
有些失敗與你無關.
但有些失敗是因為我們經驗不足.
年紀太輕,性格有些缺陷.
未能應對當年的問題.
我們的理想是對的.
但我沒有能力配合我的想法.
我也知道轉身射三分球可以反敗為勝.
但我轉身才可以.
就算我轉身也未必射得進.
道理我明白,但我真的做不到.
頂尖有時我們可能太年輕.
未懂人情世故,未懂得很多處事技巧.
我們想的方向是對的.
但未夠成熟去處理這些事.
有些失敗是讓我們更加知道.
這個世界是複雜的.
沒我想像的簡單.
有些失敗也告訴我.
我的本質是什麼.
原來有些地方我仍然可以調教自己.
我仍然可以練好那部分.
只要下一次機會再來.
我就可以應對得更好.
頂尖的成與敗不是按某一次去計算.
上帝的讚許不是出日新,不是出月新.
不是每個月評定你今天做得好不好.
你這個月做得好不好.
上帝的成敗是評定一生.
唯有我們用一生去壓上.
經歷重覆又重覆的挫敗.
經歷一次又一次的不放棄.
上帝就會對我們說.
你是忠心良善的僕人.
上帝不是按某件事做得差,做得好.

$^{321}$馬上給我們一個評語.
上帝要看我們的一生.
所以我和你不要困在人生某一次失敗裡.
不要仍然在這些沮喪的日子裡拖拖拉拉.
反倒是我們要在這些日子裡沉著.
重新去為自己定位.
究竟那一次出了什麼事.
是我人生月曆還沒夠,是我太衝動.
又或者上帝要我學懂的一些東西.
可以用在將來之上.
來總結那些看似兜圈的那些年.
我們都沒有白過.
我們經歷了很多很多不同的傷口.
我們這個群體已經有足夠的傷痛.
可以承載更多受傷的人.
我們不需要再說個別的例子.
不需要再說個別的見證.
因為在我們中間.
我們已經用每一道為主堅持的疤痕.
去述說這位上帝是真實的.
我和你都有一個故事.
是足以叫我離開這個信仰.
我和你都有一個經歷.
叫我懷疑人生.
不過我和你都一樣.
我們今天仍然坐在這裡敬拜神.
不是那個經歷沒有說服力.
而是上帝在我們人生裡壓倒性.
祂仍然色在金在.
頂姐妹如果你過去的日子覺得很疲倦.
覺得躺平是人生最快樂的.
我鼓勵你.
是時候起來.
是時候做一個像樣的基督徒.
是時候再次在這個世界裡作炎作光.
讓我們一起祈禱.
頂姐妹我知道你在你的公司.
在你的工作範圍裡.
一直堅守著信仰的底線.
你仍然在灰色地帶裡去持守上帝給你的.

$^{361}$我為你感恩.
你的堅持是有意義的.
頂姐妹你仍然面對那場不能夠治療好的病.
你仍然笑著去過這些難日子.
我為你每一次的笑容而激動.
因為你的故事正是鼓勵著我們每一個.
你仍然為很多破碎的關係努力去修補.
哪管你所盡的努力未必得到即時的回報.
但你的努力上帝看在眼裡.
我們所經歷的難日子不是徒然的.
我們在流淚殺腫的過程裡.
仍然眼睛望向歡呼羞愕的那一天.
唯有我們流過淚.
我們的笑容顯得更加寶貴.
願主藉著今天的經文.
今天的敬拜.
今天的聖餐.
對我們每一個說話.
祈禱奉主名求.
阿門.
\newpage



\section{羅馬書 16:1-16-20230624}
\label{sec:XixhhdfEXw8}
\textbf{【網上崇拜】我地唔arm beat|羅馬書16\_1-16|20230624 [XixhhdfEXw8]}
\newline
\newline
連結: \href{https://youtube.com/watch?v=XixhhdfEXw8}{\texttt{ https://youtube.com/watch?v=XixhhdfEXw8}} ~~~~ 語音日期: 2023-06-24 
\newline
\newline
\hyperref[sec:CppPjcT08EA]{\small{< < < PREV SERMON < < <}}
~
\hyperref[sec:index_chronic]{\small{[返順時目]}}
~
\hyperref[sec:index_scriptual]{\small{[返順卷目]}}
~
\hyperref[sec:I6Z1WA7E0RA]{\small{> > > NEXT SERMON > > >}}
\newline
\newline
羅馬書 16:1-16-20230624
\newline
\begin{longtable}{cl}
\hline
\hline
章節 & 經文 (和合本修訂版)\\
\hline
16:1 & \begin{tabularx}{0.7\textwidth}{X} 我對你們推薦我們的姊妹非比,她是堅革哩教會中的執事。 \end{tabularx} \\ \\ \relax
16:2 & \begin{tabularx}{0.7\textwidth}{X} 請你們在主裡用合乎聖徒的方式來接待她。她在任何事上需要你們幫助,你們就幫助她;因她素來幫助許多人,也幫助了我。 \end{tabularx} \\ \\ \relax
16:3 & \begin{tabularx}{0.7\textwidth}{X} 請向百基拉和亞居拉問安。他們在基督耶穌裡作我的同工, \end{tabularx} \\ \\ \relax
16:4 & \begin{tabularx}{0.7\textwidth}{X} 也為我的性命把自己的生死置之度外;不但我感謝他們,就是外邦的眾教會也感謝他們。 \end{tabularx} \\ \\ \relax
16:5 & \begin{tabularx}{0.7\textwidth}{X} 又向在他們家中的教會問安。向我所親愛的以拜尼土問安,他是亞細亞歸於基督的初結果子。 \end{tabularx} \\ \\ \relax
16:6 & \begin{tabularx}{0.7\textwidth}{X} 又向馬利亞問安,她為你們非常辛勞。 \end{tabularx} \\ \\ \relax
16:7 & \begin{tabularx}{0.7\textwidth}{X} 又向與我一同坐監的親戚安多尼古和猶尼亞問安,他們在使徒中是有名望的,也是比我先在基督裡的。 \end{tabularx} \\ \\ \relax
16:8 & \begin{tabularx}{0.7\textwidth}{X} 又向我在主裡面所親愛的暗伯利問安。 \end{tabularx} \\ \\ \relax
16:9 & \begin{tabularx}{0.7\textwidth}{X} 又向我們在基督裡的同工耳巴奴和我所親愛的士大古問安。 \end{tabularx} \\ \\ \relax
16:10 & \begin{tabularx}{0.7\textwidth}{X} 又向在基督裡經過考驗的亞比利問安。向亞利多布家裡的人問安。 \end{tabularx} \\ \\ \relax
16:11 & \begin{tabularx}{0.7\textwidth}{X} 又向我親戚希羅天問安。向拿其數家在主裡的人問安。 \end{tabularx} \\ \\ \relax
16:12 & \begin{tabularx}{0.7\textwidth}{X} 又向為主辛勞的土非拿和土富撒問安。向所親愛、為主非常辛勞的彼息問安。 \end{tabularx} \\ \\ \relax
16:13 & \begin{tabularx}{0.7\textwidth}{X} 又向在主裡蒙揀選的魯孚和他母親問安,他的母親就是我的母親。 \end{tabularx} \\ \\ \relax
16:14 & \begin{tabularx}{0.7\textwidth}{X} 又向亞遜其土、弗勒干、黑米、八羅巴、黑馬,和跟他們在一起的弟兄們問安。 \end{tabularx} \\ \\ \relax
16:15 & \begin{tabularx}{0.7\textwidth}{X} 又向非羅羅古和猶利亞,尼利亞和他姊妹,阿林巴和跟他們在一起的眾聖徒問安。 \end{tabularx} \\ \\ \relax
16:16 & \begin{tabularx}{0.7\textwidth}{X} 你們要以聖潔的吻彼此問安。基督的眾教會都向你們問安! \end{tabularx} \\ \\ \relax
16:17 & \begin{tabularx}{0.7\textwidth}{X} 弟兄們,那些離間你們、使你們跌倒、違背所學之道的人,我勸你們要留意躲避他們。 \end{tabularx} \\ \\ \relax
16:18 & \begin{tabularx}{0.7\textwidth}{X} 因為這樣的人不服侍我們的主基督,只服侍自己的肚腹,用花言巧語誘惑老實人的心。 \end{tabularx} \\ \\ \relax
16:19 & \begin{tabularx}{0.7\textwidth}{X} 你們的順服已經傳於眾人,所以我為你們歡喜;但我願你們在善上聰明,在惡上愚拙。 \end{tabularx} \\ \\ \relax
16:20 & \begin{tabularx}{0.7\textwidth}{X} 那賜平安的神快要把撒但踐踏在你們腳下。願我們主耶穌基督的恩與你們同在! \end{tabularx} \\ \\ \relax
16:21 & \begin{tabularx}{0.7\textwidth}{X} 我的同工提摩太,和我的親戚路求、耶孫、所西巴德,向你們問安。 \end{tabularx} \\ \\ \relax
16:22 & \begin{tabularx}{0.7\textwidth}{X} 我這代筆寫信的德提,在主裡向你們問安。 \end{tabularx} \\ \\ \relax
16:23 & \begin{tabularx}{0.7\textwidth}{X} 那接待我,也接待全教會的該猶,向你們問安。城裡的財務官以拉都和弟兄括土向你們問安。 \end{tabularx} \\ \\ \relax
16:24 & \begin{tabularx}{0.7\textwidth}{X} 願我們的主耶穌基督賜恩典給你們各位!阿們! \end{tabularx} \\ \\ \relax
16:25 & \begin{tabularx}{0.7\textwidth}{X} 惟有神能照我所傳的福音和所講的耶穌基督,並照歷代以來隱藏的奧祕的啟示,堅固你們。 \end{tabularx} \\ \\ \relax
16:26 & \begin{tabularx}{0.7\textwidth}{X} 這奧祕如今顯示出來,而且按著永生神的命令,藉眾先知的書指示萬民,使他們因信而順服。 \end{tabularx} \\ \\ \relax
16:27 & \begin{tabularx}{0.7\textwidth}{X} 願榮耀,藉著耶穌基督,歸給獨一全智的神,直到永遠。阿們! \end{tabularx} \\ \\
[1ex]
\hline
\hline
\end{longtable}
$^{1}$我們一起聽神的話之前我們一起同心同意.
神阿的國 你今天仍然做事.
你仍然在這個世界做生事.
求主你的靈現在這一刻.
囂觀在我們每一個人的心裡面.
無論我們在現場.
無論我們在不同的地方看直播.
或者我們在看重播.
我都求聖靈你現在在我們心裡面.
再一次成為我們生命的主.
去提醒我們.
讓我們知道今天你要我們做些什麼事.
以致我們能夠繼續成為主你的見證.
是在黑暗裡面的明燈.
求主你幫助我們.
教導我們去明白你自己的話.
也讓我們有勇氣有能力走在神的道當中.
我們這樣祈禱奉耶穌 聖命祈求 阿們.
今天是粵題對的最後一講.
今天我的講題就是我們不對的節奏.
我看到這個海報的時候.
有沒有人想過這個海報是怎樣看的呢?.
有沒有人想過是由.
怎樣啊?.
六合彩呢?好像那些左至右 右至上這樣.
有沒有人想過是由36244834這樣看的呢?.
還是34682344這樣看的呢?.
哎呀.
真的很不對節奏.
這個是一個實驗來的.
是我才能做到這些事.
事實就證明這個海報是不知道怎樣看的.
因為沒有人回答我.
而且我這個也跌了.
其實是怎樣看的?.
是啦 34682344.
這個是拍子來的.
謝謝弟兄.
我也知道的 因為我小時候也學過琴.
不過我怕我不對.

$^{41}$所以我這樣想.
但我第一次也想36244834是甚麼密碼呢?.
然後就發覺原來是那些節奏.
有沒有人跟我一樣以為.
謝謝 看到我這麼狼狽.
怎樣 我的樣子可以嗎?.
今天的講題是我們不對節奏.
我們不對節奏不是說我們對不上這個節奏.
有沒有人會拍三四拍?.
是這樣讀的嗎?.
三拍四.
隨便吧 我真的不會.
所以我經常在台下.
如果他說要拍手的話.
我是每一拍都拍的.
所以我是不對節奏的.
不過我今天說我們不對節奏.
不是說我們唱詩跟不上.
或者拍拍手 拍子掉了.
而是說我們.
我們也不對節奏.
聽了七個星期我們要怎樣對節奏.
最後一講我就說.
其實我們不對節奏.
但我們不對節奏其實是事實.
我們雖然也相信耶穌.
但很多時候我們也不對節奏.
我們是在同一個教會.
但我們也好像來自不同世界.
我們有不同的風格.
我們有不同的口味.
我們有不同的取向.
立場 生活方式和表現.
延伸出其實.
你不是看我多順眼.
其實我看你也一般般.
我喜歡這樣的.
不是吧 你為甚麼會喜歡這樣的東西.
你不對我的節奏.
我也不對你的節奏.

$^{81}$所以我們不對的 是不是.
真心我們不對節奏.
這件事在教會裡經常發生.
也沒有必要浪漫化.
或者刻意和諧化.
說教會不是的.
我們是一個愛的群體.
我們是合一的.
其實是的.
我們的確是合一愛的群體.
所以不對節奏我們不會大打出手.
甚至我們不會說出來.
我們表面仍然客客氣氣.
不過心裡也經常忍住.
不過今天很想跟大家說.
弟兄姊妹.
其實我們不對節奏是可以的.
教會其實是不對節奏的.
今天跟大家看羅馬書.
羅馬書可能大家都很熟悉.
因為很多耳熟能詳的金句.
我們可能也有背過.
在羅馬書.
可能我熟悉.
現在背出來可能大家也一起背得到.
一章十六節就是.
我不以封因為恥.
即封因本是神的大能.
有些人已經點頭了.
三章廿三節.
世人都犯了罪.
規決了神的榮耀.
還沒到.
我先按上去.
不讓你們看我的泡泡.
五章八節.
唯有基督撞門還俗盡的時候.
就說我們死了.
神的愛就在此.
我們顯明了.

$^{121}$一直背下去.
十二章我們很熟悉的.
所以弟兄們.
我以神的慈悲勸你們.
要怎樣.
多厲害.
將身體獻上.
我們背得到.
我們不要嚇壞這個世界.
要心意更新而變化.
十三章第一節我們也很熟悉.
再上有權病的.
我們人人都要怎樣.
信服他.
不過我們背了很多羅馬書的金句.
但羅馬書其實.
保羅寫來的原因是什麼.
保羅寫羅馬書的原因是什麼.
是為了福音.
好了.
這次我要按了.
Hey.
是左還是右.
Ey.
不是這句.
是這句.
我不以福音為恥.
即福音本是神的愛.
這句是羅馬書全章的主題.
剛才這裡.
十五章23至24節.
其實都清楚說明.
保羅寫羅馬書的原因.
好了.
不要緊.
我現在立刻問多一次.
其實我應該按左還是右.
才是下一版.
這個是右.
OK 謝謝.

$^{161}$就是這句.
十五章23至24節.
這是保羅寫羅馬書的原因.
清楚說明.
為什麼保羅要寫羅馬書.
因為他想去西班牙.
他想去西班牙傳福音.
保羅就寫信給羅馬教會.
希望羅馬教會可以.
蒙你們頌恆.
紅色了.
什麼意思.
希望羅馬教會都看到這份意象.
以至羅馬教會可以成為我的支持.
支持我的教會.
去西班牙宣教.
我們可能會想.
羅馬教會應該有一定的.
屬靈質素.
可能實力雄厚.
美好明星.
所以讓保羅看得中.
以至邀請他成為西班牙宣教的支持.
但我覺得保羅選羅馬教會.
不是因為我們心目中那份.
可能很屬靈.
所以我覺得保羅選羅馬教會.
成為西班牙的支持.
其實就是因為保羅覺得.
羅馬教會很不合格.
要認識教會有什麼方法.
最快和最直接的方法.
就是看看教會有什麼人.
或者你也試過.
剛才有位新朋友問我.
其實我今天第一次來.
很想知道教會怎樣.
現在就儘管看看.
你會看到教會怎樣.
可能你會上網.

$^{201}$可能你曾經上網.
有沒有人上過我們Flow Church HK.
但你也不太看到什麼.
你看到我們聚會地點.
連牧者的樣子都看不到.
原來阿廷是這樣的.
我們畫出來的.
你看不到什麼.
如果有社交媒體.
可能你會上我們YouTube Channel.
會上我們Facebook IG.
大約你會感覺到.
我們是一個什麼樣的教會.
你會是這樣.
但也不夠.
直接來到現場.
我們才感覺到這個教會是怎樣.
我經常在新排好揮手區.
在新排好揮手區.
經常有些新朋友.
看了很久網上宣傳.
然後終於出現.
曾經他們來過的時候.
很多時候都是這樣.
親身到現場.
他們就會覺得很特別.
因為在網上.
看到潘Sir和John很久了.
現在看到好像偶像.
真的這樣.
不知道有沒有人覺得.
原來潘Sir很高.
好像天花板那麼高.
曾經試過.
有些姊妹.
John也經常在新排好揮手區.
我介紹一下.
這位是我們創辦人陳偉安牧師.
誰啊?不認識啊.
這位就是神學士粉紅色的那位.

$^{241}$你就是神學士粉紅色的那個陳偉安啊.
很驚訝啊.
是這樣的.
甚至有些人說原來你是阿廷.
你真人比較漂亮.
今天拍我漂亮一點可以嗎?.
是啊,你要來到現場.
你才會覺得.
Full Church是怎樣的.
你才會認識到教會.
我們穿什麼衣服.
人們的風格.
我們怎麼說話.
我們的氣氛是怎樣.
我們透過一些很實際的觀察.
我們會認識教會.
今天我們看羅馬教會也一樣.
我們看看羅馬教會有什麼人.
今天跟大家看這段經文.
如果你有看我們Facebook Channel.
我們也發佈了一段經文.
是出現很多人名.
根據非正式統計.
每逢《親聖經》出現大量人名.
相信大約有80\%的弟兄姊妹.
是看了當看了.
好了,下一頁的.
按不到.
可能被我跌爛了.
對不起.
當看完吧.
我想80\%的人都有.
你看了也不想看.
加上這些人名又翹口.
出現在最後那一章.
看完前面的一至十五章.
羅馬書已經很辛苦.
你看到第十六章還有人名.
跳了,完了.
看歌不多了.

$^{281}$加上問安部分.
你會覺得只是客客氣氣.
說一下hello.
羅馬書這一串的人名.
問安說話其實非常特別.
保羅的確在每一卷書都有問安.
但其他的書卷.
保羅是向整體的基督徒問安.
在羅馬書這裡.
提及個別的信徒.
保羅刻意在這裡.
提及這二十多人名.
這二十多人名有什麼特別呢.
下一章,麻煩你.
在當時的世界,一個人名.
反映了一個人的背景,地位.
甚至生活習慣.
這段經文總共提及了26個人名.
我記下了號碼,是不是很用心.
26個人名在這裡.
有名的是24個.
另外兩個是被提名的關係.
第十三節提到.
一個叫盧夫的母親.
不知道她的名字.
總之她是他媽媽.
第十五節提到莉莉亞和她的姊妹.
總共有26個人.
提及過.
透過這二十多人名.
我們可以了解一下.
羅馬教會是一個怎樣的教會.
羅馬教會是怎樣的呢.
教會有男有女.
這26個人當中,女人佔了9個.
下一個powerpoint就看到.
那9個女人在哪裡.
第十二節.
看到我highlight了.
那三個紫色嗎.

$^{321}$就是三個女人有個「士」字.
畫本非常好.
包了個「士」字給我們.
知道就是女人名.
我們看看,很難讀她的名字.
所以我們不看很正常.
土菲娜氏和土庫薩氏.
第十四節,彼釋氏.
這些就是女人名.
有些人名一看就知道是女人.
第三節,我們很熟悉的伯基拉.
第六節,瑪利亞.
第七節,尤利亞.
這些都是女人名.
保羅特別說,第十二節.
這幾位姐妹都是為主奴婦.
第七節,他介紹尤利亞.
在使徒中有名望.
第三節,我們熟悉的伯基拉.
和保羅同工.
羅馬教會有男有女.
好像很廢.
但很重要.
在當時來說.
其實是很重要.
在當時來說.
女人可以做事是很重要.
保羅很明顯.
強調姐妹.
在羅馬教會裡.
和弟兄一樣重要.
男人和女人在教會.
都能夠獻上.
都能夠付出,承擔不同的職事.
羅馬教會有男有女.
除了有男有女之外,這教會有猶太人.
伯基拉和亞居拉.
下一頁.
這兩夫婦是.
其中兩位猶太人.

$^{361}$他們可能是這二十多個名字裡.
我們唯一認識的兩個人.
因為使徒行傳第十八章.
提及過這對猶太人夫婦.
我們看看powerpoint.
你曾經見過.
下一頁,請問.
他們曾經住過羅馬.
後來在格魯迪做王的時候.
下令趕走猶太人.
離開羅馬.
他們兩夫婦有機會.
去到哥倫多,遇見保羅.
一起去職帳拍照.
並且一起去到爾忽所.
哥倫多前書提到這兩夫婦.
開放自己的家庭.
做家庭教會.
相信這對猶太人夫婦.
在格魯迪死後,都回到羅馬.
現在再次開放自己的家庭.
再做家庭教會.
下一頁.
第七和第十一節.
我highlight了.
親屬這個字.
保羅說他們是親屬.
不是親戚.
而是同鄉.
即是猶太人.
我們能夠肯定在這二十多人名中.
至少有五個猶太人.
其他沒有特別提及的.
我們就斷估他們是外邦人.
羅馬教會有男有女.
有猶太人,也有外邦人.
這間教會.
除了這二十多人名.
有男有女,有外邦人,有猶太人.
地位階層也有些不同.

$^{401}$有些是很典型的奴隸.
或者自由人的名字.
下一頁.
我highlight了綠色的.
聖經學者推論.
這些名字.
都是奴隸的名字.
第十一節,特別是希羅天.
希羅天是一個和希律家庭.
有關的奴隸的名字.
因為中文的天.
亦叫天子.
其實原文是指屬於的意思.
屬於希律的人.
是奴隸很常有的名字.
第十二節的三個女人的名字.
都是當時很典型的女奴隸的名字.
所以這二十多人名當中.
聖經學者大約推斷.
有三分一是奴隸.
今天不會逐個多說,放心.
總而言之我們知道.
羅馬教會有些什麼人.
羅馬教會有奴隸.
同時當然有些不是奴隸.
伯羈拉亞居拉兩夫婦.
肯定不是奴隸.
羅馬教會是一個怎樣的教會.
在性別上.
有男有女.
在種族上,有猶太人.
亦有外邦人.
在階層上,有自主.
亦有遺奴.
簡單來說,羅馬教會.
是一間怎樣的教會呢.
羅馬教會是一間什麼人都有的教會.
什麼人都有的意思.
不像我們今天說的.
什麼人都有,男女性別職位.

$^{441}$甚至在外國.
有中文堂.
有普通話堂,有英文堂.
今天有其他這些,我們不是說這件事.
什麼人都有的意思.
是說到我們都好像.
來自世界,不同世界.
大家不合拍.
我們雖然相信同一位神.
我們看同一本聖經.
但因為不同背景,不同文化.
不同出身,不同傳統.
不同習俗,不同儀式.
不同喜好,等等.
這些生活表現,我們全部都不同.
你覺得正常的事.
其他人會覺得.
有沒有搞錯.
你認為一定要做的事.
有人會覺得.
你做得很無謂.
你覺得錯的事.
其他人覺得對的.
並且要堅持要做的.
簡單來說.
大家不合拍.
弟兄可不可以穿短褲.
回崇拜,大家覺得.
D導當然可以.
曾經不止一次.
不止一個.
弟兄跟我說.
有傳道人叫他不要穿短褲回崇拜.
弟兄有沒有試過.
有人舉手了.
為什麼.
對神太不尊重.
回崇拜.
衣著當然要端莊得體.
我其實都很少穿短褲回崇拜.

$^{481}$說真的.
不少人都自相教會.
回崇拜朝見至高神.
穿帥一點.
穿一條長褲迎見主.
弟兄笑我.
我心想我穿短褲才最帥.
點頭.
我穿短褲就是最帥.
為什麼不讓我穿短褲.
我穿短褲就是最尊重.
最貴就是這條褲.
什麼人都有.
不合拍.
有些教會.
年初一有初一崇拜.
有人覺得年初一有崇拜.
很有趣.
什麼意思.
如果年初一在平日.
假日崇拜.
如果那天是禮拜一崇拜.
禮拜一又要回去崇拜.
有些人覺得年初一.
你搞崇拜.
我們拜年你搞得我們很難做.
又要回教會.
又不能和家人過.
有些弟兄姐妹覺得.
年初一就是和家人過.
不就是要搞崇拜.
是不是.
什麼人都有.
不合拍.
崇拜我們唱什麼詩歌好.
傳統的生命聖詩.
青年聖歌.
等等的詩集.
歌詞非常優美.
又有神學教導.

$^{521}$曲譜嚴謹.
看譜如果你是看譜的人.
更可以即場來個四部和音大合唱.
順顯教會.
很合一.
很美妙.
但有人覺得不合音很難搞.
究竟是大慈愛還是大癡呆.
是不是.
唱一些.
近代流行詩歌.
Milk and Honey.
ACM就最好.
唱得好聽.
歌詞打動人心.
Easy.
我不唱了.
歌詞打動人心.
唱得好聽又不死板.
唱歌唱得投入.
很合感覺.
有人覺得.
唱詩歌調暗光.
感覺很好.
哭完沒人看到.
可以盡情哭.
很開心很舒服.
可以敬拜主.
但有人說我太黑黑.
什麼都看不到.
怎麼辦.
崇拜應該用白光.
夠正潔.
夠莊嚴.
什麼人都有.
不合節奏.
延伸下去.
由社會.
由教會到社會.
我們可能想到更多大大小小的例子.

$^{561}$基督徒不能說話.
社會有什麼需要當然行動.
有人說祈禱.
無論大聲.
還是小聲.
對於社會有什麼議題.
當然是要開場.
別說那麼多.
這個社會很嘈吵.
誰對誰錯.
或者根本不存在對與錯.
只是單單因為教會.
什麼人都有.
不合節奏.
我們身處同一個地方.
我們大家都是香港人.
但都好像來自不同世界.
有很多不同的想法.
羅馬教會什麼人都有.
自然都有這些不合節奏的情況.
羅馬書14章.
看到羅馬教會兩個不合節奏的情況.
第一個是.
14章第2節.
有人信百物都可吃.
但懶軟弱的只吃蔬菜.
第二個是第5節.
有人看這一天比那天強.
有人看每天都一樣.
簡單來說羅馬教會有些人.
對於食物和日子.
都有不合節奏的情況.
今天我們不會看14章.
不會詳細看.
說什麼人都有的羅馬教會.
有一批被稱為軟弱的.
吃素食菜和搜人.
這裡是講到一個.
羅馬教會很獨特的情況.
是指到所謂弱者和強者.

$^{601}$一起吃飯的時候.
即是外賢.
即是今天的聖餐.
以前的人一起吃飯的時候.
當時的情況就是.
一群所謂軟弱的.
一般認為是猶太人.
其實他們不是不吃肉.
是經過一些特別處理.
確保是潔淨的肉.
這些肉可能是在某些地方.
才能賣到.
猶太人這樣做是有原因的.
因為他們很想.
他們相信耶穌也繼續是猶太人.
他們在羅馬帝國裡面.
用食物條例.
這個潔淨條例.
去確保保存.
自己仍然是猶太人的身份.
同時確保他們仍然.
在猶太人的社群裡面.
所以他們仍然會吃.
這些有處理過的肉.
他們不是對神沒有信心.
也不是關乎那些.
拜偶像的事,是否能吃.
完全不關事,純粹是.
他們覺得這些肉不乾淨.
所以他們不吃,我們就吃菜.
大家一起聖餐.
大家一起外賢的時候,這群人就只吃菜.
好了,我們幫人沒有.
這種顧慮,所以自然.
哪裡買肉都可以.
所以當他們一起吃飯的時候,就發生了一件事.
猶太人.
不知道這些肉從哪裡來.
我不想吃了.
這些骯髒的肉,所以我就不吃.

$^{641}$就只吃菜.
可能因為這樣,那些.
什麼都吃的信徒就覺得一般般.
預備了又不吃.
甚至可能會覺得.
看不起那些吃菜的信徒.
覺得他們仍然死守規條.
甚至覺得.
很煩,我們不要再一起吃了.
不要再跟那些.
吃菜的信徒吃飯了.
本身只是.
吃肉和不吃肉.
但引伸出來,就演變成.
兩批人,大家看.
大家一般般,你覺得我死守規條.
我覺得你.
不明白我,你這些道德高地.
算了,不要煩了.
我們分開吃了.
外賢聖餐.
都不再一起吃了,守日的情況.
都類似了,猶太人繼續守.
猶太人的節日,但外邦人完全不需要.
所以就會有這樣的情況.
發生,保羅對於.
教會有不對的情況,有什麼說法呢.
保羅說了兩大原則.
第一個原則就是.
他說第三節,麻煩在這裡.
他說吃的人不要.
輕看不吃的人,不吃的人.
也不要論斷吃的人,因為神已經收納他.
只要各人心裡要意見.
堅定,保羅只是說了兩個原則.
保羅這番話是什麼意思.
就是說你不吃肉的.
就不要不吃,就繼續不吃肉.
什麼都吃的,你就什麼都吃.
你喜歡守日的.

$^{681}$你就守,你不守的.
就不要守,大家.
對自己的做法,你意見堅定.
就不需要改變.
做一樣的做法.
不過.
大家不要輕看大家.
大家也不要論斷大家.
保羅沒有嘗試.
去說服那批.
吃菜的人,跟他說.
不用這麼嚴謹,吃一兩次.
沒所謂的.
保羅沒有這樣做,保羅同意.
也理解他們.
為了保存猶太人的身份.
而繼續守食物的.
潔淨條例,同樣保羅.
也沒有說服那些愛邦的.
基督徒,要跟猶太人一樣.
要遵守這些猶太禮儀.
保羅完全沒有意圖.
要這些不同做法的人.
要有一樣的做法.
我們不需要一樣的.
我們可以不同的.
所以我們不合拍子,是可以的.
因為不是我的拍子.
不合拍子.
也不是你的拍子不合拍子.
只不過我們不同拍子.
是不是?.
三四四八,不記得了.
在神的角度.
在上帝的教會,本應有很多.
不同拍子.
因為教會什麼人都有.
教會有男有女,有自主.
有圍奴,有猶太人,有愛邦人.
這三組人,大家記得嗎?.

$^{721}$有一句我們很熟悉的金句.
在《加爾泰書》三章28節.
所提及的就是這三組人.
加爾泰書三章28節,他這樣說.
並不分猶太人,希臘人,自主.
的圍奴,或男或女.
因為你們在基督耶穌裡都成為一.
這就是福音.
這就是信耶穌最厲害的地方.
不論當時的羅馬帝國.
還是今天的香港.
有什麼團體.
有什麼群組.
可以什麼人都有.
今天做生意的商會.
講理念的政黨.
陶冶的性情的興趣班.
賣東西的店舖.
或吃東西的餐廳.
都講立場,都講理念.
都講做法.
剛好在一起.
當年羅馬帝國都一樣.
猶太人,愛邦人是不可能在同一個群體.
奴隸和主人是不可能同桌吃飯的.
男人和女人是不可能平起平坐的.
但因為耶穌.
因為福音.
所有人都沒有分別.
都可以成為一.
成為一的意思不是我們要一樣.
不是我們每個人說話要一樣.
做事要一樣.
思想要一樣.
而是像PowPo這句經典對白那樣說.
100個人可以有100個節奏.
我們即使每個人都不同節奏.
我不喜歡你的節奏.
你也不喜歡我的節奏.
我有我的堅持.

$^{761}$你有你的立場都可以.
大家看不看到這個經典對白?.
看到了.
那我不重複了.
因為福音的能力.
是要臨到每一個人.
因為神的國.
就是要什麼人都有.
只要你喜歡神的節奏.
我又喜歡神的節奏.
我們大家不喜歡節奏都可以.
不單止可以.
更加是教會應該的.
什麼人都有.
就代表著有不同的狀況.
總是有不同的想法.
不同的看法.
不同的生活表現.
我們可以不同.
我們都不需要一樣.
我們要學的是保羅給我們的原則.
彼此尊重彼此接納彼此欣賞.
其實我們每個人都知道.
不過是我們能否做到.
所以保羅也不是說.
你們分開吃飯算了.
保羅在十四章都勸.
什麼都吃的信徒.
既然你們什麼都吃.
那你就跟猶太信徒一起吃菜.
第十四章第十三節.
保羅說.
誰也不給弟兄放下半腳跌人之物.
保羅提醒他們.
不要讓這些無關痛癢的食物.
條例成為半腳石.
什麼是半腳石?.
就是不要因為這樣.
阻礙了愛邦人和猶太人.
建立屬靈的關係.

$^{801}$也不要因為這樣.
阻礙了福音的擴展.
兩批不合拍的人.
分開吃飯.
其實在今天社會來說.
是最正常.
最合宜的做法.
因為你坐在一起.
一定會面左左.
甚至可能吵架.
多一事不如少一事.
分開吃是可以的.
但分開不接觸.
原來是最不福音的做法.
合適的人坐在一起.
每個人都敢做.
有什麼特別?.
但一看就知道.
一群不合拍的人.
坐在一起吃飯.
一起聖餐.
為什麼會做到?.
為什麼這群人這麼特別?.
是因為福音的能力.
我們坐在一起.
就是要見證福音的能力.
還記得保羅為什麼要寫羅馬書嗎?.
是福音.
是想要羅馬教會.
成為他去西班牙宣教的支持教會.
羅馬教會什麼人都有.
很多不合拍的人.
聚在一起.
就是一個活生生的見證.
告訴別人福音的大能力.
但羅馬教會不是一間大教會.
不知道大家知不知道.
根據民安部分.
我們發覺羅馬.
有五個家庭教會.

$^{841}$所謂聚會點.
第五節我提到.
他問伯基拉和阿居拉在他們家的教會安.
這可能是其中一個家庭教會.
第十一節.
他問阿彌多佛家和拿其數家問安.
有可能這兩個是不同的家庭教會.
第十四十五節.
保羅一次過問候兩個組別.
就是跟他們一起的弟兄.
和他們一起的眾聖徒問安.
這些都表示.
當時信徒在不同地方聚會.
所以羅馬教會不是我們想像中央集權式的大教會.
原來不是這樣.
而是分開一個一個的家庭教會.
這些家庭教會都有不同的領袖和牧者帶領他們.
加上保羅在問安的時候.
對問安的人加了一些介紹.
例如第五節說這個二拜彌陀.
會形容他是亞細亞基督初結的果子.
為什麼要這樣介紹.
其他人不知道.
第七節說原來安多尼和尤尼亞曾經和保羅一起坐牢.
這些介紹表示了一件事.
原來這個家庭教會和家庭教會之間.
大家應該不是很熟的.
或者未必有聯繫.
第十六節保羅說.
你們要親嘴問安.
親嘴問安當時是很普遍的社交.
什麼時候和什麼情況下才可以親嘴問安.
大家知不知道.
就是見到面的時候.
如果見不到面的話是不能親嘴問安的.
意思是保羅說你們要見面.
叫這些家庭教會見面.
原來他們沒有見面.
原來這些家庭教會可能相安無事.
但保羅說你們要親嘴問安.

$^{881}$你們要見面.
你們要一起領聖餐.
你們要一起去敬拜.
保羅是期望不同的家庭教會.
可能每個教會都有自己的國籍.
背景,喜好,立場,理念.
他們各自為政.
但保羅說不要再自顧自己了.
我們要建立關係.
我們要一起崇拜.
我們要一起外延.
甚至我們要一起合作.
我們要一起分享資源.
見精神的福音.
有時有些事各自分開做.
其實更加符合國情和經濟原則.
不用和別人合作.
一起做.
還要和你們完全不合拍的人合作.
其實非常煩.
很多事要商量.
要合作才能取得一點共識.
完全不符合我們香港人的經濟原則.
但這是福音.
保羅說我們原來要這樣做.
今天我們說福音.
我們想到的是耶穌基督的福音.
釘十架.
神愛世人等等的基督徒術語.
但福音的原本意思是指好消息.
如果大家想知道更多.
我們現在直入式廣告.
可以到我們的YouTube頻道.
搜尋「神話八課」第一季的第二課.
那裡說了什麼是好消息.
大家可以重溫.
昨天我們去了第二季.
我們可以到YouTube頻道再看第一季.
保羅寫羅馬書的時候.
福音這兩個字被人聯想到什麼?.

$^{921}$聯想到羅馬帝國向國民頒布的好消息.
例如紀念某些日子.
可能我們去餐廳吃飯有七一節.
類似這些.
免費打網球.
這些可以對國民有很好的好處的消息.
當然更加包括對國家的好消息.
例如打形象.
我們可以擴展版圖.
政策我們很成功.
我們全部國民安居樂業.
羅馬和平下.
人人都開心快樂.
這些就是當時人們聽到有關的好消息.
福音的字就是這些.
就是羅馬皇帝頒布下來的東西.
歷史告訴我們.
羅馬帝國的好消息.
那些福音是用什麼方法?.
他們是用武力去擴張國界.
用權力叫人去臣服他們.
他們用政策令所有人一樣.
當羅馬帝國在那時候宣揚表面和平.
但實際上是人心很虛怯的福音的時候.
保羅就和羅馬不同的教會去呼籲他們.
我們要一起合作.
我們要宣揚真正叫人得平安的好消息.
我們要宣揚耶穌基督真正的福音.
今天這個世代我們可以怎樣宣揚.
怎樣見證耶穌基督的福音.
今天是一個分黨分派.
分裂分割的世代.
是刻意擴大人與人之間的分歧.
群體與群體之間的差異.
國與國之間的鬥爭.
為什麼要這樣做?.
某程度上是逼我們歸邊.
逼我們要講立場.
逼我們只在其中一部分.
頂尖之步我們不要跌入這個圈套.

$^{961}$耶穌基督的福音不是靠嚇的.
耶穌基督的福音也不需要靠打.
不需要靠惡.
不論是男是女.
自主的為奴的.
愛邦人還是猶太人.
什麼人都能夠保存自己的獨特性.
同時亦都能夠彼此尊重.
彼此接納.
彼此欣賞大家的差異.
神的國沒有要我們歸邊.
神的國是要我們有不同的必.
在今天要選邊站的社會.
教會如果真的有什麼取向.
立場.
生活方式.
傳統.
習俗.
什麼人都有.
甚至可以一起合作.
共同分享我們有的資源.
這就是福音的能力.
今天怎樣的人或事.
跟你不相同.
但願福音的能力再一次提醒我們.
不要跌入世界分黨分派.
分裂分割的陷阱.
我們努力令教會有.
十個.
一百個.
一千個必.
以致什麼必的人.
都能夠被耶穌基督的福音吸引.
求主幫助我們一起同心同意.
主,你是全地的主.
你愛這個世界.
你愛世界上每一個人.
求主幫助我們.
我們看的不夠闊.
我們有自己的喜好.

$^{1001}$我們有自己的眼光.
但願神的愛再一次擴闊我們的眼界.
讓我們看到人與人之間的獨特性.
也讓我們從心裡.
欣賞我們的差異.
也讓我們願意.
跟不同的人一起合作.
讓這個世界看到.
福音的能力就是這樣.
即使我們很不同.
但我們因為耶穌.
我們可以一樣.
因為耶穌.
我們可以一起合作.
因為福音是要我們這樣做.
求主幫助我們.
我們這樣祈禱.
是奉耶穌基督的聖名祈求.
阿們.
《愛的愛》 詞:陳曦 曲:陳曦.
可能你心裡也有不同的想法和回應.
可能不是剛才所說的部分.
勾起你的一些想法.
可能你也試過.
要面對一些可能.
要將你擠入一個.
的一同裡面.
將你的不同排擠了.
可能因為這樣你來到Full Church.
又或者.
可能正在看直播的你在外國.
需要和一些其他背景的信徒.
很多不同習慣的信徒.
怎樣可以一起相處.
一起去敬拜.
無論是怎樣.
讓我們這一刻.
求聖靈在我們心裡工作.
我相信.
上帝讓我們聽到今天的訊息.

$^{1041}$是有話對我們說.
是有事想我們做.
以致我們可以活出一個和這個世界不一樣.
那個生命.
那個素質.
那個群體.
讓我們這一段時間才有一段時間安靜.
想起.
剛才的訊息.
上帝藉著這個訊息和你說了些什麼.
我們有一段安靜的時候.
我們靠著自己覺得很難.
我們很容易就會和這個世界一樣.
將相對變成絕對.
將一些可以存有的不同.
可以存有的多元.
用一些冠冕堂皇的話去統一.
一定要一種的做法.
天父求你幫助我們.
活一個不同的生命.
在你的教會裡面.
我們知道我們的能力很有限.
求你的聖靈在我們心裡工作.
以致我們更大的容量.
更大的空間.
更多的愛.
和一些和我們不同的弟兄姊妹.
好好相處.
在耶穌的愛裡面.
在你的家裡面.
成為你的見證.
奉主耶穌的聖名.
阿們.
現在一首詩歌叫家.
我們聽一次.
家.
各有各背景.
有波折劇情.
請放心.
我願細心傾聽.

$^{1081}$願你共走過夜晚.
悲傷裡共參.
寄望你傷悲會消減.
閉上你眼睛.
再找到光明.
請放開.
再用心感應.
共經歷一切情況.
不管到何方.
仍夠意圖高作以傍.
家.
全憑他不怕風雨.
就在這家.
重新去建立安慰與歡說.
同經挫折與歡譽.
共建這家.
同行有著你不怕.
掃著結他和諧情.
等莫勝裡再生花.
同心血力變化.
家.
一同經歷挫折歡譽變化.
讓我們的關係可以再深化.
閉上你眼睛.
閉上你眼睛再找到光明.
請放開再用心感應.
共經歷一切情況.
不管到何方.
仍夠意圖高作以傍.
家.
全憑他不怕風雨.
就在這家.
重新去建立安慰與歡說.
同經挫折與歡譽.
共建這家.
同行有著你不怕.
掃著結他和諧情.
等莫勝裡再生花.
同心血力變化.
就算遠走仍然可跪到此處.

$^{1121}$有聚有分仍然不怕陌生.
愛中相處.
若數永遠在此處.
全為了他情可不變不愛.
這就要加.
全因這裡仲充滿你的愛.
同心血力去開.
就算遠走仍然可跪到此處.
有聚有分仍然不怕陌生.
愛中相處.
若數永遠在此處.
全為了他情可不變不愛.
這就要加.
全因這裡仲充滿你的愛.
同心血力去愛.
全為了他情可不變不愛.
這就要加.
全因這裡仲充滿你的愛.
同心血力去愛.
求主幫助我們.
讓我們心會清楚.
\newpage



\section{}
\label{sec:I6Z1WA7E0RA}
\textbf{《致餘民及流散者:給香港基督徒的神學八課》第二季第5課|20230625 [I6Z1WA7E0RA]}
\newline
\newline
連結: \href{https://youtube.com/watch?v=I6Z1WA7E0RA}{\texttt{ https://youtube.com/watch?v=I6Z1WA7E0RA}} ~~~~ 語音日期: 2023-06-25 
\newline
\newline
\hyperref[sec:XixhhdfEXw8]{\small{< < < PREV SERMON < < <}}
~
\hyperref[sec:index_chronic]{\small{[返順時目]}}
~
\hyperref[sec:index_scriptual]{\small{[返順卷目]}}
~
\hyperref[sec:J_OpyaPLYIE]{\small{> > > NEXT SERMON > > >}}
\newline
\newline
$^{1}$我只想知道.
你到底是什麼意思.
我只想知道.
你到底是什麼意思.
我只想知道.
你到底是什麼意思.
我只想知道.
你到底是什麼意思.
我只想知道.
你到底是什麼意思.
我只想知道.
你到底是什麼意思.
我只想知道.
你到底是什麼意思.
我只想知道.
你到底是什麼意思.
我只想知道.
你到底是什麼意思.
我只想知道.
你到底是什麼意思.
我只想知道.
你到底是什麼意思.
我只想知道.
你到底是什麼意思.
我只想知道.
你到底是什麼意思.
我只想知道.
你到底是什麼意思.
我只想知道.
你到底是什麼意思.
我只想知道.
你到底是什麼意思.
我只想知道.
你到底是什麼意思.
我只想知道.
你到底是什麼意思.
我只想知道.
你到底是什麼意思.
我只想知道.
你到底是什麼意思.

$^{41}$我只想知道.
你到底是什麼意思.
我只想知道.
你到底是什麼意思.
我只想知道.
你到底是什麼意思.
我只想知道.
你到底是什麼意思.
我只想知道.
你到底是什麼意思.
我只想知道.
你到底是什麼意思.
我只想知道.
你到底是什麼意思.
我只想知道.
你到底是什麼意思.
我只想知道.
你到底是什麼意思.
我只想知道.
你到底是什麼意思.
我只想知道.
你到底是什麼意思.
我只想知道.
你到底是什麼意思.
我只想知道.
你到底是什麼意思.
我只想知道.
你到底是什麼意思.
我只想知道.
你到底是什麼意思.
我只想知道.
你到底是什麼意思.
我只想知道.
你到底是什麼意思.
我只想知道.
你到底是什麼意思.
我只想知道.
你到底是什麼意思.
我只想知道.
你到底是什麼意思.

$^{81}$我只想知道.
你到底是什麼意思.
我只想知道.
你到底是什麼意思.
我只想知道.
你到底是什麼意思.
我只想知道.
你到底是什麼意思.
我只想知道.
你到底是什麼意思.
我只想知道.
你到底是什麼意思.
我只想知道.
你到底是什麼意思.
我只想知道.
你到底是什麼意思.
我只想知道.
你到底是什麼意思.
我只想知道.
你到底是什麼意思.
我只想知道.
你到底是什麼意思.
我只想知道.
你到底是什麼意思.
我只想知道.
你到底是什麼意思.
我只想知道.
你到底是什麼意思.
我只想知道.
你到底是什麼意思.
我只想知道.
你到底是什麼意思.
我只想知道.
你到底是什麼意思.
我只想知道.
你到底是什麼意思.
我只想知道.
你到底是什麼意思.
我只想知道.
你到底是什麼意思.

$^{121}$我只想知道.
你到底是什麼意思.
我只想知道.
你到底是什麼意思.
我只想知道.
你到底是什麼意思.
我只想知道.
你到底是什麼意思.
我只想知道.
你到底是什麼意思.
我只想知道.
你到底是什麼意思.
我只想知道.
你到底是什麼意思.
我只想知道.
你到底是什麼意思.
我只想知道.
你到底是什麼意思.
我只想知道.
你到底是什麼意思.
我只想知道.
你到底是什麼意思.
我只想知道.
你到底是什麼意思.
我只想知道.
你到底是什麼意思.
我只想知道.
你到底是什麼意思.
我只想知道.
你到底是什麼意思.
我只想知道.
你到底是什麼意思.
我只想知道.
你到底是什麼意思.
我只想知道.
你到底是什麼意思.
我只想知道.
你到底是什麼意思.
我只想知道.
你到底是什麼意思.

$^{161}$我只想知道.
你到底是什麼意思.
我只想知道.
你到底是什麼意思.
我只想知道.
你到底是什麼意思.
我只想知道.
你到底是什麼意思.
我只想知道.
你到底是什麼意思.
我只想知道.
你到底是什麼意思.
我只想知道.
你到底是什麼意思.
我只想知道.
你到底是什麼意思.
我只想知道.
你到底是什麼意思.
我只想知道.
你到底是什麼意思.
我只想知道.
你到底是什麼意思.
我只想知道.
你到底是什麼意思.
我只想知道.
你到底是什麼意思.
我只想知道.
你到底是什麼意思.
我只想知道.
你到底是什麼意思.
我只想知道.
你到底是什麼意思.
我只想知道.
你到底是什麼意思.
我只想知道.
你到底是什麼意思.
我只想知道.
你到底是什麼意思.
我只想知道.
你到底是什麼意思.

$^{201}$我只想知道.
你到底是什麼意思.
我只想知道.
你到底是什麼意思.
我只想知道.
你到底是什麼意思.
我只想知道.
你到底是什麼意思.
我只想知道.
你到底是什麼意思.
我只想知道.
你到底是什麼意思.
我只想知道.
你到底是什麼意思.
我只想知道.
你到底是什麼意思.
我只想知道.
你到底是什麼意思.
我只想知道.
你到底是什麼意思.
我只想知道.
你到底是什麼意思.
我只想知道.
你到底是什麼意思.
我只想知道.
你到底是什麼意思.
我只想知道.
你到底是什麼意思.
我只想知道.
你到底是什麼意思.
我只想知道.
你到底是什麼意思.
我只想知道.
你到底是什麼意思.
我只想知道.
你到底是什麼意思.
我只想知道.
你到底是什麼意思.
我只想知道.
你到底是什麼意思.

$^{241}$我只想知道.
你到底是什麼意思.
我只想知道.
你到底是什麼意思.
我只想知道.
你到底是什麼意思.
我只想知道.
你到底是什麼意思.
我只想知道.
你到底是什麼意思.
我只想知道.
你到底是什麼意思.
我只想知道.
你到底是什麼意思.
我只想知道.
你到底是什麼意思.
我只想知道.
你到底是什麼意思.
我只想知道.
你到底是什麼意思.
我只想知道.
你到底是什麼意思.
我只想知道.
你到底是什麼意思.
我只想知道.
你到底是什麼意思.
我只想知道.
你到底是什麼意思.
我只想知道.
你到底是什麼意思.
我只想知道.
你到底是什麼意思.
我只想知道.
你到底是什麼意思.
我只想知道.
你到底是什麼意思.
我只想知道.
你到底是什麼意思.
我只想知道.
你到底是什麼意思.

$^{281}$我只想知道.
你到底是什麼意思.
我只想知道.
你到底是什麼意思.
我只想知道.
你到底是什麼意思.
我只想知道.
你到底是什麼意思.
我只想知道.
你到底是什麼意思.
我只想知道.
你到底是什麼意思.
我只想知道.
你到底是什麼意思.
我只想知道.
你到底是什麼意思.
我只想知道.
你到底是什麼意思.
我只想知道.
你到底是什麼意思.
我只想知道.
你到底是什麼意思.
我只想知道.
你到底是什麼意思.
我只想知道.
你到底是什麼意思.
我只想知道.
你到底是什麼意思.
我只想知道.
你到底是什麼意思.
我只想知道.
你到底是什麼意思.
我只想知道.
你到底是什麼意思.
我只想知道.
你到底是什麼意思.
我只想知道.
你到底是什麼意思.
我只想知道.
你到底是什麼意思.

$^{321}$我只想知道.
你到底是什麼意思.
我只想知道.
你到底是什麼意思.
我只想知道.
你到底是什麼意思.
我只想知道.
你到底是什麼意思.
我只想知道.
你到底是什麼意思.
我只想知道.
你到底是什麼意思.
我只想知道.
你到底是什麼意思.
我只想知道.
你到底是什麼意思.
我只想知道.
你到底是什麼意思.
我只想知道.
你到底是什麼意思.
我只想知道.
你到底是什麼意思.
我只想知道.
你到底是什麼意思.
我只想知道.
你到底是什麼意思.
我只想知道.
你到底是什麼意思.
我只想知道.
你到底是什麼意思.
我只想知道.
你到底是什麼意思.
我只想知道.
你到底是什麼意思.
我只想知道.
你到底是什麼意思.
我只想知道.
你到底是什麼意思.
我只想知道.
你到底是什麼意思.

$^{361}$我只想知道.
你到底是什麼意思.
我只想知道.
你到底是什麼意思.
我只想知道.
你到底是什麼意思.
我只想知道.
你到底是什麼意思.
我只想知道.
你到底是什麼意思.
我只想知道.
你到底是什麼意思.
我只想知道.
你到底是什麼意思.
我只想知道.
你到底是什麼意思.
我只想知道.
你到底是什麼意思.
我只想知道.
你到底是什麼意思.
我只想知道.
你到底是什麼意思.
我只想知道.
你到底是什麼意思.
我只想知道.
你到底是什麼意思.
我只想知道.
你到底是什麼意思.
我只想知道.
你到底是什麼意思.
我只想知道.
你到底是什麼意思.
我只想知道.
你到底是什麼意思.
我只想知道.
你到底是什麼意思.
我只想知道.
你到底是什麼意思.
我只想知道.
你到底是什麼意思.

$^{401}$我只想知道.
你到底是什麼意思.
我只想知道.
你到底是什麼意思.
我只想知道.
你到底是什麼意思.
我只想知道.
你到底是什麼意思.
我只想知道.
你到底是什麼意思.
我只想知道.
你到底是什麼意思.
我只想知道.
你到底是什麼意思.
我只想知道.
你到底是什麼意思.
我只想知道.
你到底是什麼意思.
我只想知道.
你到底是什麼意思.
我只想知道.
你到底是什麼意思.
我只想知道.
你到底是什麼意思.
我只想知道.
你到底是什麼意思.
我只想知道.
你到底是什麼意思.
我只想知道.
你到底是什麼意思.
我只想知道.
你到底是什麼意思.
我只想知道.
你到底是什麼意思.
我只想知道.
你到底是什麼意思.
我只想知道.
你到底是什麼意思.
我只想知道.
你到底是什麼意思.

$^{441}$我只想知道.
你到底是什麼意思.
我只想知道.
你到底是什麼意思.
我只想知道.
你到底是什麼意思.
我只想知道.
你到底是什麼意思.
我只想知道.
你到底是什麼意思.
我只想知道.
你到底是什麼意思.
我只想知道.
你到底是什麼意思.
我只想知道.
你到底是什麼意思.
我只想知道.
你到底是什麼意思.
我只想知道.
你到底是什麼意思.
我只想知道.
你到底是什麼意思.
我只想知道.
你到底是什麼意思.
我只想知道.
你到底是什麼意思.
我只想知道.
你到底是什麼意思.
我只想知道.
你到底是什麼意思.
我只想知道.
你到底是什麼意思.
我只想知道.
你到底是什麼意思.
我只想知道.
你到底是什麼意思.
我只想知道.
你到底是什麼意思.
我只想知道.
你到底是什麼意思.

$^{481}$我只想知道.
你到底是什麼意思.
我只想知道.
你到底是什麼意思.
我只想知道.
你到底是什麼意思.
我只想知道.
你到底是什麼意思.
我只想知道.
你到底是什麼意思.
我只想知道.
你到底是什麼意思.
我只想知道.
你到底是什麼意思.
我只想知道.
你到底是什麼意思.
我只想知道.
你到底是什麼意思.
我只想知道.
你到底是什麼意思.
我只想知道.
你到底是什麼意思.
我只想知道.
你到底是什麼意思.
我只想知道.
你到底是什麼意思.
我只想知道.
你到底是什麼意思.
我只想知道.
你到底是什麼意思.
我只想知道.
你到底是什麼意思.
我只想知道.
你到底是什麼意思.
我只想知道.
你到底是什麼意思.
我只想知道.
你到底是什麼意思.
我只想知道.
你到底是什麼意思.

$^{521}$我只想知道.
你到底是什麼意思.
我只想知道.
你到底是什麼意思.
我只想知道.
你到底是什麼意思.
我只想知道.
你到底是什麼意思.
我只想知道.
你到底是什麼意思.
我只想知道.
你到底是什麼意思.
我只想知道.
你到底是什麼意思.
我只想知道.
你到底是什麼意思.
我只想知道.
你到底是什麼意思.
我只想知道.
你到底是什麼意思.
我只想知道.
你到底是什麼意思.
我只想知道.
你到底是什麼意思.
我只想知道.
你到底是什麼意思.
我只想知道.
你到底是什麼意思.
我只想知道.
你到底是什麼意思.
我只想知道.
你到底是什麼意思.
我只想知道.
你到底是什麼意思.
我只想知道.
你到底是什麼意思.
我只想知道.
你到底是什麼意思.
我只想知道.
你到底是什麼意思.

$^{561}$我只想知道.
你到底是什麼意思.
我只想知道.
你到底是什麼意思.
我只想知道.
你到底是什麼意思.
我只想知道.
你到底是什麼意思.
我只想知道.
你到底是什麼意思.
我只想知道.
你到底是什麼意思.
我只想知道.
你到底是什麼意思.
我只想知道.
你到底是什麼意思.
我只想知道.
你到底是什麼意思.
我只想知道.
你到底是什麼意思.
我只想知道.
你到底是什麼意思.
我只想知道.
你到底是什麼意思.
我只想知道.
你到底是什麼意思.
我只想知道.
你到底是什麼意思.
我只想知道.
你到底是什麼意思.
我只想知道.
你到底是什麼意思.
我只想知道.
你到底是什麼意思.
我只想知道.
你到底是什麼意思.
我只想知道.
你到底是什麼意思.
我只想知道.
你到底是什麼意思.

$^{601}$我只想知道.
你到底是什麼意思.
我只想知道.
你到底是什麼意思.
我只想知道.
你到底是什麼意思.
我只想知道.
你到底是什麼意思.
我只想知道.
你到底是什麼意思.
我只想知道.
你到底是什麼意思.
我只想知道.
你到底是什麼意思.
我只想知道.
你到底是什麼意思.
我只想知道.
你到底是什麼意思.
我只想知道.
你到底是什麼意思.
我只想知道.
你到底是什麼意思.
我只想知道.
你到底是什麼意思.
我只想知道.
你到底是什麼意思.
我只想知道.
你到底是什麼意思.
我只想知道.
你到底是什麼意思.
我只想知道.
你到底是什麼意思.
我只想知道.
你到底是什麼意思.
我只想知道.
你到底是什麼意思.
我只想知道.
你到底是什麼意思.
我只想知道.
你到底是什麼意思.

$^{641}$我只想知道.
你到底是什麼意思.
我只想知道.
你到底是什麼意思.
我只想知道.
你到底是什麼意思.
我只想知道.
你到底是什麼意思.
我只想知道.
你到底是什麼意思.
我只想知道.
你到底是什麼意思.
我只想知道.
你到底是什麼意思.
我只想知道.
你到底是什麼意思.
我只想知道.
你到底是什麼意思.
我只想知道.
你到底是什麼意思.
我只想知道.
你到底是什麼意思.
我只想知道.
你到底是什麼意思.
我只想知道.
你到底是什麼意思.
我只想知道.
你到底是什麼意思.
我只想知道.
你到底是什麼意思.
我只想知道.
你到底是什麼意思.
我只想知道.
你到底是什麼意思.
我只想知道.
你到底是什麼意思.
我只想知道.
你到底是什麼意思.
我只想知道.
你到底是什麼意思.

$^{681}$我只想知道.
你到底是什麼意思.
我只想知道.
你到底是什麼意思.
我只想知道.
你到底是什麼意思.
我只想知道.
你到底是什麼意思.
我只想知道.
你到底是什麼意思.
我只想知道.
你到底是什麼意思.
我只想知道.
你到底是什麼意思.
我只想知道.
你到底是什麼意思.
我只想知道.
你到底是什麼意思.
我只想知道.
你到底是什麼意思.
我只想知道.
你到底是什麼意思.
我只想知道.
你到底是什麼意思.
我只想知道.
你到底是什麼意思.
我只想知道.
你到底是什麼意思.
我只想知道.
你到底是什麼意思.
我只想知道.
你到底是什麼意思.
我只想知道.
你到底是什麼意思.
我只想知道.
你到底是什麼意思.
我只想知道.
你到底是什麼意思.
我只想知道.
你到底是什麼意思.

$^{721}$我只想知道.
你到底是什麼意思.
我只想知道.
你到底是什麼意思.
我只想知道.
你到底是什麼意思.
我只想知道.
你到底是什麼意思.
我只想知道.
你到底是什麼意思.
我只想知道.
你到底是什麼意思.
我只想知道.
你到底是什麼意思.
我只想知道.
你到底是什麼意思.
我只想知道.
你到底是什麼意思.
我只想知道.
你到底是什麼意思.
我只想知道.
你到底是什麼意思.
我只想知道.
你到底是什麼意思.
我只想知道.
你到底是什麼意思.
我只想知道.
你到底是什麼意思.
我只想知道.
你到底是什麼意思.
我只想知道.
你到底是什麼意思.
我只想知道.
你到底是什麼意思.
我只想知道.
你到底是什麼意思.
我只想知道.
你到底是什麼意思.
我只想知道.
你到底是什麼意思.

$^{761}$我只想知道.
你到底是什麼意思.
我只想知道.
你到底是什麼意思.
我只想知道.
你到底是什麼意思.
我只想知道.
你到底是什麼意思.
我只想知道.
你到底是什麼意思.
我只想知道.
你到底是什麼意思.
我只想知道.
你到底是什麼意思.
我只想知道.
你到底是什麼意思.
我只想知道.
你到底是什麼意思.
我只想知道.
你到底是什麼意思.
我只想知道.
你到底是什麼意思.
我只想知道.
你到底是什麼意思.
我只想知道.
你到底是什麼意思.
我只想知道.
你到底是什麼意思.
我只想知道.
你到底是什麼意思.
我只想知道.
你到底是什麼意思.
我只想知道.
你到底是什麼意思.
我只想知道.
你到底是什麼意思.
我只想知道.
你到底是什麼意思.
我只想知道.
你到底是什麼意思.

$^{801}$我只想知道.
你到底是什麼意思.
我只想知道.
你到底是什麼意思.
我只想知道.
你到底是什麼意思.
我只想知道.
你到底是什麼意思.
我只想知道.
你到底是什麼意思.
我只想知道.
你到底是什麼意思.
我只想知道.
你到底是什麼意思.
我只想知道.
你到底是什麼意思.
我只想知道.
你到底是什麼意思.
我只想知道.
你到底是什麼意思.
我只想知道.
你到底是什麼意思.
我只想知道.
你到底是什麼意思.
我只想知道.
你到底是什麼意思.
我只想知道.
你到底是什麼意思.
我只想知道.
你到底是什麼意思.
我只想知道.
你到底是什麼意思.
我只想知道.
你到底是什麼意思.
我只想知道.
你到底是什麼意思.
我只想知道.
你到底是什麼意思.
我只想知道.
你到底是什麼意思.

$^{841}$我只想知道.
你到底是什麼意思.
我只想知道.
你到底是什麼意思.
我只想知道.
你到底是什麼意思.
我只想知道.
你到底是什麼意思.
我只想知道.
你到底是什麼意思.
我只想知道.
你到底是什麼意思.
我只想知道.
你到底是什麼意思.
我只想知道.
你到底是什麼意思.
我只想知道.
你到底是什麼意思.
我只想知道.
你到底是什麼意思.
我只想知道.
你到底是什麼意思.
我只想知道.
你到底是什麼意思.
我只想知道.
你到底是什麼意思.
我只想知道.
你到底是什麼意思.
我只想知道.
你到底是什麼意思.
我只想知道.
你到底是什麼意思.
我只想知道.
你到底是什麼意思.
我只想知道.
你到底是什麼意思.
我只想知道.
你到底是什麼意思.
我只想知道.
你到底是什麼意思.

$^{881}$我只想知道.
你到底是什麼意思.
我只想知道.
你到底是什麼意思.
我只想知道.
你到底是什麼意思.
我只想知道.
你到底是什麼意思.
我只想知道.
你到底是什麼意思.
我只想知道.
你到底是什麼意思.
我只想知道.
你到底是什麼意思.
我只想知道.
你到底是什麼意思.
我只想知道.
你到底是什麼意思.
我只想知道.
你到底是什麼意思.
我只想知道.
你到底是什麼意思.
我只想知道.
你到底是什麼意思.
我只想知道.
你到底是什麼意思.
我只想知道.
你到底是什麼意思.
我只想知道.
你到底是什麼意思.
我只想知道.
你到底是什麼意思.
我只想知道.
你到底是什麼意思.
我只想知道.
你到底是什麼意思.
我只想知道.
你到底是什麼意思.
我只想知道.
你到底是什麼意思.

$^{921}$我只想知道.
你到底是什麼意思.
我只想知道.
你到底是什麼意思.
我只想知道.
你到底是什麼意思.
我只想知道.
你到底是什麼意思.
我想說這個.
Thomas Aquinas的目的是做什麼.
他想嘗試去看著這個世界.
從而去推論這個世界背後的上帝.
所以他覺得這個世界和上帝之間.
是有一定程度的結連.
你看著這個世界.
是可以找回上帝的足跡.
這是一個很重要的進路.
你看著世界里的一些事情.
你能夠找到一些理性的原因.
跡象去找回上帝的存在.
但問題是.
究竟上帝存在是什麼意思.
上帝存在是不是我們能夠理解的東西.
上帝存在和你媽媽存在是不是一回事.
上帝存在的意思是不是普通的肉眼看到.
或者是起到的意思.
這是一個很深的問題.
所以有些人覺得上帝不存在.
因為上帝不是我們平時所想的那一切的存在.
所以上帝可以說是不存在的.
因為不是按著人世間的存在理解去存在.
所以我們不要嘗試用一般我們平時理解的方式.
去理解上帝的存在.
他未必一定有一個手摸到你.
或者未必一定有一個緊緊的擁抱抱著你.
或者是一把聲音讓你聽到.
這樣的存在.
所以上帝不是那一種的存在.
所以如果我們用我們平時所理解的存在方法.
對的上帝是不存在的.

$^{961}$不是那一種的存在.
所以我們今天想說的是.
我們可以想多一層.
究竟對你來說上帝起到是什麼意思.
上帝起到因為我今天追巴士的時候被我追到.
然後你就覺得這是上帝的恩典.
你可以這樣去論證他的存在.
我很肚痛,今天上帝神父去證實我.
所以我就要認罪了.
這也是一種你自己去理解的.
所以上帝存在的意思.
如果用一般的理解可能不是這樣.
所以今天純粹想上帝的起到.
上帝的absence.
我們嘗試不是用一般你去理解一個人在哪裡.
或者一件事物在哪裡的方式存在.
因為上帝不是這個世界.
所以我們不能用世界的事情存在的方式.
來理解上帝的存在.
接著想說的是潘福華.
潘福華雖然是一生裡面去對抗希特勒.
但其實他在坐牢的時候.
寫了幾封信去討論一些很有深度的問題.
他寫給了他的朋友.
有幾封信談論到將來基督教的發展.
如果他沒有被處死刑的時候.
信仰是一回什麼事情.
基督教是一回什麼事情.
是一句很震撼的話.
其實我之前也聽過.
所謂的吸靈的世界.
或者叫做非宗教的教程.
我也說過.
但當我預備這一課的時候.
我再看那封信原本的資料的時候.
其實他所說的話會更加深入.
他說上帝告訴我們.
我們必須以沒有上帝的方式去生活.
與我們同在的上帝是離棄我們的上帝.
所以他想說的是.

$^{1001}$我們去到20世紀的德國.
去到20世紀的世界.
這個世界是一個現代世界.
他稱之為一個吸靈.
是一個成熟的世界.
我們這個世界似乎開始不需要靠上帝生活.
你不是原始人.
你不是一些澳洲的土著.
我們的世界.
我否認開始去到某個位置.
你開始能夠認知很多的東西.
你知道原來下雨是一回事.
你不會求雨神.
你知道下雨是因為天文台有個低壓槽.
下雨你會知道那個科學背後是什麼事情.
我說過Virus也是一樣.
Virus是這八年才發現的.
一千年前的人看到病毒的時候.
他們以為是瘟疫.
瘟疫是一個配合某些神秘的東西.
所以我們這個世界去到這個位置.
是開始去到吸靈的世界.
我們其實不需要有一個上帝來解釋很多的東西.
當然你現在有很多東西是需要解釋的.
解釋不了的.
譬如說癌症的出現.
你都會求一些奇跡神跡來幫助你.
有些東西你仍然解釋不了.
譬如說你的手機壞了你會祈禱.
除非你是IT人.
IT人就會重新安排你的手機.
有些東西你解釋不了的時候.
你就會開始去找這位上帝.
問題是我們這個年代.
我們越來越少的東西解釋不了.
我們越來越多.
我們明白當中背後的運作.
很多東西你都會開始發現.
原來背後某些道理某些真相是這樣.
而潘福華是怎麼看這件事呢.

$^{1041}$他說其實是上帝知道的.
上帝告訴他.
我們開始需要以一個沒有上帝去生活的方式生活.
這不是一件冒犯的事情.
人類的知識到了某個水平的時候.
我們開始不需要有上帝作為解釋原因.
而我們同時可以相信上帝.
這正是困難的地方.
我不知道你怎麼相信耶穌.
可能是某些你覺得很解決不了的事情.
所以你有一刻相信耶穌.
但有一刻有一些事情是解釋不了.
或者是處理不了.
唯有靠神去幫助你.
但是當你處理到這些事情.
那你怎麼辦呢.
所以潘福華說.
上帝是開始以一種.
好像不需要有上帝去生活的方式.
就現在這樣的年代來生活.
和我們同在的上帝.
同時是離棄的上帝.
然後潘福華就曲了十字架的道理.
所謂離棄上帝是一位被釘十字架的上帝.
我們所相信的是一位這樣的上帝.
接著下面有一句.
我在另一些講座也有說過這句話.
但原來不知道兩句話是連在一起的.
他說讓我們再沒有上帝工作的假設下生活的上帝.
就是我們不斷要面對的上帝.
我們開始面對著這樣的生活方法.
好像你不需要上帝的方式來生活.
我們在上帝面前.
和上帝同在同時.
卻在一個沒有上帝的情況下生活.
不知道你明白這句話的意思嗎.
你明白上帝同在.
但你上班.
你上班的時候不需要靠上帝來生活.
坦白說.

$^{1081}$你不做就會被人罵.
你開會有些事情要處理.
你祈禱是需要.
但你確實有些事情要做.
你開車會被人拍照.
被警察抓了就會被抓了.
這些祈禱是沒用的.
你會祈禱祈禱上帝.
但你不會有成績.
我講過很多次這個例子.
我曾經在長洲.
一大早就乘船出來.
我講過這個例子.
我去長洲碼頭乘船.
突然發現我的錢包不見了.
那時候我繼續祈禱.
剛才發現原來不是我錢包不見了.
而是我沒有帶錢包.
那我會怎麼祈禱.
如果你是我的話.
我講船.
還有五分鐘就要開船.
那我怎麼辦.
我可以有很多方法去祈禱.
但我會選擇祈禱.
可能遇到一個同事借錢給我乘船.
但你不會祈禱.
錢包會從山裡飛下來.
你不會祈禱.
瞬間轉移去中環.
你可以這樣祈禱.
但你不會這樣祈禱.
所以你會用一個現代人.
去處理生活世界的方式去祈禱.
明白我的意思嗎.
所以我們在這樣的世界生活.
你是一個很普通的人.
你會用一些世界的方式去處理世界.
這絕對不是不屬靈的事.
教會要開會.

$^{1121}$師班要練師.
講道義 社會上的講章.
全部都是很重要的事.
所以我們是以一種.
照著世界的運作方法去過生活.
同時你相信上帝的同在.
不過卻被要求.
在一個沒有上帝的情況下去生活.
上帝讓自己被驅逐出世界.
被釘在十字架上.
上帝在世界是無能和軟弱的.
正是這樣.
他才是與我們同在並幫助我們.
十字架那段經文.
基督不是憑借祂的全能來幫助.
而是憑借祂的軟弱和受苦.
這是與所有宗教的關鍵區別.
人的宗教性將他在困境中指向上帝.
在世界中的權能.
上帝是機械中的神.
即是說人的宗教性.
往往是讓我們去發明一個這樣的神.
簡單來說就是黃大仙的神.
我們人的宗教.
一個人為的宗教.
就會期盼一個這樣的神.
即是有突然間某些事情可以幫助你.
一個突然間的聖徒來解救你.
為什麼這個叫做Deus Ex Machina.
就是在以前希臘的悲劇里.
最後會有一個用機械調出來的神.
我就是神.
我幫你解決問題.
然後就完.
是一個很差的本來的.
即是說一個超烏雞的神出現就完了.
一個這樣的神.
一個大團結的神.
所以我們宗教似乎是期待一個這樣的上帝.
一個能夠來一個什麼事都是上帝解決完.

$^{1161}$完美結局的上帝.
但是他說聖經上帝不是這樣.
聖經上帝是十字架上被釘死的上帝.
是一個完全不上帝的上帝.
上帝在受苦的上帝里才能夠幫助我們.
他說正正是因為這位十字架的上帝.
才在一個靠近的世界裡面是有意義的.
不知道你明不明白我的意思.
一個靠近的世界.
一個Modern World.
不再需要一個這樣的宗教.
因為很多東西你都能解釋得到.
不需要一個上帝.
反過來說.
一個在十字架裡面受苦的上帝.
才能夠繼續在靠近的世界裡面存在.
所以十字架本身就是一個.
去對抗一切宗教的一回事.
這幾段時間我經常提這個詞.
我在Full Church講道裡面都說.
我不想Full Church成為一個搞得很大的宗教.
純粹去滿足一些宗教人士的宗教需要.
安慰你,疼愛你,寶寶.
這樣的一個上帝.
上帝在一個靠近的世界裡面.
不是純粹滿足人的宗教需要.
所以潘福華最後的結論.
就是一個非宗教的基督教.
一個再沒有宗教包裝外殼的基督教.
我覺得起碼對Full Church來說.
我是一種這樣的圖畫.
你看到我們很小事.
講道不穿西裝.
這件事本身也是這樣的事.
不講那麼多宗教儀式.
我叫穆斯普,但穆家族做的沒穿過 呼出裡面我沒穿過穆斯普的.
基本上是想將宗教的味道脫去 希望你平時所生活的香港就是你上教會的香港.
不會是兩個世界來的 所謂呼出能夠希望貼地.
希望我們的信仰能夠和大家生活一樣的信仰.
不是一個脫離香港政治困擾的地方 而是可以讓你發覺是同一個地方.

$^{1201}$所以我希望有一個非宗教的基督教.
可以和大家討論一下 怎樣是一個非宗教的基督教 不再有一種迷信任何半點人為宗教氣氛的基督教.
而是一個真真正正的基督教.
接著我們就會說 今天說的內容比較深.
說完潘福牙之後說一下莫特曼.
莫特曼在一本書 《丁十字架的上帝》裡面 其實這本書是說十字架神學.
有一個很有意思的點子 說十字架是一個對抗一切有神論信仰的事情.
什麼意思呢 十字架正正就是一個推翻人世間宗教信仰的一回事情.
所以我很喜歡聖經的一句話.
不知道大家有沒有留意過這句很有型的說話 聖經裡面保羅說的.
在《迦太史》裡面寫的.
「弟兄們,我若仍舊傳國禮,為什麼還不受逼迫呢?若是這樣,那十字架討厭的地方就沒有了.」.
這是和本翻譯.
和本不知道是不是令人感動 將這個字眼翻譯成十字架討厭的地方.
很多年前 蔡震清院長也有一本書叫《十字架討厭的地方》.
這個字本身的意思是 大家猜到的 是scandal的字.
其實就是scandal的字.
原文的意思是十字架的scandal 十字架的醜聞 十字架叛島人的地方.
所以有些翻譯成信仰的阻礙 十字架冒犯人的地方 十字架眼中釘.
總之是一個這樣的氣人的地方 用我們朝議的十字架的氣的地方.
它會卡住你 會令你跌倒 會令你不舒服.
但和本也很有意思 很有風格.
十字架討厭的地方.
所以十字架本來就是這樣的東西.
所以保羅有一個這樣的意思 如果我們仍然去傳國禮的話.
這樣十字架討厭的地方就沒有了.
保羅似乎是想保留十字架討厭的地方.
他想保留十字架很奇怪的地方 十字架令我們很不舒服的地方.
但我們這二千年來 十字架是甚麼來的.
十字架是水晶來的 紀念品來的 項鏈來的 T-shirt來的.
十字架是我們最宗教的東西來的 十字架成為了一個最有代表性的宗教記號.
就算你淘寶也是一樣 淘寶你買不到聖經的 但淘寶可以買到十字架.
請問.
你試過問 如果你不信教可以買十字架.
因為是一個很普通的宗教記號.
但我們似乎忘記了十字架的原意就是要人生厭.
本來就是要你覺得很不舒服.
十字架是一個凝聚 古代羅馬死刑的方法.
以色列人把耶穌釘在十字架上是一件很羞澀的事.
尼塞亞被釘是一件大不道的事 你明不明白.

$^{1241}$其實你不明白 因為你感覺不到.
因為你已經感覺不到十字架有多羞澀.
我們試試改一改 不要用十字架來講.
我們試試用古代中國的清末民初年代.
瑪利亞是文女 十字架是最後晚餐是最後晚餐.
當耶穌和門徒吃完最後一杯茶 被清兵捉完之後.
然後被人捉去浸豬籠 你明不明白.
耶穌被人浸豬籠 旁邊有一對姦夫淫婦在他兩旁.
所以有三個浸豬籠在那裡 死刑.
最後耶穌也被人浸豬籠死了.
早期教會說 耶穌基督被人浸豬籠是一個好消息.
並且說當你要跟隨耶穌的人要背上你的豬籠.
是很有型的事情 背上豬籠是不有型的.
十字架是一件很奇怪的事情.
一件很褻瀆的事情 一件很不型的事情.
甚至是最沒有上帝的事情.
所以無論是潘福華還是莫特曼 或者是普羅多瓦都強調.
十字架正正是要你覺得討厭.
要你覺得這正正不像有上帝的方式.
就是上帝的方式.
十字架要推翻一切宗教.
十字架是最不宗教的方法.
奈何我們人就將十字架捧上天.
變成一個宗教最強的記號 變成一種宗教.
所以我覺得很不想全聖教成為另一個宗教.
或者今天很多基督教都是一種宗教.
純粹是耶穌基督版本的宗教.
客觀來說基督教是一個高級宗教.
我們有教義 我們有文獻 我們有很多不同的體制.
是一個很強勁的宗教.
但這也是一個宗教.
十字架正正是要推翻一切宗教的外表.
所以我想說的是.
我們嘗試去實踐一個非宗教的基督教.
去活出一個非宗教信仰的信仰的時候.
正正就是將一些人為的東西去除掉.
或者說十字架是一切宗教的照妖鏡.
今天你純粹是來尋求安慰.
尋求上帝疼愛你的寶寶的時候.
實際上不是 你要背十字架.

$^{1281}$十字架是令你痛苦的.
這個信仰是令你受苦的.
如果你想尋求敬拜的時候有些良好的感覺.
或者是很在乎小眾目的關不關心你.
這些事情的時候.
這些是宗教的需求.
但我經常覺得我們嘗試活出一個非宗教的基督教.
做回一個人 做回一個社會裡的人.
對人好 有人應該要對的態度.
我們仍然去持守基督耶穌要求我們做的事情.
多於上帝這樣的喜悼.
回到那句話題 上帝喜不喜悼.
對我來說 全聖教怎麼看上帝喜不喜悼呢.
不是那種喜悼.
今天我想說的是靈恩派教會.
我不是想插兩派教會.
我只不過是想用一個例子來說不是什麼.
我們所想的上帝不是什麼.
上帝喜悼不是這樣的意思.
所以靈恩派的優點是什麼.
正正就是將上帝喜悼變得很具體.
上帝喜悼是很明顯的.
因為有聖靈的充滿.
聖靈的動作 方言等等.
參加這些教會的人是很明顯地看到上帝喜悼.
很強調上帝喜悼.
因為他們用很多這樣的方式.
去將上帝在這裡展現出來給你看到.
區域邪靈也是一樣.
這裡有一隻懶惰的靈在.
所以你這麼懶惰.
他們將懶惰變成一個很具體的屬靈現象.
你只要弄走靈就不懶惰了.
這正正就是一個不是所說的吸靈的世界.
是一個哈利波特的世界.
將這個世界變成這樣是很簡單的.
你只要多點收靈就會弄走鬼.
多點上教會就會沒事了.
這是一個這樣的上帝喜悼的方式.
我不是想拆他們 我是很尊重他們.

$^{1321}$我想說我們不是這樣的.
上帝喜悼不是這樣的.
我們很少說上帝昨天這樣對我.
那次我聽到上帝這樣對我.
我不會這樣說的.
福祉律沒有目的這樣說.
不會說上帝昨天對我說.
我禱告了二十年.
神物上帝聽我祈禱.
不是這樣做的.
不是上帝不做這些.
但我不會將上帝起到這麼活生生的.
跟我聊天的.
我希望你能夠明白什麼叫上帝喜悼.
不是那種靈恩派教會的上帝喜悼.
所以我想說的是.
我希望你明白我想說什麼.
上帝不在場是OK的.
那種不在場.
不需要每次都在你旁邊.
當然他可以這樣做.
但他不需要每次都是這樣.
為什麼說我希望大家能夠成熟一點.
能夠理解那個上帝.
剛才那個第四格的上帝.
你未必見到他.
你未必每天早上聽到他說話.
但你知道他在.
但你未必見到他.
你仍然跟隨耶穌要你做的事.
做一個人.
做一個很好的人.
將你所有的熱心忠於上主.
但不需要像好歷人派的喜悼.
所以我講了不是的.
我也講了什麼是.
我們怎樣看聖靈.
如果不是靈恩派那種.
起碼我覺得也要處理.
所以我也處理不完.

$^{1361}$我覺得.
如何看上帝的同座.
因為聖靈正正就是上帝在.
以前我們也說過.
耶穌在前團體要給他一張椅子.
這張椅子是耶穌坐的.
不要讓他坐.
強調耶穌在.
強調耶穌的靈在就是聖靈.
所以聖靈是我們今天能夠.
講上帝同在的意思.
耶穌基督在.
全靈父上帝的右邊.
耶穌的靈.
聖靈.
天父的靈也在同在.
這種同在.
所以這種同在的意思.
主會再回來.
什麼叫主再來.
就是他不能起床.
他不能起床才能回來.
所以耶穌基督今天不是那種人.
他的靈.
在那裡.
我們理解父子靈就是這樣.
很公正的三角形.
What if 聖靈其實是一種.
更加強調是一種很動態的聖靈.
他是父子之間的關係的建立者.
父子兩個的關係當中的橋樑.
就是聖靈.
不單止是這樣.
更加是上帝和世界的關係的建立者.
我們在說聖靈的時候.
他不一定是靈恩派那種.
一種很強烈的第三者的存在.
明不明白.
不是上身那種.
不一定要是.

$^{1401}$有時候說.
聖經不一定要寫東西.
我做了什麼.
聖物是寶羅在監獄裡寫信.
不需要上身.
我寫了什麼 羅拔書.
不需要這樣叫物是聖經.
物是是一個很普通的寫信過程.
我們的聖經也是這樣.
是一個平時生活上和你一起的聖靈.
但不需要有第三者.
去侵入你的聖靈.
所以你的生活.
你的世界.
你的一切.
是有聖靈在.
但都是你.
你被人解僱了.
不要說魔鬼要你.
因為你做得不好.
或者那個人不好.
有些世界的方式可以解釋到的.
但仍然有上帝在.
仍然有聖靈在.
不需要有第三者的存在.
仍然我們用一個.
及靈世的方法去明白這個世界.
所以不說那麼深.
我們覺得聖靈不是一些很奇怪的東西.
上星期清心的篇章說了這件事.
聖靈的感動.
未必是一些很突如其來.
很奇怪的感動.
而是一些很普通的.
一些平常的東西.
都可以是聖靈的感動.
所以我們嘗試去.
活在一個現代世界裡.
去理解上帝.
最後我有幾個應用.

$^{1441}$我們作為一個.
及靈世界的基督徒.
一個不迷信的基督徒.
一個非宗教的信仰的基督徒.
我們有幾樣東西可以大家一起想想.
這個回小組大家可以討論一下.
第一個.
行動和禱告.
我們很強調.
禱告很重要.
但是禱告從來都沒有和你的行動有任何的矛盾.
你仍然要做你要做的事情.
你仍然需要禱告.
所以不是那些.
你經常聽到的謠言.
什麼韓國教會祈禱八個小時.
然後禱告就復興了教會.
你就會看到我們很不熟悉的.
經常都不禱告.
我們有禱告,但不是那種禱告.
不是說我禱完八個小時就能夠復興教會.
而是我們很重視行動和禱告.
我們作為基督徒.
我們很需要禱告.
更加想重視.
作為一個現代人我們需要行動.
很多東西你知道.
是可以解釋得到的.
我們要開會的.
做事要預備.
很多事情都是這樣.
我們不會抹殺我們禱告的重要.
當然單單只是行動而不禱告不是我們的事情.
但我覺得兩者是並行的.
我們很重視行動.
很重視禱告.
第二.
我想這個也是按需要.
很多人經常問的問題.
答完了.

$^{1481}$如何明白上帝的心意呢.
當然我們很多時候.
我們可以尋求上帝的心意.
聽聽有沒有上帝跟你說話.
不知道你以前是怎樣聽.
但我想你也未必是突然間.
保羅式的光射.
讓你瞎了三天.
然後聽到上帝跟你說話.
上帝跟你說話的方法可能都是.
不知道你有沒有試過.
揭聖經那些.
揭到大紀元看看那些.
這些我們也OK的.
但也不是你長期會用的方法.
你可能會尋求人的意見.
用理性去明白判斷.
再將事情禱告交給上帝.
你會用一些及齡的方法去明白.
你想移民也好.
麻煩填表格申請簽證.
你不會祈禱那麼簡單.
祈禱不會突然間簽簽證給你.
你會做事的.
你尋求神心意也是一樣.
你會期待著上帝跟你說話.
但你仍然會預備.
仍然會去判斷.
不過更多時候可能就是.
上帝未必是這樣跟你說話.
很多時候都是你先做了一件事之後.
才發現原來這是上帝的心意.
這個也是上帝的方式.
聖經也有很多這樣的例子.
伯來河也不知道去哪裡.
走到哪裡才回頭看.
原來上帝要來這裡.
所以我們也一樣.
我們不是不聽上帝的話.
但不是靈因派去聽上帝的話.

$^{1521}$不是這樣去尋求一些奇怪的東西.
才能明白上帝的心意.
第三點我仍然強調.
雖然是這樣.
雖然今天說了這麼多.
什麼靈的世界.
什麼非宗教的基督教.
作為一個人.
我想我們跟非基督徒不同的地方.
就是我們仍然相信奧秘.
相信神秘.
我們知道世界上有很多東西.
是可以這樣解釋的.
但仍然相信上帝有些東西是神秘.
有些東西是無法參透的.
你仍然懷有一份奧秘.
仍然有一份期盼.
雖然覺得這樣的情況好像不能拔.
但我仍然相信上帝會做事.
只是你不知道怎麼做.
也不是依靠靈因派那種去做.
但你仍然相信上帝仍然有些空間.
仍然超過你解釋得到的東西.
所以我自己是這樣去理解.
這這麼多年我這樣理解上帝的存在.
上帝的存在不是那種.
抱歉,是靈因派那種.
我用它做例子.
不是那種.
好像是第四格漫畫那種.
我今天早上也見不到神.
抱歉,雖然我是牧師.
但我今天也見不到神.
未必聽到祂的說話.
但我仍然相信上帝在這裡.
但不是那種在這裡.
我也不需要去依賴那種在這裡.
而是更加要警醒自己.
不要用這麼多宗教的方式去明白你的信仰.
用十字架來做那個照妖鏡.

$^{1561}$十字架正正是推翻一切人為的構想出來的上帝.
一個最不像上帝的上帝.
在十字架裡面.
我們一起祈禱.
因為你在十字架裡面.
來表明你是上帝.
我們知道你和祂同在.
但我們知道你不是這樣的同在.
我們需要知道你怎樣.
這樣去構想你是一種昔日亞倫他們拜金網讀的事情.
他們假信一個上帝並敬拜他.
求主讓我們能夠認清你的同在.
更加去拆掉很多宗教.
我們更加去毀身於你.
不是毀身一個宗教.
而是單單去毀身於你自己.
求你讓我們能夠有你成人當中去分辨.
讓我們能夠真真正正地去跟隨你.
相信你.
帶領我們全家人.
能夠有更多的人能夠起來.
成為跟隨你的人.
馮春永求.
阿們.
餵.
今天真的.
又在這裡又不在這裡.
在這裡還是不在這裡呢.
你在這裡就對了.
你在這裡就可以說完.
但我也想問一下.
其實現場也有很多人在.
你那個上帝在還是不在.
大家應該可以有很多聯想或者感受.
可以揮揮手或者同工會給咪高峰大家分享一下.
聽完之後.
你那四格漫畫當中.
你覺得你是哪一格多一點呢.
剛才我沒有聽錯.
剛才那四格應該都有你的表達空間.

$^{1601}$那個漫畫.
我第一眼看到就是覺得.
為什麼我們一定要用這個方式去看漫畫.
可以反過來看.
莫特曼他在《盼望神話》也說.
上帝看時間.
或者潘林博.
其實也是在終末看現在.
即使我們很多時候都在第二第三格中間的時候.
但其實今天.
我們所理解這個線性的時間.
其實在上帝的眼中.
原來這個現實是由終末所斷定.
潘林博說終末是決定現在.
我覺得也很啟發.
其實我們經常用過去去證明上帝.
我們想從昨天來試驗經歷.
但原來上帝是在未來奔跑過來.
我記得之前John也在港島看過.
好像是否龍珠.
其實上帝是未來奔跑過來.
可能在場和不在場之間.
就好像莫特曼所說.
其實上帝是來臨中的上帝.
歷史是上帝自我啟示.
這個不是一個已經預設上帝存在.
而是上帝存在中.
I am.
I am是一個過程.
一直會慢慢M到最後.
會爆炸性地.
原來我就是.
是一個過程.
所以我們今天理解的.
究竟有沒有上帝.
很多時候都是有或沒有.
有一點像Process Theology.
其實是一個過程.
我們站在時間點裡.
上帝已經啟示了自己.

$^{1641}$但他還沒有完全啟示自己.
所以我們很吊詭一時間.
不過我想可能我們一生.
就是徘徊在那四格裡.
一時我們覺得.
上帝是一個完全的他者.
我們在上帝面前.
我們很軟弱很卑微的時候.
我們在第一格.
我們是一個完全的他者.
可以被我們依賴.
但可能慢慢成長的時候.
又去到第四格.
我們原來是一個活在.
一個心下God is dead的時格裡.
我們是在上帝面前.
成為我自己所是.
上帝面前的一個被造.
純粹被罪人.
不過可能去到我們年老的時候.
又發覺原來有回歸到.
上帝造人吹口氣的時候.
又是真的有一個這麼實在的上帝.
在我衰老的時候.
可以被我完全依賴.
所以我想是一個很有趣的過程.
我們的觀眾越來越高集.
我覺得很好.
認同.
沒錯,是一個already but not yet的過程.
我們說耶穌主宰來.
是一個圓滿的同在.
將來我們跟祂同在.
今天我們的同在.
總是有些不是你那種.
這裡知道那種落差.
所以上次說要等待上帝.
等候是其中一種方式.
後面.
我就是你剛才所說的.

$^{1681}$曾經是你所說的那些靈恩派的那種同在.
OD.
我自己覺得靈恩派的那種同在.
當然現在會比較落地一點.
用一些所謂世界的眼光去解釋身邊的事.
但是靈恩派的那種同在.
其實我自己會覺得是一種.
理解聖經有很多超自然和神互動之間的.
那些經文的一些比較好的體驗.
那種體驗會讓我們更加明白到.
原來耶穌去打仗.
神會因為祂的禱告.
所以會讓那一格退一格.
會讓太陽倒轉走.
在我們的經歷裡面.
就會體驗到原來不是真的長期消失.
是真的有這樣的時候.
我就會覺得靈恩派的那種經歷.
其實可能會比較容易讓我們明白.
如何實際去理解聖經和活用聖經.
就不會太多停留在理論裡面.
就會在一個更加實際裡面.
但是最容易的就好像你剛才所說.
會走過龍.
好像會依賴了那種感覺.
所以我覺得無論是靈恩派.
或者是所謂靈恩派,方恩派.
其實我覺得是各自都可能是一個.
我自己理解就是大家面對著那隻大象.
大家看著鼻子和尾巴的時候.
其實各自都有一些好處.
和各自看到一個部份.
但其實是需要平衡的.
但是我會覺得靈恩派就是解釋不了.
為什麼剛才說馬可福音14章35節.
主啊主啊你為什麼離棄我.
在十字架上耶穌說這句話的時候.
其實回想以前是沒有解釋到.
為什麼上帝會離世.
但是剛才看回原來潘博華這個解釋.

$^{1721}$其實才是整個信仰的核心.
我就補充得到.
其實沒有了那種所謂神的同在的感覺的時候.
那個信仰生活的核心.
究竟是什麼樣的本質呢.
我就會覺得整個故事就完成了.
無論是很實際的,很近身的感覺.
一些活用真理的部份.
或者是一個很理論性的真理.
和一個比較貼地的信仰層面.
整個故事我就覺得可以組合在一起.
所以剛才聽完之後.
感覺就好像將這四格裡面.
就像剛才弟兄所說的.
其實是一個循環.
其實是一個完整的故事.
一個循環.
好,謝謝你.
多謝兩位弟兄的分享.
因為當初這一堂的時候.
我和John Chit Chat就是怕弟兄姐妹.
不是很明白想帶的訊息.
聽到兩位弟兄的分享也很好.
因為說得很好.
靈感派教會.
我自己的觀察和認識.
都是太過現象學.
就是從現象當中去主導.
然後就用自己解經的方法去套路.
讓人明白到那個現象是可以解釋的.
我覺得這件事有點太過了.
方法有時候太過實用.
很著重實用性.
覺得是否容易用一個做了一些事.
未必心安理得.
但就是要做一些事.
成為那件事覺得.
我在其中.
覺得那種感覺是很實在.
但你會看到今天的訊息.

$^{1761}$不是在說現象.
也不是在說要做些什麼.
其實現象如何.
上帝也在.
現象不在.
沒有的時候也不代表上帝不在.
但過程當中也不在乎我們做什麼.
和不做什麼.
這個過程是我們最難去接受和跳過的.
再說一點分享.
剛才弟兄說的那件事.
就是那四個漫畫的心態或者感受.
我自己最大的感受.
很多時候過了之後回想.
才感受到那個時刻其實上帝在.
我反而是過了那一刻才感受到.
在做的時候不覺.
因為真的用剛才John說的話.
我自己用我的熟悉.
或者我的想法.
或者我的能力去做.
按我自己的能力去做那件事.
即是切切實實做那件事.
但回想過後.
那一刻又感受到.
好像是上帝令我想到一些東西.
所以可能是過後才回想.
從一個綜合角度去看那件事.
發生過程當中上帝參與.
所以我自己的信仰經歷.
常常和弟兄姊妹分享.
你嘗試做多一點回想.
你常覺主因的滋味就更當如此.
回想那種喝茶的回甘.
回甘那一刻.
你就回答了.
曾經那口茶是過後那一刻.
其實感覺是很實在的.
好像很虛,但感覺就是這樣.
兩位先.

$^{1801}$剛才去解釋十字架和漫畫.
對我理解現在的社會狀況.
都是一個安慰.
但其實推到極致的時候.
這個信徒他自己一個.
徬彿神是和自己同在.
或者在心裡的狀態.
其實是否和沒有信仰的人.
都可以很相似呢?.
怎樣可以區分到自己.
我做的事是否真的沒有犯罪.
或者正在走在神裡.
你們怎樣看呢?.
我第一個反應是覺得.
其實應該要像外面的人.
用另一派的例子.
太不像就不要太當.
你應該做外面又像的人.
不過我想有些質素正正就是不同.
他們覺得你多了一份盼望.
盼望其實不是純粹盼望.
因為你有你的信仰.
有一種力量.
你相信耶穌都會幫助你.
這種盼望好像沒什麼特別.
但其實這就是最大的差別.
所以不會突然我會說方言.
你不會說.
或者我突然有什麼能夠聽到你聽不到.
其實大家都不知道.
大家都好像對前面很迷惘.
但為何我們付出這幾年.
或者很多時候信仰這幾年這麼重要呢?.
盼望這個話題.
或者良心這些話題.
或者堅持這些話題.
很多人覺得信仰的力量就在這裡.
那力量不是純粹心理的力量.
而是一種從聖靈而來的力量.
這正正就是差別.

$^{1841}$但差別不應該太過誇張.
我們完全好像聖人模式.
或者完全不同他們.
所以我覺得想想像的基督徒就是應該這樣.
因為及齡世界基督徒的彭富華.
在這樣的亂世裡面.
仍然有一些價值上是持守住.
但又不是.
對不起再說一次.
你想不到很靈的彭富華.
就是這個差別.
我的看法就是.
你剛才說的是.
信主和未信主的人的差異.
我仍然相信聖靈的參與.
因為信了主之後.
聖經說得很清楚.
就是聖靈就內在我們心裡.
所以聖靈是一個會對我們說話的上帝.
祂會做一些啟發.
讓我們感受到那件事.
是否合乎聖經或耶穌教導的事情.
反而你說未信主的人.
他們是否會沒有這個觸動呢?.
我覺得未必完全沒有.
因為普遍恩典裡面都說的訊息就是.
保羅在羅馬書裡說的良知.
上帝會啟發人的良知去做分辨.
所以在過程當中你會見到.
如果沒有信主的人.
他的做事可能比我們信主的人更好.
這就是回應John所說的.
基督徒應該會有更多的改變.
但有時基督徒對信仰都沒有太大改變.
他只不過是接收了信仰.
我是一個基督徒.
但其實他接收了宗教.
而不是接收了耶穌的改變.
所以那個身份而有的能力.
應該是聖靈的改變.

$^{1881}$最不想的就是多了一份宗教習慣.
我會純粹犯善祈禱.
或者會不去教會.
或者去教會.
這些是宗教的練習.
但我想你應該有更加重要的東西在裡面.
潘福華寫了一首詩.
非宗教和宗教人.
大家都是需要一些很普通的東西.
麵包,平安,健康.
都要求主憐憫.
但唯有信主的人.
會為基督耶穌而受苦.
所以不是問你能夠得到甚麼.
而是你會做甚麼.
你會為主耶穌和他一起去受苦.
這就是差別.
我覺得另一種方法去理解這四格密碼.
就是因為神的不存在才凸顯了傳說.
就好像你每天都很開心.
那種就不叫開心.
是要有不開心才能感到開心的感覺.
所以再套用這四格密碼.
就是因為有第四格神的不存在.
才凸顯第一格神的存在.
很開心今天聽到兩位講這四格的密碼.
我的經歷就真的好像這四格密碼.
年輕的時候可能會依賴很多教會給你的訊息.
或者你已經調到宗教的模式.
你會有些依賴.
覺得你會開心一點.
但當你隨著年紀慢慢增長的時候.
你會發覺有些價值可能會和周圍的東西碰撞.
我舉個例子.
我以前總是舉例子.
好像你的車壞了.
其實有幾個方法去處理.
如果你懂得做就自己做.
你不懂就叫人做.
叫人來修.

$^{1921}$如果你有信心.
你按手在車上.
奉主耶穌基督的命令.
讓它可以重開.
其實很多東西是可以.
好像是一切.
但互相矛盾.
我覺得在我信仰的歷程裡.
我看到的教會.
無論是靈恩,灰恩派.
傳統的教會.
其實最大的問題在哪裡呢.
剛才所說的宗教問題.
其實源自於自我迷信.
我不知道各位.
我年紀比較大.
我出年也等老了.
曾經有些電視劇.
《無名無姓》.
很久以前.
模擬了一些基督徒.
突然打地鐵衝進門口.
還是你口頭禪說出來呢.
我覺得我們最大的問題就是.
在傳統的教會裡.
一直以來.
都是這樣的迷信.
我一直在人生裡.
其實我現在已經去到第四格.
我不再聽傳統教會說話.
我會用自己的經歷去感受神.
我的存在在哪裡.
以前辛苦的地方是.
當你有這種意識的時候.
在傳統教會.
我以前很喜歡說.
我自己變成一個.
我是一個異教徒.
在一份半教的基督徒裡.
我的模式就是這樣.

$^{1961}$是很辛苦的.
當然你要靠自己去思索.
去想一些問題.
我剛才也回過一些靈驗派的小組.
很簡單.
我感受到他們去山上祈禱.
叫得很大聲.
說他們所謂的方言.
然後去感受到聖靈.
其實我自己的邏輯很簡單.
如果聖靈是真的.
為什麼他們感受得到.
我感受不到.
一個簡單的邏輯.
就算我不明白他們說什麼.
如果聖靈真的在那裡.
我應該感受得到.
甚至我見過他們有些.
很奇怪.
和黃大仙沒有什麼分別.
他們有一個叫.
我不知道你們知不知道.
有一個app叫做AJAR.
靈驗派很喜歡用.
想不到什麼問題的時候.
他就揭開那個.
好像通勝一樣.
就指出今天他說的話.
是,我真的很中.
他就會去做.
我想說的是.
我們有很多痛苦.
是源自於我們的自我迷信.
不只是一個宗教的框框條條.
而是我們有沒有去.
開放我們自己去接受一些新的東西.
包括.
好像早前在互聯網說得很流行.
有些人說.
耶穌年輕的時候.

$^{2001}$其實去了印度學佛法.
我們有沒有一個批判的精神去想.
和去接受.
或者在這件事上去看.
究竟哪些東西是真哪些東西是假.
或者就算所有人都說是假.
如果你憑你自己的感受覺得是真的.
你會不會願意去.
去堅持.
好像今天福音教會那樣.
我相信一樣是真的.
但所有人都說是假.
甚至我是異端.
我是否繼續願意走下去呢.
我覺得當今基督徒.
當代基督徒很重要的一點就是要懂得思考.
不要用屁股去想.
這是一個我經常要說的東西.
包括香港教育局那時.
很多東西都是我們沒有了自我批判.
沒有了自我解除迷信的方法.
思考的方法.
回應剛才展會所說.
其實他提出的問題原意的核心在哪裡呢.
我們被傳統的教會誣陷了一樣東西.
基督徒和普通人應該有分別.
我們是分別成性的.
所以我們做的事情和他們一樣.
那就有問題.
邏輯其實就是這樣.
但你可以反過來看.
信仰做人本來給了一個.
創世紀元說了什麼.
給了我們一個管理萬物的能力.
這個能力是普遍的.
基督徒也有.
我們不要害怕一件事.
傳統教會告訴我們要分別為性.
其實我們要做的事情和他們不一樣.
其實很多東西應該是一樣的.

$^{2041}$正如剛才所說的良知.
是一樣的東西.
我覺得我們要在今天這一堂課.
除了剛才所說的宗教框框條條.
我希望大家抱著一個開放的心.
不要自我迷信.
你可以自我質疑自己.
去考慮很多事情.
可能你會看的路不一樣.
這四部漫畫其實就是在說這件事.
這是我今天要和大家分享的東西.
謝謝大家分享.
聽大家分享的時候.
我發現今天所說的原來大家都明白.
也有共鳴.
希望大家都能夠跨到成熟的地步.
我覺得這是成熟.
不是第一部漫畫不重要.
而是你起初的時候很需要.
這樣去理解.
但沒有了就不代表.
這不是我們的信仰.
所以不是否定第一部漫畫.
大家起初的時候不用教.
反而更加強調.
有時面對這樣的情況.
我們怎樣能夠知道上帝同在.
或是怎樣去理解呢.
最後給個牌子.
後面.
我想大家都明白.
但我覺得第一位弟兄和姊妹都說得很好.
我回想起來想問.
我想說我想說的東西.
為何有人說上帝不在場呢.
其實在這四部漫畫裡.
在第三部漫畫.
那個人就開始忘了上帝不在.
就是那一部才會問.
為何我聽到很多人.

$^{2081}$不論信不信.
說為何這時間上帝不在.
不在的我就不相信.
很多時候就是因為第三部.
第三部已經很堅定.
我看不到他也在.
我就跳下去沒問題.
如果你能夠提出這個題目.
我覺得第三部是重要的.
我明明學到的東西.
我知道聖靈應該這樣做.
但世界不是這樣.
怎麼辦呢.
我相信19年大家都經歷了很多.
20年之後都經歷了很多.
甚至聖經也有很多例子.
我想起.
想也想到幾個.
阿伯拉罕無端端叫.
阿伯拉罕獻了以撒出來.
然後士司祭.
不是士司祭.
總之就是有一段時間.
我去到加拿大.
我要殺光全地所有的人.
一個都不可以.
小朋友都不可以.
有慈悲之心留一個都會出事.
當你和你.
上帝要你.
那時候是上帝擺明要你做.
你都不可以質疑上帝不在.
現在是你靠感覺的.
靠感動的.
靠你自己對神的認知.
當你世界出現的東西.
和你所想像的神不同的時候.
你怎樣去自處呢.
我相信.
但正常的想法就是.

$^{2121}$你等吧.
上帝還沒做事.
不在場的上帝.
其實那些人問.
不在場其實是上帝沒做事.
或者沒做他覺得上帝應該要做的事.
當不做事的時候.
我們怎樣呢.
我說等.
等出來的結果不是.
那又怎樣呢.
我之前.
有時候會聽.
土青年圖老師.
他們經常說基督教.
他們說.
如果上帝真的在場.
你做出來的人應該是這樣.
他也是說這件事.
他心目中的上帝應該要做這些事.
但這個世界不是這樣.
我們怎樣去處理這個問題呢.
你說等.
很可能最後就是.
看到結果就安慰自己.
如果結果是好的.
那還算是自圓其說.
但如果不是呢.
其實很多人就是因為這樣.
就證明瞭我不在場.
沒有.
假的.
我覺得是.
我們相信的.
我們會有第四教出現的.
我們很堅定的相信上帝在場.
我們就走下去.
但怎樣去處理這個.
明明我覺得上帝應該要這樣做.
這些事是不跟隨聖經教導的.

$^{2161}$但依然在場的情況.
可能你回去再聽一聽.
今天想說的東西.
其實值得再想多一點.
其實正正在說第四教的情況下.
我們仍然是OK的.
剛才Boofer那段話.
值得我們再思考.
正正在十字架上.
受苦的上帝.
才是上帝.
剛才你所說.
看著他心意想像的上帝.
這只是宗教.
宗教就是這樣.
宗教不一定是迷信.
拜黃大仙那些.
這幾年裡.
很多現代人都有自己的宗教上帝.
上帝一定要是我那種.
黃的上帝.
民主,資源普選.
都是一種宗教.
所以上帝只是超過我們所說的.
一切世界的解救.
明白嗎.
一切的解救.
在其他世界裡.
正正是受苦的上帝.
才是出路.
是一個軟弱的上帝.
無能的上帝.
在這幾年裡.
香港的情況下.
是一個無能的上帝.
才是一個跟人同苦的上帝.
這句話很有深度.
正正不是要求那種.
救到或應驗到某些東西.
結局並不是說.

$^{2201}$Yes 第四格裡沒有了.
我想說這件事.
所以結局不是.
能夠應驗或等到.
而是真真正正知道.
第四格裡所謂不存在.
也是我們的上帝.
所以我想說.
那個 absence.
不是一個 code and code absence.
真是可以 absence.
所謂的 absence.
是你那種存在的方法.
仍然是我們所信的上帝.
所以潘博雅說.
今天那個同在的上帝.
是一個離棄的上帝.
這句話好像很吊詭.
但正正是我們今天可以去想的那種.
很多人因為這樣.
所以去宗教的時候就走了.
因為他想要的是這樣.
我回應一下.
剛才說的第三格.
可能他很 puzzle 的狀況.
我自己通常都會喜歡用.
信仰是一個經驗學習的過程.
經驗學習當中.
就視乎大家的經歷.
你是有什麼取態.
有些人很著重經歷當中的感受.
有些人很著重經歷當中的批判.
我就覺得.
信仰在一個經驗學習過程當中.
你的感受和批判都是一致的.
因為在悟性和感性當中.
你怎樣去調整.
其實就是你怎樣去認識自己.
怎樣去認識這個信仰.
其實不是很複雜.

$^{2241}$其實就是過程當中.
如果你沒有了.
你只用感性.
你沒有批判的話.
你很容易就跌進迷信.
別人跟著做.
你就做得很漂亮.
跟著練習去做.
從感性當中.
剛才用了幾個例子.
打開聖經.
我以前帶青年小組的時候.
都有人告訴我.
那天我剛開始打開那頁.
就是在說.
是就這樣成了.
他就覺得是就這樣成了.
我就說.
我開第一頁就是.
後來他死了怎麼辦.
這個感受上.
就是按你自己需要的話.
剛開那頁靈修.
他覺得是靈修的時候.
看了那頁.
他就覺得那件事.
他就承諾上帝給他.
我說這些感受.
不是很批判性.
或者不是很理性去衡量那件事.
所以如果說第三格那格.
就好像上帝不在.
他很不熟的話.
我覺得就是我們在這幾年的過程.
或過去信仰過程當中.
你遇到要做抉擇.
你遇到要做一些分辨的時候.
你就想一下.
你平時信仰的建構的時候.
你是著重理性,感性.

$^{2281}$還是你真的複雜了.
所以剛才值得回去聽聽.
今天的內容.
John在後面有兩個連結.
第一個就是.
講關於祈禱與行動.
我們相信上帝的意旨與人的回應.
或者上帝的意旨與人的行動.
第二就是講如何明白上帝的旨意.
你用感性去明白上帝的旨意.
還是看上帝在我們當中的互動.
或者回想上帝在我們生命當中.
參與的過程.
你如何作出判斷呢.
我覺得這件事比較實在.
不要說實在.
你會慢慢看到自己如何建立你的信仰.
或者我講得比較快一點.
我覺得這幅圖很好幫助大家去思想.
去想想和神的關係.
或者神的存在.
但我覺得這幅圖也有一點.
是很相似的.
但同一時間.
那四幅圖的主題是.
My walk with God.
所以是那個人對神的感覺.
但如果你看這幅圖的時候.
你會看到是一個神?三個神?.
還是你真的看到四個神?.
好像剛才John一開始已經說了.
The presence of God.
只不過是Presence.
不是我們預期的Presence.
而不是神真的有一個時刻.
是Absent了.
神是Always there.
只不過在這四幅圖裡面.
主題的出發點是.
由我們自己的個人出發.

$^{2321}$我們如何在第三幅圖或第四幅圖.
acknowledge神的Presence.
才是我們如何去.
作為一個成熟的基督徒.
應該可以達到目標.
我想說我們以前聽到.
有些很悶.
有些不悶.
有些是奇蹟牧師.
不知道有沒有聽過.
很多神蹟奇事的故事.
告訴你.
在建堂的時候欠六千元.
然後就祈禱.
然後發現錢是六千元.
這些奉獻建堂神蹟的故事.
我想說其實很多都不是完全對的.
不是整個Fashionable.
背後很多東西.
其實他沒有告訴你.
收拾過.
說的好像很奇妙.
我們聽了太多這些.
其實初期是可以的.
但聽了太多.
你會發現一來不是那麼真.
反而有反效果.
信仰變成只有這些.
所以我們看世界.
不是靠神蹟的故事.
我不是不相信這些故事.
而是不能只靠這些故事信仰.
你知道背後很多東西.
其實都是.
上次這樣說.
原來有借錢的.
原來有怎樣的.
有些努力都做出來的.
我們就需要去瞭解這方面.
那個Presence.

$^{2361}$就是要包括在這裡.
不只是那種神蹟的Presence.
這個我自己很重視.
我都聽過這些神蹟.
但我從來都沒有經歷過.
我是那些經常送車的人.
就算App說還有兩分鐘.
到我去到看到車子就走.
我是很實幹的人.
所以我都很清楚.
我自己的能力去到什麼就做什麼.
不過四格漫畫.
或者剛才說到My Wall with God.
這個感受.
我自己覺得.
可能例子未必是最近的.
但我想你能夠瞭解.
如果你爸爸或者家人.
帶你去樂園.
主題樂園玩.
你未必一定要你家人.
長期陪你坐Ride.
可能你自己進去坐.
他在外面等你.
那你就自己坐.
或者他整個樂園.
他說你去玩吧.
我在哪裡等你.
那你就自己去排Fast Pass.
什麼都好.
吃飯的時候才會和他一起吃.
我覺得上帝好像放在我們樂園.
或者在我們現在生活的環境當中.
你未必每次都感受到.
上帝真的在你旁邊.
但上帝事實上就在我們旁邊.
這些可能你會用你的方法去產生類比.
但我覺得大家都是討論一個方向.
就是你相信上帝會在其中.
只不過在不同的感受當中.

$^{2401}$你敢不敢受到祂的存在.
但這個我不可以說跳到滑.
這個就是你的信心.
我不是想這樣說.
不過你的信仰在經歷過程當中.
你就會慢慢得到你自己所信的上帝.
其實我自己覺得你是經歷過程當中.
你會有感受.
有情感.
你會有批判.
有分辨.
這個信仰是真實的.
盡可能都是想拿開一些太過宗教性的行為.
去主導了.
好像做了這些會心安理得.
做了這些就好像會靈一點.
做了這些就好像上帝會喜歡一點.
我覺得這個外衣其實不應該是我們要守著的東西.
接下來我們七八月放暑假.
真的飛了一次.
所以我們七八月就會放暑假.
放暑假.
第六課就九月.
就會有我們第六課.
大家都可以一起去放暑假.
OK 遲些見.
OK 拜拜.
拜拜.
\newpage



\section{出埃及記 13:1-16-20230701}
\label{sec:J_OpyaPLYIE}
\textbf{【網上崇拜】奉獻第一|出埃及記13\_1-16|20230701 [J-OpyaPLYIE]}
\newline
\newline
連結: \href{https://youtube.com/watch?v=J-OpyaPLYIE}{\texttt{ https://youtube.com/watch?v=J-OpyaPLYIE}} ~~~~ 語音日期: 2023-07-01 
\newline
\newline
\hyperref[sec:I6Z1WA7E0RA]{\small{< < < PREV SERMON < < <}}
~
\hyperref[sec:index_chronic]{\small{[返順時目]}}
~
\hyperref[sec:index_scriptual]{\small{[返順卷目]}}
~
\hyperref[sec:vN0n_pNkXlA]{\small{> > > NEXT SERMON > > >}}
\newline
\newline
出埃及記 13:1-16-20230701
\newline
\begin{longtable}{cl}
\hline
\hline
章節 & 經文 (和合本修訂版)\\
\hline
13:1 & \begin{tabularx}{0.7\textwidth}{X} 耶和華吩咐摩西說: \end{tabularx} \\ \\ \relax
13:2 & \begin{tabularx}{0.7\textwidth}{X} 「頭生的要分別為聖歸我;以色列中凡頭生的,無論是人是牲畜,都是我的。」 \end{tabularx} \\ \\ \relax
13:3 & \begin{tabularx}{0.7\textwidth}{X} 摩西對百姓說:「你們要記念從埃及為奴之家出來的這日,因為耶和華用大能的手將你們從這地領出來。有酵之物都不可吃。 \end{tabularx} \\ \\ \relax
13:4 & \begin{tabularx}{0.7\textwidth}{X} 亞筆月的這一日你們走出來了。 \end{tabularx} \\ \\ \relax
13:5 & \begin{tabularx}{0.7\textwidth}{X} 將來耶和華領你進迦南人、赫人、亞摩利人、希未人、耶布斯人之地,就是他向你祖宗起誓應許給你的那流奶與蜜之地,那時你要在這一個月守這禮儀。 \end{tabularx} \\ \\ \relax
13:6 & \begin{tabularx}{0.7\textwidth}{X} 你要吃無酵餅七日,在第七日要向耶和華守節。 \end{tabularx} \\ \\ \relax
13:7 & \begin{tabularx}{0.7\textwidth}{X} 這七日之內,要吃無酵餅;在你的全境內不可見有酵之物,也不可見酵母。 \end{tabularx} \\ \\ \relax
13:8 & \begin{tabularx}{0.7\textwidth}{X} 當那日,你要告訴你的兒子說:『這樣做是因為耶和華在我出埃及的時候為我所做的事。』 \end{tabularx} \\ \\ \relax
13:9 & \begin{tabularx}{0.7\textwidth}{X} 這要在你手上作記號,在你額上作紀念,使耶和華的教導常在你口中,因為耶和華用大能的手將你從埃及領出來。 \end{tabularx} \\ \\ \relax
13:10 & \begin{tabularx}{0.7\textwidth}{X} 所以你每年要按著日期守這條例。」 \end{tabularx} \\ \\ \relax
13:11 & \begin{tabularx}{0.7\textwidth}{X} 「當耶和華照他向你和你祖宗所起的誓將你領進迦南人之地,把那地賜給你的時候, \end{tabularx} \\ \\ \relax
13:12 & \begin{tabularx}{0.7\textwidth}{X} 你要將一切頭生的獻給耶和華;你牲畜中頭生的,公的都歸耶和華。 \end{tabularx} \\ \\ \relax
13:13 & \begin{tabularx}{0.7\textwidth}{X} 然而,凡頭生的驢,你要用羔羊贖回;若不贖牠,就要打斷牠的頸項。你兒子中的長子都要贖出來。 \end{tabularx} \\ \\ \relax
13:14 & \begin{tabularx}{0.7\textwidth}{X} 日後,你的兒子問你說:『這是甚麼意思?』你就說:『耶和華用大能的手將我們從埃及為奴之家領出來。 \end{tabularx} \\ \\ \relax
13:15 & \begin{tabularx}{0.7\textwidth}{X} 那時法老固執,不肯放我們走,耶和華就把埃及地所有頭生的,無論是人是牲畜,都殺了。因此,我把一切頭生的公的牲畜獻給耶和華為祭,卻將所有頭生的兒子贖出來。 \end{tabularx} \\ \\ \relax
13:16 & \begin{tabularx}{0.7\textwidth}{X} 這要在你手上作記號,在你額上作經匣,因為耶和華用大能的手將我們從埃及領出來。』」 \end{tabularx} \\ \\ \relax
13:17 & \begin{tabularx}{0.7\textwidth}{X} 法老放百姓走的時候,非利士人之地的路雖近,神卻不領他們從那裡走,因為神說:「恐怕百姓遇見戰爭就後悔,轉回埃及去。」 \end{tabularx} \\ \\ \relax
13:18 & \begin{tabularx}{0.7\textwidth}{X} 神領百姓繞道而行,走曠野的路到紅海。以色列人出埃及地,都帶著兵器上去。 \end{tabularx} \\ \\ \relax
13:19 & \begin{tabularx}{0.7\textwidth}{X} 摩西把約瑟的骸骨一起帶走;因為約瑟曾叫以色列人鄭重地起誓,對他們說:「神必定眷顧你們,你們要把我的骸骨從這裡一起帶上去。」 \end{tabularx} \\ \\ \relax
13:20 & \begin{tabularx}{0.7\textwidth}{X} 他們從疏割起程,在曠野邊上的以倘安營。 \end{tabularx} \\ \\ \relax
13:21 & \begin{tabularx}{0.7\textwidth}{X} 耶和華走在他們前面,日間用雲柱引領他們的路,夜間用火柱照亮他們,使他們日夜都可以行走。 \end{tabularx} \\ \\ \relax
13:22 & \begin{tabularx}{0.7\textwidth}{X} 日間的雲柱,夜間的火柱,總不離開百姓的面前。 \end{tabularx} \\ \\
[1ex]
\hline
\hline
\end{longtable}
$^{1}$頂姐妹平安.
今天見到很多新面孔.
歡迎來到Float Church的崇拜.
也歡迎網上的頂姐妹跟我們一起.
在不同時區實時跟我們一起崇拜.
剛才敬拜的時候也說過.
今天是一個很特別的日子.
今天是下半年的第一天.
今天的第一天也是想跟大家.
在崇拜當中過你的第一天.
因為對於大家來說.
我希望上帝的話語成為我們每天的亮光.
或者是成為每天的提醒.
對於大家留意我們的Facebook和IG.
每個星期崇拜之前.
小編也很用心地安排了要預備的經文.
通常呢.
通常而已.
這樣就結束了.
很快的.
或者還沒按進去就沒了.
我是不會放過大家的.
我們會一起讀經的.
所以在崇拜之前.
我們一起讀一下今天的崇拜經文.
選的經文是一段群體的經文.
很希望在下半年開始的時候.
特別說一個新月題當中.
以一段群體的經文跟大家去了解.
其實上帝如何去無造.
或者是建立這群屬祂的子民呢.
我們看的是《昌厄及記》第十三章.
第一至十六節經文.
好,一起讀吧.
(唸經).
因為是不容易的.
還有是沒有錯誤的.
這是很重要的.
因為證明大家很認真地讀.
今天這段經文不短.

$^{41}$十六節經文不是逐節跟大家一起看完的.
反而就是.
其實這段經文的出現是什麼呢.
就是他們將要離開埃及了.
離開埃及的時候.
耶和華上帝對摩西說了一個很重要的事情.
你們做這些事情的原因是什麼呢.
因為之前十二章.
其實耶和華對摩西說.
你對那些以色列人說.
有些很重要的關於禦節.
巨勢無為.
怎樣做法.
怎樣將祭身的血劃在門框.
令你們免受殺長子的災.
祂說得很清楚.
這就告訴那些以色列人.
所謂的「何事」.
今天講奉獻這個主題的時候.
我選了一段群體的經文.
也是個別的經文.
群體的經文的意思就是.
上帝要從埃及將祂屬的子民.
帶出來一個群體.
有誰願意做呢.
有誰願意跟呢.
你願不願意離開埃及呢.
你願不願意跟摩西所謂的指示呢.
真的令外殺了你的伸縮.
劃了門框避過這一劫呢.
這是個人的問題.
是個家庭的問題.
照不照做是個意願的問題.
所以是一個群體性.
也是個人性.
奉獻從來都是一個群體性.
也是個人性.
沒有人拿著槍逼你奉獻的.
應該是.
是不是.

$^{81}$也沒有人告訴你要比較.
因為都不知道.
通常都是拿著就放下去了.
沒有人知道.
但又是.
但你見群體做的時候.
你不好意思不做.
於是乎都拿一點點放下去.
但你知道不關事的.
從來都是你自己心甘情願去回應.
所以群體會做.
你個人會做.
但是看回這段經文的時候.
其實上帝要我們做什麼.
或者上帝在這段經文裡面提醒我們.
其實當你的兒子.
將來問你.
為什麼你這樣做的時候.
你要說些什麼.
所以第一節.
和大家去看看有些重點的字.
就是「投身的」.
要分別為「性」歸「我」.
然後上帝很著重一個「投身」.
「投身」是什麼意思.
就是第一個.
是不是.
就是第一個.
我不知道你成長過程當中.
你排第幾.
我排第三.
你們家裡可能都沒有三兄弟姐妹.
現在通常都是一個或者兩個.
都有的.
可能個別都有的.
我是排第三的.
我家裡有兩個兒子.
是我生的.
是吧.
即刻要先教一教.

$^{121}$我排第三的.
我記得我第一大兒子出生的時候.
有人就和我說.
他說.
你們好像在看直播.
是不是.
我大出生的時候.
有人就和我說.
第一個就照書養.
第二個就照豬養.
聽得明白嗎.
第一個出生的時候.
什麼都不懂.
就跟著書怎麼做.
怎麼學.
跟著書怎麼養大他.
到第二個了.
差不多了.
不用那麼緊張.
沒問題的.
給他吃就行了.
吃飽就行.
肥肥白白沒問題的.
打針不會有反應的.
是不是.
第一個就照書養.
第二個就照豬養.
你們兩個不是這樣的.
重點很多人都有一個概念.
就是第一個就是.
第一born.
很正.
很珍而重之.
很重要.
不要得來不易.
很緊張.
第二個就是.
都開心的.
這個概念是錯的.
也不是這樣.

$^{161}$整件事不是要分頭.
第一和第二.
今天不是說生一個兩個三個.
不是.
但其實在過程當中.
投生的.
在這件事件當中.
耶和華上帝要處理這件事.
因為當時的文化當中.
或者當時在埃及人當中.
接下來.
耶和華要幹第十災.
第九災是什麼呢.
就是沒有光.
黑暗之災.
三日三夜沒有太陽.
因為當時的埃及人.
他會將他的手身.
兒子獻給太陽神.
上帝要擊殺.
上帝其中一個很重要的表徵就是.
其實你想倚仗的.
不是太陽神.
其實反過來.
或者按順序來說.
其實太陽神不能夠幫到你.
因為第九章裡面已經說.
是使黑暗的日子.
其實如果你求的太陽神是可以的話.
你的太陽神應該會給你光.
你三日都沒有光.
但那班埃及人會不會相信.
不相信.
法老不會相信.
剛好是巧合來的.
嚇不到我的.
當然不是.
因為真正掌管的是誰.
一定是做天地的上帝.
耶和華.

$^{201}$但那班人不相信.
或者埃及人不相信.
但你們最倚重的就是.
將頭身奉獻給太陽神的時候.
但真正能夠保你頭身的是誰.
仍然是創天造地的上帝.
如果太陽神是這麼可以的話.
你就求他吧.
你會看到結果是甚麼.
不是.
但這件事不只是給埃及人.
是給埃及地當中的人.
所以耶和華上帝讓那班以色列人知道.
你願不願意跟從我.
你願意跟從我的時候.
就照摩西的方法.
讓一隻羊羔成為祭身.
用牠的血.
畫在你的門框上.
我的使者就會看到羊羔的血的時候.
就會如願帶我們離開這個死亡的宙座.
你願不願意將這件事看重.
你願不願意將主權歸還給耶和華上帝.
手心是很重要的.
大家都知道第一次很重要.
很多時候都覺得那件事是最重要的.
你真而重之.
但你所倚靠的是甚麼呢.
有沒有靠錯邊呢.
有沒有抓錯用神呢.
有沒有錯重點呢.
這個都很重要.
今天你心目中.
我們不是以色列人的環境.
但你心目中有些事對你來說.
是好像投身那麼重要的.
今天你心目當中對你來說.
有些事好像你長子名分那麼重要.
但你所倚靠的是甚麼呢.
如果當要沒有的時候.

$^{241}$要做決定的時候.
你會怎樣做決定呢.
你願不願意捨棄你投身的.
成為一個奉獻呢.
這是一個挑戰.
當時至少有六七十萬人離開埃及.
他們願意進入這個信仰的群體.
他們在埃及長大.
莫西夫召他們出來進入這個曠野.
他們願不願意.
他們願不願意再次跟隨耶和華上帝.
他們願不願意做這個走出來.
我們這幾年都問了很多你願不願意.
當有人吹哨.
當有人呼籲的時候.
有很多事你認不認同.
你覺得哪些事重要些.
你覺得哪些事要表達些.
哪些事要發聲.
哪些事要表態.
你其實就不斷衡量你自己價值觀.
你看重的事.
有些事你幾年前已經出過了.
很多人都站出來.
很多人都願意做.
因為你看重那件事.
你願意離開.
對你來說.
去到二三年下半年開展的時候.
當很多事都如常.
當很多事已經進入可以自主的時候.
今天是下半年的第一日.
你要自主的.
你要價值觀重點要重新擺位的是什麼呢?.
你要願不願意分別出來.
歸耶和華所有.
願不願意為上帝有所奉獻.
重新擺位呢?.
這是一個開始要問的問題.
而是開始你要回答的問題.

$^{281}$不僅僅是群體性.
也是個別性.
今天是每月的第一周.
都是祈禱會.
今天的崇拜講道環節當中.
會將今天的信息和祈禱會融合在一起.
在信息當中又會加入禱文.
又會加入祈禱的時間.
很希望信息帶動我們.
成為我們去回應上帝的起點.
所以第一個會和大家問的.
也是禱告的內容.
也是禱告的事項.
你願意付出什麼代價離開你的埃及呢?.
每個人都有自己的埃及.
都有你離開和不想離開的地方.
每個人都有自己的埃及.
有些是你依戀.
有些是你忌諱.
有些是你期望想離開.
但又不知道怎樣離開.
當有人說「Let my people go」.
「跟我走」的時候.
你是否願意跟?.
你的惰性會否核製了你?.
你要跟的時候.
你會跟誰?.
回想起當時.
其實是否有這麼多人想離開?.
我們不知道.
但你會看到那些以色列人.
去到曠野的時候.
他們一不滿意.
他們就埋怨摩西.
「其實讓我回去吧」.
「其實那裡挺好的」.
你心目中的埃及是什麼呢?.
如果你真的想離開埃及.
你會付出多少代價呢?.
你會否將投身看重的事情.

$^{321}$願意奉獻.
願意放下.
願意離棄呢?.
我們一起思考一下.
當你安靜的時候.
我希望你.
為自己心中的埃及禱告.
當中有什麼是核製你?.
當中有什麼是纏繞你?.
而你心中有什麼是.
像獸身一樣重要?.
你願意奉獻.
換取.
離開埃及.
開展新的埃及?.
當你安靜的時候.
我希望你.
為自己心中的埃及.
奉獻.
願意奉獻.
願意放下.
願意離棄呢?.
我們一起禱告.
最後四天.
可能很多以色列人其實都是.
懵然不知.
很多事都未必弄得清楚.
其實我們都好像他們一樣.
但當我們真的要認真.
要看待我們身邊的事情.
就好像幾年前.
那些人說要表達.
要出來.
要聚集的時候.
到今天我們經歷了幾年的時候.
我們經歷了很多選擇.
很多取捨.
我們的價值觀不斷地.
兜兜轉轉.
不斷地重新擺位.

$^{361}$但我們心中.
什麼是我們投身這麼重要呢?.
我們願不願意放棄.
而選擇我們覺得值得離開的埃及.
對我們這個群體來說.
無論移居的選擇.
留下的選擇.
都是一個新的學習.
這也是我們新的面向.
我們相信萬物都是從上帝而來的.
我們得到的福分.
我們得到的一切.
上帝叫我們在信仰群體當中一起學習.
但我們自己都認識得自處.
求主你幫助我們.
願意我們這個群體當中.
就在二,三年的下半年開始.
我們跟隨主你.
求主你幫助.
求主你垂涎我們祈禱.
從主名叩.
阿門.
萬物都是從主而來.
我們得到的福分.
都是一個新的學習.
阿門.
剛才經文的第一,二節.
就在說頭生的分別為聖歸於耶和華.
是一個取捨.
也是給予他一個新的救贖.
而當我們了解這段經文.
在後面意味著什麼呢?.
就是真正能夠令我們得著救贖的.
就像我們這群信仰群體.
能夠在當中因為耶穌基督.
成為頭生的救贖.
以致我們今天能夠得著這個福分.
到了第三和第十節的時候.
大家剛才讀得最清楚的是吃無糧餅.
因為大家知道.

$^{401}$這是預知中其中一個很重要的食物.
無糧餅是什麼意思呢?.
最主要是讓子女們明白.
要紀念當時你們在沒有準備之下.
很急著要離開.
無糧餅對於當時的人來說.
其實是未準備好.
急於是要離開.
於是就拿走.
預備離開埃及之用.
要紀念什麼呢?.
就是你無法預備.
或者你不夠時間預備.
你願不願意走這條路.
不知道大家在做決定的時候.
要考慮多少.
有些人要考慮很周詳.
有些人要有很多不同的.
配搭都預備好的時候.
或者是齊全了.
他覺得合理的狀況的時候.
他才願意開戰那一步.
他才願意做那件事情.
他才做這個決定.
正如剛才所說.
離開埃及對於你.
是在想什麼呢?.
在我做教務這段日子的時候.
都會接觸不同的弟兄姊妹.
都會和他們一起走過一些日子.
其實對於我做教務.
或者一直陪伴弟兄姊妹.
聽他們的故事的過程當中.
其實有些過程.
按他們所說多少給我知道的時候.
我回應.
是否很難做決定呢?.
有些其實不是很難.
他是不願意還是不願意.
過程當中是什麼呢?.

$^{441}$有些事情其實他知道的.
但他不想做.
有些事情其實他明白的.
但他沒有力量.
所以對我來說.
我是了解他的困難.
因為他有很多事情想要很實在.
或者要有很足夠的東西.
他才願意去走那一步路.
但是對於我們這個信仰群體來說.
其實有很多事情真的不存在.
每件事都要處理好才去做.
有些事情是知道誰帶領你.
知道哪一件事對你好的時候.
就拿著信心走出去.
我希望你不要覺得.
我是在說一些假大空的事情.
但是你看到在這幾年.
我們這個信仰群體.
特別是Flow Church這個信仰群體.
我不是說其他.
我是說Flow Church這個群體.
我們真的沒有很多東西預備.
我們真的有很多東西沒有預備.
我們覺得那件事是要做.
就做下去.
我們覺得那件事是對的.
我們就堅持.
由我們2019年1月19日開堂的時候.
其實人數我們不多.
說的是三次公開聚會.
包含那時候在2018年10月8日.
在這裡做一個child version.
試一下一個崇拜頂.
節目覺得OK不OK.
然後回覆回來又覺得OK.
其實我也不知道是不是哄我們.
因為我們說我們每個人都這樣敬拜.
這樣講道.
沒有什麼報告就完結了.

$^{481}$OK,沒有什麼報告.
是不是.
最頭痕就是我們要找地方.
星期天找教會是很難的.
借給你.
我要選擇星期六.
星期六也不容易.
因為星期六很多教會.
如果在敬拜隊練習完之後.
他就封了台.
不讓人再動.
然後第二天早上回來就立刻上台.
預備敬拜.
星期六找教會是有空檔.
也不容易.
但我們是不是要齊了章才可以開始呢.
但你會發覺有些事.
做得多少就做多少.
因為我們相信那件事是上帝喜悅的.
招聚弟兄姊妹敬拜是上帝喜悅的.
讓其他教會未落腳的弟兄姊妹.
有落腳點也是上帝喜悅的.
讓弟兄姊妹能夠重拾對教會的熱誠.
是上帝喜悅的.
讓弟兄姊妹對信仰而有盼望.
也是上帝喜悅的.
這些全部我們覺得是上帝喜悅的東西.
哪怕很多東西硬件上不準備的時候.
我們都要走出去.
這個就是我們奉獻的心智.
這個就是我們要覺得群體看到要做的心智.
毛教餅說的一個要紀念的就是.
你們滾水落腳.
這些好像很老派.
你們不準備要出去的時候.
沒有什麼特別裝備.
沒有什麼特別配套.
沒有什麼好的套裝.
但你們要去的了.
你們去不去.

$^{521}$因為你要記住帶你那個是誰.
帶你那個就是之前幹了狗災那個.
說得出做得到的耶和華上帝.
你看春一級記的時候.
你會看到那個上帝和創世的上帝.
都是同一個上帝.
不過你會看到形態是很不同.
春一級記的上帝.
他是讓人明白到他自己的屬性是什麼.
我就是自由永有的上帝.
而這個自由永有的上帝.
就是說得出也是做得到的上帝.
你看到耶和華用大人的手這句話.
在春一級記常常出現.
因為他就讓人明白到.
我就是有能力那個.
你願不願意跟從.
我們這個群體去建立的過程當中.
走到今年第四年的時候.
我們仍然相信.
沒有很多配搭.
沒有很多裝備.
但我們仍然相信.
走在上帝要我們建立信仰群體的重點的時候.
這個奉獻是最真實的.
常常在報告或者在講道當中都跟弟兄姊妹說.
你的出席.
現場也好.
網上也好.
對Flow Church來說.
就是將信息繼續廣傳的地方.
我們的訂閱者不是很多.
你們會有看很多YouTuber的.
我們的訂閱者基本上是比很多YouTuber差的.
但對我們來說.
我們的觀看次數其實也不是少的.
我們現場有300多個弟兄姊妹在現場崇拜的時候.
現在實時有多少觀看次數.
我們差不多有300多個觀看次數.
同時大概有7-8個弟兄姊妹在實時崇拜.

$^{561}$但我們每個星期有7000觀看次數.
即是有九成的弟兄姊妹都在網上崇拜的時候.
大家在不同時區當中都認同這個信仰一直開展的重點是什麼.
真的.
Full Church的出現不是一定要很多東西齊全才做.
我們相信上帝的事情的時候就做下去.
每次述說這些就是在紀念那件事.
所以想Inflow的弟兄姊妹都常常聽我講解.
由19年,20年,21年我們走過的日子.
讓你明白到上帝對我們做了很多新事.
而那些新事是因為我們知道我們要堅持的是什麼.
不是有財有勢才做.
不是存夠錢才做.
是我們覺得去做就去做.
所以要問一問你.
你的生活如何.
當然如果在祈禱會的時候.
或者大家分組祈禱的時候問你.
你近來生活怎樣.
都是這樣.
差不多.
有工作,有飯吃.
還沒去旅行.
覺得還可以,去了幾次旅館.
當你問你的生活如何的時候.
你會覺得很客套的說話.
因為有些東西齊全就OK.
但其實我要問一句.
你的生活如何.
其實你自己知道的.
如果真的飲飲食食滿足到你的話.
其實你的飲飲食食是來自什麼呢.
如果你的生活不好的時候.
你是怎樣過呢.
仍然是說一句.
你生活好的時候你的信仰擺在哪裡.
你生活不好的時候你的信仰擺在哪裡.
你看到以色列人其實生活是如何.
你看到整個信仰歷程是一個受苦民族.
就因為他在信和不信當中不斷地返來復去.

$^{601}$因為他將上帝擺入彈出彈入.
你的生活如何.
上帝是不是在你生活當中彈出彈入呢.
你的生活如何.
你有沒有將上帝擺在那裡.
很危險的.
不要去.
是不是這樣.
認真去問一問.
你生活要什麼才是你的保障.
待會在蜜桃的時候.
當音樂響起.
希望帶動你去想一想.
其實你的生活.
有什麼是你最緊張的.
我最緊張的吧.
過了半年了.
你生活如何.
好的是什麼.
不OK的是什麼.
上帝在哪裡.
在你的生活當中.
有沒有什麼是你最緊張的.
雖然不是很齊張.
不是很完備.
但你知道這是上帝喜悅的.
你堅持.
我們繼續.
今天是下半年的開始.
你跟上帝說.
你想過什麼生活.
我們相信一切都是從上帝而來.
無論你信或逆.
上帝都願意在你的生命當中.
《無教餅》再次提醒.
我們要紀念的就是.
無論一個群體不得時的時候.
上帝仍然是預備的位置.
因為祂帶領我們出來.
你願意過這種信心的生活嗎.

$^{641}$讓上帝仍然是你生命的執法人.
預備者.
你願意將每天奉獻給上帝.
過這種生活嗎.
你跟上帝說.
祂說我們每人的生命氣息.
都在你的手中.
你讓我們過每一天.
你讓我們經歷每一天新的經歷.
我們願意在下半年.
重新調整.
頂尖會跟你說的話.
都在你面前.
求主你介入在我當中.
就好像昔日以色列民.
他們相信是上帝用奇妙的手.
帶領他們來新的一頁.
就好像帶領我們一樣.
我們祈禱.
奉耶穌的名求.
阿們.
《我愛你》.
我們都是從主而來.
我們把忠主而來的必求.
阿們.
在剛才讀第三段經文的時候.
你會看到第十一章裡.
有幾個重點.
就是「待贖」.
一隻羊羔的出現.
代替了一群猶太子民.
他長子投身的生命.
就將他贖出來.
你的經文又再重現.
又是日後問你的時候.
這是什麼意思.
投身的兒子都贖出來.
這是什麼呢.
贖長子這個事件是什麼呢.
猶太人家裡.

$^{681}$爸爸有三件事在家裡代表做.
第一件事就是逾越節.
第二件事就是國禮.
爸爸負責.
第三件事就是贖長子.
就是這件事.
贖長子也是這段經文.
取埃及第十三章一至十六節的經文.
說的這段贖長子.
贖長子是什麼呢.
就是原本他的兒子是要死的.
但是你看到祭身代替他.
把兒子贖出來.
即是調換.
這件事到了拉比之後.
他離開了埃及.
他竟然要紀念的時候.
他們用這個方式是怎樣做呢.
他們會用五個赤霞勒銀紙.
成為一個贖架.
在拉比讓他贖長子的過程中.
來代替那個祭身.
意思是什麼呢.
爸爸就像我一樣.
我用五個赤霞勒銀紙.
給拉比.
成為我大兒子的贖架.
這個過程當中就是做一個紀念.
讓他明白到.
其實你兒子的出現.
或者你兒子的價值.
對於你來說.
其實一定有一個代贖的過程.
對於猶太人來說.
這件事在做什麼呢.
就是每件事都有個原因和意義.
對你來說.
可能你聽過一些說話.
每個人都有一個價值.
不過可能你聽的那些是比較負面.

$^{721}$但是對於你來說.
你覺得你自己值多少錢呢.
我不是買保險.
我第一次接觸保險的時候.
也不是負面的.
也是一個保障.
我自己看那張保單的時候.
原來斷了手指又賠多少錢.
斷食指又貴過賠什麼.
左手又貴過右手.
右手貴過左手.
視乎你左手.
我是錯的.
但我覺得人體分了不同價錢.
算起來我值多少錢呢.
我那時候拿了保單.
加我自己.
如果我全身買的時候.
保險代理說不是這樣算的.
如果你全身有事.
是另外一個計法的.
我說 哦.
其實你覺得你值多少錢呢.
你有沒有想過你值多少錢呢.
我們好像去了海.
你有沒有想過你值多少錢呢.
如果你要一個贖價的話.
你覺得你要多少贖價.
或者反過來.
劇院一點.
如果被人標心.
你覺得你值多少錢贖金呢.
這些價錢性問題.
我不會回答的.
先不要說綁架.
現在很多電話騙案.
我不夠錢了.
去海外匯多少錢來.
你有沒有計算過你值多少錢.
你不會計算.

$^{761}$你從來都不會知道.
你都沒有想過.
但我們說.
我們重價買回來的重價重在哪裡.
你知道的.
你會看到整件事.
出埃及的時候.
那些人還不知道.
耶和華用了一個祭身的方式.
來代替了以色列子民的長子的贖價.
但是你會看到.
當一個真是屬於上帝的群體.
要歸納上帝.
成為上帝新子民的名下的時候.
我們看到耶和華上帝是用什麼方法.
是用他自己獨身兒子.
這個長子的名份.
買下去來換我們這班人.
是依這個重價買回來的.
你覺得你值不值得.
耶和華.
即是上帝用他自己獨身兒子的名.
買你回來.
你覺得你值不值得.
如果你想一想.
保羅提醒我們重價買回來的時候.
這個屬長子.
我們是因著耶和華的長子.
屬了我們出來的時候.
這個價值就非比尋常.
所以使人說你拿什麼去報答.
耶和華向你所施的一切厚恩呢.
你要舉起舊恩的杯.
稱頌耶和華的名.
因為是耶穌願意放棄他自己長子的名份.
死在十字架內.
來袋贖我們的身.
以致我們能夠再得著.
這個換不是容易的.
所以你會看到.

$^{801}$當日出埃及要建立一個新群體的時候.
今天我們這個新的群體.
是耶穌基督去建立的.
是耶穌基督用他換我們回來的時候.
我們要認清.
你覺得你自己一文不值.
但耶穌是為了你覺得你自己一文不值.
但他換你回來.
以致你今天開采.
你覺得你自己不值這個價錢.
或者你覺得可能我一千萬.
但你會發覺一千萬就可以買到耶穌嗎.
你多過一千萬.
我希望弟兄姐妹認真看看.
你自己的命.
你不要自貶身價.
真的不要自貶身價.
你的命是耶穌基督買回來.
你是被重價買回來.
是創天造地的上帝.
讓他的兒子成為我們的袋贖.
這是很重要的.
我們被贖出來.
很希望大家真而重之.
當我們講奉獻的時候.
既然上帝不惜自己的兒子幹世為人.
取了奴僕的樣式.
成為我們人的樣式.
救我們的時候.
救贖我們的時候.
你拿什麼去奉獻給耶華呢.
很希望在我們月系第一講的時候.
很認真跟大家講.
奉獻從來不是付多少錢的問題.
奉獻從來不是財幹能力的問題.
奉獻其實是一個很深層的意義.
就是耶穌基督在我們沒有奉獻之前.
他已經將他自己奉獻給我們.
耶穌在我們沒有回應之前.
他已經先做了一件事.

$^{841}$袋贖了.
我們還懵懵懂懂.
都不知道發生什麼事的時候.
他已經做了這個奉獻.
而他又沒有挾持我們.
說我做了這麼多事.
為什麼你不奉獻多一點給我.
耶穌不做這件事.
耶穌仍然等他情願.
當我們去了解贖獎之的時候.
我希望大家再認真想想.
其實奉獻是一個回饋.
奉獻是一個回應上帝給我們的恩典.
奉獻是一個我們見到上帝恩典多的時候.
我們願意奉獻更多.
是一次我感受到自己被珍而重之的時候.
是我願意全情投入奉獻給上帝.
是我感受到上帝給我恩典多的時候.
我願意再給多一點恩典.
這個就是那個苦人.
那個不體面的苦人.
一個沒有名聲的苦人.
去到耶穌面前.
他被赦免多.
他被接納多.
他奉獻就多.
他拿的是真拿大向高.
去給耶穌抹腳.
親愛的姐妹.
不要自貶身價.
不要覺得自己不值.
我們今天能夠成為主內傑的群體.
每一個都是耶穌用他的補血贖我們出來.
我們的回應就是.
你感受恩典多.
你回應多.
你用你的方式去奉獻給上帝.
這個是上帝的喜悅.
所以在最後的時候.
讓我們再一次去想想.

$^{881}$上帝待贖了我們.
耶穌待贖了我們.
那個贖價是多少.
可能你看多少.
他看多少.
不用比較.
但是真正的贖價就是.
耶穌基督已經死了.
掛在木頭上.
也是死而復活.
讓我們有生生命的養成.
很希望仍然是開展下半年的時候.
日子不斷地數算.
要多少才夠我不知道.
但是無論如何.
都是你開始要想的.
你用什麼方式去奉獻給上帝.
去回應上帝給你的恩典.
你用什麼方式去看到上帝.
在你生命當中施的恩典.
當你回想你的生命的時候.
你就知道主因是什麼.
所以保羅說.
你若嘗過主因的滋味.
就更當如此.
保羅很希望你回想.
其實你恩典多.
你回饋上帝.
或者你回應上帝.
應該要更多.
希望詩歌彈的時候.
你安靜的時候.
你數算上帝的恩典.
就從今時開始.
《歌舞昇平》.
當全地裔的人.
都在《歌舞昇平》慶祝的時候.
他們覺得這些是他們奴役得久的.
他們享受其中.
但是對於我們來說.

$^{921}$我們也有奴役得久的.
但是我們是否賺到.
耶穌基督為我們所做的事情.
但是今天我們每一位坐在這裡的弟兄姊妹.
我們與會的弟兄姊妹.
我們經歷上帝根深蒂固的弟兄姊妹.
都是耶穌基督.
首先用祂自己成為贖駕.
贖回我們.
(詩歌).
你餘下的生命.
你願意奉獻什麼.
來回應上帝的贖駕呢?.
你跟上帝說.
你問問祂.
或許你沒有什麼頭緒.
你現在問問祂.
(詩歌).
將我們仍然為.
我們有生命的氣息.
可以過每一日新的日子.
我們獻上禱告.
我們仍然為我們可以在香港.
在不同境地.
享受自由敬拜的日子.
我們仍然為我們可以.
有一個自由的思想空間.
我們有機會回想.
上帝在我們生命當中所付出的.
所預備的.
而耶穌基督成為我們的挽回.
成為我們的贖罪.
我們不要自貶身價.
因為耶穌基督是無價的.
耶穌基督用無價的身軀贖回我們.
沿主教導我們怎樣過每一日.
沿主教導我們怎樣善用.
你給我們的恩賜財國能力.
去回饋你.
奉獻給你.

$^{961}$我們仍然相信.
仍然宣揚萬物都是從主而來的.
愛將從主而來的.
盡獻給你.
求主你閱立我們.
誠心誠意的獻上.
奉主名求.
阿門.
\newpage



\section{以弗所書 5:1-2-20230708}
\label{sec:vN0n_pNkXlA}
\textbf{【網上崇拜】談犧牲|以弗所書5\_1-2|20230708 [vN0n-pNkXlA]}
\newline
\newline
連結: \href{https://youtube.com/watch?v=vN0n-pNkXlA}{\texttt{ https://youtube.com/watch?v=vN0n-pNkXlA}} ~~~~ 語音日期: 2023-07-08 
\newline
\newline
\hyperref[sec:J_OpyaPLYIE]{\small{< < < PREV SERMON < < <}}
~
\hyperref[sec:index_chronic]{\small{[返順時目]}}
~
\hyperref[sec:index_scriptual]{\small{[返順卷目]}}
~
\hyperref[sec:Pn_6i9ASSV4]{\small{> > > NEXT SERMON > > >}}
\newline
\newline
以弗所書 5:1-2-20230708
\newline
\begin{longtable}{cl}
\hline
\hline
章節 & 經文 (和合本修訂版)\\
\hline
5:1 & \begin{tabularx}{0.7\textwidth}{X} 所以,作為蒙慈愛的兒女,你們該效法神。 \end{tabularx} \\ \\ \relax
5:2 & \begin{tabularx}{0.7\textwidth}{X} 要憑愛心行事,正如基督愛我們,為我們捨了自己,當作馨香的供物和祭物獻給神。 \end{tabularx} \\ \\ \relax
5:3 & \begin{tabularx}{0.7\textwidth}{X} 至於淫亂和一切污穢,或是貪婪,在你們中間連提都不可,這才合乎聖徒的體統。 \end{tabularx} \\ \\ \relax
5:4 & \begin{tabularx}{0.7\textwidth}{X} 淫詞、妄語和粗俗的俏皮話都不合宜;總要說感謝的話。 \end{tabularx} \\ \\ \relax
5:5 & \begin{tabularx}{0.7\textwidth}{X} 要確實知道,無論是淫亂的,是污穢的,是貪心的(貪心的就是拜偶像的),在基督和神的國裡都得不到基業。 \end{tabularx} \\ \\ \relax
5:6 & \begin{tabularx}{0.7\textwidth}{X} 不要被人虛浮的話欺騙了,因這些事,神的憤怒必臨到那些悖逆的人。 \end{tabularx} \\ \\ \relax
5:7 & \begin{tabularx}{0.7\textwidth}{X} 所以,不要與他們同夥。 \end{tabularx} \\ \\ \relax
5:8 & \begin{tabularx}{0.7\textwidth}{X} 從前你們是暗昧的,但如今在主裡面是光明的,行事為人要像光明的子女— \end{tabularx} \\ \\ \relax
5:9 & \begin{tabularx}{0.7\textwidth}{X} 光明所結的果子就是一切的良善、公義、誠實。 \end{tabularx} \\ \\ \relax
5:10 & \begin{tabularx}{0.7\textwidth}{X} 總要察驗甚麼是主所喜悅的事。 \end{tabularx} \\ \\ \relax
5:11 & \begin{tabularx}{0.7\textwidth}{X} 那暗昧無益的事,不可參與,倒要把這種事揭發出來。 \end{tabularx} \\ \\ \relax
5:12 & \begin{tabularx}{0.7\textwidth}{X} 因為,他們暗中所做的,就是連提起來都是可恥的。 \end{tabularx} \\ \\ \relax
5:13 & \begin{tabularx}{0.7\textwidth}{X} 凡被光所照明的都顯露出來, \end{tabularx} \\ \\ \relax
5:14 & \begin{tabularx}{0.7\textwidth}{X} 因為使一切顯露出來的就是光。所以有話說:「你這睡著的人醒過來吧!要從死人中復活,基督要光照你了。」 \end{tabularx} \\ \\ \relax
5:15 & \begin{tabularx}{0.7\textwidth}{X} 你們要謹慎行事,不要像無知的人,要像智慧的人。 \end{tabularx} \\ \\ \relax
5:16 & \begin{tabularx}{0.7\textwidth}{X} 要把握時機,因為現今的世代邪惡。 \end{tabularx} \\ \\ \relax
5:17 & \begin{tabularx}{0.7\textwidth}{X} 不要作糊塗人,要明白主的旨意如何。 \end{tabularx} \\ \\ \relax
5:18 & \begin{tabularx}{0.7\textwidth}{X} 不要醉酒,酒能使人放蕩;要被聖靈充滿。 \end{tabularx} \\ \\ \relax
5:19 & \begin{tabularx}{0.7\textwidth}{X} 要用詩篇、讚美詩、靈歌彼此對說,口唱心和地讚美主。 \end{tabularx} \\ \\ \relax
5:20 & \begin{tabularx}{0.7\textwidth}{X} 凡事要奉我們主耶穌基督的名常常感謝父神。 \end{tabularx} \\ \\ \relax
5:21 & \begin{tabularx}{0.7\textwidth}{X} 要存敬畏基督的心彼此順服。 \end{tabularx} \\ \\ \relax
5:22 & \begin{tabularx}{0.7\textwidth}{X} 作妻子的,你們要順服自己的丈夫,如同順服主。 \end{tabularx} \\ \\ \relax
5:23 & \begin{tabularx}{0.7\textwidth}{X} 因為丈夫是妻子的頭,如同基督是教會的頭;他又是這身體的救主。 \end{tabularx} \\ \\ \relax
5:24 & \begin{tabularx}{0.7\textwidth}{X} 教會怎樣順服基督,妻子也要怎樣凡事順服丈夫。 \end{tabularx} \\ \\ \relax
5:25 & \begin{tabularx}{0.7\textwidth}{X} 作丈夫的,你們要愛自己的妻子,正如基督愛教會,為教會捨己, \end{tabularx} \\ \\ \relax
5:26 & \begin{tabularx}{0.7\textwidth}{X} 以水藉著道把教會洗淨,使她成為聖潔, \end{tabularx} \\ \\ \relax
5:27 & \begin{tabularx}{0.7\textwidth}{X} 好獻給自己,作榮耀的教會,毫無玷污、皺紋等類的缺陷,而是聖潔沒有瑕疵的。 \end{tabularx} \\ \\ \relax
5:28 & \begin{tabularx}{0.7\textwidth}{X} 丈夫也應當照樣愛妻子,如同愛自己的身體;愛妻子就是愛自己了。 \end{tabularx} \\ \\ \relax
5:29 & \begin{tabularx}{0.7\textwidth}{X} 從來沒有人恨惡自己的身體,總是保養愛惜,正像基督待教會一樣, \end{tabularx} \\ \\ \relax
5:30 & \begin{tabularx}{0.7\textwidth}{X} 因我們是他身體的肢體。 \end{tabularx} \\ \\ \relax
5:31 & \begin{tabularx}{0.7\textwidth}{X} 「為這個緣故,人要離開父母,與妻子結合,二人成為一體。」 \end{tabularx} \\ \\ \relax
5:32 & \begin{tabularx}{0.7\textwidth}{X} 這是極大的奧祕,而我是指基督和教會說的。 \end{tabularx} \\ \\ \relax
5:33 & \begin{tabularx}{0.7\textwidth}{X} 然而,你們每個人都要愛妻子,如同愛自己一樣;妻子也要敬重她的丈夫。 \end{tabularx} \\ \\
[1ex]
\hline
\hline
\end{longtable}
$^{1}$好 今天早上平安.
今天穿得這麼漂亮.
因為去別人的教會講道.
如果上星期有回到church去玩.
你可能會發現.
第一個星期不是John講的嗎.
沒錯 上星期我跟潘Sir調了講道.
因為我去了一間天水圍的教會講道.
所以大家上星期在那裡敬拜的時候.
我同一時間就在天水圍講道.
這間教會是二十週年堂慶.
邀請我去講他們的培靈會.
通常以前的教會.
星期六晚我都不會接的.
因為撞了時間.
但是因為我自己在天水圍長大.
所以我立志講過任何天水圍教會講道.
我都不會推的.
所以我上星期就在天水圍.
跟幾十個弟兄姊妹一起聚會.
很感恩.
我去了之後發現了一件事.
就發現了野生捕獲了.
這個土告的距離的作者的媽媽.
原來她在那裡聚會的.
Boris的媽媽.
當天我就親自代表呼出謝謝她.
謝謝她生了個兒子出來.
讓我們呼出這麼多好的師哥.
在敬拜裡面這麼好的幫手.
上星期就不在了.
所以沒機會為這個月題解解話.
通常我第一個都會先解話.
講一下為什麼我們Full Church要有奉獻的主題.
有人問為什麼要講奉獻.
Full Church很缺錢.
我會回答是.
不過這個不是重點.
短期內一間教會四年.
有需要去過渡一些我們要去過渡的東西.

$^{41}$今天我們會講一篇有關奉獻的訊息.
不過我自己沒試過講.
我從來都沒試過講奉獻的道.
所以預備這篇道的時候.
我肯定會例牌.
就會上網看看其他行家怎麼說.
所以這星期我聽了五至六篇.
有關其他教會叫人奉獻的道.
我發覺這裡總括而言大概分幾個套路.
我嘗試去講出來.
什麼套路呢.
第一個就是奉獻就是祝福論.
奉獻就是給上帝祝福的途徑.
只要你肯奉獻.
上帝就會祝福你.
基本上用馬利基說第三章.
如果你肯奉獻的話.
就試試我是否有天上的窗戶.
傾福願你甚至無處可慾.
這是聖經的經文.
第二個就是奉獻就是感恩論.
基本上奉獻是基督徒的回應.
只要你感受到上帝對你的恩典.
你就需要以奉獻來回應上帝給你的恩典.
也是對的.
萬物都是從主而來.
我們都是從主而來的.
歸給主.
第三個就是奉獻是信心的考驗.
奉獻就是一些試煉.
看看你願不願意經歷上帝的公認.
明知是不容易的.
但你也要嘗試去經歷一下.
這是一個奧煉.
你要嘗試去堅持奉獻的試驗.
最後就是告急.
教會告急.
教會沒有錢.
教會需要什麼.
建設或賺錢.

$^{81}$大家快點幫忙.
神的家就方能了.
諸如此類.
今天不是說這四套套路.
我自己走.
剛才那些不是錯的.
奉獻是祝福.
奉獻有時候也是感恩的回應.
也是我們信心的一些考驗.
也有時候教會告急.
不過今天我們想嘗試從一個更深的層次.
去思考奉獻的主題.
我今天會說一個主要的主題.
主要是港獨.
不是在《鞭石經》說到.
雖然用了《二分楚書》第五章的經文.
但今天我們會看很多段經文.
去思考奉獻或犧牲的主題.
今天信息也很豐富.
大家都有心理準備.
我們一起祈禱.
天父多謝你.
讓我們能夠一起聚會.
我自己很想念.
我們一起去崇拜.
今天看到這麼多的弟姐妹.
我們在網上.
在白龍堂這裡.
我們願意.
今天我們沒有其他目的.
我們願意獻情給你.
成為一個生命的活祭.
求主你願意.
讓你的說話.
讓你自己的教導.
去幫助我們.
去實踐一個你所要求我們的生命.
求主你願意幫助.
幫助孩子.
孩子也是一個軟弱不配的人.

$^{121}$但你自己來去真摯說話.
奉尊永求.
阿門.
大家可能會問.
為什麼奉獻和犧牲這個題目有關係.
因為我們是說談犧牲.
大家會覺得.
明明是奉獻.
為什麼要說到犧牲這麼嚴重.
每次都說到犧牲會不會太大.
太heavy.
不過如果你細心去思考奉獻這個字的時候.
奉這個字其實.
在談金錢這個題目之前.
其實它的意思就是獻上.
就是給予.
就是offering.
這樣的意思.
所以從聖經的角度來看.
奉獻的本質就是offering.
就是你願意去向神獻上.
擺上一些你所擁有的東西.
所以與其說奉獻這個題目和金錢有關.
不如說這是基督徒為上帝去付出.
去給予奉獻犧牲一個這樣的題目.
從聖經的角度來看.
特別是猶太人這幾千年歷史.
奉獻給神其實就是獻祭這回事.
古代以色列人奉獻給上帝不是FPS.
不是Paycheck.
而是獻祭.
而獻祭這個課題.
正如是關乎於犧牲的課題.
就是宰殺牛羊獻上犧牲去呈獻給神.
希望大家明白這個邏輯.
奉獻獻上offering.
獻祭犧牲.
這幾個字是有關連的.
奉獻獻上offering.
獻祭犧牲.

$^{161}$說到這一點你明白.
其實我們看回中文.
中文犧牲這個字已經說了很多意思.
中文犧牲這個字有兩個不同意思.
首要地是一個名詞.
犧牲這個字最原本是一個名詞.
不是一個動詞.
看回中文字典的時候.
你會發現犧牲這個字是古代用的一個名詞.
什麼叫犧牲.
犧牲就是以一個祭祀為目的.
以祭祀成為祭品的牲畜.
牛羊豬雞鴨魚等等.
所以中文犧牲這個字從來都是說一個.
什麼報仇.
牲畜的報仇.
因為它本身是一個這樣的名詞.
後來近代的中文才慢慢慣常變成一個動詞.
延伸的意思是為了其他人的利益.
正義公益等等.
去捨棄自己的利益.
甚至乎是生命.
犧牲.
所以犧牲本身是一個祭祀上的動物的名詞.
然後才說出這樣的動詞.
所以我們先看看舊約裡面.
有關憲制這個課題.
如果不是去計算非自願的憲制的話.
即是贖罪制 潔淨制 賠償制的話.
舊約那些自願獻上的制包括什麼.
包括有凡制 素制 平安制.
這些都是人自願去奉獻給上帝的東西.
先說一點時間來簡單說幾句.
素制就是解作禮物.
即是進貢的意思.
用的是無烤餅 油 或乳香組成.
不是包括動物的祭品.
目的是要向神表達感恩或敬拜.
平安制的重點是共享.
即是憲制者和上帝共享得來的禮物.

$^{201}$祭物是分銷在祭壇上.
剩下的也會分享給祭司和有關的人.
目的是向神表達感恩和和平的關係.
想說多一點的就是凡制.
凡制是最高級的憲制.
A++的.
什麼是凡制.
凡字是燃燒的意思.
所以凡制就是燃燒的意思.
即是要將你的財產完全燒掉.
燒到化為烏有.
所以是凡字的意思.
完全燒到灰燼的意思.
凡制就是將你整個財產.
即是你的動物除皮後.
整個動物都燒在壇上.
然後讓煙升到上帝那裡.
凡制就是以斯文自願來獻上.
凡制是最神聖的.
因為它是一種傳言的獻上.
一個毫無保留的獻上.
最重要的是什麼.
凡制是奉獻者巨大的付出.
所謂凡制就是無論如何.
這個最重要的是什麼.
就是奉獻者自己的財產.
你不能燒別人的牛.
你要燒了你自己擁有的一隻牛.
凡制是獻上最有價值的.
國愛的.
來獻上給上帝.
這個完全燒掉了.
沒有了.
這是一個毫無保留.
傳言獻上的一個獻祭.
不過可能對大家來說.
獻上一隻牛一隻羊.
可能大家都沒有什麼感覺.
反正牛羊對大家來說不是太有價值.
現在你基本上可以去日式的黃牛燒肉.

$^{241}$幾百塊去任吃.
很便宜的東西.
你感覺不到的.
用一個比較大家可以理解的例子.
今天什麼叫凡制.
就是你拿你的iPhone出來.
燒了它.
燒到化為灰燼.
iPhone 說真的iPhone.
不是紙紮iPhone.
你知道今天那些明顯的產品.
是很多新的東西.
iPhone有了.
iPhone都出了.
我最近都研究過.
有一個很正的新產品.
是什麼呢 你猜猜.
先猜猜.
有什麼新產品.
是地虎口罩.
還有紅外線體溫測驗器.
還有酒精消毒液.
你想想.
人都死了.
還要戴口罩.
還要怕口罩靈.
是不是.
還要怕有Virus感染你再死一次.
很恐怖.
不過想強調什麼呢.
就算這個是口罩也好.
這個也是假的.
是假口罩.
用回精神.
因為是假的.
用回假口罩來做一個真正的明顯產品.
所以你發現.
所有的東西都是假的.
基本上是一個非常低成本的東西.
我剛剛上HKTV mall查過.

$^{281}$現在燒的那些銀行的錢.
38元600張.
每張5000萬.
你想想.
每張.
我算過.
一元港幣兌7億.
兌7億.
銀行下面的錢.
下面麥當勞要多少錢.
要200億.
OK.
我們編題了.
所以是真的.
上帝要求我們燒真正的牛.
真正割肉心痛的東西.
不是燒那些700億200億的東西.
所以舊約裡面的憲制.
無論是什麼形式也好.
其實基本上有一個很重要.
忽略重點.
不是說什麼功能.
什麼作用.
什麼意義.
而是全部的憲制.
都是憲制者的真正代價.
你不能夠燒一隻紙質牛.
你只能夠完全將你真正的牛.
你的財產燒了.
憲制是人對上帝付出.
犧牲割肉一般的付出.
憲制者將最寶貴的.
他視為有價值的.
覺得有些心痛的.
那種財產物品.
毫無保留的.
獻上給上帝.
所以明白為什麼聖經說.
不可以獻殘疾的那些憲制.
也不是說什麼是最好的問題.

$^{321}$而是殘疾對你來說是不值錢的.
反正我都不要了.
我燒給你吧.
不是這個意思.
是要一些你覺得是心痛的.
是要你拿以為生的.
是有價值的.
你要將它獻上.
所以這個就是憲制的意思.
這個就是舊約.
來到新約的時候.
大家知道耶穌基督成為了我們的挽回制.
世界上最大的犧牲.
就是上帝的犧牲.
上帝犧牲他的獨生兒子.
來拯救我們.
這個犧牲成為了挽回制.
所以我們看到大澤經文.
羅馬書說.
神設立耶穌做挽回制.
是憑著耶穌的血.
是憑著人的信.
要顯明神的義.
帕特羅書說.
所以神凡事該與他的弟兄相同.
為要在上帝的事上成為慈悲.
忠信的大祭司.
為白色的罪獻上挽回制.
剛剛前書說.
因為我們如意之的羔羊基督已經被殺獻制了.
然後書說.
他為我們的罪作為挽回制.
不是單為我們的罪.
也是為普天下人的罪.
聽著.
耶穌基督不單單是那位大祭司.
他更加是獻祭裡面獻上的犧牲.
他同時是祭司.
同時是獻祭裡面的羔羊.
所以我們剛才所說.

$^{361}$獻祭犧牲有兩個不同的意思.
耶穌在十字架上死是一種犧牲.
他為世人犧牲了自己的生命.
甘願捨棄生命來拯救我們.
這是現代意義的犧牲.
不過同時十字架的犧牲.
正正就是什麼.
就是獻祭的犧牲.
那個羔羊.
耶穌基督是獻上祭壇上的犧牲.
他就是上帝所獻上的那隻羔羊.
所以耶穌基督不單單是大祭司.
更加是獻上的犧牲.
我們今天就問.
我們還需要獻祭嗎.
我們今天還需要獻祭嗎.
根據先講講.
不用.
最少贖罪祭是不用的.
耶穌基督在十字架上成為了挽回祭.
一次過終結了一切人所需要獻上的贖罪祭.
最少今天你不需要在福音書中排隊.
拿一隻牛回來給我們去宰.
這個不需要的.
請不要這樣做.
雖然我們不需要帶犧牲回來崇拜.
但先強調.
我們是需要獻上給上帝.
我們仍然需要犧牲.
不過不是那種犧牲.
你要獻祭給上帝.
所講的不是牛或羊.
而是你自己.
大家都熟經文.
羅馬書十二章.
所以弟兄們.
我以神的慈悲勸你們.
將身體獻上.
當作活祭.
是聖潔的.

$^{401}$是神所喜悅的.
你不用獻上一隻牛.
你卻要獻上你自己.
你就是那個兒子.
你不需要用牛當作犧牲.
你卻要將自己當作犧牲.
所以保羅用了活祭這個字.
我講過了.
在哪裡我講過.
活祭這個字是一個很奇怪的字.
活是形容詞.
祭是名詞.
但要強行將兩個字組合在一起.
只有保羅這樣用.
明明祭是甚麼.
就是燒盡 燒死.
要殺了他.
但活祭是一個非常奇怪的組合.
保羅說你要成為活祭.
就像以前的以色列一樣.
向上帝奉獻你所有最寶貴的.
最重要的.
國慾一般的來獻上.
這是我們向上帝付出的本質.
正正就是活祭的意思.
有了以上這些背景.
我現在開始看今天的經文.
《已福俗書》第五章一到二節.
我們一起讀吧.
經文裡面所講的一句說話.
預備 1 2 3.
(經文).
很有趣.
保羅在這本書裡面.
提到我們要獻上輕香的貢物和祭物.
甚麼意思.
甚麼是輕香的祭物.
輕香的祭物只是一個很普通的形容詞.
好像沒有甚麼特別意思.
只是加一個形容詞.

$^{441}$好像小時候的作品一樣.
花袋總是加一個「美麗的」.
沒有甚麼意思.
只是加一個形容詞.
不是.
保羅不是隨便加一個形容詞.
余明斯是甚麼.
其實它不是一個普通的形容詞.
余明斯是說.
當我們獻上我們的貢物和祭品的時候.
它會發出香氣.
這些香氣會叫上帝喜悅.
所以輕香的祭物不是隨便加上去的意思.
而是很重要的.
這些貢物會發出一些香氣.
上帝會喜歡這些香氣.
當然經文裡面是在回應梵祭.
因為梵就是在救這邊.
祭物燒完之後.
發出來的味道.
都是同樣的意思.
不過你問為甚麼上帝那麼喜歡聞這些祭物.
為甚麼上帝那麼在乎這些香氣.
為甚麼有這樣的癖好.
要黏著那些東西去聞.
祭物是不是那些.
中國的奶叉燒的概念.
很喜歡聞那些叉燒的.
不是這樣的.
上帝不是真的.
這個不是那個意思.
上帝不是真的有興趣聞那些祭物的香氣.
神都不是肚餓.
要你燒一隻牛給祂吃.
或者要燒些甚麼給祂.
甚麼意思呢.
為甚麼這裡本來要說到.
一個這樣的輕香的祭物呢.
一個根本問題就是.
為甚麼上帝要我們獻祭.

$^{481}$獻祭做甚麼.
為甚麼上帝要我們獻祭燒給祂.
既然上帝不是真的要去吃你燒的東西.
或者聞你香氣的時候.
上帝是一無所缺的.
又不是告急.
又不是甚麼.
為甚麼要你這樣奉獻給祂呢.
我們為甚麼要付出.
為甚麼要犧牲呢.
上帝你知道.
這些獻祭不是賄賂祂的.
不是燒了之後祂會給你甚麼好處.
奉獻不是一個利益的交易.
所以為甚麼我們要去獻祭.
為甚麼我們要去奉獻.
為甚麼我們要犧牲.
這裡保羅為我們提供了非常重要的獻祭理由.
就是因為要.
保羅說了一個很有趣的字.
保羅叫我們要效法上帝.
效法上帝意思是我們和上帝做同樣的事情.
上帝做甚麼我們都做甚麼.
實際上心裡有很多有關效法的字眼.
效法基督.
效法保羅.
效法上帝.
你會明白保羅基督多一點.
效法保羅我明白的.
效法基督我也能理解.
但效法上帝即是效法甚麼呢.
上帝做了那麼多我們要做甚麼呢.
我們要跟隨甚麼呢.
根據上下文來說.
保羅似乎意思是甚麼.
我們要去效法天父上帝的犧牲.
上帝犧牲祂獨生兒子給我們.
所以你們該效法上帝.
好像蒙慈愛的女兒一樣.
要憑愛心行事.

$^{521}$正如基督愛我們.
為我們寫記.
所以我們要作為一個天父蒙愛的兒女.
也要效法我們的天父.
他做了甚麼我們做了甚麼.
他犧牲奉獻獻上了他的獨生兒子給我們.
因為愛的緣故.
我們同樣因為愛的緣故.
我們都獻上我們的所有.
獻上我們所有的東西.
我們的貢物和祭品.
來獻給上帝.
這才是真真正正的信仰關係.
真正的天父和天父兒女的正常關係.
天父上帝為你犧牲.
你也同樣為上帝犧牲.
犧牲是一種關係.
願意付出.
因為基於這份關係.
父母願意付出自己去愛護小孩子.
小孩子長大後願意付出犧牲孝順父母.
天經地義.
我請你喝茶.
你幫我有時照顧我的女兒.
生日送禮物給你.
新年你煮好東西給我吃.
暑假我帶你去旅行.
我病了你幫我排隊拿藥.
你幫我出公樓那份首期.
你老了之後我又照顧你.
這是一個很正常不過的關係.
因為大家是這樣的關係.
你為了我我為了你.
所以奉獻犧牲就是這樣的關係.
不用問為什麼.
就是第一個關係.
所以回到我們奉獻這個話題.
今天我們講奉獻.
為什麼我們要奉獻.
我們奉獻不是因為出於利益.

$^{561}$是因為我奉獻了.
我就試試看.
看看上帝會不會闖開天上的窗戶給我什麼.
這樣的話只不過是一個宗教投資.
你奉獻出來為了上帝給你更好的東西.
不是這樣的.
奉獻也不是純粹一種還神的概念.
因為我很感恩.
我感覺到上帝有很多好東西給我.
我買美股賺了一筆錢.
所以我就奉獻.
感恩奉獻是好事.
但這不是奉獻最基本的原因.
當然奉獻更不是純粹信心考驗.
淑玲的操練.
想挑戰的東西.
或者是高級幫手.
這也不是最主要的原因.
最重要的原因是什麼.
我們要奉獻.
因為我們就是會奉獻.
這是我們和上帝關係不可缺少的一部分.
一份正正常常和上帝關係不可或缺的部分.
沒有了這個部分.
假如我們信仰沒有了奉獻這個元素的時候.
我們的信仰就不是完整的.
我們的關係就不是完整的.
不是說要靠奉獻得夠.
也不是靠奉獻得夠.
而是奉獻是讓我們和上帝的關係.
才有一個完整的關係.
你們該效法上帝.
就像蒙慈愛的賢女一樣.
要憑愛心行事.
正如基督愛我們.
為我們寫記.
當作輕香的供物和祭物獻於上帝.
上帝很愛我們.
奉獻了我們的獨生兒子給我們.
為我們犧牲.

$^{601}$我們也做同樣的事情.
不用問為什麼.
同樣地我們願意學習.
犧牲奉獻自己給上帝.
正如我所說.
彼此犧牲是一段正正常常.
父母和兒女的一段關係.
大家想想.
沒有了彼此犧牲.
我們和神的關係就變成怎樣.
一個很單方面的交易.
我在斜宮廟裡給點錢添香油.
期望皇上給我八億.
這是一個宗教交易.
你借一百元給我.
我還一百元給你.
這是一個金錢來往.
我投資一百元給你.
期望有一百元以上的回報.
這是一種投資.
相反我犧牲一百元給你.
你也犧牲一百元給我.
這是一個正常.
大家一個愛的關係.
大家的一百元來往完全不同.
一個買一買.
一個借一個還.
最後是什麼.
是零.
但是我給你一百元.
你給我一百元.
最後是什麼.
最後是二百元.
是一個完全不同的愛的關係.
靜姐妹.
你的信仰是不是這樣.
所以奉獻是重要的.
因為犧牲付出擺上.
是我們信仰裡很重要的元素.
我以神的慈悲勸你們.

$^{641}$將身體獻上當作活祭.
是聖潔的.
是神所喜悅的.
所以基本上金錢奉獻是最低的基本.
我所說的是整個人的獻上.
如果我們信仰沒有了這些奉獻的元素.
我們只不過是一些宗教活動.
純粹拿東西撿東西.
就走了.
奉獻讓我們知道.
我們實踐了我們和上帝的關係.
不單單是單方面的去接受.
而是彼此的付出.
回應和犧牲.
你不是靠奉獻得救.
也不是奉獻換取什麼.
而是因為大家是父母兒女關係的時候.
大家就不用問為什麼.
就這樣彼此犧牲.
再說教會正正是一個彼此犧牲.
彼此付出的地方.
教會是讓奉獻發生出現的地方.
我所說的不是信徒弟子妹奉獻的那兩件事.
而是無論是上帝也好.
牧者也好.
弟兄姊妹也好.
全部人參與信徒的派對.
都是一起去奉獻犧牲自己的地方.
我知道留堂這個群體很特別.
我們每一個人都是離開自己的教會.
來到這裡.
在這裡翻起堂的你.
或者在YouTube直播崇拜.
或者翻看的你.
你可能過往在不少的經驗裡.
有很多教會不愉快的經驗.
對教會失望.
被教會當作工具人.
或者教會只想要錢.
但知道全教會不是這樣的地方.

$^{681}$對我來說.
全教會是一個理想的教會.
是彼此犧牲.
彼此幫助.
彼此付出.
彼此奉獻的地方.
無論是我們的上主.
教會的傳道.
牧者.
每一個弟子妹.
教會這三個派對.
正正都是一個願意付出犧牲的地方.
上帝的犧牲.
教會牧者無私奉獻的犧牲.
弟兄姊妹奉獻的犧牲.
三件事加在一起.
才是一個真正美麗的信仰.
缺不可的.
大家想想.
如果單單只有願意奉獻犧牲的教會傳道牧者.
單單只有願意奉獻的信徒.
而沒有那位願意奉獻犧牲的上帝的時候.
這是什麼?.
這是一個假的偶像崇拜.
大家在拜一個假神.
因為這個神不是願意為你犧牲的.
第二.
如果單單只有願意為世人犧牲的上帝.
單單只有願意奉獻犧牲的信徒.
而教會牧者卻沒有願意犧牲的時候.
這是什麼?.
這是一個攝理教.
這是一個純粹操控人心的一個牟利的宗教組織.
第三.
同樣道理.
如果單單只有願意犧牲的上帝.
單單只有願意犧牲的教會牧者.
而沒有願意犧牲的信徒.
弟兄姊妹.
這只不過是一個宗教消費的地方.

$^{721}$唯有上主的犧牲.
傳道牧者.
伯伯的無私的犧牲.
弟兄姊妹的犧牲同時出現.
才成為一個真正我們覺得美麗的信仰.
更重要的是在這個時候.
當這三樣的犧牲同時出現.
犧牲才會成為一個真正的因典.
教會是一個犧牲出現的地方.
然後才被發現成為一個因典的地方.
沒有犧牲的元素.
任何一方犧牲的元素.
這個信仰就不完整.
教會裡面沒有任何一方是free rider.
今天大家都不喜歡free rider.
崇拜的奉獻環節提醒我們.
我覺得崇拜的奉獻環節.
好像永遠都是最低潮的時間.
不知道為什麼要做.
崇拜環節不是純粹是為做而做.
它正正是提醒你.
你回來這裡不是去參與一個宗教服務.
你來這裡是要獻上一個祭給上帝.
雖然敬拜很棒 講道很好.
但你來的目的是為了奉獻給上帝.
奉獻這個環節正正是告訴你.
你不要誤會.
Full Church這個崇拜.
是真真正正的崇拜.
是奉獻給上帝的崇拜.
你就是那個祭物 你就是那個活祭.
有些人不知道.
我在流堂裡面是沒有受薪的.
因為我在建道裡面侍奉.
這四年來我一分錢都沒有收過流堂裡.
虧錢都是少之又少.
我自己每個月在Full Church裡面的講道.
甚至連講完費都沒有.
但Full Church的講道.
它是我最用心的講道.

$^{761}$甚至是唯一會寫新講章的講道.
真的不好意思 其他的教會.
其他的教會雖然有講完費.
剛才那些是講完費的.
但我只能夠獻上次好給你.
我只會將新講章給Full Church.
其他的講完費我都有錢收的.
但我不會寫新講章.
擺明的 擺明的.
你要不要就算了.
每個禮拜 每次我講道.
我花了整個禮拜的時間.
獻上自己的心血.
就是寫一篇沒錢收的講道.
如果純粹要收到一些錢的話.
我不會這樣做.
我會重複我這麼多年的道.
我已經有很多時間翻炒了.
我不是自誇.
我們流童的每一個牧者.
潘Sir每一個牧者.
本身就是一個奉獻的人.
奉獻自己的生命.
為了弟兄姊妹.
獻上自己的所有.
兩個小時電話 三個小時電話.
出來約你.
然後你遲到.
很多的事.
不要緊 因為他們是預了.
他們是奉獻出來的.
大家不要遲到.
敬拜也是一樣.
敬拜隊每個禮拜.
下午開始排練預備.
收拾東西.
這個都是犧牲.
這個都是奉獻.
流童正正是需要每一個.
每一個弟兄姊妹.

$^{801}$一起去奉獻犧牲.
這個才是真真正正的教會.
起碼是理想的教會.
我知道的弟兄姊妹.
你來到流童.
我聽過很多時候大家的回應.
Full Church是一個醫治你的地方.
一個你這幾年裡面很多的創傷.
你能夠找到上帝的地方.
一個能夠在敬拜裡面投入.
能夠在港道裡面找到上帝的地方.
但讓Full Church成為一個更加好的地方.
就是需要大家一起去付出.
大家一起去奉獻.
大家一起去犧牲.
犧牲只要大家一起犧牲的時候.
當教會沒有Free Rider的時候.
這個犧牲就是很美麗的事情.
這個就是理想中的教會.
最後我想說一個故事.
結束今天的訊息.
從前有一個受傷的女孩.
她滿身傷痕滿身鮮血.
拖著很疲倦的身軀.
在森林裡面暈倒了.
森林裡面有間屋.
屋裡面有個農夫.
看到這個女孩滿身鮮血.
於是就一手抱起這個女孩.
帶她回家裡面去幫助她.
這個農夫沉默寡言.
這個受傷的女孩都沒有發出一句聲音.
但是雖然大家都沒有出聲.
但農夫就每天幫她更換繃帶.
清洗傷口.
溫柔的 細心的.
女孩在屋裡面感到非常舒服.
一個充滿愛和關懷的地方.
女孩在痛苦裡面找到好一點的安慰.
每天清晨當第一個光線射入家裡的時候.

$^{841}$農夫就為這個女孩準備一杯新鮮的鮮奶.
鮮奶裡面非常濃郁的香氣.
滋潤著女孩整個的身體.
給她新的力量.
每當每晚夜晚的時候.
農夫在她旁邊彈奏一個非常溫柔的樹琴.
柔弱的旋律.
如同一個搖籃曲一樣.
輕輕地來叫她入睡.
如此日復一日 年復年.
我都不知道過了多長時間.
可能是幾個星期.
幾個月 一年 還是四年呢.
有一天 早上的時候.
突然傳來陣陣的「渣渣」聲的聲音.
「渣渣渣」.
好像煎香腸的味道.
然後更加是一些暗烈的味道.
旁邊有一杯很熱的咖啡.
已經預備好了.
原來是女孩子預備的早餐.
女孩將她的最好獻上.
因為這個時候.
女孩已經真真正正地來到她的痊癒.
頂姐妹期待Full Church.
正正是一個這樣的地方.
一個叫人痊癒的地方.
一個叫人快樂地奉獻的地方.
一個有恩典的地方.
一會兒我們會唱回應詩.
這首詩是我自己很喜歡的詩歌.
「我願意給你」.
這首歌其實我都說過.
要跟足歌詞唱.
認真.
要mean it這樣唱是不容易的.
因為是說「我願意給你」.
\newpage



\section{瑪拉基書 3:6-12-20230722}
\label{sec:k_DUSZ_Q45k}
\textbf{【網上崇拜】係呀,講瑪拉基書|瑪拉基書3\_6-12|20230722 [k-DUSZ-Q45k]}
\newline
\newline
連結: \href{https://youtube.com/watch?v=k-DUSZ-Q45k}{\texttt{ https://youtube.com/watch?v=k-DUSZ-Q45k}} ~~~~ 語音日期: 2023-07-22 
\newline
\newline
\hyperref[sec:Pn_6i9ASSV4]{\small{< < < PREV SERMON < < <}}
~
\hyperref[sec:index_chronic]{\small{[返順時目]}}
~
\hyperref[sec:index_scriptual]{\small{[返順卷目]}}
~
\hyperref[sec:l1I8FQov324]{\small{> > > NEXT SERMON > > >}}
\newline
\newline
瑪拉基書 3:6-12-20230722
\newline
\begin{longtable}{cl}
\hline
\hline
章節 & 經文 (和合本修訂版)\\
\hline
3:6 & \begin{tabularx}{0.7\textwidth}{X} 「我-耶和華是不改變的;所以,雅各的子孫啊,你們不致滅亡。 \end{tabularx} \\ \\ \relax
3:7 & \begin{tabularx}{0.7\textwidth}{X} 從你們祖先的日子以來,你們就偏離我的律例而不遵守。現在你們要轉向我,我就轉向你們。這是萬軍之耶和華說的。你們卻說:『我們如何轉向呢?』 \end{tabularx} \\ \\ \relax
3:8 & \begin{tabularx}{0.7\textwidth}{X} 人豈可搶奪神呢?你們竟搶奪我!你們卻說:『我們在何事上搶奪你呢?』其實就是在你們當納的十分之一奉獻和當獻的供物上。 \end{tabularx} \\ \\ \relax
3:9 & \begin{tabularx}{0.7\textwidth}{X} 因你們全國上下都搶奪我的供物,詛咒就臨到你們身上。 \end{tabularx} \\ \\ \relax
3:10 & \begin{tabularx}{0.7\textwidth}{X} 你們要將當納的十分之一全然送入倉庫,使我家有糧,以此試試我,是否為你們敞開天上的窗戶,傾福與你們,甚至無處可容。這是萬軍之耶和華說的。 \end{tabularx} \\ \\ \relax
3:11 & \begin{tabularx}{0.7\textwidth}{X} 我必為你們斥責蝗蟲,不容牠毀壞你們的土產。你們田間的葡萄樹,果實未熟以先也不會掉落。這是萬軍之耶和華說的。 \end{tabularx} \\ \\ \relax
3:12 & \begin{tabularx}{0.7\textwidth}{X} 萬國必稱你們為有福的,因你們必成為喜樂之地。這是萬軍之耶和華說的。」 \end{tabularx} \\ \\ \relax
3:13 & \begin{tabularx}{0.7\textwidth}{X} 耶和華說:「你們用話頂撞我。」你們卻說:「我們說了甚麼話頂撞你呢?」 \end{tabularx} \\ \\ \relax
3:14 & \begin{tabularx}{0.7\textwidth}{X} 你們說:「事奉神是枉然,我們遵守神所吩咐的,在萬軍之耶和華面前哀痛而行,有甚麼益處呢? \end{tabularx} \\ \\ \relax
3:15 & \begin{tabularx}{0.7\textwidth}{X} 現在,我們稱狂傲的人為有福,並且行惡的人得以建立;他們雖然試探神,卻得以逃脫。」 \end{tabularx} \\ \\ \relax
3:16 & \begin{tabularx}{0.7\textwidth}{X} 那時,敬畏耶和華的人彼此談論,耶和華側耳而聽,且有紀念冊在他面前,記錄那敬畏耶和華、思念他名的人。 \end{tabularx} \\ \\ \relax
3:17 & \begin{tabularx}{0.7\textwidth}{X} 萬軍之耶和華說:「在我所定的日子,他們必屬我,是我寶貴的產業。我必憐憫他們,如同人憐憫那服侍他的兒子。 \end{tabularx} \\ \\ \relax
3:18 & \begin{tabularx}{0.7\textwidth}{X} 那時你們必再一次看出義人和惡人,事奉神和不事奉神的人有何差別。」 \end{tabularx} \\ \\
[1ex]
\hline
\hline
\end{longtable}
$^{1}$(麥克風聲).
我是不是隨時都可以了?.
這個Live的鏡頭.
因為以前有個弟兄姊妹在這裡秀.
5 4 3 2 1 這樣.
所以今天沒有在這裡.
所以我不知道.
好吧 我當應該OK啦.
弟兄姊妹平安.
(弟兄姊妹:平安).
我多謝敬拜隊.
首先帶領我們一起敬拜.
很進入去崇拜的裡面.
我也多謝文宣組.
我覺得這張圖很可愛.
所以我們今天是講馬拉基書.
這個奉獻的閱題.
我覺得是一個好的機會給我.
也給大家一起直接認真地去看看.
馬拉基書這段講十一奉獻的經文.
本身一開始提到馬拉基書的時候.
最深的印象都是.
錢.
但是當我整本書這樣看的時候.
我發現原來是講約定.
當時上帝和以色列人原來看彼此都很不順眼.
大家想不想像到.
三章八節.
我們下一個PowerPoint.
上帝說以色列人搶他的東西.
我真的嚇到.
哪有人可以搶到上帝的東西.
但令我更加嚇到的是.
以色列人的反應.
再下一章PowerPoint.
他們說上帝我哪裡有偷你的東西.
嚇到.
當上帝知道以色列人不承認他搶東西的時候.
再下一章第九節.
他直接說.

$^{41}$你們以色列人舉國上下都搶我的東西.
還要一直搶.
到現在都沒有停過.
所以我要就坐你們.
哇.
我看《馬拉基斯》的時候.
我的心跳得很厲害.
以色列人和上帝之間.
那種劍拔弩張的關係和對話.
很嚇人.
怎麼敢.
我想問以色列人為何絲毫不怕上帝.
不停地回擊上帝.
怎麼敢.
但弟兄姊妹我們看看PowerPoint.
十一奉獻.
我們談到天上福氣的經文.
原來正正就是索羅在上帝說要就在以色列人之後.
究竟是怎麼回事.
在劍拔弩張的關係和對話之中.
《馬拉基斯》說奉獻.
其實是說什麼.
我們今天一起去看看這件事情.
首先我們去了解一下.
以色列人和上帝之間.
到底發生了什麼事.
其中一件事是.
他們覺得耶和華食言.
《馬拉基斯》記載的那個時候.
是被迷路的以色列人重新回到以色列地.
而且已經重建好聖殿.
好像我們在第一章看到.
以色列人已經可以在聖殿獻祭.
又有祭司在聖殿裡面工作.
好像一切都回歸正軌一樣.
但原來其實不是.
當時的以色列人有很多農作物都失收.
而且有很多收割的季節,季度.
農作物的產量都很少.
而在這樣的情況下.

$^{81}$就讓這些以色列人覺得.
耶和華上帝食言.
祂不守約定.
大家記不記得.
他們曾經有過什麼約定.
我們簡單說一下.
如果大家記得《啟威傳道》.
曾經說過黑蓋書.
黑蓋書裡面記載.
從前耶和華透過先知黑蓋和以色列人說.
去解釋為什麼當時的以色列地會出現乾旱.
是因為這是上帝對他們的懲罰.
因為他們對上帝的聖殿冷散.
使殿荒涼.
但自己卻住在有天花板的房屋.
大家記不記得.
《啟威傳道》說過這篇.
經文是這樣說的.
黑蓋書一章九至十一節.
這是萬君子耶和華說的.
因你們的緣故天不降甘露.
地也不出土產.
我命令乾旱臨到土地.
山崗五谷伸走.
新柔和地上的出產.
也臨到人和生畜.
以及一切人手牢籠得來的.
之後經文記載.
索羅巴巴,約書亞大祭司.
和倖存的百姓都聽了.
都聽耶和華的說話.
百姓在耶和華面前傳敬畏的心.
耶和華又激發他們的心.
他們一起興建聖殿.
而當時.
我們下一章的關鍵字.
是耶和華上帝也應許了一件事.
祂說:我必震動萬國.
萬國的珍寶都必運來.
我要使這個聖殿充滿榮耀.

$^{121}$這是萬君子耶和華說的.
他說:銀紙是我的.
金紙也是我的.
這是萬君子耶和華說的.
後來的殿的榮耀.
一定大過先前的榮耀.
這是萬君子耶華說的.
在這個地方我必賜平安.
這是萬君子耶華說的.
不單是哈該書.
撒加利亞書也是這樣說的.
撒加利亞書第八章.
也同樣記載著相同的應許.
但問題在於哪裡呢?.
就是以色列人.
他們有聽上帝的話.
他們有重建聖殿.
但聖殿興建好了.
以色列的出產仍然很少.
地依然乾旱.
聖殿也沒有比以前.
所羅門時期的輝煌.
他們依然很窮.
上帝說好的天降金路.
土地必有出產.
平安撒種葡萄樹的結果.
他們面對的是.
他們曾經相信上帝的約定.
但現實的結果是.
上帝曾經答應的事.
他沒有兌現.
我舉個例子.
我老公會陪我去英國.
去參加一個弟兄的婚禮.
我又會陪他去奧脫福的球場看球.
他第一次去主場.
他很努力地安排買票.
因為他不想買黃牛票.
所以我甚至被入會.
我入了萬聯的球會.

$^{161}$你們可否想像.
我都沒有看過多少場球賽.
但我入了萬聯的球會.
而且還要等.
要看是否可以買到票.
他每天都很期待.
有時聊天.
昨晚我寫講章的時候.
他都會播萬聯的歌.
怎麼唱.
Gory Gory Man United.
大家有沒有聽過這首歌.
可以想像.
如果.
如果我們在網上.
收到這張網上的入場券.
但最後去到奧脫福球場門口.
我一殺.
職員跟你說.
跟我老公說.
你這張票是不對的.
如果.
如果是這樣.
我老公應該會對萬聯.
大大的失望.
答應的事沒有兌現.
很可能他會因此脫萬.
他會脫離萬聯.
但這一切.
我學徐佳燕牧師說.
都是假的.
不是真的.
但.
大家有沒有試過.
就是以色列人.
他們所想的.
大家有沒有試過.
他們真的覺得上帝.
失敗了他們.
我們很多時候都會反省.

$^{201}$我們犯罪了.
我們又讓上帝失望了.
但聖經真的.
很敢寫.
他寫是人對上帝.
失望了.
我們有沒有試過.
上兩個禮拜.
John 說奉獻.
是源於關係.
但關係其實是.
很複雜的東西.
好像任何一段.
男女關係或親人關係.
如果我們真的.
進入了一段關係.
我們都會隨時間.
經歷會有信任.
失望.
我們會吵架.
我們會言歸於好.
又或者我們分手收場.
我們和上帝.
可能都是這樣.
我這幾個星期.
一直看著這本馬拉基書.
我浮在腦海的字是.
約定.
是上帝和人之間彼此的約定.
以色列人覺得.
上帝沒有保留.
他對自己的約定.
既然是這樣的話.
以色列人這樣想上帝的話.
他們會怎樣做.
分手?.
如何處理?.
馬拉基書寫了.
他們明目張膽地違反.
解明.

$^{241}$第一章的經文形容.
以色列人.
去藐視憲制.
他們憲制的時候.
用有病的,有病的,有病的.
搶回來的去獻上.
而聖殿的祭司.
睜開眼睛照燒在壇上.
第二章的經文形容.
祭司本來應該讓人.
在律法正道上.
但他們偏偏讓很多人.
在律法上跌倒.
第三章的經文形容.
以色列人欺壓貧困人.
欺壓孤兒寡婦.
大家想不想像到這是一個怎樣的畫面.
而今天的經文.
三章六至十二節.
是其中一段.
上帝對以色列人的回應.
是上帝親自去回應.
我們看的時候.
我希望我們戴著一個.
眼鏡,一個角度.
一個生命記和約的角度.
去看以下的這段經文.
因為上帝在這裡.
也跟以色列人說約定.
我們加油.
我們今晚會多看一點生命記的經文.
我們一起進入三章六節.
上帝說.
因我耶和華是不改變的.
所以你們雅各之子沒有滅亡.
如果說約定.
上帝說.
從列祖之約開始.
祂和以色列人之間.
由祂是阿巴拉罕的神.

$^{281}$以撒的神,雅各的神開始.
承諾他們的後裔.
就像海邊的沙和天上的星一樣.
以色列人.
因為是雅各之子.
上帝說我一直守著約定.
他們才不至滅亡.
我們看生命記第九章二十七節.
摩西是對那些即將要進入英雄之地的.
第二代以色列人說.
你不要忘記.
當初摩西請求上帝.
求你紀念你的僕人.
阿巴拉罕,以撒,雅各.
不看這些百姓的.
還耿,邪惡和罪言.
上帝說.
回到今天的第七節.
從你們列祖的日子以來.
你們常常都偏離我的典章.
而不遵守.
一次又一次不守約定的是哪一個呢?.
他說以色列人仍然生存.
被保留下來.
本身就是因為上帝守約的緣故.
上帝說.
我有很好的紀錄.
你們這樣.
我也會繼續守約.
到這裡.
我想大家都明白.
我們再下去第七節的下半節.
我們會發覺上帝就開始說一些很突兀的東西.
上帝說.
現在你們要轉移.
你們的信仰.
上帝說.
現在你們要轉向我,我就轉向你們.
大家知道這是什麼嗎?.
這是什麼?.

$^{321}$這是一個約.
約是雙方的.
有你有我.
兩個人才可以.
所以上帝說你們要轉向我.
我就轉向你們.
但這裡很奇怪.
面對以色列人的控訴.
以色列人說他們言而無信.
破壞約定.
但上帝就說你們要回轉向我.
以色列人.
他們真的不覺得自己離開了上帝.
可能他們覺得.
我們已經建好了聖殿.
我們已經很聽你們和先知的說話.
所以第七節最後.
以色列人說.
我怎樣才轉向?.
他的意思是.
我都沒有離開過.
我怎樣才能回轉?.
但第八節.
上帝說了一件事.
完全顛覆了他們的認知.
以色列人以為.
自己站在道理的一邊.
甚至是公義的一邊.
但上帝竟然.
去責備他們.
搶他們的東西.
這一刻.
真是始料不及.
連以色列人都問.
我在哪一件事情上.
有搶你的東西?.
大家明白嗎?.
為何上帝說了這一句?.
我不明白.
我簡直不明白.

$^{361}$人怎樣可以搶到上帝的東西?.
人可以搶到上帝的什麼?.
就算經文第八節下.
他說.
是十一奉獻.
是當獻的供物.
我都會問.
為何是這些?.
我昨天開小組的時候.
我自覺想了一個很好的舉例.
人可以搶到上帝的東西.
起碼那一樣東西.
是我們和上帝一起.
分享的東西.
人是在這件事裡.
其中一個搶劫者.
而唯一.
有一樣東西.
我們和上帝分享的.
就是一個藥.
就是生命當中.
上帝和人彼此的約定.
耶和華上帝成為.
以色列人的上帝.
以色列人成為.
耶和華上帝的子民.
而在這個約的底下.
以色列人是要.
帶十一奉獻和當獻的供物.
去到上帝面前.
我們看一段.
《生命記》第十二章十至十一節的經文.
他這樣說.
你們過了這一淡河.
住在耶和華.
神給你們承受危險的地.
他又使你們得享太平.
不受四圍一切.
受敵騷擾亂.
使你們安然居住.

$^{401}$那時你們要將我所吩咐你們的凡.
祭祀物十一奉獻.
手中的舉祭和向耶和華許願.
的一切上好的祭.
都帶到耶和華.
你們神所選擇立他名的居所.
是在這裡.
他在說的.
我們可能不是很熟.
《生命記》的十一奉獻.
是些什麼呢?.
我跟大家約略說說.
他有四個特色.
《生命記》.
我們下一張powerpoint.
《生命記》十一奉獻和當獻的供物.
有四個特色.
第一,原來他不是我們想獻就獻.
是以色列人過了約旦河.
進入了上帝.
所賜給他的地之後才可以.
第二.
進入了上帝所賜給的地之後.
還要在上帝.
所選擇立他名的地方.
才可以獻.
不可以在有其他偶像的地方獻.
第三.
大家最意想不到的是.
這個奉獻.
其實是一個慶典.
是一個慶祝.
我們看看下一張powerpoint.
第十四章二十四至二十六節.
《生命記》.
耶和華你神選擇立他名的地方.
若離你太遠.
路途太長.
令你不能將我一堆的東西.
帶過去.

$^{441}$你可以換錢.
你可以把他換成銀紙.
然後將錢包住.
拿在手上.
去耶和華你神所選擇的地方.
在那裡.
你可以隨心所欲地用錢.
你可以買牛買羊.
買清酒買烈酒.
買任何心裡所想的東西.
他說你和你全家.
都要在耶和華你神面前.
吃喝歡樂.
是不是想像不到呢.
第四.
你美人成女的寄居者.
和孤兒寡婦.
他說都要和你一起.
去慶祝.
這個十一奉獻原來是一個社會責任.
這個就是《生命記》裡面.
所說的我四個的向導.
我總結一下這個奉獻.
我們看下一章.
十四章二十三節.
《生命記》.
究竟這個奉獻是來做什麼的.
他說要在耶和華你神面前.
就是他選擇那裡.
作為他名居所的地方.
吃你所獻的.
十分之一的五穀.
新酒新的油和牛群羊群中投生的.
好讓你天天學習敬畏耶和華你神.
所以原來這個奉獻.
我們可能期待了很久.
我們所說的十一奉獻.
原來是一個敬拜上帝的慶典.
目的是讓我們天天去學習敬畏耶和華他們的神.
所以說到這裡.

$^{481}$十一奉獻當時的共物.
是上帝和人.
兩方互相.
定下的約.
的部分.
不是說天地萬物裡面任何一隻牛羊都屬於上帝.
是的都是的萬物都屬於主.
但在這裡說的是.
是由他的子民作為敬拜.
作為敬畏所獻給上帝的牛羊.
而這件事是屬於上帝的.
因為這是他們彼此的約定.
所以以色列人才有資格.
被上帝稱他們.
說他們搶上帝的東西.
因為上帝和以色列人之間.
就是有這個約定.
其實說到這裡.
我自己覺得是挺浪漫的事.
我不知道大家是否同意.
不知道大家聽到這裡會否覺得.
你們明白嗎.
他說的就是.
上帝說以色列人搶了.
本來應該他們在約裡.
答應獻給我的.
十分之一和共物.
如果我們記得的話.
馬拉基書第一章.
就形容以色列人在獻祭上.
正正就是獻上殘疾.
這裡反映的是敬畏的相反.
是藐視上帝.
而不單止以色列人獻祭.
連聖殿的祭司都接受了.
並且放在壇上燒了.
上帝說這是甚麼.
這是靈磚.
這是靈磚中.
你明不明.

$^{521}$他們覺得燒了也是這樣.
你也是這樣聞.
不是這樣.
所以今天的經文第九節.
上帝甚至說.
因為你們通國的人.
都奪取了我的共物.
周遭淋到你身上.
這是生命記說的.
上帝直接回應了以色列的指責.
他說現在你們土地貧瘠.
乾旱經濟差.
與我無關.
不是因為我失信.
反而是你們舉國上上下下的人.
都背棄了約定.
所以我就要像我的約一樣.
來就坐你們.
第九節就是這樣.
我們停一停.
我們看到這裡.
我們先回到起點.
現在局面是甚麼.
現在局面是.
以色列人控訴上帝.
說你失敗了.
你背棄了約定.
然後三章六節九節.
我們剛剛看了很久.
上帝說不是的.
是以色列人你們違背了約定.
你們違背了約定.
現在就是這樣的局面.
其實弟兄姊妹.
我寫到這裡的時候.
我在想如果這一場是激烈的.
男女的吵架.
上帝作為上帝.
他是有絕對的勝算的.
因為他一個審判一個滅絕.

$^{561}$以色列人就玩完了.
為甚麼這樣說呢.
因為整個第三章.
如果大家有機會看的話.
如果我們從早舉止.
將第九節轉到第十三節.
是毫無違和感的.
但是上帝就是這麼特別.
他很特別.
他加了一句東西.
他加了第十節.
萬君子耶和華說.
你們要將當立的十分之一傳言.
送上倉庫.
使我皆有良以此示示我.
是否為你們打開天上的窗戶.
傾福於你們.
甚至無處可容.
第十二節他說.
你們會有豐盛的田產.
萬國都稱為有福.
地會成為喜樂之地.
這些全部都是生命記的東西.
我不多說了.
我直接一點說.
我們一直以為.
這是個千載難逢的機會.
我們人竟然可以試上帝.
但其實原來.
這不是單向的.
去試上帝.
這是一個約.
這是一個出處.
都是生命記的.
是生命記所記載的約的一部分.
我們看第28章9-12節.
他說.
如果你謹修耶和華的神力誡命.
遵行他的道.
他必照他向你所起的誓.

$^{601}$立你為他自己神聖的子民.
他說地上的萬民.
見你歸在耶和華的名下.
就必懼怕你.
然後他還說.
在耶和華向你列祖起誓.
他必使你身所生的.
生畜所生的.
地所殘的都豐富有餘.
他說耶和華必為你.
躺開天上的寶庫.
安時降雨在你的地上.
他又賜福你手中所做的一切.
你不需要向其他國家借錢.
是其他國家向你借錢.
是說到這樣.
但卿姐我們再仔細一點聽.
第10節說的是.
其實是一個生命記所描寫的景象.
我特意用不同顏色.
雖然有些醜.
但分得好一點.
上帝在這裡重提.
是生命記所描寫的景象.
他其實是很想.
和以色列人重建約定.
這裡所說的十一奉獻.
其實是一個.
重建約定的引子.
是一個雙方.
雙方共同的約定.
我們就會知道.
第10節的「試試我」.
不單是試上帝.
上帝是.
叫我們先試自己.
是以色列人.
先走了第一步.
先將當立的十一奉獻送入倉庫.
先試上帝的家有糧.

$^{641}$然後才以此試試我.
而這個試.
是上帝為了解開以色列人的心結.
上帝提供了一個機會.
讓以色列人去試試.
究竟是上帝背棄他的約定.
還是人背棄上帝的約定.
是這件事.
有時我們.
未必真的知道自己是怎樣的.
但我們真的覺不覺.
我試一下講一個例子.
是我自己.
很久之前.
在大學畢業的時候.
我對上帝.
非常有信心.
我覺得上帝一定會帶領我前面的路.
上帝在哪裡.
我就跟著去哪裡.
當時我都不找.
「六」的工.
我想傳道人都很愛我.
我為了可以回教會.
我很堅持.
但我找了幾個月工都沒有.
我都沒有氣餒.
我仍然在教會裡面.
搞營.
青少年營會.
去防割石.
那些廣西斷宣.
在斷宣裡面我收到一些interview訊息.
我還覺得.
是逢上帝.
就是這樣.
這份工很可能是為我預備.
我當時就是這樣想.
其實我都沒有in上這份工.
但我覺得很OK.

$^{681}$我繼續耐心等候.
我仍然感恩有份part time工作.
可以先做.
但日子一來.
半年,七個月,八個月.
每個月都在遞增的時候.
我就開始對自己的能力.
各樣的東西都自我質疑.
直到有一天.
傳道人跟我談工作的事.
他問了我一句.
他說上帝要你在哪裡.
你就去哪裡.
如果上帝要你做一個receptionist.
你又會怎樣.
當刻我才發現.
我不願意.
不是receptionist不好.
很久之後.
我都很奇妙地,很開心地.
去做過一陣子.
但這個是後話.
我當時醒覺.
原來我一直以為.
我對神很有信心.
我願意跟著上帝走.
其實都不是真的.
真的這樣.
一個receptionist.
就已經考起了我的信仰.
最後我哭著.
跟上帝說.
我說.
是呀.
上帝我發現.
我沒有我自己所說的.
我以為的那麼跟你.
無論.
我們對上帝.
是怎樣的認知.

$^{721}$上帝究竟是怎樣的.
還是對自己的認知.
突然間我發現.
人其實很可能.
不是真的自己以為.
那麼認識.
我都以為自己對上帝很有信心.
我很願意跟隨.
但我經過幾個月.
差不多一年時間.
我一個receptionist的挑戰.
我才發現.
自己對上帝是怎樣.
所以有時候我們.
未必真的知道.
自己是怎樣.
但我想大家都明白.
當一個人很生氣的時候.
對你很失望的時候.
無論你再回應什麼.
對方都未必聽得入耳.
我都是很人性地.
理解這段經文.
上帝說多少.
我沒有背若.
反而是你們背若.
這些說話很大機會.
都沒有作為.
我們一起走一步.
唯有你試過.
你才會相信.
這是第十節.
我想說.
我看到這裡的時候.
我覺得上帝已經很努力.
為了我們的約定.
你願意去體諒了我一份.
沒有信任的出現.
去讓意識的人.
可以有機會.

$^{761}$藉著試下自己的迴轉.
去證實耶和華上帝.
是怎樣堅守他的約定.
對於意識的人.
上帝沒有說.
你不信就算了.
你不信就小信.
三章六至九節.
他首先告訴你真相是什麼.
但你未必聽得入耳.
不要緊.
十章十二節上帝說.
我讓你去試下.
我們一起用行動.
互相證明給對方看.
為的是要讓這段破碎的信任關係.
無論是基於錯誤的認知.
還是我們被困難壓到很重.
都可以因為上帝的這一步.
而去發現.
上帝一直都是當初.
跟我們納約的上帝.
經文說到這裡.
其實我一直在找.
我到昨天都一直在想.
上帝透過這段經文.
究竟想我講些什麼.
而我漸漸覺得.
在這段吵架的經文裡.
我們每個人.
可能都有自己的投射.
可能是.
我都曾經對上帝失望.
但我不敢說.
這些可以說的嗎?.
我都曾經因為拍拖的事.
教會的事.
人性的事.
社會,工作,病患.
死亡.

$^{801}$而對上帝失望.
或者貌似.
我不是很明白上帝你.
不是很有信心繼續.
又或者.
不是很清楚如何.
再去跟上帝.
在這個情況下.
以色列人都照樣獻祭.
只是他獻的是殘缺的祭.
我們都照樣回來.
教會崇拜.
我們獻上的.
會不會都是部分殘缺的信仰呢?.
而讓我.
去到這個位置.
你問我.
我都是無解的.
可以怎麼做呢?.
但只有上帝.
只有上帝.
才可以在這個無解的裡面.
給我十節的一句.
要當立的十分之一.
傳言送入倉庫.
以此示我.
我真的不知道.
屬於你.
每一個人你和上帝之間的.
那個約.
你所當立的十分之一和公物是甚麼.
這是我們需要.
每個人負上約的責任.
去找和獻上.
但對我來說.
我真的.
很想把最好的獻上.
是我每一次的講道.
每一次.
很努力去明白上帝的真理.

$^{841}$然後我們整個信仰群體.
我們一起在真理上.
堅持和努力.
然後未信的人.
可以因為上帝的真理.
而回到上帝的裡面.
我是這樣.
而去到最後.
如果我們要去到一個標準.
怎樣也要有東西.
拿著的話.
就是最後一個關鍵.
有甚麼東西可以幫我們找回.
屬於自己當立的奉獻.
對於馬拉基書來說.
我想這個一定是.
回到上帝的約裡面.
我們為了去敬拜上帝的慶典.
而帶去的一切.
我們為了.
大家記不記得.
生命記了四個向導.
我們一起去敬拜上帝的慶典.
我們去這個慶典的時候.
我們所帶去的一切是甚麼.
而這個奉獻.
是讓你天天學習去敬畏耶和華.
祂們的神.
我們的神.
而這個奉獻是我們不要忘記.
上帝刻意提倡的.
一個社會責任.
如果我們生命.
是以這件事為約的話.
那我們會奉獻甚麼呢.
我們一起祈禱.
即是鐵火求你.
讓你自己的華語.
親自進入每一個弟兄姊妹的心裡面.
如果我們裡面.

$^{881}$其實是.
有對你失望的弟兄姊妹.
有對你有很多不明白的弟兄姊妹.
最後求你.
親自去到.
捉住這一份的關係.
鐵火我們究竟.
在你裡面.
我們的約.
我們怎樣可以.
不斷地守住呢.
有時候我們都很怕.
我以為我自己.
已經在約的裡面.
但我們會不會是.
原來我們是在藐視上帝.
我們會不會像.
以色列人那樣想.
那樣以為我們哪有離棄上帝.
我們哪有離棄上帝.
主啊求你教導我們.
主啊你的亮光.
親自來照亮我們.
主啊求你捉緊我們.
奉主耶穌基督得勝的名字祈求.
阿們.
謝謝大家.
\newpage



\section{路加福音 14:1-35-20230729}
\label{sec:nQfdSDvE_CU}
\textbf{【網上崇拜】計唔掂|路加福音14\_1-35|20230729 [nQfdSDvE-CU]}
\newline
\newline
連結: \href{https://youtube.com/watch?v=nQfdSDvE-CU}{\texttt{ https://youtube.com/watch?v=nQfdSDvE-CU}} ~~~~ 語音日期: 2023-07-29 
\newline
\newline
\hyperref[sec:l1I8FQov324]{\small{< < < PREV SERMON < < <}}
~
\hyperref[sec:index_chronic]{\small{[返順時目]}}
~
\hyperref[sec:index_scriptual]{\small{[返順卷目]}}
~
\hyperref[sec:IJY_UvLqQqw]{\small{> > > NEXT SERMON > > >}}
\newline
\newline
路加福音 14:1-35-20230729
\newline
\begin{longtable}{cl}
\hline
\hline
章節 & 經文 (和合本修訂版)\\
\hline
14:1 & \begin{tabularx}{0.7\textwidth}{X} 安息日,耶穌到一個法利賽人的領袖家裡去吃飯,他們就窺探他。 \end{tabularx} \\ \\ \relax
14:2 & \begin{tabularx}{0.7\textwidth}{X} 這時在他面前有一個患水腫病的人。 \end{tabularx} \\ \\ \relax
14:3 & \begin{tabularx}{0.7\textwidth}{X} 耶穌回答律法師和法利賽人,說:「安息日治病合不合法?」 \end{tabularx} \\ \\ \relax
14:4 & \begin{tabularx}{0.7\textwidth}{X} 他們卻不說話。耶穌扶著那人,治好了他,叫他走了。 \end{tabularx} \\ \\ \relax
14:5 & \begin{tabularx}{0.7\textwidth}{X} 耶穌對他們說:「你們中間誰有兒子或有牛在安息日掉在井裡,不立刻拉他上來呢?」 \end{tabularx} \\ \\ \relax
14:6 & \begin{tabularx}{0.7\textwidth}{X} 他們對這些事不能反駁。 \end{tabularx} \\ \\ \relax
14:7 & \begin{tabularx}{0.7\textwidth}{X} 耶穌見所請的客人選擇首位,就用比喻對他們說: \end{tabularx} \\ \\ \relax
14:8 & \begin{tabularx}{0.7\textwidth}{X} 「你被人請去赴婚宴,不要坐在首位上,恐怕主人請了比你尊貴的客人, \end{tabularx} \\ \\ \relax
14:9 & \begin{tabularx}{0.7\textwidth}{X} 請了你和他的那人前來,對你說:『請讓座給這一位吧。』你就羞羞慚慚地退到末位去了。 \end{tabularx} \\ \\ \relax
14:10 & \begin{tabularx}{0.7\textwidth}{X} 你被請的時候,去坐在末位上,好讓主人來對你說:『朋友,請上座。』那時,你在同席的人面前就有光彩了。 \end{tabularx} \\ \\ \relax
14:11 & \begin{tabularx}{0.7\textwidth}{X} 因為凡自高的,必降為卑;自甘卑微的,必升為高。」 \end{tabularx} \\ \\ \relax
14:12 & \begin{tabularx}{0.7\textwidth}{X} 耶穌又對請他的人說:「你準備午飯或晚餐,不要請你的朋友、弟兄、親屬和富足的鄰舍,免得他們回請你,你就得了報答。 \end{tabularx} \\ \\ \relax
14:13 & \begin{tabularx}{0.7\textwidth}{X} 你擺設宴席,倒要請那貧窮的、殘疾的、瘸腿的、失明的, \end{tabularx} \\ \\ \relax
14:14 & \begin{tabularx}{0.7\textwidth}{X} 你就有福了!因為他們沒有甚麼可報答你。到義人復活的時候,你要得到報答。」 \end{tabularx} \\ \\ \relax
14:15 & \begin{tabularx}{0.7\textwidth}{X} 同席的有一人聽見這些話,就對耶穌說:「在神國裡吃飯的有福了!」 \end{tabularx} \\ \\ \relax
14:16 & \begin{tabularx}{0.7\textwidth}{X} 耶穌對他說:「有人擺設大宴席,請了許多客人。 \end{tabularx} \\ \\ \relax
14:17 & \begin{tabularx}{0.7\textwidth}{X} 到了坐席的時候,他打發僕人去對所請的人說:『請來吧!樣樣都已齊備了。』 \end{tabularx} \\ \\ \relax
14:18 & \begin{tabularx}{0.7\textwidth}{X} 眾人異口同聲地推辭。頭一個對他說:『我買了一塊地,必須去看看。請你准我辭了。』 \end{tabularx} \\ \\ \relax
14:19 & \begin{tabularx}{0.7\textwidth}{X} 另一個說:『我買了五對牛,要去試一試。請你准我辭了。』 \end{tabularx} \\ \\ \relax
14:20 & \begin{tabularx}{0.7\textwidth}{X} 又有一個說:『我才娶了妻子,所以不能去。』 \end{tabularx} \\ \\ \relax
14:21 & \begin{tabularx}{0.7\textwidth}{X} 那僕人回來,把這些事都告訴了主人。這家的主人就發怒,對僕人說:『快出去,到城裡大街小巷,領那貧窮的、殘疾的、失明的、瘸腿的來。』 \end{tabularx} \\ \\ \relax
14:22 & \begin{tabularx}{0.7\textwidth}{X} 僕人說:『主啊,你所吩咐的已經辦了,還有空位。』 \end{tabularx} \\ \\ \relax
14:23 & \begin{tabularx}{0.7\textwidth}{X} 主人對僕人說:『你出去,到大街小巷強拉人進來,坐滿我的屋子。 \end{tabularx} \\ \\ \relax
14:24 & \begin{tabularx}{0.7\textwidth}{X} 我告訴你們,先前所請的人沒有一個可以嘗到我的宴席。』」 \end{tabularx} \\ \\ \relax
14:25 & \begin{tabularx}{0.7\textwidth}{X} 有一大群人和耶穌同行。他轉過來對他們說: \end{tabularx} \\ \\ \relax
14:26 & \begin{tabularx}{0.7\textwidth}{X} 「無論甚麼人到我這裡來,若不愛我勝過愛自己的父母、妻子、兒女、兄弟、姊妹,甚至自己的性命,就不能作我的門徒。 \end{tabularx} \\ \\ \relax
14:27 & \begin{tabularx}{0.7\textwidth}{X} 凡不背著自己的十字架來跟從我的,也不能作我的門徒。 \end{tabularx} \\ \\ \relax
14:28 & \begin{tabularx}{0.7\textwidth}{X} 你們哪一個要蓋一座樓,不先坐下來計算費用,看能不能蓋成? \end{tabularx} \\ \\ \relax
14:29 & \begin{tabularx}{0.7\textwidth}{X} 免得安了地基,不能蓋成,看見的人都笑話他,說: \end{tabularx} \\ \\ \relax
14:30 & \begin{tabularx}{0.7\textwidth}{X} 『這個人開了工,卻不能完工。』 \end{tabularx} \\ \\ \relax
14:31 & \begin{tabularx}{0.7\textwidth}{X} 或是一個王出去和別的王打仗,豈不先坐下來酌量,他能不能用一萬兵去抵抗那領二萬兵來攻打他的嗎? \end{tabularx} \\ \\ \relax
14:32 & \begin{tabularx}{0.7\textwidth}{X} 若是不能,他就趁敵人還遠的時候,派使者去談和平的條件。 \end{tabularx} \\ \\ \relax
14:33 & \begin{tabularx}{0.7\textwidth}{X} 這樣,你們無論甚麼人,若不撇下一切所有的,就不能作我的門徒。」 \end{tabularx} \\ \\ \relax
14:34 & \begin{tabularx}{0.7\textwidth}{X} 「鹽本是好的;鹽若失了味,怎能叫它再鹹呢? \end{tabularx} \\ \\ \relax
14:35 & \begin{tabularx}{0.7\textwidth}{X} 或用在田裡,或堆在糞裡,都不合適,只好丟在外面。有耳可聽的,就應當聽!」 \end{tabularx} \\ \\
[1ex]
\hline
\hline
\end{longtable}
$^{1}$大兄姐妹平安.
今天是奉獻月題的第五講.
奉獻的道.
我都好像John一樣是第一次講.
所以當我收到這個月題的時候.
都想了很久.
想了很久的原因有幾個.
一來就是奉獻的經文好像很少.
我們一定會遇到.
所以你又要想一下.
究竟你這個星期又會講什麼.
不會跟別人遇到.
還有就是John在第二個星期的時候.
他就解話.
他聽那些行家講有幾個套路.
我在後面聽的時候就在想.
糟了.
他都差不多講完了套路.
那我還可以講什麼呢.
所以想了很久.
最後就想了這段經文.
想跟大家一起.
從不同的角度.
去思考一下奉獻這個課題.
經文就選至路加福音.
十四章一至三十五節.
先不下載.
這段經文就很長.
我今天都不會逐節跟大家讀.
如果大家有手機.
都建議大家可以拿出來.
因為字可能有點小.
為什麼會選這段經文呢.
因為這段經文.
都鼓勵了我很多.
所以今天雖然這段經文很長.
但是我們就嘗試從兩個問題.
去想一想這段經文講什麼.
第一個問題就是七至十一節.
七至十一節.

$^{41}$耶穌就說看到賓客.
爭坐首位.
於是他就講了一個比喻.
叫人們坐末位.
等家主來邀請你才坐首位.
你想像一下整個場面.
就是大家蜂擁爭那幾個位.
好像爭椅子一樣.
十幾個人這樣爭.
但是耶穌偏偏在十五到二十四節.
就講了一個筵席的比喻.
那些人是不去筵席的.
有人請吃飯.
有班人不去.
其實是一件很奇怪的事.
你想像一下如果今天.
我說要請吃飯.
我相信我的同工應該會馬上來.
或者你的朋友請你吃飯.
可能你都會放下你的東西.
去join那餐飯.
除非你很討厭他.
請吃飯是一種魔力.
我們是很喜歡吃一些不用付錢的飯.
所以我們要想一下.
為什麼耶穌在這個比喻裡面.
會說那些人不去筵席.
第二個問題就是.
經文一直講下去.
去到三十三節.
他就說你們無論什麼人.
若不撇下一切所有.
就不能作我的門徒.
究竟耶穌在說什麼呢.
如果我們按字面解釋.
撇下所有.
可能就是他在說的父母.
妻子.
兄弟姊妹兒女.
甚至是你自己的性命.

$^{81}$你的十字架.
如果是這樣的話.
我相信在座或者不要說在座.
說我吧.
我覺得我都沒有做到這個標準.
我都不是耶穌的門徒.
所以我們就要搞清楚.
究竟耶穌在說什麼.
在講經文之前.
我們首先講一些背景.
在福音書裡面.
其實有很多筵席的經文.
他們經常都吃飯.
吃飯宴請.
在那個年代是很常見的.
在那個年代.
有一種制度叫做.
因主制度.
不知道大家有沒有聽過.
有因主和受恩人的角色.
因主就是一些社會地位比較高.
或者比較有錢的人.
他就透過私恩.
去巴結一些受恩人的狀態.
最常見就是請吃飯.
所以一個因主.
例如我這樣.
我就大排筵席.
於是大家來吃飯.
就不只是吃一餐免費的飯.
你來吃飯就成為了受恩人.
我們這個關係就成立了.
於是我作為因主.
我就要對你慷慨.
可能在金錢上要幫助你.
或者可能你跌入了法律的問題.
我就要有些朋友.
可以走走後門.
或者認識一些法官.
就看看你的因主有多厲害.

$^{121}$我就會這樣去幫助受恩人.
而受恩人的責任.
就是要榮耀我.
他們就要到處跟別人說.
我有多厲害.
有多厲害.
我有多慷慨.
其實這件事在那個年代.
是一個常識.
你去吃飯不只是吃飯.
你去吃飯其實就是要.
跟這個因主締結這個關係.
然後你就在這個.
被因主去幫助的過程裡.
嘗試去提升你的社會地位.
如果我們有這個背景.
我們就可能稍為明白.
為什麼耶穌在這個比喻裡面.
會說那些人不去這個筵席.
很可能是因為這些人.
他們計算過.
覺得去這個筵席.
要給的東西.
是比起去這個筵席.
得到的東西為多.
簡單來說.
就是一盤蝕本生意.
你巴結這個因主.
你最後他會蝕的.
所以他們就作了一些原因.
去拒絕這個家主的邀請.
這個是比喻耶穌說的.
但耶穌也不是隨便說的.
這些原因.
我們可以在.
《生命記》20章裡找到.
《生命記》20章說的那些原因.
其實是一些以色列人.
拿來不去打仗的原因.
我們如果回到《生命記》那個年代.

$^{161}$他們有很多外敵.
就是經常都有人欺負他們.
所以他們經常都會有一些.
民族的叫聲.
就一起去打仗.
基本上是強制性的.
不能不去的.
所有藍丁都要去.
但律法卻是有一些.
exception的例子.
就是這幾個.
所以這一堆的例子.
或者這一堆的原因.
其實是拿來拒絕打仗的.
而耶穌擺在這班人身上.
其實就是在說.
這個賢直.
在悟悟文化底下.
就是一個因主和受恩人.
這樣的一個賢直.
這個賢直是一個天國的賢直.
是有一些不同的.
是有一些代價是需要付的.
所以當那班人.
他們計算完.
發現不太划算.
他們就不去賢直.
如果我們從計算的角度.
去理解整段經文說什麼.
我們就會發現.
其實法利賽人.
在整頓飯裡面.
都是不斷在計算一些東西.
我們下一頁一字六字.
一字六字就說.
耶穌一開始就被.
法利賽人的首領邀請.
去吃飯.
接著就有一個水鼓病人出現.
然後耶穌就問.

$^{201}$在安息日醫病可不可以呢.
法利賽人不出聲.
耶穌就醫人.
醫完人又再問他們.
他們又不出聲.
一字六字的法利賽人.
是完全沉默的.
是沒有說過任何東西.
經文沒有說過他們要.
抓耶穌的把柄去告他.
安息日醫病.
其實他們的沉默正正是反映.
法利賽人在盤算一些東西.
在計算什麼呢.
你想想.
這個水鼓病人.
是陌生人.
第一是陌生人.
第二就是.
這個病其實不是這麼緊急.
不是一些心臟病.
或者是一些會馬上死的病.
其實這個人明天來.
都是可以得到醫治的.
法利賽人在計算什麼呢.
法利賽人在計算.
究竟我為這個人.
去冒著.
肝犯安息日的風險.
有沒有好處呢.
所以當耶穌問他們.
安息日醫病.
可不可以的時候.
他們沉默.
因為他們覺得不好處.
為了一個陌生人.
去冒這個肝犯安息日的風險.
因為他們又不可以否定耶穌.
因為耶穌問的問題都很合理.
所以大家就選擇沉默.

$^{241}$所以耶穌醫治完那個人之後.
他就問了他們一個問題.
你們中間有誰的兒子.
或者你們的牛在安息日掉到井裡.
你們不會把牠拉上來呢.
意思就是.
這個就是陌生人.
就不划算.
但如果當那個是你的兒子.
當那個是你的牛.
是你的產業.
是關你事的時候.
你們每一個人.
都會冒這個風險.
去把牠拉上來.
所以去到七至十一節.
當開席了.
就是大家要選位坐的時候.
這班人就去搶首位.
搶首位其實不是要.
那個恩主多一些的恩寵.
或者要多一些的恩典.
因為那個恩主已經請了你.
你來吃一餐飯.
那個關係已經成立了.
他們搶了這個首位.
其實是想在這個筵席裡.
得到更加多的東西.
就是要得到同席的人的尊重.
你試想一下.
如果那個恩主請一班人回來吃飯.
那班人可能在社會地位.
各方面的東西.
可能沒什麼分別.
但是如果你坐到那個首席.
其他人就會覺得.
這個人應該很厲害.
於是因為大家不認識.
於是他就會多了其他人的尊敬.
所以耶穌看到他們這樣.

$^{281}$就說了個比喻.
就叫他們要坐末席.
然後就等家主邀請你才坐首席.
就是有些自取其辱的朋友.
就是耶穌的意思.
但是在法利賽人的角度.
其實耶穌的說話是很愚蠢的.
你試想一下.
你一開始選了個末席.
你已經輸了.
你已經虧了.
如果那個家主萬一真的不邀請你.
萬一如果整場的人其實是差不多的.
你選了那個位置.
你就已經是最吃虧了.
所以你怎樣都要選那個首位.
接著去到15到24節.
耶穌就說了一個言直的比喻.
就好像剛才那樣說.
就因為那班人計算過.
覺得這個恩主巴結是不著數的.
於是就拒絕他.
但是還沒完的.
經文說完這個言直的比喻之後.
就直接到26到27節.
耶穌就突然跟從他的人說.
你們無論什麼人.
若不愛我.
而恨自己的父母,妻子,兒女,兄弟,姊妹.
甚至自己的姓名.
不能作為我的門徒.
凡不背自己的十字架.
又不能作為我的門徒.
耶穌的意思就是.
這個很明顯就是門徒的標準.
耶穌的意思就是.
如果你要計的話.
這個就是那個標準.
這個就是做門徒的標準.
然後他還沒說完.

$^{321}$他就再說.
他就再說28到32節.
就是你們中間.
無論什麼人.
或者你們中間.
哪一個要蓋房子.
就不能坐計算.
或者你們中間有誰要打仗.
就不能計算.
如果用現代的說話.
他是在說你們中間有哪一個.
買車,買房子,結婚,生孩子,移民.
你們不會坐下來計算一下嗎.
你們不會坐下來計算一下花費.
計算一下你負不負擔得起.
你才去做呢.
意思就是你們每一個人都會計算.
是不是?很合理.
每一個人都會計算.
耶穌甚至有一個比喻說.
你不計算就被人取笑你.
就是你是蠢的.
所以計算是一件很正常的事情.
是我們每一個人都會做的.
在我們日常生活裡面.
我們都會用這種方式.
去衡量我們的選擇.
衡量我們不同的事情.
計算本身是沒有問題的.
不是一件這麼負面的事情.
但耶穌正正就是說.
如果你將這個邏輯.
放在信仰裡面.
就會怎樣呢.
你就會發現那個代價很高.
這一盤永遠是一盤時半生意.
因為當你一計的時候.
你就會發現那個代價.
就是耶穌26,27節說的.
就是那個標準.

$^{361}$基本上就是你所有的東西.
你想一下.
如果一頓飯.
是要你把所有東西都給出來的時候.
究竟有沒有好處呢.
沒有可能有好處的.
你怎麼計算都會虧蝕的.
所以當你用這個邏輯.
放在信仰裡面的時候.
你一計你就不會跟.
你就不會做他的門徒.
所以耶穌在33節說.
他說你們中間不論誰.
若不撇下一切所有.
就不能作我的門徒.
撇下的不一定是一些很實體的東西.
不一定是很具體的東西.
反而耶穌說的撇下是這一種計算.
正正我剛才這樣說.
一計你就不跟.
所以反過來說就是.
你要跟你就不要計算.
當你撇下這種計算.
你就會跟.
可能法利塞人在這段經文裡面.
有很多種的計算.
剛才說了.
今天我們可能都會有.
一些不同程度的計算.
可能我們會有一種.
有沒有用的計算.
我試試舉個例子.
譬如奉獻.
我們一會兒會傳奉獻袋.
不知道大家有沒有想過.
你奉不奉獻其實是沒有什麼分別的.
你奉獻和不奉獻.
其實Flow Church都會繼續.
下個星期有崇拜.
你奉不奉獻.

$^{401}$你生活裡面的難題.
其實都會繼續存在.
你奉不奉獻.
你的戶口不會有什麼影響.
除了你奉獻的錢之外.
你奉獻了不會突然間多了幾個零.
你奉獻了你的工作不會順利了.
其實是沒有什麼分別的.
在人主觀的角度裡面.
所以當我們覺得沒有什麼分別的時候.
可能我們就會想.
那個計算就會出現了.
那個結果就沒有什麼分別.
但是你做的話.
你就會有代價的.
你要給錢的.
你不做就沒有代價的.
所以我們就會傾向選擇不做.
又例如侍奉.
可能有人邀請你服侍.
你可能會想.
其實教會這麼多人.
都不差再有一個.
你就算拒絕他.
他都總會找到人去服侍的.
這個也是教會的常態.
總會有些人去服侍的.
可能你會想.
其實我去不去都沒有什麼分別的話.
那就要計算一下.
我去我要付出時間.
我要付出心機.
不去不用給東西.
那沒事的.
不去當然是不去.
又或者我們生活裡面.
可能有些東西.
我們相信的價值.
我們去堅持.
可能為上帝去堅持.

$^{441}$可能一開始你會覺得.
嘩!真是很棒.
我想為上帝去堅持一件事.
但是日子久了.
可能你會發現.
其實這個世界是不會改變的.
這個世界不會因為你堅持.
而有任何的改變.
而你堅持.
最後付代價的是你.
可能這樣的時候.
那種計算又會出現.
就是堅不堅持都沒有什麼分別.
那為什麼我要堅持呢.
如果可以不用付那個代價.
為什麼我要付這個代價呢.
又或者好像上次.
上個星期清心說的那樣.
可能我們會覺得.
上帝沒有守他的承諾.
我會和上帝計算他究竟做了多少事情.
以致我們又會做多少事情來回應他.
不知道你聽到這裡有什麼感覺.
你會不會覺得.
耶穌呼籲人撇下一切.
所有跟從他.
是一種很麻木的舉動.
如果我們真的不計算.
閉上眼衝過去.
耶穌說明了是這麼高代價的.
不知道你會不會覺得很麻木.
或者你會不會覺得.
有一種感覺是很不安的感覺.
就好像我們開張支票.
我們有個支票簿.
你想像一下.
你在支票上簽了名.
然後你給了上帝.
你說.
上帝你寫什麼我都給.

$^{481}$但是當你.
真的給了他的時候.
可能你就會擔心.
上帝可能在上面寫.
你的父母.
妻子 兒女.
兄弟 姐妹 你的姓名.
或者你的十字架.
因為這個正正就是耶穌.
26 27 節說的話.
我呢.
由信主開始.
服事.
我都覺得我好像這個狀態.
我都不會覺得.
自己跟上帝有什麼計較.
總之有人邀請.
我就會去.
或者最後呼召.
可能有些人會幾次.
我一次就會答應.
我就去.
我整個的路都是很直線.
我從來都不覺得自己.
有跟上帝計較過什麼.
但是當人長大了.
畢業了.
結婚了.
搬了出來住.
我越來越發現.
我很不想上帝在支票上面.
有些東西寫上去.
譬如是錢.
可能是一些侍奉的時間.
可能是一些牧羊的對象.
我發現.
我是會有一些東西.
我不太想上帝在支票上面寫下.
我不知道.
其實我覺得每個人.

$^{521}$到了某一個年紀.
慢慢成長的時候.
可能你總會有一些東西.
你會很怕上帝會要你的.
但是我想說一件事.
就是.
耶穌他發出這個邀請.
他邀請人撇下一切.
所有的計算去跟從他.
是因為上帝.
首先對人.
放下了這種計算.
如果我們再看回.
大賢直的比喻.
15到24節.
當那班人.
他們拒絕了.
這個家主的邀請的時候.
這個家主做了什麼呢.
這個家主就馬上寄了一個僕人.
去邀請那些黑眼.
瘸腿.
貧窮的人來進入他的賢直.
為什麼是這個僕人呢.
這個僕人正正就是耶穌在.
12到14節說的.
耶穌在12到14節說.
你們要請客吃飯的時候.
你們不要請你們的親屬.
你們不要請你們的兄弟姊妹.
甚至你們有錢的鄰舍.
因為他們可以請回你們.
你們就得著報答.
你們要請吃飯的時候.
你們要請那些貧窮黑眼瘸腿的人.
因為那些人沒有辦法報答你.
所以當耶穌說.
這個家主他決定請這班人的時候.
正正就是在說.
這個家主從來都沒有想過.

$^{561}$在他宴請這班人的身上.
拿到任何的好處.
甚至經文.
說到下面.
他說要勉強人進來.
你想一下.
貧窮黑眼瘸腿可能都是.
物理上是不能夠回報家主.
但是那些普通人.
他勉強進來的那些普通人.
那些人是有能力的.
但經文說.
家主都是勉強他們進來.
也就是說他們就算有能力也好.
家主都沒有想過要他們.
有任何的回報.
因為家主從來都不是.
想透過這個筵席.
和這班人締結一個.
利益的關係.
這個家主從來都不是要這班.
受邀請的人怎樣去益他.
怎樣去榮耀他.
怎樣去令到他.
臉上有光.
這個家主.
這個筵席的邀請.
從來都是因為這個家主.
很想和一班受邀的人.
一起分享他所擁有的東西.
弟兄姊妹如果.
上帝今天真的拿一本小氣堡.
和我們算.
就是我們每一天.
他都算了寫下.
你覺得我們真的.
有一個人.
可以進到這個筵席嗎.
我們是否真的憑實力.
做到耶穌這個標準.

$^{601}$而我們進入這個筵席呢.
我們是否真的因為.
自己做了一些很好的事.
我們真的榮耀到上帝.
來益了他.
所以我就可以.
以一個受恩人的角色.
參加這個筵席.
究竟是我們比上帝的東西多.
還是上帝比我們恩典多呢.
弟兄姊妹奉獻.
從來不是要滿足一些要求.
一些規定.
不是說上帝在聖經裡寫了什麼.
或者他要什麼.
然後我們就要滿足他.
我們就一定要給.
甚至奉獻不是一個有用的問題.
不是你要考慮.
究竟我要奉獻給哪個機構.
哪間教會.
就是因為他很等錢用.
所以教會永遠一告急就會有很多人奉獻.
奉獻.
是一個學習的過程.
就是學習怎樣.
對上帝放下這種計算.
然後我們開始去問自己.
對了.
這個上帝就是不求回報.
他從來都沒有要求過我們.
做什麼.
我們開始問自己.
究竟我想為這位上帝.
擺上什麼呢.
我想怎樣去回應他呢.
好像清心上個星期這樣說.
你信不信這個上帝.
是信守這個承諾.
由始至終.

$^{641}$他都不會和我們計算.
他不是要拿走我們什麼.
也不是要我們怎樣去.
益他.
如果你給他一張空白的支票.
你信不信這個上帝.
到最後他都會.
給你一張空白的支票.
然後他要你想.
究竟你想寫什麼上去.
去給他.
我們有一個祈禱.
因為你是那個現值的家主.
你說.
你不和受邀的人計算.
就是正正你.
首先放下這種計算.
以致我們今天.
可以得到你這個現值.
所以我求你.
你在你的愛裡.
你幫助我們.
你幫助我們生命裡經歷.
你自己的因典.
你自己的恩惠.
我們生命因著你得到改變的時候.
讓我們學習去回應你.
讓我們學習去擺上我們自己.
各人不同的.
東西去回應你.
求你這樣去轉化我們.
讓我們在這個回應裡.
繼續去經歷你自己更多.
我們這樣禱告.
奉主耶穌的名教.
阿們.
\newpage



\section{馬可福音 12:41-44-20230805}
\label{sec:IJY_UvLqQqw}
\textbf{【網上崇拜】窮寡婦的奉獻:你可能沒有想過的角度|馬可福音12\_41-44|20230805 [IJY\_UvLqQqw]}
\newline
\newline
連結: \href{https://youtube.com/watch?v=IJY_UvLqQqw}{\texttt{ https://youtube.com/watch?v=IJY\_UvLqQqw}} ~~~~ 語音日期: 2023-08-05 
\newline
\newline
\hyperref[sec:nQfdSDvE_CU]{\small{< < < PREV SERMON < < <}}
~
\hyperref[sec:index_chronic]{\small{[返順時目]}}
~
\hyperref[sec:index_scriptual]{\small{[返順卷目]}}
~
\hyperref[sec:LgSzajW7sqo]{\small{> > > NEXT SERMON > > >}}
\newline
\newline
馬可福音 12:41-44-20230805
\newline
\begin{longtable}{cl}
\hline
\hline
章節 & 經文 (和合本修訂版)\\
\hline
12:41 & \begin{tabularx}{0.7\textwidth}{X} 耶穌面向聖殿銀庫坐著,看眾人怎樣把錢投入銀庫。有好些財主投了許多錢。 \end{tabularx} \\ \\ \relax
12:42 & \begin{tabularx}{0.7\textwidth}{X} 有一個窮寡婦來,投了兩個小文錢,就是一個大文錢。 \end{tabularx} \\ \\ \relax
12:43 & \begin{tabularx}{0.7\textwidth}{X} 耶穌叫門徒來,對他們說:「我實在告訴你們,這窮寡婦投入銀庫裡的比眾人所投的更多。 \end{tabularx} \\ \\ \relax
12:44 & \begin{tabularx}{0.7\textwidth}{X} 因為,眾人都是拿有餘的捐獻,但這寡婦,雖然自己不足,卻把她一生所有的全都投進去了。」 \end{tabularx} \\ \\
[1ex]
\hline
\hline
\end{longtable}
$^{1}$好 定智妹晚安.
很歡迎你來到流唐的崇拜.
建議大家坐第一排.
其實頭排是有位置的.
這邊有五個 這邊有五個.
坐第一排的好處是當你看到後面的時候.
你會看到弟兄姊妹的敬拜樣子.
很美麗的圖畫.
剛才大家一起去敬拜的時候.
看到大家每一個敬拜時候的面孔.
我們一起來敬拜.
獻上我們的敬拜.
這也是我們的獻祭.
我們的奉獻.
來到八月了.
八月是一個熱騰騰的八月.
仍然是我們奉獻的主題.
我們的講題叫做.
窮寡婦的奉獻.
一個你可能沒有想過的角度.
很明顯後面的一段是一個.
吸引你看下去的題目.
內容內容.
是郭富城失去的一切.
後來我試過叫Chet GPT幫我寫.
請給我用內容內容誇張的方式.
給十個題目.
很正 我讀給大家聽.
第一個給我.
震驚 這位教友的奉獻如何.
令整個教會翻天覆地.
第二個是.
你不會相信這個神秘的方法.
竟然可以讓你的信仰變得更加充實.
第三個就是.
這位牧師如何用奉獻的力量改變整個社區.
你不會相信這個小教會通過奉獻實現他的夢想.
這些東西.
今天我自己的題目也挺內容化.
不過今天也要說一個大家都認識的經文.

$^{41}$窮寡婦 耶穌在聖殿裡面.
對於窮寡婦.
將所有兩個小錢都奉獻一段經文.
來更深的了解.
其實我們這兩個月裡.
也會提很多有關奉獻的經文.
希望能夠讓每一段奉獻的經文.
可以有更多的了解.
更深更闊.
更多的認識經文.
這不是唯一的角度.
不過可能也會讓你更加認識.
我們這段經文的方式.
我自己在讀經文.
這段經文在馬福音第十二章.
寫到四十四節經文.
我們一起讀 好嗎.
我們大聲地說.
我所對任何苦痛的當日.
我都用仁慈良喉之言.
它有寫在我對求要的當日詞.
由於我窮寡婦內.
我對求要兩個小錢.
就此讓我大錢.
也送你們去學書.
我實將告訴你們.
這窮寡婦求入苦裡的.
比眾人所求的更多.
因為他們都視自己有餘.
拿出來後再去求.
但這寡婦視自己不足.
但凡一切良心的.
都求得上了.
我們一起祈禱.
在你將這段.
我們聆聽你的說話的時間.
我們交代給你.
使用孩子卑微的預備.
求尊你自己的靈.
親自對你說話.

$^{81}$我們每個靈千倍能夠.
可以更加深的去思考.
奉獻這個課題.
有你的生命能夠緊貼.
回應你的心思意念.
求尊幫助我們.
奉傳明求 阿們.
要明白這段經文.
首先你要知道.
這段經文其實是說什麼.
你要知道整個的context.
整段經文的上下文.
其實是說一個什麼的故事.
其實這段窮寡婦奉獻的經文.
從來都不是一段.
獨立出來的故事.
不是從前有一個窮寡婦奉獻.
所以要稱讚她 完.
窮寡婦奉獻的故事.
其實在福音書裡面.
擺放在一個更加深.
更加闊的一個大故事裡面.
這個故事的大的主題.
其實就是耶穌和耶路撒冷.
那個宗教集團的衝突.
這個是整個福音書裡面.
更加 去到那段經文裡面.
更加宏觀的一個段落.
窮寡婦奉獻有兩個書的版本.
一個是馬福音 一個是路加福音.
不過無論是馬可還是路加都好.
其實兩段窮寡婦奉獻經文.
其實它的上下文的主題.
其實都是置放在同一個theme裡面.
就是耶穌和耶路撒冷的宗教集團之間的衝突.
無論是在馬福音第十二章十三節開始.
還是在路加福音第十二章開始.
福音書都放了很多的篇幅.
不斷地鋪陳耶穌和耶路撒冷文士之間的conflict.
所以如果你細心留意的時候.

$^{121}$你會發現窮寡婦奉獻經文.
無論是馬可還是路加都好.
其實兩段經文都是放在同一個經文的下面.
無論是路加還是馬可.
你會看到兩個窮寡婦奉獻經文.
都是緊貼同一段經文.
那經文是什麼呢.
正興正濟 耶穌去斥責耶路撒冷的文士.
我們先看經文.
基本上兩段經文都是一樣的.
耶穌是說同一番的說話.
耶穌說 你們要防備民事.
他們好穿長衣遊行.
喜愛人在街市上問他們安.
又喜愛會堂裡的高位.
然直裡的手作.
他們侵吞寡婦的家產.
假以作很長的禱告.
這些人要受更重的刑罰.
兩段經文是一樣的.
當耶穌斥責完耶路撒冷集團的文士後.
他就主動叫門徒去看.
聖殿裡的窮寡婦奉獻的情況.
為什麼耶穌斥責完這些文士集團後.
他就要提及窮寡婦奉獻這件事.
兩段經文有什麼關係.
兩段經文其實不是巧合.
路加是這樣也不奇怪.
但路加和馬河都是這樣編排的時候.
你會發現兩者是不能拆開的.
兩者不是隨隨意意放在一起.
而是有一定程度的目的.
所以在我們看窮寡婦奉獻經文之前.
我們要去認識一下.
更深認識一下文士集團是什麼一回事.
究竟文士是什麼人呢.
文士按字面來說就是一些懂字母的人.
就是一些懂寫字的人 讀書人.
所以文士就像今天秘書那些字.
可以很容易解作普通通公司的秘書.

$^{161}$也可以解作政權高級的秘書.
秘書長這樣的角色.
所以我們必須要去明白一下.
究竟方書裡面.
耶穌所譴責這些在耶路撒冷裡面的文士.
究竟是在當時的社會裡面.
是一個怎樣的人.
什麼樣的人.
我試問另一個角度問大家.
如果是你的話.
你會比較想當文士還是法律出行.
當然兩個都不想當了.
文士比較厲害還是法律出行比較厲害.
古老命那些比較喜歡.
誰比較有力量.
誰比較厲害.
你可能會說他們差不多.
反正經常都說法律出行和文士差不多.
其實不是的.
雖然經常黏在一起說.
但文士和文士 法律出行和法律出行.
兩個是完全不是一樣的東西.
雖然文士可以是法律出行.
法律出行也可以是文士.
但兩者是有一定程度的分別.
法律出行是什麼.
法律出行就是當時猶太教裡面.
一些對於猶太律法比較摧毛求疵.
很搜救很訛嚇的人.
很嚴格對於律法有很多的.
很多執著的要求.
多餘的要求 很瑣碎的要求.
所以法律出行是純粹的業餘興趣.
我對於律法一定程度的要求.
一定程度的取向態度.
所以當時不少法律出行都是普通業餘的平民.
對於律法很執著.
都可以成為法律出行.
他們不一定是宗教上的專業人士.
但文士就不同了.

$^{201}$文士是宗教上的專業人士.
新時代裡面不少文士是利未人.
甚至是祭司.
最少在以斯拉紀裡面.
以斯拉紀是祭司.
又是一個擅長摩西法律的文士.
是這樣的角色.
所以很多文士都是來自於宗教的首都耶路撒冷.
他們是出生於或來自於耶路撒冷.
可能是來自於加利利.
兩班不同的當次人.
所以文士是屬於耶路撒冷聖殿裡面.
宗教權力的人士.
他們有什麼權呢.
要低一點.
他們有什麼權呢.
他們出於宗教上的知識水平高.
他們有權力.
有權威來解釋聖經.
他們是猶太人的宗教信仰的指導者.
告訴你如何跟隨耶和華上帝.
第二除了解釋聖經之外.
他們更加擔當司法角色調查.
他們在社會的訴訟裡面.
他們有權力來主持民事訴訟.
有權力來處理糾紛的人.
所以基本上文士是一個很好做的工作.
他可能是祭司一年裡面.
他是被禁止做耕種.
他一年裡面有幾個星期在聖殿裡面當值.
其餘時間就是負責用空間時間來詮釋聖經.
或者來處理民事糾紛.
佛教上人就不是了.
佛教上人可以有一份星期一至六都要上班.
他是一個平民一個普通的宗教愛好者的角色.
所以以上就是新的年代.
最少在公元70年之前.
當聖殿被毀之前.
一個這樣的宗教權力的結構.
在公元70年前當聖殿被毀之前.

$^{241}$民事作為一個祭司.
他去獨攬了整個聖殿的宗教權.
當然當聖殿被毀之後.
這群人就沒落了.
從此佛教上人就真正興起.
所以耶路撒冷裡面的宗教系統就是這樣.
有一群民事作為民事的祭司來主管.
他們控制運作整個聖殿.
聖殿裡面的管理宗教的稅收.
信仰的話語權.
都是在這群民事手中.
他們是一群經營管理聖殿運作的一群人.
所以這是當時對於整個猶太社會裡面.
民事就是一群這樣的人.
用他們的知識來在整個宗教權力的最上層裡面.
所以你明白這段經文的背後之後.
你再看這段窮寡婦經文.
你就會明白多一點.
我們再看這段經文.
當耶穌指名道姓去斥責這群民事的邪惡.
耶穌說你們要防備民事.
他們號穿長衣遊行.
喜愛人在街市問他們的安.
又喜愛會堂裡的高位.
認識在的守衛.
他們侵吞寡婦的家產.
假意作墾場的禱告.
這些人要受更重的刑罰.
今天我們不是簡短經文.
不過我們留意這句說話就足夠了.
第四個節目是什麼.
他們侵吞寡婦的財產.
假意作墾場的禱告.
這裡說民事是侵吞寡婦的家產.
所謂侵吞寡婦的家產.
不是打劫.
不是勒索.
不是偷不是搶.
就是叫他們奉獻.
所以這群民事是假意作很長的禱告.

$^{281}$他們用一些很敬虔的宗教的外貌.
擺上一個敬虔的模樣.
來叫這個窮寡婦奉獻自己的家產.
來去成就他們的利潤.
所以就明白.
為什麼耶穌批判完這群民事之後.
就會說這個窮寡婦的奉獻.
原來是這樣的.
如果我們用這個角度去看經文的話.
就會發現.
這段經文的意思.
我再說一次.
窮寡婦經文.
第四十一節.
預備一二三.
耶穌對.
.
(賴喇人唸解釋).
如果我們將耶穌和這班民事集團的衝突.
放在我們的眼前.
再讀一段聖經的時候.
你可能會有一個新的語氣.
窮寡婦奉獻不是純粹一個普普通通.
將所有奉獻給主很有信心.
大家學一下吧 經文.
窮寡婦奉獻其實是耶穌帶著一些唏噓無奈.
有點sarcastic的一個comment.
我們試試用這樣的comment去讀這段經文.
接著看這段經文.
這班民事正一衰人來的.
將來有極重刑罰.
你看看.
窮寡婦投入這個庫裡面比所有人都更加多.
其他人都只是將他們有餘的拿出來.
但是.
這個窮寡婦本來已經是不夠的了.
他將他一生的金錢都給了這班民事.
可以是一個這樣的閱讀.
耶穌不一定是稱讚這個窮寡婦.
而是從這樣的角度去看這件事.

$^{321}$當然你會懷疑.
究竟耶穌是帶著一種稱讚的語氣.
還是一些唏噓的語氣來講這番話呢.
當然客觀來說是兩種可能.
兩種可能.
耶穌可以是帶著一個稱讚的角度來講這番話.
也可以是帶著一個唏噓的語氣去講這件事情的經過.
不過我們可以發現.
一個事實就是.
耶穌並沒有在字面上特別去稱讚寡婦.
耶穌沒有.
耶穌沒有說將來要紀念她.
也沒有叫門徒快點學學跟隨她.
沒有的 耶穌沒有叫你學.
耶穌說什麼呢.
耶穌說了一個很簡單的客觀的事實.
窮寡婦比所有人課金都課得多.
傾盡一切的都給了.
這是一個事實.
他沒有叫你跟.
也不一定是稱讚.
糟了 說到這裡我似乎叫大家不要奉獻多一點.
好像叫大家不要奉獻多一點.
不是的 我想說大家要奉獻的.
我想問題的重點是.
很在乎耶穌怎樣去理解.
這個接受金錢的對象.
究竟他是奉獻給神.
還是純粹課金給這班民事呢.
其實耶穌對聖殿要有人運作是完全沒有問題的.
聖殿是耶路撒冷說上帝的殿.
聖殿有奉獻是沒有問題的.
耶穌怎會反對人向上帝奉獻呢.
耶穌對操控聖殿的管理層是有問題的.
耶穌對要課金給這班管理層覺得有點意見.
奉獻給上帝是應該的.
海撒歸給海撒.
上帝歸給上帝.
奉獻給上帝是應該的.
問題是課金給這班民事是一件很PK的事.

$^{361}$這個就是差別.
如果這個奉獻是正常奉獻給神的話.
耶穌肯定是讚好.
肯定會稱讚窮寡婦.
正如我們以前經歷過.
這個當然是很好的榜樣.
因為他願意獻上一切給上帝.
這個窮寡婦不是問題.
但如果這個所謂的奉獻.
只不過是一種民事集團去謀取利潤來自肥.
來叫窮寡婦的錢給他們的話.
那耶穌覺得不值得了.
這個就是關鍵所在的問題.
當然問題是如何分兩者.
如何才算奉獻給神.
如何是一種宗教的課金.
我覺得這是我們今天這篇道裡.
和大家更加入獄去傾談的一個課題.
什麼叫做奉獻給神.
什麼是宗教機構的課金計劃.
我以前說過我是基督教裡的紅婦出身.
我是讀完書之後做了呼音幹事.
開始起家的.
我沒有做過任何熟悉的工作.
除了最近有做外賣之外.
我是紅婦.
我是由幹事做起做到現在.
我是從教會工作做到現在.
沒有做過外面的工作.
那時候我讀完大學之後.
我打了一份工做form幹事.
做一個教會幹事.
一個很重要的職責是什麼.
就是星期一要將教會星期日的奉獻.
拿去銀行入數.
所以我每逢星期下午.
就會拿著教會的錢.
進去匯豐銀行排長龍.
排隊入數.
這是我每個星期做的事.

$^{401}$其實我想說每次我排隊.
都看到人生百態.
很多不同行業的人都在排隊入數.
入數時看到很多人是進去很多省銀.
可能是做街市的.
可能是做茶餐廳的.
如果你看到有很多張支票.
又有一定程度的省市.
有一定程度的現鈔.
是什麼呢.
他就是form幹事.
一疊支票和不少的省市.
和一定程度的十元.
這個應該是form幹事.
所以我每個星期都是.
將大家神聖的奉獻.
拿去教會匯豐銀行入數.
很怕被人打劫的時候.
我想說什麼呢.
其實大家都知道的.
所謂向神奉獻這些錢.
是怎樣的.
從來都不是上神拜那裡.
你不是的知道吧.
你知道不是的.
上去就收了錢.
不是的.
你知道我們.
不知道大家知不知道.
每個星期崇拜之後.
在那間房裡面.
都有我們幾個牧者和幹事.
在那裡賭錢.
你知道什麼是賭錢嗎.
賭錢賭到最後就數.
這是教會經典劇.
崇拜後賭錢和犯罪.
我覺得這些劇很舊.
真心說.
奉獻給上帝.

$^{441}$從來都不是一件抽象的事.
你都知道.
奉獻給上帝.
其實就是在教會銀行戶口裡面.
原來是這樣的.
原來知道這些事.
不過是真實的.
就是流堂有14位牧者.
3個影音同工.
3個行政同工.
還有Operation 5個點.
每個月維持他們生活所需的薪金.
同工的醫療津貼.
公NPF 火險第三寶.
租這裡的租金.
MFC的租金.
上個星期我們網絡不穩定.
我們就特意要再弄一隻WIFI蛋.
那都是錢.
敬拜隊就是volunteer.
但是敬拜隊那些吃飯.
每個星期都要一千多二千元.
我想說不是update出什麼鬼.
我想說一間教會就是這樣.
全部都是需要.
這些運作需要很多很多的支出.
所以有人問我.
為什麼你開流堂的時候.
不乾脆開一間網上教會就算了.
是真的.
因為一間網上教會是便宜很多的.
又不需要請牧者.
又不需要租地方.
你自己每個星期開live就講到就ok.
那就開吧.
我明白的.
這樣去做教會是很簡單的.
事實上這個反大台.
反體制提倡去堂會化的年代.
很多人覺得開一間IG教會.

$^{481}$開一間連登教會.
開一個youtube channel叫自己叫教會.
這樣就可以了.
但是我不覺得這樣是可以的.
這樣做不到事情.
你開一個IG教會.
開一個連登教會.
開一個youtube教會.
極其量就是多一個IG.
多一個連登post.
多一個youtube channel.
不是說這些沒有用.
但是你要真真正正去長久行常地.
在香港去見證耶穌基督.
好好去牧養一班人.
長遠去運作.
這樣就不能單單去搞IG.
或者寫一下facebook.
或者在連登寫一下東西.
Full Church並不是這樣.
Full Church是一個這樣運作的教會.
所以我想說的是.
聖殿是需要金錢奉獻來維持運作.
這是一件正常不過的事情.
聖殿裡面有祭司有利美人.
要維持聖殿運作的人.
從來都不是一個問題.
耶穌不反對這件事.
事實上聖經從來都很realistic.
舊時也明白的.
寫什麼呢?.
利美人的生活開始是誰出的?.
是其他十一個支派來供養他們.
是很realistic的.
利美人的生活不是突然之間天上來.
不知道怎樣就有這樣養他們.
就是很實際的其他十一個支派來供養他們.
一間教會有運作.
有一定的體系.
需要資源來運作.

$^{521}$這是一件很必然的事情.
需要營運需要一定程度的資源.
需要來做一些事.
這是一件很正常不過的事情.
回到我們剛才問的問題.
究竟奉獻給上帝和宗教課金兩者有什麼分別?.
怎樣才是真真正正奉獻給神?.
怎樣才是耶穌所反對的.
課金給這班的民事?.
我的答案是.
耶穌不是去反對這些運作的體制.
也不是去反對人去奉獻給這些地上運作的體制.
耶穌所反對什麼?.
耶穌是反對當這些體制並沒有真正發揮應有的功能的時候.
當這些地上信仰的體制純粹為了自己的運作而運作.
並沒有發揮應有的上帝功能.
純粹為了自己的生存而去課金的時候.
這只是一個宗教外表的一個牟利機構.
耶穌是譴責這樣的牟利機構.
總而言之,用我上次所說的說法.
沒有犧牲的信仰體制.
正正就是耶穌所斥責的.
當人奉獻給這些人的時候.
就等同於浪費了你的金錢.
耶穌覺得很可惜的事情.
反過來說.
只要那個地方仍然是在發揮上帝應該要有的功能.
只要那個地方仍然是讓人可以預見上帝的時候.
那個地方就是佩德有奉獻的地方.
我相信今日香港仍然有很多這樣的地方.
很多這樣的教會,很多這樣的機構.
他們仍然在努力發揮上帝應有的功能.
很好的努力讓人每天都能夠見到上帝.
很努力地去犧牲,付出.
有關奉獻的話,我大概就說這麼多.
上帝的犧牲,教會的犧牲.
弟兄姊妹每個人的犧牲.
讓流塘這個地方成為一個美好的地方.
成為一個我們能夠預見上帝的地方.
只要你在這個地方仍然看到上帝正在工作.

$^{561}$當你在這個地方見到上帝的時候.
你就可以將你的奉獻放在那個地方.
不過你要問我要奉獻多少才夠呢?.
所奉獻是一個參考,但這從來不是我的期望.
因為我知道未必每個人都能夠做到這件事情.
也不需要像窮寡婦一樣傾盡一切去奉獻出來.
我唯一想流塘弟兄姊妹能夠做到的.
就是在你生命裡建立一個犧牲奉獻的態度.
一個犧牲奉獻的生活習慣.
從此之後讓犧牲奉獻成為你的生活方式.
你的信仰方式,你面對上帝的方式.
我知道流塘的活動是挺好玩的.
剛才那些活動是做得很漂亮的.
那本支票我講得很清楚,不像,我不收貨.
一定要像支票才可以.
不過我不想看到甚麼呢?.
就是突然間多了很多這樣的活動奉獻.
一過之後就沒有了.
我寧願你從此之後你的奉獻態度真的被改變.
將來能夠收到你每次少少少少的奉獻.
也不想在這個月裡突然間因為玩活動的緣故.
突然間一次過玩一玩就沒有了.
重點不是收你多少錢或多少奉獻.
而是你自己願意改變你的奉獻態度.
網上頂禎也一樣.
如果你覺得這個地方是一個崇拜的時候.
如果這個崇拜讓你看見上帝的話.
這個地方也可以是你奉獻的地方.
這裡是有上帝的工作進行.
因此這個也是需要你奉獻的地方.
最後,問答問題.
知不知道窮寡婦的經文之後是甚麼經文呢?.
無論是路加還是馬可都是一樣.
無論是路加或者馬福音.
窮寡婦的經文之後.
就是耶穌預言聖殿被毀的經文.
不是偶然的.
三段經文是連在一起的.
耶穌去批判當時的耶路撒冷集團.
然後去講窮寡婦的奉獻.

$^{601}$再預言聖殿將來要被毀.
很諷刺.
窮寡婦在兩個小時前.
落入聖殿集團的袋子裡之後.
最後還是化為烏有.
我想說我們的奉獻不是為了將來.
這個地方有多大有多好.
我也說不知道留堂這個地方.
其實也沒想過要營運多久.
這個奉獻也不是純粹為了擴大.
或者將來有甚麼大的項目.
教會是一個活動.
奉獻也是一個活動.
今天讓人可以見到上帝.
今天有人被牧養.
今天有人可以見到耶穌基督.
這個就是我們留堂.
這個星期我們做好做好事情.
我們有足夠的金錢.
今天能夠完成以上的事情.
這個就是我們能夠做到的事情.
這個就是我們有奉獻的原因.
讓這個地方可以繼續留堂.
可以成為一個香港見證耶穌的地方.
多謝祈禱.
再一次求你引領我們教會.
讓我們能夠走在你的方向裡.
讓我們能夠成為一個真真正正.
來發揮你的功效.
在地上見證你.
牧養你的人.
裝備他們.
來成為你的門徒的人.
主耶穌幫助我們.
讓我們能夠活在當下.
能夠將自己的金錢.
成為我們所需要的.
繼續為你見證.
求主你幫助我們.
讓我們留堂.

$^{641}$能夠成為一個願意犧牲的地方.
無論是頂尖會,牧者.
全個教會的人.
都將我們最好獻上.
奉主命求.
阿門.
謝謝大家.
\newpage



\section{利未記 3:1-5-7:11-28-34-20230812}
\label{sec:LgSzajW7sqo}
\textbf{【網上崇拜】感謝祭|利未記3\_1-5,7\_11,28-34|20230812 [LgSzajW7sqo]}
\newline
\newline
連結: \href{https://youtube.com/watch?v=LgSzajW7sqo}{\texttt{ https://youtube.com/watch?v=LgSzajW7sqo}} ~~~~ 語音日期: 2023-08-12 
\newline
\newline
\hyperref[sec:IJY_UvLqQqw]{\small{< < < PREV SERMON < < <}}
~
\hyperref[sec:index_chronic]{\small{[返順時目]}}
~
\hyperref[sec:index_scriptual]{\small{[返順卷目]}}
~
\hyperref[sec:xHhMd2gjsxw]{\small{> > > NEXT SERMON > > >}}
\newline
\newline
$^{1}$頂姐妹平安 願你平安.
今日崇拜的港島是雙月提奉獻的第七港.
接下來還有第八,九港.
我們都期望神的話語繼續滋潤我們提醒我們.
很感恩今天有機會參與港島的服事.
之前一直知道要港島的時候.
都在想港島有什麼好呢.
就想不如說說憲制 雖然在舊約的憲制我都不太熟悉.
但感恩祭好像都挺有趣的.
就想好就這個方向.
一直在等 都很擔心.
跟之前的港元都類似 擔心撞經民.
兩個星期前就問陳偉安牧師.
問了他之後 你這個星期說什麼.
原來說新約 幸運而已.
可以放心開始入二波準備港張.
如果真的要我這麼短時間內再構思港張的話.
其實對我來說有一定的難度.
雖然我本身都一個熟悉我的人.
都知道我是一個deadline fighter.
但這次這麼巧 明天又要港島.
也要開新的港張 我一定會死的.
除非好像我以前在香港十幾年前全職的全職務會.
教會的主任牧師.
又或者像John或者潘Sir一樣.
多工作 做得好又做得快.
如果用開電單車來形容的話.
我相信我連燈尾都看不到.
很感恩順主透過不同恩賜的人.
去成為教會的禮物 去造就生命.
神興起不同的屬靈領袖.
令神的子民蒙福.
神都曾經興起過摩西.
去帶領以色列人出埃及.
之後就設立了憲制的制度.
讓人去敬拜神 親近神.
在舊約聖經利美記裡面提到有五種的憲制.
包括繁制 素制 平安制 贖罪制 和贖獻制.
各有不同的功能.
今天講到的重點焦點.

$^{41}$我們就放在平安制那裡.
而平安制本身都有三個類別和功能.
到底對我們今天的信徒有什麼意義呢.
以色列人這五種的憲制裡面.
他們獻的祭物包括牛 羊 麵團等等.
這些在經文裡面稱為共物.
這個希伯來文的字眼.
音譯是葛二版.
字的意思是有禮物的意思.
換言之 以色列人是透過憲制.
向上主呈上禮物.
可以達到不同的預期目的.
正如我們去參與聖餐 銅餅 銅杯.
是表明我們完全去認同和主耶穌基督所立的新約.
我們都有經驗去送禮的.
平時都會送禮的.
我們去送禮給別人可能都同樣有不同的目的原因.
昨天才賠完罪.
詳情不說了.
人類很複雜的.
送禮給別人可能是夾雜著不同的動機.
有些人可能好像剛才說的很純粹.
不是想有什麼回報.
即使明知想請人吃頓飯也好.
都知道明知是無法報答他.
但都很想請人吃頓飯.
做一點事情.
正如神的恩情.
即使我們想還.
其實一輩子都還不完.
還不到.
幸好神並不是想我們還什麼給祂.
神是愛.
祂不會當人是韭菜.
重點不是在祭物的本身.
舊約的傳統都很強調這一點.
例如先知尼加提到我們很熟悉的經文.
耶和華喜悅千千的供養.
或是萬萬的柔和.
神期望的是人有生命的改變.

$^{81}$與神同行.
這個機制的設立.
其實是讓以色列人親近主.
以色列人的聖殿被戰火摧毀.
想獻祭也無法獻.
敬拜的自由未必能夠十年二十年不變.
又或者會移民離開.
所以我們都很珍惜敬拜的機會.
多點為我們現在有的東西去感恩.
去珍惜和我們一起去敬拜的弟兄姊妹.
顧名思義.
透過獻祭去讓人參與.
經文有沒有說呢.
我們先看看經文.
或者我們一起讀.
不是很長的經文 我們一起讀出來.
.
這種沒有產生的現在的和華面前.
乃要安守在公民的頭上.
在軟匯網門口.
發動自主作制之力.
要把血灑在癌的周圍.
從平民的安制中.
將火製建革的和華.
乃要把火裡的自由和壯相所有的自由.
並量化要恃和要恃上的自由.
就是靠要量化的自由.
與當上的旺恃和要恃.
的確取巧.
法輪的子孫要把這些燒在癌的監制上.
就是在火的柴上.
是意義的保留為輕量的火制.
好,謝謝.
經文有沒有說到.
有沒有具體說過給我聽.
有沒有呢?.
似乎沒有.
所以有理由相信.
應該凡是為了想來到神面前.
為了神所賜的任何恩典.

$^{121}$都可以去到會幕附近的祭壇.
去獻祭向神表達感恩.
我通常在每年的年尾.
或是年初的時候.
都會送一些小小的禮物給祖源.
去表達一些心意.
今年也不例外.
不過如果這樣算價錢.
今年送的就是這幾年來說.
最便宜的禮物.
不要小看它.
就是這件事.
勿輕情意重.
是什麼呢?.
當然不是護照.
它像一本護照.
但是像而已.
不是啦,不是像.
這本簿子是不簡單的.
雖然是便宜的.
但是是日本製造的.
我希望鼓勵大家.
祖源和我自己都是.
提示自己.
用來記一些感恩的事情.
老老實實.
其實隨著時間過去.
可能連上星期發生什麼事.
都可能忘記了.
喜怒哀樂,悲歡喜樂.
可能那間雞蛋仔隱世小店.
直覺很棒.
都會忘記了.
如果能夠平時記住神.
在自己生命裡的一些作為的話.
有時揭開來看都很治癒.
譬如我舉個例子.
剛才開始之前也看過.
有一兩件事簡單一點說.
這個是三月尾還是四月頭.

$^{161}$亞通停止了.
95\%沒事.
這些不方便說.
小組組員大家的關係.
都很融洽.
特別新組員的融洽都很快.
比想像中.
很棒.
類似這些.
這些不juicy.
但是都很開心.
在當中提醒自己.
其實神有很多恩典.
有很多愛在生命裡.
可能真的忘記了.
雙雙這樣做的時候.
回憶的時候.
回憶回來的時候.
比平時很簡單.
很籠統地說.
雖然都很實在.
但是有些具體一點.
如果覺得這些方法.
都不適合自己的style.
亦都可以考慮記在手機.
電腦.
或者畫在牆身.
MRC就不知道行不行.
應該不行.
總之用我們最適合自己的方法.
方式去記下.
都很有意思.
除了個人的經歷.
為了這些個人的經歷.
去獻這個感恩祭之外.
還有就是.
相信是因為群體.
那種共同的經歷.
而向神感恩.
所有獻的牛或者羊.

$^{201}$一定要在當日.
即獻祭那一天之內吃完.
而不可以好像平安祭裡面.
有一種目的是還願.
這個還願.
可以留肉多一天.
遲一天才吃.
都可以的.
在利未記七章十五節裡說到.
為感恩而獻的平安祭的肉.
是要在獻祭當日吃.
一點也不可留到早晨.
經文亦都沒有直接說得清楚.
為何要有這個安排.
不過有些經學者就指出.
其中一個可能性.
就是因為那句說話.
一點也不可留到早晨.
是很容易令人聯想起.
以色列人出埃及之前的經歷.
就是逾越節.
就是神婚婦人.
即以色列人去屠宰羊羔來吃的時候.
她說吃剩的都要燒掉.
出埃及記十二章說到.
不可留下一點留到早晨.
是可以為了一些原因.
這個解釋都很合理.
不知大家覺得怎樣.
但無論這個解釋是否完全正確.
神的子民都很應該.
去為神的恩典救恩.
其實耶穌幾多.
使徒彼得在彼得田書屋.
很熟悉的經文都有說.
你們是被揀選的一族.
是君尊的祭司.
教會和教會的信徒都是.
包括我們都在神聖的角度.
是神聖的子民.

$^{241}$這些說話對於很熟悉.
舊約聖經的猶太裔信徒.
基督徒來說.
一聽到的時候自然浮現到一個畫面.
摩西在西賴山上領受神的啟示的時候.
那段宣告.
出埃及記第十九章說.
我耶和華向埃及人所行的事.
你們都看見了.
如今你們若真的聽從我的話.
遵守我的約.
就要在萬民中.
作屬我的子民.
因為全地都是我的.
你們已歸我作祭司的角度.
為神聖的國民.
神去設立有祭司的角度.
是讓人可以透過獻祭來親近神.
建立長久的關係.
不是一次請求就完結了就再見.
而是想繼續保持聯絡.
定期出來相交.
神給了以色列人這個身份.
教會我們也有這個身份.
是由於基督耶穌的救恩和恩典.
以致我們好像以色列人.
藉著獻祭來親近神.
但現在我們更進一步.
不用每次都透過祭司的獻祭.
我們都可以來親近神.
而親近神是為了什麼呢?.
其中一個很主要的目的.
是為了使你們宣揚那召你們出黑暗.
入奇妙廣明者的美德.
所以教會存在的主要目的.
很清楚很簡單.
認識三一上帝.
追求屬靈生命的更新成長.
實踐真理.
盼望Full Church的弟兄姊妹.

$^{281}$而至到所有的教會.
我們都彼此提醒.
要小心不要不經意.
被次要的動機取代了.
返教會或少了的基本目的.
覺得好玩有趣有趣.
去交朋友交友.
雖然這些很重要.
何況我們當中有小鮮肉.
帥哥穆者Victor.
是很吸引的.
當然Pastor Tim也很棒.
但也要致於欣賞.
因為他們已經名草有主了.
是純欣賞.
當然這些不是最重要.
但願Full Church.
我們吸引人的地方.
令人引以為榮的地方.
都和其他教會一樣.
是人人都認真遵守的道.
彼此相愛謙卑.
跟隨耶穌.
上上感恩.
正如是途保羅的祈禱.
在哥羅塞書.
我們為你們祈禱的時候.
因為聽見你們對耶穌基督的信心.
並對眾聖徒的愛心.
除了向神表達感恩.
讓人有機會表達感恩.
是一件美事.
不過如果想獻祭的話.
也不是想獻就獻.
是有規矩的.
其中一個基本要求.
就是要拿一些有分量的祭物來獻.
不是把石頭混在沙裡.
是要用羊.
相對貴重的祭牲.

$^{321}$羊,牛.
在這次預備講章的過程中.
其實我遇到一些很棘手的地方.
不知道大家有沒有留意到.
在平安祭的祭物裡.
只有A餐和B餐.
剛才我們讀到的經文三章一到五節.
是講牛的.
然後繼續讀下去的時候.
是有羊的.
然後就沒有了.
就不像其他自願性的繁祭.
或者強制性的贖罪祭.
除了牛.
其他羊.
其實沒有什麼特別的.
自願性的繁祭.
或者強制性的贖罪祭.
除了牛和羊.
還有其他便宜的選擇.
例如攀溝,搓甲.
或者更便宜的細麵團.
令到經濟有能力.
即是有限的人.
都可以參與敬拜去獻祭.
正所謂.
律法面前人人平等.
但如果去到感恩祭這個位置.
窮人又如何呢.
當然是沒有份.
我質疑這個問題.
我上網查過.
現在在美國.
一隻活羊的價格.
大概是八多元美金.
即是港幣的千多元.
相信有些會便宜一點.
現在有些教會.
一直都有.
不是現在才有教會.

$^{361}$有些教會在奉獻峰那裡.
給人入錢或者支票.
或者可以去取得.
寫上自己的資料.
然後收取奉獻的回調.
他們有些格子給人取得.
例如償費,慈惠.
有一項叫做感恩.
有這個選項都很有意思.
奉獻的人可以來取得這個格子.
表示這一筆錢.
是用來向神感恩.
是很有意思.
不過通常.
雖然最後都是撥出償費.
但其實.
雖然最後都是撥出償費.
但都OK.
都是沒有什麼特別用途.
不過都是一種心意.
在我和神之間.
都是對教會的心意.
假如教會在奉獻峰那裡著名.
感恩這個奉獻.
是有免費的.
銀碼是由一千元起跳.
有什麼效果呢.
相信John,潘Sir,Millie.
或者珊珊,所有目者.
都很快收到訊息.
質疑這個奉獻的機制將人分化.
現在是否歧視窮人呢.
喂,是不是.
不是呀.
是不是歧視醫人呢.
可能是吧.
還是Full Church.
財政告急.
巧立名目.
想推高奉獻呢.

$^{401}$是什麼玩法呢.
我們可能比較一下其他的獻祭.
或者可以找到一些線索.
掃祭.
如果看第三章之前.
平安祭之前的掃祭.
就和平安祭.
或者感恩祭一樣.
都是自願獻的.
即是說掃祭所獻的祭物.
是可以獻也不可以不獻.
但他指定用的祭物不是肉.
是要指定用小麵團.
相對會便宜一點.
不過有一個形容.
在《斷經文》第二章裡.
有一個形容很奇特的.
在第二章十節裡說.
在獻給耶和華的火祭之中.
這是自成的.
即是很Holy.
Holy of the Holy.
有什麼這麼厲害呢.
有猶太的拉比認為.
其中有猶太的拉比認為.
因為對於窮人來說.
獻上這些小麵的份量.
相等於一天的食糧的份量.
對於餐飲餐穩餐食餐餐清的人.
奉獻的代價一點都不輕.
所以神特別看重窮人所獻上.
何況還沒算.
要同時間獻上的乳香.
算起來就更加不得了.
乳香和黃金同等的重量.
價值相當相同.
是很貴的.
如果按照《釋經》的進路.
去看的時候.
感恩祭的祭物羊和牛都很貴重.

$^{441}$但窮人獻上的話.
就更加顯出份量極重極其寶貴.
如果要相對窮人.
經濟不充裕的人.
要拿一隻羊來奉獻的話.
確有難度.
雖然奉獻的人可以吃一部分的祭物.
自己的羊自己吃.
但的確很重.
剛才有張圖片.
是在星期四急事要上水.
在很匆忙跑去的士的時候.
突然看到一個奇景.
可能有些人也見過.
香港有幾十隻羊.
這不是蘇格蘭.
這裡是上水.
有條河很浪漫.
很棒.
還有一幅.
我沒有視角.
不會左右.
對我來說是一個很震撼的畫面.
問起我親友.
原來是一個年輕人去放羊.
是他養的.
他貪得意就托朋友去問羊的主人.
問他如果買一隻活羊多少錢.
大家猜猜多少錢.
他說要開價三千元.
好像貴了一點.
如果出去吃全羊宴.
可能一千多元就夠了.
當然他是活羊.
又帥又乾淨.
看起來又沒有什麼味道.
假設三千元要獻.
買到一隻羊.
供應很有限.
要獻也是很重的.

$^{481}$雖然自己可以吃.
但自己平時也不捨得洗和吃.
迪士尼開了這麼多年都沒有去過.
有時收到請帖.
在五星級酒店擺酒.
其實很想出席婚宴.
最重要是開心.
突然開心也很開心.
總之開心.
但自從放寬了防疫政策之後.
馬上連環連續報復式旅行之後.
現在手頭比較緊.
除非你好像啟威牧者那麼有實力.
響應披肩的時候.
在議席上.
很快就要開始了.
隨便在名下的物業拿一個出來.
也面不改容.
不是.
是不是.
不是.
兩個才是.
現在美國的活羊價格上網查過.
一隻牛大概是美金九百元到三千元.
港幣兩萬多三萬元.
是很貴的.
如果以色列人本身有錢.
想獻一隻牛也可以.
但也不可以經常獻.
如果那時的財政真的很差.
被人懲罰也很頭痕.
有沒有人試過.
不敢說.
明白.
即使負擔不到那隻羊.
也可以努力儲錢.
一年不夠就兩年.
兩年不夠就三年.
三年不夠就五年.
一天儲一元.

$^{521}$明軍也是每天一元.
可能不用.
九毫就夠了.
應該可以達到目標.
所以當我預備經民的時候.
來到這裡.
我覺得銷售員說話也有道理.
從這個角度去看.
更加顯出.
獻祭者的真心誠意.
很重視神.
相信在神的眼中.
是很寶貴的.
你說不用這些獻祭的形式去表達感恩.
可不可以呢.
可以的.
詩人說.
詩篇第七節.
來歌唱.
用琴.
來向我們的神歌頌.
譬如彈爾利一樣.
去秘魯巴比倫和其他離散者一樣.
都沒有辦法再回到耶路撒冷.
去聖殿獻祭.
而去到波斯帝國統治的年間.
有一段時間都不能公開祈禱.
就在客房中.
去找空間祈禱.
用口頭或祈禱的形式去感恩.
都是非常有價值的.
但如果條件許可的話.
用實際的行動去表達感恩.
怎樣都會有些不同.
何況身體是最誠實的.
我們不去行山.
不去打羽毛球.
都不知道自己身體好不好.
主耶穌基督都說過.
我們很熟悉的經文.

$^{561}$你們的財寶在哪裡.
你的心都在那裡.
外在的行為.
可以流露內心的真正想法.
實體敬拜或網上敬拜.
怎樣都會有些分別.
我們去日本吃壽司.
怎樣都會和去旺角吃壽司.
都是舊如山.
但體驗和感受都可以很不同.
當然如果有限制或其他原因.
未能出席實體崇拜.
是可以理解.
但相對住得很遠.
或者很忙.
或者很多責任在身的時候.
依然優先去參與實體崇拜親近神.
那個意義在神的眼中就很不同.
感恩神興起很多弟兄姊妹和牧者.
每次都來付出很多資源和時間.
讓人去敬拜神親近神.
詩篇50篇這樣說.
凡以感恩獻祭的就是榮耀我.
那按正路而行的.
我必是他得著神的救恩.
願意我們繼續用不同的形式.
我們說到重價很容易就會想到錢.
其實怎樣才算是重價.
都未必經常用錢去衡量.
人人都會有不同的定義.
或者對每個人的情況.
重價的定義都可能很不同.
有些人可能覺得錢不是問題.
就像曾祭偉剛剛說的實體崇拜.
又或者可以像潘Sir的建議.
可以考慮不去日本旅行.
越南或其他地方.
先奉獻女婿.
如果覺得太激烈.
不如提早一兩個站下車.

$^{601}$走遠一點.
省下車費來奉獻也可以.
又是披肩.
直立式廣告.
奉獻可以給不同的單位.
Full Church, ACM, Milk and Honey.
還有孫教士.
Joanne Milker.
好像Joanne今天回來了.
或者是你所屬的教會.
重價最貴重的是什麼呢.
莫拘於聖潔的生命.
大家同意嗎.
我們的生命.
相信你都會同意.
生活的現行.
和耶穌基督的門徒身份要匹配.
將自己大大小小各方面呈獻給神.
你覺得這樣.
神是否會收貨呢.
我和我太太在生日.
情人節.
聖誕節.
每一年周年的日子.
都會和太太說.
我就是最棒的禮物.
間中她都會給面子笑一下.
今天她坐在這裡.
小心一點說話.
我們通常都很少在特別日子.
很隆重.
很刻意地去慶祝.
最多都是外出吃頓飯.
我記得幾年前.
都很棒.
去IKEA咖啡店吃龍蝦.
都很開心.
最豪那次.
應該是十幾年前.
那次是一個結婚周年.

$^{641}$我去了半島.
不是半島茶餐廳.
是半島酒店的法國餐廳吃晚餐.
整個套餐.
特價兩日一夜.
是套房.
很甜蜜.
英文老師是否很甜蜜.
都用了一整筆錢.
一整萬元.
不過很棒.
用錢.
我現在都念念不忘.
覺得最棒的是恆溫泳池.
念念不忘.
不過希望有機會.
人生盡頭之前可以再去.
大家猜一下程太對這個安排.
有什麼反應呢.
謝謝你.
聽完你講.
很棒.
吃飯.
似之以備.
他說你落重本.
無非是想補妃.
因為你做了一些對我不住的事.
不是你想的那些.
吃飯.
氣氛突然沉了.
我都借了.
他說我不是想要這些.
這些問過他才可以講.
拿了批准.
浪費了錢.
本身已經不多.
雖然他一邊說我.
一邊吃著東西.
說食物好吃.
我心想其實都不是完全沒有價值.

$^{681}$無論如何.
其實他認真的.
他想要我改變.
其實我覺得這個要求都合理.
我是智衰的.
如果我真的重視他.
就會改變.
為什麼他的要求不合理呢.
不是說用物質就一定收貨.
我預備講章的時候.
都再一次問他.
我表現如何.
他說收貨的.
大家不用太擔心我.
耶穌基督說.
可喜愛憐憫.
不喜愛制止.
是的弟兄姊妹.
我們都不要錯估本無道志.
願聖靈繼續根深我們.
用聖潔的生命呈獻給主.
當然我們奉獻禮物去奉獻.
如果不是為了補妃淑罪.
而純粹純粹純粹感恩.
要給這麼厚的禮.
不知道你會不會想.
是不是有好處呢.
就好像現在的超市.
百貨公司.
當然通常有好處.
買到便宜的東西.
正呀.
但顧名思義.
所以大家的位置不同.
所以如果我領受了恩惠.
送禮還想有回報的話.
似乎就不太合理.
即使不用獻.
不,必須強制性去獻.
都不會有什麼後果.

$^{721}$譬如神的應許在生命記.
他說:以色列人要聽.
要遵守緊守進行神的誡命律例.
使你可以在流瀨與墓之地得以享福.
正如耶和華說:你們祖的神所應許你的.
所謂要求,吩咐.
緊守進行神的誡命律例.
是不包括自願性.
既然你說optional.
那跟足是optional.
所以不獻應該也不會影響神給我的福氣.
還有另一個特色.
原來獻祭的人和祭司都會一起分享祭物.
剛才看到的經文除了三章三到四節.
其中羊的內臟脂肪腎要獻上.
要燒盡之外.
其他位置都可以被人享用.
在七章三十到三十四節.
今天我們沒有讀到這段經文.
但主要是說奉獻.
要去燒的程序的祭司.
羊的胸部就可以送給祭司.
是不是好像調轉了.
總之隨便一樣.
右腳就給祭司分.
其他沒有負責的祭司.
總之大家前前後後.
在前線或後排支持的那班祭司.
都可以一起去享用祭物.
羊不計頭皮內臟的重量.
分給祭司.
剩下的肉和骨.
如果計重量大約是兩到三成.
獻祭者只剩下七成左右.
付出都不少.
平安祭本身起碼內文.
有豐富的意思.
其中一個意思是分享和團繼.
意思是神和人之間的關係.
可以延伸到人與人之間的關係.

$^{761}$大家的關係可以很和諧.
而會幕 即獻祭的門口.
英文的翻譯是Tent of Meeting.
即會面的地方.
在《出海及記》29章說.
在會幕的門口.
即神這樣說.
我爺和華要在那裡和你們相會.
而剛才我讀到的經文在三章一到二節說.
奉獻的人在會幕門口.
親自宰殺那隻牛或羊.
經文說要獻在爺和華的面前.
在神的面前.
但不是親眼見到神.
但經文說要獻在爺和華的面前.
即在會幕的門口獻祭.
即等於與爺和華見面.
雖然看不見神.
但這裡就是與神相交的地方.
就是憑信心去經歷神的同在.
獻祭的目的很簡單.
很純粹.
只是想親近神.
如果這些都算是好處的話.
那真的可以視為一個感恩祭.
當祭司.
我們假設祭司去到獻祭的現場.
祭司見到遠處有人牽羊過來.
循例都會問.
有什麼可以幫到你.
今次有什麼想獻.
因為要問清楚.
因為祭司要看這個人是獻哪種祭.
就要用獻祭的程序去處理那些祭物.
不可以搞錯.
今晚就輪到祭司阿聰當值.
有個罪名牌就知道他叫施聰.
有位女仙知.
阿R B 走過來.
什麼事不是說到嗎.

$^{801}$問阿R B.
今天你來木土區當值嗎.
黎憲濟.
看你樣子沒有上次那麼愁.
那次被祭司阿迪誤會了你喝醉酒.
原來你正在祈禱.
只是一場誤會.
他幫你祈禱之後.
你好像先後生了兩個兒子.
亡神應允禱告.
真是恭喜你.
話說回來.
那邊的小女孩是不是你的女兒.
沒有聽過你講過.
和Metris的大女兒玩的那個呢.
對不起 看錯了.
原來是女仙知清心.
可能我最近睡眠不足.
看錯了.
阿R B 說.
先不要說那麼多.
今天我來獻祭的.
獻感恩祭 很感恩.
不分享不唱快.
上星期和幾位女仙知好姊妹.
一起去了台灣退休.
Grace 阿廷 清心 Amy.
很開心去親近神.
還在機場抽獎.
中了旅遊消費獎5000元.
可以幫補一下旅費.
阿聰回應.
哇 很感恩.
Amen.
不如先不要說那麼多.
獻祭吧.
今晚我想吃火鍋.
有黑棗羊腸.
獻祭是可以和人有互動.
分享祭物.

$^{841}$可謂一種商交.
小組組員除了每個月的定期開小組時間.
如果平時能夠分享生活的點滴.
隨時有需要去請求代禱.
感恩去分享.
剛才說寫簿仔那些.
不一定要全部拿出來說.
覺得舒服自在的就去分享.
生命有一種互動的交流.
關係就是這樣去建立.
建立關係和信任.
需要大家有來有往.
Presence 和在都很基本很重要.
之前天氣比較清涼的時候.
我所屬的小組就約了他們去西九藝術中心.
去野餐去Picnic.
一次是週末的晚上.
一次是平日的晚上.
而這兩次的野餐Picnic.
我覺得很棒.
在海邊的草地坐下.
播音樂.
沒有什麼program.
當然有些增長神經知識的program.
平時都很重要.
吃喝喝喝聽聽其他組員說話.
覺得是需要回應就回應.
覺得很自在.
關係就是在這些moment裡面一點一滴.
這樣來累積建立.
肢體希望可以做到朋友.
或者Modern朋友的肢體.
是很棒的.
我還記得有組員自己焗了麵包來吃.
不是他自食的不是自私的.
自己吃也不是自私的.
總之他自己拿來.
特意和我們分享.
嘩好吃到怎樣呢.
我現在真的念念不忘.

$^{881}$這樣說不是想推他.
暗地裡去passive active那些.
去想他會做.
不是的.
又是一件微事.
好吃到怎樣呢.
總之是很棒的.
其實到了有需要的時候.
不單止是Picnic.
到了有需要的時候.
在人生一些比較重要的moment.
感恩.
可以來到一起陪伴同行.
一起去等候神.
去經歷神.
或者這些就是去分享的回報.
各位我們既然經歷過主的恩.
主恩和主愛.
可能都不妨去考慮.
用多一些的心思.
或者奉獻給平時.
更有份量的禮物.
來給神.
不知道這樣有沒有得想呢.
當獻祭者想拿一隻羊或牛去獻祭的時候.
家人或鄰居都可能會給意見.
甚至不太熟悉的村民都會留言.
認為你這樣做不太善用資源.
反正感恩祭都不是強制性的.
不如留在更加有用的地方.
日後做一些基金.
教育基金.
還是留下來做其他東西.
更實際.
其實家人或其他人的意見.
聽一下都很重要.
始終有時影響的不是一個人的事.
資源是應該用得其所.
在六月份的時候.
我參加了一個.

$^{921}$牧者的按摩典禮.
我一直聽他.
到述字的環節.
我聽他分享.
很感動.
其中一部分我想和大家分享.
他說回憶當初去報道神學.
去奉獻自己.
去走全職侍奉的路的時候.
他特意召開了一個家庭會議.
期間跟兒子說.
爸爸現在想讀神學.
不過你不要太努力成功靠父幹.
日後如果買樓要首期的話.
就要靠自己搞定.
兒子怎樣看.
兒子就.
根據他說的就說OK.
OK.
我相信是.
家人很敬虔他們.
是一種很大的祝福.
支持.
可能兒子心想.
如果可以兩樣都有就好了.
我也是這樣說.
但要兩者取捨的時候.
怎樣將最好的給神呢.
奉獻有什麼回報呢.
其實怎樣指退稅優惠這麼簡單呢.
奉獻可以回應神的恩情.
令關係更加近一點.
深一點.
經歷那種關係的和諧滿足.
但奉獻是要付出的.
有時不是那麼浪漫.
即使不用拿出生命.
但也要交出心.
這樣做值不值得呢.
真拿達香膏貴重無比.

$^{961}$送給耶穌值不值得呢.
有什麼難阻我們去奉獻呢.
二十萬的speaker.
如果有用又有條件這樣用.
純粹焦點是向主表達心意.
如果是一件美事.
這樣做又何妨呢.
大熱天時拿著蛋糕趕去和組員慶生.
有沒有需要用幾十元搭的士呢.
是不是值得呢.
條數怎樣計呢.
計不計得點呢.
耶穌配得一切的奉獻和我們的心意.
將心歸主.
奉獻給耶穌跟隨耶穌是不能輸的.
正如譚嗣同說.
大哥你沒有輸.
願主閱立我們的奉獻.
願我們的敬拜.
包括今天的信息.
和我們心裡所想的.
所幻想的.
向主的回應.
都能夠成為一份禮物.
願主閱立.
我們一起祈禱.
主耶穌基督.
你配得一切榮耀 重讚 奉獻.
因為你是全地的主.
滿有恩惠 也很愛我們.
願主你的愛繼續透過聖靈.
來激勵我們.
打開打動我們的心扉.
愛主更深.
緊緊跟隨主.
奉主耶穌基督的名祈禱.
Amen.
\newpage



\section{哥林多前書 12:1-2-14-18-20230819}
\label{sec:xHhMd2gjsxw}
\textbf{【網上聖餐崇拜】顛覆教會恩賜的想像-奉獻為核心|哥林多前書12\_1-2,14-18|20230819 [xHhMd2gjsxw]}
\newline
\newline
連結: \href{https://youtube.com/watch?v=xHhMd2gjsxw}{\texttt{ https://youtube.com/watch?v=xHhMd2gjsxw}} ~~~~ 語音日期: 2023-08-19 
\newline
\newline
\hyperref[sec:LgSzajW7sqo]{\small{< < < PREV SERMON < < <}}
~
\hyperref[sec:index_chronic]{\small{[返順時目]}}
~
\hyperref[sec:index_scriptual]{\small{[返順卷目]}}
~
\hyperref[sec:br4_eJEULQ0]{\small{> > > NEXT SERMON > > >}}
\newline
\newline
哥林多前書 12:1-2-14-18-20230819
\newline
\begin{longtable}{cl}
\hline
\hline
章節 & 經文 (和合本修訂版)\\
\hline
12:1 & \begin{tabularx}{0.7\textwidth}{X} 弟兄們,關於屬靈的恩賜,我不願意你們不明白。 \end{tabularx} \\ \\ \relax
12:2 & \begin{tabularx}{0.7\textwidth}{X} 你們知道,你們作外邦人的時候,隨事被引誘,受了迷惑去拜不會出聲的偶像。 \end{tabularx} \\ \\ \relax
12:3 & \begin{tabularx}{0.7\textwidth}{X} 所以,我要你們知道,被神的靈感動的,沒有人會說「耶穌該受詛咒」;若不是被聖靈感動的,也沒有人能說「耶穌是主」。 \end{tabularx} \\ \\ \relax
12:4 & \begin{tabularx}{0.7\textwidth}{X} 恩賜有許多種,卻是同一位聖靈所賜。 \end{tabularx} \\ \\ \relax
12:5 & \begin{tabularx}{0.7\textwidth}{X} 事奉有許多種,卻是事奉同一位主。 \end{tabularx} \\ \\ \relax
12:6 & \begin{tabularx}{0.7\textwidth}{X} 工作有許多種,卻是同一位神在萬人中運行萬事。 \end{tabularx} \\ \\ \relax
12:7 & \begin{tabularx}{0.7\textwidth}{X} 聖靈彰顯在各人身上,是要使人得益處。 \end{tabularx} \\ \\ \relax
12:8 & \begin{tabularx}{0.7\textwidth}{X} 有人藉著聖靈領受智慧的言語;有人也靠著同一位聖靈領受知識的言語; \end{tabularx} \\ \\ \relax
12:9 & \begin{tabularx}{0.7\textwidth}{X} 又有人由同一位聖靈領受信心;還有人由同一位聖靈領受醫病的恩賜; \end{tabularx} \\ \\ \relax
12:10 & \begin{tabularx}{0.7\textwidth}{X} 又有人能行異能,又有人能作先知,又有人能辨別諸靈,又有人能說方言,又有人能翻方言。 \end{tabularx} \\ \\ \relax
12:11 & \begin{tabularx}{0.7\textwidth}{X} 這一切都是由惟一的、同一位聖靈所運行,隨著自己的旨意分給各人的。 \end{tabularx} \\ \\ \relax
12:12 & \begin{tabularx}{0.7\textwidth}{X} 就如身體是一個,卻有許多肢體,身體的肢體雖多,仍是一個身體;基督也是這樣。 \end{tabularx} \\ \\ \relax
12:13 & \begin{tabularx}{0.7\textwidth}{X} 我們無論是猶太人是希臘人,是為奴的是自主的,都從一位聖靈受洗成了一個身體,並且共享這位聖靈。 \end{tabularx} \\ \\ \relax
12:14 & \begin{tabularx}{0.7\textwidth}{X} 身體原不只是一個肢體,而是許多肢體。 \end{tabularx} \\ \\ \relax
12:15 & \begin{tabularx}{0.7\textwidth}{X} 假如腳說:「我不是手,所以不屬於身體」,它不能因此就不屬於身體。 \end{tabularx} \\ \\ \relax
12:16 & \begin{tabularx}{0.7\textwidth}{X} 假如耳朵說:「我不是眼睛,所以不屬於身體」,它也不能因此就不屬於身體。 \end{tabularx} \\ \\ \relax
12:17 & \begin{tabularx}{0.7\textwidth}{X} 假如全身是眼睛,聽覺在哪裡呢?假如全身是耳朵,嗅覺在哪裡呢? \end{tabularx} \\ \\ \relax
12:18 & \begin{tabularx}{0.7\textwidth}{X} 但現在神隨自己的意思把肢體一一安置在身體上了。 \end{tabularx} \\ \\ \relax
12:19 & \begin{tabularx}{0.7\textwidth}{X} 假如全都是一個肢體,身體在哪裡呢? \end{tabularx} \\ \\ \relax
12:20 & \begin{tabularx}{0.7\textwidth}{X} 但現在肢體雖多,身體還是一個。 \end{tabularx} \\ \\ \relax
12:21 & \begin{tabularx}{0.7\textwidth}{X} 眼睛不能對手說:「我用不著你。」頭也不能對腳說:「我用不著你。」 \end{tabularx} \\ \\ \relax
12:22 & \begin{tabularx}{0.7\textwidth}{X} 不但如此,身上的肢體,人以為軟弱的,更是不可缺少的; \end{tabularx} \\ \\ \relax
12:23 & \begin{tabularx}{0.7\textwidth}{X} 身上的肢體,我們認為不體面的,越發給它加上體面;我們不雅觀的,越發裝飾得雅觀。 \end{tabularx} \\ \\ \relax
12:24 & \begin{tabularx}{0.7\textwidth}{X} 我們雅觀的肢體自然用不著裝飾;但神配搭這身子,把加倍的體面給那有缺欠的肢體, \end{tabularx} \\ \\ \relax
12:25 & \begin{tabularx}{0.7\textwidth}{X} 免得身體不協調,總要肢體彼此照顧。 \end{tabularx} \\ \\ \relax
12:26 & \begin{tabularx}{0.7\textwidth}{X} 假如一個肢體受苦,所有的肢體就一同受苦;假如一個肢體得光榮,所有的肢體就一同快樂。 \end{tabularx} \\ \\ \relax
12:27 & \begin{tabularx}{0.7\textwidth}{X} 你們是基督的身體,並且各自都是肢體。 \end{tabularx} \\ \\ \relax
12:28 & \begin{tabularx}{0.7\textwidth}{X} 神在教會所設立的:第一是使徒;第二是先知;第三是教師;其次是行異能的;再次是醫病的恩賜,幫助人的,治理事的,說方言的。 \end{tabularx} \\ \\ \relax
12:29 & \begin{tabularx}{0.7\textwidth}{X} 難道個個都是使徒嗎?難道個個都是先知嗎?難道個個都是教師嗎?難道個個都是行異能的嗎? \end{tabularx} \\ \\ \relax
12:30 & \begin{tabularx}{0.7\textwidth}{X} 難道個個都是有醫病的恩賜嗎?難道個個都是說方言的嗎?難道個個都是翻方言的嗎? \end{tabularx} \\ \\ \relax
12:31 & \begin{tabularx}{0.7\textwidth}{X} 你們要追求那更大的恩賜。我現今把最妙的道指示你們。 \end{tabularx} \\ \\
[1ex]
\hline
\hline
\end{longtable}
$^{1}$(英語)早安.
過去兩個月去了以色列一個月.
然後去了英國一個月.
在英國的時候去了不同的城市.
在不同城市裡面都會見到Folk Church的弟兄姊妹.
所以在這裡要先跟Folk Church.
現在在英國,可以站在這個地方.
再打聲招呼.
聽了很多不同在英國香港人的故事.
有很多精彩的故事.
也有很多需要守望祈禱的故事.
但是Folk Church的名字是改得對的.
它真的有Folk在不同的地方.
就算你不預期會有Folk Church的弟兄姊妹.
它都會出現.
所以再一次在這裡問候每一位離開了香港.
在不同的地方,還在堅持很多東西的弟兄姊妹.
今天我們講一個題目.
我們講一個題目是關於《少年恩賜》.
我們看看下一張PowerPoint.
我們看看那段經文.
《少年恩賜》在高雄的教會來說.
這封信來說.
其實在第十二和十四章講恩賜.
待會會切入今天的主題.
講奉獻.
因為我想不到我還可以講什麼奉獻的.
講了第七堂了.
我為下星期的港元禱告求神.
比起它還可以講.
所以我想了另一件事來講奉獻.
不過我先講恩賜.
你會明白我想講什麼.
臨前十二章開始講恩賜.
他說弟兄們論到《少年恩賜》.
我不願意令你明白.
因為你做外邦人的時候.
隨事被牽引受迷惑.
去服侍那些啞巴的偶像.
這是你們知道的.

$^{41}$所以我告訴你們.
被神的靈感動的那些.
沒有人說耶穌可以就座.
如果你不是被神的靈感動.
也沒有人說耶穌是主.
其實如果講《少年恩賜》的話.
正常地講《少年恩賜》.
如果你看任何一本還在講《少年恩賜》的書.
很早期的話.
其實我們很少講這三節聖經.
這三節聖經不會入流成為.
講《少年恩賜》裡我們會談論的東西.
其實是九不搭八的.
這三節聖經是幾九不搭八的.
要理解為什麼講《少年恩賜》的時候.
要講這幾節聖經呢.
你做外邦人的時候.
又要服侍那些啞巴偶像.
你講《少年恩賜》的時候.
你會說我現在講《少年恩賜》.
免得你拜黃大仙.
基本上我們不會這樣理解.
到今時今日.
所以問題要處理一下.
其實《少年恩賜》.
尤其是保羅去講那麼多教會的時候.
由十二章和十四章講.
為什麼他要講這三節聖經呢.
這是我們要想的課題.
但要下到這個理解的時候.
我們先講多一點.
今天會講多一點.
不會講那麼多的.
我們先看下一張.
其實要處理的課題是什麼呢.
其實十二章和十四章是講《少年恩賜》的.
如果你用手機去看.
大部分講《少年恩賜》的經文.
都出現在十二章和十四章.
但你會覺得很奇怪.

$^{81}$為什麼十三章中間夾了十三章呢.
這個是要處理的.
因為保羅是色陰無神.
他放了十三章中間.
除非我們這樣理解.
如果是色陰無神.
寫完兩章《少年恩賜》的時候.
他講的那些外篇.
《愛是恆久忍得你耐》的那些恩賜.
就是把你放進去.
如果不是的話.
其實合理的話.
應該是十二章和十四章講《少年恩賜》.
然後十三章才講.
《愛是恆久忍耐又有恩賜》的經文.
應該是這樣吧.
應該這樣理解.
但問題是.
十二章,十三章,十四章.
其實是一併去講恩賜的課題.
我們應該這樣理解.
我不講太多.
要講太多的話要講很長.
所以暫時相信.
十二章,十三章,十四章.
其實是一併去講恩賜.
所以中間他說.
現在要把更厲害的東西講給大家聽.
就是愛.
所以應該來講.
恩賜和愛是有關係的.
除非你結婚的時候用愛篇.
《愛是恆久忍耐》.
現在很少人用.
你用過的我不得罪你.
我不是得罪你.
現在很少人用.
之前會用愛篇.
開始大家對聖經有些理解和熟悉的時候.
其實愛篇.

$^{121}$我們只用頭幾節.
你不會叫焚身.
什麼的.
你不會講很多其他東西.
愛篇來來去去只用幾句.
其實和整個十三章都用不完.
你不會有牧師在正婚講道的時候.
講十三章講完整章.
所以如果我們詳細理解的話.
十二,十三,十四章應該是有關連的.
我們這樣理解.
所以保羅要放十三章下去.
其實是在補充他想講恩賜這個課題.
一個很重要的東西.
待會最後才講.
不過起碼要有這個理解.
將來結婚的話.
不要無緣無故將愛篇在婚禮上用.
除非你想講恩賜.
結婚講恩賜是很奇怪的.
我不知道講什麼恩賜.
可以講的.
所以要有這個理解.
它是有作用的.
我們再下一個.
再下一個.
接著回到講經文.
最奇怪的地方.
什麼叫做我不願意你不明白.
你做外邦人的時候.
就被牽引,迷惑,服侍亞巴偶像.
我們試試這樣理解.
其實保羅想什麼.
其實恩賜這件事.
其實是講什麼.
恩賜是講你做得很好的事.
一定是.
你唱歌好聽.
就算你說我唱歌比誰都好.
或是全民造星快要拿冠軍.

$^{161}$你也不敢說我很厲害.
我們也會謙卑地說.
這些神秘的聲音.
恩賜是屬於上主的.
我們基督徒一定會假裝是上帝恩典的.
你很有行政的力量.
一定會引誘你.
你真的有行政恩賜.
你怎樣都加個恩賜下去.
你做得厲害的話.
在基督教界就要加個恩賜.
就成功了.
整件事就熟齡化了.
光環大一大.
你可以想像.
由神做男做女開始.
祂把形象照著祂的形象去做男做女.
其實上帝給人恩賜不是在新約講的.
應該在舊約也講了很多.
創世紀.
起碼阿當最厲害的恩賜應該是改名.
所有昆蟲植物動物.
名字都是他改的.
起碼你改不了吧.
你能改到十幾萬動物昆蟲的名字嗎.
你不會想到吧.
哎呀.
很難吧.
我相信你發十幾萬個音也很難.
所以他一定有改名恩賜.
我猜.
這應該是他分享了上帝豐富的一部分成為恩賜.
所以在整個舊聖經裡.
人有能力.
人有不同的才幹.
去做不同的事.
其實是不複雜不難理解的.
譬如聖殿的祭物.
無論是初一及二或是所羅門.
都是很多聖靈的工作幫助那個人去做的.

$^{201}$這不是新約臨前獨有的英文.
他應該在舊約裡.
每個人都有自己獨特的恩賜.
起碼不能不敢相信.
除非你說不是.
先不要指他.
對著他說你真的沒有恩賜.
那你很糟糕.
你什麼都不擅長.
你存在就存在.
沒有工作.
這不行的.
基本上在教教或神學的觀念裡.
神從來不會做一個垃圾出來.
我們應該不會這樣相信.
每個人都有些用處.
你覺得他的用處好不好是另一回事.
起碼你覺得每個人都有些用處.
有些厲害的東西.
這個理解不難理解.
其實在整個舊約裡.
但整個舊約裡.
除了說人按著上帝的豐富形象去做.
他做了很多很厲害的東西之外.
其實你發現原來在舊約裡.
都說很多人很厲害的東西.
但厲害得來是出事的東西.
譬如最簡單.
人們走在一起.
就要做個塔.
巴別塔.
雖然人們厲害是有問題的.
厲害得太多.
但他有恩賜有才幹.
他可以做個巴別塔.
到天頂上可以直接與神交易.
正如有很多人用上帝給他的恩賜.
用錯了的經驗.
很多故事放在舊聖經裡出現.
我不詳細說了.

$^{241}$所以這段經文應該這樣理解.
其實對於保羅社比哥倫多教的人來說.
每一個屬於上帝的生命的人.
都應該有些事情給上帝做.
不過很多時候.
我們不是做一些事情去榮耀上帝.
我們只不過是做一些事情去榮耀自己.
基督教更壞.
普通外面的人很厲害.
但卻是很不懂事的.
你猜Evon Musk贏了還是Facebook贏了.
他當然不打了.
打吧.
我們都會用一些事情去做一些很古怪的事情.
裝有型.
外面是很名正言順.
你厲害就厲害.
但基督教是墮落一點的.
即使你厲害也好.
你已經很厲害.
你怎樣也會說一句.
這一切都是神的恩典.
(笑聲).
我們看下一張是很虛偽的.
我們看下一張.
我想表達的那件事是什麼呢?.
我們每個人都很漂亮.
每個人都很有恩賜.
你怎樣也會有些恩賜.
你一定會有些恩賜.
就算很窄的人在家裡打遊戲.
他打遊戲的恩賜都很強.
他控制的就不是我控制到的.
總之每個人都有.
但問題是用了那些東西出來是為了什麼?.
保羅關注的不是教會裡的人有沒有恩賜的問題.
保羅關注的是每個人都有恩賜的時候.
就是出事的時候.
所以他不想你不明白.
不知道你以前做外邦人的時候.

$^{281}$你都有很多恩賜.
有很多厲害的東西.
但你們全部搞錯了.
所以保羅在這裡要表達的一點是什麼呢?.
原來我們每個人都覺得自己有些東西厲害.
有些東西才幹.
但其實很多時候我們只不過是在做什麼呢?.
我們只不過是在做一些在宗教倫理道德裡.
別人能接受我們厲害的東西.
我們只不過是在表演一些東西.
告訴別人我有多厲害.
但我戴著一個很虛偽的帽子告訴自己.
其實這一切都是上帝的恩典.
這個才是保羅說恩賜的時候.
他心裡關注的東西.
要解釋一下為什麼要說愛.
愛的中間夾住了.
愛就是要將人的恩賜.
由他很厲害的部分.
拉下來一點.
或者混合一點愛的東西.
以至這個恩賜藉著愛的時候.
可以讓上帝用得到.
如果這樣理解的時候.
其實什麼是恩賜呢?.
或者我們要理解的時候.
其實恩賜要說什麼內容呢?.
早二十多年了.
現在不知道還有沒有.
你知道有些所謂恩賜調查表嗎?.
我不想得罪那些.
得罪的人太多了.
那些恩賜調查書,表那些.
我懶得開名字.
有些教會真的做了全教會.
所有人都做恩賜調查表.
你知道邏輯是什麼嗎?.
就是做完之後.
你有信心的恩賜.
你有管理的恩賜.

$^{321}$你有傳科的恩賜.
他將所有聖經裡很無聊的.
對不起.
凡是說有恩賜的.
有category的.
都截圖出來.
大約四十幾種五十幾種.
你明白大約是這樣.
四十幾種五十幾種.
一看就知道是傻人才做的.
我們這裡幾百人.
真的只有四十五十種恩賜?.
不會吧?.
你當上帝是假的嗎?.
上帝你的豐富只有四十幾款.
你頂多五十款就給你了.
他做了那麼多千千萬萬人.
有很多不同的恩賜.
一定不會只有四十幾五十款.
這些傻事我們不要再討論.
最傻是做完之後.
你以為分了四十幾款五十幾款.
大家就會按著恩賜來服侍.
你以為是這樣嗎?.
但結果.
對不起.
可能我不知道.
我聽聞的或者我經歷過的.
真是傻的.
你做完你管理的恩賜.
有多少人去管理?.
那些做完之後沒有管理恩賜的.
他會走嗎?.
一定不會走.
幸好沒有唱過恩賜.
以前可能有些師班唱得.
他做完不會說不唱師班.
他說這是尊貴侍奉.
我要為主獻上.
最好的賢與主.

$^{361}$總之好的意思.
不要管好不好.
不是客觀的好.
我呈獻就叫好.
那還要做什麼恩賜?.
總之不要管恩賜.
我呈獻我最好的給主.
就是我現在做什麼.
都是叫最好的.
那還要做什麼?.
所以基本上.
整個恩賜調查表.
幸好最近不流行.
或者最近沒有書再做.
講這些東西.
終於香港教會有些得救的子網.
不然其實要搞什麼?.
我今天想說什麼?.
我再下一張照片.
他拋開帶著我說話.
我們再下去.
快點.
我們說這個.
你知不知道很奇怪.
這段經文是講恩賜的.
十二章的第十四節.
剛才我們說一至三節.
中間那些我不說.
因為時間關係.
我可以再說.
不過不說.
你回到十四節.
這裡再下去.
再按一張.
去到第幾節?.
第二十六節.
先看看這些.
這些基本上很面熟.
什麼不法.
不要體面的就給他體面.

$^{401}$不雅觀的就給他裝飾得雅觀一點.
我們雅觀的肢體.
自然用不著裝飾.
但神佩達的身子.
是要加倍體面給那些缺點的肢體.
這些說法.
我們經常說的說法.
這些說法我們不會放在恩賜上說.
你明不明白.
裡面講恩賜.
十二章是講恩賜.
十四章是講恩賜.
十二章下面.
由十二節開始.
殺到這一段.
無端端說.
有些人沒有一些東西.
按回十二節.
按回上面.
什麼什麼.
身體不是只有一個肢體.
有很多肢體.
譬如說我不是手.
所以不屬於身體.
他也不能不屬於身體.
假如耳朵不能跟眼睛說.
所以不屬於身體.
他也不能因此不屬於身體.
你發覺是不是很奇怪.
是不是很奇怪.
這些經文很奇怪.
其實這些經文講得這麼長氣.
是保羅刻意講的.
他在多少節.
四節到第十一節.
他講了某些恩賜.
什麼信心恩賜.
醫治恩賜.
方言恩賜.
翻方言恩賜.

$^{441}$他引述了一些.
他引述完之後.
他講恩賜就引述了.
你夠膽.
但他引述了六七款.
他引述完之後就講這裡.
這裡才是十二章裡講恩賜的核心.
眼睛不能跟手說我不屬於你.
恩賜不能屬於身體.
你沒有體面.
我要越化給你體面.
這些是我們很熟悉的經文.
這些經文講什麼.
坦白講.
對不起.
我只是說說.
譬如你唱歌很厲害.
譬如有個人唱歌很差.
假設.
你什麼時候教會唱歌.
唱到好像你這麼好聽.
你怎樣給他體面.
你明白嗎.
你會發現譬如你管理恩賜很厲害.
你的行政很強.
你看著一個管理員.
一堆東西.
你什麼時候說你沒體面.
我現在幫你.
教你有些恩賜給你.
沒有的.
教會沒有發生過這些事.
這些聲音怎樣用.
到底想講什麼.
這些經文.
我可能會這樣思考.
講教會恩賜的時候.
其實想講一件很重要的事情是什麼.
就是因為你展示你的恩賜就出事.
我先講清楚一點.

$^{481}$這樣講不是最理最好.
這樣講直接一點.
我的意思是.
如果一個地方裡面.
每一個人都在強調.
自己那些很厲害的東西的時候.
其實就會出事.
其實講恩賜要講什麼.
我切入主題了.
其實講恩賜是講奉獻的.
我認真的時候你笑啊.
你笑啊.
你明白我想講的嗎.
如果每個地方裡面.
每個人講恩賜是講什麼.
純粹是講.
你有什麼厲害.
你有什麼恩賜.
好啊,我現在開一個位給你做.
教會開十萬幾個位都做不到.
沒有這些事發生過.
真正講恩賜是講什麼.
是你知道有你的恩賜可以發揮之後.
你捉對方去發揮到.
他應該要發揮他的恩賜.
當你用心.
不是去想著自己發揮自己很厲害的恩賜的時候.
然後又謙卑.
哎呀,我這本書拿了獎.
Hallelujah,讚美主.
你寫了一本書都可以Hallelujah,讚美主.
為什麼要拿了金獎才刻意說.
哎呀,我這本書拿了金獎.
但是榮耀歸給上帝.
你不需要這麼虛偽吧.
我的禮智真的很多.
基督徒要展示自己厲害的東西.
但是在教榮耀主的光環下.
又不夠膽太得罪上帝.
我們做很多古怪的事.

$^{521}$恩賜是想講什麼.
當找到你的恩賜能夠發揮的時候.
他就說請你去找一下你身旁的幾個人.
他好像沒有什麼恩賜.
他的恩賜不是很顯眼.
他的恩賜不是很厲害.
你盡力幫他就行.
所以愛,愛偏愛.
恆久忍耐,忍得你耐.
忍得你這麼久都沒有恩賜.
看不順眼.
但是我都要慢慢幫你.
找到你有你的恩賜.
所以我才不能夠眼睛和耳朵說.
你不屬於我的身體.
原來你有的.
所以這段經文他講了什麼.
他搞了一大輪的.
請你將恩賜.
不是每一個人都盡情發揮自己的恩賜.
這個叫做教會聖靈恩賜論.
他花這麼多氣力.
講了13張愛篇.
愛篇講什麼.
哇,我叫人焚身.
哇,我多偉大.
你還敬拜我,弄個銅像給我.
他說,沒有愛的話.
這些都不算什麼.
愛篇是這樣講的.
這十幾個經文.
全部一式講一樣東西.
就是要每一個人.
都有上帝的恩賜.
但每一個人都有上帝的恩賜.
發揮是什麼意思.
是因為有人願意花時間在他身上.
讓他去經歷上帝在他身上的恩賜是什麼.
在英國之旅.
不要講太多.

$^{561}$那裡有些Folk Church 頂指妹.
有一間教會是新開的.
是一間在那裡練習的教會.
我一去到的時候.
一個年紀大的女士.
因為我只講一節聖經.
一節.
那位指妹很不開心.
為什麼你講一節聖經.
我說,糟了,我講了幾節.
她說,你為什麼講一節.
完了之後,我聽回她的故事.
原來她每一次準備去讀聖經.
她很花氣力,時間,心機.
想好怎樣去講那些聖經.
我聽完之後.
我終於明白了,她不是嚴我講一節.
她是很想多講一點聖經.
怎樣表達出來.
那個年紀大的頂指妹.
她服侍的時候很用心.
我們很明白.
有一班二十多歲的人.
在崇拜前走過來.
我看著那班二十多歲的很開心.
他們在聊天.
待會你開車去那裡.
我先送我媽媽.
然後我們今晚要去那個地方服侍.
嘩,二十多歲.
二十多歲的青年人.
在教會裡遊透.
講怎樣去服侍.
講到崇拜開始,她還在講.
完了之後,跟他們聊天的時候.
她說,我想探望這裡的香港人.
我想跟他們煮飯吃.
接著我們一班青年人去她家裡.
無論她年紀大或年紀中間.
年紀小也好.

$^{601}$我都想能夠探望這個城市裡.
所有剛剛來到的香港人.
那一刻我發現一種很美的感覺.
那種很美的感覺,我很少見到.
十多年前,很多回教會的人都是消費者.
回教會後,有沒有冷氣?椅子有沒有坐好?.
東西好不好?.
東西都要好.
剛才敬拜都做得很好.
其實我想深一層.
崇拜的重要點在哪裡?.
崇拜的重要點在崇拜完之後.
人們突然之間都知道自己在做什麼.
每個弟兄姊妹都知道自己的崗位在哪裡.
整間教會大約一百人.
每一個弟兄姊妹都知道自己在發生什麼事.
年紀中間的,年紀大的,年紀小的.
她都找到自己的崗位去做.
而最開心的那幅圖畫是.
她們每一個一代一代的幾代人.
都很欣賞每一代人奉獻著,在做的事.
如果要說教會恩賜.
是當一群人的恩賜發揮得很好的時候.
其實就是時候將你覺得發揮得很好的東西.
將那些氣力和時間放在未發揮得好的人身上.
讓他發揮得好.
直到今天教會的群體.
每個人都很清楚自己在做什麼.
每個人都很清楚在這個時候.
回到這個地方或離開這個地方之後.
他很清楚自己接下來要怎麼去做.
我相信對於一些很清楚自己的恩賜發揮的人來說.
問題要問的是.
我是老死掛念自己的恩賜要發揮得好一點.
還是我將我的眼目放在一些.
好像不太顯眼,不太體面的人身上.
我用時間氣力栽培也好,什麼都好.
令到一個生命能夠起來去服侍.
我想到這一點的時候,我有點掙扎.
我掙扎是問,到底是不是真的要服侍?.

$^{641}$我們經常說做,我們經常說doing.
完結之後去哪,之後去哪.
有很多mission groups,我們要有些mission去做些事.
沒錯,這些都可能是.
但我想說到最後這一點.
其實說他不是真的要做的.
我今天在英國旅行,一家人去.
我兒子很特別.
因為我每天都見到他.
不是,平時我都不會見到他.
不過這次是每天24小時都見到他.
我經常都很掛念一件事.
姐姐有多一點什麼.
她會經歷到上帝的一些事.
她會經歷到上帝.
你知道我最掛念的是.
她們能夠經歷到上帝是我最開心的事.
姐姐有,姐姐經常都說.
我最近去了露營,暑假去了露營.
然後一班同學一起祈禱之後.
她哭得死去活來.
神跟她們說了很多話.
說什麼我都不知道,她們不會跟我說.
總之她很感動.
我為她從小到大經歷到上帝跟她說話是重要的.
我經常都信二代最重要的是.
你遇到上帝.
我兒子就很少.
不是很少,不是很少.
他沒有,我祈禱也好,不祈禱也好.
總之他都沒有.
我經常都想問他,你有沒有的.
你有沒有什麼的.
還是你的信仰很隨便.
我經常都質疑他.
有時候他的事.
他說,不要說太多.
總之他沒有什麼.
但在最後那幾天.
我突然間上帝醒了我一個觀念.

$^{681}$他說你看看你兒子.
我看他,我嚇人.
他很好.
有些事很討厭.
我經常覺得他.
那些討厭的事能夠遇到上帝是很好的.
姐姐也這樣教他.
你試試交給上帝.
他說,我不行了.
你遇到上帝就行了.
他不肯的.
我覺得,唉呀,我擔心.
所以我看著兒子的時候.
他突然說,你看.
你兒子去到不同的城市.
遇到不同的人.
大人也好,中朋友也好,小朋友也好.
大過他的小朋友也好,小過他的小朋友也好.
他說,你有沒有發現.
很多人都牽著他.
很多人都和他玩.
很快很熟絡.
他好像一個小天使一樣.
個個都突然間會勒著他,黏著他.
很想和他玩.
我開始慢慢發現.
恩賜不一定是做的.
你生成點上給了一個甚麼性情.
其實可能這個也是一個恩賜.
我兒子一出來.
那些小過他的女孩.
不知為何就去牽他的手.
然後他就瞄人家.
嗯,這樣.
我說,你下次再牽他.
他說,老套.
那女孩又這樣牽他.
他又不知為何被人牽.
那些大朋友.
臨走的時候.

$^{721}$是要給他一個心.
那些都四五十歲了.
臨走的時候很捨不得他.
有些事.
我想最後說的是.
奉獻除了錢,物質之外.
奉獻自己得到的東西.
不是說你不做你自己覺得厲害的東西.
是說你怎樣將.
別人放在你心裡.
讓教會成為.
每一個人能夠發揮的教會.
我們所說的發揮.
不一定是說做很多事.
其實每一個生命的呈現本身.
都是一種美善的彰顯與呈現.
沒錯,有些人是很奇怪,很煩,是真的.
誰不奇怪,不煩呢?.
是吧?.
在奇怪,煩的背後.
能夠找到這個人.
他精彩的地方出來.
是需要放下我們自己.
覺得自己看得對的東西.
我兒子應該星期一回香港了.
我打算找一個時間跟他說.
我兒子,爸爸你做錯了.
可能有時候說你,吵你.
說錯了.
我希望你可以原諒我.
我慢慢發現你有很多很美的東西.
是你爸爸沒有的.
我儲著勇氣說.
但發了一次.
我怕我見到他,我不知道該說什麼.
大頂之梅.
奉獻實在是什麼?.
是你覺得你自己很多東西很強,很厲害.
你可以放下.
將對方成為你生命的裡邊.

$^{761}$他有很多很厲害的東西呈現.
這個是奉獻.
這個總結是不是很適合呢?.
我一直在祈禱.
當我們來到你面前的時候.
我們去思考.
奉獻這個課題的時候.
除了給我們金錢,物質.
或者我們一些喜好可以放下之外.
更加可以將我們的恩賜放下.
都獻上.
望著我們身邊的大頂之梅.
我們很想他們有很多很精彩的東西出現.
很想和他們在法國一起經歷.
他們生命裡各樣各樣的東西.
以致叫我們裡邊高傲的眼睛.
自恃自己的能力.
各樣的恩賜.
那種心懷能夠放下.
學習.
上帝的豐富和榮美.
永遠超乎我們的所想所求.
以致我們可以有一天來到你面前的時候.
以見到對方他精彩的生命的呈現.
成為我生命裡邊最掛念的事情.
來到Full Church的頂之梅.
我們交在你手中.
我們每一個人的生命都不一樣.
每一個人和另一個人絕不一樣.
都是一模一樣倒模出來的.
就因為這個緣故.
讓我們放下自己更多.
去成就別人更多.
這是一個看似要講很多效率.
看似要讓人去尊敬自己.
成為自己滿足的全圓的世代.
求天父你讓我們走另一條路.
就好像今天聖餐的時間一樣.
你釘死十字架.
你用一班無用的愚夫小民.

$^{801}$你不真摯去傳講福音.
你用一班無知的人.
無學識的人.
去傳這個福音.
但這班人竟然可以顛覆天下.
福音講傳.
求天父你幫助我們.
待會是聖餐時間.
你再一次思念.
你的放下你的自己.
多謝天父你聽了我們的祈禱.
附在心裡寶貴命強.
\newpage



\section{馬可福音 14:3-9-20230826}
\label{sec:br4_eJEULQ0}
\textbf{【網上崇拜】真.拿.達|馬可福音14\_3-9|20230826 [br4-eJEULQ0]}
\newline
\newline
連結: \href{https://youtube.com/watch?v=br4-eJEULQ0}{\texttt{ https://youtube.com/watch?v=br4-eJEULQ0}} ~~~~ 語音日期: 2023-08-26 
\newline
\newline
\hyperref[sec:xHhMd2gjsxw]{\small{< < < PREV SERMON < < <}}
~
\hyperref[sec:index_chronic]{\small{[返順時目]}}
~
\hyperref[sec:index_scriptual]{\small{[返順卷目]}}
~
\hyperref[sec:2n9NjI1RS9k]{\small{> > > NEXT SERMON > > >}}
\newline
\newline
馬可福音 14:3-9-20230826
\newline
\begin{longtable}{cl}
\hline
\hline
章節 & 經文 (和合本修訂版)\\
\hline
14:3 & \begin{tabularx}{0.7\textwidth}{X} 耶穌在伯大尼痲瘋病人西門家裡坐席的時候,有一個女人拿著一玉瓶極貴的純哪噠香膏來,打破玉瓶,把膏澆在耶穌的頭上。 \end{tabularx} \\ \\ \relax
14:4 & \begin{tabularx}{0.7\textwidth}{X} 有幾個人心中很不高興,說:「何必這樣浪費香膏呢? \end{tabularx} \\ \\ \relax
14:5 & \begin{tabularx}{0.7\textwidth}{X} 這香膏可以賣三百多個銀幣賙濟窮人。」他們就對那女人生氣。 \end{tabularx} \\ \\ \relax
14:6 & \begin{tabularx}{0.7\textwidth}{X} 耶穌說:「由她吧!為甚麼難為她呢?她在我身上做的是一件美事。 \end{tabularx} \\ \\ \relax
14:7 & \begin{tabularx}{0.7\textwidth}{X} 因為常有窮人和你們在一起,要向他們行善,隨時都可以,但是你們不常有我。 \end{tabularx} \\ \\ \relax
14:8 & \begin{tabularx}{0.7\textwidth}{X} 她所做的是盡她所能的;她是為了我的安葬,把香膏預先澆在我身上。 \end{tabularx} \\ \\ \relax
14:9 & \begin{tabularx}{0.7\textwidth}{X} 我實在告訴你們,普天之下,無論在甚麼地方傳這福音,都要述說這女人所做的,來記念她。」 \end{tabularx} \\ \\ \relax
14:10 & \begin{tabularx}{0.7\textwidth}{X} 十二使徒中有一個加略人猶大,去見祭司長,要把耶穌交給他們。 \end{tabularx} \\ \\ \relax
14:11 & \begin{tabularx}{0.7\textwidth}{X} 他們聽見就很高興,又應許給他銀子;他就想怎樣找機會把耶穌交給他們。 \end{tabularx} \\ \\ \relax
14:12 & \begin{tabularx}{0.7\textwidth}{X} 除酵節的第一天,就是宰逾越節羔羊的那一天,門徒對耶穌說:「你要我們到哪裡去預備你吃逾越節的宴席呢?」 \end{tabularx} \\ \\ \relax
14:13 & \begin{tabularx}{0.7\textwidth}{X} 耶穌就打發兩個門徒,對他們說:「你們進城去,會有人拿著一罐水迎面而來,你們就跟著他。 \end{tabularx} \\ \\ \relax
14:14 & \begin{tabularx}{0.7\textwidth}{X} 無論他進哪一家,你們就對那家的主人說:『老師問:我的客房在哪裡?我和我的門徒要在那裡吃逾越節的宴席。』 \end{tabularx} \\ \\ \relax
14:15 & \begin{tabularx}{0.7\textwidth}{X} 他會帶你們看一間擺設齊全、準備妥當的樓上大廳,你們就在那裡為我們預備。」 \end{tabularx} \\ \\ \relax
14:16 & \begin{tabularx}{0.7\textwidth}{X} 門徒出去,進了城,所看到的正如耶穌所說的。他們就預備了逾越節的宴席。 \end{tabularx} \\ \\ \relax
14:17 & \begin{tabularx}{0.7\textwidth}{X} 到了晚上,耶穌和十二使徒都來了。 \end{tabularx} \\ \\ \relax
14:18 & \begin{tabularx}{0.7\textwidth}{X} 他們坐席,正吃的時候,耶穌說:「我實在告訴你們,你們中間有一個與我同吃的人要出賣我了。」 \end{tabularx} \\ \\ \relax
14:19 & \begin{tabularx}{0.7\textwidth}{X} 他們就憂愁起來,一個個地問他:「不是我吧?」 \end{tabularx} \\ \\ \relax
14:20 & \begin{tabularx}{0.7\textwidth}{X} 耶穌對他們說:「是十二人中的一個,就是同我蘸餅在盤子裡的那個人。 \end{tabularx} \\ \\ \relax
14:21 & \begin{tabularx}{0.7\textwidth}{X} 人子要去了,正如經上所寫有關他的;但出賣人子的人有禍了!那人沒有出生倒好。」 \end{tabularx} \\ \\ \relax
14:22 & \begin{tabularx}{0.7\textwidth}{X} 他們吃的時候,耶穌拿起餅來,祝福了,就擘開,遞給他們,說:「你們拿去,這是我的身體。」 \end{tabularx} \\ \\ \relax
14:23 & \begin{tabularx}{0.7\textwidth}{X} 他又拿起杯來,祝謝了,遞給他們;他們都喝了。 \end{tabularx} \\ \\ \relax
14:24 & \begin{tabularx}{0.7\textwidth}{X} 耶穌對他們說:「這是我立約的血,為許多人流出來的。 \end{tabularx} \\ \\ \relax
14:25 & \begin{tabularx}{0.7\textwidth}{X} 我實在告訴你們,我不再喝這葡萄汁,直到我在神的國裡喝新的那日子。」 \end{tabularx} \\ \\ \relax
14:26 & \begin{tabularx}{0.7\textwidth}{X} 他們唱了詩,就出來往橄欖山去。 \end{tabularx} \\ \\ \relax
14:27 & \begin{tabularx}{0.7\textwidth}{X} 耶穌對他們說:「你們都要跌倒,因為經上記著:『我要擊打牧人,羊就分散了。』 \end{tabularx} \\ \\ \relax
14:28 & \begin{tabularx}{0.7\textwidth}{X} 但我復活以後,要在你們之前往加利利去。」 \end{tabularx} \\ \\ \relax
14:29 & \begin{tabularx}{0.7\textwidth}{X} 彼得說:「雖然眾人跌倒,但我不會。」 \end{tabularx} \\ \\ \relax
14:30 & \begin{tabularx}{0.7\textwidth}{X} 耶穌對他說:「我實在告訴你,今天夜裡,雞叫兩遍以前,你要三次不認我。」 \end{tabularx} \\ \\ \relax
14:31 & \begin{tabularx}{0.7\textwidth}{X} 彼得卻極力地說:「我就是必須和你同死,也絕不會不認你。」所有的門徒都是這樣說。 \end{tabularx} \\ \\ \relax
14:32 & \begin{tabularx}{0.7\textwidth}{X} 他們來到一個地方,名叫客西馬尼。耶穌對門徒說:「你們坐在這裡,我去禱告。」 \end{tabularx} \\ \\ \relax
14:33 & \begin{tabularx}{0.7\textwidth}{X} 於是他帶著彼得、雅各和約翰同去。他驚恐起來,極其難過, \end{tabularx} \\ \\ \relax
14:34 & \begin{tabularx}{0.7\textwidth}{X} 對他們說:「我心裡非常憂傷,幾乎要死;你們留在這裡,要警醒。」 \end{tabularx} \\ \\ \relax
14:35 & \begin{tabularx}{0.7\textwidth}{X} 他就稍往前走,俯伏在地,禱告說,如果可能,就叫那時候離開他。 \end{tabularx} \\ \\ \relax
14:36 & \begin{tabularx}{0.7\textwidth}{X} 他說:「阿爸,父啊!在你凡事都能;求你將這杯撤去。然而,不是照我所願的,而是照你所願的。」 \end{tabularx} \\ \\ \relax
14:37 & \begin{tabularx}{0.7\textwidth}{X} 耶穌回來,見他們睡著了,就對彼得說:「西門,你睡著了嗎?不能警醒一小時嗎? \end{tabularx} \\ \\ \relax
14:38 & \begin{tabularx}{0.7\textwidth}{X} 總要警醒禱告,免得陷入試探。你們心靈固然願意,肉體卻軟弱了。」 \end{tabularx} \\ \\ \relax
14:39 & \begin{tabularx}{0.7\textwidth}{X} 耶穌又去禱告,說的話跟先前一樣。 \end{tabularx} \\ \\ \relax
14:40 & \begin{tabularx}{0.7\textwidth}{X} 他又來,見他們睡著了,因為他們的眼睛很困倦;他們也不知道怎麼回答他。 \end{tabularx} \\ \\ \relax
14:41 & \begin{tabularx}{0.7\textwidth}{X} 他第三次來對他們說:「現在你們仍在睡覺安歇嗎?夠了,時候到了。看哪,人子被出賣在罪人手裡了。 \end{tabularx} \\ \\ \relax
14:42 & \begin{tabularx}{0.7\textwidth}{X} 起來,我們走吧!看哪,那出賣我的人快來了。」 \end{tabularx} \\ \\ \relax
14:43 & \begin{tabularx}{0.7\textwidth}{X} 耶穌還在說話的時候,忽然十二使徒之一的猶大來了,還有一群人帶著刀棒,從祭司長、文士和長老那裡跟他同來。 \end{tabularx} \\ \\ \relax
14:44 & \begin{tabularx}{0.7\textwidth}{X} 那出賣耶穌的人曾給他們一個暗號,說:「我親誰,誰就是。你們把他抓住,穩妥地帶走。」 \end{tabularx} \\ \\ \relax
14:45 & \begin{tabularx}{0.7\textwidth}{X} 猶大來了,隨即到耶穌跟前,說:「拉比」,就跟他親吻。 \end{tabularx} \\ \\ \relax
14:46 & \begin{tabularx}{0.7\textwidth}{X} 他們就下手抓住他。 \end{tabularx} \\ \\ \relax
14:47 & \begin{tabularx}{0.7\textwidth}{X} 旁邊站著的人,有一個拔出刀來,把大祭司的僕人砍了一刀,削掉了他一隻耳朵。 \end{tabularx} \\ \\ \relax
14:48 & \begin{tabularx}{0.7\textwidth}{X} 耶穌回應他們說:「你們帶著刀棒出來拿我,如同拿強盜嗎? \end{tabularx} \\ \\ \relax
14:49 & \begin{tabularx}{0.7\textwidth}{X} 我天天教導人,同你們在殿裡,你們並沒有抓我。但這是要應驗經上的話。」 \end{tabularx} \\ \\ \relax
14:50 & \begin{tabularx}{0.7\textwidth}{X} 門徒都離開他,逃走了。 \end{tabularx} \\ \\ \relax
14:51 & \begin{tabularx}{0.7\textwidth}{X} 有一個青年光著身子,只披一塊麻布,跟隨耶穌,眾人就抓住他。 \end{tabularx} \\ \\ \relax
14:52 & \begin{tabularx}{0.7\textwidth}{X} 他卻丟下麻布,赤身逃走了。 \end{tabularx} \\ \\ \relax
14:53 & \begin{tabularx}{0.7\textwidth}{X} 他們把耶穌帶到大祭司那裡,又有眾祭司長、長老和文士都來一同聚集。 \end{tabularx} \\ \\ \relax
14:54 & \begin{tabularx}{0.7\textwidth}{X} 彼得遠遠地跟著耶穌,直到進了大祭司的院子,和警衛一同坐在火邊取暖。 \end{tabularx} \\ \\ \relax
14:55 & \begin{tabularx}{0.7\textwidth}{X} 祭司長和全議會尋找見證控告耶穌,要處死他,卻找不到實據。 \end{tabularx} \\ \\ \relax
14:56 & \begin{tabularx}{0.7\textwidth}{X} 因為有好些人作假見證告他,他們的見證又各不相符。 \end{tabularx} \\ \\ \relax
14:57 & \begin{tabularx}{0.7\textwidth}{X} 又有幾個人站起來,作假見證告他說: \end{tabularx} \\ \\ \relax
14:58 & \begin{tabularx}{0.7\textwidth}{X} 「我們聽見他說:『我要拆毀這人手所造的殿,三日內另造一座不是人手所造的。』」 \end{tabularx} \\ \\ \relax
14:59 & \begin{tabularx}{0.7\textwidth}{X} 就是這樣,他們的見證還是不相符。 \end{tabularx} \\ \\ \relax
14:60 & \begin{tabularx}{0.7\textwidth}{X} 大祭司起來站在中間,問耶穌說:「這些人作證告你的事,你甚麼都不回答嗎?」 \end{tabularx} \\ \\ \relax
14:61 & \begin{tabularx}{0.7\textwidth}{X} 耶穌卻不言語,一句也不回答。大祭司又問他:「你是不是基督,那當稱頌者的兒子?」 \end{tabularx} \\ \\ \relax
14:62 & \begin{tabularx}{0.7\textwidth}{X} 耶穌說:「我是。你們要看見人子坐在那權能者的右邊,駕著天上的雲來臨。」 \end{tabularx} \\ \\ \relax
14:63 & \begin{tabularx}{0.7\textwidth}{X} 大祭司就撕裂衣服,說:「我們何必再要證人呢? \end{tabularx} \\ \\ \relax
14:64 & \begin{tabularx}{0.7\textwidth}{X} 你們已經聽見他這褻瀆的話了。你們的決定如何?」他們都判定他該處死。 \end{tabularx} \\ \\ \relax
14:65 & \begin{tabularx}{0.7\textwidth}{X} 於是有人開始向他吐唾沫,又蒙著他的臉,用拳頭打他,對他說:「你說預言吧!」警衛把他拉過來,打他耳光。 \end{tabularx} \\ \\ \relax
14:66 & \begin{tabularx}{0.7\textwidth}{X} 彼得在下邊院子裡,大祭司的一個使女來了, \end{tabularx} \\ \\ \relax
14:67 & \begin{tabularx}{0.7\textwidth}{X} 見彼得取暖,就看著他,說:「你素來也是同拿撒勒人耶穌一起的。」 \end{tabularx} \\ \\ \relax
14:68 & \begin{tabularx}{0.7\textwidth}{X} 彼得卻不承認,說:「我不知道,也不明白你說的是甚麼!」於是他出來,到了前院,雞就叫了。 \end{tabularx} \\ \\ \relax
14:69 & \begin{tabularx}{0.7\textwidth}{X} 那使女看見他,又對旁邊站著的人說:「這個人也是他們一夥的。」 \end{tabularx} \\ \\ \relax
14:70 & \begin{tabularx}{0.7\textwidth}{X} 彼得又不承認。過了不久,旁邊站著的人又對彼得說:「你真是他們一夥的,因為你也是加利利人。」 \end{tabularx} \\ \\ \relax
14:71 & \begin{tabularx}{0.7\textwidth}{X} 彼得就賭咒發誓說:「我不認得你們說的這個人。」 \end{tabularx} \\ \\ \relax
14:72 & \begin{tabularx}{0.7\textwidth}{X} 立刻,雞叫了第二遍。彼得想起耶穌對他所說的話:「雞叫兩遍以前,你要三次不認我。」他就忍不住哭了。 \end{tabularx} \\ \\
[1ex]
\hline
\hline
\end{longtable}
$^{1}$各位姐妹平安.
8月最後一個崇拜.
亦是我們月提奉獻的最後一課.
一路走來這九個信息.
希望重新描述奉獻的層面或層次.
亦希望大家可以思考奉獻這個信息.
今日選擇的經文是馬可福音經文.
這個場景其實在其他福音書中也有.
但我選擇馬可其中有些原因.
稍後會跟大家解釋.
亦在文宣中吃了些字.
跟大家分享月提結束時的內容.
經文跟大家重溫一下.
相信大家也很熟悉.
我們一起讀馬可福音第十四章的版本.
第三節開始到第九節完.
我們一起起.
天主上每當我們打開你的說話時.
你仍舊對我們說話.
當日你在門徒中展現你的心意時.
求主你今天仍然對我們展現你的心意.
以致我們看到前人的愚昧.
患得患失.
今天成為我們的借鑒.
又願我們身體力行去回應上帝.
你給我們傾福奉獻的愛.
以致我們在生命中的轉向再次歸於你.
求主預立我們成心聖意經拜.
奉耶穌的名求 阿們.
剛才大家讀完經文時.
你會發覺我自己有時說話.
飯一定會吃 看跟誰吃.
有些飯吃了很生氣.
你看到人心中不悅.
這段經文.
你會看到發生在伯大利這個地方.
在福音書中也挺有名的.
起碼他行了一個很大的神蹟.
就是叫拉撒路從死人中復活.
伯大利也是在耶穌步向耶路撒冷的日子過程中.

$^{41}$一個很重要發生的情況.
這段經文在馬拉福音第1,2節.
即是第14章第1,2節.
和在約翰福音第12章中.
預節前六天 即是耶穌進城前.
即是準備踏苦路前的日子.
在家裡坐直.
你看到一些重要的關鍵字.
我跟大家highlight了.
一群人在一個家庭中坐直.
有女人 又做了一些事情.
在阿天的講道中.
講奉獻是講那些飯局.
在飯局中有些人會看看坐在什麼位置.
同樣在飯局中也有不同的人.
有不同的帶著不同的立場.
或者是一些他想要得到的過程中.
來到飯局中.
同樣也是今天在飯局中.
飯局中有些人生氣那個女人.
那個女人做了一些事情.
但耶穌最後的三跛是什麼呢.
就是要紀念這個女人做的事情.
其實有什麼大不了.
在後世的人每當說這件事的時候.
都要記住這個女人做的事情呢.
其實就是想和大家看看.
其實做了些什麼呢.
大家都會知道.
她做了一些事情就是.
她一群人吃飯.
其實和這個場景之前也有.
之前在路加福第七章中.
有一個同樣叫西門.
不過是法理菜人西門.
都是請耶穌去她家吃飯的時候.
剛好神時暮時.
不知為什麼又有些女人走出來.
但這些女人反而是我們看的highlight.
那個女人又是去到耶穌面前.

$^{81}$哭了 用眼淚沾濕耶穌的腳.
又用頭髮為耶穌擦他的腳.
耶穌又是稱讚這個女人做了一件事.
因為就說西門.
說那個法理菜人西門.
你來到我的時候.
又沒有和我洗腳 又沒有和我親嘴.
但這個女人什麼都沒有說.
來到的時候就為我做這件事.
所以你看到同樣是有一個女人做這件事.
耶穌又是稱讚她.
今天這個女人又做了一些事.
耶穌又稱讚她.
這次的西門就不是那個.
是長大麻風的西門.
這幅圖畫就是剛才我說路加福第七章中.
那個法理菜人的西門.
你見到圖畫就未必很清楚.
下方女人前面有一個瓶.
這個就是其中考古挖出來的瓶.
可能你曾經聽過這個真拿達香膏的女人.
她所做的事.
可能你聽的訊息都會說.
真拿達有多昂貴.
或者有多重點.
但今天我就不會說這件事.
重點未必著眼於真拿達有多昂貴.
不過都可以理解為何人們那麼緊張真拿達.
因為是不是有假的拿達呢.
是有的.
Pure意思是真.
因為在做真拿達香膏的時候.
本身的植物是珍貴的.
來自波斯的一種香料植物.
要做的時候產量很少.
而香氣很好.
所以很多人想模仿.
或者是A4.
要裝的過程中要滴下去.
他們做的瓶子只是儲存後就會燒掉.

$^{121}$他開不了.
他唯一用的時候就要打爛.
所以你不可以說用完.
倒回去.
很多人不懂的.
沖些假貨.
不是.
所以每一瓶都是獨立包裝.
不能說可以偷龍轉鳳.
所以他們說這個Pure Art的真拿達.
其實就是那樣東西.
只有一支.
對他們來說只有一支.
真而重之.
他願意為耶穌去打破.
當時做香料來說.
不是說不普遍.
你見到君王會用膏.
你見到名門望族都會用這些香膏.
但唯獨是真拿達比較重要.
譬如是新婚的時候.
會為配偶的時候打破玉瓶.
代表中貞.
貞潔.
為新婚階段當中.
一個很重要的時刻.
又或者一個君王.
他要高的時候.
都會用這個真拿達.
所以你見到無論是君王的高立.
或者是婚姻.
都會用到這麼貴重的東西.
但今次那個女人.
就是用了這個東西.
如果你看平衡的經文.
唯獨是約翰福音才會開名.
開這個女人的名字叫瑪利亞.
其他方面就沒有開這個女人的名字.
所以我們知道是瑪利亞.
做了這件事情.

$^{161}$所以去到這個事情.
一玉瓶至貴的真拿達香膏就打爛了.
就淋在耶穌的頭上.
有幾個人就心中不悅.
就說了.
何用這樣枉費香膏呢.
這香膏可以賣三十多銀子.
周濟窮人.
和大家想一下這句話.
說話的人是什麼人.
就是.
吃得那餐飯的時候.
都不是我們這些閒等的人.
都是耶穌的門徒.
就.
科文書裡面.
這裡就沒有說有幾個人是誰.
其實其他科文書就有門徒.
就是出了門徒.
但沒有響過是哪一個名字.
那約翰福音就響了一個名字.
那個叫猶大.
我覺得又不只是猶大的.
不過約翰就覺得.
猶大這個人就出賣耶穌.
用三十多銀子出賣耶穌.
這裡有三十多銀子.
可能好像有點比較.
但是在其他科文書對照的時候.
他是門徒.
不是一個不喜歡.
是有幾個都不喜歡.
就沒有開名.
但是重點就是.
為什麼他們不喜歡呢.
如果在耶穌身上.
他應該不敢挖的.
你覺得是不是應該不敢挖呢.
我在想這個時候.
怎麼代人進去呢.

$^{201}$都會死的.
很難說的.
可能你公司裡面.
或者你相熟群體裡面.
比較位高權重的人.
有些東西說我想吃這件.
不是吧.
你吃了就不能吃.
你不敢挖的.
長輩吃飯都是.
有些東西你想吃那一塊的時候.
你都會等人家吃完.
你想吃多一塊.
看有沒有人再吃你才吃的.
可能你不會的.
你想吃就吃掉.
你不理會吃多少塊.
不想想吃什麼方法.
不過你會發覺.
他明顯看到是倒在耶穌頭上.
但是他覺得不OK.
其實我真的.
你覺得這個位置有點卡嗎.
耶穌是應該.
師父來的.
是不是.
用在師父裡面應該是順理成章的.
沒有人敢挖.
裡面全部都說.
不OK.
不OK的重點是枉費.
什麼是枉費.
就是浪費了.
用在耶穌上面為什麼會浪費.
應該不會浪費.
如果你覺得今天.
耶穌在這裡.
很難模擬這件事.
今天有一個很重要你喜歡的人在這裡.
他突然間說他想做這件事.

$^{241}$他就拿來用.
你會不會生氣呢.
最難被大家代入.
那幾個門徒就是覺得枉費.
因為事實上.
30元銀子可以周濟窮人.
其實不僅僅是周濟窮人.
因為真拿大.
不止30元銀子.
其實30元銀子是一個比喻.
或者一個借語.
讓你明白到.
周濟窮人.
30元銀子可以買一個奴隸回來.
30元銀子可以.
請僱工回來工作.
是可以幫忙的.
如果.
他的心態未必是不滿耶穌.
只是覺得.
其實浪費在耶穌身上.
其實可以.
不用浪費在耶穌身上.
可以找多些人工作.
可以拿些錢來做其他方式.
那你在想的就是.
可能牽涉到你的機構.
你的群體或者你所屬的單位.
有些資源運用.
你可能不認同這樣運用.
或者不應該這樣的次序運用.
可能運用在其他地方.
那你就可能容易了解.
我相信那群門徒.
不是不喜歡運用在耶穌身上.
只不過那些錢.
其實是不是真的要這樣運用呢.
如果可以做其他方法.
可以做那麼多.
我們很容易就會轉移到.

$^{281}$在事工在工作上的東西.
但耶穌跟他說.
不是這樣.
所以在這裡停一停.
你覺得那東西值不值得呢.
其實是觀點與角度.
我在想.
怎樣讓你明白觀點與角度.
我也在想說不說.
但我覺得也可以說一下.
就好像追星.
我自己.
自問沒有追星的.
習慣.
或者是一些做法.
我不知道大家會不會追星.
No offense.
先戴上頭盔.
但你會否覺得追星是可以.
是沒有程度之分的.
是只有狂熱和更狂熱.
追星本身就是狂熱.
狂熱意思是.
可能你看日程表.
可能就是.
了解一下.
用什麼方式可以不同方法支持他.
支持的程度.
就可以有好支持和極度支持.
怎會知道好支持.
這個就見仁見智.
我一開始.
出入機場兩次的時候.
我都是比較快手快腳.
我不喜歡寄倉.
很快就走的那些人.
走的時候我可能會自己走.
我發覺有人舉牌.
那不是舉給我的.
但我發覺.

$^{321}$舉牌的通常都是.
我不認識你的.
舉牌的通常都是.
捉錯用神的.
因為我認識一些super fans.
其實是不會在那裡舉牌的.
是在另一個地方.
上車的地方.
等的fans.
有些人點頭了.
所以你會發覺.
在過程當中.
你會發覺你會.
喜歡那件事.
你會用你的方法.
覺得那件事是值得的.
無論值得是時間.
心力.
有些人弄了很漂亮的濕牌.
然後就是.
錢.
雖不止是課金.
但你會買很多紀念品去應援.
對於我這個.
不追星的人來說.
我就覺得何用這樣枉費呢.
(笑聲).
對於我這個.
不追星的人我就會問.
就是.
Blackpink那張門票.
是有收藏價值的.
是不是.
但是Blackpink那張門票的價值.
就是真的不菲.
是不是.
很貴的.
我剛才曾經說過一件事.
如果教會崇拜要收門票.
收多少錢呢.

$^{361}$是不是.
你會看到這些握手位.
(笑聲).
鏡頭拍一下這個位置.
我們仍然是.
場地是空的.
是不是.
去到鏡頭的另外一邊.
(鏡頭拍下).
我現在去的這個攝影機.
這個握手位.
其實是我們攝影師.
他不是一個.
只是來崇拜的會眾.
所以這個位置是在等待你的.
你看到教會.
這些所謂的三千八百元門票.
你們是需要.
專榮的.
是要在後面請上來坐在前面.
我現在去到後面請你上來坐.
是不是.
你會看到.
有些叫做何用.
枉費.
其實是很個人的.
你覺得那樣東西值不值.
是看.
不是在乎那個物質.
是在乎心願不願意擺出來.
那些門票是在看.
其實如果請一個勞工.
我們就不辛苦.
原來那個三十多元錢.
請一個僱工請人回來工作的時候.
就應該會幫忙.
打點一些東西.
我們就可以做其他事情.
你會發覺焦點與角度.
其實不是說不可以做在人身上.

$^{401}$不過他一個人受惠.
你多人忙碌.
如果他一個人不受惠的時候.
不是更好嗎.
但你會發覺這個女人.
就把那件事打破了.
她將她最.
珍而重之的.
珍娜達香膏.
高了在耶穌的身上.
對於你來說.
什麼對你來說是珍而重之.
對於你來說.
什麼願意你all in.
其實沒有人可以comment.
只有你自己才知道.
但我深信到你現在這一刻.
目測應該沒有中學生.
你們剛剛過了中學階段.
不久.
你都有十幾二十年的權衡.
我有兩個兒子的時候.
他中學的時候.
他都有些錢.
我就問他買吧.
他說可能有些更加好.
他說不是.
可能過了就沒有了.
他們都會掙扎.
因為錢有限貨無限.
但買不買.
除了心頭好之外.
都覺得那東西值不值得.
所以仍然是說那句.
那三十多兩銀子是一個reference.
重點就是.
你看的角度和我看的角度.
不同.
奉獻不是在乎那個物質.
和那個金錢.

$^{441}$奉獻就是我覺得那東西.
是我覺得all in的.
那女人就將她覺得珍娜達.
珍而重之的東西.
all in.
這女人所做的.
要為紀念.
耶穌怎樣看那幾個人.
心中不悅呢.
耶穌就說話.
那耶穌說什麼呢.
任由他吧.
不要那麼多comment.
任由他吧.
他已經選擇了這件事.
我相信在選擇過程當中.
你的奉獻.
一定是經過你的選擇.
你這個月提做了一個.
Pay Him的活動.
你會看到我們常常拿張chat.
在那裡掩掩樣樣.
在不同的media告訴你.
有超過一百多個看完才會和我們一起.
在Pay Him的活動.
IG post有很多東西都告訴你.
其實大家用不同形式去做選擇.
在過程當中.
奉獻是一個生命的轉向.
是一個生活形式的轉變.
你每時每刻都可以做選擇.
你選擇做這個.
選擇不做這個.
由他吧.
為什麼要難為他呢.
他在我身上.
做了一件美事.
What a beautiful thing.
是寫在經文裡的內容.
這件美事美在什麼地方呢.

$^{481}$其實你要做的事.
其實很多人都會繼續做.
你們常有窮人與你同在.
要享用行事.
隨時都可以.
只是你不常有我.
其實每件事都要分先後.
我不想說裝閒這個字.
裝和閒有時都不容易去分.
誰裝閒.
但先後比較容易.
做哪件事先 哪件事後.
有些事做先不代表閒.
因為他急.
所以我希望你分先後的過程當中.
其實就是一個選擇.
你很難說對與錯.
但是在過程當中.
他所做的.
是盡他所能的.
他選擇過.
他覺得值得.
他就做這件事.
他為我安葬的.
是把香膏.
預先囂在我身上.
這個就是關鍵字.
想和大家去想.
耶穌說了這句話.
他是為我安葬的.
是把香膏預先囂在我身上.
這個是說將來式.
如果你是封書的編排.
你見到的記載.
就是耶穌從開初的時候.
特別是馬漢福音.
他登山變象的時候.
已經告訴跟隨的三個人.
下來的時候.
我是會為你們死的.

$^{521}$但他們聽不明白.
他們下去就說.
剛才在山上看到的東西.
耶穌頂他們不順就問他們.
在路上你們議論的是什麼.
他們爭誰大.
在另外一封書裡面說到.
在該撒勒比納比的時候.
他們問人子是誰的時候.
耶穌最後都說.
他會死.
這個是耶穌基督的工作.
耶穌一直以來都在說.
他不是他們想像中的那樣東西.
他會上一個.
受死待贖的路.
班門徒不明白.
但是這個女人.
做了一件事.
耶穌就說.
她是為我安葬.
她說的是中末將來式.
安葬完前.
因為如果你看約翰福音第十二.
這是平行經文.
約翰福音十二章.
但約翰福音十一章裡面是什麼經文呢.
就是伯大利拉薩路死了.
拉薩路死的時候.
他很不開心.
他就說夫子如果你早點來的時候.
我弟弟不用死.
但耶穌說他只是睡著了.
他說他會復活的.
但瑪利亞她又說.
夫子我知道末後的時候.
他會復活的.
但耶穌跟他說.
一句話.
可能安息禮拜大家都熟悉.

$^{561}$復活在我.
生命也在我.
信我的人雖然死了.
也必復活.
凡活著信我的人.
永遠不死.
接著耶穌就再也沒有出聲.
耶穌就去拉薩路前面叫醒她.
女人瑪利亞在約翰福音第十一章裡面.
她親眼見過耶穌能夠叫死人復活.
她親眼見過耶穌能夠說得出做得到.
她親眼見過她所信的耶穌是一個怎樣的人.
她覺得這個人是值得我all in.
她覺得這個人是值得我將我最寶貴的東西放進去.
所以什麼人看什麼.
什麼人做什麼.
是很真實的.
她覺得30兩銀子是可以幫你做到事工.
30兩銀子是可以幫你解決一些所謂庶務的事.
你就用30兩銀子做這件事.
但是對於瑪利亞來說.
真拿達的貴重是我願意將我所有的放出來.
奉獻從來都不是說那些價值.
就好像John說窮寡富那兩個銀錢.
和那個有錢的人做對比的時候.
從來不是銀碼.
是說他覺得全情全心全部去奉獻.
全部放進去.
那個是重點.
有什麼是值得全情投入的呢.
有什麼是覺得全部是需要放進去的呢.
在這個月提的廣道延伸的時候.
有些小組討論的時候.
其中有一個小組聽他們分享的時候.
都是很真實.
也是我自己會認同的.
就是說到當舊約要求.
十二支派有十一奉獻的過程當中.
你看到新約就沒有再重提十一奉獻這個terms.
但你會發覺新約耶穌說跟隨的人.

$^{601}$是要變賣你的所有.
然後還要跟從.
就是除了錢之外.
連個人也要.
另外新約保羅說奉獻就是將身體獻上.
現在新約的要求是比舊約還要強.
弟兄那時候分享也是說一個信息.
就是舊約就是十一奉獻.
新約還要你的性命.
事實上也是.
跟隨耶穌的過程當中.
耶穌說背起十字架跟從我.
不是說錢的問題.
是你願不願意將你覺得看重的東西.
你自己背上身跟著去做.
窮寡婦的兩個銀錢.
不是說他有兩個.
奉獻了兩個這麼簡單.
是他覺得那件事是他值得all in.
值得全情投入.
今天我們奉獻會在想什麼呢.
就是你在想你可以奉獻多少.
剩下多少.
你可能在看剩下多少.
你才在想你奉獻多少.
又或者在想.
剩下多少不是說錢這麼簡單.
可能你有時間.
你有能力.
但你會不會全情投入將你所有東西.
覺得要奉獻給上帝呢.
這是需要你自己慢慢調整.
你自己覺得那件事有多貴重.
每個人心中都有一個真拿達.
每個人心中都覺得那件事是最寶貴.
耶穌說他為我安葬.
瑪利亞看到耶穌.
你會看到其實耶穌到他復活的時候.
他是沒有香膏高抹的.
你看到七天的第一天.

$^{641}$福音書裡面.
無論是馬太福音還是路加福音.
約翰福音記載.
七天的第一天.
婦人就是很渴望要走到耶穌的墳墓面前.
是希望幫助耶穌用香膏包裹自己的身體.
你看到他們全部都找不到.
只看到空瓣.
真正能夠為耶穌高抹身體的就是這個女人.
而這個女人的奉獻是終末式的.
而這個女人的奉獻是為了耶穌身上成就的事情所奉獻.
而這個女人所做的高抹就是耶穌最終的高抹.
所以我們傳頌的時候就要述說這個女人.
她的奉獻不是在看現在這個時刻.
她的奉獻是在看將來的時刻.
她所做的事情就是.
這個人是我親眼見過他能夠叫死人復活.
如果我們今天是相信復活的訊息的時候.
我們可以all in到什麼呢.
如果今天是我們相信復活的主的時候.
你得到復活有永生的生命的時候.
我就問你一件事.
你可以all in到什麼呢.
你可以奉獻你的真那達是什麼呢.
這個是我們要想的問題.
不是在說錢.
不是在說才幹.
不是在說能力.
我想說的是你的心意.
在大祖的過程當中.
有不同歲數的弟兄姊妹.
有些還在工作有賺錢能力的弟兄姊妹.
她可能就是拿每月的月捐成為她一個心意的表達.
但是你會發覺不同歲數.
你的收入可能未必如以前那麼多.
或者你家裡轉了階段有不同的供應的時候.
可能沒有以前那麼多的時候.
如果錢不是你奉獻的話.
其他有沒有生命的轉化呢.
我仍然很希望.

$^{681}$不要像那些門徒那樣.
用三十多兩銀子這個概念.
而看不到真正的焦點.
不要看到可以做的事情.
而忽略了奉獻的真正意義.
耶穌在說.
這群人.
我們繼續去傳頌的時候.
就是在看每個取捨.
而這個女人瑪利亞所奉獻的.
是看到能夠在耶穌的奉獻.
耶穌自己的奉獻當中.
去做一件美事.
就是為耶穌高沒她.
這張照片是早前和一群Flo Church的牧者.
在台灣做一個探訪的活動.
其中一個重點的地方.
是我自己很想去的地方.
是內地的宣教士和馬來西亞的宣教士.
在台灣做了十多年的聚會點.
他們做的宣教工場就在萬華區.
台北萬華區.
台北萬華區可能對很多頂尖的會是陌生的.
但當我說西門町的時候.
你就不陌生了.
萬華區其實就是西門町站後一個站.
其實由西門町走過去.
由紅樓那邊走過去.
十分鐘路程就會到.
不過你走過去沒什麼看的.
因為是舊區.
萬華區有一個叫珍珠家園的地方.
珍珠家園就是宣教士去做一些紅燈區女性服務的中心.
去服侍那群姐妹.
跟她們聊天和建立社會關係.
在聽宣教士分享阿姨的故事的時候.
我自己是很被感動.
因為我看到很多阿姨.
她們窮她一生的生命.
她們的力氣就是為她的家庭付出.

$^{721}$為她的家庭奉獻她所有的能力.
而很多故事當中.
她們的奉獻是得不到回報的.
特別是家人不體諒她.
家人不明白她.
更加不屑.
其中一個故事我很深刻.
我跟小組員分享.
當她們字不多.
宣教士找了一個很出名的新晉作者.
寫她們的故事的時候.
筆錄成書.
還沒出書之前就讀回受訪者的故事給阿姨聽.
當阿姨聽了她的故事的時候.
她跟宣教士說.
可不可以不要這一段.
因為這一段如果我兒子聽了.
他會不開心.
但宣教士問她.
為什麼不要.
這段很重要.
我想你兒子知道.
因為阿姨窮她一生在這些區賺到的錢.
為了賣她的原生家庭.
賣她去一個地方的原生家庭.
付的賭債.
幫她還清.
她賺的錢為了她自己的家庭.
最後男人走了之後.
去供樓.
為了她兒子讀書.
讓他學有所成.
但她兒子後來知道.
她媽媽是做這些工作的時候.
他覺得媽媽的錢是骯髒錢.
你不配住這裡.
但宣教士說.
其實如果你覺得這些錢是骯髒的話.
那你就不要住這個地方.
應該是你走而不是你叫媽媽走.

$^{761}$但媽媽仍然覺得.
我盡我的責任奉獻給家庭.
做好我的崗位.
做好我覺得做媽媽要有的東西.
她不喜歡我走沒問題.
我聽到這句話.
這就是奉獻.
她覺得這件事值得.
哪怕沒有回饋也好.
她覺得值得.
可能你覺得這些故事香港也有.
或其他地方也有.
但我覺得這些故事就要記錄下來.
不是比較哪一區.
就要記錄下來.
所以宣教士說我要記得這段事.
這段事如果我們過去了.
你也過去了的時候.
就沒有人記得你為人所奉獻的東西.
要傳頌下去.
同樣是瑪利亞一樣.
她為耶穌所做的事.
要傳頌下去.
她覺得這件事值得.
宣教士其實不是說.
她信主的事要記錄下來.
她怎樣信主的時候.
才會成為一個生命見證.
不是.
其實很多人都用生命去奉獻了很多東西.
她希望仍然可以有機會去延續.
讓人知道人間有愛.
人間有付出.
如果我說到後來.
這個阿姨最後也信了主.
大家就會Hallelujah.
我相信大家不是在說她信不信主的重點.
重點就是.
每個人都在考慮那件事值不值得.
需要時間奉獻.

$^{801}$她就All in.
這本書叫《茶室女人心》.
在香港的成品是沒有進的.
很可惜.
但在台灣有很多書室可以買.
你可以買電子版.
但因為他們不太知道.
宣教士就跟阿姨說.
我們將來出了一本書.
她就說.
我們這些書有人想知道嗎.
他們沒有去過書室.
宣教士就帶他們去書室.
看他們.
你留意一下.
你這本書.
他們看到自己的書在那裡.
帶他們走到一邊去看看.
他們看到有人在揭.
揭揭下揭.
打下書釘.
以為放下了.
那個人拿去給錢.
然後阿姨就說.
我一輩子都在茶室裡工作.
很多人說了很多故事給我們知道.
說了很多故事給我們聽.
但沒有人願意聽我們的故事.
原來我們的故事是有人會聽的.
原來我們的故事是有人會買的.
她就哭了.
今天我們看這段經文的時候.
要述說這個女人的故事的時候.
就是她奉獻的故事.
她要述說的故事的時候.
就是她覺得預示了將來.
我相信這個人是能夠叫死人復活的.
我相信這個人就是我生命的主.
我相信這個人就是講得出做得到的.
我願意全情投入要all in.

$^{841}$這個就是奉獻.
這個就是日後要說要傳頌這個福音的時候.
你同樣要述說這個女人的故事.
因為她看到將來式.
而這個將來式是她真的經歷過.
親弟姊妹.
這個是一個奉獻的見證.
人間仍然有很多奉獻的見證.
是需要傳頌下來.
是需要傳下來.
我在小組裡面也分享過其他.
但是你會看到每一次生命的故事最觸動的.
我相信不是她信不信主.
每一次生命觸動的時候.
除了人間有愛就知道.
其實每個人都會選擇將生命最重要的東西彰顯出來.
從你的行為當中你可以看出你心中耶穌的價值.
當你要奉獻給耶穌的時候.
不是在說你可以拿多少給耶穌.
其實是在問其實耶穌值多少錢.
當你每次奉獻的時候.
你奉獻的是你的袋有多少錢.
你預備了多少錢.
同樣都在說你覺得耶穌值多少錢.
從你心中的耶穌的價值就決定了你怎麼看這個信仰有多實在.
你覺得這個說話可能很重.
但是對我來說是很重.
對我來說我是一個全時間奉獻的人.
是我真的選擇了放下很多東西.
我不是要自誇.
因為我不需要大家什麼肯定.
因為我肯定我自己所付出的是我覺得值得的.
是我覺得沒有枉費.
因為耶穌在我心中的價值是願意讓我自己選擇另一個生活模式.
是願意讓我選擇另一個生活價值.
這個就是生命的取態.
猶他爸.
你希望你自己在耶穌心中.
又或者你自己怎麼擺在耶穌心中的順位.
這個就是你自己重新調配.

$^{881}$相信每一個奉獻的人都是擺過的.
你看到少年的棺.
他就擺過了.
他擺不到.
所以悠悠愁愁就走了.
你會看到撒該他擺過了.
所以他歡喜快樂去接待耶穌.
同樣你也是在掙扎在少年的棺和撒該裡面.
我希望奉獻是給我們一個生命的重新投放.
真 拿 達.
你真正拿出什麼來達到上帝面前.
是我今天的講題.
整個閱題當中我希望和大家去總結.
每個人生命中都有一個真拿達.
你的香膏是你最貴重的東西.
你真正可以拿出來達到上帝面前.
其實就是反映你和耶穌之間的敗位.
也是你怎麼去選擇All in在耶穌的生命.
當你的真拿達要傾倒出來的時候.
你傾倒的是什麼呢.
一時間回應詩是傾倒這首歌.
見敗隊會在詩歌當中讓我們再聯想.
瑪利亞她的信心與行為.
在傾倒的過程當中你傾倒了什麼呢.
我不希望今天講到只是你一個情緒上的波動.
我希望傾倒這個回應詩.
會成為我們在2023年的第三季生命的轉向.
奉獻這個閱題是我們當世界都說復常.
可以做自己的事的時候.
你一個新生命的取態.
不在乎那種物質 金錢 不在乎什麼.
願意傾倒成為我們新生命的開展.
\newpage



\section{撒母耳記下 23:13-19-20230902}
\label{sec:2n9NjI1RS9k}
\textbf{【網上崇拜】大衛與三個沒有名字的勇士|撒母耳記下23\_13-19|20230902 [2n9NjI1RS9k]}
\newline
\newline
連結: \href{https://youtube.com/watch?v=2n9NjI1RS9k}{\texttt{ https://youtube.com/watch?v=2n9NjI1RS9k}} ~~~~ 語音日期: 2023-09-02 
\newline
\newline
\hyperref[sec:br4_eJEULQ0]{\small{< < < PREV SERMON < < <}}
~
\hyperref[sec:index_chronic]{\small{[返順時目]}}
~
\hyperref[sec:index_scriptual]{\small{[返順卷目]}}
~
\hyperref[sec:n7jQq3kdpkc]{\small{> > > NEXT SERMON > > >}}
\newline
\newline
撒母耳記下 23:13-19-20230902
\newline
\begin{longtable}{cl}
\hline
\hline
章節 & 經文 (和合本修訂版)\\
\hline
23:13 & \begin{tabularx}{0.7\textwidth}{X} 開始收割的時候,三個侍衛下到亞杜蘭洞,到大衛那裡。非利士的軍兵在利乏音谷安營。 \end{tabularx} \\ \\ \relax
23:14 & \begin{tabularx}{0.7\textwidth}{X} 那時大衛在山寨,非利士人的駐軍在伯利恆。 \end{tabularx} \\ \\ \relax
23:15 & \begin{tabularx}{0.7\textwidth}{X} 大衛渴想著說:「但願有人從伯利恆城門旁的井裡打水來給我喝!」 \end{tabularx} \\ \\ \relax
23:16 & \begin{tabularx}{0.7\textwidth}{X} 這三個勇士就闖過非利士人的軍營,從伯利恆城門旁的井裡打水,拿來給大衛喝。他卻不肯喝,將水澆在耶和華面前, \end{tabularx} \\ \\ \relax
23:17 & \begin{tabularx}{0.7\textwidth}{X} 說:「耶和華啊,我絕不做這事!這三個人冒生命的危險,這不是他們的血嗎?」大衛不肯喝這水。這是三個勇士所做的事。 \end{tabularx} \\ \\ \relax
23:18 & \begin{tabularx}{0.7\textwidth}{X} 洗魯雅的兒子,約押的兄弟亞比篩是這三個勇士的領袖;他曾舉槍殺了三百人,就在三個勇士中得了名。 \end{tabularx} \\ \\ \relax
23:19 & \begin{tabularx}{0.7\textwidth}{X} 他在這三個勇士中是最有名望的,所以作他們的領袖,只是不及前三個勇士。 \end{tabularx} \\ \\ \relax
23:20 & \begin{tabularx}{0.7\textwidth}{X} 耶何耶大的兒子比拿雅是來自甲薛的勇士,曾行了大事。他殺了摩押人亞利伊勒的兩個兒子,又在下雪的時候下到坑裡去,殺了一隻獅子。 \end{tabularx} \\ \\ \relax
23:21 & \begin{tabularx}{0.7\textwidth}{X} 他又殺了一個魁梧的埃及人;埃及人手裡拿著槍。比拿雅只拿著棍子下到他那裡去,從埃及人手裡奪過槍來,用那槍殺死了他。 \end{tabularx} \\ \\ \relax
23:22 & \begin{tabularx}{0.7\textwidth}{X} 這些是耶何耶大的兒子比拿雅所做的事,就在三個勇士裡得了名。 \end{tabularx} \\ \\ \relax
23:23 & \begin{tabularx}{0.7\textwidth}{X} 他比那三十個勇士更有名望,只是不及前三個勇士。大衛立他作護衛長。 \end{tabularx} \\ \\ \relax
23:24 & \begin{tabularx}{0.7\textwidth}{X} 三十個勇士中有約押的兄弟亞撒黑,伯利恆人朵多的兒子伊勒哈難, \end{tabularx} \\ \\ \relax
23:25 & \begin{tabularx}{0.7\textwidth}{X} 哈律人沙瑪,哈律人以利加, \end{tabularx} \\ \\ \relax
23:26 & \begin{tabularx}{0.7\textwidth}{X} 帕勒提人希利斯,提哥亞人益吉的兒子以拉, \end{tabularx} \\ \\ \relax
23:27 & \begin{tabularx}{0.7\textwidth}{X} 亞拿突人亞比以謝,戶沙人米本乃, \end{tabularx} \\ \\ \relax
23:28 & \begin{tabularx}{0.7\textwidth}{X} 亞何亞人撒們,尼陀法人瑪哈萊, \end{tabularx} \\ \\ \relax
23:29 & \begin{tabularx}{0.7\textwidth}{X} 尼陀法人巴拿的兒子希立,便雅憫族基比亞人利拜的兒子以太, \end{tabularx} \\ \\ \relax
23:30 & \begin{tabularx}{0.7\textwidth}{X} 比拉頓人比拿雅,迦實溪人希太, \end{tabularx} \\ \\ \relax
23:31 & \begin{tabularx}{0.7\textwidth}{X} 亞拉巴人亞比‧亞本,巴魯米人押斯瑪弗, \end{tabularx} \\ \\ \relax
23:32 & \begin{tabularx}{0.7\textwidth}{X} 沙本人以利雅哈巴,雅善兒子中的約拿單, \end{tabularx} \\ \\ \relax
23:33 & \begin{tabularx}{0.7\textwidth}{X} 哈拉人沙瑪,哈拉人沙拉的兒子亞希暗, \end{tabularx} \\ \\ \relax
23:34 & \begin{tabularx}{0.7\textwidth}{X} 瑪迦人亞哈拜的兒子以利法列,基羅人亞希多弗的兒子以連, \end{tabularx} \\ \\ \relax
23:35 & \begin{tabularx}{0.7\textwidth}{X} 迦密人希斯萊,亞巴人帕萊, \end{tabularx} \\ \\ \relax
23:36 & \begin{tabularx}{0.7\textwidth}{X} 瑣巴人拿單的兒子以甲,迦得人巴尼, \end{tabularx} \\ \\ \relax
23:37 & \begin{tabularx}{0.7\textwidth}{X} 亞捫人洗勒,比錄人拿哈萊,是給洗魯雅的兒子約押拿兵器的, \end{tabularx} \\ \\ \relax
23:38 & \begin{tabularx}{0.7\textwidth}{X} 以帖人以拉,以帖人迦立, \end{tabularx} \\ \\ \relax
23:39 & \begin{tabularx}{0.7\textwidth}{X} 赫人烏利亞,共三十七人。 \end{tabularx} \\ \\
[1ex]
\hline
\hline
\end{longtable}
$^{1}$好嗎?你們?OK吧?.
先看看你們的樣子.
很感動的,當颱風的時候.
中安澳的朋友,元朗的朋友,沙溪灣的朋友,灣仔的朋友,沙田的朋友.
都躲在家裡.
但是當風暴完結後,我們就一起來了.
我們一起來敬拜,這就是教會.
這就是我們聚集重要重視的任務.
這個月的月題是「上莊」.
為了回憶以前上莊的氣氛.
我很多部分都回到香港大學圖書館寫的.
現在香港大學裡面有很多內地人.
我發覺自己已經是他們的年紀的雙倍.
已經是無法霸佔了.
通常我一開始講月題,我都會先解一下題.
講一下月題的意義是什麼.
但這次我容讓我之後再講.
我們先正常的講一篇道.
今天講的經文是《撒慕爾的俠》.
十三章十三到十九節經文.
一個有關大衛和三個勇士.
打水給他喝的故事.
基本上題材都講出來了.
基本上差不多都是.
如果你一直都有參加流唐崇拜和好記性的話.
你應該記得的.
這個故事其實你聽過的.
在2021年5月的時候.
高銘軒牧師來過流唐講道.
他講過這個故事.
有三個勇士打水給他喝的故事.
大家可以聽聽.
講題叫做「甚遠有人打水給我喝」.
這是我的同事所講的.
他用的經文就是聶王記相.
今天我們講的就是《撒慕爾的俠》的版本.
今天的經文當然是講的很不同.
不是從大衛的角度去思考這個故事.
而是從其他角度.
我們一起祈禱.

$^{41}$不要不一起去聆聽上主的說話.
主多謝你給我們一群頂尖妹.
無論是在網上.
在不同的地方.
在北美.
在英國.
在不同的角落裡面.
我們一起聚集.
求主你的說話能夠臨到我們.
讓我們不單單一個人.
我們能夠一個一個連成在一起.
讓我們能夠成為一個極有意義的群體.
求神你這樣來帶領我們.
幫助孩子.
能夠將你的說話清楚地說出來.
逢尊名求 阿門.
我們先講一下經文的背景.
《撒慕爾的俠》第23章是整個《撒慕爾的俠》最後的部分.
如果你看《撒慕爾的俠》的話.
整本書是順著次序去講的.
從哈拿山仔開始.
到撒慕爾的事跡.
到蘇羅王被高納.
到大衛興起.
蘇羅和大衛之間的爭戰.
大衛王朝建立.
統一以色列.
之後再去成為皇帝之後.
就是不屍巴.
阿沙隆等等的故事.
你會發現整個《撒慕爾的俠》.
是按著歷史的年份順序去寫的.
不過來到《撒慕爾的俠》的最後部分.
你會發現《撒慕爾的俠》的21章到24章.
是不跟時序的.
你會發現《四頁經文》.
就好像一本書的附錄一樣.
夾雜後記去加一些不同的內容.
這個附錄包括什麼呢.
這個附錄包括了一篇詩篇18篇.

$^{81}$大衛讚美的詩篇.
加進去.
還有一段大衛臨死前.
上帝對大衛臨終的一句話.
一段神語.
24章講的是一個獨立的小故事.
就是大衛數民的故事.
還有今天所講的一段名單.
就是大衛勇士們的名單.
為什麼這班勇士.
裡面記載了30個勇士的名字和他們的事跡.
為什麼這班勇士能夠這麼有遺憾.
能夠在整個聖經裡面被記載呢.
一個字就是厲害.
30個勇士真的名符其實.
每一個都是大衛王朝裡面最大得最抽得的勇士.
如果十個字是一個三個字的遊戲.
這三個字是一些武將.
戰鬥力超過80的武將.
名符其實的戰狼30.
真的很厲害的勇士.
今天講的正正就是戰狼30裡面其中的三個.
30個裡面的三個.
我們一起看一下.
13字開始這樣說.
收國的時候有30個勇士中的三個人下到亞都蘭洞見大衛.
費利士的軍兵在利凡欽谷安營.
那時大衛在山寨.
費利士人的防營在阿伯利恆.
大衛渴想說甚願有人將伯利恆城門旁.
井裡的水打來給我喝.
這三個勇士就衝過費利士人的營盤.
從伯利恆城門旁的井裡打水拿來奉給大衛.
基本上.
用一句話來總結整個故事.
這個故事基本上是一個foodpanda.
外賣故事.
三個勇士一起接了一張order.
由伯利恆的井旁外賣杯的井水到亞都蘭洞.
沒有貼士的.

$^{121}$經文說的是大衛尚未立國的時候.
留意當中一個地點.
利凡欽谷這個地方.
人稱為巨人谷的地方.
如果你打開十美記下第五章的話.
同樣出現這個地方名.
利凡欽谷.
所以我們很大機會相信這段故事.
正正是什麼.
大衛將要統一天下的時候.
最後要擊破費利士人的伯利恆.
才能夠統一天下的故事.
快要統一天下.
如日方中的大衛.
用手機拿了一張外賣order.
我嘗試心目中的說法.
外賣的餐廳就在伯利恆門旁的井旁裡.
送貨的地址就是亞都蘭洞.
究竟亞都蘭洞和伯利恆的距離有多遠呢.
如果你猜猜的話.
你會找到的.
結果是多久.
大概是28公里.
步行的話大概需要六個小時.
一來一回.
去去倒倒水.
回來大概12小時.
如果你不休息的話.
不過OK的.
他們是勇士.
他們速度是70.
所以基本上他們是可以連日帶月來送外賣.
所以如是者.
你看到這三位是南亞裔的外賣車手.
他們是南亞地方居住的人.
由亞都蘭洞出發.
前往伯利恆.
途中經過利凡欽谷.
又稱為巨人谷的敵方陣營.
千山萬水長途跋涉.

$^{161}$去到伯利恆的城門井旁.
將水打上來.
用水壺裝著.
用保溫袋裝著.
再沿路回程.
又經過一次巨人谷的敵方陣營.
又回到亞都蘭洞.
將外賣遞給亞人簽收.
一件很震撼.
很令人值得紀念的事情.
不過最令人震撼的是什麼.
當大衛接過外賣之後.
隨手將這個在水壺冰涼的.
新鮮的保溫袋裡拿出來的.
真真正正的千山萬水.
就掉在地上.
倒了.
正如金老闆所說.
我今天講到的焦點.
不是去談論大衛的動機.
這不是我們講到的重點.
當然他的動機值得我們研究.
很多的港島都有興趣.
去談論大衛的動機和用意.
但這不是我們今天的重點.
不過我姑且講一下.
我做了一點研究.
都知道學者如何去推測大衛的動機.
第一肯定不是的.
排除的是他口渴.
大衛不是口渴.
不是出於生理的需要.
這三個勇士不是為了救他止渴.
雖然大衛身處亞都蘭東.
其實不是很淒涼.
亞都蘭東不是你想像中淒涼的山洞.
起碼是有水源的.
是一個軍事的地方.
所以排除了不是口渴.
第二個可能的原因是什麼.

$^{201}$可能就是大衛思鄉.
大衛打發人去百里行.
城門的井旁打水.
因為百里行是大衛的故鄉.
當然今天大家都記得.
百里行是耶穌出生的地方.
但大家不要忘記.
百里行本來叫什麼.
叫大衛之城.
所以百里行本來就是大衛自己成長的地方.
所以大衛特別要拿一張命令去拿井水.
是為了品嘗小時候古早味的井水味道.
當然第三個原因.
一個比較陰謀論的推測.
有人覺得大衛純粹是要去宣說自己的權力的表現.
當時大衛如日方中.
將要統一天下.
一人之下.
沒有,就是他.
他就是一人之下,上面那個.
他能夠有這樣的權力.
叫人幫他打水.
正如色神所說.
喜歡怎樣就怎樣.
這就叫做拋瓦剌.
這就是他想做的.
宣洩自己的權力.
想做就去做,這就叫拋瓦剌.
第四個原因.
學者認為,這是一篇論文寫出來的.
大衛這樣做其實是暗地裡的一個戰略.
正如我剛才所說.
是第五章裡面的戰爭的片段.
所以大衛對菲利士的時候.
表面上是叫勇士去打水.
實質是去試探軍情.
希望能夠知道敵方陣營的情況.
這就是一篇論文寫出來的.
無論如何,我們不關心大衛為何這樣做.
我們關心什麼呢.

$^{241}$我們去想一下三位勇士的觀點.
怎樣去理解整件事的本質.
無論大衛的動機是怎樣也好.
一個千真萬確無可否認的鐵板事實.
就是這三位勇士長生長水.
把自己的性命都豁出去.
幾經辛苦把井水打回來.
然後被大衛倒在地上,糟蹋了.
不知道你怎麼想.
如果你是大衛那三個勇士中的一個.
你用你的生命,走12個小時.
把水拿回來給大衛喝.
而大衛立即倒掉的話.
你有什麼感受.
基本上根據聖經記載.
聖經是沒有記載的.
沒有記載這三位勇士的反應.
不過如果是你的話.
當你付出了100\%,200\%的努力.
你的心血,付出了很多心血.
被人就這樣倒在地上的時候.
你會怎樣去理解這件事.
白做,浪費心機,徒然,無謂.
如果真的Foodpanda也有它.
那個客人吃不吃,或者寫大密碼.
基本上我就收了外賣錢.
但當你真的用你的生命.
好好地把水給大衛喝.
而他不喝,而倒在地上的時候.
是一回什麼事情呢.
整件事的本質是一件什麼事情呢.
這就是我今天嘗試去想的地方.
不過是真的.
當你年紀越來越大的時候.
你會發現這個世界裡面有很多類似的東西.
多麼的辛苦,流盡血汗,白白付出.
然後那個結果是沒有結果的.
努力付出了,放上了.
但其實是人們都沒有理會過,沒有用到.
或者是發揮不到.

$^{281}$一個運動員參加過奧運的原顧訓練了四年.
突然間出賽,突然間受傷.
你公司的項目,跟一班同事熬了半年.
很多個通宵,很多個通宵,很多個會議.
突然間上頭說資金問題,不做.
你的設計,你的項目,你的心血那五百多頁的報告書都丟掉了.
幾年前你走了這麼久.
最後雞飛狗走.
今天的閱題是「想裝」.
不知道你對你來說「想裝」是什麼.
我想未來這八個星期,不同的港元都對「想裝」有不同的體會和理解.
對我來說「想裝」是什麼呢.
對我來說所謂「想裝」就是一班人聚在一起.
一起很努力,很認真,付出很多心機.
做一些沒有什麼結果的事情.
真的,很努力,很努力,很努力.
做一些其實沒有什麼改變世界的事情.
這個對我來說就是「想裝」.
很多萬通頂,自己不斷在掃訪裡.
在那裡預備一些powerpoint,不求回報,很熱血的衝勁.
最後沒有什麼後果,沒有什麼結果.
這個就是我對「想裝」的回憶和感覺.
不算我問你自己,如果你曾經「想裝」的話.
你以前有沒有人「想裝」的呢.
我問一下,在流氓裡面「想裝」多少人,「想裝」的有嗎.
都有吧,OK.
你現在最記得是什麼,你記不記得你「想裝」做了什麼.
其實你是不記得的.
更加不要說你做了什麼改變,有什麼貢獻.
你最記得就是那班人,是吧.
那班人大家一起在掃訪裡.
都在那裡預備很多辛苦被老鬼在那裡搓,在那裡搶,很辛苦的東西.
但不知道為了什麼,不知道做什麼.
這個不是我自己的個人意見,我上網搜尋也是這樣寫的.
上網搜尋有一篇叫「大學生想裝七大原因」.
就是找一些大學生上「裝」給七個原因.
讀給大家聽,什麼七個原因叫人「想裝」呢.
第一,認識「裝」員.
認識人,整班大家是結盟的好死黨.
第二,學習如何運用軟技能.

$^{321}$就是學習,如何處理人際關係,這些東西.
第三,拓展人脈,就是認識多點人.
第四,這個很奇怪,就是展示自己的威風.
就是那些,那些政綱那些.
第六,就是拿縮分.
第七,先勁,出POOL,當然是為了出POOL.
這個就是七大原因裡,沒有一個是關「想裝」事的.
就是沒有一個原因是跟「想裝」有關係的.
當你做一件事,做一件事的重點不是做那件事.
而是為了認識人,學習,出POOL那些東西的時候.
整件事其實發覺做什麼都無所謂.
這個就是那個「想裝」的重點.
我不是負面,我不是對「想裝」有任何負面的看法.
說「想裝」不務正業,等於「吃」「做」.
相反,「想裝」的人是付出120\%的努力.
一起去做一些很熱血的事,而又沒有什麼果效的事.
這個就稱之為「想裝」.
當中回憶的就是過程,就是那份友誼.
不是那個結果.
「想裝」雖然是燒錢,燒時間,燒GPA.
但是你覺得都是值得的.
因為過程是不生難忘的.
我想這個就是「想裝」的意義.
暑假的時候,我和一群目者去了台灣退休.
有一晚我們在夜市那邊.
經過一間叫「小時候彈珠堂」的彈珠店.
這個彈珠店就彬彬自喜,也是古早味的彈珠店.
是賣一些回憶的,就是小時候玩的彈珠機.
有很多小朋友小時候玩的彈珠機.
怎麼玩呢?你創賣幾百元台幣.
一打到大堆的彈珠,就可以進彈珠機彈.
如果贏的話,就拿更多的彈珠.
然後就玩彈珠機.
就不斷地滾下去,其實是挺無聊的.
不斷地在那裡滾彈珠機,不斷地贏了.
然後就進去玩,就消費了很多時間.
那一群目者就在台灣退休了很多時間.
就在這個店裡.
不過有幾句小時候彈珠堂的口號.
我是很欣賞的,很有深度.

$^{361}$怎麼去賣這麼無聊的彈珠機呢?.
其中一頁是這樣說的.
很多事情都是這樣.
你認為值得,就是值得.
就像打彈珠一樣.
在台灣,文青感覺都出來了.
OK?.
很多事情都是這樣.
我認為值得就是值得.
就像打彈珠一樣.
是不是?那些那樣.
第二個更加有型.
就算很多事情沒有發生也沒有關係.
最重要的事情就是.
我們一起打彈珠,度過每一天.
OK的,OK的.
這種真肉,我覺得是很有意思的.
不斷地打彈珠是OK的.
大家一起在這裡打,繼續打.
重要的是大家有一個非常美好的一天.
這次目者退休.
我和一群目者在彈珠店裡玩.
我自己是一個很特別的屬靈的領悟.
所謂教會.
和打彈珠,或者像上妝一樣.
都是一些類似的地方.
一群人走在一起,做一些事情.
或者這群人做完之後.
對世界沒有什麼特別的改變.
但是,是OK的.
這個對我來說是一個很重要的信仰反省.
這次台灣退休,我們真的有工作.
我們參觀希望.
不是純粹參觀崇拜.
而是約到Peter和同工聊天.
我發現雖然看起來我們都相似.
Dukhuk和Full Church都是一個比較媒體活躍的教會.
不過其實我們兩間教會對於牧羊的理念不是很不同.
Dukhuk是一間不錯的教會.
但是它不會將很多的心血,目者的精力.

$^{401}$放在弟兄姊妹很個別,很細哀的生活細節裡.
Full Church是想花很多很多的心血和資源.
去做一些其他人不太理會的事情.
兩個弟兄姊妹吵架.
弟兄姊妹的頭暈身痛.
結婚生子,看演唱會,看醫生.
我們都覺得是一個牧羊的契機.
我們都會關心,我們都會關心,我們想知道.
你問我們做了這麼多事情,有沒有用呢?.
有沒有效呢?.
能不能改變世界呢?.
能不能改變弟兄姊妹呢?.
不要說改變不了,起碼現在未必能見到.
但是我們Full Church花了很多無比的心血.
去做一些未必一定很有結果,很多果效的事情.
其實牧羊就是這樣的東西.
你花很多時間去牧羊,但你沒有特別見到什麼果效.
如果教會真的只是做一些有效的事情.
或者一些事情都是為了有沒有效.
有沒有用,有沒有後果去做的話.
它就不是一間教會.
它只是一盤宗教生意.
我不是說結果不重要.
教會不只是計算結果.
我們都會想結果,但這不是我們唯一去想的東西.
牧羊,順遂給我們的使命,順遂給我們的命令.
這些從來都不是你要考慮太多結果,你要做的事情.
正如上次說,如果我留在長洲港島.
不開教會,我都OK的,我們YouTube的View都會.
但教會不是這樣,教會不應該這樣.
我們回到經文,十二月記經文.
我們看到故事其實沒有停在那裡.
大衛將三位勇士帶來的水,墊在地上.
然後大衛就做了一個這樣的舉動.
說了一句非常非常非常重要的話.
大衛強調他倒下來的水不是純粹倒在地上.
而是將這個水墊在耶和華面前.
然後耶和華說,只有三個人沒有死去打水.
水就像貪污的血一般.
原來一群人一起做的事情不一定是講求結果.

$^{441}$大衛將整件事情的結果的本質去改變.
或者說大衛重新去定義整件事情的本質.
將它變成一件信仰的事情.
這三個人無死去將水拿過來.
他們的血代表他們的生命.
他們的血被敲鍊在耶和華面前成為一場有生命的敬拜.
我們不能夠不承認,第一姐妹.
信仰從來都沒有什麼用.
我都說過,信仰活動對這個世界沒有什麼特別的貢獻.
對社會沒有什麼特別的GPD能夠得到出來.
我開電單車送外賣反而有些用.
起碼幫到人有得吃.
真的能夠對世界有貢獻和改變.
但你問我開一間教會出來對世界有什麼貢獻呢.
對這個社會從一個屬實來說有什麼好處呢.
不過,是可以的.
一群基督徒,將軍澳,沙頭江,元朗,沙田聚在一起去做一些事情.
做一些好像沒有什麼貢獻和改變世界的事情.
是可以的.
我知道很多人都曾經在網上聽過我這邊導.
《安法下的基督徒》,所以認識劉唐,甚至乎參加劉唐.
我在那邊導這樣說,就是羅曼十二章.
將身體獻上,當作活祭,是理所當然的敬拜.
我們基督徒的生命,我們基督徒的生活是一場敬拜.
你們如此敬拜,是理所當然的.
什麼叫敬拜呢.
所謂敬拜,就是你不用太理會他的成敗得失.
有沒有後果,最後會不會出來.
有沒有什麼用,不需要太介懷.
為什麼呢?因為他的本質就被定義為敬拜.
他就是呈現在耶和華面前的敬拜.
這個就是他做事的意義和本質.
甚至沒有一件信仰裡的事情能夠逃離到這個ultimate的上帝.
我們教會裡每一個人一起做的事.
都只能夠在他面前,作為敬拜來呈獻給他.
教會沒有一個行動不是敬拜.
當我們聚在一起,一起做一些事來附上我們的心血.
不是問那個結果,而是將我們的心血就這樣囂然在耶和華上帝的面前.
這篇道想大家做些什麼呢.
今天我日日提的是上莊.

$^{481}$將會想讓大家開始思考我們一群基督徒.
流淌的群體,我們聚在一起可以做些什麼.
在這個社會的氣壓之下.
在這個教會低迷的下降軌之下.
將你餘下的生命和你身邊的弟兄姊妹一起去獻給終極的上帝.
成為一種敬拜,無論你做什麼,做成怎樣都好.
一起做,這就是意義所在.
當然不是純粹叫大家去嘈嘈,什麼都不理,什麼都不問,什麼都不想.
而是當我們花很多心思去想,去預備,去做.
做出來就是比做不做得好重要.
沒問題.
最後我們說一下香港教會.
我知道今天這篇信息很多不同的教會的人都會聽.
你未必回流堂,你可能仍然會聽到這篇道.
19年之後,基本上教會是前所未見的艱難.
我自己在年初的時候做了一個香港教會前景會.
負責整個program.
我看到很多不同宗派的領袖,很多不同的人.
想香港教會之後會怎樣.
政治運動,疫情,移民潮.
香港教會人數大跌了三成以上,很多教會都是.
基本上崇拜人數下跌,牧者不在,弟兄姊妹不在.
一個很強的流失問題.
更重要的不是人數問題,而是士氣問題.
很多教會都不知道怎樣做.
連本身帶領的人都不在.
然後問教會怎樣做.
士氣下跌到一個非常差的地步.
面對這麼嚴峻的情況,香港教會有什麼可為呢.
經過這個前景會,很多的思考之後.
我得出一個結論.
在這麼艱難的環境下,做點事,搞點事.
無論做什麼都好,只要願意去做點事,搞點事.
這就是一個好的開始.
幸福小組也好,讀經講座也好.
堂慶聚餐也好,呼音報道聚會也好,教會大旅行也好.
總之一班人聚在一起,搞點事,做點事.
以前我曾經說過,劉韬,不搞大旅行.
我要收回這個說話.
對不起,我不是說我會做.

$^{521}$而是說,搞鬥行是OK的,沒問題的.
這件事本身是OK的.
一間教會裡面,這麼差的情況.
一班走剩的人,大家一起搞個旅行.
叫新朋友來,大家一起預備.
將這件事成為敬拜.
不要問教會的增長,怎樣怎樣都好.
不是不問,但這不是重點.
只要願意去做,做回教會要做的事.
做回教會應該要做的事.
這就是我們教會的責任.
在未來這麼多的星期裡,我們會一起去想.
教會,劉韬,我們自己的教會.
應該可以做些什麼.
但怎樣都好,我們將我們的生命.
囂顫在上帝面前,這是我們做的事.
我會祈禱.
祝你感情,因為你成為我們教會的元首.
你讓我們聚集.
你讓我們在世界,香港不同的角落.
我們連結,成為一個合一的群體.
這個群體不單是彩道.
我們願意一起去做一些事.
去做一些獻上給你的事.
我們求主你幫助我們.
讓我們教會,劉韬,成為一個行動的教會.
一個大家願意一起不斷積極.
去為主你發熱發光的群體.
為我們教會前所計劃的,所想的,所做的.
我們獻上給你,求主你閱立.
我們就明考 再見.
\newpage



\section{創世記 11:1-9-20230909}
\label{sec:n7jQq3kdpkc}
\textbf{【網上崇拜】試o下唔好咁?|創世記11\_1-9|20230909 [n7jQq3kdpkc]}
\newline
\newline
連結: \href{https://youtube.com/watch?v=n7jQq3kdpkc}{\texttt{ https://youtube.com/watch?v=n7jQq3kdpkc}} ~~~~ 語音日期: 2023-09-09 
\newline
\newline
\hyperref[sec:2n9NjI1RS9k]{\small{< < < PREV SERMON < < <}}
~
\hyperref[sec:index_chronic]{\small{[返順時目]}}
~
\hyperref[sec:index_scriptual]{\small{[返順卷目]}}
~
\hyperref[sec:h7oDnhukFbo]{\small{> > > NEXT SERMON > > >}}
\newline
\newline
創世記 11:1-9-20230909
\newline
\begin{longtable}{cl}
\hline
\hline
章節 & 經文 (和合本修訂版)\\
\hline
11:1 & \begin{tabularx}{0.7\textwidth}{X} 那時,全地只有一種語言,都說一樣的話。 \end{tabularx} \\ \\ \relax
11:2 & \begin{tabularx}{0.7\textwidth}{X} 他們向東遷移的時候,在示拿地找到一片平原,就住在那裡。 \end{tabularx} \\ \\ \relax
11:3 & \begin{tabularx}{0.7\textwidth}{X} 他們彼此商量說:「來,讓我們來做磚,把磚燒透了。」他們就拿磚當石頭,又拿柏油當泥漿。 \end{tabularx} \\ \\ \relax
11:4 & \begin{tabularx}{0.7\textwidth}{X} 他們說:「來,讓我們建造一座城和一座塔,塔頂通天。我們要為自己立名,免得我們分散在全地面上。」 \end{tabularx} \\ \\ \relax
11:5 & \begin{tabularx}{0.7\textwidth}{X} 耶和華降臨,要看世人所建造的城和塔。 \end{tabularx} \\ \\ \relax
11:6 & \begin{tabularx}{0.7\textwidth}{X} 耶和華說:「看哪,他們成了同一個民族,都有一樣的語言。這只是他們開始做的事,現在他們想要做的任何事,就沒有甚麼可攔阻他們了。 \end{tabularx} \\ \\ \relax
11:7 & \begin{tabularx}{0.7\textwidth}{X} 來,我們下去,在那裡變亂他們的語言,使他們彼此語言不通。」 \end{tabularx} \\ \\ \relax
11:8 & \begin{tabularx}{0.7\textwidth}{X} 於是耶和華使他們從那裡分散在全地面上;他們就停止建造那城了。 \end{tabularx} \\ \\ \relax
11:9 & \begin{tabularx}{0.7\textwidth}{X} 因為耶和華在那裡變亂了全地的語言,把人從那裡分散在全地面上,所以那城名叫巴別。 \end{tabularx} \\ \\ \relax
11:10 & \begin{tabularx}{0.7\textwidth}{X} 這是閃的後代。洪水以後二年,閃一百歲生了亞法撒。 \end{tabularx} \\ \\ \relax
11:11 & \begin{tabularx}{0.7\textwidth}{X} 閃生亞法撒之後又活了五百年,並且生兒育女。 \end{tabularx} \\ \\ \relax
11:12 & \begin{tabularx}{0.7\textwidth}{X} 亞法撒活到三十五歲,生了沙拉。 \end{tabularx} \\ \\ \relax
11:13 & \begin{tabularx}{0.7\textwidth}{X} 亞法撒生沙拉之後又活了四百零三年,並且生兒育女。 \end{tabularx} \\ \\ \relax
11:14 & \begin{tabularx}{0.7\textwidth}{X} 沙拉活到三十歲,生了希伯。 \end{tabularx} \\ \\ \relax
11:15 & \begin{tabularx}{0.7\textwidth}{X} 沙拉生希伯之後又活了四百零三年,並且生兒育女。 \end{tabularx} \\ \\ \relax
11:16 & \begin{tabularx}{0.7\textwidth}{X} 希伯活到三十四歲,生了法勒。 \end{tabularx} \\ \\ \relax
11:17 & \begin{tabularx}{0.7\textwidth}{X} 希伯生法勒之後又活了四百三十年,並且生兒育女。 \end{tabularx} \\ \\ \relax
11:18 & \begin{tabularx}{0.7\textwidth}{X} 法勒活到三十歲,生了拉吳。 \end{tabularx} \\ \\ \relax
11:19 & \begin{tabularx}{0.7\textwidth}{X} 法勒生拉吳之後又活了二百零九年,並且生兒育女。 \end{tabularx} \\ \\ \relax
11:20 & \begin{tabularx}{0.7\textwidth}{X} 拉吳活到三十二歲,生了西鹿。 \end{tabularx} \\ \\ \relax
11:21 & \begin{tabularx}{0.7\textwidth}{X} 拉吳生西鹿之後又活了二百零七年,並且生兒育女。 \end{tabularx} \\ \\ \relax
11:22 & \begin{tabularx}{0.7\textwidth}{X} 西鹿活到三十歲,生了拿鶴。 \end{tabularx} \\ \\ \relax
11:23 & \begin{tabularx}{0.7\textwidth}{X} 西鹿生拿鶴之後又活了二百年,並且生兒育女。 \end{tabularx} \\ \\ \relax
11:24 & \begin{tabularx}{0.7\textwidth}{X} 拿鶴活到二十九歲,生了他拉。 \end{tabularx} \\ \\ \relax
11:25 & \begin{tabularx}{0.7\textwidth}{X} 拿鶴生他拉之後又活了一百一十九年,並且生兒育女。 \end{tabularx} \\ \\ \relax
11:26 & \begin{tabularx}{0.7\textwidth}{X} 他拉活到七十歲,生了亞伯蘭、拿鶴和哈蘭。 \end{tabularx} \\ \\ \relax
11:27 & \begin{tabularx}{0.7\textwidth}{X} 這是他拉的後代。他拉生亞伯蘭、拿鶴和哈蘭;哈蘭生羅得。 \end{tabularx} \\ \\ \relax
11:28 & \begin{tabularx}{0.7\textwidth}{X} 哈蘭死在他父親他拉的面前,死在他的出生地迦勒底的吾珥。 \end{tabularx} \\ \\ \relax
11:29 & \begin{tabularx}{0.7\textwidth}{X} 亞伯蘭、拿鶴各娶了妻。亞伯蘭的妻子名叫撒萊,拿鶴的妻子名叫密迦,是哈蘭的女兒。哈蘭是密迦和亦迦的父親。 \end{tabularx} \\ \\ \relax
11:30 & \begin{tabularx}{0.7\textwidth}{X} 撒萊不生育,沒有孩子。 \end{tabularx} \\ \\ \relax
11:31 & \begin{tabularx}{0.7\textwidth}{X} 他拉帶著他兒子亞伯蘭和他孫子,哈蘭的兒子羅得,以及他的媳婦,亞伯蘭的妻子撒萊,一同出了迦勒底的吾珥,要往迦南地去;他們來到哈蘭,就住在那裡。 \end{tabularx} \\ \\ \relax
11:32 & \begin{tabularx}{0.7\textwidth}{X} 他拉共活了二百零五年,就死在哈蘭。 \end{tabularx} \\ \\
[1ex]
\hline
\hline
\end{longtable}
$^{1}$大家好,我是Flow Church的牧者Victor.
今天是我來為大家講閱題第二講.
我們閱題是「上莊」.
先看一看.
因為昨天牧者會的時候.
潘Sir很害怕一件事.
就是我的組員舉牌.
不然之後的牧者會很尷尬.
我自己也害怕.
先回到正題.
上星期John問大家有沒有上過莊.
我自己沒有上過.
大概我也知道上莊是什麼.
就是一班人一起做點事情.
做點事情的對象是什麼呢.
在大學當然是為校園做點事情.
今天我們說要一班人一起做點事情.
對象是什麼呢.
應該是教會,社區,世界.
我就想,其實我自己有沒有做過這些事情呢.
我就想,好像來來去去一班人一起都是玩.
比如去旅行,打球,吃火鍋.
都是去玩的.
有沒有一些有意義的東西呢.
我不是很想到.
然後我就嘗試拿我的手機出來.
相簿通常有些合照.
我就在這裡錄下錄下錄.
不停地錄不停地錄.
最後找到一張相.
這張相是我和一班牧者.
在今年5月6日星期六.
在中環碼頭拍的一張合照.
如果你記得那一天.
其實就是我們Flow Church的第一次戶外崇拜.
那天我記得天氣是有點悶熱的.
好像在下雨.
最重要是旁邊很嘈雜.
因為有周杰倫的世界演唱會.
但這一切都沒有了我們一班人.

$^{41}$台上台下幕前幕後.
四百多五百人.
我們一起藉著詩歌,祈禱去敬拜神.
我會說這是一件對教會,社區,世界很有意義的事情.
你有沒有想過原來你的參與.
已經是做一件這麼熱血,這麼童心.
是一件這麼榮耀主義的事.
不過聖經同樣也有一件事.
這件事也是一班人.
一起很童心,很熱血,很有使命感去做一件事.
偏偏就是這件事令他們得罪神.
然後他們被神審判.
我們一起看看我們很熟悉的記載.
《巴別塔》.
它出自《創世紀》十一章一至九節.
那時全地只有一種語言.
都說一樣的話.
他們向東遷移的時候.
在士那地找到一片平原.
就住在那裡.
他們彼此商量說.
來讓我們來做磚.
把磚燒透了.
他們就拿磚當石頭.
又拿柏油當禮章.
他們說來讓我們建造一座城和一座塔.
塔頂通天.
我們要為自己立名.
免得我們分散在全地面上.
又說光臨要看世人所建造的城和塔.
又說看啊.
他們成了同一個民族.
都有一樣的語言.
這只是他們開始做的事.
現在他們想要做的任何事.
就沒有什麼可難阻他們了.
來我們下去在那裡.
變亂他們的語言.
使他們彼此言語不通.
於是耶和華使他們從那裡.

$^{81}$分散在全地面上.
他們就停止建造那城了.
因為耶和華在那裡.
變亂了全地的言語.
把人從那裡分散在全地面上.
所以那城名叫巴別.
我們一起祈禱.
我們今天能夠一同同聚.
我們去敬拜你.
我們能夠敬拜你.
我們需要很多的智慧.
很需要我們去專一.
我們盡心盡性去敬拜你.
求你今天藉著巴別塔這件事.
去提醒我們.
當我們一同去做事的時候.
我們需要留意自己有沒有問題.
使我們原來會得罪你.
主啊我們就這樣去祈求.
求你的聖靈在當中光照我們.
帶領我們.
多謝你聽祈禱.
奉耶穌名祈求.
阿們.
《創世紀》其實主要分了兩個部分.
第一個部分就是第一章至十一章.
在第一章至十一章裡.
三到十一章其實就是說.
人類如何去犯罪得罪神.
以及他們如何被神審判.
其實大家都很熟悉的.
我們想想第三章亞當夏娃.
他們違背神的命令.
然後去到第四章.
人類第一宗的殺人事件.
該人殺了亞伯.
去到第六章就已經說人類.
邪惡到一個極點.
然後上帝藉著洪水滅世.
去到第九章就是嘉楠.

$^{121}$她對著她爸爸羅亞.
做一些對不起她的事.
雖然聖經沒有明明地寫.
而去到今天我們看的巴別塔.
就是在第十一章.
也是在說這個部分的結尾.
所以巴別塔不是跟我們說.
為什麼人會有不同的語言.
但更重要的是.
他要讓我們看到巴別塔這群人.
他們如何得罪神然後被審判.
經文有些位置可以讓我們留意到.
他們犯了什麼罪.
我們要想想巴別塔.
為什麼一群人一起建造.
然後得罪了神.
什麼原因呢.
經文有些位置可以讓我們看到.
我們未必看到的東西.
為什麼我這樣說呢.
正正就是因為我們不是當時的人.
如果我們看經文.
當時的人看經文.
他看到士拿,造磚,塔頂通天.
他們馬上想到的東西.
就是一個在他們附近.
米索不達米亞.
就是這個地方的一座宗教的廟塔.
這種廟塔是屬於巴比倫宗教系統的產物.
這些廟塔是怎樣的呢.
我們來看一下.
你會看到兩張相片.
你看到這些廟塔其實都有三個特徵.
第一個特徵就是他們都是用磚來造的.
雖然相片很小.
你未必看到.
用磚來造的.
好像樓梯一樣慢慢鋪上去.
第二個特徵就是他們都形容頂部是塔頂通天.
塔頂通天的意思不是說真的貫穿了天.

$^{161}$而是說是一個神明的門.
是神明居所的入口.
第三樣東西是很重要的.
在這些廟塔旁邊是有一座神廟.
大家可以想一下.
為什麼會有這樣的設計呢.
學者就會這樣認為.
他說這一座廟塔是用來獻給他們的神明.
那一個城的神明.
然後這個廟就是貫穿著天地.
神就可以在入口進去.
然後走樓梯走走走走走走走走.
然後旁邊就有一座廟.
他就進去接受人的敬拜.
之後他就去祝福人.
我自己小時候回教會的時候.
查經查巴別塔就問.
究竟巴別塔那班人他們犯了什麼罪.
以前教會的導師就說他們驕傲.
因為他們想建這座塔好像神那麼高.
以為自己好像神一樣.
你聽完這個考古或者學者的研究之後.
你有沒有一種覺得其實他們都很謙卑.
你想一下他們建了城還立刻建了塔給神明.
還很虔誠.
所以我們看到的經文其實不是說這一班人驕傲.
我們又要問不是驕傲他們犯了什麼罪.
OK 舊約常常說什麼.
以色列人壞什麼.
拜假神 拜偶像.
他們應該是拜假神拜偶像.
又有一個問題.
第六節.
對不起 按錯了.
不好意思.
第六節說什麼.
上帝說 這是他們開始做的事.
現在他們要做的任何事就沒有什麼可難阻他們了.
很複雜 我嘗試用我的意義.
說這班人今天只是建城建塔之後還得了.

$^{201}$似乎這句給了一個提示我們.
不是我們不應該去建城建塔.
我們應該去建城建塔.
這句給了一個提示我們.
不是在乎他們是否在敬拜假神.
是在說他們的目的有問題.
他們的目的是什麼.
大家很心碎清 因為經文有說.
經文說他們要為自己立名.
免得我們分散在全地面上.
我們嘗試研究一下他們的目的是否有問題.
我們要為自己立名.
我們要建城建塔為自己立名.
就是想自己出名.
出名有沒有問題.
如果你出道去旅行.
你跟別人說 人們會問你來自哪裡.
然後你會回答 香港人.
他不知道香港是哪裡.
那會有問題嗎.
他以為你是中國人.
那就不好意思.
所以這群人想自己出名.
有人知道他們的存在.
也很合理.
第二件事 免得我們分散在全地面上.
什麼叫分散在全地面上.
分散這個字.
如果我們看舊約多一點.
我們會知道他很多時候是在說軍事的用語.
就是當他們打仗輸了.
聖經就會形容他們被分散.
摩西曾經向耶和華祈求.
求神打散這群敵人.
所以如果我們這樣去想.
我們就會知道原來這群人.
他們不想分散.
他們要為自己立名.
免得分散在全地面上.
他們的目的似乎是.

$^{241}$我想自己出名一點.
我不想被外敵入侵.
那我問你.
如果你是這群人.
其實有這個想法是否很合理.
是不是.
那怎麼搞.
這個又不是那個不是.
他們犯什麼罪呢.
我們想想.
其實他們為什麼要建一個塔去敬拜神.
古時其實那些人.
他們都有一個概念.
他們的概念就是.
去敬拜神之後得到祝福.
他們很期望在他們自己的身體.
經濟和環境得到祝福.
我們試一下這樣想.
我們一起建一座城.
然後我們中間就要建這個塔.
然後我們敬拜上帝.
上帝就透過這座塔.
祝福這個城市.
然後這個城市會怎樣.
就會發達起來.
強盛起來.
然後這群人就會出名.
是不是.
這個就是我在想的東西.
但你想不想到這件事裡面.
有什麼問題.
原來他們在想的是.
怎樣將神明的祝福.
計算在他們的利益裡面.
他們那份自我.
大家看到嗎.
原來當人在創世紀三章.
他們吃了禁果.
他們犯罪得罪神之後.
他們開始敬拜的再不是神.

$^{281}$他們在敬拜自己.
他們裡面那份自我.
驅使到他們做很多的事.
都是為自己著想.
這群建巴比塔的人.
他們最大的問題.
其實是裡面那份自我.
他們怎樣將神明的祝福.
計算在他們的利益裡面.
正正是這份自我.
使得上帝要出手.
阻止他們所做的事.
這份自我.
在我們生命裡面.
不斷不斷地慢慢擴大.
我們的野心會越來越大.
然後我們開始做的事.
就會越來越惡.
所以上帝對這群人的評價是.
你們今天只是在建這座城.
在建這座塔.
你們想要做的事.
之後就沒有人可以阻止你們.
所以今天我要跟大家說的一個課題.
是自我.
自我是什麼呢.
我就嘗試看看chatGBT.
我也是抄一下John.
自我中心是指一個人在思想,情感和行為上.
過度關注自己的需求,利益和權益.
而不是排斥他人的需求和感受.
自我的人通常會以自我為中心.
將自己的利益置於他人之上.
巴別塔那群人就將利益置在他們的神明.
缺乏對他人的關注同理心和尊重.
所以我們看到這群建巴別塔的那群人.
他們那份自我,那份計算.
最終不是單純地敬拜神.
這是他們的問題.
當我看完這個chatGBT之後.

$^{321}$我就想起自己的一個經歷.
這個經歷不是一個開心的經歷.
大家千萬不要笑.
OK 大家笑了.
我自己很小的時候.
不要說很小.
中學的時候就接觸攝影.
那時候我自己很喜歡一個菲林牌子的鏡頭.
後來我就經常去看這個牌子的鏡頭.
去研究一下.
看看他有什麼漂亮的鏡頭.
這個牌子其實在很久很久以前.
2000年的時候已經停產了.
所以如果要找回這些鏡頭出來.
都要在二手市場裡面找.
我自己就有個習慣.
有能力出來工作之後.
就時不時看一下有沒有一些好東西.
因為停產了.
有些罕有的鏡頭.
基本上都已經沒有人再放出來.
早前我就看到有一支很罕有的鏡頭.
而且價錢是低於市價.
有人說了.
忠實粉絲當然衝出去.
我初步檢查完之後就沒有問題.
就拿回家了.
如無意外就出意外了.
當我再認真去看一下這支鏡頭的時候.
我就發現他的對焦有問題.
這個時刻我心裡面就有一個意念.
大家猜猜這個意念是什麼.
啊 沒錯.
大家和我都很相似.
就是不要告訴別人有問題.
我心裡面第一個意念就是不要告訴別人有問題.
然後放出去.
然後我有第二個意念.
大家猜猜是什麼.
大家很善良.

$^{361}$我第二個意念是市價買出去.
我去到第三個意念.
我才去想基督徒不應該是這樣的.
但我很老實告訴大家.
這就是自我.
我們想的是自己的利益.
巴比塔這群人.
他們把神明的祝福都計算在自己的利益裡面.
今天我們也許不會像他們那樣做.
但同樣的自我.
會不會都成為我們信仰裡面的一個難阻.
刺刀嗎?.
無所謂啦.
崇拜?.
一次滿錢沒有回.
OK的.
奉獻嗎?.
有錢先吧.
接納?.
不行.
饒恕?.
瘋了嗎?.
一群人一起工作?.
不要騷擾我了.
大家看到嗎?.
我不知道這些說話會不會都曾經在你心裡面響起過.
上星期包惟鯤牧師來了香港.
幫播道神學院講一個講座.
裡面有一句說話令我很深刻.
他說今天傳福音最大的難阻.
是教會不能夠活出基督的身體.
教會不能夠活出基督的身體.
一群教會的人沒有耶穌基督的樣式.
所以他說現在美國不是再傳一個倫正的福音去辯明.
耶穌基督真的復活了.
祂真是那位主.
他說現在是建立一個群體.
這個群體是有男有女.
有不同的種族,不同的階層.
他們彼此接納,彼此饒恕,彼此相愛.

$^{401}$他們活出基督的樣式.
他們要用這個群體成為一個見證,成為一個報道會.
不知道大家覺得Flow Church能不能成為一間這樣的教會.
如果我們縱容自己裡面的自我.
我們未必可以做到很多聖經要我們做的事情.
我們的生命未必可以活現到基督.
同樣地我們的教會未必可以成為基督的身體的功效.
我們被人看到這是一群真真實實的信仰群體.
巴比塔這件事讓我們看到.
這群人的自我,他們的計算正在得罪神.
今天這份自我,這份計算同樣地阻礙了很多信徒親近神.
不知道今晚聖靈有沒有在你們心裡有回響.
讓你看見我們自己很自然地也有一些自我的在裡面.
很自然地正如我剛才的第一個念頭和第二個念頭.
都不是好事來的.
但是我們願不願意活著基督.
我們願意因著福音走多一步.
我們願意好好對抗我們裡面的自我.
我們要提醒一群人一起做一些事情.
我們做的事情其實正正就是想活現基督.
我們正正想讓這個群體有基督的樣式在裡面.
正正需要大家好好去面對我們每一個人裡面的那一份自我.
我們一起祈禱吧.
讓我們看見昔日的人類怎樣去犯罪,得罪你.
我們每一個人都很軟弱.
我們裡面的自我都很容易,很自然地走出來.
同樣地今天我們去祈禱.
求神來光照我們.
讓我們看見自己仍然有需要進步的地方.
我們需要好好對抗我們裡面那一份的自我.
好使我們能夠在這個世上去見證基督.
好讓我們的生命,好讓我們的教會.
可以成為你使用的器皿.
將福音帶進這個黑暗的世代.
求主你的聖靈光照,軟化我們的心.
多謝你聽祈禱.
奉耶穌基督,明知其求.
阿波羅網記者:Mira.
採訪撰稿/金汝鑫 攝影剪輯/蘇顯榮.
\newpage



\section{路加福音 9:18-36-20230916}
\label{sec:h7oDnhukFbo}
\textbf{【網上聖餐崇拜】一齊做埋D 無聊ye 先至明白有乜意義|路加福音9\_18-36|20230916 [h7oDnhukFbo]}
\newline
\newline
連結: \href{https://youtube.com/watch?v=h7oDnhukFbo}{\texttt{ https://youtube.com/watch?v=h7oDnhukFbo}} ~~~~ 語音日期: 2023-09-16 
\newline
\newline
\hyperref[sec:n7jQq3kdpkc]{\small{< < < PREV SERMON < < <}}
~
\hyperref[sec:index_chronic]{\small{[返順時目]}}
~
\hyperref[sec:index_scriptual]{\small{[返順卷目]}}
~
\hyperref[sec:5EgvGimlwXk]{\small{> > > NEXT SERMON > > >}}
\newline
\newline
路加福音 9:18-36-20230916
\newline
\begin{longtable}{cl}
\hline
\hline
章節 & 經文 (和合本修訂版)\\
\hline
9:18 & \begin{tabularx}{0.7\textwidth}{X} 耶穌獨自禱告的時候,門徒也同他在那裡。耶穌問他們:「眾人說我是誰?」 \end{tabularx} \\ \\ \relax
9:19 & \begin{tabularx}{0.7\textwidth}{X} 他們回答:「是施洗的約翰;有人說是以利亞;還有人說是古時的一個先知又活了。」 \end{tabularx} \\ \\ \relax
9:20 & \begin{tabularx}{0.7\textwidth}{X} 耶穌問他們:「你們說我是誰?」彼得回答:「是神所立的基督。」 \end{tabularx} \\ \\ \relax
9:21 & \begin{tabularx}{0.7\textwidth}{X} 耶穌切切吩咐他們,命令他們不可把這事告訴任何人; \end{tabularx} \\ \\ \relax
9:22 & \begin{tabularx}{0.7\textwidth}{X} 又說:「人子必須受許多的苦,被長老、祭司長和文士棄絕,並且被殺,第三天復活。」 \end{tabularx} \\ \\ \relax
9:23 & \begin{tabularx}{0.7\textwidth}{X} 耶穌又對眾人說:「若有人要跟從我,就當捨己,天天背起自己的十字架來跟從我。 \end{tabularx} \\ \\ \relax
9:24 & \begin{tabularx}{0.7\textwidth}{X} 因為凡要救自己生命的,必喪失生命;凡為我喪失生命的,他必救自己的生命。 \end{tabularx} \\ \\ \relax
9:25 & \begin{tabularx}{0.7\textwidth}{X} 人就是賺得全世界,卻喪失了自己,或賠上自己,有甚麼益處呢? \end{tabularx} \\ \\ \relax
9:26 & \begin{tabularx}{0.7\textwidth}{X} 凡把我和我的道當作可恥的,人子在自己的榮耀裡,和天父與聖天使的榮耀裡來臨的時候,也要把那人當作可恥的。 \end{tabularx} \\ \\ \relax
9:27 & \begin{tabularx}{0.7\textwidth}{X} 我實在告訴你們,站在這裡的,有人在沒經歷死亡以前,必定看見神的國。」 \end{tabularx} \\ \\ \relax
9:28 & \begin{tabularx}{0.7\textwidth}{X} 說了這些話以後約有八天,耶穌帶著彼得、約翰、雅各上山去禱告。 \end{tabularx} \\ \\ \relax
9:29 & \begin{tabularx}{0.7\textwidth}{X} 正禱告的時候,他的面貌改變了,衣服潔白放光。 \end{tabularx} \\ \\ \relax
9:30 & \begin{tabularx}{0.7\textwidth}{X} 忽然有摩西和以利亞兩個人同耶穌說話; \end{tabularx} \\ \\ \relax
9:31 & \begin{tabularx}{0.7\textwidth}{X} 他們在榮光裡顯現,談論耶穌去世的事,就是他在耶路撒冷將要完成的事。 \end{tabularx} \\ \\ \relax
9:32 & \begin{tabularx}{0.7\textwidth}{X} 彼得和他的同伴都打盹,但一清醒,就看見耶穌的榮光和與他一起站著的那兩個人。 \end{tabularx} \\ \\ \relax
9:33 & \begin{tabularx}{0.7\textwidth}{X} 二人正要和耶穌分離的時候,彼得對耶穌說:「老師,我們在這裡真好!我們來搭三座棚,一座為你,一座為摩西,一座為以利亞。」他卻不知道自己在說些甚麼。 \end{tabularx} \\ \\ \relax
9:34 & \begin{tabularx}{0.7\textwidth}{X} 說這些話的時候,有一朵雲彩來遮蓋他們;他們一進入雲彩就很懼怕。 \end{tabularx} \\ \\ \relax
9:35 & \begin{tabularx}{0.7\textwidth}{X} 有聲音從雲彩裡出來,說:「這是我的兒子,我所揀選的。你們要聽從他!」 \end{tabularx} \\ \\ \relax
9:36 & \begin{tabularx}{0.7\textwidth}{X} 聲音停止後,只見耶穌獨自一人。當那些日子,門徒保持沉默,不把所看見的事告訴任何人。 \end{tabularx} \\ \\ \relax
9:37 & \begin{tabularx}{0.7\textwidth}{X} 第二天,他們下了山,有一大群人來迎見耶穌。 \end{tabularx} \\ \\ \relax
9:38 & \begin{tabularx}{0.7\textwidth}{X} 其中有一人喊著說:「老師!求你看看我的兒子,因為他是我的獨子。 \end{tabularx} \\ \\ \relax
9:39 & \begin{tabularx}{0.7\textwidth}{X} 他被靈拿住就突然喊叫,那靈又使他抽風,口吐白沫,並且重重地傷害他,不輕易放過他。 \end{tabularx} \\ \\ \relax
9:40 & \begin{tabularx}{0.7\textwidth}{X} 我求過你的門徒把那靈趕出去,他們卻不能。」 \end{tabularx} \\ \\ \relax
9:41 & \begin{tabularx}{0.7\textwidth}{X} 耶穌回答:「唉!這又不信又悖謬的世代啊,我和你們在一起,忍耐你們,要到幾時呢?把你的兒子帶到這裡來!」 \end{tabularx} \\ \\ \relax
9:42 & \begin{tabularx}{0.7\textwidth}{X} 他正來的時候,那鬼把他摔倒,使他重重地抽風。耶穌斥責那污靈,把孩子治好了,交給他父親。 \end{tabularx} \\ \\ \relax
9:43 & \begin{tabularx}{0.7\textwidth}{X} 眾人都詫異神的大能。 \end{tabularx} \\ \\ \relax
9:44 & \begin{tabularx}{0.7\textwidth}{X} 「你們要把這些話聽進去,因為人子將要被交在人手裡。」 \end{tabularx} \\ \\ \relax
9:45 & \begin{tabularx}{0.7\textwidth}{X} 門徒卻不明白這話,其中的意思對他們隱藏著,使他們不能明白,他們也不敢問這話的意思。 \end{tabularx} \\ \\ \relax
9:46 & \begin{tabularx}{0.7\textwidth}{X} 門徒互相議論,他們中間誰最大。 \end{tabularx} \\ \\ \relax
9:47 & \begin{tabularx}{0.7\textwidth}{X} 耶穌看出他們心中的議論,就領一個小孩子來,叫他站在自己旁邊, \end{tabularx} \\ \\ \relax
9:48 & \begin{tabularx}{0.7\textwidth}{X} 對他們說:「凡為我的名接納這小孩子的,就是接納我;凡接納我的,就是接納那差我來的。你們中間最小的,他就是最大的。」 \end{tabularx} \\ \\ \relax
9:49 & \begin{tabularx}{0.7\textwidth}{X} 約翰回應說:「老師,我們看見一個人奉你的名趕鬼,我們就阻止他,因為他不與我們一同跟從你。」 \end{tabularx} \\ \\ \relax
9:50 & \begin{tabularx}{0.7\textwidth}{X} 耶穌對他說:「不要阻止他,因為不抵擋你們的,就是幫助你們的。」 \end{tabularx} \\ \\ \relax
9:51 & \begin{tabularx}{0.7\textwidth}{X} 耶穌被接上升的日子將到,他決定面向耶路撒冷走去。 \end{tabularx} \\ \\ \relax
9:52 & \begin{tabularx}{0.7\textwidth}{X} 他打發使者在他前頭走;他們進了撒瑪利亞的一個村莊,要為他作準備。 \end{tabularx} \\ \\ \relax
9:53 & \begin{tabularx}{0.7\textwidth}{X} 那裡的人不接待他,因為他面向著耶路撒冷去。 \end{tabularx} \\ \\ \relax
9:54 & \begin{tabularx}{0.7\textwidth}{X} 他的門徒雅各和約翰看見了,就說:「主啊!你要我們吩咐火從天上降下來,燒滅他們嗎?」 \end{tabularx} \\ \\ \relax
9:55 & \begin{tabularx}{0.7\textwidth}{X} 耶穌轉身責備兩個門徒。 \end{tabularx} \\ \\ \relax
9:56 & \begin{tabularx}{0.7\textwidth}{X} 於是他們就往別的村莊去了。 \end{tabularx} \\ \\ \relax
9:57 & \begin{tabularx}{0.7\textwidth}{X} 他們在路上走的時候,有一個人對耶穌說:「你無論往哪裡去,我都要跟從你。」 \end{tabularx} \\ \\ \relax
9:58 & \begin{tabularx}{0.7\textwidth}{X} 耶穌對他說:「狐狸有洞,天空的飛鳥有窩,人子卻沒有枕頭的地方。」 \end{tabularx} \\ \\ \relax
9:59 & \begin{tabularx}{0.7\textwidth}{X} 他又對另一個人說:「來跟從我!」那人說:「主啊,容許我先回去埋葬我的父親。」 \end{tabularx} \\ \\ \relax
9:60 & \begin{tabularx}{0.7\textwidth}{X} 耶穌對他說:「讓死人埋葬他們的死人,你只管去傳講神的國。」 \end{tabularx} \\ \\ \relax
9:61 & \begin{tabularx}{0.7\textwidth}{X} 又有一人說:「主啊,我要跟從你,但容許我先去辭別我家裡的人。」 \end{tabularx} \\ \\ \relax
9:62 & \begin{tabularx}{0.7\textwidth}{X} 耶穌對他說:「手扶著犁向後看的人,不配進神的國。」 \end{tabularx} \\ \\
[1ex]
\hline
\hline
\end{longtable}
$^{1}$(丁字妹晚安).
剛才在前面坐.
聽到後面的聲音很好聽.
丁字妹唱歌的聲音很好聽.
不知道是跟歌聲還是你們的聲音有關.
應該是兩個聲音吧.
今天講道是有些難度的.
因為上星期有個家人離開了.
看著他離開.
其實是很不容易的經歷.
七,八個小時看著他.
呼吸越來越弱到他心跳完.
其實真的很艱難.
所以這十天都很想不說話.
昨天是我自己的老師楊醫世世三周年.
昨天看了照片聽了他說話.
今天又去了安息禮拜.
所以希望今天要做到要做到的事.
因為今天要講上妝.
我們先看PowerPoint.
一起做無聊的事才明白有什麼意義.
為了大家開心一點.
我先出賣自己.
我們看下張PowerPoint.
John說上妝希望大家拿出大學時期上妝的樣子.
但我知道上兩堂課都沒有拿到.
所以我先出賣自己.
這應該是我大學上妝時的照片.
因為今天要上山.
因為今天要上深水Po .
所以我特意選了登山變相.
所以你會發現那個樣子和現在的樣子是變了.
所以我選了這段經文.
下一張了我們笑完了.
我們看經文.
經文是這樣說的.
樹鳥姐說以後約有八天.
耶穌帶著彼得約翰雅各上山來禱告.
正禱告的時候他的面貌改變了.
衣服潔白放光.

$^{41}$忽然有摩西和伊利亞兩個人和耶穌說話.
下一個PowerPoint.
耶穌在榮光裡顯現.
談論耶穌去世的事.
就是他在耶路撒冷將要完成的事.
彼得和他的同伴睏了.
但清醒的時候看見耶穌的榮光.
和他一起站著後面的兩個人.
就是伊利亞和摩西.
兩個人正要和耶穌分離的時候.
彼得就和耶穌說.
老師我們在這裡真的好了.
我們來搭三座棚.
一座棚為你一座棚為摩西.
另一座是為伊利亞.
其實他們都不知道自己在說什麼.
下一張.
教導者說話的時候.
雲彩來遮蓋他們.
他們一進到雲彩就害怕.
有聲音從雲彩裡出來說.
這是我的兒子.
我所揀選你們要聽從他.
聲音停止之後.
只見耶穌獨自一人.
當那些日子.
門徒保持沉默.
不把他所看見的事告訴任何人.
我們再下下一張.
這句剛才讀了.
我們再下下一張看這幅圖.
這幅圖.
你見到有一個法板.
那個是摩西.
舉起兩隻手的是伊利亞.
其實這幅圖是從哪裡來的呢.
其實是來自以色列的他博山.
他博山相傳是耶穌登山變相的地方.
當然有很多討論.
或者很多人未必認同.

$^{81}$其實登山變相的地方是在他博山.
但他博山的確有一個所謂登山變相的紀念地方.
裡面很美.
但很奇怪的是.
進去的時候.
你不正門進去的話.
你從兩邊側門進去的話.
就會見到兩個小的教堂.
一個就是這兩幅圖.
一邊是伊利亞.
另一邊是摩西.
今時今日你進去也是.
所以我帶團的時候.
去到這個地方的時候.
我經常都笑.
明明剛才聖經說.
不要有三座棚.
這幅圖告訴大家.
在他博山紀念這件事情的時候.
其實他犯了彼得的時候的錯誤.
真的起三座棚.
一座是紀念中間的耶穌.
左右兩邊的是紀念摩西和伊利亞.
所以我經常笑.
去到那個時候.
根本就是耶穌說.
他們不知道自己在做什麼的時候.
但實際上.
你說一千多二千年後的人.
都是起三座棚去紀念這件事情.
所以如果這樣說的話.
其實門徒完全不知道在他博山.
或者其他.
不知道是不是他博山的登山變相.
他們完全不知道發生什麼事.
為什麼要見到伊利亞.
為什麼要見到摩西.
對於他們來說.
整個三年多的歷程裡.
跟著耶穌.

$^{121}$他們不是很掌握發生什麼事.
也不明白發生什麼事.
所以我們看下一個.
我想輕輕有一個小的結論.
我們按下一張powerpoint.
你看到我寫street memories.
所以我又要說一些無聊的東西.
好我們按下一張.
其實他們都不知道.
他們做了什麼事情.
我們再按多一張.
麻煩你.
其實我想第一個point是.
其實我們經常以為.
我們做很多事情.
都以為自己知道.
在做一些很有意義的事情.
這個是基督教一個很大的taboo.
就是說我要怎樣做.
我要有神的心意.
神的聲音跟我說.
我才知道我在做什麼.
這個很成為了我們.
很多基督徒生涯裡的一個枷鎖.
或者一個奏訟.
好像有些事情.
我不是聽得很明白.
上帝在做什麼的時候.
其實我就不敢做.
但如果你認清整個門徒.
跟耶穌經歷了很多事情的時候.
坦白說.
他在五病二愈.
都不知道耶穌在做什麼.
如果你看他.
那個什麼.
高沒耶穌那個女人.
那些珍娜大香膏.
高沒耶穌的時候.
你都跟別人發現.

$^{161}$其實門徒都不知道.
為什麼這個女人要這樣做.
更遑論耶穌要利海.
耶穌要水上行的時候.
彼得走幾步掉下去.
其實為什麼要這樣做.
其實你看清楚.
所有的史方記載.
門徒跟耶穌的經歷的時候.
大部分耶穌所做的事情.
門徒都不明白發生什麼事情.
如果要再說多一點的話.
什麼血流的女人.
血流的那個女人.
耶穌說有能力從我身上出來.
門徒說.
耶穌你是不是傻了.
這麼多人擁著你.
你怎麼知道.
誰摸著你的能力出來.
其實你可以想像.
在史方記載裡.
門徒通常都不明白.
耶穌發生什麼事.
這個是我們的常態.
我們在信耶穌的生平裡.
生涯當中.
我們以為我們經常為上帝.
做很多有意義的事情.
覺得那些事情.
哇 行啊 行啊.
但你會發現.
其實很多事情.
當你回望的時候.
我們以往覺得很有意義.
應該做的事情.
今天在這一刻來說.
我們覺得不一定.
要說一點上莊.
以前我在大學的時候.

$^{201}$是做團契職員.
在科大.
我是科大第三第四屆團契的人.
第三第四屆的紀錄.
第一第二屆那個人.
現在在美國教哲學.
第三第四屆的一個人在香港.
你可以想像.
當時我們多麼火熱.
如果你知道科大.
認識科大一點的時候.
我們基本上有福音周.
福音周那兩個星期.
基本上是做很多很多.
不同的福音御功.
我記得那時候.
最偉大那次.
在我人生裡覺得.
哇 很厲害那次是什麼呢.
就是那場報道會.
我忘了請了誰來講.
我已經忘記了.
但我記得電子科大.
如果你知道科大的話.
最大的課程就是電子科大.
坐滿所有人之餘.
樓梯都坐滿了.
如果那時候你覺得.
哇 能夠做到一件這樣的事情.
出了一份報紙.
二千多份 很快派完.
很多人看.
引起很多討論.
那一刻你覺得很有意義.
但這一刻你問你自己的時候.
那時候的事.
是不是真的在說.
很有意義.
你都會問自己一個問題.
為什麼今天這一刻.

$^{241}$你做不到這些事.
為什麼今天這一刻.
你不重複以前的事再做.
信仰歷程裡.
每一個時候.
我們都嘗試在賦予意義給自己.
我為上帝在做的事.
好像門徒登山保護.
登山變象一樣.
他見到以利亞和摩西的時候.
不如我搭座走在這裡.
去紀念你們這三個人.
然後你變象又發光很厲害.
我紀念一下這件事.
讓大家都知道這件事.
有多偉大的意義出現.
對於那一刻的彼得來說.
他覺得曉有意義.
但去到彼得.
耶穌升天以後的時候.
他在哨行傳.
從來都不會覺得.
那一刻他所做的事是有意義的.
問題是問.
這些看來好像沒什麼意義的事.
其實有什麼意義呢.
我們再看下一張泡沫.
他們在榮光裡顯現.
沙子說.
談論耶穌去世的事.
去世其實不是解去世.
其實不是說耶穌死的時候.
那個原文就是解Exodus.
是說出埃及.
就是耶路撒冷將要成就的事情.
其實門徒那時候不明白.
什麼叫談論出埃及的事.
出埃及是什麼.
出埃及就是已經出了.
在摩西時代.

$^{281}$為什麼要摩西和以利亞出來.
跟耶穌再說出埃及的事.
就等同於耶穌在耶路撒冷.
釘死十字架的事.
對於門徒來說.
由彼得那時候經歷了.
應信原理是主基督.
那個很厲害的時候.
耶穌說將來我要釘十字架死.
彼得攔了他.
耶穌馬上罵他魔鬼退去後邊吧.
所以對於門徒來說.
那些出埃及的事.
那些耶路撒冷將要成就的事.
對於門徒來說完全一知不曉.
完全不明白發生什麼事.
這個才是門徒.
跟隨耶穌三年裡的本相.
我們經常以為我們自己.
做了很多好像很有意義的事.
沒錯,你不賦予意義的話.
你人生不會再做下去.
就算你結婚的時候.
你不會說這個老婆放在這裡.
沒什麼意義的.
你也不會這樣說.
你也會覺得老婆很有意義.
她存在就令我很開心.
你一定要賦予意義在這裡.
我明白我們要賦予意義.
但實際在我們信仰的路上.
我們走了很久的時候.
什麼才是我們應該要做的事.
我們不可以說得太死.
我們不要以為我們拿著雞毛當令箭.
這是神的心意在我身上彰顯情義.
接著我就圍著做一些很厲害很偉大的事.
Come on 一起做吧.
沒錯,我們可以這樣說一下.
但可能十年後回望你這件.

$^{321}$看來很有意義的事的時候.
其實你心裡不會在想其他東西.
我們再看下一張PowerPoint.
其實這個故事的結尾.
只有一個結尾.
就是有一個雲彩有聲音.
基本上耶穌好魔西好伊利亞都沒有彩過彼得.
那三座棚的建議.
無端端在天上有個聲音說.
這是愛子,我想揀選聽從.
其實你會發現.
在我們人生信耶穌的歷程裡.
我們好像做了很多事.
我們都侍奉了.
我們都在教會裡做過的事.
我們都在上帝擺上了很多時間.
青春,光陰,金錢.
做了很多事.
但那些所謂在做的事.
意義是在乎我們賦予給它的意義.
還是做完那些事之後你發現.
其實自己所做的意義.
所賦予的意義.
其實都不是自己在想那些很有意義的事.
而那些事完結之後.
只有一樣東西.
原來我做完那些很有意義的事之後.
我回望那件很有意義的事的時候.
剩下來的不是我在想那些意義的意義.
只剩下的是.
在這件事裡.
有些事我不懂.
有些事我做錯.
有些事我做得不好.
聽從在這件事裡.
有些事我做得不夠不好.
才是我賦予這些意義之後.
我們應該要做的事.
那些複雜的事說完了.
不要說那麼複雜.

$^{361}$我的例子會最易明白.
上半年是我妹夫去世.
我也有說過.
這個暑假去英國的時候.
我帶了我妹妹去.
帶我妹妹去的時候.
我希望做哥哥的.
希望和她聊聊天.
和我妹妹聊聊天.
說說最近你怎樣.
但我想說.
整個月的旅程裡.
我一句話都說不出來.
例如有班.
通常我這班朋友.
去英國是我這班朋友多.
一起吃飯的時候.
有我這班朋友不識趣地問.
為什麼只有你和你兒子來.
你老公在哪裡.
你知道那一刻.
我尷尬到一個地步.
我不能再尷尬.
快點把話題轉了就完了.
我經常以為.
會說到很多空間和妹妹聊聊天.
你最近怎樣.
其實不知道說什麼.
只問一下怎樣.
但你知不知道.
那三個多星期.
我一句都說不出來.
我問一句你最近怎樣.
都做不到.
完全看見她就像.
純粹.
哈哈怎樣.
說其他東西.
我已經想好台詞.
想好劇本.

$^{401}$什麼什麼.
你現在怎樣.
我想好了但沒有.
最後那天我們去了約克.
在約克有個叫Castle Howard的地方.
一個很漂亮很有錢的Castle.
很大很宏偉.
坐車去看東西.
有個Tour.
Tour很特別.
因為不是公營是私營.
所以很多阿姨介紹你.
這裡是怎樣怎樣.
所以很用心.
你會聽到很多故事.
去到最後一站是一個Chapel.
那Chapel很特別.
你知道英國很多Chapel.
很大的Chapel很多.
我試過去一個很無聊的Chapel.
在Cambridge.
上去上面看Cambridge的景觀.
走百多級樓梯上去看.
又要付錢.
不夠五分鐘就說請你下來.
Do you come down.
拍照都還未完.
騙旅客.
有很多這些.
所以Chapel對我來說是最後一站.
所以它的Tour是這樣.
誰知去到最後一站.
Chapel很小不是很大.
我見我妹坐在那裡很久.
我去到的時候.
我和她的兒子一起走去.
因為我介紹一些東西給我妹的兒子聽.
怎樣怎樣.
多講一些東西聽.
因為已經是最後一天.

$^{441}$一去到的時候.
Chapel其實不重要.
在Chapel這個Stage上面.
不是Stage在後面.
在入口的位置.
有五個Screen.
即是黑色的Screen.
有五個.
1,2,3,4.
好像現在這裡這樣.
一個Screen很大.
Screen有什麼樣子呢.
我不明白Chapel為什麼有五個Screen呢.
每個Screen有些樣子.
它是錄下人的樣子.
人的樣子是很優秀的.
有些哭的.
有些很不開心的樣子.
五個Screen都是.
都是那些樣子.
我不明白Chapel不是用來敬拜嗎.
為什麼不能弄五個呢.
我去到入口的位置.
看一看.
到底那五個Screen想做什麼.
原來那五個Screen.
其實.
這一群拍下了的人的樣子.
當他失去了親人的時候.
他掛著親人的樣子.
所以五個Screen.
每個Screen有很多人的樣子.
所以有五個.
你看有幾十人掛著親人的樣子.
展示出來.
那一刻.
我看完那個介紹.
因為我只看到五個Screen的人的樣子.
不知道背後的意義是什麼.
我和小朋友走到前面的時候.

$^{481}$看完.
立刻抓到前面.
叫他看著那五個Screen.
那一刻我已經.
俯伏在上帝面前.
我心裡.
我做不到的事.
不懂得做的事.
我以為我做到的事.
其實上帝正在做.
我和小朋友說.
你失去爸爸.
其實不孤單.
我們看看五個Screen.
都是那些失去親人的人.
他們的掛念.
他說這些掛念很真實.
所以如果你很掛念你爸爸.
其實是很應該的.
你都不需要怕.
你這一刻好像失去了親人.
好像少了很多東西.
我大概說了這些類似的說法.
那一刻我妹妹聽完.
她走過來.
就和她兒子一起聊天.
你知道我也不阻.
他們兩個應該要聊天.
整件事就結束了.
我發現我也懂得安慰別人.
做牧師這行業很多年.
我也知道應該怎樣說話怎樣做事.
但當你以為你懂得做的時候.
其實你才會發現自己.
所謂做那些很有意義的事.
其實你的能力都很弱很不夠.
或者就算你說到做到.
好像應該有意義.
做得很理想的事.
其實你會慢慢發現.

$^{521}$上帝所做的.
祂比我所經歷的.
摩頂我覺得我可以做的事.
這個經驗.
令我伏在上帝面前的地步是.
我說主啊其實.
你掛念著我的妹妹.
比我以為我掛念我的妹妹更多.
當我以為我會很掛念我的妹妹.
很疼我的妹妹想表達的時候.
當我以為其他人都未必做到的時候.
上帝跟我說的是.
祂比任何一切都來得更加真實.
頂尖智慧你會發現.
在我們侍奉的歷程裡.
我們經常都覺得自己很有意義.
原來我們覺得很有意義的事.
做很多好像很行的事的時候.
其實有時候你回望的時候.
那些事是不是應該要這樣做?.
那些事是不是應該.
我想得這麼有意義?.
還是基本上所有事都是上帝保底的?.
這個powerpoint出來了.
我這樣說不是要躺平.
我這樣說不是說我們都不需要怎麼做.
恰好相反.
我們是要做的.
我這裡這麼寫.
門徒三年多在學什麼?.
他在學習自己的想法是無聊的.
但是這些無聊的想法.
很多無聊的想法的堆積.
他堆積完之後.
他想完之後發現.
他自己的想法裡.
以為很有意義的事其實都超無聊的.
唯有經歷這個過程的時候.
你才知道什麼在上帝裡有意義.
昨天大結局.

$^{561}$綻放如花.
如果你有看的話.
其實不是拍得很好.
怎麼說?.
不要得罪人.
與得罪人之間.
節奏有點慢.
不要說這些了.
但是整套劇.
令我們想到一件事是什麼呢?.
起碼昨天那一集說的是.
人經歷了很多無聊的事.
其實人都是遺忘的.
人都會選擇.
那些你應該要聽從跟從的事.
其實你都選擇不聽從不跟從.
選擇用一個忘記的方式去面對.
幾年前發生了很多事情.
慢慢回到現實環境之後.
很多以往發生的事情.
對於這一刻我們來說.
我們都不知道聽了什麼.
跟記了什麼.
所以鼎姐妹.
在亂世的日子裡.
試試離開我們以往做過的事情.
試試在亂世裡.
做一些你覺得沒什麼人做的事情.
做一些看起來好像.
不是很有意義的事情.
試試堆積很多無聊的經驗和學習.
慢慢在這個無聊的裡面.
你會慢慢看到.
上帝在做什麼事情.
我想說這兩個月的上妝.
是說我們要做一些新的事情.
不再是以往模式的裡面.
以往經驗的裡面.
做一些新走舊平.
舊走新平的事情.

$^{601}$試試在新的皮袋裡.
做新的事情.
哪怕那些事情你覺得很無聊.
哪怕你覺得這一刻.
你正在賦予它千萬個意義.
呼吁鼎姐妹.
選擇遺忘與選擇堅持下去.
是因為我們知道遺忘的路上.
是讓人可以輕散.
讓人可以有一個假象.
讓我們重新可以開始過.
彼得看著他這麼無聊的經驗.
搭三座棚.
在海裡走兩步掉下去.
最後看到耶穌的時候.
主耶穌跳下海裡游一圈.
這麼無聊.
這些這麼無聊的經驗的堆積.
就讓他知道什麼是上帝給他的意義.
但如果我們不進入無聊.
不進入那些我們眼看下去.
好像覺得很傻瓜的時候.
我們會慢慢對以往發生的事情.
只會在遺忘的裡面.
以為自己這一刻有一個新的開始.
殊不知原來一個真正新的開始.
是延續著前面你人生當中.
很多很多很無聊的事情.
唯有上帝在這些無聊事情裡面.
祂才呈現什麼叫做有意義.
天父很特別.
祂不會好像經常要我們問祂.
上帝啊,做什麼好啊.
正如如果我兒子每天問我.
爸爸去廁所好不好啊.
爸爸吃飯好不好啊.
爸爸我想去打球行不行啊.
爸爸我想看電視行不行啊.
我想玩個iPad.
大體上除非他作奸犯科.

$^{641}$殺人那些.
做要做的事情.
哪怕他多無聊.
他現在最喜歡玩的那些足球卡.
經常在那裡按啊按啊.
哇,正啊,我找到了.
巴比104啊.
天父爸爸看上去.
好像爸爸看上去玩的那些卡.
不就是我們以前的那些貼紙嗎.
你記不記得你小時候玩的那些貼紙.
拿到一張很大的貼紙.
很開心的半天.
不像我看著我兒子現在拿著那張卡.
裝得很開心.
無聊讓他成就什麼叫做有意義.
今天頂姐妹.
鼓勵你和我.
我12月希望有些新的事情做.
有些不一樣的事情做.
我覺得要做些不同的事情.
否則人生就會覺得好像很納悶.
那些事情我正在曉遇有很多意義.
但這堂課跟自己說的是.
就做一些這麼無聊的事情.
無聊一下.
花點錢,花點時間,花點氣力.
做一些傻乎乎的事情.
讓上帝笑一下.
讓自己歡愉一下.
福音書弟兄姊妹.
不是在享受崇拜.
我們不要做一個消費者.
福音書弟兄姊妹.
在這個這麼難的世代裡.
每一個都做一些很傻很笨的事情.
問一下上帝.
這些這麼笨這麼傻的事情.
可以成就到你什麼呢?.
哪怕搭三座塔,三座棚.

$^{681}$哪怕走到海裡掉下來.
哪怕看到耶穌馬上跳下海.
那些讓我們逐步逐步.
看到上帝.
心願福出弟兄姊妹.
不是停在星期六的崇拜.
而是在我們不同的崗位裡.
我們為上帝做很多很笨很無聊的事情.
我們低入禱告.
天父多謝你給我們今天這個空間.
有這個時間.
我祈求的是.
天父,我們不聰明.
我們也不需要裝得很聰明.
我們需要為自己前面的路.
曉喻有很多意義.
天父,你從來都是容許我們在你面前亂衝亂撞.
讓我們面前做一些.
你眼裡都看我們很笨很傻的事情.
但天父,那些笨的事情和傻的事情.
你卻幫我們慢慢看到你的意義是什麼.
你關注的是什麼.
你在看的是什麼.
天父,我求你的是.
祝福Fold Church的弟兄姊妹.
無論留在香港的.
或是在世界各地的弟兄姊妹.
每一個都可以在你面前.
讓你去塗灶.
因為我們生命本質.
靜下來的.
就是一個願意為你的心.
為你去做.
為你去嘗試.
為你將自己的生命擺上.
就好像彼得.
被人捉拿完,被人打完.
繼續傳講著天國的信息一樣.
求萬鈞的耶和華這一刻.
臨到在我們每一個弟兄姊妹身上.

$^{721}$在亂世艱難的日子.
更應如此.
最後求你興起我們,幫助我們.
多謝天父你聽我們祈禱.
福音書你寶貴命求.
\newpage



\section{使徒行傳 17:1-10-20230923}
\label{sec:5EgvGimlwXk}
\textbf{【網上崇拜】熱血福音戰士|使徒行傳17\_1-10|20230923 [5EgvGimlwXk]}
\newline
\newline
連結: \href{https://youtube.com/watch?v=5EgvGimlwXk}{\texttt{ https://youtube.com/watch?v=5EgvGimlwXk}} ~~~~ 語音日期: 2023-09-23 
\newline
\newline
\hyperref[sec:h7oDnhukFbo]{\small{< < < PREV SERMON < < <}}
~
\hyperref[sec:index_chronic]{\small{[返順時目]}}
~
\hyperref[sec:index_scriptual]{\small{[返順卷目]}}
~
\hyperref[sec:2QyWxsVtL8E]{\small{> > > NEXT SERMON > > >}}
\newline
\newline
使徒行傳 17:1-10-20230923
\newline
\begin{longtable}{cl}
\hline
\hline
章節 & 經文 (和合本修訂版)\\
\hline
17:1 & \begin{tabularx}{0.7\textwidth}{X} 保羅和西拉經過暗妃坡里、亞波羅尼亞,來到帖撒羅尼迦,在那裡有猶太人的會堂。 \end{tabularx} \\ \\ \relax
17:2 & \begin{tabularx}{0.7\textwidth}{X} 保羅照他素常的規矩進去,一連三個安息日,根據聖經與他們辯論, \end{tabularx} \\ \\ \relax
17:3 & \begin{tabularx}{0.7\textwidth}{X} 講解和說明基督必須受害,從死人中復活;又說:「我所傳給你們的這位耶穌就是基督。」 \end{tabularx} \\ \\ \relax
17:4 & \begin{tabularx}{0.7\textwidth}{X} 他們中間有些人聽了勸,就跟從保羅和西拉,還有許多虔敬的希臘人,尊貴的婦女也不少。 \end{tabularx} \\ \\ \relax
17:5 & \begin{tabularx}{0.7\textwidth}{X} 但不信的猶太人心裡嫉妒,聚集了些市井流氓,搭夥成群,煽動全城的人闖進耶孫的家,要把保羅和西拉帶到民眾那裡。 \end{tabularx} \\ \\ \relax
17:6 & \begin{tabularx}{0.7\textwidth}{X} 那些人找不著他們,就把耶孫和幾個弟兄拉到地方官那裡,喊叫著:「這些攪亂天下的人也到這裡來了, \end{tabularx} \\ \\ \relax
17:7 & \begin{tabularx}{0.7\textwidth}{X} 耶孫竟收留他們。這些人都違背凱撒的命令,說另有一個王耶穌。」 \end{tabularx} \\ \\ \relax
17:8 & \begin{tabularx}{0.7\textwidth}{X} 眾人和地方官聽見這些話,就惶恐了, \end{tabularx} \\ \\ \relax
17:9 & \begin{tabularx}{0.7\textwidth}{X} 於是收了耶孫和其餘的人的保證金後,釋放了他們。 \end{tabularx} \\ \\ \relax
17:10 & \begin{tabularx}{0.7\textwidth}{X} 當夜,弟兄們立刻送保羅和西拉往庇哩亞去;二人到了,就進入猶太人的會堂。 \end{tabularx} \\ \\ \relax
17:11 & \begin{tabularx}{0.7\textwidth}{X} 這地方的猶太人比帖撒羅尼迦的人開明,熱心領受這道,天天查考聖經,要知道這道是否真實。 \end{tabularx} \\ \\ \relax
17:12 & \begin{tabularx}{0.7\textwidth}{X} 所以,他們中間有許多信了,又有希臘的尊貴婦人,男人也不少。 \end{tabularx} \\ \\ \relax
17:13 & \begin{tabularx}{0.7\textwidth}{X} 但帖撒羅尼迦的猶太人知道保羅又在庇哩亞傳神的道,就往那裡去,煽動挑撥群眾。 \end{tabularx} \\ \\ \relax
17:14 & \begin{tabularx}{0.7\textwidth}{X} 於是,弟兄們立刻送保羅到海邊去,西拉和提摩太卻仍留在庇哩亞。 \end{tabularx} \\ \\ \relax
17:15 & \begin{tabularx}{0.7\textwidth}{X} 護送保羅的人帶他到了雅典,他們領了保羅的命令,叫西拉和提摩太趕快到他那裡來,然後回去了。 \end{tabularx} \\ \\ \relax
17:16 & \begin{tabularx}{0.7\textwidth}{X} 保羅在雅典等候他們的時候,看見滿城都是偶像,就心裡非常難過。 \end{tabularx} \\ \\ \relax
17:17 & \begin{tabularx}{0.7\textwidth}{X} 於是他在會堂裡與猶太人和虔敬的人,以及每日在市場上所遇見的人辯論。 \end{tabularx} \\ \\ \relax
17:18 & \begin{tabularx}{0.7\textwidth}{X} 還有伊壁鳩魯和斯多亞兩派的哲學家也與他爭辯。有的說:「這胡言亂語的要說甚麼?」有的說:「他似乎是宣傳外邦鬼神的。」這是因保羅傳講耶穌與復活的福音。 \end{tabularx} \\ \\ \relax
17:19 & \begin{tabularx}{0.7\textwidth}{X} 他們就把他帶到亞略巴古,說:「你所講的這新學說,我們也可以知道嗎? \end{tabularx} \\ \\ \relax
17:20 & \begin{tabularx}{0.7\textwidth}{X} 因為你有些奇怪的事傳到我們耳中,我們想知道這些事是甚麼意思。」 \end{tabularx} \\ \\ \relax
17:21 & \begin{tabularx}{0.7\textwidth}{X} 原來所有的雅典人和居住在那裡的外國人都無暇管別的事,只是談談或聽聽新聞。 \end{tabularx} \\ \\ \relax
17:22 & \begin{tabularx}{0.7\textwidth}{X} 保羅站在亞略巴古當中,說:「諸位雅典人!我看你們凡事很敬畏鬼神。 \end{tabularx} \\ \\ \relax
17:23 & \begin{tabularx}{0.7\textwidth}{X} 我到處走走的時候,仔細觀察你們所敬拜的,發現一座壇,上面寫著『獻給未識之神明』。你們所不認識而敬拜的,我現在向你們宣告: \end{tabularx} \\ \\ \relax
17:24 & \begin{tabularx}{0.7\textwidth}{X} 他是創造宇宙和其中萬物的神;他既是天地的主,就不住在人手所造的殿宇裡, \end{tabularx} \\ \\ \relax
17:25 & \begin{tabularx}{0.7\textwidth}{X} 也不用人手去服侍,好像缺少甚麼似的;自己倒將生命、氣息、萬物賜給萬人。 \end{tabularx} \\ \\ \relax
17:26 & \begin{tabularx}{0.7\textwidth}{X} 他從一人造出萬族,居住在全地面上,並且預先定準他們的年限和所住的疆界, \end{tabularx} \\ \\ \relax
17:27 & \begin{tabularx}{0.7\textwidth}{X} 為要使他們尋求神,或者可以揣摩而找到他,其實他離我們各人不遠。 \end{tabularx} \\ \\ \relax
17:28 & \begin{tabularx}{0.7\textwidth}{X} 我們生活、行動、存在都在於他。就如你們的詩人也有人說:『我們也是他所生的。』 \end{tabularx} \\ \\ \relax
17:29 & \begin{tabularx}{0.7\textwidth}{X} 既然我們是神所生的,就不應該以為神的神性像人用手藝和心思所雕刻的金、銀、石像一般。 \end{tabularx} \\ \\ \relax
17:30 & \begin{tabularx}{0.7\textwidth}{X} 世人蒙昧無知的時候,神並不追究,如今卻吩咐各處的人都要悔改。 \end{tabularx} \\ \\ \relax
17:31 & \begin{tabularx}{0.7\textwidth}{X} 因為他已經定了日子,要藉著他所設立的人按公義審判天下,並且使他從死人中復活,給萬人作可信的憑據。」 \end{tabularx} \\ \\ \relax
17:32 & \begin{tabularx}{0.7\textwidth}{X} 眾人聽見死人復活的話,就有人譏誚他;又有人說:「我們會再聽你講這事。」 \end{tabularx} \\ \\ \relax
17:33 & \begin{tabularx}{0.7\textwidth}{X} 於是保羅從他們當中出去了。 \end{tabularx} \\ \\ \relax
17:34 & \begin{tabularx}{0.7\textwidth}{X} 但有幾個人依附他,信了主,其中有亞略巴古的議員丟尼修,和一個名叫大馬哩的婦人,還有幾個與他們一起的人。 \end{tabularx} \\ \\
[1ex]
\hline
\hline
\end{longtable}
$^{1}$四位姐妹平安.
多謝我們這班七彩的敬拜隊.
大家有沒有留意他們的爵士曲.
是跟我一樣的 不同顏色的.
(笑聲).
今天的片是七彩的.
所以今天的講題是七彩福音戰士.
我們聽到之前一起祈禱.
萬主之主萬王之王.
我們要不失苦復敬拜你.
因為你是全地的主.
今天我們聚集在你面前.
求聖靈你幫助我們.
讓我們有能力.
你充滿我們 讓我們有智慧.
以致我們在這個世代.
我們仍然能成為你的見證.
求聖靈你成為我們的老師.
讓我們明白你的說話.
也賜給我們能力能夠走在神的道中.
我們這樣祈禱.
是奉耶穌基督的聖名祈求.
阿們.
這次雙月題的上妝.
不知道你們有沒有聽到一種.
離我很遠的感覺.
有些人沒有.
因為我前陣子才遇見一個姊妹.
她跟我說.
我下星期就轉字頭.
我轉二字頭.
我說 哦 我都生到你出了.
可能你沒有離我很遠.
但真的離我很遠.
上妝是什麼.
大學生的時候.
成為不同組織的妝員.
我聽到月題上妝的時候.
第一刻的感覺就是.
哇 這麼熱血.

$^{41}$可能有些電影姊妹沒有上過妝.
剛才好像只有一個手舉手.
上過妝.
那OK了.
我真的不會走路.
因為我都沒有上妝.
真的沒有上妝.
但是我覺得一聽到上妝.
就覺得跟熱血扯上關係.
為什麼呢.
因為我一想起當年.
剛轉二字頭的時候.
讀大學.
讀大學要有註冊.
當年我讀浸會大學.
這間屋村學校.
是這樣說的.
現在不是這樣說的嗎.
村屋?.
Anyway 沒關係.
總之浸會大學.
我一進去.
大家記得.
如果你有印象.
浸會大學有條樓梯.
我還沒去樓梯的時候.
已經聽到那群人.
那群人.
其實剛才主令.
想我們一起Damn Cheers.
大家剛才氣氛一般.
因為我們距離很遠.
是吧.
我就聽到那些人.
在Damn Cheers.
我的學系簡稱COMM.
Damn Cheers是怎樣的呢.
有個人說Give me a C.
其他人就C.
是這樣的.

$^{81}$你註冊.
我這些freshman.
去註冊沒什麼特別.
有些師兄師姐很熱切地說.
歡迎你.
你可以來這裡.
Kitty Club.
但最熱血的是什麼呢.
最熱血的是.
明明師兄師姐在介紹你.
讓你知道這間學校是什麼.
那裡是什麼.
你要去哪裡上課.
Kitty Club 什麼都好.
突然有個人說.
Give me a C.
全世界會放下.
手頭上的私人註冊.
全世界會放下手頭上的私人註冊.
我立刻拉下咪高峰.
不單止.
還要叫我一起私人註冊.
就是他一邊放下手頭上的私人註冊.
你們沒有人做的.
證明你們已經不熱血了.
有些人轉頭.
別搞我了.
想當這些年輕人做的.
這些記憶太模糊了.
當年看著這些人大叫.
我自己也有叫.
因為我當年也很年輕.
剛好轉移至頭.
但現在我沒有了.
現在回想起來.
覺得很瘋狂又好笑.
我們上課.
似乎與我們無關.
那些人太熱血了.
可以不眠不休地開AGM.

$^{121}$可以不上學.
去酒堂處理那些裝模.
有很多活動去做.
但自己的GPA是被解僱了.
對吧.
然後說自己的學期只有1.9.
然後心想.
你自己選擇的.
是這樣的.
上課是一種那些年輕人的天真純情.
有一種年少無知的傻勁.
還有一種.
已經與我無關.
那種唏噓.
想裝的是那些小孩子不懂事.
讀書不讀.
去玩.
現在還想裝嗎.
別搞笑了.
想裝這個月題.
我想似乎是應該.
想用我們當年那股天真傻氣.
青春無敵.
無限熱血.
去喚醒我們今天那麼多個.
那種又冷靜又成熟的心靈.
沒有人跟我一起喊.
剛才主令得很厲害.
高歌也沒有人跟他打.
我自己也沒有.
因為我已經過了二次刀.
所以今天跟大家看一個熱血方戰士.
保羅和他的莊園.
他的宣教團隊.
希望可以從他的熱血.
激發我們那個冷靜又成熟的心.
一起去做一些事.
今天我們看保羅.
我忘記了.
Damn cheers.

$^{161}$跟大家看保羅.
保羅沒理由不是熱血的代表.
還沒出.
不好意思.
由他還沒開始信耶穌.
他已經很熱血.
他到處找高官拿許可證去抓基督徒.
抓到大家都怕了他.
然後在第九章.
在大晚色遇見主之後.
他變成為主癲狂的熱血方戰士.
說他的勇士是因為保羅.
為了傳福音.
他甚至被人抓.
被人打.
甚至軟禁.
但無論什麼警方保羅都繼續傳福音.
熱血到今天跟大家看這段經文.
他被人說他和他的莊園.
即是他的短宣隊的人.
是搞亂天下的人.
天下都搞得亂.
就不可能不熱血.
做了什麼呢.
我們看一看這段經文.
十七章.
《自由行主》十七章一至四節.
保羅和西拉經過暗非波利亞.
波羅尼亞.
來貼殺羅尼加.
那裡有猶太人的會堂.
保羅就照他素常的規矩進去.
一連三個安息日.
用聖經和他們辯論.
講解神明基督必須受害.
從死裡復活.
又說我所傳與你們的這位耶穌.
就是基督.
他們中間有些人聽了勸.
服從保羅和西拉.

$^{201}$還有很多虔敬的希利利人.
尊貴的婦女都不少.
這裡說保羅和西拉.
經過暗非波羅尼亞.
來貼殺羅尼加.
這些地方在哪裡呢.
這個地圖大家可以看看.
主要是在今天.
希臘北部的地方.
換言之.
福音已經由本來.
耶路撒冷所謂中東地區一帶.
傳到歐洲.
保羅和西拉去到貼殺羅尼加.
連續三個星期.
拿著聖經 即是他們的舊約.
去到猶太人的會堂.
當時的人知道.
舊約所說的尼塞亞是哪位.
就是那位受害從死裡復活的耶穌.
第四節他說.
有人聽了勸.
服從了保羅和西拉.
即是跟了他們.
加入了他們的行列.
當中不單止猶太人.
還有希利利人 即是希臘人.
尊貴的婦女也不少.
保羅和西拉跟人說耶穌.
有人信 但也有人不信.
五至七節就這樣說.
他說那些不信的猶太人.
心裡疾到.
焦聚了那些使者 匪女.
搭伙成群 聳動合成的人.
闖進耶穌的家.
要將保羅和西拉帶到百姓那裡.
找不到他們.
就將耶穌和幾個弟兄.
拉到地方官那裡.

$^{241}$大聲說 搞亂天下的.
也都來到這裡.
耶穌收留他們 這些人都是違背.
該殺的命令 說另有一個.
王耶穌.
說保羅和他們那群莊園.
搞亂天下的是什麼人呢.
是這群使者 匪女.
這群人是由那群.
疾到的猶太人.
請回來的 或者買回來的.
他們疾到保羅.
於是用這些所謂下三流的方法.
找那群爛仔 黑社會來搞事.
聖經形容他們是聳動合成.
聳動合成.
這裡是不是這樣寫.
聳動合成 新漢語另一個聖經.
基本就譯作引起騷動.
擺明搞事.
這群人本身.
可以說是賊喊捉賊.
明明自己搞事.
大聲和貼薩羅尼加城的人說.
保羅和西拉這群人.
是搞亂天下的人.
而且他們還行私刑.
闖入民居 耶穌的家.
保羅和西拉應該.
在他們家舉行聚會.
但他們找不到保羅和西拉.
抓不到他們 就抓了.
耶穌和一群兄弟.
去到官場受審.
搞亂天下是什麼意思呢.
搞亂天下這個指控.
其實很嚴重 如果直譯.
其實是指upside down.
即是英文 倒轉.
天下是什麼意思呢.

$^{281}$在路加作者的用法.
是指和宗教經濟權力.
有關的主權.
所以在這裡可以解讀為羅馬帝國.
當時的羅馬帝國.
就是世界 就是天下.
所以第17章第7節.
這群市井匪類說.
保羅和西拉違背了.
該撒的命令 說另有一個王.
耶穌 當時的人.
一聽到另有一個王的時候.
其實他們聯想到什麼呢.
是聯想到內戰.
聯想到打仗.
聯想到改朝換代.
等等等等.
所以搞亂天下.
其實是指他們想推翻羅馬帝國.
保羅和西拉.
其實從來沒有想過.
這些所謂政治章程.
沒有打算要搞革命.
他們只是想讓人知道.
耶穌是尼塞亞.
他們只是想宣揚.
耶穌的救恩.
但這群爛仔就上綱上線.
說他們有政治任務.
目的很簡單.
目的其實就是想這群.
嫉妒的猶太人.
想用政治原因去威嚇.
逼迫保羅和西拉.
要他們閉嘴.
甚至消失.
離開帖薩羅尼加.
不要再搞亂他們的會堂.
搞亂他們的生活.
第八節.

$^{321}$那群官員聽到之後有什麼反應呢.
第八節說中人和地方官.
聽見就驚慌.
很奇怪.
這群人搞亂天下 但那群官驚慌.
為什麼會怕.
你可以想像.
一個搞事份子.
革命家來到自己管轄的地方.
其實很大件事.
一來很多事情要做.
二來如果處理得不好.
隨時會牽連自己的前途.
第九節.
他說他就拿了耶穌和其他人的寶藏.
釋放了他們.
什麼叫寶藏呢 簡單來說就是保釋金.
就是給錢就搞定.
擔保不再收留保羅.
他們那麼多人.
那保羅和西拉去了哪.
十七章第十節就說.
弟兄們就隨即.
在夜間打發.
保羅和西拉往比利亞去.
保羅和西拉.
可能都不想麻煩耶穌.
他們其他兄弟.
所以連夜離開鐵薩羅尼加.
去了一個地方 比利亞.
如果我們在這裡看聖經看完.
熱血戰士.
保羅似乎都一般般熱血.
完了.
面對搞亂天下的指控.
似乎迫不得已要停下他的侍奉.
但第十節還有下一句.
我故意不寫.
下一句就是二人到了.
就進入猶太人的會堂.

$^{361}$進入會堂.
說耶穌.
其實是保羅的習慣.
他去每一個地方第一件事做的.
就是進入猶太人的會堂.
跟人說耶穌是誰.
但每一次都沒有好下場.
第十三章他去到安提阿.
又是進入會堂.
但最終被人趕走了.
可以說是遞解出境.
第十四章他去到.
以哥廉.
又是進入猶太人的會堂.
但最終他被人用石頭.
打 用暴力對待.
第十八章他去到哥林多.
又是去到會堂.
然後被人毀謗.
第十九章第八節.
他去到伊弗所.
去到伊弗所.
十九章的時候去到伊弗所.
其實亦都是一樣.
這樣的情況被人毀謗.
今日看的經文都是.
十七章第二節.
寫到保羅去到貼薩羅尼加.
是怎樣的呢?.
是按照他素常的規矩.
進入猶太人的會堂.
規矩原本的意思不是.
有什麼規矩要守.
意思本身只是.
保羅按照自己的習慣.
進入猶太人的會堂.
跟別人說耶穌.
不過這次下場更厲害.
被人政治迫害.
說他搞亂天下.

$^{401}$保羅其實做了什麼被人說他搞亂天下呢?.
保羅其實只是.
做他一向做的事.
做他素常的習慣.
進入會堂說耶穌.
大家有沒有想過.
原來只是.
堅持信仰的日常習慣.
是可以厲害到.
其他人覺得.
你可以搞亂天下.
原來堅持.
進入會堂說耶穌.
是可以有這麼大的影響力.
我們一開始想的熱血.
可能是轉字頭.
二字頭的人.
那些年輕人.
不顧一切青春天真.
薄盡無悔.
但保羅讓我們知道原來我們對耶穌基督.
對信仰的熱血.
不是要做一些驚天地泣.
鬼神很厲害.
大聲叫的事.
原來我們這班人一起.
做一些基本的事.
做一些自己的習慣.
保持自己的日常.
保持著.
做自己一向做的事.
進入會堂說耶穌.
就足夠了.
足夠到一個點.
是可以搞亂天下.
什麼叫進入會堂?.
進入會堂有可能.
會堂有可能.
源自於南國滅亡的時候.
南國就沒有了.

$^{441}$猶太人就被露去巴比倫.
猶太人流散在不同地方.
開始他們在異國異族.
為了保護自己.
獨有的一神信仰.
他們的飲食習慣 他們的傳統 他們的語言.
他們的節旗等等.
猶太人自己聚集起來.
目的是為了保存 傳承.
宣揚自己的文化和獨特性.
會堂是猶太人.
敬拜神的重要場所.
只是沒有祭壇 所以他們不會在那裡獻祭.
但他們會在會堂裡.
學聖經 會討論哲學.
會學高尚的道德.
實踐信仰.
所以會堂同時是他們的學校.
另一個字形容會堂.
是禱告的房子.
所以會堂.
都是他們祈禱的地方.
對當時一群流散在不同地方的猶太人來說.
會堂不單單是他們的宗教場所.
亦都是他們聯誼社交的地方.
就好像社區中心.
會在會堂舉行不同形式的群體活動.
所以會堂可以說是一個可以學習神說話.
同時也是猶太人社交 聯誼文化的地方.
除了猶太人之外.
有一些人 外邦人.
因為很欣賞猶太人的獨特性.
很欣賞他們很高超的道德標準.
很喜歡會堂的生活.
他們都會參加.
但他們仍然未能夠參加.
但他們仍然未能夠走到國禮這一步.
所以如果聖經中你會看到.
有些人形容他們是敬畏神的人.
或者虔誠人.

$^{481}$就是這群外邦人.
他們會在猶太人的會堂聚集.
所以為何保羅進會堂說耶穌的時候.
剛才我們聽到.
就是有一些希臘人.
有一些尊貴的婦女.
他們都加入了保羅的行列.
會堂是宗教文化生活.
甚至可以是吸引人的地方.
最特別的是會堂的希臘文是這個.
不讀了.
讀了大家一樣都是這樣.
英文懂得讀了.
原來英文會堂就是吸引希臘文的.
這個字.
這個字不單止是指一種建築物.
也是在說一群回眾.
是在說一群人.
原來這個孫力軍的字.
不單單是說一棟東西.
也是在說一群人.
一群回眾.
會堂是甚麼呢?.
會堂是指一群相同信仰.
合謀合得來的人.
聚在一起談論各方面的事.
目的是為了保存.
傳承和宣揚自己的獨特性.
我們這群人一起做些事.
做些甚麼呢?.
其實做一些基本的事.
一起保持入會堂說耶穌的習慣.
保持做這個習慣.
但不是因為習慣所以做.
大家聽得明白嗎?.
我們要保持做這個習慣.
但不是因為習慣所以做.
保羅入會堂說耶穌.
也不是因為他純粹習慣了.
當然保羅入會堂是因為他是一個很虔誠的猶太人.

$^{521}$對猶太人的節旗禮儀道德很清楚很了解.
但保羅不是嚴守禮儀的人.
也不是因為規矩而習慣了入會堂.
一定不會的.
你看看他剛才的下場那麼多.
怎會為入而入呢?.
保羅勇到被人告他搞亂天下.
他也不退縮.
他也可以堅持入會堂說耶穌這個習慣.
為甚麼他可以做到?.
我猜最大原因是他自己首先被神的道搞亂.
如果我們看整個《使徒行傳》.
你會看到保羅很厲害.
很多保羅傳福音去這裡去那裡.
我們會說《熱血荒戰士》是保羅和他的一群人.
因為他不停傳福音.
但作者想強調的不是保羅很厲害.
作者想強調的是神的道很厲害.
當神的道臨到每一個地方.
每一個地方就會被反轉.
當神的道臨到我們.
我們的生命都被反轉被顛覆.
保羅也一樣是這樣經歷.
自從行傳第九章保羅親眼見到主耶穌.
生命完全被搞亂.
完全upside down.
你記得吧?.
由原本他恐嚇人不准人信耶穌.
他現在不停叫人信耶穌.
他由原本逼迫信耶穌的人.
他現在變成被人逼迫.
他由原本拉人坐牢.
拉基督徒.
變成自己要坐牢.
保羅經歷耶穌基督這一座的倒轉.
所以入會堂是習慣.
但不是因循.
他有目的.
他要向猶太人分享耶穌基督是主.
是舊約寓言的尼塞亞.

$^{561}$他很希望他的同胞和外邦人一樣.
因為耶穌基督可以一起敬拜這個獨一真神.
所以保羅入會堂講耶穌是他素上的習慣.
但不是習慣了.
保羅是願意被神的道搞亂.
以至他能夠保存自己這個信仰習慣的同時.
可以挑戰傳統.
可以接受新事.
我們如果和保羅一樣.
已經被神的道搞亂了.
生命都是一樣以耶穌為我們的主的時候.
其實我們都要一起做好入會堂講耶穌的習慣.
一方面要保持入會堂講耶穌的精義.
但同時要按著時代不同.
不斷更新我們改變我們.
入會堂講耶穌的形式.
一起入會堂講耶穌.
當然照字面解釋就是一群人.
一起回教會出席聚會.
一起崇拜.
一起小組.
一起祈禱.
一起看聖經.
一起學聖經.
一起有空回報道會.
一起街頭報道.
一起和人傳福音.
講三福.
講四律.
福音橋.
五色珠.
一起玩.
一起飲飲食食.
一起重新得力.
全部都很基本.
但很重要.
講出來有少少廢.
今日堂到原來叫大家這樣.
對呀就是這樣.
誰不知是媽媽女人嗎.

$^{601}$是不是一些呼籲.
但不代表我們每個人都做到.
是嗎.
人人都知每天運動身體好.
但我們應該不是男女老幼都做得到.
對嗎.
我們知道很多事.
但不代表我們做到.
一起回教會.
一起傳福音.
一起祈禱.
一起看聖經.
基督徒是要做的.
這些習慣.
我們有沒有保持住.
我們一群人.
要一起做些事.
我們做好這些習慣.
但當然我們不是因循教會聚會就夠了.
我們同時可能要更新一下.
我們這個入會堂.
講耶穌的模式.
坊間有很多不同關注組.
不知大家有沒有參與其中一些.
燒賣關注組有沒有人參加.
有的見到了.
咖啡愛好關注組有沒有人參加.
OK又有你.
好.
不同地區的關注組.
沙田區有沒有人參加.
OK.
尖沙咀區有沒有人參加.
不同不錯.
關注流浪狗的有沒有人參加.
有的.
環保的.
OK有的.
識飲識食.
省點食的.

$^{641}$有東西吃剩了.
有免費的有沒有人參加.
很多非飲非食的.
有的.
志同道合.
我都有參加.
很多不同關注組.
這個小賣好吃.
拍張照上載.
這個咖啡好喝.
不要做新手黨自己搜尋.
很多這些.
多點了麥當勞有沒有人要.
走過五點九油麻地站等.
你們有沒有看過這些貼文.
你們沒有.
很好看.
真的有.
有一個叫識食.
類似免費的食物.
譬如我煲多了一碗湯.
有誰想要.
六點四十五分屯門小康站.
下一天排隊.
類似這樣.
以教修.
就是這樣.
很多這些.
很好.
其實這些關注組你們覺得嗎.
這些群組很好.
我們教會有沒有這些群組.
我們教會可能可以加這些.
誰想學結他.
有一個關注組.
誰可以彈琴.
教音響幫忙.
警察隊.
又可以有一個關注組.
誰沒有住家飯吃.

$^{681}$有人煲多了一碗湯.
我也想排隊.
我也想有人給我一碗湯喝.
很多不同關注組.
交換最新資訊.
有這個地方我搬屋了.
有沒有人不要這些櫃子.
又有很多.
共同喜好的會自成一組.
喜歡跑步的有跑友.
打球的有球友.
拍照的有龍友.
全部都可以聚在一起.
當然很多群體.
都是提倡他們的理念是很好的.
關注流浪狗的很好.
黑雨的時候他們很擔心.
你們看到嗎.
那些群組.
流浪又如何呢.
很擔心.
提倡環保生活的又有.
提醒你多帶一個膠盒.
不要再用膠袋.
什麼都好.
很多這些群組.
我們這一群跟隨耶穌的人.
理應已經有相同的信仰.
我相信剛才那麼多人都舉手.
咖啡也好.
不知什麼也好.
我們都會有些人.
一定有相同的關注.
有共同的喜好.
或者想提倡的一些理念.
我們如果可以聚在一起.
加入或者建立不同的群組.
是不是理應.
分享燒賣好吃之餘.
還可以做多些事呢.

$^{721}$是不是可以分享這間咖啡店好吃之餘.
再可以做多一點呢.
跑友健身友一起.
除了跑步做健身.
追求身形的同事.
我們可不可以再做一些.
關於講耶穌的事.
身心靈平衡.
我們可不可以做到.
告訴其他人.
大自然行山的.
我們跟一些未信的人.
是不是可以分享到.
創造主的偉大.
很多藝術人教會.
可不可以分享到.
神創造的那種真善美.
關愛世界.
提倡環保.
簡約生活.
其實可能可以賴到.
社會資源如何分配.
我們有沒有想過.
其實我們可以將我們的喜好.
我們的理念.
我們的關注.
跟講耶穌.
我們素上的習慣.
連在一起呢.
其實可以繼續做我們做的事.
繼續做到我們的習慣.
繼續每天做到我們的平常事.
日常事.
其實不是那麼輕鬆容易.
或者那麼理所當然.
很多外來因素.
大環境.
其實可以威脅我們.
不容許我們再做.
這些基本的事.

$^{761}$是可以逼我們改變.
我們的日常習慣.
所以熱血不是我們要做什麼厲害的事.
不是做什麼爆的事.
也不是屬於那些二字頭.
那些人的獨有.
熱血是我們可以.
無論什麼環境.
什麼改變.
什麼逼迫.
我們都可以保持.
堅持.
保持我們進會堂.
講耶穌的習慣.
怎樣可以保持這個習慣呢.
我想其中一個方法是.
我們要一班人一起做.
我明白開始新的人際關係是很累的.
建立人際關係更累.
維繫人際關係.
是更加更加累.
付出我們很多心力.
很多精力.
但一班人一起.
這應該是我們.
這班被神的道.
搞亂了的人.
素上的習慣.
剛才我不是一開始.
二字頭的時候.
那些人就大喊.
大喊cheers.
如果我就這樣一個人.
沒人回應是很奇怪的.
我講兩句就不會再喊.
後面有人只是給我一隻手.
但都沒有喊.
是真的要這樣.
很奇怪.
但如果有人喊很不同.

$^{801}$Give me a C.
這班很大聲.
很大聲.
你們肯定是二字頭.
到你們了.
應該是一字頭.
Give me a C.
到你們了.
你們好像沒有喊.
有吧.
那我們一起喊.
我們應該喊什麼.
還是Give me a C.
好嗎.
好奇怪 無端要喊我自己的.
Give me a C.
一起喊才開心.
怎樣可以保持這些習慣.
是一班人一起.
有句很老土的話.
今天其實我很老土.
很早說這些.
他說如果你想走得快.
你就一個人走.
如果你想走得遠.
你就一班人走.
我們這班人一起走遠一點.
我們其實還有很長遠的路要走.
我們一班人一起走遠一點.
但願我們每一個弟兄姊妹.
都可以找到我們的熱血福音戰士.
我們一班人一起.
保持我們素常的習慣.
保持入會堂.
說耶穌.
一起搞亂天下.
不要記得我說Give me a C.
今天.
記得我們要保持做素常的習慣.
我選了一首很老土的歌.

$^{841}$我還很害怕地問Alex.
這首歌會不會太老土.
他說不怕.
改一改感覺.
(笑聲).
即是怕吧.
但剛才我聽了感覺不錯.
那首歌真的很老土.
但就是我二字頭唱.
所以我想唱.
就是同一首歌.
大家都知道了.
麻煩敬拜隊.
帶我們一起唱這首歌.
好好聽的同路人.
帶我們一起唱這首很好聽的《同路人》.
\newpage



\section{}
\label{sec:2QyWxsVtL8E}
\textbf{《致餘民及流散者:給香港基督徒的神學八課》第二季第6課|20230924 [2QyWxsVtL8E]}
\newline
\newline
連結: \href{https://youtube.com/watch?v=2QyWxsVtL8E}{\texttt{ https://youtube.com/watch?v=2QyWxsVtL8E}} ~~~~ 語音日期: 2023-09-24 
\newline
\newline
\hyperref[sec:5EgvGimlwXk]{\small{< < < PREV SERMON < < <}}
~
\hyperref[sec:index_chronic]{\small{[返順時目]}}
~
\hyperref[sec:index_scriptual]{\small{[返順卷目]}}
~
\hyperref[sec:JxHW7ujVbSI]{\small{> > > NEXT SERMON > > >}}
\newline
\newline
$^{1}$我只想知道.
你到底是什麼意思.
我只想知道.
你到底是什麼意思.
我只想知道.
你到底是什麼意思.
我只想知道.
你到底是什麼意思.
我只想知道.
你到底是什麼意思.
我只想知道.
你到底是什麼意思.
我只想知道.
你到底是什麼意思.
我只想知道.
你到底是什麼意思.
我只想知道.
你到底是什麼意思.
我只想知道.
你到底是什麼意思.
我只想知道.
你到底是什麼意思.
我只想知道.
你到底是什麼意思.
我只想知道.
你到底是什麼意思.
我只想知道.
你到底是什麼意思.
我只想知道.
你到底是什麼意思.
我只想知道.
你到底是什麼意思.
我只想知道.
你到底是什麼意思.
我只想知道.
你到底是什麼意思.
我只想知道.
你到底是什麼意思.
我只想知道.
你到底是什麼意思.

$^{41}$我只想知道.
你到底是什麼意思.
我只想知道.
你到底是什麼意思.
我只想知道.
你到底是什麼意思.
我只想知道.
你到底是什麼意思.
我只想知道.
你到底是什麼意思.
我只想知道.
你到底是什麼意思.
我只想知道.
你到底是什麼意思.
我只想知道.
你到底是什麼意思.
我只想知道.
你到底是什麼意思.
我只想知道.
你到底是什麼意思.
我只想知道.
你到底是什麼意思.
我只想知道.
你到底是什麼意思.
我只想知道.
你到底是什麼意思.
我只想知道.
你到底是什麼意思.
我只想知道.
你到底是什麼意思.
我只想知道.
你到底是什麼意思.
我只想知道.
你到底是什麼意思.
我只想知道.
你到底是什麼意思.
我只想知道.
你到底是什麼意思.
我只想知道.
你到底是什麼意思.

$^{81}$我只想知道.
你到底是什麼意思.
我只想知道.
你到底是什麼意思.
我只想知道.
你到底是什麼意思.
我只想知道.
你到底是什麼意思.
我只想知道.
你到底是什麼意思.
我只想知道.
你到底是什麼意思.
我只想知道.
你到底是什麼意思.
我只想知道.
你到底是什麼意思.
我只想知道.
你到底是什麼意思.
我只想知道.
你到底是什麼意思.
我只想知道.
你到底是什麼意思.
我只想知道.
你到底是什麼意思.
我只想知道.
你到底是什麼意思.
我只想知道.
你到底是什麼意思.
我只想知道.
你到底是什麼意思.
我只想知道.
你到底是什麼意思.
我只想知道.
你到底是什麼意思.
我只想知道.
你到底是什麼意思.
我只想知道.
你到底是什麼意思.
我只想知道.
你到底是什麼意思.

$^{121}$我只想知道.
你到底是什麼意思.
我只想知道.
你到底是什麼意思.
我只想知道.
你到底是什麼意思.
我只想知道.
你到底是什麼意思.
我只想知道.
你到底是什麼意思.
我只想知道.
你到底是什麼意思.
我只想知道.
你到底是什麼意思.
我只想知道.
你到底是什麼意思.
我只想知道.
你到底是什麼意思.
我只想知道.
你到底是什麼意思.
我只想知道.
你到底是什麼意思.
我只想知道.
你到底是什麼意思.
我只想知道.
你到底是什麼意思.
我只想知道.
你到底是什麼意思.
我只想知道.
你到底是什麼意思.
我只想知道.
你到底是什麼意思.
我只想知道.
你到底是什麼意思.
我只想知道.
你到底是什麼意思.
我只想知道.
你到底是什麼意思.
我只想知道.
你到底是什麼意思.

$^{161}$我只想知道.
你到底是什麼意思.
我只想知道.
你到底是什麼意思.
我只想知道.
你到底是什麼意思.
我只想知道.
你到底是什麼意思.
我只想知道.
你到底是什麼意思.
我只想知道.
你到底是什麼意思.
我只想知道.
你到底是什麼意思.
我只想知道.
你到底是什麼意思.
我只想知道.
你到底是什麼意思.
我只想知道.
你到底是什麼意思.
我只想知道.
你到底是什麼意思.
我只想知道.
你到底是什麼意思.
我只想知道.
你到底是什麼意思.
我只想知道.
你到底是什麼意思.
我只想知道.
你到底是什麼意思.
我只想知道.
你到底是什麼意思.
我只想知道.
你到底是什麼意思.
我只想知道.
你到底是什麼意思.
我只想知道.
你到底是什麼意思.
我只想知道.
你到底是什麼意思.

$^{201}$我只想知道.
你到底是什麼意思.
我只想知道.
你到底是什麼意思.
我只想知道.
你到底是什麼意思.
我只想知道.
你到底是什麼意思.
我只想知道.
你到底是什麼意思.
我只想知道.
你到底是什麼意思.
我只想知道.
你到底是什麼意思.
我只想知道.
你到底是什麼意思.
我只想知道.
你到底是什麼意思.
我只想知道.
你到底是什麼意思.
我只想知道.
你到底是什麼意思.
我只想知道.
你到底是什麼意思.
我只想知道.
你到底是什麼意思.
我只想知道.
你到底是什麼意思.
我只想知道.
你到底是什麼意思.
我只想知道.
你到底是什麼意思.
我只想知道.
你到底是什麼意思.
我只想知道.
你到底是什麼意思.
我只想知道.
你到底是什麼意思.
我只想知道.
你到底是什麼意思.

$^{241}$我只想知道.
你到底是什麼意思.
我只想知道.
你到底是什麼意思.
我只想知道.
你到底是什麼意思.
我只想知道.
你到底是什麼意思.
我只想知道.
你到底是什麼意思.
我只想知道.
你到底是什麼意思.
我只想知道.
你到底是什麼意思.
我只想知道.
你到底是什麼意思.
我只想知道.
你到底是什麼意思.
我只想知道.
你到底是什麼意思.
我只想知道.
你到底是什麼意思.
我只想知道.
你到底是什麼意思.
我只想知道.
你到底是什麼意思.
我只想知道.
你到底是什麼意思.
我只想知道.
你到底是什麼意思.
我只想知道.
你到底是什麼意思.
我只想知道.
你到底是什麼意思.
我只想知道.
你到底是什麼意思.
我只想知道.
你到底是什麼意思.
我只想知道.
你到底是什麼意思.

$^{281}$我只想知道.
你到底是什麼意思.
我只想知道.
你到底是什麼意思.
我只想知道.
你到底是什麼意思.
我只想知道.
你到底是什麼意思.
我只想知道.
你到底是什麼意思.
我只想知道.
你到底是什麼意思.
我只想知道.
你到底是什麼意思.
我只想知道.
你到底是什麼意思.
我只想知道.
你到底是什麼意思.
我只想知道.
你到底是什麼意思.
我只想知道.
你到底是什麼意思.
我只想知道.
你到底是什麼意思.
我只想知道.
你到底是什麼意思.
我只想知道.
你到底是什麼意思.
我只想知道.
你到底是什麼意思.
我只想知道.
你到底是什麼意思.
我只想知道.
你到底是什麼意思.
我只想知道.
你到底是什麼意思.
我只想知道.
你到底是什麼意思.
我只想知道.
你到底是什麼意思.

$^{321}$我只想知道.
你到底是什麼意思.
我只想知道.
你到底是什麼意思.
我只想知道.
你到底是什麼意思.
我只想知道.
你到底是什麼意思.
我只想知道.
你到底是什麼意思.
我只想知道.
你到底是什麼意思.
我只想知道.
你到底是什麼意思.
我只想知道.
你到底是什麼意思.
我只想知道.
你到底是什麼意思.
我只想知道.
你到底是什麼意思.
我只想知道.
你到底是什麼意思.
我只想知道.
你到底是什麼意思.
我只想知道.
你到底是什麼意思.
我只想知道.
你到底是什麼意思.
我只想知道.
你到底是什麼意思.
我只想知道.
你到底是什麼意思.
我只想知道.
你到底是什麼意思.
我只想知道.
你到底是什麼意思.
我只想知道.
你到底是什麼意思.
我只想知道.
你到底是什麼意思.

$^{361}$我只想知道.
你到底是什麼意思.
我只想知道.
你到底是什麼意思.
我只想知道.
你到底是什麼意思.
我只想知道.
你到底是什麼意思.
我只想知道.
你到底是什麼意思.
我只想知道.
你到底是什麼意思.
我只想知道.
你到底是什麼意思.
我只想知道.
你到底是什麼意思.
我只想知道.
你到底是什麼意思.
我只想知道.
你到底是什麼意思.
我只想知道.
你到底是什麼意思.
我只想知道.
你到底是什麼意思.
我只想知道.
你到底是什麼意思.
我只想知道.
你到底是什麼意思.
我只想知道.
你到底是什麼意思.
我只想知道.
你到底是什麼意思.
我只想知道.
你到底是什麼意思.
我只想知道.
你到底是什麼意思.
我只想知道.
你到底是什麼意思.
我只想知道.
你到底是什麼意思.

$^{401}$我只想知道.
你到底是什麼意思.
我只想知道.
你到底是什麼意思.
我只想知道.
你到底是什麼意思.
我只想知道.
你到底是什麼意思.
我只想知道.
你到底是什麼意思.
我只想知道.
你到底是什麼意思.
我只想知道.
你到底是什麼意思.
我只想知道.
你到底是什麼意思.
我只想知道.
你到底是什麼意思.
我只想知道.
你到底是什麼意思.
我只想知道.
你到底是什麼意思.
我只想知道.
你到底是什麼意思.
我只想知道.
你到底是什麼意思.
我只想知道.
你到底是什麼意思.
我只想知道.
你到底是什麼意思.
我只想知道.
你到底是什麼意思.
我只想知道.
你到底是什麼意思.
我只想知道.
你到底是什麼意思.
我只想知道.
你到底是什麼意思.
我只想知道.
你到底是什麼意思.

$^{441}$我只想知道.
你到底是什麼意思.
我只想知道.
你到底是什麼意思.
我只想知道.
你到底是什麼意思.
我只想知道.
你到底是什麼意思.
我只想知道.
你到底是什麼意思.
我只想知道.
你到底是什麼意思.
我只想知道.
你到底是什麼意思.
我只想知道.
你到底是什麼意思.
我只想知道.
你到底是什麼意思.
我只想知道.
你到底是什麼意思.
我只想知道.
你到底是什麼意思.
我只想知道.
你到底是什麼意思.
我只想知道.
你到底是什麼意思.
我只想知道.
你到底是什麼意思.
我只想知道.
你到底是什麼意思.
我只想知道.
你到底是什麼意思.
我只想知道.
你到底是什麼意思.
我只想知道.
你到底是什麼意思.
我只想知道.
你到底是什麼意思.
我只想知道.
你到底是什麼意思.

$^{481}$我只想知道.
你到底是什麼意思.
我只想知道.
你到底是什麼意思.
我只想知道.
你到底是什麼意思.
我只想知道.
你到底是什麼意思.
我只想知道.
你到底是什麼意思.
我只想知道.
你到底是什麼意思.
我只想知道.
你到底是什麼意思.
我只想知道.
你到底是什麼意思.
我只想知道.
你到底是什麼意思.
我只想知道.
你到底是什麼意思.
我只想知道.
你到底是什麼意思.
我只想知道.
你到底是什麼意思.
我只想知道.
你到底是什麼意思.
我只想知道.
你到底是什麼意思.
我只想知道.
你到底是什麼意思.
我只想知道.
你到底是什麼意思.
我只想知道.
你到底是什麼意思.
我只想知道.
你到底是什麼意思.
我只想知道.
你到底是什麼意思.
我只想知道.
你到底是什麼意思.

$^{521}$我只想知道.
你到底是什麼意思.
我只想知道.
你到底是什麼意思.
我只想知道.
你到底是什麼意思.
我只想知道.
你到底是什麼意思.
我只想知道.
你到底是什麼意思.
我只想知道.
你到底是什麼意思.
我只想知道.
你到底是什麼意思.
我只想知道.
你到底是什麼意思.
我只想知道.
你到底是什麼意思.
我只想知道.
你到底是什麼意思.
我只想知道.
你到底是什麼意思.
我只想知道.
你到底是什麼意思.
我只想知道.
你到底是什麼意思.
我只想知道.
你到底是什麼意思.
我只想知道.
你到底是什麼意思.
我只想知道.
你到底是什麼意思.
我只想知道.
你到底是什麼意思.
我只想知道.
你到底是什麼意思.
我只想知道.
你到底是什麼意思.
我只想知道.
你到底是什麼意思.

$^{561}$我只想知道.
你到底是什麼意思.
我只想知道.
你到底是什麼意思.
我只想知道.
你到底是什麼意思.
我只想知道.
你到底是什麼意思.
我只想知道.
你到底是什麼意思.
我只想知道.
你到底是什麼意思.
我只想知道.
你到底是什麼意思.
我只想知道.
你到底是什麼意思.
我只想知道.
你到底是什麼意思.
我只想知道.
你到底是什麼意思.
我只想知道.
你到底是什麼意思.
我只想知道.
你到底是什麼意思.
我只想知道.
你到底是什麼意思.
我只想知道.
你到底是什麼意思.
我只想知道.
你到底是什麼意思.
我只想知道.
你到底是什麼意思.
我只想知道.
你到底是什麼意思.
我只想知道.
你到底是什麼意思.
我只想知道.
你到底是什麼意思.
我只想知道.
你到底是什麼意思.

$^{601}$我只想知道.
你到底是什麼意思.
我只想知道.
你到底是什麼意思.
我只想知道.
你到底是什麼意思.
我只想知道.
你到底是什麼意思.
我只想知道.
你到底是什麼意思.
我只想知道.
你到底是什麼意思.
我只想知道.
你到底是什麼意思.
我只想知道.
你到底是什麼意思.
我只想知道.
你到底是什麼意思.
我只想知道.
你到底是什麼意思.
我只想知道.
你到底是什麼意思.
我只想知道.
你到底是什麼意思.
我只想知道.
你到底是什麼意思.
我只想知道.
你到底是什麼意思.
我只想知道.
你到底是什麼意思.
我只想知道.
你到底是什麼意思.
我只想知道.
你到底是什麼意思.
我只想知道.
你到底是什麼意思.
我只想知道.
你到底是什麼意思.
我只想知道.
你到底是什麼意思.

$^{641}$我只想知道.
你到底是什麼意思.
我只想知道.
你到底是什麼意思.
我只想知道.
你到底是什麼意思.
我只想知道.
你到底是什麼意思.
我只想知道.
你到底是什麼意思.
我只想知道.
你到底是什麼意思.
我只想知道.
你到底是什麼意思.
我只想知道.
你到底是什麼意思.
我只想知道.
你到底是什麼意思.
我只想知道.
你到底是什麼意思.
我只想知道.
你到底是什麼意思.
我只想知道.
你到底是什麼意思.
我只想知道.
你到底是什麼意思.
我只想知道.
你到底是什麼意思.
我只想知道.
你到底是什麼意思.
我只想知道.
你到底是什麼意思.
我只想知道.
你到底是什麼意思.
我只想知道.
你到底是什麼意思.
我只想知道.
你到底是什麼意思.
我只想知道.
你到底是什麼意思.

$^{681}$我只想知道.
你到底是什麼意思.
我只想知道.
你到底是什麼意思.
我只想知道.
你到底是什麼意思.
我只想知道.
你到底是什麼意思.
我只想知道.
你到底是什麼意思.
我只想知道.
你到底是什麼意思.
我只想知道.
你到底是什麼意思.
我只想知道.
你到底是什麼意思.
我只想知道.
你到底是什麼意思.
我只想知道.
你到底是什麼意思.
我只想知道.
你到底是什麼意思.
我只想知道.
你到底是什麼意思.
我只想知道.
你到底是什麼意思.
我只想知道.
你到底是什麼意思.
我只想知道.
你到底是什麼意思.
我只想知道.
你到底是什麼意思.
我只想知道.
你到底是什麼意思.
我只想知道.
你到底是什麼意思.
我只想知道.
你到底是什麼意思.
我只想知道.
你到底是什麼意思.

$^{721}$我只想知道.
你到底是什麼意思.
我只想知道.
你到底是什麼意思.
我只想知道.
你到底是什麼意思.
我只想知道.
你到底是什麼意思.
我只想知道.
你到底是什麼意思.
我只想知道.
你到底是什麼意思.
我只想知道.
你到底是什麼意思.
我只想知道.
你到底是什麼意思.
我只想知道.
你到底是什麼意思.
我只想知道.
你到底是什麼意思.
我只想知道.
你到底是什麼意思.
我只想知道.
你到底是什麼意思.
我只想知道.
你到底是什麼意思.
我只想知道.
你到底是什麼意思.
我只想知道.
你到底是什麼意思.
我只想知道.
你到底是什麼意思.
我只想知道.
你到底是什麼意思.
我只想知道.
你到底是什麼意思.
我只想知道.
你到底是什麼意思.
我只想知道.
你到底是什麼意思.

$^{761}$我只想知道.
你到底是什麼意思.
我只想知道.
你到底是什麼意思.
我只想知道.
你到底是什麼意思.
我只想知道.
你到底是什麼意思.
我只想知道.
你到底是什麼意思.
我只想知道.
你到底是什麼意思.
我只想知道.
你到底是什麼意思.
我只想知道.
你到底是什麼意思.
我只想知道.
你到底是什麼意思.
我只想知道.
你到底是什麼意思.
我只想知道.
你到底是什麼意思.
我只想知道.
你到底是什麼意思.
我只想知道.
你到底是什麼意思.
我只想知道.
你到底是什麼意思.
我只想知道.
你到底是什麼意思.
我只想知道.
你到底是什麼意思.
我只想知道.
你到底是什麼意思.
我只想知道.
你到底是什麼意思.
我只想知道.
你到底是什麼意思.
我只想知道.
你到底是什麼意思.

$^{801}$我只想知道.
你到底是什麼意思.
我只想知道.
你到底是什麼意思.
我只想知道.
你到底是什麼意思.
我只想知道.
你到底是什麼意思.
我只想知道.
你到底是什麼意思.
我只想知道.
你到底是什麼意思.
我只想知道.
你到底是什麼意思.
我只想知道.
你到底是什麼意思.
我只想知道.
你到底是什麼意思.
我只想知道.
你到底是什麼意思.
我只想知道.
你到底是什麼意思.
我只想知道.
你到底是什麼意思.
我只想知道.
你到底是什麼意思.
我只想知道.
你到底是什麼意思.
我只想知道.
你到底是什麼意思.
我只想知道.
你到底是什麼意思.
我只想知道.
你到底是什麼意思.
我只想知道.
你到底是什麼意思.
我只想知道.
你到底是什麼意思.
我只想知道.
你到底是什麼意思.

$^{841}$我只想知道.
你到底是什麼意思.
我只想知道.
你到底是什麼意思.
我只想知道.
你到底是什麼意思.
我只想知道.
你到底是什麼意思.
我只想知道.
你到底是什麼意思.
我只想知道.
你到底是什麼意思.
我只想知道.
你到底是什麼意思.
我只想知道.
你到底是什麼意思.
我只想知道.
你到底是什麼意思.
我只想知道.
你到底是什麼意思.
我只想知道.
你到底是什麼意思.
我只想知道.
你到底是什麼意思.
我只想知道.
你到底是什麼意思.
我只想知道.
你到底是什麼意思.
我只想知道.
你到底是什麼意思.
我只想知道.
你到底是什麼意思.
我只想知道.
你到底是什麼意思.
我只想知道.
你到底是什麼意思.
我只想知道.
你到底是什麼意思.
我只想知道.
你到底是什麼意思.

$^{881}$我只想知道.
你到底是什麼意思.
我只想知道.
你到底是什麼意思.
我只想知道.
你到底是什麼意思.
我只想知道.
你到底是什麼意思.
我只想知道.
你到底是什麼意思.
我只想知道.
你到底是什麼意思.
我只想知道.
你到底是什麼意思.
我只想知道.
你到底是什麼意思.
我只想知道.
你到底是什麼意思.
我只想知道.
你到底是什麼意思.
我只想知道.
你到底是什麼意思.
我只想知道.
你到底是什麼意思.
我只想知道.
你到底是什麼意思.
我只想知道.
你到底是什麼意思.
我只想知道.
你到底是什麼意思.
我只想知道.
你到底是什麼意思.
我只想知道.
你到底是什麼意思.
我只想知道.
你到底是什麼意思.
我只想知道.
你到底是什麼意思.
我只想知道.
你到底是什麼意思.

$^{921}$我只想知道.
你到底是什麼意思.
我只想知道.
你到底是什麼意思.
我只想知道.
你到底是什麼意思.
我只想知道.
你到底是什麼意思.
我只想知道.
你到底是什麼意思.
我只想知道.
你到底是什麼意思.
我只想知道.
你到底是什麼意思.
我只想知道.
你到底是什麼意思.
我只想知道.
你到底是什麼意思.
我只想知道.
你到底是什麼意思.
我只想知道.
你到底是什麼意思.
我只想知道.
你到底是什麼意思.
我只想知道.
你到底是什麼意思.
我只想知道.
你到底是什麼意思.
我只想知道.
你到底是什麼意思.
我只想知道.
你到底是什麼意思.
我只想知道.
你到底是什麼意思.
我只想知道.
你到底是什麼意思.
我只想知道.
你到底是什麼意思.
我只想知道.
你到底是什麼意思.

$^{961}$我只想知道.
你到底是什麼意思.
我只想知道.
你到底是什麼意思.
我只想知道.
你到底是什麼意思.
我只想知道.
你到底是什麼意思.
我只想知道.
你到底是什麼意思.
我只想知道.
你到底是什麼意思.
我只想知道.
你到底是什麼意思.
我只想知道.
你到底是什麼意思.
我只想知道.
你到底是什麼意思.
我只想知道.
你到底是什麼意思.
我只想知道.
你到底是什麼意思.
我只想知道.
你到底是什麼意思.
我只想知道.
你到底是什麼意思.
我只想知道.
你到底是什麼意思.
我只想知道.
你到底是什麼意思.
我只想知道.
你到底是什麼意思.
我只想知道.
你到底是什麼意思.
我只想知道.
你到底是什麼意思.
我只想知道.
你到底是什麼意思.
我只想知道.
你到底是什麼意思.

$^{1001}$我只想知道.
你到底是什麼意思.
我只想知道.
你到底是什麼意思.
我只想知道.
你到底是什麼意思.
我只想知道.
你到底是什麼意思.
我只想知道.
你到底是什麼意思.
我只想知道.
你到底是什麼意思.
我只想知道.
你到底是什麼意思.
我只想知道.
你到底是什麼意思.
我只想知道.
你到底是什麼意思.
我只想知道.
你到底是什麼意思.
我只想知道.
你到底是什麼意思.
我只想知道.
你到底是什麼意思.
我只想知道.
你到底是什麼意思.
我只想知道.
你到底是什麼意思.
我只想知道.
你到底是什麼意思.
我只想知道.
你到底是什麼意思.
我只想知道.
你到底是什麼意思.
我只想知道.
你到底是什麼意思.
我只想知道.
你到底是什麼意思.
我只想知道.
你到底是什麼意思.

$^{1041}$我只想知道.
你到底是什麼意思.
我只想知道.
你到底是什麼意思.
我只想知道.
你到底是什麼意思.
我只想知道.
你到底是什麼意思.
我只想知道.
你到底是什麼意思.
我只想知道.
你到底是什麼意思.
我只想知道.
你到底是什麼意思.
我只想知道.
你到底是什麼意思.
我只想知道.
你到底是什麼意思.
我只想知道.
你到底是什麼意思.
我只想知道.
你到底是什麼意思.
我只想知道.
你到底是什麼意思.
我只想知道.
你到底是什麼意思.
我只想知道.
你到底是什麼意思.
我只想知道.
你到底是什麼意思.
我只想知道.
你到底是什麼意思.
我只想知道.
你到底是什麼意思.
我只想知道.
你到底是什麼意思.
我只想知道.
你到底是什麼意思.
我只想知道.
你到底是什麼意思.

$^{1081}$我只想知道.
你到底是什麼意思.
我只想知道.
你到底是什麼意思.
我只想知道.
你到底是什麼意思.
我只想知道.
你到底是什麼意思.
我只想知道.
你到底是什麼意思.
我只想知道.
你到底是什麼意思.
我只想知道.
你到底是什麼意思.
我只想知道.
你到底是什麼意思.
我只想知道.
你到底是什麼意思.
我只想知道.
你到底是什麼意思.
我只想知道.
你到底是什麼意思.
我只想知道.
你到底是什麼意思.
我只想知道.
你到底是什麼意思.
我只想知道.
你到底是什麼意思.
我只想知道.
你到底是什麼意思.
我只想知道.
你到底是什麼意思.
我只想知道.
你到底是什麼意思.
我只想知道.
你到底是什麼意思.
我只想知道.
你到底是什麼意思.
我只想知道.
你到底是什麼意思.

$^{1121}$我只想知道.
你到底是什麼意思.
我只想知道.
你到底是什麼意思.
我只想知道.
你到底是什麼意思.
我只想知道.
你到底是什麼意思.
我只想知道.
你到底是什麼意思.
我只想知道.
你到底是什麼意思.
我只想知道.
你到底是什麼意思.
我只想知道.
你到底是什麼意思.
我只想知道.
你到底是什麼意思.
我只想知道.
你到底是什麼意思.
我只想知道.
你到底是什麼意思.
我只想知道.
你到底是什麼意思.
我只想知道.
你到底是什麼意思.
我只想知道.
你到底是什麼意思.
我只想知道.
你到底是什麼意思.
我只想知道.
你到底是什麼意思.
我只想知道.
你到底是什麼意思.
我只想知道.
你到底是什麼意思.
我只想知道.
你到底是什麼意思.
我只想知道.
你到底是什麼意思.

$^{1161}$我只想知道.
你到底是什麼意思.
我只想知道.
你到底是什麼意思.
我只想知道.
你到底是什麼意思.
我只想知道.
你到底是什麼意思.
我只想知道.
你到底是什麼意思.
我只想知道.
你到底是什麼意思.
我只想知道.
你到底是什麼意思.
我只想知道.
你到底是什麼意思.
我只想知道.
你到底是什麼意思.
我只想知道.
你到底是什麼意思.
我只想知道.
你到底是什麼意思.
我只想知道.
你到底是什麼意思.
我只想知道.
你到底是什麼意思.
我只想知道.
你到底是什麼意思.
我只想知道.
所以前纖是一種嘗試.
將上帝的國度.
耶穌基督回來的東西.
去看為一些.
時間化了它.
或者將它歷史化了.
這是一個問題.
我會說為什麼.
總之是這三種的縱末論.
我想說就是.
前纖是嘗試.

$^{1201}$將耶穌基督回來的這種盼望.
看為一種將來.
某年某月某日.
發生的事情.
而你就等待那天回來.
而毛遷就覺得.
其實就是沒有這種.
聖經的方法.
這樣理解.
當然奧斯丁就是毛遷欺的.
他不覺得聖經裡面.
的契捨祿經文.
是嘗試去撲出一個時間表.
將耶穌基督回來.
看為一個時間表.
是有些問題的.
我會說為什麼.
簡單來說.
縱末論裡面有三個不同的看法.
後纖是不用理會的.
因為是一百年前.
烏托邦的看法.
前纖是比較福音派.
字面保守解經的方法.
嘗試.
那時候聽過.
極端一點就是.
那些時代論.
波斯灣戰爭.
末世來到.
覺得耶穌回來前有大災難.
嘗試去撲出一條.
歷史時間.
而毛遷就覺得.
其實是沒有這些.
這些的偏派.
當然我們要說的是.
面對這種.
終末.
我們身處在一個.

$^{1241}$什麼狀態裡面呢.
可能大家聽過.
中文叫以言未言.
Already but not yet.
就是說.
好像雖然已經是.
發生了.
但又未來到.
這種很矛盾.
很吊詭的張力裡面.
耶穌基督已經在.
不過又未完全在.
上帝的國度已經來臨.
但又未完全來臨.
你很喜樂.
但又沒什麼喜樂.
這種很吊詭的狀態.
為什麼會這樣呢.
正是我們身處.
這種狀態裡面.
我們正是在一個.
現在的時間.
和將來的時間.
兩種矛盾裡面.
之前我講道講過.
我講那篇.
關乎於道.
因為講這件事.
但就是說.
我們其實.
還教會的盼望.
其實正是建基於.
剛才講前千的那種.
看法裡面.
他將耶穌基督.
回來.
看為一種什麼.
一種將來會發生的事.
即是今天.
還沒發生的事情.

$^{1281}$所以他覺得.
什麼叫盼望耶穌.
盼望耶穌.
就是一種我們.
純粹等住再來.
你的盼望.
就是住再來.
你對於今天.
這個世界.
我們以前唱的那些詩歌.
基本上.
到哪天.
耶穌就回來了.
今天我們就等他回來了.
這種去到極端化.
是一種像UFO那些教.
我們整個基督徒.
就不需要再理現在的事.
我們就純粹等.
耶穌回來.
很多異端都是這樣.
從奧姆真理教.
到很多其他的.
看法都是這樣.
他們覺得盼望.
就是住末日的時間裡面.
所以我們今天的世代邪惡.
我們就不要理現在的世界.
你就不要.
甚至極端點.
就不要做事.
你就辭職那份工作.
等UFO回來.
接你走.
這種是很極端的版本.
就是說.
你是越指望末日來臨的時候.
你就越對現今的世界.
是沒有任何參與.
或者改變的思想.

$^{1321}$所以當然環球會.
不比他們那麼極端.
但環球會是有這樣的傾向.
我們以前那些.
一講到朱末侖的時候.
就是什麼.
現在這個世道黑暗.
又攻打民.
我們就快點等朱末侖來.
所以這種盼望是.
盼望是什麼.
沒有的.
現在沒有盼望.
你只能等耶穌回來.
就是盼望.
所以這種盼望是對於.
現今的世界.
是沒有任何的改變.
或者是任何的參與.
他會叫你.
要不就等朱末侖來.
這種盼望是什麼.
就是你回教會.
回教會崇拜最好.
回教會崇拜.
敬拜很好.
回到外面就黑暗世代.
就不要理他.
回教會敬拜.
這種是一種方法.
所以這種叫做.
如果你很強調這種前遷.
當然我怕被人罵.
前遷不一定是問題.
前遷引出來的是一種.
對於耶穌回來.
一種完全擺在歷史上的將來的時候.
那你就和他沒有關係.
所以今天我們所說.
其實我們基督徒的盼望.

$^{1361}$並不是這樣.
不是純粹等待一個遙遠的將來.
那一天到來.
而是和我們的現在.
和你的生活有關係.
所以我們對於耶穌的盼望.
不是純粹回來.
而是希望能夠更加多的東西.
當然我們盼望有很多不同的方式.
今天可能你在小組裡面談也是.
對於盼望有很多不同的理解.
我們叫做false hope.
假的盼望.
有些盼望是建基於我們的理性.
因為我覺得.
我對於香港社會的預測是怎麼樣.
我對於整個世界的經濟預測是怎麼樣.
從而得出一個結論.
這種盼望是建基於人的理性.
我對於世界怎麼理解的時候.
我就有盼望.
計算不清就沒有盼望.
這個並不是真正的盼望.
起碼不是信仰的盼望.
信仰盼望不是純粹一種人類的計算.
確實是這樣的.
面對現在我們面臨的社會的時候.
確實是你怎麼計算.
你也計算不了有盼望.
正如剛才所說.
盼望並不是在乎於我們自己所可見的事情.
所以盼望不是一種人的計算.
也不是一種脫離現實.
純粹寄居於耶穌回來的那種終末的盼望.
而是什麼呢.
我們就要去看看.
要說盼望就不能不說莫特曼.
莫特曼寫了一本書叫《來臨中的上帝》.
The Coming of God.
他用了一句很有意思的經文.

$^{1401}$就是這個啟示錄里的第一章.
他說:今在,昔在,來臨中的上帝.
很奇怪的.
發現在啟示錄里的和本不是這樣翻譯的.
他說的是今在,昔在,而又永在.
但原文里不是永在.
而是coming的字.
回來的意思.
所以原來我們所說的上帝耶穌.
他不是一個純粹今在,昔在,永在.
而是一個正在回來的上帝.
coming的上帝.
而這個回來是什麼意思呢.
這個回來又不是那種.
除了說1984年6月9日那天回來.
而是耶穌不斷地在回來.
在未來裡面.
從未來跑回來現在裡面.
所以我喜歡畫這幅圖.
我展示裡面.
昔在,以前,今在,即是present.
而將來.
耶穌也是在將來那裡跑回來.
所以對於將來的時間.
仍然是充滿著可能性.
仍然是充滿著那種可塑性.
將來並不是和我們沒有關係.
更加不是完全定的.
將來仍然是一個充滿著可變的方法.
而耶穌正正是那位來臨中的上帝.
所以我們今天對於我們.
無論是香港的將來.
或者你生命的將來都一樣.
當我們說盼望的時候.
我們不是等主回來接我們走.
而是對於這個世界的將來.
仍然是充滿著那種可塑性的盼望.
我們希望能夠改變它.
我們知道耶穌正正是掌管著這個未來.
所以對莫特曼來說.

$^{1441}$或者對我來說.
盼望耶穌是一件很實際的事情.
不是純粹等UFO回來.
等主再來.
而是我們對於今天的世界里.
仍然是充滿著那種行動和改變.
這就是我們想強調的.
不是傳統華人教會所說.
那種終末災難性的主再來.
而是今天我們對於生活的力量.
我們對於面對黑暗實在變化.
我們如何來參與和改變.
當然我們剛才說.
當我們預見著將來的時候.
是充滿著很多的張力.
這個我講過的.
我都講過一次.
特別是在哥倫多前說里.
弟兄們我對你們說.
時間減少了.
這個時間減少了一個很特別的字眼.
就是發覺一個很弔詭的現象.
從此以後乃有七字的.
就好像每七字一樣.
哀哭的要像沒哭.
快樂的要像不快樂.
自慢的要像沒有所得.
用世物的要像不用世物.
一個很弔詭的狀態.
有老婆就等於沒老婆.
哭也不哭 快樂也不快樂.
什麼意思呢.
因為我們正正是處在兩個時間點裡面.
一個是我們熟悉的時間.
一個是我們知道上帝的時間.
所以我說過.
這個時間的意思是解作什麼呢.
時間減少了.
是解作時間的那種壓縮.
就像一個疊疊了的時間一樣.

$^{1481}$我們同時身處在兩個時間當中.
一個是世界所見到的時間.
2023年.
一個是上帝的時間.
終末的時間.
所以我們今天.
already but not yet.
我們基督徒正正是這樣.
似乎開始面對一個很艱難的世代.
同時我們又知道.
上帝的結局.
上帝的永恆.
已經在今天的裡面.
所以我上次也說過.
我們基督徒戴了兩只手上的標.
一個是我們普通時間的標.
一個是上帝的標.
所以基督徒是這樣.
你又知道今天好像沒什麼指望.
但我們仍然有一種盼望在當中.
這種正正就是我所說的.
這個盼望的開始.
所以我們見到.
當我們面對這個世界的時候.
這個世界的表象正在過去.
雖然它現在是很真實的.
但知道它正在fade away.
這個就是《仰仰書所》的經文.
真光已經照耀.
黑暗卻漸漸過去.
這個是不是我們經驗那種的經驗.
比如我們回家一開燈.
晚上回家一開燈.
一開就立刻光了.
但這個經文是什麼呢.
真光已經照耀.
黑暗是慢慢慢慢過去.
所以這個就是這麼奇怪.
一方面我們面對的是耶穌的真光.
但這個真光又不是立刻一止.

$^{1521}$一下子就光了.
這個正正就是.
already but not yet的意思.
正正是兩只手上的標的意思.
所以我們的盼望.
是建基於這個時間的觀念裡面.
我們的盼望不是純粹我們的計算.
也不是一種等待著來的看法.
而是我們知道今天黑暗.
但我們仍然可以看到.
仍然相信.
仍然是期盼.
知道這種上帝的盼望.
所以看到這種盼望是有點.
霍神經的.
不是很說得出來的.
因為確實是你解釋不了的.
正正是保祿所說的.
他在毫無盼望的時候.
仍懷著盼望而相信.
這個是保祿在羅馬書裡面所說的經文.
而經文裡面.
正正是一個很有趣的英文詞.
就叫做hope against hope.
這個盼望正正是和一切人的盼望.
是對敵的.
是相反的.
就是人所能夠期盼的.
所依靠的.
無論是靠那種理性分析的形式.
或者大局.
或者是那種純粹是宗教的盼望.
我們基督徒的盼望.
其實是一切盼望的相反.
我們不是去看現在.
面對的事情有多樂觀和悲觀.
而是我們純粹那兩只手.
根據著我們知道上帝留給我們的手.
一個終末的手.
我們身處在兩個時代裡面.

$^{1561}$所以我們仍然能夠知道我們仍然有盼望.
所以簡單來說.
任何便宜的盼望.
任何easy hope.
都不是真正的盼望.
所以就像傻強所說.
總之任何說自己是盼望的.
那些都不是盼望.
任何容易得著的盼望.
那些都不是真正的盼望.
所以布魯德說.
得救是在乎盼望.
只是所看見的盼望不是盼望.
誰還盼望所他看見得見的呢.
所以盼望的課題其實是很不容易的.
我預查的時候都和牧者談過這些話題.
說盼望很容易.
但要等到能夠掌握到這種盼望.
其實又不容易.
因為任何你會覺得很容易的盼望.
這些都未必是信仰裡面所說的盼望.
所以去年我們的研究所的那套劇很好.
他想說絕望是什麼.
絕望是很難的.
完全的絕望是很難的.
因為我們知道在任何情況之下.
完全絕望是很不容易的.
我只能這麼說.
但是那種盼望你只能夠好好地去經驗出來.
我不能夠傳給你的.
這個就是這樣,你拿吧.
盼望你只能夠慢慢地在絕境裡面掌握出來.
所以這個就是我們今天嘗試知道.
盼望是關乎於上帝的時間.
盼望是一切盼望的相反.
我們不是去依靠一切其他人類的可能性.
所以既然這麼難.
究竟有什麼可以實踐呢.
我們想說一種能夠實踐的盼望.
特地我用了背景的圖.

$^{1601}$不知道大家記不記得這幅圖.
這是幾年了.
兩三年前一場比賽.
利物浦對巴塞隆拿上半場.
上半場輸了3比0.
下半場就贏了4比0.
這個我覺得是一個很經典的盼望的例子.
其實那天我沒有看這場比賽.
我也是沒有盼望.
我預料會輸,所以沒有看這場比賽.
就是這樣.
原來我們的盼望就是這些東西.
我們嘗試去實踐一些盼望.
這種盼望其實是嘗試去變成一場球.
很容易在我們生活裡面出現.
會是嘗試.
因為我們要知道盼望其實不是等待的東西.
不是等待住在外面那麼簡單.
比特厚書說得好.
他說是切切仰望上帝日子的來到.
這種仰望,這種等待上帝回來.
沒錯,你只能等待.
你好像沒有什麼能夠做到上帝回來.
一個字很重要,就是切切.
語文裡面的切切就是hastening.
就是很急速,很催促.
一方面是等待,好像很被動.
但其實他知道我們要很迫切地等待上帝.
什麼意思呢?.
就是我們仍然用很多方法將盼望變成一些行動.
沒錯,我們不能夠逼上帝回來.
也不能夠做些什麼去改變這個世界的未來.
不過我們確實是嘗試用盡我們的心思和方法.
去做一些事,去改變一些事.
所以我想說,這個我寫了一篇課讀.
當我說實踐盼望的時候,我有兩個風格要說.
第一,我們需要有一些事是可以做的.
既然是實踐盼望,我嘗試一些事是可以做的.
這是第一點,我們不是純粹等待,不是等待耶穌回來.
我們起碼要付出盼望,希望能夠是主動的,去做的.

$^{1641}$但你知道做是一種等待,你做不了什麼,改變不了什麼.
但你也要做些事,這是第一點.
我們的應用實踐上,第一我們是希望能夠做些事出來.
第二,就是練習的意義是什麼呢?.
今天想說,以下所說的練習,所謂的應用.
其實做了不代表有的,你想說.
你不明白什麼叫練習嗎?.
練習就是說,其實你是去嘗試去做.
但那些東西其實跟後果沒什麼關係.
好像一個體操運動員,我們會去跳鞍馬,兩個半空翻落地.
你會不斷地練習出來,這個練習是有意義的.
因為你去練習,其實整件事是有關係的.
不過是否說你練完這些東西就能夠做到呢?.
其實又不一定.
所以這個練習,不是說以下的就叫做盼望.
盼望不等於我做完就有盼望.
不過盼望往往都有這些東西做出來.
明白嗎?因為我想說,這就是困難的地方.
因為你去嘗試去實踐的時候,你會不斷地重復去做這些東西.
有什麼例子呢?.
譬如說,我說的就是開電單車的例子.
你開車,你學一個運動也好,你學車也好.
你會不斷地重復一些動作行為.
轉幾個指,然後你會扭車輪,踩油門這些東西.
你狂練這些東西出來,其實不代表你會開車.
但會開車肯定會會做這些東西.
所以你不斷去練習這些東西是和盼望有關的.
但不代表你做足都有盼望.
有沒有盼望呢?其實真的沒得教的.
盼望是關乎什麼?就是上帝時間.
你對於上帝時間的那只標有多大的體會.
你看到今天香港,你能夠知道上帝的光輝已經照耀了.
你就有盼望,就不關那些動作事.
但如果要教你的話,我只能夠說盼望就是會做這些東西.
就是這一下這些東西,明不明白?.
所以實踐就是這樣做,但這些東西不等同於盼望.
只不過有盼望的人會做這些東西.
所以我說我們會嘗試去做這些東西.
嘗試去實踐這些東西出來.
第二就是我們嘗試知道這些東西不需要包括.

$^{1681}$第一個就是想象和計劃.
一個盼望的人,其實他正正是對於將來仍然有計劃和想象的人.
我說過了,我說過了.
如果你發現在凶案現場有個死屍拿著一張鏡頭會飛的話.
他應該是他殺的,不是自殺的.
因為他仍然對於將來有盼望.
所以我們嘗試去實踐我們的盼望的時候.
不妨對於你的將來,或者對於香港社會的將來.
或者這個世界的將來,你仍然可以去計劃一些東西.
透過這種計劃,其實是慢慢去學習對於我們的將來.
是可以有一些想象的.
所以這個第一個,你發現兩個字很特別.
一個字就是計劃,一個比較理性的思考.
這兩個都是從無到有的,都是你腦子里的東西來的.
將來的,未發生的.
所以對於將來,我們仍然可以嘗試去計劃或者想象.
大家不知道這幅圖是一個什麼歷史事件.
是,柏林圍牆.
柏林圍牆是人類歷史里最激發人想象力的一個事情.
為什麼呢?.
因為當突然一天之間,你知道歷史是什麼嗎?.
蘇聯突然一天之間建了一座牆,封鎖了整個柏林,分成兩邊.
一天之間,東柏林和西柏林是分開了兩邊.
人們不能夠翻牆去另一邊.
所以這個封鎖,這個牆,這個失去的自由,令人有很多的想象.
如果你去柏林的話,柏林有一個叫柏林圍牆博物館的.
有一個很有意思的說話,他說人類因為尋找自由的緣故,激發無限的想象力.
當時有些人為了要越過這一面牆,有很多不同很有創意的想象.
有人試過用家裡的床單,編織一條繩子就爬過去.
有人試過弄輕氣球飛過去.
有人試過在車尾箱里躲起來就開車過去.
有人試過扮警察走過去.
原來我們對於將來的盼望,其實和我們的想象有很大關係.
一個沒有盼望的人,對於將來是沒有任何的想象.
或者不願意去想象,也不有意不去想象任何的計劃.
所以,我想說一次,有這些東西是不代表你有盼望.
但如果你想操練一下盼望的話,你可以嘗試去做這些東西.
如果你覺得自己今天對於你自己任何一個方面沒有任何盼望的時候.
不妨嘗試去想象,不妨嘗試去計劃.
任何對於將來的想法都是一種盼望的練習.

$^{1721}$這是我想說的第一點,你嘗試不妨去思考將來.
第二個就是行善,我覺得是.
這個也是我們在《牧者御茶》的時候提出的一個很重要的點.
特別是面對著今天這個社會.
當你對於這個社會仍然有盼望的時候.
在一個黑暗的時代裡面你仍然有盼望的時候.
你做一件好事出來,這個已經是一個最好的盼望的名證.
你仍然願意在當中做一件好事出來.
這個就是盼望的起點.
我說過的也是,以前有一部電影叫做《熔爐》.
可能大家聽過,整個《熔爐》的電影是一個很黑暗的結局.
因為一個法律的不公平,整個不公義.
最後其實是沒有好結果的.
主角裡面的小朋友全部都得到不公義的對待.
其實是這樣結束的.
但是導演怎麼去結束這麼悲傷的電影故事呢?.
最後就是這群人雖然輸了.
仍然面對著很多不公義的對待.
但這群人一起在這裡過聖誕節,一起在這裡吃東西.
在黑暗裡面你仍然做一些很微小的好事出來.
沒錯,你改變不了這個世界.
但你仍然去嘗試去行善的時候.
這個是我們一些盼望的信號.
記住我說過一次,有些事情不代表得盼望.
但你去嘗試,仍然堅持自己去做美善的事.
這個是我們對於盼望的一些實踐的操練.
這幅圖我們會在以後首播.
我們一些片段.
所以我們教會仍然希望將我們的社關放在我們很重要的位置裡面.
因為這是我們作為盼望的群體.
一個很重要的實踐出來的一個表徵.
第三個我想講的仍然是一個很老土的題目.
就是土告.
土告為什麼和盼望這麼大的關係呢?.
這個我都提過.
有一個很好的天才哲學家Joseph Pieper有提過.
他說prayer is nothing other than the voicing of hope.
土告正正是我們盼望的聲音.
因為每一個土告.
全部都是你和將來去打交道的聲音.

$^{1761}$當然感恩土告關乎於過去.
但任何一個祈求.
今晚不要下雨.
到你的子女能夠出去讀書.
能夠平安到任何的東西.
任何的土告都是和將來打交道的.
你將你的將來.
你將香港社會將來放在土告裡面的時候.
這個就是最直接來將盼望的聲音放在你的面前.
這個我覺得是最重要的.
剛才所講的可能都是一些做了不代表有的事情.
但土告正正是因基於我們的相信.
更加是對於我們對於未來.
置放在我們的上帝面前.
所以仍然堅持來為未來去土告.
特別是為香港的未來土告.
我不知道你還有沒有為香港未來土告.
真的.
可能已經有兩三年沒有為香港未來土告.
當我們連土告都沒有做到的時候.
當我們連土告都不敢覺得是那些值得相信的時候.
這個正正是我們沒有盼望的開始.
或者是一個沒有盼望的表徵.
所以不妨重新來去取回這個土告.
為香港的未來去土告.
這個我覺得是我們留堂仍然很堅持的東西.
所以看到潘Sir每個月裡面月土.
很多時候都是這樣.
仍然堅持為香港去土告.
是講的不是純粹過去的事.
而是我們整個香港社會的未來.
仍然相信香港或者這個世界仍然是可以被改變的.
我們不是等到自己死了能夠回到天家.
那就叫做最好的結局.
我們仍然覺得我們的世界是此刻當下仍然可以被改變.
這個就是我們的盼望.
最後我想講Full Church是一個盼望的群體.
因為這個是我自己很特別很想提出來的東西.
當然Full Church是一個有愛群體.
一個有信仰群體.

$^{1801}$但我覺得在這個年代裡面.
Full Church更加需要被稱作是一個盼望群體.
首先每個頂尖輩是真的擁有盼望.
沒有的話立刻去操練他.
提醒自己如何可以有.
在諸如神學聖經裡面.
在土告裡面.
在實踐裡面.
令自己有盼望.
從而把這份盼望給你身邊的人.
這個我覺得是Full Church最需要.
為什麼要存在的原因.
我覺得教會是一個見證耶穌基督的群體.
我已經講過很多次了.
而我覺得盼望正正是這個年頭裡面.
最重要告訴人們.
需要人認識耶穌的那個向度.
我很期望更加多人在Full Church裡面.
看到盼望.
或者因為我們能夠將耶穌的盼望帶給他.
所以他能夠信耶穌.
所以這個我覺得是很重要的.
人們認識耶穌不是因為上了天堂.
還是因為癌症可以被醫治.
而是耶穌的盼望能夠幫助他.
所以這個我覺得是很想.
我們Full Church的教會能夠可以掌握的東西.
我們先講這麼多.
一會兒有些問答的時間.
希望大家能夠去問一下.
在你心目中如何能夠實踐盼望呢.
對於盼望的房地有什麼看法.
可以和大家談談.
很期待和大家談.
我們有祈禱時間.
我們將教會交托給你.
最後求主讓你在當中.
無論是今天直播.
或者將來每一個頂智媒去看第六科.
讓他能夠掌握到.

$^{1841}$盼望你給我們這麼大的恩賜.
因為你的緣故你給我們有盼望.
讓我們能夠在無論黑暗裡.
都仍然有一個不是人所能夠促成的盼望.
而是單單因為你的緣故.
我們所掌握的盼望.
求主幫我們這個教會.
能夠成為一個有盼望的教會.
一個能夠全陽盼望的教會.
讓更多人實踐盼望的教會.
求主幫助我們.
奉主命求.
阿們.
盼望來到了.
放了兩個月的假期.
有沒有休息.
都休息了很多.
希望教會盼望.
其實都是最困難的.
是不是也不容易.
和弟子妹說盼望的主題.
單單以最後你這張照片.
在戶外崇拜就真的很不容易.
大家都希望在公共空間繼續敬拜.
在公共空間繼續見證.
其實都是我們整個信仰群體.
想實踐這個盼望.
我為什麼會選這張照片呢.
因為這張照片正正就是.
這個我們月頭想做的事.
就是我們一班人一起去做一些事.
這件事其實.
做完之後有什麼果後是不知道的.
但我們仍然對於將來是有想像的.
對將來是有計劃的.
我們就做了.
這正正是我們去實踐盼望.
很重要的一個例子.
正正是這樣.
正正是一班人.

$^{1881}$無論是多無聊.
多不知道有沒有功用也好.
仍然去堅持去行善.
這正正是我們希望能夠做的事.
對呀.
不知道大家整晚聽了很多盼望這兩個字.
對你們來說.
這一刻你的盼望是什麼.
或者你本身有沒有什麼盼望的想法呢.
或者大家可以分享下.
和大家互動下.
或者盼望對大家難不難呀.
後面好像有些聲音.
哈哈.
是不是已經很難了.
很難有盼望了.
不是的.
OK.
或者大家可不可以分享你怎麼去有盼望的.
想問一下.
剛才提到有些操練的辦法.
但是很強調的就是.
做了那些事不代表自己有盼望.
為什麼要強調做了那些事都不等於有盼望呢.
什麼時候才知道自己很有盼望呢.
所以就說一個運動員例子.
你明白到一件事.
其實你重復那件事是有用的.
有幫助.
但是不是等同於就叫做會的.
所以一個盼望的人就有這些事.
我們嘗試有盼望就做多些這類的事.
所以我覺得是.
因為盼望不是很容易實現.
我也不知道怎麼能夠重復到有.
但是我覺得能夠說的就是.
就做這些.
接著做這些事.
你就會深刻地就能夠.
就能夠理解到.

$^{1921}$但這種理解不是說.
一些機械性的.
你做幾次就有.
只能夠你去不斷重復去摸索出來.
因為我覺得其實.
因為在《牧者會》裡面我們也談過這個課題.
真的發覺.
怎麼叫等於沒有盼望呢.
這個真的是不容易的課題.
但是我覺得起碼可以試一下做這些事.
試一下對於將來有想象計劃去行線去討告.
這些我覺得是可以試一下來到去嘗試去.
透過重復去好像肌肉記憶那樣.
令到我們能夠可以慢慢掌握到背後的道理.
其實重點就是背後那種的.
叮一聲那一下.
我覺得會是.
當然不一定是性靈.
但是如果是實踐出來的話.
我覺得又不是純粹那麼一無縹緲的.
應該是有些事可以試一下來到去嘗試去做的.
我想問一下你現在會不會還收街邊的單張呢.
都有的.
有些都不會收的.
如果你派的是呼音單張.
那些人會不會收呢.
我自己試過派單張的.
派單張的人會看一看單張的後面有沒有一些虛線.
有虛線就是證明有QPON就可能會收的.
如果是一張廣告的東西他通常都不會收的.
我一個例子.
其實都和我有一次在一間教會有一段時間參與的時候.
一個對我自己很大的提醒.
那間教會每個主日都會.
他是屋村教會來的.
每個主日他都會派呼音單張的.
但是屋村來來去去都是那些人的.
都沒有什麼生客的.
但是沒有人會收的.
但是他都繼續每個禮拜都會派單張的.

$^{1961}$不是有新朋友進來.
但是他們堅持一件事就是.
仍然告訴人家這裡還有一間教會.
這個是一個很重要的信息來的.
就是應用剛才John說的那樣東西.
做了就沒有那一刻不做好像都不是很覺.
但是有些人堅持做的時候就告訴人家.
那裡仍然是有教會.
那間教會是有開門的.
不派單張那天就是教會不開門.
只要教會開門都有人派單張.
這個就是很重要的信息.
那樣東西的存在對我們來說.
沒有加上沒有減少.
但是那樣東西繼續做就讓人有個盼望.
有個存在的意思.
我都說一下上年那套劇.
我們和Jan談最簡單的那種想法.
就是整套劇就是想去說這件事.
就是怎樣叫有盼望.
發覺是很難.
但是發覺說絕望是怎樣.
當然是容易一點.
絕望大概就是這些東西.
所以我就說只能夠反過來說.
既然絕望是這些東西的話.
那你不要做絕望的那些東西.
那你就嘗試去領悟盼望是什麼.
所以我覺得領悟盼望是一種領悟.
或者是一種從神而來給你的一些恩典.
或者是信心.
但是這些很普普票票的.
叫你信叫你禱告.
相信耶穌都會回來.
但是我們不要做那些絕望的東西.
嘗試做一些絕望相反的東西.
嘗試去希望能夠丁一聲明白什麼是盼望.
或者從神而來得到什麼是盼望.
所以這是金螃的pose.
強調好像很奇怪.

$^{2001}$什麼叫實踐什麼叫練習.
因為真的你仍然是可以沒有什麼盼望的.
你對於你會去旅行.
你會去有計劃看Mirror.
你會仍然生活.
但是不代表你有盼望.
因為你仍然可能對於這個世界.
對一般人來說不是你有期盼.
你是對於將來有一些的想法.
因為你是人來的.
但是這樣是不是有盼望呢.
我覺得真的不容易去找得出來.
但是我覺得如果真的要說到底的話.
總之你不夠就是.
總之你發現自己有絕望的時候就不要去做.
嘗試去尋找上帝的盼望.
其實我最後想說一件事.
原來盼望是一種領悟.
我們做夢研究所原來都是.
發現就是想人去領悟到什麼叫做盼望.
但是這個說不出來給你聽.
你不能夠跟著那個餐單.
跟著什麼方法什麼去做找到盼望.
所以就是這樣去尋找出來.
網上就問到當禱告的時候.
好像以讀不回.
有什麼方法可以繼續維持盼望呢.
我自己的看法就是用剛才的信息.
一個基督徒戴著兩只手錶.
你自己祈禱可能是按你自己的時間表去祈禱.
或者是表達你的意願.
帶上帝有上帝的時間表去工作.
這件事其實都不是新的東西.
你看回舊約的先知.
很多時候他都去做上帝要他做的工作.
但是他說的審判.
他說的上帝的追討.
甚至一些上帝的工作的時間.
都不是他說了就立刻做.
上帝都有他的主權和他做事的方法.

$^{2041}$有時甚至說完之後都不一定會做.
因為那些人會轉了.
所以我覺得那個過程當中.
盼望不是建立了剛才一直說的我們做了什麼.
盼望是說我們知不知道自己在做什麼.
這個是很重要.
我們Folk Church是希望弟兄姊妹看到我們.
帶領整個群體在做什麼.
而我們在做什麼都是教會存在的重要地方.
讓人得聞福音.
讓人認識上帝.
讓人感受到生命中有愛.
讓群體當中有社會參與.
這些都是我們覺得我們不斷存在.
在做一些事.
但是這個不是說你討告.
好像堅持很久都沒有果效.
反而就好像沒有盼望.
我覺得這仍然是看你自己一隻手標.
就看不到另一隻手標的東西.
不知道有沒有其他問題.
後面.
我有點生活上的分享.
不知道可不可以回應今天.
我覺得得著的點.
很體會你所說的那種行善.
是基於我在神里領受到愛.
去感受到這個世界有愛的時候.
覺得不貪大步只是小步.
看到小步自己可以做.
而帶出神的愛.
和人之間的愛在流動.
我就覺得這是很有意義的事情.
而在這個事情里.
我是不是一個想象還是計劃.
其實有時候不是想得太多.
但是我又在想.
例如我很喜歡看到的圖片.
我明明看到一個未信的親友或者朋友.
在臨終還未信.

$^{2081}$我可以做些什麼呢.
就去探望他.
在他身邊嘗試過.
他未必明白.
但是我就是帶著這個祝福.
我不是真的去想.
那個臨終的人將來會是怎樣.
我當然希望他相信.
但是那一刻他真的不相信.
我也要尊重他的選擇.
但是我去做這件事.
我也在感染身邊看著我們.
陪伴他的人.
他經歷神依然愛他.
我就去感染這種愛.
我想說的.
你想說的那種.
我沒有特別好計劃.
但是我就看到一個想象.
我就是將神期望的愛.
新天地有愛的新天地.
在地上要活出來.
要讓身邊人都感染到.
沒有人被捨棄.
那種對他的不離不棄.
這些是不是你所說的那種想象.
但我不是很實際地計劃.
我覺得做得到就去做.
我只是去按自己的那個微小的行動.
去嘗試讓身邊的人多一份愛的流動.
是不是你所說的那種.
我就有些生活上的分享.
還有剛才提到的土告.
我也試過土告落空.
是不是就沒有盼望.
我在想一個小朋友和爸爸聊天.
爸爸你為什麼不給我.
我想要那樣東西.
我想要你為什麼不給我.
可能是哭著說.

$^{2121}$但同時也會說.
你給我吧 你給我吧.
是不是我叫他給我他就會給我呢.
有時候我自己也要學習一個信服的功課.
就是天父爸爸會知道什麼適合我.
或者我們暫時要等什麼.
可能不是等一個很美的天地.
而是等我一個很美的素質.
去有一個更堅穩的信心.
相信神會帶我有更美的地方.
是在他心裡面.
不是在地裡面.
不是地上.
因為心裡面是天價的.
這些就是我所說的盼望嗎.
所以我覺得任何事情都可以.
我們很適合這個月的主題.
一起去做一些事.
任何事你肯做.
就已經是盼望的開始.
我近來對香港教會沒什麼盼望.
不想做.
不想再理不想再碰.
因為我近來在預備這些課.
我都沒什麼盼望對香港教會.
對於那件事.
沒盼望或者絕望只是無動於衷.
你無動於衷.
他影響不到你.
你又沒什麼興趣對他.
所以我覺得.
但你仍然去做一些事的時候.
那件事不在乎你計算得到是否成功.
或者是否行得通.
但你仍然肯去做的時候.
這正正就是一種盼望的開始.
所以我覺得不妨想想.
你現在的生命裡面.
有什麼仍然有心頭的事.
很想去做的事.

$^{2161}$那就一起去做.
一起結黨去做.
這正正是我們的盼望.
很重要的實踐.
所以我覺得我們的盼望是一些行動.
不是純粹等最後來.
不知道你發不發覺周邊的弟兄姐妹.
或者你自己都有些初老的症狀.
什麼叫初老的症狀呢.
你聽John說他自己曾經說.
他自己說有中年危機.
我那次聽他說的時候.
我比你大.
我覺得我做中年應該比他快.
初老或者中年危機其中一些症狀就是.
經常想舊事.
舊事衝了.
還想什麼呢.
重點又回到.
可能對自己做的事沒有什麼觸動.
沒有什麼想象.
甚至很多事都會覺得已經過了有效日期定格.
沒有什麼動力繼續做.
可能籠統地說.
這些就好像沒有了盼望.
所以如果你的歲數還沒有到中年.
就不應該有初老的跡象.
但可能環境或者周邊的聲音.
令你可能不想再想象.
我自己通常覺得.
你試一下去近不同的人.
看多一點.
很多時候你突破了平時接觸的人群體.
看多一點.
看闊一點的時候.
你會發覺你還有很多想象空間.
如果說實務一點的做法.
我自己的經歷就是.
我自己情緒覺得好像.
沒有什麼空間或者很困逼的時候.

$^{2201}$我很多時候都會約很多人.
每個星期都約不同的人見面.
其實我很喜歡聽故事.
或者我看人訪.
看見證的時候就發覺.
其實這個世界充斥了很多人不同的經歷.
其實你看不同人經歷的時候.
你就會見到很多人在自己的環境當中.
做回他自己的事.
其實你就會見到每個人都有自己的盼望.
可以繼續到.
不是要比較.
但你會發覺.
其實你有沒有看到你自己可以.
發展的空間或者做到的事呢.
這個就是由你而來的盼望.
我希望你不要覺得很遠.
但你會發覺看多一點.
看闊一點的時候.
你就會覺得.
是哦,其實我也可以的.
我們有時候都會分享喵喵的那些片子.
你會見到喵喵很多人訪.
哇,這麼有活力.
我們基督徒都沒有這麼有活力.
你會發覺故事會成為激勵.
我們的見證其實.
不一定很成功才能說出來.
失敗見證其實都可以.
成為一個很重要的能力來源.
說一個例子.
我暑假坐飛機的時候.
看一部電影.
Tom Hanks那部電影.
他很萌室的老伯伯.
我的名字是Auto.
這部電影其實很有趣.
他想自殺.
然後一個鄰居叫他幫忙.
不知道做什麼.

$^{2241}$正正就是這件事.
他想自殺.
不過他要幫人家弄那部車.
還是什麼的.
他想做完才自殺的意思.
這種做一些事.
這正正就是對將來的計劃或參與.
我都要先做完這件事.
我都起碼先做了.
做的時候不是很大件事.
不是很大件事.
但是你的生命裡面.
就正正是有些事你要完成.
有些事你要在那些事做完.
這件事就讓我們生命裡面.
仍然有一種對於將來.
仍然有一些事你要做.
你要實踐.
你要去完成.
你要去計劃.
其實人生就是這樣.
你不斷去讓自己.
能夠在將來裡面.
仍然有一些參與.
有一些想象.
那條命就這樣救回來了.
很多次幫人家弄東西.
弄一下燈.
弄一下東西.
我仍然.
原來我對將來.
仍然是有一些參與.
這個就是生命裡面的盼望.
用John這個劇目的例子.
每次Otto他自己想自殺的時候.
有些人找他做事就死不去.
但是他又聯想起.
他太太其實.
旁邊有時候碎碎念.
其實都是欣賞他這些事.

$^{2281}$做事的地方.
這個就是他的人設.
可以有這個空間.
可以繼續做些事.
我們很多時候.
沒有欣賞自己.
或者欣賞旁邊的人.
其實大家每人做一點點事.
都是一個很實際.
給對方一個盼望.
或者存在的意義.
你好.
我都覺得大家說得很好.
我想問一個問題.
盼望其實都需要長期的練習.
這個一定是一個可實踐的盼望.
因為我們有永生.
我們有時候經常與神和好.
經常回到主裡面.
就是一個很簡單的盼望.
不過因為人.
都有些罪性和貪婪.
我們可不可以利用他們來實踐這種盼望呢.
這個貪婪如何去運用一個好處.
我這樣想.
人都是想進到天國的時候.
神就讓他管很多座城池.
就不只是做一個普通的藝文.
你僅僅是能夠得救.
你在外面哭泣像一千年之後才回來.
你之前在地上僅僅是能夠得救.
你又沒有實踐盼望.
有時候我沒有應允你的禱告.
你就走了去外世界才回來.
我就想了一個方法.
不知道你們覺得可不可行.
就是當我沒有盼望很失望的時候.
我就和神說.
我這次一定要實踐盼望.
繼續去傳福音.

$^{2321}$雖然可能傳了三十年.
親戚暫時沒有興趣.
但是我要繼續.
因為我每一次去做.
每一次禱告去傳的時候.
我做了這件事.
神已經給了信用.
到時我走了的時候.
離開了這個世界.
我都有想去的.
很開心.
我繼續去實踐盼望.
但是這裡有點困難.
其實你只是想要將來的相似.
但是我覺得這個推動力.
有多符合人性呢.
去推使自己.
即使在一個很絕望的環境當中.
都去工作.
就好像上班有錢.
不打工就沒有錢.
沒有收入.
我在我很失望的時候.
這樣去處處逢生.
請問兩位覺得.
會不會有些偏.
還是你覺得都挺好的.
不如明天我們一起開始吧.
我想說.
我一向都不是看將來的相似.
我不太相信相似的東西.
因為我覺得不是.
我現在做多少有多少相似吧.
我不太理解經文裡面的相似.
是這樣相對的掌握.
所以我覺得.
今天我們做一些事情.
不是為了相似.
而是相反的.
應該是因為我們有上帝.

$^{2361}$給我們的相似恩恩典.
所以才去做.
所以我覺得.
我們應該先打破這件事.
不是一種很相似的看法.
我們的行為應該是因為.
耶穌基督的恩典.
我們去回應.
多於我們回應了之後.
做了之後就相似.
我不覺得將來牧師傳導人.
會多相似一些.
只不過我們有些人是.
願意去回應上帝的恩典多一些.
我們就去做.
都是一些無常的事情.
都是一些自己甘心情願去做的事情.
多於一些爭取相似.
雖然這個說法是一些.
以前的教會經常說的.
相似的那些.
我們找次講一下這個話題.
或者相似這個問題.
這個反而要想一想.
不關相似的事.
因為我們很軟弱.
經常靠著主的恩典.
那些恩典.
有時低谷去到最低.
那些人很功利.
特別是香港地.
我就在想.
香港有些某些功利的人.
可不可以用這個方法.
來激勵他們.
當他們很絕望的時候.
就跟他們說這個.
但其實最後上次都是神判斷的.
我這個立場只是告訴你.
可能會好一點.

$^{2401}$是不是都不是那麼好.
不是那麼好.
因為hope against hope.
這個盼望正正不是那些東西.
不是那些好處.
或者眼見的好處.
上次都是眼見的好處.
所以我們信仰所說的盼望.
應該不是那些.
不需要這樣去想.
都應該有的.
我本來還想來到這裡.
今天來到這裡.
我沒有什麼盼望.
來到這裡我開心.
我又多一點盼望.
神都很開心.
你來教會都是一件好事.
你繼續努力.
結果是怎麼樣.
我覺得很開心.
因為經過暑假之後.
大家都很絕望.
因為你看到天都下雨.
下到十號風球.
哇 所有東西都倒下.
今天風平浪靜.
神祝福第六科.
因為這是一種可實踐的盼望.
我比較少用這個方式去表達信仰.
因為這樣說很容易被人套路.
人家覺得會有功有利.
這個盼望.
這個邏輯.
其實信仰從來都不是用功利的方式去說.
所以你現在做多少.
我認同.
我都不是阿John的方法.
現在不是和一百人傳福音.
將來會比你和一個人傳福音.

$^{2441}$獎賞會多一點.
因為整件事.
我們接觸的人都不同.
你怎麼比較.
所以我仍然覺得信仰是一個很好的方法.
因為信仰是一個很好的方法.
我仍然覺得信仰的表達和傳揚給弟兄姊妹的過程當中.
都不是用你功與利的.
好像做生意的思維.
這樣逼的.
這樣逼的過程當中.
就很容易被人拿捏了.
就好像以前天主教那些.
就是你做不夠你可以買來補償.
這個補償的過程當中.
就會令到人會給下去.
其實信仰從來都不是我們做多少去賺回來的.
這個以弗所說講得很清楚.
得救是本福音.
借著信.
整件事都不是我們的工作.
做了多少東西出來.
而是上帝的恩典.
所以整件事.
這個思路就不要再想下去了.
不可行的.
我想問一個可能.
敏感一點的範疇.
這裡是講.
流散.
愚民的一個.
過程.
去到一個盼望的位置.
我就會想起.
因為很多人在走與流之間.
有很多掙扎.
很多想法.
而在很多這些掙扎.
想法裡面都牽涉盼望.
你對這個地方.

$^{2481}$還有沒有希望呢.
你對於這個地方還有沒有想象.
你想象這個地方會.
繼續更差 更糟.
那你可能會選擇離開.
或者怎麼樣.
那怎樣在.
當中去看.
無論是在.
對於信仰上面.
的一些盼望.
或者對於在地上面.
可能現在的實況.
對於地上的一些.
無論是人士還是權勢.
的一些沒有盼望.
或者還有沒有盼望的一些.
問題.
怎樣可以在我們信仰裡面.
或者實踐裡面.
去影響我們.
對於走或者流.
的一些不同的思考.
關於這方面的一些.
可以再分享多一點.
正正是等這些問題問.
是很需要問.
因為我也知道會有人問.
所以我也沒說到.
這個主題.
本身也想說.
有關移民的一個主題.
特別是流散的第二季.
我覺得移民本身.
是一件盼望.
有盼望的行動.
我想說的定義是什麼.
當你願意去計劃.
計劃移民是一個最大計劃.
是將來希望來的.

$^{2521}$去準備.
不是叫大家移民.
而是我們.
對於將來.
仍然有想象.
仍然有一種.
的嘗試.
去改變行動.
或者你去計劃.
這個就是移民.
其中一個可以實踐.
所以沒錯的.
移民的人可能對香港沒什麼盼望.
但沒問題的.
對香港是有盼望的.
但對我們對生命是有盼望的.
對世界是有盼望的.
所以我覺得.
移民不是一個有問題的.
也不是說你對香港沒盼望就不好.
這樣.
事實上你問我.
我對香港沒什麼盼望.
按照字面來看.
只不過因為我對生命有盼望.
所以我覺得我香港住.
我也是沒問題.
在香港住我也是有盼望.
我女兒也是有盼望.
對她來說.
所以我才這樣理解.
所以我覺得一個人有盼望.
可以選擇移民.
也可以選擇不移民.
兩個是沒關係的.
所以我都說.
移民只是一個嘗試去籌劃.
去計劃將來的事情.
所以你會這麼做.
當然對於.

$^{2561}$看回這一集的移民.
你也要問.
究竟我移民.
是純屬出於恐懼.
是不是.
出於絕望.
還是我嘗試去做一些.
新的事情去改變.
我生命的東西.
我想這是一個關鍵的問題.
純屬對於生命的絕望.
所以我離開.
還是我嘗試去尋找一個新的方向.
生日生活.
這個也提過了.
所以你嘗試去做一些新的事情.
也是一個盼望的表現.
所以我覺得對於移民.
對你來說.
盼望的課題.
正正就是在一個新環境里.
如何能夠去開展.
無論是為你的小朋友.
或是為你自己也好.
嘗試去.
不是等著再來.
而是可以在新的方向裡面去實踐信仰.
去尋找.
耶穌基督的豐盛.
不過是說.
耶穌的豐盛不等於移民.
但是移民是可以在一種實踐.
尋找上帝豐盛的一個方法.
所以我覺得.
兩件事是沒有衝突的.
我自己就.
看.
這個.
問題的過程當中.
回到最簡單的.

$^{2601}$就是其實我自己覺得.
每個選擇.
都是一個盼望.
你.
不要說移民這麼大步.
就算你去吃飯.
你選A餐和B餐.
你選了A餐也要當A餐有盼望.
你應該盼望A餐比B餐好吃.
你才選A餐.
所以選擇本身就是一個盼望.
我一直以來的想法.
移民這個套路.
或者這個場景是複雜了.
因為牽涉到你身份的重建.
所以你就會.
好像壓著很多東西.
認同John的做法.
和我們.
有些東西一直都在見證.
就是選擇在一個.
新的環境生活的弟兄姊妹.
其實是對.
他的選擇是有盼望的.
所以這個也是一個.
在過程當中建立.
但是.
難搞就是那種情意結.
那種情意結就牽涉到.
身份,複雜的地方就是身份.
大家好像身份不同了.
又或者好像有些東西.
情感上有些分離的時候.
就令到那件事.
感覺不好.
所以好像趕身.
移民都好像很.
敏感的話題.
但是我想你都聽過.
有些人移民之後.

$^{2641}$其實他不代表.
好過香港生活.
然後就說他自己選的.
所以過程當中.
要處理的過程當中.
不是說有沒有盼望.
這個話題那麼簡單.
所以我自己覺得那件事.
有沒有盼望或者那件事.
怎麼看盼望呢?回到你的身份.
是做什麼.
我自己的盼望仍然是.
通常都用基督徒的身份.
來處理的,就是明天會更好.
我自己覺得明天不會好.
因為這個世界會越來越差.
耶穌說過回來之前.
有很多問題,但是明天會更好.
明天會更好.
因為明天耶穌會回來.
這個好像很吊詭的方式就是.
我的身份做了決定.
我仍然沒好沒好一天.
我每一天期盼耶穌會回來.
這個是我的盼望,但是我不會盼望這個世界好.
因為世界本身就不會好.
所以移民也好.
或者做決定也好.
你去到那個地方做新的身份的時候.
就在那個新的身份當中.
做到盼望.
生活,一個基督徒原則.
做好本份就夠了.
既然你去到新的身份.
又何苦要想回舊的身份呢?.
想回舊身份的時候.
你一定很多時候會不當初.
這個就是.
手扶著泥巴往後看的人.
就是一個問題.

$^{2681}$我希望大家明白我的觀點.
有沒有問題?.
最後一個.
沒有就班機開就等下個月了.
下個月我們就會說什麼?.
說上班.
這個也是我們第二季裡面.
其中一個很實用的題目.
香港人很喜歡上班.
我們下個月再見.
拜拜.
\newpage



\section{民數記 27:1-11-20230930}
\label{sec:JxHW7ujVbSI}
\textbf{【網上崇拜】Shall We Talk?|民數記27\_1-11|20230930 [JxHW7ujVbSI]}
\newline
\newline
連結: \href{https://youtube.com/watch?v=JxHW7ujVbSI}{\texttt{ https://youtube.com/watch?v=JxHW7ujVbSI}} ~~~~ 語音日期: 2023-09-30 
\newline
\newline
\hyperref[sec:2QyWxsVtL8E]{\small{< < < PREV SERMON < < <}}
~
\hyperref[sec:index_chronic]{\small{[返順時目]}}
~
\hyperref[sec:index_scriptual]{\small{[返順卷目]}}
~
\hyperref[sec:3_8UYgGhJ0E]{\small{> > > NEXT SERMON > > >}}
\newline
\newline
民數記 27:1-11-20230930
\newline
\begin{longtable}{cl}
\hline
\hline
章節 & 經文 (和合本修訂版)\\
\hline
27:1 & \begin{tabularx}{0.7\textwidth}{X} 約瑟的兒子瑪拿西的宗族中,有瑪拿西的玄孫,瑪吉的曾孫,基列的孫子,希弗的兒子西羅非哈的女兒,名叫瑪拉、挪阿、曷拉、密迦、得撒。她們前來, \end{tabularx} \\ \\ \relax
27:2 & \begin{tabularx}{0.7\textwidth}{X} 站在會幕門口,在摩西和以利亞撒祭司,以及眾領袖與全會眾面前,說: \end{tabularx} \\ \\ \relax
27:3 & \begin{tabularx}{0.7\textwidth}{X} 「我們的父親死在曠野。他沒有與可拉同夥聚集攻擊耶和華,是在自己的罪中死的;他沒有兒子。 \end{tabularx} \\ \\ \relax
27:4 & \begin{tabularx}{0.7\textwidth}{X} 為甚麼因我們的父親沒有兒子就把他的名從他族中除掉呢?求你們在我們父親的兄弟中分給我們產業。」 \end{tabularx} \\ \\ \relax
27:5 & \begin{tabularx}{0.7\textwidth}{X} 於是,摩西將她們的案件呈到耶和華面前。 \end{tabularx} \\ \\ \relax
27:6 & \begin{tabularx}{0.7\textwidth}{X} 耶和華對摩西說: \end{tabularx} \\ \\ \relax
27:7 & \begin{tabularx}{0.7\textwidth}{X} 「西羅非哈的女兒說得有理。你定要在她們父親的兄弟中,把地分給她們為業,把她們父親的產業傳給她們。 \end{tabularx} \\ \\ \relax
27:8 & \begin{tabularx}{0.7\textwidth}{X} 你也要吩咐以色列人說:『人死了,若沒有兒子,就要把他的產業傳給他的女兒。 \end{tabularx} \\ \\ \relax
27:9 & \begin{tabularx}{0.7\textwidth}{X} 他若沒有女兒,就要把他的產業給他的兄弟。 \end{tabularx} \\ \\ \relax
27:10 & \begin{tabularx}{0.7\textwidth}{X} 他若沒有兄弟,就要把他的產業給他父親的兄弟。 \end{tabularx} \\ \\ \relax
27:11 & \begin{tabularx}{0.7\textwidth}{X} 他父親若沒有兄弟,就要把他的產業給他族中最近的親屬繼承為業。』」這要作以色列人的律例典章,是照耶和華所吩咐摩西的。 \end{tabularx} \\ \\ \relax
27:12 & \begin{tabularx}{0.7\textwidth}{X} 耶和華對摩西說:「你上這亞巴琳山脈,看我所賜給以色列人的地。 \end{tabularx} \\ \\ \relax
27:13 & \begin{tabularx}{0.7\textwidth}{X} 看了以後,你也必歸到你祖先那裡,像你哥哥亞倫歸去一樣。 \end{tabularx} \\ \\ \relax
27:14 & \begin{tabularx}{0.7\textwidth}{X} 因為你們在尋的曠野,當會眾爭鬧的時候,違背了我的命令,在取水之事上沒有在會眾眼前尊我為聖。」這水就是尋的曠野中,加低斯的米利巴水。 \end{tabularx} \\ \\ \relax
27:15 & \begin{tabularx}{0.7\textwidth}{X} 摩西對耶和華說: \end{tabularx} \\ \\ \relax
27:16 & \begin{tabularx}{0.7\textwidth}{X} 「願耶和華,賜萬人氣息的神,立一個人治理會眾, \end{tabularx} \\ \\ \relax
27:17 & \begin{tabularx}{0.7\textwidth}{X} 可以在他們面前出入,引導他們進出,免得耶和華的會眾如同沒有牧人的羊群一般。」 \end{tabularx} \\ \\ \relax
27:18 & \begin{tabularx}{0.7\textwidth}{X} 耶和華對摩西說:「嫩的兒子約書亞是一個有聖靈的人;你要領他來,為他按手, \end{tabularx} \\ \\ \relax
27:19 & \begin{tabularx}{0.7\textwidth}{X} 使他站在以利亞撒祭司和全會眾面前,在他們眼前委派他, \end{tabularx} \\ \\ \relax
27:20 & \begin{tabularx}{0.7\textwidth}{X} 又將你的尊榮給他一些,好使以色列全會眾都聽從他。 \end{tabularx} \\ \\ \relax
27:21 & \begin{tabularx}{0.7\textwidth}{X} 他要站在以利亞撒祭司面前;以利亞撒要憑烏陵的判斷,在耶和華面前為他求問。他和以色列全會眾都要照以利亞撒的指示出入。」 \end{tabularx} \\ \\ \relax
27:22 & \begin{tabularx}{0.7\textwidth}{X} 於是摩西照耶和華所吩咐他的,將約書亞領來,使他站在以利亞撒祭司和全會眾面前, \end{tabularx} \\ \\ \relax
27:23 & \begin{tabularx}{0.7\textwidth}{X} 為他按手,委派他,是照耶和華藉摩西所說的。 \end{tabularx} \\ \\
[1ex]
\hline
\hline
\end{longtable}
$^{1}$.
屬約西的兒子馬拿西的國族.
有馬拿西的元孫.
瑪吉的曾孫.
基烈的孫子.
希伐的兒子西諾菲亞的女兒.
名叫馬拉羅亞.
學拉密加德薩.
他們前來站在會幕門口.
在摩西和祭司伊利亞薩.
並眾首領與全會眾面前說.
我們的父親死在曠野.
他不可與可拉同黨聚集攻擊耶和華.
是在自己最終死的.
他也沒有兒子.
為什麼因我們的父親沒有兒子.
就把他的名從他族中除掉呢.
求你們在我父親的弟兄中分級我們產業.
於是摩西將他們的案件呈到耶和華面前.
耶和華曉語摩西說.
西羅菲亞的女兒說得有理.
你們要在他們父親的兄弟中.
把地分級他們為業.
要將他們父親的產業歸給他們.
你也要曉語以色列人說.
人若死了沒有兒子.
就要把他的產業歸給他的女兒.
他若沒有女兒.
就要把他的產業給他的弟兄.
他若沒有弟兄.
就要把他的產業給父親的弟兄.
他父親若沒有弟兄.
就要把他的產業給他族中最近的親屬.
他便要得為業.
這要作以色列人的律例.
典章是照耶和華所吩咐摩西的.
各位電影姐妹晚安.
當聽到粵語題叫「上妝」的時候.
我立即想起大學很多很好的回憶.
John在第一講說「上妝」是甚麼呢.

$^{41}$他說「上妝」是一群人聚在一起.
可能花120\%的努力.
做一些未必很有意義的事.
我不知道是否跟大學的上妝文化不同.
我自己覺得自己上妝很有意義.
我是中大某書院劇社的妝員.
每年都會參加四院戲劇比賽.
中大還是四個書院的年代.
簡直是暴露年齡系列.
John曾經在群組說.
我們要給大學時期上妝的相片.
我昨天找到了.
很快的.
我是劇社的幕後.
另一個是編劇.
我們正在排戲.
當年的長康還是用文字藝術師.
你就知道那個年代是多麼久遠.
除了參加四院比賽之外.
我們還會參加戲劇會演.
不過我們不是一群人.
做台戲,圍圍圍.
而是我們很想透過劇場.
透過舞台去表達一些想法.
曾經寫過一些劇本.
希望可以為弱勢社群發聲.
希望他們被看見.
曾經做了一齣劇.
想反映制度中某些人受到的壓迫.
人可以如何反抗.
等等.
我們經常做這些.
我想劇社當中.
不多不少都造就了我今天研究的興趣.
或者是我對某些事的取態或實踐.
我相信這些不是讀書讀到的.
正如我所說.
劇社佔了我大學生活很大部分的時間.
排戲有時排到半夜.
我會睡在掃房.

$^{81}$試過和一班莊園去看話劇.
看見別人的設備很適合我們即將比賽的那套戲.
一班莊園就在想.
不如我們寫信去劇團.
問可否在節目完結後借給我們用.
然後一班人一起寫信.
還說笑.
A級實用文寫作終於可以在現實中用得著.
結果劇團完結後.
整套設備就給了我們.
然後完結比賽後.
書院給了我們一個位置去展示這個設備.
我們當然有加工.
一班人聚在一起.
可以有很多瘋狂的時光.
我印象最深的是.
我三年級那年去了交換學校.
本來會錯過中大.
我們三月就開始拿譜拍畢業照.
但我一班莊園合資買了機票給我回來.
和他們有一個相機約會.
原來我讀大學時不介意飛來飛去.
有人付錢給我.
我就可以到處飛.
這些當然是很美好的經驗.
但同時我也看到.
劇社也改變了我們莊園的就業取向.
有些做了電視編劇.
有些繼續進修.
然後成為了駐校的戲劇老師.
或者畢業後從事劇團管理的工作.
所以我想說.
一班人聚在一起.
是可以帶來一些改變的.
我想以上這些都促使我選擇今天講道這段經文.
今天講道的經文是來自《文素記》.
在《希伯來聖經》中.
書卷的標題是來自《文素記》中第五個字.
Bamibba.
意思是一班人在曠野.

$^{121}$對於《希伯來聖經》的人來說.
當然是一個很合適的命題.
因為《文素記》大部分的敘述.
都是記載著以色列人如何在曠野度過.
這段歷史可以分為他們在西奈曠野.
巴蘭曠野和摩亞平原的三個部分.
而我們熟悉的《文素記》.
其實是出自舊約希伯.
希列文版本的標題翻譯.
74日的日本譯者為何選擇這個標題呢.
是因為摩西在書卷的一開始.
第一至第四章.
以及在下半部26章的開始.
記錄了兩次以色列人很重要的人口普查的數據.
這是他們在曠野漂流開始和結束的時候.
數點人數得來的數據.
第一次是摩西在以色列人完成了和上帝納約.
領受律法離開西奈山前進之前.
那時去數點.
40年後當他們去到約旦河東岸摩亞時.
再一次有人口普查.
在兩次普查之間.
以色列人向著迦南南部迦底斯巴尼亞進發.
但在過程中我們稍稍認識《文素記》時.
就看到他們不斷想退縮.
面對著一連串的失敗,損益等等.
以致他們那一代人無法進入迦南.
他們於是在那裡留了好幾年.
然後去到約旦河東岸.
有些人選擇留在那裡.
有其餘的人準備渡河進入迦南.
兩次人口普查有什麼特別呢?.
數據有什麼特點呢?.
兩次普查只統計男性.
婦女和兒童沒有被記載.
為什麼不記載他們呢?.
因為當時覺得婦孺不能參戰.
所以覺得這些是沒用的數據.
所以今天我們或許覺得很難.
很期望在《文素記》裡面找到對女性有什麼看法.

$^{161}$不過當我們仔細去看.
《文素記》第26章的人口普查數據裡面.
當提到馬來西亞支派的時候.
我們發現西羅菲哈五個女兒的名字.
然後我們再看第27章的時候.
即是剛才主令為我們所讀的經文裡面.
我們看到他們不單單名字被記載在聖經裡面.
同樣他們在摩西和一眾長老面前所做的.
所說的都詳細記錄下來.
我們要記住一群在當時不被重視的女性.
她們選擇聚在一起去做一些事.
結果不單單有意義.
她們更為一個制度.
為一個未來帶來一些改變.
如果從《文素記》的大綱來看.
西羅菲哈的女兒代表著什麼?.
是代表著一群猶太女性和新一代的人.
她們對上帝給予她們的地方的熱愛.
《文素記》第1至25章其實是說老一輩的人.
在曠野裡面的經歷.
我想說她們起初都雄心壯志.
很積極地為進入迦南地有很多的預備.
但是如果我們往後去看的時候.
我們發現稍後記載的就是.
她們一次又一次地發怨言.
稍稍說一下.
第11章她們因為食物問題發怨言.
她們說覺得在埃及遺留的生活更加好.
第12章則質疑耶和華所選擇的領袖.
第13章我們很熟悉的就是派探子進入迦南.
第14章就是探子回報.
最後第14章是一個更大的埋怨.
而這個埋怨招致耶和華對她們這一輩人.
一個很嚴厲的懲罰.
就是她們說她們不能進入迦南地.
然後再記述她們在曠野漂流40年.
直到第25章為止.
我想說第11至25章所記述的.
都是老一輩的以色列人那份的搏抑.
不斷遭受懲罰.

$^{201}$然後認罪.
然後面對失敗.
不斷地在循環.
從她們的怨言中我們看到.
上一代的以色列人是怎樣的.
對埃及很留戀.
信不過上帝的帶領.
不敢進入這個地方.
甚至說什麼呢.
她說不如我們選一個領袖帶我們回埃及吧.
這是她們的心聲.
但是到了第26章至36章.
文素記的後半部.
記述的是曠野出生的新一代.
她們是怎樣的.
她們是努力地預備.
去承受這個地圖.
在第27章我們看到的是.
西羅非哈的女兒是怎樣的.
她們真的很愛這個地方.
所以才這麼著緊.
要向一眾領袖去爭取土地的繼承權.
她們明白.
她們現在看到的.
她眼前的律法是說什麼.
只有男人 只有兒子.
才可以繼承土地.
而她們的處境就是.
西羅非哈沒有兒子.
所以意味著什麼.
意味著這個家族的名字.
很快就會在土地上消失.
所以當西羅非哈的女兒.
她面對著爸爸沒有兒子的局面的時候.
她們作為家裡唯一可以發聲的人.
她們是怎樣的.
她們沒有認命.
她們沒有妥協.
沒有表現出一副無能為力的樣子.
而是怎樣的.

$^{241}$是顛覆當時代的想像.
她們自己的說話權就自己爭取.
她們向領袖提出澄清.
希望可以改變.
這五個女兒就怎樣.
聚在一起.
然後鼓起勇氣.
就好像經文所說.
她們就是站在會幕門口.
在摩西和以利亞薩祭司.
以及眾領袖與全會眾面前.
去說出她們的請求.
如果用今天的說話來說.
她們就像是去到法院提出訴訟.
她們去表達自己的立場和法理依據.
為自己,為父家去發聲.
她們五個聚在一起.
去到會幕和領袖去談這件重要的事情.
對她們來說.
繼承土地在家裡是很重要的事情.
我們要知道.
當時的女性.
不被當為一個人.
不應該發聲.
所以她們能夠向領袖提出請求.
是一個不尋常和大膽的舉動.
說的時候.
都不知道會不會成功.
但就是先說了.
當這五個女人.
去到一個她們不應該去的地方的時候.
你猜當時聚在會幕的男人會有什麼反應呢?.
這些只是一些插畫.
未必能夠完全表達當時的景象.
但我估計當時的領袖,長老.
或者傳媒眾.
看到五個女人在會幕出現的時候.
會覺得很驚訝等等.
看她們究竟想表達什麼.
按理來說.

$^{281}$她們不應該在這裡.
根本沒有權和領袖去發聲.
或者是一陣沉默過後.
終於有人發聲了.
有人可能戰戰兢兢地說.
他說:我們的父親死在曠野.
他沒有和可拉同黨聚集攻擊耶和華.
是在自己的罪中死的.
他沒有遺旨.
在說什麼呢?.
在說他們的父親由始至終.
都沒有離棄耶和華.
所以怎樣呢?.
父親在這片土地上.
理應擁有合法的繼承權.
但他們擔心父親因為沒有兒子的條件.
最後被摒棄在外.
所以當領袖聽到他們最後那句.
他沒有遺旨的時候.
懂得聽的就聽得到.
就知道他們的訴求是什麼.
家裡沒有男人.
就由我們幾個女人去繼承.
本來屬於父親的土地.
我想說在我們現在的背景下.
我們可能覺得不是很震驚.
但在他們那個時代.
沒有任何一個女人可以繼承土地.
因為在上帝所頒佈的律法.
繼承權往往就是由父親傳給兒子.
所以當領袖聽到的時候.
可能第一個反應就是.
這群女人憑什麼在應許之地.
可以分一杯羹.
祭司和其他長老可能心裡嘰哩咕嚕地想.
這群女人憑什麼來到會幕.
去挑戰上帝所頒佈的律法呢?.
再過一會兒.
接著他們再提出他們的質疑.
他們說為什麼因為我們父親沒有兒子.

$^{321}$就將他的名從族中除掉呢?.
為什麼?.
為什麼可以這樣做?他們問.
再之後.
他們發出他們的請求.
他跟摩西說.
和一眾領袖說.
求你們在我們父親的兄弟裡.
分給我們產業.
他們拒絕接受現實所定下來的規矩.
他們關心自己可不可以得到這份土地.
他們要在上帝的應許裡有份.
他們很努力地做一些事.
期望可以改變他們的處境.
因為對他們來說.
土地不單只是一片土地.
更加是上帝對以色列人的應許.
加南這一片土地.
代表著上主對他們的祝福.
是對阿伯拉罕所立的應許.
終於即將實現.
加南對於他們來說.
是上帝與他們同在的一個豐富.
一個充實的生活.
西羅非哈的女兒.
很深明白以上的角點.
所以他們要怎樣?.
他們要名正言順地在這個土地佔一席位.
他們此致要成為上主的子民.
這五個女子剛剛提出了一個.
當時的人覺得聞所未聞的請求.
或許在議會裡大部分的男人都很想怎樣?.
想叫他們閉嘴.
甚至想攀他們走.
我們看看摩西怎樣處理?.
摩西聽了他們的訴求.
摩西作為領袖.
她沒有因為這個請求沒有先例而拒絕.
我們要再提醒.
女人在當時根本沒有說.

$^{361}$摩西大可以跟他們說離開.
但摩西沒有.
摩西展現了一個屬靈領袖應該有的素質.
她很清楚知道自己是上主和人之間的中介角色.
是人神之間溝通的橋樑.
所以她怎樣?.
面對著這個聽都未聽過的請求.
摩西當然沒有辦法馬上給他們答案.
所以她怎樣?.
她將這個提案原原本本帶到上主面前.
尋求他的裁決.
結果呢?.
結果是上主對西羅菲哈女兒的一個肯定.
上主對他們說了甚麼?.
肯定的是甚麼?.
他們說得有道理.
然後吩咐摩西要在他們父親的兄弟裡.
把土地分給他們.
而因為西羅菲哈女兒這次的申訴.
是否就此停止?.
不是.
上主完善了土地繼承的律法.
他再說如果一個人死了.
沒有兒子就將產業傳給女兒.
沒有女兒就將產業傳給她的兄弟.
如果沒有兄弟就將產業傳給他父親的兄弟.
他父親如果沒有兄弟.
就將產業傳給他族中最近的親屬等等.
完善了整個土地繼承的律法.
上帝的律法因為西羅菲哈的女兒而被改變.
比先前所定的再全面多些.
覆蓋範圍再大一些.
我想說這些改變正正是因為五姐妹一起走到回望.
回望是一個什麼地方?.
是一個沒有邀請他們去.
沒有期望他們去.
甚至是沒有讓他們去的地方.
但是他們為什麼可以這麼有勇氣去?.
我想就是走在一起.
五個人走在一起.

$^{401}$叫他們能夠有勇氣在那裡大膽地表達他們的想法和他們的渴求.
一人計短二人計長.
五個人呢?.
當然想不到.
但是說不到什麼.
但是從他們去看到就是五個人聚在一起是怎樣?.
令到他們的說話是很有智慧和策略的.
大家再想一下西羅菲哈的女兒是怎樣去說.
如果他們在土地的繼承問題上.
糾結在男女要平等的時候會怎樣?.
結果大概就不會是這樣.
大概就會被人攀扯了.
因為那個年代根本就沒有男女平等這回事.
但是我們回想剛才我們所看的經文.
他們所用的策略是什麼?.
他們是怎樣?.
以父親之名西羅菲哈的榮譽去說.
他們在說什麼?.
他們說父親的名字一定要留在這片土地上.
所以我們才想要繼承權.
你明不明白?.
所以他們其實是很有智慧和策略的.
這五個年輕女子走在一起挺身而出.
大膽地向上主所頒布的律例去提出質疑.
質疑一般人覺得不可以質疑的事情.
他們的行動表達什麼?.
他們拒絕接受上帝的律例.
只是一定要遵從的想法.
我想說他們很艱難.
其實他們面對著很多困難和限制.
但是他們五個人聚在一起的發聲.
為我們帶來一些啟發.
我們敢不敢質疑呢?.
對我們的現實處境.
我想說質疑是一件很重要的事情.
我們想想如果我們在實際的處境裡.
當我們想都不想別人叫你怎樣做就怎樣做的時候.
即是那個人跟你說這是上帝的旨意.
你想都不想就照做的時候.
我想說其實可能會有危險.

$^{441}$可能會令我們的信仰留於表面.
想都不想去做不是說你很信服.
而是可能反映了我們在屬靈方面有懶惰.
我想說我們應該在我們的信仰操練的時候.
要去想,要去懷疑.
在想的過程中幫我們可以想到更加明白.
更加可以更加接近合乎上主心意的答案.
甚至是可以讓我們對未知的將來更有把握.
和更加有信心.
我明白在華人社會裡.
當我們就著一些議題提出質疑.
或者提出問題的時候.
往往就被認為是對領導的不尊重,不信任.
但是在文述的27章的經文裡.
我們看到摩西作為以色列人領袖的氣量,氣魄.
他很明白眼前這五個女子所提出的質疑.
對事不是針對她的.
是針對土地繼承的律例.
所以摩西第一個反應不是起降,不是反彈.
而是將他們的質疑原原本本地呈到上主的面前.
等候祂的發落.
從這五個女子的行動裡.
讓我們看到的就是.
最重要的是要說出來.
即使我們自覺自己是一個微不足道的人.
當我們面對著不公平的時候.
都可以透過上帝所揀選的領袖.
而去向上主呼求憐憫.
我想說當時沒有人經歷過西羅非哈加的處境.
沒有人覺得這個案例很特別.
但你想想如果西羅非哈的女兒.
他們不出來做第一批發聲的人.
去爭取話語權.
西羅非哈的家族這個名字很大機會會消失.
然後有領袖願意聆聽.
並且將他們的訴求帶到耶和華面前.
他們的聲音最終被聽見.
整件事是怎樣呢?.
上帝都談了一份.
給了回應,給了肯定.

$^{481}$結果呢?.
結果是有一些原來未完善的制度被改變了.
讓我們看到未來是可以改變的.
即使這個改變很小.
為何這樣說呢?.
我們繼續看文素記的時候.
其實他們沒有顛覆到整個土地繼承律法.
整個繼承權仍然很賦權.
為何這樣說呢?.
36章對這五個女子的婚嫁有限制.
限制她們一定要嫁給同宗的人.
父親之派的人.
為甚麼呢?.
以免土地會從西羅非哈家轉到其他家.
所以其實仍然是維護著父家的名聲.
不過我想說改變始終都有出現.
他們的申訴讓我們看到.
女人可以很名正言順地承受土地違約.
由一族的女性到全以色列.
其實都可以跟領袖和上主討論.
未來是可以討論的.
可以有改變的.
最重要是鼓勵弟妹發聲.
我期望不同的群體可以對話.
並且不只是我們自己聊天.
而是容讓上主也參與其中.
我相信未來是可以改變的.
我為今天的講道選了一首回應詩.
是《彭遜跨過》.
我很期盼留堂的弟兄姊妹.
可以學習西羅非哈女兒的榜樣.
我很希望大家可以聚在一起.
將你們想要做的.
將你們在神國的負擔說出來.
或許在我們現在的處境.
未必很容易.
但我祈求上主給我們一個豁出去的勇氣.
我很心願見到大家.
和你們的牧者和組員.
和教會裡不同的弟兄姊妹聚在一起.

$^{521}$將你們心所盼望的實踐出來.
過程中我相信上主會參與其中.
和我們一起聊一份.
弟兄姊妹無論為我們的人生.
為教會.
或是為香港這片土地.
但願我們可以為他們發聲.
心願我們所做的.
可以為未來帶來更好的改變.
請敬拜對.
\newpage



\section{腓立比書 1:15-18-20231007}
\label{sec:3_8UYgGhJ0E}
\textbf{【網上崇拜】上莊的相反:結黨|腓立比書1\_15-18|20231007 [3\_8UYgGhJ0E]}
\newline
\newline
連結: \href{https://youtube.com/watch?v=3_8UYgGhJ0E}{\texttt{ https://youtube.com/watch?v=3\_8UYgGhJ0E}} ~~~~ 語音日期: 2023-10-07 
\newline
\newline
\hyperref[sec:JxHW7ujVbSI]{\small{< < < PREV SERMON < < <}}
~
\hyperref[sec:index_chronic]{\small{[返順時目]}}
~
\hyperref[sec:index_scriptual]{\small{[返順卷目]}}
~
\hyperref[sec:CYdq0eOXFpM]{\small{> > > NEXT SERMON > > >}}
\newline
\newline
腓立比書 1:15-18-20231007
\newline
\begin{longtable}{cl}
\hline
\hline
章節 & 經文 (和合本修訂版)\\
\hline
1:15 & \begin{tabularx}{0.7\textwidth}{X} 有些人傳基督是出於嫉妒紛爭;有些人是出於好意。 \end{tabularx} \\ \\ \relax
1:16 & \begin{tabularx}{0.7\textwidth}{X} 後者是出於愛心,知道我奉差遣是為福音辯護的。 \end{tabularx} \\ \\ \relax
1:17 & \begin{tabularx}{0.7\textwidth}{X} 前者傳基督是出於自私,動機不純,企圖要加增我捆鎖的苦楚。 \end{tabularx} \\ \\ \relax
1:18 & \begin{tabularx}{0.7\textwidth}{X} 這又何妨呢?或是假意或是真心,無論如何,只要基督被傳開了,為此我就歡喜。我還要歡喜, \end{tabularx} \\ \\ \relax
1:19 & \begin{tabularx}{0.7\textwidth}{X} 因為我知道,這事藉著你們的祈禱和耶穌基督的靈的幫助,終必使我得到釋放。 \end{tabularx} \\ \\ \relax
1:20 & \begin{tabularx}{0.7\textwidth}{X} 這就是我所切慕、所盼望的:沒有一事能使我羞愧;反倒凡事坦然無懼,無論是生是死,總要讓基督在我身上照常顯大。 \end{tabularx} \\ \\ \relax
1:21 & \begin{tabularx}{0.7\textwidth}{X} 因為我活著就是基督,死了就有益處。 \end{tabularx} \\ \\ \relax
1:22 & \begin{tabularx}{0.7\textwidth}{X} 但是,我在肉身活著,若能有工作的成果,我就不知道該挑選甚麼。 \end{tabularx} \\ \\ \relax
1:23 & \begin{tabularx}{0.7\textwidth}{X} 我處在兩難之間:我情願離世與基督同在,因為這是好得無比的; \end{tabularx} \\ \\ \relax
1:24 & \begin{tabularx}{0.7\textwidth}{X} 然而,我為你們肉身活著更加要緊。 \end{tabularx} \\ \\ \relax
1:25 & \begin{tabularx}{0.7\textwidth}{X} 既然我這樣深信,就知道仍要留在世間,且與你們眾人一起存留,使你們在所信的道上又長進又喜樂, \end{tabularx} \\ \\ \relax
1:26 & \begin{tabularx}{0.7\textwidth}{X} 為了我再到你們那裡時,你們在基督耶穌裡的誇耀越發加增。 \end{tabularx} \\ \\ \relax
1:27 & \begin{tabularx}{0.7\textwidth}{X} 最重要的是:你們行事為人要與基督的福音相稱,這樣,無論我來見你們,或不在你們那裡,都可以聽到你們的景況,知道你們同有一個心志,站立得穩,為福音的信仰齊心努力, \end{tabularx} \\ \\ \relax
1:28 & \begin{tabularx}{0.7\textwidth}{X} 絲毫不怕敵人的威脅;以此證明他們會沉淪,你們會得救,這是出於神。 \end{tabularx} \\ \\ \relax
1:29 & \begin{tabularx}{0.7\textwidth}{X} 因為你們蒙恩,不但得以信服基督,而且要為他受苦。 \end{tabularx} \\ \\ \relax
1:30 & \begin{tabularx}{0.7\textwidth}{X} 你們的爭戰,就與你們曾在我身上見過、現在所聽到的是一樣的。 \end{tabularx} \\ \\
[1ex]
\hline
\hline
\end{longtable}
$^{1}$上一次講道的時候,剛剛下了八號風球.
然後我們又經過水災.
明天又不知道會不會打風.
明天應該也會有八號的.
(笑聲).
那就…….
是啊,很掛念大家.
剛才聽大家敬拜的時候.
我覺得很感動.
因為好像一個小朋友的聲音.
突然間覺得是否合唱團來了.
我們一起很開心地來敬拜.
特別今天唱一些舊的詩歌.
大家會更加開心.
我們今天講道.
也知道每次講道的時候.
都要展示自己拍攝的照片給大家看.
是不是?.
那我展示給你們看.
(笑聲).
我以前曾經出過這種照片.
我以前不能叫「上莊」.
因為那時候我不是叫「上莊」.
那時候是「侍奉」.
我是香港大學基督徒團契福音文歌報道小組的.
有沒有?有沒有港大的朋友?.
我們都是一個莊.
那時候我是做師太.
我發覺那時候一上大學就立即留長頭髮.
因為我已經束縛了很久.
我一直都想留長頭髮.
那時候也想吸收一下自由的空氣.
所以就很辛苦地留了長頭髮.
然後發覺長頭髮的朋友都很難在教會生存.
特別後來我做了傳道人.
更加沒有一輪.
我已經有二十年時間沒有留長頭髮了.
我現在為什麼要留呢?.
因為我已經不怕人了.
我已經不怕任何人說我.

$^{41}$所以我就開始做自己想做的事.
今天我們會講的講題是「上莊的相反」.
「結黨」.
所謂「黨」或「莊」其實都是類似的東西.
都是一群人.
一個群體.
一個擁有共同理念的一群人.
不過我想「上莊」是一件比較正面的事情.
「結黨」特別是在聖經裡面是一個比較負面的事情.
所以今天我們會跟大家去思考「結黨」這個題目.
我們一起祈禱.
我們一起將我們的講道交給上主.
上帝求你去對我們說話.
就算你自己真正的.
在當中你自己說話.
不是孩子.
在當中你的靈.
在當中對我們每一個弟妹.
無論在網上.
或者我們看錄影的現場每一個人.
我們再一次來謙卑在你面前.
聆聽昔日在保羅的教導.
求主你這樣來幫助我們.
讓孩子的說話都清楚.
讓弟兄姊妹的耳朵都能夠敏銳.
我們主動地來呼求你.
求你對我們說話.
因為我們的生命.
我們這個群體是一個合理,心意的群體.
縱然外面有很多雜音.
我們這個群體仍然心一意.
我們願意歸向你.
求主你自己去救治我們.
求主你自己去引導我們.
幫助孩子.
洪春永求.
阿們.
我們會說「結黨」的字.
不知道大家信主這麼久.
每次在聖經看到「結黨」這個字的時候.

$^{81}$大家會想些什麼.
初信主的時候.
為什麼聖經提到「結黨」這個字.
「結黨」是什麼意思呢.
為什麼叫「結黨」呢.
為什麼聖經裡面不容許「結黨」呢.
大家有沒有這樣問過.
參與政黨是不是叫「結黨」呢.
這樣就糟糕了.
我們聖經可能會違反國安法.
當然沒事.
暫時來說都沒有關係.
在《和本》裡面.
「結黨」這個字其實是一個古時的用法.
所謂「結黨」不是在說政治上的政黨.
而是一個不懷好意的群體.
古語有云.
這個出自於《論語》裡面的說話.
「君子經而不爭,群而不黨」.
什麼意思呢.
就是作為君子.
對別人的態度是爭而重之的.
而不是彼此爭鬥的.
聚在一起都不是去結黨的.
所以「黨」這個字從來都不是一個好的事情.
就是「牛里黨爭」.
「同黨」.
「其福黨」.
「五毛黨」之類.
都不是一些好東西.
不過後來當我開始學希林文的時候.
發現一個更加有趣的事實.
原來《和本》的翻譯.
其實是一個很獨有的翻譯.
如果你回看其他的譯本.
看英文版本的話.
你找不到裡面有「黨」這個字.
其他的譯本你都找不到「結黨」這個字.
英文聖經裡面主要是翻譯成什麼.
就是叫做「Selfish Ambition」.

$^{121}$就是解作一個自私的野心.
中文的兩種譯本.
翻譯成「私刑真誠的心」.
後來的「和修本」.
翻譯成更加簡單的「自私」.
所以發覺《和本》這個版本.
是一個很特別的版本.
當時翻譯成「結黨」這個字.
不過雖然如此.
我仍然覺得《和本》的翻譯是挺有趣的.
也是挺值得.
也是挺有趣的翻譯.
因為基本上所謂「結黨」.
就是一個群體的集體的自私.
一個人的自私可以叫做自私.
但是一群人的自私.
就是叫做「結黨」.
所謂「結黨」.
就是一群人一起.
很團結地去自私.
在人類史裡面.
一群人一起.
為了鞏固集體的好處.
很齊心地去排除異己.
這就是「結黨」.
很著名的二戰時代的神學家.
拉奧尼布·尼泊爾.
在二戰前寫了一本很出名的書.
叫做《道德的人與不道德社會》.
Moral Man and Immoral Society.
這本書講述了一個很真實的事實.
一個人可以是很道德的.
但是當一群人走在一起的時候.
那個道德的力量.
往往會變成一股很可怕的力量.
民族主義就是這樣.
一群人大家一起很團結地.
去鞏固自己的好處.
團結是美好的事情.
但是團結地去排除異己.

$^{161}$就是非常可怕.
真人.
在人類史裡面.
做壞事往往比做好事更加團結.
所以「結黨」就是一個強烈地.
擁有共同信念的群體.
當這個強烈地擁有共同信念的群體.
做出一些維護自己集團利益的事情的時候.
是一件非常恐怖的事情.
才發現原來早期教會.
早已經出現了這樣的問題.
一群人一起很團結地做出一些壞的事情.
我們看看幾段有關結黨經文.
我今天掃描一下有關很多結黨經文.
基本上《聖律辯》主要用了.
epiphaia這個字.
來翻譯成結黨.
例如在《羅馬書》裡面.
唯有結黨不信從真理.
凡信從不義的.
就以憤怒,怒恨,報應貪玩.
《林前》裡面.
在你們中間不免有紛紜結黨的事.
我叫那些有經驗的人顯明出來.
《林後》裡面也提過.
大批的清單裡面.
都有結黨這個字.
《加勒斯多德》裡面.
情慾的事情.
都是包括了結黨這個字.
《阿哥書》裡面這麼說.
你們心裡若懷著苦毒的嫉妒和紛爭.
這個字背後其實都是原文.
都是結黨這個字.
所以你會發現結黨就是這批字眼.
嘗試去引起紛爭.
嘗試去一群人排斥異己.
這樣的事情.
另外也有一些希利文.
就是其他的字眼.

$^{201}$今天詳細不說了.
下一頁我只是說了另一批希利文.
都有結黨這個字的意思.
是另一個原文.
但是也代表了結黨這個字眼.
所以你會發現.
今天你會看到.
今天我們看完經文.
正正就是其中一段.
我覺得很有意思和大家想說的一段.
有關結黨經文.
就是這個《肥立比書》.
一章十五到第十八節經文.
我們一起讀吧.
一起讀 看看下一章的關鍵字眼.
我們一起讀吧.
預備 1 2 3.
等一下 放回這個.
保羅在《肥立比書》裡面.
提到一個肥立比教會的情況.
教會裡面出現了一些內鬥.
教會裡面有一些反對保羅的一班人.
當然你會覺得.
保羅有敵人.
《聖》裡面也不是新聞.
一向有很多敵人.
那裡不多前書 那裡不多後書.
加上一些書.
不喜歡保羅的人很多.
基本上保羅在每一本書裡面.
都有一些教會裡面的敵人.
所以這次在《肥立比書》裡面的敵人.
有什麼特別呢.
有的 有些特別的.
因為在《肥立比書》裡面.
這班所謂保羅的敵人.
被保羅形容為.
全基督是出於嫉妒紛爭.
全基督是出於結黨 並不單純.
意思是要增加我的苦楚.

$^{241}$你會問 究竟全基督是出於結黨.
全基督是出於嫉妒紛爭.
是怎樣解釋呢.
好端端的 全基督是怎樣傳到結黨的呢.
究竟是一個什麼層次的傳福音呢.
是一個什麼花式的玩法呢.
就是說全基督明明是一件很正面的事.
是很愛主的事.
為什麼全基督會突然傳到結黨呢.
全基督傳到結黨是一件很難以消化的事情.
就好像我傳福音傳到要去要殞你一樣.
我會十字架要去卵化你一樣.
整件事是很奇怪的.
我反复出崇巴爾 你難受啊.
明白嗎 這是一件很奇怪的事情.
我覺得很恐怖.
不單恐怖 而且是很高難度的.
很複雜的 複雜到好像人家的口供劇一樣.
傳福音傳到要算中計 公心計.
做到這樣也不容易.
不像那些口供劇的劇情.
我回Info Group是為了討潘Sir潘心.
然後說謊騙了祖元佛和慈道令他排斥他.
很複雜的 回教會怎麼會這麼複雜.
我不傾向這樣去理解這件事情.
雖然是對的 出了很多問題.
但也沒有那麼恐怖.
沒有那些公心計 暗黑的事情.
我認為所謂傳基督是出於結黨.
這句話說的不是純粹一種動機的問題.
不是說這群人傳福音純粹是為了要去搞到結黨.
而是說在傳基督這件事裡.
在他們當中還有一個瀰漫著他們的事情.
在這群人當中.
這群傳基督的人同時是結黨的.
真是一個比較客觀公允的情況.
這群人傳基督同時有結黨的事情.
發生在他們當中.
這正正就是菲律賓教會的情況.
有一群人傳基督 另一群人傳基督.

$^{281}$兩群傳基督的人當中.
有一群人想弄死另一群人.
真是這麼恐怖.
教會的紛爭當中.
兩群陣營的人都是在傳基督的.
我們經常都以為教會出現的問題.
總是一些很極端的正邪對立.
一群人就傳基督.
一群人就傳魔鬼.
一群人就走閘路.
一群人就貪愛世界.
一群人就忠心愛主.
一群人就在破壞教會.
不是一種相對主義.
不是說教會沒有壞人.
而是說結黨這些事情.
往往就在一個.
大家都很好的傳基督的日子當中.
就出現了.
一群人 一群跟隨耶穌的人.
然後就變成了兩群人.
一群一起去跟隨耶穌的人.
變成了兩群惡鬥的人.
兩群人都是跟隨耶穌.
兩群人都是宣揚基督.
一群人一起在教會裡侍奉.
然後這群人就勢不兩立.
結黨 紛爭.
結黨往往都是在愛主的事情裡發生.
你不要以為結黨就是一群人跟隨基督.
一群人就離開了基督.
兩群人都傳基督.
然後他們就結黨紛爭.
你說傳基督夠愛主了吧.
但都可以導致結黨紛爭的事情.
一個傷害極高的事情.
大家都是國內人.
曾經傷害你的基督徒.
往往都是一些很愛主的基督徒.
不是那些沒有回教會.

$^{321}$屬魔鬼的基督徒.
本來是一些很美麗的事情.
一些很神聖很有意義的事情.
正正那些事情.
變成了一些結黨紛爭的事情.
說一個我的聽聞.
不是我在神學院的.
這次不是才來的.
聽聞在神學院出現了一些事情.
一個神學生的意見跟其他人不同.
那個神學生就成為了一群神學生集體排斥的對象.
然後他們還用.
神學生吵架是用什麼.
是用聖經的.
他們說那個神學生.
單顧自己事不顧別人的事.
然後就排斥他.
說他沒有愛心之類的.
不是我的神學院.
這個故事不是我想說的.
所以我想說.
這段時間我想說幾分鐘.
有關香港教會.
不是吃花生.
而是說一下我這幾年的感受.
我覺得香港教會.
特別是香港基督教會.
真的是一個是非圈.
不知道為什麼.
天主教沒有這些東西.
佛教也沒有這些東西.
台灣基督教也沒有這些東西.
香港基督教有這些東西.
很多這些東西.
就算是那些不信耶穌的人.
外面的團體也沒有這些東西.
也不會像香港基督教那麼多小圈子.
又有很多黨派是非.
又在這裡發帖說你.
人家那些微信主群體也好好的.

$^{361}$人家那些做蛋糕的就做蛋糕.
一般人不是出去做蛋糕群體.
我加入了一個電單車群體.
就說電單車.
有空出來出車.
沒有那麼多說你的車不好.
又說開得不對.
沒有這些東西.
大家不明白嗎.
喜歡吃燒賣就做個什麼.
就均組.
大家就吃燒賣.
沒有那麼這些是是非非.
互相指控的東西.
唯獨香港的基督教會.
不知道為什麼特別多這些圈子和是非.
中派和中派之間.
不說了.
教會和教會之間.
部門和部門之間.
弟兄姊妹等等.
很多這些事情發生.
可能大家也經歷過.
崇拜部和差全部欠資源.
這個牧區和一個牧區的房子問題.
搞亂了.
這個部門舉辦活動.
沒有通知另一個部門.
然後生他的氣.
很多這些教會裡面.
大家群體之間的阻隔.
網上就更加恐怖.
可能大家也知道.
我一向有很多hater.
前陣子我們有一位同工.
在Full Church裡面講道.
有人在網上批評他的講道.
說他哪裡哪裡說得不對.
後來我知道可以上網看看.
原來是他.

$^{401}$他是一個我忠實的hater.
當然我不會開名.
大家也不要看他.
他裡面只有酸澀和仇恨.
很多年都沒有看他Facebook.
不知道他原來還有寫東西.
我自己也不寫Facebook.
但他仍然沒有停止寫我.
我回想起來.
原來他寫了這麼多東西.
還不停地寫我的東西.
基本上我每一次的講道.
他都有看.
他還會寫文章回應.
基本上Full Church每一個post.
他都有留意.
然後又說我們留堂怎樣怎樣.
他真的是一個我忠實的fans.
就算我沒有講道也好.
突然寫了一句話.
他也會想起我.
他說前陣子突然在搜尋.
找到幾年前寫的一句話.
他突然會想起我.
然後會寫東西.
他真的對我朝思暮想.
他回留堂還找到你.
他每個星期都有追我的留堂崇拜.
再去回應和寫筆記.
好了 講完了.
不講這些了.
我想講這些就是結黨.
他不會覺得這是問題.
因為他覺得他在傳揚基督.
看天行道.
相反前陣子我在MM.
一個七八萬眾生活.
看過一段令我很感動的片.
這條片叫熱血專業人士.
籌旗贊助香港世界級寵物小精靈訓練員.

$^{441}$去日本參加世界追夢賽.
這條片訪問了一群.
香港仍然用手機周街捉精靈的香港人.
大家都玩過.
但已經沒有玩.
這群人仍然繼續玩下去.
還玩得很厲害.
令我感動的是.
這群人的組合之闊.
這群人的年紀差距之大.
令人覺得很驚訝.
有小朋友 有DSE學生.
有初職 有丈夫 有爸爸.
有茶餐廳侍應 有中學老師 有律師.
有癌症科醫生 有大學教授.
有七十多歲老人家.
總之無論他們的差別有多大.
這群人走在一起.
高高興興地捉Pokemon.
他們一看Pokemon就很興奮.
他們厲害到聽到.
聽到聲音都能判斷是精神強烈.
是電擊還是重咬.
是用水鴨 水戰龜 還是穿山王.
他們是熟到不得了.
後來故事就這樣發展下去.
這群人當中.
他們其中一個打得最厲害的.
一個中五的中學生.
他叫做蝦皮仔.
他的實力是可以代表香港.
去參加日本舉辦的寵物小精靈世界賽.
不過因為他要考DSE.
蝦皮仔的媽媽不讓他參加.
要他讀書 不讓他去日本參加.
更加是他沒有錢去.
於是這群人當中.
就百折各按其.
照著身體功能彼此相助.
想辦法讓他們能夠去到日本參加比賽.

$^{481}$其中一段很感動的片段是甚麼呢.
其中一個中學老師.
他說打Pokemon玩到有三千多分難不難.
很難.
旁邊的醫生說很難 很難 很難.
但全世界只有十個八個能力打到三千多分.
就是蝦皮仔.
所以你跟我說沒有錢去日本.
哥哥幫你.
於是這個中學老師就擔當了領隊的角色.
組織了一個籌款活動.
整個日本之旅做領隊.
那個癌症科醫生幫他補習.
厲害吧.
有個醫生幫他補英文 幫他補大便.
最後最感動的是甚麼呢.
他們找到金主.
一個很喜歡玩Pokemon的有錢人.
捐了幾萬元給他們.
能夠成團去日本參加比賽.
當然這班香港代表隊一去到之後.
初賽就止步了.
不過It's OK.
許多事情都是這樣.
你認為值得就是值得.
就好像玩Pokemon一樣.
一班人一起做一些事情.
這個故事令我很感動.
因為這班人正正是比香港教會更加教會.
一班人走在一起.
本來就是一個很單純的理由.
就是喜歡玩Pokemon.
你打得比我好 我不會妒忌你.
更加樂意支持你 奮身去支持你.
你比我捉得更加多精靈.
我不會說你不對.
說你捉了我的精靈.
通街都是精靈.
通街都是上帝的恩典 你爭甚麼.
你可以去日本參加比賽 我為你感到驕傲.

$^{521}$我奮身去支持你.
不會說你怎樣捉得不對.
說你不對 說你不好.
看到Pokemon的故事 我想起教會.
為甚麼香港教會不是這樣.
我可以回到經文裡.
保羅說 這又何妨呢.
物是假意 或是真心.
無論怎樣 基督究竟被傳開了.
為此我就歡喜 或更會歡喜.
對於肥立比教會結黨的這班人.
保羅怎樣回應.
保羅說 這又何妨呢.
無論怎樣 基督究竟被傳開了.
可能你會覺得 經文的意思大概是一種實用主義.
只要基督被傳開.
這班人基督都傳開了.
就好了 有人輸就好了.
不要管教會有多黑.
黑貓白貓 捉到老鼠就是好貓的道理.
不是 我不認為保羅是純粹實用主義.
或是結果主義.
保羅是重視基督被傳開了.
因為保羅介意的是一個更高層次的事情.
或者說保羅選擇以一個更高層次的角度.
去面對他現在面對的事情.
教會裡面這班人和那班人.
在內鬥是一個層次.
但基督耶穌是一個更高層次.
坦白說 面對這些結黨的攻擊.
保羅大可以選擇在同一個層次裡面去回應他.
他們攻擊保羅.
保羅用同樣的手段去攻擊他們.
他們拉攏人去批鬥保羅.
保羅都可以拉攏另一班人去批鬥他們.
他們在Facebook取笑他.
他大可以在Facebook取笑他.
但保羅沒有.
保羅可以這樣做 但他沒有.
要真真正正去解決結黨的問題.

$^{561}$你不能夠在同一個層次裡面去解決結黨的問題.
因為面對著結黨.
你用另一個黨去對付那個黨.
不就是結黨了嗎.
你和他有什麼分別.
如果你這樣對待他的時候.
你只能夠站在一個更高的層次裡面去看待這件事情.
你就能夠越過結黨這個問題.
昨晚我和我女兒吃飯.
旁邊坐了四個說普通話的家庭.
兩個媽媽和她個別的兩個兒子.
四個人在吃麵.
那兩個小朋友大概都是四歲左右.
突然間就吵起來.
很大聲地在哭和叫.
吵到要擦天.
這兩個小朋友在爭什麼呢.
他們在爭位置坐.
一個說要坐那個位置.
一個說那個位置是他的.
然後這兩個小朋友硬是坐在小朋友旁邊.
坐在大腿那邊.
兩個人就像疊羅漢一樣坐在很窄的位置上.
在掙扎.
然後你扯我的頭髮.
我按你的頭.
你捉我的下巴.
然後我又揹你的臉.
兩個小朋友又在哭又在叫.
很吵很吵.
我就坐在兩個人旁邊.
在吃麵.
控制自己的情緒.
很吵.
最後一段時間.
那兩個媽媽竟然在聊天.
完全沒有理會他們兩個小朋友在吵架.
我心想可不可以停一下他們.
他們兩個還在吵.
可不可以幫一下忙.

$^{601}$誰知道他們打來一下.
我再打他一下.
這就是小學生打架.
沒完沒了的.
解決不了問題.
唯有家長出手.
或者老師出手.
一個更加高層次的位置出手.
或者說當你站在更加高層次的時候.
你發現原來整件事很像學雞.
真的 不知道你是不是面對這樣的問題.
在你的工作環境裡.
是不是類似這樣的問題.
一群人和一群人.
彼此互相私鬥.
你在當中.
在教會裡.
當你正常做愛主義的時候.
突然間有一些明爭暗鬥.
很多這些傷害的事情.
嘗試學習站在一個更加高的高度裡.
站在一個耶穌基督榮耀的角度裡.
去面對這件事情.
想想你去教會是為了什麼.
你信耶穌是為了什麼.
一群人走在一起.
做一些事情是為了什麼.
當你這樣想的時候.
或者你有更加多的空間.
能夠知道這件事是為了什麼.
至少這幾年.
我自己在這幾年是這樣去面對這件事情.
稱之為跳高一個層次的快樂.
站在一個更加高的層次上.
用上帝的角度去看著這群基督徒.
單單打一打.
你插我插你.
Facebook裡面罵來罵去.
其實挺好笑的.
上帝怎麼看這群基督徒在做什麼.

$^{641}$Biblically.
保羅這樣說.
為此我就歡喜.
並且還要歡喜.
保羅快樂地面對結黨的問題.
不是因為教會的結黨問題消失了.
也不是保羅擊破了他為敵的那群結黨的人.
而是保羅跳高一個層次.
去理解這件結黨的事情.
所以他就能夠仍然快樂.
想想那群寵物小精靈的訓練員.
他們為什麼聚在一起.
就是一個很單純的原因.
就是喜歡寵物.
想想他們玩寵物時開心的樣子.
這個就是你回教會的樣子.
這個就是你成為基督徒.
一群人聚在一起的樣子.
大哥我只是想玩寵物.
我不想那麼多教會政治.
我只是想玩寵物.
我不想你們罵來罵去.
我只相信耶穌.
我不想有那麼多這些東西.
保羅在最後.
在阿彼書的第二章裡.
一段大家很熟悉的經文.
他本有神的形象.
不以自己與神同等為強奪的.
反倒虛己.
取了奴僕的形象.
成為人的樣式.
既有人的樣子.
就自己卑微.
全心信服.
以至於死.
且死十字架上.
所以上帝將他升為至高.
又賜給他超乎萬名以上之名.
與一切在天上的.

$^{681}$地上的和地底下的.
因耶穌的名無不屈膝.
無不叩稱耶穌為主.
使榮耀歸於父上帝.
這段基督之歌.
正正就在這句說話的後面.
凡事不可結黨.
只要存心謙卑.
各人看別人比自己強.
你只能夠望著這位.
榮耀的耶穌基督.
才能夠越過.
我們在教會裡面.
這個結黨的問題.
最後我想說一點.
特別是全聖教的應用.
你看到我們這個月題.
叫做「上莊」.
究竟上了莊 上了什麼莊呢.
究竟一群人.
做些什麼 做些什麼呢.
遲些會說的.
我們會有些Discord.
今天報告時說有Discord.
我們會說我們的Club.
會積極地大家參加Club.
一群人一齊去做一些事.
等等.
全聖教這幾年.
其實一直都沒有這些事.
沒有多這些大家組織在一起.
除了小組和崇拜.
都沒有這些事.
好處就是沒有結黨空間.
沒有一起去侍奉 幫手.
就沒有吵架的空間.
但我們覺得.
其實大家一起去做一些事.
是應該的.
都是可以做的.

$^{721}$但希望大家都可以記得.
我們不要回到以前的舊路.
遇見結黨問題的時候.
遇見大家看法不同的時候.
遇見到這些愛主義的背後.
這些看法的時候.
大家彼此的提醒.
我們Full Church.
我們看著我們的耶穌基督.
好好地走下去.
意見不同是難免的.
但我想仍然可以克服的.
這是我們基督徒.
作為教育群體.
我們要克服的事情.
我們祈禱.
主你求你建立我們Full Church的教會.
求主你幫我們能夠在當中彼此相愛.
好像菩羅所說.
我們看別人比自己更加重要.
我們以基督耶穌的心為心.
我們來連手於你.
求主你幫助我們.
讓我們每一群弟姐妹.
當我們願意走在一起的時候.
無論在小組裡.
無論在我們的群體裡.
我們都求主你用愛.
來勝過我們當中的分歧.
讓我們能夠因為你的緣故.
越過一個一個的困難.
讓我們能夠在當中有和平.
能夠在當中有彼此相愛的關係.
求主你幫助我們.
奉主命求.
阿門.
\newpage



\section{雅各書 2:1-9-12-17-20231014}
\label{sec:CYdq0eOXFpM}
\textbf{【網上崇拜】莊員的關係|雅各書2\_1-9,12-17|20231014 [CYdq0eOXFpM]}
\newline
\newline
連結: \href{https://youtube.com/watch?v=CYdq0eOXFpM}{\texttt{ https://youtube.com/watch?v=CYdq0eOXFpM}} ~~~~ 語音日期: 2023-10-14 
\newline
\newline
\hyperref[sec:3_8UYgGhJ0E]{\small{< < < PREV SERMON < < <}}
~
\hyperref[sec:index_chronic]{\small{[返順時目]}}
~
\hyperref[sec:index_scriptual]{\small{[返順卷目]}}
~
\hyperref[sec:3YrDRTxYY2U]{\small{> > > NEXT SERMON > > >}}
\newline
\newline
雅各書 2:1-9-12-17-20231014
\newline
\begin{longtable}{cl}
\hline
\hline
章節 & 經文 (和合本修訂版)\\
\hline
2:1 & \begin{tabularx}{0.7\textwidth}{X} 我的弟兄們,你們信奉我們榮耀的主耶穌基督,就不可按著外貌待人。 \end{tabularx} \\ \\ \relax
2:2 & \begin{tabularx}{0.7\textwidth}{X} 若有一個人戴著金戒指,穿著華麗的衣服,進入你們的會堂,又有一個窮人穿著骯髒的衣服也進去, \end{tabularx} \\ \\ \relax
2:3 & \begin{tabularx}{0.7\textwidth}{X} 而你們只看重那穿華麗衣服的人,說:「請坐在這裡」,又對那窮人說:「你站在那裡」,或「坐在我腳凳旁」; \end{tabularx} \\ \\ \relax
2:4 & \begin{tabularx}{0.7\textwidth}{X} 這豈不是你們偏心待人,用惡意評斷人嗎? \end{tabularx} \\ \\ \relax
2:5 & \begin{tabularx}{0.7\textwidth}{X} 我親愛的弟兄們,請聽,神豈不是揀選了世上的貧窮人,使他們在信心上富足,並承受他所應許給那些愛他之人的國嗎? \end{tabularx} \\ \\ \relax
2:6 & \begin{tabularx}{0.7\textwidth}{X} 你們卻羞辱貧窮的人。欺壓你們,拉你們到公堂去的,不就是這些富有的人嗎? \end{tabularx} \\ \\ \relax
2:7 & \begin{tabularx}{0.7\textwidth}{X} 毀謗為你們求告時所奉的尊名的,不就是他們嗎? \end{tabularx} \\ \\ \relax
2:8 & \begin{tabularx}{0.7\textwidth}{X} 經上記著:「要愛鄰 如己」,你們若切實守這至尊的律法,你們就做得很好。 \end{tabularx} \\ \\ \relax
2:9 & \begin{tabularx}{0.7\textwidth}{X} 但你們若按外貌待人就是犯罪,是被律法定為犯法的。 \end{tabularx} \\ \\ \relax
2:10 & \begin{tabularx}{0.7\textwidth}{X} 因為凡遵守全部律法的,只違背了一條就是違犯了所有的律法。 \end{tabularx} \\ \\ \relax
2:11 & \begin{tabularx}{0.7\textwidth}{X} 原來那說「不可姦淫」的,也說「不可殺人」。你就是不姦淫,卻殺人,也是成為違犯律法的。 \end{tabularx} \\ \\ \relax
2:12 & \begin{tabularx}{0.7\textwidth}{X} 既然你們要按使人自由的律法受審判,就要照這律法說話行事。 \end{tabularx} \\ \\ \relax
2:13 & \begin{tabularx}{0.7\textwidth}{X} 因為對那不憐憫人的,他們要受沒有憐憫的審判;憐憫勝過審判。 \end{tabularx} \\ \\ \relax
2:14 & \begin{tabularx}{0.7\textwidth}{X} 我的弟兄們,若有人說自己有信心,卻沒有行為,有甚麼益處呢?這信心能救他嗎? \end{tabularx} \\ \\ \relax
2:15 & \begin{tabularx}{0.7\textwidth}{X} 若是弟兄或是姊妹沒有衣服穿,又缺少日用的飲食; \end{tabularx} \\ \\ \relax
2:16 & \begin{tabularx}{0.7\textwidth}{X} 你們中間有人對他們說:「平平安安地去吧!願你們穿得暖,吃得飽」,卻不給他們身體所需要的,這有甚麼益處呢? \end{tabularx} \\ \\ \relax
2:17 & \begin{tabularx}{0.7\textwidth}{X} 信心也是這樣,若沒有行為是死的。 \end{tabularx} \\ \\ \relax
2:18 & \begin{tabularx}{0.7\textwidth}{X} 但是有人會說:「你有信心,我有行為。」把你沒有行為的信心給我看,我就藉著我的行為把我的信心給你看。 \end{tabularx} \\ \\ \relax
2:19 & \begin{tabularx}{0.7\textwidth}{X} 你信神只有一位,你信得很好;連鬼魔也信,且怕得發抖。 \end{tabularx} \\ \\ \relax
2:20 & \begin{tabularx}{0.7\textwidth}{X} 你這虛浮的人哪,你願意知道沒有行為的信心是沒有用的嗎? \end{tabularx} \\ \\ \relax
2:21 & \begin{tabularx}{0.7\textwidth}{X} 我們的祖宗亞伯拉罕把他兒子以撒獻在壇上,豈不是因行為得稱義嗎? \end{tabularx} \\ \\ \relax
2:22 & \begin{tabularx}{0.7\textwidth}{X} 可見信心是與他的行為相輔並行,而且信心是因著行為才得以成全的。 \end{tabularx} \\ \\ \relax
2:23 & \begin{tabularx}{0.7\textwidth}{X} 這正應驗了經上所說:「亞伯拉罕信了神,這就算他為義」;他又得稱為神的朋友。 \end{tabularx} \\ \\ \relax
2:24 & \begin{tabularx}{0.7\textwidth}{X} 這樣看來,人稱義是因著行為,不是單因著信。 \end{tabularx} \\ \\ \relax
2:25 & \begin{tabularx}{0.7\textwidth}{X} 同樣,妓女喇合接待使者,又放他們從另一條路出去,不也是因行為稱義嗎? \end{tabularx} \\ \\ \relax
2:26 & \begin{tabularx}{0.7\textwidth}{X} 所以,就如身體沒有靈魂是死的,信心沒有行為也是死的。 \end{tabularx} \\ \\
[1ex]
\hline
\hline
\end{longtable}
$^{1}$對你們來說可能很新.
又要拍手又要跟歌詞.
值得在網上再看一看.
因為我們有些效果在裡面.
所以 啊!.
值得回去看看.
現場是看不到的.
網上的弟兄姊妹就看到了.
值得回去再學一下這首新歌.
我相信在基督教有很多詩歌.
對於大家來說是很闊的.
整個群體的學習過程.
整件事當中大家一起去有不同的探索.
今天就講關於樂題的內容.
關於上莊.
第一章不會出相片的.
不用太快期望.
講完就說10月就看看出什麼相片.
先讀讀經.
今天選的經文是雅各書二章的經文.
可能對弟兄姊妹來說.
初信的時候或以往都查過雅各書.
今天我們就看第二章的選段.
我們一起讀選段的經文.
雅各書第二章第一節.
請.
天上帝母當我打開你的說話的時候.
你對你的子民所教導的.
從開初到現在沒有過時.
今天我們仍然經歷在其中.
求聖靈在當中督澤.
求聖靈在當中教導.
求聖靈在當中改正我們.
以致我們在新的群體.
新的關係建立當中.
是蒙上帝你所喜悅.
在當中繼續做上帝.
你想要我們做屬靈群體的表現.
求主你幫助.
祈禱奉耶穌的名.

$^{41}$阿們.
今天你聽到這段經文.
可能重溫了一段日子.
今天講題是莊園的關係.
一講到關係.
可能有些頂梓妹就.
唉.
又或者對你來說.
關係是一個很糾纏的事.
因為你每去一個新的群體當中.
你就會建立關係.
你去一個新的群體當中.
你就期望一些你想建立的關係.
總會有些人就會在一個群體當中.
想建立關係當中.
他就會改一個新的名字.
不知道你有沒有試過.
因為我自己就比較.
來來去去都是那幾個名字.
沒有什麼特別.
但是有時候就是那些群體.
叫我那個名字的時候.
我就立刻想起那時候的關係.
不知道你有沒有這樣.
什麼人叫你什麼名字.
就想起那個關係.
什麼人叫你什麼名字.
就想起那段時光.
莊園的關係對我來說.
大學的上莊對我來說.
不是今天要分享的內容.
但是有兩個群體.
一群人一起做一些事情.
有兩個群體對我的靈命.
或者對我的生命有很大影響.
那兩個群體是什麼群體呢.
就出兩張相片給大家看看.
好了.
(笑聲).
笑吧笑吧.

$^{81}$沒什麼機會的了.
首先說說上方那張.
穿灰色衣服的那張.
那個是我來的.
現在也有很多頭髮了.
那個群體是.
是我在籃球隊的群體.
我是90年代初期的福音籃球隊.
當中信耶穌的.
在當中去認識教會.
認識一些群體.
旁邊那個是一個美籍的華人.
每年暑假.
我都會跟那些美國的.
看體試工的團體.
或者一班弟兄姊妹.
就會經香港做一些.
看體試工的訓練.
或者訊息分享.
然後就陪他們回內地.
我主要負責幫他們看中文.
中場翻譯.
幫他們講教導.
那個群體讓我感受到.
原來那個服務是用什麼心態去服務.
而他們整個服務團隊的群體.
其實來自美國不同的州省.
他們都拿他們每年一些.
年度假期.
或者放暑假的過程中.
成為一個群體.
但在過程中.
我很感受到一件事.
原來他們不是帶.
當然他們會帶技投身.
有些可能會一些樂器.
或者其他東西.
但在過程中.
他們很懂得欣賞對方的長處.
也很懂得欣賞對方的限制.

$^{121}$那限制也要欣賞的.
一會兒我會再講.
在過程中.
我覺得很有趣.
原來他們不是本身懂得.
但他們很熟悉.
每次透過服務過程中.
不斷地分享.
每次服務完結後.
就會發覺自己團隊中.
有些事與不是的時候.
每次都會去檢討.
為了下一次服務的時候.
有改進和能夠再用對方.
所以籃球群體對我來說.
也影響到我之後在籃球隊的服務.
也影響到我怎樣喜歡團體運動.
我自己很多運動我都喜歡玩.
但捉籃牌是我最喜歡玩的運動.
因為大部份都是團體運動.
拿牌的不是不懂得玩.
但我比較覺得悶.
No offence.
不關你們的事.
因為我覺得自己打.
打的都是自己打.
不是對家對牆.
但我覺得我喜歡團隊運動.
所以捉籃牌是我最喜歡的活動.
在過程中學習分享.
學習欣賞.
學習彼此輔助.
其中在籃球群體建立關係.
剛才說到限制.
懂得欣賞對方的限制.
在籃球過程中.
如果你打過就知道.
身高手長就是制胸.
短小精悍就是速度.
所以籃球不只是高.

$^{161}$你要走動.
在走動過程中.
你懂得欣賞對方的體質和體格.
其實就成為贏球的關鍵.
如果你打機.
五個都選Michael Jordan.
你都會輸.
不是.
看技術.
不過是打機.
不是真人.
你找不到五個Michael Jordan.
你知道什麼是Michael Jordan嗎.
現在年輕的LeBron James.
或者是Kobe.
無論如何.
你會發覺在過程中.
不同形態.
在整個團隊中學習彼此欣賞.
你的限制會成為團隊的祝福.
有時你不知道.
在整個過程中.
對我來說是有深刻的感受.
第二個群體.
你看到藍色衣服那個.
你明顯是老成了.
近期少了.
近期很多.
OK.
起碼那時候沒有白頭髮.
那個是什麼群體呢.
那個是拍照的群體.
很多人都不知道我喜歡拍照.
其實我自己很喜歡.
我是沒有數碼相機的.
如果手機是數碼相機.
就不當了.
我都有.
我全部都是菲林相機.
就算現在家裡也是.

$^{201}$我很喜歡拍照.
我住在深水Po 12年.
過程中.
因為輪班工作.
很多時候睡醒了.
或者沒事做.
就會去這間相機舖打燈.
拍些照片出來.
就會在那裡聊天.
相機界.
我想大家都可能會經歷的.
就是器材很重要.
一支鏡子可以價值不菲.
你看到這些鏡子都很.
你有所不知.
說來話長.
但是在我認識的群體當中.
反而大家都不是給器材.
是給你用的器材.
是否用得其所之餘.
和用的器材.
你能夠呈現到它的特質.
是很重要的.
所以很多時候.
這張相片是在試鏡.
試鏡.
人家改了鏡子.
或者將鏡子重新拋光.
讓它可以拍照.
沒有發霉.
在試鏡.
所以被人插了一張.
就拿回家了.
在過程當中.
大家都知道.
這張相片不錯.
很鋒利.
很漂亮.
拍出來很清楚.
這支是什麼鏡子.

$^{241}$你猜猜.
這支鏡子都拍成這樣.
很漂亮.
怎麼拍的.
其實不是給錢的問題.
是欣賞一些人家棄用了的東西.
你能夠給它有第二生命.
人家覺得錢買回來.
就多賺錢.
但是在群體當中.
不是說有多少錢.
是說你用得其所之餘.
又沒有將那件事發揚光大.
在拍照的群體當中.
讓我感受到.
每樣東西都有時限.
但時限當中能不能彼此去配搭.
有些人儲了很多東西.
但就儲了.
但就不用了.
但能夠令到那件事.
他覺得你合用的時候.
他可以放給你.
便宜一點放給你.
讓你可以繼續用.
在這個群體當中.
讓我發現一件事.
不是有錢就解決問題.
在群體當中的關係.
讓我明白到.
原來懂得欣賞一些人棄.
棄置了的東西.
能夠有第二生命.
其實這種關係也是很重要.
對我來說.
那群人很無私.
那群人很開心.
大家有共同興趣.
最重要的是不分年齡.
有些退休了.

$^{281}$有些像我上班.
輪班工作就時間彈性.
約出來一起去試試.
什麼都拍.
特別我們在深水Po 舊區.
拍攝人家擺攤.
拍攝人家裝東西.
拍攝人家倒垃圾.
拍攝舊樓重建.
拍來拍去的不同位置.
過程當中又可以交流很多心得.
不是因為你的身份是專業人士.
不是因為你的身份是一個漢奸.
真的有漢奸.
不是因為這些不同職級的人.
過程當中就脫去了.
社會定義的外衣.
大家有共同興趣的關係.
是很開心的.
兩個群體.
一個群體是功能性.
彼此配搭.
另一個群體是興趣欣賞.
這兩個群體都是塑造了我自己的看法.
但在看雅各書的過程當中.
其實不盡相同.
雅各書當中.
如果大家理解.
他應該是耶穌的弟弟.
寫過雅各書.
詳細就不講解了.
但他面對的是.
一群受輸的人.
他們面對的問題就是.
我的弟兄們.
我們信奉我們的榮耀主耶穌基督.
便不可戴外貌戴人.
因為他見到一群群體當中.
大家以貌取人.
這句說話.

$^{321}$年輕人和小朋友都明白.
現在怎會不看呢.
無論我兩個兒子在小學的時候.
都會認識上人.
都會看看大家的身光頸靚的情況.
這件事是整個社會界定.
什麼是good looking 好看.
或者是潮一點.
或者是型一點.
在這個群體當中.
雅各書其實被人問.
沒有什麼神學.
或者不是講信仰上的核心價值.
是的.
雅各書最主要不是講耶穌基督.
因為雅各寫信給受輸的十二支派.
第一章第一節講的十二支派.
其實他們已經assume到.
大家都有一個common factor.
我們都是認順了耶穌基督.
我不需要在裡面再跟你講耶穌基督是什麼.
他不跟保羅書信.
保羅書信會講我們信的耶穌基督是什麼.
因為保羅是target那些外邦人.
他歸順了耶穌.
他們要義正詞嚴地告訴他們.
我們信的耶穌是什麼身份.
和他為什麼會這樣做.
但是雅各是寫十二支派.
他們知道.
而在受輸過程當中已經是耶穌經歷了.
耶穌升天三十多年後的事情.
所以雅各重點是在講.
我們這個群體.
我們在做什麼.
你們這群人.
你們的行徑有什麼.
他看到出現問題的時候.
他寫給那群十二支派的人.
你們以貌取人.

$^{361}$為什麼你們以貌取人.
中間我剛才灰色了字.
雅各覺得你們為什麼要這樣做.
穿金戴銀穿得好看.
你就叫他坐前面.
但現在沒有人坐前面.
你叫他坐前面.
你覺得他穿衣衫藍柳的時候.
你就看不起他.
為什麼你們要這樣做.
雅各在問我們這群人.
不是大家都信耶穌基督嗎.
耶穌基督是不是這樣做的.
耶穌基督不是這樣做的.
為什麼我們一群信耶穌的人這樣做.
雅各在問這個群體中.
為什麼要出現這樣的情況.
你們的關係是什麼.
如果大家都是同一個血緣.
就是耶穌基督的補血.
更新了我們有新的生命的時候.
為什麼會出現這樣的情況.
雅各在問.
這豈不是你們偏心待人.
你們用你的標準去斷定人.
如果是這樣的話.
我做一個反題.
其實基督的信仰有沒有更新了你.
你仍然用舊的方式.
或者現行社會的方式去看待一個人.
還是你信了耶穌之後.
你有改變.
雅各在問其實信了耶穌之後.
你有沒有改變.
或者信了耶穌之後你有什麼改變.
你自己知不知道.
從那個群體當中.
有沒有再受教育.
以致用上帝的視覺.
看我們周遭的人.

$^{401}$周遭發生的事.
這個都是問你自己.
不一定上個星期是水禮.
就打風.
改期.
水禮其中最重要的就是.
你開展新的一頁.
你開展新的一頁的時候.
你總有很多立願.
求上帝更新我們.
我願意在眾人面前承認基督.
不以為恥.
我要開展新的信仰生活的時候.
你就問自己.
其實上帝改變了你什麼.
每個人總有些事情會改變.
每個人總有自己的限制.
你問一問自己其實你想改變什麼.
信主前和信主後.
有什麼改變.
總有些.
有些事你真的要認真去改.
還是你還是用坊間的方法去看人.
不知道你有沒有想過問題.
看人其實沒有人教你的.
是你自己懂的.
你會發現周邊的人這樣說.
你就會覺得原來大家都這樣看.
於是你就覺得這是一個社會的標準.
或者方法.
原來這樣才叫做有錢.
我明白.
原來是這樣的.
這樣就叫做高檔.
在過程當中.
說一個很實際的經歷.
我小時候也有帶他們去玩具反斗城.
現在很多人都不知道是什麼.
知道什麼叫玩具反斗城嗎.
是不是已經結業了.

$^{441}$這些爸爸不要帶他們去.
帶他們去玩具反斗城.
玩具反斗城在海運.
在海運完結後.
跟他們走出來的時候.
我還有印象.
走出來就是那些名錶店.
很多名錶店.
我走過.
我兩個兒子看到那些錶.
哇 那些錶這麼貴.
我說是啊.
他看一看.
爸爸你怎麼沒有帶錶.
他說爸爸你怎麼沒有帶錶.
他說那些錶這麼貴 爸爸怎麼會有.
接著我說你覺得很漂亮嗎.
你大一點 爸爸給你一隻.
他覺得傳導人形象很窮.
你明白我的意思嗎.
他覺得傳導人形象很窮.
這些這麼貴的東西.
不應該在爸爸身上出現.
小朋友沒有特別教他.
但他看到很多東西.
耳聞目染都會成為他自己的價值觀.
其實貌取人或偏待人.
其實我們無論信主多久都好.
一個星期有六天都不是在教會的生態環境.
甚至是七天.
可能只有這兩個小時.
其他都是.
我猜你們出去.
跟楊進進狼群當中.
大部分都是世界上發生的事包圍著你.
所以世界價值觀不斷地分逃你.
甚至影響你所有的思維.
不以貌取人是我們刻意要改的.
不以貌取人是我們刻意要更新的.
這就是我們如何用上帝的視覺去看旁邊的事情.

$^{481}$你可以聽到 你可以讀經.
你可以查經.
你可以上帝的說話.
但方書裡面記載耶穌對人的關係方式.
其實就是改變我們的處事方法和視覺.
因為我們要抵擋這個世界的價值觀.
其實是要學習的.
不會自然出現.
所以雅各就是在說.
我們是在應信耶穌基督.
但為什麼你們沒有改變.
同時你仍然用這個方式去看.
但這班人是我們教會裡面.
我們教會裡面有些人是有錢.
有些人是沒有錢.
有些人是有專業.
有些人是沒有專業.
但有沒有專業也好.
上帝根本上不偏待人.
上帝也是共惹一勞.
因為大家都是在應信耶穌基督的人.
但我們為什麼不更新.
甚至雅各問.
為什麼你們要這樣做.
為什麼我去了奉獻的呢.
可以回去.
謝謝.
所以到了第五節的時候.
雅各再說一句訊息.
我親愛的弟兄們請停.
聽這個話.
如果你有機會查雅各書的時候.
其實雅各用了很多猶太人的信仰傳統.
去表達在雅各書裡面.
聽這個字.
英文叫hear.
其實對於猶太人來說.
它是帶出一個行動.
《心理記錄》四節就是.
Hear and obey.

$^{521}$你聽和做是一起做.
所以你聽的時候.
上帝其實豈不是揀選了世上窮人.
叫他們在信上付足.
並承受所應許那些愛他之人的國嗎.
你見到福音書裡面.
耶穌周遊四方行善事.
醫治各樣的病症.
耶穌出來傳道的時候.
亦將以蔡阿書身份特徵展現出來.
就是叫被老的得釋放.
這個很多限制.
很多需要被越立的人.
耶穌就展現了出來.
所以第六節說.
但是你們反倒羞辱貧窮的人.
那付足的人豈不是欺壓你們.
拉你們到公堂去嗎.
他們不是竊盡你們所敬奉的尊名嗎.
雅各其實仍然在說.
你們做的事.
和主耶穌所做的是背道而馳.
但你們說自己是在跟隨耶穌.
但為何你們和耶穌所做的事不吻合呢.
其實你們在信甚麼呢.
耶穌說.
你聽了我的話就照樣行吧.
但你聽了耶穌你沒有做.
我們基督信仰.
或者我們基督教的信仰.
由創世紀到啟示錄.
你會看到我們的上帝是一個說了.
就是明立就立.
說得出做得到的上帝.
所以你拿著原則去看科幻書的時候.
耶穌也是說得出做得到.
我們語帶相關的說話叫做.
Speech, Act.
說了一句話其實連帶一個行動.
不是說了就不用做.

$^{561}$不是說了就算了.
如果說了就算了.
就是你能說不能行就是那些快錯人.
耶穌最不喜歡這班人只說不做.
但現在雅各正在問和責備一件事.
弟兄,你聽清楚.
耶穌或上帝是揀選了貧窮人.
去接納貧窮人.
但你反而沒有接納他們.
但那班負責人其實不是對你好.
因為你會看到在《士路恆傳》開展的時候.
其實有很多人都是有名聲有地位.
但因為他的名聲和地位受影響的時候.
他將那班認信耶穌的人趕出會堂.
甚至是拉他們下監.
其實你看到問題是.
為什麼你們還要這樣做呢?.
經上記著說要愛人如己.
你們若傳授者至尊的律法才是好的.
至尊,就是最大.
由利美記說了愛人如己的教導.
在猶太人裡.
到新約在《方書記》記載.
就是馬太福音和馬可福音都說過愛人如己.
因為當中人們問耶穌律法最大的是什麼?.
就是你要盡心盡義盡愛主你的神.
其次也相仿就是愛人如己.
都說得出來.
所以愛神和愛人.
是上帝期望神與人之間的關係.
人與人之間的關係是最重要的.
愛神愛人嘛.
愛神就是神與人.
愛人就是人與人之間的關係.
無論是中面還是橫面.
一體兩面的看法.
上帝都很緊張.
所以這是至尊的律法.
到加泰書保羅和外邦人說的時候.
都是在說一個很重要的信息.

$^{601}$就是你認信了耶穌.
耶穌怎樣接待人.
你們也要怎樣接待人.
人們接待.
因著你的接待.
就認出你是跟隨耶穌的人.
所以你怎樣做.
你聽了我的話.
照樣行.
你感受到福音改變的大能.
但是今天雅各告訴你.
我們雖然是認信同一個群體.
但我們的關係不好.
因為你沒有愛你的弟兄.
你沒有愛你的會眾的關係.
所以這句話.
或者愛人如己的話.
去到司徒約翰的時候.
你看約翰一二三書.
他都帶出一個很重要的信息.
如果你不能夠愛漢德建的弟兄.
你就不能夠愛漢德建的神.
其實無論在司徒保羅.
無論在司徒約翰.
都是在說一個律法的至尊.
就是你怎樣用愛對方.
用愛別人的心去凸顯.
你是被上帝的說話改造了.
今天我也問自己.
我沒有被上帝的說話改造.
我對人對事.
我所做的事情.
我是本著什麼本質去做.
這就是建立關係.
今天我們上一個大葬.
這個大葬叫做耶穌基督.
上大葬的時候.
我們建立一個關係.
就算你離開了之前教會.
去到flowchurch這個群體的時候.

$^{641}$你仍然在這個大葬當中.
只不過有一個sub-club.
這個club叫flowchurch.
但仍然不代表你過去學的東西.
現在這裡不用學.
這裡仍然要學.
因為我們仍然是在履行耶穌基督的教導.
無論你入了小組也好.
或者還沒入小組也好.
或者你不打算入小組也好.
都不重要.
當你一天被上帝的說話更新的時候.
無論你去到哪裡也好.
你都是披戴著耶穌基督.
去服侍你周邊的人.
都是在做愛人如己的教導.
這就是建立關係.
再近一點說.
剛才說你離開了之前的教會.
或者你很久沒有回教會了.
慢慢開始接觸flowchurch.
認識了一班弟兄姊妹.
入了小組的時候.
在過程當中.
我在不同場合.
無論是infogroup也好.
港獨也好.
我都說得很白制.
我們不會神聖了小組.
也不會美化小組的關係.
因為小組就是一個讓我們.
預先學習愛人如己的關係的場景.
你想想.
現實一點.
一個小組12至18人.
來自12至18間鄉下.
不同教會.
他的師承就是這樣.
一定要天天祈禱才可以.
所以這一個星期都不祈禱.

$^{681}$就是你不勤儉.
OK.
是不是.
他會有很多不同的評論.
有很多不同的評論.
很多時候最難處理的就是那些嘴巴.
你將你覺得很重要.
很輕鬆的東西.
加在你的組員身上.
就成為個人的重擔.
在小組裡面的落差.
或者爭拗.
源於此就繼續滾傳下去.
目者就中間要lobby一下.
其實不是這樣的.
也不一定是一致的.
於是兩邊都很難做.
但不代表目者不會做.
目者仍然會做.
但我們都要調整一下.
在過程當中.
因為在小組裡面最難處理的就是那些嘴巴.
真的.
總有些人說得白一點.
我也有四組.
可能是躺在最後中槍.
OK.
但其實那些是common factor.
總有些人是嘴巴不擇言.
沒有搞錯.
死蠢你這樣都不懂.
他口頭禪習慣了這些的時候.
別人會覺得.
你沒有搞錯.
你和我很熟.
你說我死蠢.
你會發覺馬上就瞬間踏著.
他未必和他不妥.
但你會發覺組員下星期就不回來了.
或者去吃飯就不見了.

$^{721}$目者很醒目.
為什麼每次都不回來.
或者有他沒有他.
於是就要lobby一下.
是不是有些心結.
是不是有些事情要了解一下.
有些組會成熟一點.
就說一樣米養百樣人.
公司都見不少.
你會發覺.
你慢慢將小組變成公司.
你慢慢就覺得.
回小組就像上班一樣.
你會不會享受呢.
你怕不會享受.
接著就會問問組長.
問問目者.
這個星期帶什麼周會.
你知道周會的時候是怎樣嗎.
你自己選擇.
接著就說.
看周會才消費心態.
又好像不是很妥.
其實那個謾罵就沒完沒了.
又或者是慢慢就覺得.
其實小組都沒什麼.
我對小組都沒有懸念.
我和潘Sir說轉組吧.
是不是這樣.
是不是.
已經心中有些雅門.
是不是.
有一天有人問我.
潘Sir其實做傳達人開心嗎.
接著我說.
你給我多久時間回答這個問題.
我可以告訴你開心.
又可以告訴你不開心.
我說你想聽哪個答案呢.
兩個答案都是真實的.

$^{761}$開心的.
當然是我自己選擇的路.
選擇的路就不開心.
我就重操故業.
馬上買貴錶.
我告訴你做傳達人是不開心的.
不開心的原因是.
你每天都聽了很多弟兄姊妹.
有很多傷痛的故事.
你會見到很多弟兄姊妹.
在小組裡的關係是不咬弦的.
你會見到很多小組.
在困難中沒有彼此分擔.
他們回到小組好像沒有回到小組.
他們回到小組好像沒有支持.
你不得不問.
其實我是否要承擔責任呢.
我又會問自己.
其實我是否有些事情沒有做到.
或者沒有認真去處理呢.
如果你每天都聽這些的時候.
你怎會開心呢.
你明白我的意思嗎.
我不是辭職的.
不用擔心.
馬上要笑一笑.
好像不斷壓在你們身上.
但我想讓你明白一件事.
我在港島也好.
報告也說了.
祈禱會也說了.
因為現在認真做事.
就煩了.
你越是認真做事.
你就越煩.
因為你每件事都要交代.
你每件事都要尋根.
但我們仍然在說一件事.
Float Church是希望重新去整頓.
其實什麼叫牧羊關係.

$^{801}$我們既然大家都是認順耶穌基督.
大家都是知道.
好像剛才《迷羊》那首歌.
我們是耶穌尋找了我們.
尋找是因為上帝用Restless Love.
很瘋狂的愛.
去擁抱我們.
去追我們回來的時候.
我們就看到.
其實仍然有很多事可以發生.
但反過來.
也持平的說一句.
你自己有沒有改變.
你會發覺這個世界.
常常都將問題推出去.
一定是別人的問題.
一定是社會的問題.
一定是環境的問題.
但你自己有沒有問題.
這都是認真去問問自己.
你自己有沒有改變.
你自己有沒有被同化.
你自己有沒有學習愛人如己.
你的愛是怎樣學習.
推己及人.
這都是你要問自己.
我常常都會想.
有問題找我的人.
或是有跟我討論的人.
我每次都會說.
我真的多謝他們.
我不是裝假.
也不是說一些很客套的話.
我每個星期都會見不同組的人.
我和公眾知道.
我基本上不是見人就打電話.
但我每次都會說.
我都多謝他們跟我說.
因為原因是什麼.
因為他們肯跟我說.

$^{841}$他說原因我就走了.
或者說原因我就當沒了回事.
我都是想讓你知道.
我都多謝他們知道.
因為我見到太多人.
在教會裡面.
沒有說到就怎樣.
用腳投票.
既然你來到flowchurch.
這是一個新的載體.
希望裝住你的時候.
我們都希望聆聽你的故事.
這是我們莊園的關係.
這是我們在當中一起學習的關係.
會跟我講講實況.
會跟我講當中的限制難處.
因為你沒有選擇用腳來投票.
因為你在群體當中.
你仍然希望有新的關係.
在經文下的最後一段.
你會發覺.
阿哥再用一個很具體的例子.
其實律法是什麼.
為什麼還要講律法.
我們是不是不需要再守律法.
這件事要澄清.
舊約的律法.
到新約耶穌出現的時候.
耶穌都在講律法.
祂不是要廢去.
而保羅都在講律法.
律法是一個鏡子.
去反映我們有沒有循規蹈矩.
有沒有做好行為的表現.
律法是一個影義.
不是一個終極的審判.
但它讓我們明白.
阿哥在講律法的過程中.
我們有參考的.
參考就是.

$^{881}$其實你有沒有做到.
可以連續的空間.
或者讓人去了解.
其實有什麼我可以做多一點.
我需要改變.
所以很具體的.
就在講兩個很靠近的東西.
就是有些弟兄姊妹有很大的難處.
甚至生命有危險.
赤身老體.
甚至本身起居飲食都有困難的時候.
你有沒有接濟他.
你有.
你有做的.
你不就是平平安安的去.
原地穿得飽吃得暖.
仍然是聽了沒有做.
用口頭上的文案.
這個就好像是快彩色的信仰.
沒有因著耶穌基督去做.
其中.
阿哥在講一個很重要的訊息.
就是念文.
念文在耶穌的做法來說.
耶穌用了一個特別的例子.
就是在講好撒瑪利亞人.
好撒瑪利亞人.
他就是走過.
看到有需要的時候.
他願意伸出援手.
他看到那個人.
他有別於.
他不理他是什麼身份.
他就看到他被人打到半死.
如果我這一刻不救他的時候.
他赤身老體.
他就會生命烏敷.
他願意去幫助他.
他不計較他自己.
也不計較對方是什麼想法.

$^{921}$他覺得我這一刻要做.
我就做.
他就去.
但這法國利美人和祭司.
太多的籌算.
太多的心機.
最後見死不救.
他就走.
耶穌用了一個例子.
讓我們明白什麼叫念文.
念文這個字.
其實在保羅的《迦太書》裡.
《律法愛人如己》裡.
《迦太書》裡有一個很重要的.
就是聖靈果子的九個特質當中.
第五個.
是在講恩慈.
就是念文的一個行動.
就是有需要的時候.
伸出援手.
Give a hand.
這個是.
沒有計較.
是愛倫社一個很重要的表徵.
沒有以貌取人.
沒有論之排輩.
沒有排除異己.
什麼都沒有.
但他見到需要的時候.
他就做.
你會看到.
不憐恤人或者不憐憫人的.
也要受無憐憫的審判.
就是祭司和文士.
或者利美人和祭司.
但是憐憫人原是向審判跨性.
重點就是.
那個撒瑪利亞人.
其實他是被猶太人排斥的.
他是審判他成為.

$^{961}$這些走狗.
這些打亂種的人.
不是純種的人.
他是這樣的.
但你會發覺.
他不是用這個身份去看.
反而是你這樣看我.
我願意有另外一個行徑去表現出來.
十七字就是這樣.
信心若沒有行為.
就是死的.
其實亞哥問一件事.
如果你有認信耶穌基督的信心.
你有沒有因著認信改變的行為.
如果你沒有認信耶穌基督的信心而有的行為.
其實你的信心是不是真信心.
這個你真的要認真去想想.
信仰的擺位對你來說是甚麼一回事.
因為信心是一個起始點.
就是耶穌說的.
好樹結好果子.
從你的果子.
就看到你做了多少功.
你看到末世的時候.
馬太科堂第24,25章.
說末日的聲音.
是一個很詳盡的描述.
主人王耀回來的時候.
問你做了甚麼.
他就說我甚麼時候有.
你甚麼時候有.
你在我餓的時候給我食物.
在我冷的時候給我衣服穿.
在我口渴的時候給我水喝.
在我坐牢的時候探監.
一會兒回應詩的時候.
你就會聽到.
《竹墨》這首歌.
就是在說.
因信心而有的行為.

$^{1001}$是和你的信心相輔相成.
其實今天的信息沒有甚麼新的.
從你初信的時候.
你的初信栽培完.
就會叫你這樣做.
這個信息也沒有新的.
他都告訴我們一件事.
就是你信了就去做.
你學習了就嘗試更新.
從來都不會一步到位.
從來都是說一件事.
慢慢去練習.
我在我的籃球群體當中.
學習欣賞.
學習彼此協作.
學習有些事一起去經歷.
學習欣賞對方的限制.
在拍照的群體學習.
當很多人說很好的時候.
不一定用那個方式.
可以用另一個視角看看.
當很多人說這樣.
你為甚麼可以做得這麼漂亮.
原來看不出來.
小小的鏡頭.
好像平平無奇.
他又很漂亮.
還有你用角度很好.
還有你每一張相是一個故事.
就是告訴人家.
你的視角是怎樣看事情.
親愛的姐妹.
今天的信息再一次提醒我們.
一個很重要的事.
就是我們莊園的關係.
是建立在彼此互助互愛.
我們莊園的關係.
是建立在彼此聆聽.
每個人都有自己的故事.
每個人都有自己的限制.

$^{1041}$耶穌都接納我們了.
我們都嘗試接納我們組員的限制.
組員的關係.
我們一起祈禱.
天上帝.
當日亞各成為首領的時候.
帶領教會的時候.
他向眾弟兄姐妹發出一個信仰的簽文.
問的是什麼迫使他們去到這個地步.
信仰有沒有改變他們.
他們自己有沒有更新.
這個都是很重要的.
今天我們同樣都在問我們身處的關係.
身處的環境.
我們有沒有做好自己的本份.
求聖靈去教導我們.
當提醒我們需要聆聽.
需要去更新.
需要勒住自己舌頭的時候.
求主你教導我們.
多謝你聽我們祈禱.
奉耶穌的名求.
阿們.
\newpage



\section{馬可福音 9:1-13-20231021}
\label{sec:3YrDRTxYY2U}
\textbf{【網上崇拜】上莊猶如「勿說是推理」般|馬可福音9\_1-13|20231021 [3YrDRTxYY2U]}
\newline
\newline
連結: \href{https://youtube.com/watch?v=3YrDRTxYY2U}{\texttt{ https://youtube.com/watch?v=3YrDRTxYY2U}} ~~~~ 語音日期: 2023-10-21 
\newline
\newline
\hyperref[sec:CYdq0eOXFpM]{\small{< < < PREV SERMON < < <}}
~
\hyperref[sec:index_chronic]{\small{[返順時目]}}
~
\hyperref[sec:index_scriptual]{\small{[返順卷目]}}
~
\hyperref[sec:BDg16RM34JI]{\small{> > > NEXT SERMON > > >}}
\newline
\newline
馬可福音 9:1-13-20231021
\newline
\begin{longtable}{cl}
\hline
\hline
章節 & 經文 (和合本修訂版)\\
\hline
9:1 & \begin{tabularx}{0.7\textwidth}{X} 耶穌又對他們說:「我實在告訴你們,站在這裡的,有人在沒經歷死亡以前,必定看見神的國帶著能力臨到。」 \end{tabularx} \\ \\ \relax
9:2 & \begin{tabularx}{0.7\textwidth}{X} 過了六天,耶穌帶著彼得、雅各、約翰,領他們悄悄地上了高山。他在他們面前變了形像, \end{tabularx} \\ \\ \relax
9:3 & \begin{tabularx}{0.7\textwidth}{X} 衣服放光,極其潔白,地上漂布的人沒有一個能漂得那樣白。 \end{tabularx} \\ \\ \relax
9:4 & \begin{tabularx}{0.7\textwidth}{X} 有以利亞和摩西向他們顯現,並且與耶穌說話。 \end{tabularx} \\ \\ \relax
9:5 & \begin{tabularx}{0.7\textwidth}{X} 彼得對耶穌說:「拉比,我們在這裡真好!我們來搭三座棚,一座為你,一座為摩西,一座為以利亞。」 \end{tabularx} \\ \\ \relax
9:6 & \begin{tabularx}{0.7\textwidth}{X} 彼得不知道說甚麼才好,因為他們很害怕。 \end{tabularx} \\ \\ \relax
9:7 & \begin{tabularx}{0.7\textwidth}{X} 有一朵雲彩來遮蓋他們,又有聲音從雲彩裡出來,說:「這是我的愛子,你們要聽從他!」 \end{tabularx} \\ \\ \relax
9:8 & \begin{tabularx}{0.7\textwidth}{X} 門徒連忙向周圍觀看,不再看見任何人,只見耶穌同他們在一起。 \end{tabularx} \\ \\ \relax
9:9 & \begin{tabularx}{0.7\textwidth}{X} 下山的時候,耶穌囑咐他們說:「人子還沒有從死人中復活,你們不要把所看到的告訴人。」 \end{tabularx} \\ \\ \relax
9:10 & \begin{tabularx}{0.7\textwidth}{X} 門徒將這話存記在心,彼此議論「從死人中復活」是甚麼意思。 \end{tabularx} \\ \\ \relax
9:11 & \begin{tabularx}{0.7\textwidth}{X} 他們就問耶穌:「文士為甚麼說以利亞必須先來?」 \end{tabularx} \\ \\ \relax
9:12 & \begin{tabularx}{0.7\textwidth}{X} 耶穌說:「以利亞的確先來復興萬事。經上不是指著人子說,他要受許多的苦和被人輕慢嗎? \end{tabularx} \\ \\ \relax
9:13 & \begin{tabularx}{0.7\textwidth}{X} 我告訴你們,以利亞已經來了,他們任意待他,正如經上指著他說的。」 \end{tabularx} \\ \\ \relax
9:14 & \begin{tabularx}{0.7\textwidth}{X} 他們到了門徒那裡,看見有一大群人圍著他們,又有文士和他們辯論。 \end{tabularx} \\ \\ \relax
9:15 & \begin{tabularx}{0.7\textwidth}{X} 眾人一見耶穌,都很驚奇,就跑上去向他問安。 \end{tabularx} \\ \\ \relax
9:16 & \begin{tabularx}{0.7\textwidth}{X} 耶穌問他們:「你們和他們辯論甚麼?」 \end{tabularx} \\ \\ \relax
9:17 & \begin{tabularx}{0.7\textwidth}{X} 眾人中的一個回答:「老師,我帶了我的兒子到你這裡來,他被啞巴的靈附著。 \end{tabularx} \\ \\ \relax
9:18 & \begin{tabularx}{0.7\textwidth}{X} 無論在哪裡,那靈拿住他,把他摔倒,他就口吐白沫,牙關緊鎖,身體僵硬。我請過你的門徒把那靈趕出去,他們卻不能。」 \end{tabularx} \\ \\ \relax
9:19 & \begin{tabularx}{0.7\textwidth}{X} 耶穌回答:「唉!這不信的世代啊,我和你們在一起要到幾時呢?我忍耐你們要到幾時呢?把他帶到我這裡!」 \end{tabularx} \\ \\ \relax
9:20 & \begin{tabularx}{0.7\textwidth}{X} 他們就帶了他來。那靈一見耶穌,就使他重重地抽風,倒在地上,翻來覆去,口吐白沫。 \end{tabularx} \\ \\ \relax
9:21 & \begin{tabularx}{0.7\textwidth}{X} 耶穌問他父親:「他得這病有多久了呢?」父親說:「從小的時候。 \end{tabularx} \\ \\ \relax
9:22 & \begin{tabularx}{0.7\textwidth}{X} 那靈屢次把他扔在火裡、水裡,要治死他。你若能做甚麼,求你憐憫我們,幫助我們。」 \end{tabularx} \\ \\ \relax
9:23 & \begin{tabularx}{0.7\textwidth}{X} 耶穌對他說:「『你若能』,在信的人,凡事都能。」 \end{tabularx} \\ \\ \relax
9:24 & \begin{tabularx}{0.7\textwidth}{X} 孩子的父親立刻喊著說:「我信;求你幫助我的不信!」 \end{tabularx} \\ \\ \relax
9:25 & \begin{tabularx}{0.7\textwidth}{X} 耶穌看見眾人都跑上來,就斥責那污靈說:「你這聾啞的靈,我命令你從他裡頭出來,再不要進去!」 \end{tabularx} \\ \\ \relax
9:26 & \begin{tabularx}{0.7\textwidth}{X} 那靈大喊一聲,使孩子猛烈地抽了一陣風,就出來了。孩子好像死了一般,以致眾人多半說:「他死了。」 \end{tabularx} \\ \\ \relax
9:27 & \begin{tabularx}{0.7\textwidth}{X} 但耶穌拉著他的手,扶他起來,他就站起來了。 \end{tabularx} \\ \\ \relax
9:28 & \begin{tabularx}{0.7\textwidth}{X} 耶穌進了屋子,門徒就私下問他:「我們為甚麼不能趕出那靈呢?」 \end{tabularx} \\ \\ \relax
9:29 & \begin{tabularx}{0.7\textwidth}{X} 耶穌對他們說:「非用禱告,這一類的邪靈總趕不出來。」 \end{tabularx} \\ \\ \relax
9:30 & \begin{tabularx}{0.7\textwidth}{X} 他們離開那地方,經過加利利;耶穌不願意人知道, \end{tabularx} \\ \\ \relax
9:31 & \begin{tabularx}{0.7\textwidth}{X} 因為他正教導門徒說:「人子將要被交在人手裡,他們要殺害他;被殺以後,三天後他要復活。」 \end{tabularx} \\ \\ \relax
9:32 & \begin{tabularx}{0.7\textwidth}{X} 門徒卻不明白這話,又不敢問他。 \end{tabularx} \\ \\ \relax
9:33 & \begin{tabularx}{0.7\textwidth}{X} 他們來到迦百農。耶穌在屋裡問門徒說:「你們在路上議論的是甚麼?」 \end{tabularx} \\ \\ \relax
9:34 & \begin{tabularx}{0.7\textwidth}{X} 門徒不作聲,因為他們在路上彼此爭論誰最大。 \end{tabularx} \\ \\ \relax
9:35 & \begin{tabularx}{0.7\textwidth}{X} 耶穌坐下,叫十二個使徒來,說:「若有人願意為首,他要作眾人之後,作眾人的用人。」 \end{tabularx} \\ \\ \relax
9:36 & \begin{tabularx}{0.7\textwidth}{X} 於是耶穌領一個小孩過來,讓他站在門徒當中,又抱起他來,對他們說: \end{tabularx} \\ \\ \relax
9:37 & \begin{tabularx}{0.7\textwidth}{X} 「凡為我的名接納一個像這小孩子的,就是接納我;凡接納我的,不是接納我,而是接納那差我來的。」 \end{tabularx} \\ \\ \relax
9:38 & \begin{tabularx}{0.7\textwidth}{X} 約翰對耶穌說:「老師,我們看見一個人奉你的名趕鬼,我們就阻止他,因為他不跟從我們。」 \end{tabularx} \\ \\ \relax
9:39 & \begin{tabularx}{0.7\textwidth}{X} 耶穌說:「不要阻止他,因為沒有人奉我的名行異能,反倒輕易毀謗我。 \end{tabularx} \\ \\ \relax
9:40 & \begin{tabularx}{0.7\textwidth}{X} 不抵擋我們的,就是幫助我們的。 \end{tabularx} \\ \\ \relax
9:41 & \begin{tabularx}{0.7\textwidth}{X} 凡因你們是屬基督,給你們一杯水喝的,我實在告訴你們,他一定會得到賞賜。」 \end{tabularx} \\ \\ \relax
9:42 & \begin{tabularx}{0.7\textwidth}{X} 「凡使這些信我的小子中的一個跌倒的,倒不如把大磨石拴在這人的頸項上,扔在海裡。 \end{tabularx} \\ \\ \relax
9:43 & \begin{tabularx}{0.7\textwidth}{X} 如果你一隻手使你跌倒,就把它砍下來; \end{tabularx} \\ \\ \relax
9:44 & \begin{tabularx}{0.7\textwidth}{X} 你缺一隻手進入永生,比有兩隻手落到地獄,入那不滅的火裡去還好。 \end{tabularx} \\ \\ \relax
9:45 & \begin{tabularx}{0.7\textwidth}{X} 如果你一隻腳使你跌倒,就把它砍下來; \end{tabularx} \\ \\ \relax
9:46 & \begin{tabularx}{0.7\textwidth}{X} 你瘸腿進入永生,比有兩隻腳被扔進地獄裡還好。 \end{tabularx} \\ \\ \relax
9:47 & \begin{tabularx}{0.7\textwidth}{X} 如果你一隻眼使你跌倒,就去掉它;你只有一隻眼進入神的國,比有兩隻眼被扔進地獄裡還好。 \end{tabularx} \\ \\ \relax
9:48 & \begin{tabularx}{0.7\textwidth}{X} 在那裡,蟲是不死的,火是不滅的。 \end{tabularx} \\ \\ \relax
9:49 & \begin{tabularx}{0.7\textwidth}{X} 因為每個人必被火像鹽一般醃起來。凡祭物必用鹽醃。 \end{tabularx} \\ \\ \relax
9:50 & \begin{tabularx}{0.7\textwidth}{X} 鹽本是好的,若失了鹹味,你們怎能用它調味呢?你們中間要有鹽,彼此和睦。」 \end{tabularx} \\ \\
[1ex]
\hline
\hline
\end{longtable}
$^{1}$(台語)早安晚安,有多少人先看了《密雪時推理》?有人擔心我今晚會劇透,所以這個星期就看了.其實我不是想說電影,雖然電影很好看,但電視劇的十多集才是最好看的..
今天不太快,先說經文.今天會再說transfiguration,再說一次登山變象..
但今次的進度不是上次的進度,爆炸頭那個就是九能正,我稍後會說..
我回顧上課我們說路加的transfiguration,登山變象的時候,我大約想表達的是這樣..
很多時候我們要為上帝做一些事的時候,其實不是想說我們知道很多東西,我們懂得很多東西..
很多時候我們往往為上帝做事的時候,其實我們都很無知,不知道自己在做什麼的時候,這個大體上是我們的真相..
所以當我們要說上莊這個課題的時候,我們說要為上帝做一些事,不要只是坐在這裡的時候..
請我記得,請我們都記得,其實我們好像門徒一樣,我們不太明白上帝在整個計劃裡面在做什麼,不過上帝都喜歡我們為祂擺上..
其實我們在時空裡充滿著自己的幻想,很多FF,但很多FF裡面完結之後,或者真正看到上帝有心意做的時候,其實都是在很後期的時候,我們才醒覺其實我們所做的事情,所想的事情都很無知..
這個起碼是我想跟自己說的說法..
今天我們會說馬可,因為馬可跟路加的寫作不一樣,馬可的寫法跟路加的寫法有些不同,同樣地是一個transfiguration,登山變象..
但是我希望今天從上下文的角度去再解釋一次登山變象..
所以馬可的九章一節我們叫做上文,他有一句話,耶穌對他們說,.
我實在告訴你,站在這裡的人,有人未嘗死未已前,就能見到神國,但要有能力臨到..
這裡放九章一節其實有很多討論,因為其實這句話跟下面二至八節,整個登山變象的故事我不讀了,大家都明白,.
站在塔比爾尼亞,站在塔比摩西,站在塔比耶穌,這個故事上個月已經說了,我都明白了..
但是馬可有別於路加,除了加了九章一節之外,他加了九章的九至十三節,九至十三節大約內容有五節聖經,有這五節聖經,待會我會再讀..
所以我們今天入手點是想從九章一節和九章九至十三節,這兩個跟路加沒有記載關於登山變象的時候,.
但是馬可和馬太刻意加了這些聖經上去,他想補充一下整個登山變象說什麼,我們希望今天從這個入手點去看登山變象..
我先看上文,九章一節,多給七八分鐘,複雜的事情都說完了,大家有一點耐性就行了..
耶穌對他們說,我實在告訴你們,站在這裡的人,有人未嘗死未已先就得見神國,大有能力臨到..
未嘗死未已先這個字眼其實在新約聖經很少出現,這個highlight了那個字,未嘗死未已先,就是你未死過之前..
當然我們現在廣東話會容易明白一點,但是二千年前很難明白是什麼意思,不過這個字在新約很少出現..
但是,對不起,這個,再看這個,但是如果你看一卷叫以斯拉族篇下卷,不知道是什麼,叫2nd,或者叫4th Ezra,.
就是兩約文獻裡面的一個書卷,就是兩個時期,就是耶穌出生前,就是巴比倫滅亡之後那段時期,.
寫的東西,大約是主前一世紀,大約我date他的話,4th Ezra這本書有說過未嘗死未已,所以你看到highlight是什麼呢?.
他說,他們要看見那些被接去的人,就是那些自他們出生之來未嘗過死亡的人,地上的居民要改變轉向別的意念,.
因為邪惡終必滅亡,奸計終必自釋,信心要興旺,必扭壞的要衰敗,而有長久效果的真理要顯現出來..
如果你看這裡的話,我們很簡單看,不用看完,看一點刺經說的,未嘗死未已是什麼意思呢?.
就是說,假設有一天,我見到我爺爺和我奶奶,突然間出現,你可以想像,如果你見到你爺爺和奶奶出現,.
在登山變相,通常你會求什麼呢?就是求落彩的六個number是什麼,你說,將來我的運程如何,.
是不是一生人都很順順利利,我應該人生做些什麼,大約你會想那些幾榮耀的東西,幾著數的東西,幾得勝的東西,.
好像這裡的經文一樣,它都是說將來邪惡就會完,信心的人就要得勝,大約你看著你爺爺和奶奶的時候,你就祈求祝福你家宅平安,.
就像門徒看著耶穌,見到伊利亞和摩西的時候,他大約都以為耶穌一定行,這個是未嘗死未已先的時候,.
你可以在兩約文獻理解,我未死,但現在很慘,我未死的時候就會得勝,大約觀念是這樣,但是耶穌要重新define什麼叫未嘗死未,.
所以夏文,這個就是兩約文獻的觀念,不用說完,我們說夏利,九節十三節是去解釋第一節,或者解釋整個登山變相,.
他說,下山的時候,耶穌吩咐他們說,人子說沒有重生復活,你們不要將所見告訴別人,.
這句無端端耶穌見完,當我們見完爺爺奶奶之後,突然間他說了一句很難明白的東西,叫死裡復活,.
那些門徒就將這句話傳在心中,彼此議論,他說,其實從死裡復活是什麼意思呢,他當然不明白,不是中立財嗎,不是我們打贏羅馬人嗎,.
不是我們骨林城下,進了耶路撒冷將羅馬人趕走,從此康復整個耶路撒冷,有空就得勝了,門徒大體上這樣想,.

$^{41}$但他不明白,耶穌說什麼叫死裡復活,我們不知道,什麼意思,不是要贏嗎,為什麼要死裡復活呢,未嘗死未應該是未死就贏,什麼叫死裡復活呢,.
所以門徒不明白,議論,耶穌突然間好像不理那些無知的門徒,明明他們議論的時候應該跟他們解釋,.
我會死,死了之後就復活,這樣才行,應該是這樣說,但那些門徒繼續問一件事,門徒問耶穌說,為什麼要叫以利亞必須先來,.
因為如果你知道背景的話,在馬拉基書說末世的時候,會有一個以利亞出現,所以那些以利亞被形容是,好像打人百力先知,很厲害,那些德性榮譽的說法,.
所以門徒就馬上問,其實耶穌說,你是不是很厲害,那個以利亞會來,耶穌回答他,姑利亞固然先來要復興萬事,經常不是指著人子說嗎,她要受很多的苦,被人輕慢,.
門徒不斷想那件事,耶穌也看到末世的以利亞,很厲害,你試一下看到很厲害的人,死了之後,你看到馬丁路德,還是看到約翰加爾文,還是想不知見誰,見耶穌好像比較好,.
就是不要這樣搞,你想,你見到,哇,厲害,但耶穌很掃興,他說人子要受很多的苦,被人輕慢,跟門徒想的完全不一樣,.
他十三節才是那個重中之重,告訴你們,以利亞已經來了,以利亞來了應該是說契約翰,但契約翰是誰,是被人斬頭的,死了的,任意待他,就好像經常所說的,.
登山變象,看到很厲害的東西,我們應該會想什麼,我們應該會想,希望耶穌來搞定,或者希望從此以後我的信利,侍奉上會順利一點,.
但你看,耶穌定義過,或者定義過,原來他來的時候,就好像以利亞一樣,是要受苦和犧牲,說完這句就完,這一部分完,你叫完,複雜的東西,.
路加上次說談論去世的事,如果你還記得,路加記載的在登山變象的時候,是耶穌和摩西和以利亞談論去世的事,Exodus,我們以為Exodus是紅海滅埃及人,雲柱火柱瑪那,.
但其實Exodus的意思是什麼,無論在詩篇裡重用出埃及記的觀念,或者以塞亞重用出埃及記的觀念,大部分都是在說不理想的東西,.
就好像真正Exodus是什麼意思,就是摩西在抗疫的路上,四十年來,看著百姓的時候,經常都在挑戰他,拿香爐來看上帝,火焗在誰那裡,不滿意他,.
最終他搞不定,他要敲一敲石頭,兩下,他就進不了迦南,以利亞以為他打贏完巴黎先知之後很厲害很威,.
誰知道一個女人叫耶西別的人,一聲令下,他就馬上要逃跑四十天,還要尋死,教會有一個執事跟你說一句話,.
你明天就死了,你不會走四十天,然後說主我不侍奉了,這個世界只有我一個為你忠心,我死了,這個才是Exodus,經常記著以利亞和摩西,.
耶穌在Transfiguration裡面看著以利亞和摩西的時候,他談論去世的時候,談論Exodus的時候,原來服侍上帝的一群人其實充滿著困難苦難,.
耶穌看著以利亞和摩西也是這麼辛苦,自己最終要坐十字架的時候,而那群蛋散的門徒全部都不明白,為什麼耶穌要釘十字架,.
還要說你不要這樣,你記不記得彼得說什麼,你不要這樣,耶穌馬上罵他魔鬼撒旦退後了,你試一下做一些事情,.
全部人都不明白你,你覺得多慘,但你正在受苦,你以為你做的事情應該為上帝做,但其實你受很多苦難,你問上帝為什麼要我做一些你想要我做的事情,.
堅持下去的時候,所有事情都這麼痛苦,所以他們的團契和摩西和以利亞耶穌的Transfiguration的團契,其實就是在說侍奉信耶穌的路上充滿著磨難與挑戰和痛苦,.
好了,說完複雜的事情,我們說開心的事情,這個題目叫上莊,我整個人馬上輕鬆了,你問我上莊,剛才外面不是放了一些牌叫你參加一些Club,.
你坦白說,很年輕的時候,我們不要計算,你大約在教會十年八年,教會還有什麼新的事情做,你想一下,你在教會經過十年八年洗禮完之後,你有什麼浮面沒見過的,還叫你參與一些群組,你以為他真的抽你把柜,.
放心,我等一下會說相反的事情,大家放心,通常要說這樣才相反,你叫我走完,我看看那些童工在那裡,弄一個牌來裝一下,QR Code一下,看看怎樣,.
上去做什麼,有什麼是你沒見過,沒試過,你羽毛球沒打過,我打過,你要什麼生態靈州,你沒試過去鴨州嗎,你去過鴨州,到處看過鴨的州,你生態下一定有,.
有什麼新的事情,很刺激的事情,令你上窗,各位電影姐妹,我想表達的是,要看《密室時推理》,.
我剛才要他彈《密室時推理》的主題曲,看他能不能彈到,他剛剛聽的,你有沒有,這個真的很好聽,謝謝,.
你繼續彈,你真的只有四個bar,謝謝你,其實這個是九能正和萊卡,九能正和萊卡其實是要看電視劇的,富士電視台上年2022年推出這個劇,.
你知道富士電視台是很老牌的電視台,已經沒有人看他,但是靠這套戲翻生,我明天去日本,沒有,因為10月他做第二季,現在沒得看,要去日本才能看,.
我希望能看到,這條愛情線是很浪漫的,但那些浪漫不是韓劇那些九月臨頭,灑九血那些feel,對不起,我是日劇的,謝謝,劇不吸引我,.
我不劇透,我知被人罵,但這條愛情線很值得看,第十集,你看到第十一二集就完,第十集我看了三次,一次廣東話,一次日語版,第三次我還要是什麼人都不在我身邊,.
對著大螢幕看,我要每一個表情每一個細節都要記得,這兩個人的故事是你想像不到的,令人很驚訝,我們不說,但我想說一件事,.
這兩個人的相識,他們沒有走到一起,他們一起去做一些事,像上妝一樣去做一些事,是沒有目的性的,沒有功能性的,純粹是他們兩個人在生命裡遇到很多苦,.
那些苦是他們各自的經歷裡面面對一些很難的事,是那些難的事,把這兩個人走在一起去做一些事情,推理劇,那些案是怎樣處理解決不重要,.
還是他們兩個人很純粹地,他們有些事想做,但他們心中有很多難處,很多苦難,很多難的事,這兩個人從來沒有吃過燒肉,日本的燒肉,第十集說他們兩個一起吃燒肉,.
這不是劇透嗎?我想說的故事是,為甚麼有人沒有吃過東西,他們沒有去吃過燒肉,.
那些純粹是因為苦難而造成的,不要再說了,我可以再說一個,但應該第二季會再說多一點,.
他們之間的默契是因為背後來自他們一些苦難的背景,如果你問我,要鼓勵大家出去一會兒做一個QR Code上庄,.
我不會期望在庄做一些厲害的事,希望令那個級別越來越多人,或者這個級別有甚麼得著,有甚麼厲害的事,我不會這樣跟你說,.
我想跟你說的是,你試試找一個來卡,你試試找一個跟你一起做的事的時候,不功能性,不目的性,不是要自我實現,不是要呈現自己有多厲害,.

$^{81}$有多能幹,不要搞這些事,試試在這個級別裡面遇到一些人,他跟你堅持的信仰和堅持真理是類似的,唯有這些人才會很純粹地走在一起,.
和走在一起去堅持真理,最後再看我老,九能靜和我老這對角色是很特別的,九能靜除了有來卡這個女生之外,她有另一個很BL的感覺,.
我當時看了以為這套戲是講BL的,結果不是,但是自他們兩個相遇之後,他們在巴士劫劫之後,不是劇透,沒有劇透,很容易看的,.
第二集就講了,沒有劇透可以,完全不用擔心這些,一看就知道是那些東西,他們認識了,認識之後很奇怪,他們在劇集裡面不常見面,.
沒有什麼聊天,個個都在工作,但很奇怪,我老經常想念九能靜,九能靜經常想念我老,她做了戲突然間說,阿靜今天會做些什麼呢?如果她在這裡,她會怎樣呢?.
有一天我做夢,我希望發現做夢有一個人會來跟你說一些話,我那個夢是這樣做的,有一個人走來,他跟我說,我現在侍奉很辛苦,我現在侍奉很慘,我侍奉很失望,.
他說他不知道跟誰說,他找我說,那個夢境裡面,我知道要約這個人,我特別為他祈禱,在祈禱裡面,我說,神啊,這個人很不容易,經歷了很多很難的事情,.
我知道他會灰心,會沮喪,會失望,我大體上會知道,但我說,天父你可不可以幫他走下去,在這個夢境完結的時候,我記得我跟他說這番話的時候,.
我是哭著跟他說,我說,兄弟你可以熬下去的,上帝會祝福你的,我醒來之後,我記得我自己的禱告的內容,那個禱告的內容其實不是跟他說的,是跟自己說的..
這個圈子很艱難,你要走的事情很不容易,有些人在這個圈子謀生是一個妙足,是一個說一些大家喜歡聽的事情,.
如果你要說一些不容易的事情,你會往後付出很多代價的..
最近華福的聚會,如果你有看一些報導的話,你會知道華福有做了什麼,是那個新的年輕的負責人董家華牧師,他請了一班年輕的人去,.
如果你知道的話,我不敢得罪很多人,這句話不要說,你知道很多大佬要去的,你明白我說什麼吧?.
你知道他最後的言直,當然要按照他的位置去做,他說他這次做法不是這樣,所有人來吃最後一頓飯,抽籤,你可能不太容易聽到這些,.
對我來說,我很echo,好像我奴和九能政,你echo一個人,他可以不背負很多你必須要做的大佬文化的事情,做一些正常的事情,.
而我們對那些很奇怪的事情,我們習慣了就繼續做,但當你不在那個古靈精怪的裡面,.
你站出來遠一點,不走進去的時候,你要付上多少代價?我不知道你在信仰的視縫裡面,你會相信你堅持著什麼?.
每個人都不一樣,有些人做了一些事情,你會echo,對,這個人做得很好,我很想他這樣做,我很想他做到這樣,.
我很想有這個勇氣去做那些事情,我上了耶穌之後,我會堅持看到這些事情發生,.
不是說你做的事情要很豐功偉績,很偉大,很適合我去做,我就是很想在那裡找到一些人,.
他知道你視縫裡面,信耶穌的路上有很多艱難的事情,他明白你,你會有一個這樣的人,.
我們現在那些弟兄姊妹的愛,很膚淺,吃吃好吃的,玩玩,我們就覺得,呀,就是這樣,.
可不可以在那裡找到一些人,他跟你在信仰的路上相似,每個人對信仰的堅持,他的點不一樣,.
如果你問我的話,我經常覺得這個世界上有很多亂說話的人,經常亂說話,而那些亂說話的人才是最受歡迎的,.
你知道嗎?這些是我痛恨非常的事情,但是你看著很多人都喜歡聽亂說話的人說話的時候,.
有多少人會明白這些痛苦?你的痛苦跟我痛苦是不同的,我經常會問那些亂說話的人受歡迎,.
不如我做亂說話的人,哇,這樣就好了,自甘墮落就好了,耶!.
但是有些人是不亂說話的時候,他講的那些東西,那些人永遠成為我的鼓勵,還可以堅持下去多一點點的,.
我不知道你正在想什麼,如果你出道級的時候,你試試藉著那個時俸的自然,近一點,.
你試試在那個級別裡面找一些人,跟你堅持同一樣的東西,不是要做大那件事,不是要做偉大那件事,不是要豐功偉績,.
沒有啦,你跟我經歷過很多豐功偉績之後,其實都是很虛假的,在教會生活裡面,那個已經不是重點了,.
面對著香港現在越來越古靈精怪的社會,能夠找到人,明白到你內心,很不同..
我不是劇透,劇透在一個做總結,你知道第十集,.
內卡和九能正,令我最感動的一幕是什麼?.
是他吃燒肉,他吃燒肉的時候,內卡伸了手出來,九能正發現他的手有些問題,.
發現之後,九能正是戴著頸巾的,他整套戲都是戴著很闊的頸巾,第十集裡面有一幕是,.
九能正弄開了頸巾,他整套戲都是冬天拍的,他不可以夏天拍的,不然就戴不到頸巾,.
他給他胸前一疊很大的傷口給內卡看,.
頂尖在侍奉上裝的裡面,你試一下告訴對方,在你這個侍奉生涯信仰歷程裡面,最難的是什麼?.
而你找到一個人半個,可以跟你一起走這條路,原來在現實社會裡面,.

$^{121}$我真的認識一個人,可以跟我講一番這樣的說話,大家都覺得香港教會很難搞,不想搞,.
在我們這個圈子裡面,個個都是去去去去去,而一個正常人跟你說,這個圈子很難搞,不想搞,.
你偶爾聽到這些人這樣跟你說,你的內心不是充滿著喜悅,因為真的很難搞,.
是你知道有人明白你,他可以這樣找你跟你說話,.
沿著這個上裝這個課題,待會你出去找人的時候,你試一下想一下,.
祈禱一下,可不可以上帝讓我遇到一個這樣的人,讓我在香港這個很艱難的路上,.
跟我走到,好像萊卡的手的傷,好像九能症胸前那塊很噁心的傷口,讓兩個人很純粹地一起,.
做最簡單最純粹的事,求天父在這座教會,在這麼艱難的香港教會生活裡面,.
我們以這個基礎成為我們走下去的原因,我一起禱告,.
天父多謝你今天給我們這個空間和時間,.
我們真的覺得很難,在信仰這條路上,我們有時候想做一些事,堅持一些事,.
我們很快就被打垮,我們很快就覺得沒有信心再走下去,.
最後我只希望你這一刻給我們重現一點點,很少很少的盼望,.
真的還有些人不一樣,就好像耶穌在登山變相,看著摩西,看著伊利亞,.
他們立刻怎樣經歷,成為耶穌再走下去的力量,.
最後我祈禱的是,當我們對很多事失望,難過,傷心的時候,.
最後讓我們相信,還有人和你堅持同一樣的事,.
多謝天父,求你繼續和我們心裡說話,.
我們這樣祈禱,奉耶穌你寶貴命,求你,阿們..
我想先聽完這首歌再走,我選了一首《Goodness of God》,.
《Goodness of God》這首歌是這兩三個月我經常循序漫步的歌,.
在美國,不,雖然在英國探訪的時候,在一間很小的當地教堂,.
一個印度的家庭,我不知道他們是否認識這首歌,這幾年可能很生氣,.
我好像是第一次聽這首歌,我以前唱過,.
但上次在英國,很少人,只有幾十人,全部是阿公,阿婆,阿伯,.
只有那個印度家庭是年輕的,他們有大師哥做主席,.
其實什麼都做,你知道沒有人做事的,.
他唱這首歌,這首歌是,我上星期看的觀看次數是118萬,.
這個星期看已經變成11900萬,多了100萬人一個星期,.
如果你看下面的留言,很多人因為這首歌,.
不知道為什麼信印度教會信耶穌,信佛教會信耶穌,.
什麼教都不知道為什麼會信耶穌,什麼人以前信過耶穌已經離棄了,.
突然間都信回耶穌的這首歌,.
我想唱的時候,這裡會說goodness和faithfulness,.
其實他在用詩篇23篇,詩篇23篇他說什麼?.
Tovahessal,中文是因為與此愛,但這個發音不好,.
追著你,很信實地追著你,.
其實這首歌令我感動的是,很多時候很失望,.
很不知道要怎麼做下去的時候,.
你經常說上帝的goodness和faithfulness,.

$^{161}$running after you,.
怎麼running after me啊,沒有啊,.
所有堅持的人,看著很多沒有的,沒有running after me的時候,.
就要很低能地,很阿Q地再唱,.
就是真的會running after me,.
這首歌五年前左右,能夠影響這麼多人,.
我相信今天也在影響著你和我,.
希望藉著這個靜餐,給一點點,可以嗎?.
這首歌,對不起,我想說goodness十個字,.
不好意思,我們沒有默契,.
剛才說得這麼深情,突然間都走了,.
我希望這首歌,一會兒靜餐的時候,.
當你領著餅和杯,可能這一刻你很絕望,.
你覺得上帝在虧待你,.
但我希望你也覺得,.
上帝的goodness和faithfulness,.
是繼續running after us,.
\newpage



\section{羅馬書 12:17-18-20231028}
\label{sec:BDg16RM34JI}
\textbf{【網上崇拜】正常發揮吧!|羅馬書12\_17-18|20231028 [BDg16RM34JI]}
\newline
\newline
連結: \href{https://youtube.com/watch?v=BDg16RM34JI}{\texttt{ https://youtube.com/watch?v=BDg16RM34JI}} ~~~~ 語音日期: 2023-10-28 
\newline
\newline
\hyperref[sec:3YrDRTxYY2U]{\small{< < < PREV SERMON < < <}}
~
\hyperref[sec:index_chronic]{\small{[返順時目]}}
~
\hyperref[sec:index_scriptual]{\small{[返順卷目]}}
~
\hyperref[sec:JKdFzjAsLZY]{\small{> > > NEXT SERMON > > >}}
\newline
\newline
羅馬書 12:17-18-20231028
\newline
\begin{longtable}{cl}
\hline
\hline
章節 & 經文 (和合本修訂版)\\
\hline
12:17 & \begin{tabularx}{0.7\textwidth}{X} 不要以惡報惡,眾人以為美的事要留心去做。 \end{tabularx} \\ \\ \relax
12:18 & \begin{tabularx}{0.7\textwidth}{X} 若是可行,總要盡力與眾人和睦。 \end{tabularx} \\ \\ \relax
12:19 & \begin{tabularx}{0.7\textwidth}{X} 各位親愛的,不要自己伸冤,寧可給主的憤怒留地步,因為經上記著:「主說:『伸冤在我,我必報應。』」 \end{tabularx} \\ \\ \relax
12:20 & \begin{tabularx}{0.7\textwidth}{X} 不但如此,「你的仇敵若餓了,就給他吃;若渴了,就給他喝。因為你這樣做,就是把炭火堆在他的頭上。」 \end{tabularx} \\ \\ \relax
12:21 & \begin{tabularx}{0.7\textwidth}{X} 不要被惡所勝,反要以善勝惡。 \end{tabularx} \\ \\
[1ex]
\hline
\hline
\end{longtable}
$^{1}$今日有個特別ge 活動.
歡迎大家黎到第1990屆基督徒學會諮詢大會.
歡迎你地各位出席呢個大會.
我地ge 莊園真係好需要大家ge 支持.
我地今屆ge 成員包括有ABCD.
之後大家就再慢慢認識.
佢地ge 政綱係D 乜ye 呢.
佢地今年ge 政綱好特別.
就係不要以惡報惡.
眾人以為美ge 事要留心去做.
我講jor 咁多.
不如而家就請學會成員解說一下你地政綱.
交比你地.
好得我地.
好好testing.
解說一下你地ge 政綱.
唔好意思呀.
我覺得呢唔使講架啦.
都講jor 二千年.
不如講下D 新ye .
我就覺得新同舊都唔係問題.
但係你地ge 學會一時一樣咁樣.
一時又話要尊敬神.
一時又話迎合人.
好矛盾.
你地覺得點.
其實我係想join你地個級好耐.
上年gwo 個迎新禮物.
神蹟大放送.
好鬼正好吸引.
今年可唔可以再送.
不過我唔讀經可唔可以.
唔得架喎.
不如咁啦.
等陣先啦.
你地可唔可以講下有D 咩新ye .
舊ge 都唔明講咩新.
其實讀經真係唔會有人join.
你地可唔可以考慮.
淨係送神蹟大放送.

$^{41}$但係唔讀經.
咁啦真係唔太吸引.
即係場面有D 需要受控制.
各位新生或者老鬼.
讀經真係唔得.
可唔可以淨係送神蹟大放送.
神蹟大放送.
都係我地ge 錯.
都係我地D 老鬼上一屆解釋得唔清楚.
讓到你地受苦.
不如你地都係落台啦.
即係抖一抖先.
佢地真係今日都好勞苦.
真係我解釋得唔好.
所以比D 時間我.
稍安毋躁.
頭先gwo D ge 朋友.
好投入呀佢地.
交返比我去解說下佢地今年ge 政綱.
其實佢地做得好好.
不過呢真係太激烈.
好 咁呢 完架啦.
劇場完jor .
咁我而家就係正經講到.
冇乜戲架啦.
多謝敬拜隊.
最識做戲ge 係咩呀.
就係好似梁朝偉gwo D .
一個眼神已經交代到佢地gwo 種矛盾.
無奈.
唔知點算.
睇完呢個劇場可能有D 位搞笑.
但有D 確實係我地真實生活裡面.
有D 人問我地ge 問題.
唔知你有咩感受呢.
有冇人頭先係代入jor D 莊園.
試過比人係D con dayge 時候.
比人執到好啞啞.
之前開組ge 時候都有分享下.
大家以前上莊D 狀況.

$^{81}$有人就話要廿四小時.
要通頂比人con.
又或者有冇D 人.
好似台下新生gwo 三位.
咁樣.
好多ye 質疑.
你呢樣又做唔好.
gwo 樣寫je  你之後係咪咁做架.
今日我地就同大家睇.
其中一個問題.
就係羅馬書十二章裡面.
我地一齊去睇.
我地一齊去學習一下.
今日呢個講題叫正常發揮.
存在係羅馬書十二章.
我地一齊睇十二章十七至到十八節.
我地一齊去讀出.
預備開始.
不要以惡報惡.
眾人以為美的事.
要留心去作.
若是能行.
總要盡力與眾人和睦.
呢段說話頭一句.
其實呈現jor 兩個畫面.
上次想像一下.
第一個就係一個以惡報惡ge 狀況.
唔知有冇即刻你比人報過.
定係你報過人.
有D 畫面會呈現.
但保羅就叫我地放棄.
我地信徒唔好做.
其實呢個唔係一個新ge 教導.
有D 猶太人都係遵守緊.
另一個就係眾人以為美ge 事.
可唔可以比gwo 個.
OK 喎.
交比你地幫我撳.
保羅就係第十二章裡面ge 中間.
講到弟兄姊妹之情.

$^{121}$佢先講弟兄姊妹之情.
之後黎到呢度.
佢就擴闊jor 個範圍.
就係對所有人講.
咩叫所有人.
信徒就會好容易分jor .
信徒或者非信徒.
但其實又唔需要一定咁樣分.
有時邊個係信徒.
你都未必知.
邊個唔係信徒.
可能其實佢個生命ge 特質.
或者佢心裡面相信神.
佢就講緊任何ge 人.
係任何ge 層面.
唔分種族 性別 宗教.
我地信徒都要做.
其他人認為美ge 事.
好美呀.
但係美呢.
頭先同一個姊妹講.
你ge 美同我ge 美係唔同.
好抽象.
有D 英文譯本.
就將美ge 事.
譯做right things.
黑板.
美ge 事譯做right things.
即係正確ge 事.
咁咪容易D 理解.
arm ge ye .
正確ge .
但係唔係.
有D 人話我覺得咁係arm .
你又覺得咁樣係arm .
佢仲要話眾人認為正確ge 事.
保羅其實係期待緊D 乜ye .
佢期待緊.
我地同任何人相處ge 時候.
都努力做人所認為正確ge 事.

$^{161}$甚至乎係被惡待ge 時候.
而係呢個場面裡面.
大家做正確ge 事.
唔會淨係擺係我地gwo 間.
梅鄧隊ge 學會gwo 間房.
係擺出去比世人睇到.
我地話比其他人知.
不過呢.
呢度就睇到頭先gwo 個新生.
好似係寶怡.
我記得.
寶怡gwo 位新生就話.
你地ge 學會一時就迎合D 人.
一時就話要.
一定要尊敬神ge 依神為首.
咁好矛盾.
有冇人覺得自己信ge 上帝都好矛盾.
暫時未有下笠頭.
只要我地睇埋保羅之前所講ge ye .
就唔會覺得矛盾.
其實呢保羅係十二章二節呢.
叫我地做乜ye 呢.
佢叫我地要測驗.
測驗出神ge 旨意.
測驗乜ye 係美好ge .
乜ye 係神ge 旨意.
測驗即係咩呀.
即係用返你ge 腦呀.
係我地信徒有時真係好.
有時呢直入教會離開.
可能都唔需要用腦.
我地就直接輸入jor 我D 訊息.
或者一D 教導.
咁我就去做.
但係要我地用腦去分辨神ge 心意.
係遇上人際相處ge 問題ge 時候.
我地都要去思考點樣先係正確.
點樣先係對人係一個正確ge 事.
所以唔一定係兩邊ge 鐘擺.
唔一定係gwo 邊就係世俗.

$^{201}$gwo 邊就係D 人會諗.
gwo 邊先係教會ge 人諗ge ye .
唔一定係咁樣.
點解會咁呢.
唔知你身邊有冇出現過呢.
其實係呢個世界都會蘊含住好多美好ge 事物.
因為呢個世界俾我創造.
因為係上帝所創造.
上帝ge 美好仍然存係個社會裡面.
有D 知識 有D 學問.
有一D ge 態度.
我地身邊ge 人可能你都會出現過一D .
有D 唔係基督徒你都覺得佢好好.
點解佢生命品質好好.
因為仲有上帝ge 美善係裡面.
好似我今年有上洗禮班.
幫一位姐妹去一齊分享信仰ge 時候.
有位姐妹都提到.
佢信仰裡面佢覺得自己掙扎唔係好大.
佢覺得自己係信主前同信主後.
佢持有ge 價值觀都唔係差好遠.
佢一路都係咁樣去持守.
有D 人確係咁.
有冇D 人信主ge 時候唔係驚天動地覺得.
嘩 有個翻天覆地ge 改變.
因為係社會上仍然有一D 可能唔信神ge 人.
或者未信ge 人.
一直盡力做緊一D 正確ge 事.
不過呢個世界確係有.
一D 叫做基督徒學會ge hater.
惡意扭曲一D 美好ge 價值.
惡意咁con我地呢班ge 信徒.
有權勢ge .
甚至乎拆jor 我地間學會ge 房.
限制我地係學校派單張.
所以保羅就話佢唔係講緊一個絕對ge .
頭先講緊一個絕對ge 教條.
保羅就話若是能行.
總要盡力.
佢係表達緊盡力.

$^{241}$但有D 時候真係唔可以妥協.
係有D 咁ge 時候.
我地gwo 個時候就要捉緊神ge 心意.
唔好擺jor 去人ge 喜好gwo 度.
咁既然保羅唔係講一條絕對.
一個一個好絕對化ge 教條.
咁好難跟喎.
佢係呢個愛仇敵ge 情景背後.
佢想展現一D 乜ye 訊息比我地睇呢.
佢係咪淨係叫我地對人好D .
咁你有時又諗下神ge 心意.
佢其實有一個深一層ge 思考.
首先我地諗一個問題先.
除jor 一D 我地而家都好公認ge .
迪士基督ge 人或者組織.
我地同人相處ge 時候.
你面對衝突.
面對仇敵ge 時候.
你ge 仇敵係咪一定係上帝ge 仇敵呢.
有時我地會好一口咬定.
佢敵對基督呀.
佢咁樣做唔得架.
佢無視上帝ge 話語呀.
我地將佢定性係上帝ge 仇敵.
會唔會好輕易將人去定罪呢.
其實保羅係羅馬書ge 前半部呢.
好多ge 教義好多ge 即係佢展示返神點樣對我地.
耶穌基督點拯救我地.
係向信徒顯示緊上帝對人ge 包容呀.
大家諗唔諗起一D 經文呢.
我地都會從羅馬書裡面去諗得到.
上帝對人ge 包容係超乎我地所想所求.
當我地有D 時候自以為意.
覺得自己一定arm ge 時候.
有冇奪取jor 上帝審判gwo 個權呢.
上帝唔係一定幫曬D 惡人架.
佢話會審判架佢話會懲罰架.
但我地以惡報惡呢個行動ge 時候.
係反映緊我地先有話事gwo 個.
個主權係我度呀.

$^{281}$我話佢係衰人就係衰人呀.
佢係仇敵就係仇敵啦.
如果我地認信上帝係創造主.
我地係受造物.
我地先會去接受.
並且相信神係會親自懲罰gwo D 惡人.
所以信徒係仇敵.
甚或乎係任何人ge 面前.
呢個教導可能淨係講仇敵.
但其實都係放諸係任何人ge 面前.
我地同佢地相處ge 時候.
我地都要努力去做正確ge 事.
係反映jor 我地係屬於上帝.
我地都係屬於上帝.
當我地對上帝有真正ge 信心.
我地就要思考我地同人相處ge 時候.
點做正確ge 事呢?.
點做呀?.
日日都愛心爆棚.
祝福我gwo D 同事啦.
派餅比佢地啦.
做好多行動幫佢斟水啦.
佢夾起條尾我就已經餵飯佢食啦.
係咪咁樣呢?.
嘩!愛心爆炸.
我曾經都接受過呢D 教導.
你用你ge 愛去充滿佢.
圍繞佢.
包圍佢.
讓人真係感受到基督背後ge 愛啦.
係唔係寶萊呢度所講正確ge 事呢?.
原來有D 原則架喎.
重點就係第九節.
我地一齊讀下.
好!第九節預備開始.
愛人不可虛假.
惡要厭惡.
善要親近.
原來從上帝而來ge 愛呢.
係真誠ge 愛.

$^{321}$係唔假架.
一D 都唔假架.
有時呢可能你都會聽到有D 人話.
佢ge 肌肉圖好假呀.
好似有D 包裝咁呀.
虛假呢其中一個意思就係表裡不一啦.
內心諗ge 同行為ge 好似好大落差喎.
表達緊一個人呢.
有兩個狀況喎.
其中一個狀況可能我地平時都會見得到.
好行為.
但佢ge 內心呢.
嘩!動機好唔純正.
或者呢真係甚至乎係壞人添.
甚至乎會諗好多壞事.
其實係當時ge 文化裡面呢.
有一種叫做恩主文化.
都係好分階級ge .
咁呢種恩主文化呢就係講.
私恩典同埋接受恩惠ge 一種關係.
咁呢普遍ge 社會呢.
當時ge 社會都認為.
權貴上流社會ge 人先有能力.
有資格可以私恩典比人.
即係話你想幫隔離gwo 個人.
喂!你有冇諗過你自己係邊個呀.
有冇睇返你自己身份呀.
受恩者呢佢地ge 地位呢.
普遍都會較低.
軟弱.
比人覺得軟弱無能.
係呢種關係入面呢.
受恩者呢係要回報同埋呢.
即係大家都好認同.
喂!人地比恩典你要公開去讚對方呀.
稱讚對方呀.
嘩!好似上次邊個就幫我做jor D 乜ye 啦.
比jor 幾多幾多ye 我啦.
所以好多時私恩者會有一種點樣ge 姿態呢.
高高在上啦.

$^{361}$甚至係為jor 得到榮耀稱讚先去幫助人.
呢種呢虛假ge 愛呢都反映緊.
社會上ge 人呢.
即係佢地有一D 好呀.
喂!未出.
死啦.
收埋先收埋先收埋先.
收埋先.
唔正常.
而家就係一個唔正常ge 發揮.
可唔可以幫我上返去三四頁咁樣啦.
幫我上返去.
係呀係呀arm 呀.
係唔正常呀發揮而家係.
所以呢好多時呢私恩者高高在上啦.
而呢一種虛假ge 愛都反映緊.
人同人之間有好行為.
我地成日都見到喎係個社會.
佢唔一定係黎自愛心架.
可能係黎自一D 公理ge 想法.
甚至乎呢種狀況係教會都有時會出現.
保羅就係要打破呢種ge 框架.
佢教導信徒面對弟兄姊妹ge 缺乏.
唔單止要施捨.
更加要接濟呀.
呢個呢係其他恩主唔會做架.
同埋呢請仇敵食飯呀.
大家都記得我D 講道好多講食飯.
咁呢係啦食飯呢係有咩意思呀.
就係我同你係冇一個高低之分.
我接納你同我都係平起平坐.
我地係同一groupge 人黎ge .
係打從心底接納對方咁樣去私恩點.
而保羅第十節ge 話呢.
就提到恭敬人要彼此推讓.
佢就係講緊不分高低彼此敬重.
係呢個教導呢.
係gwo 個彼此恭敬ge 教導中.
隱藏jor 一種ge 肯定呀.
肯定乜ye 呀.

$^{401}$就係信徒可能有一D 係低下階層ge 信徒.
佢都有資格佢都可以去私恩點架.
因為係上帝ge 眼中呢.
信徒ge 生命同gwo D 上流權貴.
或者普遍社會認為佢有能力.
可以私恩點ge 人.
佢ge 生命係一樣架.
我地都係受造物.
因此保羅呢都一再咁提醒信徒.
唔可以自視太高.
係十二章裡面不斷講.
唔好心高氣傲唔好自以為聰明呀.
佢係抗衡緊一種因主ge 文化.
虛假ge 因主ge 文化.
提出真誠ge 愛係唔會將人分高低.
同時呢佢亦都提到我憑住神所賜比我地ge 恩典.
佢表明因主只有邊個呢.
只有我地ge 上帝.
因主只有我地ge 上帝.
唔知係我地認識ge 信徒群體入面.
有冇見到一D 表裡不一ge 狀況.
好行為但內心真係爭D 禍.
邀請你都為此禱告.
讓真誠ge 愛去取代虛假ge 愛.
咁arm arm 講呢個狀況可能好容易明白.
黎緊呢個狀況見唔見到係掉轉jor .
但其實都係不一致.
內心好行為差.
我心地好做ge 所有一定好.
我扶阿婆過馬路幫大家洗腳.
大家有冇係教會幫人洗過腳.
有呀有D 有啦.
必做ge 行動.
愛心爆棚.
我內心好好.
呢個有D 似John成日都講.
教會好多好人.
但走埋一齊硬係有D 事發生.
唔係好arm 啞或者鬧得不愉快.
愛主ge 人其實好好.

$^{441}$為jor 實踐神ge 話語會諗好多ge 方法.
之前都諗點樣開會通宵達旦.
點樣令到我地D 羊仔聽上帝ge 話語.
點樣可以更加讀多D 經.
遵守神ge 話.
有時我地會不自覺.
好想將我地所諗ge 方法絕對化jor .
甚至乎會將好多眾人以為美ge 事.
都定義為世俗ge 事.
唔知你成長ge 經歷.
會唔會好抗拒好多.
非信徒群體以黎ge .
一D 學習,學問,知識.
會唔會好抗拒.
唔得架喎.
甚至乎我自己年代.
上莊係唔ok架.
點解?因為你唔返教會.
唔得架.
有冇一D 絕對化ge 教條.
你自己將佢成為一D 教條要人遵守.
或者要自己遵守呢.
我地都要問自己.
當我地一味只係思考教條.
只係諗我地埋手gwo 本聖經.
係好好ge 讀經.
但我地少jor 認識呢個世界.
觀察身邊ge 人.
世界上ge 人,事,學識,學問.
我地其實好難做到眾人以為正確ge 事.
點解呢?好似我arm arm 去.
我地arm arm 過去ge 中秋晚宴.
好開心.
我地今次就同好多.
其實全部都唔係基督教ge .
一個單位.
其他大部分都唔係基督教ge 團體.
有一個合作ge 機構ge 職員.
有一次開會就講笑咁問我.
Hey!你地使唔使係中途插入一個唱聖詩ge 時段?.

$^{481}$我ok架喎!唱下Hallelujah啦!.
我話o下?邊個位可以插入我地gwo 晚食飯?.
同埋有D band show係背後做背景音樂.
佢話o下?咁你唔唱Hallelujah.
人地點知你係教會呀?.
我話o下?可唔可以唱Hallelujah.
你先知我地係教會咩?.
都幾慘.
其實佢係半講笑.
因為佢自己經歷過一D .
可能同一D 機構合作.
一定要有呢D 儀式.
我唔係話呢D 儀式唔好.
因為我地ge 活動裡面實在太過硬Tag.
超級硬Tag.
食飯Hallelujah.
我地gwo 晚ge 主題係咩呢?.
中秋呀嘛!團圓呀嘛!.
就唔係好關事.
我自己呢係呢個位都悔改.
我以前都係都夾硬黎架我個人都.
係我為jor .
我都係頭先講gwo 種表裡不一.
我內心好好好愛住.
就係諗埋D 夾硬黎ge ye .
以前搞活動呀同年青人傾計ge 時候.
我覺得有機會呢就要同未信ge 人講福音.
成日係個心裡面聽緊對方講ye .
個心就係度諗呀呢個位得唔得呢?.
呢個講唔講到呢?.
呢個可唔可以祈禱呢?.
你又冇一種包袱呀?.
去探訪或者去見新朋友ge 時候.
覺得你唔祈禱就唔完全啦.
或者你講唔到福音.
嘩!好渣呀今次.
唔得喎.
但係確係有D 時候我見過.
或者我自己都經歷過過分硬踢.
咁係會有反效果.

$^{521}$可能有D 新朋友就覺得.
o下!就淨係.
點解你地都唔聽我講ye ?.
D 新朋友覺得你有冇聽我講ye ?.
你冇先聽我講ye 你回應ge ye .
全部都唔關我頭先ge 事.
最後就要黎一個神保守神帶你.
我為你祈禱祝福阿門.
咁就完jor 啦.
呢D 都確實係新朋友ge 一D 睇法.
咁當然裡面有信主架啦.
當然有ge .
或者我地呢套當中可能都係聖靈感動.
無論gwo 間幾硬踢你都信jor 主.
咁就係上帝ge 帶領.
特別係呢最近我自己都好.
好心up 啦.
大家都成日睇到一D ge 新聞ge 狀況.
咁呢就係一D 年青人ge 新聞啦.
咁我好體會到啦.
今日服侍年青人呢.
點樣去講福音點樣去傳福音呢.
在座都有班導師佢地都好知.
其實你專心聽佢分享呢.
佢可以講到佢ge 難處.
佢ge gwo D 解決唔到ge 問題.
對佢黎講係一個福音.
因為佢從來都冇辦法講到比人聽.
因為佢講親想講比gwo D 人聽.
可能都會有D ge 回應啦.
可能係社會比佢地ge 壓力啦.
好難呀佢地聽我地聽佢地講就係一個福音.
我唔係否定呢gwo D 即係有福音ge 行動.
如果你擺到明福音迎人地又黎.
咁冇乜所謂但我地就要諗.
我內心好但係我係咪真係為人好呢.
我係個行為裡面係咪同我ge 內心一致呢.
我話為人好我真係為人好.
定係好想淨係想講我自己ge ye 呢.
最後我地gwo 個中秋晚宴呢就冇唱到呢D 誠詩啦.

$^{561}$就唱jor D 即係中學生帶領唱jor D 好好聽ge 歌啦咁樣.
亦都冇分享見證.
咁但係就有組員呢就咁講.
佢事後就分享好開心咁樣一班人一齊去做一齊去上莊.
今晚呢成個即係場景好似冇上帝.
我地冇講過一個上帝祝福我地.
上帝冇令到個場本身都有D 問題ge .
上帝又點點點點點我地冇咁樣.
但係佢就係經歷到有實實在在有上帝係我地每一個信徒ge 當中.
亦都有街坊呢其實我地都冇講過教會.
我冇乜點sell過啦.
咁佢地同我地ge 距離又拉近jor 啦.
有個就請我食豉油雞啦咁樣.
我食jor 佢冇D 唔記得做乜ye 啦.
咁總之佢話豉油雞仲正呀咁樣啦.
有D 參加我地活動啦有D 黎到少年團契啦.
所以對唔同ge 人行正確ge 事都有唔同ge 方式.
有人覺得教會ge 問候關心好溫暖好有愛.
但亦都有人覺得好過界好唔舒服.
我地要常常去察驗.
如果我地有愛主愛人ge 心.
我地都要不斷ge 去學習同埋思想眾人以為美ge 事.
所以無論我地面對仇敵或者我地面對呢個世界接觸到ge 人.
我地都要用真誠ge 愛.
頭先所講就係一種表裡不一ge 愛.
我地要不斷去檢視就算我今日我地目者我地都要不斷去檢視.
真誠ge 愛去做正確ge 事.
呢個做jor 正確ge 事其實已經係一個正常ge 發揮.
已經好足夠.
當然有時我地會有一D 突然間性靈感動.
一D 好神奇ge 經歷.
呢D 唔排除ge 確實係會有.
但係比較多ge 係我地日常生活裡面.
我地每一個場景對每一個人我地做正確ge 事.
有一個正常ge 發揮.
基督徒已經係一個好好ge 見證.
你覺得做正常ge 人容唔容易呀.
你職場裡面見得多唔多正常ge 信徒呢.
有D 人就另投.
呢個就令我想起.

$^{601}$頭先唔小心出長jor .
令我想起下一版.
呢個教練.
有D 人認得佢.
令我想起呢個教練教波ge 時候.
即係我係神學院ge 時候參加籃球隊.
我地呢班人睇個高度已經知唔係本身打籃球.
好一致.
唔係ge .
其實好努力.
強身健體.
透過籃球接觸下微信ge 人.
當時ge 教練就係我地流堂黎緊PE級ge 級主.
一陣出去報名.
每次我地有比賽.
佢都唔係gwo D 好激勵ge 教練.
佢就話你打返平時練gwo D ye .
心諗平時練gwo D 其實.
我地平時最多練習兩個戰術.
記唔記得呢.
一或者二.
頂多比多個三.
三啦咁樣.
通常都係得一同二.
做返平時ge ye 好容易.
一D 都唔難.
不過呢.
我地ge 對手呢.
係邊D 人呢.
就係一D 技術精湛ge 中學生.
又或者係體能爆燈ge 正生書院ge 同學.
人地每日就跑山.
我呢就行山都好少.
所以當我地係一個比平時快jor 好多ge 速度節奏.
對手嘩點解我ge 一破解唔到人地ge 十ge .
嘩即刻放將.
即刻好簡單.
其實我講ge 一好似好勁.
就係掟個菠蘿藍底唔知傳比邊個去射.
咁樣次次都做唔到.

$^{641}$咁但係通常呢.
即係我地偶爾做到一次呢.
我地就非常之開心.
覺得呢個正常ge 發揮呢已經係超額ge 完成.
同我地日常ge 生活都好似ge .
我地唔係淨係嘩順風順水.
每日都有可能突然間隔離有人比D 聲音你.
令你覺得好煩擾.
或者嘩你同事咁快做ye .
我需唔需要再趕D 呢.
gwo 個速度需唔需要跟上.
老細好脫喎.
或者放工六點先黎交搭ye 過黎.
咁點呢咁樣.
正常發揮真係唔容易ge .
上個禮拜我地即係唔知點解.
好似D 目者知道我講咩咁樣.
係個目者會gwo 度已經講起.
嘩我地組員其實能夠係職場裡面.
做一個正正常常ge 人呢.
即係可以有一個正正常常ge 工作表現已經好唔容易.
因為實在太多其他ge 因素.
太多太多.
甚至乎有我地自身ge 一D 狀況.
可能自己ge 開心唔開心等等.
我地都需要有智慧去測驗.
職場裡面正正常常ge 人會點架.
會得到唔正常ge 工作量.
所以係呢D ge 時候我地更加要測驗神ge 心意.
有時可能我地要主動.
有時可能被動.
有時可能去.
有時可能留.
唔係好硬咁樣.
有D 有不正常工作量ge 組員就即刻.
係呀 氣憤呀咩.
我地需要有智慧呀.
信仰ge 見證真係唔一定係好耀眼ge 屬靈經歷呀.
所以大家去見證耶穌基督ge 時候.
好多人呢都會好覺得好掙扎.

$^{681}$講見證.
我冇人地gwo D 一幅圖畫呀.
即係gwo D 神.
即係神比佢一幅圖畫呀.
我gwo 次經歷到點樣點樣.
做jor 一件好大ge 事呀咁樣.
有ge gwo D 係確實係會有ge .
但更大部分時間.
係我地都同.
其他人過住一樣ge 生活.
你同隔離gwo 個人.
隔離ge 同事.
隔離位ge 同學等等.
都係過一樣ge 生活.
當我地係患難.
高低起伏.
平淡ge 日子.
甚或乎面對壓迫ge 時候.
都堅持做正常同埋正確ge 事.
就已經係一個正常發揮ge 見證.
平時我發現呢好多信徒呢.
朋友都有咁樣ge 見證.
我見過有即係我地勁白隊隊員啦.
或者以前都有勁白隊隊員.
試過經歷自己好大ge 困難.
但係為jor D 隊友呢.
一team人呢.
都會扑返黎.
好想一齊去服事.
為jor 身邊gwo 一個.
咁我又見過有宣教士啦.
係呀同你地講呀.
宣教士啦.
係停電之下呢.
都好正常咁樣煮一餐飯.
比佢愛錫ge 人.
屋企人.
甚或乎弟兄姊妹.
係海外ge 你地啦.
你地而家今日都如常咁.

$^{721}$同我地在座ge .
每一個一齊去敬拜上帝.
流塘都好努力去維持正常.
同世界連結.
黎緊我地有好多ge club.
絕對唔係單純係一D 圍爐俱樂部.
我地都好想呢.
我地係呢個.
好多壓力ge 世代同社會連結.
讓人都認識我地.
彼此認識ge 一個平台.
gwo D 都係一D 正常發揮.
我諗起一D ge club就係成日行山ge 阿迪.
佢facebook每張相九成都係行山.
清心講過話.
佢好多時都同佢老公玩boardgame.
玩到搞個club出黎.
呢D 都係佢地ge 正常發揮.
我地要正常發揮唔容易.
但我地可以祈求我地ge 上帝.
之前呢就頭先都講我好鍾意食飯.
我唔係鍾意食飯本身.
鍾意約人食飯.
以下呢張相就提醒我一個兩年前ge 飯局.
我同朋友一齊食雞煲.
仲要係冬天.
暖笠笠咁食雞煲實在太開心.
gwo 日我就如常咁返工.
放工就隨住呢個放工ge 人潮.
漂流到去呢個大西北.
好少去ge 大西北.
入到去一間大牌檔feelge 雞煲檔.
見到幾個熟悉ge 面孔同我打招呼.
咁我就行埋去參加jor 一個.
到今日我都記得ge 飯局.
當日呢.
係呢個飯局裡面呢.
有三位神學院ge 學妹.
我地如常ge 分享工作生活.
即係我地平時放工犯罪吹水都係咁.

$^{761}$講下平時工作有咩麻煩事.
講下生活.
你有冇做身體檢查.
講下呢D 啦女人有好多病.
講下身體狀況.
其中一位姐妹都有分享佢身體狀況.
佢本身自己有好多年前已經有重病.
係咁多年裡面都反反覆覆.
但係我地每次有飯局.
佢都好盡量出席.
如常去出席佢用佢ge 方法.
搭Uber又好可能縮短.
即係可能我地就下佢一D ge 地點.
咁如常去出席.
咁gwo 一晚呢就係我最後一次同佢食飯.
咁尋日呢就係佢已經返jor 天家ge 一周年.
咁我記得佢當時ge 身體呢已經好軟弱ge .
食得唔多.
其實睇我地食多.
亦都去唔到遠.
但我地從佢一向ge gwo D 歡笑聲.
同我地ge 分享.
佢分享ge 都係如常ge 生活狀況.
繼續做緊佢係教會裡面ge 服侍.
我感受到佢好享受.
佢繼續好盡力咁如常同我地吃喝.
好盡力過正常ge 生活.
佢從來係咁耐ge 病患裡面.
佢冇教我地要係患難裡面去堅忍.
但佢一直以黎堅持同我地相聚.
堅持去墓會.
同埋當日如常咁參與我地ge 飯局.
到今日仍然令我覺得非常之被愛.
非常之感動.
佢呢個正常ge 發揮激勵jor 我.
做一個可以正常發揮ge 信徒.
好唔容易.
因為實在太多唔同ge 變幻.
堅持ge 過程.
可能你突然間發現原來你自己都幾唔正常.

$^{801}$但係保羅一早就話比我地知一個重要ge 事實.
我地一齊去睇一睇.
羅馬書五章十節咁講.
預備開始.
因為我地作仇敵ge 時候.
且藉著神兒子的死.
得與神和好.
既已和好.
就更要因他的生得救了.
保羅提出要愛仇敵.
其實呢種表達呢.
我真係覺得非常之精彩.
好似電影拍攝ge 一種手法.
係自己ge 電影裡面呈現返一D 經典ge 場面.
自己鍾意ge 畫面.
唔會完全一樣ge .
但觀眾一睇到就會聯想返相關ge 電影.
係一種向電影致敬ge 方式.
今日我地ge 生命有冇呢一種致敬呢.
我地曾經都係作仇敵.
我地被耶穌接納.
同上帝和好.
保羅鼓勵我地要愛仇敵.
愛呢個世界.
愛任何一個人.
就好似向耶穌致敬.
重演某段經典ge 場面.
重演上帝對我地ge 寬容.
對我地ge 拯救.
對我地ge 接納.
真正ge 愛呢係唔會虛假.
真正ge 愛係唔會虛假.
因為係從上帝而黎.
只要我地真係接受耶穌基督.
亦都要認真思考.
有心亦都要有行動.
我地要去思想.
聖靈會帶領我地有正常ge 發揮.
任何時候保持一個正常ge 發揮.
就算我地唔拎gwo D 呢好吸引ge 迎新禮物.

$^{841}$堅持做返一個正常人.
任何人都會見到當中ge 尾線.
甚至主動問我地心中盼望ge 緣由.
我地一會會有一段安靜祈禱ge 時間.
求主去兼顧我地ge 信心.
\newpage



\section{}
\label{sec:JKdFzjAsLZY}
\textbf{《致餘民及流散者:給香港基督徒的神學八課》第二季第7課|20231101 [JKdFzjAsLZY]}
\newline
\newline
連結: \href{https://youtube.com/watch?v=JKdFzjAsLZY}{\texttt{ https://youtube.com/watch?v=JKdFzjAsLZY}} ~~~~ 語音日期: 2023-11-01 
\newline
\newline
\hyperref[sec:BDg16RM34JI]{\small{< < < PREV SERMON < < <}}
~
\hyperref[sec:index_chronic]{\small{[返順時目]}}
~
\hyperref[sec:index_scriptual]{\small{[返順卷目]}}
~
\hyperref[sec:_cxLnHL_TWQ]{\small{> > > NEXT SERMON > > >}}
\newline
\newline
$^{1}$我只想知道.
你到底是什麼意思.
我只想知道.
你到底是什麼意思.
我只想知道.
你到底是什麼意思.
我只想知道.
你到底是什麼意思.
我只想知道.
你到底是什麼意思.
我只想知道.
你到底是什麼意思.
我只想知道.
你到底是什麼意思.
我只想知道.
你到底是什麼意思.
我只想知道.
你到底是什麼意思.
我只想知道.
你到底是什麼意思.
我只想知道.
你到底是什麼意思.
我只想知道.
你到底是什麼意思.
我只想知道.
你到底是什麼意思.
我只想知道.
你到底是什麼意思.
我只想知道.
你到底是什麼意思.
我只想知道.
你到底是什麼意思.
我只想知道.
你到底是什麼意思.
我只想知道.
你到底是什麼意思.
我只想知道.
你到底是什麼意思.
我只想知道.
你到底是什麼意思.

$^{41}$我只想知道.
你到底是什麼意思.
我只想知道.
你到底是什麼意思.
我只想知道.
你到底是什麼意思.
我只想知道.
你到底是什麼意思.
我只想知道.
你到底是什麼意思.
我只想知道.
你到底是什麼意思.
我只想知道.
你到底是什麼意思.
我只想知道.
你到底是什麼意思.
我只想知道.
你到底是什麼意思.
我只想知道.
你到底是什麼意思.
我只想知道.
你到底是什麼意思.
我只想知道.
你到底是什麼意思.
我只想知道.
你到底是什麼意思.
我只想知道.
你到底是什麼意思.
我只想知道.
你到底是什麼意思.
我只想知道.
你到底是什麼意思.
我只想知道.
你到底是什麼意思.
我只想知道.
你到底是什麼意思.
我只想知道.
你到底是什麼意思.
我只想知道.
你到底是什麼意思.

$^{81}$我只想知道.
你到底是什麼意思.
我只想知道.
你到底是什麼意思.
我只想知道.
你到底是什麼意思.
我只想知道.
你到底是什麼意思.
我只想知道.
你到底是什麼意思.
我只想知道.
你到底是什麼意思.
我只想知道.
你到底是什麼意思.
我只想知道.
你到底是什麼意思.
我只想知道.
你到底是什麼意思.
我只想知道.
你到底是什麼意思.
我只想知道.
你到底是什麼意思.
我只想知道.
你到底是什麼意思.
我只想知道.
你到底是什麼意思.
我只想知道.
你到底是什麼意思.
我只想知道.
你到底是什麼意思.
我只想知道.
你到底是什麼意思.
我只想知道.
你到底是什麼意思.
我只想知道.
你到底是什麼意思.
我只想知道.
你到底是什麼意思.
我只想知道.
你到底是什麼意思.

$^{121}$我只想知道.
你到底是什麼意思.
我只想知道.
你到底是什麼意思.
我只想知道.
你到底是什麼意思.
我只想知道.
你到底是什麼意思.
我只想知道.
你到底是什麼意思.
我只想知道.
你到底是什麼意思.
我只想知道.
你到底是什麼意思.
我只想知道.
你到底是什麼意思.
我只想知道.
你到底是什麼意思.
我只想知道.
你到底是什麼意思.
我只想知道.
你到底是什麼意思.
我只想知道.
你到底是什麼意思.
我只想知道.
你到底是什麼意思.
我只想知道.
你到底是什麼意思.
我只想知道.
你到底是什麼意思.
我只想知道.
你到底是什麼意思.
我只想知道.
你到底是什麼意思.
我只想知道.
你到底是什麼意思.
我只想知道.
你到底是什麼意思.
我只想知道.
你到底是什麼意思.

$^{161}$我只想知道.
你到底是什麼意思.
我只想知道.
你到底是什麼意思.
我只想知道.
你到底是什麼意思.
我只想知道.
你到底是什麼意思.
我只想知道.
你到底是什麼意思.
我只想知道.
你到底是什麼意思.
我只想知道.
你到底是什麼意思.
我只想知道.
你到底是什麼意思.
我只想知道.
你到底是什麼意思.
我只想知道.
你到底是什麼意思.
我只想知道.
你到底是什麼意思.
我只想知道.
你到底是什麼意思.
我只想知道.
你到底是什麼意思.
我只想知道.
你到底是什麼意思.
我只想知道.
你到底是什麼意思.
我只想知道.
你到底是什麼意思.
我只想知道.
你到底是什麼意思.
我只想知道.
你到底是什麼意思.
我只想知道.
你到底是什麼意思.
我只想知道.
你到底是什麼意思.

$^{201}$我只想知道.
你到底是什麼意思.
我只想知道.
你到底是什麼意思.
我只想知道.
你到底是什麼意思.
我只想知道.
你到底是什麼意思.
我只想知道.
你到底是什麼意思.
我只想知道.
你到底是什麼意思.
我只想知道.
你到底是什麼意思.
我只想知道.
你到底是什麼意思.
我只想知道.
你到底是什麼意思.
我只想知道.
你到底是什麼意思.
我只想知道.
你到底是什麼意思.
我只想知道.
你到底是什麼意思.
我只想知道.
你到底是什麼意思.
我只想知道.
你到底是什麼意思.
我只想知道.
你到底是什麼意思.
我只想知道.
你到底是什麼意思.
我只想知道.
你到底是什麼意思.
我只想知道.
你到底是什麼意思.
我只想知道.
你到底是什麼意思.
我只想知道.
你到底是什麼意思.

$^{241}$我只想知道.
你到底是什麼意思.
我只想知道.
你到底是什麼意思.
我只想知道.
你到底是什麼意思.
我只想知道.
你到底是什麼意思.
我只想知道.
你到底是什麼意思.
我只想知道.
你到底是什麼意思.
我只想知道.
你到底是什麼意思.
我只想知道.
你到底是什麼意思.
我只想知道.
你到底是什麼意思.
我只想知道.
你到底是什麼意思.
我只想知道.
你到底是什麼意思.
我只想知道.
你到底是什麼意思.
我只想知道.
你到底是什麼意思.
我只想知道.
你到底是什麼意思.
我只想知道.
你到底是什麼意思.
我只想知道.
你到底是什麼意思.
我只想知道.
你到底是什麼意思.
我只想知道.
你到底是什麼意思.
我只想知道.
你到底是什麼意思.
我只想知道.
你到底是什麼意思.

$^{281}$我只想知道.
你到底是什麼意思.
我只想知道.
你到底是什麼意思.
我只想知道.
你到底是什麼意思.
我只想知道.
你到底是什麼意思.
我只想知道.
你到底是什麼意思.
我只想知道.
你到底是什麼意思.
我只想知道.
你到底是什麼意思.
我只想知道.
你到底是什麼意思.
我只想知道.
你到底是什麼意思.
我只想知道.
你到底是什麼意思.
我只想知道.
你到底是什麼意思.
我只想知道.
你到底是什麼意思.
我只想知道.
你到底是什麼意思.
我只想知道.
你到底是什麼意思.
我只想知道.
你到底是什麼意思.
我只想知道.
你到底是什麼意思.
我只想知道.
你到底是什麼意思.
我只想知道.
你到底是什麼意思.
我只想知道.
你到底是什麼意思.
我只想知道.
你到底是什麼意思.

$^{321}$我只想知道.
你到底是什麼意思.
我只想知道.
你到底是什麼意思.
我只想知道.
你到底是什麼意思.
我只想知道.
你到底是什麼意思.
我只想知道.
你到底是什麼意思.
我只想知道.
你到底是什麼意思.
我只想知道.
你到底是什麼意思.
我只想知道.
你到底是什麼意思.
我只想知道.
你到底是什麼意思.
我只想知道.
你到底是什麼意思.
我只想知道.
你到底是什麼意思.
我只想知道.
你到底是什麼意思.
我只想知道.
你到底是什麼意思.
我只想知道.
你到底是什麼意思.
我只想知道.
你到底是什麼意思.
我只想知道.
你到底是什麼意思.
我只想知道.
你到底是什麼意思.
我只想知道.
你到底是什麼意思.
我只想知道.
你到底是什麼意思.
我只想知道.
你到底是什麼意思.

$^{361}$我只想知道.
你到底是什麼意思.
我只想知道.
你到底是什麼意思.
我只想知道.
你到底是什麼意思.
我只想知道.
你到底是什麼意思.
我只想知道.
你到底是什麼意思.
我只想知道.
你到底是什麼意思.
我只想知道.
你到底是什麼意思.
我只想知道.
你到底是什麼意思.
我只想知道.
你到底是什麼意思.
我只想知道.
你到底是什麼意思.
我只想知道.
你到底是什麼意思.
我只想知道.
你到底是什麼意思.
我只想知道.
你到底是什麼意思.
我只想知道.
你到底是什麼意思.
我只想知道.
你到底是什麼意思.
我只想知道.
你到底是什麼意思.
我只想知道.
你到底是什麼意思.
我只想知道.
你到底是什麼意思.
我只想知道.
你到底是什麼意思.
我只想知道.
你到底是什麼意思.

$^{401}$我只想知道.
你到底是什麼意思.
我只想知道.
你到底是什麼意思.
我只想知道.
你到底是什麼意思.
我只想知道.
你到底是什麼意思.
我只想知道.
你到底是什麼意思.
我只想知道.
你到底是什麼意思.
我只想知道.
你到底是什麼意思.
我只想知道.
你到底是什麼意思.
我只想知道.
你到底是什麼意思.
我只想知道.
你到底是什麼意思.
我只想知道.
你到底是什麼意思.
我只想知道.
你到底是什麼意思.
我只想知道.
你到底是什麼意思.
我只想知道.
你到底是什麼意思.
我只想知道.
你到底是什麼意思.
我只想知道.
你到底是什麼意思.
我只想知道.
你到底是什麼意思.
我只想知道.
你到底是什麼意思.
我只想知道.
你到底是什麼意思.
我只想知道.
你到底是什麼意思.

$^{441}$我只想知道.
你到底是什麼意思.
我只想知道.
你到底是什麼意思.
我只想知道.
你到底是什麼意思.
我只想知道.
你到底是什麼意思.
我只想知道.
你到底是什麼意思.
我只想知道.
你到底是什麼意思.
我只想知道.
你到底是什麼意思.
我只想知道.
你到底是什麼意思.
我只想知道.
你到底是什麼意思.
我只想知道.
你到底是什麼意思.
我只想知道.
你到底是什麼意思.
我只想知道.
你到底是什麼意思.
我只想知道.
你到底是什麼意思.
我只想知道.
你到底是什麼意思.
我只想知道.
你到底是什麼意思.
我只想知道.
你到底是什麼意思.
我只想知道.
你到底是什麼意思.
我只想知道.
你到底是什麼意思.
我只想知道.
你到底是什麼意思.
我只想知道.
你到底是什麼意思.

$^{481}$我只想知道.
你到底是什麼意思.
我只想知道.
你到底是什麼意思.
我只想知道.
你到底是什麼意思.
我只想知道.
你到底是什麼意思.
我只想知道.
你到底是什麼意思.
我只想知道.
你到底是什麼意思.
我只想知道.
你到底是什麼意思.
我只想知道.
你到底是什麼意思.
我只想知道.
你到底是什麼意思.
我只想知道.
你到底是什麼意思.
我只想知道.
你到底是什麼意思.
我只想知道.
你到底是什麼意思.
我只想知道.
你到底是什麼意思.
我只想知道.
你到底是什麼意思.
我只想知道.
你到底是什麼意思.
我只想知道.
你到底是什麼意思.
我只想知道.
你到底是什麼意思.
我只想知道.
你到底是什麼意思.
我只想知道.
你到底是什麼意思.
我只想知道.
你到底是什麼意思.

$^{521}$我只想知道.
你到底是什麼意思.
我只想知道.
你到底是什麼意思.
我只想知道.
你到底是什麼意思.
我只想知道.
你到底是什麼意思.
我只想知道.
你到底是什麼意思.
我只想知道.
你到底是什麼意思.
我只想知道.
你到底是什麼意思.
我只想知道.
你到底是什麼意思.
我只想知道.
你到底是什麼意思.
我只想知道.
你到底是什麼意思.
我只想知道.
你到底是什麼意思.
我只想知道.
你到底是什麼意思.
我只想知道.
你到底是什麼意思.
我只想知道.
你到底是什麼意思.
我只想知道.
你到底是什麼意思.
我只想知道.
你到底是什麼意思.
我只想知道.
你到底是什麼意思.
我只想知道.
你到底是什麼意思.
我只想知道.
你到底是什麼意思.
我只想知道.
你到底是什麼意思.

$^{561}$我只想知道.
你到底是什麼意思.
我只想知道.
你到底是什麼意思.
我只想知道.
你到底是什麼意思.
我只想知道.
你到底是什麼意思.
我只想知道.
你到底是什麼意思.
我只想知道.
你到底是什麼意思.
我只想知道.
你到底是什麼意思.
我只想知道.
你到底是什麼意思.
我只想知道.
你到底是什麼意思.
我只想知道.
你到底是什麼意思.
我只想知道.
你到底是什麼意思.
我只想知道.
你到底是什麼意思.
我只想知道.
你到底是什麼意思.
我只想知道.
你到底是什麼意思.
我只想知道.
你到底是什麼意思.
我只想知道.
你到底是什麼意思.
我只想知道.
你到底是什麼意思.
我只想知道.
你到底是什麼意思.
我只想知道.
你到底是什麼意思.
我只想知道.
你到底是什麼意思.

$^{601}$我只想知道.
你到底是什麼意思.
我只想知道.
你到底是什麼意思.
我只想知道.
你到底是什麼意思.
我只想知道.
你到底是什麼意思.
我只想知道.
你到底是什麼意思.
我只想知道.
你到底是什麼意思.
我只想知道.
你到底是什麼意思.
我只想知道.
你到底是什麼意思.
我只想知道.
你到底是什麼意思.
我只想知道.
你到底是什麼意思.
我只想知道.
你到底是什麼意思.
我只想知道.
你到底是什麼意思.
我只想知道.
你到底是什麼意思.
我只想知道.
你到底是什麼意思.
我只想知道.
你到底是什麼意思.
我只想知道.
你到底是什麼意思.
我只想知道.
你到底是什麼意思.
我只想知道.
你到底是什麼意思.
我只想知道.
你到底是什麼意思.
我只想知道.
你到底是什麼意思.

$^{641}$我只想知道.
你到底是什麼意思.
我只想知道.
你到底是什麼意思.
我只想知道.
你到底是什麼意思.
我只想知道.
你到底是什麼意思.
我只想知道.
你到底是什麼意思.
我只想知道.
你到底是什麼意思.
我只想知道.
你到底是什麼意思.
我只想知道.
你到底是什麼意思.
我只想知道.
你到底是什麼意思.
我只想知道.
你到底是什麼意思.
我只想知道.
你到底是什麼意思.
我只想知道.
你到底是什麼意思.
我只想知道.
你到底是什麼意思.
我只想知道.
你到底是什麼意思.
我只想知道.
你到底是什麼意思.
我只想知道.
你到底是什麼意思.
我只想知道.
你到底是什麼意思.
我只想知道.
你到底是什麼意思.
我只想知道.
你到底是什麼意思.
我只想知道.
你到底是什麼意思.

$^{681}$我只想知道.
你到底是什麼意思.
我只想知道.
你到底是什麼意思.
我只想知道.
你到底是什麼意思.
我只想知道.
你到底是什麼意思.
我只想知道.
你到底是什麼意思.
我只想知道.
你到底是什麼意思.
我只想知道.
你到底是什麼意思.
我只想知道.
你到底是什麼意思.
我只想知道.
你到底是什麼意思.
我只想知道.
你到底是什麼意思.
我只想知道.
你到底是什麼意思.
我只想知道.
你到底是什麼意思.
我只想知道.
你到底是什麼意思.
我只想知道.
你到底是什麼意思.
我只想知道.
你到底是什麼意思.
我只想知道.
你到底是什麼意思.
我只想知道.
你到底是什麼意思.
我只想知道.
你到底是什麼意思.
我只想知道.
你到底是什麼意思.
我只想知道.
你到底是什麼意思.

$^{721}$我只想知道.
你到底是什麼意思.
我只想知道.
你到底是什麼意思.
我只想知道.
你到底是什麼意思.
我只想知道.
你到底是什麼意思.
我只想知道.
你到底是什麼意思.
我只想知道.
你到底是什麼意思.
我只想知道.
你到底是什麼意思.
我只想知道.
你到底是什麼意思.
我只想知道.
你到底是什麼意思.
我只想知道.
你到底是什麼意思.
我只想知道.
你到底是什麼意思.
我只想知道.
你到底是什麼意思.
我只想知道.
你到底是什麼意思.
我只想知道.
你到底是什麼意思.
我只想知道.
你到底是什麼意思.
我只想知道.
你到底是什麼意思.
我只想知道.
你到底是什麼意思.
我只想知道.
你到底是什麼意思.
我只想知道.
你到底是什麼意思.
我只想知道.
你到底是什麼意思.

$^{761}$我只想知道.
你到底是什麼意思.
我只想知道.
你到底是什麼意思.
我只想知道.
你到底是什麼意思.
我只想知道.
你到底是什麼意思.
我只想知道.
你到底是什麼意思.
我只想知道.
你到底是什麼意思.
我只想知道.
你到底是什麼意思.
我只想知道.
你到底是什麼意思.
我只想知道.
你到底是什麼意思.
我只想知道.
你到底是什麼意思.
我只想知道.
你到底是什麼意思.
我只想知道.
你到底是什麼意思.
我只想知道.
你到底是什麼意思.
我只想知道.
你到底是什麼意思.
我只想知道.
你到底是什麼意思.
我只想知道.
你到底是什麼意思.
我只想知道.
你到底是什麼意思.
我只想知道.
你到底是什麼意思.
我只想知道.
你到底是什麼意思.
我只想知道.
你到底是什麼意思.

$^{801}$我只想知道.
你到底是什麼意思.
我只想知道.
你到底是什麼意思.
我只想知道.
你到底是什麼意思.
我只想知道.
你到底是什麼意思.
我只想知道.
你到底是什麼意思.
我只想知道.
你到底是什麼意思.
我只想知道.
你到底是什麼意思.
我只想知道.
你到底是什麼意思.
我只想知道.
你到底是什麼意思.
我只想知道.
你到底是什麼意思.
我只想知道.
你到底是什麼意思.
我只想知道.
你到底是什麼意思.
我只想知道.
你到底是什麼意思.
我只想知道.
你到底是什麼意思.
我只想知道.
你到底是什麼意思.
我只想知道.
你到底是什麼意思.
我只想知道.
你到底是什麼意思.
我只想知道.
你到底是什麼意思.
我只想知道.
你到底是什麼意思.
我只想知道.
你到底是什麼意思.

$^{841}$我只想知道.
你到底是什麼意思.
我只想知道.
你到底是什麼意思.
我只想知道.
你到底是什麼意思.
我只想知道.
你到底是什麼意思.
我只想知道.
你到底是什麼意思.
我只想知道.
你到底是什麼意思.
我只想知道.
你到底是什麼意思.
我只想知道.
你到底是什麼意思.
我只想知道.
你到底是什麼意思.
我只想知道.
你到底是什麼意思.
我只想知道.
你到底是什麼意思.
我只想知道.
你到底是什麼意思.
我只想知道.
你到底是什麼意思.
我只想知道.
你到底是什麼意思.
我只想知道.
你到底是什麼意思.
我只想知道.
你到底是什麼意思.
我只想知道.
你到底是什麼意思.
我只想知道.
你到底是什麼意思.
我只想知道.
你到底是什麼意思.
我只想知道.
你到底是什麼意思.

$^{881}$我只想知道.
你到底是什麼意思.
我只想知道.
你到底是什麼意思.
我只想知道.
你到底是什麼意思.
我只想知道.
你到底是什麼意思.
我只想知道.
你到底是什麼意思.
我只想知道.
你到底是什麼意思.
我只想知道.
你到底是什麼意思.
我只想知道.
你到底是什麼意思.
我只想知道.
你到底是什麼意思.
我只想知道.
你到底是什麼意思.
我只想知道.
你到底是什麼意思.
我只想知道.
你到底是什麼意思.
我只想知道.
你到底是什麼意思.
我只想知道.
你到底是什麼意思.
我只想知道.
你到底是什麼意思.
我只想知道.
你到底是什麼意思.
我只想知道.
你到底是什麼意思.
我只想知道.
你到底是什麼意思.
我只想知道.
你到底是什麼意思.
我只想知道.
你到底是什麼意思.

$^{921}$我只想知道.
你到底是什麼意思.
我只想知道.
你到底是什麼意思.
我只想知道.
你到底是什麼意思.
我只想知道.
你到底是什麼意思.
我只想知道.
你到底是什麼意思.
我只想知道.
你到底是什麼意思.
我只想知道.
你到底是什麼意思.
我只想知道.
你到底是什麼意思.
我只想知道.
你到底是什麼意思.
我只想知道.
你到底是什麼意思.
我只想知道.
你到底是什麼意思.
我只想知道.
你到底是什麼意思.
我只想知道.
你到底是什麼意思.
我只想知道.
你到底是什麼意思.
我只想知道.
你到底是什麼意思.
我只想知道.
你到底是什麼意思.
我只想知道.
你到底是什麼意思.
我只想知道.
你到底是什麼意思.
我只想知道.
你到底是什麼意思.
我只想知道.
你到底是什麼意思.

$^{961}$我只想知道.
你到底是什麼意思.
我只想知道.
你到底是什麼意思.
我只想知道.
你到底是什麼意思.
我只想知道.
你到底是什麼意思.
我只想知道.
你到底是什麼意思.
我只想知道.
你到底是什麼意思.
我只想知道.
你到底是什麼意思.
我只想知道.
你到底是什麼意思.
我只想知道.
你到底是什麼意思.
我只想知道.
你到底是什麼意思.
我只想知道.
你到底是什麼意思.
我只想知道.
你到底是什麼意思.
我只想知道.
你到底是什麼意思.
我只想知道.
你到底是什麼意思.
我只想知道.
你到底是什麼意思.
我只想知道.
你到底是什麼意思.
我只想知道.
你到底是什麼意思.
我只想知道.
你到底是什麼意思.
我只想知道.
你到底是什麼意思.
我只想知道.
你到底是什麼意思.

$^{1001}$我只想知道.
你到底是什麼意思.
我只想知道.
你到底是什麼意思.
我只想知道.
你到底是什麼意思.
我只想知道.
你到底是什麼意思.
我只想知道.
你到底是什麼意思.
我只想知道.
你到底是什麼意思.
我只想知道.
你到底是什麼意思.
我只想知道.
你到底是什麼意思.
我只想知道.
你到底是什麼意思.
我只想知道.
你到底是什麼意思.
我只想知道.
你到底是什麼意思.
我只想知道.
你到底是什麼意思.
我只想知道.
你到底是什麼意思.
我只想知道.
你到底是什麼意思.
我只想知道.
你到底是什麼意思.
我只想知道.
你到底是什麼意思.
我只想知道.
你到底是什麼意思.
我只想知道.
你到底是什麼意思.
我只想知道.
你到底是什麼意思.
我只想知道.
你到底是什麼意思.

$^{1041}$我只想知道.
你到底是什麼意思.
我只想知道.
你到底是什麼意思.
我只想知道.
你到底是什麼意思.
我只想知道.
你到底是什麼意思.
我只想知道.
你到底是什麼意思.
我只想知道.
你到底是什麼意思.
我只想知道.
你到底是什麼意思.
我只想知道.
你到底是什麼意思.
我只想知道.
你到底是什麼意思.
我只想知道.
你到底是什麼意思.
我只想知道.
你到底是什麼意思.
我只想知道.
你到底是什麼意思.
我只想知道.
你到底是什麼意思.
我只想知道.
你到底是什麼意思.
我只想知道.
你到底是什麼意思.
我只想知道.
你到底是什麼意思.
我只想知道.
你到底是什麼意思.
我只想知道.
你到底是什麼意思.
我只想知道.
你到底是什麼意思.
我只想知道.
你到底是什麼意思.

$^{1081}$我只想知道.
你到底是什麼意思.
我只想知道.
你到底是什麼意思.
我只想知道.
你到底是什麼意思.
我只想知道.
你到底是什麼意思.
我只想知道.
你到底是什麼意思.
我只想知道.
你到底是什麼意思.
我只想知道.
你到底是什麼意思.
我只想知道.
你到底是什麼意思.
我只想知道.
你到底是什麼意思.
我只想知道.
你到底是什麼意思.
我只想知道.
你到底是什麼意思.
我只想知道.
你到底是什麼意思.
我只想知道.
你到底是什麼意思.
我只想知道.
你到底是什麼意思.
我只想知道.
你到底是什麼意思.
我只想知道.
你到底是什麼意思.
我只想知道.
你到底是什麼意思.
我只想知道.
你到底是什麼意思.
我只想知道.
你到底是什麼意思.
我只想知道.
你到底是什麼意思.

$^{1121}$我只想知道.
你到底是什麼意思.
我只想知道.
你到底是什麼意思.
我只想知道.
你到底是什麼意思.
我只想知道.
你到底是什麼意思.
我只想知道.
你到底是什麼意思.
我只想知道.
你到底是什麼意思.
我只想知道.
你到底是什麼意思.
我只想知道.
你到底是什麼意思.
我只想知道.
你到底是什麼意思.
我只想知道.
你到底是什麼意思.
我只想知道.
你到底是什麼意思.
我只想知道.
你到底是什麼意思.
我只想知道.
你到底是什麼意思.
我只想知道.
你到底是什麼意思.
我只想知道.
你到底是什麼意思.
我只想知道.
你到底是什麼意思.
我只想知道.
你到底是什麼意思.
我只想知道.
你到底是什麼意思.
我只想知道.
你到底是什麼意思.
我只想知道.
你到底是什麼意思.

$^{1161}$我只想知道.
你到底是什麼意思.
我只想知道.
你到底是什麼意思.
我只想知道.
你到底是什麼意思.
我只想知道.
你到底是什麼意思.
我只想知道.
你到底是什麼意思.
我只想知道.
你到底是什麼意思.
我只想知道.
你到底是什麼意思.
我只想知道.
你到底是什麼意思.
我只想知道.
你到底是什麼意思.
我只想知道.
你到底是什麼意思.
我只想知道.
你到底是什麼意思.
我只想知道.
你到底是什麼意思.
我只想知道.
你到底是什麼意思.
我只想知道.
你到底是什麼意思.
我只想知道.
你到底是什麼意思.
我只想知道.
你到底是什麼意思.
我只想知道.
你到底是什麼意思.
我只想知道.
你到底是什麼意思.
我只想知道.
你到底是什麼意思.
我只想知道.
你到底是什麼意思.

$^{1201}$我只想知道.
你到底是什麼意思.
我只想知道.
你到底是什麼意思.
我只想知道.
你到底是什麼意思.
我只想知道.
你到底是什麼意思.
我只想知道.
你到底是什麼意思.
我只想知道.
你到底是什麼意思.
我只想知道.
你到底是什麼意思.
我只想知道.
你到底是什麼意思.
我只想知道.
你到底是什麼意思.
我只想知道.
你到底是什麼意思.
我只想知道.
你到底是什麼意思.
我只想知道.
你到底是什麼意思.
我只想知道.
你到底是什麼意思.
我只想知道.
你到底是什麼意思.
我只想知道.
你到底是什麼意思.
我只想知道.
你到底是什麼意思.
我只想知道.
你到底是什麼意思.
我只想知道.
你到底是什麼意思.
我只想知道.
你到底是什麼意思.
我只想知道.
你到底是什麼意思.

$^{1241}$我只想知道.
你到底是什麼意思.
我只想知道.
你到底是什麼意思.
我只想知道.
你到底是什麼意思.
我只想知道.
你到底是什麼意思.
我只想知道.
你到底是什麼意思.
我只想知道.
你到底是什麼意思.
我只想知道.
你到底是什麼意思.
我只想知道.
你到底是什麼意思.
我只想知道.
你到底是什麼意思.
我只想知道.
你到底是什麼意思.
我只想知道.
你到底是什麼意思.
我只想知道.
你到底是什麼意思.
我只想知道.
你到底是什麼意思.
我只想知道.
你到底是什麼意思.
我只想知道.
你到底是什麼意思.
我只想知道.
你到底是什麼意思.
我只想知道.
你到底是什麼意思.
我只想知道.
你到底是什麼意思.
我只想知道.
你到底是什麼意思.
我只想知道.
你到底是什麼意思.

$^{1281}$我只想知道.
你到底是什麼意思.
我只想知道.
你到底是什麼意思.
我只想知道.
你到底是什麼意思.
我只想知道.
你到底是什麼意思.
我只想知道.
你到底是什麼意思.
我只想知道.
你到底是什麼意思.
我只想知道.
你到底是什麼意思.
我只想知道.
你到底是什麼意思.
我只想知道.
你到底是什麼意思.
我只想知道.
你到底是什麼意思.
我只想知道.
你到底是什麼意思.
我只想知道.
你到底是什麼意思.
我只想知道.
你到底是什麼意思.
我只想知道.
你到底是什麼意思.
我只想知道.
你到底是什麼意思.
我只想知道.
你到底是什麼意思.
我只想知道.
你到底是什麼意思.
我只想知道.
你到底是什麼意思.
我只想知道.
你到底是什麼意思.
我只想知道.
你到底是什麼意思.

$^{1321}$我只想知道.
你到底是什麼意思.
我只想知道.
你到底是什麼意思.
我只想知道.
你到底是什麼意思.
我只想知道.
你到底是什麼意思.
我只想知道.
你到底是什麼意思.
我只想知道.
你到底是什麼意思.
我只想知道.
你到底是什麼意思.
我只想知道.
你到底是什麼意思.
我只想知道.
你到底是什麼意思.
我只想知道.
你到底是什麼意思.
我只想知道.
你到底是什麼意思.
我只想知道.
你到底是什麼意思.
我只想知道.
你到底是什麼意思.
我只想知道.
你到底是什麼意思.
我只想知道.
你到底是什麼意思.
我只想知道.
你到底是什麼意思.
我只想知道.
你到底是什麼意思.
我只想知道.
你到底是什麼意思.
我只想知道.
你到底是什麼意思.
我只想知道.
你到底是什麼意思.

$^{1361}$我只想知道.
你到底是什麼意思.
我只想知道.
你到底是什麼意思.
我只想知道.
你到底是什麼意思.
我只想知道.
你到底是什麼意思.
我只想知道.
你到底是什麼意思.
我只想知道.
你到底是什麼意思.
我只想知道.
你到底是什麼意思.
我只想知道.
你到底是什麼意思.
我只想知道.
你到底是什麼意思.
我只想知道.
你到底是什麼意思.
我只想知道.
你到底是什麼意思.
我只想知道.
你到底是什麼意思.
我只想知道.
你到底是什麼意思.
我只想知道.
你到底是什麼意思.
我只想知道.
你到底是什麼意思.
我只想知道.
你到底是什麼意思.
我只想知道.
你到底是什麼意思.
我只想知道.
你到底是什麼意思.
我只想知道.
你到底是什麼意思.
我只想知道.
你到底是什麼意思.

$^{1401}$我只想知道.
你到底是什麼意思.
我只想知道.
你到底是什麼意思.
我只想知道.
你到底是什麼意思.
我只想知道.
你到底是什麼意思.
我只想知道.
你到底是什麼意思.
我只想知道.
你到底是什麼意思.
我只想知道.
你到底是什麼意思.
我只想知道.
你到底是什麼意思.
我只想知道.
你到底是什麼意思.
我只想知道.
你到底是什麼意思.
我只想知道.
你到底是什麼意思.
我只想知道.
你到底是什麼意思.
我只想知道.
你到底是什麼意思.
我只想知道.
你到底是什麼意思.
我只想知道.
你到底是什麼意思.
我只想知道.
你到底是什麼意思.
我只想知道.
你到底是什麼意思.
我只想知道.
你到底是什麼意思.
我只想知道.
你到底是什麼意思.
我只想知道.
你到底是什麼意思.

$^{1441}$我只想知道.
你到底是什麼意思.
我只想知道.
你到底是什麼意思.
我只想知道.
你到底是什麼意思.
我只想知道.
你到底是什麼意思.
我只想知道.
你到底是什麼意思.
我只想知道.
你到底是什麼意思.
我只想知道.
你到底是什麼意思.
我只想知道.
你到底是什麼意思.
我只想知道.
你到底是什麼意思.
我只想知道.
你到底是什麼意思.
我只想知道.
你到底是什麼意思.
我只想知道.
你到底是什麼意思.
我只想知道.
你到底是什麼意思.
我只想知道.
你到底是什麼意思.
我只想知道.
你到底是什麼意思.
我只想知道.
你到底是什麼意思.
我只想知道.
你到底是什麼意思.
我只想知道.
你到底是什麼意思.
我只想知道.
你到底是什麼意思.
我只想知道.
你到底是什麼意思.

$^{1481}$我只想知道.
你到底是什麼意思.
我只想知道.
你到底是什麼意思.
我只想知道.
你到底是什麼意思.
我只想知道.
你到底是什麼意思.
我只想知道.
你到底是什麼意思.
我只想知道.
你到底是什麼意思.
我只想知道.
你到底是什麼意思.
我只想知道.
你到底是什麼意思.
我只想知道.
你到底是什麼意思.
我只想知道.
你到底是什麼意思.
我只想知道.
你到底是什麼意思.
我只想知道.
你到底是什麼意思.
我只想知道.
你到底是什麼意思.
我只想知道.
你到底是什麼意思.
我只想知道.
你到底是什麼意思.
我只想知道.
你到底是什麼意思.
我只想知道.
你到底是什麼意思.
我只想知道.
你到底是什麼意思.
我只想知道.
你到底是什麼意思.
我只想知道.
你到底是什麼意思.

$^{1521}$我只想知道.
你到底是什麼意思.
我只想知道.
你到底是什麼意思.
我只想知道.
你到底是什麼意思.
我只想知道.
你到底是什麼意思.
我只想知道.
你到底是什麼意思.
我只想知道.
你到底是什麼意思.
我只想知道.
你到底是什麼意思.
我只想知道.
你到底是什麼意思.
我只想知道.
你到底是什麼意思.
我只想知道.
你到底是什麼意思.
我只想知道.
你到底是什麼意思.
我只想知道.
你到底是什麼意思.
我只想知道.
你到底是什麼意思.
我只想知道.
你到底是什麼意思.
我只想知道.
你到底是什麼意思.
我只想知道.
你到底是什麼意思.
我只想知道.
你到底是什麼意思.
我只想知道.
你到底是什麼意思.
我只想知道.
你到底是什麼意思.
我只想知道.
你到底是什麼意思.

$^{1561}$我只想知道.
你到底是什麼意思.
我只想知道.
你到底是什麼意思.
我只想知道.
你到底是什麼意思.
我只想知道.
你到底是什麼意思.
我只想知道.
你到底是什麼意思.
我只想知道.
你到底是什麼意思.
我只想知道.
你到底是什麼意思.
我只想知道.
你到底是什麼意思.
我只想知道.
你到底是什麼意思.
我只想知道.
你到底是什麼意思.
我只想知道.
你到底是什麼意思.
我只想知道.
你到底是什麼意思.
我只想知道.
你到底是什麼意思.
我只想知道.
你到底是什麼意思.
我只想知道.
你到底是什麼意思.
我只想知道.
你到底是什麼意思.
我只想知道.
你到底是什麼意思.
我只想知道.
你到底是什麼意思.
我只想知道.
你到底是什麼意思.
我只想知道.
你到底是什麼意思.

$^{1601}$我只想知道.
你到底是什麼意思.
我只想知道.
你到底是什麼意思.
我只想知道.
你到底是什麼意思.
我只想知道.
你到底是什麼意思.
我只想知道.
你到底是什麼意思.
我只想知道.
你到底是什麼意思.
我只想知道.
你到底是什麼意思.
我只想知道.
你到底是什麼意思.
我只想知道.
你到底是什麼意思.
我只想知道.
你到底是什麼意思.
我只想知道.
你到底是什麼意思.
我只想知道.
你到底是什麼意思.
我只想知道.
你到底是什麼意思.
我只想知道.
你到底是什麼意思.
我只想知道.
你到底是什麼意思.
我只想知道.
你到底是什麼意思.
我只想知道.
你到底是什麼意思.
我只想知道.
你到底是什麼意思.
我只想知道.
你到底是什麼意思.
我只想知道.
你到底是什麼意思.

$^{1641}$我只想知道.
你到底是什麼意思.
我只想知道.
你到底是什麼意思.
我只想知道.
你到底是什麼意思.
我只想知道.
你到底是什麼意思.
我只想知道.
你到底是什麼意思.
我只想知道.
你到底是什麼意思.
我只想知道.
你到底是什麼意思.
我只想知道.
你到底是什麼意思.
我只想知道.
你到底是什麼意思.
我只想知道.
你到底是什麼意思.
我只想知道.
你到底是什麼意思.
我只想知道.
你到底是什麼意思.
我只想知道.
你到底是什麼意思.
我只想知道.
你到底是什麼意思.
我只想知道.
你到底是什麼意思.
我只想知道.
你到底是什麼意思.
我只想知道.
你到底是什麼意思.
我只想知道.
你到底是什麼意思.
我只想知道.
你到底是什麼意思.
我只想知道.
你到底是什麼意思.

$^{1681}$我只想知道.
你到底是什麼意思.
我只想知道.
你到底是什麼意思.
我只想知道.
你到底是什麼意思.
我只想知道.
你到底是什麼意思.
我只想知道.
你到底是什麼意思.
我只想知道.
你到底是什麼意思.
我只想知道.
你到底是什麼意思.
我只想知道.
你到底是什麼意思.
我只想知道.
你到底是什麼意思.
我只想知道.
你到底是什麼意思.
我只想知道.
你到底是什麼意思.
我只想知道.
你到底是什麼意思.
我只想知道.
你到底是什麼意思.
我只想知道.
你到底是什麼意思.
我只想知道.
你到底是什麼意思.
我只想知道.
你到底是什麼意思.
我想我們今天回到.
最基本的內容.
我們看回一篇經文.
有關創世紀的.
上帝叫你勞動的經文.
他說什麼呢.
地必為你的猿固受咒助.
你必終身勞苦才能從地裡得吃.

$^{1721}$地必給你長出荊棘和泥澤來.
你也要吃田間的菜蔬.
你必汗流滿面才得糊口.
直到你歸了土.
因為你是出於土而出的.
這是上帝對於.
阿當和夏娃的懲罰.
因為犯罪墮落的緣故.
所以阿當夏娃面對這樣的情況.
是什麼呢.
就是你要有勞動才能得吃.
以前沒有的.
犯罪前阿當夏娃.
基本上人類的狀況.
你不需要勞動.
你都能夠糊口.
你不需要賺錢.
你都能夠得到食物.
所以他所說的不是純粹工作辛苦了.
而是說你沒有錢.
或者你沒有資源.
如果你有資源.
你不需要賺錢.
你就不需要工作.
所以我想重點就是.
經文不是嘗試去.
給予一個工作的神聖理由.
相反.
它不是一個賦予工作.
一種神聖意義的經文.
它只不過是間接地.
消極地解釋.
為什麼你要工作.
為什麼要工作.
先問問現場觀眾.
工作有什麼原因.
最重要的是什麼.
就是錢.
工作的定義是什麼.
你不需要辦公室.

$^{1761}$你不需要老闆.
你不需要任何放假.
有錢就行.
有錢就是工作.
沒有錢不是工作.
沒有錢就叫做義工.
沒有錢就叫做興趣.
最能夠定義的工作就是錢.
所謂工作就是.
你有回報的一個交易.
今天我想講的不是市區.
而是工作本身的定義就是這樣.
工作就是一個.
讓你能夠得到金錢回報的一件事.
所以黃子博講得很好.
就是賺錢.
賺錢或者以前固定的叫戶口.
戶口或者賺錢.
這正是工作的核心定義.
如果沒有錢的話.
這不是工作.
工作就是賺錢的一件事.
所以基民說為什麼要賺錢.
為什麼要工作.
因為你不能夠隨便得食.
你需要透過勞動.
能夠換取.
今天所講的金錢.
以前不是.
換取金錢才能夠得食.
所以就叫做賺錢或者戶口.
所以賺錢或者戶口的工作.
有沒有意義呢.
或者它是否必須有一個屬靈的意義呢.
我就問這個問題.
或者是什麼情況.
所以劉基民告訴我.
不是去嘗試去給予工作一個定義.
或者一個屬靈的定義.
而是去反面或者間接地.

$^{1801}$解釋為什麼要工作.
因為有缺陷.
因為你不是什麼都有.
因為你沒有資源.
因為你家裡沒有東西吃.
所以你需要工作來換取資源.
這個就是世界的法則.
這個也是當紅夏娃犯罪之後出現的法則.
所以簡單來講.
這個基民告訴我.
當人類墮落之後.
這個大地被咒坐之後.
工作就是間接地隨之而生.
沒錯.
工作未必一定是一個上帝直接的懲罰.
但是工作是一個間接上出現的東西.
因為上帝吸予大地被咒坐之後.
你會出現scarcity的問題.
你有資源短缺的問題.
因此你要間接地工作.
找食物戶口來換取資源.
所以關係就是這樣.
因為犯罪緣故.
你不是必然有資源.
因此你要工作.
所以你才有資源.
所以當然有錢的人是不需要的.
因為他本身有資源.
他不一定工作.
但是很多人都是這樣.
他需要工作來找食物.
所以這就是我們所面對的情況.
你要工作來換取金錢.
所謂工作的時候.
我之前也提過.
工作永遠都是帶著工具性目的性.
為什麼你要上班.
因為你為了某些目的.
就是為了賺錢.
不是說你沒有使命.

$^{1841}$不一定是你沒有屬靈意義.
你是需要來找食物賺錢.
所以當你做一件事.
為了賺錢的時候.
你會面對很多問題.
你會做一些你不想做的事.
做一些很無聊的事.
做一些你覺得沒什麼貢獻的事.
工作是否有很多貢獻呢.
我想十個裡面的工作.
其實都是為了幫老闆賺錢.
意義就是幫老闆賺錢.
所以就是這樣.
不否定工作是有意義的.
但是我們不能夠將這些變成定義.
不能夠說工作的定義就是能夠幫助人.
因為很多時候工作都幫不了人.
幫了你老闆賺錢.
就是那個人.
老闆就是那個人.
所以我想我們要讀穿這件事.
很多工作都不一定有貢獻和意義.
這個情況是荒謬的.
但正正就是因為人類墮落之後的情況.
因為大罪的緣故.
所以我們面對一個很荒謬的情況.
因為純粹是來找食物.
做一些很沒有意義的事.
一些很不合理的事.
但你能夠得到回報.
所以天堂有沒有工作呢.
天堂是沒有工作的.
意思不是說你不用工作.
而是你不需要為了賺錢換取東西去工作.
所以天堂是有無限資源.
或者沒有stress的問題.
所以我們工作裡面不需要來到去.
因為想找資源去做一些不想做的事.
這個就是天堂.
所以大家很羨慕我們的社會.

$^{1881}$這個就是我們來坦白的事情.
我認為工作是大地被咒坐之後人類的狀況.
因為資源短缺.
你被逼要做一些純粹賺錢的事.
當然不否定.
不否定你的工作有意義和使命.
這樣說不是否定一件事.
但這不是必然的.
我們不要把這看為工作的定義.
它未必一定是上帝對你的呼召.
未必是一些有意義的事.
未必是一些你很想做的事.
當然有人會這樣做.
我做一個老師.
我真的很想做一個老師.
我做一些我想做的事.
但這不是一個定義.
你不能將它看為定義.
因為很多人不是這樣.
當然我們盡可能都想做這些事.
既然我們都要上班.
為什麼不找一些你想做的事.
為什麼不找一些我自己呼召的事.
為什麼不做一些我有意義的事.
這絕對是理想.
但我們不能將理想成為一個工作的定義.
所以我嘗試分開.
不用路德那套看為天職.
或者將工作看為本身有意義的事.
那怎麼辦呢?好像很灰.
問題是.
雖然我們說工作不一定有意義.
工作不一定是上帝對你的召命.
但我想說我們的人生是可以有意義的.
我們的人生是可以有上帝的召命和呼召的.
所以我們有些人.
很多不同的情況.
有些人的工作是上帝對他的呼召.
他做了一輩子.
這是其中一個例子.

$^{1921}$有些人或者你.
上帝對你的呼召和理想.
不是在工作里.
我的工作是為了賺錢.
9至9晚51至55.
我出賣我的靈魂.
然後我賺錢回來.
我晚上可以做我想做的事.
太多這些例子.
我想說這個不是一個.
我認為這個不是一個不正常的情況.
如果你說工作是一個就在後間結果的時候.
你早上九晚五做一些沒有意義的事.
其實你就會明白為什麼.
因為這是一個不可領略的情況.
這個世界的狀況.
但你仍然要尋找你的理想.
尋找你自己想做的事.
或者你活出你生命的意義.
即是說你的工作不一定能夠實踐你當下意義的時候.
你仍然要相信和知道.
我的生命是有意義的.
我的工作未必能夠幫我這樣做.
但它能夠讓我有錢.
能夠做其他事.
所以我們嘗試將工作的意義和生命的意義來區分.
你的生命意義未必一定是實踐在工作里.
重復一次.
如果是是很好的.
但如果不是的話.
不代表你不正常.
或者你的生命沒有意義.
最近我看了一部Netflix的電影.
先講完這裡.
其實有一個很寶貴的字眼.
就是Voluntary Work.
叫做義工.
如果工作的定義是有錢的時候.
你會發現人生里有很多工作是沒有錢的.
但你很喜歡做.

$^{1961}$那些不被賠償的工作是很辛苦的.
你很喜歡去行山.
行山是很辛苦的.
但你是願意去做的.
教會的敬拜服侍.
你很辛苦來到去預備.
但你仍然覺得很滿足.
所以我想說.
工作不一定是用辛苦來定義工作.
我用賠償和不賠償來定義工作.
但一些不賠償的工作.
Voluntary的工作.
你仍然很樂於去做.
譬如你是媽媽.
媽媽是一個最強大的Voluntary Work.
你不收錢的.
但你是會願意去做的.
很多的事情你都會願意去做.
所以反而我們生命上的意義.
往往很多人都是在Voluntary Work裡面.
可能我晚上.
或者Batman.
Batman晚上做Voluntary Work.
那些不收錢的.
但很有意義的.
平時做Bruce Willis的時候.
他賺了錢.
但這不是他的生命意義.
明白我的意思嗎.
很多人都是這樣.
工作以外才是真正的生命實踐.
所以我想再重復一次.
你可以正式做意義的事.
但未必一定是這樣.
所以更加要強調.
義工其實都很重要.
重點是很多事情都辛苦.
但你想做的事是否重點.
是否收費都不是那麼重要.
而是你自己想做什麼.

$^{2001}$所以往往你為了支持.
你做Voluntary Work.
你會做一些沒有意義的工作.
近期我看了一部戲.
原來沒有了.
沒有了一張圖片.
但我看過近期Netflix有一部.
叫The Wonderful Story of Henry Su.
沒有了圖片.
上回給大家看.
故事是講一個很有錢的人.
基本上他都是無憂無慮.
不需要工作的有錢人.
他無端端看了一本書.
是一些印度瑜伽大師.
故事是講他有些方法.
可以讓他看穿背後的東西.
可以透視.
有錢人花了好幾年時間.
不斷去練透視的方法.
練成功了.
他就可以去賭場里.
透視看啤牌背後的牌.
那一刻他才開始他的生命.
記住他不是為了賺錢.
因為他已經有很多錢.
所以他才開始做他生命意義的事.
他每天都去賭場里.
賺夠好.
賭錢看穿牌.
每次賺兩萬英鎊就走.
怕被人打.
每天都很勤力.
很日常地賺兩萬五千英鎊.
然後就離開.
把錢捐給人開兒童院和醫院.
他從那時起才開始上班.
他每天都很勤力地上班.
在賭場里賺錢.
然後把錢捐給別人.

$^{2041}$我覺得這是一個尋找生命意義的故事.
一個嘗試用一個角度.
去理解什麼是工作的故事.
其實我們工作的意義.
本身更加要問的是你的生命意義在哪裡.
工作是你實現生命的其中一部分.
它可以是一個很好的部分.
但是那時候.
voluntary work才是你生命裡面更加重要的一部分.
所以我們嘗試這麼說.
上帝給我們的照明.
未必一定在工作裡面.
但是我們是有照明的.
上帝給我們的照明是有工作的.
那些工作未必是要付的工作.
但是是工作來的.
一個你可能會流汗.
你要辛苦的工作.
但是你願意做的工作.
所以我嘗試重寫這幾點.
如果我們這麼看的時候.
什麼是工作呢.
Work is an indirect result of the form.
它未必是直接因為墮落的緣故.
但是一個間接的原因.
因為墮落的緣故.
所以你的工作.
你是一個很不理想.
很荒謬的世界裡面.
做一些很不想做的事情.
所以工作裡面的那個基本質.
Work is a place to make money.
這個是名正言順.
都有道理的定義.
工作就是為了make money.
好讓你能夠生活下去.
或者你的子女能夠生活下去.
第三.
一些神聖的呼召.
不比其他的職業更加崇高.

$^{2081}$但是想說一件事.
Work is not necessarily a calling.
工作不一定是calling來的.
但是你的生命是calling的.
但是工作不一定是calling的實踐.
第四.
Redemption extends to all of life.
only in this sense including our world.
因為我們的生命是被救贖的時候.
所以工作才是救贖的部分.
所以你要看的不是工作.
而是什麼.
而是你的生命.
所以我會改後面那幾句.
There are indications that some of our lives will be present in the new heaven and the new earth.
你的生命會彰顯新天新地的模樣.
就此言言,你的工作才會這樣彰顯出來.
We are called to glorify God in our life.
所以不要把定義狹雜地放在工作里.
而是放在你的生命里.
因為你的生命是有意義的.
要榮耀上帝的.
是值得彰顯上帝的生命的.
這樣來看你的工作.
所以更加宏觀些,闊些,遠些去看你的生命.
最後我想說一些工作倫理.
我想其實有很多很困難的倫理.
公司賺錢有些不太誠實.
或者你所做的東西似乎沒有什麼特別的貢獻.
或者你遇到很多倫理問題.
我想說,今天處理不了的.
你可以嘗試在小組裡面打聽一下.
這些兩難.
工作的開單似乎又不太正確.
你發現自己在幫JPES賺錢.
我不知道.
很多這些困難.
今天不說這些.
如果我們說工作倫理是一個基督徒倫理.
就是說如果工作是一個你本身做人應該的倫理.

$^{2121}$其實工作倫理也不是太複雜.
你怎麼做人,就是在工作裡面怎麼做工作.
所以我嘗試寫一些比較無聊的三點.
工作有什麼倫理呢?基督徒.
第一,就是要負責任.
就是做事要負責任.
第二,有交代.
第三,Facebook和WhatsApp.
其實沒什麼特別的.
我看到有基督徒是基督徒.
但是不Facebook和WhatsApp.
我也是.
基督徒,但沒什麼交代.
香港就是這麼簡單.
既然工作是一個合約,是一個交換.
那你就應該做好這件事.
就是這麼簡單.
不說你是不是要在公司裡面唱歌.
還是怎麼建設上帝.
但請你先做好工作.
這個就是基督徒最基本要做的事.
應該稱職地負責工作的責任.
這些事,走的時候記住抹上水餅.
做好事才讓別人做.
這些很簡單的事.
可能太多,是基督徒來的.
但做事是要射箭的.
或者不斷地推來推去.
所以做好這些基本的倫理.
就是做好工作里應該做的表現.
再在我們的生命裡面彰顯我們的信仰.
不是說你在公司裡面唱歌不對.
是要做的.
但沒有一個工作就是為了傳福音的倫理.
因為你的人就是傳福音.
所以你自然在工作裡面會建立基督.
所以沒有一個很具體的工作倫理.
因為每一個工作倫理都是我們生命倫理的處境.
當然有很多很特別的工作處境.
像剛才說的跑數,說謊是很難處理的.

$^{2161}$但我們人生裡面的倫理其實都是很簡單的.
你既然做了基督徒.
回到達哥第一科裡面.
基督徒的呼召就是去見證耶穌.
所以我們基本上是懷著這種召命和身份去做人.
不過你會上班的.
你會做一些不同的事.
你會遇到不同的人.
所以這樣去理解的時候.
我覺得沒有一套很特別要讀的工作倫理.
而是如何做一個人.
你會很簡單地去想.
當然我也有很多問題要談.
特別是大家有很多奇能合智.
大家會問工作裡面遇到不同的困難.
怎麼辦.
當然我們可以談一下.
但我想更重要的是.
做一個好的基督徒之前.
大家要做一個好的人.
在職場裡面.
做一個正常的老闆.
或者正常的員工.
你不一定要做得特別好.
但請你負責工作裡面的責任.
等等這樣.
我們有些對話空間.
看看大家有沒有問題想問.
有關工作上的事.
我先叫上一位.
潘sir.
(潘sir的工作人員).
已經上班了.
很羨慕你的工作.
飛來飛去.
應該對於各位.
有沒有人今天不用上班.
有.
你是特別一點.
剛剛下班.

$^{2201}$就來到這裡.
應該很累了.
如果不用上班.
大家會不會來呢.
都會來的.
很多人下班就很累.
就去吃喝快樂.
去做回饋.
但有時我自己在小組討論.
如果上班這麼累的話.
我會不會來呢.
如果上班這麼累的話.
會不會選一份工作.
不要那麼累.
很多小組隊員說不要.
所以今天應該.
關於上班的題目.
你們會想瞭解些什麼.
剛剛聽完之後.
有沒有人帶了問題想討論.
應該很多.
很多不同工作上的難題.
都會答一下.
大家可以回小組裡面談一下.
今天也可以和大家談一下.
還是幫朋友問吧.
有朋友的.
(笑).
(看電視).
看下來.
(看電視).
後面有一個.
(看電視).
Hello 潘Sir.
我是John.
關於上班的時候.
我突然想起.
剛才你們說工作倫理.
我想問.
以前教會教導.

$^{2241}$如果你不要在賭場上班.
因為賭場.
沒有什麼意義.
或者負面的意義.
我之前認識一個.
大兄姐妹.
她在那地方.
找工作比較難.
外國有一間.
是賭場.
的機構.
請她.
當初我和大兄姐妹.
激烈地討論.
好像幫.
賭場的.
公司工作.
不是一個好的工作倫理.
但當時她也沒有什麼選擇.
因為她找了幾個月工.
當時也比較難找工作.
我想知道.
你們會不會.
不建議還是你們怎麼看.
她不是直接做荷官.
但她幫公司設計IT系統.
你們會怎麼看呢?.
也有些問題.
你試試回答.
我沒有在外國.
我也不太知道.
我就以.
香港賽馬會為例.
我曾經回應過一些問題.
也有些教會.
大兄姐妹.
也關於這些職場的.
倫理選擇.
我就和她說.
如果.

$^{2281}$真的是.
馬會染指的.
項目的話.
我想香港很多工作都不能做.
香港賽馬會.
徵收了.
博彩稅之後.
其實她拿到很多錢.
也做很多社會服務.
就以我們後面.
那幾間NGO.
去做紅土區的社區服務.
如果你說.
凡是馬會的錢就不打工.
其實很多人都失業.
這樣說好像.
不對題.
但其實.
我會用另一個角度去看.
雖然那些錢.
好像覺得不應該這樣賺.
或者這樣.
好像有些人.
挑釁生態,博彩.
但其實這個做法.
是資源重新分配.
因為那些錢拿出來.
是目標社會福利.
讓一些低收入.
受助人可以在.
金錢上.
得到重新分配之下.
可以得到他的生活保障.
我覺得這個也是.
工作的輔助.
我覺得這件事.
是一個.
可圈可點.
用倫理的字來看.
倫就是關係.

$^{2321}$理就是分辨.
在關係之下.
你如何作出分辨.
你想不想和他建立一個關係.
你想不想在這個關係中染指.
如果你不想在這個關係中染指.
你做了分辨就不做.
你不用在馬會工作.
那些錢.
你可以做其他都可以.
但如果你覺得.
這個關係不是.
那麼差.
或者那麼要排除.
在分辨程度下.
你把資源重新分配.
可以用回那些錢.
在正確的社會福利渠道.
當中.
用得其所.
做回應該有的決定.
我會這樣去看.
我就覺得.
馬會和.
行生是不同的.
賭場是行生.
你幫他賺錢.
我都覺得可以.
不是可以.
你不應該問我.
因為太多不同的處境.
我們不應該.
這樣做就算對或不對.
起碼不是JPES的.
個案.
JPES很明顯是騙人的.
你都做就很明顯是騙人.
但很多情況下.
其實這個是倫理問題.
我們如何去判斷倫理.

$^{2361}$我們很不容易.
不應該隨便.
去一刀切.
這樣就不對.
特別澳門人.
我都講過.
我那時候在澳門教會.
有一堂是星期日七點半崇拜.
知道為什麼嗎.
因為是方便人們崇拜後.
可以上班.
如果你是澳門人.
基本上你是沒得選擇的.
大部分人都是在賭場工作.
所以很難說.
一刀切就算對或不對.
這樣的問題.
是不應該這樣問.
所以是沒有答案的.
但我覺得.
我們作為基督徒.
在不同的生命里.
有不同的處境來判斷.
你都說了.
要賺錢.
你沒錢.
小孩沒得吃飯.
或者有很大差別.
做到可以就不做.
很多這樣的處境.
多於問應該.
做還是不做.
因為太多不同的故事.
太多不同的處境去處理.
所以除非很明顯.
做賊.
或者追賊.
很明顯是不對的.
但太多這些很模糊的例子.
我們不能夠.

$^{2401}$答到這樣就對或不對.
所以都說.
在一個墮落的世界里.
幫韓生賺錢.
但沒辦法.
這個不是呼叫.
不需要呼叫.
純粹為了賺錢.
網上那個不知道是否問你.
為什麼買散水餅.
是工作倫理的一部分.
什麼邏輯.
因為你做了一個應該要做的事情.
工作裡面.
很多例子.
做了很久的工作.
做了工作的人.
應該做的事情.
我不知道是不是買散水餅.
現在可能不是買散水餅.
多了不同的東西.
另外就是.
問關於.
是.
如果可以.
有人.
過了.
過多幾年.
就會轉工甚至轉行.
是不是代表神在不同的年和日.
當中有不同的工作.
照明.
我都說工作沒有照明.
但你想說.
生命有照明.
所以為什麼會轉工.
你不能.
用工作照明.
就解釋不了.
呼叫你做這行.

$^{2441}$但人生有上帝.
給你的呼叫帶領.
我想說的例子.
用這套工作神學.
是能夠解釋的.
上帝對一些移民了的基督徒.
有一個心意和人生方向.
所以你不做香港的護士.
去那裡就做其他的事.
拿著茶.
但那個照明不是在你的工作裡面.
所以我就說.
你轉工不是上帝給你轉了照明.
而是你本身.
上帝是你的照明.
所以就不需要將工作和照明.
去拍得那麼緊.
但我們人生裡面.
有不同的方向.
有時這十年就做這件事.
二十年之後就做這件事.
這就是我們人生裡面很多不同的生命方向.
這是我們值得去尋索的.
甚至有些.
擺明就是.
一個曠野.
這五年裡面.
很多人都沒事做.
或者很迷茫.
這不代表上帝對你沒呼叫.
但這種迷茫和曠野.
其實都是一個過程.
所以未必每一刻.
都是很明顯有意義的事.
但我們知道我們的旅程.
一個progress.
關於轉工這件事.
我曾經在教會.
散會的時候.
被家長問我.

$^{2481}$他問.
潘先生你帶開的職稱.
多不多轉工.
我說也有.
你也支持他們經常轉工嗎.
我說不是支持.
不過現在很難一份工作.
做得久.
我家裡的經常轉工.
這些情景我都不是一次見到.
不過我都跟家長說.
調整一下工作觀念.
因為上一代.
或者我也是97年前.
已經入職的人.
我們的工可以打過世.
97年前入職的炒不了你.
但我說現在不同了.
2000年.
03年和06年.
其實香港換了幾次.
簽contract的文化.
其實就算是政府工.
都是三年簽一次.
簽了兩次三年才有perm.
整件事是複雜了很多.
有機會我都跟家長說.
不要再用舊的觀念.
去說子女經常轉工.
好跳.
或者為什麼不忠於.
打一份工.
都不容易讓家長明白.
不過他們都知道.
打工辛苦.
我們年輕人都捱的.
言下之意.
好像自己子女.
不是很捱.
令父母之間的張力.

$^{2521}$都很大.
所以都給在座職稱.
知道如果你爸媽.
暗暗的你.
都可以跟他們調整.
觀念.
或者有機會.
跟你爸媽聊天.
今天這個talk.
能不能解釋.
你現在的處境.
能不能合理化.
或者叫做.
解釋現在做的事情.
令大家上班都很無奈.
剛才說的都是為了錢.
前面有.
前面都有.
最前面那位.
我想問.
很多一般人的想法.
上班就是.
有一天不用上班.
我記得John有一次.
有提過.
如果我們的夢想.
就是有一天在海邊小屋.
那天才達成就很大件事.
應該是一個.
持續在實踐中.
如果工作是這樣.
是否也可以呢.
我們上班.
都是可以為了將來.
不用上班.
第二個我想問.
在現在現今的社會.
其實工作是.
都有少少建構.
我們自己身份的一部分.

$^{2561}$如果我們不用工作.
我們用照明.
這個代入.
會是一件怎樣的事呢.
我們用照明來建立自己身份.
很好的問題.
因為我的工作.
很多時候都不用上班.
我的工作是.
如果說放假的話.
我在十二月.
我沒有出去跟別人說.
基本上.
我由六月尾放假.
放到九月頭.
十二月整個月.
都是放假.
之類的.
今天可以來做這些.
我想說是很可靠.
很多人都是這樣.
很多人是這樣.
他們不用很多obligation.
要去做很多你不想做的事.
很自由.
但你那個暑假.
或者你那個.
不用上班也好.
其實你想做事的.
只不過不是pay.
你會想收拾一下屋.
或者想無聊到找些事來搞一下.
甚至乎我打機.
打機都很辛苦.
打得好其實都不容易.
你要做.
如果是一個.
你要去做的事.
人生裡面你要做的一些事.
不過不是pay.

$^{2601}$例子.
有錢人本身都不用做事.
不過他找到自己的意義.
就好像我上班都在賭場賭錢.
賺錢回來.
所以重點.
就是這樣的意思.
當你能夠賺到什麼創造自由.
即是退休之後.
你其實仍然.
要去想下照明.
想做些什麼.
只不過你現在不需要賺錢.
去令自己養活.
你不需要這個問題.
但你仍然要問我能不能做些什麼.
所以當你賺錢.
賺到某個位置你不用上班的時候.
你就要問你人生有什麼意義.
想做些什麼.
所以都是那個做事的字.
但定義就不是賺錢的做事.
就是voluntary work.
很多人.
很多有錢人或者很多人.
都仍然找這件事.
他不需要有財務上的壓力.
但他都要問我做些什麼.
不然真的很無聊.
所以這件事.
不一定需要在你工作.
退休之後.
而是在你公寓之外.
你可以有些什麼做.
所以回到問題.
你的身份是什麼.
值得問.
我在上帝裡面.
想做些什麼.
如果上班不是的時候.

$^{2641}$某個公司裡面的會計人.
不是我的照明.
但我人生裡面.
有些什麼想做呢.
可以是一個照顧子女的媽媽.
可以是一個教會侍奉的人.
可以是一個.
對於游讀.
很有熱忱的人.
很喜歡行山的人.
很喜歡養花的人.
這些不是叫懶.
不是叫射波.
而是你的生命裡面.
上帝給你的方向就是這些.
這些不是在工作裡面.
過去十年裡面.
太多人告訴你工作就是你的照明.
而太多人不是.
其實大家都可以實踐自己的生命和理想.
我想身份正正是.
可能在工作以外裡面.
去尋索你的身份.
因為今天太多複雜的東西.
我不再是一個天職.
做一個漁夫或者消防員.
但我的天職.
仍然是一個職位.
我的照明是做什麼呢?.
可能是複雜的.
可能每年去泰國短川.
一個短川發燒友.
我就做這件事.
工作只是我能夠去生活.
值得想這件事.
每個人都有這件事.
生命才有意思.
這個問題是.
問得好.
但都相當複雜.

$^{2681}$因為牽涉到工作.
身份和照明三個很大的詞.
如果混合在一起.
其實我都在想有什麼例子.
其中一個例子就是.
以前認識了一些在北區.
做農夫的.
工作的人.
他本身不是做農夫.
但我認識他的時候.
他工作是農夫.
但他本身的職業是律師.
和精算師.
但.
他的身份.
現在是農夫.
但他又不是不做.
律師的工作.
因為他本身是律師的專業.
他幫了很多.
北區的農民.
去看文件.
以致他不要被人.
閹割了地.
和權益.
精算師就幫人.
看了一些數字.
如果收地的時候.
數字應該是這樣計算.
所以你問我.
那是他的工作嗎?.
那不是他的工作,是農夫.
但他的身份是律師.
他的照明是將他的專業.
成為一些.
可能沒有相關能力的人.
能夠得到應有的報酬.
或者是他的待遇.
所以你問我三件事.
其實來說.

$^{2721}$沒有遺和感.
大家是可以融合的.
但未必用.
像John所說.
用傳統的工作定義身份.
一定要做到這件事.
才會發揚光大.
但現在工作層面是多.
我覺得不要分得那麼細.
我自己也是.
我自己有醫學訓練.
其實我現在每個星期.
見很多弟兄姐妹的時候.
很多時候都會問我去哪裡做檢查.
還有哪裡比較划算.
應不應該這樣做.
其實我每個星期都會回答不同的.
這些相關的查詢.
但我已經離開醫院十幾年了.
但這些就是我所懂的東西.
對我來說.
我的工作是教書和木匪.
但我的身份仍然是有之前工作的能力.
所以我覺得這件事.
不一定要分割.
其他弟兄姐妹呢?.
有沒有在前面?.
(有).
讓我變聲.
因為我認識一個中學同學.
他仍然是五入歧途.
他現在的處境是.
他無法選擇自己做什麼工作.
因為他已經離不開那個位置.
他可以如何面對他這個生命的處境呢?.
可能跟這個話題未必有直接關係.
但說起工作.
有些人是無法選擇.
他的工作是幫他糊口.
但他的工作內容.

$^{2761}$可能被他的上司強行放進去.
即是他在做臥底嗎?.
臥底不會說自己是臥底的.
開玩笑的.
即是如何面對?.
譬如他跟黑社會有關.
他現在離不開了.
他無法還錢.
做一輩子工作都無法還錢.
可能今天去幫人開船.
第二天幫人派牌.
第三天幫人去游泳.
是這樣的.
無法選擇.
他的工作.
他的生活被困.
他如何面對?.
他有無辦法尋求幫助?.
簡單來說.
很多苦難的人.
其實很多這些人.
這些都是很大的話題.
我想都不只是工作的問題.
而是當人生面對這樣的情況下.
如何去面對?.
我最近看了很久的那部.
《大紅燈和菊菇掛》.
不知道大家有沒有看過.
很多年前.
那部電影基本上是說.
一個大學女孩.
因為父母無錢.
就嫁給了一個有錢人.
人生裡面就是做人家的四太太.
這部電影是說四位太太的鬥爭.
最後她瘋了.
就是這樣.
人生裡面很多人會面對.
這種情況下.
他的工作就是做人家的太太.

$^{2801}$特別是以前的年代.
文初時期.
在封建裡面做太太.
沒有什麼自由.
所以很多人都是這樣的情況.
你問我.
我都是回到耶穌給我們.
生命封城這件事.
似乎我們.
如果更加想強調.
就是說你的生命.
不是被你的工作限制住.
這個生命.
其實都可以在當中.
不能夠很精彩.
但起碼是能夠幫助我們.
在工作以外.
能夠有些出路.
是很困難的.
不會有任何好結局.
但實際上.
如果他能夠信耶穌.
我相信在這樣的情況下.
他仍然可以活得有意義.
或者快樂.
這就是我們.
所說的福音的意思.
在困難中仍然能夠活得.
有一點順利給我們的祝福和快樂.
是,扭轉不了這些.
所以就說這個工作.
這個關係.
因為我要賺錢.
所以被迫做這些.
真的不能夠稱之為照明.
不能夠稱之為順利給我們.
但我們就說.
嘗試去推高一格.
不在工作里賦予意義.
但在人生里.

$^{2841}$可以有意義.
這樣說是很風涼話.
這樣說好像很好.
但我唯有這樣理解.
我有個案.
不過不能在YouTube說.
所以我就不說那個.
那個就真的近身一點.
不過我說那個.
之後再和你談.
但暑假.
暑假就和牧者去了台灣.
我在講道也說過.
我們去了一個紅燈區.
在台北.
西門町後面的.
一個紅燈區.
探一個侍工.
那個侍工.
最後我們買了一本書回來.
我自己剛看完那本書.
叫《茶室女人心》.
我自己在講道的時候也看過那本書.
裡面記錄.
茶室的阿姨.
她的大半生.
前幾十年的人生.
其實也像你這樣說.
她從小就被人賣了.
在記載上.
或者養大了就開始陪茶.
整件事.
她人生是一片黑.
她沒得選擇.
但當孫教士去探訪她們.
和她們說.
其實你可以選擇這件事.
她們也不選擇.
她覺得.
她們的人生也是這樣.

$^{2881}$也沒什麼出路.
但孫教士一而再再而三.
去探訪她們.
她們才覺得.
其實也可以試試.
她可以試試.
於是就成為.
生命當中的一點光.
然後慢慢讓她自己發覺.
原來人生也有選擇.
如果聽你剛才講.
你朋友的個案.
她不是沒得選擇.
但她可能不想選擇.
又或者.
她覺得一選擇就要.
轉變她整個生活形態.
可能她又不想.
她為生了很久.
也是靠這個方式.
可能她轉回.
相對我們這些普通人.
她覺得太悶.
但我覺得仍然有一個網絡.
她也和你有聯繫.
你是信耶穌的人.
你就繼續給她一點光.
她希望朝著光走.
我覺得這也是.
我們在她生命當中.
的一個幫助.
所以我覺得這件事.
未必是聯繫到工作.
反而是她的人生觀.
和她的取態.
對福音的回應.
或者對於上帝的揮手.
會不會影響.
這些位置.
剛才提到.

$^{2921}$那些自願工作.
其實.
假設有個人.
對某些事物.
很有興趣.
當是去旅行.
但那件事.
其實未必.
雖然她很喜歡做.
但未必一定是能幫助人的事.
如果在這種情況下.
其實.
是否代表那件.
自願工作.
未必一定是.
照明的層次.
會不會還有其他東西呢?.
我想.
照明有幾個層面.
第一堂就說.
基督徒本身就是一個.
帶著照明的名詞.
見證耶穌.
這是一個照明.
剛才也說得對.
其實也叫做興趣.
興趣就是去旅行.
不太關我自願工作.
或照明的事.
不過我想.
剛才我說的自願工作比較闊.
不是純粹說做義工.
而是有些事.
是會放一些心思.
甚至是.
一些努力的事.
未必說得很偉大.
不是做義工.
但可能是.
我喜歡行山.

$^{2961}$經常約人行山或是搞手.
搞手也有很多事要做.
約人來.
即是說.
這些小事.
都值得我們實踐.
是否不做.
幫人的事呢?.
當然不是.
基督徒的身份就是.
幫人.
這是我們要做的事.
所以是不同層面的事.
一個是實踐自己生命中.
想做的事.
可以是興趣.
可以是你想做的事.
另一個是很大的命令.
就是你需要.
去幫助人.
是嗎?.
有些是基督徒.
去見證耶穌的事.
所以有幾個不同層面的事.
我們不會只做這件事.
但我覺得一個人.
也沒理由只是侍奉.
也有些興趣.
有些你喜歡做的事.
這些都重要.
所以我不會覺得一個人.
只做義工幫人就夠了.
不幫就不好了.
這幾件事都是很重要的.
我沒什麼想法.
因為做義工.
我會去做的.
反而我兩個兒子.
他在學校也有.
義工的事數.

$^{3001}$我不知道你學校有沒有.
我經常都擔心.
其實做十個小時.
十五個小時.
三十個小時.
其實對於想達到的目標.
我其實也很懷疑.
我覺得做義工.
應該是讓你自己去探索.
你自己是否想達到什麼目的.
所以我覺得.
很多義工其實都不是.
表面上的義工.
好像作品裡面.
很多都是義工.
不付費的東西.
我不覺得是義工.
這些都是想做的事.
父母也是一樣.
父母也是義工.
但你不會問.
為了做義工就讓你.
人裡面很多理想.
都是義工的性質.
不付費的.
但不是那些很表面的.
工程團的義工.
那些反而是很表面的東西.
很多很深層次的.
不是被稱為義工.
但不付費的東西.
後面有一個.
前輩.
我有一個很無聊的問題.
哪一個發明瞭上班?.
第二個就是.
義工和業餘是怎樣分辨的?.
他是義工還是興趣?.
有些人打遊戲是興趣.
但電競也是一個職業.

$^{3041}$誰發明瞭上班?.
應該是李老闆吧.
應該是人類吧.
人類發明瞭上班這個制度.
簡單來說就是.
以前原始人打獵.
是否叫上班呢?.
不是,只是為了賺錢.
後來發現.
不如你幫我照顧小孩.
我給你一隻羊.
那就可以上班了.
應該是這樣吧.
但應該不是上帝.
因為上帝只是給我們.
資源有限的狀況.
其他人的待羅.
所以我說不是上帝.
因為上帝沒有把工作.
成為他自己對創造心意的一種心意.
那與義工有關的問題.
我也不懂得回答.
因為我剛才說的問題是一樣的.
義工不是那些義工.
只是不付費的意思.
所以很多時候人生里.
有些是付費的,有些是不付費的.
有些事是不付費的.
可以是業餘.
可以是幫人的事.
可以是你想做的事.
可以是理想,可以是幫朋友.
所以這些全部都是義工.
所以不要太不信任.
任何不付費的事.
都是值得留意的.
就像新年的時候幫媽媽做蘿蔔糕.
這是義工.
這就是你的生命和媽媽的關係.
這就是這樣.

$^{3081}$可能也是你的業餘.
但你喜歡烹飪.
可能有點不關事.
剛才說工作上有些不是屬於你的照明.
用人生作基礎去想這件事.
我在想如果照明出事了怎麼辦呢?.
我認識一些朋友.
他至少很想做警察.
小時候參加訓練警訓.
他父母全部都是警察世家.
他很想去保安.
但你知道這段時間.
做警察有多大的掙扎.
他也知道.
有幾個這種朋友.
有些是知道這件事不好.
他就不做.
但他又忍不住.
有些經常去做護警.
但他真的很喜歡這件事.
他覺得自己很適合.
也很應該去做這件事.
又或者有些不理會.
繼續做警察.
但有些他覺得不對的就不理會.
但如果這樣就做不到.
剛才說的三件事.
負責人又用WhatsApp.
又交貨帶.
平時有些時間你會沒有交貨帶.
因為你覺得不應該要跟隨規則.
第二,有多少部隊.
第一要跟從上司的指令.
有這些矛盾的時候.
會怎麼辦呢?.
我想到一個這樣的例子.
可能這件事會比較入肉.
他自己是否這樣看?.
自己是否覺得這樣是不想做.
或者不接受?.

$^{3121}$有些是.
有幾個警察朋友.
有些是他覺得不可以.
但群眾壓力之下.
或者他覺得自己是.
或者他本身整個人.
本身的生活就是想要做這件事.
他從來沒有想過要做其他事.
或者他覺得自己真的適合.
就算我看他的人.
他的性格都真的很適合.
但他面對著這樣的情況.
怎麼辦呢?.
我想警察本身不是一件什麼.
本身警察都是一件好事.
很多事情都是需要警察來做.
都說JPEG.
警察會抓人.
所以我覺得其實.
有點像剛才賭場的情況.
有些情況下.
不是自己想做的事.
怎麼辦?.
其實都是很處境化的事.
就是掙扎.
今天才和弟兄聊天.
他以前是做保險的.
但他的保險是很不適合.
他不喜歡保險的事.
做了幾年就不做了.
所以都是一種過程.
他可能某個位置不認同警察某些事.
自然就會掙扎.
但暫時未來.
不代表他會全然敗壞.
都是一種過程.
所以很難判斷一個人.
或者判斷一個行業.
本身是怎樣的.
好還是不好.

$^{3161}$很難來到.
那麼容易去畫界線.
我的重點都是覺得.
不是爭持在那個行業上.
因為那個行業.
剛好這次有些事故.
或者事件.
就覺得那個行業不好.
我是不是要跳船.
每個行業都有不好的地方.
就正如醫生都有很多不好的醫生.
剛才張先生說保險.
我想有些人保險是好的.
有些人保險是不好的.
每次見到你都是sell down.
可能你都會負面.
我想重點未必是.
側重在那個行業上.
回到剛才John follow你.
他問你.
他本身的人設是否對做那份工作.
這個是重要的.
因為人設不對做那份工作.
一來他做得辛苦.
二來他都在問自己日日上班做什麼.
這個都很重要.
我經常都說身體是很誠實的.
我以往.
就算我自己家人都是這樣.
剛才說到97年前入職.
很多東西都是基本法保障.
不會炒到你的.
但當慢慢升職升到一個位置.
很大壓力.
工時很長的時候.
就頭髮不斷掉.
病又不好.
都會跟家人說.
錢買到的東西其實可以不要.
有些東西是錢買不到的.

$^{3201}$如果你在這份工作.
給到這筆錢.
但你沒有了家人相處的時間.
沒有身體.
沒有東西的時候.
你覺得是否還值得.
我覺得那份工作考慮的因素.
是否符合你那個階段的人設.
如果你的人設定是剛剛做的.
我覺得那個衡量是比較重要的.
但不是因為工作在那個階段.
突然被污名.
或者有些東西是違反了.
你自己接受不了.
你就選擇不做.
但如果你覺得不是那個想法.
你覺得你可以堅守到一些原則.
你就繼續做下去.
那牽涉到職業道德.
你在工作上有沒有違反職業道德.
道德就是道就是法規.
德就是行為.
在法規之下你有相應的行為.
那你就合乎道德.
但如果你說做警察.
你的職業道德你去包庇.
那就是違反職業道德.
我覺得不要用職業來.
那個環境因素影響到他的選擇.
反而他自己是否符合.
和他有沒有做回職業道德.
這個是重要的.
差不多了.
我們還有一堂課.
多飛一會就完結了.
那下一段有什麼期望呢.
下一段我們最後會講基督徒的隱藏性.
先不說這些.
大家下次來的時候就知道.
如何去理解隱藏性的題目.

$^{3241}$下個星期 下次見.
下一段 再見.
再見.
下個星期見.
\newpage



\section{撒母耳記上 3:1-10-20231104}
\label{sec:_cxLnHL_TWQ}
\textbf{【網上崇拜】內心的小孩,仍活著嗎?|撒母耳記上3\_1-10|20231104 [\_cxLnHL-TWQ]}
\newline
\newline
連結: \href{https://youtube.com/watch?v=_cxLnHL-TWQ}{\texttt{ https://youtube.com/watch?v=\_cxLnHL-TWQ}} ~~~~ 語音日期: 2023-11-04 
\newline
\newline
\hyperref[sec:JKdFzjAsLZY]{\small{< < < PREV SERMON < < <}}
~
\hyperref[sec:index_chronic]{\small{[返順時目]}}
~
\hyperref[sec:index_scriptual]{\small{[返順卷目]}}
~
\hyperref[sec:qJWlmXEzoSU]{\small{> > > NEXT SERMON > > >}}
\newline
\newline
撒母耳記上 3:1-10-20231104
\newline
\begin{longtable}{cl}
\hline
\hline
章節 & 經文 (和合本修訂版)\\
\hline
3:1 & \begin{tabularx}{0.7\textwidth}{X} 那孩子撒母耳在以利面前事奉耶和華。在那些日子,耶和華的言語稀少,不常有異象。 \end{tabularx} \\ \\ \relax
3:2 & \begin{tabularx}{0.7\textwidth}{X} 那時,以利在自己的地方睡覺;他眼目開始昏花,不能看見。 \end{tabularx} \\ \\ \relax
3:3 & \begin{tabularx}{0.7\textwidth}{X} 神的燈還沒有熄滅,撒母耳睡在耶和華的殿內,神的約櫃就在那裡。 \end{tabularx} \\ \\ \relax
3:4 & \begin{tabularx}{0.7\textwidth}{X} 耶和華呼喚撒母耳,撒母耳說:「我在這裡!」 \end{tabularx} \\ \\ \relax
3:5 & \begin{tabularx}{0.7\textwidth}{X} 他跑到以利那裡,說:「你叫我嗎?我在這裡。」以利說:「我沒有叫你,回去睡吧。」他就回去睡了。 \end{tabularx} \\ \\ \relax
3:6 & \begin{tabularx}{0.7\textwidth}{X} 耶和華又呼喚撒母耳。撒母耳起來,到以利那裡,說:「你叫我嗎?我在這裡。」以利說:「我兒,我沒有叫你,回去睡吧。」 \end{tabularx} \\ \\ \relax
3:7 & \begin{tabularx}{0.7\textwidth}{X} 那時撒母耳還未認識耶和華,耶和華的話也未曾向他啟示。 \end{tabularx} \\ \\ \relax
3:8 & \begin{tabularx}{0.7\textwidth}{X} 耶和華第三次再呼喚撒母耳。撒母耳起來,到以利那裡,說:「你叫我嗎?我在這裡。」以利才明白是耶和華呼喚這小孩。 \end{tabularx} \\ \\ \relax
3:9 & \begin{tabularx}{0.7\textwidth}{X} 以利對撒母耳說:「你回去睡吧。他若再叫你,你就說:『耶和華啊,請說,僕人敬聽!』」撒母耳就回去,仍睡在原處。 \end{tabularx} \\ \\ \relax
3:10 & \begin{tabularx}{0.7\textwidth}{X} 耶和華來站著,像前幾次呼喚:「撒母耳!撒母耳!」撒母耳說:「請說,僕人敬聽!」 \end{tabularx} \\ \\ \relax
3:11 & \begin{tabularx}{0.7\textwidth}{X} 耶和華對撒母耳說:「看哪,我在以色列中必行一件事,凡聽見的人都必雙耳齊鳴。 \end{tabularx} \\ \\ \relax
3:12 & \begin{tabularx}{0.7\textwidth}{X} 我指著以利家所說的話,到了時候,必從頭到尾應驗在以利身上。 \end{tabularx} \\ \\ \relax
3:13 & \begin{tabularx}{0.7\textwidth}{X} 我曾告訴他,我必永遠懲罰他的家,因為他知道自己的兒子作惡,褻瀆神,卻不禁止他們。 \end{tabularx} \\ \\ \relax
3:14 & \begin{tabularx}{0.7\textwidth}{X} 所以我向以利家起誓:『以利家的罪孽,就是獻祭物和供物,也永不得贖。』」 \end{tabularx} \\ \\ \relax
3:15 & \begin{tabularx}{0.7\textwidth}{X} 撒母耳睡到天亮,就開了耶和華殿的門。撒母耳害怕,不敢將異象告訴以利。 \end{tabularx} \\ \\ \relax
3:16 & \begin{tabularx}{0.7\textwidth}{X} 以利呼喚撒母耳說:「我兒撒母耳!」撒母耳說:「我在這裡!」 \end{tabularx} \\ \\ \relax
3:17 & \begin{tabularx}{0.7\textwidth}{X} 以利說:「他對你說了甚麼話,你不要向我隱瞞。你若將他對你所說的話向我隱瞞一句,願神重重懲罰你。」 \end{tabularx} \\ \\ \relax
3:18 & \begin{tabularx}{0.7\textwidth}{X} 撒母耳就把一切話都告訴以利,並沒有隱瞞。以利說:「他是耶和華,願他照他看為好的去做。」 \end{tabularx} \\ \\ \relax
3:19 & \begin{tabularx}{0.7\textwidth}{X} 撒母耳長大了,耶和華與他同在,使他所說的話一句都不落空。 \end{tabularx} \\ \\ \relax
3:20 & \begin{tabularx}{0.7\textwidth}{X} 從但到別是巴,所有的以色列人都知道耶和華立撒母耳為先知。 \end{tabularx} \\ \\ \relax
3:21 & \begin{tabularx}{0.7\textwidth}{X} 耶和華又在示羅顯現,因為耶和華在示羅藉他的話向撒母耳啟示他自己。 \end{tabularx} \\ \\
[1ex]
\hline
\hline
\end{longtable}
$^{1}$一會兒我們可以一起敬拜.
可以一起思考神的話語.
今天想跟大家看一段經文.
是撒慕爾記上第三章.
撒慕爾記是記載以色列立國的時間.
當中包括有三個領袖交接.
今天我們所看的是第一個.
第三章.
童子撒慕爾第一次領受神的呼召.
即是說一個上莊的小孩出現了.
我們會思考一個訊息.
希望我們面對上帝的呼召的時候.
少一些用我們成人的腦袋.
不斷盤算著成敗得失.
面對上帝的呼召不是用這裡.
上帝是呼召我們的心.
不是我們算盡我們做不做這件事.
可以得到什麼失去什麼.
上帝很想跟我們內心的小孩.
去說一聲.
你願不願意做一些打亂你人生的事.
你想不想做一些會打亂你人生.
但我很想你做的事.
這個是呼召.
2023年就快結束.
但願我們在結束這一年之前.
試下回應一次上帝對你個別的呼召.
我們先看經文.
經文應該是三章的一至十節.
我打錯了.
打錯了三十一章.
應該是三章的一至十節.
如果大家找到或者看到經文.
都請大家為我讀出這十節的經文.
如果看到的預備.
一 二 三.
相信這段都是我們熟悉的經文.
今天我們會嘗試透過這段經文.
去找一些細節.
讓我們明白.

$^{41}$究竟薩姆爾怎樣領受神的呼召.
薩姆爾的領受跟我們的領受.
有沒有一些相連的地方.
我希望在這段經文裡.
我們都找到一些脈絡.
首先這段經文是.
之前是在說薩姆爾的媽媽哈娜.
她不育.
她求神賜一個兒子給她.
她說如果我得到這個兒子.
我就答應將這個兒子.
送到聖殿裡.
就好像你們送了兒子到對面一樣.
帶他回到聖殿.
一起去敬拜神.
薩姆爾的媽媽.
她沒有想自己的得失.
她答應了上帝.
她就做回當做的事.
這個是她懇求的兒子.
她都願意放在神的殿裡.
在裡面成長.
到了第三章這個小孩子已經長大了.
聖經形容他是童子.
童子不是純粹說年紀小小那麼簡單.
童子就是說他沒有地位.
我們在公司叫打雜一樣.
這個打雜的薩姆爾.
就在殿裡面成長.
其中在第一二節.
他提到一個經文的背景.
就是耶和華的言語稀少.
不常有默示.
薩姆爾是生活在士司的年代.
如果大家稍為對聖經熟悉.
士司的年代最後的評語是什麼.
個人任意而行.
不將神放在眼裡.
當時的年代就是.
耶和華不想再跟這群人說話的年代.

$^{81}$薩姆爾在這個環境裡成長.
聖殿或者會幕.
祭司全部都是擺設.
沒有意思的.
神不再說話.
他們只是做樣子.
有了這個背景.
我們開始明白這段經文.
一些重要的地方.
第三節.
他講了一個很特別的事情.
整個殿裡面是昏暗.
只是有燭光在染樣.
但是在經文第三節.
他講了一個句子.
他說神的燈.
其實是駁下去還沒熄滅.
是代表那時候是晚上.
他要我們留意的是另外的一組.
就是在耶和華的殿裡面.
在約櫃附近.
薩姆爾就睡在那裡.
原文是按這個次序去看.
他想讓我們知道一件事.
薩姆爾是睡在約櫃附近.
我們開始有些感受.
約櫃不是輕易碰到的地方.
是致勝所.
你想想.
當時一個小孩.
他晚上睡覺.
睡在一個核子反應堆旁邊.
一捉即死.
代表什麼.
沒有人覺得神會出手.
沒有人覺得耶和華在約櫃附近.
根本只是傳說.
小孩睡哪裡就隨他.
致勝所就致勝所.
有什麼所謂.

$^{121}$耶和華不再出聲.
這是整段經文的背景.
然後令我們驚訝的就是.
耶和華出聲.
當我們覺得神不再說話的時候.
當我們覺得上帝不理會這個世界的時候.
當這個小孩可以隨意睡的時候.
耶和華出聲.
接著的經文讓我們看到一些對比.
就是神出聲.
究竟跟誰說話.
當時一怒一亂.
在會幕裡面睡覺.
不過在三章二節.
他形容老祭司以來.
他不是睡覺.
他是形容他餓著.
即是躺在那裡.
對比其實是.
不單只是他身體的狀況.
而是他對神的說話的狀態.
他餓著不動.
眼睛昏花.
他接收不到.
對比是另一個小孩.
那個小孩他說他還沒認識耶和華.
他不認識.
但當神的說話去呼喚他的時候.
他的反應是怎樣的.
立刻彈起來.
半夜三更有人叫你的名字.
你會怎樣的.
一鎚打下去.
他會彈起來.
撒姆爾的反應.
是對比著以來的反應.
神的話臨到.
不單只是身體的狀況.
經文是暗暗表達他們屬靈的狀況.
一個專職的祭司.

$^{161}$耶和華的話臨到.
他是分誰分不到.
一個還沒認識神的小孩.
耶和華的話臨到.
雖然他不知道是誰說.
但他的反應是怎樣的.
彈起來.
立刻跑去師父那裡.
問師父是不是來找我.
一次 兩次 三次.
我們對神有經驗.
不等於我對上帝的說話有靈魂.
我們越知道上帝的東西.
我們越懂得聽不見.
以你是這樣的情況.
我們是什麼情況呢.
直到第三次.
師父以來忠於醒覺.
他知道不是那麼簡單.
於是他教他的徒弟撒姆爾.
用一個禮儀式的回應.
他說如果你再聽到神的呼召.
你就回答.
主啊 請說.
福人 敬聽.
這是師父唯一可以教到.
撒姆爾回應上帝的方法.
果然神第四次向他說話.
聖經形容神是站著一樣.
神不是一個有形有體.
站在撒姆爾面前.
而是他很想讓我們看見.
當時的世界覺得上帝不存在.
但神就讓這個小孩.
清清楚楚聽得見.
看得見.
上帝是在的.
這是這樣的表達.
經文我們看到這裡.
我嘗試總結.

$^{201}$究竟這段經文.
想跟我們說什麼.
然後我們進入下一部分.
經文讓我們看到.
這個呼召很平凡.
相對其他先知.
火啊.
天崩地裂的呼召.
這個呼召很平凡.
就是上帝召喚了你三次.
你以為上帝召喚你.
打完一次又一次.
然後上帝說不是我召喚你.
是在上帝的上帝召喚你.
是撒姆爾的日常.
他師父可能有時候睡不著.
晚上叫他.
是習慣.
所以撒姆爾不以為師父找他.
他沒有什麼驚訝.
師父叫我我就去.
三次才發現.
原來不是這麼簡單.
是一個很平常的環境.
是我們日常生活的場景.
上帝這樣呼召.
第二.
呼召的內容.
不是很複雜.
如果我們有時間看下文.
我們知道.
上撒姆爾做什麼.
就是跟他師父.
以來說你師父的家.
的結局是怎樣怎樣.
難不難.
一點都不難.
甚至我們可以跟上帝說.
上帝你自己說就好.
不用動我.

$^{241}$勞煩我.
要我去跟師父說.
有時候我們都是這樣想.
神叫我們做一些事.
我們呼召旁邊那個就好.
我們常常為旁邊那個祈禱.
願他聆聽.
我們常常都是這樣.
為什麼.
為什麼神要.
用撒姆爾.
做一樣可以不用.
驚他都做到的事.
有沒有想過這個問題.
神的話.
要重新.
臨到地上.
唯一可以做到的.
就是有人回應.
耶和華的話.
今天不是不存在.
而是有沒有人回應.
瀑人.
敬廷.
我願意做神吩咐我做的事.
這是整段經文.
讓我們看到.
一個大時代的開始.
上帝用這個小朋友.
開展.
以色列國的新時代.
經文看完了.
我們嘗試又想一想.
究竟什麼叫呼召呢.
究竟.
我們的靈性.
是否又能夠回應呼召呢.
不論在形式上.
在內容上.
神對撒姆爾的呼召.

$^{281}$的確和摩西.
或者眾仙之比.
是比較平凡一點.
甚至是多此一舉.
不需要他都可以.
但是剛才的呼召.
仍然有三個很重要的原則.
我們看得到.
第一.
這個呼召是清晰而具體.
或者.
在靈修裡.
在剛才的詩歌裡.
在講道裡.
突然間好像觸電一樣.
上帝好像.
碰一碰你的心靈一樣.
清晰而具體.
絕不模糊.
是來自上帝.
第二.
這個呼召不是一次.
神經你聽不到.
祂再說一次.
你沒有反應.
祂再說一次.
直至你覺得.
有些不對勁.
神對我們的呼召.
是想我們收到.
持續的.
不是偶然的.
持續的.
第三.
這個呼召是個別給你的.
可能.
剛才唱詩歌.
你旁邊那個.
沒有什麼表情沒有什麼反應.
但你已經.

$^{321}$兩眼迎眶想流眼淚.
同一首詩歌.
未必感動每一個人.
但上帝用那首詩歌.
就是要對你說話.
祂就是要用那句經文.
跟你說話.
是個別的.
你聽不到的.
周圍的人不知道的.
只是叫薩姆爾.
祂是叫他的名字.
祂不是叫他小子起來.
不是的.
神叫你的名字.
祂叫得出你是誰.
祂知道你的來歷.
祂是要跟你說話.
頂智慕如果我們都同意.
呼召是成立的話.
那我們就問.
我們是否接收到.
這些呼召.
關於靈性的問題.
我發現在成長.
或者在信主不同的階段.
我對呼召的敏銳度.
都有很大的不同.
年輕時候.
或者初信主的時候.
神的呼召.
我覺得好像一種榮幸一樣.
就好像小學生.
或者小朋友.
老師選頒獎.
小學生小朋友都是舉手.
覺得你被老師選到.
是一種榮幸.
但是年年月月.
我們開始明白.

$^{361}$呼召不是那麼簡單.
我們的狀態.
慢慢沒有隨著.
年紀或者對神的認識.
而增加.
對上帝的敏銳我們越來越似耳離.
我們呃著.
聽不見.
沒有反應.
其中我想有兩個原因.
我自己去檢討自己.
都發現可能都是.
一些真實的情況.
就是我開始越來越.
了解神.
越看聖經越多.
我就知道呼召不是.
舉手做頒獎那麼簡單.
剛才你看經文.
撒姆爾.
是未認識耶和華.
所以他舉得很爽.
摩西是知道發生什麼事.
八十歲.
耶利米.
都有經驗.
萬一神呼召我.
叫我走摩西那條路.
就是一世和他同在.
頂心頂頸的人.
永不分離.
那我認了怎麼辦.
如果神呼召我.
好像耶利米那樣.
一生勞苦做一些.
徒勞無功的事.
被人神憎鬼厭.
這樣過一世.
如果我又認了.
那我怎麼辦.

$^{401}$等至我越認識聖經.
越認識耶和華.
我越來越似若拿.
我只想一走了之.
不要搞我.
我不想上帝.
打亂我的人生.
我和你不是沒有.
回應呼召.
而是在那次的經歷裡.
我們的血還沒止.
這次又來單身.
你呼召他.
請差遣他.
為什麼要鬼搞我.
我越認識神.
我們也知道.
沒有那麼多彈藥.
再走下去.
盡量避開就避開.
人年紀越大.
我們知道沒有那麼多.
本錢再失敗.
多一次.
不要再搞.
摩西是這樣的.
八十歲還說什麼.
不要搞我行不行.
找些小朋友就行.
越認識神.
呼召不是榮幸.
呼召是什麼.
不幸.
小時候想老師叫我做班長.
長大後.
我想老闆看不到我.
最好見不到我.
我聽人說.
上班最快樂的.
不是自己放大假.

$^{441}$是老闆放大假.
他見不到我.
我就真的自由.
對呼召也是這樣.
上帝見不到我.
我做什麼都行.
總之你答應我.
讓我回天家.
其他的不要理我.
越認識神.
我對呼召越遲鈍.
第二就是.
我越認識自己.
我們過去年初.
都會立志.
今年要減三磅.
讀完整本聖經.
學好外語.
但兩個星期就玩完.
我們年年月月.
這樣去立志.
年年月月這樣去答應自己.
沒有一次做得到.
我還有什麼把握.
答應上帝.
去做他要我做的事.
我還有什麼信心.
去玩領受呼召.
這個玩意.
我自己都答應不到自己.
我都答應不到.
周圍的人.
我不如放過自己.
輕輕鬆鬆過就算了.
我們開始由有自知之明.
變成自設限制.
這些不能碰.
這些不能改.
這些我不能做.
我們自己圍了一道牆.

$^{481}$上帝你不可以進入.
這個範圍.
其他OK.
這些不可以搞.
然後我們開始在靈性裡.
不想努力.
接收得太好.
我每天良心責備.
都不太妥當.
於是我開始.
逃避上帝的話語.
甚至我偷天換日.
改了耶穌那句說話.
若有人要跟從我.
就當愛自己.
若有人要跟從我.
就當做自己喜歡的事.
我跟從耶穌.
變成做我自己喜歡的事.
頂之我不是想說喜歡的事.
有什麼大奸大惡.
但耶穌已經提醒了我們.
很多時阻礙我們經歷神的.
不是環境.
是我自己.
因為我太了解上帝.
太了解自己.
於是我做了一些事情.
學校剛才那段經文.
自己製造.
耶和華言語稀少的環境.
我試過的.
有效的.
我們可以交流一下心得.
首先將我們行事力填得滿滿.
興趣多多.
每天都忙的.
但都過得開心快活.
有沒有上帝不是太緊要.
我人生真的充滿了知才.

$^{521}$然後我將餘下的精力.
全部用在報復性的娛樂裡面.
每晚弄得自己筋疲力竭.
丁點默想的空間都不需要.
我很怕我自己靜下來.
聽到耶和華對我說話.
我們仍然會回教會.
仍然會聽詩歌.
選擇性的去做.
安慰 祝福 賜予的.
當然是受.
使命 獨澤.
留給隔離.
我們自己選擇.
我想選擇一個.
我喜歡的耶和華去信.
並不是傳言 相信這位神.
我不知道這是否你的經歷.
我並不是說諷刺的話.
我真的做過.
我很想逃避上帝.
頂折會真的.
領受上帝的呼召.
有時會打亂我們的人生.
你想想 撒慕爾一個小朋友.
要跟師父說一番審判的話.
是多麼艱難.
大神偏偏用這個小朋友.
去做他說再臨到大地的第一件事.
你做得到.
雖然艱難 但你做得到.
當他這樣做.
聖經記載 撒慕爾的話.
再無落空.
神透過這個小朋友.
不斷傳講他的信息.
頂折會上帝真的會打亂我們的人生.
他的呼召可能是叫我們做一些不情願的事.
但唯有我們這樣踏上這條路.
我們發現原來人生仍然可以遇到.

$^{561}$意想之外的美麗.
我們碰上這些同樣被打亂的戰友.
一起行一趟我們未見過的旅程.
正是因為我們願意踏上.
我們又發現原來人生不是局限在我的思維裡.
以至我今天可以站在台裡跟你們說話.
是因為上帝打亂我的人生.
他叫我去做一些我不是太情願.
但走下去卻是恩典的道路.
來到最後一部分.
我想向弟兄姊妹發出呼召.
快要結束的今年.
我不知道你還有什麼卡在你的人生裡.
我希望藉著今晚.
剩下兩個月就結束的日子.
我們好好去想一想.
不要讓今年白白地流過.
2022和2023毫無分別.
至少我們在餘下的兩個月.
如果上帝對你說話.
你嘗試回應一次.
讓我們內心的孩子去做回應.
不是成人的腦袋盤算所有得失.
然後考慮一下.
最後做不做一件事.
但我希望今晚上帝是對你的心說話.
不是對你的腦袋說話.
好嗎.
容許我向你發出呼召.
你不要再為自己已經做了決定的事.
左右擺.
你已經思前想後很多次.
無論你怎樣去想.
你做A決定或B決定.
都一定有損失.
如果你從自己的成敗得失的界線.
去決定做不做這件事.
你想到地老天荒都想不通.
我今天希望你從神的角度去想.
你今天的選擇.

$^{601}$能不能榮耀上帝多一點.
哪管你會損失.
你今天或左或右的選擇.
是哪一樣讓你榮耀上帝多一點呢.
讓我再向你發出呼召.
有些糾纏在你心裡的事情.
或關係.
是否需要放下呢.
這些已經成為過去的事.
不論過去的好或不好.
既然你都讓它成為過去.
就不如張開眼睛.
看你今天身邊的人或身邊的事.
這些都是神給我們的祝福.
回頭再看.
你再唏噓都不能挽回.
倒不如我們鼓起勇氣.
去愛現在的人.
專心做現在的事.
這不是對你更好嗎.
不要再為自己自設自限.
我們有何德何能去做上帝的功.
上帝叫我們不是因為我們有何本事.
上帝是想單單用你.
就是這麼多.
上帝不是因為看中我們某個專業.
以致我們成功率比較高.
上帝說我只想用你.
就是這麼多.
沒有其他成敗得失在後面.
上帝說我想和你一起.
做一些你都未想過的事.
可以嗎.
不要再沉淪在悲傷和罪惡裡.
你可以說沒有人知道.
但你心裡的聖靈不斷督責你.
你自己都知道.
鼓起勇氣和這些事情脫勾.
不要經常愁眉苦臉.
我們的信仰是力量.

$^{641}$是讓我們有盼望.
如果眾人經常看見我們愁眉苦臉.
我們下一代如何信耶穌呢.
下一代看見我們信耶穌信得這麼慘.
他真的會選擇這個信仰嗎.
下一代看見我們在艱難中仍然笑得出.
他覺得父母的信仰是值得相信的.
我們為下一代做見證.
頂著每樣我們將神的話語放在心裡.
上帝觸碰你的心靈.
上帝要和我們每一個說話.
回應呼召不是我們不懂世情.
不是我們不懂盤算.
特別當我們人大了.
我發現一件真理.
當我想通想透.
計算過所有成敗得失之後.
我仍然選擇跟隨上帝.
這是孩童單純的美.
亦不乏我們智慧的考慮.
當我盤算一切.
我知道這樣的決定一定有損失.
甚至有相當的冒險.
但我仍然選擇跟隨上帝.
但願今晚上帝觸碰你的心靈.
對我們每一個說話.
讓我們低頭一同祈禱.
孩子.
你可否為我做叮點的事.
這些事情你早知道.
可能你猶豫不決.
可能你亦都考慮得失.
但請你相信我們的神.
祂會跟你一起經歷這些事.
願你回應我們的主.
福人敬聽.
阿門.
\newpage



\section{希伯來書 12:1-11-20231111}
\label{sec:qJWlmXEzoSU}
\textbf{【網上崇拜】孩子,我是這樣愛你的!|希伯來書12\_1-11|20231111 [qJWlmXEzoSU]}
\newline
\newline
連結: \href{https://youtube.com/watch?v=qJWlmXEzoSU}{\texttt{ https://youtube.com/watch?v=qJWlmXEzoSU}} ~~~~ 語音日期: 2023-11-11 
\newline
\newline
\hyperref[sec:_cxLnHL_TWQ]{\small{< < < PREV SERMON < < <}}
~
\hyperref[sec:index_chronic]{\small{[返順時目]}}
~
\hyperref[sec:index_scriptual]{\small{[返順卷目]}}
~
\hyperref[sec:2LJqqGa1zFo]{\small{> > > NEXT SERMON > > >}}
\newline
\newline
希伯來書 12:1-11-20231111
\newline
\begin{longtable}{cl}
\hline
\hline
章節 & 經文 (和合本修訂版)\\
\hline
12:1 & \begin{tabularx}{0.7\textwidth}{X} 所以,既然我們有這許多見證人如同雲彩圍繞著我們,就該卸下各樣重擔和緊緊纏累的罪,以堅忍的心奔那擺在我們前頭的路程, \end{tabularx} \\ \\ \relax
12:2 & \begin{tabularx}{0.7\textwidth}{X} 仰望我們信心的創始成終者耶穌,他因那擺在前面的喜樂,輕看羞辱,忍受了十字架的苦難,如今已坐在神寶座的右邊。 \end{tabularx} \\ \\ \relax
12:3 & \begin{tabularx}{0.7\textwidth}{X} 你們要仔細想想這位忍受了罪人如此頂撞的耶穌,你們就不致心灰意懶了。 \end{tabularx} \\ \\ \relax
12:4 & \begin{tabularx}{0.7\textwidth}{X} 你們與罪惡爭鬥,還沒有抵抗到流血的地步。 \end{tabularx} \\ \\ \relax
12:5 & \begin{tabularx}{0.7\textwidth}{X} 你們又忘了神勸你們如同勸兒女的那些話,說:「我兒啊,不可輕看主的管教,被他責備的時候不可灰心; \end{tabularx} \\ \\ \relax
12:6 & \begin{tabularx}{0.7\textwidth}{X} 因為主所愛的,他必管教,又鞭打他所接納的每一個孩子。」 \end{tabularx} \\ \\ \relax
12:7 & \begin{tabularx}{0.7\textwidth}{X} 為了受管教,你們要忍受。神待你們如同待兒女。哪有兒女不被父親管教的呢? \end{tabularx} \\ \\ \relax
12:8 & \begin{tabularx}{0.7\textwidth}{X} 管教原是眾兒女共同所領受的;你們若不受管教,就是私生子,不是兒女了。 \end{tabularx} \\ \\ \relax
12:9 & \begin{tabularx}{0.7\textwidth}{X} 再者,我們曾有肉身之父管教我們,我們尚且敬重他,何況靈性之父,我們豈不更當順服他而得生命嗎? \end{tabularx} \\ \\ \relax
12:10 & \begin{tabularx}{0.7\textwidth}{X} 肉身之父都是短時間隨己意管教我們,惟有靈性之父管教我們是要我們得益處,使我們在他的聖潔上有份。 \end{tabularx} \\ \\ \relax
12:11 & \begin{tabularx}{0.7\textwidth}{X} 凡管教的事,當時不覺得快樂,反覺得痛苦;後來卻為那經過鍛鍊的人結出平安的果子,就是義的果子。 \end{tabularx} \\ \\ \relax
12:12 & \begin{tabularx}{0.7\textwidth}{X} 所以,你們要把下垂的手舉起來,發酸的腿挺直; \end{tabularx} \\ \\ \relax
12:13 & \begin{tabularx}{0.7\textwidth}{X} 要為自己的腳把道路修直了,使瘸了的腿不再脫臼,反而得到痊癒。 \end{tabularx} \\ \\ \relax
12:14 & \begin{tabularx}{0.7\textwidth}{X} 你們要追求與眾人和睦,並要追求聖潔;人非聖潔不能見主。 \end{tabularx} \\ \\ \relax
12:15 & \begin{tabularx}{0.7\textwidth}{X} 要謹慎,免得有人失去了神的恩典;免得有毒根生出來擾亂你們,因而使許多人沾染污穢, \end{tabularx} \\ \\ \relax
12:16 & \begin{tabularx}{0.7\textwidth}{X} 免得有人淫亂,或不敬虔如以掃,他因一點點食物把自己長子的名分賣了。 \end{tabularx} \\ \\ \relax
12:17 & \begin{tabularx}{0.7\textwidth}{X} 後來你們知道,他想要承受父親的祝福,竟被拒絕,雖然流著淚苦求,卻得不著門路使他父親回心轉意。 \end{tabularx} \\ \\ \relax
12:18 & \begin{tabularx}{0.7\textwidth}{X} 你們不是來到那可觸摸的山,那裡有火焰、密雲、黑暗、暴風、 \end{tabularx} \\ \\ \relax
12:19 & \begin{tabularx}{0.7\textwidth}{X} 角聲,和說話的聲音;當時那些聽見這聲音的,都求不要再向他們說話, \end{tabularx} \\ \\ \relax
12:20 & \begin{tabularx}{0.7\textwidth}{X} 因為他們擔當不起所命令他們的話,說:「靠近這山的,即使是走獸,也要用石頭打死。」 \end{tabularx} \\ \\ \relax
12:21 & \begin{tabularx}{0.7\textwidth}{X} 所見的景象極其可怕,以致摩西說:「我恐懼戰兢。」 \end{tabularx} \\ \\ \relax
12:22 & \begin{tabularx}{0.7\textwidth}{X} 但是你們是來到錫安山,永生神的城,就是天上的耶路撒冷,那裡有千千萬萬的天使, \end{tabularx} \\ \\ \relax
12:23 & \begin{tabularx}{0.7\textwidth}{X} 有名字記錄在天上眾長子的盛會,有審判眾人的神和成為完全的義人的靈魂, \end{tabularx} \\ \\ \relax
12:24 & \begin{tabularx}{0.7\textwidth}{X} 並新約的中保耶穌,以及所灑的血;這血所說的信息比亞伯的血所說的更美。 \end{tabularx} \\ \\ \relax
12:25 & \begin{tabularx}{0.7\textwidth}{X} 你們總要謹慎,不可拒絕那向你們說話的,因為那些拒絕了在地上警戒他們的,尚且不能逃罪,何況我們違背那從天上警戒我們的呢? \end{tabularx} \\ \\ \relax
12:26 & \begin{tabularx}{0.7\textwidth}{X} 當時他的聲音震動了地,但如今他應許說:「再一次我不單要震動地,還要震動天。」 \end{tabularx} \\ \\ \relax
12:27 & \begin{tabularx}{0.7\textwidth}{X} 這「再一次」的話是指明被震動的要像受造之物一樣被挪去,使那不被震動的能常存。 \end{tabularx} \\ \\ \relax
12:28 & \begin{tabularx}{0.7\textwidth}{X} 所以,既然我們得了不能被震動的國度,就要感恩,照著神所喜悅的,用虔誠、敬畏的心事奉神, \end{tabularx} \\ \\ \relax
12:29 & \begin{tabularx}{0.7\textwidth}{X} 因為我們的神是吞滅的火。 \end{tabularx} \\ \\
[1ex]
\hline
\hline
\end{longtable}
$^{1}$剛才入場的時候看到上星期又今個星期的弟兄姊妹.
很開心.
另外又看到一些新面孔.
都很開心.
更加開心就是看到一些從外地回來的弟兄姊妹.
我希望在崇拜當中.
每一次都可以讓不同弟兄姊妹一起參與.
這個我常常都會重提的.
因為教會在地上最大的表徵就是.
讓可以敬拜上帝的人一起聚集.
這個也是教會要常常提醒彼此一起參與.
不知道大家是否聽完上次網上的編導.
這個星期想看一下.
那個目者這個星期沒有出現.
Sorry.
如果你聽了上星期的編導.
又有參與我們Discord的頻道的話.
你會發現出一個新的帖文.
一人一句.
不是出Pool.
不是.
一人一句徐牧師的金句.
所以很多人留言.
回顧徐牧師的金句.
都是很入心又入肺的.
因為是在邀請你.
在當中做一個信仰的表白.
在邀請你.
是否要讓上帝搞亂你的人生呢?.
我相信聽到的過程當中.
每一次觸動都是聖靈的提醒.
讓你更加靠近上帝身.
讓你更加反省.
當初你為什麼要缺志信主.
沒有人不拿槍逼你.
當初你為什麼要跟隨耶穌.
沒有人逼你.
但你可能已經放下了內心的小孩.
你留在他一個二次元的空間.
不要在現實生活中出現.

$^{41}$但我相信每一次一起崇拜的時候.
其實不單是你曾經這樣想過.
可能你身邊的人也這樣想過.
但他沒有說話.
又或者你不認識他.
在我預備今天講章的時候.
我選了希伯來書12章的經文.
這段經文對我來說.
都是我自己很入心的經文.
這段經文主要是連帶上一星期的訊息.
如果網上的你.
是今天第一次聽這本訊息.
如果你是不知道上一篇訊息是什麼.
你聽完這篇訊息.
再回聽上星期的訊息.
再聽這篇訊息.
你就會看到連動.
我自己也聽了徐慕斯的訊息.
除了現場之外.
通常晚上我會再聽一次.
這個星期預備經文的時候.
我再聽兩三次.
我希望你會看到我們的閱題的連動.
也希望成節.
我們看看這段經文是什麼.
我們先看希伯來書第十二章第一節開始.
我們準備.
希伯來書第十二章.
.
有那麼多的見證人 好像雲彩一樣圍繞著我們.
就應該擺脫各樣的重擔 和容易纏住我們的罪.
以堅忍的心 奔跑那個擺在我們面前的賽程.
定睛仰望那位信心的始創者 和完成者耶穌.
祂因為那個擺在前面的喜樂 就輕視了羞辱.
忍受了十字架的苦難.
坐在神寶座的右邊.
你們要思想這位忍受罪人 這樣頂撞的耶穌.
免得你們疲倦灰心.
你們和罪惡打仗 還沒抵擋到流血的地步.
你們又忘記了那個勸你們 好像勸子女的說話.

$^{81}$我的兒子 你不可以輕看主的管教.
被祂糾正的時候 也不要灰心.
因為主所愛的 祂一定會管教.
並且責罰祂所收納的每一個子女.
為了接受管教.
你們要忍受 神對待你們.
就好像對待子女一樣.
哪有子女沒受過父親的管教.
所有做子女的 都受過管教.
你們如果沒受過管教.
就是私生子 不是子女.
再說 肉身的父親管教我們.
我們尚且尊重他們.
何況那位 我們所有人屬靈的父親.
我們豈不是更加應該 馴服他而得生命嗎.
因為肉身的父親.
都是按著自己的意思 來管教我們.
只有短暫的日子.
但是我們所有人 屬靈的父親管教我們.
卻是為了我們的好處.
使我們在他的聖潔上有份.
但凡受到管教.
當時不覺得好受.
反而覺得很難受.
後來他為那些 經歷了這種操練的人.
結出平安的果子來.
就是二.
剛剛聽到廣東話聖經的讀出.
不知道你們之前 有沒有看過希伯來書.
但是如果你看希伯來書的時候.
你會發覺其實不容易理解的經文.
因為一來不知道作者是誰.
很難從他的背景中 了解他寫作的動機.
又或者是一些心意.
但是你從看希伯來書的時候.
你會了解作者其實是很熟悉.
希伯來文化和他的舊印歷史.
他受書的過程中.
看到書卷的人.
他要明白到我們的信仰.

$^{121}$在我們祖宗的起承轉.
這個過程中 是否真的可以合呢?.
是否真的明白到我們在等待的彌賽亞已經出現了呢?.
而我們一直在等待的祭司制度.
或者敬拜的系統 是否真的能幫助我們呢?.
所以看希伯來書的時候.
你傳統解經或者你查經的時候都會知道.
去到第十一章.
就是在說信.
有很多信心的偉人當中.
他如何經歷這個信仰的管教.
如何經歷上帝在他生命當中.
重新搞亂的麻煩.
也是看到上帝的恩典.
接著十二章就是在說望.
既然有這麼多前人做過的事情.
我們盼望的是什麼呢?.
所以十二章是在說望.
第十三章就是在說愛.
就是上帝給我們的.
仍然在這個路途上.
我們用什麼可以繼續走下去.
唯有上帝給我們的愛.
繼續彼此扶持.
所以希伯來書的十一,十二,十三章.
就是信望,愛三個信仰的表白.
對我們來說.
當我們說有很多見證人的時候.
見證從來在教會裡都不是陌生的.
因為很多時候都會聽到說見證.
但是我自己在教會這麼久以來.
我常常都覺得見證是需要說的.
但是見證是很難重複的.
因為別人的見證跟你的生命當中.
其實是相似但不是的.
你很難複製別人的見證.
因為他的經歷.
他不容易在你的生命當中.
說有碗滑碗地做一次.
是不可以的.

$^{161}$為什麼還要說見證呢.
我也是跟弟兄姊妹說.
見證要持續地說.
因為不斷地說見證的時候.
你就會知道.
其實上帝還在做事.
繼續在他生命當中做的事情.
在他生命當中做的事情.
在我的生命當中做的事情.
因為你一直經歷上帝的時候.
上帝一直這樣做.
見證是繼續寫下去的.
因為我們的信仰是一個經驗學習的過程.
所以希伯來書在十一章中.
綜合了這十五個見證人.
這十五個見證人在聖經中來說.
希伯來書是說一個信心的偉人.
但其實怎樣才算偉大呢.
其實最大的困難就是過了後才說.
通常說紀念那個人.
其實那個人可能已經過氣了.
神蹤追願拿出來表揚他.
而且做的過程不是那麼容易.
所以隨便找一個.
你看他的生命當中.
都有很多被上帝搞亂的生命軌跡.
你會看到他很多時候.
都有不同的難處當中.
想迴避上帝.
就好像上個星期.
徐牧師提醒我們.
其實你和我最大的問題.
其實不是不知道上帝在做什麼.
是不想讓上帝知道你想做什麼.
整個過程中.
是不想讓上帝搞亂你的生命軌跡.
你就想選擇性去做你想做的事情.
這班人都是這樣.
希伯來的作者.
又想讓我們明白到.

$^{201}$其實他們不是.
生出來就是一個信心偉人.
而他們不是生出來就會信服上帝.
在整個過程當中.
他有很多和上帝拉鋸的爭持.
從聖經中看到.
這班人其實有自己罪性的一面.
但是他會明白到.
一個重點就是.
有得讓你選擇.
你是不是選錯了呢.
而有得讓你跟的時候.
你會不會選擇跟回一條.
上帝其實預先已經告訴我們.
一條看為正的路.
其實你可以回頭呢.
所以在經文中.
去到第四節的時候.
我們都跟那班.
如同雲彩的見證人一樣.
我們還沒有抵擋流血的地步.
可能我們生命都是.
但是我們都會經歷過很多苦痛.
很多苦毒.
在我們成長環境當中.
都有很多不安.
很多不容易去處理的問題.
我們要處理的就是.
在那段見證人當中.
他們沒有輕看過上帝的參與.
上帝的管教.
因為上帝是愛他們.
上帝願意在他們生命當中插手.
在這件事上.
不知道你覺得.
管教對你來說是什麼事.
在過去這二十年.
牧羊環境當中.
其實聽過很多弟兄姊妹.
原生家庭的苦讀.

$^{241}$和難處.
很多時候都會問為什麼.
那時候是解決不了的.
因為不是一起成長.
也不容易透過一兩個敘事.
去拆解原因.
但是我自己的參與過程當中.
我通常用我以往的環境當中.
很多人問到我那部分的時候.
檢查出現之後.
為什麼會這樣.
我為什麼會有這樣的病.
我通常的答案都會說.
不知道為什麼.
我們可以繼續去檢查.
看看源頭在哪裡.
但是現在你要醫治了.
現在你要決定是否開始療程.
在過程當中.
管教不容易.
治療不容易.
你會發覺有很多問題.
就是問為什麼會發生.
或者那個問題為什麼會出現.
但是在短時間內不容易.
或者在過程當中插手.
要分階段.
但是現在有一個新的轉機.
你願不願意接受.
上帝當中參與的過程當中.
你願不願意讓上帝參與.
譬如聯想過程當中.
上星期徐牧師也說過.
有很多事情.
你是不是選擇不聽.
或者選擇迴避的過程.
這個就是很多時候上帝會等.
上帝一而再再而三地等.
等你過程當中.
你就會有你的彼性.

$^{281}$你就想抗拒.
你想不接受.
但是上帝的愛仍然在當中.
因為祂愛你.
祂想插手在當中.
管教過程當中.
當然可以分管有沒有教.
但是我們的上帝.
他管的過程當中.
他不是管束你.
他不是讓你無法選擇.
但是在選擇的過程當中去教你.
看看前方.
我自己很多時候說到.
都會拿家人做例子.
我家人都沒有發怨言.
不過我都不會說得很直白.
但是我自己過程當中就感受到.
上帝用人倫的方式.
在聖經當中讓我們明白到.
人被罪影響之間的關係.
或許大家對人倫是什麼.
中國的傳統人倫就是.
父子,君臣,夫妻,兄弟,朋友.
這五個人倫都是在說關係的建立.
我自己很多時候都會用家人做例子.
因為上帝給我們每個人都有家庭.
家庭不一定完美.
家庭不一定快快樂樂地生活下去.
但是家庭的破裂都會看到.
在世地上的父母不健全.
不良好.
但是有天父上帝在當中.
天父上帝告訴你.
他可以是你的套意.
施哥很好.
讓我們明白到其實有條路可以繼續走.
就是走上帝的路.
在我兩個兒子小時候.
他們不太記得.

$^{321}$不過有時候我講道的時候.
他們會說我們有這樣講過嗎?.
是不是真的?.
是真的.
(笑).
這些例子有時候在我大祖或茶經的時候.
我都會跟弟姐妹講過.
因為是要被傳頌的.
有一次我小兒子問我.
他說那天我太太不在家.
我們三個在家的時候.
我小兒子就問.
爸爸為什麼你喜歡媽媽?.
可能弟姐妹聽我說過其中一個開場白.
但是這個開場白太普通了.
其實有幾個都差不多.
接著我就心想.
難道這麼快超學生就談戀愛了嗎?.
想認識兩寸關係.
找對的人.
接著我就問.
為什麼這樣問?.
他說你不覺得媽媽很煩嗎?.
(笑).
接著我知道.
對不起,我錯重點了.
我說為什麼你覺得媽媽很煩?.
接著哥哥就說.
你現在才知道嗎?.
(笑).
我說為什麼你覺得媽媽很煩?.
為什麼說完又說?.
我說為什麼說完又說?.
你舉個例子來聽聽.
於是他就說媽媽經常提到的事情.
說完一個就說.
此氣彼落.
不斷地說.
接著我說先停一停.
你們的訴求我聽到了.

$^{361}$但我想問.
為什麼說完又說?.
小兒子就說.
因為我沒有做.
我說是呀,沒有做當然要提.
接著大兒子就說.
因為做得不好.
做不好又要提.
我說忘記了就要提.
做不好又要提.
那還有沒有.
因為那件事很重要.
重要的又怎樣?.
說三次.
所以提一次提兩次和提三次的時候.
你覺得他很煩嗎?.
接著他們的答案跟你們一樣.
我知道了.
接著.
接著他就說.
這樣很煩的嘛經常提.
我說你想清楚.
你媽媽只是提你們.
她沒有提鄰居小明.
沒有.
我說是呀,因為你們是她的兒子.
她只是提你們.
沒有提鄰居小明.
因為她愛你.
想提你.
聽到這裡通常已經覺得.
搞定了.
現在的年輕人很聰明.
他就說.
還是不見媽媽提你.
是不是媽媽不愛你.
(笑聲).
作為兩個兒子的爸爸的做法就是.
媽媽有提我.
不是用你的方法.

$^{401}$為什麼有什麼方法.
對不起,級數不同.
遲些告訴你.
真的.
其實管教是很不容易接受的.
但你有沒有小看那件事.
上帝插手在生命當中的難處.
是你經常想抗拒.
想迴避,不跟從.
因為你有你的路.
你有你的方式.
你覺得自己這麼大.
自己搞定.
但你會發覺.
為什麼上帝要插手在你的生命當中.
不斷挑釁你.
覺得你要做決定.
要在當中參與過程.
就是因為上帝愛我們.
其實上帝在我們未做之先.
祂知道有些問題會發生.
祂要你回轉.
要你去小心.
有一個例子.
我教專學的時候.
就是阿伯拉罕和羅德.
是看到簡地的環境.
一個是索多瑪.
一個是山區.
但是阿伯拉罕提醒羅德.
那個地方雖然很繁盛.
是一個很好的平原.
但那個地方不能去.
那裡會令你出事.
但羅德看不到.
愛他的緣故會提醒他.
很多時候我們都自恃自己的能力見識.
覺得我們有把柄.
可以做到自己的決定.
就像上個星期徐慕斯說.

$^{441}$有些事情我們倚仗慣了.
但是你會發現.
我們不是像羅德那樣.
看到索多瑪的過程當中.
是自己一步一步走進去.
但上帝插手管教的時候.
其實就要我們提醒一件事.
上帝常常提醒我們.
祂是愛我們.
所以不要輕看上帝的管教.
其實是很細微的提醒.
我們只是停一停.
吐一吐.
決定不要太快.
在當中你要了解一下.
為什麼事而做.
第七節的時候.
其實我們忍受到上帝的管教.
上帝對我們像兒子一樣.
因為祂要讓我們明白.
什麼是私生子.
什麼是自己的兒子.
我們比這個世界同化.
不用學的.
就像我上個月說的一樣.
以貌取人.
從你第一天開始.
成長在這個社會裡.
就會教什麼叫以貌取人.
反而不要以貌取人.
是要刻意學的.
這是上帝給我們的標準.
是聖經給我們提醒.
是刻意要改變.
但上帝的管教就是.
要我們明白到.
什麼是上帝想我們親近祂.
而有的特質.
所以教會為什麼不可以.
只靠直播.

$^{481}$要靠身體.
大家一起有接觸.
要混合.
在過程中有團體一起參與.
這是很重要的.
因為教會不只是一個純的.
是知識上的教導.
教會要有溫度.
要有相近.
一起參與的群體.
這是很重要的.
什麼叫兒子.
《約翰福音》第一章第十二節說.
「反接待耶穌的.
就是信他們的人.
他就賜他們權柄.
作神的兒女」.
我們每一個能夠成為.
上帝的兒子.
是因為我們去宣認耶穌基督.
是我們接受了耶穌基督.
我們才能夠再回復.
上帝兒子的特性.
或者身份.
但在過程中我們要知道.
這是我們一起參與.
如果之前那十五個.
被扣為信心的偉人.
他們都經歷過上帝的.
拆毀,迴轉,醫治的時候.
其實我們都是同樣的.
只不過你還沒死.
可能你死了也是信心偉人.
當然你不想這麼激昂.
但其實我們過程中.
我們不斷地經歷上帝.
因為上帝愛我們.
天父上帝上上都希望我們.
明白到我們是他的兒子.
《屠趙》這首歌是我都很喜歡的歌.

$^{521}$上星期徐模斯最後有一個呼召.
呼召我們迴轉.
呼召我們再一次將生命面向上帝.
其實我們第一個呼召.
不是說我們要做傳道人.
不是說我們要全職侍奉.
或者全時間去做一個侍奉崗位.
其實未必去到那一步.
其實第一個呼召.
就是呼召我們再一次.
回到上帝身邊.
生命當中有沒有上帝的時間表.
是很重要的.
上星期說過.
我們可以將我們的時間表排得很密.
去營造一個上帝言語稀少的生活形態.
其實就是逃避上帝再一次呼召我們回到他身邊.
就是再一次逃避上帝的愛.
不在我們生命當中出現.
因為你覺得你自己已經處理到.
但《屠趙》這首歌告訴我們.
其實你走去哪裡.
上帝認定是愛你的.
他一定有方法將你撈回來.
歌詞裡面說.
我躲不開上帝的慈愛.
是因為你不止息的愛.
不斷地搞動我.
縱山深海其實就是包圍我.
意思就是你走去哪裡.
上帝的靈都在當中.
你走不掉.
但反而你走不掉的時候.
你會問為何你還找我.
如果上帝突然開口說.
因為我愛你.
你是否真的會服服.
還是你害怕.
上帝在我們當中.
不只是頭腦認知.

$^{561}$當你感受到你生命被上帝搞動的時候.
或者旁邊有人提醒你.
其實可以找找誰.
或者可以透過一些訊息提醒.
在疫情當中.
我們多人認識Flo Church.
是因為靠很多弟兄姊妹.
包括你.
將你在網上聽到我們的訊息.
分享出去.
因為你經歷過弟兄姊妹有難處的時候.
你覺得那個訊息能夠幫助他.
或者那首詩歌能夠幫助他.
你分享出去.
其實就是搞動他.
也是告訴他.
上帝在這裡.
上帝從來都沒有.
只不過我們掩耳.
只不過我們不想聽那種聲音.
我們不斷地在吐椒.
吐椒就是逃避.
上帝召我們回去.
恢復上帝兒女的身份.
去到最後那幾節經文的時候.
其實你排一排的時候.
你會看到.
原生家庭可能對我們來說.
是很大傷害.
或者很大的限制.
我們感受不到愛.
但是你今天信了主.
你有一個屬靈群體.
你有天父.
他想我們做什麼呢.
他管教我們.
他搞擾我們人生.
想我們回轉.
是要為我們得著.
他預備給我們的福氣.

$^{601}$而在聖潔的聖情裡面.
我們重新調整我們的怨氣.
調整我們的不安.
在過程當中.
希伯來書的作者.
其實做了信心偉人一個很重要的回顧.
就是那十五個信心偉人.
其實有很多的不安.
有很多的不妥協.
但是在早期的時候.
但是你會看到那十五個偉人.
其實他們被上帝整頓的過程當中.
是以年來計算的.
所以不是要比較.
但是如果你的信仰是以年來計算的時候.
其實你試一下快點回到上帝.
無論你攤多少年都好.
好像那些信心偉人當中.
上帝仍然會整治你.
回過頭來.
經歷過這個試煉.
就會得著一個平安的果子.
平安不是說一帆風順.
平安是感受到上帝在你生命當中.
所以有幾點和大家看看上帝的管教.
今天不會說一些怎樣去了解上帝的管教方式.
我剛才說上帝管教的方式.
在他身上不一定重複在你身上.
在他身上的時候對你來說可能不適用.
但是上帝的管教方式有幾個特徵.
第一個特徵就是地上的管教是短暫的.
可能父母會隨自己的心意.
用他覺得管的方式對.
或者用他教的方式對.
但是是短暫.
我們都會離開父母.
會獨立生活.
但是天父的管教.
他是知道我們需要的.
他會讓我們感受到.

$^{641}$和讓他知道上帝給我們的好.
當我說好的時候都不容易.
因為大家的力度差異可以很大.
但是上帝很希望你回轉的時候.
或者你再參與在整個屬靈群體的時候.
你再親近上帝的時候.
你會看到上帝的心意是一步一步來.
管教的益處就是感受到上帝在我們生命當中的同在.
在過去我自己信仰成長過程當中.
經歷了很多不同的事.
其中遇到一些所謂苦難的就是.
旁邊有些人和我同工.
一起參與在福士.
或者一起在當中成為一個夥伴.
但是突然離開.
因為生命當中有些困難.
譬如患病.
譬如家裡遇到很大的難阻困阻.
就離開了.
特別有些是我自己.
如果你上個月聽我說到的時候.
你見到我有一張我做看體士工大籃球的照片的時候.
其中有一個是美國的教練.
他來香港做宣教士.
做看體士工.
教籃球.
他和我拍檔了三年多.
他認識了他太太是韓國人.
他們在香港做扎根.
準備去上海.
他太太教韓文.
他教英文和籃球.
在香港準備的過程那幾年.
其實他一直都和我拍檔.
他教籃球.
他本身在美國打球.
我負責做經理的角色.
一直配搭.
一直相處.
當和他處理好事情.

$^{681}$他準備去上海的時候.
他開始上海的士工.
太太教韓文.
他教籃球.
教英文的時候.
其實三個月.
我就沒有和他聯絡了.
因為很多原因.
有一天我在Facebook看他.
我看到他發帖的時候.
我就說.
為什麼你的照片好像在你老家.
他說是的.
我說為什麼.
我發訊息給他.
為什麼你會在老家.
他說我回西雅圖.
我說為什麼.
為什麼回西雅圖.
這麼快就述職了.
三個月就述職.
他說不是.
我在看醫生.
我今天剛剛進了醫院.
我說為什麼你進了醫院.
原來他就說這三個月.
是發生什麼事.
其實下了上海一個月之後.
他覺得身體有些狀況.
然後回到韓國.
處理了一些事情之後.
就回到西雅圖做檢查.
接著一檢查發現他有腸癌.
我聽到我覺得.
很難接受.
準備了三年.
然後去了三個月.
其實沒有三個月.
是我和他失去了三個月.
他去了一個月之後.

$^{721}$我三個月後再找他.
他告訴我他有腸癌.
我整個人是很低落.
但是反而是他不斷地鼓勵我.
他鼓勵我.
其實我們一直以來.
不是有得做就做嗎.
我們一直都不是.
從來都不是問.
那件事可以做多久.
繼續讓我們做的時候.
我們就繼續做.
我一直在過程當中.
我自己說.
你不如先不要搞這些事情.
你先處理.
你既然回到美國.
你就處理了.
他告訴我.
醫生說應該是不會痊癒.
所以倒數期間.
他幫他的兒子在Facebook設置了.
將來也可以繼續看著他的兒子成長.
當然不是他.
是他身邊的人.
因為他看不到.
他就呼籲和他同行的弟兄姊妹.
一起看著他兒子成長.
平安是什麼.
他感受到上帝的管教.
就是萬物都有事.
不是說他可以做什麼.
平安是感受到.
上帝一直都在當中看著這件事.
他不相信那件事.
因為他停了就停.
所以你問我.
有些跟我們團隊的人.
特別是年青人問.
為什麼上帝會這樣做.

$^{761}$我說我不知道.
但我就知道.
上帝仍然給我有生命氣息.
這一刻還帶著球隊的時候.
我就繼續帶.
我不懂得解釋他.
但我懂得解釋自己.
我仍然有生命氣息.
我就繼續做.
我感受到在做的過程中.
有上帝在當中.
弟兄姊妹也是.
每個人都不知道自己的生命氣息是怎樣.
但你做在上帝的事工.
你做在一個屬靈的群體.
或者做回自己要走的路.
上帝的平安.
就是上帝和我們一起.
於題是「細路」.
我今天講題是「孩子」.
「孩子」「我是這樣愛你們」.
每個人被愛.
或者感受愛的方式都很不同.
你怎樣感受上帝的愛.
上帝怎樣插手.
管教你的生命.
我這個月.
有很多時間讓自己停頓一下.
因為11月.
如果你之前聽我說過.
很多時候我用一季的時間.
去整頓自己.
預備明年的事.
我這個月也有時間讓自己停一停.
其實30年前我是一個什麼光景呢.
就是93年.
93年我已經不能叫做「細路」了.
因為我已經沒有穿校服.
我已經去到一個環境.
就是每天都可以經歷人生的八個常態.

$^{801}$是生,老,病,死,喜,怒,哀,驚.
「細路」可以不用理這些.
但我已經要經歷人生的事.
但在過程當中.
我很感受到上帝開我的眼界.
感受上帝篩污剪了.
我一些不應該再有的東西.
我感受到上帝給我一些管教.
一些插位.
我學習了一件事.
就是「知命」.
知道自己的命要怎樣做.
知道自己的命要怎樣去認識.
我就認命.
認定自己的命要怎樣做.
我就會學習「信命」.
如果上星期你聽了徐牧師的道理.
知道上帝要插手要搞動你的時候.
我盼望你今天就開始要認命.
認命就是上帝愛你.
對你不離不棄.
如果你還是一個「細路」.
覺得自己可以發脾氣的時候.
上帝會繼續插手管教你.
當你順應上帝給你的生命的時候.
我們的命就會順命做上帝.
你會感受到上帝的平安.
不是一個方法.
不過先從第一步開始.
我正在走很細的路.
我的路沒有什麼很慷張.
我一直走一條我覺得可以走得很細的路.
你有你的「細路」.
每個人都在走自己細細的路.
但在走這條路.
你會感受到上帝在其中.
結束的時候.
用詩篇第189篇第71節.
詩篇第189篇是詩篇最長的詩篇.
當中用了22個希伯來文.

$^{841}$做了一個點題.
詩人講了很多關於以色列人經歷的事情.
但重點是一個受苦的民族.
是一個有益在他生命當中的民族.
因為上帝在他過程當中插手.
為要教導他們明白到.
在你們經歷這些苦難之前.
其實上帝早就告訴你.
就是因為你還在搏役.
不跟著做.
最後就兜兜轉轉.
所以每一次受苦.
每一次在困難當中.
其實都是上帝教導我們.
回去再一次看他的說話.
上星期徐牧師說.
試試在我們生命時間表.
讓上帝佔一席位.
試試回應上帝在我們生命當中的時候.
他參與過程當中.
你去回應.
僕人敬聽.
這個呼召.
這個對我來說是很重要.
當我們不斷逃走.
想不跟隨上帝的時候.
但上帝沒有離開過我們.
上帝會跟你說.
我不知道.
但上帝已經跟我說.
當我生命不是順境.
當我遇到很多困難的時候.
我知道上帝與我同在.
上帝跟我說.
孩子,我是這樣愛你們.
當我們生命當中.
讓上帝參與.
上帝告訴我們.
你的好處不會在上帝以外.
就讓我們回轉.

$^{881}$歸去上帝.
上帝要我們走一條很小的路.
但上帝在途中與我們同行.
我們一起祈禱.
天主上帝.
希伯來書的作者再一次用.
這麼多位人.
他們的生命歷練提醒我們.
上帝是每時每刻都在做事的上帝.
今天我們同樣打開你的說話.
我們的生命當中.
我們今天能聚集去敬拜你.
是因為你已經呼召我們.
成為你的兒女.
但期間我們各自走自己的路的時候.
我們不斷逃避你的召命.
逃避我們要跟隨你的路.
但求主你讓我們在二三年.
尾聲的時候.
我們回轉.
走回上帝給我們生命的小路.
這條小小的路.
就是回歸上帝的路.
我們的好處不在你以外.
詩歌再一次提醒.
就是.
永要你是重要的.
讓我再活一次.
讓我們再一次堅守在葡萄樹.
與根本連合當中.
就是上帝你給我們結連當中.
讓上帝掌管我們的生命.
開展我們奇妙的2024年.
求主你幫助.
祈禱奉耶穌的名求.
\newpage



\section{耶利米書 50:29-30-20231118}
\label{sec:2LJqqGa1zFo}
\textbf{【網上崇拜】少年人必仆倒在地上|耶利米書50\_29-30|20231118 [2LJqqGa1zFo]}
\newline
\newline
連結: \href{https://youtube.com/watch?v=2LJqqGa1zFo}{\texttt{ https://youtube.com/watch?v=2LJqqGa1zFo}} ~~~~ 語音日期: 2023-11-18 
\newline
\newline
\hyperref[sec:qJWlmXEzoSU]{\small{< < < PREV SERMON < < <}}
~
\hyperref[sec:index_chronic]{\small{[返順時目]}}
~
\hyperref[sec:index_scriptual]{\small{[返順卷目]}}
~
\hyperref[sec:2LIl7VilU18]{\small{> > > NEXT SERMON > > >}}
\newline
\newline
耶利米書 50:29-30-20231118
\newline
\begin{longtable}{cl}
\hline
\hline
章節 & 經文 (和合本修訂版)\\
\hline
50:29 & \begin{tabularx}{0.7\textwidth}{X} 你們要招集一切弓箭手來攻擊巴比倫,在巴比倫四圍安營,不容一人逃脫。要照著它所做的報應它;它怎樣待人,你們也要怎樣待它,因為它向耶和華-以色列的聖者狂傲。 \end{tabularx} \\ \\ \relax
50:30 & \begin{tabularx}{0.7\textwidth}{X} 所以它的壯丁必仆倒在街上。當那日,它的士兵全都必靜默無聲。這是耶和華說的。 \end{tabularx} \\ \\ \relax
50:31 & \begin{tabularx}{0.7\textwidth}{X} 「看哪,你這狂傲的啊,我與你為敵,因為你的日子,我懲罰你的時刻已經來到。這是萬軍之主耶和華說的。 \end{tabularx} \\ \\ \relax
50:32 & \begin{tabularx}{0.7\textwidth}{X} 狂傲的必絆跌仆倒,無人扶起。我必用火點燃他的城鎮,將他四圍所有的盡行燒滅。」 \end{tabularx} \\ \\ \relax
50:33 & \begin{tabularx}{0.7\textwidth}{X} 萬軍之耶和華如此說:「以色列人和猶大人一同受欺壓;凡擄掠他們的都緊緊抓住他們,不肯釋放。 \end{tabularx} \\ \\ \relax
50:34 & \begin{tabularx}{0.7\textwidth}{X} 他們的救贖主大有能力,萬軍之耶和華是他的名。他必定為他們伸冤,使全地得享平靜;他卻要攪擾巴比倫的居民。」 \end{tabularx} \\ \\ \relax
50:35 & \begin{tabularx}{0.7\textwidth}{X} 有刀劍臨到迦勒底人和巴比倫的居民,臨到它的領袖與智慧人。這是耶和華說的。 \end{tabularx} \\ \\ \relax
50:36 & \begin{tabularx}{0.7\textwidth}{X} 有刀劍臨到矜誇的人,他們就變為愚昧;有刀劍臨到它的勇士,他們就驚惶。 \end{tabularx} \\ \\ \relax
50:37 & \begin{tabularx}{0.7\textwidth}{X} 有刀劍臨到它的馬匹、戰車,和其中混居的各族,他們變成與婦女一樣;有刀劍臨到它的寶物,寶物就被搶奪。 \end{tabularx} \\ \\ \relax
50:38 & \begin{tabularx}{0.7\textwidth}{X} 有乾旱臨到它的眾水,它們就必乾涸;因為這是雕刻偶像之地,人因偶像顛狂。 \end{tabularx} \\ \\ \relax
50:39 & \begin{tabularx}{0.7\textwidth}{X} 所以野獸和土狼必住在那裡,鴕鳥也住在其中,永遠無人居住,世世代代無人定居。 \end{tabularx} \\ \\ \relax
50:40 & \begin{tabularx}{0.7\textwidth}{X} 巴比倫要像神所傾覆的所多瑪、蛾摩拉和鄰近的城鎮一樣,必無人住在那裡,也無人在其中寄居。這是耶和華說的。 \end{tabularx} \\ \\ \relax
50:41 & \begin{tabularx}{0.7\textwidth}{X} 看哪,有一民族從北方而來,有一大國和許多君王被激起,從地極來到。 \end{tabularx} \\ \\ \relax
50:42 & \begin{tabularx}{0.7\textwidth}{X} 他們拿弓和槍,性情殘忍,毫不留情;他們的聲音像海浪澎湃。巴比倫啊,他們騎著馬,如上戰場的人擺列隊伍,要攻擊你。 \end{tabularx} \\ \\ \relax
50:43 & \begin{tabularx}{0.7\textwidth}{X} 巴比倫王聽見他們的風聲,手就發軟,痛苦將他抓住,彷彿臨產的婦人疼痛一般。 \end{tabularx} \\ \\ \relax
50:44 & \begin{tabularx}{0.7\textwidth}{X} 「看哪,就像獅子從約旦河邊的叢林上來,攻擊堅固的居所,我要在轉眼之間使迦勒底人逃跑,離開這地。我揀選誰,就派誰治理這地。誰能像我呢?誰能召我出庭呢?有哪一個牧人能在我面前站得住呢? \end{tabularx} \\ \\ \relax
50:45 & \begin{tabularx}{0.7\textwidth}{X} 你們要聽耶和華攻擊巴比倫所定的計劃和他攻擊迦勒底人之地所定的旨意。他們羊群當中微弱的定要被拖走,他們的草場定要變為荒涼。 \end{tabularx} \\ \\ \relax
50:46 & \begin{tabularx}{0.7\textwidth}{X} 因巴比倫被攻下的聲音,地就震動,人在列國都聽見呼喊的聲音。」 \end{tabularx} \\ \\
[1ex]
\hline
\hline
\end{longtable}
$^{1}$各位丁字妹晚安.
下個星期是Fansgiving week.
通常back friday是星期下.
下星期五在美國人們凌晨排隊.
12點就會一元大約75吋的電視機賣.
所有人就衝進去.
所以Fansgiving week是一個很隆重的日子.
所以趁這個時候我想說一些感恩的事情.
在一個艱難的世代裡面.
要說感恩的事情很難.
最近幾個月前.
兩三個月前左右.
中國美加宣佈來香港開演唱會.
其實最感恩是什麼.
最感恩是Dee迪尼問我要不要飛.
我沒有回答他,因為很尷尬.
事緣上年的時候有一個姐妹Folk Church.
她將她家裡收藏的中美加的CD.
我猜有七八隻左右.
演唱會或其他.
她整疊CD給了我.
不過我是不想要的.
因為現在已經沒有CD player.
這是一個最關鍵的問題.
你給我都不知道怎樣播.
如果大家都是YouTube或聽Spotify.
大家紀念中國美加開演唱會.
紀念我問我有沒有飛我是感動的.
上個月我們說密西斯推理.
我所知道的.
很多人去了看.
我所承知的.
很好看的第十集.
有個Folk Church的目者.
發了一條影片給我看.
那條影片是他的組員.
他自彈了一首主題曲.
然後發給我.
我將那條影片放在我的Signal的群組裡.
很重要的東西就放在那裡.

$^{41}$我是很感動的.
我聽了十次.
他彈了兩首.
他有一隻貓.
如果他今天存在的話.
即是在這裡的話.
你輕輕撥一下頭.
你彈一首這麼好聽的主題曲給我聽.
在艱難的時候.
其實有些人做一些事是很感動的.
尤其是.
就算是一些很細微的事.
我覺得這一刻我心裡面都很感覺.
很溫暖.
今天我們會說一個題目叫.
少年人必碌倒在街上.
小心一點說.
通常潘Sir不會問我說什麼.
通常大家都很信任.
這次不知為什麼.
他看完這個題目之後.
馬上打來問我.
家Sir你說什麼.
不好意思.
潘Sir真是.
我和他說一些很正面的東西.
我等一下說一些不是他能說的東西.
我們先不說這個題目.
這個題目我想是怎樣想的呢.
因為我一月的時候說耶利米蘇一章.
如果你回去FourTrack再聽一月的時候.
我是說耶利米蘇一章.
所以11月12月的時候.
我想說耶利米蘇和耶利米愛國.
成為今年的結局.
他們說耶利米蘇一章的篇道.
是說拆毀的年代.
我們不要當閉上眼睛.
復常之後就以為一切都復常.
其實我們仍然是一個拆毀的年代.

$^{81}$一個拔出的年代.
這篇道我估計今年已經說了60次.
我已經很熟悉這篇道.
熟悉的不能再熟悉.
我已經整年都說了60次以上.
所以這個月我準備了.
是耶利米蘇50章的29-30節.
我們說兩節.
不過我今天會說得大一點.
闊一點.
我們下一章.
其實耶利米蘇有三個結局.
你知道耶利米蘇是一個.
耶路撒冷被毀後寫的書卷.
如果大家知道.
耶利米蘇是一個.
耶路撒冷被毀被毀.
結果被毀.
大約30多張到44張的時候.
說被毀.
人們開始逃跑.
流散.
去了埃及.
但有幾張聖經說.
人們怎樣下到埃及.
耶利米蘇也下到埃及.
但下到埃及之後.
耶利米蘇的結局是怎樣呢.
其實聖經沒有說.
不過在聖經裡鋪陳了三個結局.
我們下一章.
下一章拍拍.
第一個結局是45章.
其實45章是整個耶利米蘇裡.
其中一個結局.
這個結局是最短的結局.
寫完這章其實是寫完的.
不過後面會再寫兩個不同的結局.
關於整個耶利米蘇.
第一個結局是.

$^{121}$我看到後面的字這麼細小.
你知道我不想看.
我只想看最大.
鋪了的三行仔.
他說約翁在位第四年.
耶利米先知對巴祿.
其實巴祿就是他的書記.
他的書記記載了耶利米說過什麼.
那個叫巴祿.
當時巴祿就照耶利米的口述.
就把這些東西寫在卷子上.
這個結局就寫完.
因為其實耶利米說什麼都好.
當時的皇帝.
不同的時代的皇帝都好.
都要抓耶利米打壓.
因為他說耶路撒冷快要玩完.
說一些不說好耶路撒冷的故事的人.
這些就要死了.
大約是這樣.
所以那些皇帝就把耶利米.
抓得抓.
想抓他就抓.
想殺他就殺.
結果他逃脫了.
或者他被人抓過.
被人打過.
但他留下的東西.
其實那些人都燒了.
不過巴祿就把他口述的東西寫下.
留下.
所以今天我們有耶利米蘇.
因為有巴祿留下.
結局就是這樣.
這樣就結束是挺好看的.
應該不是那麼長.
是45章.
很短的結局.
就是耶路撒冷玩完了.
巴祿就記下了.

$^{161}$曾經神直著耶利米蘇說的東西.
如果這個結局像什麼.
找個例子說一下.
譬如我們做社會老師.
通常有些神學生就會去找工場.
他就去這些工場.
他會問你.
加上找你做這個諮詢人好不好.
我說你去找哪一間.
通常我們見到某些教會.
某些.
我就會跟他說.
不要去.
通常神學生一開始都很熱心.
他有拯救主的情況.
等我來搞定.
結果他去不到三個月後.
就會打來跟你說.
想走啊加Sir.
有沒有介紹.
大約故事像這個.
我已經預言他會走路.
他會走不長路.
他可能記得.
不過這個人記得.
故事的結局就是.
他想起原來有老師提過他.
這份工作走不長路.
不要做.
多謝你.
所以耶利米蘇45章.
大約想說這件事.
耶路撒冷被毀.
神原來直著耶利米蘇.
多年來十幾年來說了.
耶路撒冷被毀.
一早說的.
現在真的被毀了.
所以他的話語留下.
結局大約是這樣.

$^{201}$第一個結局是這樣.
我們看看第二個結局.
第一個是46章至51章.
這幾章聖經是最爽皮的.
最開心的.
看得最舒服的.
因為結局是什麼呢.
耶路撒冷被毀後.
接著發生什麼事.
結果是46至49章.
列國都一樣玩完.
埃及 阿滿 以東 摩洛 丹麥 澀甲 基達 夏蘇 以南.
都玩完.
不過有一個國家.
用兩張聖經寫他.
寫壞他.
有這麼差的寫他.
就是巴比倫.
巴比倫攻陷了整個耶路撒冷.
所以他有一個差的結局.
他將來會被耶和華.
重重懲罰他們.
用兩張聖經寫他有多壞.
所有青年人 少年人 什麼人都要死.
這是一個很爽皮的結局.
用這麼多張聖經說.
你幸災樂禍 以色列玩完.
列國都會玩完.
用一個現代的例子像什麼.
像渣男.
譬如渣男的腳撞斷了.
很開心.
你明白嗎.
如果你認識渣男.
他的腳跌了.
這些很爽皮.
我沒有看內容.
你可以回去看.
內容寫得很差.
有這麼壞的寫下去.

$^{241}$其實像渣男撞斷腳會好起來.
不夠壞.
如果是渣男.
每次見到女生.
女生都吐一口水.
哇 爽啊.
以後都不能拍拖.
這個就做勁.
這是很爽皮的結局.
你自己不開心不行.
那些得罪你的人.
原來比你還壞.
等於很早期.
我那個年代最絕的情歌是什麼.
你沒有好結果.
你記不記得.
年輕時的歌.
李慧敏 沒錯.
沒錯.
我有一年在高街遇到她.
她在做地產.
那些歌是很絕的.
我慘了.
原來你將來的結局比我還差.
活該你死了.
所以基本上這兩個結局.
是我們平常人期望的結局.
你經歷很差的事.
我早就跟你說了.
上帝說的.
你現在才知道嗎.
第一個結局就是這樣.
得罪你的人.
傷害你的人.
他們沒有好結果.
結果比你還差.
你什麼時候見到很邪惡的人.
無端端有隻蛆在吐你的氣咬他.
那種感覺很開心.
終於有這樣的結局.

$^{281}$但這是我們平常會想的結局.
他做了一個結局.
是52章.
再看下一章.
52章不是用一些以往說過的東西.
也不是說將來會發生的東西.
52章的結局是另一類的結局.
52章的結局是耶路撒冷淪陷了.
西帝交往.
最後一個皇帝終身坐牢.
耶路撒冷裡面的祭司和領袖.
被巴比倫王在河邊擊殺了.
頭三個都很淒涼.
其實是state the facts.
52章的結局不是用之前說過的東西.
也不是說預言將來的敵人有多麼的哲學.
是說耶路撒冷被淪陷後.
所發生的每一件事.
發生成怎樣.
每一個人好像在追劇情.
神會不會突然出手.
神會不會突然做事.
所以在結局的時候.
他說了一些很差的東西之後.
他說了上帝做了一件事.
做了一件事後覺得有點不舒服.
是什麼呢.
約翰被恩代.
約翰好像是被西里加囚的.
不過西里加囚很慘.
他被人挖蠻傷眼.
被人帶去巴比倫.
約翰被捉在監獄.
他最後的日子被巴比倫王提出來.
能夠和巴比倫王一起飲食坐餐.
終於有些好的消息.
在淪陷之後.
有些好的消息聽了一下.
這個像什麼呢.
像我今早.

$^{321}$今早十點在台灣發生的事.
你知道11號至17號的民調選舉.
看到底是OP還是侯先生.
做總統還是副總統的排名.
今早又看著看著.
到底怎樣呢.
1月13日是否應該飛去台灣.
看大選呢.
當你人生開始.
從此人生沒有選舉之後.
我覺得我沒有選舉.
之後.
我沒有選舉之後.
你可不可以看看人家有選舉是怎樣做的呢.
你很想看下到底.
當地人會發生什麼事.
會想些什麼.
當地人會到底.
他們面對這些事的時候.
他們在經歷什麼.
他們做了決定還是不做了決定.
好像你看到還有50多天.
到1月13日台灣總統大選的時候.
你好像每一件事發生.
你都想關心想理解想明白.
想告訴人家.
神有沒有做些什麼呢.
神會怎樣做呢.
神其實會是怎樣的呢.
到底藍白放在一起叫什麼顏色呢.
藍和白放在一起叫淺藍色.
淺藍色一定會贏.
還是綠色會贏呢.
我相信52章的結局是最正常的結局.
是我們想看著一些事實.
一些真實會發生的事實.
每個小時每一日每一個星期.
會不會有些事情突然轉變.
會不會突然有些事情.
令一個很淒涼的耶路撒冷被淪陷之後.

$^{361}$有些好消息聽一下.
拜託.
這幾年很慘.
有些好消息聽一下.
可不可以不要.
沒有好消息.
經常都.
還有沒有好消息來.
還有這陣子.
落親街有很多旗子放在那裡.
很納悶.
很想瞪著他.
有沒有好消息.
不要經常都是.
那些不好的消息經常都走出來.
還要充斥到你周圍.
52章是說.
面對耶路撒冷被淪陷之後.
大家關心一些很真實發生了的事情.
怎麼發生.
我不知道他為什麼要停下約阿根.
在監牢裡被人抓出來.
從此以後和巴比倫王吃飯.
可能是整個被淪陷耶路撒冷之後.
唯一一件耶路撒冷的義人.
值得開心一下的事.
接著問題要問.
為什麼耶律書有三個結局.
這三個結局代表什麼.
或者這三個結局出來的時候.
其實想表達的是一件什麼事情.
我們下一章看看.
其實三個都不同重點.
不過今天我想用一個解釋一些東西.
我們再下一章.
我們先講這章.
講完這章再講另一個.
其實有三個結局.
第一個關鍵是.
其實我們社會充滿了很多不確定性和多元性.

$^{401}$Ambiguity.
不確定性是在創傷之後的人所經歷的不同環節裡.
突然間不確定性出現了很多.
假設你失業.
失業的話你怎麼辦.
你要騙家人假裝上班.
你要坐公園坐星巴克.
如果你失戀的話.
你要在IG上.
突然間刪除所有和他相片.
以前看他留言的那些很肉麻很溫馨.
很開心.
現在你要刪除那一刻覺得.
賤格變態佬死人頭.
那些突然間你心目歷程.
改變了很多東西.
就像香港人一樣.
香港人還沒走完.
我過了一個星期才有人離開.
那些小朋友是玩了十多年.
那些小朋友一分開的時候.
他知道他走了.
那七八個小朋友.
知道有兩個小朋友離開了.
哭得多淒涼.
你可以想像.
現在在我們經歷了幾年的創傷之後.
對整個香港社會的不確定性很大.
你不知道買樓還是不買樓.
買樓的人發現樓價會跌了十幾二十個百分比.
為什麼成哥可以一劈一劈.
就劈三十個百分比.
買他的樓.
為什麼我人生這麼不幸運.
為什麼我幾年前就買了他.
那麼蠢那麼愚蠢.
現在等樓價跌了才買.
我們突然面對很多不確定性.
那些不確定性是說.
我們不知道要面對什麼.

$^{441}$或者將來前面的路可以怎樣走.
很多東西突然出現.
唉 為什麼會這樣.
為什麼那些東西會走出來.
當你失戀的時候.
突然走回一個地方.
和渣男一起吃飯的時候.
你會覺得.
我的天啊 這個地方會不會消失在我面前.
你以前不會那麼毒辣.
你不會那麼讓人家倒閉.
你很溫柔 很man.
你突然發現自己變態了.
突然要將一些東西就坐到一個很淒涼的地步.
你連你自己裡面的不確定性.
你都不知道是什麼事.
就連看著很多旗飄揚的時候.
你突然間裡面很憤怒.
可不可以有一張單張的地板.
一腳踩下去.
剛好剛才我走過來的時候.
有一張單張的地板.
我就不知道為什麼.
又是一腳.
然後就假裝沒事 繼續走.
你開始認識自己.
你是在生氣的.
你是不開心的.
你是不知道怎樣表達的.
也不知道什麼時候你可以表達的.
這個成為了.
現在香港人.
烏克蘭裡邊的人.
或者在各地裡邊受苦難的人.
他們有一個很重要的特徵.
我們要說兩節聖經.
我們很快.
下一章 麻煩你.
《仆倒在地上》.
我怕我說錯了.

$^{481}$所以我找了另一個譯本.
他說為此.
他的少年人將倒在廣場上.
在那一天.
他所有的戰士都要被消滅.
然後說將宣告.
你知道50 51章是說.
神就坐巴比倫.
就坐到一個什麼地步呢.
是說巴比倫.
他太太是指巴比倫.
他的所有少年人.
都在廣場裡邊全部死了.
下一章PowerPoint.
其實那章PowerPoint說什麼呢.
剛才那章PowerPoint說什麼呢.
其實是好像.
昔日出埃及那樣.
要向他們發怒.
滅盡他們的長子.
當然.
以前出埃及的時候.
滅長子.
你看剛才50章29節的時候.
不是說長子這麼簡單.
是所有的少年人都死了.
比起出埃及的時期.
那個發怒所經歷的更差勁.
你可以想像.
51章是多麼殘忍的說話.
如果我們面對著.
創傷後遺的時候.
你面對著渣男.
面對著一個很討人厭的老闆.
炒他魷魚.
面對著人生很多苦難.
不同的事情的時候.
你很討厭.
為什麼我的遭遇要這樣.
為什麼人家可以好好的.

$^{521}$為什麼我終於要經歷這麼多.
不容易的事情的時候.
你有很多埋怨.
很多難過的時候.
沒錯.
我們其中一個出路是.
那些得罪我們的人快點死掉.
快點被人撞車.
撞死他也好.
他的兒子全部死了.
哈利亞債未主也好.
我們可以背誦整個51章.
但你知道.
不可行的.
如果只背誦50-51章.
不可行的.
你知道.
我為什麼成為一個.
這麼毒辣的人.
所以需要有.
有同時間靈感.
相信上帝的話語永存在.
我們想要45章.
就是神說了.
所以我們要聽.
所以神有.
剛才那套詩說了什麼.
他必有恩.
萬事都有結局.
你明白嗎.
我們突然間會.
想將一個神諭論的觀念出來.
Theodicy.
我們想突然間.
所有事情都皆有因.
長出信三禱告.
原來神一定掌權.
我喜歡讀聖經.
但就算你查很多關於神諭論的東西.
你查經.

$^{561}$什麼什麼.
詩篇怎麼說.
當洪水泛淡的時候.
耶和華怎樣.
坐著為王.
他坐著坐著.
你經常查完.
他坐著為王.
萬事必有因.
哇 信心要禱告.
哈利路亞.
當你信心宣告了十次八次的時候.
你看著事實的時候.
其實那些耶和華坐著為王.
那些萬事皆有因.
不確定.
都好像不是很可行.
你面對一個越來越爛的局面的時候.
你要騙自己說.
不是啊 很有信心.
沒錯.
我可以一群人一起唱的時候.
很有音樂的那種.
啪啪的時候.
耶信.
但你突然間回到家.
夜裡人靜的時候.
就算有更差的消息出現的時候.
大體上你不會很容易說得出吧.
我猜阿伯拉罕也做不到吧.
如果不是的話.
他不會跟埃及的王說.
這個是我的妹妹.
如果都不可行的話.
是什麼呢.
就是看著事實.
到底誰發生的事情.
令到事情可以改變.
所以我喜歡聽.
聽燈神說話.

$^{601}$聽一些YouTuber.
到底發生的事情會怎樣呢.
他從事實去推論一下.
將來發生的事情就會變成這樣.
其實你只要聽某些人.
一年前兩年前說的那些話.
你就知道.
憑那些事實你會推斷到的東西.
是不是真的如他所說的推斷到.
如果不是的話.
他也不會叫燈神.
我今天想說的是.
其實當一個群體.
經歷很難的時候.
我們需要有很多不同的結局.
不是結局.
不同的治療.
不同形式的安慰.
放在我們生命裡.
不知道在哪個時間.
我們需要哪一種安慰.
成為安慰到我自己.
一個很重要的出口.
如果說2023年最後的時候.
要跟一群仍然有心在.
香港教會頂展會.
一起努力的話.
我想說的是.
不要輕看過去那幾年.
對我們帶來很不容易的感情.
那種傷害.
如果不是我不會看著那張單張掉在地上.
我沒有意義.
它剛好在我腳前.
我不會突然變態地.
一踏進去很開心.
都不知道開心做什麼.
整天笑個雞.
但總之有個勝利法.
是不是?.

$^{641}$踏踏勝利法.
踏踏勝利.
連這些低B的勝利.
你都覺得安慰了自己多一點.
面對著前面.
我們不知道會怎樣的局面.
又不是要看著事實.
去看我們前面的路要怎麼看.
又不是單單說.
上帝要就坐所有人就就坐他們.
我們就滿心安慰.
我又不會靠著一定說.
神已經說了一定是這樣.
所以現在是這樣.
而那些東西就會滿足到我.
如果不是只有一樣東西.
能夠成為一個人的安慰的話.
或許我會說的是.
會不會我們需要有不同的人.
有不同的安慰呢?.
我試試用這個例子說清楚.
烏克蘭有一個Tattoo的.
叫什麼?.
專幫人Tattoo.
Tattoo是紋身.
專幫人紋身的師傅.
他在烏克蘭過去一年紋身.
你知不知道大約九成多的人.
紋身紋什麼?.
只有一個主題.
大家都知道.
因為紋得很低能.
不需要回答.
九成多的人紋身.
都是關於戰爭的主題.
No doubt.
如果去紋身的話.
他不會紋花紋心.
不會.
在烏克蘭Tattoo師傅那一年.

$^{681}$他紋身裡面紋什麼多?.
你可以想像.
其實不難猜.
他紋什麼?.
他說.
那個人的爸爸打仗死了.
他家裡有個炸彈下來.
他一家五口不在.
所以他決意紋身紋什麼?.
紋那些被炸彈炸死了家人的名字.
你可以想像是這樣.
我爸爸去打仗死了.
我紋我爸爸的名字.
在我不知道哪個部位.
我不知道.
但會紋這些東西.
你可以想像.
有些人會紋什麼?.
一個很光的十字架.
十字架裡面.
下面有一個墳墓.
墳墓裡把死了人的名字寫在那裡.
他留了一個空位.
他說到他死的時候.
希望紋身師傅也幫他紋上名字.
他說他地上的日子.
可以和這群人相處的日子.
家人相處的日子沒有了.
他盼望天家的日子可以再重逢.
還會紋什麼?.
他們會紋他們被炸的那條街.
他拿了一張照片.
以前家裡的照片.
他拍一拍.
在Facebook那邊.
不可能在Facebook或其他.
他拍那家附近的街道的照片.
被炸毀了.
他說紋身師傅.
你可不可以把那條街和我家.

$^{721}$紋在我身體裡.
雖然家裡沒有了.
但是那些大廈.
那些街道.
那張照片.
我想成為我身體裡的一部分.
我忽略其想在想.
我想如果.
那個紋身師傅可以把他紋身了的人.
不同的紋的內容.
能夠把一個PowerPoint出來.
放在這裡的話.
你知道最大的得著是什麼嗎?.
是人想不到原來.
人會用不同的方法.
不同的想像力和新思維.
去想一些東西.
是可以安慰到自己的.
你看著有二十幾三十個不同.
可以安慰到自己的方法和圖案的時候.
你覺得你身邊.
正在受到一個很大的安慰.
我最後想說的是.
我親愛的香港電影節目.
起碼我跟自己說的是.
我想擴闊.
我不想只有一招來安慰我自己.
我不想只有一樣東西.
看著一個很難的困局的時候.
兩三年都是靠著一兩招來告訴自己.
仍然有盼望.
信心必討告.
萬事必有因.
我不想看著一些Facts的時候.
又沮喪又不行.
不要緊再來一次吧.
我不想只有一兩招.
可不可以我們這個群體裡面.
有更多不同的創意.
有不同的想像.

$^{761}$讓我們這班人會覺得.
在這個時代裡面.
上帝用了不同的方法.
去安慰了不同的人.
那些不同的方法.
可以成為很多人的安慰.
而那些安慰成為了我們的力量.
繼續在不知道的將來.
我們走下去.
我這樣說完.
其實我想完這一點後.
我發覺我自己都不太convince.
不convince是甚麼呢.
因為年紀大.
就我講我.
年紀大就不太想受安慰.
很想與那些.
看著那些事實共存.
那種差下去.
不斷地差下去.
很灰磊.
那些叫甚麼.
又不是叫躺平.
那些擺爛.
很想擺爛.
跟他鬥擺爛.
那些擺爛的情緒.
很侵蝕.
很毀了心智.
我出賣了我兒子.
我就完了.
我兒子這個星期打比賽.
下一張powerpoint.
下一張完.
我兒子這個星期打比賽.
班制比賽.
班制比賽他打籃球.
兩場比賽他們那班都贏了.
有一場比賽是他入的.
他射入了一球.

$^{801}$一比零.
你知道小學生打比賽.
一比零 二比零.
一比零贏了就好.
很開心.
之後星期四 星期四 五.
跟那些五 六年班打.
他贏了那班.
星期四贏了那班.
跟那些六年班打比賽.
輸光了.
我以為輸光了.
我昨天安慰他.
想安慰他.
我兒子.
不要緊.
因為我今晚看了一些片.
那個很高大的六年班.
我兒子一射球的時候.
一下子拍下去.
當然是打球.
怎會不打他.
難道真的讓他射嗎.
那個六年班又很高大.
整個球拍下來.
我看到那些片.
我就想安慰兒子.
兒子不要沮喪.
努力加油.
前面有很多光明.
將來你高過去.
想安慰他.
怎知道他.
問完他兩句.
沒事了.
他給了我什麼看.
他給我看贏球的那場.
那個球.
他說 哇 看看.
看完之後.

$^{841}$他射了一球給我看.
射完 哇.
我讚他.
讚他也讚不及.
他給我看.
我說厲害 射得很厲害.
接著他說爸爸你看後面什麼聲音.
射球後.
我兒子的名字.
陳X$\times$加油.
陳X$\times$射球.
陳X$\times$好球.
他還笑了一笑.
他說爸爸你聽聽.
女孩子的聲音.
他還說了哪個女同學的名字.
稍安毋躁 我還沒緊張.
稍安毋躁.
女兒緊張一點.
兒子我不太緊張.
對不起.
不對.
沒有說過.
這個女孩子我在想什麼.
我在想.
是啊.
小朋友對輸了.
好像我們會覺得他很不開心.
他轉幾個頭就完了.
他看他射球的樣子多開心.
他拿給我看.
他的同學在下面寫著.
四信班加油.
同學們一起.
那個時氣.
輸了也不要緊.
有一張照片是一班同學.
輸了.
老師請他們喝東西.
一班人喝東西.

$^{881}$很開心.
耶.
我心想不是輸了嗎.
笑得這麼不開心.
在成人世界裡.
才覺得輸了是意味著很多.
小朋友.
在艱難的時代裡.
球天父讓你和我做一個小朋友.
學一個小朋友很簡單.
面對很多失敗.
很多艱難.
很多難處的時候.
沒錯.
那些很真實.
我看著小朋友也覺得意味著很多.
想像一下.
有人問你買不買票中島美加.
你想像一下.
那些中島美加的CD.
你又想像一下那位背脊.
只有背脊的姐妹.
彈那首《物說時推理》的主題曲.
那些像什麼.
像地上的擔醬.
踩一下.
耶.
讓這個情懷.
在一個艱難的香港時代裡.
成為我們一個很重要的熟齡操練.
耶利米斯三個結局.
代表著不會只有一個結局.
能夠使人心得安慰.
三個結局或三個以上的結局.
因為耶利米斯愛歌.
在說.
我們需要不同形式的安慰.
今時今日.
什麼熟齡操練重要.
除了在拆毀的年代裡.

$^{921}$我們要想什麼是拆毀年代裡.
需要的熟齡操練之外.
我相信.
在拆毀的時候.
還在拆毀的時候.
如何在繼續拆毀的裡面.
看著那些很真實的事實.
很難過的場景.
我們可以很快地.
耶一下再走下去.
最後我想跟海外的弟兄姊妹說.
我最近聽了一個.
回來在外國的木槿會的同工.
她說她過去那一年多兩年的時間.
做了很多在海外的弟兄姊妹的需要.
她說她的弟兄姊妹.
很真實地面對很多很難的處境.
她的身體很多都不是很健康.
各原因.
經歷了不同的事情.
不單身體很難好得回.
心靈很難過.
她說那些人很乖.
四點天黑開始depression的時候.
她都不找她.
她忍到十點鐘才打給她.
在哭哭哭.
說自己為什麼會在這裡.
為什麼自己回不來.
為什麼還在這裡.
好像毫無意義和生命力.
那個同工說.
你不要認真聽.
你一認真聽.
你陪她哭的話.
你哭得多少晚.
我希望將這個訊息送給海外的弟兄姊妹.
當你很難很不容易的時候.
你記得無安慰的方法不是只有幾個.
求天父給我們群體裡.

$^{961}$有不同人create不同的安慰方法.
讓很多受傷的心靈.
得到意志和安慰.
我更願意海外的弟兄姊妹們.
你活像一個小朋友.
活像一個很小的時候的自己.
享受上帝給我們很多點點滴滴的安慰.
成為我們心靈裡.
上帝真實存在的記號.
求天父使用香港教會.
祝福我們的群體.
成為很多人安慰的泉源和幫助.
我們聽到禱告.
天父多謝你給我們今天這個空間和時間.
去思想你自己.
關於耶利米蘇的結局.
我求天父你給我們.
不擺爛.
來到你面前的時候.
不是因為很多苦難和困境.
我們放棄了很多.
不應該我們放棄的東西.
我祈禱求神的是.
你幫我們每一個弟兄姊妹.
來到你面前的時候.
好像你所喜愛的小孩子.
近到你面前來.
今時今日的小孩子.
不是純粹一個天真萬難的孩子.
是一個經歷憂患,苦澀.
面對局勢困難的時代.
我們怎樣像一個孩子來到你面前來.
求你親自保守帶領著我們.
讓我們這樣去學習.
多謝天父你聽我們在你面前的祈禱.
奉基督耶穌保貴命寇.
\newpage



\section{啟示錄 12:1-17-20231125}
\label{sec:2LIl7VilU18}
\textbf{【網上崇拜】細路仔唔識世界|啟示錄12\_1-17|20231125 [2LIl7VilU18]}
\newline
\newline
連結: \href{https://youtube.com/watch?v=2LIl7VilU18}{\texttt{ https://youtube.com/watch?v=2LIl7VilU18}} ~~~~ 語音日期: 2023-11-25 
\newline
\newline
\hyperref[sec:2LJqqGa1zFo]{\small{< < < PREV SERMON < < <}}
~
\hyperref[sec:index_chronic]{\small{[返順時目]}}
~
\hyperref[sec:index_scriptual]{\small{[返順卷目]}}
~
\hyperref[sec:w1NzLUX2_GE]{\small{> > > NEXT SERMON > > >}}
\newline
\newline
啟示錄 12:1-17-20231125
\newline
\begin{longtable}{cl}
\hline
\hline
章節 & 經文 (和合本修訂版)\\
\hline
12:1 & \begin{tabularx}{0.7\textwidth}{X} 天上出現了一個大兆頭:有一個婦人身披太陽,腳踏月亮,頭戴十二顆星的冠冕; \end{tabularx} \\ \\ \relax
12:2 & \begin{tabularx}{0.7\textwidth}{X} 她懷了孕,在生產的陣痛中疼痛地喊叫。 \end{tabularx} \\ \\ \relax
12:3 & \begin{tabularx}{0.7\textwidth}{X} 天上又出現了另一個兆頭:有一條大紅龍,有七個頭十個角;七個頭上戴著七個冠冕。 \end{tabularx} \\ \\ \relax
12:4 & \begin{tabularx}{0.7\textwidth}{X} 牠的尾巴拖拉著天上星辰的三分之一,把它們摔在地上。然後龍站在那將要生產的婦人面前,等她生產後要吞吃她的孩子。 \end{tabularx} \\ \\ \relax
12:5 & \begin{tabularx}{0.7\textwidth}{X} 婦人生了一個男孩子,就是將來要用鐵杖管轄萬國的;她的孩子被提到神和他寶座那裡去。 \end{tabularx} \\ \\ \relax
12:6 & \begin{tabularx}{0.7\textwidth}{X} 婦人就逃到曠野,在那裡有神給她預備的地方,使她在那裡被供養一千二百六十天。 \end{tabularx} \\ \\ \relax
12:7 & \begin{tabularx}{0.7\textwidth}{X} 天上發生了爭戰。米迦勒同他的使者與龍作戰,龍同牠的使者也起來應戰, \end{tabularx} \\ \\ \relax
12:8 & \begin{tabularx}{0.7\textwidth}{X} 牠們都打敗了,天上再也沒有牠們的地方。 \end{tabularx} \\ \\ \relax
12:9 & \begin{tabularx}{0.7\textwidth}{X} 大龍就是那古蛇,名叫魔鬼,又叫撒但,是迷惑普天下的;牠被摔在地上,牠的使者也一同被摔下去。 \end{tabularx} \\ \\ \relax
12:10 & \begin{tabularx}{0.7\textwidth}{X} 我聽見在天上有大聲音說:「我神的救恩、能力、國度,和他所立的基督的權柄現在都來到了。因為那個在我們神面前、晝夜控告我們弟兄的,已經被摔下去了。 \end{tabularx} \\ \\ \relax
12:11 & \begin{tabularx}{0.7\textwidth}{X} 弟兄勝過那條龍是因羔羊的血,和因自己所見證的道。雖然至於死,他們也不惜自己的性命。 \end{tabularx} \\ \\ \relax
12:12 & \begin{tabularx}{0.7\textwidth}{X} 所以,諸天和住在其中的,你們都快樂吧!只是地和海有禍了!因為魔鬼知道自己的時候不多,就氣憤憤地下到你們那裡去了。」 \end{tabularx} \\ \\ \relax
12:13 & \begin{tabularx}{0.7\textwidth}{X} 龍見自己被摔在地上,就迫害那生男孩子的婦人。 \end{tabularx} \\ \\ \relax
12:14 & \begin{tabularx}{0.7\textwidth}{X} 於是有大鷹的兩個翅膀賜給婦人,讓她能飛到曠野,到自己的地方,躲避那蛇。她在那裡受供養一載二載半載。 \end{tabularx} \\ \\ \relax
12:15 & \begin{tabularx}{0.7\textwidth}{X} 蛇在婦人背後,從口中噴出水來,像河一樣,要將婦人沖走。 \end{tabularx} \\ \\ \relax
12:16 & \begin{tabularx}{0.7\textwidth}{X} 地卻幫助了婦人,開口吞了從龍口噴出來的水。 \end{tabularx} \\ \\ \relax
12:17 & \begin{tabularx}{0.7\textwidth}{X} 於是龍向婦人發怒,去與她其餘的兒女作戰,就是與那些遵守神命令、為耶穌作見證的。那時龍站在海邊沙灘上。 \end{tabularx} \\ \\
[1ex]
\hline
\hline
\end{longtable}
$^{1}$.
因為啟示了這段經文.
一來有很多人說不明白.
二來剛才有人跟我說.
原來在YouTube也有問題.
就說為甚麼這段經文會有這個講題呢.
為了讓大家對這段經文可以有深刻的印象.
所以希望透過剛才這個演繹.
大家可以對於經文的描述可以立體一點.
但最重要我們也是求聖靈去幫助我們.
希望我們可以明白聖經的說話.
也讓我們可以開眼界.
看得見祂要我們去看的東西.
我們一起禱告吧.
因為你是坐在寶座上的.
你是掌管天下的.
你的恩典是與我們同在的.
今天我們去看你的話語的時候.
求主你去幫助我們.
特別求聖靈去成為我們每一個人的老師.
去教導我們明白我們經常說看不明白的啟示錄.
也親自去感動我們.
讓我們看見主你的說話那種能力.
也讓我們看見主你是常常與我們同在.
以致我們縱使面對很多困難.
我們今天可以靠著你的力量去面對.
我們將聽你的話語的時間交給你.
願神你與我們一起幫助我們.
引導我們.
我們這樣禱告奉耶穌基督的聖名祈求.
阿們.
有一晚崇拜完結後.
我通常會去買宵夜吃的.
當天我去買魚肉碗仔翅.
如果你去過這間大圍小吃.
我真的去這間大圍小吃買的.
如果你去過這間大圍小吃.
就知道大圍小吃分三條隊.
最左邊是買魚蛋燒賣.
中間是買魚肉碗仔翅.

$^{41}$最左邊是買類似煎釀三寶的.
在排隊中.
前面有個媽媽和大約五歲的女兒.
排著排著就到了.
突然間旁邊排煎釀三寶的大叔.
突然和前面的媽媽和女兒打招呼.
原來認識的.
很開心,妹妹跟他玩.
他們兩個都到了.
去叫他們的食物,也在等.
然後妹妹就知道大叔想買煎釀三寶.
就拿了幾支竹籤.
想拿給大叔.
這個叔叔.
還沒看到他拿著竹籤的時候.
這個妹妹的媽媽就看到他拿著竹籤.
在座的媽媽.
如果你看到你的女兒拿著竹籤.
你會想說什麼呢.
這個媽媽.
不知道你聽不懂她說什麼.
這個媽媽就說,幹什麼.
拿著竹籤幹什麼.
玩得嗎.
這麼危險.
多一隻眼那怎麼辦.
妹妹就含著兩泡眼淚.
看著媽媽.
說,不玩得.
不玩得,那你還拿.
知道不可以玩,還哭.
然後呢.
這個叔叔就搞定了.
他的煎釀三寶.
轉身就看到妹妹.
哭了,無端端的.
就馬上走了.
好了.
我呢,其實呢,看到妹妹哭的時候.
很不忍心.

$^{81}$很慘.
我簡直有股衝動.
跟那個姐姐說.
媽媽,不是啊,其實妹妹.
想拿竹籤給叔叔.
不是想玩啊.
正當我.
真的很想拍拍她的時候.
我就想起我丈夫經常叫我不要那麼多事.
然後我就.
決定繼續排隊.
等我這一碗魚肉碗仔翅.
始終人家是教女兒,我這三姑六婆.
出聲幹什麼呢.
妹妹看著.
買煎釀三寶的叔叔.
終於想幫她.
拿竹籤,誰知道被媽媽罵.
媽媽看到妹妹.
拿著竹籤.
下意識覺得她貪玩很危險.
所以罵爆她.
似乎罪魁禍首的叔叔.
只顧著看著煎釀三寶.
可以了嗎.
完全不知道媽媽罵妹妹.
也不知道妹妹為什麼哭.
完全不知道發生什麼事.
而我,這個無關的.
三姑六婆.
看得很通,看得很透整件事.
但我選擇沉默.
不要那麼多事.
用不同角度看事情.
會有不同看法.
對嗎.
我不知道大家.
習慣坐地鐵的人.
會不會覺得香港.
用地鐵路線圖去理解香港.

$^{121}$曾經有人回答我.
他住綠色線.
知不知道綠色線.
是坐在哪裡,即是住在哪裡.
我再看下一頁.
他說綠色線.
原來住在觀塘.
明明綠色線很長.
你還說住在觀塘.
因為他叫觀塘線.
有人會覺得.
荃灣去大圍.
很遠.
因為他覺得要轉很多程車.
所以覺得很遠.
但其實如果你不坐地鐵.
坐巴士或是.
開車.
你會覺得荃灣去大圍很方便.
因為有城門隧道.
但最重點是.
如果我們用地鐵路線圖.
去理解香港.
其實很多地方.
是沒有地鐵站的.
我按不到.
為甚麼我每次都按不到.
我再按多一下.
要看著哪裡.
OK.
哎呀.
這裡.
很小的字.
這些地方沒有地鐵站.
但都是香港.
如果我們的腦.
用了地鐵路線圖去理解香港.
我們便會不知道.
原來這些地方都屬於香港.
這些地方都需要資源.

$^{161}$都可能要保育.
一件事的發生.
用不同眼光看.
而我們的思維.
習慣都會限制我們對一個人.
一件事.
甚至對整個世界的認識.
2023年快要過了.
我們過去這一年.
用甚麼眼光去認識自己.
認識香港.
甚至認識這個世界.
我們的眼光和思維.
有沒有限制我們去認識.
這個天賦的世界.
所以我今天講題.
叫「小朋友不懂世界」是這個意思.
謝謝Youtube的朋友.
或者我們不單單是小朋友.
不懂世界.
其實我們大人都不太懂.
剛才已經出了.
「白日之下」.
OK.
我都交給你按.
好嗎?.
「白日之下」.
大家有沒有看.
其實我沒有看.
所以你放心我不會劇透.
但就算不劇透.
我不知道,只是用戲名.
就算沒有看.
其實都大約猜到「白日之下」.
單憑名字都不會是.
開心,陽光的戲.
因為「白日之下」的.
香港大部份都是.
荒謬,難過.
令人無奈的東西.

$^{201}$如果我們用「白日之下」的.
角度去看這個世界.
我們會看得很灰的.
我們會看得很失望的.
所以我今天和大家看啟示錄.
大家轉一轉角度.
我們向上.
看我們「白日之上」的世界.
以致我們在「白日之下」的生活.
除了失望,灰心.
都可以有不同的看法.
不少人和我說.
他們看不明白啟示錄.
主要原因是因為.
經文用了很多符號,圖像.
但這些符號,圖像.
對當時的讀者來說.
其實可以一看就明白.
可以幻想.
今天我們知道小熊維尼.
不單單是小熊維尼.
粉紅色的秋我們都知道.
不是季節.
大約是這個意思.
777和721.
我們知道是數字.
但不單單是數字.
777是代表一個人.
721是一個日子.
但這些數字除了代表人物和日子.
同時我們這個年代的人.
會明白背後的代表.
或隱藏著.
一個更大的意義和意思.
今天我們看啟示錄12章.
可能剛才聽的時候.
大家對這些細節會很好奇.
可能你會想.
這個婦人為何可以.
披著太陽都不燒死.

$^{241}$龍紅色是否代表某個國家.
七頭十角究竟是甚麼樣子.
但請留意.
這些字面的意思.
不是作者想我們去斟酌.
或去強調的東西.
因為對當時的人來說.
全部都心神領會.
畫公仔不用畫出腸.
大部分的東西一聽.
大家當時已經能夠意會.
能夠明白.
啟示錄可以分為兩部分.
第一部分是第一至第十二節.
是講天上的增減.
第一節它說天上有個徵兆.
我都要下一張.
有個徵兆.
它說.
這裡有個孕婦.
很痛苦地在生育.
有一條很大的紅色的龍.
咬著這個女人.
是想等這個女人.
一生育就吃掉她.
這個小孩是誰呢.
第五節回答了我們.
它說要用鐵匠管萬國.
其實這句是引至.
詩篇第二章第九節.
講尼塞亞的王權.
所以這個小孩是在講耶穌基督.
因為耶穌基督.
才有權去管治萬國.
不過婦人.
不單單是在講耶穌.
育身的母親瑪利亞.
大部分聖經學者認為.
這個婦人是象徵神的子民.
不過當第五節.

$^{281}$這個小孩一出生的時候.
小孩就說被提到神的寶座.
這一句其實已經.
指到耶穌從死裡復活.
以至高升到神的寶座.
所以第五節可以說是.
一句講完耶穌的生平.
他的降生.
死亡,復活和升天.
所以這句同時是表示.
耶穌釘十字架.
升天,復活.
是打贏龍的原因.
也是打贏龍的基礎.
婦人在這個時間.
也到了一個荒地.
去躲避大龍.
第七節.
講到天上發生了一場戰爭.
就是米加勒,米加勒是天使長.
他和其他天使.
和龍和他的小孩.
一起去打仗.
第八節就說龍不夠打.
天上沒有他們的地方.
沒有屬於龍的地方.
龍和他的小孩.
可以說,如果用今天這樣說.
真的不是持份者.
所以第九節米加勒.
很不客氣地就一下子.
把他趕下地.
天上的戰爭.
龍被趕下地.
是表示龍是徹徹底底的失敗.
這條龍是誰呢.
他告訴我們.
這條龍是古蛇.
魔鬼,又叫撒旦.
迷惑全地.

$^{321}$用古蛇來形容.
我們大約都會聯想到創世紀.
引誘夏娃的蛇.
這套片我們有個感覺.
原來.
這個撒旦.
這條蛇,這個魔鬼.
由創世的時候.
直到現在末世.
他的攻擊,他的誘惑.
是不會停止的.
不過撒旦對耶穌基督.
是少少動搖和影響力都沒有.
耶穌仍然是.
很穩定地坐在寶座上.
所以第十節到第十二節.
如果你剛才聽的時候.
你會聽到背景音樂很大聲.
是因為他有一個宣告.
宣告就是.
耶穌基督的國已經建立.
已經響道.
是一個宣言.
告訴我們基督的國度已經響道.
第十三節到第十七節.
第十三節到第十七節是說另一場戰爭.
是在地上的戰爭.
這條龍掉在地上.
想撿回砸沙.
所以他被人一扔就掉在地上.
他殺不死耶穌.
他就唯有向他的跟隨者下手.
第十七節就說.
龍要逼迫女人.
和其餘的子女去開戰.
聖經很清楚說到.
其餘的子女是哪些.
就是那些遵守神的誡命.
有耶穌的見證的人.
龍.

$^{361}$由創世到今天.
他要做的事其實只有一個.
就是要迷惑.
引誘拉攏.
本身那些是跟隨神的子民.
拉攏他們.
改變陣營.
改變陣營去跟隨大龍.
其實啟授十二章.
很容易明白.
就是在說打仗.
說一場正邪大戰.
原因是當時的弟兄姊妹.
在信仰上.
都好像在打一場正邪大戰.
可能大家聽過.
在約翰社啟述的時候.
羅馬皇帝就近迫迫.
可能要流行拜皇帝.
基督徒不拜.
是很受迫迫的.
我們經常聽到.
不拜就慘了.
之類的.
但其實你想深一層.
明刀明槍要你拜皇帝.
要你宣誓.
其實反而沒那麼難決定.
都是做和不做.
但難在的不只是這樣.
原來當時的羅馬帝國.
沒有立例要人們拜皇帝.
相反是不同的小城市.
為了想羅馬皇帝.
見到他們.
為了想羅馬皇帝.
可以寵幸這些小城市.
所以他們要自己爭上位.
啟授就提到亞細亞有七個教會.
我們都知道.

$^{401}$寫信給七教會.
七教會其實就是七個地方的教會.
所以這七個教會.
那些城市都不例外.
他們都很想得到.
羅馬皇帝的寵幸.
要怎樣做.
就是效忠.
要讓羅馬皇帝知道.
我們這個城市對你忠心耿耿.
其中一樣東西很特別.
原來他們曾經試過.
不同城市要入紙申請.
要求你去.
建一座廟.
廟裡要有一個羅馬皇帝.
在那裡.
讓百姓去敬拜.
當時.
士美那城和薩迪.
這兩個城市.
曾經爭奪.
爭奪要建一座廟.
給提比留斯皇帝.
最終士美那這個城.
贏了.
贏了不是代表有座廟.
你明不明白.
贏了代表.
我這個羅馬皇帝寵幸你這個城市.
你這個城市真乖.
所以我現在可以給你資源.
建橋給你.
給你錢發展.
全世界都會見到你.
投放多點資源.
自然我們這個城市就有錢.
想得到資源.
有洞路賺錢做生意.
自然就要識埋堆.

$^{441}$經濟圈要搞得起經濟.
自然就要有一班.
識人好過識字的朋友圈.
其實和今天我們信徒.
面對的掙扎和處境.
幾相似.
我們不少信徒都面對一些很貼地的挑戰.
不單單是.
拜不拜皇帝.
放不放國旗.
上不上課.
而是政治社會信仰.
經濟文化教育.
等等.
每一個範疇都混在一起.
環環相扣.
以至我們這一班.
跟隨耶穌的人.
其實要解決生活困難.
和選擇.
我們都覺得搞不定,很大壓力.
還說要在信仰上.
堅持繼續有.
生命的見證.
是更加困難.
甚至可能會和神.
越走越遠.
作者就是想透過.
這一場天上的戰爭.
讓我們看見.
我們肉眼看不見.
白日之上的真相.
他想鼓勵我們.
有勇氣去面對.
在地上不同的挑戰.
和困難.
作者想我們看清一件事.
其實是.
這條龍不堪一擊.
在地上.

$^{481}$你見到他似乎是.
張牙舞爪,任意妄為.
也問也無.
但我們看見聖經很粗俗地說.
這條龍是垃圾.
是嗎?.
經文一開始形容龍很厲害.
我們會害怕,第三節.
他說龍既紅色又大條.
很有霸氣.
紅色會有人形容.
無人性,血腥,殘暴.
而且還說.
他帶著七個皇冠.
哪些人.
可以帶皇冠?.
皇帝當權者.
才可以帶皇冠.
是身份和權力的象徵.
代表他擁有.
一些權力,擁有實權.
這條龍帶出七個.
七個.
全方位他都有權勢.
全方位他都有影響力.
啟示六十二章.
告訴我們.
帶紅龍同時有很多種身份.
他是古蛇,他是魔鬼.
他是撒旦,也是迷惑人.
控告人.
目的是要我們遠離神.
漸遠,甚至轉陣營.
今天我們有什麼.
令我們和神的關係.
漸遠漸遠.
甚至明知道.
得罪神我們也會照做.
他可以化身.
做很多不同的形象.

$^{521}$可以用權力.
金錢,慾望,明星.
等等等等.
去迷惑我們.
如果我們只是看著地上的世界.
其實這條紅龍.
是很吸引的.
因為他是錢.
他是名,他是成就.
他是名氣.
他是權力.
這些對於我們來說很有安全感.
很實在.
很有渣拿.
但如果我們向上.
看天上的世界.
我們看清真相.
是紅龍只有樣貌.
暫時.
垃圾.
他一下子.
被神扔到地上.
沒有定期.
今天我們還要不要.
抱著垃圾.
當作寶.
當我們今天在地上.
看見很多蠢人.
壞人,說是.
自恃自己有權有勢,隻手遮天的時候.
我們可能.
同時也會覺得灰心絕望.
但天上的.
真相讓我們知道.
他們只是垃圾.
真正的勝利.
是屬於為我們釘十字架.
流出補血.
頭戴冠冕.
坐在寶座上的耶穌基督.

$^{561}$其實在地上.
冠冕一點也不吸引.
誰會戴冠冕.
你知不知道.
是香港小姐.
香港小姐會戴冠冕.
但你會,而且現在香港小姐的冠冕.
似乎是不值錢.
以前聽說有位港姐.
賣了一個官,有錢的,但現在似乎也不值錢.
香港小姐戴冠冕.
我們知道她甚麼也沒有.
那個官也不值錢.
做了港姐,不代表有戲拍.
也不代表她.
靚.
其實我不知道港姐現在還有沒有港姐.
Sorry,我不知道.
這樣嗎?還是我們這裡會有一個港姐?.
我們Full Church甚麼人都有.
Sorry,Sorry,我看的那些.
應該不是你那一屆.
不值錢的後官.
官冕是沒有權的.
虛榮.
地上我們是這樣看的.
但天上的世界.
讓我們認清.
官冕比皇冠.
是更大能力.
更值得我們依靠.
因為頭戴官冕的耶穌基督.
才是真正擁有.
天上,地上.
和地下的權柄.
今天我們的生命.
想戴皇冠.
還是戴官冕?.
我們想跟紅龍的陣營.
抱著垃圾皇冠當作寶.

$^{601}$還是我們跟隨.
耶穌.
跟隨港姐冠軍.
戴回官冕.
戴回真正.
得勝的官冕.
天上的徵戰.
也讓我們看清第二個真相.
原來當我們落在苦難的時候.
神不是坐在寶座上.
按按腳.
不理我們.
神不會放棄我們.
神不會不理我們.
祂看見.
而且祂一直眷顧我們.
第六節.
他說婦人走到曠野.
應該有個powerpoint.
婦人走到曠野.
在那裡有神為她預備的地方.
好讓她在那裡受供養.
1260天.
第十四節也是這樣說.
其實說同一件事.
去自己的地方供養一年兩年半年.
這裡其實一年兩年半年.
和1260天是一樣的.
都是說三年半.
三年半這個日期.
其實在第十一章.
和第十三章.
後面那章也有說到這個時間.
第十一章是說到.
教會受外邦人踐踏.
第十三章是說到.
當國勢力任意妄為的時候.
時間也是三年半.
三年半是什麼意思呢.
不是真的三年半.

$^{641}$因為我們知道啟祖不是這樣說.
七是代表完美完全的數字.
三年半.
是七的一半.
意思是讓我們知道.
受苦是有時限的.
受苦是會終止的.
受苦是不會到永遠的.
而最重要的是.
當我們.
經歷困難挑戰的時候.
神是沒有停止他的工作的.
苦人去到神.
為他預備的曠野.
有神的眷顧和保守.
十四節再次是這樣說.
他得到神的供養.
第十五節更加說蛇又出現了.
他用另一個形象.
想噴水出來淹死女人.
但是地開口吞了.
這個景象.
讓我們想起紅海.
紅海分為旱地.
神的帶領和保守.
無論如何.
這個天上的景象.
再次強調給我們這群人看見.
神是不會放棄每一個.
跟隨他的人.
神會帶領人去到他所預備的曠野.
得到他的供應.
甚至會叫地開口.
讓人避過危險.
用一些我們想不到的方式.
去為我們開路.
在困難裡面.
神的保守和眷顧.
是會停止的.
鄧小姐我不知道今天.

$^{681}$你的生命正在遇上什麼困難.
正在遇上什麼挑戰或迷茫.
可能是身體軟弱.
可能是.
心靈很沉重.
很孤單.
可能是政治環境令你感到很壓迫.
工作前景.
不明朗.
家庭婚姻可能出問題.
擔心子女的教育.
照顧父母的壓力.
人際關係.
帶給你傷害.
過去的事.
讓你很內疚自責.
不安等等.
等等.
我們眼看見白日之下.
全部都是問題.
很多的困難.
很辛苦.
很大壓力.
但聖經提醒我們.
我們向上望.
向上望這個.
白日之上天上的世界.
耶穌釘十字架的愛.
和祂同在.
是成為我們的動力.
是成為我們的勇氣.
今天我的講題是.
小朋友不懂世界.
因為我小時候.
大約是月體這張相.
這個年紀.
下一張.
有一晚.
我經過紅磡.
看到街上很多車.

$^{721}$車好像是真的.
不要那麼快開.
是我不對.
我不應該這樣說.
開出來讓他們看.
很多車我覺得很有趣.
明明是假的.
但我覺得很真.
我那時問我媽.
為何街上那麼多車.
我媽說.
因為街上很多車.
很多車.
為何街上那麼多車.
很可愛 我想坐.
我覺得我自己能坐.
因為我小時候.
我想坐 媽媽我可不可以去坐.
然後我媽說.
呸.
別亂說話 這些車不是讓你坐的.
然後我還問誰坐.
沒有理我.
後來長大了.
才知道原來這些車叫紙紮車.
你們有沒有見過.
很有趣 你們應該見過.
當年我說要坐這些車.
有些可能.
迷信的長輩.
會說一句有怪莫怪.
小朋友不懂世界.
這句話.
代表什麼.
代表他們認為.
靈界 鬼神的世界.
是真實存在.
而且比我們更有能力.
我這些小朋友不懂事.
才說要坐車.

$^{761}$所以要有怪莫怪.
這個意思是.
如果這班人.
其實真的有.
鬼怪世界.
是有的.
如果這班人對鬼神的世界.
的存在和能力.
是那麼深信不疑.
我們一班跟隨耶穌的人.
是不是理應.
應該比他們.
更加相信.
白日之上 天上的世界.
是不是應該比他們更加認定.
耶穌基督.
坐在寶座上 安定在天.
坐著為王.
是可以掌管我們.
是擁有絕對的權柄.
擁有能力.
如果你看見天上的世界.
又相信天上的世界.
相信祂的主權.
相信神的同在.
相信神的能力.
今天我們就可以更加有.
力量和勇氣.
白日之下.
全部都是困難.
但如果我們看見又相信.
天上的世界.
是真實存在的.
比我們更有能力.
我們今天就有勇氣去面對.
前幾天是感恩節.
不知道大家有沒有留意我們出了一個帖文.
過往一年在流淌.
不同方面付出過的.
不同單位.

$^{801}$感恩是提醒我們.
要多點向上望的操練.
感恩是提醒我們.
要多點向上望的操練.
當我們只是看著.
自己的困難.
我們會看不見神的眷顧.
但每當我們.
向上望.
我們就會發覺困難是很多.
但恩典.
是夠用的.
有時可能只是一句.
前幾天我開組的時候.
有組員分享.
他近來很累.
工作很忙.
放八天.
不計OT.
加上他在診所工作.
每天要照顧很多不同的病人.
他付出的不單是.
時間.
更是他的心力.
他的腦和他的勞力.
誰知這段這麼忙碌的時間.
突然不舒服.
他下班照顧完自己的病人.
他上班照顧完自己的病人.
他下班.
更加要下手下腳.
照顧自己的媽媽.
還要買菜.
照顧她的起居飲食.
但他在這段時間裡.
他覺得媽媽很不體諒他.
上班已經很疲倦.
還覺得他很老.
要照顧他.
有一天他如常上班.

$^{841}$有個伯伯來看.
他幫伯伯.
減輕了很多痛楚.
然後無理到又要看下一個症.
然後剛好.
放飯的時候.
聽到伯伯說要找他.
根據他過往經驗.
人們找他主要是問他.
要怎麼做.
他就走出去問.
原來這個伯伯說.
我很想跟你說一句謝謝.
你真的幫我減輕了很多痛楚.
他很忙.
不用了不用了.
回到家.
他當晚回到崇拜.
他說這一幕.
不停在他腦裡出現.
然後我的組員說.
他明白到.
這句謝謝雖然是出自病人的口中.
但他看見.
這句謝謝.
其實是天父跟他說的.
因為天父.
知道他需要這一句謝謝.
其實他需要媽媽.
跟他說一句謝謝.
天父知道他照顧媽媽.
的辛苦.
所以天父跟他說謝謝.
去肯定.
他照顧媽媽的付出.
一句謝謝.
沒有減輕到他.
面對的困難.
但他看見天父的看顧.
有時可能只是一份禮物.

$^{881}$我呢.
一向對收禮物.
沒有特別感覺.
大家如果想送禮物給我.
是可以的.
但你可能會覺得.
我很冷漠.
因為我從來.
從小到大.
沒有送和收禮物.
不代表你可以不送給我.
如果每個目者都有的話.
請你都送一份給我.
如果你只是送給一個.
就不用送給我了.
其實都可以不用送.
真的完全不用.
因為我本身的愛語不是收禮物.
所以其實.
別人送禮物給我.
我會覺得謝謝你.
但沒有禮物收.
我其實覺得很理所當然.
沒有問題.
所以我一向不太著重禮物.
有就有,沒有就沒有.
前陣子有個組員.
他送了一份禮物給我.
其實我一開始.
很冷漠,謝謝你.
然後就沒有了.
真的這樣.
有一天突然做了一件事.
我覺得很煩.
我要處理的事很煩.
又不知道怎樣做,怎樣解決.
覺得很頭痛,不想搞.
又想找潘Sir.
突然間我看到一份禮物.
我突然間看到一份禮物.

$^{921}$然後我就想.
為何他要把那份禮物送給我.
其實那份禮物.
他不送給我是更好的.
如果他送給他的家人.
送給他的朋友.
那份禮物對他來說.
和那份禮物來說.
總之更好,我不說那份禮物是甚麼.
但他竟然選擇送給我.
在那一刻.
突然間我看到.
其實是天父送給我的禮物.
我從來對禮物.
是沒有感動的.
但那一刻我覺得.
我很感動.
因為天父送的那份禮物.
讓我知道原來我眼前.
不只是那些很頭痛,不知怎樣搞.
很煩,不如只找潘Sir算了.
而是我身邊還有很多.
很好的人.
支持我和鼓勵我.
我從來都不會.
因為收到一份禮物.
感動到.
或是很真心.
感動到想哭.
但我想那一刻.
我就真的覺得.
天父是送了這份禮物.
透過我的組員.
去鼓勵我.
2023年.
就快過了.
你試一下向上望.
我們現在.
我們想一想.
天父今年讓你遇到什麼人.

$^{961}$讓你發生什麼事.
可能你想起的.
都是困難多的.
但有時我們向上望.
哪怕只有一個恩典.
但這個恩典.
其實就足夠了.
這個一個恩典.
就足以讓我們繼續有勇氣.
有力量.
去繼續面對我們的生活.
2024年就快到了.
弟兄姊妹.
我們不如.
由現在開始操練.
我們多點向上望.
多點看見.
天父給我們的恩典.
以致我們在生活上.
即使遇見困難.
仍然認定.
耶穌基督是那位最有能力.
亦是常常與我們同在的主.
我們一起禱告.
因為你真是.
坐著為王.
擁有最大權柄的.
那位統治者.
今天我們看見地上.
很多困難.
很多奇怪的事.
但我們看見.
當我們向上望的時候.
我們同樣有恩典.
讓我們看見你是為我們.
釘十字架.
流出補血.
求你讓我們認定.
天上的世界是真實存在.
而且比我們更有能力.

$^{1001}$以致我們今天在地上.
即使面對很多困難.
很多挑戰.
我們仍然有勇氣.
去過我們的生活.
有勇氣去面對這些挑戰.
我們這樣的祈禱.
是奉耶穌基督的名義祈求.
阿們.
\newpage



\section{}
\label{sec:w1NzLUX2_GE}
\textbf{《致餘民及流散者:給香港基督徒的神學八課》第二季第8課|20231126 [w1NzLUX2\_GE]}
\newline
\newline
連結: \href{https://youtube.com/watch?v=w1NzLUX2_GE}{\texttt{ https://youtube.com/watch?v=w1NzLUX2\_GE}} ~~~~ 語音日期: 2023-11-26 
\newline
\newline
\hyperref[sec:2LIl7VilU18]{\small{< < < PREV SERMON < < <}}
~
\hyperref[sec:index_chronic]{\small{[返順時目]}}
~
\hyperref[sec:index_scriptual]{\small{[返順卷目]}}
~
\hyperref[sec:lfg8MyM5M04]{\small{> > > NEXT SERMON > > >}}
\newline
\newline
$^{1}$我只想知道.
你到底是什麼意思.
我只想知道.
你到底是什麼意思.
我只想知道.
你到底是什麼意思.
我只想知道.
你到底是什麼意思.
我只想知道.
你到底是什麼意思.
我只想知道.
你到底是什麼意思.
我只想知道.
你到底是什麼意思.
我只想知道.
你到底是什麼意思.
我只想知道.
你到底是什麼意思.
我只想知道.
你到底是什麼意思.
我只想知道.
你到底是什麼意思.
我只想知道.
你到底是什麼意思.
我只想知道.
你到底是什麼意思.
我只想知道.
你到底是什麼意思.
我只想知道.
你到底是什麼意思.
我只想知道.
你到底是什麼意思.
我只想知道.
你到底是什麼意思.
我只想知道.
你到底是什麼意思.
我只想知道.
你到底是什麼意思.
我只想知道.
你到底是什麼意思.

$^{41}$我只想知道.
你到底是什麼意思.
我只想知道.
你到底是什麼意思.
我只想知道.
你到底是什麼意思.
我只想知道.
你到底是什麼意思.
我只想知道.
你到底是什麼意思.
我只想知道.
你到底是什麼意思.
我只想知道.
你到底是什麼意思.
我只想知道.
你到底是什麼意思.
我只想知道.
你到底是什麼意思.
我只想知道.
你到底是什麼意思.
我只想知道.
你到底是什麼意思.
我只想知道.
你到底是什麼意思.
我只想知道.
你到底是什麼意思.
我只想知道.
你到底是什麼意思.
我只想知道.
你到底是什麼意思.
我只想知道.
你到底是什麼意思.
我只想知道.
你到底是什麼意思.
我只想知道.
你到底是什麼意思.
我只想知道.
你到底是什麼意思.
我只想知道.
你到底是什麼意思.

$^{81}$我只想知道.
你到底是什麼意思.
我只想知道.
你到底是什麼意思.
我只想知道.
你到底是什麼意思.
我只想知道.
你到底是什麼意思.
我只想知道.
你到底是什麼意思.
我只想知道.
你到底是什麼意思.
我只想知道.
你到底是什麼意思.
我只想知道.
你到底是什麼意思.
我只想知道.
你到底是什麼意思.
我只想知道.
你到底是什麼意思.
我只想知道.
你到底是什麼意思.
我只想知道.
你到底是什麼意思.
我只想知道.
你到底是什麼意思.
我只想知道.
你到底是什麼意思.
我只想知道.
你到底是什麼意思.
我只想知道.
你到底是什麼意思.
我只想知道.
你到底是什麼意思.
我只想知道.
你到底是什麼意思.
我只想知道.
你到底是什麼意思.
我只想知道.
你到底是什麼意思.

$^{121}$我只想知道.
你到底是什麼意思.
我只想知道.
你到底是什麼意思.
我只想知道.
你到底是什麼意思.
我只想知道.
你到底是什麼意思.
我只想知道.
你到底是什麼意思.
我只想知道.
你到底是什麼意思.
我只想知道.
你到底是什麼意思.
我只想知道.
你到底是什麼意思.
我只想知道.
你到底是什麼意思.
我只想知道.
你到底是什麼意思.
我只想知道.
你到底是什麼意思.
我只想知道.
你到底是什麼意思.
我只想知道.
你到底是什麼意思.
我只想知道.
你到底是什麼意思.
我只想知道.
你到底是什麼意思.
我只想知道.
你到底是什麼意思.
我只想知道.
你到底是什麼意思.
我只想知道.
你到底是什麼意思.
我只想知道.
你到底是什麼意思.
我只想知道.
你到底是什麼意思.

$^{161}$我只想知道.
你到底是什麼意思.
我只想知道.
你到底是什麼意思.
我只想知道.
你到底是什麼意思.
我只想知道.
你到底是什麼意思.
我只想知道.
你到底是什麼意思.
我只想知道.
你到底是什麼意思.
我只想知道.
你到底是什麼意思.
我只想知道.
你到底是什麼意思.
我只想知道.
你到底是什麼意思.
我只想知道.
你到底是什麼意思.
我只想知道.
你到底是什麼意思.
我只想知道.
你到底是什麼意思.
我只想知道.
你到底是什麼意思.
我只想知道.
你到底是什麼意思.
我只想知道.
你到底是什麼意思.
我只想知道.
你到底是什麼意思.
我只想知道.
你到底是什麼意思.
我只想知道.
你到底是什麼意思.
我只想知道.
你到底是什麼意思.
我只想知道.
你到底是什麼意思.

$^{201}$我只想知道.
你到底是什麼意思.
我只想知道.
你到底是什麼意思.
我只想知道.
你到底是什麼意思.
我只想知道.
你到底是什麼意思.
我只想知道.
你到底是什麼意思.
我只想知道.
你到底是什麼意思.
我只想知道.
你到底是什麼意思.
我只想知道.
你到底是什麼意思.
我只想知道.
你到底是什麼意思.
我只想知道.
你到底是什麼意思.
我只想知道.
你到底是什麼意思.
我只想知道.
你到底是什麼意思.
我只想知道.
你到底是什麼意思.
我只想知道.
你到底是什麼意思.
我只想知道.
你到底是什麼意思.
我只想知道.
你到底是什麼意思.
我只想知道.
你到底是什麼意思.
我只想知道.
你到底是什麼意思.
我只想知道.
你到底是什麼意思.
我只想知道.
你到底是什麼意思.

$^{241}$我只想知道.
你到底是什麼意思.
我只想知道.
你到底是什麼意思.
我只想知道.
你到底是什麼意思.
我只想知道.
你到底是什麼意思.
我只想知道.
你到底是什麼意思.
我只想知道.
你到底是什麼意思.
我只想知道.
你到底是什麼意思.
我只想知道.
你到底是什麼意思.
我只想知道.
你到底是什麼意思.
我只想知道.
你到底是什麼意思.
我只想知道.
你到底是什麼意思.
我只想知道.
你到底是什麼意思.
我只想知道.
你到底是什麼意思.
我只想知道.
你到底是什麼意思.
我只想知道.
你到底是什麼意思.
我只想知道.
你到底是什麼意思.
我只想知道.
你到底是什麼意思.
我只想知道.
你到底是什麼意思.
我只想知道.
你到底是什麼意思.
我只想知道.
你到底是什麼意思.

$^{281}$我只想知道.
你到底是什麼意思.
我只想知道.
你到底是什麼意思.
我只想知道.
你到底是什麼意思.
我只想知道.
你到底是什麼意思.
我只想知道.
你到底是什麼意思.
我只想知道.
你到底是什麼意思.
我只想知道.
你到底是什麼意思.
我只想知道.
你到底是什麼意思.
我只想知道.
你到底是什麼意思.
我只想知道.
你到底是什麼意思.
我只想知道.
你到底是什麼意思.
我只想知道.
你到底是什麼意思.
我只想知道.
你到底是什麼意思.
我只想知道.
你到底是什麼意思.
我只想知道.
你到底是什麼意思.
我只想知道.
你到底是什麼意思.
我只想知道.
你到底是什麼意思.
我只想知道.
你到底是什麼意思.
我只想知道.
你到底是什麼意思.
我只想知道.
你到底是什麼意思.

$^{321}$我只想知道.
你到底是什麼意思.
我只想知道.
你到底是什麼意思.
我只想知道.
你到底是什麼意思.
我只想知道.
你到底是什麼意思.
我只想知道.
你到底是什麼意思.
我只想知道.
你到底是什麼意思.
我只想知道.
你到底是什麼意思.
我只想知道.
你到底是什麼意思.
我只想知道.
你到底是什麼意思.
我只想知道.
你到底是什麼意思.
我只想知道.
你到底是什麼意思.
我只想知道.
你到底是什麼意思.
我只想知道.
你到底是什麼意思.
我只想知道.
你到底是什麼意思.
我只想知道.
你到底是什麼意思.
我只想知道.
你到底是什麼意思.
我只想知道.
你到底是什麼意思.
我只想知道.
你到底是什麼意思.
我只想知道.
你到底是什麼意思.
我只想知道.
你到底是什麼意思.

$^{361}$我只想知道.
你到底是什麼意思.
我只想知道.
你到底是什麼意思.
我只想知道.
你到底是什麼意思.
我只想知道.
你到底是什麼意思.
我只想知道.
你到底是什麼意思.
我只想知道.
你到底是什麼意思.
我只想知道.
你到底是什麼意思.
我只想知道.
你到底是什麼意思.
我只想知道.
你到底是什麼意思.
我只想知道.
你到底是什麼意思.
我只想知道.
你到底是什麼意思.
我只想知道.
你到底是什麼意思.
我只想知道.
你到底是什麼意思.
我只想知道.
你到底是什麼意思.
我只想知道.
你到底是什麼意思.
我只想知道.
你到底是什麼意思.
我只想知道.
你到底是什麼意思.
我只想知道.
你到底是什麼意思.
我只想知道.
你到底是什麼意思.
我只想知道.
你到底是什麼意思.

$^{401}$我只想知道.
你到底是什麼意思.
我只想知道.
你到底是什麼意思.
我只想知道.
你到底是什麼意思.
我只想知道.
你到底是什麼意思.
我只想知道.
你到底是什麼意思.
我只想知道.
你到底是什麼意思.
我只想知道.
你到底是什麼意思.
我只想知道.
你到底是什麼意思.
我只想知道.
你到底是什麼意思.
我只想知道.
你到底是什麼意思.
我只想知道.
你到底是什麼意思.
我只想知道.
你到底是什麼意思.
我只想知道.
你到底是什麼意思.
我只想知道.
你到底是什麼意思.
我只想知道.
你到底是什麼意思.
我只想知道.
你到底是什麼意思.
我只想知道.
你到底是什麼意思.
我只想知道.
你到底是什麼意思.
我只想知道.
你到底是什麼意思.
我只想知道.
你到底是什麼意思.

$^{441}$我只想知道.
你到底是什麼意思.
我只想知道.
你到底是什麼意思.
我只想知道.
你到底是什麼意思.
我只想知道.
你到底是什麼意思.
我只想知道.
你到底是什麼意思.
我只想知道.
你到底是什麼意思.
我只想知道.
你到底是什麼意思.
我只想知道.
你到底是什麼意思.
我只想知道.
你到底是什麼意思.
我只想知道.
你到底是什麼意思.
我只想知道.
你到底是什麼意思.
我只想知道.
你到底是什麼意思.
我只想知道.
你到底是什麼意思.
我只想知道.
你到底是什麼意思.
我只想知道.
你到底是什麼意思.
我只想知道.
你到底是什麼意思.
我只想知道.
你到底是什麼意思.
我只想知道.
你到底是什麼意思.
我只想知道.
你到底是什麼意思.
我只想知道.
你到底是什麼意思.

$^{481}$我只想知道.
你到底是什麼意思.
我只想知道.
你到底是什麼意思.
我只想知道.
你到底是什麼意思.
我只想知道.
你到底是什麼意思.
我只想知道.
你到底是什麼意思.
我只想知道.
你到底是什麼意思.
我只想知道.
你到底是什麼意思.
我只想知道.
你到底是什麼意思.
我只想知道.
你到底是什麼意思.
我只想知道.
你到底是什麼意思.
我只想知道.
你到底是什麼意思.
我只想知道.
你到底是什麼意思.
我只想知道.
你到底是什麼意思.
我只想知道.
你到底是什麼意思.
我只想知道.
你到底是什麼意思.
我只想知道.
你到底是什麼意思.
我只想知道.
你到底是什麼意思.
我只想知道.
你到底是什麼意思.
我只想知道.
你到底是什麼意思.
我只想知道.
你到底是什麼意思.

$^{521}$我只想知道.
你到底是什麼意思.
我只想知道.
你到底是什麼意思.
我只想知道.
你到底是什麼意思.
我只想知道.
你到底是什麼意思.
我只想知道.
你到底是什麼意思.
我只想知道.
你到底是什麼意思.
我只想知道.
你到底是什麼意思.
我只想知道.
你到底是什麼意思.
我只想知道.
你到底是什麼意思.
我只想知道.
你到底是什麼意思.
我只想知道.
你到底是什麼意思.
我只想知道.
你到底是什麼意思.
我只想知道.
你到底是什麼意思.
我只想知道.
你到底是什麼意思.
我只想知道.
你到底是什麼意思.
我只想知道.
你到底是什麼意思.
我只想知道.
你到底是什麼意思.
我只想知道.
你到底是什麼意思.
我只想知道.
你到底是什麼意思.
我只想知道.
你到底是什麼意思.

$^{561}$我只想知道.
你到底是什麼意思.
我只想知道.
你到底是什麼意思.
我只想知道.
你到底是什麼意思.
我只想知道.
你到底是什麼意思.
我只想知道.
你到底是什麼意思.
我只想知道.
你到底是什麼意思.
我只想知道.
你到底是什麼意思.
我只想知道.
你到底是什麼意思.
我只想知道.
你到底是什麼意思.
我只想知道.
你到底是什麼意思.
我只想知道.
你到底是什麼意思.
我只想知道.
你到底是什麼意思.
我只想知道.
你到底是什麼意思.
我只想知道.
你到底是什麼意思.
我只想知道.
你到底是什麼意思.
我只想知道.
你到底是什麼意思.
我只想知道.
你到底是什麼意思.
我只想知道.
你到底是什麼意思.
我只想知道.
你到底是什麼意思.
我只想知道.
你到底是什麼意思.

$^{601}$我只想知道.
你到底是什麼意思.
我只想知道.
你到底是什麼意思.
我只想知道.
你到底是什麼意思.
我只想知道.
你到底是什麼意思.
我只想知道.
你到底是什麼意思.
我只想知道.
你到底是什麼意思.
我只想知道.
你到底是什麼意思.
我只想知道.
你到底是什麼意思.
我只想知道.
你到底是什麼意思.
我只想知道.
你到底是什麼意思.
我只想知道.
你到底是什麼意思.
我只想知道.
你到底是什麼意思.
我只想知道.
你到底是什麼意思.
我只想知道.
你到底是什麼意思.
我只想知道.
你到底是什麼意思.
我只想知道.
你到底是什麼意思.
我只想知道.
你到底是什麼意思.
我只想知道.
你到底是什麼意思.
我只想知道.
你到底是什麼意思.
我只想知道.
你到底是什麼意思.

$^{641}$我只想知道.
你到底是什麼意思.
我只想知道.
你到底是什麼意思.
我只想知道.
你到底是什麼意思.
我只想知道.
你到底是什麼意思.
我只想知道.
你到底是什麼意思.
我只想知道.
你到底是什麼意思.
我只想知道.
你到底是什麼意思.
我只想知道.
你到底是什麼意思.
我只想知道.
你到底是什麼意思.
我只想知道.
你到底是什麼意思.
我只想知道.
你到底是什麼意思.
我只想知道.
你到底是什麼意思.
我只想知道.
你到底是什麼意思.
我只想知道.
你到底是什麼意思.
我只想知道.
你到底是什麼意思.
我只想知道.
你到底是什麼意思.
我只想知道.
你到底是什麼意思.
我只想知道.
你到底是什麼意思.
我只想知道.
你到底是什麼意思.
我只想知道.
你到底是什麼意思.

$^{681}$我只想知道.
你到底是什麼意思.
我只想知道.
你到底是什麼意思.
我只想知道.
你到底是什麼意思.
我只想知道.
你到底是什麼意思.
我只想知道.
你到底是什麼意思.
我只想知道.
你到底是什麼意思.
我只想知道.
你到底是什麼意思.
我只想知道.
你到底是什麼意思.
我只想知道.
你到底是什麼意思.
我只想知道.
你到底是什麼意思.
我只想知道.
你到底是什麼意思.
我只想知道.
你到底是什麼意思.
我只想知道.
你到底是什麼意思.
我只想知道.
你到底是什麼意思.
我只想知道.
你到底是什麼意思.
我只想知道.
你到底是什麼意思.
我只想知道.
你到底是什麼意思.
我只想知道.
你到底是什麼意思.
我只想知道.
你到底是什麼意思.
我只想知道.
你到底是什麼意思.

$^{721}$我只想知道.
你到底是什麼意思.
我只想知道.
你到底是什麼意思.
我只想知道.
你到底是什麼意思.
我只想知道.
你到底是什麼意思.
我只想知道.
你到底是什麼意思.
我只想知道.
你到底是什麼意思.
我只想知道.
你到底是什麼意思.
我只想知道.
你到底是什麼意思.
我只想知道.
你到底是什麼意思.
我只想知道.
你到底是什麼意思.
我只想知道.
你到底是什麼意思.
我只想知道.
你到底是什麼意思.
我只想知道.
你到底是什麼意思.
我只想知道.
你到底是什麼意思.
我只想知道.
你到底是什麼意思.
我只想知道.
你到底是什麼意思.
我只想知道.
你到底是什麼意思.
我只想知道.
你到底是什麼意思.
我只想知道.
你到底是什麼意思.
我只想知道.
你到底是什麼意思.

$^{761}$我只想知道.
你到底是什麼意思.
我只想知道.
你到底是什麼意思.
我只想知道.
你到底是什麼意思.
我只想知道.
你到底是什麼意思.
我只想知道.
你到底是什麼意思.
我只想知道.
你到底是什麼意思.
我只想知道.
你到底是什麼意思.
我只想知道.
你到底是什麼意思.
我只想知道.
你到底是什麼意思.
我只想知道.
你到底是什麼意思.
我只想知道.
你到底是什麼意思.
我只想知道.
你到底是什麼意思.
我只想知道.
你到底是什麼意思.
我只想知道.
你到底是什麼意思.
我只想知道.
你到底是什麼意思.
我只想知道.
你到底是什麼意思.
我只想知道.
你到底是什麼意思.
我只想知道.
你到底是什麼意思.
我只想知道.
你到底是什麼意思.
我只想知道.
你到底是什麼意思.

$^{801}$我只想知道.
你到底是什麼意思.
我只想知道.
你到底是什麼意思.
我只想知道.
你到底是什麼意思.
我只想知道.
你到底是什麼意思.
我只想知道.
你到底是什麼意思.
我只想知道.
你到底是什麼意思.
我只想知道.
你到底是什麼意思.
我只想知道.
你到底是什麼意思.
我只想知道.
你到底是什麼意思.
我只想知道.
你到底是什麼意思.
我只想知道.
你到底是什麼意思.
我只想知道.
你到底是什麼意思.
我只想知道.
你到底是什麼意思.
我只想知道.
你到底是什麼意思.
我只想知道.
你到底是什麼意思.
我只想知道.
你到底是什麼意思.
我只想知道.
你到底是什麼意思.
我只想知道.
你到底是什麼意思.
我只想知道.
你到底是什麼意思.
我只想知道.
你到底是什麼意思.

$^{841}$我只想知道.
你到底是什麼意思.
我只想知道.
你到底是什麼意思.
我只想知道.
你到底是什麼意思.
我只想知道.
你到底是什麼意思.
我只想知道.
你到底是什麼意思.
我只想知道.
你到底是什麼意思.
我只想知道.
你到底是什麼意思.
我只想知道.
你到底是什麼意思.
我只想知道.
你到底是什麼意思.
我只想知道.
你到底是什麼意思.
我只想知道.
你到底是什麼意思.
我只想知道.
你到底是什麼意思.
我只想知道.
你到底是什麼意思.
我只想知道.
你到底是什麼意思.
我只想知道.
你到底是什麼意思.
我只想知道.
你到底是什麼意思.
我只想知道.
你到底是什麼意思.
我只想知道.
你到底是什麼意思.
我只想知道.
你到底是什麼意思.
我只想知道.
你到底是什麼意思.

$^{881}$我只想知道.
你到底是什麼意思.
我只想知道.
你到底是什麼意思.
我只想知道.
你到底是什麼意思.
我只想知道.
你到底是什麼意思.
我只想知道.
你到底是什麼意思.
我只想知道.
你到底是什麼意思.
我只想知道.
你到底是什麼意思.
我只想知道.
你到底是什麼意思.
我只想知道.
你到底是什麼意思.
我只想知道.
你到底是什麼意思.
我只想知道.
你到底是什麼意思.
我只想知道.
你到底是什麼意思.
我只想知道.
你到底是什麼意思.
我只想知道.
你到底是什麼意思.
我只想知道.
你到底是什麼意思.
我只想知道.
你到底是什麼意思.
我只想知道.
你到底是什麼意思.
我只想知道.
你到底是什麼意思.
我只想知道.
你到底是什麼意思.
我只想知道.
你到底是什麼意思.

$^{921}$我只想知道.
你到底是什麼意思.
我只想知道.
你到底是什麼意思.
我只想知道.
你到底是什麼意思.
我只想知道.
你到底是什麼意思.
我只想知道.
你到底是什麼意思.
我只想知道.
你到底是什麼意思.
我只想知道.
你到底是什麼意思.
我只想知道.
你到底是什麼意思.
我只想知道.
你到底是什麼意思.
我只想知道.
你到底是什麼意思.
我只想知道.
你到底是什麼意思.
我只想知道.
你到底是什麼意思.
我只想知道.
你到底是什麼意思.
我只想知道.
你到底是什麼意思.
我只想知道.
你到底是什麼意思.
我只想知道.
你到底是什麼意思.
我只想知道.
你到底是什麼意思.
我只想知道.
你到底是什麼意思.
我只想知道.
你到底是什麼意思.
我只想知道.
你到底是什麼意思.

$^{961}$我只想知道.
你到底是什麼意思.
我只想知道.
你到底是什麼意思.
我只想知道.
你到底是什麼意思.
我只想知道.
你到底是什麼意思.
我只想知道.
你到底是什麼意思.
我只想知道.
你到底是什麼意思.
我只想知道.
你到底是什麼意思.
我只想知道.
你到底是什麼意思.
我只想知道.
你到底是什麼意思.
我只想知道.
你到底是什麼意思.
我只想知道.
你到底是什麼意思.
我只想知道.
你到底是什麼意思.
我只想知道.
你到底是什麼意思.
我只想知道.
你到底是什麼意思.
我只想知道.
你到底是什麼意思.
我只想知道.
你到底是什麼意思.
我只想知道.
你到底是什麼意思.
我只想知道.
你到底是什麼意思.
我只想知道.
你到底是什麼意思.
我只想知道.
你到底是什麼意思.

$^{1001}$我只想知道.
你到底是什麼意思.
我只想知道.
你到底是什麼意思.
我只想知道.
你到底是什麼意思.
我只想知道.
你到底是什麼意思.
我只想知道.
你到底是什麼意思.
我只想知道.
你到底是什麼意思.
我只想知道.
你到底是什麼意思.
我只想知道.
你到底是什麼意思.
我只想知道.
你到底是什麼意思.
我只想知道.
你到底是什麼意思.
我只想知道.
你到底是什麼意思.
我只想知道.
你到底是什麼意思.
我只想知道.
你到底是什麼意思.
我只想知道.
你到底是什麼意思.
我只想知道.
你到底是什麼意思.
我只想知道.
你到底是什麼意思.
我只想知道.
你到底是什麼意思.
我只想知道.
你到底是什麼意思.
我只想知道.
你到底是什麼意思.
我只想知道.
你到底是什麼意思.

$^{1041}$我只想知道.
你到底是什麼意思.
我只想知道.
你到底是什麼意思.
我只想知道.
你到底是什麼意思.
我只想知道.
你到底是什麼意思.
我只想知道.
你到底是什麼意思.
我只想知道.
你到底是什麼意思.
我只想知道.
你到底是什麼意思.
我只想知道.
你到底是什麼意思.
我只想知道.
你到底是什麼意思.
我只想知道.
你到底是什麼意思.
我只想知道.
你到底是什麼意思.
我只想知道.
你到底是什麼意思.
我只想知道.
你到底是什麼意思.
我只想知道.
你到底是什麼意思.
我只想知道.
你到底是什麼意思.
我只想知道.
你到底是什麼意思.
我只想知道.
你到底是什麼意思.
我只想知道.
你到底是什麼意思.
我只想知道.
你到底是什麼意思.
我只想知道.
你到底是什麼意思.

$^{1081}$我只想知道.
你到底是什麼意思.
我只想知道.
你到底是什麼意思.
我只想知道.
你到底是什麼意思.
我只想知道.
你到底是什麼意思.
我只想知道.
你到底是什麼意思.
我只想知道.
你到底是什麼意思.
我只想知道.
你到底是什麼意思.
我只想知道.
你到底是什麼意思.
我只想知道.
你到底是什麼意思.
我只想知道.
你到底是什麼意思.
我只想知道.
你到底是什麼意思.
我只想知道.
你到底是什麼意思.
我只想知道.
你到底是什麼意思.
我只想知道.
你到底是什麼意思.
我只想知道.
你到底是什麼意思.
我只想知道.
你到底是什麼意思.
我只想知道.
你到底是什麼意思.
我只想知道.
你到底是什麼意思.
我只想知道.
你到底是什麼意思.
我只想知道.
你到底是什麼意思.

$^{1121}$我只想知道.
你到底是什麼意思.
我只想知道.
你到底是什麼意思.
我只想知道.
你到底是什麼意思.
我只想知道.
你到底是什麼意思.
我只想知道.
你到底是什麼意思.
我只想知道.
你到底是什麼意思.
我只想知道.
你到底是什麼意思.
我只想知道.
你到底是什麼意思.
我只想知道.
你到底是什麼意思.
我只想知道.
你到底是什麼意思.
我只想知道.
你到底是什麼意思.
我只想知道.
你到底是什麼意思.
我只想知道.
你到底是什麼意思.
我只想知道.
你到底是什麼意思.
我只想知道.
你到底是什麼意思.
我只想知道.
你到底是什麼意思.
我只想知道.
你到底是什麼意思.
我只想知道.
你到底是什麼意思.
我只想知道.
你到底是什麼意思.
我只想知道.
你到底是什麼意思.

$^{1161}$我只想知道.
你到底是什麼意思.
我只想知道.
你到底是什麼意思.
我只想知道.
你到底是什麼意思.
我只想知道.
你到底是什麼意思.
我只想知道.
你到底是什麼意思.
我只想知道.
你到底是什麼意思.
我只想知道.
你到底是什麼意思.
我只想知道.
你到底是什麼意思.
我只想知道.
你到底是什麼意思.
我只想知道.
你到底是什麼意思.
我只想知道.
你到底是什麼意思.
我只想知道.
你到底是什麼意思.
我只想知道.
你到底是什麼意思.
我只想知道.
你到底是什麼意思.
我只想知道.
你到底是什麼意思.
我只想知道.
你到底是什麼意思.
我只想知道.
你到底是什麼意思.
我只想知道.
你到底是什麼意思.
我只想知道.
你到底是什麼意思.
我只想知道.
你到底是什麼意思.

$^{1201}$我只想知道.
你到底是什麼意思.
我只想知道.
你到底是什麼意思.
我只想知道.
你到底是什麼意思.
我只想知道.
你到底是什麼意思.
我只想知道.
你到底是什麼意思.
我只想知道.
你到底是什麼意思.
我只想知道.
你到底是什麼意思.
我只想知道.
你到底是什麼意思.
我只想知道.
你到底是什麼意思.
我只想知道.
你到底是什麼意思.
我只想知道.
你到底是什麼意思.
我只想知道.
你到底是什麼意思.
我只想知道.
你到底是什麼意思.
我只想知道.
你到底是什麼意思.
我只想知道.
你到底是什麼意思.
我只想知道.
你到底是什麼意思.
我只想知道.
你到底是什麼意思.
我只想知道.
你到底是什麼意思.
我只想知道.
你到底是什麼意思.
我只想知道.
你到底是什麼意思.

$^{1241}$我只想知道.
你到底是什麼意思.
我只想知道.
你到底是什麼意思.
我只想知道.
你到底是什麼意思.
我只想知道.
你到底是什麼意思.
我只想知道.
你到底是什麼意思.
我只想知道.
你到底是什麼意思.
我只想知道.
你到底是什麼意思.
我只想知道.
你到底是什麼意思.
我只想知道.
你到底是什麼意思.
我只想知道.
你到底是什麼意思.
我只想知道.
你到底是什麼意思.
我只想知道.
你到底是什麼意思.
我只想知道.
你到底是什麼意思.
我只想知道.
你到底是什麼意思.
我只想知道.
你到底是什麼意思.
我只想知道.
你到底是什麼意思.
我只想知道.
你到底是什麼意思.
我只想知道.
你到底是什麼意思.
我只想知道.
你到底是什麼意思.
我只想知道.
你到底是什麼意思.

$^{1281}$我只想知道.
你到底是什麼意思.
我只想知道.
你到底是什麼意思.
我只想知道.
你到底是什麼意思.
我只想知道.
你到底是什麼意思.
我只想知道.
你到底是什麼意思.
我只想知道.
你到底是什麼意思.
我只想知道.
你到底是什麼意思.
我只想知道.
你到底是什麼意思.
我只想知道.
你到底是什麼意思.
我只想知道.
你到底是什麼意思.
我只想知道.
你到底是什麼意思.
我只想知道.
你到底是什麼意思.
我只想知道.
你到底是什麼意思.
我只想知道.
你到底是什麼意思.
我只想知道.
你到底是什麼意思.
我只想知道.
你到底是什麼意思.
我只想知道.
你到底是什麼意思.
我只想知道.
你到底是什麼意思.
我只想知道.
你到底是什麼意思.
我只想知道.
你到底是什麼意思.

$^{1321}$我只想知道.
你到底是什麼意思.
我只想知道.
你到底是什麼意思.
我只想知道.
你到底是什麼意思.
我只想知道.
你到底是什麼意思.
我只想知道.
你到底是什麼意思.
我只想知道.
你到底是什麼意思.
我只想知道.
你到底是什麼意思.
我只想知道.
你到底是什麼意思.
我只想知道.
你到底是什麼意思.
我只想知道.
你到底是什麼意思.
我只想知道.
你到底是什麼意思.
我只想知道.
你到底是什麼意思.
我只想知道.
你到底是什麼意思.
我只想知道.
你到底是什麼意思.
我只想知道.
你到底是什麼意思.
我只想知道.
你到底是什麼意思.
我只想知道.
你到底是什麼意思.
我只想知道.
你到底是什麼意思.
我只想知道.
你到底是什麼意思.
我只想知道.
你到底是什麼意思.

$^{1361}$我只想知道.
你到底是什麼意思.
我只想知道.
所以這就是我們要大概知道的框框.
潘博華說登山寶訓的時候.
其實簡單來說就是跟隨耶穌.
呼召十字架這些關鍵詞是有關係的.
然後你看到當潘博華說跟隨基督的時候.
他就很強調.
當一個人來跟隨基督的時候.
當他願意去信服基督的命令的時候.
這個命令是一個很單純的信服.
就是說你不想太多.
就像一個小朋友一樣.
你去做一個很簡單而單純的跟隨和信服.
而這個跟隨必然是可見的.
這個行動一定會被人看到.
這句話是這樣說的.
這句話的相反是這句.
我特別改出來.
我覺得是寫得挺好的一句話.
他說逃避.
就是逃避去一個不可見的地方.
當你去隱藏或者去離開人的眼光.
被人看不到的時候.
其實這個正正是否定上帝的呼召.
所以對潘博華來說.
當耶穌去呼召我們做基督徒的時候.
本身這件事情就是可見到的.
必定可見到的.
這個也是我們回到第一課.
第一課第一課我們說什麼.
就是我們作為基督徒.
如果你有上我們的八課的話.
什麼叫基督徒.
基督徒不是一個得救名單.
也不是一個宗教徒的名稱.
而是一個有意義的基督徒.
在社會裡面見證耶穌的一群人.
所以這個跟潘博華的想法是一樣的.

$^{1401}$當我們自稱做基督徒.
或者成為基督徒之後.
你就是可見的.
當你願意去跟隨耶穌的時候.
這個行徑必定會被人看到.
這是第一個.
相反來說你逃避去隱藏.
被人看不到.
這個是違反你對於上帝耶穌基督呼召你的照明.
所以你會發覺.
他整個的神學或者是教會和世界的關係.
最重要的主線就是這一條.
就是跟隨耶穌的呼召.
和信服基督的命令.
當我們去看.
當潘博華在這條主線裡面.
開始講解登山寶藏的時候.
他很簡單地將第五章和第六章.
是兩個完全不同的章節.
其實我講道講過.
我以前講第五章的時候.
如果你們有聽我編導講.
這個基督的意義.
我也提過這個點.
之後當潘博華或者是耶穌.
耶穌在講你們身上的炎和光的時候.
這個正正就是在講這個事情.
你們是身上的炎.
你們身上的光.
所以你們的光應當這樣照在人前.
叫他們看見你們的好行為.
變成榮耀.
歸咎於你們身上的福.
當耶穌在登山寶藏裡面.
講你們是炎和光的時候.
已經是在講基督徒在世界上是可見的.
當你不可見就是什麼.
正正就是鬥底下的光.
大家知道的東西.
你明明是光.

$^{1441}$但你被人收買或自我收買的時候.
這個正正是在違反你自己的本質.
不過很有趣的.
當你細心想想炎和光的時候.
你會發覺這是一個很好的比喻.
為什麼呢?.
因為炎是什麼樣的炎?.
炎是什麼樣的炎?.
炎是不能夠假裝炎的.
炎就是炎.
炎不是假裝出來的.
也不是要做出來的.
炎本身就是炎.
所以耶穌提出了一個很特別的假設.
就是說炎虐失了味.
如果炎失去味道.
那怎麼辦呢?.
怎麼會再咸呢?.
這裡有一些毒理科.
炎怎麼能夠失去咸味?.
炎怎麼可以不咸呢?.
其實我問過一些化學博士.
炎怎麼可以不咸呢?.
炎能不能不咸呢?.
或者作為智力題.
炎怎麼可以不咸呢?.
你會發覺炎就是.
我問過一些問題.
你可以把它的元素抽走.
炎是什麼元素呢?.
什麼?.
氯化鈉.
氯化鈉,對吧?.
Sodium Chloride,對吧?.
我完全不懂文科.
如果把它的元素抽走.
那它就不咸了.
但問題是.
如果把它的元素抽走.
那它就是炎了.

$^{1481}$它就不是炎了.
所以炎是堅強的.
你要摸就是.
弄到它不是炎.
但如果它是炎.
它就一定會咸.
後來有些人對這個問題有不同的答法.
他說那些工業用炎的.
可能就不咸.
但工業用炎是什麼呢?.
可能成績會死掉.
所以你也不需要.
你只需要把它抽走.
你不需要把它抽走.
你只需要把它抽走.
那些是會死掉的.
所以你也不知道是不是不咸.
因為你真的沒吃過.
所以唯一一個.
如果是智商題就會答的.
什麼炎是不咸的?.
作為智商題的話.
不是專業的那些.
但是什麼炎是不咸的?.
就是放在一個坑裡面的炎.
為什麼那些炎不咸?.
就是因為它沒被人品嘗過.
所以要它不咸.
只能夠是智商題.
就是說你只能在沒被人品嘗過的炎.
就會不咸.
正如光也一樣.
當我們放在鬥底的光會不光.
其實這個也是假設.
其實它仍然是在光的.
不過你見不到的意思.
所以這個正正就是耶穌一個很巧妙的比喻.
就是那個見證.
那個被人看到.
品嘗到的行動.

$^{1521}$其實是與生俱來的.
就是你作為基督徒.
你本身就是.
帶著這種見證在身上.
除非你將這種見證的能力.
收起來.
藏起來.
放在一個炎坑裡面.
或者是在鬥底下面.
所以炎和光的比喻正正是.
一個行動的問題.
你就是那個炎和光.
這個是你的being.
作為基督徒的being.
但是這個being同時是你的行動.
兩者是分不開的.
而且你本身就是要去.
就是會做這些事.
炎就自然會咸.
他不需要特意去想一些咸的東西.
就是不需要特意去做一些咸的東西.
就是這樣.
光就一樣.
光就是會照射出去.
他不需要特意去想一些plan.
想一些intentional的東西.
動機去這樣去過.
所以當我們去想這個比喻的時候.
撲克語也有這樣說.
不過他說得比較足夠.
會說得比較優雅.
就是炎和光就是這樣的一個本質.
他本質上就是一個見證的東西.
他不需要他特意去做一些什麼.
而那些東西是可見的.
必定可見.
當他.
就像潘鳳華說的.
逃離到不可見的地方的時候.
其實是什麼呢.

$^{1561}$就是違反了他的呼召.
違反了他的呼召.
所以這個就是潘鳳華在第五章裡面.
特別說到.
他仍然跟隨耶穌的思想來說這件事情.
當一個人來跟隨耶穌.
聽見耶穌的呼召.
做基督徒.
為他受苦的時候.
這樣的一個舉動.
必然是可見的.
他不需要特意去想一大餐.
如何見到或是什麼問題.
他就會被人見到.
所以這一話.
整個第五章.
潘鳳華就稱之為.
基督徒生活的超平凡.
這個我講到講過.
叫做Ause Authenticum.
這個字是一個很特別的字眼.
整個第五章.
想想第五章說什麼.
第五章說什麼.
就是說你們被人打完又如何.
再打錯又如何.
愛仇敵.
這些行徑.
你們要完全.
像我們天父完全一樣.
這些超級難度高的行徑.
因為跟隨耶穌的緣故.
基督徒一群門徒.
願意去這樣做.
這些事情.
愛仇敵.
為神禱告.
幫我犯姦淫.
更加不會動人臉.
這些事情.

$^{1601}$他稱之為基督徒生活的超平凡.
Ause Authenticum.
這個字我都說了.
是一個很奇怪的字眼.
Authenticum就是平凡.
Ause就是超過.
所以中文翻譯為超平凡.
基督徒生活是超平凡.
就是說是一個不正常的生活.
是一個很.
我都說了.
我那時候講到.
是一個超乎你的發揮水準.
超水準的.
超過你平時發揮的.
而今天所講的.
這些行徑.
其實是很出位的.
是很並肩的.
是一個很.
一個光影會看著.
你想想.
被人打完右邊.
就打左邊.
愛仇敵.
這些全部都是一些很.
很顯眼的東西.
明白嗎.
所以基督徒在世上.
特別當他願意跟隨耶穌的時候.
這些行徑.
整個第五章所講的.
正正就是一些超乎平凡人所做的行為.
這些行為肯定是可見的.
肯定會被人見到.
不會見不到的.
所以基督徒在世上的生活.
就是這樣.
願意跟隨耶穌.
簡單的聽從.

$^{1641}$就做出一些很radical.
很超級吸睛的行動.
肯定會讓人看到.
從而看到耶穌基督在那裡.
這就是那個原意.
所以這是第五章所講的.
今天我們不是講第五章.
是講第六章.
但這是背景.
你知道第五章整個超平凡的行徑.
是可見的時候.
然後Don Puff說.
講第六章.
是一個180度的轉向.
雖然是這樣.
但第五章所講的.
耶穌在讀中一講就反過來.
他說你們要小心.
不可將善事行在人的面前.
故意叫他們看見.
若是這樣你們就沒有天賦上前.
這是第六章裡頭十幾節所講的.
後面有三個行徑.
一個是施捨.
一個是禱告.
一個是禁食.
三個都是猶太人所講的好事.
好行為.
一些敬虔的舉動.
所以Don Puff就用了這段經文.
來講基督徒的隱藏性.
所以今天的主題叫基督徒的不可.
隱藏性和隱藏性.
就是這兩個章節所講的嚮導.
耶穌就這樣說.
耶穌就說你們的善事.
你們的好行為.
不要讓別人看到.
不可將善事行在人的面前.
故意叫他們看見.

$^{1681}$所以發覺第六章所講的.
似乎又和第五章剛剛相反.
今天我們就開始講多一些.
沿著這本書裡面的講法.
這個隱藏的意.
Don Puff說第六章所講的.
是什麼呢.
大家不用看這裡.
可以回去看YouTube.
如果想看的話.
這個字.
所以你看到就會被你看到.
總之開頭就講這件事.
第五章講了這麼多這些東西.
這些超越的行為.
一些超頻繁的行為.
一些跟隨耶穌緣故.
這麼出位的行為.
這麼好的東西.
其實會有很多的.
side effect 出來.
什麼是side effect.
就是有些人會將這些行為.
變成了一些敬虔的驕傲.
一個屬靈的驕傲.
我被人打完左邊.
左邊踩右邊臉.
一件很吸睛的事.
一件很屬靈的事.
可以成為一件很驕傲的事.
我為我的仇人討告.
這些一方面成為了一個見證.
同時也成為了某種.
另一種的見證.
我似乎是在宣傳.
或是在高舉某些東西.
所以這個就是.
在開頭那一段裡面.
帶出來的問題.
其實這些東西.

$^{1721}$很容易會讓人覺得.
他稱之為宗教狂熱者.
或者是.
今日的靈派.
意思其實就是這些.
我不是說靈派不好.
而是說當時的字.
其實有點像宗教狂熱者.
他們會將那些行為.
變成了一些高舉.
很明顯的東西.
所以這個是.
怎麼翻譯這些言著.
今日裡面其實有很多.
我稍後再說.
所以這些很敬虔的行為.
這些跟隨耶穌裡面.
很吸睛的行為.
很容易會變成這些東西.
屬靈的驕傲.
成為了眾人的焦點.
甚至偏離了.
他跟隨耶穌的目的.
所以他就說.
這句話大家可以一起看.
堅持了基督門徒的能見性.
那個visibility.
當然有他的根據.
但這個visibility.
不是他的目的.
如果成為了我們的目的.
我們就失去了我們最初的目的.
就是跟隨耶穌.
而且我們只要做過一次.
我們就決不能再在我們的別的地方.
重新再起來做.
既然我們去選擇跟隨耶穌的時候.
我們就不能夠本末倒置.
或者失去了我們的初心.
明白嗎?.

$^{1761}$即是有人去跟隨耶穌.
然後跟隨得很吸睛.
成為了焦點.
被人見到.
然後失去了跟隨耶穌的初心.
或者本質.
所以這就是Pulver所說的.
我們的行動是可見的.
但卻不是為了叫別人看見而行.
所以這就是他開始慢慢說出的分別.
基督徒在香港.
我們的見證肯定是可見的.
但被人見到不是我們的目的.
Pulver也這麼說.
不是我們的目的.
那如果不是目的呢?.
這就是他開始慢慢梳理的問題.
其中一個小小的點.
下面我也寫多一點.
其實就是要隱藏.
所以為什麼耶穌說你們要隱藏.
就是這個意思.
其實是矛盾的.
既是要被人見.
但又要隱藏.
隱藏什麼呢?.
整個第五章和第六章之間有什麼差別.
如何能夠融合兩篇經文呢?.
Pulver提出了三點.
他說兩個章例如何疏解.
第一個問題是.
我們作門徒的能見性是向誰隱藏呢?.
既然我們被人見到.
我們隱藏什麼呢?.
我們不被人見到.
是被誰不見到呢?.
那必然不是別人.
我們不是不被別人見到.
因為我們就是光.
我們就是室內的光.

$^{1801}$是被人見到的.
我們是向自己隱藏.
這個是值得我們去想的.
他說我們是向自己隱藏.
我們的工作就是跟隨.
我們就是去仰望那位耶穌.
我們必須不覺得自己是異.
而只是仰望耶穌當中.
他就好像不是超平常的.
而是十分平常又自然的了.
第五六行.
他什麼意思呢?.
就是說當我們向自己隱藏的時候.
我們做著做著去跟隨耶穌的時候.
我們不覺得自己在做什麼特別的.
我們所謂的超平凡.
其實當你向自己隱藏的時候.
你是會不覺得自己有什麼超平凡.
你只是在跟隨耶穌.
你是在做著平常又自然的事.
雖然後果是很超平凡.
但你作為一個跟隨耶穌的人.
你應該是不覺得有什麼特別.
所以簡單來說就是那個我.
是沒有了的.
我是向自己隱藏了.
如果那個超平凡成為了重點的話.
那我們就變成了吸睛的地方.
當你覺得自己跟隨耶穌.
做些很好的事情的時候.
你覺得很好的時候.
你開始想想那些人看到你有多好的時候.
其實這個就已經偏離了那個的意思.
變成了那個超平凡的事情成為了焦點.
而不是跟隨耶穌.
所以這個說法說得有點深.
但其實意思是大概這樣.
就是說當我們向自己隱藏的時候.
其實我們會做著做著.
我們不覺得自己有什麼特別.

$^{1841}$你應該這樣想.
我們只不過是跟隨耶穌.
當你考慮著嘗試用第三個角度.
看回自己的時候.
這個就是問題的開始.
不知道你值得想一想.
你平時做基督徒或者你做了一件好事.
或者你是作見證的時候.
當你見證耶穌的時候.
你會不會突然從第三個角度.
看回自己.
想想這個畫面有多美.
這個畫面有多像見證耶穌.
當你用第三個鏡頭去看回自己的時候.
其實這件事正正就是開始偏離了.
跟隨耶穌的方向.
因為你是沒有向自己隱藏.
所以這段書講得很有意思.
就是基督徒的生活的本質.
最後那一段.
是超平常的.
同時也是平常自然的.
看不出來的.
因為他也看不到自己在做什麼.
只不過跟隨著耶穌.
所以這個就是布木佛羅第一個點.
他嘗試梳理兩個的張力.
他說其實隱藏是向自己隱藏.
第二他就說.
作門徒的能見性和不能見性.
怎樣能夠結合在一起呢.
怎樣能夠成為一件事呢.
他說.
這個我也有點心.
因為他寫得很短.
所以我也沒什麼機會多理解.
他說超平常的能見性和不能見性.
正正就是那個十字架.
十字架正正就是能見和不能見的中心.
這個中文翻譯有點問題.

$^{1881}$譯得不太好.
他說十字架是必須的.
看不出的.
同時也能看見的.
他就是超平常之處.
但原文裡面的字其實更加簡單.
他說十字架.
Dark Christ.
下面是一大堆字.
所以十字架就是必然的東西.
隱藏的東西.
可見的東西.
和超平凡的東西.
所以.
大家可以看十字架和跟隨的意思.
所以唯有背負十字架.
才能讓我們同時.
真正被看見的十字架.
隱藏的也是十字架.
這部分我也覺得不容易理解.
十字架成為了我們可見和不可見的.
最核心的地方.
第三.
第五六章的矛盾.
如何疏離呢.
他說答案在作門徒的意義裡面.
作門徒的意義是什麼.
就是專一歸順他.
他所含義的就是仰望主和跟隨耶穌.
所以你做這件事很簡單.
你就是跟隨耶穌.
這正正就是我們要做的事.
所以如果他只是仰望基督徒生活的超平常特質.
他就不再是跟隨基督.
這和剛才說的差不多.
但是跟隨耶穌的本質就在這個位置.
所以你問自己.
是不是跟隨耶穌.
這就是最重要的問題和答案.
其他的後果.

$^{1921}$跟隨耶穌之後有什麼好事發生.
有什麼影響.
有什麼效力.
這些全部都是不重要的.
或者不能夠取代.
單純跟隨耶穌這件事.
所以這就是唯一我們能夠自然做的事.
所以回到那個位置.
就是簡單的聽從和跟隨.
成為了我們要做的事.
最後我們有個總結.
其實我們所做的事.
是基督的德行.
那些好事.
其實只能夠是因為耶穌基督的緣故.
所以作門徒的德行.
只要你完全不覺得.
只要你不察覺的時候才能做出來.
就是當你去察覺自己在做好事的時候.
這件事就會開始偏離了本質.
這個道理是簡單.
但是對教會來說可以有很多的用處.
我們現在會談談這件事.
當我們不察覺的時候做出來.
才是一個最純潔的東西.
當你察覺甚至有部鏡頭看到自己的樣子.
甚至想象到後果的時候.
這個正正就是我們要小心的東西.
任何你察覺到的好事.
都開始有些問題.
因為任何的好事都是你不察覺到的.
有一些例子.
假如我們想知道我們的善恨和愛的時候.
那就不是愛.
甚至對我們愛仇敵的愛都不是察覺到的.
我覺得這個故事是真的.
他說什麼叫愛仇敵.
就是當你去愛仇敵的時候.
其實你已經不覺得是仇敵了.
你愛一個人.

$^{1961}$你愛到已經忘記他是仇敵.
當你還很生氣.
我想要仇敵.
你仍然在看自己的第三個鏡頭.
看著愛仇敵的圖畫.
所以很特別的.
用了一個盲目的狀態.
當你自願的盲目.
當你看不到自己的時候.
其實這個正正就是隱藏了自己的東西.
愛仇敵.
你已經愛到一個對象.
你已經忘記他是仇敵.
那就是愛仇敵.
就是我有個自拍.
有個看著鏡頭看著愛仇敵的圖畫.
有多漂亮.
所以那個見證就是這樣.
那個見證是一個.
不含一種第三身的動機.
你甚至忘卻了自己在做這個見證.
你只是在跟隨耶穌.
所以這個就是顧佛的意思.
所以在最後寫得很好.
一個自願的盲目.
其實是基督所照的眼目.
只是讓他確知道.
在生活裡面正正是在隱藏著自己.
所以這個就是潘福華的神堂.
嘗試去理解整個的見證問題.
我們在世界上肯定是被可見的.
但是我們這個被人們看到不是我們的目的.
當我們嘗試將它成為一個目的.
甚至是一個營運的方針的時候.
可能這就變成了不簡單跟隨耶穌.
可能都是好的事情.
但這就不是一個好的.
真正最單純跟隨耶穌的行徑.
就是很簡單跟隨耶穌的時候.
你會忘記了這些事情會被人看到.

$^{2001}$而這樣你會被人看到.
今天就講了大扎潘福華.
這些1937年的這本書.
對今天我們留堂有什麼關係呢.
我們留堂這間教會.
你如何做一間見證耶穌的教會呢.
當然我們回到第一課.
第一課我們說當我們被呼召做基督徒的時候.
其實我們就已經在世界和社會裡面.
所以用一個炎和光的比喻.
你本身的呼召就是跟隨耶穌.
而在世界裡面去見證耶穌.
這件事是你整個人的行動.
第一我想講的是.
見證是一個行動.
不是一篇作出來一段一千至五分鐘的講稿或者故事.
而是你整個人的行動.
都說了你回想失見證就會想到什麼叫作見證.
失見證是什麼呢.
就是你的行動.
你不是故意很自然地做出來的.
失見證就是你生命里不需要刻意做出來的.
你不需要去失見證.
最是你流露出來的.
所以相反.
作見證也是一樣.
是你的行動.
是你整個人跟隨耶穌的時候.
整個的舉動.
所以這是我們第一樣要知道的東西.
從個別來講.
我們在流唐經常強調.
我們很想大家在社會裡面去作見證.
所以不需要特意有很多的侍奉.
而你在社會裡面的見證.
就是一個很重要的你要做的事情.
很多東西.
從你的家庭到你的公司到社會裡面.
都是我們希望大家能夠做的見證.
而所謂做.

$^{2041}$其實都是跟隨耶穌.
老實說.
都是一個這麼簡單的要求.
就是在世界裡面去跟隨耶穌.
這個就是我們最首要要做的見證.
那件好行為就在世界裡面做出來.
而我們不是特意要做給別人看.
不過是會被人看到的.
你明白我的意思嗎?.
就是他肯定會被人看到.
但這個不是你的目的.
你的目的就是做回應該你要做的事情.
我想這個是我們每弟子每妹都要去做.
和都記得我們要做的事情.
當然作為教會肯定不是只有這個.
教會有一個群體的方向.
這個是個人的.
你自己作為一個基督徒.
你要做的事情.
但作為一間教會.
肯定是會有一個不同程度的層次和考慮.
我們在牧者會談的時候都是一樣.
你教一個初訓的作見證.
總有一些教導成分.
總有一些所謂.
你就做這件事.
那就做了.
那就學會了去作見證.
作為群體.
我們是會有一些一起去做的事情.
那些事情就不會沒有計劃.
你不會突然間碰到就做.
總有一些計劃.
總有一些特意要做的事情.
但重點不是.
教會始終是有策略的.
有策劃.
但不代表這些事情是特意要做給別人看的.
所以教會的群體見證肯定會有一些計劃.
有些Emmy大家幫忙的事情.

$^{2081}$這些肯定是會預先來計劃好的事情.
所以第二個我們可以參與的.
就是我們整個教會群體的工作.
我們會有flown.
這些都是我們希望能夠大家一起去做的事情.
這個也是見證來的.
這個也是一些好事.
大家記住我們不是特意要去sell耶穌.
但也不是特意要avoid.
不講耶穌.
簡單地做一件耶穌叫我們做的事情.
我們要關心論捨.
我們見到有人需要就幫助他.
就是這樣.
最後.
這一段不重要.
大家都知道flown是我們很重要的一個元素.
第三就是我們的群體生活.
你發覺我們是沒有很多很特意要做出來的事情.
希望大家能夠透過大家的群體生活.
能夠讓人看見.
從而去見證耶穌.
所以那些生活.
首先問你生活是不是一些被人見到的生活.
是不是一個作門徒的生活.
是不是我們班基督徒所做的事情.
是不是作門徒的生活.
有兩個層次.
第一個就是我們整個敬拜群體.
可以叫人參加.
讓人看見.
從而讓人認識耶穌.
第二就是我們的club.
club絕對不是我們一群人打乒乓球.
或者行山就玩完很開心的東西.
而是我們都很想透過這些生活.
能夠讓人看得見我們本身的生命.
從而認識耶穌.
所以這件事是我們很重要的目的.
搞club其實是為了讓大家可以.

$^{2121}$更多的機會向人看見我們的生命.
讓一群行山友.
透過行山能夠認識我們的群體.
看見我們的生活.
讓一群咖啡友.
或者其他不同的block game友.
都能夠這樣來見到耶穌.
這些是我們的生命流露.
不是我們純粹要去sell什麼.
目的就是這樣做.
我想這兩件事是分不開的.
我們本身的群體.
就是我們整個展現的內容.
所以我想我們流行都開了五年.
我們經常強調.
我們要在香港裡面見證耶穌.
是時候要見證耶穌.
不是大家聚在一起.
不是純粹依次.
而是我們很需要想想.
如何能夠自然地.
透過我們的生命來見證耶穌.
這就是我們想做的事.
事後不如叫潘Sir出來.
我們可以一起有些問題和大家談談.
大家對於見證這件事有什麼看法呢.
或者你如何理解教會在基督徒.
在香港見證是怎麼樣呢.
我們一起談談.
最後一次call time.
對,還有一個.
最後的缺字.
我都講了第一堂所講.
我們說要新見證耶穌.
這是我們第一堂所講的內容.
就是叫大家的呼召.
正正就是來見證耶穌.
所以大家可以一起來.
這是他思想的事情.
這次應該很靠近了.

$^{2161}$我想第一天回教會就會講這件事.
你成為基督徒了.
要有好見證.
對於大家聽了那麼久.
或者經歷了那麼多年.
可以分享一下嗎.
舉手就是了.
我們有個麥給你.
謝謝.
對於....
是,後面.
是,Testing.
聽到John這樣講.
我會覺得是一個真正的謙虛.
可能是人家看到你一些好行為.
或者是見證的時候.
你會說.
人家會說你很好.
你就會說一句.
也算是這樣吧.
不是的.
然後他就會說一句.
你謙虛而已.
這個時候你就不是.
尖尖自喜地覺得自己很謙虛.
而是真的一份.
覺得自己還有不足的謙虛.
這樣聽下去.
不過我還有一個問題.
如果是一個長期那麼真正謙虛.
或者是一個世界那麼追求自我實現.
想建立自信.
建立多一點我們的自我價值的時候.
長期在一個那麼自我毀滅的狀態.
如何能夠有滿足感繼續下去呢.
我只能夠答就是跟隨耶穌.
如果是奔赴黃毛的話.
他就會覺得你就是在看前面的耶穌.
這樣跟隨.
少一點看自己.

$^{2201}$我剛才說的.
少一點用第三部鏡頭看著自己的畫面.
這個是他所說的.
我們有機會一起弄個讀書會.
看這本書.
就是說跟隨與個人.
就是跟隨與個別者.
他就說這件事.
其實你正正就是一個簡單的跟從.
就是這樣.
將那件事單純地去思考.
放下其他的考慮.
這個也是我們在樓堂里.
徐牧師經常說的.
就是這樣的感覺.
那種很簡單的思想.
跟隨耶穌.
其他的東西放下.
這個正是我們最關鍵的核心.
當然牧羊裡面我們有很多東西.
潘sir也說了很多這些.
自我價值.
K-Club的東西.
用Pull for Pull的話就是跟隨.
有沒有多一點.
我不是鑽著你的字.
剛才有個字是說自我抗爭.
其實我就不是太喜歡用這個字.
因為事實上你做得到.
而是做得好.
別人是欣賞你.
所以就不用說不是的.
其實你是的.
你真的是.
因為你真的做得到.
做得好.
別人欣賞你.
或者欣賞到這件事的時候.
那不就是對你一個肯定.
on the right track.

$^{2241}$繼續做下去.
你沒理由說.
你煮了些東西.
其他人說其實一般般.
不是我排到出街.
個個都等著吃這些東西.
你不要說.
隨便吧不是的.
有客觀標準.
這個很重要.
今天教會裡面.
剛才我說了很多application.
今天教會最大問題是什麼.
就是我在想客人有多少.
還有我在想客人有什麼反應.
而不是想我煮的食物.
怎麼煮食物煮得好.
今天教會最大問題就是這裡.
想很多的後果.
想很多marketing.
或者想很多其他的人.
怎麼想和怎麼多人來.
所謂多少人來.
但不想最簡單的.
就是我怎麼煮好食物.
專注在你的烹飪裡面.
才是你的目標.
當你煮得好.
都是專注在自然多人來.
明白嗎.
所以那個意思就在這裡.
今天我們教會見證就是這樣.
我們應該是專注在我們.
去跟隨耶穌裡面.
自然就有好的行為.
吸睛的一些甚至人數.
但這不是我們應該要想的東西.
我們想怎麼去做好教會.
怎麼去跟隨耶穌.
應該做什麼.

$^{2281}$應該要做的.
而不是想教會的後果.
那些都有成功的.
今天方安派教會的定義就是.
將教會變成一個生意.
這樣做多少人會來.
有什麼需要就煮什麼給他們吃.
但完全沒有想煮什麼.
完全沒有專注在烹飪裡面.
我想這就是布福華想說的東西.
問:你好.
我想問一點.
因為是關於群體和隱藏性和不能隱藏性.
我覺得作為一個基督徒.
就會覺得我要做好這個見證.
做好見證代表他欣賞我這件事.
但我們做事工某程度上也涉及這件事.
但我們不能讓別人知道我們的矛盾差異.
我就比較疑惑隱藏性和不能隱藏性和事工上的東西.
問:再說幾句事工是怎樣的.
可能我們是做一些關心社群的事工.
我們是關心社群的事工.
但也可能會被人標籤.
其實你只是想做一個見證給別人看.
我會疑惑如何不隱藏性又隱藏性的比例.
我說應該不是太想這件事.
現在很多人都不想這件事.
剛剛Emmy才發了一張照片給我看.
就是說他們和一群基督小學生考完試就請他們吃炸雞.
這很明顯不是想讓別人看到的事.
是一個愛的行為想讓他們成長.
這是一個很好的見證.
因為他不是想讓別人看到他有多好或做什麼好事.
而是從一個初心出發.
就是想幫他們 愛他們.
但這些事也不需要隱藏.
但你做的時候的動機.
你知道是為了什麼.
不是為了讓別人看到我耶穌有多好.
或者見證耶穌做了什麼好事.

$^{2321}$所以動機和次序要分得清楚.
但做完之後不妨讓別人看到.
也不妨讓人認識耶穌是可以的.
直接的例子就是.
你可能沒有見過或聽過.
我親眼見過.
有些群體去掃街.
餵 等一下.
這個位置可以嗎?清清楚楚.
這樣的.
拍了位置才開始掃.
掃完就走了.
海邊掃垃圾也是這樣.
你知道那些位置是為了那些來做.
我想重點仍然是出發.
最初用什麼方式去做到那件事.
結果是什麼其實不是我們最關心的.
有兩個問題或想法.
第一個就是.
我平時有探討基層家庭.
剛才你提到有些團體.
喜歡隱藏自己基督徒的身份.
有些打著旗號說我是基督徒.
你會不會回教會.
我記得最近有一次探討的經驗.
就是.
目者說.
其實我都探討了你一段很長的時間.
其實是否有什麼.
令你不能夠放開心去接受耶穌呢?.
我感覺你又不是抗拒.
但你又好像不太接受.
他不斷地教導.
我因為第一次接觸這個家庭.
其實我都不太認識.
但那一刻我自己都覺得.
突然間很感激.
他本身是關心他最近的需要.
工作的需要.
突然間就說這些.

$^{2361}$比較基督的內容.
其實對於我來說.
我都不舒服.
但我又覺得.
這些目者.
其實我又很尊敬.
他亦都是.
雖然我聽起來都覺得不舒服.
但我又覺得.
他又好像很有經驗去做這些事.
他自己又覺得.
他平時都是這樣.
我又覺得.
一方面我又覺得有些壓力.
另一方面我又覺得.
如果不是有些目者這樣去傳.
可能未必傳得成一個整杵的福音.
但剛才聽你這樣說.
我就會有些迷失.
我真的在那個處境里.
其實我可以怎樣做呢.
我是否.
因為你不斷地探討.
其實你每一次都是重復.
你最近怎樣.
重復禁放.
究竟應該怎樣做呢.
何時隱藏何時不隱藏.
或者目者去傳福音的方式.
或者我自己覺得.
我怎樣傳.
怎樣去作見證的方式.
是真的不同的.
但其實是否一定要說.
誰是好誰是不好呢.
因為他又真是我尊敬的人.
我覺得又不是一個錯的表達.
只不過是那一刻.
我自己都覺得.
如果我是受眾.

$^{2401}$我自己都覺得很大壓力.
第二個問題就是.
因為我最近從德國回來.
我正在想你剛才說的.
其實有時代性的.
二戰的時候.
說的是你每一天都在想.
你究竟明天會不會還在生.
都未到說要表明自己身份.
即是我明天能否吃東西.
或者明天我家人.
是否在自己旁邊都不知道的時候.
即是可能我表明自己是基督徒.
我已經被人殺了.
其實是否當時.
潘博華去寫的時候.
會不會他都顧及到.
這些考量去跟基督徒說.
其實你是由心底去散髮基督的香氣.
你不需要特別去表明.
但現在香港的社會又未去到.
我一定要隱藏我基督徒的身份.
我覺得會不會二戰的情況.
跟現在香港的情況有少少差距呢.
不是少少,是仍然有差距.
會不會都影響到我們.
如何去詮釋隱藏性的揣摩.
先回答你第二個問題.
這個比較技術性.
德國的情況就不是這樣.
因為德國的情況.
基本上每個人都是基督徒.
Hila都是基督徒.
所以不是非基督徒.
基督徒一個那麼大的問題.
不過他們當時的背景是這樣的.
這本書是寫給一群神學生.
那群就是去送死那群人.
一群地下神學院裡面傳道人的人.
所以都挺強調受苦.

$^{2441}$在困難黑暗裡面去送死.
我小時候看這本書.
覺得死是一個比喻.
我越看越覺得.
這可能不是純屬比喻那麼簡單.
而是在背景裡面.
當基督呼召你去死.
這件事不是純屬象徵性的事.
可能是一些受落生的人都要面對的事.
所以是否說要我們去受苦呢.
還是說在背景裡面.
都有這樣的背景讓他們實在地受苦呢.
我覺得是後者都有很大的機會.
其實跟香港今天不是差太遠.
不是說誰厲害些誰差些.
但肯定不是說他們沒有了基督徒和非基督徒的困難.
大家都面對著某一種信仰是很真實的情況.
我都說兩年前.
我們講《國安法》之後.
好處就是聖經寫的東西越來越真實.
很具體的逼迫或者盼望.
這些全部都不是象徵性的.
而是很真實的東西.
這就會跟聖經相似.
跟德國的時候相似.
所以我覺得時代的某些東西是相似的.
跟那種危難的時候相似.
跟內瑪帝國的基督徒.
德國的二戰基督徒和香港.
香港差一點.
香港沒有那麼危險.
但背景是類似的.
第一個問題是.
(問問德先生).
第一個問題對我來說.
問得很好.
很典型的.
有什麼時間去講.
或者什麼時間去要受眾有適量的回應呢?.
我的做法就是.

$^{2481}$用你剛才講的目者的做法.
是否很好.
或者做了之後令到很緊張.
好像令到參與的人很凶.
一定要做決定.
好像推卸責任.
我覺得都看目者對於探訪過程當中.
他在做什麼.
當然你說你都尊重目者.
我自己的做法就是.
在探訪過程當中.
當然我不會推卸責任.
但我都想瞭解一下.
有什麼問題令到他很緊張.
因為我們很多時候在探訪過程當中.
關心他全人.
除了物質.
我們有些物資供應給他.
其實心靈上有什麼可以聆聽.
在聆聽過程當中瞭解一下.
其實有些地方.
如果我們不再推一推.
要他去想多一點的時候.
他可能都灑了一個就完了.
推一推手就沒有了那件事.
所以我理解目者就是.
有些事可能他曾經發出邀請.
或者他曾經.
不過他每次都是推了去.
或者覺得遲些再說.
或者等待過程.
所以如果你覺得.
目者有點像很推卸責任.
逼他做決定的時候.
我就覺得如果時候還沒有到.
就再放下.
因為我自己都試過.
有叫做推過一下.
有些人反彈.
我通常仍然會保持關係.

$^{2521}$但是我試過有一次.
不要說幾次.
我試過推一推的時候.
再推多一點的時候.
那一下有弟兄姊妹按鍵.
他就問多了回應.
事就這樣成了.
所以我覺得這個有點不容易拿捏.
一下推那一下.
是大力勢力還是時間.
所以如果你剛才說.
那個目者都探了很久的時候.
我覺得某程度上.
目者都覺得.
存感情牌存得差不多.
當然我相信他不是帶著.
一定要探十次之後.
就要他決志的心態.
但是我覺得整件事.
是怎樣可以讓那件事.
是雙方都感受.
有一定程度的信任去做決定.
這個都是要想清楚.
或者拿捏一下大家之間的關係.
因為探訪.
我可能都不是像社工那樣.
要計算探多少次.
這個位置.
所以這個位置有點難拿捏.
但是我自己做法.
都是會讓他明白到.
探訪過程當中給他感受.
其中我自己做得最多就是.
如果我不在.
或者找不到我的時候.
或者有什麼家裡.
晚上有什麼需要的時候.
你都可以鼓勵一下他.
可以祈禱.
或者是告訴他.

$^{2561}$我們為你祈禱.
留下那個網絡.
讓他知道真的有個心.
除了分享物資之外.
都是知道帶著一個關懷.
和一個代禱的心意在當中.
不是要他決志.
這個像是最終的結果.
這樣做法.
其他人有沒有?.
不好意思.
想追問一下.
其實當講到隱藏性和不能隱藏性的時候.
我們會有一些例子.
我們做一些科研工作.
是否真的要打正旗號.
說我們是基督徒的時候.
我會看會不會是.
我們最終的目的.
是一個心甘樂意的信仰.
或者自然流露的信仰.
而不是刻意去否定隱藏性.
或者不能隱藏性.
因為我在想.
人有一個潛力.
想展示自己.
所以耶穌就叫我們要隱藏一點.
但是去到潘博華的句子.
他說如果你隱藏得太多.
你就是逃避神的召命.
其實兩邊都不是最終的目的.
而是一個自然流露的信仰.
我這麼想.
那個結論應該是這樣.
就是基督徒是不需要向人隱藏.
沒記錯吧?.
是向自己隱藏.
所以這麼說.
其實是不需要刻意隱藏.
不告訴別人你是基督徒.

$^{2601}$這個很明顯.
但是同時也不會讓別人知道.
你是基督徒成為首要的目標.
所以兩個很重要的原則.
我們不需要隱藏.
但也不會將那件事成為首要目的.
這兩個原則就會讓我們知道該怎麼做.
所以你問我是否做社官.
要刻意不說耶穌.
我覺得不是.
因為潘博華說不是.
他說向自己隱藏.
不需要隱藏.
但如果目標是為了推銷而做好事.
就不對了.
所以就做好事.
做回你要做的社官.
探訪的時候去關心他.
但不需要否定或隱藏你的身份.
甚至介意地不告訴他.
但就不是成為了首要目標.
但你會跟他說.
所以在這個規則下.
很多人有不同的看法.
我見到的老師.
我們見到有報道團.
我們這些學生有些是.
怎麼說呢?.
(讀答口上).
是報道精.
他們很開心.
在船上帶著傳呼音.
他們是很有心的人.
我不覺得那些人有問題.
他現在成為了牧師.
成為了報道會牧師.
所以這種熱情.
如果是出於自然流露.
我覺得沒問題.
所以你覺得很勉強.

$^{2641}$就不需要這樣做.
我姐妹覺得這樣做.
好像不太和她們性格一樣.
就做回你自己去跟隨耶穌.
和見證耶穌的方法.
但有些人可能是以前的神後.
或者覺得我是這樣做的.
這件事他不是推銷自己.
他不是將目的變成.
讓別人看到他多好.
就做這些事.
所以我覺得不同人有不同的.
好行為的方法.
我不覺得單純地說.
四律報道的福音書.
那些是錯的.
最重要是不要將那件事.
成為了一個純粹.
一個硬銷的目的.
我帶你來自由失魂書.
其實純粹是想告訴你.
這就變成了.
你將那件目的.
不是好行為.
所以不同人有不同的.
跟隨耶穌的方法.
跟隨耶穌的時候.
我每天都帶報道.
和人家說福音的.
我覺得這些.
不會有什麼問題.
最不可以.
兩個avoid.
avoid就是要.
取代跟隨耶穌的目的.
和要隱藏自己.
所以這兩個是.
要避免的.
我們做地區服務.
都是和地方的NGO合作.

$^{2681}$然後他們就轉介.
然後我們就去探訪.
和去分享物資.
我們過程當中.
都是做一個協調.
和一路參與在群體當中.
我們也沒有宣傳我們是教會.
我們有不同的NGO參與的時候.
後來他們問.
為什麼你們經常來.
或者你們是什麼人.
然後他們問我的時候.
我們就說我們是.
我們不是聖雅各福林會.
我們是一間教會.
叫Flow Church.
就是說流塘.
什麼流塘.
什麼來的.
人家問你心中盼望的緣由.
你就常作準備.
你就以溫柔敬畏的心回答.
就是這樣.
就算我們定期都會分享物資.
我們也不會放一張封.
放一張單張.
我們也不會做這些.
分享物資就分享物資.
探訪就是探訪.
所以整件事就是.
你的光照在人前的時候.
人家因為你的好行為.
他就會問.
那你就歸榮耀彼在天上的父.
這也是耶穌去做行徑的時候.
一個正面的反饋.
因為你做的時候.
就是做你本身光的特質.
你做的行為就是.
做你本身炎柚的影響力.

$^{2721}$所以你就做你自己的本相.
就像剛才John說的.
你用什麼形態就是什麼形態.
你用什麼形態就做你自己的形態.
適合你的東西.
所以我認同那個.
有人隨手.
我也認識有牧者.
隨時袋子里都有一些福音工具.
五色珠也好.
五色筆也好.
什麼都好.
但是小弟就什麼都沒有.
那嘴巴算不算.
就是這樣.
形態是很重要.
你自己適合自己的方式.
Hello.
其實也是和剛才大兄提問的問題有關.
有些衰竭.
就是技術上有些分別.
例如我們知道.
其實我們做侍奉最重要就是.
我們侍奉目的.
例如去派飯.
探訪.
都是為了關心有需要的人.
回想起之前有時候教會可能.
在節日會有一些小冊子.
可能寫著耶穌愛你.
或者和探訪者說.
我們教會來探你.
我在想.
如果我們說教會來的.
或者我們基督徒.
當然我們心裡不是說.
是為了侍奉自己.
而是將榮耀交給神的群體.
例如教會.
或者將榮耀交給神.

$^{2761}$這個行徑又算不算是為了自己.
就是看到自己呢.
我想會不會有少少分別呢.
因為我為什麼這麼問呢.
就是說.
很多信或者不信的人.
都不斷地做善事.
特別是很多.
其實這些侍奉.
當然長期的侍奉會.
受侍奉的對象會慢慢認識你.
會問你.
你們是哪個群體的.
我們就自然介紹.
也有很多短期的.
可能幾個月探訪一次.
大家都沒見過大家.
有時就會說.
我們是教會來的.
又或者.
如果心態純粹是.
我現在想將這件.
做善事的背後目的.
我真的想幫助你.
也同時想將.
做善事的.
類似將榮耀.
歸給教會或者神.
如果我們特地這樣介紹.
又可以嗎.
在這個情況.
你問我也會說.
那個先後在哪裡.
做事的人自己心想.
其實很難說.
有些人每個人都不一樣.
同一件事有兩個人做.
但是有兩個人的動機.
其實很少說這麼單一.
單單純粹想.

$^{2801}$榮耀神而不想幫助你.
很少這樣想.
也想幫助你.
但主要是想.
我覺得是次序.
那件事是想純粹一個.
對於sell.
為什麼對於sell這麼抗拒.
所謂sell福音就是因為.
你只是sell福音而已.
後面是沒有任何的愛心.
或者是很想幫助目標.
所以這件事是會轟動的.
所以我想要避免的就是.
把目標不能取代.
跟隨耶穌的行為的時候.
當你自己覺得是這樣想的時候.
這樣就行了.
所以我想和那件事.
很多人的看法是不同的.
剛才打卡那些.
拍照那些.
似乎是想讓人知道.
想sell多一點.
我們大約不需要去到.
抗拒我們完全.
不去提自己是基督徒.
不知道是否能回答你的問題.
Hello.
剛才也聽你的分享.
其中一件強調的事.
也是那種自然的流露.
譬如鹽本身就應該有那種特質.
咸的那種特質.
但是如果譬如.
因為我們.
剛才說到.
關於一些很超自然的事.
我猜可能.
對很多人來說.

$^{2841}$未必是一個.
很立刻就做到的一些事.
假設有個例子.
我可能真的還.
我想問一下.
如果有一些情況下.
我覺得我有些事還沒做到.
但是那些事是好的.
究竟其實我是應該.
根據我的自然狀態.
我還沒做到.
我就應該不要.
純粹是為了希望.
那個人對基督教.
或者基督徒有好印象而去做.
我應該不做.
還是.
不是.
因為有些人會說.
你都要嘗試一下.
踏出你的舒適區.
你不習慣就不做.
也做了.
這個位置.
我想問一下是如何拿捏.
對於那些你覺得.
不完全是你自然流露的事.
是應該你都為了神的緣故.
繼續去做.
還是既然我這個.
我現在這個位置.
我覺得還沒做到.
我就先不做.
是這樣.
怎樣處理這個問題呢.
所以我看FourTruck不是說了.
那篇道叫不要尋常.
就是正正講到.
這個第五章.
所以你先聽回.

$^{2881}$我們永遠都是一個.
逼高自己的位置.
就是說.
你平常你不會做的.
這些好行為.
所以永遠都是逼高自己.
你願意去做.
我想這樣去做.
我都嘗試去做.
在這些超發揮.
超水準策略的情況下.
就做了.
做到了.
所以我們基督徒的行徑.
就是這些位置.
所以你永遠都不會等同於.
在自然所做到出來.
剛才Mufan講到很擊中的地方就是.
當你嘗試去.
只是想嘗試去做的時候.
就忘記了自己能不能做到.
那一刻只是自然.
所以一向都是不自然的事.
但當你忘記了自己.
show off 或者什麼好行為的時候.
那些就很自然地做了出來.
不知你明不明白我的意思.
很複雜的位置.
就是你只是想著.
我如何做得好一點.
為了耶穌.
嘗試去做這些好事出來.
那些是不自然的.
你逼自己做.
甚至是超過自己水平.
做到出來的東西.
但當你專注去做的時候.
就帶來了一些影響.
那些人就見到了.
但你忘記了這件事.

$^{2921}$就會帶來一些影響.
或者會見到很多見證.
很多吸睛的東西.
所以問題就是.
是不自然的事.
等到自然就做不到.
我用回保羅的說話.
就是行善是要學的.
行善是一個很不自然的事.
因為你不用再選擇.
通常都是愛自己.
所以基督徒要學習行善.
是一個經驗學習.
我自己以往在.
九龍城區的幕會的時候.
我就帶著群體去做平等分享行動.
這個社會認識和參與.
總有一些弟兄姐妹.
如果你做過平等分享行動.
這個形式的時候.
總有一些弟兄姐妹就覺得.
我有潔癖的.
我不能下區的.
但我會支持物資給大家分享.
她真的每次都支持到物資.
給我們分享.
她就祝福我們出去順利.
她做回她的部分.
但她等我們回來.
久而久之她就問我一件事.
就是我這樣做.
是不是沒有突破自己的舒適區呢.
我覺得你自己去到.
不是很舒服的話.
你就做到你的部分.
但她說.
那我需要突破一下自己嗎.
我說我的原則就是.
不要驚動她.
讓她情願.

$^{2961}$她的位置就是.
我有一天說.
你出去都不用像我們這樣.
跟別人靠近.
你可以遠遠地看.
感受一下.
你覺得原來要是這樣的距離.
不用這樣摟頭摟脖子.
和你聊天.
你慢慢去感受一下.
她有一次就跟我們出去看看.
她就遠遠地看看我們原來是怎樣的.
因為她自己發覺.
我不可以走得太近.
因為我見她走得太近.
她身體就覺得很痕跡.
因為是她自己的感受.
這是真實的.
她慢慢看多幾次的時候.
就會發覺.
其實我想聽聽你們說什麼.
她越走越近.
就好像剛才John那件事.
不知不覺其實坐在旁邊.
她越做那件事就越忘記那件事.
忘記自己的身痕.
就是這樣.
所以行善是一個經驗學習.
也是違反我們常態的東西.
也是在做的過程中.
你會在迷茫的過程中.
你就不知道.
告訴你這次要做些什麼.
因為你在做那件事的時候.
你就不用告訴自己我在做什麼.
那個理解.
潘鳳婭的意思就是這樣.
所以.
那件事不是你告訴自己要突破舒適區.
是在過程中你慢慢去學習.

$^{3001}$如何可以.
或者是那件事當中.
你如何可以讓自己.
在另一個角度試一下.
不是一定用那個道路去做那件事.
我想問一問這個.
淑靈的驕傲這種情緒.
就是它好像是一種.
我們不良隱藏性或者不平凡生活.
必然會引出來的一種情緒.
那它是怎樣的性質.
和我們應該如何處置.
它是好是壞.
還是一個很正常的一件事.
很正常的失敗.
我記得我自己初信主.
第一次淑靈驕傲是什麼事.
那時候我都說我初信主.
我那時候是很敬虔的.
每天都看聖經.
每天都靈修.
每天都這樣.
第一次覺得自己是挺好的.
那時候我都說.
當你覺得自己淑靈挺好的時候.
就是淑靈低落的開始.
那你就會開始出事.
所以是很正常的失敗.
沒有了.
就是這樣.
沒什麼好說的.
總是會有這種情況.
如何避免呢.
然後你就知道自己不是挺好的.
下次你也不是挺好的.
這些不是永遠的事.
我覺得每個人的靈命靈情.
都是會不同.
開始我就會不想這些問題.
不想會不會有問題.

$^{3041}$反正都不是挺好的.
很少去衡量自己.
不會覺得自己是挺好的.
不是謙卑地說.
而是都擺明知道不是挺好的.
所以就慢慢忘卻了這件事.
當你拆掉第三個鏡頭.
當你做基督徒的時候.
經常有第三個鏡頭.
看回自己好像很謙卑的那種.
或者是.
我這次聖誕時哭了.
經常會有第三個鏡頭看著自己.
所以少一點這種鏡頭.
多一點去不要看自己.
這個也是挺實際的.
我不知道什麼叫肅靈驕傲.
真正的.
因為肅靈驕傲.
我覺得這個詞是很堆砌的.
驕傲就不會是肅靈.
什麼叫肅靈驕傲呢.
因為總會.
我覺得自己讀書厲害.
有些人不用讀書都這麼厲害.
那你就不會驕傲.
可能我身邊太多厲害的人.
所以從來都不會覺得值得驕傲.
但是你說肅靈驕傲.
我就想了想.
我身邊也有些人說.
你讀了多少次聖經.
我說讀了多少次.
整本讀了多少次.
我說一次.
他說一次.
我讀了四次.
我說聖經不是說讀多少次.
是說做了多少.
整件事你就知道他錯了.

$^{3081}$驕傲的人就會錯重點.
你錯重點.
其實我不太懂得回答這個問題.
不過我想說.
你上第三季講.
肅靈驕傲.
其實整件事是錯重點.
他根本不是做要做的事.
你看到我們今天經常都會說.
見證其實要知道焦點是什麼.
見證就是知道你自己.
你做什麼和你所在.
你知道自己是being.
你就懂得做你的doing.
這個過程是需要你認真.
你就會看到焦點.
你的being是一個question.
我之前在Q and A的時候也說.
我們不是claim自己是基督徒.
我們是proclaim自己是基督的門徒.
你做的事讓人感受到你.
認識基督.
這個是重要的.
因為那個是你的being.
以至你的doing在做的那件事.
這個是.
所以驕傲的人都不會肅靈.
那最後一個.
最後一個.
多謝各位.
我們on board.
我自己每次都沒有看自己.
教書的樣子.
但很期待第三季.
我也覺得有第三季.
因為第三季其實就是想盡心一點.
我想是更加基本的東西.
剛才大兄說得很好.
會不會第三季講一下這些題目呢.
講一下獻碑.

$^{3121}$講一下什麼和神關係多一點的東西.
或者是健議.
或者是衛生的東西.
都希望能夠.
應該最多三季都會.
我們都是一個進階.
希望大家能夠從神學到生命.
到信仰反省裡面.
都能夠是一個很有思考的人.
也有言行的人.
我們一起祈禱.
主佑多謝你.
讓我們能夠神學八科裡面.
完成了我們第二季.
當中有很多弟兄姐妹一起去參與.
他們真的每次去實體裡面參與.
或者是在網上重看也好.
我們都知道他們都是有心.
來更加去反省他們的信仰.
讓他們能夠在裡面.
做一個更加理所喜悅的門徒.
我們樓堂都願意成為一間.
更加合理心意的教會.
從牧者到弟兄姐妹.
都是一起來成長.
求主讓我們.
記得你去呼召我們.
你呼召我們成為一個基督徒.
一群見證你的群體.
讓我們一間這樣的教會.
能夠認真單純地跟隨你.
單單地愛你.
單單地在香港這個社會裡面.
在不同的地方裡面.
都能夠真真正正地來見證你.
用我們的生命.
用我們的言行來見證你.
求主你這樣去兼顧我們這間教會.
奉主命求.
拜拜.

$^{3161}$我們可以出來一起拍張照.
可以.
謝謝.
字幕志願者:劉文英.
優優獨播劇場——YoYo Television Series Exclusive.
\newpage



\section{詩篇 90:1-17-20231202}
\label{sec:lfg8MyM5M04}
\textbf{【網上崇拜】大人者,不失其赤子之心者也|詩篇90\_1-17|20231202 [lfg8MyM5M04]}
\newline
\newline
連結: \href{https://youtube.com/watch?v=lfg8MyM5M04}{\texttt{ https://youtube.com/watch?v=lfg8MyM5M04}} ~~~~ 語音日期: 2023-12-02 
\newline
\newline
\hyperref[sec:w1NzLUX2_GE]{\small{< < < PREV SERMON < < <}}
~
\hyperref[sec:index_chronic]{\small{[返順時目]}}
~
\hyperref[sec:index_scriptual]{\small{[返順卷目]}}
~
\hyperref[sec:0oiGMpkgXB8]{\small{> > > NEXT SERMON > > >}}
\newline
\newline
詩篇 90:1-17-20231202
\newline
\begin{longtable}{cl}
\hline
\hline
章節 & 經文 (和合本修訂版)\\
\hline
90:1 & \begin{tabularx}{0.7\textwidth}{X} 主啊,你世世代代作我們的居所。 \end{tabularx} \\ \\ \relax
90:2 & \begin{tabularx}{0.7\textwidth}{X} 諸山未曾生出,地與世界你未曾造成,從亙古到永遠,你是神。 \end{tabularx} \\ \\ \relax
90:3 & \begin{tabularx}{0.7\textwidth}{X} 你使人歸於塵土,說:「世人哪,你們要歸回。」 \end{tabularx} \\ \\ \relax
90:4 & \begin{tabularx}{0.7\textwidth}{X} 在你看來,千年如已過的昨日,又如夜間的一更。 \end{tabularx} \\ \\ \relax
90:5 & \begin{tabularx}{0.7\textwidth}{X} 你叫他們如水沖去,他們如睡一覺。早晨,他們如生長的草; \end{tabularx} \\ \\ \relax
90:6 & \begin{tabularx}{0.7\textwidth}{X} 早晨發芽生長,晚上割下枯乾。 \end{tabularx} \\ \\ \relax
90:7 & \begin{tabularx}{0.7\textwidth}{X} 我們因你的怒氣而消滅,因你的憤怒而驚惶。 \end{tabularx} \\ \\ \relax
90:8 & \begin{tabularx}{0.7\textwidth}{X} 你將我們的罪孽擺在你面前,將我們的隱惡擺在你面光之中。 \end{tabularx} \\ \\ \relax
90:9 & \begin{tabularx}{0.7\textwidth}{X} 我們經過的日子,都在你震怒之下,我們度盡的年歲,好像一聲嘆息。 \end{tabularx} \\ \\ \relax
90:10 & \begin{tabularx}{0.7\textwidth}{X} 我們一生的年日是七十歲,若是強壯可到八十歲;但其中所矜誇的不過是勞苦愁煩,轉眼即逝,我們便如飛而去。 \end{tabularx} \\ \\ \relax
90:11 & \begin{tabularx}{0.7\textwidth}{X} 誰曉得你怒氣的權勢?誰因著敬畏你而曉得你的憤怒呢? \end{tabularx} \\ \\ \relax
90:12 & \begin{tabularx}{0.7\textwidth}{X} 求你指教我們怎樣數算自己的日子,好叫我們得著智慧的心。 \end{tabularx} \\ \\ \relax
90:13 & \begin{tabularx}{0.7\textwidth}{X} 耶和華啊,我們要等到幾時呢?求你轉回,憐憫你的僕人們。 \end{tabularx} \\ \\ \relax
90:14 & \begin{tabularx}{0.7\textwidth}{X} 求你使我們早早飽得你的慈愛,好叫我們一生一世歡呼喜樂。 \end{tabularx} \\ \\ \relax
90:15 & \begin{tabularx}{0.7\textwidth}{X} 求你照著你使我們受苦的日子,和我們遭難的年歲,使我們喜樂。 \end{tabularx} \\ \\ \relax
90:16 & \begin{tabularx}{0.7\textwidth}{X} 願你的作為向你僕人們顯現,願你的榮耀向他們子孫顯明。 \end{tabularx} \\ \\ \relax
90:17 & \begin{tabularx}{0.7\textwidth}{X} 願主-我們神的恩寵歸於我們身上。願你堅立我們手所做的工,我們手所做的工,願你堅立。 \end{tabularx} \\ \\
[1ex]
\hline
\hline
\end{longtable}
$^{1}$丁師妹平安.
大家聽的一首歌曲叫《細佬歌》.
這首歌曲是林海峰先生在2005年出品的一首唱片.
這首唱片的名字叫《三字頭》.
不知道大家有沒有聽過這首唱片.
可能大家有聽過吧.
你們應該都是四字頭吧.
當年我聽這首唱片的時候.
大概也是二字頭吧.
我很喜歡這首唱片.
這首唱片是討論一個人三十多歲的時候.
那個心境和文化.
所以大家當中如果是三十多歲.
很值得聽這首CD.
很精彩的CD.
不知道你什麼時候第一次覺得自己長大了.
你覺得自己長大了沒有.
我二十四歲的時候.
也不覺得自己長大了.
那時候我在讀神學.
在教會裡面實習.
背著一套西裝.
在教會廁所裡面對著鏡子.
看來看去都覺得自己像小朋友背著西裝.
我什麼時候開始覺得自己長大了呢.
當然是當爸爸的時候.
這是一個長大了的記號.
不過我覺得這也不是我最深刻的時刻.
發掘出來的也不是那個時刻.
我發現那個時刻是什麼呢.
就是我開始有老花.
這是我新的眼鏡.
漸進式眼鏡.
這個時刻我覺得自己長大了.
甚至和年老牽上關係.
我不是一個年輕人.
極其量只是一個年輕人.
今天我們會看C篇90篇.
思想成長和年歲的題目.
或者看一看這段C篇90篇的經文.

$^{41}$主啊,你世世代代作我們的居所.
朱山未曾伸出.
地與世界你未曾造成.
從僅古到永遠,你是神.
你使人歸於塵土說.
你們世人要歸回.
在你看來,千年如已過的昨日.
如如夜更的一更.
你叫他們如水沖去.
他們如睡一覺.
早晨他們如生長的草.
早晨發芽生長.
晚上割下枯乾.
我們因你的怒氣而消滅.
因你的憤怒而驚惶.
你將我們的罪孽擺在你面前.
朱山未曾伸出.
地與世界未曾造成.
從僅古到永遠,你是神.
你使人歸於塵土說.
你們世人要歸回.
(字幕組:我忘記了).
不過如果我們去.
(字幕組:我忘記了).
我們人生的年歲是七十歲.
若是強壯可到八十歲.
但其中所驚誇的不過是.
勞苦受煩,轉眼成空.
我們便如煙飛而去.
誰曉得我們怒氣的權勢.
誰安著你該受的敬畏.
隨著你憤怒.
求你指教我們.
(字幕組:我忘記了).
好在我們得著智慧的心.
不過如果你細心閱讀詩篇九十篇的話.
你會發現詩篇九十篇不單單是.
講述生命的短暫,時間的流逝.
人生將要過去.
而是更加講述一個人生命的年歲.

$^{81}$一個人如何渡過他的年歲.
他的歲月究竟是怎樣的流逝,怎樣的過去.
我的意思是,如果你細心閱讀這篇詩篇.
你會發現這篇詩篇不單單是講述.
上帝向千年如同一日.
我們的生命過得很快,我們的生命很短暫.
我們很快便會年老歸去.
更加具體講述的是.
我們的年歲過得很慢,而且過得不太好.
親愛的兄弟姐妹.
如果你這篇詩篇是講述有關年老末世的時候.
這首詩歌不單單是感嘆一生的短暫.
更加是對於一個人年紀漸長的一種成熟的控訴.
甚至乎他講述詩篇九十篇.
其實正是去同步人生的歲月.
對不起,我不是特別想用英文.
找不到一個更加貼切的中文去形容這個同步的字眼.
簡單來說,詩篇九十篇是對成年長大年長的一種控訴.
詩篇九十篇不單單是講述時間過得很快.
人在世上的日子很有限.
就算是讓你活到幾十歲,五十歲也好,七十歲也好.
其實都是很廢的.
不知道當中年長的弟兄姊妹或年輕的弟兄姊妹.
在讀書的時候有沒有一種感覺.
當你面對一些比你年長的長輩的時候.
有沒有一種很敬畏,恐懼的感覺.
我們叫做權威恐懼症.
我以前也有一點權威恐懼症.
當我們面對權威的時候.
你會發現面對一些人的時候.
你會發現這些人突然間覺得很敬畏.
你會覺得他們是一班大人.
你會覺得他們是一個比你年長有智慧的人.
一個很成熟的人.
當我們這樣發現的時候.
不好意思,今天真的老化了.
讓我再檢查好一點.
你會覺得他們是一班大人.
他們很有智慧,我們要尊敬他們.
跟他們差很遠.

$^{121}$但當我自己成為了四十多歲的人的時候.
我發覺原來四十多歲也不是什麼.
也可以是一班很敷衍的人.
年紀大,其實只是年紀大.
不知道你們有沒有聽過.
也是一首舊歌.
陳昇一首的歌.
叫做《像我們這樣的人》.
歌詞正正是說一班年紀大的人.
越是年紀大就越是敗壞的一種無奈.
歌詞裡面寫.
我一天上了多少網.
我一天用了多少個塑膠袋.
我一天走了多少的路.
我一天找了多少的夢.
我一天說了多少個謊言.
我一天撒了多少個諾言.
一千個縮小的痛苦.
也換不了真心的悔悟.
我軟弱去跟好人吐槽.
他更嘔心得要人低頭.
像我們這樣的人.
是不是應該在上個世紀就死去.
詩篇大概就是一篇這樣的詩篇.
詩人去記述年老摩西.
一個八十歲被呼召的老人家.
在礦業裡面四十年.
一百二十歲離不開礦業.
而去離開世界的老人家.
當他生命將要終結的時候.
他發現年紀大其實不代表什麼.
當我們去大概想想.
整篇九十篇裡面的說話.
譬如第五第六節.
經文這樣說.
早晨他們如新長的草.
早晨發芽生長.
晚上閣下敷肝.
經文告訴我們.
人的生就像草一樣的短暫.

$^{161}$不過當我們細味經文的時候.
經文不單單是說新長的草.
生命短暫.
而是說新長的草一直在生長.
早上發芽生長.
晚上閣下敷肝.
早上仍然是小孩子.
中午就變成成年人.
晚上就步入晚年而死去.
經文不斷在描述各式各樣的生長.
不斷在描述歲月的事情.
究竟四篇九十篇如何去理解歲月呢.
我們看看下面的經文.
第七到第九節.
詩人說我們因你的怒氣而消滅.
因你的憤怒而驚惶.
你將我們的罪業擺在你面前.
將我們的忍惡擺在你面光之中.
我們經過的日子都在你怒怒之下.
我們渡盡的年歲好像在星探式.
對詩人來說.
人的生就在上帝的震怒之下.
在上帝的怒氣之中.
人類的罪業和忍惡都擺在上帝的面前.
不過經文所強調的那種罪業不是一般的罪業.
而是先強調是一個人隨著他的年歲之後積累出來的罪業.
第九節這樣寫.
我們經過的日子都在你震怒之下.
我們渡盡的年歲好像在星探式.
詩人說我們一生經過的日子都在上帝的震怒之下.
不是嗎?.
就是年紀越大.
經過的日子越多.
你經過顛覆的罪業.
必然隨著你走過的日子越久.
就會越來越多.
就好像方書裡面.
用石頭打死婦人的情景一樣.
你的年紀越是增長.
你越來越大.

$^{201}$你的罪業就必然會越重.
所以詩經這樣說.
第十節.
我們一生的年日是七十歲.
若是強壯可到八十歲.
但其中所驚誇的不過是勞苦受煩.
轉眼成空我們便如飛如去.
你以為你自己很厲害嗎?.
做了那麼多年人.
你所驚誇的都只不過是勞苦受煩.
大家不要誤會.
我不是不敬老.
事實上我不是在說老人家.
而是我們要認真反省的時候.
我們如何去理解成長這回事情.
《詩歌十篇》正正是一篇.
控訴年紀歲月長大的一篇詩篇.
你覺得自己長大了沒有?.
或者問你覺得怎樣才叫成長?.
不知道你有沒有懷疑過我們這個月的題目.
即是小孩.
小孩是否一件好事?.
上帝不是叫我們成熟一點嗎?.
不是叫我們不要做小孩子嗎?.
做一個人難道我們不成長嗎?.
我們是要做大人還是小孩呢?.
問題關鍵是你如何成長?.
你是一個怎樣的長大?.
怎樣的大人?.
我們去推一把步去思考這個問題.
當然長大是好事.
上帝創造我們.
我們有上帝形象.
理論上我們越是長大成人.
就越來越像上帝.
我們越是成熟就越是做一個更加好的人.
我們就遠離小孩子的階段.
成為一個尚有所喜悅的人.
理論上成長是好事.
成長當然是一件好事.

$^{241}$但我們在人生裡面往往不是成長.
我們只不過是勉勉強強地長大.
經過一些年.
就好像雞,豬一樣.
明明沒有足夠的空間去成長.
像吹菊打口針.
肉是長多了.
肌肉是爆出來的.
外表是大了的.
好像是成熟長大了.
但裡面其實沒有實質的內容.
我稱之為強迫長大.
面對一個問題.
你可以用某個捷徑.
某個方法繞過它.
你長大了.
你成為了一個大人.
但你並沒有真正的去成長.
在這個年頭裡面.
這個社會裡面不斷地逼出一個大人.
就好像差點我不會飛的歌詞.
求時間變慢.
不想逼於成長.
我想這是對於社會.
對於生命一種崇高的奇緣.
你叫一個小朋友長大有很多方法.
但不是每一個都是健康的方法.
我們用很多很容易的方法.
很快速的方法.
很便宜的方法.
但在我們眼中.
這未必是真正成長的方法.
我想近日大家看的電影.
即是《年少日記》.
或者是《百日之夏》.
都讓我們去反思這個課題.
真人一個人長大了.
做事變聰明.
經驗多了.
不代表真正的成長了.

$^{281}$最少你看公司的那些O.C.Food.
什麼叫O.C.Food.
O.C.Food就是一些很有經驗豐富.
很老練的前輩.
Book in a warm way.
他們的豐富經驗在哪裡.
他們的老練在哪裡.
他們的厲害在哪裡.
他們的厲害在走姐面.
射博,politics,賺好處.
這些東西裡面.
所以真正的問題是.
你是怎樣的長大.
你是一個怎樣的大人.
最近我聽一個曾經做保險的弟兄分享.
他說保險裡面有個行家的說法.
就是一年保險三年人.
不知道大家有沒有聽過.
一年保險三年人.
這句說話是什麼意思.
即是當你做了一年保險之後.
你所經驗到的人生百態.
做人處事,言談技巧.
豐富了,厲害了.
你那個人的誠苦深了.
說話圓滑了.
但是這樣是不是真的叫做成長呢.
即是一年保險三年人.
他做了保險兩年.
做了六年人.
最後他就不做了.
看不順,看不過眼.
不是做自己.
天之末日我們做了這麼多年人.
我們作為一個大人.
比起小時候的你.
你想一下你自己有什麼不同了.
說話自圓其說了.
懂得隱藏自己多了.
公關笑容漂亮了.

$^{321}$推銷自己推銷得多了.
吹吹吹得厲害了.
究竟你是圓滑還是虛偽.
自私還是自愛.
懂得疼愛自己.
還是過分放縱自己.
我們作為大人.
我們經常都有一句很厲害的話.
小朋友才有選擇.
問題是不願選擇的你.
究竟是怎樣選擇做一個大人呢.
我知道女兒今年九歲.
基本上她發覺她的成熟程度.
比我們更吃驚.
一年前我才分享過她踩滾輪.
撲噴牙的照片.
今年她是一個少女.
以前她喜歡粉紅色.
粉紫色的冰冰珠片的衣服.
Unicorn那些.
今年她買新衣服已經買黑色的.
完全是兩個不同的品味.
當然隨著年紀的長大.
都看到很多少女的煩惱衝擊.
不滿憤怒陰謀論.
有一次她回來說.
老師說謊騙我們.
原來今天的活動取消了.
她就一群人陰謀論.
老師騙她們.
不偏心.
小女孩那些.
我自己是這樣祈禱的.
求神叫她不要踩得太快.
主要踩得對.
踩得對比踩得快更重要.
寧願保持少許天真單純.
都不要踩得太快.
今天我們的講題叫.
大人者不失其赤子之心者也.

$^{361}$一個可能大家過份文就咒的講題.
大人者不失其赤子之心者也.
一句出自萬子.
即是尼留篇的一句說話.
文字的說話是什麼意思呢.
他說所謂大人.
就是一個仍然沒有丟棄赤子之心的人.
一個某方面仍然像一個小朋友的人.
不過我要下一個注釋.
其實這裡所說的所謂大人.
不是一般的大人.
所謂大人是一些能夠成就大事的人.
一些偉大的人.
即是中文法所說的君子.
用我們基督教的信仰來說.
就是一個上帝所喜悅的人.
並不是每一個大人都是真正的大人.
並不是每一個長大的人都是神所喜悅的人.
詩篇其實正正是教我們怎樣做一個大人.
雖然詩篇不斷對歲月年紀成長控訴.
不過詩篇裡最後一句說話.
教導我們怎樣做一個大人.
最後一句說話12節說.
詩人說求你指教我們怎樣數算自己的日子.
好叫我們得著智慧的心.
求你指教我們怎樣數算自己的日子.
好叫我們得著智慧的心.
怎樣數算自己的日子.
為何詩人要求主去教導他.
數算自己的日子.
怎樣才算數日子.
聽到一個很不爭的事實.
就是我們每天都在長大.
你今天度過了一天 你就長大多一天.
你的生命過多了一天 你就長大多一天.
這個說話比非洲每一分鐘有60秒過去更加真實.
一天過了 你就長大了一天.
所以詩人在冥冥道理當中的道理裡.
他去禱告上帝.
叫他可以好好去審查每一天.

$^{401}$審查他怎樣來長大.
我們要很有智慧.
你要懂得怎樣來長大.
不要隨便長大.
一些未能長大的地方.
不要隨便長大.
寧願保留一些笨拙的地方.
那些很笨拙的地方.
正正可能就是你仍然善良.
正義仍然保留的地方.
我寧願笨拙地善良.
也不想成熟地失去自己.
每個人都有自己笨拙的地方.
但不要緊.
寧願保持這份笨拙.
也不要急速地跨過長大.
我想這正正就是哲學心的意思.
我自己不是一個很會說話的人.
口齒不伶俐.
也不懂得說話.
所以有時候我會突然之間輕聲.
會突然之間的空氣.
因為我說話是我笨拙的地方.
不過對我來說.
沉默寡言也好.
空氣也好.
最少我想的是真誠地說話.
也不想言過其實.
也不想純粹討好別人.
我想這正正就是心.
往往就是這些笨拙的東西.
好好地去數算自己怎樣過日子.
好好地去選擇每一天怎樣長大.
它是在你信仰裡面.
拒絕妥協.
拒絕廉價成長的記號.
在你這些尚未成長的地方裡面.
保持著小孩子的狀態.
也不要輕易做一個廉價的大人.
如果你記得我們每次來到去.

$^{441}$寧聖餐的時候.
我們唱同餅同杯的時候.
第七節.
其實當中寫著赤子之心這個字.
赤子之心眼淚盈眶.
讓我講一下我寫這句歌詞的經歷.
當我寫到這句歌詞的時候.
我就開始想第七節最後的一節.
到底要寫什麼呢.
就是49735203.
你知道填廣東話的詞一定要對音.
所以我們就知道.
這49735203要填什麼好.
後來我就出茅招.
上了填詞網.
你知道有個網可以.
你打數字進去.
可以給你很多可能性.
對音的數字.
我就打了這個字.
49735.
那個網就給了我很多對音的四成語.
遠走高飛.
眾志之的馬可福音無功德心.
放虎歸山脆比叉燒.
今次吸煙.
很多這些字出來.
後來我就看到這個很漂亮的字.
赤子之心.
後來我就再加上.
眼淚盈眶這四個字.
赤子之心眼淚盈眶.
正正就是我們將來.
見天父時候的情景.
頂智培揚我們堅持做一個赤子之心的大人.
雖然是有些笨拙.
雖然有些單純.
雖然好像沒什麼好處.
但是當我們在明日再見到我們的上帝的時候.
我們以最珍貴最終極的身份.

$^{481}$天父的小朋友.
天父的孩子.
去迎接我們的天父.
你眼淚盈眶.
因為你知道.
你一生的堅持都是值得的.
你的成長.
都不是徒然的.
\newpage



\section{路加福音 21:25-28-34-36-20231209}
\label{sec:0oiGMpkgXB8}
\textbf{【網上崇拜】超級耶穌基督,驚奇!|路加福音21\_25-28,34-36|20231209 [0oiGMpkgXB8]}
\newline
\newline
連結: \href{https://youtube.com/watch?v=0oiGMpkgXB8}{\texttt{ https://youtube.com/watch?v=0oiGMpkgXB8}} ~~~~ 語音日期: 2023-12-09 
\newline
\newline
\hyperref[sec:lfg8MyM5M04]{\small{< < < PREV SERMON < < <}}
~
\hyperref[sec:index_chronic]{\small{[返順時目]}}
~
\hyperref[sec:index_scriptual]{\small{[返順卷目]}}
~
\hyperref[sec:sKBDQD8UIMg]{\small{> > > NEXT SERMON > > >}}
\newline
\newline
路加福音 21:25-28-34-36-20231209
\newline
\begin{longtable}{cl}
\hline
\hline
章節 & 經文 (和合本修訂版)\\
\hline
21:25 & \begin{tabularx}{0.7\textwidth}{X} 「日月星辰要顯出預兆,地上的邦國也有困苦,因海中波浪的響聲而惶惶不安。 \end{tabularx} \\ \\ \relax
21:26 & \begin{tabularx}{0.7\textwidth}{X} 人想到那要臨到世界的事,就都嚇得魂不附體,因為天上的萬象都要震動。 \end{tabularx} \\ \\ \relax
21:27 & \begin{tabularx}{0.7\textwidth}{X} 那時,他們要看見人子帶著能力和大榮耀駕雲來臨。 \end{tabularx} \\ \\ \relax
21:28 & \begin{tabularx}{0.7\textwidth}{X} 一有這些事,你們就當挺身昂首,因為你們得救贖的日子近了。」 \end{tabularx} \\ \\ \relax
21:29 & \begin{tabularx}{0.7\textwidth}{X} 耶穌對他們講了一個比喻說:「你們看無花果樹和各樣的樹, \end{tabularx} \\ \\ \relax
21:30 & \begin{tabularx}{0.7\textwidth}{X} 樹葉一長出來,你們看了自然就知道夏天近了。 \end{tabularx} \\ \\ \relax
21:31 & \begin{tabularx}{0.7\textwidth}{X} 同樣,當你們看見這些事發生,就知道神的國近了。 \end{tabularx} \\ \\ \relax
21:32 & \begin{tabularx}{0.7\textwidth}{X} 我實在告訴你們,這世代還沒有過去,一切都要發生。 \end{tabularx} \\ \\ \relax
21:33 & \begin{tabularx}{0.7\textwidth}{X} 天地要廢去,我的話卻絕不廢去。」 \end{tabularx} \\ \\ \relax
21:34 & \begin{tabularx}{0.7\textwidth}{X} 「你們要謹慎,免得被貪食、醉酒和今生的憂慮壓住你們的心,那日子就忽然臨到你們, \end{tabularx} \\ \\ \relax
21:35 & \begin{tabularx}{0.7\textwidth}{X} 如同羅網一樣,因為那日子要臨到所有居住在地面上的人。 \end{tabularx} \\ \\ \relax
21:36 & \begin{tabularx}{0.7\textwidth}{X} 你們要時時警醒,常常祈求,使你們能逃避這一切要來的事,得以站立在人子面前。」 \end{tabularx} \\ \\ \relax
21:37 & \begin{tabularx}{0.7\textwidth}{X} 耶穌每日在聖殿裡教導人,每夜出城到橄欖山住宿。 \end{tabularx} \\ \\ \relax
21:38 & \begin{tabularx}{0.7\textwidth}{X} 眾百姓清早上聖殿,到耶穌那裡聽他講道。 \end{tabularx} \\ \\
[1ex]
\hline
\hline
\end{longtable}
$^{1}$我只想知道.
你到底是什麼意思.
我只想知道.
你到底是什麼意思.
我只想知道.
你到底是什麼意思.
我只想知道.
你到底是什麼意思.
我只想知道.
你到底是什麼意思.
我只想知道.
你到底是什麼意思.
我只想知道.
你到底是什麼意思.
我只想知道.
你到底是什麼意思.
我只想知道.
你到底是什麼意思.
我只想知道.
你到底是什麼意思.
我只想知道.
你到底是什麼意思.
我只想知道.
你到底是什麼意思.
我只想知道.
你到底是什麼意思.
我只想知道.
你到底是什麼意思.
我只想知道.
你到底是什麼意思.
我只想知道.
你到底是什麼意思.
我只想知道.
你到底是什麼意思.
我只想知道.
你到底是什麼意思.
我只想知道.
你到底是什麼意思.
我只想知道.
你到底是什麼意思.

$^{41}$我只想知道.
你到底是什麼意思.
我只想知道.
你到底是什麼意思.
我只想知道.
你到底是什麼意思.
我只想知道.
你到底是什麼意思.
我只想知道.
你到底是什麼意思.
我只想知道.
你到底是什麼意思.
我只想知道.
你到底是什麼意思.
我只想知道.
你到底是什麼意思.
我只想知道.
你到底是什麼意思.
我只想知道.
你到底是什麼意思.
我只想知道.
你到底是什麼意思.
我只想知道.
你到底是什麼意思.
我只想知道.
你到底是什麼意思.
我只想知道.
你到底是什麼意思.
我只想知道.
你到底是什麼意思.
我只想知道.
你到底是什麼意思.
我只想知道.
你到底是什麼意思.
我只想知道.
你到底是什麼意思.
我只想知道.
你到底是什麼意思.
我只想知道.
你到底是什麼意思.

$^{81}$我只想知道.
你到底是什麼意思.
我只想知道.
你到底是什麼意思.
我只想知道.
你到底是什麼意思.
我只想知道.
你到底是什麼意思.
我只想知道.
你到底是什麼意思.
我只想知道.
你到底是什麼意思.
我只想知道.
你到底是什麼意思.
我只想知道.
你到底是什麼意思.
我只想知道.
你到底是什麼意思.
我只想知道.
你到底是什麼意思.
我只想知道.
你到底是什麼意思.
我只想知道.
你到底是什麼意思.
我只想知道.
你到底是什麼意思.
我只想知道.
你到底是什麼意思.
我只想知道.
你到底是什麼意思.
我只想知道.
你到底是什麼意思.
我只想知道.
你到底是什麼意思.
我只想知道.
你到底是什麼意思.
我只想知道.
你到底是什麼意思.
我只想知道.
你到底是什麼意思.

$^{121}$我只想知道.
你到底是什麼意思.
我只想知道.
你到底是什麼意思.
我只想知道.
你到底是什麼意思.
我只想知道.
你到底是什麼意思.
我只想知道.
你到底是什麼意思.
我只想知道.
你到底是什麼意思.
我只想知道.
你到底是什麼意思.
我只想知道.
你到底是什麼意思.
我只想知道.
你到底是什麼意思.
我只想知道.
你到底是什麼意思.
我只想知道.
你到底是什麼意思.
我只想知道.
你到底是什麼意思.
我只想知道.
你到底是什麼意思.
我只想知道.
你到底是什麼意思.
我只想知道.
你到底是什麼意思.
我只想知道.
你到底是什麼意思.
我只想知道.
你到底是什麼意思.
我只想知道.
你到底是什麼意思.
我只想知道.
你到底是什麼意思.
我只想知道.
你到底是什麼意思.

$^{161}$我只想知道.
你到底是什麼意思.
我只想知道.
你到底是什麼意思.
我只想知道.
你到底是什麼意思.
我只想知道.
你到底是什麼意思.
我只想知道.
你到底是什麼意思.
我只想知道.
你到底是什麼意思.
我只想知道.
你到底是什麼意思.
我只想知道.
你到底是什麼意思.
我只想知道.
你到底是什麼意思.
我只想知道.
你到底是什麼意思.
我只想知道.
你到底是什麼意思.
我只想知道.
你到底是什麼意思.
我只想知道.
你到底是什麼意思.
我只想知道.
你到底是什麼意思.
我只想知道.
你到底是什麼意思.
我只想知道.
你到底是什麼意思.
我只想知道.
你到底是什麼意思.
我只想知道.
你到底是什麼意思.
我只想知道.
你到底是什麼意思.
我只想知道.
你到底是什麼意思.

$^{201}$我只想知道.
你到底是什麼意思.
我只想知道.
你到底是什麼意思.
我只想知道.
你到底是什麼意思.
我只想知道.
你到底是什麼意思.
我只想知道.
你到底是什麼意思.
我只想知道.
你到底是什麼意思.
我只想知道.
你到底是什麼意思.
我只想知道.
你到底是什麼意思.
我只想知道.
你到底是什麼意思.
我只想知道.
你到底是什麼意思.
我只想知道.
你到底是什麼意思.
我只想知道.
你到底是什麼意思.
我只想知道.
你到底是什麼意思.
我只想知道.
你到底是什麼意思.
我只想知道.
你到底是什麼意思.
我只想知道.
你到底是什麼意思.
我只想知道.
你到底是什麼意思.
我只想知道.
你到底是什麼意思.
我只想知道.
你到底是什麼意思.
我只想知道.
你到底是什麼意思.

$^{241}$我只想知道.
你到底是什麼意思.
我只想知道.
你到底是什麼意思.
我只想知道.
你到底是什麼意思.
我只想知道.
你到底是什麼意思.
我只想知道.
你到底是什麼意思.
我只想知道.
你到底是什麼意思.
我只想知道.
你到底是什麼意思.
我只想知道.
你到底是什麼意思.
我只想知道.
你到底是什麼意思.
我只想知道.
你到底是什麼意思.
我只想知道.
你到底是什麼意思.
我只想知道.
你到底是什麼意思.
我只想知道.
你到底是什麼意思.
我只想知道.
你到底是什麼意思.
我只想知道.
你到底是什麼意思.
我只想知道.
你到底是什麼意思.
我只想知道.
你到底是什麼意思.
我只想知道.
你到底是什麼意思.
我只想知道.
你到底是什麼意思.
我只想知道.
你到底是什麼意思.

$^{281}$我只想知道.
你到底是什麼意思.
我只想知道.
你到底是什麼意思.
我只想知道.
你到底是什麼意思.
我只想知道.
你到底是什麼意思.
我只想知道.
你到底是什麼意思.
我只想知道.
你到底是什麼意思.
我只想知道.
你到底是什麼意思.
我只想知道.
你到底是什麼意思.
我只想知道.
你到底是什麼意思.
我只想知道.
你到底是什麼意思.
我只想知道.
你到底是什麼意思.
我只想知道.
你到底是什麼意思.
我只想知道.
你到底是什麼意思.
我只想知道.
你到底是什麼意思.
我只想知道.
你到底是什麼意思.
我只想知道.
你到底是什麼意思.
我只想知道.
你到底是什麼意思.
我只想知道.
你到底是什麼意思.
我只想知道.
你到底是什麼意思.
我只想知道.
你到底是什麼意思.

$^{321}$我只想知道.
你到底是什麼意思.
我只想知道.
你到底是什麼意思.
我只想知道.
你到底是什麼意思.
我只想知道.
你到底是什麼意思.
我只想知道.
你到底是什麼意思.
我只想知道.
你到底是什麼意思.
我只想知道.
你到底是什麼意思.
我只想知道.
你到底是什麼意思.
我只想知道.
你到底是什麼意思.
我只想知道.
你到底是什麼意思.
我只想知道.
你到底是什麼意思.
我只想知道.
你到底是什麼意思.
我只想知道.
你到底是什麼意思.
我只想知道.
你到底是什麼意思.
我只想知道.
你到底是什麼意思.
我只想知道.
你到底是什麼意思.
我只想知道.
你到底是什麼意思.
我只想知道.
你到底是什麼意思.
我只想知道.
你到底是什麼意思.
我只想知道.
你到底是什麼意思.

$^{361}$我只想知道.
你到底是什麼意思.
我只想知道.
你到底是什麼意思.
我只想知道.
你到底是什麼意思.
我只想知道.
你到底是什麼意思.
我只想知道.
你到底是什麼意思.
我只想知道.
你到底是什麼意思.
我只想知道.
你到底是什麼意思.
我只想知道.
你到底是什麼意思.
我只想知道.
你到底是什麼意思.
我只想知道.
你到底是什麼意思.
我只想知道.
你到底是什麼意思.
我只想知道.
你到底是什麼意思.
我只想知道.
你到底是什麼意思.
我只想知道.
你到底是什麼意思.
我只想知道.
你到底是什麼意思.
我只想知道.
你到底是什麼意思.
我只想知道.
你到底是什麼意思.
我只想知道.
你到底是什麼意思.
我只想知道.
你到底是什麼意思.
我只想知道.
你到底是什麼意思.

$^{401}$我只想知道.
你到底是什麼意思.
我只想知道.
你到底是什麼意思.
我只想知道.
你到底是什麼意思.
我只想知道.
你到底是什麼意思.
我只想知道.
你到底是什麼意思.
我只想知道.
你到底是什麼意思.
我只想知道.
你到底是什麼意思.
我只想知道.
你到底是什麼意思.
我只想知道.
你到底是什麼意思.
我只想知道.
你到底是什麼意思.
我只想知道.
你到底是什麼意思.
我只想知道.
你到底是什麼意思.
我只想知道.
你到底是什麼意思.
我只想知道.
你到底是什麼意思.
我只想知道.
你到底是什麼意思.
我只想知道.
你到底是什麼意思.
我只想知道.
你到底是什麼意思.
我只想知道.
你到底是什麼意思.
我只想知道.
你到底是什麼意思.
我只想知道.
你到底是什麼意思.

$^{441}$我只想知道.
你到底是什麼意思.
我只想知道.
你到底是什麼意思.
我只想知道.
你到底是什麼意思.
我只想知道.
你到底是什麼意思.
我只想知道.
你到底是什麼意思.
我只想知道.
你到底是什麼意思.
我只想知道.
你到底是什麼意思.
我只想知道.
你到底是什麼意思.
我只想知道.
你到底是什麼意思.
我只想知道.
你到底是什麼意思.
我只想知道.
你到底是什麼意思.
我只想知道.
你到底是什麼意思.
我只想知道.
你到底是什麼意思.
我只想知道.
你到底是什麼意思.
我只想知道.
你到底是什麼意思.
我只想知道.
你到底是什麼意思.
我只想知道.
你到底是什麼意思.
我只想知道.
你到底是什麼意思.
我只想知道.
你到底是什麼意思.
我只想知道.
你到底是什麼意思.

$^{481}$我只想知道.
你到底是什麼意思.
我只想知道.
你到底是什麼意思.
我只想知道.
你到底是什麼意思.
我只想知道.
你到底是什麼意思.
我只想知道.
你到底是什麼意思.
我只想知道.
你到底是什麼意思.
我只想知道.
你到底是什麼意思.
我只想知道.
你到底是什麼意思.
我只想知道.
你到底是什麼意思.
我只想知道.
你到底是什麼意思.
我只想知道.
你到底是什麼意思.
我只想知道.
你到底是什麼意思.
我只想知道.
你到底是什麼意思.
我只想知道.
你到底是什麼意思.
我只想知道.
你到底是什麼意思.
我只想知道.
你到底是什麼意思.
我只想知道.
你到底是什麼意思.
我只想知道.
你到底是什麼意思.
我只想知道.
你到底是什麼意思.
我只想知道.
你到底是什麼意思.

$^{521}$我只想知道.
你到底是什麼意思.
我只想知道.
你到底是什麼意思.
我只想知道.
你到底是什麼意思.
我只想知道.
你到底是什麼意思.
我只想知道.
你到底是什麼意思.
我只想知道.
你到底是什麼意思.
我只想知道.
你到底是什麼意思.
我只想知道.
你到底是什麼意思.
我只想知道.
你到底是什麼意思.
我只想知道.
你到底是什麼意思.
我只想知道.
你到底是什麼意思.
我只想知道.
你到底是什麼意思.
我只想知道.
你到底是什麼意思.
我只想知道.
你到底是什麼意思.
我只想知道.
你到底是什麼意思.
我只想知道.
你到底是什麼意思.
我只想知道.
你到底是什麼意思.
我只想知道.
你到底是什麼意思.
我只想知道.
你到底是什麼意思.
我只想知道.
你到底是什麼意思.

$^{561}$我只想知道.
你到底是什麼意思.
我只想知道.
你到底是什麼意思.
我只想知道.
你到底是什麼意思.
我只想知道.
你到底是什麼意思.
我只想知道.
你到底是什麼意思.
我只想知道.
你到底是什麼意思.
我只想知道.
你到底是什麼意思.
我只想知道.
你到底是什麼意思.
我只想知道.
你到底是什麼意思.
我只想知道.
你到底是什麼意思.
我只想知道.
你到底是什麼意思.
我只想知道.
你到底是什麼意思.
我只想知道.
你到底是什麼意思.
我只想知道.
你到底是什麼意思.
我只想知道.
你到底是什麼意思.
我只想知道.
你到底是什麼意思.
我只想知道.
你到底是什麼意思.
我只想知道.
你到底是什麼意思.
我只想知道.
你到底是什麼意思.
我只想知道.
你到底是什麼意思.

$^{601}$我只想知道.
你到底是什麼意思.
我只想知道.
你到底是什麼意思.
我只想知道.
你到底是什麼意思.
我只想知道.
你到底是什麼意思.
我只想知道.
你到底是什麼意思.
我只想知道.
你到底是什麼意思.
我只想知道.
你到底是什麼意思.
我只想知道.
你到底是什麼意思.
我只想知道.
你到底是什麼意思.
我只想知道.
你到底是什麼意思.
我只想知道.
你到底是什麼意思.
我只想知道.
你到底是什麼意思.
我只想知道.
你到底是什麼意思.
我只想知道.
你到底是什麼意思.
我只想知道.
你到底是什麼意思.
我只想知道.
你到底是什麼意思.
我只想知道.
你到底是什麼意思.
我只想知道.
你到底是什麼意思.
我只想知道.
你到底是什麼意思.
我只想知道.
你到底是什麼意思.

$^{641}$我只想知道.
你到底是什麼意思.
我只想知道.
你到底是什麼意思.
我只想知道.
你到底是什麼意思.
我只想知道.
你到底是什麼意思.
我只想知道.
你到底是什麼意思.
我只想知道.
你到底是什麼意思.
我只想知道.
你到底是什麼意思.
我只想知道.
你到底是什麼意思.
我只想知道.
你到底是什麼意思.
我只想知道.
你到底是什麼意思.
我只想知道.
你到底是什麼意思.
我只想知道.
你到底是什麼意思.
我只想知道.
你到底是什麼意思.
我只想知道.
你到底是什麼意思.
我只想知道.
你到底是什麼意思.
我只想知道.
你到底是什麼意思.
我只想知道.
你到底是什麼意思.
我只想知道.
你到底是什麼意思.
我只想知道.
你到底是什麼意思.
我只想知道.
你到底是什麼意思.

$^{681}$我只想知道.
你到底是什麼意思.
我只想知道.
你到底是什麼意思.
我只想知道.
你到底是什麼意思.
我只想知道.
你到底是什麼意思.
我只想知道.
你到底是什麼意思.
我只想知道.
你到底是什麼意思.
我只想知道.
你到底是什麼意思.
我只想知道.
你到底是什麼意思.
我只想知道.
你到底是什麼意思.
我只想知道.
你到底是什麼意思.
我只想知道.
你到底是什麼意思.
我只想知道.
你到底是什麼意思.
我只想知道.
你到底是什麼意思.
我只想知道.
你到底是什麼意思.
我只想知道.
你到底是什麼意思.
我只想知道.
你到底是什麼意思.
我只想知道.
你到底是什麼意思.
我只想知道.
你到底是什麼意思.
我只想知道.
你到底是什麼意思.
我只想知道.
你到底是什麼意思.

$^{721}$我只想知道.
你到底是什麼意思.
我只想知道.
你到底是什麼意思.
我只想知道.
你到底是什麼意思.
我只想知道.
你到底是什麼意思.
我只想知道.
你到底是什麼意思.
我只想知道.
你到底是什麼意思.
我只想知道.
你到底是什麼意思.
我只想知道.
你到底是什麼意思.
我只想知道.
你到底是什麼意思.
我只想知道.
你到底是什麼意思.
我只想知道.
你到底是什麼意思.
我只想知道.
你到底是什麼意思.
我只想知道.
你到底是什麼意思.
我只想知道.
你到底是什麼意思.
我只想知道.
你到底是什麼意思.
我只想知道.
你到底是什麼意思.
我只想知道.
你到底是什麼意思.
我只想知道.
你到底是什麼意思.
我只想知道.
你到底是什麼意思.
我只想知道.
你到底是什麼意思.

$^{761}$我只想知道.
你到底是什麼意思.
我只想知道.
你到底是什麼意思.
我只想知道.
你到底是什麼意思.
我只想知道.
你到底是什麼意思.
我只想知道.
你到底是什麼意思.
我只想知道.
你到底是什麼意思.
我只想知道.
你到底是什麼意思.
我只想知道.
你到底是什麼意思.
我只想知道.
你到底是什麼意思.
我只想知道.
你到底是什麼意思.
我只想知道.
你到底是什麼意思.
我只想知道.
你到底是什麼意思.
我只想知道.
你到底是什麼意思.
我只想知道.
你到底是什麼意思.
我只想知道.
你到底是什麼意思.
我只想知道.
你到底是什麼意思.
我只想知道.
你到底是什麼意思.
我只想知道.
你到底是什麼意思.
我只想知道.
你到底是什麼意思.
我只想知道.
你到底是什麼意思.

$^{801}$我只想知道.
你到底是什麼意思.
我只想知道.
你到底是什麼意思.
我只想知道.
你到底是什麼意思.
我只想知道.
你到底是什麼意思.
我只想知道.
你到底是什麼意思.
我只想知道.
你到底是什麼意思.
我只想知道.
你到底是什麼意思.
我只想知道.
你到底是什麼意思.
我只想知道.
你到底是什麼意思.
我只想知道.
你到底是什麼意思.
我只想知道.
你到底是什麼意思.
我只想知道.
你到底是什麼意思.
我只想知道.
你到底是什麼意思.
我只想知道.
你到底是什麼意思.
我只想知道.
你到底是什麼意思.
我只想知道.
你到底是什麼意思.
我只想知道.
你到底是什麼意思.
我只想知道.
你到底是什麼意思.
我只想知道.
你到底是什麼意思.
我只想知道.
你到底是什麼意思.

$^{841}$我只想知道.
你到底是什麼意思.
我只想知道.
你到底是什麼意思.
我只想知道.
你到底是什麼意思.
我只想知道.
你到底是什麼意思.
我只想知道.
你到底是什麼意思.
我只想知道.
你到底是什麼意思.
我只想知道.
你到底是什麼意思.
我只想知道.
你到底是什麼意思.
我只想知道.
你到底是什麼意思.
我只想知道.
你到底是什麼意思.
我只想知道.
你到底是什麼意思.
我只想知道.
你到底是什麼意思.
我只想知道.
你到底是什麼意思.
我只想知道.
你到底是什麼意思.
我只想知道.
你到底是什麼意思.
我只想知道.
你到底是什麼意思.
我只想知道.
你到底是什麼意思.
我只想知道.
你到底是什麼意思.
我只想知道.
你到底是什麼意思.
我只想知道.
你到底是什麼意思.

$^{881}$我只想知道.
你到底是什麼意思.
我只想知道.
你到底是什麼意思.
我只想知道.
你到底是什麼意思.
我只想知道.
你到底是什麼意思.
我只想知道.
你到底是什麼意思.
我只想知道.
你到底是什麼意思.
我只想知道.
你到底是什麼意思.
我只想知道.
你到底是什麼意思.
我只想知道.
你到底是什麼意思.
我只想知道.
你到底是什麼意思.
我只想知道.
你到底是什麼意思.
我只想知道.
你到底是什麼意思.
我只想知道.
你到底是什麼意思.
我只想知道.
你到底是什麼意思.
我只想知道.
你到底是什麼意思.
我只想知道.
你到底是什麼意思.
我只想知道.
你到底是什麼意思.
我只想知道.
你到底是什麼意思.
我只想知道.
你到底是什麼意思.
我只想知道.
你到底是什麼意思.

$^{921}$我只想知道.
你到底是什麼意思.
我只想知道.
你到底是什麼意思.
我只想知道.
你到底是什麼意思.
我只想知道.
你到底是什麼意思.
我要富豪得主.
家力前行.
就算遇到風浪.
不需抖震.
就算遇到困惑事.
並有著你的火柱.
困處忍耐.
我要跟你走出.
各個地域.
流淚與不理應.
去一生共贏.
看最廣闊的風景.
盡到你榮耀的國.
看最廣闊.
看最廣闊的風景.
盡到你榮耀的國.
看最廣闊.
看最廣闊的風景.
盡到你榮耀的國.
看最廣闊.
看最廣闊的風景.
盡到你榮耀的國.
看最廣闊.
看最廣闊的風景.
盡到你榮耀的國.
看最廣闊.
看最廣闊的風景.
盡到你榮耀的國.
看最廣闊.
看最廣闊的風景.
盡到你榮耀的國.
看最廣闊.

$^{961}$看最廣闊的風景.
盡到你榮耀的國.
看最廣闊.
看最廣闊的風景.
盡到你榮耀的國.
看最廣闊.
看最廣闊的風景.
盡到你榮耀的國.
看最廣闊.
看最廣闊的風景.
看最廣闊.
看最廣闊的風景.
盡到你榮耀的國.
看最廣闊.
看最廣闊的風景.
盡到你榮耀的國.
看最廣闊.
看最廣闊的風景.
盡到你榮耀的國.
看最廣闊.
看最廣闊的風景.
盡到你榮耀的國.
看最廣闊.
看最廣闊的風景.
盡到你榮耀的國.
看最廣闊.
看最廣闊的風景.
盡到你榮耀的國.
看最廣闊.
看最廣闊的風景.
盡到你榮耀的國.
看最廣闊.
看最廣闊的風景.
盡到你榮耀的國.
看最廣闊.
看最廣闊的風景.
盡到你榮耀的國.
看最廣闊.
看最廣闊的風景.
盡到你榮耀的國.

$^{1001}$看最廣闊.
看最廣闊的風景.
盡到你榮耀的國.
看最廣闊.
看最廣闊的風景.
盡到你榮耀的國.
看最廣闊.
看最廣闊的風景.
盡到你榮耀的國.
看最廣闊.
看最廣闊的風景.
盡到你榮耀的國.
看最廣闊的風景.
盡到你榮耀的國.
看最廣闊的風景.
盡到你榮耀的國.
看最廣闊的風景.
盡到你榮耀的國.
看最廣闊的風景.
盡到你榮耀的國.
看最廣闊的風景.
盡到你榮耀的國.
看最廣闊的風景.
盡到你榮耀的國.
看最廣闊的風景.
盡到你榮耀的國.
看最廣闊的風景.
盡到你榮耀的國.
看最廣闊的風景.
盡到你榮耀的國.
看最廣闊的風景.
盡到你榮耀的國.
看最廣闊的風景.
盡到你榮耀的國.
看最廣闊的風景.
盡到你榮耀的國.
看最廣闊的風景.
盡到你榮耀的國.
看最廣闊的風景.
盡到你榮耀的國.

$^{1041}$看最廣闊的風景.
盡到你榮耀的國.
看最廣闊的風景.
盡到你榮耀的國.
看最廣闊的風景.
盡到你榮耀的國.
看最廣闊的風景.
盡到你榮耀的國.
看最廣闊的風景.
盡到你榮耀的國.
看最廣闊的風景.
盡到你榮耀的國.
看最廣闊的風景.
盡到你榮耀的國.
看最廣闊的風景.
盡到你榮耀的國.
看最廣闊的風景.
盡到你榮耀的國.
看最廣闊的風景.
盡到你榮耀的國.
看最廣闊的風景.
盡到你榮耀的國.
看最廣闊的風景.
盡到你榮耀的國.
看最廣闊的風景.
盡到你榮耀的國.
看最廣闊的風景.
盡到你榮耀的國.
看最廣闊的風景.
盡到你榮耀的國.
看最廣闊的風景.
盡到你榮耀的國.
看最廣闊的風景.
盡到你榮耀的國.
看最廣闊的風景.
盡到你榮耀的國.
看最廣闊的風景.
盡到你榮耀的國.
看最廣闊的風景.
盡到你榮耀的國.

$^{1081}$看最廣闊的風景.
盡到你榮耀的國.
看最廣闊的風景.
盡到你榮耀的國.
看最廣闊的風景.
盡到你榮耀的國.
看最廣闊的風景.
盡到你榮耀的國.
看最廣闊的風景.
盡到你榮耀的國.
看最廣闊的風景.
盡到你榮耀的國.
看最廣闊的風景.
盡到你榮耀的國.
看最廣闊的風景.
盡到你榮耀的國.
看最廣闊的風景.
盡到你榮耀的國.
看最廣闊的風景.
盡到你榮耀的國.
看最廣闊的風景.
盡到你榮耀的國.
看最廣闊的風景.
盡到你榮耀的國.
看最廣闊的風景.
盡到你榮耀的國.
看最廣闊的風景.
盡到你榮耀的國.
看最廣闊的風景.
盡到你榮耀的國.
看最廣闊的風景.
盡到你榮耀的國.
以跳舞代替沮喪.
在絕地綻放希望.
WOW.
你要恨歧視.
會暗著放膽宣告.
在活後亦有指望.
WOW.
You are my miracle.

$^{1121}$WOW.
You are my miracle.
恨歧視的神.
充滿著大能.
世界的光暗.
也在你手.
神能令祂近.
神未必降臨.
斷開鎖鏈解開毒咒.
來看守於我.
濾著渴睡的生命.
看到神永遠不曾於性命.
自食掠奪的本性.
以跳舞代替沮喪.
在絕地綻放希望.
WOW.
你要恨歧視.
會暗著放膽宣告.
在活後亦有指望.
WOW.
You are my miracle.
以跳舞代替沮喪.
在絕地綻放希望.
WOW.
你要恨歧視.
會暗著放膽宣告.
在活後亦有指望.
WOW.
You are my miracle.
來看守於我.
濾著渴睡的生命.
看到神永遠不曾於性命.
自食掠奪的本性.
以跳舞代替沮喪.
在絕地綻放希望.
WOW.
你要恨歧視.
會暗著放膽宣告.
在活後亦有指望.
WOW.

$^{1161}$You are my miracle.
以跳舞代替沮喪.
在絕地綻放希望.
WOW.
你要恨歧視.
會暗著放膽宣告.
在活後亦有指望.
WOW.
You are my miracle.
我一齊宣告.
You are my miracle.
所以呢佢就係我地Miracle.
就係我地Superhero.
我地繼續停留.
墮落係一個勁派當中.
我地去高舉佢ge 榮耀.
我地繼續停留.
墮落係一個勁派當中.
我地繼續停留.
墮落係一個勁派當中.
我地繼續停留.
墮落係一個勁派當中.
我地繼續停留.
墮落係一個勁派當中.
我地繼續停留.
墮落係一個勁派當中.
我地繼續停留.
墮落係一個勁派當中.
我地繼續停留.
墮落係一個勁派當中.
我地繼續停留.
墮落係一個勁派當中.
我地繼續停留.
墮落係一個勁派當中.
我地繼續停留.
墮落係一個勁派當中.
我地繼續停留.
墮落係一個勁派當中.
我地繼續停留.
墮落係一個勁派當中.

$^{1201}$我地繼續停留.
墮落係一個勁派當中.
我地繼續停留.
墮落係一個勁派當中.
我地繼續停留.
墮落係一個勁派當中.
我地繼續停留.
墮落係一個勁派當中.
我地繼續停留.
墮落係一個勁派當中.
我地繼續停留.
墮落係一個勁派當中.
我地繼續停留.
墮落係一個勁派當中.
我地繼續停留.
墮落係一個勁派當中.
我地繼續停留.
墮落係一個勁派當中.
我地繼續停留.
墮落係一個勁派當中.
我地繼續停留.
墮落係一個勁派當中.
我地繼續停留.
墮落係一個勁派當中.
我地繼續停留.
墮落係一個勁派當中.
我地繼續停留.
墮落係一個勁派當中.
我地繼續停留.
墮落係一個勁派當中.
我地繼續停留.
墮落係一個勁派當中.
我地繼續停留.
墮落係一個勁派當中.
我地繼續停留.
墮落係一個勁派當中.
我地繼續停留.
墮落係一個勁派當中.
我地繼續停留.
墮落係一個勁派當中.

$^{1241}$我地繼續停留.
墮落係一個勁派當中.
我地繼續停留.
墮落係一個勁派當中.
我地繼續停留.
墮落係一個勁派當中.
我地繼續停留.
墮落係一個勁派當中.
我地繼續停留.
墮落係一個勁派當中.
我地繼續停留.
墮落係一個勁派當中.
我地繼續停留.
墮落係一個勁派當中.
我地繼續停留.
墮落係一個勁派當中.
我地繼續停留.
墮落係一個勁派當中.
我地繼續停留.
墮落係一個勁派當中.
我地繼續停留.
墮落係一個勁派當中.
我地繼續停留.
墮落係一個勁派當中.
我地繼續停留.
墮落係一個勁派當中.
我地繼續停留.
墮落係一個勁派當中.
我地繼續停留.
墮落係一個勁派當中.
我地繼續停留.
墮落係一個勁派當中.
我地繼續停留.
墮落係一個勁派當中.
我地繼續停留.
墮落係一個勁派當中.
我地繼續停留.
墮落係一個勁派當中.
我地繼續停留.
墮落係一個勁派當中.

$^{1281}$我地繼續停留.
墮落係一個勁派當中.
我地繼續停留.
墮落係一個勁派當中.
我地繼續停留.
墮落係一個勁派當中.
我地繼續停留.
墮落係一個勁派當中.
我地繼續停留.
墮落係一個勁派當中.
我地繼續停留.
墮落係一個勁派當中.
我地繼續停留.
墮落係一個勁派當中.
我地繼續停留.
墮落係一個勁派當中.
我地繼續停留.
墮落係一個勁派當中.
我地繼續停留.
墮落係一個勁派當中.
我地繼續停留.
墮落係一個勁派當中.
我地繼續停留.
墮落係一個勁派當中.
我地繼續停留.
墮落係一個勁派當中.
我地繼續停留.
墮落係一個勁派當中.
我地繼續停留.
墮落係一個勁派當中.
我地繼續停留.
墮落係一個勁派當中.
我地繼續停留.
墮落係一個勁派當中.
我地繼續停留.
墮落係一個勁派當中.
我地繼續停留.
墮落係一個勁派當中.
我地繼續停留.
墮落係一個勁派當中.

$^{1321}$我地繼續停留.
墮落係一個勁派當中.
我地繼續停留.
墮落係一個勁派當中.
我地繼續停留.
墮落係一個勁派當中.
我地繼續停留.
墮落係一個勁派當中.
我地繼續停留.
墮落係一個勁派當中.
我地繼續停留.
墮落係一個勁派當中.
我地繼續停留.
墮落係一個勁派當中.
我地繼續停留.
墮落係一個勁派當中.
我地繼續停留.
墮落係一個勁派當中.
我地繼續停留.
墮落係一個勁派當中.
我地繼續停留.
墮落係一個勁派當中.
我地繼續停留.
墮落係一個勁派當中.
我地繼續停留.
墮落係一個勁派當中.
我地繼續停留.
墮落係一個勁派當中.
我地繼續停留.
墮落係一個勁派當中.
我地繼續停留.
墮落係一個勁派當中.
我地繼續停留.
墮落係一個勁派當中.
我地繼續停留.
墮落係一個勁派當中.
我地繼續停留.
墮落係一個勁派當中.
我地繼續停留.
墮落係一個勁派當中.

$^{1361}$我地繼續停留.
墮落係一個勁派當中.
我地繼續停留.
墮落係一個勁派當中.
我地繼續停留.
墮落係一個勁派當中.
我地繼續停留.
墮落係一個勁派當中.
我地繼續停留.
墮落係一個勁派當中.
我地繼續停留.
墮落係一個勁派當中.
我地繼續停留.
墮落係一個勁派當中.
我地繼續停留.
墮落係一個勁派當中.
我地繼續停留.
墮落係一個勁派當中.
我地繼續停留.
墮落係一個勁派當中.
我地繼續停留.
墮落係一個勁派當中.
我地繼續停留.
墮落係一個勁派當中.
我地繼續停留.
墮落係一個勁派當中.
我地繼續停留.
墮落係一個勁派當中.
我地繼續停留.
墮落係一個勁派當中.
我地繼續停留.
墮落係一個勁派當中.
我地繼續停留.
墮落係一個勁派當中.
我地繼續停留.
墮落係一個勁派當中.
我地繼續停留.
墮落係一個勁派當中.
我地繼續停留.
墮落係一個勁派當中.

$^{1401}$我地繼續停留.
墮落係一個勁派當中.
我地繼續停留.
墮落係一個勁派當中.
我地繼續停留.
墮落係一個勁派當中.
我地繼續停留.
墮落係一個勁派當中.
我地繼續停留.
墮落係一個勁派當中.
我地繼續停留.
墮落係一個勁派當中.
我地繼續停留.
墮落係一個勁派當中.
我地繼續停留.
墮落係一個勁派當中.
我地繼續停留.
墮落係一個勁派當中.
我地繼續停留.
墮落係一個勁派當中.
我地繼續停留.
墮落係一個勁派當中.
我地繼續停留.
墮落係一個勁派當中.
我地繼續停留.
墮落係一個勁派當中.
我地繼續停留.
墮落係一個勁派當中.
我地繼續停留.
墮落係一個勁派當中.
我地繼續停留.
墮落係一個勁派當中.
我地繼續停留.
墮落係一個勁派當中.
我地繼續停留.
墮落係一個勁派當中.
我地繼續停留.
墮落係一個勁派當中.
我地繼續停留.
墮落係一個勁派當中.

$^{1441}$我地繼續停留.
墮落係一個勁派當中.
我地繼續停留.
墮落係一個勁派當中.
我地繼續停留.
墮落係一個勁派當中.
我地繼續停留.
墮落係一個勁派當中.
我地繼續停留.
墮落係一個勁派當中.
我地繼續停留.
墮落係一個勁派當中.
我地繼續停留.
墮落係一個勁派當中.
我地繼續停留.
墮落係一個勁派當中.
我地繼續停留.
墮落係一個勁派當中.
我地繼續停留.
墮落係一個勁派當中.
我地繼續停留.
墮落係一個勁派當中.
我地繼續停留.
墮落係一個勁派當中.
我地繼續停留.
墮落係一個勁派當中.
我地繼續停留.
墮落係一個勁派當中.
我地繼續停留.
墮落係一個勁派當中.
我地繼續停留.
墮落係一個勁派當中.
我地繼續停留.
墮落係一個勁派當中.
我地繼續停留.
墮落係一個勁派當中.
我地繼續停留.
墮落係一個勁派當中.
我地繼續停留.
墮落係一個勁派當中.

$^{1481}$我地繼續停留.
墮落係一個勁派當中.
我地繼續停留.
墮落係一個勁派當中.
我地繼續停留.
墮落係一個勁派當中.
我地繼續停留.
墮落係一個勁派當中.
我地繼續停留.
墮落係一個勁派當中.
我地繼續停留.
墮落係一個勁派當中.
我地繼續停留.
墮落係一個勁派當中.
我地繼續停留.
墮落係一個勁派當中.
被你所賄 榮耀遍地.
兩節我們一起禱告.
親愛的主 親愛的主.
我們知道.
眾言生活當中.
很多事情我們看到的.
八成 九成.
都是一些黑暗的事情.
但我們知道.
你就是榮耀的主.
你就是審判的主.
將來你會降臨.
將來你會帶來勝利.
因此我們的盼望.
只是在於你身上.
因此我們拒絕.
任何的絕望.
拒絕任何的失望和無力.
因為我們知道.
單單我們尋找你.
單單我們跟隨你的時候.
你就是我們生命中.
唯一的盼望.
而你是真實的盼望.

$^{1521}$我們將這一切的眾讚.
這一切的禱告.
獻情給你.
奉主耶穌基督.
得聖名字.
祈求.
阿們.
弟兄姊妹請坐.
請坐.
敬禮.
請坐.
弟兄姊妹平安.
剛才我很嘮叨.
一首慢歌都不能平靜.
很緊張的心情.
因為今天有兩個崇拜.
一個崇拜在小朋友那邊.
見到很多小朋友都在那裡.
應該在看劇.
有個劇目.
來到的時候就見到.
天使得得B.
見到排球少年.
見到網球王子.
見到小飛俠.
見到很多大家都熟悉的東西.
今天的小朋友崇拜.
在外面有攤位.
都有今天限定的.
今天第一天出的貼紙.
基本上參與過程當中.
你就會有一張貼紙.
希望和大家一起去參與.
在崇拜當中一起投入.
參與之後大家可以留下來聊天.
也可以一起去玩.
今天崇拜當中.
想和大家看看張琳琦的訊息.
可以.
出到沒有.

$^{1561}$是不是要那裡.
好.
不好意思 等一下.
還沒出就先走.
先出.
謝謝.
恰恰就打到我腦袋里了.
我還沒把手放開呢.
章林奇通常在聖誕節.
數頭四周的主日開始.
預備等待耶穌基督的降生.
或者紀念耶穌的降生.
同樣提醒我們.
等待耶穌基督的應許.
祂再一次降臨.
章林奇的經文也有很豐富的.
其中你可能會聽到.
關於耶穌基督再降臨之前.
面對的情況.
或者是一些風雷雨電.
我以前在信主的過程當中.
看三代經題.
包括章林奇的經文的時候.
我就覺得整件事很超級英雄.
因為你看到整個世界.
翻天覆地.
變了天.
環境當中很明顯就知道.
一出這些環境就知道.
是打大佬的.
整件事就應該是很準備.
我不知道你帶著什麼期望.
去看那些超級英雄的戲.
或者是有些人不看的.
但是當我帶著兩個兒子.
去看這些超級英雄的戲的時候.
他們當然很期盼.
我最記得一次他們兩個小時候.
就帶著他們去海運那裡.
你知道去海運的電影院.

$^{1601}$有些凳子會動的.
我兩個兒子都很醒目.
他說爸爸為什麼買三張票這麼貴.
比平時貴那麼多.
我說進去你就會知道.
當然他們很小的時候.
我還要教他們.
我說不行.
很震.
教到一.
一就是沒用.
因為他們小的時候.
大人坐的位置不太對.
所以震的位置對他們來說不好.
最後就是.
其實就是沒震那張凳子.
但不要緊.
一會兒會不會很緊張.
不緊張.
一定贏的.
我不知道你有沒有一個小朋友的心境.
你知道打大佬最後一定是贏的.
你的心態會不會是知道邪不能勝正.
小朋友是這樣想的.
但是當我兒子大了的時候.
他看的戲越多的時候.
他突然出一個結論.
其實沒有主角光環.
沒有主角光環是什麼意思.
主角會死的.
直到他去看Iron Man的時候就知道.
於是回頭Iron Man就完事了.
我想張臨其對於你來說.
我希望今天的信息.
和大家一起去想想.
其實你相信耶穌基督再來嗎.
這是一個很真實的.
在問自己.
你頭腦里知道還是情感會接受呢.
這是我們的信仰很靠近的.

$^{1641}$其實在新約時期的教會的弟兄姊妹.
其實都在懷疑這件事.
懷疑這件事就是.
耶穌是否說得出做得到.
而我們現在的情況.
就真的未如理想.
我們有沒有這個期盼呢.
所以今天我選擇的.
是路加福音的經文.
是關於張臨其.
耶穌基督再來之前的現象.
下一章.
我們一起讀路加福音第21章第一段的經文.
請.
好 謝謝.
你見到剛才那兩節的經文當中.
在路加要提這段經文.
其實是第二次提.
第一次提的時候.
路加福音第17章都說過.
關於人子再來的時候的場景是什麼.
這次是第二次提.
其實路加在記述的過程當中.
跟馬可福音的源頭其實很相似.
好像是在耶穌受難之前提醒監牢門徒.
就是說我再回來的時候.
或者是耶穌在這個世界終結的時候.
會出現什麼兆頭.
那幫門徒都問過.
但在這次也一章再提的時候.
很明顯這件事很重要.
而重要的一點就是.
那件事一定會出現.
而出現的重點就是.
你們如何準備這件事.
再提你如何準備這件事.
其實我不知道你過去教會成長.
你自己讀經的過程當中.
你多看重那件事是否能埋身.
我是一個很實務的人.

$^{1681}$當那件事我要做.
或者我要參與過程當中.
我已經在想我應該如何做那件事.
或者我看到這件事的時候.
我就在想下一步應該如何執行.
那件事對我來說.
我要參與就要知道埋身的程度.
我要出多少力.
或者我要預備多少時間.
就正正好像你知道耶穌會回來.
你知道和你預備過程當中.
你如何準備這件事.
在剛才的經文當中.
好像剛才提到.
可以出下一張.
再下一張.
你見到要看見人子有能力.
其實路加想提醒一件事.
就是我們相信的人子是有能力的.
他能力是讓我們明白到.
他會整頓這個世界.
整件事就好像上次.
在《小孩不懂世界》.
阿廷講的那篇信息的時候.
其實一切已經成定局.
是差在時間何時去執行.
而一切已經是會發生的時候.
我們雖然見到眼前的不是如我們所願的.
但結局都會是如技術一樣.
而重點就是這個人子是有能力.
而這個人子是有什麼能力呢.
而我們如何認識這個能力呢.
所以剛才說路加和馬可很相似的時候.
其實馬可是如何讓我們明白到.
人子的能力在哪裡.
我們下一張.
你會見到在馬可福音第四至六章的時候.
耶穌是這樣將神跡慢慢地彰顯出來.
首先見到他在第二三章的時候.
你見到他醫治了大麻瘋的人.

$^{1721}$人們就知道他有能力慢慢去擁擠他.
然後去到第四章的時候.
或者第五章開始.
你見到耶穌就平靜了風浪.
然後由過了胡之後去到落地的時候.
就能夠趕出那個格拉森的鬼.
之後就醫治了一個十二年血流的女人.
然後再繼續第六章.
就是伍秉儀的神跡.
但其中一件事就是耶穌回到鄉下.
他回到鄉下的時候.
就是回到拿撒勒的時候.
他那群村民見到耶穌的時候.
其實他的反應和我們可能會不同.
當一路走來的時候.
見到耶穌平靜風浪.
趕走污鬼.
醫治血流的人.
餵飽了五千人的時候.
但那群一直和他在玩弄的人.
其實他就說.
他不是木匠的兒子.
他的兄弟和妹妹我都認識.
其實他是誰呢.
我認識了你三十年.
你現在出去轉一轉的時候.
回頭就這樣.
其實他們接受不了.
這個人子耶穌的轉變.
他們卡在這裡.
其實為什麼這個人突然間不同了呢.
就好像剛才Alex說.
你看超級英雄的電影的時候.
其實有很多伏筆.
或者在外面的人是看不到的.
但他什麼時候會有些東西突然間轉變呢.
當他有個使命出現的時候.
他的人設就轉變了.
當然我不是說耶穌就是超級英雄.
那個等同.

$^{1761}$但你要見到.
其實這幾個神跡有他的意思.
我們下一章.
你會見到.
平靜風浪是一個掌管自然.
你會見到.
連風和海也聽從他的時候.
那個情況很有趣.
其實那群門徒都不是很熟悉耶穌.
雖然是跟了耶穌.
但是他們的環境就是.
風浪很大.
做漁夫的都很害怕.
於是就去到船尾.
向著枕頭睡覺的耶穌就說.
夫子我們放送命啊.
你還不顧嗎.
耶穌第一件事就是.
不是罵他們.
不是說.
其實最危險就是船尾.
我睡覺最危險的地方都不怕.
你們怕什麼.
我做木匠.
你們打魚的.
你們見慣風浪.
不是說這些.
耶穌什麼都不跟他們爭拗.
他反而去到船頭就說.
向著風說住了吧.
靜了吧.
風就大大的指住了.
轉過頭對他們說.
你們這些小信的人.
你們還沒有信心嗎.
耶穌在那個環境當中.
你見到整件事一說就停了.
那幫門徒最後說.
這是什麼人呢.
連風和海也聽從他.

$^{1801}$其實他有很多奇妙.
很多驚奇.
其實這個是什麼人.
為什麼有這樣的能力.
可以平靜風浪呢.
而且是立刻停了.
然後船就下到一個地方.
叫加拉森.
下船的時候.
假設是一起下船.
一行十三人下了地的時候.
有個人從墳墓走出來.
就喊叫.
跑到耶穌的根前說.
至高神的兒子.
我與你有何乾呢.
你想想整件事是怎樣.
整件事就是.
十三個人一起下去.
那時沒有照片.
什麼都好.
十三個當中.
如果假設剛才台上.
我們敬拜的十三人.
加上我十四個.
當十三個.
一下去的時候.
叫你找個潘志剛出來.
但你從來沒有見過潘志剛.
你也是十三分之一.
你很大機會錯.
但你看到那個人.
從墳墓走出來的時候.
沒有走到其他人面前.
就是走到耶穌面前.
就說至高神的兒子.
因為鬼認得耶穌.
你看到.
那個被鬼附加拉森的人.
困在加拉森一個偏遠的地方.

$^{1841}$連家人都不接濟他的情況下.
其實他是被隔絕的.
他沒有溝通.
但那個人去到耶穌面前的時候.
他就說至高神的兒子.
我與你何乾.
如果你是同行的那十二個門徒.
你看到這個情況的時候.
你在想什麼.
他剛才說什麼.
什麼名字.
至高神的兒子.
剛剛見完平靜風浪.
接著又在加拉森的情況.
接著你會見到.
去到血流的時候.
一個被隔絕的女人.
她因為血流不方便出門的時候.
也是行經期間.
定為不潔.
她不能夠那麼容易與人接觸.
她碰到的東西都成為不潔的時候.
她就自我隔離.
長期做這個自我隔離的時候.
但她本著去得醫治的心態.
你就讓我摸一摸.
就用這個小小的盼望.
去希望摸一摸根治的時候.
事就這樣成了.
耶穌是upset了宗教上.
什麼是聖潔.
也是將一個難處.
在生命當中被捆索.
將它挪開.
所以無論控制了自然.
靈界.
甚至宗教的層面.
耶穌都重新去更新了.
也是讓跟隨的人明白.
他的能力在哪裡.

$^{1881}$最後五餅二魚就很明顯看到.
史無變有.
上帝公認的那種創造.
所以你看到馬可夫音.
其中一個轉接.
在那幾章裡面.
他要讓人明白到.
今天你看到一個人子耶穌.
他是人子耶穌基督.
他有救贖的能力.
更新和拯救的能力.
這個就是希望被人更新.
但和他玩了三十年.
那群村民或舊鄉里.
其實我全家人都認識他.
他不能夠卡住.
不能轉.
他不能夠透過那件事情.
去看到耶穌的另外身份.
就是耶穌基督真正出現的身份.
這個同樣是靠近.
不知道你信主多久.
我信主超過三十年.
但過程中不是論資排輩.
不是坐得久教會.
信仰從來都是靠那種生命的經歷.
你認真去反復去想.
其實那件事是不是這樣.
不是說自己理性多厲害.
但過程中你都想想.
同一件事為什麼有不同的反應.
你看到那群門徒.
其實不斷地問.
其實是什麼事呢.
不斷地問.
其實我們跟隨的是誰呢.
所以你看到看馬克思的後面.
他在山上.
耶穌變相的時候.
他們就說竹譚.

$^{1921}$但他更加聽到一個更重要的信息.
就是這是我的愛子.
你們要聽他.
其實信仰從來都不是看見多少神跡.
信仰從來就是在見過當中.
見了當中.
你有沒有轉念.
有沒有轉化.
所以上教會不是說你上多少次崇拜.
是說崇拜當中你如何去開放去經歷.
去轉念.
去看到上帝的說話.
如何去更新你或者提醒你.
或者另外就是如何去迎受.
那個說話在你生命當中.
成為一個什麼行事的准則.
所以馬可就是看到.
由人子耶穌和人子耶穌基督的重要性.
所以用迴路加的時候.
就是說這個人子就是有能力.
在眾人面前彰顯出來.
他一定會再回來.
而他回來的時候.
就是如他所說的.
從開初他會更新了各種能力.
他就回來了.
所以我們常常都說.
特別是我自己很喜歡說.
就是在散會結束的時候.
一個星期的崇拜就結束了.
策顯崇拜的時候.
我仍然是等待著我主榮耀中再來.
因為不知道那個星期.
上帝是否會再回來.
但他回來的時候.
一定是帶著人子的榮耀回到當中.
但我們會怎麼做呢.
這個是不容易的.
在落義生powerpoint之前.
有一次我和一群年輕人.

$^{1961}$中一,二,三的年輕人上主教學.
事源是因為我那時候住堂的時候.
有位同工主教老師病了.
那天我八點半知道我要去代課.
他告訴我代那群人.
我說那群是什麼人.
他說了那些學生的時候.
我就說要認真一點.
為什麼要認真一點呢.
因為那裡有同主任的兒子.
有執事會副主席的女兒.
有主學校長的兒子.
全部都是從孩童就回教會.
說得清楚.
我不是忌惟他父母是什麼人.
正正就是我知道我要教什麼.
我要教五餅二魚的神跡.
我說OMG.
他們家裡已經聽過很多次.
我還要教五餅二魚.
因為課程是講道的.
課程是定格的.
不可以改的.
於是我就考功力.
於是我就去.
進去看到那六個年輕人的時候.
我就說今天我來代課的.
很明顯看得出.
平時都不是你教的.
第一句話就是這樣.
所以我今天也是跟著課程教.
教完.
不過他們都沒有看.
他說一個小時的主要學期間.
我給你45分鐘講.
給你15分鐘我打機行不行.
哈哈.
我就說.
看你能不能用15分鐘.
我說我開始講了.

$^{2001}$他說今天講五餅二魚.
立刻就反我眼.
我每人就派一張紙給他.
派一張紙給他的時候.
那怎麼辦呢.
重點是什麼.
就是有四格的.
我就說大家都很熟悉五餅二魚.
那你就將你熟悉的五餅二魚.
用四格漫畫去反映出來.
跟著他就說.
不做行不行.
你又說給我45分鐘.
哈哈.
於是就畫畫畫畫.
跟著畫的時候.
有些畫得真的很漂亮.
畫得很漂亮.
跟著第一格就有很多人.
跟著第二格有個小朋友.
拿了五個餅兩個魚.
跟著第三格不講了.
第四格有12個男子.
那很清楚這個modal answer.
是這樣評分這四格.
追包的是第三格.
第三格是畫什麼.
猜猜.
肚子餓.
第三格他畫了一個耶穌.
跟著前面有個盤.
跟著就這樣.
哈哈.
跟著就在boom.
boom.
跟著我說這格解釋一下.
他說畫得這麼好都要解釋.
我說我只知道boom是什麼.
很明顯就是耶穌變.
史無變有.

$^{2041}$變了boom.
我美式一點就有個boom.
跟著我就說.
你覺得耶穌是什麼.
耶穌很明顯就是一個魔術師.
我說耶穌是一個魔術師.
那魔術師可不可以是耶穌.
他們開始不出聲.
如果耶穌是史無變有.
是一個magic的話.
是否會用magic的就是耶穌呢.
他們就停在這.
你會發覺他那段經文很熟.
但我想象他明白到.
不是每件事反轉了都是一樣的.
就好像我的年代.
有叔就是你老爸.
那些想法.
這些邏輯很基本.
所以我跟初中那班.
回教會很久的弟兄姊妹.
年輕人說一句.
你習以為常的東西.
你的生命沒有轉念是很重要的.
仍然是那句.
當我們知道耶穌再回來的時候.
你知道自己要預備.
但我問你有沒有預備的時候.
除了是一個practical要預備之外.
你的心智上沒有預備.
所以在這段經文的時候.
就講到一個重點.
就是.
是不是我按得到.
就是你要謹慎.
恐怕因貪食醉酒.
今生的思慮.
你們的時候.
你要站立.
得穩這個很重要.

$^{2081}$其實你知不知道自己如何站得定.
或者覺得到.
好 我們.
是不是我按得到.
是不是.
所以我就用.
其實剛才講到.
貼守羅來加教會都遇到一個問題.
貼守羅來加教會保羅離開的時候.
保羅又寫封信給他們.
其實第一件事是贊他們.
因為他們見證很好.
很多人因為他們見證信心的時候.
就稱贊他們.
他們很開心.
但其實貼守羅來加前書第四章的時候.
就有些弟兄姐妹所愛的人死了.
他們覺得上帝說回來.
但回來幾十年都沒有.
其實是不是真的回來.
又問到一件事.
如果死了的話.
他死了的狀態會是怎樣.
所以保羅寫信就讓他們明白到.
上帝如何去看這件事.
或者上帝曾經已經告訴我們.
那件事是怎樣會發生和經歷.
在第五章的時候.
保羅都提醒一次.
就是那個日期是怎樣.
沒有人知道.
環境是怎樣會變化.
但是上帝讓我們明白到.
他預定我們不是要受刑.
預定我們直住耶穌基督得救.
他替我們死.
叫我們無論醒著睡著都與他同活.
再一次讓受輸的人明白到.
我們在什麼狀況都好.
上帝都清楚.

$^{2121}$雖然你現在生命已經暫停了.
但是你只是在上帝當中睡著.
抹後的時候他會再一次回來.
讓我們一起去和他同享受天國.
這個這一刻是證實不了的.
保羅說這是我們深信.
因為耶穌從來都是說得出做得到.
所以你們要彼此勸慰.
互相建立.
正如你們素尚所行的.
這個字素尚所行.
我覺得可圈可點.
什麼是素尚所行.
你今天素尚所行是什麼.
你還記得早前Obi目者說.
素尚對於彈爾利是什麼回事.
就是仍然在他難處當中.
做他敬拜的生活.
做他每日如常去守節.
或者是去默想上帝的生活.
對你來說什麼是素尚所行.
今天你素尚所行是什麼.
這個現在這麼問你.
不會這麼回答我.
但是你問一問自己.
平時你信仰的素尚所行.
和你生活的素尚所行有多接近.
你每日佔據的時間有多少.
你每日用的時間用在哪裡.
多些你自己在遇到困難的時候.
你找誰.
剛剛星期四小組的時候.
我和頂姐妹說.
當你有困難的時候.
我們找誰.
那個人在當中.
我可以問你.
我經常這麼問.
當你有困難的時候.
你第一個打電話會打給誰.

$^{2161}$你的腦子里可能有個人.
如果那個人彈了出來的時候.
你想掃苦.
你打給那個人.
那個人是不是在你小組里.
如果他不是你小組的人.
他是不是在你教會里.
如果他不是你教會的人.
他是不是基督徒.
如果你想有問題.
想掃苦.
想找人談的時候.
那個人又不是你小組的人.
又不是你教會的人.
又不是基督徒.
我仍然是問.
其實和有共同信仰.
能不能夠彼此勸惟.
彼此建立.
這個是很真實的.
我當然不是說.
我的朋友沒有信主的.
支持不了我.
但是我仍然覺得.
我們是有共同信仰.
有共同經歷.
在教會有不同分享的弟兄姊妹.
那種建立是很真實的.
這個也是.
這個純純都是你花多少時間.
花多少時間去分出去.
你和未信主的朋友一起.
或者和已信主的朋友一起.
我不是叫你break even.
計數不是那些.
你總會有很多.
大家一起去建立關係.
是真的.
這個就是很真實.
如何成為一個互為肢體.

$^{2201}$互為support.
是一個很重要的.
保羅仍然提醒.
貼上來的教會.
就是你們繼續彼此勸惟.
彼此建立.
就好像你們面對困難的時候.
面對失落的時候.
其實有很多見證人.
在你身邊當中出現.
他們如何經歷信仰.
其實是彼此分享.
彼此承擔.
就好像你平時那樣做.
聽我講道或者信息的時候.
你都會聽過.
其實我很少講成功建政的牧者.
因為我覺得失敗建政才是最受用的.
知道獲了就叫人不要跟著做.
你都聽過一句話.
成功沒有方程式.
失敗一定有原因.
重點就是成功是你無法複述的.
因為他的人設是可以的.
那就可以做到.
但你不行.
照板轉換都不行.
但失敗一定是有原因.
為什麼失敗的時候.
就告訴大家.
有些事是不需要做.
或者不要做.
可能他的失敗對你不是失敗.
但起碼過程當中.
人家給了你一個參考.
這個就是在當中一起去參長.
我說我們當中不一定每個人都行.
但我們一起談你不行的地方.
就是讓我們面對在等待耶穌之間的困難.
這個素尚所行是很重要的.

$^{2241}$Falsehood有很多東西是不行的.
Falsehood有很多東西是試下做下.
數下錯下.
錯下當中又試下.
一定是try an error.
走到今年第五年.
我們素尚所行就是keep trying.
keep doing.
這個就是我們所做的事.
有些什麼是重要.
弟兄姐妹之間一起投入一個信仰群體.
那個是最重要.
這個就是教會存在.
在地上一定要做到的地方.
所以我們不是追數字.
但我們緊張的就是.
你每個星期有沒有崇拜.
你來不了現場的話.
你要看回.
每個星期都一定要崇拜.
崇拜就是你每個星期預留時間去敬拜上帝.
這個是你素尚所行的事情.
我經常說求害求分數.
分數是反映你沒有做.
你得10分就因為.
你100分你得10分就因為你沒有運輸.
所以你做了那麼多就那麼多.
對我們來說素尚所行是什麼呢.
貼瘦羅爾加教會就是.
他一直勸勉一起咬實牙筋.
等耶穌基督的回來.
經文在張臨其過程當中.
我希望我們都是等耶穌基督的第二次降臨.
等的過程當中我們真的不知道.
但我仍然和頂尖妹說.
我們不斷地倒數耶穌的回來.
就像神學百科的時候.
John也說耶穌基督在遠方跑來.
我們在跑去迎接祂.
我們等相遇的那一刻.

$^{2281}$這個就是我們繼續做.
我們做回我們平時的練習.
這個就是我小時候很喜歡看的卡通片.
對很多年輕人來說.
可能沒有看過.
你看的可能是Dragon's Set.
但我仍然喜歡小時候的龍珠.
吳空由龜仙人開始長大.
到他自己打贏了龜仙人.
他把自己的衣服從龜變成了悟字之外.
我就很喜歡.
吳空其實整個班底就是.
很多team來的,很多隊員.
大家一起參與,一起很開心.
厲害和不厲害都好.
都可以一起去玩.
很著重那種群體關係.
對我來說,我真的很喜歡這個卡通.
一個很重要的人設.
但整件事你會看到他不斷地成長.
你會看到由吳空跟龜仙人.
到自己贏了武林大會.
然後開始慢慢長大.
開始變超西.
開始有不同的變化.
然後你會發現.
由一個變一次身變兩次身.
就會發現一直在變身.
但你會看到那套路又不是一直在變身.
其實你會,你可能不會,我會.
我會看到每次變身的時候.
都有些誘因.
令他在傷痛當中經歷成長.
因為我自己很喜歡吳空一個角色.
他的路是否很順利都不差.
但他一定有傷痛.
每次傷痛當中.
他就在交友公司帶著口罩.
然後泡著水.
然後好起來的時候就很厲害.

$^{2321}$每一次好起來之後.
他整個人就厲害了很多.
如果你記得他第一次變超級賽亞人的時候.
是什麼原因.
哇,立刻感覺到.
回來了.
我雖然不知道是誰.
就是他從小玩到大.
一起參加武林大會的無限.
被菲利殺死了.
然後.
然後他就.
然後看到他在轉身.
轉身的頭髮的時候.
你會發現人生會經歷傷痛.
人生會經歷分離.
人生會經歷失去的時候.
那種生命的突破.
那種毀滅的轉變.
是很重要的.
整件事對於那個做法.
是令到他成長一個很重要的突破的位置.
今天我們沒有傷痛嗎.
今天我們沒有分離嗎.
今天有沒有聽判刑.
我們在那個難處當中.
會不會經歷那個突破位呢.
很多年之後才可以再出來的時候.
你能不能感受到那種苟潔.
但是不是現在可以立刻出來.
立刻可以突破.
不是的.
很多戰鬥會繼續打的.
但是悟空的生命力讓我們明白到.
其實很多東西不是即時.
但是我們在過程當中養傷.
生命在不斷成長的時候.
會給我們很多韌力.
又給我們突破.
第一次變超西.

$^{2361}$第二次變超西.
第三次變超西.
然後不斷越強越強.
你可能說這些只是沒有辦法的卡通片.
哪有那麼神的.
但是它讓我們明白到.
其實和我們一起看這些的人很多.
在我預備講章的時候.
我就找人幫忙.
我就問人借了一件衣服.
你見到穿悟空的衣服.
就是我的好幫手.
很漂亮的Jazz.
在他讓我選擇那麼多件衣服當中.
很多件.
選擇那麼多件當中.
我就是選擇這件.
因為這件是小時候的悟空和無限.
就是這兩個.
我很喜歡這個款式.
在過程當中讓我感受到.
這個就是最原創的那種小時候的心態.
悟空和無限.
其實無限一定是不懂打架的.
但是你會發覺.
他說一起玩一起玩.
其實不是比高低.
就是那種赤子之心.
就是那種同行的感覺.
很重要.
在比拼當中一起去參與.
是很重要.
認識我的人都知道.
我很喜歡看天.
下一張.
我每次看天的時候.
我就很感受到.
就是我自己覺得有空間.
有好好.
在我看悟空很多場比賽的時候.

$^{2401}$你會看到很多風輪.
打得起起含含的.
但是通常那些都會輸.
但是耶穌.
我看到一出一個很漂亮的天空的時候.
我知道耶穌一定會贏.
不是耶穌.
悟空一定會贏.
就是悟空的時候已經打到沒有了.
當他龜波氣功都沒有用的時候.
他就會向著天.
舉高兩只手.
是怎樣的.
是的.
那些人就會了.
好了.
頂姐妹可不可以借手給我.
哇.
是哦.
我們一起來.
我們試一下.
頂姐妹.
Float出來頂姐妹.
可不可以舉起手.
分點力量給我.
不要下來.
別收.
元氣彈是一定要慢慢地收集的.
你明白嗎.
我們一個人是不行的.
你會看到.
是不斷地將我們的氣集在一起的時候.
然後悟空收集好氣的時候.
然後就將元氣彈扎下去.
好.
看到了.
手可以放下.
你會看到的.
我看悟空的時候.
我很感受到那種集氣.

$^{2441}$當他發覺自己所有的不身的絕學都打不贏那個壞人的時候.
他會發現一件事.
我一個人贏不了.
全世界都會一起幫我們.
這個對我來說.
那時候我已經回教會了.
那時候我很熟悉.
那套卡通片是12點C看的.
但我就感受到.
其實那件事很教會.
如果你覺得你有一萬塊.
如果你覺得你自己有一萬塊.
教會是要一萬個一塊.
而不是一萬個一萬塊.
你覺得你很厲害.
打完所有人是一萬塊.
一萬個power.
但其實每一個power是集在一起.
所以元氣彈是我最喜歡的東西.
而且元氣彈是沒法輸的.
最後就一定贏.
接著就天浪氣清.
我希望弟子妹.
今天大家都舉了手出來.
謝謝你.
在過程當中.
我希望教會是每個人都拿一點點出來.
不知不覺就一起去做點事.
如素尚所行一樣.
我們認識的耶穌基督.
是真真實實的人子耶穌基督.
祂會再回來.
當我們望天的時候.
你看看.
你會見到超級的耶穌基督.
那種驚奇.
那種期盼.
祂會在當中.
是真的.
我自己很不開心.

$^{2481}$或者很困難的時候.
我經常都嘆天望日.
我經常都和自己說.
或者有一天.
我會看著耶穌下來.
不知道.
但我信我主.
榮耀中在立.
你信嗎.
我願我主在來.
I am ready.
你ready嗎.
我第一次祈禱.
我們需要集氣.
集氣不是.
現實就有一個人.
我們要立刻打敗.
集氣是因為我們知道.
我們不孤單.
我們有一群同行者.
我們彼此結連.
彼此勸慰.
我們失敗的經驗.
可以彼此去支持.
讓我們可以檢討.
可以同行.
我們同行的經驗.
可以成為別人的幫助.
讓他明白到.
他不孤單.
我們需要集氣.
這是上帝讓我們看到.
也是上帝給我們的能力.
求主讓我們有更多焦聚的能力.
讓弟子們一起參與.
我們每次去敬拜禰的時候.
感受到上帝的同在.
我們也在當中望天.
我們也ready去等待禰.
再一次榮耀中在立.

$^{2521}$求主禰繼續對我們說話.
祈禱奉耶穌的名求.
\newpage



\section{馬太福音 8:1-4-20231216}
\label{sec:sKBDQD8UIMg}
\textbf{【網上聖餐崇拜】年少多好|馬太福音8\_1-4|20231216 [sKBDQD8UIMg]}
\newline
\newline
連結: \href{https://youtube.com/watch?v=sKBDQD8UIMg}{\texttt{ https://youtube.com/watch?v=sKBDQD8UIMg}} ~~~~ 語音日期: 2023-12-16 
\newline
\newline
\hyperref[sec:0oiGMpkgXB8]{\small{< < < PREV SERMON < < <}}
~
\hyperref[sec:index_chronic]{\small{[返順時目]}}
~
\hyperref[sec:index_scriptual]{\small{[返順卷目]}}
~
\hyperref[sec:dT3dN2jF8BQ]{\small{> > > NEXT SERMON > > >}}
\newline
\newline
馬太福音 8:1-4-20231216
\newline
\begin{longtable}{cl}
\hline
\hline
章節 & 經文 (和合本修訂版)\\
\hline
8:1 & \begin{tabularx}{0.7\textwidth}{X} 耶穌下了山,有一大群人跟著他。 \end{tabularx} \\ \\ \relax
8:2 & \begin{tabularx}{0.7\textwidth}{X} 這時,一個痲瘋病人前來拜他,說:「主啊,你若肯,你能使我潔淨。」 \end{tabularx} \\ \\ \relax
8:3 & \begin{tabularx}{0.7\textwidth}{X} 耶穌伸手摸他,說:「我肯,你潔淨了吧!」他的痲瘋病立刻就潔淨了。 \end{tabularx} \\ \\ \relax
8:4 & \begin{tabularx}{0.7\textwidth}{X} 耶穌對他說:「你要注意,不可告訴任何人,只要去,讓祭司為你檢查,並獻上摩西所吩咐的祭物,作為證據給眾人看。」 \end{tabularx} \\ \\ \relax
8:5 & \begin{tabularx}{0.7\textwidth}{X} 耶穌進了迦百農,有一個百夫長進前來,求他, \end{tabularx} \\ \\ \relax
8:6 & \begin{tabularx}{0.7\textwidth}{X} 說:「主啊,我的僮僕癱瘓了,躺在家裡,非常痛苦。」 \end{tabularx} \\ \\ \relax
8:7 & \begin{tabularx}{0.7\textwidth}{X} 耶穌說:「我去醫治他。」 \end{tabularx} \\ \\ \relax
8:8 & \begin{tabularx}{0.7\textwidth}{X} 百夫長回答:「主啊,你到舍下來,我不敢當;只要你說一句話,我的僮僕就會痊癒。 \end{tabularx} \\ \\ \relax
8:9 & \begin{tabularx}{0.7\textwidth}{X} 因為我在人的權下,也有兵在我以下。我對這個說:『去!』他就去;對那個說:『來!』他就來;對我的僕人說:『做這事!』他就去做。」 \end{tabularx} \\ \\ \relax
8:10 & \begin{tabularx}{0.7\textwidth}{X} 耶穌聽了就很驚訝,對跟從的人說:「我實在告訴你們,這麼大的信心,就是在以色列,我也沒有見過。 \end{tabularx} \\ \\ \relax
8:11 & \begin{tabularx}{0.7\textwidth}{X} 我又告訴你們,從東從西,將有許多人來,在天國裡與亞伯拉罕、以撒、雅各一同坐席; \end{tabularx} \\ \\ \relax
8:12 & \begin{tabularx}{0.7\textwidth}{X} 本國的子民反而被趕到外邊黑暗裡去,在那裡要哀哭切齒了。」 \end{tabularx} \\ \\ \relax
8:13 & \begin{tabularx}{0.7\textwidth}{X} 耶穌對百夫長說:「你回去吧!照你的信心成全你了。」就在那時,他的僮僕好了。 \end{tabularx} \\ \\ \relax
8:14 & \begin{tabularx}{0.7\textwidth}{X} 耶穌到了彼得家裡,見彼得的岳母正發燒躺著。 \end{tabularx} \\ \\ \relax
8:15 & \begin{tabularx}{0.7\textwidth}{X} 耶穌一摸她的手,燒就退了,於是她起來服事耶穌。 \end{tabularx} \\ \\ \relax
8:16 & \begin{tabularx}{0.7\textwidth}{X} 傍晚的時候,有人帶著許多被鬼附的來到耶穌跟前,他只用一句話就把邪靈都趕出去,並且治好了一切有病的人。 \end{tabularx} \\ \\ \relax
8:17 & \begin{tabularx}{0.7\textwidth}{X} 這是要應驗以賽亞先知所說的話:「他代替了我們的軟弱,擔當了我們的疾病。」 \end{tabularx} \\ \\ \relax
8:18 & \begin{tabularx}{0.7\textwidth}{X} 耶穌見許多人圍著他,就吩咐渡到對岸去。 \end{tabularx} \\ \\ \relax
8:19 & \begin{tabularx}{0.7\textwidth}{X} 有一個文士進前來對他說:「老師,你無論往哪裡去,我都要跟從你。」 \end{tabularx} \\ \\ \relax
8:20 & \begin{tabularx}{0.7\textwidth}{X} 耶穌說:「狐狸有洞,天空的飛鳥有窩,人子卻沒有枕頭的地方。」 \end{tabularx} \\ \\ \relax
8:21 & \begin{tabularx}{0.7\textwidth}{X} 又有一個門徒對耶穌說:「主啊,容許我先回去埋葬我的父親。」 \end{tabularx} \\ \\ \relax
8:22 & \begin{tabularx}{0.7\textwidth}{X} 耶穌說:「讓死人埋葬他們的死人。你跟從我吧!」 \end{tabularx} \\ \\ \relax
8:23 & \begin{tabularx}{0.7\textwidth}{X} 耶穌上了船,門徒跟著他。 \end{tabularx} \\ \\ \relax
8:24 & \begin{tabularx}{0.7\textwidth}{X} 海裡忽然起了猛烈的風暴,以致船幾乎被波浪淹沒,耶穌卻睡著了。 \end{tabularx} \\ \\ \relax
8:25 & \begin{tabularx}{0.7\textwidth}{X} 門徒去叫醒他,說:「主啊,救命啊,我們快沒命啦!」 \end{tabularx} \\ \\ \relax
8:26 & \begin{tabularx}{0.7\textwidth}{X} 耶穌說:「你們這些小信的人哪,為甚麼膽怯呢?」於是他起來,斥責風和海,風和海就大大平靜了。 \end{tabularx} \\ \\ \relax
8:27 & \begin{tabularx}{0.7\textwidth}{X} 眾人驚訝地說:「這是怎樣的一個人?連風和海都聽從他。」 \end{tabularx} \\ \\ \relax
8:28 & \begin{tabularx}{0.7\textwidth}{X} 耶穌渡到對岸去,到加大拉人的地區,有兩個被鬼附的人從墳墓迎著他走來。他們極其兇猛,甚至沒有人敢從那條路經過。 \end{tabularx} \\ \\ \relax
8:29 & \begin{tabularx}{0.7\textwidth}{X} 他們喊著說:「神的兒子,你為甚麼干擾我們?時候還沒有到,你就上這裡來叫我們受苦嗎?」 \end{tabularx} \\ \\ \relax
8:30 & \begin{tabularx}{0.7\textwidth}{X} 離他們很遠,有一大群豬正在吃食。 \end{tabularx} \\ \\ \relax
8:31 & \begin{tabularx}{0.7\textwidth}{X} 鬼就央求耶穌,說:「若要把我們趕出去,就打發我們進入豬群吧!」 \end{tabularx} \\ \\ \relax
8:32 & \begin{tabularx}{0.7\textwidth}{X} 耶穌對他們說:「去吧!」鬼就出來,進入豬群。一轉眼,整群豬都闖下山崖,投進海裡,淹死了。 \end{tabularx} \\ \\ \relax
8:33 & \begin{tabularx}{0.7\textwidth}{X} 放豬的就逃進城去,把這一切事和被鬼附的人所遭遇的都告訴眾人。 \end{tabularx} \\ \\ \relax
8:34 & \begin{tabularx}{0.7\textwidth}{X} 全城的人都出來迎見耶穌,見了他以後,就央求他離開他們的地區。 \end{tabularx} \\ \\
[1ex]
\hline
\hline
\end{longtable}
$^{1}$好,頂姐妹晚安.
聖誕節這幾年都很怕說馬大福音一二章和奴家福音一二章.
可以說的都說過.
凡是很怕這些日子說一些聖誕的訊息.
已經很難說了.
特別為在聖誕節裡有不同需要的人特別紀念.
無論是留下了不同的頂姐妹在不同的地方.
或者在艱難容易困所裡.
都特別紀念到不同人的需要.
所以今天只穿了一件聖誕樹的衣服.
多謝贊助商的人好像說在迪士尼Fosen那裡買的.
牌子是迪士尼的.
我猜應該是.
所以選的經文我們不用聖誕節的經文.
我們今天用馬大福音第八章.
多謝網上的頂姐妹很聰明.
她以為會說的是十八章.
因為十八章的主題是關於小子.
但是我們今天是想說第八章.
為什麼會說第八章呢.
其實也很關乎我自己最近對福音書其中一個很大的難題.
其實我想有些時候了解一下.
其實我經常問的問題就是.
為什麼耶穌記載的事情那麼多.
為什麼會選某些事情要記載.
某些事情不記載.
這是我看福音書的時候經常想問的問題.
為什麼那些東西要寫在這裡.
不寫下去其他很多很值得寫的應該不寫.
為什麼要寫一些這樣的東西.
今天我們可以看四字經文.
其實我們看完之後.
講這四字經文的機會很少.
不過我想解釋一下大麻風這個「伊」字.
其實是什麼.
或者想講這個故事背後.
其實馬太在交代一件什麼事情.
不複雜的.
耶穌下了山.
有很多人跟從祂.

$^{41}$有一個長大麻風的人來拜祂.
主約肯就必能叫我潔淨了.
耶穌伸手摸了他說.
我肯你潔淨了吧.
他的大麻風立即就潔淨了.
耶穌對他說.
你切不可告訴人.
只要去把身體給祭司擦汗.
獻上摩西所吩咐的禮物.
對眾人作證據.
這個故事其實很容易理解.
不複雜.
耶穌摸了一個大麻風的人.
通常這個故事的應用就是.
去了德蘭修女.
通常這個故事講完的時候.
我們的結論或者總結.
或者可以應用到的就是.
請我們做一個德蘭修女.
摸一些有需要的人.
很骯髒的人.
或者一些很不可愛的人.
請你擁抱他.
但實際上這個故事講完之後.
我們得出的結論是這樣的.
但我今天想問多一點.
除了這個結論之外.
會不會有多一些.
這個故事寫出來的時候.
不一樣的結論和想像呢?.
所以我們今天希望能夠從一個角度去看.
這四節聖經其實是在講一件什麼事情.
複雜的事情我們不是很想討論.
第一節.
耶穌下了山.
有很多人跟從著祂.
這個句子.
我們不要說一些希臘文複雜的東西.
這個句子背後其實想交代一件事情是什麼呢?.
5至7章是耶穌刻意在山上.

$^{81}$做了一個叫登山補糞的事情.
登山補糞就是祂在上邊.
交代所謂八福.
所謂主禱文.
被人打完左邊右邊.
講了很多很多不同的事情.
收集在一起.
叫做登山補糞.
登山補糞的故事是怎麼來的呢?.
大致上很多學者都會認為.
耶穌在那裡做了一件事.
就是重新詮釋摩西的律法.
等於很簡單的意思就是說.
如果昔日摩西在山上頒布律法的話.
耶穌做的事情很相似.
都在山上重新詮釋摩西的律法.
你會說不要殺人.
其實不是的.
心裡說別人蠢材死蠢.
你還是殺了人吧.
所以耶穌在做的事情.
很像摩西在山上領受律法一樣.
不過耶穌這次不同.
祂重新詮釋摩西的律法是什麼.
所以如果你這樣理解.
五至七章很像摩西的話.
第八章的故事.
或者第八章第九章的故事.
其實就是想表達一件事.
耶穌所做的所有事情.
尤其是在第八章第九章裡.
不是隨機地去選擇一些事情.
這個故事我記得耶穌這樣做過.
那件事耶穌做過很出名.
我又放進來了.
如果那是關於耶穌重新演繹摩西頒布律法的話.
第八至九章就是說.
耶穌如何在實際行動裡.
演繹在五至七章登山補訓裡.
所說的每一件事.

$^{121}$所以你會發覺五至七章登山補訓所說的.
其實我們很難應用.
你不會被人打完左邊右邊.
你不會看到一些你不應該看的東西的時候.
你挖自己的眼睛.
斬自己的手.
但是上登山補訓.
很難去應用.
你聽到別人說他笨一點.
你說他死蠢.
難道你又落地獄之火.
難免你就會受到.
所以登山補訓的事情.
其實是說完之後.
不知道如何實體去應用.
所以馬太她選的八至九章.
所有的故事.
其實是用來回應.
耶穌在五至七章登山補訓.
說完那些事情之後.
他找一些故事來.
告訴我們.
登山補訓其實是如何應用的.
其中登山補訓這一句也很有名.
他不是來廢掉律法.
而是承傳律法.
所以耶穌在八至九章所做的故事裡.
第一個故事.
就是他下山之後.
他遇到大麻風的人.
他治好了他.
其實這個的醫治.
正正代表耶穌要來.
承傳律法的故事.
聽到這裡.
其實我們可能不是很習慣聽這些.
你覺得耶穌如何承傳律法.
例如最近猶太人慶祝節期.
光明節 夏律卡.
你會說我都不慶祝猶太人的節期.

$^{161}$為什麼耶穌要來承傳律法.
他承傳什麼律法呢.
如果我們想多理解和明白的話.
一會兒你再聽下去的時候.
你會知道耶穌在承傳律法裡.
他在做什麼改革與更新.
這是在利美記十四章第二節.
他說長大麻風得潔淨日子.
其禮乃是者曰.
要帶他去見濟世.
其實在利美記裡也挺特別的.
利美記裡有兩張聖經說大麻風.
兩張聖經.
其中一句我們很出名的.
在這四節經文裡.
我們通常會用的.
那句是什麼.
如果大麻風來的時候.
你就要跟別人說不潔淨了.
大家快點走.
通常都要引用利美記的經文.
但你會發現利美記裡有兩張聖經.
十三十四章的兩張聖經.
專門說大麻風.
你會發現大麻風其實.
坦白說你看完利美記.
看到利美記的時候.
你會發現利美記的大麻風.
由於七天之後好一點.
再等七天.
再等七天又給濟世看.
看完之後你的身體怎麼樣.
那種煩惱複雜.
好像與愛無關.
所以聖經其實沒什麼用.
尤其是利美記說的東西.
對我們完全沒有什麼關係.
但你不能忽略的是.
這兩張聖經裡.
它說大麻風的時候.

$^{201}$重點是什麼.
其實整個大麻風.
這兩張聖經.
記載的只有一個重點.
如果你看.
你可以現在滾手機.
不要介意.
你看十三十四章.
大部分的內容.
說的是一件事情.
就是說怎樣接納大麻風的人.
在自己群體裡.
我再來一次.
我的意思是.
利美記這兩張好像很悶的聖經.
很無聊的聖經.
什麼七天七天.
又怎樣檢查它.
又要放在哪裡.
又要找祭司等七天.
肉又要看看怎樣.
七天之後又要回來.
如果沒事就獻祭.
然後就抓他回來.
它搞這麼長的篇幅.
就是想說.
凡是有大麻風的人.
人都會將他放在一旁.
將他放在最遠的地方.
利美記十三十四章寫這麼長的東西.
就是想表達一件事情.
就是我們常常將他放在一旁.
而不理會的人.
聖經執著的是.
要將他回到自己群體裡.
花很多的時間和氣力.
花很多的精神和時間.
祭司要做很多的事情.
確保這個人能夠回到自己群體裡.
所以利美記其實是說什麼呢.

$^{241}$真正接納大麻風這類型的人.
回到自己群體裡.
如果要回到今天.
不一定要去嘉義國塔.
像德蘭修女那樣.
接觸那些很骯髒很骯髒的人.
其實今天應該在說什麼呢.
其實今天這個故事.
要想說的或是想表達的是什麼呢.
如果你看回.
在一世紀的時候.
你記不記得耶穌和那些罪人吃飯.
耶穌和那些罪人吃飯的時候.
那些法利塞人.
西門家裡的法利塞人的朋友.
他們罵耶穌什麼呢.
罵耶穌是你為什麼和那些罪人吃飯.
當耶穌要趕鬼的時候.
耶穌被人罵什麼呢.
你只不過是靠著鬼王別世不趕鬼.
奉係上帝的兒子耶穌.
接納那些罪人.
接納那些鬼婦的人.
回到自己群體的時候.
其實大部分的人就不喜歡.
其實大部分的人就不知道.
突然間就說.
這班大麻瘋的人是上帝就坐的.
是上帝不喜歡的.
所以在拉比的文獻裡.
曾經形容過大麻瘋是什麼.
說他們是活著的死人.
其實根本是死的.
不過是活著的.
所以我們不用理會他.
但事實上他活著.
但他差不多死了.
用宗教的條例.
將一些人家不滿的人.
人家不喜歡的人.

$^{281}$將他放在一旁.
只接納一些同聲同氣.
好像自己談得來的人.
這是當時耶穌要突破的東西.
其實我不知道大家是否發現.
在我們的圈子裡.
我們越來越少見很多.
我應該不要這樣說.
應該怎麼說.
很特別的人.
好像我們圈子裡.
就有些很類似的人.
譬如以前.
在教會裡.
我們做人格的時候.
你發現教會裡最多的是什麼.
二號仔.
或者我這類的九號仔.
二號,九號在教會裡很多.
多到不能再多.
你知道教會裡最少的是哪類型嗎.
我的資料不準確.
八號仔是最少見的.
如果是八分之五的話.
八號仔就是意見領袖.
五號仔是思考者.
全部用腦和說話的人.
在教會裡很少出現.
教會裡多的是像你這樣.
很乖的聽道的人.
你不用想.
總之你說什麼就是.
八號和五號是教會很絕的生物.
所以一輪.
八號仔進到教會.
我就很想留住他.
希望他生存在教會裡出現.
六號仔我就不理他.
你知道六號仔是忠誠者.
你不理他.

$^{321}$他都會坐在那裡.
他很忠誠.
他一定坐在那裡.
他回來就是了.
所以我不是想說那些.
很骯髒的人.
是我們教會某個形式模式裡.
有些人在那裡生存不了.
這是我們的現況.
更重要的是什麼.
我們以為我們這樣生存是對的.
我們以為就這樣.
就這樣做.
我們就要這樣走.
這樣就適合某些人就OK了.
但需不需要你那些.
未必常在的人.
他一定要邀請他回來.
或者他容易在那裡生存得久.
這些慢慢不成為我們心裡.
要想的想法.
耶穌是一個.
整個人生裡.
祂做一件事我懷疑祂.
在很多錯誤解釋的真理.
在很多人將那個真理扭曲了一些.
表面上好像很對.
但背後其實是很錯的東西.
祂全部將那些東西扑出來.
這是耶穌想做的事.
以致能夠容納到更多不同的群體.
能夠進到上帝家裡.
全面的教會.
希望是一個什麼樣的群體.
無論是三千八個奇人怪狀.
你定義一下自己是否這樣.
我是不是這樣.
那些奇人怪狀和三千八個的人.
你都可以覺得你被歡迎.
你被接納.

$^{361}$最慘的是什麼.
當一個群體以為自己可以很穩定.
很平穩的時候.
我們慢慢就成為了很多的界線.
成為很多的牆.
將很多跟我們好像不同的人.
放在外面.
教會最難的是什麼.
是永遠都不知道自己.
在做的事其實是將一些人.
放在外面.
並且以為自己在做的事.
是很正確的.
我們不要說教會.
好像很複雜的題目.
我們說我八月的時候.
我講道壽的時候說過.
我兒子在英國的時候很煩.
我罵他一頓.
記不記得.
我還跟他說.
我要跟兒子找時間道歉.
好像八月的時候我這樣說過.
其實我沒有交代.
其實九月的時候.
我真的跟我兒子道歉了.
為什麼會道歉呢.
其實我都忘記了我要道歉.
因為我女兒聽到我去福爾摩教會講道.
(笑).
好像一家人吃飯.
吃飯的時候我女兒無聲無息地說一句.
爸爸你不是說要跟我兒子道歉嗎.
她說沒辦法.
你不道歉就好像很壞.
道歉也不知道在做什麼.
很少這樣.
然後我真的很恭敬地.
在兒子旁邊吃飯.
說兒子對不起.

$^{401}$爸爸不是很明白你那一次.
希望你原諒爸爸.
說完以為我兒子九歲十歲就當沒事.
繼續吃飯.
我就覺得這樣就完了.
可以跟你們交代.
我做了 多好.
我收到劇本大約是這樣寫.
誰知道我兒子突然.
撲到我大腿上.
整塊臉撲到大腿上.
然後嚎哭.
哭得很厲害.
哭完之後他一起來.
我的褲子兩條濕的淚痕.
鼻涕加上什麼都齊了.
我不忘記我兒子哭完之後.
他說了一句.
他說爸爸 你知不知道被人冤枉.
很慘的.
到今天我還在學什麼呢.
我現在罵他.
我也想我沒有冤枉他.
起碼我到今天.
我要很生氣去罵他的時候.
我也想我這次沒有冤枉他.
起碼我學.
所以聽完節目你會發覺.
耶穌來到世間上.
祂所做的事是什麼.
是很多我們以為我們很合得來的時候.
祂突然說了一些話.
其實你想一些事不一定是這樣.
其實祂好像登山佈墳一樣.
上次告訴你.
你以為守了規矩就可以.
搞定那些事就成為正常的基督徒.
祂告訴你不是.
祂明明可以不摸那個人.
他就可以潔淨.

$^{441}$他怎樣都要摸一摸.
祂這個動作想告訴你.
大家都覺得很礙眼.
你摸他的時候你自己都不潔淨.
你馬上要找個祭司來.
看看你有沒有發作.
要關他七天.
看看麻瘋病有沒有傳染給你.
但耶穌都不介意.
要做一些很奇怪的事.
告訴人家.
我要接納這個人回來我這裡.
在2024年很快就來.
我自己對自己一個想法.
我想給自己多一些.
以往覺得很適合的東西.
重新想一想那些很適合的東西.
是不是很適合.
基督教好像越來越是一班.
覺得自己很適合的人.
走在一起做一些好像很適合的事.
但我們很少去留意反省一下.
到底我們經常覺得我們很適合的東西.
是不是真的很適合.
2024年是一個很大的分水嶺.
分水嶺的意思是.
要麼我們繼續以往的信仰模式的群體.
要麼我們嘗試反省完.
想完.
自己和一班人也好.
不知怎樣也好.
嘗試在以往很慣常覺得很適合的裡面.
你抽出來嘗試做一些人家會說鬼你的事.
人家會說你幹嘛碰他大麻瘋.
嘗試在這個年代和不容易的環境裡.
做一些人家未必覺得你適合的事.
但你冒險告訴自己.
我想有一個不一樣的信仰表達.
如果那時候耶穌看著大麻瘋的人.
每個人都是說讓祭司來搞.

$^{481}$我們不用怎麼搞.
放在祂一邊.
祂們只是活著的死人.
都不關我們事.
耶穌就在宗教上覺得大家都適合的裡面.
就要告訴所有人.
祂碰完祂 潔淨了祂.
祂做了一件人家覺得匪夷所思的事.
所以無論祂在安息人家醫病也好.
或是安息在裡面澆墨水也好.
祂所做的每一個動作.
祂由出生到成長到傳道那幾年.
祂正在挑戰的是.
是否我們覺得我們想的事都正正正正.
如果那班二千年前的人.
是很聰明的人 很厲害的人.
在宗教上玩得很好的人的時候.
我們是否真的會比他們厲害多了.
2024年我不知道你有什麼想法.
或者2023年尾聲的時候.
當我們見到耶穌出生的時候.
祂不純粹是一個嬰兒.
祂不純粹是一個聖嬰孩.
僅此而已.
祂出來的目的.
為了要將很多人語為對的宗教的東西.
祂拆了它.
祂建立了一個叫做.
跟從耶穌群體的一班人出現了.
秋天兄姊妹.
Fold Church是一個群體.
不是因為我們有什麼獨特的東西.
如果純粹是因為Fold Church很獨特而來的話.
這只不過是說.
這些獨特的東西有一天會不再獨特下去.
Fold Church就完了.
所以獨特的東西完了一年兩年三年四年五年.
這次我在這裡說了多少年.
五年還是六年?五年吧.
好像是這樣.

$^{521}$已經說了五年.
我們沒有獨特的了.
聽了五年一個人說道.
我做了每個人一次.
沒有什麼好聽的了.
有些人不再獨特的時候.
什麼是定義Fold Church這個群體獨特的呢?.
不是Fold Church出來做的東西很獨特.
是Fold Church的群體裡.
每一個人都獨特的.
是每一個人都找了一樣東西.
是他以往覺得對的.
突然間他抹掉.
其實不一定是這樣.
我們都為上帝做了一些不同的事.
好像耶穌摸了大麻風一樣.
讓不同群體的人能夠回來我們裡面.
真正吸引人的不是Fold Church本身運作的獨特.
Fold Church獨特的是每一個個體頂尖會參與的人.
他不只是聽完明白完一些東西.
不同的東西就完了.
我希望2024年.
我自己有些東西不同.
我在2024年裡都在想一些東西是不同的.
我已經2023年準備了一年.
希望2024年有些東西會看起來是不同的.
試一下信仰不行禮如儀.
試一下不要純粹因為Fold Church.
好像很獨特而來到一個很獨特的群體.
這些獨特的已經過了幾年不獨特了.
明年也是John說到.
明年又是他又是潘Sir和我.
沒有什麼特別的.
獨特的是個別,個體.
在這個地方裡.
你可以被鼓勵,被相信,被肯定.
你可以做一些不同的事.
跟你以往宗教行為模式的表現不同的東西.
聖誕節是一個開心的日子.
很歡喜的日子.

$^{561}$你怎麼名字都可以.
我期望這個聖誕節.
認識主耶穌基督降臨的時候.
他可以將整個宗教系統.
轉變成反轉的.
今天香港教會需要一班人起來.
將我們以往覺得要這樣信耶穌.
這樣表達信仰的東西.
可以轉變成反轉.
讓人看到.
像上次所說.
這班是一班搞亂天下的人.
就好像耶穌摸過大麻蜂一樣.
不要死心.
不要覺得信仰就是這樣.
仍然對這個世界有想像.
對我們的主有盼望.
在極艱難,不容易的2024年裡.
找什麼抗衡那些艱難.
似非而是,似是而非.
連欺騙都要值得慶祝的群體.
因為我們可以做一些.
令人驚訝的信仰行為表達.
耶穌五至七章的登山部分.
不是純粹說的.
他是用八章,九章.
他所說的內容和表達的故事行為.
告訴我們.
他如何顛覆人的想像.
如何搞亂天下的人.
最後Fold Church的頂姐妹.
包括海外的頂姐妹.
我們需要一班搞亂天下的人.
你發不發覺.
小朋友經常喜歡找真理.
我兒子覺得我在冤枉他.
他很深信我在冤枉他.
你發不發覺他青少年時有種墮落感.
我做父親的.
不斷冤枉他,冤枉到他十二,三歲.

$^{601}$他不會再記在心裡.
他會覺得這個爸爸是這樣.
我懶得跟你說任何真理.
成人為何會有種墮落感.
因為青少年已經墮落了.
你發覺跟人說真理是沒有用的.
跟人說正常的事.
世界的人聽不明白.
他還會說你回頭.
所以我不理你.
為何要回去做一個小朋友.
為何天國進入要像小朋友一樣.
因為只有小朋友才會肆無忌憚.
不耐煩.
不理任何人的眼光.
覺得自己愚蠢與否.
我想說這是我還在堅持的事.
希望這個聖誕節.
少在信仰裡耕耘.
不要跟自己說其實差不多.
信仰來到這個階段,就這樣就夠了.
這就像我女兒.
她跟你多說一句,你也會浪費氣力.
她會聽你寫這句.
她也不跟你說,你也不明白她.
她跟你說完,你也不知道她在做甚麼.
她說BTS十週年.
有甚麼好慶祝的.
那五百多元的東西買來做甚麼.
我都不明白.
她跟我說完.
買了那套東西,五百多元.
CD,現在沒有人用CD機了.
買CD做甚麼?.
她是傻的.
她說我才是傻.
但她不告訴我,她藏在裡面.
但小朋友不同的是.
他會堅持.
他會把他覺得要相信的東西.

$^{641}$告訴你.
今天我需要一班小朋友.
在信仰群體裡有這樣的表達.
我一起來禱告.
天父,我求你幫我們.
要突破自己.
要告訴自己,自己所相信的東西不夠.
所相信的表達很微弱.
甚至有時候會走錯路.
都好像很艱難.
我們要再走多一步.
將我們心裡所相信的東西走出來.
更困難.
所以我祈求的是.
在聖誕節.
在這個時機,我們要在你面前.
不只是開心.
不只是吃聖誕大餐.
不只是看無人機的燈火表演.
我求你祝福我們.
在我們面對2024更艱難的日子.
我們可以實質在這個世代.
做一些不同的事.
我求天父你組合我們.
把我們的人在不同的位置組合在一起.
可以在這個世代裡.
做我們覺得應該要做的事.
Fortune有獨特的群體.
但Fortune延伸出來的.
後會有更多獨特的群體.
我求天父你保守大靈主.
聽我們在面前的祈禱.
奉禱耶穌保衛命運.
Amen.
謝謝大家.
謝謝大家收看 再見.
\newpage



\section{路加福音 2:10-12-49-20231223}
\label{sec:dT3dN2jF8BQ}
\textbf{【網上崇拜】天選的細路|路加福音2\_10-12,49|20231223 [dT3dN2jF8BQ]}
\newline
\newline
連結: \href{https://youtube.com/watch?v=dT3dN2jF8BQ}{\texttt{ https://youtube.com/watch?v=dT3dN2jF8BQ}} ~~~~ 語音日期: 2023-12-23 
\newline
\newline
\hyperref[sec:sKBDQD8UIMg]{\small{< < < PREV SERMON < < <}}
~
\hyperref[sec:index_chronic]{\small{[返順時目]}}
~
\hyperref[sec:index_scriptual]{\small{[返順卷目]}}
~
\hyperref[sec:9ztySs_vnP4]{\small{> > > NEXT SERMON > > >}}
\newline
\newline
路加福音 2:10-12-49-20231223
\newline
\begin{longtable}{cl}
\hline
\hline
章節 & 經文 (和合本修訂版)\\
\hline
2:10 & \begin{tabularx}{0.7\textwidth}{X} 那天使對他們說:「不要懼怕!看哪!因為我報給你們大喜的信息,是關乎萬民的: \end{tabularx} \\ \\ \relax
2:11 & \begin{tabularx}{0.7\textwidth}{X} 因今天在大衛的城裡,為你們生了救主,就是主基督。 \end{tabularx} \\ \\ \relax
2:12 & \begin{tabularx}{0.7\textwidth}{X} 你們要看見一個嬰孩,包著布,臥在馬槽裡,那就是給你們的記號。」 \end{tabularx} \\ \\ \relax
2:13 & \begin{tabularx}{0.7\textwidth}{X} 忽然,有一大隊天兵同那天使讚美神說: \end{tabularx} \\ \\ \relax
2:14 & \begin{tabularx}{0.7\textwidth}{X} 「在至高之處榮耀歸與神!在地上平安歸與他所喜悅的人!」 \end{tabularx} \\ \\ \relax
2:15 & \begin{tabularx}{0.7\textwidth}{X} 眾天使離開他們,升天去了。牧羊人彼此說:「我們往伯利恆去,看看所成的事,就是主所告訴我們的。」 \end{tabularx} \\ \\ \relax
2:16 & \begin{tabularx}{0.7\textwidth}{X} 他們急忙去了,找到馬利亞和約瑟,還有那嬰孩臥在馬槽裡。 \end{tabularx} \\ \\ \relax
2:17 & \begin{tabularx}{0.7\textwidth}{X} 他們看見,就把天使論這孩子的話傳開了。 \end{tabularx} \\ \\ \relax
2:18 & \begin{tabularx}{0.7\textwidth}{X} 聽見的人都詫異牧羊人對他們所說的話。 \end{tabularx} \\ \\ \relax
2:19 & \begin{tabularx}{0.7\textwidth}{X} 馬利亞卻把這一切的事存在心裡,反覆思考。 \end{tabularx} \\ \\ \relax
2:20 & \begin{tabularx}{0.7\textwidth}{X} 牧羊人回去了,因所聽見所看見的一切事,正如天使向他們所說的,就歸榮耀於神,讚美他。 \end{tabularx} \\ \\ \relax
2:21 & \begin{tabularx}{0.7\textwidth}{X} 滿了八天,他們就給孩子行割禮,又給他起名叫耶穌;這是他還沒有在母腹裡成胎以前天使所起的名。 \end{tabularx} \\ \\ \relax
2:22 & \begin{tabularx}{0.7\textwidth}{X} 按摩西律法滿了潔淨的日子,他們就帶著孩子上耶路撒冷去,要把他獻給主。 \end{tabularx} \\ \\ \relax
2:23 & \begin{tabularx}{0.7\textwidth}{X} 正如主的律法上所記:「凡頭生的男子必歸主為聖」; \end{tabularx} \\ \\ \relax
2:24 & \begin{tabularx}{0.7\textwidth}{X} 又要照主的律法上所說,用一對斑鳩,或用兩隻雛鴿獻祭。 \end{tabularx} \\ \\ \relax
2:25 & \begin{tabularx}{0.7\textwidth}{X} 那時,在耶路撒冷有一個人,名叫西面;這人又公義又虔誠,素常盼望以色列的安慰者來到,又有聖靈在他身上。 \end{tabularx} \\ \\ \relax
2:26 & \begin{tabularx}{0.7\textwidth}{X} 他得了聖靈的啟示,知道自己未死以前必看見主所立的基督。 \end{tabularx} \\ \\ \relax
2:27 & \begin{tabularx}{0.7\textwidth}{X} 他受了聖靈的感動,進入聖殿,正遇見耶穌的父母抱著孩子進來,要照律法的規矩而行。 \end{tabularx} \\ \\ \relax
2:28 & \begin{tabularx}{0.7\textwidth}{X} 西面就把他抱過來,稱頌神說: \end{tabularx} \\ \\ \relax
2:29 & \begin{tabularx}{0.7\textwidth}{X} 「主啊,如今可以照你的話,容你的僕人安然去世; \end{tabularx} \\ \\ \relax
2:30 & \begin{tabularx}{0.7\textwidth}{X} 因為我的眼睛已經看見你的救恩, \end{tabularx} \\ \\ \relax
2:31 & \begin{tabularx}{0.7\textwidth}{X} 就是你在萬民面前所預備的: \end{tabularx} \\ \\ \relax
2:32 & \begin{tabularx}{0.7\textwidth}{X} 是啟示外邦人的光,是你民以色列的榮耀。」 \end{tabularx} \\ \\ \relax
2:33 & \begin{tabularx}{0.7\textwidth}{X} 孩子的父母因論耶穌的這些話就驚訝。 \end{tabularx} \\ \\ \relax
2:34 & \begin{tabularx}{0.7\textwidth}{X} 西面給他們祝福,又對孩子的母親馬利亞說:「這孩子被立,是要叫以色列中許多人跌倒,許多人興起;又要成為毀謗的對象, \end{tabularx} \\ \\ \relax
2:35 & \begin{tabularx}{0.7\textwidth}{X} 叫許多人心裡的意念顯露出來;你自己的心也要被劍刺透。」 \end{tabularx} \\ \\ \relax
2:36 & \begin{tabularx}{0.7\textwidth}{X} 又有位女先知,名叫亞拿,是亞設支派法內力的女兒,年紀已經老邁,從童女出嫁,同丈夫住了七年, \end{tabularx} \\ \\ \relax
2:37 & \begin{tabularx}{0.7\textwidth}{X} 就寡居了,現在已經八十四歲。她不離開聖殿,禁食祈求,晝夜事奉神。 \end{tabularx} \\ \\ \relax
2:38 & \begin{tabularx}{0.7\textwidth}{X} 正當那時,她進前來感謝神,對一切盼望耶路撒冷得救贖的人講論這孩子的事。 \end{tabularx} \\ \\ \relax
2:39 & \begin{tabularx}{0.7\textwidth}{X} 約瑟和馬利亞照主的律法辦完了一切的事,就回加利利,到自己的城拿撒勒去了。 \end{tabularx} \\ \\ \relax
2:40 & \begin{tabularx}{0.7\textwidth}{X} 孩子漸漸長大,強健起來,充滿智慧,又有神的恩典在他身上。 \end{tabularx} \\ \\ \relax
2:41 & \begin{tabularx}{0.7\textwidth}{X} 每年逾越節,他父母都上耶路撒冷去。 \end{tabularx} \\ \\ \relax
2:42 & \begin{tabularx}{0.7\textwidth}{X} 當他十二歲的時候,他們按著過節的規矩上去。 \end{tabularx} \\ \\ \relax
2:43 & \begin{tabularx}{0.7\textwidth}{X} 守滿了節期,他們回去,孩童耶穌仍舊在耶路撒冷。他的父母並不知道, \end{tabularx} \\ \\ \relax
2:44 & \begin{tabularx}{0.7\textwidth}{X} 以為他在同行的人中間,走了一天的路程才在親屬和熟悉的人中找他, \end{tabularx} \\ \\ \relax
2:45 & \begin{tabularx}{0.7\textwidth}{X} 既找不著,就回耶路撒冷去找他。 \end{tabularx} \\ \\ \relax
2:46 & \begin{tabularx}{0.7\textwidth}{X} 過了三天,他們發現他在聖殿裡,坐在教師中間,一面聽,一面問。 \end{tabularx} \\ \\ \relax
2:47 & \begin{tabularx}{0.7\textwidth}{X} 凡聽見他的人都對他的聰明和應對感到驚奇。 \end{tabularx} \\ \\ \relax
2:48 & \begin{tabularx}{0.7\textwidth}{X} 他父母看見就很驚奇。他母親對他說:「我兒啊,為甚麼對我們這樣做呢?看哪,你父親和我很焦急,到處找你!」 \end{tabularx} \\ \\ \relax
2:49 & \begin{tabularx}{0.7\textwidth}{X} 耶穌對他們說:「為甚麼找我呢?難道你們不知道我應當在我父的家裡嗎?」 \end{tabularx} \\ \\ \relax
2:50 & \begin{tabularx}{0.7\textwidth}{X} 他所說的這話,他們不明白。 \end{tabularx} \\ \\ \relax
2:51 & \begin{tabularx}{0.7\textwidth}{X} 他就同他們下去,回到拿撒勒,並且順從他們。他母親把這一切的事都存在心裡。 \end{tabularx} \\ \\ \relax
2:52 & \begin{tabularx}{0.7\textwidth}{X} 耶穌的智慧和身量,並神和人喜愛他的心,都一齊增長。 \end{tabularx} \\ \\
[1ex]
\hline
\hline
\end{longtable}
$^{1}$大家知道嗎?.
三千八百元座位的你們知道嗎?.
第一句想說的話是不是說聖誕快樂呢?.
我都想了一會.
我這樣說會不會很離地?.
所以我輕輕搜尋了一下,麻煩你拍拍手.
前兩年Full Church的聖誕崇拜.
說完到底是怎麼說的?.
大家有沒有印象?.
原來John是一個很好的教師.
他教了很多人.
大家有沒有印象?.
原來John是沒有說聖誕快樂的.
他一開始就介紹他的introvert 2000 pro.
Pawn Sir也是沒有說聖誕快樂的.
他說的是很開心我們終於有場地.
不受限制,可以100\%讓大家一起聚會.
到旁邊那個,大家認不認得?.
今天不在的,天目者.
他開頭第一句都是說,弟兄姊妹平安.
其實不說聖誕快樂.
已經是我們Full Church的傳統.
有人跟我說,報佳音還唱We Wish You a Merry Christmas.
我都不好好想一想.
他說大家想報一個怎樣的福音?.
我不是第一年聽到這個問題.
這個真的是一個很值得思考的東西.
所以今天.
其實我是不是應該選擇馬太福音更加適合呢?.
應是被追殺.
一個不平安的聖誕.
似乎這些.
是跟我們現在的生活更加相近.
起碼不像路加,是嗎?.
大家有沒有看過那段經文?.
他筆下的耶穌降生是揚日著歡喜快樂的.
有天使報大喜的訊息.
有牧羊人見證.
有一大隊天兵加入去到炸尾神.
之後還有寫兩個老人家.

$^{41}$有西冕和阿娜.
他在聖殿裡面稱頌著.
這個是路加福音所說的耶穌降生.
但是弟兄姊妹大家記不記得?.
又或者大家有沒有剛剛好.
在昨天的Discord裡面的靈修遊戲室.
又一起靈修了嗎?.
路加福音一開始就表示.
他的寫成是按著次序.
他說他為的是要讓人因此更加確信.
耶穌基督.
所以今天我想問的問題是.
路加這樣去記載耶穌降生.
在2023年的聖誕節.
我們怎樣看?.
我們今天會看兩小段經文.
第一段在路加福音2章10至12節.
那部分是.
經文一開始.
天使跟一群牧羊人說.
不要害怕.
因為那個時候有主的使者站在他們旁邊.
主的榮光四面照著他們.
牧羊人見到上帝的榮光.
他們就很害怕.
如果我們記得的話.
每一次救藥的人見到上帝的時候.
他們都是害怕的.
上帝一顯現他們就害怕.
大家有沒有印象?.
二菜啊,是不是?.
二菜啊,見到主坐在寶座上的時候.
他害怕得要死.
他說我有禍了,我要死了.
而這次是一群牧羊人見到主的使者.
他們就很害怕.
他們一群人都是非常害怕.
但弟兄姊妹,這次的害怕.
加多了一重的意思.
對於牧羊人來說.

$^{81}$他們可能怎樣也想不到.
原來有生之年.
可以有這番奇遇.
因為畢竟上帝的聲音.
彷彿停了四百年.
大家記不記得?.
繼先知馬拉基之後.
已經很久很久沒有先知出來.
再傳達上帝的話.
也很久很久沒有見到上帝的工作.
沉默了幾代人的日子之後.
牧羊人見到上帝再次出手工作.
十至十二節是這樣說的.
他說:天使華康啊.
我報給你們大喜的消息.
是關於萬民的.
今天在大衛的城裡為你們生了救主.
就是主基督.
你們要找到一個鷹鞋.
花雀布,我像馬槽裡.
那就是記號了.
這群牧羊人一聽到大衛的城.
一聽到救主,一聽到主基督這些關鍵字.
他們的天線立刻豎起了.
因為他們和所有的以色列一樣.
他們一直在等.
他們一直在等上帝會實現.
曾經藉著先知所說的英許.
那個英許大家都知道.
在大衛教會有一個苗裔.
成為他們永遠的王.
成為以色列人永遠的王.
所以當天使天兵離開了之後.
我們看十五至十七節.
牧羊人就立刻跑去伯尼行.
去找到馬槽中的那個鷹鞋.
當他們找到之後.
他們就說這件事是真的.
他fact check了.
上帝真的為我們成就了這件事.

$^{121}$他們就出去跟很多人說.
弟兄姊妹.
他們見證到的是.
上帝是一個說得出做得到的上帝.
他曾經的英許.
現在真的實現了.
他們又見證到的是.
上帝工作的時代.
隔了幾百年.
他們又再一次來到了.
上帝再一次出手工作.
這個我們都很明白.
我們已經聽了幾十年.
應該沒有的.
十年?幾年而已.
有些報佳音的事情.
但我發覺可能我們都有一個迷思.
覺得耶穌降生是二千多年前的事.
是上帝二千多年前的工作.
聖誕節紀念的是什麼?.
就是紀念二千多年前耶穌降生.
聖誕節對我們現在有什麼意義?.
除了我們可以連續放假之外.
熟悉一點說的.
我們有些事是熟悉一點說的.
熟悉一點說的就是.
紀念神很愛我們.
曾經在很多很多年前.
為神的兒子.
應該說是讓神的兒子.
為全人類降生.
所以一到聖誕我們就會想.
不如我們回教會.
大家千萬不要對它合作.
在我心裡.
沒有一個畫面出現.
沒有一個人出現.
最多是我媽媽.
因為以前我經常用聖誕聚餐來理由我媽媽.
不如你一年都回來一次.

$^{161}$聖誕節好像是什麼呢?.
好像是用來紀念的東西.
紀念曾經發生過的事.
紀念一些過去的事情.
一些事情我們不想遺忘.
所以我們有聖誕節.
我們紀念它.
就好像.
我今天本身拿了兩個磁石貼過來.
我忘記拿.
那些磁石貼就是我每一次去旅行的時候.
我都會去記住那趟旅程.
例如一個是澳洲的.
一個是英國的.
在我的袋子裡.
但不要緊.
因為最重要的是什麼呢?.
我會再去.
我下一次旅行去哪裡?.
我下一次旅行是何時?.
同樣地.
頂尖我們活在此時此刻.
不是活在二千多年前.
我們活在此時此刻的時候.
我們會繼續去祈禱等候.
我們等待上帝下一次出手的工作是什麼?.
我們最有信心的應該是.
我們盼望將來的耶穌.
第二次來臨.
再回來.
我們最有信心的就是.
我們會有一個幕後的審判.
另外的就是我們生活當中的事情.
我們祈禱上帝出手.
還有這幾年.
我們有很多很多集體祈禱的時間.
我們同心為大家一起經歷的事情祈禱.
至於上帝有多少出手呢?.
都是看上帝.
看上帝的主權.

$^{201}$看上帝的時間.
這是我們這幾年一直學習的功課.
是一個很好的.
是一個馴服等待的態度.
是很好.
因為上帝才是上帝.
我們是人.
但其實當我思考.
天使報很消失的這段經文的時候.
我是有一種想法浮現.
想跟大家分享一下.
其實上帝二千多年前的一次出手.
派耶穌來.
就已經足夠了未來所有時間的人.
包括現在的我們.
我再說一次.
就是上帝二千多年前的一次出手.
派耶穌來.
就已經足夠了未來所有時間的人.
包括我們和未來的人.
足夠了的意思是什麼呢?.
足夠的意思就是.
其實我們不用等再多都可以.
不需要更多都可以.
上帝那一次出手.
就跨時代地救了我們.
直到耶穌再次回來.
第一節我們想到保羅.
他說過一段很刻骨銘心的自白.
是《肥拿比書》三章.
其中第八節是這麼說的.
他說「不但如此」.
麻煩拍一下.
「不但如此,我又將萬事當作有損的,因我已認識我主基督耶穌為至寶,我為他已經凋棄萬事,看作糞土為要得著基督.」.
我剛剛說的是.
有耶穌這一手就足夠了.
但保羅更厲害.
他說其他東西都是垃圾.
只要認識耶穌就是至寶.
之後他解釋.

$^{241}$在三章十至十一節.
「使我認識基督,曉得他復活的大能,並且曉得和他一同受苦後發他的死,或我都得以從死裡復活.」.
我要解釋一下.
對他來說.
因為有耶穌降生這一手.
讓他可以認識到耶穌是怎樣的.
認識了之後.
他才明白兩件很重要的事.
第一,耶穌復活的大能.
第二,他知道要和耶穌一同受苦,後發他的死.
他說唯有這樣才有機會在死裡復活.
弟兄姊妹,我們現在記住這件事.
我們從另一個角度來看保羅.
大家會不會問一個問題.
保羅有沒有見過上帝會再一次出手.
有.
在《史獨行傳》中記錄.
他說被困在被囚.
被兩條鐵鏈捆綁住.
然後天使拍拍他.
鐵鏈就掉下來.
然後鐵門就像《多拉伊夢》的自動門.
自動打開.
然後徐燕門.
不知道是什麼門.
神蹟一樣很自由地可以放出來.
所以保羅是知道的.
他也知道上帝要出手的時候就會出手.
但對他來說.
這卻不是他所看的重點.
弟兄姊妹,有一樣很重要的事.
我想分享的是.
他經歷被打失去自由的時候.
他不是依靠上帝的特別出手.
他是依靠早就被耶穌基督.
在《肥納比書》他自己說.
只要認識耶穌.
就明白復活的大能是怎樣的.
他說單單耶穌來了.
自己可以認識到祂.

$^{281}$他就明白自己是要和祂一起受苦的.
他保羅自己是要效法耶穌的死.
這是他寫給教會的心聲自白.
說到這裡我分享一個小故事.
每一次我講flowchurch的道.
我都會先跟老公講.
這次他也有聽我說.
然後講到這一部分的時候.
他就說「好吧,有耶穌就夠了」.
「好耶,不用回教會了」.
「有耶穌就行了」.
「我也有耶穌,那還好吧」.
他就這樣跟我說.
所以因為他.
我就覺得會不會大家也有這些想法呢.
所以補充一句.
我要補充多一件事.
這也是保羅流著淚補充給教會聽的.
就在下一節,三章十二節.
保羅說.
「這並不是說我已經得著了」.
「已經完全了」.
「而是竭力追求」.
「好像我可以得著」.
「基督耶穌要我得著」.
保羅是覺得自己不夠多認識耶穌.
不是一個耶穌不夠.
而是自己不夠認識耶穌.
所以他說他要竭力追求.
弟兄姊妹我們這幾年其實也有很多信仰反思.
我相信每一個來Flow Church的弟兄姊妹.
可能他們會有更多.
你們會有更多的信仰反思.
所以你們會來到這裡.
所以現在我問一問大家.
我也問一問我自己.
耶穌對於我們來說.
是否已經足夠.
我們的信仰的心聲自白.
會是甚麼.

$^{321}$我們看看今天的第二段經文.
我再告訴大家.
我的心聲自白是甚麼.
第二段經文.
是寫耶穌作為小孩的時候.
終於踢到他小孩了.
因為之前其實也是嬰兒.
我看過網上的留言.
都是嬰兒.
因為他寫到小孩的時候.
但非常特別的是.
為何路加會寫耶穌降生的時候.
是寫到他12歲的時候才寫完.
我會問為甚麼.
為甚麼會這樣.
大家有沒有見過我們做耶穌降生的劇的時候.
除了塑膠嬰兒之外.
還有一個12歲的小孩會出來做這個角色.
我是沒有見過.
但作者路加是按次序的.
所以他寫12歲的耶穌的時候.
一定有他的意思.
他是怎樣寫的.
他說這個小孩跟著父母去聖殿的時候.
他不跟他們說話.
自己留在聖殿裡.
要父母瑪利亞和約瑟.
很擔心地找了他們三天.
其實在原文裡三天不是真的三天.
是說一段時間.
路加就是用這件事去完成耶穌降生這個記載.
我們一起分析一下.
耶穌失蹤這件事其實有兩個高潮.
一個就是找到耶穌之後.
發現他竟然坐在聖殿的老師當中對答.
就好像這張圖.
這個高潮是想說耶穌有超乎12歲小孩的智慧.
甚至是超乎在場所有老師的智慧.
這張46節是這樣記載的.
他說耶穌坐在教師中間.

$^{361}$一面聽一面聞.
不是每個人都可以坐.
我本身也可以坐但最後沒有坐.
因為我覺得自己不配.
不是每個人都可以坐.
本來是老師才可以坐.
現在的老師才這麼慘不能坐.
在課堂上只能站.
經文很仔細地形容.
一面聞一面聽.
他在說耶穌不是受教的學生.
而是在跟他們交流.
甚至是在教這班作為老師的人.
大家有沒有曾經腦中問過一個問題.
就是他們這樣圍著圈子.
他們在說什麼.
中文老師應該是在說中文.
IT老師應該是在說電腦程式.
在聖殿的老師其實應該是在說上帝的律法.
所以最後這張47節路加寫.
所有聽見黑的人都稀奇他的聰明和應對.
對於猶太人來說.
13歲是大人.
但12歲還是個小孩.
路加上等的高潮是.
耶穌這個還是12歲的小孩.
就已經超級有智慧.
是世界上最明白上帝律法的人.
我們記住這一點.
而另一個高潮.
應該說是最最高潮的就是.
耶穌出生之後說的第一句話.
也是他小時候唯一被路加記載的說話.
我們經常都學一件事.
上帝的說話很重要.
是腳前的燈是路上的光.
創世紀說.
神說要有光就要有了光.
而路加記載.
耶穌的第一句說話是.

$^{401}$為什麼你們要來找我.
你們不知道我必須以我父的事為念嗎.
以章49節.
他說.
為什麼要來找我.
第一個高潮.
突顯耶穌的智慧.
所以我們會知道.
這裡不是想批評耶穌態度差博咀.
反而是一直住這一刻.
你猜都猜不到他會這樣說的話.
去衝擊所有聽到的人和讀者.
所以這句話.
其實是路加記載最高潮的.
耶穌說.
我必須要以我父的事為念.
第一我們會知道.
耶穌要告訴瑪利亞和約瑟.
自己真正的父是誰.
不是約瑟你.
不是你.
是父上大人.
這個我們明白的.
誰是我們真正的父.
當我們信主的時候.
我們就知道.
我們是以上帝為我們的父.
第二件事.
有少許原文避不了.
要輕輕解釋一下.
我們手頭上可能有兩個譯本.
一個是譯作我必須在我父的家裡.
另一個是我必須要以我父的事為念.
因為原文當中.
我不想改得太.
我的「的」是一個 plural form.
所以我們認為.
必須以我父的事為念.
是更有說服力的.
如果大家想了解一下.

$^{441}$可以之後再問.
但反而今天想問的是什麼.
是父的事是什麼事.
耶穌所說的必須.
又是什麼必須.
有什麼是必須想念著的事.
這個才是我們真正要著緊的.
路加到最後.
都要說一個十二歲的耶穌出來.
在這個高潮.
我們就要問.
祂說了什麼.
我們看看路加福音.
記載著什麼是耶穌的必須.
因為原來在路加的筆下.
必須是一個很重要的詞語.
經文有些多.
我們一起進入耶穌的一生.
看看祂是如何實踐.
祂的孩童時候.
說過那句說話.
那句必須.
當耶穌行了很多神蹟之後.
眾人都想留住祂.
在路加福音4:43.
耶穌就跟他們說.
我必須再別承傳上帝國的福音.
因為我奉差緣而為此.
當彼得指耶穌是神的基督的時候.
9:22 耶穌說.
人子必須受許多的苦.
被長老濟斯掌王民氏氣絕.
並且被殺第三日復活.
到了中期.
當耶穌解釋神的國是怎樣的時候.
17:25 耶穌說.
只是他必須先受許多苦.
又被這世代氣絕.
到了最後晚餐的時候.
22:37 耶穌說.

$^{481}$我告訴你們.
經常寫著說.
他被列在罪犯之中.
這話必應驗在我身上.
因為那關係我的事.
必然成就.
到復活之後.
天使讓耶穌屍體的婦女回想起.
耶穌說過的一句話.
24:7 他說.
人子必須被交在罪人手裡.
釘在十字架上.
第三日復活.
到了最後.
在二萬五師的路上.
復活了的耶穌說.
24:26-27 他說.
基督怎樣受害.
又進入他的榮耀.
豈不是應當的嗎.
於是從摩西和眾仙之起.
凡經上所指著自己的說話.
都幫他們講解明白.
原來十二歲的耶穌說.
你們不知道.
我必須以我父的事為念嗎.
我的必須.
我的應當.
是他要傳上帝國的福音.
是他要先受害.
再進入他的榮耀.
弟兄姊妹.
他由降生.
就知道自己的命是怎樣過的.
但我們好像不知道.
我們的命是怎樣過的.
一部份我們會很想知道.
未來的自己會是怎樣的.
我聽過就算是.
我們作為基督徒.

$^{521}$有時都會看看星座.
我們會看看上座.
做下心理測驗等等各樣.
有一部份我們會很害怕.
很害怕未來.
讓人恐懼的社會.
從來都未試過.
禁不下2024年來到.
從來都未試過.
禁不下新的一年來到.
往年總會有一些.
對未來期望的問卷調查.
調查我們對未來.
有多少分的希望.
還是盼望指數.
但今年去到年尾.
都好像不覺有.
我還未收到這些電話.
因為我們知道他一定不會做.
因為做了一定會破身低.
所以究竟我們的命是怎樣.
第一節我們一起看看.
《徹行傳》兩段的經文.
因為除了耶穌的必需.
路加同樣記載了.
屬於我們的必需.
9章15至16節.
主德阿拿尼亞說.
你只管去.
你是我揀選的器皿.
要在愛邦人和君王.
並以色列人面前宣揚我的名.
我也要指示他.
為我的名必須受許多苦難.
這個不用怕.
是跟保羅說的.
14章22節.
就是跟我們所有人說.
堅固門徒的心.
勸他們行守所信的道.

$^{561}$有說避不了.
我們進入上帝的國.
必須經歷許多艱難.
這個是跟我們說的.
為基督耶穌的名.
必須受許多苦難.
進入上帝的國.
必須經歷許多艱難.
我們不需要對未來感到不確定.
因為這個就是我們的必需.
這個是聖經跟我們說的.
我們有時候可能會有僥倖的心態.
未必是我的.
特別是當我們不祈求上帝的經歷.
我們就不會遇到甚麼事.
有時候我們可能會有逃避心態.
我們很不想有這些艱難.
上學院的時候有一堂.
說到人總是很想去shake off.
shake off所有的痛苦和suffering.
當我的痛苦和艱難.
好像一堆堆黃色的面膜紙.
貼到我們全身都是的時候.
我們就很想甩開它.
用盡全力甩開這些黃色的貼紙.
除非有自虐傾向.
我們也不是M底.
我們自然也不會想有艱難.
我們祈禱通常都是祈求主幫我們脫離災禍困境.
但是弟兄姊妹.
這些苦難和艱難本來就是基督的必需.
如果我們能接受它.
我們就不是祈求上帝再一次幫我們.
出手甩開它.
而是祈求耶穌基督幫我們經歷過這些艱難.
焉知道這些痛苦可以成就的是甚麼事情.
因為到現在我仍然看到.
只是經歷艱難的人的安分力量.
就好像有人叫我們繼續寫信給他一樣.
因為他知道自己為了甚麼而受苦.

$^{601}$就像每一個母親.
作為媽媽為了生孩子.
她受十級的苦都可以.
保羅面對一切的鞭打苦待.
他沒有祈求上帝再一次出手.
他教導教會的是認定認識耶穌基督就夠.
救他一生知道怎樣走.
有力量去走.
救他救到自己的命.
息經息到這裡.
大家可以聽我分享我的感受.
我們這幾年失去了很多.
有很多新聞Google現在也搜尋不到.
也搜尋不到.
有很多人事物很珍貴很珍貴的東西沒有了.
損害了.
剛剛說到一個很怕2024年來臨的人其實就是我.
我自己30歲的時候都沒有怕過這個關口.
但是明年2024年我就怕.
我不想面對.
我很怕去經歷.
但是無論我們失去了甚麼也好.
耶穌基督是永遠永遠都不會失去.
除非是我們拋棄了他.
如果不是的話.
這個最重要最重要的耶穌基督.
只要我們抓緊了.
就夠了.
這個就是我的大喜訊息.
我一生當中沒有東西比這個訊息更令我大喜.
不是結婚.
Sorry Ryan.
Sorry老老公.
不是有一份好工作.
Sorry Flow Church.
因為耶穌降生.
意志的是我們可以在一個擁有耶穌的時代裡面.
一個擁有耶穌的時代.
救我們面對香港的新時代.
這個救是因為我們能夠以必須受的苦.

$^{641}$去面對我們不想受的苦.
我多謝12歲的耶穌.
他給我生命裡面最大的智慧和方向.
人生不知為何亂過一通.
短短幾十載.
不知活成什麼樣子就死去了.
人生真的很快.
這是我睡不著覺思考這段經文的時候想的東西.
以負的事為念這樣去走.
其實都不會走很久.
但如果我們真的這樣走的話.
我們就真的抓得住這個大喜的訊息.
這個小孩告訴我了.
他的路是怎麼過.
降生時候的一句話.
就是他的一生.
弟兄姊妹.
耶穌降生的意義並不是.
或者並不只是聖誕節前後那幾天.
紀念他為人類降生.
不是24,25號.
而是人生當中的每一個時刻.
我們怎麼活.
我們有時候都會陷進一個想法裡面.
包括我自己也是.
祈禱的時候求上帝幫助出手.
工作同在.
這幾年大家在Full Church聽了幾個篇.
回應這些問題.
我們又唱了幾首歌.
敬拜裡面我們向神呼喊.
我們坦承我們的懼怕.
之後我們又堅定地相信.
不過今天的聖誕崇拜.
講耶穌降生的時候.
老家告訴我們.
這個給普天下人的大喜訊息.
上帝已經工作了.
而不是我們現在.
只是單單等待上帝的工作.

$^{681}$上帝拯救已經賜下了.
救人生命的主都給了.
那條生命的路.
耶穌都已經向我們表明了.
聖靈都賜下給我們了.
稱得上為普天下的大喜訊息.
都應該只有這樣.
打機來說.
已經齊備了裝備.
那我們還要些什麼呢.
一切都夠了.
救我們面對一切現實和未來的風浪.
救我們過我們夢照的必需.
讓我們一起祈禱.
天父上帝多謝你.
多謝你賜下耶穌基督.
來到這個世界上.
這是一份絕大的恩典.
和一份大喜的訊息.
是由我們深深處的當中去覺得.
在我們生命裡面.
沒有一樣東西是比耶穌基督更加寶貴的.
所以究竟什麼叫做有耶穌就夠.
求主你教導我們.
什麼叫做有耶穌就夠.
好像保羅一樣.
看其他東西都是有損的.
只以認識耶穌基督為至寶.
如果我們還沒經歷過.
主求你教導我們.
求你幫我們這樣去經歷.
因為我們已經得到了.
我們就不想白白好像沒有得到過一樣.
所以天父上帝求你幫我們.
我們禱告奉主耶穌基督.
得勝明志祈求.
阿們.
\newpage



\section{耶利米書 1:1-8-20231230}
\label{sec:9ztySs_vnP4}
\textbf{【網上崇拜】一道風景,一種心靈|耶利米書1\_1-8|20231230 [9ztySs-vnP4]}
\newline
\newline
連結: \href{https://youtube.com/watch?v=9ztySs-vnP4}{\texttt{ https://youtube.com/watch?v=9ztySs-vnP4}} ~~~~ 語音日期: 2023-12-30 
\newline
\newline
\hyperref[sec:dT3dN2jF8BQ]{\small{< < < PREV SERMON < < <}}
~
\hyperref[sec:index_chronic]{\small{[返順時目]}}
~
\hyperref[sec:index_scriptual]{\small{[返順卷目]}}
~
\hyperref[sec:code]{\small{> > > NEXT SERMON > > >}}
\newline
\newline
耶利米書 1:1-8-20231230
\newline
\begin{longtable}{cl}
\hline
\hline
章節 & 經文 (和合本修訂版)\\
\hline
1:1 & \begin{tabularx}{0.7\textwidth}{X} 這些是便雅憫地亞拿突城的祭司,希勒家的兒子耶利米的話。 \end{tabularx} \\ \\ \relax
1:2 & \begin{tabularx}{0.7\textwidth}{X} 亞們的兒子猶大王約西亞在位第十三年,耶和華的話臨到耶利米。 \end{tabularx} \\ \\ \relax
1:3 & \begin{tabularx}{0.7\textwidth}{X} 從約西亞的兒子猶大王約雅敬在位的時候,直到約西亞的兒子猶大王西底家在位的末年,就是第十一年五月間耶路撒冷被擄時,耶和華的話也常臨到耶利米。 \end{tabularx} \\ \\ \relax
1:4 & \begin{tabularx}{0.7\textwidth}{X} 耶利米說,耶和華的話臨到我,說: \end{tabularx} \\ \\ \relax
1:5 & \begin{tabularx}{0.7\textwidth}{X} 「我尚未將你造在母腹中,就已認識你;你未出母胎,我已將你分別為聖,派你作列國的先知。」 \end{tabularx} \\ \\ \relax
1:6 & \begin{tabularx}{0.7\textwidth}{X} 我就說:「唉!主耶和華,看哪,我不知道怎麼說,因為我年輕。」 \end{tabularx} \\ \\ \relax
1:7 & \begin{tabularx}{0.7\textwidth}{X} 耶和華對我說:「不要說『我年輕』,因為我差遣你到誰那裡去,你都要去;我吩咐你說甚麼話,你都要說。 \end{tabularx} \\ \\ \relax
1:8 & \begin{tabularx}{0.7\textwidth}{X} 你不要怕他們,因為我與你同在,要拯救你。這是耶和華說的。」 \end{tabularx} \\ \\ \relax
1:9 & \begin{tabularx}{0.7\textwidth}{X} 於是耶和華伸手按住我的口,對我說:「看哪,我已將我的話放在你口中。 \end{tabularx} \\ \\ \relax
1:10 & \begin{tabularx}{0.7\textwidth}{X} 我今日立你在列邦列國之上,為要拔出,拆毀,毀壞,傾覆,又要建立,栽植。」 \end{tabularx} \\ \\ \relax
1:11 & \begin{tabularx}{0.7\textwidth}{X} 耶和華的話臨到我,說:「耶利米,你看見甚麼?」我說:「我看見一根杏樹枝。」 \end{tabularx} \\ \\ \relax
1:12 & \begin{tabularx}{0.7\textwidth}{X} 耶和華對我說:「你看得不錯;因為我要看守我的話,使它實現。」 \end{tabularx} \\ \\ \relax
1:13 & \begin{tabularx}{0.7\textwidth}{X} 耶和華的話第二次臨到我,說:「你看見甚麼?」我說:「我看見一個水燒開的鍋,從北而傾。」 \end{tabularx} \\ \\ \relax
1:14 & \begin{tabularx}{0.7\textwidth}{X} 耶和華對我說:「必有災禍從北方發出,臨到這地所有的居民。 \end{tabularx} \\ \\ \relax
1:15 & \begin{tabularx}{0.7\textwidth}{X} 看哪,我要召北方列國的萬族。這是耶和華說的。他們要來,各安寶座在耶路撒冷的城門口,周圍攻擊城牆,又要攻擊猶大的一切城鎮。 \end{tabularx} \\ \\ \relax
1:16 & \begin{tabularx}{0.7\textwidth}{X} 這民離棄我,向別神燒香,跪拜自己手所造的,我要針對這一切惡行,向他們宣讀我的判決。 \end{tabularx} \\ \\ \relax
1:17 & \begin{tabularx}{0.7\textwidth}{X} 所以你當束腰,起來,將我所吩咐你的一切話都告訴他們;不要因他們驚惶,免得我使你在他們面前驚惶。 \end{tabularx} \\ \\ \relax
1:18 & \begin{tabularx}{0.7\textwidth}{X} 看哪,我今日使你成為堅城、鐵柱、銅牆,對抗全地和猶大的君王、官長、祭司,並這地的百姓。 \end{tabularx} \\ \\ \relax
1:19 & \begin{tabularx}{0.7\textwidth}{X} 他們要攻擊你,卻不能勝過你,因為我與你同在,要拯救你。這是耶和華說的。」 \end{tabularx} \\ \\
[1ex]
\hline
\hline
\end{longtable}
$^{1}$我們從歲首到年終都敬拜主.
上次跟大家說到,不知道大家記不記得.
我不是回來追數.
你們許了的願應該自己去還.
上次跟大家思想薩姆爾.
薩姆爾是上帝興起的先知.
他是準備建立以色列國的.
今天跟大家思想另外一個先知.
是耶利米.
他是上帝興起準備宣告.
以色列要亡國.
一頭一尾,上帝都興起先知.
耶利米先知大概二十歲左右蒙召.
他見證著國家的興衰,戰敗.
經歷了四十多年多.
他到六十歲的時候.
他與他的國民一同被擄.
這個先知又經歷人生另一個階段.
今天我會透過導讀的方式.
我們會找幾節經文.
嘗試掌握耶利米書.
甚至是這位先知的人生.
耶利米的蒙召我們無法模仿.
因為他是很特別的蒙召.
但他經歷過的事.
他如何走過這些艱難的路.
我們是可以學習的.
今天想跟大家思想一個主題.
有人說人生好像坐在一部列車一樣.
環繞我們的風景是不斷轉換.
我們的經歷,我們的際遇都不斷變.
風景轉了.
你的心境,你的心靈有沒有跟著轉呢?.
今天是最後的一天.
在敬拜裡面.
我都希望在2023年結束的時候.
跟大家思想一道風景,一種心靈.
我們先讀經文.
經文應該在PowerPoint裡面大家看到.
我們會讀這八節經文.

$^{41}$如果大家都看到的.
都請你為我讀出.
預備,1,2,3.
我們今天會先理解經文.
然後在第二個階段.
會盡心去想一想.
究竟什麼叫一道風景,一種心靈.
最後我會就著這段經文對我自己的提醒.
成為彼此的鼓勵.
我們先理解經文.
剛才經文的一至三節是全卷耶利米書的引言.
耶利米先知經歷了40年的侍奉的履歷.
裡面的一至三節有幾樣東西我們需要知道.
有幾個背景的資料.
第一,耶利米出生在阿拿德.
這是祭司之城.
是利美人所居住的地方.
上帝所選擇的是出自正統的.
耶利米是出自正統的祭司血統.
第二就是耶利米這個名字.
如果你有看舊約聖經的話.
你會發現這個名字很普遍.
大概出現了十次不同的耶利米.
大概就像阿強一樣.
有很多阿強在.
這個名字的意思就是願神尋得他的居所.
是神同在的另一種表達.
耶利米的意思就是神同在.
在場的另一種表達.
然後第三個資料.
我們會發現耶利米先知侍奉了最少有三代的王朝.
第一代就是約西亞.
大概他二十歲初蒙召的時候.
如果你熟悉聖經.
你知道約西亞是一個好君王.
他打算扭轉國家的命運.
除去偶像大興土木.
耶利米出道的時候.
他遇到一個好上司.
正是他發揮的時候.

$^{81}$不過耶利米書沒有記載這段他稍為光輝的日子.
反而他集中記載.
約西亞的兩個兒子.
約亞勁和西底加.
這兩個君王是國家最敗壞的時候.
上帝就要在這段日子宣告.
以色列國將要滅亡.
他要人民接受一個不能夠接受的信息.
正是耶利米書所記載大部分的信息.
去到三十七章裡面有兩句更加這樣說.
他說當時的人民百姓全部都不聽耶和華籍耶利米先知所說的話.
即是過去四十年他的侍奉都沒有人聽.
這個正是耶利米書所記載的一個主要內容.
然後我們看到另外一個撤婚的時段.
就是十一年五月國家戰敗.
耶利米又要成為一個被擄的人.
如果我們概括去看耶利米的人生.
二十歲到六十歲.
侍奉了四十年.
由一個好的君王.
打算大興土木.
即是一展拳腳.
到很快轉為兩位壞的君王.
他需要向人民宣告一個沉重的信息.
然後國家戰敗被擄.
他要向人民說你們要用心.
七十年而已忍一下.
這樣就回歸了.
不同的環境令到這位先知所宣告的都變得不一樣.
如果我們看耶利米書.
你會發現常有一句說話成為書卷的分割.
就是耶和華的話臨到耶利米.
每一次上帝出聲.
這位先知的人生也要作新的改變.
配合上帝做新的事.
一至三節是整本書的引言.
第四至八節是回到他第一次蒙召.
即是二十歲的時候.
神怎樣第一次呼召他.
如果我們熟悉聖經.

$^{121}$就是神每一次向人呼召.
都是很轟烈的.
耶利米也不例外.
神向耶利米呼召.
跟他說了幾句說話.
是甚麼.
神做他的.
神曉得的.
神分別他為聖.
神派他的.
即是說他不能夠擁有自己的人生.
他從母福出來開始.
他已經被分別出來.
去做神的事.
對於一個二十歲的年輕人.
他聽見這麼沉重的呼召.
他就好像很多先知一樣.
就馬上撒手.
說我不行的.
耶利米推搪的原因是甚麼.
我年幼.
我未有這樣的資歷.
未有這樣的經歷.
去面對一個這麼大的呼召.
但如果我們看得夠多.
你就知道.
所有合理的理由.
在上帝的呼召裡面.
都不成理由.
摩西說我絕口不說.
耶利米說我年幼.
所有的理由合理的.
在人間裡面是正確的.
但對上帝來說.
這些推搪的.
都不構成理由.
所以七至八節.
神再次跟他說.
是因為我猜你.
我吩咐你.

$^{161}$耶利米擔心的就是.
神給他的東西.
他做不來.
搞不定.
神怎樣跟他說.
我就是根本給了一樣.
你一生都做不來的東西.
給你做.
你一生都說不服那班人.
我呼召你.
就去做一件.
你一生都不能夠完成的事.
這個是耶利米的呼召.
神不是要先知.
為他做些甚麼.
不是神有些東西搞不定.
耶利米你好像厲害一點.
交給你幫忙.
不是.
從撒姆爾到耶利米.
我們都清楚一件事.
上帝是呼召一個聽命的人.
遠多於一個幫他完成.
某件事的人.
聽命是最重要的.
不是那件事能不能夠做到.
讓你有甚麼成就感.
呼召的本意是.
你願不願意服在上帝面前.
哪管你做不到.
你都願意聽.
這個是呼召的本意.
上帝能夠用得到的人.
這個就是最成功.
就是這麼簡單.
經文看完了.
我嘗試為這段經文做一個小結.
然後我們進入思考.
究竟甚麼是風景.
甚麼是心靈.

$^{201}$這段經文提醒我們一些事情.
第一就是.
每當耶和華的話臨到耶利米.
他人生就要轉一個場景.
他要面對一個新的局面.
他要用一種新的心態.
去應對他周圍的事情.
他要很敏銳上帝的話.
這是幾節的經文提醒我們.
第二個就是.
耶利米大部分的人生.
都在人間看來是沉重和失敗.
甚至連他自己都跟神說.
可不可以不要這樣下去.
這些生活太艱難.
但他自己又回答自己.
如果我不做神的事.
我心就好像被火燒一樣.
他為在兩難之間.
是這個先知人生所經歷.
然後還有一樣.
在他大部分的侍奉日子.
他都是勸服當時的人民.
請你們轉換一個新的心靈.
我們現在是被擄的.
我們是過艱難的日子.
他要說服周圍的人.
去接受這種環境.
這是大概整卷耶利米書給我們的提醒.
經文講完了.
我們又嘗試想一想.
在這麼劇烈的世代裡.
這麼多環境的轉變裡.
我們又在耶利米先知裡.
學到什麼呢.
首先來說一下什麼叫風景.
一道風景.
正如我們開頭所說.
人生就像坐在一部列車.
眼見的風景,經歷,際遇,友誼.

$^{241}$不斷地改變.
除了這個時代的巨輪改變.
就好像耶利米先知.
經歷,戰敗,被擄.
時代的改變.
其實在人生裡面.
你會否發現還有另外幾種改變.
是常在我們中間出現的.
第一就是順其自然的成長.
古語有云三十而立.
心理學也說我們有嬰兒期,青春期.
電視也說人生有多少個十年.
你可能還有很多個十年.
我就倒數剩下多少個十年.
我們的人生是不斷地改變.
這是自然的成長.
由升學提醒我們.
到轉坐兩元車又提醒我們.
我們已經不同了.
這是客觀的真實.
除了順其自然的改變之外.
現在的世代也鼓勵我們做人生規劃.
你也可以規劃一下自己.
在不同的年紀,不同的階段去做不同的事.
在學的時候有老師教我們人生規劃.
到了差不多退休.
我們也計算一下.
有多少老本可以繼續活下去.
還要計算自己有多長命.
這條數很難計算.
究竟長命好一點還是怎樣.
不過真正讓我們經歷到人生的轉變.
不是自然的改變.
也不是人生的規劃.
而是一些突如其來的經歷.
例如中了.
不論你是中了重病.
又或者是中了頭獎.
這些你掌握不到的事情.
它會將我們的人生一分為二.

$^{281}$以前沒有病沒有痛.
就吃吃喝喝.
大部分男士都不會去身體檢查.
誰知道檢查了什麼.
我以後怎樣生活.
我們是這樣的.
所以各位太太請你體諒.
我們一檢查就差不多玩完了.
人生就分割了.
以後想吃的全部都吃不到.
人生變得很悲哀.
我們是這樣的.
真正的改變不是那些順其自然.
而是我們掌握不到.
頂尖我想說的第一件事.
關於風景的轉變.
就是不論你喜歡不喜歡.
主動或者被動.
有心或者無意.
這些場景是不斷轉換的.
有時是不知不覺.
有時這些轉換是叫我們措手不及.
所以當我去想這個情況的時候.
人生的風景不斷轉變.
你需要問自己兩個問題.
第一 你究竟身在何處.
你周圍的環境是怎樣的.
你和我有沒有認真去看一看.
我們現身處身的環境.
不是幻想那種.
不是回憶那種.
眼前這種實況是怎樣的.
第二 你處身在這種環境裡面.
你還有什麼本領是用得著的.
有沒有一些東西已經過時了.
不適合用的.
大家還會不會用那些按鍵的電話.
電話沒有壞的 還可以按的.
但你會不會用的.
不會用的 為什麼.

$^{321}$我家裡的MacBook全新的.
我保存得很好.
但我都要賣掉它.
為什麼.
它追不到上網.
它每一次更新都搞了半個小時.
它不是有問題.
但大環境轉變了.
它不能夠再用.
這就是第一 我想和大家去想一想.
我們的風景.
我們處身的景況.
有些人與事.
是不是已經不同了.
而你經歷了不同的皇朝.
他經歷興衰.
他很清楚每個階段要配合上帝做什麼.
這是我們第一樣要發現.
看清楚周圍的環境.
究竟我們置身在什麼情況裡面.
如果你發現環境改變了.
接下來我要問第二個問題.
我們的心靈追不追得上.
或者是否能夠配合到.
這種不太心甘情願的環境.
這是重要的事情.
以往我們會用人生規劃.
人生上半場下半場去理解.
風景和心靈的變化.
不過當牧羊的日子內.
自己的年紀也成長.
我發現這個道理很難應用在真實的生命裡面.
我發現正如剛才所說.
真正令我轉變的不是我預計裡面的.
是完全出乎我意料的.
我發現當我再進入下半場的時候.
我已經不是踢同一個球.
我不是與同一班隊友.
對付同一批的對手.
甚至整個規矩都變了.

$^{361}$我不是可以輕易帶上半場的經驗.
去進入我的下半場.
是兩種的理解.
很多的改變不是輕易就這樣跨越得到.
問問前輩拿經驗.
是不可以的.
我的經驗是不適合下一代去用的.
我傳授不了很多東西給他去跟著做.
原來是不可以的.
人生未必是上下半場之分.
舉幾個例子.
假如你結了婚.
兩個人生活.
平時兩個人上班.
下班最多買盒兩sung 飯回去吃.
現在太太有嬰兒.
準備生多一個.
你會怎樣做?.
買盒三sung 飯?.
加多一個菜?.
加雙筷?.
我想不是吧.
整個人生都不同了.
前幾天我很早出門.
坐電梯碰見一位婆婆.
她回家.
在電梯口跟她聊兩句.
見她早上八點定著蒸飯回家.
她說兩句.
老公走了.
現在一盒飯三餐都吃那盒.
我猜少了一個人.
不是少了雙筷那麼簡單.
整個心靈都不同了.
就算這句說話用在職場.
都很難湊巧.
以往我們做下屬的時候.
最快樂就是吃午飯.
和一起說上司是非.
人生最滿足就是這些事情.

$^{401}$但當你做了上司的時候.
就沒有人和你吃飯.
你做了上司不是純粹加薪那麼簡單.
你將要承受的是你難以估計的事情.
你將要面對的就是沒有朋友.
你以往說人說得那麼興奮.
你不要升職了.
我保守你不要升職.
你會有報應的.
我想說是截然不同的兩回事.
不是我做下屬的經驗.
我跑前線懂得去找客.
就等於我可以應付很多複雜的人際關係.
特別我處身於中層的時候.
上面那個不長進.
下面那個不聽話.
這些是我以前做跑數的時候.
沒辦法用到的知識.
完全兩回事.
就算說自己身體也是.
年輕的時候你看醫生.
你期望醫生說什麼.
沒事了好起來了.
但到了某個年紀.
你去見醫生.
你最期望醫生說什麼.
沒有惡化.
沒有惡化得那麼快.
是兩種思考.
兩個人生的片段.
頂智文我想說.
我們的環境的轉變.
我們有沒有跟隨著這些客觀的事情.
去轉換我們的心靈呢.
我們的信仰.
我們的耶穌基督.
本來就是轉換心靈的高手.
聖經裡面常提醒我們.
舊的要變成新的.
你是新造的人.

$^{441}$耶穌經常將我們的人生一分為二.
我發現很多頂智妹或者朋友.
基本上卡在一個位置.
不進不退.
就是他不能夠接受眼前的環境改變之餘.
他也從來沒有想過調教自己的心靈.
這兩樣東西令到問題就出來了.
總覺得上帝欠了我們.
為什麼是這樣.
為什麼不給我以前的東西.
就好像出埃及那群以色列民.
去到曠野抱怨.
他幻想什麼.
我寧願像以前圍爐的時候.
吃回那些東西.
我心想你圍爐哪有那麼多東西吃.
全部都是幻想出來的.
你要抱怨上帝.
你可以塑造以前多美好都可以.
你講到天下無敵都可以.
你不喜歡現在的工作.
你可以幻想以前老闆對你好好的.
他一天都罵我三次.
現在罵四次.
你可以的.
幻想而已.
為什麼做不到.
神要耶利米向人民宣告.
請你睜開雙眼.
亡國的事實.
但不等於我丟棄你.
我希望每一個屬神的人.
我們都睜開雙眼.
望見眼前的景況.
這些是真的.
上帝要我們用新的心靈.
去面對這些事.
頂智梅是真的.
你上一次配眼鏡是什麼時候.
經過眼鏡店看到它八折.

$^{481}$很划算 先配兩副.
應該不是.
是當你發現視力有點模糊.
然後你要配合適的眼鏡.
讓你看得清.
視力模糊.
就算你閉上眼再實.
再努力.
你都是看不清.
你需要換一個客觀的配置.
讓你看得清.
就是這麼簡單.
不是好壞之分.
不是努力不努力的問題.
而是我們客觀的事實.
眼睛改變了.
變差了.
我們就要主動一點.
去配一副配合你現況的眼鏡.
去看周圍的事.
真的.
我們的心靈.
有沒有隨著我們的人生變化.
社會的大氣候.
去求問主.
我應該怎樣走下去.
我在過去的裡面.
學懂了什麼事情.
有什麼暫時我做不到.
我需要放出來.
你今天身在何處.
你有什麼東西仍然用得著.
你是需要放手.
是今天講到給我和你的提醒.
人生很多事情是轉變的.
不是簡單用上一個階段的經驗.
就可以帶入下半場.
我們趁今年未過.
還有一整天.
你嘗試定一定神下來.

$^{521}$去想一想.
今年你除了看了一場煙花之外.
你還看見過什麼事情.
我們有沒有睜開雙眼.
接受這些現實.
然後再問神.
在這樣的氣候.
這樣的環境.
這樣的情況裡.
我可以做什麼.
有什麼我需要暫時放下.
然後配合著你.
去走新一里路.
最後講一段經文.
對我自己的提醒和經歷.
回應小孩這個主題.
我有一對外甥.
我妹妹和妹夫生了一對外甥.
大概一個五年級.
一個中一.
上年十月的時候.
移民到英國.
其實過去這麼多年.
主要都是由我和太太.
去照料這對兒甥.
因為我妹妹和妹夫要輪班.
她住在我家附近.
所以過去十年.
都是我和太太去照顧她上學放學.
如果我時間許可.
不行就由外甥照顧.
所有功課都是由我負責.
十年就是這樣過.
當我知道她移民的消息.
對我來說是極沉重的事情.
因為很短時間我才接收到.
今年五月.
我和太太飛到英國探望這對兒甥.
住了兩個星期.
去英國沒有安排任何旅遊行程.

$^{561}$也沒有打算探望弟兄姊妹.
我只是在十四天裡.
每一天帶他們上學.
然後我和太太就坐車到攤位買菜.
回來接他們放學.
然後煮飯給他們吃.
這是我過去十年最快樂的事情.
大概十四天過了.
帶他們一家去曼城玩了三天.
臨到尾分手.
他們坐火車回到他們住的地方.
我和太太就坐火車去機場.
坐了半個小時.
我妹急叫我.
他們回到家沒多久.
她就說.
哥 你的外甥女哭到快崩潰.
她完全失控.
因為我上不了網.
網絡比較差.
不能夠視像.
她錄了一段音給我聽.
我太太到今天都不敢聽.
我外甥女五年級.
哭著這樣問.
她說舅父.
為什麼你們不移民過來.
她說其實在這裡做包裝都可以好好生活.
我知道我外甥女知道我有強迫症.
做包裝對我來說是很適合的.
我都說過.
我去到超級市場見到罐頭歪了.
我都會轉一轉.
我是這些人.
她哭著這樣問.
哭了很久.
我回答她留言.
她說舅母怕冷.
因為英國很冷.
我外甥女說.

$^{601}$我存錢買件外套給她.
為什麼你們不過來.
哭了三天.
外甥仔大一點 中學.
她不懂得哭.
但她媽媽知道她說不出聲.
因為外甥仔更加親我.
直至幾天之後.
她才大爆發.
我回到香港再跟她視像.
談了一段時間.
外甥女問我的問題.
為什麼我不移民過去.
這個問題我留在心裡很久.
我太太問我.
其實過去十幾年.
我和你在香港做過什麼.
究竟我們所愛又愛我們的人.
他問我一個問題.
我怎樣回答這個問題呢.
究竟我知不知道自己身在何處.
究竟我的心靈是否配合眼前的環境呢.
直至我看耶利米蘇預備這篇講章的時候.
神回答我這個問題.
環境是改變了.
他們兩個有離開香港的原因.
但你和太太也有留在這裡的理由.
我只可以跟外甥女說.
舅父不過來了.
舅父仍然在香港有一些很重要的事情要做.
你多點回來探望我吧.
頂至在這段經歷裡.
我自己也要學習放手.
對我來說.
這兩個小朋友是我人生過去十年快樂的支柱.
上帝讓他們離開一定有原因.
但對我來說.
我到今天這一刻還未適應得到.
我見我所有家人都適應得到.
但我不知道為什麼我適應不到.

$^{641}$我沒有一個支撐我人生最快樂的支柱.
上帝將它挪走了.
以前我覺得送上學接放學.
預備兩餐是理所當然.
現在都沒有了.
我要學會的是.
我要過一些不常有對方的生活.
這句話我不是寫給我的外甥女.
我是寫給我自己.
我要很努力去學一些不常有對方同在的生活.
不過我又得到什麼呢?.
我發現我現在再不需要為他們默書操心.
我不需要在假期裡幫他們想節目.
他們生日我不需要去買禮物.
我不需要為家人預備造節的飯.
我甚至不需要去老人院探望離世的爸爸.
我可以很專心侍奉耶和華.
這一年的侍奉是我這麼多年來從來沒有試過這麼專心.
上帝給了很多又大又難的事.
想到我的侍奉裡.
甚至有朋友開玩笑地問我.
為什麼你還沒有抑鬱?.
我說這段時間比較忙.
希望遲些有時間.
抑鬱也很奢侈的.
在今年裡.
上帝釋放了我很大量的自由度.
甚至可以專心去走另外一條路.
上帝很奇妙地讓我和John Poon Sir聊天.
可能他們都知道我沒事做會耍廢.
他安排了一些弟兄姊妹.
在過去一年裡給我機會和他們一起同行.
我和他們交換了很多流眼淚的故事.
我們亦交換了很多經歷神難處的事.
老實說我不能夠幫你們解決你們所面對的問題.
不過我自己這樣去評估.
我們能夠在一起的力量比單獨更大.
我在你們身上看到聖靈無聲的工作.
你們站得直立在身.
繼續走人生的路.

$^{681}$對我來說是最大的鼓勵.
與其說是我教導了什麼.
不如我們彼此同行.
我們發現原來仍然有路可行.
不需要卡住在那些你放不下的事跡.
眼前還有很多東西值得我和你去做.
弟兄姊妹是真的.
我們能夠放下的事一定是重要的事.
如果那些事常常令你猶豫不決不能夠放下.
那份量是很重要的.
我和你掙扎了很久不能夠釋懷.
甚至這些事情是構成你成為一個什麼人的事.
你不是希望國家興起.
但他也要放下.
神要懲罰.
他要面對.
如果那些事這麼輕鬆可以放下的話.
那些事早就不見了.
忘記了.
我們掙扎點就是不能夠放下.
不過我們要分清楚.
放下不是放棄.
或者是放在一邊.
神可能在某一天又要我拿起來.
又要重新開始.
當時候到了.
我們就可以再拿回這些曾經構成我們生命重要的事情的那部分.
在這個年頭裡面.
我發現原來我還有一些用法是我不知道的.
我是一個很早起床的人.
我四點多五點就在街上.
不是在家裡.
在街上.
你當我靈修也好.
神運也好.
拉筋也好.
總之我一定在街上.
通常在這個時間我就開始去工作.
我是一個絕對的神型人.
但原來我將自己交出來.

$^{721}$我不想我可以怎樣計劃.
讓上帝隨意使用.
原來神跟我說.
其實你只是一個夜奔芬.
晚上你也可以.
有些你自己的用法是你自己不知道的.
你將自己交出來給上帝.
上帝的用法可以很新奇.
是可以用得很奇妙.
總之就是.
藉著放下.
藉著這兩個我痛惜的外生.
神教曉我很多道理.
我希望我的決定都讓這兩個外生知道.
因為我走的路.
是有一位更值得我放下他們的神.
在我前面.
更值得我去找我自己舒適的生活.
在我前面.
因為這條路是上帝讓我和祂同行的路.
我希望祂望見的信仰.
是一個這樣的信仰.
我們所信的是亞伯拉罕的神.
以撒的神.
瓦國的神.
是一代接一代同樣凝信的神.
這是我很希望傳給我們的下一代.
你今天生在何處.
你還有什麼值得保留.
有沒有什麼值得放下呢.
用這樣去問.
如果你今天仍然有小朋友纏著你.
讓你失去很多meantime.
我聽說在人生最值得回憶的.
全部都是這些時間.
當他升到中學的時候.
那段日子你是不想記得的.
他還要扭要抱要問你.
要撒你的日子.
請你好好珍惜.

$^{761}$如果你的兒女又長大了.
不在身邊.
你又多了很多用不完的meantime.
多到要kill time.
不知道該怎麼辦.
你又好好珍惜這些自由.
你身邊還有很多朋友.
很多老友等著和你聊天.
這些時間你要用好.
如果你有工作壓力.
你仍然為上班而憂慮.
能夠賺錢養家.
是一件很微妙的事.
能夠有收入奉獻給教會.
讓教會繼續在不同的地方.
如登台一樣.
是值得榮耀的事.
不是每一個都有機會賺錢.
我都試過失業很一段時間.
那種徬徨我仍然記到今天.
都記得.
如果因不同的原因.
不需要上班.
很清閒.
你儘管去看看鳥語花香.
反正你上班的時候.
一直想下班.
現在長放.
你是不是很開心.
你都要享受你曾經很嚮往的日子.
不能夠反過來.
享受這些日子.
年輕力壯的時候.
禁意而行.
得罪人多.
得罪上帝更多.
恃著自己有無窮的體力.
以為自己不會病.
你去感恩.
你能夠匆匆忙忙的日子.

$^{801}$是上帝賜的.
但當你患病的時候.
你開始認真思索生命.
你開始去計算.
原來日子是可以很短.
你更認真去想.
什麼叫人生.
一道風景.
一種心靈.
今年要過去了.
你除了看了一場煙花.
吃過幾餐自助餐.
嘻嘻哈哈玩過幾輪派對之外.
你還有什麼剩下.
你學過什麼.
你有什麼本領.
可以帶入新一里路.
又或者我問.
你有沒有一些卡住你的東西.
今天仍然放不下.
如果我也學努力放下.
不如我們也一起.
反正前面還有新的風景.
等著我和你.
還有新的朋友在面前.
讓我們一起去走2024年新一里路.
我們一起祈禱.
神說前面的風景是陌生.
甚至令我們不安.
不過願你對耶利米所說的那句.
今天對我們每一個都說.
你不要懼怕.
因為我與你同在.
我要拯救你.
這是耶和華說的.
阿們.
\newpage

\allsectionsfont{\centering}

\setlength\parindent{0pt}
\setlength{\columnsep}{1.25em}
\setlength{\parfillskip}{0pt}
\setlength{\tabcolsep}{1em}
\raggedbottom

\pagenumbering{gobble}


\newfontfamily\leftfont[Path=../fonts/fell_french_canon/, Ligatures=TeX, ItalicFont=IMFeFCit29C.otf, BoldFont=AveriaLibre-Bold.ttf]{IMFeFCrm29C.otf}
\newfontfamily\leftcitationfont[Path=../fonts/frankruehl/]{FrankRuehlCLM-Medium.ttf}
\newfontfamily\centerfont[Path=../fonts/garamond/, Ligatures=TeX, ItalicFont=EBGaramond-SemiBoldItalic.ttf]{EBGaramond-SemiBold.ttf}
\newfontfamily\rightfont[Path=../fonts/averia/, Ligatures=TeX, ItalicFont=AveriaLibre-RegularItalic.ttf, BoldFont=AveriaLibre-Bold.ttf, BoldItalicFont=AveriaLibre-BoldItalic.ttf]{AveriaLibre-Light.ttf}
\newfontfamily\rightcitationfont[Path=../fonts/rashi/]{Mekorot-Rashi.ttf}
\definecolor{hcolor}{HTML}{D3230C}
\definecolor{rcolor}{HTML}{D36F0C}
\newcommand{\chfont}[1]{\centerfont{\huge\textcolor{hcolor}{#1}}}
\newcommand{\leftcitation}[1]{\leftcitationfont{\Large\textcolor{hcolor}{#1}}}
\newcommand{\rightcitation}[1]{\rightcitationfont{\normalsize\textcolor{rcolor}{#1}}}
\newfontfamily\flowerfont[Path=../fonts/fell_flowers/]{IMFeFlow2.otf}

\begin{sloppypar}

\chapter*{\chfont{編按結語}}

\columnratio{0.5,0.5}\begin{paracol}{2}

\fontsize{11}{13}\leftfont \Large \leftcitation{א} \leftfont 余少好文.宏志博覽群書而不忘.善存藏經籍文獻備後時之用。\leftcitation{ב} \leftfont 歸主年時.受友所薦.聞道網海.\switchcolumn\fontsize{11}{13}\rightfont \Large \leftcitation{ח} \rightfont 有見粵道之危.國之封講道千言亦將就至.急之何則為?\leftcitation{ט} \rightfont 嘗聞猶太者之傳承.在其力守口述之

\end{paracol}


\columnratio{0.32,0.32,0.32}\begin{paracol}{3}

\fontsize{11}{13}\leftfont \Large 尤以吳約翰遜者 \switchcolumn[2]\fontsize{11}{13}\rightfont \Large 統.以煉千載不

\end{paracol}

\columnratio{0.32,0.32,0.32}
\begin{paracol}{3}\fontsize{11}{13}\leftfont \Large 為重.其載上之粵語講道緩緩入耳.收之藏其音頻.善妥整存.反復而嚼.受益無窮。\leftcitation{ג} \leftfont 我城我國既限.歷一四一九之不測.肺疫延年.信徒靈長屢受圍創.神州燈臺數盡指日可待.粵道之求與日俱增。\leftcitation{ד} \leftfont 觀乎社、經、法、媒、言、信、網之地.愈趨受鋤.自翔不果.授受壓力.粵道聖言亦愈漸艱難。\leftcitation{ה} \leftfont 況崇基例乎.學苑講道屢逆權勢者.其言末強受壓.舊章盡刪以存其身。講道釋數失傳.徒嘆奈何。

\switchcolumn

\fontsize{11}{13}\centerfont 
\begin{tikzpicture}
    \node (0,0) [xshift=-0.10cm, yshift=-1.0cm, opacity=0.10]{\includegraphics[width=0.30\textwidth]{../ot_frontcover.png}} ;
    \node (0,0) [xshift=+0.20cm, yshift=+2.0cm, opacity=0.10]{\includegraphics[width=0.20\textwidth]{../christ_on_cross.png}} ;
\end{tikzpicture}
\Large 

\leftcitation{ס} \centerfont 詩百又廿七載:
\leftcitation{ע} \centerfont 非耶和華建屋宇.則匠人之經營徒.
\leftcitation{פ} \centerfont 非耶和華衛城邑.則守者之儆醒徒.
\leftcitation{צ} \centerfont 余獻是卷予華人社區.願為福音流通之器.願獻斯微材為祭榮耀上帝.
\leftcitation{ק} \centerfont 阿門

\switchcolumn

\fontsize{11}{13}\rightfont \Large 滅.時越次聖殿期及當今。\leftcitation{י} \rightfont 猶太者力廣納之.筆錄以卷軸.便以傳、閱、頌、攜、守、鎖、抄、譯、釋、編,得書塔木德、密示拿等經傳.家喻戶曉.傳流若芳。\leftcitation{כ} \rightfont 猶太者文以載道.傳其口述.今我輩粵道之傳應當作如是.遂力行粵音識辨之法.載言載道.以盡忠傳粵道以待興。\leftcitation{ל} \rightfont 蒙下賜恩惠.無畏海量字音文書.既馭上帝之道.今廣及粵語講道.重駛編程之技.匯導粵音遂字稿.重塑講道現場.以傚猶太卷軸之舉便以傳流。\leftcitation{מ} \rightfont 是卷乃粵音口述傳之屬.莫通華文白話之語.

\end{paracol}

\columnratio{0.5,0.5}
\begin{paracol}{2}\fontsize{11}{13}\leftfont \Large \leftcitation{ו} \leftfont 斯殺一違儆百逆.既禁壓之.我輩聞風無奈.在所難免。\leftcitation{ז} \leftfont 另有異人例乎.以版權之名.脅網絡頻道之舉.同授礙予粵道之存流。

\switchcolumn

\fontsize{11}{13}\rightfont \Large 惟待後繼來者之傚.以譯釋傳之於神州華文地。\leftcitation{נ} \rightfont 今能排程驅馭圖靈以編彙文檔,其碼長共數千千亦無逢大礙.全蒙上帝保守。

\end{paracol}



\columnratio{1}\begin{paracol}{1}

\fontsize{11}{13}\rightfont \Large
~~~~~~~~~~~~~~~~~~~~~~~~~~~~~~~~~~~~~~~~~~~~~~~~~~~~~~~~~~~~~~~~~~~~~~~~~~~~~~~\leftcitation{ר} \rightfont 二零二三年二月一日

~~~~~~~~~~~~~~~~~~~~~~~~~~~~~~~~~~~~~~~~~~~~~~~~~~~~~~~~~~~~~~~~~~~~~~~~~~~~~~~\leftcitation{ש} \rightfont 米迦勒

~~~~~~~~~~~~~~~~~~~~~~~~~~~~~~~~~~~~~~~~~~~~~~~~~~~~~~~~~~~~~~~~~~~~~~~~~~~~~~~\leftcitation{ת} \rightfont 書於香港

\end{paracol}

\end{sloppypar}
\end{document}
