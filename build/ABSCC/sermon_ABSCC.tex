\documentclass{book}
%\usepackage[letterpaper, portrait, margin=1cm]{geometry}
%\usepackage[letterpaper, bindingoffset=0.2in, left=1in,right=1in,top=.5in,bottom=.5in,footskip=.25in,marginparwidth=5em]{geometry}
\usepackage[letterpaper, left=1in,right=1in,top=.5in,bottom=.5in,footskip=.25in,marginparwidth=1cm]{geometry}
% ---------------------
% mini-table-of-content
% ---------------------
\usepackage{minitoc}
\setcounter{minitocdepth}{1}
\setlength{\mtcindent}{24pt}
\setcounter{secnumdepth}{-2}
%\renewcommand{\mtcfont}{\small\rm}
%\renewcommand{\mtcSfont}{\small\bf}
%\usepackage{setspace}
%\usepackage{tocloft}
%\setlength\cftparskip{-1.2pt}
%\setlength\cftbeforesecskip{1.3pt}
%\setlength\cftaftertoctitleskip{2pt}
%\renewcommand{\cftsecafterpnum}{\hspace*{02.0em}}
%\renewcommand{\cftsubsecafterpnum}{\hspace*{02.0em}}

% ---------------------------
% Chinese Characters Packages
% ---------------------------
\usepackage{fontspec} 
\usepackage{xeCJK}
\setmainfont{Times}
\setCJKmainfont{BiauKai}
\newfontfamily\sblgoodhebrew{SBL BibLit}[Script=Hebrew,Contextuals=Alternate]
\newfontfamily\sblgoodgreek{SBL BibLit}[Script=Greek,Contextuals=Alternate]

\usepackage{ifpdf,cite,algorithmic,url,tikz}
\usepackage[cmex10]{amsmath}

% ---------------------------
% Hebrew Characters Packages
% ---------------------------
\usepackage{polyglossia}
\setmainfont{Times New Roman}

% -------
% General
% -------
\usepackage{multicol}
\usepackage{multirow}
\usepackage{color,colortbl}
\usepackage{xparse}
\usepackage{pbox}
\usepackage{stackengine}
\usepackage{titlesec}% http://ctan.org/pkg/titlesec
\usepackage{tabularx}
\usepackage{xltabular}
\usepackage{titlesec}
\usepackage{makecell}
\newcommand{\sectionbreak}{\clearpage}

\author{
  Editor, Michael Chan\\
  \texttt{michaelchan\_wahyan@yahoo.com.hk}
}
\usepackage{tocloft}

\usepackage{hyperref}
\hypersetup{
    colorlinks=true, % set true if you want colored links
    linktoc   =all , % set to all if you want both sections and subsections linked
    linkcolor =blue, % choose some color if you want links to stand out
}

% ----------
% Afterword
% ----------
\usepackage{marginnote}
\usepackage{sectsty}
\usepackage{ragged2e}
\usepackage{lineno}
\usepackage{xcolor}
\usepackage{paracol}

\begin{document}

\clearpage
%% temporary titles
% command to provide stretchy vertical space in proportion
\newcommand\nbvspace[1][3]{\vspace*{\stretch{#1}}}
% allow some slack to avoid under/overfull boxes
\newcommand\nbstretchyspace{\spaceskip0.5em plus 0.25em minus 0.25em}
% To improve spacing on titlepages
\newcommand{\nbtitlestretch}{\spaceskip0.6em}
\pagestyle{empty}
\begin{center}
\bfseries
\nbvspace[1]
\Huge
{%\nbtitlestretch
\Large
\textbf{加拿大建道中心 粵語講道逐字稿 \\
       Youtube Channel: Alliance Bible Seminary Centre of Canada
       }}

\nbvspace[1]

{\large
Editor: Michael\\
\texttt{michaelchan\_wahyan@yahoo.com.hk}
}

\nbvspace[1]

{\large
Revision: \texttt{v1.2}\\
Last Update: \today
}


\vfill
\begin{tikzpicture}
    %remove comment for OT cover%\node (0,0) [opacity=0.03]{\includegraphics[width=15cm]{../bible_out/ot_frontcover.png}} ;
    %remove comment for NT cover%\node (0,0) [opacity=0.03]{\includegraphics[width=15cm]{../bible_out/christ_on_cross.png}} ;
    %remove comment for Bible cover%\node (0,0) [xshift=0.8cm, yshift=+2cm, opacity=0.03]{\includegraphics[width=10cm]{./christ_on_cross.png}} ;
    %remove comment for Bible cover%\node (0,0) [              yshift=-2cm, opacity=0.03]{\includegraphics[width=14cm]{./ot_frontcover.png}} ;
\end{tikzpicture}
\vfill

\end{center}

\newpage

\setcounter{tocdepth}{0}
\dominitoc
\begin{multicols}{3}
\addtocontents{toc}{\protect\hypertarget{toc}{}}
\tableofcontents
\end{multicols}

\large
%\twocolumn

% the color definition syntax is as follow:
% \definecolor{name}{system}{definition}
% example: a mono-channel color can be defined as
%          \definecolor{Gray}{gray}{0.9}
% example: an rgb-3-channel color can be defined as
%          \definecolor{LightCyan}{rgb}{0.88,1,1}
%          \definecolor{pink}{rgb}{0.68,0,0.68}

\definecolor{CUV1LightRed}{rgb}{1,0.75,0.75}     % for CUV1
\definecolor{LZZVLightGray}{rgb}{0.9,0.9,0.9}    % for LZZ
\definecolor{KJVVLightGreen}{rgb}{0.75,1,0.85}   % for KJV
\definecolor{CUV2LightYellow}{rgb}{1,1,0.75}     % for CUV2
\definecolor{CNVVLightBrown}{rgb}{1,0.85,0.7}    % for CNV
\definecolor{NRSVLightBlue}{rgb}{0.75,1,1}       % for NRSV
\definecolor{WENLLightPurple}{rgb}{0.95,0.85,0.9}% for WENL
\definecolor{TCV19PaleGreen}{rgb}{0.85,1,0.95}   % for TCV19
\definecolor{MSGVLightWhite}{rgb}{0.98,0.98,0.98}% for MSGV
\definecolor{NETSLightRed}{rgb}{1,0.75,0.75}     % for NETS
\definecolor{JPS1917LightYellow}{rgb}{1,1,0.75}  % for JPS1917
\definecolor{SBLGNTPaleRed}{rgb}{1,0.85,0.80}    % for SBLGNT

\section{目錄}
\label{sec:index}
{ \scriptsize


\begin{xltabular}{\textwidth}{|p{0.15\textwidth} p{0.6\textwidth}|p{0.07\textwidth} p{0.1\textwidth}|}
\hline
  & \hyperref[sec:UCyv23j5rVY]{講座:退而不休新境界} & 2019-10-08 & \href{https://youtube.com/watch?v=UCyv23j5rVY}{\texttt{UCyv23j5rVY}} \\
  & \hyperref[sec:JwFqk5bcKhM]{做好呢份工? - 20200407} & 2020-05-08 & \href{https://youtube.com/watch?v=JwFqk5bcKhM}{\texttt{JwFqk5bcKhM}} \\
以斯帖記  & \hyperref[sec:9pqygotXAuo]{【從聖經書卷看生命實踐系列】 主題(一) 陰霾時代的曙光:以斯帖記給我們的安慰 (粵語講授)} & 2020-05-13 & \href{https://youtube.com/watch?v=9pqygotXAuo}{\texttt{9pqygotXAuo}} \\
以斯帖記  & \hyperref[sec:qcKLit3iF4o]{主題:陰霾時代的曙光:以斯帖記給我們的安慰 (普通話講授)} & 2020-05-19 & \href{https://youtube.com/watch?v=qcKLit3iF4o}{\texttt{qcKLit3iF4o}} \\
  & \hyperref[sec:yb30yQHiYdM]{疫症大流行下的神學反思 --- 第二講:《疫症大流行下的靈性修持》} & 2020-06-19 & \href{https://youtube.com/watch?v=yb30yQHiYdM}{\texttt{yb30yQHiYdM}} \\
  & \hyperref[sec:fFkCm0QGBPw]{疫症大流行下的神學反思 --- 第一講:《疫症大流行下的禱告神學》} & 2020-06-19 & \href{https://youtube.com/watch?v=fFkCm0QGBPw}{\texttt{fFkCm0QGBPw}} \\
尼希米記  & \hyperref[sec:P0Y2lvzICsM]{【從聖經書卷看生命實踐系列】 主題(二) 從尼希米記看面對將來的挑戰} & 2020-07-31 & \href{https://youtube.com/watch?v=P0Y2lvzICsM}{\texttt{P0Y2lvzICsM}} \\
何西阿書  & \hyperref[sec:XLKUZGl9ItY]{【從聖經書卷看生命實踐系列】 主題(三) 從何西阿書看婚姻關係 (AB2008)} & 2020-08-12 & \href{https://youtube.com/watch?v=XLKUZGl9ItY}{\texttt{XLKUZGl9ItY}} \\
何西阿書  & \hyperref[sec:n5DpA1Db_0M]{【從聖經書卷看生命實踐系列】 主題(三) 從何西阿書看婚姻關係 (普通話)} & 2020-09-08 & \href{https://youtube.com/watch?v=n5DpA1Db-0M}{\texttt{n5DpA1Db-0M}} \\
  & \hyperref[sec:DOwKowWHLkM]{你在帥領\_講座題目:「蒙揀選的身份:摩西的召命與掙扎」} & 2020-10-23 & \href{https://youtube.com/watch?v=DOwKowWHLkM}{\texttt{DOwKowWHLkM}} \\
  & \hyperref[sec:0]{寫給徘徊應否進入神學學習的你} & 2021-06-02 &  \\
  & \hyperref[sec:1]{「無名無姓」的啟迪} & 2021-06-23 &  \\
  & \hyperref[sec:2]{要活得有智慧} & 2021-07-15 &  \\
  & \hyperref[sec:3]{「亂世中的小人物:舊約篇」分享回應} & 2021-08-03 &  \\
  & \hyperref[sec:4]{「亂世中的小人物:新約篇」分享回應} & 2021-08-05 &  \\
  & \hyperref[sec:5]{「亂世中的小人物」 <對談篇> 的回應分享} & 2021-08-08 &  \\
腓立比書  & \hyperref[sec:GA78znQ7bg4]{ABSCC 網上免費講座:「盡頭見耶穌:腓立比書的盼望秘笈」} & 2022-01-13 & \href{https://youtube.com/watch?v=GA78znQ7bg4}{\texttt{GA78znQ7bg4}} \\
  & \hyperref[sec:L8_DVqUvOSM]{ABSCC 講座 - 細味‧原文 ─ 簡易希伯來文 [示範 1] (AB2202 POL)} & 2022-04-20 & \href{https://youtube.com/watch?v=L8_DVqUvOSM}{\texttt{L8\_DVqUvOSM}} \\
  & \hyperref[sec:efm9yyrZOo0]{ABSCC 講座 - 細味‧原文 ─ 簡易希伯來文 [示範 2] (AB2203 POL)} & 2022-06-16 & \href{https://youtube.com/watch?v=efm9yyrZOo0}{\texttt{efm9yyrZOo0}} \\
  & \hyperref[sec:3o4omcoTUB4]{ABSCC 網上免費粵語講座:非常移民牧養\_20230329} & 2023-03-31 & \href{https://youtube.com/watch?v=3o4omcoTUB4}{\texttt{3o4omcoTUB4}} \\
  & \hyperref[sec:B5n__dtTRhE]{ABSCC 網上免費講座:YOUTH ALPHA - 青年福音新思維} & 2023-05-03 & \href{https://youtube.com/watch?v=B5n_-dtTRhE}{\texttt{B5n\_-dtTRhE}} \\
  & \hyperref[sec:6]{不要浪費這個疫情} & 2023-09-22 &  \\
  & \hyperref[sec:7]{教牧關顧的可持續節奏} & 2023-11-09 &  \\
  & \hyperref[sec:8]{主顯節的由來和反思} & 2024-01-24 &  \\
  & \hyperref[sec:9]{依納爵的心窄成因} & 2024-02-22 &  \\
  & \hyperref[sec:10]{奧古斯丁《講章 242(關於身體復活,駁異端的講道)》的節錄及反思} & 2024-03-29 &  \\
  & \hyperref[sec:11]{第一篇遺憾} & 2024-03-29 &  \\
  & \hyperref[sec:12]{依納爵的心窄的轉化方法} & 2024-04-18 &  \\
  & \hyperref[sec:13]{在後現代福音的大能} & 2024-05-02 &  \\
  & \hyperref[sec:14]{問世間,情是何物?} & 2024-07-31 &  \\
  & \hyperref[sec:15]{馬丁‧路德的心窄困擾} & 2024-09-05 &  \\
  & \hyperref[sec:16]{訂婚的變奏—掃羅遇見井旁打水的少女} & 2024-11-14 &  \\
\end{xltabular}
}
\newpage



\section{}
\label{sec:UCyv23j5rVY}
\textbf{講座:退而不休新境界}
\newline
\newline
連結: \href{https://youtube.com/watch?v=UCyv23j5rVY}{\texttt{https://youtube.com/watch?v=UCyv23j5rVY}} ~~~~ 語音日期: 2019-10-08
\newline
\newline
\hyperref[sec:code]{\small{< < < PREV SERMON < < <}}
~
\hyperref[sec:index]{\small{[返主目錄]}}
~
\hyperref[sec:JwFqk5bcKhM]{\small{> > > NEXT SERMON > > >}}
\newline
\newline
$^{1}$開始了 因為今天想和大家分享的東西其實都挺多的.
我想時間緊張一點.
很高興見到大家.
無論是認識我的 或者是未認識我的.
我都代表建立中心歡迎大家.
我今天就是一個很怪的現象.
平時我就是介紹人.
現在又介紹自己.
說的那個又是我 介紹的那個又是我.
一腳踢.
我是加拿大建造中心的主任.
陸昭明牧師.
在這個時候當然要歡迎大家.
這張slide就是一個歡迎的slide.
告訴大家我們的網站.
其實如果你是間中想一下我們的網站.
你會看到我們很多的資訊.
就是abscc.org.
如果你用facebook就看.
The Alliance Bible Seminary Centre of Canada.
裡面有很多的內容讓大家可以參考.
特別是一些新的公佈.
我們都會在那裡先做的.
現在這個世界已經是越來越網絡化.
我們都盡量用.
喂 哈囉.
盡量用很多東西.
對不起 有時候我會和我的朋友打個招呼.
哈哈.
那就 啊 糟了 說到哪裡了.
哈哈.
盡量用網絡的方式.
網絡又不用錢.
而且訊息傳得很快.
現在慢慢慢慢我們都很少印海報了.
都是用WhatsApp 用Email.
因為Email都開始多人不看Email.
我們都要與時並進.
我自己上任中心主任的位置.
還沒夠一年半.

$^{41}$過去一年半左右.
中心都有很多的改變.
如果你是新來的 第一次來的.
我也很簡單介紹一下我們的工作.
我們的工作其實就是.
咦.
Charles 啊.
Charles 說advance不到的.
為什麼 Charles.
應該說advance不到的.
Page down 你說不行.
我的cursor在裡面.
是不是.
不是.
啊 為什麼剛才你沒有去哪個按鈕.
哦 Page down.
不是這個按鈕.
不是不是.
哦 OK OK.
我是user來的.
所以很多東西都不懂.
我們的中心的目標很簡單.
因為我們不是神學院.
神學院是訓練全部人.
我們特別針對聽訊套的靈明的培育.
以及裝備他侍奉.
在教會裡面侍奉.
這個就是我們的特點.
我們的簡介這裡已經說了.
我們有兩層的課程.
供應給大家報讀的.
一個是證書.
就是certificate.
比較最簡單的.
12個小時一科.
民憑那些就是香港見到神學院的老師.
飛過來教的.
那些就是25個小時.
就是雙倍的.
大家手上這個單章.

$^{81}$就說了我們最新的季度.
就是9月至11月.
就是下個月開始.
我們有什麼科目推出.
最快的就是大家打開中間.
有一科叫做約書亞記.
是正見的人生.
這個也很推薦給大家考慮.
我知道很多大英姐妹都很喜歡讀聖經.
舊約聖經對很多大英姐妹來說.
都是比較陌生的.
我們特別推出這一科.
就是三個星期六.
就是9月7號14號和21號.
星期六的早上.
三個小時.
鼓勵大家考慮這件事.
為了多謝你們來.
如果你是報讀的.
今天是報讀這一科的.
當是早上.
已經不是很早了.
已經是晚上了.
但是你在填表格.
寫上穆勒沃斯規準.
就變成75元了.
鼓勵大家參與這個科目.
久不久我們會有神學科.
或者是一些視訪相關的課程.
你在我們的網頁裡面.
很多時候都會看到.
我們工作隨俗系這些科目.
還有講座.
就像現在這個樣子.
講座大概是一個月一次.
就好像待會我跟大家分享的講座.
也有一些是大型的講座.
大型的講座在宗教會裡面舉行.
我們不定期的.
預先講講有什麼大型講座.

$^{121}$就是under planning.
你一定會很轟動的.
明年3月的時候.
我們請了香港建道神學院的老師.
將會來多倫多這個地方教書.
順便一年之前就約定了.
3月14號一個星期六的早上.
會舉一個講座會很轟動的.
就是講到自殺.
就是有弟兄姊妹自殺的時候.
教睦或者是普通的弟兄姊妹.
親友自殺的時候.
怎樣去應對那個情緒上的轟動.
在一個危機突襲的時候.
陸續會有宣傳的東西出現.
大家留意著3月14號.
聽起來好像很久了.
其實只是眨眼而已.
就快要下雪了.
預備得差不多了.
除此之外我們也都是.
有時候我們移師到不同的教會裡面.
有一些培訓的工作.
我們覺得這個是需要的.
因為每個教會都很獨特的.
需要都很不同.
有些教會邀請我去.
我就會去的.
按著他們教會的需要.
去提供一些培訓.
2020年有一個新的發展.
這裡很簡單提到.
我們有新的證書系列.
一個就是僕人領袖.
一個就是情緒管理.
陸陸續續會有多一些的宣傳.
就會帶出來.
Hello.
建築中心將會多元化.
多元化的發展.

$^{161}$不同時間就會有不同的推廣.
我們簡單的介紹到這裡.
再一次代表建築中心歡迎大家.
今天來到我們的當中.
我們就去到正式今天的講題.
我預備這個講題的時候.
其實是很沉重的.
因為我知道這個題目是迎難而上的題目.
是要抵抗一些潮流的.
所以我很多祈禱放在裡面.
我就懷著一個感恩的心.
跟大家分享今天的題目.
因為要扭轉一個人的看法.
不是我可以做得到.
真的要聖靈在你的心裡面做工作.
你才可以做得到.
希望上帝都藉助聖靈感動大家.
今天我的講題只有兩部分.
中間有一個休息的.
有一個中場的休息.
第一部分我就講得資訊性一點.
大家是不是都有講義.
有沒有誰沒有.
因為我們印了有限的.
如果不夠我們就立刻印.
因為我很清楚知道.
我們年紀差不多的大英姐妹.
不是那麼想報名的.
所以報名人數是不足以反映實質出現的人數.
對不對 你懂得笑.
我在服侍這個事工都好幾年了.
不 已經燒盡十年了.
很明白 大英姐妹都明白.
第一部分我會講一點資訊性的東西.
然後中場休息.
接著下半場就有一點不同.
我就會講一下我具體的見證.
我自己的見證.
那些叫做真人騷.
不單止見證我還會講一個短講.

$^{201}$一個短短的講道.
期待神的靈感動大家.
在聖經裡面有一對人.
師父和徒弟.
一個師父是以利亞.
徒弟是以利沙.
大家聽過了.
可能不需要講你都知道我想講什麼.
如果你在教會生活得久的話.
以利亞未曾被上帝接走的時候.
因為神要用旋風接走他.
他就和徒弟以利沙講了一句說話.
是很感人的.
他就和以利沙講.
你想我為你做些什麼呢.
以利沙就回答得很好.
以利沙真是一個很熟悉靈的人.
以利沙就說.
願感動你的靈.
加倍地感動我.
即是說願感動以利亞的靈.
加倍地感動我自己 以利沙.
我今天希望這一次和大家分享這個題目.
能夠天賦聖靈.
他已經感動了我很久.
同樣是加倍地感動你.
因為感動你我做不到.
唯有將大家放在上帝的靈當中.
我們先做一個祈禱.
主佑我們多謝你.
因為你實在是愛護我們每一個.
你深知我們的情況.
我們的困難.
我們的路向.
我們人生的目標等等的事你都知道.
我們知道當我們人生去到某個階段的時候.
很可能都會有些困惑出現.
但我們知道當我們明白.
你在我們心的心意的時候.
其實我們可以有一個很大的轉變.

$^{241}$求主真的幫助我們.
無論今天講的聽的都同感一靈.
你繼續的直著你的話語.
直著一些的分享.
直著你的道.
真的能夠進入到大英姐妹的心裡面.
我們的人心要被改變.
不是人的工作可以做得到.
我們深深相信只要你自己是喜悅.
更多的人願意伏在你的手下.
成為你手中可以被使用的器皿.
因此我們恭敬將今天的講座交給你面前.
求主使用.
求主你用你的補血再一次遮蓋我們每一個人的罪.
以致我們可以坦然無拘地來到你的施恩補助前.
求你的恩典再一次臨到.
讓我們歡歡喜喜的離開這裡.
能夠重新得力.
成為你最得力的助手.
求主恩代奉耶穌名教.
阿們.
今天我手上派給大家的講義.
其實已經是很清楚的了.
不過真人講是很不同的.
起碼見到一個真人.
好了.
我們首先講講.
其實人生有很多階段.
你也知道.
這一部分我會講得快一點.
從出生開始到小孩子.
到成年人.
到初期成人.
大概到四十歲.
我不問你幾多歲.
你也不要問我幾多歲.
不問你大概知道.
有多少頭髮你大概知道.
我講我.
我打了個交叉在這裡.

$^{281}$因為這些歲數界線的分類.
其實已經過時了.
因為大家都知道.
現在醫療比較好.
還有健康.
每個人都想要健康的生活.
吃東西好一點等等.
那裡有位置.
人的壽命越來越長.
以前六十五歲退休.
很多場合都是這樣的.
但是現在政府的部門.
取消了必須退休的年齡.
會說技術上.
你想做多久都可以.
技術上是很好.
你想做多久都可以.
但你能不能做多久是另一回事.
做得太久會阻斷地球轉.
因為下面的年輕人等著你走.
所以不要做太久.
一會兒我講我的見證.
你就知道我何時離開.
原來一個.
我們現在去這樣的情況.
我們一個新的面貌.
應該叫做一個新的中年的觀念.
聯合國被這個定義.
其實是八十歲才真正的年長.
八十歲.
我們當中沒有人八十歲.
有嗎?對不起.
你屈指一算.
你現在還有多少才到八十歲.
我告訴你.
我還有一段時間.
我不知道你有多久.
但當你還有一段時間.
你必須要想一個問題.
就是你餘下的時間應該怎樣用.

$^{321}$我們看看下面.
最近我接觸一個新的觀念.
叫做第三人生.
有沒有人聽過這個觀念.
叫做第三人生.
第三代.
原來分年期.
有第一人生.
有第二人生.
第三人生.
成年期.
幼年期.
成年期.
成熟期.
是甚麼意思呢.
原來的意思是.
幼年期就是你讀書的時候.
你很dependent.
很依賴你的父母.
你依賴你的師長等等.
那個是你學習的時候.
成年期就是開始獨立.
開創事業.
你很獨立的.
很independent.
但到了差不多之後.
你就慢慢轉變成為一個成熟期的時候.
這裡提出一個新的觀念.
就是一個共生共榮.
就是說對人和對自己.
都有一個互相的interdependence.
就是互相依賴的時候.
就是說你互相connect.
彼此是連結.
你不可以回家就躲在家裡.
而是互相建立一個重新建立關係.
和別人建立關係.
那個建立關係再不是以前有利益衝突的關係.
或者是有高.
是我的頂頭上司和下屬的關係.

$^{361}$而是一個平等的關係.
互相建立的關係.
有一個女明星很久了.
現在已經70多歲叫Jane Fonda.
如果你是70多歲的時候.
大概有記得的.
Jane Fonda很推動第三人生論述.
她很有意思.
這裡列了她的對比.
有不同的地方.
她說生命的歷程以前就好像一個.
一個鐘,Bell shape的一個鐘.
就是一直向上向上.
去到差不多那個tip的時候.
就掉下來.
掉下來就是現在所謂一個新中年的時期.
會掉下來.
好像很慘的.
掉下來.
掉下來就是說會撞到.
就是會跌倒.
就是那個感覺很灰.
但是Jane Fonda她就說不是的.
就是有一個新的觀念.
不是這樣看的.
生命的歷程應該是要爬樓梯的.
是一直向上的.
一直向上的.
有時可能跌一點點.
但是有另一種動力將你推上高一點.
去到差不多之後又跌一點點.
但是又再推你上去.
一直推下去.
那你是總體來說都好像向上.
不單止這樣.
以前的觀念就是用一個病理的觀點來界定.
就是以你健康與否來界定你是否年長.
那是的,如果很不幸.
就是40歲或者50歲就已經有大病.
那沒辦法.

$^{401}$可能是衰老得很快.
但是現在我們會提出一個新的觀念.
就是去開發這個人的潛能.
潛能.
待會我第二部分我講到我的見證的時候.
你們會看到.
原來那個潛能被開發的時候.
是多麼開心的一件事.
在上帝面前.
原來有很多潛能是未被發掘出來的.
未被爆出來的.
有爆炸力的.
以前的觀念呢.
看年紀越來越大就很負面的.
是不是?很負面的.
通常我聽到這些大英雄會覺得.
唉,給多點機會年輕人.
是不是?可能你也這樣說.
給多點機會年輕人.
意思就是你推.
就是你不想再背一些東西.
給多點機會年輕人.
什麼叫給多點機會年輕人?.
如果上帝要你做的事情你做就推.
就是很.
講得足一點的.
就是捷福.
很負面的.
第三.
第三人生的觀念比較正面.
就是正面去看年紀漸長.
因為年紀越來越大的時候.
是一個寶貝來的.
很珍貴的.
就好像一塊.
一塊未被開發的鑽石.
你們有沒有看過鑽石?.
未被鑽磨的時候.
就是一塊東西.
沒什麼特別的.

$^{441}$但當你是一塊一塊被鑽磨之後.
非常之燦爛.
一塊一塊的.
那些瑕疵.
你就表現了一個很美的鑽石.
一個很名貴的鑽石.
我想這個道理大家都可能明白.
究竟問題就是.
我們願不願意被雕鑿成為.
一個很特別的.
一塊的鑽石.
你看看.
這個也是一個很特別的圖.
很特別的一個圖.
究竟是一隻兔子.
還是一隻鴨子.
從右手邊去看.
我先把眼鏡拿走.
從右手邊去看.
這個是兔子.
但從左手邊去看.
就是鴨子.
我想提出一個觀念很簡單.
觀點與角度.
你看今天你去年紀漸長.
你去年紀漸長.
你看今天你去年紀.
究竟現在是.
日漸衰敗.
不要衰殘.
日漸衰敗.
你可能是用病理的角度去看.
但如果你看到.
現在上帝是想雕鑿你.
磨練你.
成為一個更加厲害的器皿.
你就會看到你自己.
原來前面是很燦爛的一條路.
是觀點角度的問題.
在你的筆記上沒有這張東西.

$^{481}$為何後期我突然.
找到.
所以就加插.
同場加影短片.
我覺得很有意思.
真的是觀點與角度.
拍了沒有?拍完了我就走了.
哈哈哈.
在我自己的教育度裡.
我很喜歡用圖.
去表達.
比較輕鬆一點.
不要太悶.
走了.
需要新的觀念.
需要新的觀念是甚麼意思呢?.
很多時候我們聽到人說.
你登過六了沒有?.
入了五又登六.
有些人還說你升過先.
你聽過沒有?.
升過先?不是說先人的先.
而是先爺.
哦.
升先了.
但人是很矛盾的.
人是真的很矛盾的.
雖然說.
退休年齡沒有限制.
但大概.
cut off point都是65歲.
那叫升先.
你會很雀躍.
我去到那個歲數之後.
我坐車便宜一點.
你坐GTC.
如果是adult fare.
3.25元.
但如果是升先.
升先之後.

$^{521}$你再用presto card.
是省了1.15元.
每次.
每次省了1.15元.
你會很開心.
覺得這東西很棒.
社會主義好.
加拿大很社會主義.
香港也是很社會主義.
這方面來說.
但另一個地方.
你會覺得很不順.
你坐公共交通.
有人開始讓座給你.
有試過這個經驗嗎?.
有.
有時不是很開心.
你不要搞了.
你坐吧.
自己不肯坐.
人是很矛盾的.
得個好益處.
好 升先了.
但你升先之後.
現在多些人幫你讓座.
你會不高興.
無論是得入伍.
和登陸都好.
我覺得舊觀念要有新的思維.
舊觀念.
已經過了.
我很久都沒有追巴士的觀念.
我以前是上班.
我追巴士.
追夜巴士.
追到就怎樣.
我一定放棄.
坐下一班.
不用死.
你一定要承認.

$^{561}$某些情況已經不是那一年.
但新的觀念.
我正在思考.
入伍登陸有什麼特別的意思.
我覺得一個很積極的東西.
可以介紹給大家.
入伍我們大部份都是香港來的人.
香港來的人沒有從軍.
沒有服兵役.
如果你有服兵役.
入伍的意思是.
你要裝備作戰.
打仗是另一回事.
你要裝備.
去受訓練.
入伍是一個準備的狀態.
你裝備自己的狀態.
登陸是什麼意思.
是要搶灘.
有看過那套戲叫做《碧血長天》.
The Longest Day.
在諾曼第的時候.
登陸就是準備了之後.
衝啊.
衝上海灘.
是要打仗.
一定要向前.
不要向後走.
向後走已經是軍訓.
會罰你.
搶著走.
搶著去.
你走得快就好.
如果你走得慢就被敵人射死.
這樣的態度.
登陸是這樣的意思.
不是.
登陸.
這樣的態度就不是很積極.
不過.

$^{601}$話雖如此我們不要.
那麼離地.
要看現實.
現實告訴我們.
到了我們的年紀就要存兩樣東西.
一個是存錢.
你的銀行戶口.
沒有積蓄的時候.
其實都會有點害怕.
但仍然要存健康.
要存健康.
不然怎麼會.
健康補品那麼好賣.
現在很多人看中醫.
看中醫是調理好身體.
還有一段時間.
不是捱.
是享受.
要存健康.
各位電視節目我真的想和你們分享一件事.
分享一件事.
你可能會觀察到.
越是.
很關注很關注錢.
越是很關注很關注.
健康的人.
特別是信徒.
越是很快出事.
你有沒有留意這件事.
為什麼會這樣.
我個人.
我個人的看法.
不是聖經根據.
我個人看得多就知道.
你缺少了.
祈求上帝給你的祝福.
上帝給你的祝福.
這裡我有說.
從信仰來看.
其實你要留意.

$^{641}$我們的天賦是慈愛.
你聽過很多次.
但在現在的狀態.
其實祂會保守我們.
什麼叫保守.
就是保護你.
看顧你.
最近我的同工不舒服.
看醫生.
她的丈夫來幫她買藥.
我和她聊天.
我和她聊天的時候.
突然間靈光一閃.
令我看測一件事.
原來她說去Costco.
不是不是.
去Sophistruct Mart買藥.
買藥費其實都很貴.
我不知道多少錢.
很多年都沒有買藥.
她又建議.
不如去Costco.
我還買了廣告.
我聽得清楚.
Costco是便宜很多.
你有經驗.
我那天和她聊天的時候.
我突然間聽到這個訊息.
我就知道上帝又提醒我一樣東西.
原來我一直在保守你的健康.
但你不是很為意.
我真的很感恩.
上次買藥.
買預約的藥.
一定要醫生處方.
要付出支付費.
差不多兩三年以前.
你的身體健康壯健.
其實不是因為你有什麼厲害的東西.
你跑步跑得厲害所以能夠保住.

$^{681}$還有上帝的祝福在背後.
我覺得下面那句才是更重要.
就是原來這個符.
是永恆的符.
它會引領我們.
它會引領我們向高處行.
不是向低處走.
向低處走就滾下去.
向高處行的時候.
它會帶領我們.
是什麼意思呢?我想帶出一個觀念給大家考慮.
原來我們的天賦.
它有一個永世的計劃.
作為一個牧師.
我一定要說這件事.
各位兄姐會可能不明白.
但是透過.
我們理解上帝的心意的時候.
跟大家分享這件事.
上帝有一個永恆的計劃.
是要找人去完成.
他獨自可以完成.
他不可以麻煩我們麻煩友.
他獨自可以完成.
但是他偏偏想找人去完成.
找什麼人呢?.
就是願意被他引領的人.
願意被他引領的人.
如果他願意的話.
他就賜恩典給我們.
不是因為我們有什麼特別的地方.
而是他會給我們恩典.
一步一步走上去.
越來越走得高.
一會兒說我見證的部分.
你就明白我在說什麼.
我真的覺得這一點.
很希望能帶給大家這個觀念.
關於現實的情況.
我就有以下的領受.

$^{721}$就是.
什麼是代表我們可能有些老的心態呢?.
不想做新的嘗試.
或者不接受新的事物.
或者失去學習的動力.
等等.
對不起.
如果說中你.
不是因為我知道你的底蘊.
而是因為我觀察很多人.
包括我自己在內都是這樣.
我不是代表我已經做完全.
我只不過是看到.
這是一個困難.
對於新的東西.
我們通常都不會想再做.
再冒險.
50歲的時候或許會.
40歲30歲的時候或許會.
現在我們何必要這樣呢?.
這麼冒險呢?.
不要搞我.
通常都是不要搞我.
問題就是要翻身.
翻身的秘訣就在這裡.
就是要從神那裡找回.
我們的自尊.
我們失落的自尊.
有時候我們失落了的.
遭遇不幸的遭遇.
又或者我們經歷了.
不開心的事.
我們所謂時不可遇.
失了很多的機會.
那我們就算了.
不要搞那麼多事.
但其實.
可以翻身的.
就是靠他的恩典.
我告訴大家一個秘訣.

$^{761}$你信不信知道秘訣呢?.
秘訣就是.
求天父不單給我們恩典.
不單是恩典.
而是額外的恩典.
英文叫.
extra grace.
如果我們欠缺一些東西的時候.
求主給我們.
有額外的恩典.
他願意給我們更多的恩典.
那我們又再向前行多一步.
又再向上行多一步.
而且有動機.
這個動機從天上而來的動機.
我們就有動力.
沒有動機就不會有動力.
這個很簡單的道理.
你的動機就是.
很想完成神在我們心裡的計劃.
這個就是你的動力.
如果你沒有這個東西的話.
我們就完全沒有動機.
動力是什麼意思呢?.
就好像飛機裡面.
通常飛機有引擎.
如果沒有引擎怎麼飛?.
飛機通常有兩個引擎.
以前的飛機.
七四七有四個引擎.
現在兩個引擎等於四個引擎.
引擎沒有燃料.
怎麼飛?.
沒有飛的嘛,引擎是有動力的.
但是你一定要有燃料.
才可以更加令它發揮能量.
我們就飛得起.
那個情況就是這麼簡單.
不過很多時候我們就會有這個問題.
我們有這個問題.

$^{801}$對於一些弟兄姐妹.
有些女性經常緬懷過去很風光的日子.
想當年我怎樣威風.
想當年我侍奉多積極.
想當年我做到長老.
想當年我做過牧師.
很多人都有這個想法.
很風光.
有些人另外一個極端.
想到很多以前很不開心的遭遇.
有點頹喪.
兩類型的人我都有以下鼓勵.
不要再想以前.
不要再想以後.
想現在最重要.
這裡我也是在文章裡拿出來的重點.
昨天已經過期的支票.
扔掉吧,還想什麼.
明天那些就是未到期的支票.
你兌現不了.
只有今天才是現金.
可以立即使用.
如果你是對明天有很多憂慮.
很多傷痛.
心理上覺得很難受.
你害怕,你懼怕,不敢向前走.
這句話希望幫到你.
Why borrow sorrow from tomorrow?.
為什麼還這樣被它折磨.
只看今天.
今天我們能夠積極.
明天我們看回今天.
就覺得不錯,有進步.
如初旅途.
因為憂慮就是使用昨天的困難.
而且浪費了今天的時間.
窒息了明天的機會.
這些不是我說的.
是有些智慧的人寫出來.
給大家參考.

$^{841}$今天才是最重要.
如果你放棄今天.
其實是很可惜的.
在你筆記裡有的,不需要抄.
我覺得可以調整我們個人的心態.
有人說一個人是100歲人生.
你相信100歲嗎.
這麼多人搖頭.
是不是去得宜康多一點.
老實說,去得宜康和孟祥多.
你會看到如果你不健康.
就不要那麼長命.
你問我希望多少歲.
我說80多歲.
馬上去見主面.
你不健康.
廣東人有句話很有趣.
叫什麼牽勢.
你知道什麼牽勢嗎.
我都不知道.
時間一直在過去.
但我睡在這裡.
無謂浪費床位.
宜康和孟祥的位置很矜貴.
等一些有需要的人.
估計是100歲.
100歲有人分了五件東西.
每件20年.
有人說求學為主.
0至20歲,事業婚姻,人生精華.
61至80.
有人說應該學習放下.
一些東西.
假設是信主的人.
叫做期待永恆.
期待永恆是什麼意思.
等什麼.
等死.
如果是信耶穌的人.
期待永恆等死.

$^{881}$我覺得這樣的態度.
不太適合.
我覺得.
61至80.
這個房間的部分.
都是這個類別.
61至80.
我覺得不是學習放下.
我覺得是要.
挑選你所長.
然後繼續去做.
不一定是要上班.
要挑選你的所長.
你說恩賜也好.
服侍也好,侍奉也好.
挑選你所長.
你不是每樣都可以.
千萬不要以為你的牧師.
每樣都可以.
是不可能的.
所以不要要求自己每樣都可以.
人生已經去到一個階段.
很明白自己.
你明白自己的長處.
短處.
你就專做長處.
這個都是.
你稍後聽我的見證.
有些不同的看法.
未必一定只是做長處.
你夠膽.
向短處去行.
靠著上帝的Extra Grace.
是可以越來越.
開闊你的貢獻.
是可以的,一會兒我們詳細分享.
我覺得要挑選.
是所長.
如果你說.
期待永恆的階段.

$^{921}$我覺得如果你現在.
仍然只有五十多.
或者六十頭.
不要這麼快有這樣的心態.
你令到教會的牧師.
很傷心.
每一個都耍太極高手.
我已經去到五十多六十歲.
多給年輕人做.
年輕人又不長進.
我正在拼搏.
千萬不要找我.
多數人都在五十多歲.
你知道全世界最難的.
我不想用職業這個字.
最難的工作.
就是做牧師.
你怎樣說.
都有人不聽你的證.
剛才我初頭的祈禱.
好像以利沙一樣.
靠上帝的靈.
感動你,加倍感動我.
就是這樣.
我只要盡我本份.
講到上帝的心意.
上帝是否感動你.
與我無關,是上帝的事.
我們需要.
有一個新的改變.
為甚麼呢.
因為我們可以看到.
靠著恩典.
就好像鷹飛上去.
鷹飛上去很雄偉.
很厲害.
經文你都知道.
等候神仰望上天.
重新得力.
甚麼是重新得力.

$^{961}$就是現在沒力氣.
重新再起身.
那些白色的字.
你看到嗎.
我倒讀出來.
方下重擔.
絕去一切前類.
恢復神造我的.
榮美形象.
是恢復.
恢復神造我們.
是有一個很棒的形象.
很好的東西.
只不過當老祖宗.
犯了罪之後.
引導我們犯罪.
我們的身體沒辦法.
我們如何恢復它.
原來就是回到神的面前.
回到神有很大的恩典.
令我們有一個永遠的生命.
我們恢復形象.
我們在世上的時候.
都可以得到很大的轉變.
原來在神裡面.
是要有重新的定位.
要定位.
我相信今天在座的弟兄姊妹.
都是弟兄姊妹.
都是信耶穌的人.
我們這個講座和外面的講座.
有一個很大不同的地方.
外面的講座是講.
退休的人或將會退休的人.
如何學多點興趣小組.
參加多點遊船河.
參加多點這個那個.
那些都是好事.
都是好事.
今天講到這裡.

$^{1001}$我必須要告訴大家.
如何將你的靈命.
回到上帝的面前.
是通過什麼呢.
因為我們有.
Grace of God in Christ.
Grace就是恩典.
為什麼恩典重要呢.
原來這個恩典是從上頭而來.
是一種神.
是Unlimited Power.
即是說那個能量是無窮大.
你和我的能量.
當然不大.
因為你和我是人.
當然沒什麼力量.
睡少一點都容易病.
但是從上頭而來的能力.
是無窮大的時候就馬上改變.
所以很棒.
為什麼In Christ.
就是在耶穌基督裡面.
重新我們和上帝建立一個關係.
這個才是出路.
不是多吃些補品.
這個才是出路.
真的.
希望這一點能夠進入到你的心裡.
在今天的講座裡.
你明白到.
得到神的祝福是一件多麼棒的事情.
我舉個例子給大家聽.
亞國這位仁兄.
這位香港仔.
我常常覺得.
亞國是舊有聖經的亞國.
不是新有聖經的.
舊有聖經的亞國.
出名是實際的.
出古惑.

$^{1041}$你說這個香港仔.
有個好處.
神又寫了一個聖經告訴我們.
有一句話很厲害.
32章創世記26節.
祂說你不給我祝福.
我就不容你去.
即是說如果你不祝福我.
我就不讓你走.
是什麼意思呢.
祂很看重這種數量的祝福.
得到祝福才是最重要的.
如果你不給祝福.
我就不讓你走.
因為神的使者摔跤.
那個故事告訴我們.
我不讓你走.
祝福我之後我就不讓你走.
神很尊重這種人.
尊重這種人.
有一段聖經我一定要介紹給大家知道.
請你寫下.
在法帽2記上.
2章30節的下半節.
法帽2記上.
2章30節的下半節.
祂說尊重我的.
我必重看祂.
尊重我的.
我必重看祂.
英文聖經就比較清楚一點.
For those who honor me.
I will honor.
即是哪個人是honor我.
即是哪個人尊重我作為一個神的話.
我必定會尊重他.
必定看重他.
如果是這樣的話.
保守你的健康.
給你有更豐富的帶領.

$^{1081}$這是很自然的.
你尊重我你就會有這樣的福氣.
你不尊重我.
當神沒有到,當我被踢了一邊.
那當然沒有祝福.
不用說.
當然萬一遇到一些不太好的事.
你再求,你再悔改.
還來得及.
但是你如果真的很想.
上世祝福你.
你以後的人生.
你一定要開始尊重上帝的帶領.
尊重上帝的帶領.
OK嗎?.
這一點是我講完又講.
講完又講的重點.
如果我們沒有尊重上帝的心.
我們很難在靈明裡面.
再次再上一層.
這裡說到.
在神裡面更加上一層樓.
這樣的意思是什麼呢?.
這個意思是什麼呢?.
作為神的一個器皿.
你不要留意剛才的那個題目.
是要作為神的一個器皿.
器皿的意思叫vessel.
即是他手拿著你.
是一個工具.
是工具,給上帝使用的工具.
是什麼心態呢?.
你那個胸襟要改變.
改變就是說.
成功不一定在我那裡.
不一定在我那裡.
在神那裡.
如果在神那裡有成功的計劃出現.
當時有記錄下我的存在.
功成其中有我.

$^{1121}$成功不必在我.
功成其中有我.
這樣的胸襟.
上帝就會祝福.
因為上帝才是最重要的那個.
那你就做他的工作.
做他的工作的時候.
他一定會不單止是今世.
祝福你.
而且在來生永遠都是一個.
祝福的狀態.
我以前不是很明白這件事.
直到我開始奉獻.
開始做神的工作.
就越來越看到這個重要性.
越來越看到這個重要性.
因為我們要達到一些成功的東西.
其實是很多因素配合的.
很多因素配合.
講個很簡單的例子.
Wendy你一會數數人頭有多少.
我們.
希望第一次會報名.
其實不是想知道你這些什麼.
報名就是給我大概人數.
讓我知道印多少的廣義.
直到前天.
我們在.
我的記錄裡面是37人.
37人.
今天這麼奇怪.
剛剛有一批弟兄姊妹去考試.
所以就失了二十多人.
現在我想應該有六十多人.
三十多人我覺得.
你問我是不是滿意.
當然不是很滿意.
但我知道當我走在上帝的心裡.
這個房一定坐滿.
為什麼.

$^{1161}$因為每次都是這樣的情況.
而且我太明白弟兄姊妹的心態.
我太明白.
要報名.
要commitment.
現在人是最不想commitment.
在這種情況.
我就叫我的同工.
報名人數再印多二十份.
就差不多了.
就是這個意思.
弟兄姊妹我們很多時候.
信心都很軟弱.
我們看眼前.
我眼前是37人.
如果我看眼前我一定會很失望.
但當我們看到上帝的心意.
要完成的時候.
我們就不會失望.
因為信心將我們的平台提升.
成功與否都不是我們所concern的事.
而是上帝的心意能不能夠成就.
我舉一個例子給大家知道.
什麼叫做功成其中有.
69年的時候.
你們有沒有留意登陸月球的事跡.
現在是幾十年前的紀念.
三個太空人.
有一個太空人叫Michael Collins.
只有兩個太空人真正落到月球.
你們都記得.
一個是Neil Armstrong.
一個是Adrian.
這位Michael Collins在登陸艇上.
在月球裡面繞來繞去.
他的任務是要迎接這兩個太空人上去.
然後配合再回到地球.
而Michael Collins.
有人問他.
你這樣會不會很失望.

$^{1201}$這兩位仁兄真正成為歷史的創造者.
但他只是在太空船上繞來繞去.
這位仁兄我不知道Michael Collins是否信耶穌.
他回答得很好.
他說雖然沒有緣去登月.
但感激仍然在歷史裡面佔了一位.
這個心態就是上帝欣賞.
仍然佔了歷史的席位.
因為問題是.
萬一那兩位同伴不能回到太空船.
只有Michael Collins能回到地球.
當然很多人慶幸他.
但他們兩個人在月球裡面只好死.
在這樣的情況下.
雖然好像是一個Mark Man.
但會很悲慘.
當日的任務就是要帶他們兩個同伴回到地球.
這是歷史的任務.
如果我們服侍神的態度是有這樣的心態的話.
上帝就欣賞了.
是有一個團隊的.
上帝在歷史歷代有很多計劃.
在加拿大裡面有一個小小的地方叫多倫多.
多倫多有一個小小的地方叫建造中心.
等等我們在服侍的時候.
只不過在這個階段裡面.
一個很小的部分.
我們有一個席位.
回到天家的時候.
我們就會得到掌上.
我們不需要想得很大.
上帝就會欣賞一個.
是除去我們自我的Mindset的一個人.
因為只有他才是最重要.
只有他才是最重要.
投資.
其實投資在永恆是一點都不容易.
因為我們很多時候的困難就是我們太過疼愛自己.
我不知道這個疼愛字是不是用這個色字.
是不是.

$^{1241}$是廣東話來的.
是疼愛自己.
這個是很多人的困難.
你不能抵賴.
包括我自己在內也是.
有時候都很疼愛自己.
想睡多一點休息多一點.
其實很多原因我們是太疼愛自己.
很多原因.
有時候是看低自己的潛能.
沒有了已經玩完了.
又或者是覺得離開我們的安樂窩是困難的.
安樂窩是什麼意思.
安樂窩就是最安樂的地方.
離開安樂窩當然不好.
誰那麼傻會離開安樂窩.
另外一個原因.
真的很貼切的原因.
對不起如果我真的說得這件事.
我講得小聲一點.
就是一個字.
就是懶.
後面聽不聽到.
後面聽到是吧.
是懶.
不是我講的不是路牧師講的.
是嘉義文這個偉大的神學家講的.
人類最大其中一個問題.
就是懶.
認識神裡面的懶.
懶惰的意思是什麼呢.
只顧自己.
也不提起勁.
對於其他事情沒有起勁.
各位我們今天.
其實你教會如果有牧師.
或者你教會有主任牧師.
其實是一個很寶貴的財產.
很多教會沒有主任牧師.
沒有主任牧師簡直是沒有了一個掌舵人.

$^{1281}$其實是很危險的.
但是有一個主任牧師又如何呢.
如果這個主任牧師不是很清楚知道上帝的心意.
很願意被上帝使用.
帶領弟兄姊妹.
更加走向神那裡的話.
這個牧師只不過是在做他的職業.
做他的職業.
上班下班.
有些弟兄姊妹問我.
你現在上班多好多開心.
對不起我現在不是上班.
我現在不是上班.
我現在是在做另一個minister.
另一個minister.
我覺得人生是需要有一個更高的目標.
來推動你的.
就好像這個火柴.
一支火柴燒著.
其實不是什麼特別.
一支火柴燒著.
周圍一排的火柴.
就馬上就很熱.
就燃燒了.
所以這個.
是需要被刺激.
被刺激.
今天我想我們最大的困難.
就是不願意接受刺激.
你有時候會覺得刺激很妥當.
找這樣的刺激.
找個過山車.
你請我去過山車.
我都不願意.
但在靈裡面的刺激.
我覺得你不妨要考慮一下.
要多受刺激.
你怎樣多受刺激呢.
請你在教會裡面.
聽道的時候.

$^{1321}$張開你的耳朵.
聽聽上帝的僕人對你說什麼.
對你說什麼.
在台上對你說話.
說到那個人不再是那個人.
不再是那個牧師.
現在上帝對你說話.
你就聽.
而且是留心去行.
我相信這個提醒.
是重要的.
因為它會帶領你.
有一個新的改變.
就是你要投資.
在永恆價值裡面.
投資在永恆價值裡面.
說起來好像很虛無飄渺.
待會你會知道.
我先講原則.
投資在永恆價值裡面.
是什麼呢.
你在有限的時間裡面.
有無限價值的東西.
就叫做投資在永恆.
舉個例子.
我自己來說.
我希望能夠多活15年.
或者20年.
那就算了.
陳比我再多一個年日.
就是神的事.
我期待就是這個狀況.
在現在這個時間大概有15年.
一個短短的時間.
能夠做到一些東西.
是有永恆價值的時候.
你要投資.
你也有投資.
不要騙我.
你也有投資.

$^{1361}$投資的意思是什麼呢.
在一個很短的期間裡面.
投資.
讓錢慢慢成長.
到後期就會有很多東西可以收成.
那就很棒.
今天要投資.
其實投資在永恆.
是很值得考慮.
為什麼呢.
我自己來說.
以前我自己覺得.
我沒有投資的觀念.
我覺得我的工作是要求我寫文.
寫作就是我為了生存需要做的事.
這些不是壞事.
這是生存的要求.
但是如果過份的話.
就會被法輪功蒙蔽了我的眼睛.
我看不到更加重要的是什麼.
就是更加有永恆價值的東西.
開了我的眼.
讓我覺得我願意放下.
不再繼續追求那些短暫開心的東西.
短暫到什麼呢.
就是三個月開心的東西.
有些人不知道為什麼我這麼說.
因為出了編文.
期刊就三個月.
三個月之後就沒有人看你的東西了.
你的價值就只有三個月.
不斷地三個月三個月三個月.
其實意義不是很大.
但是當我今天能夠投資多一點.
在永恆的事的時候.
就絕對划算了.
路加坊這段經文你可以回家自己看.
不義的管家.
你應該聽過很多次了.
那個不義的管家真的很厲害.

$^{1401}$耶穌說這個人不義的管家.
今世之治.
比信耶穌的人還聰明.
因為他懂得想怎樣為將來.
為永恆的東西來鋪排.
因為他欠債.
老闆炒他魷魚.
你記不記得那個故事.
老闆炒他魷魚.
他就很聰明.
誰欠了老闆的債.
欠了一百.
他就還八十.
討好債主.
讓他以後有朋友.
多點朋友.
哇 聰明.
你做不做一個聰明的人.
我相信你很喜歡做一個聰明的人.
所以連耶穌都說這個不義的管家很厲害.
他不是叫你做不義的東西.
但說這個不義的管家.
這個人原來都有這麼聰明的做法.
你應該懂得怎樣做.
懂得怎樣選擇.
我覺得最重要的是什麼.
接受作神的用人.
我不是叫你做全道.
不要搞錯.
如果你沒有神的呼召.
我真的從心底裡說一句話.
沒有神的呼召.
千萬不要做全道.
因為你不會撐得很久.
真的.
頂心頂胃.
你不會撐得很久.
唯有你真的有上帝的呼召.
你才做全道的工作.
但不是每個人都做全道.

$^{1441}$但每個人都要做神的用人.
上帝可以使用你在某些地方.
最穩妥.
因為這個好老闆絕對不會虧待你.
絕對不會虧待你.
因為你求他extra grace就給你.
這麼簡單.
這位仁兄你一定聽過.
Jeremy Lin.
林書豪.
華裔NBA的球星.
前一陣子他在Raptors.
Raptors也贏了.
是冠軍.
他也有獎勵.
這是飄浮圖.
Jeremy去到台灣.
他本來是台灣人.
他講了一個報道會.
講到他自己身世的時候.
很感動.
他哭了.
他講了一句話.
困難的地方是.
他不是頂尖的NBA球星.
現在打完球.
要等其他球隊來招募這些球星加入.
但好像一直坐冷板.
沒有人來追他.
他很失望.
他講了一句話.
不需要失望.
當你知道你是誰的時候.
你不再是成功的某某東西.
就不再重要.
意思是.
當你知道自己是神的兒女.
身份的時候.
就算你不再是成功的NBA球星.
也不再重要.

$^{1481}$最重要的是.
你要確保你是神的兒女.
你就有一個福氣的源頭.
能否做到NBA最頂尖的球星.
不再重要.
正如我自己來說.
當年我移民到加拿大.
差不多25年.
頭幾年的時候.
我非常大的雄心.
要打回大學教書.
試過三次.
三次都是最後一關.
我過不到.
當然失望.
當然有點失望.
但是回頭想起來.
正因為這樣的難阻.
才讓我更加.
學習怎樣一步一步的去侍奉.
老實說.
如果當年我成功了.
回到大學教書.
今天的我再不是現在的我.
我不會做傳道.
更加不會做牧師.
一定不會.
因為世界的東西更加吸引你.
因為現在我們知道.
我們是上帝的兒女的時候.
有更加寶貴的東西.
我們就不會覺得這些東西那麼重要.
說來說去.
重不重要.
其實我們人有很多限制.
人有很多限制.
限制很狹窄.
我們怎樣去飛越它.
我覺得我鼓勵你.
這段經文留意一下.

$^{1521}$這段經文很幫助我們.
是保羅講的.
我們一起讀.
我們這至暫至輕的苦楚.
要我們成就極重無比永遠的榮耀.
原來我們不是顧念所見的.
乃是顧念所不見的.
因為所見的是暫時的.
所不見的是永遠的.
你看到我這裡有藍色的字.
有紅色的字.
藍色的字相對於紅色的字.
下面就列了英文聖經是怎樣講的.
更加清楚意思.
那些很不開心的東西.
很多限制的東西.
是苦的東西.
但其實是很小事.
很短的時間.
在永恆裡面不值一提.
很簡單的東西.
Momentary Light Affliction.
但相對於極重無比.
不得已.
還會是永遠的榮耀.
這句翻譯得非常厲害.
Eternal Weight of Glory.
當我們看到紅色的字的時候.
藍色的字帶給我們的衝擊.
就很大.
我們看到紅色的字.
帶給我們的衝擊.
就很大.
藍色的字帶給我們的衝擊.
就小意思.
我們怎樣看今天遇到的難處呢.
我們就要看藍色的字.
看得輕一點.
看紅色的字看得重一點.
你就有完全不同的改變.

$^{1561}$下面那句更加講到.
所能夠見的那些.
只不過是暫時.
但那些不見的是永遠的.
不見的是永遠的.
我覺得聖經是怎樣鼓勵人心呢.
就是這些說話.
這些說話真的有內裡的靈的能力.
很重要的.
希望大家都注意一段經文.
給我們的鼓勵.
或者人生有很多錯過了的東西.
已經有很多東西.
我們過錯了的機會.
其實在神裡面仍然有再起.
我覺得仍然有再起.
譬如在這裡講到.
在世界裡面我們似乎是被人打敗了.
或被人拋棄了我們.
不受我們玩.
我們沒有那個能力.
失去了那個機會.
再沒有機會可以進入某些場景裡面.
但不要緊的.
在神裡面.
實質是能夠重新有動力.
而且不單止這樣.
還可以擴展自己.
為什麼我說可以擴展自己呢.
因為原來在上帝永恆計劃裡面.
有兩樣東西是值得留意的.
第一樣東西就是.
我已經有了.
第一樣東西就是.
我已經有一個很特別的身份.
原來是上帝的兒女.
很特別很特別的身份.
是上帝的兒女.
還有就是我可以被上帝使用.
不過唯一的條件.

$^{1601}$唯一的條件就是你肯不肯.
你願意才行.
因為上帝從來不會勉強一個人.
要跟著他走.
從來不會勉強.
只不過是你跟著我走.
我鼓勵你跟著我走.
他不會逼你.
逼你沒有什麼意思.
就好像這些水晶.
水晶.
這些水晶是很寶貴的.
很寶貴的寶石.
但經過要琢磨.
要經過琢磨.
琢磨就是讓神慢慢雕琢到你這個地步.
不再是很多菱角.
一個觀念.
跟著的觀念是什麼呢.
我們真的要為退休而準備.
我不太知道有部份弟兄姐妹.
在這個房間裡面.
接近退休.
我真的很想提醒大家.
要準備.
你不準備.
你到時會倒水落蟹.
我的經驗就是.
因為我提早退休.
頭兩個星期我無奈受不了.
為什麼冬天不需要零下二十度去搭TTC.
真是不知多爽在家裡.
看著那些人去TTC.
搭TTC真是活該.
我在家裡真是不知多爽.
但過了兩個星期後.
你會發覺渾身不對財.
做些什麼都好.
難道等吃等死嗎.
要準備.

$^{1641}$因為我自己來說.
一直都有侍奉工作在教會裡.
所以我不覺得沒有來不及.
很多事情要做.
不過我想提醒大家.
要準備.
準備什麼呢.
就是財富的觀念.
財富人人都想.
有誰不想財富.
沒有.
你不要騙我說不需要財富.
我怎樣都不相信.
財富一定人人都要.
不過問題是.
只要看財富一面.
地上的財富.
現在銀行戶口有多少錢.
地上的財富.
不單看地上財富.
還要留意天上的財富.
如果.
現在這個房間裡.
的人有.
未信的朋友.
我會有另一個說法.
假設大家都是相信耶穌.
我們要留意天上的財富.
天上的財富.
是你現在.
計算不到的.
但上面有個鎖帽.
計算好.
到你上去上面的時候.
你會知道原來某年某日.
有人說了某段說話.
那個人很受感動.
因此改變了.
有個大checkmark在那裡.
歸功於你.

$^{1681}$但現在我根本不知道.
我現在只是在說話.
不知道你受了多少.
你接收多少我真的沒有辦法.
可以控制你.
但在將來的時候.
我就知道我今天所說的話.
究竟起了多少個效.
我想我們天上的財富.
是有一定增長.
大家都明白我們投資錢在銀行.
都希望有修飾.
或者買一些互惠基金.
都希望他賺錢.
投資股票都希望他不斷上升.
很自然.
財富不單止存在.
而且要投資.
投資的觀念.
很多人在退休的生活裡.
錢的投資就有.
但怎樣投資永恆.
不是很多人有.
我想給大家一個參考.
假設你地上的存款.
有N個零在那裡.
前面有個字.
假設是一字.
這個一仍然是最重要.
沒有了這個一.
你有十萬個零都好.
都是零蛋.
十萬個零都好.
都是零蛋.
所以這個一是很重要.
這個一可以被調到天上的時候.
就是說我們要.
放在耶穌的道裡.
我們整個人生.
就是一個完美的一.

$^{1721}$完美的一.
這本書.
是一個J.I.Packer寫的.
一個翻譯的本.
我覺得很有意思.
叫做榮耀神的夕陽.
夕陽無限好.
聽過吧.
是在說退休年齡的弟兄姊妹.
是夕陽.
但夕陽好像很悲.
但不是悲.
J.I.Packer的說法不是悲.
而是一個榮耀神的夕陽.
就是說.
仍然是沒有退卻的一個目標.
沒有退卻的一個目標.
你看不看到那些英文細字.
我讀給你聽.
有三句.
Finishing our course with joy.
Guidance from God.
for engaging with our agent.
第三句.
Engaging with our agent.
第三句就是.
你如何面對你慢慢漸漸年紀長大.
你要engage這件事.
你要接受這件事.
第一句是什麼呢.
Finishing our course with joy.
你一定要看到這個世界.
在你人生裡面都有一個終結的時候.
一定有的.
你不能避開.
中間那句就重要了.
就是靠上帝指引我們.
靠上帝指引我們.
我們就能夠在一個愉輝的時候.
發揮很大的改變.

$^{1761}$愉輝好像很悲慘.
但仍然有光亮.
你還沒到黑暗.
愉輝.
下面這裡.
你細心看.
你和我也買過房子.
你有沒有留意一些東西.
你房子的.
向著哪個方向.
影響你的房價.
有沒有留意到.
可能沒有留意到.
因為有一個小小的調查.
多倫多.
朝向南邊.
門口朝向南邊.
值錢.
向南向北.
中國人最喜歡.
曼錦市朝北.
什麼理由都不說了.
因為我不想side track大家的注意力.
這張slide的注意力在哪裡.
就是你今天的朝向.
你的orientation.
你的朝向目標.
決定了你在城市的價值.
你在城市的價值.
當我們上班的時候.
你和我也打過工.
都有頂頭上司.
老闆.
老闆怎麼看你.
就看你自己的業績.
動力有多少.
對公司的價值有多少.
肯定是這樣的.
不然就炒魷魚.
我們在神面前.

$^{1801}$幸好沒有炒魷魚.
沒有炒魷魚.
不過就有獎賞.
越多越少的問題.
如果你今天的朝向.
越朝向神.
你將來去到上帝面前.
上帝看你的價值.
就很豐富.
怎樣才能夠.
增加你的價值.
就是我們現在.
一個最重要的課題.
在我們現今裡面.
最重要的課題.
你的朝向.
你的orientation在哪裡.
說了很久.
現在是中場休息的時間.
大家都把握機會.
在休息的時候.
去洗手間.
喝杯水.
吃一點點餅乾等等.
好.
這個時候.
跟大家分享一句.
作為建立中心的主任.
我們都要和大英姐妹交代.
我們是.
全部都要.
依賴大英姐妹的奉獻.
我們才能夠生存.
還有.
在很多的工作上.
大英姐妹的意見.
是鞭策我們的.
還有我很努力.
在過去這一年半.
讓大英姐妹看到.

$^{1841}$在建立中心的工作是有前景的.
是不斷地.
向前推進.
就盼望大家在.
禱告裡面紀念我們.
也紀念我本人.
一會兒我會說到.
有些最基本的.
為我祈禱.
健康.
我越來越看重健康.
因為沒有健康身體.
就不用玩了.
就馬上玩完了.
身心靈都壯健.
在經濟上.
都很依靠.
大家的奉獻.
我們有兩個奉獻箱.
出門口有一個紙皮箱.
如果大家.
奉獻可以寫check.
或者現金也好.
如果超過20元.
我們會發補稅收條.
給大家.
今天派那裡.
都有一個奉獻封.
填寫卡後.
摺起來放進奉獻箱.
就可以了.
我每次.
奉獻的時候都提醒大家一件事.
今天的奉獻.
不是給.
陸牧師.
嚴格來說不是給.
電腦中心.
而是給神.
所以是一個很尊崇.

$^{1881}$很尊榮很慎重的一個.
行動.
有一個很豐富的屬靈意義.
因為我們將我們的.
心思意念.
錢財放在神的工作.
上面的時候.
上帝就會尊重我們.
祝福我們.
所以這個時候我想請大家起來.
為奉獻做一個祈禱.
我覺得奉獻的時候.
不是坐在那裡.
這樣不是很尊重神.
奉獻是我們尊榮神的時候.
請大家起來.
我們一起做一個奉獻的祈禱.
主要我們.
多謝你.
因為我們藉著救恩.
我們已經進入過門.
能夠成為你的兒女.
我們成為你的兒女的時候.
我們更加有機會可以.
參與將我們的.
奉獻放在你面前.
我們心信主義.
你自己是祝福這些奉獻.
不單止是.
借助這些金錢.
能夠使建築中心的工作.
能夠得以再繼續.
維持下去.
甚至是擴展.
而且更重要的是你是體重.
我們對你的心意.
因此我們祈求你在.
這個奉獻的時候.
將弟兄姊妹的心.
能夠再一次放在主你自己身上.

$^{1921}$讓我們有機會.
可以在奉獻的事情上.
有份來我們自己的榮幸.
我們求你.
繼續奉獻這些奉獻.
我們心信你體重的奉獻.
不單止是金錢上的奉獻.
連我們自己整個人.
已經奉獻在你的面前.
是更加寶貴.
求主都繼續.
施恩憐憫.
繼續感動我們.
我們簡單禱告奉耶穌名求.
阿門.
你好.
麥克風好像不太響.
後面聽到嗎.
好像剛才好像越來越小聲.
好.
如果大家有奉獻.
你可以遲些等待奉獻商就可以了.
我想在這個時候.
去到第二部分.
先說說我個人的見證.
因為我個人的見證.
是很具體的.
很具體放在大家面前.
讓大家參考.
因為不是每一個人都像我一樣.
但是那個路向.
原則都是一樣.
我只不過是.
看到一個很平凡的人.
好像大家一樣.
一個移民來加拿大生活的一個家庭.
當我們慢慢慢慢.
生活安定下來之後.
即是.
加拿大都不會怎樣發達的.

$^{1961}$明白嗎.
要發達就不要來加拿大.
我看自己不富又不貴.
沒有什麼現金.
加拿大人多數都是這樣.
當我們年紀開始漸長的時候.
我們就會留意.
其實我們看自己是一個很平凡的人.
但是在不平凡的人裡.
上帝又很奇怪.
給我有機會過一個.
不是很平凡的人生.
我告訴大家為什麼我這樣說.
是給我有機會的.
如果那個機會不去拿.
就沒有機會.
看不看得明白.
機會你不去拿就沒有機會.
即是你要肯去接才行.
全是讓上帝的恩典被挑選.
不是因為我自己有什麼特別的地方.
是上帝給我有機會.
見到一些機會.
就有機會.
願意承擔.
這樣就可以了.
所以明天上帝給你有機會.
你就不要那麼快去拿.
因為你拿的話.
機會就失去了.
為什麼我會覺得.
我過著一個不平凡的人生.
因為這個都是發生在我.
大概接近退休年紀的時候發生的.
大概是幾年前的事情.
幾年前的時候我還在日間工作.
在downtown上班.
其實都覺得離開.
真正退休的年日都不長.
心態上都覺得好像快要摺好包袱.

$^{2001}$停頓了.
大概四年多之前我就提早退休.
在職場上離開.
不再做了.
轉變做了我們教會義務的全部.
接著大概做了五年左右.
就按立做了牧師.
這個就是我自己的經歷.
不過特別的經歷就陸續有來.
因為有些領教不認識我.
所以我簡單講一講.
我大概過了六十歲的時候.
我就做了一個很大的決定.
經過大概三年的祈禱.
要那麼長的時間的祈禱.
這個很重要.
不是那麼簡單的決定.
經過三年的祈禱之後.
我才決定再次進修神學.
一個更高的學位.
那個是一個不容易做的決定.
因為錢,健康,有沒有一個好的題目.
有沒有好的學校等等的問題就出現了.
所以很審慎.
最後就決定進修神學.
在美國裡面.
我去美國修課.
一年去兩次.
每次大概一個星期左右.
過程是很辛苦的.
是很辛苦的.
上帝給了我很大的機會.
再一次進入大海洋裡面.
才發現原來我知道是那麼有限.
我在這裡見到中心教書.
教了有八年左右.
我一直都是教.
我那科叫做圖解系統神學.
用一些圖表.
好像現在看到的那些方法.

$^{2041}$去介紹一些比較抽象的神學觀念.
我自以為對系統神學比較有認識.
教了八年.
當我去到學校的時候.
今年初的時候.
我花了三個月的時間在那裡.
做了全職學生.
我發覺原來我知道的東西那麼少.
在大海洋裡面才知道原來自己是小坑.
更加是在更加多的厲害的同學面前.
真的連影子都沒有.
一會兒我會講一些見證.
你就知道多慘.
另外一件事要交代.
大概一年半之前.
未到一年半.
我就接受了邀請.
來到加拿大建築中心做主任.
這也是我另外一個很大的變動.
因為初時想著讀書.
其實我已經是義務職務師.
上輩子開了很多路.
我不跟教會講道等等.
我覺得已經很滿足了.
讀書了四年.
完成了學位.
寫了兩本書.
我就收攤了.
初頭的計劃就是這樣.
誰知突然間就發出這件事.
改變了整個計劃.
改變了.
我四年讀不完.
真可憐.
六年讀完已經很快了.
經常都跟大英姐分享.
我不是重出江湖.
我離開了日頭職場.
離開了幾年.
現在好像又回到工作.

$^{2081}$又有paycheck.
我最大的感受.
以前開始離開職場就沒有paycheck.
心裡的感受很大.
你習慣了paycheck就不覺得.
當你沒有的時候就很覺得.
現在又有paycheck.
心裡就有兩種感受.
但現在我不是重出江湖上班.
而是我重新調動我的侍奉方向.
讓我有很多機會受到操練.
這是什麼操練呢.
有五樣操練.
我告訴大家.
這些操練你一樣可以.
真的.
你一樣可以.
第一方面.
先講多方面的學習和感恩.
對我自己來說就是學習.
對神來說就是感恩.
你說在健康中心的侍奉難不難.
如果你問我這件事.
我告訴你.
我不會隱瞞給大家知道.
在這裡的侍奉工作來說.
是超難度.
超是誰超呢.
就是陸昭明那位.
是超難度.
我想都沒想過原來這麼難.
但差不多接近一年半.
我發覺在這麼艱難的情況下.
我可以捱得住.
而且是越來越開心.
和越來越大膽.
待會我分享一點給大家知道.
第一件事就是我退而不休.
我的經歷就是在逆場退下來.
我侍奉反而越來越多.

$^{2121}$為什麼越來越多呢.
因為我看到意義在裡面.
在計劃裡面已經找到一個競爭點.
在神學舊約裡面來說.
我是可以用到我自己學的東西.
和和弟兄姐妹分享.
我在詩書裡面做牧師.
一定是日子.
學了怎樣和弟兄姐妹相處.
退休或者接近退休人士的弟兄姐妹.
就是我服侍的範圍.
那成為我服侍的焦點.
當焦點清晰.
有更新的侍奉刺激的時候.
我知道上帝給我一個新的計劃.
我投身是很有焦點去做的.
第二就是我學到不斷的改變.
剛才說的舒適區這個觀念.
就是安全的地帶.
有一天很奇怪.
在我坐巴士旁邊有一個人.
在看一本書.
那本書的字體很大.
有一個人說舒適區.
我告訴大家.
舒適區是一個很美的地方.
但那裡沒有什麼東西長大.
原來一個人在舒適區裡面.
是很享受的.
是漂亮的.
不過沒有什麼新的東西可以做.
因為習慣了.
正如我以前.
我覺得我的舒適區是教成人主學.
如果我只是覺得教成人主學.
已經是我的原版.
就僅此而已.
我永遠都不學到.
待會我想和大家分享的東西.
我永遠都學不到.

$^{2161}$因為我要學到這個東西.
一定要離開舒適區.
才學到.
再說多一次.
這句話很有意思.
舒適區是一個很美的地方.
但沒有什麼東西可以種出來.
可能這句話誇張一點.
但這句話也有它的意義.
在我們建築中心的時範.
真的超難度.
因為我要不斷地適應.
不斷地改變環境.
我不可以用舊的思維去運作建築中心.
我一定要不斷地想辦法.
有時候我覺得.
我不是怪責神.
如果我再有機會.
我會學marketing.
marketing.
因為marketing令到整個人的活力增強.
但你要留意那個變動.
你要留意那個趨勢.
受眾的趨勢要留意.
但上帝很奇妙.
給我很新的思維.
往往是神運的時候.
我走trail.
或者是走park的時候.
一邊祈禱.
突然間就有些觀念彈了出來.
我馬上寫下來.
影響我的思維.
就有變化.
要變化舊有思維一點都不容易.
因為舊有的事人人都已經在做.
可能有成績.
現在要改變一定要justify.
但我看到原來新的變化有一個好處.
就是可以更加讓我看到原來上帝的恩典在背後.

$^{2201}$因為我是不熟悉那件事.
我常常說神學院為何不教以下的東西.
我覺得神學院應該要教那些東西.
不然人們出來做傳道都會倒寫落鞋.
特別是服務機構像我們這些.
又沒有教怎樣籌款.
又沒有教我們怎樣推廣.
又沒有教我們人事管理.
又沒有教我們很多年來整水管.
有一次我們爆水管.
我都是倒寫落鞋.
真是慘.
這些東西完全沒有教.
突發事件出現的時候怎樣教呢.
神學院不教這些東西.
神學院只教你Bible, Theology.
要實質你真的下手下腳去侍奉的時候.
這些才是最實際.
但沒有教這些東西不要緊.
就take up它.
我用回我剛才說的板斧.
求祂給我extra grace.
能夠可以渡過那個難關.
這個真是難學.
如果以前的觀念.
如果你說做一份工.
不是講侍奉,是做一份工.
我就不做了.
我辭職不做了.
沒有人拿槍戳著你一定不做.
就算我現在辭職都沒有人可以阻止我.
但必須要堅持下去.
因為這是上帝開的路.
必須要堅持下去,不斷改變.
第三,做我們這種侍奉.
你一定要有信心.
信心就是trust in Him.
一個最明顯的trust就是金錢.
我有一個例子想和大家分享.
剛才我說過我正在進修.

$^{2241}$我爸爸離開世界回天家時.
留下少少的錢.
那時我計數計到這四年的學費.
我應該ok的.
不過多去幾次後就發覺.
原來很貴.
因為有飛機票,有吃有住.
每次要千多元加幣.
現在我不是在加拿大讀,是在美國讀.
加幣對美金是怎樣的你都知道.
多去幾次後發覺.
死火了.
真的捱不捱得住呢.
還有四年這麼久.
我想既然上帝開了路.
我祈禱了三年,他都是這樣帶領.
我就玩轉彎去.
不要想太多,沒理由.
打退堂鼓,沒理由.
就照樣去.
2017年見到東森的職位就來到我面前.
我都沒有開口求神.
供應我需要,我都未開口.
祂就供應我這個需要.
那張paycheck.
是高過我想所求的很多.
極豐富的恩典.
錢是最能解決我困難的地方.
可能你有這個經驗或未有這個經驗.
一個義務全道是沒有薪水的.
即是說如果你有少少pension都還好.
如果沒有就沒有了.
在沒有的狀態下.
即是你沒有regular income的情況.
而你又經常要交學費的時候.
你可想而知那個艱難的地方有多大.
你大概可能明白.
我那個難處.
但後期我發覺上帝真的給我很大的好處.
我只相信祂.

$^{2281}$一直會帶領我下去完成那個學位.
因為已經很清楚.
第四和第三有關係.
難處不要怕面對.
我發覺現在臨到我的難處只會越來越大.
為何會越來越大呢.
因為上帝覺得我可以頂得住.
所以就給我多點難.
祂不是想虐待我.
不是想虐待你.
祂是想教你一步一步更加向前走.
我舉一個例子給大家知道我的難處在哪裡.
剛才我講過.
我年初去美國三個月長時間讀書.
和外國人鬥其實很難.
全班同學年紀最大就我.
換句話說.
思考比我快.
英文比我好.
各方面都比我豐富.
而且班特別細.
你走都走不掉.
你一定要硬著頭皮去.
什麼面都沒有了.
這樣的狀態.
我和師母和太太講.
我想退出.
我頂不住.
不如改另一個小學程式.
何必讀系統上學這麼辛苦.
和這些外國人鬥.
真是慘.
慘到不慘.
不如轉另一個小學比較容易.
人當然是這樣.
想跳開艱難的地方.
只等了兩天.
我祈禱等了兩天.
我最後決定.
死都要捱下去.

$^{2321}$為何呢.
因為我想到一個.
最大的信心功課.
我不是亂來.
我和神說了.
為什麼我會選擇這個小學.
復修.
是想幫助我的主修.
畢業論文.
有根有機去做.
是一個很重要的重要一環.
神既然這樣帶領我的話.
他沒理由中途.
弄我傻的.
我說神一定要幫我.
你不幫我就死了.
我祈禱很卑微.
你讓我合格就OK.
如果不合格要重修.
很慘.
你讓我合格就OK.
但上帝真的去到一個怎樣的地步.
我修兩科.
最後的結果.
兩科都不是僅僅合格.
而是超過我所想所求.
這是一個恩典.
令我覺得上帝的恩典很大.
他的恩典落到我身上.
不單落在我老師身上.
又落在我周圍的同學身上.
大家很開心.
那個經歷真是大.
時間不多.
所以不講太詳細.
有機會再和你們分享.
不過那個經歷.
令我有一個很大的感受.
我一定要回首.
看回上帝帶領.

$^{2361}$我安然渡過這個難關.
我經常和大家分享一句話.
是我心裡的說話.
關關都難過.
但關關都過了.
你回頭一想.
你就會有很大動力.
我覺得這個神真的會帶領我們去到一個遠岸.
我們可以繼續去.
第五很簡單.
身體沒有走下坡.
只要我肯去侍奉.
上帝就會祝福.
弟兄姊妹會有條問我.
為何為你祈禱.
第一個你為我祈禱很簡單.
為我身體祈禱.
但我發覺越來越精靈.
因為我經常都要用腦.
很多弟兄姊妹.
只要我講一句.
如果講中了就對不起.
很多弟兄姊妹因為沒有什麼目標.
所以一離開職場.
很快就.
健康就直走下坡.
不是說吃東西不好.
而是沒有動力.
有科學證明.
你的腦一直在動.
你就會越來越運作得好.
球主都讓我繼續有健康.
如果沒有健康就不用了.
對不起.
所以我這裡就講了結論給大家.
嘩很七彩.
七彩.
我現在就講一個結論.
七彩.
每一個詞都有意思在背後.

$^{2401}$今天你看到我的分享.
其他那些聽不明白或不記得也不要緊.
你只要記得這張就行了.
這張才是最重要.
在上帝恩典之下.
不斷地尋求突破.
其中有六個點.
第一就是上帝.
上帝就是我們幫助的源頭.
幫助的源頭.
你靠人要靠得多.
你靠父母要靠得多.
就算你的配偶也幫不了你很多.
你要靠神去幫你.
第二就是恩典.
具體的尋求恩典.
剛才我說過了.
很有效.
艱難到一個地步.
有一次我要寫一篇批評.
其他同學的文.
我花了三個小時.
我看不明白他們在說什麼.
我就去吃飯.
我就一路走一路祈禱.
神啊你真的要救我.
否則我明天怎麼教那篇文.
我要評論對方的文.
還要寫六百字.
一集.
怎麼勾也勾不到六百字.
上帝恩典就來了.
一路走一路祈禱的時候.
喂.
你可以避重度輕.
這麼細節的事你不要說.
當然說些空泛一點.
當然說句好.
不要說些壞話.
我回到宿舍的時候.

$^{2441}$就開始寫.
兩個小時後我寫完.
那晚睡得多甜.
因為一開始想就死了.
那晚開通宵都搞不定.
因為寫給對方的文.
我們要計分.
人家又要批評我那篇文.
我又要批評別人那篇文.
真慘.
上帝恩典extra grace就會出現.
隨時隨地都出現.
加入下一個字.
下一個字是很重要的.
我們經常都要在上帝之下.
我們要humble ourselves.
前書五章第六節講得很好.
中文的翻譯都不好.
你不要自卑.
自卑什麼.
自卑不是很慘.
其實是humble yourself.
你要自謙.
你要humble yourself before God.
神就會將你升高.
但很簡單.
不要再覺得你自己很棒.
一點都不行.
我在過去幾年的經歷.
尤其是這年半的經歷.
告訴我一件事.
如果我沒有神的幫忙.
我全部都難行.
隨時分分鐘都可以跨台.
分分鐘都可以跨台.
在我讀書也好.
在我這裡電腦中心的事務也好.
一樣會跨台.
那就慘了.
反正沒有那麼多時間告訴你們.

$^{2481}$有機會請我吃飯.
就多說一點.
第三件事就是下.
第四件事就是不斷.
意思是什麼呢.
不要停.
一停就會懶散.
不斷的.
不斷什麼呢.
尋求.
向上問.
請你給我一些資料.
好像我剛才說的經歷.
很憐憫.
阿克森也很憐憫.
給我一個機會可以解決問題.
尋求他的力量.
不單是尋求力量.
是尋求突破.
突破要break through.
因為突破是帶到我們不要守舊.
我們有一個目標可以去做.
目標可以去做.
牧師.
你說那麼多這些東西.
是你才行.
我說不是.
你一樣可以.
原理一樣.
不過我會很具體的告訴你們.
究竟你可以做什麼.
或者我可以做什麼.
我覺得這些提醒.
供應給大家參考.
其實來來去去.
回歸最基本的動作.
就是讀經和祈禱.
如果沒有這件事.
什麼都不用說.
你的個人讀經和祈禱.

$^{2521}$如果像我這樣是神運.
可以學習一下.
跑步也好.
行山也好.
祈禱可以幾樣東西.
多謝主昨天帶領我渡過某某難關.
求神給我今天面對的艱難.
我懂得怎樣去應對.
明天.
給我更精彩的目標.
昨天.
今天 明天.
這樣求.
你就會慢慢和神的關係變好.
這個不用說了 大家都知道.
第二就是對於那些已經有侍奉的弟兄姊妹.
我覺得以下的提醒.
我觀察到.
很多弟兄姊妹.
看侍奉.
是工作.
不要只是一個做.
真的不要只是一個做.
因為你的做你永世都不會怎樣長進.
而是你要看到.
藉著做侍奉這件事.
怎樣改變你和神的關係.
這個才是最重要.
和靈裡面的長進.
這個非常重要.
因此你在侍奉裡面.
當你遇到困難的時候.
不要劈炮.
走人 反而那時候.
你更加要堅持.
因為很可能.
上帝想藉著那件事.
告訴你哪件事.
你需要留意 哪件事你需要改善.
往往那件事才需要.

$^{2561}$千萬不要劈炮.
千萬不要走人.
做牧師最艱難.
看到弟兄姊妹就怕怕走人.
因為受不了.
不是受不了侍奉的工作.
而是受不了侍奉後面的人.
很難受的.
你要明白.
侍奉工作是很難的.
因為對著A B C D.
每一個都是罪人.
不是罪人也不會進來教會.
對不對.
每一個都是罪人.
罪人的罪會所教會就是這樣.
你不會覺得這條戲很不順.
有侍奉的就是這樣的要求.
但沒有具體侍奉崗位.
我就提議.
你可以這樣做.
求主帶領你.
要麼在A要麼在B.
有大類的.
你的侍奉要麼在話語上的侍奉.
要麼在服侍恩賜的侍奉.
服侍恩賜是怎樣幫人.
怎樣跟人交往.
怎樣探訪.
怎樣煮飯.
怎樣看廁所.
那些就是服侍的.
不要以為A那些是不好的.
或者A那些不重要.
A那些都是重要的.
配合的.
但有A沒有B又不行.
B那些是和上帝的話有關.
上帝話有關的又可以細分為兩類.
一類就是Bible.

$^{2601}$聖經.
讀多點聖經.
你都知道你都信了主這麼多年.
讀經讀經讀經.
另外一樣東西很少人提的.
我覺得除了讀經之外.
還要理解神的更加闊的層面的理解.
我們叫神學.
這些很嚇人的字眼.
其實是理解神的.
作為的一些知識.
其實我們無時無刻.
都被這些所謂神學的理解牽制著.
你不要以為你聽牧師講到.
甲的牧師,乙的牧師,丙的牧師.
講的東西是同一篇的聖經.
應該是完全理解一樣.
其實不是的.
剛剛可能不同.
因為神學的理解不同.
舉一個很簡單的例子.
救恩是不是會失去呢.
救恩是否會失去呢.
你不只是單是讀羅馬書就可以.
可能你讀很多古羅的書.
還有比特的書.
封印的書.
加上一疊這樣的理解.
那個難度是在這裡.
更加廣闊的理解.
我心裡面很大的熱心.
就是希望能夠在神學裡面.
用一些很顯淺的術語.
講給弟兄姊妹知道.
我們究竟信的是甚麼.
不要講那麼深的東西.
像我同學講得那麼深的東西.
全世界只有十多人知道.
那些是浪費的.
我沒理由踏在自己的台上.

$^{2641}$我只是講事實.
教會的層面是沒有用的.
教會是講道地的.
我想我們要學習這件事.
無時無刻都要留意.
這些理解原來有不同的觀點.
第四.
我覺得這個更加實際.
我知道有些弟兄姊妹是最優的.
錢財比我更加多.
我都不多的.
所以你的多可能多很多倍.
我覺得如何花金錢.
是上帝給我們一個很好的學習.
你有沒有想過.
財寶在天上.
才能夠永遠存在.
你的銀行會有很多億萬.
帶不走的.
你要想一下.
究竟我的錢.
是如何應用到最有意思.
這個你自己決定.
你要支持上帝某些工作.
是需要錢的.
如果你留意一下這段.
或者是世界宣教的知識.
你會留意到.
很多地方是不夠錢.
所以不做的.
為什麼不夠錢.
因為弟兄姊妹不奉獻.
哪有錢.
教會都還好.
機構像我們這樣的機構.
靠弟兄姊妹奉獻.
就更加艱難.
沒錢就什麼都不用說.
所以我都想告訴大家.
如果你問我.

$^{2681}$我都想告訴大家.
如果你某一種事工上有份.
你應該視為.
上帝給你一個機會.
參與協助這件事的出現.
幫助它成就.
成功推動出來.
你以後見主面的時候.
你就很膠著.
你看到表格裡面.
某某事工.
奉獻了一個人.
有個名字.
有多精彩.
這個很重要.
你問我.
金錢去侍奉什麼工作.
我當然說.
加拿大建築中心.
做推銷.
當然說自己最好.
我不可以說不好.
不過你可以考慮.
你可以考慮.
奉獻在神學教.
有一個好處.
那個目標是對平信徒的培養.
和有少少神學教育.
給一些弟兄姐妹比較尊心.
你難保他以後.
不會做傳統人.
很難說.
實質是更加多侍奉.
更加有實在侍奉的時候.
就非常之有果效.
我們覺得我們做的工作.
是有一個lasting effect.
所以你可以考慮奉獻給我們.
有些弟兄姐妹的奉獻.
是用遺產的奉獻.

$^{2721}$有沒有聽過.
遺產的奉獻.
我不是叫你將遺產的奉獻.
給建築中心.
我沒有這麼大力.
但如果你有這麼心.
你就跟我談.
打個折.
很多種金錢的幫助.
以上我所說的.
神學院有沒有教我的.
沒有教.
都是我亂撞.
撞下撞下.
學回來的.
上帝恩典真的很大.
有一次我見到一個牧師.
講一篇道.
就講到貼後.
第一章.
他有這樣的structure.
我覺得很有意思.
可以幫到我們.
好像今天的講題.
給我們一個很大的提醒.
當我們有壓力.
有困難的時候.
或者退休了的弟兄姐妹.
都有困難.
解決是怎樣呢.
原來是順.
順.
仍未冷.
未冷淡.
一冷淡的話.
很難再起來.
怎樣可以不冷呢.
就是這幾個紅色的字.
因為你知道.
末世耶穌幾多會再回來.

$^{2761}$日子不太遠.
你會知道有加以力度給你的.
因為祂的平安.
淋到你的身上.
你又會看到祂的glory.
很美的glory.
你又會起來要裝備.
裝備.
以致你可以承擔更多的事.
你不裝備.
你怎樣承擔呢.
承擔不起.
時間關係.
我想不講那邊.
只是這一點.
本來想講一整章的經文.
希望十分鐘講.
不過時間不夠.
但重點是.
以賽亞為何能被上帝所揀選.
其中一樣最主要的原因.
因為他看到上帝的榮耀.
在電裡.
Holy.
真的很厲害.
因為看到榮耀.
才能成為一個推動力.
他才能謙卑下來.
投身上帝的工作.
如果看不到榮耀.
是不會的.
人是很實際的.
最實際就是回家享受一下.
旅遊.
不是不好.
只是多了一點.
這裡不講了.
要裝備自己.
上帝怎樣用你呢.
你不裝備.

$^{2801}$那要用呢.
這是一個很好的例子.
要裝備.
軍人的衣服.
搭著的東西很多.
但帶上來就上戰場.
才有勇敢.
不然一上戰場就被打死.
同樣道理.
所以覺得.
裝備自己.
更能備神使用.
下半生能有更好的收穫.
其實這個觀念.
投資在今日.
如果你今日不投資.
明天又如何收穫呢.
如何能有收穫呢.
當你今日猶豫的時候.
你明天會後悔.
我們在建立中心.
舉辦一些科目.
尤其是文憑課程.
比較深的.
二十五小時的那種.
有些弟兄姊妹很有心去學.
但因為丟下書包太久.
學習上有難處.
我們就用另一個方法去幫助.
但如果他們再停下來.
不再把握機會去學.
明天他們一定會後悔.
因為擺在你面前的機會.
不是常有.
所以我們就用盡方法去幫助他們.
因為學習的情況.
只會越來越艱難.
我覺得.
這個擺在我自己來說.
如果我再不決定去進修.

$^{2841}$再等多五年.
我就可能會退休.
但我認為.
如果我再等多五年.
我就可能會放棄.
我的老師.
很擔心我不去.
屢次都提醒我.
為何還不來.
我又再祈禱.
其實很想去.
他也很想我去.
當然收多學生會好.
收益又大一點.
但在很多的考慮下.
就不太快去.
但他給我很大的.
一個感受.
雖然他年紀不小.
但大家都有.
很好的.
互相學習的機會.
實在太好了.
進修一點點.
有甚麼好處.
有很多弟兄姐妹會很怕.
聽到進修這個字.
讓我和大家分享一點.
其實進修的意思就是裝備.
很多等級的裝備.
不一定要很高.
要擴闊視野.
胸襟再不簡單.
只知道羅馬書說甚麼.
不過你說.
知道羅馬書說甚麼已經很不錯.
我會說.
知道羅馬書說甚麼.
他怎樣連貫.
整個保羅的思想.

$^{2881}$又怎樣呢.
進深一步.
視野就會被擴闊.
而且能得到新的思維.
往往這些新的思維扭轉你以前的看法.
扭轉你以前的看法.
牧師也不可以.
每個人都很精彩.
牧師有時也會很限制.
因為他自己不是教導的材料.
他講的東西可能有限.
當你有機會進修的時候.
你就會有新的東西走在腦裡.
第三點.
就是你堅定你所信的東西.
原來我以前不懂.
教會歷史.
原來知道2000年的教會歷史.
是息息相關和我們今天遇到的問題.
我們整個看法已經不同了.
當我們做教會領袖的時候.
如果我們沒有受過這些訓練.
我們看的東西會很窄.
或者最多只能看到我們現在教會面臨的問題.
或者社會對我們的衝擊.
看不到一個立體感.
對我們信仰的立體感.
以史為鑑.
更加是前人犯了一個錯誤.
要凸顯出來.
我們會更加小心.
你這樣踩下去.
遲早會踩到頭.
要通過歷史.
告訴我們才行.
第二是鼓勵操練.
你一定要有正確的教導.
才有正確的操練.
實質的方向.
有條理的提醒就會出現.

$^{2921}$其實我們這裡.
其實就是給.
弟兄姊妹一個選擇.
就是一些最簡單的.
培養.
最簡單的培育.
剛才我提到.
證書的課程.
這裡有一個.
具體行動的推介.
其實我正在賣廣告.
我會有一個科目在9月19至10月17.
每逢星期四晚.
就是那個時間.
這個題目就是.
金陵教會的挑戰.
所謂金陵教會的挑戰.
我們特別想講不是講老年人.
我們是講那些.
young old 新中年.
的年齡人士.
在裡面是怎樣生存.
這個教會究竟是危機還是轉機.
這個大的趨勢裡面.
視野應當是怎樣.
和出路是怎樣.
以致我們可以.
達到一個有意義豐盛的.
晚年.
你登出來的時候.
我忘記了,我打錯了.
不是萬年,是晚年.
我們侍奉的時候,我們一定要視野闊.
基於聖經.
也基於神學的看法.
我們才可以有一個.
好的培養.
為甚麼我會提出這個課程.
因為我發現.
在多倫多裡面有很多的華人教會.

$^{2961}$已經有這個現象.
老化得很快.
你能不能在我旁邊的.
你一定會見到很多金陵人士.
怎樣推動他.
上帝都給我機會.
做這個工作.
希望這個科目可以幫到弟兄姊妹.
既然大家今天來到這裡.
我再作一個決定.
如果你今天立即填了報名表.
全部都是優惠價.
75元.
省下15元.
省下15元可以買個飯盒.
你在報名表上寫著.
「六穆斯牌準」.
就變成75元.
立即填上.
就交給Christina.
我們就OK了.
我對你有幫助.
與此同時.
我們明年會推出.
其實我們這個科目.
金陵教會的挑戰.
是一個系列叫活力成年.
我會繼續教另外三科.
聖賢的操練.
靈命的操練.
和侍奉的操練.
這些科目都是.
我親身講到過去的十多二十年.
在做信徒裡面.
和做牧者的情況下.
我操練得來的東西.
和大家分享.
每一科都是12小時.
會陸陸續續推出.
給大家一個參考.

$^{3001}$明年我們還會推出一個新的系列.
就是牧人領袖.
怎樣做一個領袖.
我沒有這個料子.
做領袖.
你先做個牧人.
我們事先有特應.
不是我們.
我們都很常請國際勇牧師.
是其中一個講者.
他就會教第二科.
堂會領導.
檢視教政和更新.
我就會教是牧人領袖.
視野與實前.
很實際的東西.
其後會陸續有宣傳帶出來.
最後我想講以下的說話.
我們沒辦法離開身體健康.
我們沒有辦法離開身體健康.
這是一個規律.
你是秦始皇.
吃了長生不老的丹.
都沒什麼用.
始終前路.
總會有不穩定的狀態.
大家都理解.
求天父都引導你們.
真的有一個好的改進.
引導你和保守你.
從今時至永遠.
我們不是顧念眼前的東西.
我們顧念眼前的東西.
那些是永遠的.
那些永遠實在太好了.
我不知道怎樣用字眼形容好處.
但最低限度.
我自己來說.
先做一個示範單位.
我覺得很棒.

$^{3041}$這是一個永遠的生命.
我相信你一定喜歡祝福.
有誰不喜歡祝福.
不喜歡祝福請舉手.
肯定不會.
經上帝很大的祝福.
你就可以有機會再起.
我現在送給大家.
阿倫的祝福.
祝福大家.
我們一起祝福給大家.
我們將阿倫的祝福.
祝福大家.
祝福所有弟兄姊妹.
願耶和華賜福給你.
保護你.
願耶和華使祂的念.
光照你.
賜恩給你.
願耶和華向你們養念.
賜你們平安.
阿們.
\newpage



\section{}
\label{sec:JwFqk5bcKhM}
\textbf{做好呢份工? - 20200407}
\newline
\newline
連結: \href{https://youtube.com/watch?v=JwFqk5bcKhM}{\texttt{https://youtube.com/watch?v=JwFqk5bcKhM}} ~~~~ 語音日期: 2020-05-08
\newline
\newline
\hyperref[sec:UCyv23j5rVY]{\small{< < < PREV SERMON < < <}}
~
\hyperref[sec:index]{\small{[返主目錄]}}
~
\hyperref[sec:9pqygotXAuo]{\small{> > > NEXT SERMON > > >}}
\newline
\newline
$^{1}$有80\%的人覺得自己是大才.
但在現在這個公司是小庸.
有些人認為工作是捱累仔.
有些人覺得工作沒有挑戰.
每天回去工作是很悶的.
有些人覺得工作壓力很大.
過去兩個星期我就覺得工作壓力很大.
因為要適應很多技術的東西.
所以你發覺做工越久的時候.
我們偶然間看到.
咦?有一個新人出現的時候.
他工作很拼搏.
我們心裡自然覺得不舒服.
我們經常會這樣說.
工作而已,打工而已,不用這麼拼搏.
所以你會發覺當你懂得說這句話的時候.
就顯示你在職場打滾的日子長了.
但你會發覺我們面對的困難.
就是越是在職場久的時候.
我們越是容易失去了對工作的熱誠.
甚至我們到了一個地步就不明白.
為什麼要做好工作呢?.
對基督徒來說.
原來做好工作也是一個很適合的反思.
我認識和理解的時候.
除了少數的例外以外.
大多數的基督徒甚至在香港有好幾個職場的調查.
發現原來基督徒在工作間的工作態度和表現.
被人評分的時候.
發現原來基督徒做事一般般.
不是我們想像中那麼好.
為什麼有時候基督徒上班的時候.
我們會發覺為什麼我們的表現.
理論上應該因為信了耶穌我們會表現得好.
為什麼我們表現得不好呢?.
以下是一些我們去統計或調查的時候.
發現有的原因.
這也是同一個基督徒的屬靈觀和屬世觀有影響.
不過今天沒有時間去講.
不過我們儘管去看.

$^{41}$第一,我們一想到工作的時候.
工作是屬世的.
所以我們每天做的事情是沒有永恆價值和意義的.
第二,工作是人犯罪之後的necessary evil.
不做工是沒有飯吃的.
這是聖經在《創世記》第三章這樣說.
然後在《帖撒羅尼加前書》的第一章第二章都是這樣說.
所以對我們來說.
不做工是沒有飯吃的.
所以中國人就用「搵食」來代替了「做工」這個字.
另外一件事就是.
如果你在做工上很努力追求表現好.
人們就會說你有野心.
你貪愛世界.
你只顧著取權.
另外,我們在教會的教導裡面.
教會的侍奉是屬靈的.
應該給我們屬世的工作.
所以你會發覺很多時候.
基督徒有時候沒有辦法佔用工作的時間和資源.
所以有時候我們可能會在工作的時候.
發送一些Sunday School的資料出來.
或者將一些團體開會的議程整理.
你會發覺當你很多時間精力.
全力圍著教會的工作來做的時候.
第一,你不能專注在你的工作上.
第二,週六,週日的時候你沒有好的休息.
我發覺星期天對基督徒來說是一個很忙碌的一天.
一路由早到晚,可能三,四點.
然後我們才回家的時候.
其實星期一又要上班的時候.
你會發覺我們沒有精力去做好這件事.
另外,我們有時候傳統的教導就是.
職場是信徒宣教的地方.
我們不是宣教士,但我們上班就是一個工場.
去到職場的時候,我們最重要就是將我們的同事帶去信耶穌.
另外一個就是跟我們自己有關的.
我們在教會裡面去講侍奉的時候.
我們經常說過程很重要.
我們通過過程來認識上帝.

$^{81}$我們通過過程來認識自己.
不過你老闆不是這樣看的.
你老闆看的不是過程.
你說我已經很努力做了.
但你老闆問結果呢?.
結果呢?效率呢?.
人家兩天做完的事,為什麼你要拖四天才做完?.
所以你會發覺在教會裡面.
我們就會說弟兄姊妹彼此包容.
大家都是volunteer,不要要求這麼高.
但是上班是不行的.
所以你會發覺這些種種的原因.
從一些survey翻譯的時候.
原來基督徒在外面做工的時候.
原來我們的工作表現不是那麼理想.
所以無論是基督徒還是非基督徒.
去重新思考做好這份工作.
其實我覺得都是恰當的.
我們來看的時候.
在我的簡單的outline裡面.
我提出了三個觀念.
三個觀念不是三個裡面只選擇一種.
我覺得是一個循序漸進的.
做好這份工作的意思是什麼呢?.
第一,是否做到稱職.
第二,做好這份工作的意思就是.
你有沒有盡了自己的所能.
第三,你做好這份工作的時候.
就是你在工作上有沒有卓越的表現.
所以你會發覺不是三個觀念去選擇一種.
而是從稱職到盡己到卓越.
是一個三步曲.
然後一步一步走到最高的境界.
我們先來看看稱職是什麼.
稱職就是meeting the requirements.
就是說你去面試的時候.
你有job description.
你老闆有一個assignment給你.
什麼叫稱職呢?.
誰去決定你做事是否稱職呢?.

$^{121}$比如一些簡單的評估.
你能否達到別人的要求和目標.
或者你做這份工作的時候.
你有沒有遵守到公司的一些專業守則.
譬如做護士的.
他們不可以在工作場合傳福音.
在加拿大也是這樣.
你會說不是,我是基督徒.
我一定要講的.
你試試去教會問教會裡面做護士的弟兄姊妹.
你問問他們.
他們會告訴你.
這個是不是稱職呢?.
另外就是你能否給予服務的客戶一個合理的期望.
達成了他的合理期望.
或者你履行了工作要求的專業責任.
這些都可以告訴我們我們是不是稱職.
不過如果我們從聖經的角度去看.
什麼叫稱職的時候呢?.
你會發覺有幾段簡單的經文.
其實有很多段的,不過只能夠提幾段.
十二章,路加福音36節裡說到.
這個僕人和這個主人.
主人去了參加婚宴.
這個僕人什麼都不可以做.
就在等門.
這是當時巴勒斯坦的文化.
他要等的.
如果他不等,老闆回來後就不能進去.
等到那怎麼辦呢?.
來到他敲門.
他就立即開門.
這個就是稱職.
就是我要求你去做這件事.
你如實地做到.
路加福音17章裡說到.
當這個僕人完成了主人的吩咐的時候.
耶穌說得很好笑.
所以你們完成了主人吩咐的一切.
你唯一的反應是什麼呢?.

$^{161}$我做的是我應該做的.
換句話說,這是中國人所說的.
這就是我的本份.
是我的duty.
所以我做到我的duty的時候.
我沒有獎的.
因為這是應份的.
但我們很多時候不是.
我做了應該有bonus.
這不是稱職.
如果我們做到我們預期的.
這就是稱職.
哥林多前書四章二節說.
對一個管家的要求是什麼呢?.
Be faithful.
就是交給你的,你就做好.
所以從聖經的角度去看.
稱職就是這樣一回事.
簡單來說,如果我們擺到今天.
稱職是什麼呢?.
按照你和你老闆的合約規定.
做到恰如其分,就是剛剛好.
就是說你不會來到虧待了你的公司.
所給你所賺取的人工.
我簡單的介紹裡面說.
一個公平的交易.
他聘請你是因為你的能力.
是因為你的才幹.
是因為你某些特殊的技能.
交足功課給他.
這是合約的規定.
而且稱職是對一個基督徒的員工.
或從業員最起碼最低最基本的要求.
因為一個做公的人.
如果打工仔連稱職都做不到.
即是我們連期望都達不到的時候.
在某一個情況下來說.
其實我們是違約的.
我們是破壞了合約.
人家聘請你是因為他認為你可以做到.

$^{201}$但你做不到.
所以稱職其實是最低的要求.
不過我們今天很多時候做事的時候.
我們會發覺.
你一說你做事做得好稱職的時候.
其實我們在想另外一樣東西.
就是說你只是按照工作.
你的工作規定有什麼你就做.
你做得好稱職是什麼呢?.
廣東話最容易說.
不求有功但求無過.
無驚無險.
猶到五點.
即是說你不思進取.
所以稱職其實現在來說.
其實不是很多人來重視.
從我們的角度來看.
稱職是否足夠呢?.
你有機會就找回這段影片.
你看看劉華當年的樣子多帥.
他為香港旅遊協會做了一段短片.
他說今時今日的服務態度.
如果你只是做到媽媽夫婦.
勉強達到要求的時候.
他說不足夠.
你現在上YouTube.
我剛才也確認影片在.
我也上去看了.
不過解析度很差.
儘管可以上去看.
優質服務知識就是這段影片的宣傳.
如果你有興趣.
我可以發連結給你.
你自己上去看.
聖經對我們的要求.
從來都不是茸茸碌碌的.
Mediocrity.
而是聖經要求我們什麼呢?.
從稱職去到盡己所能.
原來稱職去到盡己是有個距離的.

$^{241}$你稱職不一定是盡己.
舉一個簡單的圖表你就會明白.
有些人稱職但不盡己.
即是他的能力大於工作要求.
所以很快就可以做完他所做的事.
然後怎樣呢?.
上去YouTube,看Facebook.
他稱職,你不死他.
他老闆要求他做的事.
他每一樣都做得起.
不過問題是他有沒有盡了自己的能力.
有些人又稱職又盡了自己的能力.
這是最好的.
不過很可惜的是.
我們發覺在職場裡有很多人.
他很努力盡了自己的能力.
但無論怎樣做都好.
他不稱職.
而最壞的情況是.
他又不稱職,但又不盡力.
所以我們今天要面對的.
在這個情況下.
對於那些能做到的.
稱職,公平交易,合乎老闆要求,合乎公司要求.
但如果我沒有盡己的時候.
沒有做到自己的能力.
那又怎樣呢?.
從聖經來看.
聖經的要求不是要求你剛剛好做到稱職為止.
聖經從來不是要求.
Just meet the minimal requirement.
如果你真的這樣做.
其實我們跟法利賽人沒什麼分別.
因為聖經裡的法利賽人被耶穌罵的原因.
就是他們meet the minimal requirement.
就是剛剛好做到那個要求.
做多了也不肯.
當然我們有時候過分苛刻對法利賽人.
其實在法利賽人當中也有很多很多的法利賽人.
他本身很努力討上帝的喜悅.

$^{281}$不過在新約的角度我們通常比較負面去看.
但在聖經裡的這個例子.
很典型的就是窮寡婦的奉獻.
一句話說完.
她把她一切養生的都放進去奉獻室.
換句話說,她丟掉那兩個小錢之後.
她走開的時候就沒有飯吃了.
在這樣的情況下,是不是盡了本份.
她不是有兩個小錢給一個.
她是兩個小錢給足兩個.
第二個就是我們去盡己所能.
向誰盡己所能呢?.
向你老闆?向我老闆?.
聖經裡最大的誡命裡這樣說.
他說盡心盡性盡義愛主你的神.
我相信沒有基督徒會質疑這個對象.
不過我們不要忘記39節說.
「其次相訪愛人如己」.
原來盡己對神是必然的.
是你和我都知道的.
但我們沒有想到原來我們盡自己的本份.
同樣放在愛人如己的範疇底下.
路德和嘉爾文在當年宗教改革.
提出一個最重要的工作功能是什麼呢?.
Love thy neighbor.
那麼上帝放你在工作崗位.
做什麼呢?.
不是Love your God.
他說God doesn't need your love through your work.
上帝不需要藉著你的工作.
來得到你對他的愛.
但你的輪舍卻要通過上帝所交付給你的工作.
去得到你對他們的愛.
所以換句話說,我們很多時候忽略了.
對神我要盡心盡義盡力.
但第39節說「其次相訪」.
在我們的工作服務角度來說.
我們盡己能力其實包括對人.
而第三就是盡己所能的目的是什麼呢?.
這就是《馬太福音》第25章的時候.

$^{321}$我們很熟悉的財幹比喻.
我們很多時候發覺.
無論港道,茶經.
我們都把焦點放在第三個僕人.
第五節沒有錯.
因為聖經耶穌說的時候.
確實把焦點放在第三個僕人.
不過,有時候從聖經學習中.
我們發覺前面兩個僕人.
其實我們也有東西學到.
舉例,我拿回這段經文.
所以我刻意切斷了第三個僕人的對話.
我只是放在前面.
第十五節請你特別注意.
「按著各人的財幹」.
換句話說,主人交託.
這些錢的份量是根據主人認識他的能力.
這個意義是什麼呢?.
這個意義就是你賺回五千是應份的.
因為這個是根據你的能力.
聽得明白嗎?弟兄姊妹.
上帝是按著財幹給你的.
所以你是一個五千的人.
你賺回五千是對的.
你賺到六千當然是好事.
如果你賺到三千的時候.
那你就是不合格的.
同樣你會看到.
領二千賺二千.
然後第三個問題就是領一千埋藏.
所以你埋藏了.
所以你會看到.
其實我們忽略了十五節.
「Each according to their gift」.
每一個人按照上帝所給他的財幹.
其實財幹是什麼我們不知道.
財幹應對的單位是多少我們不知道.
我曾經和一個香港的職場士工討論的時候.
他說耶穌這個比喻最特別的地方.
就是因為你不知道一個人的對換率是多少.

$^{361}$例如今天對美金可能是這個數量.
明天對美金可能是這個數量.
正因為你不知道的時候.
你就要拼命做到最好.
要追求最好的回報.
所以簡單回顧這裡.
按照財幹賜予僕人的比喻.
其實有些東西我們沒有從前面兩個人.
例如他們投入資源.
他們花時間花金錢花心思.
去做好主人所託.
你記住他是受託的.
他老闆交託給他.
然後他去了之後.
你有沒有留意到那兩位僱人的經文.
他們立即除職.
如果你看中文成經的學本翻譯.
如果你看英文.
immediately或者instantly.
他們的目標很清晰.
就是賺回一個開.
賺回這麼多.
五千就賺回五千.
他們要抓住機會去做.
不是說你有本錢.
那些錢就會繼續來.
許冠傑快要開YouTube演唱會.
你可以聽聽.
不是說錢會自己上來.
你要抓住機會.
做生意要抓住機會.
而且他拿著五千元.
不是一做就賺五千.
可能一出去做的時候.
他先虧一千.
然後賺五百.
然後再賺一千五.
慢慢慢慢.
所以我們做生意做工都是.
我們要付上代價的時間.

$^{401}$心機精神.
還要拼搏全力以赴.
然後我們才得到那五千的回報.
聖經可能不詳細去描述.
但更重要的做生意的事.
不會是失賺的.
換句話說.
這兩個僱人.
拿五千和拿二千的僱人.
他要take risk.
他將他手上所有的東西.
要take risk.
其實第三個僱人的另一個問題.
是他不take risk.
所以他老闆才罵他.
起碼你拿利息.
你拿不足百分之一.
你也拿幾個百分之一.
頂尖的你看到嗎.
所以盡己所能的.
包括了時間,資源,機會.
還有冒險.
你有沒有付盡全力.
所以弟兄姊妹.
如果用這個比喻給我們看.
一個盡己所能的人.
就是耶穌在評語中所說的.
你這個又忠心又良善.
原來我們常說.
We have to be faithful to God.
什麼叫做be faithful.
就是盡你所能.
做到你可以做到的最好.
這樣就是good.
good是什麼.
You have a good heart.
你的good heart是向誰.
向上帝.
你對上帝這個托管你的人.
有個good heart.

$^{441}$所以在這裡重新去定義.
什麼叫做盡己.
就是你be faithful.
就是上帝所托付你.
Whatever God entrusted you.
You just be faithful to do whatever you want to do.
And then by doing so.
你有個good heart顯現出來.
這是對上帝的良善.
兩者加起來就是盡己的聖經定義.
我們平時不會這樣看.
我問一個問題.
為什麼要盡己所能.
盡己我知道.
這個不難理解.
Why.
為什麼我們要做.
這個沒機會詳細講.
不過如果你有機會.
下個星期繼續讀.
我下一堂會講到.
工作的幾幅圖畫.
工作是什麼.
就是上帝給我們的托付.
而在《創世記》第一章開宗明義來說.
工作是上帝托付給我們去管理大地.
這個字「管理」的時候.
有很多意思,meaning.
有很多學者,有很多詮釋.
簡單來說.
如果我們說去管理的時候.
其中一個就是「照管」.
照顧管理.
所以有另外一些翻譯.
就是to look after, to take care.
是上帝所做的一切.
另外一樣東西就是說.
上帝所托付給我們的大地.
是一個potential.
是一個潛在的.

$^{481}$是一個很好的good resources.
或者我們用今天的說話來說.
是一個good assets.
但是一個good assets是怎樣.
你要develop.
你要開發.
你要完善.
你要做得好.
所以如果從這個角度來看.
為什麼我們要盡自己的本份.
因為上帝藉著工作給你去.
透過這個工作去開發,去幫助.
這樣說你就不明白.
不過我想告訴大家.
現在的under 30的很多的.
我們叫做x generation.
或者我們叫做千禧世代的人.
他們不喜歡打工.
他們很喜歡自己開公司.
而且他們很多時候開一人公司.
我認識一個年輕人.
他很厲害.
他做什麼呢.
他是做濾心的.
filter, water filter.
他公司裡面.
他不是做很貴的.
有時候我們家裡賣的濾水器.
可能要幾千元.
他要做一個濾心.
最多50元美金.
為什麼他要做的價錢要這麼低.
因為他銷售的對象是非洲.
因為他發覺全世界食水資源最缺乏的地方是非洲.
所以他就通過做一個50元美金.
其實50元美金對一個非洲人來說.
其實已經是一個很高昂的代價.
但是他通過這個50元美金的濾心做好的時候.
他就賣去非洲幫助這些沒有清潔水源的地方的人.
通過他的濾心來喝到乾淨的水.

$^{521}$鄧英姐妹,盡忌.
他有這樣的技術.
他有這樣的心思.
他有這樣的意念.
他做生意的.
他賺錢的.
你就擔心了.
我家裡的2000元.
他的50元.
所以鄧英姐妹,你不要以為他只是做,沒有錢賺.
但是他把上帝託管他.
是什麼呢?.
就是照管自然.
人沒有水,沒有乾淨的水來喝的時候.
他就要供應.
所以他看到這是他盡自己所能的方式.
另外一件事就是工作是上帝的託付.
我們在《創世記》的第二章.
在另一課,如果你有機會來的話.
下一課我會講多一點.
上帝放我們去修理看守.
修理看守這兩個希伯來原文其實是很有意思的.
他說工作當然是指耕種.
在第二章.
但其實原文裡面可以有兩個另外的意思.
一個是敬拜,一個是侍奉.
換句話說,現在裡面所講的修理這個字.
同時可以翻譯成侍奉.
也可以翻譯成敬拜.
這三個字等同的時候就很有意思了.
工作是什麼呢?.
工作是侍奉,工作也是敬拜.
你有沒有想過你可以在星期一到星期五去修行?.
你有沒有想過你可以這樣做?.
怎樣做呢?.
做好你的工作.
藉著做好你的工作去敬拜上帝,將工作委託給你.
看守,照顧,保護的意思,希伯來文的意思.
其實這個字,看守等於什麼?.
是keep the commandment的同一個字.

$^{561}$換句話說,當我們盡己所能的時候.
其實我們就是藉著工作來服侍上帝.
不單止是照顧上帝所託付給我們的世界.
那些資源我們也通過工作本身成為敬拜服侍的途徑.
第三,為什麼我們要盡己所能?.
因為上帝將工作交託給我們.
以至我們成為可以服侍別人.
所以我剛才提到路德和嘉爾文.
他們在宗教改革的時候重新提出.
Our neighbor need our work, it is not God who need our work.
所以耶穌來也是,受人服侍?.
不是,這不是耶穌要做的,是要服侍人.
Calvin Radical是一位Mennonite的實學家.
他在Waterloo教書.
他說所有的工作除了很少數的,是total isolation.
你會說我們現在每個人都在獨立,我們全都在家工作.
我在神學院教書的時候,有位學生是設計飛機.
說我從來未有一個行業是設計飛機.
我問他,你怎樣做?.
他說工作很悶,為什麼?.
因為我一天就躲在家裡做設計.
Total isolation.
他說I long to meet people, I long to talk with people.
你現在就開始熬過了.
你一開始就回家工作就很開心.
最近多做一個星期就開始不開心.
等於姐妹,他說all kind of work.
Calvin Radical也是這樣說.
To work is to relate to and serve other people.
你有想過去服務的對象包括你的同事嗎?.
很多時候當我們去工作的時候.
我們工作的時候不是各做各份.
我做的事情可能是前人去做了.
我做完的工作可能交託給另一個人去做.
所以我只不過是在工作線上的中間位一環.
前面有人他做的工作做得不好.
就決定了我的工作是否難.
If he or she did a good work, do a good work.
我就受到他的服事.
同樣到我做我的部分的時候.

$^{601}$If I am doing a good job.
我下面的人就得到我對他的服事.
你有沒有想過原來team work.
原來一起工作.
原來你的工作是建立在別人的成果.
很少人像藝術家一樣.
像一個木匠一樣.
從頭到尾把一件木雕自己一個人做完.
我們大部分現代人所做的工作.
都需要一個team work的過程.
換句話說,你被人服事.
你同時也通過你的工作去服事.
所以你做的工作的質量的好與不好.
就決定了你有沒有盡你的本份.
盡到本份,你服事到神.
服事到人,也管治到大地.
這就是我們為何要來盡己.
簡單來看這個圖表.
對地上的老闆,他出薪水給你.
你起碼要稱職,起碼要交功課.
你現在做這份工作的性質.
跟我的專長未必發揮到.
如果是這樣,你是不是也要交功課呢?.
你也不要欺詐,不要騙老闆的錢.
所以,最起碼要稱職.
但你會發覺我們現代人做工.
其實你打工的對象不只是你自己一個.
不是你老闆.
原來最終我們現在常說最新的概念.
就是一人公司.
原來你就是你自己一間公司的CEO.
你做得好其實是在建立你的聲譽.
建立你的業績.
是增加了別人對你的需求.
所以其實為自己起碼要盡自己.
我們起碼要盡自己.
你記不記得?.
為什麼他要這樣做?.
他盡他的所能.
他沒有氣場.

$^{641}$我最近有機會看他的紀錄片.
他到今天為他當時的賽事不能出賽.
覺得難過,難堪.
如果那不是出自他真心.
他不會到今時今日仍然感到難過.
等於當記者訪問他的時候.
你覺得當時有沒有盡力.
他的眼淚忍著不肯流出來.
他只點頭說我盡力了.
頂智媒,對自己有交代.
我相信劉翔沒有負別人,也沒有背負自己.
所以我們不單要盡己所能和稱職.
我們更要最後追求卓越.
現在46分了,我盡快說說.
看回這幅圖.
如果你留意,稱職是最起碼對地上的老闆.
盡己是對自己.
那卓越是為誰?.
卓越是為了上帝.
我們常說要榮耀上帝.
我們要Glorify,你憑什麼去Glorify他?.
你要獻祭.
你怎樣去獻祭?.
你怎樣去將祭物成為一個謙香的祭物?.
所以你有沒有發覺創世記最有意思的時候.
該引阿伯出來做什麼?.
該引獻上地上的土產.
阿伯獻上他所牧養的羊群的第一隻.
你先不要說他們兩個人的動機.
工作的果效就成為一個祭物.
頂智媒,你有沒有想過.
你今天每一天上班所做的事.
你的對象雖然是人.
但原來那個結果是可以呈獻給上帝.
成為一個對上帝的祭物.
所以原來從稱職到盡己到追求卓越.
是一個三部曲.
從最低限度到盡你所能.
然後就是追求最佳.
做到最好.

$^{681}$但我不知道你有沒有想過一個問題.
在這裡的時候你看看.
如果我來這裡已經盡了己的時候.
那我怎樣做這裡?.
我怎樣做到最佳?.
就是出色了自己.
頂智媒,這裡有一個很重要的因素.
如果要做到最佳,不是你做的.
我給你兩個提議.
第一是上帝的恩典讓你做到.
第二是上帝的欣賞.
令到你已經超越了自己之後.
你還可以出色到自己.
頂智媒,這樣才是真真正正我們做到卓越的動力.
不是你想做到卓越就可以卓越.
是上帝的恩典和聖靈的同在.
所以從稱職到盡自己的能力.
因為我們的才幹就是這麼多.
到最後我們能夠追求卓越,靠著聖靈的能力.
那我就問,什麼叫做卓越呢?.
什麼叫做卓越呢?.
你老闆稱讚你就是卓越.
全世界的人都說你做得好就是卓越.
超額完成,完成任務不止這樣,還要超額完成.
盡己所能,盡到一個地步就是我犧牲了個人.
以至這個project能夠成功.
我很感動,華叔司徒華說的.
如果中國將來有民主的時候,成功不必有我.
但我contribute了我的部分.
所以將來中國民主化的時候,我是在的.
I have my part.
頂智媒你看不看到?.
就是我犧牲自己.
那我就要問,什麼叫做卓越呢?.
什麼叫做excellence呢?.
如果你從商業來看.
是不是你outrun了你的competitor就叫做卓越?.
你說不是,你是satisfy你的customer.
120\%或者200\%,那是不是卓越呢?.
或者就好像我兒子做的quality control.

$^{721}$make sure你的product是meet到ISO多少.
9001,9002,9003.
不是,好像沒有9003.
只有9000,9001,9002,三樣東西.
譬如說你幫助你的股東有一個穩定的財政收益.
或者你dominate了市場的市場分配.
佔有市場率,最近zoom就很厲害了.
佔有市場率很大.
其實還有alternative的,你可以用microsoft team.
差不多的東西.
最重要是什麼?set industrial standard.
你有沒有留意到當讚美之泉剛剛出的時候.
他怎樣弄powerpoint.
他怎樣寫歌.
他怎樣lead worship.
他怎樣set industrial standard for all worshippers.
卓越.
所以全世界華語教會敬拜的團隊.
現在很多很好的團隊.
跟誰?跟著industrial standard.
米讚美之泉.
是不是這樣就叫卓越呢?.
我自己去理解,從聖經整理的時候.
和幾位神學作者的書中.
我自己整理了一個這樣的定義.
首先我們一定要明白一件事.
卓越是建基於上帝所給我們的能力財貫.
而這個能力財貫是一個given的環境.
那個環境是一個giveness.
of the giveness of the circumstances.
換句話說,那個環境是有限制的.
但你不要在那裡抱怨.
那個環境局限了我的發揮.
不是,上帝給你有財貫的時候.
同時給你一個given的circumstances,一個limitation.
換句話說,上帝要求你的就是.
在一個局限了的limited的circumstances上.
仍然是making the best.
盡用上帝所賜給你的恩賜財貫.
為了誰?為了自己?為了你個人的利益?.

$^{761}$不是,to bring about the best possible benefit.
最大的好處是為了誰?.
for the well-being of the human community.
我們很多時候在reform tradition中.
我們常說的the common good.
就是為了整體的好處,為了眾人的好處.
所以突然間我們看到,原來當我們去卓越的時候.
那個指向再不是指向自己,而是指向上帝.
和指向上帝所創造.
那你說,excellence這個字沒有出現在聖經.
是,不過excellence有它的代言人.
Joseph,你excellent嗎?.
那個才德的婦人,你excellent嗎?.
可能結了婚的弟兄,你現在擔心了.
你娶不到最excellent的那個.
但以你excellent嗎?.
保羅,你excellent嗎?.
等於我們沒有excellence這個字.
但是我們有excellence的spoken person代言人.
什麼叫做excellence呢?.
就是做到上帝的心意.
所以創世記第一章就是上帝meet自己的excellence standard.
他meet到自己的ISO9002.
為什麼?因為神說完就成就.
所有的事情按照祂所說的結果.
然後上帝說,this is good, and very good.
但最重要的時候就是上帝把印章丟上去.
他的excellence就是He bless them.
上帝賜福了.
excellence就是上帝用的賜福.
所以弟兄姊妹,如何做到excellence?.
就是去想想我們工作的內容.
如何可以去meet上帝的心意.
所以不單是去滿足了你老闆的要求.
你公司的vision或者mission statement.
最重要的是從自己的角度.
我在神學院教書的時候,第一課我問同學.
你覺得你的工作和上帝的工作有什麼關係?.
他們以前沒有這樣想.
當他們突然想到,原來我的工作和上帝的工作很有關係.

$^{801}$我就有一個做老師的.
我問他覺得他的工作和上帝的工作有什麼關係?.
他說,it's bringing out the very best of my students.
我說,對的,這個就是上帝的工作.
這個就是上帝的consummation work.
consummating work.
就是成全的工作.
神做人的時候,是做一個最完美的potentiality.
but you have to develop it.
education是什麼?.
將the best of the person develop.
所以做老師是這樣的.
做護士是什麼?.
當然是redemption.
修補.
所以如果你細心去想.
我的work和上帝的四樣工作.
神創造的工作,神維繫的工作,神救贖的工作.
和神成全的工作.
一定有關係.
所以你不要只看著你公司的mission statement.
不要看著你老闆的要求.
你要問,我做這份工作,我可以怎樣達成上帝的心意?.
另外,剛才承接那個管家的比喻結論.
既然受託,一定有那個要求.
同一時間,上帝給你各樣的gift,各樣的恩賜的目的.
不是給你放在你的組合櫃上來做欣賞.
是給你用來發展,發揮,發揮你的gift.
我發覺有時候,當弟兄姊妹盡用他的恩賜的時候.
你會看到他滿面笑容的,他開心的.
他被上帝所用.
你要工作裡面有喜悅,有快樂是怎樣?.
要被上帝用.
最後,你知不知道今天的職場是一個被敗壞,被扭曲.
我們在工作的關裡面充滿了競爭.
我們充滿了彼此的不服.
耶穌基督道成肉身,來到其中一樣東西要改變的.
就是一種工作的文化.
從競爭到互補.
從去追求自己的ambition.

$^{841}$到為那個差距來者的旨意來做.
約翰福音說這句話說了四十次.
To do the work of him who sent me.
耶穌的工作的目的是什麼?.
自己的ambition.
不是啊.
所以通過這樣的方式,他來救贖了工作.
弟兄姊妹,我們都可以用同樣的方式來救贖工作.
讓我來個很快的結論.
就是怎樣做好一份工作.
弟兄姊妹,其實做好一份工作等於做好一個人.
兩者是有一個很必然的關係.
我很少見到一個做工作做得好的人.
同一時間他不懂得來做人.
但反過來看,有很多時候在工作中做得不好的人.
其實他不懂得怎樣去掌管自己的人生.
所以弟兄姊妹,其實excellence追求卓越.
不單是工作,更包括我們的人生.
而且我們每一個人的工作都有一個calling.
我今天沒辦法說.
但我在接下來的四個禮拜的課程中.
最後一課,我才會說calling是什麼.
然後在我們所有的工作範疇中.
我們都需要盡自己所能.
藉著我們的工作.
記住,路德和嘉爾文說.
不是去祝福上帝.
是去祝福世人.
祝福這個世界.
你的目的,你的存在在這個地球.
就是為了增加社會的價值.
為了你周圍的人,為了社會.
這是最重要的.
當你願意這樣做的時候.
上帝就像對那兩個五千和二千的僕人.
來吧,這是給你加分的.
弟兄姊妹,不單是你做得好.
你老闆會給你加分,上帝也會.
但更重要的是.
當你被神所用與神同工的時候.

$^{881}$向人展示上帝的善良.
這樣就是最好的.
通過工作去見證,去傳福音的方法.
我用這個例子來做總結.
我很敬重的一位職場神學家.
Paul Stephen,我有一次跟他談.
他問我,Peter,你去找一個.
做車做得好的機械工程師.
還是找一個基督徒的機械工程師.
很簡單,你當然要做得好的機械工程師.
他說,如果你找一個基督徒的機械工程師.
他只會派單張給你.
他做你的車做不好.
弟兄姊妹,你不要笑.
我真的遭遇過這樣的事.
我的車拿回去給基督徒的機械工程師.
做了三次都做不到.
弟兄姊妹,你會不會還覺得.
他是一個可信靠的人.
弟兄姊妹,seek for excellency.
is not just for our job.
but also for ourself, but also for our God.
我盼望願上帝來祝福大家.
謝謝大家.
多謝各位收聽.
\newpage



\section{以斯帖記}
\label{sec:9pqygotXAuo}
\textbf{【從聖經書卷看生命實踐系列】 主題(一) 陰霾時代的曙光:以斯帖記給我們的安慰 (粵語講授)}
\newline
\newline
連結: \href{https://youtube.com/watch?v=9pqygotXAuo}{\texttt{https://youtube.com/watch?v=9pqygotXAuo}} ~~~~ 語音日期: 2020-05-13
\newline
\newline
\hyperref[sec:JwFqk5bcKhM]{\small{< < < PREV SERMON < < <}}
~
\hyperref[sec:index]{\small{[返主目錄]}}
~
\hyperref[sec:qcKLit3iF4o]{\small{> > > NEXT SERMON > > >}}
\newline
\newline
$^{1}$大家好,我是加拿大建造中心的總幹事,陸小明牧師.
在這裡我們歡迎世界各地的弟兄姊妹.
今晚來到這個《識經》講座.
我們真的沒有經驗,從未試過報名的人數這麼多.
超過300位,我也不知道怎麼處理.
超乎我們預期,打破了以前加拿大建造中心任何一個講座的參加人數.
我們知道疫情是很深深影響世界各地.
但是從今次講座報名的情況來看.
我們覺得神的兒女始終沒有被疫情所打倒.
反而更加愛慕神的兒女.
因為大家現在進入這個聚會.
有很多弟兄姊妹都不知道加拿大建造中心是什麼.
我們略略講一下.
加拿大建造中心是以多倫多為基地.
多倫多是一個神學機構.
今次通過Zoom的轉播.
讓全球的弟兄姊妹都能夠認識我們.
我個人很鼓勵大家.
如現在在螢光幕看到.
到我們的網站.
abscc.org.
就會看到我們最新的發展.
Facebook的Fan Page.
Alliance Bible Seminary.
Centre of Canada.
歡迎大家以後有機會進入我們的網站去看.
我要介紹一下講員.
今晚講員是何啟明牧師.
他是香港建造神學院的老師.
建造神學院是我們建造中心最堅強的後盾.
建造的老師經常與我們有合作.
雖然我們分隔幾千里.
今次很榮幸請到何牧師來幫助我們講講座.
今晚的講座.
其實是有一個題目給他的.
有一個標題.
就是從聖經書卷裡面.
從聖經書卷裡面.
看生命實踐系列的其中一個.
我們會陸續有另外兩講.

$^{41}$在七月和八月的時間會推出.
今晚是第一講.
我很高興找到何牧師來跟我們分享.
他對耳屍貼記的心得.
特別是在這個疫情底下.
讓我們真的能夠看到有曙光.
當疫情越來越嚴重的時候.
我們要看曙光才有希望.
不耽誤大家的時間.
現在請何牧師跟我們講今晚的講題.
陰霾之下的事.
何牧師帶我們做一個祈禱好嗎?.
好啊好啊好啊.
我們先做一個祈禱.
大家一起做一個祈禱.
吞佑天父我們多謝你.
我們在網絡世界裡面.
這麼多弟兄姊妹一起去.
願意聆聽你的話的時候.
求你保守我們眾人的心.
雖然我們現在分隔在世界不同的地方.
但同樣我們都願意在你的話語底下.
能夠一起受激勵.
一起受造就.
願你自己保守何牧師的分享.
讓弟兄姊妹同樣得到益處.
讓大家在這個疫情底下.
我們陰霾的時代.
我們能夠看到曙光.
願天父保守帶領今晚的聚會.
特別在網絡的傳送上.
求主你真的保守.
保守一切.
奉耶穌基督的名求.
阿們.
各位弟兄姊妹.
想不到藉著這個疫情.
有這樣的機會見到大家.
我們知道其實因為這個疫情.
令到世界各地的人.

$^{81}$生活方式都變了.
在香港這個地方.
每個人都戴口罩.
初初疫情的時候.
我想二三月.
甚至到四月.
都很多的北歐.
美國,加拿大那些人.
特別是西方人.
他們都不戴口罩.
所以人們都覺得很奇怪.
為什麼他們不戴口罩呢?.
但是現在就很不同了.
原來有一幅圖畫給我看到.
很可能是因為這樣的原因.
他說我們中國人是戴慣的.
西方人就在.
Batman那些.
全部都遮上面.
不遮下面.
跟大家說說.
今天要和大家分享的訊息.
就是從以斯帖.
給我們一些安慰的看法.
我會從四方面和大家說.
簡單來說.
在我還沒和大家說到這個情況之前.
我想告訴大家.
以斯帖是屬於舊約聖經裡的聖卷.
就是Writings.
舊約聖經就是希伯來的聖經.
是分三個Category.
摩西的律法.
先知書和斯帖.
斯帖是屬於聖卷裡.
聖卷有四個小卷.
四小卷就是亞哥,路德,耶利米亞哥,傳道書和以斯帖.
你會發覺他們每一個書卷都有節日來慶祝.
但當中只有以斯帖的慶祝日子.
和書卷有關係.

$^{121}$在以斯帖這本書裡有些特別的地方.
因為是在秘魯的地方.
巴勒斯坦以外的地方.
譬如說丹爾利,耶利米亞哥.
在那裡發生的事情.
特別的地方是這本書.
大家都很清楚.
沒有提到上帝的名字.
也沒有提到向上帝祈禱.
最多在第四章的時候.
這個沒底改.
大家可能聽過他的名字.
記得他的名字翻譯叫沒底改.
不是叫沒得改.
很多時候他都很固執.
所以有時候我們叫他沒得改.
他就和以斯帖說.
我們會被消滅.
以斯帖和他說.
我們披盲蒙灰禁食.
他只說禁食沒有說祈禱.
最接近的就是祈禱.
就是禁食.
所以整本書裡特別的地方.
就是沒有提到上帝的名字.
也沒有提到向上帝祈禱.
看上去就較為難接受.
神學家加爾文根本從來都沒有用過.
以斯帖做他的題材.
或者寫過關於以斯帖的注釋書.
當時馬丁路德甚至將他這本書.
和這個《次經》的瑪加比爾書.
覺得他不應該放在正典裡.
這也是當時改教者的誤解.
但是重點一件事.
我想和大家說的是.
其實以斯帖是一本歷史書.
是有真實歷史的記載.
不過他寫起來的時候.
是較為戲劇性來寫.

$^{161}$戲劇性的意思不是要娛樂我們.
而是他將上帝的信息.
是較為輕鬆和讓讀者.
當時不是讀者,當時是聽者.
因為當時的人不是有很多手寫的東西傳遞.
不是印刷的.
所以他們好像我們我那一代的時候.
那些你我來講故事.
讓聽的人容易聽得入耳.
所以你會發覺到.
在希伯來的歷史故事書裡面.
他們較為小刻畫.
人的思想,動機,態度和意圖.
你讀希伯來的敘事文體的時候.
我們叫故事歷史.
《殺母二記》,《上下列王記》.
你留意到作者聚焦在人物之間的行動.
和他們的對話.
很多時候敘事的人就透過他們的行動.
和兩者之間的對話.
幫助我們了解他要傳遞的信息.
很簡單,我想和大家從四方面來看《二師帖》.
我相信你對《二師帖》整卷的書.
有一個很簡略和深入的了解.
我從四方面,第一方面是從文學技巧.
看《二師帖》的佈局.
第二方面就是從言直鋪排.
看《二師帖》的情節.
第三方面就是從命運逆轉.
看《二師帖》的信息.
第四方面就是從機緣巧合.
來看上帝的照管和保守.
我主要從這四個角度.
來和大家勾畫出整個《二師帖》的內容.
精要的地方.
在說這些之前.
我用十分鐘的時間.
來和大家處理一下.
舊約聚事文體.
基本要素.

$^{201}$第一方面就是場景.
在哪個時候發生.
在哪個地方發生.
這個就是時代的場景.
和地理的場景.
比如我們讀到路德記的時候.
我們知道是事思時代.
我們基督徒也明白到事思時代.
有一句很重要的話.
那時以色列中沒有王.
各人任意而行.
如果我們現在說這個講座的時候.
在哪個時候呢?.
是2020年的5月.
在香港.
大家都知道.
是香港去年到現在發生的事情.
另一方面就是人物的刻畫.
故事當中有哪些人呢?.
還有最重要的是.
透過故事的發展.
造成一個佈局和情節.
發生什麼事呢?.
怎樣發生呢?.
最後就是觀點立場.
其實聚事的人.
他講這個故事的時候.
對不起,我說故事的字.
不是我們跟弟弟一樣說.
是神話故事.
不是.
故事其實是歷史的事實.
我用這個故事的字的時候.
在那裡.
其實就是聚事者.
透過人物的刻畫.
彼此之間的行動和對話.
表達他.
用一個我們比較常用的字眼.
就是一個神學的立場.

$^{241}$其實就是.
簡單來說.
上帝透過這個事情.
給我們一些什麼訊息呢?.
現在我們來看以斯帖的故事的場景.
我們知道這件事.
發生在馬代波斯.
你看到這幅圖畫最後.
以斯帖為波斯皇后的時候.
大概是主前479年.
你看到右邊的時代人物.
這個不要緊.
我不會詳細講的.
因為我聽說他們說.
這些powerpoint會在網上.
有一個星期的時間.
你可以仍然重溫一下.
反而我們要看的就是.
波斯帝國的皇帝.
你留意到.
我們所熟悉的.
大概有十個皇帝.
黑色字有四個.
第一個是古列.
第二個是大理烏.
Darius.
第三個是薩西.
第四個是阿達薩西.
一世.
這四個皇帝.
都是和聖經的故事.
聖經的歷史有關係.
我們留意到薩西一世.
其實他希伯來文的發音.
就是阿喀隋魯.
這個我們比較明白.
阿喀隋魯就是以斯帖記的.
一個其中的配角.
他父親大理烏一世.
是把馬代波斯的國勢.

$^{281}$推到最鼎盛的時候.
不過他壓低了南邊的埃及.
和東北邊的巴比倫.
不過最西邊的希臘.
當時興起上來.
我講一些背景的時候.
一會兒我再講.
你就會明白一點.
這個波斯帝國的皇帝.
就是阿喀隋魯王.
我們簡單給大家看看.
480年的時候.
以斯帖進入皇宮.
我弄了紅色字.
然後483年的時候.
阿喀隋魯王在位第三年.
在書山城的皇宮.
召聚貴奏首領開會.
這個第三年我們留意到.
然後474年的時候.
要殺猶太人.
就是12年的時候.
阿喀隋魯王在位12年.
很快,我們再看看一點.
阿實提被貶.
這些年份大概差不多.
當然有一年或半年之間.
不同的地方.
看完這個年代表的時候.
給你看看圖畫.
最主要你發覺到這個故事.
發生在書山城.
書山城就是波斯行政的.
宮殿.
皇帝在冬天被鼠的時候.
周圍有幾個地方.
在這裡我不會詳細跟大家說.
但是你發覺到.
這個故事的裡面.
這件歷史的事實的裡面.

$^{321}$是集中於一個城市.
就是書山城.
作者不理會其他商業.
政治如何厲害的城市.
但是這個城市.
這件事發生在書山城.
所以你看到書山城這個字.
是在很多章節裡面出現.
整個故事就在那裡.
好像延禧攻略.
就在皇宮裡面.
更加重要的一件事.
你發覺到.
這個故事的重點.
就是放在宮殿裡面.
而且你會發覺到.
敘事的人.
是非常熟悉宮廷裡面.
那些佈置也好.
那些擺設也好.
在東邊,西邊.
什麼都好.
這個人一定是裡面的人.
我們都不知道是誰.
你看到書山城.
還有書山城裡面的皇宮.
我再說一點.
以斯帖有什麼特點呢?.
就是他刻畫人物的時候.
和其他薩姆爾記,列王記,尼希米.
這些歷史書裡面.
有兩樣不同的地方.
他有兩個特徵.
第一個特徵就是.
他列明大部分角色的名字.
你看看.
七個太監.
七個大臣.
管轄女子的太監.
掌管機稟的太監.

$^{361}$還有是皇所派侍候他的太監.
哈他甲.
還有就是侍候皇的一個太監.
哈波尼.
你看看他這些路人甲的名字.
全部都在裡面.
這個就是以斯帖人物刻畫裡面的一個特徵.
大部分角色不是很重要.
但是他們的名字都記在那裡.
另外一個特色.
你可以將那些人物列成一對一對的研究.
將阿實提和阿克徐魯.
兩夫婦.
他們之間的比較.
西利斯.
這個翻譯是很好的.
哈曼的太太叫西利斯.
我們廣東話就是說她很厲害.
她的出場很少.
但是她可以和哈曼的一個比較.
然後就是這個末底改和以斯帖.
末底改就是以斯帖的叔叔.
以斯帖可以說是一個孤兒.
好了.
我給大家用了十分至十五分鐘.
給大家一個圖畫.
闊一點的圖畫來看.
然後我和大家來看看裡面的東西.
在看之前.
又要給大家看就是.
以斯帖有十張.
給大家一個重溫.
第一張就是廢後.
第二張就是以斯帖封後.
第三張就是哈曼設謀.
末底改陷於兩難之間.
第四張就是到以斯帖陷於兩難.
我們很記得當時末底改挑戰他.
或者提醒他.
我們很熟悉的一句說話.

$^{401}$見知你得了皇后的位分.
不是為現今的機會.
你要見阿克徐魯王.
沒有這個危險.
還是背負著猶太人要被滅的責任呢?.
陷於兩難.
第五張.
以斯帖為哈曼和皇帝設一個筵席.
第六張.
末底改得到報償.
第七張就是哈曼獲死罪.
第八張就是皇帝的命令逆轉.
第九張就是普爾日.
第十張就是末底改得榮耀.
現在我們從書的文學技巧.
來看故事的發展.
或者佈局.
從希伯來人寫文學的技巧.
你會看到焦點的訊息在哪裡.
我們先看.
首先很簡單.
第一第二張是引言.
第二張是情.
第三至第九張.
就是情節的核心.
然後第九至第十張.
就是結束.
我再給大家詳細看.
一至二張.
發生在富麗堂皇的波斯宮.
第三至第七張.
就是末底改和哈曼之間的個人恩怨.
你看到個人恩怨裡面.
第D那裡有一個很重要.
就是第六張一至十四節裡面.
就是皇帝睡不著覺.
好像喝了豬肉湯一樣.
眼睛乾了.
然後就是第八至第十張.
哈曼被處死.

$^{441}$這就是三大段.
你可以看到的.
然後你再看看.
這個文學技巧就在這裡.
你們讀神學或是舊約學者.
來講的時候.
他們懂得希伯來文.
當然他們就有優勢.
不懂得希伯來文的時候.
就多看一點儒家書.
在希伯來文文學的技巧裡面.
有一種叫交錯排列.
有不同的說法.
有些叫線形.
這種就像一個線.
一二三四.
然後有四拍三拍二拍一拍.
就是一個這樣的線形.
或者英文就叫CHIASM.
這個字大家都會聽過.
你看到這個一二三四.
文學的技巧裡面.
重點就在這裡.
那一夜皇帝不能入睡.
我們再看看更詳細一點.
這個做法.
就是這裡.
第五至第六章裡面.
你會發覺皇帝那一晚睡不著.
皇帝睡不著.
發生什麼事呢?.
他就叫他的隨從.
拿歷史書來看.
我們經常這樣說.
你睡不著.
你做什麼?.
有些人去看書.
我想是很少數.
大部分都是去看電視.
或者看你的手機.

$^{481}$當時那個年代皇帝睡不著.
他可以有很多事情做.
叫宮女來跳舞給他看.
唱歌給他聽.
叫太監幫他下棋.
如果有下棋.
或者說笑話.
還有他也可以去後花園走走.
看看天空的月亮.
唱一首月亮知我心.
當然他們不會唱這首歌.
但是.
不可能叫太監讀歷史書給他聽.
你知道讀歷史很悶.
你以為真是會講故事的人嗎?.
你看到那個敘事的人物.
他叫我們將注意力放在第五章至第六章.
就是哈曼的敗落和末底改革的無聲.
而在哈曼的敗落和末底改革的無聲中間.
就是皇帝睡不著的時候.
我們再看看落密的地方.
剛才我說我只用了15分鐘.
幫助大家有一個大圖畫給大家有背景.
如果我是教書的時候.
或者開一些六七堂的學習.
我就會很詳細講背景.
但是今天一個小時我只能用10至15分鐘講背景.
你看到我篇幅放在哪裡多些的時候.
你就知道我重點在哪裡.
我們看看故事當中的敘事篇幅.
從阿克徐老王登基.
到阿克徐老王七年十月.
即是七年的時間.
共有七年的時間.
但是敘事的人只用了兩張經文來敘述.
七年這麼長.
用兩張來敘述.
然後我們看到以斯帖入宮.
就是阿克徐老王七年十月.
至阿克徐老王傳旨就是十二年.

$^{521}$這裡有四年的時間.
吩咐各省人民在十二月十三日將猶太人剪除滅絕並奪取他們的財物.
這裡共有四年的時間.
但是他用了多少張經文?.
用了一張經文.
你看到兩張一張這麼長的時間.
他只是用的篇幅這麼少.
而你發覺到故事的核心在哪裡呢?.
就是阿克徐老王十二年正月至十二月一年的時間.
他用了多少張?.
用了六張經文.
所以你看到他的重點放在哪裡.
而在這六張經文裡面.
故事的核心在哪裡呢?.
就是從猶太人在第三章得知他們將會被滅.
獲得魚子和仇敵.
他們可以在仇敵身上報仇.
第三章叫第八章.
那裡他們充滿喜樂,歡喜,快樂的時候.
只是兩個月的時間.
但是這兩個月的時間發生的事情.
罪事者就用了五張經文來敘述.
然後我們看到十二月十三,十四兩日.
發生的事情.
即猶太人擊殺要害他命的人.
短短兩日的時間.
他又用了一整張經文來敘述.
所以你看到罪事的人.
他希望我們將焦點放在哪裡呢?.
就是在這一年發生的事情裡.
他用了六張的時間.
而這六張的核心就是在這兩日發生的事情.
所以我們從文學的結構裡.
我們就看到罪事的人要我們的焦點放在哪裡.
這就是從文學的技巧看以斯帖的佈局.
然後我們要從延籍的鋪排.
看以斯帖的情節.
在這裡喝一點水.
我有一個同學.
也是我的同工.

$^{561}$認識了四十多年.
和他有差不多十年的時間在建道.
因為這個講座是給加拿大和美國.
有些人都認識他.
我講他的名字都不怕.
在加拿大很出名.
英國不是加拿大很出名.
應該是全北美都很出名.
應該是全世界.
溫哥華門樂重恩堂的主任牧師.
陳耀鵬牧師.
他很多年前寫了一篇講章.
他很厲害.
用四個字將整個以斯帖包含.
我都不明白他可以這樣.
他的名字叫食完就訓.
食完當然是花好圓月亮好圓.
訓當然是說阿赫徐老王睡不著.
哈曼戈很早起身.
食完就很顯明.
其實他的內容我都沒有看到.
只是看他的題目很殺食.
內容不知怎樣.
你可以問他.
你發現在以斯帖裡有一個很特別的地方.
就是他用延直和局勢的扭轉.
來串連故事.
來突出故事要我們收到的訊息.
你發現在以斯帖裡有八個延直.
第一個就是阿赫徐老王.
有兩個延直.
就是為他的權貴擺延直.
和為書山城所有男人擺延直.
你留意到這個延直有三個月的時間.
你有沒有搞錯? 吃這麼久.
其實歷史學家研究.
其實當時他和權貴計劃怎樣攻打希臘.
因為希臘這個地方慢慢的.
馬代波斯到希臘.
希臘到羅馬.

$^{601}$他父親打不贏希臘.
所以他就計劃攻打希臘.
所以他們吃東西吃了三個月.
不只是吃.
當然有歡樂和計劃.
策動怎樣去打擊希臘.
就是第一次擺了兩個延直.
這兩個延直可以說是整個故事的開始.
然後在結尾的時候.
你可以看到.
為全國猶太人.
這個普爾日的延直.
和書山城裡的人擺延直.
所以第一個延直是為全國猶太人.
第二個也是為普爾日.
就是為書山城.
所以看到一個開始一個包尾.
然後你會看到中間的核心地方.
就是以斯帖第一次為皇帝和哈曼擺的延直.
和第二次就是第五章和第七章.
你發覺到第五章和第七章.
中間發生的事情.
就是皇帝睡不著覺.
於是就叫他的太監.
來讀歷史書.
第三和第六.
就是以斯帖加冕擺的延直.
和猶太人因為沒抵改.
得到約星而設的穿插在當中.
所以你看到從延直的鋪排.
你又看到作者的重點放在哪裡.
我把鏡頭再放小.
就在兩個延直中間.
就成為整個故事的轉捩點.
所以當你從這個角度看以斯帖的時候.
初初是文學技巧的佈局.
然後你再從延直的鋪排.
你就看到他的焦點放在第六章.
一至十一節或第五章的尾.
到第六章的尾.

$^{641}$這就是延直的鋪排.
另外一件事就是從命運逆轉.
看以斯帖的訊息.
在這裡我們看到.
你們可能有看過延禧攻略.
延禧攻略有七十集.
這是我第一次重溫看.
他有一個特別的地方.
就是下半場無線劇集播放時.
一天一小時.
下半集當然是說魏英樂.
他出了很多使喚來解決一些問題.
而下半集有些問題發生.
有些問題發生時令你很緊張很懸疑.
然後你想追下去.
明天再來.
明天上半集就讓你看到他如何解決問題.
而當他解決問題時.
有時是你想不到的.
令你驚喜的.
這就是劇集要你追.
很多時候在每一集.
下半集時有些困難.
令到他可能命完一線.
但到下一次上半集.
是命運逆轉.
說到命運逆轉.
我都覺得要說說.
峰迴路轉.
局勢的扭轉.
有時真的很難說.
第二次世界大戰之後.
冷戰時期.
我們想不到蘇東坡.
蘇聯.
東歐.
和波蘭.
一夜之間變天.
蘇聯解體.
東德圍牆拆了.

$^{681}$波蘭共產變天.
有時我們看上去.
好像天空都變暗.
我們的題目是.
陰霾時代的曙光.
其實我們在一個大時代.
我們看不到.
看不透.
忽然間可以整個局勢轉變.
這個五月份.
我們知道很多事情發生.
五月尾中國有兩會開會.
但五月二十日.
台灣有蔡英文就職典禮.
我們又看不到.
蔡英文在未選舉前.
民望很低.
民調很低.
所以韓國瑜想著可以.
選到高雄市長時.
一兩個月就不做了.
去選總統.
但我們看到整個局勢的轉變.
蔡英文多了二百多萬票.
這麼厲害.
有人說這是因為去年612.
開始的反送中運動.
令台灣人看到.
香港一國兩制的情況.
整個局勢反轉了.
是很多人意想不到的事.
我們有時看這些.
我不是和大家講政治.
但很多時候政治會跟著我們進入生命.
我只是和大家講一些歷史發生的事.
有時我們以為沒有盼望的時候.
我們發現絕處逢生.
我們看以斯帖的時候.
原本猶太人被轄制.
現在反而是他們轄制那些恨他們的人.

$^{721}$原本猶太人準備被殺.
但現在反倒是殺那些準備殺他們的人.
在這裡我想和大家講.
我沒有足夠的時間.
和大家講到.
因為有些學者會看.
馬丁路德看.
這個以斯帖好像是很民族主義色彩.
說猶太人.
還有第八第九章殺很多人.
在這裡我沒辦法和大家講這個問題.
有機會的話.
我以前的教會頂子妹.
他們查經的時候會詳細了解.
原本猶太人是準備被滅絕的一群.
現在他們成為了人們懼怕的一群.
原本猶太人是大大悲哀.
但第九章裡有句說話.
轉憂為喜,轉悲為樂.
然後我們再看下去.
原本哈曼是高高在上.
高位的,但反倒被掛在木架上.
原本由末底改在低位的裡面.
他反而成為王的喜悅,尊榮的人.
原本戴在哈曼的指頭上的戒指.
反而是在末底改的指頭上.
原本亞達月是充滿了哀愁的月份.
但成為了節宴歡樂的吉日.
我們留意到.
逆轉不是我們的時間.
或是我們的想法.
可以發生的一件事.
我們在這裡想和大家講講.
在以斯帖記第二章後尾.
就是末底改在門口.
其實他也是當中的官員.
他聽到有兩個人想試探阿赫徐老.
於是他就通風報信.
但第二章結束後.
第一章第一節說了一件事.

$^{761}$就是「這事已後」.
這件事就是末底改通風報信.
阿赫徐老逃過一劫的時候.
但你發覺到.
波斯人對於幫助他的人.
皇帝對於論功行賞的人.
是立即去做的.
很少是延遲去做的.
但這件事卻是在第六章.
幾年後才得到報償.
所以從末底改的心去想的時候.
唉,這麼遲.
都不知道是否得到他的報償.
如果你代入了末底改的時候.
你會怎樣去想呢?.
在陰霾裡.
在我的民族將要被逐的時候.
究竟有沒有反轉的機會呢?.
然後你再看第三章裡很特別的地方.
哈曼決心殺死所有猶太人.
於是他就擇吉.
定了普爾.
普爾其實就是抽籤的意思.
他抽到十三日.
十三日是什麼呢?.
我們都知道和中國一樣.
波斯人都認為這是不吉祥的日子.
最好快點吧.
但是你看看.
十三日.
但是這個十三日是什麼呢?.
是在十二月的時候.
才殺了猶太人.
所以在一月到十二月.
有十一個月的時間.
我們中國人常說.
政治一天都延長.
你怎知道忽然之間有這麼大的改變呢?.
所以猶太人知道要被殺這個命令.
到這個命令執行的時候.

$^{801}$他們仍然有十一個月的走盞.
所以香港人都說.
你怎知道九月份的選舉是怎樣呢?.
這三個月裡.
隨時會峰迴路轉.
其實我們都知道.
我們怎會想到今天.
大家會用Zoom.
來聯絡我們這班人呢?.
我真的很高興.
加拿大中心找我.
找我的時候.
我竟然可以和這麼多弟兄姊妹.
在網上打招呼.
有巴拿馬,紐約,洛杉磯,溫哥華,多倫多.
還有一樣東西我們看下去.
十三號是被滅的日子.
十三之後是十四.
十四其實就是以色列人.
守禦月節的時候.
正月十四.
所以我有一篇講道.
絕望,盼望.
看你怎樣亡.
看十三被滅的那天.
還是看十四.
我們知道十四這個禦月節.
就是慶祝猶太人.
他們逃離了埃及.
被滅的災難.
所以你看到.
當我們在陰霾籠罩著的時候.
不要忘記.
上帝其實在將中掌管.
很多時候命運,局勢.
突然之間有一個扭轉.
這個學者.
他將局勢的扭轉.
一幅圖畫.
一個V形的反彈.

$^{841}$當然我們很多人都喜歡V形反彈.
特別是買股票的人.
V形反彈.
U形也好一點.
L形就不太好.
雖然兩者之間不是完全對稱.
左邊是反面.
右邊是正面.
這裡不一定完全對稱.
重點在H那裡.
就是遊行第六章.
我只是給大家看看就算了.
還有一樣東西就是最後.
從機緣巧合.
看上帝的照管和保守.
可以說得上.
你看以斯帖記的時候.
真的可以這樣說.
飛機撞紙鷂.
真的很困難這樣去發生出來.
或者我們廣東人說.
剛剛遇著剛剛.
有什麼理由剛剛遇著剛剛.
時間關係.
我很快給大家看.
以斯帖剛剛被引進選後競賽.
以斯帖剛剛被太監寵愛.
阿黑徐老王當中得到青睞.
末底改又剛剛在城門口.
聽到刺殺王的陰謀.
他的名字又剛剛記錄在.
波斯國歷史冊裡.
當哈曼針對末底改要滅猶太族的時候.
姚籤剛剛要到十一個月.
即是十一個月之後才實踐.
然後當以斯帖要見王的時候.
他又剛剛得到王的接見.
整個月了.
皇帝都沒有見他.
皇帝可能都忘記了他的名字.

$^{881}$這麼多忌憤.
他又很特別.
第一次的筵席沒有說任何話.
到第二次的筵席才說.
在第一和第二次發生了一些事情.
那個事情就是皇帝睡不著覺.
特別的地方在這裡.
哈曼那天又睡不著覺.
皇帝那天又睡不著覺.
哈曼一早起床就去到皇宮.
回到第六章的時候.
那種文學的表達.
這個阿克徐老王又在想末底改.
要獎賞他.
哈曼又在想末底改.
要懲罰他.
兩個人都在想一個男人.
這個皇帝說的時候沒有說末底改的名字.
這個哈曼又沒有說末底改的名字.
兩個人說到最後才說那個名字.
你看到那個敘述裡面的巧妙.
我想讓大家從這個角度去看這個以斯列的時候.
你看到很多那些欺巧的地方.
這些又失眠.
這個阿克徐老王又欺巧.
沒有提末底改的名字.
以致哈曼以為皇獎賞那個人是他本人.
皇帝路發充冠.
走到後院.
皇帝回來的時候.
剛剛看到哈曼好像在凌辱皇后.
給大家看到一件事.
其實這件事就令到皇帝可以脫身.
因為其實殺猶太人這件事.
皇帝是有份的.
因為他有下命令.
但是現在皇帝要告哈曼.
那個罪不是殺猶太人那個.
是侮辱皇后.
其實你想想哈曼當時.

$^{921}$他所有發生的事情裡面.
最不可能的就是侮辱皇后.
在那個時候哪會想到這些.
去性騷擾皇后.
救自己的命.
沒有辦法.
我們經常說一句話.
棄居保稅.
現在這些疫情發生之後.
被追究的.
通常都不是最大粒那個.
都是下面的追究.
交人出來.
我們明白到.
在一個極權的波斯帝國裡面.
怎會交阿俠徐老王.
當然要想辦法.
交都交哈曼.
哈曼當了宰相的時候.
就棄居保稅.
這個就是.
給我們看到.
整個以司帖記裡面.
有一個事情就是.
恰巧.
一句說話就是.
一件恰巧的事.
可能令人感受到.
詫異有趣.
然而一連串的恰巧事.
應該相當強烈的暗示.
幕後有人.
主宰整件事的發生.
我們其實看到.
以司帖.
沒有完全提到上帝的名字.
沒有完全提到他們祈禱.
但是其實.
我們看到.
人的行動.

$^{961}$背後.
有上帝在當中.
原來上帝做事的方法.
是多樣化的.
弟兄姊妹.
你和我都很喜歡.
讀出埃及記.
讀以利亞和以利沙.
甚至我們新約的信徒.
讀主耶穌基督和仕途時代.
我們發覺這三個時代裡面.
上帝是特別多神跡.
騎士.
彰顯的.
有時上帝的工作.
會直接介入.
我們發覺在舊因歷史裡面.
出埃及的時代.
列王的時代.
和耶穌基督.
和仕途的時代.
是最多神跡.
騎士彰顯.
當然從聖經神學我們知道.
這個和舊因的歷史.
很有關係.
是講述上帝直接的介入和拯救.
我們留意到.
在你和我的人生.
或者在你和我所認識的基督徒.
都間中會聽到.
神直接的醫治.
神直接的幫助他們脫離危險.
在那個時候.
好像以賽亞書第四十章.
那等候耶和華的.
必如應展翅上騰.
令你很刺激.
不過.
很多時候.

$^{1001}$上帝是在背後動工.
我們看到聖經的故事約瑟.
聖經的作者特別提到.
上帝是與他同在.
上帝是關顧他的.
我們又看到路德的生平.
路德記四章裡面.
他一回到戎摩亞地.
回到伯利恆的時候.
就是大麥的時候.
他去到波斯的屯地.
波斯又恩待他.
在經文裡面都可以看到.
上帝是透過這些人物.
來恩待約瑟和路德.
我們又想起.
以賽亞書第四十章.
我們等候耶和華的時候.
如應展翅上騰.
很明顯看到上帝的工作.
但是奔跑又不會被罰.
然後我們看到.
其實上帝不是單單顯著的工作.
不是單單在背後的工作.
其實很多時候.
他是卻隱藏的參與.
我們看到以斯帖.
雖然整卷書裡面沒有提到耶和華的名字.
但我們看到那個.
恰巧的事情.
我們看到局勢的轉變.
我們看到以斯帖的延籍.
兩次中間所發生的事.
其實都是有上帝的參與在當中.
我們看到上帝的照管.
和上帝的保護.
很多時候好像幕後背後的人物.
雖然他看不到.
但其實他在當中工作.
你發覺到我們人生就是這樣.

$^{1041}$我們不是時常如鷹展翅上騰.
我們也不是時常去跑步.
我們人生其實就是時常一步一步走.
所以以塞亞先知這樣說.
我們走路卻不困倦.
上帝一直與我們同在.
有時我們生活會遇到困難.
經濟或有急劇.
人生裡面有不如意的事.
我們也像詩篇的作者.
那些愛歌的作者.
向上帝呼喊.
究竟何時才能見到曙光.
最痛苦的一件事.
他們經常問的.
究竟要到何時呢.
病,困難一時之間.
我們很容易跨過了.
但一個月,兩個月,一年,兩年,十年.
這是很難捱過的.
但如果我們看回以斯帖的時候.
我們見到上帝是有他的時間.
上帝沒有立刻去獎賞末底改.
有他的時間.
上帝令阿赫的迷信.
抽到十三日.
是十一個月的時間才發生的.
上帝令皇帝睡不著覺.
讀以色列歷史書.
在適當的時候.
獎賞末底改.
我們看到.
雖然我們見不到上帝.
他仍然與我們同在.
所以我想用一句說話.
來做結束.
但這句說話之前.
我要講一個故事.
有一對年青人.
男孩學完駕電單車.

$^{1081}$帶著女孩到處去.
超速.
警察追他.
追他過來時.
問他知不知道他在做甚麼.
最後給他告票.
警察跟他說.
小心點年青人.
這次給你告票.
這是超速.
年青人是基督徒.
可能信了沒多久.
超速不是很危險.
我不需要怕.
因為上帝與我同在.
警察看一看他.
這樣嗎?.
給你多一張告票.
那個年青人拿著告票.
他覺得很奇怪.
為甚麼你告我超速.
現在給我一張票做甚麼.
警察說.
因為你超載.
你說上帝與你同在.
除了你女朋友.
加上上帝就三個人.
超載.
這是一個笑話.
但是我們知道.
主耶穌是以馬來尼的位.
所以有一句說話這樣說.
近來在香港很流行.
不是因為有盼望.
我們去堅持.
而是因為我們堅持.
所以我們見到盼望.
這句說話你再回去想一想.
他說不是在陰霾裡.
真的未必見到盼望.

$^{1121}$所以不是因為有盼望.
我們去堅持.
而是我們堅持.
我們會見到盼望.
求主幫助我們.
每一位信徒.
無論你活在一個甚麼光景裡.
我們靠著主的恩典.
行路.
堅持我們要行的路.
我們就不會困倦.
求主憐憫我們.
如果你想對以斯帖有更多的認識.
你可以回去我們見道的網頁.
我在2018年9月.
有30天的時間.
跟大家詳細的研讀以斯帖.
如果你在疫情裡面.
回頭去聽的時候.
我相信你會在當中.
得到上帝給過你的訊息.
我現在把時間交給陸牧師.
多謝何牧師精彩的分享.
令我們有很大的鼓勵.
另外我們還有兩場的講座.
請你瀏覽我們的網頁.
一個是7月.
剛才提過陳耀柏牧師.
另一個是郭榮鏗.
大家要忍耐看著他們.
有個解鑒.
如果今晚你覺得何牧師給我們很大的鼓勵.
心裡覺得有很大的幫助.
按著上帝對你的感動來奉獻.
有兩個方法.
一個是在你的電郵裡面.
我們已經有一個QR Code.
或者點擊連結.
可以用PayPal來做奉獻.
或者你覺得應該用支票機.

$^{1161}$也可以用支票機的方法.
如果30元以上的奉獻.
我們會在報稅年度前.
發收據給大家.
你們的奉獻對我們來說.
是一個很實在的支持.
最後想提一提.
今晚的講座是有錄影的.
在一兩天之後.
大家可以上我們的網站.
就可以看到今晚的講座內容.
所以剛才你看不到的PowerPoint.
你完全可以看得到.
在網上,在全世界不同地區的大英節目一起.
盼望以後有機會.
在我們中心的時宮裡面.
都見到大家.
其實他在香港是早上的.
他特別早起幫助我們.
何牧師你有沒有說幾句話.
然後請你祝福.
我沒有什麼要說.
其實我打算祝福完之後.
我也想給我一點時間.
跟一些人打招呼.
沒問題.
我做一個祈禱.
天父上帝,多謝你.
讓我們在這麼困難的時段.
我們卻從另一個角度.
可以得到另外的祝福.
雖然這些問題.
令到我們,我們的家庭.
我們的教會.
甚至我們的生活.
有很多不舒服.
還有很多難處.
但是從這些難處裡面.
我們懂得去看的時候.
我們不看十三.

$^{1201}$多看十四的時候.
我們見到.
我們其實可以在當中得到很多祝福.
因為在很多機構.
神學院.
他們舉辦很多免費的課程.
我們以前沒有機會聽到.
也有這樣的機會.
藉著這個科技.
來聽到這些訊息.
更加可以.
集體的彼此見面.
我們特別為了今天.
來透過以斯帖這本書.
十章這麼短的一段訊息.
幫助我們明白到.
上帝你是那位.
好像缺席.
但卻是與我們同在的上帝.
願主耶穌基督.
你的恩惠.
天父上帝的慈愛.
聖靈的感動與團契.
與我們每一位弟兄姊妹.
來到同在.
從今時.
直到主再來.
阿們.
\newpage



\section{以斯帖記}
\label{sec:qcKLit3iF4o}
\textbf{主題:陰霾時代的曙光:以斯帖記給我們的安慰 (普通話講授)}
\newline
\newline
連結: \href{https://youtube.com/watch?v=qcKLit3iF4o}{\texttt{https://youtube.com/watch?v=qcKLit3iF4o}} ~~~~ 語音日期: 2020-05-19
\newline
\newline
\hyperref[sec:9pqygotXAuo]{\small{< < < PREV SERMON < < <}}
~
\hyperref[sec:index]{\small{[返主目錄]}}
~
\hyperref[sec:yb30yQHiYdM]{\small{> > > NEXT SERMON > > >}}
\newline
\newline
$^{1}$大家好,我是加拿大建造中心的總幹事陸昭敏牧師..
歡迎世界各地的弟兄姊妹今天晚上來參加我們這個西經的講座..
上個禮拜我們已經舉行了這個講座的粵語的部分,.
這個十分成功的一個聚會..
所以今天晚上我們就用普通話國語再做一次..
我代表建造中心歡迎大家今天晚上的到來..
現在疫情深深影響我們的地球,.
但從這個講座的報名來說,.
我看見神的兒女們仍然沒有被疫情打倒,.
仍然是喜愛上帝的話..
加拿大建造中心是以加拿大多倫多為基地的一個神學教育機構,.
但通過今次使用Zoom的轉播,.
則能讓全球的弟兄姊妹能夠認識我們..
我鼓勵你瀏覽我們的網站,.
就是www.abs.org,.
Facebook就是The Alliance Bible Seminary Centre of Canada,.
就能知道我們最新的狀況..
這一次是我們第一次舉辦國語的講座,.
我們經驗不太夠,難免有一點錯漏,請大家見諒..
為了比較方便下一次的運作,.
我們在微信WeChat裡面已經設立了一個群組,.
叫"ABSCC加拿大建造中心",.
所以大家可以在你收到的電郵裡面有一個二維碼,.
通過這個二維碼就能夠直接進入到我們這個群組裡面,.
歡迎大家加入我們的群組..
今天晚上我們的講座的講員是何啟明牧師,.
何牧師是香港建造神水園的老師,.
建造神水園是我們建造中心最堅強的一個後盾,.
他們的老師經常與我們合作..
今天晚上這一次是我們一個系列的講座的第一講,.
就是從聖經書卷裡面看生命實踐的一個系列,.
很高興何牧師能夠與我們分享他對以色列記憶的一個心得,.
特別在疫情底下,我們真需要看到曙光..
我們不耽誤大家的時間了,.
現在有請何啟明牧師開始今天晚上的講座..
何牧師,你來..
好.弟兄姐妹,你們看不看見我啊?.
我看見到你..
好,如果你們看見我就好了,.
因為我現在是在香港,.

$^{41}$陸牧師在加拿大,.
可能你們當中有一些人在印尼,在馬來西亞,.
但是我現在因為機器的問題,.
我就看不見你們的,.
所以我只好一直講,.
講完以後我希望可以看看有沒有我一些認識的弟兄姐妹,.
現在我一點都看不見你們的,.
所以你講什麼我都不知道的,.
但是你們可以在那個check room,.
可以寫下你的問題怎麼樣啊..
現在在這個疫情當中,.
實在有很多很多的改變,.
今天我看一點點的新聞的時候,.
很多人都要在家當中不可以出去工作了,.
所以今天我看那個新聞的時候,.
我看見在台灣有比較多人慢慢的懷孕,.
可能就是他們的夫妻之間沒有辦法去外面的地方,.
所以就比較親密一點,.
這個是新聞裡面哪裡講的,.
因為這個疫情的緣故,.
就是看見整個世界都要改變,.
有一點覺得很稀奇的,.
就是在香港,.
因為2003年的SARS的問題,.
所以香港人都習慣戴口罩,.
但是在美國跟北美,.
歐洲就比較沒有這樣做,.
為什麼會這樣呢?.
他們有一個這樣的看法,.
原來可能西方人,.
他們那個蝙蝠啊,.
蝙蝠俠啊,.
都是用那個弄他們的眼睛的,.
但是我們中國人哪一些的刺客,.
都是弄我們的面的,.
可能就是這樣的緣故,.
這個只是給你們一點點的輕鬆的時間,.
在這裡我要講的就是一種廣東人的普通話,.
所以如果你聽不懂,.
或者是我講得不清楚,.

$^{81}$沒有關係啦,.
就是請你們忍耐一下,.
看看我的簡報就可以了,.
就是在這裡有一個,.
在疫情當中有一個人,.
一個在香港的國內人,.
國內人,.
就是在國內來的,.
被那個十字車送到醫院,.
那麼護士就見他是國內人,.
所以他是廣東人,.
他就用香港的國語,.
或者是說我們是普通話,.
就問他了,.
他怎麼問他呢?.
他說,.
你有理由死嗎?.
還是無理由死?.
理由死,.
還是無理由死?.
那麼問了幾次以後,.
那個病人最後就很難的說,.
他說,.
我沒理由死,.
那麼護士就沒有氣的拿起筆來,.
在那個板上面寫著,.
沒有理由死,.
這個就是理由死跟理由死的一個笑話,.
這個就是我們廣東人,.
講普通話的時候,.
可能就是在那個理由死跟那個理由死,.
講得不太清楚的時候,.
就撈出這個笑話來..
我們來到這個,.
就是講到那個以斯帖記,.
這個以斯帖記的時候,.
你發現這個以斯帖記,.
就是在那個西班牙來的舊約聖經裡面的第三部分,.
那我們現在用的那個和合本,.
就是66卷,.

$^{121}$舊約是39卷,.
但是那個猶太人的西班牙來的聖經,.
就是將那個聖經分成三部分,.
這三部分裡面就是,.
好像當時門徒對耶穌問他的時候,.
主耶穌就對他們說,.
我從前和你們同在的時候,.
所告訴你們的,.
就是什麼呢?.
摩西的律法,.
先知的書和詩篇..
這個就是摩西的律法,.
就是律法,law,.
先知的書就是prophets,.
那個詩篇就是writings,.
詩篇也叫聖卷,.
那個以斯帖,.
就是在那個聖卷,.
那個詩篇裡面,.
這裡有五個小卷,.
他們稱為五個小卷,.
就是雅歌,路德記,.
哀歌,傳道書,.
還有就是那個以斯帖記,.
你發現猶太人就是用不同的節日,.
來慶祝的時候,.
他們就用不同的小卷來讀出來,.
其中就是那個以斯帖記,.
就是跟他們的普爾日,.
有這個關係的..
那個普爾這個字呢,.
就是秋天的意思,.
所以他們因為絕景逢生,.
所以他們就有這個普爾日來慶祝,.
慶祝的時候,.
他們那個時候就讀,.
那個以斯帖記,.
就是這樣..
這一本書,.
比較有一些很奇怪的地方,.

$^{161}$就是我們發現整個以斯帖記裡面,.
完全是在那個陰曙之地以外發生的事情,.
好像那個丹尼尼跟那個以斯拉,.
他們都是一樣,.
就是不是在巴勒斯坦迦南地以外的,.
在那個馬代波斯統治之下的一個地方,.
那整本書裡面呢,.
就發現到沒有提到上帝的名字,.
還有呢,.
就是就算他們有一些的,.
好像那個以斯帖告訴那個莫迪改,.
叫猶太人三天三夜禁食,.
也沒有講到他向耶和華禱告的,.
在這一段經文裡面,.
最可能他們是在禱告,.
但是完全沒有說過這一些表現..
那麼,.
所以看起來的時候,.
這一本書比較難接受的,.
當時那個加爾文,.
我們的改革家,.
就是他從來沒有用那個以斯帖作為他的素材,.
也沒有寫過有關的著實書,.
你看見在新約的經文裡面,.
從來沒有引用過那個以斯帖記的,.
還有當時那個馬丁路德,.
他覺得這個以斯帖太過民族色彩,.
很重了,.
好像是講到那個,.
只是講到猶太人,.
他們的民族的性滅亡的問題,.
好像跟那個舊約跟新約中間的時候,.
兩月中間那個瑪迦比書一樣,.
他覺得呢,.
應該將它不應該放在那個聖經的重點裡面,.
其實馬丁路德他有一些誤會,.
就是對那個新約的雅各書,.
他也有這樣的一個誤會,.
所以這一本書就比較有多一點的,.
人對他有一點懷疑的地方..

$^{201}$但是當我們讀那個以斯帖這一本書的時候,.
其實這一本書是根據當時的歷史,.
但是你要了解,.
它就是引用那個希伯來人的描述的手法,.
它要寫成一本的比較是戲劇化的歷史書,.
它的意思不是要娛樂我們,.
但是要我們讀歷史的時候,.
你知道,.
如果一些人講歷史的時候,.
就覺得它有一點的乏味,.
有一點艱辛的難讀的感覺,.
比較容易打瞌睡,.
所以它寫起來的時候,.
就用一種戲劇化的歷史的方式來寫出來,.
所以作者就沒有很詳細的刻畫人物的思想,.
動機,態度,.
還有他們的意圖..
你發現就是在舊約的敘事文體裡面,.
或者說比較簡單來說,.
在故事的題材的時候,.
通常他們將他的焦點放在那個人物的行動,.
還有他們彼此之間的對話,.
讓我們透過他們的行動,.
還有他們的對話,.
了解敘事者或者作者,.
讓我們想知道他們要知道的信息..
我現在就在這裡給你們一個簡單的解釋,.
以斯帖記的一些特別的地方..
現在我進到今天我要跟你們分享的一個信息..
我要從四方面來看,.
第一方面,.
我跟你們要從以斯帖記的文學技巧,.
看整個故事的佈局,.
第一方面..
第二,這是從以斯帖記裡面談到岩石的鋪排,.
看以斯帖的情節..
第三,我要跟你們分享的就是從命運逆轉,.
看以斯帖的信息..
最後,我要從紀元巧合,.
看神的教館保守..

$^{241}$好嗎?.
我就從這四方面來與你們分享,.
整本以斯帖記給我們的信息..
好,簡單來說,.
我要給你一個框架,.
就是在敘事文體裡面,.
就談到幾方面,.
最主要的它有幾個元素,.
來組合一個故事的..
第一就是那個場景,.
它的時代的場景,.
是在哪個時候發生呢?.
譬如說我們讀到那個《路德記》,.
因為在《聖經》裡面,.
有兩卷書,.
都是用那個女人的名字來命名的,.
一個就是《路德記》,.
另外一個就是那個《以斯帖記》..
譬如說《路德記》,.
它是在那個世世鼎盡的時代,.
當我們講到那個時代的時候,.
譬如說,.
2019年12月中旬的時候,.
到2020年6月的時候,.
我們知道整個世界都被新冠肺炎來影響,.
從這個就是那個時代了..
但是如果我要講到2019年6月到現在,.
講到那個地理的時候,.
我講到香港這樣的地方的時候,.
香港人讀起來,.
或者是其他人讀起來的時候,.
就知道不單是全世界受到疫情的影響,.
香港更加的厲害,.
就是2019年的6月,.
他們有那個土反,.
修例這個運動,.
所以香港在整個情況之下,.
整年裡面都受到很多的困難,.
這個就是那個場景..
另外,.

$^{281}$就是在故事當中有什麼的人物,.
這個故事裡面,.
這個以色列當中有什麼的人物,.
然後就是講到那個布局,.
那個情節了,.
就是發生什麼事情呢?.
怎麼樣會這樣的發生呢?.
另外就是,.
就是我講到的,.
就是那個作者敘事者,.
透過人物的行動,.
還有他們北起之間的談話,.
來傳達一個信息給我們知道的..
這個就是那個簡單的,.
整個故事裡面的,.
從有四個的元素,.
場景,.
在哪個時候,.
哪個地方發生,.
有什麼人物,.
第三就是發生什麼事情,.
怎麼樣發生,.
然後就是他透過這一些的互動裡面,.
向我們講的一個的信息..
好,.
我們看那個以色列了,.
以色列自它發生在什麼時候呢?.
你看,.
這是在,.
這個就是以色列的四大時局了,.
在左邊就是公元前,.
中間就是那個事件的發生,.
右邊就是那個時代的人物,.
這個是,.
時代的人物就是我們的聖經裡面的,.
那個是聖經書啊,.
有什麼的人物啊,.
你看就是在最下面的,.
就是以色列成為波斯王後的時候,.
就是在那個以色列自第二章第十六節,.

$^{321}$大概是479年,.
公元前479年..
好,.
我們看就是在哪個時候發生呢?.
就是在那個波斯帝國,.
在那個馬代波斯,.
就是在巴比倫以後,.
就是馬代波斯,.
馬代波斯以後就是那個希臘,.
希臘以後就是羅馬,.
羅馬就是主耶穌在世上工作的時候,.
就是在羅馬帝國統治之下了,.
在那個馬代波斯,.
他們取代那個巴比倫以後,.
大概有十個馬代波斯王帝,.
其中你看見這裡面,.
有四個比較黑色的,.
紳士的,.
就是在第一個古列,.
第三個就是那個大力虎,.
你知道他們的名字呢,.
有一點的,.
翻譯上面有一點不同,.
就是有四個王帝,.
就是古列,.
第三個大力虎,.
第四個就是歇斯斯義士,.
那個歇斯斯義士呢,.
他是希臘的名字嘛,.
但是在那個希伯來話的名字就是.
阿哈水魯,.
就是我們以謝帖講到那個波斯王,.
第四個呢,.
也是跟聖經有關係的,.
就是阿哈水魯的他的孩子,.
他的兒子,.
就是雅達歇斯義士,.
大概就是給我們知道,.
以謝帖這個故事,.
這個事實,.

$^{361}$就是發生在那個阿哈水魯,.
這個王帝,.
這個馬代王帝的時代,.
就是這裡了..
我們再看下去,.
就是很快的,.
你看見這裡就是480年,.
就是以謝帖進入王宮,.
474年,.
就是那個阿哈水魯的12年,.
我們再看,.
另外一個人物就是亞什提,.
就是王后,.
在第一章裡面所談到的,.
然後就是480年,.
就是以謝帖,.
抵達蘇珊城的時候,.
這個給大家一點的觀念,.
再看看那個地圖,.
就是整個故事發生的這個地方,.
就是在巴比倫以東的一個城市,.
就是蘇珊城,.
蘇珊城就是波斯帝國的,.
他們的王帝的一個行政的地方,.
也是他們的避暑的一個地方,.
就是在哪裡呢?.
就是很熱的時候,.
他就去到這裡避暑的一個地方,.
就是這個蘇珊城,.
你發現整個以謝帖,.
這個故事裡面,.
主要就是發生在蘇珊城,.
你看見很多經文,.
都是講到蘇珊城,.
還有一個特別的地方,.
就是不但是在城市裡面,.
還是城裡面的王宮當中,.
所以就是整個故事的焦點,.
就是在蘇珊城裡面的王帝王宮當中,.
所以你看見這裡面有很多的,.

$^{401}$很多經文都是形容王宮的,.
譬如說院子,女子,橋門,.
王帝的睡房,外院,御院,.
就是這樣的一個標準,.
主要就是集中在這裡..
好,這個就是時代跟地理了..
然後,我們就看以謝帖,.
談到人物,.
人物方面就是這個故事,.
跟其他的歷史書,.
《聖經》裡面的三母兒,.
向下立黃,.
向下一些不同的地方有兩個特點,.
第一個特點就是你發現,.
故事裡面的人物,.
大部分的腳石不是那麼重要的,.
譬如說我們在整部戲曲裡面,.
有主角,有配角,.
也有一些領銜主演的,.
但是有很多我們稱為路人甲的,.
我們廣東人就說那個Kelley Faye,.
但是呢,有一些小小的人物,.
他們的名字都記下來的,.
譬如說七個太監,.
四個大臣,.
掌管女子的太監,.
還有掌管雞棚的太監,.
還有侍候他的太監,.
還有就是侍候王的太監,.
你看見所有這些,.
你讀起來的時候,.
你就忘記他們的名字,.
就算你是一個猶太人,.
你讀到這些名字的時候,.
你讀完以後,你就忘記了,.
你最多可能就記得,.
就是七個大臣裡面,.
一個領袖的米木甘,.
還有就是侍候王的一個太監,.
哈波尼,.

$^{441}$就是在第七將,.
那個王帝很憤怒的時候,.
去到那個園裡面,.
在進去的時候就看見哈曼,.
就在王后以色列的椅子旁邊,.
所以你就知道這個名字,.
還有就是哈曼的七個兒子,.
他們的名字都記在裡面,.
所以你看見這個就是以色列,.
講到的那些不重要的人物的名字都記下來..
然後,你發現你讀以色列的時候,.
你可以從一對一對的來比較,.
就是說,皇帝阿哈隋魯,.
跟亞什提王后,.
還有哈曼跟他的太太,.
士力斯,.
然後就是摩迪改,.
跟他的養女,.
就是以歇提,.
主要你看見的整個故事裡面,.
就是以歇提跟摩迪改,.
然後就是亞哈隋魯,.
跟哈曼,.
就好像一個配角一樣,.
比較出現小的就是亞什提,.
跟士力斯,.
主要的故事的人物,.
就集中在這一些的人當中..
這個就是那個地理人物,.
然後我用一點點的時間,.
很短的時間,.
給你們問一問,.
問習以歇提十章裡面,.
第一章就是那個非曲亞什提王后,.
第二章就是以歇提封后,.
第三章就是哈曼,.
切某,摩迪改,兩難,.
第四章就是以歇提,.
他很兩難,.
就是摩迪改就跟他講了,.

$^{481}$我們都很熟識這一段的經文,.
因為我們聽到以歇提的時候,.
我們就記得這一節的經句,.
焉知你得了王后的位分,.
不是為先進的機會嗎?.
就是我們都常常聽到這個說法..
第五章,.
就是那個以歇提為王謝焉,.
也為哈曼謝焉其..
第六章,.
就是摩迪改得到那個補償,.
第七章,.
就是哈曼獲始罪,.
第八章,.
就是那個王帝的命令,.
逆轉了,.
這個就是說猶太人可以抗拒,.
或者是殺害那一些想殺害他們的人..
然後第九章,.
就是他們慶祝那個普爾日..
第十章,.
就是摩迪改他得到那個榮耀..
簡單來說,.
就是有十章的經文,.
就是這樣..
然後我們再看,.
就是我們看這個故事當中的佈局,.
簡單來說,.
可以分成三個段落,.
就是一個一到二章是個引言,.
然後是第九章到第十章是個結束,.
然後中間就是那個整個情節,.
或者是那個不絕的一個核心的信息,.
這是第三章到第九章..
現在這裡我給你一個詳細一點的,.
就是第一章跟第二章,.
就是發生在富麗堂皇的波斯王宮,.
然後第三章到第七章,.
就是那個摩迪改與哈曼之間的個人因緣..
弟兄姊妹,.

$^{521}$你們看看就算了,.
因為加拿大中心會將我的整個PPT,.
變成PDF file,.
放在網上有一個星期的時間,.
你可以下載,也可以慢慢來,.
來研讀..
現在就是聽聽,.
聽聽就算了,.
不用用筆抄什麼..
第三就是哈曼被處死,.
就是八章到十章..
我就跟你講過,.
第一我要跟你們從文學字考,.
文學字考看一些些的情節..
你看,第一就是從文學技巧,.
看故事的布局..
我們讀希伯來聖經的時候,.
經常有一個字叫交錯排列,.
英文就是CHIASM,.
你好像看起來的時候,.
一二三四,.
還有四派三派,.
就是你看見這裡好像一個線形,.
我們很熱的時候用那個線,.
拖那個線,.
就是你看見,它是第一跟一派,.
一派就是一個對稱,.
我用一個言詞來表達,.
第二就是用另外一個言詞來表達,.
第一次的玉子跟第二次的玉子..
所以你看見,.
那個說作者或者是那個敘事者,.
通常我都講兩個人,.
有些時候可能是分開,.
但是很多時候,.
聖經學者都覺得,.
那個敘事者跟那個作者,.
可能是同一個,.
比較多是同一個人..
我們看見,.

$^{561}$就是你看見那個文學技巧的時候,.
那個重點就放在哪一頁,.
《王帝不能入睡》,.
就是第六章的第一節..
我不知道你昨天是不是因為這個講座,.
睡得不太好,.
可能你覺得睡得很好,.
通常我都比較,.
今天比較早一點就醒來了,.
就醒來,.
因為比較緊張一點,.
因為今天就是用那個廣東人的普通話,.
跟你們分享,.
所以我都比較有一點點的緊張,.
緊張的時候就睡不著覺了,.
有一些時候,.
但是當然睡不著覺,.
有很多的原因,.
你看見那個《王帝》,.
其實我們很多時候都有失眠的情形,.
人生當中,.
常常都有一些失眠的經驗,.
《王帝》睡不著覺也是,.
應該是很通常發生的一些事情,.
因為他要每一天日理旺季,.
很多事情要處理的,.
所以有一些時候可能很晚才睡覺,.
很早就起來,.
那麼有很重要的事情發生,.
事情要處理的時候,.
他就比較可能睡不著覺,.
但是睡不著覺,.
失眠這件事情,.
對整個以色列的故事,.
就變成一個轉捩點,.
這個是很重要的..
我給你再看另外一個,.
這個是比較詳細一點的,.
你看見那個序幕,.
跟那個美聲,.

$^{601}$那個複雜,.
那個收藏,.
就是一點的,.
你看見那個重點,.
就是要將它框括來講,.
就是在五章第九到第六章的第十四節,.
就是那個哈瑪,.
拜勒跟那個摩提改的目生,.
在這個哈瑪跟那個摩提改的目生中間,.
就是那個《王帝》睡不著覺..
然後,你要再看那個以色列的佈局裡面,.
就是那個敘事的篇幅了..
你讀的時候,.
你看見的就是那個背景,.
就是從那個《王帝》登基,.
到他第七年十月,.
這個是有七年的時間,.
你看見的有七年的時間,.
但是那個敘事節,.
只是用了第一章到第二章,.
兩章的經文,.
講到七年那麼長時間的事情..
然後,你再看,.
以色列他進入那個王宮,.
就是《王帝》第七年的十月,.
到那個《王帝》傳那個旨意,.
要殺猶太人的時候,.
是十二年正月,.
就是要將那個猶太人檢驅,.
就是十二月的十三日..
那麼十年,十二年的七年到十二年,.
大概有四年的時間,.
但是你發現,.
在四年裡面,.
整個以色列只是用了一章的經文..
所以你看,.
第一,七年裡面用兩章的經文,.
四年裡面就是用一章的經文..
然後,我們再看下去,.
就是這個故事的核心,.

$^{641}$你看見作者就是在這裡就比較用多一點的篇幅了..
你看,《王帝》十二年的正月,.
到十二年的十二月,.
亞大月就是十二月了,.
整個是十一個月的時間,.
大概是一年,.
十二個月就是一年的時間,.
但是詩詩的人就是用了六章的經文,.
六章的經文來記述它..
所以你就看見那個核心中的核心,.
就是在哪裡呢?.
就是在猶太人得知到他們將要被殺,.
然後他們到第八章,.
他們可以在仇帝的身上報仇的時候,.
只是兩個月的時間發生的事情,.
詩詩的人就用了五章的經文,.
三四五六七八,.
就是大概五章的經文來敘述了..
然後我們知道,.
就像那個十二月的十三,.
跟那個詩詩,.
兩日當中,.
只是兩日當中,.
短短的兩日,.
那個詩詩的人就用了差不多一章的經文來記述了..
所以你讀整個《以歇帖》,.
十章的經文的時候,.
你看見那個詩詩者,.
他用的篇幅,.
放在哪裡的時候,.
你就看見那個故事的重點,那個焦點,.
他希望我們集中在哪裡..
就是在這個第三個故事的核心,.
這裡了..
這個就是第一方面,.
我要跟你們從文學的技巧,.
文學的思考,.
來看整個故事的佈局,.
第一方面..
第二方面,.

$^{681}$我要跟你們從那個言習的鋪排,.
看那個《以歇帖》的情節..
有一些學者就看見了,.
簡單來說,.
整個《以歇帖》,.
我們看見可以用那個言習,.
來串聯這個故事..
大概這裡列出有八次的言習..
一開始的時候,.
就是那個王帝,.
為他的權貴跟蘇珊城裡面的男人,.
擺設那個言習..
這個言習大概有180天,.
又是半年的時間..
我們讀的時候覺得很稀奇,.
為什麼是半年的時間呢?.
其實在那個時候就是王帝的第三年,.
到第七年,.
那四年裡面發生什麼事情呢?.
就是他要攻打那個希臘,.
因為他的父王,.
佔整個波斯的帝國,.
擴展得很大,.
但是有一件遺憾的事情,.
就是打不敗那個希臘..
所以這個壓哈水老黃,.
就是要想辦法要打那個希臘..
所以這個言習有180天,.
有幾個原因呢?.
一方面就是他要告訴他的子民,.
我們這個帝國是怎麼翻身的,.
還有要獎賞一些的權貴,.
還有對這些想叛亂的人,.
有一些的威嚇的作用..
還有一個重要的,.
就是他是一個政治的言習,.
他要跟他的權貴,.
他的大臣,.
來計劃要打西方的,.
在他四邊的希臘的地,.

$^{721}$希臘興起的軍隊..
這個就是壓哈水老黃..
然後,最後就是,.
我們知道猶太人有兩個譜,.
有兩天,.
信祖譜二日,.
一個就是在蘇珊城的,.
另外一個就是蘇珊城以外的猶太人,.
就是你看見,.
第一個,.
整個是好像一對一對,.
就是中A是同一組,.
就是為那個權貴擺脅言習,.
原來A2就是為那個,.
整個全國的猶太人擺言習,.
然後我們再看,.
就是中間,.
就是那個以色列,.
他擺了兩次的言習,.
第一次跟第二次,.
都是為那個皇帝,.
跟那個他們所擺的言習,.
就是在第五章,第七章,.
你發現就是在第六章裡面,.
中間就發生一件很重要的事情,.
然後就是在第三跟第六,.
就是一個以色列加冕的言習,.
還有就是那個猶太人,.
因為莫迪改,.
得到那個月生和謝的言習,.
所以你可以用擺言習,.
來串聯整個以色列的故事,.
你就看見,.
他從那個言習的擺脅,.
就是將那個第四,第五,.
以色列的第一次跟第二次的中間,.
所以從這個編排裡面,.
你就可以發現,.
那個故事的重點,.
就是在這裡了,.

$^{761}$第一次跟第二次的言習,.
所以你可以從言習的鋪排,.
看見這個故事的焦點,.
我有一個同學,.
也是一個同工,.
可能你們當中有一些人在劍道讀書的,.
也跟他學過,.
他就簡單來說,.
就用我們廣東人的一個什麼呢?.
就是他用一個題目叫,.
吃完就睡,.
就是你們的普通話,.
就是吃完就睡覺的意思,.
他是用吃,.
因為在整個以色列的故事裡面,.
有很多言習,.
他說從這個言習裡面,.
肯定神的恩典,.
團圓,就是整個圓滿的結局,.
你看見神的保守,.
那麼他就說,.
敬神的救人,.
這個救,.
他覺得是比較,.
不是很自然,.
不是很對稱,.
就是講到他們就做什麼就做什麼,.
肯定上帝的主權,.
然後他就說,.
那個皇帝睡不著覺,.
睡不著覺就看見神在當中的一種安排,.
這個就是用廣東話,.
比較容易明白的一個意思,.
因為那個圓,.
跟那個完了的完,.
在廣東話來說,.
是同一個音的,.
我要講廣東話就是說,.
食,圓,就,飯,.
這個圓,.

$^{801}$跟那個完全的,.
跟那個圓化好的圓,.
廣東話是同一個音的,.
圓,.
所以他是還那個字,.
這裡只是給你們一點點的輕鬆的時間,.
廣東人比較容易明白的,.
他說的意思..
好,第三,.
我們就跟你們從另外一個的情況來看,.
就是那個命運逆轉,.
看那個以色列的信息,.
你看見的就是,.
本來就是猶太人被殺害的,.
噢,.
忽然之間,.
他們反倒殺害那些恨他們的人,.
本來猶太人是準備被殺的,.
第三章,.
現在他們反倒殺那些準備殺他們的人,.
本來猶太人是準備被滅絕的一群,.
但是呢,.
忽然之間,.
他們成為,.
成為波斯人懼怕的一群,.
懼怕的一群,.
本來猶太人是大大的悲哀的,.
現在反倒歡呼快樂,.
這個,.
就是,.
這個給我們看見就是180度的一個轉變,.
就是一點點的事情,.
就是王帝那一天晚上睡不著覺,.
那你知道王帝睡不著覺,.
他可以做很多事情嗎?.
他可以叫那個,.
基本宮女跳舞唱歌,.
來娛樂他,.
唱一些音樂,.
可以他比較容易熱睡,.

$^{841}$他也可以叫他的太監,.
來跟他下棋,.
或者他去後院裡面,.
看看那個月亮,.
涼涼那個風,.
還有喝一點點的酒,.
跑跑啊,.
這可以,.
但是奇怪的一件事情,.
就是他讀那個,.
比較我們不常讀的一些,.
平淡的歷史書,.
這個就是讀那個歷史書就發現,.
第三章,.
第二章的每最後一部分,.
就是那個摸底改,.
通風不順,.
讓那個皇帝逃離那個被刺的這件事情,.
特別的一個地方就是什麼呢?.
就是本來波斯皇帝,.
波斯的歷史裡面,.
就是立刻賞賜,.
對他們有恩的一些人,.
但是這件事情很奇怪,.
就是沒有立刻的賞賜,.
就在沒有賞賜的時候,.
皇帝也很奇怪的提到,.
有沒有賞賜這個人,.
當時太監就告訴他,.
沒有,.
這個就是這個事情的發生,.
就是整個故事的一個逆轉,.
你看,.
這個的疫情,.
我們都沒有想到,.
現在還是比較普遍來說,.
都是在中國開始的,.
然後就去到那個義大利,.
有很多的溫州的人,.
他們都去到那個義大利,.

$^{881}$去到美國,.
然後就是現在差不多全世界,.
都受這個疫情的影響,.
今天我們有這樣的一個的seminar,.
能夠像不同的,.
在馬來西亞有人,.
印尼有人,.
在北美也有人,.
我在香港,.
都想不到,.
我們這麼遠,.
但是也是這麼近,.
就是整個的,.
因為這個疫情的影響,.
現在在北美,.
聽說以後,.
很多時候就鼓勵,.
哪一些人戴那個口罩,.
他說戴那個口罩,.
就是尊重別人,.
也是保護自己,.
也是保護別人,.
這個就是整個的一個的逆轉在裡面,.
原本就是那個哈曼是高貴的,.
現在反而是被掛在那個木甲上面,.
本來莫迪改是低貴的,.
現在他成為皇所喜悅尊榮的人,.
原本就是那個戒指是戴在夏曼的指頭上,.
但是現在是戴在那個莫迪改的指頭上,.
聖經的學者,.
他們就發現,.
讀那個希伯來文的時候,.
就發現這個王家沙自己的戒指,.
在那個第三章跟第八章,.
就是同一個的文字來表達的,.
所以你讀起來的時候,.
就比較明白,.
那個逆轉的意思在裡面,.
本來亞大約是一個充滿哀愁的月份,.
但是整個逆轉就變成一個.

$^{921}$缺閒歡樂的紀律,.
原本夏曼就是計劃住在猶太人的身上,.
但是現在反倒,.
他的謀害猶太人,謀害那個莫迪改,.
他自己反而被掛在那個墓甲上面,.
所以有一個學姐,.
就是那個Livingstone,.
她用一個的圖畫來表達,.
雖然這個圖畫不是完全的對稱,.
但是你也可以看見,.
中間那個第六章,.
就是那個重點,.
左邊就是一個猶太人的消息,.
慢慢慢慢越來越嚴重,.
好像疫情一樣越來越嚴重,.
但是到一個轉捩點的時候,.
就是在第六章的時候,.
他好像一個V形的反彈出來,.
慢慢向來,.
然後就是從第一章到第五章,.
第六章就是那個核心,.
一個轉捩點,.
然後就慢慢慢慢去到第十章,.
這個給你們看看的,.
就是一個以色列講到的逆轉的主題,.
弟兄姐妹,.
我先從文學的字考,.
看見那個佈局,.
然後就從言詞的鋪排,.
看見那個情節,.
現在我也講到從命運,.
命運逆轉,.
看那個以色列的信息,.
最後我要跟你們思想的,.
就是從機緣巧轄,.
這個字我昨天都問過講普通話的人,.
也是從Google裡面在講的,.
應該讀那個,.
讀什麼呢?.
這個是巧轄,.

$^{961}$我讀得不準確,.
沒有關係啦,.
你看見就是有很多巧轄的事情發生,.
就是以色列很巧轄的被引進選後的競選,.
當然這裡我們知道,.
是自願還是什麼呢?.
我們讀以色列的時候,.
如果你從那個背景,.
我們今天沒有時間來很詳細的研究,.
這個以色列記的經文,.
但是在當中裡面用了很多被動的詞句,.
其實一個以色列應該是,.
B,不是自願的去損美的,.
在那個尊崇的國度裡面,.
皇帝要損基本的時候,.
哪裡有人可以抗拒的?.
當然有一些人他們想他的兒女們,.
可以步步高升,.
但是猶太人,哪有猶太人的父母,.
願意將他的女兒,.
給一個歪幫的皇帝做他的基本呢?.
這個應該不可能的一件事情,.
應該是一個被逼進入皇帝的宮裡面,.
做那個選後的工作,.
但是很巧合的,.
他就是被引進去,.
然後,因為以色列,.
他容貌美麗,.
可能也是他的溫柔性格,.
那個太監寵愛他,.
還有呢,他就受到那個,.
在這麼多的女孩當中,.
僅仍就是皇帝看中他,.
還有他的舅舅,或者是叔叔,莫迪改,.
剛剛就是在城門口,.
聽見皇帝為被刺殺的陰謀,.
然後,他的名字就記在歷史裡面,.
但是封城了,.
在這裡呢,.
然後,你看見那個哈曼,.

$^{1001}$他要抽籤滅猶太民族的時候,.
恰巧就是一月份他抽籤,.
他抽到什麼?抽到十二月的時候,.
一月份到十二月大概有什麼呢?.
有十二個月的時間,.
然後你看見這裡,.
E跟F這個湊巧的事情,.
就是我們如果你是莫迪改,.
我們做了一些事情,.
我們覺得為什麼上帝不報償我呢?.
我們覺得那個言辭的報償,.
然後心裡面就覺得應該立刻可以得到那個賞識,.
但是我們唱一首詩歌,.
英文的叫 In His Time,.
In His Times, He Makes All Things Beautiful,.
在上帝的時間裡面是最美好的,.
有一些時候我們不明白,.
為什麼上帝的公義還沒有臨到呢?.
所以有很多的年輕人他們也說到,.
言辭的公義好像是不公義一樣,.
但是我們有沒有想到,.
上帝有他的時間呢?.
那當這個抽籤的時候,.
一月份搖到十二月,.
十一個月裡面,.
其實我們中國人,.
外面也有人講到一件事情,.
就是說政治一天都延長,.
他的意思就是說,.
峰迴路轉,.
十二個月這麼長的時間,.
你怎麼可以想到有什麼事情發生呢?.
我們在香港的華人,.
我們就明白這件事情,.
不知道那個送中的條例,.
還有這個疫情,.
哇,就是弄到我們天翻地覆,.
所以猶太人他們在一月份就知道,.
他們要被滅的時候,.
但是這個命令的實行,.

$^{1041}$就是在十二月,.
有十一個月的時間,.
所以弟兄姐妹,.
我們如果聽到一些不利的消息,.
或者是一些不好的消息,.
當然我們立刻就是很懼怕,.
很震驚,.
但是我們等一等,.
可能在這個消息,.
聽到這個消息要落實的時間,.
可能是有一段很長的時間,.
這個很長的時間,.
可能就是要改變,.
這個不好的消息,.
不會實現的,.
我們有沒有這樣的一種信心,.
就是當以色列向那個皇帝求情的時候,.
剛剛你知道皇帝已經差不多一個月,.
沒有召見那個以色列,.
皇帝有很多雞盆的,.
他可能已經對以色列忘記他的名字了,.
因為每一天晚上都不同的人,.
所以皇帝可能都忘記他,.
還有他當時被召見那個皇帝的時候,.
因為你知道波斯的律例跟我們中國的律例差不多,.
就是皇帝不叫你見的時候,.
你不可以自動的去的,.
如果你自動去的時候,.
都會很容易給皇帝的私生殺掉你的,.
但是那個以色列就是冒這個危險,.
見那個皇帝,.
就在見的時候,.
恰巧他就被那個皇帝接見,.
然後很奇怪的就是,.
在他剛剛的時候,.
就是哈曼聽到摩迪嘎,.
以及他當時如期伏藤,.
既不期待的要立刻取決摩迪嘎,.
所以他就做了很高大的十假,.
然後我們看見就是在當中,.

$^{1081}$就是哈曼他發怒,.
到他在第二次的延時,.
中間就發生一件事情,.
就是皇帝且也難免,.
然後就是他難免的時候,.
就讀到那個歷史書,.
還有沒有享受這個摩迪嘎,.
然後在這裡面,.
很有一個風趣幽默的情形,.
如果你讀到第六章的時候,.
你看見那個皇帝,.
他睡不著覺,.
然後他聽到一個人物,.
就是摩迪嘎這個名字,.
他就想到怎麼樣來享受這個摩迪嘎,.
然後另外一個男人,.
就是哈曼,.
他也可能睡不著覺,.
很早就起來,.
他也想到一個人物,.
就是皇帝所想到的摩迪嘎,.
但是他要除去那個摩迪嘎,.
但是皇帝要想起那個摩迪嘎,.
當你讀到哪段經文的時候,.
你要欣賞那個敘事節的技巧,.
就是皇帝要告訴哈曼,.
他說一個人對我有解脫得好的時候,.
我怎麼報償他,.
那個哈曼他要滅那個摩迪嘎的時候,.
他兩個人一直想到同一個男人,.
但是到最後的時候,.
才提到這個男人就是摩迪嘎,.
所以你看見那個氣氛,.
將他變成一個很緊張的氣氛在當中,.
簡單來說,.
一件蹊蹺的事情,.
你覺得很奇怪,.
但是如果這麼多的蹊蹺的事情,.
我們都覺得這個應該幕後是有一個主宰,.
在當中處理這些事情,.

$^{1121}$我們從文學技巧,.
我們從那個賢職的鋪排,.
我們從那個命運逆轉,.
還有第四方面,.
從這個機緣切巧,.
看這個上帝的教官保守,.
其實我們相信上帝的弟兄姊妹,.
我們知道我們的上帝,.
一直是教官保守我們,.
上帝的工作很奇怪,.
是多元化的事的,.
上帝有些時候,.
他做事情的時候,.
很明顯意見的,.
譬如說,.
我們看見在舊約的時代裡面,.
有兩個時代,.
特別多神級騎士的,.
在波西出埃及的時候,.
我們有時代,.
也有很多的神級,.
以利亞跟以利莎的時候,.
就是王國的時代,.
也是有很多的神級騎士,.
其實以利莎比她的師父以利亞,.
行的神級騎士多到一倍的時間,.
其實神級騎士,.
我們看起來的時候,.
很明顯意見,.
令到我們很興奮的,.
其實在你,我,.
還有我們所認識的基督徒朋友當中,.
多多少少,.
在我們的基督徒人生裡面,.
都會遇到上帝有一次,兩次,.
不是很多,.
就是一些聽你禱告,.
立刻答應你,.
還有一些是超乎人想像的神級騎士,.
特別我們知道,.

$^{1161}$1976年,.
中國開放以後,.
我們就聽到很多在國內的基督徒,.
他們都是在1949年到1976年,.
封閉的時候,.
有很多神級騎士發生,.
有很多農民,.
他們都是因為神級騎士,.
相信耶穌的..
耶穌,.
我們看見就是有這些事情,.
但是上帝不是單單用那個神級騎士,.
明顯意見的事,.
這個只是,.
不是很多,.
是見眾發生的事情..
但是,.
然後我們發現,.
很多時候上帝在背後的東宮,.
看見那個約瑟跟那個路德,.
在那裡呢,.
沒有很多神級騎士,.
反而上帝是透過人與人之間的互動,.
給我們看見上帝是在背後的工作..
譬如說,.
約瑟,.
我們看見聖經講到,.
神是與他同在的..
路德這個故事,.
他從那個墨雅第跟他的婆婆,.
去到玻璃心的時候,.
在那個第一章的米部分就講到,.
那個時候剛剛就是割大麥的時候,.
然後他去到那個農田,.
他就去到那個一個農田,.
去到那個波爾斯的農田,.
得到他的恩代..
原來波爾斯就是他婆婆,.
他是公公的遠房的親戚..
所以你看見呢,.

$^{1201}$就是在約瑟,.
在路德他們的故事裡面,.
你看見上帝在背後的東宮..
但是,.
另外上帝工作的一個方式,.
就是你好像他是隱藏的,.
他的名字不在,.
他好像你禱告,.
他也好像沒有聽你的禱告..
在詩篇裡面,.
有一些詩,.
大概有三分之一的詩歌,.
就我們稱為那個愛述詩,.
就是那個個人的,.
向上帝的一個哀求..
當他哀求的時候,.
他常常就問上帝,.
為什麼我有這樣的一個痛苦呢?.
為什麼我們的國家,.
是受到那麼多的苦痛?.
為什麼我個人,.
會受到那麼多的病痛?.
他不但是問到為什麼,.
最痛苦的一件事情,.
就是不是為什麼,.
why?.
最痛苦的就是how long?.
就是我的痛苦,.
我們國家受到哪個的苦難,.
要到幾時呢?.
原來人的最大的痛苦,.
就是那個痛苦是很長很長,.
這麼長的時間裡面,.
上帝好像聽不到,.
上帝好像是隱藏的..
其實以色列,.
就是這個故事裡面,.
給我們得到的一個信息,.
上帝其實好像是隱藏,.
但是,.

$^{1241}$祂是同在的上帝..
所以很多聖經學者,.
就用一個英文就是,.
absent, present,.
好像是隱藏,.
其實是同在的上帝..
你看哦,.
我們的人生,.
在那個以色列先知,.
四十章哪裡說,.
哪一些等候耶和華的,.
他怎麼樣?.
重生得利..
他好像怎麼呢?.
好像如鷹向騰,.
哇,鷹向騰,.
飛得很高,.
很開心..
當我們遇見一些神奇奇事的時候,.
我們都很興奮..
上帝立刻聽我們禱告的時候,.
我們很興奮..
但是,.
我們人生不是常常是如鷹向騰的,.
我們常常都是跑步的,.
跑步就是不會疲倦,.
但是,.
飛是很小,.
跑步也是不多,.
我們常常都是什麼呢?.
行路的..
我們行路的時候,.
等候耶和華,.
就不至於疲倦..
弟兄姐妹,.
我們從上帝工作的方式裡面,.
我們不要將上帝放在一個框架裡面,.
他一定要行神奇奇事,.
他一定要給我們知道,.
聽到我們的禱告,.

$^{1281}$他的手透過不同的人來工作..
其實,.
上帝有他工作的方法,.
有一些是明顯易見的,.
有一些是在背後工作,.
我們感受到的,.
但是,.
也有的時間,.
可能是很長的時間,.
長長的時間,.
他是在背後好像是隱藏的,.
其實,.
他一直在參與當中,.
好像我們讀到的一節經文,.
就是耶和華好像龍摳的水一樣,.
哪一些的君王在他手當中隨意的流轉,.
其實上帝就是那一位同在的上帝..
主耶穌祂有一個名字,.
叫什麼呢?.
以瑪內利..
弟兄姊妹,.
我們有沒有這個信心,.
相信我們的神就是以瑪內利的神..
有一個年輕人,.
他剛剛拿了摩托車的牌照,.
於是他就帶他的女朋友去跑跑,.
去開,.
他開得很快,.
有個警察就找他,.
他說,.
青年人,.
你知不知道,.
你跳熟了,.
我要給你一個高票..
那個警察給了他一個高票以後,.
那個警察很好,.
他就勸他,.
他說,.
年輕人,.
你要小心,.

$^{1321}$因為你跳熟的時候,.
不單止犯了法律,.
我要給你一個高票,.
還有呢,.
你這樣很危險的,.
你跳熟的時候,.
不單對你的生命危險,.
對你的女朋友的生命危險,.
還有對你的其他人的生命有危險..
所以要小心,.
不要開,.
跳熟,.
那個年輕人,.
信耶穌的,.
但是不是那麼常常讀聖經,.
他就跟那個警察說,.
不怕,.
因為我的上帝是以馬列利的上帝,.
他與我們同在的嘛,.
所以我跳熟也沒有關係..
那個警察看見他的時候,.
聽到他這樣講,.
他就在寫一個高票,.
然後他就說,.
青年人,.
我給你多一個高票..
那個青年人就覺得很奇怪,.
當他覺得很奇怪的時候,.
那個警察就說,.
頭一張高票是告你,.
跳熟,.
第二張的高票是告了,.
跳載..
因為摩托車只可以載兩個人,.
一個就是你,.
一個就是你的女朋友,.
現在你說上帝與你同在,.
就是第三個,.
所以你是跳載..
弟兄姊妹,.

$^{1361}$這個是一個笑話,.
但是這個笑話也是給我們一個提醒,.
我們的上帝是以馬列利的上帝,.
無論你跟我現在在這個疫情當中,.
遇到什麼的事,.
不要忘記,.
他是與我們同在,.
他與我們,.
有人說,.
不是因為有希望,.
我們堅持,.
我們因為堅持,.
才有那個希望..
在這個陰霾的時代裡面,.
我們要看見上帝與我們同在,.
在陰霾時代裡面,.
才有那個曙光..
希望大家都找到上帝的應許..
我們做一個簡單的禱告,.
然後如果有問題,.
我們可以看看,.
有什麼可以來談談,.
我們一起禱告..
主耶穌,.
給我們可以在不同的地方來,.
很簡單的讀以色列記給我們的信息..
主啊,.
我們正看見你的話語的更田,.
看見你透過你的僕人,.
寫下以色列這個故事的時候,.
真的用很多的方法幫助我們,.
從一種文學的技巧,.
從那個言習的鋪排,.
還有從那個命運的逆轉,.
最後就是用很多機緣巧合的事情,.
我們看見你給我們的一個信息,.
就是你做事有很多方法,.
你很多時候你是用那個,.
好像是不在,.
但是你都是常常都與我們同在的上帝,.

$^{1401}$好像我們不看見那個空氣,.
但是我們還是知道,.
空氣是常常在我們周圍,.
也在我們心裡面,.
上帝啊,.
你就是那一位以脈賴力的上帝,.
讓我們在這個疫情當中,.
我們還是找出這個的因緒,.
這個的事實,.
你一直與我們同在,.
讓我們因為這個信息,.
我們知道有這個盼望,.
一定可以逃離,.
一定可以度過這個危險的時間,.
我們這樣的求你,.
我們要保持著,.
保守我們有這個信念,.
禱告,.
侍奉主耶穌基督的名求,.
阿們..
好,.
將時間交給盧牧師..
好,大家好,.
希望大家在今天晚上的講座,.
能夠得到很多的安慰跟鼓勵,.
請大家留意,.
剛才我們已經講過,.
我們還有兩場的細心的講座,.
所以詳細的情況,.
請大家瀏覽我們的網站,.
就可以知道什麼時候,.
我們再有另外兩場的這個講座了..
這裡稍微一提,.
我們加拿大建大中心一直以來,.
都是依靠弟兄姊妹的奉獻,.
來支持我們的運作,.
所以如果今天晚上的講座,.
對你有幫助的,.
請你按照上帝對你的感動,.
做一點的金錢的奉獻..

$^{1441}$你的奉獻可以用兩種的方法,.
一種就是在電郵裡面,.
已經有一個二維碼,.
你就可以到這個PayPal裡面,.
做網上的奉獻,.
立刻就可以拿到這個收據,.
又或者你希望能夠以支票來奉獻的話,.
可以郵寄給我們,.
如果是30加元以上的,.
我們就可以在報稅的年度之前,.
發收據給你們..
這個地址就可以在網站的下一頁,.
已經可以看得見..
你們的奉獻是對我們一個最實質的支持..
稍微一提有兩點,.
我們就結束了..
一就是今天晚上的講座,.
我們有錄影的,.
所以過一兩天以後,.
大家可以在我們的網站上面,.
可以重新再看..
如果希望能夠下載的,.
今天晚上的PPT也可以隨便..
可以讓今天晚上不能參加的低收捐贈品,.
也可以重新再看..
另外在這個我們發給大家的電郵裡面,.
有一個二維碼,.
大家就可以隨著你的歡喜,.
你可以使用二維碼,.
就可以進入到我們的群組裡面,.
在微信裡面的群組,.
那麼方便我們以後跟大家聯絡..
願上帝祝福大家,.
一直大家能夠在這個疫情當中,.
能夠抓緊上帝對我們的應許..
謝謝今天晚上你們的參與..
\newpage



\section{}
\label{sec:yb30yQHiYdM}
\textbf{疫症大流行下的神學反思 --- 第二講:《疫症大流行下的靈性修持》}
\newline
\newline
連結: \href{https://youtube.com/watch?v=yb30yQHiYdM}{\texttt{https://youtube.com/watch?v=yb30yQHiYdM}} ~~~~ 語音日期: 2020-06-19
\newline
\newline
\hyperref[sec:qcKLit3iF4o]{\small{< < < PREV SERMON < < <}}
~
\hyperref[sec:index]{\small{[返主目錄]}}
~
\hyperref[sec:fFkCm0QGBPw]{\small{> > > NEXT SERMON > > >}}
\newline
\newline
$^{1}$各位大英姐妹,你好!.
歡迎大家來到今晚第二場的講座.
我們專題的題目是.
疫症大流行下的神學反思.
多謝大家的進來.
在Facebook Live 和我們一起聽這個講座.
我自我簡單介紹自己.
我是加拿大建造中心的總幹事.
陸昭明牧師.
今次兩晚的講座.
是由建造中心發起.
得到城北華基教會和.
加布華人宣導會兩間大教會.
一起合辦的講座.
今晚第二晚的講座.
基本上特別為城北華基大英姐妹.
這個題目是他們選的.
也歡迎所有其他教會的大英姐妹參與.
這兩間教會都是多倫多最大型的教會.
不需要多介紹.
反而加拿大建造中心有部份大英姐妹.
如果昨晚沒有出現.
今晚才第一次出現.
我簡單介紹一下建造中心是甚麼一回事.
建造中心是一個在多倫多為基地的神學教育機構.
我們提供不同程度的神學教育科目.
如果大家有興趣.
歡迎大家到我們的粉絲專頁.
現在你們看到的是我們的粉絲專頁.
或者我們的網站是abscc.org.
建造中心和香港建造神學院.
有非常密切的關係.
建造神學院的老師經常都會飛來加拿大.
特別是多倫多和溫哥華的地方來講學.
在溫哥華建造神學院已經有很長的歷史.
在多倫多現在才剛剛起步.
藉著這幾次的講座.
真的讓更多的大英姐妹認識建造神學院.
也認識建造中心.
昨晚我們說了有一個電郵地址可以給大家寫問題.

$^{41}$現在再展示一下那個頁面.
請大家記住是johncheng.workshops.
有個s.
然後是2020年6月.
@gmail.com.
昨晚已經有些大英姐妹寫了一些問題給Dr. Cheng.
我們已經交了給他.
今晚他會在大概九點開始回答這些問題.
如果今晚大英姐妹有什麼問題關於祈禱.
無論是今晚的課題.
或者是昨晚的課題.
我們電郵仍然可以開放.
我收到之後也會給Dr. Cheng.
等他有時間的時候.
私下回答大家的問題.
簡單介紹一下陳慧安博士的一些經歷.
她在德國留學.
她的PhD論文是特別專注在祈禱神學.
所以我們覺得這個題目.
你還能找到誰可以說呢.
一定是找Dr. Cheng.
所以我們很早就決定請她來.
Dr. Cheng今年是安息年.
已經去到尾聲.
下星期她就飛回香港.
她今晚跟我們講的講座不是在香港.
也不是在多倫多.
她在Texas Dallas的地方.
所以我就不浪費以下的時間.
我們先請士孫朱牧師為我們做一個祈禱開始.
然後給城北華記的Pastor Sophia.
有幾句說話.
歡迎Dr. Cheng.
現在先請朱牧師.
我們一齊同心來禱告.
我們的天父上帝.
我們去到你庚前獻上感恩.
因為昨晚我們有一個好的學習.
今晚我們有第二堂的講座.
我們求真理的聖靈在當中來引領.

$^{81}$使用你的僕人陳萬安博士所預備的.
也要我們每一位同在.
在我們裡面都能夠得到幫助.
屬靈的提醒,生命的更新.
我們就這樣恭敬將餘下的時間交給你.
求你親自來賜福大靈.
我們祈禱交託仰望.
奉靠耶穌,祈禱明治.
阿們.
弟兄姊妹,特別歡迎你來到這裡.
特別這個平台是非常之好的.
又有Facebook,又有其他.
所以歡迎你繼續在這裡學習.
其實去年我去完意大利的時候.
回來就認識了陳萬安博士.
當時他真人是在這個建築學院的.
我們有很多教務和其他的同學.
在城北華基也好,在士宣也好.
其他教會也都來參加他的學學.
我覺得非常之寶貴.
所以這次他能夠真的.
雖然在Zoom裡面.
但是都能夠和我們真人一起學習.
我認識的陳萬安博士.
是一個很謙卑的學者.
能夠有一個很包容量的.
能夠接受當日同學問不同的問題.
來自不同的派別.
他都能夠應付自如的.
所以我期望今晚會很精彩.
儘管問他多些問題.
Q and A的時候.
我都相信他是一個很現代.
做很多資料搜集的學者.
我覺得他很有智慧和有啟發性.
當年我認識他的太太和女兒.
我都很感恩.
因為整個家人都有見證.
女兒都在後面.
恭喜他太太拿到這個學位.

$^{121}$又畢業了.
我都為你們整個家感恩.
我們都將以下時間交給陳博士.
各位同學晚安.
如果你在北美的時候.
如果你在香港的話就早安.
今天我們就回到昨天的課題.
雖然兩晚的課題不同.
昨天我們講祈禱.
今天就講靈性.
不過我想這兩個題目都很大關連.
對我來說都是一個很有關係的事情.
都是等我開啟我預備的東西.
今天的題目就叫做.
一個疫症大流行下的靈性修辭.
我們會講靈命或者靈性的東西.
我想這個題目都是昨天的延伸.
雖然昨天的題目是我教書教開的.
我在建築學院教禱告神學都教了六年時間.
教了超過六七次.
靈性這個題目我自己都不覺得自己是一個.
不是一個所謂靈修學者.
或者是靈性方面的人.
不過其實都很奇怪.
我自己研究的主題是禱告.
所以很難不去和這方面有關係.
我都會講.
這是我自己在這方面思考的旅程.
但今天我期望這個題目.
是比較沒有那麼教書.
沒有那麼多資料性的東西.
而是和大家講一下靈性.
在今天這個環境下.
在疫情下.
有時候都會讓我們有些反省.
所以我們都在這個環境下.
去思考靈性的主題.
今天的內容大概分幾部分.
頭部分是我自己之前的東西.
如果你有看我寫的書.

$^{161}$我第一本書都講到靈修的東西.
怎樣看我們華人教會.
怎樣看一些屬靈傳統的東西.
當中都是我早期的一些想法.
特別是今天我們講到後半部分.
都是一些我自己這一兩年新的思考.
其實我將會出版一本書.
書名如果出得成的話.
這本書名叫浪子的靈修.
希望今年十一月十月的時間出版.
這個都是我自己去思考的主題.
所以今天下半部分都是一些我比較近期.
這兩年對於這個題目的思考.
雖然是沒有講過的.
沒有在其他的channel提過這個課題.
我可以來講一下今天想做的事情.
我嘗試先講一下自己.
可能講一下故事.
來講一下我自己的靈修歷史.
我自己由信主到現在整個的靈情.
不過特別的是我們講到靈修這個字.
其實靈修這個字是我們很特別的字.
雖然靈修是我們很重要的一個部份.
但在聖經裡面你找不到靈修這個字.
特別是你search靈修這個字的時候.
其實是沒有的.
所以我們這個一方面是有一點點.
特別是修字.
靈字大家都明白.
靈性或者是我們靈魂裡面的一個意思.
但修字其實我想有一點點.
在我們中文裡面其實是有一種向度.
修字你可能想起以前的道士.
有些很長鬍鬚的人.
然後有些修行.
然後有些修煉.
所以對於我們華人教會來說.
當我們講到靈修.
其實是有一定程度我們華人教會的思維.
或者是某種的傳統.

$^{201}$是有一定程度的概念.
所以我今天就是第一部分嘗試拆解.
我們華人教會的一個靈修的概念.
我自己也是.
我自己是一個福音派信徒.
基本上我們華人教會九成九都是福音派教會.
我自己是在99年的時候信儲.
我當時大概是18歲左右.
高中生.
所以都是和大家一樣.
一個很典型的福音派信徒.
我在香港宣導會的會友.
其實是一個很傳統的.
八九十年代的華人基督徒.
對於靈修的看法.
我們以前也有方默金傳.
或者是讀經釋義.
大概初信之後.
教會就會教你.
在靈修怎麼做.
你需要靈修.
因為靈修就是親近上帝.
每天有靈修的書籍.
我們也記得.
很小的書本.
三百多四百頁.
每天看一版.
靈修的習慣.
教會就慢慢培養你靈修的習慣.
大家都很明白.
在開始之前就祈禱.
打開聖經靈修.
然後看那段靈修物相的文章.
然後再祈禱.
我自己信主之後也很乖.
也是一個很追求的基督徒.
那時候也很快看完整本聖經.
然後也很乖.
跟著做靈修.
甚至我自己也有給自己一些靈修計劃.

$^{241}$這個不是教給我的.
而是自己也設一些靈修的計劃給自己.
譬如我那時候就將屬靈九的果子.
今天就是喜樂.
今天就是關乎喜樂.
今天就要操練喜樂.
也是一個長大的成長背景.
就是我們華人教會的靈修的意思.
所以這個我想大家也有共鳴.
然後我2004年開始進入香港的建築學院.
當時我24歲.
我大學的時候是蒙誌照.
在修院裡面當然是更加注重靈修.
因為當時在長洲聖山裡面.
長洲一百多人的群體.
大家都是很追求學武的.
但是當時我在修院裡面.
因為我自己是很追求敬虔.
我也是一個很愛主的青年人.
所以當時也是很著重.
很願意思考怎樣去靈修.
當時我在修院裡面有很多不同的靈修方法.
當時我開始會氣念文.
試過用圓文 氣念文 聖經來靈修.
試過拿著馮蔭昆的羅馬書.
四大冊 逐一來靈修.
試過在早上五點鐘起床靈修.
其實也有很多不同的嘗試.
試試怎樣去做基督徒.
怎樣去建立和神的關係.
當然也試過不靈修.
可能也是大家試過的情況.
基本上很多時候.
特別是香港教會對一些職青或者成年人.
特別是生活比較繁忙的時候.
很多頂梓妹都沒有靈修.
所以這個也是我自己的反省.
後來昨天我看到一本.
卡爾巴特的一句話.
To pray and to be Christian is one and the same thing.

$^{281}$就是說祈禱和做基督徒是一件事.
我就開始嘗試去思考這件事.
究竟我怎樣去親近上帝呢.
當時我也很習慣在長洲跑步.
跑步的時候去祈禱.
跑步的時候去親近上帝.
不知道大家有沒有試過.
如果你喜歡跑步的話.
你會發覺跑步的時候是很能夠去專心.
當你一路跑步一路流汗的時候.
你會發現到自己的心跳和呼吸.
你會覺得自己是一個純粹的受眾物.
當你是一個完全是上帝創造的人.
你只剩下心跳和血液的流動和呼吸.
然後你用你的生命來祈禱.
這時候就發覺原來靈修不是一件很靜態的事.
靈修其實不單是你每天躲藏的15分鐘.
20多分鐘的時間.
而是反而是去祈禱和去當基督徒.
是同一件事.
祈禱和整個的生命是分不開的.
這也是我昨天不斷強調的課題.
所以我就開始發覺.
靈修也不是一種純粹說我們分別為性.
時間不等.
而是我們整個人的生命.
所以我就開始去思考這方面.
我自己的博士論文就是因為這句說話.
祈禱和當基督徒是同一件事.
我可以重新思考什麼叫做靈命.
什麼叫做我的屬靈生命.
所以我就懷著這個問題來到德國.
我在德國的時間是六年的時間.
德國是一個很有恩典的日子.
因為能夠全時間讀神的話語.
去思考神的話語.
是一個很祝福的事情.
可能大家作為傳道人也有同感.
你受生可以去認讀聖經.
去鑽研預備講章.

$^{321}$或者去祈禱.
這是一個很大的恩典.
當時我也是這樣.
在德國的六年裡.
基本上我的工作就是打開聖經.
打開一些神學的著作.
去思考有關上帝自己.
所以我寫過一本書.
上帝恩典.
既然我能夠全時間去思考上帝恩典.
是一個很大的祝福.
特別我是研究一個瑞士的神學家.
那時候我也有機會去瑞士.
開一些神學會議.
如果去到瑞士.
我發覺很特別.
瑞士人Calbert.
為什麼他會想到這麼多.
充滿頌讚上帝恩典的神學呢.
當你面對這麼漂亮的環境.
瑞士的山脈.
然後去思考上帝的時候.
其實是很開心的.
覺得自己能夠全時間.
每天都默想上帝.
所以當時我就開始.
重新嘗試去定義靈修.
靈修不是單單純粹是.
我每天抽15分鐘的時間.
反而是說.
我的靈修就是我的整個Christian life.
我的基督生命.
在這個世界裡.
在經歷上帝每一分每一刻每一秒.
作為一個人.
作為一個丈夫.
作為一個爸爸.
作為一個工作上的人.
其實都是在經歷上帝的恩典.
或是上帝的遭遇.

$^{361}$所以當時我就嘗試去.
坦白說.
說一下自己的靈情.
我都嘗試去打破.
某種固有的靈修概念.
我理解我的靈修是我的整個生命.
基本上我每天都是在.
花時間去思考上帝的時候.
我的作息.
我的平時生活.
就是一種和上帝的交往.
所以對我來說.
坦白說.
需要很大勇氣.
你要打破某種.
以前我們橫教會那種.
每天打開一本靈修書.
一定程度的閱讀理解.
然後去看東西.
當中的一刻.
是一個跳躍.
嘗試離開某種靈修的框架.
嘗試一種整個生命的投入.
所以嚴格來說.
或者用那種客義定義來說.
我是沒有靈修.
沒有那種靈修.
只是看別人寫的東西.
然後去這樣做.
而是嘗試在我生命裡.
去整個生活.
整個的經驗.
作為我和上帝相遇的一種過程.
當然我自己是不斷去看書.
看聖經.
去思考.
所以我覺得這是一個這樣的過程.
那時候我大概是一種這樣的想法.
基本上我自己寫書頭那兩本書.
都是一種這樣的思考.

$^{401}$如果你看過我頭兩本書的時候.
關於基督徒和無所不其事的那本書.
當中重新去思考基督徒生活.
怎樣看靈命.
怎樣看上帝的關係.
那時候我是有點反靈修的.
對於某種靈修大師.
或者是一些傳統靈修的那種向度.
我自己不是太過去跟從.
不覺得靈修是一種.
這樣做才是靈修.
可能大家聽過的一個笑話.
有一個靈修大師在教靈修.
然後有一個學生拿著一包薯條.
來到上課.
當大家在一起默想靈修的時候.
這個人就在吃薯條.
然後那個大師就問.
為什麼你靈修的時候吃薯條.
然後那個人就回答.
其實我不是靈修的時候吃薯條.
而是在吃薯條的時候靈修.
其實整個的靈修和我們的生活.
其實是可以分不開的.
這個就是我六七年前回香港的時候.
那種信念.
嘗試去整個的Christian life.
就是我的靈修.
特別是在德國的神學傳統裡面.
Spirituality這個字.
德文當然有這個字.
Spirituality text.
這個字取源於英文.
變成德文的Spirituality text.
但其實德國一向都會將這種所謂靈修的東西.
叫做Firmicate.
就是敬虔性.
所以對於德國來說.
其實他未必是一種這樣的看法.
特別是可能是一種新教的傳統.

$^{441}$當然我們有敬虔主義.
或者有其他的看法.
但我們都比較少一種現代靈修學的一種向度.
而是說我們整個人的生命.
反而是我們整個靈修的向度.
而不是去注重某種的時間.
所以這個就是我當時去思考的事情.
當時我都寫過一篇文章.
就是關於靈修.
如果大家可以看回.
就是怎樣理解我們平時.
當我們吸收了本靈修書之後.
我們進入我們的工作和生活裡面.
我們怎樣才能真正的開展靈修的生活呢.
最後講到我在建道裡面教神學.
我在建道裡面教神學.
其實都是教學裡面不斷地學習.
因為我在建道裡面教歷史.
我就有機會重新認真地去理解整個教歷史的片段.
我對於中世紀裡面的一些所謂神秘主義很有興趣.
其實中世紀裡面的神秘主義.
其實是一些很前衛的東西.
即是今天他們是用一些本土的語言來寫作.
他們是一些女性.
是一些教會裡面邊緣的人.
傳統來說教會裡面有三種類型的人.
我們叫彼得 約翰和保羅.
彼得代表著整個教會的建制.
即是教會裡面的領袖或者權力.
保羅就代表神學.
即是教會裡面對於思考和知識的進心.
而約翰就代表著整個教會裡面的神秘傳統.
他們不是在教會的建制裡面.
都不是一種很主流的理性的神學思考.
而他們是一群傾向神秘.
傾向尋求發現上帝的奧秘的一群人.
可能對於環教會來說.
神秘這個字其實是有些負面的.
大家覺得神秘主義好像不是很正經.
或者是一些邪門的.

$^{481}$但其實你會發現.
所謂今天所說的領修傳統.
今天可能大家都經歷到.
開始在這十年二十年裡邊.
開始有領修學的研究.
其實你會發覺.
當我們回看歷史裡面的一些領修學的人.
寫領修的一些經典的人.
德蘭修女,十大若望,聖本督.
或者是十一世紀裡面的Bernard Clave.
這些人其實都是神秘主義者.
所以基本上神秘主義和靈是不分不開的.
如果你的信仰裡面是沒有神秘的話.
其實這個就不是信仰.
你的上帝總是有些東西是超越你自己的知識.
超越你自己的觀感.
即是超越你的語言.
超越你的感官裡面能夠捉摸到的東西.
所以我在教九歷史的時候.
那六年裡面.
我就重新思考所謂的領修.
我覺得今天的橫教會的領修是不夠領修的.
我們有時候都會著重某種的方法.
著重某種的即食的表面.
但是我們對於那種內涵.
其實我們不是很有興趣.
當你學很多不同的領修課程.
我們都比較傾向學那種表面的圖.
即是怎麼做.
Natu Divina 土獨.
就是怎樣的方法能夠讀聖經呢.
重點其實就不是方法.
而是當中我們作為一個新教徒.
我們怎樣重新理解上帝的奧秘呢.
不過這是一個很長的研究.
我自己是重新思考.
當然我覺得整個生命是我們的Christian life.
就是我們的靈性.
那種的靈命.
那種的場所.

$^{521}$那種生活裡面的領修.
如果你看盧雲的時候.
一些很傳統的領修的classic.
其實答案是一樣的.
當一個人從一個領修的角度來說.
我掃地的其實也是一種領修.
我整個的生活就是一種領修.
這樣的看法和我所說的一樣.
結果都是一樣.
我整個的生命其實都是一種.
到最後其實是一種共同的地方.
就是和上帝的親近.
所以我就發覺其實我也是一個領修人.
所以最後我想說的是.
雖然我不是很有興趣領修學這方面的東西.
但其實林彼得所說的東西.
其實他在一個廣義裡面.
其實都是一種領修學的探討.
可能領修學就是從一個客義到廣義.
而我就是從一個廣義到客義.
就是從整個生命裡面.
去到和上帝思考的課題.
剛才說的都是模糊的.
都是我自己的過程.
我開始說一下今天的課題.
第一部分是我六年前的事情.
都是一些對於我華人教會裡面.
一些領修傳統的反省.
所以這些是我早期的一些思考.
有些地方仍然是Reddit的.
我們就想說一下我們和上帝的關係.
我們和神的關係.
其實這個就是我們領修裡面.
很著重的課題.
所以我們和神的關係.
如果我們去理解的時候.
我會覺得分兩部分.
可以說是上帝為我們所做的事.
上帝為我們所做的事.
這個是最首要的部分.

$^{561}$都是祂首先來開展的部分.
然後就是我們對於上帝的回應.
這個就是我們領修裡面.
或者我們和神的關係裡面.
最重要的兩部分.
很明顯上帝是出了一隻雞.
我們出了一隻豉油.
上帝是做了九成九的工作.
祂自己來開展.
祂用了十二種的救恩來請教我們.
來開展這種關係.
而我們又去回應祂.
這個就是最基本的一種框架.
如果用羅馬書的結構來說.
頭十一章裡面說的是一種上帝的工作.
這個是上帝的工作.
是我們自己百百得著的東西.
然後就是我們的回應.
用羅馬書的十二章來說.
就是我們對上帝的回應.
這方面是關乎倫理的部分.
我們對於上帝的回應.
我們的生活是怎樣.
我們怎樣為主而活.
這個就是我們領修裡面.
第二部分的回響.
所以兩個的關係是互動的.
第一部分就是上帝為我們所做的.
基本上就是上帝的三個工作.
上帝請教我們.
聖子和聖靈在當中來工作.
這個就不詳細說了.
這些都是神學的部分.
但這個是客觀的.
這個不關你的事.
這個是上帝自己來做事的工作.
所以這麼說.
靈命本身是一樣上帝賜予我們的東西.
如果看回羅馬書第八章的時候.
都是昨天所說的.

$^{601}$上帝就差遣了兒子的靈去到我們生命裡面.
開始成為了上帝的兒女.
這個部分是關乎於上帝自己的恩典.
是上帝的靈.
上帝差遣兒子的靈.
這個三一的工作去到生命裡面.
所以靈命這件事.
整體來說是恩典.
不是說你靈命好不好.
而是你有靈命這件事.
本身是恩典.
所以這是一個很重要的前提.
當你去擁有靈命.
當你成為基督徒.
擁有靈命的一個人.
本身這件事是一個恩典.
很多時候我們成為了一種burden.
不知道大家有沒有這個感覺.
很多時候回到教會.
可能在團體裡面.
有一個導師或傳達問你.
你最近靈命怎麼樣.
你會覺得是一種.
一種跑數的壓力.
有沒有靈修.
有沒有做好自己.
慢慢培養出一種.
我要跟隨靈修.
我要做什麼.
當然我們會做.
但我們說本身整件事.
成為了一種很跑數式的活動.
我們要去培育靈命.
這個我會講這個課題.
但如果我們說.
救恩或整個Christian life.
是開展於上帝恩典的時候.
其實上帝給我們這條命.
這條新的生命.
靈命本身就是一個白白的恩典.

$^{641}$我們是去享受這份恩典.
多於要去.
成為了你做不成就糟糕了.
這樣的看法.
這樣的壓力.
所以我會說.
花長一點時間來說.
我們如何去回應上帝.
如果我們去看.
Karl Barth 的時候.
你會發現.
首先說這件事是我們主觀的.
這關乎於我們.
如何去回應上帝.
這件事情.
他就說.
我們對於上帝的恩典不外乎.
對於上帝不外乎有三種不同的回應.
我們對於上帝有什麼回應呢.
第一就是我們的Obedience.
我們要遵從.
我們要聽上帝的話.
我們要遵行上帝的命令.
例如禮典章.
這是一個很傳統.
從教育徒生中不斷強調.
我們要Obey.
我們要去回應上帝的命令.
這是一個很重要的一個向導.
但基督徒也不是只有做的事.
我們是要相信.
信心沒有行為是死的.
百達得過做事而無信.
是一種律法主義.
所以在我們橫交會.
我們很強調這件事.
我們要有行為.
我們要有信心.
信心成為了我們.
一個很強調的一種價值.

$^{681}$不過我想說.
其實除了遵從和信心之外.
昨天如果你要上台的話.
禱告其實是一個很重要的向導.
我想說的禱告不是純粹.
你需要去回應上帝的命令.
你要祈禱這麼簡單.
而是我們說的.
遵從信心和禱告.
是三個互為的一種關係和一種回應.
信心是遵從.
遵從是信心.
祈禱也是一種信心.
祈禱也是一種聽從.
禱告是信心.
禱告是一種信心的表達.
禱告是信心的回應.
禱告也是一種對上帝命令的回應.
這也是昨天所說的.
這就是500年前路德所強調的.
禱告是對上帝命令的回應.
所以我們今天禱告.
是因為我們願意聽從上帝的命令.
去做一個信服的人.
我們就禱告這樣.
不過這三樣的觀點其實還沒有完.
我們說禱告是信心.
禱告是聽從.
不過反之亦然.
信心同樣也是禱告.
遵從同樣也是禱告.
就是說他們的關係是可以反過來說的.
當我們願意去相信上帝的時候.
其實這種的相信.
其實都是一種禱告.
這就是我昨天所說的.
潛藏在我們生命裡面的禱告.
這未必是行為上的禱告.
而是對於上帝的懇求和倚靠的概念.
所以我們整個禱告所說.

$^{721}$有人說耶穌求主幫助我.
因為我的信不足.
我們願意去相信上帝.
但是我們的信心不足.
我們只求主幫助我們.
所以我們的信心.
也不可能沒有了這份禱告.
我們願意去相信上帝.
仍然是軟弱的.
我們是不配的.
我們對於上帝的聽從也是一樣.
我們願意遵從上帝的命令.
不過我們每一個的遵從.
其實都帶著一份禱告去遵從.
我經常說這個例子.
我們每個星期聽道的時候.
當牧師講道差不多到尾聲的時候.
就會說一句經典說話.
求主幫助我們.
當我們聽到求主幫助我們的時候.
我們會覺得這篇道差不多完結了.
差不多要做總結了.
不過這句話是很有意思的.
當牧師講道求主幫助我們的時候.
就是說今天這篇道裡面.
我們所說的那種教導.
我們仍然是以一個禱告來結束.
我們求主幫助我們.
來遵從別人的命令和教訓.
我們整個聖經裡面所做的每一件事.
都是帶著禱告來遵從.
所以當時發覺.
整個我們對於上帝的回應.
卡伯就說.
我們對於上帝的回應.
其實都是基於這三樣東西.
我們是奉行上帝.
我們是信仰上帝.
我們相信上帝.
倚靠上帝.

$^{761}$我們是禱告上帝.
所以禱告可以說是我們對於.
整個基督徒生命的底蘊.
我們一切對基督徒的行動.
對上帝的回應.
其實都是一個禱告.
因為就是說.
我們整個的生命就是一種禱告.
當我們稱呼上帝為阿爸父的時候.
其實我們就帶著一份禱告.
來回應這位上帝.
所以我想說我們的靈命.
我們的靈命最基本.
其實就是一種這樣的關係.
我們去倚靠上帝.
我們去懇求上帝幫助我們.
雖然我們願意回應他.
但是這個回應其實最基本來說.
都是一個懇求的回應.
都是懇求的回應.
所以我想說.
我們整個的基督徒生命.
就是一種禱告.
這個也像昨天所說.
無論在加泰書所說的阿爸父.
或者後期羅馬書所強調.
再重複一次的.
就是阿爸父的回應.
我們整個基督徒生命.
本來就是一種禱告.
我們所謂靈命這件事.
其實靈命這件事.
你會發覺都找不到.
聖經裡面的字眼出現.
聖經裡面沒有提到有靈命這個字.
不過我想說.
加泰書和羅馬書所強調.
我們聖靈內在的新生命.
那種生存.
那種成為上帝兒女的身份.

$^{801}$就是我們的靈靈所在.
所謂靈命的意思是什麼.
就是聖靈在我們生命裡面.
不簡單.
所謂靈命.
其實就是聖靈.
在我們生命裡面的每分每秒.
所以靈命不是一些很怪異的東西.
我們覺得.
我們環教會裡面.
靈命是一種.
好像一種.
像八楷草脊分.
就是一條靈命要養著.
在生命裡面有條命.
要吊著它.
如果你不靈修.
那條靈命就會死.
靈命就會枯乾.
我的靈命要不斷去參加培靈會.
來靈修.
來不斷養著它.
有一點點的感覺.
就是你有一條命.
要不斷去餵它.
就是這樣說.
但其實靈命不是在你生命裡面的某一部分.
靈命是你整個的生命.
你的靈命就是你整個的Christian life.
當聖靈在你生命裡面的時候.
你所面對整個的世界.
你所面對整個的生命的場所.
其實就是靈命的全部.
所以我們不去嘗試將靈命來去狹義化.
不去覺得靈命是某一件事.
是我的所謂的Spiritual aspect.
沒有所謂的Spiritual aspect.
因為你所有的Aspect.
就是在你有聖靈裡面的工作和參與.
所以我們說靈命其實是一種.

$^{841}$在聖靈裡面.
就是基督的靈在我們裡面.
我們和基督同行.
我們去跟隨基督的互動的關係.
靈命再不是我們傳統華人教所說的.
一種好像八街儲積分的感覺.
我們要不斷地去儲著儲著.
你靈修超過一百次.
就有一張積分表.
就可以去到某一個Level.
沒有的.
靈命不是這樣去直線成正比.
當然我們要靈修.
我們都需要去稱聖神.
但不是說你靈修得越多.
就靈命得越好.
每天靈修人.
但他離開上帝就沒事了.
我們就不會說.
冷存了三分釘的概念.
所以靈修是一種All or Nothing的概念.
一個罪人.
當他願意去迴轉的時候.
他就是100\%的一個屬靈人.
相反來說.
一個屬靈人.
離開上帝的時候.
就是一個零的概念.
所以沒有一種擁有的概念.
靈命不是一種我們擁有的東西.
不是一些我們可以透過某種的.
敬虔的方法.
慢慢成為一種靈命的功力.
所以為什麼我都說.
靈修這字有一點誤導.
因為修字.
令我們覺得.
好像我們中國人裡面.
有一種修行的概念.
靈命是否能夠修行得來呢?.

$^{881}$這個就是我們今天可以思考的.
靈命是否能夠修煉回來的東西呢?.
是否能夠透過某種的.
regular來做的東西.
能夠得著的東西呢?.
靈修了500次人.
是否比靈修了8次人更高級呢?.
這個就是我們嘗試去問的問題.
所以就說.
靈修,屬靈,陪靈.
這些字都令我們覺得.
靈修,屬靈,陪靈.
是一種我們能夠透過.
一種累積的方法.
來增大一個這樣的概念.
這個就似乎.
我覺得不是那麼簡單.
簡單來說.
我們可以這樣說.
我們說聖靈的九個果子.
或者九個aspect.
簡單來說.
我們要有節制這個果子.
怎樣能夠有節制這個果子呢?.
怎樣能夠有節制呢?.
節制這個果子.
不是單單靈修靈回來的.
不是單單是你在那個樂室裡面.
躲起來靈修祈禱就完.
你需要有節制果子就怎樣做呢?.
就是要在生活裡面.
慢慢去實踐節制這個功課.
你需要去忍受不要賣東西.
你需要去面對人的時候節制.
不要去那麼容易爆炸.
所以節制這個東西.
不是單單是一種內在的屬靈的價值.
而是我們在外在裡面.
我們生活裡面.
一些很平常的一種行為.

$^{921}$因此這個果子.
慈愛果子也一樣.
慈愛因此.
當然我們很重要就是和神的關係.
但也很重要就是你和人的相處.
你很難相信一個屬靈人.
但是和人的關係不好.
屬靈人不會自然和人的關係好.
他需要在世界的場所裡面.
去實踐出那個靈明.
所以說靈明不是單單純粹.
一些我們一種.
透過華人教會式的靈修的方法.
來得著的東西.
它是一個很重要的部分.
或者是一個很重要的方法途徑之一.
但就不是一種我們靈修靈得多.
就什麼都搞定了.
因為我們仍然要面對著這個世界.
所以說整個的靈明.
Christian life is a life of the Holy Spirit.
所謂的靈明.
就是我們整個的Christian life.
在聖人之下的整個Christian life.
你怎樣來做人的女婿.
怎樣來做人的媳婦.
怎樣來做人的老闆.
整個生活裡面的每一樣東西.
都是我們靈明的彰顯和操練的地方.
多於我們純粹一天裡面的十分鐘.
二十分鐘.
我經常說我們做傳道人都很好.
我們一天裡面可以有超過一半時間去看聖經.
經常可以對著聖經.
你一天裡面靈修.
能夠靈二十分鐘都很好.
我們單單去定義靈修是在你那二十分鐘的時候.
那是什麼意思.
一天二十分鐘裡面.
其他的時間就沒有靈修了.

$^{961}$所以我想說靈修就不能夠單單局限於.
這種狹義的所謂靈修的時間.
那個只是一個很好的開始.
但當你合上靈修書之後.
真正的那種靈明的場所.
才是真正的開始.
你怎樣能夠在生活裡面.
去展現你的靈性.
怎樣能夠和上帝相遇.
這才是更加重要的概念和機會.
這個我也有一個漫畫.
不知道等姐妹是什麼年紀.
可能有些比我小有些比我大.
其中一個很好的例子.
大家看到這裡有一個龍珠漫畫.
這個就說到悟空和悟飯修煉.
他們就嘗試將這個.
超殺人的狀態.
成為一種恆常的狀態.
所以變身成為超殺人.
就再不是一種特別的狀態.
不是我要爆炸那一刻才成為一個熟人.
而是嘗試將一種這麼好的狀態.
成為一種恆常的狀態.
這就是我們想說的.
靈修不再只是一種特殊的經驗.
我沒有反對過特殊經驗.
但我們不能單單依靠特殊經驗.
去量度我們的靈明.
可能你一年裡參加的那次的培靈會.
成為了你的靈明的高峰.
即是說你能夠在以色列朝聖.
那一刻就成為了你的高峰.
當然這些是很好的.
但你不能夠每個星期都去一次以色列.
所以更重要的是我們整個生命裡.
一些很平常,很沉悶的日子裡.
如何能夠捉起我們的靈性.
這反而是我們更加需要思考的課題.
所以這就是我早期的思考.

$^{1001}$這是我在德國回來後.
特別是回來香港後.
我頭兩年寫的文章就是說這些東西.
就是很強調我們的生命.
就是整個靈明的場所.
諸如此類的文章.
所以這就是我今天仍然同意的.
仍然覺得我們的靈明.
不是我們很華人教會式的跑數制度.
不是單單透過靈修的次數來量度一個人的靈明.
或者來建立一個靈明.
因為這樣的靈明成為了一種資產.
我能夠透過某種方法.
能夠得著某種靈明的質素.
我發覺靈明是一種因典.
是否有話想說.
是上帝的因典.
靈明是需要你去享受的.
靈明是不應該成為你的重擔.
而是一種不能夠操控的東西.
這就是我幾年前的看法.
不過我也說過.
後來我慢慢擴展這種看法.
我去到第二部分.
嘗試去到一些神學的建構.
我們嘗試從聖靈這個字開始說起.
因為聖靈是一種關乎於靈明的東西.
我們說靈明就是聖靈.
是我們生命裡面的全部.
所以我們就理解一下聖靈這個課題.
因為我們要去明白.
上帝的聖靈是一種什麼樣的靈.
我們需要先來一種神學的基本反省.
首先我覺得聖靈是一個關係的靈.
什麼意思呢?.
聖靈是一個建立關係的靈.
當然這是很不容易去說.
因為我們要探討到一個聖靈位格的問題.
在聖經裡面.
你會發現聖靈的位格性其實是不容易處理的.

$^{1041}$如果你認真去處理的時候.
所謂位格就是能夠分別於其他東西.
簡單來說就是一個獨立的東西.
聖靈是有位格的.
意思就是它是有別於聖父和聖子.
聖靈不是屬於聖子或聖父.
不過你會發現聖靈裡面有很多這樣的經文.
比如耶穌基督的靈.
如果這樣說的話.
耶穌基督的靈和基督的靈.
似乎是說聖靈是屬於耶穌的.
是屬於聖父的.
所以這些經文似乎是說.
聖靈是一種隸屬於耶穌的靈.
或者是耶和華自己的靈.
不過我們說.
所以這個也是根據一些探討聖靈論的哲學家去理解.
我們說聖靈有位格.
因為它是能夠捉起一個父與子的關係.
聖靈是重要的.
因為它能夠有別於父和子.
因為它能夠做到一些父和子做不到的事.
就是連結了父和子之間的關係.
所以我們說聖靈是父和子的靈.
這個似乎是沒有位格的看法.
不過我們強調.
聖靈同時是父和子之間的那份愛.
如果你看一書的時候.
當他講到聖靈的時候.
聖靈成為了一種很重要的上帝的愛.
這種愛的關係正正是聖靈裡面的那種關係.
所以我們說聖靈是一個關係的建立者.
有上師也複雜地說.
父和子之間正正是一種在靈裡面的交流的關係.
所以我們說父子靈不一定是三個型的.
而是一種互動的關係.
父和子在靈裡面彼此互動.
所以每一次提到聖靈的時候.
都是關乎於位格裡面的另外兩位.
譬如說耶穌基督的度數身是在聖靈裡面的感人.

$^{1081}$或者是聖父在耶穌的手裡插在聖靈裡面.
或者是耶穌復活的時候.
都是聖父在聖靈的力量裡面指復活.
所以父和子的交流是牽涉到聖靈自己.
今天我不嘗試複雜化.
不想拉得太遠.
不過我想說的是.
聖靈是一種關係的靈.
它嘗試捉起了內在的關係.
除此之外都是牽涉到上帝和世界和人的關係.
當耶和華的靈在世界上的時候.
神的靈在上面的時候.
整個創造主和世界正正在靈裡面的互動.
耶和華的靈和世界裡面都是這樣的關係.
所以當聖父和世界有關係的時候.
都是耶和華的靈.
靈不代表這個靈是屬於耶和華自己.
而是屬於靈.
成為了耶和華聖父和世界之間的關係.
基督徒和基督之間都一樣.
我們說不著聖靈就不能稱耶穌為主.
所以似乎當一個人和耶穌基督有關係的時候.
聖靈正正就是在當中成為一個很重要的橋樑.
所以當保羅說到基督的靈的時候.
聖靈正正就是在當中成為我們和基督之間.
一個很重要的連接點.
所以聖靈就好像一個膠水一樣.
它會連結著每一樣東西.
連結著人和人之間的同感不靈.
連結著父和子.
連結著世界和父或者耶穌基督.
所以它是一個關乎於聖靈.
每一個人都一樣.
每一個受眾物都有上帝的靈在當中.
讓我們能夠去傳播.
這就是創造出聖靈.
所以聖靈就像風一樣.
能夠連結著世界上每一樣東西和上帝自己.
所以父和子之間稱之為內在的三一關係.
當它說到世界和人的時候.

$^{1121}$就是經濟三一關係.
這一點很深.
如果大家不明白的話可以skip.
或者大概明白.
聖靈是一種關係的靈.
成為一種關係的建立者.
所以聖靈是一種捉起我們和耶穌基督的關係.
特別是在新約當我們提到基督的靈的時候.
基督的靈就是當我們一群基督徒.
和基督耶穌連結的時候.
聖靈就成為了一個很重要的橋樑.
基督的靈親自來到我們當中.
我們經常說耶穌基督仍然活著.
仍然在我們心中.
這是一個很嘉義民的說法.
基督在我們心靈裡活著.
是因為聖靈在我們心靈裡活著.
基督耶穌在上帝的右邊.
在星天裡.
將要再來.
基督的靈成為我們心靈裡很重要的一個靈.
是一個神聖的靈.
是一個和我們連結基督的靈.
所以當我們去理解基督的靈的時候.
我們就發現其實我們可以避免某種很零音派的說法.
就是說我們去強調基督的靈在我們心靈裡.
所以我們去和基督有關係的時候.
這就牽涉到聖靈的關係.
所以聖靈充滿和跟隨基督.
兩個概念.
讓大家問一下.
你覺得跟隨基督比較厲害還是聖靈充滿比較厲害?.
你覺得跟隨基督是更加高層次.
還是聖靈充滿比較厲害?.
你會覺得是無可比擬的.
因為跟隨基督是一個很客觀的東西.
我們願意在生命裡跟隨耶穌.
這同時就是基督的靈充滿.
所以兩樣東西不是兩樣東西.
是一樣的事情.

$^{1161}$當我們在愛顯上面.
我們願意去遵從基督的呼召.
去愛基督耶穌.
願意為主而活的時候.
這其實是一種很subjective的說法來講.
這就是聖靈的充滿.
聖靈充滿不是一種很零音的說法.
而是一種很感受上的.
一種很零音的外表或行為.
而很重要的是我們要跟隨基督.
所以兩樣東西是無可比擬的.
因為兩樣東西是一個銀幣的兩面.
一個客觀上的跟隨基督.
和一個主觀上的我們如何被基督的靈充滿.
聖經裡面所講的很多有關聖靈的動詞.
聖靈的充滿.
聖靈的高立.
聖靈的降臨.
聖靈的離開.
聖靈的內注.
這些動詞都是某種人法上去強調客觀的關係.
這個詳細的不講.
我們很容易會說.
每一種很零音的聖靈的看法.
都仍然有一種很客觀的東西.
可能想像一個很聖靈充滿的人.
但是不跟隨耶穌.
跟隨耶穌的人正正就是一個有基督的靈.
充滿的人.
一個圓滿有基督的人.
所以我們去效法基督.
和強調一種成聖的靈.
聖靈是一種更加讓我們成聖.
這是一個很強調.
很方牌的看法.
成聖.
很嘉義民的看法.
成聖正正就是一個聖靈的內注.
是一種聖靈的工作.
所以我們願意去效法基督.

$^{1201}$同時間也有成聖的靈.
當中去幫助我們.
另一方面就是客觀的毀身.
和我們主觀的愛上帝.
都是一樣.
我們對於上帝的愛.
其實這個可以是很大的主題.
我們對於上帝的愛.
不單單是關乎於阿加皮的毀身.
我們環教會很強調.
幫你愛上帝就要怎樣.
就要侍奉.
你要去付出.
你要為上帝去做一些事.
去犧牲某些事.
這個當然是很重要的.
但其實我們很強調.
在宗教裡面很強調.
對上帝的愛是一種愛慕.
就是character.
愛不單單是一種毀身和犧牲.
而是一種學慕和愛慕.
這個只是一種很靈修的角度.
我們愛上帝除了是遵循上帝的律法之外.
同時也是一種很學慕上帝的同在.
所以兩者都是重要的.
我們客觀的對上帝的毀身.
和我們對於上帝的感受上的愛.
其實都是一體兩面.
我們願意付出.
願意背負十字架.
也有一種對上帝感受上的愛.
稱之為Divine Eroch.
就是一種對上帝的傾慕.
很想和神一起.
很想和耶穌一起.
所以我們嘗試將靈明.
在某種程度上.
和一種客觀的跟隨基督的概念連結.
第三就是生命的靈.

$^{1241}$這個也是今天我想強調的課題.
生命的靈是甚麼呢?.
當我們說我們的靈明是整個生命的時候.
我們的生命.
所謂的靈明.
不單單是一種屬靈的東西.
我們懷疑是整個生命中.
能夠彰顯的東西.
這個也是羅馬書第八章第二節所說.
一個生命的靈.
Tom Primatos 即是Tisoué.
保羅提到一個生命的靈.
所以聖靈同時也是生命的靈.
首先作為創造主性的靈.
讓每個世界的生命得以存活.
沒有一個生命是沒有上帝的靈.
因為耶穌說靈充滿全地.
讓凡有海中的生物.
所有天空中的生物.
都能夠來到去存活.
所以聖靈本身.
或上帝的靈本身.
就是讓萬物都能得以存活.
有生命.
因為上帝是永活的上帝.
這個也是我自己.
過去剛剛做翻譯的工作.
我剛剛完成了一個翻譯.
Multuman的一本書.
叫做The Living God and the Fulness of Life.
就是一個生命的上帝.
所以我們的上帝.
被稱之為永活的耶穌說.
永活的意思不是單單的永生這麼簡單.
而是一個The Living God.
上帝是有生命的.
上帝是活著的.
上帝不是單單一個哲學上的上帝.
不是一個全能全知無所不知的上帝.
上帝是有生命的上帝.

$^{1281}$反過來說.
我們的生命.
是本源於這個生命的上帝.
所以我們不是說我們是有生命.
所以才想到上帝是有生命.
而是相反的.
上帝是有生命.
所以才有我們生命的存在.
當我們這樣看的時候.
上帝是有生命的時候.
其實是說.
上帝的生命其實是連貫著我們的生命.
所以我們所謂的活著.
我們的生命.
是關乎於一種很淑靈.
或者是一種生命的上帝自己.
當然了.
聖經裡面也說到.
耶穌說什麼.
我來了,小幾個人得生命.
變得更加的豐盛.
所以耶穌基督就是生命道路真理.
所以父子聖靈.
都是一個生命的上帝.
所以我們就嘗試去思考.
一種嶄新的靈命觀.
我稱之為生命的靈性.
生命的靈性.
這個今天就成為了.
我們最後一部分.
我們一起來思考的一種看法.
我們說Live這個字.
Live這個字本來就是一種.
這裡可以回到我的畫面.
因為這個畫面暫時停在這裡.
大家可以看到我的樣子.
不要太悶.
我們說Live這個字.
Live這個字有兩個意思.
一個是解釋生命.

$^{1321}$一個是生活.
所以我們說生命.
不是單單說一種活著.
那種生命的死亡.
而是我們的生活.
我們在世界裡面.
在時間裡面.
我們怎樣來存活呢.
怎樣去活著呢.
這是我們要探討的問題.
生命也不是單單去活著.
生命不是純粹的吃飯.
睡覺.
去廁所洗澡.
呼吸.
因為上帝是生命的上帝.
所以我們的生命.
不是單單純粹的.
不用死這麼簡單.
當我們來活著的時候.
我們其實不是單單追求不死.
而是追求更加有活得精彩.
和有生命力的那種生命.
所以這個也是摩特曼的說法.
我們所說的.
如果聖靈是生命的靈的時候.
我們就說到.
其實我們所謂的spirituality.
不是一種狹義的看法.
我們不是單單追求.
某種我們生命裡面的其中一部份.
我們有我們的工作.
我們有我們的家庭.
我們有我們的spirituality.
不是.
spirituality不是一種狹義的概念.
如果聖靈是生命的靈的時候.
我們更加能夠得到的.
是一種vitality.
是一種生命力.

$^{1361}$聖靈是讓我們能夠在我們生命裡面.
面對著一種更加有生命力的力量.
面對著今天的全球的疫情.
我們怎樣能夠彰顯這個vitality.
不是單單離開這個世界去領受.
而是怎樣能夠在這個世界裡面.
去彰顯這種vitality.
面對著困苦和艱難.
怎樣能夠去彰顯我們的力量.
這是我們很重要的課題.
所以我們很強調.
我們的靈命不單單是離開世界的靈命.
而是我們在我們生命裡面.
怎樣面對著很多艱難.
特別是香港現在的社會和世代裡面.
面對著疫情裡面.
面對著人生很多的顛簸裡面.
我們怎樣能夠在這些moment裡面去活著.
有生命力的活著.
生命力是一種對抗著生命苦難的很重要的力量.
這就是我們靈命裡面很重要的部份.
靈命不是我們傳統環球所講的道德.
屬靈人不是單單是一種講話的態度.
也不是純粹一種屬靈的表現.
而是我們怎樣能夠在逆境裡面去生存.
能夠彰顯上帝的生命呢.
另外我想講的是身體和靈魂.
很強調的是我們整個的性靈工作.
是我們整個人的生命的時候.
靈命其實不是單單是一件靈魂的事情.
也不是單單是一件靈的事情.
我們環教會比較少提靈魂這個字.
因為我們覺得靈魂好像有點奇怪.
我們比較少講靈.
但其實靈魂和靈魂是一樣的東西.
我們也不是單單講靈和靈魂.
而是講身體.
就是我們整個人.
所以當我們去經歷上帝.
或者我們去經歷靈命的時候.

$^{1401}$我們不是單單是一種內在的生命.
不是單單閉上眼睛.
那種在黑暗裡面的祈禱.
或者遠離世界的祈禱.
而是我們整個的身體和生活.
所以這也是莫特曼的講法.
他有一種叫Spirituality of Senses.
就是我們整個人.
就是我們眼耳口鼻 觸覺 味覺 聽覺.
那種觀感.
這些觀感都是我們靈性裡面重要的東西.
我們以前比較傾向.
就是我們要關閉我們的觀感.
我們要閉上眼睛.
我們要離開某種的世界.
閉上眼睛去祈禱.
但你會發覺.
如果這個世界就是上帝的存在的場所.
我們整個的身體就是我們靈性的一部份的話.
我們的眼耳口鼻.
其實都是一種經歷上帝的重要的東西.
不知道大家在這個病毒下如何生活.
當大家在家裡面的時候.
可能大家都覺得很煙悶.
當大家在家裡面的時候.
其實都覺得生活都有一些不容易的感覺.
但當你一出街的時候.
當你一出街的時候.
你會吸一些空氣是會新鮮一些的.
你會聽到外面世界的聲音.
你會看到一些新的東西.
你會發覺這些東西會令你整個人不同一些.
所以我們說我們的眼睛.
我們的耳朵.
我們的味覺.
我們的觸覺.
其實都是一種去經歷上帝的重要渠道.
不單單是我們所謂的靈性.
或者是我們的靈魂.
傳統環教會裡面很強調的就是眼睛.

$^{1441}$因為我們要讀理解.
因為我們靈修裡面就是去看著文字.
然後去讀理解.
這個靈修我覺得只是部份.
它很強調我們的眼睛和思維.
或者是每一種的閱讀.
但其實我們的靈性不單單是這樣.
我們可以透過我們的不同觀感.
來去接觸上帝.
這個世界的上帝是很豐富的.
所以這個世界裡面的上帝.
是可以經歷很多不同裡面的觀感.
透過聽覺.
透過我們的味覺.
透過我們不同的經驗來經歷上帝.
我自己很喜歡.
特別是在這年裡面.
我在安息年裡面.
基本上我日日都是煮飯.
我覺得煮飯裡面.
其實都是一個很好的.
令自己的生命.
一個很好的沉澱過程.
我覺得煮飯是一件很放鬆的事情.
煮飯是能夠讓我們從日.
預備菜sung .
然後慢慢想怎樣煮.
我覺得是一個很享受的過程.
這種都是我們去經歷上帝的一種很好的過程.
透過味覺去經歷上帝的恩典.
透過我們煮飯裡面的專注.
能夠來開展我們生命的某種部份.
所以我們有很多課題要講.
大家有機會可以再看我以後出版的書.
我想說的是我們的靈性的操練.
可以有很多種.
透過不同的觀感來開展我們的靈性.
另外我想說的是我們生命的意義.
其實這個都是一個很重要的課題.
既然我們強調上帝是永活的上帝.

$^{1481}$是生命的上帝的時候.
我們活著本身是一種Blessing.
我經常這樣說.
我們去存活.
當我們有生命的時候.
這正正就是我們去建基.
我們所謂的靈性很重要的部份.
因為我們跟神的關係.
正正就是起源於這個生命本身.
所以我一開始覺得我跑步.
是很能夠親近上帝.
因為我跑步覺得自己能夠單純地活著.
所以當我們純粹來享受生命.
來熱愛生命.
不單單是享受每一天上帝給我們的生命.
更加熱愛其他生命.
無論是其他人的生存.
或者是其他人的生命.
我們愛人.
或者是其他動物.
其他大自然的東西.
任何的生命都能夠讓我們彰顯對於上帝生命的熱愛.
這是很重要的.
我自己在翻譯《莫特曼本書》的時候.
有句說話很觸動我.
就是「We are living in this living」.
意思就是說我們活著就是為了活著.
我覺得這是一個很能夠幫助我自己的說話.
因為我十幾歲的時候就問了這個問題.
為什麼人要活著.
為什麼人要生存.
《莫特曼本書》說我們活著就是為了活著而活.
這是唯有基督教的上帝才能夠理解的說話.
因為上帝是生命的主.
所以讓我們活著.
所以我們單單來享受上帝給我們的生命.
享受活著的這一刻.
已經是我們活著的意義.
生命不是一種工具.
生命的過程也不是一種工具.

$^{1521}$不是讓我們達到某個目標.
或者去利用人達到某個目標.
而是單單來享受給我們在世幾十年的生命.
是一個很重要的價值.
當你這樣想的時候.
你會發覺我們能夠去明白.
如何能夠在生命裡面經歷這份上帝恩典.
我們的靈性正正是在我們生命裡面.
所以我們的生命也是一個愛生命.
他讓我們生命不是單單活著.
而是有愛在當中.
因為上帝就是愛.
所以我們的生命純粹可以.
就像彭華所說的.
我們是一種為別人而活的生命.
我們是為其他人而活.
這句說話是可能的.
我們為其他人而活.
因為既然我們的生命是一種恩典的時候.
就能夠讓我們為其他人而活.
我們有這樣的本錢去愛人.
我們不是工具化一個人.
這份愛同時也讓我們去承受其他人的痛苦.
我想說的今天是一種很錯的觀念.
也希望大家不要誤會.
熱愛生命不代表去避開苦難.
熱愛生命不是要將我的生命變得更加完美.
從而去避開某些苦難.
將自己的生命優化.
只擔心自己好的東西.
去避免任何的苦難.
不是這樣的.
當我們真的熱愛生命的時候.
我們去為其他人而活的時候.
我們的愛是可以讓我們傷痛的.
當我們面對著香港的情況.
或者面對著很多世界不同的人的苦難的時候.
我們的生命是會經歷苦痛的.
經歷苦痛是一個生命的靈性中很重要的過程.
我們生命靈性不是要避開苦難.

$^{1561}$純粹去享受一些好的東西.
這是自私的.
單單選擇一些對我們有利的東西.
而去避開一些對我們不好的東西.
這不是享受生命的意思.
所以享受生命不是純粹享受奢華.
而是我們去單單享受活著.
更加去經歷生命和愛所帶來的傷痛.
所以我們基督徒的心命仍然是可以封城的.
仍然是經過很多和人同哭同苦的過程.
但同時間透過生命來經歷這種苦難.
所以我們不是去逃避苦難.
而是去更加有力量去抵擋這些苦難.
然後我想說的是如祿所說的規碎.
我們面對苦難的時候.
就好像一個鹿一樣.
其實經文所說的鹿.
其實一點都不是我們所想的.
我們唱的歌曲.
其實就像一首很美麗的圖畫.
一隻鹿在溪水裡喝水.
好像很享受.
其實經文的意思並不是這樣.
經文是一個很悲慘的鹿.
是一個非常口渴的鹿.
當他面對著世界的苦難的時候.
他學了慕耶和華.
這個字其實是基學裡的學.
我們面對著生命的苦難的時候.
我們就會學慕這位生命的主.
生命學慕生命.
生命都吸引生命.
所以這個生命的上帝就會吸引我們.
更加來學慕他.
面對著一個苦難的世界.
我們更加來追求這位生命的上帝.
剩下十分鐘.
我要快一點說.
他講到修辭的部分.
靈性的修辭.

$^{1601}$為什麼我們今天的討論叫修辭.
我特意用修辭這個字.
因為不是一種修行.
不是一種修煉.
而是一種修辭.
因為不是單單純粹一種行為.
而是一種我們生命擁有的態度.
我們怎樣能夠去持守一些價值.
怎樣能夠去收養自己的生命.
這個都是很重要的東西.
首先我想說一點批判.
我們對於靈修.
我們其實不是單單去強調方法.
以下所說的都不是一些方法.
這部分是想說一些建議.
但不是一些方法.
不是說你做完ABCDEFG之後.
就能夠做到某些東西.
這也是我們今天環教裡面.
靈修很嘗試去避免的課題.
我們的靈修不是一種方法.
如果上帝是生命的上帝.
你不能夠去方法了他.
你明白我的意思嗎.
你不能夠用某種方法.
就能夠捉到這隻鹿.
當你嘗試去理論化他.
或者用某種理論去捉住他的時候.
這個就不是上帝.
而是一種道理或是一種偶像.
所以基本上我覺得靈修是沒有方法的.
我經常覺得靈修就像減肥一樣.
我們有很多不同的減肥的書.
很多不同的減肥的講座.
很多不同的減肥秘方.
很多不同的減肥清單.
你不會缺這些方法.
問題是真正能夠減肥其實很少.
你拿著一百個減肥食譜.
都不能夠幫你減肥.

$^{1641}$所以靈修也是一樣.
靈修不是一種方法.
很多人嘗試去分享他的方法給你聽.
但是這樣的方法不是重點.
靈修就像一把梯子.
你能夠在梯子上爬上去.
你就要扔下梯子.
我們也談過.
我們嘗試去研究靈修學人也是一樣.
嘗試去研究六百年前.
一些中世紀裡面的靈修人.
嘗試去寫博士論文.
然後成為一個靈修學者.
當中其實很奇怪.
六百年前的人和上帝的經歷.
嘗試用文字去記載.
然後你看回文字.
其實整件事是很易守的.
我們嘗試去用人家過往的經驗.
來成為我們的經驗.
當中可能是幫到我們.
或者是一種參考.
但是你不能夠照給人家的東西.
因為這是不能複製的.
因為這是人家當刻和上帝的關係的分享.
而你是有你自己和上帝的關係的分享.
另一方面就是上帝是奧秘來的.
他不能夠透過做出ABCDE就能夠捉到他.
當然我不是完全反對這件事.
方法是一種幫助.
就像保羅說的.
是一種幫助人去成長的系統.
成長的一種輔助.
但是你要拋棄它.
去到某個程度上就要拋棄這些輔助.
讓你能夠真正去經驗到上帝.
所以我會覺得我們要避免一種沒有思考的方法.
雖然你也知道我寫了這套字見.
我是寫靈修的東西出來的人.
但是我覺得靈修書或者其他靈修的複讀.

$^{1681}$都是作者自己當下和上帝經歷的文字.
它只不過是能夠幫助你自己去到達上帝.
所以我覺得靈修也挺考慮的.
靈修是一種你自己和上帝經歷的過程.
多於一種純粹來閱讀理解的東西.
我嘗試去避免任何填鴨式的靈修方法.
在這裡強調靜的靈性.
我仍然很重視我們靜的一刻.
我們有一種向內的靈性.
我們很強調在靈魂和靈裡面和上帝的接觸.
這是不否定的.
靜的靈性都是重要的.
特別是一些很愛向的人.
任何愛向的人都懂得去安靜自己.
相反來說一些內向的人.
需要去更加重要的去動靈性.
我們嘗試不單單自閉.
而是嘗試透過我們的心護和群體裡面.
都可以去尋找我們和上帝的關係.
所以我覺得這兩方面都很重要.
當你缺乏其中一方面.
就要去想如何能夠更加擴闊自己的靈性導向.
靈修導向.
做一些新事.
上帝是新的上帝.
新的概念是一個很重要的概念.
上帝是一種創造的上帝.
上帝給我們有新天新地.
新的創造.
新遊說覽新創藝球.
所以靈修都是很重要.
是一個關乎於新的概念.
當你每天都困在家裡.
和今天大家都隔離在家裡.
你會發覺是很冤枉的.
你會發覺也是這樣.
所以我發覺生命本身.
不能夠很機械化地重複一些事情.
我們嘗試在生活裡面.
去作出一些小小的更改.

$^{1721}$作出一些小小的更新.
這也是我們很重要的靈性裡的彰顯.
嘗試脫離一些黑板的日常生活.
嘗試在日常生活裡面尋求一些突破.
都是一些很重要的我們靈性裡的向導.
如果你是一個很創造的人.
如果你是一個創作人.
可能你在創作裡面.
寫詩歌或者填詞.
或者畫畫或者寫文章.
這些創作其實都是我們靈性很重要的部分.
所以如果你喜歡寫書法.
喜歡畫畫.
這也是一個很重要的我們靈性裡的彰顯.
都是我們今天靈修裡的一個很重要的方法.
修辭.
Theory.
Career.
這個就是我們強調的那種看見上帝.
看見上帝是一個很.
傳統裡面一個很高的境界.
特別是中世紀裡面都強調.
我們人能夠看見上帝.
小小的都說.
清心的人有福了.
因為他們不必看見神.
我們嘗試在我們的生活裡面.
看見上帝.
是一個很重要的經歷.
所以我們不是閉上眼.
不是來觀賞眼睛來祈禱.
而是在生命裡面.
在一些很細小的地方來發現上帝自己.
這個也是我們一個很重要的修辭的向導.
另外就是我們去發現美麗.
上帝的創造.
上帝是美麗.
這個也是我們一個很重要的一個向導.
不好意思今天因為很長.
所以今天沒有話題都不能夠詳細講.

$^{1761}$有興趣的話可以買我的作品.
當這本書出的時候.
所以我們能夠在不同的地方.
去思考上帝.
最後我也想說一些其他東西.
快樂我想說.
其他可能有機會再分享.
因為時間的關係.
特別是今天我們面對著苦難的時候.
無論在疫情裡面.
或者在香港的社會裡面.
遇到一些令人不能夠快樂的事情.
既然我們的靈性和生命力是一種.
我想快樂其實是一種生命裡面的彰顯.
生命本身的狀態就是快樂.
如果你看小朋友的時候.
他們開心是他們的原本狀態.
他們會哭他們會生氣.
但是快樂是上帝創造人最原本的狀態.
因為我們活著就是享受生命.
所以快樂是我們本身有的東西.
快樂不是我們追求的東西.
而是我們本身有的東西.
只不過很多世間裡面的苦難.
令我們失去快樂.
我強調我們的生命不是去避免一些不快樂.
這個我也曾經寫過書.
關乎於這方面.
快樂或者追求快樂不是去避免一些不快樂的事情.
或者是去不斷營造快樂的事情給自己.
這是一種仍然有問題的生活.
因為這是不樂得到滿足的.
不斷追求快樂的事情.
可能是買車買房子的奢華生活.
避免苦難不看新聞.
或者避免其他人的痛苦.
這個也不是快樂之道.
真正的快樂之道其實是去享受.
或者是回到明白上帝給我們的生命是什麼.
讓這種快樂.

$^{1801}$這種我們已經擁有的生命力.
成為我們不能被動搖的價值.
用這種生命來面對苦難.
讓這種生命去經歷苦難.
經歷一些不快樂的事情.
快樂的靈性不代表不開心.
我們仍然會哭和不開心.
不過我們能夠有生命力來過渡.
和有本錢去經歷這些不快樂的事情.
所以我想說.
我們這種靈性不是去避免苦難.
而是反過來我們有本錢.
在不開心的一個混亂的世界裡.
去經歷這些苦難.
所以我們能夠更加有這種力量.
去得到一種很基本上帝而來的快樂.
所以今天沒時間去說全部.
因為也給大家買一本書來看看.
因為當中有很多不同的課題.
一些比較零散的課題.
讓大家一起來思考.
大概今天分享到這裡.
我們先給Sophia或大會時間.
很豐富.
我覺得原來我們城北環基督教會.
今天在這裡也很有趣.
我們只不過問你一個很短的問題.
你能夠展示出你這麼厲害的書.
我也要學習.
因為我的題目也很廣泛.
亦在兄姊妹心裡.
昨天上完課後也有些題目.
你自己也收到一些.
我只選擇一部分說.
你昨天也有說到.
今天也有說到.
祈禱和屬靈的修持.
特別說一說我們華人教會.
很著重祈禱會.
特別在疫情中.

$^{1841}$現在不是崇拜.
(開冷氣才有點熱).
不是崇拜或小組聚會.
特別是說祈禱會.
很多人都說.
在疫情中, 祈禱會人數大增.
大家很服氣, 心裡很開心.
我想問一問.
你對這件事有什麼看法?.
你怎樣衡量復興的指標?.
如果離不開人數或氛圍氣氛.
你怎樣教我們跳出這個框框?.
有什麼建議?.
多謝這個問題.
可以補充昨天的事.
我們說祈禱會運動.
本身是復興運動的產物.
第二次復興運動時.
我們發覺有一個很特別的觀念.
就是復興主義.
認為我們人透過某種活動和經期.
就能夠大力地復興.
當時也有復興的神學.
只要做足某種事情.
就能夠復興.
包括祈禱會, 大家聚在一起.
所以有種強調.
只要做足ABCDEFG就能夠大力復興.
這是我們的背景.
祈禱會本身就是這樣的概念.
透過額外的祈禱.
格外的敬虔.
來大力復興.
這種看法是有少許.
人的祈禱能夠成為.
直接影響上帝的一種.
當然我們很相信祈禱.
但也能成為一種很理論的東西.
只要按下按鈕就能夠出第二件事.
只要按下A就有B.

$^{1881}$昨天也說過.
祈禱的本質就是承認我們不能夠.
承認我們的軟弱.
我們是基於一種呼求.
去求上帝幫助.
所以我想祈禱也是一種不能夠成為的操控.
透過我們達萬.
透過我們很多人一起祈禱.
就能夠更加強勁一點.
我發覺其實也沒有什麼聖經根據.
只要我們人多就能夠更加.
能夠做到祈禱.
我覺得這似乎不是聖經教導.
所以我們大力教會也很想多人祈禱.
但原因不是這樣.
不是因為我們覺得越疊碼就越能夠.
做到更強勁的效果.
一萬人祈禱就能夠做到.
我覺得因果不是這樣.
當然我們也很強調.
我們也要多人祈禱.
因為我們說新教會強調.
祈禱是一種服從.
所以我們願意更加多等之每.
聽話的基督徒.
從而去遵從祈禱的教導.
但我們就不是成為一種.
覺得我們人數多就能夠影響到多少.
的一種觀念.
我自己在香港教會.
在FourShot也有舉辦祈禱會.
我們也是每晚.
每晚十一點鐘在網上祈禱.
我自己作為教會的負責人.
我也想多些人.
初時有五十多個.
分別越來越少.
也有點擔心.
多些人吧.
但我也覺得.

$^{1921}$想多人的原因不是因為覺得.
這個聚會要多人才好.
也想等姐妹們都可以.
透過祈禱會來一起祈禱.
所以我們純粹可以.
各位教會的同工也可以.
放輕鬆.
祈禱會是要舉辦的.
但也不需要.
享受過程.
教會有十多二十多個.
很忠心祈禱的人.
我覺得是一個很好的福音.
這個聚會.
確實有一班人特別喜歡.
有些人未必有這樣的負擔.
但我覺得教會是需要推動祈禱.
但不是推動祈禱會.
我們要有人有人祈禱.
但不是單單透過祈禱會.
去做祈禱.
會有很多不同的東西.
而且祈禱也是一種.
我們生命裡面的表達.
我們生命裡面有很多不同的表達.
所以我覺得.
其實我們教會是.
聽從上帝的吩咐.
我們願意花時間去祈禱.
這樣就足夠了.
不要太計較後果.
多人是開心的.
但開心的原因.
不是因為覺得很復興.
我覺得復興是一種上帝的工作.
反過來說.
復興不是我們能夠操控的東西.
聖靈的來臨.
自然就復興.
我們教會其實都是.

$^{1961}$忠心做好自己要做的事.
弟弟姐妹的生命.
都是上帝所在重.
我們都是花時間去做撒種工作.
有些聚會讓弟弟姐妹去參與.
但真正叫他們能夠做到.
可能都是上帝自己.
你昨天和今天都有提到.
特別是靈因的神學.
我想看看.
我們華人教會.
不少都有很多這種包袱.
我想問一問.
對於這件事你怎樣看呢?.
有沒有值得欣賞.
又或者給我們引以為戒呢?.
我想靈因.
首先我們要尊重其他不同的.
都是弟弟姐妹.
靈因派教會.
都是我們主內的基督教會.
我覺得是練習上不同.
我們用不同的方法去表達信仰.
我們說有很多不同的動作.
以前的高等教會是.
以跪拜上帝的方式.
耶穌的時間是張開手的.
我覺得我們的表達是不同的.
對於上帝的回應是不同的.
我覺得就算是靈因的行徑.
我不覺得是假的.
但我覺得是一種宗教的表達.
我們會說阿們.
我們會做某種事.
他們會用一種不同的行徑去表達信仰.
所以我覺得這件事.
是不同宗教派對於信仰的表達.
其實都是大家有不同的方法.
我們比較輕和靜.
他們比較多說話.

$^{2001}$他們的好處是.
他們覺得聖靈是比較外顯的.
他們覺得聖靈是地獄邪靈.
雖然不說那些神學.
但他們的好處是.
他們能夠將上帝更加在世界裡面.
我昨天也說.
神不是在閉上眼睛裡面才見到的.
神是在世界裡面所見到的.
所以靈因派教會的信仰.
是比較能夠將上帝在世界裡面.
能夠外顯到.
當然我覺得好處就在這裡.
讓讓頂之輩能夠更加經歷到上帝.
因為世界上的上帝就在這裡.
但對我來說.
我覺得我們都要去承認上帝在這裡.
我們不是用這種語言去表達.
這個上帝就在這裡.
所以我覺得我們仍然是.
都要相信.
都要看得見上帝在世界裡面.
不是只是閉上眼睛的上帝.
而是在世界裡面的上帝.
但我不是去倚靠某種靈因派的神學.
或者語言去強調這件事.
所以我覺得靈因派教會.
這幾十年裡面.
人數增長都是有原因的.
因為他們能夠容易消化.
因為這個世界上就是看到.
就是比較簡單.
有病就祈禱求神.
趕走鬼.
比較容易理解他們的信仰.
但我覺得我們福音派教會.
其實仍然是一樣.
上帝就在這裡.
世界上的上帝仍然是這麼顯然易見.
我們憑著信心去經歷到他的.

$^{2041}$我都是接觸這條題目.
很多時候.
我都是比較在外面信徒的表現.
我不是說全部人都是這樣.
不過現在比較兩極化的看法.
有些信徒很著重聖經知識.
所以凡是查經聚會.
主要學習的.
或者基教部門的東西.
很嚴毒的.
在小組裡面我們都看到.
要讀很多聖經.
但他們又承認.
他們的禱告生活很貧乏.
你明白嗎?.
但另一邊.
另外一些極端.
很著重感性.
追求祈禱的興奮.
或者激情.
但你問他聖經的背景.
上文下理.
他又比較淺薄.
我想問一下.
作為目者.
你覺得怎樣可以平衡.
你剛才說過.
神是很豐富的.
在五個感官中.
可以觸感和成長.
你教我們怎樣可以平衡呢?.
我想我們都很難避免.
可能大家的恩慈不同.
我想都是的.
可能大家有不同的恩慈.
不同的內向性格.
或者自己喜歡的東西.
我覺得不需要去一式一樣.
我們說的很抗拒填鴨式的靈修.
不是每個人都要讀解.

$^{2081}$每個人都要看書.
但不是每個人都要看書.
如果我們把靈性.
單單在書本裡面.
對這一代來說很難.
特別是年青人都不看書.
聖經只是一個tools.
不是一本book.
只是查而已.
我們都需要去接受這件事.
反而是在自己的擅長和興趣方面.
去發展禱告生活.
剛才我都說了.
禱告是一種生命.
禱告不是單單是我們躲藏的時間.
譬如說我們是敬拜隊.
敬拜隊本身就是一種.
親近上帝的禱告生活.
有些是執事去開會.
我覺得開會也是一種敬拜.
開會也是一種我們透過商討事務.
去經歷上帝自己的恩典.
我覺得大家都在不同的生命裡面.
去經歷上帝.
其實我都覺得更加衝動.
開會都需要禱告.
整個會議都是一種禱告的經歷.
所以我覺得整個生命.
其實都是我們一種的禱告.
所以我們反而接近.
我們需要在不同的方面.
去經歷上帝.
當然基本的禱告也有.
不是說泛禱告主義.
什麼都是禱告.
反過來第二件事.
一些比較少.
只是神秘經驗禱告.
但又少讀經的那些.
那就看看怎樣幫助他們.

$^{2121}$就看看怎樣能夠讓他們.
更加顯淺一點去講解.
我也覺得有些事情太過深奧.
有些人也不太想聽.
例如我這些也不太多人聽.
反而會是會聽一些懂得講座的人.
我覺得每個人都有不同的長處和短處.
我覺得大家都可以成為一個好教徒.
當然牧者也很好.
想他更加平衡一點.
多餵一些給他吃.
但都很明白牧者的關心.
Sophia.
Dr.John你也是一個很包容的牧者.
我回顧一下.
不同的學習方法.
個人性.
各樣的環境.
都會令基督徒的成長有不同的取向.
但因為神也是透過五個感應器.
和我們以前的經歷.
和當時的環境.
有沒有機會讀書其實也有影響.
是一個受壓迫.
我去非洲的時候.
看到一些人很喜歡祈禱.
但你叫他讀聖經就有很大困難.
因為連讀書的機會也沒有.
我想問一下你.
因為我們的朋友也有一些電郵來信給你.
你會不會也回答一下我們.
我都挺多的.
所以我隨便地來選擇.
我都未完全去看過.
有些弟兄姊妹問到.
祈禱和禁食的關係.
我都可以說說.
禁食其實是在舊約.
特別是在第二聖殿.
被老回歸之後.

$^{2161}$以色列來賓的時候.
很強調禁食.
禁食本身是一種我們生命的表達.
因為那個人是充滿著仇苦.
所以他沒有飯吃.
沒有胃口吃東西.
所以就要祈禱.
然後更加祈禱到他吃不下東西.
所以禁食本身就是我們的生命和祈禱分不開.
我的禱告和禁食本身分不開.
因為我很傷心.
傷到我吃不下東西.
所以禁食.
更加是來祈禱.
我撕了衣服.
哭著禁食禱告.
所以禁食本身就是這樣的表達.
對於一種哀求.
當然到了耶穌年代的時候.
禁食成為了猶太人的義.
就像馬峰弟弟所說的.
一種敬虔.
施捨禁食和禱告.
成為了一種標準.
一種敬虔的表現.
但是我覺得禁食本身.
不是一種純粹的方法.
不是說我要禁食.
就能夠更厲害.
就像禁食之後.
神就會更加聽我的說話.
然後禁食是我的.
我給你一字.
我的生命和禱告是連在一起的.
我很重視這件事.
重視到一步一步.
我是要禁食.
很專注在這方面.
和禱告.
所以我們除了看成是一種傳統之外.

$^{2201}$也明白禁食的意思.
就是我很重視這件事.
不是因為我重視.
所以我禁食.
而是因為我本來.
不是因為我禁食了.
所以才重視.
兩者是一樣的.
我很重視這件事.
所以我禁食.
不是我特意要做些什麼.
而去祈禱.
所以我想禁食.
是一種教導弟兄姊妹去明白.
有時候我們也要認真去祈禱.
我們就禁食.
這是一種教導.
多於一種方法.
我多回答兩題.
有多一題.
我就多回答兩題.
那就.
信徒最常見三個對禱告的誤解是什麼呢?.
或者你們覺得.
你們覺得信徒對禱告的誤解是什麼呢?.
我問問.
問問朱牧師.
朱牧師的默會日子也很久了.
我有些人聽過.
有些人祈禱很厲害.
很厲害.
要找他去祈禱.
上帝特別聽他祈禱.
所以要找他去學祈禱的方式.
我聽過有人這樣說.
很厲害.
很渴望學到他.
我也覺得有這樣的.
大家都找牧師祈禱.
會覺得牧師祈禱厲害一點.

$^{2241}$所以這正正是中世紀時間.
某些天主教看法.
我找聖人代求.
所以找聖人代求也是一種觀念.
我的禱告是差一點的.
我要找聖人幫我祈禱.
聖經說耶穌是我們的宗寶.
我們的禱告本來就不厲害.
我們也不懂得祈禱.
所以聖經說明我們的禱告不厲害.
其實每個人都不厲害.
所以唯有奉主名求.
和聖靈幫助我們去禱告.
我覺得所謂厲害.
其實是你喜歡祈禱.
有些人是喜歡祈禱的.
我發覺教會有些.
特別是姊妹.
有些姊妹特別喜歡祈禱.
我覺得這是很好的事情.
我不會說是恩賜.
但當你喜歡禱告的時候.
這是很好的事情.
享受禱告的時刻.
享受為人祈禱.
本身是一件很好的事情.
我都覺得.
這麼說厲害的意思就是這裡.
你很喜歡祈禱.
另外.
我也想說一下禱文.
我覺得禱文是.
禱文是.
重要和不重要.
禱文不重要.
因為基本上.
任何禱告都是有價值的.
沒有什麼禱告.
哪些人會禱告得好一點.
和不好一點.

$^{2281}$有意思是.
他本身的神學.
他的觀念.
是對的還是不對的.
但這樣構不成禱告是好還是不好.
很明顯就是.
法律上的禱告.
在聖殿裡的禱文.
基本上是很正的.
但耶穌說這是一個自然自如的祈禱.
這不是禱告.
所以我覺得我們.
又不需要.
去比較禱文.
但我都覺得.
任何禱告.
都是有價值的.
但我都覺得.
上頂之妹.
做崇拜是奉獻的人.
都有禱文.
這是他的預備.
這是更加諒清諒楚的禱文.
這是我很欣賞的.
我覺得禱文是重要.
但不是那一種重要.
不是說我們的禱文.
要雕花就叫好.
但我們有預備是好的.
差不多了.
開咪先.
最後一條.
因為難得你在亞洲.
來到我們這裡.
美國和北美.
如果在一個.
這麼大環境裡.
充斥著一些.
負面消息和負面情緒.
你教我們信徒.

$^{2321}$怎樣祈禱.
經驗了神的同在和神的掌權.
在整個大環境.
負面的消息.
負面情緒下.
怎樣可以做.
這次的禱告.
這個問題很難回答.
不過.
我想大家都沒有一個方法.
不是問怎樣.
怎樣做.
沒有誰能夠教.
怎樣祈禱.
怎樣做能夠回報傷痛.
跟今天講靈修.
都是一樣.
我們弟兄姊妹.
當你不開心.
當你流淚.
其實是一個我們.
很重要的做基督徒的表現.
我們做基督徒.
其實就是為人流淚.
所以我們不是要.
避免這件事.
所以當你去.
面對艱難的情況.
去禱告.
很失望或無力.
這件事.
我都覺得可以看為.
你做的一件對的事.
所以我都不需要.
避免這件事.
當然我們有同路人.
我都覺得.
大家一齊在祈禱會裡.
為這件事祈禱.
其實是很有幫助的.

$^{2361}$我自己在Foodchart裡面.
都有這樣的經歷.
大家一齊去祈禱.
其實是有少少不同的.
當你發覺自己不是自己禱告.
而是大家一齊去祈禱的時候.
大家一齊去經歷一些.
這麼難的事情.
其實是好事.
所以方法就是這樣.
弟兄姊妹大家一齊去祈禱.
多過自己去面對.
大家一齊去.
去面對.
大家一齊去禱告.
這兩晚講座其實都是想大家.
去祈禱.
但更加是.
放膽去祈禱.
沒有誰人對錯.
反而更加重視.
和享受祈禱.
這就是我們在世的生命裡面.
靈命裡面.
一個很重要的時刻.
好多謝.
陳偉安博士.
昨晚和今晚.
和我們分享的內容.
在這裡也提提大家.
昨晚的講座.
今晚的講座.
我們已經錄影了.
我想一兩天後.
就會在建道中心的網站裡.
你可以重聽.
又或者是你介紹給.
其他的弟兄姊妹有興趣的.
都可以重聽.
完全是按大家的時間.

$^{2401}$方便去做.
我希望這兩晚的講座.
大家可以多多的.
幫助大家.
對祈禱有更深刻的認識.
和知道我們在神面前.
是很微小.
和很有限.
我們盼望以後.
還有機會.
藉著建道中心.
能夠帶給大家.
更加高質素的.
一些屬靈生命有關的講座.
如果你對於.
昨晚和今晚的講座.
有感動.
我們歡迎你.
在經濟上支持我們.
以致我們可以繼續.
去承辦這些活動.
那個奉獻的機會.
有兩個方法.
一個是用PayPal.
如果你去到我們的網頁.
第一頁裡面.
紫色的那個button.
奉獻支持神學教育.
一點進去.
就可以去到PayPal.
立即就會收到奉獻的報稅收款.
如果你喜歡用支票.
也可以抬頭.
ABSCC.
寄去我們中心的地址.
也在我們網站的首頁.
下面就可以找到.
但是如果是這樣.
我們需要30元.
或者以上的奉獻.

$^{2441}$我們在年底.
就會發奉獻的收據.
再一次提醒大家.
如果今晚.
你還有問題想問.
Dr. Chan.
仍然可以用.
我們的電郵地址.
寫給他.
johnchan.workshops.
記得有個S.
workshops.202006@gmail.com.
我都請Dr. Chan.
盡可能.
花用他一點點的.
私人時間.
個別回答你們的問題.
希望今次的聚會.
能夠對大家有幫助.
在這個時候.
我們就和大家一起.
做一個祈禱.
結束今晚的聚會.
請大家低頭.
我們做一個祈禱.
親愛的天父.
我們多謝你.
讓我們昨晚和今晚.
有陳偉安博士在我們中間.
分享到祈禱這件事情的重要性.
願意我們的靈命.
能夠被滋潤.
我們祈禱能夠和你建立更好的關係.
以致我們在面對世界.
面對我們周圍.
發生的事情.
我們懂得怎樣去依靠你.
能夠去度過.
安然度過這些艱難的歲月.
求主恩代我們每一個人.

$^{2481}$按著我們每一個人的需要.
每一個人困難的情況下.
你都能夠給我們有一個喜樂的人生.
讓我們有一個能夠活出.
你的門徒的樣子.
求主恩代.
也祈求你祝福陳醫生.
他下星期就整家回到香港.
保守他的旅程.
保守他健康回到香港.
以致他很快就能夠適應那邊的生活.
求主恩代帶領我們的禱告.
乃是奉耶穌基督的命求.
阿門.
再一次多謝大家今晚的來臨.
以後請大家去我們的粉絲專頁.
隨時都會看到我們最新的發展.
多謝大家.
晚安.
多謝陳醫生.
多謝各位.
辛苦了.
多謝主恩代.
多謝朱牧師.
多謝後面台前幕後的弟兄姊妹.
我們給一支多謝.
多謝啦.
\newpage



\section{}
\label{sec:fFkCm0QGBPw}
\textbf{疫症大流行下的神學反思 --- 第一講:《疫症大流行下的禱告神學》}
\newline
\newline
連結: \href{https://youtube.com/watch?v=fFkCm0QGBPw}{\texttt{https://youtube.com/watch?v=fFkCm0QGBPw}} ~~~~ 語音日期: 2020-06-19
\newline
\newline
\hyperref[sec:yb30yQHiYdM]{\small{< < < PREV SERMON < < <}}
~
\hyperref[sec:index]{\small{[返主目錄]}}
~
\hyperref[sec:P0Y2lvzICsM]{\small{> > > NEXT SERMON > > >}}
\newline
\newline
$^{1}$各位大英姐妹你們好,今天是6月11日,我們籌備了很久的疫症大流行下的神學反思講座正式開始.
歡迎大家來到我們當中,這個講座有三個機構一起合辦.
先由加拿大建道中心發起,然後是士嘉堡華人宣道會和城北華基教會一起邀請陳偉安博士來做分享.
大家都知道士孫和城北華基是多倫多最大的兩個教會,不需要介紹,大家都聽過.
反而加拿大建道中心有部分大英姐妹都不太認識.
我簡單半分鐘介紹一下,我是加拿大建道中心的總幹事,陸昭明牧師.
我們是一間神學教育機構,我們提供了不同類別的神學教育科目.
加拿大建道中心有一個特色的地方,就是和香港的建道神學院有非常密切的關係.
香港建道神學院的老師經常都會來加拿大多倫多這個地方,在我們的建道中心講課.
香港建道神學院或許一直以來都在溫哥華相當受人注目.
因為很多牧師和校友都在溫哥華,倒轉多倫多就比較少.
誰不知最近2020年疫情之後,整個形勢都轉變了.
反而多倫多裡面,香港建道神學院的老師都曝光很多.
我們盼望在現今和將來,香港建道神學院的老師會在多倫多更多的出現.
他們都會在建道中心的聯絡和大家見面.
有幾件事想和大家交代.
第一件事就是我們講一下今晚的程序.
我們介紹完之後,會給事先的負責人朱牧師講幾句歡迎的說話.
大概7點35分左右,會把時間交給陳偉安博士.
陳博士的講題會一直到9點05分左右.
到9點05分左右,朱牧師和我會代表其他弟兄姊妹發幾條問題給陳偉安博士回答.
內容是有關今晚課題的內容.
為什麼會這樣呢?我想簡單交代一下.
在這樣的情況下,無論如何都不能滿足所有人的要求.
所以我們就打算朱牧師擔個大橋,做了這樣的工作.
但是大家弟兄姊妹也可以用另外一個方法.
如果你願意發出問題的話,可以寫email到現在這個電郵地址.
專門收取弟兄姊妹對陳偉安博士的問題.
今晚回到家後,會將這些問題匯合,然後發給陳偉安博士.
明晚陳偉安博士會有一段時間回答.
請大家特別留意我們的運作.
因為實在人太多,我們不能完全公開所有人的問題.
所以唯有一個不是最理想的方法來處理一些問題.
我簡單介紹一下陳偉安博士的一些經歷.
或者有很多弟兄姊妹在facebook認識陳偉安博士.
有些人未必認識.
陳偉安博士是香港建築神學院神學系副教授.
他也是建築神學院研究課程的副主任.
陳博士留學德國,大家可能都知道.
德國的神學派系和美國的神學派系很不同.

$^{41}$他的專長是禱告神學.
換句話來說,今晚和明晚他所講的專題.
其實我想不到有更好的講員可以講.
陳博士在德國留學時是專攻祈禱神學.
今年是陳博士的安息年.
雖然他說是安息年,但忙到快要飛起了.
沒可能會安息了.
他們和我們講了這個講座之後.
下個星期就飛回香港,完結了他的安息年.
又回到香港服侍那邊的弟兄姊妹.
陳博士現在在德國達拉斯和大家會面.
不是在香港,也不是在多倫多.
簡單介紹了他,有什麼遺漏的就請陳博士自己說.
這個時間我們請事孫朱牧師說幾句歡迎的說話.
為什麼會這樣運作呢?.
因為今晚的題目是事孫比較有興趣.
明晚的題目是成不華基教會的弟兄姊妹比較有興趣.
所以我們今晚先請朱牧師說幾句歡迎的說話.
把時間交給朱牧師.
(朱牧師說話).
Hello,大家好.
我是朱志輝牧師.
在這裡代表士嘉堡華人宣導會歡迎每一位.
今晚來參加這個講座.
我們這裡是晚上,或者你身處的地方可能是白天.
但是無論是白天還是晚上.
我們都很感恩,一同在神的話語學習.
相信在疫情之下,我們弟兄姊妹多了祈禱.
也在信仰上有不同的反省.
前一陣子我收到一個信徒的電話.
他打來說.
朱牧師,疫情這麼嚴重.
我真的不敢出街,我的身體又不好.
我真的很怕感染了病毒.
我有祈禱,不過我還是想請你為我禱告.
聊著聊著,這位信徒說.
朱牧師,疫情這麼嚴重.
很多地方都有.
耶穌是不是快要回來了?.
不知道怎麼辦.

$^{81}$我知道自己還有很多罪.
我還未準備好回去見神.
在裡面,我和這位信徒來禱告的時候.
在禱告裡面,我們從地上的事轉到天上的事.
我們初時關心肉體健康的問題.
但是當我們思想的時候.
一個信主的人.
其實我們更加關心我們屬靈生命的健康.
在禱告裡面,我們去問與神的關係.
在疫情大流行下作禱告神學的反省.
祈禱和信仰的關係.
事不宜遲,我們將時間交給John.
我們有請陳偉安博士.
大家好,各位觀眾好.
很開心在Zoom和Facebook跟大家分享這個題目.
我也是第一次在Zoom講講座.
因為今年是安息年.
所以沒有什麼機會有講座或教學.
跟大家講一個疫情下的禱告神學.
這個題目可以說有兩方面.
主要我們會講到禱告神學這個題目.
我們一起來反思禱告這個題目.
但也有一定程度的處境.
因為現在大家無論身處在什麼地方.
都面對疫情.
所以這是一個很好的環境讓我們思考這個題目.
這個講座本身是我自己教開神學課程的一課.
叫做禱告神學.
所以今天所講的是整個課程的濃縮版.
再加上一個特別版.
因為是一個疫情下反思禱告的題目.
讓我先開啟我的電腦分享屏幕.
今天我會用一個Presi來跟大家一起看.
今天我們講疫情大流行下的禱告神學.
我想大家都會問為什麼叫禱告神學.
其實禱告神學是一個很特別的題目.
因為我們都沒有想過禱告可以是一個神學.
因為我自己博士論文的研究.
是有關神學家Karl Barth的禱告概念.
當我回到香港的時候.

$^{121}$開始在香港的建築教這一課.
其實這一課是一課神學課.
不是我們一般理解的純粹一個淑女的題目.
所以這個也說一下什麼是禱告神學.
這個題目不是叫禱告學.
因為我記得很多年前.
我回大陸教書的時候.
教這一課禱神學的時候.
其實都有點誤會.
因為當時有些同學打筆記的時候.
打漏了一個神字.
打漏了禱告學這兩個字.
他們想一個教禱告學的人是什麼人呢.
可能就是一個宿嶺的前輩.
然後教你怎麼祈禱.
誰知道是一個年輕人.
但其實這個很不同.
禱告學可能是一個.
也有的.
如果你搜尋網上.
也有人說禱告學這個字.
禱告學可能是一個教你怎麼祈禱.
教你怎麼實踐禱告的生活.
禱告神學是一個神學題目.
它是真的關乎到神學的思考.
特別是有關禱告的範疇.
雖然大家都很懂.
大家都很明白什麼叫做禱告.
但其實發現一些在我們心目中.
一些很簡單很顯淺的題目.
反而是讓我們很容易會有一個.
太快的take it for granted.
太快會覺得這件事就是這樣.
沒有什麼可以再去思考的地方.
這個正是我們今天想嘗試去重新反省.
我想大家沒有什麼人不懂禱告.
但你也懂得禱告.
可能你剛剛初信主決志的時候.
也有人教你怎麼去祈禱.
大概就是親愛的天父.

$^{161}$然後就說出你的代求.
然後就去奉主名求阿門.
禱告的方法其實我想大家都知道.
所謂的方法.
甚至聖經裡面也會教我們怎麼去祈禱.
耶穌基督教我們去祈禱.
但我們也會去嘗試想.
究竟禱告和整個神學的關係是怎麼樣.
因為如果我們說禱告和神學的關係.
神學不是一個很教義式的東西.
不是我們所說的創造論 基督論 救贖論 禱告論.
其實我們嘗試去思考禱告.
其實我們不是單單純粹去理解一個禱告論的看法.
而是說整個禱告對我們整個的信仰.
特別是我們和上帝之間的關係的思考.
當然我們發覺有很多不同的禱告定義.
但如果你說禱告是我們最基本人和上帝的對話.
或者我們和神的連結.
禱告神學不是單單關乎我們理解禱告是什麼.
而是說我們整個的信仰裡面.
我們怎麼去理解禱告這件事.
禱告這件事和我們整個的信仰.
整個和上帝的關係最基本是怎麼樣.
所以這不是單單一個know-how.
不是單單一個如何去做或操練的問題.
所以今天沒有任何叫大家去操練祈禱的東西.
反而是多一點讓大家有一個很好的反省.
讓我們去反省一些我們信仰裡面最基本的東西.
大家都很早就已經是懂得的東西.
禱告在我們整個的信仰裡面.
特別是今天我們面對這麼嚴重的情況.
整個世界都面臨著一個很大的難關.
無論是在疫情裡面.
或者是在世界大事的政治裡面.
我們面對著這個世界的時候.
我們怎麼去理解我們的禱告.
這是我們今天嘗試去思考的問題.
我們從一個人和上帝的角度.
當然上帝是首先和人去建立關係.
但是我們從一個人的角度來出發.

$^{201}$人去禱告上帝的時候.
整個的這件事情是什麼事情.
我們嘗試用一個這樣的角度去理解禱告神學.
另一個就是禱告也不是單單純粹靈修.
雖然大家不知道說不說.
通常我們會覺得你去基督教書店買書的時候.
你會發現很多禱告的書都有.
一本大概八多頁不是很厚的書.
都是一些靈修的小品.
一些去談及到我們一些屬靈的生命.
我們禱告的生活.
我們都很容易很快地覺得.
禱告是一種關乎於我們一種靈修的範疇.
所謂的Spirituality.
我要說這是其中一個部分.
我們去嘗試去理解禱告的時候.
靈修或者整個的靈修學.
是我們其中一個很重要的部分.
但是不是等於的.
當我們去思考禱告的時候.
這個禱告不是單單純粹和上帝的關係的建立.
或者靈修這麼簡單.
其實更加多.
靈修只是某部分有關禱告的事情.
起碼是我們去為我們所需要的事情祈求.
為世界的事情代求.
這些全部都是關乎於禱告.
所以這未必是純粹和靈修學的角度去理解.
所以我們嘗試不要太快地覺得禱告學.
或者禱告神學.
是一種純粹屬靈生命建造.
或者是一些靈性的東西.
或者這麼說.
靈性我們都嘗試用一個更加廣泛的定義去定義.
靈性不是一種純粹我們以往所理解.
那種純粹是一種屬靈的生命.
或者是屬靈生命的建立.
所以今天我們或者明天晚上.
我們嘗試用一種比較神學的角度.
去反映我們那種靈性.

$^{241}$或者禱告的生活.
或者禱告的活事.
所以我們今天嘗試去反省.
大家有一個critical thinking.
來思考一些神學的topic.
當然我們最後都會去反省.
今天我們在疫情之下.
我們怎樣去回應這個世界.
我們怎樣去面對我們的生命.
這個都是我們回到最後一個很重要的topic.
這個都是我們今天嘗試得到的東西.
我們嘗試來說.
今天我們會說禱告生命.
這個就是我最後的總結.
我們所說的生命的禱告.
這個不是純粹一個屬靈的很大的dragon.
我們嘗試說出來.
我們整個的生命和我們的禱告是很大的關連.
這個不是純粹一個屬靈的口號.
而是一個有意義的說話.
我們的生命和我們的禱告的關係是怎樣呢.
這個就是我們今天嘗試去做一個總結的訊息.
我們會從一些基礎開始.
我們都會花大約20分鐘的時間.
嘗試去理解一個禱告的歷史.
因為我想禱告神學或歷史.
或者我們的反省都很重要.
我們反省不是沒有了一些很客觀.
和大家去擴闊我們眼界的歷史.
所以我們都會花一段時間去認識.
我們所謂的整個禱告歷史.
我們究竟從舊約到新約.
到初期教育會中世紀.
500年前的中華教育會.
到今天我們華人教育會的傳統.
我們怎樣去看得見整個禱告的發展.
因為我們發覺其實是很大眼界的.
當你來研究一下整個基督教信仰的禱告發展時.
其實是有很大的不同.
今天我們會怎樣祈禱.

$^{281}$我們會合手 閉上眼睛.
來祈禱 奉主命 叩眼門.
但你會發覺這個禱告的姿勢.
其實是很不同的.
當然我們說的不是單單是姿勢的問題.
而是從這個姿勢裡面.
我們發現其實我們不同的年代.
從舊約或耶穌年代裡面.
他們的禱告其實是和今天不同的.
你去看聖經的時候.
例如問耶穌是怎樣祈禱的.
耶穌是怎樣來禱告的.
他似乎不是閉上眼睛的.
聖經裡面似乎沒有說到耶穌是怎樣祈禱的.
他反而是舉目望天.
張開手來祈禱.
重點不是單單說我們要跟隨哪一個姿勢.
而是從這個姿勢裡面.
我們發現其實在舊約到新約.
歷史裡面其實有一個進程.
我們要去明白.
今天我們很習慣.
從第一天決戰之後.
決戰之後的禱告.
你會發現其實是不同的.
跟聖經裡面所說的.
所以今天我們嘗試花時間去認識一下.
從聖經到歷史.
我們過往如何看得見禱告的改變.
今天我們會用WhatsApp祈禱.
很多時候我們在頂尖會裡面.
我們習慣有人為我們祈禱.
我們就會按一些emoji出來.
當作是祈禱.
我們發現其實我們的禱告有很多不同的表達.
我們說其實禱告裡面都可以有很多不同的表達.
大家看這幅圖.
這幅圖裡面好像伊斯蘭教徒的祈禱.
但其實這也是一個很猶太教.
很舊約裡面.

$^{321}$或者耶穌年代裡面的禱告姿勢.
五體投地.
整個身體都去到地面.
當然有一個意思.
對於上帝的傳言的俘服.
今天我們所說的合掌的祈禱.
都是幾百年前在歐洲裡面很強調的方法.
所以你會發覺禱告是有一個進程的.
我們作出神學思考的時候.
我們要去理解我們今天的禱告傳統是怎樣.
我們對於禱告的理解.
其實在歷史上都有不同的改變.
所以我們就嘗試花一點時間去認識一下.
我們從這個教會開始.
或者簡單問一個問題.
如果你去問.
究竟整本聖經裡面.
或者人類歷史裡面.
第一個的禱告是什麼時候.
我們說人開始向上帝禱告.
這件事是什麼時候開始的呢?.
你會發覺當你認真去查聖經的時候.
可能有很多不同的答案.
可能你會覺得在這個創世記.
很明顯在創世記裡面.
第三章裡面.
如果我們將禱告定義為.
人和上帝來說話的時候.
你會覺得.
起碼你會看到.
在聖經裡面.
第一句說話.
人和上帝說話的時候.
就是在這裡.
就是阿當對著耶和華上帝說話的那句說話.
你會發覺很奇怪.
其實聖經裡面從來都沒有定義過什麼禱告.
聖經裡面似乎一開始就assume了.
你明白什麼叫禱告.
聖經似乎就已經開始去描述一些人和上帝對話的經文.

$^{361}$他沒有嘗試為禱告作出任何神學的定義.
或者是一個最基本的共識定義.
他就去記載了很多不同的聖人物裡面.
和上帝的說話.
而第一句就是這句.
就是當人犯罪之後.
就是當阿當和耶和華犯罪之後.
阿當就被上帝和耶和華上帝去問.
Where are you? 你在哪裡?.
阿當根據聖經的記載.
去和耶和華上帝說話.
其實就在第三章第八節裡面.
他說:我在園中聽見你的聲音我就害怕.
因為我的赤身腦體在那邊藏了.
如果我們理解這句說話.
是我們人和上帝第一個的禱告的時候.
這個禱告是出於一種的協救.
就是上帝的一個明知觀問.
上帝去問阿當你在哪裡.
你知道這個是明知觀問.
上帝不會不知道阿當在哪裡.
不過他仍然來尋找阿當.
當阿當來犯罪的時候.
耶和華上帝來尋找他.
用一個問題來想阿當自己去回應.
親口說話回應上帝.
所以這個非常有意思.
來自於上帝主動的提問.
阿當親口的來和耶和華上帝說了第一句說話.
所以我們看到根據聖經記載裡面.
他記載的第一句對話是一種充滿著.
罪疚和人的軟弱.
和上帝對於人的愛的尋找.
所以這個很有意思的說話.
我們的禱告會不會建基於這種基本的關係.
就是上帝去尋找我們.
而人去回應上帝的愛和恩典的時候.
他就發出我們見到聖經裡面的第一句對話.
不單單是一種上帝自己的 monologue.
不是上帝自己的單獨的說話.

$^{401}$而是人也開始去回應上帝.
很快的,因為今天不是十堂課.
所以我們很快就會去到下一個階段.
大家有興趣可以來報一下這個課.
我們看到第二段經文其實也挺有趣的.
今天特意找了一段有關瘟疫的經文出來.
和我們Viva就有些關係.
就在這個民數記裡面.
我們發現其實往往我們見到.
當然我們說耶和華上帝其實在結局裡面.
都出現了很多次以瘟疫來作為一種.
對以色列人或者外邦人或者地上的人的懲罰.
而這種瘟疫往往是帶來禱告.
你見到下面那段裡面.
摩西就對著耶和華說話.
我用這個經文出來其實是想說什麼呢.
其實在法國舊約的早期裡面.
特別是在以色列文歷史裡面.
在摩西五經的時間.
或者是還沒有成為一個以色列帝國之前.
聖經裡面的禱告往往是一種很對話式的禱告.
一種很簡短的對話.
好像朋友之間的對話.
無論是阿伯拉罕和耶和華的對話.
一種很簡單的朋友的說話.
或者是摩西和耶和華說出來.
都是一種大家聊天的感覺.
法國希伯來文裡面的禱告是很後期的.
反而更多的就是很簡單的說.
摩西對耶和華說.
或者是去問 或者是去呼求.
沒有一個很嚴格的 因為是一個很宗教的字眼叫做禱告.
Tafina.
是沒有的.
是一個很朋友形式的日常生活的對話.
這樣說話.
這個當然是一個禱告.
他只是和上帝展開對話.
所以我們發現在初期裡面.
我們看到人和神的對話是很日常生活的.

$^{441}$是一種很日常的互動形式的說話.
不是很長的宗教禱文.
也不是很禮儀形式的禱告.
但這個事情就改變了.
當以色列民來建聖殿的時候.
你會發現出現很大的轉變.
我們就出現了一個Tafina.
一個禱告的字眼.
真正有一個希伯來文叫做禱告.
而所羅門王在列王記上第八章裡面.
你會發現那個獻殿祈禱.
在列王記上第八章裡面.
是一個很長的禱文.
所羅門王就出了一個非常長的禱告.
當中其實也承載著一些神學思想.
就是說 上帝果真住在地上嗎?.
看啊天和天上的天都不足你居住.
何況這殿呢?.
似乎所羅門王是要強調.
這個聖殿也不是說他可以住得完.
因為上帝不是在這個殿裡面.
而是在天上的天裡面.
是遠超這個聖殿裡面.
不過你會發現很特別的是什麼呢?.
就出現了一個所謂的聖殿神學.
就是說 以色列文是要透過這個聖殿.
來向上帝祈求.
他說什麼呢?.
我們說 尼泊人和以色列文.
向此處祈禱的時候.
求你在天上到水廳 水廳而生.
整個禱告的過程是這樣的.
以色列文是需要在聖殿裡面.
或者向聖殿裡面祈禱.
然後這個聖殿就成為了一個禱告的媒介.
一個medium.
將這個禱告向上達天庭.
去到天上的上帝那裡.
所以你看到是一個很重要的程序.
以色列文不是隨便禱告的.

$^{481}$他需要來到去看著.
或者在這個聖殿裡面.
向著這個聖殿祈禱.
從而這個禱告才能去到耶和華上帝那裡.
這個就是以色列文裡面.
一個很重要的禱告神學.
當中你會發現.
這個所羅門王的禱告.
都包括了一個瘟疫.
如果國裡面有饑荒瘟疫的時候.
我們一起來禱告的時候.
我們向著這個聖殿去舉手祈禱的時候.
求你在天上來到去赦免我們.
大家有沒有留意到.
天上的上帝和以色列文.
向著這個聖殿的禱告的位置距離.
上帝在天上.
不過他會隨聽這個在聖殿裡面的祈求.
所以這就是為什麼我們今天.
如果你去過以色列的話.
今天猶太人仍然會向著哭牆來祈禱.
哭牆就是被拆毀了的聖殿.
所以今天猶太人仍然保持了這樣的習慣.
要向著聖殿.
雖然聖殿沒有了.
不過仍然要對著牆來祈禱.
因為這個牆就是神人禱告很重要的媒介.
我們很快的幾分鐘就講完整個禱告.
我想說這是一個聖殿神學.
我們都看到一篇文章.
關乎於這個歷代志下的經文.
這也關乎於講到瘟疫.
當海倫王獻殿完畢之後.
他就說如果天閉塞不下雨.
有蝗蟲腐蝕.
甚至乎有瘟疫流行的時候.
他說什麼呢.
若是自卑禱告尋求我的面.
寸離惡行我必從天誰聽.
賜給你的罪 依至這地.

$^{521}$所以我們再次看到當瘟疫出現的時候.
下面的人我們是需要自卑禱告.
尋回上帝.
我想這是我們禱告裡面很基本的本質.
禱告不是單單關乎於一種忠誠的人和神的話.
而是關乎於人的自卑.
一種像是亞當犯罪之後.
就是發現自己的問題.
來向上帝自卑的很大的差距.
所以禱告的本質就是這樣.
我們看到我們的軟弱.
然後向上帝祈求.
我們很快的 因為今天要說很多東西.
我們看到到了一個新的時間.
我們有幾樣東西要說.
第一個就是主通文.
主通文裡面有一個名字.
就是耶穌所教導的禱告.
你會看到其實我們今天說不完.
你會看到其實頭三個禱告裡面.
願人穿領名為聖.
願你國降臨.
願你旨意成就.
其實我們說在地上.
這個形容不是單單是你的旨意成就那一句.
其實是三句都是.
願你的名為聖.
願你的國度降臨.
願你的旨意成就.
都是在地上.
而不是在天上.
所以我們禱告不是離地的.
不是純粹關乎天上的事情.
而是地上的淺行.
我們的禱告和我們在地上的行動是一致的.
我們去祈求上帝的名為聖.
我們也在地上有這樣的生活.
我們去祈求上帝的國度降臨.
我們也同時去活出天國的價值在地上.
我們去尋求上帝的旨意.

$^{561}$也是活出上帝的旨意.
所以我們的禱告不是純粹屬靈的事情.
而是我們基督徒生活的一個很重要的習慣.
所以下面就是這樣.
下面有另外四個祈求.
一個叫做We Prayer.
是關於我們的禱告.
我們的需要.
我們的飲食.
我們的債.
我們面對的試探.
我們的空虛的預見.
特別想說的第一句.
就是飲食的那個字.
這個字好像很簡單.
但是在原文裡面其實也不簡單.
因為當中這個字很難解的.
前面的aton的字就是bread.
就是麵包的意思.
但是後面的那個字.
這個appelation的那個字.
其實是一個非常難解的字.
因為其實是沒有其他的相同的字.
有法整本生物裡面.
或者整本生物裡面.
都沒有一個相類似的字.
所以想說找不到這個字是怎麼解的.
你看到有很多不同的可能性.
可以解作一個extremely rare的word.
就是一個非常罕有的字.
我們找不到它的意思是怎麼解.
可以解作daily.
每天.
一個necessary.
一個需要.
一個following day.
一個我們之後將來的需要.
所以我們有很多不知道的意思.
什麼叫做所謂的麵包.
其實有很多不同的意思在當中.

$^{601}$500年前路德就有這樣的意思.
就這樣解釋.
他嘗試將這個daily bread.
解作一個很深層的意思.
他就說這個所謂的daily bread.
不是單單關乎於飲食.
而是關乎於我們整個生命的需要.
所以發覺不單單是我們的營養.
身體.
更加是我們的衣著.
必需品.
和人的和平共處.
交往.
經濟買賣.
論社政策等等.
所以我們去求我們的daily bread.
這個daily bread不是純粹講飽肚.
飲食的東西.
是我們整個生命的需要.
我們關乎於我們的coronavirus.
我們國家的安穩.
國家的公義.
國家的政治.
或者是政策.
都是關乎於我們daily bread當中.
這個就是路德的詮釋.
所以包括那個字.
就是瘟疫的字.
保護我們在各方面的入侵.
無論是水災.
火災.
或者瘟疫.
甚至乎戰爭一樣.
我們都是去求天父上帝.
給我們日用的所需.
所以這個就是很明顯.
我們的daily bread的意思.
這個是我們去求的.
我們需要為我們日常所需的生活.
和整個群體.

$^{641}$整個社會.
國家的需要.
都是去祈求.
第二段我們看到馬太和路加福音的經文.
你會發現有一個比較.
左邊是馬太福音的版本.
右邊是路加福音的版本.
這個就是耶穌講到有關禱告的教導.
你們祈求給你們尋找就尋見.
求不求你們開門.
這個也是這麼說.
耶穌說你們凡是祈求的話.
必定會讓你尋見.
他說什麼呢.
凡是兒子求餅.
怎麼會給石頭呢.
這個這樣的比喻.
那時候好像說什麼.
我們說當我們求餅的話.
怎麼會不是給餅呢.
好像有這樣的感覺.
當你求A.
上帝就會給A來.
當你求B.
上帝就會給B來.
似乎有這樣的感覺.
但你會發現其實不是.
上帝只不過是說當你求餅.
怎麼會給你石頭呢.
他沒有說給餅來.
他沒有說求A有A.
他只不過是說求餅.
就不會給你一些差的東西.
有一點很奇怪.
他說何況你蒙在天上的父.
豈不會把什麼給你.
其實兩個不同的經文的版本.
是有不同的答案.
左邊的版本說什麼.
天父豈不會把名字給你.

$^{681}$就會給你好東西.
什麼是好東西我們不知道.
所謂的好東西.
右邊的路加怎麼說.
何況天父給什麼給你.
就會給你聖靈.
所以這兩個版本裡面有很不同的答案.
馬太所說的我們祈求.
所謂上帝對我們祈禱的回應.
首先不是求A就有A.
什麼東西給你好東西.
總之是好的東西.
不會是差的東西.
是好的東西.
右邊的是什麼.
就會給原來好東西是什麼.
原來就是聖靈.
你會覺得有點奇怪.
你會覺得聖靈.
你不明白為什麼要給我們聖靈.
所以我們就去想.
究竟這個禱告裡面.
所謂的好東西.
所謂禱告的答案.
原來根據路加所說.
就是聖靈的自己.
所以我們就會去問.
究竟這個聖靈.
這個好東西.
是一個什麼回事.
而這個答案.
我覺得就是出現在這個樣本裡面.
其實是一個很特別的記載.
第一句是什麼.
就是這個樣本第四章裡面.
耶穌去找井旁婦人的故事裡面.
耶穌說.
「時後將到你們拜父.
也不是在山上.
也不在猶太郎」.

$^{721}$「那真正拜父的.
以用心靈和誠實來拜他」.
似乎耶穌回答了一個.
有關敬拜和禱告的問題.
記得那個神學.
聖神學.
溪國裡面.
他們嘗試用聖殿作為.
人和神的關係的鐘堡.
我們要藉著這個聖殿.
才能夠將我們的禱告.
傳到上帝那裡去.
但是耶穌基督的彌臨.
說了什麼.
就是說原來.
耶穌的出現.
成為了我們和上帝之間的鐘堡.
我們再不需要去.
今天基督徒再不需要.
看著聖殿方向.
也不需要坐飛機到遼西里.
去到聖殿裡祈禱.
我們是奉主名求.
我們只需要在耶穌的名下.
我們就能夠.
將我們的禱告傳到上帝那裡去.
所以不是在山上.
也不是在猶太郎裡面.
我們不需要靠著一個聖神學.
為什麼.
因為耶穌基督就是那個流動的聖殿.
祂就是真正讓禱告能夠成就的鐘堡.
所以我們是.
所以我們發覺.
那個靈也是.
所謂聖靈.
作為基督的靈.
祂也成為了我們一個很重要的媒介.
我們在聖靈裡面祈禱.
我們是奉主名求.

$^{761}$我們向我們的天父祈禱.
今天我們不夠時間講神學的東西.
我們簡單地說.
我們是一個三個關係.
我們是向著天父祈禱.
我們是奉主名求.
我們在聖靈裡面去禱告.
這個也說到.
靈和耶穌基督都成為了我們很重要的禱告方法.
和回應.
所以你看到.
在這個方法裡面.
連續在這個罪案晚餐裡面.
14 15 16章裡面.
都說到奉主名求的概念.
凡奉我的名的禱告.
無論求什麼.
都必成就.
今天所說的.
上帝必定會隨聽禱告的經文是來自這裡.
不是剛才所說的求餅的比喻.
而是說我們在基督耶穌裡面.
我們在主裡面.
我們的禱告是能夠去成就的.
所以這就是我們今天很重要的基礎.
我們是因為基督耶穌和聖靈的緣故.
祂成為了我們禱告的答案和回應.
這個我們稍後會再說下去.
今天我們就說快一點.
因為今天很長.
這裡是剛才的聖經基礎.
我們會發覺.
在初期教會裡面.
我想抓一個重點來說.
就是這個不住的禱告.
在《貼薩諾尼加前書》第517節裡面.
不住的禱告.
不是那個意思.
其實不是一個這麼簡單的意思.
是一個真正不停止的禱告.

$^{801}$就是adiolatos的禱告的意思.
是一個不會停止的禱告.
一個不會間斷的禱告.
如果我們認真去看待這段經文的時候.
什麼是一個不停止的禱告呢?.
不知道你今天試過多長的禱告.
可能試過通宵祈禱.
可能試過連續三天的祈禱.
其實禱告總是會停止的.
總是會說出來的禱告會停止的.
因為你沒有那麼多東西可以說.
所以我們問.
什麼叫做不住的禱告呢?.
初期的教會裡面.
他們就嘗試這樣的詮釋.
所謂的不住的禱告.
我們可以看看一些例子.
第一個就是一些沙漠教父的撫修傳統.
一群初期教會裡面的.
第四,五世紀裡面的沙漠教父.
他們嘗試這樣實踐不住的禱告.
就是一個不停止的禱告.
他們嘗試不斷地重複一些禱文.
不斷地重複一些十分鐘左右的禱文.
在沙漠裡面不斷地晝夜地禱告.
因為他們嘗試這樣覺得.
不住的禱告是什麼呢?.
就是我們需要不停地.
不停地去嚎啞地去附中這些禱文.
這個就是一些的撫修主義裡面.
這樣的傳統.
嘗試用這樣的方法去實踐不住的禱告的意思.
這個就是一個叫做集體修道主義.
一個字就叫做Aura Elabora.
是什麼呢?.
就是說當教會出現一個集體修道會的時候.
就是說一個人當他願意進入一個修道院裡面.
去做修道士.
剃光頭.
穿著一件修道袍做修道士的時候.

$^{841}$其實整個的一種的修道院生活.
其實本來就是一個禱告.
所謂Aura Elabora.
意思就是一個禱告與工作的意思.
Aura就是禱告.
Elabora就是一個工作的意思.
對於一個集體修道主義裡面的修道士來說.
當他進入到修道院裡面.
來到所謂出家的時候.
當他要成為修道士的時候.
其實這種的生活本身就是禱告.
當他在修道院裡面嘗試去種土.
去讀經.
在修道院裡面去做清潔工作.
或者去耕田.
整個修道院的生活其實就是一種禱告.
他打掃的時候是禱告.
當他讀經的時候是禱告.
所以今天可能大家聽過一個叫做.
一個叫做禱讀.
一個叫做Lectio Divina.
就是所謂讀經的時候的禱告.
其實來自於這裡.
當他們嘗試讀經的時候.
是一個叫做禱告的心靈.
嘗試用一種禱告的態度來讀經.
所以出現了這樣的情況.
整個的生活成為了一種向上帝的禱告.
然後就是一個阿桑太學派的看法.
這個來自於這個俄里根或者格里緬的看法.
他就說什麼呢.
如果有些人嘗試去定期禱告.
譬如說三點鐘 六點鐘.
就是禁止他們祈禱.
他說這個真正的基督徒.
他說就是整個生命都在禱告.
整個的生命就是和上帝的一種連結.
發覺其實這種說話.
不是一種單單說一下就算的屬靈的Dragon.
而是他嘗試整個的Christian life.

$^{881}$整個的生命就是一種祈禱.
所以發覺其實我們看到.
開始有很多不同的看法.
他們說我們整個的生活.
我們整個的生命.
都是一種向上帝的禱告的態度.
我們很快就去到這個Reformation的時間.
當然其實500年前的Reformation.
我們發覺是很大改變的.
讓很多時候中世紀裡面.
一些很神秘主義的靈修的傳統去改了.
我們再不是天主教形式的那種物觀.
或者很神秘主義的靈魂出竅的祈禱.
現在我們回到一個最基本的禱告的原則裡面.
所以我們禱告生活其實像Reformation的時候開始.
路德嘗試強調信心和禱告的關係.
他說這是一個很舊的德文.
Glaube 信心的意思.
信心就是等於禱告.
所以對路德來說.
我們對於上帝的相信.
是帶著一種懇求的態度.
禱告不是單單是跟上帝聊天那麼簡單.
信心也不是理性上的相信那些東西是真的.
信心對路德來說是一種真正的倚靠.
和一種我什麼都沒有.
唯有去懇求你幫助的態度.
所以信心不是純粹信一些東西是真的那麼簡單.
而是一種懇求 求上帝幫助自己的態度.
所以我們新教的信心.
就是帶著一種人和上帝的關係差距.
路德覺得自己人 基督徒是完全的罪人.
我們只能夠像一個黑蓋一樣.
這是路德所說的.
我們是一個黑蓋.
我們只不過是知道哪裡有麵包的一個黑蓋.
所以當我們有信心去相信上帝的時候.
路德就帶著一種禱告的態度.
我們是懷著一種信心的禱告.
或者禱告信心來向上帝面前.

$^{921}$去懇求祂的幫助.
去相信祂 去倚靠祂.
所以禱告就是這樣的一個概念.
是帶著一種純粹的倚靠.
純粹的尋求.
純粹懇求上帝幫助自己.
這樣的一種軟弱的態度.
所以路德說.
他給了一個很重要的三個根基.
就是禱告是什麼呢.
禱告就是上帝的命令.
這個方向很重要.
就是禱告是上帝的命令.
所以他就說你不需要問禱告有什麼益處.
禱告有什麼好處.
禱告就是上帝的命令.
上帝命令你要禱告.
這也是詩篇所說的.
當你在患難的時候去求告我.
禱告你們就得著.
禱告是一個命令的語氣.
所以上帝命令我們去禱告.
這是我們新教的傳統.
上帝命令我們去祈禱.
所以我們就去祈禱.
不是去計較我們怎樣祈禱.
怎樣得著 怎樣得益.
而是本身的命令.
當你不敢做的時候.
這就是犯了上帝的命令.
你沒有聽上帝的話.
所以這是我們新教的禱告觀.
你不要問為什麼.
你去嘗試去實踐我們的禱告的生命.
禱告成為了我們信服的表達.
第二就是 應許.
上帝是應許了一些恩典給我們.
所以才出現上帝的應許.
才出現我們對上帝的禱告.
既然上帝應許給我們.

$^{961}$我們就一定要去求上帝.
第三就是患難.
德文裡面有一個說法.
患難去教導人去禱告.
不知道你同不同意這句說話.
患難去教導人去禱告.
有些人死到臨頭都不會信耶穌.
當我們面對著患難的時候.
人就看到自己的軟弱和不足.
我們就嘗試去禱告上帝.
所以當我們去患癌的時候.
當我們遇見一些災難的時候.
當我們遇見病毒的時候.
在這個疫情之下.
我們自然而然就會去禱告.
因為在這個患難當中.
我們就會發現我們需要去禱告上帝.
所以這三點就成為了我們新教的很重要的禱告原則.
或者說是基礎.
禱告是上帝命令.
禱告是上帝應許的回應.
禱告是我們自我察覺到我們人生的患難的回應.
所以穆蘭頓 奴德的另一個同伴.
強調一個Solar Day Evocation.
就是一個叫做唯獨呼求的概念.
原來除了一個唯獨的恩典.
唯獨基督的信心之外.
他就說多過唯獨.
就是唯獨呼求.
這個呼求不單單純粹是一種忠誠禱告.
而是一種叫有命.
就是人向上帝在患難裡面呼求.
是我們整個基督徒裡面很重要的態度.
都是我們新教基督教很重要的精神.
就是發現自己的軟弱和不配.
所以我們就去呼求上帝.
然後你會發現 加爾文也是類似看法.
加爾文在基督教二月第三卷裡面.
提到加爾文補充了有關聖靈的位置.
聖靈作為基督的靈.

$^{1001}$祂在禱告裡面的功效在哪裡.
第二就是強調基督的代求.
這個就是回應中世紀裡面的聖人禱告的關係和問題.
加爾文說我們基督徒新教徒.
我們不會向聖人代求.
為什麼呢.
因為基督耶穌成為我們的代求.
基督的升天是很重要的.
因為基督的升天的意義是什麼.
就是祂在天上為所有的人代求.
所以我們去奉主名求的時候.
就是這樣的原理.
奉主名求不是一句咒語.
而是什麼.
是說當我們禱告的時候.
因為我們在基督裡面.
所以基督耶穌是會聽我們的禱告.
繼而會為我們去代求.
耶穌會在天父身邊去咬耳朵.
祂在傳來佛法右邊裡面.
會親自為我們去祈禱.
因為這是一個聖旨的禱告.
是上帝移旨的禱告.
所以祂會變成咒.
所以這個就是基督的代求的意思.
基督耶穌在天上為我們代求.
是一個很重要的神學的教導.
最後就是基督耶穌就是禱告的答案.
這是正反的.
基督的發展方法的意思是.
我們禱告不是計較我們禱告的技巧有多少.
我們禱告有多厲害.
所以神父怎麼聽我們的祈禱.
不是.
基督耶穌本來就是我們禱告的答案.
所以先出現這個禱告的答案.
基督耶穌的封聖.
才出現我們的禱告.
先有應許.
然後才出現我們的禱告祈求.

$^{1041}$所以這是一個很雞文的答案.
上帝就是會怎樣.
就是禱告的回應就是基督耶穌自己.
所以先出現了禱告的答案.
然後才出現我們的祈求.
所以這是一個很加以民主意識的對禱告的回應.
我們快一點.
我們要到大概八點多.
八點多就完結了.
最後我們看到今天橫教會.
都是受某種禱告運動所影響.
去到新教大概二三百年前.
出現了一種復興神學.
就是我們認為.
我們的敬虔能夠去回應上帝復興.
我們只要聚集一群人去禱告.
所以才出現了禱告運動.
大概在二百年前.
北美出現了禱告運動.
人們就開始嘗試禱告會.
我們在一個特別的聚會.
在我們崇拜和團契.
我們特別花時間.
抽時間聚集一起去祈禱.
而這個禱告會會帶來復興.
這就是二百年前的思維.
復興運動的思維.
當一群信徒聚集在一起.
特別去呼求上帝的時候.
神就會帶來復興.
這是一個粉輕主義的態度.
而我們環球衛星是有一點受到影響.
因為一百年前的來華西教士.
這群人的政制.
在北美復興了年輕人.
來到中國傳福音.
所以整個教會都是受到這樣的神學.
通宵祈禱會.
一種禱告鏈運動.
我們都相信禱告會能夠帶來復興.

$^{1081}$這是我們環教會的傳統.
今天就說這麼多.
這是我們整個歷史.
這是我們的背景.
我們想說的就是關乎我們神的部分.
有些問題我們嘗試解答.
今天想到幾個主題和大家討論.
第一個就是禱告有用嗎?.
究竟我們禱告有沒有用?.
第二就是禱告行動的關係.
我們的禱告和行動之間.
是怎樣理解呢?.
我們禱告之後是不是不用戴口罩呢?.
這就是我們今天要討論的問題.
第三就是我們生命禱告.
想做出一個神學的論述.
第一部分就是禱告有用嗎?.
這個可能大家都有興趣.
我們今天面對這麼大的疫情.
我們想想我們的禱告是什麼意思.
特別是在大興.
當你問我們禱告有沒有用的時候.
其實都不懂得說禱告有沒有用.
我禱告之後神都沒有聽.
或者是有疫情的時候.
我們禱告是不是真的能夠改變世界呢?.
疫情是不是禱告之後就能夠被消滅呢?.
我們發現現代人和禱告是越來越抽離的.
想想古代人是這樣的.
幾千年前的人當他們面對疫情的時候.
他們是只有禱告的.
因為他們不知道有Virus.
他們沒有顯微鏡.
也沒有任何的醫學常識.
他們看到有人病了死了.
他們就自然禱告.
因為他們不懂得這個東西.
我們說Virus是會感染的.
Virus是會戴口罩之後減低傳染.
Virus是關乎你洗手的.

$^{1121}$你明白嗎?.
人類似乎會為一些迷思來祈禱.
當你很明白一件事的實際上.
你就會傾向不祈禱.
假設我都說過這個例子.
假設你的電腦上不到網.
如果你上不到網的話.
你可能會祈禱.
但你發現自己沒有開WiFi的話.
那你就開WiFi吧.
人類是會為自己懂得的東西來解決.
你不懂的東西就會去祈禱.
所以我們會為癌症去祈禱.
因為癌症仍然有些空間我們不懂.
有些癌細胞的抗發.
癌症是很神秘的東西.
感冒你都會祈禱.
但感冒就不會像很奇怪的祈禱.
你都會用藥物來醫治.
為你康復祈禱.
你會求醫生的手來醫治你.
人類是為一些迷思來祈禱.
一些你很了解的東西.
譬如說你的門有沒有鎖上.
原來沒有開電腦.
有沒有電池插電的東西.
你不會為這些東西祈禱.
你會為一些你仍然有空間神秘的東西去祈禱.
所以現代人是越來越少禱告的事情.
因為對於現代人來說.
整個世界的認識是多了.
你會發現原來病毒是關乎於這樣.
只要你發現你能夠控制到的東西.
我們就會開始越少禱告的部分.
所以在啟蒙門開始之後.
很多人發現人是有理性的.
我們的禱告就不會停在那裡.
你今天不會為時空倒流祈禱.
你今天也不會為瞬間轉移祈禱.
你會為某些事情來解釋的事情去祈禱.

$^{1161}$這是第一個.
我們覺得禱告是跟科學對立.
我們會為某些事情來祈禱.
我們似乎不會祈禱.
我們不會祈禱一些事情.
我們不會祈禱不用坐飛機飛回香港.
我們不會祈禱一些比較合理的事情.
所以其中一個說法是.
禱告就不是去影響上帝.
而是去改變禱告者.
這正是啟蒙門開始的看法.
就是禱告不再是處理一些神秘的問題.
慢慢就成為了改變我們的內心.
或者說禱告是關乎我們內心世界裡面的一些平靜.
改變我們自己.
這時候就發覺.
禱告就再不是一種祈求上帝的功能.
有些屬靈人就說.
禱告成為一種靈修.
禱告是一種什麼呢?.
就是我們不再去改變事情.
禱告是改變我們自己.
禱告是跟神親近或者是交往.
人就慢慢放棄了禱告作為一種祈求的目的.
不知道你有沒有這個看法.
其中一個答案是.
我們已經不再覺得禱告能夠改變世界.
我不知道你有多大的認真.
覺得禱告能夠改變全球大陸的運營.
你覺得禱告能夠改變現在全球加拿大美國的疫情.
你都會奇怪.
但你覺得禱告有多大用呢?.
我想說這是一個問題.
不是否定一件事.
所以我覺得這都不是一個答案.
我們不去退縮到一個地步.
覺得禱告純粹是為了靈修.
禱告仍然有用.
禱告仍然是跟隨最基本的意思.
就是懇求上帝.

$^{1201}$我們為我們所需要的生活和生命懇求上帝.
所以禱告不覺得純粹是靈修.
或者改變自己的心意那麼簡單.
而是說禱告仍然是有一種.
為世界的事情去祈求的東西.
我們是不是灰心了呢?.
我們是不是已經退讓到一個地步.
覺得禱告是沒有用呢?.
如果有的話我們怎樣去理解這個問題呢?.
禱告是怎樣有用呢?.
所以我們有另一個答案.
就是用階級民意去禱告.
就是覺得禱告是上帝不會改變的.
如果你是一個階級民主者的話.
就認為禱告是不會改變上帝的.
因為上帝有他自己的主權.
上帝有自己的旨意.
所以禱告是不會改變上帝的旨意.
所以這是一個很階級民主的說法.
上帝安排了一切.
禱告是改變不了什麼的.
所以這個仍然是將禱告的意義去推翻了.
禱告是一種命令.
所以我們不會不祈禱.
但是我覺得禱告是有點奇怪的.
覺得禱告是改變不了什麼的.
上帝也是不會聽的.
上帝是會有自己的心意的.
這樣是很勁竭.
但仍然是一種傾向是覺得禱告是沒有用的.
禱告似乎只是去聽從上帝的命令.
是一種生活習慣.
多於真的有用.
所以我們禱告其實也有些矛盾.
當你為一個人有病祈禱的時候.
你所謂的禱告好像也沒有什麼幫助.
所以我覺得似乎也有些困難在這裡.
我問的是上帝是否真正不會被改變.
或者上帝是不是傳統神學所說的不被改變.
上帝的不變是不是這個意思.

$^{1241}$這個希臘哲學的看法就是上帝是不變的.
因為上帝是一種最大的真實性.
他是不會被動搖.
不會被世界所改變.
但當我們這樣說的時候.
這是一種很哲學的上帝.
上帝是不能受苦.
上帝是不能改變.
上帝是一種沒有感情.
感情就是一種被改變.
所謂的被動和觸碰.
動和觸碰都是一種被改變的看法.
所以上帝是一種懶散散的上帝.
上帝是沒有情緒.
情緒就是一種妥變.
但這不是聖經所說的.
聖經所說的是什麼.
聖經所說的是耶穌是一種會流淚.
十家裡面的耶穌是死亡.
所以基本上上帝的來臨.
本來就是一種上帝似乎不會被哲學上的改變框住.
上帝是生命的上帝.
所以生命上帝不是一種懶散散的上帝.
不變就不會有這樣的改變.
上帝的大能是不會被他不被改變來框住.
大家明白我的意思嗎.
就是上帝的全能不會被一種很抽象的absolute power去控制住.
我不變就不變.
所以我怎樣也幫不了你.
這是一種有限的上帝.
但如果上帝是生命的主義的永活上帝.
他似乎是不會被一種不能被改變所影響.
所以上帝仍然可以去聽我們祈禱.
並且願意來聽我們祈禱.
願意開放他自己讓我們參與他的生命.
當然你問我為什麼.
我會說這是一個奧秘.
但我們不能否定這一點.
上帝仍然會聽祈禱.
仍然會開放他自己來聽我們祈禱.

$^{1281}$並且實踐我們的祈禱.
我們未必有一個很理性的理論.
去解釋這個理論的原因.
但我們仍然相信我們的祈禱是有意思的.
我們的祈禱並不是純粹去講一個不能改變的東西.
上帝是大能的.
上帝是慈愛的.
所以他能夠容納我們的祈禱在他的生命裡.
在他的旨意裡.
是超乎所謂不能被改變的概念.
來實踐我們的祈禱.
所以我們仍然是祈禱的.
我們仍然為世界的疫情去祈禱.
並且相信我們的祈禱是有意思的.
是有用的.
是能夠上帝去閱納的.
當然今天不夠時間去講很詳細的.
這是一堂的主題.
但我想說的是.
我們仍然是不會放棄我們的祈禱.
祈禱並不是被虐化成為一種純粹靈修.
改變自己想法的東西.
也不是一種純粹的嘉爾文神學.
上帝不變就不變.
而是上帝是能夠容納我們在祈禱當中的上帝.
我嘗試快一點.
因為有時間想跟大家聊聊天.
我們想說的是.
我們的祈禱與行動是分不開的.
我們究竟是要戴口罩還是要祈禱呢?.
我們是要去洗手還是去祈禱呢?.
我們的行動與祈禱有什麼關係呢?.
回到主題.
All Elaborate.
祈禱與行動.
其實是同一件事情.
當我們回看耶穌的浪子的比喻.
浪子的比喻.
你會發現我們整個的聲音.
小兒子其實很特別.

$^{1321}$十七節裡面.
他突然在豬欄裡醒不過來.
他心裡想.
我父親是一個父親.
十八節.
我要起來.
到我父親那裡去.
向他說.
父親我得罪了天.
得罪了你.
從今以後我不配為你的兒子.
這是小兒子在豬欄裡的一個醒悟.
他突然發現自己不配.
發現自己不配做爸爸的兒子.
所以他心裡就這樣想.
他說.
父啊 我得罪了天.
得罪了你.
這是一個禱告.
如果我們說父親代表天父上帝.
而小兒子代表人.
小兒子在豬欄裡醒不過來.
當他發現自己的軟弱.
發現自己的不配的時候.
這種迴轉.
這種發現自己不足的醒覺.
這就是一個最恰當的禱告.
就像瑞利的禱告一樣.
瑞利在聖經裡說自己不配.
這是上帝率意立的禱告.
很特別.
當小兒子去到父親面前的時候.
就是第22節的時候.
你會發現經文裡重覆了這個經文.
父親我得罪了天.
也得罪了你.
從今以後我不配為你的兒子.
這是小兒子再一次將他心裡的禱告.
親自說出來.
成為一個禱文.

$^{1361}$成為一個親自向他父親說的話.
我們問.
哪個才是禱告呢.
是18節所說的心裡的反省.
還是22節的親口禱告呢.
22節很特別是將它重覆了一次.
他心裡的生命的依靠.
對於上帝的迴轉.
是一個生命的禱告.
22節是一個行動的禱告.
所以我們發現.
這回到我們所說的.
初期教父所說的.
我們整個生命就是禱告.
當我們整個生命去到上帝面前的時候.
我們本來就是出於這樣的禱告.
這也是保祿所說的.
加上外面所說的經文.
當上帝猜他的兒子的靈.
進入到我們生命裡的時候.
我們就呼叫阿爸父.
這個阿爸父的禱告.
其實是我們心裡的禱告.
是我們心裡願意去承認.
上帝為父親的最基本生命的禱告.
這個阿爸父的禱文.
不一定是我們說得出來的一個禱告.
而是我們心裡承認.
這個上帝是我們爸爸.
我們願意去依靠他.
願意去尋求他的態度.
是我們整個基督徒的生命.
所以當聖靈進駐到我們生命的時候.
當我們以阿爸父來回應這個爸爸.
我們承認這個天父是我們爸爸的時候.
就成為了一個上帝的後兒.
一個新傳.
原來我們作為上帝的兒女.
整個基督徒生命的構成.
我們的靈命.

$^{1401}$這個明天會說.
我們的靈命.
我們成為基督徒整個的起點.
就是建基於道的禱告.
這個禱告不是一個說得出聲的.
一個有聲的禱告.
或者無聲的禱告.
而是我們整個生命裡的那種.
對於上帝的那種倚靠.
我們承認這個天父是我們爸爸.
我們願意去尋求他的態度.
所以這種生命的禱告.
我們對於我們基督徒.
我們在疫情裡知道要轉向上帝.
需要去尋求天父幫助.
這樣的態度.
是一種生命的禱告.
所以禱告似乎不是單單純粹.
是一種說出來的東西.
不是純粹地.
所說的那種.
向著爸爸說話的禱告.
而是我們整個生命裡.
去倚靠上帝禱告的態度.
一個Prayerful的態度.
當我們去做基督徒的時候.
基督徒最基本的定義.
就是我們願意去成為天父的兒女.
去倚靠上帝.
願意稱爸爸為爸爸.
所以這種的生存.
這種我們明明不是天父的兒女.
這是一種領養的關係.
我們就成為了天父的兒女.
所以那個過程就是我們的禱告.
一種生命的禱告.
去構成這樣的態度.
所以簡單來說.
我們整個基督徒.
我們是一個Prayer Being.

$^{1441}$就是說我們整個基督徒生命.
所謂靈命的本身.
就是一種帶著禱告的基礎.
我們以禱告來開始我們基督徒生命.
我們每分每秒都不停止的禱告.
一種不住的禱告.
就是我們整個基督徒生命.
我們在上帝面前.
我們的禱告沒有停止過.
因為我們對上帝的尋求沒有停止過.
我們在聖靈裡面.
我們在基督裡面.
那個關係是沒有停止過.
所以在這個Prayer Being下.
我們就出兩個不同的行動.
我們是會祈禱.
我們是會行動.
禱告和行動是沒有違反的.
是沒有矛盾的.
你會繼續去戴口罩.
繼續去做防禦措施.
但仍然是會禱告上帝.
不是說我戴口罩就不依靠上帝.
這個正是很多今天的異端教會.
我們覺得神會保護我們.
所以我們就不戴口罩.
這個不是的.
我們的行動和禱告是分不開的.
因為我們是整個Prayer Being.
我們的禱告生命.
就會帶來行動和禱告.
你想想今天如果頂枝妹裡面.
有人患上癌症的話.
你會怎樣做.
你會禱告.
但如果是你的兒子.
或者你的父母患上癌症的話.
你就不會單單禱告.
你會行動.
你會去看醫生.

$^{1481}$你會和他一起去過難熬的時間.
所以行動和禱告是分不開的.
我們的禱告和行動.
都是源於我們整個Prayer Being.
我們會積極地行動.
我們同時也會依靠上帝.
所以我們會禱告.
第三個我們想說的.
就是我們一種生命的禱告.
這也是我今天整個討論裡想說的.
面對著今天的疫情.
我們重新思考的.
不是我們怎樣祈禱更有效的問題.
雖然我是教土豪神學.
但我從來都不覺得禱告有任何的獨門秘笈.
一種秘方.
一種方法.
方法就是基督耶穌自己.
只要凡是奉主明求的.
這就是最好的方法.
所以重點不是方法問題.
重點是我們整個生命.
我們整個Christian Being.
本來就是一種禱告.
我們願意尋求天賦的時候.
我們就會這樣來祈禱.
這是我很喜歡的說話.
卡爾巴特說的.
To be a Christian and to pray.
is one and the same thing.
就是說做基督徒和禱告其實是一件事.
整個Christian Life本來就是一種Prayer Life.
意思不是我們不停止的說出來的禱告.
而是我們整個做基督徒這件事.
本來就是一種倚靠上帝的關係.
所以在這樣的Christian Being下.
我經常說.
什麼人就站什麼土.
這是你隱藏不了的.
你很有心宣教的人.

$^{1521}$你自自然然就會放很多時間在宣教時期.
你很關心社會政治的事情的人.
你就自自然然會為政治社會的事情祈禱.
當你面對全球疫情的時候.
你自自然然就會為疫情祈禱.
所以問題不是我們禱告怎麼祈禱.
而是我們整個生命站在什麼方向.
我們在整個上帝國度的什麼位置.
我們就自自然然為這件事情祈禱.
第二就是.
我們禱告是一種無能者的可能.
所謂禱告的大能.
就在乎於我們承認自己的無能.
我們在上帝面前.
禱告最基本的定義是什麼呢.
就是承認自己的不配.
這就是歷史上的說法.
如果你謙卑 人若自卑.
禱告來到我面前.
我必然會垂聽禱告.
必定會干預他們 將疫情除去.
所以我們的禱告本來就是帶著我們對於.
我們承認面對著這個世界的艱難.
面對著這個疫情.
我們發覺我們做不了什麼.
所以我們就祈禱上帝.
這是基督徒很基本的態度.
當我們面對著疫情的時候.
我們發覺自己真的無能.
所以我們就去禱告.
這就是我們最好面對著疫情的定義.
最後我想用詩篇第91篇.
來做一個簡單的結束.
這是一個疫情之下的詩篇.
因為這個是提及到.
當我們面對著疫情的時候.
我們來到去懇求上帝的幫助.
很特別.
我們會發現.
這篇詩篇裡面.

$^{1561}$其實如果你想看.
這篇詩篇裡面有分我們三人的.
斯文基明的第一節.
是整個詩篇的名題.
住在至高者的任物處之下.
就住在全能者的任下.
然後詩篇裡面分出幾個不同的對話當中.
第二節就是橙色的我字.
這是私人自己的對白.
我要論道爺說.
他是我的避難所.
是我的山寨.
是我的上帝.
是我所依靠的.
這是私人自己對於上帝的一種凝訊.
在面對著疫情之下.
我們對於上帝的一種凝訊.
他是我們的避難所.
是我們的山寨.
是我們所依靠的.
然後詩篇裡面有一個叫做.
你的經文.
你是第三者.
一個第三者的朋友.
向這個私人人說話.
他救你.
即是說上帝會救你做些什麼.
一個很多細節的論述.
他會救你在報料人的網絡裡面.
叫你不會怕黑夜的驚嚇.
都能夠避開這個病毒和瘟疫.
不過有趣的是.
最後這裡.
在第十四節裡面.
一個上帝自己突然間的臨格.
詩篇最後出現了上帝的出現.
上帝說.
因為他專心愛我.
我就要答救他.
因為他知道我的名.

$^{1601}$我要把他放在高處.
他要求告我.
我就應允他.
他在急難中.
我就留他同在.
我要答救他.
只要他轉跪.
這是一個上帝自己.
對於一個禱告者人的回應.
所以我想重點並不是我們怎樣祈.
而是對於上帝來說.
我們在上帝面前是一個什麼人.
他是一個專心愛上帝的人.
他是一個知道上帝名字的人.
他是一個求告上帝的人.
他是一個願意去求上帝的人.
所以我想整個的Christian being.
我們整個的生命.
就是我們的禱告.
上帝所垂聽的不單單是我們的祈求.
而是他看見我們是一個什麼人.
他是一個專心愛我的人.
他是一個知道我的名的人.
他是一個求過我的人.
所以我想整個是分不開的.
我們在疫情之下的生命.
究竟我們是一個充滿恐懼的生命.
是一個自私的生命.
只顧著自己的口罩生命.
還是我們在疫情之下.
願意愛上帝愛人.
仍然做基督徒去見證上帝人.
並且是一個禱告人.
所以我想重點不是我們怎樣去祈求.
而是我們整個人的生命.
什麼人就祈什麼禱.
這就是我們今天嘗試去學習.
在疫情之下嘗試思考禱告的課題.
我先到這裡.
跟大家聊聊天.

$^{1641}$還有一些問答環節.
如果有問題都可以寫下來問我.
我先到這裡.
大家好.
聽了這麼精彩的演講.
現在到了另一個環節.
就是我們有朱牧師.
我自己也會代表大家.
發問一些關於禱告的問題.
讓陳博士去回答.
如果這個時候請朱牧師先說.
朱牧師.
多謝John剛才的講解.
其實時間不夠.
我聽到很多內容很豐富.
剛才John也提到一件事.
陳博士說一件事.
就是在祈禱裡面.
耶穌教門徒禱告就是向天父祈禱.
剛才他也說過這一部.
也有一堆經文在《樣樣方法》提到.
就是吩咐門徒奉主的名來祈禱.
更加在其他新約聖經裡面提到.
本來提到要靠著聖靈隨時多方禱告祈求.
也提到我們的軟弱有聖靈幫助.
我們本不曉得如何祈禱.
只是聖靈親自用說不出來的嘆息替我們禱告.
這些經文在祈禱裡面.
聖父,聖子,聖靈.
好像也有不同的角色.
在這個時間.
陳博士可以講深一點.
或者講進入一點.
三一神在我們信徒禱告裡面的角色是怎樣呢?.
我們如何理解呢?.
特別是進而知.
在我們實際的禱告生活的時候.
帶來給我們甚麼學習和提醒呢?.
想請陳博士跟我們講解一下.
多謝祖母師.

$^{1681}$我想今天也不能花很長時間講三一論的問題.
其實你會發覺.
三一論裡面也很明顯有三一關係在.
當我們去看禱告的時間.
我們發覺耶穌教要我們向天上的父祈禱.
這是我們禱告的對象.
我們向著天父,上帝祈禱.
聖經裡面也有一兩句經文提到向耶穌祈禱.
例如寶羅說.
我的軟弱.
主的刺也明白我的軟弱.
我曾向主禱告.
主是耶穌基督自己.
但這是很少的.
很多時候九成九都是向天父祈禱.
偶爾也向耶穌祈禱.
耶穌的角色似乎是一個中保角色.
我們說奉主明求是一個很重要的神學.
因為他取代了整個舊約裡面的聖殿神學.
我們再也不是向著聖殿祈禱.
不是猶太教那樣.
而是我們說.
只要我們在基督裡面.
我們是一個基督徒的時候.
我們的禱告就能夠藉著基督耶穌.
去到天父上面.
所以耶穌基督是一個很重要的.
禱告關係的中保.
就像聖殿一樣.
所以我們是.
祈禱向天父.
我們是在耶穌的名字裡祈禱.
我們是靠著基督耶穌的名字.
靠著祂來禱告.
當然奉主明求這個意思.
有很多不同的解釋.
例如佳文說.
奉主明求就是基督耶穌的代求.
耶穌基督會在天上.
再次為我們的禱告事項來祈禱.

$^{1721}$奉主明求不是咒語.
不是說出來就有.
不說出來就不靈.
這是一個很重要的.
我們在基督裡面的關係.
任何的基督徒.
因為他是基督裡面的人.
所以我們的禱告.
都能夠藉著耶穌去到上帝面前.
而聖靈.
我們都會認為.
聖靈是一種.
我們稱之為上帝的主觀性.
聖靈在我們生命裡面.
是一種同在.
祂是基督耶穌的靈.
祂在我們生命裡面.
我們只能夠在祂裡面去祈禱.
譬如猶太書第十節.
在聖靈裡面禱告.
所以那個「然」的字.
可以解作「藉著」.
我們說藉著聖靈去到.
或是說靠著聖靈多方禱告.
這個「靠」字其實是個「然」字.
是一個instrumental 的「然」字.
即是說我們在聖靈裡面去祈禱.
所以似乎是一種.
強烈的聖靈的臨在.
當我們基督徒成為基督徒之後.
聖靈的臨格.
讓我們能夠和上帝.
有一個親自和互動的關係.
所以我們就能夠靠著聖靈.
或是在聖靈裡面去禱告.
所以就有這樣的關係.
我們禱告的時候是.
in spirit.
也是靠著spirit.
但我覺得更重要的是in spirit.

$^{1761}$是聖靈裡面的禱告.
是一種同在.
然後我們禱告能夠去到耶穌裡面.
成為一種橋樑.
能夠去到天父去收禱告.
所以我覺得是一個很重要的.
禱告的三個關係.
我們都比較他很強.
我覺得無所謂.
反正都是神.
其實不是的.
我們都發覺禱告裡面有很多.
三位格裡面有幾個不同的功能.
和位置在裡面.
我們向天父祈禱.
我們奉主明求.
我們是靠聖靈.
在聖靈裡面去禱告.
這是上帝去聽禱告的工作.
他聽禱告就是因為聖靈在裡面.
讓我們的禱告能夠有關係的結達.
然後我們的禱告能夠藉著.
能夠被他修正.
所以聖經說.
我們軟弱.
我們不知道怎麼祈禱.
聖靈就可以幫我們祈禱.
這也是一種基督律問題.
因為整個第八章裡面.
都強調基督耶穌的靈.
耶穌的靈去幫助我們.
所以似乎是說.
聖靈是一種基督耶穌的靈.
讓我們很不配的禱告.
很多錯誤有罪的禱告.
能夠真正在基督所說的關係裡面.
重新去垂聽.
這是一個比較神學的角度.
去理解禱告時候的三個關係.
(記者:剛才有第二個問題).

$^{1801}$剛才我提到.
對於一群信徒來說.
對我們認識到三角關係.
在信徒的禱告裡面.
或者剛才說到以靈的祈禱.
明白了這個道理之後.
對我們的祈禱有什麼學習和提醒.
我想從經驗上是沒有什麼改變.
你都是照樣去祈禱.
禱告是沒有改變.
仍然是那個祈禱.
所以禱告是很簡單.
一個婆婆.
一個臨死的人.
對禱告已經是很真誠的禱告.
禱告是不會有任何的差別.
但我們在教裡面.
都可以有一個比較正確的認識.
可能禱告裡面的經驗是沒有改變.
但我們比較知道.
比如崇拜裡面.
我們知道怎麼去祈禱.
我們的禱告對象不是聖靈.
我們是對著天父祈禱.
我們不會過分靈因.
Inspirit不代表靈因.
Inspirit是一種很正常的生活和生命.
所以你未必會突然感到震驚.
才會叫做Inspirit.
所以這些關係不會影響我們的經驗.
是一種很普通的禱告經驗.
但我們知道我們的方法.
是這樣的.
我們是禱告的End user.
基本上是沒有改變你的經驗.
但背後的事情就不同了.
所以其實都是仍然可以放膽去祈禱.
但我們知道有些錯誤的看法可以改變.
比如我們禱告一定要.
聖靈很感動這些就不一定了.

$^{1841}$甚至我覺得崇拜裡面的禱文也是一樣.
我們禱告不一定即興.
禱告裡面都可以是一些可以預備的禱告.
因為這些都是可以聖靈工作.
聖靈不一定很spontaneous.
不一定要很那種的聖靈.
聖靈是我們和我們.
這個我們明天會講多一點.
聖靈和我們的關係有什麼不同呢.
我覺得聖靈的工作不一定是很超自然的事情.
因為聖靈就是生命的靈.
所以我想靠著聖靈禱告的話.
其實我們怎樣理解聖靈呢.
聖靈不一定是那種特別的聖靈.
反而是一些很normal.
我們生活裡面很地道的聖靈同在.
所以對我們來說不是一個方法的改變.
而是我們的認識上的改變.
從而知道哪些說法是不太正確的.
去修正我們在教會裡面的教導.
回應一下.
朱牧師.
這裡提到.
我們說到我們軟弱的時候.
有聖靈幫助我們.
我想在這個期間.
神不是在做事.
我們在禱告.
其實是一個關係.
在祈禱裡面.
我們在祈禱的過程中.
可以感受到呼應.
在禱告裡面.
我相信我們很多人在祈禱裡面.
都感受到神的靈.
或者是一個感動.
不是從自己的情緒出來.
而是安慰.
在我們生命裡面感應.
所以有些環境可能沒有改變到很苦.

$^{1881}$但是那個empowerment.
在我們的生命裡面.
我覺得神在裡面.
和我們都是在一起的.
就算是在苦難的地方.
我同意.
我覺得禱告是一個很重要的靈修角度.
我們在禱告裡面.
其實都經歷到上帝的同在.
或者是一種經驗.
明天會多說一點.
我們如何理解所謂的靈修.
靈性在禱告裡面.
這也是一種很重要的事情.
我覺得兩方面都不能缺少.
我們去具體化一些經驗.
這是一個特別的時刻.
我們在靈修裡面.
這就是同神獨處的關係.
這個我們是不可以缺少的.
但是我們都不能過分依賴這些關係.
全部都要靠這個時刻.
那就很慘了.
平時有多少時刻呢?.
所以我們都著重ordinary.
平時我們很普通的生活.
這也是一個聖靈當中的同在.
但其實聖靈是上帝力量.
上帝愛的力量.
所以都讓我們在禱告裡面得力.
明天會多說一點這個話題.
謝謝.
盧牧師.
好的.
我就將這個問題帶到一個更加落實一點.
實用一點的方向.
想請教一下陳博士.
第一條問題就是.
信徒都應該知道禱告是重要的.
是應當要做的.

$^{1921}$每個人都知道.
但是最實際的問題就是.
很多人都有以下的經歷就是.
如果我們為一件事祈禱.
但是祈禱了很長很長的時間.
但是神又好像沒有什麼很明確的答覆.
對我們有什麼回應.
我想帶出一個問題就是.
究竟哪個地方出了問題.
如果有這樣的狀況.
就是祈猛祈了很久很久都沒有什麼答覆.
在這樣的狀態下.
信徒應該怎樣去繼續堅持下去.
這是一個很實際的.
很多弟兄姐妹都遇到這樣的困難.
或者陳博士可以解答一下.
好嗎.
大家都去談一下.
可能大家目者都知道.
很多這些案例.
個別不同的案例都有不同的答案.
當然我們都要問我們的.
我們都說了.
我們的禱告是我們的心靈的反思.
我們的心靈帶來很多不同的禱告.
所以我們的禱告本身就是來自我們的心靈本身.
不是說有沒有問題.
而是我們的處境問題.
所以我想我們.
一方面我們.
其實我覺得.
我自己的經驗.
特別是香港的經驗.
過去一年裡面.
很多人都為了香港去祈禱.
其實很多都很灰心.
很多都祈禱.
好像怎樣祈禱都改變不了什麼.
我覺得其實一方面.
這件事不是問題.

$^{1961}$我們不一定要用問題去判斷這件事.
因為我們的禱告就是.
我們心靈的反思.
所以我們如果是很.
對某件事很上心.
我們面對著疫情.
我們面對著戰爭.
面對著社會的動盪.
我們整個生命就是在這件事裡面.
去投放當中.
而是不成就的.
沒有改變的.
我覺得這正正就是世界的問題.
世界就是這麼悲哀.
疫情還是這麼猖狂.
其實很多人都會埋怨上帝.
很多人都會覺得失望.
但我覺得又不是問題.
其實很多.
像詩篇裡面很多哀歌都埋怨上帝.
我覺得這正正是我們.
我們那種真誠的禱告.
我覺得禱告是沒有什麼竅門.
沒有什麼說你站那麼久都站不穩.
有什麼方法可以站穩一點.
我覺得沒有這些.
沒有什麼可以教人.
反而我們的竅門就是.
只有一件事.
就是你需要真心.
真誠地祈禱.
所以當你面對著疫情.
真的面對著家人的癌症.
你為那個人的癌症去懇求祈禱.
這個問題其實沒有在這裡.
只不過是世界似乎還沒有出現.
我們會很多時候都會失望.
但這種流淚這種悲痛.
其實是我們生命裡面.
必然會經歷的事情.

$^{2001}$所以我們的禱告.
其實正正是reflect(反映)這件事.
當然你會問.
究竟上帝為什麼會不聽.
這個是每個人都會問的問題.
但我寧願去問上帝的意思.
都比去避免他好.
就是說這就是上帝的心意.
或者上帝是會有自己的成就.
你都是信服還是怎麼樣.
當然我們可以改變自己的心靈的看法.
透過禱告來改變我們的心意.
但我們都覺得.
我們不要太快避免這件事.
不要太快去控告上帝.
或者去賣上帝.
因為這是我們出於.
你埋怨是不能隱藏.
你不說出來.
神就知道你在埋怨他.
所以你反而需要真誠地面對你自己.
和上帝自己.
好過你寧願有血有肉的禱告.
好過一種正確的禱告.
我的禱告就是這樣很正確的.
就是上帝不要求我求你.
這當然是對的.
但我覺得我們太快放棄.
因為我們都覺得.
可能都要改變自己的心意.
當然我們都願意聽信上帝的心意.
但這個主題就不是關於禱告的事.
而是關於基督徒的事.
我們作為基督徒都需要去聽信上帝的心意.
去聽信上帝的旨意.
所以我覺得這是我們基督徒本身要做的事.
不斷尋求上帝的心意.
去更新自己的生命.
更新自己的眼光.
但這不是我們禱告裡做的事.

$^{2041}$而是基督徒要做的事.
我們禱告就需要真誠地.
去反映我們自己心裡的想法.
你真的很埋怨.
你就是埋怨.
你很想去求.
那你就去求.
好過覺得要強迫自己.
要聽你講.
不要聽我講.
這都是假的.
是你心裡很想成就的事.
你寧願犯錯.
例如我好想.
好像小朋友一樣.
我好想要這個玩具.
你可能覺得這是錯的.
但他真的這樣求.
這個求本身是一個真誠的禱告.
向爸爸求一個玩具.
是一個真誠的禱告.
最少他誠意向爸爸求.
好過是很假的.
我都說要讀書乖一點.
其實這些都是假的.
這樣就不是一個好的禱告.
所以我覺得.
我們的禱告都不用太介意.
我們祈求什麼.
怎樣祈求.
而是我們真的需要去問.
我們禱告背後的生命和關係.
是怎樣的.
當然有網球當中.
我經常說.
在網球當中.
我們求的其實是不好的東西.
為什麼.
因為你的生命不好.
你求的那些壞事.

$^{2081}$因為你的生命本身就是壞.
所以才求那些壞事.
所以我覺得有些事情是真的不要求.
但這不是關乎神聽不聽的問題.
而是我們整個的生命是怎樣的.
我都很同意Dr. Chan的說法.
禱告最著重的是我們的真心.
來到神的面前.
傾心講出我們的難處.
我還記得很久很久之前.
有人說到有一個很艱難的事情.
他就說了一句話.
我經常都記得.
他說通通交給主.
全部的事情都放給了神.
傾心討耳都告訴了神.
我想就會比較輕散.
整個人就會舒服一點.
就可以有堅持的能力.
我們都一起學習.
這件事情不是容易的.
在基督徒的生命中學習禱告.
這個時候我想帶另一個問題給Dr. Chan回答一下.
今晚的主題有關禱告神學.
我就有一個問題.
以你的認識.
以你的認識就著普通的弟兄姊妹.
不是說一些神學生.
或者傳道人.
或者其他信尊很久的弟兄姊妹.
我只是說普通的弟兄姊妹.
就普通的信徒來說.
剛才你曾經說過.
涉及神學的領域有好幾大類別.
如果你可以挑選一兩個重要的類別.
你會挑哪一個呢?.
你說那個有關國教奧論.
還是禱告神學的部分.
禱告有牽涉基督論,神論,末世論等等.
神學的範圍是很寬的.

$^{2121}$多多少少都和禱告有關.
我的問題是.
就著普通的信徒而言.
哪一類神學的領域是我們最關注的.
我們就向那方面進攻.
明白.
剛才我想說的.
我們都說這是享受的問題.
教會論,人論等等.
其實我覺得不是這樣分的.
因為基本的教會論.
通常都不在教會論裡面.
不是叫做道德祈禱的一種論.
或者不簡單.
是去游的.
怎樣祈禱是一種很教義式的說法.
但通常是一種人與神關係的神學.
就是我們人與上帝之間的關係.
這樣的反省.
所以我們的關係是關乎於一種.
整個Christian life.
就是我們和上帝的關係.
但這不代表靈修.
因為靈修是一種很具體的.
一種我們的經驗.
心靈的體會.
但Christian life是這麼廣闊的.
我們怎樣做基督徒.
我們怎樣去信徒生活.
所以通常都是這樣.
是一種思考人與上帝在人的層面裡.
怎樣理解人與神的關係.
所以我覺得是這樣的.
所以如果要擺位的話.
其實它是有點像ethics.
就是人應該做什麼.
但又不是純粹這麼簡單.
是關於一些關係的背後的東西.
所以我們有一些叫做靈修學.
有些叫做倫理學.

$^{2161}$就是人的行為.
和人的心靈世界.
怎樣與神的關係.
但我覺得是關於Christian life.
這幾年我寫的書.
都是被人放在Christian life的欄內.
信徒生活.
我對他的興趣是怎樣去做基督徒.
怎樣去理解做基督徒是什麼一回事.
所以基督徒本身是建基於禱告的關係.
我們本身做基督徒.
就是一種懇求上帝幫助的生命.
所以不是具體地在某一瞬間的靈修.
而是當我們決志之後.
整個Christian life的生活.
那種神學.
所以不是那一輪.
而是信徒生活的神學.
我自己有一個小小的想法.
就是如果我們對我們的救恩是認真的.
如果我們對耶穌基督的拯救是認真的.
如果我們對聖靈的引領也是認真的.
多多少少都會帶動到我們重視祈禱的動作.
我很贊同你的看法.
不是一定要很條條快快地基督論歸基督論.
什麼什麼都歸什麼.
但是混合一大堆的東西.
就表現在Christian life裡面.
基督徒的生命.
基督徒的生活裡面.
這一點我是同意的.
我有一點點的側重點放在救恩那裡.
因為從觀察當中.
很多大英姐妹對救恩不是很清楚.
那時候很自然地反映了她對於祈禱的重視情況.
有一個軟弱的地方.
我不知道這一點你自己有沒有補充一下.
根據路德所說.
我們今天很認真地重視.
因為你乖.

$^{2201}$乖的基督徒就會起哭.
就是祈禱.
這就是我們對於上帝命令的服從.
我們願意做一個聽天父的話的小孩子.
所以你會發覺基督徒會比較乖.
因為他們願意實踐禱告的生命.
所以對於新教來說.
我們願意去禱告.
純粹是因為願意聽天父的話.
多於我們覺得禱告能夠透過禱告.
去培養某種神秘的經驗.
這是宗旨的看法.
透過禱告能夠成為親近神人.
我們新教比較少這些.
純粹是因為你要聽上帝命令.
所以你願意祈禱.
我們就將這件事簡化了.
沒有那麼功能性.
我們覺得禱告是一個功效.
能夠親近上帝那麼簡單.
也有的.
但不是那麼簡單.
我們純粹是需要祈禱.
所以我想.
當然乖這件事.
最主要是因為上帝的救恩.
上帝感動我們.
上帝的救恩激勵我們去愛上帝.
所以我覺得.
對上帝的愛.
其實也是一種靈性.
也是一種上帝的命令.
也體會到上帝的救恩.
所以我想也是有這樣的關係.
相當同意.
其實今晚Dr. Chan.
總結了很多神學觀念.
焦聚在禱告這個題目裡.
我相信大家今晚參加這個講座.
也有很大的啟發.

$^{2241}$我們剛才看到.
原來超過了600多人.
參加今晚的聚會.
我們希望這個聚會能夠幫助大家.
在靈性上更加了解禱告的重要性.
如果這個聚會能夠幫助到你.
也是加拿大建築中心.
舉辦今晚和明晚的講座最重要的目的.
其實加拿大建築中心是非常樂意.
為大英姐妹提供高質素的.
有關屬靈生命的活動.
我盼望大家在以後的日子裡.
能夠在禱告上.
以及經濟上支持我們.
以致我們可以不斷地籌備.
籌劃這些活動.
如果大家有感動的話.
覺得今晚的節目能夠幫助到大家.
大家也可以為我們有些奉獻.
現在出了一個投影片.
告訴大家你的奉獻可以用兩種途徑.
一種途徑就是去到我們的網頁第一版.
我們有一個紫色的一個群組.
就是奉獻支持神學教育.
你一點進去.
就可以去到PayPal.
就可以做奉獻.
如果是PayPal的奉獻.
就立即可以收到報稅的收據.
如果你不想用這個方法的話.
也可以寫支票.
寄回我們的地址.
我們的地址都在網頁裡面.
ABSCC.org的下面.
有我們的地址.
如果是用支票去做奉獻.
就超過30元加幣.
我們就會發收據.
不過就要看你自己的情況.
我們就要看你自己的情況.

\newpage



\section{尼希米記}
\label{sec:P0Y2lvzICsM}
\textbf{【從聖經書卷看生命實踐系列】 主題(二) 從尼希米記看面對將來的挑戰}
\newline
\newline
連結: \href{https://youtube.com/watch?v=P0Y2lvzICsM}{\texttt{https://youtube.com/watch?v=P0Y2lvzICsM}} ~~~~ 語音日期: 2020-07-31
\newline
\newline
\hyperref[sec:fFkCm0QGBPw]{\small{< < < PREV SERMON < < <}}
~
\hyperref[sec:index]{\small{[返主目錄]}}
~
\hyperref[sec:XLKUZGl9ItY]{\small{> > > NEXT SERMON > > >}}
\newline
\newline
$^{1}$歡迎大家來到今晚的悉京講座.
這個悉京講座是我們在疫情期間第二次進行的.
第一次是何啟明牧師的《醫師貼記》.
今晚我們轉到另一個話題.
就是尼熙米記.
這兩個講題都有一個共通點.
就是我們上次用一個神話.
在神書卷裡面看一些問題.
可能有些觀眾未必知道加拿大建立中心是甚麼.
我簡單介紹一下就不浪費大家時間.
我們是在加拿大為基礎.
在多倫多為基礎的一個神學教育機構.
我們有很多不同類型的活動.
其中一個活動就是講座.
亦有一些課程.
包括普通訊號的訊號進修課程.
還有由香港見到老師過來多倫多.
授課的文憑甚至是師範課程.
這次疫情令我們有一個很大的改變.
我們嘗試用Zoom的平台來廣播.
讓全球的弟兄姊妹都能夠認識我們.
我想我就不浪費時間介紹太多的東西.
如果弟兄姊妹不太認識我們的工作.
請你瀏覽我們的網站.
我們的網站是www.abscc.org.
如果你去Facebook的話.
你會看到The Alliance Bible Seminary Center of Canada.
就會知道我們一切的工作.
今晚我要介紹一下一個講員.
就是陳耀鵬牧師.
陳耀鵬牧師是香港見到神學院退休前的副院長.
大家報名的時候已經看到他的資歷.
非常豐厚的.
非常豐富的學會經驗和教學經驗.
我想在這裡我不會花太多時間去介紹.
在五月的時候.
今次就是陳牧師.
在八月的時候.
我們就有郭奕雲牧師.
開始跟我們講一個新的講座.

$^{41}$這個是我們從聖經書卷看生命實踐系列的三個講座.
都是三位退休後見到神學院的中天分子.
跟我們講這些講座.
今晚陳牧師會用利希米技來講這個題目.
他很有心得.
他非常風趣.
我不阻大家的時間.
就將以下的時段交給陳牧師.
我知道有不少弟兄姐妹認識陳牧師.
因為耳邊太過熱烈.
如果開了咪高峰.
根本聽不到任何東西.
我有一個建議.
就是當陳牧師講完之後.
我最後跟大家說再見之後.
開放咪高峰給大家.
跟陳牧師打招呼.
有些不認識陳牧師的可以跟他聊幾句.
我想在講座的過程中.
我們不會開咪高峰.
否則會很吵.
在開始給陳牧師之前.
我們做一個簡單的祈禱.
請大家一起低頭祈禱.
今晚我們第二次機會在網上進行講座.
求主你紀念各地區的疫情.
讓疫情盡快停止.
讓弟兄姐妹恢復正常.
我們在這個時候願意花多點時間.
花多點心思.
在網上學習.
今晚我們學習的機會.
願意花時間跟我們在一起.
願意你自己使用他的心得.
使用他在《離天米記》裡面的體會.
能夠鼓勵我們繼續向前走.
願主你祝福陳牧師今晚的分享.
我們的禱告乃奉耶穌基督的名求.
阿門.
好,請陳牧師.

$^{81}$各位,大家應該看到PowerPoint.
可能也看到我.
還有陸牧師和同工的安排.
讓我跟大家分享.
《聖經書卷體生命實踐》系列的第二講.
我今天沒有打領帶.
當然說聖經也應該打領帶.
我通常說聖經也會打領帶.
不過因為穿著拖鞋.
好像不太協調.
希望大家不要介意.
首先向大家問安.
各位早晨.
午安,晚安.
你在香港的早晨.
你在溫哥華的午安.
你在多倫多或東岸的晚安.
我在溫哥華.
剛才陸牧師也提到.
我去年在建道退休.
現在在溫哥華過退休生活.
今天跟大家分享這個題目.
也是我的榮幸.
這次短短一小時內.
我的目的是甚麼呢?.
我目的最主要是希望介紹尼西米記.
也希望透過這個講座.
讓大家更加認識建道中心.
甚至支持建道中心和建道神學院.
我希望大家有本聖經.
因為尼西米記十三章.
一小時內要很快地看完十三章.
實在不容易.
我盡量來做.
我知道大家有些人未吃飯.
有些人未吃早餐.
所以可能不想我講太久.
我希望一小時內可以講完.
希望大家盡量有本聖經.
我跟大家講尼西米記之前.

$^{121}$首先要介紹背景.
以色列人亡國後.
有三卷歷史書.
以斯拉記,尼西米記和以斯帖記.
如果看這段歷史.
我們基本上知道.
在波蕭滅了巴比倫後.
第一個皇帝古烈.
容許猶太人回到耶路撒冷.
索羅巴伯帶領近五萬人.
大部份都是利美人.
回去重建聖殿.
之後當然是以斯拉出現.
但其實記載在以斯拉記的歷史.
都是頗長的歷史.
當中有以斯帖記成為阿俠隋老王的皇后.
然後九十四年後.
尼西米回去重建城牆.
上次何啟明牧師講過以斯帖記.
他有提過我.
所以我需要幫大家重溫一下以斯帖記.
上次沒有聽是值得聽.
因為何啟明牧師寫了一本書.
是關於以斯帖記.
大家只要到建道中心網頁.
在最新動態通訊.
點擊「講座重溫」.
將他以前的講座和很多建道中心的講座.
大家可以重溫.
他上次講以斯帖記時.
都提過我.
所以我都要交代一下.
因為我曾經用食言就訓.
整個以斯帖記用四個字來做總括.
當然我只是吸引.
只是一個噱頭.
但何牧師寫完那本書後.
對以斯帖記整個佈局,情節,訊息,保守.
都有很深入的分享.
鼓勵大家去看.

$^{161}$當然亦希望大家繼續聽.
葛玄文牧師的《河西亞書》.
好簡單,買一送一.
講尼熙米的,講以斯帖.
食言就訓.
怎樣才算食言就訓呢?.
如果看整個以斯帖記.
你會發覺那十章裡面.
基本上最少有十至十二個言.
以斯帖記沒有提到神.
不是說神沒有在.
只是沒有提到神.
但經常講言直.
如果看言直時.
發覺言直的鋪排,言直的擺放.
很多時候都是一對一對.
最出名的當然是.
以斯帖為王和哈曼設兩個言直.
第一個言直沒有提到甚麼.
第二個言直才打真軍.
很有意思.
所以看以斯帖記.
第一是看很多言直,食言.
當然不是言.
應該是言結的言.
但我用團言的言.
從猶太人,以色列人的眼光來看.
從這個角度來看.
大團言結局.
當然從亞馬力人,阿甲族人.
甚至哈曼來看.
就慘了.
大件事了.
講得俗些就大劑了.
所以團言結局.
咒其實沒甚麼好說.
咒是先後的形容詞.
或是前因後果的形容詞.
不過我相信大家最熟悉的一句說話.
就是以斯帖.

$^{201}$本來很柔順,美麗的皇后.
原本不想搞那麼多事.
但被她的養父沒底改.
散了她幾下.
她就說死就死吧.
你們一班人為我來禁食三日三夜.
我又叫工女一起禁食三日三夜.
然後死就死吧.
進去見皇.
吃完咒.
睡大家不用說.
昨晚睡得好不好.
我睡得一般.
因為見到你們都緊張.
要講尼熙米記.
尼熙這個阿黑徐老王.
睡得不好的時候.
她不是叫太監來.
跟她下棋.
或是叫工女來跳舞給她看.
她看歷史書就發現.
她沒有賞賜他們.
所以就被我們看到.
不能入睡.
睡也很重要.
所以吃完就睡.
在以斯帖記可以總括整個以斯帖記.
當然以斯帖記載有很多.
有些是欺巧的.
好像睡不著覺.
沒有特別.
人都睡不著覺.
但可以看到.
在所有欺巧的事當中.
神都有奇妙的要求.
吃完就睡.
每一個字都讓我們看到.
神的恩典.
神的保守.
神的主權.

$^{241}$神的安排.
吃完就睡.
買一送一.
上次何啟明牧師提過.
我講吃完就睡.
我都要交代一下.
因為上次有聽.
記得都未定.
我鼓勵大家回去再聽.
回到今天的主題.
尼熙米記.
見道神學院在長洲.
如果回香港.
當然你都不會回.
但如果有一天回到香港.
我都鼓勵大家去長洲.
探一探見道神學院.
入長洲可以搭快船.
可以搭慢船.
快船大概八個字就到了.
今日我和大家搭快船.
即是很快.
在餘下的大概四十分鐘裡.
和大家看完這十三章.
在以斯帖記四個字.
吃完就睡.
這個尼熙米記.
我又給你們四個字.
透過這四個字.
你就可以掌握整卷尼熙米記的訊息.
這四個字是四個英文字.
這四個英文字.
start with L O V E.
究竟甚麼是L 甚麼是O 甚麼是V 甚麼是E.
我們一齊看看.
尼熙米記十三章裡.
你可以有一個很簡單的分段.
重建,復興和改革.
再簡單些.
你如果看電腦.

$^{281}$你就可以明白尼熙米記.
那電腦如何明白尼熙米記呢.
因為電腦有一個Hard Drive.
有一個Hardware.
有Hardware是沒有用.
有一個software 有些program.
所以尼熙米記基本上有一個Hardware.
Hardware就是重建城牆.
Software就是重建百姓.
前一段一至六章就講到尼熙米的歸回.
這個極有可能是在波斯出生的猶太裔的土生仔.
他為居要職.
但他願意回到耶路撒冷.
幫助猶太人重建城牆.
之後以斯拉和他一起重建百姓.
我們看尼熙米的時候.
就發覺基本上他扮演三個角色.
第一個角色是波斯王宮的酒精.
第一章第一節至第二章第十節.
第二章第十一節至第六章第十九節.
就看到他在五十二日裡.
和猶太人一起建造城牆.
他是一個美麗的工程師.
一個很有能力.
也很有組織力.
也能夠面對困難的工程師.
由七章一節至十三章三十一節.
他就是一個管理地方的省長.
基本上我們看了尼熙米.
不過你的題目叫.
從尼熙米看面對現今的挑戰.
或者將來的挑戰.
究竟有什麼挑戰.
當然要看你住在哪裡.
在香港過往兩天的挑戰.
午飯和早上.
如果你要上班.
在哪裡吃飯是最大的挑戰.
又或者你有兒子有孫.
你的兒子和孫如果還在上學.

$^{321}$究竟九月之後能否上學呢.
這也是一個挑戰.
當然我們有面對很多挑戰.
我十多年前在香港聽過李錦雄.
大家都聽過他.
他講過一個迎向未來的十堂課.
我不講香港人去哪裡吃飯.
美國,印度,巴西如何應付COVID-19.
或者全世界的人面對經濟衰退.
我將這十件事都提出來.
當然尼熙米沒有講這些.
不過尼熙米在十三章裡面.
透過內容讓我們看到.
當我們面對人生的挑戰.
我們在疫情的影響下.
我們面對任何困難.
當我們掌握這四件事的時候.
我們在任何情況下都可以面對.
當然對我來說人口老化.
養不起的未來.
究竟我退休能否過完最後幾年呢.
這是我最大的挑戰.
最近我看回長命百二歲.
大家記得幾年前.
還有去年香港TVB有一個節目.
我喜歡看的.
看回的時候發現原來日本在前幾年.
他們老人家止尿片的銷售.
第一次超過兒童和嬰孩止尿片的銷售.
這是一個挑戰.
無論如何尼熙米沒有教我們如何面對這些.
不過尼熙米在十三章裡面.
面對有四個LOVE的挑戰.
第一個挑戰是毀身領導的挑戰.
大家都知道尼熙米.
在波斯冬天的行宮服侍亞達舍西王.
在亞達舍西王二十年的時候.
聽到一個消息.
耶路撒冷城牆城門焚毀被燒.
他安定繁榮.

$^{361}$他住在一個非常舒服的環境.
他毀身的對象是否安定繁榮呢.
他不單止安定繁榮.
他更有一個很成功的職業.
剛才也說過.
古烈王元年到亞達舍西王二十年.
基本上有九十四年.
除非他超過九十四歲.
我相信他也不是.
他應該是一個在波斯出生的猶太人.
如果不是他拿著波斯的護照.
他也不會成為一個皇帝非常寵愛甚至信任的酒精.
所以他有成功的職業.
他將來可能會拗糧.
他的RRSP一定很強.
他的公積金和強積金一定很多.
但他沒有成功的職業就很開心.
當然他是猶太人猶太裔人.
不過極有可能他拿波斯的護照.
所以他效忠哪個國家呢.
哪個是他的民族呢.
哪個是他的國家呢.
這件事很難說.
所以我們看他毀身的對象.
是否安定繁榮呢.
是否成功的職業呢.
是否國家民族呢.
似乎都不是.
為何不是呢.
因為如果你看第五節.
第一章第五節到第十一節.
他在祈禱裡面.
你發覺在祈禱裡面.
「僕人」這個字出現了八次.
有兩次是講到摩西.
有三次是講到以色列人.
有三次是講到他自己.
當然他體重安定繁榮.
當然他有一個成功的職業.
當然他效忠國家民族.

$^{401}$不過他更重要.
知道自己是上帝旨意的僕人.
弟兄姊妹.
今日你有沒有清楚自己.
是神的僕人或侍女呢.
去面對周圍給我們的挑戰呢.
這是他毀身的對象.
當然如果看第一章第一節到第一節.
看到他毀身的過程.
他面對一個需要.
需要就是耶路撒冷的城牆城門被焚毀.
耶路撒冷的居民和周圍的居民.
猶太人被外邦人欺凌.
沒有一個城牆.
沒有一個城門保護他們.
他面對這個需要.
當他面對這個需要的時候.
他沒有罵那班人.
那班人回到九十多年.
聖殿都建造了七十多年.
那班人做什麼的.
城牆城門都要搞成這樣.
他沒有.
他說我們的祖先犯罪.
我也犯罪.
他認同那個苦難.
好像香港的教會.
聽到香港的教會很多.
在過往這兩天開放.
特別開放冷氣.
甚至特別找一些童工.
給一些沒辦法找到地方去吃飯的人.
可以進去吃飯.
多麼重要.
認同他們的苦難.
弟兄姊妹我們在教會裡面.
我們有沒有認同周圍的人的困難.
苦難呢.
開放這個小小的地方.
真的很重要.

$^{441}$要承擔這個責任.
我們看到黎希敏要承擔這個責任.
究竟我們面對挑戰的時候.
我們有沒有看到需要認同苦難.
和承擔責任呢.
第二章一至十節.
過了四個月.
過了四個月的時候.
大家知道這個故事.
黎希敏在王的面前.
當然除非他四個月都是拿個No Pay Leave.
或者拿個Vacation.
但他四個月裡面都服侍皇帝.
皇帝突然見到他面帶愁容.
聖經有一個說法.
他素來在王面前沒有面帶愁容.
在四個月裡面雖然他行切祈禱.
但他仍然站穩他的崗位.
沒有面帶愁容.
他仍然知道他的身份是什麼.
他仍然肯定他自己在社會上的崗位.
在社會上的地位.
在社會上的工作.
但他仍然行切祈禱.
但他站穩他的崗位之餘.
他是悉心計劃的.
因為當王問他的時候.
他就知道究竟他需要多少時間來建造這個城牆.
知道需要什麼材料.
知道要經過大河西.
所以要拿個准許.
他是悉心計劃的.
甚至可能等到皇后在旁邊.
他才將這個要求向王講.
他站穩崗位之餘.
他是悉心計劃的.
他悉心計劃之餘.
他沒有裝醒.
他也認定神音.
最後你看第八節.

$^{481}$我們看到他在這裡說.
因為神施恩的手幫助我.
然後他默討天上的神.
當他答王的時候.
他認定神恩.
但他認定神恩之餘.
他是期待那個困難.
你想想在第十節.
他回到耶路撒冷.
應該第一句.
那裡記載什麼.
那班猶太人出來.
舖上紅地毯.
如果那時不流行.
或者找一些橄欖樹葉.
或者好像耶穌那樣進城.
和撒拿都說.
你來拯救我們幫助我們.
不是,你看第十節.
他遇到困難.
有時我們面對在社會工作.
我們做基督徒.
我們面對甚至家庭的困難.
或者我們面對教會的侍奉.
我們知道神放我們在崗位.
但我們不要想一定沒有困難.
我們認定神恩之餘.
我們要期待這個困難.
第二章第十一節.
他回到耶路撒冷.
有三天休息.
當然要休息一下.
要恢復體力.
要穩定情緒.
要站穩陣腳.
接下來我們作為一個領導.
我們都需要在我們領導的過程.
在侍奉的過程.
在我們面對挑戰的過程裡.
我們來到有這四個階段.

$^{521}$我們要有好的適應.
要站穩陣腳.
要好好學習.
要穩定我們的情緒.
要恢復我們的身心靈的力量.
然後他獨自一人.
找了一晚.
從城南破爛的門口.
一直沿著城牆.
然後向北走.
他沒有跟人說甚麼.
他只是有很好的.
要鼓勵猶太人跟你一起建造.
當然要看破爛成怎樣.
他也沒有跟人商量.
他也在最適當的時候.
看準時機.
才去鼓勵那些猶太人.
如何鼓勵他們呢.
以後我對他們說.
我們所遭的爛第二章第十七節.
我們所遭的爛耶路撒冷.
如此荒涼城門被火焚燒.
你們都看見了.
內罷我們重建耶路撒冷的城牆.
免得再受凌辱.
我告訴他們.
我神施恩的審.
怎樣來幫助我.
並王對我所說的話.
他們就說我們起來建造吧.
你看看尼熙米.
他適應了.
他觀察了.
鼓勵他所帶領的人立志.
他沒有說快點來.
你知不知道我在皇宮裡.
吃盡真羞百味.
如果你跟著我.
一起建造完城牆之後.

$^{561}$我帶你去波斯.
然後享受一下真羞百味.
包吃到你碾碾嘴.
他也沒有說.
做完之後.
我帶你去地中海.
做一個郵輪假期給你.
他沒有.
他說夠了.
我們看到這些敗壞.
看到耶路撒冷如何荒涼.
城門被火焚燒.
你們看見了.
弟兄姊妹.
我們有沒有看到現在這個世界的荒涼.
看到教會的荒涼.
看到我們在神面前.
屬靈的荒涼.
然後我們回到神那裡.
我們要建立建造重建這個城牆.
神施恩的手來幫助我們.
我們要適應.
我們要觀察.
我們要立志鼓勵.
我們更加.
看第三章.
第三章很多時候我們都會跳槍.
因為在那裡.
我們看到很多人名.
我們看到眼都不想看下去.
不過我們分析這章經文的時候.
最低限度有四點我們看到.
我們做領袖.
我們必須在建造過程當中.
互相協調.
為什麼說互相協調呢.
特別你看到.
23和29節.
尼希米當他鼓勵人去建造的時候.
那些人是對著或靠近自己的房屋.

$^{601}$來建造城牆.
換句話說.
你住在屯門.
你上班就不用去到西環.
雖然很快.
乘969都很快.
忘記了.
或者你在溫哥華.
你在西環.
就不用去到科克林上班.
又或者你在多倫多.
你在威爾曼大學就不用去到多倫多.
是互相協調的.
同時是同心盡力.
在第一節說到祭司.
第八節說到銀像和造香.
第九節說到管理.
第十七節說到利味人.
第十二節說到商人.
然後十二節有男有女.
同心盡力的來做.
而且是覺盡奇蹟的做.
你看十三節十五節.
裡面有一個幼字.
或者一個丙字.
他們做完自己的那份的時候.
覺盡奇蹟之餘.
更加會問下有哪些地方可以幫忙.
覺盡奇蹟.
多有意思.
當然.
你這樣看就是.
對我沒意思.
對你當然沒意思.
等於你參加過婚禮.
不是你自己的兒子或女兒的婚禮.
你參加過都不熟的或同事的婚禮.
成了婚之後.
當然了.
出來多謝誰.

$^{641}$多謝那個.
多謝連家裡的狗都想多謝.
你覺得我沒人認識.
但對他們來說很重要.
人家幫他插花.
又幫他佈置.
又幫他安排.
又幫他訂酒席.
又幫他那樣那樣.
又幫他送.
很多送禮.
每一個人名都是重要.
因為他要鼓勵他們.
弟兄姊妹.
我們做領袖的時候有沒有這些鼓勵呢.
所以第一個字就是.
一個領袖.
有人這樣說.
我不知道你同不同意.
他說現在寫關於領袖的書.
多過實際的領袖.
我不知道你同不同意.
當然這句說話有嘲諷現在的領袖.
我今天十二點鐘.
我花了一個多小時.
看加拿大的帥哥總理.
現在留鬍子可能沒那麼帥.
Justin Trudeau.
他在熱水中.
大家知道.
如果你在加拿大.
最近有一個所謂的爭議.
他太太阿媽和弟弟.
受一個叫V的組織.
很多的恩惠.
他會不會有一個爭議.
政客很多時候都面對這些.
今天做領袖很不同.
很不容易.
你說有什麼問題.

$^{681}$我不是領袖.
誰說你不是領袖.
有影響力的人就是領袖.
你做阿媽.
你就是影響你的孫.
你沒什麼學識.
你影響你的孫.
不要忘記.
你在教會裡.
你也是一個領袖.
你有機會影響人就是一個領袖.
一定每個人都是一個領袖.
有人這樣說.
你經常罵領袖.
每個人都可以分辨.
一隻雞蛋是好的還是臭的.
一打開雞蛋就知道是否臭.
叫你生雞蛋出來就難.
你知道臭雞蛋.
但你能不能生出來.
意思是不要批評人.
你甚至可以說.
我不做領袖不生雞蛋出來.
我就不會臭.
是或是.
不過在尼熙米的身上.
我們可以看到一個領袖的風範.
他如何毀身的過程.
和他在推動人去工作.
推動人去重建城牆的時候.
他在四方面.
這是第一章.
第三章.
尼熙米寄給我們看到的.
一個領袖.
Challenge of Leadership.
毀身領袖的挑戰.
第二個字就是O.
O就是Opposition.
危機四伏挑戰.

$^{721}$他回到耶路撒冷.
面對很多挑戰.
最低限度在這三章裡.
有六個挑戰.
六個的.
即是外面,後面,內面,前面和外面的挑戰.
首先看看外面的譏笑.
外面的譏笑就是.
金巴拉和阿滿人多比亞說了什麼.
他說這些軟弱的猶大人做什麼呢.
你們軟弱的猶大人.
你的力量是否可以.
軟弱的猶大人.
要保護自己.
你的動機是否為自己.
要保護自己.
你要一天成功.
一天你想搞定.
你的步伐是否實際.
你的材料是否堅固.
要在土堆裡拿出火燒的石頭.
沒用的.
然後多比亞再加一句.
狐狸上去也必踩到.
即是第三節.
你的成就是否持久.
今日我們想去影響人.
今日我們想去侍奉神.
今日我們很想在崗位上做一個好的.
就算別人不看我們的領袖.
但我們都能做一個好的見證.
尼熙米也想帶領著這班人建造城牆.
第一件事.
外面的譏笑.
五件事.
當然這五個問題.
我相信我們都要苦心自問.
我們都要問的.
不過有時是別人譏笑我們.
甚至有時我們自己都要譏笑自己.

$^{761}$行不行.
但跟著那幾次經文.
尼熙米的祈禱.
讓我們看到他堅持這個使命.
堅持這個使命.
城牆就做了一半.
但四章第七至第八節.
仍然要攻擊他們.
甚至出手攻擊他們.
我想我們做基督徒.
特別在不同地方做基督徒.
求神幫助我們.
無論面對甚麼政治壓力.
面對甚麼環境.
甚至有時會有前面後面.
正面或側面的攻擊.
我們都要知道.
天地有正氣.
神一定會與我們同在.
我們小心肯定工作.
不過最大的問題反而是裡面的灰心.
做到一個地步.
城牆就做了一半.
做了一半就最慘.
為甚麼呢.
因為未到完成.
但有很多灰土.
所以他們怎樣說呢.
灰土上多.
第九節說.
江台的人力氣已經衰敗.
失去意象.
只看到灰土.
看不到做了一半的城牆.
失去力量.
江台的人力氣已經衰敗.
甚至失去信心.
我們不能建造.
怎能建造呢.
做了一半怎能建造呢.

$^{801}$他沒有信心.
失去了安全感.
靠近敵人居住的猶大人.
十次從各處來見我們說.
你們必要回到我們那裡.
你們在這裡就好.
但我們近著敵人.
你們在這裡唯唯畏畏.
不單失去安全.
甚至失去團結.
外面的壓力大.
但怎樣大都不能大過裡面的灰心.
但在這環境中.
李希米最得幸有七件事提醒他們.
特別我沒時間看完這七件事.
第十三節很喜歡一句說話.
所以我使百姓割案.
中族拿刀拿槍拿弓.
站在城牆後邊.
低窪的空處.
低窪的空處是敵人最容易入侵的地方.
低窪的空處.
弟兄姊妹.
你的低窪的空處就是你的弱點.
你知不知道你的弱點在哪裡.
你的弱點是否在那天某個時間.
或是那星期某一日.
或是某一個人對你說的話.
或是某一些試探.
或是某一些困難.
讓你容易灰心呢.
求神幫助我們.
讓我們知道怎樣面對危機.
特別裡面的灰心.
裡面的灰心都是其次.
除了灰心還加上埋怨.
這班人不肯開工.
在第五章第一節.
第一他們說我們要得糧食.
我們要謀生.

$^{841}$然後我們要納稅.
甚至在第五節你會發覺.
我們的女兒已有遺痞.
我們並無力拯救.
因為我們的田地葡萄人已經歸了別人.
弟兄姊妹有時我們在教會裡侍奉也好.
或者我們在教會裡侍奉.
內地的賴院.
我哪有空呢?我要謀生.
你找別人吧.
其實我們是沒有空.
是要謀生.
不過我們回家就看很多Netflix.
看很多電視.
那些我們就不計算.
制度的要求.
這些不行的.
你們的執事你們做吧.
不過最慘的是不平待遇.
有時我們覺得我們在教會裡的服侍.
我們遇到一些我們覺得不平等的事.
我們自己不開心的事.
我們就不想做.
不想繼續做.
求神幫助我們.
當然李希米責備那些貴族.
你看第六節.
責備他們.
你們各人向弟兄取利.
不正確的.
然後你這樣做的時候.
別人會笑我們.
被敵人笑我們.
所以責備他們.
所以求神幫助我們.
不要給內面的埋怨影響我們.
侍奉的態度.
最後這兩個就是前面的失職.
如果你看第五章.
看第十四十五節的時候.

$^{881}$就看到他們怎樣失職.
前面的領袖失職.
但是李希米就不同.
李希米如果看第十五節的下半節.
他因敬畏神的時候.
他不單止不吃猶太省長的俸祿.
甚至到你看看他.
在第十七十八節.
他說每日預備一隻羔羊.
六隻肥牛.
預備一隻飛禽.
每十日一次.
多預備各樣的酒.
雖然如此.
並不要省長的俸祿.
因為百姓服役甚重.
他招待那些人.
不過他吃得絕不健康.
只有肉食.
又飛禽又野味.
又沒有菜又沒有水果.
所以不要學他.
吃得健康一點.
不過當然他只是這樣說.
不是說他沒有菜沒有什麼.
讓他看到他是放棄他的權益.
他和他的人共同生活.
丙姐妹我們在教會侍奉的時候.
我們做領導階層的時候.
我們一定要和弟兄姊妹同甘共苦.
最後第六章第一節至第十四節.
他們差不多完成城牆.
只差門扇未安.
這個森巴拉.
基善.
多比亞.
跟他們說.
你出來出來.
我們聊天.
以前的事.

$^{921}$不要.
假善出來.
想來殺他們都未來.
甚至李希敏不肯和他們同流合污.
不肯出來的時候就造謠.
好像你是造反.
如果你不出來的時候.
我就告訴王子你造反.
甚至叫祭司來恐嚇他們.
你不如進來這個聖殿.
如果李希敏進去聖殿.
他不是祭司怎能進去聖殿.
又是一個問題.
所以外面有很多的威迫.
但李希敏反對妥協.
繼續對抗.
絕對的不逃避.
很有意思.
危機四伏.
我們看到第二個字.
危機四伏.
最近大家可能都看過.
香港教會更新運動的總幹事胡志偉牧師.
寫了一篇文章.
描繪過往五年裡.
這段時間.
因為他們做教會的普查.
香港教會的普查.
用四個字來形容.
正如我用四個字來形容.
這個耳屍貼記.
食完就睡.
我用四個字來形容李希敏.
用四個字來形容香港教會.
是弱不經風.
弱是指教會比較老年化.
不單止老年化.
教會在過往五年裡.
青年人流失超過百分之十.
這件事很大件事.

$^{961}$有人說教會今天只差一代就會extinct.
失去教會的下一代.
明日就沒有教會.
將來的挑戰就是下一代的挑戰.
所以說弱不.
現在很多時候教會不牧不養.
當然我未必完全同意.
但有時教會將牧養教導責任交給機構.
經濟的滑跌.
可能年長的弟兄姊妹.
對教會的支持.
因為退休就變少了.
然後風起雲湧.
政治的風起雲湧.
弟兄姊妹我們未必個個都在香港.
不過我們在危機四伏的環境當中.
我們如何面對外面的譏笑.
後面的攻擊.
裡面的灰心.
內面的埋怨.
前面的生悸.
後面的殘積.
和外面的威逼.
第三個字.
Challenge of Vision.
七至十章.
L.O.V. Vision.
當我們看這段經文的時候.
第七章至十章的時候.
我們看到他們五十二日就建造完成牆.
當然很開心.
開心之餘的時候.
我們看到第八章.
其實第七章開始的時候.
我們看到有很多值得我們學習的地方.
有五樣東西我們可以看到.
這五樣東西是什麼呢.
第一.
在他們建造完成牆之後.
大家願意獻身和獻金.

$^{1001}$另外他們很明白.
如果你看第七章的時候.
有很多人名.
但這些人名.
如果你看以斯拉記.
其實重覆以斯拉記那班尼美人.
那班猶大人.
回去的家譜.
他們的祖先是什麼人.
為什麼這樣呢.
因為你不要忘記.
耶路撒冷周圍都是很多外邦人.
現在他們成牆建立了.
當然要進去住.
進去住的時候有個問題.
你怎麼知道是什麼人.
會不會有些無間道進去.
有些人臥底.
所以我們看到.
他們把這些家譜名字列出來.
極有可能擔心無間道.
這是我的看法.
但如果你看第七章.
第63至64節.
特別提到不單止要有一個純正的生命.
才能在教會或耶路撒冷.
更重要的是要有一個聖潔的生命.
所以在那裡提到有些祭司.
因為他們和外邦人結親.
所以失去了祭司的職份.
所以我們侍奉神.
我們都要有一個純潔.
和有一個聖潔的生命.
第八章我們看到.
眾人如同一人.
他們從清早到下午.
聽以斯拉和文士.
利未人的說話.
同心與留心.
留心聽神的說話.

$^{1041}$同心是很喜歡的說話.
如同一人.
為何如同一人.
因為我們有一個共同的生命.
我們好像枝子連於葡萄樹.
連於耶穌那裡有一個共同的生命.
同心.
我們不是波子彈來彈去.
彈人出去.
我們就像葡萄子.
連結於葡萄樹.
同心與留心.
然後感動第八章第九至十二節.
他們聽到那些說話的時候.
聖經說他們哭泣.
為何哭泣.
因為他們覺得自己失去了.
應該遵守律法.
但尼西米勸勉他們.
勸勉他們甚麼呢.
他們說你們不要憂愁.
因靠耶和華而得的喜樂.
是你們的力量第十節第八章第十節.
我們應該憂愁的時候憂愁.
應該喜樂的時候喜樂.
然後最後第八章十三至十八節.
記載他們守住棚節.
在七月十五日守住棚節.
住棚節是紀念希伯來人.
猶太人離開埃及.
雖然信心少不能立即入到迦勒.
但在四十年抗日裡.
他們仍然得到神的保守.
所以他們每年七月十五日.
都要紀念神的保守.
弟兄姊妹.
我們現在住在一個舒服的房子.
但我們不要忘記.
如果不是神的恩典.
神的供應.

$^{1081}$我們可能仍然在曠野飄流.
所以這是他們意象的一個通心.
再看下去的時候.
第九章有一個很長的祈禱.
這個祈禱在聖經舊約裡.
差不多除了詩篇一百一十九篇之外.
可以說是一個頗長的祈禱.
第九章第五節.
一直到第三十八節.
都可以說是祈禱.
特別是第五節到三十一節.
可以分成四個段落.
這四個段落.
其實最主要的重點.
在這個祈禱裡.
是想說一件事.
你們不需要擔心.
不需要害怕的那些是迦南人.
為甚麼呢?.
因為周圍的迦南人圍繞他們.
這個森巴拉.
阿拉伯人.
阿滿人.
都在他們旁邊.
他們排除萬難.
所以現在他們有一個這樣的祈禱.
要從創世一直到他們秘魯歸回的時候.
雖然過往他們的祖先.
甚至現在他們都是面對這些愛邦人.
但是他們都不需要擔心.
反而要肯定神對他們的揀選.
建立公認.
和對他們的寬容.
然後因著這個緣故.
就有一個異象的更新.
這個異象的更新在第十章第28節.
一直到第39節.
這個異象的更新是甚麼呢?.
第一.
未說所有東西之前.

$^{1121}$先說他們的家庭.
如果你看第十章第30節.
並不將我們的女兒嫁給姐弟的居民.
也不為我們的兒子娶他們的女兒.
即是為了信仰的緣故.
我們不會將我們的兒女嫁給或娶外邦人.
這個很重要.
肯定他們家庭的聖潔.
然後他們強調安息日.
修煉心神.
休息敬拜.
然後自己又訂例.
又顧念聖功.
自己訂例.
你看第十章第32節.
自己訂例.
第34節又顧念聖功.
按照規定去顧念聖功.
然後他們奉上初贖.
贖回投胎.
十章35至36節.
最後當然是十一的奉獻.
十章37至39節.
這件事對他們來說很重要.
因為是整個意象的更新.
時間關係船要差不多到岸.
我相信大家可能都暈船浪都未停.
講得這麼快.
所以一會兒我會再做一個總結.
不過最後如果這個利希米記.
在第十章結束.
或者在十一章結束.
或者在十二章結束.
那就非常好.
一級好.
不過慘在聖經不只是引落陽善.
聖經更加將人的軟弱帶出來.
在第十三章經過了十二年.
我們發覺這班猶太人有一個180度的轉變.
剛才我們講過他們答應和外邦人結親.

$^{1161}$又說要守安息日.
又說要十一奉獻.
又說要自己定例.
將這些奉獻給聖殿.
誰知十二年之後.
首先他們自己的祭司.
自己和外邦人結親.
如果你看十三章第四節.
先是蒙派管理我們神殿中庫房的祭司.
以利亞實與多比亞結親.
上行下效 上良不正 下良歪.
所以因為這個緣故群眾都仿效.
不單止這樣仿效.
如果你看第十三章第十節.
第十三章第十五到十六節.
第十三章二十三到二十四節.
完全是剛才我說的那些.
守安息日 十一奉獻.
奉獻給聖殿的祭司.
完全沒有錯.
弟兄姊妹.
我們不要以為過往的成功.
就一定擔保將來的成功.
挑戰永遠都在我們面前.
今日的挑戰亦是將來的挑戰.
求神幫助我們.
有個endurance.
不過我很喜歡尼西米.
在第十三章十四節.
十三章二十二節和十三章三十一節.
結束和撥亂歸正撥亂反正之後的說話.
十三章十四節.
我的神求你恩者是紀念我.
不要塗抹我為神的奠與其中禮節所行的善.
十三章二十二節下半節.
我的神求你恩者是紀念我.
照你的大慈愛連恤我.
十三章三十一節最後那節.
最後那句說話.
尼西米其實你說他是建造城牆的工程師.

$^{1201}$你說他是波斯王的酒精.
你說他是猶大的省長耶路撒冷的省長.
不是呀.
他更加是一個祈禱的戰士.
對不起我用L O V E.
其實如果我再想想應該用P R A Y.
用PRAY這個字來形容這本書.
因為最後那句說話.
我的神求你紀念我施恩於我.
尼西米是祈禱開始.
也是祈禱結束.
尼西米幫助我們去面對將來的挑戰.
我們是否毀身領導呢.
我們如何面對危機四伏呢.
我們如何更新我們的意象呢.
我們如何更加持久愛主呢.
在持久愛主那裡容許我用一個人物來結束.
大家可能都知道.
一個很出名的神學家J.I.Pekka最近過世.
93歲過世.
美國一個很出名的民權運動的領袖John Lewis.
最近也過世.
得到美國最高的榮譽.
在美國國會很多人都悼念他.
今晚我不行了.
因為現在過了六點.
五點半我會看ABC World News with David Muir.
NBC World News with Lester Hunt.
如果你在香港早上七點半明珠台都可以看到.
NBC World News.
有一位加拿大裔的美國人.
叫Peter Jennings.
他在上個世紀末到今個世紀開始的時候.
他是ABC World News的Anchorman.
他是Peter Jennings.
很帥呀Peter Jennings.
他癌症離開世界.
他離開世界的時候有很多人悼念他.
不過最好的悼念我相信就是這句說話.
When the world doesn't make sense, he does.

$^{1241}$中文翻譯就是.
當這個世界紛亂荒謬不可理喻的時候.
他仍然鏗鏘有力擲地有聲.
各位聽眾.
今日我們覺得這個世界真是荒謬.
紛亂.
甚至有時我們覺得不可理喻.
將來有很多挑戰.
但我們怎樣去面對呢.
我們說的話能夠鏗鏘有力擲地有聲.
最低限度我們都make sense.
為什麼我們make sense呢.
因為神make sense.
在2014年12月.
我在見道十學院的耳道自見.
寫了《尼希米記》.
我就用另一個角度去看.
三,四天裡面.
神是怎樣的神.
神是否你個人歷史的主宰.
神是否你禱告的對象.
神是否你開始的幫助.
是否你建造城牆的啟動者.
是否你被恥笑時的避難者.
是否你事工終結的扶持者.
是否你侍奉約時的賜予和啟動者.
是否你一切的供應者.
是否你一切所有的主權者.
是否你改革的後盾.
甚至是你人生下半場的倚靠.
when the world doesn't make sense.
你make不make sense呢.
求神幫助我們.
叫我們能夠繼續對神有信心.
L O V E.
鼓勵大家喜歡的時候.
因為實在很快.
我相信大家頭都暈了.
我講完了.
不過頭暈完後.

$^{1281}$想更深入了解.
不用買我的書.
去到已到自見.
在左上角的icon.
點擊經卷重溫.
找尼熙偉記.
就可以看到這些東西.
多謝大家.
我停止分享.
交回時間給陸博士.
我知道大家頭都暈了.
對不起大家.
特別是未吃早餐晚飯的.
慘了慘了.
這次頭暈了.
多謝大家.
神祝福你們.
我們吃了飯.
你吃了.
溫哥華可能還未吃飯.
非常多謝陳牧師.
今晚的分享.
我希望大家.
在今晚的講座中.
得到益處.
剛才我都講過.
我們舉辦這幾次的.
《釋經》講座.
一直以來都秉承了.
香港建道.
加拿大建道中心.
一個最重視的就是.
《聖經》的教導.
剛才都提過.
8月11日.
我們請了郭奕宏牧師.
將會講另一個講題.
這個講題更少人講.
叫做.
《從何西亞書看婚姻關係》.

$^{1321}$8月11日是廣東話.
25日是國語.
我們很快會出.
宣傳.
一切詳情.
都可以到我們的網站.
看得到.
加拿大建道中心.
一直以來都是靠弟兄姊妹的奉獻.
來支持我們的運作.
如果大家覺得今晚的講座.
對你有幫助.
請你按上帝在你心中的感動.
來作金錢的奉獻.
奉獻很簡單.
你可以用PayPal.
進入我們的網站.
按支持神學教育.
的奉獻按鈕.
就可以了.
你可以選擇每個月定奉獻.
這是我們最新的發展.
每個月少少的.
20元,25元,甚至50元.
即少成多.
給我們有更加穩定的金錢可以運用.
或者你覺得一次奉獻都可以.
如果用PayPal奉獻.
很簡單.
立即會收到收據.
或者你希望用支票奉獻.
也可以.
支票桌頭是ABCC.
以及寄去我們加拿大建道中心的地址.
在這個螢光幕內也有.
如果你奉獻超過30元.
我們會發一個奉獻收據.
給你報稅.
我們改了每年有兩次發收據的時候.
你們的奉獻是對我們最實際的支持.

$^{1361}$最後我提一提.
今晚的講座是有錄影的.
一兩天後.
你們會到我們的網站.
看到錄影的方法.
歡迎今晚.
來到我們當中的弟兄姊妹.
都可以聽到陳牧師的分享.
再一次.
多謝大家今晚來到我們當中.
願神祝福大家.
阿拉哈.
\newpage



\section{何西阿書}
\label{sec:XLKUZGl9ItY}
\textbf{【從聖經書卷看生命實踐系列】 主題(三) 從何西阿書看婚姻關係 (AB2008)}
\newline
\newline
連結: \href{https://youtube.com/watch?v=XLKUZGl9ItY}{\texttt{https://youtube.com/watch?v=XLKUZGl9ItY}} ~~~~ 語音日期: 2020-08-12
\newline
\newline
\hyperref[sec:P0Y2lvzICsM]{\small{< < < PREV SERMON < < <}}
~
\hyperref[sec:index]{\small{[返主目錄]}}
~
\hyperref[sec:n5DpA1Db_0M]{\small{> > > NEXT SERMON > > >}}
\newline
\newline
$^{1}$建築中心總幹事陸超明牧師.
今日很高興在網上跟大家見面.
今晚是我們.
自從疫情之後.
從聖經書本看生命實踐系列的第三講.
已經說了何啟明牧師的《以詩帖記》.
陳耀朋牧師的《離希米記》.
今晚我們講的是.
《荷西亞書》.
歡迎大家到來.
我們收到的訊息是今晚參加的人.
是遍佈全世界的.
遠至一些偏遠的地方,例如南半球等等.
聽到這個訊息來參加我們的講座.
相信有些人不知道加拿大建築中心是甚麼.
我用一分鐘簡單介紹一下.
加拿大建築中心的基地在多倫多.
加拿大多倫多.
我們是一個神學教育機構.
我們有一系列的課程.
無論是證書的等級.
文憑,MA,MCS等級.
歡迎大家到我們的網站.
可以看到我們全套的資料.
我們的網站是www.abscc.org.
你會看到我們一切的訊息.
還有你到Facebook的網頁.
會看到Fanpage.
The Alliance Bible Seminary.
Centre of Canada.
就會知道我們最新的狀況.
我們的基地在多倫多.
我們和香港建築神學院有合作關係.
建築神學院的老師經常會來我們這裡.
實體教學.
疫情之前是實體教學,現在轉為網上.
希望疫情過後.
實體教學的情況會陸續恢復.
我們今晚的講員是郭奕宏牧師.
剛剛在香港建築神學院退學.

$^{41}$其實他退學不等於退下去.
名義上是退學,但他還有很多會後的工作在進行當中.
以前他就是海外事工.
拓展部總監.
有很多合作的機會可以接觸他.
郭牧師最有名的專題就是.
和婚姻關係有關.
如果你看到我們的宣傳.
有一個簡單介紹郭牧師的經驗.
都是在婦道.
婚姻的範疇.
甚至到中國.
由海外事工拓展出來.
歡迎大家趁著他在這裡的時候聽聽他的專長講座.
我們希望.
能夠藉著這個講座.
大家不單認識何西亞書.
而且更深刻認識何西亞書.
如何看婚姻的關係.
在開始之前,再說一句.
這個講座是廣東話的,是粵語進行的.
郭牧師也答應了我們將這個講座.
兩個星期之後.
在25號星期二晚上.
同樣時間.
會用國語再說一次.
如果你們教會有國語部.
請你們先在口頭上.
跟我們說一聲,讓那邊的負責的弟兄姊妹知道,我們很快就會出.
宣傳的版本.
關於何西亞書和婚姻關係的國語的部分,在25號,即是兩個星期之後.
聽大家.
幫忙.
做一個口傳的情況.
最後,在那裡見到中心,因為疫情已經關門差不多五個月.
很開心,在神的恩典之下,我們決定了.
在8月31號正式重啟.
恢復我們的運作,其實我們一直沒有停過,只是在家裡運作.
8月31號就開始.
開辦公室.

$^{81}$不過我們的網上教學.
仍然會持續到.
起碼2020年12月底.
這個.
最新的消息也告訴大家.
希望大家也在討論裡.
支持我們,這個工作實在在疫情之下相當不容易.
不過我們也經歷到上帝.
很豐富的恩典給了我們.
這個時候,我們就恭敬的將.
這個講座交給郭奕弘牧師.
郭牧師,你簡單介紹一下自己,我可能忘記了你的.
豐富的資歷.
將時間交給郭奕弘牧師.
各位電影節目平安.
今天有這個特別機會在這個地方和大家透過網上一起見面.
如果是不久之前見過我.
希望你還認得,因為退休這幾個.
三個星期.
搬了.
回多倫多的Kitchener一個鄉下的地方住.
很安靜,寧靜.
有些震撼,對我自己來說,從一個很嘈吵的.
香港回來.
你看見我的圖畫,其實就是我後院的一個水塘.
很安靜,有些人看到你.
在安大略湖,其實不是一個湖,是一個水塘.
給我很多機會可以安靜,同時在不同的地方享受.
當然,所謂退休就好像我最後的文章.
在見到的通訊上寫的.
退休不退遠,這裡說退休,工作就陸續來了.
所以現在我.
見到一份工作就變成有四份工作.
但是都是要到處走,其中一個比較重的工作.
就是準備去歐洲幫助一些神學院.
做一些托錢,讓他們可以串連起來做更大的工作.
所以我前年疫情過後就開始要出國.
這次有機會和大家在這個地方來到.
講關於這個荷西亞書.
漢昏因關係.

$^{121}$雖然荷西亞書不算長,有14章的經文.
但是要在50分鐘裡面.
講完整個書卷.
特別是用釋經的方法.
再講婚姻關係的話.
幾乎是沒有什麼可能的.
所以我就很簡略地會提.
提關於聖經荷西亞書裡面的背景.
但當中有幾點,特別是荷西亞的婚姻.
在家庭裡面要看看.
有什麼原則的東西可以用.
希望不會太牽強,先解經,然後從領袖方面.
各自可以來到領袖.
每次看荷西亞書的時候,大家都會問同一個問題.
問題就是.
到底荷西亞先知的經歷是不是真的呢?.
似乎和上帝對婚姻的要求有矛盾.
因為神對婚姻的性結觀很重要,標準很高.
一開始第一章第二節就說.
耶和華初次向荷西亞說話.
耶和華對她說:你去娶一個淫蕩的女子為妻.
收納從淫亂所生的兒女.
因為這地大行大淫亂,來欺耶和華.
有些人會問,怎可能會這樣叫一個神的僕人.
一個先知去娶一個淫蕩的婦人呢?.
淫亂在原本的意思裡面.
甚至是說不當的性行為.
婚外性行為.
在宗教上的不忠也可以用這個字.
特別是用了眾數來形容淫亂.
意思是說,是反覆不斷的.
所以,為何上帝會吩咐他的僕人先知去做這件事呢?.
於是有很多不同的猜測.
有些人會覺得,這是不是一個靈異的事情呢?.
一個故事式.
來襯托出,講解神如何挽回不忠的意識鏈呢?.
我覺得這樣去看,是相當困難的.
因為,這樣看,目的只是想解釋,我們沒辦法明白的地方.
所以,我們經常有些經歷不明白的時候.
我們就嘗試用很多不同的方法.

$^{161}$來合理化和邏輯化.
這樣的話,是很危險的.
因為這樣的話,我讀的聖經和你讀的聖經不一定是同一本.
在上一代讀的聖經和下一代讀的聖經,也會有不同的內容.
因為每一個情況都不相同.
如果我個人來看,從整個聖經來看.
我們可以化解到,何西亞事實上是一個真正的人物.
他的家庭,他的婚姻,也有真正的處境.
為何他會這樣做呢?.
我真的不知道.
不過,不知道是因為我們不是活在當時的情況.
如果我們活在當時的情況,可能我們會更加了解.
首先我們要明白的是.
在聖經裡面提到何西亞的妻子,甘蜜.
二女,耶斯利.
羅路哈瑪.
然後羅亞米,全部都是真實的名字.
甚至於一章第三節更加指出甘蜜的父親是狄拉欽.
即是說,是有血有肉的歷史人物.
如果只是靈異地表明神對耶穌的愛國而冠以污名的話.
我想這就更加難明白.
我又要想其他辦法去解釋.
所以我越解釋越糊塗.
我相信這就是一個真實的故事.
為何要擺在這個地方,我們不知道.
不過我們可以從當時的背景了解.
他們的婚姻狀況如何.
第一章第一節的交代.
何西亞,當烏西亞約坦阿哈斯希西加作猶大王.
約阿斯的兒子耶羅邦安作以色列王的時候.
耶和華的華林道比利的兒子何西亞.
這裡提到有四個王是猶大王.
然後耶羅邦安是北角的王.
當然大家應該了解耶羅邦安是第二.
也是很快見到以色列將會滅亡的最後幾位皇帝.
這個何西亞本身是北角的先知.
但他所對他們說的話.
反而特別強調南角這四個王.
其實這是一個原因.
北角已經去到一個滅亡.

$^{201}$他寫在大概是750至720年之間.
幫以色列統治到滅國之前.
這是一個什麼處境呢.
從歷史看見.
當時南北角處於一個政治平靜.
經濟豐裕的昌盛期末.
不過他們當然不知道是末期.
亞述國漸漸成為超級強國.
以色列就走向滅國的命運.
而在公元前743年.
亞述帝國攻下加利利時.
米拿彥就用大批金銀向亞述進貢.
希望能夠討好他的歡心.
以後幾年.
米拿彥發現更加困難.
他糾纏在埃及和亞述之間.
一會兒親埃及一會兒親亞述.
而將他和上帝之間的關係已經完全忘掉了.
這是北角以色列離棄上帝.
甚至到滅亡的一個重要因素.
所以在這個逾期之下.
理科學在整個宗教都同樣很腐敗.
耶和華的敬拜和迦南地的宗教.
混在一起糾纏不清.
迦南地的敬拜是用很多巴力的敬拜.
巴力是雨水農作物的神.
巴力的配偶亞納.
亞納是性愛繁殖戰爭的女神.
亞斯塔諾也是女神.
所以被當作巴力的配偶.
究竟是誰呢?.
其實分不清,很混亂.
當時很多神廟.
都是特別講到掌管生育,繁殖,賜福等等.
所以迦南的神廟裡.
常有神祭.
這類神祭是和很多男性來參拜的人發生性交.
為什麼這樣做呢?.
透過性交認為神會祝福他.
甚至盼望能夠生育.

$^{241}$當他生育時證明他們在那裡得到生養眾多.
這種福氣,能力.
於是很多女生一生中會在婚前履行.
甚至後來最多一次還願.
所以這也是當時荷西亞先知的背景.
聖經學者說,在當時迦南宗教的影響下.
大概很少有女性在新婚之夜仍然是貞潔的.
這樣看下去,和我們今天去看.
你會發覺是兩回事.
今天雖然是性解放的世界.
但當時宗教和性的結合.
令到那種腐敗.
其實是隨處可見.
所以當我們要了解荷西亞書時.
我相信先知實在是取了不貞潔的女子乾月為妻.
因為聖經很明顯地告訴你.
當神要向以色列發出責備和怨恨時.
先知就用自己的婚姻家成的經歷.
如何將一個不貞潔的,不忠的妻子.
如何挽回和重建她愛的關係.
從而能在這個地方顯明神對以色列的不離不棄愛.
這就是整本書的背景.
今晚我們不會逐節解釋.
究竟經文如何說,解釋如何.
我們反而說,在這個懸殊的情況.
和一個不貞潔淫亂的女子.
如何能在後期說到她的恩愛,輕輕愛愛的情況.
一會兒你會看到,過程很傳奇性.
這個傳奇性主要是想顯出神對以色列的不離不棄愛.
所以整本書很簡單來說.
分為兩大段.
一到三章是先知親身說出自己和妻子的經歷.
家庭的狀況.
讓我們明白神對以色列的深長.
第四至十四章,比較多的份量.
在背後提及以色列傳講的訊息.
大概有十篇訊息.
想告訴以色列,她和神的關係如何.
最後落到如何被挽回的地步.
主題很明顯地在第四至十四章.

$^{281}$內涵是,我們看荷西亞書的人.
一般印象只記得荷西亞娶淫婦為妻.
其實是想見證神的愛.
但反過來用故事的個人經歷成為主題.
甚至成為我們困擾.
這是我們今天教徒讀聖經時常常出現的困擾.
我經常想起我們去傳福音時也是這樣.
當我們去傳福音時總會說見證.
去報道會時總會提及信主得到的經歷.
我們在困難中經歷上帝的幫助.
我們想告訴人神的能力.
講到神的慈愛.
然後讓他們認識.
我們出現的難處是.
反而成為整個福音的主題.
反而我們成為福音的內容.
好像不是走我們這條路的人認識主.
這是值得提醒.
荷西亞書想講神對背道的以色列.
如何不離不棄愛.
最終如何挽回他們.
這是主題.
剛才律師也提及.
我做婚姻輔導已二十多三十年.
一直在做婚姻家庭.
當我看這本書時.
當然也同樣被荷西亞家庭.
那件事引起我的注意.
如果要跟一個不忠淫亂的人去娶她.
甚至有學者說.
他的兒女不一定是荷西亞的兒女.
所以在這裡提及不忘憐憫的兒女.
不忘憐愛的人.
你想想如何過一個婚姻家庭的生活.
這引起我很多注意.
當然我們要避免.
用一些智女行間錢.
搬字過紙成為我們的子嗣.
不應該這樣做.
但從當中我們可以理解.

$^{321}$究竟荷西亞的心腸是怎樣.
她做了什麼.
能夠重建她的婚姻.
這同時告訴我們.
上帝做了什麼.
能夠把背道的意識從新挽回過來.
這是我們今晚想講的訊息.
在這個干涅的背景.
很快就看到第一章第二節.
他們經常問.
究竟這個人是婚後出場.
紅杏出場.
還是本身就是妓女.
甚至有些人以為她是妙妓的身份.
我們可以肯定.
干涅並非外邦宗教的妙妓.
因為稱呼她為淫蕩的這個女子.
希伯來文和專稱妓女的字.
是兩個完全不一樣的字.
所以我們說.
在智女行間去看的時候.
我們認定.
不是指她是一個以賣淫為生的人.
她不是一個義家的妙妓.
而是一位對婚姻不忠的婦人.
所以這個不忠.
甚至可能在他們結婚之後.
仍然是這樣.
所以多次形容她這種淫亂的行為.
是一個很難想像的.
即是跟她有什麼關係呢.
我想在我們的圈子裡.
不要說教會.
甚至我們認識的人.
或者都不會隨便經常在這方面.
在一些性的不忠裡.
背棄他們的配偶吧.
所以經常問.
究竟這個經文有什麼可以提醒我們.
我只想提醒.

$^{361}$原來何西雅所面對的處境.
比她難得多.
是大到一個我們不可以想像的情況.
不單是淫亂.
而且後來再淫亂.
生了出來的.
是私生的兒子.
不是何西雅的兒女.
但何西雅說我仍然當你是妻子.
當你是我兒女.
你都這樣看待.
究竟你怎樣做這件事.
而從當中問.
如果我今天只是跟我的妻子.
在意見上不合.
只是我跟她在情緒表達上不相同.
我們是習慣不相同.
於是我們互不往來.
究竟我們做了什麼.
為什麼一個先知可以跨越這麼大的鴻溝.
而我們只是小小的東西.
我們快到將我們的婚姻拖垮.
這也是值得我們問的一件事.
我覺得這個綱領.
肯定是婚外情.
婚姻不忠的婦人.
正如我提及.
她的婚姻不忠是在肉體上.
性上的不忠.
出賣她的丈夫.
但其實婚外情.
其實今天也是一個.
不是很常見的社會情況.
我們婚外情被視為今天在婚姻裡.
五大威脅之一.
這五大威脅我不會說完.
但婚外情其實到處可見.
你會說不是吧.
在乎你怎樣定義婚外情.
婚外情的定義有相當大的不相同.

$^{401}$但會發生在我們當中.
甚至發生在我們自己身上.
在這個學者一直探討婚外情.
因為有些人說婚外情已經殺到埋身.
在很多婚姻裡面.
不多不少都見到婚外情的影子.
那什麼叫婚外情呢.
婚外情有三個不同定義的說法.
第一個是在傳統性裡.
婚姻關係裡的夫妻.
任何一方和第三者發生性關係.
這個傳統的說法.
這個叫婚外情.
或者換另一個角度說.
以發生非婚姻性關係的男女當中.
至少有一個人是已婚.
這個就是婚外情.
即是說這個傳統定義.
都是以婚外性關係.
當作婚外情的基本界定.
涉及婚姻外異性感情關係的一方.
常以為自己都沒有發生性關係.
所以不算有婚外情.
當他的配偶說他有婚外情的時候.
他拿著這個定義.
他問自己何時有婚外情.
何時見到自己和別人有性關係.
這個就佔有傳統性.
但在合二性裡.
越來越多人開始問.
其實同樣傷害婚姻裡.
不一定要在肉體上.
才是當然是肯定.
但合二性就是.
在婚姻外情一般是說.
夫婦之間出現第三者.
可能是男性或女性.
無論只是情感交往或性活動.
有愛有性.
或有愛無性.

$^{441}$或有性有愛無愛.
這都算是婚外性.
即是有一個人夾在你們夫妻當中.
一直成為你們在一個地方.
糾纏不清.
不一定是性上的關係.
如果整個心都已經向著一個人而去.
而背棄了自己的朋友.
這肯定是婚外情.
第三樣更加講義性.
越來越多人覺得.
原來去到有性的接觸.
之前有一個人會得到你的注意.
將你們在婚姻裡的注意力轉移了.
但在這之前.
甚至在婚姻關係裡都有跡象.
告訴我們原來夫妻已經出現難處.
在婚姻以外的一切情事.
這些情事可能發生在人與人的關係上.
亦可能出現在不同的嗜好或活動當中.
就是從各種不同的活動和嗜好得到滿足.
取代了婚姻關係裡應該有的滿足.
這個意思是甚麼呢.
意思是.
可能沒有任何一個人的出現.
簡單來提.
婚外情就是婚姻之外的情事.
這叫婚外情.
為何婚姻之外有甚麼情事呢.
原來我們發現婚外情的出現.
很多時候是在婚姻以外尋求滿足感.
當夫婦已經不成為彼此的滿足.
無法談話的時候.
每天回家就像交人交職事交責任一樣.
談滿足是完全沒有.
於是開始要在婚姻關係以外.
找第二樣來填補滿足.
其實這已經是婚外情的開始.
甚至在逃避艱難的婚姻關係.
當夫妻在溝通上或是責任上.

$^{481}$很多處理衝突都無法解決.
於是想逃避婚姻關係.
去到另一種處境.
不一定是人的.
有些人成為一種習慣.
一些活動.
逃避艱難的關係.
第三.
婚姻裡不負責任的結果.
本來這是我應該在婚姻裡的責任.
但我覺得我不想承擔.
於是逃避責任.
這已經是婚外情的跡象.
因為承擔過多的義務或責任.
而導致身心不平衡不孤立.
所以如果在這麼廣義的婚外情定義來說.
很容易我們已經落在婚外情當中.
一定有第三者.
所以我們說婚外情可能是一種運動的習慣.
可能是一個嗜好,娛樂.
甚至一些教會的參與.
原來你不是真愛主愛弟兄姊妹.
而是想逃避婚姻裡不要讓老婆餓得多.
你回到教會工作.
人家又稱讚你.
牧師又喜歡你.
多開心.
很開心的一個地方.
享受著你的滿足感.
其實這是逃避婚姻的出現.
不要小看這些.
如果你說這樣就每個人都有.
在乎你如何將這些活動和婚姻作一個平衡.
如果你真的逃避婚姻去做一切.
弟兄,姊妹.
我想提醒你.
真的要小心.
因為當這已經成為一種習慣逃避婚姻的時候.
慢慢第三者出現的時候.
會拯救你.

$^{521}$會給你更具體的享受那種滿足感.
而且這種婚外關係會帶來婚姻裡更多的不滿.
推你去更遠.
所以在整個不忠的背後.
你會發現常常有情感的外溢.
在婚姻之外找滿足感而出現.
所以如果是這樣的話.
我就要問自己.
究竟我應該如何避免一切.
其實婚外情成為一個很大的衝擊和危機.
有統計上.
有時真的很難的.
就算西方也很少將婚外情作一個統計.
有多少人有婚外情.
不過從他們輔導的過程中.
我們可以知道有多少牽涉到婚外情.
反過來計算.
歐長江博士在香港很有名的一個輔導專家.
他用了美國對婚外情的統計數字.
他如何統計呢.
他說尋求婚姻輔導個案中有25-30\%.
在婚外情被發現後尋求輔導.
他就知道這個統計.
另外大概有30\%在婚姻輔導過程中被揭露有婚外情.
他是帶著夫婦關係的問題.
然後去尋求輔導.
但一路揭露就發現原來不是婚姻本身.
而是婚外情成為了阻隔.
在那份報告中同時發現.
婚外情輔導個案中最終有38\%會分開.
有62\%可以被挽回.
所以有38\%會成為婚姻利益的主因.
香港名外向晴軒輔導機構的督導主任叫作郭思英女士.
她曾經寫了一篇文章.
叫做婚外情與家庭的危機.
她寫的文章中有一個很有趣的部分.
她說婚外情問題支援服務在2001年期間.
透過電子剪報方法分類搜尋.
發現由1997年至2001年期間.
在香港總共有96宗本地新聞.

$^{561}$跟婚外情有關.
其中涉及37人死亡和65人受傷.
很奇怪的,在東方涉及婚外情.
常常有死傷發生.
在西方比較少這件事.
當然這是另一個話題.
她說自2002年至10月至2005年.
因為婚外情引起的夫婦衝突中.
需要使用向晴軒的短暫緩衝避症住宿服務.
大概有335人.
其中導致家庭暴力的有118宗.
佔整體35\%.
而涉及自殺的個案出有16\%.
總共有54宗.
所以婚外情不論在不同的社區文化當中都在發生.
有時導致婚姻的離異是跟婚外情有直接關係.
在中國的統計數字.
在離婚的數字中.
離婚率不斷上升.
在離婚的個案中.
大概有三分一是因為婚外情而造成離婚.
所以我們不只是在說婚姻到破裂的一刻.
然後再說離婚的問題.
其實作為信徒.
我們都要問一問.
究竟我們的婚姻向哪個導向.
我們會懂得為著我們的信仰.
為著我們的家庭.
我們會自虐.
但我們沒有解決婚姻的難處.
我就想問.
究竟我們的問題是否比何西少呢.
這個在信徒當中的婚外情情況.
其實是相當難去統計.
因為很敏感.
沒有人敢去問這個問題.
但從輔導的數據反過來問的時候.
我在香港九年.
這九年除了在建道的托展事工外.
我亦經常去談婦婦的工作.

$^{601}$婦婦營.
其中一個工作是建不同的輔導.
因為在輔導室裡有很多傷痛和眼淚.
當然在這種環境我不會透露任何個案.
不過我可以讓你知道整體的情況.
我所接觸的個案是越來越多.
在這九年當中見的個案不下一百人.
而全部一百個個案.
我都是做婚外情.
所以慢慢好像每個人都有婚外情.
就想起我.
就當我是專家一樣.
但其實我只是說.
在這件事上.
嚴重性很厲害.
我接觸的圈子大部分都是信徒圈子.
包括幕者在內.
所以在當中一直困擾我.
究竟有多少的基督徒.
在婚外情上不自覺地.
從夫婦開始沒法溝通.
甚至面左左.
去到一個地步.
找第二樣東西替代.
慢慢第三者的出現.
要走到一個這樣的困局.
李如泉牧師曾經發了一份文件.
90份文件給信徒.
當中有15個人.
16.6\%表示正在面對婚外情的困擾.
這些全部都是基督徒.
基督徒在面對婚姻危機的時候.
所以在他那本書最後有一個結論.
多數落入了一個不知所措的困境.
他們在這種事情裡.
如何才能夠持守信仰呢.
在教會裡.
如何正面對誰正面對著.
這婚姻的危機呢.
他們又是否得到適當的輔導呢.

$^{641}$教睦和公.
如何才能幫助他們面對危機.
在婚姻關係上作出防衛工作.
何西亞給我看完的時候.
我很害怕.
如果一個這樣的婚姻.
來到你面前.
你如何幫他處理呢.
但當他退一萬步去看的時候.
難道今天很多信徒都走同一個方向嗎.
當我們不去處理今天婚外情的時候.
結果是什麼呢.
結果就是一個死的婚姻.
有位婚姻家庭學者.
曾經在紐約時報寫了一篇關於離婚的文章.
題目叫做「未離婚」.
叫做The Undivorced.
即是說不是Divorce.
是Undivorced.
是一個創新的字眼.
這個專家的說法.
即是說未離婚的人.
不等於他不會離婚.
但他又不是離了婚.
即是說是情感上的離婚被稱為Emotional Divorce.
這種情況其實在華人圈子.
特別是在信徒圈子裡特別多.
學者指出.
這個情況是叫做沉靜的趨勢.
Quiet Trend.
即是說大家都不出聲.
大家保持制約.
保持冷戰.
你不要動我,我又不要動你.
你不要告訴別人.
大家就這樣過活.
有些人等什麼呢.
等到子女長大.
有些人等到你死.
一會兒我亡.

$^{681}$在這種婚姻的過程當中.
難道容易得過何西亞的情況嗎.
我覺得這也是一個很大的困局.
究竟我們應該怎樣面對.
我們不要用何西亞的情況來說不關自己事.
我只想說在每一段婚姻中.
都面對著前所未有的挑戰.
我們未必一定去到離婚的階段.
未必來到第三者.
又在性上的不忠.
肉體的控制不是最難的事.
反而是人的情感的歸宿.
是一個最難找到的地方.
當夫妻在同一個地方.
睡在同一張床上.
而無法親密.
無法有溝通.
一起生活的時候.
那日子難道好過嗎.
不好過.
在這個地方.
我們就要回頭去問.
究竟何西亞怎樣能勝過這件事.
將一個在婚姻上.
甚至可以說是不可能和他一起生活的.
但神竟然要他將他娶過來.
娶過來不是叫他放在一邊.
是要他建立一個愛的關係.
同樣再說.
這也是要顯出他的方法.
如何收納和重新建立.
和以色列的愛的關係.
我簡單從剩下的十幾分鐘.
我看到三件事.
是何西亞做的.
也是上帝顯示給他看的.
第一件事是何西亞書第二章二至十三節.
開始的時候.
耶和華和何西亞是一個控訴者.
戈密和以色列民眾是一個被告.

$^{721}$兩個不同的角色.
這篇經文應該不算太長.
我讀一次.
你們要與你們的母親大大爭辯.
因為她不是我的妻子.
我也不是她的丈夫.
叫她脫掉面上的淫帳和空間的淫態.
免得我脫掉她的衣服.
使她赤體.
以財生的時候一樣.
使她如抗旱之地.
因渴而死.
我必不憐憫她的兒女.
因為他們是從淫亂而生的.
他們的母親行了淫亂.
隨他們的母親做了可羞恥的事.
因為她說我要除從所愛的.
我的餅,水,羊毛,麻,油,酒.
都是他們給的.
這篇經文說的是赤裸裸地.
將干滅的行為放在日光之下.
放在兒女面前.
甚至控告他的妻子.
這個爭辯是與母親爭辯.
這是法庭的用語.
指控指出母親的那種不是.
她不是我妻子.
我也不是她的丈夫.
古代人是離婚採用的公式用語.
她不再是我妻子.
我不再是她的丈夫.
赤裸裸地將婚姻關係撕裂.
叫她除掉.
直譯是剝掉.
淫障淫態原文是複數.
用來指出她的神態神情.
也可能是代表妓女或神妓身份的裝飾品.
所以是赤裸裸地將她的不是公開出來.
當我看的時候.
我經常會忽略了這個點.

$^{761}$我會跳越了這一點.
因為這本是聖經.
這是講愛的故事.
神如何吩咐何西亞去愛一個淫亂的妻子.
所以我們很快會跳到.
因為愛改造一切.
沒錯,愛始終是最重要的答案.
但原來在未愛進入之前.
神的公義顯明.
顯露無遺.
神在一個地方.
沒有否認以色列對她的不忠.
何西亞沒有逃避指出那種行為的不是.
我想說的是.
今天好的時候.
我們將婚姻的難處.
用愛像一張皮帶包著就算了.
大家都不要看它.
收起來放在地氈底.
因為我有愛.
於是我用愛去忍受.
但心裡一直痛直至滴血.
何西亞在建立愛關係之前.
要將那個情況清楚放在面前.
這個可以算是慈愛和公義吧.
同樣地.
神對以色列也是這樣.
我們沒有時間讀那一段.
下面那一段就是神直指以色列的不是.
甚至告訴他們會停止所有的節日.
所有的節日的安息都沒有了.
享用的東西都沒有了.
因為你離棄我.
所以上帝沒有一跳就像是抹殺了所有的錯誤.
我想說的是.
當我們今天要面對婚姻的難處的時候.
起碼你知不知道你的難處在哪裡.
當你想重建一個美好的婚姻的時候.
你敢不敢去面對裡面的痛楚.
我們會很不自覺地用一個愛的床單蓋住它.

$^{801}$就什麼事都沒有了.
我們沒有辦法去重建.
我們沒有辦法去處理那個關係.
所以我在這個地方想提的就是.
弟兄姊妹不要去逃避婚姻裡面的困境.
可能你的配偶對你不忠心.
可能甚至你已經去到一個大家在冷戰裡面.
什麼原因你知道嗎.
你有沒有曾經將你在婚姻裡面的困難.
能夠表達給對方聽.
如果連這一點都沒有.
你怎樣去期望能夠在這個地方處理到你們婚姻裡面的難處.
我很強烈的感受.
原來這是一個愛的書卷.
在愛裡面又這麼殘酷.
甚至我也想不說這一段.
怎樣去脫掉他的衣服.
怎樣赤裸裸地去指控他妻子的不是.
不是說愛嗎.
為什麼要說這麼嚴厲的指控.
說情罰.
原來在愛之前.
在這件事是不能去缺乏的.
第二樣東西.
何西亞怎樣面對不正確的干預呢.
同樣神怎樣挽回不忠的意識呢.
第二樣就是付出無條件的愛.
當何西亞在這個地方指出這個行為的時候.
他所應得的一些報應或審判的時候.
第二章第十四節.
然後轉了另一個180度轉變.
那段聖經怎樣說呢.
在和修版裡面.
因此漢我要有道他.
和合本說後來我必勸導他.
令他道抗也.
我要說動他的心.
和合本說對他說安慰的話.
在那裡我必賜他葡萄園.
又賜他亞各谷作為指望的門.

$^{841}$當他面對婚姻的困境.
他的配偶那種不忠之後.
他給的是愛.
在聖經裡面特別是和修版裡面.
有道和勸導有很大分別.
有道是在聖經裡面提到.
有道是一種循循善誘.
不是用一種審判的語氣.
來讓他取捨做選擇.
而是一路幫助他進入困境.
明白難處在哪裡.
這叫有道.
然後說動他的心.
和對他說安慰的話有分別.
原本文者的意思是.
在他心裡面說話.
指的是戀愛人當中的甜言蜜語.
剛才才說他那種不是.
很嚴厲的責罰.
但他說你要留心.
上帝叫何西亞去做這件事.
不是等他妻子來面前去懺悔.
同樣地耶和華也是這樣對以色列.
所以在第四節.
有另一節聖經能夠刻畫這種感情.
就是耶和華神說.
我用慈,誠,愛,索牽引他們.
我代他們如人鬆開牛糧篩旁邊的鴨.
彎下身來餵養他們.
剛才耶和華神不是說要怎樣審判他們嗎.
不是說他們那種不是嗎.
要停止他們所有的節日嗎.
沒有安息嗎.
這都是公義連鎖彰顯的一切.
但更重要的是.
慈,愛,道的時候.
是由耶和華主動招攬以色列.
用慈,誠,愛,索牽引.
這就是關鍵的地方.
我們常常有一個錯誤的地方.

$^{881}$誰做錯事就誰要負責任.
你得罪我就要對我講道歉.
否則就沒得談.
起碼你一定要先做一些事.
何西亞不是這樣.
耶和華神也不是這樣.
原來主動地用情人的心來跟他們說話.
甚至下面說了一句話很有意思.
由次灘阿國谷作為指望的門.
阿國谷在約旦平原.
在耶利哥城西南邊的地方.
在約書記七章記載.
阿干因為貪財被處死.
所以阿國谷也是審判谷的意思.
這個審判谷卻變成了指望的門.
沒錯,一方面他的行為帶來的被審判.
被責備是完全責無旁貸.
我們是要這樣做的.
但是他提醒他.
這個阿國谷會成為一個指望的門.
就是這個審判是甚麼呢.
是祝福和交會的地方.
以色列人在阿國谷執行了對阿干的懲罰.
同樣地重新獲得艾城戰爭的勝利.
弟兄姊妹,原來這麼奇妙的地方.
不是公義去進行審判之後得出重建的關係.
公義是將他放在一個日光之下.
讓他知道他坐在那裡.
但是是誰付出.
是被得罪的人甚至用愛來挽回這個關係.
所以這個就是寶貴的地方.
他必在那裡回應.
像年輕時從埃及上來的時候一樣.
那日必稱呼我伊斯.
不再稱呼我巴力.
這是話說的.
回應是那日必稱呼我伊斯.
即是不要再稱呼我巴力.
伊斯即是男人丈夫的意思.
因為愛臨到的時候.

$^{921}$這個哥明願意心甘情願地對著何西雅說.
你是我的伊斯,是我的丈夫.
這種形容.
有人形容就像婚禮中妻子對著丈夫說.
我願意,我願意.
是一個甘心情願地將他自己重投在恩愛關係裡.
是甚麼造成呢.
不是責任.
是甚麼造成.
是愛造成.
是誰的愛.
是被得罪的愛.
如果今日在夫妻關係裡面去到緊張的情況.
讓我給你一個鼓勵.
弟兄姊妹,你被傷害嗎.
弟兄姊妹,你覺得你被疏忽了嗎.
你被虧待了嗎.
讓你將這個被虧待放在面綱當中.
我們一起去處理.
但更重要的是.
愛一定要臨到這個關係裡面才能處理.
何西雅書三章一節.
蓋爾華對我說.
你去愛那情人所愛卻犯姦淫的婦人.
正如耶和華愛那偏向別臣.
喜愛葡萄餅的以色列人.
在和學本,蓋爾華對我說.
你再去愛一個淫婦就是她情人所愛的.
好像以色列人.
雖然偏向別臣.
喜愛葡萄餅,要說還是愛他們.
在這裡首先要解釋.
這個地方再去愛一個淫婦.
有聖經學者認為這是一個載詞.
其實是連絡上的耶和華再對我說.
不是去再愛一個淫婦.
所以在和修本改了.
叫做耶和華又對我說.
不是叫我再愛一個人.
要明白.

$^{961}$在這個地方,去愛那情人所愛.
一句說話裡面有兩個愛字.
但很奇妙.
這兩個愛字一句說話前後.
你去愛那個情人所愛.
但這兩個愛字是兩個完全不一樣的希伯來文.
第一個愛字是神吩咐何西雅去愛戈敏.
這個愛字是指意志性的愛.
在意志上的.
在理性上的,在關係上的愛.
是一個毀身的愛.
第二個愛是情人所愛.
是一種喜愛葡萄餅.
葡萄餅是敬拜巴黎時所吃的餅.
同樣是一個愛字.
但這裡是說情慾性的愛.
主要是牽涉到肉體的感官方面的愛.
我們分愛情絕對肯定.
是情慾性的.
有時是心靈上,情緒上甚至肉體上的需要.
但在婚姻裡面是一個理性毀身的愛.
所以當神吩咐何西雅去跟該人說話時.
她說你愛他,不只是情感上.
你不可以只得著情感上的關係.
那種感受要維持你們的關係.
你必須有一個毀身.
這也是第三個原因.
神如何挽回以色列.
當她挽回以色列時.
她說:像在年輕時從埃及地上來的時下一樣.
很寶貴的一樣.
好像她以前年輕時.
好像她離開埃及時那種恩愛.
我就是她唯一的神.
她一直抱著我,我不會放過她.
這是一個盟約.
是過往的毀身.
盟約成為婚姻關係上被忽略的一環.
這就是我們今天說.
很常我們婚姻裡只有契約.

$^{1001}$沒有盟約這東西.
盟約成為今天很多學者都經常問.
Anderson和Dennis在書裡提到一個很有意思的話.
婚姻盟約裡的性質差異造成婚姻的不穩定性.
他強調婚姻裡的盟約的重要性.
甚至將盟約關係和血緣關係做了一個比較.
論到家庭這本書.
On Being Family這本書裡.
將家庭界線甚至超越了血緣界線.
是甚麼意思呢?.
他說從盟約書中的無條件的愛.
那種特性了解家庭的意義.
遠超過血緣的關係.
血緣關係只是提供家庭歸屬感的唯一基礎.
家庭是你可以無條件被被愛.
甚至你最不配得愛的時候.
還可以指望被愛的所在.
這是因為盟約造成的關係.
關啟文是個很出名的香港社會學家.
他說今日社會存在的婚姻關係.
已經由盟約轉到合同關係.
這是一種各人隨心所欲的婚姻關係.
他甚至強調這種轉變.
使婚姻已經到達瀕臨死亡的邊緣.
關啟文在文章裡說.
婚約由原本至死不渝的關係.
演變成以自利為基礎的合同.
這意味著一旦不能滿足.
自我的喜愛和慾望的時候.
就會壽終正寢.
這是否我們的婚姻.
我們每天都在計較.
另一個輔導學家說.
重建婚姻盟約關係.
交到信徒對盟約內涵的認識.
當中包括性禮的意義和救贖角色.
將可以重建婚姻的穩定性和長遠性.
若非有盟約.
我們無法維持.
因為夫妻根本來自不同背景.

$^{1041}$不同特性,喜好,價值觀,習慣.
怎可能你滿足我時我又滿足你.
除非我毀身在你生命中.
你同樣毀身在你生命中.
所以盟約是值得探討的課題.
可惜時間已到.
我只能簡單提醒.
聖經第一次提及盟約.
在《創世記大律章》18-19節.
就是耶和華神和諾亞來立約.
很有意思.
後來洪水過後.
他又和他們立約.
在這個盟約中有三個特性.
神主動和諾亞和阿寶拉汗立約.
沒有得到他們同意.
所以不要要對方滿足才和他們立約.
盟約就是我願意和他們立約.
我願意毀身給他們.
你就要做這件事.
神沒有得到阿寶拉汗的同意.
去和他們建立盟約.
神對他們有期望和要求.
但不會因為他們的回應.
而決定盟約的內容.
講清楚,沒時間解釋.
盟約是沒有條件.
然而當中的好處和祝福是有條件.
你得不得著這個關係的好處.
在乎你自己願不願意得著.
但盟約是沒有條件.
不會因為得著才得著.
神和人的立約沒有定下有效期.
在神的眼中.
盟約是一種狀況而不是處境.
是一種永恆的關係而不是一個交易.
這就是婚姻應有的性質.
正如今天一對新人在正婚人面前.
簽署的婚約.
又簽署了日期.

$^{1081}$但沒有寫下有效日期.
你的帳簿又沒有.
可惜人自己用了簽署離婚書的日期.
取代了婚約有效的永恆性.
弟兄姊妹.
楊蕙從何西雅.
和饒華臣和以色列的關係學習三件事.
放在我們的婚姻當中.
第一件事.
以正面去面對我們婚姻的難處.
不要逃避.
愛有其價值.
但卻要在公義面前.
讓他們知道我們難處在哪個地方.
你會否有這個機會.
將你對婚姻的難處向對方表達.
第二件事.
就是這個無條件的愛.
你從受傷害的人.
甚至被虧待的人.
你要顯出更美麗的愛.
這是一種信息.
第三件事.
你們之間的盟約永遠不改變.
最後用一段很有名的潘霍華神學家.
他對婚姻有一個很深入的體會.
他曾經說過一段很簡單的說話.
婚姻不單止是你們雙方的愛.
也有更崇高的尊嚴和能力.
是受派肩負對世界和人類的責任.
你的愛是你個人的私有財產.
但婚姻是超越個人的.
它是一個身份,一個職事.
使你兩個在神眼中結連起來.
這就是基督徒的婚姻.
這次更簡單.
五十分鐘內.
我們一起去問.
究竟你和我婚姻有什麼需要面對.
這只是一些原則性的事.

$^{1121}$當你想面對的時候.
盼望你找個機會去學習.
明白你起碼知道你的問題在哪裡.
有時我們連問題在哪裡都不知道.
於是隨意去找答案.
求主幫我們.
今天能夠有機會在直至網上.
和各位大家一起分享.
我想這是神的恩典.
祿木斯他們的團隊.
很努力去做成這件事.
我只能夠心裡分享幾句說話.
就是我總覺得上帝在不同的時代.
用不同的方法去成就他的美事.
網絡世界已經出現了一段時間.
但用在教會的圈子裡.
其實還是一個很近期.
大概七年前.
我有一個領袖.
用網上的apps.
一個流動程式.
做一個以道自見的式經領袖.
當時很多人都質疑我.
質疑到我哪裡有做的.
你怎能維持下去.
哪裡有神學院做這件事.
真是我們童工聯會院長行不行.
我只有一個領袖.
我要用更好的方法.
張神的話要傳得更遠.
七年過去了.
以道自見到今日已經超過550萬人.
每日用以道自見.
是一個很大的工作.
但我要和你分享一個很寶貴的訊息.
在整個運作裡面.
我從無欠缺一分錢.
我不會用得很難看.
要省就省.
應做的就去做.

$^{1161}$每一樣都請人去做.
連錄音的人都照錄音的價錢去收費.
我們付出不少代價.
但因為弟兄姊妹享受到.
這個網上的領袖.
他們是甘心情意.
將奉獻一筆一筆的日子給我們.
成為可以維持這個網上的以道自見.
今日我想告訴你一個心事.
加拿大警務中心終於走出這一步.
可以說是疫情帶動我們.
帶動我們走這一條路.
他們很努力.
他們團隊的專業性.
是我很佩服的.
以道自見沒有一隊這麼強勁的團隊.
我們都可以做到這麼大.
我知道加拿大警務中心一定做得更好.
不過欠缺一件事.
就是你們不知道他們的需要比見度大很多.
因為他們的工作比較少.
人手少.
他們做得很吃力.
如果你享受他們用網上去服務弟兄姊妹.
服務你們的時候.
我盼望你能夠為他們做一些定期的奉獻.
哪怕只是十元.
哪怕只是二十元.
一百元一個月.
這個定期幫到他們知道怎樣策劃前夜工作.
我自己作為託錢中心.
有很多弟兄姊妹的奉獻是很感動我的心.
其中有一個感動我的.
十年了.
他從十年前就毀身.
他說我每一個月.
他說牧師我沒有錢的.
我是一個沒有收入的人.
但我很想定期來支持神樂園.
你知道他支持多少錢嗎.

$^{1201}$他每一個月支持六十元港幣.
六十元港幣.
但十年過去沒有減過.
六十元可以這樣說.
不會造成很大的影響我們的工作.
但卻給我們很大的支持.
很大的動力.
所以請你們多紀念加拿大建築中心.
我要澄清.
私人地方.
我又不是他們的同工.
又不是他們的董事.
什麼都不是.
但我是很支持他們的工作.
很盼望在託的裡面給他們一袋錢.
多謝你們.
讓我今天晚上和你們一起分享.
多謝.
多謝郭牧師今晚在荷西亞書的分享.
令我們有很大的得著.
剛才我也說過.
今晚的講座會在兩個星期後.
用國語再說一次.
希望這個這麼好的訊息.
能夠不單給懂得聽粵語的弟兄姊妹.
更加是國語的弟兄姊妹都能夠聽到.
所以盼望大家可以告訴你們教會的.
國語部的弟兄姊妹知道.
我們將會在25日星期二晚上.
同樣時間8點至9點.
有郭牧師的國語講座.
剛才郭牧師也幫我們呼籲了.
其實我們運作一直都是相當之緊.
尤其是在疫情之下.
實在是需要操練我們對他的信心.
所以如果你今晚覺得今晚的講座對你有幫助.
請你按著上帝對你的感動來作奉獻.
現在螢光幕裡面.
你可以看到.
你可以做的是兩個方法.

$^{1241}$一個是在我們的網站PayPal裡面的奉獻.
我們最近開始實行一個.
可以每月定額的奉獻.
好像剛才所說的.
10元,20元,30元.
是沒有所謂的.
每個月都是自動轉賬.
我們能夠有一個比較穩定的收入.
或者你喜歡的一次性的金額奉獻也是可以的.
如果你喜歡支票的奉獻.
也可以寫ABCC.
寄來我們中心的地址.
這裡已經很清楚講明.
如果你是PayPal奉獻.
就可以即時拿到奉獻收據.
可以幫助你報稅.
如果你是寄來的奉獻的話.
我們每年分開兩期.
會發奉獻收據給大家.
盼望這件事能夠得到你們最實際的支持.
最後我想提一提.
今晚的講座是有錄影的.
過一兩天.
我們就可以將這個錄影.
放在我們的網站裡面的聚會重溫那裡.
有些電視節目如果今晚不方便聽.
也可以重新再溫習一次.
待會我們散會的時候.
我們就會開放給大家互相打招呼.
再一次多謝大家今天的參與.
以致這個講座能夠成功.
這個時候我們作一個低頭.
作一個祈禱.
請大家一起低頭.
香港佬,我們多謝你.
郭牧師和我們分享到何西雅書裡面.
神如何看婚姻的關係.
提醒求主你也是深深放在我們每個人的心裡.
無論我們的弟兄姊妹在這方面的認識有多深.
但我們都知道.

$^{1281}$作為一個有罪的人.
我們都很虧欠.
我們有虧欠神.
也可能有虧欠我們的配謀.
求主真的幫助.
使用今晚的訊息幫助弟兄姊妹.
在婚姻關係上.
藉著今晚的講座得到很大的鼓勵.
我們要散會的時候.
求主你繼續祝福我們眾人各部.
無論我們來自哪個州.
哪個國家.
哪個地區.
甚至是不同的教會.
求主都繼續祝福我們.
願你自己的恩典.
你的慈愛.
每時每刻都淋到眾弟兄姊妹的心裡.
我們不配禱告.
但是奉耶穌基督的名禱.
阿們.
\newpage



\section{何西阿書}
\label{sec:n5DpA1Db_0M}
\textbf{【從聖經書卷看生命實踐系列】 主題(三) 從何西阿書看婚姻關係 (普通話)}
\newline
\newline
連結: \href{https://youtube.com/watch?v=n5DpA1Db-0M}{\texttt{https://youtube.com/watch?v=n5DpA1Db-0M}} ~~~~ 語音日期: 2020-09-08
\newline
\newline
\hyperref[sec:XLKUZGl9ItY]{\small{< < < PREV SERMON < < <}}
~
\hyperref[sec:index]{\small{[返主目錄]}}
~
\hyperref[sec:DOwKowWHLkM]{\small{> > > NEXT SERMON > > >}}
\newline
\newline
$^{1}$希望大家聽得到我的聲音..
今天晚上是我們國語的部分第二次的講座,是從何時要修看婚姻的關係..
也許有些人是第一次到我們中間的,所以我很簡單介紹一下,加拿大建大中心是什麼一個機構..
其實我們是一個身處加拿大多倫多作為基地的神學教育機構..
這次我們使用Zoom來搞這個講座,是我們回應在疫情底下,弟兄姊妹對聖經的要求..
我們就努力找合適的講員給我們講這個專題.我們這個專題是叫從聖經的書卷看生命實踐的一個系列..
我們曾經搞過儀式鐵跡,彌西彌跡,今天晚上是何時要修.我們陸陸續續會在技術舉辦這一類的講座..
所以歡迎弟兄姊妹以後也常到我們的網站裡面看看我們最新的消息..
請同工們放出我們的另外一個slide好不好.這是我們的地址,是在Walden and Steele的部分..
很容易找到,是一個非常交通方便的地方..
這裡下一張,下一張的slide就有我們的網站的地址,還有我們在Facebook裡面的Fanpage,也是在這裡列出來..
歡迎大家有時間就到我們的網站,你會知道我們最新的發展是怎麼樣..
我們在加拿大建造中心與香港建造神學院是有一個非常多的合作的關係..
其實建造神學院是我們建造中心一個最堅強的後盾.建造神學院的很多老師經常到我們中間來教書的..
所以這一次的講員是郭奕宏牧師,其實也是建造神學院裡面的拓展部的總監..
他剛剛退休,我可以告訴大家,他根本沒有退休,現在還有很多工作,不過轉移到北美,到他的家裡面,他的家是在Kitchen..
郭牧師是一個非常專業的輔導員.所以今天他講這個講座,其實是他最專門的一個講座..
所以希望大家都喜歡,我們就用和西亞書作為一個起點,談到如何去應用書卷裡面,看看婚姻關係是怎麼一回事..
我們現在就歡迎郭奕宏牧師,今天晚上的講座大概是小於一個小時.請大家都能夠一直到底的聽郭牧師的講座..
謝謝大家,等一下完了以後,我再跟大家有一點點的報告.郭牧師,現在是你的時間..
弟兄姐妹平安,謝謝你們給我這個機會.想不到我們可以在這個平台上,可能在世界不同的地方,我們都可以見面來分享..
所以現在我有41位在這邊,當我講到一半,剩下10個我也不會失望,因為我知道可能你聽不懂,說不定你要跑了..
隨便,放心你們都是神的恩典,讓我們一起來學習吧.我對中國國內的施工也是特別給我很大的負擔..
我先開我今天晚上的PowerPoint,這是今天晚上的..
今天晚上我們大概用50分鐘的時間,來跟弟兄姐妹分享一下,從荷西亞書來看婚姻的關係..
這個是比較困難的一個課題,因為在短短的50分鐘要講《借卷的小先知書》,比較多的難題在裡面..
特別很多人都在問,讀這本書的時候,通常都會問一個問題,借卷書是真的還是假的呢?.
真的假的,是講到這個故事是真的還是假的呢?有荷西亞的先知的經歷,那麼是真的嗎?.
這些課題都是相當有爭議性的..
當然從福音派的教會,從福音的神學來看,我們是比較清楚的,知道我們的立場在哪裡..
所以在短短時間我不能夠將整本荷西亞書14章裡面講完它..
但是我會先提一些基本上對借卷書的背景,我們先了解一下..
了解以後,有一兩個課題我們來討論..
然後我就將借卷書裡面最常被關注的一點事情,就是婚姻的事情,婚姻的關係..
因為借卷書留給我們一個很大的爭議,就是究竟是不是一個淫亂的人,或者在婚姻裡面很多是錯的人,那麼神還是要使用他..
而且好像是上帝叫他去做這件事情..
這個題目令到很多人都在這邊有些爭論,那麼我們要先來好好討論..
那麼借卷書呢,為什麼神會吩咐先知荷西亞娶淫蕩的女子,過年會吃妻呢?.
這與神要求婚姻應該有的聖結,豈不出現矛盾和難以解釋的疑惑嗎?.
特別在聖經很清楚說,耶和華初次向荷西亞說話,耶和華對他說,你去娶一個淫蕩的妻子為妻,收納從淫亂所生的兒女,因為姐弟行大淫亂,離妻也何妨?.

$^{41}$這個事不需要去爭辯,就是這麼清楚.你可以推翻這本卷書的歷史價值,但是你不能懷疑這句話是怎麼解釋,因為太明顯了..
而且在這裡也更講到,收納從淫亂所生的兒女,淫亂,淫蕩這個字,理發是指不淡的性行為,婚外性行為,特別在宗教性不忠也可以用這個字,人文都是複數的,都是重數的..
意思就是不是講他有一個經歷,可能是多次在這方面都有失落的..
所以就帶來一個很大的問題,究竟神的立場是怎麼樣?我們要從哪一個角度來看這方面呢?.
所以我們在說,這卷書裡面,特別是佛西亞的妻子戈內,兒女耶斯內,羅路哈瑪,不謀年名的意思,羅阿彌,非我名的意思,都是真實的人物..
意思就是如果這個故事是假的話,我們就問,為什麼將這個歷史裡面真實的人物放在一個我們不能接受的故事的結構裡面來傳揚這個性行呢?.
那豈不是對這個真實的人物,如果他沒有這個事情,沒有這個行為的話,對他們就是不太公平了..
所以我們覺得從我們的角度來看,這個是真實的故事,也是真實的經歷..
當然對我們今天讀這個聖經,實在有很大的難處..
就是為什麼作為一個先知,神要吩咐他去做這件事情呢?.
當他做的時候帶來的困難有多大呢?他的掙扎是多大呢?.
這些都是我們不能夠了解..
但是我就是因為這個不能夠了解,常常成為我們看聖經裡面一個最大的困難了..
就是當我們不了解的時候,我們就想找方法去用我們能夠了解的理由來解釋這個聖經..
那這個就很危險了..
因為如果每一個人從不同的角度,不同的標準來看聖經,來解釋聖經的話,來證明聖經這段是真的,那段不是真的,那麼這就很危險了..
變成你所念的聖經,跟我念的聖經,跟我們上一代念的聖經,跟下一代念的聖經都不一樣了..
因為每一個時代,每一個人的背景都不同..
我寧願承認一件事情,可能我不能夠太了解,我就在上帝面前說,主啊,我不明白這個是什麼意思..
可能到現在我還是有一個聖經不明白,好像你都是一樣一樣..
但是我從來不會懷疑聖經是上帝給我們的話..
其實這種不了解,也有一個真正的原因在後面的..
這個原因是什麼?就是我們對當時這個聖經的背景不了解..
所以從今天的背景去看那個時候的故事,.
從以後西方的角度去看中國人,我們中國人去讀這個聖經的時候,不同的背景有不同的解釋,不同的標準..
這樣的話呢,我們就很難去了解聖經的話..
所以,讓我們先起碼,雖然我們不能夠完全了解,當今天晚上聽完以後,你還是帶著很多疑問,面對著聖經,我相信有的,不要緊..
但是起碼我們要承認一件事情,我們對這個捲聖經後面所發生的事情,可能真的不太了解..
所以我們好不好先來看看這個背景是怎麼樣..
這個聖經第一章一節,很清楚說,.
當烏西亞約探阿哈西,希西加做猶大王,約阿斯的兒子耶羅伯安做以色列王的時候,耶羅伯的畫臨到必利的兒子和希亞..
和希亞接受上帝對她的畫,是在什麼時間呢?.
就是在當烏西亞,就是在猶大四個王在那邊,他寫了四個王..
那麼,北角的四個都是南角的,北角就是約阿斯的兒子耶羅伯安..
要記得這個耶羅伯安是第二,因為第一個是所羅門,以後的耶羅伯安跟這個不同..
這個是在北角最後幾位的皇帝..
那個時候是什麼情況呢?是從公元前750年到721年之間,.
由這個耶羅伯安二世統治以色列的時候,以色列在滅國之前,.
那麼那個時候國家非常的興旺,所以他們也覺得我們很好啊,但是要了解,.
但那個時候,南北角處於政治平靜,經濟豐裕的昌盛末期,.

$^{81}$這個末期當然我們現在說是末期,對他們當時來說不知道是末期,.
他們以為是最剛剛好的開始,他們以為他們是最豐盛的時候,.
所以還有好多年,但是上帝執著,很多先知警告他們,他們走到末路了..
那個時候阿蘇格爾漸漸成為超級強國,以色列漸漸走向滅國的命運,就是這個時候..
所以公元前743年,阿蘇格爾帝國攻下加利利的時候,以色列往米拉縣,.
用大批的金銀向阿蘇格爾進攻,以後的幾年,米拉縣周傳於什麼?.
像中間埃及或者亞蘇之間,一會靠近埃及去反對亞蘇,一會靠近亞蘇去反對埃及,.
他就站在中間這個環境裡面,所以變成了他向兩方面都不討好,.
也向兩方面都開放他們的國家..
約翰華格禁拜那個時候,跟迦南帝的地方宗教巴黎的禁拜,斗前不清..
所以你就知道這個卷述,事實上,上帝用這個和西亞,.
那他發生在身上的婚姻的事情,家庭的那種不幸,.
然後反映上帝怎麼樣對他的以色列民,.
怎麼樣去警告他們,他們所走的路是怎麼樣..
所以這個是背景.迦南那個時候宗教,你一定要先了解,.
有時候我們也會發覺,哇,有沒有這麼厲害呢?其實真的..
所以你看巴黎,那個時候他們是雨水跟龍舟舞的神,.
巴黎的配偶是亞娜,亞娜是性愛繁殖跟政爭的女神,.
亞斯塔魯也是女神,常常被當成巴黎的配偶..
所以這種很澳不清的宗教,是發生在迦南帝那個地方的..
所以迦南的神廟裡面,好多神祭,就是祭女這樣,.
她是與好多來參拜的男性發生性交..
所以在迦南宗教裡面,跟性有關的,.
已經深入這個以色列人的信仰當中..
女性把自己的童真,為什麼要獻給迦南的生殖神呢?.
你陌生的人性交懷孕的話呢,.
他就表示神會已經賜福給他了,.
這個神要賜福他,以生養眾多,照顧他的宗主..
所以他會用自己的身體來帶來一個祝福,.
然後如果懷孕的話呢,他就證明那個神會愛他,.
會賜福給他,以後在他的一生裡面呢,.
就可以有好多繁殖的機會了..
所以這個通常呢,是在當時講是,.
他們現身的行為是一生一次的,分錢的來做的..
那偶爾有些呢,是為了還願多一次的也有..
在當時呢,迦南宗教的影響之下,.
大概呢,很少有女性在生殖業,仍是貞潔的,.
都是奢侈她的貞潔的..
所以這個呢,對我們今天,其實慢慢在上一個年代,.
我們就很難去想像,其實你發覺,.

$^{121}$現在在北美也好,歐洲也好,亞洲也好,.
這種的分錢的性行為,也是普遍的不得了..
在上一代的人可能發覺,哎呀,這個時候很難接受..
對,我們真的很難接受,上帝也很難接受..
但是就是這樣發生了..
所以呢,你要記得呢,當時迦南的宗教就是這樣,.
神帶領以色列人從埃及出來,然後進到迦南的時候,.
不是要跟這班人來往的,.
所以將他們隔離出來,讓他們成為分別回甥的一個民族,.
就是要影響當地的人..
反而他們離開這位帶他們離開危難的那位神,.
是一神的神,聖潔的神,.
反而去參與的是迦南的那個宗教裡面,.
來用淫亂成為他們的生活的一部分..
所以這個是對一個以色列人,或是這樣說,.
對帶領以色列人離開維奴之家,進到迦南的時候,.
神又祝福他..
所以你記得嗎?當約瑟王要帶領以色列人進到迦南之前,.
他為什麼講一句話,這句話好像今天我們常常被用了,.
但是其實背後裡面有更重要的意思..
他就是說,至於我和我家,必定是奉爺我王..
為什麼?因為那個時候,他不是講政治,不是講經濟,.
不是講糧食,不是講他們將來的生活怎麼樣,.
是講他們的家庭,他們的婚姻..
就是他太清楚了,在迦南地都是充滿這樣的婚姻..
所以就是這個背景之下,.
他宣告我要我跟我的家都要做一個敬畏上帝的家庭..
所以這個事對我們也有很重要的提醒..
其實雖然我們不能夠了解當時他們在神廟裡面的事情,.
但是我們也不能夠不承認,.
今天好多人的生活,男女的生活,婚前的性行為,.
普遍的不得了..
弟兄姐妹,要不要先停一下問問自己..
我們活在同樣的一個世代裡面,.
但是我們怎麼樣去將我們自己的婚姻,.
我們的家庭分別為生..
面對這個年代,好像婚前性行為已經慢慢被解納,.
甚至有北美一些年輕人,.
他們覺得來結婚之前沒有這種的經歷的時候,.
哎呀,他是沒有價值..

$^{161}$你的兒女怎麼樣看?.
我的兒女怎麼樣看?.
可能這個也是對我們一個很大的提醒..
所以何須要活在那個環境裡面呢?.
上帝來告訴他,你要做一件事情..
相信先知實在是娶了一個不貞潔的女子,.
過年回妻..
當神要他向以色列發出責備跟憐憫的信息的時候,.
先知就以自己的婚姻家庭當中的親身的經歷,.
作為一個最有力的比喻,.
見證如何在接納妻子的過程當中,.
體會神對以色列不離不棄的愛..
所以我們來看,.
這個活西亞書是怎麼樣將這個家庭帶在我們面前的?.
當然我們知道,整卷書裡面,十四章裡面呢,.
其實只是分開大略來說,是分開兩個部分..
第一個部分就是,神之作,先知活西亞的親身的經歷,.
使他體會神對以色列的人心長..
所以第一章到第三章,大總的來講,.
都是主要講活西亞自己的婚姻家庭是怎麼樣走過來..
當然這個不是主題..
第四章到十四章,是對被盜的以色列傳講信息..
就是用活西亞的那個經歷,.
或者我們今天的話,就是用他的見證,.
來見證什麼?見證四章到十四章裡面,.
神對以色列那種不離不棄的愛..
這其實對我自己有一個很大的提醒..
活西亞的婚姻他家庭,是引證上帝對他子民的不離不棄的那種愛..
但是我們很常讀這個活西亞書的時候,.
我們忘記第四章到十四章那個內容,.
反而我們一直在停留在活西亞自己的家庭裡面..
我們真論他為什麼這件事情發生,.
他這樣做對不對,把好多好多不同的疑問出來..
我就想到,今天很常我們在傳講神的話語的時候,.
或者特別在佈道會也好,.
我們很常會請弟兄姊妹分享他的見證,.
他信主的經歷,那這個是好事..
因為從經歷裡面,從見證裡面,我們去引證神是何等的愛我們..
但是小心,不要讓我們的見證大於福音的內容..
不要讓我們自己的經歷,接蓋了神愛內容的偉大..

$^{201}$所以這個是值得我們提醒的,提醒我們自己,.
當我在福祉裡面,我可能會將我自己的經歷放在弟兄面前,.
分享我的過去,你也會,.
但是不要將這些經歷,接蓋了神更偉大的工作..
特別有人提醒,.
他在新約裡面,浪子回頭這個故事,也是這個樣子..
有多少時候我們將這個焦點,都是放在浪子的身上..
我們講到他去怎麼離開家庭,然後在外面怎麼去經歷,.
然後回來的時候,那種勇氣,然後他的父親怎麼接納他的時候,.
他悔改過來,哇,怎麼重新得到他的位分..
但是我們忽略那個慈愛的父親,天天在那裡等候盼望..
其實那個不是浪子故事,因為這個是慈父的故事,.
但是我們這樣顛倒過來..
所以同樣的,在你的生意裡面,.
在你生意發生的每件事情,都有他最高的旨意,最大的目的,.
就是顯出神的榮耀出來..
所以你要小心,我也要小心..
我們經營我們家庭,我們婚姻,當然我們自己是享受在其中,.
但是更重要一件事情,你的婚姻,你的家庭,能夠見證上帝嗎?.
你們夫妻的愛,能夠讓上帝在你們當中得到榮耀嗎?.
如果有一些朋友看著你們的婚姻,.
講一句他說,哎呀,如果信主信到你們這樣的婚姻,.
這樣那個家庭我寧願不信,我們的見證者失去那個效用..
上帝讓我們經歷了每件事情,就好像和稀,.
雖然他的經歷是我們很難接受,但是上帝說,你看吧,.
這麼難受的一個婚姻,比起我對以色列人,對所有神愛的子民,.
他那種的愛,還是小得不得了..
這就是這本書裡面最主要的主題..
那麼我們來看,郭念的背景,到底是婚後出紅場,.
紅磡出場呢?.
又或是妓女呢?.
甚至有些人說,她是不是廟祭裡面的孫婦呢?.
我們可以肯定的從聖經原文裡面來肯定,.
她並非外邦宗教裡面的廟祭..
因為銀蛋這個字,第一章第二節裡面講的,.
他說,這個希伯來文跟專清妓女的那個字是截然不同的..
因此呢,郭念應該不是賣淫為生,.
也不是異教的那個密教,廟祭,.
而是一位對婚姻不忠的婦人..
那這個是我們很清楚知道的..

$^{241}$所以可能是他在這個宗教的文化傳統裡面,.
他將這個身體出賣了,.
然後期盼他以後的婚姻裡面得到更大的祝福..
甚至他這個後來,他結婚以後,跟活西亞結婚以後,.
他在那邊是不是再出現婚姻的問題呢?.
可能是..
因為呢,聖經裡面講到,第一個跟他生了孩子也是戀,.
然後再生兩個兒女的時候呢,.
都講到不是我的子民,不是我要憐憫的..
所以可能這個不是跟活西亞生的,也說不定..
我們不知道,太了解這個世界的時候,.
我們不需要去猜測..
但是我肯定,這個呢,就算是又如何呢?.
我們要看的是,活西亞作為神的婦人,.
他是怎樣去面對這個婦人呢?.
這個婦人呢,他是這個在婚姻裡面不忠誠的一個人..
那麼,怎樣去挽救他回來?.
這個是我們今天晚上的主題..
就是如果在我們婚姻裡面,當然我們不是說,.
今天晚上聽的人都是面對這樣的,不是..
但是婚姻出問題,是多層次的..
不一定在第三節,成為出現,才叫婚姻有問題..
其實現在很多時候,我們婚姻裡面都面對不同挑戰..
不單是妻子,丈夫,都會同樣有機會出現這個問題..
所以呢,我們就來看,我們講的這個鍋鏈,.
他在婚姻裡面出現問題,可能就是我們今天所講的,.
就是婚外情這個事情..
他在婚姻以外,有另外一些行為,.
是背棄離棄他的丈夫,背棄他的婚姻的..
所以婚外情呢,其實如果將這個婚外情,.
不是將這個行為帶進今天的婚姻關係裡面,.
而是將這個原則放在我們今天手,.
其實婚姻現在也面對同樣很多的挑戰..
婚外情,是現在婚姻裡面五大威脅之一..
在十個最大的離婚的原因裡面,.
婚外情已經越來越淺了,這個理由..
在中國國內裡面,他們統計的時候,.
他就說婚外情是在離婚裡面第三個最大的理由..
所以你發覺婚外情是在不同的世代,.
不同的背景裡面都會同樣出現的..

$^{281}$什麼是婚外情呢?.
我們對婚外情的理解也常常會出現誤差的,.
有不同的看法的..
婚外情這個定義是什麼呢?.
我們說婚外情有三個不同的定義..
這個是學者在研究,今天因為很多婚姻都面對婚外情,.
如果我們說是第三個最大的離婚的原因的話,.
我們一定要了解..
第一個呢,是傳統性的..
就是婚姻關係裡面的夫妻,任何一方,.
與第三者發生性關係..
所以講的是什麼呢?.
講的就是一定在身體上面的不忠,.
就是跟第三者在身體裡面有接觸,.
我們講的是性的接觸,.
所以這個才叫婚外情..
在傳統裡面,或者從另外一個角度來說,.
就是發生非婚姻關係,性關係的男女當中,.
至少有一個人已經結婚了..
那麼這個就是傳統定義的..
這種婚外情的定義呢,.
涉及婚外情,婚外,婚姻以外異性感情關係的一方,.
都是常常說,當我們被指為暗用第三者的時候,.
我們很會說,沒有啊,我跟他只是男女朋友,.
我跟他只是這些感情,我們從來沒有上船啊,.
我們從來沒有做什麼越軌的事情,.
我們是抓住這些傳統性,這個定義來說..
我跟他沒有性的關係,但是你不要留心啊..
就是其實婚外情到發生在船上的時候,.
已經是婚外情裡面最後的一個階段..
所以第二個定義,狹異性的是什麼?.
婚外情一般是指夫婦之間出現第三者,.
可能是男性或者女性,.
無論是只是在情感上的交流,.
或者有性的活動,有愛有性,.
或者有愛無性,或者有性無愛,.
這都算為婚外情..
就是第三者在成為你們婚姻關係的中間人,.
將這個關係,這個親密的關係,.
拉到去另外一方去了,.

$^{321}$那這已經是婚外情..
所以你不要欺騙自己,.
我從來沒有接觸別人,.
但是你的心已經跑到另外一方去了..
如果這樣的話,我告訴你,.
其實婚外情已經在你們當中..
第三個更加廣義的說法,.
第二是狹異性,還是一個人第三者出現,.
但是不一定有性的那種關係..
第三個廣義性,.
就是指婚姻以外一切的情事,.
這些情事可能發生在與人的關係上面,.
也可能出現在不同的嗜好或者活動當中,.
就是從各種不同的活動或嗜好當中所得的滿足,.
取代了婚姻關係裡面應該有的滿足..
這樣說的話,那麼好多人都有婚外情,.
可以說是,或者說是婚外情的前奏,.
或者說是婚外情的開始..
在你不覺的之間,.
你的心跑到婚姻以外的時候,.
要留心,其實慢慢一列好快的火車,.
會進到你們婚姻裡面,.
破壞你們的關係..
所以如果從第三個講義來說的話,.
你發覺婚外情那種威脅是怎樣?.
就是婚姻以外的情事威脅我們夫妻的合一..
這個定義是怎樣說呢?.
就是在婚姻之外的情事,.
就叫婚外情..
不是婚外裡面第三者,.
或者甚至不是婚外以外,.
這個第三者跟我發生性關係才是有婚外情..
在婚姻之外尋找滿足感,.
其實是婚外情開始..
或者逃避艱難的婚姻關係裡面,.
在婚姻關係裡面每一對都有難處,.
我從來沒見過一對是沒有難處的,.
包括我自己也在那裡,.
每一對也是這樣..
我們每天都在那裡掙扎,.

$^{361}$每天都在那裡學習,.
每天都在那裡成長..
所以當我不願意再面對了,.
我要逃避,.
然後在婚姻要逃避,.
找到另外一個關係裡面..
第三,婚姻當中不負責的那個結果呢,.
也是婚外情的開始..
會因為承擔過多的義務或者責任,.
而導致身心不平衡而孤立,.
這個就是婚外情的原因..
為什麼會出現婚外情?就是這樣..
在這個定義裡面呢,.
我們找不到一個人的出現,.
可能是我們的心先跑離開我們的婚姻..
這樣說的話,婚外情就不一定是一個人了..
所以在我們的研究裡面,.
在我們叫夫婦營裡面常說,.
婚外關係裡面可能包括你的運動也說不定,.
有人是利用運動逃避在家裡面發生那種衝突..
有人是用嗜好,喜歡看電影,喜歡去購物,.
喜歡去做甚至是社區裡面的服務,.
好多物質裡面的追求,工作裡面的追求,.
這些甚至於我們說在教會裡面侍奉,.
也要小心,不是所有教會侍奉都是婚外情,.
是看你怎麼去幹你的侍奉..
如果你的侍奉只是要逃避在家裡面那種困難的關係,.
然後在教會侍奉的時候被人家欣賞,.
哇,那個弟兄找人到你面前,.
哎呀姐妹呀,你做得太好,你唱詩感動我的心,.
哇,那種的滿足是從來你在婚姻裡面找不到的..
你的心會越跑越遠..
這就是婚外情的開始..
所以我們從何時那種最明顯的淫亂的關係裡面,.
我們套用在今天的時候我們要小心一件事情,.
就是不要以為我去就沒有淫亂,我沒有淫亂,.
淫亂沒事,我們是安全的,上帝不應該這樣對我們講什麼話..
要留心我們的心是怎麼樣去面對我們的婚姻,.
這是比一切都更重要..
所以究竟多少人有婚外情呢?.

$^{401}$如果在這個統計裡面,或者在嚴謹一點的,.
在第三節出現的時候我們做一個統計,有多少呢?.
可以說沒有的..
沒有是什麼?因為沒有人去做這個統計..
因為沒有人敢說我是有婚外情..
剛才你聽的時候已經有一點不舒服了吧?.
哎呀,公主你為什麼這樣說我有婚外情呢?.
這個呢,我只是說,在這個研究裡面,.
我們的關係在第三節出現之前,.
我們的心其實已經離開了..
所以當我在香港除了在劍道教學以外,.
做這個童工以外,我也做了很多不同的輔導..
我做了家庭輔導已經二十多年..
這二十多年裡面,我見了好多好多不同的困難..
好多都是在婚外情裡面出現這個問題..
是哪裡開始呢?是從他的心離開了他們的婚姻,.
他的心逃避他們家庭裡面的困難,.
然後進到另一種關係裡面..
可能是從工作狂開始,可能是從shopping開始,.
可能是從好多不同的嗜好裡面開始,慢慢越背越走得更遠,.
越掙扎的時候越跑得更遠,這個就開始出現..
所以統計是很難去統計..
在統計裡面呢,奧翔江是在香港一個很出名的婚姻研究的輔導員,.
也是一個神學的教授..
在他寫的書裡面,他引用了美國對婚外情的統計是怎樣的統計呢?.
他說尋求婚姻輔導的個案當中有25-30\%在婚外情被發現之後尋求輔導的..
另外大概有30\%在婚姻輔導過程當中被切入有婚外情的..
在那份報告裡面同時發現,在婚外情的輔導過程當中,.
最終有38\%是會分開的,有62\%是可以被挽回的..
所以不要以為,哎呀,我根本我丈夫不愛我,哎呀,我太太不愛我,她不尊重我,就處理咯..
如果你不處理,慢慢走得更遠..
如果你現在處理,其實機會是有的..
那麼這個統計裡面呢?.
在香港的另外一個統計,在民愛向前線的輔導機構的杜德主任郭智英女士,.
她在發表她的文章的時候,她怎麼說呢?.
她說婚外情問題自主管服務在2001年期間,.
透過用電子簡報分類做簡單的搜查詢..
我們發現,由1997年到2001年之間,.
總共有96宗本地新聞跟婚外情有關的..
其中涉及37人死亡,跟65人受傷..

$^{441}$由2002年10月份到2005年,因為婚外情而引起的夫妻衝突當中,.
需要使用他們的服務的,總共有335人要站出來..
其中呢,到處暴力的有118宗,佔整體的35\%..
而且呢,涉及自殺的個案佔16宗,16\%共54宗..
那奇怪,在西方婚外情,到處離婚很常見..
中國在婚外情,到處離婚其實比較少..
但是呢,如果發現的時候,處理不好的時候,通常有死傷的..
就是我們不知道怎麼去面對,好像只有死才能解決..
所以這個不是我們今天要學習的,就是怎麼樣去面對,.
如果在婚姻裡面出現問題怎麼辦?.
那麼信徒的圈子怎麼樣呢?.
你要全部是,他曾經向信徒發出調查問卷,.
收到90份的問卷,當中,你記得是向信徒發出的..
其中有15個人,16.5\%的,.
他是表示正面對婚外情的困擾..
所以李牧師呢,他在他的詩裡面,他怎麼寫呢?.
他說基督徒在面對婚姻危機的時候,.
落入這個不知所處的困境裡面,.
他們在這個事實裡面,如何才能夠,.
然後去要去守他們的信仰呢?.
在教會當中,誰正面對這個婚姻的危機呢?.
他們又是怎麼,已經得到死亡的輔導呢?.
教父同工如何才能幫助他們面對危機,.
在婚姻關係當中做出防衛的工作呢?.
那,這都是我們現在面對的,很厲害..
婚外情,坦白說,已經不是教外的事情,.
已經是社會的問題,進到教會裡面,.
因為我們活在同一個社會當中,.
我們受著同樣的威脅..
那麼,如果我們沒有正面處理婚外情的話,.
結果會怎麼樣?.
有一個輔導員在每個紐約時報裡面,.
寫了一篇關於離婚的文章,.
那篇文章的題目很特別,.
他叫《未離婚的undivorced》..
我說為什麼叫未離婚呢?.
他說,婚姻輔導專家這樣說,.
未離婚的人是情感上的離婚,.
Emotional divorce,而不是法律上的離開,.
Legally divorce,.

$^{481}$他指出這是很多現代婚姻的沉浸的趨勢,.
就是大家保持著冷戰,.
你不離開我,我也不要還你,.
那麼你住你自己的房間,.
我住我自己的房間,.
然後你去你的旅行,.
我去我自己的旅行,.
為什麼要停留在那邊?.
因為我們的文化,.
因為我們的信仰,.
因為我們的孩子的問題,.
因為我們的面子的問題,.
好多不同的理由叫我們不要離婚,.
但是你擱置精彩,.
每個人自己過自己的生活,.
這樣的話,我們的婚姻是完整嗎?.
我們就離開荷西亞的情況很遠嗎?.
我不覺得,其實我們的婚姻已經死了,.
那麼這樣的話,.
我們要面對,.
那麼怎麼去面對呢?.
以下我就提出三個很簡單的原則,.
從荷西亞,他去面對這個過濾,.
他是怎麼樣去挽回,.
而且讓他們重新在場面前,.
將不被黏你,不蒙愛的那種兒女,.
他們的關係重新成為蒙愛,.
蒙黏你的關係呢?.
第一個原則是什麼?.
悔改,悔改與悔轉,.
正視與面對,.
荷西亞第二章第二節到十三節,.
也有荷西亞是控訴者,.
過濾與以色列民族是被告,.
所以很清楚的第二章第二節開始的時候,.
他沒有避諱地將這些事情,.
關懷的事情收起來,埋藏起來,.
不是,如果要悔轉,就必須要正面對,.
如果要重建,就必須要成為,.
這是原則,.

$^{521}$所以我們不要逃避我們關心你們的困難,.
第二章第二節,你妹妹怎麼說呢?.
她說你們要與你們的母親大大爭辯,.
因為她不是我的妻子,.
我也不是她的丈夫,.
叫她去掉面上的影像和空間的那個形態,.
免得我摸她的衣服,.
使她切體與財生的時候一樣,.
使她如曠野如乾旱之地,.
因何而死,我必不憐憫她的兒女,.
因為他們是從淫亂而生的,.
他們的母親行了淫亂,.
懷他們的母親做了很羞恥的事,.
因為她說我要從隨從所愛的,.
我的餅水羊奴麻油酒都是他們給的,.
佛西亞第一段裡面講得很清楚,.
講到他妻子有什麼問題,.
造成他們婚姻的問題,.
他沒有筆尾的放在亮光之當中,.
不是我們要沒有愛,.
但是在愛之前我們也是必須有這個功能,.
上帝就是藉由這種態度來對待我們,.
上帝愛我們,.
之前他說你們必須要承認你們的罪,.
如果不承認自己的罪,.
不承認自己一個罪人的話,.
他怎麼會信耶穌為他的救主呢?.
救主就是拯救他嗎?.
如果他不覺得他危險,.
他怎麼能夠得拯救?.
我們的婚姻也是這樣,.
如果我們不能夠將我們的不願意,.
不敢將我們的婚姻難處放在面對面的,.
彼此的對症裡面,.
你知不知道在這段聖經裡面,.
很特別的講出了這句話,.
其實好像在法庭裡面用的話一樣的,.
就是在哪裡呢?.
來爭辯,來控告,.
來控告他的妻子,.

$^{561}$所以爭辯的意思,.
在這個聖經裡面,.
剛才所讀的,.
你們要與母親大大爭辯,.
爭辯是法庭的用語,.
就是指控的意思,.
不是跟他爭,.
是要指控他,.
他不是我的妻子,.
我也不是她的丈夫,.
這個是以前古代的人,.
用來離婚的時候的一個很經典的,.
很正面的,.
公實的用語,.
就是說我不再是她的丈夫,.
她也不是我妻子,.
要取掉就是脫掉,.
坦白的,坦蕩蕩的放在他們的面前,.
為什麼我們有一種觀念就是,.
我只要愛就解決一切,.
我不否認這種的觀念,.
但是愛之前,.
你愛什麼?.
你知道你愛什麼嗎?.
不管了,.
愛就是愛,.
愛就是無條件,.
愛就是全部犧牲,.
誰說的?.
愛是有條件的,.
不過這個條件不是我要對方付出,.
是我為他付出,.
所以第一方面,.
當我們要重建婚姻的時候,.
這個佛西亞神,.
吩咐佛西亞,.
你要將你的妻子所有的那種敗壞,.
放在面光之間當中,.
一起來辯論,.
第二個條件,.

$^{601}$這個無條件的付出這個愛,.
佛西亞書第二章十四節,.
那個何修版這樣說的,.
因此看我要有道他,.
就是何何版呢,.
就是我必要勸導他,.
領他到這個曠野,.
我要說動他的心,.
何何版說什麼呢?.
對他說安慰的話,.
在哪裡?.
我必是他葡萄園,.
也是他阿哥古,.
作為指望的門,.
所以我們必須要了解一件事情,.
就是這個有道跟勸導是有分別的,.
有道是怎麼用這方面,.
有道在聖經裡面的意思,.
就是我們這方面很溫柔的幫助他,.
來了解他所犯的錯誤是什麼,.
丈夫本來能用命的方式來要求妻子,.
但見你上帝卻用唇唇上咬的方法,.
用以色列民能夠接受的方式,.
來有道他們去做一些當時他們不明白的決定,.
所以原本你們不是講勸他吧,.
勸他能夠回轉吧,.
不是,是用盡一切的方法,.
來將他領回,.
來到最愛的裡面,.
所以責任不是在那個錯的人回頭,.
是被侵犯的人用愛去幫助他回頭,.
所以當神對以色列那種愛的時候,.
我覺得最寶貴的是怎麼樣,.
就是在後面的聖經裡面,.
講到神怎麼樣去愛他們的時候,.
在十一章第四節,.
神用另外一個美麗的形容,.
他說我用慈神愛說牽引他們,.
我帶他們如人,.
松開牛兩衰旁邊的鵝,.

$^{641}$彎下身來餵養他們,.
慈神愛說,我們都熟識這句話,.
原來是用在什麼,.
用在被侵犯我們的人身上的,.
不是單單上帝這樣,.
上帝說你看,.
佛西亞就是這樣,.
這個不配的妻子,.
淫亂的妻子,.
傷害他的妻子,.
但是他用慈神愛說,.
去將她牽引過來,.
如果你覺得你的配偶,.
傷害你了,.
無論是語言也好,.
或者其他事情,.
我想問,你用了什麼,.
你又有讓他知道你被傷害的事情嗎,.
你要讓他知道,.
然後我想問,.
你怎麼去引導他,.
第三方面這個愛很特別,.
又是他阿哥骨,.
作為指望的嗎,.
你知道嗎,阿哥骨是為了越淡的平原,.
也離哥城的西南部,.
在約書亞紀念的七章16到26節裡面,.
記載阿甘被處死,.
就是在這個阿哥骨,.
所以也是成為叫審判骨的意思,.
變成了你要成為他指望的門,.
一方面我們讓他受到審判,.
知道指出他的錯出來,.
阿甘被處死,.
以色列不能夠打勝仗,.
如果我們不能夠用愛將現象,.
我們的問題放在我們當中來處理的話,.
我們的愛常常不能夠真的去處理,.
化解我們的衝突,.
只有被指出我的錯出來,.

$^{681}$然後用更大的愛去包容這種錯,.
這個審判骨就變成指望的門,.
這個很寶貴,.
第三點,我們常常在婚姻裡面出現問題的時候,.
不是你死就是我活,.
我們一直在等對方道歉,.
我們一直在等他死了,問題就解決,.
是不是這樣,.
如果這樣的話,我們根本不明白,.
神在荷西亞書裡面製作荷西亞的女兒,.
那個精力到處去化解這種傷害,.
這種衝突,這種困難,.
這種不能被愛的精力,.
只有我們給他一個指望的機會,.
有一個盼望的機會,.
這才是能夠將他重新帶回我們的婚姻當中,.
所以第三件事是誤解,.
何曉明就說,.
他說他必在男兒女兒回應,.
像在年輕的時候從埃及逝上來的時候一樣,.
那日你必親赴我衣食,不再親赴我巴黎,.
這是以何說的,.
這個荷西亞是怎麼說,.
他說回應這個字,.
回應,必親赴我衣食,不再親赴我巴黎,.
這個就是,.
衣食的意思就是男兒丈夫,.
重點是判女另一半的意思,.
注重親密的關係裡面,.
回應就是,那個時候他經過愛的時候,.
他能夠重新來到我面前,.
來提醒你是我的丈夫,.
什麼時候講你是我的丈夫的,.
是在結婚的時候,.
是在沒有被傷害之前的,.
這個就是什麼,.
就是你們當初訂立你們婚約的時候,.
那種最親密的關係,.
可能在當中,過了五年十年十五年,.
二十年三十年,問題來了,傷害來了,.

$^{721}$你們都忘記你們當初那種的尾聲,.
神也同樣提醒,.
他說我要提醒你,.
你是怎麼樣,.
他說我是那個帶領你們離開埃及,.
你從埃及上來的時候,.
一樣,那個時候就是最初的愛,.
這個就是第三件事情,.
在我們的關係裡面,.
很常出現的問題是什麼,.
就是盟約這件事情,.
我現在說第三件事情,.
劉華又對我說,.
你去愛那個情人所愛,.
去放堅硬的符,.
正如劉華愛哪偏向別人,.
喜愛葡萄餅的以色列人,.
劉華對我說,.
你再去愛一個淫婦,.
我相信這個愛另外一個,.
在聖經裡面,原文裡面,.
不是去愛另外一個人,.
是葉花再一次又對我說,.
愛另外一個,愛再愛一個,.
就是講,.
葉花又對我說,.
去愛那個情人所愛,.
去放堅硬的符,.
去愛那個情人,.
你不要誤解這件事情,.
然後,.
何希亞對葡萄的說,.
最後,.
令到他們重新能夠建立他們的婚姻,.
只有一個很重要的事情,.
從現在年輕的時候,.
從埃及地上來的時候,.
好像當初的時候,.
他稱呼我這個意思,.
就是丈夫我的男人,.

$^{761}$那個時候開始,.
這是什麼,.
就是盟約的事情,.
今天在婚姻上面最脆弱的一環,.
就是婚姻只有契約,.
但是沒有盟約,.
盟約是什麼意思呢,.
有一位學者講得非常好,.
他說,.
婚姻四月當中的聖旨,.
差異造成婚姻的不穩定性,.
他們強調婚姻當中盟約的重要性,.
甚至將盟約關係與血緣關係做出了比較,.
所以《論家庭》這本書裡面,.
將家庭的界線超越了血緣的關係,.
將盟約當中無條件的愛,.
那種的特性,.
了解家庭的意義,.
遠超過血緣的關係,.
血緣關係只是提供家庭歸屬感的唯一基礎,.
家庭是你可以無條件的被愛,.
甚至在你最不配的愛的時候,.
還可以指望被愛的所在,.
什麼令到你們不能夠重建,.
你們沒有面對你們的問題,.
你們沒有被傷害的,.
一直在等著別人對他的道歉,.
什麼時候你會好像上帝這樣主動的來到我們當中,.
這個盟約其實令到今天很多婚姻都已經死亡,.
關起文,.
就是香港一個社會學家,.
他就說今天社會存在的婚姻關係,.
已經從盟約轉到和同的關係裡面,.
這是一種個人隨心所欲的婚姻關係,.
他甚至強調這種轉變,.
使得婚姻已經到達平民死亡的邊緣,.
關起文在文章指出,.
婚約由原本自始不渝的條文裡面,.
演變成以自立為基礎的和同,.
這意味著一旦他不能滿足自我的喜愛和慾望的時候,.

$^{801}$就會受終正寢,.
弟兄姐妹,.
什麼是盟約呢?.
重建盟約的內容關係,.
我們要教導信徒對盟約內涵的認識,.
包括當中勝利的意義和教訓的角色,.
將可以重建婚姻穩定性跟長遠性,.
盟約其實在《聖經》裡面講很多,.
今天時間已經到了,.
我不能再多講,.
希望有機會我們可以有一個專題講盟約就好了,.
因為盟約是一個豐富的不得了,.
如果我們今天一直在哪裡收補,.
收來收去,收不好,.
缺少什麼?.
就是我們不能收到一個條約,.
是他又滿意,我又滿意的,.
而且在不同環境都能夠生存的,.
我相信我們找不到這種的合同,.
當日,當你的妻子說我愛你,.
你是我的丈夫的時候,.
可能是她以為你可以一直愛她,.
照顧她一生,.
就會改變了,.
條件改變了,.
她現在賺錢比你更多,.
但是還是要讓你好像太上王來看的時候,.
她很難受的,.
對不對?.
所以呢,.
怎麼去表達呢?.
《聖經》第一次提到盟約,.
這我不講了,.
你可以回去看,.
在盟約裡面,.
《創世記》第六章講到,.
第一次講到,.
第九章重在重複,.
那個盟約的歌音曲,.
那個盟約呢,.

$^{841}$在神與人立約的特點裡面,.
在盟約裡面,.
有三個很特別重要的,.
神主動與羅亞,.
阿巴拉罕立約,.
並沒有達到他們的同意,.
就是盟約不是說我們同意,.
不是我愛你就是我愛你,.
無論你怎麼改變,.
那個愛一直都在我的心裡面,.
從那個婚姻那天開始,.
我們立了盟約,.
就是不論在什麼環境裡面,.
病嗎?死嗎?軟弱嗎?.
我都是還愛你,.
你真的這樣說嗎?.
第二,.
神對他們是有期望,.
而且有要求的,.
但不因為他們的回應,.
而決定盟約的內容,.
盟約是沒有條件,.
當然當中的好處與祝福是有條件的,.
我的關係沒有條件,.
我對你的愛沒有條件,.
但是我們能不能達到這個.
婚姻裡面最好的回報的話,.
那麼當然這是有條件,.
是兩方面都要進去的,.
所以神與人合約沒有定下有消息,.
在神的眼當中,.
盟約是一種關係狀況,.
而不是一個處境,.
是一種永恆的關係,.
而非一個交易,.
這就是婚姻應該有的性質,.
正如今天一對新人在正婚當中,.
面前簽署了婚約,.
有簽署日期,.
但沒有有效的日期,.

$^{881}$可是,.
人自己用了簽署離婚書的日期,.
取代了婚約裡面能夠有的永恆性,.
所以,求求幫助我們,.
在婚姻裡面面對困難的時候,.
因為何西亞做了三件事情,.
公開挑戰,.
公開爭辯,.
指控他們婚姻裡面出現問題,.
妻子怎樣不忠,.
甚至叫他的兒女,.
來,跟你的母親來爭辯,.
你們就是這樣生活下來,.
但是在爭辯之後,.
他用更大的愛,.
主動的愛,.
有盼望的愛,.
然後給他一個機會,.
最後呢,.
他在哪裡?.
回到他們原本婚姻的基本當中,.
這是盟約的開始的一個時間,.
他們那種的愛,.
婚姻就有盼望,.
最後跟你分享幾句話,.
這是著名的神學家,.
潘福華,.
他所說的,.
他說,婚姻,.
不單是你們雙方的愛,.
他擁有更崇高的尊嚴和能力,.
是受太肩負對世界和人類的責任,.
你的愛,.
是你個人的私有財產,.
但婚姻,.
是超越個人的,.
它是一個身份,.
一個職勢,.
使你們兩個在生命中,.
結連起來..

$^{921}$求主祝福我們,.
雖然在短短的五十分鐘當中,.
我們不能夠全部看完,.
因為佛西亞,.
他們整整書裡面每一個世界,.
但是整整書裡面,.
代表他面對他的婚姻困難,.
也代表上帝面對那幫背叛他的子民,.
他們怎麼樣去將重建雙方的關係,.
只有我們在神面前,.
求您,.
求神,.
讓我們進到這個聖經裡面,.
知道我們是怎麼樣去把握這三個原則,.
來面對我們的婚姻..
好,謝謝你們..
再次謝謝你們給我的機會..
將就時間,.
交還給主席好嗎?.
好,.
大家看見我沒有嗎?.
我們實在是謝謝郭牧師,.
今天晚上給我們那麼重要的分享..
大家可以理解,.
就業聖經裡面其實真的是非常寶貴..
從就業聖經裡面,.
我們更能夠理解,.
我們所相信的神,.
背後原來有那麼豐富的意義..
我們以後還有機會,.
能夠開辦更多的這一類的講座,.
讓大家都可以從聖經裡面的書卷裡面,.
看見我們怎麼樣實行一個屬靈的生命,.
這是我們劍道中心一直以來都做的事情..
所以今天晚上,.
我就覺得非常謝謝郭牧師,.
他給我們那麼寶貴的時間,.
讓我們都可以從和希亞書,.
理解神的對婚姻的關係的理解..
現在我就想告訴大家,.

$^{961}$如果以後大家要留意,.
我們劍道中心舉行的講座的信息,.
請到我們的網站裡面,.
就是www.abscc.org..
我們劍道中心一直以來,.
我們都是依靠弟兄姊妹的奉獻,.
來支持我們的工作的..
所以如果大家覺得今天晚上,.
講座對你有幫助的,.
按照你在神明面前的感動,.
你可以做金錢上的奉獻,.
你可以到我們的網站裡面,.
看見奉獻知情神學教育,.
你點擊下去,.
就可以進入到PayPal裡面的奉獻項目裡面..
你可以一次過的奉獻也可以,.
你可以更長時間來支持我們,.
每一個月奉獻一點點的金錢,.
也可以幫助我們的工作..
或者你希望用支票也可以,.
如果是你用支票的話,.
請留意我們的地址,.
你可以寄進來..
如果你是用支票奉獻的,.
你超過30塊的奉獻,.
我們就會寄給你奉獻的收據..
還有告訴大家,.
今天晚上的講座,.
我們有錄音下來的,.
再過一兩天,.
就可以在我們的網站,.
弟兄姊妹也可以重新再看..
也許你可以介紹給弟兄姊妹,.
如果今天晚上不方便來的,.
也可以在我們的網站重新再看,.
今天晚上的講座..
所以求神祝福大家,.
讓大家都可以在聖經裡面更深入理解,.
不單止深入理解,.
更能夠從當中學到怎麼樣實踐裡面的真理..

$^{1001}$現在我就做一個簡單的禱告,.
來結束今天晚上的聚會好不好?.
請大家低頭..
親愛的主耶穌,.
我們謝謝你今天晚上給我們這麼好的訊息,.
讓我們都懂得怎麼樣叫盟約,.
怎麼樣看待我們婚姻的關係,.
好讓我們更理解神,.
你是那麼豐富的一位神..
你的話語真是我們腳前的燈,路上的光..
久久恩待我們眾弟兄姊妹,.
不管我們身處世界哪一個角落,.
祝你用你的話語來幫助我們,.
好讓我們的屬靈生命能夠繼續講進..
我們簡單的禱告,.
來祝福耶穌基督的名求..
阿門..
\newpage



\section{}
\label{sec:DOwKowWHLkM}
\textbf{你在帥領\_講座題目:「蒙揀選的身份:摩西的召命與掙扎」}
\newline
\newline
連結: \href{https://youtube.com/watch?v=DOwKowWHLkM}{\texttt{https://youtube.com/watch?v=DOwKowWHLkM}} ~~~~ 語音日期: 2020-10-23
\newline
\newline
\hyperref[sec:n5DpA1Db_0M]{\small{< < < PREV SERMON < < <}}
~
\hyperref[sec:index]{\small{[返主目錄]}}
~
\hyperref[sec:0]{\small{> > > NEXT SERMON > > >}}
\newline
\newline
$^{1}$是高博士的最新著作《你的名字》.
詳細情況我們會等到高博士的訊息分享後.
再加以宣佈.
以下我們有一個短片,兩分鐘.
簡單介紹加拿大建導中心的由來和現況.
以供大家認識我們.
在事之前我們先作一個祈禱.
請大家一起低頭祈禱.
我們能夠存留在世上,真是你的恩典.
我們今晚聚集,為加拿大建導中心.
做一個網上籌款活動的時候.
我們不要忘記,你仍然是帥領我們那位主.
叫我們在疫情底下,不會下台.
也不會失去朝氣.
而是看到天上的主怎樣一直拖帶我們.
渡過一切艱難的時候.
因為你是信實那位主.
你在歷世歷代,仍然是在天上的那位主.
求主幫助我們藉助今晚的訊息.
能夠更加堅強我們對你的信靠.
好讓我們真的明白到.
跟隨這位主,實在是我們一生人的福氣.
求主因代我們今晚的所有程序.
特別是當我們要用科技.
作全球的轉播的時候.
真的要保守科技順暢.
我們簡單的禱告,是奉耶穌基督的名頭.
阿門,阿門.
請大家現在看一個短片.
《信徒領袖》.
在上世紀八十年代.
由於大量的華人移居北美.
華人教會也陸續被建立起來.
信徒領袖期待能夠在公餘時間.
修讀廣東話的神學課程.
有見感業.
香港建道神學院與加東校友會合作.
成立加拿大建道中心.
到現在已經三十年了.
初期的建道中心主要是定期提供.

$^{41}$要許和現身的課程.
為加拿大的華人信徒提供基本的神學教育.
因應信徒對於更加進深的神學教育渴求日增.
建道中心在2009年正式開辦神學文憑課程.
同時更以視像同步授課的方式.
在多倫多以外的城市授課.
為當地的信徒提供神學訓練.
現在建道中心提供不同程度的神學課程.
裝備信徒領袖.
並且定期舉辦專題講座.
探討聖經真理如何實踐信仰.
盼望信徒能夠在日常生活中活出基督的樣式.
我們堅守以訓練信徒領袖是建道中心的使命.
期望透過進深而有系統的教導.
訓練出適切教會需要的信徒領袖.
從而幫助華人教會推動事工.
拓展神的角度.
希望大家看到我們的短片.
現在我想和大家交代一下.
加拿大建道中心在未來一年的計劃.
在未來一年裡.
我們的工作重點是會推介一個新的課程.
就是由香港建道神學院.
建立的聖經研究文學碩士海外課程.
由我們建道中心負責協助推出.
大家都認同.
聖經是我們信仰的核心部分.
信徒如果認真去做耶穌的門徒.
必須通過研習聖經.
來明白上帝對我們的心意.
因此我們認為這個課程十分適合.
有心追求認識神的信徒們去修讀.
我們會以廣東話授課.
但提交作業時可以用中文或英文.
課程是五年的時間.
可以完成12科的要求.
在這個投影片中已經簡單介紹了.
那12科的內容.
暫時因為疫情的情況.
所有的科目都在網上進行.

$^{81}$等待疫情過去後.
我們會恢復以網上.
加上面授的方式進行.
我們仍然堅持一個看法.
修讀聖經和神學.
最佳的方法.
都是老師和同學有面對面的交談.
如果你們對這個課程有興趣.
或有任何問題.
歡迎大家用電郵或打電話給我們.
在投影片下方已經有電郵地址和電話.
歡迎你們與我們聯絡.
多謝大家.
各位弟兄姊妹.
各位在世界各地不同地方的弟兄姊妹.
我很高興介紹今晚的講員.
我們的講員是高永軒牧師博士.
他是舊約聖經的專家.
高博士現在在香港建道神學院任教.
是聖經研究部的主任.
也是建道神學院.
彭建輝教席的副教授.
相信大家對高博士可能有些聽聞.
他是近年建道神學院最講為人知.
和最受歡迎的講員之一.
我們非常榮幸今晚能夠請到他.
作為我們今次籌款講座的講員.
今晚高博士的講題是.
蒙簡選的身份.
摩西的照命和掙扎.
一個題目非常配合我們今次聚會的主題.
就是你在瑞靈.
在艱難的日子裡.
能夠給我們力量.
仍然是我們的神.
並且堅強地認定.
他仍然是率領我們向前的那位主.
我們有請高博士給我們一個訊息的分享.
請高博士.
各位聽眾應該晚安.

$^{121}$和大家一起思想上帝的話語.
是我非常興奮的事情.
我們知道我們面對一個很大的困難.
但我們仍然深信上帝帶領我們.
他的同在比一切重要.
他是一個壓倒性的同在.
他是很多變數當中的常數.
讓我們能夠透過他的話語.
一起去研讀.
和我們一起去認定.
我們核心的身份到底是怎樣.
今天我會用七個字的訊息.
和大家分享摩西的生命掙扎.
摩西的生命掙扎.
其實是一個很有血有肉的人.
讓我們明白.
在一切的變數.
或對自己的身份的質疑當中.
上帝是怎樣去鼓勵摩西.
從而鼓勵我們每一個人的生命.
今天我想和大家分享的經文.
是出自幾段的經文.
第一段的經文是來自《春海求其》第三章.
然後是第四章, 第六章.
為何我們選擇這三段經文呢?.
原來這三段經文都有一個次序.
首先有上帝的使命和他的召命給摩西.
然後也有摩西的拒絕.
即是摩西有三個理由去拒絕上帝.
然後上帝要回應摩西的拒絕.
就有三個肯定給摩西.
我們真的很想看看他的拒絕和上帝的肯定.
讓我們去明瞭摩西的生命是怎樣成為我們的幫助.
我們一起有個祈禱作為開始.
多謝天父, 肯求主透過你的話語.
塗灶更新我們的生命.
當我們覺得疲憊,失望, 或失去方向迷失的時候.
主你的話語永遠都是我們腳前的燈.
你提醒我們每一個核心的身份和召命.
蒙簡宣的身份, 以致我們認定你是說領著我們.

$^{161}$多謝主, 我們的禱告是奉耶穌基督的名號.
今天的訊息分為三段經文跟大家分享.
第一段就是摩西蒙召最核心的問題.
就是「我是什麼人」的問題.
經文是這樣說, 十一到十五節.
摩西對神說「我是什麼人」竟然能夠去到見法老.
其實大家也知道摩西在第三章第一節.
描述摩西進行日常的牧羊工作.
其實不是一些很特別的日子.
是極度平凡和平常的日子.
但很多時候都是這樣.
在九曰裡, 遇見每一個要蒙召的人.
通常都是在平凡,平常甚至柴米油鹽的日子裡跟他說話.
他平常這樣去到神的山.
竟然見到一個叫做「荊棘與火」的異象.
這個異象在摩西的人生裡非常特別.
成為他人生的轉捩點.
第六節就說神介紹自己是阿伯拉罕的神.
以撒的神, 雅各的神.
聽到那群以色列民在埃及的苦難和哀求的聲音.
所以就清晰地表明要使用摩西去實踐一個使命.
就是要帶領以色列民離開埃及進入迦南地.
一個這麼mission impossible的使命.
對摩西來說實在是非常困難.
所以這段經文一開始就說我是什麼人.
Who am I.
竟然能夠去到見摩西.
見這個白鷺.
把以色列人從埃及領出來.
神就說我必與你同在.
你要把百姓從埃及領出來.
你就必在這山上去到伺奉我.
這就是我打發你的證據.
摩西對神說, 我到以色列民那裡.
如果有人問我你是誰.
你是什麼名字, 我應該怎樣回答.
耶和華就說, 我就是那位自由永有的.
又說你要對以色列人說.
那位自由的, 去到打發到你那裡.
摩西就說, 你要對以色列人這樣說.

$^{201}$耶和華就說, 你列祖的神.
阿伯拉罕以撒和雅各的神.
打發我到你們這裡來.
耶和華是我的名, 直到永遠.
也是我的紀念, 直到萬代.
一如既往, 很多舊約的讀人.
他們蒙召的時候.
在故事裡, 一定有一種我們叫拒絕上帝的情況.
以塞亞的拒絕, 就是我嘴唇不見.
耶利米書第一章, 耶利米的拒絕就是.
我還是年幼.
基奠蒙召, 就是做事司.
他的拒絕就是在那個像樹底下.
在酒駕那裡, 預備食物的情況.
所以我們看到, 在舊約裡有一個模式.
那個模式就是, 蒙召或是揀選的時候.
總是有一個拒絕.
而摩西拒絕的第一個回應.
就是我是什麼人.
面對神要求摩西做的事情.
摩西, 這個就是他的拒絕.
認為自己的身份地位, 是不可以承認這個任務.
所以, 我是誰, 就是這個問題.
就好像盧雲有一段說話.
寄在他的一本書裡.
他說, 我是誰, 這個問題.
就是很多人都在問的問題.
又或者, 如果我們不是很明顯去問.
實際上, 我們經常都在問, 我是誰.
我是誰, 盧雲這樣說.
我是被喜愛, 被讚賞, 被羨慕, 被恨戶, 被痛恨, 被鄙視的一個.
無論我是鋼琴家, 生意人, 所在乎的.
就是這個世界, 會怎樣看我.
如果忙碌是一件好的事情.
我就必須去到忙碌.
如果有錢, 是爭自由的表現.
這樣, 我就必須爭取更多應得的金錢.
如果郊遊廣闊, 是證明我的重要性.
這樣, 我就必須保持和重要的人保持聯絡.
那種強制, 就是世界給我們的強制.

$^{241}$就彰顯在我們內心裡潛藏對失敗的恐懼.
以至為避免失敗, 而經常驅拆自己去囤積更多.
更多的工作, 更多的金錢, 更多的朋友, 更多的關係.
更多的學位, 更多的地位.
我們會發現, 這個世界強制我們.
要索做我們, 究竟我們的人生.
是在問的是"我是誰".
就是去到這些方向那裡.
摩西話"我是誰"大概就是受到當時文化所定義的領導.
或者領袖的質素去影響.
認為自己只不過是一個牧羊人.
自己的確曾經是埃及的王子.
但這個身份未能讓他避免法老王的追殺.
昔日曾經去到嘗試救過自己的民族.
但總是失敗而回.
倒不如安安定定就牧羊四十年, 安分守己.
但是在這裡, 摩西所說的"我是誰".
正正就是確實帶有一種叫做遺憾,自卑.
就是我嘗試過, 我努力過, 但都是失敗.
我們的人生其實都是這樣.
我們努力扮演著別人期望的我們自己.
我們迷失了.
我們真的嘗試活出那個照明.
但因為種種的失敗, 而令我們太現實.
我們意想不到神期望我們活出的"我是誰".
和這個世界要求我們的"我是誰".
的確是南轅北轍的一種情況.
而神的回應真的非常特別.
首先, 神就在第十二節那裡.
去到首先說明他自己就要和摩西同在.
其實指出人的因素不是重點.
重點就是有神同在這個因素.
十二節其實是提出一個呼召的證據.
而這個證據很特別.
就是要領以色列民到達摩西所身處的河列山.
也就是在西奈山上.
就成為神打發摩西的證據.
而最有趣的就是這個所謂呼召的印證.
並非在摩西蒙召的當下所立時能夠看到這個印證.
而是他願意踏出這個召命的道路.

$^{281}$認真去到帶領以色列民離開埃及.
到達西奈山之後.
才能夠證明到這個印證.
其實是一反我們對印證的看法.
原來摩西經驗神的打發證據.
不是現在式而是將來式.
這是很特別的.
將來才能看到證據.
我們以為看完證據才能踏上.
但摩西就倒轉了.
先踏上才能看到證據.
神第二個回應就是.
我是自有永有的.
原文直譯就是I am who I am.
或者中文直譯就是我是那位我是.
我是那位我是的描述.
對很多人來說其實是一種混亂.
因為代表什麼呢?.
代表上帝在人類的類別當中.
找不到任何的落日點.
這是一個名字.
亦是一個非名字.
摩西和以色列民竟然和這位奧秘.
和不能理解的上帝去到立約.
是一個不可思議的事情.
我是那位我是.
其實是一個霸氣的宣言.
說明只有上帝才能定義自己.
他是一位令我們驚訝的上帝.
超越了我們所理解和想像.
是這位奧秘的上帝我們只能夠謙卑.
所以就好像摩西一樣.
夢想面因為怕上帝.
這是第六節所說.
摩西的怕是因為他面對自己.
不能去到定義和控制的上帝.
在懼怕和失望當中.
摩西才真正領受上帝的呼召.
所以我介紹給大家一個非常好的.
出安及記的學者.

$^{321}$我覺得他的《釋經》書.
在出安及記是我會首選買他的那本書.
他說了一句話.
The more one understands God.
The more mysterious God becomes.
The more you know.
The more you know you don't know.
這句話是.
你更認識上帝.
你就更加知道你更多關於上帝.
或者上帝本身你完全不知道.
因為上帝是不能掌控的上帝.
我是自有永有.
沒有人能定義上帝的一個名字.
當人在界限當中才能看到上帝的真實.
近來香港熱門的話題.
我想大家都知道.
是講移民.
聽聞多倫多也是一個非常熱門的移民地方.
我教會很多弟兄姊妹.
其實都在討論.
現在不是在討論移不移民.
現在是在討論去哪裡.
有些人想澳洲.
有些人想台灣.
很多人都想英國.
不過也有很多人想加拿大.
我最近看到一個故事.
很有趣的.
我也想跟大家分享.
故事是有一個香港人.
他本身有2億美元身家.
2億美元其實是很多的錢.
他覺得香港不太安全.
所以他打算開一個離岸的戶口.
把錢撥過去.
他選擇哪個國家的離岸戶口.
他一想就想起新加坡.
新加坡的銀行看到有這樣的生意.
就派了兩個職員來香港.

$^{361}$跟他談這個安排.
辦好所有手續.
他就詢問這位2億美元身家的人士.
他為甚麼想在新加坡開一個離岸戶口呢.
這位人士說了一些說話很有意思.
他說其實我自己都已經有一個英國的護照.
即是說他是英國公民.
但總覺得自己的身份認同不是英國人.
他總覺得自己是中國人.
他就環視全球.
哪裡比較多中國人.
他的身份認同比較好.
但又安全.
他就想起新加坡.
所以他就撥了那些錢去新加坡.
然後他就跟那兩位職員.
分享了一句說話.
很有趣的說話.
就是我認為我是一個難民.
他覺得自己是一個難民.
這句說話很諷刺.
但也很有意思.
就是我們在香港即將會出現.
2億美元身家的難民.
其實是一件很有趣的事情.
人生在世.
無論是去多倫多.
或者是去伯克利斯.
又或者是去美國,澳洲.
我們認定我們都是客人.
我們都是寄居.
無論你移民去哪裡.
我們也都深信.
作為香港人.
有很多時候我們真的沒有辦法.
找到一個地方.
真的全心全意屬於那個地方.
真的覺得很舒服.
令到我們的身份認同.
真的很好.

$^{401}$但在這裡我們看到.
Who am I這個問題.
仍然都是在問我們的問題.
我們核心的身份究竟是甚麼.
如果我們相信上帝的帶領.
我們明白我們核心的身份.
並不是在世的公民的身份.
我們有一個更高的身份.
那就是天國子民的身份.
在世寄居70年,80年.
若然我們能夠認定.
那位上帝是I am who I am.
我們就知道.
這就是最拿捏得最實的一件事.
我們知道就算我們有2億美元身家.
我們都沒有辦法令到我們的心安定.
或許越多錢就越不安定.
所以我們明白一點.
就是核心身份定義我們錢的流向.
他自己在想甚麼.
他的財寶在哪裡.
他的心就在哪裡.
所以在這裡我們就發現.
Who I am.
我們就要問我們自己一個核心的問題.
但這個問題的回應是甚麼呢.
上帝對他的回應就是.
I am who I am.
真正回應「我是誰」這個身份.
是由上帝揭示你他是誰.
上帝是誰.
決定你是誰.
有很多時候我們倒轉.
我們在想我是誰.
上帝你配合我是誰.
但在這裡耶和華對摩西的回應是倒轉.
上帝就是這樣.
我們配合上帝是誰.
當我們配合上帝的時候.
我們因此就找到我們的核心身份所在.

$^{441}$這是非常弔詭又非常真實的概念.
這是第一點.
第二點就是第四章.
絕口笨舌.
這是摩西第二個理由去回應上帝.
拒絕上帝.
當我們討論絕口笨舌這件事.
我們就必須討論甚麼是先知.
因為大家明白摩西是先知.
我們每個人都知道.
很多人以為舊約的先知是未卜先知.
好像預測未來的事情.
他的工作就是預測未發生的事情.
又或者有些人覺得先知是先過利知.
他知道先然後你才知道.
先知的原文其實不是未卜先知.
也不是先過利知.
原文的B字是解作代言人.
又或者傳話者.
也就是代表上帝去到發言的一個師封的人.
先知主要的任務其實不是說話.
當然他會說話.
但先知主要的任務就是聽話.
聽清楚上帝的發言.
然後根據上帝的發言去宣講.
所以成為先知的基本條件其實不是口才.
先知的基本條件不是那把口.
先知的基本條件是他的耳朵.
他的耳朵需要有一個虛心聆聽的耳朵.
這是先知最重要的條件.
《春華及極》第四章第十到第十六節的經文.
就成為先知典範的創新.
一切舊約的先知其實都是根據第四章第十到第十六節.
這段經文作為先知.
摩西的原則是什麼呢?.
摩西的原則就是.
我必賜你口才,指教你所當說的話.
亞倫的原則就是.
你要作他的口,他要當作你的神.
因此亞倫其實就是摩西的代言人.

$^{481}$即是說亞倫是摩西的先知.
而摩西是神的代言人.
即是說摩西就是神的先知.
第四章第十節.
摩西主要用三個特徵去形容自己.
自己不適合做先知.
首先他認為自己是一個不是能言的人.
如果以豹王的翻譯.
能言的人應該譯作a man of words.
當中的言語在上下文體內代表上帝的說話.
言語的人或者man of words.
是指神的代言人.
是指明代表神說話的人.
這是先知的角色.
第二,摩西認為自己不是代言人的事實.
是永久性的.
因為他只說了一句.
從你對僕人說話以後.
我都是這樣,我都是一個不能說話的人.
他跟主說話之後.
他不會改變,也不能勝任這個先知的角色.
第三,他認為自己是絕口笨舌的人.
如果原文直譯可以譯作happy mouth.
即是你的口是非常沉重的意思.
就代表他的說話吞吞吐吐,速度非常慢.
善天不足,無藥可救.
所以以上三點.
摩西說明自己是口吃的.
他甚至連做講員的資格都沒有.
就算耶和華呼召摩西作為代言人.
都不會改變他這個絕口笨舌的現實.
摩西在耶和華面前.
曾經透過第六章十二節.
摩西在耶和華面前說.
以色列人尚且不聽我的話.
摩西怎能聽我這絕口笨舌的話呢?.
大家想清楚.
神真的有很多方法跟人說話.
他能夠用奇妙的聲音說話.
在天上整個雷電.

$^{521}$一把很威嚴的聲音跟眾人說話.
以至所有人都能驚訝和驚魂.
他都能夠藉著天使的聲音.
能夠神跡地向世人顯現.
這個效果可能更加準確,更加好.
但上帝使用的方式都不是這些.
上帝竟然藉著我們這些看不順眼的傳道人.
或者看不順眼的弟兄姊妹.
站在不能夠吸引人的講壇上.
這是不能理解的道理.
我們每一個人都是絕口笨舌.
過去是這樣,現在是這樣,將來都是這樣.
我們成為牧師,或者成為弟兄姊妹.
傳講上帝的話語的厚劈絕口笨舌的本質.
始終都沒有變過.
我們都跟摩西一樣.
只不過我們是蒙簡選的絕口笨舌.
是一個蒙恩的罪人.
卻能夠跟神聖的代言人的職事連在一起.
是一個超理性的說話.
講一個很特別的經歷給大家聽.
我還記得我剛做傳道人的時候.
那時候應該是在神學剛畢業.
大約十三年前.
在城市大學的一個基督教組織.
請我去講一個報道會.
在城市大學裡面講.
這個報道會我自己預備得很辛苦.
人生講第一次報道會就要認真一點預備.
講的時候,整個過程我見大學生都沒有什麼反應.
我沒辦法,都照樣講.
講到有一個字,有一個地方.
因為我自己,如果大家認識我.
我是一個印尼華僑.
什麼叫印尼華僑?.
我從小的時候,剛出生.
我本身不懂廣東話.
我只懂講普通話.
然後就講印尼話.
所以我從小就講印尼話和普通話去長大.

$^{561}$去到幼稚園的時候.
我才懂得講廣東話.
所以從小到大,就算在香港出生.
我的廣東話,其實有些拗字都不太好.
所以有些人都問我.
會不會你是馬來西亞華僑呢?.
又或者是印尼華僑呢?.
所以有時拗字的人心急.
又或者緊張的時候就會講錯話.
我就在報道裡面講講.
有一個字拗字不正.
以至發音是一個髒話.
然後全場都在笑.
我就覺得很面目.
因為我不是想講那個字.
但拗字不正,太緊張就講了那個字.
我就覺得很面目.
想找洞鑽.
講完之後,最後都沒有人缺字.
我就離開.
離開的時候.
我覺得這次應該沒有辦法給上帝使用.
覺得自己那把口沒辦法拗字.
心情很低落,很自卑.
然後過了第二天.
機構的負責人打電話來.
說高傳道你好.
我有個好消息告訴你.
你分享的那一晚.
有一位大學生在那一天去到缺字相信耶穌.
聽起來一位很少.
但對我來說已經是很大鼓勵.
因為有一個靈魂不救,天君天使都會搬起.
我自己覺得有時上帝使用一個人.
並不取決於他是不是住口分子.
最重要是給上帝使用.
有時正正是你的能力不足.
但上帝竟然使用,你就讚嘆.
其實有很多時不是你的努力.
而是上帝的作為.

$^{601}$你看到一種敬畏,好像站在聖地.
因為你和上帝同工.
變成你看到上帝的作為.
你竟然參與過程中.
這是很特別的情況.
所以絕口笨事不是重點.
有很多人其實是被上帝使用.
你會聽他的故事.
你都會知道他說話的口其實不是很正.
又或者不是很標準.
甚至不是琅琅上口那些.
但他被上帝使用.
哪怕是一個絕口分子的口.
仍然都是核心身份所在.
所以最偉大的先知是絕口分子.
這是摩西.
所以我們如果不會超越摩西的話.
我們都安心,因為這是上帝的使用.
第三,就是面對失敗.
摩西真的有幾次失敗.
譬如在第五章有句經文.
摩西回到耶和華那裡.
主,你為何輔待這些百姓.
為何打發我去呢.
我去見法老奉你的名說話.
你就輔待這些百姓.
你一點都沒有拯救他.
即是他做上帝吩咐他做的事.
是一點拯救都沒有.
就說明摩西根據耶和華的吩咐去見法老.
帶卻惹來不好的效果.
以至摩西有兩節說話.
叫我們不禁自問.
回應上帝的呼召.
以夢簡選的身份.
傳遞上帝的話語.
這不是傳道人的專利.
這是每一個學習上帝的話語.
是否一定會成功呢.
而最要命的問題就是.

$^{641}$為何神不幫助我去做神的事.
我已經做神的事了.
但好像上帝作弄我一樣.
既然是神的事,為何你不幫助我.
神是否失信的上帝.
22節的回應.
是摩西質問上帝的打法.
甚至質問上帝的動機.
其實不是為了拯救百姓.
其實是為了要輔待百姓.
23節就說明神一點都沒有拯救他們.
直指神言而無信,出而反而.
但是神卻再次肯定和鼓勵摩西.
他沒有正面回應摩西的質問.
只是不斷作出肯定.
事實上,神沒有義務去回應我們所有問題.
祂只是堅定地回應摩西.
告訴摩西他將要做甚麼拯救行動.
第一,在第六章的第一節說到.
以色列人要讓法老看到上帝的作為.
這是說的是佛哉,即是上帝將會賜下佛哉.
第六章的三到五節,就說要堅定向列祖所納的藥.
這是前迦南地的藥.
六到九節,就說因為這個藥.
所以上帝要拯救以色列人.
帶領他們進入迦南地.
因為藥的緣故,所以上帝要拯救他們離開埃及進入迦南地.
所以說上帝回應是說甚麼.
回應是讓人看到上帝的作為.
第一,第二就是上帝一定履行的應許和文約.
賜迦南地給以色列民.
神明白人的軟弱.
人需要具體的行動才能經歷神的帶領.
在這裡,神確是說他將會的計劃.
但這個描述不一定是說事公非常有過效.
不一定一帆風順,而是實相.
經歷失敗和失信,往往是每一個侍奉的人所必須經歷的東西.
有失敗是正常,沒有失敗是不正常.
如果沒有失敗,其實就是值得感恩.
有失敗就是值得成長.

$^{681}$反而有時我們因為太愛主,太認真去侍奉,才能經歷失敗的美好.
有時失敗是我們的常數.
失敗就是我們努力侍奉的證據.
我人生中也經歷過一個非常大的失敗.
不妨跟大家分享.
當我在2010年飛到英國讀PhD.
整個讀書過程中,有些挫折.
但到了最後,也很順利.
本來也順利.
本來博士學位要用三年完成.
Standard的課程,36個月.
但因為自己很想省錢,最後一年交學位.
所以我比平時做得更勤奮.
大約做了26個月.
20個月後就申請了PhD論文進入校試.
以致我不用繳第三年學費.
本身很順利.
我的主管也擔心我的論文成績不好.
所以他給了第二位主管去看我的論文是否適合.
最後第一和第二位都覺得合格.
可以合資格進入論文的校試.
但又不放心,所以第一位主管給了退休教授.
他比較多經驗,看看我的論文是否合格.
最後退休教授也說合格.
我帶著十拿九穩的心態.
正式申請了整篇十萬字的論文.
申請後,我回到香港.
當時是2013年5月.
然後安排2013年8月底進行考試.
當考試通過就拿到博士學位.
以致2013年9月順利在建度神學院教書.
建度和我的協議是9月或之前拿到博士學位才有教席.
否則可能就不成功.
當時我沒有想過不成功.
因為三位教授都說可以.
應該是8月底的考試非常順利.
我帶著一個必勝的心態.
8月再飛去德倫進行考試.
兩小時後,有一個外國考試生詢問我.
整個過程都很辛苦,他都問得很尖銳.

$^{721}$最後的結果竟然是不通過.
令我非常吃驚.
我還記得那一刻有一點暈倒的感覺.
因為我們全家在8月初已經搬進了長洲建度的宿舍.
9月沒有教席是比喻所思.
自己就通過不了.
他就說要用半年時間去改.
大約要重寫一萬字.
改完之後,應該在第的話.
你跟足改,應該可以通過.
那一刻我都不太聽進去.
因為明明要在9月之前有一個博士學位.
才可以順利拿到教席.
我很沮喪,很不安.
回到校,梁院長非常好.
他就跟我說不要擔心教席的問題.
現在不需要你做很多事.
你專心去改好那篇論文.
當時我的上師謝偉儀老師是《聖經學報》的主任.
他都鼓勵我,提醒我.
他都看好我,希望我能夠改得好.
其實我從一個高峰跌落一個低谷.
我覺得是人生一個很大的失敗.
我質疑自己的能力.
我質疑自己是否能夠適合做神學院的老師.
質疑很多事情.
但是因為梁院長和謝老師的鼓勵.
和身邊很多看到的同事的鼓勵.
自己在這半年改好那篇論文.
看了很多德文和法文書.
因為他要我加這些書籍進去.
要改,重寫一萬字.
都是一個漫長和痛苦的歷程.
但是很感恩.
整個過程就是最終能夠在2014年5月取得博士學位.
整個過程,回想起來.
其實都是驚心動魄的一個情況.
我們面對失敗,我很明白.
面對失敗其實是會否定自己.
我們做神的工作.

$^{761}$我們知道有時失敗就等於自己沒有做到那個價值.
就等於上帝沒有幫忙.
但是大家想清楚.
當你面對失敗的時候.
神總是在你身邊出現一些人.
去鼓勵和支持你.
而那鼓勵和支持正正就是上帝對你的肯定.
上帝對你的安慰,上帝對你的支持.
有很多時人的眼睛只看到失敗.
看不到身邊原來那麼多的天使是上帝派過來的.
這是很重要的.
所以我們有三點和大家分享.
第一點是什麼呢.
第一點就是我剛才提到的.
就是我是什麼人.
問一問自己,神要說順利我們.
我是什麼,Who am I.
這是核心問題.
而神的回應就是用I am who I am來回應.
不是說你是誰.
首先要問你認識上帝是誰.
當你認識上帝是誰.
你才能回答Who am I這個問題.
第二就是絕口憤怯.
其實這不是重點.
原來上帝有時需要使用絕口憤怯的人.
去成就他做代言者先知的角色.
第三,每一個夢照的人都要面對失敗.
失敗是常數.
我們人生面對失敗.
但最重要的是上帝的應許.
他要應許賜你迦南地.
在過程當中,人看來是失敗,但這是必經路程.
有很多時候我們面對失敗才能夠偏悲.
我們明白我們的身份不是用自我來定義.
我們明白我們的生命原來每個人都是魔鬼.
我們在Who am I的問題中迷失.
我們在能力不達,絕口憤怯的問題迷失.
甚至我們在失敗中迷失.
當我們迷失在一個這麼困難的日子.

$^{801}$最原點的問題就是問我核心身份是甚麼.
那個夢揀選的身份.
一切的掙扎我們明白有神同在.
最重要我們服侍的就是一個Mysterious God.
就是一個奧秘的上帝.
祂帥我們,祂帶領我們,祂成為我們的幫助.
我們一起祈禱.
多謝天父使用摩西這段經歷成為我們人生的借鏡.
求讓你讓我們拿捏得準「我是誰」這個問題.
取決於你是誰.
求你讓我們看得到自己自身的絕口憤怯.
不是一個重點,而是看到你的選照才是重點.
主呀,你讓我們看到我們人生縱有失敗.
但讓我們看得見你會顯示你的作為.
以及那個應去必要成就.
所以你才是重點.
所以你才是我們堅持的理由.
你再說領我們如此踏水.
所以我們貧順到相信你成為我們生命的主宰.
多謝你去賜下你寶貴的話語.
讓我們謙卑,有信心去領受.
我們的禱告,是奉耶穌基督的名禱.
阿們.
實在令人感動.
叫我們不要忘記,帥領我們繼續向前的究竟是誰呢?.
就是這位創造天地萬物的主宰.
多謝高博士給我們很大的鼓勵性.
讓我們在人生裡.
無論是什麼光景,我們都能夠把持對神的信心.
稍後我們會提到贈書計劃的詳細內容.
在之前我們會說說.
我們今天的籌款活動的意思在哪裡.
其實見到中心每年的經費達到30萬加元.
我們的收入來源主要是學費.
教會和弟兄姐妹愛心的奉獻而來.
在疫情底下,大家都沒什麼機會去旅行.
又少機會去吃大餐等等.
有沒有想過可以用你的金錢去做一些更有意義的事呢?.
又或者你覺得你個人的經濟力量很有限.
但你有沒有想過,無論多少都好.

$^{841}$你的奉獻對神來說是一種天國事業的投資.
是有永恆的價值.
又或者礙於種種的原因.
你也許不會去見到中心去修讀一些課程.
但你有沒有想過可以做一些小小的奉獻.
特別指定用在對新同學的學費資助上呢?.
如果你有這樣的想法.
你可以在你的奉獻用PayPal的時候.
當PayPal發給你的收據.
你就轉發電郵給我們.
將作為學費補助的計劃.
用途告訴我們就可以了.
在螢光幕上,大家如果想資助我們.
為我們做一些金錢的奉獻,可以有三種途徑.
第一種是鼓勵大家可以在網上的奉獻.
先進入我們的網站.
abscc.org.
你就點擊一個紫色的部分.
就是奉獻支持神學教育.
這個做法其實是用PayPal的方式去進行奉獻.
立即可以收到PayPal發給你們的報稅收據.
這是最快捷的.
第二種的方法.
有些大英哲學的喜歡用支票.
用支票奉獻也可以.
桌頭上寫abscc.
然後寄給見到中心.
見到中心的地址,剛才的頭影片也說了.
你可以回到我們的網頁的第一頁.
最下面的那裡,講到奉獻支持那裡.
你點擊一按,就找到我們中心的地址.
在今年年底,我們就會發報稅收據給你們.
第三種,也有些大英哲學喜歡的.
直接上來見到中心,交奉獻給我們.
更加能夠看到見到中心裡面的設施等等.
為了答謝奉獻給我們的大英姐妹.
我們有贈送高永軒博士最近的著作計劃.
那本書叫做《你的名字》.
這本書剛剛第二次印刷.
在香港非常暢銷.

$^{881}$這次是第二次印刷.
是很搶手的一本書.
在多倫多這個地方.
現在還不可能在坊間買得到.
我們現在有現貨.
可以郵寄給你們的奉獻者.
直至到你們的府上可以收到.
所以凡是奉獻給我們120元以上加幣的.
我們就會在一個星期之內.
寄那本書到府上.
我們決定這個贈書的計劃.
會維持一個月的時間.
直到11月20日.
最近這個星期由於很多人知道有贈書的計劃.
立刻作出奉獻.
甚至親身來到建造中心找我們拿書.
現在剩下的書的數量其實不是很多.
所以如果你有興趣真的拿到這本書.
高博士這本《你的名字》的書.
請你盡快作出行動.
手續很簡單.
當你以PayPal在網上奉獻加幣120元或以上.
在電郵會收到報稅收據之後.
請你轉發給我們以下的電郵地址.
accounting@abscc.org.
好像螢光幕顯示一樣.
accounting@abscc.org.
清楚讓我們知道你家的地址.
否則我們寄不到給你.
如果你是用郵寄支票給我們.
請你在信中說清楚.
你是希望可以拿到這本書《你的名字》.
寫清楚你的地址.
我們同樣可以寄發給你們.
拿贈書的手續詳情.
都可以在我們的網站第一頁.
贈書計劃的網頁找到.
所以不用擔心.
如果錯過了剛才的宣佈.
完全可以在我們的網站.

$^{921}$abscc.org找到.
希望大家可以用金錢支持我們這個事公.
讓這個事公得到神的喜悅.
在這個時候我們想請.
加拿大建度中心董事會主席.
譚文坤牧師.
為我們的奉獻作一個禱告.
有請譚牧師.
讓我們一同同心祈禱.
親愛的天父上帝.
我們多謝你.
你過往多年來對建度神學院.
對加拿大建度中心.
大大的祝福.
我們今天眾弟兄姊妹.
來到你自己的庚前.
我們深信在你的私人補助前.
你會持恩典.
持福分.
讓香港的建度神學院的服侍.
加拿大建度中心.
努力去培訓工人.
這些工作.
得蒙你自己的月流.
我們都願意你自己保守.
今天晚上我們眾弟兄姊妹.
同心在一起去聆聽.
一個美好的教導.
原來基督耶穌一直在說領著我們.
我們願意成為你的僕人.
成為你的侍女.
求你大大的使用.
我們今天都願意為建度中心.
在加拿大的運作.
整體的財務需要.
獻上我們的禱告.
我們深信神你必定會公認.
因為在新冠疫情當中.
我們仍然看到神你自己的私恩.
來到口袋,宗教會,宗基籌.

$^{961}$我們特別願意將建度中心.
在做一些這麼有意義的工作.
向神你自己去陳明.
願你來到私恩看顧.
也願你的靈感動眾弟兄姊妹.
我們實在有很多建度的校友.
我們有很多曾經受過.
中心神學訓練的弟兄姊妹.
我們求你幫助我們眾人.
同心合意高舉耶穌基督的聖名.
在建度中心的財務需要上.
盡力去參與.
我們相信主你在暗中.
必定會報答.
我們都願意將我們中心.
當中眾位的童工.
特別六穆師弟的交託在你的手中.
讓你使出聰明智慧.
讓整體的事工的發展,運作.
必有你自己特殊的看顧.
我們為著2021年來禱告.
我們會有一個全新的建度課程.
在加拿大推廣.
這是聖經的碩士課程.
我們求你幫助引導.
讓你自己使用.
也讓我們眾弟兄姊妹.
宗教會信徒無恩.
我們在你的面前求你垂聽.
也願意回答我們眾人.
在你面前的禱告.
禱告是放靠我主.
基督耶穌的聖名而投.
\newpage



\section{}
\label{sec:0}
\textbf{寫給徘徊應否進入神學學習的你}
\newline
\newline
~~~~ 日期: 2021-06-02
\newline
\newline
\hyperref[sec:DOwKowWHLkM]{\small{< < < PREV SERMON < < <}}
~
\hyperref[sec:index]{\small{[返主目錄]}}
~
\hyperref[sec:1]{\small{> > > NEXT SERMON > > >}}
\newline
\newline
$^{1}$陸超明牧師 - 寫給徘徊應否進入神學學習的你

講題:寫給徘徊應否進入神學學習的你

講員:陸超明牧師

場合:ABSCC General

日期:2021年6月2日

對很多人而言, 讀神學仍然是那些接受上帝呼召作傳道的人才需要.這是過時的想法了.以前當神學教育尚未普及的時候, 神學的講授, 的確其焦點是放在訓練準傳道人的身上.近20-30年以來, 神學教育已經大大普及起來.廣大的信徒們, 只要有心, 便可以在公餘時間進修簡單的神學了.

 那麼, 為什麼信徒總要唸點神學呢? 我們都知道要認識我們相信的神.原來認識有兩個層面: 頭腦上“知道祂”, 并且在經歷上“真知道祂”.前者是知識性, 後者是體驗性.若果沒有前者堅實的客觀知識, 後者很可能是虛浮的主觀體驗了.正如有名的神學泰斗J.I. Packer所言, 我們眾人都應該從knowing about God (知識上認識神), 進而knowing God (體驗上真正認識祂).若果你能細味Packer的勸導, 真正有心向上追求的你, 便不會滿足於只是聽道/查經的活動.你需要的是跟從那些在神裡有深度理解的老師們學習.

 我的神學啟蒙老師有個很生動的比喻: 進入神學學習, 你就像一塊海綿被扔到安大略湖中:那是神學的大海洋,實在太吸引了.你便盡你所能吸收湖水; 但安大略湖的水位卻沒有絲毫的下降! 當我們肯花點時間(難免也要花點金錢)去進修神學時, 屬靈上的得益卻是無限.你會立時理解到原來我們所信的是那麼豐富, 而我們懂得的是那麼少.在真道上, 你會謙卑下來.

 學海真是無涯.以我的年紀, 仍然進行更進深的神學學習.我願意作一個信徒學習的“示範單位”, 與你一起同行!

 這是我的第一個blog.歡迎你在下面留言, 大家作點交流吧! 我非常樂意與你分享學習神學的苦與樂呢(老實說, 當中的樂遠遠多過苦).
\newpage



\section{}
\label{sec:1}
\textbf{「無名無姓」的啟迪}
\newline
\newline
~~~~ 日期: 2021-06-23
\newline
\newline
\hyperref[sec:0]{\small{< < < PREV SERMON < < <}}
~
\hyperref[sec:index]{\small{[返主目錄]}}
~
\hyperref[sec:2]{\small{> > > NEXT SERMON > > >}}
\newline
\newline
$^{1}$賴潔敏 - 「無名無姓」的啟迪

講題:「無名無姓」的啟迪

講員:賴潔敏

場合:ABSCC General

日期:2021年6月23日

作者:賴潔敏 (建道中心助教)

「名字」代表生命.一個人一生總伴隨著一個名字,它代表其家庭背景和身分,也有些人的名字代表著父母的期望與寄托.在華人社會中,「有名有姓」可說是很要緊的事;然而在聖經中,名字也是非常重要的.聖經記載上帝用塵土造人,將生氣吹入其鼻孔.那個有靈的人第一時間就被起名為「亞當」(參創2:7).整部舊約聖經絕是不缺乏名字;而新約聖經就更以耶穌的家譜為始.

五月下旬,在卑詩甘碌市(Kamloops)一間原住民寄宿學校舊址發現超過二百多具兒童骸骨.令人痛心的是那些兒童都沒有被留下名字,更令人失望的是這些原住民學校是由有宗教背景的團體管理.在社會不停迴響的悲憤吶喊中,多倫多懷雅遜大學(Ryerson University) 的懷雅遜塑像因當事人曾倡議建立這些寄宿學校而遭人破壞,其校刊亦將會改名.我們可以理解這種憤怒的情緒,在《詩篇》137篇中,被擄的詩人就曾哀嘆家園慘被摧毀.在經歷迫害之後,詩人祈求耶和華紀念這仇說:「耶路撒冷遭難的日子,以東人說:拆毀!拆毀!直拆到根基!耶和華啊,求你記念這仇!將要被滅的巴比倫城啊,報復你像你待我們的,那人便為有福!拿你的嬰孩摔在磐石上的,那人便為有福!」(詩138:7-9)

雖然以上經文表面是以惡報惡,但當中的意義實際上卻正如希波的奧古斯丁(Augustine of Hippo, 354-430)在其著作 Exposition of Psalm 中所言:「詩人渴望著永恆的耶路撒冷,並且確信上帝最終會譴責敵人.」或許我們面對歷史的不公義時,總難免感到痛心和失望,只是我們仍然相信上帝的應許和審判終必來臨.盼望我們的社會能努力還原那些孩子們的故事,還他們一個公道.同時我們具有基督徒名分的人也要持守基督的愛,讓這些悲痛和不公義的事情不要再歷史重演.

作者簡介:賴潔敏(Kit Man Lai, Kimberly)於香港中文大學崇基學院神學院及多倫多大學完成神學碩士課程,現於多倫多大學攻讀博士學位,並在加拿大建道中心服待.
\newpage



\section{}
\label{sec:2}
\textbf{要活得有智慧}
\newline
\newline
~~~~ 日期: 2021-07-15
\newline
\newline
\hyperref[sec:1]{\small{< < < PREV SERMON < < <}}
~
\hyperref[sec:index]{\small{[返主目錄]}}
~
\hyperref[sec:3]{\small{> > > NEXT SERMON > > >}}
\newline
\newline
$^{1}$陸超明牧師 - 要活得有智慧

講題:要活得有智慧

講員:陸超明牧師

場合:ABSCC General

日期:2021年7月15日

在一次的講道裡,我鼓勵弟兄姊妹不要作一個糊塗的聰明人.我的意思是: 人前被認為聰明的人(正所謂「醒目的人」), 卻在神眼中看為糊塗,是很可惜的人生結局.要有神的祝福,就必須要以祂的價值觀行事.神認為有價值的東西,我們就認同和以信心來實踐.

要活得有智慧,不是以在世上能擁有多少為標準.那種智慧原來是以神的角度來決定.請不要以為什麼都要以神為准,就好像失去了自由.其實,因神是愛護我們的.所以祂的路,才是最穩妥的和最蒙福的.誠心誠意地跟隨祂而行,必定有奇妙的後果.

疫情底下,不知你會否同意一個講法:人的生命的確脆弱, 可以隨時失去 ?」 假若我們不抓著神的心意渡過這生,就有以下的後果:

· 假如我們很早就離世 (die too early),我們必定一事無成

· 假如我們的壽命過長 (live too long),我們就會空走一回

故此,有意義的人生,關鍵在於有沒有神的介入.若我們願意,神必定會加添我們的力量去面對一切.順境的時候,我們體會祂的美好帶領.逆境的時候,神也會有恩典讓我們勝過各種的難處,我們更體會到祂真是信實慈愛的主.信靠祂的人, 的確是有福的 !

作為基督的信徒,我鼓勵你要積極面對人生,好好尋求神,並使用祂所賜予的才幹.自己小小的才幹,在神的手中會成為大器,能成就超乎我們所想所求,是奇妙的恩典 !  請留意:

· “你不理【才】--- 若果你不理會神所賜的才幹

· 【才】不理你” --- 這才幹就與你無關了

最好的消息是: 神樂意使用卑微的人, 你我都有份!

總而言之, 當你願意成為一個以神為中心的人 (a God-centered person), 就是一個有真智慧的人.歡迎你在下面留言, 分享你的體會: 無論是開心的或是覺得艱難的, 都可以成為別人的參考或鼓勵.
\newpage



\section{}
\label{sec:3}
\textbf{「亂世中的小人物:舊約篇」分享回應}
\newline
\newline
~~~~ 日期: 2021-08-03
\newline
\newline
\hyperref[sec:2]{\small{< < < PREV SERMON < < <}}
~
\hyperref[sec:index]{\small{[返主目錄]}}
~
\hyperref[sec:4]{\small{> > > NEXT SERMON > > >}}
\newline
\newline
$^{1}$賴潔敏 - 「亂世中的小人物:舊約篇」分享回應

講題:「亂世中的小人物:舊約篇」分享回應

講員:賴潔敏

場合:ABSCC General

日期:2021年8月3日

賴潔敏分享回應

《約書亞記》一開首記載了摩西死後,上帝吩咐約書亞帶領以色列人進入迦南的事蹟.當中第一件大事就是攻打耶利哥城(參書1至2章).在那個亂世中,攻打耶利哥城的主角身份竟落在一個被傳統社會三重邊緣化的小人物身上——喇合;她是一名女子,一名妓女和一名貧窮的人(書2:1).究竟作者對於這種小人物的刻畫帶來什麼思考呢?

高銘謙牧師博士(下稱高牧師)在「亂世中的小人物:舊約篇」講座中,解構了喇合這個小人物在當時以色列中的角色和帶出相關的教導.高牧師藉著文化人類學家維克多·威特·特納(Victor Witter Turner, 1920-1983)對閾限概念(Liminality)的研究為基礎,從而以閾限角色(LiminalFigure)的模式來重新詮釋喇合在耶利哥之戰中的定位.高牧師在講座中的論述有著以下兩個特點:

其一,神學和人類學的跨學科討論有助我們解決閱讀經文時所產生的期望落差.作為信徒,我們往往期望聖經的故事都應該是偉大的故事,故事的主角也應有著尊貴的身分.只是出人意料地,一位為以色列人進入迦南起了關鍵作用的人物卻正正是沒有身分,地位和金錢的妓女.閾限角色概念的引入讓信徒以另一種視點來閱讀這段經文,它能啟發信徒思考聖經作者刻意安排小人物成為拯救英雄的深層意義.

其次,閾限角色的概念能豐富信徒理解聖經其他敘述的能力.我們知道喇合在身份和空間上都被邊緣化,而這種身份卻使她成為配合上帝旨意的特殊工具.縱觀聖經,當中也包含了許多閾限人物的故事.這些故事都與以色列人的轉變(Transitions)有著密不可分的關係.閾限角色往往都成為上帝與人說話的工具,當中就如:夏甲,路得,瑪利亞等;這些小人物最終都在參與歷史中上帝對我們的拯救計劃.

是次講座中,高牧師透過閾限角色為切入點,重新詮釋喇合的角色.這些教導對於今天信徒如何看待社會上被閾限的個人或群體,都能帶來不少深刻的反思.

顏佩珊分享回應

Kimmie 為高牧師的信息作了一個十分好的陳述:透過喇合這個不起眼的小角色,道出上帝如何成就祂的救贖計劃;由此引伸到今天活在大時代中,閾限角色的人,也可以同樣地被上帝所用.

筆者認為有以下幾點可以討論:

1.高牧師提出「閾限角色」的名詞似乎不是教導人去反思社會或個人如何去看待這些不被社會認同的人,反而是鼓勵那些自覺是被邊緣化的人,在基督裡可以有正面的身份價值的認同.

2.經文及高牧師似乎都沒有提到喇合的經濟狀況,她的「被邊緣化」主要是因為她的工作性質,以致她的身份,角色是糢糊或不確定性—既不完全是城牆內的耶利哥人,亦不是城牆外的人.她的名字與角色同樣是吊軌地讓人聯想到被社會排斥的人,卻被納入成耶穌家譜的一員.

3.從新約聖經中的引用,看到聖經作者對這個舊約小人物的評價,不是道德上的批判,反而是她與以色列人核心信仰的宣認.猶如在耶穌基督的家譜中其他四個閾限女性一樣,被逆向地正當化她們的角色,在上帝的國度裡擁有真正的身份認同.喇合的信仰宣認亦超越了地上民族身份的認同,「成為天國子民」才是她真正的身份價值及歸宿.對於今天難以拿捏自己身份角色的香港人,高牧師帶出「天國擁抱一切被邊緣化的人」的訊息,鼓勵人要以不一樣的眼光,對舊有有所捨割.
\newpage



\section{}
\label{sec:4}
\textbf{「亂世中的小人物:新約篇」分享回應}
\newline
\newline
~~~~ 日期: 2021-08-05
\newline
\newline
\hyperref[sec:3]{\small{< < < PREV SERMON < < <}}
~
\hyperref[sec:index]{\small{[返主目錄]}}
~
\hyperref[sec:5]{\small{> > > NEXT SERMON > > >}}
\newline
\newline
$^{1}$梁兆明 - 「亂世中的小人物:新約篇」分享回應

講題:「亂世中的小人物:新約篇」分享回應

講員:梁兆明

場合:ABSCC General

日期:2021年8月5日

甘心追隨上主在終末閾限中作個無名的基督小僕

作者:梁兆明 (Charles)

在講座「亂世中的小人物:新約篇」中,張雲開老師嘗試用人類學觀念來剖析司提反作為一個在新閾限上,[1] 站於猶太信徒與外邦信徒之間的代表人物,其所作之事,宣講的話,見證主的道,導致下一階段事情的發生.在關鍵性的新閾限階段中,他作真理與虛謊的爭戰,見證耶穌為真理的主;且也作生命與死亡之爭,其殉道有著極深遠影響:成就後來外邦人能夠順利進入神的國度.

誠然,我們都是在這終末世代閾限上的信徒,於與罪分離及等待主再來中,還沒有完全得著神所賜的永恆福樂,卻又同時領受了其應許,這個實現與應許充滿張力;但並不表示我們甚麼都不做,安舒等候,反之,從司提反身上,啟發我們在這世上應該作甚麼:永遠為主作見證,與人分享福音信息,宣辯基督真理.這就是上主給信徒的大使命.司提反的場景與我們不同,每個人都有各自周遭際遇,需要反思怎樣過一個仿主生活,為神作見證.

於此,筆者想起舊約以西結先知在新閾限上的見證.以西結由耶路撒冷被擄到巴比倫,直至離世那日還在加巴魯河邊,沒有機會看見以色列民回歸,所以他最後仍然在漆黑當中作先知.但是以西結一生就是上主於巴比倫被擄,絕望,漆黑處境裡,所燃起一線小小的亮光.誠然,在這終末亂世中,上主同樣可以如此用我們,如在工作上,教會裡,家庭中,可能處處都是黑暗,上主呼召所作的或許是芝麻小事,甚至旁人見不到果效,但上主若放你我於此場景,可否謙卑跪下,問一下神想我們在這裡做甚麼,然後讓祂使用,在此為其作光,作見證.如此堅守的信,不是過一天信心生活便可以,這是意識一種生活改變,且伴隨很多挑戰,甚至甘心把自己全然擺上,成為活祭,不論好境或逆境的,有名或無名的,作大事或小事的都隨主而行.這是神對我們發出的邀請,你願意接受嗎?

[1]  人類學的「閾限」觀念,意謂:人如何由一個階段轉變另一階段,由一個門檻跨越另一門檻.

 「亂世中的小人物:新約篇」 回應 梁兆明的分享

作者:潘澤明 (Richard)

很認同梁兆明先生的分享,我們都是在這末世閾限上的信徒,即所謂已然,未然的過渡時間.主耶穌在離世升天之前,向門徒頒布大使命,要將福音傳遍地極,這大使命不單是向當時聽見的門徒發出的,也是對現今的基督徒有效的使命.然而,我們都是在我們生活圈子內的小人物,憑自己微小的力量,怎樣將福音傳遍地極?神並沒有呼召所有基督徒都成為全職的宣教士,可是我們就是自己家庭,工作和社交領域內的一位小宣教士,以自己的生命作見證,把握每個機會與所接觸的人分享福音.

正如梁先生所說,當我們決心追隨基督,便必然產生外顯的生命和生活的改變;而這改變也可能帶來很多挑戰,甚至遭受排斥或逼迫,有時候要放棄一些眼前的利益和固有的習慣.到這裡筆者想起一首詩歌「這世界非我家」,第一節是這樣:「這世界非我家,我停留如客旅,我積財寶在天,時刻仰望我主,天門為我大開,天使呼召迎迓,故我不再貪愛這世界為我家.」願意我們都認清這個現實的存在,甘心擺上,為神所用,無論在任何境況都尊主為大,我們就會經歷神所賜的豐盛生命.
\newpage



\section{}
\label{sec:5}
\textbf{「亂世中的小人物」 <對談篇> 的回應分享}
\newline
\newline
~~~~ 日期: 2021-08-08
\newline
\newline
\hyperref[sec:4]{\small{< < < PREV SERMON < < <}}
~
\hyperref[sec:index]{\small{[返主目錄]}}
~
\hyperref[sec:GA78znQ7bg4]{\small{> > > NEXT SERMON > > >}}
\newline
\newline
$^{1}$陸超明牧師 - 「亂世中的小人物」-- <對談篇> 的回應分享

講題:「亂世中的小人物」-- <對談篇> 的回應分享

講員:陸超明牧師

場合:ABSCC General

日期:2021年8月8日

加拿大建道中心這三次講座,環繞的大主題是“亂世中的小人物”.我會在下面分享第三次聚會帶給我(們)的一點反省.期待大家能對今次的講座主題有進一步的體會.

(1) 身處“閾限”的門檻,我們可以突破!

講員們提到喇合和司提反兩個例子,表明了縱使我們在人眼中是小人物,仍可在信仰裡被提升.閾限的觀念,重點在於要體認在上帝掌管的世代裡,如果我們肯花功夫去理解祂,我們便可以在那臨界領域,離開舊有世界,等候屬天帶領,最後便與神接軌到新的境界.

喇合之例,始於她肯認識神,最後達至全家得平安; 路得之例,信心達至承認 “你的神就是我的國”; 約書亞之例,從眼前敵我看法改為降服上帝的取態; 司提反的努力,居然有份促進偉大使徒保羅的誕生.這實在是極大的鼓勵,因祂是恩典的神!

(2) 疫情下誘發末世的反思

在接近兩年的疫情下,難免有人心情低落,甚至對上帝發出埋怨: “為何我們要經歷這麼困難的歲月?” 這種看法可以理解,但撫心自問,其實只是從個人的出發點看.我覺得不要忽略以神的角度看這次疫情,就會考慮到這是末世的另一次提醒:神要喚醒靈裡已經沉睡的我們.

基督的跟隨者不要逃避應有的態度,就是調整我們信仰的盼望,乃在未來基督再臨的那一天.這也是在閾限裡的我們,可以因此疫情而被喚醒,超越自身的屬靈認識.深信這是COVID-19 能帶給我們最積極的東西.

(3) 新常態下的探索

我非常同意講員們認為疫情下使到教會生態轉到網上的運作有其優點,相信到了不能回到以前的狀態了.我更同意線上的運作,不能代替信徒之間的接觸,因為這是上帝創造我們的心意.這是生命得以交流的必須,否則我們必定受到虧損.

我個人切望弟兄姊妹們在教會不久開放實體敬拜時,一定要珍惜相聚的機會.也許是次疫情最深刻的課題,就是信徒們面對面相交的重要.兩年來這方面的失去,你能無動於衷嗎?

回應分享 : 「疫情之後,雨過天晴 ?」  黃耀華 (Ronnie)

我們由「未信之民」,成為「得救(恩)之民」,是一個改變生命的突破,但這只是開始,在人生路上,仍有許多閾限有待突破,神給我們的恩賜,不止於此.通過喇合及司提反等先賢例証,我相信憑著對神的認信和委身,靈命更新成長,將可與神接軌,同行新境界.

在神掌管的世代中,人仍要面對種種社會張力及經濟壓力,過去年半的疫情,令許多人情緒出現問題.神從沒有應許我們不再有苦難,我們需要以信仰,與主同在的生命,有信心和力量地面對現實生活的挑戰.

主耶穌喚醒沉睡的門徒,教導他們要謹慎,警醒及祈求,不要在憂慮中陷於網羅.這個是屬靈爭戰,我們要以屬靈視野應戰.正如陸牧師分享所言,在「疫境」中,神要喚醒靈裡已沉睡的我們.我們雖不能改變疫情,但可以信心和盼望「疫來順受」.

網上敬拜和聚會不能代替實體接觸,大家在教會裡敬拜,歌頌讚美,交流互動,何其美好 ? 至於是否「雨過天晴」,似乎言之尚早,因為仍有許多變種因素,但無論晴天雨天,每天都有神的護理,叫愛衪的人得到益處.在晴天之下,我們要感恩,在人生風雨中,更需要仰望神,讓「信望愛」,成為看不見但更有效的「逆苗」!
\newpage



\section{腓立比書}
\label{sec:GA78znQ7bg4}
\textbf{ABSCC 網上免費講座:「盡頭見耶穌:腓立比書的盼望秘笈」}
\newline
\newline
連結: \href{https://youtube.com/watch?v=GA78znQ7bg4}{\texttt{https://youtube.com/watch?v=GA78znQ7bg4}} ~~~~ 語音日期: 2022-01-13
\newline
\newline
\hyperref[sec:5]{\small{< < < PREV SERMON < < <}}
~
\hyperref[sec:index]{\small{[返主目錄]}}
~
\hyperref[sec:L8_DVqUvOSM]{\small{> > > NEXT SERMON > > >}}
\newline
\newline
$^{1}$各位世界各地的弟兄姊妹,你們好!.
今晚是加拿大建造中心舉行的一個免費的網上講座.
我是建造中心的總負責人,陸昭明牧師.
我們又回到疫情嚴峻的情況下.
不知是否因為這個原因,又多了人來參加今晚的聚會.
非常歡迎大家.
今晚的講座時間是比較準確的.
8點開始至9點15分.
我們按電腦時間辦事.
今晚的講座請來香港建造神學院的新約教授葉應林博士主講.
題目在螢光幕上,大家也看到.
題目是「盡頭見耶穌」,肥立比書的「盼望被給」.
我們是否已經去到盡頭呢?.
在現在疫情的狀態下,似乎大家都很洩氣.
大家都覺得這個世界是否已經到了末日的狀態呢?.
但不要緊,我們有主,有耶穌可以幫我們解決很多問題.
在未開始講座之前.
請工作人員到第二張slide.
請Charles幫忙到第二張slide.
第二張slide會講到我們加拿大建構中心的聯絡方式.
請工作人員到第二張slide.
Charles或Benson.
我們有一個網站,www.abscc.org.
網站詳細講解我們加拿大建構中心的工作.
我們基本上是一個神學教育機構.
我們總部在多倫多這個地方.
疫情下,我們的抗戰業務已經越來越廣泛.
不單止在多倫多,在加拿大很多地區,很多省份.
我們都有學生來修讀我們的課程.
歡迎大家到我們的網站去看我們的內容.
我不想花太多時間講解.
網站會告訴大家.
我們在Facebook有一個Fanpage.
如果在Facebook打Alliance Bible Seminary, Centre of Canada.
又會去到我們那裡,可以看到.
今天我們很高興請到葉博士來.
他會有些slide放給大家看.
希望大家都喜歡.
葉博士,現在可以多講一點.
葉博士是新晉的建築神學院新約老師.

$^{41}$非常年輕,我們很榮幸請到他來.
因為他的工作非常繁忙,新約的老師越來越少.
我不再繼續講下去.
我們將以下的時間交給葉博士.
他的英文名是Scott,Scott Yip.
我講的應該到這裡為止.
不如請Scott打開你的video.
開始今晚的講座.
我們恭請葉博士.
好,謝謝盧牧師.
請問大家聽到嗎?.
聽到.
聽到,好.
多謝加拿大建築中心的邀請.
各位弟兄姊妹,向大家問安.
我也不肯定應該叫晚安,早晨,午安還是怎樣.
因為我們都是在不同的時區.
剛才盧牧師介紹我,十分非常年輕.
聽到的時候我也有少許懷疑.
我是否真的非常年輕.
在盧牧師眼前,我也有少許年輕.
多謝各位參加今晚的講座.
今晚的題目是盡頭見耶穌.
菲勒比書的盼望秘訣.
我自己的PhD博士論文的研究範圍就是菲勒比書.
加上我是用敘事神學的研究人的經驗的角度.
去理解生命轉變的過程.
所以盼望作為一個人類的經驗.
其實是一個我很有興趣的題目.
在今日的投影片中.
我會圍繞這個主題和大家探討.
除了平時大家可以繼續上加拿大建立中心的Fanpage之外.
如果你經過了今晚之後.
想了解我的解經或神學更多.
你可以在Facebook,IG或上網站.
Narrating God 去了解更多.
之後希望可以和大家保持聯絡.
在講座的題目中.
首先我們要去理解的就是盡頭的定義.
一般來說盡頭包括了終點,末端,止境,邊際等意思.

$^{81}$不同的人對於盡頭這個字有不同的感受和聯想.
如果作為一個中性的字眼.
例如多倫多的西邊或東邊的盡頭.
其實只是指地理環境的邊際.
這個盡頭就很中性.
沒有什麼特別的不好或好.
但是如果盡頭是壞事的盡頭.
那就好了.
例如疫情已經兩年多了.
什麼時候疫情會到盡頭呢?.
這種盡頭就是帶來了美好的新開始.
但是有些盡頭似乎又不是太好.
如果一件好事去到盡頭.
那就帶來了威脅.
就會令人失去盼望.
換句話說盡頭和盼望其實是一個掛勾的概念.
例如你本來有職業的.
你失業了.
如果你本來有一份戀愛的關係.
失戀了.
例如你本來有一段婚姻的關係.
你失婚了.
甚至你本來有條命的.
你失去了你的性命.
這些都是一些好事.
似乎去了盡頭.
帶來了人生的一種威脅.
令我們似乎失去了盼望.
就以死亡為例.
死亡作為一個終極的盡頭.
就是一個你和我都不想面對.
特別在座的我觀察到很多是黑色頭髮的.
或者有些是沒有染過的.
我們應該看上去都是華人為主.
這裡有誰是華人呢?.
如果舉手看看是否聽到我說話.
我們華人有很多禁忌.
我們不喜歡說這些.
特別新年快到了.
還有大概兩個星期.

$^{121}$就到農曆新年.
我們華人的新年要說好彩頭.
和猶太人的新年會去到紀念他們出埃及.
整個是很不同的一種新年紀念.
我們很不喜歡說不好彩頭的事情.
死亡是一個我們往往不想面對.
但卻一定要面對的一個盡頭.
其中一位20世紀很有名.
被稱為其中一個最有影響力的哲學家.
Martin Heidegger.
他提到其實死亡我們不需要去逃避.
相反你去到正視死亡.
去到擁抱死亡.
就是我們一切可能性的一個最終結.
這個就是真人性.
似乎如果這樣說.
其實死亡不是一個什麼問題.
死亡如果不是一個問題.
也不需要有什麼.
似乎要去逃避.
但是死亡是否真的這麼簡單呢?.
死亡的意義其實不同人.
是可以給予很不同的意義.
海德格爾所提出的要正視面對死亡是對的.
但是如何去面對死亡.
卻是我們每一個人要去深思的問題.
在近十幾二十年.
其實你發現全球很多人都很關注到.
人生以至世界盡頭的問題.
在頭影片中.
你會看到有兩部電影海報.
你會發現越來越多的電影.
以一些叫做末世之後.
我們說叫做post-apocalyptic.
就是天體之後或末世之後.
講述很多地球出現了一個末世大災難之後.
人生是如何的呢?.
世界的氣候.
越來越多似乎的問題.
所以近十幾二十年.

$^{161}$在歐洲,北美.
我也留意到越來越多人留意環保.
為什麼呢?.
因為世界的盡頭.
似乎越逼越近.
近來也有另一部電影很熱.
看看這裡有沒有同學.
不是同學.
有沒有弟兄姊妹.
你有看過這部電影嗎?.
叫做Don't look up.
有的,舉手好嗎?.
因為你知道.
用Zoom去講講座.
講座的人也有時很低落.
因為看不到你的樣子.
看著一塊塊黑面.
是黑色畫面的黑面.
完全像自說自話.
像一個傻子.
很depressed.
所以你如果可以給我看你的樣子.
或者揮揮手.
你還未刷牙.
你晚上來的.
如果在加拿大.
應該不會還未刷牙.
現在是香港.
我在香港時間是早上9時2分.
你給我看你的樣子.
笑一下.
給我反應.
我開心很多.
好了,回到這裡.
這部電影也是講到世界的盡頭.
原來很多地球人不願意去到正事.
這部電影裡有一個隕石正衝向地球.
但是在當時的政權領袖和科技巨人的領袖影響下.
很多地球人都不願意去正視這個盡頭.
It's fine.

$^{201}$在這部電影裡.
不斷講說.
沒問題.
世界會更美好.
世界沒問題.
越來越好.
似乎這些領袖提供了另一些盼望.
什麼盼望呢?.
當很多地球人以為派太空船上去能夠擊碎隕石.
但是有另一些人提供了另一個更吸引的盼望.
就是不要完全射碎它們.
射碎一點點.
但是可以拿到隕石裡的化學物質.
一些礦物.
為了什麼呢?.
如果你有看過這部電影.
背後的盼望是什麼呢?.
另類的計劃就是賺錢.
賺到更多的錢.
錢在很多人類的計劃裡.
其實都是最核心的動機.
就如最近有很多元宇宙的發展一樣.
我們看到原來盼望和其他的.
除了盡頭有關係.
原來不同人會提供不同的盼望版本.
換句話說.
你相信不同的故事.
就會分別領受到不同的盼望.
所以今天和大家看腓立比書的時候.
你也要去檢視一下.
究竟你現在相信的是什麼故事呢?.
你相信的是什麼版本的盼望呢?.
在腓立比書這裡.
我們看到保羅.
其實可以用盼望盡頭錢的盼望管理.
用這個角度去理解腓立比書.
腓立比書的保羅.
我們常說他充滿什麼呢?.
腓立比書最多出現的其中一個字是什麼呢?.
就是喜樂.

$^{241}$很開心的.
很喜樂的.
Rejoice.
你們要喜樂.
但很多時候我們留意保羅喜樂之餘.
我們就忘記了.
其實寫腓立比書的時候的保羅.
其實他的生命很有可能已經到了盡頭.
當時的他很有可能被監禁在羅馬.
他會不會出監呢?.
他是不肯定的.
他很盼望可以出監.
再去菲律賓堅固信徒.
但他是否肯定呢?.
他不肯定.
他的前路其實不在他的掌握之中.
個人生命去到一個彷彿的盡頭.
不單止個人生命去到盡頭.
其實他的事公的延續.
也就是他的legacy.
我們英文說legacy.
他的那種對耶穌基督的理解.
他的神學.
他的那種福音的事公的傳承.
都面對很大的危機.
如果你讀過腓立比書.
你會記得原來在腓立比書的第一章.
那裡說到有些人甚至要藉著傳福音去加害他.
這真是一個很特別的現象.
竟然有人可以用傳福音來攻擊保羅.
在當時的馬其頓省.
除了菲律賓教會之外.
也沒有任何其他教會去支持保羅.
換句話說.
保羅為什麼在腓立比書.
他那麼恩切.
他就說他所盼望的.
他無論是生是死.
總叫基督在他生命照常顯大.
是不是?.

$^{281}$大家這些經文應該很熟悉.
是不是?.
為什麼?.
因為保羅的處境.
他為了福音要坐牢.
甚至去到被人斬頭的可能性.
正在削弱他自己在教會裡的名聲.
當時有不少其他領袖.
不少其他猶太人信了耶穌的領袖.
都紛紛質疑.
保羅你這樣傳福音.
你這樣去宣教.
其實你是不是傻了?.
保羅的傳承.
保羅的福音事供.
在那一刻似乎在馬其頓省.
只剩下一間教會.
會繼續紀念他.
而這間教會似乎最近.
都好像有一段短時間沒有再支持他.
直到以巴弗提.
以巴弗提在第二章去找保羅.
代表了菲律賓教會對保羅的繼續支持.
腓立比書在這種事供延續.
是不是已經到了盡頭?.
保羅的傳承是不是已經到了盡頭?.
保羅有沒有白做?.
保羅會不會做了這麼多.
最後什麼渣都沒有?.
保羅不僅個人生命到了盡頭.
他個人事供的影響力.
也似乎到了盡頭的危機.
就在這兩個背景下.
保羅要去處理信徒.
如何面對眼前的苦難.
眼前的逼迫.
因為菲律賓的信徒也跟他一樣.
正在面對各式各樣的困難.
保羅也在最後.
曾經有處理過世界中局的盡頭問題.

$^{321}$我們今天短短的一小時時間.
看看盼望的盡頭.
究竟在腓立比書有什麼內容.
說到一個盼望的秘笈.
或者秘訣.
我會提出很簡潔的四點.
這四點我希望能夠讓你.
完成今天的講座後.
都能夠借鏡成為你面向眼前.
不可知的未來的一個秘訣.
第一.
我們要熟讀聖經.
所以舊約的故事.
如何成為保羅在新約的一些資源.
原來你和我都一樣.
要熟讀新舊約聖經彼此的共鳴.
成為我們的基礎.
這種共鳴原來成為了保羅.
在他當時的處境.
去辨別到上主足跡的方法.
所以原來新約聖經去引用舊約.
這件事大家應該都聽過.
不會陌生.
但這不是一個文學技巧這麼簡單.
其實是一個鑒古之金.
辨識上主在歷史中足跡的手法.
正正在辨識之後.
鑒古之金才能夠將自我的故事代入.
然後才能改寫你人生的視野.
也就是重塑你應該探望的角度.
這種結合聖經和經驗.
也就是將詮釋學和現象學結合的一種解經.
就是我自己博士論文的方向.
這些可以說100至200小時都可以.
今晚說1小時.
OK.
在這個所謂舊約的經文共鳴中.
其實在腓立比書我們起碼找到.
有ABCDEFG七處.
我們說舊約明顯的暗人.

$^{361}$時間關係.
今晚只集中看一節.
就是腓立比書三章三節的這一段.
看看他如何引用烈王寄上和耶利米蘇.
好讓我們去體會一下.
原來這種舊約的共鳴.
是時代的平排.
將現在的故事和昔日的故事放在一起.
如何能夠為保羅提供一種動力.
今天所說的不單是一些教義.
不單是跟你說一些知識.
最重要當處理到盼望問題的時候.
究竟讀聖經能否為你帶來動力.
有一種力量.
有一種讓你辨別到.
和有一種心追求聖經所說的盼望.
我們要處理的是這一點.
在這裡我們代入一下當時保羅的處境.
就是一個走近盡頭關在牢中的保羅.
他如何回想昔日和耶利米蘇的故事.
在這裡可以補充一句.
如果我沒有記錯.
路穆斯提醒了我.
我今天所說的powerpoint.
頭影片是否可以分享給大家.
所以你一邊聽的時候放心.
除了可以一邊抄之外.
是可以分享給大家.
所以大家不用擔心.
這裡提到菲勒比書的三章一到三節.
你聽我讀出好嗎.
我為了不想在Zoom上大和唱.
feedback到無痕無倫.
還是我讀好一點.
否則大合唱一起讀就很慘了.
聖經這裡提到.
沒有了我的弟兄們.
你們要靠主喜樂.
我把這些話再寫給你們.
對我並不困難.

$^{401}$對你們卻是妥當的.
應當防備犬女.
防備作惡的.
防備妄自行各的.
因為真受各禮的.
就是我們這席著上帝的靈敬拜.
以基督耶穌為誇耀.
不依靠肉體的.
在這裡我們集中去看三段話.
第一.
什麼叫妄自行各的.
第二.
什麼叫真受各禮的.
第三.
什麼叫以基督耶穌為誇耀.
在這裡我們看到保羅在做什麼呢.
就是他自己在盡頭面前.
他在回想以往河西亞的時代.
特別是耶利米的時代.
原來昔日他們也有面對盡頭.
換句話說.
當你面對盡頭的時候.
你可以透過別人的故事.
他們也在面對盡頭.
去借鏡取智慧.
讓你知道你應該如何面對盡頭.
明白了嗎.
這種所謂盡頭的互相借鏡.
就是背後的方法.
這裡說到原來妄自行各的.
你看到這個紅色字.
原文叫做Katatome.
妄自行各的Katatome.
這個字很特別.
原來我們看回猶太人的希臘文舊約版本.
也就是七十四的版本.
這個字當我們想查究一下.
什麼叫妄自行各.
保羅你用這個字的時候.
其實是什麼意思.

$^{441}$很多時候我們要翻查七十四的版本.
看看猶太人在以往的文中.
他們有什麼慣性.
我們查一查.
原來這個字從來沒有出現過.
不僅如此.
在斐羅和約瑟夫.
也就是跟保羅差不多同期的猶太人著作中.
也找不到妄自行各Katatome這個字.
但是還好.
他的同源動詞Katatemlo.
就在七十四的版本中我們看到.
分別在《尼米基》,《聯王記》上.
以暢亞書,何錫亞書都看到.
究竟這個Katatemlo.
妄自行各的同源動詞.
有什麼意思呢?.
如果我們看一看英文的譯本.
我們看到這個妄自行各.
英文可以翻譯為.
False Circumcision.
或者是.
Those who mutilate the flesh.
我們試試看.
看一看其中一個舊約的記載.
就是《聯王記》上.
這個《聯王記》上.
我們看到原來當時.
Katatemlo妄自行各的這個字.
原來當時是記載一群巴力先知.
和以利亞正面比拼.
七十四的譯本就說到.
原來Katatemlo這個字.
是用來形容那群巴力先知.
拿他們的刀槍去刺割自己.
在經文這裡看到.
他們大聲求告.
按著他們的儀式.
刀槍刺割自己.
直到渾身流血.

$^{481}$他們狂呼亂叫.
直到獻晚祭的時候.
卻沒有聲音.
沒有回應.
沒有理睬.
原來這群巴力先知.
他們嘗試用這些刀槍.
去割傷自己.
以致能夠控制巴力.
在當時.
原來以色列人也漸漸用了這些方式.
去確保自己能夠控制上帝.
這種參雜了愛邦異教的祭祀儀式.
就是直著割傷自己的身體.
或者似乎能夠感動.
控制天上的神子去賜福.
例如在荷西亞書中.
我們看到那群以色列人.
不誠心哀求上帝.
他們為求五穀伸走聚集.
原文提到聚集.
其實另一個意思是割傷自己.
換句話說.
當保羅在這裡形容他的敵人是妄自行割.
然後又提到自己是真受割禮的時候.
很特別.
他不像在羅馬書.
加拉太書說割禮是不需要的.
如果割禮是不需要.
那就很簡單.
但這裡一方面說你們亂割自己.
但另一方面又用割禮這個傳統.
去稱呼他和菲拉比信徒.
是不是很有趣?.
如果什麼都可以按照羅馬書去解的話.
那就很簡單.
但在這裡保羅用貶義詞.
Catatumé 你們亂割自己.
然後用Peritomé 我們是真正正確地割自己的意思.
去包養自己.

$^{521}$就是正正反映原來保羅和他的敵人在爭論.
究竟割禮這個傳統應該如何去形容.
原來菲拉比書的關注.
不是教人如何脫離罪惡和死亡而得救.
菲拉比書的信息是說盡頭苦難死亡或將來臨之際.
我們信徒如何能夠按上帝的心意去為自己為信徒.
指出一個正確的盼望方式.
對於保羅來說.
他才是上主地上合法的代言人.
他才能夠代表上主.
不是叫人接受割禮的猶太基督徒領袖.
他才是 為什麼?.
因為他願意正視盡頭.
他願意正視基督徒必須要走過苦難.
他沒有逃避盡頭.
或者逃避迎見盡頭的受苦過程.
原來 講一點歷史背景.
在第一世紀的羅馬帝國.
人都慣常參與敬拜羅馬皇帝.
但當時 如果作為猶太教的分支之一.
只要你接受了割禮.
你就能夠成為一個被看為猶太教的一部分.
猶太教有一個特權.
就是可以領受到其他市民的諒解.
就可以豁免參與帝王崇拜.
所以 當時在歐洲 菲律賓的地方.
有一群猶太人去到菲律賓.
他們知道菲律賓的信徒正面對張力.
我們信了耶穌 但我們又不是猶太人.
我們又要繼續信耶穌 又拜羅馬皇帝嗎?.
怎麼辦呢?.
有一群人就說 其實有另一個盼望.
有另一個可能性.
就是你不用面對受苦.
不用面對張力.
你割一割 你就可以不去受苦.
基督徒是不用受苦的.
其實你不知道嗎?.
我們自從阿巴拿開始.
上帝已經藉著國禮與我們立約.

$^{561}$所有人都要守國禮才可以參與逾越節.
紀念上主的拯救.
你不記得嗎?.
在耶利哥的致疫之前.
我們所有人都是集體受國禮.
然後才打聖仗.
所以在這裡我們看到.
就如近代歷史.
在主前167年.
也就是說對於初期教會的信徒來說.
百多年前.
當時你不知道嗎?.
有些猶太人放棄了國禮.
去拜宙斯.
所以就成為了上帝很討厭的一群人.
所以堅守國禮.
是一個我們民族.
我們猶太教大是大非的層次.
你堅守國禮.
就可以繼續信靠上主保護你.
上主會賜你盼望.
所以我們現在一起鼓勵大家都接受國禮.
就會得到社會的認同.
這就是上主賜你盼望的方式.
在這裡我們隱約看到.
兩個以色列人.
有關國禮.
或者叫做國.
的一些彼此相憎的傳統.
一方面.
國好像代表上帝的保護.
但是就在前面說到.
原來有一些的國.
是不合神心的國.
是一些不誠心求告的國.
那種的國.
不是上帝所期望的國.
究竟在這個菲律賓的處境.
我們應不應該.
藉著國禮去經歷安全.

$^{601}$經歷到一種盼望呢?.
如果藉著國禮.
就可以成為猶太教的一部分.
那就可以不受苦了.
我們就可以得享安全了.
對於保羅來說.
藉著國禮.
去到這樣逃避受苦.
原來就和以前以色列人.
妄自行各去操控上帝一樣.
因為這樣忽視了上帝.
已經藉著耶穌基督.
受苦人生路的啟示.
耶穌基督來到這個世界.
不單止成為了你和我的救主.
為得救入天國.
給了一張入場券你和我.
菲勒比斯的耶穌.
不單止是一個救主.
他更加是一個受苦人生的典範.
是一個不逃避面對人生盡頭的典範.
所以保羅就是在耶穌的人生裡.
找到他也要去到正是生命的盡頭.
正是有苦難的必經人生過程.
以這種面對死亡的態度.
以耶穌基督不逃避.
為上帝受苦的態度去面對死亡.
所以菲勒比斯其實是見證保羅.
和一些其他教徒領袖.
在爭論應不應該受苦.
應不應該守國禮.
其實在爭論這件事.
根據猶太其他領袖的故事.
用國禮去獲保障.
It is fine, it is fine, no problem, ok?.
就和昔日以色列人守國禮得王保守一樣.
繼續舞照跳,馬照跑.
但是根據保羅的敘事.
你用一些人的方法去逃避受苦.
逃避正是的所謂盡頭.

$^{641}$只是會加入了引領信徒.
操控上帝的行列.
是偏離了上帝在末世開始了.
藉著耶穌基督以受苦承擔別人生命.
而帶來受苦的典範.
所以我們循著這種角度.
再去探討一下究竟以基督耶穌為誇耀是什麼意思.
在這個菲勒比書的三章三節.
以基督耶穌為誇耀.
我們胡合本以前譯作以基督耶穌誇口.
誇口也好誇耀也好.
傳統來說我們就將這些誇口誇耀.
一般理解為我們要靠耶穌的恩典得救.
不要靠行為得救.
我們都很根深蒂固.
覺得猶太人,法利賽人就是靠行為.
我們就是靠恩典.
這也有它的道理.
但是這樣的解讀有一個危機.
就是將特別羅馬書和加拉太書的處境.
不理三七二一就讀入了菲勒比書.
忽視了保羅在菲勒比書的一些獨特關懷.
在這裡正正就是當我們去到理解以基督耶穌為誇耀的時候.
我們不能夠忽視的就是.
原來保羅在這裡就是在暗引耶利米書的經文.
在這裡我們看到耶利米書九章聖經怎樣說呢?.
它說:耀華這樣說.
智慧人不可誇耀自己的智慧.
勇士不可誇耀自己的勇力.
財主不可誇耀自己的財富.
誇口的卻要因了解我.
認識我而誇口.
認識我是耀華.
要在地上施行慈愛公正公義.
因為我喜悅這些事.
這是耀華的宣告.
請問智慧你知不知道去到一個時代.
或者一個政權或者一個國家的結束的時候.
通常有一個什麼現象呢?.
其中一個現象就是那些銀紙會全部變成廢紙.

$^{681}$我沒有經歷過.
但我記得我小時候.
我婆婆都會拿一些軍票去看.
知不知道什麼是軍票?.
有沒有人知道?.
軍票差不多是美國100年前.
以前的時代.
那些政權或者是日本人也好.
中國人也好.
國民黨也好.
什麼黨也好.
他們那些票據.
但是一個時代的終結.
那些你以前可以誇口的.
很恐怖的.
本來是可以給你盼望的.
我們一打開那本紅包.
很多個零.
很可以有盼望.
突然間盡頭來臨.
全部變廢紙.
很可怕.
例如我們香港.
都會說每個銀行戶口會保你50萬.
我不知道加拿大多少.
是不是不同銀行有什麼可保障保你50萬.
原來保你50萬.
最終那50萬都是給廢紙.
好像大富翁那些.
你拿去吧.
試試看園藥店有沒有給你麵包.
哇.
不可以誇耀.
你本來可以誇耀的.
智慧.
廢紙.
大力.
廢紙.
財富.
廢紙.

$^{721}$很恐怖.
為什麼呢?.
為什麼耶利米.
要說到這些現象呢?.
我們知道耶利米蘇是一個什麼時代?.
耶利米蘇是一個.
以色列人.
他們的.
國權.
他們的國家.
去到一個盡頭的時代.
換句話說.
當保羅.
他自己面對一個今天不知道明天的處境.
他為什麼要一一的引用各個舊約聖經呢?.
原來他引用約伯也好.
引用以塞也好.
引用出阿及記.
引用利美記.
引用耶利米蘇.
正正是因為.
他在這些歷史裡.
他看到.
上帝的僕人.
昔日如何面對一個有歷史的重要時間.
他們如何去到合適地.
正確地回應上帝.
就在這些盡頭面前.
在這些危機面前.
他們找到上帝.
在這裡.
我們再看看.
在這裡我們看到.
再看下去.
很特別.
前面23,24節.
我們通常會背的金句.
但是背金句有一個危機.
背金句的危機是什麼呢?.
背金句有什麼危機呢?.

$^{761}$背金句的危機就是不看上下文.
有理沒理.
合適就好了.
拿出來.
就立刻墮入自己的處境.
但是原來.
耶利米蘇的處境.
是說到一個政權很混亂的處境.
第25節.
我們再看下去.
看眼而日子將到.
這時耶華宣告.
我必懲罰所有旨在身體受過割禮.
內心卻未受割禮的人.
就是什麼呢?.
埃及.
猶大.
以東.
阿滿.
摩亞.
和所有剃除殯法.
住在曠野的人.
因為列國的人.
都沒有受過割禮.
甚至以色列全家的內心.
也沒有受過割禮.
阿迪姐妹.
你留意到很有趣嗎?.
這裡再一次出現了割禮的概念.
換句話說.
當保羅引用以基督耶穌為誇耀的時候.
這裡又是誇耀自己的智慧.
誇耀認識耶和華.
原來誇耀不僅是重複出現的主題.
割禮也是.
真受割禮.
未受割禮.
真的割禮.
假的割禮.
重複的在菲勒比書第三章和耶利米書第九章.

$^{801}$重複一起出現.
很有趣.
我們一般學者也是這樣.
去靈為都看到這個暗印.
但是究竟暗印來做什麼呢?.
保羅去引用耶利米書.
做什麼事呢?.
一般來說.
就是對比人類虛榮.
Anthropocentric的Boasting.
和Theocentric的Boasting.
的一個對比.
如果是這樣.
就是靠恩典得救.
和靠行為得救.
是不是?.
我看到也很難以接受.
這種隨隨便便.
用其他經卷來解釋菲勒比書.
這種籠統大路的解釋.
忽略了一個關鍵.
就是其實這兩段聖經.
都分別記載上帝的子民之間.
他們對於如何辨識而和華的作為.
其實在爭論.
這兩段主題.
我們都分別看到.
有真假割禮.
有如何確保蒙上帝賜福的爭論.
換句話說.
我們要問的問題就是.
其實保羅暗引耶利米書.
究竟對他當時的處境有什麼幫助呢?.
這兩段經文原來都見證著.
一個類似的故事的開始.
就是上帝的子民被威脅.
盡頭似乎來臨了.
耶利米書的上帝子民.
正正面對一個.
巴比倫即將攻陷耶路撒冷.

$^{841}$我們上帝的國度會被人打低嗎?.
有可能嗎?.
不可能的.
你不記得嗎?.
耶利米.
你不記得你的前輩.
以賽亞.
就在大概一百年前.
當時亞述不是要來攻打我們南國耶路撒冷嗎?.
你不記得上帝如何用他的天使天君.
去殺了西拿基納.
殺了亞述的皇帝.
以至保護了我們南國嗎?.
你為什麼這麼沒有信心?.
你不記得上帝如何保護以賽亞.
保護希西加嗎?.
今天你竟然這麼沒有信心.
你竟然叛國.
你竟然說我們要向巴比倫投降.
你真是大逆不道.
耶利米的信息.
對於當時上帝的子民.
如何面對這個國家的盡頭.
他們很受到極大的衝擊.
他們不能接受.
他們不願意面對.
他們記得的是.
It's fine.
沒問題.
在菲律賓輸.
保羅也在跟菲律賓信徒面對很大壓力.
如果保羅真的被斬首怎麼辦?.
是不是代表福音會完結?.
是不是代表教會會完結?.
是不是代表我們菲律賓教會要結束?.
是不是?.
會不會呢?.
其實保羅後來有沒有被斬首?.
有啊.
史提反有沒有被斬首?.

$^{881}$沒有啊.
史提反被石頭扔死.
有啊.
我們教會會面對困難嗎?.
會啊.
It's fine.
真的It's fine.
有時候不是什麼都It's fine.
是隕石正在扔下來.
兩個故事的情節發展也有些相似.
原來無論是《耶利米書》和《肥牙比書》都一樣.
包含著如何面對當前危機的爭論.
就如昔日《耶利米》的時代.
他們說.
回到前面.
你知不知道巴比倫是沒有受國禮的?.
國禮是絕招.
我們找其他國家.
其實你知不知道.
不只是我們以色列人有受國禮.
埃及,以東,阿滿,摩亞都有受國禮.
我們組織一個國禮大聯盟.
一個國禮政黨.
國禮政權成為一個反對巴比倫沒有受國禮大聯盟.
上帝一定站在我們這邊.
一定可以打贏巴比倫.
結果是如何呢?.
當時那些以色列人.
就是這樣用了這些虛假的國禮.
去取替真國禮.
因為真國禮從來不只是指皮膚的國禮.
更加是指一種內心對上帝的信靠.
他們藉著這個國禮兵團.
以為就可以得到平安.
但原來沒有聽上帝的話.
這些國禮大聯盟根本不會得到上主的祝福.
這種只是一種自欺.
而你不要責備的就是這種國禮.
昔日上帝拯救了以塞亞的時代.
但上帝的計劃在不同的時代.

$^{921}$都有點不同.
在菲律賓有一班基督徒領袖.
都以為苦難可以逃避.
不用面對.
在他們眼中不用受苦.
信耶穌何須受苦呢?.
為了駁斥這些假見證.
本來稱呼這些人為狗.
作惡的人.
割損身體,妄自行國的人.
原來在地上合神心意的盼望.
就是一班願意正視會有苦難.
面對它的一班人.
這兩個故事同樣充滿張力.
對於耶利米時代的人來說.
輸給巴比倫是不可被接受的.
輸給巴比倫還可以歌頌上帝嗎?.
還可以歌頌他是那位喜愛在世上.
施行慈愛,公平和公義的主嗎?.
菲律賓的信徒或有部分人.
面對受苦的預期.
已經失去了一定的信心.
要面對這種盡頭.
會有受苦.
這竟然是上帝計劃的一部分嗎?.
腓立比書其實就是要告訴你和我知道.
耶穌基督已經創啟了一個末世時代.
藉著受苦的僕人的故事.
耶穌不單是我們的救主.
還是你我面對受苦人生的典範.
耶穌的受苦人生的典範.
成為了我們面對未來的基礎.
兩個故事 耶利米和菲律賓.
都有面對一個盼望和審判的主題.
一邊是耶利米書沒有受割禮的人.
另一邊是甘願在亡國的裡面.
繼續信靠耶和華上主.
是掌管歷史的人.
耶利米就是這樣的人.
同樣在腓立比書.

$^{961}$一邊是打算可以避開受苦.
避開受苦 不面對受苦.
另一邊是一些用平安的心.
正是會受苦的 就是這樣.
所以如果用一個故事回答故事的角度.
你可以看到上面的故事就是保羅.
他怎樣把自己的一個開始,中和,結局的時間流.
如果有些頂智媒.
有留意後現代的一些所謂意識的研究.
你會留意故事時間流.
是一個當代很重視的概念.
你和我人生有沒有盼望.
就是視乎你內心在用一個什麼故事去發展.
去經驗一種什麼時間流.
來到差不多講座的最後15分鐘.
總結我今天幾點的分享.
腓立比書是一卷講關於盡頭的經典.
也是因為這樣.
死亡是一個不斷出現的主題.
腓立比書三章之前.
你不記得保羅也提到一個差點死的榜樣嗎?.
以巴弗提.
這位菲律賓教會差遣來幫助關心保羅的戰友.
在保羅的眼中.
他就好像耶穌一樣.
走近死亡.
當然我們知道耶穌是死亡.
但是死亡原來在以巴弗提身上.
我們也看到一樣出現在他的故事裡.
第27節說他病了幾乎死去.
第30節因為他曾經為基督的工作冒著生命危險.
幾乎致死.
這個幾乎致死.
正正就是耶穌基督.
他在腓立比書說.
你們當以基督耶穌的心為心.
他本有神的形象.
他不以自己與神同等為強奪.
反倒虛己取了老僕的形象.
幾何人的樣子就自己卑微.

$^{1001}$全心馴服.
以致於死.
那個以致於死.
就是這裡所說的幾乎致死.
耶穌基督是我們面對一切可能有的危機.
苦難逆境的一個典範.
他致死忠心.
他沒有逃避死亡.
他沒有逃避苦難.
他降世走向死亡.
二百佛提也一樣.
二百佛提也是一樣.
所以如何能夠見證盡頭原來是可以超越的呢?.
頂智妹.
原來盡頭是可以超越的.
海德格爾.
馬丁·海德格爾見不到.
但是我們相信耶穌的人見到.
超越盡頭的盼望來源是什麼呢?.
就是原來雖然在耶穌受苦的一生.
上帝好像沒有出現過.
菲勒比書的二章五到八節.
上帝不知道去了哪裡.
但是二章九到十一節.
上帝出現了.
因此神高舉了他成為至高.
由次吸他那超乎萬名之上的名.
使天上地上和地下的一切.
全都屈膝跪拜在耶穌名字前.
全都口裡宣認耶穌基督是主.
把榮耀歸給父神.
頂智妹.
這幅圖畫其實都是引用.
《以賽亞書》第四十三章.
時間關係.
我今天就不能夠去演繹那一段經文.
但是在這裡我想告訴你.
這裡不只是說一個普通的敬拜.
這裡是上帝為他的僕人耶穌平反.
平反他甘願去到為了你我的好處.

$^{1041}$受苦至死的一個經驗.
這種的受苦平反.
就成為了保羅都一樣期待.
他最終會得到上帝平反的盼望.
如果不是有耶穌的故事.
你和我面對盡頭是沒有盼望的.
同不同意嗎?.
如果不是有耶穌面對盡頭.
不如就信海迪加.
就是說到死亡就是所有可能性的盡頭.
不用想那麼多.
你正是死亡之後就渣都沒有.
那就是真人性了.
但是因為耶穌.
我們看到積著十字架.
原來盡頭是有得超越的.
而這一種也都成為了保羅.
他去到見證世界的盡頭.
他說三章二十節說到.
我們是天上的公民.
並且熱切等候一種盼望.
救主耶穌基督從天上降臨.
他要按著那臨死萬有馴服他的大臨.
將我們卑賤的身體變得和他榮耀的身體相似.
有沒有想過為什麼在這裡.
說到耶穌降臨就算了.
為什麼要將我們卑賤的身體.
變得和他的身體相似呢?.
為什麼要特別說身體呢?.
如果你知道前面.
耶穌剛剛在第二章說到.
他為我們釘死.
或者你能夠明白多一點.
原來身體受苦.
就是保羅,爾伯弗提和耶穌.
共有的一種受苦經驗.
今天你和我.
或者不一定會身體為主受苦.
但是當你和我在我們短暫的一生.
為別人得福音的好處.

$^{1081}$面對苦難.
菲勒比書不是受苦主義.
菲勒比書的受苦是一種為別人得福音的好處.
而要承擔的苦難.
這些苦難會帶來盼望上主的平凡.
反過來說.
弟兄姊妹.
如果你從來都沒有為主受苦.
你通常不會很盼望上主降臨.
明白我的意思嗎?.
如果你不是在地上為主受若干程度的苦難.
其實耶穌基督的降臨.
對你的意義不一定很切身.
相反,我正面再說一次.
為了那些為了耶穌的福音.
曾經經歷大大小小的苦難.
負上大大小小的代價.
或多或少的去承擔了一些危機.
這些人他們才會最期待耶穌的降臨.
最期待耶穌的平凡.
因為這個世界不能夠提供到他值得去追求的東西.
為了這些人.
他們會很熱切盼望耶穌基督的降臨.
耶穌基督才是他們盼望的那一位.
所以死亡或是我們人生裡.
你我都要面對盡頭,是嗎?.
但是耶穌在盡頭等著你和我.
海德格爾的Being in Time.
其實從哲學的角度.
已經被Paul Ricoeur的Time and Narrative破了.
這些是哲學的東西,說兩句而已.
這位基督徒的哲學家引用奧古斯丁.
說到其實我們藉著聖經的敘事.
是能夠為我們前面的盡頭提供另類的意義.
所以我們作為時間裡的生命體.
盡頭不是死亡.
盡頭是Judgement.
是一種審判.
是我們生命平凡的開始.
這是你的故事嗎?弟兄姊妹.

$^{1121}$這是你的盼望嗎?.
引用另一句.
菲勒比書被人背得爛的金句.
菲勒比書四章十三節.
和合本和合修大概是這樣譯的.
不是這裡的版本.
我靠著那家給我力量的.
凡事都能作.
一個不太好的翻譯.
有一點點It is fine的感覺.
新漢語譯得比較好.
靠著那次給我力量的.
什麼情況我都能應付.
最後讓我用羅馬書.
講到盼望的經文結束今天的分享.
因為我們得救的時候.
是存著這盼望的.
上文是說這個世界都與我們一同身入.
整個世界.
整個受造之物.
都面對著受苦.
可是看得見的盼望.
就不是盼望了.
因為看得見的事.
誰還會盼望呢?.
但我們若盼望那看不見的事.
就要耐心等候.
願主的話語成為你和我.
腳前的燈.
路上的光.
開我們的眼睛.
叫我們盼望耶穌基督的降臨.
前面的路會有多少危機苦難呢?.
我不知道.
我不是什麼印度神童.
但耶穌說在世上你們會有苦難.
但你放心.
他帶領著我們已經勝過了.
我們可以正視苦難.
正視盡頭.

$^{1161}$唯有不要隨便說.
It's fine.
我們才能找到真正的盼望.
真正的平安.
我今天的分享就到這裡.
我們一起祈禱結束.
多謝天父上帝.
你給我們今天的分享.
願你的話語成為我們盼望的基礎.
加力給我們.
願你去引導我們.
我們面對著前面很多的危機.
很多的盡頭.
大大小小的.
求你堅固我們.
讓我們誇口的.
當子著主誇口.
多謝主.
天地祈禱.
奉耶穌名求.
將時間交給龍.
給我們一個全新的理解.
肥立秘書.
我盼望今晚的訊息能夠幫助到大家.
在理解肥立秘書的時候.
視野更加闊一點.
新舊約也一起去研讀.
我們的眼界就會闊很多.
盼望這些盡頭的觀念.
不再成為我們的障礙.
而是更加鞭策我們向前走.
剛才我們漏了這張slide.
讓大家知道我們聯絡的方向.
我們的網址.
Facebook等等.
最近我們在YouTube.
開了一個帳戶.
你打進ABSCC.
就會看到我們過去很多的項目.
都在那裡出現了.

$^{1201}$今晚的講道.
很快可以在YouTube看回.
你可以介紹給你相識的朋友.
或是影子們.
今晚錯過了的.
都可以在我們的YouTube頻道裡看回.
在這裡最後.
我們呼籲大家.
繼續支持加拿大建造中心的工作.
我們作為一個non-profit making的組織.
我們都很需要大家經濟上的支持.
也在禱告上繼續鞭策我們.
如果你有些奉獻想給我們的話.
可以看下面這張slide.
就會說得到.
請童工展示下面這張slide.
我operate不到.
你的奉獻可以寫支票.
寄來我們中心的地方.
你也可以到我們的網站.
abscc.org.
下一張再下一張.
去到網站的時候.
在PayPal去奉獻都可以.
看到奉獻支持神學教育.
這個icon.
你點進去.
就可以告訴你如何奉獻.
如果用網上的奉獻的話.
你就立即會收到奉獻收據.
你可以在報稅的時候有些減免.
如果是支票.
你就可以寄來我們這個地址.
順帶提一提.
加拿大建造中心.
今年2022年將會有些新的發展.
現在先賣了一個關子.
很多在計劃當中.
盼望你們多點到我們的網站.
看我們最新的公佈.

$^{1241}$還有我們新的科目都會在這裡出現.
希望以後有更多機會和大家在一起.
我們不久又會有新的講座出現.
我們決定了以後的講座.
絕大部分都是以聖經的目標為主.
希望在3月的時候.
用另一個方式和大家見面.
有一個講座會講到.
你如何使用希伯來文來讀聖經.
這件事我們遲些會更詳細的公佈.
在我們當中.
時間也差不多了.
晚安.
謝謝各位.
拜拜.
\newpage



\section{}
\label{sec:L8_DVqUvOSM}
\textbf{ABSCC 講座 - 細味‧原文 ─ 簡易希伯來文 [示範 1] (AB2202 POL)}
\newline
\newline
連結: \href{https://youtube.com/watch?v=L8_DVqUvOSM}{\texttt{https://youtube.com/watch?v=L8\_DVqUvOSM}} ~~~~ 語音日期: 2022-04-20
\newline
\newline
\hyperref[sec:GA78znQ7bg4]{\small{< < < PREV SERMON < < <}}
~
\hyperref[sec:index]{\small{[返主目錄]}}
~
\hyperref[sec:efm9yyrZOo0]{\small{> > > NEXT SERMON > > >}}
\newline
\newline
$^{1}$包括我們這個聚會裡面.
我們根據記錄發現原來不單止是加拿大.
不單止是多倫多裡面的弟兄姊妹來參加.
甚至是加拿大其他各省都有人來參加的.
也都在美國裡面.
也都在世界各地有部分的弟兄姊妹是不同的國家.
我們都是歡迎大家來到我們當中.
我知道你們現在在不同的時區.
可能都在未天黑的地方也不一定.
或者是在另外一些時間我們也不知道.
歡迎大家.
有些弟兄姊妹可能是第一次參加我們建築中心裡面的講座.
用一個很簡單的介紹一下.
建築中心的一些情況是什麼.
加拿大建築中心就是一個基於多倫多.
根據多倫多的一個機構提供不同程度的神學教育.
我們是和香港建築神學院有一個很密切的關係.
香港建築神學院是我們堅強的後盾.
這裡在螢光幕大家可以看到一個網站.
就是www.abscc.org.
請大家也記下它.
你收到我們的電郵其實也會大概有些印象.
這個網站裡面有很多資料關於我們中心的工作.
最近有什麼新的科目提供.
我們提供的神學教育科目.
有很多不同的水平.
有正書的水平.
有中級的水平.
也有碩士水平的科目.
你進入這個網站裡面就會看到課程裡面的簡介.
我們最近期的科目有牽涉到教會歷史.
有牽涉到基督教教育.
也有中國教會史.
也會牽涉到一些神學的課程.
歡迎大家進入這個網站裡面.
就會詳細知道我們的狀況.
今晚的講題.
就是和希伯來文有關.
你看到講題你會發現有一個字是括弧的.
就是示範一.

$^{41}$就是說有示範一就是有示範二.
我想大家都可以想到.
我們大概在兩個月之後.
就會再邀請今晚的講員李浩然牧師.
再講示範二.
李牧師的簡介你也會收到的.
這個消息.
他在不同的地方也教授希伯來文.
大家都知道希伯來文是《舊約聖經》裡面一個重要的原文.
今晚我們很榮幸有李牧師在我們當中.
李牧師是非常年輕的.
大家可以看到.
他今晚大概會用一個小時.
來分享耶利米哀歌第三章.
來講簡易希伯來文.
如何幫助我們去研讀聖經.
我估計大家都已經印了講義.
印了講義出來.
然後聽李牧師講解.
就會相得益彰.
我想我介紹李牧師到這裡.
不如請李牧師自己加上什麼.
請他自己加.
現在我把時間交給李牧師.
李牧師.
謝謝,謝謝李牧師.
絕對不年輕的我.
不是你,大家差不多年紀.
不過很開心,謝謝李牧師,謝謝見到中心.
有這個機會給我去分享.
如果有什麼可以加上.
我是很喜歡看原文.
無論希伯來文或者希列文.
我自己特別都比較偏向喜歡希伯來文.
或者希望你都可以咀嚼多一點.
其實你再去看多幾眼的時候.
或者我相信大家都懂得讀中文.
起碼懂得聽得懂我說什麼.
你會發覺那個詩意,意象其實很濃厚.
所以亦代表那個空間.

$^{81}$你可以去想像.
可以去看得見.
真實一點.
聖經背後想帶出來的意思.
其實是空間很大.
這件事是吸引我的地方.
所以我自己無論有時講道,查經也好.
其實我都會很願意抽一些時間.
extra的時間來看原文.
因為我發覺那個能夠去啄真的味道.
是遠遠超過你看一個翻譯本.
我希望透過今晚的示範.
都可以給到大家淺嘗一點.
亦都希望可以鼓勵大家.
在未來日子裡面.
今晚會介紹一些工具.
你都可以有方法.
起碼去啄多一點希伯來文.
好嗎?.
好,我們就事不宜遲了.
今晚時間不長.
所以我會用《別人的愛》第三章來做一個示範.
而這個的.
大家看到了.
這個的示範.
我主要其實是這樣的.
(怎麼去了這裡).
這個是大家看到的簡介.
希望幫到大家去看得見用原文的好處.
兩個目的.
所以第一,明白認識原文的重要.
正如介紹裡面所講.
很多時候我們不認識原文的時候.
其實只能夠透過翻譯本.
我不是說翻譯本翻譯得不好.
而是很多時候它有它的限制.
而我們是需要知道它的限制.
當你去更加仔細地去解釋一段經文.
甚至你比較不同的譯本也好.
你要知道翻譯出來的版本.

$^{121}$它的限制.
以至你繼續知道經文要帶出來的重點是什麼.
那個你是希望透過認識原文幫到你.
所以你就可以透過這樣.
更加明白到作者的用意.
可以幫到你更加去應用相關的經文.
這是第一.
當然今晚能夠做的不是幫你完全掌握一切.
但起碼去認識這一點.
為什麼在這個世代.
我們有這麼多的資源.
其實我們是很值得花多一點時間去看原文.
第二,正如剛才所說.
就是因為這個世代有很多線上的工具.
如果你說相對可能十年二十年前.
你可能需要買一些軟件.
價值也不少的.
譬如我有兩套.
便宜的可能也要二三百元.
這樣你才有一個比較完善的系統.
我知道舊時有一些比較免費的軟件.
今時今日其實也不需要下載.
因為基本上網上已經有一切的資料.
這也是我希望今晚第二個想和大家分享和介紹.
有一些網站其實是很幫到大家.
現在大家基本上只要去那些網站.
甚至乎我當年學希伯來文.
一個很重要的步伐.
整個拆字.
其實大部分網上的工具.
它已經幫你做到.
當然有它不足的地方.
但是已經免卻了大家很多小時的痛苦.
也可以因為你已經有這些工具.
其實簡化了你要去看原文.
需要放上去的時間.
所以我覺得.
為什麼不在這個世代.
我們就善用我們有的資源.
來學習和使用.

$^{161}$所以這個是今晚最主要的兩個目的.
所以就算是示範.
質重點也不是解釋.
第三個只是一個示範的例子.
我相信也說不完66節這麼長.
但是希望透過這個例子.
給大家一些認識原文的重要.
和可以學習如何使用一些工具.
好,既然要工具.
所以我想首先和大家分享.
這個很有用的網站.
我放在chat上,大家可以看看.
或者直接點進去.
我未必會詳細解釋每一個.
因為時間所限.
我今晚也會集中主要用一兩個.
不過我會很簡略地說幾個值得看一看的.
第一個是趙老師.
他是一個其實是.
如果你有去過神學院.
或者有接觸過希伯來文.
這一篇,在screen看不到.
Basics of Biblical Hebrew.
是一個classical的書本.
這本是英文的.
趙老師的網站.
基本上將裡面的內容變成.
由中文翻譯.
這個網站是非常好的.
或者今晚我們也會看.
因為這個是字母詩.
基本上希伯來文22個字母都看完.
你說我不知道是什麼.
你可以進入字母表.
你起碼可以大約掌握到讀音.
或者認到符號.
起碼你不會覺得完全不知道在說什麼.
當然我沒有時間可以逐個高估.
但是這個網站有很簡單和重要的資料.
已經提供給你.

$^{201}$另一個呢.
今晚我就不會用的.
不過是一個叫Interlinear Bible.
其實網上也有很多.
但我覺得這個比較好的.
就是它的資料相對來說比較齊全.
和容易看.
不過唯一就是英文.
如果你熟悉英文,不介意用英文.
這個網頁是好的.
它會按照那個次序.
不過你要知道希伯來文是由右到左.
平時英文或者現代的中文都是由左到右.
它是由右到左.
所以你要看的時候.
基本上是由這個黑色不知道是什麼的符號開始.
向著這個方向看.
下面的英文.
當然按照英文讀.
但是它都是對著那個字.
這個I M.
其實M是括著的.
這個I這個字由這邊開始.
Demand.
Demand回過來.
好處是它逐個英文翻譯了.
你自己大約其實拿一個特別英文的翻譯本.
你可以逐個對照.
你就知道其實原文的次序是怎樣.
因為無可避免的.
你一翻譯一定用你那種翻譯出來的語言的次序.
所以這個網站是幫到你去認識原文的次序.
和每一個字大約是怎樣解釋.
還有它的拆字它都做了.
但這個就不是今晚的焦點所在.
另一個.
這個今天會用多一點的.
叫做Bible Bento.
這個最大的好處是它可以有和合本.
你看到我並排了和合本.

$^{241}$你在這裡選擇.
BHS就是代表舊約的原文.
你選擇和合本.
你就會看到這樣的東西.
它原先是這樣的.
有沒有記錯.
它分了一行.
有括弧.
我不詳細說了.
如果你想和我現在做的一樣.
你在這個加號按一下.
第二個加號按一下.
你就會看到這樣一模一樣.
我今晚主要會用這個網站和大家一起看.
《耶利米亞哀歌》第三章.
另外你說我都是不.
因為這個它的資料都是用英文的.
不是用中文的.
因為我都需要有關中文特別拆字那裡去解釋是一個什麼字.
這個信望愛的網站就非常好了.
這個信望愛的網站.
它就將那個希伯來文.
你可以即時顯示.
又或者下面它列出來.
剛才我看到我字.
那個I.
它這裡列了中文.
我.
它有其他資料.
今天不會詳細使用這個.
不過好處就是它有中文的翻譯.
最大的挑戰是什麼呢.
它這個原文直譯的字.
用的字.
和下面這個字典.
給的字未必是一樣的.
就是說你要中文熟到一個程度.
你知道它上面所譯的.
原來是下面哪一個字.
這個是有時候會比較挑戰.

$^{281}$譬如我給一個例子.
它這裡說遭遇.
OK.
一會兒我們都會看看.
其實是漢建那個字.
但是你不知道的時候.
你就不知道原來遭遇了漢建.
我真的漢建不了.
怎麼辦呢.
它這裡就算給的不同的解釋.
漢建,漢固,察覺,顯現,顯明.
都沒有.
上面哪個是漢建那個字.
它給不了你.
這個是少許的限制.
所以可能你說我想用中文又想英文.
又想看明.
你可能就將這個中文.
加上剛才我說的.
那些Internet Bible的網站.
並排去用.
可能就會幫到你.
OK.
另外一個叫做Step Bible.
那個我不詳細介紹了.
和剛才所說的都類似.
你可以去看看.
它有多少不同的翻譯和資料在那裡.
最後想說的是Next Bible.
我不知道大家有沒有用過.
這個和原文沒有直接關係.
不過它裡面很多的內容.
都會說到關於原文的.
我們一會兒會看.
大約是這樣.
每一節的經文.
應該不是每一節.
它有六萬多個筆記.
關於一些試驗的筆記.
或者一些研究上的筆記.

$^{321}$它是幫你去明白聖經.
這個我覺得就算你不用原文去看一段經文.
我強烈推薦你去用這個網站.
它有中文翻譯的.
不過中文翻譯最大的限制和挑戰.
就是它沒有六萬多個筆記.
可能它只剩下三分之一至一半.
所以很多時候.
英文看到的.
你打算找回翻譯.
找不到.
又或者英文是這麼長的.
可能中文只剩下這麼短.
它是濃縮版.
這是一個挑戰.
不過起碼它有中文.
還有它的翻譯.
你看到中文的聖經.
其實大致上是和合本.
但是當它發覺和合本的翻譯上.
和原文有出入.
它就會有自己的翻譯.
所以我覺得挺好的.
作為一個參考.
你可以看到和合本翻譯上的限制.
大約這幾個網站.
我覺得非常有用.
特別你想鑽研看原文.
用原文來看.
它絕對幫到你.
今晚主要用這個Bible Bento.
所以你開定它.
和這個Next Bible的筆記.
我主要用英文的.
這是我們有的工具.
我們就進入今晚的戲肉.
你手上有印出來的附錄.
我發給大家的.
有希伯來文的英文名字.
其實有希伯來文的字母.

$^{361}$不過你們的版本有.
我一會兒出的.
你可以認一認希伯來文的符號.
你也可以讀一讀它的英文拼音.
你就大約有一點點概念.
因為我們今天看的第三章.
是一個字母詩.
或者有些叫彌合詩.
或者我不妨展示給你看.
其實這裡有寫的.
譬如我用這個Next Bible中文的.
本處彌合詩句的格式變換了.
因為第一,第二章都是.
每三節順序以相同的字母作起手.
而不是每節以一字母作起手.
這些其實是Next Bible已經說了給你聽.
所以其實是很方便.
你不需要太過深入去找一個識經書.
或者你家裡沒有識經書.
它這個網站已經提供了這樣的資料.
最基本,你最需要知道的資料.
所以我們知道.
每三節都是用同一個希伯來文的字母.
所以你一會兒會看到第一段叫Alab.
好像一個X.
其實英文的A.
頭三節都是用A來開頭.
如此類推.
這個是我們今天會看的經文.
可能這個耶和米哀歌第三章.
大家也不算熟悉.
最熟悉的是我這裡highlight了22和23節.
「我們不致消滅,是出於耶和華諸般的慈愛」.
歌也有一唱英文歌.
「Their new every morning」.
這個可能是我們對耶和米哀歌第三章的印象.
不過其實在說什麼呢?.
它的核心信息是什麼呢?.
我不知道你有沒有想過.
我希望今晚和大家去看.

$^{401}$剛才說的兩個目的.
同一時間或者利用對原文少少的認識.
或者用這些工具.
幫我們對經文多一層的了解.
以致或者可能會找到.
dicode到多少少這一章的經文.
其實和我們說什麼呢?.
OK.
好,那麼.
所以就想和大家逐個去看.
我會這樣的.
我會和大家去看每一句.
OK,每一節.
因為這個是尼合斯.
其實每一節第一個字是哪個字.
你說不是中文,我他他,就是了嗎?.
原來其實不是的.
因為剛才說希伯來文.
由右到左讀.
不單止是這樣.
它的次序和翻譯是不同的.
大多數九成九的時間都是不同的.
有相同的時候.
但是譬如最基本.
通常是動詞先行.
接著主詞,接著受詞.
這是最基本的.
特別是平時最市民.
在詩歌體裁就比較不同.
但是你一會兒繼續看下去的時候.
你就會發覺原來.
平時我看耶利米亞哀歌第三章.
其實是純粹中文翻譯.
所以大多數是你我他先行.
不過既然這首是一個字母詩.
既然他刻意每三節用一個希伯來文字母開頭.
是一個很有心思的安排.
那麼究竟每一個第一個字.
每一句第一個字.
他選了什麼字呢?.

$^{441}$我們會不會多明白一點.
他選的字.
可以令我們更加清楚這首第三章.
想帶出來的核心信息是什麼呢?.
我希望從這樣的角度和大家一起看.
也去學習一些希伯來文.
也順著次序.
由阿拉伯到塔爾維爾.
22個字母.
來淺嘗希伯來文.
我會….
這個你看到是和合本.
你手上有的筆記.
我會和大家看一看BibleBent的網頁.
這個網頁怎麼去用呢?.
第一.
如果你想去看第一個希伯來文字母.
或者字.
它是哪一個和合本的字.
你可以這樣.
你放一個老鼠.
放在第一個字.
叫做由右邊到左邊.
你放在第一個字的時候.
你會看到翻譯英文.
I.
這是其中一個最簡單的翻譯.
有些字.
這個情況不會是代名詞.
但有些情況你會發現.
需要知道多一點.
這裡未必能夠給你.
這是最簡單的翻譯.
然後下面是拆字.
我今天不會詳談.
有需要才會說.
但起碼你會看到.
這是一個I.
而你會看到右邊的我字.
它也有顯示.

$^{481}$你也可以自己試.
你開一個網站.
那個顯示了一個.
其實告訴你.
這個.
Ernie.
就是那個希伯來文.
就是等於那個我字.
有限制的.
不是沒有限制的.
一會兒繼續會說.
不過起碼這個網站有一個好處.
你大約知道.
每一個希伯來文.
相對出來的中文字是什麼.
剛才說.
英文比較多的.
Interlinear Bible.
直接自跟自翻譯.
你容易跟很多.
中文基本上.
我不覺得有.
除了這個和剛才說的信望外那個.
信望外那個.
因為用的字有時跟不到.
有時會帶一些挑戰.
這個直接highlight和合本.
你可以跟到.
當然這個也限制在和合本有的翻譯.
Anyways.
我們現在嘗試用這個方法.
去找出每一個希伯來文.
每一句.
每一行.
第一個字.
它相對的中文字是什麼.
你看到.
這個是我的字.
它說什麼呢.
如果很快地逐個字翻譯.

$^{521}$你看到其實是我.
事其實沒有出現.
為什麼呢?.
因為它沒有動詞.
簡單來說.
它紅色的就是動詞.
或者橙色.
紅色橙色的.
也不知道.
紅紅的.
紅紅的色那個.
第二個字其實是我.
其實這裡有一個der.
我是哪個人.
什麼人呢?.
是一個看見或者遭遇.
OK.
困苦的人.
是在也好.
因也好.
你看到和合本用了因字.
因為憤怒的象.
其實有一個他者在後面.
我們不詳細看.
我想給你大約一點看到怎麼去用它.
這裡其實就是說.
如果你看回我給你的和合本.
你可以這樣highlight的.
紅色了它.
就是第一個字.
OK.
你可以現在即時拿支筆去這樣做.
紅色的我字.
我是因.
如果你剛才深水清.
你看到其實耶和華的名字沒有出現.
這個是翻譯的時候.
它加進去.
它怕死你不知道.
不過你喜歡的話.

$^{561}$好像我這樣豁出去.
好像我給你的聲音豁了.
豁出去.
暫且拿走它.
為什麼呢?.
待會到最後你就會知道了.
我是因.
你可以代入它.
它的憤怒.
或者它憤怒的象.
遭遇困苦的人.
這個是它第一句.
也想和大家說一下.
剛才提過的遭遇.
其實是看見的.
我想給你看到的是.
原來在一個翻譯上面.
如果它這樣翻譯.
我是因.
放回耶和華.
憤怒的象看見困苦的人.
你會覺得有沒有搞錯.
這是什麼翻譯.
所以你看到翻譯上有它的限制.
但是這個字Raa.
基本上大部分時間就是講看見.
可以講經歷的.
待會給你看Net Bible.
它有提過一點.
但是它的字最簡單最本身就是看見.
所以.
你又看到翻譯是因為想我們明白.
看見困苦你不明白.
是要講遭遇困苦的.
但是這個不等同.
這個字就是遭遇.
這個字其實是看見.
你看到這個邏輯.
所以很多時候.
有時候人解錯經.

$^{601}$就是因為他純粹拿了翻譯的字.
就等同了這裡用遭遇.
第二個經文是用遭遇.
那就同一個字了.
其實你是需要查一下原文.
是不是同一個字.
也很多時候原文要翻譯成另一個字.
原因是因為語言上的限制.
當你一翻譯出去.
有時候是無可避免的.
一定要這樣翻譯的.
你要找過原文.
可能才能夠明白和了解到的情況.
給你看看那個Net Bible.
它是.
好像中文也有的.
Net Bible你可以這樣.
我懂得看Net Bible其實很重要.
那個筆記.
它寫著三遭遇.
希伯來文動詞就是.
Ra.
看見.
To see.
有很廣的意義.
包括一.
看見.
如從經驗學習.
二.
看見.
是什麼?.
如體驗.
給了一堆經文.
本說解作.
發言人遭遇這些事.
本字義用在二章二十節.
要求耶和華看.
它給了一些這樣的背景資料.
其實是很需要的.
很多時候我們不知道.

$^{641}$或者起碼我只是看翻譯本的時候.
你靠Net Bible這個筆記.
其實是幫到你.
起碼有一些.
你要小心.
不要亂按去解錯的地方.
它是會提醒到你.
一些很重要的事情.
它是會告訴你.
所以我很鼓勵你.
基本上你去查經的時候.
自己去學習的時候.
你去Net Bible的筆記.
去看一看.
你會看原文的.
你就會發覺.
多很多東西.
這是最簡單的.
它只是提一提動詞的名字.
Ra.
但是.
再有一些比較進深的筆記.
你知道原文.
特別是文法上的規律.
你就會幫到你.
去看這些筆記.
是多很多的.
好,我們回來了.
這是第一個.
我們繼續可以去看.
究竟第二個字.
究竟是什麼呢?.
你去回到BibleBent的網站.
你會看到.
它這裡有些複雜的.
它都是寫著「我」字.
但是你會發覺.
和上面的「我」字是不同的.
這個寫著「I」.
這個寫著「object marker」.

$^{681}$是什麼東西呢?.
很簡單的說一說.
在希伯來文.
如果是一個直接受詞.
或者它有一個term叫.
指定的直接受詞.
Direct.
我忘記了.
DDO Marker.
Direct.
我忘記了第二個字.
因為習慣了用Short form.
OK.
它就是.
A definite direct object marker.
OK.
是指定的直接受詞.
就是說動詞.
它要直接是那個object.
它很多時候不一定用這個東西.
而它可以有一個詞尾放進去.
詞尾在這個case是叫first person.
就是「我」.
大約說到這裡.
有興趣的.
我們在往後的課堂可以繼續討論.
在這樣的情況下.
所以這個雖然和上面的字不同.
但其實都是在說「我」.
所以「我」的字.
是第二行的第一個字.
所以你看到.
不是它引導.
不是動詞.
不是主詞.
而是受詞.
「我」就是第二行.
和第一行一樣.
都是說「我」.
OK.

$^{721}$好,回到這裡.
第三行.
它是用了.
英文也叫做only.
其實中文是high了真詞.
它high了不是它.
其實是不對的.
因為它其實基本上是跟動詞的.
如果它沒有直接用一個代名詞的話.
它是會跟動詞的.
所以不是跟這個.
它寫adverb或者一些particle.
它不是跟它.
它是要跟動詞.
所以如果真正要high.
這個是個真詞.
OK.
你見到.
high了紅色是個真詞.
這個就是第一個字母.
adverb.
就是「我」.
「我」和真詞.
有什麼用呢.
我們要放長雙眼去看.
我們繼續下去.
去到B了.
B站.
現在拉下去第四節.
第四節那個.
它就是一個動詞.
很複雜的動詞.
我也不打算深入去探討.
往後的日子希望有機會跟大家分享.
但是你起碼見到.
它是說.
起碼右邊.
它是high了個「膚乾」.
其實正如剛才說.
其實要加一個「他」.

$^{761}$因為它的verb person.
在這裡寫著.
第三生的那個「他」.
因為這裡沒有主詞.
其實它應該包含的.
所以如果正式要.
我已經說了.
我不是說正不正式.
我自己去high.
我就會好像這樣.
它是.
OK.
「膚乾」.
它是什麼「膚乾」呢.
那個是受詞了.
那個是裡面的東西了.
但是它是.
「膚乾」.
是個動詞.
也是第一個字.
OK.
第五字.
其實和上一行很類似.
這次另一個動詞.
就是「他足裡」.
你看到它這次high了個「他」字.
所以這些網站有時候是這樣的.
你要容許.
它不是perfect.
不過它幫到你.
OK.
「他足裡」.
就是第三個.
我不show我自己的版本了.
最後才show給你看.
這個.
Sorry我去錯地方了.
這裡.
第六字.
OK.

$^{801}$其實這個是一個preposition.
介係詞.
這個是其中一些最頭痛的翻譯.
剛才我忘記了說.
你看到它high第一節也有.
這個「ber」.
OK.
in.
第一節它用了和本.
用了做「因」.
其實你可以算譯成「在」.
可以譯成.
「敗」.
OK.
透過或者.
我不知道怎麼翻譯中文.
你把它放進去的時候.
其實那個.
那個nuance.
那個微細的意思.
其實挺不同的.
所以做這些preposition是很痛苦的.
也想讓你明白.
為什麼很多時候在不同的翻譯本上.
有這樣的出入呢.
很多時候.
有其他原因.
不過很多時候.
是這些的介係詞.
或者是一些的particle.
或者是對一些次序上的理解的出入.
所以會形成幾種不同的翻譯.
OK.
但是特別值得留意的是這些preposition.
就是.
究竟是做「在」.
「敗」.
我不深究,有興趣可以再學.
這個值得留意.
但是這個case就容易很多.

$^{841}$它是「在」什麼呢?.
如果用「在」.
「在幽暗之處」.
是幽暗的地方.
是開頭第一個字.
所以你大約會看到這樣的東西.
四至六節.
「他屎枯乾,他捉女,在幽暗之處」.
好了,接著到Gimel.
這個第三個字母.
七至九節.
它也是類似用一個動詞.
就是「他用了泥巴」.
我會加上「圍住」.
其實它是在說用泥巴去圍住.
OK.
接著呢.
這個第八節.
你會看到有顆星星.
其實是沒有列出來的.
有時候是這樣的.
就是不能直接翻譯也好.
翻譯不出來.
或者用不到.
這些字放在一起翻譯也好.
在這個情況下.
其實我會加上「甚至」在前面.
它是一個強調.
甚至.
我去哀號求救.
接著就比去不到上面好.
第三句,第九節.
它原來是用同一個字.
中文看不到.
或者很多版本都看不到.
因為幾不同.
一個是用泥巴圍住.
這個呢.
它擋住.
為什麼會這樣呢.

$^{881}$第一,有時候有些動詞.
是可以有幾不同的意思.
Depends on它說的上下文.
或者Depends on它說的主詞或者首詞.
也有些時候.
可能是同一個字.
可能它有兩套不同的意思.
等等.
但是值得留意的是.
原來是同一個字.
一個是圍住.
一個是擋住.
其實林心玉經常都很類似.
她Hip stones.
即是拿石頭來做一個牆.
她也可以來擋住這條路.
不過她現在似乎.
專門在同一段.
用同一個字.
來表達兩個意思.
記住它.
因為接下來你會看到更多的例子.
所以在第三段.
你大約會看到這樣的東西.
我用了紫色代表相同.
就是它用泥巴圍住和下面它擋住.
相同同一個字.
第二句.
第八節.
是一個甚字.
是它沒有.
我用了square bracket代表我自己加上去.
在和學本那裡沒有的.
但是原文有的.
好,到Dialect.
第四個字母了.
第十節.
其實它highlight多了一點.
其實它只是一個紅.
不是向我預.

$^{921}$那個是下文繫上的意思.
所以只是一個紅字.
紅人的紅.
然後這個是路.
我的路.
其實直接就是我的路.
不過它要譯成正路.
那也OK.
它似乎說我要走的路.
你看到那個經文.
它轉移了.
它不讓我走.
應該走的路.
OK.
你看到其實在上文也有出現.
這個highlight有好處.
它即時展示給你看.
OK.
不過更加值得看.
其實這個字.
OK.
將公.
將開拉公.
拉.
它拉的意思.
其實你看到和上面這個字是一樣的.
不過你說不是.
是因為最後那個.
其實是一個K.
當它在最後的時候.
它會拉長了.
所以和上面好像有一點不同.
又不詳談了.
不過其實是同一個字.
這個它.
所謂我的正路.
和這個它將.
一個是動詞.
一個是名詞.
同一個字根.

$^{961}$所以你會看到.
第四段就是這樣.
紅.
我正路.
和它將.
就是每句的第一個字.
第五個.
Hey.
每一次走過來.
它都會移動了.
我要走回去.
OK.
這個Hey.
它第一個.
即是第十三節.
就是它.
它正路就打了射入.
其實是它把.
什麼什麼射入.
它把射入.
我就會這樣打.
頭兩個它把.
然後射入.
它將一些東西.
然後後面又說.
就是.
箭.
它說的是一整袋的箭.
射入我的肺部.
OK.
它射入.
然後十四節.
就是我.
不是它做的.
subject.
現在是我.
需要寫First Person.
我成了.
十五節.
它充滿我.

$^{1001}$因為這個.
它是第三身.
是那個主詞.
First Person.
它所以.
但它有個詞尾.
你會看到它寫著First Person.
在詞尾.
所以就不只是它充滿.
是它充滿我.
OK.
這個是希伯來文動詞的特性.
它可以有一些東西加在後面.
所以它充滿我.
就是第一個字.
所以你會大約見到這樣.
它把射入.
我成了.
它充滿我.
最後.
十六到十八.
這個是第一個大段落.
十六到十八.
這個是Word.
它其實是一個.
它其實是一個and字.
OK.
不過呢.
它希伯來文很特別.
有一些字和剛才的proposition.
in一樣.
可以黏在一起.
所以你會見到.
這個其實是and.
它黏在一起的動詞.
所以你會見到.
它由.
讀錯了.
三段.
我不知道.

$^{1041}$右邊讀邊.
它由三段.
由下面.
其實就不是.
就這樣遠離.
其實理字沒有在這裡.
我想不詳細解釋.
你去看Next Bible.
你會見到其實是七十字的本.
希伯來文的翻譯.
和希伯來文原裝的手叉本的出入.
不詳細談了.
你有興趣看Next Bible的note.
不過這個字本身.
是遠離的.
之前都有個and在這裡.
所以其實我.
你會見到我加上一個右字.
and.
OK.
and.
什麼什麼遠離.
還有這個.
又是一個右字.
然後這個是我說.
我就說.
所以你會見到.
我是這樣的.
它由三段.
由.
我就遮住了理字.
我比較支持原文.
馬索拉抄本的理解.
沒有一個理字.
是說平安遠離的.
也都.
由我就說.
似乎他刻意.
或者這麼說.
就用這個and字開頭.

$^{1081}$用這個wab開頭.
因為這個字開頭的字很少.
所以他可能用這個方法.
但是似乎都是一個.
有意思的.
為什麼我這樣說.
我們在這裡停一停.
我現在走了六個字母.
由alep去到wab.
這六個小段落.
其實你會見到.
他有一些意思想表達出來.
綜合來看.
就算我們特別.
看回第一個字.
你更加可能見到那個.
他有一點刻意想帶出來的意思.
為什麼我這樣說.
如果你留意一下.
你嘗試將highlight的字看一看.
我.
一開始.
我是一個什麼處境.
我是落入一個什麼地方.
他好像再強調.
真的.
還要說是再三反手攻擊我.
強調他所遭遇的事.
然後第二段就說.
他.
你記住我還沒開估是誰.
他令我枯乾.
他觸雷.
在幽暗之處.
其實是他令我在那裡.
你感覺到嗎?.
他開始就好像.
來到我去說他所遭遇的事情.
而那個令這一切發生的是他.
然後第三段就繼續了.

$^{1121}$他.
尼巴圍住.
甚至.
令我禱告不可以上達.
他擋住.
如果你又記得是同一個字.
似乎他又很強調.
他弄了一堆石頭.
圍住我.
又擋住我.
其實是一個甚至.
似乎是一個加強去說.
其實發生了什麼事.
我遭遇的是什麼事.
第四段說紅.
這個可能沒什麼特別意思.
或者可能.
他就好像一隻熊.
你遇到一個熊來襲擊.
我的正路.
被轉來.
你又見到他帶回了一點點和我的關係.
不只是說他.
然後說他將功.
來當我做箭靶.
有時只有第一個字不知道.
但我想強調.
第一個字似乎想帶出方向.
帶出他那個段落.
或者整組的紀錄段落.
想帶出的方向.
然後第五段.
他繼續說下去.
一些東西射入我的閉骨.
我成為笑柄.
他繼續充滿我.
不過是苦楚.
最後他又怎樣.
可能這樣看右字就更有意思.
不單止上面說的東西.

$^{1161}$又這樣.
又平安遠離.
又.
最後一個是我說的.
說什麼.
我的力量衰敗.
我在耶和華那裡毫無指望.
這才是第一次耶和華這個名字出現的地方.
第一.
第二.
似乎他就帶出.
他落入去到這樣的境地.
一句總結.
我什麼力都沒有.
耶和華.
我在耶和華那裡連指望都沒有.
我不知道你感受到嗎.
他似乎刻意這樣引導下來.
來帶出一個這樣的味道.
讓我們感受他落入的處境.
甚至到這一刻的結論.
我毫無希望.
在耶和華那裡都沒有拍攝.
在這樣的情況.
跳下去就幾不同了.
時間所限.
我就不和大家逐一去看了.
我找到的.
你可以回去繼續.
你都可以告訴你.
有幾節我想和大家看多一點.
但是最後的.
我們一起看看.
會不會我們這樣去看.
看到一個更加核心的訊息.
19度有一節Science.
一開始他沒有耶和華.
耶和華那天是他加上去的.
求你紀念.
紀念什麼呢.

$^{1201}$他的困苦.
不過他第20節又用同一個字.
Remember.
其實是一個九月很重要的字.
他說我心.
這次不是紀念.
同一個字可以有不同的翻譯.
因為要看想表達的是什麼意思.
同一個字.
不過這次不是說紀念.
紀念可能是以往的事.
他現在說想念.
即是說現今.
我想起這些事的時候.
心裡幽悶.
同一個字.
記得我剛才有試過.
有兩個不同帶出來的味道.
他在同一段拿出來用.
然後他繼續說.
他說這事.
什麼事呢.
他說當我想起這件事.
今次不是幽悶了.
今次是有指望.
看到這個對比嗎.
即是他在說.
我求你去紀念.
我這些困苦.
我想起這些事.
就很慘.
不過這件事.
我想起的時候.
有指望.
那件事是什麼事呢.
就落到我們最熟悉的那句.
Hat.
這個他就是說.
原來是想起什麼呢.
不是我們不自消滅.

$^{1241}$那個一開頭是說.
諸般的慈愛.
他說的是耶和華.
諸般的慈愛.
其實值得看看.
我想和大家分享.
這個其實有不同的翻譯和理解.
我自己比較傾向的.
是這樣去解讀.
這個是Hat.
諸般的慈愛.
這個是總數.
說的是耶和華.
諸般的慈愛.
但他用了一個最難翻譯的字.
Particle.
是什麼呢.
可以解作因為.
不過呢.
有時候也可以解作debt.
好像這裡這樣寫.
如果我這樣理解.
可能你會明白多一點.
或者我覺得這樣也合理.
他說想起什麼.
上一節說.
我想起耶和華諸般的慈愛.
或者是一個.
英文通常是loyal love.
OK.
永遠不變的那種愛.
這個愛是什麼呢.
什麼是這種愛呢.
就是我們不自消滅.
就是他的憐憫不自斷絕.
我嘗試告訴你.
他想起什麼.
他想起這件事的時候.
他又展望.
我想起耶和華諸般的慈愛.

$^{1281}$什麼是耶和華諸般的慈愛.
原來我仍然到今天我們.
你想起這個背景.
耶利米所作的.
是一個秘魯的字.
落入巴比倫的地方的時候.
我今天仍然存活.
原來這個就是耶和華的慈愛.
諸般的慈愛.
然後再加一句.
就是因為他的憐憫不斷.
如果我們這樣解讀.
似乎更加合理.
因為很多版本是說.
耶和華諸般的慈愛.
沒有斷絕,沒有小到.
似乎我學本今次也比較準確.
不過次序是.
耶和華諸般的慈愛.
就是我們今天仍然未至於消滅.
就是他的憐憫不斷斷絕.
這樣合理.
為什麼呢?.
繼續看下去.
這個我們很多時候都理解錯了.
不是每個早晨.
早晨是新的.
他是說.
如果你看下去.
這個形容詞是MP.
陽性,眾數.
即是說.
不是一句.
不是單一.
而他需要希伯來文.
去配合這個陽性,眾數.
其實就是這個字.
那個Hazard.
那個耶和華諸般的慈愛.
就是陽性,眾數.

$^{1321}$在這樣的情況下.
其實這裡像是說.
這個耶和華諸般的慈愛.
他上面22節說了怎樣.
23節說.
這個這樣的慈愛.
是每天都是新的.
他們是新的.
所以我highlight了新的.
新的那個字.
新.
什麼是新呢?.
就是形容耶和華諸般的慈愛.
每天都是新的.
每天我都繼續不至於被消滅.
每天仍然經歷他的憐憫.
這個是新的.
繼續下去.
都說說這個.
這裡也都形容他的信實.
耶和華的誠實.
他是一個信得trustworthy.
faithful的那一位.
這裡很多時候譯作廣大和諧般.
其實是說很豐富.
abundant.
ok.
很足夠.
你這樣看就明白了.
仍然可以傳承.
仍然經歷他的憐憫.
是因為他很信實.
不是大.
而是很豐足.
ok.
接著第24節.
他就繼續說.
我的份.
是誰呢?.
就是耶和華.

$^{1361}$ok.
我要因為這樣去仰望他.
我自己說.
我說.
雖然不是我先說.
我是耶華的份.
我就說.
因此.
這個比較複雜.
我可以仰望他.
我想說的是.
你再看這裡.
可能你會更加看到.
這一段.
ok.
他上文說要紀念.
想起一樣東西.
又指望.
他這裡其實是.
告訴我.
他想起什麼東西.
哪一樣東西.
是耶和華朱潘的慈愛.
這一樣東西.
每天都是新的.
他正在經歷.
就算在秘魯之地.
他都在經歷.
以至到.
他更加清楚.
是因為耶和華是我的份.
我的份.
我能夠有的portion.
就是神自己.
所以我要去仰望.
我可以去.
其實和上文的指望是同一字.
可以.
那個盼望就在他那裡.
你看到.

$^{1401}$這兩段其實原來有關連.
雖然他start with不同的字母.
但是有關連.
他在告訴我們.
原來這一切.
是他仍然.
可以在一個困苦裡.
有指望可以繼續去走.
的原因.
接著25到27.
這個tag.
我說不看這個.
不夠時間.
這裡.
其實都是.
都看看.
因為很重要.
這個是.
一個很特別的一段.
每行都是start with同一個字.
good.
中文完全不是.
看不到.
因為開始用的因.
其實他interpret了.
其實他在說.
耶和華是好的.
是不是.
為什麼.
不是為什麼.
耶和華是好的.
對那些人.
對等候他的人.
尋求他的人.
是好的.
中文的詩恩.
其實是好的.
對他好.
是好的.
其實26,27都是start with.

$^{1441}$繼續說.
他能夠去.
仰望.
他這個通常證物.
其實是有耐性地.
去仰望.
他的救恩.
這樣是好.
如果他小時候.
有不同解釋.
仍然富得起.
或者仍然有日子等候.
可以回歸的時候.
所以他幼年去富額是好.
但你看到這一段.
三句都是start with.
好字.
能夠繼續去等候耶和華.
繼續去仰望.
這樣是好.
繼續下去.
他繼續說.
他當坐.
他當貼.
他當遊.
他當貼和他當遊.
其實是同一個動詞.
有時用動詞.
可以翻譯出來.
是很不同的.
你可以這樣理解.
他在說.
當他見到這一切.
你可以怎樣回應.
原來你可以這樣繼續仰望他.
在他面前.
其實是同一個字.
默然無聲.
去等候.
耐性地等候他.

$^{1481}$甚至受凌辱.
都可以捱過.
有少少這樣意思.
最後.
這段落.
三個都是同一個字.
叫keep.
是因為.
不過在不同場合有不同翻譯.
主要是因為.
下面三十二節比較複雜.
不過用rather.
英文和日文.
翻譯成衰.
最後都是因.
其實如果這樣綜合整段.
這五個小段落.
你就會見到一個這樣的景象.
上文說到.
他落入一個這麼痛苦的困境.
現在他就說.
我想起不單止困難.
而是想起一些好.
是甚麼呢?.
是耶和華的慈愛.
是他的憐憫.
所以.
等他是好的.
靜靜地.
忍耐地等候他.
是好的.
就算現在要去負壓.
都是好的.
不是災難是好的.
但能夠等候經歷他.
是好的.
所以他可以這樣去回應.
因為.
主是不丟棄人.
因為.

$^{1521}$他雖然會洗人憂愁.
但也會照他的慈愛.
其實和上文.
22到23節.
裡面有的字是一樣.
都是因為.
他不甘心洗人愁苦.
你看到嗎?.
如果這樣看.
他原來是有拉筋的.
他嘗試去說出.
落入一個這樣的處境.
你可以怎樣去回應.
為何會分第一個做六段.
第二個做五段呢?.
這個.
我自己想的.
我不確定有沒有其他識經書.
參考.
分段經常有很多不同.
不過我自己嘗試這樣去分的時候.
發覺.
多幫到我去消化.
理解整張的中文.
所以我用最後.
五分鐘時間.
看完下面.
我未必能夠高呼所有.
我當是雜作來做.
但可能你更加看得明.
為何我會這樣分.
和他想帶出來的意思.
繼續看下去的時候.
你會看到他用了三個動詞.
Infinitive, To, 什麼什麼.
來說.
有被人踩在腳下.
在至高者面前屈往人.
做一些不對的事.
這些都是主看不上的.

$^{1561}$亦有翻譯.
可以翻譯成?.
這個主是否看不到呢?.
看不看得見呢?.
可以這樣意思.
跟著MEM.
就是三個問題.
誰能說成就就成呢?.
除非是主明的.
其實上帝說的.
就能夠成就.
不是他說的就不能成就.
三十八節.
口,於口.
其實從他口而出的就是和福.
雖然一切都是在主的命令當中.
三十九節為何.
當你因自己的罪受罰.
為何要發怨言.
四十到四十二節.
很特別的在這裡.
用了我們.
其實是一個祈願式.
讓我們去深深考察.
讓我們去舉起手.
其實沒有禱告.
只是舉起手.
向天上的神做一個回應.
跟著說了一樣東西.
我們犯罪活躍.
你並不赦免.
說得好像很負面.
但其實在說的.
就是他們正正經歷的事.
因為他說.
你是.
遮蔽.
一個是被怒氣遮蔽.
一個是以黑暗遮蔽.
以致禱告入不去.

$^{1601}$也讓我們成為污穢.
渣滓.
說的是.
他因為犯罪.
上帝不赦免.
去到這麼嚴重的地步.
所以上帝就這樣.
來遮蔽自己.
以致他們落入如此境地.
下面這個pay.
剛才說漏了.
summit, pay.
就是結局.
或者現今的境況.
仇敵大大張口.
恐懼陷坑.
等等.
臨近我們.
以致我淚如河流下.
如果你看回整個圖畫.
就是在說.
其實這裡可能不是.
其他人如何對待我.
可能是在說他們看回自己.
深深考察自己.
原來我們一向就這樣.
踩人在腳下.
屈往人的顛倒是非.
原來主是看不上.
不是主所定.
誰能夠去定呢.
引導他.
主要去懲罰的時候.
我還能夠說什麼呢.
不過核心.
中間.
其實特別.
你記得那三個good.
good, good hope, 好的.
其實就是五個小段中間.

$^{1641}$似乎核心就是說上帝.
就算一切都是好.
而這裡第三.
我們要怎樣去做.
我們要看回自己的行為.
我們要去求.
向上帝祈禱.
雖然我們知道我們犯罪.
理不赦免.
即是一個覺悟.
一個回轉.
去到上帝面前.
其實我們真的做錯了.
然後他解釋.
在這樣的情況下.
上帝就不理會他們.
仇敵地對待他們.
不過還未完.
去到最後那段.
似乎就是resolution.
起承轉合.
其實接回上文的我的眼.
也重覆了.
流淚.
流到什麼時候?.
直等到耶和華誰沽.
我的眼不單止流淚.
令我傷痛.
因為我的眼就看到本城眾民的遭遇.
這個Iron.
沙迪.
遭遇是甚麼?.
被人迫迫.
然後他們洗.
我的命斷絕.
在牢獄中流過.
好像水一樣流過.
我頭領落命也斷了.
淹死了.
不過在這樣的情況下.

$^{1681}$我去求告耶和華.
說的是上帝聽見我的聲音.
求告你的日子臨近.
上帝臨近.
到Ratio.
58到60節.
他說的是.
上主去申明了我的怨.
他見到我的委屈.
也見到.
同一個字眼看見.
他們受敵.
怎樣對我.
他們所出現的一切謀害.
其實這裡.
他是用一個.
叫perfect tense.
不過其實.
未必一定是.
像英文一樣.
有識經.
Net Bible也說.
可能是一個奇緣.
或者呼喚的意思.
不過不詳談了.
不過你看到.
他向耶和華求告.
也見到上主的回應.
他申明了我的怨.
見到我的受屈.
見到受敵怎樣對我.
所以他就說.
你聽見我的聲音.
你聽見.
或者祈求你聽見.
類似剛才的情況.
他們辱罵我的話.
口中所說的話.
攻擊我的人.
他們怎樣.

$^{1721}$做他們座下.
無論坐或起.
都以我為歌.
其實是說.
上主你看看他們對我.
在做甚麼.
其實是連續58到60節.
最後.
他對上主有一個.
這樣的祈求.
或者.
不是命令.
一個這樣的呼求.
你要去施行.
這個TARF.
SIN和SHIN.
好像有三差.
SIN是那一點在左邊.
SHIN在右邊.
通常在這些字母.
他用回.
他那一點可以.
無所謂,當作同一個字母來理解.
不過平時你會發覺.
是兩個字母,不過字母.
他當作是一個字母.
TARF就是說.
你要.
去施行報應.
你要洗他們心肝眼.
你要追趕.
你見到這是最後.
這個耶利米.
在這一切裡面.
他說完.
他的眼.
見到的情況.
他繼續等候耶和華的時候.
他.
重覆.

$^{1761}$他所面對的處境.
不過他就求告耶和華.
亦都看得見或是在呼求上主的回應.
亦都相信上主.
他見到他們所做的一切.
他就去祈求.
或者去哀求.
或等候.
上主要施行的東西.
我覺得是一個很完整的.
由頭到尾.
亦都是一個設計得很好.
66節.
22個字母.
亦都.
去嘗試分.
我嘗試分了四大段.
由他講他自己遭遇的處境.
重新去提醒.
其實.
耶和華是一個怎樣的神.
或者你可以說.
過往.
我知道.
他是一個怎樣的主.
到現在我可以怎樣去回應.
特別是一個.
禱告,一個迴轉的.
祈求.
回望原因.
到最後.
他等候.
好像上文這樣說.
等候耶和華.
亦都哀求求告他.
而主亦都看見的時候.
就等他去出手.
66節的經文.
講出.
耶利米哀歌.

$^{1801}$你記得有五章.
這是第三章,亦都是最長的一章.
其實我相信是整個耶利米哀歌的核心.
去嘗試.
這樣去帶出這一切.
我亦都希望.
今日這個示範.
就是透過.
看到每一個.
每一節,第一個的希伯來文.
嘗試去突出.
特別重複的.
這個他要帶出的訊息.
有些其實.
你單看翻譯.
如果你.
明白他講甚麼.
其實你都會理解到.
不過有些我相信.
去看原文是很值得的.
譬如一些重複的地方.
特別如果你看.
這裡.
三次專登都是用.
因為,三次都是用.
特別這一段.
是很多重複的.
除了這三節.
其他都一定有重複.
我相信不是偶然.
似乎這個作者刻意.
這樣去寫,來突顯.
那個核心.
他要講的訊息.
這些我相信需要你.
對原文有少少掌握,或用這些網頁.
幫到你看深入少少.
明白一些.
譬如Next Bible的筆記裡所講的東西.
以致可以見到.

$^{1841}$這些比較.
要釘針的地方.
亦都嘗試去代入那個.
私人作者想講的東西.
以致.
我們能夠明白,甚至.
可以形容在我們的處境.
能夠去呼應.
希望這個可以幫到大家.
或者我就停一停,交回給.
時間給魯牧師.
聽不到你說話.
你還未開咪.
多謝.
魯牧師今晚的.
有趣味的介紹.
令我想到.
聖經的說話叫我們離開.
道理的開端.
要進入到完全的地步.
其實就必須要下功夫.
其中一個下功夫.
就是要在聖經的原文裡.
有些涉獵.
我們未必是專家,但是這種功夫.
其實是幫助我們理解.
經文是很有作用.
今晚就真是多謝.
魯牧師開了個頭.
好像我剛才所講.
他現在作了第一個示範.
他答應了給我們兩個月後.
做第二個示範.
第一個示範和第二個示範.
都同一個共通點.
就是趣味性.
帶領大家明白.
開始碰一碰.
希伯來文究竟是甚麼一回事.
盼望大家再留意.

$^{1881}$我們的宣傳.
第二次的示範.
同樣可以呈現給大家.
面前.
這個時候我們就向大家.
一個呼籲.
見到中心都很樂意.
服侍大家.
我們的機構.
主要是.
依靠大英節目的奉獻.
這裡已經講了奉獻的方式.
可以到我們的網站.
click in支持神學大學教育事工.
已經可以知道.
一切奉獻的方式.
無論支票或用paypal.
或用其他方式.
都無任歡迎.
很多謝大家今晚.
來到我們當中.
亦都會問一問魯牧師.
剛才他的handout.
可不可以.
有些彩色的版本.
給我們.
如果得到的話.
我們會用email寄給大家.
給大家可以自己.
可以的.
我稍後寄給benson.
另外.
有人問.
如何看到希伯來文.
他說他看到.
幾行,不是我看到的一行.
我想再展示一次.
很快.
你通常進來.
這個網站會看到.

$^{1921}$這個應該是default.
這個其實是.
每一個phrase.
我就沒有用.
我就嫌它阻礙.
所以按了中間一個.
這個最後是銀色.
我覺得有用.
他將動詞變成橙色.
好像夜和華的專有名詞.
就是綠色.
第一個就是這個.
他會和你分句.
分行數.
有時有用,有時沒用.
看這個.
我覺得這樣好一點.
因為每一個字母.
每一個節都是同一個字母.
這樣看會容易一點.
平時我就會這樣.
所以你按一按.
就可以做一行個.
很精彩.
我們大家都努力.
用這幾個網站.
來豐富我們對於.
氣話來文的掌握.
我們今晚的聚會就到此為止.
我們收到.
color coded version.
我們會發給大家.
今晚到此為止.
祝福大家.
有一個平安的晚上.
以後繼續支持.
建築中心.
無論你是奉獻上支持給我們.
或者是禱告支持我們.
我們都無任歡迎.

\newpage



\section{}
\label{sec:efm9yyrZOo0}
\textbf{ABSCC 講座 - 細味‧原文 ─ 簡易希伯來文 [示範 2] (AB2203 POL)}
\newline
\newline
連結: \href{https://youtube.com/watch?v=efm9yyrZOo0}{\texttt{https://youtube.com/watch?v=efm9yyrZOo0}} ~~~~ 語音日期: 2022-06-16
\newline
\newline
\hyperref[sec:L8_DVqUvOSM]{\small{< < < PREV SERMON < < <}}
~
\hyperref[sec:index]{\small{[返主目錄]}}
~
\hyperref[sec:3o4omcoTUB4]{\small{> > > NEXT SERMON > > >}}
\newline
\newline
$^{1}$Selvin你聽到我說話嗎?.
你聽到嗎?.
OK,好.
現在已經是8點01分了.
我代表加拿大的建造中心.
謝謝大家今天進入到我們這個講座.
今天這個講座.
承接大約兩個月前.
同樣的講員李鴻義牧師講過一個.
用希伯來文做解經的講座,今晚是一個示範的第二講座.
很歡迎大家到來我們這個講座.
我相信有些弟兄姊妹是上一次有來過的.
你這次又再來,顯示你是很有心的.
能夠再進一步了解希伯來文對於我們的解經有什麼好處.
亦有些弟兄姊妹可能是第一次來聽.
我鼓勵你去YouTube我們的頻道.
ABSCC.
你打ABSCC.
你會找到上一次的講座.
同樣也是李牧師.
給我們講論的.
當然今晚.
這個聚會亦會錄影了,明天會放在YouTube.
無論如何都是非常歡迎大家.
相信用希伯來文解經,很多人都覺得.
好像很難,或者不知道怎樣著手,所以我們特別請李牧師.
能夠用兩個示範.
給我們一些鼓勵.
先去除我們一些恐懼感.
以致到大家有機會.
再一次進深到.
舊約聖經的原文.
這一方面的東西.
大家報名的時候,你也看到我們有問到你們一個問題.
假如有足夠的人有興趣.
我們開一個班.
學習初階希伯來文.
初級希伯來文.
有收費的.
是不高的,一個短短的課程.

$^{41}$我們發覺很奇怪,原來也有很多人有興趣.
我們會在適當的時間.
再研討在什麼時候推出.
我們也要問問李牧師,什麼時候方便才行.
所以.
大家也要留意我們的網站,剛剛已經寫了.
www.abscc.org.
看看推出的日期是如何.
以及一些詳細的資料.
我們很希望大家能夠踴躍支持.
我們這種effort.
我們的effort也是很簡單.
希望能夠幫助評論圖.
能夠在聖經的原文方面有更多的追求.
現在我們做一個簡單的祈禱,請大家一起雞頭.
我們仰望你.
今晚的講座.
能夠順暢的傳播,你使用李牧師在領袖上.
給我們有眾多大英雄有學習的機會.
願主你自己祝福.
他的講論,也祝福聽的同學.
因為可能有些同學身體疲倦,可能日間上班.
到現在晚上.
聽課的時候也有些困難.
願主你自己繼續引領,繼續祝福,我們簡單禱告,奉主明求.
我將時間交給李浩然牧師.
再次邀請.
有機會跟大家分享.
第一,我們不要太緊湊.
課堂不要太緊湊.
如果你上次有來,不妨在這個channel寫個hi.
讓大家可以有些.
互動.
在這個時候.
我想.
講幾件事,第一就是.
今次我們嘗試.
看看最後能否有幾分鐘的時間.
可以跟大家有一個.
Q and A的時候.

$^{81}$如果你在課堂上.
你有什麼問題,不妨寫在chat.
我希望.
未有時間.
看一看.
嘗試回答.
你也要明白,時間所限,有些比較複雜的問題.
未必完全能夠幫到大家.
如剛才盧牧師所說,將來如果有一個課程.
可以在到時的時候.
回答和處理.
我希望.
不只是我去講.
就算時間上大部分都是我講的時間.
希望大家可以透過這樣的.
對話和交流.
文字上也好.
起碼我看到你.
如何去想,有什麼掙扎,有什麼挑戰.
這是對學習來說.
很重要.
也希望可以幫到大家.
如果上一次有來,你可能也會有少少.
印象.
重點在哪裡?.
今天.
和上次有些不同,上次我們用這個《源米哀歌》第三章.
如果有上過,你就記得了.
重點是因為可以透過希伯來文.
你看到每一行.
字母詩的第一個字是哪一個字.
從而看到一些可能在翻譯上.
很難的,跟不上原文次序.
比我們看不到.
作者想透過字母詩有精心的設計.
來表達.
帶出來的重點.
今晚有少少不同,我重返《創世記》第六章一到八節.
一個短的經文,可能也不是最熟悉.
因為也比較.

$^{121}$不知說什麼,通常這些可能.
聽過,不過可能跳過的一些經文.
我希望.
透過這個經文做兩件事.
第一.
認識原文的重要.
我有幾個例子,在這段經文裡.
給大家去看看,原來是需要用.
對原文.
有個理解和認識.
才能夠明白到.
不過這個.
我要再三強調.
如果我用.
一般的.
OIA.
嚴經的概念.
很多時.
我們想著看完原文.
我們就很快跳到A.
應用.
我看完原文,我就找到答案.
我可以更加.
將那個.
經文要講的東西,因為我認識原文.
我就可以很快地畫出來.
這個其實是一個頗大的誤解.
我.
會這樣理解.
認識原文.
其實最多.
最能夠幫到你的.
其實是一個O.
幫你observe.
觀察更深入.
平時在一個.
譯本.
甚至乎你用幾個譯本做個比對的時候.
都未必能夠見得到或者可以確定得到.
的一些訊息.

$^{161}$是observation.
觀察.
有些時候.
OK.
都未必是.
兩成三成的時候.
可能.
你會.
因著你對原文的認識.
對那個interpretation.
對那個解釋.
可以有確定多少少.
但是.
我很想強調的是.
其實你去認識原文.
你看回原文.
去明白那段經文.
其實最能幫到你的是observation.
有時我們要.
自我收斂少少,有時跳得太快了.
去應用.
停下來慢一慢.
首先的.
幫下自己.
可不可以看到一些.
平時.
我當大家看中文譯本或英文譯本.
未必能夠看到的東西.
這個是第一點.
帶著這個心態.
我相信.
今晚的示範.
或者將來你繼續學習希伯來文或者希臘文也好.
其實是會幫到大家.
不是直接去到應用.
而是幫我們.
豐富我們的觀察.
觀察一些平時看不到的東西.
第二.
亦都好似上次一樣.

$^{201}$幫大家用一些線上的工具.
在這裡.
我會將.
上次.
同一個表來的.
裡面一些我覺得比較.
有用的.
website.
給大家參考.
我可以在這裡.
都展示給大家看.
這個都很重要,因為很多時候學習.
原文.
如果我沒有這些工具.
其實是漫長的.
我忘記上次有分享過.
我們以前在神樂院.
去翻譯一節經文.
要大約一個小時.
今晚八節.
就要八個小時.
今日上班的日子就是在這裡看原文.
但在我們這個世代,因為有很多的網站.
上次說過有它的.
不足的地方.
不過.
這麼多的網站.
resources在這裡.
其實我們很應該善用.
當我們善用的時候.
其實是.
幫到我們.
特別.
我們強調我們不是要從 scratch.
重返原文去翻譯.
我不是去做一個宣教.
的.
的事,去做翻譯員.
我們未必是要這樣.
我們更加需要的是幫到我們更加觀察到.

$^{241}$一個平時可能在譯本裡面看不到的東西.
所以我們其實可以善用這些工具.
去幫助我們.
我今日都會透過一些例子做一些示範.
你可以回去繼續.
鼓勵去用,你去看聖經的時候.
除了你最喜愛的幾個譯本之外.
你都可以去.
運用這些.
工具網頁.
去看一些.
原文本身和原文背後的參考資料.
我們就開始進入今晚的課題.
今日主要和大家看一個.
世紀第六章一至八節,我將和合本和.
和修訂版.
放在一起.
黑色的是和合本,藍色的是和合本修訂版,簡稱和修.
我將兩個並排.
想給大家看看.
因為兩個翻譯.
比較.
和修是根據和合本做一些.
修改.
所以.
這樣做可能會更加突顯到一些.
和合本修訂版裡面的編輯,看到有需要.
刪減的地方.
我們從一個這樣的參考去看一看.
讓你有多一點看到原來有這樣的不同.
我從而進入原文裡面看看.
究竟其實.
看到原文可以幫助我們更加看到.
一些什麼東西.
我會讀一讀.
這段經文,我會讀和合本,可以參考和修.
來看看不同的地方.
在世紀第六章一至八節,第一節開始.
當人在世上多起來.
又生女兒的時候.

$^{281}$臣的兒子們看見人的女子.
美貌.
就隨意挑選.
取來為妻.
耶和華說.
人既屬乎血氣.
我的靈就不永遠住在他裡面.
然而他的日子還可到一百二十年.
那時候有偉人在地上.
後來臣的兒子們.
和人的女子們交合身至.
那就是上古英武有名的人.
耶和華見人在地上罪惡很大.
終日所思想的盡都是惡.
耶和華就後悔做人在地上.
心中憂傷.
耶和華說.
我要將所做的人.
和走獸.
並昆蟲.
以及空中的飛鳥.
都從地上除滅.
因為我做他們.
後悔了.
惟有羅亞在耶和華眼前蒙根.
如果你聽我讀的時候,再看一看.
你可能都看見.
大部分都是一模一樣的.
但也看見一些地方.
有一點不同.
看見有不同的地方是一個.
很好的事,你就可以盡心去看.
為什麼有不同.
特別如果你有興趣看原文.
那是一個很好的位置.
去看什麼令到.
翻譯有不同之處.
不過時間所限,我沒有可能.
將短短八字經文.
全部.

$^{321}$逐一去解釋.
裡面亦都.
牽涉到很多.
不單止.
釋經的問題.
而是背後一些神學的問題.
其他範疇都會.
關乎到.
所以今晚我只能夠舉.
6個例子.
來講講.
也去凸顯.
看見原文.
可以看見.
在.
平時翻譯上看不到的地方.
第一個.
剛才你見到.
和合本和和修.
有少少不同,一個是地上.
地面上第一節.
之後大部分地方都是地上.
除了第七節.
和修.
用了地面上.
很少.
出入的地方,但如果你仔細看原文.
原來都幾.
不同.
原來用了兩個字.
兩個.
看上去不像樣的字.
再提一提.
希伯來文是由右到左的.
像舊式中文,由右到左.
所以現在有英文,為何要放英文呢?因為這個字體.
不是最方便,我出到希伯來文,出不到中文,所以.
我把翻譯用英文.
是.
ground, soil.

$^{361}$剛才和修是地面.
如果你記得,這個是說.
亞當.
他是如何被造的.
從塵土而出.
塵土是另一個字,但塵土是在地面上.
在地面上的塵土.
而造出來的.
那就是at the mark.
如果深水清.
和亞當,Adam那個字.
是很.
關聯的.
你可以說是同一個字根.
所以你見到兩者,亞當.
和地面.
似乎有一個比較密切的關係.
第二個字,Eros.
Earth.
就比較普遍.
其實在創世記一章一節已經有講神創造天地.
就是Eros.
兩者有什麼分別呢?有什麼關係呢?.
我看到.
如何呢?.
我想在這個例子裡面給大家.
一個小小理解.
不過首先我想給你看看,我怎樣可以知道是兩個不同的字呢?.
我給你.
用一個工具,你有興趣可以即時去開.
我主要就用.
BibleBentel.
你打BibleBentel.com,你就會見到.
你去BHS,即是那個.
黑白論文的聖經簡稱.
你就會見到一個叫BHS.
和合本.
你就可以開到,就見到這個網頁,當然要選擇創世記第六章.
這裡有三個加號的.
OK,什麼來的呢?.

$^{401}$第一個,其實是將它變成分開一句句.
我又喜歡.
現在這樣的,即是分開做.
紙句.
容易看一點.
還有它的分白.
有時候我未必最認同.
不過大部分時間,起碼幫到你容易處理那段經文.
就不是.
一句那麼長,你就不知道怎樣.
剪開,不知道怎樣停.
這個頗方便的.
第二個,我通常會弄走它.
因為其實是說一個,它是一個什麼的紙句,什麼性質.
這裡很學術性或者grammatical.
我就沒有什麼需要,我就弄走它.
第三個,其實是將動詞highlight.
或者一些專有名詞,例如葉和華的名字.
我覺得都好,方便你去.
locate,認一下位置.
你可以玩一下.
你去見到這裡的時候.
你會說,我不懂看原文,完全不知道是什麼來的.
沒問題的.
我覺得這個網站最理想,最好,最幫到的一個東西.
就是它.
highlight,或者你放一隻老樹在中文字的時候.
它就說相關相對的.
希伯文是哪個字.
這個很好的.
功能.
不過有兩個問題.
第一.
有錯誤的,有些地方我弄錯了,不是我覺得,是弄錯了.
我覺得這個你要.
要知道,始終是免費網站.
第二.
就是.
有時候.
他怎樣去.

$^{441}$弄呢.
譬如在他裡面.
其實是在.
在.
亞當裡面.
或者在.
我一會處理,在哪一人裡面.
他這個只能夠基於和合館的翻譯,去告訴這個希伯文等於.
那個和合館是哪個字.
不過.
我不知道他是.
設計網站的人的中文程度,我不知道他的來源是哪裡來的.
所以其實有些地方是會有.
這樣的.
錯誤.
所以.
不能夠.
怎樣說呢,完全.
盡信.
你要自己再去做一些.
參考.
或者進一步的.
verification.
有什麼方法呢,不如多說一個網站.
這個呢.
大部分時間.
他也會.
準確一點,信網外.
bible.fhl.net.
那你.
進去,你可以選不同語言的.
我不會詳談了.
我想給你看到的是.
他的翻譯.
就.
多數時候會準確一點.
為什麼呢.
因為其實他是用.
一個.
字典的模式.

$^{481}$展示給你看.
他就沒有像剛才那樣,你去原文.
highlight了他,然後他就展示給你聽那個和合館是哪一個字.
不過你可以用這個做一個參考,原來這個字是這樣解釋的.
譬如.
下單.
00120.
是解釋人的,那為什麼那邊會在他裡面呢,是不是有些問題有些不同了呢.
這些你可以去問,或者你未必知道答案,但是你知道.
原來會不會是那個網站未必做得最好,那你check下另一個網站.
多數時候.
你可以find out到.
那個答案的.
這個網站最大的挑戰是.
他有個原文直譯.
不過問題是.
有時候你在上面看的.
中文翻譯.
和下面這個字典用的字是不同的.
那你有時候就map不到.
你不知道那個原文.
在上面.
是哪個中文字.
下面他給了你,不過找不到那個字.
這個是這個網站的短處.
不過我多數會鼓勵學生.
將bible bundle和信望愛兩個比對一起.
通常.
那個答案都.
你會見到出來的.
我們回到正題.
剛才我們為什麼會知道是兩個不同的字呢.
因為如果你去到再細上.
這個highlight錯了.
其實是這個,他變了一個星星符號.
不知道為什麼.
其實是細.
即是.
地上.
或者地面,所以我收譯了做地面.

$^{521}$他是.
The map.
你見到.
他這裡的highlight.
你見到那個希伯來文.
由一到第八節.
他只是.
見到.
兩次.
地上這個字,還有多過兩次,所以你會知道似乎不是同一個字來的.
當你去到第四節.
你見到在地上的時候.
你見到就出現了IRIS這個字.
我看不明白希伯來文的鬼下符號,你認number.
OK.
剛才H.
127.
OK.
這裡下面.
H776.
即是不同的了.
通常將不同的字.
他將每一個的生字.
他附了一個number,所以見到不同number.
你可以知道不同了.
最好最理想.
看完那個原文.
起碼學了一個希伯來文的字母.
最好連響音.
的符號你讀.
這個就會幫到你.
這個可以在將來的課程有機會去學習.
你大約認到,看到鬼下符號,這個和這個不同的,你就知道原來是兩個字來的.
你見到這個IRIS出現了三次.
在第四,第五和第六節.
和合本都是做在地上的.
在字其實是個preposition來的.
五暢談.
我想給你看到.
原來是兩個字有不同.

$^{561}$為什麼有不同有什麼重要呢?我知道又如何呢?.
如果你將.
剛才我們找到這個資料.
即是說.
adama.
出現的地方.
你去看看,你就會發覺是一個很特別很精心的安排.
記得adama這個字出現了兩次.
第六章一節和第六章的七節.
IRIS出現了三次.
第四,第五,第六節.
好像三文字一樣夾住了.
特別是這個經文有八節.
基本上就真的好像一個三文字一樣.
開頭結尾.
是adama.
IRIS.
中間.
可能是這麼巧而已.
有時是的,你不知道的,不過你找到這些資料.
將它們湊在一起.
去到尾.
可能圖畫就更加清晰.
這些是功課,這些是功夫來的.
我會鼓勵大家如果想盡心去.
看一段經文,這是一個很重要的過程.
都需要時間.
但是.
收穫是豐富的.
未必即時.
不過是會幫到你.
去看到一些.
絕對在翻譯上看不到的東西.
因為如果你記得.
我們在.
這個.
和合本和和修.
你見到他用了.
和修好一點,他用了開頭結尾用地面上,中間用地上,和合本你就會.
很confuse,因為在世上下,再從地上.

$^{601}$上面那三字的出現是一樣的,他就給不到你看原來.
那個.
用字上的不同之處.
第一個例子.
用希伯來文.
你可以見到究竟是不是用同一個字.
我相信這個可能大家聽過.
很多時候,有些目者在講道的時候都會用這個方法.
一個bible study.
我見到.
這個字和這個字的不同.
會幫助到.
因為很多時候在翻譯.
為了要我們明白.
或者一個.
出於那個的理解.
用了的字.
未必會.
同一個字未必翻譯成同一個中文.
不同的字可能翻譯成同一個中文.
這就是一個例子.
所以看到原文的字.
絕對幫到.
他明白多少少.
第二個例子.
我就想.
其實有時候.
你發覺可以看原文是用哪個字.
跟著做一個.
是不是同一個字來的.
做一個比較.
不過.
其實有.
一定的限制.
為什麼呢.
因為很多時候.
他.
那個怎麼說.
他可能未必用同一個字.
但是用同一個字跟.

$^{641}$這個就是那個挑戰.
我這裡有一個例子就是這樣.
在六章第一節.
他是說.
當人在世上多起來.
多.
多起來.
即是動詞.
這個叫畫花.
是一個word.
to be numerous.
多.
所以你見和本和修.
多起來是真多.
你見呢.
你說我看不懂,但起碼有三個字母.
你可以在這裡找到的.
他會show給你看有三個字母,這個是他查字典的時候那個.
那個index.
但是呢.
他第二次出現第五節.
這裡他也是.
第五節.
他用了.
一個形容詞.
Rough.
同一個字根.
但是.
是不同的字.
所以如果你說純粹我查.
是不是同一個字來的,你會發現不是,你就會miss了.
純粹看翻譯也看不到,一個字做和,畫本很大.
這個和修也做惡極.
同上面是不同的.
中文字.
但其實.
背後.
其實是一個相關的字.
不是.
exactly 一樣.

$^{681}$但是.
相關,一個動詞一個形容詞.
兩個其實都是.
多.
或者.
根據形容的原因.
下面的做大.
惡極.
理解的.
罪惡多.
不是代表嚴重性.
這是翻譯上的限制.
翻譯要讓你明白.
但是你看不到.
原來那兩個字.
相連的地方.
surprisingly,這個bible bundle.
竟然把這兩個放在一起.
比較少有的,多起來.
下面的很大,也highlight了.
不知為何他出不到.
他看到.
希伯來文上面.
Rove.
下面.
畫花,兩個都highlight了.
可能你會問,是不是同一個字來的?.
雖然是不同的.
因為這個是.
動詞,7231.
7227形容詞.
不過這個就是.
要學了原文才能夠overcome的.
地方.
有什麼辦法可以.
入到去明呢?.
有時可能你就去.
進入字典,或者.
在信網的網站.
你可能去看他的原形.

$^{721}$翻譯.
看看他會不會.
幫到你.
給電源.
兩個字相連.
有時有些字典有說,有些情況沒有說.
有些study note有說,有些沒有說.
這個也要看情況.
不過我想在這裡introduce一件事.
原來希伯來文.
他和我們平時.
中文就沒有這個concept.
英文有少少.
不過都not exactly the same.
因為原來希伯來文用的root,字根.
這是另一個例子.
他這個叫Mallet.
Mallet.
是一個和皇帝有關的東西.
他的root,字根是這樣.
然後.
他說.
皇帝.
是一個名詞.
皇后.
Malacca.
就是一個.
將他.
音性了,就變成皇后.
另外這個.
Malcute就是.
這個kingdom.
和皇帝有關,是他統治管治的地方.
都是noun,都是名詞.
但是同一時間,也還可以有一個Mallet.
就是動詞.
to reign,作王.
你見到全部這四個,其實還有其他.
都和這個字根.
是關聯的.

$^{761}$這個是希伯來文的一個特性.
他一個字根.
可以演變出不同的動詞名詞.
相關.
related.
但不是同一個字.
很多時候.
我們去看的時候就未必會.
第一,在翻譯本上未必看到.
第二.
就是就算你.
查字典,上網找我找一個.
1560的number.
都收不到,因為他編了不是同一個number.
你真的要對原文.
盡心了,你譬如在我這個case.
Rath是一個動詞.
原來Rath.
畫花是一個動詞,Rath是一個形容詞.
兩者其實是出自同一個字根的.
你才能夠看得見.
原來中間的關聯.
這是第二個例子.
我想說的是原來.
在.
純粹靠字本身.
是很有限的.
只是看到很少的一部分.
你能夠要再看深入多一點,看到這些情況.
你需要對原文多一點認識.
或者起碼那個system多一點的理解.
才能夠做到.
我們下去第三個例子.
還有另一個情況.
這個更加是和原文.
希伯來文一個獨特的地方.
在第六章一節又生女兒的時候.
和第四節.
這個.
神的兒子們和人的女子們交合.

$^{801}$OK.
生了孩子.
看下去.
生女兒生孩子.
一樣都是一個動詞.
都是說生了.
小朋友.
不過如果你看回原文.
其實兩個.
是有.
一個很細微的不同.
如果你看他的.
冊字.
一個叫.
Cal Passive Perfect.
3CP,今天不詳細說了,沒時間去.
交代所有東西了.
課堂可能才有機會.
我只想給你看到.
一個是叫Cal Passive.
一個叫Cal Perfect 3CP.
3CP一樣,除了一個是Passive,一個不是.
單單從這個你可能已經猜到.
其實上面那個是一個被動式來的.
下面.
是主動的.
換句話說,其實更加正確的翻譯.
好像Next Bible那樣說,上面那個是叫Daughters were born to them.
女兒是.
被生了,中文不會這樣說,所以沒有人這樣翻譯的.
女兒是生了出來的,給他們的.
有少少這樣的意思.
中文真的看不到這個.
這個是你要明白語言上的限制.
這個你可以參考.
特別英文,因為英文有這個概念多一點.
但是同一時間你都不知道哪個是對哪個錯.
所以其實是很需要去原文那裡.
去理解的.
我不在那個.

$^{841}$Bible Bench上展示給你看,你可以去六章一節和四節那裡看一看.
這個字的動詞.
你就看到原來他展示兩個這樣的東西.
你說什麼是Cal,什麼是什麼,很快看一看.
希望有一個這樣的.
簡單的概念.
他有分.
主動.
被動.
還是反動.
那個做動作的又是你.
又是自己身上.
那個叫做反動.
然後有.
簡單的動詞.
有一個.
加強的動詞.
有一個.
使役.
的動詞.
他就分開了.
上面的圖表.
基本上.
八大學.
還有其他.
微細的其他的類別,主要的.
常見的.
這八大類.
下面這裡用英文翻譯,中文比較難,英文翻譯了,你就大約看到了.
這個.
Cal最簡單的主動就是He heard,他聽見.
Napal就是他被聽見.
Pial.
就是那個加強.
用了另一個字,所以不是和聽見沒有關係.
他是說.
打碎.
但是smash into pieces.
打得很碎.
如此類推.

$^{881}$使役的.
他就是.
不是自己作王.
是他使得某人作王.
OK.
這些例子.
亦都給你看到原來.
同一個動詞.
其實可以有不同的解釋.
是因為他用了.
字幹.
字幹不同了.
你可以說這是一個希伯來文很獨特的語言系統.
連英文都沒有,所以難理解,這是其中一個.
希伯來文最複雜去學習的東西.
但是.
當你明白了,你就明白.
原來為何要用同一個動詞.
為何翻譯不同.
剛才的case.
都看為簡單.
因為剛才的case.
純粹是.
一個是被動.
一個是主動.
其實他那個.
很多時的情況.
如果他是一個Cal.
一個是一個Pl.
可以完全不同的意思.
所以這個是很多時.
一個沒有.
盡心去明白.
希伯來文.
的人會最容易出錯地方,同一個動詞來的,所以解釋可以由這裡搬去那裡.
原來是不行的.
因為他用了不同的字幹.
Stem.
這個是.
亦都在翻譯上面.

$^{921}$很多時未必會.
見到,有時翻譯他會做到.
不過很多時他會用另一個翻譯,所以你亦都未必會.
知道原來那兩個.
字呢.
其實是同一個.
動詞的字.
字根.
但是因為因著他的字幹不同.
他演變出來的意思.
變得很不同.
這個是需要明白.
但其實很重要的,因為當你知道的時候,你就會見到那個.
中間互相的關係.
的地方.
OK.
這個可能複雜一點,不過我想.
簡單.
其實不單止剛才說.
那個.
你純粹明白那個希伯來文的字.
是哪個字,找他是不是同一個字.
但原來亦都有一些情況.
你發覺.
是因著.
一個動詞,一個形詞.
一個名詞.
以致到.
不同的字.
但原來是同一個.
字根.
但亦都有時候因著他是同一個動詞也好.
因為他的字幹不同.
原來那個意思.
又可以有不同.
但又相關的.
這個就是在希伯來文的一些.
比較複雜的地方.
好,然後去.
第四個例子,我想給一個.

$^{961}$再闊一點的層面.
剛才說的都是主要環繞在希伯來文.
當你明白了希伯來文的時候,你可以.
知道多一點.
那個資料,觀察上你可以豐富多一點.
但我有另一個例子想說的.
其實原來.
去到尾,為什麼我剛才記得.
很多時候都是.
只能夠.
豐富你的觀察.
但當你說,我是真的想去.
解釋甚至應用.
其實有去限制.
純粹一個grammatical understanding.
對於原文的理解,其實未必可以給到答案.
以下就是一個例子.
神的兒子們,可能大家都好curious,什麼意思來的呢?.
這個原文能夠幫到你的,是以下的東西.
Bernet Elohim.
OK,朝曰.
The sons of God.
很簡單,神的兒子們.
但是第一.
Elohim.
後面那個字.
其實本身.
如果從一個.
文法上.
是一個眾數來的.
但是.
他很多時候.
你見.
在聖經裡面看到他的.
翻譯.
其實是用了.
大草G.
God.
單數.
因為他還要配合一個單數的動詞.

$^{1001}$這個是希伯倫怎樣說神.
這個不是耶和華的字,這個純粹是神.
神明.
所以其實.
純粹Elohim這個字.
其實可以譯作.
大草G,God.
或者細草G,God.
假神,神明.
這個是要知道.
通常.
你從那個動詞.
是眾數還是單數,你可以分辨到.
多數時候,或者從上下文.
但是有時候都複雜,因為他有一個叫做.
Grammar上他就叫做.
Majestic plural.
即是因為他很豐富,他很偉大.
這個這麼偉大的神.
他用了一個複數來表達,甚至有些時候他用複數的動詞.
所以都是要看上下文.
不過這個純粹讓我們看到就是,究竟那些掙扎.
要譯作The sons of God.
還是The sons of God呢.
其實給不到答案.
純粹希伯倫給不到答案.
你可以甚至說他有時刻意給一個這麼ambiguous的選擇,讓你掙扎.
能夠怎樣呢?我就給一個例子,其實很多時候.
看原文也好,其實你需要其他的工具來配合.
以至你能夠盡心明白那個經文.
我非常鼓勵的是看.
Net Bible的筆記.
如果你手上有的,你去開一開.
我這個是從.
創世記第六章.
OK.
然後呢.
就在這個.
第二節.
你按出來,你就會看到.

$^{1041}$我一會兒會讀一讀給你聽.
不過我想說.
這個是中文版的Net Bible的筆記.
不過.
中文版的Net Bible的筆記.
其實是.
原裝英文的那個,其實會豐富一點.
齊全一點.
有些筆記會沒有了,有些會.
削短了,所以我鼓勵你如果看一看英文.
去英文那個.
英文那個,他就會齊全一點.
會好一點,這個是中文的限制.
不過我會用中文來做一個例子.
他在說什麼呢?我把這個筆記削了出來.
他說神的兒子對新索哥.
原文.
片語.
就是.
只出現在本著六章二節四節.
和約伯記.
那幾個經文.
這片語有三個主要見解.
可能你都聽過了.
一,在約伯記裡面清楚地指明是天使.
創世記第六章神的兒子明顯與人類有別.
建議這些並非人類.
因為他有神的兒子.
人的女子們.
所以他在這裡說.
大家有分別.
兩書的用法.
三字.
本著技術.
及.
與女人同居.
與這些神的兒子.
門必是以形體出現.
或具有男人的身體.
錄於舊約.

$^{1081}$《次經二諾一書》.
六七節.
早代猶太傳說.
詳細描述這場天使的叛變.
甚至提及和守的名字.
即是重返希伯來文.
《耳經解經》.
約伯記或《次經》.
裡面都有提及一些這樣的.
用法.
Sons of God.
即是天使.
或者.
有些可能再闊一點,Divine Beings.
OK.
這是第一個理解.
第二個,不是所有學者都同意神的兒子是天使說法.
有些變稱神的兒子.
是.
瑟特一物.
亞當其中一個兒子.
從亞當上賢至神.
在第五章中這樣說.
而人的女兒就是該人的後裔.
不過正如前述.
本著經文明顯把神的兒子.
與人類分開.
而人類亦包括瑟特與該人的後裔.
有見人的女兒是一般的女人.
不僅是該人的一體.
因為他所說的.
上文沒有.
提及.
人有分開兩大類.
一個是瑟特.
一個是該人的後裔.
所以.
似乎不是最.
解得通.
不過的確有人這樣理解.

$^{1121}$第三個.
另有說法說神的兒子是勇武的霸王.
可能是鬼父.
他看自己是神.
又以拉密.
為榜樣神多妻的婚姻.
不過約伯記對神的兒子的使用否定了這個觀點.
即是可能說他是一些.
好像.
六章四字裡提及.
即是.
很強.
很有能力的人.
或者甚至是.
皇帝.
權貴.
即是他們自稱為神.
神的兒子.
的確古人近代都有些這樣的.
去看.
一個皇帝.
究竟哪一個是.
正確呢?你純粹看下一個禮物.
他的preference.
就在約伯第一個.
見解.
神的兒子就是天使.
他給了一個參考.
Gordon Bramham.
一個出名的神學家.
聖經學家.
在WBC.
一個.
比較.
technical的.
聖經書.
裡面.
會有詳細的教導.
我想說的不是要解答這個問題.
究竟神的兒子們.

$^{1161}$是哪一個?你可以自己慢慢再去想.
選擇.
但我想說的是.
希伯來文.
你純粹看了那個原文.
答不到你這些問題.
他可以當然就好像他第一個選項.
比較其他的經文.
在那裡有這樣用.
可以.
但等不等於.
那個是一定的答案呢?.
是否符合在這裡常客文的理解呢?.
或者甚至乎.
我們看古代更多其他.
參考的文獻.
會不會.
幫到呢?.
可能是可能不是.
我想說的是.
其實原文.
純粹看原文.
不要以為他給了你所有答案.
他幫到你的是.
知道.
原來.
The sons of God.
究竟他在用什麼字.
你可以找到有沒有類似的用法.
不過在這裡.
你只能夠發現幾處.
這個片語的運用的地方.
再下一步你要怎樣呢?.
你只能夠.
去讀一些參考書.
那些才能夠幫到你.
這些是進入一些神學的問題.
討論.
我想給你看到的.
原文都有自己的限制.

$^{1201}$OK.
第五個例子.
另一個呢.
剛才說到神的兒子們.
另一邊就是人的女兒.
但我不知道有沒有看到.
其實這個人.
的XX或者關乎人這件事.
其實有.
兩個不同的.
表達.
出現了.
不過呢.
這個基本上.
起碼在這一章裡面.
翻譯就.
看不到.
為什麼呢?因為他是一個叫做Adam.
OK.
通常是叫做Adam.
或者人.
或者再統一.
就是一個Human kind.
不過同一時間.
這一章裡面其實有一個叫Adam.
其實是一個Demand.
Human kind.
因為是一個貫詞.
什麼來的?很簡單看一看.
他這個叫做.
是.
我先看一看.
Definitive article.
其實他.
加了一個東西在前面.
這個也是希伯來文的一個特色.
加了一個東西附屬在一個字的前面.
這個叫做Hey.
你當.
通常翻譯的.

$^{1241}$Transliteration是H.
你當有一個H在前面.
戴了頂帽子.
你翻譯一下那個符號.
這個.
他下面有一畫.
一個Short a的Wild sign.
再加一點.
他將這個放在.
原先那個Noun.
名詞的前面.
就是一個對字.
跟我們平時理解英文很不同.
可以黏在一起.
而這件事.
給一個例子.
剛才那個皇帝.
The King.
就是這樣,他將它加在前面.
這件事原來在這裡發生了.
他基本上全部的出現.
都是有一個對字.
除了.
在第七節第二次的出現.
是沒有了對字,沒有了貫詞.
有什麼分別呢?有什麼關係呢?.
就這樣看回.
翻譯本.
和俠本和修號.
其實沒有的.
他也是在說.
人啦,他沒有show出這個情況.
人啦,其實每一節都有出現的,除了第八節.
然後就.
偉人,這個不是的,不要亂說,其實是Nephilim.
是另一個名詞.
這裡是說人的女子們.
第五節.
見人在地上.
後悔做人.

$^{1281}$所做的人.
這裡出現兩次,不過他沒有譯出來.
我想說的是.
其實你在翻譯本裡面看不到.
在原文你才看到.
但是.
許多人不是譯成了那人,是嗎?.
那有什麼關係?.
和我的經文理解有什麼.
分別,會產生什麼不同的情況呢?.
我這裡想說另一個情況.
其實很多時候我們看一個經文.
甚至我用原文看一個經文.
也好.
一個重要的關鍵.
是你要去看上下文.
走不去.
不要以為看完原文就不需要,這個反而更加要擴闊了你去看上下文的需要.
因為在這個情況.
其實你是需要看.
Adam.
或者夏丹.
有沒有貫詞的用法.
throughout整個.
不要說創世記,起碼創世記第一章到第十一章.
他的用法.
是一個過程來的.
你要找出所有,如果你有一些software要付錢的,那就容易很多.
純粹網站.
也很痛苦的一個過程.
其實是需要的.
這樣你才能夠理解,我為什麼會這樣說呢?.
我給一個Netfiber的註腳來看.
他在.
創世記第一次出現.
Adam這個字.
一章26節.
他是這樣寫的.
人,Mankind,Habermann,Adam.
這字有時用作指男人,本處指人類.

$^{1321}$包括男和女.
為什麼會這樣呢?.
單數式顯然是統稱.
參下半節,使他們管理,原來27節.
是用了一個.
重數,他們.
所以第27節特別聲明神做男做女.
所以他.
起碼Netfiber的理解.
26節不是說Adam.
而是說人類.
上帝.
來按著自己形象做人類.
創世記第一到.
十一章其他經文都是同樣用法.
五章二節並且做男做女.
神賜福給他們,稱他們為人類.
其實人類就是Adam的字.
翻譯成做人類.
但是因為他做男做女,神賜福給他們.
所以.
似乎翻譯上應該譯成做人類.
這名詞在六章一節,我們剛剛看到五至七節.
九章五至六節都是用作指人類.
其實中間.
二三四節都有出現這個字.
那裡呢.
主要是說.
人的類子們.
似乎.
照他這個理解.
都是在說統稱人類.
我想給你看,不一定是說Netz Bible說,那你一定是譯成做人類.
不是沒有掙扎的.
都是一個很值得去探究的問題.
不過我最想指出的是.
原來.
就算看原文也好.
不能夠只是停在第六章.
雖然我們看第六章一到.

$^{1361}$八節.
我們需要去看回上下文.
特別如果你說.
在這些情況下.
這個作者.
何時用這個寫法,何時用另一個寫法.
為什麼會有這樣出入.
是不是有一個東西想表達.
這個又是一個.
即是.
當你學了原文.
可能你的責任就大了.
更加值得.
拿著對原文的認識.
盡心去.
進入經文裡面.
去明白去理解.
不能夠只是停留在.
那一字是這樣用的,那個字是這樣解釋的.
所以.
這句一定是這樣解釋的.
其實是需要看回很多其他東西.
好,最後.
我想給大家.
可能你會問.
看完五個例子.
和我這個經文有什麼關係呢.
好像都是一頭煙.
我希望給大家一點安慰.
也有幫助的.
所以最後用一個例子.
羅亞.
羅亞這個名字.
他可能會給多一點.
我們.
潔亮光,看到.
當我們能夠看到原文的時候.
或者可能對.
能夠觀察得到那個主題.
就可能.

$^{1401}$幫到多一點.
或者起碼你明白.
有些適應學者帶出來的時候,你知道他在說什麼.
Noah.
英文,即是羅亞用中文.
其實是一個音譯,所以你會讀的.
其實在.
第一次出現,五章二十九節.
他有說的.
他叫Comfort.
為什麼呢?.
他說吸他起名叫羅亞.
說.
這個兒子必為我們的操作和手中的勞苦.
安慰我們.
就是這個.
The harm.
那字你見了.
你不懂,你只能認,但一個是.
N.
和一個H.
一樣的.
最後一個M.
一個M.
有時是這樣的,他就拿他那個.
大部分的字,或者相同的音.
然後呢.
通常的名字他都有交代,為什麼要叫他這個名呢?.
那個名字和.
他想帶出來的訊息有什麼關係呢?.
在這個case我們見到.
他叫他做羅亞.
他的父母叫他做羅亞,因為.
他希望.
他能夠為我們的操作手中的勞苦.
可以帶來安慰.
這個操作手中的勞苦是因為要就坐地,這個其實是地,The mark.
你見到,原來在這個這樣的情況下.
他幫他改名做羅亞.
他希望得到安慰.

$^{1441}$這個是羅亞本身的名字,起碼是他父母安他這個名字的時候的意思.
同一個動詞呢.
就出現在六章六和七節.
不過你見到他不是譯作安慰,而是譯作be sorry.
後悔,說上帝後悔.
為什麼會這樣呢?.
這是另一個例子.
剛才我們很弱的說過,你記得吧.
因為這個上面叫PL.
即是一個加強的.
動詞,加強主動.
下面那個呢.
是一個.
叫Nivelle.
就是這個簡單.
但被動.
一個簡單,一個是加強.
你就見到其實動詞的意思很不同.
一個是主動,一個是被動.
所以上面那個叫安慰.
下面那個呢.
就是後悔.
詳細的解釋,或者中間的關係,你去字典那裡可能會看到多一點.
不過我想帶出的是同一個動詞,同一個字根,不同字幹.
這個值得留意.
上帝說他.
後悔了做人.
不過不單止是這樣.
你繼續看.
Maha.
雖然不是.
三個字的N,是一個M.
但那個音.
起碼有聲音,或者說.
其實都像.
Noah.
Maha.
ok.
Wipe out.
他要滅絕.

$^{1481}$將人類從地上滅絕,六章七字.
還有一個.
Hing.
音不是最像.
不過.
其實是將那個Noah的字母調轉了.
Hat.
在後面,現在變成前面.
Nun.
因為他在尾,變成長尾.
這個字體不是最漂亮,如果看回比較漂亮的原文字體.
你會發覺兩個是像的,只不過是將尾拉長了.
他將那個.
兩個字母調轉了.
由Noah.
變成Hing.
這裡說是Favor,Grace.
說什麼?.
我啊在.
耶和華眼前蒙恩.
如果將它們放在一起,由五章二十九節.
在第六章之前的幾節.
到第六章.
你去看的時候.
或者可能你看到整個.
主題.
其實是相關的,六章一到八節不應該抽出來看的,跟上兩節沒有關係.
主題是說.
他們希望Noah.
可以帶來這個安慰.
但是結果是什麼呢?.
我抄Gordon Brangham,剛才說的聲音學者.
他說了.
以下兩件事.
第一.
他說其實是一個.
單文字.
的表達.
OK.
他說第五節,說的是上帝看到人類是怎樣.

$^{1521}$到第八節.
上帝.
其實看到Noah是怎樣,他不是用看到這個動詞.
他說Noah.
在耶和華眼前蒙恩.
所以眼前蒙恩,就是看到他.
被蒙恩.
看見他的好.
就是這個意思.
所以.
他.
這個線型的倒影就是這樣,上面是看到人類的惡.
下面是看到Noah.
那個.
的義.
然後.
第六節,第一次出現.
耶和華後悔.
第二次.
到第七節的時候再出現,因為他後悔.
中間夾住的.
就是我要.
將人類從地上滅絕.
這個似乎是.
第五到第八節的主題,三個字中間突顯了上帝要.
將人類滅絕,從地上,也是之後.
下文.
為什麼有洪水來的原因.
不過如果繼續再看深一層.
他就說出.
其實他似乎想帶出Noah這個名字.
其實是告訴我們.
上帝在做什麼.
他說首先上帝.
是記得我.
他後悔.
Noah的名字是安慰.
不過同一個字根,不同字幹的字.
可以在另一個情況下.
上帝後悔.

$^{1561}$他後悔做人.
因為人的罪,人的惡.
然後.
上帝就.
想要去將人類從地上.
剷除,滅絕.
剛才說了.
Mahat.
另一個.
和Noah的音或字型有類似的字.
但是.
最後上帝原來.
是要.
放過Noah,要拯救Noah,因為他在上帝眼前蒙恩.
Grace.
Faith.
Hing.
蒙上帝的恩.
似乎這樣看.
刻意.
這個.
Wordplay.
他是用Noah這個名.
原先說他是安慰.
原來那個安慰.
其實是透過上帝.
對人類罪惡的.
後悔.
人類的罪惡,人類做的事情,他後悔做人.
到他要決定.
將人類從地上剷除.
不過.
最後他看見Noah.
那個安慰.
而上帝.
看到那個恩典.
他要給那個恩典.
Noah.
要.
透過Noah.

$^{1601}$將那個救贖帶出來.
如果這樣看似乎.
這一章.
很多看不明白的東西,又說什麼神的女子,誰打誰.
不太清楚.
我們也有興趣,誰是鬼人,為什麼有勇士.
那些需要時間去探討.
不過.
我想起碼從這個希伯來文.
從Noah的名字.
我相信一個.
在這段經文裡面最重要的訊息.
似乎就是這個.
要帶出來的.
是在人類的惡.
上帝要.
施行審判.
洪水要來的.
之際.
不只是純粹要.
剷除人類.
而是Noah.
代表了上帝.
那個的.
後悔從意大利的憂傷.
代表了上帝要去.
施行審判的決心.
但更加突顯出.
上帝要.
施恩.
那個恩典.
希望這樣可以幫到大家理解多一點.
經文的重點.
也可以透過原文.
發覺其實可以多一點.
經文上說什麼.
正如一開始所說.
觀察上可以豐富到大家,幫到大家.
來看到.
起碼.

$^{1641}$一般翻譯.
或者做翻譯對比的時候.
看不到的東西.
最後多說兩句,希望有時間可以看到Q and A.
剛才Gordon Vanhan.
他.
將經文分成.
一到四節,五到八節.
我通常.
看過很多經文,發現很多時分段都是參考的.
當然你需要對原文有理解,才做一個比較準確的判斷.
但我也見過很多時.
都很.
Depends on學者.
怎樣去理解.
大家有不同的演繹.
所以就.
不是一致性的,也有空間去思考.
我自己對於八節的經文.
我會這樣理解.
你見到我其實將它分開做五大段.
1,2,3,4,5,1到2節,3節.
4節,5到6節,7到8節.
為何我會這樣分開呢?.
某程度上.
當看完原文的時候.
特別在那個order上.
原文的次序和中文的分別很大.
多數時候不一樣.
所以如果你能夠看回原文的次序,或者可能都會更加清楚地看到.
那個.
分段.
凸顯出來多一點.
我自己看到是這樣.
一開始.
其實是說.
神的兒子們.
和人的女子,中間那個情況,特別是說人多起來.
神的兒子們看到人的女子的美貌,就來找他們.
然後.

$^{1681}$中間第二段,就不說他們了,就說耶和華的說話.
一開始是耶和華說.
他說了一個說話.
然後到第四節.
又說回神的兒子們和人的女子們.
你見到嗎?重覆回來,不過這次帶出了一班偉人,似乎是他們所伸出來的.
因為他們的教學,產生了一個古代的勇士有名的人.
然後又回到耶和華.
耶和華看到人的罪惡,然後說後悔.
來到做人.
去到最後呢.
其實重覆,再一次說耶和華的說話.
耶和華說了一班說話,就是從地上除滅這班人.
因為他後悔,不過.
好像剛才所說.
原來.
狼在耶和華眼前無因.
其實這裡你看到.
通常三文治的做法,重點似乎放在.
這裡.
中間,有偉人,上古英明有名的人.
什麼意思呢?又沒有說是哪個人,為什麼要凸顯他們呢?.
不過如果再想深一層.
似乎.
在那個情況更加要凸顯的.
其實是.
羅亞.
因為.
原來,你可能可以這樣看.
一開始的是他們.
看見.
人的.
女子的美貌.
不過耶和華卻看見人的.
罪惡.
然後耶和華.
說了一個.
說話.
下面也說了耶和華一番說話.
說出他的.

$^{1721}$回應.
不過要去對比的,最明顯的.
他們似乎生了一些.
上古英.
有名的人.
但是呢.
更加重要,那個不是重點.
重要的卻是.
羅亞在耶和華眼前無因.
這個才是有了很大的contrast.
是的.
好像這班人.
很.
厲害.
不過,這裡也有說.
他們的日子.
只能夠120年.
其實也有不同理解,有些說只能夠去到120歲.
有另一個理解就是.
還有120年,洪水就來了.
無論是怎樣也好.
似乎他們有限制.
更加的是.
當上帝決定要將洪水帶到來.
將這個.
除滅的事帶到來的時候.
因為後悔做人.
他最後看見的.
一切能夠改變的.
卻是因為.
羅亞.
在上帝眼前.
如果這樣去看.
更加可能就突顯到.
原來.
這一.
五段的經文.
在做一個.
不同的比較.
但更加最後突顯的.

$^{1761}$原來就是.
羅亞,就符合剛才我們理解.
原來羅亞似乎在這裡做一個.
很重要的貫穿.
的角色.
帶出這個經文,原來想告訴我們.
是.
上主.
的心場.
他看見的是什麼.
他心裡面怎樣想.
剛才我們沒有時間去說他心中憂傷是什麼意思.
然後.
亦都說出.
當他要去除滅的時候.
除滅大地的時候.
在一切他創世的一章裡面所說他所創造的東西的時候.
他.
卻看見羅亞.
羅亞亦都是他眼前的原因.
這個似乎就是這一章.
的經文的信息.
亦都.
透過那個原文.
或者可能幫到我們.
更加.
affirm.
肯定.
這個理解.
這樣的分段.
似乎是.
可以突顯得出.
那個註指和信息.
給多一個簡單的例子.
我想帶出的,例如第七到第八節.
我最後分的那段.
你會看見其實.
在螢光幕最底那裡.
其實耶和華出現了兩次.
中文都有這樣翻譯.

$^{1801}$不過.
你看回原文.
你就會看見耶和華這個名字.
其實是.
在第一行.
最後那個字.
和最後那個字.
我純粹一個例子.
為什麼我可以將它.
我覺得這樣分段是合適的.
除了因為上面有出現耶和華說之外.
原來這裡.
一開頭是耶和華.
做是.
第一句最後一個字.
和最後那句.
亦都是耶和華的名字做結束.
似乎都是一個上下的呼應.
這些.
你見到中文的翻譯.
做不到因為.
始終礙於翻譯上.
要通辭達理.
不能調文字的次序.
但看回原文.
沒有這個限制,通常會把重要的東西放在前面.
不過有些時候,像現在的情況.
似乎刻意.
把耶和華的名字.
在.
這一段.
一開始一結尾.
都用耶和華.
的名字做結尾.
這些希望透過原文.
當你看到時,你可能更加看得到,咀嚼得到.
亦都可以更加看得到.
要帶出來的訊息.
我停一停,看看大家有沒有什麼問題想問.
希望可以幫到大家.

$^{1841}$這個.
對原文的理解多一點.
特別.
見到.
學習原文.
的好處.
幫助.
我見到有兩個問題,我可以很快回答.
這個Bible Bento.
那個.
字.
十字架.
對我理解應該是翻譯不到原文出來的時候.
放在這裡的.
即是告訴你有一個原文的存在,不過沒有翻譯到中文.
就是這個意思.
另外六章一節.
Well perfect.
這個我不回答了,因為我留待.
遲些有機會開.
課堂的時候.
特別是講動詞的時候,再詳談.
,大家可以在下面留言,我會盡快回答的,多謝各位,再見!.
,多謝各位,多謝各位!.
趁這個機會,還有兩三分鐘,讓李牧師可以喝口水.
回答你們任何關於希伯來文的問題.
我明白的,有些東西是很難問的,要問到是幾個字的,我會盡快回答的,多謝各位,再見!.
總結一句,我覺得李牧師.
真的很深入淺出,我希望他的解說.
能夠.
拉著大家的興趣.
最難的一件事,原文最難的一件事,不是很技術性的說.
最難的就是.
引起人們更加深入的興趣,這個才是最難的.
我相信.
剛才的講論,再加上上一次的講論.
都達到這個目標.
尤其是今晚我聽到一句話,很受鼓勵的.
不要期望懂得語言,可以解釋很多,可能.
幫助你細心的觀察.

$^{1881}$這個觀察是微觀,很細微的觀察.
這樣就能夠.
有一個把握,掌握到一些線索.
以致我們可以推敲到一些.
連翻譯版本都翻譯不到的東西,我相信這一點.
是對大家最大的得益.
不要阻礙大家的時間,如果沒有問題的話.
今晚就大概是這樣,我又覺得.
提醒大家,如果我們真的.
舉行一個學科,我們會在我們的網站公佈.
請大家.
來我們的網站,abscc.org.
這些講座,我們時不時都會有推出的.
希望大家都支持我們,經濟上支持我們,你可以去到我們的網站.
點擊支持神學教育,就可以.
懂得怎樣去奉獻給我們,支持我們的工作.
深入展出,介紹給我們知道.
創世記的八字經文.
他講得很少.
提綱結論,講得非常少.
簡單做一個結束,今晚就散會了,請大家一起低頭.
我們要一生學習.
在你的話語上用功.
借助老師們的學識.
能夠深入淺出.
帶動我們.
先是有興趣.
然後加上自己的努力.
一步一步進入到通透.
去理解,願意在以後日子裡面.
繼續使用李牧師,也給弟兄姊妹有一個心智.
在你的話語上更加用功.
我們.
簡單禱告,奉耶穌基督名求.
阿門.
\newpage



\section{}
\label{sec:3o4omcoTUB4}
\textbf{ABSCC 網上免費粵語講座:非常移民牧養\_20230329}
\newline
\newline
連結: \href{https://youtube.com/watch?v=3o4omcoTUB4}{\texttt{https://youtube.com/watch?v=3o4omcoTUB4}} ~~~~ 語音日期: 2023-03-31
\newline
\newline
\hyperref[sec:efm9yyrZOo0]{\small{< < < PREV SERMON < < <}}
~
\hyperref[sec:index]{\small{[返主目錄]}}
~
\hyperref[sec:B5n__dtTRhE]{\small{> > > NEXT SERMON > > >}}
\newline
\newline
$^{1}$今晚的講座會被錄影,遲一兩天會在YouTube帳戶內,可以鼓勵大家告訴其他教務同工知道,如果有興趣可以在YouTube看今晚的錄影.
當然在場的可以發問問題,互動的情況,如果看YouTube就沒有這件事.
歡迎大家來到我們當中,首先請大家聽我介紹一下,建造中心是甚麼一回事.
建造中心是我們總部位於多倫多這個地方,在我的背景後面的那座樓,不是全棟都是我們,只是其中兩個單位.
加拿大建造中心是一個機構有兩個目標,一個目標是提供粵語的神學教育,給平信徒進修.
我們提供的科目從證書開始到碩士課程的水平都有.
大家稍後會看到我們的網址,我歡迎大家經常可以到我們的網站,看我們最新的發展,因為我們更新得很快.
鼓勵大家來到我們當中,abscc.org,第一個目標是提供粵語神學教育.
建造中心最近一年多了一個很清楚的目標,就是為教務童工的支援.
我們知道有很多新來的教務童工進入加拿大這個場景,我們希望能夠在建造中心做一些工作,幫助大家認識加拿大的環教會.
所以這就是我們第二個目標,歡迎大家經常來到我們的網站,寫email給我們,在網站內全部都有資料,我們能夠有些溝通.
今晚我們講題叫做非常移民的牧羊,未開始之前,我想先和大家看看一張slide一節的經文.
這節經文給我很大的鼓勵,這節經文來自士德行傳13章36節的上半節.
如果你看和修版,和合本的修訂版,會看到大衛在世時進行了神的旨意,長眠了,如果是和合本版本,就睡了.
但下面也有一個footnote,有些譯本是大衛按著神的旨意,服侍了那一世的人,就長眠了.
這一句的翻譯,我在翻譯過很多英文版本,都是這個翻譯的,ESV也好,NASB也好,NRSV也好,這幾個版本是我們經常使用的英文版本.
同樣帶有footnote的意思,For David, after he had served the purpose of God in his own generation, fell asleep.
即是說,大衛在服侍了他當時的世代,跟從了上帝的心意的時候,完成了這個工作的時候,他就離開了這個世界.
為什麼我會選擇這段經文,我們彼此勉勵呢?主要的原因就是最近這幾年的移民潮,從香港來的移民潮,其實你和我都不知道會持續多久.
可能會是兩年,或者三年,又或者更長的時間,也可能在很短的時間內就停止了.
當我們做神的工作的時候,我們就會做好一個心理的準備,既然上帝帶領了一群人來到加拿大這個地方,我們作為教會的領袖,我們就有一個機會去牧養這一批的信徒.
向一批未信的人傳福音,這就是這個世代的人,帶到我們面前,很明顯就是上帝在工作,所以我們的職責就是要服侍這一班人.
如何服侍呢?就是我們今晚講題裡面的核心要探討的部分,服侍了這一班人,我們服侍了多久我們不知道,但是我們有一個心去服侍,就可以得到上帝的喜悅.
我盼望這次的經文能夠我們彼此激勵,就算我們現在各自的教會可能做得很辛苦,在移民的工作裡面,又或者我們做得很成功,但是我們都能夠彼此激勵,能夠繼續向前推,以至到這個新移民的工作能夠使得上帝喜悅.
所以今晚我們這個講題其實就是一個很特別的地方,我們不想是一些很簡單的,隨便就說到新移民牧羊,這個不是什麼太突出的地方,我們挑選了三間教會的牧者,在當中和我們有一個交談,這個是座談會,有一個交談,而通過這些交談,我相信會給大家參加者有些啟發.
這三間教會,一會兒他們都會各自介紹他們教會一些特色的地方,一些獨特的東西,但是基本上這三間教會,在一個平面來講,或者是外表來看,在一個罩眼上來講,他們那個位置都不是一些很特殊的位置的地方,在地理上來講,似乎吃虧一點,針對一些好的教會來講,似乎吃虧一點.
又或者是這三間教會的規模,匯眾的規模都不是一些巨型的教會,究竟這些教會是怎樣找出一條路去服侍新移民呢?就是我們今晚的座談會的特徵,希望大家能夠摸得到的.
我簡單介紹這三位的目者是哪三位,然後我們就進入座談的時候,首先有一位,我多倫多先介紹多倫多,我們介紹的是恩泉浸信會,Stouffville Great Baptist Church的吳盤德牧師,他代表了Stouffville,Stouffville在哪裡呢?可能有些人都不知道,一會兒他會講的.
第二間教會離我們遠一點的就是Winnipeg的Manitoba省的Pastor Nathan Lee,在Winnipeg的地方,揮手過就是Winnipeg的Pastor Nathan Lee,你知道Winnipeg都挺遠的,都挺墮國的,一會兒他會講一下他們怎樣做新移民工作.
第三位離我們更遠的就是在加西那邊,在加西那邊,是大溫地區,不過他不是Wancouver最多香港人去的地方,他是高桂林這個地方,高桂林是哪個地方我也是最近才知道的,請謝耀恩牧師揮揮手,謝牧師是很年輕的牧師.
一會兒他會講一下他們在那個地方是怎樣能夠發揮到他們的特色做新移民的工作,我相信三位牧師來自不同的宗派,亦都不同的地區,在加拿大不同的地理位置,我相信會給大家一些啟發.
以下的程序我想這樣,我發幾條問題,有些問題我希望三位牧師都能夠簡單地去講一講,尤其是第一條問題會介紹他們究竟在各自的教會,在新移民的模樣上有甚麼特點可以分享的呢?.
尤其是他們位於哪個地方,我們很熟悉的,因為我們今晚的聚會是在加東,加中,加西都有的,未必每個人都知道你們三位是在哪裡的,不如先請牧師講一下.
好,我用地圖講一講我們的位置,我們位置在多倫多的北面,所以如果開車由多倫多市中心去蘇寮,大概是35到45分鐘之間,看誰是中數.
我們現在比較多的,我們身處的位置是在馬克森,可能大家比較熟悉,我們南面是馬克森,再南一點,馬克森下面是斯加堡,所以傳統的區就是這樣.
可能建築中心在斯加堡,在馬克森那邊,我們現在其實這個區本身,我們環繞著的就是阿羅拉,新市場,伊斯格魯瑪利,以及阿斯利康,其實都挺多在我們附近的地方是沒有華人教會的.
所以我們最多的新移民其實就是在東面,在康奈爾的區域,跟我們斯多福共同住一個醫院,所以在那個區域開車上來大概十分鐘,所以比較近的,對我們教會來說.
而其他的地方,包括Richmond Hill都挺多會有的,阿羅拉,新市場,其實四方八面都有的,為什麼我們會搬到這裡比較正式一點呢?.
因為這邊比較便宜一點,馬克森和斯加堡已經很貴了,所以他們就搬上來,其實我們的特色,如果你說新移民的特色,第一個問題,因為斯多福現在大概五萬人左右.
但他本身是多倫多的後花園,所以他有很多東西可以玩的,冬天有滑雪,夏天有騎馬,所以很多人上來都會玩,空間感比較大一點.

$^{41}$而且現在新的方向就是城市應該在路邊建立一些新的社區,大概有七萬五千人左右,還有一些戰嘴子的地方,其實加起來有六七十萬人,我們圍住的地方.
所以我們的建堂現在還在進行中,所以我們希望年底可以開始建堂,現在我們是租一個堂址,我們的人數大概是二百五十左右,每個星期最近.
我們在疫情前,在回流移民潮之前,也是一百人左右,所以在回流移民潮之後,增加了很多.
我們的地區反而是斯多福有很多年輕人,大多數的華僑都是四分之一的華人,我們是華人屬於最大的非白人種類的人,在我們現在的城市裡面,四個人裡面有一個華人.
如果說我們有什麼模樣呢?我們設計的模樣是湖光山色,小鎮風情,多人多事的後花園,所以整個環境比較休閒,多些戶外活動,比較著重人情味,個人關係那些.
所以我們自己就看我們的教會和我們所處的地方的社區特色,教會一直在殖堂之後,我們都想和社區結連,所以我們不會小看自己的資源,因為我們相對來說是比較小的教會.
但我們就著重把握發揮自己的特色,包括我們的環境,人力資源,數人恩賜,我們奧義還有什麼嗜好,生活方式是怎樣的,順其自然地發展我們的新移民志工.
我們發展的原則有幾樣,簡單來說就是我們有一個好客之家,比較盡情去熱情款待人,也幫助一些新移民,其實我們第二個原則就是幫助新移民融入社區,因為其實他們來到這一區,空間感比較大,怎樣去投入這裡的生活模式.
另外我們教會本身是一個簡約教會,simple church的觀念,所以以人為本的牧養原則,我們發覺比較容易將新移民融入教會,因為我們很簡單,崇拜小祖和主一學.
所以牧養的資源,我們就可以在這三方面配合新移民,其實他們來到,我覺得他們很大程度上想和神相遇多一點,可能心靈裡面有很多需要.
所以我們都是集中在這方面,與神相遇,特別幫助弟兄姊妹建立一些自己的靈性成長的方向,所以我們的資源擺在門訓或者靈性成長那裡.
建立一些平台,世教會有時候最大的頭痛就是我們可不可以給弟兄姊妹這麼多的侍奉機會去投入教會呢?所以我們就開發很多不同的侍奉機會,減低我們的侍奉門檻,他還沒有做會友都可以參與侍奉.
所以我們牧者和領袖團隊就會盡快把握到新移民和新會友加入教會之後,如何成為我們團隊的一部分,侍奉團隊的一部分.
最後就是牧養方面,我們三年前開始請了兩位香港移居來加拿大的牧民,這是一石二鳥的,因為一方面可以將教務年輕化,因為香港來的牧者比較年輕,亦都可以配合到新移民,因為他們自己都是新移民,所以可以配合到,可以很自然地把關係發展.
另外我們有好客款待的配對,將會眾和新來的人配對,共同嗜好開始,變成他們可以一起生活,一起同行建立.
所以其實很快說到剛才的,我想我和盧老師都說,我們真的沒有什麼特色,所以如果你問有什麼特色,我不知道該說什麼特色,但就是沒有特色.
因為我們沒有一個專責的新移民事工,但我們就將整個教會的強項盡量發揮,做一些加強基本牧養的intentional shepherding,比較有意識地策劃配對.
我想我們教會比較小,可以做多一些比較深入的個人的牧養工作.
非常好,多謝你吳牧師,大家聽得到嗎? 吳牧師剛才已經說了很多點,我不知道你記不記得,如果記不記得的話,不要緊,你後期看YouTube,你會拿到很多.
其實他們教會已經用了很多,他就很客氣,說沒有什麼特色,其實這些沒有什麼特色,裡面就是特色,你能夠捉摸到幾點,已經是很能夠發揮的.
不如請Pastor Nathan你講一下,冰天雪地,你如何做到新移民事工,請你分享一下.
我先透過PowerPoint讓大家了解一下我教會的地理位置,以及我教會的移民事工的發展.
好,我們Winnipeg其實是在Manitoba,過去中文翻譯叫做面傘,在我們早期的留學生,以及在90年代移民,隨著多了很多中國內地,來自Mainland China.
現在官方有兩個名字,大家可以叫面傘,或者是萬傘,而我們的首輔就是Winnipeg.
雖然是首輔,但實際上Winnipeg相對,因為我們中文譯作叫做溫尼帕,所以有時候說溫哥華就叫大溫,我們就叫小溫.
不像華人圈子裡,溫川,可能是很小的地方.
不過我想說的是,究竟Winnipeg在整個華人社區裡面的情況是怎樣呢?.
我其實來了加拿大是7年的時間,我2016年才來到Winnipeg,我之前是在中東參與宣教的體驗,有8年時間.
所以來到Winnipeg,當時我也去了解Winnipeg是一個怎樣的北美的中部城市.
我會回答,放心.
其實當時我嘗試了解加拿大2016年的統計局,究竟在我城市的群體有多少是Cantonese.
這個很重要,因為我最主要的目標是粵語群體,我的教會是三文兩語,我要了解一下我目標是粵語的人數有多少.
大家可以看到,如果你是PR,甚至是Citizenship,即是在加拿大永久居留,成為加拿大公民.
其實實際上在Winnipeg只有6625,因為是以廣東話為主的.
所以我們可以看到,其實可以目樣的對象,大部分以廣東話為主的是香港人.
有小部分可能是早年的台山,甚至是一些可能是廣東省他們當做統計的時候,他們說他們的家庭是以廣東話為主.
所以究竟在過去2019年之後產生的移民潮是怎樣呢?.
其實在我們NB省裡面,我都認真去做一個初步的評估.
初步的評估,其實我們已經翻了一倍,無論在留學生,甚至是自從出了Hong Kong Pathway之後.
大家知道有Stream A, Stream B,無論是透過讀書,透過工作,甚至我的省其中一個優勢.

$^{81}$是在整個加拿大裡面能夠透過省提名,我不跟Hong Kong Pathway,只要我能夠找到工作半年.
我都可以入紙申請成為加拿大的永久居留,所以其實這個政策在我們省,甚至加上Hong Kong Pathway.
令到更多的家庭,如果他們想在第一步的移民,他們考慮無論是Living Standard,生活的開支各方面.
相對比一些大城市,可能是便宜一些的時候,我們俗稱的開平費,即是去移民.
他們往往有一些就會考慮來到一個城市,甚至留學,甚至去發展移民的路徑.
所以在一步的時候帶來了一個很大的契機,以致這幾年在整個翻天覆地,除了Pandemic.
甚至是湧現了2019年衍生到2020年到2022年的移民潮.
我都通過我自己的中派裡面的教務同居分享,如果要我用兩個的Capital Letter,就是HP.
無論香港的移民潮,甚至全球的疫情,都成為了在整個模樣裡面的機遇和翻天覆地的改變.
如果我再簡單一點來說,對我教會,我猜我們一班的教會領袖,特別是教會裡面的長老執事.
都一起看到,這是一個Change的時候,甚至會帶來一個Trance.
所以願意去體現這個機遇,而願意去改變,就能夠對教會的事工,甚至是粵語的事工,有一個突破點.
我講一下回顧,從2016年我當年來到到2022年,究竟我教會出現了什麼樣的改變呢?.
當時只有三個家庭,甚至我都不是想在加拿大移民,那時也沒有一個出現的移民潮.
所以當時是有一些回流,有兩個家庭和我一起在2016年的7月來到.
所以當時其實最鞏固,我們在城市裡面,其實最重要要做的就是留學生的事工.
所以當時要看準要做好留學生,要做好校園裡面,是牧養香港的年輕人.
去到2018年,突然出現了多了累積的10個家庭來到我的城市.
當時因為我們的城市,我們的省比較容易門檻去移民.
所以省提名這個優惠政策,希望吸引更多新的一些家庭,年輕的家庭來到發展.
甚至年輕人都可以,省政府如果我們畢業的時候.
留學生可以在我們那個省三年批他們尋找工作的簽證.
所以這些優惠政策令到更多的年輕人和一些移民家庭願意考慮在我們城市.
2018年當我訪問這10個家庭的時候,不少是因為2014年當時香港出現了Unbearable Revolution.
所以就開始成立了第一個的移民事工的一個牧養的一個專為香港移民家庭的小組.
去到2020年的時候,大家知道我們進入了2019年香港出現了社會的一些動盪.
以致後來在那個暑假,也在之前3月去到打後開始進入了lockdown的時候.
但是竟然在我們的城市裡面,是很多的香港家庭.
他們闖關,都希望能夠來到加拿大裡面發展.
以致當時候在那個暑假,竟然是湧現到我們已經接觸到有25個家庭.
從香港而來,可以說是因為香港那個社會動盪的情況下.
帶來一個來得很急的移民潮.
在2020年6月,那時經過兩年的移民潮.
結果我們的教會出現了什麼樣的改變呢?.
就是去到去年的暑假,我們發覺在我們一直去接觸.
而在這兩年裡面,我們來了有60個家庭.
60個家庭曾經參與教會的一些聚會.
甚至嘗試體驗一些新移民的一些活動.
結果這60個家庭有一半的已經很穩定的參與在我們教會裡面的聚會.
成為了三個半在移民事工的小組.

$^{121}$為什麼會有半個小組呢?.
是因為我們都有面對培育新組長的人手上的緊張.
所以有一些半個組,是半個移民為主的.
和我們既有的一些已經在這個城市生活了一段日子的一幫.
資深的會友一起願意的一起在小組.
所以過去在這幾年的移民潮.
我們結果就來了30個家庭.
而30個家庭之中,令到我們雖然我們都有.
aging population,在粵語課的事工有人口老化的情況.
大家可以想像如果90年代的移民.
其實已經到了今天他們進入退休之年.
有部分也在COVID期間影響了他們身體健康的狀況.
所以現在這一幫的香港移民.
其實是令到在COVID-19之前.
是有120個姐妹,甚至朋友來到粵語課的聚會.
包括一些兒童可能有10個,少於10個以下.
包括我的子女.
但是在過了COVID這三年之後.
在我們粵語課現在加上移民家庭.
帶上子女,年輕人.
我們整體上在粵語課.
如果連同小朋友和年輕人.
我們已經增長到200個的聚會在星期天的主日.
而在接下來,因起的是在這幾年的模樣.
已經有部分的香港移民家庭.
是準備在這個Easter會轉會成為我們的會友.
甚至是洗禮.
當中有年輕人,也有移民的一些婦女.
所以我所說的關鍵是.
其實在2021年的時候.
當我們看到2020年來了這麼急的移民潮.
我們在短短半年內.
是馬上啟動要成立香港移民事工委員會.
而我想去講我教會其中在這個移民事工委員會裡面.
我猜我們很強調的兩個重點.
就是我猜我們是願意的一起.
一班教會領袖是同心的分擔.
如果我用這個Team這個字.
是Eat Together.
願意每個人都貢獻.

$^{161}$他們可以貢獻的時間,力量,恩賜.
Everyone.
還有我們希望能夠建立這一班香港移民的家庭.
能夠重新投入來到這個城市裡面.
有一個淑齡的家庭是我們的achievement.
而在事工裡面.
我們其實都有面對人手上的緊張.
但是我們就會優始去定位.
Priorities.
我們怎麼優始定位呢?.
在我教會裡面.
我們的移民事工委員會.
會定位就是來了很多移民.
我們第一優先的就是.
信主多年而希望來到這個城市.
有穩定的教會生活.
我們希望將這些教徒.
先帶回來這個城市裡面.
有一個淑齡的家.
是我們優先的對象.
至於第二就是.
我們會接觸到一些過去它自稱是教徒.
它說它去過教會.
但是卻不是很穩定的.
我們希望藉此與它結緣.
以致鼓勵它們回教會.
而第三我們的Outreach.
是盼望這些移民家庭.
透過社交媒體.
透過我們一些活動.
接觸到而希望它們都能夠.
這些福音的對象.
可以來更體驗教會的生活.
所以我們在開拓網上事工.
除了因為疫情的緣故.
我們用大數據發現.
原來很多很多家庭.
第一次接觸我們教會是透過網上.
然後我們設立我們的Facebook.
和一些核心的會友.

$^{201}$他們用社交媒體.
年輕人Instagram.
現在我們再在其他的社交媒體.
去建立一些Signal Group.
WhatsApp Group.
然後盼望它們能夠去接觸到教會.
所以這些網上平台.
是很重要.
在我們過去幾年裡面發覺.
之後是定時定刻.
因為初初移民他們都很關心.
例如是牧者.
還有教會的一些獎執的探訪.
我們會組織.
而我們當時過去這兩年.
就是能夠兩個星期.
怎樣都去探一些來到教會的新來的.
現在在各時各節.
無論母親節,父親節,中秋,新年.
疫情很多的限制.
但是我們都親自去作博.
表達教會對他們的愛和關懷.
甚至如果當時疫情開始寬鬆.
可以做卡夫的時候.
我們都嘗試協助.
初初來的家庭.
如果他們是需要接送回教會的時候.
我們都盡量在我們能力範圍.
甚至結緣一些是Uber.
是我們認識的一些人.
是結緣的Uber.
然後去給一個優惠.
幫助他們能夠去到教會.
最後我想去說的就是.
戶外活動是很重要的.
很多移民家庭.
其實他們都想去體驗.
在我城市一些郊外旅遊的景點.
甚至冬天的活動.
我們都去組織.

$^{241}$以致帶他們去體驗一下在加拿大生活.
不過我都想去補充.
在疫情期間.
我們都有一些因為防疫的緣故的限制.
但是若是能行的.
我們都能夠無論大規模,小規模.
甚至我們那邊可以Ice Fishing.
給他們體驗一下.
我們都組織幾個家庭.
和一些香港移民家庭一起去體驗.
在加拿大無論夏天冬天的一些生活.
給他們感覺.
其實這裡有很多生活的樂趣.
好.
我這些是我整體上一些的事工計劃.
至於一會兒我會有機會再談的時候.
我會再去說多一點.
我們教會在現在這幾年.
究竟這些剛才吳牧師都所說的.
我們盡了很多努力.
累積了很多經驗.
其實都是在做90年代的Hospitality.
當時的移民條.
而這些只不過在定位裡面.
我們再定位得清楚一點.
但是What's Next和What's Challenge.
我估一會兒如果有時間.
我估我會再去分享多一點.
好.
多謝Nathan Lee.
一會兒會陸陸續續回轉不同的牧者.
去凸顯一些特色.
又再找Buster Ray.
說一下高桂林你們那邊的情況.
請Buster Ray.
好.
大家看得到BPT.
聽得到嗎?.
好.
首先抱歉.

$^{281}$因為我可能說話沒那麼快.
有少少後音.
因為廣東話不是我的母語.
無論如何.
Westwood Alliance高桂林村徒會.
就在高桂林.
這個就是我的.
我的母會.
我就去了香港建道讀信學.
接著就回來部會服侍奉禮.
看到這個.
剛才盧牧師都說了.
我們是在加西.
在BC省.
這裡就是Metro Vancouver.
大溫市區.
城市裡面.
大家如果能夠熟悉.
大溫裡面的城市.
都有聽過溫哥華.
列治文.
Burnaby.
我們就在Burnaby.
大概一兩個城市.
就叫高桂林.
高桂林裡面.
其實在2021年.
就大概有14.8萬人口.
在一年之內.
就增長到16萬.
我本教會大約是7,8人.
有連同430粵語.
120英語和120國語的弟兄姊妹.
這430粵語的弟兄姊妹.
包括我們在隔離兩個城市.
直堂的弟兄姊妹.
還沒有獨立得到.
所以都是屬於高桂林宣道會的.
我們的團隊.
大概目者同工.

$^{321}$大概是7個全職.
3個半職.
還有5個義務.
這個大概就是我們的情況.
說到大家可能去溫哥華.
列治文.
Burnaby.
可能你不懂怎樣說.
在列治文.
列治文有香港人.
香港教會.
可以一條街都有4,5間華人教會.
其實我們就沒有那麼多.
我們這條街都是一個.
我們城市都有.
高桂林有華人教會.
包括普通話國語教會.
都有3,4間.
在地理來說.
我們佔有一點點的優勢.
就是我們教會.
這裡是我們教會.
教會對面就是LeForge Park.
有很大的湖.
有些籃球場.
足球場.
籃球場.
你想到什麼都有.
另外一個優勢就是.
我們有天車站.
天車站如果你沒來過溫哥華.
我們是沒有地鐵的.
我們是天車的.
路軌是在空中.
走到一條街就有天車.
我們的教會.
如果看大溫區.
英文和法文普遍多講的語言.
很多城市都有Punjabi.
第二大可以說是Mandarin.

$^{361}$Mandarin是國語.
普通話.
第三看在哪裡.
總括在哪裡.
可以說是廣東話和韓語.
在那個區裡面.
大部分都是Mandarin和Cantonese.
就是After English and French Speaking People.
這個就是大概我們的地理.
和教會人數各方面的東西.
至於時工來說.
在2020年疫情之前.
或者是社運之前.
2019之前.
我們看到我們的教會.
都是在路化中.
無論是國語堂也好.
英語堂也好.
廣東話也好.
只講廣東話.
我們都是一個移民教會.
意思就是在1994年開始.
一群香港移民來的弟兄姊妹.
開始這間教會.
那時候就五個人在我們創辦人牧師.
一個叫做Basement下面.
開茶經.
經歷了差不多30年的時間.
就成長到差不多700多人.
在疫情和社運之前.
我們都看到我們的弟兄姊妹.
因為1994年.
剛才李全都說了.
都是已經接近或者進入退休年齡了.
我們很多都是.
教會已經老化了.
年輕的就搬出去Vancouver.
或者在Burnaby去住.
上班等等.
都是住在外面的教會.

$^{401}$我們就想著.
其實我們的將來.
都會在英文堂或者國語堂.
但是誰不知就有社運發生.
然後又有疫情發生.
我們就.
疫情之前.
我們在語堂大概有300人出席.
在疫情前和疫情現在的期間.
就增長了大概100人左右.
其實我不敢說其他教會.
我們教會在疫情當中.
就沒有了一個百分比的會友.
去了其他教會也好.
或者從此不回教會也好.
什麼都好.
就少了大約25\%.
到30\%左右.
在疫情期間來的弟兄姊妹.
或者社運後來的弟兄姊妹.
來的人已經填補了.
甚至是超越了.
那個走了的人.
就意志到我們需要.
直堂.
也不是單單為了.
教會地方飽滿的.
是因為我們在.
隔離.
Sorry.
隔離兩個城市.
就是Meadows.
在東一個城市.
就是Maple Ridge.
在那裡是沒有廣東話.
廣東話的教會.
所以就知道很多.
香港的朋友.
弟兄姊妹也好.
都afford不到.

$^{441}$給不起在溫哥華.
列治文.
或者是不拿比去住.
或者是想用同一個價錢.
同一個價錢去買更加大的房子.
在向東住.
我們就在那裡.
就是直堂.
同一個西人宣導會教會合作的.
這個就是大概教會具體的情況.
如果說到時工來講.
其實具體來講.
我們是沒有開展過一個新移民時工的.
OK.
好像毛遂說的.
我們就多一些人與人關係.
牧羊,關心等等.
這樣進行的.
但是雖然沒有新移民時工開展.
但是帶來我們原有的時工團體有很大的轉變.
什麼轉變呢?.
就是比如說.
因為我們很多年輕家庭.
來了高貴能住.
因為都是說多貴各具.
溫哥華,列治文.
或者不拿比太貴了.
或者是想用同一筆錢去買大間的房子.
所以小朋友能夠去玩得自由一點.
有一個後院等等.
可以說是一個20人左右的年輕家庭團體.
可以成長到一百多人.
在疫情期間.
以致要改變那個團體的進行模式.
第一就是要分開.
因為已經太大了.
要分團.
分開那個團體.
第二就是要多一些大團小組的模式.
因為不能夠再是20個人.

$^{481}$就是每一位去查經等等.
因為已經有80多個.
甚至100個人在一個團體裡面.
但是這個轉變.
我是一個青少年目者.
但是青少年目者都是很大的轉變.
不要說比較年輕的那些.
就是高中或者大專那些.
我們一直以來都有一個.
高中,大專,粵語青少年團體.
其實在疫情之前.
是五,六個人.
可以說是掉到鹽水裡面.
差不多要結束了.
在疫情當中.
甚至在近這六個月.
五,六個人是增加到三,四十個人的.
所以在短短.
尤其是這一年來.
是增加了大約三十個人.
所以我們看到不單是年輕家庭的來.
也是很多青年人都早來.
或者是選擇來到高桂林溫哥華去生活.
去讀聲大學等等.
大致上都是這樣.
很多時候.
我怎麼說到對面這個公園.
因為很多時候小組聚會.
或者是崇拜園.
很多家庭都會去叫外賣.
或者是賣外賣.
去外面聚會.
和幾個家人一起聚會.
去一個聚餐.
很多時候我們都不會教會特意設一個事工.
會幫助新移民的.
反而是弟兄姊妹很踴躍.
很自動幫忙.
去幫助新移民來的弟兄姊妹.
去settle down的.

$^{521}$譬如有一個truck.
他就會幫忙搬傢俬.
搬家.
或者是組成一個group.
有二手電磁爐.
或者是水煲.
你明白我意思嗎?.
在現有的家裡沒有用過的.
已經沒有用的了.
就可以給新移民來添加一些東西.
所以他們的使費不會太高.
我們教會呢.
你看這個地圖.
其實那個地方很大.
如果沒有車都很不方便.
尤其是住在山上.
因為我們靠山.
搭巴士等等.
加拿大搭巴士.
我不知道其他省.
溫哥華搭巴士都很不方便.
不像香港.
都很困難的.
尤其是去做證件.
或者去做某些東西.
考車牌等等.
都需要人幫忙的.
雖然有天車在附近.
但是天車都不是處處都可以去的.
很多人來到我們教會當中.
都需要大型資本去幫忙.
甚至有時候是牧者都會幫忙.
不過天車就會幫到很多.
暫時就說到這裡.
因為還有機會問問Pastor Ray.
我想補充一下.
我們一會兒會有一個Q and A的時間.
請大家如果有什麼問題.
可以在聊天室寫出來.
我們一會兒會邀請三位牧者.

$^{561}$個別去答覆的.
現在我想帶領大家討論.
因為時間也不是太多.
剛才其實.
我也不想總結.
因為已經太多點了.
三位牧者都說到他們國慈教會.
有很多特色.
其中我覺得最合聽的.
我不知道你覺得合不合聽.
就是走自動波.
教會可能不是很大.
或者人手不是很厲害.
或者沒有什麼program.
但一樣都可以搞到.
新移民的工作搞到有聲有色的.
就是弟兄姊妹願意去投身.
或者應該說覺得.
道地的願意奮身下去.
這樣就可以做成.
所以有些教會是很structure的.
一些很完整的program.
可能剛才Pastor Nathan那個.
就已經是那一類.
但其他.
吳牧師或者Pastor Ray.
那個情況就是.
可能不是一些很特別的.
一些的事工的.
project的形式.
都一樣可以做到.
新移民的工作.
而且就令到參加的人數.
是會飆升的.
我聽到這個數字的飆升.
我覺得是很興奮的一件事.
有哪個牧者不會因為.
人數飆升而覺得.
沒什麼所謂的.
一定是很開心的.

$^{601}$所以這一點.
我相信是一個關鍵點.
所以鼓勵各教的牧童工.
不要因為你的教會可能比較小.
或者可能比較沒有什麼特色.
但一樣都可以.
做到很實質的東西.
現在我想帶領大家一個討論.
就是什麼呢.
我們事物一定會有.
正面反面的.
剛才三位牧者都說到.
很興奮的一部分.
我想大家調節一下.
說到一些困難的部分.
困難的部分.
我想說的就是.
你覺得.
三位牧者覺得.
你在過去這兩年.
三年移民的事工裡面.
感覺到最艱難.
去突破的點是什麼呢.
可不可以具體去分享一下呢.
先請Fasten Nathan 先說.
你覺得最困難的地方.
你如何解決這些困難呢.
好,謝謝樂牧師.
我希望很真誠地分享.
在牧羊移民事工的情況.
其中我剛才都帶到一個重點.
就是.
來了已經是一批.
香港這麼多的湧現的移民.
我當在2020年.
當時疫情也是防疫.
最緊張的時候.
華美又滑仙.
當時已經接觸了.
無論在社交媒體.

$^{641}$甚至在不同渠道.
都已經知道來了很多移民.
甚至他們.
人數限制都來教會.
去嘗試接觸.
所以我們都在問.
好像很多的行會.
我們會辦移民的講座.
無論去到一些重要關鍵的季節.
例如Test Return.
報稅的講座.
又或者是大家.
因為這次來很多的家庭.
他們很關心.
加拿大教育的系統.
在我們的省.
無論是Private School.
Home School.
有些人提出.
擔心加拿大的性傾向.
很多元.
他們都會多考慮.
Public School.
Private School.
和Home School.
的一些問題.
甚至於可能大家也會關心.
怎麼找家庭醫生.
這些講座.
我們全部都很努力.
去找一些專業的人士.
透過第一智慧網絡.
去辦講座.
但我們在問.
來的人都多.
甚至用Google Form.
我們很系統的.
可以做Database.
成為去跟進.
不過我們問.

$^{681}$其實這一次來的.
新的移民.
和90年代的移民潮.
有什麼分別呢.
我嘗試在PowerPoint裡.
再和大家分析多一點.
大家看到頭影片.
我想帶出一件事.
剛才說的教會.
已經有移民事公委員會.
大家都有一個定位.
甚至有Strategy.
做了很多剛才說的.
譬如保險.
報稅的講座.
吸引移民家庭.
人流都多了.
去接觸教會.
甚至網上平台.
辦這些Online Meeting.
但我們都要問.
當我們在接觸.
已經由2020年的暑假.
去到成立移民事公委員會.
我們正在問一個問題.
其實是很重要的問題.
其實90年代都很努力.
已經大家很習慣.
很有經驗做Hospitality.
這個也是加拿大教會的強項.
甚至剛才我聽到謝牧師說.
我們都有弟兄姊妹會組織一些.
幫人搬家.
有些弟兄姊妹有出.
他格外的愛心.
去幫一些移民家庭.
甚至去機場Pick up.
但這些Hospitality.
甚至是Radical Hospitality.
我們都想問一件事.

$^{721}$What's next?.
來了人了.
但來了的時候又好像.
似乎不是很留到他們.
我們開始去問.
如果有一些已經穩定回教會.
會不會是我們定位裡面.
第一批真的要牧羊到他們.
所以我們正在問一個問題.
我當時帶領移民事工委員會去探討.
究竟過去1997年那次移民潮.
和2019至2022年這次移民潮.
本質上有什麼分別.
其中的分別我想去說.
在2021年.
在明報裡面有一位.
其實在香港一個退休.
在Hong Kong U做.
是Social Work的一位學者.
叫周永新博士.
他的文章裡面寫得很好.
他說在1997年的移民潮.
是基於想像中的恐懼.
但是2019至2022年的移民潮.
是基於體驗的恐懼.
所以當我例如這個新聞剪輯.
我都和宣導會裡面的教務同工去分享.
這一次Hong Kong's Exodus.
在西方的媒體裡面.
是It's real and painful.
是這一班的移民他們正在經歷的.
是一種他們要年金拔起.
身於師長於師的地方.
又來得很急的.
去到加拿大甚至到英國.
或者其他國家.
而這一種他們主觀經歷的感受.
在Council裡面來說.
他們是經歷過一些叫做不同程度上的政治傷害.
所以當我提到這一點.

$^{761}$和移民事務委員會的祈禱.
究竟可以怎樣在牧羊之中.
能夠幫到這些家庭.
甚至是年青人呢.
所以在2021年9月.
我在暑假開始去籌劃一個查經材料.
隨其是現在要查經.
而來到加拿大.
又或者移民去到居家.
無論去到什麼地方.
我們盼望這個查經材料.
能夠幫到來到我城市裡面的家庭.
去重新思考究竟可以怎樣.
能夠聆聽他們的心聲.
所以我們就以阿伯拉罕的家庭.
移民的事蹟由創世記.
一直到信仰的傳承到以撒.
我們就很systematic.
在剛才說到的三個移民小組裡面.
我們就一起的.
先由我帶頭帶了第一次.
然後再去帶另一組.
去到第三組已經可以培育到.
有其他的查經組長可以去帶.
所以在過去我們在試驗性的去查的時候.
我們其中是鞏固一些家庭.
他們能夠去講他們所面對移民的掙扎感受.
甚至除了一些scenario.
例如在機場從香港移民之前.
和家人講這個急的決定那種內心的掙扎.
以致讓他們表達多一些感受.
又或者一些在過去.
無論90年代到現在.
我們都會面對著仍然會有太空人.
甚至有一些我們說是shortcut.
是 捷徑 希望可以盡快拿到PR的方法.
例如買工 其實你給了錢給移民公司.
但實際上那個工作你是不需要去做的.
這些所謂的shortcut.
是否作為在我們討論之中.

$^{801}$去再看我們在移民的倫理裡面.
那個道德的課題.
所以我們很靠近 是不會去迴避.
而是盼望我們能夠在信仰裡面.
真是建立這一班來到香港的朋友.
所以當中也有不少的香港朋友透過這個課程.
而他們更加打開心扉.
甚至在眼淚之中願意決志祈禱.
所以我們盼望這個.
是我們面對挑戰之中困難.
在頭一年的模樣而去開展一個課程.
然後更盼望這個柴經材料.
是能夠去一個很認真的信仰的反省.
至於 盧牧師我想多說一點.
請簡介.
至於我想說 去到當下.
其實三年的移民潮.
我的challenge在我的教會是甚麼呢?.
除了父母現在的模樣可以透過柴經,對話.
甚至一些counseling可以去幫助一些家庭.
但是我們也面對另一個挑戰.
這一次來到的 是一家大小來到.
特別是孩子的模樣.
我想說的是 接下來我預示我的教會.
甚至更多其他在加拿大.
我們一起面對的次移民潮.
Children and Youth Ministry.
究竟這一班已經可能鞏固了.
一些過去在香港裡面的文化.
今天在Internet的世代.
他們的結連香港仍然是很深.
現在在移民之中.
他們自己的identity.
甚至他們也有一些父母的political trauma.
影響他們怎麼看自己香港人的身份.
所以求上帝也幫助.
宗教會甚至.
我教會正正面對一個模樣裡面的挑戰.
而我感覺到現在的年輕人.
甚至是一些去到青春期的香港移民.

$^{841}$其實他們也有hesitate回到教會的情況.
所以特別邀請大家.
為這個需要 為我的教會禱告.
多謝Pastor Nilton.
因為是非常之rich的information.
所以也不想打斷Pastor Nathan的分享.
不過我也很想聽聽.
例如Pastor Ray.
剛才Pastor Nathan說的Youth Ministry.
可否Pastor Ray又說一下.
你們如何開始或開拓.
在新移民裡面的青年的事工裡面.
你們有什麼板斧在那裡做.
你們教會的特色在這方面.
在團體或小組裡面.
都是同行.
因為我在這裡長大.
我多了解這裡當地的youth文化.
在香港移民潮來的時候.
那些youth或children是另一種文化.
其實我不是幫得太多 做得太多.
所以我們就要去精選.
很purposefully去選一些導師.
有相同的經歷.
或者是late 20s 或 early 30s 這樣的導師.
去幫助移民來的youth.
尤其是youth.
因為最主要是比起兒童來說.
他們是很大的打擊.
另外一件事就是.
對家長的一些教導和分享.
就是性教育這件事.
你看到這個畫面都很特別.
我和我太太.
我太太也是我教會的女教會女教會.
我們一起合教 懂得怎樣教.
其實是21世紀基督教性教育.
是專門在說在加拿大或溫哥華上學的小朋友.
會遇到什麼.
小朋友我不是說青少年.

$^{881}$而是children.
裡面的topic也有.
也有這樣的東西.
其實有一個家長就說到.
怎麼搞的.
在pre school 或 kindergarten.
也會和小朋友說這些.
說到性教育.
但其實也不能怪這位家長.
可能他在香港或亞洲地區來說.
遲一點說都可以.
但來到加拿大.
如果我們不早點和下一代.
青少年也好小朋友也好.
開始說性教育這件事.
我們的學校或社會就會和他們說.
合和聖經的教導是兩回事.
如果說到challenge挑戰.
就是性教育這件事.
青少年或兒童方面.
非常好.
我都質問你對於基督教性教育推行這件事.
很有心得.
以後有機會讓大眾請教你.
我相信在移民父母的心目中.
不單止是為了子女的教育做得好一點.
讓他們容易有機會成材.
也有很多的信不信耶穌.
性教育的挑戰.
社會的多元文化挑戰.
一大堆東西就出現了.
所謂新移民時宮不單止為他們settle down.
也不單止為父母信耶穌.
帶領了一個很大的問題出來.
每個教會的目者都逃避了.
我相信這是一個相當大的題目.
在我繼續說的時候.
請大家留意我們的chat room.
可以給大家寫一些問題.
待會我們會請嘉賓的講員回答.

$^{921}$請大家盡量使用chat room的效果.
我想回應一件事.
剛才Nathan提到查經的資料.
我亦有兩項宣佈.
等到Q and A session完結後.
會有一些新的資訊提供給大家.
剛才Nathan提到查經是用來牧養訓練門徒.
Westwood Alliance Church是用青年的侍工.
而且有這麼特別的地方.
旁邊有這麼好的picnic area.
元祖崇拜可以做picnic.
這是得天獨厚的地理環境.
我想是溫哥華才有這麼理想的地方.
我們這裡就很難了.
不過吳牧師那裡我都去過.
是非常優美的.
我希望能夠在那裡住.
太過優美了.
countryside的生活非常優柔自在.
其實三間教會的牧者都給我們一個體驗.
就是每個教會都有獨特性.
我相信神會特別使用不同的教會.
有不同的mindset.
有不同的大英占會的恩賜.
而tailor-made到一些新的牧養方向.
來接待這班新移民.
我相信這是一個很特別的地方.
今晚我們的座談會其實就是想帶出這一點.
沒有一間教會是可以獨佔新移民事故.
也沒有一個one size fits all的model.
而實際上就是各司各法的做法.
所以我鼓勵在座的教皇同工.
來自不同的教會.
都能夠彼此得著激勵.
這一點是我們很想帶出的.
這個時候我還沒有看到新的問題.
還沒有看到新的問題的時候.
我就繼續發問一些東西.
帶領這三位牧者的一些討論.
說到新移民發展.

$^{961}$請大家給一些意見.
你覺得最關鍵的key success factor是哪裡呢?.
假如有另外一間教會問你們這三位牧者.
現在開始想教新移民事故.
究竟最關鍵或有甚麼提醒可以給到呢?.
不如請胡牧師分享一下.
如果有人問你這個問題你會怎樣回答呢?.
我想幾樣很簡單的說.
對我來說就是保持彈性.
保持彈性可以有速度的回應.
特別是我們中小型教會.
其實回應速度可以快一些.
而且不用像李姑姑那樣.
一定要有太多既定原有的包袱.
另外一樣就是其實香港移民.
很多時候他們在香港都很慣用app.
app或者social media.
我知道時間沒有了.
我只是給大家看一下.
我們教會就弄了一個app.
其實裡面有很多資源給新移民.
包括他可以engage.
我想其中一個next step.
新移民engage教會.
怎樣engage呢?.
我們就提供裡面各種人有不同的興趣.
無論是數人生命.
無論是零收材料.
無論是牧羊配套.
其實每一個人都有看過所有東西.
但是可以在不同的部分.
在app裡面可以適應到.
給予一些頂尖媒體.
我想其中一個key.
就是我們訓練了一些組長和教務.
就是just in time.
例如報稅.
短的video.
按進去app裡面.
已經可以看回那個video.

$^{1001}$然後在那裡面.
就有一個人.
可以在那裡面.
就有一個人.
可以在那裡面.
就有一個人.
可以在那裡面.
就有一個人.
可以在那裡面.
就有一個人.
可以在那裡面.
就有一個人.
可以在那裡面.
就有一個人.
可以在那裡面.
就有一個人.
可以在那裡面.
就有一個人.
可以在那裡面.
就有一個人.
可以在那裡面.
就有一個人.
可以在那裡面.
就有一個人.
可以在那裡面.
就可以看到一個人.
就可以在那裡面.
就可以看到一個人.
可以在那裡面.
就可以看到一個人.
可以在那裡面.
就可以看到一個人.
可以在那裡面.
就可以看到一個人.
可以在那裡面.
可以看到一個人.
可以在那裡面.
就可以看到一個人.
可以在那裡面.
就可以看到一個人.

$^{1041}$可以在那裡面.
就可以看到一個人.
可以在那裡面.
就可以看到一個人.
可以在那裡面.
就可以看到一個人.
可以在那裡面.
就可以看到一個人.
可以在那裡面.
就可以看到一個人.
可以在那裡面.
就可以看到一個人.
可以在那裡面.
就可以看到一個人.
可以在那裡面.
就可以看到一個人.
可以在那裡面.
就可以看到一個人.
可以在那裡面.
就可以看到一個人.
可以在那裡面.
就可以看到一個人.
可以在那裡面.
就可以看到一個人.
可以在那裡面.
就可以看到一個人.
可以在那裡面.
就可以看到一個人.
可以在那裡面.
就可以看到一個人.
可以在那裡面.
就可以看到一個人.
可以在那裡面.
就可以看到一個人.
可以在那裡面.
就可以看到一個人.
可以在那裡面.
就可以看到一個人.
可以在那裡面.
就可以看到一個人.

$^{1081}$可以在那裡面.
就可以看到一個人.
可以在那裡面.
就可以看到一個人.
可以在那裡面.
就可以看到一個人.
可以在那裡面.
就可以看到一個人.
可以在那裡面.
就可以看到一個人.
可以在那裡面.
就可以看到一個人.
可以在那裡面.
就可以看到一個人.
可以在那裡面.
就可以看到一個人.
可以在那裡面.
就可以看到一個人.
可以在那裡面.
就可以看到一個人.
可以在那裡面.
就可以看到一個人.
可以在那裡面.
就可以看到一個人.
可以在那裡面.
就可以看到一個人.
可以在那裡面.
就可以看到一個人.
可以在那裡面.
就可以看到一個人.
可以在那裡面.
就可以看到一個人.
可以在那裡面.
就可以看到一個人.
可以在那裡面.
就可以看到一個人.
可以在那裡面.
就可以看到一個人.
可以在那裡面.
就可以看到一個人.

$^{1121}$可以在那裡面.
就可以看到一個人.
可以在那裡面.
就可以看到一個人.
可以在那裡面.
就可以看到一個人.
可以在那裡面.
就可以看到一個人.
可以在那裡面.
就可以看到一個人.
可以在那裡面.
就可以看到一個人.
可以在那裡面.
就可以看到一個人.
可以在那裡面.
就可以看到一個人.
可以在那裡面.
就可以看到一個人.
可以在那裡面.
就可以看到一個人.
可以在那裡面.
就可以看到一個人.
可以在那裡面.
就可以看到一個人.
可以在那裡面.
就可以看到一個人.
可以在那裡面.
就可以看到一個人.
可以在那裡面.
就可以看到一個人.
可以在那裡面.
就可以看到一個人.
可以在那裡面.
就可以看到一個人.
可以在那裡面.
就可以看到一個人.
可以在那裡面.
就可以看到一個人.
可以在那裡面.
就可以看到一個人.

$^{1161}$可以在那裡面.
就可以看到一個人.
可以在那裡面.
就可以看到一個人.
可以在那裡面.
就可以看到一個人.
可以在那裡面.
就可以看到一個人.
可以在那裡面.
就可以看到一個人.
可以在那裡面.
就可以看到一個人.
可以在那裡面.
就可以看到一個人.
可以在那裡面.
就可以看到一個人.
可以在那裡面.
就可以看到一個人.
可以在那裡面.
就可以看到一個人.
可以在那裡面.
就可以看到一個人.
可以在那裡面.
就可以看到一個人.
可以在那裡面.
就可以看到一個人.
可以在那裡面.
就可以看到一個人.
可以在那裡面.
就可以看到一個人.
可以在那裡面.
就可以看到一個人.
可以在那裡面.
就可以看到一個人.
可以在那裡面.
就可以看到一個人.
可以在那裡面.
就可以看到一個人.
可以在那裡面.
就可以看到一個人.

$^{1201}$可以在那裡面.
就可以看到一個人.
可以在那裡面.
就可以看到一個人.
可以在那裡面.
就可以看到一個人.
可以在那裡面.
就可以看到一個人.
可以在那裡面.
就可以看到一個人.
可以在那裡面.
就可以看到一個人.
可以在那裡面.
就可以看到一個人.
可以在那裡面.
就可以看到一個人.
可以在那裡面.
就可以看到一個人.
可以在那裡面.
就可以看到一個人.
可以在那裡面.
就可以看到一個人.
可以在那裡面.
就可以看到一個人.
可以在那裡面.
就可以看到一個人.
可以在那裡面.
就可以看到一個人.
可以在那裡面.
就可以看到一個人.
可以在那裡面.
就可以看到一個人.
可以在那裡面.
就可以看到一個人.
可以在那裡面.
就可以看到一個人.
可以在那裡面.
就可以看到一個人.
可以在那裡面.
就可以看到一個人.

$^{1241}$可以在那裡面.
就可以看到一個人.
可以在那裡面.
就可以看到一個人.
可以在那裡面.
就可以看到一個人.
可以在那裡面.
就可以看到一個人.
可以在那裡面.
就可以看到一個人.
可以在那裡面.
就可以看到一個人.
可以在那裡面.
就可以看到一個人.
可以在那裡面.
就可以看到一個人.
可以在那裡面.
就可以看到一個人.
可以在那裡面.
就可以看到一個人.
可以在那裡面.
就可以看到一個人.
可以在那裡面.
就可以看到一個人.
可以在那裡面.
就可以看到一個人.
可以在那裡面.
就可以看到一個人.
可以在那裡面.
就可以看到一個人.
可以在那裡面.
就可以看到一個人.
可以在那裡面.
就可以看到一個人.
可以在那裡面.
就可以看到一個人.
可以在那裡面.
就可以看到一個人.
可以在那裡面.
就可以看到一個人.

$^{1281}$可以在那裡面.
就可以看到一個人.
可以在那裡面.
就可以看到一個人.
可以在那裡面.
就可以看到一個人.
可以在那裡面.
就可以看到一個人.
可以在那裡面.
就可以看到一個人.
可以在那裡面.
就可以看到一個人.
可以在那裡面.
就可以看到一個人.
可以在那裡面.
就可以看到一個人.
可以在那裡面.
就可以看到一個人.
可以在那裡面.
就可以看到一個人.
可以在那裡面.
就可以看到一個人.
可以在那裡面.
就可以看到一個人.
可以在那裡面.
就可以看到一個人.
可以在那裡面.
可以看到一個人.
可以在那裡面.
可以看到一個人.
可以在那裡面.
可以看到一個人.
可以在那裡面.
可以看到一個人.
可以在那裡面.
可以看到一個人.
可以在那裡面.
可以看到一個人.
可以在那裡面.
可以看到一個人.

$^{1321}$可以在那裡面.
可以看到一個人.
可以在那裡面.
可以看到一個人.
可以在那裡面.
可以看到一個人.
可以在那裡面.
可以看到一個人.
可以在那裡面.
可以看到一個人.
可以在那裡面.
可以看到一個人.
可以在那裡面.
可以看到一個人.
可以在那裡面.
可以看到一個人.
可以在那裡面.
可以看到一個人.
可以在那裡面.
可以看到一個人.
可以在那裡面.
可以看到一個人.
可以在那裡面.
可以看到一個人.
可以在那裡面.
可以看到一個人.
可以在那裡面.
可以看到一個人.
可以在那裡面.
可以看到一個人.
可以在那裡面.
可以看到一個人.
可以在那裡面.
可以看到一個人.
可以在那裡面.
可以看到一個人.
可以在那裡面.
可以看到一個人.
可以在那裡面.
可以看到一個人.

$^{1361}$可以在那裡面.
可以看到一個人.
可以在那裡面.
可以看到一個人.
可以在那裡面.
可以看到一個人.
可以在那裡面.
可以看到一個人.
可以在那裡面.
可以看到一個人.
可以在那裡面.
可以看到一個人.
可以在那裡面.
可以看到一個人.
可以在那裡面.
可以看到一個人.
可以在那裡面.
可以看到一個人.
可以在那裡面.
可以看到一個人.
可以在那裡面.
可以看到一個人.
可以在那裡面.
可以看到一個人.
可以在那裡面.
可以看到一個人.
可以在那裡面.
可以看到一個人.
可以在那裡面.
可以看到一個人.
可以在那裡面.
可以看到一個人.
可以在那裡面.
可以看到一個人.
可以在那裡面.
可以看到一個人.
可以在那裡面.
可以看到一個人.
可以在那裡面.
可以看到一個人.

$^{1401}$可以在那裡面.
可以看到一個人.
可以在那裡面.
可以看到一個人.
可以在那裡面.
可以看到一個人.
可以在那裡面.
可以看到一個人.
可以在那裡面.
可以看到一個人.
可以在那裡面.
可以看到一個人.
可以在那裡面.
可以看到一個人.
可以在那裡面.
可以看到一個人.
可以在那裡面.
可以看到一個人.
可以在那裡面.
可以看到一個人.
可以在那裡面.
可以看到一個人.
可以在那裡面.
可以看到一個人.
可以在那裡面.
可以看到一個人.
可以在那裡面.
可以看到一個人.
可以在那裡面.
可以看到一個人.
可以在那裡面.
可以看到一個人.
可以在那裡面.
可以看到一個人.
可以在那裡面.
可以看到一個人.
可以在那裡面.
可以看到一個人.
可以在那裡面.
可以看到一個人.

$^{1441}$可以在那裡面.
可以看到一個人.
可以在那裡面.
可以看到一個人.
可以在那裡面.
可以看到一個人.
可以在那裡面.
可以看到一個人.
可以在那裡面.
可以看到一個人.
可以在那裡面.
可以看到一個人.
可以在那裡面.
可以看到一個人.
可以在那裡面.
可以看到一個人.
可以在那裡面.
可以看到一個人.
可以在那裡面.
可以看到一個人.
可以在那裡面.
可以看到一個人.
可以在那裡面.
可以看到一個人.
可以在那裡面.
可以看到一個人.
可以在那裡面.
可以看到一個人.
可以在那裡面.
可以看到一個人.
可以在那裡面.
可以看到一個人.
可以在那裡面.
可以看到一個人.
可以在那裡面.
可以看到一個人.
可以在那裡面.
可以看到一個人.
可以在那裡面.
可以看到一個人.

$^{1481}$可以在那裡面.
可以看到一個人.
可以在那裡面.
可以看到一個人.
可以在那裡面.
可以看到一個人.
可以在那裡面.
可以看到一個人.
可以在那裡面.
可以看到一個人.
可以在那裡面.
可以看到一個人.
可以在那裡面.
可以看到一個人.
可以在那裡面.
可以看到一個人.
可以在那裡面.
可以看到一個人.
可以在那裡面.
可以看到一個人.
可以在那裡面.
可以看到一個人.
可以在那裡面.
可以看到一個人.
可以在那裡面.
可以看到一個人.
可以在那裡面.
可以看到一個人.
可以在那裡面.
可以看到一個人.
可以在那裡面.
可以看到一個人.
可以在那裡面.
可以看到一個人.
可以在那裡面.
可以看到一個人.
可以在那裡面.
可以看到一個人.
可以在那裡面.
可以看到一個人.

$^{1521}$可以在那裡面.
可以看到一個人.
可以在那裡面.
可以看到一個人.
可以在那裡面.
可以看到一個人.
可以在那裡面.
可以看到一個人.
可以在那裡面.
可以看到一個人.
可以在那裡面.
可以看到一個人.
可以在那裡面.
可以看到一個人.
可以在那裡面.
可以看到一個人.
可以在那裡面.
可以看到一個人.
可以在那裡面.
可以看到一個人.
可以在那裡面.
可以看到一個人.
可以在那裡面.
可以看到一個人.
可以在那裡面.
可以看到一個人.
可以在那裡面.
可以看到一個人.
可以在那裡面.
可以看到一個人.
可以在那裡面.
可以看到一個人.
可以在那裡面.
可以看到一個人.
可以在那裡面.
可以看到一個人.
可以在那裡面.
可以看到一個人.
可以在那裡面.
可以看到一個人.

$^{1561}$可以在那裡面.
可以看到一個人.
可以在那裡面.
可以看到一個人.
可以在那裡面.
可以看到一個人.
可以在那裡面.
可以看到一個人.
可以在那裡面.
可以看到一個人.
可以在那裡面.
可以看到一個人.
可以在那裡面.
可以看到一個人.
可以在那裡面.
可以看到一個人.
可以在那裡面.
可以看到一個人.
可以在那裡面.
可以看到一個人.
可以在那裡面.
可以看到一個人.
可以在那裡面.
可以看到一個人.
可以在那裡面.
可以看到一個人.
可以在那裡面.
可以看到一個人.
可以在那裡面.
可以看到一個人.
可以在那裡面.
可以看到一個人.
可以在那裡面.
可以看到一個人.
可以在那裡面.
可以看到一個人.
可以在那裡面.
可以看到一個人.
可以在那裡面.
可以看到一個人.

$^{1601}$可以在那裡面.
可以看到一個人.
可以在那裡面.
可以看到一個人.
可以在那裡面.
可以看到一個人.
可以在那裡面.
可以看到一個人.
可以在那裡面.
可以看到一個人.
可以在那裡面.
可以看到一個人.
可以在那裡面.
可以看到一個人.
可以在那裡面.
可以看到一個人.
可以在那裡面.
可以看到一個人.
可以在那裡面.
可以看到一個人.
可以在那裡面.
可以看到一個人.
可以在那裡面.
可以看到一個人.
可以在那裡面.
可以看到一個人.
可以在那裡面.
可以看到一個人.
可以在那裡面.
可以看到一個人.
可以在那裡面.
可以看到一個人.
可以在那裡面.
可以看到一個人.
可以在那裡面.
可以看到一個人.
可以在那裡面.
可以看到一個人.
可以在那裡面.
可以看到一個人.

$^{1641}$可以在那裡面.
可以看到一個人.
可以在那裡面.
可以看到一個人.
可以在那裡面.
可以看到一個人.
可以在那裡面.
可以看到一個人.
可以在那裡面.
可以看到一個人.
可以在那裡面.
可以看到一個人.
可以在那裡面.
可以看到一個人.
可以在那裡面.
可以看到一個人.
可以在那裡面.
可以看到一個人.
可以在那裡面.
可以看到一個人.
可以在那裡面.
可以看到一個人.
可以在那裡面.
可以看到一個人.
可以在那裡面.
可以看到一個人.
可以在那裡面.
可以看到一個人.
可以在那裡面.
可以看到一個人.
可以在那裡面.
可以看到一個人.
可以在那裡面.
可以看到一個人.
可以在那裡面.
可以看到一個人.
可以在那裡面.
可以看到一個人.
可以在那裡面.
可以看到一個人.

$^{1681}$可以在那裡面.
可以看到一個人.
可以在那裡面.
可以看到一個人.
可以在那裡面.
可以看到一個人.
可以在那裡面.
可以看到一個人.
可以在那裡面.
可以看到一個人.
可以在那裡面.
可以看到一個人.
可以在那裡面.
可以看到一個人.
可以在那裡面.
可以看到一個人.
可以在那裡面.
可以看到一個人.
可以在那裡面.
可以看到一個人.
可以在那裡面.
可以看到一個人.
可以在那裡面.
可以看到一個人.
可以在那裡面.
可以看到一個人.
可以在那裡面.
可以看到一個人.
可以在那裡面.
可以看到一個人.
可以在那裡面.
可以看到一個人.
可以在那裡面.
可以看到一個人.
可以在那裡面.
可以看到一個人.
可以在那裡面.
可以看到一個人.
可以在那裡面.
可以看到一個人.

$^{1721}$可以在那裡面.
可以看到一個人.
可以在那裡面.
可以看到一個人.
可以在那裡面.
可以看到一個人.
可以在那裡面.
可以看到一個人.
可以在那裡面.
可以看到一個人.
可以在那裡面.
可以看到一個人.
可以在那裡面.
可以看到一個人.
可以在那裡面.
可以看到一個人.
可以在那裡面.
可以看到一個人.
可以在那裡面.
可以看到一個人.
可以在那裡面.
可以看到一個人.
可以在那裡面.
可以看到一個人.
可以在那裡面.
可以看到一個人.
可以在那裡面.
可以看到一個人.
可以在那裡面.
可以看到一個人.
可以在那裡面.
可以看到一個人.
可以在那裡面.
可以看到一個人.
可以在那裡面.
可以看到一個人.
可以在那裡面.
可以看到一個人.
可以在那裡面.
可以看到一個人.

$^{1761}$可以在那裡面.
可以看到一個人.
可以在那裡面.
可以看到一個人.
可以在那裡面.
可以看到一個人.
可以在那裡面.
可以看到一個人.
可以在那裡面.
可以看到一個人.
可以在那裡面.
可以看到一個人.
可以在那裡面.
可以看到一個人.
可以在那裡面.
可以看到一個人.
可以在那裡面.
可以看到一個人.
可以在那裡面.
可以看到一個人.
可以在那裡面.
可以看到一個人.
可以在那裡面.
可以看到一個人.
可以在那裡面.
可以看到一個人.
可以在那裡面.
可以看到一個人.
可以在那裡面.
可以看到一個人.
可以在那裡面.
可以看到一個人.
可以在那裡面.
可以看到一個人.
可以在那裡面.
可以看到一個人.
可以在那裡面.
可以看到一個人.
可以在那裡面.
可以看到一個人.

$^{1801}$可以在那裡面.
可以看到一個人.
可以在那裡面.
可以看到一個人.
可以在那裡面.
可以看到一個人.
剛才提到的上課的情況,督導小組一部分是在Zoom的,無論你住在那裡都是在Zoom的..
但是上課的時候,如果在Richmond Hill, Toronto,或者Malcolm,就是實體在電腦中心上課..
如果對於學科有任何的詢問,你們可以寫email給我,我已經寫了email address在這裡,或者打電話給我,也可以的..
我們特別將學科的定位放多了在公眾人士的旁聽,.
以及教務同工的旁聽方面,特別放寬..
希望這方面能夠吸引更多人關注..
事實上這些學科是不會常常開設的,.
只有這個時代有這樣的需要,我們才會offer..
特別教務同工如果是做旁聽的,我們只收他200元,.
31個小時的運作,只收他200元的旁聽費,實在非常便宜..
如果是收學分的同學,或者試讀,我們的program就是490元..
如果是公眾人士就這樣旁聽,就是300元..
歡迎大家到我們的網站詳細看資料,我們昨天已經將所有資料放在網站,供大家參考..
在這裡我看到有一個人有問題,他說江濤,溫多華未夠一年了,.
未看完,這個問題希望能夠回答..
教會忙於接待牧羊組織的階段來看是屬於蜜月期,教會有沒有預備下一階段的衝突期?.
預期有什麼衝突,如何避免這些衝突,或者預備面對?.
這個問題很尖銳,三位牧者誰想回答?.
有沒有預計到衝突這個問題?.
三位誰想回答?.
不要客氣,隨便誰..
其實對我們來說,就算在加拿大教會也經常遇到有藍有黃,也會有些匯種..
所以預防對我們來說只做一件事,首先表明立場,我們是椰絲,不是黃絲,不是藍絲..
第二,我想是教導本身也是重要的,所以我們總是會教會合一,.
主要是預防措施做得好一點,我們過去也是這樣做..
其他兩位呢?.
或者有沒有預計到衝突出現了?.
如何處理呢?.
我分享一點,今天在社交媒體,匯種他們的意見,看法,貼文,我們作教目,其實都不是在我們可以控制範圍之內的..
反而我認同吳牧師所說,在我移民事工委員會,我們會有一些大家一起去談成立的時候,.
當中我在移民事工委員會,我估計大家,因為他們有部分都在政府部門,甚至一些公營機構工作,.
所以他們會提出一個很好的觀點,大家可以去,至少在領導層面,就是叫做政治敏感度,.
例如在一些立場上,其實我估計在弟兄姊妹,我們都盡量先關心他們在移民的需要..
第二,我們亦多一些站於,我們都會在輔導的角度,和一些幫忙的,一些核心的成員,.

$^{1841}$到底我們去幫這些香港朋友的時候,有時候會說一些感受,我們先聽多一些,聽多一些他的感受,.
所以不要立刻有很大的反應或回應,我估計在我的移民事工委員會,他們這方面會做得比較同心,.
但是確實我們會接觸一些年輕人,他們會有一些很強烈的看法,.
但是我估計,我都會和一班教會的信徒領袖,因為我都會特別去關懷年輕人,.
所以我會特別去主動,亦是聽一下他們的心聲,我估計在一些感受上去表達,.
我估計在一些我稱為coffee time,即是和年輕人去到校園裡面,.
我會特別去關懷他們的適應,甚至有時領袖去表達,我估計他們會說一些他們的經歷,.
我估計我會作一個好的聆聽者去聽他們,然後一起轉化去祈禱,.
所以我發現反而是當你願意去先作聆聽的時候,因為他們都會看看,究竟你肯不肯聽我去講一下我的看法和感受,.
如果我們越能夠在教會裡面,特別是教會的無論是牧者還是領導的團隊,.
可以更願意以耶穌的恩愛去聆聽,然後去站於多一些他的角度去想他們的一些體會經歷,.
而不要立即太快去作一個有我們的看法,我覺得我們都在學習中,.
當然我也認同在教會裡面我們也要去點在愛的功課裡面,.
甚至耶穌所說的愛受敵的功課裡面,我們其實也會在講壇裡面去分享,.
不過不容易的,確實我們會在私底下在社交媒體裡面表達一些不同的意見,.
但是也沒有去到一個層面,變成面紅耳熱的情況在我學會裡面..
Ray 教授,有沒有什麼補充,衝突方面?.
我想到另外一個衝突,不過回應一下兩位牧者的回應,.
其實也對的,很多時候都像巴森·亞森利(Bastion Nathan Lee)所說,.
剛剛來到加拿大的弟兄姊妹,尤其是家人,要解決很多事情,.
家裡,車,小孩上學,工作等等,都沒有機會讓自己的傷口復原,.
多一些作為牧者也好,長子也好,都需要多一些聆聽,.
就像我太太,我18年來到加拿大,她也沒有親身經歷過社運,.
最瘋癲的時候,社運或是那個階段沒有親身經歷過,.
我更加不用說,因為我跟香港基本上沒有什麼關係,.
所以也不能夠體會到他們的傷,只能夠去聆聽和同行..
反而我覺得另外一個衝突,將來有機會是什麼衝突呢?.
就是新移民和原有的弟兄姊妹,有時功上或崇拜上的工作,.
或者是侍奉或敬拜的模式,很簡單的,.
如果傳統的事情,拿一本生命聖詩,.
香港教會基本上沒有用過生命聖詩,都是用現代化的詩歌,.
我教會也沒有用過生命聖詩,只是其中一個例子..
我教會有一個優點,就是有些人在97年之前來讀書,.
然後回到香港工作,結婚生子,再回來,.
這些弟兄姊妹,有些已經做了執事,有些已經做了長老,.
是幫助新一批香港移民的弟兄姊妹和舊一批的你,.
去融合在一起,作為一個橋,這樣就比較好..
如果貴教會有這樣的人,可能會發掘更多,.
很可能幫助新舊會眾的衝突..
多謝三位嘉賓的講完,.

$^{1881}$今晚來到我們當中,其實講了很多東西,.
我一直在寫,已經寫了一大張紙,.
可能有少許零碎,零碎得來也看到一些脈絡,.
好像我剛才所講,在新移民的事工上,.
其實大家都可以參與,不要浪費神給我們這個機會,.
很可能藉著這個機會,能夠令貴教會人數增長,.
質素可以有改進,是一個很大的祝福..
不單祝福你們的模樣,也祝福你們教會整體的發展,.
也祝福新來的移民,實在是非常好的..
如果我們看到,有人問起錄影,我剛才已經講了,.
今晚的講座是我們錄影過的,過一兩天之後,.
我們就會在YouTube裡面,大家可以自由去看,.
也歡迎你們轉發給其他的教務同工,可以參考..
今天晚上,其實我們已經講了很多東西,.
祝福大家在新移民的事工上,繼續再上一層樓,.
能夠蒙上帝的祝福..
我們加拿大建導中心,其實也盡我們的綿力,.
在新移民的事工上做一些事情,.
剛才已經講了,今年年底我們會協助推廣,.
一個香港建導神學院的出版,是加拿大版的,.
如何落實新移民事工的查經材料..
另外,5月的一個科目,新移民的模樣挑戰,.
大家請你們去我們的網站,詳細看那個course outline,.
就會知道的了..
不如我們請吳牧師為我們做一個祈禱結束,.
好嗎?吳牧師..
好啊,請我們一起同心祈禱..
彼此的學習,也深信主你的愛,.
在我們宗教當中,讓我們能夠加倍的,.
先從你那裡得著經歷,.
以及滿滿的得著主你自己的恩典和愛,.
彼此在教會當中,我們能夠化解很多的不同的意見,.
不同的底線,他們有不同的政治立場,.
但我們就深信,當我們在基督的愛裡面紮根的時候,.
我們能夠無論是新移民,無論是本地的會友,.
已經是很久在加拿大成長的,彼此的磨合,.
都能夠在愛裡面去互相的體諒,互相的學習,.
去認識之後,能夠更加同心的,.
作為一個屬靈的友伴,我們彼此的關係不再是分你我,.
而是真的可以是,讓友伴當中彼此同行,.

$^{1921}$主我們就求你給我們宗教目,.
有從你而來,屬天的智慧,.
真的幫助我們怎樣去幫助我們每一個來到我們教會,.
或者我們接觸我們社區的人,.
去能夠彼此的去建立起一種這樣的屬靈友伴的友情,.
以致我們能夠在主你的腦裡面能夠同行,.
我們今天天父,你給我們今晚的時間,.
我們祈禱感恩,奉靠主耶穌基督的名求,.
阿們,阿們..
好,我們今晚的聚會就到此為止,.
多謝大家的參與,願神祝福大家,.
OK,拜拜,拜拜..
多謝再一次,多謝三位講員,多謝你們..
\newpage



\section{}
\label{sec:B5n__dtTRhE}
\textbf{ABSCC 網上免費講座:YOUTH ALPHA - 青年福音新思維}
\newline
\newline
連結: \href{https://youtube.com/watch?v=B5n_-dtTRhE}{\texttt{https://youtube.com/watch?v=B5n\_-dtTRhE}} ~~~~ 語音日期: 2023-05-03
\newline
\newline
\hyperref[sec:3o4omcoTUB4]{\small{< < < PREV SERMON < < <}}
~
\hyperref[sec:index]{\small{[返主目錄]}}
~
\hyperref[sec:6]{\small{> > > NEXT SERMON > > >}}
\newline
\newline
$^{1}$好,歡迎大家今晚,我們在多倫多的時間是晚上8時半.
在地下地區的地方,甚至其中一位港員是在香港的.
又是白天,我也不知道該說早上還是晚上,晚安.
我是建造中心的負責人,陸昭明牧師.
讓我們簡單介紹一下加拿大建造中心是一個什麼機構.
因為未必每個人都知道.
進入到我們的會場,可能有些是新的朋友.
建造中心在加拿大的多倫多一個地方.
多倫多是一個很大的城市.
建造中心有兩個目標.
第一個目標是提供粵語的神學教育.
不同水平的神學教育我們都做的.
尤其是現在多了很多online的科目.
所以在整個加拿大的地區,不單是多倫多.
整個加拿大地區我們都經常有學生報道我們的科目.
甚至遠到澳洲都有.
所以第一個目標我們已經達到了.
第二個目標就是建造中心近一年我們特別關注到教務的需要.
我相信在今晚的聚會裡面有部分的參加者是教務的同工.
歡迎大家.
我們對教務希望能夠提供一些目標的支援.
今晚這個講題就是其中一個例子.
我們會就著加拿大華人教會裡面特別有興趣的一些題目.
關注到一些課題,我們會舉行一些像今晚一樣免費的講座.
希望能夠幫到大家.
這是我們兩個主要的目標.
請Benson放第二張slide.
這裡有我們的網站,abscc.org.
歡迎大家任何時間都可以上網站看到我們最新的公佈.
我們update都很密的,請大家以後都多點來我們的網站.
我們有Facebook的專頁.
我們最近一年多兩年特別在疫情底下.
很多的聚會都會放在我們YouTube裡面的帳號.
如果你在YouTube裡面打abscc.
全名叫Alliance Bible Seminary Centre of Canada.
你就會看到很多以往我們舉行過的聚會錄影.
任何時間都可以歡迎大家去看.
大概可以了解我們作為一個基督教教育的機構是做些什麼的.
歡迎大家常常來到我們當中.
這裡有一個email address就是我自己.

$^{41}$chillook@abscc.org.
歡迎大家有任何的查詢.
寫個email給我,我會盡快答覆大家.
有任何疑問,任何的建議.
有什麼我們做錯了的,你告訴我們.
我們都很歡迎的.
讓我講講今晚的題目的背景是什麼.
請Manson看第一張slide.
今晚的題目是Alfred Hubert青年福音的新思維.
為什麼會有這個講題呢?.
主要的原因就是最近我們做了不少與新移民有關的課題.
在基督教中心來說.
特別是進入到2023年的時候.
我還記得1月的時候我們舉辦了一個講座.
專門給教務的.
講到兩文三語裡面的教會的服事.
3月的時候,剛好對上一個月左右.
我們又講到一個叫做非常移民的牧羊一個專題的講座.
講到一些教會,不是很大的教會,通常是很小的教會.
或者是處於很偏遠的地帶.
他們怎麼做新移民的工作呢?.
我們又舉行了一個講座.
甚至是明晚開始.
我們有一個課程是講到新移民的牧羊.
我是一個總召集人.
這個題目新移民的牧羊.
其實是一個很闊的一個大的標題.
我們察覺就是.
現在在香港移民過來.
就是講廣東話移民過來的弟兄姊妹或者朋友.
如果進入到教會的時候.
另外一個很熱門的話題就是.
他們的子女.
甚至他們自己本身都是屬於比較青年的情況.
究竟教會如何去將福音傳給一些未信的青年呢?.
又或者教會的年輕的會眾.
牧羊這一班年輕會眾的時候有什麼需要注意的地方呢?.
這個就是今晚我們講題的背景.
以往我的了解就是.
在我很短的牧會的過程裡面.

$^{81}$我也很了解教會對青年的侍工.
很簡單就是請一個年輕的牧羊.
找一個年輕一點的牧羊工作的年輕傳道.
似乎就搞定了.
其實如果你有留意一下.
是絕對搞不定的.
情況不是這麼簡單.
因為新移民過來的一班青年.
或者是移民過來的一班青年帶動他們的小孩過來的時候.
其實已經是跟加拿大文化背景下的.
都是講廣東話的人是有很不同的情況.
所以孕育了我們有一個心意就是.
探討一下有什麼新的思維可以跟大家分享.
刺激了我們對於以後的這種福音工作多加注意.
我們今晚請了兩個嘉賓.
這裡說就是麥浚思先生.
就是香港Alpha這個課程機構的一個同工.
他就是負責Alpha Youth的影片的主持人.
他最適合分享Alpha Youth這個課程.
是如何幫助教會能夠好好的做好向青年傳福音的工作.
在這裡我略略提一提.
加拿大建造中心不是說一定要用Alpha Youth才可以向年輕人傳福音.
沒有這個意思.
不過我發覺Alpha Youth這件事很可能提供了一些刺激給我們.
刺激我們在思維上再進一步.
所以我們今次就請麥浚思先生來.
你叫他Matthias就行了.
Matthias很容易記的.
他一會兒會自己說為什麼會叫Matthias.
Matthias就會說Alpha Youth這個課程是如何幫助到向年輕人傳福音.
我們也請了譚子信牧師.
譚牧師就是加拿大Talgary的地方.
South Talgary的播道會.
粵語部分的一個主要人物.
為什麼會請譚牧師呢?.
主要原因就是一方面一看就知道他做青年工作.
看樣子就知道.
譚牧師原來在他的教會裡面用了Alpha Youth.
在implement之後有很大幫助.
我相信是一個很成功的案例.

$^{121}$請他分享一下有什麼獨特的地方.
一個是理念,一個是實踐.
一會兒先請Matthias分享他的部分.
然後譚牧師分享他的部分.
然後我會和這兩位嘉賓有些對談.
有幾個問題讓他們分享一下.
都是圍繞一個青年福音的工作的部分.
與此同時我們也開放一些Q and A的時間.
大家可以寫出一些問題.
用這個chat room.
大家用慣Zoom就知道chat room是如何使用.
你打出一些問題我們看得到.
然後我就交給Matthias或譚牧師Jason.
讓他們解答.
我們大概會在東岸時間.
即是多倫多時間大概9點45分.
我們基本上就完結了.
我們會請Matthias和譚牧師總結一下.
究竟從他們的角度來看.
處理青年福音的工作關鍵在哪裡.
讓他們再reiterate一次.
讓大家有深刻的印象.
他們說完之後.
我最後會說說加拿大建造中心在我的構思之下.
再有什麼後續的事情可以繼續做.
讓大家可以思考一下.
希望今晚的聚會是很豐富的.
讓大家能夠思想上有一個啟發.
能夠對青年福音的工作裡面.
能夠有一些新的思維出來.
造福教會.
我們都是一個這樣的期待.
不過還沒開始之前.
幫我們做一個祈禱開始.
請大家低頭我們有一個祈禱.
主席我們謝謝你.
今天晚上我們在網上進行這個講座.
實在體會到青年福音的工作的需要.
我們有這麼多弟兄姊妹.
甚至教務同工.

$^{161}$進入到這個會場的時候.
其實大家對這個事工都有熱切.
也都很想能夠辦得更加好.
更加識切青年福音的需要.
願主你自己使用Matthias.
使用Jason.
在這兩位嘉賓講完.
分享了之後.
我們能夠在對談當中.
能夠擦出一些火花.
好讓你能夠使用我們.
更加在前衛.
更加在前面已經做了這個工作的.
我們就可以在一些什麼情況下.
可以有些借鏡.
求主你祝福今晚的聚會.
特別祝福我們的轉播能夠順利.
好讓更加多人能夠很開心的.
能夠在這個半小時內得到最大的祝福.
我們簡單祈禱.
奉耶穌基督的名頭.
阿們.
這個時候我們就將時間交給Matthias.
請Matthias這個時候是你的.
OK, Go ahead.
Hello, 大家好.
各位目者,各位同工大家好.
我是Matthias.
我像剛才陸牧師介紹.
我是香港啟發的義務同工.
因為過去我大概有八至九年的時間.
我都是全職在Alpha(超級)裡面去服事的.
這段時間也累積了很多不同的經驗.
很多體會,很多成長.
簡單做一點自我介紹.
我自己就是一直都在香港裡面去成長的.
我父母一個是目者.
一個是傳道人.
所以他們過去.
我都是一個信義代的背景.

$^{201}$小時候已經在教會去成長.
翻書,學習,聽聖經故事.
這些都是最基本會成長到,接觸到的東西.
亦都是青年的時候.
可能教會亦都會有青少年崇拜.
會在當中去服事,去拿經驗.
但去到真是高中的部分的時候.
我都在教會裡面經歷過一次Alpha(超級).
令到我重新有一個有系統的方法.
去讓我可以了解到整個基督教的信仰.
Alpha(超級)一會兒我會再講多一點.
究竟它是一個怎樣的東西呢?.
以致到為何我可以作為一個信義代.
聽了這麼多年聖經故事.
我居然都可以透過這個有系統.
十多個星期的Alpha(超級).
去覺得信仰裡面有重新重整的機會.
我的信仰能夠有一個新的體會呢?.
剛才亦都說了.
究竟Matches(Mathias)你為何會叫Mathias呢?.
其實原因都很簡單.
其實我父母當然都會有一個英文名給我.
但我不是不喜歡.
但我亦都很羨慕一些坊間的人.
例如他英文名叫William.
原來他中文名叫韋林.
他中英文名是有些差的.
所以我最後就將我的中文名的讀音.
英文拼音Matches讀得快一點.
就變成Mathias.
這個就是我為何會叫Mathias的原因.
因為我希望我的中文名和英文名.
發音上是有些差.
另外我都很想像火柴人一樣.
是為主來燃燒的.
只有一根柴很快就燒完.
所以眾數+ES.
就是可以燒久一點.
事不宜遲,我都和大家分享一下.
關於Alpha究竟是什麼.

$^{241}$我有一個PowerPoint.
如果大家一會有什麼問題.
或者一直聽的時候覺得有些東西.
好像不太掌握到.
都可以在留言區打低.
或者在Q and A的時間.
希望都可以回應到.
究竟Alpha是什麼來的呢.
Alpha我們希望可以從.
如果是堂會的身份.
或者是個別用家的身份.
我們希望可以和你走過一些不同的歷程.
開始探索一下認識一下Alpha是什麼.
以致令你有興趣.
或者你有實際的需要.
我們一起有實戰的經驗.
繼而透過不斷和我們Alpha Hong Kong有聯繫.
或者是我們Alpha不同國家的一些總部.
不同國家的分部.
和我們的同工一起去了解.
盡心實踐的時候.
以致我們可以有不斷的進步.
以致我們的Alpha能夠有一份活力.
我自己覺得Alpha是一種.
將教會的群體回歸.
好像初期教會.
很單純的信徒互相彼此.
透過交流,互動.
有經驗的一些信徒.
或者有經驗的牧者.
能夠有一個互動的關係.
一起成長的信仰群體.
是一個活力的,是一個有機的.
又不是一個像教書式那樣.
一對many的方法.
為什麼我們要傳福音呢.
我們用風扇吹過.
大家都很明白.
其實我們都很想將神的愛.
耶穌的愛.

$^{281}$聖靈的恩賜.
能夠用一個別人明白.
體驗得到的方法.
來將這件事和別人分享.
更加重要的是.
神在這個時代.
給了教會這個信仰群體模式給我們.
讓我們能夠在這個信仰群體下實踐.
我相信.
基督教不是一個個人修煉的信仰.
而是透過你自身在某一個群體.
某一個對象下.
將在信仰裡學習到的.
將耶穌所教導的東西.
能夠在群體裡實踐.
所以Alpha也會是一個.
希望有很多實踐.
有很多互動.
和讓你和讓新朋友.
自身在一個群體下.
來產生的一個關於福音向導的活動.
究竟Alpha是甚麼呢?.
讓我播一段影片.
讓大家了解一下.
謝謝大家.
要和人談談有關人生信仰和耶穌.
可能很困難.
但有趣的是每一個人到了某些時候.
都會在一些人生重大問題上掙扎.
例如人生有甚麼盼望.
有甚麼目的,意義,如何找到愛.
試想想.
如果我們營造一個空間.
讓我們身邊的人.
例如朋友,鄰居,同事.
可以一起真誠,沒有壓力地聊天.
組長不用有所有的答案.
每個人都可以問一些比較複雜的問題.
或者坦白地分享他們的想法.
這就是啟發.

$^{321}$啟發是很多年前在一間倫敦的教會開始.
一個簡單的理念去吸引那些未必常回教會的朋友.
很多人的生命有轉變.
而啟發也發展到世界各地.
今天你會在辦公室,學校,咖啡店,教會,監獄,家.
甚至在網絡上找到有人在開辦啟發.
全球超過100個國家.
超過100種語言.
以千萬計的人已經體驗過啟發.
那啟發是甚麼來的呢?.
啟發是一系列互動性的聚會.
讓人探索信仰的基本內容.
每一次的聚會都會有時間一起聊天.
聽一個短講.
還有在小組討論中分享疑問和看法.
無論是面對面一起喝杯茶,吃頓飯.
還是隔著螢光幕聊天.
都是想有個時間讓大家一起輕鬆和建立友誼.
短講從基督信仰角度去解答一些.
有關人生和信仰的核心問題.
討論時間是讓參加者在一個不怕被糾正.
或者批評的環境下.
更加深入探索這些課題.
這些環節全部都在一個輕鬆,有趣.
每個人都被歡迎的環境裡進行.
短講方面主要有六套影片系列可以選用.
啟發新世代系列,啟發肖青系列X.
華人啟發視頻,啟發影片系列全文版.
啟發影片系列青年版,啟發課程金力黑版.
每一套影片系列都是為不同的觀眾而度身訂造.
一般為期8至12個星期.
包括一次啟發週末.
在那裡大家可以有機會經歷音樂敬拜.
和還有祈禱的時間.
啟發亦有你需要的資源.
去協助其他人一起參與.
例如是討論指引和培訓短片.
去裝備你和你的團隊.
所有短講和其他資源都可以在網上下載.
而且費用全面.

$^{361}$透過開辦啟發.
你就會營造一個空間.
去讓人互相認識.
而且去認識人.
登記,啟動.
今日就開辦啟發.
透過簡單的短片.
相信大家對Alpha有一個基本的認識.
和基本的概念.
接下來的時間.
我會再講多一點Alpha實際是怎樣的.
我們很希望.
好像我剛才所說.
我們是一個用互動的方法和互動的空間.
去讓新朋友或對信仰有興趣的朋友.
能夠他們一起來有一個簡單的交流時間.
我們很希望從單向的信仰.
過去一些會用的報道會.
或者一些回來教會活動的方法.
能夠變成一個.
其實他都有機會去表達的方法.
我不是說那些方法不OK.
但其實這個世界都有很多種不同的人.
百樣米養百樣人.
所以我們都希望教會能夠有不同方向.
或者不同面向的一些活動.
或者是能夠讓新朋友.
能夠感覺到自在自如的一些活動.
我相信Alpha是教會可以選用的一種.
另外,除了解釋我們的信仰之外.
我們都希望可以在這個群體裡面.
去創造一些他們能夠體驗和經驗的機會.
繼而就是我們從很多家裡.
或者是社區裡面分散了的地方.
我們可以透過聚集來經歷那份合一.
我們各人有不同的生命歷程.
我們各人有不同的看法.
但是當我們這些相似的看法.
或者是不同的看法撞擊在一起的時候.
究竟我們在一個信仰群體裡面.

$^{401}$我們是可以怎樣去互動的呢?.
這個就是Alpha裡面.
大家有機會會經歷得到的東西.
我們Alpha是很重視有四個R.
這四個R是在我們的Alpha的影片.
在我們裡面.
我們可能之前會一起聊天,吃飯.
或者是後面的一些討論環節.
這些我們都希望教會能夠一起去擁有這些元素.
這些元素究竟是甚麼呢?.
就是Real,我們很重視真實的.
很重視關係的.
我們是依靠聖靈的帶領的.
另外就是Alpha這東西是可以持續發展的.
因為它有一個很清楚的一些的.
可能是syllabus或者是一個module.
大家去實踐出來的時候.
大概都可以能夠有不同的人都相信很容易掌握得到.
當然大家都可以加入不同的元素.
一會兒Jason牧師可能都會分享一下.
他的堂會在基本的內容和元素底下.
他究竟加了些甚麼進去呢?.
以致到他這個Alpha都可以成為了很適合他教會文化.
很適合他社會,社區.
可以接納到社區人的一種模式呢?.
一會兒可能Jason牧師都會分享一下.
這四個元素.
其實我們今天集中講關於青少年.
其實青少年是更加重視頭兩個R.
Real和Relation.
其實他們都是很重視關係.
可能大家會很快覺得年輕一代只顧著上網.
顧著打機.
好像把自己也背起來.
但其實他們不會覺得網絡上的世界的自己和現實生活的自己是不同的.
他們都會當自己是同一個人.
他們去玩遊戲也好.
或者在網絡上做不同的社交媒體去展現他們自己.
其實他們都是很想將真實的自己的其中一面去展現.
另外亦都是希望透過互動去賺取關係.

$^{441}$Alpha我們這裡所有的核心元素,核心的理念.
我們正正回應年輕人在這方面的需要.
另外就是他究竟放在了信仰群體裡面.
究竟是一個怎樣的體會呢?.
究竟Alpha真真實實會是怎樣的呢?.
但在我一直講的時候.
大家可以想一想這個問題.
或者在留言區裡面去分享.
其實你很想你的堂會能夠成為一個怎樣的堂會呢?.
我覺得這個都是我們當我們去想教會有很多活動.
我們安排或者策劃整個牧養系統的時候.
我們需要想的一樣東西.
如果你未想過或者曾經想過.
歡迎你去想一想究竟你的教會成立和運作的時候的初衷.
或者是你的願景究竟是什麼.
在一間新朋友來到的時候.
都會感覺到舒適的教會.
還是信仰神的話語是非常非常扎實的教會.
這些都會影響到你去想的活動.
他們的面向是會有些不同的.
大家可以去想一想.
剛才我已經講了.
Alpha是一個很好的平台去展現基督教信仰.
它是一個想法.
因為它裡面是幫助教會去轉化.
或者是啟動一些新的文化.
幫助整個堂會變得更加有活力.
是其中一個方案.
另外Alpha也是一個工具.
是很實踐地幫助堂會去面向新朋友.
我們可能有很多東西.
在我們基督教信仰裡面.
我們都覺得重要的.
可能我剛才講的神的話語.
我們要忠於聖經.
我們要把好的教的理念傳遞給新朋友.
他們不能信有差錯.
我覺得這些都是重要的.
當你再去想一層.
我們這個活動是面向新朋友的時候.

$^{481}$可能有些元素我們就要降低.
有些元素我們會提升.
究竟新朋友對於教會的活動.
他們有什麼前節,有什麼想法.
我們可能就要想清楚.
剛才大家都會想.
Alpha如何有系統地將信仰的基本內容.
和新朋友分享呢.
大家會看到Alpha的基本內容.
無論是我們面向什麼年齡層的觀眾.
而製作的影片系列.
基本內容都是接近的.
我們會分三個部分左右.
第一個部分其實已經將所有基本的信仰內容.
跟大家分享.
究竟有一個introduction.
耶穌是誰? 為什麼會死?.
如何去信耶穌.
其實這裡已經將一個最基本的核心的福音內容.
頭三集,頭四集已經跟大家分享了.
另外會講究竟基督徒這個群體常常會做的.
祈禱,讀經,我們去求問神的指引.
究竟這些方法或這些東西是什麼呢?.
第二個部分我們希望會比較輕鬆.
平時在教會的環境裡進行.
第二個部分我們會有一個Alpha Weekend.
可能大家開放期間約弟兄姊妹的家.
比較舒服,輕鬆.
或者找一個retreat camp來做.
就是講關於聖靈是誰和聖靈會做什麼.
這個部分相信很多堂會過去.
少少認識Alpha.
都會在這個地方去探討.
因為知道不同的宗派對於聖靈的教導.
是會有少許不同.
所以他們除了可以預先看影片.
覺得適不適合之外.
其實都可以拿裡面的核心內容.
或者拿script去做一些調整.
如果影片不太合適你們宗派對於聖靈的教導的話.

$^{521}$也可以透過一些剪輯.
或者你們做live talk來表達.
第三個部分就是關於信徒.
當他們進入信徒群體的時候.
他們和世界面向的時候.
那些張力,那些他們要面對的事.
究竟我們怎樣去處理呢.
可能關於邪惡,魔鬼,煞騰的孕幼.
關於苦難,醫治的問題.
關於教會和傳福音的事.
關於他信了耶穌的時候.
怎樣實現信徒生命有關的事.
就會在第三個部分處理.
很多人都問,那些堂可以剪輯.
run D或不run D.
或者可以調動.
今天官方的答案當然就是.
我這麼努力一起製作這麼多條影片.
都很希望大家有機會用得到.
但如果時間上有限的話.
第一個部分和第二個部分.
都是很希望大家有機會run得到.
因為這個都是最核心的.
就是聖父,聖子,聖靈的一些內容.
就會在頭一,二部分處理到.
第三部分那些都可以做一些選取.
或者調動次序都是可以的.
剛才我的題目是一個.
面向即青或大約三十多歲的人.
來製作的新世代系列的一些題目.
這個X就是Alpha Cantonese Teen Series.
這個名字真的很簡單.
但也很直接.
就是將我們的福音傳給年青人.
這個系列拍攝的內容和題材.
和選取的內容.
都會更加貼合青少年人.
大家的題目都是差不多.
但有什麼不同呢?.
就是根據不同年齡層的需要.

$^{561}$我們在內容和例子都會有些許調整.
例如青少年這個系列.
我們會用很多青少年有關的生活的情境.
可能是學校,可能和家庭的關係.
等等.
來做一些例子或演繹方法.
來幫助他們對影片的內容有共鳴.
即青那套可能會加多一些職場.
可能在人生階段.
可能要想要不要結婚.
愛情這些元素.
可能都會加進去.
所以這個就可以想想.
你的群體究竟是怎樣呢?.
所以我們會選取的.
究竟真真實實有什麼分別呢?.
今天我都選取了一些精華片段.
讓大家了解一下這兩套系列.
究竟是怎樣的呢?.
我們首先可以看這個.
少青系列.
少青系列都是近年由Alpha Hong Kong製作的.
可能有些人會問.
我教會不是那麼廣東話.
有沒有其他選擇呢?.
一會兒我都會介紹.
以致大家可以用那些題材.
用那些影片的內容.
來接觸新朋友.
就這樣.
以上言論不代表本台立場.
市場要戰場.
要解決對手.
我們就需要.
天能力.
臨急抱隻腳.
之大全.
你在這裡做什麼?.
你在洗牌嗎?.
我在吸收這本書的精華.

$^{601}$震懾之氣場.
唉,沒運輸.
不過不要緊.
最重要是氣場.
老師,加紙,謝謝.
兩張.
三張,謝謝.
你還是把所有紙都給我吧.
無用紙攻勢.
我算到了.
1+1=3.
時間停止,可以做很多事.
睡多久都可以,沒所謂.
意念控制.
用意念可以做很多事.
那你就可以懂得飛.
我會選擇時空穿越.
因為回到過去就可以修正我以前的錯誤.
如果祈禱都是超能力.
那一定不是基督徒的專利.
世界各地有很多人.
都有試過祈禱的經驗.
講自己的需要.
例如健康,學業等等.
當我們面對危急關頭.
我們自己沒有辦法的時候.
我們就會求誰去保佑我們.
或者打救我們.
有些人就算沒有宗教信仰.
當他面對一些自己都控制不了的環境的時候.
他們就會嘗試祈禱.
所以祈禱真的一點都不陌生.
但基督徒的祈禱有什麼分別呢?.
剛剛這個展現了青少年系列的影片的味道.
讓大家看到青少年.
特別是剛剛在香港背景下.
來到加拿大或是不同地方的年輕人.
他們都很容易有共鳴的場景.
接著再看一看.
關於這個即稱系列.

$^{641}$其實內容的感覺大概是怎樣呢?.
因為時間所限.
我選一個小段給大家看看.
如果有一天.
有人送了一輛車給我.
停泊在我樓下.
我第一時間衝過去.
打開車門.
伸手進入車頭的箱子裡.
拿出一本白雪雪的說明書.
很興奮地逐頁逐頁揭.
還拿回家.
每天一早起床和臨睡前都拿出來看.
然後看看.
我拿出螢光筆.
打亮我最喜歡的部分.
接著我心想.
要背熟一點才行.
所以就抄了幾句出來.
貼在鏡子上.
一邊刷牙一邊背.
還不止這樣.
我還拿出我的結他.
用說明書裡的內容作出幾首歌.
然後我心想.
可能我應該學點日文.
那我就可以用圓圓來看說明書.
這樣就可以原則了.
如果我做齊了以上那麼多東西.
但唯獨沒有開過車的話.
那我就錯過了重點.
說明書的目的是為了幫助我.
按照原本設計的概念來開這輛車.
同樣有時候當我們看完整本聖經之後.
仍然有機會捉到鹿脫不了角.
錯過了最重要的東西.
聖經的主角是耶穌.
聖經的目的是為了幫助我們和耶穌建立關係.
好,那麼大家都大概掌握到.
其實我們Alpha裡面不同的影片.

$^{681}$他們的特色大概會是怎樣.
我覺得用汽車說明書比喻我們看聖經.
其實我覺得是非常有趣.
每一次看我都會心微笑.
好,希望大家上我們網站.
大家可以了解多一點.
你想看完全部東西才決定.
用不用Alpha我們都會有方法.
一會兒我會介紹一點.
我們透過一些問卷調查.
發現年青人裡面有很多問題.
不是問題的問題,而是很多疑問.
對世界有很多疑問.
他想不到自己怎樣和世界連結.
這些都是我們收集了一些歸類了的意見.
我們希望能夠透過一個有信仰價值的群體.
和他們一起成長.
所以我們就會有Alpha.
一些調查的背景發現.
自稱基督徒的年青人.
有57\%已經離開教會.
但其實在問卷裡也發現.
有44\%的他們.
是發覺信仰對他們來說是重要的.
為何他們雖然離開教會.
但對於信仰又覺得是重要的呢?.
這個我們就要問了.
是不是教會本身是有些框架.
有些模式,或者文化.
和他們本身有一些張力.
他們不能夠留在那個群體裡.
但他們依然很想尋求.
在信仰上,在超自然的上.
怎樣和他們的人生有連結.
這個就是我們可能需要想的東西.
究竟我們教會群體有沒有一些東西.
我們可以改變,或者變得活力.
以致我們可以回應到青少年的需要呢?.
所以我們Alpha這個影片系列.
我們都希望能夠製造一個空間.

$^{721}$讓他們能夠探索自己和神的真實關係.
和自己的真實關係.
和朋輩人的真實關係.
他們在一個自由的空間裡去探索基督教信仰.
很簡單地用三個H.
我們很希望他們能夠在耶穌裡找到盼望.
很希望有一個安全的環境被聆聽.
他們有機會發聲.
聽到別人的聲音.
自己也有分享的機會.
另外就是感覺到一個有互動,有愛的.
好像家一樣的群體.
我相信這些都是全世界的年輕人.
他們很渴望得到的元素.
以致Alpha都是回應著這些焦點的.
剛才分享了兩套系列.
大家可以簡單地用.
我們會有一個.
如果是外國人的背景.
我們有Alpha You Series.
如果是成年人的,我們有Alpha Theme Series.
這些都是剛才影片裡大概感受到的東西.
因為時間不多,我會跳一點來講.
我們很希望Alpha能夠幫助教會去掌握一些改變文化.
或者是一些實踐基督教的意念的時候.
我們都可以進步或是一起提升的東西.
因為要一起提升.
可能要牽動整間教會.
或者是細細地搞也可以.
其實我們很多不同的部分.
都可以有弟兄姊妹去參與的.
不同信仰歷程的弟兄姊妹都可以參與.
Alpha在中間,我們可能需要有人負責物資.
外國我知道可能會有車隊去幫忙接送.
可能準備食物.
可能是搞Online Alpha的時候.
可能需要IT Support.
可能有叔叔姨姨負責做關顧.
或者是哥哥姐姐負責陪伴.
陪他們成長,負責帶走.

$^{761}$這些不同元素的弟兄姊妹.
其實他們都可以有機會有部分參與在Alpha裡.
去建立空間讓新朋友能夠好好體驗信仰.
很快地,如果在教會裡應用Alpha.
可以簡單舉辦一些的餐會.
跟他們一起看影片,一起討論.
這是最基本的模式.
我們香港也有很多堂校合作.
在學校裡有一位傳導人.
其實每一次關心年輕人.
他四五個左右,他就立刻跟他們去Alpha.
小小的,細細的.
雖然用的時間多了.
但他們交流的時間質量更加好.
所以他們很珍惜這些機會.
同步時間,可能星期一開一組.
星期二開一組.
星期六早上可能開一組.
然後下午就帶他們一起回團契.
所以其實有很多模式可以選擇.
給大家一些啟發.
有些人就拿著iPad在cafe裡開Alpha.
他們一起去行山,在山上一起看Alpha.
或者是搞一些活動,遊戲日.
跟他們一起玩完,建立關係之後.
就開Alpha.
其實我覺得其中一個部分就是.
有些common interest.
或者是令到他們在關係上.
能夠互相建立.
其實那些東西都可以代替吃東西的部分.
關係是最重要的.
有關係下就帶他們去認識耶穌.
在認識耶穌的時候.
他們有一個自由的空間能夠討論.
不只是你講的.
因為這都是新世代或後現代弟兄姊妹.
他們很想找到的知識.
而不是純粹接收的知識.
他們希望在探索的過程中認識到信仰.

$^{801}$我們會有一個比喻.
有些人就如你去買iPhone.
因為這個都很貴的.
很多cost.
所以他要多點時間去了解.
可能他會去電器店試電話.
認識一下,去一下論壇,開一下YouTube.
看看人們開箱,看看人們介紹電話.
甚至按Apple的發佈會去認識.
最後才去做購買的行動.
我相信Alpha很適合這一類.
需要多點時間去探索.
因為他可能聽了別人的見證.
他可以體驗到信仰.
用一個像旁觀者的角度.
體驗一下這群基督徒究竟做了什麼.
這群人聚在一起.
究竟玩的遊戲是什麼.
為什麼會讀聖經.
為什麼會祈禱.
當他了解過後的時候.
他可能會嘗試將他的生命.
押在,放在主耶穌那裡.
所以這個其實對於教會來說.
是一個信仰和新朋友.
生命的投資.
未必一試就會有效果.
可能我們需要等一等.
短則十多次的歷程.
或者透過第一次Alpha.
第二次Alpha.
第三次Alpha.
才有機會順利.
所以在這裡鼓勵弟兄姊妹.
鼓勵同學們.
我們可以積極去嘗試.
好,最後我用多幾分鐘去演示一下.
究竟我們怎樣去拿到影片.
或者我們的網站.
怎樣讓大家拿到Alpha這些資源.

$^{841}$如果大家看到的就是.
我們Alpha Hong Kong的網站版面.
其他部分大家可以自己探索一下.
這個時候就可以按登記課程.
或者登入.
網頁就會跳到這個地方.
就會有Alpha的按鈕.
或者是我們有個婚姻課程的系列.
這個我們今天就不詳細說了.
按Alpha的東西.
Alpha的平台.
我們就會跳到Alpha的網上資源平台.
我先按登出.
因為想和大家一起用那個步驟.
去了解一下.
就會有一個登入的版面.
如果沒有帳戶.
就在這裡按Create帳戶.
這些基本流程.
相信大家都很容易掌握得到.
按登入.
登入到的時候.
就會看到我們網上平台的一個版面.
這個版面很簡單.
你找一些紅色就可以了.
我就想開辦Alpha.
可能我網頁的字不是太大.
我今天放大一點.
按開辦Alpha.
我們這裡是沒有AI程式的.
我們只是純粹邀請你去回答幾條問題.
讓我們能夠透過這些filter.
去介紹一個最適合大家的Alpha內容.
例如我想開給青少年的.
這個是一個現場實體的Alpha.
我想用影片的.
這個Alpha就在教會裡面進行的.
按完這些內容.
我想用粵語的.
按完這些簡單的問題.

$^{881}$我們背後的程式就會將一個最建議的系列給你.
例如我用了X.
這裡很簡單.
我今天按了Test.
因為這個是一個測試.
所以就按了Test.
但你可以自己命名你的啟發.
然後就選你開辦Alpha的日子.
打一些時間.
將你的地址打進去.
我現在是香港.
所以我才打香港.
這裡也有些remark給你打.
另外就是究竟你是屬於什麼教會的呢.
這裡可以打回一堂會去找回你自己的堂會.
我自己現在回的是一間叫做Crow Church的堂會.
所以我就這樣打.
不過現在是試的.
我就選我這個Alpha是和教會無關的.
然後我就開辦啟發了.
開辦完啟發之後.
其實我都是說一兩分鐘就完成登記.
就會看到這個課程的版面.
大家第一時間通常會問.
影片怎麼下載呢.
就是按下載資源.
這裡就可以找到影片.
我們會提供一些對攻訓練的影片.
每條大概五至十分鐘左右.
有些基本關於Run Alpha的時候.
你可以下載下來.
點擊它.
可能下載你想要HD版本的.
下載下來.
你可以叫你的對攻一起去看這些訓練短片.
又可以.
或者你是分享一個連結給他看.
在網上看又可以.
然後那些基本的十三條骨幹影片就在這裡.
都是按個鍵就下載得到.

$^{921}$旁邊這裡我們有Cut Half的版本.
可能將本身20分鐘的影片.
變成10分鐘的意思.
另外也會有一些支援的文件.
幫助大家去Run Alpha.
這些都可以在這裡找到.
另外我們也會有一些宣傳資源.
你可以探索一下.
我們會有一些遊戲包.
有一些建議的詩歌的包.
或者是有一些做茶點的食譜包.
或者是有一些Invitation Card.
或者是有些Roller Banner的Template.
或者是已經做好了的東西.
都可以在這裡下載得到.
這個對攻就開始有些特別了.
這裡除了你去Run Alpha之外.
其實你可能有IT或者其他Admin的同事.
一起幫你去Run Alpha.
難道他們又再Create一個新的Alpha.
他才可以Access到那些影片嗎?.
不是的.
你只要第一個人.
在這裡輸入你其他幫你一起Run Alpha的人的Email.
他就可以透過一個成員的角色.
去看到你已經登記的課程.
也可以在當中拿到影片的資源.
他就不需要自己再Register一個新的Alpha Course.
這裡也是幫助我們Alpha.
在後面的Customer Service的System.
我們可以分清楚.
究竟真實Run Alpha的人有多少.
以致我們Alpha Office.
能夠在課程裡面去Contact你.
Send Email給你的時候.
就更加的中了.
今天可能用多了一點時間.
但我覺得這些都是很需要和弟兄姊妹去分享的.
我相信今天也是一個基本的介紹.
但如果大家想了解多一點.

$^{961}$認識Alpha多一點.
或者看看有沒有什麼訓練我們可以做.
那些我們未來都會有的.
一聲飛了很多的時候.
這個就是我們Alpha的一些聯絡方法.
可以去Contact我們.
也可以拿到一些資訊.
或者未來可能透過建度中心.
我們也會Arrange一些Alpha相關的活動和大家分享.
我今天的分享就到這裡.
謝謝Matthews.
非常詳盡.
大開眼界.
我自己也參加過很多次Alpha.
原來青少年的Alpha這麼精彩.
希望自己也可以滅掉這個需要.
現在就找譚牧師去分享一下.
他們在教會成功的體驗如何.
馬上給你譚牧師.
謝謝老牧師.
謝謝Matthews.
我的名字叫譚子信.
我的英文名和子信的音是對的.
所以叫Jason.
是我媽媽改的.
謝謝我媽.
我是一個新移民.
大概來了加拿大Calgary兩年半.
我現在在南卡城環過渡會來侍奉.
我是語堂的領導教授.
我們語堂現在大概有六百多穩定的聚會人數.
其實我們從這幾年的移民潮.
我們是大概多了200人.
我們原本400人左右的教會.
現在是600多人.
我們加上英文堂國語堂.
我們就過了1000人的教會.
說一下青少年Alpha.
其實南部這間教會有很長遠的歷史傳統.
是會用Alpha的.

$^{1001}$所以弟兄姊妹對Alpha這個名字非常熟悉.
也非常有口碑.
很多我們的信徒都是在禮法那裡信主.
也都在那裡參與侍奉來被興起.
所以我們OneAlpha沒有任何的疑問.
因為大家都很接受.
只不過我們這次是特別想針對一個年齡層.
來開Alpha.
待會我會說一下我們有兩種開Alpha的方法.
一種是一大群人,幾桌子.
我們這次開年青人的YoungAlpha.
對象是針對移民潮的年輕人.
因為看到他們過來讀書和工作.
我們由去年的11月中開.
開到剛剛過了這個星期六.
我們大概是八個星期開的.
所以為期是五個半月,接近半年.
由11月中到上個星期.
我們有六桌子,平均每桌子有八至十人.
我們剛剛統計了一個profile.
在我們的參加者裡面.
其實有23個參加的人是未信主的.
因為我們有不同的network.
我們認識了他們,邀請他們來教會等等.
所以我們有些人際關係,邀請他們參加Alpha.
我們都明白要過半年是很困難的.
所以我們真的祈禱.
經過這麼多個星期之後.
我們統計回來的缺席人數.
23人裡面有7個缺席.
有10個仍然未信.
另外有6位是中途退出了.
因為種種原因不能夠完成整個Alpha.
所以大概信主人數是三分之一.
即是33\%是大概的信主人數.
除了這些未信的人.
其實我們也要計算那些幫手帶組.
我們幫手帶組總共有20個頂梓妹.
是年輕人,都是同年紀的.
大概20人帶著這些人.

$^{1041}$另外我們還有一個團隊.
是負責幕後幫我們做每次的食物.
那裡有33人.
那些是上一年級的人.
大概50多歲到60歲.
這班頂梓妹每次幫我們預備食物,茶點等等.
另外我們還有一班人是我們的年輕團體.
他們信了主的.
也有11人參與我們Alpha.
我們當它是一個培靈的聚會.
因為他們可能有很多從小到大返校會.
但其實他們沒有參與Alpha.
透過Alpha其實有很多比他們好的討論.
有些未信的台,有些是信了的台.
但我們這樣也建立了一個社區.
因為他們信了或是很自然有關係.
就可以進入到我們青年的團體.
所以剛才你們聽我的profile.
我大概每一次開組就是接近60人開組.
其實我們是跟足剛才Natural所說的Alpha的運作.
包括之前我們是招聘團隊的時候.
我們一起用了Alpha的訓練材料.
那兩三條片我們一起看,一起分享.
一起祈禱.
然後我們開組也照跟足剛才的次序去開.
也有做一個Alpha Weekend.
有做代託的服侍等等.
我們也知道食物是一個很重要的Alpha元素.
所以剛才我說邀請了這班成年人來參與.
這也是另外一個考慮.
因為對於成年人來說.
現在在移民潮他們未必有親身感受.
因為現在移民潮主要是在夫婦的年齡層.
即是帶著小朋友的年齡層.
也是初職的年齡層比較明顯.
Stream A,Stream B是這班人來的.
所以對於那些年紀較大的成年人.
接近Baby Boomer那班人.
其實他們不太感受到.
所以我們也很想製造一個機會.

$^{1081}$讓他們真的有份參與現在的服務.
因為他們聽了很多,但他們沒有機會做.
所以我們一叫的時候,我們很感恩.
因為這班弟兄姊妹他們是很衛生.
每次都很有心機預備我們的小食.
我們按一個大的預算給她.
總之這十多次,她就幫我預備食物.
其中有兩餐是要全日食的.
因為After Weekend是八個小時的.
所以中間要吃一餐,可以飽得肚子.
另外我們剛剛跨過了聖誕節.
我們那晚開早是平安夜.
所以我們那晚又是有一個比較豐富的晚餐.
所以整體來說,她要安排十次的茶點.
有兩次的全日食,給她一個心儀的錢.
她就幫我們預備,每次的食物都很精彩.
例如,我簡單說一下.
他們有主題的,剛剛星期六就是吃壽司,鰻魚卷.
然後就是喝日本綠茶.
然後就是綠茶紅豆芝士蛋糕.
這個就是星期六的晚餐.
有一餐是香港之夜.
就是吃咖喱魚蛋,豬皮,香港奶茶.
我只是舉其中兩個例子.
你會明白,弟兄姊妹花心機去參與.
她們花很多時間去試菜,試味.
怎樣做出來,就做好了,放在桌上.
所以年輕人很感動,很被愛.
年輕人是自發的.
特別是我們順著主的弟兄姊妹.
她們又做了一些食物給整個食物團隊.
她們做了一些手製曲奇和很好的標誌.
包裝好,來向我們致謝.
當然是默契情義重,我想心是很重要的.
所以我剛才說這些的時候.
我不是只想量化有多少人信主.
而是我想生命的建造,參與.
能夠在神的角度開了眼界.
我覺得這個火的燃點,也應該要計算下去.
這就是那個「年輕人」.

$^{1121}$我們用的那段片,剛才Metris也有播放.
就是《新世代》的片.
因為我在香港也用過,比較熟悉那段片.
但我們也計劃,如果我們將來.
明年有更多的學生,也想試試那段片.
《新世代》的片,我們也有看過那些片段.
我們也想覺得適合我們用.
我其實很欣賞香港Alpha的侍奉.
因為大家知道,我們報名了.
其實就是有費用全面,資源又很好.
所以我們很感恩神興起了Alpha這個機構.
來服務全世界的教會.
所以我們在禱告上支持Alpha.
也在奉獻上支持Alpha.
除了這是一個大圍的Alpha之外.
我還想說,我們教會也有化整為零.
我們叫做Mini-Alpha,我們用小的Alpha.
因為特別是經歷了兩年半的COVID之後.
我們已經不能夠,很難一班人回來聚會.
所以我們在COVID的尾聲,我們開始推出Mini-Alpha.
那個想法很簡單,就是四人成行.
只要你找到四個微信主人.
有兩個Christian能夠為身做組長.
我們就可以開班.
你可以多過兩個Christian,三個四個都可以.
就可以開Mini-Alpha.
我們教會做一點推動.
我們在預算上,在我們Alpha的預算上.
預了一些錢,現在大概是四百元.
或者四百多元,預整個Mini-Alpha.
給那組人,他可以預備一些小食,茶點等等.
有一個sitting,給他不需要自己出時間,出力,出錢.
起碼給他一樣東西,推動他,讓他可以比較無後顧之憂地去侍奉.
我講一個例子.
我們去年的時候,有一個夫婦,也是新移民來了一年多兩年.
是我們的組長,她就實踐Mini-Alpha.
她在自己家裡做.
在她家附近,有些夫婦,有些年輕人.
大概每次,算上她夫婦本人.
大概是八個人,九個人,每逢星期五晚在她家聚會.

$^{1161}$其中主要只有兩位是微信主的弟兄.
但經過這個Mini-Alpha之後.
就100\%信主,那兩個微信的男士都信主了.
我覺得這個是非常有效.
因為人少,地點,很容易去哪裡都可以.
剛才Natasha說了,在家裡做就行,在餐廳做就行.
彈性很大,不需要每次都在教會預訂房間,有幾十人.
你又要安排很多東西.
所以彈性大,關係也會很好,那班人很熟.
所以我很感恩他們的那個學校,很好,帶到他們信主.
很容易那些人就會投入在我們的小組和團體.
這個也是我想會提出的.
另外我想特別高興多一點就是關於那個Alpha Weekend.
因為那兩課是介紹聖靈是誰,還有是說關於怎樣給聖靈充滿.
我知道這兩課可能在某些教會裡有些關注.
但是我是聽過香港Alpha的同工說.
其實我們的信仰是三一神的信仰,聖父,聖子,聖靈.
其實我們不需要特別去反思聖靈的部分.
還有我看過他們的內容,其實是相當神奇的.
我想是拿了中間位置,他未必是最肯明恩那邊.
但也不是最保守那邊,但應講的我覺得都是要講.
我想特別提對我們的弟兄姊妹來說,他們都是一個很好的學習.
因為我們當中要學習怎樣為人禱告.
特別是在聖靈禱告,這個聖經教的.
Pray in the spirit.
我們怎樣去為他禱告,我們也有這個經歷.
我們不是說要講方言或者暈倒.
但我們真的很真誠地開放自己的心靈.
按手為一個新的參加者禱告的時候.
我們很多人一按手,參加者就哭了.
不知道為什麼莫名的感動.
彷彿聖靈也教我們怎樣為一個人去禱告.
所以我覺得這個對我們的義工來說.
都是一個嶄新的學習,我為此感恩.
我先分享這麼多,待會我們有機會多些討論.
但我大概講一個案例,讓大家立體化我們經歷的事情.
多謝譚牧師的分享,這個時間我有一兩個問題帶出來.
讓Matches或譚牧師有些回應.
因為我們身處加拿大,現在有一個尷尬的地方.
就是這些新移民,從香港出發.

$^{1201}$來到加拿大的時候,多多少少都受到以下兩種張力.
他們必須要面對的.
在我們教會來說,也必須要做好準備.
第一件事就是,究竟加拿大裡的元素和香港的元素.
有沒有一個融合的機會呢?.
因為剛才Matches也提到,無論是職青的份量.
或者是青少年的部分,都是很香港風格.
但是來到加拿大的時候,他讀書也好.
他就開始面對加拿大那種多元文化的體會.
很開放性的體會等等.
我相信很多父母,包括一些目者.
都很留意加拿大帶出來的衝擊.
如果我們要做青少年的工作的時候.
又用Outer Youth這個詞語的時候.
有沒有一些需要調節的地方,或者加上去呢?.
最好請Matches講一下.
首先分享Alpha,是字母裡的第一個.
所以其實是一個陽生朋友接觸教會的第一個起始點.
剛才的老師分享的東西,其實都是Beta和Gamma之後他們會面對的東西.
我相信這些都是重要的.
但是Alpha他做到的東西是有限的.
我們很希望集中火力幫助教會做信仰群體和新朋友的第一個接觸點.
讓他們,特別是加拿大的新移民.
因為他們可能剛剛離開香港一兩年,甚至三年.
其實他們一直以來都是在香港華人的背景下成長.
他們要進入一個新的國家環境來適應.
其實他們都會思鄉,沒有那麼快做到轉換.
所以在教會這一面,如果選取一些香港背景的製作.
其實幫助他們對於教會的文化.
可能首先是一個軟化的作用.
讓他們容易聽得懂,容易掌握.
然後也可以透過和弟兄姊妹的真實關係和後續.
我們都可以舉辦很多不同的活動來幫助他們.
豐富接軌當地文化的東西.
我都聽過一些例子,可能是舉辦一些家長的講座.
作為人數的焦聚.
可能是關於入學的方法,或者認識社區等等的活動.
焦聚了他們,和他們一起經歷了一些文化的適應下.
他們建立了關係之後,也邀請他們來聽Alpha.
這是其中一個方法.

$^{1241}$聽完Alpha之後,我們教會有些後續的.
可能是旅行,或者不同的小組.
就可以引流進入教會整個模樣系統裡面.
我覺得這些就可以配合.
我們要想清楚,Alpha放在教會,堂會.
他們整個新朋友歷程裡面的哪一站呢?.
我們計劃清楚,我覺得是可以的.
剛才你們教會有沒有類似的後續工作做?.
除了推Alpha出來.
我們其實是有不同的.
剛才的新朋友流程,其實都是慢慢有些步驟.
因為我們有一個青年的崇拜,其實那裡是歡迎所有人.
所以我們久不久都會有些新朋友,即使還未相信的都會來.
相信的都會當作一個接觸點,他可以進來.
另外我們有其他的program.
我們有Sports Life,我們有去健身室打球.
定期,隔多久我們就會舉辦.
我們每一季都會接觸到新朋友.
另外我們有一個英文班,都有接觸到新朋友.
我們有一個接觸點,跟那些不相信的人有接觸.
然後我們會邀請他來Alpha.
因為Alpha真的會說福音.
即他不是一個pre-evangelism的東西.
我回到我老牧師的問題.
我也是一個新移民,都還在思考加拿大環教會的自處.
其實不只是關乎Alpha,而是整體加拿大環教會的立足點想是怎樣.
我相信我們今天這班都是說廣東話,我們都是環教會.
因為我的教會在名字的招牌都有華人的字,都沒有撕掉華人的字.
所以我覺得我們這件事不需要自己來懷疑自己我們在做環的工作.
因為我們已經擺明居馬,是給他一個西方文化的異地.
用一個本土文化的焦點幫他們聯合.
所以這已經是一個事實.
當然對於在這裡長大的人,所謂CBC,或者在這裡長大的青年人.
他們本身已經可能在我們英文堂那邊.
所以英文堂,我覺得用Alpha,他當然可以用英文那邊的影片,或者是西方的影片.
但現在我們最重要是清楚自己的對象.
好像我們這次做,我們是很清楚是香港來的新移民.
我們是粵語的作為焦點.
我自己看到特別是初中那班,雖然我們這次的Alpha沒有什麼初中生.
他們是從16歲左右到35歲.

$^{1281}$但是初中我自己這兩年留意,其實他們的適應是比較難的.
你說小朋友英文學很快,他很容易可以進入加拿大.
高中去到大學也是很快會適應.
但是初中這個年紀,他剛剛人生這個階段,有些身份危機,有些少面皮薄一點.
總是很難,所以很多這個年紀是流失的.
他本身在香港教會也很容易流失的,我們做過一個墓會就知道,初中是難做的.
他初中加移民,其實是兩個壓力因素放在一起是難的.
所以如果你對這班人,你還要給他一條英文影片,你融入加拿大文化,其實你也是推他走的.
所以我覺得對於這些年齡層的人,其實我覺得用香港文化的影片是適合他們的.
甚至我也認識很多這裡的教友,他們在加拿大很多年,原本來教會之前是完全英文人.
但是他們是說了廣東話的,他們回到教會後,他們能夠說廣東話,很開心,原來有這麼多人可以說廣東話.
雖然他們不是這幾年才來香港,但是其實那個文化的根在這裡.
所以你一勾起他們的時候,他們是會共鳴得到的.
很多謝譚牧師這個分享,因為事實上文化這件事是很複雜的.
我們在找機會,一些新來的朋友,無論是信主還是不信主.
他也好像剛才Matthews說的,他也有一些思鄉的元素在裡面.
就好像我想起我二十多年前,來到多倫多的時候,很感受心的能夠吃到蛋撻.
這些就是文化帶來的感受,如果教會有很多蛋撻吃,那就最精彩了.
起碼我們有一個接觸面,他會覺得feel good,be accepted.
然後才說到一些更加沉重的東西,我很欣賞剛才Matthews你show的那些片.
真的很適合那些年輕一輩的人,起碼他不會有抗拒感.
對於福音不是這麼老土,這麼悶的東西,又跟我說耶穌.
他不會有這樣的態度,這是一個很effective的方法去傳福音.
當然如果你去到更加深入的層面的時候,可能也需要一些幫助.
現在想帶出一個問題,在加拿大也是很common提出的.
就是那種後現代衝擊,現在的情況就是在加拿大.
西洋教會也好,或者是華人的教會,但是說英文那批的弟兄姊妹.
一個最大衝擊就是後現代,兩位講完有沒有什麼回應呢?.
對於後現代衝擊,我們向他們傳福音,有沒有什麼板斧可以注意到?.
Matthews你先說吧.
好,這樣呢,就是.
我猜Alpha這種互動其實已經在回應後現代他們.
期待他們進入某一個社群,聽到別人的聲音.
也能夠展現自己聲音的其中一個方向.
這個也是他們去尋索.
因為我們相信其實不是要很硬去為一個信仰去給予人.
而是他們真的透過一個探索的過程.
他們找到,掌握到,明白到這個信仰.
究竟內容是什麼,他們才去相信.
這個當然是站在新朋友的角度.

$^{1321}$但站在教會的角度,我們其中一個R就是很依靠性.
其實我們都是真的,如果你們參加一些我們Online Training.
我們都會分享到,其實我們都會很想開口去.
可能我們不是Run Alpha的時候.
我們組長邀請他們按著一些人名出來.
按著每一個人去為他們祈禱.
我們祈求神去幫助他們在他們生活的空間.
因為他們可能Run Alpha只有一個小時,兩個小時的時間.
你和他們接觸到的時間.
但其實有更多的時間,我們深信聖靈在他們生活的空間裡面.
在他們的歷程裡面,都一樣這麼愛他們,保護他們.
讓他們能夠體會到信仰.
所以其實,索靈的說,我們都是會為他們去祈禱.
有些堂會,一路在Run Alpha的時候.
同步地有禱告隊在旁邊,同步為這些人去祈禱.
所以我們一方面去容讓他們有自由,空間,互動.
能夠表達到自己.
但另一方面,我們也很索靈地去祈求神.
保護他們這些自由的思想.
保護他們這個空間.
在自由探索的旅程裡面,依然有聖靈陪伴著他們.
這個就是我們Alpha在索靈上.
如何去面對後現代他們這麼豐富,這麼多思維,百放的張力.
這個真的很有實踐性.
因為後現代的衝擊,其中一環,我觀察就是.
提倡沒有真理,也很相對.
一班年輕人聽到這些教導的時候.
他也會覺得跟一般教會的教導好像格格不入.
但如果一個Alpha的形式,給他開放性的.
讓他可以抒發他的感受.
還有讓他講,覺得沒有壓力下去講.
我自己覺得這是一個不錯的方向.
最低限度最能夠break the ice.
他能夠很自由地講.
在這個環境下,他可以得到一些幫助.
雖然譚浩思你來了加拿大不久.
但後現代的衝擊你也有了解.
因為我知道你在美國讀書.
你有什麼看法,對於後現代衝擊我們教會的東西.
我現在的體會,我自己也有看書.

$^{1361}$我從來不把後現代看成一種威脅.
我看它是一個現實,是一個機遇.
因為每一種文化,每一種思潮是有它的特色.
我們現在身處這個年代.
我們不可以把它的時光倒流.
變回現代主義,或者以前的時代.
剛才阿隆斯提到,那種是沒有絕對的.
大家都是相對的.
所以在Alpha的設計,剛才Matthews說了.
某程度是照顧了這件事.
所以是透過對話和大家討論.
而且那些問題不是close end question.
是一種open end question.
不是說耶穌是不是神.
一個就答完了,沒有討論空間.
所以他慢慢討論,其實是符合現在人的口味.
那種思維.
但我自己覺得是一個平衡.
因為Alpha本身也有一定的內容教導.
而且它的作品的贊成.
我知道有很多神學院老師的團隊贊成.
所以是非常德基督教育,是標準的內容.
在一個平衡的基礎下,我們可以對話.
我覺得反而我們自己要調教一下我們的心態.
當然耶穌的真理是絕對的.
但我不敢說我是完全掌握這個絕對的真理.
所以有一種神學的說法就是.
沒有人絕對的,其實大家在一個dialogue裡面.
大家一起接近真理.
所以如果我們肯放下自己的身段.
我們會容易一點,不需要每次都高高在上地告訴你真相.
而是可以在對話裡面.
有時候我也要留白,我也要忍口.
不是說我說完說了就算了,我是莫斯,你不要反駁我.
年輕人不是這樣.
反而更加要想,Alpha是一個帶他信主的工具.
信了之後他進入你的教會團契.
你的團契也要有這種文化才行.
否則接不到鬼.
他明明Alpha可以對話,但一回到團契就像傳統一樣.

$^{1401}$他就趕拍了.
即是說,如果我們真的要做青年的模樣.
是要整個思維扭轉過來.
即是更加是那種真誠的關鍵性.
我也是受害者的,我也有我的疑惑,我有我的脆弱.
我可以顯露出來.
在那個對話裡面,耶穌當然是真.
但是我們怎樣透過我們的疑問,我們的掙扎.
我們的掙扎也是真實的,大家都可以哀羞.
越年輕的人越珍惜這種真實的關係.
這種是能夠大家互動的,躺開的關係.
這是值得我們做青年模樣的人去思考的.
很好,我收到在chat room裡面.
有一位姐妹這樣說的.
我姑且這樣讀出來.
這位姐妹的看法是說.
既然這班是新來的移民.
他們也是在適應的過程裡面.
也沒有那麼大的需要硬性.
要他們盡快融入加拿大或者外國的文化.
會嚇到他們.
令到他們錯覺他們不是很被愛或者被接納.
不理解他們真正的感受或者需要.
這個其實提到很重要的一點.
就是我們的敏感性有多高.
當然我們接受新移民的時候.
我們不只是希望我們教會多些人.
如果是這樣的態度就很糟糕.
而是出於真誠的希望能夠介紹到.
這麼好的福音給他們.
是我們最大的一個推動力.
也不是硬性要他們適應這裡的文化.
不過毫無疑問.
好像我們這些移民了久一點的人.
慢慢你就知道.
你喜歡也好不喜歡也好.
你也要適應.
過了五年之後.
你現在還是覺得新移民這樣的心態的時候.
就很可能很多事情都很不方便.

$^{1441}$很多事情你會發覺好像格格不入.
但是我也同意這個看法.
在初期的時候我們就盡量去適應他們的需要.
一個好的態度就是要包容他們.
因為事實上很多事情他們都不知道.
回想起二十多年前我們也可能經歷過這個階段.
這個時候就是一個很好的機會訓練我們的愛心.
是否真是地道的愛心.
我想我收不到有什麼問題.
如果真的有問題歡迎你現在寫出來.
我想請Matches總結一下.
如果讓你說一點.
你覺得你今晚說的話最關鍵.
希望打進我們的聽眾的心.
One point你覺得是哪一點.
其實很多人都分享.
他有機會都未必很記得影片的內容是什麼.
但是他參加完Alpha最記得的.
就是怎樣被弟兄姊妹去接待.
他很記得教會的群體是怎樣去愛他們.
怎樣去透過這些對話的空間去讓他們表達可以自在.
無論他最後信不信耶穌也好.
我覺得影片可能都是其次.
裡面的內容都是其次.
但是怎樣給新朋友去體驗到我們這個信仰群體.
應該展現的那種愛.
應該展現的那種款待.
其實是很重要的.
其實他有機會對信仰的內容未必完全掌握或相信.
但他可能是基於你對他的那份愛.
這個群體的那份味道.
而留下在這個群體.
很多人未必繼續信耶穌.
但他繼續保持你們的崇拜.
但因為他喜歡這個群體.
他覺得這個群體是有愛的.
我覺得這是對於我們信仰群體裡面.
最需要展現到的.
無論在Alpha裡面.
在我們教會裡面.

$^{1481}$其他的活動裡面.
希望鼓勵到大家.
可以透過Alpha去練習到這件事.
多謝Matthews.
譚牧師,你覺得最關鍵的那一點可以和大家分享嗎?.
我就從教會牧羊的立場來說.
我想強調的重點是動員弟兄姊妹的參與.
這是其中一種培育他們很重要的一個階段.
因為Alpha相對來說.
其實他入場的門檻是比較低的.
他不是很難的.
簡單來說.
你說三科.
你也要上一輪訓練.
要背一堆根據.
你栽培百科又要上訓練.
又要考一張試卷.
你才可以做栽培百科.
但是Alpha你看兩條片15分鐘.
30分鐘已經搞定了.
其實我想說.
是任何人都可以做的.
只要他想做的話.
當然我這樣說.
其實也不是很容易.
他怎樣和新朋友開放,聆聽.
反而是一種遠的功.
但是是一個good start.
因為他的門檻低.
所以透過這樣也是一種門徒培育的過程.
多謝譚牧師.
我又分享一下我覺得一個關鍵點是什麼.
我覺得今晚提出一個觀點.
就是Alpha的Youth Education Program.
牽涉到人與人之間的關係.
relational的那一點是最strike我的.
因為我們怎樣能夠拉到他們.
以致他們有機會信耶穌.
他也要進來聽福音.
如果他不聽也沒辦法.

$^{1521}$這個就是最巧妙的地方.
就是和他建立了一個穩定的關係.
就像剛才也說過.
聖靈會有工作的.
他可能現在這一刻沒有反應.
對於福音沒有反應.
但是他願意留下就有希望了.
希望聖靈會繼續做這個工作.
我覺得Alpha.
無論你是職青的層面.
又或者是青少年的層面.
都可能幫到大家.
我現在又收到一個問題.
Alpha能不能幫助年青人.
deal with social issues.
例如同性戀這個問題.
一個社會性的問題.
不知道是否一個matress.
你給一個官方答案.
官方其實我們沒有涉及這些問題.
例如像香港.
我們因為針對香港的環境去製作.
所以這些在香港的教會裡.
暫時還沒有完全去面對.
因為我們華人社會裡.
相對在這些議題上.
是比較保守和慢.
令我們集中在基督教信仰的核心內容裡.
所以接下來的部分.
就很需要在堂會裡.
就著你們的主心的文化來補充.
我收到有另一個問題.
就是教會如何避免推行Alpha Youth的時候.
令到某些群體的事工.
而沒有辦法整合整個教會的參與.
我相信譚牧師剛才也分享了.
你很成功.
雖然你是在搞Alpha Youth.
但是你能夠mobilize到.
其他年齡層的弟兄姊妹.

$^{1561}$都可以幫到忙.
你有沒有什麼特別的板斧.
可以做到這件事呢?.
你如何說服他們呢?.
所以我剛才說了.
我想幾樣的背景.
是我的教會獨特的背景.
可能未必能夠應用在各位同道裡的教會設定.
剛才第一件事我想highlight的.
就是我的教會已經有超過十年是run Alpha.
所以整體教會對Alpha是很接受的.
這是第一個條件.
另外就是移民潮這一年多兩年的移民潮.
其實教會已經看到.
所有年齡層的人都看到移民潮.
剛才只不過是說.
那些比較年長的弟兄姊妹.
他都沒有移民潮.
他的團體都沒有新人湧來.
聽到你們團體都沒有新人湧來.
經常覺得牧師說這麼多新移民.
他又做不到什麼.
你叫他跑又跳.
他又不可以跟你跑又跳.
青年崇拜他又嫌醜又嫌黑.
他有什麼參與呢.
當然要找一些位置給他參與.
所以我們一做YouFarmer的時候.
我們問有沒有人可以幫忙煮東西吃.
人們就舉手.
就來招聘他們.
所以其實是一個.
我想是可以應用到進來.
但另外我剛才想說.
由於Alpha在我的教會設定已經很接受.
所以我們除了是Young Alpha.
我們剛才也說推出Mini Alpha.
其實Mini Alpha是給任何年齡層的人.
你老人家你都可以自己在太極班那裡.
找一些太極之友自己開Mini Alpha.

$^{1601}$其實我們是歡迎.
我們是鼓勵他們這樣做.
所以就不會覺得支離破碎.
就是說你獨立一個群體.
只不過是大家都有同等的機會去把握.
用好這個工具.
我知道有些堂會.
我覺得是需要跟弟兄姊妹說的.
特別可能用崇拜的空間,報告的時間.
或者教會的壁報板.
或者自己的社交媒體.
或者其他媒體,網頁.
去將Alpha的一些理念跟弟兄姊妹分享.
或者是Run過一次Alpha之後.
我們相信裡面會有一些好的故事或者是好的見證.
去節錄出來讓弟兄姊妹看到.
如果教會不是只是Run一次Alpha.
而是想連續Run幾次.
我覺得透過這些跟弟兄姊妹接觸的渠道.
就可以告訴他們Alpha他們可以怎樣參與.
或者這個Alpha的時候是target什麼弟兄姊妹.
有什麼果效.
或者有什麼方法.
我覺得跟弟兄姊妹溝通都是重要的.
非常好.
這個時候我也略略簡單分享一下.
究竟建築中心在這件事的事工裡面.
可以有什麼計劃呢?.
我們具體策劃是沒有的.
不過我們看到一件事.
就是青年的事工.
其實是一個非常值得去推行的事工.
今天給我們有這個啟發.
我也鼓勵中弟兄姊妹或者同工.
如果你有什麼建議.
你覺得建築中心可以做到一些工作.
幫助宗教會在Alpha.
或者是普遍青年的事工裡面.
我們能夠做些什麼.
你寫個Email告訴我.

$^{1641}$讓我們再詳細去思考.
在我們的簡單的策劃裡面.
我們希望能夠再多幾個月之後.
我們再搞多一次的一個講座.
是和青年的福音事工有關的.
或者在明年的年初也會有.
我們希望能夠在青年的事工裡面.
我們建築中心都能夠盡一點力量.
能夠為宗教會有些合作.
所以剛才說Alpha裡面.
一些培訓.
雖然在網上的培訓.
只是幾套影片去看培訓.
但是在地的實體裡面.
可能組織幾個小型的教會.
會不會建築中心都可以做一些事.
我真的很樂於去做.
總之有些想法能夠幫助宗教會.
已經是非常之樂意的.
在這個時候再重複.
我們如何可以聯絡到我們.
請Benson展示我們的slide出來.
我們的位置是在Markham.
80號Acadia.
就是Walden and Steel.
搭公共車都可以來到.
不需要開車.
很方便找到我們.
請大家特別留意我們的網站.
因為我們的網站和Facebook的Fanpage.
經常都會發佈我們最新的一些發展.
這個就是查詢的電郵.
你寫給我.
我會盡快回覆大家.
加拿大建築中心其實是一個.
Self-supporting的機構.
歡迎大家如果有感動的.
都可以金錢奉獻給我們.
如果你進入到網頁.
你點擊支持神學教育的事工.

$^{1681}$你可以用支票.
或者e-transfer等等.
都可以的.
希望大家藉著這些事工.
得到上帝的祝福.
也在奉獻方面支持我們.
以致我們有經濟上可以維持下去.
我們很樂意服侍大家.
今晚時間差不多了.
不阻礙大家休息.
這個時間不如請Matches為我們做一個結束的祈禱.
好嗎?Matches.
好啊.
結束祈禱之前很快.
一分鐘內就回答最後一個問題.
我們其實是沒有英文的資源去支持的.
所以很抱歉.
但是我們西人版本的Alpha.
是有中文字幕的.
如果真的需要同時兼顧兩個語言的弟兄姊妹.
可能可以在這些資源裡面去作出選取.
我們一起來祈禱吧.
Matches.
去讓我們一眾的教會.
能夠在主你的愛裡成長.
我們在整個教會的歷程裡面.
我們都需要經歷不同的up and down.
可能經歷過不同時代的挑戰.
文化的衝擊.
但是主我們深信.
主你的聖靈在我們當中賜我們永恆的力量.
賜我們活水的盼望.
叫我們能夠展現出教會應該有的特質.
今天我們見到很多一眾的教會的領袖.
他們都面對著一些的張力和困難.
但是他們依然很想去找一些機遇.
找一些方法去和弟兄姊妹同行.
主人首先將這件事交在你手上.
主人你去恩待他們.
愛他們.

$^{1721}$讓這些教會能夠得力.
主人我們也期望.
我們未來日子可以成為一個基督的身體.
我們彼此互相幫助.
彼此互相協助.
我們讓基督的身體在這個世代裡面.
能夠展現出榮耀的光芒.
主人願你帶領我們.
我們以上的禱告.
乃是奉主耶穌基督得勝的明智祈求.
阿們.
再一次多謝大家今天的到來.
亦多謝兩位嘉賓講完.
我會再個別和你們兩位再接觸.
這個時候和大家在節目過程再見.
拜拜.
拜拜,多謝.
拜拜.
謝謝.
多謝.
我再和你再接觸.
好,沒問題,拜拜.
拜拜.
拜拜 拜拜.
\newpage



\section{}
\label{sec:6}
\textbf{不要浪費這個疫情}
\newline
\newline
~~~~ 日期: 2023-09-22
\newline
\newline
\hyperref[sec:B5n__dtTRhE]{\small{< < < PREV SERMON < < <}}
~
\hyperref[sec:index]{\small{[返主目錄]}}
~
\hyperref[sec:7]{\small{> > > NEXT SERMON > > >}}
\newline
\newline
$^{1}$陳耀鵬牧師 - 不要浪費這個疫情

講題:不要浪費這個疫情

講員:陳耀鵬牧師

場合:ABSCC General

日期:2023年9月22日

溫哥華建道講座日於9月16日在本拿比宣道會舉行.我被邀負責其中一個環節,分享「後疫情教會你我關注 - 後逆情下教會面對的新挑戰」.另外兩位講員是陳偉明博士及陸超明牧師,他們的題目分別是「願可面對面見你和你 - 從聖經及歷史角度思考後疫情中如何真實地面對神與人」和「以不變敬虔面對萬變 - 在失焦的新常態下更要回歸神的敬虔生命」.容許我稍為介紹從我自己預備這講座及聽取另外兩位講員的分享中所學到的功課.

我不是第一次講關於教會怎樣面對疫情這個題目.大家若留意我這份月訊的內容,在三月那期,我已經分享過一個類似的題目,大家可以參考一下 .我這次在這個講座分享的內容與那一份簡報大同小異,不過為免因為只重複過往所講過的材料而令自己感覺上完全沒有新鲜感,我便從維真神學院的圖書館中借了今年才出的一本書 -  Hybrid Church ,閱讀後在舊的材料裡加添一些新意.作者是我剛剛暑假第一次參觀過的Gordon Conwell 神學院 (Charlotte 校園) 的院長 James Emery White,他也是一間大型教會的主任牧師.他在書中提出疫情後的教會必須成為一間實體與網上並融的教會.教會在疫情後若要保持活力,就必須結合實體與數碼的氛圍.他提出一個以數碼作教會前門的觀念.他認為這一代的人很多會先在網上嘗試檢視甚至考核一間教會的聚會,甚至歷時幾個月,才考慮會否參與那教會的實體聚會.所以教會的網頁與社交平台是非常重要.教會的策略最初可能並非是要人進入教會,而是希望更多人先瀏覽教會網頁.另外,教會需要更好的運用 YouTube 與不同的社交媒體,甚至考慮發展自己教會的手機程式. 在一個互聯網充斥的世代及人工智能影響的世界,這本書無可否認提出很多教會應該考慮的方案.不過教會不一定需要無分辨地就接納他所有的提議, 反而要量力而為,識時度勢找出適合自己教會可行及可持續的應對策略,去面對後疫情對個別教會不同的「洗牌效應」.

陳偉明博士在講座的第二部分將他的題目分為三個部分 - 面對面見你和你的本相(向內看),面對面見你和你的群體(向外看)和面對面見你和你的未來(向前看). 他分析聖經中所記載的幾個瘟疫(出9:1-7,摩4:10,路21:9-11及啟6:7-8)的深層意義, 然後用魯益斯的說話作一個總結 - 「我們甚至連快樂都可以忽略,但痛苦仍堅持要被關注.上帝在我們的快樂中對我們耳語,在我們的良知中說話,卻在我們的痛苦中呼喊 : 這是他喚醒聾子世界的擴音器」.陳博士跟著從歷史中的幾個瘟疫 - AD252年的亞歷山大城瘟疫, AD249 - 262年的居普良瘟疫,1347 - 1353年的黑死病,1854年倫敦的霍亂及1918 - 1920年的西班牙流感帶出幾個教訓. 我最深刻的領受是陳博士提到西班牙流感肆虐時,很多教堂都關閉,但家庭就成了一個祭壇 .今日教會是否更應教導父母怎樣帶領家庭崇拜去抗衡疫情?最後,我亦由陳博士從聖經中看「面對面」這個分詞(創32:30,出33:11,申34:10,士6:22,結20:35,林前13:22,林後10:1,西2:1,約二1:12及約三1:14)的應用,提醒自己要鼓勵信徒多些「面對面」見神,以致不會「面阻阻」對人.

陸超明牧師以一個進程的角度來看疫情對信徒靈性的衝擊 - 從恐慌,害怕死亡到退縮而有一種無力感, 靈性開始滑落甚至漸漸感到生命無目標,以至最後懷疑神在哪裡.他強調信徒需要有一種復原的韌性,因為信徒在21世紀裡面已經面對很多危機,其中包括信仰與生活的分割,多元文化的衝擊及後現代的思潮,以致教會只淪落為一個社團,再加上疫情的拖累,我們的信仰可能會更內向和乏力,後疫情更加令一些信徒覺得信仰變為可有可無.陸牧師更提出信徒要追求一個以神為中心的生命,這生命基本上包括敬畏與敬虔,就是全心向神的畏懼及基督生命的彰顯.他特別介紹16與17世紀一位英國清教徒 Lewis Bayly 所寫的一本書 The Practice of Piety 裡面提出信徒不單要有內在的敬虔,也要有外在的表現.我在陸牧師分享中領受最深刻的就是他提出清教徒四個的敬虔方向  -  敬畏上帝,嚮往天堂,痛恨罪惡及高舉基督.

無論疫情是否已經結束,又或者在這個秋冬之後再來,願意活像基督的見證及榮耀上帝的目標成為我們在疫情前或後屬靈生命的內在追求與外在見証.
\newpage



\section{}
\label{sec:7}
\textbf{教牧關顧的可持續節奏}
\newline
\newline
~~~~ 日期: 2023-11-09
\newline
\newline
\hyperref[sec:6]{\small{< < < PREV SERMON < < <}}
~
\hyperref[sec:index]{\small{[返主目錄]}}
~
\hyperref[sec:8]{\small{> > > NEXT SERMON > > >}}
\newline
\newline
$^{1}$陳耀鵬牧師 - 教牧關顧的可持續節奏

講題:教牧關顧的可持續節奏

講員:陳耀鵬牧師

場合:ABSCC General

日期:2023年11月9日

我事奉的教會是門諾弟兄會基督教頌恩堂.在英屬哥倫比亞省有一百多間門諾弟兄堂會,而華人門諾弟兄會亦有十五間堂會,當中頌恩堂就有九間.在這個省,所有門諾弟兄會的堂會都隸屬於區會名下,而區會則負起保持堂會信仰純正,配合政府運作,提供不同資源,提升堂會牧者質素等責任,每年亦會召開年會去釐定計劃及通過預算.華人門諾弟兄會亦有一個華聯會,每年舉辦及統籌一些活動,最主要是受苦節的聯合崇拜,秋季的牧者執事研討訓練班及聖誕節的牧者聚餐 .最近我亦招聚了包括我自己共六位頌恩堂廣東話會眾中我熟悉的牧者,一同在網上交流「吹水」,看看有什麼可以合作做的事工.

門諾弟兄會華聯會(Mennonite Brethren Chinese Churches Association MBCCA) 於9月23日在基督教頌恩堂舉行了牧者執事研討訓練班,其中有三個不同的講座,我負責其中一個.不過,我很想在此簡單介紹華聯會主席陳博仁傳道負責那一個講座所分享的內容及附上簡報.

他的題目是「教牧關顧的可持續節奏」,在他的分享中,他不只從教牧的立場看這個題目,當中亦提醒執事怎樣幫助教牧去培養一個可以讓他們持續的事奉氛圍.他首先提出教牧事工與教牧關顧需要與教牧休息同步協調,另外可持續的教牧關顧必須在休息中邁步.我認識一位與我年齡相約,但仍然全職在教會事奉的牧者,他最近突然需要進院開刀動一個手術.我之後向他提議,若教會許可,他不妨考慮可以轉為部份時間,甚至像我一樣每年取兩個月無薪假期,多一點時間休息及與家人相聚,甚至出外旅行,亦可以藉此培養後輩,準備自己一天可以完全離場,能夠順暢交接.當然不是所有人都會像我這樣「懶惰」,但無論如何每一個人都需要適當的休息.

陳傳道在講座中跟著提出沒有健康的休息通常會產生四個可能的狀況.首先教牧可能會耗盡,以致他們在三至五年後要暗淡離職.其次,教會外面的問題與失敗可能不必要地變成教牧裡面的抗拒與批評,另外當教會事工衰落,領袖會過份操勞甚至承擔不起,在最差的情況下,教牧會離開他們的崗位,失望離職,甚至以後不再參與事奉.當然很多教牧仍然不忘初心,就算在身心俱疲的情況下,仍然站穩崗位,努力事奉.不過他們若在一種個人情緒不穩定,身心靈不和諧與人際關係不協調的情況下,他們的事奉不一定有長遠的果效,對他們自己與其家人亦不會有益處.

陳傳道很清楚的列出教牧可以考慮在短期,中期與及長期的一些行動建議,包括每一日要知道什麼可以令自己提昇或頹喪的事,時常禱告祈求活在主裡.中期的行動,則可以每一週或每一個月找到支持自己的群體,

認識自己繁忙的節奏與及看重實踐對別人的寬恕.在長期方面,陳傳道首先肯定教會在教牧同工事奉六年後,有一個四個月至一年的「安息年假」的重要.他自己亦提到在他安息年前對事奉的失落與煩躁,而安息年在原住民當中的短宣服侍令他重拾服侍的熱情, 他巨細無遺的提到第七年安息年之前六年應該做的事.

他提出第一第二年教牧應該先適應事奉,建立信任,向會眾流露甚至分享自己的長處及弱點,然後在適當時候引進微小的改變及掌握自己事奉的節奏.第三第四年教牧可以考慮邁向帶領更大的轉變,但無論成果如何,仍繼續忠心的裝備信徒與耐心的領導,不介意接受一個更全面對自己事奉的評估,然後努力的改善,更應該願意在需要時適當地轉變自己事奉的方向與步伐.最後在安息年之前第五第六年教牧需要更積極的裝備領袖及幫助他們同意與擁有自己的異象與文化,更需要在安息年前放置成熟的領袖在崗位上學習服侍,以致在第七年因教牧離開事奉崗位,仍然有人彌補事奉崗位上的需要.

誰說「一代不如一代 」? 陳傳道應該還未到不惑之年 ,但對教牧事工已經有這樣深入的分析, 令我非常佩服.

陳耀鵬

2023-10-29
\newpage



\section{}
\label{sec:8}
\textbf{主顯節的由來和反思}
\newline
\newline
~~~~ 日期: 2024-01-24
\newline
\newline
\hyperref[sec:7]{\small{< < < PREV SERMON < < <}}
~
\hyperref[sec:index]{\small{[返主目錄]}}
~
\hyperref[sec:9]{\small{> > > NEXT SERMON > > >}}
\newline
\newline
$^{1}$陳偉明博士 - 主顯節的由來和反思

講題:主顯節的由來和反思

講員:陳偉明博士

場合:ABSCC General

日期:2024年1月24日

陳偉明博士出生並成長於香港,中學畢業後負笈加拿大溫哥華,先後畢業於UBC取得應用理學士(化工及化學),理學士(計算機科學,副修商科)和文學士(心理學)及多倫多大學之理學碩士(物理化學).在資訊科技界工作十年後,蒙神呼召回港,於建道神學院取得道學及神學碩士,牧會七年後回多倫多大學攻讀並取得哲學博士,主修神學研究(研究奧古斯丁神學及教父學),期間也在本地牧會並於加拿大建道中心作客席老師.於Carey Theological College任職神學助理教授已近三年,期間也在職進修教育學碩士學位,並不時應邀於不同教會作講道及主日學服事.陳博士和太太 May 育有一名青少年兒子 Koen.

主顯節的由來和反思

2024年開始了半個月,世界不同角落仍然彌漫著各種黑暗,爭戰和困擾,跟二千多年前神的兒子主耶穌降世時在本質上沒有分別,大家還記得馬太福音幾位東方來的博士跟隨夜空的星光要來朝拜主耶穌那一幕嗎? 那權迷心竅,嫉妒成狂的希律王與那生來作猶太人的王,跟謙卑降世,道成肉身的主耶穌基督成了強烈對比,前者的假意說要去敬拜(加上合城的猶太人心裡不安)更是突顯身為外邦人的眾東方博士 (相信他們是波斯/中東一帶的祭司/賢士)排取萬難去見新生王的可貴.今天我們有多盼望主的顯現?

從早期教會到今天不少教會也以「教會年曆」(不是教會每年的行事曆,而是以基督事件為中心及相關節期為靈性操練和讀經年曆) 來默想主耶穌在世上的工作,而記念主的顯現的節期稱為「主顯節」或「顯現期」(Epiphany),由每年1月6日(聖誕節後第十二日,稱為「主顯日」)至聖灰日(Ash Wednesday)前的主日.

主顯節的名稱源於希臘文epipháneia.在古希羅世界,這字在國王或皇帝對其領土內的城市進行國事訪問,特別是當他向人民公開顯現時,也被稱為epipháneia (「顯現節」).這個節日最古老的名稱是 「神顯節」 (Theophany)(東正教會至今仍在使用這個名稱),這表明這一天的起源是為了紀念神的顯現,就是主耶穌「道成肉身」降世為人.在東方教會,默想基督的主顯節強調其神學性,因此,東方教會選擇以「耶穌受洗」而非基督降生這一事件來說明上帝在耶穌基督裡向世人顯現的教義.在這一天大量使用乳香特別適合紀念神的顯現.

早在主後二世紀末,埃及就有了基督教慶祝 1 月 6 日的活動,這在早期教父埃及亞歷山大的革利免(Clement of Alexandria)的著作Stromateis (《雜論》)中可找到端倪.當時敘利亞諾斯底主義中的一個「巴西利德」教派 Basilidians 在 1 月份慶祝耶穌的洗禮.這本來無可厚非,但問題在於這些諾斯底派的人主張耶穌是在受洗時才真正成為神的兒子. 這當然是違背聖經的教導,因耶穌是「道成了肉身,住在我們中間,充充滿滿地有恩典有真理.我們也見過他的榮光,正是父獨生子的榮光.」(約翰福音 1:14)故此,革利免為了保護教會信仰,駁斥諾斯底派這個扭曲基督神性的錯誤教導. 於是他呼籲當時的教會信眾一起用正統的神學來慶祝這個主顯節. 於是,亞歷山大教會強調耶穌的洗禮是主顯節的主要內容,是耶穌神性的顯現(卻不是那時才有).

到了四世紀,尤其是在高盧和義大利北部,耶穌在迦拿婚禮上的第一個神跡和他神力的顯現——以水變酒,已成為主顯節主題的一部分.在慶祝主顯節的早期,基督神性顯現的三件事——東方博士來伯利恆朝拜耶穌,基督的受洗和耶穌在迦拿婚宴上的第一個神蹟——彙於一起來於主顯節慶祝.而在東方教會,1 月 6 日(主顯節)主要是慶祝耶穌受洗和他在迦拿行第一個神蹟的節日;在西方教會,12 月 25 日(聖誕節)是慶祝耶穌降生和東方博士來朝拜耶穌的節日.四世紀下半葉,兩個節日發生了交換,不少東西方基督徒都慶祝這兩個節日.

於361 年,羅馬歷史學家阿米亞努斯·馬塞林努斯(Ammianus Marcellinus)指出,基督徒們稱一月的節日為 「主顯節」.因此,主顯節是繼復活節和聖靈降臨節之後教會年中最古老的節日.著名的教父及神學家奧古斯丁在他五世紀的講章Sermones 202-203,他指出東方博士是神的兒子顯在外邦人中的初熟果子(“firstfruits of the Gentiles” , 而牧羊人乃為猶太人最先見證主顯在猶太人當中).

而隨著慶祝基督誕生的聖誕節的傳播,在西方,主顯節開始與東方博士來訪聯繫在一起,部分原因相傳可能與東方博士的遺物在五世紀從君士坦丁堡轉移到了西方有關.與St. Bede (A.D. 735) 寫的Excerpta et Collecteana一書中記載了三個哲人,並提供了他們的名字 (Melchior, Caspar, Balthasar) 以及對每個人的描述和對每個人帶來的禮物的象徵性解釋.這文獻影響後世描繪東方有「三個」博士(聖經沒有說是「三」個),以及以象徵性解釋黃金,乳香,沒藥分別代表耶穌的王權,神性和受死.

最後,讓我們透過主顯節作兩點反思和禱告回應:

我們的主不只是猶太人的王,更是萬國之光,東方賢士真的奉上黃金和乳香(他們以此尊祟王及神明),我們若是神的兒女,也就是萬國中的一分子,在追隨主的路上如何排取萬難,甘心奉上將最好呈獻給萬王之王,萬主之主?

二,主顯節宣佈基督在外邦人中顯現,也是預示著基督將以祂所有的榮耀再次降臨,將萬民聚集在他的統治之下.今天,主的榮耀已經照耀屬祂的子民,黑暗不能遮蔽祂 (參:詩139:12),並應許將再來並永遠在他們中間顯現.願我們屬神的人記住主耶穌的話:「到那日,你們就知道我在父裡面,你們在我裡面,我也在你們裡面. 有了我的命令又遵守的,這人就是愛我的.愛我的必蒙我父愛他,我也要愛他,並且要向他顯現.」(約14:20-21).

讓我們學習主顯節的真義,深信祂神性的光輝——祂賜下引導博士的星光,祂的受洗和把水變酒也為此作見證——能驅走人間黑暗,以神聖的愛拯救世人,並施行公義的審判.願我們的愛,感恩與讚美歸於耶穌基督,那昔在,今在,來在的時間與歷史之主,直到永永遠遠,阿們.
\newpage



\section{}
\label{sec:9}
\textbf{依納爵的心窄成因}
\newline
\newline
~~~~ 日期: 2024-02-22
\newline
\newline
\hyperref[sec:8]{\small{< < < PREV SERMON < < <}}
~
\hyperref[sec:index]{\small{[返主目錄]}}
~
\hyperref[sec:10]{\small{> > > NEXT SERMON > > >}}
\newline
\newline
$^{1}$陳顯宗博士 - 依納爵的心窄成因

講題:依納爵的心窄成因

講員:陳顯宗博士

場合:ABSCC General

日期:2024年2月22日

專題文章 \# 03

【依納爵的心窄成因】

作者:陳顯宗博士

陳顯宗博士(Dennis H. C. Chan)簡介:

中國神學研究院哲學博士,香港大學社會科學碩士(心理學),香港建道神學院神學碩士,香港建道神學院道學碩士,香港大學文學士(歷史和德文);現於加拿大美城華人浸信會擔任傳道;期刊文章有〈人類擁有可作他選的自由意志〉,〈探討當代莫林那主義如何有助消解上帝主權和人類自由意志之論爭〉,〈莫林那主義者回應湯瑪斯主義者及宗教哲學家的挑戰〉,〈依納爵的心窄問題與辨別神類如何轉化靈性〉,〈心窄問題如何演變為宗教強迫症〉.

【依納爵的心窄成因】

一,心窄折磨

耶穌會會祖依納爵‧羅耀拉(Ignatius of Loyola, 1491—1556)曾受「心窄」(scruple/scrupulosity)[註1]折磨,其《自述小傳》有以下記載:

「但在這事上他也被多疑病頑強地困擾.因為他在蒙賽辣所辦的告解,雖然照他所說過的用了很大的心,而且一切都是寫出來的,但仍好像有些事沒有說明,這使他很困擾……」[註2]

他受困擾至萌生自殺念頭.後來,一次突如其來的光照令他不再為以前的罪告解,並使他不再受困擾.[註3]

二,心窄成因

(一)屬靈經典

依氏在家養傷時,看了《基督行實》和《聖人傳記》.兩本書強調很高的屬靈標準,致使他在悔改初期為追求聖潔而擁抱苦修主義.在《自述小傳》裡,依氏曾提及「聖道明做了這事,我也當做;聖方濟各做了那事,我也要做.」 [註4] 兩位聖人曾極力苦待己身,深深影響了依氏.例如,他也將自己名貴的衣服送給一個窮人,自己卻穿上乞丐的衣服.[註5]上述書籍令他在悔改初期極力效法聖人而克己.

(二)靈修傳統

 1. 「現代敬虔運動」


$^{41}$中世紀晚期出現了「現代敬虔運動」.這運動強調人要過聖潔和敬虔的生活,其中的代表作就是影響至今的《效法基督》. [註6]依氏在茫萊撒默想基督的生平及閱讀《效法基督》.這時,他也致力於自我鞭打,禁食,祈禱及其他十六世紀也認為極端的苦行.他不再追求美好的外表,甚至不修剪指甲和頭髮. [註7]這種極端的苦修主義令他不容易放過自己以前的罪過,決心認清它們.

 2. 傳統三階:煉,明,合

偽丟尼修建議的靈命成長分為三個階段:煉,明,合三路.唯有透過這途徑,人才可以與上主聯合. [註8]這三路對後世的影響很深,而依氏也很可能受這概念影響而在煉路裡致力認清自己的罪.

3. 本篤修院

在十六世紀初,本篤修院院長西斯內羅斯已推行了改革,其中一項是新進者要花上十天或更多的時間準備告解.這告解要將由出生至當時所犯的罪全部告明.[註9]依氏受指導下進行了這項措施的修訂版,他用了三天的時間寫下自己的罪,然後告解.這些行動象徵他與舊我斷絕關係.[註10]本篤修院這種做法也令依氏要極力認清過往的罪,不想有遺漏.

(三)負面神觀

「神觀」是指人如何看上主,而「負面」則指其過於嚴厲而令人畏懼.[註11]中世紀教會高舉上主的公義和審判,令人戰兢地透過善行補贖自己的罪,以求獲得救恩.前述的《聖人傳記》也反映了上主是極度聖潔和嚴苛的,人們動輒得咎.悔改後的依氏視上主為他的「學校老師」.基於他小時候曾受老師賞善罰惡的教導,依氏在悔改後便將老師的形象投射在上主身上:祂是位極度賞善罰惡的神,故他要不斷努力地討祂喜悅. [註12]

(四)完美主義

依氏原是一個鍾愛宮廷服飾和美好生活的男人,他不能忍受突起的腳骨阻礙他穿喜歡的帥氣和緊身的長靴, [註13]其性格由此可以窺見.他知道若再做手術將要承受的痛苦較之前更甚,但為追求完美仍然決定再做.[註14] 他也曾經對他的指甲和腳甲極度挑剔. [註15]可見,他具完美主義的傾向.悔改後的他雖然不再注重外表,但在靈性方面卻追求完美,故容不下自己有罪遺漏未認.

編者語:

依納爵的心窄相信是每一個認真追求基督的人也曾有過的經歷,經常因罪的纏磨而苦無出路,正如保羅所言:「我真是苦啊!誰能救我脫離這取死的身體.」(羅七24)

陳顯宗博士在專文中指出:依納爵的屬靈觀是由他的個人性格及信仰歷程所模塑而成:屬靈經典,屬靈傳統,神觀及個人完美主義的追求,這種種的原因形成他的心窄觀念.

試想想你自己的屬靈觀念如何形成?教會傳統,牧者傳道的教導,你所仰慕的屬靈偉人,影響你的成長伙伴,屬靈書籍,以及你自己個人的性格.試反思當中有哪些是需要努力堅持的信念?有哪些是需要改變的觀念?你又如何辨別哪些是出於聖靈的教導?哪些是來自個人性格所使然或者是那惡者的控訴呢?

你的神觀是正面還是負面居多?你的神觀的形成,當中或許是受到家中父母的形象,或成長時影響你最深的人所影響.這對於你認識自己有何新發現?對於上帝又有何更深的認識呢?下期陳博士將繼續分享如何轉化心窄的過程.

1 詳參喬治·剛斯:《神操新譯本:剛斯註釋》,鄭兆沅譯(臺北:光啟,2011),345—351號及入門書籍Joseph W. Ciarrocchi, The Doubting Disease: Help for Scrupulosity and Religious Compulsions (New York: Paulist Press, 1995).

2 依納爵:《聖依納爵自述小傳‧心靈日記》,侯景文,譚璧輝譯,再版三刷(臺北:光啟,1999),22號.

3 Philip Caraman, Ignatius Loyola: A Biography of the Founder of the Jesuits (San Francisco, CA: Harper  and  Row, 1990), 38—39;依納爵:《自述小傳》,24—25號.

4 依納爵:《自述小傳》,7號.

5 依納爵:《自述小傳》,18號.

6 Bernard McGinn, The Varieties of Vernacular Mysticism, The Presence of God: A History of Western Christian Mysticism, vol. 5 (New York: Crossroad, 2012), 96.


$^{81}$7 John W. O’Malley, The First Jesuits (Cambridge, MA: Harvard Univ. Press, 1993), 25.

8 吳國傑:《築樓蓋頂》,頁159—160.

9 Anselmo M. Albareda, “Intorno alla scuola di orazione metodica stabilita a Monserrato dall’ abate Garsías Jiménez de Cisneros (1493—1510),” Archivum Historicum Societatis Iesu 25 (1956): 254—316, quoted in O’Malley, The First Jesuits, 24.

10 O’Malley, The First Jesuits, 24.

11 Ana-Maria Rizzuto, “Critique of the Contemporary Literature in the Scientific Study of Religion,” paper presented at the annual meeting of the Society for the Scientific Study of Religion, New York, NY, quoted in Richard T. Lawrence, “Measuring the Image of God: The God Image Inventory and the God Image Scales,” Journal of Psychology and Theology 25, no. 2 (1997): 214.

12 James Brodrick, Saint Ignatius Loyola: The Pilgrim Years 1491—1538 (San Francisco, CA: Ignatius Press, 1998), 66—67.

13 RibVita, I, i, in Fontes narrativi de Sancto Ignatio, 4 volumes in MHSJ, IV, 85, quoted in Cándido de Dalmases, Ignatius of Loyola, Founder of the Jesuits: His Life and Work, trans. J. Aixalá (St. Louis, MO: The Institute of Jesuit Sources, 1985), 42.

14 de Dalmases, Ignatius of Loyola, 42.

15 de Dalmases, Ignatius of Loyola, 58.
\newpage



\section{}
\label{sec:10}
\textbf{奧古斯丁《講章 242(關於身體復活,駁異端的講道)》的節錄及反思}
\newline
\newline
~~~~ 日期: 2024-03-29
\newline
\newline
\hyperref[sec:9]{\small{< < < PREV SERMON < < <}}
~
\hyperref[sec:index]{\small{[返主目錄]}}
~
\hyperref[sec:11]{\small{> > > NEXT SERMON > > >}}
\newline
\newline
$^{1}$陳偉明博士 - 奧古斯丁《講章 242(關於身體復活,駁異端的講道)》的節錄及反思

講題:奧古斯丁《講章 242(關於身體復活,駁異端的講道)》的節錄及反思

講員:陳偉明博士

場合:ABSCC General

日期:2024年3月29日

專題文章 \# 05
【奧古斯丁《講章 242(關於身體復活,駁異端的講道)》的節錄及反思】
作者:陳偉明博士

===============================

陳偉明博士(註1)

教父神學家及牧者奧古斯丁講章 242是他寫於主後411年復活節期間關於身體復活的講道,主題是基督真的肉身從死裡復活了,以駁斥當時異端.以下是當中的一些節錄和當中的反思.

第一段的重點是「基督真的以肉身從死裡復活了」:

1. 在這記念主復活的神聖日子裡,讓我們在主的幫助下盡可能地討論肉身的復活.你看,這就是我們的信仰,這就是我們的主耶穌基督在肉身中應許給我們的禮物,這禮物的樣本首先出現在他本人身上.你看,他不僅要預言他應許給我們的結局,而且還要展示出來.當時和他在一起的人確實看到了這一切;在驚慌失措之間,以為自己看到的是一個靈魂之後,他們領悟到了他的身體是實實在在的.畢竟,他不僅對他們的耳朵說了話,還對他們的眼睛顯現了形體;而且,他不滿足於向他們的視覺展示自己,還主動提出讓他們觸摸和感受自己.你看,他說:「你們為甚麼愁煩?為甚麼心裡起疑念呢?」我的意思是說,他們以為自己看到了一個靈魂.他說:「你們為甚麼愁煩?為甚麼心裡起疑念呢?你們看我的手和腳;你們摸摸看,靈魂是沒有骨和肉的,你們看,我是有的.」(路 24:36-39).
有人反駁這一顯而易見的真理.畢竟,你還能指望那些品嘗人間事物的普通人怎麼做呢?我的意思是,祂是神,而他們不過是人.但神知道人的意念,是虛妄的(詩 94:11).對於世俗的,擁抱物質主義的人來說,他們的觀察習慣完全決定了他們的理解方式.他們習慣了看到甚麼,才能相信甚麼;他們不習慣看到的,也就不能相信它.然而,上帝創造的神蹟超越了我們的習慣,因為祂是上帝.事實上,每天都有那麼多以前不存在的人出生,這比幾個曾經存在的人復活更偉大的神蹟;然而,這種神蹟並沒有被認真地考慮和欣賞,而是因為[被視為]太普通而[感到]沒趣.基督復活了. . . 他 . . . 肉體掛在十字架上,放棄了靈魂,被放在墳墓裡.他把肉體活生生地呈現出來,因為他活在肉體裡.我們為何驚訝,為何不信?這是神做的.反思那使事情發生的主,你就會消除一切懷疑的可能性.(註2)

「他們習慣了看到甚麼,才能相信甚麼;他們不習慣看到的,也就不能相信它.」可能也是我們活在世上的天然模式.基督真的以肉身從死裡復活了,衝擊不少認為身體,物質是沒永恆價值,因而可以放縱肉體於今生的一種諾思底主義的二元論思想.今天我們在這方面需要更新呢?例如:我們對身體的態度是為預備將來榮耀的身體,好好看待它,不殘害,不放縱,用身體作對神國有益的事?

第二段的重點是「主復活後為甚麼吃東西?」

2. 因此,有人會問,他們在自己的肉體中經歷過的這種身體衰殘,在死人復活時是否也會出現.我們說不會.他們回答我們說:「如果不會有任何身體衰殘,為甚麼有『吃』這回事呢?」如果沒有「吃」這回事,為甚麼主復活後還要吃東西? 剛才讀福音書的時候,我們聽到,當他把自己活生生地呈現在門徒的眼前和手中時,他認為這還不足以證明他的身體是真實的;但他又說,你們這裡有甚麼吃的嗎?他們給了他一份烤魚和一塊蜂房,他吃了,把剩下的給了他們(路 24:41-43).

所以有人問我們:「. . .那麼主耶穌為什麼還要吃呢?」你能讀到他吃了東西;你能讀到他餓了嗎?他吃東西是他的能力,而不是他的需要.如果他想吃東西,他就會有需要.同樣,如果他不能吃東西,那就意味著他的能力下降了.. . .

復活後的耶穌吃東西是因為他表明自己有能力吃.他並不因餓而吃.想起主耶穌在地上時曾在曠野,受魔鬼的試探.「他禁食四十晝夜,後來就餓了.」(參太4:1-4)人活著,不單靠食物(乃是靠神口裡所出的一切話);將來在新天新地,屬主的人活著再完全不靠食物,只靠救恩,但他們還能吃,與基督共享筵席,願主給我們這視野,靠祂復活的大能,今生不被食物或其他慾望操控.

第三段的重點是「主為什麼帶著傷痕復活?」

3. 他們又說:「人死後身體上的缺陷也會出現在復活的身體嗎?」我們回答說:「不,缺陷不會在復活的身體上有份.」有人問我們:「那為甚麼主會帶著傷痕復活呢?」 我們該怎麼回答呢? 這也是能力的問題而不是需要的問題?他希望以這樣的方式復活,他希望以這樣的方式向一些心存疑慮的人展示自己.肉體上的傷疤治癒了不信的創傷.

耶穌展示自己的傷口也是因為祂選擇這樣做,以便安慰他的跟隨者,讓他們相信真的是耶穌顯現了.復活的耶穌沒有受傷,我們復活後的身體也不會有任何腐敗或缺陷.「因他受的刑罰我們得平安:因他受的鞭傷我們得醫治.」(賽53:5b)羔羊婚宴主也帶他復活身體上的鞭傷和釘痕跟祂坐席.這事實如何影響你面對自己的傷痕?

$^{41}$
「天使對我說:『你要寫下來:凡被請赴羔羊婚宴的人有福了!』他又對我說:『這些都是神真實的話.』」(啟19:9)

註1筆者為Carey Theological College 神學助理教授,亦為加拿大建道中心候任總幹事暨建道神學院候任神學系助理教授.

註2中譯文為筆者Jimmy Chan譯自Augustine, “Sermon 242,” Sermons 230–272B on Liturgical Seasons, ed. 註John E. Rotelle, trans. Edmund Hill, vol. 7, The Works of Saint Augustine: A Translation for the 21st Century (Hyde Park, NY: New City Press, 1993), 77–78. 下同.
\newpage



\section{}
\label{sec:11}
\textbf{第一篇遺憾}
\newline
\newline
~~~~ 日期: 2024-03-29
\newline
\newline
\hyperref[sec:10]{\small{< < < PREV SERMON < < <}}
~
\hyperref[sec:index]{\small{[返主目錄]}}
~
\hyperref[sec:12]{\small{> > > NEXT SERMON > > >}}
\newline
\newline
$^{1}$李詩琳博士 - 第一篇遺憾

講題:第一篇遺憾

講員:李詩琳博士

場合:ABSCC General

日期:2024年3月29日

專題文章 \# 04
【第一篇遺憾】
作者:李詩琳博士

詩絲細語

大家好!我是李詩琳,就以「詩絲細語」作為這個專欄的名稱吧!希望透過專欄跟大家分享不同的內容,從聖經到文學,從教會到日常,細細分享,娓娓道來,希望能引起共鳴與迴響!

第一個系列會分享<撒母耳記>,這一卷書伴隨了自己十多個寒暑,終於在2023年完成了博士論文的研究.當中的大人物,小角色,人生的悲歡離合,人性的美與醜……道盡人生的無奈與無常.

先從撒母耳記中的女性作為分享的內容,大家還記有哪些女性角色出現在書卷中呢?哈拿,米甲,亞比該,隱多珥的女巫,拔示巴,他瑪,提哥亞婦人,愛雅的女兒利斯巴.讓我們先看看哈拿的人生吧!

==============================

遺憾

撒上一1-8
1以法蓮山區有一個拉瑪的瑣非人,名叫以利加拿,他是蘇弗的玄孫,託戶的曾孫,以利戶的孫子,耶羅罕的兒子,是以法蓮人.2他有兩個妻子:一個名叫哈拿,另一個名叫毗尼拿.毗尼拿有孩子,哈拿卻沒有孩子.3這人每年從本城上到示羅,敬拜萬軍之耶和華,向他獻祭.在那裡有以利的兩個兒子何弗尼和非尼哈當耶和華的祭司.4每逢獻祭的日子,以利加拿把祭肉分給他的妻子毗尼拿和毗尼拿所生的兒女.5他給哈拿的卻是雙分,因為他愛哈拿.耶和華卻不使哈拿生育.6她的對頭毗尼拿因耶和華不使哈拿生育,就常常惹她發怒,要使她生氣.7年年都是如此.每當她上到耶和華殿的時候,毗尼拿就這樣惹她發怒,以致她哭泣不吃飯.8她丈夫以利加拿對她說:「哈拿,你為何哭泣?為何不吃飯?為何傷心難過呢?有我不比有十個兒子更好嗎?」

第一個在撒母耳記出場的女性是哈拿.哈拿的出場是典型的中國傳統女性所面對的悲劇,一個沒有生育的女性,正所謂:「不孝有三,無後為大.」可以體會哈拿所受的壓力.敘事者清楚指出:以利加拿有兩個妻子:一名哈拿,一名毗尼拿.毗尼拿有兒女,哈拿沒有兒女.或許正正因為哈拿沒有兒女,以利加拿再娶毗尼拿為他傳宗接代.
以利加拿卻不是典型的傳統男人,以兒女為先,他沒有特別寵愛有兒女的毗拿尼.敘事者強調:「4每逢獻祭的日子,以利加拿把祭肉分給他的妻子毗尼拿和毗尼拿所生的兒女.5他給哈拿的卻是雙分,因為他愛哈拿.耶和華卻不使哈拿生育.」(撒上一4-5)

哈拿有丈夫的愛,無奈卻沒有兒女;毗尼拿有兒女,卻得不著丈夫的愛.人生就是如此的無奈,得到這,卻失去那! 這是人生的遺憾 !

毗拿尼或許以為恃著有兒有女必定可以成為丈夫心中的第一,然而卻事與願違,難免心中充滿了苦澀,她藉著奚落哈拿而發洩心中的苦澀:「6她的對頭毗尼拿因耶和華不使哈拿生育,就常常惹她發怒,要使她生氣.」(撒上一6)

大家試想想毗拿尼如何惹怒哈拿呢?「你這個沒兒子的女人,真是毫無價值!」「我都不知道丈夫為什麼愛你這個沒兒子的女人!」「你看我兒女成群,是不是很羨慕呢?你食雙分的祭肉又如何?」或許大家可以想出更多潑辣的說話,如果你是哈拿又會如何應對呢?以牙還牙,以眼還眼?「我哈拿雖然無兒無女,但有丈夫愛我,你不要妒忌了!」「我仍是丈夫心中唯一最愛!」
人的說話反映人內心的實況,苦澀的人發出苦澀的言語,正如雅各所說:「11泉源能從一個出口發出甜苦兩樣的水嗎?12我的弟兄們,無花果樹能生橄欖嗎?葡萄樹能結無花果嗎?鹹水也不能流出甜水來.」(雅三11-12)

哈拿面對毗尼拿的惡言惡語,她沒有選擇以惡報惡,但毗尼拿的話也大大刺痛她的心,讓她感到無奈與傷痛:「7年年都是如此.每當她上到耶和華殿的時候,毗尼拿就這樣惹她發怒,以致她哭泣不吃飯.」(撒上一7)每次上聖殿獻祭之時,是讓毗尼拿感到苦澀之時,同時是哈拿面對奚落之時,年年如是 .可以想像一家人去聖殿敬拜神,原先應該是滿心歡喜的時候,然而卻是兩個女人最傷痛的時候.毗尼拿因著未能得到丈夫的疼愛而傷痛,狠狠吐出惡言;而哈拿因著毗尼拿的冷嘲熱諷而傷痛,滴滴的眼淚流不完.

$^{41}$
愛哈拿的丈夫毗尼拿似乎只看見哈拿的眼淚,並沒有完全體會哈拿內心的傷痛:「8她丈夫以利加拿對她說:『哈拿,你為何哭泣?為何不吃飯?為何傷心難過呢?有我不比有十個兒子更好嗎?』」(撒上一8)丈夫的心也很難過,因他看見所愛的妻子難過,他也感到難過,他很想安慰妻子,然而他的安慰卻進一步在哈拿的傷口上洒鹽.在哈拿心中,兒子確實是比愛她的丈夫更為重要,這是丈夫以利加拿的無奈,他希望因著自己的愛而能令哈拿感到滿足,然而在哈拿心中, 只有兒子才是她真正的滿足與快樂.

在這個家庭中,各有各的期望,各有各的遺憾.毗尼拿渴望丈夫的愛卻得不著;以利加拿渴望妻子因他而滿足卻感到不足;哈拿渴望兒子卻不能生育.每個人都有著自己的無奈與遺憾,每個人面對人生的不完滿時,都有著不同的反應與感受.

在我們的人生中,你有什麼渴望?美滿的婚姻?兒女成群?成功的事業?健康的身體?這些渴望也許是美好的,然而,人生也並非事事如願,得到這或失去那,這就是人生.求主讓我們為所擁有的而感恩,在遺憾中仍看見主恩手的拖帶,仍能聽見主對我們說:「我的恩典是夠你用的,因為我的能力是在人的軟弱上顯得完成!」(林後十二9)
\newpage



\section{}
\label{sec:12}
\textbf{依納爵的心窄的轉化方法}
\newline
\newline
~~~~ 日期: 2024-04-18
\newline
\newline
\hyperref[sec:11]{\small{< < < PREV SERMON < < <}}
~
\hyperref[sec:index]{\small{[返主目錄]}}
~
\hyperref[sec:13]{\small{> > > NEXT SERMON > > >}}
\newline
\newline
$^{1}$陳顯宗博士 - 依納爵的心窄的轉化方法

講題:依納爵的心窄的轉化方法

講員:陳顯宗博士

場合:ABSCC General

日期:2024年4月18日

依納爵的心窄的轉化方法
April 18, 2024|ABSCC General

專題文章 \# 06
【依納爵的心窄的轉化方法】
作者:陳顯宗博士

依納爵的心窄的轉化方法

(一)辨別神類
「辨別神類」是天主教會的核心概念,被歷代靈修學者所看重.初期教會早已提及這概念,如:《十二使徒遺訓》第一章就提到人要分辨「生命之路」和「死亡之路」.註1 眾教父零散地談論辨別神類,以「辨別」(Discretio)這詞去表達約翰和保羅書信在這方面的教導.註2

到了中世紀,伯爾納鐸認為辨別神類是聖神所賜的恩賜.註3 阿奎那指出辨別神類可分為「普通的辨別神類」及「神恩性的辨別神類」兩種.第二種是聖神白白的恩賜,能分辨人心裡的秘密和真假先知.第一種則與「明智之德」結合.因此,他不稱第一種為辨別神類,只稱第二種為辨別神類.日爾松寫有《論辨別神類》,認為辨別神類分為「普通的」及「神恩性的」兩種.德尼也著有《論辨別神類》一書.他集各家大成,並寫有辨別神類之規則.註4 由此可見,在依氏之前已有辨別神類這個概念出現,註5但沒有《神操》「辨別神類的規則」那麼有系統.註6

杜納指出,依氏提及的辨別神類是教導人辨別他們的經歷令他們更趨向上主還是遠離祂.註7 巴克利提到依氏在第二週辨別神類的規則教導人如何分辨化為善者的惡神:某現象表面上是善的,但其實是惡的,因它令人與上主疏遠.註8

心窄令依氏感到「很困擾」,「不安」,「痛苦萬分」,「很有害處,最好置之不理,但辦不到」,甚至「多次受強烈的誘惑」而想自殺.註9這令他辨別到他與上主的關係越來越遠,從而察覺到那是惡神偽裝成光明的天使去攻擊他.因此,他便決心不再順從牠的誘惑而不再重複認罪,繼而擺脫了心窄.註10

(二)改善神觀
依氏獲得的三一神視 註11,卡陶內河畔的經歷及作聖子僕人的神視等恩寵令他感受到上主的愛,接納和差遣,使他的神觀被改變,致使他不再過份地認為上主是位嚴厲的主,而是充滿慈愛的主.因此,他不再懼怕自己因遺漏認罪而不被上主所愛.由此看來,這些神秘經歷拆除了他的聖人英雄主義,苦修主義和完美主義,讓他單純用愛去回應上主的愛而與祂建立親密關係.

若將他在卡陶內河畔經歷前後的神觀作比較,可見強烈的對比.在此之前,他認為人要努力做得很好,上主才施恩.故此,他花盡努力去擺脫心窄,例如他在長時間考慮後決定禁食多天,甚至直至自己頻臨死亡或上主救拔他才停止.註12此後,他認為上主是充滿慈愛和憐憫的,甚至無須前因便賜下恩寵,他就由一個效法先賢而從事祈禱和苦修的孤獨朝聖者轉化成一位服侍人的領袖.註13 神觀的突破令依氏在靈性上指導別人時,均不再提及那些聖人英雄主義,苦修主義和完美主義等思想,轉為強調上主的慈愛.

小結
依氏的心窄經歷是可怕的,但他辨別到那些心窄並不是來自上主,而是來自撒但,最終使他得釋放.註14縱使他受著屬靈經典,靈修傳統,負面神觀及個人性格所限,上主的慈愛和恩典也可以超越它們而作工,拆除了他的聖人英雄主義,苦修主義和完美主義,並將他轉化成靈性指導者.即使他沒有完全被醫治,也可以被祂使用.註15

反思:
1.參照依納爵的經歷,你曾經歷過心窄(良心對罪過份敏感)的困擾嗎?

2.在你人生的旅途裡,你曾經歷上帝帶領你走出困局嗎?

$^{41}$
3.承上題,若上帝曾經帶領你走出困局,對你的生命有帶來轉變嗎?

註釋
1.高士傑:〈辨別神類簡史〉,載光啟編輯室編:《分辨神類》,再版(臺北:光啟,2018),頁5.
2.高士傑:〈辨別神類簡史〉,頁6.
3.「聖神」即基督教的「聖靈」.
4.高士傑:〈辨別神類簡史〉,頁8—12.
5.參高士傑:〈辨別神類簡史〉,頁5—8「一,教父時代的辨別神類」,頁8—12「二,中世紀的辨別神類」.其他關於「辨別神類」的書籍可參拉蒙‧鮑狄斯塔神父:《避靜,祈禱與分辨:依納爵神操101問答》,謝詩祥,鄭兆沅譯(臺北:光啟,2012);Timothy M. Gallagher, The Discernment of Spirits: An Ignatian Guide for Everyday Living (New York: Crossroad, 2005); Stefan Kiechle, The Art of Discernment: Making Good Decisions in Your World of Choices, The Ignatian Impulse Series (Notre Dame, IN: Ave Maria Press, 2005).
6.參剛斯:《神操新譯本》,313—336號;高士傑:〈辨別神類簡史〉,頁12.
7.Jules J. Toner, A Commentary on Saint Ignatius’ Rules for the Discernment of Spirits: A Guide to the Principles and Practice (St. Louis, MO: The Institute of Jesuit Sources, 1982), 10—11.
8.M. J. Buckley:〈神操中分辨神類的規則的結構〉,胡國楨編譯,載光啟編輯室編:《分辨神類》,頁119.
9.依納爵:《自述小傳》,22—24號.
10.依納爵:《自述小傳》,22,24,25號.
11.「神視」即基督教的「異象」.
12.依納爵:《自述小傳》,24號;de Dalmases, Ignatius of Loyola, 59.
13.de Dalmases, Ignatius of Loyola, 63.
14.Barry, Finding God in All Things, 47.
15.他日後仍有心窄,參Brodrick, Saint Ignatius Loyola, 109—110.
\newpage



\section{}
\label{sec:13}
\textbf{在後現代福音的大能}
\newline
\newline
~~~~ 日期: 2024-05-02
\newline
\newline
\hyperref[sec:12]{\small{< < < PREV SERMON < < <}}
~
\hyperref[sec:index]{\small{[返主目錄]}}
~
\hyperref[sec:14]{\small{> > > NEXT SERMON > > >}}
\newline
\newline
$^{1}$陳耀鵬牧師 - 在後現代福音的大能

講題:寫給徘徊應否進入神學學習的你

講員:陳耀鵬牧師

場合:ABSCC General

日期:2024年5月2日

在後現代福音的大能 Power of the Gospel in the Postmodern society
May 2, 2024|ABSCC General

專題文章 \# 07
【在後現代福音的大能】
作者:陳耀鵬牧師

在後現代福音的大能
各位同行者

葛培理佈道團在四月六日上午九時至下午四時十五分於大溫哥華聖道堂舉辦 了一個 2024 華人福音大會.我不能參加早上的聚會,因為要負責教會網上 的耆英崇拜.不過很開心,我可以在下午一時三十分至三時三十分分參加了 這個福音大會在下午的聚會,與接近一千位的弟兄姊妹一同聆聽了兩個講座
- 新加坡聖經學院院長謝木水牧師博士的「在後現代社會中讓佈道事工煥發 新生」及葛培理牧師的孫兒葛衛理牧師的「福音的大能」. 我從中獲益良 多,在此希望簡單與你分享,彼此勉勵.

後現代是這一代對之前被稱為現代的那一代一個相對的形容.後現代對現代以理性為主導非常不滿,以致對高舉理性的基督教的傳道方式反感.一些後現代的人會視傳福音是一種宗教狂熱的現象,侵犯自由信仰的權利,強迫洗腦的方法及手段,甚至是破壞宗教和諧的暴力行為.謝牧師提醒信徒需要反省基督教信仰是否趨向甚至被外人認為等同現代的理性主義?基督教制度是否服侍現代霸權?基督教文化是否表達這種現代精神? 謝牧師首先用提摩太 後書第二章 11 至 13 節指出耶穌的「生」可以醫治現代理性,耶穌的「死」 可以醫治現代霸權,耶穌的「復活」可以醫治現代心靈.福音叫現代唯我獨 尊的理性回歸上帝,道成肉身可以打破現代化心靈的鐵籠,復活生命能夠更 新個人與自然,叫信徒更有道德勇氣. 他提醒作為一個生活在後現代的傳道人,首先需要曉得怎樣做「人」,要好好認識時代,自己經歷福音,擁有福 音的人格及活像基督 .在「傳」的技巧上切忌驕傲 ,態度要謙卑,他人為 先及言行一致. 最後「道」的內容需要使人覺得有價值意義及感受到神的真 實與寶貴.謝牧師博士的分享資料非常豐富,理論非常強,可能會有些人覺 得有一種難以消化的感覺.這使我記起我的建道同學與前建道同事何啟明牧 師對前現代,現代與後現代的分析.

前現代(600-1500)
特徵: 接受權威及服膺傳統            
名句: 安密瑟:「我相信以致我可以明白 
代表人物: 牧師                          

現代(1500-1960)     
特徵: 獨立恩老和客觀分析  
名句: 笛卡兒:「我思故我在 
代表人物: 科學家              

後現代(1960--)                   
特徵: 主觀經歷及群體認同               
名句: 「我歸醫故我在」或「我消費故我在 
代表人物: 搖滾樂手                         


$^{41}$跟著大會的壓軸講員葛培理牧師的孫兒葛衛道牧師的「福音的大能」是一篇簡單易明,充滿故事,很多見證的講章.他首先提到自己的太公太婆多年前是中國的宣教士,在江蘇省傳道.他的婆婆亦即葛培里牧師的太太 Ruth 甚 至在中國出生.他說若他們仍世,甚至今日在場,必定喜極而泣,因為在 他們一生中,特別是太公太婆一定沒有見過這麼多的華裔基督徒聚集,更何 況是一班願意參與傳福音講座的基督徒聚在一起學習怎樣傳福音.

他跟著從使徒行傳第八章腓力向安提阿伯太監傳福音的故事指出福音大能的 四方面.傳福音必須從神的命令開始,其次傳福音的人亦需要聽從神的指 示,另外傳福音必須以個人為對象甚至只以一個人為對象,最後傳福音的內 容必須是神的話語.他提到兩個關於他爺爺葛培里牧師佈道的故事去引證他 的第三及第四點,令我印象很深刻.1959 年葛牧師在澳洲舉行大型佈道會, 有百分之二十六的澳洲人在那次佈道會聽到葛牧師的信息,百分之四的澳洲
人歸向基督,其中一位是原本鄙視基督教及憎恨基督徒的巴士司機,因為他 想知道為什麼他接載的一班基督徒聽到葛牧師的信息後那麼開心,所以自己 便好奇的去聽,然後信主.跟著全家得救,甚至後代成為影響澳洲的一班基 督徒.另外在 1953 年葛培里牧師在美國達拉斯市的一晚大型佈道會後,第 二天早上他與一位屬靈長者散步,在聊天時,葛牧師表白他昨天晚上雖然有
人信主,但他自覺好像缺乏一種屬靈深度,能力與興奮.那位屬靈長者提醒 他因為他昨天的信息中,沒有提到十字架.

這使我記起這首聖詩 -

1.求主使我靠十架 在彼有生命水 \\
寶血由十架流下 \\
白白賜人洗罪 \\
十字架十字架 \\
永是我的榮耀 \\
我眾罪都洗清潔

2.我來到主十架前 \\
蒙救主的愛憐 \\
祂賜我聖靈亮光 \\
照亮我的心田 \\
十字架十字架 \\
永是我的榮耀 \\
我眾罪都洗清潔 \\
唯靠耶穌寶血

3.求主使我依十架 \\
思念昔日情景 \\
常在十架蔭庇下 \\
緊緊跟主前行 \\
十字架十字架 \\
永是我的榮耀 \\
我眾罪都洗清潔 \\
唯靠耶穌寶血

4.儆醒等候十架前 \\
盼望信心加增 \\
直到走完世路程 \\
天家永享安穩 \\
十字架十字架 \\
永是我的榮耀 \\
我眾罪都洗清潔

$^{81}$
這首讚美詩提醒我們,十字架在福音信息中的中心地位,以及它給個人和社 區帶來的轉化力量.

陳耀鵬
2024-04-29
\newpage



\section{}
\label{sec:14}
\textbf{問世間,情是何物?}
\newline
\newline
~~~~ 日期: 2024-07-31
\newline
\newline
\hyperref[sec:13]{\small{< < < PREV SERMON < < <}}
~
\hyperref[sec:index]{\small{[返主目錄]}}
~
\hyperref[sec:15]{\small{> > > NEXT SERMON > > >}}
\newline
\newline
$^{1}$李詩琳博士 - 問世間,情是何物?

講題:問世間,情是何物?

講員:李詩琳博士

場合:ABSCC General

日期:2024年7月31日

問世間,情是何物?
July 31, 2024|ABSCC General

專題文章 \# 08
【問世間,情是何物?】
作者:李詩琳博士

問世間,情是何物?

在撒母耳記中出現了女子,整體上都是比較正面的角色,而且是較為主動,有主見,有行動的女子.今次介紹的米甲,也可算是女中豪傑.

在撒母耳記中第一句介紹米甲出場的說話是:「掃羅的次女米甲愛大衛」(撒上十八20)究竟米甲為何愛大衛 ?聖經沒有記述.或許是因著大衛打敗歌利亞的英勇表現,他的勇武吸引了米甲;或許是因為大衛俊俏的外貌:「面色光紅,雙目清秀,容貌俊美」(撒上十六12);或許是當大衛為父親掃羅驅鬼時,彈出優雅悅耳的琴聲;又或許是大衛做事精明,富有領袖魅力……總而言之,米甲就是深深地,並且採取主動地愛著大衛.

在聖經中唯獨此處是開宗明義地指出一個女子愛上一個男子.米甲是現代版愛情主動的追求者,相信,如果米甲生於現在,一定是愛得轟轟烈烈的女子!可惜,米甲生於古時的以色列,她雖然主動追求所愛,然而她的生命卻是悲劇結束.

至於米甲所愛的大衛是否也愛她?看看敘事者的說話:「26掃羅的臣僕把這話告訴大衛,大衛就歡喜作王的女婿.日期還沒有到,27大衛和跟隨他的人起來前往,殺了二百非利士人,將包皮足數交給王,為要作王的女婿.於是掃羅將女兒米甲嫁給大衛.」(撒上十八26-27)

敘事者沒有說大衛愛米甲,只說大衛喜歡做王的女婿(son-in-law).

米甲採取主動愛大衛,她用行動表達自己的愛,當大衛被父親追殺,是米甲提醒大衛死期快到,於是建議大衛當夜從窗戶逃命:

「大衛的妻子米甲對大衛說:『你今夜若不逃命,明日就要被殺.』12於是米甲將大衛從窗戶縋下去,讓他走;大衛就逃走,躲起來了.」(撒上十九11-12)

曾冒著生命危險救出大衛的米甲,當大衛離開王朝時,她卻被父親掃羅轉嫁給另一個男人,究竟當時米甲的心情是如何?是出於無奈,還是看見丈夫大衛應該無機會再回朝,因而選擇再嫁,聖經沒有詳細交代.

米甲愛未成王的大衛,大衛成王後,卻不再愛髮妻米甲.當押尼珥主動提出要統一以色列國,大衛王提出的第一個條件竟然是取回米甲,可見大衛王知道妻子米甲改嫁了另一個男人,耿耿於懷.在大衛王的眼中,米甲在他的心中最重要是什麼?

13大衛說:「好!我與你立約.但有一件事我要求你,你來見我面的時候,除非把掃羅的女兒米甲帶來,就不必來見我的面了.」14大衛派使者到掃羅的兒子伊施‧波設那裡,說:「你要把我的妻子米甲歸還我;她是我從前用一百非利士人的包皮所聘定的.」(撒下三13-14)

米甲是大衛用一百非利士人的包皮所聘定的.大衛沒有記念米甲曾救他逃離掃羅的手,他沒有記念米甲曾經深深愛著他,在大衛心目中,只記掛著曾冒死闖入非利士人陣地聘娶的聘禮,米甲是他用血換來的妻子!

$^{41}$當米甲重回大衛身邊,不是歡天喜地,不是破鏡重圓,大衛不是因愛再取回米甲,而是因為不甘心被人奪走自己的妻子,他誓要奪回來,以示主權.米甲從第二任丈夫帕鐵手中被奪回,敘事者沒有形容米甲的心情,只是側面描寫了丈夫帕鐵的反應:

「15伊施‧波設就派人去,把米甲從拉億的兒子,她丈夫帕鐵那裡帶來.16米甲的丈夫跟著她,一面走一面哭,直跟到巴戶琳.押尼珥對他說:『你回去吧!』帕鐵就回去了. 」(撒下三15-16)

此時,敘事者記述米甲的丈夫不是大衛而是帕鐵.帕鐵送別妻子,一面走一面哭,可見情深淚深,何等悲涼,何等無奈:「你回去吧!」大權在握的大衛王及押尼珥面前,一個無權無勢的帕鐵只能屈從.帕鐵對米甲的深情,是大衛身上沒有的.

米甲再出現在大衛面前,大衛王已經不是昔日曾經讓她怦然心動的年青大衛,此時此刻,在米甲心中,不再有愛,只有恨.

昔日曾在窗戶救出大衛,今日在窗戶卻鄙視大衛王.當大家歡喜快樂地迎接約櫃時,米甲用輕視的眼神望著跳著舞的大衛王.心想:這個大衛王就是昔日打死歌利亞的大衛?就是我曾冒死救出的丈夫?是用二百非利士包皮將我聘娶的大衛?米甲或許終於看清楚大衛不是因著愛而為她冒死,看清楚在大衛心中,對她不曾有愛.

20大衛回去要為家裡的人祝福,掃羅的女兒米甲出來迎接他,說:「以色列王今日有好大的榮耀啊!他今日在臣僕的使女眼前露體,如同一個無賴赤身露體一樣.」(撒下六20)

當大衛為家人送上祝福時,掃羅的女兒(不再稱為大衛的妻)卻狠狠地批評大衛的無恥,言下之意,在米甲心中大衛如同一個無賴的惡人.

23掃羅的女兒米甲,直到死的那日沒有孩子..(撒下六20-23)

這是敘事者對米甲最後的敘述,米甲以掃羅的女兒作為她一生的稱呼,並且是一個不育的女人.

最後送上,金代詩人元好問《摸魚兒.雁丘詞.邁陂塘》:
問世間,情是何物,直教生死相許.天南地北雙飛客,老翅幾回寒暑.歡樂趣,離別苦,是中更有癡兒女.君應有語,渺萬里層雲,千山暮景,隻影為誰去. 橫汾路,寂寞當年蕭鼓,荒煙依舊平楚.招魂楚些何嗟及,山鬼自啼風雨.天也妒,未信與,鶯兒燕子俱黃土.千秋萬古,為留待騷人,狂歌痛飲,來訪雁丘處.
\newpage



\section{}
\label{sec:15}
\textbf{馬丁‧路德的心窄困擾}
\newline
\newline
~~~~ 日期: 2024-09-05
\newline
\newline
\hyperref[sec:14]{\small{< < < PREV SERMON < < <}}
~
\hyperref[sec:index]{\small{[返主目錄]}}
~
\hyperref[sec:16]{\small{> > > NEXT SERMON > > >}}
\newline
\newline
$^{1}$陳顯宗博士 - 馬丁‧路德的心窄困擾

講題:馬丁‧路德的心窄困擾

講員:陳顯宗博士

場合:ABSCC General

日期:2024年9月5日

專題文章 \# 09
【馬丁‧路德的心窄困擾】
作者:陳顯宗博士

一,心窄困擾

1. 學生時期
學者指出馬丁‧路德(Martin Luther, 1483—1546)的心窄經歷有助他發展出「唯獨因信稱義」這教義.註1在《青年路德:一個精神分析與歷史的研究》一書裡,艾瑞克森(Erik H. Erikson, 1902—1994)從「心理史學」(Psychohistory)的角度去看路德的青年時期.註2 他指出,路德認為自己沒有盡心愛主,觸犯了十誡的第一誡.註3原因是他和父親之間的緊張關係令他產生了負面的神觀,甚至令他對上帝缺乏敬意.註4 他懷疑父親的責打並不是因著慈愛和公義,而是因著無理及仇恨,他將這種懷疑投射在天父那裡.註5 除了被父親責打,他在學校又面對乏味和嚴格的教育.兩者嚴謹的管教加上教會強調最後的審判,令他的人生充斥罪疚感和憂愁的情感,這些情感促使他走進修道院.註6 路德會因一些宗教畫(基督在彩虹上,身邊配有百合花及長劍)而懼怕施行審判的基督.註7

2. 修道院時期
路德在修道院裡焦慮和恐懼日增:
「在至高至聖的上帝面前,路德感到一片惶惑.他以Anfechtung這個字來形容自己的體驗……這個體驗可能是指上帝對人的考驗,也可能是魔鬼對人的攻擊.其中包括了懷疑,混亂,痛苦,激動,駭懼,失望,孤單和絕望;這一切一同起來侵襲人的心魂.」註8

在路德的年代,上主被描繪成一位冷酷無情及不斷嚴懲罪人的審判官,罪人要得救便要行出各樣功德補贖.路德當時相信這主張故長期禁食,卻令自己身體受盡折磨仍未能令內心平安.註9他說:
「我是好修士,我嚴謹地恪守修道會的一切規則;嚴謹到了一個地步,如果曾經有修士因為修行的緣故而得以進入天堂,我敢說那就是我了.所有在修道院認識我的弟兄可以為我見證.假若我繼續這樣下去,終會給守夜,禱告,閱讀和其他工作奪去我的性命.」註10

路德極度努力,為要獲取上帝的恩典.他頻繁地告解,差不多每天都如此.有次他認罪至六小時.他認為每項罪都要告明才可獲赦免.故此,他必須搜遍靈魂每個角落,察考自己的記憶及反思每個動機.七罪宗和十誡對他來說是有幫助的.為免遺漏,路德會重複認罪,甚至檢視他整個人生而令聽告者厭煩且說:「孩子,上帝並沒有生氣,只是你與上帝過意不去.難道你不曉得,上帝吩咐你要存有盼望嗎?」註11

路德將「小罪化大」,使他的導師施道比次(Johann von Staupitz, 1460—1524)也覺得厭煩,故後者說耶穌對雞毛蒜皮的事並無興趣,若要認罪應找些真正的大罪如殺人和姦淫來認.註12 路德痛苦得曾經想自殺,註13但導師的一個重要決定阻止了他.他被委派在大學裡教授聖經,期望神的話能夠解除焦慮.註14

3. 覺悟唯獨因信稱義
羅馬書一章17節令路德受困擾多年,而「義」這個字的解釋最為關鍵.註15 路德覺悟到是憐憫人的上帝賜予人得救的恩典,而不是人靠自己的功德去賺取.註16心窄困擾刺激他發現了一個重要真理:「唯獨信心」(sola fide),使他煥然一新.註17

註釋

1.Joseph W. Ciarrocchi, The Doubting Disease: Help for Scrupulosity and Religious Compulsions (New York: Paulist Press, 1995), 16; Ian Osborn, Can Christianity Cure Obsessive-Compulsive Disorder?: A Psychiatrist Explores the Role of Faith in Treatment (Grand Rapids, MI: Brazos Press, 2008), 46.

2.艾瑞克森:《青年路德》,頁39.

3.Martin Luther, Werke, vol. XXXIII (Weimarer Ausgabe, 1883), 507, quoted in Erikson, Young Man Luther, 27.

$^{41}$
4.Erikson, Young Man Luther, 27.

5.Erikson, Young Man Luther, 54.

6.Erikson, Young Man Luther, 47.

7.羅倫培登:《這是我的立場──改教先導馬丁路德傳記》,古樂人,陸中石譯(香港:道聲,1987),頁14.英原文參Roland H. Bainton, Here I Stand: A Life of Martin Luther (Nashville, TN: Abingdon Press, 1978), 22.

8.羅倫培登:《這是我的立場》,頁27;英原文參Bainton, Here I Stand, 31.

9.楊東川:《馬丁路德的痛苦與狂喜》(臺北:永望文化事業,1996),頁33..

10.羅倫培登:《這是我的立場》,頁31—32;英原文參Bainton, Here I Stand, 34.

11.羅倫培登:《這是我的立場》,頁44;英原文參Bainton, Here I Stand, 41.

12.Scheel, Dokumente zu Luthers Entwicklung, 487, quoted in Erikson, Young Man Luther, 150—151.

13.Martin Luther, D. Martin Luthers Werke: kritische Gesamtausgabe, 31. Band, Zweite Abteilung (Weimar: Hermann Böhlaus Nachfolger, 1914), 348.

14.Osborn, Can Christianity Cure OCD, 58.

15.Osborn, Can Christianity Cure OCD, 59.

16.Martin Luther, D. Martin Luthers Werke: kritische Gesamtausgabe, Tischreden, 3. Band (Weimar: Hermann Böhlaus Nachfolger, 1914), 228.

17.Osborn, Can Christianity Cure OCD, 67.
\newpage



\section{}
\label{sec:16}
\textbf{訂婚的變奏—掃羅遇見井旁打水的少女}
\newline
\newline
~~~~ 日期: 2024-11-14
\newline
\newline
\hyperref[sec:15]{\small{< < < PREV SERMON < < <}}
~
\hyperref[sec:index]{\small{[返主目錄]}}
~
\hyperref[sec:code]{\small{> > > NEXT SERMON > > >}}
\newline
\newline
$^{1}$李詩琳博士 - 訂婚的變奏—掃羅遇見井旁打水的少女

講題:訂婚的變奏—掃羅遇見井旁打水的少女

講員:李詩琳博士

場合:ABSCC General

日期:2024年11月14日

訂婚的變奏—掃羅遇見井旁打水的少女
November 14, 2024|ABSCC General

專題文章 \# 10
【訂婚的變奏——掃羅遇見井旁打水的少女】
作者:李詩琳博士

在撒母耳記中第三處出現有關女子的情節是在掃羅出場時,大家記得掃羅是如何在故事中登場呢?敘事者介紹了掃羅的家譜:
1有一個便雅憫人名叫基士,是便雅憫人亞斐亞的玄孫,比歌拉的曾孫,洗羅的孫子,亞別的兒子,是個大能的勇士.2他有一個兒子名叫掃羅,又健壯,又英俊,在以色列人中沒有一個可以與他相比;他比眾百姓高出一個頭. (撒上九1-3)

敘事者刻意形容掃羅的外表:健壯,英俊,高大,簡直是超群絕倫,鶴立雞群,用現代術語:很吸睛!

初步了解了掃羅的家庭背景及外貌,接著是他出場時第一個行動是協助爸爸尋找幾頭驢.大家可以留心一下,在聖經中,有很多人物出場時都是按著爸爸的吩咐去找人或找物,接著就出現了人生重大的轉捩點.大家想起了有哪些人物呢?

創世記中的約瑟,爸爸吩咐他去找哥哥們,後來就被賣為奴;大衛也是聽爸爸吩咐找戰場上的哥哥們,後來就單人挑戰歌利亞.在這裡,掃羅因著爸爸吩咐尋驢,而最後尋得撒母耳.這是敘事文體中的其中一個寫作手法:典範場景.

典範場景是聖經敘事者慣用的手法,讓讀者熟識當中的場景,他們會預料必定有某些事情發生.奧爾特在他的經典著作《聖經敘述文的藝術》中指出:「典範場景多會用來表現一些日常發生的情況,而聖經卻總在即將實踐一些異常的行動之時,才會觸及一些日常發生的事. 」(頁90)

在掃羅的出場,除了尋驢的典範場景,還有訂婚場景的變奏版.當掃羅未能成功尋驢,心情開始焦急,他擔心爸爸不為驢掛心,反而為他而掛心,他有意回家去了.他的僕人建議去尋找先見的指引.在此情景中,奠下了掃羅人生的基調,他一生不斷尋求指引,直到死前,也要尋找逝者撒母耳的指引.

正當掃羅進城要尋找先見,他們遇到幾個打水的少女:

11他們上坡要進城,遇見幾個少女出來打水,就問她們說:「先見有沒有在這裡呢?」(撒上九11)

這班少女們看見玉樹臨風的掃羅,相信是有點怦然心動,聽聽她們的回話,可以推想她們的心情:

12她們回答說:「有的,看哪,他就在你們前面.快!他今日正來到城裡,因為今日百姓要在丘壇獻祭.13你們一進城,他還沒有上丘壇吃祭物之前,就會遇見他.因為他沒有到,百姓不能吃,必須等他先為祭物祝謝,然後受邀的人才可以吃.現在就上去吧,因為這時候你們會遇見他.」(撒上九12-13)

其實,她們只需要簡單回答: :「有的,看哪,他就在你們前面. 」已經可以寫上句號.然而,為了多看這個英俊高大的掃羅,她們是興奮的,雀躍的,我們可以想像一下她們說話的語氣及心情:


$^{41}$「有的,看哪,他就在你們前面.」看哪!是一個親臨其境,趕急的說話用詞;

「快!他今日正來到城裡,因為今日百姓要在丘壇獻祭.」快,也是焦急的;

「你們一進城,他還沒有上丘壇吃祭物之前,就會遇見他.因為他沒有到,百姓不能吃,必須等他先為祭物祝謝,然後受邀的人才可以吃.現在就上去吧,因為這時候你們會遇見他.」

何等詳細的指引,先見在什麼地方,先見將做什麼事,吃祭物的規矩,都細細道出.也間接說出了將要發生在掃羅身上的預言:掃羅是被撒母耳邀請吃祭內的上賓.

奧爾特指出了這段經文是聖經中訂婚場景的變奏:

「訂婚的典範場景是這樣的:未來的新郎,或者是他的代理人,必會千里迢迢地到了外地.然後他會在那裡的某口井旁,遇上一個女子——總會用上נער 的字眼……有人從井裡打水,這若不是那個男的,也準是那個女的.之後,那個少女或者那班少女總會趕緊跑回家中,說有陌生人來了……最後,這個陌生人總會跟那個少女締結婚約.」(頁91)

在掃羅的故事中,這個訂婚場景卻出現了變奏:

「主角離開了井旁的少女,匆匆上路去找尋那一位將要驅使他踏上悲慘命途的神人……主角的婚約本有著圓滿之意,但這一位主人翁卻喪失了這一份圓滿;我們所預期的典範場景出現了偏離,從某些角度看來,掃羅已經遭到孤立了……」

人生就是在主旋律與變奏中進行著,沒有了變奏,又如何突出主旋律的優雅與獨特,求主讓我們在人生的變奏中,仍然看見上帝施恩的手在指揮著,在彈奏著,讓我們生命中的樂曲變得豐富悅耳!
\newpage

\allsectionsfont{\centering}

\setlength\parindent{0pt}
\setlength{\columnsep}{1.25em}
\setlength{\parfillskip}{0pt}
\setlength{\tabcolsep}{1em}
\raggedbottom

\pagenumbering{gobble}


\newfontfamily\leftfont[Path=../fonts/fell_french_canon/, Ligatures=TeX, ItalicFont=IMFeFCit29C.otf, BoldFont=AveriaLibre-Bold.ttf]{IMFeFCrm29C.otf}
\newfontfamily\leftcitationfont[Path=../fonts/frankruehl/]{FrankRuehlCLM-Medium.ttf}
\newfontfamily\centerfont[Path=../fonts/garamond/, Ligatures=TeX, ItalicFont=EBGaramond-SemiBoldItalic.ttf]{EBGaramond-SemiBold.ttf}
\newfontfamily\rightfont[Path=../fonts/averia/, Ligatures=TeX, ItalicFont=AveriaLibre-RegularItalic.ttf, BoldFont=AveriaLibre-Bold.ttf, BoldItalicFont=AveriaLibre-BoldItalic.ttf]{AveriaLibre-Light.ttf}
\newfontfamily\rightcitationfont[Path=../fonts/rashi/]{Mekorot-Rashi.ttf}
\definecolor{hcolor}{HTML}{D3230C}
\definecolor{rcolor}{HTML}{D36F0C}
\newcommand{\chfont}[1]{\centerfont{\huge\textcolor{hcolor}{#1}}}
\newcommand{\leftcitation}[1]{\leftcitationfont{\Large\textcolor{hcolor}{#1}}}
\newcommand{\rightcitation}[1]{\rightcitationfont{\normalsize\textcolor{rcolor}{#1}}}
\newfontfamily\flowerfont[Path=../fonts/fell_flowers/]{IMFeFlow2.otf}

\begin{sloppypar}

\chapter*{\chfont{編按結語}}

\columnratio{0.5,0.5}\begin{paracol}{2}

\fontsize{11}{13}\leftfont \Large \leftcitation{א} \leftfont 余少好文.宏志博覽群書而不忘.善存藏經籍文獻備後時之用。\leftcitation{ב} \leftfont 歸主年時.受友所薦.聞道網海.\switchcolumn\fontsize{11}{13}\rightfont \Large \leftcitation{ח} \rightfont 有見粵道之危.國之封講道千言亦將就至.急之何則為?\leftcitation{ט} \rightfont 嘗聞猶太者之傳承.在其力守口述之

\end{paracol}


\columnratio{0.32,0.32,0.32}\begin{paracol}{3}

\fontsize{11}{13}\leftfont \Large 尤以吳約翰遜者 \switchcolumn[2]\fontsize{11}{13}\rightfont \Large 統.以煉千載不

\end{paracol}

\columnratio{0.32,0.32,0.32}
\begin{paracol}{3}\fontsize{11}{13}\leftfont \Large 為重.其載上之粵語講道緩緩入耳.收之藏其音頻.善妥整存.反復而嚼.受益無窮。\leftcitation{ג} \leftfont 我城我國既限.歷一四一九之不測.肺疫延年.信徒靈長屢受圍創.神州燈臺數盡指日可待.粵道之求與日俱增。\leftcitation{ד} \leftfont 觀乎社、經、法、媒、言、信、網之地.愈趨受鋤.自翔不果.授受壓力.粵道聖言亦愈漸艱難。\leftcitation{ה} \leftfont 況崇基例乎.學苑講道屢逆權勢者.其言末強受壓.舊章盡刪以存其身。講道釋數失傳.徒嘆奈何。

\switchcolumn

\fontsize{11}{13}\centerfont 
\begin{tikzpicture}
    \node (0,0) [xshift=-0.10cm, yshift=-1.0cm, opacity=0.10]{\includegraphics[width=0.30\textwidth]{../ot_frontcover.png}} ;
    \node (0,0) [xshift=+0.20cm, yshift=+2.0cm, opacity=0.10]{\includegraphics[width=0.20\textwidth]{../christ_on_cross.png}} ;
\end{tikzpicture}
\Large 

\leftcitation{ס} \centerfont 詩百又廿七載:
\leftcitation{ע} \centerfont 非耶和華建屋宇.則匠人之經營徒.
\leftcitation{פ} \centerfont 非耶和華衛城邑.則守者之儆醒徒.
\leftcitation{צ} \centerfont 余獻是卷予華人社區.願為福音流通之器.願獻斯微材為祭榮耀上帝.
\leftcitation{ק} \centerfont 阿門

\switchcolumn

\fontsize{11}{13}\rightfont \Large 滅.時越次聖殿期及當今。\leftcitation{י} \rightfont 猶太者力廣納之.筆錄以卷軸.便以傳、閱、頌、攜、守、鎖、抄、譯、釋、編,得書塔木德、密示拿等經傳.家喻戶曉.傳流若芳。\leftcitation{כ} \rightfont 猶太者文以載道.傳其口述.今我輩粵道之傳應當作如是.遂力行粵音識辨之法.載言載道.以盡忠傳粵道以待興。\leftcitation{ל} \rightfont 蒙下賜恩惠.無畏海量字音文書.既馭上帝之道.今廣及粵語講道.重駛編程之技.匯導粵音遂字稿.重塑講道現場.以傚猶太卷軸之舉便以傳流。\leftcitation{מ} \rightfont 是卷乃粵音口述傳之屬.莫通華文白話之語.

\end{paracol}

\columnratio{0.5,0.5}
\begin{paracol}{2}\fontsize{11}{13}\leftfont \Large \leftcitation{ו} \leftfont 斯殺一違儆百逆.既禁壓之.我輩聞風無奈.在所難免。\leftcitation{ז} \leftfont 另有異人例乎.以版權之名.脅網絡頻道之舉.同授礙予粵道之存流。

\switchcolumn

\fontsize{11}{13}\rightfont \Large 惟待後繼來者之傚.以譯釋傳之於神州華文地。\leftcitation{נ} \rightfont 今能排程驅馭圖靈以編彙文檔,其碼長共數千千亦無逢大礙.全蒙上帝保守。

\end{paracol}



\columnratio{1}\begin{paracol}{1}

\fontsize{11}{13}\rightfont \Large
~~~~~~~~~~~~~~~~~~~~~~~~~~~~~~~~~~~~~~~~~~~~~~~~~~~~~~~~~~~~~~~~~~~~~~~~~~~~~~~\leftcitation{ר} \rightfont 二零二三年二月一日

~~~~~~~~~~~~~~~~~~~~~~~~~~~~~~~~~~~~~~~~~~~~~~~~~~~~~~~~~~~~~~~~~~~~~~~~~~~~~~~\leftcitation{ש} \rightfont 米迦勒

~~~~~~~~~~~~~~~~~~~~~~~~~~~~~~~~~~~~~~~~~~~~~~~~~~~~~~~~~~~~~~~~~~~~~~~~~~~~~~~\leftcitation{ת} \rightfont 書於香港

\end{paracol}

\end{sloppypar}
\end{document}
