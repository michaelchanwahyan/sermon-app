\documentclass{book}
%\usepackage[letterpaper, portrait, margin=1cm]{geometry}
%\usepackage[letterpaper, bindingoffset=0.2in, left=1in,right=1in,top=.5in,bottom=.5in,footskip=.25in,marginparwidth=5em]{geometry}
\usepackage[letterpaper, left=1in,right=1in,top=.5in,bottom=.5in,footskip=.25in,marginparwidth=1cm]{geometry}
% ---------------------
% mini-table-of-content
% ---------------------
\usepackage{minitoc}
\setcounter{minitocdepth}{1}
\setlength{\mtcindent}{24pt}
\setcounter{secnumdepth}{-2}
%\renewcommand{\mtcfont}{\small\rm}
%\renewcommand{\mtcSfont}{\small\bf}
%\usepackage{setspace}
%\usepackage{tocloft}
%\setlength\cftparskip{-1.2pt}
%\setlength\cftbeforesecskip{1.3pt}
%\setlength\cftaftertoctitleskip{2pt}
%\renewcommand{\cftsecafterpnum}{\hspace*{02.0em}}
%\renewcommand{\cftsubsecafterpnum}{\hspace*{02.0em}}

% ---------------------------
% Chinese Characters Packages
% ---------------------------
\usepackage{fontspec} 
\usepackage{xeCJK}
\setmainfont{Times}
\setCJKmainfont{BiauKai}
\newfontfamily\sblgoodhebrew{SBL BibLit}[Script=Hebrew,Contextuals=Alternate]
\newfontfamily\sblgoodgreek{SBL BibLit}[Script=Greek,Contextuals=Alternate]

\usepackage{ifpdf,cite,algorithmic,url,tikz}
\usepackage[cmex10]{amsmath}

% ---------------------------
% Hebrew Characters Packages
% ---------------------------
\usepackage{polyglossia}
\setmainfont{Times New Roman}

% -------
% General
% -------
\usepackage{multicol}
\usepackage{multirow}
\usepackage{color,colortbl}
\usepackage{xparse}
\usepackage{pbox}
\usepackage{stackengine}
\usepackage{titlesec}% http://ctan.org/pkg/titlesec
\usepackage{tabularx}
\usepackage{xltabular}
\usepackage{titlesec}
\usepackage{makecell}
\newcommand{\sectionbreak}{\clearpage}

\author{
  Editor, Michael Chan\\
  \texttt{michaelchan\_wahyan@yahoo.com.hk}
}
\usepackage{tocloft}

\usepackage{hyperref}
\hypersetup{
    colorlinks=true, % set true if you want colored links
    linktoc   =all , % set to all if you want both sections and subsections linked
    linkcolor =blue, % choose some color if you want links to stand out
}

% ----------
% Afterword
% ----------
\usepackage{marginnote}
\usepackage{sectsty}
\usepackage{ragged2e}
\usepackage{lineno}
\usepackage{xcolor}
\usepackage{paracol}

\begin{document}

\clearpage
%% temporary titles
% command to provide stretchy vertical space in proportion
\newcommand\nbvspace[1][3]{\vspace*{\stretch{#1}}}
% allow some slack to avoid under/overfull boxes
\newcommand\nbstretchyspace{\spaceskip0.5em plus 0.25em minus 0.25em}
% To improve spacing on titlepages
\newcommand{\nbtitlestretch}{\spaceskip0.6em}
\pagestyle{empty}
\begin{center}
\bfseries
\nbvspace[1]
\Huge
{%\nbtitlestretch
\Large
\textbf{中國神學研究院 粵語講道逐字稿 \\
       Youtube Channel: China Graduate School of Theology
       }}

\nbvspace[1]

{\large
Editor: Michael\\
\texttt{michaelchan\_wahyan@yahoo.com.hk}
}

\nbvspace[1]

{\large
Revision: \texttt{v1.1}\\
Last Update: \today
}


\vfill
\begin{tikzpicture}
    %remove comment for OT cover%\node (0,0) [opacity=0.03]{\includegraphics[width=15cm]{../bible_out/ot_frontcover.png}} ;
    %remove comment for NT cover%\node (0,0) [opacity=0.03]{\includegraphics[width=15cm]{../bible_out/christ_on_cross.png}} ;
    %remove comment for Bible cover%\node (0,0) [xshift=0.8cm, yshift=+2cm, opacity=0.03]{\includegraphics[width=10cm]{./christ_on_cross.png}} ;
    %remove comment for Bible cover%\node (0,0) [              yshift=-2cm, opacity=0.03]{\includegraphics[width=14cm]{./ot_frontcover.png}} ;
\end{tikzpicture}
\vfill

\end{center}

\newpage

\setcounter{tocdepth}{0}
\dominitoc
\begin{multicols}{3}
\addtocontents{toc}{\protect\hypertarget{toc}{}}
\tableofcontents
\end{multicols}

\large
%\twocolumn

% the color definition syntax is as follow:
% \definecolor{name}{system}{definition}
% example: a mono-channel color can be defined as
%          \definecolor{Gray}{gray}{0.9}
% example: an rgb-3-channel color can be defined as
%          \definecolor{LightCyan}{rgb}{0.88,1,1}
%          \definecolor{pink}{rgb}{0.68,0,0.68}

\definecolor{CUV1LightRed}{rgb}{1,0.75,0.75}     % for CUV1
\definecolor{LZZVLightGray}{rgb}{0.9,0.9,0.9}    % for LZZ
\definecolor{KJVVLightGreen}{rgb}{0.75,1,0.85}   % for KJV
\definecolor{CUV2LightYellow}{rgb}{1,1,0.75}     % for CUV2
\definecolor{CNVVLightBrown}{rgb}{1,0.85,0.7}    % for CNV
\definecolor{NRSVLightBlue}{rgb}{0.75,1,1}       % for NRSV
\definecolor{WENLLightPurple}{rgb}{0.95,0.85,0.9}% for WENL
\definecolor{TCV19PaleGreen}{rgb}{0.85,1,0.95}   % for TCV19
\definecolor{MSGVLightWhite}{rgb}{0.98,0.98,0.98}% for MSGV
\definecolor{NETSLightRed}{rgb}{1,0.75,0.75}     % for NETS
\definecolor{JPS1917LightYellow}{rgb}{1,1,0.75}  % for JPS1917
\definecolor{SBLGNTPaleRed}{rgb}{1,0.85,0.80}    % for SBLGNT

\section{目錄}
\label{sec:index}
{ \scriptsize


\begin{xltabular}{\textwidth}{|p{0.15\textwidth} p{0.6\textwidth}|p{0.07\textwidth} p{0.1\textwidth}|}
\hline
    & \hyperref[sec:AoYPWrVX2A4]{18⧸19年度研究院開學崇拜:文明背後的血與記號 -李耀坤博士} & 2018-11-01 & \href{https://youtube.com/watch?v=AoYPWrVX2A4}{\texttt{ AoYPWrVX2A4}} \\
    & \hyperref[sec:Jp_xjonF0nk]{2019年研究院開學崇拜「真係要饒恕?」-- 雷競業牧師} & 2019-09-10 & \href{https://youtube.com/watch?v=Jp-xjonF0nk}{\texttt{ Jp-xjonF0nk}} \\
    & \hyperref[sec:YKdJJovZoTc]{2020年研究院開學崇拜崇拜「你準備好未?」- 李適清博士} & 2020-09-09 & \href{https://youtube.com/watch?v=YKdJJovZoTc}{\texttt{ YKdJJovZoTc}} \\
    & \hyperref[sec:wfhoZve0GwI]{CGST Magazine Vol 5 憂質教育} & 2019-08-12 & \href{https://youtube.com/watch?v=wfhoZve0GwI}{\texttt{ wfhoZve0GwI}} \\
    & \hyperref[sec:fnpbCi1eLhU]{CGST Magazine 創刊號} & 2017-03-07 & \href{https://youtube.com/watch?v=fnpbCi1eLhU}{\texttt{ fnpbCi1eLhU}} \\
    & \hyperref[sec:WhsjJODKaTI]{CGST Magazine 第三期 土地公公} & 2019-08-07 & \href{https://youtube.com/watch?v=WhsjJODKaTI}{\texttt{ WhsjJODKaTI}} \\
    & \hyperref[sec:JYDFJ9wnjB4]{CGST Magazine 第二期 尋思:三代中港・情・意・結} & 2019-08-03 & \href{https://youtube.com/watch?v=JYDFJ9wnjB4}{\texttt{ JYDFJ9wnjB4}} \\
    & \hyperref[sec:tYIP6koYn_I]{CGST Magazine 第二期 尋思:三代中港・情・意・結 - 情懷} & 2017-06-08 & \href{https://youtube.com/watch?v=tYIP6koYn-I}{\texttt{ tYIP6koYn-I}} \\
    & \hyperref[sec:2d8c5hyS5LI]{CGST Magazine 第二期 尋思:三代中港・情・意・結 - 意識} & 2019-07-08 & \href{https://youtube.com/watch?v=2d8c5hyS5LI}{\texttt{ 2d8c5hyS5LI}} \\
    & \hyperref[sec:L5cJtWqteCI]{CGST Magazine 第二期 尋思:三代中港・情・意・結 - 意識「信仰與身分認同」} & 2017-06-10 & \href{https://youtube.com/watch?v=L5cJtWqteCI}{\texttt{ L5cJtWqteCI}} \\
    & \hyperref[sec:82HZNC1kimk]{CGST Magazine 第二期 尋思:三代中港・情・意・結 - 糾結} & 2019-08-01 & \href{https://youtube.com/watch?v=82HZNC1kimk}{\texttt{ 82HZNC1kimk}} \\
    & \hyperref[sec:v55_9RXOWv0]{CGST Magazine 第二期 尋思:三代中港.情.意.結 - 想像} & 2019-08-08 & \href{https://youtube.com/watch?v=v55-9RXOWv0}{\texttt{ v55-9RXOWv0}} \\
    & \hyperref[sec:8IQTX4IveJg]{CGST Magazine 第二期 尋思:從流離到安身 黃嘉樑} & 2017-06-10 & \href{https://youtube.com/watch?v=8IQTX4IveJg}{\texttt{ 8IQTX4IveJg}} \\
    & \hyperref[sec:cQDr9IMmvaA]{CGST Magazine 第八期 SHALOM} & 2019-08-01 & \href{https://youtube.com/watch?v=cQDr9IMmvaA}{\texttt{ cQDr9IMmvaA}} \\
    & \hyperref[sec:_ISrkAG_KkM]{CGST Magazine 第四期 1% 99%} & 2019-07-24 & \href{https://youtube.com/watch?v=_ISrkAG_KkM}{\texttt{ \_ISrkAG\_KkM}} \\
    & \hyperref[sec:NKv_tFpmlEM]{CGST magazine 第六期 絕處• 逢生} & 2019-07-29 & \href{https://youtube.com/watch?v=NKv-tFpmlEM}{\texttt{ NKv-tFpmlEM}} \\
    & \hyperref[sec:GGZt_muKlhs]{Public Lecture: A Culture Of Life In The Dangers Of Our Time (Q&A)} & 2018-06-27 & \href{https://youtube.com/watch?v=GGZt_muKlhs}{\texttt{ GGZt\_muKlhs}} \\
    & \hyperref[sec:B1NjObnaFkg]{Public Lecture: A Culture Of Life In The Dangers Of Our Time (Response: Dr Hong Liang)} & 2023-05-31 & \href{https://youtube.com/watch?v=B1NjObnaFkg}{\texttt{ B1NjObnaFkg}} \\
    & \hyperref[sec:MKd__CpGhKU]{Public Lecture: A Culture Of Life In The Dangers Of Our Time (Speaker: Prof. Jürgen Moltmann)} & 2018-06-28 & \href{https://youtube.com/watch?v=MKd-_CpGhKU}{\texttt{ MKd-\_CpGhKU}} \\
    & \hyperref[sec:Ps_hGAmvVcA]{The Crisis of American (White) Evangelicalism (Day 1)} & 2021-01-21 & \href{https://youtube.com/watch?v=Ps_hGAmvVcA}{\texttt{ Ps\_hGAmvVcA}} \\
    & \hyperref[sec:F73AP9vE2KM]{The Crisis of American (White) Evangelicalism (Day 2)} & 2021-01-21 & \href{https://youtube.com/watch?v=F73AP9vE2KM}{\texttt{ F73AP9vE2KM}} \\
    & \hyperref[sec:A4_dD_EAMHg]{The Crisis of American (White) Evangelicalism (Day 3)} & 2021-01-21 & \href{https://youtube.com/watch?v=A4-dD-EAMHg}{\texttt{ A4-dD-EAMHg}} \\
    & \hyperref[sec:NlL3iKQIHto]{The Crisis of American (White) Evangelicalism (Day 4)} & 2021-01-21 & \href{https://youtube.com/watch?v=NlL3iKQIHto}{\texttt{ NlL3iKQIHto}} \\
    & \hyperref[sec:0eVKvD_Q9zg]{The Old Testament and Christian Ethics: How should we live? (3) — Dr Edwin Mung | Dr Chun-Luen Wu} & 2023-03-09 & \href{https://youtube.com/watch?v=0eVKvD-Q9zg}{\texttt{ 0eVKvD-Q9zg}} \\
    & \hyperref[sec:5bv9U4Tdhps]{The Old Testament and Christian Ethics: How should we live? (3) — Q & A Session} & 2023-02-16 & \href{https://youtube.com/watch?v=5bv9U4Tdhps}{\texttt{ 5bv9U4Tdhps}} \\
    & \hyperref[sec:cUiQvS6fE24]{The Old Testament and Christian Ethics: How should we live? (3) — Speaker: Dr Christopher Wright} & 2023-02-16 & \href{https://youtube.com/watch?v=cUiQvS6fE24}{\texttt{ cUiQvS6fE24}} \\
    & \hyperref[sec:XZUhcyDQxTc]{The Old Testament and Christian Hope: Where will it all end? (4) — Dr Stephen Lee} & 2023-03-08 & \href{https://youtube.com/watch?v=XZUhcyDQxTc}{\texttt{ XZUhcyDQxTc}} \\
    & \hyperref[sec:c9xzZQkJF0c]{The Old Testament and Christian Hope: Where will it all end? (4) — Q & A Session} & 2023-03-09 & \href{https://youtube.com/watch?v=c9xzZQkJF0c}{\texttt{ c9xzZQkJF0c}} \\
    & \hyperref[sec:0xAv1_RZmUM]{The Old Testament and Christian Hope: Where will it all end? (4) — Speaker: Dr Christopher Wright} & 2023-03-08 & \href{https://youtube.com/watch?v=0xAv1-RZmUM}{\texttt{ 0xAv1-RZmUM}} \\
    & \hyperref[sec:LWCkMGG0tpo]{The Old Testament and Christian Identity: Who are we? (1) — Dr Leo Li | Dr Lilian Li} & 2023-03-08 & \href{https://youtube.com/watch?v=LWCkMGG0tpo}{\texttt{ LWCkMGG0tpo}} \\
    & \hyperref[sec:mzg9S8Q9BQ8]{The Old Testament and Christian Identity: Who are we? (1) — Q & A Session} & 2023-02-15 & \href{https://youtube.com/watch?v=mzg9S8Q9BQ8}{\texttt{ mzg9S8Q9BQ8}} \\
    & \hyperref[sec:_96yWYR16Vo]{The Old Testament and Christian Identity: Who are we? (1) — Speaker: Dr Christopher Wright} & 2023-03-08 & \href{https://youtube.com/watch?v=_96yWYR16Vo}{\texttt{ \_96yWYR16Vo}} \\
    & \hyperref[sec:XyScbip7koI]{The Old Testament and Christian Mission: What are we here for? (2) — Q & A session} & 2023-03-11 & \href{https://youtube.com/watch?v=XyScbip7koI}{\texttt{ XyScbip7koI}} \\
    & \hyperref[sec:BCZSaeNGuKE]{The Old Testament and Christian Mission: What are we here for? (2) — Speaker: Dr Christopher Wright} & 2023-03-09 & \href{https://youtube.com/watch?v=BCZSaeNGuKE}{\texttt{ BCZSaeNGuKE}} \\
    & \hyperref[sec:iWfhwhP8KkA]{The Old Testament and Christian Mission: What are we here for?(2) — Dr Lawrence  Ko | Dr Johnson Yip} & 2023-03-09 & \href{https://youtube.com/watch?v=iWfhwhP8KkA}{\texttt{ iWfhwhP8KkA}} \\
    & \hyperref[sec:QNbcTFot66g]{Towards a Shared Land Theology: Palestinian Christian Reading of the Land Promises} & 2016-04-27 & \href{https://youtube.com/watch?v=QNbcTFot66g}{\texttt{ QNbcTFot66g}} \\
    & \hyperref[sec:7cxD3Fsxces]{Towards a Shared Land Theology: Palestinian Christian Reading of the Land Promises (Q&A)} & 2023-11-25 & \href{https://youtube.com/watch?v=7cxD3Fsxces}{\texttt{ 7cxD3Fsxces}} \\
    & \hyperref[sec:4mzmfSAacaU]{Unpacking the Credo of Christ through the Temptation of Jesus (盧允晞老師)} & 2022-12-17 & \href{https://youtube.com/watch?v=4mzmfSAacaU}{\texttt{ 4mzmfSAacaU}} \\
    & \hyperref[sec:2WDL8N3ZBFk]{「受苦的耶穌——受苦教會的使命實踐」座談會} & 2023-09-30 & \href{https://youtube.com/watch?v=2WDL8N3ZBFk}{\texttt{ 2WDL8N3ZBFk}} \\
    & \hyperref[sec:_SfD9ovSJ1E]{「經濟嚴冬下 重塑教會使命」公開講座} & 2020-12-17 & \href{https://youtube.com/watch?v=-SfD9ovSJ1E}{\texttt{ -SfD9ovSJ1E}} \\
    & \hyperref[sec:hfF50g4VbXo]{「逆」後餘生心理教育輔導短片系列 - 表達藝術治療(二)} & 2020-05-28 & \href{https://youtube.com/watch?v=hfF50g4VbXo}{\texttt{ hfF50g4VbXo}} \\
    & \hyperref[sec:oKCp7lVV9g0]{「關愛受造世界與福音」香港會議 「一同歎息 迎向更新」:答問環節} & 2021-07-12 & \href{https://youtube.com/watch?v=oKCp7lVV9g0}{\texttt{ oKCp7lVV9g0}} \\
    & \hyperref[sec:6_vdz8RQKsg]{「關愛受造世界與福音」香港會議 「一同歎息 迎向更新」:約拿單 ‧ 穆爾博士} & 2021-05-31 & \href{https://youtube.com/watch?v=6-vdz8RQKsg}{\texttt{ 6-vdz8RQKsg}} \\
    & \hyperref[sec:a0nrOscVmnU]{「關愛受造世界與福音」香港會議 「一同歎息 迎向更新」:關浩然牧師 (回應)} & 2021-09-13 & \href{https://youtube.com/watch?v=a0nrOscVmnU}{\texttt{ a0nrOscVmnU}} \\
    & \hyperref[sec:qPYjE7lJ5fw]{「關愛受造世界與福音」香港會議 「一同歎息 迎向更新」:黃國維博士 (回應)} & 2023-09-10 & \href{https://youtube.com/watch?v=qPYjE7lJ5fw}{\texttt{ qPYjE7lJ5fw}} \\
    & \hyperref[sec:wCdSimQhtMM]{「關愛受造世界與福音」香港會議 「門徒生命與關愛受造世界」:胡志偉牧師 (回應)} & 2021-09-09 & \href{https://youtube.com/watch?v=wCdSimQhtMM}{\texttt{ wCdSimQhtMM}} \\
    & \hyperref[sec:9AdkPpUw1wU]{「關愛受造世界與福音」香港會議 「門徒生命與關愛受造世界」:鄺偉文博士 (回應)} & 2023-07-31 & \href{https://youtube.com/watch?v=9AdkPpUw1wU}{\texttt{ 9AdkPpUw1wU}} \\
    & \hyperref[sec:EDiGHY_cDLQ]{「關愛受造世界與福音」香港會議「門徒生命與關愛受造世界」: 答問環節} & 2023-08-04 & \href{https://youtube.com/watch?v=EDiGHY_cDLQ}{\texttt{ EDiGHY\_cDLQ}} \\
    & \hyperref[sec:dJ7VfX3ypUE]{「關愛受造世界與福音」香港會議「門徒生命與關愛受造世界」: 約拿單 ‧ 穆爾博士} & 2021-08-02 & \href{https://youtube.com/watch?v=dJ7VfX3ypUE}{\texttt{ dJ7VfX3ypUE}} \\
    & \hyperref[sec:U2MibYFulYg]{「面對同性戀 教會何去何從」:答問時間} & 2017-10-12 & \href{https://youtube.com/watch?v=U2MibYFulYg}{\texttt{ U2MibYFulYg}} \\
    & \hyperref[sec:O8VAiCx1rx4]{【公開講座|擁抱殘疾的教會 — 群體中經歷醫治和牧養】} & 2023-04-14 & \href{https://youtube.com/watch?v=O8VAiCx1rx4}{\texttt{ O8VAiCx1rx4}} \\
    & \hyperref[sec:kR2ujHQel1E]{【工作坊|擁抱殘疾的教會 — 群體中經歷醫治和牧養】} & 2023-04-14 & \href{https://youtube.com/watch?v=kR2ujHQel1E}{\texttt{ kR2ujHQel1E}} \\
啟示錄   & \hyperref[sec:_p1k_J5ZNow]{【疫有嘢學 │ 延SUN在線 】啟示錄三個七看天災|辛惠蘭博士} & 2020-05-31 & \href{https://youtube.com/watch?v=-p1k-J5ZNow}{\texttt{ -p1k-J5ZNow}} \\
    & \hyperref[sec:Hs1Y_XrlxkM]{【疫有嘢學 │ 延SUN在線】以西結看瘟疫|結5章|葉希賢博士} & 2020-04-27 & \href{https://youtube.com/watch?v=Hs1Y_XrlxkM}{\texttt{ Hs1Y\_XrlxkM}} \\
    & \hyperref[sec:xiMH3MdBCkY]{【疫有嘢學 │ 延SUN在線】從歷史看政教分離的多種含義|雷競業博士} & 2020-06-07 & \href{https://youtube.com/watch?v=xiMH3MdBCkY}{\texttt{ xiMH3MdBCkY}} \\
    & \hyperref[sec:2dfqNGcDljE]{【疫有嘢學 │ 延SUN在線】愛在差異蔓延時-婚姻成長講座|林添德先生} & 2020-06-20 & \href{https://youtube.com/watch?v=2dfqNGcDljE}{\texttt{ 2dfqNGcDljE}} \\
    & \hyperref[sec:OmTXVUsNi_8]{【疫有嘢學 │ 延SUN在線】當急難敲門時|王下4章8-37節|張智聰博士} & 2020-04-18 & \href{https://youtube.com/watch?v=OmTXVUsNi-8}{\texttt{ OmTXVUsNi-8}} \\
    & \hyperref[sec:V3Cpa_Vt9fE]{【疫有嘢學 │ 延SUN在線】疫境中的抉擇|代上21章|李思敬博士} & 2020-04-06 & \href{https://youtube.com/watch?v=V3Cpa-Vt9fE}{\texttt{ V3Cpa-Vt9fE}} \\
    & \hyperref[sec:O9blI5PB1Ss]{【疫有嘢學 │ 延SUN在線】疫境中遇見復活主|約20章11-18節|余振陵博士} & 2020-04-12 & \href{https://youtube.com/watch?v=O9blI5PB1Ss}{\texttt{ O9blI5PB1Ss}} \\
    & \hyperref[sec:AHC7wVTdk1o]{【疫有嘢學 │ 延SUN在線】窘迫與釋放:殘疾處境之啟迪|陳關韻韶女士} & 2020-06-28 & \href{https://youtube.com/watch?v=AHC7wVTdk1o}{\texttt{ AHC7wVTdk1o}} \\
    & \hyperref[sec:KKN4CX_CTEM]{【疫有嘢學 │ 延SUN在線】談判買地葬妻:研讀創廿三章|朱光華博士} & 2020-05-21 & \href{https://youtube.com/watch?v=KKN4CX-CTEM}{\texttt{ KKN4CX-CTEM}} \\
    & \hyperref[sec:G6y4aNW1WhY]{【疫有嘢學 │ 延SUN在線】講故事、做神學|李適清博士} & 2020-05-17 & \href{https://youtube.com/watch?v=G6y4aNW1WhY}{\texttt{ G6y4aNW1WhY}} \\
    & \hyperref[sec:gGn96F2CeGs]{【疫有嘢學 │ 延SUN在線】貼地同行,破舊立新 - 近代中國教會的社會服侍|莫陳詠恩博士} & 2020-05-03 & \href{https://youtube.com/watch?v=gGn96F2CeGs}{\texttt{ gGn96F2CeGs}} \\
    & \hyperref[sec:qtqZXfdLO9c]{【疫有嘢學 │ 延SUN在線】雙翼齊飛:信仰和正向心理學如何助我們跨越逆境和疫情|區祥江博士} & 2020-05-10 & \href{https://youtube.com/watch?v=qtqZXfdLO9c}{\texttt{ qtqZXfdLO9c}} \\
    & \hyperref[sec:QdUuSNEVBIc]{【疫有嘢學 │ 延SUN在線】靈旅四季的調適與群體中的同行|潘怡蓉博士} & 2020-06-14 & \href{https://youtube.com/watch?v=QdUuSNEVBIc}{\texttt{ QdUuSNEVBIc}} \\
    & \hyperref[sec:4FekoGe9H60]{一根杏樹枝:職場的看見 (李適清博士)} & 2015-05-19 & \href{https://youtube.com/watch?v=4FekoGe9H60}{\texttt{ 4FekoGe9H60}} \\
    & \hyperref[sec:OH2MWqshqHE]{不能逃避的戰爭:人性大戰 - 善念與邪情之爭戰} & 2019-07-29 & \href{https://youtube.com/watch?v=OH2MWqshqHE}{\texttt{ OH2MWqshqHE}} \\
    & \hyperref[sec:eN8W8le4E48]{不能逃避的戰爭:問答環節} & 2019-07-29 & \href{https://youtube.com/watch?v=eN8W8le4E48}{\texttt{ eN8W8le4E48}} \\
    & \hyperref[sec:uudfjjhd4hA]{不能逃避的戰爭:終末大戰 - 光明與黑暗之爭戰} & 2019-07-29 & \href{https://youtube.com/watch?v=uudfjjhd4hA}{\texttt{ uudfjjhd4hA}} \\
    & \hyperref[sec:81BflBsZN_g]{中神40周年院慶培靈會:張智聰博士} & 2023-11-19 & \href{https://youtube.com/watch?v=81BflBsZN_g}{\texttt{ 81BflBsZN\_g}} \\
    & \hyperref[sec:UWFUcTme2GY]{中神40周年院慶培靈會:李耀坤博士} & 2015-10-14 & \href{https://youtube.com/watch?v=UWFUcTme2GY}{\texttt{ UWFUcTme2GY}} \\
    & \hyperref[sec:S_AT9Lsna28]{中神40周年院慶培靈會:沈祖堯教授} & 2018-10-21 & \href{https://youtube.com/watch?v=S_AT9Lsna28}{\texttt{ S\_AT9Lsna28}} \\
    & \hyperref[sec:g45zbSUTL0c]{中神40周年院慶培靈會:鍾氏兄弟} & 2015-10-14 & \href{https://youtube.com/watch?v=g45zbSUTL0c}{\texttt{ g45zbSUTL0c}} \\
    & \hyperref[sec:DregqwIedRk]{中神40周年院慶培靈會:陳關韻韶老師} & 2015-10-14 & \href{https://youtube.com/watch?v=DregqwIedRk}{\texttt{ DregqwIedRk}} \\
    & \hyperref[sec:yNpGxFfqF_k]{中神40周年院慶感恩暨開學崇拜(證道:榮休院長余達心牧師)} & 2015-09-16 & \href{https://youtube.com/watch?v=yNpGxFfqF_k}{\texttt{ yNpGxFfqF\_k}} \\
    & \hyperref[sec:RU4oBv0Wasg]{中神45周年北美培靈講座(西岸)} & 2021-01-20 & \href{https://youtube.com/watch?v=RU4oBv0Wasg}{\texttt{ RU4oBv0Wasg}} \\
    & \hyperref[sec:_3eTxXlzPX0]{中神45周年北美培靈講座(東岸)} & 2021-01-21 & \href{https://youtube.com/watch?v=-3eTxXlzPX0}{\texttt{ -3eTxXlzPX0}} \\
    & \hyperref[sec:zyLpufeGBYs]{中神北美培靈講座} & 2016-05-20 & \href{https://youtube.com/watch?v=zyLpufeGBYs}{\texttt{ zyLpufeGBYs}} \\
    & \hyperref[sec:2TEwldoXzT8]{中神北美培靈講座 2016 「合神心意的事奉」-- 李思敬院長} & 2023-12-02 & \href{https://youtube.com/watch?v=2TEwldoXzT8}{\texttt{ 2TEwldoXzT8}} \\
    & \hyperref[sec:JHQ2Beaggow]{中神北美培靈講座 2017 「你們說,我是誰?- 作門徒的召喚」 - 余達心牧師} & 2017-06-22 & \href{https://youtube.com/watch?v=JHQ2Beaggow}{\texttt{ JHQ2Beaggow}} \\
    & \hyperref[sec:ml6Ww0ZQVk8]{中神北美培靈講座 2018 「我和我家必定事奉上主」 - 黃國維博士} & 2018-10-31 & \href{https://youtube.com/watch?v=ml6Ww0ZQVk8}{\texttt{ ml6Ww0ZQVk8}} \\
    & \hyperref[sec:XGZrzl_HY54]{中神北美西岸培靈講座 2019} & 2019-06-28 & \href{https://youtube.com/watch?v=XGZrzl-HY54}{\texttt{ XGZrzl-HY54}} \\
    & \hyperref[sec:jGn4gOBS1HA]{中神墨爾本培靈會:用我一生} & 2018-11-12 & \href{https://youtube.com/watch?v=jGn4gOBS1HA}{\texttt{ jGn4gOBS1HA}} \\
    & \hyperref[sec:EtbCz7LXbWY]{中神澳紐培靈講座 2019} & 2019-07-25 & \href{https://youtube.com/watch?v=EtbCz7LXbWY}{\texttt{ EtbCz7LXbWY}} \\
\end{xltabular}
}
\newpage



\section{}
\label{sec:AoYPWrVX2A4}
\textbf{18⧸19年度研究院開學崇拜:文明背後的血與記號 -李耀坤博士}
\newline
\newline
連結: \href{https://youtube.com/watch?v=AoYPWrVX2A4}{\texttt{ https://youtube.com/watch?v=AoYPWrVX2A4}} ~~~~ 語音日期: 2018-11-01 
\newline
\newline
\hyperref[sec:code]{\small{< < < PREV SERMON < < <}}
~
\hyperref[sec:index]{\small{[返主目錄]}}
~
\hyperref[sec:Jp_xjonF0nk]{\small{> > > NEXT SERMON > > >}}
\newline
\newline
$^{1}$從1996年開始.
中國神學研究院的開學崇拜.
負責正道.
就是我們教授團老師輪流的責任.
今年輪到李耀坤博士.
當然這幅才是正經的.
李耀坤博士是我們忠臣趙淑榮.
霍佩芳教席副教授.
也是我們信仰及公共價值研究中心的主任.
這些我都可以在網上知道.
有沒有其他的介紹呢.
李耀坤博士是90年代.
我們忠臣的傑出校友之一.
他是93年入學.
聽說入學的時候.
他說他要讀聖經科.
畢業之後.
他在甘巴倫長老會侍奉.
公元2000年.
丹尼爾老師去到愛丁堡大學.
開始他的博士研究.
師從Professor David Wright.
是愛丁堡奧古斯丁和嘉爾文的權威學者.
所以他回來2005年.
從忠臣開始任教神學科.
也質問他是余牧師的接班人.
如果你查一下忠臣的網站.
這是我們同工有一次祈禱會考我們的冷知識.
原來忠臣網站最多講座的影片的.
不是院長.
是丹尼爾老師.
所以今晚我們很高興.
終於輪到他了.
剛剛抹了汗.
他要準備上來解釋.
為什麼他改了他的題目.
今晚的題目不是印出來的.
我們把時間公敬交給丹尼爾老師.
No.
多謝院長的介紹.

$^{41}$這樣介紹想不抹汗都很難.
我是改了題目.
最初的題目是長康裡面的血的呼號.
一個這麼驚嚇的講題.
我怕嚇到新同學就不太好.
因為今天才剛開學.
雖然忠臣的確有很多Killers.
大家還是要小心一點.
但我選擇創世紀這段經文.
最主要是想和大家在今晚.
透過聖經思想一下.
在創世紀給我們看的.
關於聖經如何看文明的視角.
所以最後我想.
不如轉回一個比較長一點.
比較clumsy一點.
不過比較回應得準確一點.
我想講的東西.
是文明背後的血和記號.
我們在聆聽上主的話之前.
我們一起再做一個禱告.
創造天地的主要是.
我們在座每一個的天父.
你的說話每一句都是帶著能力.
也都帶著你的恩典.
所以我們在這裡等候.
你向我們每一個發出.
你自己寶貴的說話.
我們懇求你.
因著你愛子耶穌基督的緣故.
口口的囂冠你的聖靈.
運行在我們當中.
賜給我們有悟性.
叫我們能夠聽得明白.
也都聽得入心.
並且我們聽得到之後.
我們思考在我們的生命.
在我們的社會.
在這個世界當中.
如何去回應你.

$^{81}$和見證你的真實.
天父願我口中的說話.
我們眾弟兄姊妹.
在你面前的心思意念.
都蒙你自己親自的監察.
洗淨和閱立.
我們這樣的禱告.
交託奉耶穌基督的命求.
阿們.
如果你看《創世紀》第四章的話.
你必須要看第三章.
第三章是說到什麼呢.
就是亞當和夏娃.
因為犯罪.
然後被上帝逐出伊甸園.
《創世紀》第四章是記述.
人類被逐出伊甸園之後.
接著下來族群的繁衍發展的情況.
如果我們從第四章的第一節去看.
第一節就提及到亞當和他的妻子夏娃同房.
然後夏娃懷孕.
然後生了該孕.
然後再接著就是生了父親.
如果我們從第一節和第二節讀完之後.
我們馬上快速跳到第十七節的話.
你就會發覺到整個經文是一氣呵成的.
你一邊讀下去的感覺是很一氣呵成的.
並且你會感覺到裡面給你的印象是.
有一種很生生不息的感覺.
因為你看到兩段的句子.
都是有一些字眼是重複的.
就是同房,夫婦同房.
然後懷孕.
然後生一個兒子.
到了第十七節也是一樣.
就是生命的循環.
不僅如此.
如果我們再看下去.
到了第十八至二十二節的話.
你就會發覺到.

$^{121}$在第十八至二十二節.
也繼續交代在該人以後的.
就是由該人開始計算.
總共有七代的人.
就是整個生命和家族一直延續下去.
如果我們再留心一些.
再看第十八至二十二節裡面的記載的話.
我們就會發現.
聖經這裡不僅說到.
生命是在繼續延續.
它更加說到.
人類的文明似乎是蒸蒸日上.
該人是聖經裡面.
第一個是建城的人.
他是一個城市建築師.
他開始有建城的行動.
然後我們再數下去.
你會看到.
在該人的後裔裡面.
在拉麥的那一系裡面.
他三個兒子.
都是非常傑出的人才.
一個是開創了畜牧技術.
然後一個是發明了樂器.
再有一個是開創了野金技術.
你可以說.
該人一系一門三傑.
用我們今天的說法.
那三個都是諾貝爾獎得主.
完全是開一個新的領域.
一顆顆巨星.
是一個偉大文明的開拓家族.
如果我們看一看.
在世界各個古文明的話.
你會發現對於城邦的建立人.
或者文明的開拓者.
通常是非常景仰的.
通常會看他們是英雄.
是聖王.
甚至很多時候.

$^{161}$會將他們神明化.
譬如這幅圖所說到的.
就是羅密歐和里姆斯兩兄弟.
這兩兄弟是雙胞胎.
是雙胞胎的.
他們在羅馬的傳奇裡面.
說他們是建立了羅馬城.
羅馬帝國的開拓者就是這兩兄弟.
這兩兄弟他們被描述為.
Mars.
一個戰神的後裔.
所以他們是半神半人.
Demigod.
不是普通人.
半人半神.
並且在這個傳記裡面.
他們是英雄.
我們再看看另一個古文明.
希臘.
希臘裡面有一位.
你可以說是很重要的神明.
就是普羅米修斯.
普羅米修斯是泰坦族的神明.
在希臘的神話裡面.
是普羅米修斯將火傳遞給人.
就是他將火給了人.
然後人因著有火之後.
人類的文明就得以開展.
所以對於希臘人來說.
普羅米修斯是他們很崇敬的.
或者是很感謝的一個神明.
如果我們再看看中國也不例外.
中國這幅圖是我在網上找到的.
古籍裡面畫的就是夏雨.
我們通常比較熟悉的就是大雨.
大家記得大雨嗎?.
就是大雨自水的那一位.
大雨自水在中國民間的信仰裡面.
這個要問一下宋君博士就更加清楚了.
已經是被神明化了.

$^{201}$有一些傳記或者神話.
已經是說大雨會變身等等.
而在儒家的道統裡面.
大雨也被尊崇為其中一個典範的君王.
如果我們看小康的書裡面.
他就一直數雨湯文武成王周公.
這樣排列.
而雨是排在這裡的首位.
所以你會發覺到大雨是一個什麼人呢?.
其實他是一個水利工程師.
他處理了洪水的問題.
一個對於他們的族群.
對於那個社會很大貢獻的人.
被升格為一個聖君.
被升格為一個神明.
但是如果我們看回剛才所讀的那一段聖經.
你就會發覺到.
創世紀偏偏在一個光輝亮麗的文明史裡面.
在第四章三節至十六節.
加插了人類世界第一宗的血案.
這是一個令人無法釋懷的家族悲劇.
家族的惡夢.
在這裡.
你可以說創世紀是古代的文明中.
一個很奇特的例子.
舊約學者在這裡提醒我們.
他說四章三至十六節這裡.
有很多地方呼應第三章.
就是亞當和夏娃犯罪的記載.
例如上帝在這兩處地方.
都有問到犯罪的人在哪裡的問題.
在哪裡.
你在哪裡.
在哪裡的問題.
兩處都有出現.
然後也有質問他們.
你做了什麼.
你作的是什麼.
在這兩處裡面.
也有提到地是受咒詛.

$^{241}$而懲罰也是趕盡離開他的故土故地.
所以從這裡看來.
我們知道創世紀其實是想告訴我們.
該隱和阿八的故事.
可以說是人類墮落的續集.
第一集就是亞當夏娃的犯罪.
然後續集就是該隱和阿八的犯罪.
主要是該隱的犯罪.
在這個故事裡面.
就正如在這段經文的記述裡面.
提到罪就是一個伏在文明的門前.
一直纏繞著人類社會的一個噩夢.
而每一代的社會都要打醒精神.
如何去制服罪的那種前類.
人類社會面對著罪的威脅.
這是我今晚想和大家.
首先思考的一個部分.
罪究竟是怎樣.
究竟造成什麼對人類社會的威脅.
然後從這裡.
我會和大家一起思想.
上帝在這個處境.
在這個困局裡面.
上帝又給了我們什麼保護.
所以我首先會和大家一起思想.
文明要面對的這個噩夢.
我嘗試由三個向度去思考.
這段經文讓我們看到的.
罪對於我們這個世界.
對文明,對社會的威脅.
第一重的威脅.
我會說是忌恨的威脅.
在第二節.
我用的版本主要是基於和合本成經.
但是有某些地方我覺得需要改動.
我就作出一些輕微的改動.
所以我盡量跟隨和合本的版本.
在這裡第二節提到.
阿伯是牧羊的.
然後該人是種地的.

$^{281}$這個記載雖然是簡短.
但是提出了一件很重要的事情.
就是人類社會開始分工.
這是文明發展的重大里程碑.
但是這個也同時隱含著.
人類社會將要面對一些尖銳的挑戰.
分工意味著生活模式開始有分殊.
意味著勞動的成果會不同.
也意味著大家的際遇可以很不一樣.
差異可以造成張力.
而這段經文最令人意想不到的就是.
張力的爆發點竟然在最淑齡的活動中發生.
獻祭.
所以我們當中的弟兄姊妹同學.
你受訓就是去到教會.
去到上帝的祭壇面前去服侍.
這是高危行業.
獻祭壓力爆煲的點竟然就在獻祭中.
兩兄弟各自將工作的成果獻呈給上帝.
但是換來兩個截然不同的結果.
經文很簡單.
在第四節尾然後到第五節那裡交代.
就是耶和華看中了阿伯和他的共物.
只是看不中該人和他的共物.
上帝看中了阿伯和他的共物.
但是沒有看中該人所獻.
值得注意的地方是.
經文在這裡只是很平衡地報道兩邊的情況.
如果沒有任何前設預設.
或者暫時放下你主要學老師講的版本.
然後看經文的話.
你會發覺經文其實是很冷靜.
很平衡地報道兩邊的情況.
完全沒有作出任何價值判斷.
換言之.
經文在這裡沒有用一個成敗功過的角度.
去理解上帝看中或者看不中.
明白我的意思嗎?.
經文完全沒有這樣的味道.
在往後的解經家對於聖經這種沉默.

$^{321}$往往是窩窩攣,按耐不住.
所以我們看到很多的版本.
可能你主要學也聽過很多不同的版本.
就是千方百計試圖要去幫上帝解話.
為什麼上帝會對一個看中一個看不中呢?.
其中一個很流行的版本就是.
阿伯獻上的是最好的羊.
因為那裡講到他將頭身的羊和資油獻上.
所以他是最好的.
倒過來那個人當然是求求其其敷衍了事.
所以就失敗了.
上帝看中一個,沒有看中第二個.
但其實如果我們這樣想的話.
會不會正正跌入比較這個陷阱裡呢?.
我們會不會走不掉「成大得失」的心癮呢?.
而這個豈不是正正就是該癮要面對的心魔.
在第五節,聖經很生動地描述.
該人在這種糾結的情況下.
他受到那種煎熬.
他有些時候想著想著,心有不甘.
很激氣,憤怒.
有些時候又轉為垂頭喪氣.
很鬱悶,很鬱結.
所以我懷疑那個人是世界上第一個bipolar的人.
我們最理智的時候,最冷靜的時候.
我們大概都會明白到.
「你得不等於我失」.
我們都會明白到「你成功不代表我失敗」.
但是人性裡的理智.
在群體裡似乎是格外脆弱和無力.
在上個世紀,一位對這件事思考得最深刻的神學家.
要數的就是尼泊爾.
他的《Moral Man and Immoral Society》.
一個個體可以很理智.
但是在一個群體,在一個社會裡.
似乎突然之間有些很不理智的事情.
就可以在裡面發生.
「你得我失」.
當我們不斷沉溺在這種比較.
「你成功我失敗」的時候.

$^{361}$是可以伸出致命的忌恨出來.
你的成功造成我的失敗.
我今天的屈辱,我今天的失意.
完全是你一手造成.
那個人沒有留心去聽上帝的勸告.
制服他的心魔.
結果犯下彌天大罪.
這種怨毒忌恨的悲劇.
不斷在人類的歷史裡面重演.
不是停在該隱亞伯這件事當中.
上個世紀爆發了兩場.
影響人類歷史深遠的戰爭.
在第一次世界大戰的時候.
德國的軍隊非常奮勇地作戰.
當時他們要面對兩條戰線.
一條是西戰線一條是東戰線.
德國軍隊特別在西戰線的表現.
是十分之出色.
但後來因為各種戰略錯誤的緣故.
戰況急轉直下.
最後以戰敗告終.
戰敗之後.
德國要面對一個非常苛刻的凡爾賽條約.
令德國人的生活在水深火熱.
自此之後.
軍隊開始流行一種論調.
那種論調是這樣的.
認為德國軍隊是很優秀的軍隊.
所以根本不是德國的軍隊打敗仗.
為什麼會戰敗呢.
因為有鬼在後面抽我們後腳.
那個鬼是誰呢.
是猶太人在我們背後刺我們一刀.
這種論調.
深信不疑的人都不少.
其中一個名字叫阿道夫希特勒.
往後發生什麼悲劇.
這個我們都知道.
阿伯的血今日仍然不斷發出他的呼號.
第二重威脅我們的.

$^{401}$我稱之為是卸責.
irresponsibility.
當阿當犯罪之後.
上帝找回他.
呼喚他.
然後上帝問他.
你在哪裡.
同樣地在這段經文裡.
當該人犯罪之後.
上帝也找他.
也呼喚他.
向他發出一個非常類似的問題.
不過這次這個問題.
由你在哪裡.
轉變成為你的兄弟在哪裡.
你在哪裡這個問題.
迫使人回應上帝對我的關注.
要我向上帝交代.
我現在生命的境況.
你的兄弟在哪裡.
這個問題迫使人回應上帝.
對我的人類兄弟的關注.
要我向上帝交代.
我的人類兄弟.
他現在的生命境況是怎樣.
在這裡.
伴隨著人和上帝中向的關係.
不能分割的.
是人和他人類的兄弟姐妹.
和他的鄰舍橫向的關係.
我的鄰舍在我面前出現.
這件事本身已經是向我.
發出一個生命的索求.
要求我盡上對這個兄弟或者姐妹.
要盡的那種情誼和責任.
就算特別是《創世紀》這裡所說.
就算這個鄰舍的兄弟.
他的聲音非常微小.
甚至已經被人滅聲.
上帝仍然代他去發出呼喚.

$^{441}$該人怎樣回應這個呼喚.
回應這個生命的結吻.
他的回答是.
「我不知道」.
劈頭「我不知道」.
不過不止一個埋沒良心的不知道.
他還補上一句.
現在在網上我學會這個字.
叫霸氣.
一句很霸氣的反文句.
我這裡很喜歡新日本的翻譯.
所以我參考了新日本的翻譯.
「難道我是我兄弟的看守者?」.
這句話.
如果我們聽真一點.
你會知道這是一句有骨的說話.
這句話裡面特別提到看守者.
看守其實是一個目者.
一個目人的工作.
即是說.
阿伯才是目人.
他才是看守者.
上帝你有沒有搞錯.
你找我問我.
這件事.
我是看守過看守者的人嗎?.
你明白這句話的霸氣在哪裡嗎?.
更加厲害的言外之音其實在哪裡呢?.
就是上帝.
豈不是你才是他的目者嗎?.
耶和華是我的目者.
你才是他的目者.
他在哪裡.
他現在怎麼樣.
你問我.
他在哪裡.
關他的事.
關上帝你的事.
但與我無關.
你問我.

$^{481}$對於他的人類的兄弟.
對他人類的鄰舍的責任.
該人推得一乾二淨.
這一類的回答.
大概我們都不會太陌生吧.
看新聞.
最近都聽到一個這樣的例子.
你的太太生小孩.
關我這個老闆的什麼事呢?.
以前我們的太太生小孩.
都沒有產假的了.
又如何?.
與該人的回應不相伯仲.
不過老實說.
這樣的回應.
在今時今日的標準來看.
其實是很低層次的.
今日我們的文明.
已經可以進步到另一個境界.
我們推裝卸責.
已經可以用一種.
更加有品味的方式來發聲.
在今日的世界裡面.
已經有很多結構性的暴力.
是可以令到人殺人.
而沒有任何一個人.
是需要直接操刀和染血.
我們可以理直氣壯地說.
我不知道.
而在當中是完全沒有說過.
一句的謊言.
譬如我們可以心安理得地.
享用很多價廉物美的電子產品.
衣服,球鞋.
但是完全不知道.
在血汗工廠裡面.
那些不人道的工作環境.
是如何殘害很多年輕工人的生命.
甚至令到他們當中有些人.
情緒崩潰到自尋短見而自殺.

$^{521}$阿伯的血.
今日仍然在我們的世界.
在我們的文明.
在我們的社會當中.
不斷發出呼號.
第三重的威脅.
我想了很久.
都發覺用這個字眼是最合適.
雖然比較抽象一點.
就是異化.
當上帝向該人宣告他的審判之後.
該人馬上的回應是.
「我的刑罰太重,過於我所能當」.
比起他的父母.
我們看到該人進步了很多.
你記得亞當和夏娃.
被上帝趕走之後.
頭一剎那離開.
連一句話都沒有回答過.
不過該人就不同.
該人聽完刑罰之後.
回了這句很重要的話.
該人似乎更加清晰地意識到.
將要淋到他身上的懲罰.
是多麼嚴重的一個情況.
在經文中特別提到流離,飄蕩.
這裡當然是指他將要面對的是.
連根拔起離開他本來熟悉的社群.
和生活的環境.
不過更加大的問題是什麼呢?.
是自此之後天下雖大.
但是再沒有一個地方.
該人可以安頓下來.
「阿伯的血」的呼號.
該人可以乍聽不到.
但是他沒有辦法阻止.
其他人聽到「阿伯的血」的呼號.
聽到以後因為這樣而激起.
要為「阿伯」報血仇的情況.
所以他接著說.

$^{561}$「凡遇見我的必殺我」.
你可以看到他多迫切.
也未必一定會殺.
不過在他心裡.
那隻鬼是怎樣說都說不走.
該人啟動了一個.
他沒有辦法自己去煞停的連鎖反應.
殘害生命的尊嚴.
激發起那種憤怒.
反過來威脅著加害者.
作為人的尊嚴.
「阿伯的血」令該人永無寧日.
任何的淪陷都可以是他潛在的威脅.
一個置身在任何人類社群裡.
他都沒有辦法安頓.
沒有辦法「Feel at home」的人.
他正在過的.
究竟還是不是一個人的生活呢?.
該人面對的不是一時一地的流離失所這麼簡單.
他面對的是人性本身非常基本的異化問題.
在現代歷史裡.
最強烈地呈現出這種異化的.
莫過於是各場種族的衝突.
南非聖公會的主教道徒.
曾經在他的著作裡.
描述在種族隔離政策下的南非.
不僅長期受歧視欺壓的黑人.
他們覺得自己生命的尊嚴被傷害.
就算是那些處於優勢當中的白人.
其實按照道徒的觀察.
他們長期活在一種惶惶不可終日.
不安恐懼和自欺的當中.
他們不斷要找理由去證明.
或者去合理化自己所作的一切事情.
道徒看到其實整個社會.
無論是黑人還是白人.
都是在這種處境當中.
他們也見證了不少的情況.
就是昨天明明還是該忍的一個人.
但是轉眼之間可以成為另一個的阿伯.

$^{601}$而且在過程中又生出了很多新的該忍.
種族衝突其實只不過是.
現代社會的一個大特寫.
把問題很清晰地浮現出來.
17世紀的思想家霍布斯.
在他的名著《利維坦》當中.
描述到在現代社會當中.
基本上是一個所有人與所有人的戰爭.
他提出的social contract 社會契約.
基本上是將這種所有人與所有人的戰爭.
變成用一種隱藏的方式繼續呈現出來.
所以人與人的互相傾雜.
在現代社會裡面其實是不斷存在.
只不過它是用一種隱藏的形式存在而已.
阿伯的血今天仍然不斷在我們的社會當中.
在我們的文明當中發出呼號.
如果我們看聖經到這裡停在第十四節的話.
我們大概會對人類社會和當中的文明.
感覺到非常絕望.
用現在的潮語的說法就是灰暴.
罪性的威力是非同小可.
在《創世紀》裡面是完全沒有掩飾這一點.
我們知道人類的社會或者文明本身.
單靠它內在的資源.
根本沒有能力去對抗罪的威脅.
如果由得文明自己去運作自己的邏輯的話.
最後只會踏上自毀的命運.
所以想到這裡的話.
或許我們會有一個這樣的想法.
就是教會帶著凱爾攪在一起就手旁觀.
任憑這個世界自生自滅.
詩歌第一唱「這世界非我家」.
我們救人上我們這艘船是最重要的.
這個世界由得它.
它是不是也會如此黑暗和毀滅.
但是在這裡教會必須要謹記.
《創世紀》還有第十五節.
四章十五節這裡.
我比較喜歡新譯本的翻譯.
因為和合本就將一個字剪掉.

$^{641}$這個字很要命.
就是開頭那裡.
其實耶和華對該人說.
「絕不會這樣」.
當該人很灰飽說.
「這次一定死了」.
「我去到哪裡都有人追殺我」的時候.
耶和華對他說.
「絕不會這樣」.
然後耶和華向該人做了一個很重要的動作.
上帝為該人立了一個記號.
「免得有人遇見他殺他」.
這個當然是指對該人的愛和保護.
但同時之間.
該人的記號也是上帝對人類社會.
整個社會的愛和保護.
這裡有趣的地方是.
上帝特別警告人.
如果殺該人必遭報七倍.
遭報七倍的警告.
你想深一層你會知道.
目的就是要人放棄報復.
很有趣的一個安排.
其實就是要人放棄報復.
雖然人罪孽深重.
但上帝定義不讓人的尊嚴全面瓦解.
不讓人類的社會.
人類的文明進入自毀的壓榮當中.
為了這件事情.
上帝沒有按照人的過犯所應得來懲罰我們.
上帝的記號.
將罪的破壞力加以約束.
令到人類的社群可以延續.
亦給予人有時間和發展的機會.
聖經這裡接著告訴我們.
該人仍然可以有自己的遺址.
仍然可以在他去到新的地方建成.
成家立室.
在這個位置裡.
不能不提的是.

$^{681}$上世紀一位很重要的法國社會學大師.
他寫了一本《Meaning of the City》.
他對於聖經裡.
特別是在上世紀這段.
他有很深的感受.
讀到這一段.
特別是該人去建成這裡.
他認為該人建成.
是表示該人信不過上帝的應許和保護.
所以該人要用自己的方法.
建成這個方法來保護自己.
不過雖然我很欣賞.
以祿他很多的見解.
在這裡我就沒辦法認同他這個釋經.
因為如果我們看上世紀這裡的話.
經文對於該人建成.
並不像之後的巴別塔那裡的描述.
聖經完全沒有對於他建成有任何的批評.
相反接下來的經文.
是交代了該人以降的七代生命的延續.
而我們知道七代.
七這個數字.
在聖經裡是有完整.
甚至乎有神聖的涵義在當中.
所以更加可能的是.
聖經要告訴我們.
上帝的應許和保護.
的確已經成就在該人的家當中.
該人所建的城.
城外面的圍牆.
以至城內的法治.
豈不是對於受罪威脅的人類社會.
一種的保護嗎?.
聖經要告訴我們.
要提醒我們.
我們必須要擺脫.
好像外邦或者異教當中.
對文明的盲目崇拜.
我們要擺脫那種的迷思.
但同時之間.

$^{721}$聖經提醒我們.
切勿魯莽地.
全盤否定文明的價值.
放棄守護文明的責任.
因為文明仍然是帶著.
上帝苦心為該人所立的記號在當中.
來到結語.
阿伯的血的呼號.
打破了古代以至現代社會.
對所謂偉大文明的迷思.
任何的文明.
你可以說創世紀無情地告訴我們.
都是受罪的威脅.
面對各種各樣的破碎和不義.
但是創世紀第四章.
同時提醒我們.
文明仍然帶著阿伯的記號.
在上帝的召管當中.
他仍然有一個俠如其分的位置.
以他有限和破碎.
繼續保存人類的生命.
所以面對文明其實是一個很複雜的事情.
信徒的群體既要有一種深刻對他的批判.
但同時之間亦都不可以放棄.
保護保守文明的責任.
當中非常需要上帝所給予我們的智慧.
給予我們的勇氣和忍耐.
並且每一代的教會.
每一代的信徒群體都不能夠.
你可以說整個尋找如何處理.
是沒有一勞永逸的方案.
是要每一代不斷地摸索.
在我們的時代裡探討.
在這裡我們要知道.
文明無法補滿阿伯血的虧欠.
但它起碼可以給予我們時間.
認識到上帝為這個世界所預備.
真正補滿的出路.
這個出路就是希伯來書裡所記載的.
那裡是這樣說.

$^{761}$「新約的宗保耶穌.
以及所灑的血.
這血所呼號的.
比阿伯的血所呼號的近利」.
教會在任何一個時代裡.
都不斷要向它處身的社群當中.
文明當中.
說清楚給它知道.
它真正盼望的來源.
是在這個近利的血的呼號當中.
我們一起低頭有個禱告.
天父上帝在新的一年當中.
求你激勵我們當中的老師,同學.
讓我們能夠在你裡面立志.
按照你的心意去學習.
裝備好我們自己.
特別在這個時代當中.
求你讓我們有那種.
由你而來的智慧,勇氣和堅持.
以致我們能夠如何辨識.
在文明當中需要批判的地方.
但我們同時也要懷著愛鄰舍的心.
去保守我們當中.
那些很重要的價值和文明.
天父求你在新的學年當中.
將藏在基督裡面.
一切的美善和智慧厚厚的賜給我們.
叫我們為你的緣故.
努力地去思考,努力地去實踐.
願你賜福給在座每一個人.
每一個你所愛的兒女.
我們這樣的禱告交託.
奉耶穌名求.
阿們.
(字幕由 Amara.org 社群提供).
\newpage



\section{}
\label{sec:Jp_xjonF0nk}
\textbf{2019年研究院開學崇拜「真係要饒恕?」-- 雷競業牧師}
\newline
\newline
連結: \href{https://youtube.com/watch?v=Jp-xjonF0nk}{\texttt{ https://youtube.com/watch?v=Jp-xjonF0nk}} ~~~~ 語音日期: 2019-09-10 
\newline
\newline
\hyperref[sec:AoYPWrVX2A4]{\small{< < < PREV SERMON < < <}}
~
\hyperref[sec:index]{\small{[返主目錄]}}
~
\hyperref[sec:YKdJJovZoTc]{\small{> > > NEXT SERMON > > >}}
\newline
\newline
$^{1}$今晚開學崇拜的訊息.
是由中國神學研究院天恩洛佑教席神學科的教授.
雷敬業牧師為我們宣講.
雷牧師的題目.
「真係要饒鼠?」.
我們將時間公敬交給雷牧師.
好.
等了十六年才能站在這裡.
很辛苦的.
我在中神教了十六年.
這個開學崇拜是輪流.
老師按某些次序去排.
詳細情況你問一下院長.
總之排了十六年.
到底開學崇拜是做什麼的呢?.
其實我最初來中神的時候.
開學崇拜應該是給一個lecture.
就是老師的專長學養.
然後不知道什麼時候.
我都不記得了.
開始變成一篇講道.
不過我又在想.
如果我今天又做《釋經講道》.
其實用我們政教部長李偕青的說法.
其實我每個星期早上都去送外賣.
所以今天不想又做一個「是朝炒河」給你.
我通常喜歡在中神做一些在教會.
我沒有膽量做的事情.
為什麼沒有膽量做?.
另一件事.
所以今天我又試試做另一件事.
為什麼選擇這個題目呢?.
其實這個題目和香港最近發生的事情.
都有些關係.
不過我不是為了香港最近發生的事情選擇這個題目.
還有如何應用這個題目.
就由你自己對號入座去決定.
我之所以選擇這個題目.
其實是因為我很喜歡這個故事.
其實我在《著人意道》的書.

$^{41}$那個sermon已經用過這個故事.
不過那次是很簡短的.
這次就詳細一點跟你說.
我今天要講的故事.
就是來自這個人的一本書.
Simon Wilson Ford.
Simon Wilson Ford是一個猶太人.
他在集中營經過一段時間.
當二次世界大戰結束之後.
他就成為一個Nazi Hunter.
就是他專門去抓那些藏在土裡的納粹黨人.
然後就抓他.
然後就繩之以法.
他在60年代出了一本書叫《The Sunflower》.
你一定會知道為什麼這本書叫《太陽花》.
這本書你未必看到.
不過副題就寫著.
On the Possibilities and Limits of Forgiveness.
就是說到底饒恕的可能和有沒有限制呢.
他就在這本書裡講了他自己的一個故事.
Simon Wilson Ford也是一個很重要的人.
其實他差點就拿到諾貝爾和平獎.
據內幕人士所說.
不過他最後就輸給了Elie Wiesel.
不過他也是一個很出名的猶太人.
經歷大屠殺的猶太人.
所以我今天跟大家分享.
上半部講了他的故事.
這本書下半部就邀請了五十多個不同的人.
去回應他這個故事.
基本上他講完故事就問.
你會不會饒恕故事裡的人呢.
他請了五十多個人去回應.
我今天分享的下半部.
就會回應一下那些回應.
看看我怎樣去理解那些人的回應.
既然玩玩東西.
所以開始講故事.
我就假設我是Simon Wilson Ford.
我現在開始跟你講這個故事.

$^{81}$當那些德國人開始去抓猶太人的時候.
我也不例外.
我就被抓到一個集中營裡.
不過在我這個集中營裡.
我每天都有機會出去一個醫院裡工作.
我做的事情其實很簡單.
我把垃圾堆在一起.
然後運到醫院外.
我每天早上就會離開集中營.
晚上工作完就回到集中營.
當我去到集中營的路上.
我都會看到一些Sunflower.
看到一些向日葵.
因為德國的士兵.
喜歡在他們每一個死了的士兵墳墓旁.
都種一朵向日葵.
當我每次經過向日葵的時候.
我很羨慕那些德國人的士兵.
我覺得那個向日葵.
就好像是埋在土地之下的那個德國士兵.
向著這個世界.
繼續展示著他的生命.
他雖然好像離開了這個世界.
又好像還沒有離開世界.
這個向日葵好像是一個死亡的世界.
和一個活人世界之間的橋樑.
所以我很羨慕那些德國士兵.
我想到假如有一天我死了.
也不用假如了.
我多數都會死在集中營裡.
我死了之後.
我就會葬在一個墓地.
葬在一個很多人堆在的洞裡.
不會有一朵花去紀念我.
我在想.
為什麼那些德國人做了這麼多事情.
依然有一朵向日葵去紀念他的死亡呢?.
有一天當我又去醫院裡工作的時候.
其實這間本來不是醫院.
這間本來是我讀書的一間學校.

$^{121}$是一間工業學校.
所以裡面其實每一間班房.
我以前有記憶.
我都走過.
不過這間學校現在變成了德國人的醫院.
我覺得對我的學校是一個羞辱.
無論如何.
有一天當我又去工作的時候.
有一個護士忽然間就來找我.
他走到我的面前就問我.
你是不是一個猶太人.
我一聽他這樣說的時候.
我心裡就有很多疑問.
是凶?.
他要做一些事情去欺負猶太人.
是急?.
是好事?.
因為我試過.
其實遇過一個德國士兵.
他就將一些他不想吃的東西.
他吃飽吃剩的東西.
放在我面前.
然後就叫我去拿.
我說為什麼你不給我.
那個德國士兵跟我說.
因為我沒有親手給你.
當我的上司問我.
有沒有將一些食物給猶太人.
我可以坦白跟他說.
沒有.
你喜歡就拿吧.
他就走開了.
我就拿了一些食物.
是不是又有些德國士兵想給我一些東西呢.
不過當我跟著那個護士一路走的時候.
走了一段時間之後.
我終於去到一個房間.
那個房間很暗.
我隔了一會才看到.
原來房間裡面只有一張床.

$^{161}$床上就睡了一個人.
我看到這個人已經全身被一些紗布包著.
我也看到這個人應該是一個很軟弱.
是一個快要離開世界的人.
那個護士將我帶到這個人的面前.
然後他就離開了.
這個人雖然他好像瀕臨死亡.
但是當我走到他身邊的時候.
他就用他的手抓住我的手.
他就不讓我走.
我很驚奇地發覺.
原來他一個臨死的人.
他的手也可以很有力.
接著這個病人就開始跟我說他的故事.
他說我的名字是Carl.
現在我要做兩個人.
我的名字是Carl.
我是一個SS Soldier.
我是德國的特種部隊.
我現在快要死了.
我心中有些事情.
我覺得我需要找一個猶太人跟他說.
我現在很希望聽到一個猶太人.
跟我說饒恕的話.
當一個士兵跟我說的時候.
我第一個印象就是.
我為什麼要饒恕你呢.
我見盡的是德國士兵所做的各種傷天害理的事.
有什麼理由我要饒恕你呢.
接著Carl就繼續說.
你要習慣一下.
如果我是一個好一點的演員.
我就假裝兩把聲.
不過我不想做一些那麼搞事的事情.
接著Carl就跟我說.
其實你看我是一個SS Soldier.
你可能以為我是一個很殘忍的人.
其實不是的.
我小時候我是一個天主教徒.
我父母都是一個很敬虔的天主教徒.

$^{201}$我甚至在天主教會裡.
我是那些Order Boy.
就是幫神父收拾聖餐.
遞東西給小朋友.
神父也很喜歡我.
不過當我去到中學的時候.
我就開始加入了Hitler's Youth.
就是希特勒的納粹黨的青少年部隊.
當我加入了青少年部隊的時候.
其實我爸爸很生氣.
我爸爸是一個Democrat.
是一個所謂的Socialist.
就是社會主義者.
他覺得納粹德國.
納粹黨是一些邪惡的事情.
他很不喜歡我加入納粹黨.
不過我的朋友個個都加入了納粹黨.
因為納粹黨有一些很威風的制服.
然後你在學校裡.
別人都會好像尊敬你多幾分.
所以我就加入了.
當我加入了Hitler's Youth的時候.
我的家裡的氣氛都變了.
我爸爸本來對納粹政府有很多批評的話.
不過現在他都不再在我面前說.
我知道他依然是很生氣的.
當我好像上了床.
他以為我上了床睡了的時候.
我就會聽到他跟我媽媽.
說很多投訴納粹政黨的東西.
不過在我的面前.
他已經靜默.
所以當二次世界大戰開始之後.
很自然的我就加入了SS Troop.
因為很多Hitler's Youth都是.
夜多歲就加入了SS Troop.
SS Troop其中一個的工作.
就是去找那些所謂賣國賊.
我記得我跟著軍隊一路走的時候.
當我去到波蘭的一些地方的時候.

$^{241}$其實那時候打仗剛剛開始.
當我們去波蘭進攻俄國的軍隊.
蘇聯的軍隊的時候.
他們節節敗退.
不過當我去到一個地方的時候.
有一個軍官已經在那裡的軍官告訴我們.
那天有一個蘇聯的軍隊.
做了一個地雷.
炸死了幾十個德國士兵.
所以他們很憤怒.
然後我跟我的隊友.
就一起去到一間屋的面前.
在這間屋那裡.
我看到有很多的一些猶太人在那裡.
其實我以前對猶太人都沒有什麼接觸.
我唯一的接觸就是.
我媽媽的醫生其實也是一個猶太人.
不過後來因為德國政府的政策.
我媽媽的醫生被人抓去集中營.
我媽媽經常都跟我投訴.
她說沒有一個醫生像我以前那個醫生那麼好.
那次是我第一次見到.
數以百計的猶太人.
聚集在一間屋的原地.
然後我看到那些士兵.
將那些猶太人一個一個趕進去屋裡.
起初我都不知道他們在做什麼.
然後有一個軍官跟我說.
他說我們要報復那些蘇聯地雷.
炸死了我們幾十個德國士兵.
我在想為什麼報復蘇聯的軍隊要找猶太人呢.
不過我沒有時間想那麼多.
因為當我見到有些軍官開始要求猶太人.
帶一罐一罐的汽油進入那間屋裡.
我就知道一定有些很慘烈的事情要發生.
當那些德國軍官將所有的猶太人都趕進屋裡.
還有叫那些男人將那些一罐一罐的汽油.
都搬進了屋的地下那一層的時候.
然後軍官就跟我們說.
你拿起你的手榴彈.

$^{281}$將那個安全鎖匙拔出來.
扔進去那間屋裡.
所以當我們這樣做的時候.
我們就見到那間屋很快就紅紅的大火燒了上來.
很多的猶太人因為忍受不住那些火的痛.
所以他們就在窗口跳出來.
我們就圍著那間屋拿著我們的窗.
如果有哪個人走到窗口或者想跳出來.
我們就會向他開槍.
我們想肯定每一個猶太人都是以死亡終結.
當這些一個個猶太人跳下來的時候.
其中一幕令我是很難忘的.
其中我見到有一個爸爸.
應該是一個爸爸.
他一手抱著兒子.
他身邊有一個女人.
應該也是個媽媽.
她一手抱著兒子.
用另一隻手掩著兒子的眼.
然後爸爸的背部已經燒著了.
她就抱著兒子跳下來.
然後媽媽就一起跳下來.
我想那天我也不知道死了多少人.
我也不知道開了多少槍.
不過在那天裡面最難忘的.
就是這個男孩子的臉紅.
一個有大大眼睛的男孩子.
我覺得今天他就好像站在我面前一樣.
不過終於那些猶太人都死了.
我們就繼續隨著軍隊出發去其他的地方.
其實做士兵的生命是很悶的.
很多時候我們就躲在一些戰壕.
或者躲在一些地方等候.
來開始攻擊.
當我們去到另一個地方.
沒多久幾個月之後.
我們又去到另一個地方.
其實經過那一次火燒那間屋的時候.
我見到我和我的同伴.
很多都很不開心.

$^{321}$不過我們的軍官跟我們說.
我們在打仗.
打仗怎麼可以心軟.
心軟就一定輸了.
所以要記住.
這些猶太人是走狗.
死多少個都不可惜.
知道了嗎.
他們是很邪惡的人.
死一個世界就好一點.
軍官不斷跟我們洗腦的時候.
過了兩個月.
我們都沒有再想這件事.
不過在去到兩個三個月之後.
有一次我去到一個地方.
準備開始打仗.
我們等了很久.
終於那個軍官跟我說.
我們要衝出去.
向著你的敵人開槍.
當我在戰壕跳出來.
當我向前衝的時候.
忽然之間.
我發覺我整個人停下來.
有些東西好像抓住我的腳一樣.
然後在我眼裡.
我看到的就是這個男孩.
和他的父母.
好像在戰場那裡走過來.
他跟我說.
你不可以再殺我了.
當我看著這個男孩的時候.
我忽然之間就失去了知覺.
當我醒過來的時候.
我就在醫院裡面.
就好像你現在看到我一樣.
被很多的紗布包紮著我.
醫生告訴我.
有一個炸彈在我身旁爆炸.
他在我身上拔了幾十塊.

$^{361}$或者超過一百塊的碎片.
他也很驚訝.
我居然可以生存下去.
不過說真的.
其實我也真的不太想生存.
因為我現在全身都痛.
他們將我從一間醫院送去另一間醫院.
其實我很想回到家裡.
我想我死之前.
都可以跟我爸爸媽媽說聲對不起.
不過他們沒有這樣做.
他們只是將我從一間醫院.
送去另一間醫院.
我的身體就越來越痛.
我知道我快要離開這個世界.
我有很多時間去想起當天所發生的事情.
或許這就是上天對我的懲罰.
他沒有叫我馬上死.
他要我回想我所做過的事情.
我想起我當年小時候的信仰.
我想如果我還有那種信仰的話.
或許我可以去到耶穌面前求祂饒恕.
不過我沒有了.
我沒有了我的信仰.
我現在去到這個光景.
我只是在這裡等死.
但是我很想我死之前.
能夠有一點平安離開這個世界.
你是一個猶太人.
你可不可以跟我說.
你饒恕我吧.
當卡爾一直跟我說這個故事的時候.
我心裡不是有很大的感受.
因為我見到的這些殘忍的事情太多了.
這個德國士兵現在快要死了.
於是他就求人饒恕他.
但是很多的猶太人.
他們死的時候.
被人一槍打死的時候.
沒有人聽他們最後的申訴.

$^{401}$他們就像一隻狗一樣被人打死了.
我為什麼要跟他說饒恕呢.
不過這個也是一個快要死的人.
我跟他再去責備他又有什麼意思呢.
我對著他看了他很久.
最後我掙扎那隻手.
我拉開自己.
然後我一聲不吭就出去了.
他跟我說話.
我最後看著他的時候.
他的手合起來.
就好像在做一個禱告一樣.
當我離開醫院之後.
過幾天那個護士來找我.
她告訴我Karl已經死了.
不過她有一些他生前的衣服和一些信.
她很想我幫她帶給她媽媽.
我看到那個包裹上的地址和名字.
然後我就把信交給護士.
我不想做這些事.
我和我的集中營裡的其他朋友分享.
他們都說為什麼要饒恕Karl呢.
如果他覺得內疚的話罪有應得.
但是我始終沒有辦法叫心裡平安.
到底我對Karl是不是說了.
到底我對Karl應該說些什麼呢.
到底我這樣走了對不對呢.
打完仗之後.
我就終於滴起心肝.
我要去找他的家人.
其實我也不知道想做什麼.
不過總之我覺得這個故事還沒完.
所以我要去見他的父母.
當我去到Karl住的地方的時候.
我發覺其實他住的那一區已經被炸得熔融爛爛.
我很辛苦才找到他父母住的家.
然後我去敲門.
其實那間屋二樓三樓都爛了.
只剩下地下那層.
當我敲門的時候.

$^{441}$一個老人家就出來開門.
我就問他你是不是Karl的媽媽.
他說是的.
然後他就請我進去.
我問他Karl的爸爸呢.
他說打仗沒多久爸爸就死了.
你是哪位啊.
其實我之前沒想過怎麼說.
不過那一刻我覺得很多東西都說不出來.
我只能夠跟他媽媽說.
我在醫院見過你的兒子Karl.
然後他媽媽就開始跟我說話.
Karl小時候其實很乖.
他是一個讀書很厲害的小孩子.
他又很聽我的話.
是一個很乖的小孩子.
不過後來他加入了Hitless Youth.
然後他就跟他爸爸吵架了.
當Karl離開家裡去參軍的時候.
那一天他離開的時候.
他都沒有跟他爸爸說再見.
他爸爸也沒有理會他.
躲在房裡.
Karl就走了.
他爸爸走了沒多久.
我就聽到Karl離開世界.
我的兒子離開世界的消息.
然後他媽媽說完這些.
她就繼續拿他的照片出來.
拿他小時候的照片出來.
她只給我看青靚白淨的男孩.
就是她兒子.
她繼續說很多她兒子小時候的故事.
不過我都沒心機去聽.
我坐了一會之後.
然後我就多謝她.
我就離開.
我去的時候有想過.
將在醫院裡發生的事情跟媽媽說.
但是當我去到見到媽媽.

$^{481}$聽到她跟我說的一切.
她給我看的那些照片.
我覺得我沒辦法跟媽媽說她兒子的故事.
雖然在對話當中.
我覺得媽媽好像有點想推卸責任.
所以我曾經罵過媽媽.
不過你有沒有想過.
你不出聲.
你容許立稅黨當權.
其實你也有份殺那些猶太人.
不過媽媽看著我.
她的答案是.
我都不懂那麼多.
其實我有聽過猶太人被殺的消息.
不過那時候我都不敢相信.
我老公也知道這些.
不過我這個女人.
我怎知道哪些是真哪些是假.
當我看著她的時候.
我覺得我都沒心情繼續罵下去.
我就離開了那個媽媽.
離開了那個媽媽之後.
始終這個問題一直纏繞著我.
我應不應該染樹膠呢.
假如你那天在這個病房裡.
你會對卡爾說些什麼呢.
回到現實.
當Simon Winstenfors寫完這個故事之後.
他出版這本書.
邀請很多人去回應.
既然這是關於猶太人的故事.
所以當然第一個.
我就會看看裡面一些猶太人的學者.
他們會怎麼說.
第一個我看的是Abraham Hussle.
如果在中神比較久的同學就知道.
以前我們現在還是榮休院長.
余達森牧師很喜歡說Abraham Hussle.
經常說他寫了一本書叫The Prophet.
那本先知.

$^{521}$說到我們基督徒要在這個社會有先知的角色.
不過坦白說一句.
我還沒空看那本書.
有一天我會看的.
所以我首先要看看Abraham Hussle.
他怎麼回答這個問題呢.
老實說他的答案令我也有點驚訝.
我先讀一次英文.
No one can forgive crimes committed against other people.
According to Jewish tradition.
Even God himself can only forgive sins committed against himself.
Not against man.
我對猶太教不熟悉.
當然師敬給我.
他是專家.
我一點也不熟悉.
不知道原來猶太教裡面有這樣的想法.
Abraham Hussle繼續要解釋.
他講了一個故事.
他說以前有一個很出名的Rabbi.
叫做Rabbi Bisk.
有一天Rabbi Bisk在火車上坐.
旁邊有一個男人.
可能火車很擠擁的緣故.
那個男人開始和其他人聊天.
唱歌很開心.
Rabbi Bisk被迫坐在旁邊的時候.
那個男人就在罵他.
為什麼我們唱歌.
我們玩牌.
你不和我們一起.
你一聲不響就坐在那裡.
很沒人性.
不理我們.
罵了Rabbi Bisk一段時間之後.
Rabbi Bisk最後就下車.
不理他們.
後來那個男人終於知道.
原來坐在他旁邊的就是大名鼎鼎的Rabbi Bisk.
所以這個男人最初寫信.

$^{561}$後來去到他家裡找Rabbi Bisk.
不過Rabbi Bisk都不饒恕他.
不肯見這個男人.
Rabbi Bisk的兒子就很奇怪.
就問他爸爸.
為什麼你不肯饒恕這個人呢.
Rabbi Bisk的答案就說.
就算他和我說對不起都沒有用.
因為現在他和我說的對不起.
是和Rabbi Bisk說.
他知道我是誰.
但是當日他中傷的是Mr. Common Man.
是一個普通人.
所以他應該和一個普通人道歉.
不過這樣說就沒有辦法.
有機會有饒恕.
因為那個普通人已經不見了.
就算他和另一個普通人道歉.
也不是那個普通人.
第一我覺得驚訝的是.
原來猶太有些傳統是這麼硬的.
這麼鷹派的.
即使是神自己都只能夠.
饒恕得罪他的罪.
而不能夠饒恕得罪人的罪.
什麼是得罪神的罪.
我不知道他在指什麼.
可能就坐了神.
你可以求神饒恕.
不過如果一會你生了李思敬的氣.
走來罵了李院長兩句的話.
那就對不起了.
神都幫不了你.
你一定要和李院長饒恕你.
為什麼猶太教有這個傳統呢.
我想和基督教有不同的看法.
不過我們一會再說基督教的看法.
但Abraham Husserl有一點要欣賞.
他明白饒恕是一種客觀關係.
這是兩個人的關係.

$^{601}$他的中心思想是.
假如我傷害了A先生.
沒有別人可以代替A先生.
因為這是客觀的關係.
不是一個隨便主觀感情的改變.
關係破壞了就一定要重建.
這才是真正的饒恕.
我想這是很重要的.
因為現在有些人說饒恕.
就以為饒恕純粹是一種主觀的東西.
你想開一點就行了.
Abraham Husserl的女兒.
也是一個出名的猶太教老師.
她在書中說了這句話.
「不論是要求原諒.
納西人的子孫.
都應該繼續聽到.
猶太人的哭聲.
並且保留自己的人性」.
她的女兒強調不應該給饒恕.
為什麼呢?.
因為給了饒恕的時候.
就好像現在大家都平等了.
我又不欠你,你又不欠我.
她覺得這樣是不公道的.
德國人要永遠聽到猶太人的血的呼聲.
否則他們又會再做一次.
這些毫無人道的事情.
同樣對這個看法.
我贊同也有不贊同的地方.
贊同的地方就是說.
的確饒恕不等於忘記.
饒恕了的人.
你饒恕了我.
其實我也要繼續不斷地道歉.
譬如大家聽過前幾年.
曾經中國要求日本.
在二次世界大戰對中國南京大屠殺等事去道歉.
當時日本首相就說.
我們日本已經道歉過很多次了.

$^{641}$不過不是首相道歉.
但不知道這個大臣那個人又道歉過.
這個人又道歉過.
道歉到什麼時候呢?.
二次世界大戰那批人都快死了.
其實如果我有機會見過日本首相.
我就會跟他說.
道歉到什麼時候?.
我的答案就是道歉到永遠.
但不是為了我.
我沒有生日本人氣.
其實他道歉不是為了我.
他的殘酷的事情跟我沒什麼關係.
他道歉是為了他們自己.
因為日本人願意繼續道歉.
就即時表示他承認.
其實他跟他的祖先都沒什麼分別.
當年他們能夠做到一些殘酷的事情.
其實今天我們都會的.
今天的日本人跟當年的日本人沒什麼分別.
今天的中國人跟文革的中國人都沒什麼分別.
我們要不斷地去道歉.
提醒我們自己能夠做到的殘酷的事情.
不過對於Susanna的說法.
我有些疑問.
Susanna這樣說的時候.
好像只有德國人需要道歉.
Jews.
我們這些被人擊打的人.
被人虐待的人.
我們就不需要有任何覺得歉疚的事情.
我們可以堂堂正正.
永遠都不饒恕那些德國人.
我覺得這樣將人分為兩種.
一種叫做施害者.
一種叫做被害者.
這個紅溝是永遠過不了的.
我覺得這不是基督教給我們的教導.
再看看.
其實不是那麼多.

$^{681}$只有四個左右.
不會跟你說完五十九個.
放心.
接著我們看看第三個猶太作者的.
一個很出名的Rabbi.
Harlot Kushner.
他最出名寫了一本書.
Where is God when we hurt?.
寫了一本書.
看看這個Rabbi又說什麼.
有些東西是很對的.
但有些東西是有問題的.
To be forgiven is a miracle.
It comes from God.
And it comes when God chooses to grant it.
Not when we order it up.
他首先強調.
饒恕是一個神職.
是神做的事情.
我想基督徒絕對認同.
不過這個神職是什麼呢.
Kushner就說.
God's forgiveness is something that happens inside us.
Not inside God.
Freeing us from the shame of the past.
原來饒恕是一種在我裡面的東西.
所以如果我是猶太人.
我饒恕的德國人.
這是我內心裡.
純粹是我內心的一個心路歷程.
這個歷程讓我們能夠從過往的那種羞愧.
釋放出來.
Forgiving is not something we do for another person.
It represents a letting go of the sense of grievance.
And perhaps most importantly.
A letting go of the role of victim.
Kushner來說.
其實某程度上.
Kushner我覺得和Husserl有些相反的看法.
Kushner說.

$^{721}$饒恕純粹是我個人裡面的東西.
你可以想像.
以前有一個人.
Mortwether 家良哥.
曾經打我三槌.
打得我很痛.
我現在決定饒恕家良哥.
這純粹是我心裡面的一種態度的改變.
為什麼要饒恕呢.
Kushner就說.
如果我不饒恕家良哥.
我經常有一種被害者的心態.
我就會覺得.
我就是那麼軟弱.
被家良哥打了三槌的人.
所以我走來走去都覺得自己是一個.
很沒用的人.
被家良哥.
家良哥也打了我三槌.
Annoysusnager 就說.
我要從這種身份釋放出來.
這是說內在的釋放.
同樣地.
我覺得從基督教的立場來看.
饒恕的確有內在釋放的層面.
輔導老師更加熟悉這些東西.
不讓輔導老師說這個層面.
但我覺得聖經也強調.
饒恕不是單單一個人裡面的東西.
神的賜予是在神裡面發生的.
不是在神裡面.
如果你讀過基督教神學.
你就會說肯定不是.
神與耶穌要死在十字架裡面.
如果你讀過Moldman.
你就會知道十架的事情.
其實是神的 Trinity.
神的三一.
神的歷史的一部分.
很明顯.

$^{761}$饒恕是涉及神某程度的改變.
或者神裡面的本質的流露.
所以有對的部分.
又有問題的部分.
最後這個.
想給你看看幾種.
與基督教很不同的觀點.
Sean Emery.
他是一個 Atheist Jew.
他是一個猶太人.
不過已經不信任任何宗教的猶太人.
他也是經歷二次世界大戰.
他就是因為二次世界大戰的經驗.
讓他覺得無法相信神.
他又說了另一個觀點.
Psychologically forgiving or not forgiving.
in this specific case.
is nothing more than a question of temperament or feeling.
他基本上來說.
你要饒恕Karl.
你就饒恕吧.
你不喜歡不饒恕就不饒恕吧.
有什麼大不了的.
沒什麼重要的.
這個純粹是看你個人.
你個人好一點.
你個人沒什麼所謂.
那就饒恕他吧.
你個人計算得比較清楚.
那就不要饒恕.
也沒所謂的.
因為他覺得Political.
從一個政治的立場來看.
Forgiving or not forgiving.
is quite relevant.
個人的饒恕是多餘的問題.
不重要.
重要的是什麼呢.
Sohn繼續說.
他的意思是.

$^{801}$你個人要不要饒恕立稅黨.
我不管你.
這些個人事情不重要.
重要的是社會層面.
我是絕對拒絕任何從入立稅黨復和的行為.
因為立稅黨的人.
或者廣泛一點來說.
所有做了一些atrocities.
做了一些傷天害理的人.
一定要追究到底.
這個才是重要的一點.
其實這一點我們也看到.
如果大家記得前幾年.
柴玲曾經說過她饒恕.
天安門的人.
饒恕李鵬.
立刻被人炮轟.
其實某程度上.
我覺得這個炮轟是有誤解的.
我相信柴玲是在說個人的饒恕.
另外那些攻擊她的人.
其實是在說在一個社會層面.
我們可不可以有復和.
對於Sohn的看法.
我覺得兩個問題都重要.
社會如何處理.
固然重要.
今天不說了.
反而集中說.
到底我個人是否饒恕.
一個做盡傷天害理的人.
對我對那個人.
都是一種很重要的東西.
不是單單政治層面.
也是個人層面.
對我來說.
基督教如何看饒恕呢.
我覺得基督教告訴我.
第一,饒恕不是單單內心態度的改變.
是一種客觀關係的改變.

$^{841}$當我們學習去饒恕一個人.
其實就是嘗試重新建立和那個人的關係.
同時也透過神去建立這個關係.
也是在建立與神的關係.
我相信剛才說的那句經文.
「我們要饒恕人,正如天父饒恕我們一樣」.
我在說.
假如我們不懂得去饒恕別人.
其實我們與天父之間的關係.
都是一種缺陷.
當然我們饒恕別人的時候.
別人未必會接受我們的饒恕.
或者覺得他做錯了甚麼.
所以真正完全修補的關係.
是需要有人肯饒恕.
有人肯接受饒恕.
真正接受悔改的饒恕.
不過無論如何.
就算那個人不理會我.
我覺得我主觀願意去饒恕.
都是要建立我與神的關係.
和開始修補我與那個人的關係.
重要的一步.
第二,我覺得從基督教的立場來說.
你記得剛才Kushner說的.
「原諒是一件奇蹟」.
基督教告訴我們.
其實這個奇蹟.
這個禮物已經給了我們.
主耶穌死在十字架的時候.
已經給了我們饒恕的能力.
假如你坐在靠隔離.
你覺得很難去饒恕他的話.
你其中一個可以做的禱告.
就是求聖靈幫助你.
正如像羅馬書所說.
當我們不知道怎樣禱告的時候.
聖靈已經為我們作出無聲的嘆息.
第三點.
同樣地我覺得要反對Kushner的看法.

$^{881}$純粹是一種主觀的東西.
我相信從我們信仰的角度來看.
你某程度是我的一部分.
我某程度是你的一部分.
這個就是新生未知.
舊生就知道了.
新生就會開始學這個忠神很偉大的名詞.
叫做co-humanity.
中文翻譯成為共人性.
饒恕不是單單為了讓我好過一點.
饒恕是因為我明白.
如果有一天你還是我的敵人的話.
我的humanity 我的人性還是缺少了一部分.
所以不像Susanna所說.
猶太人和德國人之間.
有一個不可跨越的鴻溝.
所以德國人的責任就是不斷地求饒恕.
猶太人就是不斷地訴說他被人害.
其實不是的.
你想想當年德國人害猶太人.
但如果轉過一個情況.
猶太人有沒有害其他人呢.
猶太人對巴勒斯坦人.
又是不是這麼無辜呢.
我們可不可以這麼清楚地分到.
有些是罪犯,施惡者.
有些純粹是被害者.
是沒有施惡的人呢.
如果我們知道大家都是同一條船.
那我就要承認.
我饒恕你也是為了我的緣故.
因為我的生命少了你的話.
我的生命就缺少了一部分.
另一點我想提醒的是.
饒恕和forgiveness.
饒恕不是所謂的完全忘記.
饒恕是我回頭看.
我雖然知道這件事.
但這件事已經不成為我的答案.
我依然有我的記憶.

$^{921}$但沒有一個boundage.
上主.
其實我們所做的所謂的壞事.
上主都寫在書上.
不過他不再計算這些是惡行.
其實說到饒恕的時候.
某程度上哈利·克里斯納就說得對.
就是讓我可以看回我過去的生命.
無論別人對我好或不好.
那些都不會成為我的捆綁.
我學會對昨日所發生的事情.
懷著一個感恩的心去接受.
我能夠這樣做的時候.
我明天才可以充滿盼望.
如果我執著過去一些傷痛不放的話.
就會成為我今天的捆綁.
最後一點我想說的是.
我覺得作為主的門徒.
我們學習饒恕是一種屬靈的操練.
Not merely free.
這個free就指自由.
沒有心理捆綁.
更加是一種spiritually free.
為什麼這是diet.
因為我說的是在聖靈裡面的自由.
主的靈在哪裡.
哪裡就有自由.
這種自由不是說我大大方方算了.
這種自由是基於與主的生命連結而來的自由.
如果主是饒恕了卡爾.
我們與主的生命連結.
我們又怎能夠說我們不去饒恕他呢.
這是我們跟隨聖靈需要做的行為.
所以結論的時候.
我想多說幾句就完結了.
其實大家有沒有想過.
Forgiveness and justice.
Justice without love makes a God.
如果我們只是把著一些公義.
我們不肯饒恕的話.

$^{961}$我們會變成一個神.
不過我們變成神的時候是很恐怖的.
這句話可能很抽象.
所以我想引用Elie Wiesel的一句話.
我講過Elie Wiesel.
是拿到諾貝爾和平獎的.
他最重要的作品就是《Night》這本書.
1958年第一次出版.
裡面他提到有這一句話.
就是當他看完一些.
Elie Wiesel說到他小時候.
其實他很喜歡猶太經典.
他研究一些Carpenters.
即是他研究一些猶太神奇的書寫.
不過當他去到集中營的時候.
他看盡那些殘酷的事情.
他失去了他的信仰.
其中他《Night》那本書.
講到有以下一段說話.
(這段是Elie Wiesel的第一次翻譯).
(這段是Elie Wiesel的第二次翻譯).
(這段是Elie Wiesel的第三次翻譯).
(這段是Elie Wiesel的第四次翻譯).
(這段是Elie Wiesel的第五次翻譯).
(這段是Elie Wiesel的第六次翻譯).
(這段是Elie Wiesel的第七次翻譯).
(這段是Elie Wiesel的第八次翻譯).
(這段是Elie Wiesel的第九次翻譯).
(這段是Elie Wiesel的第十次翻譯).
當你執著一些怨恨,你不放手的話.
你會很堅強.
不過這個堅強,你要付什麼代價呢?.
這個堅強的代價就是一個沒有神,沒有人的世界.
你就是執著你那份仇恨.
其實Elie Wiesel在這段之前.
為何他會說這段呢?.
其實故事很快就說完了.
因為在集中營裡面.
有些是波蘭的地下軍隊.
有一天軍官抓了幾個反立粹的地下軍隊.

$^{1001}$其中一個是一個十一,二歲的男子.
Elie Wiesel就說到當時的德國人.
懲罰這些所謂賣國賊或者地下軍隊的方法.
就是他們要在集中營的中間弄一支棍.
然後吊頸,要把那些人吊在營的中間.
然後整個營的人要圍著圈子.
看著那個人被人吊,看著他死.
這就是德國人嚇那些猶太人的方法.
他說到因為那次他吊的是一個十一,二歲的男子.
因為他的身體太輕,所以他很難死.
如果你是肥佬就死得很快.
但他是一個十一,二歲很輕的男子.
Wiesel就說到.
他就這樣呆了半小時.
在那裡吊在營的圈子裡半生半死.
他還活著,當我過了他.
那些德國軍隊要求每一個猶太人走過那些死囚的面前.
我經過的時候,他還在生存.
他的嘴巴依然是紅色的,他的眼睛還沒有熄滅.
我聽到後面的我問.
「為了神的好處,神在哪裡?」.
我後面有人在問,神去了哪裡?.
我裡面聽到一個聲音回答.
「神在哪裡?神在這裡,從這監牢裡」.
很可惜,Elie Wiesel不知道.
其實他這個答案正正就是一個真正的答案.
主耶穌在哪裡?.
就是吊在監牢裡,吊在吊頸死的男子裡面.
也是在被警察打的示威者裡面.
也是被家人揭穿了.
就是被警察家人揭穿了.
在受苦的警察裡面.
正正因為主耶穌在每一個受苦的人裡面.
所以他有這樣的權利去饒恕所有的罪.
我們也要學習去饒恕.
我們低頭做一個禱告.
主耶穌我們感謝你.
當我們人生去到最黑暗的時候.
當我們覺得沒有人在我們身邊的時候.
我們知道你在我們身邊.

$^{1041}$當我們覺得生命完全沒有力的時候.
我們沒有辦法再走多一步.
我們希望白天不會升出來的時候.
我們知道你扶著我們.
你每一天與我們同行.
你告訴我們罪惡終會過去.
所以我們不需要怨恨.
我們只需要饒恕和盼望.
給我們這份勇氣.
禱告奉耶穌基督的名.
(掌聲).
我猜我不懂得弄一個powerpoint.
(笑聲).
謝謝.
多謝雷牧師.
預備了16年.
(笑聲).
有一篇很觸動我的.
我們叫做address.
不是lecture 也不是sermon.
我是1993年12月1日加入忠臣的.
輪到我說開學崇拜.
是2013年9月.
我等了20年.
也沒有你說得那麼好.
感謝主.
雖然崇拜我們很少拍手.
今天我們應該多謝雷牧師.
也要感謝.
(掌聲).
(字幕由 Amara.org 社群提供).
\newpage



\section{}
\label{sec:YKdJJovZoTc}
\textbf{2020年研究院開學崇拜崇拜「你準備好未?」- 李適清博士}
\newline
\newline
連結: \href{https://youtube.com/watch?v=YKdJJovZoTc}{\texttt{ https://youtube.com/watch?v=YKdJJovZoTc}} ~~~~ 語音日期: 2020-09-09 
\newline
\newline
\hyperref[sec:Jp_xjonF0nk]{\small{< < < PREV SERMON < < <}}
~
\hyperref[sec:index]{\small{[返主目錄]}}
~
\hyperref[sec:wfhoZve0GwI]{\small{> > > NEXT SERMON > > >}}
\newline
\newline
$^{1}$今日開學崇拜為我們主講訊息.
是李色清博士.
主因擇深教職神學科副教授.
其教務長.
Gin老師十七年前.
都是同大家同學一樣.
第一次參加忠臣的開學崇拜.
MDiv畢業.
Gin老師回到自己的舞會侍奉.
亦都成為忠臣的儲備師資.
到外丁堡大學完成他的博士課程.
回來用了十年的時間.
開墾我們晚上職場神學的訓練課程.
2013年Gin老師成為副教務長.
2016年她擔任教務長.
2018年余牧師正式從她的教席卸任.
教員人事委員會委任Gin老師成為主因擇深教職的副教授.
今日她分享的題目直接了當.
你準備好了嗎.
我們恭請Gin老師.
Gin老師的開學崇拜文件.
弟兄姊妹平安.
今年是一個很特別的開學崇拜.
短短一年多香港以至整個世界.
都因為社會事件,疫情,國際關係的變化下.
情況完全逆轉.
急劇的改變不單是外在.
我們每日新聞可以看到.
更加是很靠近.
進入了每一個家庭.
影響我們的日常生活.
起居飲食.
就在這個特別的時間空間裡.
上帝讓我們在忠臣相遇.
一起接受裝備.
可能你向上帝委身.
說那句我願意的時候.
從來都沒有想像過.
會有這樣的巨大改變.
你準備好迎向這個未來嗎.

$^{41}$我今天選的經文.
在西班牙書第三章.
我想和大家溫習一下.
在一個動盪不安的世界裡.
先知給我們甚麼訊息.
西班牙書的訊息.
和其他小鮮書都有很多相似.
但它有一個取向很不同.
書裡似乎沒有其他先知書.
那麼直接叫人去悔改.
有人說.
似乎悔改已經太遲了.
西班牙的宣告更多的是.
鼓勵和堅固那些.
仍然未在苦難當中.
但將要面對困境的人.
西班牙書第三章.
我將它大概分為三個段落.
第一個段落一至七節.
是對耶路撒冷和列國的審判.
八至十三節是重大的轉變.
十四至二十節是一首新歌.
我們先看第一個段落.
一至七節是連接西班牙書頭兩章.
三章一至七節是這樣說的.
「這搏逆污穢欺壓的城有禍了.
他不聽從命令.
不領受訓誨.
不倚靠耶和華.
不親近他的神.
他中間的首領是咆哮的獅子.
他的審判官是晚上的豺狼.
一點食物也不留到早晨.
他的先知是虛浮詭詐的人.
他的祭司是毒聖所.
強解律法.
耶和華在他中間是公義的.
斷不做非義的事.
每早晨顯明他的公義.
無日不言.

$^{81}$只是不義的人不知羞恥.
我耶和華已經除滅列國的民.
他們的城樓毀壞.
我使他們的街道荒涼.
以致無人經過.
他們的城邑毀滅.
以致無人也無居民.
我說:你只要敬畏我.
領受訓誨.
如此你的住處.
不至照我所已定的除滅.
只是你們從早起來.
就在一切事上敗壞自己.
先知在第二章上文宣佈.
上帝對外邦各國的判詞.
而在第三章一至五節.
就轉向耶路撒冷.
經文說:上帝所愛的.
所揀選的耶路撒冷.
是不聽從命令.
不受訓誨.
不倚靠耶和華.
不親近他的神.
當中包括他的領袖.
首領審判官.
先知制師.
都是貪婪詭詐的人.
但對比耶和華.
上帝卻是公義的.
他的公義其實一直都存在.
只是不義的人.
不知羞恥.
列國都犯罪面對審判.
而上帝特別向他一直都揀選.
保守的百姓來說話.
不過他們都沒有聽.
他們背信棄義.
連他們的領袖都背棄上帝.
去到第六至七節.
直至連接第八節.

$^{121}$是一連串的事件.
說到上帝在歷史的行動當中.
讓列國受到審判.
然後呼籲耶路撒冷要敬畏.
敬拜耶和華.
領受訓誨.
過程中有一個警告.
或者是盼望.
不自照我所已定的除滅.
事實是以色列民.
並沒有因為一眾先知的警告而回轉.
他們反而變本加厲.
最後上帝的審判.
臨到消滅全地.
在審判之下.
整個人類包括我們都沒有希望.
不過奇妙的事情發生.
經文在三章九節.
出現了一個明顯的轉折位.
這裡上帝的憤怒消滅.
突然扭轉了.
上帝做了新的不同的事.
就是第二個段落.
第八節開始.
耶和華說.
你們要等候我.
直到我興起努力的日子.
因為我已經定義.
朝聚列國聚集列邦.
將我的憤怒.
就是我的烈怒都傾在他們身上.
我的憤怒如火.
必消滅全地.
那時.
我必使萬民用清潔的言語.
好求告我.
耶華的名.
同心合意地侍奉我.
祈禱我的.
就是我所分散的民.

$^{161}$必從古實河外來.
給我獻共民.
當那日.
你必不因你一切得罪我的事.
自覺羞愧.
因為那時.
我會從你們中間除掉.
那些高傲之輩.
你都不再在我的聖山狂傲.
我卻要在你們中間.
留下那些困苦貧寒的民.
他們必投靠我.
耶華的名.
以色列所剩下的人.
必不作罪孽.
不說謊言.
口中也沒有鬼咋的舌頭.
而且吃喝躺臥.
無人驚嚇.
第九至十節.
萬民分散了的上帝兒女.
都要來求告和敬拜耶華.
當那日.
高傲狂傲的那些.
都要被上帝除去.
留下的是甚麼人呢?.
在十二至十三節.
就是那些困苦貧寒的.
不作罪孽.
不說謊言.
口中也沒有鬼咋的.
這些剩下的剩餘之民.
他們不再懼怕.
他們投靠耶和華.
在西班牙書這個段落開始.
在上帝的滅路和滅絕當中.
上帝親自留下一些人.
這些不是我們認為很神聖的人.
更加不是領袖或者智者.
很厲害的人.

$^{201}$都不是.
經文說.
這些是困苦貧寒的人.
這些是單單依靠上帝的恩典.
去存活的人.
在西班牙書十三章十一節.
特別說.
這班剩餘之民是謙卑的人.
他們靠耶和華得到力量和平安.
而不是靠其他東西.
不單這樣.
他們更加是一班具道德品格的人.
不作罪孽.
不說謊言.
口中也沒有鬼咋的舌頭.
在這個前提下.
我們就去到最後第三段.
第三段第十四節起.
是記載了一首頌歌.
是上帝的子民得勝.
勝過仇敵的新歌.
十四節起是這樣說.
「錫安的民啊應當歌唱.
以色列啊應當歡呼.
耶路撒冷的民啊應當滿心歡喜快樂.
耶和華已經除去你的刑罰.
趕出你的仇敵.
以色列的王.
耶和華在你中間.
你必不再懼怕災禍.
當那日必有話向耶路撒冷說.
不要懼怕.
錫安啊不要手軟.
耶和華你的神是施行拯救.
大有能力的主.
他在你中間.
必因你歡欣喜樂.
默然愛你.
且因你喜樂而歡呼.
那些屬你為無大會受煩.

$^{241}$因你擔當羞辱的.
我必聚集他們.
那時我必罰辦一切苦待你的人.
又拯救你瘸腿的.
聚集你被趕出的.
那些在全地受羞辱的.
我必使他們得稱讚有名聲.
那時我必領你們進來.
聚集你們.
我使你們被老之人歸回的時候.
就必使你們在地上的萬民中有名聲得稱讚.
這是耶和華說的.
十四至二十節這個段落.
是包括了歌唱歡呼.
這裡原本的文字是命令式.
為什麼歡呼歌唱呢?.
因為耶和華已經除去你的刑罰.
趕出你的仇敵.
他們不用怕了.
因為耶和華與他們同在.
先知宣告耶和華是得勝的那位.
而且愛他的百姓.
還有就是一個應許.
那些瘸腿的.
被趕出的.
受羞辱的.
都可以平安歸回.
三章十四至二十節.
這一首頌歌.
在一個紛亂絕望的世界當中.
給我們有新的盼望.
小占詩書所說的.
耶和華的日子.
都是大而可畏.
上帝審判萬民的訊息.
都是很難啃的訊息.
如果不是有盼望和回歸的應許.
人實在不知如何是好.
西班牙這裡很清楚地告訴我們.
將會有新的一日.

$^{281}$再無恐懼和羞辱.
取而代之是歌頌和讚美.
不過在那一日之前.
上帝的百姓.
都要面對這個過程裡的苦難.
人生的歷程高低起伏.
人是沒辦法掌控.
人更加是落在罪裡.
更加是自以為很行.
直到災難來到了.
人沒辦法自救.
但是上帝有恩典.
在人沒辦法的時候.
終於我們學會要完全依賴上帝.
或者有些人在另一個極端.
可能一早已經覺得自己沒什麼好.
很不濟.
當你來到中神遇到那些殺手老師.
原本儲了很多年的自信.
全部都粉碎.
當你決定要主動認識其他同學.
卻遇到疫情.
連見面都有困難.
當我們已經習慣了.
很安穩自由的生活.
短短的一年多.
一切都改變都消失.
這個時候我們要記得.
上帝在這個時空帶你來到中神.
一定不是偶然.
我們不是需要科科裸A.
也不是需要一定做到什麼.
而是我們預備好.
把握上帝在我們面前.
每一個學習的機會.
迎接不同的改變.
讓上帝按他的心意.
來塗鑿我們每一個.
中神是一個學習的地方.
當中我們有老師.

$^{321}$有同工 同學.
我們有校友等等.
我們都來自不同的背景.
不同的宗派.
我們對於社會急速變化.
都有不同的看法和立場.
在這裡我們學習開放的對話.
可以有學術的交流.
可以和而不同.
我們唯一相同的.
就是那份對於上帝的委身和追求.
這件事令我們能夠在主裡面合而為一.
正正因為這樣.
面對不同的議題.
我們都要準備好.
身邊的弟兄姊妹.
和你有不同的意見.
我們可以學習如何對話.
一起來尋問.
可能接下來的日子.
我們都會面對很實際的困難.
有些環境的限制.
可能令我們很不習慣.
很不舒服.
疫情突然間變化帶來的限制和要求.
都只是其中一個可能的例子.
我們都要準備好.
去面對限制.
選擇合適的回應和行動.
和來自不同群體的弟兄姊妹聊起.
包括同學.
教會弟兄姊妹.
微信的朋友.
其實各人都有不同的想法和考慮.
我想大概大家只有一點是共通.
就是對於未來.
有一種的不確定.
覺得真的要改變.
無論是家庭生活.
教會的模樣.

$^{361}$工作的活動.
一切都會更新.
但如何改變呢?.
其實我們都在一起摸索中.
西班牙書最後這裡.
對上帝有一個很特別的.
出人意表的形容.
在審判和消滅的處境裡.
上帝是拯救的上帝.
在三章十七節裡.
這節的經文講出一個最基本.
又最美麗的事實.
上帝默然愛你.
施行拯救.
當我們大多數人可能在想.
我有沒有犯罪得罪上帝呢?.
又或者.
其他人如何犯罪得罪這位創造主呢?.
這節經文的重點.
卻不是我們有沒有犯罪.
或者是否需要上帝拯救.
經文講的是上帝已經決定了.
是祂採取主動.
要施行拯救.
是上帝主動去愛我們.
為何上帝要拯救我們呢?.
經文這裡給了一個很簡單的原因.
因為上帝喜悅我們.
默然愛我們.
祂不單只愛我們.
祂還因我們而喜樂.
如果你家裡有小朋友.
你或許比我更加明白這種弔詭性.
曾經有父母跟我分享.
他小朋友兩歲多.
講道理又未必講得明白.
又對所有事情都很好奇.
很難照顧.
不過當小朋友玩到累了.
睡著的時候.

$^{401}$看著他.
就好像看著一個小天使一樣.
這位大有能力的主.
在人類理當滅亡.
因為罪而走向末路的時候.
他堅持愛到底.
因為他的愛是充滿歡樂的.
他為他的兒女而歡呼.
這種這樣的愛.
收窄到我們一個小小的實際環境.
我不知道你準備好.
環境帶來跟你預期當中.
很不同的變化了沒有.
疫情反覆.
可能我們都不知道.
這個學期有多少實體課堂和活動可以進行.
疫情就是常態.
起碼有兩個大範圍我們可以注意.
第一就是我們埋身的生活.
第二就是我們對應環境的一些思考.
和選擇.
埋身的防疫工作.
不用多說了.
最重要的是保障身邊的人.
可能我們都要面對一個不同的生活作息的安排.
隨著環境的因素.
這些東西會不斷地改變.
讓我們操練更加能夠貼地.
當情況變化的時候.
我們能夠對應那種限制.
在過程當中.
其實這些改變是會幫助我們發現自己更多.
面對自己的感受和一些情緒的反應.
我們認識自己更加多.
在網上上課.
需要我們更多的主動性.
學習如何把握科技的用處.
和一些不同的機會.
在不同的模式中.
發掘我們可以如何創意地一起連結.

$^{441}$其實群體生活的確只靠線上.
網上是很困難的.
尤其是在這裡認識新朋友.
不過與其因此沮喪.
我們就不如更加積極主動去發揮創意.
由offline去到online.
再由許可的情況下.
online去到offline.
我們善用科技的話.
或者幫助我們看到不同的景象.
可以繼續前進.
從經文我們看得到.
審判當中有恩典.
上帝為我們開出不同新的路.
先知對以色列.
以至對列國都一直發出判詞.
但是當人來到盡頭.
上帝卻宣告祂的拯救.
西班牙書中間有一種超出人所能理解的互動或張力.
就是上帝的公義.
和上帝的慈愛喜樂之間有一種張力.
不知道你認識的上帝是怎樣的上帝.
如果對你來說.
上帝好像一位宇宙警察.
要抓我們犯的罪.
我們就會生活在一個恐懼當中.
如果上帝對你來說.
好像一位消防員.
你可能只會在有事的時候去找他.
如果上帝對你來說.
是一個能夠解決問題的幫得上忙的工人.
你一定會碰過他.
直至你遇到無法控制的絕望的情況.
我們才會放手去跟隨上帝.
不過如果像先知一樣.
他遇到的上帝.
就是那位大能的征戰者.
掌王權的君王.
他是會為他的百姓歌唱歡樂的.
無論眼前的逆境是短暫還是很長.

$^{481}$多麼的惡劣.
上帝才是真正掌權和得勝的那位.
他關心的不單是我們是否存活.
而是他和人渴望有一種親密的愛的關係.
我們可以因為上帝的喜樂而喜樂.
從上而來的恩典會與我們同在.
令我們生命在無助不安當中.
仍然有方向經歷上帝的實在.
經文裡說上帝就像慈愛的父親.
看到小孩子一樣.
他因我們而喜樂.
我們又問下去.
我們又如何回應這位上帝呢?.
看回聖經裡很多人物.
上帝的兒女都是在困難,不明白.
甚至在絕望中學習經歷上帝的同在.
在當中歌頌和喜樂.
就像瑪利亞一樣.
她未結婚就要生孩子.
面對不少壓力.
在她等待耶穌出生的時候.
她唱的頌歌被稱為「尊主頌」.
瑪利亞說:我心尊主為大.
我寧以臣我的救主為樂.
正如保羅和西拉在監獄裡.
仍然唱詩歌.
《士林傳》16章25節.
約在半夜.
保羅和西拉禱告.
唱詩讚美神.
眾囚犯也則以而聽.
哈巴谷書裡.
因著主的同在.
人面對絕境都可以歡欣喜樂.
雖然無花果樹不發黃.
葡萄樹不結果.
橄欖樹也不效力.
田地不出糧食.
圈中絕了羊.
棚內也沒有牛.

$^{521}$然而我要因耶和華歡欣.
因救我的神喜樂.
今天我們也來讀神學.
教會對你有期望.
老師也有期望.
你對自己也有期望.
但更重要的是.
上帝對你有什麼期望.
祂想你用什麼態度迎接.
接下來的學習.
在困難和限制當中.
我們也會有沮喪.
失落的時刻.
但在經文裡.
對於面臨困境的人.
有一個恢復信心的秘訣.
就是天父親自向我們啟示.
祂的大愛.
耶和華你的神是施行拯救.
帶有能力的主.
祂在你中間.
必因你歡欣喜樂.
默然愛你.
且因你的喜樂而歡呼.
這是一個榮耀的啟示.
上帝愛祂的百姓.
是堅定不移的.
無論環境如何.
我們也不需要懷疑.
或許我們看不到什麼跡象.
但其實上帝正正在動功.
祂每時每刻都看顧愛我們.
針對我們自以為是.
有各種提醒和管教.
針對我們軟弱無助.
祂為我們開路.
在忠臣的一年三年六年裡.
你的課程不單表面學院要求的科目.
靈培等等.
更加是上帝為你度身訂做的經歷.

$^{561}$我們學的不單止是知識.
更加重要是你遇到老師.
同工同學在群體裡互動.
和環境的改變來互動.
在過程裡不單止學院.
給我們一個學習的環境.
其實上帝也為我們度身訂做.
包括你的家庭.
你的堂會.
你的工作.
社會處境.
各種的變化.
當中你會有不同的遭遇.
和一些時機.
西方書三章九至二十節.
揭示了很重要的一點.
上帝祂審判最終的目標.
不是毀滅.
而是潔淨和更新.
祂要讓祂的百姓.
回歸祂的懷抱.
成為祂所喜悅的人.
所以當我們準備迎接.
新學年那些未知的變化的時候.
我們知道三件事.
或者我們看不到什麼跡象.
但是上帝正在動工.
祂的應許是不會落空的.
但是對於渺小的我們來說.
必須要帶著這個應許.
我們預備好迎接環境的改變.
那些衝擊或者困難.
第二件事就是上帝因為你而喜樂歡呼.
祂帶領你此時此刻來到這個地方.
必定有美好的心意.
讓我們預備好迎接上帝.
在這個學年為我們設計的學習.
第三件事.
上帝就是我們最大的喜樂.
祂默然愛我們不離不棄.

$^{601}$所以我們預備好.
無論在什麼環境.
我們都存有這份喜樂的盼望.
我們仰望上帝的掌權和祂的工作.
願主祝福我們這個學年的學習.
(音樂).
\newpage



\section{}
\label{sec:wfhoZve0GwI}
\textbf{CGST Magazine Vol 5 憂質教育}
\newline
\newline
連結: \href{https://youtube.com/watch?v=wfhoZve0GwI}{\texttt{ https://youtube.com/watch?v=wfhoZve0GwI}} ~~~~ 語音日期: 2019-08-12 
\newline
\newline
\hyperref[sec:YKdJJovZoTc]{\small{< < < PREV SERMON < < <}}
~
\hyperref[sec:index]{\small{[返主目錄]}}
~
\hyperref[sec:fnpbCi1eLhU]{\small{> > > NEXT SERMON > > >}}
\newline
\newline
$^{1}$(音樂).
(字幕製作/時間軸:秋月AutumnMoon).
(字幕由 Amara.org 社群提供).
\newpage



\section{}
\label{sec:fnpbCi1eLhU}
\textbf{CGST Magazine 創刊號}
\newline
\newline
連結: \href{https://youtube.com/watch?v=fnpbCi1eLhU}{\texttt{ https://youtube.com/watch?v=fnpbCi1eLhU}} ~~~~ 語音日期: 2017-03-07 
\newline
\newline
\hyperref[sec:wfhoZve0GwI]{\small{< < < PREV SERMON < < <}}
~
\hyperref[sec:index]{\small{[返主目錄]}}
~
\hyperref[sec:WhsjJODKaTI]{\small{> > > NEXT SERMON > > >}}
\newline
\newline
$^{1}$(音樂).
(字幕由 Amara.org 社群提供).
\newpage



\section{}
\label{sec:WhsjJODKaTI}
\textbf{CGST Magazine 第三期 土地公公}
\newline
\newline
連結: \href{https://youtube.com/watch?v=WhsjJODKaTI}{\texttt{ https://youtube.com/watch?v=WhsjJODKaTI}} ~~~~ 語音日期: 2019-08-07 
\newline
\newline
\hyperref[sec:fnpbCi1eLhU]{\small{< < < PREV SERMON < < <}}
~
\hyperref[sec:index]{\small{[返主目錄]}}
~
\hyperref[sec:JYDFJ9wnjB4]{\small{> > > NEXT SERMON > > >}}
\newline
\newline
$^{1}$(音樂).
(字幕製作/時間軸:秋月AutumnMoon).
(字幕由 Amara.org 社群提供).
\newpage



\section{}
\label{sec:JYDFJ9wnjB4}
\textbf{CGST Magazine 第二期 尋思:三代中港・情・意・結}
\newline
\newline
連結: \href{https://youtube.com/watch?v=JYDFJ9wnjB4}{\texttt{ https://youtube.com/watch?v=JYDFJ9wnjB4}} ~~~~ 語音日期: 2019-08-03 
\newline
\newline
\hyperref[sec:WhsjJODKaTI]{\small{< < < PREV SERMON < < <}}
~
\hyperref[sec:index]{\small{[返主目錄]}}
~
\hyperref[sec:tYIP6koYn_I]{\small{> > > NEXT SERMON > > >}}
\newline
\newline
$^{1}$今天我們三個坐在這裡.
代表三代人去討論一下.
關於中國,香港之間一些關係的問題.
看看我們三代人對於這個問題.
有什麼相同的地方.
也都可能有一些分歧的地方.
我們是三代人.
我就是60後.
余牧就是50前.
我們這位阿宇就是未到30.
尚未立的一位年輕人.
我們想反省一下一些所謂國情.
想想信仰和我們的國情.
對我們愛國的感情有什麼關係.
所以就一起討論一下.
很感謝兩位來的.
我們就由第一個問題開始.
我們想一下.
請不吝點贊訂閱轉發打賞支持明鏡與點點欄目.
\newpage



\section{}
\label{sec:tYIP6koYn_I}
\textbf{CGST Magazine 第二期 尋思:三代中港・情・意・結 - 情懷}
\newline
\newline
連結: \href{https://youtube.com/watch?v=tYIP6koYn-I}{\texttt{ https://youtube.com/watch?v=tYIP6koYn-I}} ~~~~ 語音日期: 2017-06-08 
\newline
\newline
\hyperref[sec:JYDFJ9wnjB4]{\small{< < < PREV SERMON < < <}}
~
\hyperref[sec:index]{\small{[返主目錄]}}
~
\hyperref[sec:2d8c5hyS5LI]{\small{> > > NEXT SERMON > > >}}
\newline
\newline
$^{1}$香港過去20年發生了很多事情.
如果我想簡單地形容.
你覺得過去20年是一種進步還是一種退步?.
基本上中港關係和我們互動是退步的.
負面的比較多.
這是很可惜的.
我覺得一個很重要的事件是23條的處理.
我用今天的眼光回看以前.
很明顯是一種退步.
最重要的是我們開始發現.
我們香港的政府都不是為香港的人著想.
這一點是本土最根本最起初的一個想法.
我自己的經驗是覺得某程度上是退步.
而這個退步不只是溝通的問題.
是整個北京政府態度的問題.
我自己的感覺是2001至2003年回來的時候.
\newpage



\section{}
\label{sec:2d8c5hyS5LI}
\textbf{CGST Magazine 第二期 尋思:三代中港・情・意・結 - 意識}
\newline
\newline
連結: \href{https://youtube.com/watch?v=2d8c5hyS5LI}{\texttt{ https://youtube.com/watch?v=2d8c5hyS5LI}} ~~~~ 語音日期: 2019-07-08 
\newline
\newline
\hyperref[sec:tYIP6koYn_I]{\small{< < < PREV SERMON < < <}}
~
\hyperref[sec:index]{\small{[返主目錄]}}
~
\hyperref[sec:L5cJtWqteCI]{\small{> > > NEXT SERMON > > >}}
\newline
\newline
$^{1}$在白話最後那句.
我們愛我們的國家.
國家愛不愛我.
打了一個很大的問號在雪地裡.
有一種欲哭無淚的感覺.
當時的感覺是.
無論再差的都是自己的國家.
我是其中一份子.
我不可以逃避.
我是有責任的.
現在我們所面對種種的困難.
我將它拋開.
好像與我無關.
我有我自己的路要走.
怎可以這樣.
我覺得你有你的路.
我有我的路.
我有深感受.
剛才余福利說了很多話.
黃河.
我從來沒有提過黃河.
他提的.
中國的神州大地.
我都沒有提過中國的神州大地.
那種土地情懷.
我自己的感受是.
我們只看到的地圖是香港.
我可能對城門河會有些感情.
因為我住沙田從小住到大.
可能對維多利亞港的感情會更深.
對於香港人的感覺.
我要回到那裡.
我要去了解那裡.
原來我要去給回鄉證.
所以很多時候.
那種土地的情懷.
我只住到那個地方.
去到一個比較遠的地方.
雖然現在交通方便了.
但那種感覺是真的遠.

$^{41}$你都要好像去外國一樣.
對.
我是香港人.
你是中國人.
我要你認可我.
你要一國兩制.
你不要一國兩制就不用.
問題是我們選擇了什麼.
你說如果我們.
不要九龍.
不要這樣的安排的話.
即是說大陸可以一千萬人湧入來.
我們都不應該擋住.
我們都要這樣去看.
你覺得是好事嗎.
一千萬人可以湧入來香港.
這件事我不知道.
我真的不知道.
那個的意義是什麼.
我不知道.
不過我肯定不同意.
你的手不用顫抖.
我沒有留意我的手在顫抖.
無論如何不是因為你.
多謝您收睇時局新聞,再會!.
(字幕由 Amara.org 社群提供).
\newpage



\section{}
\label{sec:L5cJtWqteCI}
\textbf{CGST Magazine 第二期 尋思:三代中港・情・意・結 - 意識「信仰與身分認同」}
\newline
\newline
連結: \href{https://youtube.com/watch?v=L5cJtWqteCI}{\texttt{ https://youtube.com/watch?v=L5cJtWqteCI}} ~~~~ 語音日期: 2017-06-10 
\newline
\newline
\hyperref[sec:2d8c5hyS5LI]{\small{< < < PREV SERMON < < <}}
~
\hyperref[sec:index]{\small{[返主目錄]}}
~
\hyperref[sec:82HZNC1kimk]{\small{> > > NEXT SERMON > > >}}
\newline
\newline
$^{1}$(音樂).
感一定要有歸屬感.
我們應該要覺得我們屬於某個community.
我們不是一個individual floating around.
你覺得我們的信仰.
對於你去回答這個問題.
圈要劃多大.
即是怎樣去建立身份.
你覺得對你有什麼影響呢?.
即是從信仰的角度來看.
My being here.
我在這裡絕對不是accidental.
是上帝有祂自己的心意.
將我放在這裡.
叫我怎樣在這個群體裡面.
是和這個群體一同來同行.
祂將我放在這裡.
祂是給我一個很重要的任命.
是叫我將我現在身處的那個群體.
和上帝的角度連結起來.
有打算移民的牧師問我.
為什麼在這個時候你還說留呢?.
我的答案很簡單.
我的羊在哪裡.
我就在哪裡.
現在我的羊是在香港.
我就留在這裡.
牧羊我的羊.
其實我們這一代人是更加擁抱全球.
我想其實擁抱全球.
可能因為現在交通方便了.
通訊亦方便了.
其實有更加多人願意投身一個海外宣教.
我們是有更加闊的胸襟去接受自己.
可能就算我黃皮膚.
但可能我的一生是奉獻在不同的地方.
所以我覺得這個都是成為了我們這一代的可能性.
嗯.
我也感受到在教會裡面.
其實有這個典範轉移.

$^{41}$可以這樣paradigm shift.
沒有這種assumption.
覺得我和中國人傳福音是特別urgent(緊急)的.
不是這種想法.
我有時也感受到夾在兩代中間.
我很多時候夾在這個落差.
就是我有時要掙扎怎樣去幫助年輕一代.
多些明白所謂對中國的情意結.
MING PAO CANADA | MING PAO TORONTO.
(字幕由 Amara.org 社群提供).
\newpage



\section{}
\label{sec:82HZNC1kimk}
\textbf{CGST Magazine 第二期 尋思:三代中港・情・意・結 - 糾結}
\newline
\newline
連結: \href{https://youtube.com/watch?v=82HZNC1kimk}{\texttt{ https://youtube.com/watch?v=82HZNC1kimk}} ~~~~ 語音日期: 2019-08-01 
\newline
\newline
\hyperref[sec:L5cJtWqteCI]{\small{< < < PREV SERMON < < <}}
~
\hyperref[sec:index]{\small{[返主目錄]}}
~
\hyperref[sec:v55_9RXOWv0]{\small{> > > NEXT SERMON > > >}}
\newline
\newline
$^{1}$現在政府是在推動.
無論香港政府 內地政府都想推動這件事.
但事實上香港有很多人是很反感這種路向.
你覺得在這種張力之下.
香港社會有沒有出路呢.
我想其實我們又覺得中港融合未至於.
我們要成為敵對.
未必的.
如果我們真的覺得要敵對的話.
我們就會覺得其實太困難.
你這麼近 我們這麼小.
國家機器所有資源這麼大.
我想在這個融合上.
正如鍾神經常說的.
Unity in diversity.
那種diversity我們可以怎樣保存得好些.
我經常有一句說話就是.
勉強就沒有幸福.
剛才我們開場的時候都說到.
其實香港為什麼會搞到這樣.
就是因為在過程當中有幾次硬銷.
如果共融的話都是硬銷的話.
那就先這樣.
在尋求共融的過程當中.
我覺得最重要的一件事就是什麼.
尊重.
不要小看我們.
我們都是其中一股改革的動力.
最初你開始的時候.
你講到香港當時都有一個期望.
就是我們可以成為一個中國.
改革的一個力量.
我覺得我們不能夠放棄.
這種願望和理想.
MING PAO CANADA | MING PAO TORONTO.
(字幕由 Amara.org 社群提供).
\newpage



\section{}
\label{sec:v55_9RXOWv0}
\textbf{CGST Magazine 第二期 尋思:三代中港.情.意.結 - 想像}
\newline
\newline
連結: \href{https://youtube.com/watch?v=v55-9RXOWv0}{\texttt{ https://youtube.com/watch?v=v55-9RXOWv0}} ~~~~ 語音日期: 2019-08-08 
\newline
\newline
\hyperref[sec:82HZNC1kimk]{\small{< < < PREV SERMON < < <}}
~
\hyperref[sec:index]{\small{[返主目錄]}}
~
\hyperref[sec:8IQTX4IveJg]{\small{> > > NEXT SERMON > > >}}
\newline
\newline
$^{1}$你個人看著2047的時候.
你有什麼願景呢?.
你希望香港去到2047.
跟中國的關係是怎樣呢?.
香港是怎樣走到去407.
或者47有什麼跟現在有什麼分別呢?.
我的願景是香港可以仍然是一個很有特色的地方.
無論你……我想其實無論最後的政權.
因為其實你知道政權的事情.
現在都沒有選特首.
你的政權其實你都不能夠敵.
我覺得反而是索求一種.
香港仍然可以保持到它一定的獨特性.
而那種獨特性是可以.
讓香港仍然可以很自由地.
例如今天可以達到一些.
它本身仍然有它自己的一些制度.
或者其實它對於香港整個宗教群體的那種空間.
我就給你大膽一點.
2047的時候我就98歲了.
可能都不在了.
基本上.
我說大膽一點.
就是我真的希望2047中國的整體.
是能夠分享到香港很多我們現在很重視的價值.
要做到這一步.
我們香港人需要努力.
有我們努力的角色去扮演.
我會相信上主都真的看著香港.
亦都看著中國的社會.
我理性上不能夠證明.
不過我會相信47會比現在好的.
因為我相信我真的相信.
好像余木說的.
現在香港有的一些.
一個所謂比較開明自由的社會.
我會期待47的中國會比現在開明.
多謝您收睇時局新聞,再會!.
(字幕由 Amara.org 社群提供).
\newpage



\section{}
\label{sec:8IQTX4IveJg}
\textbf{CGST Magazine 第二期 尋思:從流離到安身 黃嘉樑}
\newline
\newline
連結: \href{https://youtube.com/watch?v=8IQTX4IveJg}{\texttt{ https://youtube.com/watch?v=8IQTX4IveJg}} ~~~~ 語音日期: 2017-06-10 
\newline
\newline
\hyperref[sec:v55_9RXOWv0]{\small{< < < PREV SERMON < < <}}
~
\hyperref[sec:index]{\small{[返主目錄]}}
~
\hyperref[sec:cQDr9IMmvaA]{\small{> > > NEXT SERMON > > >}}
\newline
\newline
$^{1}$當我們目睹香港一處處承載本土歷史文化的地區.
因種種原因被拆毀重建.
當我們見到香港一些核心價值.
例如自由,平等,法治.
屢受到破壞,消耗.
所以我們雖然仍然好像在香港這個地方當中.
但我們彷彿好像經歷到被擄的處境.
感覺到疏離和割裂.
我們如何在這樣的環境下安身作為回應呢?.
先知以西傑這個片段提醒我們.
在流離的地方應該與其他流離者休戚與共.
放下自己.
顯明上帝忠心宣講和守望.
在生活層面上要活出神的心意.
我們或者可否這樣問呢?.
假如那位神甘願與我們流離.
成為我們接下來可以安身的基礎.
我們就會問今時今日我們的流離.
我們的安身,我們的同在.
能否同樣地成為其他人可以安身的基礎呢?.
多謝您收睇時局新聞,再會!.
(字幕由 Amara.org 社群提供).
\newpage



\section{}
\label{sec:cQDr9IMmvaA}
\textbf{CGST Magazine 第八期 SHALOM}
\newline
\newline
連結: \href{https://youtube.com/watch?v=cQDr9IMmvaA}{\texttt{ https://youtube.com/watch?v=cQDr9IMmvaA}} ~~~~ 語音日期: 2019-08-01 
\newline
\newline
\hyperref[sec:8IQTX4IveJg]{\small{< < < PREV SERMON < < <}}
~
\hyperref[sec:index]{\small{[返主目錄]}}
~
\hyperref[sec:_ISrkAG_KkM]{\small{> > > NEXT SERMON > > >}}
\newline
\newline
$^{1}$Zither Harp.
\newpage



\section{}
\label{sec:_ISrkAG_KkM}
\textbf{CGST Magazine 第四期 1% 99%}
\newline
\newline
連結: \href{https://youtube.com/watch?v=_ISrkAG_KkM}{\texttt{ https://youtube.com/watch?v=\_ISrkAG\_KkM}} ~~~~ 語音日期: 2019-07-24 
\newline
\newline
\hyperref[sec:cQDr9IMmvaA]{\small{< < < PREV SERMON < < <}}
~
\hyperref[sec:index]{\small{[返主目錄]}}
~
\hyperref[sec:NKv_tFpmlEM]{\small{> > > NEXT SERMON > > >}}
\newline
\newline
$^{1}$(音樂).
(字幕製作:貝爾).
(字幕由 Amara.org 社群提供).
\newpage



\section{}
\label{sec:NKv_tFpmlEM}
\textbf{CGST magazine 第六期 絕處• 逢生}
\newline
\newline
連結: \href{https://youtube.com/watch?v=NKv-tFpmlEM}{\texttt{ https://youtube.com/watch?v=NKv-tFpmlEM}} ~~~~ 語音日期: 2019-07-29 
\newline
\newline
\hyperref[sec:_ISrkAG_KkM]{\small{< < < PREV SERMON < < <}}
~
\hyperref[sec:index]{\small{[返主目錄]}}
~
\hyperref[sec:GGZt_muKlhs]{\small{> > > NEXT SERMON > > >}}
\newline
\newline
$^{1}$那對於那些年輕的年輕人,覺得教會太令人失望,太沉迷於它們的存在權力,你對他們的看法是?.
我只是香港的一位嘉賓,我不知道香港的年輕人的情況,但是我認識一些在德國的年輕人..
在德國,我希望說,如果你對教會失望,希望在神的國度,祂的國度將會來臨..
這是一片生命和生命的王國,自由和真理..
所以,要投入自己,不要看著別人,否則你會失望,而不是自己..
好的..
多謝您收睇時局新聞,再會!.
(字幕製作/時間軸:秋月AutumnMoon).
(字幕由 Amara.org 社群提供).
\newpage



\section{}
\label{sec:GGZt_muKlhs}
\textbf{Public Lecture: A Culture Of Life In The Dangers Of Our Time (Q&A)}
\newline
\newline
連結: \href{https://youtube.com/watch?v=GGZt_muKlhs}{\texttt{ https://youtube.com/watch?v=GGZt\_muKlhs}} ~~~~ 語音日期: 2018-06-27 
\newline
\newline
\hyperref[sec:NKv_tFpmlEM]{\small{< < < PREV SERMON < < <}}
~
\hyperref[sec:index]{\small{[返主目錄]}}
~
\hyperref[sec:B1NjObnaFkg]{\small{> > > NEXT SERMON > > >}}
\newline
\newline
$^{1}$We can now start our Q and A session..
OK..
Kin is always the first one..
(Laughter).
I always begin the first question..
OK. You talk about the ecological crisis..
OK..
For us Chinese people,.
I think our culture tend to treat the other world,.
non-human world as object of consumption..
So we eat shark fin, you know,.
and we eat abalone,.
and we eat everything, OK?.
Everything edible to Chinese people..
So it's a reflection that we consume.
the non-human world for our benefit..
That's part of our culture..
So in this context,.
how do you talk to non-Christian.
about, you know, the community of life.
that, you know, the shark fish is part of our life,.
the abalone is part of our life..
How do you communicate this to the general public?.
You know, what's your suggestion?.
(Laughter).
(Clears throat).
May we first thank Dr. Hong Liang..
I trusted him without understanding..
(Laughter).
I think the ecological crisis.
affects every human being on earth..
And therefore,.
planetary solidarity is in hand.
and is for everybody understandable.
for meeting this danger..
The danger is universal.
and every human being.
has a duty to save life..
(Clears throat).
We have the order of God, human beings, the earth..

$^{41}$And the Tao Te Ching of Chinese wisdom,.
we have the order of the divine, the earth,.
than the human being..
And we have to learn Chinese wisdom of Lao Tse.
to respect the earth as our mother..
Tao Te Ching second part..
OK..
(Silence).
This is a question sheet from....
(Chinese).
An old confession of the Orthodox Church in Aquileia, Italy..
It says to Christ,.
(Chinese).
I had an experience in my desperation.
behind barbed wire in the prison camps after the war..
And with Christ, the suffering Christ,.
what I not have hope and a years to life.
in this dangerous situation..
(Silence).
(Chinese).
She would like to know the genetically modified food.
and how would it....
The GM food, the genetically modified food,.
how does it related to the ecological crisis?.
Do you have any comments on the....
Do you understand the genetically modified GM food?.
(Silence).
I am for natural food and not for artificial food..
And my grandsons are vegan..
(Chinese).
(Laughter).
我想問一問,.
剛才你說人類中心是其中一個問題,.
其中一個對於自然環境是一個經濟危機,.
人類中心可能是之前卡爾巴克的問題,.
如果是這樣,.
我們作為人類,.
你最後問的問題,.
究竟我們是否應該是,.
或者不是,.

$^{81}$還有這件事是否無論是Sherry也好,.
Lobby也好,.
是否可能,.
還有怎樣可以成為可能?.
(Chinese).
(English).
天,夜,光芒,創造的天皇是第七天的食糧.
這就是著名的考察問題:基督是否根據基督的說法在六天或七天內創造了世界呢?.
(回答:回答給自己).
(笑).
(回答:回答給自己).
(這位先生:我會想問一個問題,關於時間的概念).
(這位先生:我知道您在您的《希望的道理》中提到基督是根據基督的說法).
(這位先生:您提出我們應該以循環的時間來取代現代的概念).
(這位先生:您如何解釋您對於時間的看法?).
(回答:您每年慶祝生日,我越來越老了).
(回答:所以,生日的循環是在一個時間線上).
(回答:時間的循環是在一個時間線上).
(回答:我們正在朝著未來的方向).
(這位先生:這兩個問題是關於經濟的,我可以將它們結合並向您介紹).
(這位先生:您提到社會正義是一個非常重要的問題).
(這位先生:如果我們要解決社會正義,貧困問題,我們應該透過公共管理來解決問題).
(這位先生:您如何建議我們如何幫助解決社會正義).
(這位先生:我們可以從我們的位置開始,以及我們有的社會角色,或是我們可以做一些政治和政治的事情).
(這位先生:您可以更詳細地解釋我們可以做的事嗎?).
(回答:我可以更詳細地解釋我們可以做的事嗎?).
(這位先生:您可以更詳細地解釋我們可以做的事嗎?).
(這位先生:我們可以更詳細地解釋我們可以做的事嗎?).
(這位先生:在提出其他問題之前).
(這位先生:我想提醒那些在Dossett的學校).
(這位先生:因為我們的助手需要收集資料).
(這位先生:所以這將是最後一堆問題).
(這位先生:這兩條問題是關於科技).
(這位先生:我們如何處理).
第二個問題.
我已經回答了.
聰明的人學習在內.
聰明的人學習在外.
在大浪中.
(你對這方面有什麼意見?).

$^{121}$我對這方面有什麼意見?.
我希望,但不太有信心.
因為我是一個貪心的人.
我也是一個有信心的人.
(笑).
有信心和貪心.
沒有分別去處理未來.
(這兩個問題是關於現代的個人主義).
(第一).
(在一個極度個人主義的自由社會).
(我們面臨著嚴重的社會崩潰).
(我們如何應對).
(以及營造社會的協調).
(為一個正義和關心的社會).
(第二).
(在現代世界).
(有強烈的強調).
(追求個人的快樂和個人的身份).
(例如).
(社會主義).
(他們也聲稱).
(正面宣揚生活).
(有許多種類).
(社會主義的社會主義).
(我們作為基督徒).
(如何看待這些社會主義?).
(請尊重自己的選擇).
(這是基督徒Paul的建議).
(但我沒有回答).
(這是最後一個問題).
(說「對」生活可能勇敢).
(但有時也可能不公平).
(當一個人身體受到極度的痛苦).
(而生活沒有正義).
(你認為自殺是否有幫助).
(這對某些人來說).
(可能會更加有價值).
(高齡的人).
(他們有百分之百的).
(希望健康的疾病).

$^{161}$(讓他們選擇去瑞士).
(自殺).
(我不會指責他們).
(但對我來說).
(這從來不是問題).
(兩年前).
(我太太因癌症死亡).
(我們沒有討論自殺).
(謝謝莫子曼教授).
(謝謝大家).
(感謝).
謝謝大家.
\newpage



\section{}
\label{sec:B1NjObnaFkg}
\textbf{Public Lecture: A Culture Of Life In The Dangers Of Our Time (Response: Dr Hong Liang)}
\newline
\newline
連結: \href{https://youtube.com/watch?v=B1NjObnaFkg}{\texttt{ https://youtube.com/watch?v=B1NjObnaFkg}} ~~~~ 語音日期: 2023-05-31 
\newline
\newline
\hyperref[sec:GGZt_muKlhs]{\small{< < < PREV SERMON < < <}}
~
\hyperref[sec:index]{\small{[返主目錄]}}
~
\hyperref[sec:MKd__CpGhKU]{\small{> > > NEXT SERMON > > >}}
\newline
\newline
$^{1}$(英文).
謝謝耀坤的介紹.
尊敬的莫特曼教授.
尊敬的李思穎院長.
尊敬的各位來賓.
大家晚上好.
歡迎來到中國神學研究院.
剛才我坐在這裡聽.
想起很多在Tübingen的場景.
可以說剛才在這裡發表演講.
這個老人家不是我們通常意義上所說的神學明星.
或者神學超級明星.
而是深刻塑造了第二次世界大戰之後.
世界新教神學的基本面貌的一個人.
而且也可以說.
他是代表了二十世紀新教神學基本精神的一個人.
所以剛才可以我們在這裡講.
剛才在這裡是二十世紀的新教神學本身.
在開口講話.
作為聽眾我們剛剛一起在經歷了一個新教神學的一個事件.
可以說在上個世紀五十年代末到六十年代初.
在試圖掙脫出戰爭意識形態廢墟的種種神學探索當中.
莫特曼的神學重新把弱者和被壓迫者的生命尊嚴.
放置到了一個中心的地位.
那麼他的批判性人道主義.
代表了歐洲新教神學的公共性的重生.
這個意思是什麼呢.
如果神學不能傳達出上帝對人的生命的無條件的肯定和接納.
以及人在這個基礎之上對自身的肯定和接納.
基督教神學就錯失了它最重要的任務.
剛才我相信大家在聽這個演講的過程當中.
一定會有一個感受.
反復出現的一個主題就是.
我剛才說的這種人道主義意義上的對生命的肯定和接納.
但是如果換作另外一個人.
比如如果是我剛才坐在這裡在講的話.
如果我來說這種對生命的接納和肯定的話.
就非常的可疑.
為什麼呢.
因為我生活和成長的這個時代.

$^{41}$是一個以新自由主義為基礎的一個消費時代.
讓我們這一代人去談尊重接納肯定生命.
很可能無意識當中我們所講的東西.
只是在肯定和美化自己消費生命的慾望而已.
很有可能是這個樣子.
但是如果換作這個老先生來講對生命的肯定.
那個意思是完全不一樣的.
如果大家讀過他的傳記.
或者說看過這個網絡上關於他的介紹的話.
你會知道這個老年人在少年的時候是被迫參軍的.
然後在戰俘營裡被關押了三年.
而且親身經歷了自己的出生地漢堡被夷為平地.
而且自己的朋友和戰友在一夜之間消失的無影無蹤.
而且親身經歷過自己的第一個孩子在出生的過程當中死去.
所以當這個老年人在說接納和肯定生命的時候.
他表達的不是一個中產階級式的一個消費的理想.
而是一種抗爭的態度.
這個抗爭的態度的內涵是什麼呢.
不是滑進享受和自我美化的深淵當中.
而是要在充滿死亡和腐爛的地方存活下來.
而且更重要的是要活出一個人的樣子.
所以可以說剛才在這個演講當中最核心的一個議題.
人要活出人應該有的樣子.
因為上帝無條件的肯定和接納了人的生命.
他是一個讓人活而不是一個讓人死的上帝.
可以說這種包含著抗爭元素的人道主義關懷.
具有強烈的批判性和先知氣質.
因為他反對一切蔑視,漠視和敵視生命的東西.
尤其是當這種東西顯得非常強大,不可撼動.
讓人絕望,讓人活不下去的時候.
我怎麼用來表達這種先知性和批判性的基本詞彙.
大家知道是盼望.
在基督教信仰當中.
盼望從來都是跟絕望聯繫在一起的.
沒有絕望作為前提.
所謂的盼望只是人自己希望達成的目標.
跟上帝無關.
當我們一切的計劃和籌算被清零的時候.
盼望才剛剛開始.
所以我們可以在這個意義上去講.

$^{81}$真正的盼望是要把絕望放到嘴裡細嚼蔓延.
盼望不是對未來收益的預期.
而是在滿盤皆輸的時候.
我們要把目光轉向啟示錄第20章第5節中的那個上帝的作為.
看啊,我把一切都更新了.
大家可能都知道中世紀有位偉大的神學家叫安瑟倫.
他有一個很有名的講法.
在他的Postlogium這本書裡提出一個很有名的講法.
叫信仰尋求理解.
這裡的意思是什麼呢.
就是他要求信仰具有這樣一種特徵.
就是信仰必然蘊含著對信仰對象的認識.
可是我們在回顧Mortimer教授在過去幾十年的整個的思想探索當中.
尤其是在這個批判性人道主義當中.
為這個安瑟倫的很深刻的見解添加了新的思想音符.
不僅僅是信仰尋求理解.
而且是盼望尋求理解.
他的意思是盼望不斷支持著.
不斷產生著對信仰對象的認識.
如果我們可以用一個比喻來表達.
安瑟倫和Mortimer之間的思想差異的話.
一個是信仰尋求理解.
一個是盼望尋求理解.
那麼我們可以說在安瑟倫那裡.
信仰者被帶進一個莊嚴肅穆的聖殿當中.
這是一個靜態的圖像.
他對這個聖殿充滿敬畏和好奇的瞻望.
就是信仰尋求理解.
可是在Mortimer這裡.
信仰者發現的是自己的腳下出現了一條.
看上去幾乎是走不下去的路.
而他的前方是《出埃及記》第十三章.
第二十一節當中提到的雲柱和火柱.
他要在不斷向前邁步當中去見識上帝的信使.
這是盼望尋求理解.
盼望所要認識和理解的對象.
是這個信使上帝的作為.
是他的日日星 苟日星.
這是信仰者可以真的放下過去.
但不是忘記過去.

$^{121}$忘記過去是不可能的 也是不應該的.
過去只能被轉化.
這是信仰者可以真的放下過去.
面向未來的根據所在.
我們如果回顧剛才的演講.
可以說其中批判性的人道主義.
是一個貫穿史中的基本線索.
無論是對原子武器威脅的反思.
還是對新自由主義腐蝕社會生存環境.
以及自然生態問題的思考.
還是對以伊斯蘭國為代表的宗教恐怖主義的觀察.
他們背後的基本關懷是完全一致的.
那就是他們都威脅到了.
人對生命本身的肯定和敬畏.
生命不是可以隨意上下的交通工具.
而是要被開墾的土地.
人和生命之間的關係不是支付與購買.
而是勞作和被允許勞作.
而這個關係的特徵不是消費而是責任.
要在盼望當中活下去.
而且要活出人應該有的樣子.
這是一個任務而不是一個選項.
所以透過剛才的演講.
我們可以很清楚地看到.
Waterman在上世紀六十年代到二十世紀初期的.
一條基本的思想線索.
而這條批判性人道主義的思想線索.
是第二次世界大戰之後.
新教系統神學自我重建的一個重要的方向之一.
但是今年是2018年.
我們在這裡可以問.
Waterman的這種批判性人道主義.
跟我們今天的生活有什麼樣的關係.
我們今天生活在其中的這個世界秩序.
是在第二次世界大戰之後奠定的.
而它現在正在加速地土崩瓦解.
很顯然我們在這個解體的過程當中.
面對兩組問題.
首先是以協商,以共識,以共通感.
為導向的國際政治倫理.

$^{161}$正在迅速地退出歷史舞台.
取而代之的是各類趨向地方性的權力想像.
和對差異的恐懼.
它的表現形態大家看得很清楚.
威權主義,民粹主義和保守主義.
其次是以人工智能和社交媒體為代表的.
技術空前發展所帶來的社會信任危機.
當然大家知道我指的肯定是世界範圍內的.
對信息的儲存,販賣和監控.
這兩組問題其實並不是什麼新的問題.
而是過去數年每一個人.
只要你使用手機都可以感受到的問題.
而如何在這兩組問題語境當中.
講好生命的價值和它背負的責任.
講好對生命的肯定和接納.
這非常不容易.
它是一個非常非常重要的神學任務.
也是做出新的神學人類學發現的機遇所在.
毫無疑問,Mortimer的批判性人道主義.
將會持續給予我們深刻的啟發.
所以也是在這個意義上.
我們要再次對Mortimer教授的到來.
表達最由衷的感謝.
謝謝大家.
(掌聲).
謝謝大家收看 再見.
\newpage



\section{}
\label{sec:MKd__CpGhKU}
\textbf{Public Lecture: A Culture Of Life In The Dangers Of Our Time (Speaker: Prof. Jürgen Moltmann)}
\newline
\newline
連結: \href{https://youtube.com/watch?v=MKd-_CpGhKU}{\texttt{ https://youtube.com/watch?v=MKd-\_CpGhKU}} ~~~~ 語音日期: 2018-06-28 
\newline
\newline
\hyperref[sec:B1NjObnaFkg]{\small{< < < PREV SERMON < < <}}
~
\hyperref[sec:index]{\small{[返主目錄]}}
~
\hyperref[sec:Ps_hGAmvVcA]{\small{> > > NEXT SERMON > > >}}
\newline
\newline
$^{1}$Good evening, ladies and gentlemen..
My name is Daniel Lee..
I teach systematic theology at CGST here..
It's my great privilege to introduce our honored speaker,.
Professor Nguyen Van Mugman tonight..
Actually, Professor Nguyen Van Mugman.
does not need an introduction..
He has long been distinguished.
as one of the most acclaimed theologians of our time.
since the publication of his celebrated.
Theology of Hope in 1964..
He is the architect of post-Auschwitz theology.
as well as a pioneering voice in political theology,.
ecological theology,.
feminist and Jewish-Christian dialogue..
He demonstrates a life that has dedicated itself.
to seeking a Christian response.
to the turmoil and opportunities of our time..
His publication list is breathtaking in length.
and many of his books are immensely influential..
So just name a few.The Crucified God in 1974..
The Church in the Power of the Holy Spirit,1977..
The Trinity and the Kingdom of God in 1981..
God in Creation in 1985..
The Way of Jesus Christ in 1990..
The Spirit of Life in 1992..
The Coming of God in 1996..
Experiences in Theology in 1999..
Ethics of Hope,2010..
Professor Mugman truly has great delight.
and vitality in writing..
We've been told that he's going to publish.
a new book on Christian patients this year..
So in his profound theological works,.
there has been a great passion for life..
This reminds me of a beautiful prayer.
I have read in Professor Mugman's autobiography..
It serves a quotation here..
Quote, "When I love God,.
I love the beauty of bodies,.

$^{41}$the rhythm of movements,.
the shining of eyes,.
the embraces,.
the feelings,.
the sense,.
the sense of all these protein creation..
When I love you, my God,.
I want to embrace it all,.
for I love you with all my senses.
in the creations of your love..
In all the things that encounter me,.
you are waiting for me.".
The experience of God.
deepens the experiences of life..
It does not reduce them,.
for they are ways.
the unconditional yes to life..
The more I love God,.
the more gladly I exist..
The more immediately and wholly I exist,.
the more I sense the living God,.
the inexhaustible source of life.
and eternal livingness.".
End quote..
Such a great passion for life.
is why we are so thankful.
to have him in our midst tonight.
and to listen to his insights.
on the topic,.
a culture of life.
in the dangers of our time..
So please join me.
and offer our warm welcome.
to Professor Yungel Mokhmar..
Thank you, Professor Daniel Lee,.
for your kind words of introduction.
and of welcome..
In this lecture,.
ladies and gentlemen,.
I would like to introduce.

$^{81}$the famous professor,.
Professor Daniel Lee..
Ladies and gentlemen,.
I grapple with what have been.
my most urgent concerns.
for some time,.
a culture of life.
stronger than the terror of death,.
a love for life.
that overcomes.
the destructive forces.
in our world today.
and a confidence.
in the future.
that overcomes anxiety.
and fatalism..
These issues are for me.
most urgent.
because with a poet,.
Friedrich Hölderlin,.
I believe strongly.
that where there is danger,.
salvation also grows..
We should inquire.
whether it,.
to what extent,.
this hope will wait.
as we explore the possibilities.
of a culture of life.
and of a future..
In the face of the real annihilation.
with which our world is threatened,.
I will begin by addressing.
some of the dangers of our time.
in part one..
And in part two,.
I will offer some answers.
by considering dimensions.
of the world.
capable of supporting life.

$^{121}$and in a quite literal sense,.
a world that is worth of our love..
And in the end,.
I'll return to the first verse.
of the poem by Hölderlin..
Life is God,.
but difficult to grasp..
Part one..
The terror of universal death..
Number one..
Life is today in danger..
It is not in danger.
because it is mortal..
Human life has always been mortal..
It is in danger.
because it is no longer loved,.
in respected and affirmed and accepted..
The French author Albert Camus.
wrote after World War II,.
This is the mystery of Europe..
Life is no longer loved..
The 20th century.
was a century of mass exterminations.
and mass executions..
State terror from above..
The beginning of the 20th,.
21st century.
saw private terror from below.
of senseless killings.
with suicide assassins..
In the terror rest of the 21st century,.
a new religion of death is confronting us..
I do not mean the religion of Islam,.
but rather the ideology of terror..
Your young people love life,.
said the late Mullah Omar of the Taliban in Afghanistan..
Our young people love death..
After the mass murder in Madrid on March 11, 2004,.
there were acknowledgments by the terrorists.
with the same message..

$^{161}$You love life, we love death..
A German who joined the Taliban in Afghanistan.
declared, "We don't want to win..
We want to kill and be killed.".
Why? I think because they view killing as power.
and they experience themselves as gods over their enemies.
and they love publicity and this they get..
This seems to be the modern terrorist ideology.
of suicide assassins..
I remember we had this love of death in Europe.
some 70 or 80 years ago..
"Viva la muerte" cried an old fascist general.
in the Spanish Civil War..
Long live death..
And the German SS troops in the Second World War.
had the saying, "Death gives and death takes away.".
And wore the symbol of the skull and the bones..
It is not possible to deter suicide assassins.
for they have broken the fear of death..
They don't love life anymore..
They want to die with their victims..
Second, behind this terrorist ideological surface.
a greater danger is hidden..
Peace, disarmament, and non-proliferation.
treaties between the nations.
share an obvious assumption.
namely that on both sides there is a will to survive.
and a will to live..
Yet what happens if one partner.
does not want to survive.
but is willing to die is through death..
That partner can destroy this whole wicked.
or godless world..
Until now we have had to deal only.
with an international network of suicide assassins.
overcome by a death wish..
What happens when a nation possessing nuclear weapons.
becomes obsessed with this religion of death.
and turns into a collective suicide assassin.
against the rest of the human world.

$^{201}$because it is driven into a corner.
and gives up all hope..
Deterrence works only so long.
as all partners have the will to live.
and want to survive..
When it is of no matter whether one lives or dies.
one has lost the fear that is necessary for deterrence..
Whoever is convinced for ideological or religious reasons.
that he or she must become a sacrifice.
in order to save the world.
can no longer be threatened with death..
The one who clamors for the great war or the end war.
even if it means one's own destruction.
is beyond deterrence..
The attraction of destroying a world.
that is considered rotten or disordered or godless.
can obviously grow into a universal death wish.
to which one sacrifices one's own life..
Death then becomes this fascinating divinity.
inflaming a desire for destruction..
This apocalyptic religion of death.
is a real enemy of the will to live.
the love of life and the affirmation of being..
Number three, the suicide program..
Behind this present political danger.
endangering the common life of the nations.
there is still an older threat lurking.
the nuclear threat..
The first atomic bomb dropped on Hiroshima.
in August 1945.
brought the world war to an end..
At the same time is marked the beginning.
of the end time for the whole of humankind..
The end time is in the age.
in which the end of humankind is possible at any moment..
No human being could survive the nuclear winter.
that would follow a great atomic war..
Remember humankind was at the cusp.
of such a great atomic war.
for more than 40 years.

$^{241}$during the cold war time..
It is true that since the end of the cold war in 1990.
the great atomic war is not as likely..
We live in relative peace..
Yet there are so many atomic and hydrogen bombs.
stored up in the arsenals of the great nations.
and some smaller ones as well.
that the self annihilation of humankind.
remains a distinct possibility..
The Russian atomic scientist Sakharov.
called it collective suicide of humankind..
Whoever fires first dies second..
For those 40 years.
we depended on mutual destruction for security..
Most people had forgotten this atomic threat.
until President Barack Obama in 2009 speech.
lived in Prague.
revived the old dream of the world free of atomic bombs.
and started new disarmament negotiations with Russia..
But in 2017 Donald Trump threatened North Korea.
with fury and fire with atomic bombs..
Then many of us became aware again of this destiny.
hanging like a dark cloud over the nations..
Strangely enough we feel the presence of the nuclear threat.
publicly in what American psychologists call nuclear numbing..
We repress our anxiety.
try to forget this threat.
and live if this danger were not there..
Yet it is gnawing at our subconsciousness.
in impairing our love of life..
Number four, the social conditions of misery..
A general impairment of life also exists.
in the miserable social conditions..
For more than 40 years we have heard repeatedly.
and everywhere the charge that despite of all political efforts.
the social gap between the rich and the poor is widening..
It is not just in the poorer countries of the third world.
that a small rich sector of the population.
rules over the masses of the poor..
In the democracies of the developed world.

$^{281}$the financial asset gap between financiers on the one hand.
and low income workers, welfare recipients, the unemployed.
and those not able to work on the other hand takes on obscene proportions..
Yet democracy is grounded not only in the freedom of the citizens.
but also in the equality without social justice in life opportunities..
In the comparability of life circumstances.
the common will dies and with it what holds a society together falls apart..
Trust is lost..
Since the democratic revolutions in England, the United States and France.
the political task in the European states.
has been the balancing of individual freedom and social equity..
The deregulation of the economy and fiscal institutions.
wrought by American politics with all its destructive consequences.
has led to an imbalance between freedom and equality.
that has become life threatening for many people..
It has led to the disempowerment and poverty..
A capitalism that is no longer politically controllable.
through the common wealth becomes an enemy of democracy.
because it destroys the common meaning of a society..
Climbing on the social ladder brings anxiety..
In the modern competitive society.
the loser creates the force of the winner..
The winner creates the anxiety of life..
The anxiety of life creates nothing.
but the anxiety of existence for modern human beings..
Yet is anxiety a good incentive for life, for work and for happiness?.
I doubt..
At the end, the ecological conditions of world destruction.
unlike the nuclear threat, climate change is not only a threat.
but already an emerging reality everywhere..
It is not only a latent problem.
but also very much a matter of public consciousness..
People know it because they can see it, feel it and sometimes smell it..
The biosphere of the planet Earth is the only space we have for life..
The globalization of the human civilization has reached its limits.
and is beginning to alter the conditions of life on Earth..
The destruction of the environment.
that we are causing through our present global economic system.
will undoubtedly, seriously jeopardize the survival of humanity.
in the 21st century..

$^{321}$Modern industrial society is thrown out of the balance,.
the equilibrium of the Earth organism.
and is on the way to the universal ecological death..
Unless we can change the way things are developing..
Year after year, vulnerable species of animals and plants die out..
Scientists have shown that certain chemical emissions are destroying the ozone layer.
while the use of chemical fertilizers and a multitude of pesticides.
is polluting our drinking water and making the soil infertile..
They have shown that the global climate is already changing.
so that we are now experiencing an increasing number of.
so-called natural catastrophes.
such as droughts and floods, expanding deserts and incessant storms..
These are not simply natural but also caused by human activities..
All in all, life on Earth is under threat..
Why is this so?.
This is some irony, I say, some do not know what they are doing.
while others do not know what they are knowing..
This ecological crisis is fundamentally a crisis.
wrought by the Western scientific and technological civilization..
Yet it is a mistake to think that our environmental problems.
are problems for the industrialized countries of the region..
On the contrary, ecological catastrophes are increasing even more.
in the midst of a rising global economic and social crisis in their development..
Indira Gandhi was right when she said poverty is the worst pollution..
Despite the well-known document of limits of growth,.
the ideology of permanent growth continues..
We know all this, but we are paralyzed.
and do not change our economy or our lifestyle..
This paralysis will be called ecological numbing..
Nothing accelerates an imminent catastrophe.
so much as the paralysis of doing nothing..
We do not know whether humanity will survive this self-made destiny..
This is actually a good thing..
If we knew with certainty that we would not survive,.
we would do nothing..
If we knew with certainty that we will survive,.
we would also do nothing..
Only if the future is open for both possibilities.
are we forced to do today what is necessary to survive tomorrow..
We cannot know whether humankind will survive,.

$^{361}$so we must act today as if the future of life depends on us.
and trust at the same time that our children will survive..
But must a human race exist and survive.
or are we just an accident of nature?.
We can ask cynically,.
didn't the dinosaurs come and go?.
So it ends with the question of existence,.
whether humanity should be or not be..
That is a humbled question of our days..
More than 8 billion human beings already live on earth today..
This number will likely grow rapidly..
An alternative for the future.
is that the earth could be uninhabited..
The earth existed without human beings.
for more than millions of years.
and may survive perhaps for millions of years.
after the human race disappears..
This raises the even deeper question,.
are we human beings on earth only by chance.
or are we human beings a necessary result of the evolution?.
If nature would show a strong anthropic principle,.
we could feel at home in the universe with Stuart Kaufman..
If such a stark anthropological principle cannot be proved,.
the universe gives no answer to this existential question of humankind..
Looking to the universe for an answer to the question of a reason for being,.
we encounter the sad conjecture of Steven Weinberg..
The more the universe seems comprehensible,.
the more it also seems pointless..
The silence of the world's expanses.
and the coldness of the universe can lead to our despondence..
In any case,neither the stars nor our genes.
say whether human beings should be there or not..
How can we love life and affirm our being as humans.
if humanity is only an accident of nature,.
so perfidious and without relevance for the universe,.
but is only a mistake of the evolution?.
Is there a duty to be as Hans Jonas claimed?.
Is there any reason to love life and affirm human beings.
if we find no answer to this existential question.
every culture of life is uncertain in its fundamentals.

$^{401}$and built on shaky grounds?.
With this I start part two..
A culture of life must be a culture of common life.
in the human and the natural world..
First point,can we live with a bomb?.
Are the dangers growing faster.
than what can save us?.
I think we can grow in wisdom,but how?.
President Obama's dream of a world without atomic weapons.
is an honorable one,but only a dream..
Human beings will never again become incapable.
of what they can do now..
Whoever has learned the formula of atomic fission.
will never forget..
Since Hiroshima in 1945,.
human kind has lost its atomic innocence..
Yet the atomic end time.
is also the first common age of the nations..
All the nations are sitting in the same boat..
We all share the same threat..
Everyone can become the victim..
In this new situation,.
human kind must organize itself.
as the subject of common survival..
The foundation of the United Nations in 1945.
was a first step..
International security partnership.
can serve peace and give us time to live..
And someday,.
perhaps a transnational unification of humankind.
will keep the means of nuclear destruction under control..
By science and technology,.
we learned to gain power over nature..
But by wisdom,.
we learned to gain moral control of our powers..
The development of public and political wisdom.
is as important as the scientific progress..
The first lesson we learned is this..
Deterrence does not secure peace anymore..
Only justice serves peace between the nations..

$^{441}$There is no way to peace in our world.
except through just action.
and harmonious balance of interests..
Peace is not the absence of violence,.
but the presence of justice..
Peace is a process,.
not a property of one nation..
Peace is a common way of reducing violence.
and constructing justice.
in the social and global relationships of humankind..
Point two..
Social justice creates social peace..
The gap between the poor and the rich widens..
But the alternative to poverty is not poverty..
The alternative to poverty and poverty is community..
One can live in poverty.
when it is born in common with others.
as was the case in Europe.
in the years of hunger after the World War II..
It is injustice that makes poverty insufferable..
The spirit of communal solidarity.
and mutual help was demolished.
by the flight from Texas.
which in turn aroused the anger of the people..
If everyone is in the same situation.
then all give mutual help..
Remove equality because one wins and the other loses.
then mutual help also vanishes..
By community I here mean.
the visible community of solidarity.
as well as the inner togetherness of society.
in social balance and social freedom..
It is not in the end the football games.
that unite a society..
It is social justice.
that creates lasting social peace..
The individualism that says.
everyone is his or her own neighbor.
looking out for himself and herself.
makes the human beings powerless..

$^{481}$The fragmenting of work.
by making it temporarily insecure.
and without benefits.
harms the life planning of those.
at the mercy of the system.
and destroys their future..
In communities of solidarity.
human beings are strong and wealthy.
that is wealthy in relationships.
with neighbors and friends.
companions and colleagues.
of which one can depend..
They are thus made strong.
by being recognized.
and by being esteemed as worthy..
Many helpful actions.
emerge in such communities.
as child care, care of the sick.
and care of the aged, associations.
of the handicapped and the health peace movement..
Market position and competition.
are certainly strong incentives for work.
but they remain humane.
only in the framework of a common life.
and that means only in the bounds.
of social and ecological justice..
There are dimensions of life.
that may not be determined.
by the market logic.
because they follow other laws..
Patients are not customers of doctors and nurses.
and students are not consumers of science.
and research in our universities..
Security must not be a private property.
and the reverence for life.
because human society.
is a natural environment.
for a total life system.
where there is a crisis of dying in nature.
a crisis of the whole life system.

$^{521}$emerges as well..
We call today the ecological crisis.
is not only a crisis in our environment.
but also a total crisis of our life system.
and cannot be solved by technological means alone..
It also demands a change in our lifestyle.
and a change in the basic values.
and convictions of our society..
Modern industrial societies.
are no longer in harmony.
with the cycles and rhythms of the earth.
as was the case in pre-modern agrarian societies..
Modern societies are predicated on progress.
and expansion of the projects of humanity..
We reduce the nature of the earth.
to our environment.
and destroy the life space of other forms of life..
Nothing works so much destruction.
as with using nature to more than an environment for humans..
We need a change from modern domination of nature.
to the reverence for life.
as Albert Schweitzer teaches us..
Reverence for life is respect for every single form of life.
and for our common life in the human and the natural world.
and for the great community of all the living..
The post-modern life centrism.
will have to replace the western and modern.
anthropocentrism of culture..
Of course we cannot return to the cosmos orientation.
of the ancient and pre-modern agrarian world.
but we can begin the necessary.
ecological transformation of our industrial world..
For this we must change our concept of time..
The linear concept of progress.
in production, consumption and waste.
has the future, the present and the past.
must give way to the concept of the cyclical time.
of renewable energy.
and the recycling economy..
Only this cycles of life.

$^{561}$can give stability to our world of progress..
Yet as long as the children of Ghana.
bear the burden of recycling.
or electronic scrap.
we must say that the recycling economy.
is still the economy of the poor people..
The 2000s charter points in the right direction..
Humanity is part of nature.
or other life forms of nature.
have their worth independent of their worth.
for human beings..
We are part of nature.
and can only survive.
by preserving nature's integrity..
The last part, the love of life in times of danger..
Human being is not only a gift of life.
but also a task of being human..
To accept this task of humanity in times of terror.
requires the strength of life.
and the courage to be..
Life must be a front against terror.
to say simply.
life must be lived.
and then the beloved life.
will be stronger.
than the threat of universal annihilation..
I see three major factors.
for this courage to be.
and this courage to live..
First, human life must be affirmed.
because it can also be denied..
As we know, a child can only grow.
and live in an atmosphere of affirmation.
and an atmosphere of rejection.
that child will fade away in soul and body..
If experiencing affirmation.
is the occasion for a child.
to affirm himself or herself..
What is true for a child.
is true for human beings.

$^{601}$throughout their lives..
We are accepted, appreciated and affirmed..
We are motivated to live.
where we feel a hostile world of contempt.
and rejection and mistrust..
We retire into ourselves and become defensive..
We need a strong affirmation of life.
that can deal with such negations of life..
Each year to life is stronger.
than every negation of life.
because it can create something new.
against the negations..
Second, human life is participation..
We become alive.
where we feel the sympathy of others.
and we stay alive.
where we share our life with others..
As long as we are interested, we are alive..
The counterproof can be easy to make..
Indifference leads to apathy.
and apathy is a sickness unto death..
Complete lack of participation.
is completely unlived life..
Third, human life is a life.
in the pursuit of fulfillment..
Human life gains its dynamic.
from this inborn striving..
The pursuit of happiness.
is since the writing of the Declaration.
of Independence of America.
one essential human right.
to pursue one's happiness.
is not only a private human right.
but also a public human right..
We speak of a good life.
and we mean a life.
that lives out its best potential.
in the public life of a good society..
When we take this pursuit of happiness seriously.
we encounter the misfortune.

$^{641}$of the masses of poor and sick people.
and to begin to suffer with the unfortunate..
The compassion by which we are drawn.
into their passion of life.
is the reverse side of the pursuit of happiness..
The more we become capable of the happiness of life.
the more we become also capable of sorrow and compassion..
This is the great dialectic of human life..
But where there is danger, salvation also grows..
How is salvation growing?.
I've tried to show how being can take in non-being.
and how life can overcome death through love.
and how deadly contradictions can change into a life..
And how death can change into a protective differences.
and higher forms of living and community..
Near is God and difficult to grasp..
Here at the end I will allow the theologian.
with declarations of Christian faith..
Should humanity be or I be superfluous?.
Is there a duty to survive or our life and death.
simply take or leave it matter?.
In the evolution of life,.
are we an accident or a mistake of life?.
The existential questions of humankind.
are not only answered by rational arguments..
But first of all by pre-rational assurance.
or a lack of assurance that leads the interest of our reason..
Difficult to grasp is God..
Not because God is distant from us human beings.
but rather near and therefore difficult to grasp..
What is near indeed nearer to us than we ourselves.
is not to be grasped by us..
We would need distance for that..
If we were however grasped by the nearness of God.
we would know the answer to our existential questions..
In the eternal yes of the living God.
we affirm our fragile and vulnerable humanity.
in spite of death and say yes to life..
In the eternal love of God.
we love life and resist its devastations..

$^{681}$In the ungraspable nearness of God.
we trust in what is saving even if dangers are growing..
Here ends the lecture..
(掌声).
謝謝大家.
\newpage



\section{}
\label{sec:Ps_hGAmvVcA}
\textbf{The Crisis of American (White) Evangelicalism (Day 1)}
\newline
\newline
連結: \href{https://youtube.com/watch?v=Ps_hGAmvVcA}{\texttt{ https://youtube.com/watch?v=Ps\_hGAmvVcA}} ~~~~ 語音日期: 2021-01-21 
\newline
\newline
\hyperref[sec:MKd__CpGhKU]{\small{< < < PREV SERMON < < <}}
~
\hyperref[sec:index]{\small{[返主目錄]}}
~
\hyperref[sec:F73AP9vE2KM]{\small{> > > NEXT SERMON > > >}}
\newline
\newline
$^{1}$(國語).
Ladies and gentlemen,.
Christian brothers and sisters,.
warmest greetings in our Lord Jesus Christ to you,.
no matter where you are now geographically located..
Welcome to join us.
at our China Graduate School of Theology.
Josephine So Culture and Ethics Lecture Week..
My name is Simon Cheung,.
your host this morning..
Let me first say a few words about this lecture series..
Instituted in 1985,.
this biannual lecture series.
is for commemorating Josephine So,.
the pioneering Christian thinker and leader.
here in Hong Kong.
after her untimely death in 1982..
In the past,.
this lecture series has covered.
a broad swath of topics,.
ranging from a contemporary reading of.
Augustine's "The City of God".
to pressing cultural and ethical issues.
concerning creation care,.
marketplace theology,.
or contextualizing Christian faith.
in the current Chinese context..
This year,.
our lecture series is set to take a look,.
and I may add,.
a scrutinizing and critical look.
into the American white evangelicalism.
and the crisis it faces today..
And our distinguished speaker is.
Reverend Dr. Mark Levitin,.
the President of Fuller Theological Seminary.
in the U.S..
Dr. Levitin,.
it's great to have you here virtually with us..
Would you like to say hi to our audience here?.

$^{41}$Thank you so much for this invitation..
Thank you, Dr. Levitin..
And before we hear your first lecture today,.
may I first say a few words.
to introduce you to our general audience?.
Dr. Levitin is a prolific writer,.
and his work appears in various kinds of platforms,.
with the Lord Jesus Christ as a center of passion,.
as Dr. Levitin once expressed in an interview..
His work is aimed to wake people up.
as to what kind of life and practice.
they are called into by their Savior and by their Lord..
Just by reading the evocative titles of his books like.
"The Dangerous Act of Worship".
and "The Dangerous Act of Loving Your Neighbor,".
his readers will immediately get the feel that.
the complacent and insipid understanding of our faith.
falls far short of the true calling.
to be the followers of Jesus..
Out of a deep conviction in the practice of genuine dialogue,.
Dr. Levitin has been hosting,.
as the president of FOLA,.
a forum called "Conversing,".
in which he will speak with a wide spectrum of leaders,.
I quote,.
on issues at the intersection of theology and culture,.
end quote..
This dialogue is meant,I quote again,.
to evoke both the power of conversation.
and the turbulence of our times,end quote..
So one of his latest additions features his interview.
with a Christian writer and activist.
who champions racial justice.
and seeks to expose the Christian,.
especially white Christians' complicity.
in racism in the U.S..
Today,Dr. Levitin will begin his reflection.
of the issue at hand with the title.
"The Start, the Complicity, the Tsunami and Today,.
the Birth, Original Scene, Development, Growth,.

$^{81}$Consumption and Betrayal.
of American White Evangelicalism.".
Without further ado,let's welcome Dr. Levitin..
Thank you very, very much..
It's a great joy and privilege to be here..
Presently, distinguished faculty, students and guests,.
it's a great honor for me to be participating.
in these lectures in 2021..
We had, of course, all hoped that the lectures.
would be delivered in person,.
but 2020 and the COVID pandemic.
regrettably made that impossible..
I hope that the Canadian philosopher.
Marshall McLuhan is wrong,.
that this medium will not be the message.
and that you will remember far more.
about what I'm trying to say.
than the medium by which I will be saying it..
I bring all of you very warm greetings.
from Fuller Theological Seminary,.
especially from our board of trustees and faculty,.
and pray for you, for the church.
and for the world of which we are a part.
and for which we serve as we do..
Before formally starting,.
let's let first things be first..
In Romans 8, the Apostle Paul traces.
the pervasive, unfinished and agonizing groans of God,.
the church and creation itself..
In these lectures, there will be cause for joy,.
but there will also be evidences of groaning,.
reasons for arguing, causes for crying,.
repenting and lamenting..
Yet in the face of all that is wrong and incomplete,.
let us first hear the Apostle Paul's words.
toward the end of that well-known chapter..
He writes, "What then are we to say to these things?.
If God is for us, who can be against us?.
He who did not spare his own Son.
but gave him up for us all,.

$^{121}$how will he not also with him.
graciously give us all things?.
In all these things,.
we are more than conquerors.
through him who loved us..
For I am sure that neither death nor life.
nor angels nor rulers,.
nor things present nor things to come,.
nor power nor height nor depth.
nor anything else in all creation.
will be able to separate us.
from the love of God in Christ Jesus our Lord.".
I offer these lectures as those with hope,.
a hope that is gratefully far beyond ourselves.
or these days..
In the name of the Father and the Son.
and the Holy Spirit. Amen..
Shall we pray together?.
Oh God, how thankful we are.
for the privilege and gift of gathering in this way..
We are in places throughout parts of Asia.
and perhaps around the world,.
aware of your presence and faithfulness,.
which is our only hope,.
the only hope of the Church.
and the only and final hope of the world..
May our days together this week.
be ones that you will bless,.
that our conversation and my reflections.
will combine in useful ways,.
that our understanding of your will.
and the future to which you call us.
will be clearer and that our faithfulness will grow..
I ask it to your glory in the name of Jesus..
Amen..
To speak about the crisis.
of American white evangelicalism.
is to speak about a crisis.
within just part of the body of Christ.
and its mission in the world..

$^{161}$As I hope and know you will see,.
I have no difficulty trying to honestly describe.
my understanding of the situation,.
but in doing so,.
I want to remind myself and you.
that I am after all talking about.
part of our Lord's beloved bride..
The journey of God with God's people.
from Abraham to today.
has been endlessly fraught with crises.
and still our ever-loving.
and persisting Good Shepherd.
does not let us go..
I believe in the legitimacy.
and in fact the necessity of self-critique,.
while conscious that that can easily.
and glibly fall prey.
to its own arrogance and presumption..
So I want genuinely to repent even in advance.
and commit to you that I will try.
to be both candid and loving,.
honest and fair..
The topic is large enoughand complex enough.
to necessarily state that I'm not addressing.
all dimensions of evangelicalism,of course,.
and certainly not addressing the related.
but very different realities.
of global evangelicalism..
Nor would I pretend to adequately trace.
therefore the impact and importance.
of American white evangelicalism.
on the China Graduate School of Theology.
or the Chinese Church..
Instead,I will offer you my thoughts.
about American white evangelicalism.
and ask during our question and answer sessions.
for your partnership.
as brothers and sisters in Christ,.
with whom the Scriptures call us.
to both rejoice with those who rejoice.

$^{201}$and weep with those who weep..
Your reflections,critiques,.
and perhaps even weeping.
are truly valuable to me,.
and I look forward to hearing.
your voices and perspectives as well..
We are in one communion in Christ..
Presently,encourage me to deliver lectures.
in a way that would include.
my own personal experience.
and not only to offer them.
through the lens of a merely academic.
set of categories or ideas..
I was grateful for this and pleased.
since it allows me to do in public.
what we all know to be true in private,.
namely that almost all scholarship.
is also narrative or narratival..
That is,our commitments and interests.
are inevitably oriented by.
and reflective of our own stories..
A few examples..
I point to Dr.Francis Collins,.
the director in the United States.
of the National Institutes of Health.
in Washington, D.C.,.
and the leader of the Human Genome Project,.
which worked out the first.
complete sequencing of human DNA,.
forever changing the way.
human biological science would proceed.
and affecting the lives of people everywhere..
It was he who said.
that the underlying conviction.
of God's creative and recreative love.
and salvation in Jesus Christ.
is the daily rock on which he seeks.
to do the very best science.
and public health that he can provide..
Dr.Collins can distinguish science from faith,.

$^{241}$but his commitment to science.
and to the evangel.
are together part and parcel.
of one life and one project..
Dr.Martin Luther King,.
the great American civil rights leader,.
was not incidentally black..
It was his blackness.
that informed his experience and his theology.
and also motivated his rhetoric.
and his analysis of the past and present..
His personal story made him more capable,.
not less,of perceiving reality clearly.
in his study of American racial history..
His embeddedness in his story.
and in the larger black narrative.
in the United States,.
including his demonstrations.
in some of the great marches.
and movements of the '60s,.
gave him clearer vision, not less..
In a multiracial nation,.
race is never a neutral factor..
Any claim of neutrality about race in America.
is a warning sign.
that misunderstanding is already at hand..
Jesus was not incidentally.
a first-century Jewish man..
The very character of his divine revelation.
and of the love and healing,.
teaching and sacrifice,.
death and resurrection.
are simultaneously objective.
and subjective realities..
This is what we affirm about the incarnation,.
the universal particularity of God..
Postmodernism has succeeded.
in driving us to face.
the inescapable particularity of life..
In the case of Christian faith,.

$^{281}$the particular, however,.
is the lens by which we can see,.
know and encounter the God of the universe..
This is what we hold dear..
This is what we hold together..
Just a moment..
Excuse me..
All of this is to say.
that when President Lee.
invited me to speak personally.
on the subject of.
American white evangelicalism,.
he was allowing me to express in public.
what otherwise is privately.
and inescapably present..
I am allowing,.
I am being allowed through his invitation.
to pixelate the story.
with an acknowledgment of the ways.
my own experience.
has ordered the arrangement.
and colors and shades.
of my telling.
of the American evangelical story..
On a personal level,.
I come to you to this topic.
first and foremost,.
simply as a brother in Christ,.
seeking, as you do,.
to hear and respond.
to what the Holy Spirit.
may want to say to us.
in these sessions together..
The primary context.
out of which I speak.
is what we hold in common,.
our life in Christ..
This defining context.
of God's loving redemption.
holds any and all details.

$^{321}$about me,.
my gifts and weaknesses,.
my sins and sorrows,.
my hopes and longings,.
my social location,.
my birth and my new social location.
as part of God's one new humanity..
The same, of course, is true for you..
We find our life's principal location.
together in this one new communion,.
made possible and active.
as we are held in the love.
of the Father, Son and Spirit..
In this common life,.
we also bring our distinct similarities.
and differences..
I am a tall, white,.
educated American.
born and raised.
on the west coast.
of the United States,.
and I am 67 years old..
I've been married.
for nearly 37 years,.
and my wife and I.
have two adult sons..
My family of origin.
had a low or moderate income.
and lived in.
an agricultural community.
adjacent to a nearby.
Native American community.
and in a place known.
as the Yakama Nation,.
a designation of a.
Native American tribe,.
namely the Yakamas,.
and in a community.
served by itinerant farm workers.
who traveled each season.

$^{361}$to labor in the agricultural harvest.
around us..
My life as a white American.
was not their life..
We lived in proximate, parallel,.
but utterly different worlds..
I grew up in a tight-knit family.
in which my father saw religion.
and religious devotion.
as one of the most dangerous.
influences in the world,.
one that he wanted his two sons.
to avoid if at all possible..
So my Christian journey.
began as I started college,.
where I became a Christian.
through reading the Gospels..
I simply had an intellectual.
and social curiosity.
about the Bible.
and felt that since I was.
on the road to becoming.
I hoped an educated person,.
I should read the Bible,.
and I began with.
the Gospel of Matthew..
My father had committed.
to do all he could.
to keep his two sons from religion.
and especially from religious devotion..
His argument from history.
and experience.
was that religious people,.
particularly Christian people.
in his assessment,.
by the influence of their faith,.
take great things,.
the universe,.
the mystery of being human,.
and turn them into something.

$^{401}$that's actually very small..
My father illustrated this.
through the history of science.
and Christianity,.
but also through social.
and political history,.
as well as moral philosophy.
and practice..
To my great shock,.
what I discovered.
in the New Testament.
was first that Jesus.
shared this same concern.
as my father..
Many of his expressions.
address that very concern,.
and yet Jesus' antidote.
to small-making.
is what he announced.
and displayed to be.
the kingdom of God,.
the reality that cracks.
the whole universe wide open..
This was the gift.
that I was being offered.
through the New Testament,.
the gift that shocked me.
and ultimately has come.
to change my whole life..
My life as a disciple.
to this day.
has been constantly shaped.
by the dialogue.
between my father's anxiety.
about Christian faith,.
which I believe.
was accurate and fair,.
and how it has been.
constructively shaped now.
by the dialogue.

$^{441}$between that anxiety.
and the comprehensive.
and personal character.
of God's loving reign.
in Jesus Christ.
by the Holy Spirit.
through one new humanity.
that we call the church.
for the sake of the whole.
created order..
The first Christians.
I came to know were and are.
some of the most positive.
spiritual influences in my life..
They referred to themselves.
as evangelical,.
and the rest is, well, history..
I stepped into a biblical story.
I had not known.
and into a community.
that became home to me..
My spiritual life and journey.
has been full of God.
and full of a ministry.
of resisting smallness..
Let's go back to the beginning.
and all of these sections.
of this lecture.
will be very, very introductory,.
and I apologize.
if they seem repetitive.
to what you may already know..
By the 17th and 18th centuries,.
the so-called birth of America.
was already by that time.
long since underway..
Only in the last quarter.
of the 18th century.
would it become.
a formally united project,.

$^{481}$but for two centuries before that,.
Europeans and European colonies.
had established.
what became the United States..
Evangelicalism was a spiritual.
and theological renewal movement..
Some would say one.
that emerged from.
the Protestant Reformation.
of the 16th century..
Evangelicalism was both.
a spiritual movement.
and a way of conceiving.
of the Christian life..
It was by no means then a party,.
let alone an organization.
or an institution..
It was instead.
a fresh experience or movement.
of the Holy Spirit.
that cut across denominational lines,.
that spoke out into town squares,.
that brought influence.
uncontrolled by formal.
Christian structures or histories,.
while declaring and nurturing people.
in what they saw as the first order.
matters of faith..
By the last quarter.
of the 17th century,.
the 18th century,.
when the declaration of independence.
and the War of Independence occurred,.
the colonies were clearly hungry.
for change, expansion, freedom and hope..
This, along with various.
Christian awakenings in Europe.
and in America in the 1730s.
and the 1740s.
by groups such as Anglicans.

$^{521}$and Methodists, Reformed and Puritan,.
meant that institutional religious teaching.
and authority were being modified.
and challenged in various ways..
Together it created a readiness.
for a kind of spiritual awakening.
that was motivated by a readiness.
to receive the spiritual identity.
and life that could freshly ground them.
in this new context, in this new land..
The great awakenings in Europe.
came to and triumphed in American soil..
In retrospect,.
a British historian, David Bebbington,.
in his defining work in 1989,.
entitled Evangelicalism in Modern Britain,.
a History of the 1730s to the 1980s,.
set forth that evangelicalism.
was centered on a quadrilateral set of traits,.
that it was Bible-centered,.
that it was cross-centered,.
that it was conversion-expecting.
and that it was activist-oriented..
This movement unfolded with undulations.
that responded to external changes,.
to the presence and practice of slavery,.
for example,.
to the critical study of the scriptures,.
to the anxieties spawned.
during the Enlightenment by science,.
to the realities of world wars,.
to the impact of pandemics,.
to the growing impact of higher education..
It claimed a spiritual, theological.
and experiential legitimacy.
more than an intellectual,.
denominational or formal identity..
It was a movement.
more than it was an institution.
or an organization..

$^{561}$And therefore,.
it was adaptable to the East Coast,.
but it was highly flexible enough.
as the westward expansion.
in the United States occurred..
In the 1920s,.
the fundamentalist-modernist controversy.
drew and debated lines.
that were the fruit of this long history.
that had already been developed.
over that period of time..
By the middle of the 20th century,.
when Fuller Theological Seminary.
and many other evangelical organizations.
and institutions were founded,.
evangelicalism had become a word.
that could serve as a pathway.
between two extremes..
In the case of Fuller Theological Seminary,.
it was a way between fundamentalism.
and a wooden understanding of the Bible.
and its interpretation.
and modernism,.
namely,potentially at least.
an adaptation, if not an abandonment,.
of orthodoxy at times.
for the sake of being "modern.".
Evangelicalism, as Fuller saw it,.
was an embrace of Christian orthodoxy.
that was open to more critical study.
of the scriptures.
and more open to engagement.
with realities around us.
in context and in culture.
than was typically true of fundamentalists.
or wholeheartedly true of modernists..
So evangelicalism quietly and broadly.
remained in this position.
until the late 1970s,.
to which we will comment to the 2000s,.

$^{601}$including the last four years.
in the national and political life.
of the United States..
Please note, again, several things..
The lack of an institutional.
or organizational center,.
the lack of a defining.
or controlling role.
or set of voices.
for what was in fact.
a diffuse and multiply defined.
and growing movement..
The claim to knowledge.
that can and must.
define what is true and good.
was clarified by evangelicalism,.
but not with the emphatic certainty.
of fundamentalism..
It had an orientation.
towards who is in.
and who is out in fundamentalism,.
whereas in growing evangelicalism,.
it was a sense that it was the center.
that was primary.
and not the boundaries..
We'll return to this as well..
We turn next to the complicity..
Here is complicity.
in a number of social realities.
which create a major problem.
for evangelicalism in America..
From the earliest days in North America,.
Native American tribes were being displaced.
by formal and informal versions.
of the so-called doctrine of discovery.
in which lands and peoples.
could be claimed under the authority.
of various European countries..
The majority of this.
was theologically justified.

$^{641}$as part of a sense of the mission of God.
that reigned over all..
The presumption was.
that this was legitimate legally,.
that it was morally right,.
that it was socially beneficial,.
and that it was, most importantly,.
spiritually redemptive.
for those whose lives.
were now held inside.
and under Christian.
and European authority..
The Christian church was manifest.
in a European-defined.
and embodied set of terms..
This meant that there were Roman Catholics.
and various Protestant churches as well,.
Anglican and Puritan in particular,.
that were involved in these matters..
The church in America.
imitated the church in Europe,.
while it also opened the way.
for many who were defining.
religious persecution in Europe.
to be a motivation for immigrants.
to the colonies..
Regional plots along the East Coast.
tagged with different primary European.
and Christian identities..
From this intertwined connection.
between the doctrine of discovery.
and Native Americans,.
that is, the readiness.
to dominate Native peoples,.
there developed an abominable extension..
An intertwined connection.
between white Christians.
and African slavery.
was early and well established.
in the United States..

$^{681}$The doctrine of discovery.
established by the Pope in 1493.
created the legal, political.
and religious claim of land.
and people that were not Christian..
This plain and historic doctrine.
became the driving theological.
and missiological force.
underpinning colonialism.
and its mutating motives.
and expressions to this very day..
What was fundamentally.
a political and economic vision initially.
was justified and defended.
even more passionately.
as a theological one,.
trumping and masking.
the political and economic benefits..
The doctrine of discovery.
provided the theological cover.
and the unlimited possibilities.
for economic expansion.
that were simply too much to resist..
Colonialist condescension.
and rationalization.
looked away from.
and denied the horror.
of the transatlantic slave trade.
and the practice.
that denigrated and dehumanized.
African men, women and children.
by the millions,.
all to satisfy the greed.
of colonial and post-colonial families.
in the name of the gospel..
The unpaid labor of black Africans.
was the workforce.
that built America..
Christians saw the expansion.
of agricultural and manufacturing capacity.

$^{721}$as it was neutrally called.
as the blessing of God..
The reports, as you probably know,.
in West Africa,.
where kidnapped Africans.
were imprisoned.
after being taken in horrific conditions.
before rather being taken.
in horrific conditions.
across the Atlantic Ocean,.
the Middle Passage,.
for sale in the Caribbean.
and in the colonies..
In those ports,.
Christian worship happened.
to ask for God's blessing.
in the so-called Middle Passageway..
One such chapel in Ghana, for example,.
was built across the bridge.
upon which manacled slaves.
would later walk to the place.
where they would be tied and bound.
for the cross-Atlantic trip..
The chapel above this gangplank.
taking slaves on board the ship.
would be the captors,.
the slave masters,.
the ship's captains and crew.
singing hymns still familiar today..
They would read the scriptures..
They would pray for safety.
for all the crew.
and for the trip to be successful.
in the eventual sale.
of fellow human beings..
The church would be the place.
of the servitude.
that would define them.
for the rest of their lives.
and the lives of generations.

$^{761}$and generations of slaves.
and slave families to come..
All this being done.
in the name of Jesus Christ,.
as they would say,.
the Lord and Savior of the world..
Among the perpetrators.
on both sides of the Atlantic.
were white evangelicals..
The convergence of the psychological.
and spiritual readiness of colonials.
in relation to this narrative.
of America's original sin.
is hard to trace..
The ability to claim to love God.
while so obviously hating their neighbors.
was of course not invented.
in the Atlantic slave trade,.
but it was destructively.
and politically practiced.
for over 250 years..
This practice involved.
the death of 2 million lives.
in the Atlantic crossing,.
the deaths of millions.
en route to their plantations,.
and the sale of 12 to 14 million human lives..
These are the ones.
who carried out the nation building..
That was the growing, emerging,.
uniting, and Christian reality.
that became the United States..
In the Constitution of the United States,.
the phrase "we the people".
meant white, land-owning men..
We the people.
are the people that are referred to.
in our Constitution,.
not the slaves..
African slaves were in fact classed.

$^{801}$by our Constitution.
as three-fifths human..
All this brought the United States.
to war in the mid-19th century,.
and the death of more than 650,000 lives.
during that Civil War period..
The war divided families and friends.
and congregations and denominations,.
and continues to this very week.
to define and explain.
the American church.
and political rallies and realities..
The interwovenness for many evangelicals.
of faith and slavery.
now seems like an unbelievable aberration,.
and yet it reveals the ways.
that culture and economics.
speak more loudly.
than our theology and faith,.
and that we are more capable.
than we might imagine.
of wrapping desire with faith.
and calling it God's gift..
The economic labor engine.
that built the United States.
was unpaid labor,.
unpaid labor for 250 years of slavery..
It was claimed to be God's blessing.
for the owner and God's blessing.
for the uncivilized slave and family.
who, because of having been taken captive.
and now "employed" as a slave,.
giving them the opportunity and privilege.
to live in a "Christian nation.".
The cruelty and sin.
that was right in front of them.
was apparently invisible..
In other words, many, by no means all,.
white evangelicals.
were seamlessly in the church community.

$^{841}$and fully complicit in 250 years of slavery,.
followed by 100 more years.
of extended slavery and oppression.
into the assumptions of culture,.
economics, and evangelical faith..
Debates continue to this day.
about the cause of the American Civil War.
from 1861 to '65..
The central matter,.
the main point of concern,.
was, however, slavery..
In Romans 1, Paul talks about.
our human capacity to hold down.
or to suppress the truth.
in unrighteousness..
We do this in our own lives..
And the culture of Christian churches.
and congregations and institutions.
reflects this in various ways..
But for a very large and focal part.
of white evangelical church in America,.
this habit became deep,.
multigenerationally ingrained..
And the white evangelical movement.
has at times challenged.
and at some times completely conceded.
to this kind of vision..
There were, of course,.
in the 18th and 19th century,.
white evangelical public leaders.
who spoke and acted loudly.
and sustainably against slavery..
But the lowest baseline.
associated with and defended.
in Christian terms.
was the defense of slavery..
Evangelical complicity.
with slavery,.
with reconstruction.
following the Civil War,.

$^{881}$with imprisoned and indentured labor,.
with Jim Crow,.
with bank control of real estate and money.
erupted in periodic riots and protests..
As an example,.
100 years ago this very year.
in Tulsa, Oklahoma,.
in the heart of southern America.
and south, southern American.
white evangelicalism,.
the most thriving black community.
in the whole country.
was decimated.
and 300 black lives were slaughtered.
because a black man was alleged.
to try and kiss a white Christian woman..
The most economically thriving black community.
was burned to the ground,.
wiping out an entire community.
who confessed that Jesus Christ is Lord.
Only this year will this be fully faced,.
lamented and recognized..
In different but parallel ways,.
this black narrative.
is not only the Native American story,.
but also in the United States,.
the Latin American story.
and the Asian story as well..
In all these instances.
and many others besides,.
many in the evangelical church,.
confessing Christ.
were also participating,.
supportive or passive.
about these abuses,.
cruelties and injustices..
Being perpetrated by those.
our Bible calls.
are made in the image of God,.
knit together in their mother's womb,.

$^{921}$wounds, wounds,.
fearfully and wonderfully made..
Complicity is too mild a word..
It showed that many evangelical Christians.
had enslaved themselves.
to the injustice and abuse.
of fellow human beings for centuries..
People ask,.
where was the church in Nazi Germany?.
The answer is the same place.
that many white evangelical Christians.
in North America have been.
during the past 400 years,.
right here,.
right in the midst of mainstream culture..
This needs to be kept in mind,.
I might add as a footnote,.
in understanding the events.
of the United States in the last week..
We move next to the tsunami..
The civil rights movement in America.
was primarily led by.
the black church in America..
It has evangelical theology and doctrine,.
but did not choose for the most part.
to identify as evangelical.
because they were not white,.
and because the white evangelical narrative.
excluded black and brown people..
The narrative the black church knew.
was the God who knew.
how to set his people free,.
and that narrative,.
led by exceptional black Christian leaders.
and pastors,.
was the narrative.
to which their civil rights movement.
was committed..
Enormous and positive changes occurred..
But in symbol and in reality,.

$^{961}$the murder of the Reverend.
Dr. Martin Luther King Jr..
slowed and diverted the narrative,.
along with the sacrifice.
of many others as well..
For evangelicals,.
King was too liberal,.
and only very gradually,.
as the black Christian critique.
of the white story of America.
became domesticated enough.
to accept Dr. King.
into some white Christian.
evangelical narratives,.
but only some,.
and only relatively recently..
By the late 1960s,.
the election of Jimmy Carter.
as a "born-again" Christian,.
evangelicalism began to quickly.
step into the public light.
in a different way than before..
By the 1980s,.
we have the birth of a movement.
called the Moral Majority.
and the rise of various media ministries.
that can propagate arguments of faith.
in continuous feeds of ideas..
Simultaneously,.
cable television news channels.
are developing and growing.
in passive popularity,.
and this development mixes together.
in a thick media presence.
of culture and religion..
It's all presented as real,.
but what emerges is closer to reality TV,.
that is,.
not so much an expression of Christian faith.
as much as a combination.

$^{1001}$of entertainment and production..
Audience ratings.
are the most important measure of success..
This becomes a globally spreading.
reality and infection..
The experience of popular.
cable religious programs,.
most of them fundamentalist.
and Pentecostal,.
hosted by many different TV evangelists,.
turned to the political sphere in the '80s.
and unleashed a tsunami.
that flooded the religious airwaves..
This went on for some time.
during the '80s and early '90s.
and increasingly addressed politics,.
especially related to abortion.
and sexuality debates.
and claimed an enormous.
both American and global audience..
In 1996,.
Fox News was established.
by Robert Murdoch,Rupert Murdoch,.
and further amplified the voices.
of popular fundamentalist.
conservative pastors,.
TV evangelists,.
organizations,.
churches and political groups..
Along the way,.
these became known.
as the evangelicals by the media..
It was a title claimed by those.
that the media amplified.
and made more culturally acceptable.
than the tag that had been used.
as fundamentalist or Pentecostals.
who were seen more to be.
at the margins of culture,.
whereas evangelicals.

$^{1041}$became a more acceptable.
mainstream term..
This was the tsunami.
that took over the term.
evangelical in the United States,.
and it remained so.
through the late '90s.
and certainly during the first.
two decades of the 2000s..
This all happened.
as the growth of megachurches.
in the United States also expanded.
from a few hundred.
to thousands of such congregations..
The pervasive.
lack or minimalization.
of theological education.
in most of these movements.
was quite appalling..
It created a large collection.
of communicators.
who were not marked.
by careful thought.
or by accountability..
It became a tribe of its own.
with its own regard.
and disregard for boundaries..
Effect, popularity, emotion.
all dominated.
with a popular,.
unselfcritical frame and attitude..
Typically, a single personality.
in the four ways.
dominated the message..
All this is too simplistic..
Two parallel narratives.
or realities also need to be acknowledged..
First, the media prominence.
of white evangelicalism.
was similar,.

$^{1081}$but always distinct.
from the rise and growth.
and development of the black church..
Some, if not all,.
of the same theological markers.
describe the rise of the black church,.
but the social associations.
do not, quite the opposite, in fact..
The white evangelical church.
and the black church.
were distinct, separate.
and had very different characteristics..
The most important one.
might be assumed.
to be integration of faith.
and context.
of the love of God.
and the justice of God.
in Jesus Christ.
and the suffering of blacks,.
including black Christians,.
who suffered from the brutal racism.
and the white supremacy.
and normativity.
of the American landscape..
More on this in our next lecture..
The second layer.
that needs to be acknowledged.
is a smaller reactive movement.
of neo-evangelicals.
who specifically and stridently.
rejected the sociology, policy.
and politics.
of conservative, republican, white.
American evangelicals..
This cross-current is essential.
to understand the state of affairs.
today in the U.S. evangelical landscape..
The rabid, condemning response.
of conservative evangelicals.

$^{1121}$to the HIV/AIDS crisis.
in the '80s and '90s.
added another layer of judgmentalism.
and cultural division and separation,.
all justified by, quote,.
the good news of the gospel, unquote..
The crisis of today.
has a long development.
that precedes it..
So finally, today..
Today, American white evangelicalism.
is publicly and privately.
in a shambles and in confusion..
Publicly and broadly,.
the identification of evangelicalism.
with Trumpism.
and the loss of the 2020 election.
was a resounding repudiation.
of American white evangelicalism.
by the broad American public,.
not least by African Americans,.
but also by many whites..
Many up to this point,.
evangelicals were silent.
or deflective of the racial issues.
and associated with the appointment.
under Trump of conservative judges.
at all levels.
and the dismantling.
of various social safeguards,.
the aggressive politics.
and policies of America First,.
the anti-immigration and anti-LGBT,.
anti-abortion, anti-racial justice.
and anti-science stances.
that have all been part.
of the Trump administration..
And then the post-election unwillingness.
since November to acknowledge.
Biden's defeat of Trump..

$^{1161}$Over 73 million Americans.
who voted for Trump.
in this last election.
included still 80%.
white evangelical voters..
The summer of 2020.
before the election.
will go down as one of the most prominent.
and provocative seasons.
of racial demonstrations.
in the United States to date,.
triggered most directly.
by the murder of a string of.
and a string of murders.
of unarmed African Americans.
and especially the killing.
of George Floyd by white policemen..
Somewhere between 15 and 25 million people.
were active in demonstrations.
over this past summer.
in cities and towns across the country,.
people of all races and ages.
and political affiliations..
Virtually nothing was said or done.
by many that the press turned to.
and called evangelical leaders..
The height of offense to evangelicals.
should have been taken.
when Trump aggressively dismissed.
protesters for a photo opportunity.
in front of a church in Washington, D.C..
in order to hold up there.
an upside-down Bible..
But instead, Trump's actions.
were heralded by prominent.
white evangelical voices..
The political and public rejection.
of science in response.
to the coronavirus pandemic.
added a third layer.

$^{1201}$to the problems of this moment,.
more fuel yet to the fire.
that put the voices of political.
and evangelical forces.
on the side of being,.
frankly, reality-denying..
Denial of racism and white supremacy,.
denial of the coronavirus.
and trust in science,.
denial of the election.
and its outcome,.
all has many evangelical supporters.
in common..
Typically when these denials.
are offered as an expression.
of Christian faith,.
then it appears to the wider.
American culture.
and to many Christians.
and non-Christians.
inside the United States.
that it appears that evangelicalism.
has less and less to do.
with ordinary reality,.
the reality that everyone else.
lives full time..
The tradition that has elevated.
the possibilities of spiritual.
and theological knowing.
in Jesus Christ.
appears to have abandoned.
the possibility of knowledge..
This historic theological.
and cultural stew.
is the crisis of American.
white evangelicalism..
For me personally,.
it is a crushing,.
maddening and tragic story..
But far more importantly,.

$^{1241}$and I hope that I am wrong,.
this highly public narrative.
holds every sign.
of driving a multigenerational.
decimating disaffection.
with many people,.
not least many young adults.
in the United States..
For if this,.
this that has been demonstrated.
in the last four years.
and in the last months.
and in the last weeks,.
if this is the fruit.
of white American evangelicalism,.
how could it ever be claimed.
that its gospel.
could be received as good news?.
So in light of all of that,.
what is the hope?.
Jesus said,.
"I am the way,.
the truth and the light.".
This is the word.
that defines and holds.
all other words..
This is the plum line,.
the true vine,.
the rock that is our origin,.
our redeemer and our sustainer..
Behold,.
all things are being made new,.
the apostle says,.
even white evangelicalism.
that promotes that hope for others,.
that hope for others.
when it is a time.
when we who are white evangelicals.
in America.
need it more than ever..

$^{1281}$On many levels,.
the white American evangelical population.
does not want to live out.
its theological identity.
as much as it wants to justify.
its social and racial identity..
And more on this.
in the next lecture..
There is no joy in sharing.
what I have said in this lecture..
The story could be told.
in the positive light.
of God's true and faithful bounty..
In truth,.
it is a both/and story..
God is faithful.
to white American evangelical church.
and the white American evangelical church.
is a deep and painful crisis..
In this lecture,.
I have tried to show the scope of the problem..
Tomorrow,.
I will turn to some.
of the more underlying reasons..
Let me stop there.
and see if there is comment or questions.
as we engage together..
Thank you, Dr. Labberton,.
for walking us through.
the past three centuries,.
giving us a quick.
but succinct historical survey.
of how the present evangelical crisis.
came to be..
So in what follows,.
two members of our faculty.
will give their response.
to Dr. Labberton's speech..
Before introducing them,.
I would like to remind the audience.

$^{1321}$that a question and answer session.
will immediately follow the responses..
All the questions can be posted.
on our website in the live streaming page..
And you may scroll down the webpage..
After the live streaming in Mandarin,.
there is a Q and A session,.
and you are most welcome.
and strongly encouraged.
to input your questions there.
after inputting your name..
Your questions or comments.
can be either in Chinese or English..
Now, our two responses....
our two response speakers today,.
they are Dr. Chen Wenyan,.
our visiting professor in practical studies,.
and Reverend Dr. Song Jun,.
our Philip Tan associate professor.
in theological studies..
Their responses will be given.
in English and Putonghua, respectively..
Let's welcome them with our applause..
Thank you, Dr. Labberton,.
for your wonderful lecture,.
and I feel so honored.
to give my personal response here..
I was drawn to the Fowler School of Theology.
long before I entered the CGST..
I think most of the Hong Kong churches.
know Fowler not only as a seminary.
and a training ground.
for counselors and theologians,.
but also as the school.
for training intercultural workers.
and teachers in mission..
The two CGST mission faculty members.
prior to me both got their D.MITS degrees.
from Fowler School of World Mission,.
which has now changed her name.

$^{1361}$to the Fowler School of Intercultural Studies..
To me, Fowler's intercultural studies.
excel on issues like gospel and culture,.
intercultural communication,.
cultural anthropology,.
and so on and so forth..
And we have all benefited.
from the easy-to-grasp concepts.
like 10/40 window,.
unreached people groups,.
people movement in church growth,.
et cetera,.
developed by the Fowler faculty..
We are also on the receiving end.
from a lot of American missionaries.
coming to this part of the world..
Personally, I had great friends.
among the American Southern Baptist missionaries.
when my family was in Sri Lanka.
three decades ago..
I had very mixed feeling,.
experience with them.
who worked closely with me.
in that country..
We were great friends,.
so much so that we had asked.
an elderly couple.
to be the godparents of our infant son..
But at the same time,.
we avoided too close a connection.
because of the way they carry themselves..
Coming in with obvious financial power.
and human resources,.
locals commented that they lived like kings..
Another difficulty between us.
was our differences.
in seeing the meaning of the gospel.
in a world more torn land..
To our American colleagues,.
their job was to get the message.

$^{1401}$of the gospel across.
without ever touching.
on the problems of inequity and injustice.
that were rampant in the country..
A lot of the mission theories.
and conceptual framework.
were generated from social sciences.
which were excellent tools.
for reaching people outside the church..
But unfortunately,.
such techniques did not touch.
on the socioeconomic political aspects.
of the gospel..
And the good news of God's mercy and justice.
to a broken world has been missed out..
What is the gospel to a world.
plagued by systemic oppression.
against the minority and the disadvantaged,.
especially when privileged middle-class Christians.
were being implicated.
as part of the unjust process?.
After reading Dr.Leviton's lecture script,.
I could suddenly solve the puzzle I have.
regarding the focal areas of American missiology..
Now I understand.
why the packaged Kairos mission course.
did not mention the special needs.
of the poor and the marginalized.
and why in another classic package.
called Perspectives,.
out of 169 articles,.
only 6, that is 3.5%,.
directly talk about the issues of poverty.
and social ills..
Now I understand.
why there is no mentioning of liberation,.
of ethnic equality,.
and of suffering and persecution,.
because the majority of authors.
belong to a majority of white evangelicals.

$^{1441}$that have actually turned a blind eye.
on these issues on their own doorsteps..
Then the problem is getting serious.
because we have imported the whole paradigm.
and repeated the same patterns.
of missionary engagement.
in our Hong Kong and Mainland churches..
These days,.
we talk a lot about the Silk Road initiatives.
that have enabled Chinese missionaries.
to venture far beyond the country border,.
going to countries.
that badly need Chinese input in their businesses..
But in all these Asian and African countries,.
do we have to ask.
if our political supremacy.
and economic advantage.
reflect a God that care for sins.
generated from power imbalance?.
How much are we implicated.
in some of the irregular Chinese trading methods.
and exploitation of people and the land?.
The way we carry our presence.
speaks volumes about the gospel we preach..
The contemporary Chinese mission movements.
like Back to Jerusalem,.
the Homecoming Chinese Gathering,.
or Mission China 2030,.
have gathered momentum in recent years.
to mobilize Chinese Christians.
to serve the global church,.
which is very good indeed..
However,.
despite all their good intentions,.
they all carry certain air of conquest and entitlement,.
painting the Chinese church as the country.
used by God at the end to finish the task.
to preach the gospel to the ends of the earth..
Such stance does energize.
and motivate people in positive ways,.

$^{1481}$but are we going out as conquerors and saviors.
or as fellow sinners needing redemption?.
Do we dare to mention that.
we are also being seen against by the social system?.
The same holds true within the Chinese soil.
when a sizable group of Chinese Han Christians.
enter minority groups without any attempt.
to embrace a world of weakness,.
vulnerability and marginality..
Han Christians often fail to identify.
with the struggles of being minorities,.
refuse to learn minority languages and cultures,.
and expect ethnic Christians.
to merge into the majority Han churches..
Thus,.
they have inadvertently adopted.
a mindset and lifestyle that support inequality..
Critical self-reflection is truly needed.
for missionaries and their sending bodies..
Another problem coming out.
from this superiority mentality.
is the lack of partnership.
with the weak in the advancement of the gospel..
We pride ourselves when we play the savior mode,.
sending out workers to places.
where the church is poor and depleted.
and sometimes even persecuted..
But we are not ready to encounter truth.
in those contexts.
which may change and challenge.
our current position in our theology..
Are we ready to engage with cultures.
that may be superior to ours,.
not in the modern day economic technical sense,.
but in the sense of classic moral standards.
that sometimes have even worked on the archaic,.
like caring for the elderly,.
communal sharing,.
the demand to uphold family name and honor,.
the emphasis on resilience in suffering.

$^{1521}$instead of success, etc., etc..
Are we repeating the failures.
of American white evangelicalism?.
Coming back to lecture today,.
we might not be able to feel.
the kind of pain and anguish.
Dr.Leberton has on American white Christians.
because the whole scenario seems so distant..
But we know we are all bundled together.
in this global village..
And it's time for us to confront.
the whole array of misrepresentation of the gospel.
and the darker side of human nature.
being dressed under the cloak of evangelicalism..
I pray that together.
we might be able to find a way out of this mire..
Thank you..
尊敬的Leberton博士,李思敬院长,各位嘉宾,.
established guests, colleagues and students,.
I am most honored to be given the opportunity.
to respond to Professor Leberton's great speech today..
I'd like to talk about the stories.
of American mission workers in modern China,.
but it would be a totally impossible task.
to do justice to this rich legacy.
within seven minutes' time..
However, I am much encouraged by the example.
of the late Reverend Dr. Cheng Chengyi,.
one of the representatives of Chinese churches.
at the World Missionary Conference.
in Edinburgh in 1920..
Rev. Cheng was able to clearly articulate.
the voices of the Chinese Church.
within a short span of just seven minutes..
Tracing the footprints of the many.
American mission workers in China,.
from Elijah Coleman Bridgman,.
the first one to arrive in China in 1830,.
through to Loyal Holding Bartow,.
who finally died in prison in China,.

$^{1561}$we can clearly see the dedication.
and contribution of the American Church in China.
in those one and a half centuries..
The impact made was enormous and imperishable,.
so much so that even after over 70 years,.
the new China could not totally remove.
all the marks and traces they left..
If we push the clock back.
another 30 years from 1949,.
we will find that the Chinese Communist Party.
had been a competition to Christianity.
since the Party's inception in 1921..
The competition was vigorous on many, many fronts..
Both sides had wins and losses at different times..
In respect of a number of reasons,.
in respect of a number of resources and operations.
and other things,.
in the first 30 years of the 20th century,.
Christianity seemed to gain the upper hand,.
except that it didn't have a military presence..
But by the end of the second 30 years,.
Christianity had lost almost everything..
And during the third 30-year period,.
the competition ended in a draw.
if we look at numbers alone..
There is so much that I want to tell about this period,.
so in the remaining five minutes,.
I should be very, very focused..
In mid-20th century,.
the China Committee of the Division of Foreign Missions.
of the National Council of Churches of Christ in the USA.
conducted a study on the 150-odd American mission workers.
from 22 sending organizations.
who withdrew from mainland China..
The report entitled "Lessons to be Learned.
from the Experiences of Christian Missions in China".
was published in 1951..
It represented the common reflections.
of that group of American missionaries in China..
One of the issues raised in the report.

$^{1601}$caught my special attention..
Whilst almost all missionary respondents agreed that.
services in the urban and rural areas,.
common education,.
improvements in agriculture and so on,.
were all needed to meet the needs of the people of China,.
86% of the respondents pointed out that.
the Church had done very little in helping workers and farmers,.
and 31% agreed that.
the Church's action would build up a community.
that is totally detached from the poor people,.
and 31% agreed that.
missionaries tended to associate much more closely.
with government officials, intellectuals and rich people..
Since the mid-19th century,.
the most urgent priority for China.
and the most desperate dream of her people.
had been survival and rebuilding the country..
For this, different organizations and individuals.
had put forward various proposals and solutions..
At the same time,.
they were fighting for their share of influence and power.
which was needed to change the country..
That the Communist Party finally won this power struggle.
might be attributable to many reasons,.
one of which was the ability to hone in precisely.
to the crux which was the farmers and the land..
So the Communist Party's.
penetrated into farming communities,.
enticing them with fair distribution of land,.
and tirelessly fought on their behalf,.
thus winning over their support..
Let me clarify here.
that I'm not trying to say that.
the Chinese Communist Party's goal and efforts.
are not related to the Communist Party's endeavors.
without commenting on the legitimacy of their tactics and actions..
By contrast,.
many American mission workers.
tried to showcase the Chinese version of America's success story.

$^{1641}$in becoming a strong nation,.
extending the American westward expansion.
all the way to China,.
with the promise that.
this gospel from across the Atlantic.
would bring about democratic political system.
in the American style,.
free market economy,.
and the Christian ethos.
that was rooted in the founding of the United States..
I must admit that this is still.
the dreams of many Chinese Christians today..
But as an agricultural country like China,.
what about the most pressing social issue,.
namely the farming people and the land?.
During the 1920s to 1930s,.
some American mission organizations.
did send some agricultural mission workers to China,.
and in some Christian universities in China,.
agricultural departments were set up..
Some Chinese Christian leaders had championed.
some rural development movements.
and managed to get some financial.
and policy support from the then Kuomintang government..
But their contribution was limited to symptomatic measures..
These are like starting education in the rural areas,.
improving hygiene and farming technologies,.
but without tackling the roots of the problem,.
which was the land issue..
But these efforts were not embraced.
by mainline Chinese churches,.
and indeed,just like a glasshouse flower.
which bloomed only for a short while.
and soon to be destroyed by the Sino-Japanese War..
Just as advocates for rural development.
were trying hard to persuade the mainline Chinese churches.
to change their focus,.
some mission workers in the tertiary education sector.
cried out in alarm.
that the Communists had stolen from Christian thoughts..

$^{1681}$The Communists had already put into practice.
the very concept of fighting for social justice.
for grassroots people..
Pastors at that time also found that.
they were losing the young people.
to the Communist movements..
In the summer of 1933,.
special meetings were held among mission workers.
to discuss how Christianity could be as influential.
among young people as Communism..
The issue that I wish to bring up today.
is not whether Christianity could win in this competition,.
but rather whether the Church.
is following the path of life.
after Christ and His cross..
Then Jesus said to the disciples,.
"If you want to be a Christian,.
you must be a Christian,.
and you must follow the way of the Lord.".
Then Jesus said to His disciples,.
"Whoever wants to be My disciple.
must deny themselves.
and take up their cross and follow Me.".
The young official in the Gospels.
had great wealth,.
so the meaning of the cross.
was to challenge the status.
and challenging the social order.
on which He was established..
Jesus answered,.
"If you want to be perfect,.
go, sell your possessions,.
and give to the poor,.
and you will have treasure in heaven..
Then come, follow Me.".
Christ has called His followers.
to tread on a downward path..
The issue is not whether the Church.
is in the middle class,.
but rather whether she is living out.

$^{1721}$a different life,.
separate from the world trends.
and in accordance with Christ's commands..
We can list four or more characteristics.
of evangelicalism,.
but each one may have the peril.
of becoming just a deceiving packaging..
The crux is whether we are determined.
to internalize it as our inner value.
and to be recognized as such.
through our earnest practice..
Thank you..
(掌聲).
Thank you..
Thank you, Dr. Chen and Dr. Sung.
for responses..
And now in the rest of the....
we have about 45 minutes of questions and answers..
And in fact, there have been several questions.
being sent in..
Dr. Levitin, can we start with a question.
perhaps most directly related to your presentation?.
So I think the first question is.
what factors made evangelical Christians.
so blind to be unable to live out the truth of gospel?.
What can be done to change?.
What suggestions do you have for Chinese churches.
in America, which tends to so blindly.
follow the white churches in many or in all ways?.
Thank you very much for that good question..
Let me start simply by also saying thank you.
to the respondents..
I appreciated very much their comments.
and I hope I'll have a chance.
to interact with them at some stage..
The question is penetrating.
and I think maybe one way of summarizing it.
is that we are all aware that the gospel.
is far more radical than the church.
that names that gospel as usually ever wanting to be..

$^{1761}$What Jesus did and what he said.
and how he taught.
and what he called us to.
is just much more difficult and uncomfortable.
than we are usually willing to actually accept.
or even acknowledge..
I think Christian history has shown.
that there is a great tendency, a common tendency.
to accommodate the gospel.
to something that we find more acceptable.
than to live in the radical way.
that Jesus exemplified and taught..
So in talking about American white evangelicalism,.
I'm talking about one way.
that the church throughout history.
has simply accommodated the good news.
to the life that we want to go ahead.
and live apart from the gospel..
And we want to use the gospel.
as a framework for various purposes,.
for eternal life, for moral clarity,.
for social relationships, for hope,.
for love, for forgiveness..
But the radical call to take up our cross.
and follow Christ has never been.
a popular part of the life of the church..
And throughout church history,.
the number of people, as Jesus said,.
that are willing to do that are few..
So let me begin first with saying that..
We're talking about the broadness of the church.
and its unwillingness to actually be.
radical followers of Jesus..
And in many ways, the white evangelical church.
in America becomes a case study.
in what for different reasons could be said,.
perhaps as the two respondents have indicated,.
in some ways of the Chinese church,.
but it certainly could be true.
of countries all around the world,.

$^{1801}$each in their own particular context and culture..
But in the American setting,.
I would commend a book.
that I'm going to refer to again tomorrow.
called "The Democratization of Christianity in America.".
"The Democratization of Christianity in America.".
It's an exceptionally good book,.
which is an expression of the way.
that American Christianity changed.
as it moved from the east coast.
to the west coast of the United States.
and further into the south as well..
And a great deal of the theme of the book.
has to do with the way that Christianity.
became something which people made up for themselves..
So they had the Bible,.
but they often didn't have any educated leaders.
they didn't have any kind of Christian community.
around them necessarily..
They didn't have any educational institutions.
that were helping to guide and form their leadership..
They just instead took the gospel.
or whatever it is that they thought.
they knew of the gospel.
and put it to work for their own designs,.
sure humbly and earnestly,.
but often very, very specifically and personally..
Rather than as a response to.
the lordship of Christ over all things..
And out of that came.
what now we would recognize, for example,.
as consumer Christianity..
That is, I'm going to take from the gospel.
what I want,.
and I'm going to keep the parts that I like.
that serve me,.
and I'm going to more or less ignore.
or reject the parts that I don't like..
And that pattern,.
which is one that emerges early in American history,.

$^{1841}$is one that continues to this day..
So the antidote to that,.
coming to the question asker's point,.
what is the antidote to that?.
I think the antidote to that.
has to do with how we form genuine disciples,.
not people who are part of.
denomination and a movement.
or an organization or a structure,.
but people who are truly devoted.
to being followers of Jesus..
And then do it in community,.
because of course we're not meant.
to live this kind of life..
We can't live this kind of life.
individualistically as Americans also try to do..
We'll come to that more tomorrow..
We have to do it in community..
And in doing it in community,.
then we have to check and balance.
of other people around us..
And the difficulty is that so often.
what we gather around us.
is a community that looks like us..
So then the church starts looking.
like the social setting.
into which we were born,.
rather than a community.
like the apostle Paul,.
I would want to argue,.
suggests is meant to be.
an entirely new humanity..
One humanity, a new humanity.
that reflects the death.
and resurrection of Jesus.
by the death and resurrection.
of ourselves in community.
of unlike people..
That's the place that the church.
has often broken down..

$^{1881}$And that's exactly what happened.
in American church culture.
in historical terms..
But it is largely still the case..
So when a pull was done by the church.
a pull was done by the Pew Foundation.
two years ago,.
about how many white Americans.
have close relationships.
with a person of another race,.
nearly 85% of white Americans.
said they did not have.
any close relationships.
with people that were not of their race..
Now that's true of society in general..
But if that's true of society in general,.
then it is tragically certainly true.
of churches who have a white normativity.
that often excludes people of color.
and does so in cultural and church terms..
And then we just become a mirror,.
a hall of mirrors.
that we all look alike.
and act alike and talk alike..
And therefore we think that's the gospel.
instead of letting the gospel.
radically redefine us..
Thank you, Dr. Levitin..
I think you have raised several themes.
like taming the radical nature of the gospel,.
the consumerism,.
and also bringing together.
unlike people in the new created community in Christ..
So these themes sound also familiar.
here in actually in the Chinese churches.
and the churches around here in Hong Kong as well..
So I think this leads very naturally.
to a second question..
Actually, it's a small agenda in nature..
It says maybe our two response speakers.

$^{1921}$have to fill in some of the gaps as well..
It asks,.
"In what way does American white evangelicalism.
affect churches in Hong Kong now?".
So, yes, I think we need wisdom.
from across the Atlantic Pacific..
The respondent should speak first..
OK..
So perhaps I translate the question.
for Dr. Sung-Chun first..
OK..
蔣元所說的美國白人福音派的信仰.
對於現在香港的教會有多大影響?.
Maybe we may start with a born Chinese,.
a Hong Konger..
Sorry..
Well, I noticed that there are.
quite a few faculty members on the floor..
So I'm expecting some of your response as well..
OK..
We need your input..
And, yeah, because of my location,.
I mean, I feel more and more distant.
from the Hong Kong current scene, to be sure..
But I'm just thinking that.
just following up on Dr. Lepperton's remark,.
I think we need to be aware of our inclination.
or this kind of tendency.
to group together with people who are like with us..
So I mean, this is something very natural.
because you feel so comfortable..
We will stay in our comfort zone..
But right now, because of the divisions,.
the divisiveness among the churches in Hong Kong,.
I think we need to sort of alter our mentality..
We need to keep reminding us.
that we have to keep reading from the other side.
and then to listen from opposing angles and views..
So I think this is kind of a discipline..
I call this even spiritual discipline for us.

$^{1961}$that we need to constantly keep learning.
and trying to see what we have missed.
in our own framework..
So this is something that I think should be done..
But if talking about this whole concept of.
whole tendency of white evangelicalism.
and their influence on the Hong Kong church,.
I'm feeling that whatever American is doing now.
would have global effects..
Like it or not..
So we have to acknowledge that.
because of all this online information.
and all because of all the most of the theological material.
we study, imported from the West.
and because of the majority of written material.
from the States,.
I mean, we are really affected.
by the whole concept of.
theology that would underplay the.
emphasis or the look on the weak and the marginalized..
This is what I'm talking in my response..
Because of this whole idea of success.
that you can do it, freedom and all this idea.
that is very positive thinking.
and it's really good for us..
But on the other hand, I think we really lack.
this whole concept of suffering and persecution.
in American theology..
And we import them..
And then we tend to talk about,.
say when we talk about suffering,.
we talk about personal suffering as such..
Instead of social suffering..
We talk about this whole concept of.
suffering because of illness, of age..
But we know that.
in the end, actually, most of the suffering.
actually comes from a social setting..
But because Hong Kong churches are in general.
also very middle class..

$^{2001}$And so we can hide in our small niche.
and then we can feel comfortable..
And again, we can sort of explain our whole being.
with a very limited kind of framework..
And so I'm feeling that if we look broader.
into the whole global world.
and read more about the kind of experiences.
and narratives from more of the third world countries,.
from Africa, from Latin America,.
and from say South Asia, etc..
I think we'll really try to wean ourselves, okay,.
from American evangelicalism..
Thank you very much..
Thank you for the question and the response from Wenyan..
I just want to respond to one point..
From what Dr. Levitin talked about.
American evangelicalism.
true to the critical and put into action..
I think the churches in Hong Kong.
do have the DNA of evangelicalism..
But there are three things.
which are also pointed out in the question..
And as someone coming from mainland China,.
I have such observation..
The first one, from 1980s onwards,.
churches in Hong Kong.
more and more members of them.
are becoming more concerned about social issues,.
especially among young Christians..
And this is a very clear sign..
The second point is.
in Hong Kong there are also many ethnic problems..
But among Christians in Hong Kong,.
very few of them would voice out their response..
Most of them do not admit that there is an ethnic problem..
But as a very international and open community here,.
with many, many people from many different ethnic backgrounds living here,.
their conditions and their problems,.
their livelihood is not something.
that is a concern of the Church of Hong Kong..

$^{2041}$The third point is.
many churches in Hong Kong are middle class,.
neglecting concern about the grassroot people..
But admittedly there are many organizations,.
Christian NGOs,.
who are helping the underprivileged..
But they also have characteristics of the middle class church..
Thank you..
Dr.Levitin, any further comments after hearing the two?.
Augustine was convinced that sin is disordered love,.
and every culture has an order of loves that it prioritizes..
And in American culture,.
loves that tend to get prioritized are loves like.
the love of personal happiness,.
the love of pleasure,.
the love of success,.
which can be monetary,.
but it can be other things as well..
The love of power,.
the love of beauty,.
these are all things that are part of our loves..
And we live in a world of disordered loves,.
where we're meant to live and sustain ourselves,.
first and foremost out of the love of God,.
and in response showing our love to God as our first love..
But of course the nature of the human story.
and the nature of the American white evangelical story.
is the story of us attaching our lives to other loves,.
and then trying to make peace with the way that they do.
and don't fit the call of a disciple..
And part of what we're called to is to reorder our loves..
And I think this is where Christian worship.
reveals both our sin and the places that American culture.
has influenced many places around the world,.
and the ways that we are meant to be redeemed..
So ideally we could say public worship.
is meant to be an experience of reordering love,.
to be regrounded again in our love for God.
and our love for our neighbors.
and our love for ourselves..

$^{2081}$And within the ordering of loving our neighbors,.
it doesn't mean the neighbors that are like us or who like us,.
but as Jesus says in the Sermon on the Mount,.
it even can include those who are our enemies..
So to live into that kind of reordered love.
means that our public worship has to feed those appetites,.
nurture that desire that we would be those kinds of people..
But instead,the American worship that we've exported.
is often a worship of entertainment..
It's about celebrity..
It's about money..
It's about beauty..
It's about fame,etc.,etc..
And not,as Dr. Chan said so beautifully,.
it's not about the first law of God,.
particularly for the poor and for the neediest..
So we are not cultivated to love what God most loves,.
and instead we're recultivated in the church.
to love what we already love.
and then to just go on loving it even more.
rather than actually reflecting the love of God for the world..
So I fear that we've exported that,.
and it is its own musical and worship entertainment industry.
that has often been exported,.
and tragically that forms people to love the wrong things..
Thank you for your comment..
So your comment has encouraged me to study more closely your two books,.
The Dangerous Acts of Worship and The Dangerous Acts of Loving Your Neighbor..
I think I will take out more answers from your two works..
Well,actually,bring that to,.
I think your mentioning of loving your neighbor,.
actually it has quite some impact on our present Hong Kong churches as well..
So there are two questions a bit related,.
but actually there are some details there,.
so maybe it can help Dr.Leviton to help giving your wisdom.
about what we may do,.
how we may react in the certain situation..
One question is from a pastor,.
and he says,.
Dr.Leviton, your views on complicity.

$^{2121}$is something we can relate to greatly in Hong Kong..
Given the recent circumstances as experienced by Hong Kong,.
most churches are very divided..
It is those who feel we should speak out.
against present happenings,.
against authority,.
and those who do not wish to at all..
As a pastor,.
I can say that we struggle greatly with the balancing act.
of wishing to speak against that which we feel is evil,.
and of not wishing to offend those members of the church.
who are strong supporters of authority..
But you seem to present the view that.
we are either complicit and condone evil acts like slavery,.
or we speak out against it,.
which at least to war,.
and there's highly frowned upon in the Chinese community,.
which is all about harmony..
So how should we juggle this balancing act,.
speaking or not speaking,.
and not to be complicit?.
So that is the very particular question..
And in the more general nature.
it's about how to communicate effectively.
and in a loving way.
to those who hold opposite opinions from you..
So these are, I think, two very related questions..
Can we hear your wisdom on that?.
Yes, I'll offer what I have..
Whether you consider it wisdom,.
I'll leave you to decide..
Let me say first that.
what you've just described by that example.
is precisely what has been going on.
in the American white evangelical church,.
where you end up having some members of a church.
believe that, for example,.
that America and its government.
is not in any way an expression of God's will or desire..
And in the same congregation,.

$^{2161}$you have people who look at what's going on.
and say this at last is the best.
and most supportive government for Christians.
that the United States has ever had..
So how do they share in the same congregational life?.
So this problem that we're all describing.
manifests under different issues perhaps,.
but the dynamic is really the same..
So when I think about, first of all,.
how Jesus responds to this dilemma,.
he moves toward the stranger, toward the other,.
and also actually toward the so-called enemy..
These are such dramatic examples,.
but one of them, for example,.
would be at the end of the Sermon on the Mount.
when Jesus says that we build our house on rock or on sand..
He's clearly saying that we build our house on rock.
by actually doing the truth,.
not by just affirming it or confessing the truth..
We live the truth..
That's how we build our house on rock..
And what's interesting is that right after that,.
he encounters a person with epilepsy..
Sorry, not with-- yes, with epilepsy..
No, sorry, excuse me, with leprosy..
Excuse me..
And in that moment, he's invited to heal the person,.
and Jesus moves toward this person.
that was considered unclean,.
touches them and heals them..
And then if that's not enough of a shock,.
then the very next part of this.
is that a Roman centurion comes toward Jesus.
and says that a servant of his is in great need..
Could Jesus come and heal him?.
And Jesus says he's happily willing to do so..
And the man says, "No, no,.
you don't need to come and do so.
because you just say the word and it will be done.".
And then he pauses everyone around and said,.

$^{2201}$"Did you just hear what this man said?.
I have not heard," these are the critical words,.
"I've not heard faithfulness like this in all of Israel..
There are some people outside the kingdom of God.
who will be in and some who you presume to be in.
who will be on the outside.".
Now I share those stories because for me,.
they are examples of at least some of the things.
to consider in answering the question..
I want to first of all move toward people.
that I might consider to be uncomfortable..
They might not be "unclean,".
but they're not like me or they are not easy.
for me to be with or they are a problem.
in quotes in some way or another..
And Jesus gives an example in this case.
and in so many other cases as we know.
about what it means to move toward people.
who are difficult, marginalized, awkward, needy,.
potentially unclean..
I think there's a lot of tendency.
in contemporary culture around the world.
to label people clean and unclean,.
to be mindful that summer should be included.
and others don't deserve to be included..
What's amazing about these texts.
in the early part of chapter 8 of Matthew's Gospel.
is that Jesus moves toward the man with leprosy,.
touches him, calls him a name of attachment,.
heals him, and does this completely counter.
to the pressure of the moment..
The social pressure..
But then more radically,.
when the centurion appears,.
the representative of dominating Roman authority,.
he receives the man's witness,.
hears it for what it is,.
acknowledges it and lifts it up.
and then goes forward to actually heal the man's servant..
Now, in both cases,.

$^{2241}$I think we learn lessons about what it means.
to try to live together in a different way..
In the first case,.
how do I learn to lean toward people.
that I might marginalize.
because of different opinions in churches.
over these very issues that you've just raised?.
That has to take practice to develop that ability..
And I think often churches have small groups.
that meet or prayer groups that meet..
Our ability to pray for one another.
to grow in our capacity to love people.
inside the body of Christ.
is one of the kinds of prayers.
and the sort of work.
that we need to hold one another accountable to doing,.
not just becoming our own little small tribe within a tribe,.
but to actually learn to love one another,.
to see each other,.
to value each other,.
to honor in whatever way we can..
And then the gold standard of Jesus.
is not whether we love those who love us,.
but whether we love those who don't love us.
or don't like us.
and who may even be our enemies..
How do we also learn to practice.
loving people like that?.
And I think that happens in that case in Matthew 8..
It happens first of all by Jesus.
just very carefully listening to what the man said..
The most amazing part of that text.
is the way that Jesus.
hears his actual declaration of faith.
and then focuses on that,.
letting it fully be absorbed into his own heart and mind.
and then responds back to that affirmingly..
That is deep and profound faith.
unlike anything I've heard in all of Israel..
Well, that was scandalous, of course, to Jews..

$^{2281}$It was crazy to think that Jesus would do that..
And yet Jesus listened very carefully.
and affirmed what he could affirm..
Now, I think often in divided churches.
like the ones that we're talking about tonight,.
we have to learn to really listen much more carefully..
And often, not always,.
but often within the differences.
is a failure of really hearing the other person,.
really understanding why they're motivated,.
what it is that drives them,.
what their fears are,.
what controls their anxiety..
Why are some people so insistent.
on holding on to certain forms of authority?.
And why do other people find it necessary,.
almost required for them.
to reject certain kinds of authority?.
Those are conversations.
that ultimately the body of Christ has to have..
But we have to first be able to simply carry it out.
in a very respectful, mutual way..
And that's going to require that our hearts be remade..
It's going to require, in Paul's language,.
that we be transformed by the renewing of our minds,.
that we have the heart of Christ.
and that we do not regard equality with God,.
that is, taking our own prerogative and views as first order,.
but actually lay them down for the sake of loving.
and in some bizarre way,.
identifying with the person that we most disagree with..
Those are just some small initial thoughts,.
but I think it's in that kind of direction.
that some of this work can be sustained..
But as a pastor, we have to decide,.
and I was a pastor for 35 years.
before I became involved.
in formal theological education as I am now..
And I was very, very aware every day as a pastor.
that it was about restoring community and communion..

$^{2321}$The word "communion" for me.
is one of the most valuable Christian words.
in the whole Christian vocabulary..
And it's through that lens,.
communion with God, communion with one another,.
communion across the economic, social, political divides,.
that communion in Christ is meant to give us.
our ground on which to stand and build our lives..
I think I do have something to add along this line..
From my experience with the Mainland churches,.
because it's divided into the institutional.
and non-institutional churches,.
and I'm finding that this one major attitude.
that we need to have is to be humble.
and to know that in spite of all my conviction.
and understanding of truth,.
I don't have all the truth,.
that God is bigger and higher..
I know I have certain conviction.
that I'm following the right way in my choice,.
either it's institutional or non-institutional..
But it's okay,.
because I see the other party,.
the other churches as brothers and sisters.
and not as satanic work..
So this is important..
And another way,.
another solution is for me to be engaged.
together in some joint projects..
Usually I find that some social projects is good,.
like some community work projects like that,.
together with people from different groups.
and denominations..
They don't have denominations,.
but institutions, outside institutions..
And if we work together,.
through living together,.
we come to appreciate what they have.
and what I don't have, okay?.
And then by communing together,.

$^{2361}$I think this is a way for us to soften our divide..
And then I have to acknowledge.
that the divide is always there..
I mean, I have no way to sort of....
You can't pretend..
Yeah, I can't..
I can't dissolve that, okay?.
But I can, to a certain extent,.
hold one another's arms.
and then we say that we can work together..
And I always tell my friends.
that our enemies are outside,.
not inside this circle..
Right. Thank you..
So even though you mentioned.
it's only your small initial thoughts,.
it takes a heart which is big.
and that can generate.
a large amount of courage.
and also so as to overcome a barrier.
which is as high as a mountain, actually..
Right. That's the stuff..
We really appreciate that..
And, well, actually,.
this related, but not that political.
or maybe political,.
sexuality has become a political issue as well..
So one of our faculty members.
raised this question..
Suppose you are talking to.
the conservative white Christians.
who feel threatened by the LGBTQ movement,.
who feel that they are losing.
their once-dominant social influence..
So how would you encourage them to change?.
Well, I think it's along the very same lines.
that we were just discussing..
I think it has to do with beginning,.
first, not with theological assessments,.
but with interpersonal relationship building,.

$^{2401}$letting people actually tell you their story..
Often I find in settings like that,.
this is less and less true, by the way,.
but it has been true,.
that people didn't believe.
that they knew anyone who was LGBTQIA..
They didn't have any idea.
where those people were,.
but they were not in their life..
Well, it turns out,.
as more and more people.
have been willing to explain.
and declare who they are.
and how they approach the world,.
they are discovering that their children,.
their grandchildren,.
their neighbors, their friends,.
their colleagues may either be LGBTQ.
or they may be related to people who are..
And when that occurs,.
it's a privilege to actually welcome.
the knowledge of who someone.
actually understands themselves to be..
So even though I maintain myself.
a traditional view of the theological.
and moral understandings of human sexuality.
that have been classic in Christianity,.
I don't do so as a wall against LGBT people.
or as a battering ram to defeat them..
I want to hear them..
I want to know them..
I want to love them..
I want to invite them into communion.
with me and with other people..
I want them to not be isolated..
I want them to not be self-loathing..
I want them to feel as much welcome.
as I can extend to them..
And so I would encourage.
such a community, for example,.

$^{2441}$to think carefully about who they really know.
and how they are going to love people..
And I think every pastor has to work out.
in their own particular setting.
how they're going to help people.
do this kind of self-formational work..
But it really has to begin.
by the same things we've just talked about,.
laying down power,.
laying down certainty,.
laying down self-righteousness,.
not righteousness, but self-righteousness,.
laying down a presumption of a theological exam.
at the very beginning,.
or a moral exam.
at the beginning of every relationship,.
which is not the way that any of us,.
in fact, actually live,.
nor would we want to live that way..
So we want to know one another..
We want to love one another..
And there may be differences..
Of course, there are differences.
in how we see the world.
and relate to one another..
But I want to start with,.
as Dr. Chan said,.
I want to start with an attitude of humility.
and of deference..
I want to listen very, very carefully..
I want to hear their story..
I want to understand and empathize.
with the way that their story has unfolded..
Now, I believe that all of our stories.
are held by a larger story that's God's story..
So eventually, in some way,.
verbal and nonverbal both,.
I want to help people who are LGBTQ2,.
if they don't already,.
understand how to hold their story.

$^{2481}$in light of the greater story of God..
That's a longer and more theological question..
But on the pastoral level,.
I think it has to do with inviting people.
into honesty,.
into their own vulnerability,.
into their places of pain and shame.
and say, can you imagine.
what it's like to live in a community.
in which those realities.
are reinforced by the church..
People feel worse about themselves..
People feel more marginalized about themselves.
by the way the church has related to LGBT people..
And frequently, the church has been.
a failure in loving people.
who don't live up to what we believe.
are biblical standards, perhaps..
I'm deeply struck by your comment.
that it is a matter about.
loving and respecting your neighbors.
as creating the image of God..
It's not about warring or battling.
for theological or ideological correctness..
So this is a very important thing.
to remember, actually..
And maybe another question,.
or perhaps that will be.
the one last question..
Dr.Levitin, you talked about.
the influence from the media.
in the 1990s..
So could you comment on the impacts.
on the social media in white evangelicals?.
Yes. Well, it's of course a problem,.
as I would say it,.
at least all around the world,.
or at least in places.
where social media dominance.
controls things,.

$^{2521}$as social media does dominate things.
in the United States..
So you have public media.
of various kinds,.
but then you have just the social media.
and all the networks.
and communication of messages.
which can be used as battering rams..
And this is why.
it has been such a big deal.
in the last few days.
since Twitter no longer allowed.
President Trump to be able to.
post his declarations of various kinds,.
as you've probably.
or pleasantly experienced.
or seen or heard about..
And he has a following on Twitter.
of something like 84 million people..
So he would typically tweet.
usually several dozen messages.
over the course of a day.
to 84 million people..
So let's just imagine.
he was the only person.
that was engaging in social media..
And he would be the only person.
who would be able to.
be able to communicate with people..
And he would be the only person.
who would be able to.
be able to communicate with people..
And he would be the only person.
who would be able to.
be able to communicate with people..
And he would be the only person.
who would be able to.
be able to communicate with people..
And he would be the only person.
who would be able to.

$^{2561}$be able to communicate with people..
And he would be the only person.
who would be able to.
be able to communicate with people..
And he would be the only person.
who would be able to.
be able to communicate with people..
And he would be the only person.
who would be able to.
be able to communicate with people..
And he would be the only person.
who would be able to.
be able to communicate with people..
And he would be the only person.
who would be able to.
be able to communicate with people..
And he would be the only person.
who would be able to.
be able to communicate with people..
And he would be the only person.
who would be able to.
be able to communicate with people..
And he would be the only person.
who would be able to.
be able to communicate with people..
And he would be the only person.
who would be able to.
be able to communicate with people..
And he would be the only person.
who would be able to.
be able to communicate with people..
And he would be the only person.
who would be able to.
be able to communicate with people..
And he would be the only person.
who would be able to.
be able to communicate with people..
And he would be the only person.
who would be able to.
be able to communicate with people..

$^{2601}$And he would be the only person.
who would be able to.
be able to communicate with people..
And he would be the only person.
who would be able to.
be able to communicate with people..
And he would be the only person.
who would be able to.
be able to communicate with people..
And he would be the only person.
who would be able to.
be able to communicate with people..
And he would be the only person.
who would be able to.
be able to communicate with people..
And he would be the only person.
who would be able to.
be able to communicate with people..
And he would be the only person.
who would be able to communicate with people..
And he would be the only person.
who would be able to.
be able to communicate with people..
And he would be the only person.
who would be able to.
be able to communicate with people..
And he would be the only person.
who would be able to.
be able to communicate with people..
And he would be the only person.
who would be able to.
be able to communicate with people..
And he would be the only person.
who would be able to.
be able to communicate with people..
And he would be the only person.
who would be able to.
be able to communicate with people..
And he would be the only person.
who would be able to.

$^{2641}$be able to communicate with people..
And he would be the only person.
who would be able to.
be able to communicate with people..
And he would be the only person.
who would be able to.
be able to communicate with people..
And he would be the only person.
who would be able to.
be able to communicate with people..
And he would be the only person.
who would be able to communicate with people..
And he would be the only person.
who would be able to communicate with people..
And he would be the only person.
who would be able to communicate with people..
And he would be the only person.
who would be able to communicate with people..
And he would be the only person.
who would be able to communicate with people..
And he would be the only person.
who would be able to communicate with people..
And he would be the only person.
who would be able to communicate with people..
And he would be the only person.
who would be able to communicate with people..
And he would be the only person.
Thank you..
And today....
Can I say one thing?.
Yes, sure..
The difficulty of speaking on this subject.
is that I arrive in your presence.
with this terrible news..
It feels like I've arrived.
with this terrible news..
It feels like I've arrived.
with this terrible news..
It feels like I've arrived.
with this terrible news..

$^{2681}$It feels like I've arrived.
with this terrible news..
It feels like I've arrived.
with this terrible news..
It feels like I've arrived.
with this terrible news..
It feels like I've arrived.
with this terrible news..
It feels like I've arrived.
with this terrible news..
It feels like I've arrived.
with this terrible news..
It feels like I've arrived.
with this terrible news..
It feels like I've arrived.
with this terrible news..
It feels like I've arrived.
with this terrible news..
It feels like I've arrived.
with this terrible news..
It feels like I've arrived.
with this terrible news..
It feels like I've arrived.
with this terrible news..
It feels like I've arrived.
with this terrible news..
It feels like I've arrived.
with this terrible news..
It feels like I've arrived.
with this terrible news..
It feels like I've arrived.
with this terrible news..
It feels like I've arrived.
with this terrible news..
It feels like I've arrived.
with this terrible news..
It feels like I've arrived.
with this terrible news..
It feels like I've arrived.
with this terrible news..

$^{2721}$It feels like I've arrived.
with this terrible news..
It feels like I've arrived.
with this terrible news..
It feels like I've arrived.
with this terrible news..
It feels like I've arrived.
with this terrible news..
It feels like I've arrived.
with this terrible news..
It feels like I've arrived.
with this terrible news..
It feels like I've arrived.
with this terrible news..
It feels like I've arrived.
with this terrible news..
It feels like I've arrived.
with this terrible news..
It feels like I've arrived.
with this terrible news..
It feels like I've arrived.
with this terrible news..
It feels like I've arrived.
with this terrible news..
It feels like I've arrived.
with this terrible news..
It feels like I've arrived.
with this terrible news..
It feels like I've arrived.
with this terrible news..
It feels like I've arrived.
with this terrible news..
It feels like I've arrived.
with this terrible news..
It feels like I've arrived.
with this terrible news..
It feels like I've arrived.
with this terrible news..
It feels like I've arrived.
with this terrible news..

$^{2761}$It feels like I've arrived.
with this terrible news..
It feels like I've arrived.
with this terrible news..
It feels like I've arrived.
with this terrible news..
It feels like I've arrived.
with this terrible news..
It feels like I've arrived.
with this terrible news..
It feels like I've arrived.
with this terrible news..
It feels like I've arrived.
with this terrible news..
It feels like I've arrived.
with this terrible news..
It feels like I've arrived.
with this terrible news..
It feels like I've arrived.
with this terrible news..
It feels like I've arrived.
\newpage



\section{}
\label{sec:F73AP9vE2KM}
\textbf{The Crisis of American (White) Evangelicalism (Day 2)}
\newline
\newline
連結: \href{https://youtube.com/watch?v=F73AP9vE2KM}{\texttt{ https://youtube.com/watch?v=F73AP9vE2KM}} ~~~~ 語音日期: 2021-01-21 
\newline
\newline
\hyperref[sec:Ps_hGAmvVcA]{\small{< < < PREV SERMON < < <}}
~
\hyperref[sec:index]{\small{[返主目錄]}}
~
\hyperref[sec:A4_dD_EAMHg]{\small{> > > NEXT SERMON > > >}}
\newline
\newline
$^{1}$(音樂).
歡迎各位先生女士.
各位在香港的朋友們,早上好.
還有早上好,我是Mark Labberton.
歡迎在網上觀看的朋友們.
歡迎來到中國教學大學第二天的.
Josephine Zhou 文化與道德課程週二.
我們昨天有很關心和分享的討論.
如果這是你第一天參加課程.
我會歡迎你來到我們第二天的課程.
關於美國白人主義主義的危機.
如果你想參加課程的話.
在廣東話或普通話.
你可以到網頁的下面.
找到相同語言.
在英文,廣東話或普通話.
今天我們很榮幸.
有Mark Labberton 和我們一起.
和我們分享基督教學和我們的文化.
今天的課程的題目是.
美國白人主義主義的危機.
在基督教學和文化之間的焦點.
我們先不說了.
讓我們歡迎Mark Labberton.
非常感謝.
我又一次的榮幸和榮幸.
我們今天的課程.
有機會有問題和討論.
我期待會有機會.
也期待會有回應.
基督教學和我們的文化.
我們會反映美國白人主義主義的危機.
它激發基督教學.
而它的文化吃了它的靈魂.
2004年.
在George W. Bush的政府高度.
和新的媒體波動.
探索了基督教的意義和性格.
David Brooks.
一位紐約時報的記者.

$^{41}$寫了一篇文章.
叫做《John Stott是誰?》.
那篇文章很簡短.
但卻給了John一個很美妙的畫像.
他寫的作品有著令人印象深刻的看法.
但也有John Stott的性格和影響.
而這都是Brooks說的一個理由.
他認為這就是最適合.
基督教思想的一方.
它吸收了John的信仰.
強硬和慎重的信仰.
教會,文化和道德的心.
和靈魂.
當然John不是來自美國.
所以我今天要請他離開.
但他代表和推廣的基督教思想.
是美國和全球的基督教思想.
最強勁和最重要的聲音.
它形成了美國和全球的基督教思想.
John的主要作家.
不僅是在很多書籍.
也在《聖經》.
都是他具有獨特的性格的見證.
雖然他出生在一個.
真正的高級英語家庭.
但John的生命中的教會.
破壞了他出生的基督教思想.
和態度.
他成為了一個基督教主義的.
全球主義者.
和一個深刻的朋友.
我已經有了30年.
在John的教會工作和工作.
的個人權利.
我可以見證John的基督教思想.
是如同他所說的.
他更是一個真正的基督教徒.
更深刻地了解了他.
我希望你們記住的第二張圖片.
就是Franklin Graham.

$^{81}$Billy Graham的兒子.
Franklin Graham的父親仍然活著.
而他對Franklin的私人和公眾影響.
Franklin的角色和故事.
是由他的父親建立的.
美國和全球的偉大偉大.
美國人和白人的平等主義.
Franklin Graham創立了.
Samaritan's Purse.
一個基督復興組織.
是一個表達基督的慈悲.
相似於其他基督復興.
和發展機構的創立.
例如世界視野.
和慈悲國際社會.
和其他.
當Billy Graham的健康下降.
並且Billy離開公眾生活時.
Franklin的聲音在不同的方式上增加.
不幸的是.
尤其是他的詩詩詩詩的聲音和聲調.
當川普政府.
在美國白宮實行的時候.
Franklin看起來已經成為了真正的信徒.
在使用川普總統府的服務上.
對教會的目的上.
而Franklin就盡了所有的力量.
他為川普的支持.
作出了一種殘酷的.
而常見的原因.
並且明確地定義了.
支持川普.
作為一個信仰的基督.
包括川普的"美國第一"政策.
我將這兩位聲音.
提升為兩位非常不同的.
"美國人"的相反的複製人.
因為John不是美國人.
但是美國人的種族主義.
John Stott和Franklin Graham.

$^{121}$共享一種共同的信仰.
和基督主義.
他們保護了共同的信仰.
但是他們展現了.
兩種不同的看法.
宗教和文化.
而在這第二節.
我想我們要討論的.
是關於這兩點的主題.
Stott是一名英國人.
出生於在倫敦的.
Alsos教堂的家庭.
他母親逐年逐年帶他去.
而他父親是一名.
一名猶太人和一名醫生.
他們住在幾公里的家裡.
他一生都住在倫敦.
他從未結婚.
他經常與公眾.
和私人的社會.
進行交流.
由於神的恩典.
和一種活生生的基督教.
也就是基督教徒.
基督教讓他自由脫離.
許多可能困擾他.
或保留他文化上的困擾.
主要是基督教徒.
他們的觀點是.
神的極端重新實現的現實.
被人類的靈魂所包裹.
強調,服務.
不,是敦促.
服務,治癒,痛苦.
死亡和昇華.
讓他生活和社會的位置變化.
基督教讓他.
在世界上生活和移民.
是另一種方式.
他在一個簡單的.

$^{161}$兩房單位.
做教會的.
專注於學習,寫作,教導和旅行.
基督教讓他聽.
信,理解和行動.
在世界各地的兄弟姐妹在基督教中的角度上.
發明了信念.
我最近與Rene Padilla.
一個基督教的創始人和領導人.
Misio Integral.
在南美國家的哥倫比亞.
他說他對西方白人的領導人.
是一種信念.
他聽了.
他想了解我們.
而不是控制或使用我們.
儘管格林頓·克林頓·克林的對世界教會的揭露.
儘管他父親對世界的理解和憐憫.
他也不再是一個小鎮.
在北卡羅來納州生活.
儘管世界各地的聖人.
都做了很好的工作.
但我深深地認為.
格林頓·克林的生活.
最主要是他的社會地位.
而不是他的基督教.
我自我批評.
格林頓·克林的聲音.
很像一個白人.
南美的男性.
尊嚴的人.
他相信基督教語言.
來批評和抵制美國的特殊性.
和獨立性.
換句話說.
他的基督教學.
似乎沒有破壞.
和重新定義他的生活和社會地位.
我可以認為我誤會了他.
我可以誤會了他.

$^{201}$但我的印象是.
他的基督教學.
從未擴大了.
他的基督教觀點.
或是鄰近.
或是自我.
他的觀點.
權力.
自我.
甚至是國家主義.
從未被他的基督教學重新定義.
而是相反的.
他深信美國的特殊性.
不愧是在紐約市中心.
建立了新冠病毒醫院時.
這起了重大的影響.
因為他不願意服務LGBT人.
他的政治.
阻礙了他對受苦的人的同情.
坐在座的基督教徒和稅收人.
與弱者和需要的人.
看起來都很吸引.
Franklin的社會理念.
他的慈悲的名義.
看起來是被川普操縱了.
也就是被川普的觀點操縱了.
所以我今晚的關注.
或是今早的關注.
是一個觀察和一個信念.
就是.
美國白人教會人士.
告訴我們我們正在生活著基督教.
這是我們的公開信息.
但更有可能的事實.
是我們正在生活著我們的文化位置.
而不是基督教.
當然我們都知道.
基督教和文化之間的關係.
是一場歷史性的基督教問題.
我們都知道.

$^{241}$神進入了文化.
而要做出這個表達.
並不簡單.
雖然它的熟悉.
可能會提醒我們不然.
當神進入了人類的經驗.
雖然與神同在.
但他並不認為與神同在.
是一種能夠被掌握的東西.
他卻把自己放棄了.
並在人類形態中找到自己.
他謙卑了自己.
甚至死亡也變得全面的遵循.
因此神極度地祈禱他.
並給他一個名字.
是超過每一個名字的.
就是在耶穌的名字之下.
每一隻腿都會靠近.
每一張嘴巴都會自認.
耶穌基督是主的.
祂是神父的榮耀.
在理論上.
我們要問的是.
我們現在要面對的.
是我們身體的思想問題.
我們現在要問的是.
我們身體的思想問題.
我們現在要問的是.
我們身體的思想問題.
我們現在要問的是.
我們身體的思想問題.
我們現在要問的是.
我們身體的思想問題.
我們現在要問的是.
我們身體的思想問題.
我們現在要問的是.
我們身體的思想問題.
我們現在要問的是.
我們身體的思想問題.
我們現在要問的是.

$^{281}$我們身體的思想問題.
我們現在要問的是.
我們身體的思想問題.
我們現在要問的是.
我們身體的思想問題.
我們現在要問的是.
我們身體的思想問題.
我們現在要問的是.
我們身體的思想問題.
我們現在要問的是.
我們身體的思想問題.
我們現在要問的是.
我們身體的思想問題.
我們現在要問的是.
我們身體的思想問題.
我們現在要問的是.
我們身體的思想問題.
我們現在要問的是.
我們身體的思想問題.
我們現在要問的是.
我們身體的思想問題.
我們現在要問的是.
我們身體的思想問題.
我們現在要問的是.
我們身體的思想問題.
我們現在要問的是.
我們身體的思想問題.
我們現在要問的是.
我們身體的思想問題.
我們現在要問的是.
我們身體的思想問題.
我們現在要問的是.
我們身體的思想問題.
我們現在要問的是.
我們身體的思想問題.
我們現在要問的是.
我們身體的思想問題.
我們現在要問的是.
我們身體的思想問題.
我們現在要問的是.

$^{321}$我們身體的思想問題.
我們現在要問的是.
我們身體的思想問題.
我們現在要問的是.
我們身體的思想問題.
我們現在要問的是.
我們身體的思想問題.
我們現在要問的是.
我們身體的思想問題.
我們現在要問的是.
我們身體的思想問題.
我們現在要問的是.
我們身體的思想問題.
我們現在要問的是.
我們身體的思想問題.
我們現在要問的是.
我們身體的思想問題.
我們現在要問的是.
我們身體的思想問題.
我們現在要問的是.
我們身體的思想問題.
我們現在要問的是.
我們身體的思想問題.
我們現在要問的是.
我們身體的思想問題.
我們現在要問的是.
我們身體的思想問題.
我們現在要問的是.
我們身體的思想問題.
我們現在要問的是.
我們身體的思想問題.
我們現在要問的是.
我們身體的思想問題.
我們現在要問的是.
我們身體的思想問題.
我們現在要問的是.
我們身體的思想問題.
我們現在要問的是.
我們身體的思想問題.
我們現在要問的是.

$^{361}$我們身體的思想問題.
我們現在要問的是.
我們身體的思想問題.
我們現在要問的是.
我們身體的思想問題.
我們現在要問的是.
我們身體的思想問題.
我們現在要問的是.
我們身體的思想問題.
我們現在要問的是.
我們身體的思想問題.
我們現在要問的是.
我們身體的思想問題.
我們現在要問的是.
我們身體的思想問題.
我們現在要問的是.
我們身體的思想問題.
我們現在要問的是.
我們身體的思想問題.
我們現在要問的是.
我們身體的思想問題.
我們現在要問的是.
我們身體的思想問題.
我們現在要問的是.
我們身體的思想問題.
我們現在要問的是.
我們身體的思想問題.
我們現在要問的是.
我們身體的思想問題.
我們現在要問的是.
我們身體的思想問題.
我們現在要問的是.
我們身體的思想問題.
我們現在要問的是.
我們身體的思想問題.
我們現在要問的是.
我們身體的思想問題.
我們現在要問的是.
我們身體的思想問題.
我們現在要問的是.

$^{401}$我們身體的思想問題.
我們現在要問的是.
我們身體的思想問題.
我們現在要問的是.
我們身體的思想問題.
我們現在要問的是.
我們身體的思想問題.
我們現在要問的是.
我們身體的思想問題.
我們現在要問的是.
我們身體的思想問題.
我們現在要問的是.
我們身體的思想問題.
我們現在要問的是.
我們身體的思想問題.
我們現在要問的是.
我們身體的思想問題.
我們現在要問的是.
我們身體的思想問題.
我們現在要問的是.
我們身體的思想問題.
我們現在要問的是.
我們身體的思想問題.
我們現在要問的是.
我們身體的思想問題.
我們現在要問的是.
我們身體的思想問題.
我們現在要問的是.
我們身體的思想問題.
我們現在要問的是.
我們身體的思想問題.
我們現在要問的是.
我們身體的思想問題.
我們現在要問的是.
我們身體的思想問題.
我們現在要問的是.
我們身體的思想問題.
我們現在要問的是.
我們身體的思想問題.
我們現在要問的是.

$^{441}$我們身體的思想問題.
我們現在要問的是.
我們身體的思想問題.
我們現在要問的是.
我們身體的思想問題.
我們現在要問的是.
我們身體的思想問題.
我們現在要問的是.
我們身體的思想問題.
我們現在要問的是.
我們身體的思想問題.
我們現在要問的是.
我們身體的思想問題.
我們現在要問的是.
我們身體的思想問題.
我們現在要問的是.
我們身體的思想問題.
我們現在要問的是.
我們身體的思想問題.
我們現在要問的是.
我們身體的思想問題.
我們現在要問的是.
我們身體的思想問題.
我們現在要問的是.
我們身體的思想問題.
我們現在要問的是.
我們身體的思想問題.
我們現在要問的是.
我們身體的思想問題.
我們現在要問的是.
我們身體的思想問題.
我們現在要問的是.
我們身體的思想問題.
我們現在要問的是.
我們身體的思想問題.
我們現在要問的是.
我們身體的思想問題.
我們現在要問的是.
我們身體的思想問題.
我們現在要問的是.

$^{481}$我們身體的思想問題.
我們現在要問的是.
我們身體的思想問題.
我們現在要問的是.
我們身體的思想問題.
我們現在要問的是.
我們身體的思想問題.
我們現在要問的是.
我們身體的思想問題.
我們現在要問的是.
我們身體的思想問題.
我們現在要問的是.
我們身體的思想問題.
我們現在要問的是.
我們身體的思想問題.
我們現在要問的是.
我們身體的思想問題.
我們現在要問的是.
我們身體的思想問題.
我們現在要問的是.
我們身體的思想問題.
我們現在要問的是.
我們身體的思想問題.
我們現在要問的是.
我們身體的思想問題.
我們現在要問的是.
我們身體的思想問題.
我們現在要問的是.
我們身體的思想問題.
我們現在要問的是.
我們身體的思想問題.
我們現在要問的是.
我們身體的思想問題.
我們現在要問的是.
我們身體的思想問題.
我們現在要問的是.
我們身體的思想問題.
我們現在要問的是.
我們身體的思想問題.
我們現在要問的是.

$^{521}$我們身體的思想問題.
我們現在要問的是.
我們身體的思想問題.
我們現在要問的是.
我們身體的思想問題.
我們現在要問的是.
我們身體的思想問題.
我們現在要問的是.
我們身體的思想問題.
我們現在要問的是.
我們身體的思想問題.
我們現在要問的是.
我們身體的思想問題.
我們現在要問的是.
我們身體的思想問題.
我們現在要問的是.
我們身體的思想問題.
我們現在要問的是.
我們身體的思想問題.
我們現在要問的是.
我們身體的思想問題.
我們現在要問的是.
我們身體的思想問題.
我們現在要問的是.
我們身體的思想問題.
我們現在要問的是.
我們身體的思想問題.
我們現在要問的是.
我們身體的思想問題.
我們現在要問的是.
我們身體的思想問題.
我們現在要問的是.
我們身體的思想問題.
我們現在要問的是.
我們身體的思想問題.
我們現在要問的是.
我們身體的思想問題.
我們現在要問的是.
我們身體的思想問題.
我們現在要問的是.

$^{561}$我們身體的思想問題.
我們現在要問的是.
我們身體的思想問題.
我們現在要問的是.
我們身體的思想問題.
我們現在要問的是.
我們身體的思想問題.
我們現在要問的是.
我們身體的思想問題.
我們現在要問的是.
我們身體的思想問題.
我們現在要問的是.
我們身體的思想問題.
我們現在要問的是.
我們身體的思想問題.
我們現在要問的是.
我們身體的思想問題.
我們現在要問的是.
我們身體的思想問題.
我們現在要問的是.
我們身體的思想問題.
我們現在要問的是.
我們身體的思想問題.
我們現在要問的是.
我們身體的思想問題.
我們現在要問的是.
我們身體的思想問題.
我們現在要問的是.
我們身體的思想問題.
我們現在要問的是.
我們身體的思想問題.
我們現在要問的是.
我們身體的思想問題.
我們現在要問的是.
我們身體的思想問題.
我們現在要問的是.
我們身體的思想問題.
我們現在要問的是.
我們身體的思想問題.
我們現在要問的是.

$^{601}$我們身體的思想問題.
我們現在要問的是.
我們身體的思想問題.
我們現在要問的是.
我們身體的思想問題.
我們現在要問的是.
我們身體的思想問題.
我們現在要問的是.
我們身體的思想問題.
我們現在要問的是.
我們身體的思想問題.
我們現在要問的是.
我們身體的思想問題.
我們現在要問的是.
我們身體的思想問題.
我們現在要問的是.
我們身體的思想問題.
我們現在要問的是.
我們身體的思想問題.
我們現在要問的是.
我們身體的思想問題.
我們現在要問的是.
我們身體的思想問題.
我們現在要問的是.
我們身體的思想問題.
我們現在要問的是.
我們身體的思想問題.
我們現在要問的是.
我們身體的思想問題.
我們現在要問的是.
我們身體的思想問題.
我們現在要問的是.
我們身體的思想問題.
我們現在要問的是.
我們身體的思想問題.
我們現在要問的是.
我們身體的思想問題.
我們現在要問的是.
我們身體的思想問題.
我們現在要問的是.

$^{641}$我們身體的思想問題.
我們現在要問的是.
我們身體的思想問題.
我們現在要問的是.
我們身體的思想問題.
我們現在要問的是.
我們身體的思想問題.
我們現在要問的是.
我們身體的思想問題.
我們現在要問的是.
我們身體的思想問題.
我們現在要問的是.
我們身體的思想問題.
我們現在要問的是.
我們身體的思想問題.
我們現在要問的是.
我們身體的思想問題.
我們現在要問的是.
我們身體的思想問題.
我們現在要問的是.
我們身體的思想問題.
我們現在要問的是.
我們身體的思想問題.
我們現在要問的是.
我們身體的思想問題.
我們現在要問的是.
我們身體的思想問題.
我們現在要問的是.
我們身體的思想問題.
我們現在要問的是.
我們身體的思想問題.
我們現在要問的是.
我們身體的思想問題.
我們現在要問的是.
我們身體的思想問題.
我們現在要問的是.
我們身體的思想問題.
我們現在要問的是.
我們身體的思想問題.
我們現在要問的是.

$^{681}$我們身體的思想問題.
我們現在要問的是.
我們身體的思想問題.
我們現在要問的是.
我們身體的思想問題.
我們現在要問的是.
我們身體的思想問題.
我們現在要問的是.
我們身體的思想問題.
我們現在要問的是.
我們身體的思想問題.
我們現在要問的是.
我們身體的思想問題.
我們現在要問的是.
我們身體的思想問題.
我們現在要問的是.
我們身體的思想問題.
我們現在要問的是.
我們身體的思想問題.
我們現在要問的是.
我們身體的思想問題.
我們現在要問的是.
我們身體的思想問題.
我們現在要問的是.
我們身體的思想問題.
我們現在要問的是.
我們身體的思想問題.
我們現在要問的是.
我們身體的思想問題.
我們現在要問的是.
我們身體的思想問題.
我們現在要問的是.
我們身體的思想問題.
我們現在要問的是.
我們身體的思想問題.
我們現在要問的是.
我們身體的思想問題.
我們現在要問的是.
我們身體的思想問題.
我們現在要問的是.

$^{721}$我們身體的思想問題.
我們現在要問的是.
我們身體的思想問題.
我們現在要問的是.
我們身體的思想問題.
我們現在要問的是.
我們身體的思想問題.
我們現在要問的是.
我們身體的思想問題.
我們現在要問的是.
我們身體的思想問題.
我們現在要問的是.
我們身體的思想問題.
我們現在要問的是.
我們身體的思想問題.
我們現在要問的是.
我們身體的思想問題.
我們現在要問的是.
我們身體的思想問題.
我們現在要問的是.
我們身體的思想問題.
我們現在要問的是.
我們身體的思想問題.
我們現在要問的是.
我們身體的思想問題.
我們現在要問的是.
我們身體的思想問題.
我們現在要問的是.
我們身體的思想問題.
我們現在要問的是.
我們身體的思想問題.
我們現在要問的是.
我們身體的思想問題.
我們現在要問的是.
我們身體的思想問題.
我們現在要問的是.
我們身體的思想問題.
我們現在要問的是.
我們身體的思想問題.
我們現在要問的是.

$^{761}$我們身體的思想問題.
我們現在要問的是.
我們身體的思想問題.
我們現在要問的是.
我們身體的思想問題.
我們現在要問的是.
我們身體的思想問題.
我們現在要問的是.
我們身體的思想問題.
我們現在要問的是.
我們身體的思想問題.
我們現在要問的是.
我們身體的思想問題.
我們現在要問的是.
我們身體的思想問題.
我們現在要問的是.
我們身體的思想問題.
我們現在要問的是.
我們身體的思想問題.
我們現在要問的是.
我們身體的思想問題.
我們現在要問的是.
我們身體的思想問題.
我們現在要問的是.
我們身體的思想問題.
我們現在要問的是.
我們身體的思想問題.
我們現在要問的是.
我們身體的思想問題.
我們現在要問的是.
我們身體的思想問題.
我們現在要問的是.
我們身體的思想問題.
我們現在要問的是.
我們身體的思想問題.
我們現在要問的是.
我們身體的思想問題.
我們現在要問的是.
我們身體的思想問題.
我們現在要問的是.

$^{801}$我們身體的思想問題.
我們現在要問的是.
我們身體的思想問題.
我們現在要問的是.
我們身體的思想問題.
我們現在要問的是.
我們身體的思想問題.
我們現在要問的是.
我們身體的思想問題.
我們現在要問的是.
我們身體的思想問題.
我們現在要問的是.
我們身體的思想問題.
我們現在要問的是.
我們身體的思想問題.
我們現在要問的是.
我們身體的思想問題.
我們現在要問的是.
我們身體的思想問題.
我們現在要問的是.
我們身體的思想問題.
我們現在要問的是.
我們身體的思想問題.
我們現在要問的是.
我們身體的思想問題.
我們現在要問的是.
我們身體的思想問題.
我們現在要問的是.
我們身體的思想問題.
我們現在要問的是.
我們身體的思想問題.
我們現在要問的是.
我們身體的思想問題.
我們現在要問的是.
我們身體的思想問題.
我們現在要問的是.
我們身體的思想問題.
我們現在要問的是.
我們身體的思想問題.
我們現在要問的是.

$^{841}$我們身體的思想問題.
我們現在要問的是.
我們身體的思想問題.
我們現在要問的是.
我們身體的思想問題.
我們現在要問的是.
我們身體的思想問題.
我們現在要問的是.
我們身體的思想問題.
我們現在要問的是.
我們身體的思想問題.
我們現在要問的是.
我們身體的思想問題.
我們現在要問的是.
我們身體的思想問題.
我們現在要問的是.
我們身體的思想問題.
我們現在要問的是.
我們身體的思想問題.
我們現在要問的是.
我們身體的思想問題.
我們現在要問的是.
我們身體的思想問題.
我們現在要問的是.
我們身體的思想問題.
我們現在要問的是.
我們身體的思想問題.
我們現在要問的是.
我們身體的思想問題.
我們現在要問的是.
我們身體的思想問題.
我們現在要問的是.
我們身體的思想問題.
我們現在要問的是.
我們身體的思想問題.
我們現在要問的是.
我們身體的思想問題.
我們現在要問的是.
我們身體的思想問題.
我們現在要問的是.

$^{881}$我們身體的思想問題.
我們現在要問的是.
我們身體的思想問題.
我們現在要問的是.
我們身體的思想問題.
我們現在要問的是.
我們身體的思想問題.
我們現在要問的是.
我們身體的思想問題.
我們現在要問的是.
我們身體的思想問題.
我們現在要問的是.
我們身體的思想問題.
我們現在要問的是.
我們身體的思想問題.
我們現在要問的是.
我們身體的思想問題.
我們現在要問的是.
我們身體的思想問題.
我們現在要問的是.
我們身體的思想問題.
我們現在要問的是.
我們身體的思想問題.
我們現在要問的是.
我們身體的思想問題.
我們現在要問的是.
我們身體的思想問題.
我們現在要問的是.
我們身體的思想問題.
我們現在要問的是.
我們身體的思想問題.
我們現在要問的是.
我們身體的思想問題.
我們現在要問的是.
我們身體的思想問題.
我們現在要問的是.
我們身體的思想問題.
我們現在要問的是.
我們身體的思想問題.
我們現在要問的是.

$^{921}$我們身體的思想問題.
我們現在要問的是.
我們身體的思想問題.
我們現在要問的是.
我們身體的思想問題.
我們現在要問的是.
我們身體的思想問題.
我們現在要問的是.
我們身體的思想問題.
我們現在要問的是.
我們身體的思想問題.
我們現在要問的是.
我們身體的思想問題.
我們現在要問的是.
我們身體的思想問題.
我們現在要問的是.
我們身體的思想問題.
我們現在要問的是.
我們身體的思想問題.
我們現在要問的是.
我們身體的思想問題.
我們現在要問的是.
我們身體的思想問題.
我們現在要問的是.
我們身體的思想問題.
我們現在要問的是.
我們身體的思想問題.
我們現在要問的是.
我們身體的思想問題.
我們現在要問的是.
我們身體的思想問題.
我們現在要問的是.
我們身體的思想問題.
我們現在要問的是.
我們身體的思想問題.
我們現在要問的是.
我們身體的思想問題.
我們現在要問的是.
我們身體的思想問題.
我們現在要問的是.

$^{961}$我們身體的思想問題.
我們現在要問的是.
我們身體的思想問題.
我們現在要問的是.
我們身體的思想問題.
我們現在要問的是.
我們身體的思想問題.
我們現在要問的是.
我們身體的思想問題.
我們現在要問的是.
我們身體的思想問題.
我們現在要問的是.
我們身體的思想問題.
我們現在要問的是.
我們身體的思想問題.
我們現在要問的是.
我們身體的思想問題.
我們現在要問的是.
我們身體的思想問題.
我們現在要問的是.
我們身體的思想問題.
我們現在要問的是.
我們身體的思想問題.
我們現在要問的是.
我們身體的思想問題.
我們現在要問的是.
我們身體的思想問題.
我們現在要問的是.
我們身體的思想問題.
我們現在要問的是.
我們身體的思想問題.
我們現在要問的是.
我們身體的思想問題.
我們現在要問的是.
我們身體的思想問題.
我們現在要問的是.
我們身體的思想問題.
我們現在要問的是.
我們身體的思想問題.
我們現在要問的是.

$^{1001}$我們身體的思想問題.
我們現在要問的是.
我們身體的思想問題.
我們現在要問的是.
我們身體的思想問題.
我們現在要問的是.
我們身體的思想問題.
我們現在要問的是.
我們身體的思想問題.
我們現在要問的是.
我們身體的思想問題.
我們現在要問的是.
我們身體的思想問題.
我們現在要問的是.
我們身體的思想問題.
我們現在要問的是.
我們身體的思想問題.
我們現在要問的是.
我們身體的思想問題.
我們現在要問的是.
我們身體的思想問題.
我們現在要問的是.
我們身體的思想問題.
我們現在要問的是.
我們身體的思想問題.
我們現在要問的是.
我們身體的思想問題.
我們現在要問的是.
我們身體的思想問題.
我們現在要問的是.
我們身體的思想問題.
我們現在要問的是.
我們身體的思想問題.
我們現在要問的是.
我們身體的思想問題.
我們現在要問的是.
我們身體的思想問題.
我們現在要問的是.
我們身體的思想問題.
我們現在要問的是.

$^{1041}$我們身體的思想問題.
我們現在要問的是.
我們身體的思想問題.
我們現在要問的是.
我們身體的思想問題.
我們現在要問的是.
我們身體的思想問題.
我們現在要問的是.
我們身體的思想問題.
我們現在要問的是.
我們身體的思想問題.
我們現在要問的是.
我們身體的思想問題.
我們現在要問的是.
我們身體的思想問題.
我們現在要問的是.
我們身體的思想問題.
我們現在要問的是.
我們身體的思想問題.
我們現在要問的是.
我們身體的思想問題.
我們現在要問的是.
我們身體的思想問題.
我們現在要問的是.
我們身體的思想問題.
我們現在要問的是.
我們身體的思想問題.
我們現在要問的是.
我們身體的思想問題.
我們現在要問的是.
我們身體的思想問題.
我們現在要問的是.
我們身體的思想問題.
我們現在要問的是.
我們身體的思想問題.
我們現在要問的是.
我們身體的思想問題.
我們現在要問的是.
我們身體的思想問題.
我們現在要問的是.

$^{1081}$我們身體的思想問題.
我們現在要問的是.
我們身體的思想問題.
我們現在要問的是.
我們身體的思想問題.
我們現在要問的是.
我們身體的思想問題.
我們現在要問的是.
我們身體的思想問題.
我們現在要問的是.
我們身體的思想問題.
我們現在要問的是.
我們身體的思想問題.
我們現在要問的是.
我們身體的思想問題.
我們現在要問的是.
我們身體的思想問題.
我們現在要問的是.
我們身體的思想問題.
我們現在要問的是.
我們身體的思想問題.
我們現在要問的是.
我們身體的思想問題.
我們現在要問的是.
我們身體的思想問題.
我們現在要問的是.
我們身體的思想問題.
我們現在要問的是.
我們身體的思想問題.
我們現在要問的是.
我們身體的思想問題.
我們現在要問的是.
我們身體的思想問題.
我們現在要問的是.
我們身體的思想問題.
我們現在要問的是.
我們身體的思想問題.
我們現在要問的是.
我們身體的思想問題.
我們現在要問的是.

$^{1121}$我們身體的思想問題.
我們現在要問的是.
我們身體的思想問題.
我們現在要問的是.
我們身體的思想問題.
我們現在要問的是.
我們身體的思想問題.
我們現在要問的是.
我們身體的思想問題.
我們現在要問的是.
我們身體的思想問題.
我們現在要問的是.
我們身體的思想問題.
我們現在要問的是.
我們身體的思想問題.
我們現在要問的是.
我們身體的思想問題.
我們現在要問的是.
我們身體的思想問題.
我們現在要問的是.
我們身體的思想問題.
我們現在要問的是.
我們身體的思想問題.
我們現在要問的是.
我們身體的思想問題.
我們現在要問的是.
我們身體的思想問題.
我們現在要問的是.
我們身體的思想問題.
我們現在要問的是.
我們身體的思想問題.
我們現在要問的是.
我們身體的思想問題.
我們現在要問的是.
我們身體的思想問題.
我們現在要問的是.
我們身體的思想問題.
我們現在要問的是.
我們身體的思想問題.
我們現在要問的是.

$^{1161}$我們身體的思想問題.
我們現在要問的是.
我們身體的思想問題.
我們現在要問的是.
我們身體的思想問題.
我們現在要問的是.
我們身體的思想問題.
我們現在要問的是.
我們身體的思想問題.
我們現在要問的是.
我們身體的思想問題.
我們現在要問的是.
我們身體的思想問題.
我們現在要問的是.
我們身體的思想問題.
我們現在要問的是.
我們身體的思想問題.
我們現在要問的是.
我們身體的思想問題.
我們現在要問的是.
我們身體的思想問題.
我們現在要問的是.
我們身體的思想問題.
我們現在要問的是.
我們身體的思想問題.
我們現在要問的是.
我們身體的思想問題.
我們現在要問的是.
我們身體的思想問題.
我們現在要問的是.
我們身體的思想問題.
我們現在要問的是.
我們身體的思想問題.
我們現在要問的是.
我們身體的思想問題.
我們現在要問的是.
我們身體的思想問題.
我們現在要問的是.
我們身體的思想問題.
我們現在要問的是.

$^{1201}$我們身體的思想問題.
我們現在要問的是.
我們身體的思想問題.
我們現在要問的是.
我們身體的思想問題.
我們現在要問的是.
我們身體的思想問題.
我們現在要問的是.
我們身體的思想問題.
我們現在要問的是.
我們身體的思想問題.
我們現在要問的是.
我們身體的思想問題.
我們現在要問的是.
我們身體的思想問題.
我們現在要問的是.
我們身體的思想問題.
我們現在要問的是.
我們身體的思想問題.
我們現在要問的是.
我們身體的思想問題.
我們現在要問的是.
我們身體的思想問題.
我們現在要問的是.
我們身體的思想問題.
我們現在要問的是.
我們身體的思想問題.
我們現在要問的是.
我們身體的思想問題.
我們現在要問的是.
我們身體的思想問題.
我們現在要問的是.
我們身體的思想問題.
我們現在要問的是.
我們身體的思想問題.
我們現在要問的是.
我們身體的思想問題.
我們現在要問的是.
我們身體的思想問題.
我們現在要問的是.

$^{1241}$我們身體的思想問題.
我們現在要問的是.
我們身體的思想問題.
我們現在要問的是.
我們身體的思想問題.
我們現在要問的是.
我們身體的思想問題.
我們現在要問的是.
我們身體的思想問題.
我們現在要問的是.
我們身體的思想問題.
我們現在要問的是.
我們身體的思想問題.
我們現在要問的是.
我們身體的思想問題.
我們現在要問的是.
我們身體的思想問題.
我們現在要問的是.
我們身體的思想問題.
我們現在要問的是.
我們身體的思想問題.
我們現在要問的是.
我們身體的思想問題.
我們現在要問的是.
我們身體的思想問題.
我們現在要問的是.
我們身體的思想問題.
我們現在要問的是.
我們身體的思想問題.
我們現在要問的是.
我們身體的思想問題.
我們現在要問的是.
我們身體的思想問題.
我們現在要問的是.
我們身體的思想問題.
我們現在要問的是.
我們身體的思想問題.
我們現在要問的是.
我們身體的思想問題.
我們現在要問的是.

$^{1281}$我們身體的思想問題.
我們現在要問的是.
我們身體的思想問題.
我們現在要問的是.
我們身體的思想問題.
我們現在要問的是.
我們身體的思想問題.
我們現在要問的是.
我們身體的思想問題.
我們現在要問的是.
我們身體的思想問題.
我們現在要問的是.
我們身體的思想問題.
我們現在要問的是.
我們身體的思想問題.
我們現在要問的是.
我們身體的思想問題.
我們現在要問的是.
我們身體的思想問題.
我們現在要問的是.
我們身體的思想問題.
我們現在要問的是.
我們身體的思想問題.
我們現在要問的是.
我們身體的思想問題.
我們現在要問的是.
我們身體的思想問題.
我們現在要問的是.
我們身體的思想問題.
我們現在要問的是.
我們身體的思想問題.
我們現在要問的是.
我們身體的思想問題.
我們現在要問的是.
我們身體的思想問題.
我們現在要問的是.
我們身體的思想問題.
我們現在要問的是.
我們身體的思想問題.
我們現在要問的是.

$^{1321}$我們身體的思想問題.
我們現在要問的是.
我們身體的思想問題.
我們現在要問的是.
我們身體的思想問題.
我們現在要問的是.
我們身體的思想問題.
我們現在要問的是.
我們身體的思想問題.
我們現在要問的是.
我們身體的思想問題.
我們現在要問的是.
我們身體的思想問題.
我們現在要問的是.
我們身體的思想問題.
我們現在要問的是.
我們身體的思想問題.
我們現在要問的是.
我們身體的思想問題.
我們現在要問的是.
我們身體的思想問題.
我們現在要問的是.
我們身體的思想問題.
我們現在要問的是.
我們身體的思想問題.
我們現在要問的是.
我們身體的思想問題.
我們現在要問的是.
我們身體的思想問題.
我們現在要問的是.
我們身體的思想問題.
我們現在要問的是.
我們身體的思想問題.
我們現在要問的是.
我們身體的思想問題.
我們現在要問的是.
我們身體的思想問題.
我們現在要問的是.
我們身體的思想問題.
我們現在要問的是.

$^{1361}$我們身體的思想問題.
我們現在要問的是.
我們身體的思想問題.
我們現在要問的是.
我們身體的思想問題.
我們現在要問的是.
我們身體的思想問題.
我們現在要問的是.
我們身體的思想問題.
我們現在要問的是.
我們身體的思想問題.
我們現在要問的是.
我們身體的思想問題.
我們現在要問的是.
我們身體的思想問題.
我們現在要問的是.
我們身體的思想問題.
我們現在要問的是.
我們身體的思想問題.
我們現在要問的是.
我們身體的思想問題.
我們現在要問的是.
我們身體的思想問題.
我們現在要問的是.
我們身體的思想問題.
我們現在要問的是.
我們身體的思想問題.
我們現在要問的是.
我們身體的思想問題.
我們現在要問的是.
我們身體的思想問題.
我們現在要問的是.
我們身體的思想問題.
我們現在要問的是.
我們身體的思想問題.
我們現在要問的是.
我們身體的思想問題.
我們現在要問的是.
我們身體的思想問題.
我們現在要問的是.

$^{1401}$我們身體的思想問題.
我們現在要問的是.
我們身體的思想問題.
我們現在要問的是.
我們身體的思想問題.
我們現在要問的是.
我們身體的思想問題.
我們現在要問的是.
我們身體的思想問題.
我們現在要問的是.
我們身體的思想問題.
我們現在要問的是.
我們身體的思想問題.
我們現在要問的是.
我們身體的思想問題.
我們現在要問的是.
我們身體的思想問題.
我們現在要問的是.
我們身體的思想問題.
我們現在要問的是.
我們身體的思想問題.
我們現在要問的是.
我們身體的思想問題.
我們現在要問的是.
我們身體的思想問題.
我們現在要問的是.
我們身體的思想問題.
我們現在要問的是.
我們身體的思想問題.
我們現在要問的是.
我們身體的思想問題.
我們現在要問的是.
我們身體的思想問題.
我們現在要問的是.
我們身體的思想問題.
我們現在要問的是.
我們身體的思想問題.
我們現在要問的是.
我們身體的思想問題.
我們現在要問的是.

$^{1441}$我們身體的思想問題.
我們現在要問的是.
我們身體的思想問題.
我們現在要問的是.
我們身體的思想問題.
我們現在要問的是.
我們身體的思想問題.
我們現在要問的是.
我們身體的思想問題.
我們現在要問的是.
我們身體的思想問題.
我們現在要問的是.
我們身體的思想問題.
我們現在要問的是.
我們身體的思想問題.
我們現在要問的是.
我們身體的思想問題.
我們現在要問的是.
我們身體的思想問題.
我們現在要問的是.
我們身體的思想問題.
我們現在要問的是.
我們身體的思想問題.
我們現在要問的是.
我們身體的思想問題.
我們現在要問的是.
我們身體的思想問題.
我們現在要問的是.
我們身體的思想問題.
我們現在要問的是.
我們身體的思想問題.
我們現在要問的是.
我們身體的思想問題.
我們現在要問的是.
我們身體的思想問題.
我們現在要問的是.
我們身體的思想問題.
我們現在要問的是.
我們身體的思想問題.
我們現在要問的是.

$^{1481}$我們身體的思想問題.
我們現在要問的是.
我們身體的思想問題.
我們現在要問的是.
我們身體的思想問題.
我們現在要問的是.
我們身體的思想問題.
我們現在要問的是.
我們身體的思想問題.
我們現在要問的是.
我們身體的思想問題.
我們現在要問的是.
我們身體的思想問題.
我們現在要問的是.
我們身體的思想問題.
我們現在要問的是.
我們身體的思想問題.
我們現在要問的是.
我們身體的思想問題.
我們現在要問的是.
我們身體的思想問題.
我們現在要問的是.
我們身體的思想問題.
我們現在要問的是.
我們身體的思想問題.
我們現在要問的是.
我們身體的思想問題.
我們現在要問的是.
我們身體的思想問題.
我們現在要問的是.
我們身體的思想問題.
我們現在要問的是.
我們身體的思想問題.
我們現在要問的是.
我們身體的思想問題.
我們現在要問的是.
我們身體的思想問題.
我們現在要問的是.
我們身體的思想問題.
我們現在要問的是.

$^{1521}$我們身體的思想問題.
我們現在要問的是.
我們身體的思想問題.
我們現在要問的是.
我們身體的思想問題.
我們現在要問的是.
我們身體的思想問題.
我們現在要問的是.
我們身體的思想問題.
我們現在要問的是.
我們身體的思想問題.
我們現在要問的是.
我們身體的思想問題.
我們現在要問的是.
我們身體的思想問題.
我們現在要問的是.
我們身體的思想問題.
我們現在要問的是.
我們身體的思想問題.
我們現在要問的是.
我們身體的思想問題.
我們現在要問的是.
我們身體的思想問題.
我們現在要問的是.
我們身體的思想問題.
我們現在要問的是.
我們身體的思想問題.
我們現在要問的是.
我們身體的思想問題.
我們現在要問的是.
我們身體的思想問題.
我們現在要問的是.
我們身體的思想問題.
我們現在要問的是.
我們身體的思想問題.
我們現在要問的是.
我們身體的思想問題.
我們現在要問的是.
我們身體的思想問題.
我們現在要問的是.

$^{1561}$我們身體的思想問題.
我們現在要問的是.
我們身體的思想問題.
我們現在要問的是.
我們身體的思想問題.
我們現在要問的是.
我們身體的思想問題.
我們現在要問的是.
我們身體的思想問題.
我們現在要問的是.
我們身體的思想問題.
我們現在要問的是.
我們身體的思想問題.
我們現在要問的是.
我們身體的思想問題.
我們現在要問的是.
我們身體的思想問題.
我們現在要問的是.
我們身體的思想問題.
我們現在要問的是.
我們身體的思想問題.
我們現在要問的是.
我們身體的思想問題.
我們現在要問的是.
我們身體的思想問題.
我們現在要問的是.
我們身體的思想問題.
我們現在要問的是.
我們身體的思想問題.
我們現在要問的是.
我們身體的思想問題.
我們現在要問的是.
我們身體的思想問題.
我們現在要問的是.
我們身體的思想問題.
我們現在要問的是.
我們身體的思想問題.
我們現在要問的是.
我們身體的思想問題.
我們現在要問的是.

$^{1601}$我們身體的思想問題.
我們現在要問的是.
我們身體的思想問題.
我們現在要問的是.
我們身體的思想問題.
我們現在要問的是.
我們身體的思想問題.
我們現在要問的是.
我們身體的思想問題.
我們現在要問的是.
我們身體的思想問題.
我們現在要問的是.
我們身體的思想問題.
我們現在要問的是.
我們身體的思想問題.
我們現在要問的是.
我們身體的思想問題.
我們現在要問的是.
我們身體的思想問題.
我們現在要問的是.
我們身體的思想問題.
我們現在要問的是.
我們身體的思想問題.
我們現在要問的是.
我們身體的思想問題.
我們現在要問的是.
我們身體的思想問題.
我們現在要問的是.
我們身體的思想問題.
我們現在要問的是.
我們身體的思想問題.
我們現在要問的是.
我們身體的思想問題.
我們現在要問的是.
我們身體的思想問題.
我們現在要問的是.
我們身體的思想問題.
我們現在要問的是.
我們身體的思想問題.
我們現在要問的是.

$^{1641}$我們身體的思想問題.
我們現在要問的是.
我們身體的思想問題.
我們現在要問的是.
我們身體的思想問題.
我們現在要問的是.
我們身體的思想問題.
我們現在要問的是.
我們身體的思想問題.
我們現在要問的是.
我們身體的思想問題.
我們現在要問的是.
我們身體的思想問題.
我們現在要問的是.
我們身體的思想問題.
我們現在要問的是.
我們身體的思想問題.
我們現在要問的是.
我們身體的思想問題.
我們現在要問的是.
我們身體的思想問題.
我們現在要問的是.
我們身體的思想問題.
我們現在要問的是.
我們身體的思想問題.
我們現在要問的是.
我們身體的思想問題.
我們現在要問的是.
我們身體的思想問題.
我們現在要問的是.
我們身體的思想問題.
我們現在要問的是.
我們身體的思想問題.
我們現在要問的是.
我們身體的思想問題.
我們現在要問的是.
我們身體的思想問題.
我們現在要問的是.
我們身體的思想問題.
我們現在要問的是.

$^{1681}$我們身體的思想問題.
我們現在要問的是.
我們身體的思想問題.
我們現在要問的是.
我們身體的思想問題.
我們現在要問的是.
我們身體的思想問題.
我們現在要問的是.
我們身體的思想問題.
我們現在要問的是.
我們身體的思想問題.
我們現在要問的是.
我們身體的思想問題.
我們現在要問的是.
我們身體的思想問題.
我們現在要問的是.
我們身體的思想問題.
我們現在要問的是.
我們身體的思想問題.
我們現在要問的是.
我們身體的思想問題.
我們現在要問的是.
我們身體的思想問題.
我們現在要問的是.
我們身體的思想問題.
我們現在要問的是.
我們身體的思想問題.
我們現在要問的是.
我們身體的思想問題.
我們現在要問的是.
我們身體的思想問題.
我們現在要問的是.
我們身體的思想問題.
我們現在要問的是.
我們身體的思想問題.
我們現在要問的是.
我們身體的思想問題.
我們現在要問的是.
我們身體的思想問題.
我們現在要問的是.

$^{1721}$我們身體的思想問題.
我們現在要問的是.
我們身體的思想問題.
我們現在要問的是.
我們身體的思想問題.
我們現在要問的是.
我們身體的思想問題.
我們現在要問的是.
我們身體的思想問題.
我們現在要問的是.
我們身體的思想問題.
我們現在要問的是.
我們身體的思想問題.
我們現在要問的是.
我們身體的思想問題.
我們現在要問的是.
我們身體的思想問題.
我們現在要問的是.
我們身體的思想問題.
我們現在要問的是.
我們身體的思想問題.
我們現在要問的是.
我們身體的思想問題.
我們現在要問的是.
我們身體的思想問題.
我們現在要問的是.
我們身體的思想問題.
我們現在要問的是.
我們身體的思想問題.
我們現在要問的是.
我們身體的思想問題.
我們現在要問的是.
我們身體的思想問題.
我們現在要問的是.
我們身體的思想問題.
我們現在要問的是.
我們身體的思想問題.
我們現在要問的是.
我們身體的思想問題.
我們現在要問的是.

$^{1761}$我們身體的思想問題.
我們現在要問的是.
我們身體的思想問題.
我們現在要問的是.
我們身體的思想問題.
我們現在要問的是.
我們身體的思想問題.
我們現在要問的是.
我們身體的思想問題.
我們現在要問的是.
我們身體的思想問題.
我們現在要問的是.
我們身體的思想問題.
我們現在要問的是.
我們身體的思想問題.
我們現在要問的是.
我們身體的思想問題.
我們現在要問的是.
我們身體的思想問題.
我們現在要問的是.
我們身體的思想問題.
我們現在要問的是.
我們身體的思想問題.
我們現在要問的是.
我們身體的思想問題.
我們現在要問的是.
我們身體的思想問題.
我們現在要問的是.
我們身體的思想問題.
我們現在要問的是.
我們身體的思想問題.
我們現在要問的是.
我們身體的思想問題.
我們現在要問的是.
我們身體的思想問題.
我們現在要問的是.
我們身體的思想問題.
我們現在要問的是.
我們身體的思想問題.
我們現在要問的是.

$^{1801}$我們身體的思想問題.
我們現在要問的是.
我們身體的思想問題.
我們現在要問的是.
我們身體的思想問題.
我們現在要問的是.
我們身體的思想問題.
我們現在要問的是.
我們身體的思想問題.
我們現在要問的是.
我們身體的思想問題.
我們現在要問的是.
我們身體的思想問題.
我們現在要問的是.
我們身體的思想問題.
我們現在要問的是.
我們身體的思想問題.
我們現在要問的是.
我們身體的思想問題.
我們現在要問的是.
我們身體的思想問題.
我們現在要問的是.
我們身體的思想問題.
我們現在要問的是.
我們身體的思想問題.
我們現在要問的是.
我們身體的思想問題.
我們現在要問的是.
我們身體的思想問題.
我們現在要問的是.
我們身體的思想問題.
我們現在要問的是.
我們身體的思想問題.
我們現在要問的是.
我們身體的思想問題.
我們現在要問的是.
我們身體的思想問題.
我們現在要問的是.
我們身體的思想問題.
我們現在要問的是.

$^{1841}$我們身體的思想問題.
我們現在要問的是.
我們身體的思想問題.
我們現在要問的是.
我們身體的思想問題.
我們現在要問的是.
我們身體的思想問題.
我們現在要問的是.
我們身體的思想問題.
我們現在要問的是.
我們身體的思想問題.
我們現在要問的是.
我們身體的思想問題.
我們現在要問的是.
我們身體的思想問題.
我們現在要問的是.
我們身體的思想問題.
我們現在要問的是.
我們身體的思想問題.
我們現在要問的是.
我們身體的思想問題.
我們現在要問的是.
我們身體的思想問題.
我們現在要問的是.
我們身體的思想問題.
我們現在要問的是.
我們身體的思想問題.
我們現在要問的是.
我們身體的思想問題.
我們現在要問的是.
我們身體的思想問題.
我們現在要問的是.
我們身體的思想問題.
我們現在要問的是.
我們身體的思想問題.
我們現在要問的是.
我們身體的思想問題.
我們現在要問的是.
我們身體的思想問題.
我們現在要問的是.

$^{1881}$我們身體的思想問題.
我們現在要問的是.
我們身體的思想問題.
我們現在要問的是.
我們身體的思想問題.
我們現在要問的是.
我們身體的思想問題.
我們現在要問的是.
我們身體的思想問題.
我們現在要問的是.
我們身體的思想問題.
我們現在要問的是.
我們身體的思想問題.
我們現在要問的是.
我們身體的思想問題.
我們現在要問的是.
我們身體的思想問題.
我們現在要問的是.
我們身體的思想問題.
我們現在要問的是.
我們身體的思想問題.
我們現在要問的是.
我們身體的思想問題.
我們現在要問的是.
我們身體的思想問題.
我們現在要問的是.
我們身體的思想問題.
我們現在要問的是.
我們身體的思想問題.
我們現在要問的是.
我們身體的思想問題.
我們現在要問的是.
我們身體的思想問題.
我們現在要問的是.
我們身體的思想問題.
我們現在要問的是.
我們身體的思想問題.
我們現在要問的是.
我們身體的思想問題.
我們現在要問的是.

$^{1921}$我們身體的思想問題.
我們現在要問的是.
我們身體的思想問題.
我們現在要問的是.
我們身體的思想問題.
我們現在要問的是.
我們身體的思想問題.
我們現在要問的是.
我們身體的思想問題.
我們現在要問的是.
我們身體的思想問題.
我們現在要問的是.
我們身體的思想問題.
我們現在要問的是.
我們身體的思想問題.
我們現在要問的是.
我們身體的思想問題.
我們現在要問的是.
我們身體的思想問題.
我們現在要問的是.
我們身體的思想問題.
我們現在要問的是.
我們身體的思想問題.
我們現在要問的是.
我們身體的思想問題.
我們現在要問的是.
我們身體的思想問題.
我們現在要問的是.
我們身體的思想問題.
我們現在要問的是.
我們身體的思想問題.
我們現在要問的是.
我們身體的思想問題.
我們現在要問的是.
我們身體的思想問題.
我們現在要問的是.
我們身體的思想問題.
我們現在要問的是.
我們身體的思想問題.
我們現在要問的是.

$^{1961}$我們身體的思想問題.
我們現在要問的是.
我們身體的思想問題.
我們現在要問的是.
我們身體的思想問題.
我們現在要問的是.
我們身體的思想問題.
我們現在要問的是.
我們身體的思想問題.
我們現在要問的是.
我們身體的思想問題.
我們現在要問的是.
我們身體的思想問題.
我們現在要問的是.
我們身體的思想問題.
我們現在要問的是.
我們身體的思想問題.
我們現在要問的是.
我們身體的思想問題.
我們現在要問的是.
我們身體的思想問題.
我們現在要問的是.
我們身體的思想問題.
我們現在要問的是.
我們身體的思想問題.
我們現在要問的是.
我們身體的思想問題.
我們現在要問的是.
我們身體的思想問題.
我們現在要問的是.
我們身體的思想問題.
我們現在要問的是.
我們身體的思想問題.
我們現在要問的是.
我們身體的思想問題.
我們現在要問的是.
我們身體的思想問題.
我們現在要問的是.
我們身體的思想問題.
我們現在要問的是.

$^{2001}$我們身體的思想問題.
我們現在要問的是.
我們身體的思想問題.
我們現在要問的是.
我們身體的思想問題.
我們現在要問的是.
我們身體的思想問題.
我們現在要問的是.
我們身體的思想問題.
我們現在要問的是.
我們身體的思想問題.
我們現在要問的是.
我們身體的思想問題.
我們現在要問的是.
我們身體的思想問題.
我們現在要問的是.
我們身體的思想問題.
我們現在要問的是.
我們身體的思想問題.
我們現在要問的是.
我們身體的思想問題.
我們現在要問的是.
我們身體的思想問題.
我們現在要問的是.
我們身體的思想問題.
我們現在要問的是.
我們身體的思想問題.
我們現在要問的是.
我們身體的思想問題.
我們現在要問的是.
我們身體的思想問題.
我們現在要問的是.
我們身體的思想問題.
我們現在要問的是.
我們身體的思想問題.
我們現在要問的是.
我們身體的思想問題.
我們現在要問的是.
我們身體的思想問題.
我們現在要問的是.

$^{2041}$我們身體的思想問題.
我們現在要問的是.
我們身體的思想問題.
我們現在要問的是.
我們身體的思想問題.
我們現在要問的是.
我們身體的思想問題.
我們現在要問的是.
我們身體的思想問題.
我們現在要問的是.
我們身體的思想問題.
我們現在要問的是.
我們身體的思想問題.
我們現在要問的是.
我們身體的思想問題.
我們現在要問的是.
我們身體的思想問題.
我們現在要問的是.
我們身體的思想問題.
我們現在要問的是.
我們身體的思想問題.
我們現在要問的是.
我們身體的思想問題.
我們現在要問的是.
我們身體的思想問題.
我們現在要問的是.
我們身體的思想問題.
我們現在要問的是.
我們身體的思想問題.
我們現在要問的是.
我們身體的思想問題.
我們現在要問的是.
我們身體的思想問題.
我們現在要問的是.
我們身體的思想問題.
我們現在要問的是.
我們身體的思想問題.
我們現在要問的是.
我們身體的思想問題.
我們現在要問的是.

$^{2081}$我們身體的思想問題.
我們現在要問的是.
我們身體的思想問題.
我們現在要問的是.
我們身體的思想問題.
我們現在要問的是.
我們身體的思想問題.
我們現在要問的是.
我們身體的思想問題.
我們現在要問的是.
我們身體的思想問題.
我們現在要問的是.
我們身體的思想問題.
我們現在要問的是.
我們身體的思想問題.
我們現在要問的是.
我們身體的思想問題.
我們現在要問的是.
我們身體的思想問題.
我們現在要問的是.
我們身體的思想問題.
我們現在要問的是.
我們身體的思想問題.
我們現在要問的是.
我們身體的思想問題.
我們現在要問的是.
我們身體的思想問題.
我們現在要問的是.
我們身體的思想問題.
我們現在要問的是.
我們身體的思想問題.
我們現在要問的是.
我們身體的思想問題.
我們現在要問的是.
我們身體的思想問題.
我們現在要問的是.
我們身體的思想問題.
我們現在要問的是.
我們身體的思想問題.
我們現在要問的是.

$^{2121}$我們身體的思想問題.
我們現在要問的是.
我們身體的思想問題.
我們現在要問的是.
我們身體的思想問題.
我們現在要問的是.
我們身體的思想問題.
我們現在要問的是.
我們身體的思想問題.
我們現在要問的是.
我們身體的思想問題.
我們現在要問的是.
我們身體的思想問題.
我們現在要問的是.
我們身體的思想問題.
我們現在要問的是.
我們身體的思想問題.
我們現在要問的是.
我們身體的思想問題.
我們現在要問的是.
我們身體的思想問題.
我們現在要問的是.
我們身體的思想問題.
我們現在要問的是.
我們身體的思想問題.
我們現在要問的是.
我們身體的思想問題.
我們現在要問的是.
我們身體的思想問題.
我們現在要問的是.
我們身體的思想問題.
我們現在要問的是.
我們身體的思想問題.
我們現在要問的是.
我們身體的思想問題.
我們現在要問的是.
我們身體的思想問題.
我們現在要問的是.
我們身體的思想問題.
我們現在要問的是.

$^{2161}$我們身體的思想問題.
我們現在要問的是.
我們身體的思想問題.
我們現在要問的是.
我們身體的思想問題.
我們現在要問的是.
我們身體的思想問題.
我們現在要問的是.
我們身體的思想問題.
我們現在要問的是.
我們身體的思想問題.
我們現在要問的是.
我們身體的思想問題.
我們現在要問的是.
我們身體的思想問題.
我們現在要問的是.
我們身體的思想問題.
我們現在要問的是.
我們身體的思想問題.
我們現在要問的是.
我們身體的思想問題.
我們現在要問的是.
我們身體的思想問題.
我們現在要問的是.
我們身體的思想問題.
我們現在要問的是.
我們身體的思想問題.
我們現在要問的是.
我們身體的思想問題.
我們現在要問的是.
我們身體的思想問題.
我們現在要問的是.
我們身體的思想問題.
我們現在要問的是.
我們身體的思想問題.
我們現在要問的是.
我們身體的思想問題.
我們現在要問的是.
我們身體的思想問題.
我們現在要問的是.

$^{2201}$我們身體的思想問題.
我們現在要問的是.
我們身體的思想問題.
我們現在要問的是.
我們身體的思想問題.
我們現在要問的是.
我們身體的思想問題.
我們現在要問的是.
我們身體的思想問題.
我們現在要問的是.
我們身體的思想問題.
我們現在要問的是.
我們身體的思想問題.
我們現在要問的是.
我們身體的思想問題.
我們現在要問的是.
我們身體的思想問題.
我們現在要問的是.
我們身體的思想問題.
我們現在要問的是.
我們身體的思想問題.
我們現在要問的是.
我們身體的思想問題.
我們現在要問的是.
我們身體的思想問題.
我們現在要問的是.
我們身體的思想問題.
我們現在要問的是.
我們身體的思想問題.
我們現在要問的是.
我們身體的思想問題.
我們現在要問的是.
我們身體的思想問題.
我們現在要問的是.
我們身體的思想問題.
我們現在要問的是.
我們身體的思想問題.
我們現在要問的是.
我們身體的思想問題.
我們現在要問的是.

$^{2241}$我們身體的思想問題.
我們現在要問的是.
我們身體的思想問題.
我們現在要問的是.
我們身體的思想問題.
我們現在要問的是.
我們身體的思想問題.
我們現在要問的是.
我們身體的思想問題.
我們現在要問的是.
我們身體的思想問題.
我們現在要問的是.
我們身體的思想問題.
我們現在要問的是.
我們身體的思想問題.
我們現在要問的是.
我們身體的思想問題.
我們現在要問的是.
我們身體的思想問題.
我們現在要問的是.
我們身體的思想問題.
我們現在要問的是.
我們身體的思想問題.
我們現在要問的是.
我們身體的思想問題.
我們現在要問的是.
我們身體的思想問題.
我們現在要問的是.
我們身體的思想問題.
我們現在要問的是.
我們身體的思想問題.
我們現在要問的是.
我們身體的思想問題.
我們現在要問的是.
我們身體的思想問題.
我們現在要問的是.
我們身體的思想問題.
我們現在要問的是.
我們身體的思想問題.
我們現在要問的是.

$^{2281}$我們身體的思想問題.
我們現在要問的是.
我們身體的思想問題.
我們現在要問的是.
我們身體的思想問題.
我們現在要問的是.
我們身體的思想問題.
我們現在要問的是.
我們身體的思想問題.
我們現在要問的是.
我們身體的思想問題.
我們現在要問的是.
我們身體的思想問題.
我們現在要問的是.
我們身體的思想問題.
我們現在要問的是.
我們身體的思想問題.
我們現在要問的是.
我們身體的思想問題.
我們現在要問的是.
我們身體的思想問題.
我們現在要問的是.
我們身體的思想問題.
我們現在要問的是.
我們身體的思想問題.
我們現在要問的是.
我們身體的思想問題.
我們現在要問的是.
我們身體的思想問題.
我們現在要問的是.
我們身體的思想問題.
我們現在要問的是.
我們身體的思想問題.
我們現在要問的是.
我們身體的思想問題.
我們現在要問的是.
我們身體的思想問題.
我們現在要問的是.
我們身體的思想問題.
我們現在要問的是.

$^{2321}$我們身體的思想問題.
我們現在要問的是.
我們身體的思想問題.
我們現在要問的是.
我們身體的思想問題.
我們現在要問的是.
我們身體的思想問題.
我們現在要問的是.
我們身體的思想問題.
我們現在要問的是.
我們身體的思想問題.
我們現在要問的是.
我們身體的思想問題.
我們現在要問的是.
我們身體的思想問題.
我們現在要問的是.
我們身體的思想問題.
我們現在要問的是.
我們身體的思想問題.
我們現在要問的是.
我們身體的思想問題.
我們現在要問的是.
我們身體的思想問題.
我們現在要問的是.
我們身體的思想問題.
我們現在要問的是.
我們身體的思想問題.
我們現在要問的是.
我們身體的思想問題.
我們現在要問的是.
我們身體的思想問題.
我們現在要問的是.
我們身體的思想問題.
我們現在要問的是.
我們身體的思想問題.
我們現在要問的是.
我們身體的思想問題.
我們現在要問的是.
我們身體的思想問題.
我們現在要問的是.

$^{2361}$我們身體的思想問題.
我們現在要問的是.
我們身體的思想問題.
我們現在要問的是.
我們身體的思想問題.
我們現在要問的是.
我們身體的思想問題.
我們現在要問的是.
我們身體的思想問題.
我們現在要問的是.
我們身體的思想問題.
我們現在要問的是.
我們身體的思想問題.
我們現在要問的是.
我們身體的思想問題.
我想再問一個問題.
我覺得.
在美國的教會.
有時會太著重.
具體的問題.
例如離間.
他們常常在想.
如何轉變角色.
但我認為.
福音書教導了很多.
關於正義的東西.
並不是只有一個.
我認為沒有一個政黨.
完全滿足了聖經的模式.
所以這需要多元化.
有些基督徒.
學習共和黨.
有些基督徒學習民主.
我覺得這會是一個健康的情況.
所以你只要追求.
你心中最親近的東西.
但沒有一個問題.
定義了.
所謂的基督徒的政治立場.
第二.

$^{2401}$我建議.
個人基督徒.
當然你可以支持民主黨.
甚至加入他們的政治機構.
或是代表他們的黨派.
等等.
但對教會領導人.
特別是對教會主席.
我建議教會主席.
應該直接退出政治活動.
我指的是.
你可以去見議員.
或是議員.
和祈禱.
這也沒問題.
但如果是政治活動.
我認為.
教會領導人.
應該要記住.
主要的職責.
就是見證真相.
我們不是一個興趣群.
主要的工作.
就是見證神的正義.
如果這正義.
能夠在世界上成功.
那就是神的工作.
而不是我們努力.
我們要見證神的真相.
但有時候.
我們也要相信神.
才能在地球上.
實現神的真相.
我同意你剛才說的.
你的觀點.
非常有道理.
但我只想確定.
至少從我的角度.
如果我有.
政治領導人的接觸.

$^{2441}$我會想和他們討論政治.
在耶穌基督的名義.
和神的正義標準.
這意味著.
我們討論.
政治分裂.
我的興趣不是政治分裂.
而是要見證神的正義.
這就是你剛才說的.
但我想說的.
我們剛才說了一樣的話嗎.
還是你覺得我說的.
跟你說的不一樣.
不,我們說的一樣.
只是在公開的活動.
我們不想支持.
或是說支持民主黨.
但當然.
他們可以提出這些問題.
對,謝謝.
謝謝.
我想要.
同意你.
嘗試.
分別在.
政治中的正面.
和.
政治黨.
的正面.
我覺得.
這分別.
非常有幫助.
我覺得這分別.
對香港的教會也很有幫助.
因為這裡很常見.
他們想要.
堅持.
分別在.
教會和政府.
的原則.

$^{2481}$很多人.
也明白.
教會.
應該.
保持.
正面.
對政治的態度.
但我覺得.
這對.
我們的.
命令.
是愛我們的鄰居.
因為政治.
是對我們的鄰居.
最大的影響.
在很多.
情況下.
所以謝謝你.
讓我明白.
謝謝.
非常感謝.
有關問題.
關於.
有觀察說.
有很多少數民族.
甚至中國人.
他們今年.
參加川普的運動.
他們不是.
為了.
白人的反對.
你怎麼解釋.
他們有.
川普的態度.
你覺得他們的.
態度是怎樣的.
我覺得.
他們的態度是.
在重新命令權.
我們如何理解.

$^{2521}$世界上的權利.
有很多人.
對川普.
因為他.
的態度.
強人的觀點.
在於.
他使用力量.
去運作.
並展示.
決心.
和.
方向.
這是其中一個原因.
例如在非洲.
有很多非洲基督徒.
他們非常.
熱情支持川普.
這因為.
在非洲的.
民族生活的剩餘.
以及.
強人.
他會.
領導民族.
他有時會.
說話和行為像川普.
這是其中一個原因.
我認為很奇怪.
在伊朗.
這週.
會有教會.
如果他們跟隨過去的.
會是基督徒.
他們會是.
70歲女性最強的支持者.
為什麼.
70歲女性是強人.
她很強.
很多人會稱她為一個政黨.

$^{2561}$誰支持她.
有趣的是.
我認為是教會.
70歲女性.
不關心.
貧窮.
需要的女性和孩子.
受到政府的侵略.
而.
教會.
想支持70歲女性.
因為她的語言是基督徒.
她的語言是強人.
我不知道.
這是否正確.
在亞洲和亞洲人.
的解釋.
但我認為.
川普的語言.
力量,成功,力量,金錢.
影響.
並且.
與一些熱門的問題.
結合.
如同.
人性.
或者移民.
這些問題.
都結合在一起.
首先.
我們要求的東西.
然後我們要求的領導人.
要保護我們想要的東西.
保護對方.
然後我們要保護.
我們的價值.
例如.
性別暴力.
或對LGBTQ的歧視.
我不知道.

$^{2601}$你會否告訴我.
這是否一個合理的解釋.
對於一些中國人.
對川普的吸引力.
是否合理.
是的.
我認為強人的.
形象是一部分.
另一部分.
你已經提到.
社會主義.
川普是一個.
保守的聲音.
在很多社會問題上.
家庭等等.
我必須說.
川普做了一些.
我同意的事情.
他強調保護宗教自由.
這是我們可以.
認同的.
但在一般來說.
我認為很多中國人.
想保留一個保守的社會規則.
例如尊重父母.
這些東西.
我可以解釋一些.
川普的.
拉丁支持.
至少在川普的.
批評.
所謂的保守派.
民主黨.
被塗上了.
批評.
他們想要.
促進社會的一切.
讓所有人都能夠.
平等.
所以我認為.

$^{2641}$他們的恐懼.
對社會的壓迫.
是讓.
中國人.
或其他民族人.
支持川普的.
是的,謝謝你.
我同意.
我不知道.
我不認為我能夠.
對很多人說.
但對我所知的一些人.
可能是因為川普的.
外國政策.
因為他對.
一些政府.
的.
強硬.
例如.
中共.
我想這可能是.
其中的一部分原因.
而且在美國.
我不知道.
我能否說得對.
基督教.
也是在.
塑造的.
大致上.
在川普時代.
很多工作都被.
給了墨西哥和.
中國的國家.
很多人都失去了工作.
所以可能.
他們想要.
讓美國再次偉大.
讓他們可以.
重新找到工作.
讓他們可以.

$^{2681}$重新找到工作.
我想這可能是其中的一部分原因.
我也同意.
是的.
但通常.
我認同你兩位的回答.
我認為.
通常在這些回答中.
最難看到的.
是一個.
保護自己的意願.
而這有一點.
在基督教.
這就是我選擇了這個詞.
是關於危機.
和關於.
對另一方有興趣.
而不是.
我們有責任去理解.
所以我並不是說.
這會導致.
經濟或工作安全的.
直接計劃.
但我認為.
這根據一個假設.
一個文化假設.
而不是一個教會假設.
有興趣的東西.
或有機會.
有興趣的東西.
從基督教的角度來看.
這至少要爭辯.
如果我看到.
川普周圍的人.
在批評這件事.
甚至在批評它.
在設立邊界.
在批評它.
我會有不同的感覺.
我認為.

$^{2721}$這些有意義.
我們要仔細考慮.
謝謝.
接下來的問題.
如果現在的流行.
是白人的基督教.
你認為.
基督教會會變成什麼樣的?.
另一個相關的問題.
你認為.
現在是白人的基督教.
的轉折點.
來反映他們的價值嗎?.
還是你認為他們會.
走向極端的方向?.
根據我們現在的情況.
我必須說.
我比對「有信心」還要懷疑.
我並不是.
有任何的喜悅.
或是有任何的感覺.
我只是覺得.
我們現在的情況.
是有意義的.
盡管現在的情況.
仍然有.
人在反映.
反對選舉的.
訴訟和.
被用暴力的方式.
等等.
所以我並沒有.
深入的思考.
或反思.
這件事.
是更加有意義的.
更加反應的.
而不是.
自我批判.
我認為人們在展現.

$^{2761}$自己的真正的顏色.
他們更加適應這個議題.
他們在向神祈求.
神給我們新的眼睛.
新的心和新的意識.
讓我們可以看見自己.
我們的鄰居和我們的國家.
和我們周圍的世界.
是另一種方式.
我認為.
有些學院.
例如今天.
我認識的.
大概有四個學院.
他們發表了.
一些非常強烈.
的聲明.
他們對這個議題的.
自我批判.
我上週發表的信.
我正在說.
我們必須把自己.
放在首位.
雖然富勒不想.
自己成為我所說的.
這個議題.
但我們不可能不被困在.
這件事上.
我們可能在.
不同的社會地點.
和不同的社會批判.
我們仍然會被困在.
相同的問題上.
所以.
我們要承認.
我認為我被邀請到這裡.
和邀請到其他人.
是進入了.
一個持續的討論.
與不同的人.

$^{2801}$不同的文化.
道德和.
名義的情況.
討論.
這個危機.
因為.
我們必須.
向一個.
我們可以.
共同分享的新的故事.
去做出.
共同的表達.
我們可以.
更公開的表達.
現在.
有無數的.
故事.
我懷疑你們.
知道一個很棒的.
視頻.
我不知道它能否翻譯成中文.
但是它是.
一個TED講座.
由一個尼西里人女士.
名叫.
Adichie.
她的名字是Chimamanda Adichie.
\newpage



\section{}
\label{sec:A4_dD_EAMHg}
\textbf{The Crisis of American (White) Evangelicalism (Day 3)}
\newline
\newline
連結: \href{https://youtube.com/watch?v=A4-dD-EAMHg}{\texttt{ https://youtube.com/watch?v=A4-dD-EAMHg}} ~~~~ 語音日期: 2021-01-21 
\newline
\newline
\hyperref[sec:F73AP9vE2KM]{\small{< < < PREV SERMON < < <}}
~
\hyperref[sec:index]{\small{[返主目錄]}}
~
\hyperref[sec:NlL3iKQIHto]{\small{> > > NEXT SERMON > > >}}
\newline
\newline
$^{1}$(音樂).
Good morning, welcome everyone.
Michael Chu here.
It's great to have so many people.
hundreds of people here in our lecture.
We know we have people from Hong Kong.
from America and from all over the world.
And this morning we continue to host our.
Josephine So lecture week.
And today is the third lecture.
It's a great honor to have Dr. Mark Lepperton.
to be our keynote speaker.
Dr. Lepperton has spoken to us.
on the overall title.
The Crisis of American White Evangelicalism.
Personally for the last two days.
I learned a lot as Mark walked us through.
the American history and outlined.
the specific perspective on the American culture.
And Mark also highlights the certain deficiency.
of evangelism or even the deficiency.
of evangelical churches.
I'm particularly impressed by Mark's honesty.
and courage in the path of self-critique.
Mark, if you don't mind, I would like to address you.
by your first name.
Because I feel after these two lectures.
I feel so close to you.
We are just like good old friends.
As if you are talking to me in person.
So Mark will speak to us this morning.
on the third topic.
Evangelical Perplexities about Power and Leadership.
We will shift our discussion.
to the leadership part of the churches.
So Mark, we are much looking forward.
to your speaking to us.
And now we welcome Mark to speak to us.
Thank you very much, Michael.
I appreciate your words.

$^{41}$I'm grateful again to be here.
Just before I start my lecture.
I want today to do two things briefly.
First, just to acknowledge that.
I find it perhaps providential.
that on the day when.
the Congress of the United States.
voted to impeach for the second time.
the President of the United States.
we're talking about the issues of power and leadership.
and evangelical perplexities about those things.
It's been a very sobering and painful day.
a divisive day in the United States.
and the implications of today.
will be unfolding for some considerable time to come.
I secondly want to just acknowledge that.
in this decision of the Congress.
there is for so many different reasons.
ways that the American white evangelical church.
is implicated and involved.
in the things that led up to this moment.
and certainly in the implications of this moment.
as we go forward from here.
So I wonder if you would mind.
letting me just offer a prayer.
for the church in the United States.
as well as for all of us.
Lord of all, it's to you that we come.
You are the one who reigns.
in mercy and justice over all peoples and places.
You are the one who sees us as we are.
You understand the places in which we see clearly.
And you understand the places.
in which our sight and knowledge are so confused.
This day in the United States is a very dark day.
It's a day in which we acknowledge.
some painful realities about our national life.
And as we go forward.
your church is implicated in both.
some of the causes of why the problems exist.

$^{81}$And also we pray by your grace.
a church that might be able to be used.
by you for the renewal of the church.
and the renewal of a nation.
Tonight as we pray here.
and in the morning as people pray throughout Asia.
and in Hong Kong in particular.
we acknowledge that it's your lordship.
your power, your leadership.
that we need far above and beyond any human agency.
So in humility, oh God, we turn to these issues tonight.
acknowledging how much we need you to speak.
to guide, to confront, to transform us.
In the name of Jesus Christ.
who reigns with humility and power.
In Jesus' name, amen.
So we come tonight to consider.
evangelical complexities about power and leadership.
We will reflect on issues of evangelical power.
that underline, crown, and even subvert its gospel.
and its witness.
The history of the church has typically been told.
as the story of so-called big men.
that is, a man usually, if not almost always.
through the power of personality, gifts, opportunity.
stealth, privilege of birth, and influence.
to have written the story of the world.
I dare say this is true in the east or the west.
it's true in the south and the north.
whether the past or the present.
We seem to be in a time of exceptional global power disorder.
and redefinition.
Every continent has evidences of the turmoil.
that's going on inside the United States.
It is not a time of world war.
and yet in so many directions.
we see national and international collision.
aggression and hostility.
Battles of war, including within and among evangelical Christians.
are clearly afoot.

$^{121}$All of this is, of course, in dramatic contrast.
to the description of Jesus.
that we see in Paul's letter to the Philippians, chapter 2, verses 1 through 8.
Hear these words again.
Have this mind among yourselves.
Which is yours in Christ Jesus.
Though he was in the form of God.
Did not count equality with God a thing to be grasped.
But emptied himself by taking the form of a servant.
Being born in the likeness of men.
And being found in human form.
He humbled himself by taking the form of a servant.
And being born in the likeness of men.
And being found in human form.
He humbled himself and became obedient.
To the point of death.
Even death on a cross.
Jesus as the ultimate "big man".
Lays down the assumptions and prerogatives of power.
To live and lead in vulnerability and in weakness.
Nevertheless, the big man.
assumptions occupy a very significant place.
in the life of the American Evangelical Church.
From the 18th to the 20th century.
to this day in the 21st.
we can point to the quiet, prayerful, faithful, often exemplary lives.
of members of the white Evangelical Church in the U.S..
and see many true signs of the fruit of the Spirit.
Albeit, this can now only be said.
if we acknowledge that this positive legacy.
is often still intertwined.
with the subtle and horrific compromises.
and the absence of self-criticism.
or repentance or lament.
and transformation that could have occurred.
It's often a story of hidden piety.
inward piety.
rather than public piety.
matched with justice.
Susceptible as human beings are.

$^{161}$to seek a savior and a hero.
it has been an easy assumption.
that God will raise up the leader.
in a U.S. context.
that should be understood primarily.
typically as being white and male.
to be a leader, pastor.
an authoritative, visionary voice.
This archetype is so pervasive and influential.
it's assumed rather than examined.
This is necessarily accompanied by an image.
of a blonde, blue-eyed Jesus.
who serves the dominant interests of an Evangelicalism.
that insists on seeing its image of Jesus.
seeing in its image of Jesus itself.
rather than letting Jesus see his image in us.
On the other hand, it can be acknowledged with gratitude.
that the Evangelical movement in the 20th century.
with all of its blessings and sins.
perhaps ironically and unselfconsciously.
both contributed to and resisted.
the geographic repositioning.
of the center of the global Christian movement.
from being a story centered in the global north.
to one that's now clearly centered in the global south.
and in Asia.
This is a cause for rejoicing and for prominence.
that is primarily about the Spirit of God.
meeting the people of God.
This is really the story of God.
calling, forming, and using leaders.
around the world for such transformative change.
not least throughout Asia.
During the last quarter of the 20th century.
in America and elsewhere.
we saw the rise of the megachurch phenomenon.
that has risen and also fallen.
because of the influence of the big man paradigm.
story of largely unidirectional.
typically white male power in the United States.

$^{201}$This means the need to acknowledge as such.
as part of the observations of the 20th and 21st century.
white Evangelicalism still operates.
largely in a colonialist framework.
both in the U.S. and globally.
Several factors explain this.
First is dominion, the theme of dominion.
The specifically Christian theme of dominion.
has been one that has been informed, motivated.
shaped and subverted by our sense of the mission of God.
the missio Dei.
The sense of the mission of God in the world.
as an embodied and proclaimed announcement.
that the reign of God is at hand.
for the sake of all people and places.
a bringing to consummation of all things.
by our Creator, Redeemer and Remaker.
This exercise of divine dominion.
is for the sake of bringing to an end.
the groaning of all humanity.
and of Jesus and of the Spirit.
and of the people of God.
and of the whole created order.
for the sake of God's shalom.
and the well-being of love.
and mercy and justice, harmony.
The so-called doctrine of discovery.
which came into existence in 1493.
was the act of Pope Alexander VI.
and played a highly influential role.
in justifying and shaping the Spanish conquest.
and the colonial project from that point on.
The papal bull that the Pope.
put out gave permission and power.
to discoverers, as they were called.
of any land that was not inhabited by Christians.
that if they discovered such a land.
uninhabited by Christians.
then it was considered to be available.
to be "discovered," claimed.

$^{241}$and exploited by Christian rulers.
and declared that the Catholic faith.
and the Christian religion.
be exalted and be everywhere increased and spread.
that the health of souls be cared for.
and that barbarous nations be overthrown.
and brought to the faith itself.
This doctrine of discovery.
justified all European claims in the Americas.
and the premise justifying.
the Western expansion of the United States.
A book that I would commend to you.
is a book written by Mark Charles and Choon Sa.
Sung Chan Ra, entitled "Unsettling Truths".
which explores this in much more detail.
Many other books, likewise, have been written about it.
This aligns with what had been seen.
in the Roman Empire.
Constantine's influence, medieval expansionism and war.
Yet, through the doctrine of discovery.
the Church is itself the initiator.
in the further coalescence.
of Church and government collusion.
to use power in the name of faith.
for the sake of political, economic and religious ends.
It is the full fruit.
of what had long been operational.
and it was unleashed.
in what continues in less obtrusive ways.
to the dominion mindset.
of some American white evangelicals to this day.
The ferocity of the division.
between the American evangelical church.
and the American evangelical church.
is in many ways.
unlocked by the doctrine of discovery.
If you believe, as many do,.
that the ideological, political and religious battles.
going on among white evangelical voices in the U.S..
has little to do with theology.

$^{281}$it has more to do with power.
Its roots for how that power expresses itself.
is evident in the doctrine of discovery.
The assumption is that these battles.
are an extension of the doctrine of discovery.
that is, they now go on.
like an ever-playing fugue line in music.
that undergirds and reassures many.
that this white normativity is the right track.
It is a form of aggressive nostalgia.
that longs for an earlier, more racially homogenous.
less racially diverse.
more unquestionably defined.
and controlled white normativity.
that dominates other so-called intrusive voices.
I accept Dr. Willie Jennings argument.
that being "white".
is not strictly about the melatonin level of one's skin.
but about a heart and mindset.
that can take captive any person of any race.
Whiteness holds power.
and the claims and rights of power.
simply as an expression of the way.
that reality has established it.
This at least is the blindness of whiteness.
the presumption and normativity.
that is so damaging to white people.
and even more to those who are its victims.
Whiteness is the personalized and social form.
of the doctrine of discovery.
holding internally the assumption.
and assertion.
that the individual can claim in relationship to others.
attitudes of conquest, superiority and dominance.
whether it is about owning a gun.
wearing a face mask.
attending an inter-public worship service.
or anything else.
I hope all of these themes.
are ones that you've seen.

$^{321}$and given evidence to.
in the American press.
Dominion then is our first major problem.
The second problem that's related to it.
is the notion of American exceptionalism.
This as you know.
has a long tail in the narrative of US history.
It starts with the writing.
of a French man of letters.
Alexis de Tocqueville.
who described the United States as "exceptional".
as a "new country".
and in the convergence of a kind of mindset.
and instincts.
that would allow it to hold unique power.
and influence in North America.
and eventually the world.
One of the deepest assumptions in nationalism.
is that American nationalism.
is not only best for it.
but also best for the world.
that it wants to influence.
It is endlessly self-admiring and adulating.
and it is full of the idolatry.
of a nation over any and all other forces.
especially outside forces.
In this model.
the phrase "America first".
has periodically emerged.
as the battle cry.
in relationship to other nations.
While the internal dangers.
and enemies of powers.
must be limited and opposed as well.
All this has been an intense resurgent.
in recent years in the United States.
certainly since the 1980s.
growing in the 1990s.
and very powerfully present.
throughout the 2000s to date.

$^{361}$And it's attached to the Dominion vision.
that like Napoleon.
simply crowns itself emperor.
America has done that.
in so many painful.
and inappropriate and unjust ways.
When the church adds to it.
the elements of manifest destiny.
that this exceptionalism.
this nationalism.
is like America being a new Israel.
and in relation to which.
this nation is first and primary.
in God's and in the church's priorities.
It is an extremely dangerous.
collection of idolatries.
precisely and passionately held up.
as the only true national worship.
This creates a perfect closed circle.
dressed up in Christian speech.
which avoids Christian identity.
and character.
in matters that are both public.
such as the equality of diversity.
peace with justice.
structural and personal racism.
and in private.
such as which voices are heard.
which lives and needs are seen.
which narratives or stories.
are actually believed.
So Dominion, nationalism.
and third is individualism.
White evangelicals in the United States.
seamlessly pair the doctrine of discovery.
with deep-seated autonomous individualism.
That is, the individual autonomous self.
gets to walk through the world.
like nations did.
under the doctrine of discovery.

$^{401}$and make claim on peoples and places.
jobs, voices, power, influence and so forth.
It's a tragic story.
It's one of America's best and worst traits.
The best side of this kind of individualism.
is the way in which it can be.
attached to a sense that.
individual independent rights.
are a reflection of being made in the image of God.
They are the center in personal, political.
structural, familial, social and religious life.
in the United States.
It must be the individual.
Unlike Ubuntu in Africa.
"I am because we are".
the American slogan is more nearly.
"I am because I make myself".
This means that power is finally personal.
and that the community.
whether in church or society at large.
are always in third place.
behind family and tribe.
In the end, the culture in which.
American evangelical lives and often thrives.
is on the grounds that each person.
and their own will and personality.
and priorities and values.
should be placed at the center of everything.
So much of America's ethical and moral history.
even to this day.
around issues of human sexuality.
and many other things.
are all reflections.
of this doctor discovery oriented.
racially biased.
individual pleasing form of Christian life.
is into a consumerism.
that's measured by the satisfaction.
of church members.
and not by wisdom or by faithfulness.

$^{441}$This kind of distorted ecology of faith and mind.
heart and will.
to call Jesus Lord, Lord.
but completely separated.
from actually doing the will of God.
The inward private conversion of a disciple.
or the response to the altar call.
to getting saved.
is an especially white evangelicalism.
The Bible is of course clear.
about the value of the individual.
and the need for each of us in the body of Christ.
to seek and find God.
But the weight of the scriptures.
rests on the calling of a people.
not just an individual.
to be God's own people.
and for the call of the ecclesia.
that is those that called out community.
to be communally called out into one new humanity.
is a sign that we collectively.
not just I.
who once was dead.
have not just been revived.
but have actually been made new.
as evidenced through our living.
into a new community of unlike.
The immigrant escape.
from Europe and many other countries since.
has infused in the evangelical movement.
in the United States.
the preference for finding and maintaining.
your tribe.
America can at times be a melting pot.
but that is not the common experience of many.
Instead we are a collection of social, racial and economic tribes.
with the white evangelical tribe.
presupposing itself to be always in the lead.
the communion.
that crowns the communion.

$^{481}$that is really the way that is often organized.
I will confess that for me personally.
one of the most painful parts of the history.
and influence of Fuller's Seminary.
is the church growth movement.
In a well-researched and motivated beginning.
Donald McGovern on Fuller's faculty.
desired to help the gospel reach more of America.
not just one by one.
but in a larger, broader, more communal sense.
He knew this meant a gospel that was attuned to context.
This led him to the discovery, sociologically.
that people come to faith faster and simpler.
if they do not have to cross a cultural threshold.
And if the gospel is embodied in a contextually appealing.
and recognizable way.
In some ways these are not only legitimate.
but valuable insights.
On the other hand, what these insights become.
is a rationale for what is known as the homogeneous unit principle.
And what later becomes the marketing motivation.
for the church growth.
through gathering simply people who are similar.
not gathering people through the lens.
of a diverse group of people with a kingdom vision.
a brokenness and remaking.
that moves from nothing less than death to life.
Instead, we call people to join the church.
as an accommodation to their social context.
assuring them that they will find there.
many who are just like themselves.
What is offered in this way, then.
is simply a gospel that is about appeal.
a warm, positive, life-affirming appeal.
and a community of people of those who together.
believe that "their needs are being met".
Conversion comes to mean that we are putting trust.
in God's eternal hope.
and having a personal relationship with Christ.
that can become nothing more than getting God.

$^{521}$to give us good gifts for our sake.
and for the sake of others that we know and love.
Pointedly, what is not being offered.
or discipled into many people's lives.
is the fundamental, comprehensive revolution.
of kingdom recreation.
We take the phrase in Corinthians.
all things being made new.
and reattach it to the impact.
of simply saying yes to an affirmation of faith.
But then that gets cut off.
from the deep spiritual transformational process.
by which by word and spirit.
our minds and our lives are meant to be renewed.
in order that we may actually do, that is, perform.
enact, embody the will of God.
Romans 12, 1 and 2.
The next factor in this is what I'm calling agency.
In Matthew 9 we read.
All things are possible through them that love and serve God.
This rich but humble affirmation of God's ability.
to accomplish whatever God may design.
through the power of the Holy Spirit.
can become nothing more than a mantra.
like rubbing the genie lamp.
at the will of the disciple.
God becomes the power.
for the exercise of my own personal agency.
I seek God for the power to do my will.
not my will to seek the humility.
to serve the power of God.
In the name of the unassailable.
and incomparable power of God.
there is a kind of teaching about the exercise of faith.
that is really about the role that one can have.
in being the willful believing agent, hence agency.
that can accomplish what the prayer may will.
as though by prayerful agency.
the individual or two or three others.
can by themselves usher in the kingdom of God.

$^{561}$I would argue that if it is the kingdom of God we are naming.
not the kingdom of Christian leaders or nations or individuals.
then human beings are never the primary agent of that sentence.
The inference in some white American evangelicalism.
is that human agency.
represented by what is seen as the "truly faithful".
the true believers.
can bring forth the manifestation of God's will and reign.
It is prayer without doubt.
often meaning without humility or self-critique.
or acknowledged finitude or fundamental bias.
that produces blessing.
The best history I know of the prosperity gospel as an aside.
is written by Kate Bowler.
and is in a book that she has written entitled "Blessed".
She examines and studies firsthand.
how this agency theme is managed.
and described by prosperity gospels.
by prosperity churches and preachers.
in the United States and beyond.
Agency in the way that I'm talking about it at the moment.
is really just a personalized form of dominion.
that is exerting the power of our vision, personality or desire.
The important quality of how we have been made in the image of God.
can be another good gift.
turned into little more than self-interest.
We are made to exercise agency in the world.
when we hold it in the way that often has happened.
it plays into what I call consumer Christianity.
in which human agency and desire.
occupies the private place.
with the general assurance that this is appropriate.
because God wants to give us "the desires of our heart".
Psalm 37.
But agency in Israel's life.
is at the heart of the prophet's critique in Isaiah 58.
Israel has taken the worship practices of God's people.
and used them, the prophet says, for their own purposes.
such that worship ends up being about them.
and not about God.

$^{601}$No wonder God says this is "as if" worship.
that is, worship that pretends or claims to be about God.
as if they were going to actually live in a righteous way.
Instead, instead of honoring and using their agency to serve Yahweh.
in fact, their worship, the prophet says, is about themselves.
This is a story of agency gone to seed without any fruit.
The fruit of the Spirit is meant to be productive.
and it's meant to be productive that shows itself in ways far beyond.
productive agency in the bounty of our own self-interest.
Instead, the fruit of the Spirit is, you know.
love, joy, peace, patience, kindness, goodness, gentleness, and self-control.
So dominion, nationalism, individualism, and agency.
cannot and do not seek or produce the fruit of the Spirit.
Those are all elements that by themselves.
bear the fruit of a lack of faithfulness.
rather than an evidence of it.
Another factor for a movement that is founded.
Sorry.
Christianity and evangelicalism.
Sorry.
is a movement that is founded on the belief that God gives us the peace that passes understanding.
And yet fear is a significant feature that motivates American white evangelicalism.
particularly so for its leaders.
There are different fears that correspond to different possible threats.
We could start with theological or doctrinal fear.
Part of a danger of favored status is then the possibility of losing that status.
Fear and anxiety grips a lot of evangelical leaders and pastors.
who then wonder about speaking the truth when the stakes are high.
And if they get it wrong or if people don't like what they say.
the loss can be aggressive and destructive.
In some cases, it is built into a self-dependent theology of salvation.
as though a person is really saying, "I am saved by having the right theology.".
And that can mean that my salvation is mine to win or lose.
Will I or won't I be saved depends on me, not depending on God and what God has done in Jesus Christ.
So the issue then becomes, do I have the doctrine right?.
Or am I too dangerously close to quote wrong doctrine, wrong belief, wrong communion, wrong practices?.
All of those things being things that Jesus himself was accused of.
This kind of an approach to fear is about boundaries.
A boundary that is about who is in the circle.
versus who and what is outside the boundaries of God's grace.

$^{641}$Minding the boundaries for oneself or one's family or church or denomination or one's nation and so on.
means that we are fearful about getting it wrong.
The ever-present need to be sure that we are getting it right.
that our people are right, that our tribe is right, that our church is right, that our denomination is right, that our seminary is right.
Within one week recently.
I received five phone calls from very experienced pastors.
who called to talk about the fact that these pressures around fear.
were going to mean that they themselves feared that they would be fired.
and that they would not be able to find another job.
because of being judged unacceptable by leaders around them.
Within a couple of weeks of receiving those phone calls myself.
three had been fired.
and not only fired but then maligned.
This is the evidence of what I'm talking about.
Another category of fear has to do with the ever-present danger of "the world versus the flesh".
The premise here is that faithful disciples are vulnerable to being consumed or shipwrecked.
by the ubiquitous presence of sin and evil in the world.
At any moment our enemy is seeking someone to devour.
Therefore disciples need not just the tools of spiritual warfare.
but the continual readiness for spiritual battles.
This tends to stimulate in our brains the kind of neurobiology.
and the moral worry that makes finding, naming, and embattling enemies.
an obvious and continuous necessity for many.
History has shown the likelihood of this progression.
and the historical evidence, not least in exploding numbers of demonimations in the 20th and 21st century.
In the century of God's greatest wars, the 20th, of the world's greatest wars in the 20th century.
it is plain that enemies are real.
and that the church must be alert and aggressively self-protective.
It's a fearful posture that ironically pushes away and down.
the very lost in need of the evangel.
This fear plays readily to a spiritual fight or flight response.
which fear often evokes.
Both a fear of fighting or fleeing.
both toward those inside the church and toward those in the culture.
It fosters the likelihood of approaching discipleship as embattlement.
Now clearly the New Testament uses this imagery.
I'm not in any way arguing that.
I'm arguing for the way that it's a dominant theme.
and for what it is that Christian leadership should do about that theme.
The first response of fighting takes up force of various kinds in order to defeat the opponent.

$^{681}$A recent public example of this.
was when the Southern Baptist Convention of Seminary Presidents.
opposed to, rejected and publicly repudiated.
the academic study that's known as Critical Race Theory.
This school of thought, Critical Race Theory.
attempts to understand the dynamics of racial justice and oppression.
When the seminary presidents of the Southern Baptist Convention.
condemned Critical Race Theory.
it did so out of a fear that the critique will do damage to the church.
But by condemning Critical Race Theory.
they expose at the same moment unintentionally.
the underlying fear that the denomination's originating racism will be exposed.
Ironically, their condemning action reveals so pointedly.
that they see CRT, that is Critical Race Theory, as the real enemy.
rather than the underlying controlling racism that debilitates them and their gospel.
By drawing the battle line as these seminary presidents did.
they fought what they saw as an enemy.
that is a stand-in for the real opponent.
maintaining the embattlement profile that is so important to their theological and cultural sensibility.
And it was done by white, male, evangelical leaders.
displaying their power to name and control their sense of the reality narrative.
while masking their fear and their denomination's fear of losing their white supremacy.
The chief problem revealed here is what the Apostle Paul says.
is the fear of the weaker brother.
The wrongly fears what does not need to be feared.
while failing to fear what deserves to be feared.
The fears of many American white evangelicals is a set of fears that center on them.
on their perceptions, their social location, their beliefs, their tribal interests.
rather than on what matters most to the kingdom of God.
Is it too much to think of Jesus' words to the Pharisees.
that they "tithe dill and cumin, but avoid the wavier matters of the law, mercy and justice".
Fear management can seem to be one of the defining features of American white evangelicals.
and for many leaders, this is one of their chief preoccupations.
American and leadership are words that many in the United States proudly believe belong inextricably together.
The American white evangelical church in most of the last half century.
has tried to assert its voice in the cultural, legal and yes, political maelstrom that I've been trying to describe.
American culture has long since been defined by media and secular culture than ever before.
There is a genuine battle to acknowledge and to engage in relation to the controlling influences.
shaping children and teens and young adults.
The reasons to fear the implications of a highly influential secularism.

$^{721}$and anti-religious bias, not least in the west and east coast of the United States.
Now nations around the world have marveled at the pervasive religiosity of the United States.
and the high percentage of Americans who believe in God and worship regularly.
The liberty and protections of religious freedom are long established and also continuously under attack.
What American white evangelicalism has not done is to find a nationally credible, thoughtful leader practitioner.
who incarnates and speaks into the public sphere in ways that can capture the imagination and engage the heart.
I've tried here to suggest factors that I believe contribute to why this perplexity of leadership has not emerged.
and perhaps would not even be received as a good idea or a legitimate hope.
And of course what is needed, far more than a singular voice, is for the church in the United States to live into its true Christian identity.
Jesus' strategy for disciple-making is a new community that lives out its surprising eclectic diversity.
This servant-living communion would dismantle and redefine dominion and nationalism and individualism and agency and fear.
This kind of Christian community that is so deeply needed would be an authentic, credible communion.
that lives as the body of Christ in the world and for the sake of the world.
Not inward turning, but living sacrificially and freely for the sake of the world.
especially the poor and the marginalized.
The American white evangelical crisis is not so much in the sense of theological crisis.
as it is a church in crisis over a failure to live out the church's identity as a reflection of Jesus Christ.
Instead, it has turned out to live as a reflection of the culture in which it is placed.
As simple and plain as that sounds.
it is the story of great blessings of life put under a basket of American thirst.
or of salt losing its saltiness for a desire instead to conform.
Will that change?.
On this day, when the President of the United States has been treated in the way that he has by Congress.
in such a legitimate and defensible way.
shows the barren fruit of the American evangelical story being identified more with power.
secular, political, social, economic, cultural power.
than by the life, death and resurrection of Jesus Christ.
It's only the latter that is the hope of the world.
Let me pause there and welcome my two respondents.
Thank you, Mark.
Thank you for your very perceptive discussion of leadership.
particularly the leadership of the American white evangelicalism and the churches.
You also outline quite a few factors which governs or defines our identities as a leader in the Bible.
I'm particularly impressed by the last point, the fear, the factor of fear.
that really always shapes how we behave, how we perform, how we relate to one another.
So thank you very much.
There's a lot of insight into that.
So now may I ask Dr. Gene Lee and Dr. Bernard Wong.
our dean and our associate dean to come forward to give their response.
I would like to ask Gene first.

$^{761}$Thank you, Professor Levitin, for an insightful portrayal of power and leadership in this lecture.
In particular, I appreciate the personal and contextual narratives you include in these lectures.
allowing us to integrate theories with reality for a deeper understanding of the matter.
Indeed, the big man paradigm has been apparent for a long time.
and all the more in today's world.
No matter what the dominant political system is.
people do have a natural tendency to look for heroic leaders.
I lift the narrative of a little lady in a big man's world.
In my late teens and early twenties, I spent nine years in North America.
studying and working in a big man's world.
My first job was in external audit, a profession where you are pushed every year to advance a grade in the ranks.
So quickly, I found myself a short, non-white, Asian female leading an audit team.
The audit team is tall, handsome-looking young white males.
I remember once when we walked into a client's office.
the client immediately spotted the leader among us.
greeting one of my white male subordinates with a warm handshake.
I still recall how embarrassed the client was.
when my colleague redirected everyone's attention to the short Asian female among them.
Indeed, we are framed by our culture and context.
We are bombarded daily with values and expectations that may only be partial truth.
Not only do we need the firm grasp of our identity in Christ.
but also continuous vigor in training for theological instinct.
that can help us sense the distortions we face.
Professor Levitin pointed out a number of perplexities.
in the concepts behind leadership and power.
Dominion, nationalism, individualism, agency concepts, and even fear.
can quickly become tools or justifications for power and dominance.
In our context, we face similar compromises and distortions.
especially when different pressure groups may use the same words and phrases.
to mean different things.
twisting people's minds and thought processes to suit their agenda.
While Christians from the West may appreciate the relational culture in Asian communities.
we know that relationships in Chinese actually translates to 'guānxì'.
a term expressing an obligation of one party to another.
with a continuing reciprocation of favours in Chinese business culture.
Such practice also penetrates other walks of life.
taking forms of excessive or corrupted gifting or social exchanges.
The prosperity gospel is another example of faith distortion.
that presents itself so vividly in our culture.
Chinese folk religions have long preyed on human desire.

$^{801}$and a tendency to create a God.
to whom one can give gifts in exchange for favours.
Thus we are very much attracted to Matthew 7:7.
"Ask, and it will be given to you; seek, and you will find;.
knock, and the door will be opened to you.".
This Bible verse is obviously valuable.
only that we tend to stop at 'ask and be given'.
without truly seeking and finding God's will.
and discerning the door that gets opened.
In recent years, Hong Kong society undergoes drastic changes.
both politically and under COVID-19.
Adversities and pressures tear us further apart.
splitting groups of brothers and sisters.
some down the compromising route, some to fighting in the extremes.
Those taking various middle stances are attacked by both sides.
We no longer see leadership and power in the traditional sense.
but rather see these with a negative feeling of distrust.
We are challenged to go back to the very basic.
asking the question.
What does the gospel mean to our culture.
and the situation we are facing?.
All of us are confronted with our own fear.
Be it spiritual, as in the Christian faith.
or political in the societal sense.
Under the changing political climate.
the people in Hong Kong are influenced by fear.
in much deeper ways as never before.
Fight or flight takes on another meaning.
as individuals and families decide.
whether to redomicile abroad.
under the changing political climate.
Hopefully, this decision process involves spiritual discernment.
and that the Lord continues to work in our midst.
that our faith be strong wherever we go.
We continue to pray for God's wonder.
in this special time of drastic change.
The big man paradigm is indeed a sharp contrast.
to the description of Jesus in Philippians 2, 1-8.
The humble leader is a subversive paradigm.
Even here, we encounter the easy distortion of James 4:10.

$^{841}$and a few other verses that say.
"Humble yourselves in the presence of the Lord,.
and He will exalt you.".
It is our natural tendency to live.
according to the system of this world.
so that we can get exalted or rewarded.
Nevertheless, in discerning the Lord's work.
during this day and age,.
we observe the rising up of a generation.
of collaborative leaders.
rather than individual heroes.
The next generation faces an increasingly complex.
and hostile world.
Yet they are individually gifted.
in various new and creative ways.
Rather than trusting those at power.
to resolve all the problems.
young leaders today rise to face challenges.
in their own sphere of influence.
in many different ways.
There may be limits to what one can do.
while not in power.
But in collaboration and in discerning God's will.
in this day and age.
there can be no limits to the possibilities.
that open up.
Obviously, collaborative leadership.
also has its limits.
Decisions may be delayed or out of sync.
Actions may be fragmented.
and chaos may result.
Nevertheless, collaboration reminds us.
to humble ourselves and listen to each other.
to discern God's will.
in a story that not us.
but Him is the real hero.
So I stay contented.
as a little lady in a big man's world.
continuing to find my identity in Christ.
and to learn to be a faithful follower of Christ.

$^{881}$(掌聲).
(英文).
(笑聲).
(英文).
而是自我懺悔和憎恨.
如果任何一個宗教的人.
不願意做這些基本的事.
來實現真實的靈性生活.
那我就不認為我們對這件事是有興趣的.
這件事在一般情況下是死定的.
我認為這件事是如此的自然.
我用自然來形容.
這件事是如此的自然.
一個宗教不願意向自己說真相.
不願意向別人道歉.
這不是一個對耶穌基督的靈性的宗教.
我們就變成了社會運動.
它會做那些那些人想做的事.
他們也許會在耶穌的名下做.
但如果他們不願意去活在宗教.
那就像耶穌說的.
那就是一座建築物的故事.
而不是一座石頭.
在時間的過程中.
那座石頭就是我們的故事的結局.
所以我覺得我們在這時期.
有著非常重要的事情.
我不是想要作弊.
我認為在美國宗教教會.
有著一個第一次的危機.
也許是真的.
在其他教會和其他地方.
我沒有被要求討論.
我只是在討論.
白人的宗教現實.
我想問問.
如果有些人想參與我剛才說的話.
我會很高興聽到.
我認為Bernard想說幾句話.
其實我讀出這個問題.

$^{921}$我會這樣看待.
看來這個人的問題是.
如果一個人是自身的危機.
那他就不能成為世界的影響力.
但我覺得這是一種雙重思考.
如果我是真實的.
那我就要告訴你真相.
你就要聽我所說的.
我認為這是一個不可思議的.
或是一個錯誤的神的任務.
我們在帶給世界的一種冒險.
我會想像.
基督徒在做革命.
我們其實是真正的真正的求真相者.
我們是真正的真正的求真相者.
而對於那些還沒有信仰的人來說.
我們是真正的真正的求真相者.
所以我認為.
這就是一個可以克服這種危機的思考.
你只要說我有真相.
你就不會有這種思考.
是的,非常有幫助,謝謝.
我覺得真正的真相.
是你對著對方說真相.
我們一起做.
這會幫助我們再往前走一步.
好,馬克.
還有一個問題給你.
在今天的世界.
世界競爭造成了國家主義.
和一些重要的和互相爭議的意義.
美國白宮如何教導.
健康的友善.
也就是愛和熱情.
對自己的國家.
以及慶祝我們的族群的身份.
是的,我覺得這是一個非常好的問題.
這是美國白宮一直以來.
一直在做的一個計劃.
我們要愛和熱情.

$^{961}$我們要為國家服務.
我們不需要將我們的國家.
變成偶像.
甚至是像這樣.
我們不需要將它變成偶像.
謝謝.
沒有其他問題嗎?.
Mark,剛才您在講座中提到.
"融化盤".
我們有一個問題.
請您給我們一些實際的建議.
如何組織一個.
在教會內的多元社區.
不讓它變成融化盤.
但沒有死亡.
我在教會內的一些事業中.
我亦在Fuller的教育學院加入.
Fuller的某個形式.
我們稱為"麥克群組".
是不同種類的人.
不同背景,不同種族,不同組別.
集體聚集.
在一個小群組內過兩年.
我們在教會內的事業中.
\newpage



\section{}
\label{sec:NlL3iKQIHto}
\textbf{The Crisis of American (White) Evangelicalism (Day 4)}
\newline
\newline
連結: \href{https://youtube.com/watch?v=NlL3iKQIHto}{\texttt{ https://youtube.com/watch?v=NlL3iKQIHto}} ~~~~ 語音日期: 2021-01-21 
\newline
\newline
\hyperref[sec:A4_dD_EAMHg]{\small{< < < PREV SERMON < < <}}
~
\hyperref[sec:index]{\small{[返主目錄]}}
~
\hyperref[sec:0eVKvD_Q9zg]{\small{> > > NEXT SERMON > > >}}
\newline
\newline
$^{1}$(音樂).
Good morning or good evening everyone.
Depending on where you are.
Welcome to the final day of.
CGST Josephine's Soul Culture and Ethics lecture week.
I am Esther Xue.
Assistant Professor of Biblical Studies.
We are so honored to have.
our keynote speaker Dr. Labberton.
with us at the start of 2021.
via this virtual platform.
Besides he is known as president of.
Fuller Theological Seminary.
Dr. Labberton also serves in several.
significant world ministry institutions.
Particularly he is known as.
leadership with scholar leaders.
international and John Stott ministry.
NATI's Lindheim partners.
Actually I can survive through my PhD.
partly because I have been supported.
spiritually and financially.
by these two institutions.
Thanks Dr. Labberton.
I believe we were so inspired.
by your fruitful lectures.
in the past three days.
You have clearly shown the cultural locations.
and the key factors of the crisis.
of American white evangelicalism.
And your reflections on them.
shed light on the situation of Hong Kong.
or Chinese evangelical churches.
Well, after the crisis.
finally we move forward.
to today's topic on hope.
Please join with me to welcome Dr. Labberton.
to present on evangelicalism.
still or not yet.
to explore how the crisis of.

$^{41}$American white evangelicalism requires.
the restoration its soul.
by discovering its source and hope.
Thank you very much.
I appreciate again being here.
I have been very warmly welcomed.
by all of you and I'm grateful for that.
I want to say again my special thanks.
to your president Dr. Stephen Lee.
for the invitation to come.
and be part of these lectures.
And I look forward I hope to one day.
being able to visit CGST in person again.
It's a community that I have had.
connection with for over 30 years.
And I'm extremely grateful for your ministry.
and for your life together.
I'm aware of the significant challenges.
that many churches throughout Asia.
and particularly in Hong Kong.
are facing these days.
And therefore my own particular gratitude.
for your willingness to allow me.
to be part of your conversation.
And I hope and pray as I have from the beginning.
that these days together.
will be useful to you.
So the title of this final lecture.
as we've heard is Evangelicalism.
Still or Not Yet.
You'll notice in the English spelling of this.
that there's a hyphenation.
between evangel and icalism.
And one of the arguments that I want to make.
is represented by that simple construct.
that is that it's the evangel.
Jesus and the kingdom of God.
which is central.
Icalism names a movement.
a body of people.

$^{81}$a set of cultural practices.
and theological and spiritual.
and ethical stances.
that have been taken in light of the gospel.
But I want to argue tonight.
that our hope is found in thinking again.
about the evangel.
and letting that be what defines.
who we are and how we live in the world.
So let me begin then by saying this.
If there's going to be hope.
which I definitely believe there is.
Our American evangelical.
is going to find it by claiming.
that we are "still evangelical".
or asking the question.
Are we yet evangelical?.
In the first construct.
Are we still evangelical?.
The claim can often be.
that it's like a destination.
at which some have already arrived.
or a trajectory, by contrast.
a direction, but by no means yet reached.
If we're claiming that we're still evangelical.
it seems to me that we're in danger.
of claiming that we have arrived.
and that we need to hold on.
to the terrain that was defended.
and claimed before.
But if we ask the question.
Are we yet evangelical?.
We're asking.
Is our life and our ministry.
individually, personally and institutionally.
is it a reflection of the evangel itself?.
That puts us on a very, very different trajectory.
In other words, is evangelical an identity.
that we claim like a badge.
that we could put on our chest?.

$^{121}$Or is evangelicalism an invitation?.
Is evangelical mission.
an exercise in dominion.
with territory claimed.
and a flag firmly planted?.
Or is evangelical mission.
a witness to the invitation.
to find life in the kingdom of God.
and to become part of a new community.
that seeks to be an embodied reflection.
of Jesus Christ?.
To pose these binary alternatives.
reflects the ways that I find.
aspects of American evangelicalism.
to be tragic and offensive.
to the gospel itself.
Encroaching on the unique supremacy.
and authority of Jesus.
is sadly a deceptive duplicity.
that subverts the primary.
and the primacy and character of Jesus.
and supplants it with American culture.
dressed in gospel garb, gospel clothing.
It can make the sounds of the gospel.
while it lives a different and divergent story.
Of course it must be said.
that no expression of the gospel.
is free from compromise.
and human distortion.
And in Romans, or rather in Matthew 28.
it is an astonishing thing.
that Jesus gives the great commission.
that we call the mission.
that defines the mission of God's people in the world.
to 11 disciples, the text points out.
not the perfect number 12, but just to 11.
And the text says that some believed and some doubted.
And in Greek it's that some believed and doubted.
It's that the same people both believed and doubted.
The church is always fewer numbers than we would hope.

$^{161}$And it's always far from being the true and.
doubtless people that we might imagine.
To Jesus, this band of 11 believer doubters.
11 believer doubters.
This band was the right group.
to be God's witnesses to nothing less.
than the hope of the world.
This is all part of the shocking evangel.
God's redeeming love and justice in Jesus Christ.
should ground and define a new communion.
or community of disciples.
on a journey of transformation.
If the question is not still evangelical.
but rather yet evangelical.
then American white evangelicalism.
could begin the kind of reset.
that only the evangel itself offers.
Some of the primary elements of.
Bebbington's quadrilateral depiction of evangelicalism.
which we discussed earlier.
could contribute to the reclamation and recreation.
of American white evangelicalism.
So let me begin by making several points.
The first is this.
The hope for American white evangelicalism.
lies in the evangel.
It does not lie in evangelicalism.
The hope of evangelicalism in America.
is born by the evangel.
not by evangelicalism itself.
The hope is not that the icalism of evangelicalism.
past or present, that is not our hope.
Agendas, personalities, structures, assumptions.
prerogatives cloud and encrust the evangel.
and attach to it like barnacles.
to some kind of port.
The icalisms encase God's living word.
inside a particular set of language games.
regarding abortion, human sexuality, race, gender.
mannerliness, control and doctrine.

$^{201}$So around the evangel, in other words.
are these barnacles or these attachments.
that are claiming that they are.
virtually equal to the evangel itself.
A similar set of religious codes.
conscripted first century Israel.
It was into that context.
that Jesus announced the arrival of the kingdom of heaven.
or the kingdom of God.
and called unlikely disciples.
and generously taught and healed.
Jew and Gentile, righteous and unrighteous alike.
Jesus embodied a scandal.
that refused the boundaries of acceptability.
that confronted the abusive control of Sabbath keeping.
that honored the good Samaritan.
and that saw in the faith of a Roman centurion.
a faith that exceeded that which he had seen.
in all of Israel.
In other words, Jesus subverts the practices.
and claims of dominion and nationalism.
and individualism and agency and fear.
not least those things done in the name of God.
All of this meant paring away.
what was assumed to be primary.
in order to actually hear what was primary.
the good news that was and is truly good.
because of Jesus Christ.
American white evangelicals need the evangelism.
the evangel to save us from ourselves.
We need to be delivered, rescued from evangelicalism.
Upsetting the money changers in the temple.
seems like the right picture.
We stand inside the faith.
but we in one sense or another.
need the faith that has been practiced.
to be overthrown.
Procuring and providing implements of worship.
but distorting and distracting people from true worship.
was the problem in the temple.

$^{241}$When Jesus flips the tables.
and calls them thieves and a den of robbers.
He exposes what had been unassailable.
and it changes everything.
about identity, power, money, control and more.
It was a religious, social and economic system.
that was being dismantled at that moment.
as well as confronted.
Is it too much to believe.
that American white evangelicalism.
is in need of the same confrontation.
The purpose is not deserved judgment.
but instead the discovery of the good news.
again and as it were for the first time.
The evangelicals need to be converted.
back to the evangel.
The presumptions of evangelicals.
undermines the ability.
to allow the good news.
to do its fundamental confrontation.
conviction, repentance, lament and reconstruction.
Zacchaeus got this to the core.
and it changed his inner life.
and his public life.
It meant tangible actions of reparation.
in place of acts.
that up to that point.
had been all about self-preservation.
He got the word Jesus was making.
that Jesus was making all things new.
and that meant all of him.
for the sake of all who were around him.
As evangelicals we are known.
for talking conversion.
but we do not necessarily.
practice it ourselves.
The ongoing work of letting.
the actual evangel be our life.
is not the same as living.
inside evangelical culture.

$^{281}$Allowing Jesus Christ to confront.
and to break American evangelicalism.
of its distractions and its idols.
would be essential for its revival.
Do we need a road to Damascus confrontation?.
Saul was after all so assured.
that his spiritual, theological.
and political pedigree were in place.
Dominion, nationalism, identity, agency.
and fear were in full flight.
And then on the road to Damascus.
the risen Christ confronts him.
when he is right in the middle.
of his own self-righteous crusade.
with a life-redefining question.
"Saul, Saul, why do you persecute me?".
Cast down and blinded for three days.
but also fed and kept by God's provision.
And then he was unexpectedly welcomed.
by the touch of Ananias' words.
"Brother Saul, the Lord Jesus.
who appeared to you on the road.
by which you came has sent me.
so that you may regain your sight.
and be filled with the Holy Spirit.".
What confronts and redeems Saul.
is the evangel, it's not evangelicalism.
Through his blindness, through later.
13 years of study and reflection.
Paul lives as a mature.
and perfect reflection.
of an unexpected evangel.
that was still in the process.
of making him, Paul, new.
American white evangelicalism.
needs to hear and receive.
Jesus' confronting question.
"Why do you make my good news.
look and sound so bad?".
We need the evangel to rescue us.

$^{321}$The next point is this.
that American white evangelical.
needs the evangel.
to help evangelicalism die.
The apostle's birthright, his personal pride.
his persecuting dominion.
over Jesus' followers.
his religious stature.
as a Pharisee of Pharisees.
his political identity.
as a Roman citizen.
had to become, as he says in Philippians.
"dumb".
It was this dismantling of Saul.
that allowed him to become Paul.
The mark of his conversion.
was not in holding on to the prerogatives.
and privileges of which he had.
so understandably been proud.
while learning to profess Jesus.
Rather, his proud religious life.
his public life.
had to die.
so the new life in Christ.
could be born.
Laying down his trophies.
to identify, as he says.
with the fellowship of Christ's sufferings.
led Paul to speak and embody.
an entirely different life.
completely unlike the life.
that he had led before.
his conversion on the Damascus Road.
and the events that followed.
The success of American white evangelicalism.
has exposed, ironically, its failure.
By choosing to pursue the road of victory.
that has not led to human, ecclesial.
or social thriving.
for American white evangelicalism.

$^{361}$But too often, as we've explored.
to a barren and shameful display.
of self-serving appetites.
and self-protective interests.
This is hardly a mirror of the Lord.
that we claim to trust and follow.
American white evangelicalism.
is a part of the body of Christ.
that cannot be remade without dying.
to significant elements of our common life.
The death that is needed.
must confront its idolatrous assumptions.
as we've thought about communion.
about dominion, nationalism and individualism.
about whiteness and agency and fear.
These are the primary posts.
that define the broken and scandalous architecture.
within which American white evangelicalism.
has been built.
Life in Christ cannot thrive.
in this prison of privilege.
These things are a betrayal.
of an evangelistic life.
and they substitute human categories.
and privileges.
that are bent against the humble purposes.
of God's redemptive reign.
Failure to put these things to death.
exposes their power.
and our false allegiance.
No wonder American churches.
are being vacated by young adults.
and teenagers.
who fail to see a basic integrity.
of life that matches faith.
Why should these young people not leave?.
When we make them the problem.
that is the young adults and teenagers.
and fail to see that we.
that is the holders of the treasures.

$^{401}$of white evangelicalism in America.
that we are the problem.
the scale of the crisis only grows darker.
We fail to see what we have to first admit.
so that we can be changed.
If for example.
American white evangelicalism.
had truly repented.
of its deep complicity in racism.
and its participation in centuries.
of torturous and abusive slavery.
that would utterly change.
what has become the nearly impossible.
conversation today.
about reparations.
that is paying back in some way.
the people who are descendants of slaves.
This willingness.
of course our secular nation.
is not interested in that.
But the people of God.
would it not be the natural outcome.
of a true dying to the offense.
and scandal of slavery.
and the contemporary entanglements.
with racism and white supremacy.
for at least the people of God.
in the 21st century.
to work on behalf of appropriate reparations.
for slaves' descendants.
Perhaps reparations from the nation.
may not come.
But for the role of American white evangelicals.
to raise reparations for their role.
in the defense and benefit from slavery.
The promises of Zakiya echo.
Look Lord, here and now.
I give half of my positions to the poor.
And if I cheated anybody out of anything.
I will pay back four times the amount.

$^{441}$Jesus said to him.
Today salvation has come to this house.
Because this man too.
Is a son of Abraham.
For the Son of Man came to seek.
And to save the lost.
What is the point of American white evangelicals.
Defending and arguing against.
The witness of black and brown Christians.
And others in the United States.
Who testify to suffering.
Because of whiteness in general.
And Christian whiteness in particular.
Would not a spiritual humility.
Empower God's people.
To seek the truth.
To name reality.
To take responsibility.
To confess wrong.
To repent.
And to truly start again.
It is hard to find this.
As a deep track in the history of the world.
Of the American white evangelical church.
It has not profoundly and pervasively happened.
And the proof lies in the ongoing legitimacy.
Of the Christian movement.
That can so appropriately be labeled.
American white evangelicalism.
That total phrase would never be needed.
If the American church.
White evangelical church had responded.
The unnamed but ever-present reality of whiteness.
Shows the willingness of millions of Christian people.
To die to their cultural and racial privilege.
White evangelicals profess that dying leads to living.
But that is true.
Only if we live are dying.
The next point is this.
That the death of American white evangelicalism.

$^{481}$Could actually be its birth.
Revivals are typically defined.
By church historians as a combination of two.
Simultaneous factors.
Evidence both of personal transformation.
And of social transformation.
The need for each contributes to the necessity of the other.
The admission that personal change is required.
In order to produce social change.
And social change without personal change.
May be too superficial to make enduring change last.
Let's hypothetically imagine.
That the portrait of American white evangelicalism.
That I've given you.
Is sufficiently accurate to bring many to their knees.
And by God's grace to usher in a revival.
Of a deep and transformative kind.
Please join me in an imaginary exercise.
Of how a story of revival in America.
Not of secular and multi-religious society that America is.
But of the American white evangelicals themselves.
The story might read like this.
Dateline January 2020 or even better now January 2021.
American white evangelicals in several cities and towns.
Are showing themselves in an entirely new way.
In place of their typically proud and adamant rhetoric of confidence.
We are hearing now new voices with an entirely different tone.
Rather than them pointing blame at others.
These evangelicals are realizing the responsibilities.
That they have failed to practice.
Quietly at first.
But now building in congregations and other gatherings.
This community that we have heard from with such stridency.
Is now making confession.
People who described themselves previously.
As unwilling to hear and accept the experience of others.
Especially others who do not share their racial social background.
Are normally listening to their brothers and sisters of color.
It's like they are actually hearing these stories for the first time.
The report could go on perhaps like this.

$^{521}$Immigrants seeking justice.
Having had their child taken from them and caged before.
Are telling their stories.
And in place of defending borders.
And denying compassion and care for vulnerable people.
American white evangelicals are at the borders weeping.
And acknowledging the fears that have kept them so angry.
And defensive that they would want to keep other people out.
Gatherings of white evangelicals the report might go on.
Are now happening at the southern US border.
And people who fought for the wall to be built.
Are now lamenting at the suffering that they have helped cause.
They are gathering with Latino leaders in different parts of the country.
They are trying to start over in hearing the stories.
And grasping the injustice perpetrated against Latinos.
Too often previously unsupported by white evangelicals.
But now supported and cared for.
The report could go on.
Tours of the sites of lynchings against black men and women in America in the past.
Are now being swamped by gatherings of white evangelicals.
As are some of the primary sites of the Underground Railroad.
And of the Civil Rights Movement in the 1960s.
Reading groups seem to be springing up everywhere.
In towns and cities and in evangelical white churches.
Who are listening to the profound words of Frederick Douglass.
And W.E.B. Du Bois and of Martin Luther King.
And of Ruby Bridges and of Fannie Lou Hamer.
And of James Baldwin and of Malcolm $\times$and of Ta-Nehisi Coates.
These are all African Americans in America.
Publishers are reporting significant increases the report would go on.
In the sales of such writers.
And the prominence of these voices is finally continuing to grow.
It is like something that was never really seen or understood or trusted.
Is now front and center in American evangelical circles.
What's more than this.
Is that there are scenes of home and community gatherings.
Just for confession.
People are admitting the ways that they've defended themselves.
And rejected the voices of others who are not like them.
A related but different movement seems to be developing in churches.

$^{561}$That have treated women as second-class people.
And likewise need to be repented of.
They are being invited by their male pastors to tell their stories with trust.
So these women's voices can now be heard and believed.
And where actions are needed, including legal action.
They will take such steps and where appropriate, admit their complicity.
The theme of those that we have met.
Is that what is happening is personal, the report would go on.
But also institutional.
People realize that it is not just about a change of hearts and minds.
But about changes in the structures and habits of our public lives.
That need to be changed as well.
This feels revolutionary.
A black pastor and a Latino leader.
Said this was deeper and broader than they have ever seen it before.
It is not measured at this point, they said, by demonstrations.
But by gatherings or services of confession and lament and healing.
Led by black, Latino and Asian leaders of their churches.
Stay tuned for more.
Now this little imaginary exercise of mine is not a pure dream.
Such things are happening among some American white evangelicals.
But by too small a percentage.
Only a deep work of the Spirit of God.
Would make it possible for such events to develop.
As an organic reset of American white evangelicalism.
I share this dream with gospel hope.
But also with deep sorrow.
I think Jesus is the one who weeps over Washington, D.C. in recent weeks.
Or New York or Boston.
Chicago or Los Angeles.
And wants to see revival.
God's fulfillment will come.
But we are in the already and not yet in between.
Following the one in Jesus Christ who is with us on a long journey.
Which is never complete in this life.
Where faith will finally become sight.
And where we see fully and no longer.
Through a glass darkly.
We are given confidence but not certainty.
We are straining on at this point.

$^{601}$But not yet arrived.
We live in trust.
For what we have not yet fully seen.
Let's do something more modest if nonetheless profound.
After more than two decades of ministry.
Surrounded by the University of California, Berkeley.
I can say that it is an icon of secularism in America.
I was there serving as a pastor in Berkeley.
Surrounded by the campus.
And I very seldom found people unwilling to engage.
With issues and debates of religious meaning.
As long as there was room for mystery and humility.
For an acknowledgment of unfinishedness.
And uncertainty.
Christian faculty at the University of California, Berkeley.
They may not have had an easy time communicating with colleagues and students.
About how to understand the world through the lens of the Bible.
But even so, it was imaginable if two things were present.
If the professor gave true evidence of genuine humility.
And second, if they gave evidence of an honest life.
Consistent with a complicated world.
In the midst of their profession of faith.
If either humility or true action of honesty and candor were absent.
The result from students would be derision.
In the presence of both of these things.
There could be debate.
But there could also be respectful silence.
I remember one time when there were huge campus demonstrations.
On the campus of the University of California, Berkeley.
That were largely led by Jewish students.
Against what they saw as Islamic fascism.
And on the central plaza of the campus.
There had been daily gatherings during this one week.
Of thousands of students.
From Monday to Friday.
On the Friday of that week.
During the last day of demonstrations.
A remarkable undergraduate.
Who was a white evangelical woman that I know.
A student who had been present.

$^{641}$To support and encourage the dialogue between Jews and Muslims.
Took the microphone in the closing moments.
As the sun was beginning to go down over the plaza.
She was known by people there.
And she was known as a Christian.
But she had credibility.
To be able to say to the audience of Jews.
How grateful, and Muslims, how grateful she had been.
For the week of dialogue.
As passionate as it had been.
And then she surprised everyone.
By unexpectedly asking a group of Jews and Muslims.
A very straightforward question.
She simply said.
Please raise your hand.
If you believe that Jesus was the Son of God.
Not one person in this crowd of thousands did so.
She would have known that.
And in the stillness of this moment.
And with a very humble and earnest heart.
She slowly said to them all.
I love you.
The typically cynical editor of the student newspaper.
Later muted something like this.
I saw, I saved the last word of this editorial.
To honor Tinley Ireland, the young woman who had spoken.
And for what she did on that final afternoon.
I don't know or believe in her God.
But whatever that is.
And whatever love that is.
I need.
This is an authentic witness.
A credible evidence of the presence.
Love, justice and respect.
That the gospel, the evangel itself.
Is meant to produce.
If American white evangelicals.
Could begin a process of rebirth.
Then American white evangelicals.
Must recover a genuinely Christian identity.

$^{681}$The crisis of American white evangelicalism.
Can be summed up in the inadequacy.
If not the absence of these two qualities.
In place of humility.
Often an overreaching confidence.
That they and they alone know and define the truth.
In a place of honesty.
An admission of weakness and limitation.
They proclaim and over-assert.
What they don't know fully.
But claim to know and enact consistently.
This is what exposes.
The unacceptable hypocrisy to so many.
What aspects of Christian identity.
Of white evangelicals seem to have lost.
We might start here.
That as it says in the Heidelberg Catechism.
The answer to question one.
What is your only hope in life and in death.
To which the answer is.
We belong not to ourselves.
But that in body and soul.
In life and in death.
We belong to our faithful Savior Jesus Christ.
If you read this in light of the gospel narratives.
We do not then embrace a mere theological frame.
But a new way of being in the world.
That challenges and reprioritizes.
All other assumptions of our social, economic, racial, ethnic, and educational identity.
Our life is a life now "hidden in Christ".
And in relationship to which.
We come by the power of the Spirit to live into this new reality.
That God says is new.
Christ died to make this.
Which is necessary, namely our redemption and reparation.
Possible since we cannot redeem ourselves.
This is the identity into which we are meant to grow.
As maturing disciples.
When we are baptized.
And we acknowledge entering into Christ's death and resurrection.

$^{721}$And our own dying and rising.
It is not a mere religious ceremony.
But a sacred and public sign of the beginning of a new life.
This is a communal reality.
Not just an individual one.
And it happens before the church.
And in some sense before the world.
It is our identity, not my identity alone.
And it is with and for God's glory.
Including but not defined by my desires.
And it's for the building up of the body of Christ.
Including but not defined by my gifts.
And the evidence of God's life-transforming reign.
In whatever context we consider.
Including but not bound by my own social location.
As Paul says, for to me to live is Christ.
And to die is gain.
If American white evangelicals were to live this identity.
And find new life, find hope.
It means letting some things die.
Letting die our sense of self-definition.
Our sense of individualism.
Our sense that we hold primary agency.
Our sense that our Christian identity is primarily private and inward.
Our sense that we are followers of Jesus.
Marked with his death and resurrection.
And filled with his spirit and joined to others who share that new life.
And that look like us.
Our sense, in other words, that we are first white or American or male.
Rather than realizing we are first Christ's people.
American white evangelicals would also.
Not only need to recover a Christian identity.
But recover a Christian sense of location.
According to the Apostle Paul.
Our first and foremost location as Christian disciples.
Is that we live "in Christ".
This picture of our lives is both mystical and theological.
But it is also more incarnational.
Than those categories alone might suggest.
To be "in Christ" is to grow up in the capacity to see and think and feel.

$^{761}$And engage in communion with others.
To re-socialize one another in Christ.
To be "in Christ" is to dwell in our divine communion of Father, Son and Spirit.
This is deeply inward but not private and not only personal.
The need to realize our Christian social location.
If American white evangelicals were to live "in Christ".
It means letting some things die.
That this place is ours.
That our social location is first and foremost where we choose to live.
That we, that is white evangelicals, are in the top location.
That our social location is neutral when it's clearly not.
That we know the truth and the wisdom and the decisions to make.
And that our only life in Christ.
Knows how to learn to read our home, our town and our nation.
American white evangelicals.
In addition to identity and location.
Need to recover Christian communion.
God's people, we together here.
Right at this moment.
In Hong Kong, around the Chinese world.
Wherever that you might be listening.
And around the United States and throughout the whole globe.
We are to be the living evidence that Christ died and rose.
According to Philippians 2.
We do so by showing that we have died and risen.
To be part of a new humanity, a new communion.
Of unlike people.
This is how and why the church is to be.
If American white evangelicals were to live out this.
What would need to die?.
We would need to crucify the pride and presumption of our social location.
We would need to repent and lament and reconcile and restore our past.
And be willing to understand what we need to release.
About the expectation of finding a church.
That is basically finding comfort instead of finding transformation.
Finally, the future and hope of American white evangelicalism.
God's repertoire of recreating anything.
Whether it's the Chinese church or CGST or Fuller Theological Seminary.
Or American white evangelicalism is without limit.
God has every capacity and desire and will and purpose to recreate us.

$^{801}$God can and God will do what God can and will do.
Depending on the theological tradition that we are in.
We have presuppositions about what God might do.
I'm here not as the prophet to declare what God will do.
But to suggest what seems needed.
All of these comments that I've tried to share with you.
Are conditioned by God's choosing to be and do what he does in Jesus Christ.
Perhaps the least likely possible strategy for the redemption of the world.
That could have been imagined.
That a Jewish first century male.
Who incarnates nothing less than the whole of God's character in life.
Is the one by whom the whole world.
In all of its difference and diversity.
In all of its crisis and urgency.
Can be saved by Jesus Christ.
This is not a strategy any of us would have devised.
But it is at the very core.
It is the heart and the heartbeat of the evangelist self.
Therefore God's true heart and truth and imagination.
Continue to hold our goodness.
And our brokenness.
And any hope for our redemption.
Including hope for the white American evangelism we've been discussing.
I do not presume God will do what seems to my perspective to be so needed.
What I trust is that God is good.
That God is near.
And that God will complete what he has begun.
So for the sake of scattering.
The understandable despair.
Or resignation that I find.
In some of the crisis of God's church in the 21st century.
Let's remember that the enterprise that we are a part of.
Is that we know as reality.
To be God's wondrous project.
This mess that I've described.
As white American evangelicalism.
Is also part of the body of Christ himself.
We do not come to the table.
That is about human enterprise and effort.
But to the table of God's unending love.

$^{841}$And mercy and justice.
The purposes for which the world has been made.
The goodness with which God has created.
The severe brokenness that God is redeeming.
In the sacrifice of Jesus Christ.
And the new creation that is initiated in the resurrection.
The end to which God is moving all things.
By the presence and power of the Holy Spirit.
In the people of God and in the world.
These are the solid grounds of our collective hope.
And to that we give thanks and praise every day.
In Jesus' name, amen.
(英文).
\newpage



\section{}
\label{sec:0eVKvD_Q9zg}
\textbf{The Old Testament and Christian Ethics: How should we live? (3) — Dr Edwin Mung | Dr Chun-Luen Wu}
\newline
\newline
連結: \href{https://youtube.com/watch?v=0eVKvD-Q9zg}{\texttt{ https://youtube.com/watch?v=0eVKvD-Q9zg}} ~~~~ 語音日期: 2023-03-09 
\newline
\newline
\hyperref[sec:NlL3iKQIHto]{\small{< < < PREV SERMON < < <}}
~
\hyperref[sec:index]{\small{[返主目錄]}}
~
\hyperref[sec:5bv9U4Tdhps]{\small{> > > NEXT SERMON > > >}}
\newline
\newline
$^{1}$(音樂).
謝謝Dr. Wright的光明的講座.
這對我們來說,基督教的道德是比起一切的對與錯更重要的.
它是無法分別於我們的身份和目的.
現在是回應時間,我們請Dr. Edwin Moon, 老經教授,來為我們介紹一下.
他是中國使命教會的第一回應者.
接著是Dr. Chun-Lun Wu, 基督教教授, 會為我們介紹一下.
請Dr. Moon.
(掌聲).
謝謝Dr. Wright的光臨香港.
為我們介紹一系列的精彩和有意義的講座.
謝謝Pastor Wong和CGST邀請我來接受Dr. Wright的講座.
Dr. Wright是基督教道德的一個前進者.
他也是這篇文章的政治寫者,特別是在古代的文化界.
我們很多人都熟悉Dr. Wright的原理.
他將神,人類,地球的連結成一個橫幅.
並且將古代的耶穌基督和新教會的教會.
連結成一個小橫幅, 以太多方式和數學方式來形容.
我會在接下來的三個方面向Dr. Wright致謝.
首先,我非常欣賞Dr. Wright的穩妥和有意義的方式.
在教育主題中,古代和基督教道德,我們該如何生活?.
Dr. Wright明顯地展示了基督教道德的定義和基督教教義的表達.
Dr. Wright刻意選出了四個代表性的詞語.
從每一本基督教道德的書中取出數字.
我完全同意Dr. Wright的基督教道德和道德的相連性.
在亞伯拉罕選舉中的二個目標的列印.
在約翰遜18:18-19中的列印.
是為了一個正義的目標.
並且讓神的目標在選舉中顯得清晰.
亞伯拉罕選舉中的二個目標.
並且在約翰遜18:1-3中的列印.
是為了一個正義的目標.
並且讓神的目標在選舉中顯得清晰.
亞伯拉罕選舉中的二個目標.
並且讓神的目標在選舉中顯得清晰.
亞伯拉罕選舉中的二個目標.
Dr. Wright 確切地連結了.
亞伯拉罕選舉中的二個目標.
並且讓神的目標在選舉中顯得清晰.
亞伯拉罕選舉中的二個目標.

$^{41}$並且讓神的目標在選舉中顯得清晰.
亞伯拉罕選舉中的二個目標.
在亞伯拉罕選舉中的二個目標.
亞伯拉罕選舉中的二個目標.
在亞伯拉罕選舉中的二個目標.
\newpage



\section{}
\label{sec:5bv9U4Tdhps}
\textbf{The Old Testament and Christian Ethics: How should we live? (3) — Q & A Session}
\newline
\newline
連結: \href{https://youtube.com/watch?v=5bv9U4Tdhps}{\texttt{ https://youtube.com/watch?v=5bv9U4Tdhps}} ~~~~ 語音日期: 2023-02-16 
\newline
\newline
\hyperref[sec:0eVKvD_Q9zg]{\small{< < < PREV SERMON < < <}}
~
\hyperref[sec:index]{\small{[返主目錄]}}
~
\hyperref[sec:cUiQvS6fE24]{\small{> > > NEXT SERMON > > >}}
\newline
\newline
$^{1}$(音樂).
所以,你可能正在思考一些問題,或在寫一些文章.
我會跟隨Simon的步驟,並採取行動,來問這首問題.
謝謝Dr.Wight,您的光輝的演講.
您談到主的方式.
我非常欣賞您嘗試向我們解釋.
您只需要跟隨神的步驟,嘗試模仿祂,看清楚,並遵循服裝.
這讓我回想起,您也提到過.
耶穌在《聖經》中教導我們.
祂說:愛你的敵人.
祂說:祈求那些將你判處.
因為你的天父將使子女從邪惡與善惡中升起.
而聖靈降臨於正義和不正義.
我無法幫助您,但我會想.
難題不在這裡.
我們與神處於非常不同的位置.
我們處於敵人與同類人的位置.
但神是創造者,我們是祂的物質.
我希望您能分享您的想法.
如何幫助我們縮小了對於天父和天父的差別.
謝謝,這是一個很好的問題,謝謝你,Grace.
當然,在談論神的模仿或走祂的路時.
並不意味著我們在模仿神.
我們並非神,所以我們並非如此.
但當您正確地提到耶穌的行為.
我認為我們可以從祂的行為中得知.
當您正確地提到耶穌.
並不認為祂的徒弟們.
可以是神,但他們可以模仿祂的父親.
因為祂在聖經中經常提到祂的父親.
和你的天父.
就像耶穌說的.
如果你想知道如何在神的王國生活.
這就是這件事的主題.
那你就像我的父親.
就像你聽到天父的聲音.
就像我的兒子.
在其他詞語上,我們說的像是像是耶穌.
就像Paul說的.
你像我一樣模仿耶穌.

$^{41}$他為自己做了一個例子.
因為他認為自己在跟隨耶穌.
所以我認為說法是不對的.
只要你不假設我們在模仿神.
這就意味著我們問.
神的性格在聖經中顯示出什麼樣的性格.
最著名的是在福音書中.
是神的性格.
神是愛,恩恩,和正義的神.
三者都如此.
他看到耶穌在埃及的需要和痛苦.
他對他們擔憂,他對他們有恩恩.
然後他就行動去賠償他們.
所以當神啟動了耶穌.
讓他們像他一樣行動.
出雲之時,是被重複使用為了啟動.
你知道那是怎樣的.
當你身處那片土地,你會像外星人一樣.
而神愛你,救你.
因此你也要像那樣.
對那些處於那樣的位置.
在《詩》15:1-2中.
當一名以色列農民被告知.
要讓他的兒子離開.
他已經在六年間做了一種合作.
然後在七年間,你要讓他離開.
然後它說:祝福他,就像是你的神祝福了你.
這些話,可能是從耶穌的嘴裡傳來的.
是神在說.
你知道在這片土地下,神祝福你.
所以當你釋放這個奴隸時.
不要只把他送到不明的地方.
用很多的酒,油,橄欖,羊,和其他東西.
讓他能夠活著.
你要對他和神的行為.
我認為我們不需要扮演神.
我們需要問神的想法.
然後尋找他的方法.
這對你的問題有幫助嗎?.
我覺得這對我們自己的體驗.

$^{81}$神的恩典和祝福.
和神的兒女.
正是這樣.
Paul說過兩次.
Paul說:原諒彼此.
他不只是說原諒其他人.
而是說:如同神在耶穌的身邊原諒你.
所以神的原諒是為我們而設的.
而耶穌說:這是我的命令,你要愛彼此.
我認為你提到過,你強調了.
要愛彼此,但他還說:.
要愛彼此,如同我愛你.
而John說:我們愛,因為祂是第一次愛我們.
所以我們的愛,是為了神的愛設的.
我認為,一個重要的點是.
我們不需要要求要延伸神的存在.
而是要延伸神的性格.
或是要反映神的性格在我們的生活中.
所以,就像Dr. Wright說的.
我們不是要作弊於神.
但我們要反映神.
要反映神的形象在我們的日常生活中.
回到創造的問題.
神的目的.
是要讓我們.
活在神的形象和形象之下.
所以,在這個意義上.
我們應該活在.
或許在理論上來說.
是一個特徵.
而一個特徵.
是神自己說明了.
在我認為的.
《聖經》33:1.
祂是怒氣的雪人.
所以,我認為這就是為什麼.
我們要先與敵人相處.
第二,我認為.
在Hosea的書中.
這顯示了神的原諒.

$^{121}$對於不信任的人.
就像Hosea的神父.
甚至還賜予了.
我認為.
不信任的妻子.
我們可以從這張筆記上的問題.
來回答一個問題嗎?.
這關於基督教的行為和宗教的宗教宗教的宗教的分別.
我認為這位基督教兄弟或姐姐.
他或她在國外服務.
他們說,他們還在店裡.
他們被捕了.
他們去教會.
然後又被人偷走了.
所以,看來基督教人.
並沒有做出很好的偏見.
就像德國醫生所說的.
教會應該是世界或國家的模型.
您對此有什麼建議?.
我認為這個問題.
也能夠解答問題.
我認為.
這並不是一個證明.
神的正義和聖誠和愛.
如果那些人.
假設是他的人.
他們會做出.
世界也會懷疑的事.
所以.
這就是基督教人說的.
他們並不是.
不及國家的好.
他們甚至比國家更差.
這就是伊斯克利所說的.
有時候,這也可能是真的.
有時候,基督教人.
假設是基督教人.
他們也可能比非基督教人更差.
這很不公平.
這不應該是真的.

$^{161}$但很遺憾地,是真的.
但我想補充一點.
這在個人層面上.
很常發生.
但很遺憾地.
在機構層面上.
有時候是真的.
很可憐的是.
至少在西方.
我不太清楚香港的情況.
但在英國和美國.
我們有很多.
教會受到暴力.
暴力.
性暴力.
精神暴力.
他們對教會的反應非常嚴重.
難怪那些人.
從基督教中退出.
但我想補充一點.
就是.
承認和懺悔.
仍然是說.
世界上有很多地方.
基督教人.
做出了很好的事.
他們愛人.
他們犧牲.
他們關心窮人和難民.
他們是補救者.
基督教人一直這麼做.
這回到.
早期的教會.
在羅馬帝國.
有一個很美好的書.
我推薦給你.
是一位澳洲人的作者.
叫做John Dixon.
他的名字叫.
「暴力和聖人」.

$^{201}$這是基督教歷史.
他提到.
是的.
我們可以在歷史上.
說出很多對基督教的說話.
比如說.
「大戰」.
「領土」.
「獲得」.
基督教人在各種方式.
都行動得很差.
但也有一個.
經過基督教歷史的線條.
就是基督教人.
帶來善良和愛和同情.
人類的尊嚴和平等.
這在過去並不存在.
所以我推薦這本書給你.
「暴力和聖人」.
這本書給你兩邊的故事.
我想從另一個角度.
或從觀點上分享.
就是從.
「暴力和聖人」的角度.
來看.
因為.
我覺得.
基礎的情況.
我們有一個罪惡的人性.
所以就算是Paul.
他也哭了.
我認為是在《羅曼》.
他知道什麼是好事.
但他無法做到.
他知道那些不是好事.
他一直在做.
所以.
一個人.
或許是一個真正的信徒.
或是一個真正的正義的人.

$^{241}$需要時間來成長.
我有些學生.
他們也有喝了很多藥.
或是玩了很多遊戲.
他們被困在.
很多罪惡的習慣之中.
但即使他們自認.
成為信徒.
也需要時間.
需要時間來努力.
去成為聖人.
或學習成為聖人.
這是第一點.
第二點是從社會的角度來看.
某種盜竊.
或是所謂的犯罪行為.
並不是因為是一種罪惡行為.
也許是因為某種疾病.
或是.
我們現在需要去分辨.
到底是什麼最終的問題.
還是最根本的問題.
這是我的看法.
謝謝.
我只想補充最後一點.
是的.
福音書也認同.
福音書說.
神給我兩件事.
第一.
不要讓我變得富有.
讓我忘記你.
但也不要讓我變得貧窮.
所以福音書認同.
貧困有時候可以導致.
人們做一些他們不應該做的事.
甚至在英國.
我們也有過一些案例.
在法庭上.
法官們都說.

$^{281}$不要太過於堅持.
對於有些人.
因為他們沒有錢.
而不能給他們的家人吃飯.
因為現在的國家.
人們的貧困非常嚴重.
這並不代表盜竊是正確的.
但也證明了.
福音書也認同.
有時候有些情況.
是可以延長的.
但問題就像是.
某個人.
很快樂地.
違反了神的法律.
然後盜竊.
然後在教會上.
再次犯罪.
那樣就像是.
在耶路撒冷的七月.
當神對亞洲人說.
你會盜竊,殺害,犯罪.
然後來我的教堂.
說我們被救了.
我們安全了.
那種快樂的不遵從.
是不一樣的.
我會再提醒大家.
一個非常有概念性的.
香港人.
香港人的問題.
在2019年.
很多人受到傷害.
教會想要.
給這些受傷害的人.
更多的溫暖.
但這些受傷害的人.
還是基督徒.
他們仍然犯罪.
他們仍然沒有.

$^{321}$活下來.
或是完全違反了神的法律.
他們可能會.
對其他人帶來更多傷害.
所以受害者.
他們會再次受到傷害.
所以問題是.
我們應該如何防止.
或保護其他人.
從更多傷害其他人.
同時也要給予他們.
溫暖.
所以這就是一個問題.
這是一個難題.
我沒有一個簡單的答案.
當然.
歷史上都顯示.
受傷害的受害者.
被釋放.
很容易成為.
受傷害者.
古代文明.
從基督徒.
被釋放成為教會教會.
到蘇利曼成為教會教會.
他對自己的民族受到壓迫.
然後.
傑瑞波爾蒙.
要求他.
讓自己的民族離開.
正如穆斯林所說.
然後傑瑞波爾蒙.
成為受傷害者.
這是我們.
受傷害者的一部分.
我們重複犯罪.
然而.
這有點像.
問題是.
我們只要做好.

$^{361}$就能夠.
讓我們想要做好.
的人.
自己也會變得完美.
或者我們期望他們不會犯罪.
我認為.
在個人層面.
在倫敦的街上.
如果有人求救.
在那裡.
無家可歸.
你知道.
有些人在那裡.
是在欺騙.
有些人是一群人.
你知道.
所以你知道.
如果你向求救者投降.
你可能會被帶走.
可能.
我一直想.
那我應該投降嗎.
因為.
比較好.
投降.
知道你投降給.
需要的一些人.
要有溫暖.
而不是要放棄心.
說我不會投降.
因為他們都是欺騙.
我從來沒有想過.
這樣去批判.
因為我覺得.
就像耶穌說的.
神是不批判.
祂的寄生.
和祂的降臨.
今天香港有.
數百萬人.

$^{401}$不愛上神.
但神寄生了他們.
祂不只寄生.
那些愛祂的人.
所以我們需要.
謹慎.
愛護.
善待需要的人.
如果他們.
後來誤會.
我們會憂慮.
我們可能會.
向他們道歉.
或向他們報仇.
我們需要謙慎.
正如耶穌說的.
我們需要有.
理解和智慧.
而不是被當作傻子.
但我仍然覺得.
神的愛.
是更願意.
不被批判.
而是更加小心.
我們不會.
向任何人展示.
任何的愛和慈悲.
所以.
我想這是我.
想要回答的方式.
如果人們受傷.
愛他們.
照顧他們.
就像Paul說的.
如果你的敵人餓了.
餵他.
他可能不會.
變成你的敵人.
但無論如何餵他.
所以我覺得.

$^{441}$神的指令.
是愛和慈悲.
這會很常被傷害.
很常被不尊重.
但我們會繼續.
這也是基督的信息.
我想我們有問題.
從桌上.
謝謝Dr.Wright.
我的名字是Luke.
我是CJST的學生.
是我的榮幸.
可以問一個.
讓我感到非常有趣的問題.
就是我們討論的問題.
是關於古代的道德.
很容易就談到.
權力問題.
古代的教會.
是否有權力的一種.
是否有定義的一種.
在香港的情況下.
一些傳統的中國教會.
我們總是談論.
「回到聖經」.
「回到權力」.
我們面臨不同的.
道德問題.
包括一些.
政治或同性戀問題.
我們來聖餐禮.
學習聖經的使用方式.
學習希伯文.
學習德國.
以便我們可以有一個心.
只有道德的回答.
這種問題和諸如此類的.
我看過你的書.
《古代的道德》.
《2004年為人類的神》.

$^{481}$在你上一章提到的問題.
是關於和諧和權力的問題.
你也提到.
意義上和理論上的衝突.
是關於一體性的問題.
看來.
權力和道德.
並非從一個獨特的.
標準來源而來.
但是否有一個.
一體性的道德標準.
或是一個獨特的標準.
來源於古代的教義.
你也提到.
你會與.
新教的作者.
和他們的教義.
同步的回答.
但是.
這怎麼可能呢?.
你能否詳細解釋一下?.
我想問一個更實際的問題.
也給文中先生.
你也提到.
《古代的道德》的總結.
是要遵守和做事.
但是.
你提到的問題.
是現今的情況.
是一個很傷人的情況.
許多年輕的基督徒.
甚至是年輕的世代.
他們對權力非常敏感.
他們想要.
中心化權力.
他們說.
教會不應該有.
一個很好的權力.
他們想要的.
是他們要抗衡壓迫.

$^{521}$要獨立.
要反對所有的規則.
要避免受到壓迫.
他們會選擇.
用一些方法.
例如.
以後代的方式.
或以家庭的方式.
來正義地批評.
當然.
我不是在說CGST的學生.
CGST的學生都非常聰明.
但是.
文中先生.
怎麼可能.
在教會的舊教義.
在現今的香港.
你會怎麼對他們說?.
我還有兩個問題.
謝謝.
天啊.
這裡有很多問題.
我認為.
你剛才說的第一部分.
我很高興你讀了我的書.
《舊教義的道德》.
《為人類的神》.
因為它是在嘗試.
假設.
做一個預測.
我們基督徒.
我們認真地接受.
聖經中的教導.
當他說.
所有的經文.
也就是舊教義.
所有的經文.
都是受到神的啟發.
受到神的啟發.
舊教義.

$^{561}$是有利於教導.
批評.
訓練正義.
等等.
所以Paul假設.
舊教義的權力和重要性.
問題是.
他如何使用它?.
他使用它.
有很多方式.
如你所見.
例如.
他有時可以.
直接使用它.
例如.
天主教人.
不要盜竊.
他告訴盜竊人不要盜竊.
這是一條橋.
性的道德.
他告訴我們.
不要犯罪.
不要做這些事情.
他很清楚地.
基於這一點.
在其他方面.
他可以使用舊教義.
在一些.
我所形容的.
像是一個定義性的方式.
例如.
他在爭論.
基督教教會.
有責任.
支付和支持基督工作人員.
在《九個基督教》中.
他說.
法律也這樣說嗎?.
你會說.
在《圖書》中.

$^{601}$它在說什麼?.
在說支付基督工作人員.
它在說什麼?.
所以.
Paul 寫了一條法律.
關於牛.
法律說.
不要用牛來牽羊.
牠會把羊群弄出來.
換句話說.
一隻工作動物.
應該可以從.
牠的工作生產出來.
Paul 說.
是關於動物的問題嗎?.
還是關於我們?.
他在法律中.
應用了一種原則.
來應對.
一位基督工作人員.
如果神想要.
牠們的工作動物給牠們吃.
那麼神也想要.
給基督工作人員吃.
Paul 也有時可以.
用這個原則.
來應對.
而不是.
直接的服從.
我認為.
關於你的第二條.
關於法律的問題.
這很複雜.
首先.
在《九個基督教》中.
我記得.
你曾經說.
Dr. Wu.
我的翻譯.
《我們如何生活》.

$^{641}$或《我們如何生活》.
在中文中.
是寫了.
「遵守」或「做」.
之類的.
我實際上.
並不太高興.
因為.
當我用了「遵守」.
當然我指的是.
我們必須遵守神.
但是.
神要我們遵守.
是指.
我們必須.
遵守.
祂的教導.
我們必須遵守.
祂的教導.
我們必須遵守.
祂的教導.
是指.
我們必須遵守.
祂的教導.
我們必須遵守.
祂的教導.
我們必須遵守.
祂的教導.
我們必須遵守.
祂的教導.
我們必須遵守.
祂的教導.
我們必須遵守.
祂的教導.
我們必須遵守.
祂的教導.
我們必須遵守.
祂的教導.
我們必須遵守.
祂的教導.

$^{681}$我們必須遵守.
祂的教導.
我們必須遵守.
祂的教導.
我們必須遵守.
祂的教導.
我們必須遵守.
祂的教導.
我們必須遵守.
祂的教導.
我們必須遵守.
祂的教導.
我們必須遵守.
祂的教導.
我們必須遵守.
祂的教導.
我們必須遵守.
祂的教導.
我們必須遵守.
祂的教導.
我們必須遵守.
祂的教導.
我們必須遵守.
祂的教導.
我們必須遵守.
祂的教導.
我們必須遵守.
祂的教導.
我們必須遵守.
祂的教導.
我們必須遵守.
祂的教導.
我們必須遵守.
祂的教導.
我們必須遵守.
祂的教導.
我們必須遵守.
祂的教導.
我們必須遵守.
祂的教導.

$^{721}$我們必須遵守.
祂的教導.
我們必須遵守.
祂的教導.
我們必須遵守.
祂的教導.
我們必須遵守.
祂的教導.
我們必須遵守.
祂的教導.
我們必須遵守.
祂的教導.
我們必須遵守.
祂的教導.
我們必須遵守.
祂的教導.
我們必須遵守.
祂的教導.
我們必須遵守.
祂的教導.
我們必須遵守.
祂的教導.
我們必須遵守.
祂的教導.
我們必須遵守.
祂的教導.
我們必須遵守.
祂的教導.
我們必須遵守.
祂的教導.
我們必須遵守.
祂的教導.
我們必須遵守.
祂的教導.
我們必須遵守.
祂的教導.
我們必須遵守.
祂的教導.
我們必須遵守.
祂的教導.

$^{761}$我們必須遵守.
祂的教導.
我們必須遵守.
祂的教導.
我們必須遵守.
祂的教導.
我們必須遵守.
祂的教導.
我們必須遵守.
祂的教導.
我們必須遵守.
祂的教導.
我們必須遵守.
祂的教導.
我們必須遵守.
祂的教導.
我們必須遵守.
祂的教導.
我們必須遵守.
祂的教導.
我們必須遵守.
祂的教導.
我們必須遵守.
祂的教導.
我們必須遵守.
祂的教導.
我們必須遵守.
祂的教導.
我們必須遵守.
祂的教導.
我們必須遵守.
祂的教導.
我們必須遵守.
祂的教導.
我們必須遵守.
祂的教導.
我們必須遵守.
祂的教導.
我們必須遵守.
祂的教導.

$^{801}$我們必須遵守.
祂的教導.
我們必須遵守.
祂的教導.
我們必須遵守.
祂的教導.
我們必須遵守.
祂的教導.
我們必須遵守.
祂的教導.
我們必須遵守.
祂的教導.
我們必須遵守.
祂的教導.
我們必須遵守.
祂的教導.
我們必須遵守.
祂的教導.
我們必須遵守.
祂的教導.
我們必須遵守.
祂的教導.
我們必須遵守.
祂的教導.
我們必須遵守.
祂的教導.
我們必須遵守.
祂的教導.
我們必須遵守.
祂的教導.
我們必須遵守.
祂的教導.
我們必須遵守.
祂的教導.
我們必須遵守.
祂的教導.
我們必須遵守.
祂的教導.
我們必須遵守.
祂的教導.

$^{841}$我們必須遵守.
祂的教導.
我們必須遵守.
祂的教導.
我們必須遵守.
祂的教導.
我們必須遵守.
祂的教導.
我們必須遵守.
祂的教導.
我們必須遵守.
祂的教導.
我們必須遵守.
祂的教導.
我們必須遵守.
祂的教導.
我們必須遵守.
祂的教導.
我們必須遵守.
祂的教導.
我們必須遵守.
祂的教導.
我們必須遵守.
祂的教導.
我們必須遵守.
祂的教導.
我們必須遵守.
祂的教導.
我們必須遵守.
祂的教導.
我們必須遵守.
祂的教導.
我們必須遵守.
祂的教導.
我們必須遵守.
祂的教導.
我們必須遵守.
祂的教導.
我們必須遵守.
祂的教導.

$^{881}$我們必須遵守.
祂的教導.
我們必須遵守.
祂的教導.
我們必須遵守.
祂的教導.
我們必須遵守.
祂的教導.
我們必須遵守.
祂的教導.
我們必須遵守.
祂的教導.
我們必須遵守.
祂的教導.
我們必須遵守.
祂的教導.
我們必須遵守.
祂的教導.
我們必須遵守.
祂的教導.
我們必須遵守.
祂的教導.
我們必須遵守.
祂的教導.
我們必須遵守.
祂的教導.
我們必須遵守.
祂的教導.
我們必須遵守.
祂的教導.
我們必須遵守.
祂的教導.
我們必須遵守.
祂的教導.
我們必須遵守.
祂的教導.
我們必須遵守.
祂的教導.
我們必須遵守.
祂的教導.

$^{921}$我們必須遵守.
祂的教導.
我們必須遵守.
祂的教導.
我們必須遵守.
祂的教導.
我們必須遵守.
祂的教導.
我認為我們可以看整個古代教學的範圍.
我認為古代教學的範圍是整個範圍.
它是一種不同的種類.
有些教學會採取更多的權力.
但有些教學會採取更多的權力.
他們要重複他們的權力,但同時,他們也要承擔自己的責任..
他們要完成命令,要做出決定,要面對音樂,還要承受後果..
所以,從這個角度來看,當一些人,或一些年輕的基督徒,抵抗了教會的權力時,.
他們的底層問題,或他們的想法,或問題是否有誤解..
我認為,可能是基督徒,教會的教師,他們不明白年輕的基督徒的需要..
所以,我想我們需要有些交流和溝通..
但除此之外,如果我們只想要否認或轉變所有的權力,那會發生什麼?.
我認為,這會變得非常混亂..
所以,誰會去照顧教會,或去重建新的權力結構?.
我們必須面對這個問題..
我不知道教會學生,或教會學生聯合會,他們有很多的計畫,.
去改變教會,或改變自己..
但我覺得,這也是一個教會的問題..
當我成為教會的成員時,我真的不明白教會的教師的角色..
因為我並不在教會的職位之中..
我真的不明白,當教師,我必須面對很多事情,面對很多責任..
當然,我可以是一個非常不負責任的教師..
但這又是另一個問題..
我不知道我能否嘗試去解決,也許這不是一個答案,或是一些答案..
但讓我們思考一下,關於權力的問題..
謝謝,如果有時間,我可以,謝謝,非常有幫助..
再提一下,我們的兄弟,.
就像您所說的,權力有很多不同的意義..
我們通常會以軍事的角度去思考權力,.
或是誰告訴你該做什麼,老闆..
還有另一個角度,更加有動力的,.
就是那些能夠允許你做事的..

$^{961}$我擁有駕駛執照,那張駕駛執照對我來說是權力,.
但它不告訴我該駕駛哪裡..
我可以選擇駕駛,我遵守道路規則,.
但聖經不只是告訴我們該做什麼,.
因為很多聖經都不是這樣,.
它沒有任何指令,像您所說的,.
它充滿了智慧,充滿了故事,充滿了詩..
但是它是一本書,.
它允許我們,如果我們活在神的故事中,.
作為神的民族,.
我們有權利在詩中思考,.
以神的意識,.
以決定我們該如何活下去,.
以神的形象,.
以我們的故事,.
以我們的呼喚,.
以選擇行動..
這就是基於基督的自由,.
正如Paul所說,.
自由在聖靈中,.
透過聖靈的禮物和聖靈的果實,.
我們有權利在聖靈中活下去..
它是詩的權利,.
但它是一種權利,.
它允許我們活在世界的神的民族中..
它不是一種權利,.
它只是告訴我們該做什麼,.
即使有某些權利..
所以我們可以把它視為一個更加有動力的形式,.
不僅是在軍事上,.
也在有權利上..
我相信我們有很多很好的答案,.
也在香港處理很多時刻的問題..
但是我必須在這裡結束..
我想在這裡做幾個宣傳..
在樓下有一家書店,.
售賣Dr. Wright的書籍和文章,.
所以你們最好去看看..
我們下一個大總結,.
這個講座的系列,.

$^{1001}$今天晚上7點30分,.
會在大堂再次開會..
我們下次見..
謝謝..
(音樂).
\newpage



\section{}
\label{sec:cUiQvS6fE24}
\textbf{The Old Testament and Christian Ethics: How should we live? (3) — Speaker: Dr Christopher Wright}
\newline
\newline
連結: \href{https://youtube.com/watch?v=cUiQvS6fE24}{\texttt{ https://youtube.com/watch?v=cUiQvS6fE24}} ~~~~ 語音日期: 2023-02-16 
\newline
\newline
\hyperref[sec:5bv9U4Tdhps]{\small{< < < PREV SERMON < < <}}
~
\hyperref[sec:index]{\small{[返主目錄]}}
~
\hyperref[sec:XZUhcyDQxTc]{\small{> > > NEXT SERMON > > >}}
\newline
\newline
$^{1}$(音樂).
各位先生女士,先生們,各位基督徒弟妹們,各位主人們,各位觀眾,歡迎來到我們CJSC的文化與道德講座..
各位先生,先生們,各位基督徒弟妹們,各位觀眾,歡迎來到我們CJSC的文化與道德講座..
自1985年起,CJSC在一年多來,一直在宣傳這個講座,以紀念約瑟琳·索,在1982年她的不幸死亡後,香港的一位信仰主義者和領導人..
今年,我們的講座題目是《寫給我們看的古代教義如基督教》.我們將來要看看古代教義如何對我們基督教而有關..
我們很榮幸有許多人來參與.我們也很榮幸有許多人來參與..
我們也很榮幸有許多人來參與..
謝謝,博士.這是一個非常溫暖的歡迎.今天很高興能在這麼陽光的天氣下來..
這不像是在倫敦,但還是像是在香港,我記得之前的訪問..
我再次在今天下午的同樣情況下,我再次在倫敦的所有靈性教會的Langham教會的家庭教會,我向大家致謝..
Langham Foundation的名字來自於Langham Foundation的名字,在倫敦的那條街道,John Stott曾經是一位主席,.
他創立了Langham Partnership,在香港的Langham Foundation..
所以,我們向您致謝..
就像博士所說的,我們今天將進入第三個課堂,.
在我們第一課堂中,我們看到了我們基督教的身份,.
是基於古教的身份,而形成了基督教的身份..
在基督教中,我們,多國的信徒,基督徒和基督徒,.
博士說的,是阿伯拉罕的種子,.
我們在那次課堂中觀察到,.
在那次課堂中,我們觀察到,.
在那次課堂中,.
我們看到了基督教的身份,.
在那次課堂中,.
我們看到了基督教的身份,.
在那次課堂中,.
我們看到了基督教的身份,.
在那次課堂中,.
我們看到了基督教的身份,.
在那次課堂中,.
我們看到了基督教的身份,.
在那次課堂中,.
我們看到了基督教的身份,.
在那次課堂中,.
我們看到了基督教的身份,.
在那次課堂中,.
我們看到了基督教的身份,.
在那次課堂中,.
我們看到了基督教的身份,.
在那次課堂中,.
我們看到了基督教的身份,.

$^{41}$在那次課堂中,.
我們看到了基督教的身份,.
在那次課堂中,.
我們看到了基督教的身份,.
在那次課堂中,.
我們看到了基督教的身份,.
在那次課堂中,.
我們看到了基督教的身份,.
在那次課堂中,.
我們看到了基督教的身份,.
在那次課堂中,.
我們看到了基督教的身份,.
在那次課堂中,.
我們看到了基督教的身份,.
在那次課堂中,.
我們看到了基督教的身份,.
在那次課堂中,.
我們看到了基督教的身份,.
在那次課堂中,.
我們看到了基督教的身份,.
在那次課堂中,.
我們看到了基督教的身份,.
在那次課堂中,.
我們看到了基督教的身份,.
在那次課堂中,.
我們看到了基督教的身份,.
在那次課堂中,.
我們看到了基督教的身份,.
在那次課堂中,.
我們看到了基督教的身份,.
在那次課堂中,.
我們看到了基督教的身份,.
在那次課堂中,.
我們看到了基督教的身份,.
在那次課堂中,.
我們看到了基督教的身份,.
在那次課堂中,.
我們看到了基督教的身份,.
在那次課堂中,.
我們看到了基督教的身份,.

$^{81}$在那次課堂中,.
我們看到了基督教的身份,.
在那次課堂中,.
我們看到了基督教的身份,.
在那次課堂中,.
我們看到了基督教的身份,.
在那次課堂中,.
我們看到了基督教的身份,.
在那次課堂中,.
我們看到了基督教的身份,.
在那次課堂中,.
我們看到了基督教的身份,.
在那次課堂中,.
我們看到了基督教的身份,.
在那次課堂中,.
我們看到了基督教的身份,.
在那次課堂中,.
我們看到了基督教的身份,.
在那次課堂中,.
我們看到了基督教的身份,.
在那次課堂中,.
我們看到了基督教的身份,.
在那次課堂中,.
我們看到了基督教的身份,.
在那次課堂中,.
我們看到了基督教的身份,.
在那次課堂中,.
我們看到了基督教的身份,.
在那次課堂中,.
我們看到了基督教的身份,.
在那次課堂中,.
我們看到了基督教的身份,.
在那次課堂中,.
我們看到了基督教的身份,.
在那次課堂中,.
我們看到了基督教的身份,.
在那次課堂中,.
我們看到了基督教的身份,.
在那次課堂中,.
我們看到了基督教的身份,.

$^{121}$在那次課堂中,.
我們看到了基督教的身份,.
在那次課堂中,.
我們看到了基督教的身份,.
在那次課堂中,.
我們看到了基督教的身份,.
在那次課堂中,.
我們看到了基督教的身份,.
在那次課堂中,.
我們看到了基督教的身份,.
在那次課堂中,.
我們看到了基督教的身份,.
在那次課堂中,.
我們看到了基督教的身份,.
在那次課堂中,.
我們看到了基督教的身份,.
在那次課堂中,.
我們看到了基督教的身份,.
在那次課堂中,.
我們看到了基督教的身份,.
在那次課堂中,.
我們看到了基督教的身份,.
在那次課堂中,.
我們看到了基督教的身份,.
在那次課堂中,.
我們看到了基督教的身份,.
在那次課堂中,.
我們看到了基督教的身份,.
在那次課堂中,.
我們看到了基督教的身份,.
在那次課堂中,.
我們看到了基督教的身份,.
在那次課堂中,.
我們看到了基督教的身份,.
在那次課堂中,.
我們看到了基督教的身份,.
在那次課堂中,.
我們看到了基督教的身份,.
在那次課堂中,.
我們看到了基督教的身份,.

$^{161}$在那次課堂中,.
我們看到了基督教的身份,.
在那次課堂中,.
我們看到了基督教的身份,.
在那次課堂中,.
我們看到了基督教的身份,.
在那次課堂中,.
我們看到了基督教的身份,.
在那次課堂中,.
我們看到了基督教的身份,.
在那次課堂中,.
我們看到了基督教的身份,.
在那次課堂中,.
我們看到了基督教的身份,.
在那次課堂中,.
我們看到了基督教的身份,.
在那次課堂中,.
我們看到了基督教的身份,.
在那次課堂中,.
我們看到了基督教的身份,.
在那次課堂中,.
我們看到了基督教的身份,.
在那次課堂中,.
我們看到了基督教的身份,.
在那次課堂中,.
我們看到了基督教的身份,.
在那次課堂中,.
我們看到了基督教的身份,.
在那次課堂中,.
我們看到了基督教的身份,.
在那次課堂中,.
我們看到了基督教的身份,.
在那次課堂中,.
我們看到了基督教的身份,.
在那次課堂中,.
我們看到了基督教的身份,.
在那次課堂中,.
我們看到了基督教的身份,.
在那次課堂中,.
我們看到了基督教的身份,.

$^{201}$在那次課堂中,.
我們看到了基督教的身份,.
在那次課堂中,.
我們看到了基督教的身份,.
在那次課堂中,.
我們看到了基督教的身份,.
在那次課堂中,.
我們看到了基督教的身份,.
在那次課堂中,.
我們看到了基督教的身份,.
在那次課堂中,.
我們看到了基督教的身份,.
在那次課堂中,.
我們看到了基督教的身份,.
在那次課堂中,.
我們看到了基督教的身份,.
在那次課堂中,.
我們看到了基督教的身份,.
在那次課堂中,.
我們看到了基督教的身份,.
在那次課堂中,.
我們看到了基督教的身份,.
在那次課堂中,.
我們看到了基督教的身份,.
在那次課堂中,.
我們看到了基督教的身份,.
在那次課堂中,.
我們看到了基督教的身份,.
在那次課堂中,.
我們看到了基督教的身份,.
在那次課堂中,.
我們看到了基督教的身份,.
在那次課堂中,.
我們看到了基督教的身份,.
在那次課堂中,.
我們看到了基督教的身份,.
在那次課堂中,.
我們看到了基督教的身份,.
在那次課堂中,.
我們看到了基督教的身份,.

$^{241}$在那次課堂中,.
我們看到了基督教的身份,.
在那次課堂中,.
我們看到了基督教的身份,.
在那次課堂中,.
我們看到了基督教的身份,.
在那次課堂中,.
我們看到了基督教的身份,.
在那次課堂中,.
我們看到了基督教的身份,.
在那次課堂中,.
我們看到了基督教的身份,.
在那次課堂中,.
我們看到了基督教的身份,.
在那次課堂中,.
我們看到了基督教的身份,.
在那次課堂中,.
我們看到了基督教的身份,.
在那次課堂中,.
我們看到了基督教的身份,.
在那次課堂中,.
我們看到了基督教的身份,.
在那次課堂中,.
我們看到了基督教的身份,.
在那次課堂中,.
我們看到了基督教的身份,.
在那次課堂中,.
我們看到了基督教的身份,.
在那次課堂中,.
我們看到了基督教的身份,.
在那次課堂中,.
我們看到了基督教的身份,.
在那次課堂中,.
我們看到了基督教的身份,.
在那次課堂中,.
我們看到了基督教的身份,.
在那次課堂中,.
我們看到了基督教的身份,.
在那次課堂中,.
我們看到了基督教的身份,.

$^{281}$在那次課堂中,.
我們看到了基督教的身份,.
在那次課堂中,.
我們看到了基督教的身份,.
在那次課堂中,.
我們看到了基督教的身份,.
在那次課堂中,.
我們看到了基督教的身份,.
在那次課堂中,.
我們看到了基督教的身份,.
在那次課堂中,.
我們看到了基督教的身份,.
在那次課堂中,.
我們看到了基督教的身份,.
在那次課堂中,.
我們看到了基督教的身份,.
在那次課堂中,.
我們看到了基督教的身份,.
在那次課堂中,.
我們看到了基督教的身份,.
在那次課堂中,.
我們看到了基督教的身份,.
在那次課堂中,.
我們看到了基督教的身份,.
在那次課堂中,.
我們看到了基督教的身份,.
在那次課堂中,.
我們看到了基督教的身份,.
在那次課堂中,.
我們看到了基督教的身份,.
在那次課堂中,.
我們看到了基督教的身份,.
在那次課堂中,.
我們看到了基督教的身份,.
在那次課堂中,.
我們看到了基督教的身份,.
在那次課堂中,.
我們看到了基督教的身份,.
在那次課堂中,.
我們看到了基督教的身份,.

$^{321}$在那次課堂中,.
我們看到了基督教的身份,.
在那次課堂中,.
我們看到了基督教的身份,.
在那次課堂中,.
我們看到了基督教的身份,.
在那次課堂中,.
我們看到了基督教的身份,.
在那次課堂中,.
我們看到了基督教的身份,.
在那次課堂中,.
我們看到了基督教的身份,.
在那次課堂中,.
我們看到了基督教的身份,.
在那次課堂中,.
我們看到了基督教的身份,.
在那次課堂中,.
我們看到了基督教的身份,.
在那次課堂中,.
我們看到了基督教的身份,.
在那次課堂中,.
我們看到了基督教的身份,.
在那次課堂中,.
我們看到了基督教的身份,.
在那次課堂中,.
我們看到了基督教的身份,.
在那次課堂中,.
我們看到了基督教的身份,.
在那次課堂中,.
我們看到了基督教的身份,.
在那次課堂中,.
我們看到了基督教的身份,.
在那次課堂中,.
我們看到了基督教的身份,.
在那次課堂中,.
我們看到了基督教的身份,.
在那次課堂中,.
我們看到了基督教的身份,.
在那次課堂中,.
我們看到了基督教的身份,.

$^{361}$在那次課堂中,.
我們看到了基督教的身份,.
在那次課堂中,.
我們看到了基督教的身份,.
在那次課堂中,.
我們看到了基督教的身份,.
在那次課堂中,.
我們看到了基督教的身份,.
在那次課堂中,.
我們看到了基督教的身份,.
在那次課堂中,.
我們看到了基督教的身份,.
在那次課堂中,.
我們看到了基督教的身份,.
在那次課堂中,.
我們看到了基督教的身份,.
在那次課堂中,.
我們看到了基督教的身份,.
在那次課堂中,.
我們看到了基督教的身份,.
在那次課堂中,.
我們看到了基督教的身份,.
在那次課堂中,.
我們看到了基督教的身份,.
在那次課堂中,.
我們看到了基督教的身份,.
在那次課堂中,.
我們看到了基督教的身份,.
在那次課堂中,.
我們看到了基督教的身份,.
在那次課堂中,.
我們看到了基督教的身份,.
在那次課堂中,.
我們看到了基督教的身份,.
在那次課堂中,.
我們看到了基督教的身份,.
在那次課堂中,.
我們看到了基督教的身份,.
在那次課堂中,.
我們看到了基督教的身份,.

$^{401}$在那次課堂中,.
我們看到了基督教的身份,.
在那次課堂中,.
我們看到了基督教的身份,.
在那次課堂中,.
我們看到了基督教的身份,.
在那次課堂中,.
我們看到了基督教的身份,.
在那次課堂中,.
我們看到了基督教的身份,.
在那次課堂中,.
我們看到了基督教的身份,.
在那次課堂中,.
我們看到了基督教的身份,.
在那次課堂中,.
我們看到了基督教的身份,.
在那次課堂中,.
我們看到了基督教的身份,.
在那次課堂中,.
我們看到了基督教的身份,.
在那次課堂中,.
我們看到了基督教的身份,.
在那次課堂中,.
我們看到了基督教的身份,.
在那次課堂中,.
我們看到了基督教的身份,.
在那次課堂中,.
我們看到了基督教的身份,.
在那次課堂中,.
我們看到了基督教的身份,.
在那次課堂中,.
我們看到了基督教的身份,.
在那次課堂中,.
我們看到了基督教的身份,.
在那次課堂中,.
我們看到了基督教的身份,.
在那次課堂中,.
我們看到了基督教的身份,.
在那次課堂中,.
我們看到了基督教的身份,.

$^{441}$在那次課堂中,.
我們看到了基督教的身份,.
在那次課堂中,.
我們看到了基督教的身份,.
在那次課堂中,.
我們看到了基督教的身份,.
在那次課堂中,.
我們看到了基督教的身份,.
在那次課堂中,.
我們看到了基督教的身份,.
在那次課堂中,.
我們看到了基督教的身份,.
在那次課堂中,.
我們看到了基督教的身份,.
在那次課堂中,.
我們看到了基督教的身份,.
在那次課堂中,.
我們看到了基督教的身份,.
在那次課堂中,.
我們看到了基督教的身份,.
在那次課堂中,.
我們看到了基督教的身份,.
在那次課堂中,.
我們看到了基督教的身份,.
在那次課堂中,.
我們看到了基督教的身份,.
在那次課堂中,.
我們看到了基督教的身份,.
在那次課堂中,.
我們看到了基督教的身份,.
在那次課堂中,.
我們看到了基督教的身份,.
在那次課堂中,.
我們看到了基督教的身份,.
在那次課堂中,.
我們看到了基督教的身份,.
在那次課堂中,.
我們看到了基督教的身份,.
在那次課堂中,.
我們看到了基督教的身份,.

$^{481}$在那次課堂中,.
我們看到了基督教的身份,.
在那次課堂中,.
我們看到了基督教的身份,.
在那次課堂中,.
我們看到了基督教的身份,.
在那次課堂中,.
我們看到了基督教的身份,.
在那次課堂中,.
我們看到了基督教的身份,.
在那次課堂中,.
我們看到了基督教的身份,.
在那次課堂中,.
我們看到了基督教的身份,.
在那次課堂中,.
我們看到了基督教的身份,.
在那次課堂中,.
我們看到了基督教的身份,.
在那次課堂中,.
我們看到了基督教的身份,.
在那次課堂中,.
我們看到了基督教的身份,.
在那次課堂中,.
我們看到了基督教的身份,.
在那次課堂中,.
我們看到了基督教的身份,.
在那次課堂中,.
我們看到了基督教的身份,.
在那次課堂中,.
我們看到了基督教的身份,.
在那次課堂中,.
我們看到了基督教的身份,.
在那次課堂中,.
我們看到了基督教的身份,.
在那次課堂中,.
我們看到了基督教的身份,.
在那次課堂中,.
我們看到了基督教的身份,.
在那次課堂中,.
我們看到了基督教的身份,.

$^{521}$在那次課堂中,.
我們看到了基督教的身份,.
在那次課堂中,.
我們看到了基督教的身份,.
在那次課堂中,.
我們看到了基督教的身份,.
在那次課堂中,.
我們看到了基督教的身份,.
在那次課堂中,.
我們看到了基督教的身份,.
在那次課堂中,.
我們看到了基督教的身份,.
在那次課堂中,.
我們看到了基督教的身份,.
在那次課堂中,.
我們看到了基督教的身份,.
在那次課堂中,.
我們看到了基督教的身份,.
在那次課堂中,.
我們看到了基督教的身份,.
在那次課堂中,.
我們看到了基督教的身份,.
在那次課堂中,.
我們看到了基督教的身份,.
在那次課堂中,.
我們看到了基督教的身份,.
在那次課堂中,.
我們看到了基督教的身份,.
在那次課堂中,.
我們看到了基督教的身份,.
在那次課堂中,.
我們看到了基督教的身份,.
在那次課堂中,.
我們看到了基督教的身份,.
在那次課堂中,.
我們看到了基督教的身份,.
在那次課堂中,.
我們看到了基督教的身份,.
在那次課堂中,.
我們看到了基督教的身份,.

$^{561}$在那次課堂中,.
我們看到了基督教的身份,.
在那次課堂中,.
我們看到了基督教的身份,.
在那次課堂中,.
我們看到了基督教的身份,.
在那次課堂中,.
我們看到了基督教的身份,.
在那次課堂中,.
我們看到了基督教的身份,.
在那次課堂中,.
我們看到了基督教的身份,.
在那次課堂中,.
我們看到了基督教的身份,.
在那次課堂中,.
我們看到了基督教的身份,.
在那次課堂中,.
我們看到了基督教的身份,.
在那次課堂中,.
我們看到了基督教的身份,.
在那次課堂中,.
我們看到了基督教的身份,.
在那次課堂中,.
我們看到了基督教的身份,.
在那次課堂中,.
我們看到了基督教的身份,.
在那次課堂中,.
我們看到了基督教的身份,.
在那次課堂中,.
我們看到了基督教的身份,.
在那次課堂中,.
我們看到了基督教的身份,.
在那次課堂中,.
我們看到了基督教的身份,.
在那次課堂中,.
我們看到了基督教的身份,.
在那次課堂中,.
我們看到了基督教的身份,.
在那次課堂中,.
我們看到了基督教的身份,.

$^{601}$在那次課堂中,.
我們看到了基督教的身份,.
在那次課堂中,.
我們看到了基督教的身份,.
在那次課堂中,.
我們看到了基督教的身份,.
在那次課堂中,.
我們看到了基督教的身份,.
在那次課堂中,.
我們看到了基督教的身份,.
在那次課堂中,.
我們看到了基督教的身份,.
在那次課堂中,.
我們看到了基督教的身份,.
在那次課堂中,.
我們看到了基督教的身份,.
在那次課堂中,.
我們看到了基督教的身份,.
在那次課堂中,.
我們看到了基督教的身份,.
在那次課堂中,.
我們看到了基督教的身份,.
在那次課堂中,.
我們看到了基督教的身份,.
在那次課堂中,.
我們看到了基督教的身份,.
在那次課堂中,.
我們看到了基督教的身份,.
在那次課堂中,.
我們看到了基督教的身份,.
在那次課堂中,.
我們看到了基督教的身份,.
在那次課堂中,.
我們看到了基督教的身份,.
在那次課堂中,.
我們看到了基督教的身份,.
在那次課堂中,.
我們看到了基督教的身份,.
在那次課堂中,.
我們看到了基督教的身份,.

$^{641}$在那次課堂中,.
我們看到了基督教的身份,.
在那次課堂中,.
我們看到了基督教的身份,.
在那次課堂中,.
我們看到了基督教的身份,.
在那次課堂中,.
我們看到了基督教的身份,.
在那次課堂中,.
我們看到了基督教的身份,.
在那次課堂中,.
我們看到了基督教的身份,.
在那次課堂中,.
我們看到了基督教的身份,.
在那次課堂中,.
我們看到了基督教的身份,.
在那次課堂中,.
我們看到了基督教的身份,.
在那次課堂中,.
我們看到了基督教的身份,.
在那次課堂中,.
我們看到了基督教的身份,.
在那次課堂中,.
我們看到了基督教的身份,.
在那次課堂中,.
我們看到了基督教的身份,.
在那次課堂中,.
我們看到了基督教的身份,.
在那次課堂中,.
我們看到了基督教的身份,.
在那次課堂中,.
我們看到了基督教的身份,.
在那次課堂中,.
我們看到了基督教的身份,.
在那次課堂中,.
我們看到了基督教的身份,.
在那次課堂中,.
我們看到了基督教的身份,.
在那次課堂中,.
我們看到了基督教的身份,.

$^{681}$在那次課堂中,.
我們看到了基督教的身份,.
在那次課堂中,.
我們看到了基督教的身份,.
在那次課堂中,.
我們看到了基督教的身份,.
在那次課堂中,.
我們看到了基督教的身份,.
在那次課堂中,.
我們看到了基督教的身份,.
在那次課堂中,.
我們看到了基督教的身份,.
在那次課堂中,.
我們看到了基督教的身份,.
在那次課堂中,.
我們看到了基督教的身份,.
在那次課堂中,.
我們看到了基督教的身份,.
在那次課堂中,.
我們看到了基督教的身份,.
在那次課堂中,.
我們看到了基督教的身份,.
在那次課堂中,.
我們看到了基督教的身份,.
在那次課堂中,.
我們看到了基督教的身份,.
在那次課堂中,.
我們看到了基督教的身份,.
在那次課堂中,.
我們看到了基督教的身份,.
在那次課堂中,.
我們看到了基督教的身份,.
在那次課堂中,.
我們看到了基督教的身份,.
在那次課堂中,.
我們看到了基督教的身份,.
在那次課堂中,.
我們看到了基督教的身份,.
在那次課堂中,.
我們看到了基督教的身份,.

$^{721}$在那次課堂中,.
我們看到了基督教的身份,.
在那次課堂中,.
我們看到了基督教的身份,.
在那次課堂中,.
我們看到了基督教的身份,.
在那次課堂中,.
我們看到了基督教的身份,.
在那次課堂中,.
我們看到了基督教的身份,.
在那次課堂中,.
我們看到了基督教的身份,.
在那次課堂中,.
我們看到了基督教的身份,.
在那次課堂中,.
我們看到了基督教的身份,.
在那次課堂中,.
我們看到了基督教的身份,.
在那次課堂中,.
我們看到了基督教的身份,.
在那次課堂中,.
我們看到了基督教的身份,.
在那次課堂中,.
我們看到了基督教的身份,.
在那次課堂中,.
我們看到了基督教的身份,.
在那次課堂中,.
我們看到了基督教的身份,.
在那次課堂中,.
我們看到了基督教的身份,.
在那次課堂中,.
我們看到了基督教的身份,.
在那次課堂中,.
我們看到了基督教的身份,.
在那次課堂中,.
我們看到了基督教的身份,.
在那次課堂中,.
我們看到了基督教的身份,.
在那次課堂中,.
我們看到了基督教的身份,.

$^{761}$在那次課堂中,.
我們看到了基督教的身份,.
在那次課堂中,.
我們看到了基督教的身份,.
在那次課堂中,.
我們看到了基督教的身份,.
在那次課堂中,.
我們看到了基督教的身份,.
在那次課堂中,.
我們看到了基督教的身份,.
在那次課堂中,.
我們看到了基督教的身份,.
在那次課堂中,.
我們看到了基督教的身份,.
在那次課堂中,.
我們看到了基督教的身份,.
在那次課堂中,.
我們看到了基督教的身份,.
在那次課堂中,.
我們看到了基督教的身份,.
在那次課堂中,.
我們看到了基督教的身份,.
在那次課堂中,.
我們看到了基督教的身份,.
在那次課堂中,.
我們看到了基督教的身份,.
在那次課堂中,.
我們看到了基督教的身份,.
在那次課堂中,.
我們看到了基督教的身份,.
在那次課堂中,.
我們看到了基督教的身份,.
在那次課堂中,.
我們看到了基督教的身份,.
在那次課堂中,.
我們看到了基督教的身份,.
在那次課堂中,.
我們看到了基督教的身份,.
在那次課堂中,.
我們看到了基督教的身份,.

$^{801}$在那次課堂中,.
我們看到了基督教的身份,.
在那次課堂中,.
我們看到了基督教的身份,.
在那次課堂中,.
我們看到了基督教的身份,.
在那次課堂中,.
我們看到了基督教的身份,.
在那次課堂中,.
我們看到了基督教的身份,.
在那次課堂中,.
我們看到了基督教的身份,.
在那次課堂中,.
我們看到了基督教的身份,.
在那次課堂中,.
我們看到了基督教的身份,.
在那次課堂中,.
我們看到了基督教的身份,.
在那次課堂中,.
我們看到了基督教的身份,.
在那次課堂中,.
我們看到了基督教的身份,.
在那次課堂中,.
我們看到了基督教的身份,.
在那次課堂中,.
我們看到了基督教的身份,.
在那次課堂中,.
我們看到了基督教的身份,.
在那次課堂中,.
我們看到了基督教的身份,.
在那次課堂中,.
我們看到了基督教的身份,.
在那次課堂中,.
我們看到了基督教的身份,.
在那次課堂中,.
我們看到了基督教的身份,.
在那次課堂中,.
我們看到了基督教的身份,.
在那次課堂中,.
我們看到了基督教的身份,.

$^{841}$在那次課堂中,.
我們看到了基督教的身份,.
在那次課堂中,.
我們看到了基督教的身份,.
在那次課堂中,.
我們看到了基督教的身份,.
在那次課堂中,.
我們看到了基督教的身份,.
在那次課堂中,.
我們看到了基督教的身份,.
在那次課堂中,.
我們看到了基督教的身份,.
在那次課堂中,.
我們看到了基督教的身份,.
在那次課堂中,.
我們看到了基督教的身份,.
在那次課堂中,.
我們看到了基督教的身份,.
在那次課堂中,.
我們看到了基督教的身份,.
在那次課堂中,.
我們看到了基督教的身份,.
在那次課堂中,.
我們看到了基督教的身份,.
在那次課堂中,.
我們看到了基督教的身份,.
在那次課堂中,.
我們看到了基督教的身份,.
在那次課堂中,.
我們看到了基督教的身份,.
在那次課堂中,.
我們看到了基督教的身份,.
在那次課堂中,.
我們看到了基督教的身份,.
在那次課堂中,.
我們看到了基督教的身份,.
在那次課堂中,.
我們看到了基督教的身份,.
在那次課堂中,.
我們看到了基督教的身份,.

$^{881}$在那次課堂中,.
我們看到了基督教的身份,.
在那次課堂中,.
我們看到了基督教的身份,.
在那次課堂中,.
我們看到了基督教的身份,.
在那次課堂中,.
我們看到了基督教的身份,.
在那次課堂中,.
我們看到了基督教的身份,.
在那次課堂中,.
我們看到了基督教的身份,.
在那次課堂中,.
我們看到了基督教的身份,.
在那次課堂中,.
我們看到了基督教的身份,.
在那次課堂中,.
我們看到了基督教的身份,.
在那次課堂中,.
我們看到了基督教的身份,.
在那次課堂中,.
我們看到了基督教的身份,.
在那次課堂中,.
我們看到了基督教的身份,.
在那次課堂中,.
我們看到了基督教的身份,.
在那次課堂中,.
我們看到了基督教的身份,.
在那次課堂中,.
我們看到了基督教的身份,.
在那次課堂中,.
我們看到了基督教的身份,.
在那次課堂中,.
我們看到了基督教的身份,.
在那次課堂中,.
我們看到了基督教的身份,.
在那次課堂中,.
我們看到了基督教的身份,.
在那次課堂中,.
我們看到了基督教的身份,.

$^{921}$在那次課堂中,.
我們看到了基督教的身份,.
在那次課堂中,.
我們看到了基督教的身份,.
在那次課堂中,.
我們看到了基督教的身份,.
在那次課堂中,.
我們看到了基督教的身份,.
在那次課堂中,.
我們看到了基督教的身份,.
在那次課堂中,.
我們看到了基督教的身份,.
在那次課堂中,.
我們看到了基督教的身份,.
在那次課堂中,.
我們看到了基督教的身份,.
在那次課堂中,.
我們看到了基督教的身份,.
在那次課堂中,.
我們看到了基督教的身份,.
在那次課堂中,.
我們看到了基督教的身份,.
在那次課堂中,.
我們看到了基督教的身份,.
在那次課堂中,.
我們看到了基督教的身份,.
在那次課堂中,.
我們看到了基督教的身份,.
在那次課堂中,.
我們看到了基督教的身份,.
在那次課堂中,.
我們看到了基督教的身份,.
在那次課堂中,.
我們看到了基督教的身份,.
在那次課堂中,.
我們看到了基督教的身份,.
在那次課堂中,.
我們看到了基督教的身份,.
在那次課堂中,.
我們看到了基督教的身份,.

$^{961}$在那次課堂中,.
我們看到了基督教的身份,.
在那次課堂中,.
我們看到了基督教的身份,.
在那次課堂中,.
我們看到了基督教的身份,.
在那次課堂中,.
我們看到了基督教的身份,.
在那次課堂中,.
我們看到了基督教的身份,.
在那次課堂中,.
我們看到了基督教的身份,.
在那次課堂中,.
我們看到了基督教的身份,.
在那次課堂中,.
我們看到了基督教的身份,.
在那次課堂中,.
我們看到了基督教的身份,.
在那次課堂中,.
我們看到了基督教的身份,.
在那次課堂中,.
我們看到了基督教的身份,.
在那次課堂中,.
我們看到了基督教的身份,.
在那次課堂中,.
我們看到了基督教的身份,.
在那次課堂中,.
我們看到了基督教的身份,.
在那次課堂中,.
我們看到了基督教的身份,.
在那次課堂中,.
我們看到了基督教的身份,.
在那次課堂中,.
我們看到了基督教的身份,.
在那次課堂中,.
我們看到了基督教的身份,.
在那次課堂中,.
我們看到了基督教的身份,.
在那次課堂中,.
我們看到了基督教的身份,.

$^{1001}$在那次課堂中,.
我們看到了基督教的身份,.
在那次課堂中,.
我們看到了基督教的身份,.
在那次課堂中,.
我們看到了基督教的身份,.
在那次課堂中,.
我們看到了基督教的身份,.
在那次課堂中,.
我們看到了基督教的身份,.
在那次課堂中,.
我們看到了基督教的身份,.
在那次課堂中,.
我們看到了基督教的身份,.
在那次課堂中,.
我們看到了基督教的身份,.
在那次課堂中,.
我們看到了基督教的身份,.
在那次課堂中,.
我們看到了基督教的身份,.
在那次課堂中,.
我們看到了基督教的身份,.
在那次課堂中,.
我們看到了基督教的身份,.
在那次課堂中,.
我們看到了基督教的身份,.
在那次課堂中,.
我們看到了基督教的身份,.
在那次課堂中,.
我們看到了基督教的身份,.
在那次課堂中,.
我們看到了基督教的身份,.
在那次課堂中,.
我們看到了基督教的身份,.
在那次課堂中,.
我們看到了基督教的身份,.
在那次課堂中,.
我們看到了基督教的身份,.
在那次課堂中,.
我們看到了基督教的身份,.

$^{1041}$在那次課堂中,.
我們看到了基督教的身份,.
在那次課堂中,.
我們看到了基督教的身份,.
在那次課堂中,.
我們看到了基督教的身份,.
在那次課堂中,.
我們看到了基督教的身份,.
在那次課堂中,.
我們看到了基督教的身份,.
在那次課堂中,.
我們看到了基督教的身份,.
在那次課堂中,.
我們看到了基督教的身份,.
在那次課堂中,.
我們看到了基督教的身份,.
在那次課堂中,.
我們看到了基督教的身份,.
在那次課堂中,.
我們看到了基督教的身份,.
在那次課堂中,.
我們看到了基督教的身份,.
在那次課堂中,.
我們看到了基督教的身份,.
在那次課堂中,.
我們看到了基督教的身份,.
在那次課堂中,.
我們看到了基督教的身份,.
在那次課堂中,.
我們看到了基督教的身份,.
在那次課堂中,.
我們看到了基督教的身份,.
在那次課堂中,.
我們看到了基督教的身份,.
在那次課堂中,.
我們看到了基督教的身份,.
在那次課堂中,.
我們看到了基督教的身份,.
在那次課堂中,.
我們看到了基督教的身份,.

$^{1081}$在那次課堂中,.
我們看到了基督教的身份,.
在那次課堂中,.
我們看到了基督教的身份,.
在那次課堂中,.
我們看到了基督教的身份,.
在那次課堂中,.
我們看到了基督教的身份,.
在那次課堂中,.
我們看到了基督教的身份,.
在那次課堂中,.
我們看到了基督教的身份,.
在那次課堂中,.
我們看到了基督教的身份,.
在那次課堂中,.
我們看到了基督教的身份,.
在那次課堂中,.
我們看到了基督教的身份,.
在那次課堂中,.
我們看到了基督教的身份,.
在那次課堂中,.
我們看到了基督教的身份,.
在那次課堂中,.
我們看到了基督教的身份,.
在那次課堂中,.
我們看到了基督教的身份,.
在那次課堂中,.
我們看到了基督教的身份,.
在那次課堂中,.
我們看到了基督教的身份,.
在那次課堂中,.
我們看到了基督教的身份,.
在那次課堂中,.
我們看到了基督教的身份,.
在那次課堂中,.
我們看到了基督教的身份,.
在那次課堂中,.
我們看到了基督教的身份,.
在那次課堂中,.
我們看到了基督教的身份,.

$^{1121}$在那次課堂中,.
我們看到了基督教的身份,.
在那次課堂中,.
我們看到了基督教的身份,.
在那次課堂中,.
我們看到了基督教的身份,.
在那次課堂中,.
我們看到了基督教的身份,.
在那次課堂中,.
我們看到了基督教的身份,.
在那次課堂中,.
我們看到了基督教的身份,.
在那次課堂中,.
我們看到了基督教的身份,.
在那次課堂中,.
我們看到了基督教的身份,.
在那次課堂中,.
我們看到了基督教的身份,.
在那次課堂中,.
我們看到了基督教的身份,.
在那次課堂中,.
我們看到了基督教的身份,.
在那次課堂中,.
我們看到了基督教的身份,.
在那次課堂中,.
我們看到了基督教的身份,.
在那次課堂中,.
我們看到了基督教的身份,.
在那次課堂中,.
我們看到了基督教的身份,.
在那次課堂中,.
我們看到了基督教的身份,.
在那次課堂中,.
我們看到了基督教的身份,.
在那次課堂中,.
我們看到了基督教的身份,.
在那次課堂中,.
我們看到了基督教的身份,.
在那次課堂中,.
我們看到了基督教的身份,.

$^{1161}$在那次課堂中,.
我們看到了基督教的身份,.
在那次課堂中,.
我們看到了基督教的身份,.
在那次課堂中,.
我們看到了基督教的身份,.
在那次課堂中,.
我們看到了基督教的身份,.
在那次課堂中,.
我們看到了基督教的身份,.
在那次課堂中,.
我們看到了基督教的身份,.
在那次課堂中,.
我們看到了基督教的身份,.
在那次課堂中,.
我們看到了基督教的身份,.
在那次課堂中,.
我們看到了基督教的身份,.
在那次課堂中,.
我們看到了基督教的身份,.
在那次課堂中,.
我們看到了基督教的身份,.
在那次課堂中,.
我們看到了基督教的身份,.
在那次課堂中,.
我們看到了基督教的身份,.
在那次課堂中,.
我們看到了基督教的身份,.
在那次課堂中,.
我們看到了基督教的身份,.
在那次課堂中,.
我們看到了基督教的身份,.
在那次課堂中,.
我們看到了基督教的身份,.
在那次課堂中,.
我們看到了基督教的身份,.
在那次課堂中,.
我們看到了基督教的身份,.
在那次課堂中,.
我們看到了基督教的身份,.

$^{1201}$在那次課堂中,.
我們看到了基督教的身份,.
在那次課堂中,.
我們看到了基督教的身份,.
在那次課堂中,.
我們看到了基督教的身份,.
在那次課堂中,.
我們看到了基督教的身份,.
在那次課堂中,.
我們看到了基督教的身份,.
在那次課堂中,.
我們看到了基督教的身份,.
在那次課堂中,.
我們看到了基督教的身份,.
在那次課堂中,.
我們看到了基督教的身份,.
在那次課堂中,.
我們看到了基督教的身份,.
在那次課堂中,.
我們看到了基督教的身份,.
在那次課堂中,.
我們看到了基督教的身份,.
在那次課堂中,.
我們看到了基督教的身份,.
在那次課堂中,.
我們看到了基督教的身份,.
在那次課堂中,.
我們看到了基督教的身份,.
在那次課堂中,.
我們看到了基督教的身份,.
在那次課堂中,.
我們看到了基督教的身份,.
在那次課堂中,.
我們看到了基督教的身份,.
在那次課堂中,.
我們看到了基督教的身份,.
在那次課堂中,.
我們看到了基督教的身份,.
在那次課堂中,.
我們看到了基督教的身份,.

$^{1241}$在那次課堂中,.
我們看到了基督教的身份,.
在那次課堂中,.
我們看到了基督教的身份,.
在那次課堂中,.
我們看到了基督教的身份,.
在那次課堂中,.
我們看到了基督教的身份,.
在那次課堂中,.
我們看到了基督教的身份,.
在那次課堂中,.
我們看到了基督教的身份,.
在那次課堂中,.
我們看到了基督教的身份,.
在那次課堂中,.
我們看到了基督教的身份,.
在那次課堂中,.
我們看到了基督教的身份,.
在那次課堂中,.
我們看到了基督教的身份,.
在那次課堂中,.
我們看到了基督教的身份,.
在那次課堂中,.
我們看到了基督教的身份,.
在那次課堂中,.
我們看到了基督教的身份,.
在那次課堂中,.
我們看到了基督教的身份,.
在那次課堂中,.
我們看到了基督教的身份,.
在那次課堂中,.
我們看到了基督教的身份,.
在那次課堂中,.
我們看到了基督教的身份,.
在那次課堂中,.
我們看到了基督教的身份,.
在那次課堂中,.
我們看到了基督教的身份,.
在那次課堂中,.
我們看到了基督教的身份,.

$^{1281}$在那次課堂中,.
我們看到了基督教的身份,.
在那次課堂中,.
我們看到了基督教的身份,.
在那次課堂中,.
我們看到了基督教的身份,.
在那次課堂中,.
我們看到了基督教的身份,.
在那次課堂中,.
我們看到了基督教的身份,.
在那次課堂中,.
我們看到了基督教的身份,.
在那次課堂中,.
我們看到了基督教的身份,.
在那次課堂中,.
我們看到了基督教的身份,.
在那次課堂中,.
我們看到了基督教的身份,.
在那次課堂中,.
我們看到了基督教的身份,.
在那次課堂中,.
我們看到了基督教的身份,.
在那次課堂中,.
我們看到了基督教的身份,.
在那次課堂中,.
我們看到了基督教的身份,.
在那次課堂中,.
我們看到了基督教的身份,.
在那次課堂中,.
我們看到了基督教的身份,.
在那次課堂中,.
我們看到了基督教的身份,.
在那次課堂中,.
我們看到了基督教的身份,.
在那次課堂中,.
我們看到了基督教的身份,.
在那次課堂中,.
我們看到了基督教的身份,.
在那次課堂中,.
我們看到了基督教的身份,.

$^{1321}$在那次課堂中,.
我們看到了基督教的身份,.
在那次課堂中,.
我們看到了基督教的身份,.
在那次課堂中,.
我們看到了基督教的身份,.
在那次課堂中,.
我們看到了基督教的身份,.
在那次課堂中,.
我們看到了基督教的身份,.
在那次課堂中,.
我們看到了基督教的身份,.
在那次課堂中,.
我們看到了基督教的身份,.
在那次課堂中,.
我們看到了基督教的身份,.
在那次課堂中,.
我們看到了基督教的身份,.
在那次課堂中,.
我們看到了基督教的身份,.
在那次課堂中,.
我們看到了基督教的身份,.
在那次課堂中,.
我們看到了基督教的身份,.
在那次課堂中,.
我們看到了基督教的身份,.
在那次課堂中,.
我們看到了基督教的身份,.
在那次課堂中,.
我們看到了基督教的身份,.
在那次課堂中,.
我們看到了基督教的身份,.
在那次課堂中,.
我們看到了基督教的身份,.
在那次課堂中,.
我們看到了基督教的身份,.
在那次課堂中,.
我們看到了基督教的身份,.
在那次課堂中,.
我們看到了基督教的身份,.

$^{1361}$在那次課堂中,.
我們看到了基督教的身份,.
在那次課堂中,.
我們看到了基督教的身份,.
在那次課堂中,.
我們看到了基督教的身份,.
在那次課堂中,.
我們看到了基督教的身份,.
在那次課堂中,.
我們看到了基督教的身份,.
在那次課堂中,.
我們看到了基督教的身份,.
在那次課堂中,.
我們看到了基督教的身份,.
在那次課堂中,.
我們看到了基督教的身份,.
在那次課堂中,.
我們看到了基督教的身份,.
在那次課堂中,.
我們看到了基督教的身份,.
在那次課堂中,.
我們看到了基督教的身份,.
在那次課堂中,.
我們看到了基督教的身份,.
在那次課堂中,.
我們看到了基督教的身份,.
在那次課堂中,.
我們看到了基督教的身份,.
在那次課堂中,.
我們看到了基督教的身份,.
在那次課堂中,.
我們看到了基督教的身份,.
在那次課堂中,.
我們看到了基督教的身份,.
在那次課堂中,.
我們看到了基督教的身份,.
在那次課堂中,.
我們看到了基督教的身份,.
在那次課堂中,.
我們看到了基督教的身份,.

$^{1401}$在那次課堂中,.
我們看到了基督教的身份,.
在那次課堂中,.
我們看到了基督教的身份,.
在那次課堂中,.
我們看到了基督教的身份,.
在那次課堂中,.
我們看到了基督教的身份,.
在那次課堂中,.
我們看到了基督教的身份,.
在那次課堂中,.
我們看到了基督教的身份,.
在那次課堂中,.
我們看到了基督教的身份,.
在那次課堂中,.
我們看到了基督教的身份,.
在那次課堂中,.
我們看到了基督教的身份,.
在那次課堂中,.
我們看到了基督教的身份,.
在那次課堂中,.
我們看到了基督教的身份,.
在那次課堂中,.
我們看到了基督教的身份,.
在那次課堂中,.
我們看到了基督教的身份,.
在那次課堂中,.
我們看到了基督教的身份,.
在那次課堂中,.
我們看到了基督教的身份,.
在那次課堂中,.
我們看到了基督教的身份,.
在那次課堂中,.
我們看到了基督教的身份,.
在那次課堂中,.
我們看到了基督教的身份,.
在那次課堂中,.
我們看到了基督教的身份,.
在那次課堂中,.
我們看到了基督教的身份,.

$^{1441}$在那次課堂中,.
我們看到了基督教的身份,.
在那次課堂中,.
我們看到了基督教的身份,.
在那次課堂中,.
我們看到了基督教的身份,.
在那次課堂中,.
我們看到了基督教的身份,.
在那次課堂中,.
我們看到了基督教的身份,.
在那次課堂中,.
我們看到了基督教的身份,.
在那次課堂中,.
我們看到了基督教的身份,.
在那次課堂中,.
我們看到了基督教的身份,.
在那次課堂中,.
我們看到了基督教的身份,.
在那次課堂中,.
我們看到了基督教的身份,.
在那次課堂中,.
我們看到了基督教的身份,.
在那次課堂中,.
我們看到了基督教的身份,.
在那次課堂中,.
我們看到了基督教的身份,.
在那次課堂中,.
我們看到了基督教的身份,.
在那次課堂中,.
我們看到了基督教的身份,.
在那次課堂中,.
我們看到了基督教的身份,.
在那次課堂中,.
我們看到了基督教的身份,.
在那次課堂中,.
我們看到了基督教的身份,.
在那次課堂中,.
我們看到了基督教的身份,.
在那次課堂中,.
我們看到了基督教的身份,.

$^{1481}$在那次課堂中,.
我們看到了基督教的身份,.
在那次課堂中,.
我們看到了基督教的身份,.
在那次課堂中,.
我們看到了基督教的身份,.
在那次課堂中,.
我們看到了基督教的身份,.
在那次課堂中,.
我們看到了基督教的身份,.
在那次課堂中,.
我們看到了基督教的身份,.
在那次課堂中,.
我們看到了基督教的身份,.
在那次課堂中,.
我們看到了基督教的身份,.
在那次課堂中,.
我們看到了基督教的身份,.
在那次課堂中,.
我們看到了基督教的身份,.
在那次課堂中,.
我們看到了基督教的身份,.
在那次課堂中,.
我們看到了基督教的身份,.
在那次課堂中,.
我們看到了基督教的身份,.
在那次課堂中,.
我們看到了基督教的身份,.
在那次課堂中,.
我們看到了基督教的身份,.
在那次課堂中,.
我們看到了基督教的身份,.
在那次課堂中,.
我們看到了基督教的身份,.
在那次課堂中,.
我們看到了基督教的身份,.
在那次課堂中,.
我們看到了基督教的身份,.
在那次課堂中,.
我們看到了基督教的身份,.

$^{1521}$在那次課堂中,.
我們看到了基督教的身份,.
在那次課堂中,.
我們看到了基督教的身份,.
在那次課堂中,.
我們看到了基督教的身份,.
在那次課堂中,.
我們看到了基督教的身份,.
在那次課堂中,.
我們看到了基督教的身份,.
在那次課堂中,.
我們看到了基督教的身份,.
在那次課堂中,.
我們看到了基督教的身份,.
在那次課堂中,.
我們看到了基督教的身份,.
在那次課堂中,.
我們看到了基督教的身份,.
在那次課堂中,.
我們看到了基督教的身份,.
在那次課堂中,.
我們看到了基督教的身份,.
在那次課堂中,.
我們看到了基督教的身份,.
在那次課堂中,.
我們看到了基督教的身份,.
在那次課堂中,.
我們看到了基督教的身份,.
在那次課堂中,.
我們看到了基督教的身份,.
在那次課堂中,.
我們看到了基督教的身份,.
在那次課堂中,.
我們看到了基督教的身份,.
在那次課堂中,.
我們看到了基督教的身份,.
在那次課堂中,.
我們看到了基督教的身份,.
在那次課堂中,.
我們看到了基督教的身份,.

$^{1561}$在那次課堂中,.
我們看到了基督教的身份,.
在那次課堂中,.
我們看到了基督教的身份,.
在那次課堂中,.
我們看到了基督教的身份,.
在那次課堂中,.
我們看到了基督教的身份,.
在那次課堂中,.
我們看到了基督教的身份,.
在那次課堂中,.
我們看到了基督教的身份,.
在那次課堂中,.
我們看到了基督教的身份,.
在那次課堂中,.
我們看到了基督教的身份,.
在那次課堂中,.
我們看到了基督教的身份,.
在那次課堂中,.
我們看到了基督教的身份,.
在那次課堂中,.
我們看到了基督教的身份,.
在那次課堂中,.
我們看到了基督教的身份,.
在那次課堂中,.
我們看到了基督教的身份,.
在那次課堂中,.
我們看到了基督教的身份,.
在那次課堂中,.
我們看到了基督教的身份,.
在那次課堂中,.
我們看到了基督教的身份,.
在那次課堂中,.
我們看到了基督教的身份,.
在那次課堂中,.
我們看到了基督教的身份,.
在那次課堂中,.
我們看到了基督教的身份,.
在那次課堂中,.
我們看到了基督教的身份,.

$^{1601}$在那次課堂中,.
我們看到了基督教的身份,.
在那次課堂中,.
我們看到了基督教的身份,.
在那次課堂中,.
我們看到了基督教的身份,.
在那次課堂中,.
我們看到了基督教的身份,.
在那次課堂中,.
我們看到了基督教的身份,.
在那次課堂中,.
我們看到了基督教的身份,.
在那次課堂中,.
我們看到了基督教的身份,.
在那次課堂中,.
我們看到了基督教的身份,.
在那次課堂中,.
我們看到了基督教的身份,.
在那次課堂中,.
我們看到了基督教的身份,.
在那次課堂中,.
我們看到了基督教的身份,.
在那次課堂中,.
我們看到了基督教的身份,.
在那次課堂中,.
我們看到了基督教的身份,.
在那次課堂中,.
我們看到了基督教的身份,.
在那次課堂中,.
我們看到了基督教的身份,.
在那次課堂中,.
我們看到了基督教的身份,.
在那次課堂中,.
我們看到了基督教的身份,.
在那次課堂中,.
我們看到了基督教的身份,.
在那次課堂中,.
我們看到了基督教的身份,.
在那次課堂中,.
我們看到了基督教的身份,.

$^{1641}$在那次課堂中,.
我們看到了基督教的身份,.
在那次課堂中,.
我們看到了基督教的身份,.
在那次課堂中,.
我們看到了基督教的身份,.
在那次課堂中,.
我們看到了基督教的身份,.
在那次課堂中,.
我們看到了基督教的身份,.
在那次課堂中,.
我們看到了基督教的身份,.
在那次課堂中,.
我們看到了基督教的身份,.
在那次課堂中,.
我們看到了基督教的身份,.
在那次課堂中,.
我們看到了基督教的身份,.
在那次課堂中,.
我們看到了基督教的身份,.
在那次課堂中,.
我們看到了基督教的身份,.
在那次課堂中,.
我們看到了基督教的身份,.
在那次課堂中,.
我們看到了基督教的身份,.
在那次課堂中,.
我們看到了基督教的身份,.
在那次課堂中,.
我們看到了基督教的身份,.
在那次課堂中,.
我們看到了基督教的身份,.
在那次課堂中,.
我們看到了基督教的身份,.
在那次課堂中,.
我們看到了基督教的身份,.
在那次課堂中,.
我們看到了基督教的身份,.
在那次課堂中,.
我們看到了基督教的身份,.

$^{1681}$在那次課堂中,.
我們看到了基督教的身份,.
在那次課堂中,.
我們看到了基督教的身份,.
在那次課堂中,.
我們看到了基督教的身份,.
在那次課堂中,.
我們看到了基督教的身份,.
在那次課堂中,.
我們看到了基督教的身份,.
在那次課堂中,.
我們看到了基督教的身份,.
在那次課堂中,.
我們看到了基督教的身份,.
在那次課堂中,.
我們看到了基督教的身份,.
在那次課堂中,.
我們看到了基督教的身份,.
在那次課堂中,.
我們看到了基督教的身份,.
在那次課堂中,.
我們看到了基督教的身份,.
在那次課堂中,.
我們看到了基督教的身份,.
在那次課堂中,.
我們看到了基督教的身份,.
在那次課堂中,.
我們看到了基督教的身份,.
在那次課堂中,.
我們看到了基督教的身份,.
在那次課堂中,.
我們看到了基督教的身份,.
在那次課堂中,.
我們看到了基督教的身份,.
在那次課堂中,.
我們看到了基督教的身份,.
在那次課堂中,.
我們看到了基督教的身份,.
在那次課堂中,.
我們看到了基督教的身份,.

$^{1721}$在那次課堂中,.
我們看到了基督教的身份,.
在那次課堂中,.
我們看到了基督教的身份,.
在那次課堂中,.
我們看到了基督教的身份,.
在那次課堂中,.
我們看到了基督教的身份,.
在那次課堂中,.
我們看到了基督教的身份,.
在那次課堂中,.
我們看到了基督教的身份,.
在那次課堂中,.
我們看到了基督教的身份,.
在那次課堂中,.
我們看到了基督教的身份,.
在那次課堂中,.
我們看到了基督教的身份,.
在那次課堂中,.
我們看到了基督教的身份,.
在那次課堂中,.
我們看到了基督教的身份,.
在那次課堂中,.
我們看到了基督教的身份,.
在那次課堂中,.
我們看到了基督教的身份,.
在那次課堂中,.
我們看到了基督教的身份,.
在那次課堂中,.
我們看到了基督教的身份,.
在那次課堂中,.
我們看到了基督教的身份,.
在那次課堂中,.
我們看到了基督教的身份,.
在那次課堂中,.
我們看到了基督教的身份,.
在那次課堂中,.
我們看到了基督教的身份,.
在那次課堂中,.
我們看到了基督教的身份,.

$^{1761}$在那次課堂中,.
我們看到了基督教的身份,.
在那次課堂中,.
我們看到了基督教的身份,.
在那次課堂中,.
我們看到了基督教的身份,.
在那次課堂中,.
我們看到了基督教的身份,.
在那次課堂中,.
我們看到了基督教的身份,.
在那次課堂中,.
我們看到了基督教的身份,.
在那次課堂中,.
我們看到了基督教的身份,.
在那次課堂中,.
我們看到了基督教的身份,.
在那次課堂中,.
我們看到了基督教的身份,.
在那次課堂中,.
我們看到了基督教的身份,.
在那次課堂中,.
我們看到了基督教的身份,.
在那次課堂中,.
我們看到了基督教的身份,.
在那次課堂中,.
我們看到了基督教的身份,.
在那次課堂中,.
我們看到了基督教的身份,.
在那次課堂中,.
我們看到了基督教的身份,.
在那次課堂中,.
\newpage



\section{}
\label{sec:XZUhcyDQxTc}
\textbf{The Old Testament and Christian Hope: Where will it all end? (4) — Dr Stephen Lee}
\newline
\newline
連結: \href{https://youtube.com/watch?v=XZUhcyDQxTc}{\texttt{ https://youtube.com/watch?v=XZUhcyDQxTc}} ~~~~ 語音日期: 2023-03-08 
\newline
\newline
\hyperref[sec:cUiQvS6fE24]{\small{< < < PREV SERMON < < <}}
~
\hyperref[sec:index]{\small{[返主目錄]}}
~
\hyperref[sec:c9xzZQkJF0c]{\small{> > > NEXT SERMON > > >}}
\newline
\newline
$^{1}$(音乐).
让我开始,用一句热烈而溫暖的口谕,.
欢迎我們的好朋友,.
護士醫生,.
OT Wright,.
不是NT Wright,.
他來自Langham Partnership International,.
是1969年由John Stort 创立的Langham Trust..
我仍然深刻記录.
是第一次Chris 來到香港.
作為Langham 的國际領导人,.
而不是他第一次來到香港,.
他告诉我的時候是80年代..
那是19年前的2003年,.
我有榮譽和权利.
欢迎他來到香港.
作為Langham 的學者和第一次.
Langham Foundation 的國际領导人..
在一世記的转折中,.
Chris 成功地帶領John Stort.
向Langham 推进三個重点目标,.
那就是學业,公开發行和宣传,.
為大多数世界宣传..
当然,這是一個很高的期望,.
让任何人都能在John Stort 身上感到滿足..
然而,过去兩十年,.
這让我們都能证明.
Chris 不僅是一個聪明和榮譽的領导人.
在Langham 宗教家庭的格式中,.
但也有著一种宗教和任何宗教的影響.
在全球的路障运动中,.
由一個持续和集中的工作.
在研究和写作.
英语语言教會全球和外界的工作中..
我认為您现在应该同意.
John Stort 确实找到了.
一個無法实现任何任务的选举选手..
我今天晚上的出席是.
與我們的优秀评论员.
在最后一堂課的主题中.

$^{41}$古代宗教和基督教的希望.
我将假設說話是一個相反的单词.
或者是一個单词.
所以這是一個聪明的对話.
與意義的交流观.
我甚至可以承认.
古代宗教的多词一句的宗教原理.
因為我們正在探索古代宗教的基督教宗教.
所以,尽管我对這些.
我必须在這一系列的四個连续課課中.
說出這些词中的.
严重和严重的同意.
我仍然尽力地挑选蛋.
正如中國的說法.
我提出這些主题.
是通过三個连续的评论.
或是關于方法的问题.
在展示了新教會如何建立在古代的基础上.
以告诉我們我們是誰.
為什么我們在這里.
以及我們如何生活.
在他今晚的第四個課課的开始.
他提出了最終的问题.
它将到哪里來.
他提出了.
读书是一個物理的故事.
它是一個大故事.
從基督教到記念.
從创造到新创造.
他还利用了詩中的詩句.
作為一個大戏剧.
扮演了大戏剧的连续.
在七個行為中.
创造,反抗,保证,基督,任务,判決,新创造.
或是在紙巾上.
記住.
在這本詩中的詩句.
作為一個物理的故事.
或是七個行為的戏剧.
我首先想提出的就是.

$^{81}$在他的《基督教和宗教》.
一本記念书.
《希伯语詩》的宗教詩.
在2003年出版.
由阿波羅斯和英特瓦羅斯的新聞社.
和《希伯语詩》的新聞社.
與《希伯语詩》的新聞社.
最后,第三部分.
认為這部分的故事.
\newpage



\section{}
\label{sec:c9xzZQkJF0c}
\textbf{The Old Testament and Christian Hope: Where will it all end? (4) — Q & A Session}
\newline
\newline
連結: \href{https://youtube.com/watch?v=c9xzZQkJF0c}{\texttt{ https://youtube.com/watch?v=c9xzZQkJF0c}} ~~~~ 語音日期: 2023-03-09 
\newline
\newline
\hyperref[sec:XZUhcyDQxTc]{\small{< < < PREV SERMON < < <}}
~
\hyperref[sec:index]{\small{[返主目錄]}}
~
\hyperref[sec:0xAv1_RZmUM]{\small{> > > NEXT SERMON > > >}}
\newline
\newline
$^{1}$(音樂).
我坐在兩隻巨人的後面.
非常令人感到害怕.
你不看起來很害怕.
我只是在假裝.
只是在假裝.
所以,李博士已經非常完美地總結了李博士的講座.
我想簡單地講一下李博士的講話.
我認為這三個肋骨.
我只是在假裝.
我沒有想到他會這樣說.
所以我可能會錯誤.
但是李博士選的這三個肋骨.
有些許跟劇本有關.
第一是關於神的形狀.
第二是關於影響力的神的方向.
不是說到最後.
而是回到中心,回到神的圖案.
最後可能是關於神的權力.
李博士,關於這三個肋骨.
你覺得他們好吃嗎?.
我今天午餐的時候.
我正在想著.
我希望中國人會知道.
有別的方法去切雞肉.
而不是每一塊雞肉都要切成一塊.
我想說的是.
我嘗試說的.
李博士說的話.
在很多方面都很合理.
因為我嘗試說的.
這整個神話的理論流程.
從一個開始到一個結局.
當然不是否認.
在這個理論的結構.
有很多的轉變.
有很多的事情發生.
這些事實上.
在某些時候.
是一個故事的懸殊.

$^{41}$例如,耶穌進入了死亡,然後再回來.
還有,在神話和詩中.
有很多的評論.
這些都是好的.
但我認為.
雖然第一部分.
斯蒂芬的觀點是.
在古代的神話.
這就是它的看法.
但當你看到.
這是一個故事.
它在某個地方.
它其實是帶領著耶穌.
到結局的時候.
我有時候對學生說.
我們不應該去想.
這個神話的故事.
即使我把它放在.
一個很長的直線.
它只是一個大渠道.
兩個直線在前方.
在中間流動的水.
它就像亞馬遜河一樣.
亞馬遜河的渠道.
那是一個非常複雜的水系統.
水流向後.
向後.
渠道,湖,等等.
但最後一天.
水流都會向同一方向流動.
那就是.
那些水.
在亞馬遜渠道.
會在亞洲海洋中.
這就是命運.
我想說的是.
在古代.
就像亞馬遜河流.
水流都會向著耶穌.
和神的使命流動.

$^{81}$這是我首先想說的.
然後.
我第二點.
我認為和斯蒂夫所說的相符.
當然.
當聖經寫家.
神父.
說者.
聖經寫家.
寫到聖經.
當他們想要看未來.
他們總是回到過去.
他們總是說.
我們有一個神.
已經做了一些很奇妙的事.
但祂還是一樣的神.
所以祂要再做一次.
那最明顯的地方.
在聖經最後一章.
你無法理解聖經.
不說祂一直.
使用耶穌的語言.
和古代的語言.
來表達.
我們有一個神.
所以.
這就是我前面的講座.
一種講座.
我說了.
神的民族.
是以記憶和希望為主.
的.
沒有記憶.
你不知道自己是誰.
這就是為什麼.
在後成代的時期.
內瑪雅人.
會做很多.
以保存記憶.
這個人是誰.

$^{121}$這是亞維的民族.
是出口的神.
當你有記憶.
你會知道.
你會有希望.
因為這是你擁有的神.
所以.
那種不斷的被轉移.
使用過去.
為了未來的希望.
是基礎.
我認為我會同意.
斯文的說法.
這可以嗎?.
你提到的一點是.
希望不是基於.
我們在遙遠的未來.
但也有部分的希望.
是基於.
我們對神的工作的記憶.
我對嗎?.
是的.
是基於記憶.
但也有部分的.
是在想像未來.
這就是神父們所做的.
他們以神的方式.
創造未來.
未來還沒有.
當聖經96召喚.
所有的創造.
來慈禱.
來慶祝.
因為神正到來.
他們正在描述世界.
實際上.
聖經描述世界.
是一個平等的地方.
正義和正義.
和穩定.

$^{161}$和慶祝.
我們知道世界並非如此.
但.
聖經的預言.
和諸侯的預言.
能夠預見.
未來的世界.
基於他們知道.
神的性質.
和他們想要做的事.
所以這是一種組合.
記憶和想像.
這總是會有.
我可以繼續嗎?.
抱歉.
我必須抱歉.
我把時間給你短了.
不不不.
沒關係.
請說.
我們無法想像.
新的創造會是怎樣.
除了聖經告訴我們.
在聖經65召喚之外.
這並不驚訝.
因為我們還沒有到那裡.
我們無法想像.
我們已經超越了.
我們生活在三個層次.
的宇宙時間宇宙.
和時間.
耶穌.
耶穌的復活體.
已經在那三個層次裡.
因為祂可以觸摸.
祂可以吃.
祂可以烤魚.
但是祂也能夠.
在那裡活動.
消失.

$^{201}$祂已經存在.
如新創造的第一果實.
新創造已經在這裡.
這就是我和斯蒂芬的最後的意見.
所有聖經說的.
已經開始了.
我們已經在那三個層次裡.
生活了.
我的最後的三角形.
新創造.
我再做一次.
我把它再次傳遞到.
聖經.
我們已經在聖經和聖經之間.
生活了.
在神的新世界.
已經在那裡.
雖然我們無法想像.
現在我用我的說法.
我喜歡在聖經8.
Paul在說.
生命的形式.
他說.
全創造正在長大.
是的.
但是它正在長大.
因為孩子的孕育.
換句話說.
新創造已經在.
這個創造的胎兒裡.
生育了.
這是一個非常有趣的說法.
因為這裡有一個原因.
Paul在說.
復活身體.
你能想像.
兩個胎兒.
胎兒在胎兒裡.
一個人對另一個人說.
這是想像.

$^{241}$一個人對另一個人說.
你相信生命的復活嗎.
我們聽到很多.
關於光和氣息的謠言.
你能呼吸嗎.
你相信這些嗎.
然後他就說.
不不不.
這只是一個謠言.
只是為了讓你開心.
這個胎兒就是所有的東西.
沒有什麼比無精靈液更多.
只要享受它.
因為.
吃喝住.
結婚.
因為明天我們會出生.
我們結束了.
OK.
現在.
你看.
胎兒在胎兒裡.
無法想像.
胎兒在胎兒外面的情況.
即使它有所有的能力.
因為胎兒在胎兒裡.
他們有眼睛.
他們有耳朵.
他們有腦子.
他們有能力.
活在一個世界.
他們還沒想像到.
這就是為什麼生育這麼傷心.
OK.
這就是為什麼我們會在胎兒裡.
喊哭.
因為我們突然.
在光和氣息.
和呼吸.
有人在背後打擊你.

$^{281}$讓你哭.
現在.
Paul說.
新的創造.
與生育不同.
它是同一個人嗎.
當然是.
胎兒在胎兒裡.
是同一個人.
會生育.
活在世界裡.
但是生育是一個巨大的.
變化的時刻.
這就是為什麼我們說.
生育的日子.
我們知道你的生育日.
不是你的出生日.
即使你知道.
生育的日子.
即使你開始的時候.
已經是九個月了.
我們現在正在生活在.
新創造的世界裡.
的一種生育.
我們還沒想像.
它會是怎樣的.
但這也不是原因.
否認它會存在.
因為神已經保證了.
這就是我喜歡用的形容詞.
幫助我們思考.
這世界和下一世界之間的持續.
同時也有著不持續的.
這就像是.
生育的胎兒.
與生育的人.
之間的持續.
同時也有著不持續的.
在生育的時刻.
有幫助嗎.

$^{321}$是的.
是一種想像的想像.
謝謝.
這是一個很有幫助的例子.
所以.
李博士.
我非常高興.
繼續你對於.
生育的說法.
因為.
我只不喜歡的一點.
就是在你的題目中.
你說.
它們會在哪裡.
結束.
是的.
所以對於生育.
這不是結束.
我喜歡CS Lewis的書寫.
我第一次讀.
他的《善於基督教》.
是在我大學一年級.
我不懂這個詞語.
但是我後來.
繼續讀《南亞》的《論述》.
最後一段.
最後一場戰爭.
他描述了死亡的那一刻.
是夏季的第一天.
然後我才明白.
這不是學校的年底.
這是夏季的一年初.
所以.
它們在哪裡開始.
所以謝謝你對於.
這個生育的說法.
我覺得這對兩隻胎兒的交流.
對話很有幫助.
我以為你說的是想像.
我們要開始.

$^{361}$而不是結束.
所以這確實是.
我認為.
生育的關鍵.
所以我不是說.
這個世界並不重要.
所以記憶.
或整個聖經.
是關於這生命.
並不是關於未來的生命.
但我們仍然在等待.
這個生育的開始.
我喜歡.
謝謝.
我們可以在這裡拍一部.
荷里活電影.
但是.
你剛才說了.
它什麼時候會開始.
讓我開始懷疑.
我會在哪裡.
所以這是什麼.
我認為.
這就提出了.
我們現在為了什麼而生存.
你看.
這就是我們的基督教學.
其實非常重要.
因為.
如果我們看到的是新的創造.
在這裡.
如同聖經所說.
不會有欺騙,謀殺,謊言,邪惡,腐敗.
如果沒有了.
就不會有了.
換句話說.
我們現在要活在.
為了這生命而準備的那一刻.
這不僅是在道德上.
我必須要小心.

$^{401}$如何活下去.
因為我將會在天堂.
或新的創造.
這也意味著.
這就是我.
我沒有給足夠時間.
對於我所說的.
在《聖經》21:1.
約翰說.
他看到了國王們.
帶著他們的光榮.
他們的富貴.
他們的榮耀.
進入了神的城市.
這意味著什麼.
我認為這並不意味著.
這只是會發生在.
國王們和首相們.
他們會活著.
當耶穌回來了.
他們將會前來.
前往聖堂.
在我們之間.
我認為.
它正在說.
什麼是國王們和國家的.
光榮和美麗.
這是人類的.
人類的.
人類的文化成就.
我們人類.
能夠創造.
巨大的.
光榮和美麗.
都被罪和自私.
和自豪和憂愁.
和無道.
和慾望.
但人類的成就.
在神的形象中.

$^{441}$是奇妙的.
我認為這段話.
告訴我們.
新的創造.
並不是一個.
空白的座席.
在那裡.
神看著人類.
所做的一切.
在過去的十萬年裡.
祂說.
把所有的東西.
放進垃圾桶裡.
那是煙火的一切.
讓我們開始新的創造.
我認為.
這會發生的是.
神會把.
人類的文化的光榮.
和所有的邪惡.
都吸進去.
然後被毀滅.
清洗和滅絕.
在祂的判決中.
這就是2彼得.
三語.
它會被燒掉.
但燒掉的東西.
並不是一個.
宇宙的燭台.
裡面沒有東西.
只剩下煙火.
而是一個.
宇宙的煙火.
裡面的東西.
被吸進去.
就是垃圾.
就是罪.
就是邪惡.
而剩下的.

$^{481}$就是人類創造的.
光榮和美麗.
在神的形象中.
現在在新創造中.
正在祝福神.
這意味著什麼呢.
意味著你的工作.
每天的工作.
對於人類的.
會有所貢獻.
在新創造中.
我在這生的工作中.
會有所貢獻.
在神的新創造中.
這給予了價值.
給予了意義和意義.
對於普通的每天工作.
這本書寫了.
由達爾·科斯頓寫的.
叫做.
"天地之好".
我認為這是它的名字.
我們的每天工作.
在這生中.
會有所貢獻.
在神將帶給我們的.
新創造中.
不要問我為什麼.
我不知道.
我明天要寫什麼報告.
或者說.
你需要做什麼.
你需要煮什麼.
我不知道.
但是神有能力.
來賦予和恢復.
我認為這是一個.
很勵志的思想.
不只是.
我們死後都會飛到天上.

$^{521}$這比此更有信心.
比此更有地道性.
比此更有滿足感.
所以我們在準備裝飾.
這可能.
你可以這麼說.
我們肯定是.
新創造的人口.
已經在這創造中生活.
這就是教會.
我們是未來的人.
我們在歷史的右邊.
我們在神的故事中生活.
所以現實世界並不在那裡.
我們有時會談到.
現實世界在那裡.
不是.
現實世界是神的世界.
新創造.
我們在生存中.
我們已經在裝飾中.
我們已經在貢獻.
我們在賦予.
國王和國家的榮耀.
他們將會帶來他們的榮耀.
這對一年級的理論學學生來說是非常有趣的.
但我可以深刻地記住.
我們在課堂上的討論.
這給予了意義,意義.
以及永恆的善良價值.
給予了我們亂亂糾纏的感覺.
這也是為何.
我今天早上想與你們進行一個對話.
我會提到路德的《道德法》的靈性使用.
因為您已經非常深入地探討了.
路德的原始法律.
但路德先出手.
他說:你無法做任何事.
這是一場失敗.
懺悔是指不要相信自己.

$^{561}$這在於基督,是在於聖體.
我覺得這就是希望的共同實現.
我們盡力而為,但卻失敗了.
但這就是聖經.
這是天性.
這是永恆.
不是遠遠的未來.
是天空的派.
或是人類的opium.
不,這是現實.
我們今天正在體驗.
我們知道這是真的.
因為基督已從死亡中復活.
這定義的時刻.
並非在未來.
這正是勒斯林·雅畢根說的.
在復活中.
復活只有意義在於.
它是從何而來的故事.
如果你嘗試去理解.
基督的復活.
如同新教義所描述.
在任何其他世界觀.
如在物質世界觀.
死亡的身體不會復活.
你只能以神話來理解.
基督在復活的心中.
或是在聖經中.
有過一些靈性體驗.
但墓地是空的.
是基督的身體.
從死亡中復活.
這就是聖經所說的.
你見到一個人復活.
我們見到一個人復活.
他們碰到他,吃了他,走過他.
基督的身體復活.
是新創造的開始.
是基督已經開始的.
因此,在基督教中.

$^{601}$Paul可以說.
我們已經與祂同在.
Paul說的,在天堂.
這要靠一大步的想像力.
我還活在這個世界.
我仍然是一個死去的罪惡人.
在這個時代的複雜.
我仍然活在這個古代.
但我已經在這個緊張的世界.
我已經在天堂的市民中.
我已經在我的身體中.
有了基督的生命復活.
就如同Paul所說的.
這也是為何.
不僅是John堅持.
基督的靈魂.
說話變成靈魂.
他也被提起了身體.
但Paul也堅持.
這就是他現在所活的.
我所活的生命.
是基督在我身體中的.
我們是基督在我身體中的.
我可以回到你的問題.
關於法律的問題.
我同意你的說法.
當然,我認為.
Luther和Calvin.
在這裡彼此彼此彼此.
Luther當然認同.
法律的重要功能.
就是它可以曝光我們的罪惡.
但這不是因為法律本身不好.
而是因為我們不好.
這一點我們需要非常小心.
因為Luther的懷疑.
是要把法律曝光我們的罪惡.
然後讓法律本身.
變成不好的.
我們必須要守護法律.

$^{641}$Paul從來沒有這樣說.
我們必須要守護罪惡.
在聖經七章.
他重複地描述法律.
是那些是善,活,聖的.
法律不是問題.
問題是.
當法律對付了罪.
當法律對付了死.
我們就會對它造成死亡.
我重新想到一個說法.
對於這個說法的意思.
Paul其實問了一個問題.
那是如何對生命造成死亡的.
然後我想到蝦子.
蝦子.
抱歉,寶寶在胸前.
但這是另一個畫面.
因為我們都知道蝦子是好東西.
是食物.
是神的創造物.
是充滿蛋白質的.
但對某些人來說.
蝦子會殺死你.
但問題不是蝦子.
問題是你的疾病.
是在你身上.
有些東西在你的身體裡.
會反應蝦子.
然後會殺死你.
我想這就是Paul所說的.
問題不是神的法律.
這是善,活,聖,正,完美的.
但問題是我們是罪惡的.
我們拒絕遵守.
所以法律在靈性的意義上.
它會引導你去聖經.
它會閉上你的嘴巴.
告訴你你需要被救.
但感謝彼得神.

$^{681}$祂會讓我們得到這一點.
然後是Calvin.
他也說了.
是的,這是真的.
但法律也有三種用法.
第三種是.
法律仍然操作為指引.
指明了我們如何活著.
法律告訴我們.
如何愛一個人.
法律給我們設定的標準.
模式,模式.
如何活著.
這就是我主要用法律的方式.
在我的書中.
我問.
如果Paul說.
所有的文章都是有用的.
正如你所說.
為正確訓練和正義.
我沒有必要.
說自己是一位教會人士.
我相信所有的文章.
都是由神啟發的.
是有用的.
除非我準備了.
寫一章關於伊達基.
或一章關於聖經.
或聖經.
這本文章.
是神吸引的.
是有用的.
對訓練和正義的.
我必須找到.
這裡有些東西.
在我對神的聲音.
對神的呼吸中.
在我對正義的訓練.
能夠做到的方式中.
能夠使我能夠.

$^{721}$做到的方式中.
即使我沒有.
做到正確的訓練.
因為我無法.
我沒有牛角尖.
在我的後座.
在我背後.
所以我無法遵守這個法律.
正如它所寫.
但是.
這裡必須有些東西.
神想通過它.
對我說話.
使我能夠.
有神的聲音.
這就是.
法律.
作為一個牆壁.
作為一個鏡子.
讓你展現你的罪行.
讓你展現你的失敗.
你需要聖經.
但也成為了一個導導.
讓你能夠.
活在神的方向.
謝謝.
我想你們也提出了.
各種話題.
都值得追究.
例如.
我的觀點.
是要.
要被形容.
但其實.
我喜歡你的觀點.
關於想像.
是的.
這是很真實的.
記憶能夠幫助我們.
有一個.

$^{761}$有一個交換的想像.
並不是固定在現實之中.
但是.
神是更加強大的.
祂可以.
做任何事.
祂可以做更大的事.
比你能想像的更大.
所以.
這就是我們.
期待的一件事.
但同時我們也知道.
這不是我們能夠得到的結局.
是的.
老天爺也說過.
神說.
不要想起過去的事.
就算你提醒他們.
過去的事.
祂說.
我會做一些.
會讓你們感到驚訝和驚訝的事.
但是過去的事.
仍然是祂所做的模式.
所以.
當耶穌第一次來.
很多人都認出了.
很多人沒有.
即使他們有了經文.
他們仍然驚訝.
神會成為.
納粹的耶穌.
這個探頭的兒子.
洗別人的腳.
觸碰妖精.
和原諒女性.
你知道.
神怎麼會.
變成這樣.
所以很驚訝.

$^{801}$神會這樣做.
所以這就是.
我所想說的.
當耶穌再次回來.
會有很多驚訝的事.
即使我們.
知道經文的知識.
我們仍然會發現.
神會做一些.
讓我們驚訝.
我們從來沒有想到.
謝謝神.
祂比我們的想像更大.
但祂給了我們我們的想像.
就像那本.
《諾貝爾》的書.
《神的想像》.
我們是神的形象.
我們是神的想像.
神要想像宇宙.
然後說出它成為實相.
神給了我們.
這個想像的能力.
我覺得這很明顯.
是人類的.
我們知道.
狗和大象.
不會想像.
其他東西.
很難說.
他們會夢想.
但我不知道.
我們人類的能力.
是站在宇宙和時間之間.
去想像一些不同的東西.
在人類歷史的偉大時刻.
當時有很多的壓迫和痛苦.
例如黑人奴隸.
是由耶穌基督.
和神所傳授的.

$^{841}$寫信的想像.
帶來的.
尤其是在美國.
黑人奴隸的流行.
黑人的靈性.
他們所說的.
想像了約旦的未來.
想像了去到康寧地區.
想像了去到水裡.
有非常多的.
想像自由和自由.
都是從.
從古代的耶穌.
尤其是從耶穌的死亡.
這在歷史上.
有很多次發生.
神的人類.
是記憶的人類.
因此他們是希望的人類.
當你有了過去的記憶.
和希望未來的希望.
你就是一個任務的人.
因為你會問.
我該怎麼生活.
如果那是.
我過去的事.
而那是我未來的目標.
那是一個.
道德和任務的挑戰.
我想我們應該邀請觀眾.
參加我們的討論.
你認為呢?.
有問題的觀眾.
請用麥克風.
來問問題.
或寫下問題.
在紙上.
傳給我.
讓我可以.
翻譯出來.

$^{881}$(麥克風轉換中).
(英文).
謝謝你們.
非常感謝你們.
默哲偉.
你們給我們畫了一張圖片.
是一種想像.
這是現實.
但那是未來的事.
我們被圖片觸動了.
我們真的想要去那裡.
但是當然.
對於基督徒來說.
如果我們只將希望放在未來.
問題就會變成逃脫.
我們想逃脫.
所以這世界並不重要.
所以我想默哲偉的訊息.
是重要的.
我們今天也看到了希望.
即使是在過去.
所以如果我想將你們兩人的形象.
可能是像.
我們的基督信仰.
是基於一個.
已經發生了.
但還沒有實現的事.
信仰是已經發生了.
透過死亡和復活.
但也並非完全滿足.
所以我們已經在那裡.
但還沒有實現.
我想我們需要兩個.
但當我讀《新經》時.
作者們總是希望耶穌的復活.
我認為原因是.
其一是因為他們在受苦.
所以他們無法找到希望.
在地球上.
我認為.

$^{921}$當我們回到整件事的基督性應用.
當我們面對一個真正的受苦的人.
我們應該如何.
如何與他們交談基督的希望.
就像.
現在你看到地球上還有希望.
我們應該如何.
我們應該如何更加強調未來.
你明白我的意思嗎?.
「已經」但還沒有的方面.
已經在耶穌的教導中.
在他的講道中.
他將兩段講道放在一起.
在一篇章中.
我認為是在《路克》中.
或者是在《馬克》中.
他在講述基督的國王是怎樣的.
他說.
一方面.
就像是種了柑橘子.
它會長得很大.
所以事情會開始變得小.
所以不要擔心.
如果一切看起來很不重要.
只有耶穌和一位工匠.
它是小的.
但它會長大.
另一方面.
就是種了柑橘子的人.
他起來.
他去睡覺.
他不知道發生了什麼.
而柑橘子本身已經長大.
並產生了種子.
所以耶穌似乎在說.
柑橘子已經種了.
而神的國王也會長大.
也會產生種子.
所以已經有了.
工匠已經做了他的工作.

$^{961}$也有了未來.
我認為新教的作家們.
並沒有提到.
耶穌的復活.
因為他們只是在痛苦.
並沒有希望.
他們是在痛苦.
但我覺得是因為.
他們有了.
復活的信心.
這個耶穌.
你基督基督.
祂將祂提出.
從死去.
並使祂成為耶穌.
因此.
祂現在是.
天上的國王.
如同聖經所說.
未來也屬於祂.
現在和未來.
所以他們可以.
走出一個世界.
那時總是說.
耶穌是主的.
並說不.
耶穌是主的.
然後去死.
因為他們願意.
認同.
世界的國王.
已經被基督基督復活.
所以我認為.
新教的.
已經和未來.
你去看.
在《聖經》中.
那裡的約翰.
在帕瑪斯的監獄.
或許他自己在監獄.

$^{1001}$他可能只認識.
幾個基督徒.
在伊布萊斯斯.
或者是其他地方.
但耶穌給了他.
這個視覺.
基督復活.
以及一種視覺.
是所有的國家.
族群.
語言.
It's a kind of tearing back of the veil to say this is reality..
There is a throne at the center of the universe and god is on it,.
and there is a lamb who was crucified,.
in whose hand is the scroll of the whole of human history,.
and human history is under the control of the lamb of god..
And so he gives hope to the martyrs,.
the suffering churches that he writes to,.
because he sees the reality of the present glory of Christ..
I think that's the way the New Testament expresses it..
I don't know whether Stephen you want to add..
For the last 12 months or so,.
or to be more precise,.
since the national security law comes into place,.
I kept asking myself,.
why are we evangelical church leaders.
are so afraid of suffering and persecution,.
that we leave so quickly?.
Of course there are exceptions,.
but I remember,I recall back in the 80s,.
when Eastern European Christians came out,.
and the first question they asked,.
why are you Western Christians not being persecuted all the time?.
It's the other way around..
So in our faith,.
is it a kind of do-it-yourself,.
IKEA furniture building game?.
Is all we are doing things,.
when we cannot do anything,.
then we gave up and leave?.

$^{1041}$Or is it a trust in the one true God,.
in spite of our own,.
was it frustrations, failures?.
I think the Old Testament talks more about failures than success..
I don't know whether this is true in the Psalms..
I mean, you know, laments more than praises..
So you talk about the meta-narrative,.
it's a story of failure..
It's not a story of success..
And you read the book of Acts,.
it's not the church,you know,practicing mission,.
but it's persecutions,it's arguments,.
it's seal it,or whatever..
Among these realities,.
the way of the Lord conquers the Roman Empire,.
that seems to be the hope,.
which again I would say not where are we going to end,.
but here and now,.
here in Hong Kong today,.
or putting aside the sociopolitical,.
moralistic reading of scripture,.
but as a cancer patient,.
I'm faced with this reality every day..
Oh, I hope someday,.
no, no, no, no, no,.
it's every day, every hour, every minute..
And so, yes, Christian hope..
I think I always want to ask New Testament colleagues.
or theology colleagues,.
ascetology, is it always in the future?.
No..
I remember Carver Yu, our president emeritus,.
once answered questions on our curriculum,.
because some senior pastors raised his hands and queried,.
CGSD doesn't have a course on eschatology..
And Dr. Yu stood up..
I was a dean..
I don't know how to answer as an Old Testament scholar,.
but he is the theologian at CGSD..
So he stood up and said,.

$^{1081}$"Eschatology permeates every lesson.".
Has he spoken the same to you?.
I remember that statement very clearly..
So I suppose that is the Reformed tradition,.
or the legacy..
And I think our church here in Hong Kong.
really undergoing something that the Lord has prepared for us.
and pushed us to reflect and review and self-criticism,.
not criticizing the other side, but ourselves..
Can I add one thing about how,.
during the present suffering or persecution or frustration,.
how we may maintain hope?.
I think the psalmic literature in this respect.
helps us or provides us with a language.
how we can talk to God in those times.
so that we may uphold our hope.
and uphold our hope in the one sovereign God.
at the same time without.
subcoming to the external pressures..
And I think this is one of the contributions.
I think that's why psalms is so important..
I take it you teach the psalms?.
Oh, sorry..
I don't mean to let you know about that..
Yes, why did you say what you really think?.
I want to agree with you,.
because I think the psalms are tragically neglected.
in the Christian church..
When I speak, even sadly, of my own church,.
All Souls Church, I keep on trying to get them.
to use the psalms more often, to pray the psalms,.
to actually sing the psalms from time to time..
Because the church through the ages,.
the psalms were the hymn book for the church..
And as you rightly said,.
the title of the book is "The Praises,".
and yet the largest category of psalms.
within the book of psalms are laments..
But lament is not an absence of praise..
Lament, praise in the biblical sense,.

$^{1121}$is recognizing the reality and presence of God,.
saying, "God, you are there,.
and I will worship and glorify you.".
But actually, God, at the moment,.
life is pretty tough right here..
People are laughing at me, I'm suffering,.
and you're not doing anything about it..
When are you going to do something about it?.
I praise you, I love you, I trust you, I worship you,.
but it's pretty bad..
So the psalms are willing to protest to God.
about what's going on,.
and there's nothing wrong with that..
I think, in fact, the idea that Christians.
should always just be happy,.
and never be angry with God.
or never say anything negative,.
that's not the way the Old Testament saints behaved..
They were quite willing to, it says,.
beat on God's chest or to cry on his shoulder.
and then to acknowledge him, to say,.
"But God, I still trust you..
You're still there,.
even though I don't understand what you're doing.
or why you're letting this happen.".
And so that is this balance.
between recognizing that there's evil in the world.
and wrong things are being done.
and things are not the way we want them.
and it's not comfortable.
and we may be suffering,.
and yet somehow being willing to say,.
in all of that, God remains sovereign,.
God has his purposes,.
and God will remember his own people.
and preserve them, ultimately,.
even through death, even through martyrdom,.
that you're safe in the hands of God..
So I think we do need to use the psalms more,.
and that's what Revelation does as well..

$^{1161}$It fills the book with the songs of the people of God,.
celebrating the sovereignty of God,.
even in the midst of the beasts and the empire.
and the wrong things that are being done.
by the kings and the nations.
before the end of the world..
Thank you..
So, the next question..
Thank you, Dr. Rice and Dr. Lee..
The sharing of Dr. Rice on the FIFO forms.
of hope, the hope of sinners,.
of the nations and of creation.
is truly inspiring..
And as you mentioned about new creation.
repeatedly in this lecture,.
and one important agent of the new creation.
in the New Testament is the Holy Spirit..
But maybe due to lack of time,.
you rarely mentioned the Spirit.
in the previous lectures..
Yet, I believe the Holy Spirit.
is the important agent of new creation.
who reborn the Christians,.
the John 3,.
and transformed them into new creation in Christ,.
2 Corinthians 5, verse 17..
And my question is,.
I have two questions..
The first question is,.
is the Spirit of God in the Old Testament.
referring to the same agent.
as the Holy Spirit in the New Testament?.
Some biblical scholars prefer.
to treat them separately.
rather than equating them..
What do you think?.
And the second question is,.
you mentioned about the metanarrative,.
the mission of God,.
the story of God..

$^{1201}$Can you outline the story of the Spirit,.
the narrative of the Spirit.
within your metanarrative?.
Briefly, thank you..
Thank you..
My answer to the first part of your question is yes..
Yes, the Spirit of God.
is the same Spirit of God.
as the Spirit of Yahweh in the Old Testament,.
who is sometimes called the Holy Spirit,.
the Spirit of God's holiness,.
and the Holy Spirit as we see in the New..
I'm absolutely convinced of that..
God is only one Spirit..
It is the Trinity..
We believe God is Father, Son, and Holy Spirit..
I've actually written a book called.
Knowing God the Spirit,.
Knowing the Holy Spirit.
Through the Old Testament..
So you might want to look for that.
because he is the Spirit of creation..
He's there..
He's the Spirit of justice..
He's the Spirit of....
Do you know who the first people.
who are said to be filled with the Spirit are?.
If you want to be filled with the Spirit,.
the first people who are filled with the Spirit.
are in the Old Testament,.
in the Book of Exodus,.
and they're called Bezalel and Aholiab..
You should know those guys..
They were artists..
They were craftsmen..
They were there to work with wood and stone.
and fine metals and precious stones.
to build a tabernacle..
The God whose Spirit had done creation.
with all its beauty.

$^{1241}$now pours the same Spirit on these men.
to be the artisans of the tabernacle..
It's a beautiful thought..
And then he's the Spirit of justice..
He's the Spirit of the prophets and so on..
And it's also the Spirit who is poured out.
in the eschatological vision of new creation.
because Isaiah speaks about God.
pouring out His Spirit on the land.
and restoring His people..
And of course, so does Joel..
And Peter quotes from Joel in the Day of Pentecost.
that this is that which the prophet Joel spoke about..
So the idea that Spirit is different.
in the Old and New Testament.
wouldn't stand up to the Day of Pentecost.
where Peter equates..
In relation to the metanarrative,.
I would want to put my hand up and say,.
yes, you are right..
That is an element that I should bring in more clearly,.
namely the role of God's Spirit in the story.
and the role of the Holy Spirit in Christian mission..
So forgive me for not emphasizing that as much as I should.
because I think you're right..
But I want to say that the Spirit is there.
throughout the whole of the narrative, isn't it?.
The Spirit is there in the second verse of the Bible..
The Spirit of God is hovering over the waters,.
over the deep there in creation..
And the Spirit of God is there in the ministry of Jesus..
He is filled with the Spirit..
He is conceived by the Spirit..
If I, the Spirit of God is doing the work..
And also, Hebrews says that even the work of atonement.
is the work of the whole Trinity.
because it's through the Holy Spirit.
that Jesus offered Himself to the cross..
So the Father, Son, and Holy Spirit.
are united in the work of atonement..

$^{1281}$And then, of course, the Spirit comes.
in the book of Revelation..
The Spirit is there before the throne of God.
and engaged in the work of new creation and so on..
So that's something I should factor in..
And thank you for the reminder..
And I wouldn't want in any sense to have overlooked it,.
Him, the Spirit..
But I should have perhaps mentioned it more than I did..
But it's certainly something which, again,.
is cross-testamental..
It would be another uniting theme.
within who we are and why we're here.
and how we should live..
Well, who we are, we are created by the Spirit of God..
And how we should live?.
Well, by bearing the fruit of the Spirit,.
by being filled with the Spirit,.
by living our lives in the Spirit and so on,.
walking in the Spirit..
Thank you..
(英文).
Steven, you are the one who talked about.
biblical theology and dogmatic theology..
And, yeah, there is obviously a difference,.
but you explain it..
Well, I think at the risk of being.
over-simplifying things,.
I would say dogmatic theology or Christian theology..
Christian theology, yeah..
By theology, logos of theos,.
we are talking about logic..
So today we talked about systematic theology..
Systematic is logical..
But going back to the Old Testament,.
I don't claim to know the New Testament.
as good as the Old Testament..
But talking about the Old Testament,.
it's never logical, never..
Who stopped?.

$^{1321}$What do you say?.
The Amazon River with all the undercurrents upside down..
Yes..
That's a challenge..
My understanding of the phrase biblical theology.
means the theology of the biblical texts.
seen in their own context and background..
So one could speak about, say,.
the theology of Ezekiel.
or the theology of the Pentateuch.
or the theology of the Old Testament..
It's thinking what is the Bible saying.
about God and God's action in the world..
And you were right, as James Sanders said..
Was it James Sanders?.
The whole is about what God is doing,.
not about us..
God is the main character..
That's why I like the idea of it being a drama.
in which God is the director.
and the chief author of the whole story..
So biblical theology is the theology.
as articulated when we try to explain.
what the Bible is saying..
Systematic theology is the way.
in which we then try to put that together.
into categories such as, well,.
what is our doctrine of God, of creation,.
of humanity, of sin, of salvation?.
And so we create a kind of conceptual framework.
for the Christian faith.
based around those biblical theology.
So systematic theology has to be rooted.
in what the Bible says,.
but in a sense it's trying to organize it.
into a more conceptual framework..
But here is what I want to say.
is that to me, all that we put.
into our systematic theology.
flows from the grand meta-narrative.

$^{1361}$of the Scripture.
because it all comes out of that great story..
It tells us there is one living God..
He created this world..
He made us in His image..
We rebelled and fell against Him, fell into sin..
God chose to redeem us,.
so He had to atone for our sin..
So it leads you to Christ,.
leads you to the cross, the resurrection,.
and ultimately the future..
So our doctrines are in a sense.
an articulation of the implications.
of the biblical story.
and what it is telling us.
about God and the world..
The two need to mesh together..
There is also a sense in which,.
and maybe this is why CGST moves to put,.
teach systematic theology,.
teach dogmatics first,.
and then see how the Bible critiques.
and explains it,.
is because the early Christian fathers.
talked about the rule of faith..
They said that there is a way.
of understanding the Scriptures,.
which comes from the Scriptures,.
but it's possible to read the Scriptures.
in a way which distorts their meaning..
You could, as we all know,.
you can make a Bible text.
mean whatever you want..
You can take verses and passages.
and you can really say things.
based on a single verse,.
which would be quite contrary.
to what the whole Bible says..
Prosperity Gospel does that all the time,.
for example..

$^{1401}$And what the fathers were saying is.
we need to read each passage of Scripture.
and each book of the Scripture.
in consistency with what the whole.
of the Bible has to say,.
the whole canonical rule of faith..
That's why there's a commentary series.
by Zondervan just in the process.
of being produced called.
The Story of God Bible Commentary..
I've contributed the volume on Exodus..
I don't know if it's in the bookstall downstairs,.
but the idea of the series is.
that each commentator will work.
on the text of that book.
in the light of the story.
that has gone before.
and the story that follows..
In other words,.
setting the text in the light.
of the canonical Scripture,.
especially, of course,.
as centered on Christ..
And I think that's a very biblical way.
of reading the Bible itself,.
because it's what the Bible does..
They keep on telling us the story.
as they go through..
So biblical theology, Christian theology,.
they are ultimately the same,.
but they're just different ways.
of organizing and expressing.
what we believe as Christians from the Bible..
Is that fair enough?.
Something like that..
I just want to add one footnote..
A long one?.
No..
Is it logical?.
No, it is not logical,.

$^{1441}$because you talk about metanarrative,.
the Bible story..
I would just want to add biblical poetry.
or in Brueggemann's Old Testament theology,.
the counter testimony.
or Robert Davidson's.
The Courage to Doubt..
You know what I'm saying..
I mean, it is not just the logical,.
the positive,.
or even the triumphalistic way.
of reading Scripture into theology.
and faith and work..
No..
It's actually,.
I like your word, struggle..
It's a struggle..
And during the struggling,.
you discover it's God,.
the only one true God..
So you are not working for your salvation..
It's grace..
It's sola gratia..
So this is the Bible..
Yes..
I don't want to give the impression.
that I'm looking down upon.
systematic theology..
My first theological degree,.
divinity training,.
I did a double major in Old Testament.
and systematic theology..
And then I moved on to my THM.
in historical theology,.
then coming back after playing truant.
to Old Testament doctoral research..
So I suppose dogmatic theology,.
historical theology,.
systematic theology,.
really helps, as you say,.

$^{1481}$categorizing the content of Christian faith..
But still, when you open the Old Testament,.
it doesn't begin with, you know,.
systematic theology,.
existence of God..
Instead it says,.
only fools will say there is no God..
So....
I think, Stephen,.
with this word "metanarrative",.
I would emphasize the meta bit of it,.
because although I drew that diagram,.
a nice straight line,.
nice symbols and so on,.
I don't want to say that it's all just story,.
that it's all just narrative..
The whole point about a metanarrative is that.
here is an overarching canonical shape.
to the scripture,.
which has a beginning and an ending,.
which is a new beginning at the end..
And within that canonical metanarrative,.
of course,.
there are all kinds of counter movements.
and questions and problems and struggles..
And that's why I love a book like Ecclesiastes,.
which I've written a small exposition of..
It's coming out next year..
Because what Ecclesiastes does,.
and to some extent Job,.
but especially Ecclesiastes,.
it says,.
Look, I know that this is what God says,.
and I know this is what wisdom should be,.
and I know it's better to be righteous.
than to be wicked,.
and it's even probably better to be alive.
than to be dead,.
although you can't really tell.
when you look at a dead dog..

$^{1521}$It's just as dead as a dead human..
So, you know,.
he's facing the fact that he knows.
with his head what's true,.
and yet he struggles.
with all the stupidities,.
the absurdities,.
the futilities of life,.
and he just puts the two together.
and says,.
What do you think?.
And it's a wonderful book.
of sort of questioning faith..
And I use the word questioning.
both in the sense of it is.
questioning the faith he knows he has,.
because the book still comes to the fear of God,.
but it's also a faith.
which has the ability to question..
And so in my book I say.
just because we're Christian believers.
doesn't mean that we don't have any questions,.
but also just because we're Christian believers.
doesn't mean we have all the answers.
to all the questions we have..
Sometimes we don't have answers..
That's why I wrote a book.
called The God I Don't Understand,.
because there are many things.
you don't understand about God,.
but it's okay..
There's a lot I don't understand.
about my wife,.
but that doesn't mean I don't know her.
and love her and trust her..
She's a person,.
and you don't always understand other people,.
but you can love and trust them..
And so you can love and trust and know God,.
even though there are things.

$^{1561}$beyond what we can really understand,.
even when we have to say,.
God, why did you allow that?.
We don't know why..
And so that's true..
The Old Testament has this wonderful capacity.
to affirm the truth,.
but also to put the truth into the furnace.
of questions and doubt and suffering.
and lament and protest.
and still come out the other side.
as faith in the sovereign goodness of God..
I want to say a final word.
before I stop..
Thank you for quoting very often.
from the Book of Isaiah,.
the prophecy..
The poetic commentary.
rather than the storyline part.
of the Old Testament..
I think that shows your meta narrative..
It's not simply narrative or storyline..
So thank you very much indeed..
Yes, I think that is a good question.
to close the whole discussion..
Well, actually, I suspect.
I'm just doing a reading response reading.
of that question..
The person asking,.
a Christian, our Christian theology.
actually, may have something in mind.
that a theology that has to be unifying,.
unitary, and that can maybe.
can include everything..
But perhaps what we are led to explore.
is actually we may have.
political theologies, if I may say..
And there are actually different trajectories.
within this meta narrative..
And we do look forward.

$^{1601}$to your coming books.
to lead us to scale.
along all these trajectories again..
So once again,.
our warmest applause to Dr. Wright.
and also Dr. Liu..
謝謝大家..
(CC字幕製作:貝爾).
MING PAO CANADA | MING PAO TORONTO.
\newpage



\section{}
\label{sec:0xAv1_RZmUM}
\textbf{The Old Testament and Christian Hope: Where will it all end? (4) — Speaker: Dr Christopher Wright}
\newline
\newline
連結: \href{https://youtube.com/watch?v=0xAv1-RZmUM}{\texttt{ https://youtube.com/watch?v=0xAv1-RZmUM}} ~~~~ 語音日期: 2023-03-08 
\newline
\newline
\hyperref[sec:c9xzZQkJF0c]{\small{< < < PREV SERMON < < <}}
~
\hyperref[sec:index]{\small{[返主目錄]}}
~
\hyperref[sec:LWCkMGG0tpo]{\small{> > > NEXT SERMON > > >}}
\newline
\newline
$^{1}$(音樂).
Good evening, ladies and gentlemen,.
fellow sisters and brothers in Christ..
Welcome back to the fourth and final lecture.
of this year's Josephine So Culture and Ethics lectures..
In this session, our distinguished speaker,.
Reverend Dr. Christopher Wright,.
Langham's Global Ambassador and Ministry Director,.
will continue to speak to us.
on the relevance of the Old Testament to our Christian faith..
In the previous three lectures,.
he finally elucidated for us.
how the Old Testament may bear on our understanding.
of our Christian identity, missions and ethics..
Tonight, he'll stretch our horizon a bit forward.
and cast a look into the future..
Tonight's topic is.
The Old Testament and Christian Hope,.
Where Will It All End?.
Over the dinner, I confessed to Dr. Wright that.
the Chinese translations are written by me..
So all the distortions and mistakes were on me..
Accompanying our speaker on stage tonight.
is our Emeritus President, Dr. Stephen Lee..
Dr. Lee shares the same enthusiasm as our speaker.
in helping Christians to reckon the Old Testament.
as an indismissible part of their faith..
Some 20-odd years ago,.
I listened to one of his talks on the cassette tape..
Do you know cassette tapes ever existed on this earth?.
No, okay..
And the title of his talk is.
A Bible with Only the Upper but No Ground Floor.
只有樓上沒有樓下的聖經.
in which he explained that.
our Christian faith is grounded.
in both the Old and the New Testament..
As for tonight,.
Dr. Wright and Dr. Lee.
will have a dialogue on this interesting topic.

$^{41}$Old Testament and Christian Hope..
Dr. Wright will give his first lecture.
will give the lecture first.
a 25-minute speech.
followed by Dr. Lee's response.
of about the same length..
Then, they will further exchange their views.
for another half an hour..
And then the discussion.
will be open to the floor..
So for the time being,.
you may ask our ushers.
for a piece of paper or a piece of note.
to write down your questions or thoughts..
And without further ado,.
let's welcome Dr. Wright..
Thank you so much, Simon, Dr. Chung..
And welcome again..
And if you have been with me.
all through these four, well done..
It's a kind of a marathon for all of us,.
but it's good to see you again..
Let me just get to my right page on the notes..
There we are..
So yes, our title this time is.
Where Will It All End?.
We've seen, haven't we,.
in these previous three lectures.
how the Old Testament shapes.
the identity and the mission.
and the ethic of God's people..
Those of us who are in Christ,.
as Paul says in Galatians,.
are in Abraham..
We are part of that people.
believing Jews and Gentiles..
And we are shaped together.
by the scriptures as a whole,.
the scriptures of our Lord Jesus Christ,.
of course, what we now call the Old Testament..

$^{81}$So the Old Testament then,.
topped by the New,.
tells us who we are,.
why we are here,.
and how we are to live..
That's what we've been thinking about.
over the three lectures..
But what's the point?.
Where is it all going to end?.
Where does this story go?.
All human religions and philosophies.
and ideologies.
express some elements of narrative or story.
to explain how things are.
and how they got to be the way they are.
and how we could make them better.
if we want to..
And the Bible as a canonical whole,.
the whole scripture that we have,.
is also in that sense a narrative,.
indeed, a metanarrative..
It tells the big story.
that runs from creation,.
right through to new creation,.
from Genesis to Revelation..
But it's not just a story..
It claims to be the story,.
the true story of the whole universe..
Now,in other places.
and in some of my books,.
I've tried to pick up an idea.
that I first got from,.
well,originally from Tom Wright.
and then from Michael Goheen.
and Craig Bartholomew,.
who talk about the Bible.
as rather like a great drama,.
that is a great cosmic play.
that has taken several acts,.
you know how dramas.

$^{121}$have different acts,.
in which God is the author.
and, of course, the chief character,.
but there's a cast of thousands as well..
And one way in which.
I've tried to portray this.
is in a kind of diagram.
which I put into.
one of my other books.
called The Great Story.
and the Great Commission,.
which is published by Baker,.
like this, with sort of symbols.
that express the different acts.
of the Bible..
So here it is.
as a drama in seven acts..
I'll try to go through this.
fairly quickly, watching my time,.
because I'm very aware.
that there are two of us this evening..
Act one of the Bible story,.
of course, is the creation..
In the beginning,.
God created the heavens and the earth..
And I've put this as a triangle.
because there is God..
He created the earth.
and he put human beings in it..
Act one, all a good start..
But then, of course,.
we know that everything went wrong.
because we chose.
to rebel against our Creator,.
to distrust his goodness,.
to disobey his instructions..
And so that symbol expresses.
the sense of rebellion..
Act two of the story.
is where all the problems begin..

$^{161}$And then act three is basically.
the rest of the Old Testament..
It's a very long act.
and there's an awful lot in it..
And it's not just a story..
Of course, it's full of other things.
like poetry and psalms.
and wisdom and everything else..
But the arrow pointing forward.
has this sense of movement.
that it is going somewhere.
because it's based on.
God's promise,.
the promise that he made to Abraham.
that he would be a nation.
and through him the nation.
is to be blessed.
as we've been thinking.
repeatedly through these.
last couple of days..
Act three, therefore,.
is a story which ultimately.
leads up to act four,.
which, of course, is the coming.
of the Lord Jesus Christ.
when the God of Israel.
keeps his promise,.
enters into human history.
as a human being,.
the Word becomes flesh.
and then lives and teaches,.
dies the atoning death.
on the cross.
and rises again.
and returns to his father..
So that cross symbol at the centre.
is not just the crucifixion.
but is intended to represent.
everything we have in the Gospels,.
from the conception.

$^{201}$by the Holy Spirit.
through to the ascension.
of the Lord Jesus Christ..
But the Bible doesn't end there,.
does it? Of course not..
It moves on through the book of Acts.
into the outpouring of God's Spirit.
on his people,.
sending the people of God.
out in mission.
to the ends of the earth..
We live within that era of mission.
begun in the New Testament.
and continuing to this day,.
which leads to the last two.
great acts of the story..
I put a tick here.
for the final judgment..
It might seem a wrong symbol,.
but in a sense,.
that is when God puts everything right..
It's when God deals with evil,.
so evil will not have the last word..
Thank God..
The judgment is actually good news.
in Paul's thinking.
because it is when God rectifies.
all that went wrong.
in the story..
And that then leads.
to the new creation,.
the new heavens and the new earth.
that we read about there.
in Revelation 21 and 22,.
but of course actually promised.
in the book of Isaiah..
Now,that's simply trying to portray.
the whole Bible story.
on the back of an envelope,.
in a sense,.

$^{241}$which is where I first saw it done,.
actually with a friend of mine.
in the US with some other.
slightly different symbols,.
but the same idea..
I've even done this.
on the back of a napkin.
in a restaurant.
sitting beside somebody.
and say,.
what do you think the whole Bible is?.
Well,just draw it like this..
Here's the whole story of the Bible.
in one piece..
And this helps us to see,.
of course,.
that we now,.
we are somewhere in Act 5,.
that is to say,.
we are in between.
the resurrection of Christ.
and the return of Christ..
We are in the Bible..
Did you know you're in the Bible?.
We are..
We are actually participating.
in this drama..
We are part of the story..
This is the story we are in..
And of course,.
this Act 5 is not the end..
We know that ultimately.
this story is leading.
to a glorious future,.
which gives us hope..
That's why I use the word hope.
in the title of the lecture..
Now,hope,of course,.
is one of those massive.
New Testament words..

$^{281}$In fact,.
the Apostle Paul puts it.
in his favourite three,.
doesn't he?.
Now abide faith,.
hope and love..
But Christian hope.
is not just optimism..
It doesn't just mean,.
oh,we like to think.
that everything's going.
to get better and better..
That's the myth of modernity,.
that somehow human progress,.
science and wisdom.
will eventually.
create a better world..
Well,Jesus told us.
that things may well get worse.
as they get better..
In other words,.
the wheat of the kingdom of God.
and the weeds of the evil one.
work together.
all the way through history..
So Christian hope.
is not optimism..
Rather,it is the certainty,.
the biblical certainty.
that the future of this planet,.
indeed,the future of the universe.
is in God's hands.
and God's sovereign purpose.
will be accomplished.
in the reconciliation.
of his whole creation.
and the ingathering of people.
from every tribe.
and nation and language.
and that that hope.

$^{321}$has indeed already.
been accomplished.
in anticipation.
through the cross.
and resurrection of Christ,.
as we shall see..
Now,what I want us to see.
is that this sense of Christian hope.
didn't just sort of drop.
into the Apostle Paul's mind.
from nowhere..
It actually comes.
very deeply rooted.
in his scriptures.
even before his conversion..
In other words,.
Paul tells us,.
doesn't he,.
that he was a Hebrew-speaking Jew,.
that he was a Torah-observant Pharisee..
He tells us that in Philippians..
So he knew the scriptures..
He probably chewed masses.
of the Old Testament by heart..
So therefore,.
he knew about the God of hope.
as he speaks in Romans 15..
Indeed,I think if you had asked.
Saul of Tarsus.
before his experience.
of the Damascus Road,.
what gave him hope for the future,.
he would have taken you.
to some of the texts.
that we're just going to look at.
in a moment,.
to multiple scriptures..
He would have believed.
that the future would include.
God's restoration of Israel..

$^{361}$That's what would happen..
And then after the restoration.
of Israel would come.
the ingathering of the nations.
and then ultimately would come.
a new heaven and a new earth,.
which he would have read about.
in Isaiah..
And if you were to ask.
the same question.
to Saul of Tarsus.
after he had become.
the Apostle Paul,.
after his experience.
on the Damascus Road,.
then he would have done the same.
because he actually did..
When he wants to talk.
about the hope that we have,.
he uses the Old Testament.
scriptures to do so..
And that's why I wanted.
to use this text.
as a kind of control text.
here from Romans chapter 15.
because hope figures.
again and again in it..
Romans chapter 15 verse 4.
says this..
He says, Paul says,.
that everything that was.
written in the past,.
and of course he means.
the scriptures of what we call.
the Old Testament,.
was written to teach us.
so that through the endurance.
taught in the scriptures.
and the encouragement.
they provide,.

$^{401}$we might have hope..
Paul says the whole Old Testament.
was to give us hope..
Have you ever thought of that?.
The Old Testament.
gives you a lot of problems.
and a lot of classes.
and a lot of exams.
and everything else..
But Paul says.
the Old Testament.
was given to us.
to give us hope..
And then immediately.
in the following verses,.
verses 8 and 9,.
he summarizes.
what I just described.
as Acts 3, 4 and 5.
of the biblical narrative..
Because he says,.
I tell you, says Paul,.
that the Messiah Christ.
has become a servant.
of the Jews,.
that's the story.
of the Gospels, isn't it?.
on behalf of God's truth.
so that the promises.
made to the patriarchs.
might be confirmed.
and moreover,.
that the Gentiles.
might glorify God.
for his mercy..
So Paul says,.
Jesus came.
as the servant of the Lord,.
that's Act 4,.
the Gospel story there,.

$^{441}$the cross narrative,.
as portrayed in Isaiah,.
but he had come.
in order to fulfill.
God's promises.
to the patriarchs,.
meaning, of course, Abraham..
So that was Act 3.
of the drama..
And he says,.
that was so that.
all nations,.
the Gentiles,.
would come to find.
their hope.
and their blessing.
and the glorifying God.
for his mercy,.
which was now happening.
in Act 5,.
because Paul was taking.
the Gospel to the nations.
and seeing the.
responding to God's mercy..
And then,.
in order to.
reinforce that.
climactic point,.
which is,.
let's remember that.
Romans 15.
is the climax.
of the whole letter..
Paul didn't write.
a postcard to the Romans,.
which ended at Romans 8..
He wrote a whole letter.
to the Romans,.
which comes right up.
to chapter 15 and 16..

$^{481}$So let's hold it all together..
He said that God's intention,.
his intention,.
from chapter 1,.
was to bring about.
the obedience of faith.
among the nations..
That's what the Gospel.
is all about..
And so Paul,.
when he gets to this point,.
that it's for the glory of God.
among the nations,.
throws in.
four Old Testament quotations.
immediately afterwards..
I referred to this.
in an earlier lecture.
from the Psalms,.
the Torah,.
and the Prophets,.
all of which.
speak about the Gentiles,.
the nations,.
rejoicing along with Israel..
Because now,.
under the reign of.
the Messiah,.
King Jesus,.
Son of David,.
they have hope.
for the future of the world..
Here's how Paul.
concludes that list..
It's in,.
again,.
still in Romans,.
sorry,.
in Romans,.
chapter 15..

$^{521}$He says,.
"Therefore I will praise you.
among the Gentiles,.
rejoice you Gentiles,.
praise the Lord,.
you Gentiles..
The root of Jesse.
will arise,.
and in him.
the Gentiles will hope.".
That's his climax..
Hope for the world,.
hope for the nations,.
is the climax.
of the Book of Romans,.
because Jesus of Nazareth.
has fulfilled the promises.
of the Old Testament Scriptures,.
and the mercy.
and the glory of God.
is now going to the ends.
of the earth..
That vision is now being fulfilled.
through Christ and his victory..
And it's being extended.
to every sinner.
and to all nations.
and indeed to the whole.
of creation..
So that was simply.
a kind of a glimpse.
at Romans 15..
It's a significant passage..
Please don't overlook it.
because you can't get past.
Romans 9 to 11.
or something..
It's a very important way.
in which Paul draws his letter.
to a conclusion..

$^{561}$So I want us to think of this.
hope then, the God of hope,.
briefly in three areas..
As I said, hope for sinners,.
hope for the nations,.
and hope for creation..
So first of all then,.
hope for sinners.
goes, takes us right back,.
of course, to the big problem.
that we have,.
which is in Genesis 3,.
which is because we chose.
in Adam and Eve.
to distrust and disobey God.
The result was sin, evil.
and death.
and that penetrates.
to every dimension.
of the human personality..
It's remarkable.
in that little story there.
in Genesis chapter 3.
of how when Eve is,.
you know, confronted.
by this temptation.
from the serpent.
that the evil.
that is somehow there.
and it's not explained.
where this has come from.
or why it has happened,.
but it speaks into her.
spiritual relationship.
with God..
Did God really say.
you don't want to believe him?.
It penetrates her mind.
because she now rationalizes.
the fruit of the tree..

$^{601}$It's good, beautiful..
It's good to eat..
It makes you wise..
And then also it enters.
her physical body..
She takes, she eats,.
and then it enters.
into her social relationship.
with her husband.
because she gives it to him..
He was with her,.
the text says..
Please notice, you men,.
including myself..
Adam was not.
somewhere else in the garden..
He was with her.
while this was happening..
So he was a witness to it..
So sin enters.
into every dimension.
of the human person.
and indeed to all of us.
because as Paul would say,.
all of us have sinned in Adam..
And so that leads, of course,.
to the promise of God.
in Genesis 3, verse 15,.
sometimes called.
the first gospel.
of the Proto-Evangelium,.
where God says.
even in the context.
of that, the fall,.
God says to the serpent.
that I will put enmity.
between you and the woman.
and your offspring and hers..
He, that is,.
the seed of the woman,.

$^{641}$a human being,.
he will crush your head,.
the head of the serpent,.
and you will strike his heel..
So there will be warfare.
between the serpent.
and what the serpent represents.
and the human race.
represented by the seed of Eve..
But it will be a human being.
who ultimately.
will be the serpent crusher,.
as it were,.
and destroy that evil..
Now, very quickly to say,.
the rest of the Old Testament.
shows not just the fact.
that we are all sinners,.
but that there is an awareness.
that God's forgiveness.
and mercy and grace.
is available to sinners..
It's there in the Psalms.
at an individual level.
when the psalmist.
celebrate the fact.
that when they confess their sin,.
God forgives it..
Psalm 32, Psalm 51,.
Psalm 130, for example..
God, if you kept a record of sin,.
who could stand?.
But with you there is forgiveness..
Amazing words.
of personal, individual,.
forgiveness..
But how is that happening?.
Well, you look at Isaiah 53,.
and as we saw earlier.
in an earlier lecture,.

$^{681}$the servant of the Lord,.
the climax of those songs.
is when he will bear.
the sins of many,.
and because he would.
bear their sins,.
they would then be.
received the righteousness.
of God or be right with God..
That's in Isaiah 53,.
that he took up our pain,.
bore our suffering,.
crushed for our transgressions.
and the punishment.
that brought us peace.
was on him.
and by his wounds.
we are healed,.
which of course is picked up.
in the New Testament.
by Jesus and by Peter..
But the passage.
that I wanted particularly.
to take us to.
is that one on the screen.
of Romans chapter 5,.
where Paul explicitly.
makes the link.
between the sin.
that entered the world.
through Adam.
and the forgiveness.
that enters the world.
through the Lord Jesus Christ..
Romans 5, verse 12..
Therefore just as sin.
entered the world.
through one man.
and death through sin.
and in this way.

$^{721}$death came to all people.
because all have sinned,.
that's what comes.
into the world..
Then says Paul.
in verse,.
a little bit later.
in verse,.
have I got it?.
19..
He says,.
therefore just as.
the disobedience.
of the one man,.
many were made sinners,.
so also through.
the obedience of the one man,.
Jesus of course,.
the many will be made righteous..
So that just as sin.
reigned in death,.
so also grace.
might reign.
through righteousness.
to bring eternal life.
through Jesus Christ our Lord..
So that's the first point.
of the hope.
of the future.
of eternal life.
which we have in Christ.
is because we can have.
our sins forgiven..
There is hope for sinners.
and it's promised.
in the Old Testament.
and accomplished by Jesus..
And for some people.
that's it,.
that's the gospel,.

$^{761}$that's all we ever preach,.
which I think is true.
but very short-sighted.
because it's not all the Bible says.
about what God has accomplished.
through the cross of Christ.
because there's also,.
of course,.
secondly,.
hope for the nations..
You see,.
the opening chapters of Genesis,.
that Act 1 and 2,.
present not just that.
all human beings are sinners.
but also the problem.
of the nations,.
that the nations too.
are scattered.
and broken and divided..
God's got a very big problem.
on his hands..
It's not just that we are sinned.
and the earth is cursed.
but that nations are also.
scattered and divided..
And the chapters 4 to 11.
of Genesis show very clearly.
that sin isn't just personal,.
it affects society.
and cultures and generations..
So much so that by the time of Noah,.
we are told several times.
in those chapters.
that God saw that the earth.
was filled with corruption.
and violence..
Now those are social things..
That's what people do.
to each other..

$^{801}$And so that's why,.
as we've been seeing earlier on,.
the promise of Genesis 12.
is so important.
that God says to Abraham.
in the context of the Tower of Babel.
in chapter 11,.
God says that through you,.
all nations on earth.
will be blessed,.
not just experiencing that curse..
The blessing of God.
is to go far as the curse is found.
in Isaac Newton's great hymn,.
Joy to the World..
Unfortunately,.
the modern translation of that hymn.
has spoiled Isaac Newton's meaning.
because you know that.
where it says,.
"No more at thorns and thistles.
grow or sin infest the ground..
He comes to make his blessings flow.
far as the curse is found.".
That's explicitly relating.
Genesis 12, 11 and 12..
The more recent version of that hymn.
says,far as guilt,.
wherever guilt is found,.
which again reduces.
theological reductionism,.
as if somehow God's blessing.
was only because of human guilt,.
whereas actually the text.
and what Isaac Newton was saying.
was that God's blessing.
is for the whole of creation.
as well, wherever the curse is found,.
not just wherever guilt is found..
Anyway, that was an aside..

$^{841}$It isn't in my text..
So sorry, Stephen..
You didn't have that earlier..
Now what we see then.
is that God's promises.
you see then speak of.
his blessing for nations.
as well as for individuals..
And we saw some of that in the Psalms.
in the second lecture,.
Psalms which speak about.
all the nations you have made.
will come to worship you,.
all the families of the nations..
But in Isaiah,.
which of course is so influential.
on Paul,.
it's particularly the case.
that this is seen as part of.
the hope of the reign.
of the Messiah, Son of David,.
that when he comes to reign,.
then it's good news.
for the nations specifically..
So Isaiah chapter 2,.
when the mountain of the Lord.
is lifted up.
and all the nations.
will stream to it,.
many peoples will come and say,.
"Hey, let's go up to.
the mountain of Yahweh, the Lord,.
the temple of the God of Jacob..
He will teach us his ways..
We will walk in his ways.".
And then he will judge.
between the nations.
and settle disputes..
Wouldn't that be great?.
And they will beat.

$^{881}$their swords into plowshares.
and their spears.
into pruning hooks..
And nations will not.
take up sword against nation..
They won't train for war anymore..
What a hope for the nations,.
the ending of war..
And that's what.
Isaiah specifically says..
And then he repeats it.
in Isaiah chapter 11,.
that the shoot of Jesse,.
i.e. the Son of David,.
will come and it will bring.
blessing and righteousness.
and justice to the nations.
and indeed to creation..
Isaiah chapter 60.
speaks about all the nations.
coming to worship God..
And at the very end of Isaiah,.
in Isaiah chapter 66,.
we read God saying this..
God says that I,.
because of what they have.
planned and done,.
I am about to come.
and gather the people.
of all nations and languages.
and they will come.
and see my glory.
and I will set my sign.
among them..
And then it says,.
they will proclaim my glory.
among the nations..
So you see,.
when Paul talks.
in Romans 15,.

$^{921}$about the hope of the nations,.
of the God of that hope,.
it's speaking about all the nations.
and it comes from the Old Testament.
and it comes,.
of course,.
to its climax,.
as you can see on the screen.
in Romans chapter,.
in Revelation chapter 7,.
verses 9 and 10..
I've referred to this earlier.
where John says that.
there I saw a great multitude.
from every tribe,.
nation,.
people and language.
standing before the throne.
and before the Lamb..
That's there in Revelation 7.
and it isn't forgotten.
in the rest of the book.
because in Revelation 21,.
we read this.
in verses 24 to 26.
that the nations.
will walk by its light,.
that is the light of the city of God.
and the kings of the earth.
will bring their splendor.
into the city of God..
The glory and the honor.
of the nations.
will be brought.
into the city of God..
Now that's astonishing.
because all through.
the book of Revelation,.
the kings and the nations.
are the bad guys..

$^{961}$They're the ones.
who are in rebellion against God..
They're persecuting.
the people of God..
They've got the blood.
of the martyrs on their hands..
So in some radical transformation,.
the kings and the nations.
of the world.
are there in this new creation.
as nations,.
as kings bringing their glory.
into the kingdom of God..
So much so that.
the book of Revelation.
can end almost.
in its closing verses.
in chapter 22.
with the tree of life.
and the leaves of the tree.
are for the healing.
of the nations..
An incredible statement.
there right at the end.
of the Bible..
So from the old.
through to the new,.
there is hope..
Hope not only for sinners,.
but hope for the nations..
Now,.
I don't know what that means..
I don't know how to.
get my head around that.
in terms of.
what is that going to mean.
in our human history..
It's bigger than I can imagine..
That's because it comes from God..
And God is always bigger.

$^{1001}$than we can imagine..
But it gives us hope.
for the future,.
for this planet,.
that God ultimately has a plan.
that will bring blessing,.
peace,.
healing.
and the end of war and strife.
to the nations of this world..
Isn't that good?.
Good news..
Hallelujah..
Thank you..
But more than that,.
finally,.
there is hope for creation itself..
You see,.
again,.
God's big problem.
wasn't just that Adam and Eve.
had sinned..
It wasn't just.
that the nations were scattered,.
but that he said,.
"Cursed is the earth,.
the ground because of you.".
There is something in creation.
that is not the way.
it was intended to be..
Now,.
there's all sorts of ways.
of talking about that.
that I don't want to go into,.
but the apostle Paul.
in Romans 8 agrees..
He says that creation.
was subjected to frustration..
He uses the word "mataiotes,".
which is the same.

$^{1041}$in the Greek translation.
of Ecclesiastes' Hevel,.
which was translated "vanity.".
It's all something.
about this world.
which is,.
you know,.
frustrated.
and not the way it should be..
Yes,.
we can say that creation.
gives glory to God,.
and indeed,.
the Psalms will tell us that..
The whole creation.
is singing the praises.
and glory of God,.
and yet,.
even though it is doing that,.
it's still,.
in some sense,.
frustrated in it..
And I have to say.
that there are times.
when I look out.
at the glories of our world.
on a wonderful day.
or the mountains.
or the grandeur.
of some great scene,.
and I say to myself,.
"Wow,.
if this is what it looks like.
when it's frustrated,.
what's it going to look like.
when it's liberated.
in God's new creation?".
So,.
the creation itself,.
then,.

$^{1081}$is frustrated,.
but God says,.
in both the Psalms.
and in Isaiah,.
that his plans are indeed.
for a whole new heaven.
and a new earth,.
not in the sense.
of a complete replacement.
heaven and earth,.
but a renewal,.
a restoration.
of what the earth.
was intended to be..
So read Isaiah 65:17-25..
Yes,.
there it is..
See,.
I am creating,.
says God..
It's a participle..
I am creating new heavens.
and a new earth..
And then it describes.
that new heavens.
and new earth.
in the most wonderfully.
earthy language..
It talks about human life,.
about family..
It talks about work..
It talks about.
building houses.
and vineyards.
and satisfaction.
and no more frustration.
in our work.
and so on..
It's a wonderful picture.
of the new creation.

$^{1121}$there in Isaiah 65..
It's repeated in chapter 66..
And of course,.
it influenced John.
in Revelation chapter 21.
of the new creation..
But it's also there.
in the Psalms.
and elsewhere.
where the psalmist.
portrays the whole of creation.
as rejoicing..
And,.
you know,.
yes,.
it's poetic..
And we need to understand it.
in that sense..
But it is still.
pretty marvelous.
that when you have said.
among the nations.
that the Lord reigns,.
the kingdom of God.
in among the nations,.
then let the heavens rejoice,.
let the earth be glad,.
let the sea resign.
and all that's in it..
Let the fields be jubilant.
and everything in them..
Let all the trees of the field.
sing for joy..
Let all creation.
rejoice before the Lord..
Why?.
Because he's coming..
Because he comes..
And what's he going to do?.
He comes to judge.

$^{1161}$the world with righteousness..
In other words,.
he's coming to put things right..
Because that's what it means.
when God acts as judge..
He gets things sorted out..
He puts things right..
And the whole of creation.
is rejoicing.
and waiting for that day.
when God puts everything right..
That's,.
I think,.
what influenced Paul.
in Romans 8.
when he says in Romans 8.
verses 18 to 25.
that the whole of creation.
is just longing for that day.
when it will be liberated.
at the same time as we experience.
the redemption of our bodies..
Paul puts together.
the resurrection of the body.
with the redemption of creation..
That's a fascinating parallel..
It's actually very important.
when people try to think of,.
well,.
what is the new creation.
going to be like?.
Well,.
it's going to be like.
our resurrection bodies..
And our resurrection bodies.
are like the body of Jesus.
who was risen from the dead..
And when you ask,.
well,.
the risen Jesus,.

$^{1201}$was he the same.
as the earthly Jesus.
or different?.
Yeah,.
both..
It was the same Jesus..
They recognized him..
They saw the wounds in his hands.
and his feet and his side..
So it was the same Jesus..
And yet,.
it was a resurrected Jesus..
He was a whole new mode of being.
in the new creation,.
the first fruits of the new creation..
And so in that sense,.
I think,.
when we talk about the new heavens.
and the new earth,.
will it be just like this old heaven?.
Well,.
yes and no..
It will be this earth..
It will be this heaven..
But it will be now restored,.
refreshed,.
renewed,.
cleansed,.
purged.
in the new creation..
That's what I think Paul.
was talking about in Romans 8..
And he says that's already.
been accomplished.
in Colossians 1:15-20..
That marvelous passage.
where Paul,.
it's just incredible.
what he says..
About five times.

$^{1241}$he talks about all things,.
all things in heaven and earth..
He couldn't make it any more clear.
that he's talking about.
the whole of creation..
And he says Christ is the origin..
He's the head..
He's the source..
He holds it all together..
And through him,.
through Christ,.
God was pleased to reconcile.
to himself all things,.
whether things in earth.
or things in heaven,.
by making peace.
through his blood shed on the cross..
Have you ever thought about that?.
That the cross is cosmic.
in its scope..
It brings me salvation,.
brings you salvation..
It also brings salvation.
to the whole of creation..
So we're not going to be saved.
out of this world..
I wish we didn't have.
so many hymns and songs.
which talk about.
going somewhere else..
You know,.
they're really not biblical..
In fact,.
I get my music leaders.
at All Souls.
to change the words sometimes..
Even great hymns like,.
you know,.
How great thou art!.
When Christ shall come.

$^{1281}$with shout of acclamation.
and take me home..
He's not coming.
to take me anywhere..
He is coming to make his home here..
So we sing that.
when Christ shall come.
and make his home..
That's the way we sing it.
at All Souls.
because that's what it says.
in the Bible..
The Bible does not end.
with us going up to heaven..
The Bible ends.
with God bringing heaven to earth,.
bringing the city of God here..
Then the dwelling place of God.
will be with us,.
it says in Revelation 21,.
three times..
Revelation 21.
is the great Emmanuel fulfillment,.
God with us..
Emmanuel means God with us,.
not us going somewhere.
to be with God,.
but God coming to restore.
his creation.
in Revelation 21 and 22..
We could go on about that..
Perhaps the last thing.
to say then is.
very close to the end.
of the Bible.
in Revelation 22,.
verse 3,.
immediately after it says.
that the leaves of the tree.
will be for the healing.

$^{1321}$of the nations,.
the next verse says.
no longer will there be.
any curse..
The curse on the earth.
will be gone.
in the new creation..
So therefore,.
we need our whole Bible,.
I will end by saying.
to tell us who we are,.
to tell us why we're here,.
to tell us how we should live,.
and to tell us.
what we can look forward to.
in the hope of sinners,.
of the nations,.
and of creation..
So may the God of hope.
fill you with all joy and peace.
as you trust in him.
so that you may overflow.
with hope.
by the power of the Holy Spirit..
Amen..
中文字幕:YK.
CC字幕:貝爾.
\newpage



\section{}
\label{sec:LWCkMGG0tpo}
\textbf{The Old Testament and Christian Identity: Who are we? (1) — Dr Leo Li | Dr Lilian Li}
\newline
\newline
連結: \href{https://youtube.com/watch?v=LWCkMGG0tpo}{\texttt{ https://youtube.com/watch?v=LWCkMGG0tpo}} ~~~~ 語音日期: 2023-03-08 
\newline
\newline
\hyperref[sec:0xAv1_RZmUM]{\small{< < < PREV SERMON < < <}}
~
\hyperref[sec:index]{\small{[返主目錄]}}
~
\hyperref[sec:mzg9S8Q9BQ8]{\small{> > > NEXT SERMON > > >}}
\newline
\newline
$^{1}$[音樂].
謝謝Dr. Wright.
為我們帶來一場.
為我們作為神的人民.
反映自己的身份的旅程.
我們現在有兩位接待者.
為我們提供回應.
首先有Dr. Liu Kuan-Hong Lee.
教授中芝大學的神學教授.
香港中國大學的教授.
然後我們有Dr. Lillian Li.
教授政治學研究.
以及CGST的副教授.
謝謝Dr. Wright.
這次的講座非常明顯和有趣.
我非常榮幸能在這裡.
因為我是CGST的副教授.
還有很多朋友和朋友們.
我們聽了Dr. Wright的講座.
《古實書》和《基督的身份》.
我們知道.
這次講座的目的是探索.
我們基督的身份.
通過古實書的鏡頭.
就如講座的字幕所示.
我們的基督的身份是兩倍的.
我們是神的故事中的人.
我們也是神的榮耀中的人.
這點非常值得紀念.
在講座的第一部分.
Dr. Wright顯示.
古實書中的神的故事.
是由三個故事中的神.
人類的故事.
阿伯拉罕的故事.
和以色列的故事.
這些古實書的故事.
能夠幫助我們定義.
新教會的身份.
在第二部分.

$^{41}$Dr. Wright講到.
古實書中的說法語言.
這些說法能夠讓我們看到.
我們對神的榮耀是誰.
從古實書中的許多說法中.
Dr. Wright選擇了三個.
人類,家庭和寺廟.
或神的住所.
來作為他的描述.
從我的簡短總結中.
可以看到Dr. Wright的講座.
如此的整潔.
和容易跟隨.
這並不難以阻礙他的觀點.
Dr. Wright的語言是.
直率,明確,有趣.
我可以這麼說嗎.
沒有學術或哲學的語言.
而他的講座.
也很適合我們用的語言.
來形容聖教教會.
因此我相信今天的觀眾.
無法無故吸收.
這堂課的訊息.
然而.
當然,總有些缺點.
Dr. Wright的第一堂課的概念.
可能對我們來說.
太廣泛和普遍了.
最後我們身處教堂.
這裡有很多教堂學生.
所以這廣泛和普遍的.
舊穆斯林故事和.
反教會的關係.
可能對我們來說.
這只是不足夠.
有些人會懷疑.
知道還有兩堂課要來.
我相信我們會更滿意.
在這兩天之後.

$^{81}$很遺憾.
因為我的家庭工作.
和其他責任.
我無法參加其他課程.
但我期待第一天的會議和晚餐.
作為回應者.
我會想給予一些實際的反映.
在這堂課上.
特別是在這裡的觀眾.
可能正在尋找.
他們的身份.
作為一個基督教社群.
在香港的今天.
在本堂課的開始中.
讓我印象深刻的是.
基督教和教會的身份.
不僅是神的選擇人.
也就是神創造的人類.
一般來說.
這是一個非常不公平的說法.
香港的主流教會.
從數據上來說.
是從基督教和聖經上.
從我們的傳統上來的.
有時許多香港教會的教學.
都集中在個人的救贖和神學.
結果就掌握了.
基督教社會和人類的社會和人類的層面.
近年來.
香港教會也曾經注意到.
基督教社會的社會和政治層面.
但我還是想強調.
基督教社會的人類層面.
當我回想到羅生老師的講座時.
《古書》的創造故事.
是香港的基督教和教會.
一個重要的提醒.
我們人類的身份.
是由神創造的.
讓我們認出人類的歡樂和痛苦.

$^{121}$我們可以享受神的美麗.
並且受到神的恩賜.
我們可以成為人類.
我們與我們同類的兄弟姐妹.
共同分享更多的共同地位.
我不是只在教會中.
談及環境保護.
但我們也有時會在文化上.
在教會的教學中.
承諾我們要正確地.
在基督教的光明下.
做出人類化的事.
我們作為基督教徒.
我們可以在香港的文化.
藝術 文化.
或其他文化的表現中.
為人類做出好事嗎?.
我將在這堂課中教授.
基督教的文化傳說.
學生和我將討論.
基督教的意義和文化.
基督教的文化.
有更多的種類和種種.
比起新教的文化.
是我們對我們自己的身份.
我們對我們自己的身份.
作為神的創造人.
我之前與Lillian討論過.
古代教學的流行.
你會看到在外國的教育學院.
歐洲 英國 美國.
大多數的古代教學.
集中在古代教學或希臘教學.
作為文化藝術品.
它的接收.
古代教學的文化意義.
傳遞到西方世界.
西方文明和文化.
甚至在中國.
我所知道的.

$^{161}$很多古代教學的研究.
都是以文學的名義.
因為這就是他們.
如何能夠存活在中國的政權.
我希望這番話.
也與Dr. Wang的講座中心合作.
就是為了顯示古代教學.
是對新教會非常重要的.
新教會.
香港的基督教社會.
在評論如何.
亞伯拉罕和以色列的故事.
在香港的基督教身份上.
閃耀了我們的基督教身份.
我希望不只專注於.
他們的總體代表.
作為一個信仰的社會.
而是他們的特定地點.
作為一個在亞伯拉罕的故事中.
和以色列的社會.
作為一個共同體.
Dr. Rice講述了.
亞伯拉罕的信仰.
使他面對.
我提到的.
早期的文章.
或早期的文章.
"賦予兒子的最高任務".
但仍然有其他測試.
亞裔的解釋者.
在亞伯拉罕的十個試驗中.
可能對我們香港人的基督教身份.
更加驚訝.
同樣地.
Dr. Rice也詳細地談到.
以色列的故事的爭端.
會更加與香港的基督教和教會相關.
我從收到的文章中提到.
他們內部失敗.
而外部受到攻擊.

$^{201}$這聲響了一聲響鐘.
我希望如果有可能.
Dr. Rice可以多多回應.
亞伯拉罕的身份危機.
記得他對他的妻子的處境.
以及他對他身份危機的思考.
在這個昭和時期.
古代文章的內容.
是非常具有詮釋性的.
但坦白說.
我不能判斷一位講者.
他所說的.
但如果古代文章.
對香港基督教社會的身份.
有幫助.
我認為亞伯拉罕和以色列人.
解決他們的身份危機.
對我們觀眾來說.
會更加吸引.
我想我時間差不多了.
我不會再提到最後一部分.
讓更多的時間給倫敦人.
但最後一句話.
特別是用古代的.
說法和說法語言.
幫助新教會.
或香港的教會.
表達自己的身份.
這是非常實際.
也非常重要.
因為有時在香港的教會.
是非常不可想像的.
有些說法可能在我們語言中缺乏.
這些古代的說法.
是我們必須改變.
或重新形容.
我們在現在面對的.
不可思議的時期.
的教會身份和任務.
謝謝.

$^{241}$(掌聲).
(英文).
謝謝Dr. Wright給我們帶來.
一個有意義的講座.
並展示如何回答.
我們是誰.
或更加具體地.
我們是誰.
在古代的教會.
其中四個講座.
關於基督的身份.
基督的任務.
基督的道德和希望.
Dr. Wright選擇了.
從我們的身份開始.
不可思議地.
我們是誰.
是最基本的.
在問我們的問題.
我們的任務是什麼.
或我們的任務是什麼.
我們無法避免.
先要定義我們的身份.
透過展示兩個關鍵的想法.
我們是神的故事中.
創造出來的人.
我們是神的榮耀.
Dr. Wright強調.
我們的身份.
以及教會的身份.
與古代的歷史.
有著深刻的連結.
在第一部分.
Dr. Wright顯示.
古代歷史的不同部分.
包括創造.
歷史的出現.
以及以色列的故事.
都涉及到.
我們作為教會的身份發展.

$^{281}$尤其是以色列的故事.
選擇.
復仇.
和神的同盟.
以及戰爭.
我們都被賦予了.
信仰的共同體的真實性.
在第二部分.
我們顯示.
一些形象和說法.
在兩本福音書中使用.
如神的民族.
神的家庭.
和神的寶藏.
可以展示基督教的身份.
並且證明.
兩本福音書的.
有機的連結.
當然.
Dr. Wright.
也將教會的身份.
展現得更加穩定.
但儘管兩本福音書.
有著明顯的連結.
有些人仍然會質問.
真正的缺點在於.
在我們認為的新福音書中.
我們對於舊福音書的故事.
對於主義的想法.
如選擇.
復仇.
和神的同盟.
在兩本福音書中都顯現了.
換句話說.
我們對於基督教的身份.
會否被誤解.
如果我們不認同.
舊福音書的文字.
因為基督教人們同意.
基督教人們同意.

\newpage



\section{}
\label{sec:mzg9S8Q9BQ8}
\textbf{The Old Testament and Christian Identity: Who are we? (1) — Q & A Session}
\newline
\newline
連結: \href{https://youtube.com/watch?v=mzg9S8Q9BQ8}{\texttt{ https://youtube.com/watch?v=mzg9S8Q9BQ8}} ~~~~ 語音日期: 2023-02-15 
\newline
\newline
\hyperref[sec:LWCkMGG0tpo]{\small{< < < PREV SERMON < < <}}
~
\hyperref[sec:index]{\small{[返主目錄]}}
~
\hyperref[sec:_96yWYR16Vo]{\small{> > > NEXT SERMON > > >}}
\newline
\newline
$^{1}$[音樂].
所以我們來看看有沒有人想要先問一個問題.
謝謝Dr. Wright和兩位採訪者.
Dr. Wright您提到我們在基督教教會中承受著亞伯拉罕的身份.
您還提到我們是一個祝福社區.
我們也是伊拉克的社區.
這是我認為古代教學研究所一直在爭論的一個重點.
選舉和非選舉的差別.
尤其是在獲勝的故事中.
選擇的亞伯拉罕被命令要除去所有其他種族.
所以這種暴力是有合理的理由.
但同時我們也選擇了為多國帶來祝福.
所以這兩種身份之間的關係.
Dr. Wright您能否解釋一下這個問題.
謝謝我會嘗試.
這是直播嗎?.
是的.
謝謝.
這是一個常見的問題.
所以這並不意外.
我先把這個拿走.
我寫過一本書叫做《我不明白的神》.
在這本書中.
有幾章叫做《你問的是什麼?》.
這就是您所要問的問題.
我想這本書有幾個角度可以從中學習.
首先我想說的是.
選舉和非選舉的差別.
是一種很系統的理論.
我認為這一點並不在古代教學中.
因為他們被叫做「阿伯漢」.
但這只是為了其他國家的好處.
所以這並不像是說.
以色列的選擇是對其他國家的拒絕.
而是為了祝福他們.
但這仍然是您的問題.
這是一個好問題.
因此,在長期的目的下.
神可以選擇祝福國家.
但同時也有權利.

$^{41}$在某些國家實行判斷.
我們需要了解.
這與我們在人類生活中所見的很相似.
作為父母.
我長期的希望.
是讓我的孩子們成為祝福.
我想他們成長.
我想他們成為其他人的祝福.
我想他們成長得健康.
我的長期目的是好事.
但這並不意味著.
我無法判斷他們犯錯.
他們不可能想要祝福.
而在不遵守規則時.
實行判斷和指導.
現在,我們需要了解.
為何神明明確說.
以色列人會把肯尼亞人.
和共和黨帶走.
是因為那一整個文化.
那一整個國家.
在神的前面變得如此惡毒.
他們現在有權判斷.
他們在亞伯拉罕時期並沒有.
所以神說.
亞伯拉罕,我會把你們帶進這個土地.
但現在不可以.
因為亞馬遜的非正義還沒有完全滿足.
這是在基督教15世紀.
換句話說.
神明明確說.
他們還不夠惡毒.
讓我來毀滅他們.
但那一天將來.
所以我們需要視此為一項判斷.
這將是第一點.
第二點是.
神警告了以色列人.
告訴他們.
如果你們走肯尼亞人的路.

$^{81}$我會跟你們一樣.
我會把你們帶走.
所以,只是神的選民.
並沒有免了他們神的判斷.
事實上,神對他們說.
在亞馬遜和亞馬遜9年.
你們是我所有的國家中.
我所認識的或選的.
你們是我的選民.
因此我會判斷你們.
不是說.
因此你們永遠都會好.
和所有人都一樣.
選民也有受到神的判斷.
他們也有.
你們也知道.
因為神把他們引入了死亡.
我認為最後一點.
我們不應該說.
神總是反對以色列.
和反對外國人.
因為《約翰書》.
反映了這一點.
有趣的是.
《約翰書》.
在後來的《聖經》中.
提到神駕駛了.
堪薩斯人.
我們在《約翰書》中.
遇到的首個堪薩斯人.
是一個復原的.
被轉變和救了的人.
那就是雷哈.
後來的一章.
我們有一個以色列人.
雖然他是選民之一.
但他卻不守護神.
把自己和家人都放在.
神的判斷和憾怒之下.
他被毀滅了.

$^{121}$所以你會有一個外國人.
進入了救贖和祝福.
你也會有一個內地人.
體驗了神的憾怒.
所以《約翰書》.
在某種意義上.
是在反映了.
我們是正義的.
我們是選民的.
它在說.
不不不.
在這裡的信仰.
是神的信仰.
所以.
最後我會說.
我們必須看到.
這是一個故事的一部分.
最後帶領我們.
到耶穌的跨越和復活.
這並不是一個.
神的民眾.
他們不想要.
毀滅自己的敵人.
可惜的是.
在基督教歷史中.
這種方式已經被使用了.
在北美.
在澳洲.
在南非.
在其他地方.
人們覺得.
我們是耶穌的民眾.
而這些原住民.
就是基督教的民眾.
他們只是在毀滅他們.
我覺得這是一個完全錯誤的.
和諧的文章.
所以這是一個.
很深奧的回答.
這也是我教學生.

$^{161}$在我的課堂中的一部分.
我回應我的回應.
《喬治亞》.
《喬治亞》的書籍.
有一個很長的歷史.
歷史的接待.
是關於如何處理.
喬治亞書中的.
基督教問題的問題.
我們會談論.
讀書的道德.
因為在後期.
特別是在北美.
他們駕駛印度人.
他們用《喬治亞》書籍.
作為測試.
來證明他們的行為.
這是一個非常危險的行為.
所以我們會談論.
讀書的道德.
我們有責任.
在測試中讀書.
所以我們必須意識到.
選舉並不是綠色的.
選舉是代表更多的責任.
這是我老測試老師.
史蒂芬尼.
叫我們做的.
所以我們讀測試.
我們必須非常小心.
不要將測試放到自己的口袋裡.
並且侵犯它.
這可能會對我們作出傷害.
作為人類.
作為讀者.
作為一名總理.
作為一名讀寫文章的讀者.
我回應一下.
這是很重要的一點.
我們並不是用它.

$^{201}$作為一個具體的測試.
讓社會有所應對.
其實對於普通人來說.
這也是很重要的.
所以我們在批評.
我們讀書的方式.
而不是在批評測試.
這是一個很好的方式.
回應這些問題.
是的,我認為.
我喜歡文章.
由羅生先生寫的.
是關於選舉.
而不是關於自由.
而是關於責任.
我覺得這對我來說.
是非常有意義的.
我認為今天教會也一樣.
我們被選為教會.
並不是為了自由.
而是為了責任.
為我們服務社會.
為我們服務地球.
為我們服務社會.
而不是為了自由.
我認為自己比其他人優勝.
但這並非如此.
我們必須更謙虛.
就像古代的教會.
我們必須更謙虛.
我認為我們必須更謙虛.
非常感謝Dr. Wright.
帶來這次的講座.
我認為感謝您.
為了選擇身份.
作為第一次進入這個系列.
因為現在人們.
正在掙扎身份.
我認為因為.
有太多個身份.

$^{241}$我們不知道.
哪個身份可以承擔.
您的講座其實提到一個很重要的問題.
但我們看到.
就算是基督徒也有身份的危機.
所以說.
基督徒應該是你的.
主要身份.
並不代表你的問題.
有時我們的想像.
在古代.
事情是更簡單的.
你的身份是神的人.
你的文化,宗教.
所有的東西.
都屬於一個身份.
而你是一個靈.
所以您能否解釋一下.
這是否真正的事實.
或是對於.
現代人.
我們正在掙扎的多個身份.
如何基督徒.
能夠幫助我們.
或提醒我們.
是的,他們正在回應您的回應.
我非常欣賞您的回應.
因為我認為.
身份在香港.
這明顯是一個重要的問題.
因為在世界上.
你屬於哪個地方.
我非常欣賞您的回應.
我來自北海.
您可能猜到.
我會說中文.
我們和北海人.
爭論的是.
我們是俄羅斯人.
還是英國人.

$^{281}$我當然不是英文.
美國人很難理解.
就像你和我一樣.
你會說我是德國人.
但我不是英文.
我是俄羅斯人.
我支持俄羅斯人.
打英國人的籃球.
我完全不在乎.
誰打贏英國人.
在籃球或柒球或其他領域.
我的身份.
也有這種複雜的性質.
事實上,我來香港.
我帶著我的俄羅斯護照.
我現在在展示我的證據.
當英國人笨蛋.
決定離開歐洲.
有很多人說.
這是一個蠢的想法.
但我還沒有在意大利出生.
我現在可以拿到俄羅斯護照.
我現在有了.
我現在有俄羅斯護照.
我非常高興.
回到問題.
我認為從聖經的角度.
不僅是新教.
也有老教.
身份.
是神.
是國家的創造.
意味著.
你的基督性身份.
是.
亞維的.
是第一.
但它並不消除.
它並不決定你.
是一種族裔.

$^{321}$或者另一種族裔.
在以色列的情況下.
是族裔.
但雷哈布成為了.
神的一部分.
她是一個基督教人.
謝謝.
現在是七點三十分,謝謝.
謝謝.
\newpage



\section{}
\label{sec:_96yWYR16Vo}
\textbf{The Old Testament and Christian Identity: Who are we? (1) — Speaker: Dr Christopher Wright}
\newline
\newline
連結: \href{https://youtube.com/watch?v=_96yWYR16Vo}{\texttt{ https://youtube.com/watch?v=\_96yWYR16Vo}} ~~~~ 語音日期: 2023-03-08 
\newline
\newline
\hyperref[sec:mzg9S8Q9BQ8]{\small{< < < PREV SERMON < < <}}
~
\hyperref[sec:index]{\small{[返主目錄]}}
~
\hyperref[sec:XyScbip7koI]{\small{> > > NEXT SERMON > > >}}
\newline
\newline
$^{1}$[音樂].
Good afternoon..
Honorable guests, speakers,.
CGST family, brothers and sisters..
It is my pleasure to welcome everyone to the.
CGST Josephine Soh Culture and Ethics Lecture Week..
The CGST Josephine Soh Culture and Ethics Lectures.
are sponsored by the Josephine Soh Foundation..
Every two years, we invite overseas scholars.
and church leaders to speak on issues.
regarding Christianity, culture and ethics.
that are of concern to Chinese believers..
This year, we have the honor and privilege.
in welcoming Dr. Christopher Wright,.
global ambassador and ministry director.
of the Land-Game Partnership..
He will be our keynote speaker.
for all four sessions of the lectures..
We are grateful that Dr. Wright is present in person.
to speak to us, which we know.
is no longer given during COVID times..
The theme for his lecture this week is.
"Written for Us, the Old Testament.
as Christian Scripture.".
In this afternoon's first lecture,.
Dr. Wright will talk about.
the Old Testament and Christian identity..
Who are we?.
After his speech, I will introduce.
our two correspondents.
who will give their responses.
followed by time for Q and A..
Now, please join us in welcoming Dr. Christopher Wright..
(掌声).
謝謝.
非常感謝,李教授.
謝謝你欢迎我來到CGST.
很高兴能再次在這里.
我記得在过去几次.
站在這個台上.

$^{41}$享受了一些教育和演讲.
所以很高兴能再次來到香港.
感謝神让我能在這些日子.
能夠做到這些事情.
当然,我也很榮幸.
能在我住在的英國城市.
London,英國.
不是在澳大利亚,是在英國.
我住在的教堂,All Souls Church, Langham Place.
Langham Foundation在香港取名.
不是酒店或商店.
而是在John Stott的教堂.
Langham Church, All Souls Church, Langham Place.
我代表這些教堂.
向他們致辞.
当然,我也很榮幸.
能在這里接受這些演讲.
這些是Josephine So的演讲.
Simon Chung給了我一些背景.
她是誰,她是多么的有影響力.
她也是Breakthrough Centre的创始人.
我也曾经在过去几年.
與她合作过.
我們现在來到.
就像李教授所說的.
我們的名字.
《古典是基督教》.
有些人可能知道.
這是我的專业.
古典學习.
特別是在基督教价值.
和任务上.
我會在演讲中读.
因為我认為這是我能夠.
保持話题和時间的最佳方式.
所以我的榮譽回答员.
不會聽到.
他們已经读过的东西.
我相信這是他們的安慰.
所以這里是我們的第一個题目.

$^{81}$《古典和基督的身份》.
我們是誰.
你看,问题是.
這些日子,很遗憾地.
很多基督徒都在想.
古典是否对他們自己.
有任何的關系.
甚至如果他們承认.
古典是基督教的一部分.
他們应该认真考虑.
他們在想.
它和他們认為.
是基督教的信仰有關.
古典是否和.
基督教有關.
這不僅是伤心.
也很令人愤怒.
因為把问题放到那里.
是反过來的.
对于最早的.
耶穌的信徒面对的问题.
他們的问题不是.
如今天很多人所說.
古典和基督教有關.
而是他們在反过來的问题.
面对的问题.
在他們的信仰和诉求上.
就是我們建立的基督教.
是否和基督教有關.
也就是說.
他們并沒有判断.
基督教的宗旨.
他們在我們现在称之為.
古典的信仰.
是否和基督教有關.
而是反过來的.
他們在我們现在称之為.
古典的信仰.
面对的问题.
他們的信仰.

$^{121}$尤其是在.
這一整個新的.
多族主義的真实性.
在與基督教人一起.
宣称自己是神的國家.
而沒有观察基督的身份.
例如約束.
或食物法的观察.
你怎么能夠证明這些.
在信仰上.
在其他話里.
他也在與基督教人.
與羅马人的读者.
與其他人的读者.
這就是我們自己.
对自己的信仰.
而神也需要.
被這些詩词.
強调的.
這些詩词是我們的詩词.
当然不是写給我們的.
但就像是莫非說的.
是写給我們的.
因為古代的詩词.
是如此的伟大.
有很多方法可以解決.
我想要把這一個主题.
誰是我們.
我們作為神的民族.
我們的身份.
我們的任务.
我們的价值.
我們的希望.
在這四個方面.
我想让我們看到.
新教會在.
古代的詩词上.
強调了這四個主题.
所以這里是第一個.
古代的詩词.

$^{161}$和我們的身份.
我們作為神的民族.
我們的价值.
我想在這兩個方面.
在這兩個時刻中.
先說我們是.
神的故事.
形成的民族.
第二.
我們是神的榮耀.
所创造的民族.
首先.
我們是神的故事.
第二.
我們是神的榮耀.
我們想象了這兩個.
這是第一個.
我們是神的故事.
形成的民族.
神的故事.
开始在一开始.
神创造了天空和地球.
所以我們所在的.
整個宇宙.
是神创造的.
一位生命的.
個性的神.
祂创造了我們.
人类在祂的图像中.
以治理地球.
在基督之临之時,是神以主的族人.
在基督之临之時,是神以主的主人.
在基督之临之時,是神以主的人.
所以,那一段早期的故事.
在第十章第十章的結局.
在基督之临之時.
在使人們不再沮丧祂的目的.
就是要撒云添雾地撒地.
這在基督之临之時是這樣的.
我們想要阻止這些.

$^{201}$基督之临之時的這种.
勇敢的同谈语言的一致性.
神會把人类语言混乱.
不僅會把國家混乱.
也會把他們混乱到.
对方的分裂和误解.
所以,在這段基督之临之時的.
結局的一部分.
我們會發现人类被分裂.
被分裂.
被沦落在被诽谤的地球上.
所以,人类还有希望嗎?.
但是,故事还在继续.
传說中的传說.
從创造到降落.
到在歷史中的赎罪.
這就是新的创造的最終結局.
這就是我們從基督之临之時.
從基督之临之時的.
整個故事的結局.
從基督之临之時到.
新的创造.
這当然是基督之临之時.
和耶穌基督的降落.
這将是基督之临之時.
重新继续传承新的创造.
這段故事的好处是.
它牵涉到人类.
我們是人.
在這世界上.
在基督之临之時.
這世界上的國家.
是团結在爭辩中.
但在混乱中相互相互相互.
而在《約翰·約約》第七章.
他們會聚集在.
一個大群体.
每個國家,族群,语言.
都不可计数.
但他們會聚集在一体.

$^{241}$來祈求神的位置.
神的神父.
在《約翰·約約》第七章.
這幅画像.
是教會的幅画.
是我們.
神的救世主的多元社會.
這就帶你回到.
神最初对亚倫的承诺.
祂說.
所有的國家.
都會受祂的祝福.
我們就是那些人.
在這故事中.
被创造.
這故事是神的.
祂的救世主.
祂的天堂.
祂的创造.
祂的所有的國家.
都會受祂的祝福.
這就是我們.
我們是神的救世主.
在這故事中.
我們來看這一点.
從兩個角度來看.
我們是.
我們學到的.
耶穌的社會.
在《約翰·約約》第七章.
祂的祝福.
祂的所有的國家.
祂的子孫.
然后.
從耶穌的社會.
我們學到的.
是他們的故事.
這就幫助我們了解.
我們的故事.
那么我們可以從耶穌的声音.

$^{281}$和早期的故事.
學到的.
四個东西.
我們是.
信仰和希望.
我們看到的這四個东西.
在阿伯罕的故事中.
他們也是.
阿伯罕的民族.
阿伯罕的树.
就像我們所說的.
首先是祝福的社會.
祝福是.
耶穌的第一個言论.
祂的创造.
除了祂的创造.
大家記得在一世之后.
祂成功地.
在地球上成為了神.
然后在洪水之后.
神祝福了诺亚.
祝福了地球上所有的生命.
所以祝福的意思是.
祝福了诺亚的成就.
但是.
大家也知道.
這次的犯罪和失敗.
在下一個章节中.
似乎是在重复了.
第三章的說法.
在地球上的神的诅咒.
在神的诅咒下.
祂的诅咒.
甚至诺亚的父亲也擔心.
祂會把诅咒.
從地球上扛起.
所以在阿伯罕.
当祝福的语言.
回到天上.
然后神說.

$^{321}$祝福你.
然后你也會成為祝福.
然后所有的國家也會受到祝福.
在這章节中.
這就是一种新的开始.
但是我想強调的一点是.
祝福是受到的.
但是通过你.
所有的國家也會受到祝福.
也會成為祝福.
所以這里有一個社群.
神的民族.
他們都体验了神的祝福.
也被传承給了其他人.
受到祝福.
和分享祝福.
是神的民族.
這就是神的民族的真实性.
如果我們要生活在神的故事中.
那就是.
如果我們要生活在神的故事中.
那就是.
如果我們要生活在神的故事中.
那就是.
如果我們要生活在神的故事中.
那就是.
如果我們要生活在神的故事中.
那就是.
如果我們要生活在神的故事中.
那就是.
如果我們要生活在神的故事中.
那就是.
如果我們要生活在神的故事中.
那就是.
如果我們要生活在神的故事中.
那就是.
如果我們要生活在神的故事中.
那就是.
如果我們要生活在神的故事中.
那就是.

$^{361}$如果我們要生活在神的故事中.
那就是.
如果我們要生活在神的故事中.
那就是.
如果我們要生活在神的故事中.
那就是.
如果我們要生活在神的故事中.
那就是.
所以,那些來自于阿伯的社區.
必须是一個會相信神的承诺的人.
而不是相信他們自己的能力.
去建造他們自己的安全.
這就是他們在巴勒斯坦尼斯的想法.
在這里我們會看到一些人.
想要自己建造自己的城堡.
而神說,我會祝福你.
如果你相信我.
你将會受到祝福.
所以,阿伯是一個信仰的人.
他相信神的承诺.
這也是一個原因.
為什么我們是最早的.
為了基督教人的名字.
為了為了基督教人的名字.
我們是信仰人.
我們是為了.
為了神的故事.
為了神的计划.
為了人类的计划.
這故事有未來.
未來的新的创造.
当我們进入這個故事時.
我們将信仰在故事中的神.
因為他已经证明了.
在耶穌基督的跨世和折磨中.
所以,這是一個信仰的共同传承.
第三,是一個信仰的奉行.
因為他的信仰,阿伯就奉行了神.
所以,在他的信仰中.
所以,這兩點相同的意義.

$^{401}$所以,如果我們想要考慮選舉.
作為我們的身份.
作為神的民眾.
首先,這完全是神的恩賜.
沒有任何東西.
讓我們覺得值得.
神選擇我們.
這是神的恩賜.
第二,這基本上是.
在目的上的任務.
也就是說,我們選擇.
被神選擇.
不僅僅是我們.
享受救恩.
我們也會成為神.
為別人帶來救恩的方式.
這就是彼得在第二章.
在第二章中強調的.
我們將在國際上.
在神的光榮之中.
看到一些東西.
然後來祝福神.
所以,教會就像古代的.
耶穌基督所做的.
祈求神的任務.
使國際社會.
從他們在.
基督的反抗.
在基督的反抗.
在基督的反抗中.
從他們的反抗.
在基督的反抗.
在基督的反抗中.
在1 Corinthians 3.
他說「你」意思是你基督徒.
在《康倫斯》作為當地的信徒.
你就是神的寺廟.
因此他警告他們.
不要參與在偏僻的寺廟裡發生的事情.
因為我們是生命的神的寺廟.

$^{441}$然後他也用了.
在宇宙教的意義上.
在《伊斯蘭教》2:1-2.
對所有的基督徒說.
他們已經與.
基督徒和基督徒聯合起來.
並建立在一起.
他說,成為一個生命的寺廟.
讓神能夠被他的靈所所住.
所以在古代的寺廟的形象中.
他們不僅是一個教會.
他們是整個地球的一座教會.
是生命的神的教會.
祂的一座住所.
是祂的一種靈魂.
是所有國家人民的住所.
我不會繼續說到基督徒.
因為我會在下個講座中.
提到這些.
這裡提到神的民族.
以及以色列為神的基督徒.
也提到彼得的自己.
當然我們可以加上其他人.
我必須在這裡結束.
但是有許多其他的說法.
在古代的說法中.
以色列人用的.
在新教會中.
以色列人用的.
是一個葡萄.
耶穌用的.
一個橄欖樹.
Paul用的.
一個婦人.
它們都在各地.
還有一群牛.
以色列人為神的鷹.
所有這些.
以色列人的想法.
在古代的說法中.

$^{481}$都具有不同.
但是它們都能夠傳遞到新教會.
所以我希望.
至少現在.
我已經說了足夠的.
如果我們要回答.
基督徒的問題.
我們是誰.
我們無法不去.
承認古代的.
信仰和文化.
的豐富的.
道德和歷史真相.
謝謝.
字幕志願者:陳志全.
(CC字幕製作:貝爾).
\newpage



\section{}
\label{sec:XyScbip7koI}
\textbf{The Old Testament and Christian Mission: What are we here for? (2) — Q & A session}
\newline
\newline
連結: \href{https://youtube.com/watch?v=XyScbip7koI}{\texttt{ https://youtube.com/watch?v=XyScbip7koI}} ~~~~ 語音日期: 2023-03-11 
\newline
\newline
\hyperref[sec:_96yWYR16Vo]{\small{< < < PREV SERMON < < <}}
~
\hyperref[sec:index]{\small{[返主目錄]}}
~
\hyperref[sec:BCZSaeNGuKE]{\small{> > > NEXT SERMON > > >}}
\newline
\newline
$^{1}$(音樂).
我們歡迎大家前來.
但在此之前.
我想問問Dr.Wright.
會否有回應.
當然可以.
謝謝你們.
Dr.Johnson和Dr.Lawrence.
首先是謝謝你們的評論.
因為他們對於.
Abraham和Exodus的回應.
是非常有幫助的.
我想首先要說的是.
這是一堂簡短的課.
你無法詳細地解釋.
所有的詳細細節.
包括了我的書中的一些.
你也知道的.
但有幾點.
我會非常感謝你們.
首先.
我不是想說.
在《馬爾地亞人》28節.
Abraham的回應.
只是一個中間的文字.
我更想說的是.
《基督教》的文字.
是一條.
基督教的傳統路線.
一條思想的船.
就是神是關於.
祝福所有的國家.
它包括了一個必要的"go".
而Matthew是Port of Entry.
是他的同伴.
這並不是我所想說的.
所以我不想強調這個觀點.
或者強調這個觀點.
因為在他的意義上.
Abraham必須離開他的國家.

$^{41}$但那些基督教人還留在了他們的土地上.
所以當我們用"使命"這個詞.
我們必須認同.
我並不是在想使命.
而是在中間的意義上.
離開國家.
使命的目的.
是為了神而成.
他們是為了吸引國家.
而不是為了那些比較中性的.
所以這就是我所想的.
在《基督教》19條.
我想我會想問一點.
因為你說得對.
《1彼得》2條.
是加了一句.
"你可宣佈祂的譽言".
但是他在做什麼呢?.
我認為.
在那些詩.
彼得有了一整個的.
古代教的思想.
他有了《 Exodus 19》.
他有了《和尚》.
那另一個.
宣佈祂的譽言.
可能是一種反思.
在《以下》.
我讀到.
神說.
"你是我的使徒,我選擇了你".
"我建立了你,你可宣佈我的榮譽".
所以宣佈國家的榮譽.
是古代教的主題.
即使不是《 Exodus 19》.
我還會說.
以色列人宣佈國家的榮譽.
在《撒冪》中.
很常見.
在《撒冪96》中.

$^{81}$宣佈國家的榮譽.
宣佈神的榮譽.
當然.
以色列人並沒有被送到國家.
他們不需要到國家去.
但是.
有些評論家說.
耶路撒冷是一個.
宇宙化的城市.
國家一直在那裡.
自古以來.
你知道.
你有那些國家在耶路撒冷.
在那裡宣佈.
但是.
在那裡.
宣佈的.
在那裡.
是一個國家的榮譽.
所以.
我同意你.
《Exodus 19:46》.
並不是說了所有的.
我並不想.
把它當成.
神的使徒的簡單的類型.
但是我認為.
這是一部分.
基督教傳承的旅程.
最後帶領我們到那裡.
所以你說得對.
如果我想延續的話.
我必須說.
這不是一個文字的回響.
但是這裡有一個.
基督教傳承.
在這裡.
這樣合理嗎?.
所以你現在不太緊張了.
我不會在你眼中.

$^{121}$告訴Walter Mobley.
因為有趣的是.
Walter和我會有不同.
Walter Mobley是一個.
很著名的古代學家.
在德爾蒙.
在英國.
他比我更像是一個學者.
他和我討論了.
《基督教傳承》.
你也知道.
當它說.
所有的國家都會受到你的祝福.
《希伯來文》.
其實是一種非洲語.
意思是.
他們會祝福自己.
也就是說.
國家會用.
阿伯拉罕的名字.
作為祝福.
祝福自己.
以阿伯拉罕的名字.
而不是以.
他們會受到阿伯拉罕的祝福.
所以他沒有.
任何的使命意義.
所以我對Walter的報告.
並不太害怕.
你絕對對.
因為我的主管.
總是告訴我.
一種名字叫.
「復興教育」.
那一種名字.
會是古代的.
早期傳統的用法.
混合在一起.
進行不同的復興教育.
當傳統.

$^{161}$發展到全面.
那就是你在這裡.
所用的哲學主義.
會在新教會.
和整個教會.
中.
解釋出現的現象.
我認為.
我在這一點上.
和你在一起.
然後.
你用了「傳統」.
我用了「復興教育」.
那些都是同一種名字.
基本上.
非常棒.
好的.
抱歉.
我基本上同意了.
他所說的一切.
謝謝.
這裡一切都很好.
有任何問題嗎?.
你會想到.
我可以從這裡問一個問題.
你提到.
以色列為國家.
即使他們有很多的苦難.
他們不是完美的族群.
也不是他們的善良.
能夠反映神的榮耀.
但這就是神的任務.
那會有任何的影響嗎?.
你認為.
這會對我們.
現代基督教.
新西蘭基督教.
如何看待自己.
作為神的御所.
因為很多時候.

$^{201}$我們覺得.
我們已經有了一切.
我們是從.
光的土地.
帶給人們光明.
在黑暗中.
你能否延續.
那些觀察和光明.
作為現代基督教的使者?.
謝謝.
我認為神給我們了.
古代的基督教.
正如Paul所說.
希望我們可以.
透過文字來得到希望.
很有趣.
就如同羅馬人.
在說給基督教人.
說到結局.
所有的文字.
都是為了讓你們有希望.
所以有一個.
有一個正面的未來.
但他在基督教中.
說的另一件事是.
這些東西.
是為了警告而寫的.
我們從.
以色列的故事來學習.
什麼是.
挑釁.
和瓦礫.
和偽造.
和錯誤.
是神的民眾所做的.
以致.
在耶穌基督中.
他想要.
將它們全部毀滅.
在耶穌基督32.

$^{241}$在寄宿中.
他把它們.
從他的土地中帶走.
只有.
由神的奇妙恩典.
使得以色列的生命繼續.
所以.
古代的基督教.
知道.
這個故事是我們的故事.
它提醒我們.
我們.
我們是.
和任何國家都一樣.
在地球上.
是神的罪人.
我們只是.
神的民眾.
因為神的恩典.
和神的原諒.
並通過耶穌基督的聖體.
所以.
古代的基督教.
有著深刻的.
慰藉效果.
我們覺得.
為什麼.
以色列人.
總是要去追求.
譬如說.
我們讀到.
法官.
我們讀到.
先是前聖經的.
這些人.
總是在犯偽造.
什麼.
什麼都在他們身上.
為什麼.
我們在香港.

$^{281}$從星期天到星期日.
是怎麼生活的.
你知道.
我意思是.
比爾是在西方世界.
和資本主義世界.
生活得很好.
我意思是.
比爾.
比爾是錢的神.
他是性的神.
他是生育的神.
他是土地的神.
他是商業的神.
他是一切都很重要的神.
亞威是一個很棒的神.
因為他贏得戰爭.
他贏得戰爭.
你能在聖誕節當作是一個休息一天,你可以在三天的假期當作是一年假期,如果你做了耶穌教,就算是三個星期假期,但是在每天生活當中,你還是只能依照基督的教條.
這正是教會在過去的多世紀都顯示出的神經的方式,我們同樣依照基督的教條,我們堅持依照耶穌,但卻沉浸在世界各地,卻又看起來不太不同.
所以,最後,神必須帶來判斷,我個人的信念是,在五百年來,西方世界在歐洲的發展,自歐洲決定將其轉移到其他國家,西方世界基本上是受到神的判斷,神說,我已經有足夠的人,現在又有其他事情發生了.
而古代教會顯示出,在神的計劃和目的之下,帝國的領袖也在這個故事中出現了一些出現,也體驗了一些成功和成長的時期,也體驗了一些受判斷和失敗的時期.
我們必須謙虛地認可這些時期,當丹尼爾在丹尼爾9年認可,當時的時期是神的判斷,在巴布林,我們在巴布林仍然生活,他並沒有舉行一個祭祀,並說,很棒,巴布林快要結束了,不,他坐在自己的腿上,並向他人道歉,他向一群懺悔的人道歉,神給了他們承諾,在聖經中的29:11,我知道我對你們有的計劃,計劃是要求你們認可這些計劃,但我相信你們也有自己的計劃.
計劃是要求你們成功,計劃是要求你們受到祝福,這些計劃是神給了他們的很驚訝的承諾,他們受到判斷的那些人,所以我認為在現代教會,我們需要利用古代教會作為一個方式來理解和探索和揭露那些教會仍然存在著的偶像,我認為西方基督教是信仰中最合法的一種形式,所以我們不能真的坐在古代教會的耶穌基督的判斷上,.
除非我們在自己的偶像中指出了巨大的手指,這是一個警告,但也是一個希望,因為這也提醒我們,神永遠會在祂的民眾身上,神將會達成祂的目的,.
而教會也在兩千年來都繼續在那裡,即使在罪惡和判斷和偽善之間,所以這就是我認為古代教會要說的一些東西..
所以目標和訊息不僅是對於非主義者,而是對我們也有很大的影響..
當然,我的天啊,大部分古代教會都寫給了神的民眾,您提到的那些反對國家的文章,那些都是很好的章節,但是我們要記住,古代教會的真正問題不在於外國國家,他們還沒有真正認識神,所以不可能他們是在慈愛偶像,他們被懲罰了對待貧窮人的那種態度,他們的不公平,他們的壓迫,他們的暴力,他們並沒有被懲罰了對待偽主義者,.
這就像是,當然,他們是在懲罰偽主義者,但是當神的民眾,當他們認識了真正的神,然後他們轉換了真正的神為他們的偶像,這就是神的父親..
所以,古代教會的真正問題是在於自己的民眾,所以對我們作為基督徒來說,當然,我們擔心這個世界,我們為了壓迫和暴力而戰鬥,.
但是我認為神會說,你呢?你對教會有什麼看法?你現在在哪裡生活?我認為古代教會的真正問題就在於這一點..
有任何問題嗎?.
首先,我要感謝你們三位,你們三位一起帶給我們一個非常深奧的精神上的運動,我非常享受..
我有兩個問題,首先是關於克里斯,在您的介紹中,您提到您帶給我們這兩個非常重要的概念,基督的任務,基督的身份..
我想要說明一下您的意思.基督的身份是否代表基督的任務,還是基督的任務是否定義基督的身份?.
我明白您的意思,但是我認為這問題是不可回答的,如果我們想要說明基督的真正真正的真相,我們必須說明我們是基督所創造的,為了祂的榮耀,為了祂的愛,為了祂的服務,為了祂的享受,我們將永遠都做到..
即使是新的創造,我們也會服務和愛祂.所以在新的創造中,我們可以說,我們不會有任何任何的任務,在施行救贖任務的意義上,我們不需要再教會了..
因為你們的律師會有相對較為薄的選擇在新的創造中.所以在這個意義上,我們的永遠的命運就是我們的身份,就是基督的妻子,在新的創造中,我們是那位偉大的宇宙性的天使的婚姻,是由神和祂的子女和祂的民眾..
但是因為在新的創造中,我們有著永遠的命運,我們仍然活在最終的時刻,我們仍然活在這個歷史上的破壞,已經被神復活,已經被基督基督所賦予的..
因為我們是那些民眾,我們現在有一個理由存在,所以我們的歷史任務來自我們的身份,就是我們所謂的民族..

$^{321}$但是我也想說,他們是不可分開的.也就是說,你不能說自己是神的民族,除非你其實是為了神的目的而活在當下..
所以我所說的就是,你不能說什麼是為了什麼,而不說什麼時鐘..
我不是說你能夠定義神的民族,也不能說我們是為了什麼..
我們現在為了的,是為了服務和榮耀神.我們現在為了服務神的任務,就是為了帶領國民到祂,帶領宗教到地球,並傳承到創造..
所以這對你的問題有所幫助嗎?你剛才說你有兩個問題..
第二個問題是關於練習,我教政治理論,所以要談練習..
你知道我喜歡教會,我喜歡教會我的教會成員,我真的很想要把我的基督徒教育發揚出去..
但是我想像一下,可能在幾年時間內,我會變成有疾病,我可能明天會有心臟病..
我會失去我的說話能力,我不會教會,我不會向我的教會成員發聲,我不會祈禱給他們..
所以,這是一個簡單的問題..
讓我來結束.抱歉..
第二點更糟糕的是,你剛才說的,我正在想起我的兄弟..
我的兄弟幾歲比我大,他生病了,心理上是失敗的..
他一生都無法接受基督教教訓,無法做任何見證,服侍,無論如何..
現在他也要依靠自己的基督教教訓..
所以,我在想我的兄弟,一個這樣的人,他有什麼基督教的使命?.
非常深奧,非常有幫助,謝謝你..
我對你的兄弟有些擔憂..
我有幾種方法想要解釋..
第一,我回到我剛才講的課堂,我不知道是否是你,.
但有人對我說,你在說的是我們,而不是我..
我認為,如果是我們,我們可能會遇到意外,.
無法做任何事情,或是你的兄弟一生都這樣,.
問題在於這個人的個人身份,.
這個人是否能夠成為基督教的形象,.
這是基督教的愛,而在這個地球上,.
基督有一個目的,即使我們無法看到..
這就是一個個人的目的,.
而我正想談論的是,我們作為一個社群的任務,.
我們為何能夠存在為基督的民族..
我認為我們可以正確地分辨這兩點,.
因為我自己也有同樣的問題,你也說得很好,.
但有時候,當我思考自己,.
我會問自己,如果我失去能力去做所有讓我感到值得的事情,.
例如寫作,說話,有症狀,等等,.
我會否仍然覺得自己是一個重要的人,.
在神的計劃和目的之中,我會否有能力了解神,.
神愛我,神珍惜我,我對我的生活有一定意義,.
不僅是因為我寫過書,但也因為我是我自己..
我希望我能夠回答這個問題,.

$^{361}$神的任務不僅是關於聰明的人做聰明的事情,.
因為他們很擅長做任務,.
神的任務在一個破碎的世界中,.
也包括了破碎的人,.
而神教的美麗和榮耀,.
在古代的歷史上,.
也包括了基督教的美麗和榮耀,.
它們是地球上的一個社群,.
它們是社會不敬的,.
不知不覺的,盲的,孩子的,胎兒的,.
從最早的時期,.
基督教人在羅馬帝國中成為了知名的,.
因為他們在瘟疫中照顧了孩子,.
他們拯救了遺棄的孩子,.
女性的孩子被扔進垃圾桶,.
因為有關基督教社群的一些事情,.
是關於人類的生命,.
是以神的形象來做的..
因此,有關教會的一些事情,.
是關於教會,.
它們一直都包括了這些人,.
我認為這是一部分的教會..
我提到這一點,.
也是因為古代的文章,.
其實是在《德里格斯19》中.
指出了盲人和不知不覺的溫和,.
並且關注了孩子,.
孩子的遺棄,.
所以我想回答你的問題,.
就是在說我們的身份和任務是一體的時期,.
我只是在說教會,.
並不是說只有能夠「做任務的人」.
對神的價值是有價值的,.
這會是錯誤的,.
因為我們都是以神的形象來做的,.
他愛每一個人,.
而所有被基督救贖的孩子,.
都是他的孩子,.
他愛他們,.
他有一個理由讓他們在這個地球上存在..

$^{401}$我不知道這回答了你的問題,.
但目前為止,這是我能做的最好的事情..
(英文).
所以我浪費了15分鐘,.
您剛才在Q and A中提到.
有兩個不同的方式去做任務,.
您所說的,.
在古代和新教會,.
任務是神的一部分,.
但在古代,.
您提到的,.
做任務的方式是在以色列作為目標,.
作為一個好國家,.
吸引人們來以色列,.
但在新教會,.
這實際上是在走出了..
我認為這兩個方式的任務.
其實帶來非常不同的結果,.
相比以色列在歷史和在教會的情況下..
所以我想知道,.
在兩國之間,.
這兩個方式是否如此的獨特?.
還有一個問題是,.
這對我們來說是什麼?.
謝謝,.
我立即想說,.
這兩個方式並不像我們所想的那樣獨特或相反..
因為,當然,.
在古代,.
以色列被稱為神的民族,.
並且活在神的民族之中..
但是,有兩點,.
第一,.
神說這是為了國家的好,.
所以儘管是以色列,.
但這並不獨特..
例如,.
在聖經中,.
聖經中的寫意是,.
當外國人來到這個地方,.

$^{441}$他們會做任何外國人要求的事情..
為什麼?.
因為這樣,.
你的名字就會被稱為地球的盡頭..
所以,.
以色列人們對神的民族的慈悲,.
是為了讓神在各國之間被稱為..
所以,.
它是以色列人為主的,.
你明白我的意思嗎?.
這也在聖經中,.
以色列是那片光明的城市,.
那座山,.
那座彩虹,.
它會把國家畫出來,.
但是,.
神的名字會在地球的盡頭..
所以,.
在古代,.
以色列只是在那裡,.
但是,.
它在那裡為了所有人的好..
它就像是一個航班..
相同的,.
在新的誦經中,.
雖然,.
當然,.
我們被叫去國家,.
但是,.
我們被稱為國家,.
但是,.
我們被稱為國家,.
但是,.
我們被稱為國家,.
但是,.
我們被稱為國家,.
但是,.
我們被稱為國家,.
但是,.
我們被稱為國家,.

$^{481}$但是,.
我們被稱為國家,.
但是,.
我們被稱為國家,.
但是,.
我們被稱為國家,.
但是,.
我們被稱為國家,.
但是,.
我們被稱為國家,.
但是,.
我們被稱為國家,.
但是,.
我們被稱為國家,.
但是,.
我們被稱為國家,.
但是,.
我們被稱為國家,.
但是,.
我們被稱為國家,.
但是,.
我們被稱為國家,.
但是,.
我們被稱為國家,.
但是,.
我們被稱為國家,.
但是,.
我們被稱為國家,.
但是,.
我們被稱為國家,.
但是,.
我們被稱為國家,.
但是,.
我們被稱為國家,.
但是,.
我們被稱為國家,.
但是,.
我們被稱為國家,.
但是,.
我們被稱為國家,.

$^{521}$但是,.
我們被稱為國家,.
但是,.
我們被稱為國家,.
但是,.
我們被稱為國家,.
但是,.
我們被稱為國家,.
但是,.
我們被稱為國家,.
但是,.
我們被稱為國家,.
但是,.
我們被稱為國家,.
但是,.
我們被稱為國家,.
但是,.
我們被稱為國家,.
但是,.
我們被稱為國家,.
但是,.
我們被稱為國家,.
但是,.
我們被稱為國家,.
但是,.
我們被稱為國家,.
但是,.
我們被稱為國家,.
但是,.
我們被稱為國家,.
但是,.
我們被稱為國家,.
但是,.
我們被稱為國家,.
但是,.
我們被稱為國家,.
但是,.
我們被稱為國家,.
但是,.
我們被稱為國家,.

$^{561}$但是,.
我們被稱為國家,.
但是,.
我們被稱為國家,.
但是,.
我們被稱為國家,.
但是,.
我們被稱為國家,.
但是,.
我們被稱為國家,.
但是,.
我們被稱為國家,.
但是,.
我們被稱為國家,.
但是,.
我們被稱為國家,.
但是,.
我們被稱為國家,.
但是,.
我們被稱為國家,.
但是,.
我們被稱為國家,.
但是,.
我們被稱為國家,.
但是,.
我們被稱為國家,.
但是,.
我們被稱為國家,.
但是,.
我們被稱為國家,.
但是,.
我們被稱為國家,.
但是,.
我們被稱為國家,.
但是,.
我們被稱為國家,.
但是,.
我們被稱為國家,.
但是,.
我們被稱為國家,.

$^{601}$但是,.
我們被稱為國家,.
但是,.
我們被稱為國家,.
但是,.
我們被稱為國家,.
但是,.
我們被稱為國家,.
但是,.
我們被稱為國家,.
但是,.
我們被稱為國家,.
但是,.
我們被稱為國家,.
但是,.
我們被稱為國家,.
但是,.
我們被稱為國家,.
但是,.
我們被稱為國家,.
但是,.
我們被稱為國家,.
但是,.
我們被稱為國家,.
但是,.
我們被稱為國家,.
但是,.
我們被稱為國家,.
但是,.
我們被稱為國家,.
但是,.
我們被稱為國家,.
但是,.
我們被稱為國家,.
但是,.
我們被稱為國家,.
但是,.
我們被稱為國家,.
但是,.
我們被稱為國家,.

$^{641}$但是,.
我們被稱為國家,.
但是,.
我們被稱為國家,.
但是,.
我們被稱為國家,.
但是,.
我們被稱為國家,.
但是,.
我們被稱為國家,.
但是,.
我們被稱為國家,.
但是,.
我們被稱為國家,.
但是,.
我們被稱為國家,.
但是,.
我們被稱為國家,.
但是,.
我們被稱為國家,.
但是,.
我們被稱為國家,.
但是,.
我們被稱為國家,.
但是,.
我們被稱為國家,.
但是,.
我們被稱為國家,.
但是,.
我們被稱為國家,.
但是,.
我們被稱為國家,.
但是,.
我們被稱為國家,.
但是,.
我們被稱為國家,.
但是,.
我們被稱為國家,.
但是,.
我們被稱為國家,.

$^{681}$但是,.
我們被稱為國家,.
但是,.
我們被稱為國家,.
但是,.
我們被稱為國家,.
但是,.
我們被稱為國家,.
但是,.
我們被稱為國家,.
但是,.
我們被稱為國家,.
但是,.
我們被稱為國家,.
但是,.
我們被稱為國家,.
但是,.
我們被稱為國家,.
但是,.
我們被稱為國家,.
但是,.
我們被稱為國家,.
但是,.
我們被稱為國家,.
但是,.
我們被稱為國家,.
但是,.
我們被稱為國家,.
但是,.
我們被稱為國家,.
但是,.
我們被稱為國家,.
但是,.
我們被稱為國家,.
但是,.
我們被稱為國家,.
但是,.
我們被稱為國家,.
但是,.
我們被稱為國家,.

$^{721}$但是,.
我們被稱為國家,.
但是,.
我們被稱為國家,.
但是,.
我們被稱為國家,.
但是,.
我們被稱為國家,.
但是,.
我們被稱為國家,.
但是,.
我們被稱為國家,.
但是,.
我們被稱為國家,.
但是,.
我們被稱為國家,.
但是,.
我們被稱為國家,.
但是,.
我們被稱為國家,.
但是,.
我們被稱為國家,.
但是,.
我們被稱為國家,.
但是,.
我們被稱為國家,.
但是,.
我們被稱為國家,.
但是,.
我們被稱為國家,.
但是,.
我們被稱為國家,.
但是,.
我們被稱為國家,.
但是,.
我們被稱為國家,.
但是,.
我們被稱為國家,.
但是,.
我們被稱為國家,.

$^{761}$但是,.
我們被稱為國家,.
但是,.
我們被稱為國家,.
但是,.
我們被稱為國家,.
但是,.
我們被稱為國家,.
但是,.
我們被稱為國家,.
但是,.
我們被稱為國家,.
但是,.
我們被稱為國家,.
但是,.
我們被稱為國家,.
但是,.
我們被稱為國家,.
但是,.
我們被稱為國家,.
但是,.
我們被稱為國家,.
但是,.
我們被稱為國家,.
但是,.
我們被稱為國家,.
但是,.
我們被稱為國家,.
但是,.
我們被稱為國家,.
但是,.
我們被稱為國家,.
但是,.
我們被稱為國家,.
但是,.
我們被稱為國家,.
但是,.
我們被稱為國家,.
但是,.
我們被稱為國家,.

$^{801}$但是,.
我們被稱為國家,.
但是,.
我們被稱為國家,.
但是,.
我們被稱為國家,.
但是,.
我們被稱為國家,.
但是,.
我們被稱為國家,.
但是,.
我們被稱為國家,.
但是,.
我們被稱為國家,.
但是,.
我們被稱為國家,.
但是,.
我們被稱為國家,.
但是,.
我們被稱為國家,.
但是,.
我們被稱為國家,.
但是,.
我們被稱為國家,.
但是,.
我們被稱為國家,.
但是,.
我們被稱為國家,.
但是,.
我們被稱為國家,.
但是,.
我們被稱為國家,.
但是,.
我們被稱為國家,.
但是,.
我們被稱為國家,.
但是,.
我們被稱為國家,.
但是,.
我們被稱為國家,.

$^{841}$但是,.
我們被稱為國家,.
但是,.
我們被稱為國家,.
但是,.
我們被稱為國家,.
但是,.
我們被稱為國家,.
但是,.
我們被稱為國家,.
但是,.
我們被稱為國家,.
但是,.
我們被稱為國家,.
但是,.
我們被稱為國家,.
但是,.
我們被稱為國家,.
但是,.
我們被稱為國家,.
但是,.
我們被稱為國家,.
但是,.
我們被稱為國家,.
但是,.
我們被稱為國家,.
但是,.
我們被稱為國家,.
但是,.
我們被稱為國家,.
但是,.
我們被稱為國家,.
但是,.
我們被稱為國家,.
但是,.
我們被稱為國家,.
但是,.
我們被稱為國家,.
但是,.
我們被稱為國家,.

$^{881}$但是,.
我們被稱為國家,.
但是,.
我們被稱為國家,.
但是,.
我們被稱為國家,.
但是,.
我們被稱為國家,.
但是,.
我們被稱為國家,.
但是,.
我們被稱為國家,.
但是,.
我們被稱為國家,.
但是,.
我們被稱為國家,.
但是,.
我們被稱為國家,.
但是,.
我們被稱為國家,.
但是,.
我們被稱為國家,.
但是,.
我們被稱為國家,.
但是,.
我們被稱為國家,.
但是,.
我們被稱為國家,.
但是,.
我們被稱為國家,.
但是,.
我們被稱為國家,.
但是,.
我們被稱為國家,.
但是,.
我們被稱為國家,.
但是,.
我們被稱為國家,.
但是,.
我們被稱為國家,.

$^{921}$但是,.
我們被稱為國家,.
但是,.
我們被稱為國家,.
但是,.
我們被稱為國家,.
但是,.
我們被稱為國家,.
但是,.
我們被稱為國家,.
但是,.
我們被稱為國家,.
但是,.
我們被稱為國家,.
但是,.
我們被稱為國家,.
但是,.
我們被稱為國家,.
但是,.
我們被稱為國家,.
但是,.
我們被稱為國家,.
但是,.
我們被稱為國家,.
但是,.
我們被稱為國家,.
但是,.
我們被稱為國家,.
但是,.
我們被稱為國家,.
但是,.
我們被稱為國家,.
但是,.
我們被稱為國家,.
但是,.
我們被稱為國家,.
但是,.
我們被稱為國家,.
但是,.
我們被稱為國家,.

$^{961}$但是,.
我們被稱為國家,.
但是,.
我們被稱為國家,.
但是,.
我們被稱為國家,.
但是,.
我們被稱為國家,.
但是,.
我們被稱為國家,.
但是,.
我們被稱為國家,.
但是,.
我們被稱為國家,.
但是,.
我們被稱為國家,.
但是,.
我們被稱為國家,.
但是,.
我們被稱為國家,.
但是,.
我們被稱為國家,.
但是,.
我們被稱為國家,.
但是,.
我們被稱為國家,.
但是,.
我們被稱為國家,.
但是,.
我們被稱為國家,.
但是,.
我們被稱為國家,.
但是,.
我們被稱為國家,.
但是,.
我們被稱為國家,.
但是,.
我們被稱為國家,.
但是,.
我們被稱為國家,.

$^{1001}$但是,.
我們被稱為國家,.
但是,.
我們被稱為國家,.
但是,.
我們被稱為國家,.
但是,.
我們被稱為國家,.
但是,.
我們被稱為國家,.
但是,.
我們被稱為國家,.
但是,.
我們被稱為國家,.
但是,.
我們被稱為國家,.
但是,.
我們被稱為國家,.
但是,.
我們被稱為國家,.
但是,.
我們被稱為國家,.
但是,.
我們被稱為國家,.
但是,.
我們被稱為國家,.
但是,.
我們被稱為國家,.
但是,.
我們被稱為國家,.
但是,.
我們被稱為國家,.
但是,.
我們被稱為國家,.
但是,.
我們被稱為國家,.
但是,.
我們被稱為國家,.
但是,.
我們被稱為國家,.

$^{1041}$但是,.
我們被稱為國家,.
但是,.
我們被稱為國家,.
但是,.
我們被稱為國家,.
但是,.
我們被稱為國家,.
但是,.
我們被稱為國家,.
但是,.
我們被稱為國家,.
但是,.
我們被稱為國家,.
但是,.
我們被稱為國家,.
但是,.
我們被稱為國家,.
但是,.
我們被稱為國家,.
但是,.
我們被稱為國家,.
但是,.
我們被稱為國家,.
但是,.
我們被稱為國家,.
但是,.
我們被稱為國家,.
但是,.
我們被稱為國家,.
但是,.
我們被稱為國家,.
但是,.
我們被稱為國家,.
但是,.
我們被稱為國家,.
但是,.
我們被稱為國家,.
但是,.
我們被稱為國家,.

$^{1081}$但是,.
我們被稱為國家,.
但是,.
我們被稱為國家,.
但是,.
我們被稱為國家,.
但是,.
我們被稱為國家,.
但是,.
我們被稱為國家,.
但是,.
我們被稱為國家,.
但是,.
我們被稱為國家,.
但是,.
我們被稱為國家,.
但是,.
我們被稱為國家,.
但是,.
我們被稱為國家,.
但是,.
我們被稱為國家,.
但是,.
我們被稱為國家,.
但是,.
我們被稱為國家,.
但是,.
我們被稱為國家,.
但是,.
我們被稱為國家,.
但是,.
我們被稱為國家,.
但是,.
我們被稱為國家,.
但是,.
我們被稱為國家,.
但是,.
我們被稱為國家,.
但是,.
我們被稱為國家,.

$^{1121}$但是,.
我們被稱為國家,.
但是,.
我們被稱為國家,.
但是,.
我們被稱為國家,.
但是,.
我們被稱為國家,.
但是,.
我們被稱為國家,.
但是,.
我們被稱為國家,.
但是,.
我們被稱為國家,.
但是,.
我們被稱為國家,.
但是,.
我們被稱為國家,.
但是,.
我們被稱為國家,.
但是,.
我們被稱為國家,.
但是,.
我們被稱為國家,.
但是,.
我們被稱為國家,.
但是,.
我們被稱為國家,.
但是,.
我們被稱為國家,.
但是,.
我們被稱為國家,.
但是,.
我們被稱為國家,.
但是,.
我們被稱為國家,.
但是,.
我們被稱為國家,.
但是,.
我們被稱為國家,.

$^{1161}$但是,.
我們被稱為國家,.
但是,.
我們被稱為國家,.
但是,.
我們被稱為國家,.
但是,.
我們被稱為國家,.
但是,.
我們被稱為國家,.
但是,.
我們被稱為國家,.
但是,.
我們被稱為國家,.
但是,.
我們被稱為國家,.
但是,.
我們被稱為國家,.
但是,.
我們被稱為國家,.
但是,.
我們被稱為國家,.
但是,.
我們被稱為國家,.
但是,.
我們被稱為國家,.
但是,.
我們被稱為國家,.
但是,.
我們被稱為國家,.
但是,.
我們被稱為國家,.
但是,.
我們被稱為國家,.
但是,.
我們被稱為國家,.
但是,.
我們被稱為國家,.
但是,.
我們被稱為國家,.

$^{1201}$但是,.
我們被稱為國家,.
但是,.
我們被稱為國家,.
但是,.
我們被稱為國家,.
但是,.
我們被稱為國家,.
但是,.
我們被稱為國家,.
但是,.
我們被稱為國家,.
但是,.
我們被稱為國家,.
但是,.
我們被稱為國家,.
但是,.
我們被稱為國家,.
但是,.
我們被稱為國家,.
但是,.
我們被稱為國家,.
但是,.
但有趣的是,.
在古代的大多數神父.
謝謝你提出這個問題.
並且帶領我們.
進入明天的講座.
我們非常期待.
德國醫學家教授.
在明天的講座.
會議的討論.
基督教道德.
講座將於.
2點30分和7點30分開始.
我們非常期待.
再次看到您在這裡.
還有一個宣布.
就是書店還在樓下開放.
還有一些書.

$^{1241}$有關德國醫學家的.
報告還在.
如果您想要拿一份.
今天就到這裡.
謝謝大家的光臨.
謝謝各位的聽眾.
大家晚安.
謝謝.
感謝觀看.
\newpage



\section{}
\label{sec:BCZSaeNGuKE}
\textbf{The Old Testament and Christian Mission: What are we here for? (2) — Speaker: Dr Christopher Wright}
\newline
\newline
連結: \href{https://youtube.com/watch?v=BCZSaeNGuKE}{\texttt{ https://youtube.com/watch?v=BCZSaeNGuKE}} ~~~~ 語音日期: 2023-03-09 
\newline
\newline
\hyperref[sec:XyScbip7koI]{\small{< < < PREV SERMON < < <}}
~
\hyperref[sec:index]{\small{[返主目錄]}}
~
\hyperref[sec:iWfhwhP8KkA]{\small{> > > NEXT SERMON > > >}}
\newline
\newline
$^{1}$(音樂).
各位先生女士.
歡迎來到第二節.
Josephine So的文化與道德課程.
2023年.
由中國大學教學院主持.
本年我們的主題是《寫給我們聽的》.
The Old Testament as Christian Scripture.
Our keynote speaker is Reverend Dr. Christopher Wright.
the Global Ambassador and Ministry Director of Langham Partnership.
The two response speakers are Reverend Dr. Lawrence Koh.
and Dr. Johnson Yu.
Similar to the arrangement of the previous session.
there will be time for question and answer.
at the end of tonight's session.
Our ushers have memo pads ready for you.
to write down your questions.
or you can choose to come forward in person.
in the middle of the hall to ask.
Let me give you a little bit more introduction about Dr. Wright.
Dr. Wright is a messiologist.
He is an Anglican clergyman.
and also Old Testament scholar.
He has published extensively.
and two of his books.
namely 《宣教中的上帝》.
The Mission of God.
Unlocking the Bible's Grand Narrative.
and the other one.
《上帝子民的宣教使命》.
The Mission of God's People.
A Biblical Theology of the Church's Mission.
These two books have either been in the textbook.
or on the reading list.
of many of the mission-related courses.
taught here in CGST.
So it's a real pleasure to meet you in person, Dr. Wright.
and hear you lecture.
Earlier this afternoon.
Reverend Dr. Wright guided us.

$^{41}$in a reflection on how the Old Testament.
informs us about our Christian identity.
answering the question of who we are.
Upcoming in this session.
we'll be looking at what we are here for.
the Old Testament and Christian mission.
Mission to the Nations.
is not just a scholarly pursuit for Dr. Wright.
who is a missionary kid.
and had experience of teaching.
and living in India with his wife and kids.
as missionaries.
We are very much eager.
to hear his perspectives and findings.
It's very cold out there.
Let's extend a very warm welcome.
to Reverend Dr. Christopher Wright.
Thank you.
Thank you so much, Swans.
for that very warm welcome indeed.
Yes, actually this is the first time.
I've ever felt cold in Hong Kong.
I've been here many times.
but it's nearly always been very hot.
So it's actually quite a pleasure.
to be in a temperature a bit more like London.
where I left behind a few days ago.
Now in our first lecture.
we explored that theme.
as Professor Kwan said.
the theme of Christian identity.
And it was interesting that.
someone said to me just as we were going out.
that I kept on saying about identity.
is that who are we.
who are we as God's people.
And they said it's good to ask the question.
in that way rather than simply asking.
who am I.
Because so often that is the way we ask it.

$^{81}$as individuals.
we want our individual identity.
And just thinking about that.
Chance comment as we were leaving.
that of course who am I.
does depend as a Christian.
on who we are as the people of God.
But I argued then in that last lecture.
that we can't answer that question.
who are we as Christians as the church.
without going back to the texts.
and the images and the story.
and the metaphors of the Old Testament.
the scriptures of the Lord Jesus himself.
But there is a further dimension to that question.
asking that question who are we.
which has to do with the purpose.
of our existence as God's people in the world.
And there's some analogy here.
to things, human artifacts.
Because the only way to explain.
what something is.
if it's been manufactured.
is to understand what it's for.
You could have a clock.
and you could describe it.
I have a little one here.
It's a little box.
It's got a funny little face.
It's got numbers up to 12.
Why only 12?.
It's got two hands that go round and round.
But you haven't said what it is as a clock.
Until you say, ah, well, it's for telling the time.
That's its purpose of existence.
Is to help you know what time it is.
What is the purpose for which.
it's been designed and produced.
And so therefore we must ask.
if we claim that our identity.

$^{121}$is a people who are created by God.
as we said in our first lecture.
We have to further ask, yes, for what purpose.
What are we here for.
What is the reason for our existence.
and the purpose of being.
this unique people on the planet.
Or perhaps better to ask.
what was the purpose and goal.
objective of God himself.
the God of the Bible.
for which and within which.
he chose to create us as his people.
What is the mission of God.
to quote the title of a book.
that was just mentioned.
Well, the apostle Paul, of course, gives us.
what I think is perhaps the most.
succinct statement of the mission of God.
in Ephesians 1, verses 9 and 10.
where God says.
just a moment, I need to bring that on there, yes.
where God says this, I'm quoting.
or Paul says this, I'm quoting him.
he says, "God has made known to us.
the mystery of his will.
according to his good pleasure.
which he purposed in Christ".
So when Paul's talking about.
the will of God there.
he doesn't mean his will for my life.
He's talking about God's great plan and purpose.
"to be put into effect.
when the times reach their fulfilment.
to bring unity to all things.
in heaven and on earth under Christ".
And when Paul says.
all things in heaven and earth.
he means it.
He's talking about the whole creation.

$^{161}$And so God's plan, says Paul.
is to bring unity, reconciliation, healing.
to the whole creation.
in and through Christ.
That's the mission of God.
It's the redemption, the reconciliation.
of the creation that has been broken.
and fractured by sin and evil.
into the new creation.
which will be populated by the redeemed.
from every culture, tribe and nation.
through the cross and the resurrection.
of our Lord Jesus Christ.
I think that that.
that expression there in Ephesians 1.
is what Paul probably meant.
when he talked about in Acts 20.
about the whole counsel of God.
which he said he had preached.
to the people in Ephesus.
in Acts 20, verse 27.
He means the plan of God.
the purpose of God.
from what we would say.
from Genesis to Revelation.
the whole grand narrative of Scripture.
creation, fall, redemption.
and new creation.
all embraced and held together by Christ.
So the mission of God then.
is fundamentally the activity of God.
driving this story forward.
from its beginning.
through its multiple facets.
in the scriptural story.
bringing it to its conclusion.
Now that's the reason why.
some of you may know about.
the Lausanne movement.
I hope you do.

$^{201}$here at CGST.
and Billy Graham and John Stott.
and the first Lausanne Congress in 1974.
Well, at the third Lausanne Congress.
in South Africa in 2010.
we produced a statement.
which became known as.
the Cape Town Commitment.
which I dare say.
some of you would have heard of.
and I believe has also.
been translated into Chinese.
And here is how it defines mission.
Let me quote to you from it.
and it will come up.
on a couple of screens.
I hope you can read that.
It says this.
And as I read this.
please listen for the biblical echoes.
the echoes of the Scriptures.
that you, I hope, will discern through it.
It says.
We are committed to world mission.
because it is central.
to our understanding of God.
the Bible, the Church, human history.
and the ultimate future.
The whole Bible.
reveals the mission of God.
to bring all things in heaven and on earth.
into unity under Christ.
reconciling them through.
the blood of His cross.
There's Colossians.
In fulfilling His mission.
God will transform the creation.
broken by sin and evil.
into the new creation.
in which there is no more.

$^{241}$sin or curse.
God will fulfill His promise to Abraham.
to bless all nations on the earth.
through the gospel of Jesus.
the Messiah, the seed of Abraham.
God will transform.
this fractured world of nations.
that are scattered under the judgment of God.
into the new humanity.
that will be redeemed by the blood of Christ.
from every tribe, nation, tongue and language.
and will be gathered to worship.
our God and Savior.
God will destroy the reign of death.
corruption and violence.
when Christ returns to establish.
His eternal reign of life.
justice and peace.
And then God, Emmanuel.
will dwell with us.
and the kingdom of this world.
will become the kingdom of our Lord.
and of His Christ.
and He will reign forever.
I hope I might have heard an "amen".
somewhere around the room.
I read that in Africa.
they go "Hallelujah!".
but then you're Chinese.
so I don't suppose you'll do that.
Now what I hope is clear in that statement.
which I hope you heard the scriptural echoes.
is that God is the principal actor.
Although it begins with our mission.
that we are committed to.
it reminds us that it is God.
who has been driving this story forward.
from its beginning.
promise right through to its conclusion.
But within that story.

$^{281}$or that drama of the scripture.
God has chosen to have a partner.
He has chosen to bring together.
those who will participate with Him.
in this great project.
As Paul puts it.
we are co-workers with God.
which is quite a remarkable expression.
And that covenant people of course.
is the people of Israel in the Old Testament.
and believers in the Messiah Jesus.
Jews and Gentiles in the New.
So what are we here for?.
What is our purpose?.
We are called to partner with God.
in the fulfillment of God's intentions.
for the world and for creation.
to do that within the course of human history.
and then to go on serving.
worshipping and glorifying God.
for all eternity.
in the world to come.
in the new creation.
God's new creation.
Now in our first lecture.
the one if you were here earlier this afternoon.
you'll remember that we surveyed.
a number of Old Testament texts.
that related to the story of Israel.
I want to do much the same thing again.
but this time to show the reflection of them.
in mission.
how they affect the purpose of God's people.
And I want to speak about four.
Blessing in a slightly different way.
from the first.
Blessing, priesthood, witness and servanthood.
So here is the Old Testament Christian mission.
and the first thing we want to think about.
is missional blessing.

$^{321}$Now when did the church begin?.
Well some people say the church began.
on the day of Pentecost.
I hope that at least after my first lecture.
you'll realize that the church didn't begin.
on the day of Pentecost.
but began right back with Abraham.
when God called the people into existence.
But just as the church didn't begin.
on the day of Pentecost.
Christian mission didn't begin.
on the Mount of Ascension only.
at the end of Matthew's Gospel.
The mission of God's people began.
when God announced his intention.
through Abraham.
to bless not only him and his people.
but all the nations on earth through him.
That's God's mission.
That's really where we trace it back to.
Let me remind you of those verses.
Genesis chapter 12 verses 1 to 3.
very quickly read them.
The Lord said to Abram.
Go from your country, your people.
and your father's household.
to the land that I will show you.
I will make you into a great nation.
and I will bless you.
I will make your name great.
and be a blessing.
or you will be a blessing.
I will bless those who curse you.
and whoever who bless you.
I will bless those who bless you.
and whoever curses you I will curse.
and bottom line.
all peoples on earth.
will be blessed through you.
Now it's not hard to hear.

$^{361}$what the key word in those verses are.
It's the word bless and blessing.
which shines through about five times.
in three verses.
And this is very good news.
because this is Genesis 12.
which comes after Genesis 11.
You need to know these things.
You don't get a PhD for nothing.
But it's not just the arithmetic.
because as I was saying this morning.
Genesis 11 is the sort of climax.
of the story of human fallenness.
and sin and rebellion.
coming to its climax.
in the Tower of Babel.
and the scattering of the nations.
Genesis 12 suddenly is good news.
all over again.
that God is going to bring blessing.
into the world.
So it's no wonder therefore.
the Apostle Paul quotes this text.
as what he calls.
in Galatians 3 verses 8 and 9.
the gospel in advance.
That's what he said.
that the scriptures.
preach the gospel in advance.
So Abraham saying.
through you all nations will be blessed.
The Greek literally is.
he pre-evangelized Abraham.
Here's God's great surprise.
good news.
that in spite of all that has happened.
in chapters 4, 3 to 11.
God still intends.
to bless the nations of the earth.
And he would launch this mission.

$^{401}$through Abraham and his people.
So therefore as we were thinking earlier.
there is a universal purpose.
to the existence of particular Israel.
Israel as a nation exists.
for the sake of God's ultimate purpose.
for the whole of humanity.
And this fundamentally missional intention.
of the election of Israel.
echoes on and on.
through the Old Testament.
Now you kindly referred to my book.
The Mission of God.
And there's a whole chapter on this.
in there looking at these texts.
But let me just read you.
two or three of them very quickly.
Verse Psalm 22 verse 27.
All the ends of the earth.
will remember and turn to the Lord.
And all the families of the nations.
will bow down before him.
Isn't that an echo of Abraham?.
Psalm 47 verse 9.
The nobles of the nations.
assemble as the people.
of the God of Abraham.
For the kings of the earth.
belong to God.
And he is greatly exalted.
Psalm 47.
Psalm 86 verse 9.
All the nations you have made.
will come and worship before you, Lord.
And bring glory to your name.
I sometimes wonder.
what was going through the mind.
of an Israelite.
when they sang those words.
Even just thinking of the nations.

$^{441}$that they knew about in their world.
Mind you sometimes I wonder.
what's going through the mind of Christians.
when they sing our hymns and songs as well.
So it's not surprising.
But there it is in the text.
All the nations that God has made.
will ultimately worship him.
And turn to me and be saved.
All you ends of the earth.
For I am God and there is no other.
Says Isaiah 45 verse 22.
Many nations will be joined.
with the Lord in that day.
And will become my people.
Zechariah 2 verse 9 and so on.
Many many more.
Those are just a sampling.
So when we turn to the New Testament then.
It's fascinating isn't it?.
The very first verse of Matthew's Gospel.
Traces the ancestry of Jesus.
Not just back to David.
Israel's messianic king in prophecy.
But to Abraham.
So Matthew begins with Abraham in the first verse.
And ends with a strongly Abrahamic word of Jesus.
In the Great Commission.
Where he tells his disciples.
To go and make disciples of all nations.
Rather like Paul who speaks about.
The obedience of faith among all nations.
It's very Abrahamic.
Here's how a New Testament scholar.
Richard Baucom.
A very fine New Testament scholar.
Puts it on our quote.
He says Matthew frames.
The whole story of Jesus.
Between the identification of him.

$^{481}$As a descendant of Abraham.
In the opening verse of the Gospel.
And in the closing words of Jesus.
At the end of the Gospel.
With the commission to the disciples.
To make disciples of all nations.
For Matthew.
Jesus is the Messiah.
Not only for Jews.
But also for Gentiles.
He is the descendant of Abraham.
Through whom God's blessing.
Will at least at last reach the nations.
So mission to the nations.
As we would put it.
Evangelization of the world and all of that.
Is not some kind of afterthought.
That occurred to Jesus.
On the Mount of Ascension.
When he suddenly thought.
Hey I'm going off back up to heaven.
What are these guys going to do.
With the rest of their life.
Hey why don't you go and be missionaries.
That's a good idea.
A sort of afterthought.
At the end of the Gospels.
No, it's the climax.
It's the point.
It's what the Gospel is all about.
That what God had now accomplished.
Through the death and resurrection of Jesus Christ.
Must now become the source of blessing.
To all nations.
So that God could keep his promise to Abraham.
So in other words therefore.
There's this theological affirmation.
That if we are in Christ.
Then not only as Paul says in Galatians.
Are we the seed of Abraham.

$^{521}$We share the blessing of Abraham.
We are also commissioned to spread.
The blessing of Abraham.
Indeed I would argue that there is something.
Great commission-ish.
That's a word.
About the Genesis 12 1-3.
I'm not suggesting that it's sort of exactly the same.
But the language certainly bears that sort of flavour.
It starts with command go.
It also says be a blessing.
That middle line is actually an imperative.
Not just you will be a blessing.
But be a blessing.
And then it ends with.
All nations will be blessed through you.
So here is a kind of a dynamic impulse.
In which Abraham has launched into the world.
In order to become as it were the launch pad.
Of God bringing blessing into the world.
So the people whom God creates in Abraham.
Are not only a people for God.
Belonging to God as we were thinking earlier.
But also a people for the world.
Our identity is also our mission.
In other words Genesis 12 1-3.
Is both ecclesiological.
It's the beginnings as it were of the people of God.
It's also missiological.
Because it describes the very purpose.
Through which they exist.
And this Abrahamic dimension of Christian mission.
Impacts not just that Gentile mission.
In the book of Acts.
Where Paul sees the moving out to the Gentiles.
As God keeping his promise to Abraham.
But I would say would be the kind of mantra.
That one could put over all the following centuries.
Of Christian mission ever since.
You could argue that Christian mission at its simplest.

$^{561}$Is simply a matter of God keeping his promise to Abraham.
Which he began to do of course.
In the book of Acts.
And then in the New Testament era.
So that the gospel went south.
Down to Africa with the Ethiopian eunuch.
It goes east according to the church in India.
With Thomas who brought the gospel to India.
And there's certainly documentary evidence.
Of Christianity in India from the second century.
It also of course went eastwards to Syria.
And then northeast.
And it was Syrian missionaries.
Who were bringing the gospel to China.
As I'm sure you know.
By the 6th to 8th century.
Went west with the apostle Paul.
Cross the Bosporus into Europe.
And eventually reached the very edges of the world.
Ireland where I come from.
On those little islands just off the edge.
About to drop off the edge of the world.
And of course ever since.
Every time the gospel crosses a barrier.
Reaches new people.
God is keeping his promise to Abraham.
Bringing the blessing of salvation.
To new nation after new nation.
Until the day comes.
That we read about in Revelation.
Revelation 7 verse 9.
Where John says.
I saw a great multitude.
That no one could count.
From every tribe, nation, people and language.
Standing before the throne of God and the Lamb.
And I have a kind of mental picture of God.
Poking Abraham in the ribs and saying.
There you are, I kept my promise.
All nations I said.

$^{601}$All nations it is.
Mission accomplished.
So that's the first, this missional blessing.
The Abrahamic dimension of Christian mission.
And there's a lot more about that of course.
In my two books.
That leads us to the second theme.
That I wanted to bring.
Which is missional priesthood.
Moving on from Genesis to Exodus.
We won't go all the way through the Old Testament.
But at least we do the first two or three books.
God's address to Israel at Mount Sinai.
Comes after the Exodus.
After they get out of Egypt.
In Exodus chapter 19.
And here again we pick up.
One of those identities of God's people.
We mentioned it.
I mentioned it in passing in the earlier lecture.
Of being God's priesthood.
But let me read these verses to you.
Exodus chapter 19 verses 3 to 6.
And the first two verses of the chapter.
Give us both the date and the location.
It says this was just three months.
After they had come out of Egypt.
When they had reached Mount Sinai.
And camped there at the foot of Mount Sinai.
And verse 3.
Then Moses went up to God.
And the Lord called to him from the mountain.
Mount Sinai and said.
This is what you are to say.
To the descendants of Jacob.
And what you are to tell the people of Israel.
You yourselves have seen what I did to Egypt.
And how I carried you in eagles' wings.
And brought you to myself.
Now then if you will obey me fully.

$^{641}$And keep my covenant.
Then out of all nations.
You will be my treasured possession.
For the whole earth is mine.
And you will be for me.
A kingdom of priests.
And a holy nation.
These are the words you are to speak.
To the Israelites.
End of quote from verse 6.
So the very first thing that God does here.
Is that he points to his own initiative.
In verse 4.
You have seen what I have done.
Says God in verse 4.
In other words this is the priority of God's grace.
As I said also earlier about election.
God redeems his people out of Egypt.
And then talks about obeying him.
And keeping the covenant.
Redemption, salvation comes first.
God's initiative of grace.
And then obedience as a response to that.
In the keeping of the law.
And so the motivating power for all that follows.
Including the Ten Commandments and the rest of the law.
Is always what God has done for them.
His redemption from slavery.
The Exodus becomes the motivating power.
Why should we keep this law.
Because God redeemed us out of Egypt.
That's the theme that flows through the rest of the law.
But then, and one could go on a lot about that point.
But then the text there in Exodus 19.
Moves on from that backward looking motivation.
What God had already done.
In 18 chapters of salvation.
Moves on to a forward looking motivation.
That is what God planned for Israel.
To be in the world.

$^{681}$In the midst of the nations.
Here's my agenda for you.
This is what you will be for me.
This is a future.
This is a looking forward.
It is indeed what we might call a missional perspective.
Because God's plan for his people.
Is that they should be a priestly and holy people.
In the midst of all nations in the whole earth.
Verse 5.
Which as I hope you will detect by now.
Has a certain Abrahamic ring about it.
God had said to Abraham.
I'm going to bless all nations in the earth through you.
And here is Yahweh God as it were.
At the top of Mount Sinai.
Saying to the people of Israel so to speak.
From up here as it were in the top.
I can see the whole earth and it's all mine.
And all the nations of the earth belong to me.
But you will have a particular purpose.
A particular identity and agenda in that world.
To be a priestly and a holy people.
Now that identity of being priestly and holy.
Is something which echoes on.
Through other parts of the Old Testament.
And indeed is inherited by us as Peter says.
And will still be there in the book of Revelation.
But we need to ask.
What does it mean for Israel.
To be God's priesthood in relation to the nations.
And in order to understand that.
We need to ask.
Well what were Israel's priests for them.
How did the priests function.
And here is one way that I sometimes.
Try to illustrate that for you.
Priests basically were middle men.
They stood in between God on the one hand.
And all the rest of the people on the other.

$^{721}$Sometimes we use the word mediator.
Or in the middle.
And in that mediating position between God.
And the rest of the Israelites.
They had two main jobs.
One which we don't often think about.
Was that they were to be teachers.
Teachers of God's law to the people.
That's actually what is said about them.
Right at the very ordination of Aaron and his sons.
The Lord says to Aaron.
This is in Leviticus chapter 10.
He says they must distinguish.
Between the holy and the common.
The unclean and the clean.
So that you can teach the Israelites.
All the decrees that the Lord has given them.
Through Moses.
The priests were to be teachers.
So that the people would come to know.
The ways of God.
That's also there in the blessing of Moses.
On the tribe of Levi in Deuteronomy chapter 33.
Where it says about Levi.
Which was the Levites tribe.
That the priests came from.
He teaches your precepts to Jacob.
And your law to Israel.
And he offers incense before you.
And whole burnt offerings on your altar.
So the teaching function actually comes first.
That through the priests.
God would become known to the people.
Which is why when the people went so badly wrong.
In later life of Israel.
Who did the prophets blame?.
The priests.
That's what Hosea says very clearly.
He says the land.
There's no knowledge of God in the land.

$^{761}$The land is full of idolatry.
And stealing and robbery and murder and adultery.
And then he says.
Who is my quarrel with?.
Who am I accusing?.
He says it is you, O priest.
Because the priests were failing to teach the people.
So that's the first task.
Through the priests.
God would become known to the people.
And then the other direction of course.
That we're perhaps more familiar with.
Is that it was the job of the priests.
To bring the sacrifices of the people to God.
Which you read in the first seven chapters of Leviticus.
So you know this of course.
That if you're an Israelite.
And for some reason.
Whether some sin or some uncleanness.
That you were unable to come into the.
Worshipping community of God.
The temple and so on.
What would you do?.
Well you would bring an animal.
To the sanctuary as prescribed in the law.
You would lay your hand on its head.
The priest would be then slaughtered.
And the blood of the sacrifice.
Would be thrown against the altar.
Representing God.
And the priest would declare to you.
That your sin is atoned for.
Covered or possibly cleansed.
The word atoned has two possible meanings.
And so with the sin dealt with.
Through the sacrifice.
You could then come back.
Into covenant fellowship with God.
And if you offered a fellowship offering.
As it was called.

$^{801}$Then you're going to have a barbecue as well.
And enjoy a good family meal with the meat.
That was what it signified.
Was that you and I are back in fellowship.
With God and his people.
So you can see then that the job of the priest.
Was twofold.
It was bringing God to the people.
And bringing the people to God.
That was their role within Israel.
So it's very significant isn't it.
That God says to the Israelites.
As a whole people.
As a community for him.
He said you will be for me.
To the rest of the nations.
What your priests are for you.
Through you my people Israel.
I will make myself known to the world.
And through you ultimately.
I will draw the world to myself.
That's what it meant for Israel to be God's priesthood.
In the midst of the nations.
At least that's part of the meaning.
There are other ways of thinking of it as well.
So as the people of Yahweh God in the Old Testament.
They would have this task of.
Bringing the knowledge of God to the nations.
Which they've done.
Because we've got this book.
Three quarters of which are the scriptures of Israel.
Through whom we know the living God.
So God has made himself known as it were.
Through the priesthood of Israel.
Simply being the people that brought us the scriptures.
And of course we know.
That ultimately it was through Israel.
The Messiah of Israel.
The Lord Jesus Christ himself.
That God has drawn the nations to himself.

$^{841}$As Jesus put it.
That I if I be lifted up.
Will draw all men to myself.
In his revelation of God.
And in his atoning work at the cross.
God through the Messiah Jesus.
Has made himself known.
And has brought the world to himself.
So that's the priestly task of this people.
And even in exile.
Even when they were thrown out of their land.
Under God's judgment.
Jeremiah writes them a letter.
Do you remember in Jeremiah chapter 29.
And says that even in the midst of that calamity.
They were instructed to be a blessing.
To the nation where they were.
That is not the precise words he uses.
But it's by implication.
Where Jeremiah says in chapter 29 verse 7.
Seek the shalom of the city to which I've carried you into exile.
Babylon their enemies.
Pray to the Lord for it.
Because in its shalom is your shalom.
In other words you can be a blessing.
Even to the people who are your enemies.
In the midst of the nations.
So that's what I mean by saying.
That it does seem to me that the priesthood of God's people.
Is a missional function.
Which stands in continuity.
With the Abrahamic election.
For the sake of the nations.
So the mission of God's people.
Then includes being this priesthood.
In and for the world.
And that fits exactly with the way.
One way in which the New Testament.
Speaks about our mission in the New Testament.
Certainly it's how the apostle Paul.

$^{881}$Saw his task.
Here I'm quoting from Romans chapter 15.
Where Paul reminds the Romans.
He says I want to remind you.
Of the grace that God gave to me.
To be a minister of the Messiah.
Christ Jesus to the Gentiles.
To the nations.
So he's a minister of the Messiah.
To the nations.
He gave me the priestly duty.
Of proclaiming the gospel of God.
So that the Gentiles.
The nations.
Might become an offering acceptable to God.
Sanctified by the Holy Spirit.
It's a fascinating verse.
In fact it's the only verse in the New Testament.
Where anybody speaks about their personal work.
In priestly terms.
The pastors in the church.
They're called pastors or elders or shepherds.
But not specifically priests.
But Paul says he had a priestly job to do.
Now Paul could never have functioned as a priest.
In Jerusalem in the temple.
Because he was the wrong tribe wasn't he.
The soul of Tarsus was of the tribe of Benjamin.
Not of Levi.
But he says yes but I had a priestly job.
Not in some ordained function as it were.
Within the church.
But precisely in his evangelistic work.
Among the nations.
He says through my role.
My lifetime's work.
Was to bring God to the nations.
In order to bring the nations to God.
That's a priestly function.
I don't know whether he had this verse.

$^{921}$Of Exodus in mind when he wrote that.
Or whether it was just as it were.
Part of the furniture and assumptions of his mind.
That his evangelism.
Was effectively priestly work.
Bringing the knowledge of God.
The gospel to the nations.
So the nations could come to God.
And we might think.
Wow that's all very well.
That was the apostle Paul.
He was an apostle.
He was a missionary.
I'm just an ordinary Christian.
Well we don't get off quite so easily.
Because as the text says on the screen.
The apostle Peter applies precisely.
This very same verse.
Exodus 19 verse 6 to all believers.
Writing almost certainly to a mixture.
Of both Jewish and Gentile believers.
Who were scattered across the various provinces.
Of Asia Minor.
Where the churches were in his day.
And you notice how in 1 Peter 2 verses 9 to 12.
Peter combines a number of Old Testament references.
Including Exodus 19 verse 6.
And insists that this is how we.
We Christian believers.
Are to live among the nations.
They're fairly familiar verses.
But I'll read them anyway.
Peter says you.
Plural, you are a chosen people.
A royal priesthood.
A holy nation.
Is Exodus 19.
God's special possession.
That you may declare the praises of him.
Who called you out of darkness.

$^{961}$Into his wonderful light.
There's the Exodus imagery again.
Out of, into.
Once you were not a people.
Now you are the people of God.
That's Hosea.
Once you had not received mercy.
But now you have received mercy.
So therefore dear friends.
On the basis of that.
Unfortunately many of our English Bibles.
Put a paragraph division in there.
With another heading.
Which is a shame.
Because it spoils the flow of Peter's thought.
He says this is who you are.
This is what you've experienced.
Therefore I urge you as foreigners in exile.
To abstain from sinful desires.
Which war against your soul.
And to live such good lives.
Among the nations.
And again our English Bibles.
Sometimes say among the pagans.
Or the heathen or something.
But it is among entwise ethnic.
Among the nations.
It's Genesis language.
So that even though they accuse you of doing wrong.
They may see your good deeds.
And glorify God on the day he visits us.
Where Peter is echoing the Lord Jesus Christ himself.
You says Peter.
You are that priesthood.
You, we.
We are meant to be God's representatives.
Among the nations.
To be the living proof of the living God.
To bring God to people.
And to bring people to God.

$^{1001}$That's our priestly mission.
And how do we do that.
Well of course we can only do that.
By the second phrase that Moses used.
That is not only by being a priestly kingdom.
But by being a holy people.
And that's what we will think of tomorrow.
If you can come back for lecture number three.
When we ask about how then should we live.
What are the ethical standards.
Of being a holy people.
So that's therefore missional blessing.
And missional priesthood.
I think there are just two other ones.
That I want to put before us.
And that is witnesses and servants.
They occur actually together in one verse.
In Isaiah chapter 43 verses 9 to 11.
This is what we read.
God says to the Israelites.
He says you are my witnesses.
Declares the Lord.
And my servant whom I have chosen.
And I just want us to think about.
Each of those two words.
Now the context of those verses there.
In Isaiah chapter 49 are of course.
That Israel had been in exile in Babylon.
And many of them had completely lost hope.
For any future for themselves.
Or indeed for the purposes of God.
You know from Ezekiel for example chapter 37.
That the Israelites in exile thought of themselves.
As dead bones in the grave.
We might as well be finished.
There's no more future for us.
But God's word through these chapters of Isaiah.
Repeatedly reassures them.
That God's purposes go on.
And that God still has a place within them.

$^{1041}$And that their mission was to bear witness.
To Yahweh the God of Israel.
As being the only true and living God.
That they were God's servant.
In order to bear witness to the living God.
So missional witness.
You see as again you probably know.
If you've done some old tests.
And study and read these chapters of Isaiah.
That there's a great kind of conflict going on.
In these chapters between.
Yahweh the God of Israel.
And the other nations and their gods.
It's like a kind of court case.
That's happening between them.
And in Isaiah chapter 43 verse 9.
There's a great assembly of the nations pictured.
It's like a sort of great court.
Here's Isaiah 43 verse 9.
All the nations gather together.
And the peoples assemble.
Which of their gods foretold all this.
And proclaimed to us the former things.
Now you see God has been saying earlier.
That he was the God who had predicted the exile.
And he was the God who had explained it.
And interpreted it in the past.
So which of their gods did that.
Let them bring in their witnesses.
To prove that they were right.
So that others may hear and say.
Oh yeah yeah that's right.
So there's this sense in which.
In this assembly of the nations with their gods.
Who and how will we decide which God or gods is real.
Which God is right.
Now Yahweh the God of Israel has made his claim.
To be the only true God.
In all the way through these earlier chapters.
By his ability precisely to predict the future.

$^{1081}$To interpret the past.
And to explain the present.
That he's the sovereign God of all these things.
So the other nations then.
The other gods rather.
The other gods are invited to bring in their witnesses.
If they can prove that they're any better.
Or even anywhere near as good.
Now who are the witnesses of the other gods.
Well the other nations.
But they don't say anything.
Because they've got nothing to say.
Because their gods are weak and powerless.
And less than nothing.
So then okay who's left.
Yahweh God.
So who will speak for him.
In this international court of nations and gods.
Who will testify to the power.
The existence the reality of Yahweh as God.
Who.
God says verse 10.
You.
You are my witnesses declares the Lord.
You Israelites.
You are my witnesses.
And my servant whom I have chosen.
In other words this is God saying to Israel.
You know you're in exile.
But you're still my witnesses.
In fact it's actually very surprising.
Because God had just described Israel.
In the previous chapter.
The Israelites in exile as blind and deaf.
And is not with all due charity.
Not a lot of point bringing into a courtroom.
As a witness.
Someone who couldn't see anything.
And can't hear anything.
Blind and deaf is sort of precisely.

$^{1121}$Not what you want in a witness.
And the reason of course why.
The Israelites were blind and deaf.
As it says in chapter 42.
Is because they had rejected God's words.
And his warnings in the past.
And he had had to drive them into judgment.
But now by his restoring grace.
By the miracle of a new exodus.
Which was just around the corner.
God says you're going back.
Back to your original Abrahamic mission.
In the midst of the nations.
And when I've restored you.
You will bear witness.
To the saving God of Israel once more.
God's people must be witnesses for their God.
That is a major missional responsibility.
You are my witnesses says God.
Now the word itself is a significant one.
In Old Testament Israel.
Just being a witness of anything.
Was a very serious responsibility.
Especially in a courtroom.
Or in the gate in the city.
Where the cases were discussed.
In fact there are a number of.
Significant laws in Old Testament law.
About the importance of being a truthful witness.
And how serious it was to be a false witness.
In fact one of the Ten Commandments.
Is exactly that.
You shall not bear false witness.
Another thing of course is that.
Bribery was hated.
Because bribery would twist people.
To tell untruths.
And indeed the Old Testament law.
Had about the most severe.
Perjury law anywhere in the world.

$^{1161}$Even today.
If you gave false witness in court.
And it could then be proved that you had.
Then you would be punished.
With whatever punishment.
The person that you were falsely accusing.
Would have got.
If your witness had been believed.
Even up to the death penalty.
So if you mistakenly accused somebody.
Of something for which they could be executed.
And you were shown to have.
Told the untruth.
You would be executed.
Now that's pretty severe.
It's a very deterrent law.
It shows how seriously God took the importance.
Of being a true and truthful witness.
The integrity and the truthfulness of witness.
Was important in Old Testament society.
So here you see God is.
Calling the Israelites to exercise.
In the international arena of nations and gods.
A responsibility.
That was deeply rooted in their own social culture.
The task of bearing witness.
Now what would it be.
That the Israelites were to bear witness to.
Well these verses that we are just starting on.
Spell it out.
Let me now read to you.
What they were to bear witness to.
This is Isaiah chapter 43.
Verses 9 to 13.
You are my witnesses declared the Lord.
And my servant whom I have chosen.
So that you may know and believe me.
And understand that.
I am he.
I am God alone.

$^{1201}$Before me no God was formed.
Nor will there be one after me.
I'm not a temporary God.
In other words.
Alpha and omega kind of thing.
I even I am the Lord.
And apart from me there is no saviour.
I have revealed and saved and proclaimed.
I am not some foreign God among you.
So you are my witnesses declares the Lord.
That I am God.
It's a wonderful language.
Israel then is being called to bear witness.
To several great foundational truths.
About Yahweh the God of Israel.
All of which we will see in a moment.
Apply to Jesus Christ.
When you get to the New Testament.
And therefore help us to understand.
What it means for us to be witnesses to Jesus.
The three things that are clear in this text are.
First of all.
Israel was to bear witness.
That Yahweh alone is the transcendent eternal God.
I am he.
I am God.
There is no other God.
He is the transcendent eternal living God.
There is none before and none to come after.
That's Old Testament monotheism.
It's not just saying there is only one God.
Doesn't really matter which God you have.
As long as you only got one.
No, this is saying.
Not only is there only one God.
This is who he is.
It's the identity of the living God.
Secondly.
That Yahweh alone is in sovereign control of history.
I have revealed and saved and proclaimed.

$^{1241}$This again is recollecting the story of the Exodus.
That God revealed it.
This is what he's going to do.
Then he did it.
Then he interpreted it.
Then he explained it.
And teaches his people.
And only the God who is in control of events.
Can do all of those things.
Can say this is what's going to happen.
That's what just happened.
This is why it happened.
God interprets, explains.
It's the author of the story.
Who controls the story.
God is sovereign in history.
And that Yahweh alone is not only God.
And sovereign but also saviour.
The I alone and the God who say.
There is no other saviour than the living God.
That indeed is his name.
He is the God then.
Whose righteousness had sent the Israelites into exile.
And it's the same God.
Whose righteousness will deliver them from exile.
There is no other saving God.
Because there is no other God.
Period.
You said God to the Israelites.
You are my witnesses that I am God.
You said Jesus twice to his disciples.
The risen Jesus to his disciples.
You will be my witnesses.
Luke makes that phrase by Jesus.
A kind of verbal velcro.
Between the end of his gospel.
And the beginning of Acts.
It's there in Luke chapter 24.
In the day of resurrection.
Where Jesus says.

$^{1281}$You know what the scriptures have said.
This is what is written.
Messiah will come and suffer and rise again.
In the third day.
That repentance and forgiveness of sins.
Will be preached in his name to all the nations.
And you are witnesses of these things.
And then also in Acts chapter 1.
Verse 8.
On the day of his ascension.
Where Jesus says to the disciples before he left.
You will be my witnesses.
In Jerusalem, Judea, Samaria.
And to the ends of the earth.
And I have no doubt in my mind anyway.
That Jesus was consciously echoing this text from Isaiah 43.
That just as Yahweh God of Israel had said to the Israelites.
You are my witnesses in the world of the nations.
So Jesus the risen Christ now calls his disciples.
To do the same task.
To bear witness to the living God among the nations.
Because after all.
What greater truth could there be in the universe.
Than the identity of Jesus of Nazareth.
As the living God who has come and lived among us.
The sovereignty of Jesus as Lord.
He is the Lord Jesus Christ.
Not kurios kaisar, Caesar is Lord.
But Jesus is Lord.
They went out into the world to say.
And that he is the one and only saviour.
That there is no other name given among men under heaven.
By which we must be saved.
Than the name of Jesus himself.
So he alone is God and Lord and saviour.
And we are his witnesses.
That's why it seems to me on the Mount of Ascension.
That Jesus says these amazing words.
That Matthew records in Matthew 28.
He almost in his way.

$^{1321}$Matthew tells us that when they came to Jesus.
At the Mount of Ascension they worshipped him.
Though some doubted.
He's very honest about that.
Not everybody was convinced.
But for those who worshipped Jesus.
This is Jesus, this is Jesus of Nazareth.
The carpenter's son.
The man they'd walked around with for three years.
And these were Jews who knew.
That they must not worship anybody or anything.
But the living God.
But before the risen crucified Jesus.
They worshipped him.
And Jesus accepts it.
And he assumes the Yahweh position.
And claims all authority in heaven and on earth.
Has been given to me.
And those men and women who were there.
Would almost certainly have heard the echo of Deuteronomy again.
Acknowledge and take to heart this day.
That the Lord is God in heaven above.
And on the earth below.
There is no other.
Deuteronomy 4 verse 39.
It's as if Jesus is saying to his disciples.
Look guys, now you know who I am.
That this God you worshipped all your lives.
Yahweh God, the Lord God, the Holy One of Israel.
Has been walking among you.
In the flesh of Jesus of Nazareth.
And also that Jesus has accomplished.
All that Yahweh the God of Israel said he would accomplish.
To bring about the salvation of the world.
So you know that I am he.
And therefore you are my witnesses of these things.
Of my life, my teaching, my death.
My resurrection.
So how then will the nations come to know.
What God has done through the Lord Jesus Christ.

$^{1361}$How will the nations come to that Abrahamic blessing.
Of the saving knowledge of these truths.
About the living God.
You are my witnesses says Jesus.
You now stand in the place of Israel.
Bearing witness to the living God.
And his Son the Lord Jesus Christ.
And all he has accomplished.
And so once again that seems to me.
That we seem that.
One of these primary dimensions.
Of what it means to be a Christian.
Namely that we are called to bear witness.
To the Lord Jesus Christ.
Is deeply deeply rooted.
In the soil of Old Testament Israel.
And the identity and mission that was entrusted to them.
So blessing and priesthood and witness.
And last of all in the same verse.
Missional servanthood.
That's the other phrase that's there in Isaiah chapter 43 verse 9.
You will be my witnesses and my servant.
Whom I have chosen.
Now that word there in chapter 43.
Actually repeats what God had said to the Israelites.
In chapter 41 that's on the screen.
Where God had said to Israel.
Isaiah chapter 41.
You Israel my servant.
Jacob whom I have chosen.
You descendants of Abraham my friend.
Lovely word.
I took you from the ends of the earth.
From its farthest corners I called you.
And I said you are my servant.
I have chosen you.
I have not rejected you.
So don't be afraid I'm with you and so on.
So back there in Isaiah 41.
Israel is identified as God's servant.

$^{1401}$But there's a massive problem.
I referred to earlier for God.
Was that at this precise moment.
Israel was in exile under God's judgment.
A failed disobedient blind deaf servant.
As God says to them.
Hear you deaf.
Look you blind and see.
Who is blind but my servant.
And deaf like the messenger I send.
Who is blind like the one supposedly in covenant with me.
Blind like the servant of the Lord.
And he goes on this is chapter 42.
Who handed over Jacob to become loot.
And Israel to the plunderers.
Was it not the Lord against whom we have sinned.
For they would not follow his ways.
They didn't obey his law.
So he poured out on them his burning anger.
And the violence of war and so on.
And so here is Israel.
Called to be the servant of God.
But languishing in exile.
But says God.
He introduces the servant of the Lord in chapter 42.
And as again you possibly know.
If you studied these chapters.
That theme of this servant of the Lord.
Flows through these chapters.
In a mysterious way.
As both being in a sense Israel.
But also being an individual.
Who has a mission to Israel.
And indeed through Israel for the nations.
Because in chapter 42.
He's going to bring justice to the nations.
But then that becomes not enough.
Because this is chapter 49.
The Lord says he who formed me in the womb.
To be his servant.

$^{1441}$To bring Jacob back to him.
To gather Israel to himself.
So the servant of the Lord.
Has a mission to Israel.
To bring Israel back to God.
But in chapter 49.
The servant seems to be struggling.
Because he says I have laboured in vain.
I spent my strength for nothing at all.
He has this sense of frustration.
And it gets worse in chapter 50.
Where the servant experiences rejection.
And abuse and people are.
Plucking out his beard and stuff.
So it almost seems.
As if the servant's mission.
Is failing in frustration and opposition.
And then God answers him.
In chapter 49.
With a very startling word.
The Lord says to me.
In answer to that.
It's too small a thing for you to be my servant.
Just to restore the tribes of Jacob.
And to bring back those of Israel that I've kept.
I will also make you a light to the Gentiles.
To the nations that my salvation.
May reach the ends of the earth.
I don't know if you've ever had God.
Do that to you.
You know you complain to God.
This is far too hard.
I can't do this.
Because oh really.
Well let me give you something harder.
God sometimes does that.
That's what he does to the servant.
He's going to bring Israel back to God.
But also bring the salvation of God to the nations.
So to sum this up.

$^{1481}$This servant language then.
Israel as a people.
Was a servant of God.
Chosen upheld by him.
In order to be a light to the nations.
But historically in exile.
Israel was failing in that role in that mission.
Israel as the servant was blind and deaf.
And under God's judgment.
So this individual servant figure.
Mysteriously in the prophet.
Is at one level distinct from Israel.
Because he has a mission to them.
And yet at another level.
Is somehow identified with Israel.
Same language is used of both.
So that the servant figure.
Will actually fulfill the original mission of Israel itself.
That is to bring the salvation.
The justice.
The liberation of God.
To the ends of the earth and to all the nations.
The universal purpose of the election of Israel.
Will be accomplished through the servant of Israel.
Now when we come to the New Testament.
There's abundant evidence.
Which I won't try to go into.
To suggest that Jesus saw himself.
In that identity as the servant of the Lord.
In the texts of Isaiah.
For example when he says.
I came not to be served but to serve.
Or I am among you as the one who serves.
And he quotes Isaiah 53.
At the Last Supper and so on.
I haven't time to go there.
But that mission of the servant.
In the Messiah Jesus which he quotes.
Doesn't end with Jesus himself.
Because what we then discover is that.

$^{1521}$The Apostle Paul reflected very deeply on these texts.
Also in his mission to the nations.
He saw theologically a link between.
Jesus being the servant who was sent to Israel.
On the one hand.
And the ingathering of the nations.
With great rejoicing to the people of God.
On the other hand he saw them linked together.
That indeed is how he brings his letter to the Romans.
To a climax.
A climax which we sometimes ignore.
But it's actually very important in chapter 15.
Where Paul has been saying to the Jewish and Gentile believers.
In chapters 14 through to 15 7.
He's been telling them.
Jews and Gentiles in Christ.
That they must accept one another.
So he says.
I'm now quoting from Romans 15 verse 7.
Accept one another then.
Just as Christ Messiah accepted you.
In order to bring praise to God.
For he says I tell you.
Christ the Messiah has become a servant.
The servant of the Jews.
On behalf of God's truth.
So that.
There's the purpose.
So that the promises made to the patriarchs.
I.e. to Abraham.
Might be confirmed.
Moreover that the nations might glorify God for his mercy.
As it is written.
And then he quotes four Old Testament texts.
Therefore I will praise you among the Gentiles.
And sing praises of your name.
Again it says rejoice you Gentiles with his people.
And again praise the Lord all you Gentiles.
That all the peoples extol him.
And again Isaiah the root of Jesse will spring up.

$^{1561}$And he will rule over the nations.
And in him the Gentiles will hope.
Four texts from the law the prophets and the writings.
All of them speaking about God's ultimate purpose.
That the nations the Gentiles will come to rejoice.
And praise God.
And Paul says all of that in a sense.
Is the fulfillment of the fact that.
Jesus became the servant of Israel.
He was the servant.
So that the glory of God.
And the praise of God could go to the nations.
That's how he ends Romans.
It's the climax he brings it to.
And of course Paul could do this.
Because this had been part of his calling.
At the very beginning.
In Acts chapter 26 verse 16.
He knew that he had been called to be a servant and a witness.
The very words that Jesus spoke to him echoed that.
Here's what Jesus said to Saul of Tarsus.
As he still was on the road to Damascus.
Get up stand on your feet.
I've appeared to you to appoint you.
Paul Saul of Tarsus.
As a servant and as a witness.
Of what you have seen and will see of me.
I will rescue you from your own people.
And from the Gentiles.
I'm sending you to them.
The nation the Gentiles.
To open their eyes and turn them from darkness to light.
And from the power of Satan to God.
So that they may receive forgiveness of sins.
And a place among those.
Who are sanctified by faith in me.
Paul himself.
Was commissioned to be servant and witness.
And that's why.
When Paul was in a synagogue.

$^{1601}$In Pisidia in Antioch.
And he had given the whole history of Israel.
In about seven verses.
To the people there.
And led it from Abraham right up to Jesus.
And said this is what has now been fulfilled.
And some of the Jews believed him.
And some would not.
And the following Sabbath.
They came back and some of them opposed him.
And Paul what he does is that he takes this text.
From Isaiah.
Which of course he knew had been fulfilled by Jesus.
And he applies it to himself.
And his little band of missionary friends.
He moves from Jesus as the true son of David and Messiah.
And his resurrection.
Through to the God-fearing Gentiles in the audience.
This is his words, Acts 13.
We now turn to the Gentiles.
For this is what the Lord has commanded us, he says.
I have made you a light for the Gentiles.
That you may bring my salvation to the ends of the earth.
That's what God has commanded us, says Paul.
And we've been doing our exegesis and hermeneutics class.
We might say, just a minute Paul.
Isaiah actually said that about Old Testament Israel.
Paul said, yeah I know that.
And Jesus we understand.
Jesus applied the language of the servant to himself.
And Paul said, yeah I know that too.
I've just said it in chapter 15.
But here Paul says.
That this command of the servant to Israel through Jesus.
Is now also explicitly the command for God's people.
And we are to be that servant people.
Bringing the light of God to the nations.
It is a missional text for the church.
As the servant of the Lord bearing witness to him.
So there we are.

$^{1641}$Once again I hope what we've seen is that.
In the scriptures of Israel and the Old Testament.
We find some essential understandings.
Of what it means not just to be.
The people of God.
But for the purpose for which we are.
The people of God.
Rooted in God's missional purpose for Israel.
Like them we are called in partnership with God.
For the nations and to the nations.
To be a blessing, to be a priesthood.
To be a witness and to be God's servant.
May God help us to do that.
For his name's sake, amen.
Thank you.
謝謝觀看.
感謝觀看.
(CC字幕製作:貝爾).
\newpage



\section{}
\label{sec:iWfhwhP8KkA}
\textbf{The Old Testament and Christian Mission: What are we here for?(2) — Dr Lawrence  Ko | Dr Johnson Yip}
\newline
\newline
連結: \href{https://youtube.com/watch?v=iWfhwhP8KkA}{\texttt{ https://youtube.com/watch?v=iWfhwhP8KkA}} ~~~~ 語音日期: 2023-03-09 
\newline
\newline
\hyperref[sec:BCZSaeNGuKE]{\small{< < < PREV SERMON < < <}}
~
\hyperref[sec:index]{\small{[返主目錄]}}
~
\hyperref[sec:QNbcTFot66g]{\small{> > > NEXT SERMON > > >}}
\newline
\newline
$^{1}$(音樂).
謝謝Dr. Wright.
為我們進行精彩的探索.
教會的使命是如何深受古代的歷史.
並且被傳承到新的一代.
我們現在有兩位接待者.
首先是羅倫斯·科爾教會主任.
教會教學分部的主席.
教學學院協會教授.
Alliance Bible Seminary.
接著是Dr. Johnson Yip.
教學學院協會教授.
CGST.
如剛才宣布.
接待者接待後.
會有時間給予問題和回答.
所以你們可以開始寫下問題.
並將問題寄給教會.
隨時.
或者可以前往主席室.
並親自發言.
你們可以選擇用中文.
我會盡我所能為你們翻譯.
所以是時候給羅倫斯·科爾教會的回答.
這是我的榮幸.
為Dr. Christopher Wright提供回應.
在《基督教使命》和《古書》的報告中.
我與他吃過晚餐.
我知道他最好的朋友是Walter Mulberry.
他是我在德羅倫斯的主任.
這讓我感到很緊張.
因為我要回應我的教授最好的朋友.
這是一個非常值得回應的時刻.
這是我的回應.
這份報告目的是.
探索耶穌基督的使命層面.
在各個古代的經文中.
第一.
Dr. Wright認為耶穌宣布了.
祂對亞伯拉罕的使命意願.

$^{41}$以便祂會成為全國的使命祝福.
如同在《基督教使命》12:1-3中描述.
第二.
他帶領我們到《基督教使命》19:3-6.
這中提到耶穌宣布了.
以耶穌為使命的基督教徒.
為耶穌的使命的基督教徒.
這是第二點.
最後.
他探索了《基督教使命》12:1-6.
以為耶穌的使命意願.
是以耶穌為使命的使命.
為全國的使命的使命.
在他的探索中.
Dr. Wright嘗試連結這些OT語句.
和他們在新教會的使用或不使用.
以為想爭取.
在古教中的基督教使命.
有關聯.
大部分的文章.
都非常吸引人.
並給我們一個很好的觀察.
來了解.
如何耶穌在古教中的使命.
已經在新教會之前.
發動了使命.
但我會想在這十分鐘內.
提出一些有關.
Dr. Wright的解釋.
OT語語的基督教文章.
這是我的第一次回應.
我的第一次回應是.
Dr. Wright認為.
基督教使命的使命.
在《基督教使命》12:1-3中.
可以用幾個定義來形容.
第一個定義是「目標」.
第二個定義是「不可思議」.
這是兩個定義.
第三個定義是「所有的國家都會受到你的影響」.

$^{81}$Dr. Wright在這裡確確實實地.
用兩個基督教定義來形容.
「神」和「被祝福」.
他嘗試將這個信息.
傳遞給大使.
Matthew 28.
他認為.
基督教使命在.
Matthew 28:18-20中.
是基於基督教使命的.
基督教使命的基礎.
但我認為.
這兩個定義的國際關係.
並不是如此關鍵.
雖然「祝福所有的國家」的概念.
已經存在於基督教使命中.
但我們不能自信地.
依賴這個概念來爭取.
「使命」的意義.
「所有的國家」的意義.
在《Matthew 28:18-20》中.
是基於古代傳遞.
信仰耶穌會開放給所有的信眾.
但在古代的情況下.
這不一定意味著.
所謂的「使命」.
被給予耶穌的民眾.
以及「保證」的意義.
此外.
除了「所有國家」的定義.
我認為基督教使命的.
基督教使命的相似關係.
在《Matthew 28》和《Genesis 12》中.
是如此的稀有.
它並不具有.
基督教使命的.
堅定的文字上的依賴.
而我們可以看到.
在這裡的文字.
確實是在《Matthew 28》和《Genesis 19》中.

$^{121}$但它並不是一個必要的.
而是一個參與的.
而必要的.
是《Matthew 28:19》的主要文字.
「使使徒」.
《Matthew 28:19》中.
顯示了使徒給予所有的國家.
成為了「偉大使命」的主要概念.
在於「使徒」的目標.
並不符合《Genesis 12》和《1-3》中的對比.
看來,羅伯特·萊特(Dr. Wright)的聲明.
《Genesis 12-1-3》.
是「基督教」和「基督教」.
不能夠正確地解釋.
「使徒」的概念.
是「全世界」和「所有國家」的概念.
這是我的第一次回應.
第二次回應.
羅伯特·萊特(Dr. Wright)認為.
《Exodus》19:5-6.
含有一個使命的角度.
神救了以色列人民.
並指引他們到馬爾西尼.
並稱呼他們為「使徒」和「聖人」.
他確實有權.
探索「全世界」和「所有國家」的元素.
如《Exodus》19:5所提到的.
這具有「使徒」的概念.
但這不一定意味著.
使徒是「聖人」和「聖人」.
他們有使命的任務.
這次考試的國際觀點.
嘗試強調「使徒」的「使命」.
而不是「使徒」的「使命」.
「使徒」的「使命」.
「所有國家」.
而不是所謂的「使命」.
羅伯特·萊特(Dr. Wright)確實有權.
說「聖人」.
我現在在螢幕上重覆.

$^{161}$《Exodus》19:2-9.
以描述基督的使命.
但如果我們仔細地比較.
1 Peter和《Exodus》19.
我們可以注意到.
Peter(1 Peter)確實在這裡提出了一個聲明.
這個聲明可以在這裡看到.
「以免你以為你被他所叫的.
『你被黑暗所喚醒』.
在他的奇妙的光明之中」.
這個聲明會在.
《Exodus》的引述後.
「你是聖神」.
「你是皇帝」.
「你是聖國」.
這明顯地顯示.
Peter(1 Peter)提出了.
使命的聲明.
在《Exodus》19的引述中.
這個聲明並不在《Exodus》的原始文章中出現.
因此,我無法看到.
《Exodus》19的使命聲明.
或使命層面.
即使1 Peter描述了.
使命和基督的使命.
是同一種和諧的使命.
所以,總結一下.
我欣賞了Rice的文章.
《使命和基督的使命》.
在傳遞基督的使命中的使用.
我嘗試在這裡提出一些和諧的問題.
以示意圖引起一些關注.
關於使命的使命的使用.
這並不一定完全符合使命的概念.
雖然我同意了文章的關於.
《使命和基督的使命》.
是基督的使命.
這肯定有使命層面.
但我認為.
《使命和基督的使命》.

$^{201}$和《Exodus》.
並不能夠有信心.
傳遞出使命的聲明.
我們應該更加關注.
新的理解.
在不同的解釋層面.
或不同的和諧層面.
之間的新和舊經文.
所以,我非常緊張.
這是我回答的結尾.
非常感謝.
謝謝.
謝謝Dr. Wright.
給我們帶來的靈感演講.
我感到榮幸.
可以在這裡發表我的個人回應.
Dr. Wright 認為.
古代教義.
以一個牽連的故事來看.
關於神的使命.
和耶穌基督的參與.
在他的觀點.
使命是整個福音書的主要主題.
使命和基督的和諧.
是基於了解福音書的主要主題.
Dr. Wright 作為學者.
做出了重要的貢獻.
就是為了提供最好的指導.
為基督的和諧.
在解讀古代教義時.
中國教會教會人.
經常讀到新和舊經文.
也就是從新和舊經文.
到古代教義.
這經常會導致.
在解讀古代教義.
在新和舊經文的詞彙中.
失去了正義的觀點.
內容和現實.
古代教義的內容.

$^{241}$在此之外.
有些基督徒.
認為新和舊經文的教會.
是以以色列為代替.
他們通常認為.
新和舊經文的信徒.
把以色列視為.
神的救贖和保證的新接待者.
而Dr. Wright認為.
古代教義並不依賴.
新和舊經文的意義和意義.
他認同古代教義和新和舊經文的差別.
他不認為以色列是新和舊經文的.
出生的一種重要共同體.
Dr. Wright 顯示.
正如神所說.
透過亞伯拉罕.
把以色列變成了.
他的保證夥伴.
為國家的祝福.
新和舊經文描述.
神透過耶穌.
亞伯拉罕的子孫.
把所有的國家人民.
相信耶穌.
變成了.
亞伯拉罕的保證.
和世界的祝福源.
透過從基督教角度來看.
Dr. Wright 成功地.
與以色列的呼喚.
為全國的祝福器.
和教會的身份和任務.
如同新和舊經文描述.
教會是以色列的延伸.
和共同的任務.
根據Dr. Wright的說法.
基督教任務.
最簡單地說.
是神保留了.

$^{281}$祂的保證給亞伯拉罕.
他指出.
即使在死亡.
和受到神的判斷.
以色列仍然被呼喚.
回到他們的亞伯拉罕身份.
如同聖經中的.
二十九話.
Dr. Wright 更加強調.
在死亡和被執法的時期.
以色列被呼喚.
為國家的祝福.
就算以色列.
在巴勒納敵人的土地上.
Dr. Wright 強烈地提出.
神的使命觀.
神呼喚了以色列.
成為世界人民的.
祝福樂器.
這從古代教的.
常識中看來.
是天主的.
判斷和災難的宣告.
我猜.
這兩種方式.
不一定是互相反映.
但會比較喜歡.
和互相讚賞.
幫助我們.
謙虛的人類.
更了解.
神的真誠和信仰.
我會用兩種例子來描述.
第一.
Dr. Wright 觀察到.
雖然以色列.
在死亡的時期.
失去了希望.
為神的未來.
但神會通過.

$^{321}$迫害者 以色列.
確保他們知道.
以色列仍然有.
神的目的.
希望和神的信仰.
成為了.
在死亡的後期迫害者中.
最著名的.
在此.
我想提到.
迫害者.
如謝瑞和伊西克.
看到死亡.
是以色列的機會.
為他們的身份.
為世界的和平而獻上.
在《新的遵循》的.
傳說中的.
傳說31:31-34.
神將寫上他的福音.
在人民的心中.
使得.
神的知識.
成為了內在的.
這份傳說的下層.
是神的新的心.
和新的靈魂.
在伊西克36:26.
神將寫上他的福音.
在人民的心中.
意味著.
神的福音的觀察.
將成為了以色列的.
內在的反應.
以色列能夠.
提升他們的.
使命身份.
他們被稱呼為.
並提升他們的職責.
以色列成為了.

$^{361}$以色列的例子.
即使在.
無希望的情況下.
神仍然使用以色列.
為人民展示他的光榮.
他會成為榮耀.
因為他給予的.
以色列的法律.
以色列的和平.
會保持在.
在死亡中.
我們看到.
神將實現.
他的意圖.
為人民.
為他們做出.
他們無法做到的事.
神的使命.
將會達到.
無論如何.
第二.
默克爾對世界的.
以色列的職責.
和如何.
以色列擁有.
對所有國家的.
重要影響.
以色列的參與.
神的使命.
非常明確.
以至於.
以色列可能.
在使命成功中.
扮演了決心.
但是.
我們也看到.
神使用其他國家.
在他的聖事上.
哇.
以色列的恩典和慈悲.

$^{401}$給予了所有國家的.
動機轉向神.
當神帶來.
他的知識和經驗.
他將使用以色列.
直接帶給所有國家.
而不使用以色列.
作為祝福的工具時.
例如.
以色列的.
國家在.
以色列的.
25-32章.
提到.
神是.
擁有.
世界的所有王國.
判斷他們為迷信者.
但更重要的是.
保持道德的.
定義.
並保持他們的.
最終的信仰.
在神的.
使命的.
世界中.
或是說.
通過神的判斷.
以色列的.
抗爭國家.
提出.
所有國家都會.
認識或慈悲.
神.
不需要.
以色列的.
參與.
在迷信的.
宣傳中.
使命.

$^{441}$是超過.
人類的.
專業.
它是神的.
實際活動.
在事情的.
超越性中.
神.
正在.
工作.
回到今天的講座.
今天講座.
是值得一看的.
它會提升.
我們在世界中的.
信仰.
它的工作.
值得.
關注的.
是信仰.
對於.
參與.
最佳.
宣傳書.
的一切.
謝謝.
字幕製作:貝爾.
感謝觀看.
\newpage



\section{}
\label{sec:QNbcTFot66g}
\textbf{Towards a Shared Land Theology: Palestinian Christian Reading of the Land Promises}
\newline
\newline
連結: \href{https://youtube.com/watch?v=QNbcTFot66g}{\texttt{ https://youtube.com/watch?v=QNbcTFot66g}} ~~~~ 語音日期: 2016-04-27 
\newline
\newline
\hyperref[sec:iWfhwhP8KkA]{\small{< < < PREV SERMON < < <}}
~
\hyperref[sec:index]{\small{[返主目錄]}}
~
\hyperref[sec:7cxD3Fsxces]{\small{> > > NEXT SERMON > > >}}
\newline
\newline
$^{1}$(英文).
各位兄弟姐妹.
各位先生女士.
歡迎來到今晚的公開演講.
是由中國學院教學.
以及蘭陽基金會香港組織.
邀請到今晚的嘉賓.
教授 文特·埃塞克.
我的名字是Simon Cheung.
我是教育學院的舊約翰文教學院教授.
今天是我的榮幸介紹我們的嘉賓.
文特·埃塞克.
目前是巴菲特教育學院教授.
1979年.
是由巴勒斯坦教會基督教教會.
組織的教堂.
他擔任教堂的教授.
同時也是教堂樂團的導演.
在晚餐時間.
他告訴我們.
他曾經為樂團作曲.
所以他也很有才華.
他也是現任會議主席.
名為「基督在關口」.
我請文特·埃塞克寫下.
他在2010年在會議上.
寫下了一篇文章.
來看看他如何形容此事.
他開始寫.
「巴勒斯坦教會組織的教堂.
是由愛國者,愛國者,愛國者.
和愛國者的同胞.
為他們的愛而建立的」.
「這次會議是為愛的工作.
為那些願意尋求正義和和平.
在不斷發生的衝突中.
為那些願意尋求正義和平.
而不斷發生的衝突中.
為那些願意尋求正義和平.
在不斷發生的衝突中.

$^{41}$為那些願意尋求正義和平.
在不斷發生的衝突中.
為那些願意尋求正義和平.
在不斷發生的衝突中.
為那些願意尋求正義和平.
而不斷發生的衝突中.
為那些願意尋求正義和平.
而不斷發生的衝突中.
為那些願意尋求正義和平.
而不斷發生的衝突中.
為那些願意尋求正義和平.
而不斷發生的衝突中.
為那些願意尋求正義和平.
而不斷發生的衝突中.
為那些願意尋求正義和平.
而不斷發生的衝突中.
為那些願意尋求正義和平.
而不斷發生的衝突中.
為那些願意尋求正義和平.
而不斷發生的衝突中.
在這裡,我認為.
我們必須要重新思考.
這次會議的主題.
是「共同的土地」.
這是一個非常重要的議題.
這次會議的主題.
是「共同的土地」.
這次會議的主題.
是「共同的土地」.
這次會議的主題.
是「共同的土地」.
這次會議的主題.
是「共同的土地」.
這次會議的主題.
是「共同的土地」.
這次會議的主題.
是「共同的土地」.
這次會議的主題.
是「共同的土地」.
這次會議的主題.

$^{81}$是「共同的土地」.
這次會議的主題.
是「共同的土地」.
這次會議的主題.
是「共同的土地」.
這次會議的主題.
是「共同的土地」.
這次會議的主題.
是「共同的土地」.
這次會議的主題.
是「共同的土地」.
這次會議的主題.
是「共同的土地」.
這次會議的主題.
是「共同的土地」.
這次會議的主題.
是「共同的土地」.
這次會議的主題.
是「共同的土地」.
這次會議的主題.
是「共同的土地」.
這次會議的主題.
是「共同的土地」.
這次會議的主題.
是「共同的土地」.
這次會議的主題.
是「共同的土地」.
這次會議的主題.
是「共同的土地」.
這次會議的主題.
是「共同的土地」.
這次會議的主題.
是「共同的土地」.
這次會議的主題.
是「共同的土地」.
這次會議的主題.
是「共同的土地」.
這次會議的主題.
是「共同的土地」.
這次會議的主題.

$^{121}$是「共同的土地」.
這次會議的主題.
是「共同的土地」.
這次會議的主題.
是「共同的土地」.
這次會議的主題.
是「共同的土地」.
這次會議的主題.
是「共同的土地」.
這次會議的主題.
是「共同的土地」.
這次會議的主題.
是「共同的土地」.
這次會議的主題.
是「共同的土地」.
這次會議的主題.
是「共同的土地」.
這次會議的主題.
是「共同的土地」.
這次會議的主題.
是「共同的土地」.
這次會議的主題.
是「共同的土地」.
這次會議的主題.
是「共同的土地」.
這次會議的主題.
是「共同的土地」.
這次會議的主題.
是「共同的土地」.
這次會議的主題.
是「共同的土地」.
這次會議的主題.
是「共同的土地」.
這次會議的主題.
是「共同的土地」.
這次會議的主題.
是「共同的土地」.
這次會議的主題.
是「共同的土地」.
這次會議的主題.

$^{161}$是「共同的土地」.
這次會議的主題.
是「共同的土地」.
這次會議的主題.
是「共同的土地」.
這次會議的主題.
是「共同的土地」.
這次會議的主題.
是「共同的土地」.
這次會議的主題.
是「共同的土地」.
這次會議的主題.
是「共同的土地」.
這次會議的主題.
是「共同的土地」.
這次會議的主題.
是「共同的土地」.
這次會議的主題.
是「共同的土地」.
這次會議的主題.
是「共同的土地」.
這次會議的主題.
是「共同的土地」.
這次會議的主題.
是「共同的土地」.
這次會議的主題.
是「共同的土地」.
這次會議的主題.
是「共同的土地」.
這次會議的主題.
是「共同的土地」.
這次會議的主題.
是「共同的土地」.
這次會議的主題.
是「共同的土地」.
這次會議的主題.
是「共同的土地」.
這次會議的主題.
是「共同的土地」.
這次會議的主題.

$^{201}$是「共同的土地」.
這次會議的主題.
是「共同的土地」.
這次會議的主題.
是「共同的土地」.
這次會議的主題.
是「共同的土地」.
這次會議的主題.
是「共同的土地」.
這次會議的主題.
是「共同的土地」.
這次會議的主題.
是「共同的土地」.
這次會議的主題.
是「共同的土地」.
這次會議的主題.
是「共同的土地」.
這次會議的主題.
是「共同的土地」.
這次會議的主題.
是「共同的土地」.
這次會議的主題.
是「共同的土地」.
這次會議的主題.
是「共同的土地」.
這次會議的主題.
是「共同的土地」.
這次會議的主題.
是「共同的土地」.
這次會議的主題.
是「共同的土地」.
這次會議的主題.
是「共同的土地」.
這次會議的主題.
是「共同的土地」.
這次會議的主題.
是「共同的土地」.
這次會議的主題.
是「共同的土地」.
這次會議的主題.

$^{241}$是「共同的土地」.
這次會議的主題.
是「共同的土地」.
這次會議的主題.
是「共同的土地」.
這次會議的主題.
是「共同的土地」.
這次會議的主題.
是「共同的土地」.
這次會議的主題.
是「共同的土地」.
這次會議的主題.
是「共同的土地」.
這次會議的主題.
是「共同的土地」.
這次會議的主題.
是「共同的土地」.
這次會議的主題.
是「共同的土地」.
這次會議的主題.
是「共同的土地」.
這次會議的主題.
是「共同的土地」.
這次會議的主題.
是「共同的土地」.
這次會議的主題.
是「共同的土地」.
這次會議的主題.
是「共同的土地」.
這次會議的主題.
是「共同的土地」.
這次會議的主題.
是「共同的土地」.
這次會議的主題.
是「共同的土地」.
這次會議的主題.
是「共同的土地」.
這次會議的主題.
是「共同的土地」.
這次會議的主題.

$^{281}$是「共同的土地」.
這次會議的主題.
是「共同的土地」.
這次會議的主題.
是「共同的土地」.
這次會議的主題.
是「共同的土地」.
這次會議的主題.
是「共同的土地」.
這次會議的主題.
是「共同的土地」.
這次會議的主題.
是「共同的土地」.
這次會議的主題.
是「共同的土地」.
這次會議的主題.
是「共同的土地」.
這次會議的主題.
是「共同的土地」.
這次會議的主題.
是「共同的土地」.
這次會議的主題.
是「共同的土地」.
這次會議的主題.
是「共同的土地」.
這次會議的主題.
是「共同的土地」.
這次會議的主題.
是「共同的土地」.
這次會議的主題.
是「共同的土地」.
這次會議的主題.
是「共同的土地」.
這次會議的主題.
是「共同的土地」.
這次會議的主題.
是「共同的土地」.
這次會議的主題.
是「共同的土地」.
這次會議的主題.

$^{321}$是「共同的土地」.
這次會議的主題.
是「共同的土地」.
這次會議的主題.
是「共同的土地」.
這次會議的主題.
是「共同的土地」.
這次會議的主題.
是「共同的土地」.
這次會議的主題.
是「共同的土地」.
這次會議的主題.
是「共同的土地」.
這次會議的主題.
是「共同的土地」.
這次會議的主題.
是「共同的土地」.
這次會議的主題.
是「共同的土地」.
這次會議的主題.
是「共同的土地」.
這次會議的主題.
是「共同的土地」.
這次會議的主題.
是「共同的土地」.
這次會議的主題.
是「共同的土地」.
這次會議的主題.
是「共同的土地」.
這次會議的主題.
是「共同的土地」.
這次會議的主題.
是「共同的土地」.
這次會議的主題.
是「共同的土地」.
這次會議的主題.
是「共同的土地」.
這次會議的主題.
是「共同的土地」.
這次會議的主題.

$^{361}$是「共同的土地」.
這次會議的主題.
是「共同的土地」.
這次會議的主題.
是「共同的土地」.
這次會議的主題.
是「共同的土地」.
這次會議的主題.
是「共同的土地」.
這次會議的主題.
是「共同的土地」.
這次會議的主題.
是「共同的土地」.
這次會議的主題.
是「共同的土地」.
這次會議的主題.
是「共同的土地」.
這次會議的主題.
是「共同的土地」.
這次會議的主題.
是「共同的土地」.
這次會議的主題.
是「共同的土地」.
這次會議的主題.
是「共同的土地」.
這次會議的主題.
是「共同的土地」.
這次會議的主題.
是「共同的土地」.
這次會議的主題.
是「共同的土地」.
這次會議的主題.
是「共同的土地」.
這次會議的主題.
是「共同的土地」.
這次會議的主題.
是「共同的土地」.
這次會議的主題.
是「共同的土地」.
這次會議的主題.

$^{401}$是「共同的土地」.
這次會議的主題.
是「共同的土地」.
這次會議的主題.
是「共同的土地」.
這次會議的主題.
是「共同的土地」.
這次會議的主題.
是「共同的土地」.
這次會議的主題.
是「共同的土地」.
這次會議的主題.
是「共同的土地」.
這次會議的主題.
是「共同的土地」.
這次會議的主題.
是「共同的土地」.
這次會議的主題.
是「共同的土地」.
這次會議的主題.
是「共同的土地」.
這次會議的主題.
是「共同的土地」.
這次會議的主題.
是「共同的土地」.
這次會議的主題.
是「共同的土地」.
這次會議的主題.
是「共同的土地」.
這次會議的主題.
是「共同的土地」.
這次會議的主題.
是「共同的土地」.
這次會議的主題.
是「共同的土地」.
這次會議的主題.
是「共同的土地」.
這次會議的主題.
是「共同的土地」.
這次會議的主題.

$^{441}$是「共同的土地」.
這次會議的主題.
是「共同的土地」.
這次會議的主題.
是「共同的土地」.
這次會議的主題.
是「共同的土地」.
這次會議的主題.
是「共同的土地」.
這次會議的主題.
是「共同的土地」.
這次會議的主題.
是「共同的土地」.
這次會議的主題.
是「共同的土地」.
這次會議的主題.
是「共同的土地」.
這次會議的主題.
是「共同的土地」.
這次會議的主題.
是「共同的土地」.
這次會議的主題.
是「共同的土地」.
這次會議的主題.
是「共同的土地」.
這次會議的主題.
是「共同的土地」.
這次會議的主題.
是「共同的土地」.
這次會議的主題.
是「共同的土地」.
這次會議的主題.
是「共同的土地」.
這次會議的主題.
是「共同的土地」.
這次會議的主題.
是「共同的土地」.
這次會議的主題.
是「共同的土地」.
這次會議的主題.

$^{481}$是「共同的土地」.
這次會議的主題.
是「共同的土地」.
這次會議的主題.
是「共同的土地」.
這次會議的主題.
是「共同的土地」.
這次會議的主題.
是「共同的土地」.
這次會議的主題.
是「共同的土地」.
這次會議的主題.
是「共同的土地」.
這次會議的主題.
是「共同的土地」.
這次會議的主題.
是「共同的土地」.
這次會議的主題.
是「共同的土地」.
這次會議的主題.
是「共同的土地」.
這次會議的主題.
是「共同的土地」.
這次會議的主題.
是「共同的土地」.
這次會議的主題.
是「共同的土地」.
這次會議的主題.
是「共同的土地」.
這次會議的主題.
是「共同的土地」.
這次會議的主題.
是「共同的土地」.
這次會議的主題.
是「共同的土地」.
這次會議的主題.
是「共同的土地」.
這次會議的主題.
是「共同的土地」.
這次會議的主題.

$^{521}$是「共同的土地」.
這次會議的主題.
是「共同的土地」.
這次會議的主題.
是「共同的土地」.
這次會議的主題.
是「共同的土地」.
這次會議的主題.
是「共同的土地」.
這次會議的主題.
是「共同的土地」.
這次會議的主題.
是「共同的土地」.
這次會議的主題.
是「共同的土地」.
這次會議的主題.
是「共同的土地」.
這次會議的主題.
是「共同的土地」.
這次會議的主題.
是「共同的土地」.
這次會議的主題.
是「共同的土地」.
這次會議的主題.
是「共同的土地」.
這次會議的主題.
是「共同的土地」.
這次會議的主題.
是「共同的土地」.
這次會議的主題.
是「共同的土地」.
這次會議的主題.
是「共同的土地」.
這次會議的主題.
是「共同的土地」.
這次會議的主題.
是「共同的土地」.
這次會議的主題.
是「共同的土地」.
這次會議的主題.

$^{561}$是「共同的土地」.
這次會議的主題.
是「共同的土地」.
這次會議的主題.
是「共同的土地」.
這次會議的主題.
是「共同的土地」.
這次會議的主題.
是「共同的土地」.
這次會議的主題.
是「共同的土地」.
這次會議的主題.
是「共同的土地」.
這次會議的主題.
是「共同的土地」.
這次會議的主題.
是「共同的土地」.
這次會議的主題.
是「共同的土地」.
這次會議的主題.
是「共同的土地」.
這次會議的主題.
是「共同的土地」.
這次會議的主題.
是「共同的土地」.
這次會議的主題.
是「共同的土地」.
這次會議的主題.
是「共同的土地」.
這次會議的主題.
是「共同的土地」.
這次會議的主題.
是「共同的土地」.
這次會議的主題.
是「共同的土地」.
這次會議的主題.
是「共同的土地」.
這次會議的主題.
是「共同的土地」.
這次會議的主題.

$^{601}$是「共同的土地」.
這次會議的主題.
是「共同的土地」.
這次會議的主題.
是「共同的土地」.
這次會議的主題.
是「共同的土地」.
這次會議的主題.
是「共同的土地」.
這次會議的主題.
是「共同的土地」.
這次會議的主題.
是「共同的土地」.
這次會議的主題.
是「共同的土地」.
這次會議的主題.
是「共同的土地」.
這次會議的主題.
是「共同的土地」.
這次會議的主題.
是「共同的土地」.
這次會議的主題.
是「共同的土地」.
這次會議的主題.
是「共同的土地」.
這次會議的主題.
是「共同的土地」.
這次會議的主題.
是「共同的土地」.
這次會議的主題.
是「共同的土地」.
這次會議的主題.
是「共同的土地」.
這次會議的主題.
是「共同的土地」.
這次會議的主題.
是「共同的土地」.
這次會議的主題.
是「共同的土地」.
這次會議的主題.

$^{641}$是「共同的土地」.
這次會議的主題.
是「共同的土地」.
這次會議的主題.
是「共同的土地」.
這次會議的主題.
是「共同的土地」.
這次會議的主題.
是「共同的土地」.
這次會議的主題.
是「共同的土地」.
這次會議的主題.
是「共同的土地」.
這次會議的主題.
是「共同的土地」.
這次會議的主題.
是「共同的土地」.
這次會議的主題.
是「共同的土地」.
這次會議的主題.
是「共同的土地」.
這次會議的主題.
是「共同的土地」.
這次會議的主題.
是「共同的土地」.
這次會議的主題.
是「共同的土地」.
這次會議的主題.
是「共同的土地」.
這次會議的主題.
是「共同的土地」.
這次會議的主題.
是「共同的土地」.
這次會議的主題.
是「共同的土地」.
這次會議的主題.
是「共同的土地」.
這次會議的主題.
是「共同的土地」.
這次會議的主題.

$^{681}$是「共同的土地」.
這次會議的主題.
是「共同的土地」.
這次會議的主題.
是「共同的土地」.
這次會議的主題.
是「共同的土地」.
這次會議的主題.
是「共同的土地」.
這次會議的主題.
是「共同的土地」.
這次會議的主題.
是「共同的土地」.
這次會議的主題.
是「共同的土地」.
這次會議的主題.
是「共同的土地」.
這次會議的主題.
是「共同的土地」.
這次會議的主題.
是「共同的土地」.
這次會議的主題.
是「共同的土地」.
這次會議的主題.
是「共同的土地」.
這次會議的主題.
是「共同的土地」.
這次會議的主題.
是「共同的土地」.
這次會議的主題.
是「共同的土地」.
這次會議的主題.
是「共同的土地」.
這次會議的主題.
是「共同的土地」.
這次會議的主題.
是「共同的土地」.
這次會議的主題.
是「共同的土地」.
這次會議的主題.

$^{721}$是「共同的土地」.
這次會議的主題.
是「共同的土地」.
這次會議的主題.
是「共同的土地」.
這次會議的主題.
是「共同的土地」.
這次會議的主題.
是「共同的土地」.
這次會議的主題.
是「共同的土地」.
這次會議的主題.
是「共同的土地」.
這次會議的主題.
是「共同的土地」.
這次會議的主題.
是「共同的土地」.
這次會議的主題.
是「共同的土地」.
這次會議的主題.
是「共同的土地」.
這次會議的主題.
是「共同的土地」.
這次會議的主題.
是「共同的土地」.
這次會議的主題.
是「共同的土地」.
這次會議的主題.
是「共同的土地」.
這次會議的主題.
是「共同的土地」.
這次會議的主題.
是「共同的土地」.
這次會議的主題.
是「共同的土地」.
這次會議的主題.
是「共同的土地」.
這次會議的主題.
是「共同的土地」.
這次會議的主題.

$^{761}$是「共同的土地」.
這次會議的主題.
是「共同的土地」.
這次會議的主題.
是「共同的土地」.
這次會議的主題.
是「共同的土地」.
這次會議的主題.
是「共同的土地」.
這次會議的主題.
是「共同的土地」.
這次會議的主題.
是「共同的土地」.
這次會議的主題.
是「共同的土地」.
這次會議的主題.
是「共同的土地」.
這次會議的主題.
是「共同的土地」.
這次會議的主題.
是「共同的土地」.
這次會議的主題.
是「共同的土地」.
這次會議的主題.
是「共同的土地」.
這次會議的主題.
是「共同的土地」.
這次會議的主題.
是「共同的土地」.
這次會議的主題.
是「共同的土地」.
這次會議的主題.
是「共同的土地」.
這次會議的主題.
是「共同的土地」.
這次會議的主題.
是「共同的土地」.
這次會議的主題.
是「共同的土地」.
這次會議的主題.

$^{801}$是「共同的土地」.
這次會議的主題.
是「共同的土地」.
這次會議的主題.
是「共同的土地」.
這次會議的主題.
是「共同的土地」.
這次會議的主題.
是「共同的土地」.
這次會議的主題.
是「共同的土地」.
這次會議的主題.
是「共同的土地」.
這次會議的主題.
是「共同的土地」.
這次會議的主題.
是「共同的土地」.
這次會議的主題.
是「共同的土地」.
這次會議的主題.
是「共同的土地」.
這次會議的主題.
是「共同的土地」.
這次會議的主題.
是「共同的土地」.
這次會議的主題.
是「共同的土地」.
這次會議的主題.
是「共同的土地」.
這次會議的主題.
是「共同的土地」.
這次會議的主題.
是「共同的土地」.
這次會議的主題.
是「共同的土地」.
這次會議的主題.
是「共同的土地」.
這次會議的主題.
是「共同的土地」.
這次會議的主題.

$^{841}$是「共同的土地」.
這次會議的主題.
是「共同的土地」.
這次會議的主題.
是「共同的土地」.
這次會議的主題.
是「共同的土地」.
這次會議的主題.
是「共同的土地」.
這次會議的主題.
是「共同的土地」.
這次會議的主題.
是「共同的土地」.
這次會議的主題.
是「共同的土地」.
這次會議的主題.
是「共同的土地」.
這次會議的主題.
是「共同的土地」.
這次會議的主題.
是「共同的土地」.
這次會議的主題.
是「共同的土地」.
這次會議的主題.
是「共同的土地」.
這次會議的主題.
是「共同的土地」.
這次會議的主題.
是「共同的土地」.
這次會議的主題.
是「共同的土地」.
這次會議的主題.
是「共同的土地」.
這次會議的主題.
是「共同的土地」.
這次會議的主題.
是「共同的土地」.
這次會議的主題.
是「共同的土地」.
這次會議的主題.

$^{881}$是「共同的土地」.
這次會議的主題.
是「共同的土地」.
這次會議的主題.
是「共同的土地」.
這次會議的主題.
是「共同的土地」.
這次會議的主題.
是「共同的土地」.
這次會議的主題.
是「共同的土地」.
這次會議的主題.
是「共同的土地」.
這次會議的主題.
是「共同的土地」.
這次會議的主題.
是「共同的土地」.
這次會議的主題.
是「共同的土地」.
這次會議的主題.
是「共同的土地」.
這次會議的主題.
是「共同的土地」.
這次會議的主題.
是「共同的土地」.
這次會議的主題.
是「共同的土地」.
這次會議的主題.
是「共同的土地」.
這次會議的主題.
是「共同的土地」.
這次會議的主題.
是「共同的土地」.
這次會議的主題.
是「共同的土地」.
這次會議的主題.
是「共同的土地」.
這次會議的主題.
是「共同的土地」.
這次會議的主題.

$^{921}$是「共同的土地」.
這次會議的主題.
是「共同的土地」.
這次會議的主題.
是「共同的土地」.
這次會議的主題.
是「共同的土地」.
這次會議的主題.
是「共同的土地」.
這次會議的主題.
是「共同的土地」.
這次會議的主題.
是「共同的土地」.
這次會議的主題.
是「共同的土地」.
這次會議的主題.
是「共同的土地」.
這次會議的主題.
是「共同的土地」.
這次會議的主題.
是「共同的土地」.
這次會議的主題.
是「共同的土地」.
這次會議的主題.
是「共同的土地」.
這次會議的主題.
是「共同的土地」.
這次會議的主題.
是「共同的土地」.
這次會議的主題.
是「共同的土地」.
這次會議的主題.
是「共同的土地」.
這次會議的主題.
是「共同的土地」.
這次會議的主題.
是「共同的土地」.
這次會議的主題.
是「共同的土地」.
這次會議的主題.

$^{961}$是「共同的土地」.
這次會議的主題.
是「共同的土地」.
這次會議的主題.
是「共同的土地」.
這次會議的主題.
是「共同的土地」.
這次會議的主題.
是「共同的土地」.
這次會議的主題.
是「共同的土地」.
這次會議的主題.
是「共同的土地」.
這次會議的主題.
是「共同的土地」.
這次會議的主題.
是「共同的土地」.
這次會議的主題.
是「共同的土地」.
這次會議的主題.
是「共同的土地」.
這次會議的主題.
是「共同的土地」.
這次會議的主題.
是「共同的土地」.
這次會議的主題.
是「共同的土地」.
這次會議的主題.
是「共同的土地」.
這次會議的主題.
是「共同的土地」.
這次會議的主題.
是「共同的土地」.
這次會議的主題.
是「共同的土地」.
這次會議的主題.
是「共同的土地」.
這次會議的主題.
是「共同的土地」.
這次會議的主題.

$^{1001}$是「共同的土地」.
這次會議的主題.
是「共同的土地」.
這次會議的主題.
是「共同的土地」.
這次會議的主題.
是「共同的土地」.
這次會議的主題.
是「共同的土地」.
這次會議的主題.
是「共同的土地」.
這次會議的主題.
是「共同的土地」.
這次會議的主題.
是「共同的土地」.
這次會議的主題.
是「共同的土地」.
這次會議的主題.
是「共同的土地」.
這次會議的主題.
是「共同的土地」.
這次會議的主題.
是「共同的土地」.
這次會議的主題.
是「共同的土地」.
這次會議的主題.
是「共同的土地」.
這次會議的主題.
是「共同的土地」.
這次會議的主題.
是「共同的土地」.
這次會議的主題.
是「共同的土地」.
這次會議的主題.
是「共同的土地」.
這次會議的主題.
是「共同的土地」.
這次會議的主題.
是「共同的土地」.
這次會議的主題.

$^{1041}$是「共同的土地」.
這次會議的主題.
是「共同的土地」.
這次會議的主題.
是「共同的土地」.
這次會議的主題.
是「共同的土地」.
這次會議的主題.
是「共同的土地」.
這次會議的主題.
是「共同的土地」.
這次會議的主題.
是「共同的土地」.
這次會議的主題.
是「共同的土地」.
這次會議的主題.
是「共同的土地」.
這次會議的主題.
是「共同的土地」.
這次會議的主題.
是「共同的土地」.
這次會議的主題.
是「共同的土地」.
這次會議的主題.
是「共同的土地」.
這次會議的主題.
是「共同的土地」.
這次會議的主題.
是「共同的土地」.
這次會議的主題.
是「共同的土地」.
這次會議的主題.
是「共同的土地」.
這次會議的主題.
是「共同的土地」.
這次會議的主題.
是「共同的土地」.
這次會議的主題.
是「共同的土地」.
這次會議的主題.

$^{1081}$是「共同的土地」.
這次會議的主題.
是「共同的土地」.
這次會議的主題.
是「共同的土地」.
這次會議的主題.
是「共同的土地」.
這次會議的主題.
是「共同的土地」.
這次會議的主題.
是「共同的土地」.
這次會議的主題.
是「共同的土地」.
這次會議的主題.
是「共同的土地」.
這次會議的主題.
是「共同的土地」.
這次會議的主題.
是「共同的土地」.
這次會議的主題.
是「共同的土地」.
這次會議的主題.
是「共同的土地」.
這次會議的主題.
是「共同的土地」.
這次會議的主題.
是「共同的土地」.
這次會議的主題.
是「共同的土地」.
這次會議的主題.
是「共同的土地」.
這次會議的主題.
是「共同的土地」.
這次會議的主題.
是「共同的土地」.
這次會議的主題.
是「共同的土地」.
這次會議的主題.
是「共同的土地」.
這次會議的主題.

$^{1121}$是「共同的土地」.
這次會議的主題.
是「共同的土地」.
這次會議的主題.
是「共同的土地」.
這次會議的主題.
是「共同的土地」.
這次會議的主題.
是「共同的土地」.
這次會議的主題.
是「共同的土地」.
這次會議的主題.
是「共同的土地」.
這次會議的主題.
是「共同的土地」.
這次會議的主題.
是「共同的土地」.
這次會議的主題.
是「共同的土地」.
這次會議的主題.
是「共同的土地」.
這次會議的主題.
是「共同的土地」.
這次會議的主題.
是「共同的土地」.
這次會議的主題.
是「共同的土地」.
這次會議的主題.
是「共同的土地」.
這次會議的主題.
是「共同的土地」.
這次會議的主題.
是「共同的土地」.
這次會議的主題.
是「共同的土地」.
這次會議的主題.
是「共同的土地」.
這次會議的主題.
是「共同的土地」.
這次會議的主題.

$^{1161}$是「共同的土地」.
這次會議的主題.
是「共同的土地」.
這次會議的主題.
是「共同的土地」.
這次會議的主題.
是「共同的土地」.
這次會議的主題.
是「共同的土地」.
這次會議的主題.
是「共同的土地」.
這次會議的主題.
是「共同的土地」.
這次會議的主題.
是「共同的土地」.
這次會議的主題.
是「共同的土地」.
這次會議的主題.
是「共同的土地」.
這次會議的主題.
是「共同的土地」.
這次會議的主題.
是「共同的土地」.
這次會議的主題.
是「共同的土地」.
這次會議的主題.
是「共同的土地」.
這次會議的主題.
是「共同的土地」.
這次會議的主題.
是「共同的土地」.
這次會議的主題.
是「共同的土地」.
這次會議的主題.
是「共同的土地」.
這次會議的主題.
是「共同的土地」.
這次會議的主題.
是「共同的土地」.
這次會議的主題.

$^{1201}$是「共同的土地」.
這次會議的主題.
是「共同的土地」.
這次會議的主題.
是「共同的土地」.
這次會議的主題.
是「共同的土地」.
這次會議的主題.
是「共同的土地」.
這次會議的主題.
是「共同的土地」.
這次會議的主題.
是「共同的土地」.
這次會議的主題.
是「共同的土地」.
這次會議的主題.
是「共同的土地」.
這次會議的主題.
是「共同的土地」.
這次會議的主題.
是「共同的土地」.
這次會議的主題.
是「共同的土地」.
這次會議的主題.
是「共同的土地」.
這次會議的主題.
是「共同的土地」.
這次會議的主題.
是「共同的土地」.
這次會議的主題.
是「共同的土地」.
這次會議的主題.
是「共同的土地」.
這次會議的主題.
是「共同的土地」.
這次會議的主題.
是「共同的土地」.
這次會議的主題.
是「共同的土地」.
這次會議的主題.

$^{1241}$是「共同的土地」.
這次會議的主題.
是「共同的土地」.
這次會議的主題.
是「共同的土地」.
這次會議的主題.
是「共同的土地」.
這次會議的主題.
是「共同的土地」.
這次會議的主題.
是「共同的土地」.
這次會議的主題.
是「共同的土地」.
這次會議的主題.
是「共同的土地」.
這次會議的主題.
是「共同的土地」.
這次會議的主題.
是「共同的土地」.
這次會議的主題.
是「共同的土地」.
這次會議的主題.
是「共同的土地」.
這次會議的主題.
是「共同的土地」.
這次會議的主題.
是「共同的土地」.
這次會議的主題.
是「共同的土地」.
這次會議的主題.
是「共同的土地」.
這次會議的主題.
是「共同的土地」.
這次會議的主題.
是「共同的土地」.
這次會議的主題.
是「共同的土地」.
這次會議的主題.
是「共同的土地」.
這次會議的主題.

$^{1281}$是「共同的土地」.
這次會議的主題.
是「共同的土地」.
這次會議的主題.
是「共同的土地」.
這次會議的主題.
是「共同的土地」.
這次會議的主題.
是「共同的土地」.
這次會議的主題.
是「共同的土地」.
這次會議的主題.
是「共同的土地」.
這次會議的主題.
是「共同的土地」.
這次會議的主題.
是「共同的土地」.
這次會議的主題.
是「共同的土地」.
這次會議的主題.
是「共同的土地」.
這次會議的主題.
是「共同的土地」.
這次會議的主題.
是「共同的土地」.
這次會議的主題.
是「共同的土地」.
這次會議的主題.
是「共同的土地」.
這次會議的主題.
是「共同的土地」.
這次會議的主題.
是「共同的土地」.
這次會議的主題.
是「共同的土地」.
這次會議的主題.
是「共同的土地」.
這次會議的主題.
是「共同的土地」.
這次會議的主題.

$^{1321}$是「共同的土地」.
這次會議的主題.
是「共同的土地」.
這次會議的主題.
是「共同的土地」.
這次會議的主題.
是「共同的土地」.
這次會議的主題.
是「共同的土地」.
這次會議的主題.
是「共同的土地」.
這次會議的主題.
是「共同的土地」.
這次會議的主題.
是「共同的土地」.
這次會議的主題.
是「共同的土地」.
這次會議的主題.
是「共同的土地」.
這次會議的主題.
是「共同的土地」.
這次會議的主題.
是「共同的土地」.
這次會議的主題.
是「共同的土地」.
這次會議的主題.
是「共同的土地」.
這次會議的主題.
是「共同的土地」.
這次會議的主題.
是「共同的土地」.
這次會議的主題.
是「共同的土地」.
這次會議的主題.
是「共同的土地」.
這次會議的主題.
是「共同的土地」.
這次會議的主題.
是「共同的土地」.
這次會議的主題.

$^{1361}$是「共同的土地」.
這次會議的主題.
是「共同的土地」.
這次會議的主題.
是「共同的土地」.
這次會議的主題.
是「共同的土地」.
這次會議的主題.
是「共同的土地」.
這次會議的主題.
是「共同的土地」.
這次會議的主題.
是「共同的土地」.
這次會議的主題.
是「共同的土地」.
這次會議的主題.
是「共同的土地」.
這次會議的主題.
是「共同的土地」.
這次會議的主題.
是「共同的土地」.
這次會議的主題.
是「共同的土地」.
這次會議的主題.
是「共同的土地」.
這次會議的主題.
是「共同的土地」.
這次會議的主題.
是「共同的土地」.
這次會議的主題.
是「共同的土地」.
這次會議的主題.
是「共同的土地」.
這次會議的主題.
是「共同的土地」.
這次會議的主題.
是「共同的土地」.
這次會議的主題.
是「共同的土地」.
這次會議的主題.

$^{1401}$是「共同的土地」.
這次會議的主題.
是「共同的土地」.
這次會議的主題.
是「共同的土地」.
這次會議的主題.
是「共同的土地」.
這次會議的主題.
是「共同的土地」.
這次會議的主題.
是「共同的土地」.
這次會議的主題.
是「共同的土地」.
這次會議的主題.
是「共同的土地」.
這次會議的主題.
是「共同的土地」.
這次會議的主題.
是「共同的土地」.
這次會議的主題.
是「共同的土地」.
這次會議的主題.
是「共同的土地」.
這次會議的主題.
是「共同的土地」.
這次會議的主題.
是「共同的土地」.
這次會議的主題.
是「共同的土地」.
這次會議的主題.
是「共同的土地」.
這次會議的主題.
是「共同的土地」.
這次會議的主題.
是「共同的土地」.
這次會議的主題.
是「共同的土地」.
這次會議的主題.
是「共同的土地」.
這次會議的主題.

$^{1441}$是「共同的土地」.
這次會議的主題.
是「共同的土地」.
這次會議的主題.
是「共同的土地」.
這次會議的主題.
是「共同的土地」.
這次會議的主題.
是「共同的土地」.
這次會議的主題.
是「共同的土地」.
這次會議的主題.
是「共同的土地」.
這次會議的主題.
是「共同的土地」.
這次會議的主題.
是「共同的土地」.
這次會議的主題.
是「共同的土地」.
這次會議的主題.
是「共同的土地」.
這次會議的主題.
是「共同的土地」.
這次會議的主題.
是「共同的土地」.
這次會議的主題.
是「共同的土地」.
這次會議的主題.
是「共同的土地」.
這次會議的主題.
是「共同的土地」.
這次會議的主題.
是「共同的土地」.
這次會議的主題.
是「共同的土地」.
這次會議的主題.
是「共同的土地」.
這次會議的主題.
是「共同的土地」.
這次會議的主題.

$^{1481}$是「共同的土地」.
這次會議的主題.
是「共同的土地」.
這次會議的主題.
是「共同的土地」.
這次會議的主題.
是「共同的土地」.
這次會議的主題.
是「共同的土地」.
這次會議的主題.
是「共同的土地」.
這次會議的主題.
是「共同的土地」.
這次會議的主題.
是「共同的土地」.
這次會議的主題.
是「共同的土地」.
這次會議的主題.
是「共同的土地」.
這次會議的主題.
是「共同的土地」.
這次會議的主題.
是「共同的土地」.
這次會議的主題.
是「共同的土地」.
這次會議的主題.
是「共同的土地」.
這次會議的主題.
是「共同的土地」.
這次會議的主題.
是「共同的土地」.
這次會議的主題.
是「共同的土地」.
這次會議的主題.
是「共同的土地」.
這次會議的主題.
是「共同的土地」.
這次會議的主題.
是「共同的土地」.
這次會議的主題.

$^{1521}$是「共同的土地」.
這次會議的主題.
是「共同的土地」.
這次會議的主題.
是「共同的土地」.
這次會議的主題.
是「共同的土地」.
這次會議的主題.
是「共同的土地」.
這次會議的主題.
是「共同的土地」.
這次會議的主題.
是「共同的土地」.
這次會議的主題.
是「共同的土地」.
這次會議的主題.
是「共同的土地」.
這次會議的主題.
是「共同的土地」.
這次會議的主題.
是「共同的土地」.
這次會議的主題.
是「共同的土地」.
這次會議的主題.
是「共同的土地」.
這次會議的主題.
是「共同的土地」.
這次會議的主題.
是「共同的土地」.
這次會議的主題.
是「共同的土地」.
這次會議的主題.
是「共同的土地」.
這次會議的主題.
是「共同的土地」.
這次會議的主題.
是「共同的土地」.
這次會議的主題.
是「共同的土地」.
這次會議的主題.

$^{1561}$是「共同的土地」.
這次會議的主題.
是「共同的土地」.
這次會議的主題.
是「共同的土地」.
這次會議的主題.
是「共同的土地」.
這次會議的主題.
是「共同的土地」.
這次會議的主題.
是「共同的土地」.
這次會議的主題.
是「共同的土地」.
這次會議的主題.
是「共同的土地」.
這次會議的主題.
是「共同的土地」.
這次會議的主題.
是「共同的土地」.
這次會議的主題.
是「共同的土地」.
這次會議的主題.
是「共同的土地」.
這次會議的主題.
是「共同的土地」.
這次會議的主題.
是「共同的土地」.
這次會議的主題.
是「共同的土地」.
這次會議的主題.
是「共同的土地」.
這次會議的主題.
是「共同的土地」.
這次會議的主題.
是「共同的土地」.
這次會議的主題.
是「共同的土地」.
這次會議的主題.
是「共同的土地」.
這次會議的主題.

$^{1601}$是「共同的土地」.
這次會議的主題.
是「共同的土地」.
這次會議的主題.
是「共同的土地」.
這次會議的主題.
是「共同的土地」.
這次會議的主題.
是「共同的土地」.
這次會議的主題.
是「共同的土地」.
這次會議的主題.
是「共同的土地」.
這次會議的主題.
是「共同的土地」.
這次會議的主題.
是「共同的土地」.
這次會議的主題.
是「共同的土地」.
這次會議的主題.
是「共同的土地」.
這次會議的主題.
是「共同的土地」.
這次會議的主題.
是「共同的土地」.
這次會議的主題.
是「共同的土地」.
這次會議的主題.
是「共同的土地」.
這次會議的主題.
是「共同的土地」.
這次會議的主題.
是「共同的土地」.
這次會議的主題.
是「共同的土地」.
這次會議的主題.
是「共同的土地」.
這次會議的主題.
是「共同的土地」.
這次會議的主題.

$^{1641}$是「共同的土地」.
這次會議的主題.
是「共同的土地」.
這次會議的主題.
是「共同的土地」.
這次會議的主題.
是「共同的土地」.
這次會議的主題.
是「共同的土地」.
這次會議的主題.
是「共同的土地」.
這次會議的主題.
是「共同的土地」.
這次會議的主題.
是「共同的土地」.
這次會議的主題.
是「共同的土地」.
這次會議的主題.
是「共同的土地」.
這次會議的主題.
是「共同的土地」.
這次會議的主題.
是「共同的土地」.
這次會議的主題.
是「共同的土地」.
這次會議的主題.
是「共同的土地」.
這次會議的主題.
是「共同的土地」.
這次會議的主題.
是「共同的土地」.
這次會議的主題.
是「共同的土地」.
這次會議的主題.
是「共同的土地」.
這次會議的主題.
是「共同的土地」.
這次會議的主題.
是「共同的土地」.
這次會議的主題.

$^{1681}$是「共同的土地」.
這次會議的主題.
是「共同的土地」.
這次會議的主題.
是「共同的土地」.
這次會議的主題.
是「共同的土地」.
這次會議的主題.
是「共同的土地」.
這次會議的主題.
是「共同的土地」.
這次會議的主題.
是「共同的土地」.
這次會議的主題.
是「共同的土地」.
這次會議的主題.
是「共同的土地」.
這次會議的主題.
是「共同的土地」.
這次會議的主題.
是「共同的土地」.
這次會議的主題.
是「共同的土地」.
這次會議的主題.
是「共同的土地」.
這次會議的主題.
是「共同的土地」.
這次會議的主題.
是「共同的土地」.
這次會議的主題.
是「共同的土地」.
這次會議的主題.
是「共同的土地」.
這次會議的主題.
是「共同的土地」.
這次會議的主題.
是「共同的土地」.
這次會議的主題.
是「共同的土地」.
這次會議的主題.

$^{1721}$是「共同的土地」.
這次會議的主題.
是「共同的土地」.
這次會議的主題.
是「共同的土地」.
這次會議的主題.
是「共同的土地」.
這次會議的主題.
是「共同的土地」.
這次會議的主題.
是「共同的土地」.
這次會議的主題.
是「共同的土地」.
這次會議的主題.
是「共同的土地」.
這次會議的主題.
是「共同的土地」.
這次會議的主題.
是「共同的土地」.
這次會議的主題.
是「共同的土地」.
這次會議的主題.
是「共同的土地」.
這次會議的主題.
是「共同的土地」.
這次會議的主題.
是「共同的土地」.
這次會議的主題.
是「共同的土地」.
這次會議的主題.
是「共同的土地」.
這次會議的主題.
是「共同的土地」.
這次會議的主題.
是「共同的土地」.
這次會議的主題.
是「共同的土地」.
這次會議的主題.
是「共同的土地」.
這次會議的主題.

$^{1761}$是「共同的土地」.
這次會議的主題.
是「共同的土地」.
這次會議的主題.
是「共同的土地」.
這次會議的主題.
是「共同的土地」.
這次會議的主題.
是「共同的土地」.
這次會議的主題.
是「共同的土地」.
這次會議的主題.
是「共同的土地」.
這次會議的主題.
是「共同的土地」.
這次會議的主題.
是「共同的土地」.
這次會議的主題.
是「共同的土地」.
這次會議的主題.
是「共同的土地」.
這次會議的主題.
是「共同的土地」.
這次會議的主題.
是「共同的土地」.
這次會議的主題.
是「共同的土地」.
這次會議的主題.
是「共同的土地」.
這次會議的主題.
是「共同的土地」.
這次會議的主題.
是「共同的土地」.
這次會議的主題.
是「共同的土地」.
這次會議的主題.
是「共同的土地」.
這次會議的主題.
是「共同的土地」.
這次會議的主題.

$^{1801}$是「共同的土地」.
這次會議的主題.
是「共同的土地」.
這次會議的主題.
是「共同的土地」.
這次會議的主題.
是「共同的土地」.
這次會議的主題.
是「共同的土地」.
這次會議的主題.
是「共同的土地」.
這次會議的主題.
是「共同的土地」.
這次會議的主題.
是「共同的土地」.
這次會議的主題.
是「共同的土地」.
這次會議的主題.
是「共同的土地」.
這次會議的主題.
是「共同的土地」.
這次會議的主題.
是「共同的土地」.
這次會議的主題.
是「共同的土地」.
這次會議的主題.
是「共同的土地」.
這次會議的主題.
是「共同的土地」.
這次會議的主題.
是「共同的土地」.
這次會議的主題.
是「共同的土地」.
這次會議的主題.
是「共同的土地」.
這次會議的主題.
是「共同的土地」.
這次會議的主題.
是「共同的土地」.
這次會議的主題.

$^{1841}$是「共同的土地」.
這次會議的主題.
是「共同的土地」.
這次會議的主題.
是「共同的土地」.
這次會議的主題.
是「共同的土地」.
這次會議的主題.
是「共同的土地」.
這次會議的主題.
是「共同的土地」.
這次會議的主題.
是「共同的土地」.
這次會議的主題.
是「共同的土地」.
這次會議的主題.
是「共同的土地」.
這次會議的主題.
是「共同的土地」.
這次會議的主題.
是「共同的土地」.
這次會議的主題.
是「共同的土地」.
這次會議的主題.
是「共同的土地」.
這次會議的主題.
是「共同的土地」.
這次會議的主題.
是「共同的土地」.
這次會議的主題.
是「共同的土地」.
這次會議的主題.
是「共同的土地」.
這次會議的主題.
是「共同的土地」.
這次會議的主題.
是「共同的土地」.
這次會議的主題.
是「共同的土地」.
這次會議的主題.

$^{1881}$是「共同的土地」.
這次會議的主題.
是「共同的土地」.
這次會議的主題.
是「共同的土地」.
這次會議的主題.
是「共同的土地」.
這次會議的主題.
是「共同的土地」.
這次會議的主題.
是「共同的土地」.
這次會議的主題.
是「共同的土地」.
這次會議的主題.
是「共同的土地」.
這次會議的主題.
是「共同的土地」.
這次會議的主題.
是「共同的土地」.
這次會議的主題.
是「共同的土地」.
這次會議的主題.
是「共同的土地」.
這次會議的主題.
是「共同的土地」.
這次會議的主題.
是「共同的土地」.
這次會議的主題.
是「共同的土地」.
這次會議的主題.
是「共同的土地」.
這次會議的主題.
是「共同的土地」.
這次會議的主題.
是「共同的土地」.
這次會議的主題.
是「共同的土地」.
這次會議的主題.
是「共同的土地」.
這次會議的主題.

$^{1921}$是「共同的土地」.
這次會議的主題.
是「共同的土地」.
這次會議的主題.
是「共同的土地」.
這次會議的主題.
是「共同的土地」.
這次會議的主題.
是「共同的土地」.
這次會議的主題.
是「共同的土地」.
這次會議的主題.
是「共同的土地」.
這次會議的主題.
是「共同的土地」.
這次會議的主題.
是「共同的土地」.
這次會議的主題.
是「共同的土地」.
這次會議的主題.
是「共同的土地」.
這次會議的主題.
是「共同的土地」.
這次會議的主題.
是「共同的土地」.
這次會議的主題.
是「共同的土地」.
這次會議的主題.
是「共同的土地」.
這次會議的主題.
是「共同的土地」.
這次會議的主題.
是「共同的土地」.
這次會議的主題.
是「共同的土地」.
這次會議的主題.
是「共同的土地」.
這次會議的主題.
是「共同的土地」.
這次會議的主題.

$^{1961}$是「共同的土地」.
這次會議的主題.
是「共同的土地」.
這次會議的主題.
是「共同的土地」.
這次會議的主題.
是「共同的土地」.
這次會議的主題.
是「共同的土地」.
這次會議的主題.
是「共同的土地」.
這次會議的主題.
是「共同的土地」.
這次會議的主題.
是「共同的土地」.
這次會議的主題.
是「共同的土地」.
這次會議的主題.
是「共同的土地」.
這次會議的主題.
是「共同的土地」.
這次會議的主題.
是「共同的土地」.
這次會議的主題.
是「共同的土地」.
這次會議的主題.
是「共同的土地」.
這次會議的主題.
是「共同的土地」.
這次會議的主題.
是「共同的土地」.
這次會議的主題.
是「共同的土地」.
這次會議的主題.
是「共同的土地」.
這次會議的主題.
是「共同的土地」.
這次會議的主題.
是「共同的土地」.
這次會議的主題.

$^{2001}$是「共同的土地」.
這次會議的主題.
是「共同的土地」.
這次會議的主題.
是「共同的土地」.
這次會議的主題.
是「共同的土地」.
這次會議的主題.
是「共同的土地」.
這次會議的主題.
是「共同的土地」.
這次會議的主題.
是「共同的土地」.
這次會議的主題.
是「共同的土地」.
這次會議的主題.
是「共同的土地」.
這次會議的主題.
是「共同的土地」.
這次會議的主題.
是「共同的土地」.
這次會議的主題.
是「共同的土地」.
這次會議的主題.
是「共同的土地」.
這次會議的主題.
是「共同的土地」.
這次會議的主題.
是「共同的土地」.
這次會議的主題.
是「共同的土地」.
這次會議的主題.
是「共同的土地」.
這次會議的主題.
是「共同的土地」.
這次會議的主題.
是「共同的土地」.
這次會議的主題.
是「共同的土地」.
這次會議的主題.

$^{2041}$是「共同的土地」.
這次會議的主題.
是「共同的土地」.
這次會議的主題.
是「共同的土地」.
這次會議的主題.
是「共同的土地」.
這次會議的主題.
是「共同的土地」.
這次會議的主題.
是「共同的土地」.
這次會議的主題.
是「共同的土地」.
這次會議的主題.
是「共同的土地」.
這次會議的主題.
是「共同的土地」.
這次會議的主題.
是「共同的土地」.
這次會議的主題.
是「共同的土地」.
這次會議的主題.
是「共同的土地」.
這次會議的主題.
是「共同的土地」.
這次會議的主題.
是「共同的土地」.
這次會議的主題.
是「共同的土地」.
這次會議的主題.
是「共同的土地」.
這次會議的主題.
是「共同的土地」.
這次會議的主題.
是「共同的土地」.
這次會議的主題.
是「共同的土地」.
這次會議的主題.
是「共同的土地」.
這次會議的主題.

$^{2081}$是「共同的土地」.
這次會議的主題.
是「共同的土地」.
這次會議的主題.
是「共同的土地」.
這次會議的主題.
是「共同的土地」.
這次會議的主題.
是「共同的土地」.
這次會議的主題.
是「共同的土地」.
這次會議的主題.
是「共同的土地」.
這次會議的主題.
是「共同的土地」.
這次會議的主題.
是「共同的土地」.
這次會議的主題.
是「共同的土地」.
這次會議的主題.
是「共同的土地」.
這次會議的主題.
是「共同的土地」.
這次會議的主題.
是「共同的土地」.
這次會議的主題.
是「共同的土地」.
這次會議的主題.
是「共同的土地」.
這次會議的主題.
是「共同的土地」.
這次會議的主題.
是「共同的土地」.
這次會議的主題.
是「共同的土地」.
這次會議的主題.
是「共同的土地」.
這次會議的主題.
是「共同的土地」.
這次會議的主題.

$^{2121}$是「共同的土地」.
這次會議的主題.
是「共同的土地」.
這次會議的主題.
是「共同的土地」.
這次會議的主題.
是「共同的土地」.
這次會議的主題.
是「共同的土地」.
這次會議的主題.
是「共同的土地」.
這次會議的主題.
是「共同的土地」.
這次會議的主題.
是「共同的土地」.
這次會議的主題.
是「共同的土地」.
這次會議的主題.
是「共同的土地」.
這次會議的主題.
是「共同的土地」.
這次會議的主題.
是「共同的土地」.
這次會議的主題.
是「共同的土地」.
這次會議的主題.
是「共同的土地」.
這次會議的主題.
是「共同的土地」.
這次會議的主題.
是「共同的土地」.
這次會議的主題.
是「共同的土地」.
這次會議的主題.
是「共同的土地」.
這次會議的主題.
是「共同的土地」.
這次會議的主題.
是「共同的土地」.
這次會議的主題.

$^{2161}$是「共同的土地」.
這次會議的主題.
是「共同的土地」.
這次會議的主題.
是「共同的土地」.
這次會議的主題.
是「共同的土地」.
這次會議的主題.
是「共同的土地」.
這次會議的主題.
是「共同的土地」.
這次會議的主題.
是「共同的土地」.
這次會議的主題.
是「共同的土地」.
這次會議的主題.
是「共同的土地」.
這次會議的主題.
是「共同的土地」.
這次會議的主題.
是「共同的土地」.
這次會議的主題.
是「共同的土地」.
這次會議的主題.
是「共同的土地」.
這次會議的主題.
是「共同的土地」.
這次會議的主題.
是「共同的土地」.
這次會議的主題.
是「共同的土地」.
這次會議的主題.
是「共同的土地」.
這次會議的主題.
是「共同的土地」.
這次會議的主題.
是「共同的土地」.
這次會議的主題.
是「共同的土地」.
這次會議的主題.

$^{2201}$是「共同的土地」.
這次會議的主題.
是「共同的土地」.
這次會議的主題.
是「共同的土地」.
這次會議的主題.
是「共同的土地」.
這次會議的主題.
是「共同的土地」.
這次會議的主題.
是「共同的土地」.
這次會議的主題.
是「共同的土地」.
這次會議的主題.
是「共同的土地」.
這次會議的主題.
是「共同的土地」.
這次會議的主題.
是「共同的土地」.
這次會議的主題.
是「共同的土地」.
這次會議的主題.
是「共同的土地」.
這次會議的主題.
是「共同的土地」.
這次會議的主題.
是「共同的土地」.
這次會議的主題.
是「共同的土地」.
這次會議的主題.
是「共同的土地」.
這次會議的主題.
是「共同的土地」.
這次會議的主題.
是「共同的土地」.
這次會議的主題.
是「共同的土地」.
這次會議的主題.
是「共同的土地」.
這次會議的主題.

$^{2241}$是「共同的土地」.
這次會議的主題.
是「共同的土地」.
這次會議的主題.
是「共同的土地」.
這次會議的主題.
是「共同的土地」.
這次會議的主題.
是「共同的土地」.
這次會議的主題.
是「共同的土地」.
這次會議的主題.
是「共同的土地」.
這次會議的主題.
是「共同的土地」.
這次會議的主題.
是「共同的土地」.
這次會議的主題.
是「共同的土地」.
這次會議的主題.
是「共同的土地」.
這次會議的主題.
是「共同的土地」.
這次會議的主題.
是「共同的土地」.
這次會議的主題.
是「共同的土地」.
這次會議的主題.
是「共同的土地」.
這次會議的主題.
是「共同的土地」.
這次會議的主題.
是「共同的土地」.
這次會議的主題.
是「共同的土地」.
這次會議的主題.
是「共同的土地」.
這次會議的主題.
是「共同的土地」.
這次會議的主題.

$^{2281}$是「共同的土地」.
這次會議的主題.
是「共同的土地」.
這次會議的主題.
是「共同的土地」.
這次會議的主題.
是「共同的土地」.
這次會議的主題.
是「共同的土地」.
這次會議的主題.
是「共同的土地」.
這次會議的主題.
是「共同的土地」.
這次會議的主題.
是「共同的土地」.
這次會議的主題.
是「共同的土地」.
這次會議的主題.
是「共同的土地」.
這次會議的主題.
是「共同的土地」.
這次會議的主題.
是「共同的土地」.
這次會議的主題.
是「共同的土地」.
這次會議的主題.
是「共同的土地」.
這次會議的主題.
是「共同的土地」.
這次會議的主題.
是「共同的土地」.
這次會議的主題.
是「共同的土地」.
這次會議的主題.
是「共同的土地」.
這次會議的主題.
是「共同的土地」.
這次會議的主題.
是「共同的土地」.
這次會議的主題.

$^{2321}$是「共同的土地」.
這次會議的主題.
是「共同的土地」.
這次會議的主題.
是「共同的土地」.
這次會議的主題.
是「共同的土地」.
這次會議的主題.
是「共同的土地」.
這次會議的主題.
是「共同的土地」.
這次會議的主題.
是「共同的土地」.
這次會議的主題.
是「共同的土地」.
這次會議的主題.
是「共同的土地」.
這次會議的主題.
是「共同的土地」.
這次會議的主題.
是「共同的土地」.
這次會議的主題.
是「共同的土地」.
這次會議的主題.
是「共同的土地」.
這次會議的主題.
是「共同的土地」.
這次會議的主題.
是「共同的土地」.
這次會議的主題.
是「共同的土地」.
這次會議的主題.
是「共同的土地」.
\newpage



\section{}
\label{sec:7cxD3Fsxces}
\textbf{Towards a Shared Land Theology: Palestinian Christian Reading of the Land Promises (Q&A)}
\newline
\newline
連結: \href{https://youtube.com/watch?v=7cxD3Fsxces}{\texttt{ https://youtube.com/watch?v=7cxD3Fsxces}} ~~~~ 語音日期: 2023-11-25 
\newline
\newline
\hyperref[sec:QNbcTFot66g]{\small{< < < PREV SERMON < < <}}
~
\hyperref[sec:index]{\small{[返主目錄]}}
~
\hyperref[sec:4mzmfSAacaU]{\small{> > > NEXT SERMON > > >}}
\newline
\newline
$^{1}$所以我們要做的就是要讓他們知道我們在做什麼..
(英文).
(掌聲).
(英文).
\newpage



\section{}
\label{sec:4mzmfSAacaU}
\textbf{Unpacking the Credo of Christ through the Temptation of Jesus (盧允晞老師)}
\newline
\newline
連結: \href{https://youtube.com/watch?v=4mzmfSAacaU}{\texttt{ https://youtube.com/watch?v=4mzmfSAacaU}} ~~~~ 語音日期: 2022-12-17 
\newline
\newline
\hyperref[sec:7cxD3Fsxces]{\small{< < < PREV SERMON < < <}}
~
\hyperref[sec:index]{\small{[返主目錄]}}
~
\hyperref[sec:2WDL8N3ZBFk]{\small{> > > NEXT SERMON > > >}}
\newline
\newline
$^{1}$(廣東話).
我們現在會從翡翠台變成明珠台.
這個講座會用英文來講的.
請大家不要介意.
(英文).
(字幕由 Amara.org 社群提供).
\newpage



\section{}
\label{sec:2WDL8N3ZBFk}
\textbf{「受苦的耶穌——受苦教會的使命實踐」座談會}
\newline
\newline
連結: \href{https://youtube.com/watch?v=2WDL8N3ZBFk}{\texttt{ https://youtube.com/watch?v=2WDL8N3ZBFk}} ~~~~ 語音日期: 2023-09-30 
\newline
\newline
\hyperref[sec:4mzmfSAacaU]{\small{< < < PREV SERMON < < <}}
~
\hyperref[sec:index]{\small{[返主目錄]}}
~
\hyperref[sec:_SfD9ovSJ1E]{\small{> > > NEXT SERMON > > >}}
\newline
\newline
$^{1}$(主席).
(主席) 各位弟兄姊妹 晚安.
(主席).
(主席) 代表中國神學研究院歡迎大家參加.
(主席) 今天晚上舉行的四福音神學座談會.
(主席) 受苦的耶穌受苦教會的使命實踐.
(主席) 我是今晚座談會的主持葉希賢博士.
(主席) 今晚座談會會由忠臣四位新約老師.
(主席) 透過互動交流.
(主席) 讓我們認識四福音各自對耶穌基督的表述特色.
(主席) 從而盼望我們能夠更豐富體會.
(主席) 教會如何在受苦的處境下實踐福音的使命.
(主席) 今次座談會會有三個互動交流環節.
(主席) 每個交流環節之後.
(主席) 我會讓台下弟兄姊妹有發問的時間.
(主席) 而座談會結束之前.
(主席) 我亦會邀請每位老師為他負責的書卷作出總結.
(主席) 事不宜遲,我介紹今晚為我們有講解的四位老師.
(主席) 他們分別是坐在我旁邊的Janna博士.
(主席) 忠臣的退休教授成英波.
(主席) 今天穿得很漂亮的薛霞霞博士.
(主席) 周永健教席副教授.
(主席) 今天也穿得很漂亮的余振寧博士.
(主席) 楊石昌教席助理教授.
(主席) 最後當然是很穩重的申慧蘭博士.
(主席) 余達森教席副教授.
(主席) 你看我的介紹似乎不太論資排輩.
(主席) 其實由張略牧師開始到申慧蘭博士.
(主席) 大家有沒有買那本書《四福音神學》.
(主席) 由幾位老師撰寫.
(主席) 張略博士負責撰寫《馬太福音》.
(主席) 薛夏夏博士負責《馬漢福音》.
(主席) 余振寧博士負責《路加福音》.
(主席) 最後就是申慧蘭博士負責《約翰福音》.
(主席) 我們開始今天的講座座談會.
(主席) 座談會的第一條問題.
(主席) 我想問問幾位老師.
(主席) 在耶穌受苦的敘述裡.
(主席) 四卷福音書各自的描述.
(主席) 其實有什麼獨特的地方呢.

$^{41}$(主席) 先請馬太張略博士講解.
(張略) 各位傳媒平安.
(張略) 其實我剛才跟夏老師說.
(張略) 馬漢福音是最難講的.
(張略) 原因是因為很多福音書的研究裡.
(張略) 都是看馬太,潘太和路嘉.
(張略) 建基於馬可的描述.
(張略) 所以這裡基本上說獨特.
(張略) 就變成跟什麼比較獨特.
(張略) 所以馬可有提及.
(張略) 我在講及馬太的時候就不會提及.
(張略) 就免得說他那一份.
(張略) 馬太福音其實我們一開始看馬太福音的時候.
(張略) 特別馬太有五段的講論.
(張略) 第一段講論我們大家很熟悉的.
(張略) 就是登山寶訓.
(張略) 登山寶訓其中一個很重要的主題.
(張略) 就是講公義.
(張略) 大家很熟悉的.
(張略) 就是你們的義要勝過民事和法理錯人的義.
(張略) 公義事實上在馬太福音裡面.
(張略) 有一個很重要的主題.
(張略) 這個字的名詞出現七次.
(張略) 就在登山寶訓裡面出現五次.
(張略) 很多人留意這個字很重要.
(張略) 這個主題很重要.
(張略) 因為要對比耶穌的教訓和法理錯人.
(張略) 他們行律法的時候的公義.
(張略) 但另外一個字和公義這個字是同字根的字.
(張略) 就是Dikaios.
(張略) 公義不過是形容詞.
(張略) 形容詞我們知道在希臘文裡面.
(張略) 可以當作名詞用的.
(張略) 這個形容詞出現17次.
(張略) 其中有12次是用作名詞.
(張略) 是二人.
(張略) 在馬太裡面描述耶穌是一個二人.
(張略) 其中包括記載到比拉多的太太她做夢.
(張略) 你見到馬太第二章講法夢.
(張略) 最後講耶穌受死的時候.

$^{81}$(張略) 亦描述到比拉多的妻子.
(張略) 她在夢裡面體會到這個.
(張略) 她的講法是說這是一個二人.
(張略) 馬太裡面亦用這種方式去描述耶穌.
(張略) 例如在第23章兩次指出.
(張略) 法利塞人流他們祖先的血.
(張略) 最後的講法是.
(張略) 叫世上所流二人的血都歸到你們身上.
(張略) 從二人阿伯的血起.
(張略) 直到你們在殿和壇中間.
(張略) 所殺巴拉加的兒子撒加利亞的血為止.
(張略) 我實在告訴你.
(張略) 這一切的罪都要歸到這個世代了.
(張略) 所以在馬太裡面很強調的一點.
(張略) 就是耶穌是一個二人.
(張略) 他在整個歷史的長河裡面.
(張略) 承接了由阿伯開始的作為二人.
(張略) 以至後來的先知.
(張略) 他們因為宣講上帝的道.
(張略) 而被以色列人所唾棄.
(張略) 甚至被他們所迫害.
(張略) 以至於死.
(張略) 耶穌就是在這樣的行列裡面.
(張略) 這是馬太一個很獨特的描述.
(張略) 他是作為二人而死在當時宗教的領袖底下.
(張略) 這個二人的說法其實重要的地方在哪裡呢?.
(張略) 如果大家記得八福的時候.
(張略) 最後一福就是為二受迫害的人有福了.
(張略) 到了第九福就變成.
(張略) 如果因為我的緣故受迫害.
(張略) 這個人是有福的.
(張略) 把他自己和公義等同.
(張略) 公義到底是說甚麼呢?.
(張略) 公義一方面是說人與人之間相處的所謂仁義.
(張略) 但另一方面也是在說上帝的公義.
(張略) 耶穌是要承傳上帝的義.
(張略) 這個公義亦帶有救恩的意思.
(張略) 耶穌作為二人.
(張略) 祂犧牲了自己.
(張略) 其實是要帶來上帝的公義.

$^{121}$(張略) 亦要帶來救恩.
(張略) 當然這個與祂的名字有密切的關係.
(張略) 因為祂的名字是甚麼?.
(張略) 耶和華的拯救.
(張略) 要把百姓從罪裡拯救出來.
(張略) 這個字其實在整個婦女福音裡.
(張略) 其實在婦音書裡.
(張略) 馬太福音是用得最多的.
(張略) 149次.
(張略) 跟著就是路加夫音88次.
(張略) 而且大量集中在耶穌受苦的描述裡.
(張略) 亦即是說耶穌作為二人.
(張略) 祂的死其實就是要把百姓從罪裡拯救出來.
(張略) 這是祂很獨特的身份.
(張略) 和祂要完成的使命.
(張略) 是以二人的身份為罪人死.
(張略) 你見到寶蘭馬上拿到這件事.
(張略) 第二方面就是.
(張略) 我們知道一開始.
(張略) 由第一章開始就講到.
(張略) 耶穌的家譜.
(張略) 祂是阿伯拉罕的後裔.
(張略) 是大衛的子孫.
(張略) 大衛的子孫.
(張略) 這個說法在死灰音裡.
(張略) 馬太也是出現得最多的.
(張略) 有十次.
(張略) 有兩次在家譜.
(張略) 有一次是指約瑟的兒子.
(張略) 是大衛的子孫.
(張略) 其餘七次.
(張略) 七是完全沒有數字.
(張略) 所以公義是七次.
(張略) 這個是七次.
(張略) 是出現其他人稱呼耶穌.
(張略) 祂是大衛的子孫.
(張略) 即是祂是彌賽亞.
(張略) 但更重要的是.
(張略) 這個彌賽亞.
(張略) 祂是要完成上帝在祂身上的旨意.

$^{161}$(張略) 如果我們看.
(張略) 《馬太福音》26章.
(張略) 耶穌在赫西瑪利園禱告.
(張略) 第二次禱告時.
(張略) 祂說:.
(張略) 「我父啊,這杯若不能離開我.
(張略) 必要我喝,就願你的義子成全」.
(張略) 為甚麼說願你的義子成全呢?.
(張略) 就是我們很熟悉的.
(張略) 在《主禱文》裡所說的.
(張略) 「願你的子義行在地上」.
(張略) 「如同行在地上」.
(張略) 願你的子義行.
(張略) 所以耶穌不單止教門徒.
(張略) 教我們祈禱.
(張略) 而且祂是身體力行的.
(張略) 祂是要完成父上帝的心意.
(張略) 所以祂作為大衛的子孫.
(張略) 祂有一個很重要的任務.
(張略) 就是完成父上帝的心意.
(張略) 祂怎樣完成呢?.
(張略) 馬太的描述其中一個獨特的地方.
(張略) 就是祂將耶穌和大衛作一個比較.
(張略) 而且用「壓沙龍追殺大衛」這個事件.
(張略) 來講及耶穌受當時的大祭司和法利賽人所迫害.
(張略) 特別是在《撒姆爾記下》第17章.
(張略) 大衛有一個謀士.
(張略) 叫做希羅人阿希多弗.
(張略) 阿希多弗原本是大衛的謀士.
(張略) 但他幫助壓沙龍追殺大衛.
(張略) 在《撒姆爾記下》第17章記載.
(張略) 阿希多弗對壓沙龍說.
(張略) 「求你准我挑一萬二千人」.
(張略) 「今夜我就起身追趕大衛」.
(張略) 「趁他疲乏手軟,我忽然追上他」.
(張略) 「使他驚惶,跟隨他的民必都逃跑」.
(張略) 「我就單殺亡一人」.
(張略) 這個描述.
(張略) 當中的重要用詞包括疲乏,逃跑.
(張略) 這些字眼.

$^{201}$(張略) 其實在形容猶大背叛耶穌的情景中出現.
(張略) 意思是他用了這些字來描述.
(張略) 耶穌就像昔日大衛般被追殺.
(張略) 而且詩篇更指出.
(張略) 他是我的朋友,我講得太長了.
(施生) 你再講多一點,大家很想聽.
(張略) 另外一點很有趣的是.
(張略) 大衛面對亞沙隆對他的追殺.
(張略) 雖然他打贏了,但他說了一句話.
(張略) 「願耶和華憑自己的義子待我」.
(張略) 所以他不殺亞沙隆.
(張略) 願耶和華按他的義子.
(張略) 這是我們跟耶穌所說的.
(張略) 願上帝的旨意成就.
(張略) 同樣是這樣的表述方式.
(張略) 另外一件事就是.
(張略) 當耶穌在亞西瑪利園的時候.
(張略) 祂有一個門徒削了祭司僕人的耳朵.
(張略) 耶穌說難道我不可以猜12型天使來幫我嗎?.
(張略) 意思是說祂要建立的彌賽亞國度.
(張略) 不是用暴力的方式來建立.
(張略) 祂選擇的是以非暴力的方式來建立上帝的國度.
(張略) 是流血.
(張略) 不過是甘心樂意地流自己的血.
(張略) 來將百姓從罪裡面拯救出來.
(主持) 馬可怎樣去看耶穌受苦的獨特地方呢?.
(馬可) 好啊.
(馬可) 弟兄姊妹平安.
(馬可) 你們可能聽得懂我的廣東話.
(馬可) 聽得懂嗎?.
(馬可) 聽一下我轉成普通話.
(馬可) 大家可以嗎?.
(馬可) 可能中間有些位置聽不懂.
(馬可) 可能聽一下普通話.
(馬可) 可能我轉回普通話.
(張略) 如果不說得這麼快就應該可以的.
(馬可) 哦,ok,那你提醒我吧.
(馬可) 好,剛才也謝謝張博士說.
(馬可) 因為馬可是弟弟對公婦姻書.
(馬可) 和婦女婦姻書的基礎.

$^{241}$(馬可) 所以說到獨特.
(馬可) 我也很努力找到他的獨特之處.
(馬可) 因為自己生的兒子.
(馬可) 我也覺得他很獨特.
(馬可) 所以也有些位置想跟大家分享.
(馬可) 首先受苦這個主題.
(馬可) 其實對馬可婦姻很重要.
(馬可) 為什麼呢?.
(馬可) 馬可婦姻是紀卷婦姻書中最短的一卷.
(馬可) 她記了幾章?只有16章?.
(馬可) 她說到受苦的地方有多少?.
(馬可) 其實從第11章開始.
(馬可) 到最後一個星期的耶路撒冷.
(馬可) 從第11章到第16章.
(馬可) 如果這樣去記的話.
(馬可) 幾乎是佔了三分之一的篇幅.
(馬可) 所以她說到受苦的地方佔了相當多篇幅.
(馬可) 這是一個.
(馬可) 第二個她的寫作對象.
(馬可) 當然是有爭議的.
(馬可) 我傾向覺得是一個.
(馬可) 當時在羅馬當中受逼迫的基督徒.
(馬可) 所以她面對的這群基督徒.
(馬可) 其實她說到如何面對.
(馬可) 面對受苦這個主題.
(馬可) 這個是她為何受苦.
(馬可) 是馬可婦姻很重要的一個主題.
(馬可) 第二個就說到.
(馬可) 馬可婦姻說到耶穌受苦的.
(馬可) 獨特的地方是甚麼呢.
(馬可) 第一個就說到.
(馬可) 其實有一個很明顯的對比.
(馬可) 其實一方面說到耶穌是君王微醉.
(馬可) 同時又是受苦的僕人微醉.
(馬可) 君王的部分其實可能.
(馬可) 剛才有提到.
(馬可) 可能可以大為君王的模式去說.
(馬可) 馬可婦姻都有.
(馬可) 不過她還突出了.
(馬可) 突出了她是僕人受苦的微醉那一面.

$^{281}$(馬可) 我說幾點.
(馬可) 第一點先說馬可婦姻.
(馬可) 其實她介紹耶穌的時候.
(馬可) 和馬太路家很不同.
(馬可) 她沒有說到她的家譜.
(馬可) 一個僕人她是沒有家譜的.
(馬可) 她沒有說到她的家譜.
(馬可) 沒有說到耶穌的出生.
(馬可) 沒有說到她少年成長期的時候.
(馬可) 通通都沒有的.
(馬可) 她一開播她怎樣介紹耶穌呢.
(馬可) 一章一到十三節的時候.
(馬可) 透過三件事去說耶穌.
(馬可) 第一是說她和施洗約翰的關係.
(馬可) 約翰是她的開路先鋒.
(馬可) 這些大家都很熟悉.
(馬可) 第二個是說洗禮.
(馬可) 洗禮.
(馬可) 洗禮當中被神聆聰明.
(馬可) 天上有個聲音宣告.
(馬可) 這是我的外子.
(馬可) 洗禮當中被宣告為神的外子.
(馬可) 第三個是狂野當中.
(馬可) 射食日受撒旦的試探.
(馬可) 當然她曾經試探過.
(馬可) 有個特別的地方是什麼呢.
(馬可) 這個地方和馬太路更不同.
(馬可) 大家可以留意.
(馬可) 首先先說一方面很高.
(馬可) 天上有宣告她是神的外子.
(馬可) 但是接下來她就被引導到狂野受試探.
(馬可) 受試探在馬可是唯一一個地方提到.
(馬可) 耶穌受試探的時候.
(馬可) 與野獸同在一處.
(馬可) 有沒有印象.
(馬可) 是對這句話沒有說的.
(馬可) 為什麼這樣呢.
(馬可) 你可以想到一個哥哥.
(馬可) 一方面剛剛宣告.
(馬可) 他是被神聆聰明的兒子.

$^{321}$(馬可) 接下來他就被撒旦試探.
(馬可) 與野獸同在一處.
(馬可) 跌到塵埃裡的那一位.
(馬可) 看到那個對比.
(馬可) 這個是馬可介紹她是一個怎樣的人.
(馬可) 第一個對比.
(馬可) 第二個是到了.
(馬可) 她十一章開始.
(馬可) 剛才說了.
(馬可) 她上耶路撒冷.
(馬可) 十一到十三章.
(馬可) 去到耶路撒冷.
(馬可) 她進入耶路撒冷的時候.
(馬可) 大家還記不記得.
(馬可) 剛剛耶穌到耶路撒冷.
(馬可) 當時群眾的情緒是怎樣的.
(馬可) 相當高漲.
(馬可) 很多人呼喊.
(馬可) 賀聖啦賀聖啦.
(馬可) 那份主名來的.
(馬可) 應當是被稱頌的.
(馬可) 很多路邊很多人歡迎她.
(馬可) 同時進來的時候.
(馬可) 當她進入聖殿的時候.
(馬可) 進入聖殿的時候.
(馬可) 那裡是沒有聲音的.
(馬可) 悄然無聲.
(馬可) 這個是有一點仿高潮的表達.
(馬可) 當她一開門.
(馬可) 聲音很高漲.
(馬可) 到最後其實沒有聲音的.
(馬可) 進入聖殿的時候.
(馬可) 沒有聲音.
(馬可) 很快就走了出去.
(馬可) 這個其實是仿高潮的表達.
(馬可) 其實已經預示了.
(馬可) 高高在上的君王.
(馬可) 凱旋式的彌撒亞.
(馬可) 卻是被人嘲笑的.
(馬可) 無人在意的受苦的彌撒亞.

$^{361}$(馬可) 這個其實是馬可夫人有意的說的.
(馬可) 另外一個很有趣的點.
(馬可) 可以看十五章.
(馬可) 十六三十二節.
(馬可) 那個記載她被領十字架的時候.
(馬可) 上面寫著猶太人的王.
(馬可) 這個大家都知道.
(馬可) 她左邊右邊都有一個牆道.
(馬可) 其實有古卷記載多一句話.
(馬可) 就說她被列在罪犯之中.
(馬可) 那句話她被列在罪犯之中.
(馬可) 其實是出自彌撒亞書馬三章十二節.
(馬可) 那裡是很有意思.
(馬可) 彌撒亞書馬三章大家都熟悉.
(馬可) 我相信.
(馬可) 她說的是受苦僕人的受苦僕人之歌.
(馬可) 其實第一個方面.
(馬可) 我覺得她有獨特的地方.
(馬可) 她是一個光華彌彩亞.
(馬可) 同時很突出她是受苦僕人彌彩亞.
(馬可) 這個是第一個.
(馬可) 這麼快?.
(馬可) 很快兩段很快講完.
(馬可) 另外我可以再講多一點.
(馬可) 耶穌在施工的開始.
(馬可) 除了說到第二個.
(馬可) 其實在施工過程中.
(馬可) 祂一直受到很多人的質疑.
(馬可) 可以看到祂受到宗教領袖的質疑.
(馬可) 甚至壓迫.
(馬可) 從二章一到三章十二節.
(馬可) 祂受到和宗教領袖的十五次衝突.
(馬可) 輸到一百二十七頁.
(馬可) 如果大家有的話可以打開.
(馬可) 這裡我會提到.
(主持) 廣告賣完.
(馬可) 很快.
(馬可) 唯一我想講多一點.
(馬可) 最初的衝突可能是大家心裡覺得.
(馬可) 什麼不開心不滿意.

$^{401}$(馬可) 最後慢慢慢慢衝突越來越面難看.
(馬可) 最後講到法利善人和希律一黨的人.
(馬可) 三意怎樣可以除滅耶穌.
(馬可) 他們想除滅耶穌.
(馬可) 這個地方有什麼特別的地方呢.
(馬可) 大家可以要留意.
(馬可) 其實無論是馬太也好路加也好.
(馬可) 沒有提到希律一黨.
(馬可) 都是只是提到法利善人.
(馬可) 但是馬可提到希律一黨.
(馬可) 知道這意味著什麼.
(馬可) 除了受宗教領袖這塊的逼迫.
(馬可) 同時也面臨暗示.
(馬可) 或者說是所經歷的.
(馬可) 來自於政權或者什麼樣的壓迫.
(馬可) 這個是我的一個解讀.
(馬可) 因為我覺得他太特別.
(馬可) 除了三章.
(馬可) 應該12節提到同希律一黨.
(馬可) 馬可八章那邊.
(馬可) 14到21節也提到.
(馬可) 就是在馬可八章.
(馬可) 醫治博塞大麻人之前.
(馬可) 耶穌吩咐要反對法利善人的教.
(馬可) 和希律的教.
(主持人) 路嘉你要搶麥克風.
(馬可) 和希律的教的這塊路嘉沒有.
(馬可) 馬太也沒有.
(馬可) 所以呢只有馬可有.
(馬可) 好了好了 我還有但我不說了.
(馬可) 哈哈哈哈.
(馬可) 好了 講完了.
(主持人) Kevin.
(Kevin) 不如我們讓台下先問一輪問題.
(Kevin) 我們可以回來再.
(Kevin) 我可以將頭兩條.
(Kevin) 串在一起說都可以.
(Kevin) 不如你先說吧.
(Kevin) 如果弟兄姊妹你們有問題.
(Kevin) 其實你們可以在WhatsApp.

$^{441}$(Kevin) 你們會寫一些問題到WhatsApp.
(Kevin) 螢光幕應該會轉頭出現.
(Kevin) 是不是?.
(Kevin) 啊有.
(Kevin) 我們看到一些問題的時候.
(Kevin) 就會問幾位老師.
(Kevin) 讓他們發表意見.
(Kevin) 不過這個時間還是交給.
(Kevin) Kevin先說吧.
(Kevin) 好的 那個問題是.
(Kevin) 每一卷科幻書對耶穌受苦的敘事.
(Kevin) 有什麼獨特的地方?.
(Kevin) 我想我.
(Kevin) 當然每一卷科幻書.
(Kevin) 如果我們深入去看細節的時候.
(Kevin) 即每一卷科幻書都有很多獨特的細節.
(Kevin) 所以再問多一次廣告.
(Kevin) 詳細的可以自己看書.
(Kevin) 哈哈哈哈.
(Kevin) 我想我嘗試選幾點好點提息.
(Kevin) 去和大家分享一下.
(Kevin) 如果說到路加記載耶穌受苦的敘事的過程.
(Kevin) 我特別想從路加的選材和鋪排.
(Kevin) 來看看有什麼特別的地方.
(Kevin) 發覺其中一個特別的地方.
(Kevin) 就是路加記載耶穌在十字架或釘十字架的過程中.
(Kevin) 所說的話.
(Kevin) Joyce老師很喜歡說.
(Kevin) 我們平時說十架七言.
(Kevin) 但其實沒有任何一卷科幻書是記載十架七言的.
(Kevin) 馬太和馬可都只記載一句.
(Kevin) 基本上是同一句說話.
(Kevin) 路加記載另外三句.
(Kevin) 約翰記載另外三句.
(Kevin) 路加記載耶穌在十字架說哪三句說話呢?.
(Kevin) 不知道大家馬上問你們會不會回答.
(Kevin) 路加記載耶穌在十字架說哪三句說話呢?.
(Kevin) 第一句是.
(Kevin) 父啊赦免他們,因為他們所作,的他們不要得.
(Kevin) 這個說話很有意思.

$^{481}$(Kevin) 說到耶穌受苦.
(Kevin) 釘十字架帶來一種赦免.
(Kevin) 如果我們把這個說話放回路加福音整個脈絡裡面看.
(Kevin) 就更加有趣.
(Kevin) 因為路加福音很強調耶穌為地上帶來紛爭.
(Kevin) 記得耶穌和門徒說.
(Kevin) 祂來不是要地上太平.
(Kevin) 是要地上有紛爭.
(Kevin) 叫門徒跟隨主人會受到迫迫.
(Kevin) 受到攻擊甚至殺害.
(Kevin) 這個是耶穌在地上帶來的一個果效.
(Kevin) 門徒也會面對這些紛爭迫迫.
(Kevin) 耶穌是第一個在這個迫迫當中面臨死亡.
(Kevin) 雖然祂面對這個紛爭.
(Kevin) 但祂不是選擇以一個敵對的態度進入這個紛爭當中.
(Kevin) 祂選擇.
(Kevin) 紛爭的處境改變不了.
(Kevin) 周圍的人攻擊祂,要釘死祂.
(Kevin) 改變不了,但祂選擇以一個赦免的態度回應這件事.
(Kevin) 這是路加記載耶穌受苦的一個很有意思的地方.
(Kevin) 受苦當中,原來耶穌如何選擇去面對.
(Kevin) 祂選擇赦免.
(Kevin) 其實祂整個受苦要帶來的也是神對人的赦免.
(Kevin) 第二句說話,耶穌跟祂身邊求祂憐憫的犯人說.
(Kevin) 我實在告訴你,今天要跟我再落圓裡了.
(Kevin) 當然帶來一個盼望,在受苦當中.
(Kevin) 釘在十字架上,下不來了.
(Kevin) 雖然耶穌有能力下來,但祂不下來了.
(Kevin) 下不來,在一個無助無望的狀態.
(Kevin) 但是耶穌帶出一個盼望,你今天要跟我再落圓裡了.
(Kevin) 這個盼望是超越了眼前的處境.
(Kevin) 雖然可能要到死了才能經歷.
(Kevin) 囚犯也是斷氣了才能進入落圓.
(Kevin) 但是是一個很實在的盼望.
(Kevin) 第三句說話,耶穌臨斷氣的時候.
(Kevin) 父我將我的靈魂交託在你的手裡.
(Kevin) 當然反映耶穌在受苦當中,到最終.
(Kevin) 祂仍然堅定地將自己交託給神.
(Kevin) 我相信耶穌將交託給神的不是單單祂的生命或靈魂.
(Kevin) 而是將祂一生的侍奉交託給神.

$^{521}$(Kevin) 耶穌在地上所作的,到祂最後受苦.
(Kevin) 到最終其實有什麼意義呢?.
(Kevin) 其實意義在神的手裡.
(Kevin) 耶穌將一切交託給神.
(Kevin) 深信神會在耶穌身上.
(Kevin) 使用耶穌的信服所成就的事情.
(Kevin) 去完成神的旨意.
(Kevin) 不如我先分享一點.
(Kevin) 我將麥克風交給Joyce.
(Johnny) 或者我想看看,Joyce先等一下.
(Johnny) 我想看看台下的同學或弟兄姊妹.
(Johnny) 你們有沒有什麼問題.
(Johnny) 是關於剛才三位的老師他們分享的.
(Johnny) 你們可以用WhatsApp去問問題.
(Johnny) 或者在禮堂的中間有一支麥克風.
(Johnny) 你們可以直接問問題.
(Johnny) 如果你們沒有問題,我就有問題了.
(Johnny) Joyce,你可不可以分享一下.
(Johnny) 耶穌受苦的敘事當中.
(Johnny) 在《約翰福音》有什麼特別的地方?.
(Johnny) 哈哈.
(Joyce) Johnson這樣問我.
(Joyce) 其實他已經帶出了《約翰福音》獨特的地方.
(Joyce) 大家是否留意到.
(Joyce) 剛才你只聽到《馬太·馬可路加》.
(Joyce) 沒有人提起《約翰福音》.
(Joyce) 因為《約翰福音》在四卷福音書裡.
(Joyce) 正正是一個被留下餘下的三卷叫做《芙蕾福音》.
(Joyce) 所以《約翰福音》的確有很多東西是很獨特的.
(Joyce) 我就分享一下敘事裡的耶穌有什麼特點.
(Joyce) 因為時間不多.
(Joyce) 我只想說一個點.
(Joyce) 《約翰福音》有幾部分.
(Joyce) 其中一部分叫做《榮耀之書》.
(Joyce) 就是耶穌說我得榮耀的時候到了.
(Joyce) 那裡就開始了敘事裡最後的部分.
(Joyce) 所以在《約翰福音》裡耶穌的受苦.
(Joyce) 是耶穌得榮耀的時間.
(Joyce) 是耶穌得榮耀也要榮耀父的一個時間.
(Joyce) 為什麼耶穌的受苦是對父神的一個榮耀.

$^{561}$(Joyce) 因為耶穌的來到.
(Joyce) 祂受苦是祂的使命的一部分.
(Joyce) 大家記得在《約翰福音》裡.
(Joyce) 其中一個耶穌的稱號很重要的就是「子」.
(Joyce) 是相對於父,差,子到世間來.
(Joyce) 是《約翰福音》經常說的.
(Joyce) 因為在《約翰福音》裡整個敘事世界.
(Joyce) 是來自一個使命.
(Joyce) 這個使命的源起就是神愛不是世人是世界.
(Joyce) 神愛世界甚至將祂的獨生子賜給他們.
(Joyce) 叫一切信祂的不自滅亡反得永生.
(Joyce) 耶穌是帶著這個使命來到世界.
(Joyce) 受苦本是耶穌使命的一部分.
(Joyce) 以至到祂釘十字架完成了神的心意.
(Joyce) 是神救世的心意被成就的一個時間.
(Joyce) 在《約翰福音》裡受苦是和耶穌的復活.
(Joyce) 和升天捆綁在一起.
(Joyce) 是人子被高舉的一個時間.
(Joyce) 是一個榮耀的時刻.
(Joyce) 所以在《約翰福音》裡.
(Joyce) 耶穌帶著這個使命.
(Joyce) 在《約翰福音》裡.
(Joyce) 耶穌是受苦的逾越節羔羊.
(Joyce) 《約翰福音》裡最特別的地方.
(Joyce) 就是耶穌不單在耶路撒冷過了一次逾越節.
(Joyce) 最少是過過三次.
(Joyce) 逾越節是《約翰福音》敘事的一個框架.
(Joyce) 我被撩了.
(Joyce) 如下看本書吧.
(羊群笑聲).
其實不是撩你的 是提我.
有個問題是問的.
問題是張牧師要回答的.
就是《馬太福音》十一章第三節.
為什麼施洗約翰會問這個問題.
就是問《馬太福音》十一章三節的問題.
他已經知道耶穌的身份了.
請張牧師.
《馬太福音》十一章三節說.
第二節是說耶穌在監獄裡聽見基督所作的事.

$^{601}$打發兩個門徒問.
「將要來的是你嗎?還是我們等候別人?」.
這個我想正是當代猶太人的一個煩惱.
他們往往知道.
或者他們所等候的尼塞亞.
就算宣言耶穌是基督 是神的兒子.
他們所期待的尼塞亞.
不是好像他們所見到的耶穌那樣的尼塞亞.
他想著尼塞亞來的時候.
就翻天覆地第一件事就是推翻羅馬的統治.
建立一個大一統的以色列的帝國.
萬國來朝.
一個你可以說是昔日大衛和所羅門的復興的國度.
這個是他們的看法.
我想約翰雖然他是一個耶穌的先鋒.
但是他對耶穌的整個服侍的理解.
仍然是受到很多限制.
另一點也是他正在坐牢.
即是正在坐牢的人其實有時是會感覺沮喪.
在監牢裡面有很多事情他在想.
他自己的服侍.
他所指向的他認為比他自己更大的人出來的服侍.
現在來到這個情節.
又說以色列的復興.
去了哪裡.
我們可以想像為何他會問一個這樣的問題.
是完全容易理解的.
所以耶穌才要非常清楚地澄清給他知道.
我作為尼塞亞我的服侍是一個甚麼的服侍.
所以他底下的答覆是很清楚的.
不知道弟兄姊妹還有沒有其他的問題.
大家可以出來禮堂的中間問.
或者繼續用WhatsApp發問你們的問題.
好,又另一個問題.
就是講到福音書當中有解釋為何耶穌會以受苦來完成上帝的心意嗎.
還是受苦本身並非實踐使命的方法.
只是過程當中的副產品呢.
好,幾位講完.
Kelvin,是嗎?.
剛才你說得不夠.

$^{641}$如果按路加福音的說法.
受苦或拯救絕對不是一個副產品那麼簡單.
因為當耶穌復活了之後.
特別是耶穌向門徒顯現.
路加記載了幾次耶穌向門徒顯現.
其中一次是路加獨有的記載.
就是講到耶穌在一個往二馬五司的路上.
向兩位門徒顯現.
向他們解釋聖經.
因為那兩位門徒都很沮喪很失望.
耶穌向他們解釋聖經.
當中耶穌就說了兩人無知.
他們先知所講一切你們信得太遲.
接著24章26節.
耶穌說:基督者讓受害又進入他的榮耀.
豈不是應當的嗎?.
其實這裡.
Kelvin用了一個字.
其實是一個必須.
耶穌是必須受苦.
所以這不是一個意外.
剛好耶穌受苦.
神就藉著耶穌的受苦來完成救恩.
不是這樣的.
耶穌自己的解釋是.
祂的受苦是必須.
而必須在聖經.
特別是在路加福音裡.
往往是用來表達.
這就是神所定旨的事情.
神的旨意就是要透過耶穌的受苦.
去成就救贖.
所以這不是一個意外.
不是一個選擇.
這是必須的.
我想在路加福音裡會說.
就是人子被舉起.
就像摩西在曠野舉銅蛇的一個意象.
就是說在路加福音裡.
受苦是一個拯救.

$^{681}$而在路加福音裡.
剛才沒時間說.
就更加好了.
順便說一下.
在耶穌受難的敘事裡.
約翰福音強調.
耶穌的苦是最少的.
因為馬可可能要解釋多一點.
為什麼要離棄我呢?.
在約翰福音裡不是這樣.
耶穌受苦的敘事.
是耶穌以一個君王的姿態.
來統管整件事.
大家記得.
約翰福音冰冰來到的時候.
是耶穌自己知道一切要發生.
祂走出來說你們找誰.
然後那些冰冰就說.
我們找垃圾人耶穌.
然後耶穌說我是.
原文是Ego Emi.
我是神.
然後嚇得那些人全部跌倒.
這是一個約翰很特別的敘事.
十加七言.
剛才Kevin說.
其實我們沒有一卷書是七言.
約翰福音的十加.
耶穌上面的說話.
是把母親交給約翰.
安排媽媽.
然後耶穌是要完成的.
於是他說我學了.
完成聖經的應許.
耶穌最後那句說話是搞定.
成了的.
在福音裡面.
是耶穌完成了.
作為子道成育身.
祂的目標.

$^{721}$Joyce.
既然剛才我給你時間不多的時候.
不如我進入第二個問題.
先由Joyce去回答.
如果剛才所說.
受苦是基督侍奉的中心.
受苦對於信徒和教會.
有什麼意義呢.
我先請Joyce補充.
我比較接納約翰福音.
他成書的背景.
就是當時有.
可能是猶太.
原先是皈依猶太教的.
一些信徒.
他們被趕出會堂.
我覺得對於我來說.
都是一個.
我比較接納的解釋.
如果是這樣的話.
其實約翰福音.
他正是要寫給一些.
在苦難.
因著信仰的緣故.
受壓力的一班信徒.
約翰福音為什麼要成書.
他就說.
但記者些時.
要叫你們信耶穌是基督.
是神的兒子.
以至你們得生命.
所以其實耶穌的受苦.
耶穌是誰呢.
約翰福音是要.
真的告訴受苦的信徒.
你要知道你正在信的是誰.
我們已經是正在信.
那個真是神的兒子.
還要選擇什麼呢.
苦難的來臨.

$^{761}$不是要讓人動搖.
當苦難來臨的時候.
人就要記得他正在信的是誰.
他是一個很強的神的兒子.
這是第一個點.
第二件事.
在約翰福音裡說到.
什麼是信.
約翰福音裡經常提到.
這個信了耶穌.
那個又信了耶穌.
但有時耶穌說.
耶穌沒有把自己交給.
那些說信給他的人.
因為在約翰福音裡.
信是要信到最後才是信.
苦難很容易令人不想信.
在約翰福音裡.
信的模特是那個盲子.
那個盲子是很重要的故事.
那個盲子就是.
當人要把他趕出會堂的時候.
他說我什麼都不知道.
我只知道這個人治了我.
然後耶穌說你信不信我.
我信人子.
人子是誰.
然後耶穌說我是.
然後耶穌說這個盲子我信.
他是願意付出被趕出會堂的代價.
他是肯為信仰付出代價.
但我要快一點.
因為我還要留很多東西給別人說.
其中一個我覺得很重要的說法.
耶穌說了.
他在離別之言裡說.
世人恨你們之先.
其實是恨了我.
耶穌在告別之言的時候.
他已經說了.

$^{801}$在世你們有苦難.
所以其實苦難不是讓人感到意外.
耶穌一早說了.
世人恨你們之先.
是因為他們先恨了我.
所以如果我們覺得受苦是令人意外.
這件事是很令人意外的.
我希望大家聽得到.
耶穌一早就說了.
你們在世有苦難.
不過你們放心.
我已經勝過這個世界.
其他三位有沒有什麼補充.
我們是交流座談會.
你們可以搶咪講的.
不如Kevin來說說.
基督的受苦對信徒教會有什麼意義.
其實每一本福音書都是.
路加夫音文也很強調.
門徒其實是跟隨主的腳蹤.
耶穌告訴他的門徒.
正如剛才Joyce也說到.
門徒也要像耶穌基督一樣.
去經歷受苦.
在路加夫音文裡.
他用了一個很獨特的圖畫.
他突出一個旅程的圖畫.
就像耶穌在地上的時逢.
是一個旅程.
而在路加夫音9:51.
講到耶穌知道他被接上升的日子章度.
因為被接上升.
你可以說是基督的升天得榮耀.
但是.
正如剛才所說.
是透過他的受苦.
才能夠進入榮耀.
然後那裡的經文是說.
耶穌知道他被接上升的日子章度.
他就定義去耶路撒冷.

$^{841}$他很堅定地向著耶路撒冷前進.
而耶穌基督早就說了.
耶路撒冷在等他.
是被拒絕和被殺.
我們看到耶穌的生命.
就是一座明知山有虎.
偏向虎山行.
他明知死亡.
拒絕在那裡等他.
但是他定義要去那裡.
我想這也是一封書告訴教會.
教會在地上要走的路就是這樣.
我們知道有什麼在前面等我們.
但是我們不去逃避.
我們仍然像耶穌一樣.
堅定地向著受苦的前景進發.
因為我們知道受苦背後.
有神應許的榮耀.
我想這是一個很重要.
耶穌去受苦給教會的意義.
而且剛才也提過.
耶穌說過的話.
「赦免他們所作的,因為他們所作的,他們不曉得」.
受苦是改變不了.
敵對是改變不了.
但是我們可以選擇.
帶著一個什麼心態進入紛爭當中.
耶穌選擇赦免.
當然這對我們來說是一個.
不是很行的事情.
我們被人這樣拼的時候.
我們很難去赦免.
很多時候我們說.
耶穌可以,我們怎麼可以.
我們只是區區的人,我們做不到.
但是耶穌給了我們這個榜樣.
我們可以帶著一種赦免.
神恩慈進入受苦當中.
而不是讓受苦在我們裡面.
栽種那種苦毒.

$^{881}$這也是耶穌受苦對教會一個很重要的意義.
當然還有就是.
路加一封十四章也記載.
也是路加一個獨有的記載.
其他一封書沒有的.
就是耶穌說到比喻.
說到我們作為主一門徒.
我們要計算好代價.
你知道你去跟從主,你去信耶穌.
其實你不是只是拿一張天堂入場券.
你是跟從主走在受苦的路上.
麻煩你搞清楚你在做什麼決定.
你才好做這個決定.
所以你聽完可能要回去想清楚.
你是不是要信耶穌.
我特別喜歡鎮定老師.
他283頁.
沒有買書就記下這個頁數.
受苦和禱告的生命一流.
我猜路加其中一個很重要的.
所有在福音書裡面有關於禱告的獨特的比喻.
都是在路加福音.
說到禱告和受苦.
你就看到不單止在路加.
你一跨下去試圖行轉也非常清晰.
那馬太呢.
馬太,馬太我講完.
馬太和路加也有相似的地方.
路加也有提到耶穌是二人.
耶穌是二人.
我剛才提過八福所說的為二受逼迫.
第九福就是為我原故受逼迫.
所以跟隨耶穌的人.
追隨這個字在馬太裡面.
所有福音書中出現最多的26次.
其實它強調作門徒.
很簡單.
耶穌走的路你跟著走.
這個就是作為門徒的意義.
剛才鎮磷老師也有提到.

$^{921}$但這個全天國的福音.
其實就是將真理擺出來.
說是就是是.
不是就是不是.
是為了公義的緣故而鬥爭.
你是站在公義的立場.
你知道你站在哪個位置.
這個很重要.
我們站在耶穌那邊.
我們站在公義那邊.
我們站在真理那邊.
你就沒辦法不面對它的對立面.
我覺得不留意.
馬太福音其實是記載邪惡.
或者這麼說.
邪惡這個字Polygnost.
Polygnost這個希臘名字.
在馬太裡面出現最多.
26次.
當然再加上13次假冒偽善.
這個其實是.
跟公義只是成為一個很大的對比.
正是因為要面對邪惡的緣故.
所以要面對迫害.
正如路加.
路加.
鎮磷.
正如路加福音所說.
其實饒恕是很重要的.
路加也一樣有提到.
路加用另一種方式說.
也是為迫不利的祈禱.
馬太的說法正如說要愛你的仇敵.
這是矛盾的.
仇敵不是你愛的.
是你恨的.
是跟你作對的.
是你的對頭.
是你的死對頭.
你恨不得他死.

$^{961}$但他也說要愛.
這也是耶穌在這個過程裡面.
最獨特的一種表達方式.
你要消滅敵人.
不.
你愛他就消滅他.
不是用任何暴力的手段.
報復的手段.
消滅你的敵人.
而是愛他.
你一愛他.
他就不是你的敵人.
他是你的鄰舍.
而這種方式來表達.
上帝就是對著我們這群.
這麼不可愛的他的敵人.
他就是這麼愛我們.
如果這個敵人你很恨的話.
你就知道.
上帝可以很恨你和我.
因為我們做的事.
在他眼中是可惡的.
是邪惡的.
但他就是這麼愛我們.
馬太這種演繹.
也是馬太裡面.
說謠說最多的一段書.
應該華晨麒刊.
又不是告白.
十二月應該會出我的一篇文.
就是馬太福音裡面.
怎樣講到耶穌.
饒恕人的教訓.
這是一個非常重要的教訓.
另外一點也很重要.
耶穌說教會是建立在盆石上.
陰間的權柄.
不能勝過他.
世界上沒有一樣東西.
連撒旦都不能夠.

$^{1001}$將教會從地上除去.
因為教會有耶穌基督的授權.
天上地下就是所有的權柄.
都在耶穌基督的手上.
我還是決定用普通話好不好.
我發現我這個小女子搶麥.
是搶不過大家的.
所以我只能等.
我下一回會幫你的.
然後這個題目是說.
剛才問的是說.
受苦是基督示憤的中心.
那麼受苦對於信徒和教會有何意義.
我是特別乖的就問什麼問題.
我答什麼問題.
然後先說就是.
為什麼在馬可福音當中.
他是怎麼來記載受苦.
為什麼信徒要受苦.
其實也回應剛才有一個問題.
就是福音書當中.
關於受苦的意義.
馬可福音他先.
他集中的部分在八章27到十章45節.
就是要在上耶路撒冷這個途中.
這中間的部分.
那他談到受苦.
就是受苦的基督跟做門徒的意義.
集中在這塊來談.
首先就是前面我們談過.
耶穌的身份來是受苦的基督.
對吧.
但是作為他的門徒.
作為追隨者.
我們是追隨他的腳蹤行.
所以他的教導裡面是一次又一次的出現.
我就拿他三次預言.
自己受苦這個部分來說.
第一次預告耶穌受苦復活.
在八章31節之後.

$^{1041}$緊接著他就指出.
耶穌的追隨者也要受苦.
舍己背十字架.
跟從耶穌的道路.
八章34大家可能很清楚說.
若有人要跟從我.
就當舍己背起他十字架來跟從我.
第一次預告完就談到.
跟從他的人.
就要一樣背起十字架來跟從.
第二次預告.
耶穌受苦復活第九章30到32節.
就是耶穌預告完.
其實門徒還在爭論.
將來在神國裡面誰為大.
誰要得召尊榮.
誰要為首.
耶穌對他們的教導是什麼.
是說恰恰你想要在前面的.
你要在後面.
要做眾人的僕人.
所以在九章35節他又提了一句.
若有人願意做首先.
他必做眾人幕後的眾人的用人.
之後大家可能比較.
他又用了一個關於青年財主的比喻.
大家應該就比較熟悉那個比喻.
那個比喻的最後.
談到什麼.
其實他談到一條十字架受苦之路.
就是要捨棄我們看為最重要的.
他要青年捨棄他所看為寶貴.
所有的財富要他捨棄.
所以在第三次.
然後這是第二次.
就是要他捨棄所有.
第三次預告耶穌的受死跟復活.
就是在逼近耶路撒冷的時候.
關於受苦的預告就更加的詳細.
裡面大家如果去看.

$^{1081}$他是談到他受鞭打.
比較詳細的去介紹.
但是門徒的反應都是.
將來你得榮耀的時候.
我怎麼樣可以坐在你的左邊.
坐在你的右邊.
耶穌的教導就是在神國度裡面.
不是憑著權力和控制來治理.
而是透過奉獻與服侍.
這個時候出現了全書的一個要節.
十章四十五節.
應該是馬可福音的一個keyverse.
大家都很熟悉.
是什麼.
因為人子來並不是要受人的服侍.
乃是要服侍人.
並且要捨命做多人的塾駕.
這個是整個馬可福音的一個要節.
所以其實這個也回應剛才.
剛才那個問題是說什麼.
福音書受苦是不是必然的.
其實這裡就談到耶穌來.
他就是要服侍人並且要捨命.
其實受苦的極致.
就是實質加上犧牲的這個捨命流血.
所以他來就是捨命做多人的塾駕.
這個是馬可福音裡面很明顯就談到.
我們所有追隨耶穌的人.
追隨他的腳中.
必然要經歷這份的苦.
必然要經歷這樣的一個苦路.
到13章談到類似有時候墨的時候.
也很明顯談到具體的.
就是要會由領導這個苦難與逼迫.
提到會在會堂裡受鞭打.
人會把你們拿去見官.
並且弟兄要把弟兄.
父親要把兒子送到死地.
兒女要與父母為敵.
為主的名被眾人恨惡等等.

$^{1121}$他都很明顯地提到.
這個是要經歷的.
必須經歷的苦.
這個的話就是基本上是這麼談到.
當時對第一世紀的.
當時耶穌的追隨者.
那麼對今天的我們.
對今天的信徒跟教會有什麼樣的意義呢.
這個就是可能給大家留下的一個思考.
首先就是信仰.
它不是我們進入或者成功之門.
反而信徒是在這個世界當中.
我們會經歷各樣的難處困難.
在今日的社會裡.
我不知道大家所面臨的.
這個環境的變遷可能帶來的這種壓迫感.
壓逼感壓抑感.
也可能是個人生活的種種不如意.
破碎的家庭.
不幸的成長經歷等等.
當然我們就是說.
信徒肯定會經歷各樣的難處.
但不一樣的是什麼呢.
在這個難處當中.
在黑暗當中的我們是有同行者.
我們有勝過死亡.
與黑暗中的主可以信靠.
這個是我們可以思考的.
第二個就是在苦難當中.
其實苦難為什麼我們要去經歷呢.
其實在教會門徒.
其實在整個過程當中.
他經歷過很多的失敗.
後來就被轉化.
失去傳扶嬰.
這跟他的經歷有關.
那弟兄姊妹一同去經歷這種.
在苦難當中.
大家一同去經歷這種成長.
transformation這種改變.

$^{1161}$最後一點快了.
就是說.
耶穌談到耶穌的身份.
一開始大家如果還有印象的話.
馬可一開始談耶穌的身份.
他是以聖靈來貫穿的.
談到他三個去聖靈.
有天上來.
或者聖靈帶他去到曠野.
到了13章11節.
談到信徒在面臨逼迫的時候.
其實要我們敏感於聖靈的工作.
談到說.
能把你們拉去見光的時候.
要預先思慮說什麼.
到那時候.
賜給你們什麼話.
你們就說什麼.
因為說話的不是你們.
乃是聖靈.
就是說.
在我們處境當中.
在艱難的環境當中.
我們更需要是敏感於聖靈的工作.
是什麼.
最後就是.
好.
我的時間到了.
不是.
你都.
哎呀.
錄了.
你都.
還有問題可以問你的.
其實.
還有問題問我.
因為有個挺有趣的問題.
是.
在馬克福音第14章51到52節裡面.
就說到有個少年人赤身逃走.

$^{1201}$這個記錄其實會不會和耶穌受苦.
有沒有一些關係.
啊.
OK.
好啊.
謝謝你.
因為.
這個可能是我剛才第一條回答.
應該是剛才有一條問題.
不過時間.
就是可以來.
就是說是這樣.
其實.
耶穌的.
其實這跟耶穌的受苦有關係.
是什麼樣的關係呢.
首先就是說.
耶穌他在這個他的命理事宜當中.
他所經歷的苦是什麼.
其實除了剛才我提到說宗教宗教領袖.
或者是甚至他提到心理一黨.
對吧.
有政治上.
政治上的這個壓迫也好.
還有呢.
他面臨的是親朋好友的一個的.
背離背叛.
說一個就是.
他在六章的時候.
應該是提到說.
我看提到說是什麼.
我看一下.
就是呢.
他本主本.
六章二到三節提到他本主本相的人.
厭棄耶穌.
這個是第一個.
接下來耶穌身邊.
耶穌悉心教導的這個門徒.
他們不明白耶穌是誰.

$^{1241}$直到這個最親近的門徒.
他們都離棄.
這個呢就跟剛這個問題就是十四章.
是吧.
這個當然有人不同的有猜測呢.
這個少年人.
有沒有可能是約翰啊.
總之就是耶穌親近的一個追隨者.
其實他最到什麼時候.
他他你就是逃走呢.
就是在耶穌被捕的時候.
所以在耶穌被捕.
他最需要就是有人支持他的時候呢.
他最親近的門徒的紛紛離開.
紛紛逃走.
這個就是.
然後包括克西瑪尼恩的禱告.
耶穌在流內流血禱告的時候呢.
門徒卻沉睡了.
所以呢.
這個就是他所經歷的另外一種苦.
就是清倫的貝尼.
他最親近他悉心教導的門徒呢.
不不不理解他.
不明白他.
甚至彼得三次呢.
否認主.
這種種的都是他最需要支持的時候.
他在走最艱難路的時候呢.
身邊的親人不理解.
你最在意的人呢.
他離棄你.
他背叛你.
他否認你.
這個是耶穌所經歷的另外一種苦.
所以剛才提到的這個十四章這塊呢.
這個少女只是一個影子.
其實是表達的是.
耶穌最親近的人的這個貝尼.
離開他.

$^{1281}$給他帶來這個痛跟苦.
謝謝.
好了.
有另外一個問題呢.
可能四位老師都可以答的.
不過是關於《仕途行傳》的.
(笑).
問題是.
有關福音書當中.
受苦耶穌對教會的意義.
路加和《仕途行傳》的序士.
在裡面.
路加當中的受苦主題.
在《仕途行傳》裡面有沒有被發揮.
或者怎樣去表述出來呢.
我答一點點.
然後由路加去答.
有兩個路加.
一個寫路加.
一個英文名叫路加.
然後讓兩個路加回答.
《仕途行傳》是路加福音的下集.
他很快.
他有一個《仕途行傳》很特別的地方.
就是路加已經完了整個耶穌的序士.
他開始的時候.
講回耶穌升天.
有很多的討論.
但是我比較接納的一個說法.
就是《仕途行傳》餘下的序士.
教會的建立.
是要建基在耶穌基督之上.
所以為什麼路加完得這麼美麗的時候.
去到《仕途行傳》要再寫多一次.
耶穌升天復活的時間.
因為他很強調的就是.
福音傳到地極.
這件事是建基在耶穌基督的序士之上.
這是第一.
《仕途行傳》其中一個主題.

$^{1321}$當然就是環球宣教.
福音由耶路撒冷.
猶太傳遞薩瑪利亞直到地極.
其實未必是地極.
但是起碼福音傳到羅馬.
大家都記得.
其實是沒有完結.
因為是在講保羅在羅馬被軟禁的時候.
他還在傳福音.
所以其實.
如果你給我看.
我比較相信《仕途行傳》.
它是在講一個福音怎樣傳開.
裡面其實是在講.
福音怎樣可以排除萬難.
可以由耶路撒冷去到地極.
當然當中還有耶路撒冷教會的逼迫.
繼續就是很多的困難.
最後就是保羅.
他面對在猶太地.
他去到哪裡都被猶太人打.
他回到耶路撒冷就要上告該撒.
他就去了地極.
我想《仕途行傳》很想告訴我們.
福音好像是一個很大的成功.
但是這種成功.
其實裡面是有很多人的血和汗.
而引達至的一個成功.
裡面是有斯提反.
裡面是有保羅.
他們不斷地為福音傳開.
到底我要信服人還是信服神呢.
很多人恐嚇他們.
他們排除萬難.
他們跨越很多這些壓力.
福音最終傳開.
而它是沒有結局.
是有一個忠心的使徒.
他在羅馬就算軟禁.
他仍然宣講神的國.

$^{1361}$這個故事是要留給之後的教會.
就是保羅.
Joyce 剛才也說了很多.
《仕途行傳》如何承接路加福音.
各方面的主題.
包括福音的見證廣傳.
當然也包括受苦的主題.
剛才提到五重司令降臨.
當然是很有大能力的彰顯.
但是轉頭去到第四章就已經說.
彼得仕途被人捉拿.
所謂的成功.
所謂很大能的彰顯.
沒錯,他真的勝利很大能的彰顯.
但同時伴隨著的.
繼續是人的拒絕.
人的攻擊.
之後剛才提到斯提反尼的殉道.
雅各的死.
保羅在宣教旅程中受到種種的逼迫.
他最後回到耶路撒冷.
猶太人想殺他.
後來千夫長救了他等等.
我們看到教會仕途不斷受到攻擊.
第八章.
也說到當時耶路撒冷教會大受迫迫.
所以仕途就分散.
我們看到受苦的主題在《兆行傳》中.
也是一個很明顯的.
也是貫穿整卷書的主題.
還有很有趣的是.
我們回看《兆行傳》的開始.
那段對答是很有趣的.
一章六到八節.
當時門徒問耶穌.
你復興以色列就在這時候.
如果我們放回當時的歷史背景.
我們很自然理解門徒在想的.
就是耶穌你都復活了.
死亡你都勝過了.

$^{1401}$是時候復興以色列了.
他們很期待.
最終復興要來到.
但耶穌卻回答聖廉哥倫.
在你們身上你要得著能力.
並要在耶路撒冷猶太傳帝國.
撒瑪利亞直到基地作為見證.
沒錯神要復興以色列.
這是肯定的.
但是門徒不是現在就來看那種.
很榮耀得勝的復興.
門徒經歷其實也是復興.
不過是透過他們受苦的見證.
讓神的國度在地上興起.
所以其實最開頭這段話.
雖然他沒有直接提到受苦.
但其實受苦的主題已經隱含在裡面.
張牧師有話想說.
我們現在經常想移民.
究竟是否移民.
你看到施平行傳是很有趣的.
因為他說的碧白林道.
其實是集中在耶路撒冷的人分散.
這個分散事實上是帶來福音的傳播.
也就是說可能他們在耶路撒冷.
很多人信主很開心.
他們建立教會.
碧白林一來的時候.
他們迫不得已要將福音帶出去.
你可以說這是上帝用的方式.
叫福音傳開.
另外有另一件事是有趣的.
就是保羅面對著碧白林.
你看到在接受萊迪加也好.
菲律賓也好.
菲律賓被害受辱.
甚至坐牢.
他在接受萊迪加的時候.
因為迫害的緣故所以離開.
不過在另外一次.

$^{1441}$他明知道要上耶路撒冷.
他會受迫害.
他就走去耶路撒冷.
你會說他是不是不一致.
迫害又走.
另外一次迫害又走過去.
其實對保羅來說很簡單.
尤其是他對阿基布預言.
他會上耶路撒冷的時候會受迫害.
就好像那班大影子說的.
願上帝的旨意成就.
最重要的不是他去哪裡.
最重要的是他怎樣體會.
上帝在他生命裡的大令是什麼.
不如我問第三個你們要交流的問題.
不如張牧師先回答.
你說話少.
耶穌即使其中一個向度是開展神的國度.
四卷方書是怎樣描述神終末國度的特質.
而信徒今天又怎樣能夠經驗.
或者回應這個終末的特質呢.
馬太是很特別的.
因為他有用上帝有國.
但只出現四次.
最主要用的是天國.
天國以前說天是一種忌諱.
不想叫上帝所以叫天.
但其實這個說不通.
因為馬太福音裡.
神這個字已經出現50次.
為什麼其餘的不忌諱.
只用到國度的時候就忌諱呢.
所以近代的研究認為.
其實是講及到上帝透過耶穌帶來的國度.
不是一個地上的國度.
不是一個屬地的國度.
亦都不是地上的國度.
以權力以暴力去爭取的一個政權.
他帶來一個屬天的國度.
這是上帝的旨意.

$^{1481}$行在地上如同行在天上.
是講及到上帝的權柄怎樣地上彰顯出來.
如果你看八福.
他講到心靈貧窮的人有福了.
因為天國是他們的.
為義仇逼迫的人有福了.
因為天國是他們的.
這是第一和第八福.
明顯地講天國是帶給一班人.
八福基本上就是.
這一班人就是天國子民的特質.
所以他不單要強調天國是從上面而來.
而且講及到神的子民的身份.
他是天父的兒女.
作為天父的兒女.
是和其他人有分別的.
包括在第十三章講到.
他們是二人出現的.
然後說他們要發光好像太陽一樣.
可能大家沒有留意這段.
其實是引用《但願你書》十二章三節.
智慧人必定發光如同天上的光.
那使多人皈依的必發光如星.
直到永永遠遠在地上成為炎陽光.
這是明顯地講及到.
作為神的子民在地上的時候.
怎樣能夠表達天國的降臨呢?.
就是透過他們活出上帝所要求的公義.
神的國和神的義.
在他們中間來彰顯.
亦在整個《馬太福音》裡面.
很強調忠惡要分成兩類人.
有些人走闊路.
有些人走窄路.
有兩種不同的樹結出兩種不同的果子.
有真先知,有假先知,有假基督.
亦都講到.
看你將你的生命建造在沙土上還是在磐石上.
然後不斷提及到.
有些人要被趕出去.

$^{1521}$在外面哀哭切齒.
或者在第十三章講到比喻的時候.
講到田地的比喻.
最後一些被拿出去.
然後被雕在火裡面燃燒.
所以很強調我們要知道我們的身份是什麼.
我們要很清楚知道.
你是誰.
你知不知道你是誰?.
你知不知道你是天父的兒女?.
如果你知道你是天父的兒女.
你就知道你應該有的生活方式是什麼.
如果你的生活方式.
和耶穌所描述的工業的生活方式不同.
你要問清楚你自己.
你站在哪一條隊裡?.
所以你見到忠末審判的意味是很重的.
你要看清楚你究竟是不是上帝的兒女.
好像剛才已經想說話了.
多謝北極有能.
就講到最後關於神國度這個問題.
其實在馬可福音當中.
他有談到神國度的地方.
首先從耶穌口中.
大家如果記得的話.
就一章十四節左右.
開始傳揚神的國境了.
其實這個就是大家很熟悉的.
關於以然未然的神國忠末的狀態.
以然指向神國已經在耶穌的指示.
他的ministry宣講當中已經開始.
在經文當中.
譬如三章二十二到二十七節.
談到跟文士辯論的時候.
耶穌提及那裡面已經暗指.
就是撒旦國的瓦解.
三章二十七節提到說.
沒有人能進鑽石的家裡搶奪他的家具.
必先捆住那鑽石才可以搶奪.
所以也就是說.

$^{1561}$耶穌能夠進到這裡去醫病趕鬼.
其實撒旦某一個程度上.
他是被捆綁的.
這正是宣告神的介入.
將撒旦捆綁.
神的國度已經來臨.
這個是經文上的一個記載.
同時在很多地方他談到.
比如說耶穌平靜分娜的比喻.
在水面上行走等等.
表明耶穌本身.
大人勝過大自然的力量.
這個在在都表明.
神的國已來臨到.
當然在四三章可能還.
神國有談到.
他來到但並未全然的實現.
即即未即的狀態.
談到讓日子要來.
將來來的時候他有大人.
大人立大榮耀駕雲佔領等等.
但何時來並不完全知曉.
因此我們當下.
在這個即已來未來的時候.
我們要更加的去警醒.
所以這個呢.
所以談到強調.
這個時候我們中末.
門徒在中末的時候.
不能放棄傳揚福音的使命.
福音必須先傳給萬民.
四三章十節提到.
這個是一個關於.
已來未然的神國度的狀態.
第二個談神國度如何擴展.
其實他是透過這個比喻來說明.
在神的手中.
第四章談神國的比喻當中.
談種子如何長大.
如何從微不足道的芥菜種.

$^{1601}$長成參天大樹.
不是我們人為可以控制的.
神國度的擴展.
他在神的手裡.
最終神的國度全然臨到.
提到你們必看見仁子.
坐在全仁者的右邊.
代表君王米賽亞.
所以這個第二個層面.
第三個就是神國度中的人.
是怎麼樣的.
在這裡頭.
其實他談到.
這也是我覺得.
是馬可福音的一個特色.
在神國裡面的人.
正是那些很普通的人.
那些遵行神旨意的人.
就是我的兄弟姐妹.
這個是耶穌在.
馬可福音三章三十五節.
耶穌裡面提到的.
然後第十章緊接著他提到.
耶穌對孩子的祝福.
當中他提到說.
因為在神國裡面正是這樣的人.
凡要承受神國的.
若不向小孩子斷不忍進去.
這什麼.
也就是說在神國裡的.
就是那些無權無勢的小孩子.
小孩子就是.
他是一個代表的表達.
是一個普羅大眾的代表.
正是在神國度裡的.
正是這些人.
而不是那些像青年財主.
在故事當中.
那樣的一個人.
耶穌要他反而要他.

$^{1641}$去變賣所有來跟從.
所以他又要走了.
所以在神國度裡的.
不是憑藉著我們所擁有的.
而是單純的.
好像如這個小孩子那樣的.
單純無所一的.
是雲雲眾生之中的小人物.
這個正是神國度當中的人.
神所寶貴的人.
這個是馬可福音比較.
我覺得是他比較強調的.
馬可福音跟其他福音.
不重複的地方.
很少很少有三處.
那其中有一處就是.
三章七到十二節是其他福音書.
沒有提過的.
裡面就是談到.
一般的人怎樣在加利利海邊.
從不同的地方來跟從他.
耶穌和門徒退到海邊去的時候.
許多人從加利利來跟隨.
聽見他所做的事.
從猶大耶路撒冷以圖買.
從各個地方來跟從.
都是一個雲雲眾生的.
普通人來跟從他.
然後耶穌看見他們.
憐憫他們.
治好了他們許多的人.
我覺得馬可福音他們強調的是.
這個小人物.
一會兒會再談.
我怕你又勸我.
我很快完了.
所以我們應該如何回應呢.
簡單說.
神的國已然來臨.
他憑著信心經歷.

$^{1681}$上主的大人.
經歷他的醫治.
經歷他的轉化.
就像福音書裡面.
很多人經歷那些小人物.
經歷耶穌的手被他醫治.
第二個就是我們帶著.
諸默的期盼活在當下.
因為最終得勝的是主.
無論我們當下覺得如何.
最終是相信.
掌權的是主.
我們是那位勝過黑暗的權勢.
最終的掌權者.
所以我們在當中所經誇的.
我們是有盼望的.
我們不是沒有盼望.
苦難不是終點.
它只是我們旅程中需要去經歷.
使我們更加強大的一個過程.
所以這個是我們可以去談.
最後一個就是.
或者我們給Calvin.
就一句話.
看見身邊人的需要.
因為在神國的.
都是那些不起眼的小孩子.
經歷過傷痛的云云眾生.
所以看見身邊人的需要.
也是說我們可以去回應.
就是說在馬可福音提到神國的時候.
我覺得我們可以做的一些回應.
對不起我話多了.
兩位可以隨便了.
我分享兩點.
就是講到.
天國的特質.
和我們怎樣去.
經驗天國.
路加.

$^{1721}$我認為路加比其餘幾篇分析書.
是更多去問一個.
誰?.
到底誰是在神的國裡面.
當然其他分析書也有提到.
但是路加是更加強調這個問題.
到底誰是屬於神的國.
他有幾次提到.
特別是那個終末的筵席.
很多人要從四方來到.
和亞伯拉罕和以色列的座席.
這些想著自己一定有份的人.
寫我旗帥的人.
卻是沒有份.
路加有好幾次提到.
就是很強調.
到底誰是有份於神的國.
你很有信心.
自以為一定有份.
我這麼虔誠.
一定有份.
但是路加說不一定.
就是剛才.
兩位提到.
唯有那些天秘地遵行神旨意的人.
謙卑地領受耶穌基督的恩典的人.
他們能夠進入神的國裡面.
難阻人進入神的國.
往往就是人自己的驕傲.
這是路加一個很突出的訊息.
另外說剛才我們怎樣去驚艷這個國呢.
《紀元藩書》也有提到.
天國的已然未然.
當然就是在等待基督再來.
天國傳言的實現.
同時也是透過耶穌在地上的侍奉.
其實已經開始.
路加有一段很有趣的經文.
在路加17章20節開始的一段經文.
在裡面提到.

$^{1761}$其實天國已經在你們中間.
天國已經在這裡.
很明顯地說到天國已經臨到.
但是我們怎樣去驚艷這個天國呢.
很有趣地說到.
天國不是眼所能見.
什麼意思呢.
這裡用的見字.
它不是普通的常見見字.
原文用的是觀察的字.
就好像醫生斷症.
醫生斷症.
看病人的病徵.
就能夠推敲那個病人是什麼病.
天國不是可以這樣去觀察的.
不是透過你去看一些很特別的徵兆.
好像猶太人求耶穌見一個天上來的神蹟.
很震撼.
我就看到天國了.
天國不是這樣去看的.
反而是天國就在你們中間.
透過可能在我們周圍一些.
看似很平凡的神一點一滴的作為.
就好像耶穌在地上所行的.
雖然耶穌行很多醫病趕鬼的神蹟.
但我們要明白當時的背景.
這些神蹟其實不算得上很獨特.
所以為什麼人們還要求天上的神蹟.
有時候就是透過我們生命當中一點一滴.
所經歷的神的作為.
我們看到這個天國.
然後其實繼續的就是.
耶穌跟門徒說了一句很相似的話.
耶穌說你們會渴望見到人子的一個日子.
卻不得看見.
人子的一個日子.
我們簡單理解.
都是神一種拯救的作為.
但耶穌竟然說門徒不得看見.
這句話其實很奇怪.

$^{1801}$門徒為什麼看不到.
其實很多人都在說門徒.
都是用錯的方法去看.
他們期望用某種他們心目中的方法.
去經歷神的救贖的作為.
但原來神的救贖.
神的拯救.
有時候不是用他們心目中的方法.
來呈現.
所以他們看不到.
不是神不拯救.
是因為他們用錯了方法去看.
所以路加說到我們怎樣體會這個天國.
不是要去找一些很大的事情.
有時候反而是在我們生活當中.
一些看似很平凡的事情當中.
我們看到天國在我們中間.
Joyce.
約翰是獨特的.
在約翰福音中.
神國只出現了兩次.
所以應該是沒有什麼可以說的.
神國只在尼哥底姆的片段中出現過兩次.
所以有學者提出.
因為約翰福音是說很多生命.
所以有人說.
在約翰福音中是沒有天國的比喻.
順便也告訴各位.
裡面是沒有天國的比喻.
裡面有的是說生命.
所以不如我從生命中.
介紹少少約翰的末世論.
在約翰福音中.
有一個很特別的特色.
就是學者留意到.
約翰特別提到的.
有一種說法叫做實現末世論.
因為耶穌在裡面說的是.
「時候將到,現在就是了」.
「誰聽我的話,又信那差我來的,就有永生」.

$^{1841}$「而是已經出死入生」.
聽起來好像是.
末世的生命已經發生了.
是一個實現了的末世.
不過近年是越來越多學者.
發覺在約翰福音裡.
除了語言之外.
還未言的因素也有很多.
但是給我的反省是什麼呢.
就是如果我們從生命.
已經出死而入生.
他有一個現在已經發生.
也有一個將來要成就的兩個層次.
今天我們面對苦難.
其實也給我們有很多的反省.
首先耶穌說生命.
就是我來要揚得生命.
並且得的更豐盛.
永生不單只是說長短.
是說一種永生的豐賢.
而這種永生豐賢在一個實現末世論的框架裡.
還告訴我們.
我們現在已經有很多的掌握.
今天有很多人說無力無奈無能.
我們的反省就是.
為什麼那種的語言.
在聖經的角度裡.
我們本應是已經拿了首期.
或者是50%.
我們的下注本應是拿在手裡.
為什麼今天這麼多人會說無力無奈無能.
永生的豐盛.
其實已經本應在我們的生命裡發生.
所以已經不是怎樣去經驗.
或者我們應該問.
為什麼我們經驗不到.
這個其實是反省多於如何的問題.
而另一個我的反省就是.
既然如果語言有一大部分.
還有一部分未言.

$^{1881}$其實教會永遠都是一個等候的群體.
它要有耐性.
它要等.
今天又有多少人肯好好的等.
這個是我留下給大家想的.
好,我們不如問一下台下弟兄姊妹的問題.
經常說觀眾,不好意思.
有一個問題就是.
請問信徒應如何實踐為義受苦.
而另一邊又在地上彰顯上帝的公義.
我想四位老師都可以回答一下.
而另一個問題就只是問張牧師.
所以張牧師你看完問題就連帶回答.
好不好.
我想那些問題.
一方面為義受苦.
另一方面在地上彰顯神的意義.
就好像兩者是對立.
但其實正正就是因為為義受苦.
其實就是彰顯我們站在上帝那邊.
所以公義才會彰顯出來.
兩者不是矛盾的.
但是我剛才說的馬德威很強調.
為什麼我們這個世界這麼複雜.
就是因為有邪惡行事.
正正就是因為面對邪惡.
即是公義的對立.
所以才有面對迫害.
但消滅邪惡的方法不是用邪惡的方法去消滅邪惡.
你只會將自己變成邪惡的一部份.
你是要站在公義那邊.
然後情願好像耶穌一樣為義受苦.
然後來表達上帝在我們生命裡面.
所完成的所結出的那種美善的果子.
所以這種才是真正的公義的發出.
是如同正日一樣將光輝發明出來.
照耀出來.
正是在最黑暗的時候.
第二條.
「你說我們要活出神公義的方式.

$^{1921}$另一方面要愛仇敵.
用愛便能消滅仇敵」.
這個感覺很像我阿Q.
「只是思想上的.
如果我們連對錯也沒有說出來和弄清楚.
神既是愛也是公義.
那麼公義如何彰顯.
其實是否會盲目縱容惡和錯誤.
受苦是自招.
而這個苦並非由神要我們經歷的」.
最後一句其實我看不到有什麼關係.
不過無論如何.
我想這裡所提出的問題是.
第一,我猜這裡是說得對的.
我們一定要弄清楚.
什麼是站在公義那邊.
公義是透過愛和憐憫而顯露出來的.
你看到上帝的公義.
如何從福音裡透過愛和憐憫而顯露出來.
你就明白這是什麼意思.
我再說一次.
如果你明白上帝在救恩裡的公義.
如何透過愛和憐憫來得以實現.
你就明白愛和公義之間的關係.
這裡的確我們要很清楚知道.
我們是站在公義那邊.
我們是站在真理那邊.
這個是我們需要知道的.
然後.
問題在哪裡.
彈上去了.
那麼.
我叫他回答了沒有.
我也不是很清楚.
略略又.
略略就是我的名字.
略略回答了.
完全跟不上你.
會不會縱容呢.
這是一個很重要的問題.

$^{1961}$我覺得這是一個很重要的問題.
我的文章裡說的搖樹.
究竟是有條件還是沒有條件.
就是要處理這個問題.
會不會縱容.
你覺得上帝愛我們.
赦免我們去做.
有沒有縱容我們.
你回答.
是不是Joyce有東西想補充.
是不是Joyce說縱容.
哈哈哈哈.
我想.
這條題目就是這樣的.
問題就是.
我們要回答的就是.
因為很多.
現在去了哪裡.
很多問題.
我想回答後面.
另一方面要在地上彰顯上帝的公義.
我想首先回應這一句.
我想說的是.
彰顯上帝的公義是上帝.
我不知道大家明不明白.
上帝的公義是在神的時間.
他要彰顯他的公義.
我想第一樣.
我們是遵行神的旨意的人.
我想這件事.
所以在聖經裡.
雅各書說.
你是誰.
你會審判人.
即是施行審判.
施行公義的是上帝.
這是第一樣.
不代表不講真理.
剛才好像有一句.
可能是張藍茉蕭的.

$^{2001}$為義受苦不代表不講真相.
為義受苦其實來自彼得前書.
所以我想用彼得前書來回答.
為義受苦來自彼得前書.
彼得前書裡不是不講真相.
是你要以謙卑溫柔的心回答各人.
這樣是為義受苦.
有人問你們盼望的緣由.
你要講清楚.
你的盼望的緣由.
拔以謙卑溫柔的心回答各人.
上帝的公義是上帝施行的.
我們會遵行神的旨意.
我們會學習神的義.
在聖經裡.
聖經的義是講righteousness.
不是講justice.
righteousness是遵行神的旨意.
如果神的time還沒到.
為什麼我們要有自己的time.
上帝已經說不如我們也想一想.
在彼得前書裡沒有說不講真相.
不過以溫柔謙卑的心回答各人.
為義受苦.
我想時間也差不多.
既然Joyce你講了為義受苦.
能不能總結一下.
約翰福音在受苦的課題上.
有什麼地方你很想再強調一下呢.
在我要寫到耶穌告別之言的時候.
我心裡有一份感動.
所謂耶穌告別之言就是farewell discourse.
在第十四章開始.
背景是什麼呢.
耶穌要走了.
在祂離開門徒之先.
祂很掛心門徒.
死了我走了.
你們在地上怎麼辦.
所以我寫到那裡的時候.

$^{2041}$我覺得我很感動.
耶穌祂掛念掛心祂的門徒在地上.
於是耶穌在farewell discourse裡說.
我必不撇下你們為孤兒.
然後祂應許的是聖靈.
你們在世會有苦難.
不過聖靈會接續耶穌基督.
去導引我們指引我們真理.
聖靈在約翰福音裡叫保衛師.
原文是paracletos.
即是被呼喚在身旁.
耶穌答應了我們.
我不撇下你們為孤兒.
你們放心.
在世你們會有苦難.
不過你們放心.
我已經勝過世界.
耶穌勝過世界.
並不是像剛才那個張顯.
自己走上帝的公義.
他勝過世界用十字架.
他用十字架來勝過世界.
但耶穌也已經為我們.
在大祭司的禱告.
祂跟神說.
神啊,我不求你叫他們離開世界.
我只求你叫他們離開惡者.
我寫到這裡的時候.
我覺得我很感動.
耶穌離開的時候.
最掛心我們的是這個.
耶穌更掛心我們一樣東西.
就是祂為門徒洗腳.
你們要彼此相愛.
眾人就認出你們是我的門徒.
當苦難的來到.
教會只會撕裂.
這是我的感覺.
Kevin有沒有什麼感動的地方要回應?.
剛才最後一個問題.

$^{2081}$你們有沒有機會回應?.
我先說兩句.
我想關注的問題是.
教會或者信徒.
其實有沒有能夠分辨到是非?.
剛才張老師或Joyce也有提到.
這是一個很重要的問題.
我們需要明白.
什麼是對什麼是錯.
教會裡面出現爭論.
往往就是大家對什麼是對什麼是錯.
有不同的看法.
所以你說受苦,你說饒恕.
我想張教授剛才說得很好.
是不是等同縱容呢?.
張教授剛才問的問題很精彩.
神赦免我們是不是代表縱容我們呢?.
神赦免我們.
但是神對我們有很高的要求.
你看聖經.
所以沒錯.
我們要分辨對錯.
我們要擇善而行.
我們要走在神的道中.
但是這樣不代表我們不會受苦.
不代表我們做對了就不會受苦.
這絕對是一個錯誤的假設.
反而我們越做得對.
我們受的苦可能越多.
就正如舊約先知.
正如耶穌自己.
所以剛才一開始已經說.
路加很強調耶穌的受苦是別墅.
耶穌沒有做錯任何事情.
但是祂的受苦是別墅.
教會的受苦也是別墅.
最後總結路加.
你提到有什麼獨特的地方.
路加很強調.
我們要.

$^{2121}$很強調.
我剛才提過.
誰在神的國裡.
很強調我們要去反省.
我們怎樣去看待自己.
怎樣去看待我們周圍的人.
這是路加一個很凸顯的訊息.
到底我們是否真的覺得.
自己這麼好.
是否真的覺得自己拿著真理.
會不會到頭來.
我們就好像耶穌所說.
周圍的人來和阿伯拉罕和坐直.
我們卻是在門外.
哀哭切齒那一群.
這是路加強調我們要反省的一個問題.
我們最看不起的人.
他有什麼可能在天國.
到最終可能.
你就會看到他坐在耶穌旁邊.
路加其實很提醒我們那種謙卑.
不要想著自己這麼了不起.
受苦其實也是提醒我們.
要在神面前謙卑.
我們沒有能力.
靠自己勝過邪惡.
但是耶穌已經勝過了邪惡.
剛才Joyce說等候.
我們等候著耶穌的德性.
最終傳言的彰顯.
我們今天見證.
用我們的生命.
用我們受苦的生命.
去見證耶穌的德性.
我就是談一點點.
就是關於之前沒機會談的.
或者一些特別的地方.
馬可福音我覺得整體來說.
他特別重視對小人物的描寫.
特別是談到他們得的這個醫治.

$^{2161}$這個共同的可能有一些.
很多不提.
我就提兩個就是.
其他馬太路加沒有的.
一個就是在馬可七章31到37號.
耶穌治好聾啞人.
跟馬可八章22到26節.
治好博塞大的麻人.
這兩個醫治是在馬太跟路加是沒有的.
這兩個地方的描寫.
就是他的描寫很特別.
就是耶穌怎麼醫治他.
就是把他們領到一個地方.
然後吐唾沫在他們身上.
按手在他們身上.
是耶穌跟他們很親密的.
這種身體上的接觸.
有探頭.
探指頭探到他耳朵裡等等.
這些呢.
耶穌就是親自的進入到這個小人物.
或者說邊緣群體的人物的生命當中.
真正的去影響他們.
醫治他們.
轉化他們.
改變他們的生命.
這個是馬可福音.
我想可能比較強調的.
另外一個我覺得這個.
另外一個我再提一點關於門徒.
門徒我可能時間不夠點簡單點.
就是他一而再再而三地描寫到門徒的失敗.
不明白耶穌是誰.
他真正的受苦彌撒亞的身份.
到最終他才發生一些轉化.
他描寫了很多關於門徒的失敗.
但另外一個呢.
因為我想正好看到一個問題.
那我就回應這個問題.
然後做一個結束.

$^{2201}$就是其中有一個問題問.
特別關於馬可福音的.
他說不少順經研究表示.
馬可福音並沒有第16章.
看似一個失望失敗的結局.
我想這個我不知道哪個同學.
或者是弟兄或者姊妹.
就不是說馬可福音沒有第16章.
應該是說不少研究是認為.
馬可福音是在16章第8節結束.
如果是在第8節結束呢.
那就是一個失敗的結局.
確實呢.
很多最古老的.
最古老的那個manuscript.
就是他的抄本呢.
其實確實都是在第8節結束的.
那第8節結束就意味著什麼.
他們就出來從墳墓那裡逃跑.
又發抖又驚奇.
什麼也不告訴人.
因為他們害怕.
就在他們害怕那邊結束了.
那這個是不是可能的呢.
我其實之前寫過文章的.
那我覺得就是說.
他很大可能是在這裡結束.
但是呢.
這個結束不是一個失敗的結局.
這為什麼要這麼去描寫呢.
這背後有他的意義是什麼.
我先跟我想跟他分享一個.
關於就是他這裡描寫.
這一幫的婦人.
其實剛才說到描寫門徒的時候呢.
他也描寫門徒.
一直是一個不斷失敗的一個過程.
但是相對來說呢.
這幫婦人呢.
他的描寫相對是正面的.

$^{2241}$對吧.
你看啊.
比如說是從馬克思四章.
他們一開始是耶穌忠實的追隨者的身份.
到耶穌臨喪時至.
門徒都四散的時候呢.
提到瑪利亞用那個.
正那竿相高高耶穌.
得到耶穌的稱讚.
普天之下無論往哪裡去呢.
都要傳這個女人所做的以為紀念.
然後呢.
十五章四十到四十一號門徒都跑掉了.
但是呢.
是這幾個婦人呢.
一直跟到最後.
看到耶穌斷了氣.
他們遠遠的觀看.
然後呢.
他們來到耶穌墳墓前.
就這幫婦人呢.
他前面的描述是相當相當正面的.
對吧.
一直都很忠實的追隨耶穌的人.
但是呢.
就是這幫很正面.
像這一幫人.
就在結局呢.
就在結局描寫之前呢.
他們都很正面.
但是呢.
到了最終的結局.
他們描寫呢.
似乎是一個非常負面.
他們害怕.
那這個為什麼經文要去這麼去描寫.
就是他們害怕逃跑.
這個逃跑呢.
跟最初門徒那個逃跑是同一個字.
所以呢.

$^{2281}$其實一方面表達呢.
就你看似呢.
就是這一幫堅持到最後的婦人呢.
他很忠誠.
但是呢.
他最終呢.
確實跟門徒也是差不多.
只是呢.
他們逃跑.
被你的時間比門徒晚了一點.
所以呢.
但是為什麼.
就是要以這個失敗作為結局呢.
如果是十六章八點結束的話.
你會怎麼回應.
如果你是你會怎麼回應.
那從我這麼這麼說.
從就是最終福音被被宣揚的時候.
我們可以知道.
後面後面無論什麼時候.
他肯定是有發生了一些轉化.
所以呢.
這個是莫納霍克說的.
就是以這樣的一個結局呢.
他其實是留下一個留白.
他是邀請所有的讀這個福音書的人呢.
去遇見這位復活的主.
你自己去遇見.
這個福音書的這個復活.
就是透過什麼.
就是邀請.
無論是第一世紀的.
當時的這個領受這個福音書的人.
或者在這個歷史長河中的人.
就是所有的當閱讀這樣的時候.
就是你看到這裡.
你自己親自上路去完成這個福音的故事.
你自己親自去encounter with Jesus.
跟耶穌相遇.
這個呢是說他特意留下的一個懸念.

$^{2321}$故意埋下的一個伏筆.
這個是一個.
就是有一個學者.
這個你可以.
怎麼說.
可以是這樣一個理解.
我覺得我也很喜歡這個.
Mohawk這樣的一個描述.
但是我同時認同呢.
其實正典呢是有9到20節的.
這邊呢.
因為我估計我沒時間.
你可以看我的文章.
在應該是.
中神期刊53期.
我應該寫過一篇文章.
關於他的殘階尾.
是怎麼回事.
雖然可能不是最初的.
但是呢我覺得他可以算是正典一部分.
因為他的整個的描寫是.
其實跟他八節呢.
其實有一曲同工之妙.
那我也留下伏筆.
那自己去看.
我就到這裡結束.
你幾位學生都留下了一點時間給你.
那最後就交給你去總結這個馬太福音.
苦難不是一個輕鬆的題目.
就是.
因為我看有些問的問題.
苦難絕對不是一個.
我們可以當家常便飯聊天的題目.
我最近在寫八幅.
心靈貧窮人有福了.
如果我們看他的背景.
二十話書.
61章是在說大流散.
說他們.
仰望的日子.

$^{2361}$就是他們那段貧窮.
那段謙卑的時候會過去.
馬太加上心靈更加顯示出.
我們在一種.
很惡劣.
甚至你可以說一貧如洗.
我們什麼都沒有.
我們有什麼.
有什麼我們人生可以把持住.
心靈貧窮人有福了.
因為天國是他們的.
哀痛的人有福了.
因為他們必定得到安慰.
馬太和馬可都是這樣說.
都是記載到耶穌.
他在被釘十字架的時候.
所呼喊.
用詩篇12篇所說.
我的神我的神.
你為什麼離棄我.
Why.
我們經常問的問題.
Why.
所以苦難不是一個輕鬆的題目.
就算面對著信仰而來的苦難.
更加不是一個輕鬆的題目.
我想我過往幾年在內地.
當然也不適合說太多.
但無論如何.
這不是一個輕鬆的題目.
是我們自己面對著.
因為信仰的緣故.
而面對著艱難的時候.
我們要做選擇.
怎樣做選擇.
當然你告訴你.
如果你說你是基督徒.
你就不用做研究生了.
如果你說你是基督徒.
大學這個教席就沒有你的份了.

$^{2401}$你要不要繼續堅持說你是基督徒.
這些選擇是不容易的.
是歧視.
早期教會.
我想不少信徒一樣面對著歧視.
當然到後來.
無論是尼祿的時候.
或者是多米田的時候.
不單止是歧視這麼簡單.
還是會因為這樣的緣故.
而失去自己的生命.
《舌行傳》也記載到.
試題犯是第一個殉道的信徒.
不是一個容易的事.
但既然上帝容許教會經歷這些.
我也深信是為了教會的好處.
人數少了.
不要緊.
我們在這些人當中.
我們看到有人做了不同的選擇.
有些人知道他自己是誰.
有些人不知道他自己是誰.
同樣我覺得馬太福音不斷地說.
我們是要做選擇.
你喜歡也好.
不喜歡也好.
就是要做選擇.
但馬太也在說一個很重要的訊息.
馬太福音中用上帝父親.
用了44次.
接著的老家只用了17次.
當然約翰用了很多次.
但在婦女福音中.
馬太用的次數是最多.
甚至當你逼白的時候.
不是說到有聖靈幫助你.
而是有我們的父.
我們在天上的父的靈.
會在我們裡面說話.
仍然是用父親.

$^{2441}$我們是天父的兒女.
我們是誰?.
很簡單.
是天父的兒女.
是天父所愛的兒女.
所以無論在哪種情形下.
我們哀動也好.
甚至他說我們在逼白中.
我們要歡喜快樂.
無論我們在哪種情境下.
我們要記得一點.
正如剛才有人提出.
我都不….
約翰福音所說的.
祂不會撇下我們為孤兒.
上帝是我們的父.
所有的信徒的生活.
以這個為起點.
但我們怎樣能夠成為.
上帝的家裡面的成員.
對於馬太來說是很簡單.
耶穌就是把百姓從罪裡面拯救出來.
耶穌就是二馬來尼.
耶穌就是那個允許與我們同在.
直到世界的末了的.
那位有權柄的.
是祂使我們成為天父的兒女.
祂的同在.
不只是和當時的教會.
也和我們世世代代的信徒同在.
約翰福音第八章說到.
記載到一件事就是.
門徒上船的時候.
然後狂風大作.
那個字,暴風的字.
其實就是地震的字.
是說中末的時候那種震撼.
耶和美的解釋就是.
教會,歷世歷代以來的教會.
就是在這條船上面.

$^{2481}$面對著很大的風暴.
祂只需要知道一件事.
就是耶穌在哪裡.
耶穌在那條船上.
祂是那位平靜風浪的主.
也是那位我們要承認.
祂就是我們的主.
一切在祂手上.
我想在座談會結束之前.
也要多謝四位.
借著大家的掌聲.
多謝四位的老師.
大家會看到.
其實可能.
我想很多時候我們看訪問書的時候.
我們也會把四卷訪問書.
合起來看.
我想今天晚上.
四位的老師.
他們分別將四卷訪問書.
獨特的地方呈現出來.
有時候不是一定要.
把它們和諧地.
合起來看.
我想今天晚上.
或者看到他們所寫的.
那本四呼音神學.
那個賣點.
就是他們能夠很突出.
每卷訪問書裡面.
很獨特的一些神學思考.
我相信也很值得.
各位弟兄姊妹.
你們可以花時間去看.
如果剛才.
可能幾位老師說得不夠厚.
是因為小弟.
阻礙了大家.
也是因為小弟.
阻礙了各位弟兄姊妹的問題.

$^{2521}$所以很抱歉.
但是因為時間差不多了.
我們不如就用一個禱告.
去作為今天晚上的座談會的結束.
我們一同低頭祈禱.
天父上帝我們感謝你.
今天晚上藉著.
四位老師的互相交流.
讓我們聆聽到.
四卷訪問書各自的聲音.
更深感認識到.
耶穌基督你受苦的意義.
求真理的靈.
你能夠讓今天晚上.
我們所聽到的訪問書裡面的真理.
成為我們繼續思考的起點.
並且堅固我們.
特別在受苦的處境底下.
仍然能夠實踐福音的使命.
我們同心禱告.
是奉救主耶穌基督的名祈求.
阿門.
阿門 阿門.
\newpage



\section{}
\label{sec:_SfD9ovSJ1E}
\textbf{「經濟嚴冬下 重塑教會使命」公開講座}
\newline
\newline
連結: \href{https://youtube.com/watch?v=-SfD9ovSJ1E}{\texttt{ https://youtube.com/watch?v=-SfD9ovSJ1E}} ~~~~ 語音日期: 2020-12-17 
\newline
\newline
\hyperref[sec:2WDL8N3ZBFk]{\small{< < < PREV SERMON < < <}}
~
\hyperref[sec:index]{\small{[返主目錄]}}
~
\hyperref[sec:hfF50g4VbXo]{\small{> > > NEXT SERMON > > >}}
\newline
\newline
$^{1}$各位網上朋友各位電影姊妹.
歡迎大家參加今晚的聚會.
我是朱光偉今晚的主持人.
今晚的講座題目是.
經濟寒冬下重塑教會使命.
我猜大家留意到過去一年.
香港的疫情新型冠狀病毒.
對整個香港社會的經濟.
甚至全世界的經濟都有很大影響.
我留意到失業率一直惡化.
有些公司倒閉.
倒閉.
我們猜想情況會一直嚴重下去.
我想某個程度都會影響教會,堂會.
所以今晚很開心.
希望透過這個題目.
研討究竟在這個情況下.
教會的使命應該怎樣去想.
今晚邀請到三位嘉賓.
在我左手邊.
黎廣德宣教師.
中華基督教優雅麻地基督堂宣教師.
我叫他Alan.
打個招呼說hello.
大家好.
第二位陳全華牧師.
博導會雅庚堂主任牧師.
我叫他John.
大家好.
最後最遠的叫蘇慕如老師.
是中國神學研究院實踐科助理教授.
大家晚上好.
雖然今晚隔著螢幕.
但希望不會影響分享交流討論的熱情.
今晚的情況就是這樣.
我們會就相關的課題.
請三位一路有些交流討論分享.
在我們分享交流的過程中.
請你們透過YouTube,Facebook.
或留言表達意見.

$^{41}$問問題.
我們收集問題後.
我們會在適當時候回答大家的問題.
今晚講座大約一個半小時.
但九時半左右完結.
事不宜遲.
我們開始.
不如我們請蘇.
我們說經濟的嚴冬.
很嚴嗎?.
你講一下嚴重的.
如果要講嚴冬的境節.
我認為要先從一些就業情況.
教會的奉獻去想.
今天在我們講這些數字之前.
我們都想知道.
我們現在在螢幕後的參加者.
他們是怎樣.
所以我們在Facebook.
預備了一些Polling.
不過在YouTube就未必有.
Facebook有四條問題.
第一條是我們想知道.
各位參加者在現在的堂會中.
你的侍奉崗位是甚麼呢?.
剛才朱牧問我.
嚴冬的景節有多寒.
我們可以看一個就業情況的數字.
暫時來說已經比較新的.
在九月份.
香港的失業率是6.8%.
其實是25萬人.
你這樣看的百分比好像不高.
大約25萬人失業.
是勞動人口的25萬人.
是勞動人口.
朱牧提到我們的定義.
失業率的份額是勞動人口.
就業預算有3.6%.
但如果你說最受影響的行業.

$^{81}$大家可能都心中有數.
旅遊業,飲食業的失業率是達到15%.
零售業的失業率達到8.6%.
但我想再聚焦一些貧窮住戶.
香港的貧窮人口的貧窮線.
是定在我們住戶收入的一半.
不是用國際標準的絕對貧窮.
在這裡是樂施會最新的.
是十月的一個調查.
他說貧窮住戶的失業人數升到11萬.
你可以對比剛才25萬.
原來大部分落在貧窮住戶裡面.
其中45%至59%的失業率近四成.
但年輕人的失業人數比去年同期增加2.4倍.
是很明顯的.
不知道我們能否看到大部分參加的人是甚麼呢.
我們可能都關心的就是教會又如何呢.
很多人都說教會的奉獻肯定都低.
之前的政主他們都做了一個調查.
原來他們真的根據9月份的調查.
10月份出來的時候.
9月份他們有170間堂會.
他們就發覺其實有70%的堂會.
奉獻是維持或沒有減少.
甚至乎是多了.
而支出有大約20%說他們可能少了.
機構又如何呢.
香港基督教機構協會就做了兩次的調查.
其實他們在3月和10月都出過.
10月的時候他們說.
由4月至9月有45%的機構出現虧損或嚴重虧損.
但是這些機構他們如何看待他們的前景呢.
他們有90%的機構認為他們不會結束或暫停.
亦都有超過60%的機構.
是在看這次的經濟情況出現.
其實未必是負面的.
即是機構的奉獻情況是比教會差.
其實一開始出現就看到.
其實年初已經有機構減肥.
減肥瘦身.

$^{121}$你還說我.
還有七折或八折之薪.
其實他們第一批是遇到的.
我們其實都有三個問題.
那三個問題其實是想知道.
我們在座參加者的堂會對未來和一些奉獻的情況.
第二條我們想知道.
就你所體會.
你自己覺得經濟寒冬對你的堂會的影響會有多大呢.
第三條就是你的堂會.
大家可以看看.
不知道Alan或John你怎樣看呢.
你說堂會的情況嗎.
說兩句.
我們堂會.
我想領袖方面都是謹慎的.
因為都看到現在外面的經濟下滑.
都會緊張教會夠不夠資源.
去我們童工的薪金.
試工的發展.
或者是堂會本身要運作的支出.
我覺得是謹慎.
帶著祈禱的心去看上帝.
那奉獻有沒有跌到呢.
奉獻又不是這麼厲害.
因為有一段時間.
政府都有補就業.
所以混在一起的時候.
經濟壓力都不算太大.
John呢.
我都回應剛才蘇老師說的.
失業率令到教會都有一個恐懼.
原因是現在失業率是6.8%.
大家知道過往香港大概是3%.
3%是什麼概念呢.
就是一個我們知道失業率有兩個概念組成.
一個叫結構性.
結構性是什麼呢.
有些人想轉工.
有些人可能回去讀書.

$^{161}$所以香港失業率是非常低的.
但今天去到6.8%.
當然又是不同的行業.
特別是領袖.
或者是旅遊業,飲食業.
去到8%多,10%多.
其實對堂會來說都帶來一定的恐懼和壓力.
為什麼呢.
我們知道疫苗是否還可以呢.
是否還有一年呢.
在中間究竟什麼時候來.
打完疫苗或是疫苗究竟什麼時候完.
種種因素的未知的確據.
都令到很多教會知道.
一方面是有不同程度的影響.
另一方面因為很多時候都是預期的.
在那個情況.
剛才看到那個數據都很有趣.
不影響.
但支出已經開始影響.
即是他期待明年會是怎樣.
那個位置我都看到很多不同的教會.
都在那裡比較審慎.
你們堂會算是大堂會.
都有這樣的壓力嗎.
其實我們的奉獻是少了.
的確是的.
還有特別有幾種.
第一就是中產.
第二就是基層.
特別是為什麼呢.
因為零售業.
餐飲業等等.
我們教會都服務相當一部分的基層的人.
那個有不同程度都有影響.
奉獻這樣.
我自己的堂會是比較中小型.
我們都是和剛才政主的調查報告出來差不多.
但我和John有一次談.
都是知道其實大的堂會.

$^{201}$無論在人數.
我都未講.
人數或者奉獻都是有影響的.
雖然政主日日出了那個報告.
都說其實就算網上崇拜.
原來香港的信徒.
都是頗穩定出席自己教會的.
或者一定會有崇拜的.
但我聽到原來在大的堂會.
他們面對的情況又很不同.
我又想問一下.
還有第三條.
第三條就是.
其實我不知道大家自己的堂會.
會不會因為預計.
即將來世空空的經濟的下滑.
而大幅度減少來年時工的計劃.
你剛才說教會的獎執開始審慎.
計劃明年的時工會不會都遷就著.
當然會審慎一點.
亦都不敢拖欠得太厲害.
但我自己的感受就是.
我覺得是兩方面的.
一方面教會就因為那個奉獻.
或者對經濟的觀察.
都是比較審慎.
很小心.
或者甚至很擔心.
但另外一面我們又見到.
其實如果時工肯去做.
肯去發展和特別回應.
社區裡面的需要的時候.
我又看見其實有很多供應在裡面.
甚至堂會本身未必需要預很大的預算.
但都能夠運作到.
甚至是超乎我們所想像的資源.
所以我見到的是兩方面.
內部有些擔心.
但你真的肯去做的時候.
其實那些的時工的發展.

$^{241}$又不需要太擔心那些資源.
這麼厲害.
可以做又不用擔心資源.
可以做又不用錢.
為什麼你會這樣想.
執事長執不會這樣想呢.
要回去問一下長執.
John.
我回應一下.
剛才大家的投票數出到第二.
他們都覺得只有11%.
覺得經濟下滑對教會不大影響.
其他都是出現一定的影響有63%.
影響大和有待觀察是27%.
我估計都頗貼近.
因為越來越長時間.
當然是10月做11月和12月.
因為10月第三波第四波.
現在期待可能還有第五波第六波.
特別是比較大的堂會.
我都有幾多接觸.
影響的奉獻10%至30%不等.
我估計都一定.
還有很多的移民潮.
但我都很欣賞Alan.
也都很同意Alan說.
我估計教會面對的挑戰.
相對我們教會其實.
或者教會本身的Excellence.
有它一定重要的角色.
是不可以減退的.
我記得當時朱牧剛才問.
我們都商討過.
你都知道.
在我們教會一開始在二月的時候.
很感恩.
我們目者和執事.
即刻開了執事會.
即刻通過了有兩件事.
第一.

$^{281}$就是我們今年向外的奉獻.
是要比去年雙倍.
因為知道嚴冬.
我們的想法就是.
如果外面嚴冬的話.
教會是否還有責任.
去分享資源.
雙倍.
雙倍了.
已經用完了.
即是雙倍.
我們剛剛早兩天立即問.
一些同工.
特別有些困難的機構.
或者我們即刻對會友.
如果他即刻有收入.
或者是街坊.
剛才我和Alan有小休.
Alan正在服務三百多個.
我們有大約六七百個.
如果街坊一旦.
他說他的收入跌了百分之三十.
我們即刻就有三個月補貼.
先不理會 先出了.
在那個感動就是.
目者和執事都一致.
其實帶來壓力.
因為我都說.
如果這樣下去.
目者可能薪金要減.
我們都說了.
二月.
減的話.
減成怎樣.
如果一中.
我就減五成.
然後我們有三個比較高級的目者.
就減三成.
即是越飢養就越不減.
譬如我們清潔姐姐.

$^{321}$或者我們基本幹事就不減.
但如果比較資深的幹事就減.
十個百分比.
有些基本目者十五.
我們就很想給電影姐妹.
有一個感覺.
就是我們要一起走.
一起不要.
不要收在自己袋.
先說一個故事.
我聽到其中一個執事都很感動.
他說目者就減.
那個執事本身提早退休.
他說我應該出來找工作.
你們就要減.
給回工資.
他已經很舒服的退休.
很愛主.
很忠心.
以前馮憲都是服侍.
亦有些突然間感染到.
他家中的中點姐姐.
他知道有很大影響.
他竟然本來一百元一個小時.
他加到120元.
但他自己兩夫妻一個已經沒工作.
你沒工作為何這樣.
他說昨天我沒工作.
但我生活到.
我還有一個可以工作.
我們現在還在走.
那些故事即刻在港島.
在門徒分享.
很多電影節目開始回應.
我自己心都很感動.
Ellen你教會有減薪水嗎?.
有減薪水嗎?.
我們有這樣的計劃.
但我們沒有實踐.
剛才聽完陳木說.

$^{361}$我都覺得奉獻裡面.
其實有些啟示.
第一其實當嚴冬來到的時候.
如果弟兄姐妹.
好像陳木他們觀察到.
本身內部的弟兄姐妹都真的減薪.
他的奉獻都會減.
我想這個層面就共度時間.
一起去面對.
但另一個情況我體會.
其實有弟兄姐妹都會看教會.
做的事工方向,意象.
是否能夠對應這個時代的需要.
以至他奉獻的心態上.
如果你真的在回應.
他是樂於奉獻.
亦都不需要覺得是一個很大的責任.
一定要給.
但如果很坦白說.
如果教會在這些困難裡面.
他給不到弟兄姐妹看到意象方向和行動.
其實很多時候自然奉獻是沒有的.
我會體會.
所以我覺得一件事就是.
教會如果能夠實踐一個使命.
弟兄姐妹如果他基本上有能力奉獻.
我覺得這個不需要擔心.
不過現在出現第三個情況.
就算你有意象,你有方向.
你真的做的話.
你的帳目和帳戶都面對另一個挑戰.
所以我覺得今天我們說奉獻,錢.
如何回應上帝給我們的使命的時候.
那裡其實有很多層面需要反省和討論.
你多說兩句.
教會又如何服務有需要的人.
我們在油麻地.
在無論是疫情或經濟.
一直下滑的時候.
或疫情一直嚴峻的時候.

$^{401}$我們就看見社區裡面有很大的需要.
剛才說過.
本身我們沒有預算或錢去回應這些需要.
但是因為那個需要這麼大.
以致有弟兄姐妹出來.
去回應的時候.
其實教會就算沒有錢也好.
那些錢是會來的.
有其他人願意去.
無論是出錢,出力.
在當中去參與.
不單止是我們教會.
更加動員到有其他不同的.
堂會的弟兄姐妹一起參加.
所以圖畫很有趣.
教會沒有預算去做這件事.
但是事工會令到那些資源.
無論是錢,人,物資,有心人.
一起注入在事工裡面.
所以這幅圖畫有趣.
既是嚴冬的處境.
但是我們的事工是很活潑的.
是有動力的.
而且不覺得沒有資源.
所以其實聽剛才幾位分享.
奉獻多多少少都受影響.
但是如果我們服務我們的社區.
有需要的人.
那裡一直維持.
資源好像不缺乏.
好像我聽到幾位都一樣.
蘇你那邊怎樣?.
其實一直都想回應.
我們在Facebook的貼文.
我們當中有42%的人是長執的.
或者是傳道.
所以我們覺得.
當然還有其他侍奉崗位的弟兄姐妹.
都覺得一起分享是很好.
但是我們看到.

$^{441}$當我們有差不多80%的人.
認為嚴冬是有影響.
但是第三條問題.
只有10%的人的教堂會.
才會大幅度地減低他的事工.
這就和兩位的目者.
我覺得看到一樣.
其實同樣都是嚴冬的景致.
但是解讀是因為你的使命觀是不同的.
我自己也要看回整個香港教會的歷史.
神是怎樣藉著香港教會.
去服侍整個香港.
有幾樣東西其實都是一些回憶.
也是香港市民的共同回憶.
例如開埠的時候的派奶粉.
這個就是剛進來的宣教士.
他們是藉著社會服務.
當時的香港政府.
其實在社福或者醫療教育.
還沒有很全面投入的時候.
教會就做了這件事.
而大家的市民都記得.
就是派奶粉.
這個很久以前的事.
這是不是說你還沒有出生的時候.
不知道我出生了沒有.
我不說了.
第二就是天台教會.
到後面1980年.
或者60年代.
香港的政府已經開始在社福界.
做了很多的支出預算.
但是香港教會也仍然是.
跟進社區做社區服務.
還有無數的機構.
我們做出來的.
這些機構就好像當時的.
基督教協奏會的郭牧師說.
機構的出現就是做一些.
教會獨立做不到.

$^{481}$堂會獨立做不到.
又或者堂會不是很方便做的事.
例如有工業委員會.
我們也有一些性文化的機構等等.
我就在看.
好像每一段時間.
無論是堂會或者機構.
都是在社會裡面有生影.
而市民大眾不會分你是堂會還是機構.
對他們來說這些就是基督教.
我看到的可能新的一個歷史就是.
會不會2019年.
2020年.
就因為堂會和機構有更多的結合.
而我們在順著上帝.
在香港的使命.
我們香港教會有新的一頁呢?.
剛才你說了一句.
我覺得有點奇怪.
有些事情是教會不方便做.
不適合做.
所以機構就代勞.
什麼意思?我不明白.
有些什麼事情教會是不應該做的?.
我剛才說是.
或者是堂會.
堂會,什麼堂會?.
慢慢地,其實我也想說.
其實我們很多時候.
看整個教會的生態是怎麼樣.
其實教會的生態裡面.
我們說是有堂會.
也有機構組織.
我就想引用Nubitin.
就是柳碧珍.
這位宣教學者所說.
她說自從18世紀.
就已經有一個東西出現.
就是教會沒有再做一些.
關於使命的社群的東西.

$^{521}$而那些出了.
由教會分出去.
做一些使命社群的東西.
它又不再稱自己是教會.
為什麼18世紀?.
因為18世紀就是工業革命.
整個很多的組織.
是有分工.
還有專業化.
當時其實.
因為整個世界的變化很大.
那個多元化.
就出現了很多不同的機構.
可能以前的神學教育都不同了.
或者退休,修院不同.
但是那個時候可能因為.
各種不同的使命.
有很多專業的弟兄姊妹.
出來做機構.
他們更快地回應.
慢慢就形成了.
堂會就變成了在後方.
它主要就是做崇拜.
還有培訓,信徒.
而堂會因為它有最多的人才.
它始終是整個教會生態裡面.
最基本的單元.
但是我們說很多使命的.
所謂的功能.
就出去了外面.
是的,這個就是我覺得最不可思議的.
有時堂會又忘記了.
其實這些組織原本出去.
其實本來那些機構是稱為.
逆風教會.
是代表先鋒的意思.
其實不是不方便.
機構可以很快地回應.
就等於這次也是.
但是越來越久的時候.

$^{561}$機構和堂會就已經忘記了.
對方彼此的獨特性.
和大家的夥伴關係.
我反而看到這次來勢洶洶的疫情.
好像令堂會重新再看到機構.
機構又更加看到堂會.
我有些小懷疑.
我不知道堂會是否真的看到機構.
在做這件事.
我想教會或者堂會.
如果不去踐行使命.
那就是做一下崇拜.
搞團契,小組.
做來幹什麼呢?.
什麼原因呢?.
John或者Evan.
你們兩個是堂會中人.
你和我們分析一下.
為什麼會這樣.
有人說堂會裡面有所謂中產思維.
堂會有很多中產人.
原本不是中產.
後來變成中產.
為什麼中產思維會把堂會搞成這樣呢?.
我不知道真假.
你們兩個comment一下.
或者我講一講.
先說回剛才.
我同意Alan說.
當有儀仗示名.
弟兄姊妹奉獻就來.
剛才我說.
當二月我們分享教會有這個需要.
亦有準備了減薪.
結果到今天一毛錢都沒有減過.
不好啊.
奉獻來得很快.
比去年來很相似.
我很同意.
不單這樣.

$^{601}$還有很多外面教會以外.
在其他教會的弟兄姊妹拍門.
他們知道我們在深水Po 做得多.
他們問你們有沒有因為這個緣故.
做不到社區工作.
他們就開始幫助.
我很同意這件事.
所以回答剛才朱牧問.
為什麼有些機構.
蘇老師.
有些情況會是這樣.
例如我們在pandemic.
在疫情.
其實我們都要冒一個風險.
除了金錢奉獻.
我想Alan都知道.
我們去到前線.
跟他們接觸.
例如今天是限聚令.
但我們不可以不繼續做.
因為你不去探望他.
不去跟他聊天.
不供應.
那是會危險.
但你不去到他們那誰理呢.
所以我想我們有一樣東西.
跟一些教會和機構不同.
機構可以很單純去.
但我們搞了整個群體.
我們還跟關懷無家者的會議說.
我們一直開會.
我們每個星期五.
他們已經很多人來拿.
他們的露宿者來拿飯票.
我們自己教會在服務100個露宿者.
即是全香港大概1000個100個.
我們不能停.
會繼續跟他們聊.
關心他們.
那個情況.

$^{641}$所以有些關注就是.
會不會機構簡單些.
幾個人有什麼.
可以動員地點.
就不會來到教會中間.
第二樣的思維.
剛才朱博也問了.
在資本主義.
或者在教會已經.
有部分教會已經.
給一些比較資本化.
或者比較富裕的人.
在那裡中間參與.
所以對他們來說.
我見到有些基層.
跟我一起去崇拜.
我有一個上市公司的董事.
隔離一個基層露宿者來崇拜.
他行不行呢.
我說不如機構做.
其實剛才蘇老師說得對.
教會好像忘記了自己本身的使命.
是些什麼.
所以教會本身一定最少有兩件事.
第一就是內在.
我們要敬拜要禱告.
信徒培育成長.
門徒訓練一定要.
第二就是對外.
我們不要忘記了大使命的福音書.
關心社群.
愛人大誡明.
愛倫社.
那個好撒瑪利安.
他跟那個人不認識.
這件事怎麼會不是教會呢.
原來有種種的限制挑戰會發生.
其實你剛才說的對內對外.
金科玉律.
我們神學生已經知道.

$^{681}$為什麼要這樣.
為什麼到教會侍奉的時候.
這件事就好像忽略了.
什麼原因呢.
你來吧.
我自己覺得.
多謝陳木的分享.
我覺得是有很多提醒在內.
因為教會.
剛才蘇老師也說了.
從歷史給我們看見.
最開頭都是西教士來到.
他們做的都是服務社群的侍工.
他們來到教學扶貧等各方面.
但現在慢慢教會中產化了.
又或者是那個年代的人.
好好讀書.
找到一份好工作.
自然就上了位.
很多剛才說的中產思維.
進入了進來.
我們很看計劃.
我們很看帳目.
我們要跟黑數師交代.
我們要跟律師交代.
這些種種種種.
令到教會一直看的.
是內部很多的需要.
或者我們以往的增長.
全方位的策略增長方面.
都是希望找到.
跟大家很相似的人一起.
所以我想越來越.
一直擴大的時候.
如果說今天有些教會面對的財困.
其實很大部份可能都是因為.
他根本自己營運教會的成本都是很高的.
無論人工,地方各方面.
最貴人工.
教會去到一個位置.

$^{721}$好像有一套電影叫做.
Too Big to Fail.
大到不能倒.
我們不能夠叫教會倒閉.
所以我們就要.
把資源放進去.
繼續令到它生存.
但就忘記了教會外面.
眾生的狀態和需要.
我覺得這個是今天.
在教會歷史裡面.
我認同去到今天這個位置.
是很值得我們.
堂會去再想想.
教會的使命是什麼.
我想說一個片段.
就是當我們說很多物資的短缺.
堂會都很怕不夠預算的時候.
我有一天星期日收到一個電話.
那個電話是我不認識那個人的.
那個人跟我說.
你們是不是在落區派口罩.
我說是呀.
他說你要多少口罩.
我說你有多少口罩.
他說我是一間口罩廠.
你要多少我就給多少.
問題就是我沒有地方放.
那個困難.
但我看到就是.
當我們很怕沒有資源的時候.
其實上帝.
都是那句.
你肯去做的時候.
那個資源和安排.
就在你計劃以外會出現.
自己送上門.
所以我就是在說.
中產思維就是我們很著重.
要有預算要有計劃.

$^{761}$要有時間表.
是一路按著計劃.
一步一步.
我不是說不好.
不過在一些情況出現的時候.
我想那種彈性.
特別我們有時候.
教會不是很大.
大就說too big to fail.
但是我們都未big就fail了.
那個情況就是.
我們的思維是大教會的思維.
但我們身處一間很小的堂會.
其實應該很有彈性.
可以沒有那麼多程序.
但有時候因為我們.
中產的思維進來的時候.
我們所有事情都要按規矩去做.
我不是說沒有規矩.
但現在我們有一種這樣的限制在裡面.
不如兩位分享就挺好.
雖然教會都奉建一般般.
你們都明明在做事.
我剛剛看到我們有人問問題.
他說現在的奉獻不算很大影響.
因為移民潮還沒開始發生.
會不會一兩年的移民潮.
一路厲害的時候.
才嚴重打擊到我們教會的奉獻.
你怎麼看.
你怎麼展望前面這個情況.
我本身真的沒有這些展望.
因為移民潮是否會出現.
我覺得這件事是自然流露.
但到時我都是這樣說.
當教會如果面對有錢的弟兄姊妹離開了.
只剩下窮的在這裡.
我們就一起窮.
如果教會本身的成本真的很大的時候.
那些有錢的又不在的時候.

$^{801}$我們就自己減少成本.
剛才牧師說得很好.
由主任開始一直減少.
你好像不是主任.
所以我很贊成牧師的建議.
不是由下面減少.
但我覺得這件事是大家都能夠理解的.
我反而著重的都是那句.
無論大教會或小教會.
老牌教會或新興教會.
你究竟有沒有意象和方向.
其實奉獻或移民不是我最大的考慮.
剛才真的有人這樣說.
你有意象有使命.
那些奉獻就會說服人們奉獻.
你教會有沒有移民情況厲害嗎.
這件事我們真的有討論.
剛才說到中產在疫情的影響.
或是未來的移民影響.
對我們來說其實也很大幫助.
因為我們剛剛在2018年建堂.
建了堂我們還要供.
還錢.
一億的我們欠別人一億.
但我們弟兄姐妹特別是中產.
我們也有一個文化.
中產有一個情況就是.
其中一個free market.
profit maximization.
多好新.
因為其中一個資本主義的看法.
就是在free market.
less safe fair.
在competitive market.
最好可以最高的回報.
你突然說了很多英文.
你再說一次.
包括幾樣東西.
第一是自由市場.
自由市場也是自由放任.

$^{841}$不管的時候.
資本主義其實有一個好處.
就是需求和供應會取一個平衡.
令資源分配有一個不浪費的情況.
這是好事.
當然相對來說會產生一件事.
因為這樣的情況.
他鼓勵資本家.
可以在裡面有一個profit maximization.
即是最高的回報.
而最高回報一定是金錢上的回報.
monetary return.
而這件事就帶來一定的問題.
我們看到環境也是.
我們要多一點.
要好一點.
要靚一點.
環境的影響就開始發生.
而很多時候.
誰會受苦呢.
就是基層.
因為我賺得多.
你基層讀書少.
所以我可以很便宜地請工人.
來為他們工作.
而他可以賺到最多.
你不做,下一個.
你不做,再下一個.
所以香港產生一個很奇怪的情況.
為什麼這麼有錢的社會.
經歷系數這麼高.
有一班這麼有錢的人.
這一班一百萬.
一百一十萬.
一百二十萬的人.
這麼貧乏.
而最低的工資.
只有這樣.
我每天做十個小時.
做六天.

$^{881}$都不可以有一個住的地方.
這樣產生了一個問題.
就是很多人覺得.
這件事就是這樣走的.
我們可以在部份.
在教會裡面.
我們就.
中產繼續敬拜.
在裡面.
但是問題是.
他見不見到那個社會責任在哪裡.
這一班人.
其實是一個犧牲品.
某程度上.
他的責任應該是怎樣回應.
怎樣看呢.
所以我們很想.
一直在教會.
過去幾年一直想產生一個新的文化.
就是不是多好靚身是好.
是我們分享是好.
同行是好.
而在過程裡面.
特別我剛才說到.
我們都見堂.
都辛苦的.
我們都說在2019年.
我們都有十一奉獻再加一個月.
再說一次.
十一奉獻再加一個月奉獻.
即是十一奉獻以外.
再一個月薪金再奉獻.
那為甚麼我們來長沙.
我們是太子.
因為我們很想在長沙環心水Po .
這裡有個地方.
這個地方可以服侍貧窮人.
所以我們教會其中一樣東西.
就是我們所有機構.
科技機構來我們用.

$^{921}$我們是不收錢的.
任用.
Open我們教會任開.
這是第一個.
我們一來就已經做了.
但是第二當然就是.
我們這個地區有很多.
比較.
你都知道板間房.
越來越多新樓.
舊的樓越來越壓縮.
舊的樓所以板間房越來越貴.
因為舊樓越來越少.
但是貧窮人也有一定的數目.
所以他們逼迫租金越來越貴.
所以我們就去.
去和他同行.
去聊天等等.
剛才說到一個好問題.
咦.
我們今年就說十一奉獻再加一個月.
就不是建堂了.
上年是.
就是一個月就拿去給你.
見到自己的親友街坊不夠的.
你那個月的糧.
就拿過去.
所以我們要.
Demaximization(降低).
聽到很多很特別的情況.
就是我們有些人做迪士尼.
大英姐妹.
我們知道迪士尼.
出人工都是五分四.
因為沒得開工.
他們就說.
陳牧師我有五分四人工.
不過我回來.
那五分四沒有了.
我奉獻了.

$^{961}$我真的很感動.
他們都不是多餘的.
是分享的.
還有一個很有趣的情況.
就是.
有一個他是做銷售的.
頭三個月.
他基本家庭的開支有了.
接著他就可以從四月份.
全部奉獻了.
我覺得這個動作很特別.
剛才當我問到.
在資本主義等等.
這個教會文化.
我反而問.
怎樣可以將文化改變.
好了,接下來我們知道會有人移民.
已經有了.
已經出現了.
已經在談.
很多不同的教會.
我都在談.
估計十至二十個百分比中產.
我們的傳道.
還有我們的教會都說.
我們不需要有堂會.
必要時沒有了.
賣掉.
租回.
第二.
我們在問傳道.
我都很感動.
很多都是忠實不要.
很感動.
我問他們什麼呢.
你們的照明是不是做牧者.
沒有工資.
如果日後.
你做不做.
如果要自酬.

$^{1001}$就是說有可能會.
所以有些人說話.
我都很同意.
這樣的情況.
如果我們照明真的.
你信不信就算是這樣.
你自酬.
突然之間會公認.
信.
所以我們準備.
準備明年如果真的有這樣的情況.
我們的課資.
我們都預測可能真的會.
我們怎樣回應呢.
我們回應.
不可以沒有了教會的使命.
不可以沒有了教會一定要做的事.
內裡敬拜成長培育滿途.
向外使命.
大誡明.
到明年什麼情況.
我們再整合告訴你.
上帝有恩典.
不過你這個說法就是.
無論我們有多少資源.
無論是奉獻數字跌.
或者因為移民的情況.
所以奉獻都會少了.
你就一定堅持.
不會縮減.
你們一直對外的服侍.
對有需要的人的關懷.
情願去到情況.
我覺得你好像有點pre-emptive.
先跟你的同工說.
萬一減薪.
萬一沒薪水.
萬一要自籌薪金.
會不會呢.
這個是一個.

$^{1041}$你可以說是很聰明的做法.
但亦都是比較預先想好將來會怎樣.
我覺得這個是很好.
會不會因為這個緣故.
結果同工離開我們不知道.
不過最低限度.
我們正在走這條路.
蘇,你是不是有什麼想補充.
其實我只是說理論.
但我感動聽到John或Ellen.
他們做到出來.
我始終覺得剛才朱牧民.
如果很多人移民又怎樣呢.
我想說.
我教會有個執事.
他就這樣說.
不要緊.
如果移民了.
他就祝福那邊教會.
我們少了一個人.
我們辛苦了.
但整體上帝的家都是好的.
我仍然聽到.
其實我經常覺得.
最主要中產的思維.
影響了我們.
是以堂會的企業發展為本.
我們怎樣都要保住我們的人數.
企業發展為本.
多說一點.
企業發展為本.
我們不記得我們是受差的.
我們很多時候.
我們是在一個差權那裡.
我們是Sending Church.
我們是差遣教會.
堂會慢慢以為自己真的是教會.
而且不記得自己是受差.
但如果換回來.
其實堂會是整個教會裡的其中一份.

$^{1081}$我們的人才是可以互相流通的.
甚至實際上.
原來有教會真的這樣說.
我們沒有堂會都可以.
我們以企業發展.
人少不可.
我們的堂會一定會越來越大.
弟兄姊妹奉獻.
就是希望我們的堂會舒服一點.
然後給我們生老病死的宗教服務.
我要有專業人士跟我們說.
這就是整個中產的思維.
我們用這樣的方式去經營教會.
但我看到.
其實我剛才說的這些反思.
其實是那些目者自己說的.
目者已經有這樣的看法.
所以重要的是.
我又要來回拋書本.
重要的是.
其實我們記得.
什麼叫做教會的本質呢.
不是上帝為了教會去設立使命.
而是使命是教會的一個事工.
而是上帝是為了他在世界的使命設立教會.
教會的身份.
教會的存在目的.
正寫著就是使命.
所以他必興旺.
我必衰微.
我想在目者是要很記得.
我想在目者機構也是.
我知道有些機構.
其實願意在這段時間會合併嗎.
是否一定要保住自己的堂會或機構這麼重要.
而最重要是看怎樣去達到使命呢.
這個我又很想看.
因為Alan好像有問題.
我想回應我們Facebook的問題.
問題很長.

$^{1121}$不過值得我們討論.
奉獻會否與受眾使命形同化掛鉤.
他說如非作基層事工.
而有機構或堂會議長.
真的是作教社會地位較高層的工作.
高端人士.
奉獻會否真的不一致呢.
是教會介動議長使命.
還是有私心令教會引發.
我開始不明白他想問甚麼.
簡單來說.
如果教會不是以服務基層.
你們兩個一個是長沙灣一個是油麻地.
不是以服務基層為目標.
使命是服務中產或以上的人.
會否影響這個構思呢.
有甚麼回應呢.
這個問題其實都想過.
有些節目都問過.
第一件事就是究竟使命是誰定的.
使命是我自己按自己的感動.
按自己的計劃.
或是我們這班人一起討論.
這是我們的使命.
是否可行呢.
我覺得是可行的.
如果不是在教會裡.
我覺得是可行的.
但在教會裡.
我們真的要回到聖經裡.
去看上帝如何吩咐.
由舊約至新約.
由蒼世紀至啟示錄.
我們如何看貧窮.
其實這樣的眼界.
我自己覺得.
真是上帝給我們一個很重要的吩咐.
我自己的體會就是.
有些人未必看到.
就算看到也好.

$^{1161}$他的心.
用回耶穌的好鄰舍的故事來說.
他的心未必有感動.
就算他有心有感動也好.
他做的時候.
他也覺得是沒有用的.
因為好像在大海裡撿海星.
我做了多少呢.
我覺得如果從這個眼界去看使命的話.
其實就是你自己定一個計劃.
看你能否做到.
但如果按著上帝給我們的吩咐.
就是你看到你的弟兄當中.
是有貧窮的.
或是社群裡面.
是有人患難當中的.
其實你不是因為你的感動.
是因為上帝的吩咐.
你就要去做.
我相信無論在長沙灣.
在油麻地.
或是其他區裡.
教會在其中.
其實是有一個上帝給我們的任務和使命.
所以回應問題.
我都要問.
究竟那個使命是誰定的呢.
如果是上帝定了.
我們真的要關心貧窮的時候.
我們就要問.
我們如何看這個使命呢.
不過你都是以一個關心貧窮的角度.
來回答這個問題.
John呢.
我們教會的任務有兩方面.
一方面就是作門徒.
作門徒一定包兩邊.
一個就是自己成長.
第二就是Disciple making.
第二就是關顧貧窮人.

$^{1201}$所以我自己的看法就是.
如果有部分人覺得.
我想服侍職場.
或者我想在醫生團隊.
但整體教會裡面.
不可能所有人都是在一方面.
我們就在分配.
你說這樣可以.
你說這樣.
所以我們自己就定了.
如果我們教會大概沒有五成人.
不斷去探訪.
我覺得就已經開始出現了一些問題.
對我來說.
我們教會現在的比例.
當有15個人.
10個人行常去教會.
8個是中產.
2個是基層.
有5個還沒有去教會.
我有15個人.
就是我們行常去探訪的.
那5個行常去探訪.
就是教會這10個人裡面.
其中一半每個月都去探訪.
所以我們就鼓勵他們.
如果是這樣的話.
我們會調撥教會資源.
中產奉獻.
醫生奉獻.
上市公司人奉獻.
對.
我們就可以調撥了.
你就覺得這件事我們會.
錢會到.
但我們都很想人都要到.
因為其中一件事.
我想我和Allen都很感受到.
或者我們看完節目都感受到.
當去探訪的時候.

$^{1241}$你只是給.
和你一起去碰它.
那個對象是不同的.
那個對象是感受到.
你買一樣東西放下.
和他談.
關心.
又很不同.
所以我們曾經有一次出隊.
我們幾十人去洗樓.
有一個都是專業人士.
我就問專業人士.
你為什麼.
你明明分享你真的沒有興趣.
為什麼還想來試行常呢.
他說一樣東西都很特別.
他說我真的不想也沒有興趣.
不過聖經要.
如果聖經要.
為什麼我不去學呢.
為什麼我不去試呢.
我眼淚就流了.
原來有些很真誠.
第一就是剛才我說的不同的資源分配.
第二有些他都想試過去那個位置.
一路去經驗.
這兩個情況一路互動就很特別.
我有一個問題.
我覺得很有意思.
外國有大積木者的觀念.
會否這是一種的路向呢.
可不可能呢.
可能要自酬身甘.
我的照明是要做木工.
木工工作.
不過是教會給不到薪水.
可不可以.
這裡所說大積木者.
怎麼看.
我覺得現在的趨勢都會是這樣.

$^{1281}$趨勢.
如果背著教會的名字.
其實越來越有不同的挑戰.
甚至危險.
同時我們有很多營商宣教的宣教士.
其實都在用這個模式去做.
我問他背後的意思是.
其實木者是否自己能夠有一份工作.
可以賺到錢.
所以他能夠自己生活呢.
我想背後的動機是重要的.
你好啊.
我做了幾十年木師.
除了講道什麼都不懂.
怎樣大精呢.
重新學習.
真的重新學習.
有另外一個問題.
這個是YouTube問的問題.
這個很好.
我想這個是信徒問的問題.
蔣執不贊成減教木工薪金.
不贊成減宣教經費.
不贊成減差權經費.
個個時都不能減.
這樣怎麼辦呢.
他有一點覺得.
什麼都不准減.
這樣怎麼辦呢.
是好是不好.
他好像覺得不太好.
他問的語氣好像.
我猜不知是他覺得不滿足.
覺得不減或減好.
應該加.
我猜很值得在教木團隊和蔣執.
很清楚定教會的意向方向.
而我自己看很多教會都會問.
如果教會都不想行.
行不到.

$^{1321}$不要緊.
你自己去行.
你帶一班弟兄姊妹去做.
即是在服務教會全心全意.
服務教會以外.
你有很多侍奉時間.
你想想我們在職的信徒.
除了上班40至45小時.
回來教會侍奉又要探訪.
又可能要帶小組.
又要崇拜.
額外5至10小時.
不止.
額外不止.
每個星期真的不止.
我們會不會同工.
有時你很有火熱的心.
但教會本身的意象未有.
但你都全心做了教會比例.
你還有10小時8小時.
你就推動去行.
有些人真的會行.
開始去不同機構.
剛才聽Alan說.
他的位置都有不同教會.
突然會去.
牧者都可以去.
我估這個位置可以在一開始的時間.
可以去嘗氣水溫.
當他感受到那種火熱.
並且那種影響力.
他帶回去.
跟他長洲分享.
一步步希望在融合協作裡面.
可以有新的景象.
剛才我覺得你很想關心貧窮人.
或基層的朋友.
你介紹多一點.
我聽到你有四個大字.
由淺入深.

$^{1361}$你介紹多一點.
首先我想強調一件事.
不只是我個人去關心貧窮人.
我覺得聖經教導我們所有信徒.
都應該關心貧窮人.
不過每個人的眼界和時間.
在什麼時候.
是上帝感動.
但我覺得在這個疫情裡.
再加上我一直的牧養經歷裡.
我看到社區裡有一個很大的需要.
所以跟弟兄姊妹一起去關心.
在公園裡面的一些街坊.
那個由淺入深是甚麼呢.
其實我們有一條片.
有片看嗎.
是有片看.
不如我們去一去這三分鐘的片.
省下我說的東西.
我們先看看.
好 看片.
油麻地.
一個充滿不同需要的地方.
在2017年左右.
我們開始定期在榕樹頭公園排塘水.
在果欄幫婆婆推紙皮.
探訪獨居長者.
漸漸體會到這個社區.
有很多被社會遺忘的人.
被邊緣化的人.
活在孤獨世界的人.
我們發現自己原來是很無知的.
不認識繁榮背後的真實世界.
但是我們可以為這個社區的人和事.
做些甚麼呢.
心裡浮起很多四字詞語.
生耕細作.
細水長流.
由淺入深.
後來我們將由淺入深.

$^{1401}$轉化為同音不同意思的四個字.
由淺入深.
簡單說就是在油麻地實踐上帝的道.
讓反省進入心裡.
由淺入深地帶來更新和變化.
讓我們看看這班由淺入深的義工.
營養照顧區內有需要的人.
早一兩天義工會打電話.
約街坊來拿物資.
義工也會按他們不同需要去預備物資.
亦有義工去餐廳和酒店拿飯.
有義工在教會附近接待街坊.
再通知其他義工預備物資.
另外我們有外送服務.
給年長或行動不便的街坊.
星期三晚.
我們有更多義工.
一起去到油麻地不同地方.
探訪露宿者和分享聖物.
我們每星期的服務已經維持了一段時間.
不少街坊都做了我們的好朋友.
甚至一起參與服務.
看完這個簡介.
你有沒有感動一起參與?.
歡迎你加入由淺入深.
這個以展行信仰.
轉化生命和社區的行動.
就好像聖經《馬太福音》25章40節說.
「我實在告訴你們.
這就是你們既做在我者弟兄中.
一個最小的身上.
就是做在我身上了」.
在這一刻的香港.
我們是不是什麼都做不到呢?.
香港人我們又看到什麼呢?.
我們又看到什麼呢?.
Helen 這個好東西.
由淺入深你拿到精髓.
你這個構思.
不會完全沒有一個所謂的教會觀.

$^{1441}$剛才蘇老師說過.
教會觀很棒.
你介紹一下你這個由淺入深.
整個背後的意念是什麼?.
其實我想說教會觀.
我已經有20年了.
我自己的感受就是.
在堂會裡面.
我們現在有一幅圖畫.
其實在堂會裡面.
我們要不斷負責很多不同的事務.
所以我想今天很多的.
參與的弟兄姊妹.
分享的弟兄姊妹都是目者.
相信大家都很明白.
其實在堂會裡面.
真的有很多不同的事務.
我們需要去服務.
需要去關心.
需要去管理很多的東西.
所以我們的心思其實就去到.
以堂會為中心.
這種想法.
但是久而久之發現.
堂會以外的世界.
那個情況我們是沒有空間.
沒有心力.
沒有動力去真正關心.
我自己的體會就是.
特別在疫情當中.
當我們面對三百多個街坊來到的時候.
其實堂會是不知道怎樣回應的.
因為堂會本身都有很多的事情.
目者需要去處理.
要去關心.
做不及.
根本做不及.
所以我自己都有一個體會和想法.
就是另一幅圖畫.
就是其實教會不是我們自己為中心.

$^{1481}$不是我們自己想做什麼就做什麼.
而是我們應該是一個.
我自己看為.
由一個自我中心.
轉為一個基督中心的關懷群體.
而以基督為中心的就是.
究竟耶穌基督將教會放在哪裡呢?.
就是在社區裡面.
這幅圖畫就是看到.
其實社區裡面除了教會之外.
還有很多的機構.
有心人.
對於社區關注的人.
在服務的人.
這些人和機構.
我們怎樣能夠結連在一起.
去服侍上帝將教會擺放在社區裡面呢?.
由「錢入心」這個.
我覺得是使命行動.
就開始在裡面去運作.
而我們覺得既然.
不應該是一間教會去承接所有的街坊.
這樣的話.
我們就需要結連其他的.
堂會和機構.
去一起去做.
所以剛才說.
你說傳道人會不會走出來.
自己做另一份工作.
去繼續服侍呢?.
其實我就是有這樣的經歷.
所以我重新學習.
就是我都要在裡面去學習.
怎樣去結連堂會.
一起回應社區裡面的需要.
所以由「錢入心」的概念.
我覺得是一個.
現在來到這個位置.
我覺得是一個使命的行動.
讓堂會一起去關心和結連.

$^{1521}$因為我覺得如果剛才那個圖.
看到各個教會都是自己.
要花這麼多心力去照顧自己的話.
其實我們和其他教會.
同區的其他教會.
都沒有什麼機會合作和交流.
或者那種交流和合作的關係.
都是只留於我們一個團契式的關心.
就未能夠一起去回應使命.
這個很好.
你服侍社區裡面的貧窮人.
我這裡有一個問題.
他說現在可能有不少年青人失業.
沒有錢.
他們變成窮人.
如果你們這個油麻地的教會群體.
可以怎樣服侍他們呢?.
或者John你也可以回答一下.
或者Sue你們幫忙回答一下.
我覺得貧窮的年青人.
我這個問題問得很有意思.
越來越大群.
那怎麼辦呢?.
其實我講一點點親身的經歷.
現在我們其實逢星期二,三,四,日.
我們都會到區去服侍.
無論是派飯.
關心街坊.
原先大部分都是長者.
但現在越來越多年青人.
有些是很年輕的.
有些是在打工.
但他因為真的付不起錢去租tong 房.
所以就睡在天橋底.
我們都有這些街坊去服侍他.
特別現在這麼冷.
我們都會拿衣服給他.
我自己有個發現.
其實現在很多街坊.
他在工作市場裡面.

$^{1561}$可能他被裁出來.
或者他很想努力找工作.
但都找不到的話.
我們有時會慨嘆.
我很努力幫他找到工作.
送他回到工作市場裡面.
但剛才所講的資本主義的氛圍.
其實令到他們現在就算找到工作.
都仍然要睡在天橋底.
因為他現在的處境.
他就算找到工作.
他都被壓到.
人工,工資方面都被壓榨得很厲害.
所以有時我們都會覺得.
我們這麼辛苦建立他個人.
希望他找到工作.
其實這個所謂向上流動的想法.
是很正常的.
但在一個不正常的社會裡面.
我以為他能夠向上流動.
但他出來的時候.
他都在食物鏈裡面的最底層.
這樣的話我就覺得.
其實作為信徒.
剛才講的中產化的信徒.
其實都要另外一個想.
當你做了老闆.
當你做了一個機構主管的時候.
你會不會能夠有另外一種的思維.
是能夠帶動到.
我們不一定要跟那種賺到盡.
或者是用公司的利益為本的想法.
而是真的讓人性有回一種憐憫的心.
我們是否一定要跟隨現在的最低工資.
我們能不能夠有一個.
所謂生活的成本的概念.
建立一個人.
所以現在我們的做法是.
有時不把他馬上推回去工資市場.
我們聘請他回來做社區大使.

$^{1601}$我們幫他處理食物,住宿的問題.
但不會這麼急於幫他找工作.
最低限度是先熬過這個疫情.
甚至我們可能在等候一些.
比較良心的僱主之類.
是願意用一個比較好的工資去幫他們.
不過我覺得你剛才說的這個情況.
不只香港.
早幾天我好像在外帶都提過.
北美.
我前幾天看Washington Post.
他們提到大大隻的標題.
Newshead Night說.
Stealing to survive.
More American short-living food.
很嚴重.
在超級市場他們很多人.
沒有東西吃.
要高賣.
如果看內文.
超級市場的老闆都不想打電話報警.
算了.
隨便他了.
Claim保險算了.
如果看另一句.
26 million doesn't have enough to eat.
26個million即是2600萬的人.
他說不夠東西吃.
這是北美.
富裕社會.
所以我想補多一句.
其實我們對貧窮這件事.
我覺得我們不陌生.
堅尼系數.
其實我們一直都知道.
但問題是我們看不看得見.
其實這不是貧窮的問題.
我覺得這是一個零性的問題.
我們面對一個處境.
其實我講多一個例子.

$^{1641}$現在我們星期三.
有時我們有些機構.
在果欄問店舖拿一些水果.
我們下午派給街坊.
我曾經去過一次.
我見到一幕很震驚.
就是他們將爛了的水果.
丟進垃圾站.
而很多人衝進去.
立即拿.
立即開.
立即吃.
這些場景.
很坦白說.
我只在菲律賓.
柬埔寨的垃圾山裡面才見到.
我想不到在香港的處境裡面.
是見到.
所以我覺得不是一個貧窮的問題.
因為貧窮.
我們每個人都知道.
但問題是一個眼界的問題.
我們看不看得見.
看得見之後.
有沒有動了持心.
動了持心之後.
我們能不能夠放手.
讓上帝去成就那個工作.
現在我們在這些位置.
我覺得就是斷得很緊要.
以致我們很著重的就是.
究竟有沒有成效.
是否我們要幫他.
還是社福教會幫他.
教會會不會做得太多.
或者這個人之後.
我幫了他.
他會不會.
其實反過來.
其實他只是.

$^{1681}$take advantage.
其實我們太多這些想法.
以致我們去關心.
不要說是貧窮人.
關心有需要的人.
裡面我們有很多的障礙.
所以我覺得貧窮.
這個不是我們不認識貧窮的問題.
而是我們靈性的問題.
你想想那個比喻.
很純碎.
其實不是說.
真是有個人受害.
而是說.
最開頭是說.
怎樣才有永生.
其實整個論述.
根本就不是說那樣東西.
五餅而魚.
不是說沒東西吃的問題.
是說我們在困難的裡面.
是否信靠上帝.
我覺得我們今天信徒.
怎樣看聖經.
和怎樣去對應處境.
我覺得這件事.
可能我說得太多.
我開始想說道.
剛才我覺得你那兩幅圖畫.
教會的使命.
我覺得是很棒的.
Sue 你那邊有補充一下.
教會觀這件事.
我不是想說教會觀.
你說吧 你想說什麼.
其實我也做過扶貧的項目.
我也想.
剛才很贊成.
兩位朱牧和Allen都說了.
其實我們中產的人.

$^{1721}$有時不太看到.
貧窮是結構性的.
我們也不太看得起.
那些找到工作.
但還是那麼窮的人.
因為其實2012年開始.
已經有一個term叫窮亡族.
他有工作.
在職貧窮.
但是怎樣做.
他都不夠.
因為生活成本很高.
現在全球化.
經濟下滑.
在這個階層的人.
更加遭受到那種慘況.
而年輕的窮亡族是增加了.
我想除了教會.
我們是要一起去做.
所以我覺得.
為什麼貧窮人是我們首要去幫忙.
他們真的沒得吃.
他們真的會冷.
但另外一件事.
就是需要專業.
我那時候去做扶貧.
我看到是要做社區扶貧.
要做一些倡導性的.
即是advocacy.
在政府的政策是怎樣.
或者幫.
其實從小時候.
由小朋友開始.
我們開始有些教育.
我以前在內地.
我們會有一個叫做.
農二代的一些計劃.
再說一次.
什麼農二代.
農民.

$^{1761}$以前農民.
在以前的社會還有機會.
你去到城市.
還可以攀上.
攀上這個社會階層.
但現在.
其實那個社會結構已經很明顯.
你其實根本連租屋都沒有.
即是沒錢.
你怎會有錢去進修.
你沒錢.
怎會從小到大.
你就可以去其他的外國去交流.
那你當然.
你的社會資本自然少過人.
你一定很難攀升到社會.
上不到社會樓.
所以我們有些人就在想.
去幫一些農二代去做.
所以我覺得.
其實除了堂會做一些.
制貧式的.
其實都可以是想.
如果一些專業人.
他們想去做一些社區發展.
或者小朋友.
這一類的政策去做.
我覺得基督徒其實.
是需要去想一些貧窮的問題.
John補充.
我自己的補充就是.
我用另一個方向去看.
特別是中產或者資本家.
我自己的看法就更加要在.
大事明償.
去將福音給到他.
原因就是.
當他經歷一個.
我們出生在資本主義的文化浸淫.
但我們怎樣可以將價值觀扭轉.

$^{1801}$讓他見到上帝要他見到.
讓他行到上帝要他行.
他必須要經歷福音徹底的更新.
那個很大的扭轉.
叫到他真的可以開始明白.
其實他的責任.
是要治理上帝的資源.
資源放在他那裡.
他怎樣可以好好分配.
怎樣用公義和愛可以走出來.
所以我們都有些基督徒.
在他的生意裡都不少.
他確保他給的人工.
是七成.
比這個平均.
七成或以上的人工價格.
最低人工現在三十多元.
你知道中間有多少.
可能有人給四十五十六十.
有些七十有些八十.
他一定要確保.
在那個貧乏階段裡面.
那個階層.
他的人工是要高於七成.
現在的百分比.
有些是在國內.
五十元的.
最少.
有些是在國內開廠的.
他在國內有個統計.
他是超過國內工人的人工八成以上.
為甚麼呢.
因為他知道他開始原來.
不是需要資本主義的利潤最佳化.
我們是要天國最佳化.
在神國裡面.
怎樣可以將神的福音和真正永恆.
改變生命.
所以當他們扭轉.
我自己經歷到.

$^{1841}$很多都是.
他開始發現.
原來我人生的目標.
不是在地上多.
是要忠心.
作神忠心的管家.
很難得.
這樣的想法很少有.
所以慢慢成為教會文化.
一見到這些故事.
我們就分享出去.
一見到這些故事就分享出去.
見到他們發現原來.
他們第一不是一個人走路.
第二他們發現原來這樣.
他們裡面有一個滿足.
是由神而來的.
那個位置就很不同.
沒錯.
這些故事分享得很好.
剛才你分享.
很多人服務幾百個有需要的人.
有人都很認同.
我們見到有人很認同.
不過現在疫情這麼嚴重.
有沒有人受過感染.
會不會在封市區.
暫時沒有.
有突然之間會擔心.
會害怕.
怎麼辦.
我覺得都是一個眼界的轉化.
就這樣說疫情來說.
其實在社區裡面.
我覺得社區的公共衛生問題.
我們不只是隔離自己.
保護自己.
戴口罩.
不要出街.
社區裡面其實很多有需要的人.

$^{1881}$他們沒有口罩.
他們要出街.
我們接觸很多隱蔽長者.
平時沒有機會見到他們.
為什麼會出來.
他食物都沒有.
出來口罩都沒有.
在這樣的環境裡.
你在社區裡面.
你看見的不只是自己的安全.
我想這都是聖經教我們的.
你看見有人有需要的時候.
或者教歷史給我們看見.
在疫情裡面.
其實是基督徒走出來的.
我想在今天的處境裡面.
我們正在學習功課.
當然我們都要很小心.
我們都要不要變成某些群組.
但是我們見到有這麼多人.
不夠這些資源的時候.
其實今天的處境.
很多本身在服務的機構.
都因為疫情而停止了.
為什麼我們有三百多人.
其實故事是.
本身有些宗教團體.
有些機構.
每個星期日都在照顧這些人.
但因為疫情而停止了.
這班人怎麼辦呢.
這班人裡面有三十多人.
是我們平時照顧的.
我們打算照顧三十多人的時候.
他就將其餘的二百多人帶過來.
在這個環境裡面.
我就看到上帝一方面給我們很多資源.
至少口罩都不會缺.
第二方面就是人手又來了.
我想整件事.

$^{1921}$我們都是要靠上帝恩典.
有沒有感染這件事.
但我覺得就算在這個風險裡面.
我們仍然靠著主.
我們要去做.
所以你的說法就是.
你超越了自保的心態.
是有受感染的風險.
其實都害怕的.
我們教會的大廈.
都有兩次樓上有人中招.
在那個環境裡面.
其實我們要停是很容易的.
但等姐妹都說.
我們停了.
剛才陳木都說.
我們不吃都好.
街坊都要吃.
欲罷不能.
那怎麼辦呢.
又有心人去.
我們有酒店.
有些社企.
他們每個星期都有飯盒和飯票.
我們只是差點去派.
我們就做這個流通管治.
我還有一個問題.
回答這個問題.
問題是社會需要和公認的比例實行.
社會需要更多教會關心貧窮人.
但很多教會是服侍中產.
要服侍的人多.
貧窮人多.
但很多教會都放了私人在服侍中產.
這個失衡.
有人扭轉.
剛才John牧師說.
教會的文化改變了.
為什麼這麼多故事發生在你教會.
為什麼故事發生在你教會.

$^{1961}$其他教會都可以發生.
有什麼情況下可以做到.
我猜耕教一定可以.
很好的專門問題.
為什麼有些教會可能會忽視.
我自己看中產是一個非常好的群體.
當他信主.
他扭轉.
他有資源.
他有知識.
他有能力.
他有思想.
他有計劃.
所以我們教會一開始成立的時候.
就定了其中一個方向.
就很想推動中產.
落地.
有錢有資源一起去走.
而這個文化我們常常鼓勵他們.
其實教會如果沒有他們.
我們有一句口號.
就是沒有義工就沒有事工.
目者的比例很少.
我們目者比起我們常常去服事的人.
是一對五十.
他們是五十倍.
所以我們不停訓練弟兄姊妹.
訓練領袖.
去到一方面門路訓練成長.
為什麼.
因為當他轉換信上帝的角度.
很多人信主都是.
我被祝福.
我今世我賺錢.
即中產.
永生我又盼望.
我贏了.
但如果真正經歷上帝扭轉門徒訓練成長.
他知道作為門徒背起十字架跟從我.
要寫記.

$^{2001}$當他改變思想.
原來信上帝給他祝福的資源.
是叫他去分享.
叫他去祝福.
他的心態就很不同.
曾經有一次.
第一次露宿者來崇拜.
都好幾年前.
那次我都覺得歷歷在目的記憶.
當時有一個拿著一代二代露宿者.
衣衫藍柳.
那天還是聖餐的崇拜.
坐在距離我三四行.
我都聞到很濕的氣味.
但他就坐在一個上市公司的CFO.
即一個財務總監的旁邊.
其中一個是一個三十歲的姐妹.
崇拜完後吃完聖餐.
我就問弟兄姐妹.
你覺得怎樣.
他就問我.
牧師你的意思是怎樣.
我說你覺得有味嗎.
因為我都覺得不是好意思.
因為當時我們去外展的時候.
我們想不到真的開始來.
一來的時候怎樣融合呢.
結果坐在旁邊的那個.
亦都是在門徒成長.
經歷了兩年一起成長訓練.
他的回應我很感動.
他說我聞到有味.
不過是基督的香氣.
我突然間就流淚.
然後我就將他的分享見證.
在港台講.
我們是要這樣歡迎.
我們是一家人.
我們不分你是中產你是超幼稚你是醫生.
我們都是門徒.

$^{2041}$我們在主的國裡都是神的兒女.
所以這個很清晰的思維改變.
加上作為領袖要先行.
我們一定要先行.
因為如果不是我們想你去你去你去.
但自己不去.
姐妹姐妹就好像看電影一樣.
你不去我怎去.
去完一次我沒有感覺.
但那個的意象和方向.
一定是從領袖落水開始.
這兩個元素就帶動了教會文化的改變.
我不知道還有沒有少少的事.
來來來分享一下.
我都想說當我們說一個是使命導向的教會.
其實就需要有使命導向的領袖.
有使命.
我講的是長則.
不是只是說堂主任.
但其實都需要有使命導向的架構.
或者制度.
剛才有人問可不可以全部人都是目者.
我想告訴你.
我在國內其實很多是自己有份工作去做.
但教會是全動員的.
即是有很多義工.
如果你的目者是兼職去做.
但你們還想著他打份工回來又全職.
是不可行的.
但我聽到剛才John或Ellen說.
其實很重要的就是.
你的組織架構是要總督.
在動員的.
在我們教會.
教會沒錯.
堂會崇拜是很重要的.
但在崇拜裡面.
我們幫他們去經歷到做門徒.
和與基督有個團契.
但之後是基督可以讓他猜獻出去的.

$^{2081}$有時我們的崇拜可能流程就在於.
有沒有好的師哥.
有沒有好的講道.
令他覺得被服侍了.
但沒有了那個可以猜你出去的.
如果整個架構組織.
或是組織也是我們決定權.
我們決策的過程.
是鼓勵領袖同心一起去.
領受聖靈給他的意象和使命.
然後一起去擁有這個使命去做.
這樣就很不同.
其實就是全部去總動員.
即是使命就是整個教會.
進入全世界去整全福音.
這個就是樂喪所說的.
這樣就真的做到.
我自己也覺得無論你是中產或貧窮人.
都可以參與的.
我覺得今晚很棒.
眨眼過了一個半小時.
剛才我們一直在說.
我們的教會面對經濟的嚴冬.
奉獻或是資源一定會越來越少.
所以我們需要有種準備.
譬如說教務同工減薪.
或者是自籌薪金.
或者是Ten Maker什麼都好.
但是我們的使命就不可以忽略.
無論如何一定要堅持.
甚至乎我們就想著資源少.
所以就做少一點服侍.
不是.
我們需要一種逆向思維.
反過來想.
剛才有張牧師的說法.
就是扭轉了文化.
這個過程裡面包括很多東西.
我們教會觀不可以只維持一種對內的考慮.
就必須有一個向外的使命.

$^{2121}$而領袖就很重要.
堂主任,教務同工,長執,弟兄姊妹,義工.
剛才提到沒有義工就沒有士工.
這些是幫我們想教會成為一個士工.
當然我們今晚討論的方向.
似乎比較集中多一點.
服侍貧窮人或者是基層的朋友.
當然現在香港社會經濟嚴冬.
其實任何人都受打擊.
上至老闆,中層,基層全部都受影響.
所以我們教會就需要想.
我們怎樣全方位每一個都有機會服侍到.
即使我們服侍中產.
服侍他們,他們也能夠服侍其他人.
所以這是一個很美麗的思考.
不過我們就需要堅持.
堅持我們的使命,一定不放手.
時間差不多了.
不過我想問你們三位.
我每人給你們半分鐘.
有沒有最後半分鐘想不說不可以的事.
有沒有?.
來吧,Ellen先說.
其實這個題目,經濟嚴冬.
我自己想在香港的處境裡.
嚴冬又起子經濟.
還有很多現在教會要面對的挑戰都是嚴冬.
但當我面對這個題目的時候.
我都在想嚴冬裡面究竟什麼才能夠生存.
我現在也翻查一下資料去了解一下植物.
怎樣能夠經歷到嚴冬還能夠生存.
普遍來說,它的根能夠有心有狀.
我自己盼望教會在各方面的嚴冬都好.
其實希望信徒,我們的信仰群體.
能夠有心有狀的根在裡面.
一起經過.
多過我們去祈禱.
嚴冬快點過去,快點回復平靜.
快點天下太平.
讓我們的生命能夠有經歷.

$^{2161}$能夠強壯去面對不同的處境.
你那句經濟起子嚴冬.
不是,是起子經濟嚴冬.
我覺得是可圈可點.
不過我更加開心的是.
我們不是冬眠,而是有根要心.
John呢?.
我自己看,不要浪費了疫情.
很多人覺得疫情,像剛才Ellen說的.
快點過去,快點走.
但不是的,這是上帝將它愛.
將它光的時間.
絕對要把握.
所以真的跟隨主.
我們作為主的門徒.
我們很禱告究竟中間上帝要我們做些什麼.
在今天講座,我講到最後.
教會是要deep change.
or slow death.
不是深的改變,就是慢慢死去.
死去動力,死去使命.
死去上帝給我們的大解明,大使命.
死去視野.
如果我們不更新的改變.
我們怎樣去面對未來21世紀.
香港當然起子是疫情的嚴冬.
經濟的嚴冬,社會的問題.
很多問題,我們教會懂得回應.
很厲害,不要浪費了.
Don't waste the pandemic.
很棒,厲害.
蘇老師.
我相信,差遣我們的上帝.
在經濟很順境的時候.
我們教會有很多發展.
但上帝的能力很大.
我們在艱難的時候.
還可以更加好的發展.
或者有一個deep change.
所以我們都要參與在裡面.

$^{2201}$憑信而行.
多謝,多謝.
我今晚很開心.
和你們三位一起在這裡.
說名是講座.
其實我是和你們三位來學習.
很多很有insight.
很多很inspiring的故事.
亦都同心幫我們思考教會的使命.
這個非常好.
各位網上的朋友.
今晚我們的討論.
我體會到今時今日.
我們做一個教會的目者.
或者機構的領袖.
我遇見很多困難.
我們要思考很多不同的問題.
要懂得不單是聖經,神學.
並且要懂得分析整個社會的狀況.
處境文化,研究教會群體等等.
中國神學研究院很想訓練全職.
為教會,為機構訓練全職的侍奉的領袖.
希望能夠做到我們所謂.
做一個有效的反思協作者.
所以就向大家.
藉今晚的機會向大家宣佈一件事.
2021年,即明年9月.
中國神學研究院就開辦.
教務學博士的課程.
D-Mean.
我們希望無論你是.
到學碩士畢業.
或者基督教研究碩士畢業的都好.
你全職侍奉五年或以上.
就可以報讀這個教務學博士的課程.
這個課程很簡單.
六年裡面完成七個科目.
四個必修科.
三個選修科.
另外再加一篇十萬字的論文.

$^{2241}$多不多或少不少.
其實都是一個很好的鍛鍊.
如果大家對這個課程有興趣.
你可以立即瀏覽忠臣的網頁.
我已經在忠臣的網頁裡面.
有關教務學博士的詳細消息.
你喜歡可以開始思考.
和教會商量,然後報名.
我們截止報名在今年五月底.
希望今晚我們所討論的課題.
作為一個小小的題材.
我們如何訓練多些時代工人.
來服侍香港教會和社會.
再一次多謝你們三位今晚的分享.
非常精彩,多謝多謝.
在結束的時候我們請陳廚華牧師.
為我們做一個結束祈禱.
我們一起禱告.
天父我們多謝你.
因為你給我們福氣.
可以服侍你,可以參與你神國的事.
是我們何等美好光榮的福氣.
主要我們知道我們服侍你不是姦身.
是恩典.
我們今天可以走出來.
可以互相緊扣.
無論什麼背景,無論什麼學識.
但我們都是主的兒女.
我們都可以一起各盡其職.
可以建立基督的身體.
讓我們重拾在今天.
在一個不同艱難的時刻.
我們懂得可以怎樣.
在什麼時候我們聽到主的聲音.
我們就按照主的心意.
去彰顯主你的榮美.
帶著主你自己交付我們.
成為你的炎,你的光.
讓我們永不忘記.
主,因為我們先經歷到.

$^{2281}$你大能的拯救.
讓我們就像聖誕節的日子.
將大喜的信息宣揚出去.
這個大喜的信息.
不單止在教會裡敬拜.
這個很重要,我們感謝主.
同樣我們就走出去.
就好像當天天使在牧羊人野地裡.
在外面宣告這麼重要.
耶穌降生,這個美好的信息.
讓這個愛,這個道成肉身的祝福.
藉著我們卑微的身份.
卑微的僕人.
按照主的心意.
我們進到社區中.
讓人見到耶穌是主.
我們感謝你.
讓我們今天晚上.
我們互相分享,互相學習.
讓你繼續引領我們.
聽我們的禱告.
奉耶穌名求.
Amen.
多謝大家,祝大家晚安.
(音樂).
\newpage



\section{}
\label{sec:hfF50g4VbXo}
\textbf{「逆」後餘生心理教育輔導短片系列 - 表達藝術治療(二)}
\newline
\newline
連結: \href{https://youtube.com/watch?v=hfF50g4VbXo}{\texttt{ https://youtube.com/watch?v=hfF50g4VbXo}} ~~~~ 語音日期: 2020-05-28 
\newline
\newline
\hyperref[sec:_SfD9ovSJ1E]{\small{< < < PREV SERMON < < <}}
~
\hyperref[sec:index]{\small{[返主目錄]}}
~
\hyperref[sec:oKCp7lVV9g0]{\small{> > > NEXT SERMON > > >}}
\newline
\newline
$^{1}$(音樂).
不知道大家知不知道.
當我們覺得很緊張,很煩躁的時候.
其實聽音樂是會幫到我們的.
音樂往往會有一些很固定的節奏.
很固定的旋律.
例如是這樣.
(音樂).
我們就是跟著這些節奏,旋律.
可以令到我們的心情平靜下來.
你不懂彈結他?.
不要緊.
接下來Chris會介紹一個表達藝術的方式.
我們在家裡很方便.
就可以透過表達藝術.
能夠平靜我們的心情.
我們將時間交給Chris.
(音樂).
面對不安可以做什麼呢?.
在這一集想跟大家透過一些簡單的藝術創作.
去表達我們的內心感受.
也透過跟這個作品的互動.
希望可以做多一點自我的覺察.
還有當中會不會有一些輕微的心境轉化.
我們都可以透過這個簡單的藝術創作去體驗一下.
可能觀眾聽到說要畫畫.
有什麼反應.
第一個反應可能是.
我不懂畫畫.
或者第二個反應就是.
我畫畫很噁心.
第一個問題的假設就是關於技巧的問題.
第二個反應就是關於審美能力的問題.
回答這個問題我想從一個字開始.
就是我們感受美的能力.
美感.
美感英文是Aesthetics.
翻查字典的字根是Senses.
就好像上一集跟大家分享過.
用五官去感受當下的練習.

$^{41}$當我們有感受.
就會有感而發.
所以創作原理很簡單.
當我們用五官去經驗身邊的人事物的時候.
我們會有感受.
在當中會激發很多的聯想力,創造力.
透過藝術創作去表達.
這個就是有感而發的創作原理.
大家不知道有沒有看過近代的畫家作品.
我其中很喜歡的兩位就是馬提斯和米洛.
他們在馬年的時候的作品都是最受讚賞的作品.
大家如果有機會看一下.
看到的作品其實很像小朋友的一些作品.
在當中揚入了小朋友那份赤子之心.
那份對生命的熱愛,真摯.
和熱切在當中.
所以相對來說技巧就相對次要.
如果大家覺得可以嘗試一下.
我都邀請你可以有一個開放的心.
好奇的心.
願意嘗試的赤子之心.
跟我一起去經驗簡單的藝術創作.
材料很簡單.
一張圖畫紙.
一支鉛筆.
一盒顏色筆就可以了.
開始之前我邀請大家可以合上眼.
有五至十秒鐘我們安靜一下.
感覺一下這一刻.
你身體的感受.
或者你用一個情緒的字.
去描述你這一刻的感受.
身體的感受.
或者一個情緒的字.
然後就打開雙眼.
把感受或者情緒的字寫在紙上.
寫完之後就把紙反轉.
下一步就是我們創作的部分.
我邀請大家用平時不太習慣用的手.
我是用右手拿筆.

$^{81}$所以這次用左手.
以下的環節就用身體去畫畫.
我們閉上雙眼.
用我們的觸感.
感受一下紙的邊界在哪.
然後用我們不太習慣用的手.
帶動畫筆.
在紙上留下一些痕跡.
可以暫時放下腦袋的運動.
用你的手指帶動筆.
在紙上留下一些痕跡.
可以試一下一些直線.
曲線.
或者一些深色的筆觸.
淺色的線條.
或者一些斷軸的線.
讓你的手指帶動畫筆.
在紙上留下一些線條,痕跡.
不需要想畫到甚麼.
最重要是用一個開放好奇的心.
探索一下這枝筆在紙上可以留下甚麼線條.
當你覺得差不多了.
可以停下筆.
打開雙眼.
可以看到紙上的線條.
我邀請各位觀眾用不同的角度.
看看紙上的不同形狀,線條.
或者符號的浮現.
或者不同深淺的顏色,線條的交疊.
有沒有一些形狀或線條特別吸引你.
可以打開顏色筆.
選擇一些你喜歡的顏色.
把一些你覺得吸引你.
或者你覺得很有趣的線條.
打亮出來.
或者可以幫一些形狀填色.
打亮一些你喜歡的顏色.
繼續懷著一份好奇和開放的心.
看看有沒有哪些形狀,線條吸引你.
你就可以用顏色筆勾畫輪廓或填色.

$^{121}$因為時間關係,我做了一張.
這張就是完成的作品.
我想邀請觀眾用手指.
嘗試在這些吸引你的形狀或線條中.
用手指去遊走一會.
嘗試感受一下你的手指.
有哪個形狀,線條.
特別讓手指有一種被支持,保護,承托的感覺.
可以暫時放下腦袋的運作.
相信我們的身體.
讓手指做體驗的部分.
我想邀請觀眾選擇一個.
你覺得特別吸引你的形狀或線條.
讓手指在線條中遊走.
給你一種支持,承托的感覺.
我選擇了這條紅色曲線.
繼續用手指在線條中遊走.
嘗試不同的速度.
可以慢一點或快一點.
讓手指感受一下當中的分別.
找到一個你覺得舒服的速度.
嘗試把動作放大.
讓手指慢慢動.
嘗試用手掌的動作.
嘗試用手臂的動作.
在過程中,有幾個形容詞形容這個動作嗎?.
我會想起一個持續不斷.
的承托力和推動力.
讓手指去經驗持續不斷的承托力和推動力.
嘗試不同的角度.
不同的速度.
讓身體記住這個被承托,被推動的經驗.
直到你覺得身體已經記住.
可以不斷重複.
慢慢把動作停下來.
我們可以閉上眼睛,安靜一會.
沉澱一下剛才的經驗.
可以問自己兩個問題.
第一個問題,在過程中有沒有一些很深刻的地方呢?.
第二個問題,這一刻.

$^{161}$你有沒有察覺到你身體的感受或情緒是怎樣的?.
當你打開雙眼的時候.
我邀請你看看你的紙.
開始的時候寫的字.
我分享一下.
我心跳,因為有鏡頭對著我.
所以我有緊張的情緒.
這是我開始的時候寫下的.
剛才跟大家做藝術創作的經驗.
我感覺心跳得沒那麼快.
有一種比較平和的感覺.
特別給我深刻的經驗.
就是我的手在爬升的時候.
經驗到更多承托力.
壓下來推動的力量.
透過手有這種感覺.
不知道各位觀眾有什麼經驗.
我邀請大家做完這個練習.
可以跟你的朋友分享一下.
或者在Facebook留言.
彼此分享一下你的經驗都可以的.
今天很高興和大家透過藝術創作去面對不安.
希望將來有時間我們再見吧.
拜拜.
參與藝術創作可以將我們的專注.
暫時離開我們覺得很緊張.
很煩躁的情緒或思想.
給我們更大的空間.
可以檢視我們的想法或感受.
當我們很緊張的時候.
往往用說話去表達自己都很困難.
表達藝術和參與創作藝術.
可以將我們由語言的表達.
轉移到視覺的表達.
又給我們更大的空間.
可以平靜我們的心情.
(音樂).
\newpage



\section{}
\label{sec:oKCp7lVV9g0}
\textbf{「關愛受造世界與福音」香港會議 「一同歎息 迎向更新」:答問環節}
\newline
\newline
連結: \href{https://youtube.com/watch?v=oKCp7lVV9g0}{\texttt{ https://youtube.com/watch?v=oKCp7lVV9g0}} ~~~~ 語音日期: 2021-07-12 
\newline
\newline
\hyperref[sec:hfF50g4VbXo]{\small{< < < PREV SERMON < < <}}
~
\hyperref[sec:index]{\small{[返主目錄]}}
~
\hyperref[sec:6_vdz8RQKsg]{\small{> > > NEXT SERMON > > >}}
\newline
\newline
$^{1}$在今晚去到以下的環節.
亦是一個最重要的環節.
就是大家有互動的時間.
我首先請三位講者坐到台前.
我們有一個預備.
在這個對答的時間中間有一支咪.
大家遵守一些原則.
大概每人用一分鐘的時間.
你不用分享太多.
大概點出你想問的問題.
然後如果有人在問問題的時候.
其他有的時候亦可以在中間的時間.
Hello, I am Johnny.
I graduated from this school.
Thank you speakers for all the contribution.
I would like to ask a question for Dr. Moon.
我是這裡的畢業生.
很感謝三位講員.
You mentioned eating is also a thing that we can do.
你剛才提到吃東西也是我們可以做的一個行動.
Dr. Moon, you mentioned that we can eat less meat.
你看到我們可以吃少一點肉.
你是否推動我們要吃素呢?.
Inevitably seen in this consumer world.
unless we stay at home making food from raw material.
So for example, when we go to the market.
除了我們沒有辦法的一定會犯罪.
除非我們是自己在家裡做農夫.
從原始的東西出來.
否則所有東西都是被處理過的.
例如我買東西喝.
全部都是塑膠的瓶裝.
其實都非常困難.
不識喝.
我們這個細小的個人.
如何能夠做到這件事呢?.
世界是沒有辦法避免的.
一定會越來越差.
多謝你的提問.
我想你所說的.

$^{41}$其實我們都是在不同的系統裡面居住.
而這個系統都給予很多限制我們.
我不想我們做的就是.
我們很想去.
如何去.
吃東西是如何特別到.
我們忘記了.
是要慶祝神給我們的食物.
這件事本身.
有一個新的吃東西的方法.
不應該是一個條文主義.
所以我們需要去明白.
其實很多人都有限制.
我很幸運地.
我居住的地方.
有很多地方可以選擇如何去吃.
但就算在我居住的國家裡面.
都不是每個人都可以選擇.
有些人可能要吃很多是.
處理過的食物.
已經是被處理過的食物.
而不是一些有益的食物.
所以我們是需要在整個系統裡面去做.
例如如何看待農作物系統等等.
例如我不知道香港是怎樣.
但很容易地.
我們去問你的食物是從哪裡來.
例如我自己.
我的背景裡面.
我所想的.
因為我的信念.
所以我在家裡真的不多吃肉.
如果我在家裡吃肉的時候.
我會知道那些肉是從哪裡來.
是甚麼農場.
農夫是怎樣去對待土地等等.
我們說肉.
因為我們知道養一隻牛.
其實有很多的資源.
很多的碳排放.

$^{81}$所以其實對我們很多人來說.
去少吃牛肉.
已經是幫助碳排放.
我們吃的時候.
去想有甚麼後果呢.
其實對美國人來說.
可能好像是一些犧牲來講.
其實可能是一個更加.
有益健康的生活方式.
去令到食飯本身是一個.
很感恩.
大家可以一起去感恩那些幫手.
去自有土地.
以致我們可以來到.
去吃得好的農民.
對我來說.
我自己也是去耕作的.
因為在我住的地方裡面.
其實不是真的那麼貴.
去種植東西.
但譬如在我的國家裡面.
如果我種不到的.
我可能去買一些有機的食物.
或者去講一點點.
農場是怎樣的.
如果我們開始去問這些問題的時候.
其實在美國.
現在開始是這樣.
可能在香港.
也可以是這樣.
開始有一些新的.
將美國飲食的文化去改變.
有些人可能說.
其實只是一個潮流.
但其實潮流也不是不好.
你想想我幫的.
就是那些很多在農場上的農夫.
他們其實去能夠.
對他們的土壤更加好.
或者對他們的農夫.

$^{121}$他們的工人更加好.
其實也有一些代價是要付出.
我是一個教母.
我想在吃東西的部分想問一問.
關於食物公義的題目.
剛才我想Dr. Moo有提到.
但是在聖經裡面有說.
其實它不是完全否定.
人可以有一個享受食物的權利.
在洪水之後.
上帝給人吃肉.
好像吃果蔬一樣.
並且在傳道書的時候就說.
人莫強如吃喝.
且在牢籠中享福.
我看,這也是出於神的手.
剛才提到一種新的生活模式的時候.
我就擔心這個新的生活模式.
和一種苦修的感覺有多大的差距.
因為我們不是特別強調一種苦修的時候.
上帝賜福給人.
食物.
甚至其實神創造人.
是用一個以進食為生的生命體的時候.
本身吃任何東西都是在終結其他生物.
或者其他受造物的生命.
來延續我們自己的生命的時候.
這個食物選擇.
或者食物公義.
或者食物的問題.
在我們今天的食記錄圖實踐裡面.
可以怎樣去理解.
第二個問題就現實一點.
因為我作為一個牧者.
剛才關牧師提到一個.
Dr. Moo解釋過的Paradigm Shift的時候.
我們作為牧者.
在一種特別方派教會裡面.
那種很高舉要傳福音.
要門訓的一種context底下.

$^{161}$我們說這些的時候.
會不會被人視為一種.
好像上一世紀的社會福音的感覺.
甚至可能被人稱為妖牧.
諸如此類.
被視為一種不是傳福音的牧者的感覺.
多謝你的提問.
第一個問題就關於食物的方面.
第二個問題就特別是關牧師可能要回答.
看誰好.
他在翻譯.
讓我試一下回答你的問題.
因為我都提出了一種不同的生活形態.
當然聖經裡面真的說.
人可以feasting.
或者人有一種.
就是筵席是一件好事.
不過有時候我們問.
究竟我們今天的生活習慣是怎樣.
我自己比較少經歷這件事.
不過我上一代經常都提醒我.
其實以前是過年之後雞吃的.
今天是很慘還是甚麼呢.
他們是很慘還是.
他們是苦行的生活呢.
我想又未必是這樣的.
我深刻印象就是.
我以前讀Richard Foster的書.
《屬靈通人哩站》.
他說到提及.
其實我們平日如果是吃得簡樸的時間.
反而你每個星期有一天.
有一餐機會吃得豐富一點.
那些才真正是叫做.
明白甚麼叫做feasting.
因為feasting和fasting其實是一對的.
我們的問題就是.
究竟我們今天有沒有fasting呢.
還是我們feasting feasting feasting.
接著就勁feasting.

$^{201}$然後又feasting feasting feasting.
究竟我們是一個甚麼的生活模式.
還是我們其實吃得fasting和feasting.
其實是一對的去看.
在屬靈的生命裡面.
我對於第一題是有少少這樣.
還有如果我趁機會.
要搞清楚社會福音是甚麼.
社會福音除了好像社會行動改變.
很重要的元素就是.
覺得人能夠透過人的進步去將福音帶來.
今天我們說的整全福音不是這樣的概念.
是說福音是整全地彰顯.
不是說人能夠將天國帶進人間.
是一個很重要的分別.
我們要明白其實不是說上一代的社會福音.
我有少少補充.
(主持人:我們請關牧師先跟著再請Dr. 俊良).
說一些不是共識的東西.
或者不是一些主.
大家都是接受的東西.
一定會有些風險的.
但是進步不是由共識開始的.
所有進步.
所有進步都是從偏方開始的.
(主持人:很正經的回應).
我嘗試回答剛才我聽到的一些問題的背景.
(主持人:究竟新的另類的食物方式和苦行的修為有什麼不同呢?).
其實我所提議的是相反的.
我們可以歡慶吃到一些很好的食物.
好像上帝賜給我們的一些禮物一樣.
在路加福音裡面記載了很多耶穌和人一同吃.
在不同的筵席裡.
當我們禱告的時候.
經常感謝上帝賜給我們食物.
還有感謝幫我們預備食物和耕作食物提供食物的人.
事實上在古代的禮儀都會說很多這些東西.
例如我們說主禱物.
或者是初期教父所說的.
如何慶祝主餐.

$^{241}$他們說吃麵餅的時候.
其實是周圍收集墨紙來做麵餅.
就好像眾多的肢體走在一起來處理麵餅.
因為我們現在的年代.
可能在香港就沒有那麼差.
但在美國那些東西是從哪裡來的.
它的來源地是哪裡.
我們完全不知道.
我們只是吃這個東西.
但我們跟地上的其他人完全沒有結連.
所以對我來說.
我覺得是一個慶祝.
需要我們慢一點來慶祝神賜給我們的食物.
(多謝您回應這兩個問題).
(我想我們都值得再繼續思考和探討).
(請其他同工或弟妹).
(如果有需要可以不得客氣).
(大家製造的垃圾).
(其實環保有很多方面).
(但我們每天大量製造垃圾).
(我剛才去食堂吃飯).
(有一成的垃圾).
(我吃完之後).
(收起自己用的膠盤和餐具).
(順便用了五分鐘收起隔壁的餐具).
(原來也不少).
(就有這一包).
(我已經把它們凝固了).
(我不知道大家是怎麼看).
(我很開心今天可以在這裡大家相聚).
(雖然不是太好意思).
(可能會有些味道).
(但如果我們能夠有同一顆心).
(自己在身邊做一點點).
(或者幫助隔壁的做一點點).
(其實已經是一個影響).
(香港的困難是回收率太低).
(教會也沒有什麼推動).
(唯有自己先開始).
(多謝你的分享).

$^{281}$(提供了我們一些具體的行動).
(提問可以繼續).
我想跟著這個展會有一些繼續的問題.
當我們回收不同的膠製品.
或者是瓶裝或者紙.
我們想將這些分類.
不過很多工人或收集的人.
都會將這些混在一起.
所以重點問題是在政府的政制那裡.
所以對於我們來說.
我們怎樣可以解決這個問題呢?.
其實政府不是處理得很好.
在這個議題上.
可能對於我們來說.
我們自己用少一點膠樽.
又或者像這個展會那樣.
幫忙收集人家的膠樽.
或者是那些紙.
各樣回收的物品.
(提問關於環保和我們可以怎樣做).
(我們的講員都可以回應).
(我們準備思考的時候).
(後面有一位光復的弟兄).
(可以先提問問題).
(我看了一些訪問統計).
(幾年前就說).
(在美國的教會).
(主要是白人的教會).
(方塊教會).
(有一個比較高的百分比).
(是會).
(不是很贊成氣候改變).
(而是會有一個比較高的百分比).
(不是很贊成氣候變化).
(他們是會更多的去).
(支持特朗普的預算案).
(大刀去斬一些).
(給氣候變化).
(保護的方案).
(我想知道你對美國的教會).

$^{321}$(有什麼說法).
(謝謝你的問題).
(如果真的想有一個).
(好一點的答案).
(我會嘗試去回答).
(對我來說).
(有兩個原因).
(為什麼有這麼多的).
(美國的風險派 保守派).
(不肯承認有氣候變化).
(就算他們覺得).
(氣候是有變化的).
(他們也覺得教會).
(跟他們沒有關係).
(為什麼).
(所以很多時候).
(我們特朗普那些).
(非常興奮的支持者).
(覺得那些不再支持).
(那些法例的原因是).
(為什麼呢).
(其中一個原因是).
(幾十年前).
(這些保守派的基督徒).
(很多時候去).
(去畏著).
(在美國的共和黨政黨).
(裡面所說的).
(一些的 譬如墮胎).
(或者是同性婚姻等等).
(這些家庭的看法).
(他們對於這些個人性).
(他們覺得重要的議題).
(覺得共和黨).
(是更加接近他們的想法).
(我想我們其實很多時候).
(都會對不同的理想主義).
(我們覺得那些東西是更加的).
(跟聖經是有關的).
(就算是對環境主義).

$^{361}$(都是其中一種主義).
(特別是).
(比較保守派的基督教).
(他們跟共和黨的聯繫).
(是接近到).
(共和黨無論說什麼).
(譬如一些社會主義的).
(就算他們做的一些事).
(是不再支持).
(保護氣候變化的).
(他們覺得都是好的).
(那個時候是).
(我們修例的背後的想法).
(就是我們能夠去處理).
(所有這些問題).
(就是增長 繼續增長).
(例如如果有).
(經濟的增長).
(所以就更加好了).
(如果我們能夠打低中國的).
(人均的生殖率更高).
(我們可能就很多事都做到).
(我們就可以提升).
(很窮的人).
(可以幫到貧窮).
(經濟增長就是我們的辦法).
(我覺得這是第一樣東西).
(第二個原因其實是).
(早些的一個關於).
(一個言論來的).
(就是一個).
(一是就有).
(有靈的人).
(或者是一個無靈的體).
(就好像我們和神的關係).
(和我們人和).
(受造世界的關係之間).
(是沒有關係的).
(神是好像分開的).
(這種的語言論).

$^{401}$(是令到我們有一種).
(完全是無知).
(是不去看氣候變化的事情).
(曾經有一個).
(有一個說話).
(去說當你看到).
(看到天體的東西).
(比所有東西更重要的時候).
(你說的就是這種的語言論).
(你只是想說).
(天上的東西).
(就不用說地上的東西).
(第三樣就是最後說的).
(綜合論).
(就是這個綜合).
(你看到世界是被遺棄的).
(同樣是一個言論的現身).
(一個不再).
(來去接受).
(神的一個更新).
(這件事).
(而是).
(原諒就是).
(既然世界都被毀壞).
(我們現在不用去做什麼).
(第二個問題就更加難答).
(我覺得).
(對我來說).
(就算是).
(特在美國).
(所以就算在現在).
(在美國).
(民主黨).
(在民主黨).
(雖然他們可能比共和黨).
(更加貼近氣候變化).
(等等的事情).
(比共和黨更加上進).
(但對我們來說).
(其實沒有一個的).

$^{441}$(政黨是完全能代表你的).
(因為基督徒其實是).
(不會是).
(對我們的文化來說).
(基督徒永遠都是).
(抗衡著).
(這個世界裡面的文化).
(所以有時候可能是).
(去跟別人聊天).
(讓他們來聽).
(給自己來傾聽).
(已經是一個回到).
(聖經裡面所說).
(我們應該有的).
(文化是怎樣的).
(對我來說).
(在一個這麼基本上).
(看到我的學生).
(發現這些事情的時候).
(他們跟他們的父母).
(想的事情很不一樣).
(對於他們未來想的事情).
(很不一樣).
(我已經覺得很受鼓舞).
(對他們來說).
(能夠去改變這個世界).
(甚至是改變他們父母的想法).
(是一個非常令我).
(雀躍的想法).
(有時候可能離不開).
(關於政治).
(各方面的社會考慮).
(時間關係).
(我想最後一兩個問題).
(問完我們就一次過回應).
(如果有的話).
(有沒有問題?).
(或者最後一個問題).
(請出來).
(最後一個問題).

$^{481}$既然說到政治.
不如我也說一下政治.
如果你們有看報紙的話.
最近有些報紙說香港的教會.
因為有些教會是開門.
讓示威者進去避難.
《人民日報》的紙媒報.
叫什麼名字?.
《環球時報》.
它說美國的勢力滲透到香港的教會.
變成有幕後黑手.
變成很多這些都是一些.
conspiracy theory.
都是政治的.
我覺得他們主要的指責.
就是一個美國的宗教右派.
因為我在外國住.
我接觸過一些比較右派的人.
被他們歧視過.
我領教過他們的東西.
我也不喜歡他們這些.
所以我不斷地在香港.
為什麼我們的教會.
很多地方都太像基要派.
很多東西都很基要.
我一提出來就被人罵.
吵架吵到幾乎拍桌子.
我都想藉此機會.
不要再瓜田李下.
被人說我們不是我們.
總之教會和美國人有關係的時候.
我都很贊成你推動環保.
因為美國的宗教右派是不講環保的.
其實我在教會推動環保.
推動了十幾年.
被人罵了十幾年.
吵架也吵了很多.
被人罵我什麼都不知道.
社會福音什麼的.
我覺得真的很冤枉.

$^{521}$我都藉此機會.
快點不要再瓜田李下.
讓人覺得有基要.
多謝你的分享.
也算是提醒.
最後這個時間看看.
關牧師和黃國偉博士.
和Dr. Jonathan Wolf.
有沒有補充最後的回應等等.
其實在座大部分人.
來得都很明白大家說什麼.
不過我想要有說服力在教會裡面.
其實現在香港的教會生態是很好的.
也是我們的困難.
好的生態其實香港教會是很.
說自己很信聖經.
也都真的很信聖經.
要說聖經的.
不過也都有一種很強的框架去讀那本聖經.
所以我想我鼓勵.
我感覺到大家來到.
其實很有心的在Christian Catholic議題裡面.
我就鼓勵我們能夠做些什麼呢.
就是很有說服力的讀本聖經.
讓人看得到其實福音是什麼.
我們不經過那套徑.
其實沒有人理我們.
真的沒有人理我們.
其實今天Professor Wu提出那種嚴謹的解讀.
他還有書在下面買.
有些書其實是很好的.
幫我們不偏頗地去讀聖經.
去看整傳福音.
看關於受罪世界.
有什麼聖經神學上的支持.
我想這個對我們來講在教會裡面是很重要的.
我覺得是關牧師.
關牧師有沒有.
香港人加油.
關牧師真是.

$^{561}$我想加很多東西.
或者再補充.
今天很高興和大家一起.
我覺得很受鼓舞.
就像剛剛說的.
很多人都已經為身在關牧世界.
本身有很多人已經這麼做.
今天大家不要用一些限制.
或者說一些其他挑戰的不好.
有時候在很多前書.
最尾的部分說.
我們所做人會徒勞的主義.
就算現在的事情.
看上來可能現在.
我們知道上帝的力量多數是在我們老虎的腳裡面.
而不是在桃園裡面.
所以請大家繼續努力.
謝謝你站在這裡.
就是Be Water.
接下來也要Be Water.
我們今天很高興.
由下午到晚上.
特別是Dr. Jonathan Wu給我們一些分享.
給我們一點掌聲.
也謝謝黃國偉博士和關浩然牧師的精彩回應.
謝謝大家來到這個會議.
有三本書推介給大家.
一本是《起初關愛受造世界》.
有很多很好的文章.
特別是入門有很好的讀本.
另一本比較新出版的.
《耶穌的環保學》.
第三本是《關愛受造世界》.
在門口有一個書攤.
大家爭取機會購買書本.
最後時間要預告.
明早我們有六個工作坊.
歡迎大家Walk-in報名.
我們已經預備了.
題目不詳細在這裡講解.

$^{601}$大家上網可以看到.
希望我們最後的時候有一個結束.
將我們整晚的聚會.
我們所得著的交託給上帝.
我們請關浩然牧師.
我們同心的祈禱.
創造天地的主.
我們感謝你的創造.
我們的呼吸.
我們的生活.
全流動作.
一切都是你的恩典.
你的權能的托住.
我們為整個世界.
陷在虛空敗壞當中.
我們求主.
藉著今天聖經的說話.
透過聖靈的根深.
轉化我們.
透過教會的根深.
世界得以轉化.
我們求主.
給我們信心.
我們不是靠自己.
因為我們相信是主你自己的工作.
是你的感動.
臨到這個世界.
臨到香港.
我們為自己去禱告.
因為我們都在經歷那種軟弱.
經歷那種困乏.
但我們相信你是我們的力量.
你是我們的盾牌.
我們就勇敢的.
為你去作供.
我們感謝你今天.
給我們的訊息.
願主在宗教會當中.
將你自己的話語.
將你的恩典.

$^{641}$去作出影響.
願教會走在你的道裡面.
我們同心的祈禱.
奉耶穌的名求.
好,今晚的聚會完滿結束.
大家可以隨意離開.
(字幕由Amara.org社區提供).
(感謝支持Amara.org社區的支持).
\newpage



\section{}
\label{sec:6_vdz8RQKsg}
\textbf{「關愛受造世界與福音」香港會議 「一同歎息 迎向更新」:約拿單 ‧ 穆爾博士}
\newline
\newline
連結: \href{https://youtube.com/watch?v=6-vdz8RQKsg}{\texttt{ https://youtube.com/watch?v=6-vdz8RQKsg}} ~~~~ 語音日期: 2021-05-31 
\newline
\newline
\hyperref[sec:oKCp7lVV9g0]{\small{< < < PREV SERMON < < <}}
~
\hyperref[sec:index]{\small{[返主目錄]}}
~
\hyperref[sec:a0nrOscVmnU]{\small{> > > NEXT SERMON > > >}}
\newline
\newline
$^{1}$主席.
在這兩天的會議中,我們有四個主辦單位.
分別是中國神學研究院,香港教會更新運動.
和新福士工協會,施達基金會.
還有11個協辦單位.
背後我們很高興.
在這次籌備中,其實有很多關心這個議題的弟兄姊妹和機構.
在背後都出了很多力.
今晚我們很高興有Dr. Jonathan Wu在我們當中.
Dr. Jonathan Wu是來自美國威特斯沃斯大學副教授.
他的研究主要是新約早期猶太教,啟示錄和天啟文學等.
科學與信仰,生態及研究環境.
著作包括Creation Care, A Biblical Theology of the World.
Let Creation Rejoice, As Long As the Earth Endures.
Creation, Nature and Hope in 4 Israel.
等書.
所以今晚我們能夠有Dr. Jonathan Wu在我們當中分享.
一銅爛色影響更新是不可思議的.
我們今晚除了講完之外,有兩位會議的講員.
分別是黃國維博士,中國神學研究院副教授.
副教務長及助理教授.
關浩然牧師,中國報道會尖沙咀加拿大堂堂主任.
我們都有即場的傳譯.
大家如果有什麼需要幫忙,點擊QR Code.
就可以用你的耳筒聽到傳譯.
現在的時刻我都是閒話休提.
我們有請Dr. Jonathan Wu.
請給Dr. Jonathan Wu一個熱烈的歡迎.
謝謝你今晚與我們分享.
非常感謝.
在我早前的演講中.
我分享了我感到非常榮幸.
在我今天較早前的演講中.
我分享到我非常感恩可以第一次來到香港.
可以親眼看到一直風雲的一個這麼繁榮的城市.
我住在太平洋的另一端.
在那裡的文化,氣候和生態系統都與這裡很不同.
不過當我在反思我住的地方和香港之間的關係的時候.
我想起一種特別的海龜叫做靈皮海龜.
這種奇妙的生物會在我家鄉華盛頓州沿海的地方.

$^{41}$那些涼水裡覓食.
在那裡游泳的海龜已經游了幾千英里.
並且很有可能在印度尼西亞的熱帶海灘上被孵化出來.
而根據GPS追蹤的資料顯示.
其實牠們很可能在香港沿海的水域中游過了好一段日子.
這些長度不屑越洋的生物只是其中一個例子.
讓我們重新發現地球上所有生命是怎樣互相影響.
和交織在一起的.
而可能我們比任何人在任何年代都更加留意到這些相互影響的關係.
不過在座每一位最重要的關聯是.
我們都是在基督裡面的弟兄姊妹.
我們都在這裡彼此聯繫上.
並且在神的家裡一起彼此合一聯繫.
我希望今天我們可以透過研讀我們共同有的經文.
反思地球這個我們共同的家.
一起敬拜上帝.
我在華盛頓州的韋特沃斯大學教書.
有幸我可以教到聖經科 希臘文和環境學.
我自己也有野生生物學的背景.
所以我有幸教授有關上帝的兩本書.
一是上帝在聖經裡面的啟示.
二是上帝在自然界的啟示.
教授環境學是一個很好的藉口.
讓我可以和學生一起在戶外度過很多的時間.
幾年前我為韋特沃斯大學環境學中心.
編寫了一本自然歷史指南.
這個地方有300公頃大.
位於大學北部小山丘的一條河流之上.
不過由於我完全沒有藝術天份.
我就請了一位韋特沃斯大學的畢業生.
當畫家來為這本指南做插圖.
他畫的畫會在我稍後的投影片裡面的背景出現.
當我和他合作的時候.
我感到很驚訝.
因為我留意到不同的人.
可以帶給世界有不同的看法和看見.
例如你在這個投影片上見到的插圖.
是顯示著洛基山峰的景觀.
畫家要做的其實只是畫洛基山峰的景觀.
不過當然他也可以決定用特定的方法去構圖.

$^{81}$即是將某些事物包括在這幅景觀裡面.
也剔除了某些事物.
因此在這幅插圖面前出現了兩種特別的植物.
是瘦豬梅和茅茴.
順帶一提這些是很美味的野會.
歡迎你來這裡的時候我會和你分享.
在我眼前這個是一個很美麗和和諧的景觀.
反映了上帝創造的美好.
不過有些人也可能對這個景觀有不同的看法.
例如我們可能看到這個景觀而想起適者生存.
又例如我有一個鄰居.
曾經他是一個伐木工人.
他會看到採伐這裡的樹木有什麼價值.
並且他很希望可以拿著那輛全地形車和雪地摩托.
可以在這個地方穿梭.
另外有一個環保主義的朋友.
可能也會留意到不同的東西.
我們在這裡看到的是一個森林曾經被砍伐過的結果.
背景那裡有很多枯樹.
雖然是伐木公司留下的一些落葉松樹.
他們也是被很多其他樹木都砍伐了的時候.
令到乾熱而死亡.
這些砍伐的痕跡.
令到我們知道美國太平洋西北地區.
有多少古代的森林已經被砍伐.
被消失的森林也意味著.
在最近十年裡面曾經吃過這些老樹上的地衣生物.
例如好像那些林地 純綠一樣已經消失了.
我們可能會看到這個景象而感到其實都很悲哀.
重點很明顯.
我們其實有很多不同的向度去設想我們在自然界當中的位置.
我們對世界的看法也都影響我們怎樣生活在其中.
對於基督徒來說 聖經給我們一種怎樣看待這個世界的方法或眼光.
這種眼光不是用來回答所有問題的.
不過聖經呼召我們用特別的眼光來看待受造世界.
並且呼召我們採取一種特別的生活方式.
所以我們今晚要做的事是看看聖經對受造世界的看法.
我們也會花一些時間去了解科學怎樣啟示創造的情況.
了解我們盼望的本質.
然後思考我們對今天的一些實際意義.

$^{121}$第一樣我們在聖經裡面學到的東西.
亦都可能是最重要的東西就是這個受造世界是美好的.
在創世紀第一章裡面 神七次稱呼創造為好.
不過要留意的是其中有六次是在人類未出現之前.
其他生命 其他生物在上帝面前都有自身的價值和美善.
甚至在第六天結束的時候.
受造世界沒有僅僅因為人類而變得非常好.
之所以非常好 是因為創造工作已經完成了.
上帝見到他創造的一切所有 然後覺得非常好.
其他生物的價值不僅僅在於他們能夠為人類提供什麼.
這是基督教的世界觀對大多數當代環境主義者的挑戰之一.
即是除了個人的喜好和功利價值以外.
環境主義並沒有任何理由支持要珍惜其他生命.
至於其他生命的價值 其實聖經給我們一個相當激進的看法.
這個價值是我們不能夠簡單直接賦予或者剝奪的.
這個價值是我們應當認知自己是被呼召來尊榮神的人.
這個主題在聖經其他地方也有呼應.
不過沒有詩篇第104篇更加優美.
儘管我們在這裡只是節錄一個部分.
我仍然鼓勵你花點時間回去看看這篇詩篇.
以及延伸對上帝奇妙創造的熱切重讚.
很多時候我們只是在創世記建立我們的創造神學.
不過其實聖經其他部分也有形容被造的世界.
詩篇是一個很好的起點.
詩篇第104篇讚美上帝創造的多樣性和美好.
請聽以下幾節經文.
[節節文].
從聖經其他的部分我們知道.
黎巴嫩的香柏樹可以用來建造所羅門的聖殿.
但是在這裡我們了解到他們不是僅僅為人類服務.
黎巴嫩的香柏樹也是鳥類的家園.
松樹也是一樣.
這些其他生物住的地方.
同樣高山為野山羊的居所.
岩石為石棺的長生處.
創造充滿了其他生物的家園.
不單單是屬於人類的.
在這首詩篇裡.
人類是創造的節奏.
晝夜間在循環當中確實找到自己的位置.

$^{161}$其他生物也是這樣.
甚至於作為獵物咆哮的獅子也是這樣.
對於獅人來說.
獅子必定是對生命和財產有極大的威脅.
然而獅人也對獅子是上帝喜悅的動物.
是上帝所在乎的動物.
就好像上帝在乎我們一樣.
若白面對著充滿野獸的創造世界.
這個世界充滿了.
好像獅子和力威亞森般一樣可怕的生物.
但卻發現上帝很喜悅他們.
若白看到一個旨意給他和他的朋友.
想像更加偉大的上帝.
一個榮耀比若白所理解更加大的上帝.
若白在荒野世界之中辨別到.
一切不是為他而存在.
而是為上帝而存在.
在詩篇104篇之中.
力威亞森象徵著狂野和混亂.
甚至威脅著已經安定的人類生活和文明.
卻只是上帝在創造海洋中的稀稀生物之一.
若白面臨著力威亞森.
還面臨著和力威亞森相約的陸地生物巨獸.
不過這個巨獸的描述.
上帝告訴若白.
我創造了巨獸也創造了你.
其實它是一個同伴生物.
我們很難理解.
如此狂野和可怕的事物.
為什麼和我們一樣.
都是上帝美好的創造的一部分呢.
但上帝告訴若白.
天下萬物都屬於我.
有時我們基督徒被指控.
我們相信世界存在唯一的意義.
就是給人類去使用甚至利用.
不過其實聖經沒有這種想法.
聖經沒有任何地方說上帝將大地賜給人類.
是可以讓他們隨意使用的.
受造世界不是只是為我們而存在.

$^{201}$而是為了神的榮耀而存在.
在詩篇24篇第一節這樣說.
地和其中所充滿的.
世界和住在其中的都屬耶和華.
希伯留書二章十節這樣說.
上帝那為萬物所屬那萬物所本的.
啟示錄第四章十一節.
因為你創造了萬物.
萬物因你的旨意而被錯.
做而存在.
哥羅西書一章十六節.
因為萬有都市在他裡面做的.
一切都是藉著他.
為著他而做的.
由於受造世界是屬於神的.
所以他也啟示了神的某一些屬性.
創造者的屬性.
CP19篇說.
諸天述說上帝的榮耀.
保羅在羅馬書第一章裡.
認為受造之物.
揭示了上帝是誰.
神永恆的神性的本質.
而這個啟示是受造之物可以觸及的.
在《使徒行傳》十四章.
保羅在路斯德宣講福音的時候說.
我們傳福音給你們.
是要你們歸向創造天地海和其中萬物的永生神上帝.
保羅說的意思是.
上帝沒有讓自己在沒有見證人和沒有見證的情況下離開.
在保羅和巴拿巴出現之前.
上帝就已經向路斯德的人做了見證.
那上帝是怎麼做這個見證呢?.
就是透過充足的雨水和豐收的季節.
向他們展示美善.
令他們衣食豐足.
使他們心裡充滿歡樂.
保羅說上帝的創造乃至他們內心的喜悅.
都是對永生神的見證.
當我們的行為導致上帝的受造這本書的書頁.

$^{241}$被撕爛撕下來的時候.
其實我們是不是應該為此而悲哀感嘆呢?.
上帝希望他的榮耀是透過被造的美麗.
而讓世界更多人知道.
我們應該好好照顧這個世界.
努力保存它對永生神的見證.
不過我們人類又在這個美好的受造世界之中.
有什麼位置呢?.
人類被造的一天不是獨享的.
我們在第六天和陸地動物一起被造.
因為我們屬於的類別就是陸地生物.
創造的故事的頂峰不是在第六天我們出現.
而是第七天休息的時候.
上帝坐在整個受造世界之上.
在《創世紀》第二章那裡.
人類由塵土而造.
在亞當就是希伯來文裡面.
土地這個字義.
在英文裡面也是這樣說.
我們可以說人類是由土而造出來的.
我不知道粵語裡面有沒有表達到.
這個希伯來文的文字遊戲.
根據《創世紀》第二章.
我們是泥土生物.
在英文版《聖經》King James裡面.
說我們是一個活著有靈魂的生物.
不過在《創世紀》的背景之下.
這個實際上和我們.
實際上並沒有令人與眾不同.
事實證明其實所有生物都有生命氣息.
上帝的靈給了他們生命和給他們存在.
因此他們都以相同的方式被描述.
他們是有靈的活物.
生養眾多的祝福.
不單單賜給人類.
還有賜給其他的受造之物.
在《創世紀》第一章的序詩裡面.
魚,鳥類都有生養眾多的祝福.
洪水過後所有生物都再次被祝福.
在上帝的創造裡面也有空間去結果.

$^{281}$現在我們來看看.
在《新約聖經》裡面.
有時候說幫助我們明白.
發現自己人有多重要.
不過請聽一下彼得如何引用賽亞書的說話.
凡血肉之軀的盡都如草.
他的一切榮美像草上的花.
花必枯乾草必枯乾花必凋謝.
雅各又說你們的生命是什麼呢.
原是一片雲霧出現片刻就不見了.
不過當然這不是故事的全部.
如果我們再看舊約體詩篇103篇.
我們再次被提醒到.
其實我們就如地上的塵土.
我們就如草.
今天在這裡明天就沒有了.
但是在這兩節經文前後都提醒我們.
我們的價值是來自上帝選擇創造我們.
上帝選擇與我們建立關係.
上帝憐憫我們.
就如父親憐憫他的兒女一樣.
從僅古到永遠.
主的慈愛與我們同在.
因此我們本質上沒有什麼.
可以令我們自身的價值.
又或者變得更有價值.
但是我們的身份.
我們的價值.
甚至我們的尊嚴.
其實都是上帝恩典的禮物.
這是基於我們和上帝的結連.
發現我們真正的身份.
並且在我們自理萬物的過程裡.
反映上帝形象的意思.
人之所以有尊嚴.
是因為上帝選擇宣告人是按神的形象而做的.
而不是因為我們自身的能力或者才能.
在詩篇第八篇裡.
我們發現斯人在晚上.
在看著星空的時候.

$^{321}$他觀看這個廣闊的星空.
他很驚歎地說人算什麼.
有時可能我們都會這樣想.
在我們自己一個人的時間裡.
我們才意識到其實這個宇宙有多廣闊.
驚歎人類是顯然渺小.
但是我懷疑我們的問題.
和古時近東的人不太一樣.
他們抬頭仰望的行星.
一定多過我們現在看到的很多.
他們感到人類生活是多麼的謙卑.
微不足道和脆弱.
斯人體會到在遼闊的宇宙裡.
似乎我們並不是那麼重要.
所以說人算什麼.
而且斯人也知道.
上帝呼召我們在創造裡.
扮演極其崇高的角色.
上帝創造了我們來統治其他受造物.
詩篇第八篇也說.
你派他管理你手所做的.
使萬物都服在他的腳下.
所以這個規則並不是獨立於上帝的規則.
其實整個詩篇的開頭和結尾.
都算應對上帝主權.
以這個為框架.
耶和華,我的主,我們的主.
你的名在全地,何其名.
因此在上帝的統治之下.
我們的統治才能找到自己的位置.
而必須仰望上帝才能知道如何統治.
詩篇的作者其實在呼應創世紀.
表達即使人類是其中一種受造物.
但是他們按照上帝的形象.
獨特地被創造的生物.
我們與上帝聯繫的方式和其他受造物不同.
並且在創造中被賦予獨特的角色.
其他受造之物.
只要造神創造他們的樣式.
就是榮耀神.

$^{361}$就好像很多詩篇中形容受造物對上帝的讚美一樣.
我們被呼召以信服和忠心.
來加入這個宇宙讚美大合唱.
而且在受造的合唱團中擔當特別的角色.
我們的角色其實很活躍.
他包括去治理大地.
和有智慧地管理其他受造物.
去耕作管理全地.
創世紀第二章幫助我們了解我們的角色.
這個看起來好像一個祭司的角色.
去耕種和看管伊甸園.
這些字眼是形容祭司在殿內的工作職務.
我們透過耕種,看守,保護大地來侍奉上帝.
正如上帝看顧我們.
就像大祭司的祝禱一樣.
願耶和華賜福與你保護你.
在文數記錄將二十四節這樣說.
因此我們要保護大地.
讓我們用羅亞來做一個例子.
羅亞的召命是保存生命.
這個就是羅亞的角色.
故事提到要保存生命.
而且這個章節不停重複了很多次.
不過如果你知道這個召命.
是在上帝向世界發出災難性洪水的情況下.
你會覺得羅亞這個召命很有趣.
上帝可以選擇他想要的方法來拯救世界.
不過他是選擇使用羅亞.
用人的才智和技術來建立這個方舟.
以至其他的生命可以在洪水中得以保存.
無論這片大地遭遇毀滅的原因是什麼.
上帝呼召羅亞.
上帝呼召他的子民.
都是為了使其他受造之物能夠存活.
如果我們有更多時間.
我們可以更詳細研究聖經中以色列人的生活.
去了解土地本身和以色列人的生活是如何與神連結.
他們對上帝的忠心或不忠.
其實反映在大地本身的情況之中.
在以色列的律法中.

$^{401}$其實有一些律法是用來確保可以持續修成和保護動物.
而以色列的王應該是更能夠體現到.
代表其他人的統治是如何.
不過最重要的是.
我們從耶穌基督這個人.
參透到什麼是有著神的形象的真正意義.
希伯來書的作者都引用了詩篇第八篇.
啟示了耶穌是神的真實形象.
和向我們展示神是誰.
耶穌的統治是如何呢.
從他在地上的事公看來.
他代表他所統治的人在十字架上死了.
這個耶穌為他人獻出自我犧牲的愛.
眾所周知的是.
其實我們並沒有按照上帝的旨意生活.
好像《創世紀》第三章所形容的.
我們拒絕了上帝對我們的旨意.
並且離棄了創造我們那一位主.
我們拋棄唯一能賜予我們生命愛存在意義的神之後.
我們陷入了死亡和黑暗.
我們和上帝的關係都破裂了.
我們彼此之間以及和大地之間的關係都破裂了.
我們和上帝的關係破裂對整個宇宙萬物都有影響.
我們是有著上帝形象的人.
是要看管受造之物.
但是現在被造之物遭到破碎.
有罪的受造物陷入了徒勞和空虛.
值得留意的地方是.
《創世紀》第三四章強調土地不結果子的後果.
現在工作不是常常可以達到預期的結果.
我們的勞碌也不是一定能夠達到預期的目標.
有趣的是《創世紀》之中的就坐其實也有一個偶然性.
就坐的後果可能會根據人對上帝的服從和忠心而有所不同.
因此該人殺了他的弟兄之後.
必須以一種特殊的方式再次向他發出就坐.
該人發現當他耕作的時候地不再為他效力.
相反在第五章《創世紀》中記載到信服的羅亞.
反而可以舒緩這個就坐.
就坐並不只是一種被接受的東西.
其實它的影響很大程度取決於人與上帝的關係.

$^{441}$人與人彼此的關係以及與其餘受造物之間的關係.
我們與土地的關係反映了我們與上帝的關係.
這個觀念貫穿整個舊約.
對於先知來說江汗和土地被毀.
是以色列對上帝不忠的徵兆.
這表明上帝的子民和地之間的盟約已經被毀了.
大地陷入了人類罪惡叛亂帶來的影響.
它受到人類的污染並且因為我們的罪而變成苦.
根據《河西亞書》第四章說.
因此這地悲哀其中的居民也都消滅.
《以賽亞書》第四章說地被其中的居民所污穢.
因為他們犯了律法廢了律例背了永約.
他們違反了律法.
破壞了永久的永恆的永恆.
這是我們在《新教堂》中提到的主題.
如果您是在這裡的話.
我們會在《聖經》第八章提到的.
這是《使徒保羅》在《摩馬書》第八章提出的新一日主題.
受造之物切忘等候神的眾志續現出來.
因為受造之物伏在虛空之下.
不是自己願意乃是因那叫得如此的.
但受造之物仍然指望脫離霸王的克制.
得享神兒女自由的榮耀.
我們知道一切受造之物一同嘆息.
勞苦直到如今.
我們等會會再看這段經文.
怎麼討論基督教的盼望.
但現在我希望我們去看看.
他怎麼說受造之物嘆息這件事.
保羅首先說.
他說因為受造之物屈服在虛空之下.
但誰令受造之物屈服.
唯一能夠這樣做.
仍然令受造之物有指望的是神.
神令受造之物屈服在虛空之下又是什麼意思.
有些人認為這段經文是在說.
《創世紀》第三章的就座.
但這個說法有一個問題.
就是在《創世紀》三章沒有顯示.
神以指望為基礎來就座受造之物.

$^{481}$所以用比較好的理解.
保羅描述神將受造之物屈服於人之下.
保羅已經在《羅莫書》第一章告訴讀者.
人類因為敬拜受造之物.
而不是敬拜我們的創造者.
在思想上變得虛空而逃離.
事實上兩段經文所說的虛空.
保羅都是用同一個希伍利文的字根.
他認為人類是越來越虛空.
所以受造之物現在屈服在虛空之下.
他永遠都沒有辦法達到神原本給他的目的.
從整個聖經由《詩篇》到《啟示錄》.
我們都知道所有受造之物.
即使如今也繼續讚美和光榮耀給神.
但他去不到終點.
就是神所定義的終極目的.
因為他屈服於有罪,虛空和徒勞的活物之下.
因此受造之物在敗壞的隔制下嘆息.
學問得到釋放.
我認為這段經文挑戰我們去檢視自己的處境.
去思想受造之物如何被敗壞所克制.
在人的罪之下嘆息.
所以現在我想離開一陣子的經文.
去看看今天我們地球的狀況.
我們根據聖經向我們揭示的.
關於這個美好而美麗.
但同樣在嘆息之中的受造之物.
來去思想我們周圍的世界.
首先我要大膽的宣稱.
我們的時代有一些獨特之處.
馬丁里斯是皇家學會的會長.
皇家學會是英國最重要的出盡自然科學發展的組織.
馬丁里斯和很多的科學家一樣這樣說.
他說無論從宇宙或地質的時期來看.
我們這個世紀仍然有些獨特之處.
為什麼這麼多科學家會這樣說.
正是由於人類對於地球所有系統的功能.
產生了前所未有的深遠影響.
就在保羅的時代我們知道了很多地區性毀壞的故事.
通常都是由於羅馬推動經濟發展.

$^{521}$和經營殖民帝國的方式.
我們都知道這些更早期的文明的國家.
由於環境的退化或不能好好管理自己的資源而瓦解.
但我們的時代在我們共同對受到世界所產生的影響.
無論是規模和嚴重性都是獨一無二.
很多的科學家都形容我們進入了一個新的地質時期.
叫做人類世 即是人類的時代.
我們的時代與眾不同的其中一個原因.
是世界人口的直情指數增長.
我們必須要立刻去說.
每一個人無例外都是按照神的形象去做.
是應該值得被慶祝 被愛.
但同樣我們都必須要考慮.
就快有80億人口共享地球是怎麼一回事.
在耶穌時代地球上可能有2.5億人.
1000年後已經達到約3億.
直到1800年工業革命開始之後.
我們才達到第10億.
到1900年這個數字是15億.
而現在我們的人口約為77億.
所有這些人都需要食物 住所.
並且應該擁有良好工作 富有成果.
和能夠有尊嚴的生活的機會.
我們必須要承認這個不可避免.
會對其餘的受造之物產生深遠的影響.
作為基督徒我們確認這種影響既可以是良性.
亦可以是消極.
因為我們認識到人類文明和文化的美好之處.
以及我們有著作為受造之物.
反映神的形象的角色.
但正如保羅和舊約才提醒我們.
我們的影響亦可能具有深遠的破壞性.
所以據估計.
全球地球上武兵的陸地表面.
有85%已經受到人類的直接影響.
當然我們考慮間接的影響也差不多是100%.
人類現在是一股如此重要的力量.
以至我們所能移動的土石.
比所有的天然侵蝕力量加起來更加多.
我們共同所用的地球上的植物.

$^{561}$大約是三分之一的生產力.
人類所造成的孤單量.
比所有的天然資源加起來還要多兩倍.
在過去25億年裡.
我們對氮貯滿的影響是最大的.
當然我們從空氣中能夠固氮的能力.
是養活了我們世界上很重要的一件禮物.
氮氣漸漸大約是空氣的八成.
通常沒有固定氮細菌的植物是用不到它.
如果植物有足夠的水分.
氮通常是限制它的養分.
能夠從肥料裡固定空氣中的氮.
我們開發新的農作物品種.
就預備在最近幾十年裡.
一直有足夠的食物來養活世界.
問題是它的分布不均勻並且不公義.
所以很多人都叫餓.
但我們是不應該錯過.
神賦予我們的科學和科技恩賜.
問題是我們是否使用得當.
因為當這些恩賜被濫用.
錯誤的使用或者誤用的時候.
它們可能對地球能夠持續繁榮.
和健康的運轉造成很大的負擔.
因此肥料的濫用和工業化.
農業體系種種的問題.
經常忽視土地的長期健康.
我們有一個情況就是.
我們的表層土出了問題.
表層土是養活我們所有人的很偉大的元素.
但我們正在以驚人的速度失去它.
一厘米的表層土形成.
需要四百年那麼久.
但在全球我們損失表層土的速度.
是它能夠形成的速度.
是十至四十倍那麼多.
而我們對食物,農場,牧場.
以及我們自己的房屋的需求.
意味著我們繼續取代自然棲息地.
和其他活物的自如處.

$^{601}$例如世界上七成的草原.
已經被人造的景觀來代替.
而且就好像最近幾個月的新聞.
樹林都正在消失.
在熱帶地區.
無論是在東南亞還是在南美.
在生物多樣性最高的地方.
森林仍然以驚人的速度.
被人砍伐和燃燒.
去年在2018年.
我們失去了一千萬公頃的樹.
但2018年已經是近年來損失比較少的一年.
例如在2016年.
我們損失了近三千萬英畝空傾的樹木.
今年亞馬遜洪洪大火.
看來都是另一個惡劣的年度.
我們的海洋在很多地方.
繼續被過度保留.
世界上將近九成的漁業.
是已經被過度開發.
或者快要這樣了.
如果我們不再停一停這些漁業.
有很多正處於瓦解的軌道.
我們向大氣中排放的二氧化碳.
其中一個問題.
就是吸收了大部分二氧化碳的海洋.
變得越來越酸性.
這就是導致世界上很多珊瑚礁死亡的原因.
並且在某些情況下頻臨絕種.
所有這些都影響到其他的生命.
影響到我們被要求管治和照顧的所有受眾物.
我們不是在保護他們.
而是在使他們絕種.
據估計作為一種物種.
我們是在使其他的受眾物活絕種.
這個絕種的速度是天然背景率的近一千倍.
與近四十年前相比.
我們現在發現自己處於一個被急劇減少的世界.
這個世界的野生動物數量比以前少了六成.
我對這件事感到極其驚訝和深深的悲傷.

$^{641}$在我們的日常生活中.
這不是很多人會注意的東西.
因為我們經常都在忙其他東西.
或者我們好像沒有受影響.
而作為神的子民.
就好像那些上司愛我們的靈社.
愛神並且關心神創造的人.
我們不應該很安心地不理會地球上所有生命.
還在持續減少.
目前各種生物的豐富和多樣性下降.
主要是由於棲息地的喪失.
和對其他動物的直接過渡的開發所致.
但氣候變化都是一個重要的因素.
而它的影響只會在下個世紀越來越強.
我想我們都知道.
我們生活在一個地球變暖的時代.
最近四年是有紀錄以來最熱的一年.
今年都不會例外.
地球比工業化前時代的溫度高大概一度左右.
聽起來好像不是很多.
但是我們在說的是全球平均數字.
而在這樣的情況下.
這是一個重大和戲劇性的變化.
確實在人類的文明興起和繁榮的整個過程中.
變化的不只是溫度.
而且值得觀察的是就算是冰河時代.
與我們現在的時代之間.
溫差是大概四度左右.
在上一個冰河時代的高峰期.
海平面比今天低了350英尺.
當時地球三分之一被冰覆蓋.
因此當我們說全球的平均水平時.
只是幾度已經很不同.
除了人類的行為外.
已經沒有任何其他合理的原因.
解釋地球為什麼會變暖.
當科學家研究我們所知道的.
影響全球氣候的所有因素.
並且根據上一個半世紀以來.
溫度的實際變化來製圖的時候.

$^{681}$他們發現如果排除任何因素.
模型就不對數據了.
你可以在右邊的圖表看到.
唯有當計算人為造成的溫室氣體排放時.
實際發生的事情和物理學的預測才會配對.
而這個人為因素.
中國和美國對這個人為因素的貢獻.
就超過任何其他國家了.
中國現在是全球排放總量最大的國家.
而美國是至今為止人均排放最大的國家.
而且由於我們的社會.
從這些化石燃料的燃燒中受益.
作為基督徒我們有最大的責任.
來解決和減輕.
因著我們的生活方式.
對全球的淪陷和地球上所有生命所產生的害處.
現在我們知道地球歷史上的氣候一直在變化.
曾經比現在溫暖得多.
也曾經比現在寒冷得多.
那今天有什麼不同呢.
正如我們剛剛看到的.
其中一樣不同的就是.
我們現在是推動這些變化的人.
第二件事就是.
氣候變化的速度是人類歷史上從未有過的.
只是這種變化系統.
令到生態系統和人類社會難以適應.
特別我們已經是生活在一個破碎.
並且一直在減少人的減產的受災世界裡面.
其實已經是面對各種的挑戰.
第三個今天的氣候變化不同.
就是它對人類生活的影響.
海平面上升.
令到很多地方沒有辦法居住.
冰川的喪失.
意味著很多地方失去了可以利用的淡水.
極端的氣候事件將會越來越普遍.
更極端更持久的熱量.
大氣變暖.
意味著水文循環的加劇.

$^{721}$這個意味著更加強烈更加持久的水災事件.
但諷刺的是.
也都意味著更嚴重更持久的乾旱.
所有這些的變化都挑戰我們.
種植足夠糧食以養活世界的能力.
氣候變化影響的複雜性.
也都意味著更多的難民.
對基督徒來講我建議的.
除了是努力減輕氣候變化.
並且幫助其他人適應不斷變化的世界的時候.
尤其是最貧窮和最脆弱的人.
我們必須更加準備好歡迎難民和外來者.
很多人的家園可能因為氣候變化而沒有辦法居住.
而且因著氣候變化和政治貧困和現有不平等現象.
有種種的互動.
加劇了不公義和苦難.
但讓我們以盼望作結束.
因為這個就是聖經帶我們去的地方.
對整個灘塞的受災世界都仍然有指望.
我有很多不是基督徒的朋友.
他們都知道我們剛剛看過的東西.
他們充滿了深深的焦慮和絕望.
我想我們要樹立榜樣.
並且邀請這些人加入福音中所給我們的盼望.
這是神的意象.
祂一直通過聖靈和我們一同在基督裡面.
這位神的目的是復活和更新整個在灘塞的受災世界.
我們的希望是植根於耶穌基督.
祂透過道成肉身來重新這個世界的美好.
祂就由祂的肉身.
將自己和整個受災世界聯繫在一起.
由此進入了這個世界的康莊血腥.
塵土飛揚,汗流瓦背的生活.
祂通過道成肉身來祝福受災物.
並且表明祂是適合神啟示自己的工具.
而在耶穌的復活裡面.
祂指出受災之物能夠體現生命的盼望.
不是逃避世界而是要更新.
保羅在《哥羅西書》上說.
耶穌是一切與祂自己和好.

$^{761}$由於基督的死和復活.
每一件事都得以恢復並且得著更新.
在《啟示錄》記載有關受災物的更新.
我們看到神不是就這樣重新做那批所謂的新事物.
相反承擔受災的神自力將一切都更新.
這又令我們回到《羅馬書》第八章.
保羅很明確地指出.
現在嘆息和渴望得釋放的受災之物.
在神的旨意中是有未來的.
祂是不會被迷惘的.
我們知道將會有改變,有轉變.
在更新創造中有些東西是連續.
有些東西不再連續.
我們只能想像這種轉變.
耶穌的身體復活是我們唯一對這種轉變和更新的受災之物.
可能會是怎樣的唯一的標記.
但當這個世界的角度成為神和祂離開的角度.
必然會有些東西不再連續的方面.
但無論這些不再連續的方面是怎樣.
被恢復過來和更新的受災之物都是同樣的受災之物.
亦都是同一個受災之物如今在嘆息.
但將會在神的耳裡的啟示中得到釋放.
其實你可能會留意到這是神做的工作.
我們不是這個星球的救星.
但我們再次思考羅馬書第八章.
受災之物等待什麼.
等候的是得著兒子的名分.
我們可能會認為這是在未來才會發生.
但保羅在前面的經文中剛剛提醒了我們.
當我們在基督裡.
當我們被神的靈引導.
我們已經是神的兒子.
所以我們建議我們面臨的挑戰.
就是現在開始這樣的生活.
現在開始活得像神的孩子一樣.
我們要按照神在基督裡更新創造我們的樣式而活.
就好像保羅在哥倫比亞剛才所說.
若有人在基督裡他就是新造的人.
我們自己是不會帶到國度的降臨.
但我們作為基督徒和在基督裡的群體.

$^{801}$作為教會被呼召在當下展示和顯示更新的受災物生命.
這個必須是包括地球復和.
看來會是怎樣呢.
對我們來說.
他透過關愛我們所有人.
靠他身處的受災世界來到愛倫是怎樣.
對我們來說.
透過照顧神所創造和所愛的受災物.
令愛神是怎樣.
首先世界上已經有很多人和組織.
可以幫我們採取什麼行動.
並幫我們做得更加好.
我認識的一些最好的基督教機構.
有份贊助這次的會議.
我都鼓勵你參與其中.
如果我們要神呼著我們成為子民這樣生活.
我們每個人都需要在自己的生活上.
反映神的國度原則.
但我們面臨的挑戰規模.
也意味著集體的行動是必須的.
所以我們和其他人一起完成這項任務是很重要的.
我和我爸爸所寫的關愛受災世界裡面.
我提出了我們所有人在這個旅程開始的時候.
可能用不同的方法.
我用英文首字母縮寫詞awake.
表明我們所有人的生命.
都需要好像仕途保羅那樣.
我們完全的清醒.
不幸的是我不懂得說廣東話.
所以我沒有一種巧妙的方法.
來組織這些想法.
或者令你覺得他難忘.
但以下就是這五個方式.
首先專注力.
讓聖經對神創造的世界的儀仗成為一個邀請.
讓我們在周圍自然世界看到.
體驗神的榮耀.
無論是在礦野農場 花園 城市公園.
還是在遍佈在我們心目中.
甚至在香港這樣的市中心.

$^{841}$讓我們了解自己的位置.
並且在敬拜神的時候.
承認神創造的恩賜.
讓我們也注意種種的挑戰.
讓我們不要忽視地球上的生命.
都要真實的面對的問題.
不單在世界各地.
也在我們每個人的家庭裡面.
第二就是走路.
我選擇了步行這個詞的.
就是一般的交通.
還有因為我在美國寫這本書.
看來好像很多的美國人.
都覺得有必要開車到處去.
駕駛和飛行.
當然對碳排放有深遠的影響.
所以我希望我們可以更多的反思.
這些活動所產生的影響.
或者我們應該更多的選擇.
放慢速度.
關注我們周圍的世界.
更多地區性的生活.
我想好像在香港這樣的地方.
實行的可能性可能比在我住的.
華盛頓州的斯波坎.
這個廣闊的地方更加高.
當我飛行的時候.
我們飛行就好像我來到這裡要做一樣.
我們就要承認這件事的真正代價.
你可以給錢.
給例如 Climate Stewards.
這個很好的基督教機構.
去抵銷碳排放.
這個不是逃避責任.
而是承認責任的其中一個方式.
第三就是行動.
正如我之前所說.
現實是我們面臨的挑戰.
最終是需要整個社會.
都有重大的變革.

$^{881}$而作為基督徒.
我們可以和遍佈世界各地的城市的弟兄姊妹結連.
當我們團結在一起.
為公義和正義發出自己聲音的時候.
有很大的潛能.
第四就是消費主義.
我讀燈是用英文的時候串錯這個字.
就是要我們質疑我們的社會.
和富裕的國家.
怎樣導致我們消費過多我們不需要的東西.
我們有必要去仔細考慮.
我們所消費的物品的影響和來源.
但更重要的是.
可能只是有必要簡單地停止消費很多東西.
最後就是飲食.
這種消費方式.
令我們每個人都和其他受造之物緊密地聯繫.
可能的情況之下.
我們吃的食物.
是否能夠表彰那些對土壤善於耕作.
並且為工人支付高薪的農民.
我們可不可以學會少吃一些肉.
因為我們知道飲食對地球的深遠影響.
我們可以討論很多其他應用的方式.
我鼓勵你利用這次會議的資源.
來了解更多的訊息.
但有時似乎好像很沒盼望.
往往似乎做什麼都無濟於事.
但這就是為什麼我們必須經常返回在基督裡面所擁有的喜樂.
回到受造世界的美麗.
即使在嘆息中仍然可以展示出神的榮耀的受造世界.
有時我們都必須哀悼.
但即使是哀悼.
亦都讓我們記得我們在基督裡面的最終盼望.
就是萬物的更生.
我想用耶利米哀歌第三章來結束今晚的講座.
這是承認苦難的一段經文.
預備我們面對有可能發生的瓦解.
但這段經文亦指向我們終極盼望的源頭.
我用此來結束.

$^{921}$耶穌說:我求你紀念我.
如恩塵和苦難的困苦困迫.
我心想念這些就在裡面幽門.
我想起這些心裡就有寂寞.
我們不是消滅是出於耶和華諸般的慈愛.
是因祂的憐憫不自圖絕.
每早晨這都是新的.
你的誠實極其廣大.
我心裡說耶和華是我的份.
因此我要仰望祂.
(掌聲).
\newpage



\section{}
\label{sec:a0nrOscVmnU}
\textbf{「關愛受造世界與福音」香港會議 「一同歎息 迎向更新」:關浩然牧師 (回應)}
\newline
\newline
連結: \href{https://youtube.com/watch?v=a0nrOscVmnU}{\texttt{ https://youtube.com/watch?v=a0nrOscVmnU}} ~~~~ 語音日期: 2021-09-13 
\newline
\newline
\hyperref[sec:6_vdz8RQKsg]{\small{< < < PREV SERMON < < <}}
~
\hyperref[sec:index]{\small{[返主目錄]}}
~
\hyperref[sec:qPYjE7lJ5fw]{\small{> > > NEXT SERMON > > >}}
\newline
\newline
$^{1}$(字幕:梁美萍).
剛才Dr. Moon說了一些重點.
就是勾劃福音信仰.
它從受造世界在聖經裡有甚麼部份.
位置在哪裡.
作為信仰的關注點.
受造世界是彰顯上帝的榮耀.
為上帝自己受造,不是為人類去受造.
這也將我們的焦點從人類中心.
轉到以受造世界為中心.
人類只是受造世界的其中一個部份.
與其他生靈是一個淪陷的關係.
中間也提到上帝的恩典去揀選人類.
使人在萬物中有一個獨特的角色.
人是上帝的形象.
是一個恩典的揀選.
道成肉身的耶穌基督重新肯定受造世界的好.
和盼望在基督的道成肉身.
讓世界成為祂的一部份.
基督信仰雖然關注個人終極的命運.
但在Dr. Moon的說法裡.
個人終極的命運不是重心.
而是關心勞苦嘆息的受造世界.
如何在基督裡得著根深.
所以他提到基督徒要有集體行動.
最後他提到一個Awake的說法裡.
中間的A是Activism.
這種行動如何幫助我們了解教會的生活.
我們一般稱之為教會生活.
我今天想從這裡和大家分享.
Dr. Moon提出的理解基督信仰的套路.
並不是以人類如何得救為軌跡.
而是以受造世界得到根深為軌跡.
這兩個是很不同的進路.
Dr. Moon提出的進路有別於我們坊間熟悉的套路.
我們用這幅圖去看.
信徒是從永恆中唯一的存在.
他創造了這個世界.
他最後也將人帶回了天堂.
剛才黃博士也有描述我們這種信仰.

$^{41}$道成肉身的耶穌基督.
在這種說法裡很重要的就是.
他的神人異性帶來一個有效的救恩.
如果是凡人.
我們死N次也沒用.
有N個人死也救不了所有人.
但耶穌基督的神人異性.
所以他一個人死.
全世界所有人都因著他而得救.
受造世界是一個提供人在那裡得到救恩的場景.
而最終的目的就是將人帶來墮落的受造世界.
剛才黃博士也提到.
中末論關注如何得救.
如何上天堂.
當然也有新生命新生活.
倫理生活很特別很關注.
所以在香港的華人教會.
我相信可能很多教會一樣.
我們很特別關注的就是道德生活.
因為我們在說的是新的生活新的生命.
多德姆提出的一種論述.
就是以受造世界去到他的根生.
就是V1到V2.
中間經過可能有很多的過程.
在多德姆的講座裡他不是很特別說到.
但大家應該都看得出.
他講到墮落,揀選以色列,耶穌基督聖靈.
然後去到教會.
最後等待著的是新天新地.
現在我們就在最右手邊的教會.
和上高受造世界3,2.0中間的箭咀.
這個箭咀的位置大概就是我們現在的位置.
在這種描述裡面.
世界是一個等待拯救的對象.
所以多德姆提到羅馬書第八章.
受造世界是在勞苦嘆息.
等候著神的兒子的榮耀自由.
我們和合本也做得自由的榮耀.
這個論述是將世界放在中間的.
是關注的焦點.

$^{81}$傳統的教會生活.
由於這個套路的關係.
是講人如何從受造世界,墮落的世界得到永生.
所以我們往往教會的行動.
如果有活動性的說法.
教會的活動性是什麼呢?.
就是透過報道,門訓去壯大教會.
當然門訓不單止一定是堂會.
有些教會對門訓的意識包括了家庭,包括了職場.
有部分的教會可能較為注重堂會.
忽略職場的宣教或職場的見證等等.
不過無論如何.
這個論述裡面.
報道門訓,報道門訓.
不斷用報道產生門訓,門訓產生報道.
缺陷的就是論述裡面沒有關乎受造世界.
它是一個以人類活動為中心的論述.
它的特徵或口號方面.
有兩個聖經的說法.
一個叫大使名.
這個大家可能很熟悉.
就是以洗萬民作為蘇基督門徒的說法.
另一個就是.
當耶穌回答律法師的詢問的時候.
祂回答最大的誡命.
就以大誡命和大使命去說這個論述.
大誡命就是愛上帝和愛倫社.
這兩個口號的特徵.
同樣是相當以人類為中心的.
不是說它不對.
而是說那個特徵是以人類為中心.
或者由愛倫社這個說法.
可以進入到一些社會行動.
但是多數都是關乎民生的事務.
而今天特朗普提出的套路.
所關注的是生態.
是人類生活對生態的影響.
一方面有人類生活.
人類社會的生活.
但是另一方面.

$^{121}$是將受造世界放在眼界當中.
我們剛才所說的套路.
左邊的套路.
它以人類為中心之外.
它也是以個人實踐的.
我們說到報道,門訓.
很多時候是說個人報道.
和個人的門訓.
不知道大家有沒有這樣的印象.
它的實踐是可以個體的.
一個人做的.
它的焦點是很道德的.
很倫理道德的.
因為我們說的是生生命,新生活.
雖然大使命真的需要很多很多信徒去實踐.
大革命都是需要所有人.
任何人都需要去進行.
但是它大體上都是放在個人層面去理解.
在個人層面去發揮影響力.
譬如我們說的道德影響力.
多德姆提到的那種教會的行動.
就是關心人類社會的運作.
他提出過一些建議.
包括了他說交通.
或者他那邊真的行的.
不要開那麼多車了.
有關乎食物的.
關乎消費的.
這幾方面黃博士都在香港的層面有些分享.
他提到的聖經裡面說到.
「古地因著人類的敗壞而哀號.
受造世界因人人的罪惡.
被伏在敗壞之中」.
所以人類的運作.
人類社會的運作如何去.
透過一個行動改變.
帶來的是受造世界得到釋放自由.
現在就因為做得不好.
人類人口膨脹.
我們的偏差帶來生態的危機.

$^{161}$很多不同的生命被消滅.
這個是我們的罪惡.
值得留意.
Dr.Mu提到的那種叫做.
帶來重要社會層面轉變.
他的說法就是.
Major societal wide change.
不是個人生命的轉變.
他是在說社會層面.
重要的社會層面轉變.
我們在華人的社會.
華人教會裡面.
這一方面是比較陌生的.
如果我們比較這兩個行動.
在左邊所關注的是道德問題.
主要是一個道德的形象.
右邊關注的是受造世界的復原.
或者受造世界的更新.
它關乎的是一種統治.
人類如何治理這個大地呢.
是一個Royal image.
一個是Moral 一個是Royal.
不同的行動.
或者怎樣的行動.
才能夠帶來一個重要的社會層面轉變呢.
這裡就去到社會.
城邦的事.
眾人的事.
又是政治.
當我們在說社會層面的轉變.
一個重要的社會層面轉變.
就會涉及到政治方面的問題.
香港教會過去較多的政治參與.
是簽名運動.
有些是靠我們叫做朝女友人好辦事.
就是得到在上者的歡心.
可以有直線搭通政府的核心.
能夠見最高層開會.
在由上而下去帶來改變.
這種路線.

$^{201}$仍然有一些教會領袖是很重視的.
至於較遠的.
我不清楚說的.
就是簽名運動.
一人一信等等.
但還有沒有一些強硬一點的呢.
有沒有一些強硬一點的行動可以做到呢.
但在香港的環境裡面.
或者在華人社會裡面.
很重視尊卑的.
還有一種父母官.
做官就好像媽媽或者爸爸.
這種父母官觀念裡面.
我們任何強硬的行動.
會被視為不信服掌權者.
例如涉及土地發展.
在香港的社會運輸的壟斷.
大家知道現在的鐵路有問題.
大家不知道怎樣上班.
乘甚麼車.
壟斷.
食物的供應的問題.
當有甚麼豬甚麼行.
大家知道加價減價.
不會減價的.
大家會發現.
背後可能涉及的是很多我們叫做.
原來還有一些.
涉及一些官商鄉黑.
或者加上警的勾結.
軟的行動.
一人一信.
簽名運動.
有沒有效果呢.
能否帶來社會性改變.
但如果香港的教會.
我們問的是.
可不可以強硬呢.
強硬到哪個地步呢.
在菜園村的事件.

$^{241}$或者今日現在還在發生的.
明日大漁填海計劃.
我們真的涉及到很多土地供應.
和生態的問題.
中華白海豚已經差不多要整個墳墓給他了.
西貢的珊瑚.
剛剛好像有些起色.
因為畫完之後.
現在好一點了.
我們面對一些這樣的處境.
我們的活動.
究竟是以什麼方式去進行呢.
個人在華人教會.
福音是個人的.
救恩是個人的.
滿分是個人的.
所以很多時候行動都希望是個人的.
我們是行動還是教會行動呢.
當我們說大使命大誡命.
個人緩分個人救恩的時候.
我們一直以來都是強調個人的.
所以當有人叫一群什麼人.
去嘗試做一些不是個人為單位.
但他又不想代表整個宗派或整個教會.
他用一群什麼人.
很多的.
大家之前都記得.
究竟這種行動既不是個人.
又不是集體或整體.
這種名義在華人教會的活動中.
我們有沒有位置.
或怎樣為他開闢一個位置呢.
當日出現了很多不同的一群什麼人之後.
我們出現了幾種態度.
一種態度是馬上割蓆.
表達的是什麼人不代表什麼會.
另外一種情況就是.
我很欣賞一個大宗派.
一個總會.
雖然我們知道他是很猶豫或不開心.

$^{281}$不過他主動接觸這個聯署的肢體.
和那些去參與聯署的人表示欣賞.
和自己辦祈禱會.
為所關注的事情去做一些事.
這個不是最近的.
這個如果大家知道.
是早一點說大陸拆十字架的時候.
在華人教會裡面.
個體和整體一直存在很大的張力.
當你觸碰到集體的時候.
他就問你是不是代表我.
我為什麼讓你代表我.
所以我們如果不是一個個人為中心.
而是一個介乎個人和集體中間的小社群.
我們可能要發展這方面.
因為單單的個體沒有力量.
整體又怕背鍋.
又怕被牽連.
地方堂會關愛受到世界範圍的活動.
能不能夠保持一種內部的對話.
和內部觀點的多元性.
容許在個人和集體中間.
有一些單位的存在呢?.
我剛才提到.
這個.
跳了不要緊.
教會要在政府之外的行動.
不去靠政府.
我們在下面那裡.
不是靠上高政府.
不是叫政府做事.
不用說強不用說硬.
又不用說潮流.
有人不要辦事.
我們自己做.
自己做的時候.
我們的力量在哪裡呢?.
可能我們涉及的一些是民間其他的團體.
環境保育有很多團體.
受到世界關懷.

$^{321}$受到這個觀念的可能是基督徒.
基督徒能不能夠和其他宗教.
其他人一起去做這些事呢?.
尤其是在香港.
香港不是基督教為主的地方.
在亞洲有很多不同的宗教.
我們如果能夠改變這個社會.
我們很可能需要和其他人一起合作.
而華人教會卻很少在合作上.
和其他宗教團體合作.
因為我們以前說的就是這個部分.
這個報道和緩分的部分.
這個部分不需要有其他人參與.
不需要政府甚麼都不需要.
但當我們說這個部分.
我們可能要踏出和其他宗教群體.
其他的社會群體合作的部分.
因為我們在說生態.
生態是在說reconnect.
我忘了說漏了甚麼.
今天回應就這麼多.
謝謝.
字幕:MG.
(字幕由 Amara.org 社群提供).
\newpage



\section{}
\label{sec:qPYjE7lJ5fw}
\textbf{「關愛受造世界與福音」香港會議 「一同歎息 迎向更新」:黃國維博士 (回應)}
\newline
\newline
連結: \href{https://youtube.com/watch?v=qPYjE7lJ5fw}{\texttt{ https://youtube.com/watch?v=qPYjE7lJ5fw}} ~~~~ 語音日期: 2023-09-10 
\newline
\newline
\hyperref[sec:a0nrOscVmnU]{\small{< < < PREV SERMON < < <}}
~
\hyperref[sec:index]{\small{[返主目錄]}}
~
\hyperref[sec:wCdSimQhtMM]{\small{> > > NEXT SERMON > > >}}
\newline
\newline
$^{1}$今天下午的公開演講.
胡志偉牧師上來回應.
我向他學習.
他說大家坐了這麼久.
應該站起來.
享受一下上帝給我們的創作.
我們站起來.
用十秒鐘伸展一下自己.
讓你感受一下上帝給我們的身體的感覺.
各位請坐.
很感謝Professor Moon.
特別在北美洲這麼遠.
熬著jet lag飛來香港.
分享這麼重要的訊息.
剛才看到他帶來了他家鄉.
很美麗的生態圖畫.
令我很羨慕可以在那個環境居住.
不過也不要妄自菲薄.
其實香港很美麗.
香港有很美麗的自然生態.
我最希望在我還沒死之前.
都能夠保留得到.
不要讓我們經常說.
怡山填海,發展才有未來的口號去破壞.
Professor Moon.
他剛才那段時間.
把上帝的創作和基督的救贖工作.
其實解釋得很清楚.
提醒我們上帝怎樣愛他創造的世界.
而且差遣基督來救贖整個宇宙.
所以我們不可以貶低上帝的創造.
和我們今天身處的世界.
他這樣強調塑造世界的訊息.
其實反照出我們的教會.
尤其是華人教會.
有一種很他世的傾向.
即是將眼光放在一個終末的時間.
很多時候我們說.
做人最重要是怎樣?.
得到永生.

$^{41}$意思是死了之後可以上天堂.
或者傳福音都是鼓勵人.
信了耶穌.
將來就不會在上帝審判台前被定罪了.
我們也認為教會要很努力宣教.
為什麼?.
因為當福音傳到萬民之後.
基督就回來了.
我們很注目未來終末本身可以沒問題.
不過同時我們傾向貶低了今天生活的意義.
和上帝創造這個物質世界的價值.
有很多信徒抱一種所謂毀滅性的終末觀.
即是相信上帝在終末的時間.
要來毀滅這個物質世界.
到時得救的人是怎樣呢?.
就好像災難片一樣.
上了一隻救生員飛到另一個世界居住.
所謂就是啟示錄的新天新地.
不過我們要問的問題是.
上帝是否最終要做這個動作.
毀滅這個世界.
如果我們看聖經說.
他創造是甚好.
還猜顯基督來救贖這個世界.
為什麼上帝在終末的時候要毀壞了它呢?.
其實Professor Moore寫了另一本書.
即是今天我們賣的.
是一本新的.
它較舊的一本書叫做.
《Let Creation Rejoice, Biblical Hope and Ecological Crisis》.
它提出.
聖經其實不支持這種毀滅性的終末觀.
它說的確在聖經裡面.
有些暗示終末的災難經文.
好像彼得厚斯說的.
例如有形質的都要被烈火去消化.
地和其上的物都要燒盡.
這些經文其實不是說神要毀滅他的創造.
而是說他要審判那些破壞世界的罪.
包括今天我們不斷破壞環境的罪.

$^{81}$所以那些神降火不是說要毀滅,消滅這個物質世界.
而是要煉淨這個世界.
更新這個世界.
所以所謂這些終末災難經文.
看Professor Moore的解釋.
其實那裡不單止考慮了經文和文法前文後理.
更加吻合了整本聖經描述的上帝.
他怎樣說這個創造甚好.
和基督救贖的工作.
其實想清楚一點.
羅亞那時候叫做洪水毀滅大地.
有沒有毀滅大地?其實不是.
洪水沒有毀滅這個物質世界.
而方舟也不是一隻離開世界的救生月船.
而是上帝用洪水來審判人的罪.
最終方舟要回到地面.
讓人在一個結淨的土地上繼續管理大地.
而羅亞不單止拯救了他家人.
也救了創造的動物和飛鳥.
所以羅亞其實是一個生態保育者.
Professor Moore說得很對.
基督來是要更新這個物質世界.
修復上帝的創造.
這是福音.
當基督再來的時候.
他要完全更新這個物質世界.
不是要毀滅它.
我們怎麼知道呢?.
其實我們看看基督的復活.
預示了這個物質世界應該怎麼去更新.
基督的復活不是變成一個侮辱新的靈魂.
他也不是拋棄了舊的身體再造一個.
而是將釘在十字架上被罪摧殘的身體更新過來.
成為一個榮耀的身體.
所以福音是上帝結淨罪.
更新世界的工作.
而基督徒就是被呼召.
此時此刻參與這個更新的工作.
所以我們知道其實我們都很流行.
想一種叫毀滅式的宗門觀.

$^{121}$其實是不合乎聖經的.
不過當這個毀滅式宗門觀.
都這麼影響我們整個信仰表達的時候.
我們發覺這件事其實不斷地影響我們的信仰生活.
我提出究竟我們怎麼影響.
我們想想其實如果反正這個物質世界最後都要被毀滅.
那為什麼要愛惜它呢?.
就好像我們拿一張用完即棄的紙巾出來.
即是掉了也不可惜.
這樣其實某程度上我們會發生什麼呢?.
就是這個世界發生什麼事.
其實我可以不太理會.
可以避開世界上的承擔.
我想這件事不單止在環境保護上.
更加在我們會不會落實很努力地改善今天的社會.
又或者一些政治的參與上.
我們都可以不理會.
我們都要過去.
只要等那個終末的災難來到.
基督開一隻救生船救我們走就行了.
教授沒有提醒我們.
終末的災難不是救生船.
它是用來審判我們的罪.
所以上帝可以用很多的方法.
包括今天會不會已經叫到生態環境去破壞.
讓我們去經歷我們自己手上的一個狀況.
透過這件事去提醒我們其實在犯罪.
第二個我們如果相信毀滅性的終末觀的反應是怎樣呢?.
我們就會以為自己不貪愛世界.
基督徒很重要不貪愛世界.
或者我們不留戀世界的事情.
不過實在的是我們不愛護上帝創造的世界.
其實我們想我們是不是真的不貪愛世界呢?.
其實不是的.
我們仍然很喜歡這個世界.
不過只是當它是一張用完即棄的紙巾.
我們其實是這樣看的.
所以今天對我仍然有用.
我在肉身那裡可以盡情地在這個世界裡吃喝玩樂.
開心用完.

$^{161}$就留給下一代.
只要我的心歸向上帝就行了.
不過其實想清楚一點.
這不是真的不貪愛世界.
這其實更加可以是放縱自己傷害鄰舍的一個行為.
第三如果我們想物質世界要去到毀滅的時候.
我們重視什麼呢?.
我們重視靈魂.
或者重視屬靈的事情.
物質的東西反正我們不應該去那麼緊張.
所以我們的說法很多時候都將全福音改成.
或者說成救靈魂.
彷彿人的肉身好像不用救的.
剛才Professor Moon有提出創世紀第二章說.
他說到人和其他活物的分別.
他說其實沒什麼分別的.
在《正經敘述》裡面.
救靈的活人不是說沒有身體的靈魂.
而是說有生命氣息的活物.
所以其實人是有生命氣息的活物.
不只是一些沒有身體的靈魂.
所以我們的福音不是救一些非物質的靈.
更加是救贖整個人.
而當我們覺得自己最重要的是一些靈命追求.
通常我們會了解成什麼呢?.
就是要多一點靈修.
靈修,祈禱,崇拜,侍奉,傳福音這些活動.
當然我完全沒有否定這些的重要性.
而靈命的追求是很重要的.
不過我們要問.
我們的這些靈命追求.
是不是會直接帶來我們生活的選擇和改變呢?.
不過有時候我們也因為著重靈命去輕視物質.
就會將靈命和物質生活去分割.
可能有時候我們遇到經常回教會積極參與的人.
不過他在日常生活裡完全不像耶穌的基督徒.
也不一定.
不過當我們細心讀聖經.
會發覺聖經所說的屬靈人是什麼呢?.
意思其實是我們的生活形態.

$^{201}$按著聖靈帶領的人.
所以其實屬靈人是一個最落地.
在世界的事情裡最順從聖靈的指引.
又或者可以說是和這個世界和這個世俗最有分別的人.
我帶來一個問題就是.
究竟在香港生活.
今天21世紀的香港生活.
什麼叫做屬靈人呢?.
我就借用Professor Moore的最後兩個實踐.
我相信沒有什麼人會否認香港是一個消費的城市.
我們工作是為了什麼呢?.
工作其實就是為了消費.
很多人都是這樣去想的.
甚至我們決定一個人對社會有沒有貢獻有沒有stake.
都是在說他有沒有賺錢和消費的能力.
所以最後這兩個實踐.
消費主義和飲食這件事對我們其實很重要.
在香港生活消費主義和飲食享受.
是我們的生活形態其實相當難去抵抗.
我想基督徒也是很難去抵抗這種世俗.
在這個文化裡面其實我們是用飲食和消費來安慰自己.
很辛苦去做這件事.
鼓勵自己我要達到什麼目的.
我又做這件事.
獎勵自己我做得很好.
我又去做這件事.
我們很習慣去做.
其實教會和基督徒是不是都是這樣呢?.
我有兩個小朋友兩個都在讀小學.
每次在教會主日學習暑期聖經班之後都帶一堆禮物回家.
究竟我們基督徒是屬靈還是什麼呢?.
我們都是很習慣用這些去鼓勵獎勵和安慰自己.
不過當小朋友在教會收得多巧克力.
長大了就要吃自助餐.
小朋友收紙飛機長大了就要坐飛機去旅行.
我們的習慣是怎樣去養成的呢?.
又或者每到假期教會裡面不太熟悉的弟兄姊妹.
就會問放假去哪裡旅行.
好事坐飛機去旅行是一定要的.
一定要做的.

$^{241}$甚至我們有些基督徒會以為.
聖經所說的豐盛生活其實就是逛街吃飯購物旅行.
當我們努力工作或者努力侍奉之後.
就應該用這些去獎賞自己.
甚至這些是上帝賜福我們的禮物.
我們為著可以做這些事去感謝上帝.
有時候我想在消費主義的香港.
其實教會有時候也很世俗化.
基督徒不貪愛世界.
其實是說世界的人愛的東西追求的東西.
我們是不愛不追求的.
其實這是一個不貪愛世界的真正的意思.
所以我們要有一種淑齡的眼光.
看穿消費主義不能夠帶來的豐盛的生命.
想想我們平時買東西.
買那一下其實是最開心的.
拿著袋子走就很開心了.
吃好東西那一下也很享受.
不過可以維持多久呢.
一下子滿足過之後.
下次就再來過.
搞到我們家裡其實很多東西.
有多少東西放在家裡其實等待一天.
當夠時間就等我們拿到箱子裡.
或者送給別人.
我們很多都是這樣的生活模式.
又或者我們有時候到了坊間.
之前幾個月就很雀躍地.
很期待地安排怎樣去旅行.
去到很開心地又喝又吃.
還要搬一大堆手信回家.
其實你想想我們買手信開心一點.
還是收別人手信又喝的感覺呢.
是不是真的這麼開心呢.
為什麼呢.
還是反過來.
假期為什麼我們不可以簡單地留在家裡.
平時很少見面的朋友見一下面.
平時忙到這麼厲害的生活.
可以真的休息一下.

$^{281}$簡單地去聚一下.
究竟哪個生活型態.
給我們更大的滿足呢.
還是我們其實習慣了做.
剛才以上很多的事情.
究竟這些活動.
令我們很開心很滿足.
還是我們沒想過.
因為世界這樣做.
每個人都這樣做.
我就這樣做.
如果是這樣.
我們是世俗還是屬靈呢.
(可能是屬靈).
好,接著我就要結束了.
我想我的總結就是.
環境破壞和生計危機.
剛才Professor Moo說.
我們今天其實是在一個非常獨特的時代裡.
21世紀.
在這個世紀.
人類面對一個很嚴峻的挑戰.
我們說基督徒其實應該成為世界的光.
世界的明燈.
我們都知道今天的處境.
是需要活出一個.
如何和世界不同的屬靈人的樣式.
如果這樣的話.
我們要明白基督救贖的意義.
信服聖靈的帶領.
在最落地.
最生活上的細節的選擇裡.
我們參與福音的工作.
這是我的回應.
(掌聲).
\newpage



\section{}
\label{sec:wCdSimQhtMM}
\textbf{「關愛受造世界與福音」香港會議 「門徒生命與關愛受造世界」:胡志偉牧師 (回應)}
\newline
\newline
連結: \href{https://youtube.com/watch?v=wCdSimQhtMM}{\texttt{ https://youtube.com/watch?v=wCdSimQhtMM}} ~~~~ 語音日期: 2021-09-09 
\newline
\newline
\hyperref[sec:qPYjE7lJ5fw]{\small{< < < PREV SERMON < < <}}
~
\hyperref[sec:index]{\small{[返主目錄]}}
~
\hyperref[sec:9AdkPpUw1wU]{\small{> > > NEXT SERMON > > >}}
\newline
\newline
$^{1}$我相信在座的各位都坐了一段時間.
聽Dr. Moo講一篇很神學,很理論的講座.
這段時間大家起起身,有些伸展.
我們去享受上帝的創造,就是我們的身體.
所以我們都可以起起身.
請大家可以坐下.
我剛才大概一點半鐘由九龍塘站走過來的時候.
都感受到原來在十一之前,現在過了中秋.
原來還是這麼熱的,在熾熱的,當然不是紅太陽.
熾熱的太陽下,我由九龍塘站走到中晨.
都覺得熱力很驚人.
所以在這段簡短的時間裡.
特別在現在社會的氛圍裡.
今天下午難得有這麼多的教母.
弟兄姊妹,同道一起來關心上帝的創造.
確實是一件好事.
所以我就按著Dr. Moo所分享的.
跟大家有一點點的回應.
Dr. Moo在他的講論裡有兩個重點.
第一個重點就是指到福音派的教會.
也不是福音派,教會很多時候忽略了.
我們真實的活在現有的地球裡.
特別我們華人教會,我相信大家都知道.
我們很多時候都是有一種諾斯底主義的熟悉的色彩.
我們是鄙視,不喜愛我們的世界.
所以如果愛世界就一定是錯的.
所以大家就會明白為什麼楊福音在所有華人的譯本裡.
特別是三章十六節.
一定是神愛世人,不可以是神愛世界.
大家明白了,是不是.
那裡的希臘文很清楚是神愛世界.
但是我們的神學或者翻譯者.
是受他很保守的神學思想的影響.
是不可以有世界的.
我們不可以愛世界的.
所以這個就是我們的問題.
我最近看了一本書.
這本書是講亞倫特的傳記.
是一本很厚的書.
這本書很吸引我.

$^{41}$就是它的書名.
它的書名是講什麼.
是愛這個世界.
For love of the world.
他不是一個基督徒.
他是一個思想家.
但他也說可以愛這個世界.
確實叫我們這群作為神的子民.
是有一點點慚愧.
或者很多的慚愧.
當然第二個忽略.
在Dr. Moo的講論裡告訴我們.
原來整傳福音和使命.
是必然是包括.
當我們說愛上帝的時候.
我們必然愛上帝所創造的世界.
就是我們現在所身處的這個地球.
今天很多時候.
環教會,香港教會.
都很受一種錯誤的中末論的影響.
於是我們強調.
這個世界不是我們的家.
我們最好就是一天.
我們每個人都是坐太空船羅亞號.
接我們去天堂裡.
但聖經或者是在Dr. Moo的講論裡.
說得很清楚.
我們不是救贖脫離這個受造的世界.
我們本身是成為夢救贖的.
這個受造世界的一部分.
所以不是將來整個地球毀滅.
這裡已經有很多的神學討論.
我就不詳細說.
當我有機會出席.
2010年.
在開普敦舉行第三屆的.
這個洛桑世界峰會議的時候.
後來就出了這個cap time commitment.
在裡面Chris White說得很清楚.
整團的使命.

$^{81}$就是指導上帝這個好消息.
是藉著十字架.
和復活的耶穌基督.
傳給我們個人,社會.
和這個受造之物的好消息.
是個人,是社會.
和包括這個受造的世界在裡面.
而這三方面都是因為罪.
帶來破碎和痛苦.
而這三方面都需要神的救贖的愛.
和使命在裡面.
於是這個就成為了神指紋.
使命的那個內容.
所以在cap time commitment裡面.
就說到我們要愛神創造的世界.
這裡我就不作太多的論述.
這些就因為如果再說.
就變成了很學術性.
回過頭來看香港教會的現實.
最新2019年.
教會的數據還沒出.
因為我們現在還是收緊那些回卷.
但很明顯可以看到環保.
是由2009年開始.
當時的關注是有13.3.
就是說在不同的回應裡面.
但去到2014年的時候.
你看到那個回應是6.1.
盼望經過這次香港會議.
和更多的教會關注之後.
在未來的日子裡面.
有關的這一方面是會提升.
當然我不會說是放在第一的優先次序裡面.
至少我們都應該學一下聖公會.
所以聖公會我是很讚賞.
聖公會在他們的使命宣言裡面.
至少排第五.
排第五也好.
排第五就是說他們除了宣講上帝國的好消息.
在第五裡面.

$^{121}$特別是強調是自力保護受造物的整全性.
保護地球的生命.
你說他有沒有做?.
另一件事.
但至少在聖公會的使命宣言裡面.
是有包括這一點.
請問在座.
在你的教會裡面.
在你的使命宣言裡面.
有沒有這一項在你六點.
七點.
八點.
十五點裡面的其中一點呢?.
這個我們就要去思想了.
所以今天很簡單.
在這十五分鐘的回應裡面.
我就嘗試用care這個字.
四個層面.
作出呼喚.
給我們的同工.
頂至每一同的思考.
第一個我們需要celebrate.
就正如剛才說完.
Dr. Moo說.
他引用Bloombergman所說.
安息的慶祝.
就是抵抗和另類選擇的行動.
是一個行動.
這個行動就表示.
resistance和另一個alternative.
所以第一樣我們需要的.
就好像一個在中世紀的.
一個修道的一位姊妹.
他所分享.
Hildegard他所說的.
所有創造都是一手向神讚美的思覺.
剛剛上個星期.
我們有機會有兩次的靜修營.
帶領的一位日本的.
屬靈道師Oktawa.

$^{161}$就告訴我們.
今天我們需要celebrate.
celebrate很重要的.
就不是因為對方.
在你身上做了什麼.
而是因為對方的presence.
就已經值得我們celebrate.
所以我們感謝神.
至少我們在香港.
我們仍然可以有郊野公園.
仍然是可以.
如果你不想聞那些催淚煙的時候.
你可以去到這個郊野公園裡面.
至少那裡不會有警察.
至少那裡不會有其他.
你可以在那裡享受大自然.
這個就是香港的好處.
我們celebrate.
今天有些時候.
有些人有一種錯誤的理解.
這個理解就是.
天堂只是.
只是人類.
或者只是靈魂存在的空間.
這一點我就不同意了.
我就想如果是新天新地.
在上帝將來的救贖的世界裡面.
只是人類.
只是靈魂就很悶的.
我是需要有動物.
有貓狗.
花草樹木.
陪伴我.
在一起.
而這個就只是在伊甸園裡面告訴我們.
在將來新天新地裡面.
是必然都會有的.
所以我們歡喜.
是為了整個上帝的創造.
是彰顯三一神的榮耀.

$^{201}$是為了人類的美善.
所以第一樣.
我們可以思想的就是每年.
每一間教會.
都至少有一次關愛受造的主日.
你可以參考比方.
用Earth Day.
有這個地球日.
就是每年在4月22號的時候.
你會在前後或者在一些日子裡面.
你有一個關愛.
這個世界的一個主日.
來到這方面.
來到為上帝所創造的一切來慶祝.
在2010年的時候.
那次我們在聖公會朱聖堂.
就有一個冬日與動物同讚美的.
一個特別的崇拜.
那次是有貓有狗有蛇.
有不同的.
在一起.
在裡面.
當然在平時的崇拜裡面是比較困難的.
但如果一個特殊的主日.
我相信.
都是讓所有人去認識.
我們是可以為上帝所創造的一切.
來獻上我們的慶祝.
第二個我們可以advocate.
就是昌道.
今早我都想過.
如果我從元朗跑步出來忠臣.
要多少時間呢?.
後來我想如果我跑步過來的話.
我應該不夠氣去說這15分鐘.
所以我就放棄了.
大家都知道.
這個通貝里.
瑞典16歲的女孩子.
她就不坐飛機.

$^{241}$所以下次就叫Jonathan不要坐飛機了.
坐船.
坐船由美國來香港.
不知道要多少天了.
所以我們看到他就在Dr. Moo的講論裡面.
就說兩樣東西不是必然的.
人類的福祉和環境的保護不是對立的.
福音的宣講和關愛的受造也不是對立.
所以今天我們不一定說.
救人的靈魂比救樹還重要.
我們不需要必然存在這種衝突.
所以這個都是以往在2007年的時候.
教新辦過一些全球暖化的講座.
當時參加的人不是很多.
只有三十多個我不多說了.
第三我們說了.
第一celebrate.
第二advocate.
我們繼續教會.
不同的機構不同的教徒去倡導.
我們要愛這個受造的世界.
第三我們需要來過一個簡樸的生活.
我們來散用.
所以我們在過往一些遊行示威裡面.
或者一些是行街的一些活動裡面.
行街的活動裡面.
我都會見得到他們都做得很好的.
他說在這些行街的活動裡面.
他出得來閒就預備了要回收.
是不是你預備了回收了沒有.
如果要回收的時候.
他們就有一個實際的.
我想一下今天在教會裡面.
特別是大的教會.
會不會都需要思想.
所以現在我們看到聖公會或者一些教會.
他們都來到是履行管家的職責.
現在特別是最近都很多的那些飲水.
膠瓶可以回收屈臣氏或者其他.
如果大的教會我覺得其實是可以的.

$^{281}$在你那裡裝一部這樣的機.
或者可以禮拜日的時候.
收集他們的膠瓶.
然後將這些膠瓶拿去回收機裡面.
來化作一些優惠券或者等等.
衣物的回收或者等等.
這個都是今天教會可以來實踐的.
來reduce這一方面.
散用這一方面.
這些原則我就不詳細講了.
很快跳過去.
很快第四.
就educate.
也幫協進會賣廣告.
他們有一個不錯的課程.
有一套關愛受造世界成人主學的查經課程.
你在網上可以download.
甚至你可以跟他們購買.
所以我會覺得今天教會是需要更多這方面的教育.
每年可能兩年一次有主學課程.
團契小組有關愛受造的活動.
甚至你可以參觀安排.
你去中大參觀氣候變化的博物館.
或者加道理農場.
他們有不同的活動.
這些活動都是幫助鼎姐妹.
去深化他們環保的意識和行動.
最後.
這些我不詳細講了.
我們很快跳過去.
在這方面天主教是做得比我們好.
所以這個就剛剛上個禮拜.
在天主教的一個退休營裡面.
就看到.
他們在這個月份裡面.
鼓勵教會對受造界有更多不同的活動.
在當中的參與.
很快最後.
最近一首歌很流行.
就是願榮光歸香港.

$^{321}$我如果這樣講就不是很好.
願榮光歸地球.
行不行呢.
他其中第一句我覺得很有意思.
何以這土地淚在流.
正是配合我們這次的主題.
一同嘆息.
我們現在.
我相信每一個香港人.
你是環保人士也好.
不是環保人士也好.
我們現在都在嘆息.
可能我們嘆息要過了十一之後.
看看不知道怎麼樣.
所以何以這土地淚在流.
所以祈求上帝賜福我們每一個.
以至我們願意在地上見到.
上主的榮光.
向當中彰顯.
多謝各位.
字幕志願者:陳志全.
(掌聲).
\newpage



\section{}
\label{sec:9AdkPpUw1wU}
\textbf{「關愛受造世界與福音」香港會議 「門徒生命與關愛受造世界」:鄺偉文博士 (回應)}
\newline
\newline
連結: \href{https://youtube.com/watch?v=9AdkPpUw1wU}{\texttt{ https://youtube.com/watch?v=9AdkPpUw1wU}} ~~~~ 語音日期: 2023-07-31 
\newline
\newline
\hyperref[sec:wCdSimQhtMM]{\small{< < < PREV SERMON < < <}}
~
\hyperref[sec:index]{\small{[返主目錄]}}
~
\hyperref[sec:EDiGHY_cDLQ]{\small{> > > NEXT SERMON > > >}}
\newline
\newline
$^{1}$多謝Dr. Mool和胡牧師.
他們分享了很多聖經或神學上對這個議題的討論.
我就跳開這個層面.
我想今天用少許時間去回應.
或者從一個世界的角度去看這個議題究竟發生了甚麼事.
我會探討兩點.
剛才Jonathan也提到.
第一點是關愛受造世界和貧窮的關係.
第二點是關愛受造世界和福音的關係.
剛才提到這是福音的一部分.
是整全福音的一部分.
我也會回應這兩點.
首先我想從關愛受造世界和貧窮的關係開始說.
這樣我才能解釋我今天站在這裡.
因為施達就是做扶貧救災的工作.
首先我想進入的時候.
就用詩篇19篇12節進入.
大衛的這個禱告就算不常常.
我們久不久也會說.
求主耶穌去寬恕我們一些隱然未見的罪.
隱然未見的罪是過犯得罪了上帝得罪了人.
Jonathan提到.
我們不可以只說愛人愛神.
但不愛我們的世界.
我們在日常生活裡.
事實上很多時候我們有意無意地.
我們在破壞我們的環境.
破壞我們的生態.
從上帝創造的角度.
當然我們這樣破壞的時候.
無論我刻意也好.
不覺意也好.
或者完全做了也不知道也好.
我們當然得罪了上帝.
因為這些是祂的創造.
是祂的東西.
我打爛了別人的東西.
我就是錯了.
但是何解不只是得罪神.
同樣得罪人.

$^{41}$我想和大家分析一下一些數據.
很快的.
在網上很容易找到.
不過我想指出一件事.
根據世界銀行的一個研究.
它的統計數字.
在由1960年到2018年.
差不多50年裡.
全世界的人口增長了.
由大概30億增長到超過70億.
即是說是1.3倍左右的增長.
但是在同期裡.
全世界的二氧化碳排放量.
由1000萬的KT增加到超過3500多萬.
即是差不多大概是3600到3700中間.
即是說是3倍多的增長.
而這個增長.
其實我們都會明白.
這幾十年裡世界的發展是很快.
我們無論在工業各方面都很快.
但是其實其中一方面的增長是什麼呢.
是為了要達到一個什麼目的呢.
不是因為譬如我們要幫人延長壽命.
那些有一部分.
但是其中一個方面.
就是我們在追求舒適方便.
我小時候.
我住政府樓.
一層樓是約莫有20戶.
20戶裡面基本上只有一家或兩家有電視機.
但是今天大家回家數一數.
你家裡有多少部電視機.
我們去追求這個方便舒適的時候.
我們從經濟的角度.
我們產生了一些經濟的動力.
產生了即製造了提升了發展.
但同樣也都帶來很大的二氧化碳的排放.
從另一個圖我們看得到.
這個人均的耗電量.
在同期是由1200度一個人一年.

$^{81}$上升到今天的超過3000度.
兩倍多的消耗用電.
我們看到其實我們都明白了.
不需要詳細去再解釋.
是我們的消耗能源方面.
又有多少是必要有多少是不必要.
或者是可以減省的.
我們可以自己再思想.
但是無論如何.
其實所有這些的二氧化碳的排放.
或者其他的氣體排放.
我想大家都明白.
今天就產生了我們常常說到的議題.
就是氣候變化的問題.
氣候變化有很多學者去研究.
基本上可以簡略地歸納了.
氣候變化對人類的影響是什麼呢.
就是天災.
這些是我們在南亞做救水災拍到的一些照片.
其實現在我有同工在印度.
和當地的夥伴一起去服務一班.
做一個救援的工作.
因為早前有水災.
受影響的人超過100萬.
每年都有很多這些工作需要去做.
天災是一個氣候變化.
為我們帶來一個很直接的問題.
我們大家去年都受過了.
今年還沒有來.
但其實也不只是天災.
好像剛才胡牧師說.
在火車站走到中午都出汗了.
氣候越來越熱.
天氣越來越熱.
上個星期一.
聯合國就開了一個氣候行動的峰會.
在這個峰會開始之前.
他們出了一個報告.
過去這五年.
氣候變化的溫度是很高的.

$^{121}$有史以來最高的五年.
我們都可以想像到.
氣候變化對我們的影響.
當然氣候變化也帶來很多其他問題.
海岸的侵蝕水位的上升.
我們有一天都很害怕.
大家去馬爾代夫玩.
可能沒什麼機會去.
但又要坐飛機.
又要去太平洋排放.
去還是不去呢.
食物短缺是一個問題.
其實在我們服務的.
譬如在非洲.
曾經出現一些情況.
在短短半年裡.
之前是乾旱.
你要做救災.
就是乾旱的救災.
半年後你要做水災的救災.
我們可以想像到.
對食物供應的影響有多大.
另外公共衛生.
很多災害.
無論氣候,天氣,溫度的問題.
或者水的問題.
其實帶來很多傳染病.
各方面的問題.
也因為有很多人.
因為天災.
或者剛才說的四點問題.
而被迫要流離失所.
要離開他們自己本身的地方.
聯合國在六月份的時候.
出了一個報告.
是一個預測.
預計如果我們今天什麼都不做.
到2030年.
我們會推動一億二千萬的人.
原本不貧窮.

$^{161}$推動他們變成貧窮.
而且到2030年.
我們過去的五十年.
所有的經濟增長GDP.
全部會抵消.
因為氣候變化的問題.
從這個大家可以看到.
氣候變化對我們其實是很大影響.
但其實這樣.
氣候變化是說.
二氧化碳排放,溫室氣體排放.
我們還沒有提到一些大自然上的破壞.
譬如砍伐樹木.
亂用土地.
很多海洋的破壞.
我們很多的飲管,塑膠.
對海洋生物,對海洋的污染.
都還沒計算在剛才我說的後果裡.
氣候變化對人人都受影響.
但是對有錢人和對窮人又不一樣.
剛剛士丹福大學就出了一個研究.
他說由1961年到2010年.
即是五十年裡.
因為氣候變化的緣故.
貧窮國家裡的人.
基本上每個人損失了17-30%的GDP經濟得益.
但這些損失去了哪裡呢?.
去了富裕國家的人身上.
在這個報告裡出了一個名單.
我稍微說一下.
所謂裡面有贏家,輸家.
贏家是誰呢?.
排第一名的是挪威.
排第二名的是加拿大.
輸家是誰呢?.
輸得最多的是蘇丹.
第二名是印度.
因為這個氣候變化.
令到貧富的國家裡的人.
受到的影響原來是有正有負的.

$^{201}$所以因為這個緣故.
這個報告也計算了出來.
在這五十年裡.
貧富國家的貧富懸殊.
因為氣候變化的緣故增加了25%.
換句話說.
氣候的破壞.
是帶來同樣的.
我們很多時候在社會其他經濟的行動裡.
同樣的後果就是.
富者越富,貧者越貧.
貧窮人呢.
我們中國人有一句話.
叫做靠山吃山,靠水吃水.
對於貧窮人來說.
特別是在他們身上.
因為他們沒有其他的工具.
或者其他東西可以幫到他們.
在這樣的情況下.
對氣候變化對貧窮人的傷害是很大的.
也因為這個緣故.
聯合國早幾年定出來的可持續發展目標.
17個目標裡的第一個目標.
就是No Poverty,就是要消滅貧窮.
而在這個目標裡.
其中一個點.
就是要處理氣候的問題.
從這些數字.
我們看到.
環境的破壞對貧窮人影響很大.
我們有意無意地去破壞.
影響了我們的環境,影響了我們的生態.
其實不單止得罪神.
也得罪了人.
特別是貧窮人.
可能這些就是我們隱而未見的罪.
另外一點我想討論的就是.
關於受造世界和福音使命.
我們的使命的關連是什麼呢?.
剛才Jonathan讀了羅馬書八章十九到二十三節.

$^{241}$在裡面其中有一節.
我覺得特別想提一提.
中文是這樣說的.
是二十一節,八章二十一節.
但受造之物仍然指望脫離敗壞的合制.
得享神兒女自由的榮耀.
在這裡保羅在說.
受造世界正在面對,正在經歷一個敗壞.
而這個敗壞也是因為人的問題而帶來的.
所以它就像我們人一樣.
我們仰望榮耀的那一天來臨的時候.
這個受造世界同樣仰望著這一天來臨.
不多說背後的神學思想和討論.
因為剛才Jonathan和胡牧師稍稍提過.
如果大家想知道更多.
可以到樓下買Jonathan的書.
我幫他買了植入式廣告.
但我在想.
如果我們要實踐的時候.
我知道地球的破壞.
無論是有意還是無意.
在影響很多人,特別是貧窮人.
也是一個不合上帝心意的動作的時候.
那我怎樣去實踐關愛受造世界呢?.
在實踐上,我發覺.
放回我們今天的世界,今天的社會裡.
其實我覺得.
我們今天的社會,我們的主流文化.
是遷就不了我們去實踐關愛受造世界.
剛才Jonathan提到.
最後的時候提到有兩大點.
我們去實踐的時候有兩個大方向.
第一個就是.
我倒轉了次序.
其中一個就是.
把我們放回創造裡.
讓我們回到創造裡.
我就想.
把我們放回創造裡的時候.
其實有兩個大層面.

$^{281}$我們要跨越今天的文化洪流.
第一個就是時間的觀念.
我們今天很講求效率.
是不是什麼都要講效率?.
不只是商業世界.
在我們的教會.
在我們機構的運作裡.
我們有意無意地,我們不覺的.
我們都講求效率.
今天的主題是門徒生命.
和關愛受造世界.
近十年我們就刻意.
很多教會.
將它做一個主題就是門徒訓練.
我們怎樣做門徒訓練呢?.
就做課程.
每一次講哪一課.
怎樣讀完六十六卷聖經.
全部有時間表.
我們要講效率.
我們要達到一個指標.
這就是我們今天的社會的主流.
但在這樣的主流情況下.
我們的時間的運用.
就被這個主流牽著.
如果我們要親近上帝所創造的大自然.
我們要撇開.
星期六日都要加班開工.
我們要有些時間出來要去.
這些是我們要跨越.
我們今天主流的生活習慣.
我們要做這件事.
所以我們要調節我們的時間觀念.
我們的時間觀念是怎樣呢?.
在很多外面的世界.
特別是貧窮國家的朋友.
他們的時間觀念跟我們不一樣.
不是誰對誰錯.
只不過給大家參考和分享.
我想大家都有經歷.

$^{321}$中國大陸也是.
有很多地方都是.
譬如星期日崇拜.
它沒有時間限定.
我們崇拜就會分.
早,午,晚.
我們講得很艱難.
半小時就半小時.
已經過了.
因為還有下一堂.
來不及接軌.
交接都接不到.
那怎麼辦呢?.
但在很多其他地方.
誰來就敬拜.
牧師就一路講.
整個崇拜由開始有人來.
到最後那個人離開.
可能是四,五個小時的事.
甚至我見過有些差不多一天的時間.
他們是這樣的.
我們去想想.
我們對時間的觀念是怎樣呢?.
可能我現在都過了.
大家不要挑釁我.
我要調節大家的時間觀念.
第二點是.
我們怎樣運用我們的時間.
剛才提到.
Jonathan也提到.
我們要分離我們今天.
很多電子工具.
我們的電話,ipad,電腦.
不容易的.
是很難的.
是很艱難的一件事.
我們不能的.
因為工作各方面.
以我個人的經歷.
我真的曾經試過.

$^{361}$是電郵癮.
星期六,日.
因為那時我在美國機構工作.
美國人星期六,日不開工.
其實我根本就不會有電郵.
但一起床自此而言.
洗面換好衣服後.
第一件事就是開電郵.
但每次都沒有電郵.
根本不關事.
我們會很習慣這件事.
但我們怎樣跳出被奴役的習慣呢?.
我們怎樣運用我們的時間呢?.
Jonathan提到第二點.
就是把創造放回新創造.
我們要思考多點.
從舊約裡的創造觀念.
如何與新約裡的新天新地.
融合在一起去理解.
然後在教會裡面去教導.
但另一方面.
其實另一個很主流的文化.
今天阻礙我們去做這件事.
就是我們很多時.
有一個很自我中心的情況.
我們追求舒適,方便,物質.
我們買的東西.
可能有很多我們都不太需要.
我們所謂的needs and wants的分別.
這是個人的層面.
我們知道香港.
今天香港的手機滲透率超過100%.
換句話說.
平均每個人多過一部手機.
是否真的有這樣的需要呢?.
我不知道,可能有的.
但有些人會有很多部手機.
我們有沒有這樣大的需要.
還是純粹是一個wants.
或者盼望,desire呢?.

$^{401}$這是個人層面.
但甚至今天近這十年.
或者這七,八年.
甚至走到一個在國際層面.
都很個人主義.
這個個人主義是甚麼呢?.
就是我們所說的.
其中一個展現出來的.
就是所謂的民粹主義.
即是nationalism.
2015年簽訂了巴黎協議.
關於環保,關於氣候變化.
眾多參與國承諾盡量減排.
希望控制到全世界平均的氣溫.
由工業革命做一個底線.
不會增加超過1.5度.
2015年簽了.
但是美國總統特朗普.
一上台的時候.
第一件事就是反悔.
到去年聯合國的氣候行動峰會.
巴西都幾乎想退出.
我們知道現在亞馬遜森林.
還燒得很熱.
今年峰會剛剛在星期一才開完.
沒有甚麼具體結果.
現在其實都很擔心.
不要說1.5度.
可能很快超過2度,3度.
科學家計算.
如果我們平均氣候溫度增加2度的話.
所帶來的經濟損失.
大概是15%.
而增加3度是25%.
25%是多少呢?.
如果大家讀歷史就知道.
美國經濟大蕭條的時候.
就是25%.
我們可以想像到全世界大蕭條.
只要全世界的溫度提升3度.

$^{441}$我們就全世界大蕭條.
到時都不知道情況是怎樣.
從這些角度去看.
我們去實踐關愛受眾世界.
去學習愛護這個世界.
去保護這個世界的時候.
其實它不單止是一個福音的使命.
我會加上.
它是一個跨文化的福音使命.
因為我們要跨越我們今天.
慣常主流的文化.
是不容易的.
我們說門徒生活,門徒生命,門徒訓練.
很多教會都有去做扶貧的工作.
我們奉獻.
我們可能去派飯.
我們去做很多的服事.
但其實在同一個時刻.
我們去學習實踐關愛這個世界的時候.
其實我們就已經是.
因為我們不會每一分每一秒都要奉獻.
我們都會有一個時間.
我們不是每一分每一秒都去服事有困難的人.
但原來我們去實踐關愛環境的時候.
我們是每一分每一秒都在間接服事貧窮的人.
而這個服事正正就是我們基督徒.
我們的信仰裡其中一個很重要的特點.
其實我們常常是一個跨越主流文化的服事.
我的總結就是.
我們在這個地方香港.
其實我很感恩.
大家可能想像不到.
世界上有很多地方的人.
他們有海就沒有山.
有山就沒有海.
或者兩樣都沒有.
只有沙或者只有雪只有冰.
我們很好.
我們有山又被海包圍著我們.
也有四季我們可以去體會上帝的創造.

$^{481}$但我們同時也有一個很現代化的城市.
但我們可能一般一個多小時車.
我們已經可以接觸到山和海.
但我們有沒有珍惜.
有沒有去享受上帝的創造.
正如剛才牧師提到.
我們去慶祝的時候.
我們有沒有去慶祝呢.
我自己覺得很簡單的事情已經可以做到.
首先我們去享受和珍惜我們的大自然.
我們也要去保護它.
我們知道接下來的填海計劃.
各方面其實都有很大的影響.
我們怎樣去調節我們的發展步伐.
生活步伐呢.
值得我們去反思.
值得我們去鼓勵弟兄姊妹去反思.
最後其實是很挑戰我們每一個人.
我們能不能控制自己對舒適方便.
甚至物質上的追求.
我不夠膽從元朗跑出來九龍塘.
但我通常因為施達的辦公室就在佐敦.
我坐地鐵就會經過荃灣線的油麻地.
有時候我會在旺角出.
有時候會在油麻地出.
我就會走回去.
我們去學習去訓練自己.
減低我們對舒適方便的需求.
今天就到這裡.
謝謝.
謝謝大家.
(字幕由 Amara.org 社群提供).
\newpage



\section{}
\label{sec:EDiGHY_cDLQ}
\textbf{「關愛受造世界與福音」香港會議「門徒生命與關愛受造世界」: 答問環節}
\newline
\newline
連結: \href{https://youtube.com/watch?v=EDiGHY_cDLQ}{\texttt{ https://youtube.com/watch?v=EDiGHY\_cDLQ}} ~~~~ 語音日期: 2023-08-04 
\newline
\newline
\hyperref[sec:9AdkPpUw1wU]{\small{< < < PREV SERMON < < <}}
~
\hyperref[sec:index]{\small{[返主目錄]}}
~
\hyperref[sec:dJ7VfX3ypUE]{\small{> > > NEXT SERMON > > >}}
\newline
\newline
$^{1}$我們的Q and A是這樣的.
現在我們的同工就設置了咪高峰在中間.
如果我們當中的弟兄姊妹有問題.
我都邀請大家可以來到前面.
用咪高峰發問.
我們想爭取時間讓弟兄姊妹多點發問.
所以問題盡量簡短.
是發問不是發表.
一分鐘問一個問題.
我把時間交給大家.
可以用中文或英文發問.
(英文).
我是在這間實學院教書的.
一個很簡單和實際的問題.
對很多人來說.
用一個很簡單的生活去生活.
是非常之難去做得到.
我的女兒也會和她的朋友去旅遊.
(英文).
比如說香港人吃很多牛肉.
但如果要做一個素食者就很困難.
(英文).
實際上怎樣去遊說人去環保一點呢.
(英文).
其實我應該由其他人去就香港處境回應.
我發現在我自己和其他學生的生命中.
其實我們以為犧牲自己.
但其實我們重新拾回一些基督教的價值.
就是我們的快樂和喜樂是從哪裡來的.
其實一個更簡單的生活.
會更加令我們更加有喜樂.
也可能會吸引到其他人加入我們.
包括你女兒.
(英文).
其實我用中文回應吧.
不用同工翻譯.
我拿起咪高峰其實沒有答案.
不過我們私立做扶貧.
我們發現其實是真的.
貧窮國家裡的人可能被迫.

$^{41}$但他們的生活遠遠比我們簡樸.
簡單過我們很多.
所以有時候我自己的禱告是很害怕的.
(英文).
這樣祈禱也好.
但我想我們要一步一步學習.
我們從心底開始的時候.
我們可以做到一些具體的事情.
每個人可能都不一樣.
例如剛才Jonathan提到要截斷社交媒體和手機.
其實我聽過很多教會近幾年.
他們在青少年團體裡都會實踐.
例如一晚的retreat.
那一晚真的全部充公了.
充公之後.
原來第二天他們的回應是真的很好.
團友之間的交流是真的多了時間.
讓他們慢慢體會到.
我覺得教會是一個場地.
其實是值得開始去走.
教會我們怎樣可以去減排呢?.
怎樣可以學習在教會的運作裡去愛惜這個世界呢?.
有很多張力的,我明白的.
一方面我們不想弟兄姊妹崇拜的時候拿手機出來.
但另一方面如果我不印周刊.
我全部都做QR Code給他們看的時候.
他們又要拿手機出來.
很多這些問題.
但要靠教導讓弟兄姊妹去明白.
我覺得我們要一步一步去走.
但可以由教會開始.
(廣東話).
今天聽到大家的回應.
教會如何去實踐關愛創造這件事.
但我感覺這個很重要.
提醒大家很重要.
在個人方面如何去做好工作.
但這似乎只是個人的面向.
在教會個人生活上的面向.
但更大的問題是公共面向的事項.

$^{81}$究竟教會或教務同工在這個角色裡.
其實我們應該怎樣做.
剛才大家提議的方式.
我相信二三十年前.
我們面對很多情況下.
提出很多環保的概念.
都是說個人如何去節能.
如何做好回收的工作.
這些是從個人角度去說.
但現在我們面對整個社會.
整個世界是很多經濟發展的主導下.
令很多政策在破壞這些.
我們做多少個人的事.
都沒有辦法去補償這些破壞.
而這些破壞更加是我們應該.
參與當中扭轉發展.
我聽不到大家在這個面向裡.
是否教會或教務同工.
是不應該提出這些事.
我想大家回應一下.
這是一個教會和信徒.
在公共政策上的角色.
基本上我相信剛才這位同工所提的.
都是我們需要關注的課題.
我想今天下午在Jonathan的演講裡.
很強調宣講和教導.
所以當我們在堂會裡實踐的時候.
第一就是教務.
我剛才也說了.
我不是要求很高.
一年有沒有一次在講道的信息裡.
其實在舊約新約很多經文都有提及.
不過很多時候.
我建議大家用英文的譯本會比較好.
華文的譯本很不幸.
把很多關於創造的東西.
都變成了死物.
大家明白了.
變成了物.
將它完全物化.

$^{121}$變成了沒有生命.
變成了很多時候.
我們仍然會將人.
當然肯定人的獨特價值.
在Jonathan的演講裡也說得很清楚.
不過我們不是因為肯定人的價值.
而需要否定其他.
受造物的或者受造生命.
它本身的價值.
我想在堂會裡.
當我們去做這方面的教導的時候.
我們本身可以去慶祝的時候.
然後我們自己來實踐.
就能夠提供另外一個alternative.
來給予這個世界和社會.
如果我們.
比方說有些教會在外國也會有.
他有綠教會.
或者在他教會的建築物裡.
他也是怎樣去實踐一種環保的概念.
於是或者在教會.
現在我在我一間的分堂.
宣道會會幕堂.
他們每一次完了崇拜之後都聚餐.
但每一次聚餐都是自己家庭帶他們自己的食具回來.
然後自己洗.
是不用交碗,交碟,交杯的.
換句話說.
教會很簡單.
在你聚餐裡.
同樣來到實踐的時候.
我相信當我們有更多的個人和群體來實踐.
就能夠向社會去展示另外一套的價值.
當我這樣說的時候.
不是否定我們可以在公共課題裡.
比方說教會裡有一群人.
當他對某一個課題.
比方說對海洋污染.
或者對現在這個郊野公園.
現在政府要準備收地來建屋.

$^{161}$或者明日大嶼的建屋計劃.
當然最近沒有了.
因為很多地產商都懂得做.
每個人都捐地.
所以就不用說明日大嶼.
這是政治現實.
多於良心發現.
但無論如何.
我相信教會或基督徒繼續做這些倡導.
都是一件好事.
如果教會有一天.
出現一個16歲的弟兄或16歲的姐妹.
都像那位碎短的女生一樣.
這就是好事.
教會可以出現未來的青少年.
現在大家都明白.
第一代正在上街的青少年.
全部都是重視環保的.
我真的很贊成你剛才所說的.
其實聽到我的問題.
我今天所說的.
我是很想我們當中的領導人.
我也很喜歡我同事所說的.
教會是一個讓我們能夠訓練我們.
在文化新運動的一個地方.
今天我會說.
這件事是怎樣去改變所有的基督徒.
其中一樣就是我們確實要有行動.
因為我們很容易將自己的生命改變.
但對世界一點影響都沒有.
除了我們知道神能夠做的事.
是超越了我們所能夠明白的.
而我們確實想要的生命.
就是我們在公眾裡面去推廣的生命.
所以不只是我們自己的生命.
例如曾經在英國解放黑龍運動的時候.
他們不去吃糖的原因.
是因為他們記得那個來源在哪裡.
和知道解放黑龍是一個多麼重要的事情.
這種行動是會令到整個社會都能夠得到改變.

$^{201}$所以去問到究竟.
穆仔是否應該直接參與政治的事情.
我覺得都是一個困難的問題.
我認為如果問題是一個很明顯的福音的問題.
我覺得是關於受災世界的問題.
對你來說 你應該要講真正的真理.
你不需要一些政黨.
但你需要講福音的真理.
如果你是真實地信實地去教導他們的時候.
我希望你的教會的人會付諸於行動.
在這個世界裡面去做一些我們需要改變的事.
在這個世界裡面.
我們有很多的方法和途徑去做到這件事.
因為我們全部都是在神的世界裡面.
在神的家庭裡面結連在一起.
每一次我聽到的時候.
我都會覺得神很有盼望.
神可以用我們來改變這個世界.
我想我們可以從一個小小的故事去看待這件事.
好幾年前.
我們在緬甸幫一些山區裡面的村民.
其實他們都是流浪的人.
流浪就流浪吧.
哪裡有一個地方他們可以搭建屋.
一起住在那裡生活.
可以種一些東西或者養隻雞,豬.
但是昂山素姬上台之後.
其實之前都已經開始了.
緬甸政府要推動經濟發展的時候.
他要借助外力.
其中一個外力就是中國.
緬甸和雲南接壤.
當時有很多中國大陸的投資者.
不是像聖馬那樣大.
那些是很小的.
但是他可能拿著一兩萬美金.
他已經可以在那裡買一塊很大的地.
即是去到緬甸那邊買一塊很大的地.
其中一條村裡面.
當然政府是想賣的.

$^{241}$因為有錢收.
他們就想趕走那班村民.
而賣了那塊地.
當然我們都是從服務那班村民的角度.
但另一方面.
這些投資者來買地的目的是什麼呢?.
只是兩件事.
第一就是掘地.
掘地就是緬甸出肉.
全部在地下.
他都要斬樹,掘地.
去掘肉出來.
又或者斬樹而平地建工廠.
無論如何都是破壞樹木等各方面.
因為需要的錢不是很多的時候.
我們都去籌到一些錢去幫助.
我們就幫忙賣了那塊地.
爭取贏了.
有時候可能.
當然你在香港你要買一個大嶼山.
我想全香港的基督徒.
拿了那些錢出來都未必可以.
但我想我們要.
我們的視野超出香港.
我們其實是全世界的弟兄姊妹.
我們是全世界的教會.
我們是一個群體去看的時候.
其實可能有些事.
我們在幫其他人的時候.
其他人也會幫我們.
當然剛才兩位都提到.
都是很重要的.
我們很多所謂的倡議.
Advocacy的工作是需要.
無論是在個人.
弟兄姊妹在他們的範圍內.
能否做得到.
甚至可能教會都可以發聲.
對一些環境的議題我們都可以發聲.
可能我們覺得做不到甚麼.

$^{281}$這幾個月教會都經常發聲.
都是這樣.
但是不發聲.
在上帝.
我們看聖經就知道.
不發聲都是一個罪.
有時候消極一點.
或者靈活一點去說.
我發聲了.
我以後就跟上帝交代.
不過我想.
因為牽涉到很大的一個經濟.
政治各方面的問題.
是很複雜的時候.
我們可以做的.
我們可以影響的.
未必是一下子.
我今天做.
或者今天說.
明天就會出現甚麼轉變或結果.
但是我們不做的時候.
我們想想.
結果就是我們想像到的.
我們不想出現的結果.
有沒有弟兄姊妹.
或者弟兄姊妹都可以問.
如果心目中想某一位講者回答.
都可以說清楚.
我自己本人從事環保二十多年.
過去十年.
我希望很積極.
推動基督徒參與環保.
但實際上.
比我三十年前做環保.
我的失落和失望是更大的.
因為發覺.
你向基督徒說環保的時候.
沒有人會反對這件事.
全人類都認可贊同.
教會給你的答案.

$^{321}$是說他們的議題很多.
我們很忙.
他們有些會告訴你.
我們有做過環保.
不過是一個項目.
例如我們收過膠瓶.
做過這些.
但我想既然我們.
大家都認同.
創造關心是很重要.
我們是否可以有策略去推動呢.
我過去的做法.
我比較多是做底層工作.
例如在教會開講座.
鼓勵他們成立環保小組.
但我發覺.
過去有些經驗告訴我.
這個工作原來是要上下.
所以我希望.
可否討論下有甚麼策略.
在教會裡面形成上層.
因為上層不改.
我們只可以做一些項目.
即底下的項目.
間中做一下這件事.
例如一年做一次講座主題.
但這件事是否足夠回應.
上帝告訴我們.
每個基督徒.
尤其是基督徒.
我們是需要關心環境的.
另外我有第二個問題.
當我推動環境保護的時候.
我發覺有兩個名詞.
都變成對立了.
環保和所謂的創造關心.
中文譯本.
關愛受造世界.
如果是基督徒.
就必須講關愛受造世界.

$^{361}$但一講關愛受造世界.
其實沒有實際的實踐.
行動模式.
可以展示或教導別人.
但當你提到環保.
別人會覺得很低落.
甚至環保裡面.
現在有些風吹得很妖艷.
鼓勵環保人士.
去崇拜月亮和太陽.
去一些最原始的方法.
但兩者之間.
我想我們基督徒或基督教.
都需要清晰定位.
不要妖魔化環保.
因為不是所有環保的工作或團體.
它可以是我們的夥伴.
但如何可以利用他們的豐富經驗.
然後可以跟我們合作得更好.
我想這兩個議題.
都需要我們思考.
謝謝.
哪位講者有興趣回應?.
我回答了剛才那位姐妹的第一件事.
所以我很認同.
姐妹剛才所說.
所以這個就只是.
在下午Jonathan的講座裡.
很強調.
牧者在宣講教導.
整傳福音和使命.
這件事是非常非常重要.
因為今天我確實說.
很多的教牧.
當他對福音的理解.
只是領人歸主.
教會人數增長.
他就是一個很靈性的.
一個Gospel.
一個能夠令到.

$^{401}$Church Marketing.
以至可以經營一間.
所謂Successful Church.
這種思想.
我想他一定不會將這個.
無論你叫他Creation Care也好.
你叫他環保也好.
放在他的議題裡.
所以.
用回我們當中的黃福二長老也在.
他也有提到.
第一件我們要的就是悔改.
我們所有做教會的領袖.
我們都要認罪.
悔改.
為了我們在宣講上.
在教導上長期.
對上帝所創造的世界.
那個的虧欠.
我都要悔改.
(音樂).
說到要將我們的創造.
回歸到上帝的創造裡.
的時候.
你可不可以.
去建議我們.
怎樣去生活在這個.
更新了的創造裡面呢.
第二個問題就是.
我之前是一個宣教士.
我在海外基督使團工作.
老實說都不明白.
為什麼關愛受造世界.
很多人都問.
為什麼關愛受造世界和宣教有關呢.
我們又.
不是環保環保.
我們怎樣去跟他說.
其實關愛受造世界和宣教有關.
和怎樣去做呢.

$^{441}$我會在第二點開始.
其實這個更加重要.
就是要看回福音的內容.
和去擁抱這個地球.
和去看回.
這個宣教的內容.
其實這個更加重要.
就是要看回福音的內容.
和去擁抱這個地球.
和去看回.
這個宣教的內容.
和去看回.
這個宣教的內容.
和去看回.
這個宣教的內容.
和去看回.
這個宣教的內容.
和去看回.
這個宣教的內容.
和去看回.
這個宣教的內容.
和去看回.
這個宣教的內容.
和去看回.
就是所有的東西.
都會與基督復和.
就是要將這個.
擁抱這個東西.
也都體現這個東西.
當這個教會的生活.
能夠體現這個東西的時候.
我們可以看看什麼會發生.
我會出席一些這樣的講座.
因為我要提醒自己.
和其他.
就是福音的性質.
就是一個完整的福音.
如果一個更新的創造.
就是我們和上帝.
和其他的受造物.

$^{481}$去復和的時候.
我們就要在現在.
去顯示這一種的復和.
其實科學也都會告訴我們.
究竟這個有什麼含義.
在今晚.
我也會嘗試去.
去講一些.
實際上我們可以怎樣去體現.
也都怎樣去活出.
今天我所講的.
怎樣去在這個更新的創造裡面.
回歸去這一件事.
也都可以和一些不同的機構去談一下.
如果對於第二個問題.
就是說.
有一個人.
他想申請.
成為宣教士.
他想申請.
成為宣教士.
他想申請.
成為宣教士.
然後我們和他講.
關於.
受造世界的時候.
而他挑戰關於受造世界.
關於宣教什麼事.
我們怎樣去說服他.
其實很簡單.
我相信我只要回答陳牧師.
他一定不會申請那些警察會.
第一件事.
因為他都知道警察會是不做這些的.
他應該現在在世界.
國際.
他會去羅查.
或者其他那些.
或者去斯特.
他都會派一些同工在海外做.

$^{521}$他會選擇一些合適的計劃.
合適的機構裡面去做.
我一定是不會叫.
叫那個宣教士候選人.
去那個機構裡面.
舉例說宣導警察會.
我也很熟悉宣導警察會.
他是關注傳福音的.
我當然叫他不要申請.
你OMF做嗎.
那OMF吧 都一樣.
都一樣的.
所以這個就很清楚了.
什麼教目就去什麼教會.
什麼宣教士就去什麼機構.
不要錯誤就行.
所以我又不會期望.
叫那些機構改變.
改變原來的宗旨.
這個不是我要做的.
這個就很簡單.
就回答了這個問題.
因為時間關係.
我想最後一個問題.
請.
請到前面.
最後一個問題.
我不是來問題的.
不過我想回應.
牧師和很多朋友.
我覺得我們基督徒.
我們的用字.
是非常窄.
我們只會說Creation Care.
就不會將它.
改變很多東西.
我最近和一群.
不是基督徒的朋友.
大家Mingle得很好.
他們就不是用.

$^{561}$什麼Creation Care.
或者什麼的.
他們就很簡單.
很落地地說.
我就是想幫一些人.
就是很想他們.
能夠從土地.
可以聯繫.
可以和大自然.
可以聯繫.
在這個過程中.
人可以被發展.
因為這樣.
整個Community都可以被發展.
我覺得這件事.
我們基督徒為什麼不可以做.
為什麼我們不可以.
去想一想.
我們只不過是將.
人和大自然和土地.
整個上帝的創造.
裡面的動物和靜物.
一起來去連結.
其實我們不只是.
用一些人.
我現在來和你講.
三福音就算了.
我們真的讓他們.
去到整個大自然裡面.
的發展裡面.
去看到上帝的創造的奇妙.
今天我看到佛教朋友可以做到.
其他沒有宗教的朋友.
為什麼我們基督徒.
這麼熟悉上帝.
這麼熟悉聖靈的時候.
我們是做不到的.
而且我覺得最重要的是.
我和你去講福音的時候.
就好像最早期最初期的.

$^{601}$宣教士來到.
我們中國的地方.
他是教你養豬.
教你農牧.
然後就因為這樣.
傳福音給社群.
社群就順主.
今天我們看到雲南很多部落.
他們是這樣順主.
今天我們看到台灣很多部落.
都是這樣順主.
思想可以這麼狹窄.
還有我們是不是一定要靠機構去做.
我認識有些朋友.
他不是靠機構.
他們可能是一兩個人很有心.
這樣去做.
今天我們看到佛教裡面的朋友.
他不是說靠他們有佛教機構去做.
他們就是一至兩個人很有心.
這樣就開始了.
開始一個計劃.
他們是小而美.
這樣就做了.
為什麼今天我們基督徒.
都要講什麼大機構.
大差會.
這樣去做呢.
這件事我真的覺得很奇怪.
不好意思.
我真的覺得我沒辦法.
可以想像.
為什麼人可以這麼狹窄.
你說在實際的生活層面.
其實你每一樣東西.
你只要注意一點.
你去問你自己.
其實今天的資訊是爆炸.
我們如果在網絡那裡.
你可以找很多的資料.

$^{641}$是關於什麼叫環保.
生活環保.
問題是我們會不會用一點點的時間.
去真的去學習.
多謝.
多謝姊妹的分享.
就是說其實很多很有心的弟兄姊妹.
已經自己be water這樣做了.
或者最後幾位講者.
會不會有最後的一個回應.
特別是剛才都再提到.
另外一位姊妹問的問題.
就是和一些外面的非基督教團體.
等等的一些collaboration.
一些合作等等.
會不會最後大家都有一點點想回應.
我贊同最後這位姊妹所講.
其實不需要機構.
大家都命運.
現在每個人都這樣講.
命運自主.
每一個人都可以自主自發.
去創造一些空間.
當我們的政府disconnect的時候.
我們就reconnect.
好.
我想講一樣東西.
就是我們的生活裡面.
是可以有一些行動的.
我之前在大學的時候.
有一幫不同的人.
就來講我們怎樣去關愛這個世界.
當時就有一個去幫更新資源.
美國的一個不是信徒的科學家.
他站起來就說.
我聽上來很好.
不過我看不到會造成.
基督徒是根本我做所有的事.
最大的阻礙的東西.
我覺得這是一個很大的挑戰.

$^{681}$就是如果我們活出來是做不到的.
所以我覺得你講的所有的東西.
都是沒有任何意義的.
所以我多謝大家去講.
究竟其實在我們教會的生命.
和我們自己的生命裡面.
其實是做出來怎樣.
非常豐富的一個訊息.
創造萬有的主.
亦是創造我們生命的神.
來到你面前的時候.
在這個時候.
我們相信你所創造的一切的受造之物.
包括我們.
都是一同探惜.
香港在探惜.
很多人的心靈都在探惜.
我們為了今天.
我們被各樣的罪惡.
污染整個城市.
整個地圖.
特別是為了我們當中一群的教會領袖.
來到你面前的時候.
我們需要認罪悔改.
我們為了我們在宣講上.
教導上.
我們很多時候忽略了你重要的真理.
而是選擇那些對我們有利益關係.
對於我們個人有好處的.
我們來宣講.
主是求你赦免我們.
主幫助我們.
今天我們承認基督徒.
很多時候我們說愛這個世界.
但很多時候我們的愛.
是給很多其他信仰人士.
給那些不信主的人.
我們對這個世界的愛.
我們是更加的慚愧.
求你赦免我們.

$^{721}$因為很多時候.
我們用我們的那些無知.
我們的方式.
來實踐我們的使命.
但有些時候我們所造成的.
是帶來對別人的傷害.
對大地的傷害.
對環境的污染.
主是求你赦免我們.
主就幫助我們.
我們承認我們在一個勞苦嘆息的階段裡.
但我們感謝你.
因為你仍然今天給香港人.
有海,有山,有郊野公園.
有藍天,有白雲.
是一切受造之物.
以致我們為這一切的臨在.
我們獻上感恩.
因為當我們看到這些受造之物的時候.
我們知道上帝你看顧他們.
你仍然是看顧我們.
以致我們看到這些大地.
仍然去讓日著生命.
仍然去讓日著喜樂的時候.
我們的心都同樣可以在裡面.
有這種復活的盼望.
這種的喜樂.
主就求你不斷去更新我們的生命.
不斷去更新整個受造界.
整個香港的社會.
以致我們在這個痛苦,陣痛之後.
我們就經歷上帝.
你給過我們這個復活所帶來的盼望.
求你更新地圖.
求你更新香港整個的社會.
求你更新我們每一個人的心靈.
我們一同在你面前.
獻上我們的禱告.
是奉耶穌基督的聖名而求.
阿門.

$^{761}$多謝各位.
(字幕由 Amara.org 社群提供).
\newpage



\section{}
\label{sec:dJ7VfX3ypUE}
\textbf{「關愛受造世界與福音」香港會議「門徒生命與關愛受造世界」: 約拿單 ‧ 穆爾博士}
\newline
\newline
連結: \href{https://youtube.com/watch?v=dJ7VfX3ypUE}{\texttt{ https://youtube.com/watch?v=dJ7VfX3ypUE}} ~~~~ 語音日期: 2021-08-02 
\newline
\newline
\hyperref[sec:EDiGHY_cDLQ]{\small{< < < PREV SERMON < < <}}
~
\hyperref[sec:index]{\small{[返主目錄]}}
~
\hyperref[sec:U2MibYFulYg]{\small{> > > NEXT SERMON > > >}}
\newline
\newline
$^{1}$(廣東話).
這個聚會其實是延續在2011年.
在開普敦舉行的第三次洛桑運動大會的議帳.
這個會議的目的是要呼籲香港的教會和我們的信徒.
去關心我們的受造世界.
一起反思和實踐如何關愛受造的世界.
我們這個會議能夠舉辦.
多得有很多主內的弟兄姊妹一起去幫助.
除了剛才提及的那四個主辦單位之外.
我們另外還有11個本地和國際的機構一起去協辦.
也都是除了這11個協辦的機構之外.
還有很多在這些機構以外的.
很關心這個課題的無論是弟兄姊妹義工.
或者是一些機構.
在當中有很多的參與和努力.
才能夠令到這個會議能夠進行.
所以我們在這裡多謝他們的參與和幫助.
我們今天的大會的講員是Dr. Jonathan Moo.
他是美國維特沃斯大學的新藥及環境研究副教授.
Dr. Moo的研究的興趣和專長是在新藥的研究.
特別是早期的猶太教,啟示錄,天啟文學.
以至到科學和信仰之間的關係.
還有是生態和環境研究.
所以由他來說今天的課題是最適合不過的.
Dr. Moo的著作都非常豐厚.
雖然你稍後會看到他是很年輕的一位學者.
但是他的著作是非常豐富的.
2011年的時候他已經有一本著作.
就是以色列四書裡面的創造,自然和盼望.
2014年有另一本著作叫做As Long As the Earth Endures.
2014年也有另一本著作叫做Let Creation Rejoice.
最近在2018年的時候.
他和他同樣是新藥學者的爸爸Douglas Moo.
一起合著的一部著作叫做Creation Care.
A Biblical Theology of the Natural World.
這本著作在我們的書攤裡面都可以購買.
大家如果有關注這個課題.
很值得去看看這本著作.
Dr. Moo不單止是學者.
剛才我們一起和他吃午餐的時候.

$^{41}$他告訴我他每隔年.
他會和他的學生去到山區的地方.
有一個月的時間在山裡面.
做一些生態和信仰的反思,研究.
在裡面一起思考.
所以今天這個課題.
他講的訊息是門徒生命與關愛受造世界.
這個肯定不單止是一個學術,研究,思考這麼簡單.
還是Dr. Moo他自己生命的一個呈現.
我們非常期待一會兒去聆聽他的訊息.
當Dr. Moo講完他今天的主題訊息之後.
我們有兩位嘉賓回應當中的訊息.
分別是香港教會更新運動總幹事胡志偉牧師.
和斯達基金會總幹事鄺惠民博士作出回應.
我們在回應以後會有一些Q and A的時間.
在當中很歡迎等一會兒.
可以將你聽訊息的時候.
你有的問題或者你自己有一些反思.
都可以跟大家一起分享.
大會是設有即時傳譯的.
如果你在需要即時傳譯的時候.
應該你已經知道我們有電郵告訴你.
是怎樣去下載那個app等等.
倘若你有任何的需要要去支援的話.
請你去到禮堂後面.
我們有同事有同工會幫助大家去解決這個即時傳譯的問題.
我想事不宜遲.
請大家加入我們.
請給我們一場熱烈的歡迎.
我們的主持人Dr. Jonathan Moo.
(掌聲).
非常感謝.
(英文).
在這個會議裡跟大家一起.
在這個美麗的前線有極大的優待.
我希望能夠去認識大家.
聽到你們的故事.
也知道多一些你們在香港和世界各地的服事.
基督的志工和經驗.
縱使我們在不同的處境.

$^{81}$無論在這裡和我和華盛頓州的路加斯坡是多遠的距離.
我也極度感到鼓舞.
為的是因為我們作為在基督裡的姊妹和弟兄.
我們共同分享福音的程度.
和一位上帝的兒女分享同一個創造的因子.
在基督裡同為後子.
就跟著同一個福音.
和同一個聖經的引導.
所以我滿懷信心向上帝禱告.
上帝會施恩.
今天對我們所有人說話.
去挑戰和鼓勵甚至激發我們.
就是當我們探索聖經.
如何展示福音並思考這個揭示.
對我們生命和事工的觀念.
這次是我第一次到訪香港.
我的兄弟曾經在深圳住過好幾年.
他告訴我他愛上了香港.
但可惜當我在這邊的時候我沒有機會去探望他.
所以我很高興能夠來到這裡.
我可以告訴你.
我和我的教會和很多其他人.
都一直為你們在香港所有人祈禱.
在最近的月份都經常這樣做.
香港是經常吸引我的注意的地方.
部分原因是它獨特的歷史和文化.
也因為我在所有的照片裡看到.
這個偉大的城市高聳入雲的高樓大廈.
被茂密的山緣和藍色的深海所包圍.
在今天我們一起的下午和晚上.
我希望打個比喻來說.
是將我們的目光指向群山.
就好像CP104呼喚我們一樣.
我希望我們之後大而廣闊的海.
擠滿了無數的生物.
我希望我們再一次看到.
我們身旁非凡奇妙的創造.
而我們的城市和所有人類的活動.
都是受盡世界的一部分.
最重要的是.

$^{121}$我希望我們再一次.
被上帝的創造和救贖工作的宏偉規模所迷住.
我們才能夠再一次以適當的角度.
去看我們自己的生命.
並且去問我們對於這個受造世界.
和其他的部分有什麼責任.
那到底是什麼呢.
就是當我們作為獨特.
按照上帝的形象所造的受造物.
去關愛上帝的世界.
同時也作為正是按照我們救主耶穌的形象.
被更新被呼召去使萬民作門徒的人.
我今天的論點.
直接的說就是.
如果要忠於福音.
我們必須好好地關愛上帝的創造.
這個不是一個選項.
不是在我們被呼召去做.
其他有價值的事以外.
在額外加上的計劃.
也不是一個聰明的報道策略.
關愛上帝的創造.
是我們被呼召作為上帝指紋的重要部分.
也是我們今天如何活出福音的重要部分.
是一種生活的方式.
是需要滲透我們所有行動和宣講.
我們時代最迫切的需要.
可能就是讓世界各地的基督徒.
頂身而出.
去回應環境的危害和氣候變化.
一起體現上帝的愛和公義.
就是透過好好地去關愛.
人和其餘的受造世界.
主要有兩個原因.
我認為這是我們今天呼召的重要部分.
第一個原因.
我們現在生活在世界歷史上.
一個獨特的時間.
一般來說.
我們會合理地懷疑.

$^{161}$任何聲稱我們的時代是獨特的說法.
這種聲稱傾向反映我們誇大了.
自己世代的重要性.
或者對歷史的無知.
但是在今天.
只不過是因為對歷史的無知.
才會阻礙我們看到當今世代的獨特性.
人類一直都對世界產生影響.
也面對環境的挑戰.
但是這些影響的規模.
和這些挑戰的全球性.
到了今天已經是無可比擬.
在今晚的公開講座.
我會用一些時間去看看這個聲稱.
看看科學對於我們世界的本質.
有什麼啟示.
並且我們會就地球生態的健康.
去簡單審視我們所知道.
希望今晚你們都會在場.
對我來說很沮喪的就是.
重要的環境思想家.
很多時候都認為基督徒.
都是那些很快去圍繞他們的信念.
就是說上帝創造世界.
但是徹底都沒有能夠按著這個信念去生活的人.
有少數了不起的基督徒群體.
就好像今天發起的會議那些.
他們已經顯示出基督徒能夠做些什麼.
就是當他們嘗試去生活.
愛上帝也都愛上帝的創造.
並且愛他全球的淪陷.
就好像愛自己一樣.
但是很多人都對基督徒壓倒性的印象就是.
我們對時代最重要的道德意義.
但是沒有東西可以貢獻.
就是說我們怎樣可以讓.
可能有100億人在這個世紀的中期.
不夠支撐生命就興旺.
同時間確保我們的星球健康.
還有能力去繼續支持生命.

$^{201}$特別就是在我的國家美國.
基督徒都經常被視為阻擋環境危害.
和在後面發主要力量.
但是我們不是這樣的.
基督徒早就應該.
在關愛上帝的地球方面就在前頭.
我們應該體現.
更加廣泛顯示基督徒面對環境議題的獨特進步.
這是特別需要的.
這對關乎我為什麼認為關愛受造世界.
是今天教會迫切和重要的使命的第二個原因.
就是說關愛受造世界是福音的完整組成部分.
今天我們會再探索有關創造涵蓋面.
更加廣闊的聖經神學.
並且考慮如何呼召所有上帝的子民去參與其中.
但是現在我希望我們首先集中仔細地看新約.
特別是耶穌和保羅有關福音的啟示.
它對於受造世界的涵蓋.
以及對我們作為基督徒的領袖.
教師 牧者和宣教士的關聯.
我們是福音人.
我們對於受造世界的關愛.
必須是直根於我們對於福音的維新.
直根於我們叫萬國成為門徒的使命.
直根於我們去愛上帝和跟隨耶穌的呼召.
在第一世紀念.
保羅寫信給在菲臘比跟隨基督的人.
他們當中有些人是有理由.
為著他們羅馬公民的身份而驕傲.
當時保羅將菲臘比人指向一個更加極端.
不同和重要的多個公民身份.
目光不是望向羅馬皇帝.
而是仰望在天堂.
加冕作為真正的主和皇的基督.
在菲臘比書一章27節.
保羅呼籲菲臘比人.
要活得像和基督的福音相稱的公民.
現在如果我們要活得像和基督的福音相稱的公民.
我們最好確實掌握到基督福音是什麼意思.
今天基督教會很多人.

$^{241}$未能夠認真去看待.
對於創造貧窮和弱勢的人的責任.
是可以追溯於我們未能夠掌握.
基督福音故事的重要性已經完全不豐富.
其實基督他自己向我們顯示福音的本質.
根據路加對耶穌的心靈的記載.
在一個安息日.
就是在時宮的開頭.
耶穌走上了家鄉亞撒勒的寶堂.
會堂.
他打開了耳塞和書的卷軸.
並且連給聚會的敬拜群眾聽.
在路加福音四章18節.
有一節耶穌是這樣唸出耳塞和書.
主的令在我身上.
因為祂用高高我.
叫我傳福音給貧窮的人.
差遣我報告被擄得釋放.
核眼的得看見.
叫過一個受壓制的得自由.
報告神越立人的欺憐.
於是耶穌將書卷起來.
交給執事就坐下.
會堂裡的人都定睛看著他.
耶穌對他們說.
今天已經應驗了你們已中了.
福音是什麼呢.
就是以塞亞所宣告.
並且耶穌所升在正聖就的福音.
這就是神越立人的欺憐.
神指定他帶來.
醫治自由和恢復的時間.
藉著聖靈的力量.
耶穌即是被告那位.
彌賽亞.
帶來以塞預言的時代.
就是上帝會行動喜量.
去拯救他的子民.
就是他的國度開始在地上實現的時候.
上帝的統治是有一些標記.

$^{281}$就是靈性和身體上的醫症.
是從被壓迫中拯救和恢復.
如果我們將《預創的書》從頭看到尾.
這些標記都會包括整個更新的創作.
《路加福音》的其餘部分.
都顯示這些是怎樣已經透過.
耶穌和他在地上的事供被見證和表明出來.
我們從這些事件中看到.
一個男人萎縮的手恢復.
被綑綁而得釋放.
就是當耶穌的一個身體歪曲的女人.
變回直和很多醫治的個案.
和那些原來被排拒的被融入社群.
包括了麻風病人.
被邪靈附身和還血流的女人.
就好像以塞所預見的.
上帝打算在基督裡面回應人類.
因為遠離上帝所帶來的破碎.
以及所引致在身體 社會 靈性和宇宙層面的後果.
在《路加福音》第七章.
施洗約翰的門徒去到耶穌那裡.
他回應耶穌是否真的將來要來那裡.
耶穌的回應指向他在事供裡面.
能夠看到的有關上帝拯救的使命的標記.
耶穌這樣說.
你們去將所看到的.
所聽見的告訴約翰.
就是「乞子看見,騎子行走,.
掌大麻風的潔淨,.
龍子聽見,死因復活,.
窮人有福音傳給他們,.
凡是不因我跌倒的就有福了.」.
上帝國度的記號.
是在耶穌的說話和行動中顯而易見.
這些向貧窮人宣告的好消息.
就是上帝已經來幫助他的子民.
但要留意.
在耶穌回應中所引含的警告.
就是提示到有些人會因他而跌倒.
畢竟事實約翰在監獄裡.

$^{321}$如何和耶穌正帶來上帝國度的事實回覆.
這是拯救和恢復的時間.
要得到答案.
路加的讀者要等到書卷的尾聲.
在那裡我們了解到耶穌的使命.
不僅是醫治和恢復.
耶穌剛好在他遊走在加里利和耶路撒冷期間.
遇到一些人.
耶穌的使命.
是實現上帝一個極端的計劃.
就是要將所有人從他們的罪拯救出來.
以至於寬恕,復和,恢復新的生命.
和完整的根深蒂固的創造成為可能.
最後這些都是十字架上的成就.
在那裡.
耶穌無辜的人.
上帝公義的兒子.
被折磨並被釘死.
就像保羅將會更加全面地展示.
在十字架上.
世間的邪惡被決定性地處理.
在這裡.
上帝在基督裡.
將世界的罪都放在自己身上.
在這裡不公義的力量被擊敗.
在這裡死亡被承誤過.
十字架並不是終結.
正如彼得後來對他的同胞.
以色列人宣告在《聖堂傳》4章.
彼得說.
「你所釘十字架的那撒尼耶穌.
神叫祂從死裡復活.
除了祂之外.
並無拯救因為天下人家.
每次下別的名.
我們可以得罪得救」.
在復活之後.
耶穌向那些感到憤怒的門徒解釋.
上帝一直以來的計劃.
如果他們正確地理解聖經.

$^{361}$他們就會明白.
基督是必須受害.
在第三天從死裡復活.
並且人要奉祂的名.
要傳悔改.
赦罪的道.
從耶路撒冷直到傳到萬邦.
基於耶穌對自己對於好消息的宣告和體驗.
我們不怪得知.
在新約中.
福音就是指向上帝.
透過基督所行出來的.
和持續地向全世界宣告他們的福音.
例如在《關林多前書15章》.
保羅總括記福音就是.
基督是照聖經所說.
為我們的罪死了.
並且埋葬.
又照聖經所說的第三天復活.
並且獻給基督.
然後獻給十二使徒.
這是一個很簡單的總括.
但我們看到的就是.
這個好消息的故事.
高潮是在耶穌的死埋葬和復活.
福音就是來塞耶穌的故事.
保羅強調這是照聖經所說的.
就是說這個好消息.
是透過先知在聖經中所應許.
保羅的聖經當然就是我們稱為舊約.
根據保羅和耶穌自己.
舊約已經應許基督會來.
對保羅來說.
耶穌的故事就是福音.
但是我們無法全面明白福音.
除非我了解它是一個更廣闊的故事部分.
一個從創世紀開始的創造故事.
假若任何人要全面明白.
耶穌和福音的重要性.
他們需要知道這個更廣闊故事的一些東西.

$^{401}$因此每當保羅向那些完全不知道.
希伯來聖經中的聖經的人說話的時候.
就是那些沒有聽過耶穌上帝的人.
保羅總是覺得要補充這個故事.
在《路加記載》中.
保羅對雅典人說話中.
在《舍堂傳》第17章中.
我們會發現保羅是如何開始宣告福音.
強調上帝是宇宙獨一的創造主.
也是所有被造物的主權的統治者.
保羅不希望他的聽眾誤會上帝是誰.
他不是另一個本土的神明.
可以在拜祭其他神明以外再多拜一個神.
也不是在雅典中保羅看到他周圍的其他神明.
意思是上帝是那位創造一切的存有.
他是管治全創造.
是存在超越的.
也就是不存在任何受俗物那裡.
保羅說上帝同時也是靠近每一個人.
也能夠讓所有人知道.
上帝是內在的.
和他的創造同在.
也是參與其中.
對保羅來說.
這個首要就是在耶穌的身上彰顯出來.
他就是上帝的子子.
進入創造中.
啟示上帝的作為.
創造主和救贖主的屬性.
按照這樣啟示出來的上帝.
保羅呼籲他的聽眾要回應.
所有人都是被呼召去崇拜偶像中回轉.
悔改歸向上帝.
創造萬有同時審判萬有.
根據保羅.
耶穌這個人物就是復活的主.
透過上帝的公義和正義得到最終的確立.
盧加對於保羅講話的記錄.
顯然是簡短和具選擇性.
雖然如此.

$^{441}$我們得到一些提示.
就是當保羅向猶太人同胞.
和猶太人基督徒講話時.
他能夠假設他們知道一些東西.
但當他和那些知道很少.
或完全不認識聖經中上帝的人講話時.
他需要更加全面地說明.
這個提醒我們.
特別在我們今天的處境.
我們服事的人通常都不認識聖經中的上帝.
耶穌的故事需要成為整體聖經故事的一部分.
被講述出來.
而這個故事就是以創造的上帝開始.
在新約的另一次我們見到保羅.
不認識聖經中上帝的人講道.
就在《舌痰主》14章.
那時保羅和巴拿巴在四處調查路斯德人.
他們看到一個騎腿的人得到醫治.
感到驚奇.
他們開始當保羅和巴拿巴是神明.
這樣去歡呼.
保羅和巴拿巴就撕開衣裳挑針送神.
中間強調他們都是人.
並且宣告我們傳福音給你們.
是要叫你們離棄這些虛妄.
歸向創造天地海和其中萬物的永生神.
這裡的宣告令人驚奇的是.
保羅和巴拿巴不僅強調上帝是創造主的開始.
更加是假設路斯德人已經得到一些.
關於上帝的見證.
因為上帝「未嘗不起出證據來.
就如常思恩惠從天降如賞賜豐年.
叫你們飲食飽足滿心喜樂」.
有時被稱為上帝普及的恩典.
他管治和受到世界所帶來的祝福.
以至喜樂的經歷都是從上帝而來.
這些全部都是成為上帝啟示自己的方法的一部分.
就是向未來的民福音宣告的人.
當我們今天考慮基督徒使命的時候.
值得考慮的除了是照顧和佛人.

$^{481}$和宣告福音之外.
同時我們也要考慮到.
特別是我們對於上帝的受到世界的態度.
以及我們對於創造主和救贖主的讚美.
是能夠邀請其他人進入聖經提出的現實.
並且是因著上帝的恩典去認識和敬拜基督.
我會感到悲傷.
就是每當我想起我們怎樣.
每不在乎地去對待上帝美好的創造.
想到我們粗心地在地球上生活.
想到我們減少生命的豐富和多樣性.
這些一切不單止叫我們成為不忠的管家.
更加是難阻我們成為上帝在基督裡面榮耀的見證.
我們究竟是怎樣去難阻其他人認識.
這個受到世界對於上帝的見證呢?.
當我們未能好好地去關愛.
受造的世界行公義去確保一切都能夠分享地球的豐富呢?.
古羅宣說福音的中心是耶穌基督的王權.
祂掌管一切受造物和萬物的福報.
這個對於整個世界來說都是一個好消息.
因為祂決定了神怎樣命令所有受造物要在基督裡面成全.
這個是萬物都能夠得到更新並且和他們的創造主復和的方法.
福音是關於神在耶穌身上所承載的故事和宣告.
祂要求所有聽到這個好消息的人以信心作回應.
福音邀請所有人成為神的兒女這個群體.
透過耶穌基督與神復和並且以祂為王.
這個福音也會驅使我們來做行動.
在《多倫多後書》裡面.
保羅稱這個行動為「宣認基督,信服他的福音」.
對保羅來說,效忠耶穌基督為王是很重要的信心.
也要信服.
當福音在第一世紀被宣講的時候.
他是很挑戰當時羅馬帝國所宣傳的所謂平安的福音.
這種平安是透過戰爭和暴力所得到.
帶來的並不是聖經裡面的神.
而是羅馬帝國的君王.
這個君王甚至有時被稱為神的兒子.
但如果耶穌是全力的主.
沒有任何其他人或事物可以稱為主.
而今天福音仍然迫使我們去問.

$^{521}$究竟我們真正的效忠對象是誰.
福音迫使我們去問.
在哪裡找我們的救贖和身份.
如果我們尊稱耶穌基督為獨一的主.
福音容不下我們去同時間追求我們的社會和文化所著將.
福音排除了將我們的身份,自我價值和安全.
置於金錢,消費主義,科技,民族主義,經濟增長或其他任何方面之上.
既然只有上帝才能拯救基督教的福音.
福音就挑戰了人類在世地球的救星或萬物的量度.
或最終一切都取決於我們這種種的觀念.
因為在中末的時候.
正如起初的時候.
福音告訴我們一切都取決於基督裡面的神.
保羅在《羅馬書》第八章中為我們勾畫了福音的宇宙輪廓.
在這段經文裡他讓我們看到圍繞著我們整個的創造.
讓我讀《羅馬書》第八章19至23節.
受造之物切忘等候神的眾子顯出來.
因為受造之物伏在虛空之下.
不是自己願意乃是因乃叫他如此的.
但受造之物仍然指望脫離敗壞的痕跡.
得享神兒女自由的榮耀.
我們知道一切受造之物一同嘆息.
勞苦直到如今.
不但如此.
就算我們這有聖靈初見過之的.
也是自己心裡嘆息.
等候得著兒子的名分.
乃是我們的身體得贖.
在現今保羅說.
受造的世界屈服在虛空之下.
受敗壞的痕跡.
好像他之前的舊約先知一樣.
保羅所望見的是整個受造世界.
於人的罪之下一同嘆息.
神叫受造世界屈服在人之下.
讓人管治這個世界.
而當人拒絕神.
當我們因為對偶像的崇拜而變得虛空的時候.
受造世界就屈服在虛空和徒勞之下.
如今受造世界在嘆息中渴慕得著.

$^{561}$神原本給予他的目的.
遠遠在保羅之前先知.
好像以賽亞 何西亞.
都曾經用地的嘆息這種形容.
他們從城市的滅亡 土地的毀滅.
乾旱 農作物的貼收.
去看到基本的宇宙失調的跡象.
何夢耳問的是.
保羅在他的時代也將天然災害.
視為所有受造物繼續遭受痛苦和嘆息的證據.
羅馬帝國的一世紀居民.
和其他任何時代的人一樣遭受天然災害.
他們知道很多其中的災害.
是因為他們沒有辦法控制的力量造成的.
但也很清楚有其他因素.
顯然和人類對當地環境的過度開發有關.
但當我們今天去看羅馬書第八章的時候.
我們不禁看到我們的生態危機中.
反映了受造之物的嘆息.
如果你覺得聖英所描述的人類.
在受造世界的角色.
這幅圖畫好像很幼稚.
竟然將這麼深遠重大的責任.
交給地球區區一個品種.
在人類對地球這麼廣泛影響的時代.
這件事已經不再幼稚了.
甚至有科學家形容成人類世代.
或者人類時代.
正如我所說.
我們從來沒有這麼認真的對待過.
我們在受造世界之中的責任.
而如果我們不能夠履行這個責任.
所帶來的後果.
也從來沒有現在這麼嚴重.
福音所帶來的好消息.
就是儘管我們不信實.
但神仍然是信實的.
因此保羅可以在羅馬書第八章中說.
叫受造之物延延子望.
神的目的是所有的受造之物.

$^{601}$不單止應該分享人類墮落的結果.
也會分享我們的救贖.
保羅說受造之物.
有一天會從敗壞的核心下得到釋放.
得享神而榮耀的自由.
在基督裡面.
就是在新的亞當和神的真理形象裡面.
受造世界在未來得到了保障.
保羅沒有給我們這些自由的全部細節.
他只是說這個自由.
是會消除目前受造世界的套路.
它將會達到神一直打算的目的.
它會將從敗壞的核心下得到釋放.
因為神的兒女終於能夠在基督裡面.
做回他們被造時神給他們的角色.
這個時代裡面的生命被破碎所標誌.
這些破碎包括人和神的關係破碎.
人和其他所有受造物的破碎.
在基督裡面都能夠得到醫治.
在基督裡面神的平安能夠得到保障.
神期望我們有平安完整繁榮.
在其他聖經的經文中可以找到更多類似的圖畫.
這種盼望被描述為萬物的復活或更生.
一個新天新地或這個世界的國.
變成神的國和彌賽亞的國.
在這些經文裡面受造世界光明的未來.
是完全超越人想像.
以至只能夠用符號和隱喻來形容.
但是羅馬書第八章很明確表明.
這個現在正在嘆息的受造世界.
在神的旨意裡面是有未來的.
所以關於聖經的盼望的細節.
無論我們能夠得出什麼其他結論.
我們必須要根據羅馬書第八章來肯定.
這個受造世界這個地球是不會被遺忘.
我們在這裡找到驅動聖經環境觀念的意象.
就是受造世界得到釋放.
這種宇宙的盼望重新了所有受造物.
在神最終目的裡面的價值.
也呼籲那些自稱在基督裡面的人.

$^{641}$開始這樣生活.
我們被召生命要好像成為神兒女的人的生命.
以至能夠反映神這個新創造的優先次序.
保羅在羅馬書第八章裡暗示.
神的兒女正在慢慢等待未來的復活.
我們的身體得救贖.
但他在同一章都肯定.
在第十四到第十七節.
我們所領受的已經是神兒女的名分.
並且已經期待我們的生命反映這個身份.
如果整個受造世界都渴望.
看到我們成為我們原本應該有的身份.
即使是現在.
我們的生命都應該反映出神最終的目的.
就是受造的世界從徒勞和毀滅中得釋放.
神的創造 新的創造在基督裡面已經進入了現今.
也進入了所有在基督裡成為神的兒女的人.
所以我們的工作不是通過自己的努力.
來拯救地球或實現神的國度.
而是被照以神國度的實例去生活.
我們是抵抗一切會破壞神創造的成員.
我們也相信神的恩典能夠成就我們的工作.
以及成就福音所應許的未來.
對於我們今天所有的人是什麼意思呢?.
對於基督徒領袖和福音傳導者又意味著什麼呢?.
具體細節可能隨著我們的背景和呼召有所變化.
聖經不是一本所有答案都能給予我們的手冊.
但它給予一些原則引導我們.
包括對所有受造物在神面前的價值認識.
人類作為受造世界成員的特殊和獨特價值.
是我們所有關係裡反映基督的犧牲之愛的呼召.
我們必須利用神賦予我們的能力去研究受造物.
並做出我們可以照料和人類居民最佳的決定.
而且非常明顯的就是.
我們不可能只關心我們的人類的兄弟姐妹.
而不關心他們和我們所居住的環境裡的情況.
但我們是必須保持警惕.
因為我們都是偏向找一些適合我們的社會和文化觀念的答案.
也不要逃避作為門徒徹底和激進的挑戰.
例如我們必須警惕我們所說的.

$^{681}$人類的福利相對於環境的優先排序.
這種說法尤其是在世界上比較富裕地區使用的說法.
往往是源於一些錯誤的取捨.
而這些取捨是忽略了長期的後果.
或者根本沒有想清楚所謂照常開展業務是怎樣的承諾.
它經常都是代表著一個世界的同化.
在這個世界上金錢和未經審查的發展.
成功,美好生活等等的概念都是不會受質疑的.
簡而言之它代表了一些想像力上的失敗.
我們必須要記住根據正經的見證.
非人類的受造物以其各種奇妙的多樣性.
在神面前具有自己的價值.
因此在任何可能的情況下.
我們都必須努力尋求能夠改善人類和非人類群眾的健康的解決方案.
我們還必須警惕將宣講福音排在照顧窮人和關愛地球之先的說法.
我們如何能夠在不留意福音其實是在說什麼情況下宣揚福音.
正如我們所說的和開普通承諾所表達的.
在宣揚福音時宣告耶穌是主.
宣告包括地球在內的福音.
因為基督的主權是在一切受造之物之上.
因此照料受造之物就是基督主權範圍內的福音議題.
在今晚的演講中我會說一些我們人人都會用的應用.
希望在今次會議上的研討會和其他資源都能夠幫助大家.
總結一下我想說的兩個我覺得能夠幫助我們的原則.
是我和我爸爸一起寫的書中得出的原則.
第一就是將受造之物重新投入更新的創造.
第二就是將我們自己重新投入創造.
第一件事對我們所有人特別是對領導層來說.
第一必要做的做法就是教導和傳揚整本的福音.
無論何時何地我們都需要強調.
耶穌基督的福音如何包含所有的現實.
換言之我們需要將受造之物重新投入新的創造.
其中一個導致教會的創造神學會變弱的原因.
就是對舊約的忽視.
就是未能夠閱讀和教導整本的聖經.
我們需要效法新約使徒的榜樣.
恢復概念整本聖經的聖經教義.
聖經中很多描述創造的出處都可以在舊約中找到.
而其實對所有新約的作者來說.
他們的聖經就是舊約.

$^{721}$舊約所教導的神和創造是他們所設定的背景的一部分.
而他們本身就是這個故事的一部分.
因此不單止是新約我們必須更加全心全意地.
致力於舊約聖經的閱讀,學習和教學.
這是因為各樣的原因.
但其中一點很重要就是當代的教會裡面.
太多的人全港溝探了的福音.
而忽略了舊約所揭示的非人類受造物.
以及我們對它的責任.
正如我們所看到的.
我們只有兩次保留向那些沒有舊約背景知識的人傳福音紀錄.
而這兩次保留都是從創造者和他的創造而開始講福音.
當我們花時間去分享蓋咸所有生命和創造的聖經敘述的時候.
我們經常都會發現在人當中是能夠引起深刻的共鳴.
你們人通常都在尋求一個他們可以成為其中一部分的故事.
一個比他們自己更加大的故事.
一個永恆的故事.
讓我們這個時代最重大的事情有關的故事.
牧者每週最主要的職責.
就是傳講和教導聖經.
施行聖禮.
牧人他們會眾.
而在不斷有新的呼聲要處理新的事情或者事工的時候.
我覺得這些責任都確實是很重要.
視乎你教會或者事工的結構.
如果我們的行動是要合乎我們關愛受造世界的神學.
很可能是在平信圖安排和執行這些行動.
但如果基督徒看不到他們的信仰和關愛世界之間的聯繫.
這件事是不會發生的.
他們必須能夠有想像力在這個嘆息的受造世界之中.
如何想像到完全成為神的兒女生活.
這是一個怎樣的一回事.
如果我們要被動員去做神呼召我們去做的事情.
我們需要牧師需要教師.
在他們迅速地施行聖禮的時候.
提醒我們聖禮和耶穌的道成肉身.
生命死亡復活和再來有關.
以及這些對整個受造世界的意義.
我們需要傳講神完整的道的傳導人和宣教士.
並且告訴我們神的道和我們關愛受造世界和貧窮的人有什麼關係.

$^{761}$我們需要領袖讓我們看到這樣做生命是怎樣的.
教會處於一個獨特的位置去處理地球上生活所面臨的種種地區性和全球性的挑戰.
作為全世界在神裡面的大家庭.
我們可以通過不同的宗教宣教救援和發展機構的網絡.
和全世界的弟兄姊妹結連在一起.
在基督裡面我們因著聖靈結連在一起.
而且靠著神的能力我們能夠完成超越我們所想的事情.
難道神不可以使用教會.
去新的創作裡面的堡壘來指向他們整個地球的旨意.
並且透過我們彼此之間.
和整個受造世界之間的關係轉變來展示神的榮耀嗎.
這是我對你的祈禱盼望和挑戰.
第二樣我認為對我們所有人來說不可或缺的.
其實嚴格來說並不是要做什麼一件事.
而是對神的創造而感恩喜樂和敬拜的一種姿態.
我們除了要將受造之物重新投入更新的創造.
也要將我們自己重新投入創造.
重新聖經對福音的全面理解.
可以讓我們重新認識世界.
將它視為歸榮耀給神的舞台.
也向讓我們讚美各位四周受造物都見證著的創造者.
我們喜樂地去接受我們作為有限的人類受造物這個角色.
也是整個受造世界的一部分.
並且與其他生命息息相關.
我們被邀請去承認我們對地球和對那位創造.
和維持地球的創造者的依賴.
面對這麼龐大和偉大的創造.
我們可能會因著自己的隱約有限和微小而覺得很謙卑.
我們更能夠為這位跟我們有關係.
在這個廣大的角度給予意義我們的神.
和他無法測度的因典而慶祝.
這個感恩喜樂和敬畢的姿態.
能夠透過在神的創造內花時間而得到幫助.
當然實際上這是我們唯一有花時間的地方.
香港的中心作為神的創造.
不礙於最荒涼最偏遠的山頂.
你是不能夠離開他的創造.
但是在我們這個時代.
很容易假裝我們與受造世界的其他部分是分開的.
假裝我們不依賴地球.

$^{801}$假裝我們已經超越了自然.
眾所周知我們現在可以使用的科技的眾多好處.
也可能帶來很大的麻煩.
我們可以讓科技在我們身上施一些咒語.
讓我們習慣現實.
讓我們對自己的身份和價值覺得憂慮困惑.
並且我們和他人和周圍的世界切斷.
我們的科技可以讓我們忘記我們是誰.
我們在哪裡我們屬於誰.
我覺得要消除這種健忘的對策.
不僅是要花時間在神的創造裡面.
而且也要有目的地思考和積極參與我們周圍的非人類世界.
一切奇觀和多元性.
對我們中某些人而言.
這個有時可能會是在野外的時光出現.
可能是以戲劇性的方式發現我們自己的極限.
以及看見神在創造中.
有時所展示的美麗.
令人感到害怕的美麗和威脅.
但對於其他人而言.
可能只是種植和收集我們自己吃的東西.
可能是種植一個很適合的香草.
從事園藝收集食物植樹和農民漁民生物學.
加在一起去了解我們周圍土地和水域更多的知識.
研究我們的生態.
從無盡的忙碌上退下來.
抵制那些破壞我們的地方.
服侍我們的群體.
有無盡的方法.
可以讓我們更加充分地參與地球的生活.
並且與我們與他人和神的創造之間結力.
挑戰是我們是否願意.
與那些隨意可見種種世界的美麗和破碎日益接觸.
我們需要的是我們對這個歸榮耀給神.
和祂創造的世界.
有著專注和好奇的態度.
呼籲我們回到真實的身份.
也許我們願意在我們的生命中.
打開種種歡樂的可能性和盼望.
讓我們可以邀請他人進入我們的生命.

$^{841}$如果要喚醒我們看到四周非人類受造物的美麗和美好.
我們必須在生活中抽時間脫離科技.
與我們很多時間的生存當中.
經常存在的文化.
定期脫離連線已經成為我們時代必要的屬靈操練.
尤其是那些自稱基督徒領袖和牧者.
這是安息日休息原則的延伸.
與所有的屬靈操練一樣.
由於種種原因這可能是很有挑戰性.
但它太重要不能忽略.
我們單純有需要定期讓各種.
讓我們從此時此刻分心的科技抽離.
科技令我們與我們的屬靈者與地方有距離.
科技令我們不能持續反思聖經.
令我們無法慶祝或親密關懷自己周圍的受造物.
如果耶穌自己有時也獨自有人上山去禱告.
為何我們覺得自己不需要這些時間呢.
波魯格曼他曾經寫過.
慶祝安息日既是抵抗亦是另外的選擇.
這是抵抗 因為這是一種明顯的堅持.
即是我們的生活不受商品的生產和消費所限制.
提供的另一種選擇是意識到和實踐.
我們接收神的恩典.
如果我們為我們所侍奉的人樹立一個這樣的榜樣.
對我們和他們的生活有什麼意義呢.
我們大多數人所需要的.
不僅僅是簡單有什麼方法關愛受造物.
就算那是一個很實用的建議.
我們也不需要一些新的計劃.
加在我們本身已經在做的事情上.
儘管可能會有些幫助.
我們需要的是讓我們再次向山舉目.
再次對福音著迷.
再次愛上這位創造和救贖一切的神.
用聖經的整個故事來重新看我們的世界.
再次與基督的復活相遇.
祂是我們的一切.
如果我們愛神.
我們將愛他所創造的世界.
如果我們愛我們的鱗石.

$^{881}$我們將關愛 大家都有份受造世界.
這個需要艱苦的工作.
照著神的意思去關愛受造世界.
意思就是要犧牲重新整領我們的慾望.
以及更新和改變我們的心靈.
而且我相信你們每個人都已經有很多時間拉扯.
在面對很多事工上的挑戰 沮喪和忙碌之時.
我懷疑我們很難去做多一件事.
而且很容易去委任那些特別熱衷關愛受造世界的人去做這件事.
認為這些只是某些人的呼召.
但讓我們記住使徒保羅.
保羅的呼召是向外邦人傳福音.
為了這樣做 他將整個回轉後的生命都給了這個使命.
然而保羅也因為其他信徒紀念窮人而被鼓勵.
而且保羅是很樂意的.
他說 這是我一向熱心在做的.
如果保羅看到他這麼緊急的宣教任務.
必然是包括了對窮人的照顧.
我們幾乎不能讓我們自己的各種恩賜 呼召和忙碌.
讓我們忽略了我們對窮人的責任.
如果我在這裡提出的論點是正確的.
我們今天必然需要以關愛受造世界的方式.
去愛他人和愛神.
我們絕對不能忽略我們對整個探世中的受造世界的責任.
我的禱告是 我們會看到信實地關愛受造世界.
這是一個呼召 不是重擔 而是一種生活方式的邀請.
重新的得著好好照顧受造世界這個角色.
是一個能夠進入神國度喜樂 安息和和平的邀請.
是神所定義的平安和圓滿的生命.
這是我們提醒自己福音故事的廣闊.
它是重新的得著神所定義宇宙的美麗.
並且作為新的創造裡的標誌和實例.
是進入神的國度裡的安息日.
神的國度的工作是神的工作.
透過聖靈的能力在基督裡成就.
而當我們發現自己是神的指紋時.
應該做的工作和召命.
我們確實可以與耶穌的話產生共鳴.
就是 我心裡柔和謙卑.
你們當乎我的額 向我學習.

$^{921}$這樣你們的生靈就必得安息.
因為我的額是容易的.
我的擔子是輕鬆的.
謝謝.
字幕志願者 李宗盛.
感謝.
\newpage



\section{}
\label{sec:U2MibYFulYg}
\textbf{「面對同性戀 教會何去何從」:答問時間}
\newline
\newline
連結: \href{https://youtube.com/watch?v=U2MibYFulYg}{\texttt{ https://youtube.com/watch?v=U2MibYFulYg}} ~~~~ 語音日期: 2017-10-12 
\newline
\newline
\hyperref[sec:dJ7VfX3ypUE]{\small{< < < PREV SERMON < < <}}
~
\hyperref[sec:index]{\small{[返主目錄]}}
~
\hyperref[sec:O8VAiCx1rx4]{\small{> > > NEXT SERMON > > >}}
\newline
\newline
$^{1}$我們收到的提問都頗踴躍和尖銳性.
其中首先想對於李博士和江博士的講論裡.
有些提問.
現在的科學研究或不同的說法.
同性戀的傾向其實是天生的.
所以如果他們沒有選擇又怎可以是一種罪呢.
這方面可否請兩位回應一下呢.
(笑聲).
喂喂,聽到嗎?.
這是一個比較難回答的問題.
性傾向是否天生的,是否罪.
我相信教會或宗神不會在網頁列出一個list of sins.
大家可以看到哪些是罪,哪些不是罪.
這不是我們應該有的態度.
在聖經的教導當中.
我自己的了解是很強調我們活在上帝的恩典當中.
而在這個恩典當中我們了解到.
我們整個的存在是在罪中的.
一種人性在罪中的存在.
我們需要上帝的恩典.
我覺得現在很多時候我們會有一種錯覺.
我們以為我們做出來的外在行為就是罪.
而我們內在的存在是天生的.
但沒有關係,這種內外的分別我相信不是聖經所說的.
我記得有一個人問我.
是同性戀者,但我沒有性行為,那我拍拖是不是罪呢?.
如果我們這樣想的話,就會很複雜了.
保羅說在罪人當中我是大罪魁.
那為什麼保羅這樣的人都會說自己是大罪魁呢?.
因為他活在恩典當中.
在恩典當中我們會回轉了解到.
原來他整個的存在.
都是一個墮落亞當之後離開上帝恩典的存在.
剛才我再問傾向的問題.
容許我這樣插開一點.
你問我媽媽,我小的時候就有說謊的傾向.
我不知道是不是事實,我從來沒有說謊.
但我相信很多人都知道.
從小到大都會有這樣的傾向.
這是天生的,就算在醫學當中.

$^{41}$都不知道同性戀的基因是怎樣的.
但我要說的是.
我們有很多其他的傾向.
男性有多性伴侶的傾向.
這些傾向很多時候我們感覺當中.
我們了解當中都似乎是天生的.
我生出來很快就會說謊.
我是男性,我發育之後就會對很多女性有興趣.
但就算是天生的.
聖經都讓我們知道我們很多的慾望.
我們生活的狀態是墮落之後的亞當的狀態.
是需要上帝的恩典.
聖經怎麼說呢?.
其實剛才江博士半小時的討論.
正正就是澄清創造論.
不可以從遺傳基因出發去討論.
因為聖經是有說原罪墮落罪性的事實.
無論你用什麼名稱去說它都好.
所以舊約聖經裡面罪這個字.
不只是一個外在的標準.
而是人心裡面的扭曲軌跡.
所以什麼是創造.
今天大部份的人討論的時候.
都認為是醫學生物學.
而不是從神學去說創造.
所以這個我想江博士剛才說得很清楚.
我希望再強調.
我也很同意剛才提過.
其實傾向不是限於同性的吸引.
The King and I 尤伯年立最出名的一句話.
男人就好像蜜蜂.
不是Beast do it.
而是要在花間採蜜.
我們中國人就叫做如獵.
這個傾向聖經提醒我們.
不是停在行為而已.
耶穌討論奸淫.
從動淫念開始.
其實是幫助我們如何去勝過試探.
所以我收到另外一些問題問得很直白.

$^{81}$我沒有同性性行為.
我有同性戀的戀愛.
這樣合不合聖經呢.
即是某個美國總統說.
我沒有跟別人上床.
我只有某一種接觸.
這樣算不算呢.
我們不是用最低的底線去讀聖經.
這樣是一個非常危險的心態.
聖經從來不是容許我們關上門.
胡思亂想.
只要我沒有做.
我就是聖傑.
所以我想補充剛才培成所說得很清楚.
補充多一句.
我發現很多人在談同性戀的時候.
發現絕大部分的論述都是來自外面.
人家說性傾向.
但說到聖經只有一句.
聖經說是罪.
這樣就完了.
我希望教會在教導當中.
今天我們跟大家分享.
我們對聖經的思考要深入.
我們真正明白.
要了解到聖經是怎樣去說.
知其言 知其所願.
所以教會在這方面的教導是重要的.
跟進一下是否罪的問題.
好像大家都對這兩個字很有興趣.
甚麼是罪.
如果說同性的性行為是否罪.
剛才我們有回應了.
如果同性的傾向又如何.
即是仍然有傾向.
又或者沒有這個行為.
但依附的 相親的.
譬如說繼續跟同性拍拖.
但沒有這個性行為.
這種關係是否可以接納.

$^{121}$可否補充.
可否接納這件事.
好像要畫一條線.
如果推得極端一點.
就好像你過了這條線就落地獄.
在這條線入面就可以了.
你上到天堂.
我想不要有這樣的想法.
我們是要追求整個人的性生活的性結.
所以如果你說.
譬如我見到一個男人.
我是男人.
我見到男人時.
我會想男人的身體有很多的遐想.
我應不應該這樣做.
當然是最好不要.
但你說我是否這樣想了一次.
我落地獄 也不是.
所以接納這個字到底是甚麼意思.
如果你的接納意思是.
這些是ok的 沒問題的.
No 這個不是沒問題的.
我們要追求全人的性結.
但如果你接納的意思是.
有時我們要面對我們會有這樣的軟弱.
但這些軟弱不能夠.
我們不會因為有這樣的軟弱.
就代表和神的愛隔絕了.
In that sense 我覺得是可以接納.
Depends 甚麼叫做可以接納.
有否補充.
你拍拖為了甚麼.
如果兩個人真正拍拖的時候.
應該有下文的.
如果沒有下文的拍拖.
就只是玩玩而已.
我這裡要講清楚.
究竟拍拖是甚麼.
你又不要和一個人吃飯就叫做拍拖.
如果你真的serious拍拖.

$^{161}$其實你有下文.
這就牽涉到.
如果我們了解男女之間的拍拖.
是真的seriously.
很嚴肅的尋找.
我和你還有沒有下文.
我希望尋找我們的明天是怎樣.
這個傾向.
怎麼會不希望更加親密的身體的關係呢.
在拍拖的過程當中.
我們是drawing each other closer.
是不是.
所以如果我們是這樣去看的時候.
聖經是讓我們看到.
慾望是必須在聖潔當中.
在上帝創造的意念當中.
我想我們下一個很熱門的範疇.
就是牧養的範疇.
有人提問.
是否需要在教會設立一些特工.
去牧養同性戀或同性戀傾向的人士.
看到有一定的挑戰和難度.
也提到牧養者有什麼恩賜或資格.
一般信徒在信仰的框架中.
可能會越幫越忙.
牧養可能幫人變成傷害人.
或者幾位有什麼建議呢.
我是神學老師.
所以經常說看問題是怎麼看的.
教會是否需要一些特工去關顧同性戀者呢.
在意義上你可以說是需要.
在意義上是什麼呢.
就是關顧同性戀者的人.
起碼是可以接受.
他不會看到同性戀者.
哈哈這樣.
是一些可以接受同性戀者.
可以當同性戀者也是一個人.
有這樣的心態的人.
然後避免一些不必要的衝突.

$^{201}$但是你說需要特工嗎.
不需要.
千萬不要這樣想.
不要覺得你要受過什麼特殊訓練.
去讀了忠臣的counseling course.
然後才會關懷同性戀者.
No.
同性戀者也是人.
你問候他你關心他.
他和其他人一樣.
都會覺得多謝有人關心他.
所以一個普通人的關心.
你請他去吃飯.
或者和他一起去看電影.
普通人的交往.
本身已經是對他一種鼓勵.
表示其實我們是接納他的.
所以在這個意義上.
不需要特工去做.
所有人應該可以做的一起做.
當然有一些比較技術性的.
譬如同性戀這種習慣.
怎樣形成.
有沒有什麼方法可以.
所謂輔導的東西去幫助他.
去改變一些生命.
這個不單止是同性戀的人.
如果一個人有些心理特殊的問題.
其實可能都需要一些專業的輔導.
涉及到他那個深層的問題.
是不是需要一些專業的人輔導.
這個不一定是同性戀的問題.
在這樣的情況下.
有時可能一些外來的輔導.
一些專業的輔導員.
是可以有幫助的.
我講兩句.
在我們想要特工之前.
我覺得教會一定要做內部的教育.
我覺得教會一定要.

$^{241}$真誠地去看一看自己.
是不是真的有那種歧視的傾向.
如果我們讓這件事隱藏.
在我們自己的裡面.
自己都不自知地受到影響的話.
就算我們有些特工.
做了一些很好的事.
來到群體裡面.
只會令到那些同性戀者.
覺得在教會裡面受傷害.
我覺得內部的教育是很重要的.
講得簡單一點.
教會是需要一些很成熟的人.
來跟同性戀者同行.
我們教會有沒有一些很成熟的人.
不是想要特工.
在他身上做些甚麼.
令他信主.
怎樣怎樣.
不是用這樣的方法.
余牧師可不可以繼續講多一點.
關於怎樣去看自己.
教會有沒有歧視的情況.
因為有些問題都是這樣說的.
如果同性戀者認為.
教會對他們的自由選擇權利的壓迫.
如果你認同他們的取向.
才叫做不歧視他.
才叫接納他.
怎樣跟這些朋友對話.
和牧養他們呢.
我想我們沒有辦法.
我們沒有辦法改變他們的態度.
我們能夠改變我們自己.
我們自己怎樣將自己對他們的關懷.
在關懷當中.
我們很恩慈地將我們的立場講出來.
首先要我們自己有所謂的靈性操練.
我們自己都要有靈性操練.
我們才能夠做到這樣.

$^{281}$KW有沒有補充.
其實你剛才問的問題是.
教會未必能幫到所有人.
我想我們都接受限制.
如果有個人說.
這間教會一定要認同同性戀不是罪的.
我才能回這間教會.
那我就承認了.
我自己的教會牧養不到這些人.
不過我覺得我主要擔心的都不是這些.
而是假如有一個同性戀者.
真的來到我的教會.
結果他得到的不是幫助.
而是更多的傷害.
我想這個是我要關心的問題.
所以正如剛才余牧所說.
我們要教育我們的會友.
不要對同性戀者有些奇怪的思想.
或者不要說同性戀者.
舉一個例子.
譬如我以前在紐約牧會.
紐約都有很多精神困擾的人.
會來到教會.
我牧者通常會跟這些人聊天.
你會發覺當然他有精神困擾的時候.
但大部分時候他跟普通人可能沒什麼分別.
但有些人就.
我跟有些會友說.
你關心一下A君.
會友就說A君有精神困擾.
要一些特殊訓練的人才可以關心他.
牧師你去關心他.
我就覺得有些迷惘.
其實普通人來的.
可能有某些特殊需要.
但大部分情況下其實是普通人.
只需要普通人的關心.
我再說一句.
教會其實就這個議題.
要稍為多一點討論.

$^{321}$以致到我們教會的弟兄姊妹.
不會迴避.
不會把這個問題收起來.
自己有什麼感覺.
有什麼感受.
又不說出來.
我想有討論才有機會.
加深了解.
還有某些想法需要釐清.
這個其實很重要.
或者雷牧師.
你心目中的普通人都可以模樣.
其實你的普通人可能已經是特工.
也有一些很實際的問題.
其實是看我們接納的程度.
譬如說教會可不可以安納.
一個仍然有同性戀傾向.
但沒有同性性行為的人.
做傳道人或牧師呢.
可不可以讓他寧靜餐.
做執事或其他教會侍奉呢.
這個我也想說.
第一.
還有桌上有問題.
就是離開罪.
第一我想說.
離開罪不是一刀切的事.
今天然後明天我就離開了.
往往都是一個過程.
所以一個人離開同性戀生活.
是一個旅程.
不是今天我是同性戀者.
明天我就變成異性戀者.
通常不是.
除非是做一些很特別的工作.
所以.
譬如剛才的問題.
如果有一個人仍然有同性戀傾向.
什麼意思呢.
是否這個人.

$^{361}$譬如他是男人的話.
是否還經常在網上看一些.
剝光珠的男人的照片.
假如一個人仍然有這樣的習慣.
他可能不適合做教會的牧者.
或長老的角色.
但怎樣才叫離開了同性戀的傾向.
其實很難說.
可能我現在.
如果我是同性戀者.
可能我現在生活已經很.
所謂單身.
一般都沒有這樣的思想.
但偶然三個月.
忽然間有閃過一些念頭.
我是否代表.
還有同性戀傾向呢.
有時這些問題不能一刀切.
所以譬如一個傳道人.
要考慮一個傳道人.
我覺得要考慮他整體.
他是否表示了一個整體性的生活.
如果他整體的生活.
是成熟的基督徒的表現.
他亦沒有明顯地犯一些.
同性戀的行為.
起碼一段時間沒有做.
我覺得我們不能因為一個人.
曾經有過同性戀的歷史.
所以這個人以後不適合做長輩.
不可以的.
是要看那個人成長.
但成長到甚麼地步.
正如我剛才所說.
這個就真的要看個人的情況去決定.
很難給一些很客觀的條件.
回應剛才院長提到.
其實教會需要多去探討和多去溝通.
有提問都是說.
在這麼多的爭議.

$^{401}$和在社會裡面有很多公開的表白等等.
在教會的教導裡面.
怎樣可以讓年青和未成熟的信徒.
可以維持傳統真理的教導.
不至於失去分辨的能力.
有甚麼實際的建議呢?.
我都是請年青的.
我自認年青.
我想跟進剛才雷牧.
雷牧,你有兩分鐘去想想怎樣回答.
我想跟進一下剛才的題目.
關於「暗目」.
去年在英國.
有一位,我下次忘記他的名字.
是一個很有影響力的婚姻派牧師.
他有一本書.
不過我現在忘記書的名字.
因為我還沒看過這本書.
書因為出得很好.
所以五周年就再版.
再版是講到教會面對的問題.
再版他就在序言當中.
輕輕說了一句.
這本書的大九章.
是我個人的掙扎.
突然傳媒嘩然.
原來這個人是同性戀者出軌.
馬上追問.
是不是婚姻派教會接受.
一個同性戀者,即是牧師.
簡單來說.
這個牧師知道他是同性戀者.
他接受,他講得很清楚.
我是掙扎,我到現在還是掙扎這個問題.
但我知道聖經的教導.
我知道聖經的真理是什麼.
我都願意在這樣的情況下.
就好像剛才李博士所說.
Holy sex.
我決定是這樣的生活.

$^{441}$所以其實很多年了.
十年前他的同工都知道.
到現在我還是掙扎.
但我是這樣服侍.
我想說的就是.
剛才我們幾位.
可能在我們幾位稍微不是一致的答案.
但一致的答案是什麼呢.
就是你不可以一刀切.
一個問題就給了答案.
每一個不同的個案.
在牧師當中.
他掙扎了很多年.
現在仍然在掙扎.
因為在一個罪性的存在當中.
但我覺得一個人是按目.
或者是掌執.
他是代表教會.
他第一件事就是要接受教會的真理.
而不是一個人進來.
我要去改變教會.
要尊重教會的傳統.
尊重聖經的真理.
在這種情況下.
你發現我們一致的立場.
是真理當中的印典.
我覺得這是希望大家注意的.
而不是每次講到都是在說罪.
或者是怎樣行.
教導聖經.
其實不應該是一個很消極或善的做法.
就用剛才我們講的利未記十八二十二章.
做例子才算.
很難教一章的聖經.
就算在崇拜要讀也很尷尬.
我相信沒有多少間教會.
星期日崇拜讀聖經.
會選這兩章聖經來讀.
所以因此我很欣賞有些教會.
真的輪著一章一章的聖經讀.

$^{481}$你就會讀到這些經文.
讀的時候你怎樣去解釋呢.
你會發覺其實聖經的教導是很好的.
它只是將我們從男女的性慾的試探當中釋放出來.
我見到一個異性的時候.
我不需要被我的天生的傾向去flirt.
這是我作為一個基督徒.
我經歷上帝的恩典的改變的學習掙扎.
所以如果在教會裡面你不教這件事.
其實外面教他很多東西.
放了工一起去飲杯.
甚麼都好.
Alcoholic, Non-Alcoholic.
目的是甚麼.
Flirting.
為甚麼要讓社會教我們年輕的一代.
都不一定是心智未成熟.
我懷疑四十歲都仍然可以未成熟.
男人四十你沒看過嗎.
可能雷牧師太年輕了.
沒看過這套經典的電影.
我的意思是其實聖經是釋放我們的.
清清潔潔.
它不信任我們的傾向.
所以神學院所有老師的辦公室門都有塊玻璃.
教會辦公室的門有沒有玻璃.
為甚麼你不信得過牧師.
是呀,清清潔潔.
所以在我們的掙扎.
在我們的成長的過程當中.
聖經的教導其實是很正面很積極.
不是Do and Don't.
Full Stop.
這樣停在那裡.
所以如果我們是這樣去教的話.
難怪聽的人會覺得歧視.
覺得受傷.
覺得沒意義.
覺得老土.
我希望作為教牧童工.

$^{521}$我們從這個最基本.
有些時候從釋經開始.
就是這樣的意思.
我很難得才有機會插進去.
這些牧者多年經驗的討論當中.
用多了時間.
不好意思.
現在不用回答了.
或者我們轉一轉話題.
去對外.
即是現在面對社會和世界.
剛才余牧師的講論裡面.
提到有關對於逆向歧視的立法.
歧視的法例.
他有一些講論是想有澄清的.
這個問題就是這樣.
十分贊同院長對於同性戀的包容和諒解.
但是不贊同他認為性傾向歧視立法.
如果要落實的話就不應該阻止.
或者這方面院長有沒有一些澄清或補充呢?.
我在另外一個場合.
我很清楚地表明.
我是反對立法的.
不過反對立法不是因為我們怕受到逆向歧視.
我覺得不應該逆向歧視.
這個的確是可能.
我自己也分享過一個經歷.
是受到被人誣衊.
被一個同性戀者誣衊.
強行說我滋擾他.
但其實我完全沒有做到.
作為一個基督徒.
我們不應該只是為了保障自己.
去反對立法.
我的意思是.
如果我們要反對立法的話.
我們是要提出一些.
其他更加合理的理由.
我必須要澄清.
我是反對立法.

$^{561}$其中一個理由.
很簡單.
女人站出來是女人.
我很清楚地看到.
知道她是女人.
她作為一個女性受歧視.
這個其實是比較容易辯認的.
一個同性戀者.
她說自己是同性戀者.
我怎樣辯認呢.
很難辯認.
我分享的經歷是這樣的.
我在這裡也很想和大家分享.
我有一次和兒子在家走下來.
我住在山上.
有人放狗.
放狗的時候.
狗沒有繩子牽著.
狗看到我正在下泥的時候.
衝上來.
撲在我身上.
結果我跌倒.
撞在牆邊.
割傷了手.
流血.
我起來的時候.
當然有些不悅的情緒.
你應該綁住狗.
那個是一個德國人.
他講英文.
我用英文跟他說.
你應該綁住狗.
他不以為然.
我經常都是這樣.
你可能只是我的狗.
我兒子後來很生氣.
馬上說.
你應該至少道歉.
他就說我不需要道歉.
我兒子就衝動.

$^{601}$你不道歉.
我就報警.
因為他受了傷.
他真的打電話.
打電話來.
警察也過了一輪才來.
我們就上回家.
接著警察來的時候.
敲我門.
就說有人投訴.
你滋擾他.
因為他的性傾向的緣故.
你滋擾他.
到底是不是有回感的事.
我就把整個故事再講一次.
那個是我第一次的經歷.
第一次經歷到.
是這樣的.
被人說我滋擾一個同性戀者.
基本上.
他是不是一個同性戀者.
我們沒有辦法.
很容易地去辯論.
或者一個人不是同性戀者.
他要謀害另一個人.
他也可以.
說他自己是同性戀者.
這些都是一些理由.
我們可以反對立法.
我只是舉一個例子.
當然反對立法這件事.
我們可以提出什麼理據.
這就要長加討論.
未必一定可以在這裡.
三兩分鐘可以講得清楚.
或者跟進對院長的請教.
他說如果你不贊成.
用逆向歧視成為一個反對的理由.
但是如果這件事成事的時候.
其實非基督徒也會被逼迫.

$^{641}$你是不是認為.
非基督徒也應該跟基督徒一同.
為二手被迫呢.
我基本上不會反對有人.
因為恐怕逆向歧視.
而提出作為反對的理據.
我只是說基督徒的關注.
不應該是這樣.
我想說的是什麼.
基督徒可以反逆向.
反立法.
我們可以提出很多不同的理由.
當然逆向歧視也可以是其中一個理由.
不過如果我們將這個變成一個焦點的話.
我們真的要想一下.
我們的靈性到底是怎樣.
到底我們是不是真的這麼關心自己.
我舉一個很簡單的例子.
我作為牧師.
我有一次confront一個會友.
這個已經是二十年前了.
confront一個會友.
他有婚外情.
並且屢勸不改.
到最後我們就要跟他說.
如果是這樣的話.
我們就要啟動教會discipline.
就是紀律的機制.
他說啟動這個機制就啟動這個機制.
不要緊.
我隨時都可以離開你教會.
不在你教會.
我跟他說得很清楚.
我說因為你在教會是一個領袖.
我們啟動這個機制的時候.
我們都會將這件事跟會友交代.
他是一個法律界的人.
他就警告我.
他說這是一件非常私人的事.
你不可以將這件事告訴任何一個人.

$^{681}$如果你不是的話.
我一定會告你.
一定的一定告你的.
我當時作為一個牧者.
我怎樣說呢.
我說我明白你的感受.
但是我作為一個牧者.
我沒有選擇.
我做你認為需要做的事.
但我作為一個牧者.
我只能夠做我應該做的事.
在我教會的規章裡面.
我一定要按照整個程序.
整個手續來做.
我說如果你告我的話.
如果我又入罪的話.
我是準備坐監的.
我們都要有這種情操和心態.
不要只是為了保護自己.
來表達自己的立場.
說到可能要坐監.
也有些法律問題.
不是請大家當自己是律師來回答.
但看看有什麼可以給教會提醒和參考.
其中一個就是.
若果同工.
我想應該是教務同工.
發生了同性戀的問題.
教會在行政上.
如果可以自己處理的話.
有什麼提示.
以便避免抵觸法律呢.
另外一個不是同樣的.
是有關教會辦的學校.
應否聘用有同性戀傾向的老師呢.
大家對這兩個問題.
背後反映的想法有什麼回應呢.
同工說兩句.
首先我想我要關心的是.
遵守聖經的教導.

$^{721}$所以會根據馬太福音的程序.
個別跟他說.
如果解決不了.
才慢慢升級.
不是一下子就要攤開.
我想這個其實根據聖經的程序.
由小到大.
可以不說知的人.
由小到大.
影響從小到大.
我想應該都合乎法理上的要求.
不是無緣無故將私隱的事公開.
你說會否觸犯法例.
真的不知道.
可能法律人士才可以更加清楚解答.
不過一般我的常識是.
教會對於道德一向的立場.
不是隱蔽的一件事.
所以如果教會因為同性戀的行為.
而作出某些紀律的行動.
起碼那個人不可以說.
這件事是不知道的.
事先沒有說過.
你忽然間跳出來的.
所以起碼要告的話.
都是一個要爭論很久的事.
不是明顯有一個法律的案件.
我想起碼這個按部就班去處理.
是重要的.
江博士有什麼補充.
可不可以聘請一個同性戀傾向的人.
教會辦的學校.
專家未曾考慮清楚之前.
我做CALIFER.
其實如果按聖經的看法.
你不將他集中看同性戀.
一個這麼窄的範圍.
有沒有幫助呢.
有一個來申請做老師的.
他有病態倒圖的傾向.

$^{761}$你請不請他呢.
你可不可以將他放寬一點來考慮.
其實每個辦學團體.
有不同的政策.
會不會.
你的校監或校長或校董會.
討論這些問題的時候.
可以有不同的界線.
所以不是只是討論一個很窄的一點的事情.
這是起碼你未寄給我們見到的範疇.
我給了一分鐘的江博士.
希望大家注意.
李博士經常提出.
我們要從闊一點的角度.
這是一個很好的提醒.
很多時候我們以為爭論點就是同性戀.
為什麼不行呢.
忽略了其實.
很坦白說我們現在是面對一個文化價值之爭.
後面你看是自由主義後面所推動的.
就是有選擇的自由.
我們有等等的東西.
但這裡其實.
如果我們放長遠一點看.
自由主義其實自己是陷在一個困局當中.
因為他經常玩著.
究竟社會當中有什麼受到的限制.
有什麼受害者.
出來說.
受害者的時候.
其中一個問題就是立法.
又不可以這樣.
結果自由主義者沒有注意的問題.
慢慢形成一個大政府.
政府一直要介入.
立法去禁止所謂的歧視或者受害者.
但同時我們需要知道.
教育是有教育的理念.
在這樣的情況下.
教育團體過去在香港也出現過一個問題.

$^{801}$教育團體發現在這樣的情況下.
我們再辦教育.
有沒有意義呢.
我們教育團體教育學校.
是否按照教育的理念去辦教育呢.
大家要注意.
教育團體在公共的事.
教育這方面是出了很多力.
今天有很多反對教育的人.
其實是教育學校畢業的.
這是很重要的.
我們需要擴大來看.
不要懷疑一件事.
你不可以不請他.
我請他之後.
我給學生的教導是怎樣呢.
是否混亂呢.
我們不容易在一個平台.
有一個渠道.
教會可以按照自己的立場去教導.
譬如我們在學術界.
或者我們在實學教育.
我們希望學生是開通的.
學生是可以怎樣說呢.
是很critical mind.
但我們也有立場的.
我們教導學生就是在這樣的.
不同的價值觀當中.
為何這間學院.
這些老師是這樣教導呢.
不是因為他們只是畫地自限.
他們想過.
他們有這樣的立場.
但如果你要求所有的教育.
都是要甚麼都說.
我今天很多時候問大學的老師.
他們說不行的.
我不可以跟學生說.
要怎樣選擇的.
我們說了ABCD他們自己選擇.

$^{841}$這是一個問題.
結社自由也是很重要的.
我們在某些情況下.
有些人因為有共同的理念.
他們譬如說教會.
特別是教會.
有一定的價值觀.
如果我們社會忽略了這件事.
教會沒有這種選擇.
而這種選擇都是為了在公共的線.
表達這些價值觀的意義.
我覺得對社會來說是很大的傷害.
我補充一點.
他說得很大.
收回一點.
教會學校如果有同性戀的人.
來申請做老師.
我們當中其實有法律專家.
但無謂起他們底.
所以我這個不懂法律的人說兩句.
我想如果那間學校是拿公帑的話.
不請同性戀人士的理據.
可能不是很強.
你教那間學校.
為甚麼只可以請同樣道德觀念的人呢.
很難的.
或者有個佛教徒申請基督教學校.
你是不是因為他是佛教徒.
所以就一定不請他呢.
這些我想如果是拿公帑的學校.
是比較難的.
但是從另一個角度來看.
如果有個同性戀的人.
來一間基督教學校做老師.
是不是一定是一件壞事呢.
我看又未必是.
因為第一.
基督教的學校.
不清楚他的德育教育是教甚麼.
所以老師入職之前.

$^{881}$跟老師講清楚.
這是我們的德育課程.
當然你不需要一定認同我們所講.
不過因為為了教育的順利.
希望你不要在課堂上.
特意講一些東西跟我們作對.
我相信就算是一個同性戀的人.
可能都願意接受這件事.
這是他的身份.
但他未必一定要在課堂講這件事.
是好事.
如果學生知道又如何.
我覺得這也是一種見證.
我們在這個社會裡.
就是要跟不同價值觀的人相處.
學校就是一個micro-christian.
即是一個大社會的縮影.
在這個社會裡.
即是在學校裡.
學習如何跟不同價值觀的人相處.
不過學校也要教導清楚.
基督教的價值是怎樣.
當然基督教的價值.
不代表全世界的人的價值.
但幫助學生明白.
我們基督教為何有這樣的價值.
然後幫助學生在這個大社會裡.
選擇他們認為對的去過生活.
我認為社會教育應該教育一件事.
就是我們應該尊重.
不同團體的價值觀.
我覺得我小時候.
我印象很深刻.
有很多不同的人來教會.
會自覺覺得我是吸煙的.
我在教會我就不吸煙.
當時沒有什麼二手煙的概念.
只不過知道教會是這樣.
當時很尊重.
知道是這樣.

$^{921}$我現在完全沒有提及健康的問題.
但現在的人是這樣的.
覺得我是這樣就是這樣.
一定要接受我.
我覺得一方面我們討論到.
教會是可以開放.
使到不同的人在教會當中都可以參與.
但另一方面我覺得.
其他人都要學習尊重.
所以我剛才說的.
一間學校最重要是辦學的理念是什麼.
如果辦學的理念受到影響.
我覺得應該尊重.
不應該干涉影響一間學校的辦學理念.
如果你加入這間學校.
你教書.
不論你的背景是什麼.
都不只是同性戀的問題.
但在這樣的情況下.
不影響辦學的理念.
不影響學校教育的目的.
我覺得是有開放的空間的.
你也看到職場的示範.
是可以有不同的意見.
可以有交流.
可以很開放地接納.
這個可以看到.
其實這個情況.
也是一個對話的開始.
其實今天不是很對話.
因為你沒有機會出來拿咪高峰說話.
不過希望是一個起步點.
說到對話.
有人提問江博士.
又是你.
你剛才提到.
不要一說就說平等.
公義 權利等等.
但這位朋友說.
其實這些是一般人可以理解到的.

$^{961}$一些普世價值.
所以用這個入手點.
也是可以溝通的.
但他說.
根據你剛才用基督教的人觀.
對於兩性的complementarity.
和reciprocity的神學.
不是一說就明白.
甚至對基督徒也很難掌握.
這方面.
其實教會在社會上.
很難用這些溝通方式.
不像自說自話那樣.
那怎樣將我們的信仰神學.
作為我們反對的理由呢?.
你或者去表白一下.
你的選取.
有沒有考慮到這些方面呢?.
因為他們跟我說.
我只有30分鐘說.
再給我30分鐘.
我就說第二段.
我不是反對平等.
說自由 多元.
我是說.
如果我們面對同性戀.
我們其中一樣東西.
就是不可以只說這些.
不說性.
特別在我們說到.
同性和異性有什麼不同的時候.
特別我注意到.
在我們的context之下.
很多時候很多人先假設.
同性 異性 相性.
全部都是一樣.
我在這一堂特別提出來.
你不可以將兩者橫行一談.
沒有作任何區別.
第二.

$^{1001}$其實我剛才也說過.
雖然我說的是聖經的基礎.
從聖經開始.
因為之前說的是Sola Scriptura.
但是其實我所說的.
我只是說生態.
其實在Natural Law.
很多人可以明白.
男男的身體的關係.
女女的身體和男女的身體的關係.
親密關係是有不同的.
這個我相信.
不難在公共空間去說.
如果你問我有什麼難.
我自己過去這麼多年.
在中神之前在大學教一科.
我就是一直這樣說.
我說了13個星期.
我的了解當中.
學生是可以明白.
身體的關係是什麼.
我不知道你剛才.
我沒有聽清楚.
是不是有問到性和愛的關係.
是不是有提到.
這一段沒有.
等你下次有30分鐘再解釋.
有沒有.
這方面有沒有其他回應.
在對話的公共空間.
我只是想說一句.
因為舊約聖經教導我們.
為孤兒身冤為寡婦便屈.
所以我要跟江博士.
我自己最後也說了.
你不要誤會他.
是一直在中神象牙塔裡.
閉門造車的神學教授.
他是在香港社會最前線的.
大學生當中教倫理學.

$^{1041}$很多我們教牧童宮.
都沒有這樣的機會.
站在多少個性伴侶.
我都不敢引用你.
在這樣的群體當中.
用一個聖經神學.
福音信仰的立場.
但不是全福音的報道會.
來討論性倫理.
所以這個是我們需要學習的.
不是每個人都需要.
不是每個人都有機會這樣做.
不過我們需要作為童宮.
我們要明白.
我們要想通整件事情的過程.
說到溝通和對話.
也有一些提問.
似乎很多在不同取向的教會.
剛才雷母幫我們.
放了幾個派別的名字出來.
其實裡面好像是有一些溝通.
但這位朋友覺得.
看不到有真正對話的可能.
好像在雙軌進行.
這裡有這裡說.
那邊有那邊說.
可以怎樣去突破這件事呢.
尤其是在教會裡有不同想法的信徒之間.
另一位朋友也有一個問題.
怎樣可以整合到教會.
信徒參與社會空間裡.
對同運所帶來的衝擊呢.
整合到大家.
這裡有千多人報名.
有很多關注在裡面.
可以怎樣去整合呢.
而不是每人自己一堆說自己的話呢.
其實上個星期五.
上個星期五.
有一班神學生在永光堂聚會.

$^{1081}$他們其中一個議題.
就是關於同性戀的問題.
在其中兩個不同立場的人.
同台發表意見.
我自己覺得這樣的平台.
其實是創造了一個非常好的機會.
不同立場的人.
可以將他們的立場和背後的理據說得清楚.
以致教會的弟兄姊妹.
比較了解對方的想法.
或者如果要作出判斷的時候.
應該怎樣作出一個很合理的判斷.
我不知道江博士.
有沒有想不想分享一下.
因為我是最新加入中神.
所以很容易被欺壓.
我也承認了.
所以誰要加入中神的時候.
你先想清楚.
如果說對話.
原諒我可以說幾句.
我過去差不多有二十年的時間.
是從東正教正式對話的.
是很正式的.
一年前每一邊派三個人.
差不多有二十年.
在對話當中.
也有不少的誤字.
東正教的神學家對Couture說.
(英文).
是很強烈的.
(有沒有拍桌子?).
也有我在科學與宗教的對話.
我覺得在對話當中.
我注意到幾件事.
當然面對面的對話是很重要的.
對話其實是一個process.
如果嚴格來說.
我們東正教和東正教的對話.
二十年是一個真正的對話.

$^{1121}$因為是同一班人一直不斷.
一直有一個concern.
一直不斷的對話.
過去在大學.
很多時候對話的人一直不同.
沒有一個焦點.
大家說自己的立場.
沒有什麼聆聽.
完了之後.
拍拍屁股就走了.
這些就是對話.
但我們的社會當中.
對話的確是很困難.
我注意到一個問題.
對話不是只講立場或結論.
其實很多對話的平台.
都是聲音,語氣,技巧就完了.
我覺得對話需要有一個空間.
你要講出自己的思想是什麼.
是慢慢的.
要給大家一段時間.
對話是一個很長的process.
如果你沒有打算用五年的時間.
就沒有對話了.
不過我從我個人的經驗.
我想說一件事.
我在大學教書的時候.
我就注意到一個問題.
很難或不要說道德的問題.
因為說道德的問題.
很容易put off.
學生就不說了.
我試過有一次.
有一個跟幾個學生聊到.
教員師就聊了好幾個小時.
關於性的問題.
其中一個學生生得很漂亮.
他就跟我說.
一個女生幾個男生都沒所謂.
同時幾個男生都沒所謂.

$^{1161}$他說開心就可以了.
摸索文化就可以了.
我完全沒有跟他說道德的問題.
令我驚訝的是.
他說我是一個基督徒.
我說你是我教過的學生當中.
最前衛的一個.
那天晚上我們聊了兩三個小時.
幾個學生跟我聊到半夜十一點.
之後我就說很簡單.
我說性是很intimate的body relation.
如果你跟每一個人都這樣說.
例如我對每一個人都說你最了解我.
聽到都很開心.
但知道原來我對每一個人都說你最了解我.
你就沒什麼興趣了.
但如果遇到一個人.
我說身體的關係是一種身體的語言.
我們知道身體語言.
即是我們身體的關係.
如果有一個人你真的覺得很特殊.
那你有什麼身體語言跟他說呢.
表達就是這個身體語言是我跟你說.
不是跟其他人說的.
有沒有一個特殊的身體語言呢.
他回答不出來.
我說還有今天跟你說這個語言的人.
交換這個身體語言.
今天跟你這些男同學很開心.
都很開心他這麼靚.
那他五年之後十年之後.
是不是都想跟你說這個語言呢.
或者他跟其他人說了.
你明白嗎.
即是十五年二十年之後.
今天很喜歡跟你交換這個語言的人.
十年十年之後.
兩個星期之後他寫了一封信給我.
他說對不起我這兩個星期很忙.
所以我沒時間寫.

$^{1201}$兩個星期之後他寫了一封信.
他說那天跟你談了之後.
我回去大喊一輪.
大哭一場.
我希望你所說的是正確的.
我要重新思考我的身體語言.
這個就是知其言知其所以.
我們在公共空間中的對話.
是給對方一個空間.
我一句道德的執迷.
我跟很多弟兄姊妹說.
他說不可以不講道德的.
我們一定要講道德的.
我當然講道德.
但就不是一般人想像的道德.
而是講身體語言的意義是什麼.
我自己在這方面.
是實際進行很多對話.
說到對話和道德.
這裡有兩個問題.
想提問一下教會.
當他在公共空間.
很高調地表明自己的立場的時候.
會不會像雷牧師所說.
先檢視一下自己有沒有偽善.
會不會如何回應.
社會或基督徒.
對高調的表達.
是一種偽善或忽然公義.
或是針對同性戀者的指責.
相關的問題就是.
有些堂會在聖經向歧視條例.
未必有不同的想法.
不過對於如何表達自己的立場.
是有不同的取態.
具體說他可能會覺得.
如果是舉辦祈禱會.
向社會宣示立場.
會有借屬靈的事情.
去干涉政治之言.

$^{1241}$而且會覺得有一點霸道.
這兩方面的提問.
大家會如何回應.
我講兩句.
早前我給一位傳媒人.
是朋友來的.
晚追問我.
教會為什麼這麼霸權.
對同性戀者.
採取這麼壓制性的.
態度和立場.
我問他.
你在哪裡看到.
聽到教會有霸權的表達呢.
他說不出.
我想很多時候.
其實都是一些印象.
我們想當然.
教會有這樣的立場.
於是他們一定是很保守.
很封閉的人.
這些保守封閉的人.
面對著和他不同的人.
一定會採取霸權的手段.
這些我想是印象.
我覺得香港的教會.
我剛才都講得很清楚.
我自己的印象.
真的對同性戀這個議題.
其實很少公開出來.
有什麼表達.
不要說強烈的表達.
我們講來講去.
都是一兩個人.
真的.
講來講去都是一兩個人.
到底香港教會.
是不是真的很霸權呢.
我自己覺得.
在我自己的經歷裡.

$^{1281}$我一點都不覺得是這樣.
我反而希望教會.
是更加清晰的來表達.
他的立場.
我想補充一點.
偽善這件事.
我想每個人自己都要問.
所以我要問.
我是不是只是.
支持某一種議題.
其他就不關心.
這個自己回答.
譬如113的聚會.
我有去.
之後有一位.
都幾出名的人就說.
為什麼基督徒.
只是來113的聚會.
就不去七一遊行呢.
我可以.
如果有機會和這位.
知名人士聊天的話.
可以告訴他.
七一遊行我去了幾次.
我還帶我兩個女兒去.
113我都沒有帶我的女兒去.
所以我有做到.
當然教會做這些事.
不是一定要.
打著旗號七一遊行.
要麥克斯堂.
堂堂打著旗號七一遊行.
未必要這樣做.
所以我們要接受.
我們自己問.
但有時一些偽善的指控.
是沒有辦法的.
人們是要這樣指控.
你不能夠因為.
要滿足人家的要求.

$^{1321}$我沒有偽善.
我才出來說話.
結論就是你不用說話.
每個人都找到理由說你是偽善.
不過說話的態度.
當然都要重要的.
我們就要強調.
要解釋清楚.
我們在公共空間.
是為大眾的好處說.
好像余牧師所說.
是為大眾的利益去說.
我們不是要保護.
基督徒少數的利益.
我們去說話.
這個我想我們就要.
反而是這件事我們要小心.
其實我們在短短一個小時裡.
都回答了好幾方面的問題.
雖然有些個別的問題.
未曾很詳細地去回應.
由於時間關係.
剛才都說了.
今天只不過是一個開始.
我想我們還有很多聲音.
需要去聽.
還有很長的路要走.
我想問問幾位講者.
有一些是剛才給你們的問題.
有沒有一些很想現在馬上回應的.
不回應的話.
待會吃飯都不安心.
我可以多給兩分鐘.
我不是不安心.
不過可以趁這個機會.
等他們預備的時候.
我補充剛才時間所限.
我沒有提到在你們手上的大綱.
我列了四本書.
我手上有十幾個問題.

$^{1361}$所以我想介紹你們看這幾本書.
如果你問的是《釋經》的問題.
就第三本Robert Gannon.
2001年出版的.
The Bible and Homosexual Practice.
Texts and Hermeneutics.
就是你很值得去看的.
如果這個是你關心.
釋經.
還有弟兄姊妹問很具體.
問到某一段經文怎樣解.
這本書幾乎是好像手冊一樣.
什麼都包羅萬有.
如果你的問題是對於神學上.
和近代很多問題都在問.
現在看同性傾向有不同的看法.
最後這本書Paul King Jewett.
好像比較舊一點.
但如果你是我那個年代.
對神學開始有一點點興趣.
大概這個人你不會陌生.
他是美國佛羅倫斯學院.
我不知道余牧師在那裡的時候.
他是否教授.
他是余牧師的老師.
他第一本震撼整個福音信仰世界的書.
就是Men as Male and Female.
那時候還沒有inclusive language.
他質疑保羅的教導.
是否只是文化的處境.
而不是永恆的真理.
這個人在福音信仰神學院.
他的看法他的思考絕對不保守.
我建議你看看他這本書.
他用了很長的篇幅.
特別去討論homosexuality.
你不看完也好.
Karl Barth那些fine print很難看.
你看他的結論很精彩.
這是一個美國心思熟慮的.

$^{1401}$福音信仰的神學家.
他對這個問題有一個很深入的反省.
Christopher Yuen這本書.
剛才我提過就不再重複了.
Mark Yarhouse的這本書很薄.
放在書的原因就是.
我知道可能有弟兄姊妹會覺得.
今天說的話很硬.
我們的題目是需要處理一些很硬的問題.
我是個媽媽.
我是個姐姐.
這本書說.
A Guide for Parents, Pastors and Friends.
這本書是值得你上網可以買得到的.
我停在這裡.
不好意思因為那些問題.
時間所限不可以逐一回應.
我先說兩分鐘.
給你多一點時間.
第一如果你上網看我的essay.
我會多加幾本書.
不過不介紹了.
這本書和施敬沒有overlap.
因為這些書很多.
我基本上是集中說教會如何面對這個紛爭.
如果你有興趣上網看我的essay.
第二件事我想回應.
剛才Vance問了問題.
不過沒有時間回答.
給江大俠一點時間.
然後我就拿了時間.
就是對下一代如何講性這個問題.
因為我有三個女兒.
所以很關心這件事.
我的一個建議就是.
你多一點和你的子女或者年輕人.
多一點討論時事.
和和他分析這些時事代表了什麼.
我今天看新聞.
好像有一個19歲的男孩.

$^{1441}$不知道如何介紹他的女朋友.
我有時和女兒聊天.
為什麼現在這麼多這些現象.
現在的人如何看性.
和他解釋.
讓他明白第一現在的人如何想.
第二明白其實想法是很重要的.
如果你有某些不合乎聖經的想法.
對你的生命可能會有很大的破壞.
於是就希望年輕人能夠明白.
世人有世人的想法.
但世人的想法不一定是對的.
你要知道我們如何看.
我如何看.
我會和我的女兒說.
我希望聖經的看法是怎樣.
還有說完這句.
其實你的模特兒.
我們基督徒的生命的模特兒.
我做了什麼生命出來.
譬如其實我的女兒.
如何去了解兩性之間的關係.
很多時是看我和我太太的關係.
如果我做得不好.
我經常舉例去打我太太.
其實是鼓勵她對男人沒有信心.
所以我們在教會有沒有模特兒.
一個好的聖經倫理.
給我們下一代看呢.
這是我們要自己問的.
我這裡有一個問題.
他是這樣問的.
為什麼今天我這一堂講.
性是要講到身體.
而不是講愛呢.
為什麼不講愛呢.
所以我覺得這個問題應該回答一下.
他說是不是講性就等於已經講了愛呢.
還有你愛一個人.
是不是執著一定要講身體.

$^{1481}$要講性別呢.
當然今天我講的.
主要是在同性戀的context當中.
我想大家更加清楚講明.
同性之間的性的關係.
和異性之間的性的關係有什麼不同.
很多時候我們講愛的時候.
很容易很抄襲.
我們說大兄姐妹彼此相愛.
什麼是愛呢.
父母兒女之間的愛.
家庭當中.
所以愛是可以變成一個很抄襲的東西.
但是我們今天社會講愛的時候.
很多時候都忽略了.
很多時候你可能愛上一個不可愛的人.
所以很多很複雜的問題.
今天在另外一個context我們可以講愛.
但今天希望從身體這方面.
因為我們講的性的確和愛有關係.
但我們講的是embodied sexuality.
是說身體男女之間.
和同性戀人之間有什麼不同.
所以我想這裡需要澄清一下.
就這樣吧.
在我面前有一位聽眾.
他提到科恩派教會.
是不是會帶著政教分離的擋箭牌.
對政治的潔癖.
不願意濕身.
跳入同性戀政治的重要戰場.
我只想回應一下.
他問這個問題的時候.
採取的一個視野.
他有關同性戀這個問題.
我們面對他就好像在戰場.
我真的要提醒大家.
我們不要用這種心態去看待這個議題.
不要用一種好像要打仗的心態.
真的要保持一個諒解和對話的取向.

$^{1521}$來面對這件事.
多謝幾位講者.
今天我們是一個問題帶到我們一起.
面對同性戀教會何去何從呢.
我們未必有所有答案.
我也不希望你覺得自己有所有答案.
但正如余牧師提醒我們.
我們希望去培養一個對話的情操.
有諒解,尊重,包容.
但是講真話的.
希望今天是一個開始.
然後我們看看我們怎樣可以一起向前行.
接著我們請忠臣教務長張良牧師.
我們有致謝和祈禱結束.
讓我們一起起來.
我們有一個禱告.
首先多謝你讓我們今天早上.
有這個機會可以在這裡聽我們四位教會的老師.
長者向我們所說的話.
他們願意明白上帝你自己的心意.
同樣呼籲我們.
服膺在上帝你自己的話語底下.
主求你幫助我們.
我們知道我們裡面有很多慾望.
不只是同性戀的人士有他們裡面的慾望.
我們每一個人都有我們自己裡面的慾望.
當我們看到我們自己的時候.
其實我們都能夠明白.
同性戀的人士他們內心裡面很多的掙扎.
主求你叫我們時時懷抱著這種心.
知道怎樣去關懷這些人.
怎樣去愛這些人.
或者不只是我們教會裡面我們會面對這些人.
或者甚至今天早上我們中間都有人.
在這些掙扎當中主要我們求你憐憫.
我們求你的聖靈不斷地在我們生命裡面工作.
讓我們的生命不斷地因著我們的成長成熟.
能夠知道怎樣來將我們的慾望.
能夠在聖靈的管制底下.
能夠被你改變.

$^{1561}$主我們很希望我們能夠聖潔.
好像上帝你自己聖潔一樣.
主你幫助我們.
也給我們勇氣給我們智慧.
知道怎樣去牧養我們身邊的.
你放在我們身邊的弟兄姊妹.
放在我們身邊的人.
知道怎樣去關心他們愛他們.
讓他們明白上帝你自己的真理.
明白你的福音.
體會你的福音怎樣可以改變我們的生命.
主同樣幫助我們有那種胸襟有那種智慧.
知道怎樣跟外人交往.
在其中要常常地帶著調和.
以溫柔敬畏的心來回應各人.
主求你幫助我們.
讓我們領受祝福.
願耶穌基督因為赴上帝的慈愛.
聖靈的感動,加力和團契.
尚與我們眾人同在.
從今時直到永遠.
阿門.
各位請坐.
今天聚會到這裡.
(字幕製作/時間軸:秋月AutumnMoon).
\newpage



\section{}
\label{sec:O8VAiCx1rx4}
\textbf{【公開講座|擁抱殘疾的教會 — 群體中經歷醫治和牧養】}
\newline
\newline
連結: \href{https://youtube.com/watch?v=O8VAiCx1rx4}{\texttt{ https://youtube.com/watch?v=O8VAiCx1rx4}} ~~~~ 語音日期: 2023-04-14 
\newline
\newline
\hyperref[sec:U2MibYFulYg]{\small{< < < PREV SERMON < < <}}
~
\hyperref[sec:index]{\small{[返主目錄]}}
~
\hyperref[sec:kR2ujHQel1E]{\small{> > > NEXT SERMON > > >}}
\newline
\newline
$^{1}$(字幕製作:貝爾).
好,等一會晚安.
大家吃飯了嗎?.
很開心可以見到大家.
也代表中國神學研究院和福音正主協會.
歡迎大家參加一個講座.
講座叫做「擁抱殘疾的教會:群體中經歷醫治和牧羊」.
我相信我們的上主不單止讓我們用口去彼此問安.
我剛剛跟了Carol傳道學了一個手勢.
我一起學一下用這個手勢去跟其他弟兄姊妹問安.
第一個手勢叫做「平安」.
我們跟身邊的弟兄姊妹問安.
平安.
手語,手語,手語,手語.
對了,對了,對了.
第二個手語動作就是「你好」.
對嗎?.
大家一起試一下.
跟身邊的弟兄姊妹表達你好.
很開心我們可以用不同的方法跟身邊的弟兄姊妹彼此連結.
要介紹講完之先,有兩個報告.
第一個報告就是.
如果弟兄姊妹需要看手語傳譯.
可以坐在前面這個位置.
就可以看得清楚一點.
如果弟兄姊妹需要去洗手間的話.
我們地下和一樓都有洗手間.
而一樓特別有一個無障礙的洗手間.
如果弟兄姊妹有需要的話.
也可以出去外面有一部升降機.
上一樓就可以去到.
事不宜遲,我也介紹今晚的講完.
也很感恩有我們實踐科關允兆老師.
楊石昌教席副教授.
去分享一個主題的訊息.
接著就有兩位回應講完.
分別是我們學院神學科助理教授楊思賢博士.
和中華基督教會基督堂王凱恩傳道.
和我們分享.
將時間首先交給Vance老師.

$^{41}$各位,平安.
你好.
很開心今天可以在這裡和大家分享.
今天的題目是.
「擁護殘疾的教會,群體中經歷醫治和無恙」.
這是一個我很想和香港教會分享的題目.
首先我要說明一下.
雖然在這裡分享的是我.
但其實背後有很多和相見群體一起的生命經歷.
去啟發了我今天的訊息.
當中每個人,包括我自己都是很獨特的.
各有不同的殘疾.
在這裡首先要多謝他們.
多年來容讓我和他們同行.
一起分享生命.
今天的題目也挺長.
牽涉很多方面.
有殘疾,教會,醫治,無恙,擁抱經歷.
接下來的四十分鐘,我會集中在幾方面去分享.
有三方面.
首先會和大家看看何謂擁抱殘疾呢?.
另外一方面,在這個何謂擁抱殘疾.
我想提出一個願景,是雙向共融的願景.
然後,描繪這個願景後,看看有什麼障礙要達到.
最後,會和大家看看在這些障礙和張力的群體生活.
是如何讓大家一起經歷醫治和無恙呢?.
曾經有人問我.
其實殘疾人士都算是小眾.
為什麼你會有興趣為這個課題寫書和搞講座工作坊等等呢?.
沒錯,我發覺上帝都會用我去搞偏門的事情.
其實這個課題在教會未必會引起很多關注.
除非你本身和殘疾群體有聯繫.
或者你教會本身已經開展了這些事工或打算開展.
其他人覺得這個課題和我沒有什麼關係.
事實上,是否真的那麼偏門呢?.
我認同殘疾人士是小眾,他們是隱藏的群體.
由於種種障礙,我們很多時候都看不到他們.
聽不到他們的聲音.
在社會是這樣,在教會也是這樣.
我在神學院教書很多年.

$^{81}$我留意到在教會的圈子裡.
對於殘疾群體的認識其實都很缺乏.
如果我本身不是一個職業治療師.
其實我根本不會留意到.
這群朋友其實有很多的需要各方面.
特別包括他們的屬靈需要都被忽略.
其實你不要以為我們健全人士就不需要關心殘疾人士的需要.
神學家Nancy Iceland,她自己本身也有殘疾.
她指出一個我們很多時候忽略了的事實.
這個事實就是提醒我們.
其實我們這些所謂健全人士.
Able bodies.
其實只不過是暫時健全的人士.
Temporarily able bodies.
即是我們還沒有殘疾.
你想清楚一點.
就算不是有什麼病患或意外令你殘疾.
事後到了開始衰老的時候.
你的視力,聽力,行動能力,思考能力都會慢慢減退甚至消失.
神學家Moteman甚至說.
根本就沒有殘疾人士和健全人士之分.
這個很值得大家思考.
你覺得自己是健全還是殘疾呢?.
一邊聽你一邊思考這個問題.
另一方面,雖然我們說殘疾人士是少眾.
但其實他們的人數也不少.
根據2020年香港殘疾人士統計數字.
香港大約有53萬4千多位殘疾人士.
佔全港人口7%多.
這些殘疾人士有一項或多於一項的殘疾類別.
包括列出的身體活動能力受限制.
視覺或聽覺有困難.
溝通能力有困難.
精神病,情緒病,自閉症,特殊學習困難,專注力不足等等.
以上這些數字其實還未包括智障人士.
為什麼呢?.
因為統計處發現他們搜集資料的方法.
經常會低估智障人士的數目.
所以他們索性不計算在這個數目之內.
而根據他們粗略的統計評估.

$^{121}$全港大約有7萬多至9萬位各種程度的智障人士.
事實上,除了這些統計數字.
在一個基督教圈子裡.
2020年樂喪第三屆宣教會議裡.
都留意到殘疾人士這個群體.
在他們的會議中發表了一份文件.
「開普頓承諾」.
提醒普世教會.
其實殘疾人士是世界上最大的邊緣群體之一.
估計他們的人數超過6億.
而這個群體是福音未得之民.
文件也指出殘疾人士每天所經歷的.
除了生理或心理上的缺陷之外.
也會面對社會中種種的歧視和不公平.
我曾經參與教會差派的海外醫療服務.
回想當年,是90年代.
當年的宣教概念是去到遙遠的地方.
接觸一些不同宗教信仰的群體.
將神的愛帶給這些未得之民.
不過我越來越看到.
其實未得之民就在我們身邊.
雖然他們是隱藏的一群.
但神並沒有忘記他們.
神聽到他們的呼聲.
亦是這種呼聲驅使我關注這個議題.
值得注意的是這份文件中提出了一個新的向導.
帶領教會思考殘疾人士這個課題.
文中提到,殘疾人群.
如果服侍他們,並非單單服侍.
而是接受他們所給予的服侍.
教會群體的重要任務是協助殘疾的信徒.
去完成他們自己宣教的呼召.
原來擁抱殘疾人士.
不單單是單向地服侍他們.
向他們傳呼音.
而是看到有殘疾的人也和大家一樣.
可以去服侍,亦可以是呼音的使者.
這正是我希望在這兩天帶出的願景.
就是雙向共融的願景.
甚麼是擁抱殘疾呢?.

$^{161}$就是活出這個雙向共融的群體生活.
在此,很想分享一下這本書的封面設計.
亦是源於這個雙向共融的願景.
首先用了紫色,代表尊貴.
而這個圖案是表達一種叫做「金繼藝術」.
又稱為「金線藝術」.
是一種修補破損器具的日本傳統工藝.
工藝師會運用漆和金粉.
去黏合那些崩壞了的器具.
而經過藝術家的重新塑造.
原本被視為遺憾的裂痕.
反而成為整件事最美的位置.
完成品的價值可能比原本未破爛的器具更高.
同樣,雙向共融的願景.
就是破損的生命在信仰群體的擁抱同行當中連結.
就像碎片連結在一起.
承載基督尊貴的器皿.
大家可能會問.
這個雙向共融的概念.
和社會上所說的「雙建共融」有沒有分別呢?.
事實上,我們有很多東西可以從公民社會學習.
首先,在社會上所說的「雙建共融」.
其實都有幾個層次.
香港商建協會在2022至2023年發表的.
「共融教育計劃」文件中.
引用了澳洲商建共融專業顧問Debra Fullwood的理論.
描述共融的四個層次.
第一個層次是「出席」.
可以離得到大家中間.
輪椅人士要出席.
可能要安排復康巴.
還有適合的輪椅通道.
所在的樓層要有電梯.
也是這個原因.
今天我們特別由禮堂轉到這裡.
讓大家不用搭電梯.
場地最好有商城人士用的洗手間等等.
第二個層次是「參與」.
參與就是participate.
不單單是出席坐在這裡做旁觀者.

$^{201}$可以參與到大家的活動.
你見到今天我們當中有手譯翻譯.
有些教會在崇拜當中會為視力障礙的朋友.
準備點字版的程序表和詩歌集.
還有聖經.
讓他們可以和大家一起參與崇拜.
這個第一和第二層次.
出席和參與.
是很著重硬件的設備和配套.
和環境的配合.
而去到第三個層次.
「交流」.
是指有人與人之間的關係.
接觸.
和產生一些情感的連繫.
例如在教會裡.
不同能力的人士.
可以參與團契的活動.
甚至可以參與一些的侍奉崗位.
其實如果教會去到這個層次.
是很難得的.
不過其實還有進步空間.
因為我們可能沒有太大留意.
大部分情況下.
他們參與和服侍的模式.
並不是按照一般人.
習以為常的模式.
而不是按照他們的獨特性.
而展現出來的模式.
這些模式.
未必可以發揮到他們獨特的恩賜.
去到第四個層次.
社會上說「相見共融」.
最高層次是「協作」.
Interdependence.
了解對方的特性.
互相尊重.
甚至可以去到互相依賴.
欣賞和鼓勵.
這幾個層次的「相見共融」.

$^{241}$背後有一個基礎.
就是一個平等的概念.
社會上說「相見共融」.
是基於平等概念.
每個人都是平等的.
是一種普世價值.
所以你給予對方機會.
不是一種憐憫或施捨.
而是一個公義的表彰.
所以推展「相見共融」.
不單是維護相親人士的權益.
而是維護整個社會的公義.
透過共融的環境.
社會才可以真正達到.
公民互相接納和尊重.
我想大家注意.
這裡的第四個層次.
「協作」和「獨立」.
是最接近我剛才所說的.
擁抱殘疾.
相向共融的願景.
在教會群體.
共融有什麼不同呢?.
有信仰基礎的「相向共融」.
所說的相見之間的關係.
就像身體裡的手足.
沒有誰幫誰.
其實大家都需要對方.
沒有對方是不行的.
「相向」的不是一味服侍對方.
在這個願景裡的信仰基礎.
不單是基於人人平等.
而是關乎基督教的人觀和教會觀.
我們相信人是按三一神的形象去做的.
所以基督教的人觀是有群體互動的向導.
教會觀亦是一個.
連於基督的身體各人互為肢體.
所以說沒有殘疾人士是不行的.
因為基督的身體沒有某些部份.
會是怎樣呢?.

$^{281}$變成殘疾.
那時真是殘疾的教會.
所以除了人人平等之外.
作為信徒我們要知道.
我們有這些信仰基礎.
還有一點就是一個更大的推動原因.
聖經說到提醒我們.
不要欺壓弱勢的人.
因為不能忘記他們背後是有一位主.
所以基於信仰的相見共融.
亦是出於對神的敬畏.
患有間歇性肢體癱瘓的神學家Cathy Black.
在她的一本書.
Healing Homiletics.
Preaching and Disability 裡面.
提出Theology of Interdependence 的概念.
我將它翻譯為「相互依賴神學」.
Black 指出教會慣常.
以服事,照顧的出發點.
是接待殘疾人士.
基本上是假設了是單方面的付出.
所以你經常聽到教會說.
「我不夠資源」「我不夠專業」.
所以就覺得不懂得做了.
這種單向服務的固有模式.
其實是反映教會受西方文化高舉獨立的思想所影響.
忘記了人類真正的本相是怎樣呢?.
原來在社群當中.
沒有一個人是完全獨立的.
不是他們需要我們.
其實我們彼此都是需要.
我們不是走到一個.
我是健全的,我獨立.
比別人高一等.
不是這樣的.
正如聖經以互為肢體形容信徒的群體.
在臨前十二章說.
眼不能對手說.
我用不著你.
頭又不能對腳說.

$^{321}$我用不著你.
不但如此.
身上肢體人以為軟弱的更是不可少的.
雙向共融是指不可以沒有了你.
沒有了殘疾人士不可以的想法.
其實在我們的社會是很難明白的.
在一個高舉個人主義和目標違本的社會.
根本想像不到我們會這樣想.
大家覺得教會能否掌握呢?.
舉個例子你很快會明白教會能否掌握.
假如我們教會發現沒有年青人.
大家會很緊張.
快點檢討一下.
看看有什麼改變.
以至我們能夠接待年青人在我們當中.
但如果我們教會沒有了傷殘人士.
我們會有什麼反應?.
就是沒有反應.
原來教會未必能夠揣摩到.
這種相見互為肢體的概念.
我們不覺得沒有了這群人.
我們會變成殘缺.
沒有了殘疾人士是不行的.
當我們不去接待當中的殘疾人士的時候.
我們的教會會變成殘疾.
我們自己也無法擁抱真正有缺陷的自己.
神學家侯父士也呼應這種概念.
侯父士對於殘疾人的論述.
大部分是關於智力障礙人士.
他認為智障人士是神擺放在我們中間的先知性記號.
要提醒我們不要再活在自給自足的假象中.
智障人士獨特的角色是什麼呢?.
就是他們很自然地.
能夠毫無遮掩地展露自己的需要.
他們可以堂堂正正地活在一個依賴的狀態下.
不需要遮掩自己的需要.
侯父士說,如果我們懂得和這群朋友相處.
我們會學習到如何和不完美的自己相處.
會學習到接納自己也有需要.
在群體當中,我們會學習到.

$^{361}$以相互依賴的狀態互相連結.
學懂如何生活.
否則,侯父士說,我們肯定會成為遲緩.
如何遲緩呢?有什麼做不到呢?.
就是沒有能力為自己的需要轉向神和轉向人.
通常,我講書講到這些理論.
同學都會覺得,聽起來很對.
不過很理想,很超現實.
相互依賴可以如何發生呢?.
想不想像到殘疾人士可以如何服侍其他人呢?.
有時,有點難去想像.
所以,我決定在這個講座之後.
一定要搞一個有血有肉的工作坊.
這是直入式廣告.
明早的三向共融真人圖書館.
將會有五個單位.
邀請了15位不同身份角色的分享嘉賓.
展現不同殘疾人士的生命經歷.
如何體會到,就算有張力,有障礙.
但仍然體會到雙向共融.
我很期待,希望大家不會錯過.
我們剛才討論過何謂擁抱殘疾.
提出了雙向共融的願景.
共融不單止是為殘疾人士的好處.
而教會作為一個基督群體.
在這個願景引導之下.
才可以活出教會的本相.
我們接著看看雙向共融的障礙.
其實有很多方面的障礙.
不過,今天我會集中看看.
在我們的信仰系統裡.
出了甚麼扭曲的解經或神學.
以致構成障礙呢?.
我們做個小測試給大家.
你面對病患或殘疾時.
會否產生以下的想法呢?.
你會否想,一定是自己信心不足.
所以祈禱醫治不好.
又或是想,一定是被鬼附呢?受咒咒呢?.
有時會想,有這個殘疾是否因為犯罪呢?.

$^{401}$亦有時聽到別人勸自己.
或自己也會這樣勸人,不要傷心.
受苦一定是有神的旨意.
要信佛,要上上喜樂才是好的基督徒.
又或是想,有殘疾的人是不潔的.
不可以接近祭壇去侍奉.
最近我跟一位姐妹分享.
她累累癌症病危時.
身邊有很多弟兄姐妹為她們祈禱.
弟兄姐妹的祈禱都是出於好意和關心.
卻沒有為她們的內心帶來平安.
反而帶來很多困惑和受苦.
為甚麼呢?.
當中有不斷質疑她是否犯罪.
追問是否有偶像未拆清.
又或是勸她不要傷心.
因為要信得過,苦難一定是神的美意.
有信心就不應該傷心.
要信佛,要喜樂.
聽起來很正路,是嗎?.
為甚麼會帶來這麼大壓迫感呢?.
甚至成為雙向共慾的障礙呢?.
其實,這裡列出的幾個說法.
某程度上都有聖經根據.
有時候,情況真的是這樣的.
不過不是一定.
當我們把它當作定律.
隨便應用在其他人身上.
或是自己身上的時候.
就變成了一種扭曲的解經和神學.
很可能會帶來傷害和障礙.
因為當我們扮演神.
我們要判斷發生甚麼事.
判斷身邊的人遇到的情況.
一定是怎樣怎樣.
這些想法得出來的結論.
很多時候就是將殘疾人士當成犧牲品.
沒有理會到他個人的真實狀況是怎樣.
強行要把一套自己判斷的答案.
加諸他們身上.

$^{441}$群體這種所謂的關心.
反而成為了壓迫.
令人不想和這個群體連結.
直接阻礙了雙向共融.
Nancy Lane 是一位有殘障的神學家.
她提醒我們.
殘疾不一定是障礙.
不等於障礙.
不過我們要提防那些製造障礙的神學.
她稱這些壓迫殘疾人士的神學為.
Victim Theology.
不過其實要提防也不是那麼容易.
因為最麻煩的就是她的想法.
表面上看來也有道理.
例如第一行扭曲.
認為病人信心不足.
所以病患不得醫治.
你可能會想不是嗎?.
在《馬太福音九章》.
耶穌醫好患有12年血流的女人的時候.
之後明明是跟她說.
你的信救了你.
沒錯,這個女人的信心救了她.
她得到醫治.
但不代表所有祈求但得不到醫治的人.
是因為沒有信心.
也不代表所有有信心的人.
祈禱就會得到醫治.
保羅是一個很好的例子.
他三次求神拿走他身上的一根刺.
解經家認為極有可能是說.
他身上的殘疾或病患.
但神也沒有拿走.
當我們高舉信心禱告一定得醫治的時候.
我們要小心不要墮入成功神學的圈套.
成功神學是崇拜財富和健康的.
認為財富和健康是等同神的祝福.
沒錯,我們可以盡管憑著信心去求.
但能否拿走我們的殘疾病患.
是由神決定的.

$^{481}$時間也是由祂決定的.
醫治有時,不醫治有時.
其實當我們細心看.
耶穌在科興書記載了很多醫治的神蹟.
但祂有沒有治好所有病人和殘疾人士呢?.
是沒有的.
有解經家認為,耶穌的醫治事件.
重點其實不是病得醫治.
而是尼塞亞身份的表徵.
原來當你看《以賽亞書》35章5-6節.
裡面提及幾個指向神必定來報仇的標記.
那裡說神必來報仇.
必定施行極大的報應.
必定來拯救以色列民.
接著就有以下一個清單.
何時會發生這些事呢?.
那時,核子的眼必睜開,龍子的耳必開通.
那時,瓊子必跳躍著鹿.
瓦巴的舌頭必能歌唱.
原來耶穌的醫治事件.
可以說是報仇的日子的標記.
是耶穌尼塞亞身份的表徵.
這些記載的重點不是所有病得醫治.
更不是代表神要取走我們的困苦和病患.
我們信的神說明了.
我們在世界上繼續會有苦難.
不要忘記,我們的神就算被釘在十字架上.
仍然可以成就救恩.
連死亡都阻攔不了神的祝福.
更何況疾病和病患呢?.
第二種妨礙相向共鳴的扭曲.
就是隨便把病人當作被鬼附或被咒詛.
當中犯的錯誤.
亦是把一段經文的描述.
隨便套在其他情況下.
《馬太福音》第九章三十二至三十三節.
記載有人把被鬼附的啞巴帶到耶穌面前.
耶穌把啞巴鬼趕出去.
那個人就說話了.
在這裡我們看到一個關於啞巴鬼的記載.

$^{521}$今天我們不會隨便看到啞巴鬼被鬼附.
我們不會這樣想.
不過當我們遇到一些解釋不了的疾病.
例如嚴重的精神病.
而病人的病徵又讓人感到不安或驚慌的時候.
我們有沒有不經考究.
就將它歸納為邪靈教要呢?.
在這裡分享一下.
一位歐陽太的經歷.
歐陽太是一位基督徒.
其中一個女兒有嚴重智力障礙.
加上有自閉症和過度活躍症.
平時的照顧已經十分困難.
試過有一段時間.
女兒回到家裡.
一踏入門口就大叫大叫.
情緒近乎失控.
但是她離開了家裡出外.
或者回到宿舍又好好的.
如是者,幾個星期都是這樣.
媽媽也沒有辦法了.
身邊有些親戚朋友開始這樣跟她說.
她說:你家裡是不是有些髒東西?.
要不要找人回來處理?.
歐陽太覺得她還要研究清楚一點.
於是她就很細心地看大廈附近周圍.
有什麼環境和異樣.
因為她之前好好的.
終於被她發現她家附近有個單位.
窗口位置伸出了很多發射器.
相信是被電信公司租來做市區發射站.
一般人不會有什麼特別反應.
但是她女兒特別敏感.
於是受到訊號的影響.
歐陽太就用盡她的方法.
包括去立法會請願.
結果被她成功地.
要這間電信公司搬走了.
之後女兒回到家裡就沒有再大叫了.
安靜了.

$^{561}$原來我們對事物的理解真的很有限.
雖然科學已經很進步.
但是對於某些身心靈的狀況.
是一些非一般的現象.
例如思覺失調.
我們其實還有很多東西需要去探究.
或者我們可以向這位太太學習.
用一個開放的心.
求知的心.
去嘗試認識和理解.
不同殘疾群體.
他們經歷的世界究竟是一個怎樣的世界.
我們接下來看第三種妨礙雙向共融的扭曲.
是宣判你有殘疾.
一定是你犯了罪.
其實聖經有很多經文.
是在說神用病患去懲罰罪人.
其中一個例子是歷代之下21章.
才知道伊利亞達遜痛於難忘.
因為他犯了罪.
他建了一些休壇.
去引導猶太人和耶路撒冷的居民去拜偶像.
所以神要降大災.
並且要用病患懲罰他.
才跟他說你的腸必患病.
日加沉重.
以致你的腸子墜落下來.
除此之外還有其他例子.
神是用病患去懲罰人.
這裡舉一兩個.
歷代之下和列王紀夏.
都是這兩個王患大麻瘋.
因為得罪了神.
沒錯,神是可以用病患去懲罰人.
但不代表人一定因為犯罪而成為殘疾.
約翰福音9章記載.
生來核眼的人就是一個例子.
旁邊的人包括門徒都在想.
他們問耶穌.
你被這個人生來核眼.

$^{601}$究竟是誰犯罪呢?.
不是問他是否犯罪.
已經是假設了有人犯罪.
究竟是他還是父母呢?.
耶穌幫他們平反.
不是他犯罪.
也不是他父母犯了罪.
而是要在他身上顯出神的作為.
當然,在病患和殘疾當中.
我們都要在神面前醒察.
讓神的亮光去光照我們.
讓我們看到自己有什麼需要去轉向.
事實上人生遭遇的苦難.
都是邀請我們轉眼看上帝.
去聆聽祂要跟我們說的話.
但是,如果我們還沒有聽清楚.
神在這個獨特處境裡跟我們說的話.
就自以為是妄下判斷.
我們就好像約伯的朋友一樣.
未審先判.
成為落井下石的.
幫不了,反而是加害的朋友.
關於罪和病患殘疾的關係.
我們還要思考另一個層面.
就是我們生活在墮落的世界裡.
人類的罪惡製造了各種不理想的環境因素.
使人去患病.
包括空氣污染,食物,食水有毒.
這些病可以說是由人類的罪所導致.
並不是單純源於個人犯罪所引致.
剛才我們看這三種扭曲.
它們有一個共通點.
就是人總要為殘疾人士的狀況.
找一個自己認為合理的原因.
總覺得應該有個原因.
還要我能夠明白,能夠接受的原因.
才能收貨.
殘疾人士除了要為應付自己的狀況.
已經很不空閒,很困難.
還要為未必是真正原因的原因去負責.

$^{641}$最慘的是.
他們不能表達自己的疑惑,傷心,憤怒和抗議.
第四種扭曲是.
不要傷心,信服.
受苦就是神的心意.
背後是假設了.
悲傷,愛好,憤怒.
這些一定是沒有信心的表現.
這裡一定要提一句.
我們經常用來安慰人.
但其實可以帶來壓迫的聖經.
有人稱之為「淑寧萬金油」.
你知不知道是哪一句?.
可能你也猜到.
萬事都互相效力.
萬金油是甚麼事都好.
茶少萬金油.
萬事互相效力.
叫愛神的人得益存.
其實用這句經文本身是沒有問題的.
但當你用這句經文叫人不要再傷心,憤怒.
不准問那麼多問題.
這樣就出問題了.
可能對方的憤怒和悲傷令你不安.
但你不要忘記.
那就是他現在真實的自己.
神從來都沒有禁止人向他申訴.
連耶穌在十字架上.
他都大聲喊著問神.
我的神我的神為甚麼離棄我.
事實上.
唯有當我們誠實地面對神.
在我們的苦況裡.
誠實地向他和他對話.
我們才可以和他相遇.
而和他相遇.
就會帶來醫治和更新.
最後我們看第五種扭曲.
就是認為殘疾人士是不潔的.
不可以參與接近祭壇的侍奉.

$^{681}$事實上教會歷史裡的確有這些想法.
認為殘疾人士不配做某些侍奉崗位.
例如不可以成為牧師.
不可以講道.
帶領崇拜等等.
甚至有一間教會.
因為有個施班員走路.
覺得他進入禮堂時.
因為要列隊進入.
覺得不體面.
就勸退他.
不過其實這些做法.
已經觸犯了很多地方的反歧視法.
這種扭曲其實直接成為雙向共和的障礙.
因為你不讓他侍奉.
剝奪了殘疾群體實踐使命的機會.
為什麼會有這些扭曲的想法呢?.
其中一個原因是.
舊約訂立的祭祀的憲制規矩.
除了吩咐以色列人.
不可以拿殘疾的祭物去獻給神.
在《利美記》21章16至21節.
神確實吩咐.
不可以讓殘疾人士前來獻祭.
然後對摩西說.
你告訴亞倫.
你世世代代的後裔.
雖然亞倫是先知.
可以做祭司的這群人.
但如果他有殘疾.
都不可以來獻神的食物.
凡有殘疾的.
無論是黑眼的,瘸腿的,鼻子的.
肢體有餘的,折手指腳的.
駝背的,矮錯的,眼睛有毛病的.
即是我這些.
長善的,長介的或損壞神志的.
都不可近前來.
祭司亞倫的後裔.
凡有殘疾的都不可近前來.

$^{721}$將火祭獻給耶和華.
他有殘疾.
不可近前來獻神的食物.
很清楚的.
很多次「不可不可」.
這些規定帶來一個聯想.
就是殘疾人士不潔.
所以他們覺得不可以侍奉.
其實這個說法也不完全錯.
你說殘疾人士不潔.
為什麼?.
因為我們所有人都不潔.
無論有沒有殘疾.
我們本身都不配來神的壇前侍奉.
我們有什麼資格來神的壇前呢?.
完全是因為基督耶穌救贖了我們.
用祂的補血遮蓋我們的罪.
我們是因為披戴基督.
才可以來到神的面前侍奉.
所以要注意.
這些祭祀的規定.
是用來預表基督是完美無缺的祭品.
和毫無玷污的祭司.
而不是用來引申到殘疾人士是次等.
當日我們以為身體上沒有殘疾.
就可以配得侍奉.
這樣就是扭曲了聖經的教導.
一切都是因典.
我們根本沒有一個人配得侍奉的位份.
主的能力在人的軟弱上顯得完全.
所以保羅甚至說.
「我更喜歡誇自己的軟弱.
好叫基督的能力奉給我」.
總結一下第二點.
扭曲的解經和神學.
構成對殘疾人士的歧視.
信仰群體不單止不能擁抱殘疾.
更加雪上加霜.
要他們自己獨自承擔責任.
在此我想指出一點.

$^{761}$原來有時我們都會用這些歪理去壓迫自己.
去擅自批判自己.
當自己覺得自己不完美的時候.
去指控自己.
其實這些壓迫都會令我們同神疏遠.
或者同群體疏遠.
甚至因為不能真正面對自己.
而同自己疏遠.
直接或間接成為雙向共融的障礙.
說到第二點.
可能你會想接下來會不會是說.
應該是說怎樣去拔走這些障礙.
讓教會成為理想的教會.
可以擁抱殘疾.
達到雙向共融.
那就完滿大結局.
是不是這樣呢?.
不好意思.
其實是沒有這麼簡單的.
沒錯,有些障礙我們可以藉著硬件的配套.
可以努力去排除.
好讓殘疾人士可以來到我們當中.
可以出席,參與,參與.
不過,去到一些根深蒂固的概念.
你會發覺我們就是沒有能力去改變自己.
教會事實上是有殘疾,有障礙.
所以我想提出的實踐的向導是.
首先我們要擁抱這個.
不是那麼完美.
有障礙,有殘疾的教會群體.
不要因為他不懂得接待殘疾人士而嫌棄他.
要相信聖經說.
我們彼此都需要對方.
雖然有障礙.
大家仍然要學習堅持在破損當中互相連結.
不需要對方完美才去接納他的群體生活.
就是我們在連結當中.
可以創建一個有因典的空間.
而在這個因典的空間裡.
人是不會隨便被批判的.

$^{801}$可以按著我們的本相同神相遇.
而這種相遇會為我們帶來醫治和牧養.
今天的分享是想為相見群體的連結打個底.
大家要留意.
我們不知不覺之間.
會有時為他人的殘疾狀況下定論.
有時還覺得自己在幫他.
就好像若白的朋友.
其實我們有時這些判斷.
是扼殺了對方和神相遇的空間.
我們有時低估了.
我們不需要那麼多判斷.
其實單單是一種群體的生活.
一起去連結已經很有力量.
當信仰群體用愛接納和擁抱殘疾人士的時候.
會提供一個與神相遇的空間.
在神選擇的時候.
人會在這個空間裡找到屬於自己的答案.
在他與神之間對話之後的答案.
在我的經驗裡.
我曾經見證有特殊兒童的家長.
分享他的領受.
其中一位分享說.
他的小朋友的狀況.
令他悔改歸向神.
因為他說之前他信主.
但他忘記了神.
直至小朋友出現提醒他.
他自己想就行.
你不要將他的答案告訴另一個人.
你忘記了神.
所以神給了你這個小朋友.
這樣就非常不可以.
要記住.
是他自己從神領受才可以.
我們要做什麼呢?.
就是給他一個空間.
接納他.
以愛與他同行.
不要幫他拿判斷.

$^{841}$另一個家長又有不同的領受.
他說.
神告訴他.
他要經歷這些是要裝備他.
去幫助其他遇上同樣困難的人.
原來領受到這個答案.
不單止他得了醫治.
他還有一份使命感.
但不要隨便將這個使命感.
套在別人身上.
否則他會說.
這個使命留給你.
最近我認識了一位媽媽.
她領受了答案.
我從來沒有聽過.
她說她的領受是神告訴她.
神要賠償給她.
其實我自己也在揣摩這個答案.
但我觀察到.
當人自己從神領受.
屬於自己的答案的時候.
他不單止得了醫治.
他有時還會領受到一個獨特的使命.
當然,不一定每個人都要有答案.
擁抱殘疾的群體.
就要容許每個人.
在自己生命的苦況裡.
繼續向神申訴.
繼續問問題.
不要告訴他.
你問了這麼久.
你應該收手吧.
不是的.
要給他空間.
我們千萬要記住.
不可以將A的答案套在B身上.
否則會變成另一種扭曲和壓迫.
在破損裡.
堅持要彼此連結.
其實是需要很大勇氣的.

$^{881}$大家的生命都是硬硬的.
連結在一起又如何呢?.
不是一拍即合的.
還會時不時的被慘.
在完結的時候.
我想和大家分享一個真實的事例.
希望帶動大家繼續思考.
以下是用化名的例子.
主角是兩姐妹.
妹妹葦葦有嚴重智力障礙.
但她能走路.
亦很聽姐姐的話.
姐姐信了耶穌很多年.
她很想妹妹可以回教會.
信耶穌和受浸.
她回教會.
因應她獨特的情況.
作出了一些安排.
讓她兩姐妹一起上一個.
度身訂造的洗禮班.
之後.
葦葦就在一群弟兄姐妹的見證下受洗.
姐姐就很開心.
最初他們參與教會聚會的時候.
大家是有些擔心的.
不過都很快習慣了.
葦葦是間中會叫一兩聲.
間中興奮起來會彈起身.
大家都適應了.
不過.
有一次有位外來的港元來港報道會.
就出事了.
其實姐姐都很仰慕這位港元.
等了她聽她講很久.
那天兩姐妹一如既往就坐在大堂正中.
因為我跟她說.
這裡任你坐.
不用別人告訴你坐在玻璃房.
你喜歡坐中間就坐中間.
因為你自己選擇.

$^{921}$就坐在大堂正中.
因為港元很有經驗.
她不用講稿.
還有她有個習慣.
喜歡拿著咪高峰在港台前行來行去.
跟人有互動.
問題出在哪裡呢?.
葦葦就見到港元行來行去.
她就很興奮.
興奮到彈起身.
大聲叫.
還維持了一段時間.
當時港元的反應令到大家都很驚訝.
港元就說這位妹妹情緒不穩定.
不如請家人先帶她出去.
當時大家都不懂得反應.
沒辦法.
姐姐就帶了妹妹出去看直播.
當時我覺得教的電影節目都很有型.
有班人跟她一起走出來.
之後就有人請我跟姐姐聊天.
姐姐就傷心.
覺得被人拒絕.
而且是一個很仰慕的港元.
不過她說教的電影節目都很好.
走出來安慰她.
跟她說我們都很愛你們.
我跟姐姐聊天的時候提醒她一件事.
你有沒有想過這位港元.
可能都有她的限制呢?.
港元都有上一年多.
雖然她很有經驗.
但葦葦這麼興奮的表現.
有可能令她無法集中.
想不到繼續講下去.
那報道會怎樣收場呢?.
可能是她迫不得已.
唯有這樣做才可以繼續講下去呢?.
我們很想其他人擁抱接納葦葦的殘障.
我們可不可以學擁抱這位港元的殘障呢?.

$^{961}$結果這種想法的導致.
令姐姐突然豁然開朗.
即時釋懷.
不單原諒對方.
還體諒到別人有時看似不接納妹妹的表現.
可能都是因為對方某方面的限制.
群體牧養不是為殘疾提供解釋或批判.
而是在障礙破損中堅持彼此要連結.
擁抱和同行.
讓大家在生命當中互相看見對方.
聽到對方的聲音.
彼此用愛去接納對方的限制和障礙.
彼此饒恕,聆聽.
連結成為盛載基督尊貴的器皿.
好像今屆的藝術一樣.
這種醫治不是還原或取走殘障.
而是以愛去擁抱破損成為一件新的創作.
什麼叫做擁抱殘疾的教會呢?.
今日我和大家提供一個雙向共融的願景.
嘗試了解我們遇到的一些障礙.
雖然專業知識和資源都很重要.
但最重要的是我們願意踏出一步.
在我們的局限,障礙中嘗試擁抱彼此的障礙.
一起連結.
期待經驗神的同在.
祂臨在的時候就會轉化我們群體的生命.
讓我們經歷從祂而來的醫治,牧養.
我們的盼望不在乎我們多有愛心,多有能力.
而是仰望在困苦殘疾中和我們同行的主耶穌.
祂是化腐朽為神奇的生命藝術家.
醫治,牧養從祂而來.
而我們也可以用謙卑的心.
放下自己對生命應該是怎樣的執著.
開放,接納來到我們當中的每一個人.
說不定這個我們接待的人.
會帶領我們去到耶穌的面前和祂相遇.
願神憐憫,祝福我們.
多謝大家.
好,大英姐妹平安.
好,感謝Vance老師的分享.

$^{1001}$很開心今天可以做一個回應.
我的題目是選了剛才Vance老師提到的一個詞.
因為我作為一個教神學的人.
其實都很觸動到我.
我是很認同.
其實神學好像是一些理論.
或者是我們的信仰.
我們會覺得是真理.
但是可能我們所說的神學.
其實只能反映我們某一些問題.
我們一些文化.
或者甚至是一些偏見.
所以我的題目就是.
剛才Vance老師提到的製造障礙的神學.
我純粹在這裡都是舉一些例子.
因為我覺得剛才Vance老師都已經說得很好.
都說了很多.
我想說的例子就是教會觀和人觀.
例如教會觀.
有些什麼是我們沒有意識.
但可能製造了一些障礙.
就是我們很多時候.
傳統教會或者福音派慣常.
我們去理解教會就是一個得救的群體.
怎樣加入教會就是受洗加入教會.
怎樣可以經過受洗的步驟呢?.
首先就是要「缺志」.
其實華人教會或者福音派.
其實就是將很大的重要性.
給了這個缺志的這一刻.
而缺志其實就是.
很需要那個人真的有那個智力.
他有那個能力.
是可以做到這個信心的決定.
不單止是要去做那個缺志祈禱.
甚至很多時候教會為那些想受洗的人.
其實安排了一條很長的路.
首先就是他要認信.
他要上很多的主一學.
主一學很多時候都是一個課室的設定.

$^{1041}$要看看他是不是舒服在那個設定那裡.
要寫見證.
甚至我聽過要懂得寫文的人.
要通過一些的考問.
很多一些這樣的要求.
那個原意是好的.
在教會歷史裡面.
為什麼有一些傳統發展出來.
是對加入教會的人有這樣的要求.
是因為都是想確保教會的每一個會友.
都是清楚信仰,清楚得救.
但是就很不經意地.
其實就是我們將很多的障礙.
放在人的面前.
聖餐也是.
聖餐因為我們傳統以來.
都是覺得只是給已經受洗的信徒.
所以我剛才說的所有的障礙.
其實都是同樣適用.
就變成同樣有這個問題.
什麼人可以守聖餐呢?.
也是剛才能夠通過到那麼多要求的人.
才可以和教會的大家庭一起去守聖餐.
甚至更加慘的是什麼呢?.
就是傳統教會很多時候一去想聖餐.
就是慣常只是看保羅,哥倫多前書的教導.
特別是說那些不能夠分辨主的身體的人.
他們就是在吃喝自己的罪.
其實是很大的一件事.
就令到教會有時很緊張.
究竟是什麼人可以來守聖餐呢?.
有什麼人是否我們應該要攔阻他們呢?.
有時就是基於神學的原因.
就是不自覺地,沒有意識地.
但其實是將一些不必要的障礙.
放在人的面前.
特別是聖餐和受洗.
除了說是一個堂會所認可的會友.
和一些我們覺得已經清楚得夠的人一起參與之外.
其實我們也是在說一個大公教會.

$^{1081}$或者是神的監.
但當我們將某一些人排除在這個過程.
這個加入教會的過程的時候.
其實原來我們是不畏異的.
但其實我們在給一個訊息給人.
就是神的監或者基督的身體.
原來是排除了一些人.
例如排除了小朋友.
因為我們成年人.
經常都覺得他們不能夠真心的決志.
或者是不能夠真正明白到聖餐的意義.
或者一些有認知障礙的人士.
其實原本應該是將教會的家.
基督的身體連在一起的洗禮和聖餐.
其實就是將某些人排除在外面.
人觀也是.
基督教人觀有時我們會想.
神的創造,創世計劃是美好的.
美好我們很多時候有一個馬上的假設.
就是這個美好其實等於完美.
所以任何我們看到的限制.
好像是需要解釋的.
就是怎樣都好像剛才Rez老師說的.
不同的迷思.
就是我們覺得有病或者有限制的殘疾.
就是需要找一個屬靈的解釋.
是不是犯了罪呢?.
是不是有些什麼東西呢?.
總是覺得任何的限制.
都好像不是符合神的心意.
另外人觀我們很多時候就是講神的形象.
就是創世計劃.
人是以神的形象被造.
在教會歷史裡面其實很不幸地.
有一個很長很長的傳統.
就是我們去理解什麼叫做神的形象的時候.
方法是怎樣呢?.
就是去找人和動物有什麼不同的地方.
也很不幸地因為這樣.
我們基督教有一個很長的傳統.

$^{1121}$就是以為人有思考能力.
有分析的能力.
這就是神的形象.
所以他們就會覺得.
很多時候我們就會覺得.
是不是人之所以有這麼尊貴的身份.
有這麼高的價值.
是因為我們有思考能力.
是因為我們有一些特別於動物的能力.
所以我們才是尊貴呢?.
如果你真的去想深一層.
很多人都是這樣想的.
但其實這樣想是很歧視性的.
因為你是用能力.
特別是用思考能力去定義一個人的價值是多少.
所以這給我們的提醒也是.
其實神學我們是需要不斷去反省的.
在教會裡面有時我們一些很習以為常.
已經很久的傳統的一些神學理念.
無論是有關教會也好.
有關人的定義也好.
其實有可能是反映了我們的文化.
例如我們香港人是很緊張讀書的.
很緊張成績的.
很自然我們有時會想.
究竟一個好的基督徒.
或者一個好的人.
我們就會將一些成就和某一方面的能力.
去跟一個人的價值拉上了關係.
其實這真是我們的偏見.
這未必是聖經的教導.
例如我們要去理解神的形象.
為什麼不選用其他的特質呢?.
人和動物其實都有很多不同的東西.
例如愛,我們懂得去愛.
笑容,如果我們用笑容去定義人的尊貴身份.
應該反而是某一些弱智人士.
可能他們的笑容是更加真,更加美.
為什麼不用這些去定義人呢?.
例如謙虛,小朋友他們的謙虛也是.

$^{1161}$都是和動物不同的.
為什麼不用這些特質去定義人呢?.
其實我們都要反省.
有時我們一些神學的框架.
其實是反映了我們自己的問題.
至於剛才Wes老師所說的.
雙向共融的出路是怎樣呢?.
我不敢說我有什麼出路.
不過只不過作為一個神學的人.
去建議一些可以重新去思考的方法.
例如剛才其實都有提到.
我們真的不要忘記.
我們教會裡面每一個人其實都是不配的.
沒有人是配得來到神的面前的.
為什麼我們有時會覺得.
例如我們可以決定另一個人.
究竟他是否有資格加入教會.
或者他有沒有資格去接受聖餐呢?.
其實根本決定那個人.
他自己都是不配的.
在我教洗禮和聖餐的神學的時候.
我都很鼓勵同學.
就算你的傳統是不認同嬰兒洗禮也好.
其實是很值得我們去思考.
為什麼在教會歷史裡面.
一直以來都有很多傳統是很擁抱這個理念.
其實嬰兒洗禮的意義就是.
不是要那個人能夠付出到什麼.
或者要達到什麼人所定的標準.
或者一些我們對人有的期望.
他才能夠加入教會.
本來教會作為基督的身體.
基督如果沒有去排斥任何人去到祂面前.
其實我們都不應該排斥任何人.
能夠去到神的面前.
所以這個我都會邀請同學.
都嘗試去理解一下嬰兒洗禮.
其實你可能有一些神學不認同他的做法.
但是他都有他的意義.
他都是在表達福音.

$^{1201}$原來是白白的.
另外就是我自己的教會.
我們都有這一方面的討論.
我們其實都是希望做到一件事.
就是將洗禮或者浸禮.
這個聖禮和會籍脫勾.
因為很多時候傳統教會是將這兩件事.
放在同一個動作裡面的時候.
我們會將一些我們對一個.
能夠參與教會運作的會友的要求.
放在我們對那個人能不能夠洗禮.
能不能夠獲得通過去接受他.
去受洗.
就是這樣拉上了關係.
但其實如果我們相信洗禮是.
那個人去加入神的家.
或者是加入大公教會.
其實是不應該受一些不必要的障礙去阻止他.
人觀都是.
其實人必然是有限的.
這其實是創造觀的一個很核心的信仰.
很多時候我們看到限制好像是需要解釋.
但其實反而如果你看到一個人沒有任何限制.
反而才是需要解釋的.
還有聖經從來都沒有任何的特徵去定義神的形象.
我覺得這個也是創世的一個很有智慧的做法.
就是特地不用任何的特徵去定義.
因為你一用任何的特徵去定義.
其實結果就是一定會有些人是多一些或者少一些的.
我覺得我們的人觀都是有一樣重要的.
就是我們人的價值不是由我們自己去定義.
是由神去定義.
而神是很樂意去使用軟弱的去啟示他自己.
在耶穌基督作為一個鷹鞋.
這個在事跡上已經很明顯了.
我最後想去分享兩段的經文.
我們看到神很親密地和人相遇.
去揀選人.
但正正就是因為這個相遇.
是很有趣的.

$^{1241}$我們剛才說.
就算人是有某些方面的限制或者殘疾也好.
我們都是需要.
作為一個群體其實我們是很需要他們的.
甚至他們的狀態.
其實完全不是一些我們覺得是要去找罪.
或者找其他的原因去解釋.
亞各我覺得是一個很有趣的例子.
其實正正就是因為他和神相遇.
正正是因為他被神揀選.
所以聖經這裡這樣說.
所以他的大腿窩從此扭了.
他的傷勢不單止不是因為他犯罪.
甚至聖經很誇張地說.
是他和神相遇.
有一個很親密的相遇.
甚至說他是勝過了神.
而這個他的大腿窩的傷.
其實你可以說是一個記號.
就是他和神相遇.
和神揀選了他.
甚至說他幫那個地方起了一個名字.
就是比魯尼.
就是說我曾經面對面見到神.
最後那個例子就是摩西.
摩西也是在著火的荊棘同層相遇.
很多時候我們對摩西的印象.
都是一個很大膽,很勇敢和能言的人.
但是聖經其實強調給我們聽.
其實他是絕口不說的.
甚至他去對神埋怨.
他都說其實自從你和我說完之後.
我都是這樣.
就是神給他的揀選.
神在這個荊棘向他顯現.
是完全沒有拿走他的問題.
甚至乎後來因為時間關係.
我們不再多說.
但都鼓勵你去看英文.
就是《初一及二》第四章和第七章.

$^{1281}$其實神是兩次.
當摩西再去埋怨他絕口不說的時候.
神是跟他說你要代替神.
你要代替我去跟他們說話.
其實神是啟示他自己.
他竟然樂意利用一個人.
看來是絕口不說的人.
去成為他的聲音.
將神的話去告訴別人.
甚至乎其實當時的以色列人.
是不認識耶和華的名字.
其實他們第一次聽到耶和華的名字.
是透過這位絕口不說的人跟他們說.
亦都是在他們犯罪的時候.
是這位絕口不說的人去幫他們去禱告.
他們才得以在曠野生存下去.
以上都是我的分享.
(掌聲).
大家好.
介紹一下教會背景.
很感恩今天教會有弟兄姊妹在當中.
四個嗨.
教會大約有二百多位敬拜群體.
二百多人裡面有二十多三十是聾人.
其中懂守禦的弟兄姊妹十多個.
有三分一是打了幾下.
除了早安,平安之外.
小心地話都會說.
對於老師說關於雙向共融的分享.
我用了一個這樣的角度.
神藉著農建共融給我很多禮物.
我想起有四方面.
第一就是.
我們的故事.
回想二十多年前.
我們原是一間叉燒舖樓上的教會.
當第一個聾人來到.
我們後來是沒有人懂守禦的.
我們的弟兄姊妹慢慢跟隨聾人學守禦.
陸美珍在當中.

$^{1321}$她是第一個教我們守禦的人.
在這個故事之下.
我們正在經歷聖靈的帶領.
當時沒有人想過我們會發展到今天.
但當日牧者和教會的領袖.
看到有聾人來.
我們不懂溝通.
我們就嘗試用他們的方法.
他們做的第一件事就是一起祈禱.
既然神帶不同的弟兄姊妹來.
他們想跟隨我們.
一起跟從主.
一起學習真理.
我們就嘗試一下.
這個真的是群體的領袖.
在裡面慢慢學習.
加守禦翻譯.
在崇拜聚會.
或成立一些小組.
我都見證了二十多年來.
神不停帶聾人.
帶懂守禦或對守禦有興趣的人來.
我自己在盲中跟同學回來教會的時候.
都不知有聾人.
如果大家知道我背景.
父母是聾人.
所以我天生就懂守禦.
就像你們學廣東話那麼自然一樣.
很多同學去佐敦那間教會的近學校.
但我不知為何跟同學去了當時.
在長沙灣的那間教會.
原來在我來之前.
神已經一直塑造這間教會.
成為一個弄見共融的群體.
今天我覺得很特別.
當我去準備的時候.
我想起很多恩師.
除了Rance老師.
石老師,Sam老師.
其實我都想起楊醫.

$^{1361}$楊醫經常都會說.
「是的,神想這樣的時候.
祂就會做到你這樣」.
後來我就越來越明白.
原來神想我們這間教會是弄見共融.
祂就會自己做到我們這樣.
不單止我一個聾人子女在當中.
今天有另一個聾人子女也在當中.
當我想起聾人的就業問題.
甚至開一間工廠.
會請到聾人來工作.
其實這也是我從小到大的祈禱.
也是夢寐以求.
不好意思,說多了.
是的,經歷了二十多年的共融.
不代表我們很懂.
我們都經常犯錯.
我們都經常會出事.
不記得聯絡.
或者經常有新的挑戰.
當轉到網上崇拜的時候.
我們怎樣去預備版面.
很多討論要花很多時間.
甚至很多艱辛.
但就像Rance老師所說.
雙向共融的願景是很豐富.
是很多禮物.
有一次我問見聽的弟兄姊妹.
你怎麼看突然間.
教會慢慢有越來越多聾人.
有聾人在我們當中.
他回想起當他在少年時代.
看到有聾人.
他的感覺竟然是.
原來信仰是真的.
從小到大聽到神愛每一個人.
這些說法.
看到有一群人會學手語.
慢慢去服侍.
很樂地.

$^{1401}$如果告訴別人.
雙向共融是吸引年青人的方法.
我不知道大家會否努力一點.
當然我們不是用這個來招來.
但我想給我們的提醒是.
其實每個人都在問意義.
我們都在問.
我們為什麼要回教會.
我們為什麼要跟從主.
不一定是學手語.
但每一個跟從基督的人.
都是在學怎樣活出.
為他的生命.
雙向共融存在的.
本身已經是一個訊息.
都在挑戰我們.
你衛生的對象是誰.
你有什麼關注的課題.
是神放在你心裡.
我很好奇地問聾人弟兄姊妹.
其實為什麼你要留下來.
有時我們手語翻譯不是特別好.
有時都會失落.
或者有很多活動都不是很能夠共融.
我們安排不到手語傳譯.
那就不好意思了.
其實有些時候我們都.
令到可能聾人弟兄姊妹.
有不好受的地方.
但是他們都很鼓勵.
和很真實地跟我說.
他說在這裡,在外面.
在教會以外.
我們都有遇見很多.
一直都是主流的,健聽的人.
但這裡的健聽人是有點不同.
他們專門學我們的語言.
他們想了解我們.
和我們做朋友.
我們就在這裡一起跟從耶穌.

$^{1441}$學聖經,一起去侍奉.
在中晨一起說.
Diversity and Unity.
其實不只是看不同.
我們同樣受教跟從主.
聾人弟兄姊妹來到.
都是一起探索他們有什麼恩賜.
不單是用他們最耍家的東西.
去教手語.
我們是一起帶健聽的小朋友.
帶兒童聚會.
而我也是跟很多聾人去探訪.
聾人家庭的糾紛,調解.
其實我是很需要聾人跟我做搭檔.
我們一起去聆聽,一起去關心.
插花服侍.
在崇拜裡面做主席.
帶領,祈禱等等.
聾人都是慢慢慢慢地去參與.
也有很多聾人弟兄姊妹上司.
很早起床一起收水果,派水果.
我們參與由淺入深的工作.
我覺得很有恩典.
我可以在這個教會長大.
老師引述Howard的一段說話.
在書裡面.
有另一段我覺得也很感動.
他說:教會不是一個個人聚集的地方.
而是一群旅途上的子民.
他們願意在途中互相擔負.
縱然有障礙.
弟兄姊妹需要大家多花一些時間.
但神絕對不會要我們.
因為忙於改善世界.
只顧著造福人群.
而放下他們,不顧及.
甚至想深一點.
其實這兩件事不一定要對立.
與傷殘弟兄姊妹同行.
改善世界.

$^{1481}$其實更加需要一起去做.
唯有當我們與軟弱傷殘弟兄姊妹同行.
我們自己本身軟弱的生命.
才會慢慢經歷神的改變.
我們整個群體才會被改變.
我們才會知道神對這個世界的心意是甚麼.
第二點想跟大家分享.
龍勁共融給我的禮物.
就是欣賞到神創造的多元與豐富.
可能你也聽過一個說法.
正如我們初初有龍人在我們當中.
我們成立的小組叫做「失衝小組」.
只是著眼於他們聽不到.
就像兩位老師剛才提到.
我們很著重能力.
聽不到的狀況.
「不行了」.
「失衝」.
英語叫做「Hearing Loss」.
但是一些傷殘學者.
他提出另一個概念.
「Deaf Game」.
龍可以獲得,可以賺取.
有些東西是龍人才有的.
是我們沒有的.
我也很感恩.
當年在讀書的時候遇到Sam老師.
經常帶我們看畫.
他也帶過教會的龍人弟兄姊妹去看.
「真的,原來說Deaf Game」.
「龍人的視覺特別敏銳」.
「特別看得到」.
「你看得到有多少人嗎?」.
「他們除了數」.
「多了一個」.
「十二個門徒加耶穌」.
「不是十三嗎?為什麼是十四?」.
「他還會看到有一個特別看著你」.
「是呀」.
是龍人弟兄姊妹告訴我的.

$^{1521}$這些圖像,這些畫.
怎樣幫助他們明白經文.
其實我們平時去理解聖經的訊息.
都是停留在認知層面上.
其實在畫裡,在各樣的媒體裡.
雙殘弟兄姊妹.
幫助我們開闊了另一個世界.
另外是查經.
有時手語也幫助我們很多的發言.
可能你們可以看看這位姊妹.
或者看我也可以.
十個大蠻蜂的故事.
你們不會不認識.
如果你們記得經文的敘事.
開始是「遠遠的站著」.
耶穌在這裡.
「他們很遠的呼喊」.
「群體得到醫治了」.
「大家都不回來」.
「除了一個」.
「他們就跪在耶穌的腳前」.
手語其實我剛才很簡單.
一群人在這裡.
最後是一個.
「病得醫治」.
「問題得到解決」.
不一定叫人可以來到耶穌的根前.
唯有感恩的心.
可以叫人和耶穌連結.
我覺得大家有時會畫畫.
畫畫四甲.
去理解經文.
手語本身已經是這麼圖像.
這個視覺語言.
就是幫助我們更深刻去理解.
其實神藉著聖經.
想我們留意些什麼.
甚至很多我們說得爛的術語.
或者是神學概念.
恩聖青衣,天國.

$^{1561}$我們怎樣表達好呢?.
我覺得一直很感激嘉良哥.
在我還未讀中神的時候.
他已經幫忙研發聖經手語.
我們經常說天上的群體.
天國.
以前是一群人.
或者更差一點.
天家也試過.
那些是主懷安息的.
但天國的重點是什麼?.
是神的治權.
我們後來就有了手語.
手語或者是.
雙殘兄姊妹.
很多想法,很多角度.
其實都豐富了我們對信仰的理解.
而我自己呢.
都是那些異異異異的人.
但手語這種語言是很直接的.
不異異異異的.
是呀,很有趣的.
我不知道讀語言學的朋友.
是否也會知道.
手語是沒有被動句的.
那種直接的.
我覺得這種語言也在塑造民族性.
所以我跟聾人朋友的相處.
都很直接,很坦誠.
可能也影響了我.
說話也像現在這樣.
當家人一樣.
跟兩位老師很優雅.
是很不同的.
不明白笑聲是不要緊的.
是另一種優雅.
雖然有時候那種直接也傷心.
例如我也試過翻譯一下.
有聾人說不知道你在說什麼.
我記得我十多歲.

$^{1601}$在港九培靈園經大會還有一個營.
一邊翻譯一邊哭.
其實那種直接是難受的.
但你也會想知道自己翻譯得好不好.
你也不想別人騙騙哄哄你.
其實這種真誠直接的.
我姑且稱為民族性.
都是教我大膽一點.
接納多一點真正的自己.
還有不要太多修飾.
不要太多忌諱.
剛才說教會大公.
Catholic.
應該是All Embracing.
但為什麼我們教會是這麼少不同的人呢?.
同質,差不多的呢?.
接下來幾點分享.
我都有很多自己的難處和自己的發現.
有一次有位聾人媽媽.
預備帶她患自閉症的兒子.
來參與敬拜.
十多二十歲的年輕人.
他只告訴我這件事.
已經令我睡不著兩天.
我心想那該怎麼安排呢?.
他在崇拜的時候會怎樣呢?.
會不會走來走去?.
我不懂得接待自閉的人.
也很快就想到.
我知道哪間教會有自閉的群體.
可以一起敬拜.
會不會適合他呢?.
但問心其實我也是迷信專業.
昔日我教會目者,教會領袖的初心.
去了哪裡呢?.
我就在擔心.
也很真實地發現.
原來我的安全感,我的信靠.
都是自己的知識,自己的經驗.
對於聾人我就很熟悉.

$^{1641}$我從小到大都打手語的.
但對於自閉的朋友.
原來我很快就會想拒絕.
我又再想深一點.
為什麼我會這麼害怕呢?.
我害怕什麼呢?.
原來我也很想操控,很想掌握.
很想一切在自己知道的範圍內.
但回看生命的本相.
除了殘破之外.
其實生命本身也是一個奧秘.
你以為我們是同一樣的語言.
差不多的文化.
大家都是健全的,我就了解你嗎?.
其實也不是的.
反而是相殘朋友.
一個極端,很不同的身體狀況,生活經驗.
都帶我回看,其實每一個人都獨特.
我不要以為我很明白你.
就很暴力地假設你是怎樣怎樣.
或者是批判你.
但其實和不同的人相處.
相殘朋友不斷提醒我去學尊重.
而事實上,這種奧秘的生命形態.
令我們無法不承認自己的無知.
無法不謙虛.
所以,今天我也覺得.
Wenso 給我機會去反省.
給機會我們整個群體一起去學.
有時可能大家的步伐不同.
我們都可以彼此去體諒,去調節.
故事有個快樂的結局.
就是那位自閉的朋友.
來了,跟著媽媽一起敬拜.
很開心,就很開心地敲窗.
但我就坐在前面,我不太清楚.
完了之後,我問大家覺得怎樣.
很奇怪,我又覺得很多人覺得無所謂.
同工又說,下次學習怎樣接待.
原來我也傾向了我們群體的承載力.

$^{1681}$對於陌生的他者,不同能力人士.
我不是覺得我們教會是特別好或最好.
但原來神用了二十多年去訓練我們.
我們慢慢被聖靈去教導,塑造出來.
我想,正如剛才兩位老師提到.
在共融裡面最大的學習.
不是接待其他人.
其實是接納這個自己.
接納這個叫做牧養農人.
但其實不是很共融的我.
這個會沒有安全感.
當失去可以掌控的經驗的時候.
這個的我,我也很需要弟兄姊妹幫助.
剛才也說了很多,可能大家也不陌生.
我們身處香港的社會文化.
都是很高舉獨立自主,推崇能力.
我們崇拜強者,其實全球都是.
大家看Marvel,都是長大的.
即是小朋友,對不起,只是大家的小朋友.
你們沒有那麼小.
我們很難去被人幫助.
很難去呈現軟弱.
因為我們習慣用能力去看待其他人.
亦都是這樣去看待自己.
我自己最深的痛苦是.
我發現自己有一種暴力.
存在於自己的內心.
是很想毀滅一個沒有能力的自己.
我不知道是否與我的強勁的學習有關.
整個教育,整個成長.
或者後來我在FEF的運動.
甚至我在終身的學習裡.
我都不斷地問.
我是否一直在追求一個很有能力的自己呢?.
於是,就習慣逞強,死撐.
很難去認不懂,認不懂.
很難表達需要.
所以更加不懂得領受別人的幫助.
不懂得領受恩典.
與龍人一起開心的地方是.

$^{1721}$他們很直接地說.
我需要翻譯.
我需要有寫白板.
雖然寫白板是寫幾個字.
或者是.
我想參加這個訪談.
我想有手語翻譯.
很直接,不卑不亢地表達需要.
亦都不是說.
我們這些懂手語,口語,聽得到的人.
可以厲害一點,可以幫到他們.
其實我們是很需要倚賴龍人.
給予我們語言和我們分享.
尤其是很多時事,社會新聞的新詞彙.
大家知不知道這個人是甚麼人呢?.
都是我突然間.
「對呀!新特首的名字是甚麼呢?」.
台下的人就馬上.
你知道在崇拜裡面.
要忍笑都很難的.
但真的很開心.
學了新的手語.
就是在互相依存,互相承傳的關係裡面.
聖經已經說了.
基督的身體是多元而合一.
問題是我們如何體現這份多元和合一呢?.
真的.
龍人和譯者的關係都是很微妙.
但彼此不是消費對方.
亦都不是說.
我譯得不好.
龍人就像老闆那樣批評.
或者要換人.
在教會的龍見共融群體.
是超越消費主義.
亦都超越社會科學裡面講充權的概念.
不是去爭權.
雖然我們都很支持很多倡議.
但我們都學習在愛和真理裡面.
去彼此扶持.

$^{1761}$一起用愛心說誠實話.
但這種學習.
不是頭腦上的.
我們在忠臣學習很多東西.
有很多的credit.
我發現是用了七年.
我進入目的.
在這七年裡面.
不斷經驗.
不斷去撞板.
在生活裡面去被塑造.
去被改變.
今天我覺得.
我很感謝Vance老師.
和一班忠臣弟兄姊妹.
一班同工.
我們這個群體.
就是很站身去經歷這種.
原來要共融這麼辛苦.
要聯絡這麼多人.
安排手語傳譯.
或者地方又要換.
有很多的張力和掙扎.
但其實也許.
在這些挑戰和壓力裡面.
我們問自己的內心.
為什麼我們要做這件事.
我們心底的渴求是什麼.
我們渴求和上帝相遇.
在什麼情況下和上帝相遇呢.
剛才兩位都提到.
我們和殘疾弟兄姊妹的分別.
其實不是那麼多.
我們不是相殘和健全.
是相殘和未相殘.
我們遲早都會硬盲.
而龍會幾力衰退.
擁抱相殘其實就是擁抱自己.
不只是未來的自己.
而是今天我們都有不同.

$^{1801}$隱藏著我們殘缺破損的生命.
我所經歷的醫治.
是有很深的友誼.
無論是和我們譯者.
一班懂手語的見聽朋友.
和龍人.
有一段的.
引文我想和大家分享.
說到有關在相見之間的友誼.
Eugene Peterson 在.
《Live Over a Wall》這樣寫.
每個人一生都會接觸數百個人.
在和他們開始有眼神接觸的時候.
他們就會衡量我.
是否有利用價值.
就會將我分門別類.
好像處理一些事務一樣.
換算.
我都是這樣看待其他人.
但有些人進入我們生命並不是這樣.
他們只是盼望更了解我們.
他們不會暴露我們的弱點.
不會妒忌我們的優點.
他們明白我們的內心.
體恤我們的難處.
尊重我們的感受.
他們願意成為我們的朋友.
就是在這種同行裡.
我們慢慢就是.
你幫我,我幫你.
原來你是這樣的.
原來我有這樣的需要.
彼此去發現彼此認識.
知道雙殘弟兄姊妹.
成長於很多困難的家庭.
他們經歷很多幫助來自家人.
家人同行都很不容易.
但有時那份被幫助都很辛苦.
因為家人無法選擇.
好像都是要幫你.

$^{1841}$友誼不同.
做朋友我可以選擇.
我喜歡和你做朋友.
我喜歡和他做朋友.
我喜歡主動學手語.
在這種出於自由選擇的愛和同行裡.
我看到醫治在我們的群體裡發生.
對我來說,如果沒有龍人弟兄姊妹.
如果沒有傳業的弟兄姊妹.
這些友誼.
我想我已經很快放棄神對我的呼召.
或者我都會變成一個很腐毒的人.
對於拒絕龍人的人,我會產生敵意.
我都會變得只懂得一路批評.
很多憤怒.
或者很沮喪地埋怨.
或者很無力地退縮.
回望其實.
無論成長的十多年.
在這個教會.
我有十年是逃走過的.
覺得太辛苦了.
或者剛巧又結婚.
好像有第二個呼召.
我做學生服務工作.
但再回來默會.
艱難是很多.
我覺得經常面對自己內心的不安.
甚至是憂暗.
但神是用主類的友誼保護了我.
讓我可以對人繼續努力.
保持善意.
讓我有力量走下去.
所以最後一點.
想和大家分享的是.
受苦是真實的.
恩典也是.
今天看到在座這麼多人.
你們對這個題目有關注.
有渴求.

$^{1881}$我已經覺得.
上帝在做新事.
我深信聖靈也在你們所屬群體裡.
已經塑造生命.
預備一些東西.
預備我們進入.
相見共融的場景裡.
我們會和殘疾人士.
面對很多.
剛才老師也提到.
被歧視.
有很多困苦.
被誤解.
聾人也是.
讀書的時候.
有些在主流學校.
被禁止用手語.
或者是在求職.
或者持續教育裡.
你想去讀Ivy嗎?.
你想去讀大學嗎?.
你要自酬.
一些費用.
去請手語傳譯.
經歷很多被歧視.
被誤解.
最孤單.
甚至最孤單.
有時是來自自己最親密的人.
家人的不明白.
或者是當你像貓貓的小狗一樣.
養大.
其實是.
很多聾人弟兄姊妹告訴我.
也有些暴力的衝突.
其實是很傷.
我們.
預備自己.
和殘疾弟兄姊妹同行的時候.
我們的心靈.

$^{1921}$其實也很容易被蠶食.
很多無奈.
很多失望.
很多憤怒.
也會發現自己.
其實沒有什麼可以做.
我改善不了教育制度.
我改善不了翻譯服務.
我也改變不了.
家裡的慘況.
是很有限.
也很容易變得越來越討厭這個世界.
這麼冷漠無情的.
甚至.
很多逃群體.
這個說得很散.
雖然有錄影.
我幫某一個群體.
每一年的培靈會.
也有翻譯.
十多歲到現在.
其實有很大改善.
但原來.
跟他們說.
其實手語框這麼小.
等於很小聲.
聽不到.
要大一點.
才有正常的音量.
原來要讓對方了解.
也要很多時間.
所以.
在裡面發現自己那種無奈性.
自己那種.
很容易批判別人.
也是很貧窮,很缺乏.
我們跟傷殘人士一樣.
也是破損.
也是很需要倚靠神.
今天.

$^{1961}$最後用這段引文.
作為一點結束.
我發現上帝藉著.
龍人給我的禮物.
根本地去.
顛覆我的信仰.
這不是一個.
很厲害.
什麼都做得到.
很神職醫治的耶穌.
在裡面.
這位是十價受苦的耶穌.
記得以前在讀書.
占卜宗老師跟我們看過一本書.
作者引述潘復華的說話.
這樣說.
我們的上主是為人受苦的上主.
就像保羅見證神所說.
我的能力.
在人的軟弱上顯得完全.
當一個人在軟弱當中.
察覺自己有神的形象.
就在那裡.
他分享著神的形象.
就在那裡.
他感受到神的同在.
就在這種軟弱破碎那裡.
他對神的能力開放.
這是恩典.
是神的愛.
是安慰.
也許如果我沒有你們.
我不明白神的愛.
我繼續都是靠自己的Carol.
時間有限.
我們稍後可以在Q and A裡面.
分享多一點.
今天我的分享到此.
多謝三位講員的分享.
我們現在有一個答問的環節.

$^{2001}$如果有想要提問三位講員的話.
你可以到中間的咪高峰.
去問問題.
或者你可以舉手.
我們有工作人員會將一些紙和筆.
交給你.
你可以寫下來.
再交給我們台上的講員.
我們就等三位講員可以上台.
看看有沒有頂層的問題.
你可以到咪高峰那裡去講.
舉手問我們工作人員拿一些紙筆.
或者你不方便走出來.
你也可以舉手.
我們可以將咪高峰拿到你那裡.
或者如果是有手語的發問.
我們台上都會有翻譯員幫我們翻譯.
時間就交給大家了.
我們現在就開始答問的環節.
有兩個問題.
剛才Rans老師提到.
有幾個色經上的問題.
我想請問面對這些觀念.
剛才袁老師提到的觀念問題.
是怎樣去破除這些迷思呢?.
第二個問題.
當中有幾位講者所講的.
都是將殘疾人士標籤.
或者是呈現一個負面的形象.
我不在這裡講太多自己的事了.
但是一個很真實的例子.
我之前在某些公開試之後.
是上過報紙的.
之前是沒有什麼人理我的.
尤其是在堂會裡面.
但是當我上了報紙之後.
上了一些中派的見證集的時候.
那個牧師突然間出聲.
就說這些就是成功的例子.
這些就是融合了的例子.

$^{2041}$接著就不斷地去消費成功的例子的時候.
我想請問.
今天面對這些模樣的問題的時候.
你作為紀委會不會有些建議.
或者是怎樣去到讓我們這些.
可以真是去到融合到.
而不是被消費的呢?.
謝謝.
多謝這位朋友的提問.
剛才提到一個就是.
怎樣去破除這些扭曲呢?.
我想其實好像我剛才一直這樣說.
我們自己好像在那些泥淡的猴子那樣.
你要自己救自己脫離那個泥.
其實是沒有可能的.
因為其實這件事不是我們.
好像來到這個世界.
被這個錯誤的觀念污染.
而是我們根本就生在一個扭曲了的世界裡面.
其實那些概念是怎樣來的呢?.
都是我們整個文化.
先說一下.
就是吹捧一些成就,成功例子.
我們自然就會去到那一邊.
就算你說.
剛才你提到的那個例子就是.
一個傷情人士.
如果他有一個正面的,正向的例子.
就突然間好像那個光線就去了他那裡.
所以,我剛才一直在想.
其實我們都是一起.
在看見自己的扭曲.
下一步就不是怎樣去撕走那些扭曲.
而是真的在神面前.
我們需要一起去悔改.
我想這裡就算我們是什麼狀態.
其實都會被這些捆綁.
影響著我們.
所以我想擁抱殘疾.
就是擁抱我們其實都是殘疾.

$^{2081}$在思維上,在觀念上都是殘疾.
但是,剛才一直提.
你提問的時候.
我相信可能背後你沒有很詳細地表達.
但可能中間都會受到一些傷害.
或者覺得自己很不甲.
很大疑惑.
這個群體為什麼會這樣呢?.
我相信就是為什麼我說.
為什麼這麼難連結呢?.
因為事實上,真的慘手慘腳.
他不來搞你.
他一買到你的時候.
就算好像很容易.
結果出來的效果都會令你.
好像你剛才說的被消費.
我想,我一直聽的時候.
更加是我們都一起需要悔改.
一起去需要神在我們當中.
但我很欣賞你可以在這裡.
將你的經歷和我們分享.
因為唯有我們將我們.
群體的現況呈現出來的時候.
我們自己的聲音被人聽到的時候.
我們才可以有機會一起去相遇.
所以,多謝你.
我都是比較悲觀.
我覺得我在這方面.
我都是很認同素斯說.
其實無論什麼人.
包括教會都是.
包括牧者都是.
包括神學院都是.
我們都是要悔改的.
我們可能叫做開始留意到.
例如我們過往可能識經.
或者講神學.
有一些東西原來其實是很歧視.
帶有很多偏見.
現在開始成覺.

$^{2121}$但我相信其實還有很多位置.
我們還是要悔改.
也很需要其他人去提醒我們.
所以,都很多謝你的聲音.
我關心的是.
你有沒有機會和你教會.
一些弟兄姊妹講你的感受.
甚至如果你覺得都準備好了.
都讓他們有機會知道.
他們需要悔改的地方.
或者不是他們,是我們.
我們這些以為很懂得的教牧.
這種傷害,上帝知道.
多謝大家的回應.
不知道還有沒有弟兄姊妹的提問.
三位講完,你好.
這個題目給楊老師.
剛才你分享雙向共融的教會觀.
第四點你講.
不將洗禮與會籍掛勾.
其實我想知道這個要點.
如何促進雙向共融呢?.
因為其實自己都是牧會的.
我想其實那個會籍.
對教會的生態環境有些意義.
其中一個可能是基督教墳場的門檻.
所以,這個都很失望.
所以其實我想.
雙向共融其實是.
從前在醫院,比如羽木實習的時候.
其實羽木其中一個呼喊聲.
就是很多教會都未必會幫到一些臨終者施洗.
所以其實比如當我們說沒有會籍的時候.
其實如果你的眼界放在天國.
其實真的沒有所謂.
但我就不明白為什麼.
寫這個要點可以促進雙向共融.
是,多謝你.
可能我都需要澄清一下.
其實我的意思是.

$^{2161}$在一個平時的狀態下.
如果我們將洗禮和會籍掛勾.
其實是加了一些要求給.
希望純粹表達他信耶穌.
和加入大公教會的決定.
例如我教會之前.
其實後來都是.
堅持要到某一個年紀.
因為其實那個就是牽涉到.
當有會籍的時候就有投票權.
很多行政,教會運作的參與.
那個期望.
也是加給想洗禮的人.
我都明白.
剛才梓媚提到的關注.
其實我剛才的分享.
我的意思是.
其實洗禮.
我的質疑是.
洗禮是不是應該有這麼多的門檻呢?.
又要上諸學,又要考文.
要寫見證.
要寫一篇文出來.
其實是加了很多的負擔.
是一些.
例如臨終.
希望相信耶穌的人.
或者一些有特殊需要.
而教會是不是應該可以.
其實平時不是說這麼多.
這些要求的時候.
其實就會更加降低門檻.
而去為那個人去施洗呢?.
其實我剛才的意思.
不是說洗禮和會籍是完全分開處理.
而是至少容許是分開處理.
如果那位朋友.
他是想一起加入那個會籍.
當然是可以的.
只不過是不想那個會籍的要求.

$^{2201}$加在一起.
要那個人要符合.
他要投票.
認識教會.
例如某一些公司登記.
就要參與運作的那個要求給他.
所以剛才我說的.
其實不是否定一個有需要洗禮的人.
他同時想加入那個會籍.
多謝思妍老師的回應.
還有沒有其他的提問?.
(回應觀眾).
我剛剛收到一個.
他說不算是提問.
是一些回應.
他這樣說的.
我不知道自己是否一般.
或者值得被關愛或者被接納的殘障.
他說有情緒病.
有精神患病.
當年不只教會.
連家人都認為這是罪.
是錯的.
明白你們所說的.
要體諒對方的不足.
但是我都沒有辦法原諒.
我們都做不到.
或者Vance老師你看完之後.
有沒有什麼回應?.
情緒病,精神病事實上是很大的挑戰.
但是這件事有很多標籤.
很多判斷.
還有很多誤解.
所以剛才我提到.
其實有很多東西我們要知道.
其實我們基本常識可能都不夠.
曾經有一些外國的研究.
做過一些關於教務.
究竟有多了解精神病患的一些.
很基本的醫學知識.

$^{2241}$其實很多都是肥佬.
不只是教務.
其實是一般人.
所以很多時候更加容易被判斷.
尤其是你為什麼這麼憂鬱.
不是應該上堂喜樂嗎?.
因為那個形態和我們心目中覺得.
一個基督徒怎樣展現.
我經常想我們有時候都被社會覺得.
你是一個好的基督徒.
你就會怎樣?.
你就會常常喜樂.
就算打不死.
其實我們有時候在教會聽見證都是這樣.
聽見見證很慘.
不過靠主我又站起來.
好像你要去到那種狀態.
你才有資格見證到神和榮耀到神.
其實最終我經常給他一個詞.
我們形造了一種信徒.
就好像一朵塑膠花.
塑膠花是怎樣的呢?.
你什麼時候看它都是這麼漂亮.
雖然它是假的.
真花會怎樣?.
會很醜的.
有時候會很殘的.
但它真的有生命才會這樣.
我覺得其實剛才這位朋友.
看到的,多謝你分享.
你分享的一個其實很多人都在經驗的現實.
就是身邊的人沒有辦法去看到你真實的你.
在你自己本身的困難裡面.
還加上一些傷害.
我想裡面,我剛才說要體諒.
其實不是說不要覺得你要馬上放下那些東西.
其實我們是一個很長的過程.
很長的過程.
就算在這一刻裡面.
在你的狀態裡面去連結.

$^{2281}$我絕對不會覺得你放不下就是你的錯.
如果剛才聽到這些,對不起.
那是一個神的時間.
因為原諒不是我們自己能力做得到的.
真的是一個恩賜.
是一個很大的.
不是我們自己很強勁,很包容.
而是一種神在那個時候給你放鬆.
開了你的鎖.
所以你說放不下.
我自己去看可能我們祈求的就是.
祈求神,我現在不行.
不過如果你給我這個恩典.
我願意接受這個可以原諒的恩典.
我想最重要是很真地來到神面前.
不要做一朵索加回到神面前.
其實神知道你是怎樣.
祂亦都不介意你是怎樣.
其實情緒,精神病患.
其實我一些很親密的家人都經歷.
其實我有時面對這個情況.
我有時都會想神為何不讓他有一刻可以.
他都是一個很好的徒.
亦都幫人,對人很好.
我好像覺得為何不讓他經驗一下.
可以很舒適地去了解身邊的世界呢.
因為裡面是很大的困苦.
裡面經常有很多的憂鬱.
不過在過程裡面.
我又經驗到另一個向度.
就是看到家人之間.
就算在這樣的情況下.
那種不離不棄的愛.
神好像讓我看到.
他有他自己的心意.
但如果好像這位朋友裡面經驗.
我自己都會覺得.
我覺得很需要神特別的因典.
在你的生命裡面怎樣去繼續走下去呢.
我都鼓勵你.

$^{2321}$其實有很多希望可以找到同行者.
香港都有一些機構.
譬如愛協的協會.
當中有很多願意去和大家同行.
提起愛協.
其實他們有一個很欣賞的概念.
大家有沒有聽過愛協團體.
即是專門服務精神病或情緒病的團體.
他們的董事裡面.
會有一些一定的數目.
好像三分之一.
曾經在這方面有經驗的復原人士.
他們很想他們做出來的計劃.
是真的落地的.
不是自己另一班沒有這個經驗的人想出來的.
他們很推崇一件事.
叫做殘疾之力.
什麼是殘疾之力呢.
有些事是因為你有這些經驗.
你才有資歷.
如果你沒有這些經驗.
其實用不著你.
所以在裡面.
我覺得這些概念,這些群體.
都可以成為我們在經歷裡面的鼓勵.
我都想有少少回應和分享.
家裡和教會裡面的傷害其實是很深的.
原諒不了就原諒不了.
也許有一種的憤怒.
都是神告訴我們.
我們內裡的需要和渴求.
我們需要被尊重.
當尊嚴受損.
感受到別人不尊重的時候.
我們的憤怒是在保護我們的人性.
傷心,我們的悲傷.
其實是在表達我們需要愛.
需要被肯定,需要被珍惜.
這些都是神給你的禮物.
我都紀念你,為你祈禱.

$^{2361}$其實最近我教會都有一位農人.
是精神情緒有困擾.
來了我教會幾個月.
我見到他不斷掙扎.
他最初是報讀一個課堂.
他盡量來到就來.
有時他崇拜都很不安.
要周圍走.
亦有時每天都要發訊息.
不斷和人連結.
不能外出,買不到食物.
很多驚恐.
但我見到他的掙扎裡面.
他很努力.
都讓我明白到.
其實我剛才沒有說的.
這些弟兄姊妹.
就是讓我明白榮耀和苦難是同一件事.
他的掙扎都彰顯著神的榮耀.
都要賣廣告.
這本書真的很好看.
《擁抱殘疾》裡面真的不嘗氣.
和我不同.
很精簡,清晰.
重點講了最後幾個章節.
我自己是很喜歡十字架神學的部分.
我都在我們的群體裡面一起掙扎.
我賠償了多少.
我弟兄姊妹賠償了多少.
我都有一種嘆息.
很想大家一起去試夢.
然後就將重擔放在別人身上.
但不是的.
其實我們教的掙扎都是一起去學.
如何和他同行.
如何和他的家人聯絡.
如何尊重他.
想看醫生還是不看醫生.
是否需要服藥還是見輔導.
我們整個群體一起去體會.

$^{2401}$我們這個十字架的信仰.
好,謝謝.
我也想分享一件事.
我都是因為教神學.
我也覺得有一個概念.
我和別人分享.
或者別人和我分享.
我也覺得是很幫助我們去面對現實.
而現實是包括教會的現實.
我們現在處於一個已然未然的狀態.
我們很多時候想起.
例如聖經說我們要起狼.
或者我們已經是無請救.
為什麼會這樣呢?.
或者為什麼教會的群體.
還帶來這麼多的傷害.
或者為什麼我的身體.
或者我的精神.
還是有這麼多的問題呢?.
但其實的確是因為我們還沒有去到終末.
我喜歡有一個哲學家說.
其實聖經很多的教導.
他沒有說你五分鐘就要做到.
是我們迫自己覺得我們已經去到終末了.
已經是去到耶穌再回來的聖潔的那個狀態.
其實我們全部人包括教會,包括牧者.
其實我們都是在一個旅程上.
我們都是在尋求,在等待將來的那個狀態.
但的確其實我們那個真實的我們.
我還是喜歡剛才Rans說的.
那個真實的人性.
其實我們還是在很多軟弱裡面.
的確聖經有教導我們饒恕.
但是真的沒有說你五分鐘就要去饒恕.
很多的傷口.
只是傷口埋口可能都要很多很多年.
這個神是知道的.
多謝三位的回應.
時間都差不多了.
不如我們三位先入座.

$^{2441}$我們有些報告.
謝謝.
(音樂).
(字幕由 Amara.org 社群提供).
\newpage



\section{}
\label{sec:kR2ujHQel1E}
\textbf{【工作坊|擁抱殘疾的教會 — 群體中經歷醫治和牧養】}
\newline
\newline
連結: \href{https://youtube.com/watch?v=kR2ujHQel1E}{\texttt{ https://youtube.com/watch?v=kR2ujHQel1E}} ~~~~ 語音日期: 2023-04-14 
\newline
\newline
\hyperref[sec:O8VAiCx1rx4]{\small{< < < PREV SERMON < < <}}
~
\hyperref[sec:index]{\small{[返主目錄]}}
~
\hyperref[sec:_p1k_J5ZNow]{\small{> > > NEXT SERMON > > >}}
\newline
\newline
$^{1}$各位弟兄姊妹早安.
再一次代表中國神學研究院和方正柱協會.
歡迎各位弟兄姊妹再一次參與今天的聚會.
想問問有沒有昨天參加了公開講座.
歡迎你們,好耶好耶.
今次有很多弟兄姊妹都是第一次參加我們這個系列的講座.
也歡迎大家.
我們也很想除了透過說話去表達對大家的問候之外.
我今天也學了一個新的手語.
跟大家說早晨應該是這樣的和這樣的.
多謝多謝多謝,我們再做一次好嗎.
早晨,對,早晨.
跟你們身邊的弟兄姊妹說一下.
早晨,早晨,早晨.
你好就是.
你好是伸出來的.
不是拍胸口的.
多謝你們的指導,很開心.
我們又試一下,你好.
你好,你好,真的很開心.
透過不同的方法都可以去表達對我們身邊的弟兄姊妹的關愛.
也想跟大家介紹一下我們今天有很多不同的工作坊的嘉賓.
第一個工作坊是叫做難不住的路.
有吳祝寧傳道,樂思惠女士和石健華女士負責.
對不起,多謝多謝.
這些經常都是.
所以我有殘障的了.
這麼近也看不到,看錯字.
還有語言也有些殘障.
吳傳道是回聲谷雙健康協會和教會的傳道人.
樂思惠女士和她已經回天家的丈夫.
一同在回聲谷雙健康協會侍奉多年.
同心推動雙健共融,擁抱殘疾的使命.
余健華女士是西區福音堂同工會副主席.
也在教會推展無障礙敬拜.
先給她掌聲,多謝她們的分享.
第二個工作坊是漢不見的美.
有關志偉傳道,黃承軒先生和林君輝先生負責.
(掌聲).
關傳道是視像者團契中心侍奉多年.

$^{41}$他喜歡閱讀,音樂,電影,交友和旅行.
黃承軒先生,Steven,自幼讀心光學校.
喜歡彈奏夏威夷小結他和大提琴.
也經常游泳,行山,跑步,也會製作甜品.
不知道有沒有機會可以試試他的手藝.
林君輝先生是一位音樂創作人和老師.
在2014年開始,指導一隊以心光學校視像畢業生為骨幹的敬拜隊.
開心唔敬拜隊,掌聲歡迎.
(掌聲).
第三個工作坊是跨文化友誼.
分別由黃海恩傳道,余偉晶女士和吳樂熙女士負責.
黃海恩傳道Carol,7年前進入木職,牧養農人,學習同行.
余偉晶女士Debrie,農人譯者,近年受教,投身香港手語聖經工作.
吳樂熙女士Phoebe,希望用紀錄片,共融舞蹈的方式.
探索各種與不同能力,殘疾,社群同行的可能性.
先給他們掌聲.
(掌聲).
我忘記說,QR Code有詳細介紹.
如果你想了解更多工作坊嘉賓的資料,可以掃描QR Code.
第四個工作坊嘉賓是軟弱的力量.
分別由羅啟康先生,李家緹傳道,范瑞芬女士和陳錦梅女士負責.
羅啟康先生是寧實恩光成長中心總監,負責嚴重智障學生40年.
李家緹傳道Gloria,目前已有15年,負責無障礙敬拜,病患者關顧.
范瑞芬女士和陳錦梅女士是智障人士的家長,是生命勇士的小天使.
先給他們掌聲.
(掌聲).
最後一個工作坊是無阻隔的愛.
分別由陳麗嫻牧師,邱秉華醫生和曾偉賢先生負責.
陳麗嫻牧師是基督中心堂荃灣主任牧師,牧養特殊學習需要的兒童十多年.
邱秉華醫生和曾偉賢先生Jacob是特殊學習需要兒童的家長.
我相信他們更能體驗家長的需要.
顧此Jacob在2021年創立了社企OneCent為特殊學習需要的家庭.
在身心,社齡上提供不同的支援.
先給他們掌聲.
(掌聲).
所以今天的工作坊,陣容鼎盛.
亦可以讓不同的弟兄姊妹和同工有好的交流.
在開始我們今天聚會之前,都有幾個報告.
如果有肢體想看手語,亦可以坐前面的座位,可以看得清楚一點.
如果有弟兄姊妹需要洗手間的話.

$^{81}$我們在多瑟街地下和一樓都有洗手間.
而一樓是一個無障礙的洗手間,你可以在外面乘升降機就能到達.
有些工作坊會在對面的德雲道校舍進行.
那裡也有無障礙的洗手間,到時也可以乘升降機到達.
工作坊完成後,希望各位弟兄姊妹可以填寫回應表.
回應表有QR Code,可以張貼在不同的工作坊課室.
希望弟兄姊妹可以給予我們更多意見.
今天繼續有一個書攤,可以發售不同的書籍.
特別推薦由關樂兆老師的著作.
擁抱沉寂的教會,群體中經歷的意志和模樣.
今天購買有特別的折扣.
我的報告就這樣,將時間交給Vance老師.
希望稍後可以繼續和大家分享很多學習.
謝謝.
各位早晨,歡迎大家.
今天的雙向共融真人圖書館.
是這次擁抱沉寂的教會講座工作坊的下半場.
接下來的20分鐘,我希望藉著一些分享.
和大家做一下這個工作坊的導引.
如何進入這個工作坊.
今次我分享的內容是有一首歌.
所以我邀請了一隊我很欣賞的勁拜隊.
和大家一起分享這首歌.
我先介紹一下他們.
台前的就是海參勁拜隊.
隊員有心光學校的畢業生.
心光學校是香港唯一的市長人士學校.
這三位都是畢業生.
Steven,請你站起來.
Steven今天是彈Ukulele.
還有Jeremy,是打鼓的很帥的.
前面是打卡康的卓龍.
你可以坐下.
卓龍帶了他的悟道麻犬Dong Dong在前面.
很乖的等待他的主人.
也有兩位音樂導師,輝Sir和G Sir.
也有老師Miss Chan帶他們來.
昨晚我分享過.
在我的訊息背後有很多相見群體的生命經歷.
和他們的連結啟發了我的反思和整合.

$^{121}$所以對我來說.
相見共融不是從腦裡入心.
而是從生活中體驗到和相見連結所帶來的祝福和學習.
所以今天這個工作坊的設計.
我很想讓大家親身體驗一下.
一起去聆聽這些相見群體.
互相連結的時候發出的力量.
如果大家記得講座的內容.
就會留意到裡面也為今天工作坊的主題做了一些鋪排.
主題就是相向共融.
昨晚我們問了一個問題.
就是何謂擁抱殘疾呢?.
整個兩天的主題是擁抱殘疾的教會.
何謂擁抱殘疾呢?.
我們昨晚說過.
擁抱殘疾不是單方面憐憫和幫助他們.
而是一種相向共融的群體生活.
而相向共融的四個層次裡面.
最高層次就是相見共融.
最高層次就是相向共融.
不單止是可以出席,參與,交流.
而是彼此互相依賴.
沒有了大家是不行的.
而我們有信仰基礎的相向共融.
不單止是高舉人人平等.
不單止是做平權運動.
而是關乎基督教的人,官和教會官.
所以我們不論相或見.
每個人都好像手足一樣.
大家都需要對方.
沒有對方是不行的.
所以擁抱殘疾的教會願景.
亦都是相向共融的願景.
多謝你的回應.
什麼叫做相向呢?.
相向即是有來有往.
所以教會不單止服侍他們.
亦都接受他們的服侍.
或者你說我們教會都有給機會.
殘疾人士服侍的.

$^{161}$他們有什麼能力做到的.
我們就讓他們做.
例如如果他們唱歌可以.
那就唱詩班.
或者你還記得有一位一出世就沒有手沒有腳.
叫什麼名字?.
叫做Nick.
他真的很厲害.
他祝福了很多人.
還到處去講見證.
但如果我們單是想他們還做到什麼.
這個例子.
我們始終都有個問題.
因為我們正在用一般人覺得.
正正常常的侍奉模式.
什麼叫做到東西.
要做到東西才叫做侍奉.
所以就算某方面有殘疾,沒有能力.
你總有些事情做到.
始終都要做到東西才可以侍奉.
所以一去到一些例子.
例如嚴重的精神病患.
或者嚴重的智力障礙.
就很棘手.
我們就站在這裡.
究竟他們有什麼侍奉崗位呢?.
他們什麼都做不到.
可以提供什麼服侍給其他人呢?.
今天的工作坊就很希望大家.
可以跳出這個要做到東西才是侍奉的框框.
在不同的殘疾人士群體當中.
其實他帶給我們的祝福.
是他們是怎樣用非一般的模式去服侍我們.
我都希望大家進入這種的經歷.
進入經歷之前.
我們首先要放下我們一些單靠理性的思維模式.
所以以下就想和大家分享一首歌.
這首歌叫做《溫柔的生命》.
是我為相片裡的這位小朋友.
希希和他外婆梅姐所寫的.

$^{201}$梅姐今天也在我們當中.
先給梅姐掌聲.
我和梅姐說我要分享這首歌.
然後她說你又弄哭我了.
我說其實你也弄哭很多人了.
希希是一個嚴重智力障礙的男生.
他十一年前已經安息住位.
我認識他的時候.
他十幾歲.
是寧石恩光學校讀書的.
希希有很多重的障礙.
他不懂得走路,不懂得說話.
也完全沒有自我照顧的能力.
吃東西,大小便都要別人照顧.
外婆很疼愛他.
雖然梅姐當時未信主.
但她看到希希很喜歡.
教會和學校合辦的無障礙敬拜.
所以她一直都有帶希希參加.
一直參加了幾年.
後來希希離世之後.
外婆問我們.
教會可不可以為希希辦安息禮呢?.
如果你是教會傳導人或是教會長執領導.
你會怎樣回應呢?.
通常我們這樣想很簡單.
有多難呢?.
但如果你做到那些位置.
你就知道很多事情要想.
因為在這個情況教會要想些什麼呢?.
就是要知道.
代表離世的人是信徒.
你才可以為這位弟兄辦安息禮.
因為牧師會說我們和弟兄天家相聚.
但就因為這個原因.
就算你信了主和接受了水禮.
但如果你自殺的話.
有些教會都不肯幫你辦安息禮.
因為他們覺得都不知道你能否上天堂.
希希是沒有表達能力的.

$^{241}$她亦沒有足夠的認知能力.
去了解我們一般傳福音的內容.
我們怎知道她是否信主呢?.
其實當你清醒一點去想.
希希的情況.
明明反映了我們制度的盲點和缺陷.
教會結果都是有為希希辦安息禮的.
過程怎樣處理我一會兒再說.
現在不如先聽聽歌.
《溫柔的生命》這首歌.
我在籌備希希的安息禮的時候.
外面下傾盆大雨.
我坐在車內.
這首歌的歌詞和音樂就像雨水一樣.
一次過在我心裡湧了出來.
首先請開心記拜隊的隊員.
為大家慢慢讀一次歌詞.
謝謝.
啟發我的心.
是你的生命啟發我的心.
原來愛無障礙伸出堅韌.
是你的生命開我眼看得清.
原來信抗超越眼所見耳所聽.
溫柔的生命發出無窮強勁.
越過傷心越過困惑靠主心堅定.
短暫的生命成就不死使命.
讓我高呼讓我唱和.
我願來做證.
請大家聽一次.
是你的生命啟發我的心.
原來愛無障礙伸出堅韌.
是你的生命開我眼看得清.
原來信抗超越眼所見耳所聽.
溫柔的生命發出無窮強勁.
越過傷心越過困惑靠主心堅定.
短暫的生命成就不死使命.
讓我高呼讓我唱和.
我願來做證.
是你的生命啟發我的心.
原來愛無障礙伸出堅韌.

$^{281}$是你的生命開我眼看得清.
原來信抗超越眼所見耳所聽.
溫柔的生命發出無窮強勁.
越過傷心越過困惑靠主心堅定.
短暫的生命成就不死使命.
讓我高呼讓我唱和.
我願來做證.
溫柔的生命發出無窮強勁.
越過傷心越過困惑靠主心堅定.
短暫的生命成就不死使命.
讓我高呼讓我唱和.
我願來做證.
繼續說剛才的情況.
婆婆戴希希參加的無障礙敬拜.
是由西區科音堂和寧實恩光學校合辦的.
從2006年開始到現在.
當時我也有份參與開展這個事工.
其實我們想和嚴重智障的朋友一起敬拜.
當時我們是不懂得怎樣和他們敬拜的.
也不知道他們可以怎樣敬拜.
不過我們相信如果神真是神的話.
祂應該可以向這些小朋友啟示自己.
沒有什麼會難阻祂.
我們會有障礙我們不明白.
但我們相信神的愛沒有障礙.
反而我們心裡有一個渴望.
很想多了解更加認識這位無障礙的神.
在嚴重智障的小朋友和家庭裡所做的事.
我們也是福音教會很著重傳福音.
弟兄姊妹來參與無障礙敬拜.
很多時候都會問.
怎樣向他們傳福音呢?.
怎樣知道他們是否明白和相信呢?.
怎樣知道希希是否明白福音.
是否願意接受救恩呢?.
其實如果我告訴你他明白.
你也不要相信.
現在說明是嚴重智力障礙.
你還要問他明不明白.
究竟現在是誰智力障礙呢?.

$^{321}$誰應該明白但不明白呢?.
我們總覺得自己需要判斷得到.
我記得當時我問希希婆婆.
為什麼她會想為孫子辦安息禮呢?.
婆婆的回應很直接.
我覺得希希很喜歡無障礙敬拜群體.
如果真的有個天堂的話.
我想希希會喜歡去你們的天堂.
很直接吧?.
希希選了我們的群體.
但現在問題是這個群體.
是否當希希是他們的一份子呢?.
我們教會的章程規則裡.
如果申請成為會友.
首先要寫得救見證.
就像昨天老師說的.
要寫文章.
就算你不懂寫文章.
你口說也要說得到重點.
才算合格.
很明顯這個規定對於希希這類人來說.
是一個歧視性的規定.
因為她一生都沒有辦法做到這件事.
於是教會群體就想.
希希沒有辦法為自己做見證.
我們當中有沒有人願意為她做見證呢?.
不是證明她是否信耶穌.
我也說了沒有人可以做到這個判斷.
而是見證她和希希一起敬拜的時候.
是怎樣經歷到神的同在.
很感恩.
我們當中的第一節目很踴躍.
有長短的見證.
在希希的安息禮小冊子裡.
不少留言都說.
希希是我們的弟兄.
事就這樣成了.
我記得希希的安息禮裡.
有不少同學的家長都來參加.
當世界上大部分的人都說.

$^{361}$這些小孩不值得我們投放這麼多資源.
我為願教會有點不同.
我為願教會的行動.
可以大聲地跟大家說.
耶穌歡迎這些小朋友.
耶穌也聽到他們父母和照顧者的呼求.
溫柔的生命發出無窮強勁.
為何說希希的生命很強勁.
因為她一句話都不會說.
但她成為了我們的神學老師.
挑戰我們.
要我們重新反省我們自己定出來的規矩.
好像一塊鏡子照到我們自己的無知和無稽.
希希呼喚我們.
從過份崇拜理性這種迷思當中悔改過來.
她和我們的連結.
帶動了教會重新以愛去連結.
回到基督群體的核心.
就是以基督的愛彼此相愛.
在當中見證神的臨在.
我想問問大家.
希希有什麼侍奉崗位?.
一個這麼弱的小朋友.
她在群體裡的角色是什麼呢?.
是否只可以接受服侍呢?.
希希是人體為小子中的小子.
但我覺得她是我們的屬靈導師.
這就是她的侍奉崗位.
而希希的侍奉不是靠做的.
她沒有什麼學位.
她的侍奉是單單靠存在就行了.
溫柔的生命.
挑戰大家再唱一次.
今次我請大家重新讓這些生命去挑戰你.
去在神面前反思.
首先聽紀美隊唱第一部分.
然後我們才入.
是你的生命啟發了的心.
讓愛和寂寞伸出言論.
是你的生命啟發了的心.

$^{401}$溫柔的生命.
化做我共戀的.
別放手心的復活.
靠著心堅定.
決戰場的生命.
拯救不死是命.
讓我高飛揚昌.
我願來造證.
溫柔的生命.
化出無窮風景.
別放手心的復活.
靠著心堅定.
決戰場的生命.
拯救不死是命.
讓我高飛揚昌.
我願來造證.
今天有很多溫柔而強勁的生命在我們當中.
雙向共融真人圖書館.
將會有五個單位.
邀請了15位不同身份角色的嘉賓.
他們當中有本身是殘疾人士.
有殘疾人士的配偶.
家人.
也有當中有牧者.
有不同機構服務的弟兄姊妹.
在裡面我們看到.
這些不同的角度.
如何見證雙向共融是如何發生的.
當中有很多位都是我多年的好友.
我見證著大家的生命裡.
都是有淚水,有汗水.
但實在都很有恩典和祝福.
真人圖書館是三位三位一起分享的.
這個設計亦都很想讓大家看到.
我們的故事是在連結裡發生的.
其實每一個分享嘉賓都很不同.
但他們有一個共通點.
就是他們堅持在破損當中去連結.
當中會有被傷害,失望,憤怒的時間.
但堅持不要離群.

$^{441}$與群體連結,與神連結,與自己連結.
今天我們的課題是「雙向共融」.
所以在這個工作坊裡.
我邀請大家即時實踐「雙向共融」.
如何實踐呢?.
我知道當中有很多有心人.
很想去服務殘疾群體.
我希望你們首先學習被服務.
邀請大家在聆聽他們的故事的時候.
不只是想怎樣去解決他們的問題.
而是首先開放你的心靈.
去讓他們的故事服侍你.
讓他們的生命故事成為一面鏡.
一道橋或一扇門.
去幫助你更加認識神.
更加帶動你與上帝有對話.
所以今天的真人圖書館.
與一般真人圖書館有點不同.
是不會有你與分享者的答問.
不過仍然有Q and A.
就是你與上帝的Q and A.
在每一個工作坊的最後幾分鐘.
你會有些安靜時間與上帝聊天.
這一頁PowerPoint.
就是今天工作坊的工作紙.
你有需要的話可以拍下它.
提醒自己.
一邊聆聽的時候.
問自己這兩個問題.
真人圖書館的分享.
有沒有啟發你更加認識神呢?.
神在跟你說什麼呢?.
就著今天的主題.
這一刻.
你有沒有問題想問神呢?.
有沒有什麼想跟神說的呢?.
在這個環節結束的時候.
很想邀請工作坊的15位嘉賓.
一齊出來與我們唱這首歌.
請你們來到台前.

$^{481}$不要害羞,在前面.
(音樂).
其實.
就這個陣容.
我覺得已經很震撼.
這麼多的生命.
我們看到原來被看見的生命.
被連結的生命.
就是有力量的生命.
我們一起唱吧.
是你的生命.
啟發我的心.
原來愛無障礙.
伸出堅韌.
是你的生命.
害我眼看得清.
原來順覺超越.
眼所見已所聽.
溫柔的生命.
發出普通強勁.
如果率性.
如果輕鬆.
靠主心堅定.
短暫的生命.
成就不死使命.
讓我高呼.
讓我唱.
我原來造就.
溫柔的生命.
發出普通強勁.
如果率性.
如果輕鬆.
靠主心堅定.
短暫的生命.
成就不死使命.
讓我高呼.
讓我唱.
我原來造就.
讓我高呼.
讓我唱.

$^{521}$我原來造就.
請大家低頭祈禱.
師父你是最溫柔的生命.
你來到我們當中.
謙卑服侍.
被釘在十字架上.
還有當我們連結於你的時候.
我們求你幫助我們.
也以這個無障礙的愛.
去連結身邊的人.
還有就求你.
在我們今天的工作坊裡面.
親自跟我們每一個說話.
讓說的聽的.
都感受到.
你在當中呼喚我們.
我們祈禱交托.
奉主明求.
阿們.
請你們留步.
阿們.
(字幕由 Amara.org 社群提供).
\newpage



\section{啟示錄}
\label{sec:_p1k_J5ZNow}
\textbf{【疫有嘢學 │ 延SUN在線 】啟示錄三個七看天災|辛惠蘭博士}
\newline
\newline
連結: \href{https://youtube.com/watch?v=-p1k-J5ZNow}{\texttt{ https://youtube.com/watch?v=-p1k-J5ZNow}} ~~~~ 語音日期: 2020-05-31 
\newline
\newline
\hyperref[sec:kR2ujHQel1E]{\small{< < < PREV SERMON < < <}}
~
\hyperref[sec:index]{\small{[返主目錄]}}
~
\hyperref[sec:Hs1Y_XrlxkM]{\small{> > > NEXT SERMON > > >}}
\newline
\newline
$^{1}$各位同學大家好.
承接上個主日講過聖經的第一卷書《創世紀》.
今個星期的《易有野學》延伸再線.
是輪到講聖經的最後一卷書《啟示錄》.
課堂邀請了忠臣聖經科新約的副教授.
申惠蘭博士負責.
Joyce老師的研究興趣很廣泛.
有彼得前書,早期基督教歷史,早期教父著作等等.
當然亦包括今日要講的《啟示錄》.
接下來她會和我們分享《啟示錄》3.7漢天災.
順便賣下廣告.
Joyce老師會在七月份的延伸課程會教授提摩太前書.
現在已經可以報名.
雖然這個星期我們的課堂是關於聖經的最後一卷書.
但並不是預告《易有野學》延伸再線會停在這裡.
忠臣延伸部六月份已經邀請了四位老師.
在主日這個時間裡面.
繼續和我們分享他們的教學和研究心得.
這幾位老師包括雷敬業博士,潘怡容博士,林添德先生,陳關運兆老師.
盼望弟兄姊妹你們都能夠繼續在《易有野學》延伸再線裡面.
延伸我們對基督教信仰的各方面認識.
沿著今日藉著Joyce老師的教導.
讓我們更深去認識上主的管教和啟示.
我們的系列叫做《易有野學》.
我今日都想和弟兄姊妹思想一下瘟疫.
想一下天災.
雖然我們面對肺炎疫情已經好幾個月.
現在疫情也都已經慢慢減退.
再說天災或者我們會覺得好像已經是沒有了一份的迫切性.
不過我想當疫情這樣爆發.
每天都有新的消息要我們立即去應對.
或者當時我們根本就沒有空間去想清楚究竟是發生什麼事.
今日當疫情慢慢過去.
我想亦都是時候我們真的停下來.
好好地在信仰上去反思去領受.
究竟我們這位掌管著世間發生的一切事情的上帝.
其實在這段時間在疫情裡面.
他可能在做些什麼.
他亦都想我們信徒去體會些什麼.
我今日想和弟兄姊妹一起去讀.

$^{41}$約翰啟示錄裡面記載.
自己在天上見到的那三個七.
即是七印,七號和七碗的異象.
或者我們都知道在那裡大部分都是驚天動地.
有很多人在裡面苦不堪言的災難.
有時我們想看都真的挺難看下去的.
不過當約翰他發出這三個七的警告的時候.
他並不是像我們過往很多人以為的那樣.
只是要記下末日的密碼.
來滿足一下我們對將來的好奇心.
約翰其實是要鼓勵當時在小亞細亞裡面的信徒.
縱然周圍的確充滿著對基督信仰的挑戰.
甚至是對自己生命的威脅.
他們仍然要持守神的道.
仍然要堅持繼續為基督去做見證.
所以約翰見到的那三個七的災禍.
並不是一些沒有意義的偶發性的事件.
也不是一位大能的神.
他只是在玩弄人類.
這對令人觸目驚心的和環裡面.
反而有著安慰,有著警告,有著悔改的呼籲.
裡面更加是充滿著對人類的硬心那種無奈.
當然我今天並不是要嘗試去解釋.
為我們為何會有肺炎疫情來提供一個答案.
我想我也沒有資格去揣測究竟神在做些甚麼.
不過我很想邀請弟兄姊妹跟我一起去讀約翰這三個七的經文.
看看裡面會否對我們各自在面對的處境.
對我們也有些啟發.
在我進入經文之先.
我想首先交代一下約翰寫啟示錄的背景.
啟示錄我們比較認識.
也可能比較容易明白的部分.
相信就是基督要約翰寫給小亞細亞.
即是今天在土耳其一帶七間教會的七封信.
這七封信也揭示了當時在這個地區生活的信徒.
他們正面對周圍的多神文化.
正面對著很多的壓力.
要他們在信仰以至在生活上去做一些妥協.
去融入周圍社會的文化.
因為當周圍的異教徒.

$^{81}$他們就像我們中國人的傳統文化.
可以去滿天神佛去甚麼神都拜.
基督徒卻要堅持拒絕去拜其他的神.
而在古代的希臘城市.
他們每次的宗教節日和活動.
都是在一些全城的城市.
任何的社交活動,飲宴.
也通常包含了一些祭祀.
又或者對其他神明的宗教.
尊崇其他神明的一些儀式.
而作為異教多神文化的一部分.
當時在城市裡面的地方官員.
和本地的鄉紳貴州.
他們為了去討好羅馬帝國的皇帝.
希望羅馬的君王能夠給一些好處.
又或者一些方便給他們自己的城市.
他們還會在城市裡面.
很積極地去推動.
在城市裡面向帝王的崇拜.
如果能夠為皇帝建省級的皇廟.
得到看守皇廟的地位.
這些都會成為整個城市的榮譽和驕傲.
而在約翰社,啟示錄給他們的.
七個城市裡面.
還是特別出名,特別熱衷帝王崇拜的.
在這七個城市裡面生活的信徒.
他們在異教文化,政治和宗教的要求挑戰底下.
他們為了生存,為了有正常的社交生活.
甚至為了商業的利益.
他們應該採納.
如何去採納周圍哪些異教徒的生活習俗.
又或者他們如何堅持.
為自己的信仰去做見證.
在基督給這七間教會的信裡面.
我們固然見到.
基督稱讚仕美娜教會.
寧願遭受經濟上的杯葛和損失.
搞到自己貧窮.
他們都仍然堅持拒絕去參與周圍異教的經濟和社交活動.
我今天會用的是新譯本的經文.

$^{121}$另外,基督也稱讚菲拉鐵非教會.
雖然他們只是略有一點的綿力.
不過他們都是不問結果地.
公開堅持自己基督的信仰.
進行基督關於忍耐的教導.
不過在其他的教會.
出現了一批假仕徒,假教師.
包括在以弗所教會裡面的假仕徒.
在以弗所和別加摩教會裡面的尼哥拉黨.
在別加摩教會裡面的巴蘭.
和在推亞推拉教會裡面的耶駒別.
他們鼓勵信徒可以自由去參與.
去吃那些祭過偶像的食物.
讓他們可以再沒有保留地去參與.
去融入周圍羅馬帝國的社會系統.
基督尤其責備得嚴厲的有撒迪教會.
他們外表朝氣勃勃.
亦都很積極地去參與周圍城市的生活.
去和周圍的社會打交道.
於是他們得到一個活的名聲.
不過看在基督的眼裡.
他們在屬靈上其實是死的.
基督還叫這班睡了的基督徒去醒一醒.
去警醒等候他們回來.
而我們很熟悉的魯迪加教會.
他們很明顯有把握到.
城市提供給他們的經濟機會.
分享得到城市裡面的財富.
以至到教會亦都有錢起上來.
於是他們自滿.
以為自己是一樣都不缺.
不過基督給他們的信卻是說.
他們真正的屬靈光景.
其實是貧窮,是盲,是赤身露體,是羞辱.
所以當時教會的信徒.
他們有很多根本就沒有在意周圍社會.
異教的文化對自己信仰的威脅.
他們沒有準備好去受苦.
去為自己的信仰付出代價.
基督說他們睡了.

$^{161}$他們盲.
約翰他自己卻是見到.
當時在別加摩的教會裡面.
已經有一個叫安提柏的信徒.
因為見證基督而被殺.
還可能是死在羅馬官府的手下.
而約翰他是一張糾結.
他說自己是因為神的道和耶穌的見證.
在分享患難和忍耐.
他在別加摩島上.
很可能同樣被官府在流放.
當約翰他見到情況越來越急.
挑戰和逼迫.
越來越真實,越來越靠近.
但很多信徒沒有準備好.
他要怎樣寫啟示錄.
呼籲信徒堅持.
守住信仰的位置.
好好等候主回來.
他又怎樣在異象裡面.
見到的災禍去堅固.
去安慰那些信仰付出代價的信徒.
他亦要怎樣警告那些.
跟隨異教文化習俗去拜偶像.
包括去參與帝王崇拜.
而不肯以神為神的人.
他怎樣去催促他們去悔改.
不如讓我們首先看看七印.
七印的經文.
在書卷的第六章.
是延續之前第四和第五章.
天上敬拜的場景.
在裡面高揚基督.
他接過用七個封印封起的書卷.
這書卷是要揭示.
神對眼前這個世界去到最後.
即是歷史終末的計劃.
神這個計劃.
是要為我們這個扭曲的世界去撥亂反正.
是要將我們帶進新天新地裡面.

$^{201}$而這七個封存書卷的印.
在他們七個全部打開之前.
打開之前.
書卷裡面載著的末日事件.
是還未展開的.
換言之.
這七個印要揭示的事件.
並不是關乎萬物最後終末的結局.
而是約翰對於眼前的局勢.
繼續演變下去.
他要發出的先知預言.
他這個七印是要為書卷真正打開之後.
要揭示的事件去鋪路.
是七號和七碗的前奏.
他們是要解釋為何神最終要出手.
用七號和七碗去干預.
甚至去叫停地上發生的事.
其實七印,七號和七碗.
他們在佈局上有一個共通點.
就是頭四個可以獨立成一組.
第五和第六個有較多的相似.
而第七個就比較獨立出來.
所以每個七的系列.
都是四,二和一的佈局.
我們首先看頭四個印.
是那四個騎士.
六章一至八節.
約翰頭四個印的預象.
很可能是來自撒加利亞書第一章和第六章.
不過約翰這四個騎士.
他們是代表著佔領征服,戰爭,饑荒和死亡.
是帶來一件人禍,多禍,天災.
第一個騎士是騎著白馬.
他手上是拿著弓.
這是古代征戰通用的象徵.
經文說有官勉.
即是有權柄和能力.
去比過這個騎士讓他出去.
即是出征.
他還不停地出征成功.

$^{241}$不斷地擴展自己的版圖.
所以第一個騎士是代表地上的強國.
在約翰當時正值四周東征西討.
在霸權高峰的羅馬帝國.
羅馬的帝王.
他們靠著自己的軍事實力.
四處攻佔,征服.
不斷地將越來越多的地方.
納入自己控制的裡面.
野心征服帶來下一步就是戰禍.
第二個人打開是騎著紅馬的騎士.
他血腥殘酷是代表著戰爭.
《六章四節》說他讓人自相殘殺.
奪去地上僅有的和平.
而只有連連的戰爭和世界大亂.
他手上的是一把大劍.
代表著人類最大的武力.
不過是用來流人的血.
去令其他人的生命悲慘,恐怖.
更慘的結果是第三個人.
出現的是黑馬.
這次騎士握著的是天秤.
天秤不是代表公義.
而是在連連的征戰動盪的底下.
地上的糧食非常短缺.
而要去配給.
所以這個騎士是代表著饑荒.
他也帶來經濟的剝削和不公.
第六節.
一個銀幣又或者一個他連得.
即是當時工人一日的工資.
不過只能夠買到很少的食物.
而當饑荒必需的糧食價格飆升.
奢侈品.
好像橄欖油和酒的供應.
卻是維持正常.
繼續拉開緊.
帝國裡面的貧富懸殊.
貧者越被剝削缺乏.
有錢人卻是越能夠得到他們想要的享受.

$^{281}$在驟眼看來是天災的背後.
可能更多是人禍.
最終就是第四個騎士.
約翰在第八節開宗明義地說.
他的名字叫做死.
他騎的那匹馬的顏色.
在原文是一種黃黃綠綠.
好像人有病.
甚至是腐爛屍體的顏色.
經文還說陰間是跟著他的.
在連綿的功佛,戰禍,饑荒和剝削之後.
剩下的就只有屍橫遍野.
死亡和陰間得到他們的權勢.
四周到處都是死亡,毀滅和哭號.
都是以人命來餵養自己的野獸.
所以雖然這四個騎士.
是由神面前的活物呼喚出來.
是神容讓他們發生.
不過在天災看來是不可抗疫的裡面.
或許更多的是人禍.
是人類沒有制約的驕傲,野心,貪婪.
甚至是自作自受的結果.
約翰要發出先知的聲音.
他要求自己的讀者.
要求所有那些以為能夠跟周圍世界做好朋友.
可以從帝國社會得到一些好處的基督徒.
約翰要他們認清.
自己其實是生活在一個怎樣的世界裡面.
是一個充滿著驕傲,貪婪.
充滿著欺壓,剝削.
一個視人命為草芥的世界裡面.
基督徒要堅持向神效忠.
守住神的道.
堅持為基督做見證.
的確會為自己帶來很多麻煩.
甚至是性命的危險.
這樣第五個印也自然要揭開.
其實第五和第六個印正在提醒我們.
今天在地上生活的只有兩種人.
一是高洋基督的追隨者.

$^{321}$那些聖徒.
他們在第五個印揭開的時候出現.
另一種就是在第六印出現的那些住在地上的人.
當第五印打開.
是一群祭壇下的亡魂.
是一群堅持以基督為主的高洋追隨者.
第九節說他們是為了神的道.
和為了自己所作的見證被殺的.
所以這一群是一群殉道者.
他們已經將自己的生命獻上給神.
是在祭壇的下邊.
在舊約以色列子民的獻祭裡.
在祭壇下邊是要收集祭身的血.
這群在祭壇下的殉道者.
他們向神問了一個書卷裡面很重要的問題.
正正也是之後七號和七碗要回答的.
就是第四第十節.
聖潔真實的主啊.
你不為我們身流血的冤要到何時呢?.
而流他們血的人.
就是那群在啟示錄裡面.
被約翰喊了做住在地上的人.
不過很奇妙.
這群在祭壇下邊的殉道者的亡魂.
他們不是問神會不會.
而是問何時.
因為他們認得神是那位真實的主.
縱然他們的確是經歷著自己無辜被殺害.
惡人卻並未遭報的荒謬.
不過他們仍然抓住對一位他們認識是真神.
真實而聖潔的主的盼望.
他們知道也相信.
這位聖潔而真實的神.
最終是會為他們伸冤.
是會審判那一群要為他們所受的害.
要負責任的人.
問題不是會不會.
而是何時.
不過這群在祭壇下的亡魂.
他們在第五天得到的答案.

$^{361}$可能的確沒有辦法給他們即時的安慰.
第十一節.
有聲音叫他們多等一會.
因為他們要等那些.
和他們一樣被殺害的人.
他們要等湊夠數為止.
原來神的伸冤審判.
並不是像人們常期望的會即時發生.
反而情況只會越來越差.
約翰知道.
別加摩殉道的安提柏.
他不會是唯一一個.
在羅馬帝國鐵腕統治下.
當其他人都熱衷帝王崇拜.
殉道者會陸續有來.
情況只會越來越危急.
苦難也只會越來越多.
而不是老底家教會.
又或者殺敵教會的人.
以為的那麼樂觀.
當要追隨一位.
在羅馬帝國手下被釘十字架的主的君王.
有誰可以逃避.
不用像他這樣去受苦呢?.
不過啟示錄作為天啟文學.
它就是要揭示.
在看得見的現實.
在眾多令人沮喪.
在看到絕望的情勢的背後.
肉眼看不見的事實的真相.
真相就是.
神仍然坐著為王.
掌管著地上的一切.
要怎樣和何時發生.
所以第六人出場的是.
住在地上的人.
要預示神對他們的審判將要來到.
揭示的是地震.
在舊約先知書的寓言裡.
地震是神憤怒.

$^{401}$又或者得審判的象徵.
而在啟示錄.
地震更加是神審判的前奏.
即是說第五人受苦.
聖徒他們的哀告.
最終是由真實的神.
藉著天災而得到回答.
所以那些殺害神中心的聖徒.
那些住在地上的人.
他們最終會知道.
是神要向他們去發烈怒.
第十四節說.
天好像捲起了一樣.
是不見了.
山嶺和海島都從遠處被移走.
這些都是地震直接的結果.
於是第十五節.
那些住在地上的人.
包括那些在本土掌有權柄和財富的人.
他們都躲在山洞和岩月裡.
經文說得很特別.
他們並不是要逃避地震.
而是要逃避坐在寶座上.
即是他們要逃避神的面.
他們也要逃避羔羊的震怒.
他們寧願在地震裡.
被石頭和大山摧毀.
寧願接受死亡.
他們都不願去面對永恆的上帝.
和羔羊的基督.
第十七節.
這班住在地上的人.
他們還問了書卷另一個很重要的問題.
就是他們.
即是神和羔羊基督.
他們震怒的大日子來到.
誰能夠站立得住?.
這個問題很重要.
首先是因為.
當第五天人為神殉道的聖文.

$^{441}$他們問神.
神為我們犧牲流血的冤要到何時?.
第六章這裡.
是要扭轉之前的問題.
問題並不是.
神沒有為子民伸冤要到何時.
真正的問題卻是.
當神真的出手.
去審判那些住在地上的人.
當神和羔羊基督的憤怒.
真的發出.
有誰能夠站立得住?.
所以這個問題.
在三個七節裡.
也起了關鍵的作用.
如果七人.
是要為書卷真正展開.
世界終末要發生的事.
來鋪路.
這樣.
神和羔羊的震怒真的來到.
誰能夠站立得住?.
也是七號和七碗的主題.
雖然目前的世界.
看到的貪婪,慾望,野心的征討.
暴力戰爭,饑荒,惡削.
令到多人死亡.
全部其實並不是天災.
而是人禍.
雖然基督徒要效忠基督.
要堅守主的道.
只會面對仇視,排擠.
甚至是殉道.
也只會看到.
不斷有人為主犧牲擺上.
當眼前的邪惡力量.
四騎士的人禍.
只在越演越烈.
未見絲毫消退的跡象.
七號和七碗的應許.

$^{481}$就是要肯定基督徒.
這個地球仍然是神的世界.
神將要用七號和七碗的天災.
為我們的世界去撥亂反正.
將邪惡推翻.
預備我們最終要進入新天新地的落景.
所以去到第八章.
當高揚基督打開第七人.
他手裡的書卷.
傳言的開啟.
也是人類目前地上世界的終局.
要正式的展開.
所以七人一打開.
好像甚麼都沒有發生.
就見到七之號的出現.
「吹號」在舊約先之書裡面.
其中一個用法.
就是要警告,審判,張曉,杜林.
不過在「吹號」之先.
約翰在八章第三至第五節.
為我們加插了一段前奏.
異象的場景.
也都轉換成一座天上的聖殿.
一位天使.
他好像祭司一樣.
將聖徒的祈禱和香.
一起獻上.
升到神的面前.
即是神指引的禱告.
尤其是第五印祭壇下的殉道者.
那些被苦害致死的聖徒.
他們的哀告已經直達.
神寶座的面前.
接著第五節.
天使他拿著香爐.
裝滿了壇上的火.
然後扔在地上.
之後就是第七號.
就是七號天災的預警.
雷轟,響聲,閃電和地震.

$^{521}$即是說七號要降在地上的天災.
正正就是神對自己子民的禱告.
甚至是哀告的回答.
所以七號.
其實之後七碗都是一樣.
是在呼應摩西出埃及的十災.
在七號裡面呼應的十災.
包括水變血,冰雹,蝗蟲和黑暗.
正如神當年.
他是怎樣將以色列民.
從埃及法老的奴役.
和逼迫的裡面拯救出來.
今日.
神同樣都應運了聖徒.
他們的禱告.
神要照樣以天災去擊打世界.
去向住在地上的人去示警.
神要將自己的基督的子民.
從地上帝國的逼迫和苦害裡面拯救出來.
不過.
七號有一個很特別的地方.
就是它只是七碗最後審判的警告.
所以每次號筒的響起.
地上世界就只有三分之一受影響.
因為每一次號筒的吹起.
每一輪的天災都只是一次的管教.
其實是神的憐憫又一次的展現.
又一次的發動.
剩下三分之二.
就是每次都是一個悔改的機會.
但是當住在地上的人.
他們最後始終都不肯悔改.
憤怒的碗就會被傾倒.
他們就不會再有機會.
七號和七引一樣.
都是呈四,二和一的佈局.
頭四災的鋪排是一樣.
都是非常簡潔.
都是天使吹號.
然後就是災殃.

$^{561}$最後就是三分之一的結果.
最後那三個號是由八章十三節.
在高空飛翔的鷹大叫三次.
「有禍了」來引入.
所以我們首先看頭四個號.
第一支號.
冰雹和混雜著血的火.
目標是地.
說的是雷暴.
以至當冰雹和閃電.
即是火打在地上.
就燒毀了三分之一的樹和草.
即是地上出產的三分之一.
第二支號.
是一座好像在燒的大山.
擲在海裡.
應該是指火山爆發.
尤其是公元79年.
在羅馬的確發生過.
一場很出名的火山爆發.
甚至著名的龐貝古城.
都在這次火山爆發裡被摧毀.
無論如何.
如果第一支號是摧毀地的三分之一.
這次就是海的三分之一變成了血.
就好像出埃及十災裡的尼羅河一樣.
除了令到海裡的三分之一的生物.
因為污染而死.
還有三分之一的船毀壞了.
我們在啟示錄第十八章.
我們知道羅馬.
即是經文裡面說的巴比倫.
他們是靠著其他地方進口物資.
給他們維生的.
即是這次的天災.
是直接對羅馬的審判.
第三號的審判.
再次是來自上面.
一粒燃燒著的大星.
好像火一樣從天掉下來.

$^{601}$應該是類似隕石的墜落.
不過這粒星被約翰叫做「苦茙」.
「苦茙」又或者和合本叫「茵塵」.
又或者「苦艾」.
是一種沒有毒.
不過有苦味的植物.
「苦茙」是要反映當這粒星.
這粒隕石掉落江河裡的時候.
三分之一的水源就變成了苦.
在約翰寫書第一世紀的時候.
當時的醫藥還是很落後.
喝了受污染的水源的苦水.
的確可以讓人致病.
甚至是致死.
所以這次災禍的目標.
不單只是清水.
而是要指向人.
第四枝展開的號的天災.
是和出埃及的第九災.
即是黑暗之災是一脈相承.
因為白天和晚上.
日月星的三分之一都變成了黑.
《白章》第十二節說.
白天的三分之一沒有了光.
晚上也是這樣.
說的應該不是地上的光度少了三分之一.
而是一天之內多了三分之一的時間.
完全黑暗.
如果很多人.
就算是基督徒都以為.
殉道,死亡已經是人間終極的悲劇.
神讓世界全然黑暗的那種恐怖.
或者更加令人生不如死.
不過在先知.
例如約爾書裡面說.
有血,有火和有煙柱.
太陽變為黑暗.
月亮變為火紅.
只是在耶和華偉大.
可畏的日子臨到之前.

$^{641}$要發生的事.
即是他們不是終極審判和死亡的刑罰.
而是有時間性的.
是天上的神.
祂不喜悅地上的情況.
祂要表達和祂的警告.
豪和裡面的天災.
其實是要呼喚並且要留下悔改的門.
讓人趁著仍然有機會.
去停止犯罪.
停止繼續去得罪神.
是要讓人回轉悔改的.
在一連串四個豪之後.
第十三節經文說.
有一隻鷹在空中飛行.
鷹在原文也可以解作「禿鷹」.
無論如何兩隻都是吃腐肉的.
都是猶太人視為不潔的.
所以一隻不潔的吃腐肉的鳥.
大叫三次「有禍了」.
又或者和合本翻譯的「禍災」.
已經給了我們很多不祥的預感.
知道接下來的第五和第六號.
一定是非同小可.
當第五人在祭壇下的聖徒亡魂.
他們問神.
你不審判住在地上的人要到何時.
現在就由吃腐肉的鷹宣告.
如下這些豪的天災.
就是要對準住在地上的人.
他們的審判和警告.
「吹號」是要提醒住在地上的人.
雖然他們住在地上的確未必是問題.
但當住在地上而又當天沒有到.
又或者以為天沒有眼.
是不會干涉在地上的人在做甚麼的話.
這樣其實真是太過驕傲和愚蠢.
「吹號」就是要宣告.
神仍然在統管地上的世界.
住在地上的人不可以當他沒有到.

$^{681}$所以在第九章的第五和第六號.
非常之恐怖.
第五號有一顆天體.
約翰只能叫它一顆星.
經文說這個天體被猜險.
來到地上打開了一個無底坑.
裡面像火爐一樣充滿著煙.
第十一節還說.
這個無底坑有一個天使作王統治.
他的名字叫阿巴頓.
又或者是阿波倫.
即是一個破壞者.
這個作王的天使應該是指魔鬼.
所以這個充滿著煙的無底坑是指地獄.
更加令人連雞皮都豎起了.
當無底坑被神猜險的那顆星打開之後.
有很多蝗蟲跑了出來.
雖然是在呼應埃及的第八災.
不過這群蝗蟲卻不只是蝗蟲.
因為它們還可以像蠍子那樣去叮人.
說著說著它們還成為了一支魔鬼兵團.
披上了盔甲.
全副武裝地在煙霧裡面出動.
去傷害住在地上的人.
所以整件事其實都是非常之不自然.
我很喜歡學者賴特的說法.
他說:一個這樣有整支怪物兵團走出來.
四周去傷害和擾亂.
還是由一個破壞王來掌管的一個無底坑.
就好像一處反創造,反物質.
只有破壞和混亂的地方.
世界雖然本來是神所做和所愛.
現在卻變成叛逆成性.
一個反創造,有一味破壞的世界.
不過第五號裡面.
這個蝗蟲加蠍子混合體的怪物兵團.
它們又好像訓練有素.
它們是嚴格受到限制.
不准去傷害植物.
這些其實才是蝗蟲天生的對象.

$^{721}$它們也不可以去傷害那些有神印記的人.
一會兒我會介紹多一點.
這群受神印記保護的人.
無論如何.
這個蝗蟲怪物兵團.
它們要傷害.
就只能夠傷害那些沒有神印記.
就是那群住在地上的人.
不過就算這群住在地上的人.
他們要受痛苦.
也只會是五個月.
也就是一場蝗災通常維持的時間.
所以第五號同樣只是警告.
而不是最後人們不能回頭的審判.
就好像《約爾書》第二章那支蝗蟲預警的軍隊.
同樣只是在耶和華偉大可畏的日子.
臨到之前要發生的事.
所以去到九章第十二節.
約翰就說.
第一樣災禍過去了.
還有兩樣災禍要來.
繼續就是第六號.
第六號響起.
就好像是最後的戰爭一樣.
有四個原先被綁住的天使放了出來.
要去殺害地上三分之一的人口.
約翰的讀者對於「又法拉底河」不會陌生.
因為「又法拉底河」正是羅馬帝國東邊的邊界.
對面就是帕提亞帝國.
歷史上羅馬人是曾經兩次被帕提亞帝國打敗.
所以當約翰寫到四天使被釋放.
還帶來了一支超乎想像的龐大軍隊.
第九章第十六節.
馬兵的數目是二萬萬.
其實就是不知道有多少.
總之就是很多.
約翰的讀者他們看到.
自然就知道是在說大軍壓境.
對象甚至可能是羅馬帝國這個世界的霸主.
這支馬兵同樣像一群怪物.

$^{761}$那些馬的頭像獅子的頭.
還有火,煙和硫磺在他們的口裡噴出來.
他們的尾巴又像一條蛇.
還有一個蛇頭要來傷人.
所以又是一個混合體的怪物.
他們是由馬用像獅子的口去殺害.
還有用他們像蛇的尾巴來傷害人.
而不是由坐在上面的人去出手.
約翰將一個驚心動魄,兵荒馬亂的場景.
擺在一個其實已經夠了恐怖絕倫.
令人毛骨悚然的大蝗災之後.
他就是要提醒所有的人.
或者都在提醒我們.
其實世界的恐怖和痛苦.
正是不斷地在升級.
住在地上的人已經到了非要悔改不可的時間.
不過很遺憾.
神怎樣放手被邪惡傾巢而出.
被壓力人催促人去悔改.
神怎樣為人留下機會和悔改的空間.
並沒有得到住在地上的人去領情.
第20和第21節.
人的惡真是根深蒂固.
留下的住在地上的人.
他們根本就沒有理會天災的警告.
而只是繼續去拜偶像.
不肯以神為神.
更加不肯慰問他們在帝國生活裡面的殺戮傷害.
那些邪惡謊話淫亂濫交和偷渡貪婪去悔改.
他們反而變本加厲去迷戀.
就好像當日埃及的法老一樣的硬心.
看著面前那麼多的天災事件.
看著創天造地的上帝.
在這些災難裡面的作為.
他們竟然可以是無動於衷.
令到自己真的要面對最終的審判的時候.
就只剩下自食其果的罪有應得.
所以三個七.
神對世界的審判去到七碗.
亦都到達了高潮.

$^{801}$當天使在第十一章第十五節.
祂吹響了第七之後.
我們再次看不到有什麼發生.
而只是聽到天上的聲音說.
世上的國已經屬於我們的主.
和祂的忌妒.
之後又是一段的等待.
中間是第十二和第十三章.
只是看到邪惡的羅馬帝國.
即是第十三章的海獸.
和它的地區爪牙.
即是第十三章的地獸.
他們繼續如日中天.
他們繼續橫行無忌.
繼續殺害逼迫神的聖民.
到了七碗.
就是帝國的君王和黨羽.
要最後得到他們應有的審判.
所以第十五章.
在天使將莊主神憤怒的七碗.
倒在地上之前.
第二至第四節的流離海之歌.
是源於主.
你公義的作為已經顯明出來.
而這七個碗.
同樣是對應十災裡的五個.
包括水變血.
災,滄災,國災和黑暗.
不過這次在四個碗之後.
約翰特別加入了一句.
第十六章九節.
約翰說他們並不悔改.
也不將榮耀歸還給神.
所以七碗是裝滿了神的烈露.
而一個碗是倒在地上.
對象已經是那些有受的記號.
和拜受像的人.
即是第十三章裡的那些.
向帝國和帝國生活妥協.
已經被帝國收買.

$^{841}$甚至參與帝王崇拜的人.
他們身上是要掌了毒瘡.
而第二碗是海水變血.
第三碗和第三支號一樣.
都是倒在清水的水源.
不過和第三號不同.
這次連清水也變成血.
是要以血還血.
是要報帝國流人血的罪.
第六節.
長官中水的天使說.
他們曾經流聖徒和先知的血.
現在神要給他們血喝.
所以公義的神是佩德讚美.
原來神是不會忘記.
受屈聖徒的苦情.
神自然要紀念他自己的子民.
在這個時候.
第七節.
就連天上的祭壇.
都有聲音來發出.
「是呀,主呀,全能的神.
你的審判真實,公義」.
如果祭壇是聖徒的禱告被獻上的地方.
第五天.
祭壇下是因為持守神的道.
和自己的見證而被殺聖徒的亡魂.
他們稱呼神是真實的主.
現在他們最終是受到安慰.
他們確認得到.
神真的有聽到他們的哀聲.
神最終也有顯明他們的真實和公義.
因為已經不再是三分之一.
一切在海裡的生物都死了.
所有河川都變成了血.
再沒有悔改的空間.
是神最終的審判和懲罰.
所以在第四個碗.
太陽就像沙漠裡的沙漠一樣.
要將人灼傷.

$^{881}$當人被高熱燒燒.
他們根本看不到.
神的手呼籲他們要悔改.
他們反而去褻瀆神的命.
約翰在這一句的撮要裡.
他為我們說出了.
住在地上的人一個很普遍的現實.
就是儘管人事上都當神沒有到.
事上都想完全離開神掌管的生活.
不過當有事不順.
當有什麼風吹草動.
頭暈身痛.
他們第一時間就去埋怨神.
去褻瀆那個.
其實真的給了很多次機會他們.
真心愛他們的神.
這樣也將我們帶去之後的兩個碗.
碗最後一輪的天災.
是直指第十三章的海獸.
即是羅馬帝國.
第五碗是倒在獸的寶座上.
是直倒獸的心臟.
於是整個獸的王國就陷在黑暗的裡面.
呼應埃及的第九災.
那些追隨羅馬帝國生活和政策的人.
雖然他們已經痛苦到咬著自己的舌頭.
不過約翰再次重覆.
他在書卷留下的一份遺憾.
就是第十一節.
這班住在地上的人.
他們仍然是褻瀆天上的神.
在一個不順的世界.
當人順境的時候.
固然是不會想起神.
但是當災害到臨.
當情況是強差人意.
他們就會將一切的責任推給神.
在世界裡面發生的任何壞事.
都可以成為人埋怨神的藉口.
經文說.

$^{921}$他們並沒有為自己所作的去悔改.
任何悔改的機會.
都是被人踐踏成對神的褻瀆和詛咒.
所以很遺憾.
到最後當然就是一場最後的戰爭.
真正的戰爭是記載在書卷的第十九章.
第六本揭示的是這場戰爭的序幕.
就是「幽法拉底河的河水乾涸.
讓羅馬帝國的盟友可以來會合」.
為什麼他們會來?.
因為第十四節.
在第十二和第十三章裡的邪惡三一.
即代表魔鬼的龍.
代表羅馬帝國的海獸.
和代表地方推動帝王崇拜的地獸.
有人在這裡叫他們做「假先知」.
從他們這個邪惡三一的口裡.
有污淩出來.
形狀好像青蛙.
又再提醒我們.
埃及的十災.
不過這次這些青蛙.
他們好像青蛙的污淩.
他們是要施行歧視.
去用他們此時而非.
跳來跳去的悠康滑調.
去迷惑地上的蛛王.
引誘他們在這個已經喊了做.
全能神的大日.
這個日子.
他們聚集.
參與這場他們根本就不會贏的戰爭.
他們集合的地點叫做哈米吉多頓.
在啟示錄的全息歷史裡.
給了很多人很多的想像.
不過哈米吉多頓.
這個名字的意思.
其實只是米吉多的山.
而在舊約聖經.
米吉多正正就是很多戰爭發生的地方.

$^{961}$也很適合約翰用這個地方名.
來象徵這一場終極的世紀大戰.
總而這個地方名.
又或者這個地點.
本身並不重要.
最重要的是.
所有邪惡的勢力.
他們都集中在同一個地方.
他們都在等候.
給神去應罰.
所以當這場戰爭還沒有正式打.
第十六章第十七節.
三個七的最後一個碗.
第七碗.
將神的憤怒再一次傾倒出來的時候.
隨即就是神在寶座上.
發出的一句「成了」.
因為基本上一切都被摧毀.
大公告成.
天使在空中倒出碗.
第十八節是自從地上有人以來.
從未發生過的地震.
除了當日在世上的大城羅馬.
得到他們應有的審判.
一分為三.
其他在第十七和第十八章裡面.
和他們同流合污.
追隨羅馬政策的城市.
都一併倒下.
所有的海島和山都要被移平.
不過神的審判.
是不會停留在地和海.
自然更加會包括那些經常褻瀆.
根本就當神並不存在.
這樣生活的人身上.
第二十一節.
當巨型的冰雹打在這班人身上.
約翰再次重覆.
他在七碗不斷地強調的主題.
就是神的天災.

$^{1001}$祂出手的審判.
並沒有帶來那班.
硬心住在地上的人的悔改.
而只能夠帶來.
他們對神的惱羞成怒.
約翰七碗的描述.
是很遺憾地.
沿在人就竊俗神.
因為這些災太過嚴重.
這些災太過嚴重.
或者都是很多弟兄姊妹.
第一次看三個七的時候的感覺.
不過當我們越體會.
周圍被最污染的世界.
是真的有多麼險惡.
或者我們都會越來越明白.
為什麼我們真的需要七號和七碗.
唯有神首先將地上這個扭曲的世界.
徹底地去更新.
第二十一和第二十二章的新天新地.
才能夠全面地展開.
唯有這樣.
神對我們人類世界.
本來有的美好心意.
才可以重新.
完完全全地體現出來.
在這一段.
要迎向神在地上.
要施行七號和七碗的時間.
我們基督徒在地上.
可以怎樣去自處.
我最後想介紹.
三個七之間的兩段間場.
其實在啟示錄裡面.
我覺得間場故事才是最重要的.
不過其他要再有機會.
才去跟弟兄姊妹分享.
我今天只是想看兩段.
十四萬四千的段落.
在錄音之後.

$^{1041}$第七印就要打開.
七號就要發動的時候.
約翰在第七章第一至第八節說.
有一位天使.
向其他四個天使呼籲.
不要去發動神的審判處.
要等他首先將神的印.
印在神的眾僕人的額上.
第四節說.
受恩僕人的人數是十四萬四千.
即是說十二乘十二乘一千.
我們都知道.
十二是神指紋的數目.
十二支牌,十二門徒.
而一千在啟示錄裡面是指很多.
也有人說.
這裡是強調完整.
無論如何.
約翰聽見所有神的僕人.
都會在神將要來的審判裡面.
得到保護.
這樣不代表我們可以免於受苦.
而是我們最終都會安抵彼岸.
與基督同死同復活.
而當第七號響起.
七門未發動之前.
這班會被神的印保護的十四萬四千人.
他們在第十四章再次出現.
這次他們是一班基督的精兵.
第十四章第四節說.
他們是要不讓自己被玷污.
更加是無論高揚基督去到哪裡.
他們都會緊隨其後.
所以最後.
就讓我用約翰在看見第六門發動的時候.
他加入的基督的一句說話.
來完結我今天的分享.
在契斯祿第十六章第十五節.
基督說.
看啊.

$^{1081}$我來是要好像賊一樣.
個個有警醒.
有看守自己衣服的.
是有福的.
這樣他就不用赤身而行.
也不會讓人看見他的羞辱.
祝福各位.
謝謝.
(字幕製作:貝爾).
(字幕由 Amara.org 社群提供).
\newpage



\section{}
\label{sec:Hs1Y_XrlxkM}
\textbf{【疫有嘢學 │ 延SUN在線】以西結看瘟疫|結5章|葉希賢博士}
\newline
\newline
連結: \href{https://youtube.com/watch?v=Hs1Y_XrlxkM}{\texttt{ https://youtube.com/watch?v=Hs1Y\_XrlxkM}} ~~~~ 語音日期: 2020-04-27 
\newline
\newline
\hyperref[sec:_p1k_J5ZNow]{\small{< < < PREV SERMON < < <}}
~
\hyperref[sec:index]{\small{[返主目錄]}}
~
\hyperref[sec:xiMH3MdBCkY]{\small{> > > NEXT SERMON > > >}}
\newline
\newline
$^{1}$主席.
現在我們看到的聖經裡面的以西結書.
是很有組織和緊密結構的.
第一個比較容易觀察到的結構.
就是這本書可以按主題分為三個段落.
第一至二十四章是關於審判猶大和耶路撒冷的神諭.
第三十三至四十八章是講到以色列復興的神諭.
在這兩個段落中間.
二十五至三十二章是一個關於審判列國的神諭.
容許我這樣去形容.
以西結書其實是一本很古怪的書.
書裡面描述的意象包含了很多罕見和怪誕的言語.
是其他先知書所沒有的.
相對其他先知書.
以西結先知亦都反覆使用很多很長很複雜的比喻.
至於今日我們討論的第五章.
裡面提及先知的象徵行為.
就更加是以西結書裡面最突出古怪的視覺效果.
以西結被他同年代的人稱呼做.
「講比喻的人」和「唱情歌的人」.
以西結就好像一個在街上演出的表演者.
他的象徵行為就有些像我們玩遊戲「有口難言」.
只能用動作去描述字卡上的關鍵字或事情.
而先知的象徵行為.
就是以他的身體去表達上主的訊息.
或者可以這樣說.
象徵行為就是讓上主的訊息能夠被視覺化.
從而加強他的說服力.
以西結書四章一至五章四節.
就報告了有四個象徵的行為.
他們都是描述耶路撒冷將要受到圍困.
雖然第四,第五章沒有明顯的時間註明.
不過按照一章一至三節的日期標誌.
第四至五章所論述的歷史背景.
應該就是在耶路撒冷在公元前597年被擄巴比倫.
以及最後在公元586年.
耶路撒冷被毀這一段時間裡發生.
而經文五章五至十七節.
就是為了解釋以西結之所以之前所做象徵行為的意思.
讓人們對這次象徵行為的含義更清楚和明白.

$^{41}$第四至五章的象徵行為.
呈現了一個耶路撒冷事情的次序.
四章一至三節先說到耶路撒冷被圍困的開始.
然後四章九至十一節就說到因為被圍困而引發的饑荒情況.
最後就是今日經文要提及的五章一至四節.
是在描述圍困之後耶路撒冷居民的結局.
而這三個象徵行為的開場白.
其實都有上主命令先知去拿一些東西的經文.
就好像四章一節要拿一塊磚.
四章九節就說要拿小麥,大麥,紅豆等等的食物.
最後五章一節就是今日我們要討論的經文.
一開始的時候上主就命令以西結.
你人子呀拿一把快刀作剃刀.
拿剃刀做甚麼呢.
先知要自己剃自己的頭髮和鬍鬚.
然後就用天平去將鬍鬚和頭髮分成幾份.
三分之一要放在城裡面用火焚燒.
三分之一要放在城的四周用刀砍碎.
而三分之一就任風吹散.
留意第二節說到圍困的日子滿了.
原來焚燒這些蘇法的象徵行為的時間.
是定在當圍困的日子滿了.
於是我們就明白了.
三分之一用火焚燒.
三分之一用刀砍碎.
三分之一任風吹散.
就是代表耶路撒冷居民.
他們在圍困結束.
在城牆被毀之後的命運.
頭三分之一放在城裡面用火焚燒.
一般來說火象徵毀滅審判.
視覺上就告訴我們.
圍城的時候大火處處.
有些人身陷大火之中.
不過之後的經文在第十二節.
會更加具體告訴我們.
圍城大火對耶路撒冷居民的實際影響.
另外有些人他們就能逃過圍城的大火.
不過他們卻被敵人追殺.
死在刀劍之下.

$^{81}$最後的三分之一蘇法就任風吹散.
意味著他們會被分散到列邦之中.
按照第三節上主就吩咐以西傑.
你要從其中.
即是以西傑要從任風吹散的蘇法那裡.
拿幾斤的蘇法用衣服的邊包起來.
那就意味著上主的審判過程裡面.
他同時亦都容許有部分倖存者的存留.
弟兄姊妹你們懂得計數的.
三分之一加三分之一再加三分之一.
即是全部.
上主的審判對耶路撒冷全部的居民是無一倖免的.
不過第三節告訴我們.
上主眷顧了一些的存留者.
一些的倖存者.
在這裡我想插開一點去講一下背景.
我們看到第五章的象徵行為.
表面的對象仍然是住在耶路撒冷的猶大人.
但我們亦都知道其實傳捲以西傑書開始的時候.
一章一至三節告訴我們.
以西傑當時是身處在迦勒底人之地的加巴魯河邊.
明顯的就是以西傑現在做的一些象徵行為.
只有身處巴比倫的秘魯群體可以看見.
所以這班秘魯的群體才是先知要做象徵行為的即時受眾.
雖然以西傑是侍奉在巴比倫這些秘魯的群體.
但是他講論的焦點卻是集中在耶路撒冷裡面.
於是或者你就會問.
秘魯去巴比倫的人為甚麼還會關心耶路撒冷呢?.
會的,其實你會明白的.
因為之前武漢有肺炎的時候.
香港人都很關心的.
甚至有些人會做出一些標語.
就講到武漢要加油這些鼓勵性的說話.
秘魯的群體可能在想像自己只是短暫留在外邦的地圖上.
上主很快會讓他們回到家鄉耶路撒冷.
所以在耶利米書29章21節裡面.
亦都曾經有先知在巴比倫的確這樣告訴耶路撒冷的秘魯之民.
所以他們仍然是很關心耶路撒冷的情況.
不過,以色列的象徵行為就是要讓秘魯的人看見.
事情是不會變得好轉的.

$^{121}$只會越來越差.
第四節,繼續這個象徵行為.
神主命令先知要從其中取一些「瘟災火類」.
在火中焚燒.
必有火從其中出來燒盡以色列全家.
這裡再從其中的「災」字呢?.
意思不是說先知再在第三節裡所講.
藏在衣服裡面的蘇花.
拿幾條丟在火裡面.
不是,不是這個意思.
第四節的「災」字是「再一次」的意思.
是一個重複之前動作的意思.
那「再一次」做甚麼呢?.
就是重複之前處理「淫風分散」的三分一的蘇法動作.
和修本這裡翻譯得很好.
他說「再從其中」.
這個「其中」就在平衡第三節.
你要從其中取幾根蘇法的那個翻譯.
於是我們就明白了.
上主吩咐先知再一次在第二節所講.
最後那三分一份被風吹散的蘇法裡面.
拿一些丟在火裡面.
因此第四節被以色列丟在火裡面的蘇法.
是來自「淫風分散」的那一份蘇法.
而不是包裹在衣服邊裡面的蘇法.
或者你又會再問.
明明剛剛上主才說要拯救一些倖存者.
為甚麼第四節又說.
必有火從其中出來燒盡以色列全家呢?.
理性上我們是知道的.
我們知道並不是所有以色列人都被殺的.
起碼就是現在你看著以色列表演的那些秘魯群體.
他們就肯定自己是沒有被殺的.
正如我剛才說.
秘魯的人他以為自己只是短暫被流放到巴比倫.
甚至可能他們認為自己就是那些倖存者.
所以以西傑澄清.
這個火即是審判是會從耶路撒冷開始.
擴散到以色列全家.
亦都會擴散到秘魯的人當中.

$^{161}$這裡的經文要強調.
將要來臨的審判是沒有辦法阻止的.
亦都是沒有人能夠逃脫的.
你問我上主是否真的那麼殘忍.
要燒掉以色列全家.
我可能會這樣回答你.
你小時候有沒有試過被媽媽打.
我就有了.
我小時候讀小學的時候.
有一次試過出貓.
就被媽媽知道了.
她就拿著藤條一直追著我來打.
我記得我媽媽那時候說什麼.
她說我今天不打死你這個壞小子.
我不是姓葉.
我今天仍然站在這裡.
你就知道我媽媽是沒有真的打死我.
我猜你應該明白.
這裡上主其實是沒有真的想殺光他們的意思.
事實上上主已經告訴我們.
在這個沒有人能夠阻止.
沒有人能夠逃脫的審判裡面.
其實已經拯救了某些人的生命.
不過經文說.
審判是燒盡以色列全家.
縱使那些被擄的人.
他們沒有在耶路撒冷被圍困的時候死亡.
他們在巴比律所面對的掙扎.
或者他們見到耶路撒冷最終被毀所帶來的恐懼.
其實都可以算是上主對他們的審判和懲罰.
所以審判不一定指到肉體上要受到災難刀劍的折磨.
審判可以是心靈上的煎熬.
經文就進入第五至十七節的解說部分.
第五節以一個使者公式主耶和華如此說作為開始.
善智在古近東從來都不是預測未來的微卜先知.
他們只不過是上主的發言人.
就好像古代頒佈聖旨的使者一樣.
他們所說的一切就等同當朝天子的說話.
善智就像一班運用純屬修辭技巧的演說家.
他們很小心地鋪排自己的講辭或者象徵行為.

$^{201}$目的就是要傳說上主的說話.
所以每當這些說話要開始的時候.
通常都會有這種使者公式.
第五章這裡使者公式出現.
除了要告訴受眾聽以下的說話是從上主而來的權威說話之外.
亦都標誌著經文是從象徵行為.
轉到上主直接對耶路撒冷發出神諭.
以洗潔書四章一節到五章四節的象徵行為.
是關乎審判的信息.
接著五章的五至十七節的經文.
就是為這些象徵行為提供審判的原因和結果.
我們可以藉著使者公式來了解第五至十七節的經文結構.
首先第五至六節就是宣告這是耶路撒冷這一句說話.
然後七至十節就是第一次的指控和審判.
十一至十二節就是第二次的指控和審判.
然後十三至十七節就是結局.
結局裡面包括耶路撒冷的人知道上主跟他們說話.
然後耶路撒冷就會成為列國的借鑒.
最後十六至十七節就是上主親自的私行審判.
第五節開始的時候是使者公式.
主耶和華如此說.
上主說「這就是耶路撒冷」.
在這裡以西傑清楚地指出.
耶路撒冷就是之前他所做的象徵行為所要描述的城市.
經文形容耶路撒冷曾經是上主賦予一個很尊貴的地方.
很尊貴的位置.
因為他被安置在列國的中間.
列邦也是在他的四圍.
然後第六節.
經文進一步指出耶路撒冷的惡行是過於列國和列邦.
第七節又是另一個主耶和華如此說的使者公式出現.
經文之後就對耶路撒冷作出了第一個指控.
裡面說「因為你們混亂過於四圍的列國.
不遵行我的律例.
不順從我的典章.
甚至也不順從四圍列國的規條」.
第七節似乎好像在重複第六節有關耶路撒冷的罪行.
「因為你們混亂」.
「混亂」這個詞匯在全本希伯來聖經裡面.
有三分之一出現在以西傑書當中.

$^{241}$而且大部份都是在審判埃及的神諭裡面出現.
當中其實是有暗示埃及他自大高傲的意思.
於是我們就明白.
耶路撒冷不遵從上主的律例和典章.
甚至他不順從四圍列國的規條.
是因為他們高傲自大.
第八至十節就給了這個自大所帶來的結果就呈現了出來.
所以主耶和華如此說.
「看!我必與你為敵.
必在列國前在你中間施行審判」.
弟兄姊妹.
你不要誤會這裡上主說話好像癡里根一樣.
「我與你為敵」.
不是這樣的意思.
這裡重複用「我」字.
是要強調是上主自己在列國的眼前要去審判耶路撒冷.
不知道大家有沒有留意到.
在第五至十節的段落裡.
除了用了很多「因為所以」這些連接詞之外.
其實經文節與節之間的用詞顯字是互相緊扣的.
剛才提到第七節的指控.
其實是重複第六節有關耶路撒冷不遵從上主的律例典章.
第七節更加補充.
耶路撒冷甚至不順從四位列國的規條.
然後第八節就說.
既然耶路撒冷比四位列國還差.
那麼上主就必定在列國的眼前.
在你耶路撒冷中間施行審判.
然後第九節.
並且因你一切可憎的事.
我要在你中間.
又是在你中間.
行未曾行過將來亦不會行的事.
第十節繼續說.
在你中間.
父親要吃兒子.
兒子要吃父親.
在你中間.
父親要吃兒子.
兒子要吃父親.

$^{281}$這一節的恐怖程度.
就是兒子要吃了父親.
是超越了利未記二十六章二十九節裡面那種恐怖.
利未記裡面這樣說.
你們要吃你們兒子的肉.
也要吃你們女兒的肉.
不過沒有說子女要吃父母的肉.
另外耶利米書十九章九節.
也說到囚敵和尋索其命的人追逼他們.
使他們落在困逼之中.
我必使他們各人吃自己兒女的肉和朋友的肉.
然後耶利米愛哥二章二十節.
和耶利米愛哥四章十節.
都是說到有婦人親手烹調她自己兒女作為食物.
所以你們看到的.
以西傑書五章十節所描述的.
父親要吃兒子.
兒子要吃父親.
是從來沒有發生或從來沒有被描述過的.
我們可以這樣說.
第十節其實也是在重複第九節.
我要在你中間走 未曾走過.
將來也不會走的事.
這一句說話的意思.
頂姐妹 為何上主要這樣節與節.
彼此緊扣地說話呢.
你們有沒有看過周星馳以前的一套片.
《九品芝麻官》.
在影片裡面.
周星馳扮演的角色.
包龍星練成了一個三寸不爛之舌.
被他在公堂上駁到方唐鏡和李蓮英.
一句說話也說不出來.
你們應該明白我想說甚麼.
上主就好像包龍星那樣.
除了上主沒有說髒話和歪理之外.
上主在第五至十節對耶路撒冷的指控.
一個道理接著一個道理.
一字一字地緊扣.
讓耶路撒冷毫無喘息和反駁的餘地.

$^{321}$所以你說耶路撒冷他們是否該死呢.
為何說了這麼久還未說到瘟疫這兩個字呢.
現在說吧.
上主的審判是公平的.
他要讓疑犯耶路撒冷去知罪.
變無可變的時候才去宣判他的審判.
經文第十一至十二節裡面這樣說.
主耶和華說.
因為耶路撒冷用一切可憎之物可厭的事.
玷污聖所.
所以三分之一的百姓必遭瘟疫而死.
因饑荒而消滅.
三分之一必死在刀下.
最後三分之一分散四方.
並由刀追趕他們.
這節經文跟第二節很相似.
整個十二節都是第二節進一步的解釋.
這裡基本上跟第二節三層的災難格式一樣.
之前我已經說過.
火不單止可以指耶路撒冷.
當時被圍城的時候大火處處.
同時更廣泛的理解可以是象徵毀滅和審判.
第十二節沒有提及火這個字.
卻用上我們特別關心的瘟疫和饑荒.
來具體的去解釋.
或者代替了第二節的火字和審判的意思.
以西傑書第五章提及的審判.
起碼有四種.
有瘟疫,饑荒,刀劍和放逐.
但你不要忘記剛才我在第四節所說.
審判對於某些人來說.
是不需要親歷其境的.
心靈上的煎熬,恐懼的折磨.
其實已經夠他們受的了.
當我們問以西傑他怎樣去看待瘟疫的時候.
以西傑書告訴我們.
瘟疫是上主審判的其中一種災難.
亦是人類普遍災難的其中一種.
當我們說第十二節的瘟疫.
是對第二節的火字或者審判所作出的解釋的時候.

$^{361}$我們不要忘記.
五章一至四節的那個象徵行為.
曾經提及有三個的三分一.
還有火,刀,任風吹散等等的審判形式.
另一方面.
第十一至十二節所提及的指控和審判.
亦都包括了有三個的三分之一.
有瘟疫,饑荒,刀,分散四方等等的字彙.
於是第五章一至四節和十一至十二節.
就成為了一個倒影.
彼此前呼後應.
而夾在中間的五至十節.
就告訴我們上主審判的基礎.
而所有的指控其實都是有根有據的.
換句話來說.
五章五至十節的陳述.
就是上主要耶路撒冷他們死.
都死得清清楚楚.
就以西傑來說.
所有耶路撒冷的背叛.
其實都驅使上主藉著暴力的審判.
去彰顯他公義的主權.
而瘟疫,刀劍,饑荒和放逐.
就是上主他重複使用的災難格式.
當我們從第十節開始.
我們逐節逐節地回到前面看的時候.
我們會發現審判其實由那裡開始呢?.
就是第七節所說.
因為耶路撒冷混亂.
為甚麼會混亂呢?.
就是因為第六節裡面說.
耶路撒冷他們自大.
不遵行上主的律例和典章.
今日如果你問我.
瘟疫在世界蔓延的今天.
我們應該做些甚麼呢?.
我不懂得去判斷.
今次的疫情是否上主的審判.
不過我會說以西傑看瘟疫的角度.
似乎提醒我們要回去認識.

$^{401}$並且遵行上主的律例和典章.
第十三節.
我要這樣發盡我的怒氣.
我向他們發的憤怒停止以後.
自己就得到平息.
這裡就用了三句意思平衡的句子來宣告.
上主的憤怒最終都會平息.
然後經文就說.
他們就知道我耶和華所說的是出於妒忌.
就知道我是耶和華.
我們通常會稱之為一個認知公式.
這個公式以西傑書出現了超過七十次.
全部都是出現在記載上主做完某些事情之後.
目的就是要讓人類能夠認識上主的本性.
讓人知道無論審判或拯救的作為.
全部都是上主權能的彰顯.
你看得沒錯.
以西傑書這裡所呈現的上主.
是會令到我們覺得不舒服,不自然的.
經文就是說到上主就是要將他對耶路撒冷所有的憤怒.
植柱,瘟疫,刀劍,饑荒和放逐等等的災難.
要全部宣洩了才能夠平息.
於是有人就形容以西傑書裡面的上主是一個空殘的神.
不過弟兄姊妹.
我們不要看漏之後那一句的經文是說甚麼.
之後他們就說.
他們就知道我耶和華所說的是出於妒忌.
我就不是那麼喜歡妒忌這個翻譯.
我覺得以前和合本的翻譯.
他們就知道我耶和華所說的是出於熱心.
更貼近原本這個字的意思.
熱心或者熱情.
本身是反映著上主對和以色列所立的約那份執著.
和耶路撒冷的以色列人不同.
以色列人棄絕了上主的律例典章.
也沒有遵行他上主的律例.
上主卻是堅持遵行他和以色列所立的約.
所以他狠狠地懲罰了耶路撒冷.
弟兄姊妹.
你應該明白這裡另外一重的意思.

$^{441}$經文告訴我們.
我耶和華所說的是出於熱心.
不是告訴我們.
上主依然是那位守約.
思慈愛的神嗎?.
不單止耶路撒冷知道上主是出於熱心.
去堅持去遵行他所立的約.
經文第十四節提到.
在四圍的列國當中.
這些列國都知道.
是上主使他的子民耶路撒冷成為芳良.
這裡是重複第五至八節.
必在列國眼前對耶路撒冷施行審判的意思.
在第五節.
上主強調耶路撒冷位於列國的中間.
有列邦圍著他.
以色列沒有提及過.
上主會將以色列差派到列國去宣揚上帝自己.
上主卻定義要以色列成為列國認識他的方法.
以色列假定了這種列國對上主的認識.
是需要以色列和耶路撒冷.
他們藉著遵從上主的律例和典章來達到.
因此以色列就是上主祝福列國.
祝福世界所流向的管道.
又因為他們作為已經被升高的國家.
卻沒有遵守這個律例典章.
他們也失去了在上主面前的尊貴地位.
第十四至十五節裡面提到.
以色列的懲罰就是要受到列國列邦的羞辱.
耶路撒冷失去了他的身份.
責備和辱罵成為了公開的羞辱.
但在第七節裡其實已經指出.
現在以色列的問題是出於他們已經被列國的行為.
意識形態所同化.
上主甚至認為以色列連列國的標準都不達標.
你說死不死?.
現在不是以色列的忠誠成為列國認識上主的途徑.
反而是以色列的不忠.
導致他們要承受災難在列國的眼前發生.
這個審判成為了列國參考的實例.

$^{481}$或者成為了列國的借鑒.
瘟疫或者其他的災難總有會過去的一日.
但我們作為上主的子民.
其實我們身份是沒有變的.
教會作為某一方面以色列的延伸.
我們就是放在列國當中的燈台.
世人的參考例子.
若果其他人看我們和列國的行為是沒有辦法分別出來.
甚至我們比他們更加不可靠.
我們就要問.
我們要問這一段以西傑去看或者去提及瘟疫的經文.
會不會就是提醒我們.
要再次去審視自己在這個約當中.
所應該有所應該負的責任和表現呢.
第五章的結尾.
16到17節.
用了三個重複的詞彙去形容上主射出一些災難的動作.
頭兩個射出.
和合本修訂版已經翻譯了出來.
就是在第16節.
我向滅亡的人射出饑荒的惡戰.
將他們射出毀滅你們.
而第三個射出.
就是在第17節.
和合本修訂版就將它翻譯為.
我要令饑荒和惡獸臨到你.
原本的翻譯其實是可以.
我要向你射出饑荒和惡獸.
那誰射出饑荒和惡獸呢.
並且誰比到瘟疫刀劍要臨到耶路撒冷呢.
上主說是祂自己.
你猜也猜不到.
來到五章結尾的經文.
上主也沒有想過放過以色列人.
最後上主仍然要強調.
就是祂就是那位審判.
施行審判的.
整個第五章的宣講.
對以色列當時的受眾來說.
其實真的很淒涼.

$^{521}$和對他們來說也是很難受的.
但對我們今天來說.
其實以色列第五章裡面的情景.
仍然讓我們難受.
我們經常說.
耶和華是愛.
歌仔也有得唱.
會不會就是因為這樣.
我們就沒有辦法去接受.
以西傑書裡所描述的這位上帝呢.
以致我們會不會好像教會歷史裡的馬吉安.
凡是指出上主是審判的神.
我們就避開不去看這些經文呢.
以致我們將上主.
局限在一個我們自己預設的形象裡面呢.
我很喜歡看以西傑書.
正正就是因為以西傑書.
向我們去揭示一位施行殘酷審判.
令到我們不能夠接受.
或者令到我們覺得不舒服的神.
可能我講完這番說話之後.
就會收到很多投訴.
為什麼你要這樣去形容上帝呢.
弟兄姊妹.
正正是以西傑書所描述的這位神.
祂提醒我們.
上主是會令我們有反感.
有不舒服的一個面向的.
所以我們要更加認真地去認識.
上主這個公義審判的一面.
第五章第一次在以西傑書裡面.
反映出人從罪裡面受苦.
和從耶路撒冷的毀滅.
被擄等等的災難當中受苦.
以西傑首先向他的受眾.
呈現出耶路撒冷將要面對的命運.
第二就用一種倖存者的罪疚感.
來使得這班被擄的受眾.
去控訴他們自己.
並在這個災難所帶來的創傷裡面.

$^{561}$去明白自己的罪疚和責任.
以西傑是要準備他們有可能.
成為會被更新的人.
去迎接將來的復興.
弟兄姊妹.
你們會這樣想.
就算以西傑真的出於好意.
以災難作為被擄群體的提醒.
再以倖存者的罪疚感.
來使得被擄的受眾去更新生命.
但始終第五章所描述的災難和審判.
的確是讓受眾再沒有動力捱下去.
就好像有些弟兄姊妹說.
過去大半年的反修例運動.
再加上一月底至到現在的肺炎疫症的煎熬.
在這種處境底下.
我們真的不知道怎樣捱下去.
先知以西傑除了看瘟疫是災難.
是刀劍饑荒和放逐等等審判的其中一種之外.
以西傑才知道他有沒有記錄下.
上主為以色列人所留下的一些鼓勵說話或出路呢.
難道我們真的要一直看.
以西傑書一直看一直看.
直到33到34章開始.
我們才看到那一點點的探望.
就正如難道我們要等很久很久很久的時間.
等到疫情完結的消失.
政府再次重視市民的訴求.
勞工重新投入市場.
百業有復甦的跡象之後.
我們才再次有那種安舒快樂的感覺嗎.
既然要面對這麼殘酷的災難就快要來臨.
以西傑才知道他有沒有預先作出準備.
或他有沒有為他的讀者受眾去戴頭盔呢.
這樣我們就要從以西傑書第一章的經文開始看起.
因為第五章的歷史背景.
就是按著一章一至三節所標示的日期發展出來的.
很多弟兄姊妹都會留意到.
以西傑書其中一個特色.
就是列出了14個詳細的日期標誌.

$^{601}$去說明在某年某月某日將會發生的事情.
或先知所要宣講的訊息.
一章一至三節.
就是全卷以西傑書的第一個和第二個日期標誌.
在30年4月初五.
以西傑在加巴魯河邊被擄的人當中.
天就開了.
得見神的異象.
正是若雅根王被擄去第五年4月初五那天.
大部份人都會認為.
一章一至三節這兩個日期標誌.
是關於以西傑被呼召.
以西傑在加巴魯河邊看見神的異象.
是在30年4月初五那天.
對於以西傑的即時受眾.
即是那些和先知以西傑一起被擄到巴比倫的群眾來說.
他們當然明白.
第30年4月初五那天是指到哪一件歷史事情.
但對於被擄群體的第二代或第三代.
甚或是後來以西傑書的讀者來說.
30年4月初五的意思似乎是有些模糊.
所以後來以西傑先知或他的門徒.
就在第二節加插了一個澄清的日期標誌.
就是若瓦根王被擄去第五年4月初五日這句說話.
也就是我們現在在聖經看見就清楚明白.
當以西傑在加巴魯河邊看見神的異象那一天.
就是30年4月初五日.
也就是若瓦根王被擄去第五年的4月初五那一天.
但我們仍然會問.
第一和第二節始終有些不協調.
第30年和若瓦根王被擄的第五年.
似乎兩件事好像指向著不同的一個歷史事件.
教父俄利根就非常有智慧.
又或者他本身也是按照教會一貫的傳統去解釋.
第一和第二節.
俄利根就指出.
第30年其實就是指到.
以西傑他看到第一個異象的年歲.
這樣你就明白了.
以西傑被擄第五年4月初五那一天.

$^{641}$在巴比倫他看見神的異象.
正正就是他30歲那一天.
之後第三節就說.
在那一天在迦勒底之地迦巴魯河邊.
耶和華的話就突地臨到布西的兒子.
祭司以西傑.
現在我們知道了.
原來以西傑是在祭司的家庭裡面長大的.
原來他是祭司.
如果你根據《文素記》第四章的經文.
利未祭司在30歲那一年.
就會開始在會幕裡面侍奉.
於是你又明白了.
以西傑在30歲那一年.
被上主呼召做先知.
按照傳統.
以西傑在30歲那一年.
就要在聖殿擔當祭司的職份.
作為祭司家庭的成員.
30歲象徵他過去成長過程.
所接受的裝備和訓練.
終於在這一天大排長龍.
誰知以西傑30歲之齡.
卻是身在巴比倫.
他沒有辦法再進入耶路撒冷的聖殿侍奉.
無論是以西傑.
還是他的門徒後來將日期標誌記載入書裡.
以西傑書一開始描述的這件事.
就是以西傑在30年4月初五那一日.
亦是約瓦根王被擄第五年那一日.
在加巴魯河邊的經歷.
其實就反映著.
以西傑已經失去了他侍奉的崗位.
按今天的標準.
以西傑失業了.
對於以西傑來說.
祭司不單止是一個職業.
亦是他與生俱來的一個職份.
我們可以這樣說.
以西傑在被擄的環境裡.

$^{681}$他連自己的身份都變得模糊.
甚至嚴重地說.
他失去了自己的身份.
弟兄姊妹.
此時此刻.
我們或許和以西傑當時所面對的情況.
真的有點相似.
我們同樣地對自己的身份.
都有點模糊.
過去香港社會所持守.
珍重的核心價值.
言論自由.
民主.
法治.
等等令到我們當中有部份人.
感覺到非常憂心.
以前在獅子山下.
我們彼此互相照顧.
我們共同打造更美好生活環境之下.
成長的部份人.
現在在面對社會不斷的紛爭.
亦都會令到他們顯得無所適從.
我們就好像被擄的以西傑.
對於我們身處的環境.
價值的轉變.
我們真的感覺到很陌生.
再加上最近疫情的持續.
為醫療體系.
經濟前景帶來不明朗的因素.
或者我們當中有不少的人.
亦都承受相當的壓力和困擾.
我們擔心會失去工作.
沒有工作.
甚至我們現在對自己的身份.
在社會上的責任.
亦都漸漸感覺模糊.
不過.
以西傑書並沒有停在.
以西傑他失去祭司身份和侍奉機會這裡.
第三節他講.

$^{721}$在這個危機裡面.
上主的說話就來到以西傑那裡.
並且上主將他的手.
按在以西傑的身上.
這就令我們想到.
上主的手其實亦都曾經按住.
按在先知以利亞和以利沙的身上.
當年上主怎樣加力給兩位先知去侍奉.
今日上主在以西傑灰心.
落寞的時候.
亦都同樣讓他看見神的意陣.
而得到鼓勵.
並且呼召他做先知.
弟兄姊妹.
今日我們或者因為社會的狀況.
或者疫情令到我們生活習慣.
工作改變.
我們會有情緒.
我們會沮喪.
但是我們作為基督徒的身份.
是上主所賦予的.
是不會改變的.
當上主他呼召以西傑.
這個失業沮喪的以西傑.
去做先知.
去傳講他說話的時候.
會不會亦都是上主提醒我們.
在這個動盪不安的環境裡.
我們要忠心地在各自的崗位裡.
我們要用勇氣去說出真相.
我們要宣講所要宣講的真理.
對於今日我們基督徒去看以西傑書.
尤其是今日我們所讀到的第五章的時候.
內容尚且會顛覆我們對上主的認識.
可想而知.
對於秘魯群體的第二,第三代子孫來說.
縱使其實他們沒有親歷秘魯的過程.
或者他們只不過是憑經文聽見,看見.
耶路撒冷淪陷聖殿被毀的事實.
不過這種國破家亡的震撼.

$^{761}$和對他們自己信仰的顛覆.
依然會為這些身處在巴比倫的以色列人.
帶來無法估計的創傷.
難道我們要看見上主呼召失業的以西傑.
作為先知幫助以西傑去肯定自己的身份.
就足以安慰或者去撫平大多數以色列人的創傷嗎?.
不過,不知道大家還有沒有留意到第二節.
正是《約瓦根王被擄的第五年四月初五》這句經文.
作為傳捲書的開頭.
約瓦根被擄的那一年.
成為了以西傑書所有日期標誌的參考年份.
約瓦根被擄的那一年是公元前597年.
是以色列被擄巴比倫的那個年份.
而以西傑亦都是在那一年.
和約瓦根和耶路撒冷的宗教,政治,社會領袖.
一起被放逐到巴比倫.
按照《列王記》下24章的記述.
約瓦根在耶路撒冷作王很短的時間.
只有三個月.
當約瓦根被擄巴比倫.
尼泊爾加尼撒立即在西底加在耶路撒冷做皇帝.
他做了11年.
按道理,約瓦根被擄之後.
西底加才是耶路撒冷的王.
其實年號是應該要改變的.
就好像2019年日本天皇名人退位.
年號都是由平成轉做令和.
況且約瓦根只是短短做了三個月的皇帝.
他對以色列的歷史其實都不是有很大的貢獻.
為甚麼以西傑不採用西底加作王那一年.
作為全本書日期標誌的參考年份呢?.
可能你會想.
會不會是因為西底加行耶和華眼中看為惡的事?.
以色列書就不想跟這位應該被奏助的領袖連上關係.
但《列王記》下記載約瓦根.
都是行耶和華眼中看為惡的事.
以色列書之所以選用約瓦根被擄那一年.
作為全本書的參考年份.
是因為以色列書認為.
約瓦根被擄標誌著大衛家的正統王位繼承被打斷了.

$^{801}$雖然西底加是約瓦根的叔父.
以色列和被擄的人卻認為.
西底加並非王位繼承的正統.
更重要的是以色列認為.
西底加是巴比倫王所立.
而且以色列和約瓦根一起被擄這件事.
的的確確對他們和被擄的群體實在有太深遠的影響.
為何以色列要加插約瓦根被擄這件事.
作為日期標誌的參考年份呢?.
或許還有另一個原因值得我們細心思考.
對於以色列和它即時的受眾來說.
約瓦根被擄其實是一種羞辱.
但對於被擄的第二和第三代.
甚或是那些正在看希伯來聖經的以色列人來說.
他們對約瓦根被擄這件事會有甚麼看法呢?.
我們作為基督徒或許我們不察覺.
原來希伯來聖經是將以西傑書編排在耶利米書之後.
並不是我們常常看到的聖經次序.
耶利米書之後就是耶利米哀歌.
然後才到以西傑書.
當我們將希伯來聖經的以西傑書揭前一頁.
耶利米書52章最後幾節記載了甚麼事情呢?.
耶利米書52章31節.
猶大王約瓦根被擄後37年.
相信這個時候以西傑應該有大約60多歲.
巴比倫王爾美米羅達元年12月25日.
使猶大王約瓦根抬頭提他出監.
猶對他說恩賢使他的位高過與他一同在巴比倫縱王的位.
給他脫了囚服.
他終身在巴比倫王面前吃飯.
巴比倫王賜他所需用的食物.
日日賜他一份.
終身是這一日直到他死的日子.
善至耶利米記載了約瓦根被釋放.
雖然他沒有被批准回到耶路撒冷.
但巴比倫王換走了他的囚衣.
賜他終身的飲食.
並且使約瓦根的位分高過其他王.
對於秘魯的第二第三代.
約瓦根蒙恩這件事.

$^{841}$就好像復興會很快開始的信號.
對於閱讀希伯來聖經的以色列人來說.
以色列宣一章二節約瓦根被擄的第五年四月初五這句說話的背景.
其實就是一頁之前耶利米書的約瓦根蒙恩這件事.
以色列原來為他的讀者已經打了底戴了頭盔.
當我們讀到以色列書第五章的時候.
耶路撒冷被毀.
瘟疫這些災難依然對秘魯的第二第三代是他們心裡的一條刺.
依然牽動著他們包括希伯來聖經的讀者他們的情緒.
雖然第四十到四十八章所描述的復興.
距離第五章這裡的審判和指控仍然有一個很長的距離.
但當讀者帶著約瓦根蒙恩這件事.
去看回以約瓦根被擄為背景的第五章的時候.
以色列書的受眾是懷著一個盼望底下.
他們不再用一個倖存者的罪疚感來控訴自己.
反而成為他們的提醒和動力去等待這個真正復興的來臨.
甚至當你這樣看的時候.
你會留意原來在第五章的審判裡面.
上主應許會保護一些倖存者.
以及他對所納的約那份執著和熱心.
都會成為以色列人在患難裡面的安慰.
弟兄姊妹.
過去這個星期肺炎的疫情好像有些回落的跡象.
但肺炎對世界的衝擊.
令我們的生活充滿了不安,焦慮甚至沮喪.
我們今天看以色列書.
第五章的確是很難受的.
就正如耶路撒冷的居民他們不能夠逃避災難.
我們也不能夠預計肺炎疫情和它所帶來的衝擊會怎樣停止.
但以色列教我們.
你看闊一點.
不需要闊到第四十章.
其實你看闊一點.
揭前一頁耶利米斯最後幾節的經文裡面很短的復興預告.
就是我們走過瘟疫的第五章的動力.
今天我們問疫情究竟何時完結.
我們埋怨政府不封關.
我們好像沒有甚麼希望.
但看闊一點.
你看今天我們周圍還是有一群很忠心的醫護人員.

$^{881}$我們有一些人會自發地將口罩派給前線的清潔工人.
有些民間的企業他們用高價買來製造口罩防護裝備的機器.
然後他們用良心的價錢去賣給市民.
甚至在這個疫情的底下奉獻資助緊絀的時候.
有社福機構他們沒有減少過他們的服務.
社工聽到赤貧的家庭的需要.
就幫他們去買米.
幫他們去買清潔的用品.
這些就是香港人的盼望.
你問我可以怎樣捱過瘟疫的煎熬.
我答你.
用以西傑去看瘟疫的這個角度.
我們看闊一點.
我們就可以看到原來人本為善的質素.
仍然在我們身處的這個社會裡面.
這種以愛心彼此相顧的情操.
仍然在我們周圍裡面出現.
或者你轉頭看闊一點.
你轉頭一看你會發現比我更多.
這些就是我們在疫情裡面的鼓勵.
亦都是帶我們能夠走出困境的盼望.
時間過得很快.
又是下課的時候了.
結束之前.
或者我們都用少許的時間.
我們去沉澱今天課堂所領受的說話.
記住.
下星期同樣主日的時間.
我們會再次在「亦有ye 學」.
「延伸再線」裡面再次見面.
原主的說話.
祝福大家.
多謝.
字幕:J Chong.
(字幕由 Amara.org 社群提供).
\newpage



\section{}
\label{sec:xiMH3MdBCkY}
\textbf{【疫有嘢學 │ 延SUN在線】從歷史看政教分離的多種含義|雷競業博士}
\newline
\newline
連結: \href{https://youtube.com/watch?v=xiMH3MdBCkY}{\texttt{ https://youtube.com/watch?v=xiMH3MdBCkY}} ~~~~ 語音日期: 2020-06-07 
\newline
\newline
\hyperref[sec:Hs1Y_XrlxkM]{\small{< < < PREV SERMON < < <}}
~
\hyperref[sec:index]{\small{[返主目錄]}}
~
\hyperref[sec:2dfqNGcDljE]{\small{> > > NEXT SERMON > > >}}
\newline
\newline
$^{1}$主席.
今日小蕾老師會和我們分享一個近期比較熱門的課題.
就是從歷史看政教分離.
願主開闊我們的眼界.
讓我們能夠透過歷史去反省.
今日我們作為基督徒在社會上應有的公民責任.
歡迎大家來到《亦有ye 學》.
今日我選了一個最近都很爭議性的問題.
就是政教分離.
不過我首先要警告大家一句.
你聽完這個講座.
我不會告訴你一個所謂聖經模式的政教分離.
不會給你一個模式答案.
反而希望透過這個歷史的進程去思考這個問題的複雜性.
所以我定了這個題目.
叫做從歷史看政教分離.
當我們有時提到這些題目的時候.
很多時候我們很快就想跳到一個結論.
請你告訴我正確答案.
但我想告訴你.
如果你只是跟從所謂聖經模式的政教關係.
我自然就要問.
到底有沒有一個很簡單的所謂聖經模式的政教關係呢?.
我們要明白.
其實聖經講的東西.
很多都是很處境化或者本色化.
即是聖經所表達.
譬如舉例在舊約裡面.
那些先知和王的關係.
是很反映舊約近東的文化.
當然有些東西我們是可以去學習的.
但我們又不會直接搬過來.
譬如舉個例子來講.
舊約的先知可以走到王面前大罵王.
今天我們不會說你做先知的角色.
所以林鄭開記者招待會.
你衝到林鄭面前大罵她一頓.
就算你覺得你是先知的角色.
你都不會學足舊約那套.
新約同樣也是.

$^{41}$有些人說新約不講政教.
其實不是的.
新約也有新約的講法.
因為當然在舊約那個年代.
舊約我們說是一個所謂的同質的社會.
以色列整個社會的人.
應該都是同一個信仰.
所以你可以說當先知去批判政權的時候.
倫理的標準可以假設大部分人都認同.
能否做到另一回事.
但大部分人都起碼理念上知道應該是這樣的.
所以先知走去責罵皇帝說.
你不應該拜偶像.
不應該拜巴比倫的神.
舉個例子.
他不需要去所謂justify.
他不需要去解釋一輪.
為什麼你不可以拜巴比倫的偶像.
他已經假設了大家都認同.
這個起碼理論上你應該做的.
實際上做不做是另一回事.
先知就會譴責皇帝.
去到新約的時候.
當然那個信徒就變成了一個所謂的邊緣群體.
新約的信徒是在羅馬帝國裡面.
最初當然在新約時代.
很多時候對一些的愛邦人來說.
他們就會覺得基督教其實是猶太人信仰的一部分.
猶太人就在羅馬帝國裡面是一個少數民族.
所以新約的教會某程度是少數派中的少數派.
用今天人的說法是一個邊緣群體.
所以很自然新約.
保羅不會直接做舊約的事情.
如果保羅舉個例子.
他要模仿舊約的先知.
他就應該跑到凱撒面前.
然後大罵凱撒.
你違反了羅馬人的傳統.
羅馬的神明說不應該這樣做.
當然保羅不會做這些事情.

$^{81}$一來保羅不是信羅馬的宗教.
保羅信的宗教不是羅馬皇帝.
或者羅馬整個社會信的宗教.
二來保羅也沒有這樣的機會.
他就算想學舊約先知.
跑到他的皇面前罵他的皇一堆.
他做不到的.
所以你明白新約有新約的背景.
所以保羅所做的事情.
是從他的背景出發.
舉個例子.
保羅沒有特別去批判奴隸制度.
有些人就說為什麼保羅不寫信.
保羅的信裡不大罵奴隸制度呢?.
一來當然是羅馬時期的奴隸制度.
和後來美國的黑奴制度.
我們稍後會提及.
是兩樣的東西.
所以奴隸制度本身是很複雜的.
第二樣更重要的是.
譬如保羅在他的信裡寫.
奴隸制度很邪惡.
在他的處境是沒有意思的.
他寫了什麼呢?.
羅馬皇帝不會聽的.
就算他寫了一句.
奴隸制度很邪惡.
對社會是完全沒有影響的.
你看到保羅所做的事情.
你說他有沒有講政教呢?.
其實他有提到政治.
譬如他在《肥理門書》.
叫肥理門不要視奴隸再是奴隸.
你視他作為你的弟兄.
你看到某程度保羅已經對奴隸制度作了一些批判.
不過他的批判不是弄一個旗號.
「廢除奴隸制度」.
他做這些事情是沒有用的.
他做的那些是那時候可以有用的事情.
他對一個奴隸的主人說.

$^{121}$在主內你不要再視奴隸制度.
是主人的財物.
你要視他作為你的弟兄.
你會發覺他對社會不公義的批判.
是有在裡面的.
不過是用他在他的處境.
可以說是有意義的做法.
去講出這些批判.
所以我們要譬如說政教關係.
我們不斷要問.
到底在不同的社會的處境裡面.
到底基督徒怎樣的關心政治.
或者怎樣的關心社會不公義的事情.
才是一個合適的做法.
隨著社會的模式的變動.
就要不斷地變.
譬如舉個例子來說.
我們說政治的制度才算.
你可以簡易言之.
新約時代是一些軍權的社會.
基本上羅馬的皇帝.
通常是打仗很厲害的將軍.
羅馬的皇帝又不一定是父位的子弟.
有時是有時不是.
最重要是兒子要打仗打得厲害.
才可以承繼父親的皇位.
起碼耶穌的年代是這樣.
所以在這個政治體系裡面.
譬如舉個例子.
你想像保羅在一封信裡面寫.
追求民主完全沒有意義.
那個年代的人沒有人明白什麼叫民主.
說來也多餘.
那個年代整個古代社會都是這樣的.
誰打仗厲害就做皇帝.
這是整個社會文化的假設.
所以我們不要期待保羅會說一些現代人的問題.
或者討論民主的問題.
保羅不會知道這些問題.
就算他說出來.

$^{161}$他的觀眾也不知道他在說什麼.
我們今天就不這樣了.
今天就改變了.
我們的政府不是說軍權的.
不是誰打仗厲害就做國家的元首.
我們今天說民權.
我們透過人民的意志.
人民想要A做總統或者國家主席.
A就做國家主席.
我們今天無論你是美國,中國.
什麼國家都好.
大部分的國家.
都會說國家的元首.
應該是表達人民的盼望.
因為人民希望這個人.
或者這個政黨.
或者這個權力是對國家好的.
所以他才能夠上位.
這種假設不是新約的假設.
新約的時候人們不是這樣去解釋政權.
所以你要明白.
什麼叫做信服.
我們稍後會說信服.
其實不同的社會背景.
你可以解釋不同的信服.
信服都可以有不同的意義.
譬如說.
比如保羅舉個例子.
沒有提到你可不可以在報紙上寫一些東西去批評政府.
因為那時候沒有報紙.
要批評都沒有報紙登出來.
比如保羅沒有提到叫人去投票.
當時當然沒有人去投票.
但你可以問.
比如說信服政權.
我投一票是反對的.
比如說川普競選.
我現在投票.
我不選川普.
我選川普反對拜登才算.

$^{201}$這些是不是叫做不信服政權呢.
其實保羅沒有回答這個問題.
因為保羅當時沒有投票.
換句話說.
因為今天對政府的期望不同.
所以今天我提出一些議題.
我覺得香港政府不好.
林鄭的政策A有問題.
這個是不是代表我們不信服政權呢.
我覺得我們要想清楚.
不可以說新約沒有做這件事.
比如保羅都沒有叫我們去投票.
因為當時都沒有投票這件事.
不可以說新約沒有做這件事.
所以我們今天就不應該做了.
我們要想清楚.
在不同的社會的場景.
特別是在政教分離.
我們對一個政府.
為什麼憑什麼理據.
能夠成為一個國家的政權呢.
當社會對這個問題有不同答案的時候.
很自然基督徒在社會的參與.
什麼叫做信服政權.
什麼叫做盡了公民責任.
自然有不同的理解.
所以為什麼我們要從歷史去看.
而不是說單單一個所謂聖經的模式.
其實我想說一件事.
我相信歷代的聖賢.
包括中世紀.
我們很多時候.
所謂基督新教.
就不太熟悉中世紀.
或者對中世紀評價都很低.
但對我來說.
讀歷史的人來說.
其實我覺得歷代的先賢.
都是很嘗試去將他們了解的.
所謂的聖經的政教關係去實踐出來.

$^{241}$包括中世紀的教皇.
他們大體上都是嘗試去實行.
他們怎樣去理解聖經講教會和政權的關係.
所以我們看歷史.
其實是一個謙卑的功課.
我覺得如果你說.
我不用理會別人做過什麼.
總之我直接聖經說什麼.
我跟著做就行了.
對我來說.
我覺得這是一種很驕傲的說法.
你這麼厲害.
或者我們這一代人這麼厲害.
歷代的人都搞錯了.
就是我們這一代人可以找出來.
聖經就是原來我這樣教導政教關係.
我覺得很難不接受這件事.
其實我覺得歷代的信徒.
正在努力去實踐.
什麼叫做聖經模式的政教關係.
當然過去的人.
會有他的偏見.
會有犯錯的地方.
正如我們今天的了解.
都會有我們的錯.
都會有我們的偏見.
所以我希望透過歷史的.
所謂的一個範圍.
一些闊一點的角度.
讓我們看到這個問題的複雜性.
還有聖經的不同的.
教導裡面有不同的著重的地方.
或者不同的詮釋的方法.
按著這些不同的詮釋方法.
其實歷代就有不同的模式.
就是政教關係的模式.
讓我們去理解.
其實政教分離這件事.
我想強調一件事.
就是其實問題是很複雜的.

$^{281}$有時候我覺得人家.
有時候坊間說政教分離.
好像一說政教分離.
你應該知道是錯誤的.
其實說政教分離.
我覺得有時候真的不知道是什麼.
譬如舉個例子來說.
在有些信徒的論述之間.
政教分離的意思是什麼?.
教會不談政治.
是不是什麼都不談呢?.
那又不是.
通常說政教分離.
教會不應該談政治的人.
都會引用羅馬書13章.
就是要信服.
於是我們逆來順受.
政府就算是不公義.
我們都要繼續信服.
當你去教導信徒.
我們要根據羅馬書13章.
信服政權.
信服政權的意思就是.
教會不談政治問題.
其實這種說法.
本身是不是一種真正.
所謂真正.
政教分離的說法呢?.
或者從另一個角度來看.
對我來說.
你可以這樣想.
如果真正的分離的意思.
其實可能指沒有話好說.
分離的意思就是完全沒有關係.
舉個例子來說.
在基督教的信仰裡面.
基本上就有所謂的食教分離.
就是食物和我們的信仰.
是真的分離的.
所以你喜歡吃什麼就吃什麼.

$^{321}$除了當然有些基督徒認為.
基督徒不可以吃血.
不要理會.
假設你沒有這種禁忌.
所以我們在信仰和吃什麼之間.
有食教分離這件事.
因為完全沒有關係.
基督徒吃什麼都可以.
從這個角度來看.
如果你要說基督徒應該信服政權的話.
已經不是政教分離.
已經混在一起了.
因為你告訴我.
做基督徒對政府應該有一種正確的態度.
這個正確態度叫信服.
總之你每次說.
聖經告訴我們對政府.
是應該有某一種態度.
信服也好不信服也好.
爭取公義什麼態度都好.
有一個正確的態度的時候.
其實已經不分離了.
你覺得信仰其實已經有一個標準.
既然你這樣說.
我們就要繼續去探索.
到底你的標準.
是不是一節經文就可以說完整個聖經的標準呢?.
你的標準從哪裡來呢?.
所以.
我希望你聽完這個講座.
你要明白.
其實歷代的教會.
從來都不會完全的政教合一.
也不會完全的政教分離.
基本上歷代的教會從來沒有說過.
會好像我剛才說的食教分離.
基督徒對政府是完全任何看法都可以的.
無政府也可以.
你很信服政府也可以.
基本上沒有一個教會.

$^{361}$沒有任何的時期會分到這個地步.
是完全無意見的.
但另一方面教會也從來沒有試過.
政教合一到一個地步.
教會就是政府.
教會的主教.
舉個例子.
應該做屬世政府的頭.
也從來包括中世紀的教皇.
都從來沒有這樣說過.
合一到兩者是identical(相同)的.
教會就是政權.
所以其實歷代的不同時段.
教會其實什麼時候都有說.
所謂政教分離.
不過合和離的點在哪裡呢.
不同的時代就有不同的著眼點.
哪些地方是需要合的.
意思就是說基督徒是要強調.
政府是要這樣那樣的.
哪些地方要離的.
基督徒是不應該干涉政府某些的東西.
或者政府不應該干涉教會某些的東西.
其實希望透過今天.
跟大家看看不同的時代.
對於合和離的關鍵到底在哪裡.
其實今天一方面.
我希望你聽完這個訪談.
你覺得有些混亂是非常好的.
我也想你混亂一點.
不要覺得政教分離是一種很簡單的東西.
政教分離就是這樣.
不過混亂之餘.
我想當然也有一點底線.
我希望你混亂之餘.
你明白一個底線.
第一就是我會強調.
信仰和政治生活是不可分離的.
我的確相信聖經給一些原則.
關於我們的政治生活.

$^{401}$這個當然你要明白.
我特意要選政治生活.
因為所謂政府的institution.
政府的機制.
到底政府要做些什麼.
政府是透過什麼方法去成立.
這些問題其實隨著時代去轉變.
變動是很大的.
譬如舉個例子.
又說回剛才.
新約時代的政府.
基本上是負責去打仗.
新約時代沒有人會期望.
政府 羅馬的皇帝.
譬如舉個例子.
羅馬皇帝會給你失業救濟金.
新約時代沒有人會有這種期待.
但是今天我們是很自然的.
譬如COVID-19.
肺炎出來了.
我們很多人失業.
我們期望政府會幫助失業的人.
無論是香港 無論是大陸.
無論是美國.
哪裡都會有這樣的期望.
我們對於政府要做些什麼.
其實隨著歷史不斷演變.
但是政治生活是什麼時候都有的.
政治生活基本上就是我們一個群體.
我們住在一起.
你叫一個國家也好.
一個民族也好.
總之作為一個群體.
有些事情我們有些集體決定.
我們要做的.
譬如舉個例子.
譬如有沒有民主那才算.
香港所謂有真普選.
如果香港有真普選.
全部人都有.

$^{441}$如果香港沒有.
全部人都沒有.
應該就是這樣.
一個普選的制度.
市民都享受.
如果沒有的話.
全部人都沒有.
政治生活就是說.
整個群體要集體做的決定.
集體生活的一些原則.
什麼時代 什麼社會.
都需要有政治生活.
於是我就說.
其實信仰或者聖經的教導.
關於我們的集體生活.
有沒有一些標準.
所謂集體生活裡面的公義.
什麼叫做公義.
什麼叫做平等.
有些的教導.
所以什麼時候都不可以完全分離.
不像吃東西那樣.
但是教會在社會裡面.
怎樣去見證公義.
怎樣見證群體生活的標準是什麼.
隨著社會的機制.
或者隨著政府的角色不同.
教會就會扮演不同的角色.
所以其實我們讀歷史.
一方面希望了解.
原來教會過去扮演過這麼多不同的角色.
同時當我們明白教會扮演過不同的角色的時候.
就幫助我們今天視野更加闊.
原來教會不是只有一個角色.
教會可以講信服.
但又不是單單講信服.
希望當我們明白.
有更多角色的可能性的時候.
當我們回看今天 回看明天的時候.
我們可以更豐富的想像.

$^{481}$教會可以扮演什麼不同的角色.
甚至在今天我們香港社會.
不同的教會可以扮演不同的角色.
我希望不要單一個模式.
模式要套在所有的教會裡面.
說了這些其實都可以說是隱言.
接著就開始真的進入歷史.
其實這一點我們剛才已經提過.
有些人就說.
如果你看網上的評論你都知道.
有些人就說.
因為我們要舊約的先知.
我們基督徒要舊約的先知.
所以我們要指出政權的不公義.
於是有些人就反駁說.
不是舊約的模式不適合我們用的.
因為舊約的王.
以色列王都一定要信聖經.
或者信猶太教.
但是現今社會不是.
雖然林鄭月娥剛巧.
她是天主教徒.
但是現在的君王.
或者政府的當權的人.
不需要一定信基督教.
所以舊約不適合用.
先知的模式.
所以我們要跟新約學習.
於是你看新約.
保羅有沒有叫你去抗議.
保羅有沒有去羅馬的官長面前.
去指責羅馬的不義.
羅馬也是軍權統治.
你以為羅馬政府很公義嗎.
保羅有沒有什麼時候.
你不知道保羅.
整本新約聖經.
你有沒有看到一些信徒.
使徒去譴責羅馬的不公義.
當然有羅馬書十三章.

$^{521}$時間有限.
大家都應該讀過那段經文.
不會跟你讀那段經文.
羅馬書十三章.
保羅說地上的政權都是神所設立的.
所以我們要信服它.
這就是新約的教導.
於是有些人說.
根據新約的教導.
我們不應該批評地上的政權.
教會就不要講政治.
總之羅馬書十三章.
一句話就順和.
這個說法.
我覺得要小心看新約的見證.
你記住對保羅最貼身的.
可以這麼說.
那個政權其實不是羅馬帝國的政權.
對保羅最貼身的政權.
是猶太政權.
那些猶太人.
法利賽人.
達到個人主義.
是猶太人的政權.
保羅有沒有抗爭的行為呢?.
對我來說.
我覺得很明顯有.
如果你看《使徒行傳》.
有一個階段.
猶太人的領袖.
很想把保羅從塞斯利亞.
即是蓋薩尼亞.
帶回耶路撒冷.
途中想找人殺了保羅.
保羅搞搞震.
於是保羅知道猶太人的計劃的時候.
他不是信服.
當猶太人提出這個要求的時候.
保羅沒有說.
好啊 我就回耶路撒冷.

$^{561}$我要信服掌權的.
那些猶太領袖是猶太社團的掌權者.
我作為一個猶太人應該信服他.
死就死吧.
沒有啊.
保羅就跟這些羅馬官長說.
不行 你不應該送我回耶路撒冷.
因為那些猶太官長.
跟猶太人的首領有些詭計.
我要求羅馬人的保護.
最後他就說.
我要去見凱撒.
如果你套用今天的場景.
借用一些由古代翻譯到現代.
其實保羅作為一個猶太人.
他去要求羅馬人的勢力去保護他.
其實就是要求外國勢力.
用今天的說法.
保羅是要求外國勢力去保護他.
因為他自己族裔的人的領袖.
要威脅他做一些不公義的事情.
所以我覺得.
就講回《士徒行傳》這個例子.
你說有沒有抗爭.
保羅當然是有抗爭的行為.
於是我們又要明白.
跳到羅馬書十三章講兩句.
當然詳細的解釋.
就留給新約的專家.
例如我們的霞霞老師.
保羅的專家.
詳細解釋.
你來忠臣交了學費.
叫霞霞教你.
簡單講一句.
我對羅馬書十三章的了解.
其實保羅那裡不是要講.
政權公不公義的問題.
保羅十三章不是說.
政權多不公義也好.

$^{601}$他叫你撞頭就撞頭.
因為保羅自己也不是這樣做.
剛才講了《士徒行傳》的例子.
保羅那裡關心的一個問題.
就是羅馬的政權.
很明顯羅馬的政權.
就是拜羅馬的宗教.
所以答案問題就是.
我作為一個基督徒.
我知道羅馬的政權.
拜的東西是假的.
我是否要信服他呢.
保羅講的那句話其實很大膽.
保羅說羅馬的政權.
都是耶穌設立的.
你明白從當時的處境去看.
保羅講這句話其實很招架.
你可以想像.
你跟一個共產黨黨員說.
中國的共產政權.
都是神所設立的.
對於一個不信神的共產黨員.
你可能會覺得很招架.
你這個耶穌是甚麼.
我們共產黨毛澤東.
這麼偉大的領袖.
你說毛澤東也是耶穌的僕人.
你真的很招架.
你明白嗎.
保羅講的說話.
他要強調一件事.
就是當時最強的帝國.
羅馬帝國.
其實只不過是主耶穌讓他做到.
有甚麼不招架.
有甚麼不偉大.
就是耶穌讓羅馬帝國.
能夠管治當時的世界.
所以保羅想信徒明白.
其實保羅已經做了某一種的政教分離.

$^{641}$因為當時的文化.
每一個國家都是他的神明.
所謂的保佑他這個國家強.
譬如A國打贏了B國.
那時候的了解可能就是.
所以A國的神明比較強.
A國的神明打敗了B國的神明.
所以A國的軍隊.
就可以砌低B國的軍隊.
保羅強調不是.
政教分離.
原來所有的政權.
都是耶穌設立的.
我不管你A,B甚麼國都好.
所以羅馬帝國.
其實都是耶穌設立的政權.
所以保羅就說.
OK,你都要信服他.
雖然他不信猶太教.
不信耶穌.
所以我覺得最直接的應用.
其實就是說.
譬如有一天.
你不知甚麼原因.
走到伊朗居住.
伊朗的政權.
說明是伊斯蘭教的.
不是基督教.
你去到伊朗.
那你要不要信服那個政權呢.
要.
因為就算伊朗這個.
信伊斯蘭教的政權.
其實都是耶穌設立的.
雖然他不知道.
但事實上.
是耶穌讓他能夠做到.
現在伊朗的政權.
做到伊朗.
管治到伊朗.

$^{681}$OK,所以重點不是.
政府做甚麼你都信服他.
而是你明白.
所有政權都是出自神的.
所以你不可以忽視那些政權.
你信服他的目的是甚麼呢.
保羅說得很清楚.
為了懲罰惡人.
上善罰惡.
其實保羅沒有回答的問題.
就是當那個政權.
不是上善罰惡.
上惡罰善.
好人他就去懲罰.
壞人他就給他們很多好處.
如果政權是反過來.
那怎麼辦呢.
保羅其實十三章就沒有回答這個問題.
因為這個不是他當時.
那些人不是在問他這個問題.
所以他也沒有想這個問題.
所以也沒有回答這個問題.
你可以說啟示六十三章.
講到地上的獸.
海中出來的獸.
那裡就涉及到.
你知道啟示六十三章.
有些《釋經學家》估計.
可能是90年代.
多米田的時代.
羅馬政府迫不及待的教會.
寫啟示六十三章.
那裡就強調.
政權是可以很邪惡的.
當政權很邪惡的時候.
你就要拒絕政權的命令.
你要見證神的真理.
寧死不屈.
這個就是啟示六的另一幅圖畫.
所以新約進入後.

$^{721}$就是無底深坑.
新約的教導.
不過我想簡單地告訴你.
就算是早期的教會.
早到新約的教會.
其實都有不同的見證.
當離開了新約教會.
去到君士坦丁.
我會提出一些人名.
你知就知道.
不知就有機會讀讀.
《宗臣》教會歷史研修課.
你就會知道更多的人名.
總之去到第四世紀初.
有一個皇帝叫君士坦丁.
他信了耶穌.
於是教會的角色.
由一個抗爭者.
變成一個夥伴.
夥伴的意思.
以後教會的了解.
就算是皇帝也好.
其實都是一個信徒.
所以教會要有責任去牧養皇帝.
而教會也有責任去牧養所有的人.
整個社區的人.
早期教會.
君士坦丁信了耶穌之後.
就開始有所謂大學士.
其實你知道我們拔萃學校.
Diocesan School.
其實就是來自早期教會的觀念.
一個教區.
Diocese.
教區的意思就是說.
於是一個主教.
要牧養整個區內的人.
當然某程度你可以說.
這些是一些混亂.
教會等於社區.

$^{761}$對不對.
我們今天可以有很多批判.
不要理會.
當時後的人覺得.
應該要牧養整個社區.
所以你可以說.
開始有些合理.
就是說.
他們會看主教的角色.
就是一個沒有代議制的人民代表.
什麼意思呢.
你知道在羅馬社會裡.
沒有投票的.
我講過了.
打仗就做皇帝.
所以所謂人民的聲音.
去哪裡去說出來呢.
你知道當時後.
不單是皇帝不是投票選.
所謂管治地方的人.
都不是代議制選出來的.
管治地方的人很多時候是貴族.
那時候是封建制度.
所以有個貴族.
貴族A就管省A.
所以貴族A很自然.
他最關心的東西就是.
他自己的權力.
未必是他省裡.
或者區裡的人的福祉.
於是當時早期教會.
即是第四世紀之後的教會.
主教就開始有這個觀念.
於是主教應該有一個角色.
就是代表人民的疾苦發聲.
主教不發聲.
誰發聲呢.
沒有人發聲.
貴族不是要代表人民的.
還有主教.

$^{801}$當然理論和實際.
當然有落差.
不過在觀念上.
主教是神職人員.
不會結婚.
所以他不用關心.
千秋萬代兒女的生計.
他死了回到天家就很榮耀.
所以理論上.
他關心人民的福祉.
符合他自己的利益.
因為他不用留遺產給子孫.
就算他是一個很自私的主教.
他怎樣表達他的自私呢.
就爭取人民的福祉.
讓人民都稱讚他.
主教你真偉大.
人又得到讚美.
上到天又得到神的讚美.
你為人民謀福祉.
很好.
所以當時就有所謂的政教.
你說合也好.
混淆也好.
就會有一個期望.
主教會代表一個地區的人.
向當地的貴族.
或者甚至是羅馬的皇帝.
去表達當地人民的疾苦.
譬如最近這三年都旱災.
可不可以減稅.
主教就會做這些事.
除了牧養一個區的人之外.
他也要牧養一個皇帝.
所以其中一個很出名的例子.
有一個人叫做Ambrose.
這個詳細不說.
第五世紀的Ambrose.
有一個皇帝叫Theodosius.
Theodosius有一次.

$^{841}$故事沒有時間詳細說.
總之Theodosius有一次.
因為某個城市有些人造反.
他報復的緣故.
將一批人.
據說大概七千人.
困在一個所謂的經濟場.
一次過殺了七千人.
皇帝做了這件事.
Ambrose就說.
你這樣做很殘忍.
違反聖經的原則.
所以就不讓皇帝Theodosius.
進入他的教會.
當Theodosius去到Ambrose.
米蘭的地方.
就不讓他去教會.
皇帝當然很沒面子.
他相信了耶穌.
所以最後Theodosius.
皇帝就要公開認錯.
然後Ambrose就接受他的認罪.
於是就讓他進入教會.
讓他領聖餐.
這個故事曾經是.
教會一個很傳頌.
廣為傳頌的一個故事.
講到一種政教的關係.
你明白在這個政教的關係.
不是合一的.
Ambrose不是要坐了皇帝的位.
Ambrose不是要做政府的元帥.
或者是Head.
但是又不分離的.
因為他覺得皇帝作為一個信徒.
換句話來說.
在政治上.
他就是最厲害的一個.
不過在教會或神的面前.
或者在教會機制裡.

$^{881}$皇帝也不過是一個會友.
他跟任何一個.
A,B,路人甲沒有分別.
他不過是一個會友.
所以如果路人甲是會友.
他做了一些不合聖經的事情.
要受到紀律.
皇帝一樣.
如果皇帝犯了教會的紀律.
都要受到教會的紀律.
這個你就明白.
中世紀.
或者由早期教會開始承繼到中世紀.
其實他們不是要強調.
教會要取代屬世政府的權力.
不過他們要強調.
屬世政府的元首.
都依然是一個信徒.
所以他做的事情.
應該要subject to.
應該要受到教會的倫理教導去規管.
這個就是早期教會.
一直給中世紀的教會.
都有這種擁抱的看法.
中世紀的教會.
如果詳細的話.
你買我的書看看.
講中世紀教會.
不過簡而言之.
中世紀教會的政治理論.
基本上就是所謂的雙劍理論.
來自《路加福音夜議》.
第38節.
就快要去赫西瑪利了.
門徒說有兩把劍.
耶穌說足夠了.
那一節的經文是為什麼呢?.
你又是給了錢進來跟進中神的新猶老書學.
總之中世紀的人對這一節的了解.
就說於是乎神就在地上設立了兩把劍.

$^{921}$一把屬靈的劍.
教會拿著.
一把屬世的劍.
或者世俗的劍.
那一把就是殺人的劍.
金屬的劍.
那一把就是地上政權去拿的.
所以也都用今天人的說法.
皇帝關心的就是國安.
就是國家安全.
國家呢.
譬如你是A國的皇帝.
你要確保A國不會被其他國家侵略你.
國土被侵蝕.
所以你A國的皇帝.
你要懂得打仗.
你要鑄造劍給你的士兵去打.
這個是皇帝的角色.
教會呢.
不會走去打仗的.
理論上是有差距的.
不過起碼理念上呢.
教會不是訓練人做士兵的.
教會也都不會鑄造劍,刀這些東西的.
教會或者特別.
你知道當時中世紀的歐洲.
就是整個歐洲.
起碼西歐.
東歐那些先不要理會.
東正教另一部分.
總之起碼羅馬天主教那部分的歐洲.
整個歐洲都是一個教會.
所以譬如教宗呢.
理論上就不會關心到底A國打贏B國.
還是B國打贏A國的.
除非B國是很邪惡的.
B國的皇帝吃人肉.
舉個例子.
我們要討伐這麼邪惡的皇帝.
但是一般的情況之下.

$^{961}$A國和B國打仗.
教宗是不需要支持A國還是B國的.
他不是關心國安的問題.
一個國家.
哪個國家可不可以保護他的疆土的問題.
教宗關心的是普世公義.
一個屬靈的劍的意義.
就好像剛才Fyodosus那個例子所說.
不是Ambrose對Fyodosus這個人有什麼偏見.
理論上任何一個國家的皇帝.
如果他做了一些很殘忍的事情.
教宗應該罵他.
或者教會應該罵他.
那個國家的教會應該要紀律那個國家的皇帝.
如果他做了一些很壞的事情.
所以你看到.
有合有離的.
我想跟你說.
就算中世紀也是有合有離的.
合在於教會有權去批判皇帝做的事情.
正確不正確.
聖經的原則.
但是也有離的.
離在於教會不會拿著劍.
而教會和國王的視野.
應該是不一樣的.
國王就是保護他自己的國家.
教會就是整個基督王國.
所謂他的聖潔.
理論上這是不同的視野.
所以中世紀的教宗.
譬如實際上他很多時候做的角色.
就是他想做一個終審法庭.
實際上中世紀教宗的角色.
就是一個歐洲社會的終審法庭.
譬如我是A皇帝.
你是B皇帝.
現在有條村.
我說是我的.
你說是你的.

$^{1001}$我們兩個解決方法.
一個就是打仗.
看誰手牽強一點.
如果不是我們很溫柔的.
我們不想打仗.
於是我們想找一個仲裁人.
找誰是仲裁人呢.
找教宗吧.
假設他是全歐洲最聖潔的那個.
於是他按照公義去判斷那條村到底是屬於你還是我的.
教宗於是.
所以你明白嗎.
有時候我們在我們的論述.
有時候在一些坊間.
說到中世紀好像是政教合一.
其實不是完全合一的.
我強調有合的地方.
也有離的地方.
當然實際上是混亂很多的.
但永遠都是這樣.
現實和理念永遠有些差距.
當然我們今天回看中世紀的時候.
就會說其實當時的離可能是做得不夠了.
因為中世紀的教會.
譬如舉個例子.
我剛才說過.
那個Bishop代表那個人民的區.
那個教區的人民的福祉.
他要關心那個區的人民的福祉.
他一定要有實力才行.
譬如舉個例子.
錢是一件很重要的事.
有些人譬如旱災或者什麼都好.
有意外山崩地裂.
有些人受了傷.
教會要救濟這些所謂災民.
都要有錢的.
所以去到中世紀.
其實教會本來有很好的理由.
教會就越來越有錢.

$^{1041}$總之簡而言之.
中世紀後來就是教會成為全歐洲最大的地主.
結果因為這些財產.
譬如其中一個重要原因.
因為教會這麼有錢的緣故.
於是就和這個熟世的政權.
就纏鬥不清的關係.
你有錢的話.
政府自然就很想拿你的錢.
無論是古代中世紀的政府還是這樣.
今天的政府都是這樣.
北京政府還是這樣.
美國政府也是這樣.
你有錢的話.
他們就想拿一部份你的錢.
所以就搞得很複雜.
總之這些東西.
不說了細節.
總之這個點就是說.
教會中世紀的教會.
理念上它是嘗試在地上推行神的公義.
它為了在地上推行神的公義.
它就要累積權力和財力.
但當它累積權力和財力的時候.
就越發覺原來推行公義是一種很複雜的東西.
於是教會有時理論上應該是為受壓者去發聲.
自己就變成一個欺壓人的人.
因為它要保護自己的財產.
保護自己已經擁有的權力.
不過這個故事我們又沒有時間說了.
我們就跳過它了.
下一點我們也跳過它了.
開始要說一些基督新教的故事.
路德對於政教的關係其實很有趣.
先說「合」那件事.
即是政教不分離那件事.
路德中世紀的教會.
很想推行某一種形式的政教分離.
中世紀教會那種形式的政教分離.
就是說教會裡面發生的事情.

$^{1081}$政府完全不要管.
譬如教會裡面有個神父.
他強姦了一個女人.
政府的地上政權是不可以干預這個神父的.
不可以審這個神父的.
那個神父犯了什麼錯.
教會自己審他.
這個是中世紀形式的政教分離.
教會裡面的內政的事情.
完全是教會處理.
所以教會的錢又是留在教會.
跟政府完全沒有關係.
不交稅給屬世政府的.
這種是institutional.
即是所謂機構上的分離.
完全兩個獨立的機構.
這個是中世紀的政教分離的模式.
路德提倡打破這一種的分離.
其實我們看歷史有時頗過癮的.
譬如當時宗教改革.
其中一個口號就是說.
神職人員都要交稅.
神職人員自己說的.
我要交稅給政府.
有時我會想為何他們會喊這個口號.
無論如何.
當時人們喊這個口號.
就是要打破機構上的分隔.
神職人員都不過是一個市民.
做市民的責任.
分離不是在機構上.
身份上完全變成兩種政治.
完全變成兩個分離.
所以路德強調有合的地方.
第一神職人員都不過是一個普通的公民.
第二就是當教會領袖失職.
當然在路德的筆下.
特別是羅馬教宗.
當羅馬教宗失職的時候.
政府是有權去干涉的.

$^{1121}$所以他鼓勵德國的Nobel Prize.
德國的貴族.
來帶頭改革德國的教會.
因為羅馬教宗不肯做這件事.
這就是所謂的合.
所以信教會的機構分離.
這件事我想說甚麼.
路德是很有趣的.
路德他會強調教會.
教會可以從神學上來說.
就是基督的身體.
但從另一個角度來看.
就是一個組織.
教會和鄉親會.
和任何一個地上公司.
沒有分別的.
都是一堆人組成的一個機構.
路德就會強調.
如果從一個機構的觀點去看.
教會是和其他機構沒有很明顯的分別.
所以教會裡面一樣有罪人.
公司會貪婪.
教會裡面的人都會貪婪.
公司有不守法的地方.
教會也可以有不守法的地方.
作為一個機構.
教會不應該覺得自己和其他機構是完全不一樣的.
是分開的.
不過他就會強調.
有分的地方反而在於信仰上.
於是他就強調.
雖然教會未必是.
不一定比其他機構.
特別性的.
舉個例子來說.
我們作為信徒.
我們知道我們的聖潔不是源於我們的生活.
明顯和其他人不同.
你一做了耶穌.
於是你就從來都不犯罪.

$^{1161}$這是眼見的.
信徒可能和其他人都很多類似的.
都會貪婪.
都會說髒話.
不過分在於.
在神面前.
我們知道我們的聖潔是來自基督.
教會的聖潔也是來自基督.
教會的聖潔不是因為教會不會犯罪.
而是在耶穌女神已經饒恕了教會的罪.
或者在耶穌女神已經饒恕了你和我的罪.
所以我依然會犯罪.
因為我是一個軟弱的人.
但是我又同時是一個聖潔的人.
你明白嗎?.
所以分就不是在於.
我作為一個基督徒和旁邊的非基督徒.
完全是兩種不同的人.
分是在於我在神面前的身份.
我和他一樣都會說髒話.
可能上了黃色網頁.
舉個例子.
分別不是在於.
我所做的事情都比旁邊的聖潔好.
而是在於.
我雖然會在世上犯錯.
但是在神面前我是聖潔的.
所以教會要服於政權.
說過交稅的事情.
但是國家元首.
路德也說.
目者有權去指責國家元首.
但是分反而在於.
所謂天上和地上公義的分離.
意思是什麼呢?.
路德會說.
舉個例子.
一個社會不公義.
舉個例子.
用美國最近的例子.

$^{1201}$黑人被人歧視.
於是你作為一個信徒.
於是你爭取黑人的權益.
路德會說這些都是好事.
不過你要記住.
你去到神的面前.
你離開世界了.
去到神面前.
神說為什麼我要接你上天堂.
舉個例子.
天堂有一個人在門口問你.
你不會跟天堂門口的人說.
因為我爭取民權.
因為我地上做了很多好事.
路德會說.
你應該說.
為什麼我應該上天堂.
因為耶穌為我死.
其實我上不了.
我做足壞事.
抵死有餘.
不過耶穌為我死了.
讓我上去.
他會強調.
地上無論我怎樣去努力.
爭取地上的好事情.
公義.
社會公義.
個人的公義.
各樣的事情都好.
這件事不是讓我能夠去到神面前.
誇口的事情.
這就是所謂地上公義.
路德會強調.
地上公義都是好事.
你繼續去做.
不過跟你能否討月神.
是沒有什麼關係的.
你能夠討月神.
就唯一靠求主饒恕.

$^{1241}$所以他明明是一個合的地方.
他覺得政府可以干預教會的事情.
不過他覺得.
最終你跟神的關係.
不在乎地上發生的事情.
路德的角度.
是所謂兩個角度.
有些人說路德是鼓勵政治被動.
這是另一個問題.
我們沒有時間去討論.
我想討論一下跟路德.
你可以說是剛剛相反的.
某程度.
這是另一個模式的分離.
就是重振派的模式.
重振派的模式.
重振派就是安娜·伯特蒂斯.
記住.
少許一個重點.
浸會不等於重振派.
我們今天的浸會.
其實是源自英國的一些清教徒的改革運動.
不是完全等於重振派.
我現在說的是重振派.
重振派.
今天最明顯的後人.
最直接的後人.
就是門諾會.
如果大家有聽過的話.
重振派對《羅馬書》十三章.
有另一種的了解.
大家明白.
大家都看聖經.
大家都想信服聖經.
不過什麼叫信服聖經呢?.
可能有不同的詮釋.
重振派的傳統體.
《羅馬書》十三章.
它的了解是.
保羅說過.

$^{1281}$政權是為了懲罰那些罪人.
不公義的人.
所以重振派的傳統.
認為《羅馬書》十三章.
和教會是完全沒有關係的.
因為十三章說.
政府是為了不公義的世界設立.
我們這批信徒.
理論上已經是神的信徒.
我們不犯罪的.
就算犯罪也是很小事.
我偷吃了你的蘋果.
這些小事.
我不會殺人放火的.
所以政府已經和我.
或者和神的教會.
是沒有關係的.
因為神的教會.
是不屬於不公義的國度.
所以重振派的分離.
是一個institutional的分離.
有些像中世紀.
某程度上回復中世紀.
所以路德經常形容重振派的人.
其實是另一批羅馬天主教徒.
這個指控是不公道的.
但某程度上他們是有點像.
重振派要追求的.
就是從屬世政權分離.
不過他們的分離.
比天主教或中世紀更加radical.
更加完全的分離.
他們希望重振派做的.
就是教會自己成立一個社群.
最好搬到一個地方.
例如山卡拉.
搬到香港.
不知道還有沒有山卡拉的地方.
搬到香港印堂海一個荒島上.
舉個例子.

$^{1321}$於是我們自己建立一個社群.
我們和政府不需要再來往了.
因為政府是拿劍的.
他們很強調.
基督是一個和平的國度.
所以基督的國度是不用劍的.
但世俗的國度是有罪的.
所以他們用劍的.
我們建立一個peaceable kingdom.
做回現在的著作.
一個和平的國度.
但因為世俗是真的有罪惡的.
你活在世俗裡面.
你沒有辦法只是講和平.
正如你活在香港.
你說警察全部沒有槍.
警察暴力這是另一回事.
但你說我廢除警隊.
於是現在社會上沒有人可以拿槍.
沒有人可以用暴力.
是否可以呢?.
我相信是不可以的.
於是那些賊就去搶東西.
所以你要完全沒有暴力.
那怎麼辦?.
你離開這個邪惡的世界.
你自己建立自己的社群.
所以從另一個角度來看.
你要明白.
重振派是另一種形式的政教合一.
它的政教合一模式是.
我這個性徒群體裡面的政治.
或者這個性徒群體裡面的權力問題.
完全由這個性徒群體裡面去解決.
你可以想像.
例如像這樣.
重振派裡面.
我們是重振派的人.
現在我們搬到葡萄島.
建立了一個重振派社團.

$^{1361}$我和香港政府沒有關係.
我不交稅給香港政府.
沒有任何來往.
你想像我們葡萄島這個社群裡面.
有一個我們的性徒犯了罪.
他居然揭開.
另一個弟弟揭開姐妹的裙.
看他的裙帶.
舉起來犯了這樣的罪.
於是誰去懲罰這個人呢?.
這個就是政治的問題.
犯了社區的規矩.
誰去懲罰他.
這個其實是一個政治的問題.
在重振派的模式裡面.
重振派的社群自己去懲罰這個人.
不讓他領聖餐.
什麼都好.
什麼時候這個人會受到世俗政權的懲罰呢?.
除非這個人不悔改.
於是我這個重振派葡萄島社群.
將他逐出教會.
找一艘船將他從葡萄島再返回香港.
然後把他扔到香港.
把船返回葡萄島.
當我把他扔回他熟悉的社會的時候.
他才受熟悉的社會的那種法律所管轄.
所以你可以說.
重振派是要求政教分離.
他和世俗的政府分開.
但是他們又是另一種的合.
教會的領袖要做社區的政治領袖.
所以重振派給我們今天一個很重要的.
世俗政權的一個很重要的挑戰.
就是一個反暴力傳統.
不過這些東西又沒有時間去說了.
我唯一補充一句.
我覺得重振派值得給我們思考的一樣東西.
就是說.
信服行動對政權榮耀的批判.

$^{1401}$意思是什麼呢?.
其實當時候那些重振派的信徒.
他們去到哪裡都會被人抓的.
因為他們鼓吹教會和社會的分離.
在當時候的十六世紀,十七世紀.
這是不能夠容忍的東西.
所以他們無論在羅曼天主教的地方.
或者在基督新教的地方.
即是路德會,信義中,改革中的地方.
都會被人抓.
都會被人處決.
很多時候這些重振派的人信服.
他們表達信服的方法.
就是他被人抓的時候就被人抓.
然後人家要處決他.
他就所謂的「從容就義」.
他們的信服的方法就是從容就義.
我說這種他們就認為透過這一種的.
你明白嗎?.
你可以說信服也可以.
你可以說不信服也可以.
因為他講明他不接受世俗政府的標準.
他講明他不承認世俗政權.
對他其實有任何的所謂的「authority」.
對他有任何的權柄.
不過你要用暴力來抓我.
你要用暴力來殺我.
我就讓你殺.
因為我不會用暴力去對抗你的暴力.
我的國度不是透過暴力方法建立的.
即是重振派的信念.
於是乎他其實是透過一個不認同.
你可以說透過一個叫做「passive resistance」.
一種被動的抗爭.
一種抗爭的信服.
怎樣也好.
一種被動的抗爭.
好像表面上是信服.
實際上就是一種抗爭.
去表達對所謂政權榮耀.

$^{1441}$即是說為政的人覺得他很威風.
為政的人覺得甚麼叫做有權力.
就是你被迫信服我.
你不聽我說我殺了你.
那我就很有權.
重振派就說我讓你殺.
你以為這是權力的象徵.
其實這是表示你邪惡的象徵.
即是他透過這個方法去表達的一種批判.
這個其實值得我們今天繼續去思考.
不過沒有時間和你慢慢思考.
因為又要去下一頁了.
我們就繼續加快速度.
我們要明白我們今天的人.
很多時候說政教分離.
其實根就來自啟蒙運動.
中國傳統就不說政教分離.
所以今天就算國內說政教分離.
很多都是源自西方的傳統.
西方說政教分離其實是來自啟蒙運動.
啟蒙運動為甚麼說政教分離呢.
最主要就是因為經過宗教改革之後.
歐洲經歷了很多內戰.
於是西方的知識份子就在想.
為甚麼你要說到宗教改革就打了這麼多仗呢.
於是就提出一個結論.
我們要把宗教信仰和公共空間分開.
我們今天經常說公共空間.
其實中世紀沒有人說過這個.
公共空間所謂public square.
這是啟蒙的時候發明的一種東西.
公共空間的特色是甚麼呢.
我是單單說理性.
啟蒙的人為甚麼會這樣說呢.
啟蒙的人相信一件事.
就是認為人的理性是放之四海都可以的.
所以我去到公共空間.
我不說宗教信仰.
為甚麼不說呢.
因為你有你的信仰.

$^{1481}$我有我的信仰.
我們說宗教信仰就吵架.
你覺得要支持佛廟.
我覺得要支持教堂.
說不定就打架.
於是佛廟又放下.
教會又放下.
我們去到公共空間.
只說理性.
理性就你和我都一樣.
起碼啟蒙運動的人這樣看.
中國人,英國人,德國人,火星人.
都是同一樣的理性.
於是我們談得來.
於是我們可以透過一個所謂和平理性的方法.
去解決社會的衝突.
這個基本上是啟蒙的一個夢想.
他們說政教分離是希望這樣.
我只能夠說一句.
時間所限.
跳到最後那句.
公共空間的理性和融合性.
我們今天發覺一件事.
原來理性不是啟蒙那班人說得那麼普世.
原來不同的人可能有不同的理性標準.
我們開始發覺原來.
今天的人就好像.
一個哲學家叫做Megenter.
原來所有的所謂理性的標準.
背後都有一個故事.
為什麼我說這種理性呢.
因為我對整個世界.
這個世界是什麼的世界.
其實有一個Metanarrative.
有一個宏大敘事.
其實我的理性就是宏大敘事的一部分.
其實我們今天的社會.
例如伊斯蘭教.
他們都講人權的.
他們覺得不讓女人開車.

$^{1521}$保護女人的人權.
因為女人讓她開車.
隨時搞很多車禍.
為了保護女人的安全.
就不讓她開車.
就是人權.
你明白伊斯蘭教的國家.
一樣講人權.
但他對人權的理解.
對西方社會完全不同.
於是現在人們又發覺.
其實分不清.
不是像啟蒙那班人那麼樂觀.
以為我講理性.
就將宗教信仰全拋下.
原來不是.
原來理性背後都有一個世界的圖畫.
這個世界的圖畫其實就是涉及宗教.
所謂我講的宗教信仰.
所以到今天西方人又會發覺.
原來啟蒙那班人太樂觀.
我們不可以真的講到.
宗教分離去到一個地步.
純粹拋開信仰就可以解決公義的問題.
所以其實有些福音派參與.
時間有限.
又不跟你講了.
香港今釋又不跟你講了.
有機會第二個talk再講.
最後幾分鐘想跟你講.
再複雜化的問題.
其實今天.
我們現在這個時候.
香港人講宗教分離.
其實都有好幾個方面.
所以希望下次你跟人講的時候.
問清楚.
你講什麼類的宗教分離.
有些人講宗教分離.
就是說教會不應該講政治.

$^{1561}$OK 我們待會傳福音算了.
因為教會的使命不是講政治.
不過對於這些人.
我想我有一個問號就是這樣.
你講不講信服呢.
如果你講信服的時候.
其實已經在講政治.
因為你講你對政府需要有某一種的.
態度.
你一講信服的時候.
其實你就應該繼續講.
信服有沒有限制呢.
如果你說是沒有限制的.
我覺得就不合乎聖經的教導.
舉個例子.
保羅說信服政權.
保羅用同一個字眼說.
妻子要信服丈夫.
但教會不會這樣教導.
妻子說信服丈夫.
就是說丈夫講什麼你都要信服他.
丈夫吸毒.
於是丈夫跟老婆說.
老婆麻煩你買些毒品給我.
老婆說聖經教導我信服丈夫.
好 我幫你去買毒品.
沒有一個教會會這樣教導的.
妻子信服丈夫不是什麼都聽.
所以我們信徒.
或者市民信服掌權者.
都不是什麼都聽的.
所以限制去到哪裡呢.
什麼叫直接與聖經教導有衝突呢.
這句話其實都很多意思.
你要去討論這些東西的時候.
其實就是在討論政治.
所以不容易分辨的.
有些人就說為了福音的緣故.
避免談政治.
談政治就會教會分裂.

$^{1601}$人就會走了.
或者進不了大陸傳福音.
我覺得這些說法.
某一個程度是合理的.
但是否完全解決到呢.
又未必的.
因為其實到今時今日.
你不講政治又有一些人走.
你講政治又有另一些人走.
所以其實你怎樣都死的了.
可以這樣說.
所以對我來說.
我想主要的重點.
不是說為了一些人不走.
不講政治.
而是要培育信徒.
怎樣可以坦誠接受分歧.
我覺得這個才是重點.
譬如你說去大陸傳福音.
其實你要記住一樣東西.
就是說.
OK 你換了今天一時的方便.
會不會其實失去一些見證呢.
原來基督教都是一個拉攏政權的宗教.
會不會這方面在長遠更大的損失呢.
不知道.
我只是想提出一個問號.
我想每個人的審判都不一樣.
第三種的政教分離.
就是講利益的分割.
譬如舉個例子.
教會應不應該收政府的錢去辦學校呢.
舉個例子.
我認識一些浸會的朋友.
對於浸會大學當年決定要政府的資助.
他們覺得是錯誤的選擇.
因為你拿了政府的錢.
就要跟政府做事的.
標準.
你可以譬如舉個例子問.

$^{1641}$今天我們這麼多間大學.
只有一間浸會大學.
掛名是基督教.
但是浸會大學有多少基督教化呢.
這個真是一個問題.
所以有些的政教分離.
就說教會應該不要拿政府的錢.
利益分割.
你會發覺譬如一講這個.
就導致那些親政府的人.
通常就很敏感.
我們講的政教分離不是這種分離.
講什麼分離呢.
我想告訴你.
分離都有很多種分離.
有些人就會強調分離就在於神職人員分泌為性.
所以神職人員不可以參與任何政治運動.
Facebook都不可以講政治.
不過當然又涉及到一個問題.
神職人員是不是一個市民呢.
如果神職人員不可以Facebook講政治.
神職人員可不可以投票呢.
不如去到一個地步.
好像從浸派一樣.
神職人員票都不要投.
神職人員交稅給政府.
是不是支持一個暴政呢.
如果政府不好的話.
我的point我覺得你要去問.
神職人員分泌為性是正確的.
但是分泌去哪一個地步呢.
在哪一條線上畫呢.
其實是很含糊的.
最後一點就是政教分離.
其實是在講教會的視野.
應該不是限於一個國家裡面.
這個通常就是所謂的「黃」.
通常強調這樣的.
那些「藍」就比較有傾向.
所謂「藍」的傾向就是.

$^{1681}$教會應該要支持國家.
表示愛國的行動.
但是是不是要這樣呢.
如果教會要愛國的話.
是不是已經是一種政教不分離呢.
這個都是要問的問題.
總括來說.
其實這五點都想告訴你.
其實有些東西.
政教分離就是「黃」那邊強調.
有些東西就是「藍」那邊強調.
其實兩邊都有講政教分離.
兩邊都有講政教不分離的地方.
所以我們對話的時候.
有時要很小心.
真的要聽.
問清楚對方.
你講政教分離.
是什麼分離呢.
哪方面分呢.
哪方面離呢.
有沒有哪方面其實你是講合的.
要這樣細心對話的時候.
教會才可以信徒.
才可以互相聽到大家講話.
今天的時間到這裡.
多謝大家觀看.
(字幕由 Amara.org 社群提供).
\newpage



\section{}
\label{sec:2dfqNGcDljE}
\textbf{【疫有嘢學 │ 延SUN在線】愛在差異蔓延時-婚姻成長講座|林添德先生}
\newline
\newline
連結: \href{https://youtube.com/watch?v=2dfqNGcDljE}{\texttt{ https://youtube.com/watch?v=2dfqNGcDljE}} ~~~~ 語音日期: 2020-06-20 
\newline
\newline
\hyperref[sec:xiMH3MdBCkY]{\small{< < < PREV SERMON < < <}}
~
\hyperref[sec:index]{\small{[返主目錄]}}
~
\hyperref[sec:OmTXVUsNi_8]{\small{> > > NEXT SERMON > > >}}
\newline
\newline
$^{1}$各位同學平安.
歡迎大家再次來到「亦有ye 學」延伸在線.
先跟大家做個報告.
五月份「亦有ye 學」的五個主日課堂已經配上字幕.
會在來近的星期二開始重新放在忠臣的YouTube Channel.
歡迎大家隨時重溫.
說說這個主日的安排.
課堂是由忠臣輔導科副教授林添德先生負責.
Edmond老師過去幾年都有參與忠臣延伸部一些輔導科目的構思.
這些課程很多時都是關於心理輔導和靈命成長的整合.
今天他會跟我們分享一個關於婚姻成長的題目.
愛在差異蔓延時.
沿著林添德先生今天的分享.
讓我們去準備並且操練一個有上主同在的婚姻生活.
再一次歡迎大家參加我們的延伸在線.
今天我們的主題是愛在差異蔓延時.
我想跟大家探討一下夫妻之間的差異.
究竟是怎樣影響他們的婚姻關係.
差異的意思是指人與人之間的不同或人與人之間的分別.
例如我們有個性的不同,個性的分別.
有些人較為外向,有些人較為內向.
外向的人一般會將他們的不開心跟別人分享.
而內向的人則比較習慣將感受留給自己去處理.
你說這個外向與內向的差異.
如果放在夫妻的相處裡面會有甚麼發生呢?.
可能會令他們不容易去明白對方也說不定.
例如外向的人會覺得明明看到內向的也很不開心.
不斷問他也說沒甚麼,不斷問他也說不如給我一點時間.
所以真的很不明白.
又或者反過來,內向的人會覺得很奇怪.
覺得為何這個外向的重複又重複地跟我分享一件他不開心的事.
究竟有甚麼幫助呢?.
又或者有另一個差異也很常見.
就是性別上的差異.
如何影響夫妻之間的溝通呢?.
一般來說我們會說女性是比較多憑感覺和直覺去做判斷.
而男性會比較多用他們的理性或者有事實的根據去分析事情.
例如太太跟先生說.
喂,你好久沒有跟我談戀愛了.
先生的反應如果是很客觀分析的話.

$^{41}$他可能會這樣跟太太說.
上個月不是約過你嗎?.
其實太太想表達的「很久」其實是一種感受.
絕對不是在搞清楚時間或者先生約了自己的次數.
她其實是很想丈夫能夠主動去關心自己.
從這兩個例子我們可以想像到.
差異在夫妻關係裡甚至差不多一切人與人之間的關係裡是無可避免的.
夫妻之間的不同其實是需要大家一起努力去互相配搭去互補不足.
差異有時成為夫妻之間的分歧就會產生衝突.
有衝突的時候也可能是反映大家越來越能夠坦白.
將自己的看法,自己想怎樣,喜歡不喜歡的東西.
直接一點來跟對方表達.
所以適量的衝突是有建設性的.
它可以加深夫妻之間的彼此了解.
不再單單關心自己想要的東西.
也會考慮對方想要的東西.
今天的延伸在線.
我很想跟大家分三個部分去分享一下.
怎樣愛在差異萬年時.
第一個部分就是其實差異可以帶來給夫妻一個成長的機會.
第二點就是差異同時不能給予我們成長.
可能會令我們受傷害.
第三點就是這些傷痛也很可能會帶領我們去耶穌基督那裡得到救贖.
我們先集中講第一點.
在美國有一個已故的婚姻家庭治療師.
她叫做Virginia Satir.
她專門研究人的溝通和互動模式.
在她的這本聯合家庭治療裡.
她認為夫妻之間的差異是一個成長的契機.
可以令夫妻一起成長.
我們先看看Virginia Satir覺得可以帶給夫妻的成長.
她認為差異可以幫助夫妻之間開放一些去聆聽和自己不同的建議,看法或者選擇.
她設了一個場景.
她說夫妻一個想吃漢堡包,一個想喝茶,吃點心.
大家都很想說服對方去選擇自己的喜好.
或者最起碼是去跟對方表達自己想要的東西.
介紹一個跟對方不同的選擇.
Virginia Satir認為夫妻可以藉著對於食物有不同選取的機會.
學習開放自己.
去考慮跟自己不同的選擇.

$^{81}$Virginia Satir將學習聆聽放在成長裡的第一位.
可能她知道要處理夫妻之間的差異很關鍵的一個地方.
就是大家都願意開放聆聽跟自己不同的看法.
我想用另一個學者的資料.
另一個學者叫Stephen Coffey.
他專門研究人的心理發展和人際關係.
他有一本很多人可能都看過的作品.
叫做Seven Habits of Highly Effective People.
他提到.
很多人都不是專心聽別人說話.
他有一個名言.
We listen to reply.
意思是我們聽別人說話的時候.
其實我們已經在想我們要怎樣回答.
往往沒有一個很開放的態度去嘗試了解對方.
今天我想介紹一下.
Coffey提到有兩個阻礙我們開放聆聽的因素.
第一個就是覺得.
有些時候我們會覺得自己對於自己的配偶已經很熟悉.
甚至我們會覺得我們比配偶更熟悉.
去認識他們自己.
我們真的很清楚他們的人.
我們很清楚他們需要什麼.
所以我們不著重去聽他們說話.
我們反而著重去告訴他們.
應該怎樣選擇.
應該怎樣做.
另一點是Coffey認為.
第二個令我們不容易去聽別人說話的.
可能我們有一種意義分發的觀念.
如果我聽你說.
如果我跟隨你.
那你不就贏了.
我不就輸了.
又或者我聽你說.
我跟隨你.
那我想要的東西就沒有了.
Coffey就覺得.
其實這個世界是很大的.
很多資源.

$^{121}$我們的資源都是足夠我們去用的.
所以他覺得人與人之間的合作.
是不需要那麼計較.
我們可以多一點不怕吃虧.
去跟別人分享.
他說完兩個障礙之後.
他也提到有兩個出路.
第一個會幫助我們開放一點.
去聆聽不同意見的出路.
就是我們要知道.
如果人與人之間是需要合作.
需要好好地相處.
首先是要建立信任的.
Coffey就覺得.
要建立信任.
首先一個很重要的方法.
簡單來說就是.
你先給了別人.
什麼叫先給了別人呢?.
就是說.
如果你想別人聽你說話.
你不如先聽別人說話.
如果你想別人考慮你的意見.
不如你先了解一下他的看法.
甚至乎如果你想去影響別人.
不如先給別人影響.
因為透過這個過程.
對方就會發覺.
咦?.
你不單單是想要自己的東西.
你也很關心我想要什麼.
就是在這個地方開始.
我們就能夠建立信任.
另外一個出路.
這個出路就是說.
其實我們每個人.
都有很多不同的情感需要.
Coffey就選擇了.
或者他認出了.
有四種情感需要.

$^{161}$是每個人很基本的.
基本到他形容為.
好像空氣那麼重要.
是可以維持我們的心理健康.
這四種情感需要.
包括了解,明白,肯定.
以及被欣賞.
我試試舉例.
黃先生很積極,很忙碌地.
在預備自己的專業考試.
黃太太就覺得他實在太緊張了.
緊張到連家裡的責任都放棄了.
當然最不滿意的就是.
忽略了太太.
所以黃太太可能也看過.
Coffey她的書.
雖然她對先生有一點不滿.
很想她不要那麼緊張.
很想她能夠顧及自己.
但是她也選擇先去聆聽丈夫.
同時關心丈夫的情感需要.
當黃太太這樣去選擇.
聆聽丈夫的時候.
她了解到其實黃先生這次.
很想考好這個專業試.
是希望能夠發揮自己.
以及對於他的專業貢獻更多.
在談話中黃太太也明白黃先生.
其實這次的專業試給他很大的壓力.
因為他不是第一次考.
其實他之前也已經胖過頭.
還有就是原來比他遲出道的人.
也有很多已經考到這個專業試.
所以他很大壓力.
也很害怕自己這次考不到.
所以黃太太就去肯定他.
黃太太就說.
你今天的努力是不會白費的.
無論如何你都是幫到人幫到自己.
黃太太更加去欣賞丈夫.

$^{201}$也覺得他對於自己的專業.
是很嚴肅,很認真的.
如果黃先生聽到太太給你這些情感.
給你這些支持.
你會覺得怎樣呢?.
我們也希望黃先生.
在吸入了太太給他的心理空氣之後.
有足夠的空間容量.
去聽聽太太對他的不滿.
我們又要回到Virginia Satir.
去看看她繼續告訴我們.
有甚麼差異可以帶給我們.
婚姻關係的成長.
我們說了第一點.
就是學習開放的聆聽.
第二點就是學習妥協.
其實我們也可以試試輪流.
去將自己的喜好.
這次不如讓你先.
我跟著你去喝茶.
下次我們就要吃漢堡包.
又或者我們可以試試大家的創意.
我們可以又不是選擇喝茶.
又不是選擇吃漢堡包.
我們會不會想到一個選擇.
是你和我都會很喜歡.
你和我都會想去吃呢?.
這裡就可以幫我們.
發揮一下我們的創意.
原來處理差異.
其實我們都要實際一點.
當我們考慮我的想法.
和你的想法的時候.
我們都要考慮實際的因素.
可能吃漢堡包是比較近.
是比較節省時間.
事實上我們根本上都沒有那麼多時間.
坐下來慢慢喝茶.
不如下次時間多一點.
其實都要考慮實際的因素.

$^{241}$學習面對現實.
最後一個.
Sajiya提到成長的可能性.
就是學習尊重不同的個性.
既然你有你的選擇.
我有我的喜好.
不如這樣吧.
我們這次可以.
你選擇你喜歡吃的漢堡包.
我選擇我喜歡去享受一下茶.
一個小時之後.
我們在這個商場.
再走在一起.
不知道你覺得這個方法怎樣.
可能你會覺得.
這麼沒有時間走在一起.
能夠走在一起也不容易.
還要分開.
都真的很難.
不過這一點.
其實Sajiya可以給我們.
去看看.
其實我們有很多成長的可能性.
當我們去.
運用.
運用我們不同的想像力.
去處理我們的分歧的時候.
其實是可以有很多的可能性.
而這些可能性.
都帶給我們成長.
如果你說.
幾點成長的要素.
其實我都不是記得很清楚.
可不可以重覆呢.
與其重覆.
不如我將Sajiya.
一個很重要的理論.
來跟大家分享一下.
可能她很多成長的項目.
都在她這個理論上.

$^{281}$去演變出來.
Sajiya就覺得.
其實人與人之間的相處,溝通.
最重要是能夠顧及三方面.
第一方面.
就能夠顧及自己.
第二方面.
就能夠顧及其他人.
第三方面.
就能夠考慮到當時的處境.
她就是用這三方面.
來分析到.
人可能有幾種不同溝通的模式.
是會障礙我們.
跟一個好的人際關係.
第一個.
她就叫做討好型.
你想像一下.
其實她說的.
這些人.
他們很不顧及自己.
他可能看得自己很輕.
看得自己很小.
看得自己很渺小.
相對地.
他就看得其他人很大.
很多時候.
他對人的形態.
都好像自己要蹲在那裡.
伸起雙手.
去討好其他人一樣.
Sajiya就覺得.
她只是顧及了其他人.
顧及了別人.
沒有顧及到自己.
第二種指責型.
就跟討好型相反.
反過來.
指責型就是.
看得自己太大.

$^{321}$看得自己太重要.
而忽略了其他人.
覺得自己高高在上.
有什麼不舒服.
有什麼問題.
她都會很容易覺得.
不是自己的問題.
自己不會錯的.
不會有問題的.
一定是其他人的問題.
所以就很容易去指責其他人.
第三種超理智型.
Sajiya的意思就是說.
這種型格的人.
他們在溝通相處的時候.
只是著重理性.
他們忽略了自己的感受.
也忽略了其他人的感受.
固然他明白不到自己.
也明白不到別人.
跟他分享的遭遇.
是怎樣的一回事.
最後一種.
Sajiya叫做打岔型.
就是說.
是三方面他都忽略的.
他忽略自己.
忽略其他人.
也忽略了當時的處境.
完全是一些無理拉扯.
不識切的反應.
我把Sajiya最主要的理論.
跟大家分享了.
其實這個理論.
都跟健康婚姻的指標.
是很一致的.
一個很重要的健康婚姻的指標.
最重要的是在婚姻關係裡.
能夠照顧自己的需要.
同時都可以考慮對方的需要.

$^{361}$透過這個又考慮你又顧及我的關係裡.
我們就可以建立彼此的關係.
彼此的情感.
這個就是健康婚姻關係.
很重要的結構.
說到很多差異帶來給我們的正面影響.
我們都轉一轉台.
去看看差異帶來給我們一些負面的影響.
Sajiya認為當夫妻處理差異的時候.
變成了一些很經常的爭執.
其實就反映了.
大家可能接受不到對方.
跟自己是有不同的.
不可以容納差異的存在.
就會用一些很破壞性的方式.
試圖去消滅差異.
我們看看一些怎樣破壞性的方式.
第一個方式.
就是將我們之間的差異拖延.
去逃避.
避開它.
而不是去處理.
第二個.
是用強迫性的方法.
怎樣都要對方去跟自己的選擇.
自己的喜好.
不理會對方想怎樣.
第三個方式.
是將差異模糊化.
我覺得也有一點哄對方.
其實吃漢堡包和喝茶.
都是差不多的.
不如你跟我吧.
接著這個方法.
就是我們近乎去嚇對方.
去說對方.
你不要搞錯.
你喜歡吃雞.
你喜歡吃漢堡包.
你喜歡喝茶.

$^{401}$你從來都不喜歡的.
為何你會這樣呢?.
是去質問對方.
最後這個就重要了.
Satire就說.
有些人是不容許差異到一個地步.
覺得如果你不跟我的意思.
就等於你不愛我.
所以就用責備和道德的判斷.
去怪責對方不跟自己.
這些方法全部都是希望.
來消滅我們之間的差異.
不去處理.
最後就要你跟我.
為何會這樣呢?.
為何有些人不可以容納差異呢?.
Satire就認為.
我們不能夠容納.
大家有不同,大家有差異.
是因為我們的自尊感出現了問題.
Satire就覺得.
如果我們的自尊感是足夠的.
我們就會對對方有足夠的信任.
就不會很容易覺得.
如果我跟了你.
我就好像沒有自己一樣.
其實自尊感是一個很重要的心理健康指標.
我們每個人都需要某程度上.
覺得自己是有價值.
覺得自己是重要的.
我們需要有這種足夠的自尊感.
去面對我們的成功.
又或者去接受我們的失敗或者限制.
當我們能夠肯定自己的價值.
就可以重視自己.
亦都不會那麼容易害怕.
如果我放下自己不要的東西.
很容易就好像被人搶走了.
一些屬於自己的東西一樣.
當夫妻不能夠容納大家的差異.

$^{441}$不能夠運用差異去令大家成長.
差異就很可能會帶來給大家一些傷害.
Satire是幫助我們明白得到.
如果我們能夠好好運用差異.
當我們不能夠好好運用差異的時候.
其實是因為我們缺乏了信任.
而我們不能夠建立到和對方的信任.
是因為我們的低自尊.
接下來我們嘗試參考王勵章博士.
他描述了夫妻關係的幾種相處模式.
我們試一下去看看這幾種相處模式.
是為什麼會處理不到差異.
而為什麼處理不到差異.
到了一個地步是帶來傷害.
我們先看看第一種.
第一種就叫冷戰型夫婦.
這一類的夫婦會給你一種感覺.
就是覺得他們很安靜.
他們好像沒有什麼話說.
他們的常用語就是.
「你先選擇吧,我無所謂的」.
「我跟隨你,我什麼都可以」.
這些就是他們時常溝通的方式.
當我們這樣溝通的時候.
其實你會發覺.
他們把自己收得非常近.
第一點就是說到這一類冷戰型的夫婦.
他們表面是很和諧.
不過其實他們也把很多他們的不滿.
收起來了.
好像他們把自己收起來一樣.
他們的個性也會習慣去忍讓.
去包容和避免衝突.
這些質素其實也能幫助夫妻建立關係.
只不過他們把自己的自我.
過份隱藏,過份收起來.
令對方沒有足夠的空間.
去了解和認識自己.
也因為這樣很難建立親密的關係.
第二點就是說他們的特性.

$^{481}$沒有經過澄清的假設.
不如我們了解一下陳先生,陳太太.
他們的相處.
會幫助我們更加明白.
冷戰型的夫妻的互動模式.
陳先生,陳太太.
在他們結婚的初時.
陳先生為了想跟陳太太.
有更多相處的機會.
因為他也發覺其實大家好靜.
大家好像沒什麼話說.
陳先生就提議.
不如跟陳太太一起週末去踩單車.
看看可不可以建立一個.
大家共同都喜歡的興趣.
這樣可以多點時間在一起.
陳太太因為自己的低自尊.
她不相信自己的價值和重要性.
所以她不太相信.
不太有把握.
丈夫真的想自己.
想跟自己建立共同的興趣.
丈夫不是真的想多點機會跟自己一起.
她很有質疑.
她很沒有把握.
所以她就這樣發福先生.
到時看看怎樣.
陳先生聽到陳太太說到時看看怎樣.
心裡很疑惑.
究竟太太想怎樣?.
什麼叫到時看看怎樣?.
究竟她是想去還是不想去?.
可惜陳先生也是低自尊.
也令他不敢問清楚太太.
其實他想不想去呢?.
因為他怕如果他問太太.
就迫太太做些他不想做的事.
那就不好.
其實陳太太心裡就這樣想.
陳太太就想.

$^{521}$如果陳先生真的想我去就算了.
他如果想跟我去.
想多點時間跟我在一起.
他會再叫我去.
結果去到周末.
什麼都沒有發生.
陳先生見陳太太沒有再提.
他就當她不想去.
只是沒有直接拒絕她.
陳太太見到陳先生沒有再問她.
她就更加相信自己的想法.
自己的假設是真的.
她更加認為丈夫不是真的想跟她去.
她也覺得她估計的是沒錯.
他們的相處就是嚴重缺乏溝通.
他們只不過是經常估計對方.
假設對方.
慢慢這些估計和假設.
就變成他們以為對方真的好像他們想的.
第三點就是說.
當他們越來越不去開放表達溝通的時候.
他們的關係就變得越來越遠.
越來越生疏.
他們好像覺得婚姻裡的愛.
就是讓你去做你喜歡做的事.
這種就是我對你的愛.
越來越他們生活在忙碌的自己的天地裡面.
陳先生繼續去發展他的單車隊.
陳太太就繼續忙於她自己的工作.
為什麼會繼續下去.
為什麼會成為了大家對大家的憤怒呢.
因為在完全沒有交流溝通的情況下.
各自以為自己不斷為對方犧牲.
從來都不要求對方為自己做些什麼.
一直以為我不煩你.
我不阻礙你.
就是愛你.
其實這類型的冷戰型的夫婦.
因為他們的低自尊.
他們覺得自己的價值和重要性都不高.

$^{561}$所以他們是不會相信有人會對自己好的.
他們覺得他們的需要只不過是會帶給別人麻煩.
自己是覺得很內疚.
我們看看另一種.
我們看看熱戰型的夫婦.
熱戰型的夫婦和冷戰型的夫婦剛剛相反.
冷戰型的夫妻覺得他們的自我太過微小.
太過隱藏.
熱戰型的夫妻自我強勁.
他們雙方都看到自己很大很重要.
所以兩個強勁的人走在一起的時候.
不甘示弱.
產生很多差異.
我們看看第一點.
他們經常爭吵.
因為他們的自我膨脹.
他們覺得自己很重要.
很容易覺得自己一定是對的.
多數都是對方有問題.
對方出錯.
所以溝通的時候.
經常會指責對方.
我們看看李生李太的經歷.
李太曾經有兩次小產.
最近懷孕了.
所以李生和李太心情都比較緊張.
患得患失.
很害怕這次又再小產.
但因為兩人的個性都很強勁.
不容易將心裡的憂慮.
來跟對方分享.
只是透過指責和控制對方.
去安慰自己心裡的不安.
我們看看他們是怎樣指責和控制對方.
我們試試聽他們的一段對話.
李先生有一天跟太太說.
「你小心點吃東西」.
「我叫你不要吃沙冷的東西」.
「你還吃雪糕?」.
李太聽到當然很不開心.

$^{601}$她就這樣回答李生.
「我吃了多少?」.
「只是小杯裝而已」.
「你用不著這麼緊張吧?」.
李先生聽到當然很不服氣.
覺得太太為何不聽自己的提醒.
他就跟太太說.
「我就是之前不夠緊張」.
「由得你亂吃東西」.
「你不是小產了兩次嗎?」.
如果你是李太.
你聽到先生這樣說的時候.
你會有甚麼感受呢?.
李太的反應是這樣的.
「你說甚麼?」.
「是你這麼緊張?」.
「每天晚上都追著我」.
「給我壓力」.
「令我無法保胎」.
「我不是想這樣說的」.
「是你逼我的」.
大家是否感受到他們那種熱戰的氣氛呢?.
不斷互相指責和反擊.
要用這樣的對抗性方式來相處.
其實是很傷感情的.
我們看看熱戰型的夫婦.
他們的深層問題在哪裡.
其實他們的深層問題.
深層的恐懼.
是不相信有人會接納到自己的軟弱和限制.
所以他們一定要埋藏自己.
特別是埋藏自己不行的東西.
埋藏自己擔心的東西.
埋藏自己的軟弱.
來很強勁地去指責控制對方.
去令自己的恐懼減少.
我們看了冷戰型和熱戰型的夫婦之後.
我們去看第三種夫婦的形態.
這種就叫做「你追我」的地形夫婦.
是一種互補不足的夫妻.

$^{641}$「你追我」地形的夫婦.
其實是典型的.
我們有時也會這樣說.
一個「粒」配一個「突」的組合.
是絕對發揮互補不足的功能.
我們先看看第一點.
一個「粒」一個「突」.
一個是諸多不滿.
一個是以沉默去回應.
我們看看張先生張太太的經歷.
張太太是諸多不滿的一個.
她認為做人要真.
不滿意的時候.
或者就算滿意也好.
也要說出來.
不要放在心裡.
張先生很怕衝突.
最重要是以和為貴.
所以就算對太太有多少不滿.
也只會藏起來.
絕對不會說半句出來.
讓太太知道.
本來張太太喜歡說.
而張先生喜歡聽.
這個一個「粒」一個「突」的組合.
是不錯的.
但是如果這個互補不足的模式.
長年累月.
都不可以輕輕的調轉一下.
我們就會說.
他們各自都困在自己的安書區.
是不能夠成長.
令到夫妻在相處裡.
被這種很固執的模式困住了.
絕對是缺乏一些彈性.
去應付人生很多很多的轉變.
事情就是當社會運動開始的時候.
張先生發覺自己和太太.
在政見上有很大的分歧.
但可惜張先生一直都停留在.

$^{681}$藏起來不說這個安書區裡.
從來都沒有試過和太太表達.
自己的不滿.
而太太也從來沒有試過.
去聽一下先生的意見.
所以當大家處於政見分歧的時候.
他們互補不足的相處模式.
是不能夠去消化這個這麼大的分歧.
張先生產生很大的憤怒.
但他不能夠和太太去表達他的不滿.
唯有將這些憤怒變成和太太保持距離.
而這些距離令到張太太覺得.
張先生是越來越不愛自己.
張先生和張太太的互補不足.
其實經常都要投訴的.
就是要去肯定對方不會遺棄自己.
對方是喜歡自己的.
而不斷逃避的就是很害怕.
控制不到自己.
很害怕如果她真的說出自己的不滿.
會令到夫妻有一些情緒.
大家是控制不到.
所以他們一個是追一個是避.
如果他們這個互動模式.
能夠嘗試給多一點彈性.
調換一下位置.
經常追那個可以等一下.
經常避那個可以主動一點.
其實他們就會有成長的機會.
亦都令到他們互補不足的模式.
是可以成長.
可以有足夠的彈性.
去應付人生很多不同的轉變.
我們看了這三種夫妻相處的模式.
陳李璋這三對受傷的夫妻.
其實他們想明白了.
原來他們面對的婚姻問題.
是和自己的低自尊有關的.
他們開始對配偶有多一點的明白.
多一點的體諒.

$^{721}$大家都願意放下大家的防衛.
放下大家的保護罩.
嘗試真誠一點去和對方相處.
去重建大家的信任.
和關心大家的情感需要.
這一步,小小的一步.
其實都是一點都不容易的.
可能你現在的婚姻關係.
是沒有這三對夫婦那麼困難.
又或者你覺得自己比他們更加惡劣.
令你覺得筋疲力竭.
我也很想和你一起去呼吸一下.
屬靈的空氣.
去給一些安慰.
給一些鼓勵大家.
我們看看《馬太福音》十一章二十八至三十節.
這三節的經文.
第一節說:煩惱負擔重擔的人.
可以到我這裡來.
我使你們得安息.
耶穌和當時的以色列人說了一句話.
因為當時的以色列人.
被律法主義壓得他們透不過氣.
耶穌在這裡向他們發出一個邀請.
耶穌說:你來吧.
你來吧.
你來我這裡.
你就可以放下現在壓得你很辛苦.
甚至壓得你很痛苦的重擔.
我想問.
現在令你很勞苦.
壓得你很辛苦的重擔.
是什麼呢?.
陳李璋這三對夫妻.
他們的重擔.
其實是他們的恐懼.
他們很害怕在親密關係裡.
去表現真正的自己.
他們只可以很勞苦地.
重複又重複.

$^{761}$用很多防衛的方法.
去保護自己脆弱的自我.
他們差不多用盡所有的方法.
希望可以在婚姻關係裡.
去肯定自己的價值和重要性.
說到底.
是希望自己能夠成為一個.
值得被人愛的人.
我們回到.
《馬太福音》十一章二十八節.
這一節的經文.
耶穌說.
你來我這裡跟隨我吧.
你來我這裡依靠我吧.
你就可以放下.
這個要證明自己值得人愛.
值得人重視的重擔.
一本英文的翻譯聖經.
叫《Message》.
作者Eugene Peterson.
認為二十九節.
是一節耶穌向我們說.
祂是一個怎樣的人.
耶穌在這一節裡.
很想邀請我們.
去經歷一下祂的恩典.
Eugene Peterson形容.
耶穌的恩典.
是一個unforced.
unforced rhythm of grace.
是一個不催逼.
不勉強的恩典節奏.
意思就是說.
耶穌說.
我心裡柔和謙卑.
意思就是說.
祂會用祂的恩典.
來接納我們的軟弱.
接納我們的限制.
祂不會勉強我們.

$^{801}$祂不會強迫我們.
第三十節.
耶穌繼續去吸引我們.
邀請我們.
去讓我們有一些.
更加大的肯定.
Eugene Peterson的翻譯.
在第三十節裡說.
耶穌給我們的擔子.
是不會過於我們能力.
所能夠承受的.
耶穌給我們的擔子.
是會適合我們的.
祂不會給一些.
不適合我們的擔子我們.
換句話來說.
耶穌很想來肯定我們.
叫我們安心去跟隨祂.
因為祂給我們的擔子.
是為我們度身訂做的.
我盼望.
無論你今天.
是一個怎樣的婚姻關係.
是一個怎樣的家庭關係.
或者人際關係.
或者和神.
是一個怎樣的光景.
怎樣的關係裡面.
我都希望.
今早.
耶穌基督的邀請.
是來吸引你.
是來讓你願意回應祂.
以致你可以.
不再倚靠.
你一直倚靠的擔子.
這個擔子.
只是令到你勞苦.
耶穌想邀請你.
去倚靠祂.

$^{841}$讓祂給你.
有一個.
是你能夠承擔.
也都適合你的擔子.
我繼續希望.
耶穌的說話.
來吸引你.
來祝福你.
我今天和大家的分享.
到這裡.
願神用祂的話語.
繼續來賜福給大家.
多謝大家.
(字幕製作:貝爾).
(字幕由 Amara.org 社群提供).
\newpage



\section{}
\label{sec:OmTXVUsNi_8}
\textbf{【疫有嘢學 │ 延SUN在線】當急難敲門時|王下4章8-37節|張智聰博士}
\newline
\newline
連結: \href{https://youtube.com/watch?v=OmTXVUsNi-8}{\texttt{ https://youtube.com/watch?v=OmTXVUsNi-8}} ~~~~ 語音日期: 2020-04-18 
\newline
\newline
\hyperref[sec:2dfqNGcDljE]{\small{< < < PREV SERMON < < <}}
~
\hyperref[sec:index]{\small{[返主目錄]}}
~
\hyperref[sec:V3Cpa_Vt9fE]{\small{> > > NEXT SERMON > > >}}
\newline
\newline
$^{1}$各位同學平安.
歡迎你們來到《亦有ye 學》.
延伸再線的第三個課堂裡面.
今個主日的課堂.
是由張志聰博士負責講授.
Simon老師是忠臣的舊藥科.
許書楚教席副教授.
亦是我們忠臣的副教務長.
Simon對詩篇.
以及聖經文類的研究是頗有心得的.
不過今日Simon老師不是跟我們講詩篇.
他會分享《列王記下》四章八至三十七節.
題目是「當急難敲門時」.
袁主席他自己的說話向我們敲門.
使我們能夠躺開心思意念.
讓主進入我們的生命裡面.
今日我想和大家看一段經文.
記載在《列王記下》第四章八至三十七節.
我想藉著這段經文.
同樣地和大家一起學習.
如何再一次更加仔細地觀察一段經文.
特別是聖經學者如何將一些.
所謂重複或對比的技巧放在當中.
來突出他想帶出的一些重點.
除此之外,我也很想藉著聖經作者.
如何用他的手法來塑造這個故事裡面.
兩位角色的性格.
還有一樣東西也很想和大家一起探討的是.
聖經作者如何藉著描述一件急難的來臨.
以致帶來了當中兩個故事人物.
他們生命的改變.
今日我會用的經文版本會是和合版.
修訂版,所以弟兄姊妹可以特別留意.
這段經文所記載的故事.
和以利沙其他的故事不太一樣.
我們見不到鄰國一些很厲害的戰爭.
我們也見不到一些大將被醫治的神蹟.
我們見到的是以利沙和一個婦人.
甚至連名字也沒有的婦人.
他們之間的一個相遇的故事.

$^{41}$故事一開始的時候.
是講述以利沙如何認識這個婦人.
「一日,以利沙經過書簾.
在那裡有一個富有的婦人強留她吃飯.
此後,以利沙每次經過就轉到那裡吃飯.
婦人對丈夫說.
看吶,我知道那常從我們這裡經過的是神聖的神人.
我們可以為她蓋一間有牆的小閣樓.
裡面安放床塔,桌子,椅子,燈台.
每當她來到我們這裡就可以住在那裡.
一日,以利沙來到那裡轉進那閣樓.
躺臥在那裡」.
如果你剛才留心聽我讀的時候.
你會發現經文一開始短短幾節.
就已經重複了好幾次一個字眼.
那個字眼就是「那裡」.
第十節裡面「裡面」那個字.
其實應該都應該翻譯成「那裡」這個字眼.
為甚麼聖經作者那麼強調「那裡」這個字眼呢?.
如果我們再仔細去看的時候.
你會發現其實這六節的「那裡」.
其實不是在說同一個地方.
一開始的時候第八節「那裡」.
其實就正正在說上文書念這個地方.
書念是一個甚麼地方呢?.
如果我們看回聖經的地圖的時候.
我們會發現書念其實在加密山東南以東的一個地區.
而它離開加密山大概有三十五公里的距離.
三十五公里大概是十幾個小時的路程.
加密山正正就是以利沙他所居住的地方.
所以我們見到書念不是以利沙他居住的地方.
而書念並不是以利沙他的家.
以利沙不知為了甚麼原因.
他來到書念這個地方.
是一個可能他沒有人認識.
亦沒有地方可以去投靠的一個地區.
敬文一開始說的「那裡」.
就是在說這個不是以利沙的家的書念.
不過當他再出現的時候.
我們會發現這個「那裡」已經成為了.

$^{81}$由書念這麼大的地方進入了一個婦人的家.
再下來的時候.
第三次的「那裡」出現.
已經是在說婦人的家裡面.
一個所謂有牆的小閣樓.
如果你正在用和合本舊版的聖經的時候.
你會發現和合本翻譯成為牆上蓋的一間小樓.
可能你會有點拗頭.
究竟這句說話是說甚麼.
其實和合本修訂版比較正確.
說的是原來一個小的閣樓是有牆的.
有牆的地方代表這個地方已經成為了一個.
可能是一個比較長期會設立的房間.
不是朝寒晚差.
等以利沙來就突然間起氣出來.
而是已經長期擺定在這裡.
讓到以利沙隨時來.
他都可以進入的一個房間.
第四次再出現的「那裡」.
亦再一次是在說這個有牆的小閣樓.
然後之後十一節就說.
以利沙真的來到「那裡」了.
可能這裡說的是書念.
不過亦都在說的是婦人的家.
然後到最後一次.
就說以利沙躺臥在一個有牆的小閣樓.
我想你發現的是.
第一次以利沙踏足這個地區的時候.
「那裡」是在說書念.
一個不是他家裡,沒有人投靠的地方.
逐漸一步一步.
這個「那裡」就成為了婦人的家.
這個「那裡」就成為了一個婦人為他預備.
一個有牆的小閣樓.
這個有牆的小閣樓裡面有些什麼呢?.
有床,有桌子,有椅子,有燈台.
在當時的以色列人家裡面.
其實有家具,並不是每一個家人.
每一個人家都會有的一些裝備.
能夠為一個這麼小的房間.

$^{121}$擺了這麼多所謂的家具.
我想就大概好像我們今天家裡面.
擺了一張奧禪的愛摩爾一樣.
其實是一個非常豪華對於以利沙的招待.
所以對於以利沙來說.
他衷心地感受到.
在一個不是他自己的地方.
但他找到一個可以落腳休息.
而且甚至乎有很優厚的條件.
去招待他的地方.
我想你就很明白.
其實以利沙心裡面.
對於這位書唸的婦人.
他的慷慨所帶來對他心裡面.
那份感恩的情感.
所以十二至第十七節.
就不斷地說以利沙怎樣找一個方法.
來報答這位婦人.
十三節和十四節.
他兩次都問.
我可以為這個婦人去做些什麼呢?.
以利沙很希望可以藉著他為這個婦人.
去滿足她的一些要求.
以致去報答這位婦人.
他接待她的一些恩情.
不過最後這個婦人.
他沒有說出他有什麼要求.
只是他的以利沙的僕人.
希西卡西.
他知道了這個婦人原來她沒有孩子.
所以第十六節.
以利沙他就對這個婦人說.
明年這個時候.
你必抱一個兒子.
以利沙應許了這個婦人.
她人生裡面一個最重要的需要.
不過你不要小看明年這個時候.
這句說話.
聖經作者都特別要我們去留意這句說話.
因為十六節這句說話在以利沙的口中.

$^{161}$十七節聖經作者再一次重覆.
婦人果然懷孕到了明年那個時候.
為什麼這句說話這麼重要呢?.
如果我們看回希伯來文裡面.
這句說話其實明年這個時候.
有四個希伯來文字.
其實這四個希伯來文字裡面.
其中有三個出現在另一段經文.
甚至可以說在舊約聖經裡面.
只是在這兩個地方有出現.
那地方就是《創世紀》第十八章第十四節.
日期這個字眼.
就是明年這個時候其中一個字的字眼.
接著在《創世紀》第十八章.
明年這個時候.
其實是在說獵王紀夏第四章第十六節.
明年這個時候的最後兩個希伯來文字.
「以利沙對這個婦人所做.
應許她生兒子的一個神蹟」.
其實好像彷彿當年.
上帝應許亞伯拉罕.
他會生一個兒子的一個神蹟.
以利沙做這件事不是一件很平凡的事.
他的奇妙之處.
就好像當年上帝應許以色列的列祖.
怎樣可以生一個孩子一樣.
所以這是一個很特別的神蹟.
這個神蹟特別到一個地步.
當這個婦人聽到以利沙這樣說的時候.
婦人嚇得她顫抖地說話.
中文聖經翻譯不到她顫抖的感覺.
而文聖經說不是的.
我主啊 神人啊.
不要棄控你的婢女.
這個婦人她為了這個以利沙對她的應許.
她驚喜得突然之間.
不太知道可以怎樣去說話.
不過聖經告訴我們.
以利沙所應許這件事.
明年那個時候就實現了.

$^{201}$不過當這件事實現了之後.
突然之間急難就來到這個家庭.
十八至二十節.
孩子長大一日出去.
到他父親和修國的人那裡.
他對父親說.
我的求啊 我的求啊.
他父親對婦人說.
把他抱到他母親那裡.
婦人抱去交給他母親.
孩子坐在母親的膝上.
到中午就死了.
一個孩子得了一場急病.
他去找他的父親.
他父親說.
就好像一向做男人都一樣.
小孩子有什麼事去找他的母親.
於是婦人就抱他帶到他母親那裡.
半日之內.
這個本來還是沒有生命力的孩子.
他就死了在他母親的膝上.
一件急難,驟然來到這個婦人身上.
短短半日的時間.
他失去了這個孩子.
但是我們回想.
他本來其實不會經歷這件事.
因為本來他是一個沒有孩子的婦人.
他為什麼現在會失去了他的兒子.
正正是因為當日.
以尼沙他應許他可以生一個孩子.
這個苦難的起頭.
是當日以尼沙的一個善念.
如果不是有這個.
好像慧烈祖來應許有後代的神蹟.
臨到這個婦人身上的時候.
這個婦人她不會經歷這一次的苦痛.
所以婦人去到二十八節.
當她見到以尼沙的時候.
她和以尼沙這樣說.
她說我何嘗向過我的主去求個兒子呢?.

$^{241}$我何嘗不是和你說過.
你千萬不要騙我.
你給了我.
為什麼又要容讓這個事情臨到我身上.
以致我失去了我這個兒子.
不如你不要讓我也中紅.
這件事情臨到這個婦人的身上.
不過其實這件事情.
也帶來了這個婦人一些很特別的改變.
聖經作者他在描述這個婦人的時候.
其實剛剛讓我們去明白.
或者去體驗這個婦人是一個什麼性格的婦人.
我想先和大家再看回第十二至第十七節.
婦人她和以尼沙的一段對話.
第十二節這樣說.
以尼沙吩咐僕人基哈西說.
你叫這孫念婦人來.
他把婦人叫了來.
婦人就站在以尼沙面前.
和盤修訂版.
他就說這個婦人她站在了以尼沙的面前.
不過在原文聖經.
我們發現這裡其實沒有以尼沙的名字.
這裡只是說一個代名詞是她.
他把婦人叫了來.
婦人就站在他面前.
如果我以一個中文閱讀理解的題目.
去問問大家.
第十二節最後那個他.
其實應該是說誰呢?.
大概你就會說.
應該是說之前那個他.
之前那個他.
他把婦人叫了來.
又是誰呢?.
沒錯.
這個不是以尼沙.
這個是他的僕人基哈西.
所以當基哈西叫了這個婦人來的時候.
婦人她站在了哪裡呢?.

$^{281}$她站在了他基哈西的面前.
如果是這樣的話.
我們就明白下一節所發生的事了.
因為以尼沙就吩咐基哈西說.
你跟他說.
如果這個婦人已經站在了以尼沙面前的話.
其實以尼沙不需要再跟基哈西說.
你跟婦人說這句話.
正正因為婦人不是站在了以尼沙面前.
婦人她站在了以尼沙的僕人基哈西面前.
所以第十三節.
以尼沙才需要要求基哈西幫他.
來向這個婦人傳話.
這個婦人她和以尼沙保持了距離.
不單止一次是這樣.
當以尼沙第二次再叫婦人來的時候.
婦人站在了哪裡呢?.
第十五節.
「於是他叫了她來,她就站在門口」.
婦人也不是站在了以尼沙的面前.
她和這個神人一直保持相當的距離.
這是婦人在這件事發生之前.
她和這個神人所有所謂的一份關係.
除此之外.
我們更加可以看到這個婦人的另一個特性.
當以尼沙第一次叫基哈西問她.
「你費了這麼多心思來幫我們」.
「其實有什麼我可以為你做呢?」.
婦人的答案是.
「我已住在自己百姓之中」.
其實婦人的意思就是.
「我一無所求,我什麼需要都沒有」.
就好像大家回到每個禮拜的祈禱會的時候.
接著這個組長問你.
「你有什麼需要祈禱嗎?」.
你就說「感謝主,我這個禮拜過得很好」.
婦人在跟以尼沙說.
「我沒有什麼需要你幫忙」.
「我沒有什麼需要特別的祈求」.
婦人和以尼沙在這個位置上有一個距離.

$^{321}$婦人對於以尼沙要對他的善意.
她也保持了一種心理上的距離.
如果你再看看夏文的時候.
你又會發現一件事.
其實這個婦人後來這個兒子出世之後.
她也不時會繼續來探望以尼沙.
不過她有什麼時間去探望以尼沙呢?.
當她的兒子死了之後.
她就趕快去叫她的僕人去預備一匹驢.
然後跟著她和她的丈夫去說.
「我現在需要去探望這個神人」.
她的丈夫就回應了一句話.
「今天不是初一,也不是安息日」.
「你為何要到他那裡去呢?」.
從丈夫這句話我們大概可以明白到.
這個婦人一向去探望以尼沙是有一個規矩的.
就是初一和安息日.
她才會去探望以尼沙.
她不會無端白事地去打擾這個神人.
縱使這個神人為她做了一件特別的事情.
但這個婦人仍然是一個非常有規有矩.
和以尼沙保持著一個很規矩之間的關係.
這三點我們發現在這個急難之前.
這個婦人保持著和以尼沙在一個空間上的距離.
第二方面,她尋找以尼沙.
都是一個定時,定候,初一和安息日.
當以尼沙問她「你有沒有一些需要?」的時候.
婦人沒有向以尼沙發出任何的請求.
不過急難之後,這個婦人她有一些改變.
來到這裡,我想邀請你來做一件事.
我想你看看經文從第十八節開始到第三十七節.
這個聖經作者如何描述這個婦人.
至少在這三方面,她和以尼沙之間的距離.
她尋找以尼沙的時間.
和她有沒有向以尼沙發出她的請求.
我想你試一下在故事的下半部分.
這個婦人她有沒有一些改變.
我邀請你先停止這段錄影.
然後你拿一支筆嘗試看這段經文.
有沒有這三方面的不同改變.

$^{361}$你找到之後,然後你繼續播放下去.
我就會在下半部分說這個婦人有什麼改變.
給大家看.
我想當大家看完之後,你會發現.
其實當急難發生之後.
這個婦人的確她有一些東西是很不同的.
第一方面,我們剛才說過她和以尼沙的距離.
我想你在剛才那段時間.
你應該會找到她的一些改變.
在剛才那段時間你應該會找到.
經文二十七節描述了.
婦人上了山之後,到了神人那裡.
她就抱著神人的腳.
由剛才站在神人的前面.
第二站在神人的房間門口.
突然之間,這裡說的是.
婦人抱著了神人的腳.
抱著這個字其實是一個很特別的字眼.
因為一開始的時候,第八節.
還記得嗎?.
婦人強留了以尼沙在她家吃飯.
其實抱著和強留在希伯來文裡.
都是同一個字眼.
我們發現,由最開始.
這個婦人是一個她要去施禦.
她要成為一個接待以尼沙的人.
去到第二十七節急難來到的時候.
她成為了一個同樣的字眼.
但她成為了一個她需要求告.
請求以尼沙去幫忙的一個人.
第二方面,我們剛才提到.
婦人是定時定候才去找以尼沙.
不過,去到她的兒子死了之後.
我們發現,真的有改變.
其實二十五節都反映了這個改變.
對於以尼沙來說,都是很驚訝.
因為當婦人驅車來到加密山的時候.
神人遠遠看到她.
然後他就派他的僕人基哈西去跟她說.
「漢娜,書殿的婦人來了」.

$^{401}$「現在你跑去迎接她」.
「對她說,你平安嗎?」.
「你丈夫平安嗎?」.
「孩子平安嗎?」.
為甚麼以尼沙要基哈西走去問這個婦人.
「你平安嗎?」.
「你的丈夫平安嗎?」.
「你的兒子平安嗎?」.
其實,以尼沙正正心裡面.
他都感覺到.
這個婦人在一個這麼不尋常的時候.
來到找她.
一定是有些不尋常的事情發生了.
所以這個婦人.
她在開初之前是定時定候來找以尼沙.
當急難來到的時候.
這個婦人就再沒有按著這個規矩.
她就在一個很不尋常的時候.
她就驅車直往找以尼沙.
最後我們說到.
這個婦人在之前她跟以尼沙說.
「我已經住在我自己的百姓之中」.
「我一無所求」.
不過,去到後來.
我們發現這個婦人.
她都同樣有這個改變.
第28節.
婦人她第一次跟以尼沙說.
「我何嘗向我主求過兒子呢?」.
「我豈不是說過不要欺控我嗎?」.
不單止求.
她還埋怨以尼沙.
埋怨以尼沙為何要這樣.
讓她經歷這麼大的失去.
之後,以尼沙想拆敗這個婦人.
帶著她的手杖.
把手杖放在這個孩子的上面.
我想以尼沙大概的意思是.
「基哈西,你就跟隨這個婦人」.
「和她一起坐車回去吧」.

$^{441}$婦人明白以尼沙這個心意.
所以第30節她再一次跟以尼沙說.
「我只著永生的耶和華」.
「又只著你的性命起誓」.
「我必不離開你」.
婦人強烈向以尼沙表達.
「如果以尼沙你不走的話」.
「我一定不會走」.
婦人向以尼沙表達.
她這個很直接,很大的需要.
和她的哀求.
聖經作者在這裡很特別.
如果你看看.
這次聖經的作者.
如何提及這個婦人.
她如何稱呼這個婦人.
第30節她用了什麼字眼.
「孩子的母親」.
差不多這是這段經文裡.
唯一一次提及這個婦人.
說話或做事的時候.
是以一個孩子的母親的身份去做.
正正因為她這個身份.
正正因為這個事情.
令到這個婦人從以往大大改變了.
其實我們回想.
她沒有小朋友.
其實對她來說是一個很重要的困難.
基哈西說她的丈夫都年老了.
其實當這個丈夫年老過世之後.
這個婦人又沒有孩子.
其實她就成為了一個無依無靠的寡婦.
但是對於婦人來說.
她都不覺得這是一個問題.
她都不把這個需要放在以尼沙的面前.
直到她成為了一個孩子的母親.
直到她失去了這個孩子.
這件事情就帶來了她生命一個180度的改變.
聖經作者彷彿就是在等待這個婦人.
她這樣強烈地向以尼沙發出她的要求.

$^{481}$因為去到當這個婦人在30節.
她說完這句話之後.
接下來這個故事裡面.
婦人她一句話都沒有再說過.
整個故事彷彿就在等待這個婦人.
將她的需要.
將她所求的陳明在以尼沙的面前.
如果我們再想想.
其實什麼令到這個婦人經歷這麼大的困難的時候.
會有這麼大的改變.
除了她這個失去之外.
其實還有另一件事件成就了這個婦人這麼大的改變.
這個事件就是第27節.
以尼沙所說的話.
神人說:由他吧,因為他心裡受苦.
但耶和華向我隱瞞這事,沒有告訴我.
這句詔義聽起來沒有什麼特別的說話.
不過我們想清楚一點.
其實以尼沙她不知道發生了什麼事情.
直到這個婦人來到她面前.
然後以尼沙才知道原來這個小朋友出了事.
為什麼?.
就是因為上帝沒有告訴這個神人.
究竟發生了什麼事.
我們想像一下.
如果耶和華將這件事情透露給了以尼沙.
以尼沙以他和這個婦人的交情.
以他對於這個婦人的感激.
他一定會在第一個時間.
他就會和基哈西趕去這個婦人的家.
但正正是因為以尼沙不知道這件事情的發生.
以至婦人要自己在一個不是定時定候的時間.
來找到以尼沙.
正正是因為以尼沙不知道這件事情的發生.
以尼沙才會經歷到這個婦人.
經過這麼長一段時間.
驅車來到,心急如焚.
一見到以尼沙.
就抱著她的腳不放.
正正是因為以尼沙的不知道.

$^{521}$才帶來了這個婦人一個這麼大的改變的關鍵.
但是很諷刺.
經文說她是神人.
神人應該要將上帝的訊息向人去講.
但是神人在這裡竟然收不到上帝給他的訊息.
如果我們再看清楚一點的時候.
其實經文在第十八節到第三十一節.
其實他每一次去講以尼沙的時候.
都是強調她是神人.
你說不是啊.
我的聖經是翻譯了有以尼沙這個名字的.
譬如我們看第二十九節.
第三十節和三十一節.
不過其實這三節裡說的以尼沙.
都是和合本修訂版加上去的.
三次其實都是用他這個字眼.
在這段經文裡.
他五次都是以神人來描述以尼沙.
不過這位神人並不是我們想像.
能夠像剛才一樣.
做一個很偉大神跡的神人.
也不是一個所謂踏通天地線.
他明白完全收到上帝訊息的一個神人.
他既收不到訊息.
也面對這個孩子的死亡.
他什麼辦法都沒有.
我們繼續看下去.
第三十一節.
我們剛才說了.
以尼沙叫基哈西.
你快點去.
拿著我的手杖.
然後放在這個孩子的身上.
當我們後來知道基哈西.
趕過這個婦人和以尼沙.
然後嘗試將這個手杖.
放在這個孩子的身上.
不過結果是怎樣呢?.
三十一節.
他說沒有聲音.

$^{561}$沒有動靜.
以尼沙其實他叫基哈西.
將手杖放在這個孩子的面上的時候.
其實以尼沙心裡都毫無勝算.
以尼沙都不知道究竟是否有效.
不過實驗證明了.
原來以尼沙這一招是不行的.
之後唯有以尼沙親自出馬.
三十二節就說.
以尼沙進入屋裡.
我想看看經文怎樣描述.
以尼沙做了甚麼事情.
他伏在孩子的身上.
口對口 眼對眼 手對手.
然後又伏在孩子的身上.
你會問究竟口對口 眼對眼 手對手.
伏在這個孩子身上.
以尼沙在哪裡學了回來的一個祖傳的地方.
可以令一個死了的人再次復活呢?.
其實答案是.
我想對於以尼沙來說.
他都不太知道應該怎樣做.
他這樣做.
我想我們大概可能會有印象.
其實另一個人物.
他都曾經在類似的處境裡.
做過類似的事情.
那個人物就是以尼沙的師父 以尼亞.
當撒勒伐的寡婦.
她的兒子死了的時候.
然後在《獵王紀》上第十七章.
都描述到以尼亞三次伏在孩子的身上.
不過以尼沙進步了一點.
他再描述一點.
或者更加精確地.
他想孩子的口要對著孩子的口.
自己的手要對著孩子的手.
他的眼要對著孩子的眼.
不過這個祖傳地方.
以尼沙是否都很有把握呢?.

$^{601}$我們再留意.
當以尼沙去做之前.
他要向耶和華祈禱.
做了第一次伏在孩子身上的時候.
他感覺到孩子漸漸暖和.
不過他都沒有什麼把握.
他就在屋裡度來度去.
在想究竟有沒有其他方法.
可以真真正正.
可以讓這個孩子真的復活呢?.
不過他想不到其他方法.
於是第二次又再一次伏在孩子的身上.
我們在這裡見不到一個指揮約定的神人.
我們見不到一個很淡定.
很知道下一步應該怎樣做的先知.
我們見到的是一個.
面對這個孩子的死亡.
他都沒有什麼辦法.
也都不是很知道可以怎樣做的以尼沙.
可能你會問.
以尼沙這麼沒有把握.
為什麼當初他又會跟著這個婦人來度去呢?.
如果我們還記得.
以尼沙他決定要跟著這個婦人.
回到他家裡之前.
其實這個婦人她說了一句話.
第三十節我們剛才看過.
我指著永生的耶和華.
又指著你的性命起誓.
我必不離開你.
如果你對以尼沙的故事熟悉的話.
可能你會發現這句話好像有點面善.
沒錯.
這句話其實在獵王幾下的第二次.
當以尼沙知道他要被耶和華接走的時候.
他跟著三次都跟他的徒弟以尼沙說.
耶和華要差遣我去某一個地方.
你可以留在這裡.
三次.
以尼沙他都跟他的師父說.

$^{641}$我指著永生的耶和華.
又指著你的性命起誓.
我必不離開你.
當日以尼沙他堅決告訴以尼亞.
我必定要承繼你成為以色列的先知.
無論如何我都不會離開你.
這就是我回應上帝給我的呼召.
今天這個婦人在一個不知就裡的情況下.
她用了同樣的一句話.
她告訴以尼沙.
我指著永生的耶和華.
又指著你的性命起誓.
我必不離開你.
好像重提以尼沙.
他當日的夢照見證一樣.
以尼沙聽到這句話.
他就彈起身來.
雖然他不知道有什麼方法.
來面對這個急難.
但是他就跟著這個婦人去.
來回應上帝當日給他的呼召.
如果我們再詳細一點去看.
比較以尼沙和以尼亞的故事的時候.
你會發現其實兩師徒之間.
很多事跡都有所謂的對應的事件.
譬如一開始的時候.
以尼亞出場.
他就說他要令到以色列地不下雨.
然後之後第十八章.
雨就再一次下下來.
其實同樣在以尼沙.
他接過了以尼亞.
他這個所謂的先知的身份的時候.
他接下來做的事情.
也是同樣令到一個很乾旱的地方.
來到有雨來到下雨.
以尼亞他向亞哈王.
說了這個國家不會下雨之後.
然後他逃到薩納法的地方.
在那裡他在寡婦的家.

$^{681}$令到麵粉和油可以源源不絕地供應.
同樣的故事.
其實就正正發生在我們今天.
書唸婦人的這個故事之前的一個小故事.
一個所謂先知門徒.
他的遺孀.
他們家裡沒有錢還債.
按著以尼沙的吩咐.
他們將油不斷地倒出來.
說到令到一個小孩來到死裡復活.
我們剛才提過.
以尼沙,以尼亞他們同樣做了這個事蹟.
正正就在這件事件上.
一個以尼沙開初希望帶給這個婦人.
一個所謂報答她.
結果好心做壞事.
令到這個婦人經歷一個這麼大的困苦.
同樣一個以尼沙束手無策.
不知道應該怎樣去解決的困難.
在一個先知彷彿很無能,很無助的處境裡.
但是上帝就藉著這個事件.
讓到以尼沙她就和她的師父的侍奉.
來到一個圓滿的對應.
很諷刺吧.
一個彷彿我們看為好像都很失敗.
很不完美的一個處境.
但是那個就是上帝對於先知.
她的一個呼召,最終的一個確立.
今天我們看到的這件事.
是一個突然之間.
在一個人沒有預想過的情況之下.
困難來到.
但是正正在一個困難這樣來到的處境裡.
我們發現當中兩個故事的角色.
他們的生命都出現了一個很大的改變.
第一,一個婦人她本來生命有很多性格上.
或者她覺得和神人之間的關係.
甚至她和上帝之間的關係.
有很多的限制,很多的規矩.
很多彼此之間的隔閡.

$^{721}$但是因為這件事.
這個婦人她突破了這些距離.
她將心裡所願,所想,所要求的.
她坦白傾倒在上帝,在以尼沙的面前.
另一個人就是以尼沙.
在一段困難,急難的處境裡.
上帝最終給了這個先知.
他的侍奉,他的召命.
一個最終極的確立.
弟兄姊妹,或者今天.
我們都面對著各樣不同的急難.
但是當我們的焦點放在急難上的時候.
也許我們都可以找些時間.
看一看上帝怎樣在你自己.
在你身邊的人,在這個世界上.
他原來是藉著這些急難.
承受著一些在日常的時間.
我們不可能看到的改變.
心願我們今天面對我們的困難的時候.
我們除了見到困難帶給我們的痛苦之外.
我們更加可以有另一隻眼.
看到上帝在當中的作為.
我們可以看到上帝的真面目.
\newpage



\section{}
\label{sec:V3Cpa_Vt9fE}
\textbf{【疫有嘢學 │ 延SUN在線】疫境中的抉擇|代上21章|李思敬博士}
\newline
\newline
連結: \href{https://youtube.com/watch?v=V3Cpa-Vt9fE}{\texttt{ https://youtube.com/watch?v=V3Cpa-Vt9fE}} ~~~~ 語音日期: 2020-04-06 
\newline
\newline
\hyperref[sec:OmTXVUsNi_8]{\small{< < < PREV SERMON < < <}}
~
\hyperref[sec:index]{\small{[返主目錄]}}
~
\hyperref[sec:O9blI5PB1Ss]{\small{> > > NEXT SERMON > > >}}
\newline
\newline
$^{1}$各位電影姊妹主內平安.
真係冇諗過.
肺炎疫情係香港已經出現jor 超過兩個月ge 時候.
更加估唔到ge 係疫情ge 蔓延.
竟然會轉變為大流行.
我地大部分人亦都係因為咁樣.
而需要差唔多每一日都留係屋企ge 裡面.
明白係疫情ge 期間.
信徒仍然需要有學習上主說話ge 機會.
眾神延伸部嘗試去回應呢一方面ge 需要.
之前我地已經推出jor 延伸在線亦有得讀.
帝國霸權下ge 一神信仰.
二冊書28到39章釋義呢一個免費ge 課程.
讓到電影姊妹可以係4月6號至到5月3號呢段期間.
任何ge 時間裡面參與學習.
我知道報名ge 情況係相當之踴躍.
另一方面呢我地亦都會提供一個全新製作ge 課程.
就係大家依家參與緊ge 亦有ye 學延伸在線.
顧名思義呢一個課程就係係Sunday主日ge 裡面.
比大家藉住網上ge 平台去學習聖經.
課程首先就會由李院長披甲打頭陣.
為大家去講解呢歷代至上21章課題係逆境中ge 抉擇.
之後ge 幾個主日ge 課堂.
將會由忠臣新藥科ge 老師助理教授余振寧博士.
忠臣ge 副教務長舊藥科ge 老師副教授張志聰博士.
同埋小弟輪流負責ge .
盼望我地ge 服侍能夠讓到弟兄姊妹可以係主日ge 時間裡面.
繼續保持學習上主說話ge 習慣.
此外新一季ge 延伸課程亦將會係五月份開課.
鼓勵弟兄姊妹你地能夠走多一步.
走去了解有關ge 課程.
課程ge 詳細資料可以瀏覽忠臣ge 網頁.
亦都可以留意今日課堂之後ge 宣傳資料.
願主係疫症ge 困難當中.
藉住佢自己ge 說話.
勉勵同埋安慰大家.
主席.
停課嚴陣以待.
好似事情又開始有少少失控.
不過我收到一個年輕ge 醫生.

$^{41}$係月初ge 時候傳比我呢一幅ge 圖表.
就係話比我聽原來香港ge 流感季節.
忽然之間係2月1號就結束jor .
咁呢個點解感恩呢.
因為醫院因此可以得到緩衝喘氣.
病床冇咁緊張.
我有朋友前個星期入jor 伊麗莎白醫院.
發覺急症室幾乎一個人都冇.
上到病房佢話好好招呼.
呢個背後其實有上帝ge 保守帶領.
另外一個圖表呢.
就係Johns Hopkins美國流行病學ge 權威.
佢地早前發表ge 呢一個統計數字.
如果你睇斜gwo 條虛線.
就係如果每日增加三分之一ge 確診病例.
你就會睇見西方幾乎所有ge 國家.
包括伊朗都係呢一個平均數之上.
一之下就係四個亞洲ge 地區.
南韓,日本,新加坡同香港.
咁呢個係好感恩ge .
上帝恩代憐憫.
以香港一個人口咁密集ge 城市.
能夠守住過去呢兩個月ge 時間呢.
的確唔係理所當然ge 事.
所以呢我睇到呢幅相ge 時候.
我好感動醫護人員係前線.
值得我地表示衷心ge 敬意.
但係其實呢一場ge 抗疫.
唔係只係佢地係度做緊最緊要ge 事情.
有時我地都有咁ge 錯覺.
佢地做曬就得啦唔係ge .
呢個圍堵ge 政策.
呢個傳染病學ge 方法.
其實係我地留係屋企.
係一個好關鍵ge 做法.
所以唔好氣餒.
前面我地仍然有連場ge 硬仗.
需要弟兄姊妹懇切祈禱.
亦都要守規矩自律.
今日我地選擇討論ge 一張舊約經文.

$^{81}$係歷代至上第21章.
選擇呢一張經文ge 原因.
當然係因為呢一張ge 聖經提到瘟疫.
其實瘟疫呢個詞彙係聖經當中呢.
唔算經常出現.
呢個唔係聖經一個焦點.
事事講到唔係.
嚴格黎講可能你都會覺得好稀奇.
就係新約聖經27卷書裡面.
只有路加福音21章11節提過一次瘟疫.
再冇第二次.
而係gwo 處主耶穌係講到.
民攻打民 國攻打國.
多處必有地震 饑荒 瘟疫.
呢句說話唔係係符類福音.
馬太馬可都出現咩?.
冇錯.
馬太24章第7節.
馬可福音第13章第8節.
卻係arm arm 正好略去jor 瘟疫呢個字.
如果你心水再清呢.
你話我記得好似係啟示錄第6章第8節.
有提過瘟疫.
咁就真係要讀原文.
希臘文啟示錄第6章第8節.
其實gwo 個字係死亡.
中文英文ge 聖經翻譯.
點解咁大膽譯做瘟疫呢.
係因為舊約聖經屢次出現刀劍 饑荒 瘟疫ge 公式.
所以去到啟示錄ge 時候.
明明係死亡都翻譯或者理解為瘟疫.
舊約聖經提到瘟疫62次.
咁就好多囉.
如果我地減jor 其中28次.
差唔多一半係重複.
剛才我講呢個刀劍饑荒瘟疫之外呢.
其實剩返ge 經文呢真係唔算係好多.
我地選擇歷代至上呢.
係因為呢一章ge 經文係舊約聖經記載.
三次瘟疫ge 事件ge 最後一次.

$^{121}$前兩次都係出現係文掃記.
今次係大衛數點百姓.
而呢一章ge 經文呢.
又同時係薩姆爾記下24章重複出現.
好似舊約Synoptic負雷福音咁樣平衡ge 記載.
出現在舊約ge 歷史書.
薩姆爾記下24章係薩姆爾記總結大衛生平ge 結論.
佢竟然用jor 呢一件事情.
黎到總結大衛ge 一生.
歷代至就唔係啦.
歷代至上21章係記載大衛ge 立國ge 過程當中.
就發生呢一件事.
所以如果我地只係讀薩姆爾記呢.
我地可能有個錯覺.
以為大衛係係佢晚年ge 時候.
先至犯jor 呢個錯誤.
點解呀.
因為薩姆爾記將呢件事情擺到最後.
擺到最後唔係gwo 個時間先後次序ge 取捨.
而係作為一個總結.
作者係薩姆爾記ge 作者.
將大衛呢一件事同素羅.
另外一次又係犯jor 錯誤.
帶黎國家三年ge 饑荒相提並論.
咁就比我地見到.
作為一國之君.
作為領袖人物.
冇個人ge 得失.
其實佢ge 選擇呢.
係難免影響佢身邊好多ge 人.
呢個係薩姆爾記比我地見到ge 總結.
歷代至上21章又講乜ye 呢.
我地一打開呢一章ge 經文呢.
立即就會碰到第一個ge 難題.
如果你唔熟識呢一章ge 聖經呢.
或者你可以撳停jor 呢一個ge 廣播先.
拎出聖經讀一次歷代至上21章.
我地就唔會一齊黎到去讀啦.
歷代至上ge 開場白話比我地聽.
係撒旦起黎攻擊以色列人.

$^{161}$係撒旦激動大衛去掃點百姓.
但係比較薩姆爾記下24章ge 開場白呢.
卻係好清楚提到係耶和華激動大衛去掃點百姓.
難題就係究竟舊約聖經講乜ye 架.
究竟係耶和華定係撒旦呀.
如果我地停落黎平心靜氣問一問清楚呢.
我地就發覺咁樣問問題呢就得唔到答案.
因為你只能夠選擇係咪.
如果係就歷代至ge 記載係正確.
係撒旦搞鬼.
如果唔係ge 話呢就薩姆爾記ge 記載係正確.
唔關撒旦事.
後來唔知點樣賴jor 落佢個樹.
無論點樣睇都好呢.
兩段ge 經文唔可能都arm 架喎.
所以捉住呢D ge 難題呢.
我地係冇辦法能夠搵到答案.
因為呢D 論就只能夠話聖經有錯囉.
有矛盾囉.
有出入囉.
所以呢學習去明白聖經第一件事出發點.
我地需要謙卑.
我地需要退一步重組我地ge 問題.
即係唔係我地去問聖經.
係聖經問我地.
係我地發出個問題背後ge 假設有錯誤.
唔係聖經有錯.
千祈唔好搞錯.
所以邊個激動大衛數點百姓架.
答案其實唔係歷代志或者薩姆爾記.
答案係舊約聖經另外一段好出名ge 經文.
gwo 段ge 經文就係全本新舊約聖經唯一一段.
係同時提到撒旦同耶和華.
你就即刻記起啦.
係邊一段呀.
約伯記第一章第二章.
要明白有關耶和華同撒旦.
我地唔可以從民間宗教.
甚至如果容許我咁樣講.
神學ge 理論出發.

$^{201}$因為係我地ge 文化當中正邪不兩立.
所以楚河漢界上帝係天堂.
撒旦係地獄各據一方.
冇可能佢地之間有任何瓜葛.
呢個係徹頭徹尾.
係哲學係宗教係文化上面ge 二元論.
即係有光必有暗.
有黑必有白.
善良必有邪惡.
舊約聖經對唔住.
從來都不吃這一套.
我地可以咁講.
從來都唔係咁講.
舊約聖經只有獨一ge 真神耶和華.
約伯記點樣介紹上帝.
佢坐係寶座上面.
然後撒旦係邊處呀?.
撒旦唔係係對岸搖搖相拒.
咁樣罵陣.
撒旦係事立係耶和華寶座前ge 其中一員.
當然呢度約伯記用緊ge 比喻.
係我地都好熟悉ge 一個朝廷.
宮廷.
我地講天廷ge 比喻.
係天上有寶座.
有文武百官事立.
另外一段好出名ge 係列王記.
米該雅先知.
睇到係耶和華寶座前.
有神靈出黎獻祭以謊言.
引誘阿哈王上陣.
不過約伯記講得好清楚.
睇完之後你一定唔會忘記.
撒旦係耶和華面前必恭必敬.
耶和華問佢你從哪裡來往哪裡去.
撒旦係要回覆.
換一句說話講.
係舊約聖經ge 神學裡面.
從來都唔係耶和華同魔鬼對立.
你話新約係.

$^{241}$古蛇,龍,撒旦,魔鬼,啟示錄.
你留心啟示錄gwo 處講jor 一句說話.
係我地必須要掌握.
弟兄勝過撒旦係乜ye ?.
係因為羔羊所流ge 血.
同埋教會所見證ge 真道.
唔係一個神話ge 世界.
魔鬼撒旦從來都冇上帝ge 權柄能力.
呢個係我地係聖經明文ge 記載裡面.
我地必須學習ge 真理.
所以其實撒姆爾記同歷代寺ge 記載並無衝突.
正如我地要問約伯壽虎究竟係出於撒旦.
抑或出於耶和華.
你ge 答案係乜ye ?.
大概唔使我教你.
你都識得答.
呢個唔係一個選擇.
呢個係同時發生ge 事情.
最後當然係上帝掌管撒旦係佢面前都要得到批准.
先至可以動手去對付約伯.
今次歷代寺ge 記載同撒姆爾記ge 記載.
我地都可以用約伯記去幫忙解釋.
咁就唔需要係呢個難題上繳錢不清.
可以繼續.
數點百姓做錯jor D 乜ye ?.
可能呢個係我地查考聖經出現ge 第二個問題.
摩西係文素記都兩次數點以色列百姓.
又唔見帶黎乜ye 上帝ge 刑罰.
係,我地要好誠實.
因為歷代寺又好撒姆爾記又好.
並冇交代點解大衛咁樣.
吩咐數點百姓係得罪上帝.
係犯jor 錯.
不過兩段ge 記載.
都同樣話比我地聽.
約克勸jor 大衛唔好做呢件事.
約克是誰?.
約克係大衛軍隊ge 元帥.
約克係忠於大衛ge 一個軍人.
約克殺人唔眨眼.

$^{281}$你讀撒姆爾記.
你就明白.
佢為大衛除去jor 幾多個敵人.
包括烏尼亞.
約克係一個咁樣ge 魔頭.
係一個粗人.
原諒我咁講,你明我意思.
佢唔係一個文士.
佢唔係一個神學家.
佢唔係一個牧師.
佢唔係祭師先知.
但佢都知道唔應該做呢件事.
所以佢力勸大衛唔好咁做.
免得使百姓陷在罪中.
咁我地就需要接受經文ge 交代.
已經足夠.
即係任何ge 讀者.
讀完約克ge 說話之後.
都應該明白.
連佢都知道呢件事唔可以做.
大衛卻一意孤行.
歷代至上記載.
約克都冇執行大衛ge 命令.
留jor 利美人,綿瓦敏人冇數在其中.
後來歷代至上27章.
再補充記載同一件事.
亦都好清楚交代jor .
有兩個支派.
迦德和阿徹.
並無係數點百姓ge 支派名單當中.
所以其實大衛係有選擇.
我地唔好又係以為.
宿命論.
今日提D 比較深D ge 字眼.
因為主日學.
唔係主日崇拜港獨.
二元論.
我地需要抗拒.
宿命論.
亦都唔係舊約聖經所贊成.

$^{321}$即係上帝激動大衛.
殺彈出黎攻擊.
大衛死緊啦.
冇得走啦.
認命啦.
一定錯啦.
上帝掘個坑.
你仲唔一腳踩落去.
唔係喎.
經文arm arm 相反.
大衛聽完約翰ge 說話之後.
其實佢有得揀架喎.
記得那八ge 事件.
大衛點起佢ge 勇士.
要去殺害那八ge 時候.
阿比蓋係路上截停大衛.
大衛結果懸崖立馬.
大衛可以唔咁做ge .
不過佢選擇jor 一意孤行.
呢個係經文好簡單ge 記載.
大衛冇聽約翰ge 勸阻.
但其實佢係可以停手.
佢係可以唔做呢件.
佢自己跟住講落去.
係第七節第八節.
上帝唔喜悅呢件事情.
降災俾以色列人.
大衛禱告神說第八節.
我行者是大有罪了.
因為我行的甚是愚昧.
講遲jor 啦.
做jor 之後先講.
應該未做之前就識得迴轉.
呢個係經文頭三分一.
俾我地見到事情ge 開始.
坦白講未入正題.
瘟疫都未曾發生.
經文跟住講落去ge 記載.
就係耶和華吩咐先知迦德.
黎到見大衛.

$^{361}$呢段記載又係.
全本新舊約聖經六十六卷.
從來未曾發生過.
前無古人後無來者.
上帝叫先知同大衛講.
唔係拿丹先知gwo 一次.
唔好撈聯華.
先知黎親就指住你個鼻哥鬧.
你就是賴人.
唔係唔係唔係.
今次係大衛認罪之後.
佢承認佢所行ge 係愚昧.
先見就黎到話俾大衛聽.
耶和華如此說.
大衛豎起耳仔聽.
上帝點樣聽我祈禱.
我有三樣災難.
隨你選擇一樣.
我地聽得多童話故事.
就覺得係啦.
次次神仙出現.
主角就一定有三樣ye 可以揀.
呢個係童話故事.
聖經從來無咁講.
你從創世紀讀到啟示錄.
上帝幾時有俾人揀.
你做錯jor 啦.
我俾你揀啦.
加德係呢處講咩.
三年ge 饑荒.
三個月ge 刀劍.
三日ge 瘟疫.
有得揀架咩.
俾你會點樣揀.
你係大衛.
你聽到先知傳達上帝ge 回覆.
你心裡面你點樣作決定.
饑荒係咪.
我地搵藥室.
先積穀防饑.

$^{401}$咁就解決難關.
太遲囉.
一係你就平時儲定啦.
否則ge 話.
邊有即刻可以有足夠ge 糧食.
而且饑荒係一年疊一年咁發生落去.
刀劍有藥壓係處幫手.
大衛ge 勇士.
如果出於上帝ge 話.
咁就連大衛ge 軍隊.
都一定兵敗如山倒.
瘟疫其實都可以搵大祭司.
記得亞倫昔日.
拎住香爐.
企係死人活人中間.
佢ge 子孫後代.
仲係我地中間.
好似菲尼蝦.
係巴力皮爾gwo 場ge 瘟疫.
佢為耶和華大發熱心.
總有辦法去解決.
作為領袖.
大衛係咪即刻召開內閣會議.
選擇點樣應對上帝比佢揀ge 後果呢?.
經文就跟住記載大衛對迦德說.
十三節.
大衛點樣講.
我甚為難.
我願落在耶和華ge 手裡.
我不願落在人ge 手裡.
大衛有冇揀到?.
經文跟住就話.
於是耶和華降瘟疫.
以色列人死jor 七萬.
我地好快就假設.
大衛係揀jor 瘟疫.
佢落係耶和華ge 手裡.
饑荒刀劍瘟疫.
瘟疫就係上帝ge 天災.
其實饑荒都唔係落係人ge 手中.

$^{441}$都係天災.
無論係因為大旱.
定係因為蝗蟲.
咁樣睇.
其實刀劍都可以出於上帝.
唔一定係因為人.
我地開始ge 時候提過.
係舊約先知書.
耶利米跟以西傑兩本書裡.
重覆jor 28次.
刀劍,饑荒,瘟疫.
聽到嗎?.
三個選擇.
其實係舊約聖經睇黎.
都係我地會遇到ge 災難.
唔一定有特別ge 原因.
有時我地就好鍾意.
拎住一段經文.
就去尋根究底.
今次係大衛犯罪.
上次係邊個犯罪.
我地同時都要記得.
主耶穌曾經係福音書.
兩次提醒佢ge 門徒.
路加福音13章第1至第5節.
唔係因為gwo D 遭災難ge 人.
比gwo D 冇遭災難ge 人更有罪.
你地若不悔改.
都要如此滅亡.
呢個係路加福音13章主耶穌ge 說話.
約翰福音第9章1至3節.
你會有印象.
生來核眼ge 人.
門徒就問.
係佢父母犯罪.
還是佢本人犯罪.
生來核眼.
即係未出世就已經犯罪.
唔可能.
父母犯罪.

$^{481}$直到三,四代.
留心主耶穌ge 回答.
耶穌話唔係佢父母犯罪.
亦都唔係佢本人犯罪.
係要係呢件事上.
展出神的榮耀.
歷代至上第21章ge 敘述.
似乎唔係停留係追究大衛ge 責任.
而係將gwo 個經文ge 焦點.
係轉移到大衛呢一個ge 選擇.
揀邊樣.
揀刀劍.
揀饑荒.
定係揀瘟疫.
如果我地將呢三件事.
放返係舊約聖經ge .
上下文.
耶利米書.
以西傑書.
有呢個背景.
歷代至.
呢個記載.
呢個故事可能好短.
但係都係係以色列當中.
流傳ge 一個事實.
我地就明白到.
其實三件事無乜分別.
講得唔好聽D .
揀邊樣都死.
識解呀.
講得唔好聽D .
都係大災難.
主耶穌提到地震.
饑荒.
瘟疫.
背後ge 意思.
啟示錄提到.
饑荒.
地震.
刀劍.

$^{521}$死亡.
大災難.
其實.
冇乜好揀.
上帝.
玩大位.
冇得揀.
比佢揀.
咁我地就要.
明白.
上帝ge 心.
即係你捉住一件事情.
你好容易會解錯.
我成日都記得.
我arm arm .
返黎香港.
1983年.
住係沙田ge 村屋.
樓下係一個小康之家.
爸爸做貨櫃車司機.
阿媽好錫個仔.
九歲大.
日日一路煮飯.
一路同個仔溫書.
個仔一唔生性.
阿媽鬧.
唔止呀.
打.
呢個仔嗌到好大聲.
阿媽.
你打死我啦.
我地住係樓上.
聽到.
我地笑ge je .
阿媽錫到個仔窿.
點會打得死佢呀.
呢個仔.
嗌得好大聲je .
係.
我地先要明白.

$^{561}$阿摩斯書第四章.
係舊約聖經先知書.
另外一次提到.
瘟疫.
阿摩斯書gwo 書係點講.
先知唔係淨係提.
瘟疫.
先知提到好幾樣ge 災難.
阿摩斯書第四章.
第六節.
第一樣ge 災難.
上帝話我洗你地.
係一切ge 城中.
牙齒乾淨.
係你地各處糧食缺乏.
你地仍不歸向我.
這是耶和華說的.
第二.
係修國前三個月.
我洗雨水亭子.
不降在你門那裡.
我降雨在這城.
不降雨在那城.
這塊地有雨.
那塊地無雨.
無雨ge 就枯乾了.
兩三個城ge 人.
湊到一個城去搵水.
搵唔到渴不足.
上帝話你地仍不歸向我.
這是耶和華說的.
第九節.
上帝話我以旱風霉難攻擊你地.
你地許多ge 菜蔬葡萄樹.
無花果樹橄欖樹.
被剪蟲.
大概係蝗蟲ge 意思.
所吃.
上帝話你地仍不歸向我.
這是耶和華說的.

$^{601}$然後.
我降瘟疫在你門中間.
像埃及一樣.
用刀殺戮你地ge 少年人.
洗你地ge 馬匹被擄掠.
刑中廝守ge 臭氣撲鼻.
上帝話你地仍不歸向我.
這是耶和華說的.
最後.
我傾覆你門中間ge 城邑.
如同我從前傾覆索多瑪,.
俄摩拉一樣.
洗你地好帳從火中抽出黎ge 一斤柴.
你地仍不歸向我.
這是耶和華說的.
你一定聽到.
先知重覆上帝最多ge gwo 一句說話.
係講乜ye ?.
歸向我.
所以.
上帝ge 心腸.
唔係迎罰.
唔係趕盡殺絕.
唔係打落十八層地獄.
永世不得翻身.
針研第三章.
好出名ge 金句.
我兒不可輕看耶和華ge 管教.
上帝好似一個慈愛ge 父親.
針研呢一句金句.
精彩ge 地方.
就係上帝ge 管教唔係出自佢ge 公義審判.
係出自佢ge 愛.
佢會鞭打佢收納ge 兒子.
希伯來書引用七十時ge 翻譯.
係父親對兒子ge 管教.
呢個管教只有一個目的.
就係讓百姓識得回轉.
所以.
你睇到大衛ge 回覆嗎?.

$^{641}$大衛揀jor 乜ye ?.
大衛無係三個災難當中.
選擇任何一樣.
因為佢明白.
其實邊一樣都係相同ge 結果.
無一樣係好事.
但佢選擇落係耶和華ge 手裡.
因為佢有豐盛ge 憐憫.
佢唔願意落係人ge 手裡.
大衛揀jor 佢不離開耶和華.
或者再講白D .
大衛俾神去揀.
佢唔會操控上帝.
因為神有豐盛ge 憐憫.
呢個就係大衛選擇ge 基礎.
17年前.
係沙田威爾士醫院試藥.
3月19日.
一群前線ge 醫生.
感到最無助乏力.
開始每日早上巡訪前的祖禱會.
沈祖堯醫生係教會弟兄.
見到面先知道原來你都係基督徒.
點樣祈禱呢?.
沈醫生見證《星島日報》3月31日刊出.
向神認罪.
承認原來自己所知道.
所能做到ge 都好有限.
承認我地ge 驕傲其實幾無知.
過去習慣依賴ge 係自己ge 知識.
忽略jor 真正醫治ge 係上帝.
於是我地將醫治ge 方法.
將大家ge 擔心憂慮.
將種種ge 壓力.
交託係上主ge 手中.
頭三次祈禱會.
到有人忍唔住喊.
比神揀.
因為佢有豐盛ge 憐憫.
呢個係大衛ge 選擇.

$^{681}$大衛第一個選擇錯jor .
約伯冇錯.
約伯係撒旦ge 攻擊之下.
佢話賞賜ge 係耶和華.
收取ge 係耶和華.
耶和華ge 名始終係應當稱頌.
大衛錯jor .
但係跟住佢再有機會揀.
佢話我情願落在神的手裡.
當然我地最緊張ge 問題.
可能唔係聖經呢一個最核心ge 選擇.
我地始終都要追問.
瘟疫到底點樣可以停止呢.
冇錯.
歷代至上如果只係停係上半章.
咁我地就停係大衛ge 選擇.
不過冇.
我地又必須要面對事實.
作者用jor 下半章.
從第十四節一路去到三十一節.
記載整件事瘟疫如何停止.
一定係祈禱認罪.
大家約埋係同一個時段一齊祈禱.
咁就能夠消災解難.
唔係太平清照.
唔係颱風襲港.
於是弟兄姊妹就跑去海灘.
就吩咐風ge 方向要轉移到台灣.
唔好黎香港.
呢D 係民間宗教祈福.
控制上帝.
唔係,你聽到嘛.
大衛係信任上主.
大衛選擇謙卑.
認罪等候.
其實經文記載.
唔係因為大衛所做ge 選擇.
或者佢ge 祈禱.
而係耶和華後悔.
瘟疫一日之後.

$^{721}$以色列人死jor 七萬之後就停止.
十五節.
神差遣使者去滅耶路撒冷.
剛要滅ge 時候.
耶和華看見就後悔.
不幹這災.
吩咐滅城ge 天使說.
夠了,駐守吧.
那時耶和華ge 使者.
站在耶布斯人.
阿爾蘭ge 和場那裡.
上帝後悔.
又係一個舊約神學ge 難題.
我地又係呢一處繳錢不清.
又讀唔到經文想要講比我地聽ge 主題訊息.
係呀.
舊約聖經.
一而再.
再而三提到.
耶和華係可以後悔.
第一次出現.
春埃及記三十二章.
第十二到第十四節係金牛毒ge 事件.
若爾書第二章.
同若拿書第四章兩卷十二仙之書.
都引用春埃及記ge 典故.
話比我地聽.
耶和華有恩典有憐憫.
不輕易發怒.
且有豐盛的慈愛.
並且後悔不幹所說的哉.
耶和華ge 後悔.
其實同我地ge 懊悔.
悔改.
唔係同一件事.
雖然動詞係希伯來文相同.
悔轉.
如果我用另一個翻譯.
或者可以幫助弟兄姊妹.
快少少去掌握清楚聖經ge 意思.

$^{761}$耶和華漢堅就悔心轉意呢.
意思即係佢唔係一個定律.
一個因果的定律.
人犯罪做錯事.
種下惡因.
就必收惡果.
果無人能夠攔阻.
咁先叫做定律.
所以上帝係公義的.
因果定律.
呢個唔使聖經講.
好多其他宗教都解釋得淋漓盡致.
舊約聖經要告訴我們.
正正就是耶和華上帝.
在祂傾福所多瑪俄摩拉之前.
祂都特登去找阿伯拉罕.
告訴他要做什麼事.
並且容許阿伯拉罕.
跟祂講價.
不是講公義.
審判全地的主豈不施行公義.
無問題.
阿伯拉罕在那裡求情.
如果有十個疑人.
你會不會毀滅.
創世紀的經文.
好爽快.
上帝每一次都說.
好.
我就不會毀滅.
我常常都想.
如果阿伯拉罕問到最後.
一個都沒有.
上帝你可不可以都不毀滅.
你估上帝會怎樣答呢.
其實上帝答了的.
沒有疑人.
連一個也沒有.
耶穌基督在我們還作罪人的時候.
為我們死.

$^{801}$就顯出神的愛.
羅馬書特別加上.
就在此向我們顯明.
所以這裡所講的.
上帝回心轉意.
他看到大衛的選擇.
他看到百姓的苦難.
他吩咐天使救了.
駐守吧.
但其實事情未曾停止.
上帝的回心轉意.
沒錯.
是整個瘟疫的轉折點.
你叫它做拐點.
還是轉裂點.
不要緊.
不過那是最重要的關鍵所在.
但是跟著經文下半章記載.
大衛做了一連串的行動.
他不單止祈禱認罪.
在前面第八節.
他不單止回應先見迦德提出的選擇.
他說我願落在耶和華的手裡.
在第十七節.
大衛見到滅命的天使手裡.
有拔出來的刀.
站在天地之間.
在耶路撒冷之上.
大衛和長老身穿麻衣.
面伏於地.
大衛再次祈禱.
其實他正在等待.
否則怎會有長老身穿麻衣.
在那裡呢.
大衛不是在皇宮裡睡覺.
我倚靠上帝.
我交託上帝.
上帝你做一切.
我照舊可以去飲茶.
我照舊可以在社區走來走去.

$^{841}$不是.
他嚴陣以待.
大衛和眾長老身穿麻衣.
俯伏在地.
然後耶和華的使者又再吩咐迦德.
去告訴大衛.
叫他在這個和場上.
為耶和華祝壇獻祭.
大衛很忙碌.
大衛很多事要做.
到底做這一切有沒有用.
有沒有幫助.
如果沒有幫助.
做和不做有沒有分別.
我們就是這樣實用主義.
我們要有用才肯去做.
看不到有用.
我們就不做了.
不是的.
猶太的聖經學者提醒我們.
在上帝要消滅邪惡的歷史當中.
他樂意讓人參與.
這是沒有用的.
是上帝做完的.
不用我們.
回到為父的心腸.
或者我們再進一步明白多一點.
我也做過父親.
當我的孩子還小的時候.
他幫我一起收拾辦公室.
搬那些很重的書.
其實真的他幫忙嗎?.
不是.
其實他有沒有用?.
阻礙著.
不過作為一個父親.
你明白吧.
他參與.
他說我不去打千秋.
今早我要幫爸爸在辦公室撿書.

$^{881}$你明白天父的心腸嗎?.
大衛連串的行動.
其實坦白說.
如果我們細心讀經文的記載.
你會發覺是滅命的天使拔出刀.
站在和牆上等待他.
在第十六節.
講這個使者站在那裡.
伸出手裡的刀.
在耶路撒冷之上.
當大衛最後獻完祭.
有火從天降在壇上.
然後二十七節說.
耶和華吩咐使者.
使者就收刀入朝.
特意等他.
上帝等我們.
在這個過程當中.
你選擇什麼?.
你做什麼?.
各人有各人的崗位.
前線的醫生護士.
他們信任上帝.
因為在神的手裡.
有豐盛的憐憫.
於是他們自告奮勇.
進去ICU.
Dirty Team.
我教會也有年輕的姐妹.
是這樣選擇.
不回家過夜.
不回家吃飯.
信得過上帝的保守.
不同的崗位.
信得過上帝的保守.
我沒有崗位.
我只不過在家裡.
信得過上帝的保守.
我們各人有我們需要做.
需要選擇的關鍵.

$^{921}$求主幫助我們.
不過我們必須要最後看到.
歷代至上21章.
還有總結.
就是當耶和華的使者.
收刀入朝之後.
還有28到31節.
這四節的經文.
你說這個是Appendix.
附錄來的.
沒什麼重要的.
不需要的.
這是歷代至的作者.
九尾族雕.
不是的.
你這樣讀.
起碼要謙卑.
或者虛心問一問.
或者這個就是整章經文的結論.
高潮壓軸.
最後四節說什麼.
最後四節就是停在大衛的結論.
大衛說.
這就是耶和華神的殿.
為以色列人獻梵濟的壇.
上文夏利告訴我們.
耶和華神沒有容許大衛建造聖殿.
這是所羅門要做的.
但是大衛在歷代至的記載.
他想盡千方百計.
為耶和華殿的工程.
預備物料,財富,人手.
只是不開工.
上帝不讓我開工.
沒有禁止我預備.
所以來到這一章大衛的經歷.
就沒有好像薩姆爾記下二十四章那樣.
只是停留在他作為一國之君.
做錯了決定.
帶來給百姓一個嚴重的後果.

$^{961}$不是.
歷代至告訴我們.
大衛所經歷上帝手中豐盛的憐憫.
就成為了百姓日後可以來到聖殿.
來到祭壇.
同樣經歷上帝豐盛的憐憫.
猶太聖經學者告訴我們.
在古代近東所有的聖殿.
其實都是生人勿近的.
那裡是供奉神明.
獻祭讓他們吃得飽開開心心.
就不會帶來給我們災難.
所以聖殿不是普通平民老百姓.
可以隨便來的地方.
只有祭司可以出入.
但是舊約聖經所記載的聖殿.
卻是萬民禱告的殿.
所羅門建好聖殿之後.
歷代至下第七章十三十四節.
很出名的金句.
幾乎不用打開聖經也會背.
上帝對所羅門說.
我的百姓如果懂得謙卑.
禱告尋求我的面.
轉來他們的惡行.
我必定從天上垂聽.
赦免並且醫治這個地.
聖殿是什麼地方.
我們今天只懂得想.
如何找到真正建聖殿的地址.
那個地址不是大衛所說的.
聖殿是一個怎樣的地方.
聖殿是一個禱告耶和華的地方.
聖殿是一個經歷上帝豐富憐憫的地方.
聖殿是一個迴轉認罪謙卑禱告的地方.
我們是否需要回到耶路撒冷呢?.
不要忘記.
保羅在羅馬書第八章告訴我們.
其實我們在耶穌基督裡.
就正正是一個這樣的應許.

$^{1001}$一個這樣的事實.
保羅問誰能使我們與基督的愛隔絕.
不是要去到聖殿才一無所隔.
而是在耶穌基督十字架寫生留血的事實裡.
我們就明白無論是死是生.
都不能叫我們與神的愛隔絕.
患難,瘟疫,刀劍,饑荒呼喚我們迴轉.
不是呼喚上帝迴轉.
上帝從來沒有離開.
所以我們喜歡在這時候呼喚.
上帝去了哪裡?.
問錯了,其實是我們去了哪裡?.
我們的自我中心.
我們的驕傲.
我們的自以為是.
我們指責別人.
這些令我們與上帝的愛.
祂的平安.
祂豐盛的憐憫隔開了.
要迴轉的可能是我們.
不是上帝,上帝沒有離開.
上帝豐盛的憐憫.
由始至終都在這裡.
當然,頭一堂的查考聖經.
我們不可以處理所有的問題.
所以還有下一個主日.
四月份我們還有其他的經文.
可以互相補充前後的呼應.
今天我們就停在歷代至上二十一章.
我們問逆境中的抉擇.
你選擇甚麼?.
我們與上帝的關係.
是否只是指上談兵?.
沒事的時候,虛虛慢慢.
高高興興,快快樂樂.
一有甚麼事情發生.
上帝呀,你去了哪裡?.
求主幫助我們.
將祂的說話.
恭敬,藏在我們的心中.

$^{1041}$在網上,我不和大家一起祈禱了.
我想邀請大家.
有一段短短安靜密討的時間.
你親自和主說出你心裡的心聲.
感謝主,我們今天就停在這裡.
(字幕製作:貝爾).
(字幕由 Amara.org 社群提供).
\newpage



\section{}
\label{sec:O9blI5PB1Ss}
\textbf{【疫有嘢學 │ 延SUN在線】疫境中遇見復活主|約20章11-18節|余振陵博士}
\newline
\newline
連結: \href{https://youtube.com/watch?v=O9blI5PB1Ss}{\texttt{ https://youtube.com/watch?v=O9blI5PB1Ss}} ~~~~ 語音日期: 2020-04-12 
\newline
\newline
\hyperref[sec:V3Cpa_Vt9fE]{\small{< < < PREV SERMON < < <}}
~
\hyperref[sec:index]{\small{[返主目錄]}}
~
\hyperref[sec:AHC7wVTdk1o]{\small{> > > NEXT SERMON > > >}}
\newline
\newline
$^{1}$各位主內的弟兄姊妹平安.
在復活節的主日裡面.
相信大家已經參加過自己教會裡的網上崇拜了.
歡迎你們再次來到.
「亦有ye 學 延伸在線」的課堂裡面.
在上一個星期裡面.
李思敬院長已經為我們講解了歷代至上21章的經文.
這個主日的課堂延伸部邀請了余振寧博士為我們去教授.
約翰福音二十章十一至十八節.
題目是「逆境中遇見復活主」.
余振寧博士Kelvin是忠臣的新藥科助理教授.
他是忠臣和愛丁堡大學合辦的博士課程的第一位畢業生.
他的研究興趣包括新藥的牙角書和典外文獻與新藥背景.
就讓我把以下的時間都交給Kelvin.
求主都幫助大家在心思意念上高.
抓緊上主說話的應許.
弟兄姊妹.
今年的復活節很特別.
因為對於很多地方的弟兄姊妹來說.
今年我們不能夠回到我們平時聚會的地方.
一起去舉行復活節的崇拜.
可能我們因著疫情的緣故.
被迫留在家裡看崇拜的傳播.
不知道這個會不會讓你有少少覺得有一些失落.
好像錯過了教會一個很重要的日子一樣.
的確是的.
我們知道自從早期教會開始.
復活節已經是一個相當重要的節期.
早期教會有一個傳統.
他們會在復活節那一天來施行洗禮.
而且在洗禮之前.
從早期教會的文獻我們看到.
那些即將接受洗禮的信徒.
他們會有一至兩天的時間來禁食.
不單止他們.
教會裡面其他的弟兄姊妹.
如果他們情況許可的話.
都會和這些即將要接受洗禮的信徒一起禁食.
他們選擇在復活節的日子進行洗禮.
甚至隆重其事地在洗禮之前來禁食.

$^{41}$這代表著他們很重視在這一天舉行的洗禮儀節.
因為洗禮就是代表著這些受洗的信徒.
他們進入到一個新的生命當中.
借用保羅的說話.
「現在活著的不再是我.
乃是基督在我裡面活著」.
在復活節施行洗禮.
很明確地象徵著信徒在洗禮當中.
和復活的耶穌基督來到去聯合.
去分享復活主的生命.
既然復活節是教會傳統當中一個這麼重要的節日.
在今天這個復活節的日子.
我也想和大家再一次思想一下.
耶穌基督的復活對我們到底有什麼意義.
在聖經裡面我們看到耶穌的復活.
對我們有一些相當重要的意義.
第一.
耶穌的復活表明了.
祂有那個勝過死亡的大能.
這個復活的大能.
也成為我們每一個跟隨主的人.
我們的盼望.
我們知道我們在基督裡面.
同樣能夠勝過死亡.
我們有這個復活永生的盼望.
雖然我們肉身會經歷死亡.
但我們都同樣會經歷復活.
進入那個永恆的生命.
另外耶穌基督的復活也表明.
天國已經降臨到地上.
已經進入到歷史當中.
即使天國的完全的顯現.
要到將來耶穌基督再來的時候.
這個才會完全的彰顯.
但今天天國已經進入到歷史當中.
我們這群藉著耶穌基督的恩典進入天國.
成為天國子民的人.
我們已經活在一個新的國度裡面.
所以我們需要跟從的.
不是地上很多的遊戲規則.

$^{81}$我們唯一需要依從的.
就是神的那個順傳,善良,可喜悅的旨意.
我相信對於很多弟兄姊妹來說.
耶穌的復活這幾方面的意義.
或許我們已經相當熟悉.
起碼我們不會陌生.
不過面對著今時今日.
我們正在經歷這個疫情.
可能我們需要進一步去問的問題是.
就在這個全城,全球.
關注疫症的日子當中.
耶穌基督的復活.
如何影響我們面對疫情的態度.
可能我們最直接,即刻想到的就是.
假如有弟兄姊妹不幸感染疫症.
甚至因此離開世界.
他們也有復活的盼望.
就像保羅在《貼薩萊加前書》裡說到.
當時的信徒都面對有些弟兄姊妹離世.
他們因此而很悲傷.
這也是人之常情.
是很正常的.
不過保羅說我們在悲傷.
不要像那些沒有盼望的人一樣.
我們知道雖然地上的生命會過去.
但我們仍然有復活的盼望.
我們要和耶穌基督一同在新天新地裡.
去享受神的恩典.
但除了將來的復活盼望之外.
耶穌基督的復活.
對於今天正身處在疫症陰霾中的我們.
還有甚麼重要的訊息?.
又或者我們換個方式.
換個角度去問這個問題.
在這段時間我們很感受到.
疫情改變了我們生活很多的事情.
我們的工作.
我們的娛樂.
我們的社交等等.
在這段時間裡都改變了很多.

$^{121}$我們很深刻感受到.
疫情對我們的影響.
對我們的衝擊.
如果疫情也有這麼大的能力.
那麼耶穌基督這份復活的大能.
又如何影響和衝擊我們日常的生活?.
除了我們盼望將來的天家的復活.
我們今天又如何去過一個復活的生命?.
要去思想這個問題.
可能其中一個最好的方式.
是我們回到聖經裡面.
我們看看聖經裡面所記載的那些人物.
他們如何因著耶穌基督的復活.
帶來他們的生命一些改變.
今天我特別想和大家去看的是.
《約翰福音》二十章裡面.
提到瑪利亞遇見這位復活的耶穌基督的記載.
主要的經文是在《約翰福音》二十章.
十一到十八節.
當然我們知道四卷福音書.
都有記載耶穌的復活.
都有記載到耶穌復活之後.
向不同的人顯現.
不過四卷福音書選擇記載的事情.
各自有不同的選材.
他們強調不同的細節.
不同的角度.
以至他們帶出不同的訊息重點.
今天我要和大家看看《約翰福音》裡面.
提到耶穌基督復活顯現.
和瑪利亞相遇.
這個要強調一個怎樣的訊息.
首先我們留意到的是.
其實《約翰福音》二十章.
記載耶穌復活的時候.
這個的確和另外三卷福音書很不同.
傳統我們將馬太,馬可,路加三卷福音書.
我們會叫做「芙蕾福音」.
或者叫做「對觀福音」.
兩個不同的名字.

$^{161}$其實是不同的翻譯.
都是來自英文.
Synoptic Gospel.
意思是這三卷福音書.
是可以甚至應該一起來看.
Synoptic這個字.
其實本身的意思就是「一齊看」的意思.
因為馬太,馬可,路加這三卷福音書.
他們內容上的確有很多重複的地方.
他們之間有很多很相似的東西.
所以我們知道這三卷福音書之間.
他們有比較密切的關係.
但是《約翰福音》就特別不同.
它裡面記載耶穌生平的事件.
祂選擇的事件.
和另外那三卷福音書有比較大的分別.
祂當中介紹耶穌基督的角度.
也和另外三卷福音書比較不同.
在耶穌復活的記載裡面.
《約翰福音》的特別地方又在哪裡呢?.
其中一個我們很直接.
從《約翰福音》第20章.
我們很容易留意得到的.
是《約翰福音》將耶穌復活顯現的記載.
聚焦在兩個人物身上.
一個是我們今天要講的瑪利亞.
另一個是在20章下半章提到的多瑪.
《約翰福音》是唯一一卷.
有提到多瑪疑惑.
祂說祂要用手探入耶穌手上的釘痕.
和耶穌肋旁的傷口.
唯有《約翰福音》有記載這件事.
至於瑪利亞呢?.
這裡說的是莫大拉的瑪利亞.
四卷福音書都有記載她.
都有提到她在主復活那天的清早.
去到耶穌的墳墓那裡.
但發現耶穌的屍體不見了.
於是她就回去找門徒.
告訴門徒這件事.

$^{201}$當中亦都有提到瑪利亞在墳墓那裡遇到天使.
向她說出耶穌復活的消息.
這個過程是四卷福音書都有記載.
但唯獨《約翰福音》.
它很詳細地描述.
瑪利亞在墳墓外遇上復活的耶穌基督.
你可以說透過瑪利亞遇見復活主這段記載.
約翰很活生生的.
很生動的讓我們看到.
一個正在經歷悲傷困難逆境的人.
他怎樣因為遇上這位復活的主.
他整個生命改變了.
而這個對於當初約翰寫這本福音書的時候.
那個早期的教會.
這個都是一個相當重要的訊息.
因為我們知道在早期的教會當中.
那些信徒面對很多的困難.
迫不還難.
他們都是生存在逆境當中.
在逆境當中.
他們除了要捉緊將來復活的盼望之外.
其實他們同樣都是要靠著耶穌基督的大能.
去面對他們仍然在地上所經歷的日子.
約翰透過瑪利亞和主的相遇.
帶出了一個訊息.
復活主的大能怎樣改變我們此時此刻的生命.
當然這個對於今天我們面對的逆境.
同樣都是一個很重要的提醒.
耶穌基督的復活.
怎樣改變我們今日的生命.
當我們要更深入去了解.
到底約翰這一段的記載.
他要帶出甚麼訊息,重點的時候.
或許我們就要先退後一步.
我們去問.
整卷的約翰福音.
當約翰去介紹耶穌基督.
為耶穌作見證的時候.
約翰將重點放在哪裡.
其實這個復活顯現的記載.

$^{241}$和整個約翰福音.
約翰要帶出的重點.
是緊密地連在一起的.
所以我們要問.
整卷約翰福音.
約翰到底想講甚麼重點.
而要回答這個問題.
某程度上你可以說.
不算太過困難.
因為約翰福音第一章.
約翰寫了一個相當精闢的引言.
這個引言就告訴我們.
整卷的約翰福音.
到底約翰想帶出甚麼重點.
所以我們就先來看看.
約翰福音一章一到十八節的引言.
這個引言相信我們不會很陌生.
當中有一些我們很熟悉的金句.
「太初有道,道成肉身」等等.
我們可能都琅琅上口.
不過當我們真的深入去看.
這十八節約翰福音的引言.
可能我們就會發現.
這一段的說話.
有時我們讀起來似明非明.
當中有些東西好像我們很熟悉.
但有些細節.
到底約翰想表達甚麼.
我們又未必一定掌握得到.
在這裡我嘗試和大家仔細去看看.
約翰福音這一句引言.
首先這個引言一開頭第一節.
有幾句我們相當熟悉的說話.
「太初有道,道與神同在,道就是神」.
約翰這裡開宗明義.
說出了耶穌一個很獨特的身份.
祂是太初的道.
在這裡其實約翰借用了.
當時希臘哲學裡面一個觀念.
借用希臘哲學來說信仰的真理.

$^{281}$其實這個不是約翰先做的.
很多人都做.
其中一個我們熟悉的是保羅.
如果大家有印象.
保羅在《士徒行傳》第十七章記載.
保羅去到雅典的時候.
他看到滿街都是神像.
其中有一個叫做「未識之神」的神像.
保羅就借用雅典人所敬拜的「未識之神」.
向他們介紹.
以色列人所信奉,所跟從的.
這位創造天地的上帝.
保羅也做過這樣的事.
約翰這裡也類似.
他在這裡借用了希臘哲學的觀念,語言,字眼.
來嘗試說明關於耶穌基督的真理.
在希臘哲學裡面.
我們要明白.
希臘哲學其實有很多不同的流派.
不同的思想.
是很眾說紛紜的.
不過其中一種比較主流的希臘哲學思想.
他們認為這個世界的運行.
背後是一種至高的原則秩序去規範著.
這個世界不是一個偶然的結果.
不是隨意地去發生.
而是背後遵從著一種至高的原則秩序.
而某程度上.
如果我們和中國的文化比較一下的話.
這個觀念和中國傳統道家的觀念都有點相似.
道家同樣認為天地萬物.
是遵從著一種自然的法則.
他們就稱這個法則為道.
道是天地萬物一切秩序的根源.
所以你就明白.
為什麼《中文聖經》的翻譯.
會將這裡譯成「太初有道」.
約翰借用希臘哲學的字眼.
其實那一字就是logos.
可能有些弟兄姐妹都會聽過這個字.

$^{321}$約翰借用這個字去說關於耶穌基督的道理.
當翻譯成中文的時候.
那些翻譯聖經的人就借用了中國傳統.
這個道這個字來翻譯約翰這裡的說話.
的確這個背後的觀念.
和中國道家所說的道是有點相似.
是說世界運行的秩序.
不過約翰雖然借用了希臘哲學的觀念.
或者他用了這個字眼.
但是他在當中加上了一個全新的意義.
他說道就是神.
約翰這裡說得很清楚.
這個道並不是一種無形的.
掌管著這個世界運行的一種法則這麼簡單.
他說這個道是有神性的.
這個道就是神.
就是以色列人傳統所敬拜的一位上帝.
約翰在這裡.
他將希臘哲學那個道的觀念.
和希伯來.
以色列人傳統的一神創造主的觀念.
來結合在一起.
借用道來解釋神.
在希伯來的傳統當中.
這位上帝除了是創造主之外.
他也是救贖主.
他拯救他的百姓.
昔日他將以色列人從埃及圍爐之地帶出來.
在西乃山和他們納藥.
將他們帶入迦南英許之地.
現在藉著這個道.
耶穌基督.
神要繼續施行拯救的工作.
所以我們看到約翰接著繼續說明.
這個道他作為創造主和救贖主.
這兩個角度的身份.
第三節這裡說萬物是藉著他做的.
這句話有時候我們讀下去.
有些像名非名.
什麼叫萬物是藉著這個道.

$^{361}$我們知道這個道是說耶穌基督.
什麼叫萬物是藉著耶穌基督所做的呢?.
說到創造的時候.
我們比較印象深刻的就是.
《創世紀》第一章的創造記載.
我們很深刻的說到神說.
「要有光就有了光」.
神用他的說話.
說一句話.
事情就發生.
基本上六天的創造都是這樣.
神用說話來成就創造的工作.
《創世紀》第一章的記載相當簡潔.
我們印象很深刻.
但當約翰說到萬物是藉著他.
藉著耶穌基督所做.
是不是在說耶穌基督等於.
《創世紀》第一章.
神所說的那些說話.
是不是就這麼簡單?.
如果我們明白背後.
其實約翰是借用希臘哲學道的觀念.
去解釋耶穌基督的身份的時候.
我們就可以明白多一點.
約翰想表達什麼.
剛才我們說到.
希臘哲學裡面的道.
其實是說世界運行的法則秩序.
神的創造其實不單止是從無變為有這麼簡單.
創造更加是說.
神將他的心意.
放在他所創造的物質世界裡面.
神的心意就好像一個設計藍圖一樣.
他要讓這個世界.
按照著他所設計的藍圖.
去成就,去運作.
而道就好像一個中間人一樣.
借用今天的比喻.
就好像一個設計師.
他設計了一個建築物.

$^{401}$畫了所有的圖樣出來.
但是需要有人幫他去將這個設計實現出來.
需要有工程師.
需要有建築的工人.
來將這個設計實現出來.
所以約翰說的基本上就是.
道就成為了神超越的設計的意念.
和在這個受造物質世界中間的一個中間人.
將神的意念實現在受造物質世界當中.
這個某程度上.
基本上就好像舊約裡面說到智慧的觀念一樣.
在《針研》第八章有一段經文說到.
智慧參與在神的創造當中.
當中正正就是用了工程師的比喻.
智慧就好像一個工程師一樣.
將神對世界的心意落實在世界當中.
讓世界能夠按照著神設計的藍圖.
那些法則秩序來運行.
或者借用希伯來書的說話.
耶穌基督用祂權能的命令去托住萬有.
讓這個世界不會落入混亂的當中.
讓這個世界能夠走在神的旨意裡面.
約翰接著又說.
「生命在他裡頭」.
這裡就說到神要賜的生命.
我們知道神在創造當中.
祂賜了給人.
賜了給萬物有生命.
不過在《創世紀》也告訴我們.
神所賜的生命.
因著罪的緣故受到威脅.
因為罪.
死就進入世界.
所以神要施行拯救.
神要再一次將生命賜予給他所創造的世界.
讓世界脫離罪的核制.
再一次回到神的美善旨意當中.
這裡說到神的拯救.
所以我們看到約翰介紹耶穌基督.
第一個重點.

$^{441}$祂是太初的道.
祂賦予這個世界有秩序.
有生命.
讓這個世界.
特別是其中的世人.
可以回到神的美善旨意當中.
可以得回神要賜給我們的美善生命.
接著約翰帶到第二個重點.
他說耶穌是世人的光.
這個光照在黑暗裡面.
照亮一切生在世上的人.
光和生命是有很密切的關係.
甚至乎你看到第四節的下半節.
他直接將這兩件事等同在一起.
這裡說生命就是人的光.
將光和生命等同在一起.
這個某程度上又是一個很古老的觀念.
古人已經很明白.
光對生命是非常重要.
有了太陽的光.
生命才能夠在地上存活.
所以為什麼我們看到很多古代的民族.
他們會敬拜太陽.
太陽神.
他們很明白.
太陽的光對於生命是很重要.
用今天的科學觀念去說.
就是太陽的光和熱.
來賦予了地球.
這個生態圈的運作所需要的能量.
有了能量.
整個生態系統才能夠運作.
生命才能夠在當中存活和延續.
當然光還有另一個很重要的作用.
就是讓我們能夠看到東西.
這個也是我們日常很熟悉的觀念.
有光我們才能看到東西.
我們困在黑房裡.
黑漆漆.
我們伸手不見五指.

$^{481}$什麼都看不到.
約翰說耶穌是世界的光.
光照在黑暗裡.
這個當然是一個象徵性的說法.
不是說我們真的什麼都看不到.
約翰說的是.
我們有一些重要的東西.
我們應該看到的.
我們應該知道的.
我們卻是看不到.
我們卻是不知道.
從上文下理.
我們知道約翰在說的是真理.
我們不認識真理.
我們不認識神.
所以我們好像活在黑暗中一樣.
耶穌基督的光照在我們當中.
就讓我們能夠看得到.
能夠認識真理.
能夠認識神.
能夠步向神.
於是我們好像十二節也說.
我們能夠作神的兒女.
這是約翰介紹耶穌第二個重點.
他是光.
讓我們看得到.
讓我們能夠接連在真理當中.
第三個重點.
約翰說.
這個太初的道成為了肉身.
十四節也是我們很熟悉的金句.
道成了肉身處在我們中間.
充充滿滿的有恩典有真理.
我們也見過他的榮光.
正是父獨生子的榮光.
其實在當時來說.
這又是一個很突破的觀念.
因為特別是以色列人傳統的觀念裡面.
那位超越的上帝和受造的世人之間.
有一個不能跨越的鴻溝.

$^{521}$雖然以色列人.
他們知道他們和神有很密切的關係.
但同時.
神和人之間仍然有一個不能跨越的鴻溝.
就正如十八節這裡說.
從來沒有人看見神.
按照著人受造的本質.
我們是不可能看到那位超越的上帝.
但耶穌基督是跨越了這個不能跨越的鴻溝.
祂本身是道.
祂本身是神.
但祂成為肉身.
成為人.
處在我們中間.
使我們原本和神那種不能跨越的接觸.
變成可能.
就是十八節最後這裡說.
父為女的獨生子.
將祂表明出來.
我們能夠更加實在.
更加貼身地去經歷這位上帝.
這當然也讓我們更加毫無阻隔地.
去領受神的豐盛的恩典.
所以約翰強調.
耶穌住在我們中間.
祂充充滿滿有恩典,有真理.
到了十六節他再說.
我們都領受了祂這個豐滿的恩典.
而且「恩上加恩」.
我們很簡單地去總結了約翰福音的引言.
當中我們看到約翰強調耶穌.
有幾個方面.
祂是太初的道.
祂是神和世界之間的中間人.
是讓神的旨意能夠實現在.
祂所創造的這個世界裡.
第二,祂是世界的光.
祂讓我們認識真理.
認識神.
步向神.

$^{561}$第三,祂是成為肉身的道.
讓我們更加貼身去經歷神的同在.
領受神的恩典.
掌握了這幾個重點之後.
我們回來看約翰福音二十章.
耶穌復活的記載.
首先我們看一看二十章一到十節.
這裡交代的.
我們剛才提過.
就是主復活那天的清晨.
瑪利亞去到墳墓那裡.
發現耶穌的屍體不見了.
於是她就回去找門徒.
約翰福音提到.
她遇見兩位門徒.
一位是彼得.
另一位.
這裡沒有說他的名字.
不過我們一般相信.
這裡提到的耶穌所愛的那個門徒.
應該是指約翰.
瑪利亞告訴他們.
耶穌的屍體不見了.
大概我們想像.
很可能是瑪利亞想兩位門徒.
幫忙去找耶穌的屍體.
因為去到後面我們看到.
其實瑪利亞一直很想.
可以找回耶穌的身體.
可能瑪利亞就想兩個門徒幫忙去找.
於是兩個門徒聽到瑪利亞這樣說.
他們就跑去墳墓那裡.
一先一後去到墳墓那裡.
都發現耶穌的屍體真的不見了.
只剩下包裹屍體的細麻布.
其實這裡記載的細節都挺有趣的.
我們看到約翰跑得快.
他先去到墳墓.
但他站在墳墓外面不進去.
只在墳墓外面望進去.

$^{601}$然後彼得隨後來到.
二話不說就跑進了墳墓裡.
然後約翰就跟著他後面進去.
這些細節我們不詳細去討論了.
因為我們主要想說的是瑪利亞.
反而我們想留意一點.
在第八節這裡說到.
門徒進去看到空墳墓的景象.
這裡說先到墳墓的門徒.
就是約翰進去看見就信了.
第九節再補充.
他們之前不明白耶穌必須從死裡復活.
這裡有多少細節我們留心.
第一.
我們留意到第八節說到看見就信了.
這個看中文的翻譯.
我們不是很清楚看得到.
因為中文的動詞是沒有單數眾數的分別.
如果我們參考英文或原文.
我們就會看到看見這個動詞其實是單數.
只是說一個人.
一個人看見就信了.
很明顯就是說先到墳墓的那門徒.
約翰他看到就信了.
那彼得呢?.
彼得有沒有信呢?.
經文沒有告訴我們.
我們其實不知道.
另外就是他信什麼呢?.
他是信耶穌復活了.
還是信瑪利亞說的話.
耶穌的屍體真的不見了?.
其實我們看清楚一點.
經文這裡都沒有說清楚.
第九節告訴我們.
他們還不明白聖經的意思.
或許他們看完空墳墓之後.
他們仍然未明白.
直到那一晚.
在後面的經文十九節.

$^{641}$我們就看到那一晚耶穌向他們顯現.
他們看見復活的主就起來.
或許他們當時看見空墳墓.
還未明白耶穌的復活.
所以我們看到第十節記載.
他們就各自回自己的住處去.
你想清楚.
這是一個挺有趣的反應.
他們去到墳墓看見耶穌的屍體不見了.
無論他們認為耶穌是復活了.
還是認為耶穌的屍體被人搬走了.
照理說都不會若無其事地回家.
似乎這句話讓我們感受到.
兩位門徒還未消化整件事.
他們就回到家裡.
你可以說這十節記載這兩位門徒.
記載得很簡短.
匆匆忙忙跑到墳墓那裡.
看見空墳墓再回家.
這個很簡短很匆忙的記載.
正好對比著接著那一段.
瑪利亞在墳墓外遇見耶穌的記載.
所以我們看到.
說完兩個門徒之後.
十一節.
經文就將焦點放回瑪利亞身上.
說她留在墳墓外.
她站在那裡哭.
這個正好和第十節說兩個門徒.
回家成為一個對比.
瑪利亞留在墳墓外.
可能正如剛才所說.
她原本指望兩個門徒.
幫她搜尋耶穌的身體.
兩個門徒沒有幫忙.
甚至從這裡的記載.
似乎我們有感覺.
兩個門徒都沒有和瑪利亞說什麼.
他們就回家了.
留下瑪利亞一個人在這裡.

$^{681}$不知道該怎麼辦.
經文這裡很強調.
瑪利亞在那裡哭.
經文一連四次提到瑪利亞哭.
首先第十一節一開始說.
瑪利亞在墳墓外哭.
隨即再說她哭的時候.
往墳墓裡看.
所以只是第十一節已經接連兩次提到瑪利亞哭.
接著第十三節天使問瑪利亞.
十五節耶穌問瑪利亞.
都是同一個問題.
你為什麼哭?.
經文一連四次提到瑪利亞哭.
這當然讓我們感受到.
瑪利亞當時的心境.
她是很傷心.
從當時的情景我們不難明白.
為什麼瑪利亞會這麼傷心地哭.
不久之前她才親眼看著.
她所敬愛的這位夫子耶穌.
他受盡酷刑死在十字架上.
這個時候她來到墳墓這裡.
她想為耶穌盡最後一點心意.
來去告末耶穌的屍體.
誰知道耶穌的屍體不見了.
不知道去了哪裡.
她連想為耶穌做的最後一件事都做不到.
那種傷痛.
那種失落.
那種無助.
我們不難去明白.
另外這段經文有另一個細節.
我們可以稍為留心一下.
這段經文由十一節到十八節.
瑪利亞由頭到尾.
都是用「我」來自稱.
你說自稱做「我」有什麼特別呢?.
我們每個人都是說「我」的.
難道說「朕」嗎?.

$^{721}$要明白「我」這個稱呼有什麼特別.
我們就要對比前面的經文.
第一,第二節.
說到瑪利亞去到墳墓那裡.
不見了耶穌的屍體.
她就回去找門徒.
第二節她跟門徒說.
有人把柱從墳墓裡挪了去.
我們不知道放在哪裡.
雖然二十章第一節.
只是提到瑪利亞去到墳墓那裡.
但第二節瑪利亞的說話是說「我們」.
我們參考其餘三段方書.
我們就知道.
其實那天早上去墳墓那裡的.
不只是瑪利亞.
其實還有其他的婦女一起去.
所以瑪利亞所說的「我們」.
可能包括其他.
跟她一起去墳墓那裡的婦女.
但來到第十一節.
瑪利亞只是說「我」.
她不再說「我們」.
這就更加突顯了.
瑪利亞當時只是自己一個人在那裡.
面對整個的情景.
找兩個門徒來幫忙.
兩個門徒都走了.
其他的婦女也不知道去了哪裡.
只剩下瑪利亞一個人在那裡.
這或許就強調了瑪利亞.
她獨自面對這個困境.
那種孤單的感覺.
我們再留意多一個細節.
這段經文裡面其實兩次提到瑪利亞「體見」.
首先第十四節.
這裡說到瑪利亞轉身來「體見」耶穌.
站在那裡.
然後第十八節.
瑪利亞回到門徒那裡告訴他們.

$^{761}$「我已經看見了主」.
兩次的經文都提到「體見」.
不過其實這兩個「體見」是不同的.
首先是原文所用的字是不同的.
兩次的「體見」是不同的動詞.
十四節的「體見」動詞.
稍微比較多一點.
是強調「體」的動作.
如果用英文去說的話.
就是「look at」.
就是「體」的動作.
十八節用的字稍微多一點.
是表達「體」的結果.
用英文說就是「see」.
就是「真的看見了」.
當然單單看這兩個不同的動詞.
其實分別都不是那麼清晰和絕對.
這兩個動詞有時候都可以交替使用.
所以我們不能單憑這兩個不同的動詞.
去拿一個很確切的結論.
不過更加清楚的是.
十四節告訴我們.
瑪利亞看到耶穌站在那裡.
卻不知道是耶穌.
這裡就很清楚告訴我們.
當瑪利亞轉過來.
看到耶穌站在她面前的時候.
她其實是視而不見.
她親眼看著.
但其實她看不到真相.
她以為眼前那個只不過是一個圓丁.
我們嘗試換個角度.
換個心情.
看一看十五節瑪利亞和耶穌的對話.
我們換個心情去看的時候.
其實那個頗滑稽的.
瑪利亞和耶穌說.
「是不是你搬走了耶穌的屍體?」.
「你告訴我你搬到哪裡去了?」.
「讓我去找他吧」.

$^{801}$瑪利亞對著耶穌說要找耶穌.
基本上就是那些人戴著眼鏡.
然後去找眼鏡.
其實是頗滑稽的.
這個是瑪利亞.
她當時的困境.
她承受著失去摯愛的傷痛.
她身邊沒有人幫她.
她自己一個人不知怎麼辦.
那種孤單無助.
她也對眼前的真相視而不見.
她沒有辦法回應.
她當時面對的環境.
這個會不會是今天.
我們在疫情當中.
我們的經歷.
我們的感受.
在疫情當中.
我們可能有很多傷痛.
可能我們染病.
我們的親人染病.
我們熟悉的朋友染病.
甚至可能離世.
這個都會帶給我們很多傷痛.
面對著疫情.
可能我們都會有很多徬徨無助.
孤單的感覺.
我們亦都可能在疫情當中.
我們被很多訊息包圍.
我們卻看不到真相.
在這個困境當中.
復活的主怎樣能夠幫助我們.
怎樣能夠改變我們.
當瑪利亞深陷在這個困境的時候.
經文來到一個轉捩點.
在十六節.
耶穌叫了她一聲.
叫她瑪利亞.
就是這一聲.
讓瑪利亞整個人不同了.

$^{841}$我們可以留意.
經文兩次提到瑪利亞轉過來.
首先是十四節.
瑪利亞和墳墓裡的天使說完話之後.
經文說她轉過來.
就看到耶穌站在那裡.
不過不認得是耶穌.
然後第十六節.
經文說她轉過來.
就稱呼耶穌叫她拉波尼.
我們留心一個細節.
如果第十四節已經告訴我們.
瑪利亞已經轉過來了.
她已經看到耶穌.
十六節瑪利亞還怎樣轉呢?.
她還轉到哪裡去呢?.
其中一個可能性就是.
十四節說瑪利亞轉過來.
可能還未完全轉過來.
她原本對著墳墓.
然後察覺到後面有人.
所以她稍微轉過頭.
還未完全轉過去.
然後十六節耶穌叫她的名字.
她才完全轉過去.
正面面對著耶穌.
這是其中一個可能性.
但是從整個經文的鋪排.
尤其是十四節強調瑪利亞.
認不出眼前的人是耶穌.
所以十六節這裡說她轉過來.
可能不是說瑪利亞的身體轉過來.
而是說她的心思意念轉過來.
「轉」這個字.
是可以用來表達「回轉」的意思.
例如在《約翰福音》十二章四十節.
引用以塞亞的說話.
「免得他們眼睛看見心裡明白回轉過來.
我就醫治他們」.
「回轉」這個字.

$^{881}$其實和瑪利亞「轉」過來.
用的是同一個字.
或者在《馬太方案》十八章三節.
耶穌和門徒說.
「你們若不回轉變成小孩子的樣式.
斷不得進天國」.
那裡的「回轉」用的都是同一個字.
所以經文這裡說瑪利亞「轉」過來.
第十六節第二次說她「轉」過來.
說的可能不是說她的身體轉過來.
面向耶穌.
而是她的心思意念轉過來.
她從不認得眼前這個人.
她變為認得出耶穌.
這是瑪利亞遇見這位復活的主之後.
第一個最直接的轉變.
從不認得.
從看不到眼前的真相.
她忽然之間認得眼前這個人.
原來是耶穌.
耶穌已經復活了.
為什麼之前瑪利亞會不認得耶穌呢?.
這個經文沒有告訴我們為什麼.
所以不同的人有不同的猜測.
有些人認為會不會是耶穌復活之後.
她戴著那個復活榮耀的身體.
所以樣子不同了.
那瑪利亞就不認得她了.
這個解釋有一個弱點.
就是我們知道耶穌復活之後.
其實她仍然戴著復活之前很多身體上的標記.
包括那個釘痕的手.
包括那個被刺傷的肋膊.
如果耶穌復活之後.
仍然戴著復活之前這些傷痕的話.
可能更加大的可能性.
就是她復活之後.
其實她的樣貌和復活之前.
未必會有太大的分別.
所以有人提出另一個可能性.

$^{921}$會不會因為瑪利亞哭得太厲害了.
剛才也說經文一連四次強調瑪利亞在哭.
瑪利亞哭到淚眼朦朧.
淚水遮住了她的眼睛.
所以她沒有看清楚眼前這個人.
這的確是一個可能性.
又或者經文還給了我們一個提示.
經文讓我們看到瑪利亞很著緊地尋找耶穌的身體.
她找門徒幫忙尋找.
她看到墳墓裡的侍者.
問侍者知不知道耶穌的身體去了哪裡.
她轉過來看到眼前她以為是圓釘的人.
她問他知不知道耶穌的身體去了哪裡.
問令到當時整個心思意念只有一件事.
我要去尋找耶穌的屍體.
她只想著這件事.
所以她根本沒有好好看清楚眼前這個人.
更加不會想到他就是耶穌.
其實我們知道人有時也是這樣.
我們的心思意念的注意力.
泰國放在某一件事上的時候.
我們會忽略了其他事物.
有時即使是眼前的事物都可以視而不見.
我記得我小時候小學上中文課.
其中有一本書就講牛頓.
我們知道牛頓是一個很偉大的科學家.
那本書其實我不是很清楚這個故事是真是假.
不過那本書是這樣說的.
就是牛頓有一次做實驗.
做到很投入很專心.
做了一會兒就肚子餓了.
他就想煮一隻雞蛋來吃.
將雞蛋放進鍋裡煮.
但因為太專心做實驗.
一個不經意放了一隻手錶進去.
他也不知道.
直到想著拿隻雞蛋出來吃的時候.
他才發現原來放錯了一隻手錶進去.
人就是這樣.
泰國專注於一件事.

$^{961}$可以忽略很多其他很理所當然的事情.
這裡瑪利亞可能也是這樣.
她的心思意念集中在耶穌的屍體上.
她沒有看清楚眼前這個人.
不過當耶穌叫他的名字.
叫他一聲瑪利亞的時候.
瑪利亞就認得他了.
那為什麼瑪利亞突然又會認得呢?.
這裡又有不同的猜測.
會不會是瑪利亞認得耶穌的聲音.
認得他叫瑪利亞的獨特語氣,口音.
這些都有可能.
不過其實更加重要的是.
是耶穌叫他的名字.
因為我們發現其實不只是到.
十六節耶穌才跟他說話.
十五節耶穌已經跟他說了一句話.
問他你為什麼哭.
當時瑪利亞也沒有從耶穌的聲音認出他.
但到了十六節耶穌叫他的名字的時候.
瑪利亞就認出耶穌.
因為當耶穌叫他的名字.
她就突然間驚覺到.
眼前這個人不是一個三不識七的園丁.
眼前這個人認識他.
甚至是跟他有很密切的關係.
這個讓她看真眼前這個人.
她就認出耶穌.
接著瑪利亞就叫他拉波尼.
拉波尼這個字其實源自拉比這個字.
拉比這個字我們很熟悉.
我們知道拉比是老師的意思.
當時很多人都會稱呼耶穌做拉比.
稱呼他做老師.
但瑪利亞不只是稱呼耶穌做拉比.
瑪利亞稱呼他拉波尼.
拉波尼這個字其實是拉比這個字.
加上一條尾巴.
將老師變成我的老師.
瑪利亞說出了.

$^{1001}$她跟耶穌之間有一個很密切的關係.
這個不只是一個老師這麼簡單.
眼前這個是我的老師.
瑪利亞她很感受到.
她跟耶穌之間有一個很個人.
很密切的關係.
這裡我們看到的是.
瑪利亞遇見這位復活的耶穌.
耶穌稱呼他.
她不單只認得耶穌.
更加是她跟身邊整個現實重新接軌.
之前她基本上是跟現實脫節.
耶穌站在她面前.
她都認不出.
她都看不到.
但現在她認出了.
她明白她眼前發生的.
到底是一回甚麼事.
所以她不會像剛才所說的笑話.
戴著眼鏡去找眼鏡.
對著耶穌問去哪裡找耶穌.
我們還記得.
《約翰福音》第一章說.
耶穌是世界的光.
祂照亮我們的黑暗.
祂叫我們真正看見真理.
看見事物真實的面貌.
這是看見我們跟神所創造的這個世界.
跟祂擺在這個世界裡的旨意.
不再脫節.
我們能夠跟世界的主結連.
以致我們能夠正確地回應我們的主.
當我們因著一些事情.
很困擾我們就像現在我們面對的疫情.
我們很自然被各種情緒去充斥.
我們會有難過,有悲傷,有擔心.
可能有憤怒.
這些情緒很自然.
我們都很自然會關注一些我們認為重要的事情.
可能我們每一天打開新聞.

$^{1041}$我們就留意著數字.
新增多少確診.
升幅是越來越嚴重還是開始放緩.
我們可能關注的死亡率有多高.
關注防疫物資的供應是怎樣.
或者有些人可能更加關注.
什麼時候可以復市.
恢復經濟的活動.
我們有我們很關注的事情.
於是我們就像瑪利亞一樣.
繼續不斷追問我們心目中認為最重要的問題.
瑪利亞不斷追問.
誰搬走了耶穌的屍體.
搬到哪裡去了.
這是她認為最重要的問題.
但其實這個問題根本不需要去問.
在疫情當中.
我們需要遇見復活的主.
我們不要對祂視而不見.
不要祂站在我們眼前.
我們都當看不到.
我們看不看到.
當我們正在經歷這個疫情的時候.
其實主就站在那裡.
沒錯我們生活我們的工作會被疫情打亂.
但主的工作卻是不會被疫情打亂.
我們看不看到這個現實.
我們回看瑪利亞.
當她認出耶穌之後.
當然她就不需要再去找耶穌了.
之前她逢人就問.
我應該去哪裡找耶穌.
現在她不需要到處去找.
因為主已經與她同在.
她能夠在耶穌裡面.
在耶穌的面前.
得著滿足.
得著安穩.
我們又記得《約翰第一章》.
說耶穌是道成了肉身.

$^{1081}$住在我們中間.
豐豐滿滿有恩典.
有真理.
我們很需要神的恩典.
因為我們離開了神.
我們的生命就永遠是一種缺欠.
是一種不安的當中.
我們會很想找方法去填補這種缺欠.
但憑我們自己的能力.
我們填補不到.
所以神親自藉著聖子耶穌基督.
來到我們中間.
讓我們可以領受祂的恩典.
我們不需要繼續徬徨不安地去找.
不需要被這種徬徨不安缺欠的感覺.
牽扯著我們.
即是說環境.
即是說我們今天面對的疫情.
的確會讓我們有不安.
有徬徨的感覺.
但既然復活的主已經與我們同在.
我們能不能夠經歷到這份安穩?.
最後經文讓我們看到.
瑪利亞遇見復活的主之後.
有什麼轉變呢?.
剛才提到經文一開始.
很強調瑪利亞在墳墓裡哭.
她很想找耶穌.
但她只能夠站在那裡哭.
什麼都做不到.
一方面因為她真的不知道去哪裡找.
她不知道誰搬走了耶穌的屍體.
想找也無從找起.
除了這個原因之外.
也因為她那種很強烈的悲痛情緒.
讓她整個人癱瘓了.
除了哭.
除了問人耶穌去了哪裡.
她站在那裡.
我們俗語說「站在那裡」.

$^{1121}$瑪利亞這裡名副其實.
她站在那裡.
什麼都做不到.
但當她遇見這位復活的耶穌.
當她認出耶穌之後.
耶穌叫她回去弟兄那裡.
於是這段最後十八節我們看到.
瑪利亞不再站在墳墓外哭.
她回去跟門徒說.
她見到復活的主.
她回去將耶穌吩咐她的話.
傳達給門徒聽.
其實之前經文頌交代了一個細節.
十七節那裡耶穌先跟她說.
「不要摸我」.
「摸」這個字.
如果我們參考一些其他的譯本.
有些譯本將這裡譯成「拉著」.
「摸」這個字的確有「拉著」的意思.
在當時這個情景當中.
或許翻譯成「拉著」.
是更加的貼切.
我們不難代入當時的場景.
瑪利亞一直以為耶穌死了.
她已經失去了耶穌.
她來墳墓這裡原本只是想.
高沒耶穌的屍體.
但突然之間耶穌活生生站在她面前.
讓她喜出望外.
我們不難明白瑪利亞那種.
失而復得的喜悅.
又或者再加上一點點.
她害怕再次失去耶穌.
所以她很著緊地捉緊耶穌.
但耶穌對她說.
「不要拉著我」.
是的 雖然瑪利亞和父主的重逢.
是很喜悅的一件事.
但這不是他們互相擁抱慶祝的時候.
因為耶穌說她還有要做的事.

$^{1161}$她要升上去見她的父.
同時耶穌亦交給瑪利亞一個任務.
我們剛才提到.
耶穌要瑪利亞回去告訴門徒.
耶穌已經復活.
告訴門徒.
向門徒傳達耶穌的說話.
這個時候我們看到瑪利亞.
她不再被悲傷癱瘓.
她做回她應該做的事.
做回主吩咐她做的事.
也是身為一個跟從主的門徒.
她應該要做的事.
我們看到的是.
瑪利亞為主作見證.
楊福音開始說.
耶穌是太初的道.
萬物是藉著祂做.
祂托住萬有.
祂賜予生命.
祂叫世界可以按照著神尾線的旨意去運行.
在瑪利亞身上我們看到.
她遇見復活的主.
讓她的生命原本是被悲傷控制.
甚至誇張一點說.
幾乎被悲傷摧毀的生命.
能夠重回正軌.
能夠回到神的旨意當中.
能夠做神要她做的事.
從這裡我們看到.
楊福音一開始的引言裡.
介紹耶穌基督的身份.
祂是太初的道.
祂是世界的光.
祂是成了肉身.
住在世人中間的道.
讀下去似乎是一些很高深的神學道理.
神學的宣稱.
但是在瑪利亞和復活主相遇的這段記載當中.
我們就很具體看到.

$^{1201}$瑪利亞如何因為主而改變過來.
楊福音第一章引言裡所說的.
原來不單止是一些超越的.
屬靈的神學的大道理.
更加說的是.
我們在生命最低落.
最難過.
最困難的逆境當中.
我們可以很切身.
很真實.
很具體地經歷到的生命的轉變.
這是我們從瑪利亞身上所看到的.
作為一個總結.
我們知道主的復活是一個客觀的事實.
復活的主與我們同在.
這也是一個客觀的事實.
當你今天身處在逆境當中.
我們面對著這個疫情.
我們又有沒有真真正正遇上這位復活的主.
復活的主為我們的生命帶來怎樣的衝擊.
怎樣的改變.
首先當然.
我們要認定主已經復活的客觀事實.
說的不是兩千年前耶穌復活的歷史事件.
而是今天這位復活的主.
他復活的大能.
仍然勝過死亡的權勢.
在今天這個疫情當中.
復活的主仍然臨在世界.
仍然臨在我們中間.
無論我們的處境令我們有多擔心.
有多沮喪.
我們要抓緊這個客觀的事實.
當我們看的不再只是眼前的疫情.
不再只是數字.
不再只是裝備.
當我們轉眼望向這位復活的主.
我們所宣認的客觀事實.
就要成為我們主觀的經歷.
復活主所賜的喜樂.

$^{1241}$平安.
能力.
就要進入我們裡面.
這也讓我們可以繼續去做我們應該做的事.
就是我們為主作見證的大使命.
是的.
疫情或許改變了我們日常的生活.
疫情或許叫很多事情停頓了下來.
但讓我們在復活的主裡面.
我們的生命不會停頓.
不會癱瘓.
讓我們在復活的主裡面.
我們的使命不會被改變.
在節目中我同樣邀請大家.
在禱告當中.
轉眼望向這位復活的主.
讓祂的平安.
讓祂復活的大能.
就進入你的裡面.
帶領我們去面對這個疫情.
帶領我們在疫情當中成為主復活的見證.
謝謝大家.
字幕由 Amara.org 社群提供.
感謝收看.
(字幕由 Amara.org 社群提供).
\newpage



\section{}
\label{sec:AHC7wVTdk1o}
\textbf{【疫有嘢學 │ 延SUN在線】窘迫與釋放:殘疾處境之啟迪|陳關韻韶女士}
\newline
\newline
連結: \href{https://youtube.com/watch?v=AHC7wVTdk1o}{\texttt{ https://youtube.com/watch?v=AHC7wVTdk1o}} ~~~~ 語音日期: 2020-06-28 
\newline
\newline
\hyperref[sec:O9blI5PB1Ss]{\small{< < < PREV SERMON < < <}}
~
\hyperref[sec:index]{\small{[返主目錄]}}
~
\hyperref[sec:KKN4CX_CTEM]{\small{> > > NEXT SERMON > > >}}
\newline
\newline
$^{1}$主席.
我相信這也是把主日教導的工作交回教會.
同時也是易有野學結束的合適時間.
雖然今天是易有野學的最後一課.
但弟兄姊妹仍然可以透過忠臣的網上平台去學習聖經.
因為忠臣延伸部七至九月的網上課程仍然接受大家的報名.
在這裡借這個機會衷心地去感激曾經為易有野學付出的老師,同工.
以及一直為我們打氣的弟兄姊妹.
多謝你們.
今天為易有野學課堂壓軸的老師是陳關運兆老師.
Vance是忠臣實踐科副教授.
他曾經在阿富汗參與國際救援組織的復康工作.
Vance老師一直都是致力推動教會關注特殊群體的需要.
並且身體力行地在教會開展無障礙敬拜的事工.
今天他會與我們分享一個課題名為「困壁與釋放殘疾處境的啟迪」.
今天我們也邀請了兩位忠臣的校友.
是王海恩傳道 Carol和他的拍檔謝慧璇姐妹阿姐.
在我們當中為課堂作即時的手語翻譯.
就將以下時間交給Vance老師 Carol和阿姐.
願更多的弟兄姊妹你們能夠因著陳關運兆老師的教導.
再次被主的說話觸動和激勵.
弟兄姊妹平安.
今日亦有ye 學延伸再線的題目是「困壁與釋放殘疾處境的啟迪」.
先來看看這個題目.
主要的標題是困壁與釋放.
困壁是人生處境裡面經歷到的一些狀態.
一種令人覺得很不安很困難很大壓迫感.
會有傷心失望憤怒混亂受創.
很想快點可以離開的狀態.
對於困壁裡面的人得到釋放當然是最好的消息.
令人經歷困壁的處境可以有很多種.
例如可以是經濟困難.
人際關係的糾結和鬥爭.
病患殘疾親人離世戰爭等等.
當然還有我們很近身最近經歷的世紀疫症.
和作為香港人經歷的社會變遷.
都是可以帶來困壁的處境.
為困壁的人帶來釋放.
正正是以色列人期待尼塞亞來臨將要帶來的盼望.
亦都是耶穌開始傳道的初期.

$^{41}$祂的使命宣言.
《路加福音》四章有這樣的記載.
耶穌來到祂長大的地方拿撒勒.
在安息日照祂平時的規矩進入會堂.
站起來要讀聖經.
有人將以塞亞的書交給祂.
祂就打開找到經文這樣說.
第十八節.
主的靈在我身上.
因為祂用膏告我.
叫我傳福音給貧窮的人.
差遣我報告被擄的得釋放.
乞訝的得看見.
叫那受壓制的得自由.
報告神越立人的欺憐.
之後耶穌坐下來.
會堂裡的人都眼盯著祂.
耶穌對他們說.
今天這經應驗在你們耳中了.
耶穌基督帶給困壁世代的福音.
就是報告神越立人的欺憐.
欺憐會發生什麼事.
你未記二十五章描述的欺憐.
可以說是一種還原.
恢復和免債.
之前因為生活艱難.
賣了出去的地.
可以地歸原主.
賣身的老僕可以重獲自由.
這裡提的欺憐.
應對以塞亞書六十一章.
一至二節裡說的耶和華的因憐.
耶穌的使命宣言.
表達祂的福音.
是觸及很多範疇的釋放.
包括傳福音給貧窮的人.
被擄得釋放.
乞訝的得看見.
受壓制的得自由.
不過很快問題和張力就出現了.

$^{81}$路加福音七章記載著一段類似的經文.
不過有些重要的部分就不見了.
第七章的背景是這樣的.
耶穌周圍傳道醫病趕鬼.
要死人復活的事跡.
傳遍了猶太和周圍的地方.
當時施洗日漢被昏封的王希律.
囚禁在監獄裡.
原因很簡單.
因為他公開批評希律做的一切惡事.
路加福音七章記載施洗日漢.
差派兩個門徒去問耶穌.
「將要來的是你嗎?.
還是我們等候別人呢?」.
這一問真是可圈可點.
說得白一點俗一點.
就是在問你究竟是堅還是流呢?.
你說你是道路真理生命.
但是現況看來不對路.
耶穌的回答更加可圈可點.
七章二十二節耶穌回答說.
「你們去,將你所看見的.
所聽見的事告訴約翰.
就是客子看見,騎子行走.
長大麻風的潔淨,龍子聽見.
死人復活,窮人有福音傳給他們」.
你有沒有留意到這裡所說的.
和第四章所列出的有什麼分別呢?.
這個清單有些東西多了.
也有些東西不見了.
少了些什麼呢?.
這裡沒有說到被擄得釋放.
受壓制的得自由.
對施洗約翰來說.
就算耶穌厲害得可以令死人復活.
但是作為一個囚犯.
對他來說最道地的福音.
就是得釋放得自由了.
為什麼這位英雄的尼塞亞.
還不出手救自己呢?.

$^{121}$可能這種張力和困惑.
比起被囚禁本身更加難受.
帶來心靈裡的困迫更加大.
聖經所應許的釋放.
究竟在哪裡看得到呢?.
究竟還要等到什麼時候呢?.
耶穌絕對能夠明白.
約翰的困惑.
明明有能力的.
但是他不但沒有做到他答應做的事.
就連一句話都沒有說.
怎麼說都說不通.
難怪耶穌說.
凡不因我跌倒的就有福了.
我們雖然不是施洗約翰.
但是相信我們或我們身邊的人.
經歷人生各種困迫的處境.
都會和約翰一樣去問上帝.
為什麼還沒見到你出手拯救呢?.
究竟還要等到什麼時候呢?.
你會怎樣回應這些問題呢?.
當然你可以輕輕帶過.
萬事都互相效力.
叫愛神的人得益處.
又或者你會出王牌.
我們不明白的了.
要乖乖地信服在上帝的主權之下.
對於在困苦當中的人.
這些標準答案.
他心裡又怎會不知道呢?.
但是這些標準答案.
往往是帶來更加大的困迫.
因為答案的背後.
要求困迫裡的人.
將自己心裡的哀動割裂.
所以曾經有人說.
如果他不是基督徒.
他還沒那麼辛苦.
為什麼呢?.
因為除了他要應付處境裡帶來的實際困難.

$^{161}$和壓迫之外.
他還要應對將他撕裂的信仰張力.
沒辦法明白上帝在做什麼.
看不到教會群體在做什麼.
在做什麼.
所以今天的題目.
困迫與釋放.
聽起來應該是一個很有盼望.
很有力量的題目.
因為我們的救贖主已經來到.
已經得勝了.
天上地下的權柄已經屬於他.
沒錯的確是這樣.
新天新地裡得勝就要完全彰顯了.
但是在今天的處境裡.
我們要怎樣面對這種令人撕裂的張力呢?.
我們是否要完全將心裡的張力放在一旁.
摺起來.
也強迫身邊的人和我們一樣這樣做.
才算是一個有信心.
有喜樂的基督徒呢?.
今天的課堂.
就是要在困迫和釋放的張力裡學習.
因為不是嘗試提供一個從上而下的標準答案.
而是嘗試用一個從下而上的角度.
去和大家思考.
什麼叫做從下而上呢?.
意思是從處境出發.
在處境裡去聆聽困迫的聲音.
去發掘得釋放或不得釋放的經驗.
過程裡會做信仰的整合.
也就是用聖經的話語.
神學的論述去和這些聲音和經歷對話.
希望藉著這種和處境對話互動的神學反思.
讓我們更加了解福音的意義.
更重要的目的.
是要去認識那位向我們展示福音的神.
我選擇了用禪室的處境.
作為今天思考的場景有幾個原因.
其中一個是因為我和這個群體有千絲萬縷的連結.

$^{201}$有很多機會以第一身或很靠近身的同行者身份.
去聆聽這個群體的聲音.
另外你也應該留意到.
在耶穌的使命宣言裡.
為有殘疾的人帶來醫治.
是宣告神越立人的欺凌其中一個重要部分.
我相信聆聽殘疾當中.
的困迫和釋放的經歷.
是可以幫助我們更加明白福音在此時此刻的意義.
西非加納的處境神學家Bernard Quagshi說.
每一個不同的文化場景.
他們的群體的屬靈經驗和他們的神學反省.
都好像鑽石的其中一個切割面.
讓我們可以更加了解基督信仰.
作為一個普世信仰的豐富.
雖然殘疾未必是你有興趣或你會接觸的處境.
但我邀請你像拿起一顆鑽石一樣.
前後左右多角度去看.
從殘疾處境裡去欣賞這顆鑽石所發揮的光輝.
今天我們先會從困迫開始觀察和思考.
我們會從殘疾處境去看一個縱容歧視的文化.
所衍生的製造障礙的神學和欺凌弱小的制度.
如何形成多方面的困迫.
接著我們會從聖經話語去看耶穌模式的釋放.
「道成肉身成為殘障,醫治殘疾,整全釋放」.
「以馬內利困迫同行」.
今天你會看福音書裡的一些醫治殘疾的事蹟.
也會觀察耶穌和今天有殘疾的信徒群體.
如何在未得釋放的處境中展現著一種什麼樣的生命.
藉著今天的學習.
希望你對困迫有更深入的了解.
從而再重新聆聽你自己和你身邊的人.
今天所經歷的處境.
你所需要的釋放.
或者去發現你已經正在經歷的釋放.
其實除非我們死得早.
殘疾這個處境遲早都是我們要面對的事實.
所以不妨早點去了解一下.
從處境出發.
首先我們看一些客觀的資料.

$^{241}$以下是一些香港的殘疾人士的殘疾類別的數字.
紅色字列出的類別.
是指任何人在統計的時候.
認為自己有以下四項的.
其中一項或者多於一項的情況.
並且持續或者預料會持續最少六個月的時間.
當中包括有身體活動的能力受限制.
視覺有困難,聽覺有困難.
或者言語能力有困難.
因為是可以多於一項.
所以總數加起來是會不同的.
藍色字體列出來的是.
經過認可的醫療人員.
例如西醫,中醫他們診斷出來的殘疾類別.
有以下的幾項.
精神病,情緒病,自閉症,特殊學習困難.
注意力不足,過度活躍症和智力障礙.
智力障礙人士的統計數字.
往往發覺有低估的情況.
所以為了避免誤導的結果.
統計報告是分開計算的.
大家看到就算是說幾年前的統計.
殘疾人士是佔了全港人口的百分之九點多.
在疫情裡面.
大家也經歷了困難和挑戰.
不知道大家有沒有留意到.
有些殘疾群體是有特別的困難.
早前《鏗鏘集》也探討過一群自閉症小朋友.
在停課時間的情況.
一般小朋友要留在家中學習.
已經很難適應了.
自閉症的小朋友.
還要他們改變自己習慣了的每天流程.
要應付遙距的學習.
對小朋友和家長來說.
是一個很大的困擾.
不過有各種能力的限制.
不代表那個人一定是經歷很大的障礙和困迫.
例如我是有視力障礙的.
但是戴了眼鏡.

$^{281}$大多數我想做的事.
我都可以正正常常地做.
又例如農人未必每個人都覺得.
用手語去溝通是一種障礙.
手語可以視為另一種言語.
當中承載著一種獨特的文化.
所以希望我們記得.
撐則不一定等於障礙.
困難未必一定帶來困迫.
我們要尊重和聆聽.
當事人自己的體會和經驗.
不可以將自己在某個處境和經驗裡的感受.
去強行套在別人的身上.
香港有不少教會都有實際行動.
去回應不同能力人士的情況.
為了減少他們殘疾處境帶來的障礙.
例如香港有三十多間教會.
每個星期都會安排將他們的程序表.
詩歌歌詞送去視障人士呼音中心.
預先為他們製作點字板.
好讓有視力障礙的參加者.
可以拿著點字板一起唱詩參與敬拜.
亦有十間教會會為農人提供現場手語翻譯.
坐輪椅翻教會是一個挑戰.
因為香港地方狹窄.
很多時都有樓梯.
不過教會在可行的情況下.
有些會安排輪椅通道.
甚至在會場內畫出指定的輪椅區.
不擺椅子.
讓輪椅人士感受到他們受歡迎.
教會予他們一份.
另外有教會會與機構合作.
探討如何回應不同特殊群體的實際情況.
例如特殊學習需要的小朋友.
或者情緒精神病病患康復者等等.
這些外在的硬件和安排.
若果有心做也不難做到.
不過往往這群朋友經歷困迫的.
其中一個核心問題.

$^{321}$就是社會甚至教會.
習慣了只看到殘疾的標籤.
而看不到裡面的人.
以下是一個什麼社會現象呢?.
用一個標籤去定斷一個人.
用一個人的某一部份去定義和判斷他.
這個現象是什麼?.
你也知道了.
這種就是歧視.
不單止殘疾人士經歷歧視.
如果有人認為某種性別的人.
一定不適合做領導的工作.
那就是性別歧視.
如果心裡有一個預設.
認為某某種族的人一定很不衛生.
又或者一定不是好人.
這些想法就等於種族歧視.
《開普敦承諾》第2項宣言.
其中宣言的第二部份.
第2項第4點.
是關於基督給殘疾人的平安.
裡面是呼籲世界各地的基督徒.
排除文化上對殘疾人的固有觀念.
因為正如司徒保羅所教導的.
從今以後不憑著外貌認人.
我們都是按著神的形象所做.
有神所賜的恩賜.
用於服侍他.
憑外貌認人即是歧視.
原來歧視一個人.
等於看不到他按著神的形象被做.
亦不尊重他背後的創造主.
咦 教人不要歧視?.
這個亦有野學的課堂.
為什麼變成好像教育電視一樣?.
這麼簡單的道理.
花了這麼多聖經章節去講.
就連21世紀樂喪會議.
又是講這些基礎的東西.
幾千個教會的宣教領袖會議裡.

$^{361}$居然要提這些教育電視的議題.
台上講的道理可以天下無敵.
不過呢.
這就是我們的現況.
其實我們也不會有心要歧視別人.
不過今天想和大家看看問題的複雜性.
要知道有文化神學制度幾方面.
互相鞏固的金剛索.
影響我們的思維判斷和言語行為.
不是簡單地說一句你們要彼此相愛.
我們就可以自然地跨越衝破這些東西.
我們口裡承認.
心裡相信主耶穌是我們救主的那一刻.
並不是我們悔改歸主的終點.
路還很長.
我們還要藉著聖經的話語幫助我們.
光照我們.
要好像照鏡子一樣看到自己的本相.
看到自己如何被壓迫需要被釋放.
同時壓迫別人需要悔改.
我們是在一個縱容歧視的文化下長大的.
其實我們不用刻意去瞻掩世俗.
我們根本就是在這個大艷光長大的.
我們的思維基本的設定.
自然就會以世俗的眼光判斷人.
按著外貌待人.
小時候我們學會如何用一些外在的標籤去判斷一個人.
也學會了很注重外表和包裝.
在香港這個社會.
什麼叫做外表包裝OK呢?.
要表現得有能力.
要成功.
成功也是一個很看得的包裝.
所以我們會崇拜成功輕視弱勢.
社會大眾 族群 家庭 甚至自己.
都會用這些東西去量度人的價值.
回看殘疾的處境.
殘疾人士因為各種能力的挑戰和限制.
用這樣的標準.
他們自然會輸在起跑線.

$^{401}$在這個文化環境下.
殉途如果不提高警覺.
一個不留神.
我們會不加思索地.
接受了社會對殘疾人士既有的概念.
忘記了一個人的價值.
不在於他有多有能力 多成功.
舉一個簡單的例子你會更加明白.
教會一般都很在意.
自己的年青人的事工發展成怎樣.
如果突然發覺年青人的數目驟降.
就會大為緊張.
會開會 想一些方法.
例如特別請委員 程導同工去發展這方面.
又或者主動改變自己的環境空間.
讓年青人來到教會覺得舒服一點.
這樣做是沒有問題的.
不過有沒有教會會問.
為什麼這麼少殘疾人士在我們當中呢.
不行了 我們要做些事.
讓我們改善一下自己的空間.
讓他們來到覺得受歡迎一點.
你的經歷是不是這樣呢.
背後又反映了什麼呢.
縱容歧視的文化另外一方面.
都會孕育出仇外的心理.
甚至是排外主義.
Xenophobia.
意思是面對和自己不同的人.
自然的反應是會有恐懼.
批判 排擠.
甚至會發出有敵意的訊息.
不少帶著有特殊需要的家長.
他們經常覺得一件很懊惱的事情.
就是每當他們帶小朋友出街的時候.
很多時候都要面對鄰居 街坊.
甚或陌生人對他們的目光.
那些批判的目光.
甚至驚恐的表現.
或者攻擊性的言語.

$^{441}$其實歧視文化是慢慢潛移默化.
從小從小學回來的.
試想一下 如果你小時候出街.
你媽媽一見到某一類人.
就拉你走遠一點 避開一點.
那一種教育.
比起你在教會聽到你.
可能更加有影響力.
Sympathophobia.
Sympathophobia 出於恐懼.
是一種擺錯位的恐懼.
這種恐懼令我們忘記了.
我們排擠歧視的小子.
其實他是按著神形象所做的.
他背後有一位主.
而這位主是不會按著外貌去偏待人.
也會審判那些欺負弱小的人.
他才是真正值得我們去敬畏的.
不過 要說一句公道的話.
有殘疾的人.
雖然他經歷了被人歧視.
被人嫌棄的困苦.
但不代表他不會嫌棄其他人.
歧視其他人.
不同程度或者不同類別的殘疾人士.
都有可能會互相歧視.
所以 在一個縱容歧視的文化下.
成長和存活.
我們都是受害者 壓迫者和共犯.
其實 當我們歧視別人的時候.
我們也會歧視自己.
或者我們不以為意.
很多時候 令自己困苦壓迫的聲音.
都是在自己的思想裡面發出的.
我們往往會以社會的角度去評核自己.
社會崇拜成功.
我們就會鞭策自己.
做多點 做好點.
我們相信世界對我們的評估.
多過相信聖經的說話.

$^{481}$我們很難真心地相信神愛世人.
不只是愛那些達標的.
值得他愛的世人.
神愛我們總是要有些附加條件.
才可信的.
原來我們自己的思想信念.
成為了我們自己人生的困壁.
我們也有份壓迫自己.
真是不值得.
困壁多面體.
文化會影響我們的神學和制度.
有時候我們聚焦去看.
某一個處境令我們陷入困壁裡.
我們以為拿走這個處境.
踢走那些壞人.
我們就會得到釋放.
不過事情是複雜很多的.
在一個扭曲罪惡的世界裡.
文化制度甚至神學.
環環緊扣.
都是在墮陷的狀態下.
一個縱容歧視的文化.
會衍生一套製造障礙的神學.
舉個例子.
從來在教會聽人講見證.
都是病得醫治去讚美神.
見證病得醫治本身是沒有問題的.
但當我們認為得醫治.
才是神祝福的記號.
才有見證可以講.
我們就令到沒有經歷醫治的殘疾人士.
面對極大的困苦.
又或者當我們認為.
需要上上起落才是一個好的見證.
我們就會將困苦的人.
發自內心哀動的祈禱聲去滅聲.
縱容歧視的文化.
又縱容了我們對自己對立的人.
妖魔化.
對我們將他們的壓迫合理化.

$^{521}$這些扭曲的文化和神學.
亦都會支持一個欺壓弱小的制度.
因為當我們看不到.
這些少數的殘疾人本身的價值的時候.
你就會對於他們在制度底下.
所受的壓迫和損害.
變得無動於衷.
這些長期沒有受到正視的扭曲和壓迫.
有一天會爆發成為社會裡面的衝突.
近期因為弗洛伊德事件.
在美國各地爆發的抗議和示威潮.
正是揭示了社會裡面.
根深蒂固的種族歧視這個問題.
歧視令人看不到對方是一個人.
不要以為基督徒有免疫力.
當我們縱容歧視文化.
將人標籤化的時候.
有一天我們可能再也看不到自己的惡.
單單在某個處境的環切面.
可能會壁壘分明一點.
受害者和壓迫者之間.
可以畫到一條很清晰的界線.
但是在這個被罪惡扭曲的世界裡.
每一個人都是受害者,壓迫者和共犯.
其實我們都需要上帝的醫治和釋放.
在困壁裡面壓迫者是惡人.
但是這個會不會是其中一個標籤呢?.
有權有勢的人.
如果從來都沒有人因為他是他.
而尊重聆聽他的話.
他又從何去學會聆聽尊重其他人呢?.
無論一個人多有權力.
未曾經歷過無條件的愛.
未曾經歷過神的愛.
就好像袋裡面沒有錢的乞丐一樣.
你怎可能要求他可以分錢給你呢?.
你為何會期望他可以放下自己去為你著想呢?.
在社會的撕裂,疫情的挑戰裡面.
我們出於道德力量的愛.
很快就會見底,會乾塘.

$^{561}$今天我們可能是受害者.
但是轉過頭來.
很快我們可以成為別人傷害的源頭.
我們都需要上帝的醫治和釋放.
醫治釋放可以去那裡找呢?.
去到這裡我們都是時候謙卑下來.
告訴神我們真的搞不定這個世界.
搞不定自己.
因為根據你看到這個局面.
我們每一個人就好像一個從大染缸出來的小孩.
自己都弄得糊裡啖道.
我們很想按照聖經的吩咐.
做我們應該做的事.
過合乎聖徒體統的生活.
但是我們就好像一部搭錯線的機器一樣.
我們做什麼,說什麼.
都會為人帶來困迫.
但是偏偏我們的信仰預設了我們.
是一個受差遣的群體.
正正要面向他者.
我真是苦.
誰能夠我脫離自己呢?.
我們極需要主耶穌的醫治和釋放.
我們對釋放的概念,期盼和行動.
都要被上帝去轉化.
否則有一天.
就算讓我們成功地用自己的方法.
改變眼前的困局.
但是可能會令我們和身邊的人.
陷入更深的困迫裡面.
帶著種子之知名.
接著下去金堂的第二部分.
去說釋放.
我們會看耶穌模式的釋放.
如何回應文化,神學和制度的扭曲.
第一點.
「道成肉身成為殘障」.
有沒有想過.
當耶穌道成肉身成為人的時候.
其實祂是解除了自己.

$^{601}$意思是祂自願成為殘疾人士.
你會說:不是啊.
耶穌可以走路.
視力,聽力,言語能力和智力都沒有問題.
但是你不要忘記.
祂是神.
祂的常態和我們不同.
當耶穌進入人的處境的時候.
祂就接受了一種非比尋常的限制.
這種自限.
你可以看為祂讓自己成為殘障.
比尼·聖繆指出.
耶穌道成肉身.
本來就是一個上帝自限的行動.
祂更加人身,殘疾和人性的整全並沒有衝突.
基督的自限就是最好的範例.
如果用這個向度再看路加福音第四章.
當耶穌打開.
以塞亞書宣告神月立人的欺凌已經來到的時候.
祂是說一個怎樣的欺凌呢?.
一種怎樣的釋放和醫治呢?.
我們的文化,我們的神學.
教我們劃清界線,敵我分明.
使我們期待一種拿走限制的釋放.
不過耶穌選擇了逆文化去回應我們.
上帝展示的釋放.
不單止不會被處境限制所局限.
而且還是在耶穌道成肉身.
選擇和我們的限制連結當中去展示出來.
釋放福音的能力.
道成肉身本身就是一種關係的醫治和釋放.
跨越了原本劃得清清楚楚的界線.
全然聖潔的神沒有嫌棄我們.
按照祂的聖潔.
人類是全然可憎.
但是祂居然沒有和我們割席.
還選擇來到我們當中成為我們一分子.
這件事本身已經是對我們.
人類最核心的醫治和釋放.
今天你接不接受耶穌這種模式的釋放和醫治呢?.

$^{641}$祂不嫌棄你.
對你是不是已經足夠呢?.
耶穌在意你.
祂來到尋找你.
有祂不離不棄的愛.
對你是不是已經足夠呢?.
耶穌模式的釋放.
第二方面.
醫治殘疾 整全釋放.
以下會看書中記載耶穌醫治殘疾的事件.
我們會集中看其中四個殘疾類別的醫治.
第一是身體活動能力受限制.
第二視覺有困難.
第三聽覺有困難.
第四言語能力有困難.
如果你有記性.
你會看到剛才我們在香港殘疾人士類別列出來的頭四項.
其實藍色字寫的五項.
都可以提供很多反思學習的空間.
不過今天的時間有限.
看耶穌醫治殘疾的事件裡.
我們很多時候只會注意耶穌如何醫治了那個人.
令他可以脫離殘疾的處境.
沒錯 這是釋放的其中一個方面.
以下我們會看看.
在這些事件裡的敘述裡.
耶穌如何回應人類在扭曲的文化 神學和制度裡.
所需要的種種方面的釋放.
扭曲歧視的文化.
會令我們只看到標籤 看不到人.
耶穌展示他如何聽到 看到每一個人.
製造障礙的神學.
將人強行塞在條文之下.
甚至將神也強行屈就在一套人所定的宗教條文框架當中.
耶穌的醫治事件裡.
應對這種扭曲的神學.
要展示高舉條文之上的那位神.
我們也會看耶穌在醫治的事件裡.
如何為制度下被壓迫 被邊緣化的弱者解困和發聲.
從這幾個方面我們學習什麼叫做整全釋放.

$^{681}$縱使我們知道今天不會在每一個情況裡都看到這些東西.
但是我們學習去辨識甚至去參與上帝在做的事情.
以至我們在一個已然未然的世界裡.
可以仍然有力去前行.
在困迫張力當中 知道上主仍然在.
他的國度仍然全速地去推進.
我們首先看第一組.
耶穌醫治身體活動能力受限制的人.
包括在馬福音2章 亞伯隆那裡.
有四個朋友拆了屋頂把他吊下來的毯子.
然後我們會看馬可福音3章.
在會堂裡敷乾了一隻手的人.
路加福音13章 會堂裡駝背的女人.
還有我們會看楊福音5章.
在不士大池邊一個病了38年行動不方便的人.
時間關係我會將我對經文的觀察整合和大家分享.
我不會仔細和大家看逐段的經文細節.
留待你自己慢慢研究和發掘.
首先我們看耶穌如何應對扭曲的歧視文化.
這幾段經文裡有三個情況.
一是耶穌主動接觸需要醫治的人.
試想像如果你是一個領袖.
你想為這個時代帶來一些改變.
恐怕你不會像耶穌那樣去找那些老弱傷殘.
你更加沒有時間去逐個問他們.
你需要什麼?你想不想好起來?.
四個人當中有兩個人的活動能力是嚴重受到限制的.
他都很依賴身邊的人去照顧幫助.
嚴重肢體活動的能力受到限制是一個外顯的殘疾.
而這個標籤絕對可以掩蓋你整個人的性格.
和你各方面的獨特你的能力你的需要.
也都掩蓋了在你身邊的人的需要.
馬福音2記載耶穌看到坦子身邊的照顧者.
第五節說耶穌見到他們的信心.
就對坦子說「小子,你的罪赦了」.
耶穌看見他們的信心.
這個小小的情節絕對是有很大的意義.
今天有很多的殘疾人士.
他們的照顧者都是默默地去服事.
社會上對他們的支援極之少.

$^{721}$也都不會有心去看他們的需要.
耶穌見到他們還肯定他們的信心.
反過來看那位躺在不士大祠旁邊的.
病了三十八年的人.
他的痛苦不單止是他的病.
而是他身邊沒有人幫他.
耶穌沒有立刻去醫好他.
而是首先去聆聽他內心裡沒有人幫助的困苦.
在四個敘述裡.
其中有兩個是提及赦罪或者叫他不要再犯罪.
一個是在身體的醫治之前提及.
而另一個是在身體醫治之後才提及罪的問題.
耶穌的整全釋放不是跟著一張藥單.
照單去整理這麼多種藥.
也不是跟著食譜去煮食.
一定要先放這個再放那個.
耶穌的釋放是度身訂做.
他真的看得到,聽得到.
殘疾背後的人的狀態和需要.
耶穌看得到殘疾標籤背後的人.
他醫治身體殘疾的痛事.
也回應這個人身心,社,靈,群.
所需要的釋放.
接著我們看耶穌如何應對製造障礙的神學.
在經文裡觀察到.
耶穌不單止是醫治釋放那個人.
他還在意地展示高舉宗教條文以上的神.
四個敘述裡有兩個沒有提及社罪.
是否代表另外兩個人不需要社罪呢?.
是否等於他的殘疾和罪沒有關係呢?.
經文沒有說.
甚至連房頂掉下來的毯子.
耶穌雖然說小子你的罪社了.
之後再治好它.
也不等於他被解除是因為他犯罪.
聖經真的沒有這樣說.
不過我們很喜歡自己砌出方程式.
弄來弄去就變成壞鬼神學.
或規矩去框著神和壓迫其他人.
沒錯,世人都犯了罪.

$^{761}$都需要從罪的綑綁裡得釋放.
但耶穌不是每次都第一時間叫人認罪.
耶穌模式的釋放.
表達的是祂才是釋放的主.
有祂的判斷,主權,時間.
耶穌要展示,要高舉的是條文之上的神.
祂要叫人知道人子有社罪的權柄.
四段記載裡有三段是在安息日裡治病的.
耶穌似乎要專挑戰安息日的條文.
當中關係到主權的問題.
祂要讓人知道祂才是安息日的主.
也牽涉到釋法的問題.
究竟安息日有什麼可以做.
有什麼不可以做呢?.
是建基於如何詮釋安息日的真正意義.
最後我們看耶穌如何面對欺壓弱小的制度.
如何為弱者解困發聲.
《路加方》13章裡.
當耶穌的敵人為安息日治病而為難耶穌的時候.
耶穌為被鬼附的女人發聲.
祂提醒大家.
不要忘記這個女人是亞伯拉罕的後裔.
為什麼不可以在安息日解開她的綑綁.
耶穌沒有說要廢除安息日的制度.
祂要挑戰的是實踐這些制度的人.
那種欺壓弱小的手段和態度.
祂沒有忘記被制度遺忘的人.
並且祂毫不猶豫地為這個女人發聲.
另外在《不事大辭》有一個制度.
是一個沒有用白字黑字寫出來的制度.
這個制度的功能是用來輪候醫療神蹟.
這個制度很簡單.
誰反應最快.
當池裡的水一動的時候.
誰最快跳進水裡.
誰就得到醫治.
醫治是一個因點.
這裡還有一個神蹟.
只不過原來神蹟的運作.
也會有不公平的制度.

$^{801}$要鬥快的話.
就差點病得嚴重.
就吃虧了.
或者像這個人孤苦無依.
沒有人幫他.
就只有一個等字.
耶穌如何應對這件事呢?.
祂有沒有改變這個制度呢?.
不如從此以後排隊拿籌吧.
或者找醫務社工來逐個見一見.
看誰需要大一點就醫誰吧.
不好了 不如大方一點.
On the house 今天慶者有份.
全部人都得到醫治.
耶穌沒有直接這樣做.
祂沒有直接處理這個制度的情況.
不過我們看到的就是.
祂沒有忘記這個在制度裡被遺忘的人.
祂看見他走過去聽他的故事.
耶穌施行了跨越制度的因點.
祂醫好了在這個制度裡.
永遠都輪不到他得醫治的人.
今天世界有很多聲音告訴你.
你應該如何看待自己.
或者如何看待某一個人.
求神幫助我們聽到他的判斷.
相信他的時間.
讓我們和我們的社群.
經歷新心靈的醫治和釋放.
亦讓世界藉著我們這個殘障.
得醫治的群體.
去經歷一種耶穌模式的釋放.
一路去預備的時候.
經常想起我一個好朋友.
他叫林路德.
路德十幾歲就突然之間癱瘓了.
頸以下全身癱瘓.
從此他預輪以為五.
當時他剛剛信主.
他不斷這樣祈禱.

$^{841}$雖然他沒有經歷到身體的醫治.
但是路德一生卻為很多人帶來新心靈的醫治.
他創辦了回聲谷相見呼音協會.
設立熱線去聆聽其他殘疾人士的心聲.
和他們同行為他們祈禱.
雖然按照制度的條文.
沒有認可到他目者的身份.
不過也沒有了他繼續參與.
耶穌基督施行的醫治和釋放.
路德後來拍拖結婚.
他太太施慧也是我的好朋友.
路德分享有個朋友問他.
林路德,為何你一個嚴重殘障的人.
要和一個正常人拍拖結婚呢?.
你不覺得會連累他嗎?.
他要幫你做所有的家務.
去服侍你.
而你卻什麼也幫不到他.
如果你是路德,你會怎樣回答?.
路德反問這個朋友.
為何你會這樣問我呢?.
究竟你怎樣看婚姻呢?.
如果你只是想對方或自己.
在這個關係裡提供什麼功能的話.
那你乾脆請一個菲傭就行了.
你呢?你會怎樣想呢?.
如果你的女兒告訴你.
喜歡了一個坐輪椅的男生.
你會不會反對他呢?.
無論你同不同意路德對於婚姻的想法.
你不得不承認.
路德起碼沒有歧視自己.
如果讓你今天突然失去了行動的能力.
你會不會歧視自己,嫌棄自己呢?.
要小心.
歧視自己帶來的困迫.
往往是最棘手,最難搞的.
路德的生命故事裡.
我們看不到像這幾段經文所說的身體得醫治.
但是我們看到耶穌在他身上.

$^{881}$顯示各種層次的釋放.
而那種不被醫治的狀態.
也在挑戰他身邊的人重新思考生命.
我們看第二組耶穌醫治視覺有困難的人.
這裡列出的有.
《馬可福音》八章《白菜大》的核子.
《楊福音》九章《生來核眼》的人.
《馬可福音》十章《耶利哥》的核子.
叫做巴底馬.
在《馬太福音》二十章和《路加福音》十八章.
有類似的記載.
可能是說同一件事.
這裡和第一組的經文有類似的觀察.
關於耶穌應對文化,神學和制度.
我就不再重複了.
我特別想提的是.
當中《楊福音》九章記載一個生來核眼的人.
引發了門徒去問.
拉比 者人生來是核眼的.
是誰犯了罪呢?.
是這個人 是他父母嗎?.
這裡是一個很經典的製造壓迫和障礙的神學.
耶穌在這裡為這個人發聲.
祂說兩人都不是.
而是要在他身上顯出神的作為.
《路加福音》九章這段敘述.
最後帶出一個很特別的問題.
就是究竟誰才是盲呢?.
耶穌說:我審判來到這個世界.
不能看見的可以看見.
看到的反而成為核眼.
法利賽人聽到言外之音.
說:難道我們也盲嗎?.
耶穌對他們說:你們如果是盲的就沒有罪.
但現在你們說我能夠看見.
所以你們的罪還在.
如果今天主流的論述將你標籤為不正常.
有缺陷請你不要太過氣餒.
謙卑在上帝面前的人.
缺陷還能醫治.

$^{921}$只會指責人的.
看不到自己的缺陷才是最可憐.
這個段落說到視力有障礙的人.
我想介紹大家認識一位信道者.
他在一個封閉的地方信主和傳道.
大家都稱他為約翰.
他是當地第一位信主的傳聖名弟兄.
但他信主的是盲孝的老師.
約翰很厲害.
他很快便摸完整本聖經.
為何說摸完呢?.
因為他是用點字來看聖經.
所以他不是用眼睛看.
而是用手摸那本聖經.
一般信徒看聖經都有很多限制.
因為始終是一個封閉的國家.
不過約翰就沒有這個限制.
因為周圍沒有人知道他拿著那本書來看.
因為他拿著那本點字書.
所以他便可以明目張膽地讀聖經.
想起來也挺吊詭的.
因為他是盲孝.
所以他可以很自由地看聖經.
困難的處境未必等於全部都是困壁.
用少許創意和幽默感.
亦可以創建另一種海闊天空.
約翰很有語言天份.
他懂得很多國的語言.
他把新約聖經翻譯成自己的文字.
有人想供他去外國讀神學.
讓他可以翻譯舊約.
不過約翰說我要留在自己的地方.
最後他被一些極端分子所殺害.
約翰超越了殘障的限制.
亦甘願承受信仰的壓迫.
以下是約翰寫的一首詩歌.
在歌詞裡你會明白.
其實有些什麼在他心裡承托著他呢.
來呀來呀 來到主耶穌面前.
不要怕 不要怕世上的人.

$^{961}$神是教會的看守者 保護者.
不要怕 不要怕世上的人.
約翰雖然對眼看不到.
但他心裡看到並專注那位.
有能力看顧保護教會的人.
但願約翰的信心鼓勵我們.
接著我們看第三組.
耶穌醫治聽覺有困難.
言語能力有困難的人.
因為這兩個類別聖經記載不多.
以下情況是兩方面的困難.
就是馬可福音第七章記載.
以龍切結的人.
和馬可福音九章記載.
被龍牙鬼所附的孩子.
請大家留意第七章記載.
耶穌是帶人丸來人群才醫治他.
之後吩咐他千萬不要到處說.
但聖經說越吩咐他就越多說.
當時眾人的反應是憤外嬉其.
有聖經學者嘗試解釋他們為何憤外嬉其.
原來當你看二冊書三十五章三至六節時.
講到幾個指向神必定來報仇的標記.
神必定來報仇.
必定施行極大的報應.
必定來拯救以色列民.
接著有下一個清單.
就是「瞎子的眼必憎害.
聾子的耳必開痛.
瘸子必跳躍著鹿.
啞巴的舌頭必能割腸」.
所以以龍切結的人得到醫治.
大家是否會憤外興奮呢.
因為突然間這份清單多了兩個「剔」.
又說報仇的日子的標記來.
耶穌很快便會來懲罰侵略我們的外邦人.
解放我們的民族.
眾人若因此而憤外興奮.
耶穌可能令當時很多人失望.
耶穌醫治這個以龍結舌的人.

$^{1001}$他的角度是怎樣呢.
我們從有記載的去看.
經文說耶穌舉目望天嘆息.
望天他仰望天父.
他感受到眼前這個人.
有時候我們會想多了.
我們心目中對於釋放的期盼.
籠罩著我們的思想.
甚至阻礙了我們了解.
耶穌所帶來的釋放.
這個釋放其中必經的道路.
是十字架的道路.
是進入困壁的道路.
另外《瑪學福音》九章.
說到被鬼附的小孩.
他醫治的過程很轉折.
他父親來找耶穌.
因為耶穌的門徒處事不妥.
門徒卻是不能.
耶穌對這個父親說.
「你若能信,再信的人凡事都能」.
這個父親很老實.
他承認「信心我也有,但不足夠」.
「求主你幫我」.
或許在這麼多醫治的事件中.
我們也在呼召自己的不足.
學會如何轉向神.
求祂幫助.
我們藉著四個殘疾類別的醫治事件.
看到耶穌揭示的.
那些隱藏壓迫我們的扭曲和困壁.
施行醫治釋放這些事件背後.
其實也在呼召我們悔改.
轉向和求助.
去認清我們在各個層次上的殘疾.
承認我們不但不能自救.
未必能幫助別人.
甚至連求醫治的信心也不足夠.
接著我們看耶穌模式的釋放第三方面.
「以馬來尼困壁同行」.

$^{1041}$釋放其中必經之路是十字架.
聖經記載耶穌是一位常經幽幻.
很熟悉困壁的一位主.
他經歷人在扭曲世界的苦況.
他看到人的苦況.
他感受到他會動持心.
他會哀動.
他會大聲哭 會生氣 發脾氣.
會推倒一些東西.
甚至會質問神為何要離棄他.
這位復活主不是高高在上.
揮一揮神仙棒去釋放我們.
又或者舉起一支拳杖去猜險我們的生命.
他是那位能夠體恤我們軟弱的同行者.
他甚至凡事受過私貪.
和我們一樣.
以馬來尼的神經歷過被人厭棄.
針對 誣衊 出賣 鞭打 囚禁 甚至處死.
馬福音14章記載耶穌去赫西瑪利園祈禱那一幕.
他帶著彼得瓦國約翰一起去.
經文記載 耶穌驚恐起來 極其難過.
對門徒說:我心裡甚至憂傷 幾乎要死.
這裡用的字眼很重.
是用來形容情緒極其低落或者不穩定的字眼.
耶穌面向十字架.
他也會驚恐 極其難過.
他的心靈痛苦到快要死.
我們說 耶穌是那位釋放我們.
猜險我們和我們同在 找我們代求的神.
有時我們忘記了.
我們信的帶來的釋放是耶穌.
他的釋放是這樣的.
神閱立人的欺憐是說這位尼塞亞.
抱著我們這群罪人不放 情願被我們釘死.
求天父放過我們.
在十字架上的耶穌沒有掙扎.
沒有釋放自己.
直到他斷氣那一刻 他仍然在困壁當中.
「以馬來尼 神在你困壁當中與你同在 休戚與共」.
本身就是福音.

$^{1081}$這個福音對你是否足夠呢?.
耶穌在十字架上成就的釋放.
可以被視為被動式.
最終是神叫他從死裡復活.
另一方面 亦是主動式的抗爭.
耶穌在最不可能的情況下.
在神的形象都被污衊的情況下.
仍然堅持活出自己的本相.
亦即是神就是愛這個本相.
這種愛一方面疾惡如仇.
不會徇私枉法.
但卻願意讓最邪惡的人有機會悔改 得釋放.
耶穌說「父怎讓差我 我也照樣差你」.
照樣差我?.
不可能做得到.
耶穌是神 祂可以 我不可以.
恭喜你.
你承認自己不可能像耶穌那樣堅持活出神的形象.
那你就有希望了.
就像《醫治神蹟》裡向耶穌求醫治.
就得到祂為你度身訂做的釋放.
因為只有耶穌可以釋放我們裡面.
按著神的形象做的本相.
我們可以在最不可能的情況下.
仍然堅持活出這種生命的質素.
耶穌說「你們在世上會有苦難.
不要以為稀奇.
但也不用太害怕 因為祂已經勝過世界」.
靈子們 我們今天背負著十字架.
經歷驚恐 憂傷 無力和困迫的時候.
請你明白這是旅程的一部分.
不過也請你記住 面對十字架的道路.
耶穌自己也需要停下來.
在黑色夢裡與神對話.
接受天使的幫助 加添祂的力量.
耶穌也需要同伴的支援和守望.
雖然祂身邊的人也令祂失望.
當身邊的人也體會不到我們的情況.
也不能支援自己.
與自己一起去警醒禱告的時候.

$^{1121}$請你好好照顧自己.
要守住心靈裡與神獨處對話的空間.
讓祂親自加力量給你.
這裡我們總結一下.
今天我們從困迫與釋放的張力開始.
施洗右向的提問.
表達著神的僕人在困迫裡.
苦苦等候神的大能彰顯.
今天你正在承受什麼困迫呢?.
你有沒有將你的提問帶到主面前呢?.
藉著殘疾的處境.
讓我們看到文化,神學和制度如何環環緊扣.
在扭曲和墮陷裡構成四方八面的困迫.
我們又看到自己在當中既是受害者.
也同時是壓迫者.
當我們認為身邊的人都好像又盲又聾.
沒有行動能力的時候.
我們是否也懂得承認自己極其量都是弱視.
偏聽和四肢不協調呢?.
我們都極需要上帝的醫治和釋放.
今天從殘疾角度學習耶穌釋放的模式.
從道成肉身成為殘障.
到耶穌福音書裡的醫治事件.
帶來對殘疾人的整全釋放.
都宣告神對我們的愛是無可動搖的.
「以馬內利,神與困迫人同行,憂戚與共」.
本身就是福音.
耶穌這個愛對你有沒有吸引力呢?.
耶穌在困迫和你同行.
對你是否已經足夠呢?.
十字架上耶穌模式的釋放.
是一種啟動和邀請.
差應主邀請我們經歷醫治.
在祂的醫治裡釋放我們按著神形象做的本相.
我們心裡有沒有渴望得到耶穌這個模式的釋放呢?.
今天聽了一個小時.
聽完你是更加困迫還是有些釋放呢?.
我邀請大家在神面前安靜和神傾下話題.
將你內心的張力向祂傾訴.
等候祂為你加力.

$^{1161}$我的期盼是在大時代的信徒.
面對張力裡面經歷醫治.
帶著釋放的心靈去迎向生命歷程裡種種的困迫.
也向世界展示什麼叫釋放.
其實香港人很頑強的.
轉數很快的.
耐力也很高.
這年多以來面對各方面的挑戰.
我們的盼望和出路的見機在什麼上面呢?.
要知道我們什麼時候看到自己能力的盡頭.
轉向耶穌求祂幫助醫治釋放的時候.
你總會看到祂一直在你身邊等著你.
耶穌說「父怎樣猜我,我也照樣猜險你們」.
我們的人生只活一次.
願我們都認清誰是我們的創造主.
認清我們聽誰的猜險.
讓我們可以有力量在這個扭曲的世界裡.
活出我們被造的本神.
頭主幫助我們.
多謝您的觀看.
(字幕製作:貝爾).
(字幕由 Amara.org 社群提供).
\newpage



\section{}
\label{sec:KKN4CX_CTEM}
\textbf{【疫有嘢學 │ 延SUN在線】談判買地葬妻:研讀創廿三章|朱光華博士}
\newline
\newline
連結: \href{https://youtube.com/watch?v=KKN4CX-CTEM}{\texttt{ https://youtube.com/watch?v=KKN4CX-CTEM}} ~~~~ 語音日期: 2020-05-21 
\newline
\newline
\hyperref[sec:AHC7wVTdk1o]{\small{< < < PREV SERMON < < <}}
~
\hyperref[sec:index]{\small{[返主目錄]}}
~
\hyperref[sec:G6y4aNW1WhY]{\small{> > > NEXT SERMON > > >}}
\newline
\newline
$^{1}$主席.
各位同學平安.
本集的「亦有ye 學 延伸再現」.
是由忠臣副院長朱光華博士負責.
我很喜歡稱呼他為朱牧師.
因為他既是忠臣實踐科的助理教授.
他亦有很多年牧養經驗的一位牧者.
我覺得他跟任何年紀的人都很投契.
朱牧師的負擔是要去建立新一代的牧者和信徒領袖.
所以他今天說的聖經課題.
談判買地壯妻 研讀《創世紀》23章.
其實都是跟我們分享他自己在這方面的牧養體驗.
順帶一提.
四月份的「亦有ye 學」的四個課堂已經配上字幕.
亦都重新放在忠臣的Youtube Channel那裡.
歡迎大家隨時再去重溫.
另外七至九月的延伸課程.
現在已經接受報名.
詳細可以瀏覽忠臣的網頁.
亦都可以留意今天課堂結束之後的宣傳資料.
就將以下的時間交給朱光華博士.
願主再次藉他自己的說話.
透過朱牧師的分享.
激勵及造就我們.
大家好.
歡迎大家觀看「亦有ye 學」延伸再線.
隔著螢幕跟大家分享聖經.
別有一番滋味.
今天跟大家研讀的一段經文是.
《創世紀》第23章.
我給的題目是「談判買地壯妻」.
稍為解一解題.
是提到阿巴拉罕透過談判.
買一塊墓地埋葬他的妻子撒拉.
《創世紀》第23章.
這段經文我有時會忽略了.
我估計很多時候原因是因為.
第23章的前一章.
和第23章的後一章.
搶去了我們的注意力.

$^{41}$前一章第22章.
是阿巴拉罕獻以殺.
自然是拿了我們很多注意力.
而第24章是阿巴拉罕派他的僕人.
去為以殺物色妻子.
偏偏在第23章夾在中間的時候.
我們很多時候就不留意了.
所以今天我們一起來研讀這章聖經.
看看阿巴拉罕怎樣買一塊地.
埋葬剛剛離開人世的妻子撒拉.
大家收看《亦有ye 學》.
目的就是要學.
所以我希望大家不要只坐著.
被動地聽.
我期望大家能夠專注地.
投入地進取地.
一起研讀這段聖經.
所以我會希望大家打開聖經.
我會打經文出來.
不過我仍然希望大家打開聖經.
預備筆 預備紙或寶.
聽到重要的地方你記下來.
反覆思想.
不明白的地方你記下來.
有機會再查考一下.
所以希望大家今天能夠和這段聖經.
有交流和互動.
首先我會用和合本修訂版來讀.
大家可以打開聖經聽著我讀.
第一第二節是整段經文的引言.
或提供了經文的背景.
第一節說.
撒拉享壽一百二十七歲.
就是撒拉一生的歲數.
撒拉死在迦南地的基列亞巴.
就是希伯倫.
阿伯拉罕來哀悼撒拉.
為他哭泣.
然後第三節進入戲肉.
阿伯拉罕談判埋地.

$^{81}$第三節說.
然後阿伯拉罕起來.
離開死人面前對黑人說.
我在你們中間是五外人.
是寄居的.
請給我你們那裡的一塊墳地.
我好埋葬我的亡妻.
使她不在我的面前.
黑人回答阿伯拉罕說.
我主請聽.
你在我們中間是一位尊貴的王子.
只管在我們最好的墳地裡.
埋葬你的死人.
我們沒有人會拒絕你在他的墳地裡.
埋葬你的死人.
這是第一回合的對話.
第七節開始.
就是第二個回合.
第七節說.
於是阿伯拉罕起來.
向當地的白姓黑人下拜.
對他們說.
你們若願意讓我埋葬我的亡妻.
使她不在我的面前.
就請聽我為我求所黑的兒子.
以忽倫把他田地盡頭的墨比拉洞賣給我.
他可以按照足價錢賣給我.
作為我在你們中間的墳地.
第十節.
那時以忽倫正坐在黑人中間.
黑人以忽倫就回答阿伯拉罕說.
給所有出入城門的黑人聽.
不 我主請聽.
我要把這塊田送給你.
連田間的洞也送給你.
在我同族的眼前都給你.
讓你埋葬你的死人.
十二節.
就是第三個回合.
十二節說.

$^{121}$阿伯拉罕就在當地的白姓面前下拜.
對以忽倫說.
也當給當地的白姓說.
你若應允請你聽我.
我要把田和田價錢給你.
請你收下.
我就在那裡埋葬我的死人.
十四節.
以忽倫回答阿伯拉罕說.
我主請聽.
四百射克勒銀子的地.
在你我中間算什麼呢?.
只管埋葬你的死人吧.
阿伯拉罕聽從了以忽倫.
阿伯拉罕就照著他說給黑人聽的.
把買賣通用的銀子.
請了四百射克勒銀子給以忽倫.
十七節.
於是以忽倫把那塊位於萬里對面的麥比拉田.
和其中的洞以及田間周圍的樹木都成交了.
在所有出入門口的城門的黑人眼前.
賣給阿伯拉罕作為他的產業.
來來回回三個回合.
終於成交了.
結果如何呢?.
第十九節.
阿伯拉罕把他妻子撒拉安葬在.
迦南地萬里對面的麥比拉田間的洞裡.
萬里就是以忽倫.
從此那塊田和田間的洞.
就從黑人移交給阿伯拉罕.
作墳地的產業.
對完這段聖經.
我的好奇心驅使我問三個問題.
第一.
按道理阿伯拉罕可以三言兩語交代.
阿伯拉罕妻子撒拉死了.
阿伯拉罕為她哭泣然後埋葬她.
這樣就可以簡單交代了.
為什麼聖經用了一整章的篇幅.

$^{161}$記載阿伯拉罕埋葬一塊墓地的妻子呢?.
為什麼要用這麼長的篇幅.
來記載這件事?.
用意是什麼呢?.
這是第一個疑問.
第二個疑問就是.
整章聖經記載阿伯拉罕埋葬一塊地.
用的就是埋葬一個女人.
一個大半生不能生育的女人.
到九十歲才生一個.
或者一生人只生一個孩子的女人.
在《創世紀》的記載裡.
基本上是以男性做主導.
妻子多數只是附屬品.
為什麼作者處理這件經文的時候.
跟《聖卷》《創世紀》的精神面貌不同呢?.
這麼大出入呢?.
第三個疑問就是.
這次談判買賣的雙方.
一方是客人.
當地加拿大原居民.
是地主.
那裡是他的地頭.
另一方就是阿伯拉罕.
他是加拿大的新移民.
他的家鄉.
或者我們香港很多時候.
我們的直觀.
直觀就是.
他是在丘法拉底河和底格里斯河.
即是現在伊拉克的地方.
吳爾的地方.
移民過來的.
所以買賣雙方.
是分屬於兩個不同文化.
不同的種族.
他們是怎樣談判的呢?.
他們是怎樣做生意的呢?.
不同族裔的人怎樣來談生意呢?.
如何討價還價.

$^{201}$而結果是達至各自所需.
各取所需的一個目的呢?.
希望我們讀完這三張聖經.
我能夠回答以上的問題.
剛才我讀經文的時候.
你留意到我不是一口氣讀完.
好像是分段讀.
其實我希望大家能夠捕捉到.
經文的結構.
故事的佈局.
其實結構不是很複雜.
你看這個圖表.
第一節和第二節是引言.
是說撒拉的死.
十九二十節是說結語.
埋葬撒拉.
中間第三到十八節.
就是談判買賣那塊地的過程.
三輪三個回合.
這個就是一個很簡單的結構.
我們進入經文.
第一節.
撒拉享壽一百二十七歲.
就是撒拉一生的歲數.
享壽一百二十七歲.
其實原文不是這樣寫的.
是用一個很特別的方式.
來表達她的歲數.
她的說法反而是.
撒拉活了一百歲.
二十歲.
七歲.
一百二十.
二十二.
七歲.
創世紀一般記載男性的歲數.
記載女性的歲數.
是非常罕有的.
撒拉差不多是唯一的例外.
如果你有機會留意.

$^{241}$以撒的妻子尼伯加.
和雅各的妻子拉潔的死.
聖經只是輕輕帶過.
甚至略過.
撒拉的死卻是很仔細地記載下來.
創世紀對撒拉的死的技術.
其實很特別.
我想指出兩點.
第一.
你有機會翻查聖經.
撒拉在九十歲生了以撒之後.
創世紀沒有再提及撒拉.
就好像撒拉九十歲.
一直到一百二十歲.
這三十七年的日子是空白的.
沒有任何的作為.
這是很特別的一點.
第二點就是.
聖經記載撒拉.
只是生一個兒子以撒.
生一個兒子有什麼特別呢?.
如果你有機會翻查前一點.
第十二章最後一段.
那裡記載阿伯拉罕的兄弟拿學.
一共生了十二個兒子.
然後撒拉記載他死的時候.
一位十七歲.
其實他只有一個兒子.
拿學那一方人生了十二個.
阿伯拉罕這一方人.
就只生了以撒一個.
兩方人後裔的比例.
對比是十二比一.
兩方人的人丁的數目非常懸殊.
這麼懸殊.
好像那邊比較旺.
那邊很大福氣.
但又不是.
你看回二十二章的上半章.
提到阿伯拉罕獻以撒.

$^{281}$那件事件裡.
其實阿伯拉罕確認.
他最後確認一件事.
就是耶和華已立.
這句話不單指出.
耶和華為阿伯拉罕預備了一隻羊.
取代以撒成為祭物.
耶和華必有預備.
使耶和華對阿伯拉罕的應許.
其實正逐步實現.
阿伯拉罕將成為大國.
上帝應許的端倪正在展開.
沒錯.
撒拉死的時候.
他仍然見不到.
耶和華的應許.
百子千孫.
他仍然見不到成為大國.
不過憑著.
耶和華必預備這句話.
就已經足夠了.
經文再下去.
他這樣說.
撒拉死在迦南地的基列阿伯.
就是希伯倫.
我們讀者不太知道.
基列阿伯是什麼地方.
不要緊.
作者加了一個註腳.
就是希伯倫.
撒拉死在迦南的核心地帶.
希伯倫.
雖然撒拉死的時候.
阿伯拉罕還未承受.
這個應許之地.
但他死在耶和華應許的境界裡.
然後聖經說.
阿伯拉罕來哀悼撒拉.
為他哭泣.
阿伯拉罕進入帳篷.

$^{321}$在撒拉的遺體面前哀悼.
大家可能都知道.
在當時的背景.
大戶人家通常可以聘請專人來哀哭.
來哀悼.
特別是阿伯拉罕這些有錢人.
他絕對請得起專人來哀哭.
不過阿伯拉罕沒有這樣做.
他自己親自哀哭哀悼.
我猜我們這些在看這個節目的頂姐妹.
老夫老妻那些.
當我們其中的配偶離世的時候.
我們少不免會回憶過去.
和配偶一起的日子.
阿伯拉罕和撒拉都是.
過去大概一個世紀.
我們不知道他們何時結婚.
大概我猜想一個世紀.
一百年的夫婦關係.
他們從大河那邊.
從吾爾跟隨阿伯拉罕來到迦南.
阿伯拉罕下到埃及.
撒拉陪著他.
阿伯拉罕帶著家丁去營救羅德.
回來和撒冷王墨基西德相遇.
撒拉在那裡.
耶和華多次應去阿伯拉罕的時候.
撒拉在那裡.
阿伯拉罕帶著家人逃離.
所多瑪俄摩拉.
撒拉也同時在場.
回憶過去一幕一幕的情景.
兩個人同甘共苦.
這麼多年來.
撒拉跟隨阿伯拉罕.
完成耶和華上帝給他的召命.
或者我們說撒拉跟隨阿伯拉罕.
追夢.
尋找那更美的家鄉.
其實當年起步的時候.

$^{361}$他不知道下一步會怎樣.
總之這麼多年來.
兩個人一起走過人生的高潮.
低谷困難的日子.
一起面過家庭糾紛.
譬如和下鴿那些糾結的關係.
總之兩個人互相支持.
一步一腳印.
守一輩子.
終於走到這一步.
阿伯拉罕失去了一個生命的同行者.
一個soulmate.
差不多一個世紀.
不離不棄.
那種感情那種不捨.
你大概可以想像到.
阿伯拉罕當時的百感交集.
所以雖然聖經作者這裡只是輕輕一句.
阿伯拉罕來哀道撒拉為德哭泣.
不過我們外國來人.
我們不會輕輕放過.
默想念經文背後的心意.
經文提到阿伯拉罕哭泣完.
哀悼完.
就著手開始進行身後事.
正式進入第一輪買地的談判.
第三節開始.
阿伯拉罕才發現.
他初步提出他買地的意向.
第三節他說.
然後阿伯拉罕起來.
離開死人面前對黑人說.
我在你們中間是五外人是寄居的.
請給我你們那裡的一塊墳地.
我好埋葬我的亡妻.
使她不在我的面前.
阿伯拉罕和黑人來到磋商買地.
他在什麼地方什麼地點討論談判呢?.
祠堂?大會堂?拍賣場?.
會議室?不是.

$^{401}$如果看第十節就看到一些線索.
第十節他說.
黑人以忽倫就回答阿伯拉罕說.
給所有出入城門的黑人聽.
原來他們談判的地點是城門口.
是當時的商業活動中心.
是法律仲裁的地點.
是中環金鐘.
經文沒有提及當時阿伯拉罕有沒有預約.
因為突然間到訪.
稍後經文我們大概可以揣摩到.
阿伯拉罕是有備而來.
他不是路過偶然辦身後事.
偶然經過遇見一班黑人.
順便談兩句.
他是有目的而來的.
是特意而來的.
我相信當時談判員可以拿現金買賣時.
我估計他帶著一大箱現金來談判.
看阿伯拉罕如何打開話盒子.
第四節他說.
我在你們中間是外人是寄居的.
請給我你們那裡的一塊墳地.
阿伯拉罕很知地.
他才稱自己是外人.
是寄居的.
在這個地方誰是主誰是客.
講得足夠.
誰是莊誰是閒.
清楚定位.
然後阿伯拉罕就講明他的來意.
他表示他買地.
買一塊墳地葬他的妻子.
其實他沒有具體講明他想買哪一幅地.
但講明是買的.
他想講的重點是他要買.
不是借用不是租用.
是要買斷的.
是產業來的.
更好的翻譯是.

$^{441}$給我一塊做墳地的產業.
給我你們之間的一塊地.
給我一個積極的地位.
意思是交易後阿伯拉罕用了這塊地的業權.
雖然說是墳地.
其實當時的文化.
就不同他今天的文化所謂墓地.
今天我要墓地.
在墳場買六尺地.
然後把棺木放在地底.
然後蓋上泥土就算了.
當時的墓地其實是墓室.
或者是中國以前的墓陵.
面積很大.
可以是一個大山丘.
裡面可以安葬很多人.
好幾代人.
裡面除了擺放遺體.
棺木之外.
還可以擺放陪葬品.
擺設.
或者他生前用的武器.
個人物品.
貴重財寶等等.
阿伯拉罕在這個地方.
說明了他的來意.
他來買地的用意後.
黑人的回應是甚麼呢.
第五節.
黑人回答阿伯拉罕說.
我主請聽.
你就是我們中間市的尊貴王子.
說得很客氣.
說得很得體.
先稱呼阿伯拉罕是我主請聽.
給阿伯拉罕一個很高的評價.
你是我們中間尊貴的王子.
然後是他的回應的核心.
他怎樣說呢.
他說.

$^{481}$只管在我們最好的墳地裡.
埋葬你的死人.
我們沒有人會拒絕你在他的墳地裡.
埋葬你的死人.
表面的信息很簡單.
隨便選擇.
隨便選擇你要的地.
不用客氣.
沒有人會拒絕你的.
意思是開綠燈.
表示可以商量商量.
歡迎我們進一步的商討.
阿伯拉罕見過世面的人.
他當然聽得出.
黑人的言外之音.
第一.
我這個地方的墳地.
是上好的墳地.
意思是阿伯拉罕你識貨了.
你知道來我們這裡找墳地.
厲害.
第二就是.
普通人一般人.
我們就不會跟他談生意.
不過既然你是尊貴的王子.
大家都是當地的首輔.
僑領.
或者顯赫的家族.
我就當你是自己人.
我們可以談下去.
當然背後還有一個隱含的意思.
信息.
既然大家都有頭有臉.
你就不要出售這麼低.
不要壓價.
你開一個好的價錢.
沒有人會阻止你買一塊好的土地.
所以你看.
阿伯拉罕的對手.
很有禮貌.

$^{521}$很得體的回應.
其實都看到.
他的對手不是省油的燈.
經過第一輪的交手.
阿伯拉罕知道可以談下去.
進入第二輪.
第七節.
第七節阿伯拉罕具體說出.
他想買哪一塊地.
第七節.
阿伯拉罕起來.
向當地的百姓客人下拜.
你們若願意讓我埋葬王妻.
使她不在我面前.
就請聽我.
為我求所下的兒子爾忽倫.
把他田地盡頭麥比拉洞賣給我.
他可以按照足價賣給我.
作為我在你們中間的分地.
阿伯拉罕未說清楚.
他買地之前.
你看阿伯拉罕的誠意.
他尊重他.
他先下拜.
很有東方色彩.
反映阿伯拉罕對對手客人的文化.
有相當的了解認識.
到這個時候.
阿伯拉罕說.
他想要什麼地方.
他說為我求所下的兒子爾忽倫.
把他田地盡頭的麥比拉洞.
顯然阿伯拉罕.
似乎也做過資料搜集.
可能阿伯拉罕曾到現場.
做過視察.
看過環境.
甚至到土地註冊處.
做過查查.
知道業主是誰.

$^{561}$阿伯拉罕說.
我要的地是屬於爾忽倫的地.
那個無餘土地盡頭的麥比拉洞.
麥比拉的意思是一雙.
一對.
兩個.
大概的意思是雙洞.
一個地方.
可能那個地方的墓穴有兩個洞室.
他說他願意付足價錢.
照足市價來買.
表示他有誠意.
他不會壓價.
他志在必得.
看看對方如何回應.
第十節.
黑人爾忽倫回答阿伯拉罕.
說給所有出入城門的黑人聽.
不.
我主請聽.
我要把這塊田送給你.
連田間的洞也送給你.
在我同族的人眼前都給你.
讓你埋葬你的死人.
原來阿伯拉罕說買地的時候.
那塊地是屬於爾忽倫的.
剛好爾忽倫當時在場.
於是他現身.
由他來回答.
他如何回答呢.
他說我要把這塊田送給你.
在田間的洞也送給你.
在我同族的人眼前都給你.
讓你埋葬你的死人.
你看到英文聖經.
I give you.
三次出現.
三次給你.
如果我們睡眠不足.
聽起來就像送給我.

$^{601}$發達了.
想清楚.
阿伯拉罕剛才要求.
買地.
你要賣地給我.
現在對方回應是送給你.
是不收錢.
是免費.
不過不是賣.
不是賣斷給你.
意味著地你可以用.
可以借給你用.
可以租給你用.
不過業權仍然屬於我的.
我才是業主.
我不會想賣給你.
當然表面上說不賣.
表面上說送.
可能是談判中的所謂以退為進.
阿伯拉罕你要賣這塊地.
可以.
不過你要付出相當多的價錢.
符合我心中的底價.
你出到的底價.
我不賣給你.
否則我不會賣給你.
當然以弗林的回答.
還有另一個意思.
他意思是.
阿伯拉罕你想買一個墓穴.
買一個山洞.
以弗林怎麼回答.
他說你買那塊地.
那塊田.
我就連那塊山洞的墓穴.
都給你.
意思是.
反過來想.
你明白了.
如果你想買那塊山洞.

$^{641}$你就買山洞附近周邊的田野.
即是說.
阿伯拉罕想要的一個山洞.
但以弗林要求的.
是一個所謂捆綁式的銷售.
連地連山洞一併賣出.
意思是反正都賣了.
不如多買一點.
意味著以弗林.
他想賣多一點的錢.
買一個更好的價錢.
一個山洞的價錢.
和一幅田地的價錢.
是不同的.
當然背後還有一些考慮.
什麼是實際考慮呢.
就是你買那塊洞穴.
那個洞穴作為墓地.
你要買那片土地.
如果不是以後每年.
清明重陽來掃墓來拜山.
你不要經過我.
以弗林的屬地才可以.
其實都有這個可能性的考慮.
無論如何經過兩輪的來回對話之後.
轉入第三輪.
第三輪第十二節.
阿伯拉罕怎麼說呢.
阿伯拉罕就在當地的百姓面前下擺.
你若英文請你聽我.
我要把田的價錢給你.
請你收下.
我就在那裡埋葬我的死人.
你聽到這句說法.
到了這個地步.
阿伯拉罕其實沒有討價還價的本錢.
沒有什麼爭議力.
阿伯拉罕沒有發脾氣.
他沒有說沒得談.
阿伯拉罕在這個地步.

$^{681}$阿伯拉罕他知道他需要什麼.
他需要一塊地.
看阿伯拉罕怎麼回應.
他再一次向當地百姓下擺.
表示他的敬意和尊重.
然後說請你聽我.
聽我什麼呢.
他說我要把田的價錢給你.
言下之意.
他同意埋下全幅田地.
連同其中的墓穴.
雖然沒有說價錢多少.
不過阿伯拉罕在這個地方表示.
他會付足價錢.
其實意思在這個地方.
你看到阿伯拉罕開了一張.
沒有銀碼的支票.
你隨便開聲.
你開個價.
我都會給你.
對方怎麼回應.
來了.
第十四節.
以弗倫回答阿伯拉罕說.
我主請停四百射克勒銀之地.
在你我中間算什麼呢.
只管埋葬你的死人吧.
雖然以弗倫即將要開天殺價.
不過他仍然稱呼阿伯拉罕是我主.
厲害吧.
做生意人真是很厲害.
他終於開價了.
四百射克勒銀紙.
四百射克勒大概等於三點三公斤的銀紙.
當時來說一點都不便宜.
但最可圈可點的就是.
以弗倫還要加多一句.
在你我中間算什麼呢.
意思是隨便吧.
你和我都是大財主.

$^{721}$區區四百射克勒.
三十歲吧.
暗示他開這個價已經很便宜.
開這個價已經很大方.
半賣半送.
阿伯拉罕你不要再和我講價了.
這樣的意思.
結果怎樣呢.
阿伯拉罕當然接受.
第十六節.
阿伯拉罕聽從了以弗倫.
阿伯拉罕就照著他說給客人聽.
把買賣通用的銀紙稱了四百射克勒銀紙給以弗倫.
於是以弗倫把那塊地於萬里對面的麥比拉田和其中一棟.
以及田間周圍的樹木都成交了.
在所有出入城門的客人眼前.
賣給阿伯拉罕作為他的產業.
成交.
交.
貨銀兩銀.
講好數了.
我想我們深水清的頂尖可能會注意到.
阿伯拉罕花那麼多時間來談判.
要買這一幅地.
其實這一幅地.
是耶和華應許給他的土地.
這片土地將會是屬於他的.
遲早都是他的.
為什麼阿伯拉罕仍然用真金白銀.
購買這一幅遲早屬於他的地呢?.
這是一個我們要思考的問題.
可能看另一段聖經我們有多少線索.
創世紀第十九章.
二十九到三十年前.
那段經文提到亞國人生走到盡頭.
臨死之前.
他對他的兒子有這一番吩咐.
他又吩咐他們說.
亞國吩咐他的兒子.
我快要歸到我祖先那裡.

$^{761}$你們要將我葬在.
黑人以弗倫田間的洞裡.
與我的祖先在一處.
就是在迦南地萬里對面的.
麥比拉田的洞裡.
那田是阿伯拉罕向黑人以弗倫買來的.
作墳地的產業.
阿伯拉罕和他的妻子撒拉葬在那裡.
以撒拉和他的妻子尼伯加也葬在那裡.
我也在那裡葬了尼亞.
那塊田和田間的洞是向黑人買的.
原來在很多年之後.
葬在這個洞裡的人.
不單止撒拉還有阿伯拉罕.
還有以撒尼伯加.
還有亞國的妻子尼亞.
還有現在即將離世的亞國.
亞國將要離世.
臨終之前他吩咐.
他給兒子的遺命就是.
我都要葬在我祖先這個墓裡.
我們明白了.
原來你發覺阿伯拉罕往後的後裔.
他的家族.
真的進佔了.
佔領了這塊當年他買的地.
沒錯,那塊地是英許之地的部分.
但意味著這塊地.
整大片土地的始終擁有者.
是屬於阿伯拉罕和他的家族.
將撒拉葬在那裡.
其實預示了英許之地.
得到實現的第一步.
一小片的土地.
象徵著一大片的迦南地.
耶和華對阿伯拉罕的英許.
現在已經開始部分應驗了.
更大更多更廣的土地.
將要臨到阿伯拉罕.
他現在買下這小部分的土地.

$^{801}$提醒他自己.
將來他何等大英許將要來臨呢?.
他要憑信心來等候.
看完節目大概了解了創世紀十三章.
創世紀十三章我們讀到一次成功的談判.
一個大團圓結局.
但你想想.
其實這次的談判成功.
也不是必然的.
按當時的形勢.
談判失敗的因素也不少.
談判失敗的機會也高.
我最低限度從經文中找到三個很簡單的因素.
第一.
阿伯拉罕感情脆弱.
什麼意思?.
他剛剛失去他的配偶.
他正在傷痛之中.
或者用今天的說法.
他在悲傷期.
可以說他人生最低的一點.
感情最脆弱的時候.
在這個時候要扮身後事.
很多大大小小的事情要做決定.
可行的是阿伯拉罕在這個時候.
他仍然控制他的情緒.
他不會讓他的情緒蓋過他的悲傷.
也不會蓋過他的理性.
這是第一.
第二.
阿伯拉罕的需要是很急切的.
什麼是急切?.
我不知道薩拉奇離世多久.
才開始來談判.
但最低限度應該是一段日子.
當然人過世.
最好在那個地方盡早完成埋葬.
入土為安.
不然遲些遺體可能會發臭.
腐爛.

$^{841}$所以阿伯拉罕當時有某種緊急性.
不容當時的對手和他抗日持久地談判.
沒有時間和對方糾纏.
但很有趣的是阿伯拉罕雖然急.
但他沒有因為心急而壞了大事.
他沒有因為失去耐性而失言.
說錯話.
第三.
阿伯拉罕身處弱勢.
從何說起?.
當時的場面.
阿伯拉罕是一個人.
面對著一群的黑人.
那群黑人是當地原居民.
彼此熟絡.
別人說笑.
阿伯拉罕不懂笑.
別人說自己的暗語.
他自己獨有的表情.
你摸不通對方的意向.
你一個人去到對方的地盤.
阿伯拉罕要客客氣氣.
一靠近就輸了氣勢.
而且你又知道對方分明會敲你一巴掌.
但奇怪的是阿伯拉罕沒有失方寸.
他依然在整個談判過程處理得妥妥貼貼.
不過回看整個談判過程.
有三方面值得我們思考.
第一 整個談判過程.
在陽光之下進行.
公開的.
在眾人面前進行.
沒有桌底交易.
一切都在陽光之下進行.
其實那個時代沒有文件書信往來.
沒有文件工作.
但一句就一句.
說出口就真的.
阿伯拉罕沒有因為悲傷方寸大亂.
而說錯話.

$^{881}$沒有氣急敗壞失去了條理.
以致壞了大事.
你會發覺阿伯拉罕仍然可以開誠報公.
很真誠地攤開來談.
我要那個地怎樣怎樣等等.
阿伯拉罕這樣談判.
是不是我東方人亞洲人的習慣.
我是東方人傾向內斂一點.
不喜歡攤開自己的心底話.
我認識一些很關心年青兒女的父母輩.
他們有時希望和曾經有誤會的兒女和和氣氣.
或者他們很希望和曾經吵過架的兒女家人有談有講.
做兒女當然都很想和父母和好如初.
不過年青人你說他要解決問題嗎.
他就很希望將事情說清說錯.
有話直說.
將問題的核心一一攤開來講.
我發現很多父母輩的人不想來這一套.
很多父母五六十歲但求適時令人.
不想重提舊事.
過去了不要再說了.
不想回舊帳.
總之一家人聚在一起和和氣氣.
吃飯吃火鍋喝湯就算了.
做父母的可能夾隻雞腿給兒女.
就算是解決了問題.
所以即使父母在西方受教育.
或者在西方居住一段日子的父母輩.
一去到這些要疏解關係的結的場面.
很多時候都是採取一個比較傳統的解決手法.
你說年輕的一代.
怎能容忍父母那麼含含糊糊地處理問題呢.
所以陽光之下進行談判.
可能需要一個歷程.
需要分階段.
可能需要談很多次.
每次處理一點點問題.
每次深入一點點.
每次談一點點.
然後慢慢好幾次之後.

$^{921}$才去到問題下修.
甚至有時候需要第三者.
雙方面都信得過的第三者.
來幫忙調停.
做中介調解.
協調大家溝通.
拉近雙方的差距.
這是第一方面陽光之下進行談判.
第二你會發覺.
在整個阿不拉罕和黑人的談判裡.
捕捉對方需要.
阿不拉罕雖然當時在悲傷期.
但他沒有情緒不穩定.
他很清楚雙方的需要.
所謂談判目的.
談判底線.
他知道的.
其實你想想.
在這種形勢下.
最忌諱的就是掌握不到自己.
掌握不到對方的需要.
他的目標是什麼.
阿不拉罕知道對方需要的錢.
阿不拉罕知道自己需要的墓地.
只要各取所需.
只要拿到大家要的東西.
談判就算成功.
阿不拉罕知道自己的需要.
他不介意說出自己的底牌.
我需要一塊地.
我需要這塊地埋葬我的愛妻.
我想要你這塊地.
我甚至願意不惜代價.
投放多少錢都好.
我願意買這一地.
這種姿態有什麼好處.
好處就是對方不用猜.
不用猜測你的用意.
沒錯.
對方可能知道你的底牌.

$^{961}$他可能會作你一大筆.
不過阿不拉罕無所謂.
你剛才從頭到尾心平氣和.
有理有節.
進退有道.
不卑不亢.
這種各取所需.
說做容易.
現實上未必容易做到.
有一個基督教家庭長大的大學生.
因為不同見解.
跟父母吵架.
跟幾個同學搬到外面住.
經過大半年.
雙方心平氣和.
回來了.
終於有機會大家說出心底話.
好難得.
大家可以說出自己心中的期望.
心中的目標是什麼.
原來發覺.
大家的關注都不同.
父母表示.
他關注的不是不修例.
他關注的不是修不修例的問題.
他只不過是不贊成守法.
他不明白做兒子的.
很關心社會公義.
民主.
法治.
新聞自由.
做父母的.
他不是關心這些.
做父母的關心就是.
兒子不要太前.
不要太勇武.
不要衝擊政府或警察.
做父母是擔心兒女一旦出事.
留在案底.
將來影響他前途的就業出國.

$^{1001}$甚至更慘的是.
不小心被人斬傷.
被人起底.
所以當時做父母是阻止兒子.
進行反修例的活動.
其實他是關心.
底線是期望的目標.
是關心兒子的前途.
但他不明白.
兒子的兒子需要什麼.
父母的需要是.
兒女個人的前途光明.
但年輕一代的需要是.
社會整體的前途光明.
這些心聲.
這些需要.
在雙方火紅火綠的時候.
大家情緒激動的時候.
怎樣能夠說清楚呢?.
心平氣和.
所以這很難.
兩代特別是兩代.
各走各路越走越遠.
有時做旁觀者的我.
在這些情況下.
都會有些心傷.
第三.
阿伯拉罕和黑人的談判.
是尊重對手的文化的談判.
對方是黑人.
對方是主人.
來到對方的地頭.
阿伯拉罕不斷揣摩對方的文化.
他的習慣.
你看到他的禮貌.
你看到他的分寸.
你看到他說話的口吻.
他的下拜.
body language.
身體語言.

$^{1041}$你注意到對方的文化習慣.
你發覺雙方初期.
談話是有些試探性質的.
大家客客氣氣.
在外圍繞圈.
但是阿伯拉罕在這個地方.
很明顯的吻合對方.
他說話背後的言外之音.
然後聽得出對方話中有話.
然後他就按部就班地去應對.
對方可能一步一步試探.
阿伯拉罕的誠意.
虛實.
但基本上整個談判過程是順利的.
沒有被拘捕.
大家保持應有的尊重.
營起一個decent的.
大方得體的談判.
坦白說.
這次不單止是一次成功的談判.
其實他們彼此都賺了一個朋友.
將來仍然是朋友.
將來仍然維持良好的關係.
不然你可以想像.
這次即使買到地.
不過大家很不開心.
罵得很僵.
阿伯拉罕每年清明.
重陽去掃墓的時候.
遇見對方.
在村口尷尬尷尬.
面面點點.
很可惜很無謂.
我們在教會.
今天我們在教會.
在信徒之間.
在教目和長執之間.
我們信徒和領袖之間.
都有時面對這種情況.
海外華人教會也好.

$^{1081}$香港即使基本是華人社區也好.
有時我們都要面對不同文化背景.
不同年齡,世代.
不同觀點立場的人討論,傾談.
其實大家都愛教會.
大家關心教會事貢.
不過在議事決策的時候.
少不免會產生很大的張力.
教會長執和年青領袖的爭拗.
有時很簡單.
權威問題,自由問題.
兩代領袖一起議事討論.
年輕的人問問題.
問完一個又一個.
不斷問問題.
年長的覺得.
你搞錯了.
你挑戰權威嗎.
你不信任我.
年長的就不想說很多.
總之你信我吧.
我愛教會這麼多年.
總之你信我吧.
年輕的覺得.
你搞錯了.
你談都不讓談.
這麼粗疏就決定.
未經審視過.
你就去闖出通過.
行不行.
所以年長的覺得.
年輕的一天到晚爭取自由.
要有選擇.
有得選擇.
但你追問一下.
你有機會追問一下年青人.
其實他需要什麼自由.
他的自由不是一定說.
我一定決定權,話事權.
不是.

$^{1121}$你問清楚.
他需要什麼自由.
有時他可能只需要表達意願的自由.
表達感受的自由.
表達憂慮的自由.
他只是想有機會說出他的憂慮.
或者他只是想說出他的道理.
說出背後的理據.
說出邏輯.
可能他沒有想過挑戰你的決定權.
他只是想提出有什麼選項.
除了這個做法還有什麼更好的做法.
或者他只是想問問題的自由.
有沒有被聆聽的自由.
有沒有反對的自由.
你知道現在年輕一代.
他明知他不會改變你的決定.
不過最低限度你給他機會表達.
我是不喜歡,不贊成.
弟兄姊妹.
今天我們這一堂.
其實差不多到尾聲.
我們創世紀.
讀23章.
我讀到阿伯拉罕.
他用真金白銀.
買了一幅墓地.
埋葬撒拉.
於是耶和華對他的應許.
神的應許逐步實現.
代表這個應許會繼續延續下去.
這是經文給我們一個很重要的訊息.
我們可以借用阿伯拉罕和黑人的交易.
學習如何談判.
陽光下進行的談判.
捕捉對手需的談判.
尊重對手文化的談判.
今天我只是想用這個簡單的經文.
一方面大家學習研讀聖經.
我希望大家愛上這段我們平時忽略的聖經.

$^{1161}$另一方面希望趁機會讀這段聖經時.
學習如何與不同的人相處.
一起同工.
一起走主的路.
多謝大家.
(字幕由Amara.org社匯入).
(感謝收看).
(字幕由 Amara.org 社群提供).
\newpage



\section{}
\label{sec:G6y4aNW1WhY}
\textbf{【疫有嘢學 │ 延SUN在線】講故事、做神學|李適清博士}
\newline
\newline
連結: \href{https://youtube.com/watch?v=G6y4aNW1WhY}{\texttt{ https://youtube.com/watch?v=G6y4aNW1WhY}} ~~~~ 語音日期: 2020-05-17 
\newline
\newline
\hyperref[sec:KKN4CX_CTEM]{\small{< < < PREV SERMON < < <}}
~
\hyperref[sec:index]{\small{[返主目錄]}}
~
\hyperref[sec:gGn96F2CeGs]{\small{> > > NEXT SERMON > > >}}
\newline
\newline
$^{1}$主席.
梁繼昌議員.
主席.
梁繼昌議員.
主席.
梁繼昌議員.
主席.
丁子妹.
梁美芬議員.
主席.
梁美芬議員.
主席.
梁美芬議員.
主席.
丁子妹.
主席.
丁子妹.
主席.
我們今天出生於故事的其中.
在中間的中段.
很渺小的一點裡.
我們可以藉著聖經看到整個歷史.
然後我們發現自己的位置.
既渺小又很特別.
在尋求真理的過程當中.
我們不斷調教我們的視野.
力求發現如何更貼近上帝.
如何配合上帝的計劃來生活.
同時我們讀聖經.
不是單憑自己的理解能力.
是聖靈的同在.
讓我們繼續在今天的處境和經歷裡面.
去發現真理.
耶穌曾經跟門徒說過.
在楊福音16章12-14節.
祂說:我還有好些事要告訴你們.
但你們現在擔當不了.
只等真理的聖靈來了.
祂要引導你們明白一切的真理.
因為祂不是憑自己說的.

$^{41}$乃是把祂所見的都說出來.
並要把將來的事告訴你們.
祂要榮耀我.
因為祂要將屬於我的告訴你們.
其實上帝可以通過任何事物.
來啟示真理.
告訴我們祂是誰.
在文字普及之前.
人類透過口述,故事.
和流傳一些詩歌,圖文.
來將信仰承傳下去.
後來有了圖像.
可以畫畫.
聖經有不少詩歌都很藝術化.
用了一些比喻和不同的形式和手法.
全部都是上帝透過我們能夠理解的事物.
來將真理啟示給我們.
例如聖城耶路撒冷.
其實在地上可見.
但它有時也可以寓意未來或終末.
因為人類的能力和認知很有限.
所以要通過我們可以理解的事物.
來洞察上帝永恆的旨意.
上帝很特別.
祂就是這樣給予我們一種想像力.
亦是將我們需要知的賜予我們.
讓我們在故事,歷史延續當中.
在整個教會群體和弟兄姊妹一起.
來洞察,發現上帝的心意.
所以按照這個邏輯.
我們發現歷史承傳的傳統是必須的.
但同時持續更新,改變,對應時代.
是同樣重要的.
作為基督徒.
我們需要善用創造主給我們每個人的智慧.
我們都有不同的資源.
一起去認識上帝.
去反思我們的生命.
去編寫我們在上帝裡面的生命故事.
我們發現我們要讀通聖經.

$^{81}$並不是隨便的.
不是一時三刻就能夠急來.
而是我們一生的追求.
所以當我們有神學訓練.
有機會去學習的時候.
我們會學到歷史,文化背景.
有系統的神學思維.
對於生命的反思等等.
全部都幫助我們好好讀上帝的話語.
而且將神給我們的知識.
化成生命的行動.
或者你心裡有個疑問.
我沒有讀神學.
我沒有神學訓練.
聖經就讀不明白.
很多東西不知道.
怎麼算好呢?.
其實在不知不覺的順主之後.
上帝很愛我們每一個人.
祂為我們每一個人都度身訂造.
一個不同的神學訓練的旅程.
透過我們每個星期聽到.
你回教會上主日學.
你在團契裡有生活.
也在你生活裡經歷這位主.
我經常鼓勵弟兄姊妹.
不是必定要去哪裡.
或者讀什麼神學.
而是要善用上帝給你的.
在你眼前的每個機會.
無論是教會的主日學.
網上不同的資源.
團契小組的學習等等.
就按照你現時人生階段.
不同的需要.
去繼續追求認識上帝.
這裡我特別想提一提.
有一樣很重要的東西.
神學不是單單講知識.
不是說我們多懂一點.

$^{121}$我們就好一點又叻一點.
不是的.
神學是一個令人更謙卑的過程.
因為我們越讀.
我們就越發現.
其實自己是不懂的.
其實我們越來越發現.
上帝的偉大的時候.
我們就知道我們更加渺小.
我們就必須要預備好.
一個向上帝開放的心.
讓神的道進入.
哲學家蘇格拉底曾經這樣說.
他說未經審查的生命.
是不值得活著的.
當然我們不是每個人.
都是思考型的哲學家.
但我們每個人都有上帝給我們的智慧.
不同程度,不同的性情,不同的角度.
幫助我們去分辨和選擇.
怎樣去活出合神心意的生命.
所以神學反思就涉及我們的認知.
我們的思維,我們的理念.
但說到要做神學.
就不單單是這樣.
因為做神學涉及我們整個人.
在我們反思的時候.
和我們的行動.
和我們整個人的回應.
是不可以分割的.
神學反思是要讓我們.
帶來一個具體的改變的力量.
影響到我們繼續怎樣去處事.
行事為人.
而且信仰是要觸碰到我們的生活.
是要內化成為生命的道.
而且是活出來的.
讓我們所知的和所行的合一.
我們就成為表裡一致的人.
聖經是這樣說的.

$^{161}$憑著他們的果子就可以認出他們.
信念是很重要.
但最終我們都會用行動來顯明出來.
這些行動和經歷.
就成為我們生命故事不同的情節.
值得我們去回想,去反思.
去整理,去詮釋.
以致我們可以在經歷裡.
從實踐當中繼續學習.
這是全人的整體.
是表裡一致去做神學.
耶穌他「道成肉身」.
就是很具體地發生在歷史的時空裡.
這個故事.
神學家的神學.
很多時候都是神學家.
他處於某一個時空當中.
就在那個問題裡紮根.
所以神學不是停在理念和抽象的假設.
我們是需要敏銳地去觀察每個處境.
文化,時間.
做一些識切和帶有行動的神學.
優秀的神學不是說有關上帝的事.
而是進入人的心裡去建立一種習慣.
讓我們經常處身在上帝的臨在當中.
所以如果我們不禱告,不靈修.
我們都可以思考神學.
我們可以用腦袋.
但我們沒有禱告,沒有靈修.
我們就做不到真正的神學.
神學因而是從一個聆聽開始.
是一種安靜在上帝面前的操練.
我不知道你有沒有想過.
你願不願意放下一些既定的想法.
和人生的經歷.
讓聖經的話語改變我們.
可能你也喜歡查考聖經.
甚至和弟兄姊妹一起想想.
經文是如何應用.
但是做神學是要讓我們多走一步.

$^{201}$讓經文背後的真理.
成為我們生命的一部分.
操練成一個生命裡自然的反應.
我很喜歡觀察我身邊的故事.
有一天我回家乘電梯的時候.
電梯裡有一個爸爸.
牽著一個小朋友.
那個小朋友大概五,六歲左右.
他看電梯掣.
他說「41樓燈開了」.
「哎呀,不能按了」.
你知道小朋友喜歡按電梯.
爸爸說「是的」.
「還有多少號燈開了」.
小朋友看了一看.
他說「49」.
爸爸說「那49和41要上多少層」.
小朋友想了一會.
他說「7層」.
爸爸說「哎呀,不是,你錯了」.
「49減41是8」.
小朋友呆了一會.
指著電梯掣.
一二三四五六七.
是7層.
你說爸爸對還是小朋友對呢.
是小朋友對.
因為現代住宅大廈的號碼是沒有的.
這部電梯沒有44樓.
所以41和49之間只有7層.
這個片段在我腦海裡再三徘徊.
其實我們每個人都有既定的想法.
加上我們做了這麼多年人.
經驗告訴我們一定是這樣.
我們從小就這樣計算.
但是有一天我們相信了耶穌.
又或者你突然發現.
聖經說的東西和你想的東西很不同.
聖經說的東西.
有時是你最不想做.

$^{241}$覺得最不可行的事.
你願不願意仍然堅持相信.
用行動去相信.
以至你能夠經歷上帝呢.
我想只有我們願意被上帝的話語.
去改變我們的時候.
聖經才能夠在我們生命裡活起來.
我們不單只需要神學思辨和反省能力.
去讀通聖經.
我們更加需要上帝的話引導我們.
所以回應第二點.
聖經是否可行呢.
說就說吧.
事實是生活的困境正正在挑戰我們.
是否真的相信耶穌呢.
是否真的有信心去經歷呢.
現代的資訊越來越多.
環境越來越複雜.
突發的事情一下子就出現.
影響到我們的生活.
甚至影響整個世界.
今年的疫情就是一個這樣的例子.
天災人禍在我們預計不到的時候就出現了.
我們接收的資訊有時候真假難分.
沒有人可以掌握到所有的東西.
路易斯他曾經這樣說.
他說每一刻都太複雜了.
以至我們沒有辦法全面去掌控.
有無數的事物從我們的認知當中流走了.
問題不是我們不知道一切.
而是我們知道的少到差不多等於沒有.
我們都發現在自己的生命中間.
時間都被填滿了.
每秒鐘我們都被感官,情緒,思想等等來衝擊.
沒有辦法招架.
令到90%的訊息都接收不到.
這個對於每個人都是一樣的.
以往就好像很多訊息集合而成的一聲巨響.
斯亞蘇利斯他繼續說.
他告訴我們.

$^{281}$信心就是相信我們真實地認為是真相的力量.
即使眼前有切實的理由.
令到我們真正改變心意.
如果我們的理性不單只是以往或者現在的理性.
而是保持穩定的理性思維和判斷.
我們就必須要去祈求上帝賜下信心這份禮物.
讓我們有力量不單只按照理性去信.
而是當我們眼前充滿慾望和恐懼.
有驚嚇,窒道,沉悶.
甚至覺得無所謂的時候.
在這些不同的時刻和主觀感受之間.
我們都能夠保持清醒和理性.
在權威或經歷給我們各種觀念之間.
我們能夠分辨甚麼才是更加貼近真理.
人類最基本的問題不是不知道而是罪.
不是頭腦上的問題而是心靈裡的問題.
所以我們拒絕上帝.
我們傾向犯罪.
罪不單止令到心靈缺乏.
亦都會影響我們的思考.
令到我們的思想扭曲.
我們期望看到的,聽到的,想聽到的.
亦都可能會先入為主而不容易改變.
為了這樣,信徒更加要站穩在真理之上.
神學正正就是要讓我們學習好去分辨是非.
尤其是世界裡的人和事.
而且當中經歷的故事背後.
我們是否忠於真理和指向美善呢?.
神學家普蘭丁格說.
生命就是在未知,不肯定和不差當中.
我們相信的很多東西.
都是其他同樣認真的人不會相信的.
事實上有不少我們認為很重要的信念都是這樣.
我明白我可能很錯,徹底的錯.
甚至在非常重要的問題都看錯了.
但這就是人的警方.
最後我都要表明立場.
告訴大家世界對我來說就是這樣.
我們不能控制環境.
但我們可以選擇用哪些資源去回應眼前的問題.

$^{321}$我們亦都被賦予神給我們有智慧去理解.
去詮釋眼前的事物.
我們可以用神給我們的.
即使多或少的才幹和資源.
來分辨他的行動.
而在這個過程中我們是需要負上責任的.
如果上帝只給我們知識和能力.
可以解決一切問題.
不用想了就成功的話.
這個對我們的信心是沒有幫助的.
在不肯定和不完美的空隙裡.
我們就學習去依賴和仰望上帝.
而且我們相信他一定會賜給我們足夠的能力去實踐.
和有足夠的恩典.
所以這就是我們一生學習的過程.
我們在沒有辦法掌握的處境裡.
我們仍然很堅定地面對困難.
而且做我們能做的眼前最好的決定.
當我們做這個決定的時候.
我們會知道這個決定可能對或錯.
可能引致好的或不好的後果.
不過這一次所謂的成功或失敗.
都只是人生故事裡其中一個環節.
一個橫切面.
我們的故事仍然繼續.
而且基督徒的故事是很特別的.
我們可以後悔.
可以回轉.
可以經歷上帝的愛和教導.
所以我們不需要擔心.
我們就按照我們能夠的去走.
問題是我們願不願意擺上信心.
走這條另類的.
按天國的價值來決定的路.
甚至當我們眼前好像很不可行.
其實是這個世界的經驗或價值觀.
告訴我們很難去做.
當我們祈禱也沒有改善.
但你又很清楚上帝的心意.
你會否願意堅持下去呢?.

$^{361}$有一次跟一位姐妹聊天.
她最近工作的情況.
她跟我說:最近很缺乏信心.
祈禱後也沒有轉機.
所以越來越缺乏信心.
我想有時我們也會有這樣的期望.
做了成功有我們預期的後果.
但往往眼前很多時候都不是這樣.
不過當姐妹這樣分享的時候.
我突然覺得不是這樣.
這裡有些地方不對.
信心不是這樣的.
信心只是祈禱後就有反應.
這樣才有信心.
我們祈求有回應.
然後成功.
這是什麼信仰呢?.
裝著香神會庇佑.
我們就去還神.
其實這個邏輯只是民間信仰.
是一個有求必應的那種.
於是我就跟弟兄姊妹說.
就是求也沒有.
要等.
要發現究竟神怎樣出手.
未必我們想的最好.
我們可以努力去求.
但我們繼續去發現.
神在什麼時間給我們什麼不同的機會.
這樣我們才知道.
其實上帝的意念是超乎我們想像.
祂是又真又活的那位.
我們求了沒有.
我們仍然繼續求.
這樣才叫信心.
信心就是要信得過耶穌跟我們一起.
我們忍耐等候祂的時間,方法和心意.
而我們對於信心的理解.
只有當我們在生活故事的情節裡.
我們遇到.

$^{401}$好像這位姐妹這樣.
挑戰我們對於上帝有沒有信心的時刻.
我們才有機會去發現.
其實信心是什麼.
我們是否真的有信心.
在你每一天的生命故事裡.
都會有這些.
和真理面對面觸碰的時刻.
就是在這些故事的情節.
在你生命的經歷裡.
你去發現自己更多.
你去認識上帝更多.
當日耶穌復活升天之後.
早期的仕途.
他們不是像耶穌那樣.
講很多道理和比喻.
而是在他們身上活出一種很獨特的.
無法替代的生命.
他們就到處見證.
宣講耶穌復活了.
這個訊息.
這個不是單單出於他們自己的選擇.
而是來自內在的一種動力和能量.
令到他們在面對迫不及待.
甚至面對死亡的時候.
他們仍然可以靠神賜給他們的聖靈.
心裡的力量.
去高聲宣講.
耶穌基督活著.
這個訊息.
今天我們其實都同樣需要.
這種熱忱和力量.
好像早期的先賢一樣.
在一些迫不及待和困難當中.
我們學習去堅守真理.
我們通過生活的行動.
來告訴別人.
我們用行動來宣認.
將福音在現今世代的處境.
文化裡面.

$^{441}$用我們眼前很獨特的方法.
來展現在人的面前.
我們要堅守我們信仰的核心訊息.
但是我們不是在講一些教條式的演繹.
從舊約的聖經開始.
我們已經發現.
上帝用的方式.
是透過人類和人類互動的一個故事.
將自己歧視.
而這個歷史故事的高峰點.
就在耶穌基督.
他渡成肉身.
受死復活的事實裡.
不過故事並沒有停在那裡.
上帝隨時都可以.
讓你和我設計.
在這個巔峰點.
一切都改變.
搞定所有的事實.
但是故事沒有停.
路加就用《仕途行傳》.
繼續寫出教會群體的形成.
聖靈降臨在他們中間的事實.
今天我們的教會群體.
仍然帶著見證上帝.
耶穌已經復活的訊息.
這個使命.
要讓耶穌活著的訊息.
透過聖靈在我們生命裡的工作.
來彰顯出來.
所以弟兄姊妹.
你就是通過你的生命的經歷.
那個故事去宣講.
耶穌活在你的生命當中.
在這裡我們知道.
作為教會群體的一份子.
同樣是耶穌基督的門徒.
我們跟隨的主.
早就已經將祂的足跡.
留在我們的生命裡.

$^{481}$祂在等我們繼續去發現祂的心意.
讓我們在一個反思和實踐的過程裡.
更深去經歷和明白祂.
我們的生命故事.
就好像早期的《仕途》一樣.
藉著聖靈的臨在.
祂的工作和帶領.
繼續編寫著門徒的行傳.
理論上我們的內在生命.
是會衍生出外在的行動.
但事實是.
當我們認識真理.
但又未完全被上帝得著.
我們就會經歷很多不少的內心的掙扎.
保羅說.
我們很熟悉的經文.
他說「我都知道在我裡面.
在我肉體裡是沒有良善的.
因為立志為善由得我.
只是行不出來.
卻由不得我」.
反思和辨別自己的經歷.
就是要將事實真相顯明.
有時我們會後悔.
有時我們會喜悅.
讓我們面對這種矛盾和爭戰的狀況.
我們靠著上帝可以在這裡繼續分辨.
在這種張力當中走出來.
建立內外一致的生命.
故事有個特點.
它不一定是展示真理.
聖經故事的人物都有成功.
有得失 有失敗.
我們的故事都一樣.
我們有正面 有負面的.
在不同的行為和抉擇當中.
我們需要去檢視.
而且我們需要去分辨.
我們現時面對的環境.
和各種的可能性.

$^{521}$在當中洞察上帝的帶領和心意.
我們不一定會成功.
我們必定會經歷失敗.
會有軟弱的時候.
但正正因為這樣.
我們的生命故事.
必須要有持續的詮釋和反思.
讓我們的視野不斷被調教.
以至能夠更加準確地.
對準上帝的心意和聖靈的帶領.
所以不要讓你聽的道.
學的知識停留在你腦袋的層面.
我記得有一次.
我早就約了一位做生意的基督徒.
去分享他的故事.
過了幾個月.
聚會的時間快到了.
他突然找我.
他說:老師.
我不知道我還能不能上台分享.
因為我捲入了一宗官司.
可能我會因此而申請破產.
我詳細了解了他的情況.
最後我跟他說:沒問題.
請你分享這個部分.
包括你的故事還未完的.
將來也不知道將會怎樣的部分.
原來是他受到生意夥伴的牽連.
他沒有參與當中的決定.
但卻要負上責任.
這個在商業社會有時也會發生.
於是他身邊的弟兄姊妹.
不少人都給他一些意見.
叫他早些做一些財政和其他安排.
這樣就可以避免這宗官司.
搞到上身甚至可能要破產.
而且這件事理論上也與他無關.
他大可以不理會.
他心裡有很大的掙扎.
他掙扎了很久.

$^{561}$也跟家人商量.
最後他決定不會這樣做.
他還決定接觸一些有損失的人.
去解釋整件事.
當然他真正要負責的生意夥伴.
就不見了.
於是他就伸出頭來被人狂罵.
還可能會破產.
影響到他自己.
還有全家大小.
在人看來.
這件事絕對是一件失敗的事.
但是這件事也正正在上帝的眼裡.
可以看到不同角度.
甚至是蒙主喜悅的一件事.
難得他的家人都支持他的決定.
於是他們全家大屋半小屋.
放棄了生活裡很多的東西.
但他們選擇了上好的福分.
他們選擇了忠於上帝.
與其他人不同的路.
基督徒的生命要演繹的.
不是世人眼中必定的順利.
一定成功的故事.
更加絕對不是成功神學.
而是當我們經歷人生的起伏裡.
在仍然堅定和相信跟隨主的過程裡.
去經歷這個環境.
高低生命的起伏.
無論在什麼環境或威嚇裡.
我們仍然選擇合乎真理的路.
所以又回到聖經是否可行的問題.
如果只是想要很快地解決眼前的問題.
主要的目標是保住工作.
或者是得到好處.
聖經是真的不可行.
又不夠快.
又不夠世俗之子那麼聰明.
不過如果你想要回歸創造主的設計和懷抱.
能夠長遠地駕馭你的生活.

$^{601}$你的工作.
活得豐盛精彩.
走一條另類充滿恩典的路.
這樣就不同了.
這樣就不是一時三刻的事.
而是長遠地演活上帝給你的生命平台.
長遠地讓人認識你.
知道你是一個怎樣的人.
有時候你會不怕吃虧.
很遷就別人.
很願意幫忙.
但有時候別人不知道為什麼.
你就會很堅持.
半步都不讓.
因為你的生命是一個很不同的故事.
是一個跟隨耶穌基督徒的故事.
我們面對一個世界.
這個世界是一個後現代支離破碎的世界.
跟我們剛才一直說的故事,歷史.
剛剛是對立的.
因為我們的世界被灌輸一種想法讓我們知道.
不在乎天長地久.
只在乎片刻的歡愉.
不珍惜擁有的東西.
而只追求一時的滿足.
這樣的話就很簡單.
我們的故事不需要有延續性.
或者不需要有更深層的意義.
過到骨就算了.
但問題的所在正正就是.
在這種破碎的片刻的衝動裡.
我們的感動是停留不了的.
只會一會兒像飄過一樣.
留下的是一種空虛,失落.
我們就連下一次的激情會不會再來.
我們也無法預期.
在這個零落的碎片中.
我們必須要重新發現生命的規律.
讓我們人生的拼圖.
重新展現在人的眼前.

$^{641}$而且認識這位背後設計萬有的創造主.
對於那些認識上帝,擁抱生命的人來說.
生活就是經歷神聖的帶領.
與上帝互動.
令我們的生命重尋一種美善.
我們就毀身在創造主的秩序裡.
這個狀況為我們的生命帶來一種全新的思維.
讓我們去想起我們人生的經歷.
重新去理解,去詮釋這位創造主的美善作為.
從我們的過去一直指導的未來.
我們可以用信心去回應.
以及迎向未來的前路.
因為我們相信的這位上帝.
祂已經應許了給我們足夠的盼望.
所以一切在祂的手裡就足夠了.
三一上帝祂作為那位活著的.
是我們的生命之主.
不單止在我們的故事或歷史裡出現.
也成為我們生命故事真正的帶領者.
祂才是故事的主角.
雖然祂仍然是外在的.
那位超越我們很多的上帝.
但祂同時神的靈也內在我們生命裡.
掌管我們生命的主.
令我們的故事蘊含著一種奇妙的.
一種引人入勝的奧秘.
我們的生命被罪扭曲.
但我們在上帝的恩典裡重新被建立.
在這個故事裡我們與創造主相遇.
我們經歷掙扎,尋求出路,學習去倚靠上帝.
作為有限的被造之物.
雖然我們是基督徒.
但我們的理念和目標仍然被扭曲.
抉擇和行動當然也有可能會錯誤.
所以人生故事的時間進程.
為這些誤差提供了空間.
讓我們去演繹和變化.
甚至人生的進程.
樂意將誤差成為我們與弟兄姊妹.
一起關注,同行,學習的過程和經歷.

$^{681}$當故事發展出突如其來的衝擊.
例如很稱職的員工突然被解僱.
或者關係很好的同事突然出賣你.
上司突然吩咐你做違反道德的事情.
家裡出現了變化.
家人患病.
我們原本認定的軌跡突然起了變化.
這就是故事裡的危機.
在這些時刻.
邏輯思維似乎無法完全處理眼前的危機.
因為它涉及到很多非理性的感受和關係.
在這個過程中要求我們繼續分辨.
繼續選擇.
繼續演繹故事的情節.
在事情過去後.
我們可以回顧,反思.
在我們的經歷中找出故事的基本結構.
然後在這個過程中重新認識自己.
面對當中的起伏,對錯,感受.
擁抱我們的經歷.
讓它成為我們生命中更新的一種動力.
來面向未來.
這時候我們的生命似乎不斷地建立一個故事的活動.
在這個過程中.
當我們敏銳地對準上帝.
與祂互動的時候.
是會幫助我們不要被外在的壓力.
或者外加的身份,角色.
很多人想定義你是誰.
不要被這些東西影響或支配.
而是我們常常對準上帝.
在祂的身上捉緊我們的角色.
讓祂來得著我們.
第三點是聖經沒有說.
那怎麼辦呢?.
聖經沒有直接說.
但我們要繼續演活這個故事.
我會說每一項生活的處境和困難.
都要求我們同時有思考和辨別的能力.
以及有信心去實踐.

$^{721}$有情感的經歷.
所以思考,辨別和感受的經歷.
是兩者同樣重要的.
對我們來說.
世界就是這個場景.
就是一個學習的處所.
是一個征戰的地方.
是一個乘勝之旅的場地.
我們每一天都去到社會不同的角落.
我們都遇到不同人的價值觀.
有時候我們的罪性都會浮現出來.
我們發現邪惡勢力是想佔據我們所有.
但上帝卻在這個過程裡保護和看守我們.
令到這些困難甚至苦難成為我們成長的場景.
在這個過程裡創造主給我們思想和感受.
我們需要全人去經歷上帝.
去經歷人生.
所以理性思考和感性的經歷.
在這個追求真理的過程裡.
兩者都同樣重要.
我想最近大家面對的大環境.
就是疫情的困擾.
現在好像有些曙光.
在這個過程裡.
不少人都很努力地在聖經裡面.
找到面對瘟疫的一些傳說內容.
不過基本上.
我們今天面對的疫情和整個大環境.
都是前所未見.
上帝就在不同的時代.
做著很多不同奇妙的工作.
聖經真的沒有教我們直接怎樣做.
這個時候其實大家問的問題.
很多都是神學問題.
我舉一兩個例子.
在疫情開始的時候.
不少弟兄子會問.
上帝何時出手呢?.
何時我們才可以回復以往日常的生活呢?.
不過當我們繼續去沉澱和反思.

$^{761}$我們對這位上帝的認識.
又或者說是我們的神學觀.
可能會令到我們察覺到.
上帝其實一直都在.
祂的恩手不單止在人信警.
安書的時候出現.
也在最艱難的日子.
上帝仍然掌權.
我們可以理性去分析.
可以講解不同的經文.
去講述上帝在疫情當中與人同在.
但同時我們必須要用信心去回應.
我們最真實地.
當我們去經歷疾病,死亡,失業.
經濟的困難,家庭問題.
情緒的困擾等等.
我們都仍然堅持去相信上帝.
另外很多人都會問一個問題.
一個例子.
究竟網上可以做些什麼呢?.
可以怎樣去演繹聖餐呢?.
或者我們因著不同的傳統.
神學觀,教會觀.
我們亦會有不同的演繹方法.
但其實環境的限制.
給了我們一個非常好的機會.
讓我們重新去檢視.
教會的核心價值.
我們不能缺少的是什麼呢?.
然後我們好好調教我們的視野和行動.
我們是否願意這樣去開放自己.
特別在疫情當中.
當整個環境都在搗亂的時候.
我們會否都願意承認自己的渺小.
重新發現上帝的偉大呢?.
我其實很期望我們的教會群體.
在我們經過了疫情的經歷.
這段故事,獨特的情節之後.
我們的教會生活會起一些正面.
更貼近上帝心意的變化.

$^{801}$有一天,當我離開家裡上班的時候.
又搭電梯.
電梯門一開.
裡面有一個爸爸帶著一個女兒.
我猜她大概四,五歲左右.
爸爸有一個公事箱.
我進去的時候.
那個女兒正正用雙手拿著公事箱.
爸爸說「很重的,你拿不起的」.
那個女兒說「我可以的,我可以的」.
然後她兩隻手一起.
拿起那個足足有她半個身高的公事箱.
我看到她拿起來離開地大概只有一吋.
爸爸說「是不是很重呢?」.
「我可以的,我可以的」.
然後女兒就用單手拿著那個公事箱.
氣定神閒地把公事箱放在地上.
那個高度是她半個身高.
爸爸就說「把公事箱還給爸爸吧」.
那個女兒很乖,她想給.
於是她用盡力拿起那個公事箱.
不過她連給的氣力都沒有.
於是爸爸就蹲下來.
就拿起那個公事箱.
突然間那個女兒發現了一個事實.
她說「爸爸的手很大,爸爸拿得起那個公事箱」.
然後她伸出自己的手.
爸爸又伸出手跟她拼一拼.
果然爸爸的手大很多.
女兒就說「嗯,是喔,爸爸的手大很多」.
於是這個聰明的小朋友就發現了.
爸爸能夠有力拿起那個公事箱.
是因為自己不夠力.
是因為爸爸的手大很多.
這個時候電梯就剛剛到地下.
電梯門打開.
那個小女兒是氣定神閒.
很有自信地走出去.
她認為大概掌握到事情的真相.
這個短的片段小故事.

$^{841}$又令我想了很久.
很多時候我們都想.
我們為神做一些事.
我們很努力.
好像女兒一樣.
她很想幫爸爸拿起那個公事箱.
不過可能我們連重擔交給天父的能力.
有時連力氣都沒有.
我相信上帝是不會介意的.
其實天父在意的只是我們的心.
究竟我們為自己去想去做一些事.
以為自己可以.
還是我們是有心去為神呢?.
天父就好像磁場的父親一樣.
他其實很明白我們心裡想著什麼.
他就給我們空間去嘗試去努力.
然後在適當的時間.
他把身體放下去幫助我們.
又或者有時候我們都以為自己找到真相.
我們就擁有真理.
就是因為爸爸的手大一點.
女兒很聰明.
她分析過.
她伸出手跟爸爸比較過.
的確是爸爸的手大很多.
所以爸爸就拿得起那個行李箱.
我們以為自己的經驗加上努力.
我們掌握到真理.
我們很有自信.
有時候我們會忘記.
我們其實只是很有限的人.
沒有人可以說自己是百分之一百正確.
我們就要保持開放的心.
向上帝的話語開放.
然後繼續去貼近神的心意.
我記得我讀神學的時候.
每個星期我們都有一個時間開小組.
每一組有位老師和幾個同學.
一起來同行一個學年.
當我第一年一年班.

$^{881}$我開小組的時候.
一開始組裡面就討論.
我們要怎樣善用開小組一年的時間.
當時有位老師帶著我們.
他是一位輔導科的老師.
他說千萬不要選擇一本書一起看.
因為平時讀書已經看很多書.
每逢有人問他看什麼書.
有什麼好書推介.
他就會說你看你自己那本書.
看你自己那本書.
他的意思是我們每個人的生命就是一本書.
你也有你和神和人一起編寫的故事.
一路組成你人生的故事書.
故事裡有處境.
有家庭,有社會,有教會,有工作.
和各樣日常活動的場景.
這些都是我們生命故事裡面發生的平台.
當中的人和事都在挑戰你.
你是做一個怎樣的人.
你怎樣去演繹上帝給你的人生.
你堅持什麼呢?.
弟兄姊妹.
我很願意我們一起編寫合神心意的故事.
在我們人生整個生命的過程裡.
我們不單在個人的層面.
也在教會群體的層面.
我們彼此一起同行.
讓我們一起配合上主在這個時代要做的事.
面對一個更新,改變的環境.
和各種突發的事情,衝擊.
我們都決心讓上帝成為我們故事的主角.
我們就對準祂的心意來前行.
我今天的分享就來到這裡.
祝福大家.
謝謝.
字幕:J Chong.
(字幕製作:J Chong).
(字幕由 Amara.org 社群提供).
\newpage



\section{}
\label{sec:gGn96F2CeGs}
\textbf{【疫有嘢學 │ 延SUN在線】貼地同行,破舊立新 - 近代中國教會的社會服侍|莫陳詠恩博士}
\newline
\newline
連結: \href{https://youtube.com/watch?v=gGn96F2CeGs}{\texttt{ https://youtube.com/watch?v=gGn96F2CeGs}} ~~~~ 語音日期: 2020-05-03 
\newline
\newline
\hyperref[sec:G6y4aNW1WhY]{\small{< < < PREV SERMON < < <}}
~
\hyperref[sec:index]{\small{[返主目錄]}}
~
\hyperref[sec:qtqZXfdLO9c]{\small{> > > NEXT SERMON > > >}}
\newline
\newline
$^{1}$歡迎大家再次來到「亦有ye 學」延伸在線.
五月份的主日為課堂打頭陣的.
是忠臣實踐科的訪問教授 莫陳榮恩博士.
榮恩姐是一位宣教學的實踐型老師.
在忠臣侍奉這二十年.
她一直致力於推動和實踐跨文化宣教.
以及城市宣教.
榮恩姐今日教的課題是.
「貼地同行 破舊立新 近代中國教會的社會服事」.
求主使用榮恩姐的教導.
讓我們對近代中國教會的社會服事.
有更多的了解並且從中得到鼓勵.
各位弟兄姊妹你好.
很開心能夠在網上的地方和大陸交流.
一個對我來說很重要的題目.
就是中國教會的社會服事的題目.
我今日的題目叫做「貼地同行 破舊立新」.
是講到近代中國教會的社會服事.
開始之前我想和大家一起祈禱.
天父多謝你給我們機會在疫情的時刻.
仍然繼續有機會一起去研讀.
一起去看我們前人所做的所有事情.
以及檢討我們今日所做的.
我求你也帶領我們.
求聖靈在這裡雖然是一個網絡的傳播.
但也知道聖靈在我們當中願你掌管.
願你名得榮耀 願你感動.
奉主的宿命祈禱 阿們.
很開心和大家討論這個題目.
為什麼會講這個題目呢.
為什麼會講教會的社會服事呢.
就是在過去的日子當中.
當我在教宣教學 或是教會使命的時候.
近年來我經常講這個整全使命的課題.
其實整全使命這個課題.
在最近半個世紀開始有人講.
現在越講越廣泛.
我想特別講講整全使命是什麼東西.
我就用最簡單的方法去形容.
我很喜歡Andrew Wals.

$^{41}$Andrew Wals是現今的宣教歷史學者.
老人家九十多歲.
仍然繼續在大學裡講宣教的歷史.
他用一個很簡單的方法.
一本書叫做《21世紀的宣教》.
暫時沒有中文翻譯.
他講到全球的使命.
教會的使命有五個標誌.
他五個標誌.
我很簡單地講.
宣講福音 培育門徒 愛心服事.
社會轉化和保育地球.
我用更簡單的方法去看.
宣講門訓 服事改革和環保.
如果這些全部包在一起.
就是我所講的整全使命.
教會的使命是很多元化的.
我今次想講什麼呢.
我希望能夠從歷史.
看近代中國教會.
如何實踐教會在社會上的使命.
我們說名古珠金.
讓我們明白福音的社會意義.
另一方面我們從前人的過失和成功.
兩方面都有.
獲取教訓.
懂得怎樣衡量我們目前的方向.
我所說的近代中國教會是指什麼呢.
我是指19世紀初至20世紀中的教會.
特別是19世紀中至20世紀中.
那一百年的時間.
這裡有個圖畫.
中間是以前教會的圖畫.
一個相片.
在19世紀中國教會的門口.
有一道對聯.
一邊寫著送診醫藥.
一邊是宣傳聖道.
原來它是教堂.
又是診所.

$^{81}$下面的圖畫.
是教會的地方用來辦學.
特別留心看的是一些女孩子接受教育.
教會辦學將西方的教育引進中國.
如果你想一下.
我有什麼資料可以跟進.
其實這段時間的中國教會歷史是很多很多的.
我建議你.
如果你要選的話.
這本書我都相當不錯的.
是羅偉雄著的.
中國基督教新教史.
是在上海人民出版社出版.
2014年出版的.
這本書我都寫得相當不錯.
是羅偉雄教授寫的.
我今天主要不是一個很全面系統性的講述.
我真的這樣講都不相信.
一個小時時間都講不到.
我主要是講故事.
特別是找一些好聽的故事.
引你繼續去追尋.
我會從七個項目.
七個方向去看教會的社會服事.
包括教育,醫療,慈善,公益服務.
災難中的人道救援.
少數民族文化保育.
轉化文化和改革社會.
我講得很闊.
所以每一處都只是講一些故事.
教育方面.
大家知道西方的教育注重知識和科學.
補充了中國傳統四書五經.
在哲學和倫理上的編輯.
我想講幾個特別的例子.
一個叫艾迪胥.
或者叫愛爾德塞.
是不同的翻譯.
他翻譯成中文.
是Mary Ann Aldersey.

$^{121}$她其實是第一個來中國的女孫教師.
我不計算那些是做太太.
接著先生過來.
做女兒.
接著是父母.
是單身來到中國的.
第一個她自己決定的女孫教師.
就是這個人.
她在寧波開辦了一個叫寧波女屬.
那是中國第一個女子學校.
這個很特別.
所以是第一個.
所以我提及.
其實女孩子.
在當時來說.
開辦女子學校不容易.
因為女子無才便是德.
如果我們這樣來開辦.
就如風逆蹤.
人們會很懷疑.
很多的偏見.
很多的攻擊.
甚至很多的風險.
但她願意做.
其實最初她是在印尼做.
後來就來到中國的地方去做.
這個寧波女屬.
後來就變成了雍江女子學校.
我特別提及.
為什麼呢.
因為今天去寧波.
仍然可以找到雍江中學.
我認識有人在那裡讀書.
所以我就覺得.
很精彩.
這麼多年的歷史.
另外一個我想提及的.
就是山西大學堂.
這個也是歷史價值.
其實是1900年義和團之亂.

$^{161}$八國聯軍.
接著就是賠款.
賠款的時候.
就有不同的人有不同的做法.
當時內地會.
大德生他們是不收的.
不收賠款.
但是另一個派的人.
就是李提摩太太就收的.
你讓我收.
但他不是用給自己.
不是去重建自己那些被破壞的東西.
而是他用了這筆錢.
來做教育的功夫.
因為他知道教育很重要.
山西大學就是在庚子賠款.
創建的一個中西大學堂.
裡面有中學.
中學其實不是說到我們中學校.
是中國的學術.
和西學.
就是西方學術.
兩方面是共融的.
就希望用教育的方式.
來改變人心.
這個很重要.
其實當時山西大學堂.
是中國最早的三個大學當中的一個.
很重要.
今天如果你去到山西大學的話.
你今天去太原.
你都會見到山西大學裡面的.
李提摩太像.
這個歷史的一個很好的見證.
另外一個我覺得是很見證的.
我選擇了對大家有興趣的.
有關係的那些.
就是一個叫金陵文理女子學院.
我遲些會再說這個宣教士.
她的名字叫做魏特琳.

$^{201}$她就是當時她曾經很長的時間.
在當中做老師.
做教務長.
做校長.
為什麼提這個呢.
因為這個金陵女子文理學院.
她就到後來的時間.
就停了一段時間之後.
就分開了兩批的.
當時到40年代末的時候.
1940年代末的時候.
就是國民政府就遷去台灣.
就把金陵文理女子學院遷到香港.
而在香港的金陵女子文學.
對不起.
金陵文理女子學院.
就是成為香港當時的金陵女子大學.
也是當時崇基學院裡面的一個小部門.
後來就再合併.
到現在的香港中文大學.
原來現在的香港中文大學.
最原始是當中有小部份.
這個金陵文理女子學院.
這樣的歷史.
她分開兩部份.
一部份就遷到香港.
另一部份就留在南京.
南京在改革開放之後.
就是她復校之後.
就依附在當時的金陵女子學院.
就變成現在南京師範大學的一部份.
這個歷史是很寶貴.
提起這個學校.
講學校就很多很多.
我只不過提幾個故事都講不完.
因為你知道我有這麼多東西要講.
沒辦法講這麼多.
但是我會講一些.
一會兒還會講更精彩的故事給大家聽.
其實你都知道.

$^{241}$基督教最多的參與就是什麼.
教育 接著就是醫療.
你看看香港很多的基督教的醫院.
基督教學校都知道的.
第二樣東西.
我講醫療.
醫療方面其實都很多的例證.
不過我特別提一樣東西.
就是第一個西醫學堂的建立.
1894年的.
有故事講的.
為什麼呢.
因為當時的清朝的大臣李鴻章.
大家聽過吧.
李鴻章的太太她就生病了.
而她生病.
是有兩個的宣教士醫生.
合力治好她的.
就是John Mackenzie.
和Leonora Howard 兩個.
為什麼需要兩個呢.
其實主要的主診是John Mackenzie的.
因為她是男人.
李鴻章的太太是女人.
所以中間有一個女的醫生.
而接著之後.
這兩個就是這兩個的醫生.
John Mackenzie都有傳記寫她的.
其實他的貢獻就是.
他治好了李鴻章太太之後.
就說不如我們引進西方的醫學.
不只是我們去治.
我們是能夠將醫學交流給.
中國整個的國家.
那不就可以有更多的發展.
就開了北洋醫學院.
當時是中西合璧的.
是第一個醫學院開出來.
這是這樣的情況.
但其實同期的時間.

$^{281}$我們見到其實又有.
傳教士在中國也開辦精神病院.
這個是在中國沒有的.
精神病人每人治他.
並不會完全將他流放.
但有人會開精神病院.
有婦孺醫院.
因為男女授受不親.
所以女人就沒有辦法治.
所以就要開婦孺醫院.
還有女醫學院.
都是那個時間後一點開始的.
所以你見到在醫療方面.
都有新的和很多的突破.
講完教育醫療之後.
我想講下一個.
就是慈善公益服務.
我都是提一些來講.
可能大家覺得最多做的就是孤兒院.
因為你知道在中國的時間.
沒有計劃生育.
沒有節育這件事.
所以貧窮人就沒有辦法養這麼多小孩.
當時很多饑荒年年的時候.
很多時候就丟出來.
令到女孩子更加慘.
所以在1856年之後.
清政府就已經給外國人開孤兒院.
為什麼外國人這麼喜歡開孤兒院呢.
當然是因為基督的愛延伸.
但其實當中有很重要的.
基督教是人觀的.
人觀就是說.
人是神所做的.
有神的形象.
有神的尊嚴.
我們是完全平等的.
所以特別當時很多被驕氣的小孩.
有殘障 有病 有女性.
他們覺得是不平等的.

$^{321}$所以這是一個很重要的原因.
流浪兒童等等.
就是被一些無論是天主教的.
還是基督新教的信徒.
和傳教士收養的.
這些是圖片.
很有意思的.
其實到了今天為止.
今時今日.
中國家庭仍然會遺棄小孩.
因為殘障.
這是一個很大的問題.
因為殘障不讓社會接納.
社會沒有足夠資源.
讓殘障人可以正常地生活.
所以這是一個很大的問題.
除了孤兒院之外.
另外一個就是很多的滯糧所.
在1901年的時間.
美國的傳教士包慈禎.
這個女士.
就於上海開設一個滯糧所.
她說是收養失足婦女.
其實就是妓女.
她稱之為失足婦女.
和她們所生的小孩.
還有女生學習做工藝品.
讓她們以後有機會繼續維生.
但她們主要是走兩條路.
一個是回娘家.
就是回家.
另一個就是嫁人.
這些滯糧所.
都會幫助這些女孩子繼續.
這是一個慈善公益服務的例子.
你看到這個洋娃娃.
其實就是當時的女生去做的.
她叫做多利滋糧.
多利滋糧就是滯糧所.
滯糧所就是用來做這些洋娃娃.

$^{361}$運到歐西地方賣的.
其實這種方法到現在都在做.
我知道現在也有類似的滯糧所.
我們不是這樣稱呼.
但也是用來做手工藝品.
給女孩子重新生活.
特別是做首飾.
我認識這樣的社會組織.
是基督教的.
除了這些之外.
我下一步就說災難當中的人道救援.
在災難的時間裡面.
教會怎樣去回應呢.
這方面我想說三大件事.
一個是震災.
第二和第三其實是戰事.
一個是七七事變 盧溝橋.
另一個就是南京大屠殺.
下面說的有少少三級一點.
不宜兒童.
因為是比較血腥的.
在震災中.
我選了一個故事.
這個故事我真的覺得很好看.
很感動.
我剛剛最近看的.
這個故事.
我其實是最近一年來.
發覺原來瑞典的瑞華差會.
有很多宣教士在中國.
原來我們不太知道他們.
因為他們很多時候寫的文章.
故事是瑞典文.
中國人認識瑞典文的不多.
不過我最近很感恩.
我認識了劉鴻老師.
這個翻譯的專家.
他專門將瑞典文的書籍.
翻譯為中文的.
他翻譯了一堆瑞華差會裡面的故事.

$^{401}$當中有一個故事.
我真的很感動.
一個叫做惠潤世穆斯.
《我的祖父在中國》的書.
他在1903至1930年.
不是祖父的生和死日.
而是他在中國的日子.
足足27年的.
瑞華差會的工作地方.
主要在山西,河南,陝西三個省份工作.
在19世紀末到20世紀初的時候.
大家知道是黃土高原.
長年是農業失收.
有時候會洪水為患.
主要是乾島洪水為患.
人民經歷過好幾次大饑荒.
當時每次災難死亡人數.
是成千上萬的.
比現在所講的疫症還要厲害.
而瑞華差會的船教士深入民間.
他們在歐洲各地籌款.
讓船教士在最前線的地方鎮災.
當中有講這個故事.
就是衛牧師的事.
衛牧師在中國服侍了27年.
他第一任太太.
在他結婚一年半之後.
就死於惡性吏疾.
第二任太太和他的小女兒.
就在一次疫症鎮災當中.
死於腦膜炎.
而他自己最後.
亦都是在一次鎮災當中.
受到傳染病去世.
其實是三次的鎮災.
三次的災難.
大家知道當時鎮災期間當中.
很容易會有傳染病.
很容易有傳染病.
這四個人都是死於.

$^{441}$饑荒災難的疫點.
疫症裡面.
他們沒有用防護裝備.
走到災民的中間.
跟他們分享吃的.
和接受他們的病毒.
我看完之後.
很感動.
原來教會去實行慈善公益服務.
一點都不是廉價的施贈.
裡面是有血有淚的.
這些工作.
看上去好像對全方位無關.
不過其實就好像.
李提摩泰對福音的理解一樣.
這些信徒他們是面對社會當中的苦難.
他們必然要負的責任.
亦都是我們今天所說的.
整傳福音的核實前.
就是這個意思.
說起這麼感動的故事.
七七事變.
七七事變的時間.
即是盧溝橋事變.
在中國裡面當時.
基督教協進會.
中華基督教青年會.
中華基督教女青年會.
救世軍.
和各城市的基督教聯會.
都是來參與的.
主要是在最前線的當中.
提供醫療.
幫助傷兵.
還傳了福音.
最重要的是救濟難民.
因為很多難民逃離前線的時候.
他們會設立收容所.
和臨時的難民營.
讓難民在當中去暫步.

$^{481}$另一方面.
他們有輿論的反應.
就是這些聯會.
當時包括全國基督教協進會.
寫信給日本的基督教協進會.
和世界基督教協進會.
來譴責日本的侵略.
在海連城上.
亦都是發佈割難的宣言.
在當中.
這些都是當時有做的事.
說到戰爭.
當然大家都想到.
南京大屠殺.
其實我最近才去過南京.
看過南京大屠殺博物館.
所以這件事對我來說.
是很深印象的.
說回Minnevotrin 魏特林.
1937年的時候.
其實當時又打仗.
而金陵女子文理書院.
搬了去四川成都.
當時留校的魏特林.
就是見證南京大屠殺.
和他利用當時的校園.
保護了大批的中國婦女.
免受日軍的暴行.
他當時一共曾經收容了.
一萬個婦女.
即是先後.
而其實他不是當中唯一的人.
因為其實南京大屠殺的時期.
有24位宣教士留守.
是沒有離開的.
如果你是去到南京大屠殺博物館.
你就會看到這些圖畫.
這是一個像.
有像的人不多.
圖畫很多.

$^{521}$有像的人不多.
但魏特林本身是很特別.
因為他所做的事.
這書說的是.
他是婦女和兒童的守護神.
他當時被稱為活菩薩.
這個圖畫.
下面最低的圖畫.
其實就是他在那裡.
在學校當中.
收容的難民.
即是婦女的一部分.
這個故事其實是有血有淚的.
因為你可以明白.
可以想像.
魏特林當時.
他有困難.
他心裡有壓力.
所以他到了中期的時間.
到1941年.
他受不了.
他就生病了.
然後他就回到美國.
來治病.
而在美國的時間.
他是抑鬱.
他覺得他無法去接受自己.
在美國是可以平安的.
但他所愛的婦女.
在南京受苦.
他接受不了.
有一天.
他在自己所住的地方.
開煤氣自殺.
他自殺而死.
而在他的墳墓.
這是在美國的墳墓.
他在28年在中國.
在墳墓上寫著幾個字.
就是金陵永生.

$^{561}$南京的金陵.
金陵永生.
近期有人拍出戲.
金陵永生是說他.
是在國內拍的.
而拍的時候.
其實不是用信仰角度.
他沒有淡化信仰.
但他主要是說.
為特林所做的一切.
是在南京大屠殺時所做的.
所以令我們覺得.
是很感動的.
這些故事.
我經常說.
看到這些有血有淚的故事.
很沉重.
你問我為何這樣呢?.
如何評估呢?.
我不說了.
大家慢慢再想吧.
當時有很多其他人.
有24位宣教士留守.
當時有一個約翰·馬吉穆斯.
他做的事是.
秘密拍了很多照片.
每天都寫日記.
這些藏著的日記和照片.
後來偷運到境外.
成為後來聯合國審判.
日本戰爭暴行的證據.
這是馬吉穆斯的貢獻.
所以我說.
當中在南京大屠殺中.
銅像不多.
但當中有幾個.
當中有馬吉穆斯.
和很明顯的韋特林.
這些都是那些照片.
留守的西方人士.

$^{601}$裡面有很多絕大部分.
是基督徒和宣教士.
說完災難中人道救援.
我下一個說到.
少數民族的文化保育.
如何保育少數民族的文化呢?.
當然有很多方面.
但我想提一個.
一個很重要的方法.
就是翻譯聖經.
翻譯聖經的第一步.
因為很多少數民族.
只有語言.
沒有文字.
所以翻譯聖經的第一步.
就是將少數民族的語言.
變成文字.
如何保育文化呢?.
只是除呼音.
不是的.
因為當你的語言.
變成文字寫下來的時候.
就很不同了.
它可以將傳統的故事.
傳說.
詩歌.
很多歷史.
都寫下來.
你會發覺原來翻譯聖經.
或者製造文字.
是將少數民族的文化.
保存下來的.
這是很重要的.
因為你們問.
中國有多少少數民族呢?.
55個.
當然是假的.
怎會只有55個呢?.
印度也有幾百個.
其實最早期.

$^{641}$在解放的時間.
那時候.
曾經做過一個研究.
說中國有四百多個少數民族.
只不過我們太難管理.
就將他們收起來.
大家是歸類的.
所以你會發覺.
算是一個少數民族當中.
藏族.
有不同的語言.
完全不同的語言.
其實不是一個少數民族.
彝族更加厲害.
有很多不同的語言.
如果你沒有自己的語言.
你那個少數民族.
本身的定位.
特色就慢慢沒有了.
所以原來製造語言.
翻譯這件事.
是很重要的.
這本聖經.
叫做《拉姑族》.
拉姑我也不太清楚.
拉虎族的聖經.
是我自己拍的.
它就是.
例如.
律戌族.
苗族.
雅族.
我也見過這些聖經.
都是文字.
宣教士先做文字.
然後去翻譯聖經.
還有.
其實你發覺.
少數民族教會的禮儀當中.
都會穿著他們的傳統禮服.

$^{681}$載歌載舞.
用他們的方法去敬拜神.
這本身就是一個.
少數民族的保育.
好了.
講到後面.
轉化文化.
一講到轉化文化就比較長.
因為有很多故事要跟大家講.
首先.
怎樣轉化文化.
先講第一個故事.
未講的先講.
這是民國十六年.
即是1927年的時間.
中國的海報.
宣傳基督革新中國.
你見不見.
原來基督的精神.
包括基督化生活.
光明 真理 博愛 犧牲.
等等.
去驅趕.
侵略 貪婪 殘暴 罪惡.
等等.
是這樣的轉化.
你說有什麼實際的例子.
具體例子.
首先因為我很有興趣.
少數民族的故事.
所以我先講少數民族的故事.
第一個就是叫做.
砍頭族.
砍頭族是怎樣的呢.
其實如果去普爾市.
現在雲南普爾市博物館.
你會見到.
在左手邊那裡.
你猜這是什麼.
其實是砍頭刀.

$^{721}$前面是真人的頭髮.
頭髮加上了刀.
給人砍頭.
很厲害.
砍頭族.
你會見到.
右上角的相片.
其實是他們將一些.
竹腔.
下面你會見到竹腔.
竹腔下面.
和一個竹筒.
很大的竹連著.
頭部.
砍了之後.
在竹腔上面.
很恐怖.
然後讓他的血.
滴滴滴滴滴.
跟著那條竹.
滴到地上.
為什麼那些人那麼殘忍.
原來是這樣的.
原來相傳.
這樣的民族已經有.
很多年的歷史.
其實是在緬甸.
和雲南邊境的.
一些叫做亞族.
亞族.
他們世世代代都是去獵頭.
為什麼獵頭呢.
他們相信每年都要用人頭為祭.
如果能用人頭為祭.
有什麼好處呢.
就是換取豐收.
豐條與順.
防災防難.
一條大村.
在當地.

$^{761}$每年總要殺十幾人.
來獻祭.
他們村裡面有些叫做.
「砍頭英雄」.
是誰呢.
他們專門去砍頭.
當然是他們的敵人.
所以千萬不要得罪他們.
還有陌生人的頭.
都會被砍.
所以不要亂闖入他們當中.
可能會被砍.
有時沒有頭.
他們這些砍頭英雄.
就會去走幾天的路.
去其他地方.
去偷頭.
在晚上偷頭.
然後拿走.
另一方面.
有些人不做.
買少年人.
回來.
說我買回來.
幫我們做勞工.
但其實只是砍頭.
這些故事.
一直都.
好到近代.
其實這些故事.
是啞族的牧師.
親口跟我說.
他說他小時候.
見證過.
好恐怖.
相傳.
這個風俗.
大約公元300年前.
已經流傳.
有千多年歷史.

$^{801}$直到1950年代.
解放中國後.
才被取締.
但被取締後.
1970年代才完全消滅.
不砍人頭.
而是砍牛頭.
用牛代替.
但在上世紀初.
已經有一群啞族人.
因為相信耶穌.
不再砍頭.
改變了文化.
有些啞族的村落沒有砍頭.
為何會這樣.
有很多不同的版本.
相傳的版本.
在1902年.
有個啞族的先知.
他名叫柏趙門.
臨死時.
吩咐族中長老.
跟著白馬走.
就可以找到真神.
這個白馬帶著一群.
尋道團隊.
走回山越嶺.
走進森林.
來到緬甸東南的邊境.
美國傳教士.
叫做永維尼.
是William Marcus Young 的家.
停留在那裡.
是這樣的故事.
長老一中.
永維尼牧師.
學到幾年.
1908年.
他就將福音帶回.
雲南的啞族鄉村.

$^{841}$當時很多村莊.
信了主.
全村歸居的地方.
已經沒有了砍頭的習俗.
不用等到1950年代立法為止.
這是一個文化的轉化.
另外.
第二方面.
我想說的是.
有關於女性地位.
在女性地位當中.
我特別提出兩點.
一個是.
說到「砸腳」.
我們叫做「前足」.
「砸腳」.
大家知道.
「砸腳」是三寸金蓮.
不知道大家有沒有見過「砸腳」的人呢?.
我小時候見過.
我記得小時候.
有「砸腳」的老婆婆.
可能大家.
絕大部份人都未見過.
你見到「砸腳」.
其實是.
很小的時候.
將腳剖開.
然後讓它「砸」住.
慢慢痊癒後.
讓它可以.
「砸腳」很小.
「砸腳」是很美的象徵.
「砸腳」小的時候.
「砸腳」就代表大家歸瘦.
「砸腳」就代表嫁得好一點.
原來到今時今日.
在雲南的地方.
仍然有老人家.
租用「砸腳」.

$^{881}$大家看的這兩個圖畫.
是近日拍攝的.
是2019年拍攝的.
是去年拍攝的.
所以「砸腳」的風俗.
其實流傳很久了.
「砸腳」這東西.
成為婦女很大的.
「綑綁」.
最初西方的.
女孫教士覺察到.
前足的問題.
很多問題.
開始反對.
而反對的口號.
很有趣.
「一個女基督徒.
砸了一雙基督的腳.
基督的腳沒有砸.
耶穌基督在世的時候.
所以我們就應該要放腳.
在1902年.
發起中國婦女天竺會.
一個全國的組織.
很有意思.
在二次大戰的時候.
仍然有女孫教士.
被政府聘用到.
百姓家裡.
檢查婦女有沒有暗自砸腳.
為什麼要用女孫教士去呢.
因為他們深入民間.
去到女性的地方.
去檢查.
所以關於砸腳方面.
天竺會是改變文化的一件事.
另外.
就說到「妹仔運動」.
「妹仔解放運動」.
香港1844年的時候.

$^{921}$就已經立例禁止奴隸的制度.
因為在英國.
當時已經是.
廢除奴隸制度.
引導香港.
去禁止奴隸制度.
不過.
雖然如此.
其實「妹仔買賣」一直都有.
維持.
1879年的時候.
保良局.
這是保良局的污點.
當時上書的.
清廷的政府.
維護「妹仔買賣制」.
即是.
聯同一些工商業人士.
就說.
「妹仔買賣」不是奴隸制度.
當然.
可以自圓其說.
說到不是為止.
但這是事實.
到1922年的時候.
孫中山就發佈.
嚴禁被領.
到民國.
就禁被.
不可以買賣奴隸.
即奴婢.
而香港教會響應.
基督教女青年會.
和其他社團一起.
鼓吹禁被.
所以在香港當中.
「妹仔」的制度.
在1920年代才停止.
而是因為有教會.
合力的緣故.

$^{961}$在當中.
推翻這個制度.
這都是說到.
轉化文化.
如果說到.
禁被方面.
有另外的故事都很好聽.
廈門有一個.
基督徒.
他叫許春草.
1920年代.
孫中山已經禁被.
但上面說.
下面做不做另一件事.
當時許春草.
他未信主之前.
都是江湖老大.
他也是做地產.
有財有勢.
他自己信主之後.
他覺得不單要廢除.
家裡的「妹仔」制度.
禁被.
他都需要建立.
中國婢女救拔團.
和婢女收容院.
而當他.
做這些事.
你會發覺不容易.
你不要以為我去做這些.
改善文化.
當然不是.
因為.
很多的既得利益者.
既得利益者.
全部都有錢有面子.
他不讓你這樣做.
當時的時間.
他用法庭上訴.
有人告他.

$^{1001}$說他搶走婢女等等.
而.
我看他最近的一個.
是他的舅仔.
張勝才口述.
實錄他當時發生的事情.
原來他當時.
試過有人要陷害他.
要殺他.
但神保守他.
所以他有個.
「聖徒與戰士」.
他都去征戰.
成功救了婢女.
左下角.
在山上.
在山邊.
建立了收容院.
給婢女.
很小的.
裡面很小的.
女孩.
幫他們.
自由的一天.
不過.
很可惜的就是.
這個婢女收容院.
只是幾年的時間.
就開始打仗.
在日本打仗的時間.
在那段時間.
大家亂的時候.
到平安.
和平之後.
其實那個.
收容院都沒有再重開.
其實繼續那個.
婢女的制度是沒有改變的.
直到解放之後.
OK.

$^{1041}$好了 說到轉化文化.
我想說一件事.
一個具爭議性的項目.
就是倡導禁煙戒煙.
我剛才說的都是好事.
說得好像我們很偉大.
但其實過程有些事情.
都不是很光榮.
為什麼這樣說呢.
因為全教士和鴉片的貿易.
前期有認同的.
打倒馬勒信.
我們經常說馬勒信是.
第一個來華的宣教士.
但我們知道在基督教教會當中.
很少說這件事的.
其實馬勒信來了中國第二年.
在澳門就已經做了.
東印度公司的職員.
而他做東印度公司的職員.
是直到他死之前幾個月.
1834年.
所以這是一個.
我承認是.
中國教歷史的污點.
當時羅冠中.
一個歷史學家.
來形容說.
馬勒信來華第二年.
就受聘於東印度公司.
為什麼呢?他有住的地方.
有固定收入 減輕測估負擔.
當東印度公司的大班.
可以發現.
全教士不是說.
敵黨他們的.
對全教士的厭惡等等.
而他又說到.
引述另一個.
人叫做.

$^{1081}$其實他也是馬勒信.
不過是他翻譯成摩里遜.
他寫過歷史的時間.
他說基督教的全教士.
都不反對鴉片貿易.
特別我.
說個例子 例如郭實立.
當時的例子不反對.
不單止不反對.
他們乘坐販賣.
鴉片的飛展船到中國去.
他們還從.
販賣鴉片的.
公司商人的手中.
接受捐款.
他們都說鴉片對中國人.
是無害的 就像酒.
對美國人是無害的一樣.
這個的確.
是我們要去承認.
最早期的無知.
你還要去.
用他們的船來往往沿岸的地方.
這個是羅冠中.
他的書叫做前事不忘.
後事之詩裡面寫的.
他的分題就說.
帝國主義利用基督教侵略.
中華史述的.
論述.
這個我們要去.
承認.
我們是有錯的.
但另一方面.
鴉片的非法偷運是從18世紀末.
直到就開始.
直到19世紀中期.
英國的宣教士才開始.
倡議禁煙.
這是英國宣教士開始做的.

$^{1121}$宣教士對鴉片早期的默許.
和容忍.
成為了中國教育歷史上的大污點.
羅偉雄.
他講了另外一些.
他說其實.
全教士在鴉片貿易後期.
是反抗的.
在1850年代之後.
特別的反抗.
就是在英國政府裡面.
做輿論.
用一個方法讓英國人知道.
你這樣做是不行的.
他是用背後知道的.
操控鴉片的.
或者是鼓勵鴉片的政府.
在廣州.
基督徒組成的反鴉片會.
到頭來.
另一個歷史學家.
王志祺怎樣去評論呢.
他就說在反鴉片的.
運動當中其實全教士.
分別要面對來自中國.
和英國政府的敵視.
但是他們是不斷堅持下去.
他就說其實沒有什麼.
運動可以堅持50年.
而他說.
反鴉運動堅持了50年.
是兩代的宣教士.
他也說其實.
我們都需要.
給宣教士一個.
公道.
後期的時候他們悔改了.
他們真的好好地做了.
這本書也有人意.
王志祺就說.

$^{1161}$其實他們也有.
到後來的時間.
去改變這個事實.
說到鴉片.
就說一個故事.
直性魔.
直性魔本來是.
一個讀書人.
16歲當秀才.
很厲害的.
16歲.
入大學.
很厲害的.
他父母行醫.
他自己吃鴉片.
窮到窿.
窿的時候遇到李修善.
宣教士.
為什麼遇到李修善.
原來李修善想要反鴉片.
他就有一個精文比賽.
就說你們寫文章.
就說鴉片如何毒害人民.
我會.
冠軍的話會給你一筆錢獎金.
直性魔.
就因為想要.
這筆錢.
他太窮.
他寫文章是可以的.
就拿到冠軍.
其實是雙冠軍.
就去李修善那裡拿錢.
見到李修善.
結果長話短說.
就信主.
然後就戒毒.
因為他父母有醫術的背景.
他自己做了一個.
自製藥丸.

$^{1201}$可以幫人戒毒的藥丸.
開了天招局.
天招局是叫人戒毒的.
就像我們國福音戒毒.
全國都開了45個.
直性魔牧師.
他後來改變.
故事插曲.
我很喜歡胡適的說法.
說到我們有時的錯誤.
但這方面我們所做的一切.
他說胡適說話.
他不是基督徒.
但他比較幽默.
我們深深感謝帝國主義者.
把我們從黑暗的夢裡.
驚醒起來.
我們焚香鼎禮感謝基督教的傳播.
帶來一點點西方新文明.
和新人道的主義.
讓我們知道我們這樣對待小孩子.
是殘忍的,慘酷的,不人道的,野蠻的.
十分感謝這班文化侵略者.
提倡天竺會.
不前祝會.
開設新學堂,開設醫院,開設輔英醫院.
我們從現在的角度來看.
他們可能不算好學堂.
不算好醫院.
但他們是中國新教育的先鋒.
他們是中國新教育的先鋒.
他們是中國持有運動的開拓者.
他們當年的缺陷.
是我們應該原諒寬恕的.
我借胡適.
一個不信主的人的評論.
做一個暫時的小結.
好的,最後.
我講改革社會.
原來剛才所說的一切.

$^{1241}$教育,醫療,公益等等.
但最後說到改革.
改革社會方面.
我講幾個例子.
一個是1901年受王乃常.
福建的信徒.
其實王乃常在1898年.
已經考試.
他是知識份子.
加入百日圍新.
他當時是.
倡議漢字拼音化.
和簡體化.
在周有光.
我們叫漢字之父.
的幾十年前.
他做這件事.
但生不逢時,太早了.
被人罵他,以風易俗.
背早背終,追斬他.
他又要逃,要逃亡回福建.
他對當時的清政府完全失望.
他帶了一千人.
一千個教會信徒.
去了東馬沙鹿越的地方.
開建華人社區.
所以東馬在今天.
是全馬來西亞.
基督徒人口最多的地方.
是他們的結果.
後來他又覺得.
不行,國家緊要.
又回去.
偷回.
中國內陸.
參加革命.
革命運動.
革命完結後才去世.
這是王乃祥的故事.
另外.

$^{1281}$是講辛亥革命.
我們知道黃花崗72烈士.
當中有五個專職傳道人.
大家聽過了嗎.
據說還有幾十個.
基督徒.
我們不知道是甚麼人.
但我們覺得.
如果想查是可以查到.
另外在辛亥革命時.
在武漢.
武昌起義.
武昌起義時.
在聖公會.
聖公會裡面.
有一個革命組織叫日支會.
日支會的會長.
就是劉靜暗.
這兩張照片.
是很難找得到的.
在武昌市的.
萬華林裡面.
是一個很難找得到的地方.
在景點裡面找不到的.
Google或Google對白都沒有出來.
這間屋.
就是當時日支會的會址.
外面有一個.
牌子.
形容當時他做的事.
劉靜暗.
就是當時的會長.
他就在這裡去.
付出革命.
他做得很多地步.
做到.
無懈可擊.
被他稱為.
革命元人.
到今時今日在武漢的.

$^{1321}$首爾博物館裡面.
有革命元人劉靜暗的事跡.
當然他的信仰部分.
就很低調處理.
他後來屈死在監獄裡面.
36歲.
是這個故事.
另外一個是胡蘭廷牧師.
胡蘭廷牧師其實是.
當時劉靜暗的同伴.
也是他的師父.
他同樣在當中.
鼓吹革命.
在當中去做.
但革命成功後.
他沒有在當中取任何的官半職.
他繼續回去做牧師.
中為牧師.
講改革社會的時間.
講到革命.
我發覺其實有些書講了很多.
光與炎.
兩冊書其實裡面講了很多.
特別是.
20世紀初19世紀末的時間.
一些基督徒的事跡.
包括石美玉.
早期的第一位的女醫生.
唐國安 清華第一任的校長.
這些全部都是基督徒.
顏陽初.
我很喜歡的 中國平民教育.
第一人 很重要的.
譬如剛才講的王乃常.
等等 劉廷芳.
燕京大學的教授.
這些都是很著名的人.
我不能一一跟大家講.
不過我先講總結歷史.
到最後的時候.

$^{1361}$到底我們怎樣去看呢.
我只能講到幾點.
我覺得我們可以反省.
我說 要實踐聖經的真理.
教會應該與受苦的人群同行.
出世的屬靈 操練.
與入世的社會服務.
兩者是相容的.
一件事的兩面.
我們要活出整全的使命.
退隱和行動.
其實是生命的兩面.
所以我經常用行和忍.
如果用普通話.
是善與賢 兩個.
其實是生命的兩部分.
另一個就是.
教會是可以從基督教的價值觀.
來改變社會文化的.
我們看到的.
一方面 實踐過程是不容易的.
不要以為很開心.
大家一起去做.
不是的.
教內有反對.
教外有難阻.
甚至要賠上你的性命.
而且是可以死得很慘的.
最後我說.
教會有倡議的能力.
而保持緘默.
也會受到歷史的批判.
我再說 教會有倡議的能力.
而保持緘默.
也會受到歷史的批判.
所以最後.
當我說整傳使命的時候.
其實我們在我們的生命.
每一個範疇當中.
都可以表現出來的.

$^{1401}$包括我們口傳福音.
我們去栽培門徒.
我們用愛心服侍周圍的人.
我們去轉化改革社會.
還有一點我剛才的故事當中沒有說的.
就是保育地球.
做香港的環保.
做管理大地的管家.
都是整傳使命的多元發揮.
希望大家能夠好好參考.
也希望大家.
我們大家一起去反省我們的生命.
反省我們的行動方式.
一起去實踐整傳的使命.
多謝大家.
(字幕製作:貝爾).
(字幕由 Amara.org 社群提供).
\newpage



\section{}
\label{sec:qtqZXfdLO9c}
\textbf{【疫有嘢學 │ 延SUN在線】雙翼齊飛:信仰和正向心理學如何助我們跨越逆境和疫情|區祥江博士}
\newline
\newline
連結: \href{https://youtube.com/watch?v=qtqZXfdLO9c}{\texttt{ https://youtube.com/watch?v=qtqZXfdLO9c}} ~~~~ 語音日期: 2020-05-10 
\newline
\newline
\hyperref[sec:gGn96F2CeGs]{\small{< < < PREV SERMON < < <}}
~
\hyperref[sec:index]{\small{[返主目錄]}}
~
\hyperref[sec:QdUuSNEVBIc]{\small{> > > NEXT SERMON > > >}}
\newline
\newline
$^{1}$主席.
各位忠臣的朋友.
你們好.
無論你在香港,北美或其他地方.
在疫情之下.
都有空間一起學習課題.
今天想和大家分享一個課題.
叫做「相逆齊非」.
信仰和正向心理學.
如何幫助我們跨越疫情和疫情.
我想為這個題目點題.
甚麼是相逆呢?.
我是指信仰和正向心理學.
我本身在忠臣教輔導.
我發覺心理學和信仰.
都有很多整合的機會.
現在面對疫情和疫情之下.
信仰發揮了很重要的元素.
而我自己在研究心理學的過程中.
有一個叫正向的心理學.
它和一般的心理學有少許不同.
它比較著重我們人有一種內在想好的動力.
在疫情之下如何正向面對前面的問題.
今天我會從兩方面和大家分享.
先談談疫情.
在疫情之下如何仍然有盼望.
下半段我會和大家分享關於疫情.
在疫情要家居隔離.
或者在不太自由的狀況下.
我們的情緒也會受到影響.
在疫情之下如何調適好自己的情緒.
我們也有很多憂慮.
希望透過心理學和聖經.
讓大家在這方面有出路.
我先和大家分享關於逆境.
逆境在心理學中會看為.
當我們面對巨大的壓力時.
我們原本能夠面對困難的能力.
突然間好像掉了一圈.
如果大家看這幅圖.

$^{41}$左手邊是壓力和逆境.
一出現的時候.
就令我們好像掉進了一個低谷.
如何才能從低谷上來呢.
這就是研究逆境或抗逆力.
英文叫resilience.
在我們看resilience的過程中.
一個人被一件事衝擊掉下來.
其實有一種反彈.
站起來的能力.
在逆境中.
我們有三個被逆境打擊後的可能性結局.
有些人會一蹶不振.
被逆境來臨.
令他情緒和精神出現困難.
很多時候回不去原本的能力狀態.
有些人在逆境完結後.
回復了水平.
和逆境之前沒有分別.
但有些人經過逆境後.
可以提升自己.
我們叫做經一事長一智.
我們希望在逆境中能夠.
跨越之後.
好像增強了我們自己面對困難.
逆境的能力.
這幅圖其實說了.
面對逆境有三個很重要的元素.
一個是sense of competence.
聲音感.
有逆境來臨.
我們都需要去面對它.
我們有一個問題解決的能力.
我們都要找到一些資源或支持.
所以我們需要有社交的能力.
另外在逆境下我們有很多情緒.
我們下半節會說情緒管理是怎樣做.
在逆境中我們都要定一些目標.
所以這裡有四個C.
C1 C2 C3 C4.

$^{81}$其實就是說.
在逆境中如果我們有這種能力的話.
面對逆境會好一點.
第二個面對逆境的出路.
我們叫做sense of belonging.
我想在這次疫情.
對我們挑戰很大.
譬如我們沒有了實體的境界.
我們不能回到教會.
好像病了自己.
做了宅男宅女.
幸好我們都見到很多善用資訊科技.
所以很多人都會.
教會都會用Zoom去團契.
有一個人與人之間的接觸.
因為這種sense of belonging.
有一種歸屬感.
你不覺得孤單.
是面對逆境很重要的元素.
這裡其實是說.
當我們有一種sense of belonging的時候.
我們身邊的人會關懷我們.
支持我們.
對我們會有一些期望.
我們放屁.
有人會鞭策我們.
所以有一個高期望.
還有人與人之間多一些互動的時候.
就可以多一些參與.
在那裡人與人之間接觸.
其實有很多正能量給我們.
第三方面就是sense of optimism.
即是樂觀感.
這個就是正向心理學.
幫助我們看一個人.
他是樂觀還是悲觀呢?.
其實是很能夠幫助他面對.
當前的逆境是怎樣.
在這裡樂觀感.
我想簡單和大家介紹一下樂觀感.

$^{121}$其實是Martin Silliman.
他自己建構出來的一個模特兒.
在這裡分辨到樂觀和悲觀的人.
其實他們思考的方式是很不同.
這裡有三個向度.
一個叫恆久性.
一個叫普遍性.
一個叫個人性.
是怎樣去分辨呢?.
一個樂觀的人.
其實他看面前的逆境.
不是一生一世的.
不是恆久不變的.
他會覺得是短暫的.
會過去的.
譬如這段時間.
其實香港的氣氛好一點.
因為我們零確診的都有14個日子.
所以我們會比較輕鬆一點.
但如果一個悲觀的人就覺得.
糟了,不知道什麼時候才能夠上學上班.
好像很長久也解決不了.
所以你問一問自己.
你的思考方式.
看問題,看困難是很長的.
還是很短暫的.
其實就能夠分辨你是一個悲觀還是樂觀的人.
第二方面就是普遍性.
普遍性是什麼意思呢?.
就是說當逆境來到的時候.
是對你整個人的打擊.
還是只影響你一部分.
我想這次疫情真的牽連到我們的情況也很厲害.
經濟,就業,家人之間的關係,情緒.
但一個樂觀的人.
其實他會比較局限了那些影響.
譬如一個人就算短暫失業.
失業他會想.
有些男人很看重事業的話.
失業就等於失去了一切.

$^{161}$但一個樂觀的人會說.
我雖然短暫失業.
但我還有很多事情要做.
我是一個好的爸爸.
我還有很多不同的角色可以扮演.
我不是整個人的失敗.
只是一個局部性的失敗.
局部性的困難.
所以一個悲觀的人會覺得.
一件事都不行,一件事都不行.
第三個向度就是.
一個樂觀的人.
其實他會看.
問題是自己出現問題.
還是一個外在環境的問題.
一個悲觀的人多些會怪責自己.
我時不與我.
我被社會淘汰.
我一無是處.
一個樂觀的人多些會想.
其實這個問題是大為的事.
很多人都會經歷.
很多人都會面對.
我和別人是同一條船.
不是說自己最差,最不行.
所以正向心理學提示我們.
當我們去思考的時候.
你是一個樂觀還是悲觀的人.
其實是很影響我們如何面對逆境.
其實樂觀感.
如果我們看.
其實在正向心理學裡面.
還有另一個範圍是相對應的.
他提到一個叫盼望的心理學.
Psychology of Hope.
這位叫Rick Sider.
他提一個人.
他有沒有盼望的能力呢?.
盼望我們記憶圖經常提及.
原來心理學也有研究盼望.

$^{201}$不過他將它放在一個.
如何在逆境裡面能夠跨越困難.
如果你的盼望能力高.
你就能夠跨越這些困境.
他建構這個Hope.
由兩個元素組成.
一個叫Way Power.
Way Power的意思是找出路的能力.
第二個叫Will Power.
即是他有沒有一個堅毅的能力.
舉個例子.
市區大塞車.
疫情下很少發生這些事.
但你趕時間.
你要由A點去到B點.
如果一個有盼望能力高的人.
其實Way Power的意思是.
他會找盡方法.
即是巴士塞車.
他會下車或者行或者轉地鐵.
他有很多辦法.
有很多方法去面對困難.
譬如這段時間.
香港人在疫情下.
其中一個逆境沒有口罩.
其實我們香港人也很聰明.
自製口罩.
或者做一些布的口罩.
可以耐用一點.
其實我們有很多方法.
很多辦法.
以致我們仍然有這種動力和盼望去做好.
第二個Power就叫Will Power.
Will Power就是那種毅力.
不會很快放棄.
會堅持下去.
譬如這段疫情.
我們不要鬆懈.
我想香港人在這次抗疫的過程裡面.
說疫情的抗疫.

$^{241}$其實也算暫時有一點小勝.
不過我們不要掉以輕心.
我們要繼續有堅毅的能力去面對.
好.
但是這個盼望的心理學.
其實就帶給我們.
Way Power可以說是problem solving的能力.
Way Power說的是有堅毅的能力.
但是疫情這件事有一個特點.
疫情不是在短暫的時間內能夠解決的.
譬如你家裡有一個長期病患的人.
或者你要陪伴一個欠債累累的人.
家人.
那個環境的因素不會馬上消失.
其中有一個基督徒的心理學家.
他整合盼望心理學和信仰.
加多了一個Power.
我想和大家看看.
這個叫做Worthington.
一個基督徒的心理學家.
他就.
Based on Cider.
加多了一個Power.
Way Power.
等待的力量.
我想這個和我們信徒的.
看這個世界.
神在我們生命裡的參與很有關連.
我們會覺得.
我們人是很有限.
我們人手其實可能.
很多東西都不在我們的控制下.
這次的疫情其實都.
讓我們去反省我們自己.
原來是不是每樣東西都能夠靠我們雙手.
能夠打拼.
能夠解決所有問題.
這裡等候的力量.
是我們基督徒.
在有盼望的過程裡面很重要.

$^{281}$所以我很想和大家簡單.
看一段經文.
耳塞亞書40章28到31節.
這裡說.
你不曾知道嗎?你不曾聽見嗎?.
永在的神耶和華.
創造地的主.
並不疲乏也不困倦.
他的智慧無法測度.
這裡是說我們的神.
為什麼我們基督徒.
在這些逆境下有盼望呢?.
因為我們相信.
這個世界是神掌管的.
是祂創造的.
而祂照管著我們.
祂不會疲乏和困倦.
而且祂有祂自己的智慧.
是我們人的智慧不能夠測度.
基於我們對神有這份信任.
我們的盼望來源就在這裡.
然後他說.
疲乏的他似能力.
軟弱的他加力量.
我們在逆境下.
有時都會失去堅毅的能力.
會灰心,放棄.
抗爭的底下都會乏力.
這裡說.
原來我們乏力的時候.
祂會賜能力.
我們感覺到軟弱的時候.
祂會給力量我們.
是一個被加力量.
我們叫做Empower的警方.
然後他說.
少年人也要疲乏困倦.
強壯的也必全然跌倒.
我們不可以靠自己.
你多年輕,多強壯.

$^{321}$都會有軟弱的時候.
但是等候耶和華的.
必重新得力.
他們必如鷹展翅上頭.
他們奔跑卻不困倦.
行走卻不疲乏.
我想這個等候.
其實是很考我們屬靈的操練.
我們信不信神.
祂看顧著我們.
在我們還沒出手.
在我們以為要靠自己力量去做什麼的時候.
有時候我們會說.
神你去了哪裡.
你好像沒參與.
沒伸出援手.
其實很多時候.
好像Eugene Peterson.
我看他一些屬靈的著作.
他特別提到.
其實神往往在我們出手之前.
祂已經在工作.
祂沒有疲乏.
祂沒有困倦.
祂看著這個世界.
我們信不信神是這樣的神呢.
所以我們基督徒盼望的源頭.
多了一項叫做.
Wait Power.
而Wait Power其實是來自什麼呢.
我想我們在這段時間.
應該停下來.
校正自己的視野.
相信神是創造者.
祂的智慧無法察覺.
接受我們人自己的限制.
要看到神在工作.
祂不疲乏.
我們反而要學會放手.
休息.

$^{361}$等待.
等候神.
所以詩篇也有說.
你們要休息.
要知道我是神.
祂有祂自己的時間表.
我希望透過等候的力量.
和正向心理學講.
盼望.
有一個融匯的時候.
我們就能夠相逆齊飛.
即是說兩樣東西.
能夠整合.
互相給我們有不同的啟迪.
下半節時間.
很想和大家說說.
疫情下的情緒處理.
我想在疫情的情況下.
我發覺.
我們有很多情緒.
面對情緒.
應該有一些正確的態度.
一方面我們要留心自己有什麼情緒.
最好能夠給情緒一個名字.
我們叫做「Laming」.
你能夠給自己一個名字.
心情就會比較安頓.
有時候我們會.
好像不太承認自己有某些情緒.
其實你承認了它.
你才能夠擁抱它.
或者叫做給自己有這些情緒.
給予許可.
然後接納.
然後找一個安全的環境.
或者對象可以去分享.
我希望我們在疫情下.
都有很多不同的情緒.
大家記住.
面對情緒.

$^{401}$我們應該有正面的態度.
我想說幾個.
我看到這段時間.
在疫情下.
很多人都會有這些情緒.
包括憂慮.
今天會說多一點憂慮.
有一種很無助.
好像沒想過.
病菌真的像瘟疫一樣.
散播得這麼快.
它都很隱藏.
沒有病徵下都會傳染.
有時候我們都覺得很無助.
不知道怎麼辦.
如果你獨居又不能外出.
其實你很多時候都會有孤單.
在香港的電視.
其實都有做一些特輯.
說在疫情下的人.
譬如沒工作.
只有一個人.
在家裡經常上網.
看電視.
可以整個星期都沒外出.
其實經常這樣屈自己.
都會屈到孤單寂寞.
很大的困難.
還有沉悶.
很無聊.
做家長就很頭痛.
因為小朋友不用上學.
好像今天宣布.
6月8日可以復課.
我們都很開心.
其實都會面對沉悶.
所以在這個情況下.
我們都有很多情緒.
我們怎樣去面對這些情緒呢.
有情緒其實我們需要去調適.

$^{441}$我想建議一個叫做.
Process Model of Emotion Regulation.
情緒調適的過程模式.
即是說原來我們去調適自己的情緒.
是可以分開一個過程.
首先是處境.
Situation.
譬如疫情就說.
不要外出.
因為很容易被傳染.
第二就是Attention.
即是說你會將你的心思,意念,關注放在哪裡.
我都聽到有些太太向我投訴.
她老公很緊張.
半小時檢查一次新聞.
看看確診多少.
令到她越看越焦慮.
看著確診數字高高低低.
自己的情緒都受著高高低低.
第三就是.
面對這些情緒令我們困擾的處境.
我們怎樣去理解呢.
Appraisal.
這個我想我們作為基督徒.
在這方面有多些貢獻.
即是我們用神的角度去看一些困難逆境.
其實幫到我們能夠豁然開朗一些.
最後就是當我們有情緒的時候.
這裡有個字很有趣.
就是Suppression.
即是我們將情緒發出來.
有時會影響到其他人.
有時我們都要學跟自己的情緒共存.
要擁抱它.
能夠多些活在當下.
我將這四個過程.
試下放在現在的疫情底下.
處境方面.
我們怎樣可以調適自己情緒呢.
例如我們少去一些高危的地方.

$^{481}$別人叫你不要去酒吧.
去多過四個人的地方.
或者吃火鍋.
起初中招都多都是一起吃火鍋.
即是我們那個環境是會幫到我們.
不去的環境就少了困難和焦慮.
第二就是你將你的心思意念放在什麼地方.
我發覺我們經常追新聞.
再早一段時間.
香港社會事件.
很多人都看直播.
看著那些暴力衝擊的場面.
其實都令到自己的心情很不開心.
或者都很緊張.
很恐懼.
很多情緒.
所以其中一個方法.
心理學家都會提議.
少看一些有影像的即時的報導.
新聞不是不理.
可能你一晚或一夜看一次新聞.
不要全天候去追.
因為你越將你的專注放在那些地方.
你會越給那些資訊.
特別有影像的資訊.
就影響了你的情緒.
第三就是.
看大家怎樣去看疫情.
剛才我說.
有盼望的角度去看事情的時候.
我們會樂觀一點.
這個疫情會過的.
現在都有點曙光.
香港的情況好像受控一點.
我們盼望的源頭.
剛才說了.
我們相信這個世界是神管理的.
疫情都在它的掌握下.
它會幫我們去面對這件事.
另外就是有不開心我們要學習.

$^{521}$忍受不快.
這個一會兒會和大家說多一點.
正向心理學.
其中一個很重要的就是mindfulness.
就是活在當下.
在當下下.
我們怎樣可以剔除這些情緒.
或者憂慮和困擾呢?.
有些心理學家提議一個很簡單的字.
叫做improve.
用七個英文字母砌出來.
改善這一刻.
簡單逐個和大家分享.
第一是影像.
正如剛才所說.
你看什麼其實會影響你的心情.
我們基督徒可能可以用一些聖經的.
譬如詩篇23篇.
給我們的影像.
好像是一個牧者會帶我們去青草地溪水泊.
多一些有這些影像在你的腦海裡.
你的心情就會好一點.
如果你的影像是暴力的衝擊.
或者疫情的散播得很厲害.
數字不斷上升.
你經常看著這些就會影響你.
第二是剛才說了.
和盼望方面.
我想基督徒禱告其實都很重要.
在這個疫情的過程.
我都陪過一位弟兄.
每星期四晚.
我透過視像和他聊半小時.
因為他在沙士期間都順著.
當時他受這件事影響到心裡都很焦慮.
所以找到信仰作為他的出路.
這次疫情再來.
其實他的焦慮又出來了.
他都會防疫的措施做得相當足.
比他老婆厲害.

$^{561}$有時還要監控著老婆.
回來要做足這件事.
要噴些什麼 洗些什麼.
弄到整個人都很焦慮.
我陪伴他的過程裡面.
都會叫他.
譬如看回詩篇91篇.
說到耶和華是我們的保護者.
他的保護是全天候保護得最好的.
大家可以看回詩篇91篇.
其實是一篇很好的詩篇.
我鼓勵他多些和神禱告.
透過禱告將這些憂慮交託給神.
其實他已經做得很足.
接下來的事要仰望交託給神.
他不是放鬆自己.
他只是很緊張.
另外第四個.
Relaxation.
這段時間很多人都盡量找方法去放鬆自己.
香港人疫情好一點就去行山.
現在的體育設施開放一點.
可能去做運動.
這些都是很重要的.
怎樣可以隔膜.
譬如對我來說.
我聽聽古典音樂.
放鬆一下自己.
第五方面.
此刻只做一件事.
其實我覺得都幾重要的.
因為當我很專心投入去做一件有意思的事情的時候.
那些其他情緒的困擾.
自然就會放在一邊.
你很專心地.
特別有意思的事情.
你投身去做其實很重要.
第六就是Vacation.
放假.
現在都不能去外地旅行.

$^{601}$其實一放鬆一點.
人們就開始想.
會不會去深圳旅行.
還是去那裡旅行.
其實我這裡只放假.
不單是外出去旅行.
而是說你為你自己憂慮緊張的事情.
不如放下它.
放下自己.
放下假.
不要那麼擔心.
意思就是這樣.
最後.
E字就是Encouragement.
鼓勵.
其實是很希望.
我們人與人之間.
例如剛才說逆境.
Sense of belonging.
這件事很重要.
這段時間.
其實我覺得.
信徒都很能夠發揮一個.
彼此關顧的能力.
例如透過用Zoom.
電話.
WhatsApp.
互相有一種慰問.
鼓勵.
其實都成為了我們.
面對困難的一個很好的助力.
所以大家.
當你有情緒不好.
或者有焦慮.
憂慮的時候.
試試記下這七個英文字母.
Improve.
即是改善我這一刻的情況.
好.
信仰怎樣看待我們這段時間的憂慮呢?.

$^{641}$所以我都選了一段經文.
主耶穌教導我們不要為甚麼憂慮.
我覺得它裡面的提議.
很值得我們參考.
所以我們很想.
對一些可能你受疫情影響.
無論你憂慮疫情.
憂慮經濟.
憂慮你的工作.
很多事情可能都煩擾著我們內心.
我很想看回耶穌給我們的一些提示.
成為我們面對憂慮的一個出路.
我先說說其實憂慮是甚麼呢?.
憂慮其實是一種對將來的思考.
通常想起將來.
即將來的日子.
會令你很焦慮.
好像覺得自己很困擾.
如果你過份焦慮或過份擔憂.
可以成為一種心理的病.
這叫做Generalized Anxiety Disorder.
有甚麼特徵呢?.
這裡都是說.
譬如你會不同的範疇.
你都會持續有很多的憂慮.
為了解決問題.
你會過份思考.
這裡說Overthinking Plans and Solutions.
根本不需要你解決.
你會想得過濃.
第三.
你會看當時的困境很有威脅性.
其實未必像我們想像中那麼厲害.
第四.
就是面對將來很多的未知.
Uncertainty.
影響到你做決定的時候.
你都很猶豫不決.
又不可以暫時放下.
好像剛才說Improved Moment.

$^{681}$這樣的做法你都做不到.
即是很多焦慮.
不能夠放鬆自己.
經常都好像很繃緊,很緊張.
甚至影響到你不能夠集中精神.
又或者睡不著覺.
這裡其實都有一些叫Physical Symptoms.
即是累,難睡覺.
肌肉繃緊.
甚至好像坐立不安.
很憂慮,出汗,想吐,很煩躁等等.
如果大家有這些情況出現.
其實我們都看為嚴重.
所以其實這段日子如果大家有憂慮的話.
不妨去找人幫幫忙.
另外就是希望透過耶穌在馬太福音.
6章25節開始.
在這裡說我們人如何面對憂慮.
找到一些出路.
好,很快和大家分享一下這段經文.
馬太福音6章25節.
這裡說:所以我告訴你們.
不要為你們的生命憂慮吃什麼喝什麼.
或為你們的身體憂慮穿什麼.
生命不勝於飲食嗎?.
身體不勝於衣裳嗎?.
這裡耶穌挑戰我們.
我們擺放的東西的次序是否正確.
他覺得有些重要性.
應該優先次序放得高一點.
你說不是,這次疫情要我的命.
我的憂慮其實很確切.
其實我想在我們信徒裡.
我們肉身有生命.
但更重要的是整全的屬靈生命.
我們的擔心會否過了一部分.
我們沒有看到.
原來我們的生命在神的掌握下.
祂會托住我們.
所以你要想一想.

$^{721}$哪些是你這段時間.
在神的眼光裡看為更重要的東西.
這裡耶穌繼續說.
你們要看一看天上的飛鳥.
也不種也不收.
也不在倉裡存糧.
他教我們去看看外面的世界.
其實你發覺一個憂慮的人.
有一個特點就是inward looking.
很多時候只顧著自己.
在自己的世界裡不斷轉動.
很多擔憂.
主耶穌叫我們出去看看.
看看大自然.
看看飛鳥.
走走海灘.
因為當你看著大自然的時候.
你會看到大自然都在神的照管下.
運作得很正常.
很順暢.
我們用不用這麼擔心呢.
對一些情緒低落的人.
在心理治療上.
有時候都會建議他們.
譬如種一棵花.
出去走走.
不要經常盯著自己.
我覺得這個outward looking.
對我們的焦慮感有幫助.
所以加上耶穌提到.
看這些東西的時候.
他想帶出一件事.
就是比較.
在比較之前我們再看一件事.
他就說.
你們哪一個能藉著憂慮使受掃多加一刻呢.
或者使身量多加一爪呢.
其實他在說憂慮是沒用的.
沒作用的.
做不到什麼.

$^{761}$我想把憂慮沒用這件事.
放在一個時間的框架裡說.
如果一些事情已經過去了.
都發生了.
你都不能改變.
為過去憂慮其實沒什麼意思.
此時此刻的事情.
其實我們要接受自己人的限制.
很多事情都不在我們掌握.
操控下.
我們要接受.
所以此時此刻.
現在的事情未必能夠受你控制.
我們能夠控制.
就是我們做好自己防疫的措施.
其他事情.
疫情什麼時候走什麼時候來.
不受我們控制.
第三.
對於將來來說.
我們憂慮的事情.
其實很多時候未必一定會發生.
我覺得這是我們人生的經驗裡.
經常都會體會到.
例如你下星期要讓你老闆.
做那個表現評估.
他要見你.
你可能想來想去.
怎麼回答.
如果他挑戰我的時候.
我應該拿什麼證據告訴他.
我怎樣.
我們用了很多心思意念.
去準備.
當然準備不是不好.
但如果你過份的話.
你會有一個發現.
你花了很多精力去憂慮的事情.
其實到那天.
你老闆可能都沒空.

$^{801}$見你五分鐘就走了.
其實很大機會.
未必會發生.
所以我們看到憂慮的沒用的一面.
就是因為我們花了精神去預計.
其實那些事情.
到那個時間來的時候.
可能未必會發生.
所以我們不需要為那件事憂慮.
另外耶穌就做個比較.
我們比飛鳥,野地的花更加寶貴.
上帝尚且都照顧這些飛鳥,野地的花.
何況我們呢.
我想我們做信徒.
其實要信得過.
我們的生命在神眼中是寶貴的.
祂會珍惜我們.
有這個安全感我覺得都很重要.
我們不要心慌慌.
覺得自己生命不保.
所以耶穌就跟著說.
你們這些小信的人.
野地裡的草.
今天還在明天就丟了爐裡.
上帝還給他這樣的裝飾.
何況你們呢.
原來憂慮.
嚴重一點說.
原來是一種小信的表現.
我們會不會提醒自己.
我會不會小信了一點呢.
我是不是應該對神有多點信心呢.
好,接著.
這個金句大家都很熟悉.
不要為那些事憂慮.
吃什麼穿什麼.
然後你們要先求祂的國和祂的義.
這些東西都要加給你們.
原來將我們的專心.
是一些有意義的目標.

$^{841}$回應神的對我們的照明.
這段日子.
我發覺我們很多信徒都把握這個逆境.
例如發展一些事工.
用多一些科技幫我們傳遞信息.
其實是擴展一些神的角度.
或者對敬拜.
用一些不同的方式去達至效果.
我們都用了很多創意.
很多的力量.
如果我們專心去回應.
先求神的國和神的義的時候.
其實好像我剛才所說的一種昇華的作用.
我們相信.
我們擔心或焦慮的事.
神早就知道.
祂會為我們預備.
或許祂會為我們準備更好的東西.
可能我們只想要一個口罩.
但原來神要給我們一個保護.
給我們出人意外的平安.
豈不是更加好嗎?.
所以我們不要花太多時間.
只在憂慮裡.
選擇一些有意思的事去做.
做好它.
在神的角度裡有份參與.
最後就是.
不要為明天憂慮.
因為明天自有明天的憂慮.
一天的難處一天當就夠了.
我們都是Live by Grace.
One day at a time.
既然憂慮是沒有用的.
例如今天已經有很多事要做.
你先做好今天的事.
不要想到下個星期,下個月.
這段時間我都.
本來每年暑假.
我都會有一個家庭旅行.

$^{881}$本來想著.
六月要去紐西蘭.
興之勃勃.
怎料疫情來.
現在都取消了.
原來我們為將來計劃了很多事.
有時都要順應環境.
要作出一些調適.
今年去不到就明年吧.
我們不要為將來太多憂慮.
今天就搞定了.
一天的難處一天當就.
我們怎樣去Live by Grace.
好,所以很簡單和大家總結一下.
如何面對疫情裡的憂慮.
主耶穌給我們的提示就是.
我們排好生命的次序.
不要太深入.
多點去看看這個世界.
看看大自然.
要知道憂慮是沒用的.
我們的生命天賦是寶貝.
比飛了野地的花還要多.
而這種憂慮.
過份的憂慮是一個小信的表現.
我們要調適自己.
要回應.
要先求神的國和神的義.
我們專心在有意義的工作裡.
有一種昇華的作用.
靠著主的恩典.
Live by Grace.
好,今天就和大家分享到這裡.
希望我把正向心理學.
把信仰融匯來.
和大家說說如何面對疫情和疫情.
對大家有一點幫助.
祝福大家.
今天就和大家分享到這裡.
拜拜.

$^{921}$(音樂播放).
(字幕由 Amara.org 社群提供).
\newpage



\section{}
\label{sec:QdUuSNEVBIc}
\textbf{【疫有嘢學 │ 延SUN在線】靈旅四季的調適與群體中的同行|潘怡蓉博士}
\newline
\newline
連結: \href{https://youtube.com/watch?v=QdUuSNEVBIc}{\texttt{ https://youtube.com/watch?v=QdUuSNEVBIc}} ~~~~ 語音日期: 2020-06-14 
\newline
\newline
\hyperref[sec:qtqZXfdLO9c]{\small{< < < PREV SERMON < < <}}
~
\hyperref[sec:index]{\small{[返主目錄]}}
~
\hyperref[sec:4FekoGe9H60]{\small{> > > NEXT SERMON > > >}}
\newline
\newline
$^{1}$主席.
未來七月份 安妮·斯姆會在延伸部教一科名為.
生命工程 靈情培行者的課程.
歡迎大家報名參加.
而今天安妮·斯姆亦會首先和我們分享一個相關的課題.
就是靈女四季的調色與群體中的同行.
沿著透過潘怡蓉博士今天的分享.
讓我們在面對社會多方面挑戰的環境下.
學習與耶穌同行同工.
作一個靈性塑造的生命陪伴者.
親愛的弟兄姊妹 平安.
今天要和大家分享的主題.
是過去這一年放在我心中的一個題目.
藉著今天短短的一個小時.
希望可以和我們的同學和弟兄姊妹一起思考.
藉著我小小的分享.
我想說兩個焦點.
就是生命好像一個旅程.
有四季.
四季雖然是不同.
但最主要的是我們如何去面對.
如何去適應.
所以調色是我今天第一個的重點.
第二個就是在一個忙碌的時代.
在一個多變的世界.
我們如何和教會的弟兄姊妹一起去同行.
所以同行是我今天的第二個焦點.
在十年前我和我先生在歐洲回來.
我們住在德國差不多有十年了.
最初回來香港.
有一個感覺就是.
這個城市很忙.
大家要知道.
我們住在德國.
但我後期就去了比利時.
進修我的靈修學.
所以由那個世界回到香港.
我就感覺這個城市不單單人多.
個個是很匆忙.
甚至是有些疲倦.

$^{41}$回到教會有很多的活動.
但慢慢我在幕會的場合.
面對我的弟兄姊妹.
在忠臣的教書和同學的相遇裡面.
我就發覺原來這麼多人的城市.
大家這麼忙碌的時候.
原來是很孤單的.
所以孤單這兩個字.
就令我想.
如何是我們教會的群體.
那個聖徒的相通.
是可以更加實際的.
借助一個的寧靜同行.
令我們可以一起在面對時代的挑戰.
我們不感覺這份的孤單.
甚至是享受孤單中.
和上帝的親近.
另一個就是.
過去的這一年.
香港的處境有很多的變化.
不單單社會的運動.
是多元化.
甚至過去這半年的疫情.
是全世界的.
我們是面對一個新的時代.
甚至這些的挑戰.
不是我們過去熟悉的.
亦都可以這樣說.
就是我們走的一條新的旅程.
未必是可以用以前.
我們的經驗.
或者上一輩的.
很多的類似的方式去應對的.
我們需要一起尋求.
聖靈的帶領.
教會是耶穌基督的身體.
我們如何在這樣的時代.
處境.
繼續去和耶穌基督去同工.
去同行.

$^{81}$以致我們可以在神的角度.
回應到神在基督裡面.
對我們的呼召.
所以今天的一個題目.
借著一個調適.
和同行這兩個焦點.
我希望可以鼓勵到我們的同學.
和弟妹.
學習.
如何在群體中.
我們可以豐富彼此的生命.
一起經歷聖靈的帶領.
甚至可以一起尋求.
基督對我們的呼召.
而可以活出我們的召命.
容許我用四個大點.
跟著四個小點.
和大家去分享.
第一個我會和大家分享的.
什麼是靈性的旅程.
第二方面就是.
和大家分享.
靈裡的四季.
其實就是旅程中有變化.
而這個變化.
是會帶出我們的轉變.
轉變就是一種成長.
第三方面就是.
面對一個未知的旅程.
我們如何去適應.
如何去調適.
第四方面就是.
群體中的同行.
究竟是如何的同行.
我會分四個小點.
第一個我想強調.
這種同行不是走在一起.
這種同行是在天國裡.
我們一起走路.
一起朝著永恆的角度.

$^{121}$會有共難和成全的一天.
第二個就是.
什麼是靈性的同行者.
這一類的人是怎樣的呢.
第三方面我們先提及.
同行的關係.
是如何去經營.
如何去進行.
第四個就是.
一些比較詳盡的.
和我們的同學或姊妹分享.
我們如何進行一種同行的靈修對談.
而這種對談就是一種相遇.
和自己相遇.
和聖靈相遇.
和對方相遇.
以致我們彼此在基督的角度相遇.
首先靈性的旅程.
我小小的定義.
這個旅程就是一種成長的旅程.
我用它來讀出我寫下的幾句話.
靈修的旅程是追求進入與上主的愛的相交.
所以是一種共同的旅程.
我們最終的就是與上帝的相交.
成為了一體生命.
或者說靈性的最深層的基礎.
第二方面就是.
我們的生命不是生活的.
這個生命只有在上帝的角度中.
去進行,去彰顯.
可以在基督的理邊.
藉著聖靈.
越來越走出自己的自我中心.
因著上主的接納,盼望,連續.
我們可以轉化,更新.
進入與上主,與人,與世界的共融.
所以我們在靈修的旅程中.
可以稱自己是一個朝聖者.
這個旅程會令我們不停地.
在自我中心,在基督的理邊.

$^{161}$藉著聖靈.
轉化成為一個更加像耶穌基督的生命.
而可以合法基督貢獻在上帝的角度.
這就是我們的旅程.
所以這樣的旅程.
是有一個不停地進展的過程.
還有它有特定的方向去進行.
我們先是稱之為靈修的旅程.
而在這個旅程的過程中.
四季代表春,夏,秋,冬.
但更加在靈修傳統的理邊.
我應該這樣說.
除了用春,夏,秋,冬.
去形容我們與上主的關係.
更加多的簡單分類就是.
在我們旅程的過程中.
我們會經歷與上主的關係更加近.
但間中又會感覺到有些遠.
甚至是疏離.
這種變化會不停地糾結在我們旅程裡.
不單有這種變化.
還有我們行的旅程.
原來不是一條直路.
是有彎彎曲曲.
甚至有時候會走一些冤枉路.
甚至我們會走到低谷.
走到高山.
正正是這個旅程.
有春,夏,秋,冬的變化.
有彎彎曲曲的變化.
加上有高山,低谷的經歷.
所以我們就會很豐富.
這種豐富令我們在不同的季節.
經歷不同的場地.
春天的時候.
我們會發覺像夢幻的場地.
給我們的生命有一種希望.
有一種夢想展現.
夏天的時候我們感覺.
場地恩待我們的時光.

$^{201}$我們的事業.
我們的生命.
我們的生活.
有很多很燦爛的展現.
令我們感覺這個世界充滿恩典.
秋天的時候.
是一個修成的時刻.
無論是我們努力過的.
我們培育過的.
我們曾經用心去耕耘的.
神賜恩.
我們看到那個成果的展現.
看到上帝的國度.
因為我們的參與.
而令我們的生命不單有意義.
我們更加看到.
是有一種喜樂.
修成的喜樂在中間.
冬天的時候.
我們發覺.
盼望消失了.
而很多的.
我們積蓄的糧食.
或者我們曾經付出的努力.
現在消耗著.
甚至是見底了.
力量用完了.
黑暗充滿.
在冬天的時候.
我們所經歷的上帝.
似乎是隱藏的.
所以正正就是這一種.
似乎看到上帝有隱藏.
似乎上帝有展現.
似乎上帝豐富的同在.
似乎上帝在安揭中.
給我們歡樂.
不同的近的遠的關係.
令我們體驗上帝.
是非常之豐富的.

$^{241}$因著體驗上帝的豐富.
我們可以安然的.
慢慢接納人生中間的春夏秋冬.
因為我們知道.
不離祥的上帝.
所以我們的感受會有不同.
在生活的經歷.
會有不同的面貌.
但是我們可以見到.
原來真理是這麼豐富.
原來上帝的恩典.
在不同的場景.
是以不同的面貌.
對著我們去展現.
於是我們的信仰.
慢慢成熟了.
見得到的時候.
我們有信心的.
見不到的時候.
我們是在基督的裡面.
憑著在基督裡面的信心.
我們可以繼續的.
去行我們的旅程.
繼續去參與在上帝的國度.
甚至這一種的豐富與成熟.
令我們對自己的體驗.
都是豐富了的.
在平順的時候.
我們覺得教會的每個人都很可愛.
但是遇到一些被拒絕.
甚至是一些受損的經歷.
我們發覺自己就收縮了.
我們把自己關上門.
封閉.
於是我們生活在一個寒冷的冬天裡.
但是正正在寒冷的冬天裡.
我們有一種冷靜.
看回自己過去.
究竟什麼叫做人際關係好.
什麼叫做土人歡喜.

$^{281}$什麼叫做吃得開.
和服侍有果核.
就在這種冰冷的裡面.
原來我們可以看到很多不同的自己的面貌.
甚至這一種對自己的認識.
都令我們可以冷靜一點.
去看清楚對其他人的認識.
所以這一種的豐富都包括.
我們對於自己和他人的人性.
有多面態.
這一種的多面態.
令我們可以重新謙卑去做人.
有非常大的因典.
你和我都是罪人.
所以我們也都在寒冬的裡面.
發覺那份因典的美麗.
是因典令我們可以看到對方的美麗.
所以也都寒冬.
令我們期望自己的心靈.
再一次的蘇醒.
所以這一種的認識上帝.
認識自己.
認識他人.
就是靈情的豐富.
帶來靈情那種的盼望.
和那種繼續向前行的期待和轉化.
而這一種的轉化.
正如我剛才所說的.
慢慢慢慢.
我們就會在那種有血氣的自我中心.
慢慢就會黑暮基督.
慢慢就會希望進入上帝相交的美麗.
這就是靈情三個階段.
也是古典的靈修學所說的.
生命,成長,靈情進行的三個階段.
不停的對自我的明白的蘇醒.
對耶穌基督因典的驚訝.
和被上帝的愛的吸引.
但是我們的靈情.
一直就會在這三個階段裡.

$^{321}$春夏秋冬的循環.
和高山低谷的起伏.
經歷自己.
經歷上帝.
看清楚這個世界.
真正靈情是向前行的.
所以我稱之為一個未知的旅程.
在靈修傳統裡.
甚至會用一個字.
叫做脈觀自己.
脈觀的意思就是.
不是自己在那裡自我分析.
也不是自己在那裡評估處境.
而我們可以由上帝的心懷意念.
由上帝的眼光.
去看回自己.
也試試由上帝的眼光.
借助聖經的真理.
我們去看回上帝所看到的世界.
所以一個行靈修旅程的人.
他需要這種開放的眼光.
其實我們對自己是未知的.
甚至我們對所愛的人.
所熟悉的人.
我們也需要這種未知的眼光去看他們.
我們才會有驚訝.
我們才會看到.
創造的上帝.
一直在我們自己的生命.
其他人的生命做新事.
對於處境.
對於世界.
我們都要知道.
舊因的歷史是不停向前的.
從來沒有一個時刻.
我們可以回頭.
歷史不停地向前進.
直到新天新地那天的來臨.
所以歷史是不會重覆的.
所以我們需要這種新的眼光.

$^{361}$當我們用這種新的眼光.
去看自己.
看人.
看世界的時候.
這個未知就不會變得這麼恐怖.
所以這個未知.
令我們有一種期待.
就是看到上帝正在做新的事情.
我們開始看到自己的成長.
我們也開始看到別人的變化.
我們甚至看到世界.
是不停地更新.
不為停頓.
不為打頓.
不為睡覺的上帝.
日夜帶領歷史.
朝著他的心意.
繼續前行.
我們的信心.
就是在這種新的眼光裡.
看到上帝不停地成就新的事情.
所以當我們說調適.
這兩個字的時候.
我們自己.
如果可以在這兩個字裡.
想一想.
什麼叫調適.
我們在鏡頭前的同學.
或者頂字位.
都想問一下你.
雖然你不在現場.
但是很感受你聽見我說話.
你對調適有什麼定義呢?.
譬如.
我以前讀輔導的時候.
調適就有它很專門的名詞.
如果你調適得不好.
甚至會出現一個.
叫做Adjustment Disorder.
就是會有一個適應上的.

$^{401}$偏離或者適應不良產生.
什麼時候你會適應不良呢?.
通常就是有新的事情發生.
就是這些新的事情.
新的人事.
令你適應不了.
很簡單.
用我解釋的幾個角度.
你就會明白的.
你這麼聰明.
譬如你轉工了.
你去到新環境.
你就發覺.
這裡的文化和舊公司是不同的.
你經常會和舊公司去比較.
然後就發覺.
為什麼會不同呢?.
為什麼它不會和以前的不同呢?.
你就會發生一些調適上.
想要以前的.
不能接納現在新的.
就會出現一些問題.
還有.
調適什麼時候.
都會出現在我們人生不同的階段.
譬如新婚.
本來剛剛結婚很開心的.
結婚一年就生了一個寶寶.
有些人就會很開心.
但開心沒多久就發覺.
原來晚上不能睡覺.
原來老婆的第三者.
你們的第三者就是那個寶寶了.
我沒有想過的.
本來是幸福家庭的圖案.
誰知道老婆現在不理你了.
這個都要調適的.
什麼時候還有調適呢?.
很簡單.
譬如你要退休了.

$^{441}$以前有人稱呼你什麼老師.
以前有人稱呼你什麼主任.
現在什麼先生都沒有了.
直接叫你的名字.
或者叫你Annie.
就這麼多了.
Annie是我的英文名字.
所以.
我們生命中間.
出現很多的調適.
過去香港這一年.
特別是每一個人都要適應.
無論是對社會的改變.
對疫情的未知數.
我們都在走一個未知的旅程.
在一些心理報告裡.
有這樣的一個角度.
那些叫做彈性.
就是比較有彈性的人.
比較容易轉一轉彎.
起步就走新的路.
但那些覺得什麼時候會回到以前.
就是比較少一點有彈性.
就經常等啊等啊等.
想等回又回到以前.
事實上很多東西都回不到以前.
所以靈活度或者彈性度.
跟這個調適是很有關的.
不過今天既然我是比較以靈修的角度.
去跟我們的頂姐妹和同學分享.
我提出有六點.
簡單的去解釋.
第一個就是既然上帝是帶領我們.
走新路的上帝.
還有上帝在我們生命中.
是不停的借著聖靈創造我們新生命的上帝.
我們一向都是天天有新事.
天天有新路要走.
既然這樣的歷史是天天有新的進程.
我們真的要預備.

$^{481}$教會是一個走新路.
尋找上帝新的心意.
在世界展現的一個群體.
所以第一件事反而不是想著走新路.
或者有彈性的去面對或者回應.
第一個是敬畏的心.
因為做新事的是上帝不是我們.
我們就需要去聆聽.
聆聽聖經的真理對著一個新的時代.
聖靈要啟迪我們.
以什麼樣的新的角度去理解.
聆聽我們的群體在新的處境.
有什麼樣的痛楚.
有什麼樣的歡樂.
有什麼樣的新的體驗在中間.
我們彼此的聆聽.
我們個人借著對上帝的尋找.
我們聆聽究竟在新的處境.
新的時代.
上帝要我們這個人.
怎樣找到我們在他角度的定位.
我們可以對上帝做出更好的回應.
有更好的選擇.
所以我們天天都做新的決定.
我們都有新的變化.
新的決定在中間.
所以聆聽.
無論是個人.
無論是群體.
第二個就是.
在靈修的傳統裡.
其實很強調奧秘的上帝.
雖然我們是認真去尋求.
聆聽聖經.
還有我們借著教會歷史.
在歷史中間我們聆聽歷史的經驗.
以致我們可以去尋找.
在我們新的處境怎樣去行.
不過大家要知道.
上帝的豐富.

$^{521}$完完全全超過我們一切的掌握.
所以那個豐富的真理.
值得我們繼續去聆聽.
上帝的真理繼續的啟迪.
借著聖靈.
令我們的明白.
或者那個頓悟.
可以更加的豐富.
第二個上帝繼續的做生事.
在歷史的中間.
他奧秘的展現.
我們需要有一個熟悉的眼光.
去明白.
去看到.
所以開放就很重要.
特別.
我剛才一直強調的.
就是新事新路.
而加上我們對奧秘的上帝.
那種不能掌握.
不能完全掌握的.
這個開放就要知道.
不能以以前我們舊有的框架.
去完全的框住一件新的事情.
或者一個新的社會的現象.
或者一個世界的新的一個疫情的展現.
我們需要尋找新的解讀的神學框架.
或者一個聖經的事業.
或者一個群體的領袖.
以致我們可以更加的好的去解讀這些新的事物.
通常沒有靈活度的意思就是.
我們很想呢.
就吸引過去的框架.
最好就完全套現現在新的處境.
新的經驗.
新的解讀.
而這樣的時候其實我們讀不出一些新的角度.
甚至是比較難以新的回應去回應上帝.
所以需要有一種的開放.
要放下過去的一些框架.

$^{561}$以新的眼光去看自己.
以新的眼光去看處境.
看身邊的人.
甚至看我們的教會.
看上帝要在教會做的新的事情.
在這樣的一個過程.
這個調適呢.
仍然有兩個很簡單的熟悉的操練.
一個就是.
我們既然不能夠命令上帝.
又不能框著上帝.
又不能完全叫上帝按著我們的心意去行.
這種開放表現在哪呢.
就是一種馴服.
就是上帝我願意這樣去做.
但是你永遠是有空間.
將我的旅程轉彎.
你永遠是有權.
有主權.
揭示一些新的角度.
令我可以悔改.
所以這種樂意悔改.
又樂意隨時被神去調節的心情.
才是叫做開放.
如果沒有的話.
哇 上帝我對你很開放.
但是你不要動我.
我是不會變的.
這個開放就沒有它的意思了.
開放都包括剛才我所說的悔改.
就是如果上帝你願意按著我的心意.
我很感恩.
但是不要按著我的心意.
按著你的心意.
甚至經常都是調整我的眼光.
特別是那種自我中心.
自我自意.
又驕傲.
又保護自己的眼界和思想方法.
可以慢慢讓你去調整.

$^{601}$這個都叫做開放.
第四個就是開放.
另一個動作就是放手.
有沒有人祈禱說.
上帝我將一切都交給你.
不過第一步你要這樣做.
第二步你就是這樣做就行了.
我就將一切交給你了.
我想這種放手.
是我們安排了.
然後讓上帝吸引.
或者讓上帝執行.
開放是很重要的.
是真的我們盡力了.
有一部分就被動了.
你什麼都做了.
我們真的是信得過.
祂是掌管歷史的上帝.
你的小小的手放開.
祂大大的手仍然是承托著.
所以更多的仰望.
更多的倚靠.
這種倚靠表現在哪裡是開放的態度呢.
就是你盡力了.
不過路這一步就是這樣.
我們是忠心的.
靠上帝給我可以走這步路的力.
就走好這步路.
忠忠心心的走好它.
第二步路呢.
一天因典一天用.
步步都有因典.
這個就是一天的開放.
每一天的日用飲食.
靠上帝賜給我們.
就是這種倚靠.
倚靠的意思就是.
不能夠一次靠兩十天的智慧.
甚至是天上的力量.
天上的喜樂.

$^{641}$每一步路我們靠上帝.
甚至可以這樣去鼓勵我們的同學和頂姐妹.
過去這一年我們是辛苦.
但是在靈修歷史裡面.
我們看到特別很多.
很寶貴的靈修經典.
它出現的處境都是很艱難的.
就是因為艱難.
所以很多很特別的熟練的功課就學得到.
在幾年前我們的處境比較平穩.
還有世界的波動.
無論是經濟.
無論是疾病.
無論是戰爭.
相對來說我覺得.
我在歐洲的時候.
就是十年前.
我感覺是非常穩定.
甚至我住的城市是哈德布.
我感覺是古城是十年不變的.
我讀書是比利時的老文.
那個都是中世紀的古城.
所以對我來說.
十年前我的世界是千年不變的感覺.
但是回到香港.
這十年其實香港變化得很快.
所以我就學到在歐洲那個安然的環境.
是沒有學到的.
特別是在香港的處境.
昨天的智慧是沒有的.
昨天的課程.
今天的社會環境.
今天課程要討論的.
完全跟我昨天預備的.
是真的不同的.
心底的說話和感動.
一篇講章.
很辛苦預備了兩個星期.
很不容易輪到我說的.
我的教會.

$^{681}$我現在也在做教會顧問.
最近還在連播.
因為有些變化.
限聚令變化.
或者是疫情有什麼變化.
你的信息.
裡面的感動是不同的.
很體驗.
以前我做傳道.
或者老師不需要學的功課.
以前PPT.
就是我們的筆記.
打完應該安然睡覺.
現在安然打完.
第一天要看有什麼新聞.
昨天可能不能用.
所以步步要靠上帝.
這種功課在淑玲的初年.
我們在平信裡是學不到的.
不過很體驗神是信實的.
這種體驗神的信實.
比起以前的體驗.
深刻很多.
第二方面.
如果你問我.
做老師做牧者這一年多.
我對同學弟妹.
最多感覺需要依靠上帝的地方.
或者幫助弟妹依靠上帝的地方.
是什麼呢.
我想就是很多的焦慮.
很多的沮喪.
但是那些焦慮.
不是一兩個人說安慰的話就可以.
因為面對一些處境.
或者全世界的轉變.
不是一個人兩個人說.
我們這樣改變就可以解決.
明天就不需要擔心.
現在我們很多的挑戰大到.

$^{721}$其實不是一兩個人可以解決.
在這個焦慮裡.
真的就是學到.
靠上帝怎樣有天上的盼望.
靠上帝怎樣有喜樂的.
有意義的.
每一天做一些小事情.
而令到我們心裡的平安.
在這些小小的喜樂.
小小的天上的盼望的展現的時候.
我們可以有一種.
叫做世界不能夠給到我們的平安.
面對一些的沮喪.
就會發覺.
原來世界以前我們覺得很安然的.
可以掌握的.
是這麼快就可以失去的.
這種失去.
英文叫Loss.
其實就是Depression.
或者是沮喪.
或者是憂鬱.
一個很重要的源頭.
過去所擁有的.
所依靠的.
所穩固的.
可以一夜之間就沒有了.
過去你覺得是你手中的.
可以很快就被人拿走的.
所以這種沮喪.
當我們慢慢發覺.
原來我們所有的投資.
所有的努力.
如果不是因上帝的賜予.
如果不是因為上帝對我們的肯定.
我們人世間一切的肯定.
一切的擁有.
會很快過去的.
於是我們重新的.
在靈修的旅程.

$^{761}$發覺這些的.
波動的背後.
都有一份上帝的等待和邀請.
就是邀請我們再一次思考.
我們的價值觀.
我們一切的投資.
我們一切的肯定.
從哪裡而來.
當我們再一次去定精.
一切的根基.
在上主那裡.
從上帝的角度.
看回自己今天一切的努力.
你才會更加穩妥的.
無論憂或喜.
得事或不得事.
即或不然.
我們都充滿一種喜樂.
和盼望.
繼續在上帝賜予的.
你的立根之地.
就是你現在站在的地方.
繼續將自己的生命.
在天國裡去努力.
所以你就可以和不安去共處.
慢慢的.
你會視為這種不安.
或者帶給我們一切的波動.
就是上帝對我們的邀請.
不停的淨化我們.
不停的去奧煉我們.
以致我們的生命.
越來越在一些掙扎.
苦難.
痛苦.
思念的裡邊.
我們可以慢慢的去成熟.
有新的喜悅.
和盼望.
好.

$^{801}$來到最後的一地點.
就是我想分享的群體中的同行.
下面仍然是有四點的.
所以容許我.
有少許的講解.
逐點逐點的和大家分享下去.
很多人說我是你的同行.
所以我每天和你聊天.
我可以幫你解決一些問題.
我這樣就可以和你同行.
這的確是一個層次的同行.
不過今天所說的同行者.
我更加要強調的是.
朝向上主角度.
在你這邊參與的同行者.
因為生命的旅程.
如果我們說是一個靈修的旅程.
或者是一個靈性成就的旅程.
它應該是一個投資的方向.
甚至是一個永恆的天國的盼望的方向.
我們這樣去行才會有目標.
所以很多人行行下.
同行就變成我和你的友誼就是這麼多.
所以如果沒有了天國的horizon.
那個遠景.
那個意象.
行一行就變成我和你的感情是這麼多.
在朝向天國的靈情的同行.
我們最重要的有幾個焦點.
就是因我和你的相遇.
我們可以更加見到.
上主在你的生命做著什麼事情.
藉著我和你的交談.
我們可以聆聽.
神在你的生命中做著什麼事情.
藉著我和你的祈禱和分享.
我們可以分辨.
神在我們的教會做著什麼新的事情.
而在上主的角度.
我們的教會如何更好地去參與.

$^{841}$祂做的新的事情.
而在我們每一天的生命的旅途.
我們可以藉著交談.
可以見到上帝的族人.
祂對我們的照明.
是要怎樣去展現.
所以藉著這種同行.
我們慢慢地可以分辨.
上主的靈.
在我們裡面微小的聲音.
在我們身邊的弟妹.
在我們教會的群體.
祂在發展著什麼事情.
在這個時代.
祂對我們教會的挑戰.
呼召.
邀請.
所以這種同行的交談.
或者這種同行的分享.
就更加幫助我們個人.
教會群體在時代中間.
更加好地去回應上帝的角度.
這才是靈情的同行.
同行的焦點所在.
第二個同行者.
這個同行者.
我今天所分享的不是.
我們出去靈修中心.
去見靈修指引或者導師.
那些很重要的.
弟兄姊妹.
那些我完全沒有否認.
是很重要的.
因為很多的牧者.
很多弟兄姊妹.
都經歷了這種幫助.
甚至自己的人生階段.
很多時候要.
想一想.
去推行的時候.

$^{881}$這些靈修中心的導師.
對我的幫助都很大.
不過今天我想強調的是.
群體中.
因為群體中的牧者.
我們的團體導師.
我們的小組長.
甚至你是教會中間比較成熟的.
你會來修中神的現身報的課程.
我相信你都信用你比較追求.
甚至你是在服侍中的.
甚至我們的同學中間.
都已經有些牧者回來.
討一討進修的.
所以我更加想說.
你們很重要的.
在靈修傳統裡.
靈修指引.
是群體中間.
發生中的.
甚至這個責任.
是給牧者.
給團體導師.
給小組長.
甚至給有心.
生命比較成熟的頂智妹.
正正因為我們是性徒相通.
所以這一種的同行.
是擺在更加成熟的.
生命的頂智妹的.
木頭上面的.
因為你比那些剛剛幼嫩的初信者.
你經歷上更加豐富.
你對人性的體驗更加有深度.
所以你的經驗.
是可以幫助那些人.
借著你分享生命的故事.
生命的經驗.
而他們可以反思他的經驗.
反思他生命的故事.

$^{921}$而這樣的同行的美麗就是.
我和你虛假性是低一點的.
因為你就看到我怎樣做人.
我和你一起的茶經.
我和你一起在團期的中間.
我和你一起經歷教會不同時宮的同工.
甚至你的靈修分享.
我都知道你是要說什麼.
我們一起經歷教會的春夏秋冬.
我經歷你生命不同的階段.
從學生到初入職.
到結婚到生子.
我知道你經歷的掙扎.
而這樣彼此的扶持.
生命的相交的深度.
而將來同工的默契.
在教會是很需要去建立的.
所以鼓勵我們中間的頂姐妹和同學.
願意我今天所提出的方向.
在一個忙碌的城市.
很多人都孤單的.
現在在教會中.
我的體驗就是.
Hi and Bye的人多.
打個招呼.
然後就Bye Bye了.
很多人是坐崇拜的.
不是返崇拜的.
坐一坐就走了.
我們怎樣將那個聖徒相通去加深.
將我們每一天的靈修心得.
借著這種一兩個的同行關係.
無論是小組.
或者是約一兩個朋友.
固定的一個月.
分享你的靈修心得.
是你生活.
是你職場的掙扎.
是你給予手忙.
是你給予祈禱.

$^{961}$這種紮實的功夫.
在忙碌的大都會.
慢慢失傳了.
甚至我鼓勵很多的教會.
在過去的五六年.
我都鼓勵很多教會.
在小組的分享.
每一個月去一次.
大家就拿一段經文.
或者就拿一個.
你生活中在想的.
掙扎的位置.
或者就拿一個經驗.
回教會的小組.
和你的頂智妹真誠的分享.
讓他們對你有些回應.
和你一起尋找.
上帝現在借著這段經文.
或者這個經歷.
在你生命中在做什麼.
然後大家就一起為你祈禱.
我要說回這五六年.
我就這樣鼓勵.
職場的頂智妹.
教會很多的小組.
我現在自己親自都在帶一個職場團體.
我就用這個方法.
大家是很享受的.
因為在這個時代.
可以這麼真誠認真的.
談我們和上帝的經歷的互動.
尋找上帝對我們生命的塑造.
帶領.
這種的聊天已經很少了.
所以同行者.
同行者就是這些.
會和你認真聊.
上帝借著經文.
借著你的處境.
借著你的教會生活.

$^{1001}$現在對你呼喚著什麼.
在你生命中.
收集著你什麼地方.
或者祂塑造著你什麼.
或者祂呼召著.
你要怎樣回應服侍祂.
而這種的交談.
認真的交談的人.
已經不是很多了.
所以今天我們聽我講.
這個短講的頂姐妹.
你願不願意.
有人這樣陪伴你.
而你也願意.
在教會裡面.
慢慢頂姐妹關係.
補育的時代.
可以做這樣的人.
去幫助培育.
紮實的.
去陪伴我們頂姐妹的成長.
好.
我會講一段.
關於同行的關係.
調劇的是.
當我們可以.
以尋找耶穌基督對我們的塑造.
去明白耶穌基督的福音的真理.
為中心的時候.
我們會出現一種.
我和你的屬靈關係越來越好.
但是我發覺.
我是和耶穌基督的關係越來越好.
我們之間越來越自由.
因為.
你喜不喜歡我.
我真的有一種安全感.
就是因為耶穌基督.
你會很用心很真心地對我.
你不會以你自己的眼光.

$^{1041}$跳過耶穌基督看我.
你看我都會有一種.
耶穌基督對我的盼望.
所以這種靈情同行者的關係.
一方面可以容許你的真我展現.
二方面也是透過耶穌基督去看你這個人.
希望和你一起尋找.
更加怎樣回應耶穌對你的心意.
所以對你有一份期待.
所以這一種的真力.
就令到人.
既是接納你.
又是對你充滿盼望.
還有因為我們的相聚.
我們就更加愛耶穌.
不是更加愛對方.
這種跟一般的頂子妹經營.
頂子妹關係是有點不同的.
有些時候我們在教會.
越來越變成為關係而關係.
而中間已經沒有了耶穌基督.
所以希望這個已經有一點.
被遺忘的一個很簡單的同行關係.
可以重新的.
給我們教會的目者.
或者有心去做生命培育的.
一些成熟的頂子妹再一次去重視.
好,我的時間差不多要結束之前.
簡單的就和大家介紹.
既然是我們將同行的關係.
是要在天國裡面參與.
而為了參與天國彼此同行.
而同行既然是以耶穌基督為中心.
這種同行的對談.
對話就很重要了.
以後要分享的十點.
你少少學到.
你的人際關係已經可以改善.
婚姻關係也可以改善.
因為這個可以用在基本輔導技巧.

$^{1081}$婚前輔導.
彼此的交談關係.
這個也可以用在.
目者和青少年的交談方式.
不過今天主要要和你分享的就是.
這個就是在同行關係裡.
大家彼此.
兩個人或者三個人.
學習這樣去同行.
同行的對談.
或者同行的交通交流.
所以這個交流就界定了.
你是否建立了一個寧情同行的關係.
第一個就是.
既然是對話交談.
既然是希望對方借著跟你說.
他的經驗,他的處境.
他的靈修心得.
而去分辨上帝在他生命中在做什麼.
所以我們一個去招待對方的心靈很重要.
怎樣招待呢?.
就是給他一個耳朵.
聽一聽他在說什麼.
他借著慢慢說.
他就可以整理到他的言語.
和他的經驗.
他就慢慢拼出一個圖畫.
他就可以看到神在動他在哪裡.
慢慢聽,不要太快去評論.
最後你的感動都會中肯地去回應對方.
第二個就是交談.
交談是給予回應.
交談就不是對方說一個經驗.
你就說一個你的經驗.
這個就不是交談的焦點.
這個交談最主要的就是.
希望借著我們幫他澄清一些問題.
或者是有興趣的角度.
幫對方釐清他想什麼.
幫對方好像是一個放大鏡頭.

$^{1121}$將一個很快長大的鏡頭放大他看清楚.
這裡是不是其實你是有些迴避神的.
這裡是不是他有些美化自己.
令到他借著交談可以揭示自己.
不要那麼奉慰自己.
去坦然地面對神的.
去看看神在做什麼事情.
第三個就是聚焦.
通常我們認真去交談.
40到45分鐘都只是談一個焦點.
所以不要天花亂飛.
繞到不知去哪.
所以少少的聚焦.
如果可以的話.
借著我們問對方一些問題.
引導他問.
發生這樣的事情.
你覺得耶穌會想跟你說什麼.
發生這樣的事情.
你覺得聖靈提醒你什麼.
借著這段經文.
為什麼對這一節那麼有感動呢.
是不是上帝是想提醒你.
對自己的生命有什麼反省.
用一些好的問題去幫對方沉澱.
因為他講得太快了.
你問他問題.
他突然之間答一答.
然後就冷眼.
他就會冷眼問題.
對質.
你前後不一致.
你剛才不是這樣說的.
我感覺你好像很掙扎.
這個人本來不感覺他有掙扎.
但你一說.
就說.
是呀,其實我在上帝面前想.
我究竟要不要轉工.
要不要移民.

$^{1161}$特別是我轉工.
想一想不如移民.
原來我是想移民.
譬如這些掙扎的位置.
你就問他.
你究竟想轉工還是想移民.
他後面就知道.
他最深層次掙扎的位置在哪裡.
不用怕沉默.
讓對方有一點靜默.
給他空間跟自己聊天.
讓他自己沉澱.
跟聖靈聊天.
所以沉默是一些靈修對談.
靈修分享裡.
是好的事來的.
太多人的話.
聖靈就沒有空間說話.
多點鼓勵.
像做夢的上帝對你有夢想.
做心事的上帝對你有新的美意.
願意我們一起尋求.
所以多點鼓勵.
鼓勵他踏出一小步.
不要往後走.
不要將自己的心靈封閉.
為對方祝福.
雖然他今天說的話你不太喜歡.
你可能不同意.
不過一個同行者.
不是扭轉對方的心意.
是跟對方一起尋求上帝的心意.
所以如果可以的話.
雖然我不同意你.
你可以這樣跟對方說.
不過我真的願神祝福你.
我也願意藉著這件事跟你一起尋求.
神都祝福我.
有新的體驗.
所以彼此的祝福.

$^{1201}$不要聊一聊聊不完.
彼此在那裡吵架.
吵架.
祝福對方.
開放給上帝.
最後大家可以一起祈禱.
我更加鼓勵的就是.
這種靈修關係.
無論是小組.
無論是一對二.
一對一.
盡量在結束的時候.
將我們掛心的那件事.
或者要尋找上帝的那件事.
就用主禱文去做一個結束.
最後就是戒善.
知道我們自己不是上帝.
我們沒有掌控對方.
特別我們作為一些比較成熟的.
基督徒或頂智妹.
知道唯有上帝.
才是他生命的主.
我們將他交託給主.
雖然我們有些擔心.
我們有些掛心.
將他交給主.
而這樣的時候.
有些時候這種同行的關係.
未必是即時找到一些答案.
不過我們慢慢會喜悅.
生命的旅程是屬於你自己的.
那條路唯有你自己走.
旁邊的同行只可以支持你.
鼓勵你.
陪伴你.
最後那條路走出來的.
就是你的照明.
那個是你的責任.
所以這種同行.
不是因為我們走出一條成功的路.

$^{1241}$而是我們走出一條回應呼救.
活出照明的路.
我們因此而喜悅.
所以簡單結束一句話.
同行的關係之下的靈修空間.
令我們的相遇的空間.
我們可以遇見自己.
遇見上帝.
以及遇見將要走的那條路.
那條路上的自己.
這種相遇.
就是同行關係要開出來的.
在多變的時代.
結束的時候.
鼓勵我們的弟兄姊妹.
這一年令我覺得.
有力量的時刻.
不是聽了一個講座.
我得到一個永恆的答案.
一個穩定的方式.
或者是一個.
包你成功的方案.
不是的.
過去這一年.
我跟很多弟兄姊妹分享.
我發覺我們都有共通的經驗.
反而就是大家一起真誠分享.
開放給上帝.
攜手同行.
仰望上帝的時候.
我們反而有那種力量.
所以面對未來.
我們經濟是不穩定的.
疫情現在暫時穩定.
但下半年都很難說.
現在世界的局面.
包括政治經濟都一樣.
都不穩定的.
什麼是一個可行的.
零收的出路.

$^{1281}$簡單的一個中的講座.
我就提出.
或者我們應該分分.
零收傳統基本的一個.
很簡單的一個點.
就是重新建立.
教會群體中的.
同行關係.
時間差不多到了.
最後.
我只是介紹兩本書.
過去這六年.
我寫了幾本書.
不過你想.
相遇都會寧晴路.
你已經看到我的關注.
相遇.
都是寧晴路.
地本.
寧晴同行者.
又是寧晴.
同行者.
如果說得不清楚的地方.
讓姊妹們.
願意多點明白.
你可以去看看.
哪些人買了這兩本書.
都可以一起去閱讀.
接下來的暑假.
多謝延伸部.
給我一個機會.
直接就是會開.
寧晴同行者的這一課.
如果頂姊妹願意.
希望我們一起度過一個.
一起尋找.
一起同行的暑假.
歡迎你來到中晨.
我們一起成長.
願神祝福你.

$^{1321}$祝福你的教會.
也都祝福.
你們的群體.
一起要走的前面的道路.
謝謝.
\newpage



\section{}
\label{sec:4FekoGe9H60}
\textbf{一根杏樹枝:職場的看見 (李適清博士)}
\newline
\newline
連結: \href{https://youtube.com/watch?v=4FekoGe9H60}{\texttt{ https://youtube.com/watch?v=4FekoGe9H60}} ~~~~ 語音日期: 2015-05-19 
\newline
\newline
\hyperref[sec:QdUuSNEVBIc]{\small{< < < PREV SERMON < < <}}
~
\hyperref[sec:index]{\small{[返主目錄]}}
~
\hyperref[sec:OH2MWqshqHE]{\small{> > > NEXT SERMON > > >}}
\newline
\newline
$^{1}$(廣東話).
今天的主題「人生上半場」.
令我很快想起我初初信主的時候.
那時候我剛出來工作.
有一句經文很吸引我.
經文記載在約翰福音十章十節.
耶穌說「我來了是要叫揚得生命.
並且得的更豐盛」.
耶穌應許我們的生命要更豐盛.
我當時就相信這句話.
不是在報道會上叫人信主才會這樣說.
也不是要等死後見主面才有那種豐盛.
而是此時此刻就可以經歷到.
所以我就反覆思想這句話.
這句經文究竟是什麼意思呢?.
而這句經文也成為我生命裡的一個標記.
好像神給我的一句話,一個記號一樣.
讓我無論經歷什麼.
我都可以記得原來主耶穌給了我一個應許.
就是可以活得更豐盛.
漸漸我發現.
上帝會在我們不同人生階段.
不同經歷裡.
給我們一些記號或者把握.
有時可能是某句經文.
有時可能是某些經歷.
有時可能是某次祈禱.
或者某個人的說話很觸動你.
一些神在我們身上好像度身訂造的記號.
讓我們面對我們前面將要面對的困難.
或者我們剛才說的那些度身訂造的壓.
原來不單止有那些壓.
神也給我們一些把握.
一些應許,一些標記.
去幫助我們面對這些挑戰.
聖經裡的耶利米.
也有一個人生的標記.
當上帝呼召年輕的耶利米做先知的時候.
他給耶利米見到一個異象.
我們首先來看看當時的背景.

$^{41}$在耶利米書第一章.
就是這樣說.
那裡耶利米說.
「耶和華的話臨到我說」.
「我未將你做在福中」.
「我已曉得你」.
「你未出母胎」.
「我已分別你為聖」.
「我已派你作列國的先知」.
經文記載年輕的耶利米.
他當時被神呼召.
但是耶利米其實他一個很快.
他有一個反應.
他覺得自己不行.
耶利米書接著就說.
耶利米那裡說.
主耶和華.
我不知道怎麼說.
因為我是年幼的.
經文說.
「耶和華對我說」.
「你不要說我是年幼的」.
「因為我猜顯你到誰哪裡去」.
「你都要去」.
「我吩咐你說什麼話」.
「你都要說」.
「你不要懼怕他們」.
「因為我與你同在」.
「要拯救你」.
「這是耶和華說的」.
當上帝他呼召耶利米的時候.
耶利米看到的是他自己年輕.
不懂得說話.
不行.
他很害怕.
這個時候上帝不僅讓耶利米有口才.
更讓耶利米看到他人生裡.
第一個的異象.
第十一至十二節.
聖經記載.

$^{81}$那裡說.
「耶和華的話由臨到我說」.
「耶利米你看見什麼」.
「我說」.
「我看見一根杏樹枝」.
「耶和華對我說」.
「你看得不錯」.
「因為我留意保守我的話」.
「使得成就」.
通常我們想想.
看到異象這件事.
我們都會想一些很特別的東西.
尤其是我們期待看到一些超自然的東西.
但是耶利米看到的這個異象不是這樣的.
杏樹在當時是很普通的一種樹.
所以杏樹枝在耶利米的生活環境裡.
是經常會遇到的一件事.
不過經文這裡.
上帝加上他的解釋.
他解釋得很清楚.
其實這裡是在玩字.
是一個wordplay.
在原文裡杏樹枝這個發音.
和原文的留意保守這個字.
是一個同音字.
所以當耶利米回答說.
他看見一根杏樹枝.
那個發音就等於留意保守.
神就對他說.
「是的沒錯」.
「我會留意保守我的話」.
「使得成就」.
杏樹枝這個異象似乎是一個微不足道的異象.
尤其是我們看下去.
我們看到後來的異象.
都是關乎國家民族變遷的異象.
對比來說.
這個只不過是一個好像很簡單的異象.
但是這個異象背後.
對於耶利米來說.

$^{121}$有一個很重大的意義.
先知耶利米將會為神宣講很多話.
而且大家知道當他宣講的時候.
人們不理會他 不聽他的話.
在耶利米要面對的那麼多困難之前.
在上帝呼召他的時候.
上帝就給他這個異象.
一根杏樹枝.
成為一個記號.
肯定了神的話必定會成就.
耶利米看到的是他自己不足.
不配 不能.
但是神要耶利米看到一些很不同的東西.
神要耶利米看到.
是耶和華神親自地揀選他.
差遣他.
吩咐他去講說話.
跟他同在 拯救他.
耶和華留意保守他的說話.
使得成就.
我又回想起.
神給我這個豐盛生命這句經文.
其實回想起來正正就是對應我當時的處境.
就是今天我看到最新的研究.
叫做Quarter Life Crisis.
青年危機.
據說這個危機不是二分一 三分一.
是四分一.
25至30歲出現的.
或者現在的人長命了.
不過我覺得這樣形容不好玩.
應該叫四分一危機好玩一點.
就是當我們人生走到四分一的時候.
這個時候正正就是我們進入這個叫做.
恐怖的工作世界的時候.
我們可能遇到很多的事.
我們可能遇到金融海嘯.
有得上班沒得下班.
或者好像最近有一套電視劇叫做.
《衝亞壽新兵團》.

$^{161}$不知道有沒有人看.
裡面就講很多複雜的人事關係.
這些這樣的環境.
其實是不斷挑起我們一些負面的情緒.
很容易讓我們在每一天這些環境當中.
我們會覺得失落.
會覺得不安.
甚至我們會覺得抑鬱.
如果你也試過這樣.
你不要怪自己.
也不要覺得出奇.
因為你並不孤單.
當我們在職場裡面的時候.
其實我們都面對著同類型的一些衝擊.
近來西方的研究就發現.
很多年輕人都會這樣想.
就是30歲的時候.
想結婚 想有錢買樓.
想有自己的事業.
但是事實上這些渴望和期望.
未必可以實現.
於是乎就有些人.
想一些不同的方法去處理.
例如持工 轉工.
例如放下工作.
拿起一個背包去環遊世界.
甚至有些人說.
把頭髮染成金色.
改變一下形象等等.
其實目的就是做一些自己想做的事.
當然我們不否定.
改變有時會令人有些反省.
甚至有些思考和成長.
但是這個世界給我們的答案.
往往就是從我們自己做出發點.
而且只能夠停留在問題的表面.
例如旅行會結束.
染金色的頭髮很快就打回原形.
我們很需要更深層的改變.
我們需要有真正的生命.

$^{201}$就是那位原本創造我們的上帝.
賜給我們的生命.
當我們這班很平凡的人.
跟隨這位很不平凡的主.
我們的生命就會變得再不平凡.
聖經裡面也有很多平凡人.
他們回應上帝的呼召.
整個生命就被扭轉過來.
變得很豐盛 變得奇妙.
我們一起來看一些例子.
(字幕:Sara).
今天上帝同樣呼召我們每一個人.
成為跟隨耶穌基督的人.
我不知道你呢.
你可能是一個行政人員.
可能是一個廚師.
可能你是做銷售的.
可能你是IT人.
可能你是醫護人員.
可能你是老師 工程師.
關鍵不是我們做些甚麼.
做哪一行.
關鍵是上帝要我們做些甚麼.
祂要怎樣使用我們.
而且祂的心意.
往往是我們意想不到的.
在《約翰福音》20章21節.
當主耶穌復活之後.
祂離開門徒之前.
聖經記載耶穌跟門徒說.
祂說.
「父怎樣差遣了我.
我也照樣差遣你們」.
耶穌基督在世上的工作.
包括醫治 饒恕 釋放 救贖.
使人的生命完全.
也就是說.
要使人的生命回復.
回到創造主那種美善的原意裡.
我們當然不是耶穌.

$^{241}$但是耶穌說.
祂同樣差遣我們去服侍人.
就在你身處的工作崗位裡.
透過我們的生命和言行.
將人帶回神的面前.
職場不是一個藍天白雲的地方.
更加不是一個浪漫的地方.
當基督徒進入這個世界.
我們每一天都不停被世俗的價值觀.
和一些其他工作的方法所衝擊.
所以只要你站在真理那一方面.
你一定會和其他人不同.
《馬太福音》十章十六節裡說.
我差你們去.
如同羊進入狼群.
所以你們要靈巧像蛇.
純良像甲子.
世界所看重的.
就是金錢 工作的績限和權力.
用這些東西來衡量我們的價值.
所以這個世界為成功下了一些定義.
結果造成我們可能會強行要上位.
做不到也死定.
變成裡外不一致.
其實是在傷害我們自己.
上帝國度裡的價值觀就很不同.
雖然我們都和其他人一樣.
會面對很多困難 會一樣辛苦.
但是背後的目標和原因都很不同.
我們一起來看看當中的對比.
《馬太福音》.
(魅力前五).
(又一划).
(戰鬥前六).
(補興前七).
(戰鬥後八).
(戰鬥前九).
(戰鬥後十).
(戰鬥後十一).
(戰鬥後十二).

$^{281}$(戰鬥後十三).
(戰鬥後十四).
(戰鬥後十五).
(戰鬥後十六).
(戰鬥後十七).
(戰鬥後十八).
(戰鬥後十九).
我們很需要為成就重新定位.
我記得我初初出來工作的時候.
在職場裡面我看見一些東西.
我看見扭曲了的價值觀.
我看見很難受的上司.
我看見很惡劣的環境.
而且數不盡的人事關係的問題.
但是我發現.
神要我看見一些很不同的東西.
神要我看見上司他發脾氣.
他亂說人.
其實代表著他背後一些很不安的情緒.
我又看見神在困難在困境當中.
很奇妙的保守和開路.
我更加發現在職場裡面很多人都不為意.
同事之間其實可以建立關係.
可以有更加深交的關係.
在當中基督徒生命見證是很重要.
原來我看見的東西就是關乎我的事.
就是我眼前的事.
但是神要我看見的東西.
全部都是神主導的事.
全部都是關乎神的事.
讓我從另外一個角度去看見一件事情的真相.
在工作崗位裡面跟隨神.
做一個職場的門徒.
這是對我們每一個基督徒最終極的挑戰.
對著一份我們不太喜歡的工作.
或者上帝好像跟工作沒有什麼關係的時候.
我們有時候又會覺得這樣不是很理想.
甚至我們會有些罪疚感.
有人用了一個比喻來形容這個情況.
就好像一間屋有兩間房.

$^{321}$一間是信仰一間是工作.
兩間房中間有道門可以打通.
我們就將工作的房間細軸化了.
就在那裡按照這個世界的方法來賺錢.
去追求我們想要的生活.
然後按著其他人說好的事我們就去做.
但是有時候我們因為在這間房太久了.
我們會覺得內疚.
而另外一間房就是信仰的房間.
裡面是很舒服的.
可以給我們安歇休息的時間.
不過有時候我們又覺得.
這間房的要求好像太高了一點.
甚至覺得它有點不切實際.
我們怎樣將中間這面牆拆掉.
以致我們成為一個整全的人呢.
我想我們很需要問一些最基本的問題.
究竟上帝的心意是怎樣的呢.
他做我們的時候他創造的時候.
原本他設計我們是怎樣的呢.
如果工作只是為了兩餐.
只是為了賺多點錢.
我們的經濟思維是沒有錯的.
工作的價值就是在乎人願意給多少錢.
來購買你的勞動力.
但只要我們想深一層.
我們就會發現.
如果我們只是為了賺錢來工作.
我們是得不到滿足的.
因為其實上帝做我們的時候.
工作或者活動,創作.
彼此合作,服侍的能力.
這些行動已經是我們生命的一部分.
已經是創造的一部分.
是我們要得到發揮.
然後才得到滿足.
世界的狀況好像很不理想.
我想神都知道的.
但職場卻是培育我們屬靈生命的一個很重要的地方.
很重要的土壤.

$^{361}$因為什麼呢.
因為我們的信念.
我們聽到的道.
必須要經過實踐.
在我們聽到之後.
在實際的環境裡面.
當我們真的面對上師很無理的要求.
同事的誤解.
工作本身的壓力等等.
最後挑戰我們.
真的能不能夠堅持去做神所喜悅的事呢.
我們真的能不能憑著信心去做合乎真理的決定呢.
最後我們就經歷到.
我們發現到.
信心不是口說的.
不是聽了就算了.
而是我們真的在我們的生活裡面.
是一個很實際要活出來的.
是一個實踐的流露.
讓我們的屬靈生命.
就經過這個土壤的培育.
不斷地成長.
在這個過程裡面.
其實我們不是一個人.
我們不是孤單的.
因為我們不單止有神同在.
而且我們有弟兄姊妹.
我們彼此一起去鼓勵.
用我們今天35歲以下的術語.
就是互撐.
互撐就不是死撐.
不是亂撐.
而是在合乎真理的事上.
你要和你身邊的人說.
我支持你.
我撐你.
一起去活出基督徒的身份.
然後同樣重要.
我們有工作.
有停頓.

$^{401}$有一種生活的節奏.
我們盡力工作之後.
我們有空間.
回到神面前.
我們一起去經歷.
上帝裡面對我們生命的更新.
我們停頓之後.
我們再次面對挑戰.
怎樣才可以在職場裡面有能力呢.
其實答案一早在耶穌基督.
差遣我們的時候.
祂親自就告訴我們.
剛才的經文.
約翰福音20章.
父怎樣差遣了我.
我也照樣差遣你們.
其實後面還有一節.
說了者說.
耶穌就向他們吹一口氣.
說你們受聖靈.
原來聖靈住在我們裡面.
隨時隨地帶領著我們.
所以神不止差遣.
祂又給足夠的能力.
給一些記號.
更加祂和我們一起.
住在我們裡面.
還有最後很重要的一點.
就是我們要有一個.
國度的視野.
有一個廣闊的國度觀念.
以致我們看到.
上帝的國有多麼的大.
就是我們整個的教會群體.
一起與神同工.
會出教會在世界裡面.
而不同光的身份.
我們一定要堅持.
因為當我們見到這個扭曲.
沒什麼希望的世界.

$^{441}$很多負面的情緒出現的時候.
耶穌基督是我們世界裡.
唯一的盼望.
而祂就是選擇了.
使用我們每一個.
將我們猜派到每一個崗位裡面.
成為這個扭曲世代裡面的.
炎和光.
為這個世界帶來希望.
所以只要你很簡單.
你在神猜險你去的崗位裡面.
你每一天.
你就走在真理當中.
你自自然然在職場.
不同的價值觀當中.
你會遇到一些挑戰.
一些撞擊.
不斷地挑戰你的信仰.
這些時候.
就是你為主去活出.
基督徒身份的時候.
是你堅持守住神的道的時候.
行樹之.
成為耶利米生命裡面的.
一個記號主題.
我不知道你的生命裡面.
又有什麼經歷.
可能在你信主的時候.
又或者神給你一些心裡面的負擔.
甚至乎你說我都沒想到.
都不要緊的.
就在你日常的經歷裡面.
不同的處境裡面.
求神給你一些肯定和記號.
在你的困難當中.
去活出信仰.
所以不要等了.
你今天.
你每一天.
你就可以去實踐.

$^{481}$這個職場門徒的使命.
無論你在什麼位置.
你只需要做一個忠心的職場門徒.
把握今天神給你的時間.
機會.
強偉.
在上帝裡面.
你可以活出真正的成功.
走上這條誠信之路.
謝謝.
(字幕由 Amara.org 社群提供).
\newpage



\section{}
\label{sec:OH2MWqshqHE}
\textbf{不能逃避的戰爭:人性大戰 - 善念與邪情之爭戰}
\newline
\newline
連結: \href{https://youtube.com/watch?v=OH2MWqshqHE}{\texttt{ https://youtube.com/watch?v=OH2MWqshqHE}} ~~~~ 語音日期: 2019-07-29 
\newline
\newline
\hyperref[sec:4FekoGe9H60]{\small{< < < PREV SERMON < < <}}
~
\hyperref[sec:index]{\small{[返主目錄]}}
~
\hyperref[sec:eN8W8le4E48]{\small{> > > NEXT SERMON > > >}}
\newline
\newline
$^{1}$(字幕:梁繼昌).
大家好,我是雷敬業,在忠臣教神學.
首先代表忠臣歡迎大家來到這個講座.
最近香港風風雨雨.
現在外面又在下雨.
不過坐在我們當中的應該都是下雨之前來到這個會堂.
希望今晚我們暫時將香港的風風雨雨.
暫時放低一點.
我們討論一下聖經的問題.
雖然也有些關係.
你看到這個正邪大罩燈.
會和今天的香港處境有些關係.
不過我們就不是直接討論香港處境.
我們就討論一下聖經.
我們今晚很高興有兩位忠臣傑出的老師做講員.
第一位是張略,其實也不需要多介紹.
張略是我們的周永健教席教授.
聖經科的教授.
我們也是副院長.
不過大家有沒有留意.
張略橫跨四個州都讀過書.
他在加拿大讀大學.
在香港讀過神學院.
澳洲讀過神學院.
最後在St Andrews University拿了PhD.
所以我們的張教授很有國際視野.
他對兩約之間特別有研究.
所以今天他也會按照他的專長和我們分享.
另外一位就是我們的研新部主任.
余振寧博士.
他是我們的後起之秀.
也是後起的.
因為我也有機會教過他.
其實我已經變成很有輩分的老師.
所以有青出於藍而勝於藍的後起之秀出來.
是很高興的.
余博士在中神讀過書.
他也是我們中神和愛丁堡的聯合博士課程的第一個畢業生.
所以也是北部可以說是中神的果子.
所以今天也很高興能夠讓我當年的老師.

$^{41}$現在做他的學生.
在他身上學習.
不過也不再說那麼多不需要說的話了.
我現在就請張良博士出來.
和我們分享他的主題.
(掌聲).
今天和大家分享的題目.
是這個學展部給的題目.
人性大戰善念和邪情的爭戰.
可能我們也沒有聽過什麼美善的傾向.
善念.
但猶太教裡面.
有善念和惡念兩者的對比.
說到善念的時候.
說到美善的傾向.
他們是說由陀拉所指導和帶動.
要我們的理性得以控制我們的身體.
什麼時候善念啟動呢?.
就是當一個人進入某一個年紀的時候.
當他成為所謂的陀拉之子.
對於他來說是十二歲的時候.
十三歲的時候成為陀拉之子.
所以在這個時候美善的傾向就會啟動.
因為他那時候就會開始很專心的要研讀陀拉.
他們要經過一個過程.
讓他能夠成為成人禮.
不過在米士拿.
這個米士拿是猶太教第二世紀末的時候.
所有的一份我們叫做口傳陀拉.
在裡面的祝福篇有句說話.
盡性愛上帝.
盡心就是盡你自己的生命.
盡力就是盡你自己所擁有的所有的東西.
當說到盡性愛上帝就很有趣的.
代表要以美善和邪惡的傾向去愛上帝.
除了美善的傾向之外.
還有一個叫做邪惡的傾向.
就是惡念.
這個日沙厚拉是什麼呢?.
根據猶太教的解釋.

$^{81}$源於當我們看創世紀第六章第五節.
他說:而又說見人在地上罪惡很大.
終日所思想的日沙的盡都是惡.
或者創世紀第八章二十一節.
人從小時心裡懷著惡念.
日沙厚拉.
氣亞丹.
惡.
所以心裡蘊含著惡.
在人的心裡已經存在.
其實大家可以在網上找到很多這些資訊.
這是其中一個拉比解釋什麼叫做邪惡的傾向.
有七個名字.
第一個是邪惡,第二個是未受割禮.
第三是不潔,第四是憎恨.
第五是叛腳,第六是石頭.
第七是隱藏.
你懷疑什麼呢?.
邪惡的傾向究竟是怎麼來的呢?.
對於拉比,邪惡的傾向是與生俱來的.
很多時候我們都聽到外面有很多人用猶太式經.
不過當我們說猶太式經的時候.
我們必須要非常小心.
因為當有人跟我說這是猶太教的.
我一定要問.
你指的是哪個時期的猶太教?.
不同時期有不同的看法.
我們也一樣要問.
你指的是猶太教裡面哪個拉比這樣看?.
有人跟你說是基督教.
你指的是哪個時期的基督教?.
你指的是哪個傳統的基督教?.
有很多的.
所以當我們來看邪惡和美善的傾向的時候.
我們也要問一個問題.
沒錯,猶太教有這個概念.
但其實不是所有正統猶太教的拉比和學者.
是接受這樣的看法的.
而且這個概念很清楚這兩個的對比.
其實是在新約之後才出現的.

$^{121}$我們就要問.
究竟這些觀念在新約裡面是怎樣的呢?.
或者甚至我們問回頭.
究竟舊約裡面是否真的有這些概念.
由舊約到新約中間這一段.
我們叫第二聖殿時期.
究竟有沒有這樣的看法呢?.
所以我們要清楚的.
就是究竟這些觀念是出於哪一個時期.
另外一個就是我們很多時候.
當一聽到這些概念的時候.
我們就會用我們所熟悉的想法.
我們基督教裡面我們所有原罪的想法.
究竟邪惡的傾向是不是即是原罪呢?.
或者和原罪有什麼關係呢?.
我們必須要回頭去問一個問題.
其實當我們看舊約的時候.
雖然拉比引用《創世紀》的兩段經文.
但事實上我們看舊約的經書.
不是很多是強調人裡面有這種所謂邪惡的傾向的.
反而到了第二聖殿時期.
對邪惡這個問題是有特別的關注.
我們知道第一聖殿被毀.
以色列人被擄到巴比倫.
或者我們香港人都沒有經歷過國破家亡.
而且不單止國破家亡.
而且被遷徙到異域.
在別人的地方寄人籬下.
幾乎做奴隸.
對於以色列人來說.
這是對他們的民族自尊.
對他們整個民族的存亡.
是一個非常重大的打擊.
他們問一個很簡單的問題.
究竟我們以色列人.
究竟我們猶大國犯了什麼罪.
以至上帝要用巴比倫這個殘酷的角度來懲罰我們.
究竟我們出了什麼問題.
所以在第二聖殿時期.
你可以說很多著作都在處理我們叫做神異變的問題.

$^{161}$為什麼邪惡會存在.
因為以色列人要面對這些邪惡所帶來的種種痛苦.
在整個當代的著作裡面.
有兩個很重要的傳統.
一個是天啟的傳統.
另一個是智慧的傳統.
邪惡的傾向是屬於智慧的傳統.
然後姚博士會說的是另一個傳統.
是天啟的傳統.
我們想去看邪惡存在在這個世界上.
我們在這個邪惡底下受到苦難的問題.
智慧的傳統追溯到什麼呢.
當然是追溯到亞當和夏娃.
我們看當時的一本.
是猶太人當中相當流行的一本智慧書.
叫做《變西拉自訓書》.
是公元前二世紀的一本書.
裡面這樣說.
是上主在起初的時候做人.
又賜給他們自由意志.
使得他們可以按自己的意願做選擇.
你可以選擇進行上帝的誡命.
是否誠實地行出來完全是你個人的抉擇.
你的火和水都擺在你面前.
你只要伸手去選取你所要的.
生死已經在你面前陳明.
人選擇什麼就什麼.
上主的智慧無法測度.
他的大能大力能夠洞悉一切.
他的眼目看顧敬畏他的人.
他知道每個人所作的作為.
他沒有吩咐什麼人作惡.
也未曾准許什麼人犯罪.
這與我們現在所用的神異變.
和自由意志的新變.
Free Will Defense 是很相似的.
罪惡為什麼會存在這個世界上.
人為什麼會因為罪惡的緣故而受苦呢.
很簡單的,原因是非常簡單的.
就是因為人有選擇的能力.

$^{201}$人選擇罪惡帶來的是什麼呢.
就是帶來的就是死亡.
是很簡單的.
如果有人說是神叫我犯罪的.
他就是欺詐之子.
他是殺方者的兒子.
希臘版本還有一句這樣的說話.
很清楚地將犯罪這個責任.
放在人的身上.
另外一個是有趣的.
罪忌從女人轉入世界.
死就臨到我們眾人.
有沒有熟悉的感覺.
其實有另一本書叫做.
亞當夏娃生平.
特別是說夏娃怎樣犯罪.
一個很詳細的描述.
在這本書我們已經看到.
有這些觀念.
但這裡沒有講及到.
邪惡的傾向.
我們看另外一個文獻.
叫做4Q393.
3B-5A.
你笑的話.
你就可以想像.
有些未信的人來到教會.
你說馬太福音第六章第五節.
他都不知道你在說什麼.
這個就是了.
這個是死海古卷裡面的第四動.
其中第393號文獻的第三個段片.
裡面的第三行下半節到第五行上半節.
「不要拋棄你的子民和你的產業.
不要任憑個別依從自己邪惡的心還耿而行.
根據你的子而我的臣.
這都要過去.
並且你亦撇棄你的子民和你的產業.
他們個別依從自己邪惡的心還耿而行」.
這裡其實是引用或解釋.

$^{241}$詩篇八四十一篇第十二節.
「我便任憑他們心裡剛硬.
隨從自己的計謀行事」.
當然他就加了邪惡的心.
「Yashar harak」這個詞.
另外4Q又是第四個動.
1號文獻.
第1段第十行至第2段第4行.
這個不是章節,是段和行.
「你責備我們裡面的錫心」.
你熟悉經典就知道這是二世傑書.
「放置潔淨的心取代它.
你責備著我裡面深處邪惡的傾向.
你為我去除好色的眼.
為我驅走頑梗的頸.
使它變得謙卑.
為我除去匆匆的憤怒.
傲慢的心和自大的眼」.
很明顯這裡他要解釋錫心.
然後說到邪惡的傾向.
邪惡的傾向有什麼表現呢?.
就是好色的眼,頑梗的頸,匆匆的憤怒.
傲慢的心和自大的眼.
是說到邪惡的傾向.
在一個人的生命裡會怎樣呈現出來.
另外一本伊斯蘭四書.
伊斯蘭四書很有趣.
因為它有一本天啟文獻.
但裡面引用不少智慧的傳統.
第七節第三章.
這是公元81至96年的著作.
他說:「上帝只給了亞當你的一條命令.
他違背了命令即使你將死亡.
加於他和他的後代.
從他而出的有各國各族不可勝數的人民和族群.
然而你並未有剔除他們邪惡的心」.
又來了.
「以至妥拉可以從他們當中結出果子.
因為第一位亞當受虐於一個邪惡的心.
犯罪和被制服正如我們所有從他而來的.

$^{281}$而這成為固定的疾病.
妥拉和邪惡的根都存在眾人的心中.
但那良善的離去.
邪惡的存在」.
很有趣.
因為他不是說美善的傾向在人裡面.
美善其實是離開了.
「你為他們創造了世界.
當亞當違反我的律例.
這被做的被審判.
進入這個世界的路變得狹窄.
憂傷和路苦.
他們甚少而邪惡充滿危險.
並參與於極大的苦難」.
你看到他要很仔細地.
或者你可以說他很深刻地去描述.
罪惡所帶來的結局是什麼.
我們知道這段書是回應第二聖殿被摧毀的.
在公元70年.
整個耶路撒冷的聖殿.
被提多將軍攻入整個產平.
同樣,這一重深重的苦難.
為什麼會再一次發生在猶太人的身上.
他就是要解釋這一點.
他說原因就是因為我們裡面有邪惡的心.
這個邪惡的心是邪惡的根.
根植在我們生命的深處.
《以色列四書》講到三方面.
第一,上帝最終對人邪惡的傾向負責.
這個很有趣.
因為他沒有除掉人邪惡的心.
第二,邪惡的傾向存在於人的心.
也可以說這就是人的心.
作者甚而說.
「自始有邪惡的種殺在亞當的心中」.
那是誰殺的呢?.
這個邪惡的根居於人的心.
第三,這個邪惡的傾向誘惑人犯罪.
人是有責任控制這個邪惡的傾向.
人要致力克制承認在他們裡面邪惡的傾向.

$^{321}$才可以不被誘惑,偏離生命,進入死亡.
所以人的一個很重要的任務.
就是要克制邪惡的傾向.
我們不其然會問.
邪惡的傾向來自上帝.
為何上帝不消滅我們裡面邪惡的傾向呢?.
難道他沒有能力這樣做嗎?.
在後期拉比的解釋裡是有的.
為何上帝沒有去除人的邪惡傾向呢?.
因為如果人沒有邪惡傾向,會有很大劑的.
為何會有很大劑呢?.
因為人不會生而育類.
人也不會築橋建路.
他沒有生命的動力去建屋,去發展.
所以邪惡傾向是人有的慾望.
而我們沒有慾望,就沒有生存的動力.
所以人的問題不是人有慾望.
而是人控制不了自己的慾望.
這才是最大的問題.
慾望本身不一定是不好的.
這也是為何上帝創造人是有邪惡傾向的.
其實是指人裡面會有的一種動力.
這種動力沒有了的時候.
基本上是不能有創造性的.
這是很有趣的.
如果我們看《新約》.
「慾望」這個字.
你只要看BDAG,現在仍然是最有權威的聖經希臘文字典.
當《新約》裡面用到「Epiphany」這個字的時候.
它有兩種不同的用法.
第一種用法是講述到渴望,熱望,渴求.
是忠誠的,甚至是對好的事情的一種渴望.
人的問題不是沒有渴望.
人的問題是渴望錯了東西.
渴望好的事情是好的.
所以某種意義上人有這種慾望是好的.
所以我們就叫做渴求,渴望.
是好的.
但這個字也有另外一個用法.
就是講述到慾望,情慾是不好的.

$^{361}$是一種負面的用法.
而事實上在《新約》裡面.
耶穌講述到從裡面,從人的心裡面的惡念.
其實就是邪惡傾向這個字.
是苟合偷渡,凶殺,姦淫,貪婪,邪惡,詭詐,淫蕩,嫉妒,蚌毒,驕傲,狂妄.
你數一數十二.
十二個之快.
十二個代表性.
人類都是從人裡面的惡念而出.
保羅教訓我們.
情慾的事包括姦淫,污衛,邪蕩,拜偶像,邪惡,仇恨,憎恨,忌恨,擄擄,結黨,紛爭,異端,嫉妒,醉酒,謊言等.
十五個加個等.
即是還可以繼續數下去.
比拉前書也說我們要避開肉體的私用.
這個私用常常跟生命爭戰.
和合本叫靈魂.
不過我覺得靈魂的譯法不是很準確.
雖然那個字可以譯成靈魂.
但這個會危害我們的生命.
在我們裡面會產生一種爭戰.
我覺得講得最清楚的就是亞國書.
亞書第一章說人被誘惑.
不可以說我們被上帝誘惑.
這裡明顯地是神異變.
你不要一遇到問題就說上帝你用不著這樣作弄我.
不是這樣的.
我們被誘惑的時候.
我們不可以說是上帝誘惑我們.
因為上帝從來都不被邪惡誘惑.
不被誘惑和上帝是沒有關係的.
每個人被誘惑其實是被自己的私慾所牽前所引誘.
然後私慾懷胎就誕下罪孽.
罪孽長成就生出死亡.
私慾然後出來.
經過過程包括選擇的過程出來的是死亡.
在辯釋拉前的分書已經看到.
不單止講述這些邪惡傾向.
只是我們心裡有的東西.
這些全部是內心的東西.
不是的.

$^{401}$你看四章有什麼說.
你們中間的衝突是從哪裡來的呢?.
當中的爭執又是從哪裡來的呢?.
難道不是來自你們體內.
體內這個我覺得不應該這樣翻譯.
是來自你們中間.
即是在他們群體中間那些膠箭著的私慾.
你們貪戀得不得就殺人.
你們嫉妒不能獲得就彼此爭執衝突.
神對抗驕傲的人卻次因比謙卑的人.
講到私慾的時候產生出來的.
貪婪,嫉妒,驕傲.
這些都是人裡面的私慾.
當我們控制不了它的時候.
它會出來的那些表現.
而且這些表現不單只是我自己.
而且會蔓延到整個群體裡面.
對群體產生出撕裂.
這個詞很熟悉了.
所以這種所謂的「剋惡科」拉扯.
其實不只是一個人.
一個天使,一個魔鬼.
不單只是講到個人.
而且也講到群體裡面所出現的紛爭.
而亞文書有一點很重要的.
亦和亞文書的主題有很密切的關係.
「邪惡的傾向」會引發出一些我們需要關注的現象.
這個叫做「兩面派」的現象.
它用到的字是「心懷異議」.
這個字是由希臘文兩個字.
Hadēpsēkos兩個字.
合併在一起的.
意思是兩個靈魂.
是講到一個人的生命裡面是分裂的.
特別是在對比什麼呢?.
我們以世俗為友的就是與神為敵.
亞伯拉罕是與神為友的.
也就是說,如果我們以上帝為我們的朋友.
如果我們愛上帝的時候.
我們就不可以愛這個世界.

$^{441}$我們愛這個世界的時候.
我們就是站在上帝的對立面.
其實大家很熟悉的.
因為登山補訓也是這樣講的.
同樣是說我們不能夠又愛上帝又愛馬門.
而我們往往在上帝與馬門.
世界與上帝之間.
我們在糾纏.
一時說當然是相信上帝.
但不是的.
當我跟隨這個世界的方式去做的時候.
我也得到不少利益.
當然上帝的應許比我們永生好.
不過在這個世界我也要在這裡生活.
我也要做某些事.
讓我可以生活得比較好.
我們往往就是因為這樣的衝突.
而將我們的靈魂一分二.
或者你可以說將它們扯開了.
所以這個人基本上他的忠誠是分裂的.
用到深懷義怨這個字.
其實在亞國之前從來沒有人用過這個字.
是他作出來的.
來表達一個人裡面出現這種所謂的Doubleness.
他不知道是間諜還是反間諜.
總之就是他出現一種這樣的狀態.
另外用到另外一個字.
就是一張六字.
他說我們當中如果有缺少智慧的.
就當求那厚切於眾人而不斥責人的神.
主必定賜給他.
我們要憑信心求.
其實信心是有忠心的意思.
但他說我們只要憑著信心求一點不疑惑.
疑惑這個字.
Diakonoma其實是說一個人裡面.
都是一種所謂分裂的思維.
只要我們不分裂地想.
我們不是像長頭草一樣一時一樣.
我們是專心一意地相信上帝.

$^{481}$忠於上帝.
那上帝就會把智慧賜給我們.
同樣是用這種方式去講及到.
這個所謂兩面派的表現.
第三個就是兩面派的一個很明顯的表現.
就是那個人搖擺不定.
一張六字說.
心懷異議的人在他所行的一切事情上.
都沒有定見.
定見就是搖擺.
不斷地搖擺.
如果我們看三章八字.
講到一個人.
他有的如果是屬地的智慧.
他說造成的是什麼呢.
就是在何處有嫉妒紛爭.
何處有擾亂和各樣壞事.
擾亂就是這種搖擺不定.
就是一個人所謂腳踏兩板橋.
一腳踏兩船.
好像長頭草一樣飄來飄去.
擺來擺去.
這是一種分裂而不專心.
第二種表現.
兩面派的第二種表現.
就是一種不一致,不真誠和欺詐.
第二章講到偏心待人.
很多時候我們會想.
那裡是講一個人進入會堂的時候.
他身光頸靈.
你就請他坐在高位.
他衣衫褸褸.
你就叫他坐在那裡.
很多時候我們會想.
那當然是在講崇拜.
但其實不是.
那裡其實是在講法庭.
當在法庭裡面.
你因著那個人的社會地位.
因為那個人有錢.

$^{521}$你恩待他.
而那個人沒錢.
你就踩他.
他說這樣.
你怎麼可以說.
在審訊的過程裡面.
你可以很公正地做事呢?.
所以這裡所講到偏心待人.
其實是講到司法公正.
也很有趣.
因為第二章是亞國第一件所討論的事.
他用這件事.
為什麼用這件事呢?.
很簡單.
因為最終那個人要面對上帝的審判.
審判是亞西國很重要的題目.
那些在地上屈枉人.
妨礙司法公正.
他們將要面對上帝的審判.
另外信心和行為的不一致.
只是有信心而沒有行為.
這種不一致的現象.
同樣是一種兩面派的表現.
第三章講到.
人不可以一方面稱讚上帝.
另一方面就咒詛.
按上帝形象被造的人.
這樣的話.
同一把口.
一方面長勝斯.
Sing Hallelujah to the Lord.
另一方面就說髒話.
這樣罵人.
這樣是不可以的.
就好像我們很愛那個人.
我太太很喜歡.
對著他有很多稱讚的話.
然後拿著他的相機.
很想罵他.
這樣可以嗎?.

$^{561}$按著神的形象.
相就是形象.
這樣是不可以的.
這樣是一種不一致的現象.
人有內在的情慾.
引發兩面派的表現.
但如果我們想將兩面派的表現.
和情慾兩者連在一起.
我們要問一個問題.
這件事是怎樣可以發生的呢?.
人有慾望.
有兩面派的現象.
人可以對著你一個面孔.
對著另一個人又有另一個面孔.
怎樣可以做到呢?.
《雅克書》講到一個很重要的概念.
就是自我欺騙.
在一章十三節說.
人被試探的時候.
別人給予誘惑的時候.
不可以說是給予上帝誘惑.
有些事情我們不應該這樣想.
如果我們這樣想的時候.
是想錯了的.
或者在一章十六節說.
不要看錯各種美善的恩賜.
各種傳遞的賞賜是從上帝而來的.
不要將上帝看成是專制魔王.
不是的,上帝是最好的東西給過我們.
你不要看錯.
其實是說你不要被人騙了.
或者你可說不要被你自己騙.
一章二十四節說到.
人如果聽道而不行道的時候.
就是自己騙自己.
然後一章二十九節更有趣.
他說人如果聽道而不行道的時候.
就好像人對著鏡子.
看了自己本來的面目.
看了之後.

$^{601}$走了之後就忘記了自己的相貌如何.
我很相信你今天早上.
也不差今天早上照過鏡子.
剛才可能來之前也要照鏡子.
你不照鏡子.
我剛剛記起.
我太太我剛才看一看她.
她的頭髮亂了.
你也找人看一看你.
你現在是很整齊的.
我們其實照鏡子是做什麼呢.
照鏡子不是就這樣照完就算了.
照完鏡子之後.
但是看了自己的容貌.
你的貓樣如何.
你要整一下.
現在不用一段鏡子.
手機也可以.
但是作用是很簡單的.
你這樣照的時候.
你要看清楚.
但為什麼看完之後.
好像沒有看一樣呢.
是不是騙了自己呢.
為什麼會看不到呢.
然後二章二十節說到虛浮的人.
你願意知道沒有行為的信心是死的.
你想不想知道呢.
有些人不想知道.
你有你說.
他聽完之後.
聽完就算了.
水過鴨背.
然後四章四字.
豈不知世俗為有的就是與神為敵.
難道你不知道嗎.
五章十九節.
你們中間有失迷真道的意思.
就是你們中間有些人被騙.
離開了真道.

$^{641}$可能是被人騙.
也可能是被自己騙.
你又可以說.
欺騙.
自我欺騙.
是一種方式.
使我們產生出一個.
兩面派的懲罰.
如果你們有讀過心理學101.
就會知道這是弗朗伊德說的.
一個人.
有三個部分.
一個是驅力.
eat.
第二個是ego.
是自我.
eat的意思是本我.
上面那個是自我.
還有一個叫super ego.
叫超我.
用另一個圖去解釋.
驅動力的eat是甚麼呢.
驅力就是一隻馬.
ego就是控制一隻馬.
然後後面有一個指揮人.
當然對弗朗伊德來說.
後面的指揮人是甚麼呢.
很多時候是講文化裡的道德價值.
這個就是後面的超我的指導力.
然後弗朗伊德說.
因為我們要控制那隻馬.
但那隻馬有時候就不太聽你的話.
當超我叫自我.
你求求你控制那隻馬吧.
你的肉王那樣.
你控制他吧.
但你自我又覺得控制不來.
所以當中.
或者你控制得很辛苦.
當中出現一種爭戰.

$^{681}$出現一種衝突.
在這衝突裡面.
我們開始出現某種焦慮.
為了解決我們焦慮的問題.
我們會怎樣做呢.
我們會有所謂的意識防衛.
這裡是網上找到的.
意識防衛有很多種.
有逃避.
其中一種叫自我欺騙.
自我欺騙是合理化.
抵消.
反向的表現.
這些要找讀心理學的人說.
或者找我太太說也不一定.
自我欺騙是可以有不同的方式表現.
如果我們看藝術.
不要說我這樣是因為上帝對我不好.
是上帝作弄我.
當你這樣說的時候.
其實是甚麼呢.
我因為上帝.
你沒理由把我放在這樣的環境裡.
你把我放在這樣的環境裡.
我哪有辦法.
我不是想的.
你逼我.
環境造成的.
我們很會說.
每個人都這樣做.
我不排隊.
我怎樣生活.
我們開始合理化.
嘗試合理化理性化自己的行為.
不要看錯.
基本上有些人會否定.
不是的.
這樣可以的.
沒問題的.
直接不想看.

$^{721}$看都不想看.
自我欺騙自己很清楚.
看見.
忘記自己的相貌.
根本完全把他壓抑.
虛浮的人.
你願意知道嗎.
有時候是否認.
有時候是壓抑.
我們就用這些.
你可以說.
自我防衛的機制.
使得我們.
在有意無意之間.
不知道一些東西.
這個自我欺騙的厲害的地方.
你不是完全不知道.
就在模糊的界線裡.
所以我們很喜歡灰色地帶.
灰色地帶是甚麼意思呢.
由我決定.
所以黑白都是不好的.
灰灰色.
總之就是暗暗糊糊.
然後可以幫到我們.
我們就可以發揮.
這種所謂奇長的做法.
這個也是一個很有趣的漫畫.
反映真正的自己的鏡子.
割價三折都沒有人要.
可能割價一折都沒有人要.
那個鏡子樣樣像照妖鏡.
把你全部照出來.
不好看.
反映理想自己的鏡子.
全部都賣光.
照到最好看的自己.
為甚麼要賣光呢.
因為我們不想看.
很有趣的就是.

$^{761}$大家知道佛洛依德是猶太人.
所以我說.
他的東西是從猶太教借過來的.
他只是看到.
人這種所謂自我欺騙的現象.
這種自我欺騙可以幫到我們.
看不到某一些東西.
我們有盲點.
有些時候我根本就不想看.
以致可以幫到我站在一個立場.
其實我知道背後我知道.
其實我不應該.
不過我起碼給我自己一些理由.
讓我做起來的時候可以安心一點.
另外一個.
耶穌經常說的就是假冒為善.
假冒為善最大的問題是甚麼呢.
就是認不出自己是假冒為善.
你認不出自己是假冒為善.
你就會想我不應該假冒為善.
那我不假冒為善.
但最大的困難就是.
我認不出我自己.
在大家都熟悉的登山補佛裡面.
其實說得很多.
很有趣的就是.
《馬太福音》這本書.
是全卷新約裡面說的邪惡.
用邪惡這個字用得最多.
其中集中在哪裡呢.
登山補佛.
邪惡使事情變得很複雜.
雖然在登山補佛裡面.
邪惡這個字有不同的譯法.
然後當說到假冒為善的時候.
我舉其中一個例子.
七章三至五節.
其實大家都很熟悉的.
耶穌說我們不要論斷人.
然後這裡說.

$^{801}$為何看見你弟兄眼中有刺.
卻不想你眼中的良目.
為何你只看見人家眼中的刺.
看不到自己的呢.
眼中其實有良目.
你眼中有良目.
為何能對弟兄說.
用我去掉你眼中的刺呢.
結論是.
你要先去掉你眼中的良目.
然後才能清楚去掉弟兄眼中的刺.
這裡基本上是說.
我們很多時候.
我們對自己有的問題.
我們完全忽視的.
那是一大棵木.
你竟然看不到.
很奇怪.
沒有可能的.
可能是那棵木遮住你.
甚麼都看不到.
但對別人的東西.
看得很清楚.
你明不明.
你眼中.
或者你看看你.
我看得很清楚.
這肯定是你的問題.
肯定不是我的問題.
在這種自我欺騙裡.
你會發現.
我們會有傾向.
看我們自己想看到的東西.
然後.
還有趣的是.
同一類型的人.
就會一起去看自己喜歡看的東西.
所以你會發現他們討論來討論去.
都是得到的結果.
都是他們早就有的立場.

$^{841}$你不見到網上是這樣嗎.
全部都是這樣.
大家走在一起討論.
這是陰謀論.
背後有甚麼勢力.
這是空城計.
你問我我怎知道.
我常說請拿證據出來.
大家都沒有證據.
一定是這樣.
怎會不是這樣.
我們聽不聽到另一面說.
當然聽不到.
我們選擇地聽不到.
耶穌為何說不要論斷人.
其實很簡單.
當你這樣做的時候.
一定會產生事例.
沒有辦法.
你怎樣溝通.
所以耶穌給出來的.
你要先去掉你眼中的良目.
然後才能看得清楚.
去掉你眼中的.
請你問問你自己.
是不是真的全部都對.
有沒有真的全部都對.
某些人.
這亦是一種自我欺騙.
而發生出來的現象.
所謂兩面帶來甚麼.
帶來分裂.
所以兩面不只是說.
一個人裡面分裂的問題.
他呈現出來的.
就是群體.
教會的社會的分裂.
只看到自己有利.
看不到別人有利.
就是造成撕裂.

$^{881}$越來越嚴重的原因.
亞哥提出甚麼解決的方法.
第一他說.
神的道可以歇斥我們的生命.
一方面他說.
上帝其實.
各樣美善的恩賜.
各樣傳遍的賞賜.
都是從上頭而來的.
從重光之父那裡來的.
他用真道生了我們.
我們要如何脫離欺騙.
脫離兩面派這種感恩的現象.
我們要的是真道.
我們需要上帝的話語.
成為我們的提醒.
好像光一樣照耀在我們的生命當中.
當然這個道跟甚麼有關.
也跟智慧有關.
因為當中有缺少智慧的.
就要求厚切於眾人.
也不斥責人的神.
這種智慧是讓我們有分辨的能力.
知道甚麼是自己裡面的慾望.
知道甚麼是上帝的心意.
知道甚麼其實是.
整個環境所產生出來的一種現象.
分辨是非常重要的.
而上帝的道就給了我們標桿.
讓我們知道如何去做分辨.
而且不只是聽,而是行.
第二就是信仰群體的提醒.
其實不單是這本書的總結.
在這個過程裡面.
他經常說你們中間.
我的弟兄們.
你們中間.
弟兄們.
親愛的弟兄們.
你們中間.

$^{921}$不斷強調群體有種彼此守望的責任.
所以你們要彼此認罪.
互相代求.
使你們可以得到醫治.
有甚麼用呢?.
祈禱有甚麼用呢?.
那種無助感很重.
但祈禱有件事很重要的.
就是把整件事帶回上帝面前.
求上帝讓我們看清楚.
告訴我們.
我們彼此認罪,互相代求的時候.
我們在這個過程裡面就能得到醫治.
群體的守望.
如果你看《雅各書》.
只看到信心和行為.
或者只看到信心必須有行為.
把重點放在行為上.
你會看錯了.
《雅各書》從頭到尾都是講恩典.
所以一章第五節講到.
我們要求厚切於眾人.
亦不斥責人的神.
然後一章第十七節講到.
各有美善的恩賜和各有傳備的賞賜.
都是從上頭來的.
第四章講到.
人裡面有慾望帶來群體的爭鬥的時候.
然後就說上帝因為他能夠賜更大的恩典.
不是更多.
我本人的翻譯不是很對.
是更大的恩典.
這個恩典能夠昇高我們裡面的慾望.
然後在總結的時候.
明顯主是滿有憐憫.
大有慈悲.
所以最後我們能夠克服我們的慾望.
所帶來的種種的負面現象.
當然我們要有信心.
當然我們要回到上帝面前.

$^{961}$當然這個群體是可以幫助我們.
但最重要的仍然是.
你上來的時候見到.
是上帝的恩典.
上帝的恩典才能夠承托著你和我.
承托著教會.
所以我們要回到上帝的面前.
求上帝幫助我們.
我今天分享到這裡.
謝謝大家.
字幕志願者:陳志全.
感謝大家.
(字幕由 Amara.org 社群提供).
\newpage



\section{}
\label{sec:eN8W8le4E48}
\textbf{不能逃避的戰爭:問答環節}
\newline
\newline
連結: \href{https://youtube.com/watch?v=eN8W8le4E48}{\texttt{ https://youtube.com/watch?v=eN8W8le4E48}} ~~~~ 語音日期: 2019-07-29 
\newline
\newline
\hyperref[sec:OH2MWqshqHE]{\small{< < < PREV SERMON < < <}}
~
\hyperref[sec:index]{\small{[返主目錄]}}
~
\hyperref[sec:uudfjjhd4hA]{\small{> > > NEXT SERMON > > >}}
\newline
\newline
$^{1}$(廣東話).
我先問第一個問題.
OK.
(笑聲).
首先補充一句.
其實張略剛才跟我們分享的訊息.
已經綜合了他一生的功力.
因為他的博士論文寫亞國書.
然後他又研究兩約之間.
他又教過教育.
教務輔導.
他剛才的論文已經將三件事混合.
一次過教給你.
所以很厲害.
OK.
不過你剛才說到自我欺騙.
你講到自我欺騙.
最難的地方是你不知道自己在自我欺騙.
你覺得其實在亞國書或者新約.
經卷也好.
有沒有什麼教導.
幫助我們去明白.
怎樣去發現我們有沒有自我欺騙我們自己.
我以為我已經講了.
(笑聲).
我蠢嘛.
講多次.
不是蠢.
我不是講得很清楚.
我想三方面.
從亞國書的角度.
我覺得新約也是很強調.
我們發現我們自己.
當然我們要自省.
但這個自省.
因為我們有自我欺騙的能力.
所以自省是有很多限制的.
譬如我們學退休.
學自省.
其實不只是坐著想想自己怎樣.

$^{41}$而是很多時候我們透過上帝的話語.
也透過聖靈的工作.
用上帝的話語來提醒我們.
所以亞國書很強調.
從天上而來的智慧.
我們求上帝給我們智慧.
我們有上帝所賜的真道.
能夠幫助我們在反省的過程裡.
透過聖靈的工作.
讓我們看清楚我們自己是怎樣.
其中一樣.
我們在上帝裡面常常講的.
一個人沒有自我醒覺的能力.
或者我們不能夠透過聖經和聖靈.
去反省我們自己的生命.
這個人是沒得救的.
是沒得救的.
因為我們有太多可以騙自己的機會.
所以我們必須要向上帝開放.
第二當然就是透過群體.
可以讓我們提醒.
有時候我們看不到.
但群體正如亞國書所說.
我們彼此認罪互相代求.
是能夠幫助我們.
所以這兩方面都是重要的.
第三個是什麼呢?.
就是上帝的恩典.
下一個問題是有人問的.
這次問Kevin 余振寧.
余振寧的博士論文都是研究啟示錄的.
你是亞國書的?.
搞錯了,對不起,我不認識我的同事.
這個都是我的錯.
不要緊,問啟示錄.
有一位朋友問的問題.
我想是不少教會教主一學的.
負責主一學的老師.
可能會這樣問的問題.
他就說現在我知道.

$^{81}$應該這樣說.
問題就是如果你在主一學裡面教啟示錄.
你又說了《終末的盼望》.
你怎樣令組員不會對意象的象解讀過多.
怎樣去解讀意象,適合你去解讀意象.
你問我怎樣教啟示錄嗎?.
我真的在教會教過啟示錄.
我想你問怎樣令到弟兄姊妹不會過度解讀意象.
老實說這個問題很難回答.
因為要看你班弟兄姊妹是怎樣的弟兄姊妹才行.
我記得我在教會教啟示錄的時候.
有一個弟兄第一堂就這樣問我.
他說,余先生.
到底我們現在去到啟示錄第幾章呢?.
第一堂而已,第一堂他已經這樣問我.
你就看到整個問題背後是.
我們首先怎樣去理解啟示錄到底是一本什麼書.
你聽到剛才弟兄問我的問題.
其實你就明白他看啟示錄基本上是一本語言書.
有時你看電視節目都會說.
發現一個中世紀的語言書.
不知道幾年就發生什麼事.
然後就說他很準確.
有時都會有這樣的八卦節目.
事實上在教會傳統裡面的確是有一種頗強的看法.
是這樣看啟示錄的.
認為啟示錄就是一個這樣的歷史的語言.
很仔細地告訴你將會有些事情怎樣發生.
所以剛才的弟兄就這樣問我.
所以我猜你問的是.
怎樣不會令他們過分解讀圖像那些語言的時候.
我覺得最重要是首先一開始.
跟大家討論清楚一個問題.
到底啟示錄是什麼.
我剛才說啟示錄我沒有特別說.
但我講1QM的時候.
1QM雖然裡面有很多很仔細的戰爭描述.
但其實它不是教你打仗的.
它是要藉著一個想像.
想像這場中美戰爭.

$^{121}$當中結合了很多舊弱的傳統.
剛才我沒有時間說.
但其實是要問一個問題.
到底我們是一個怎樣的群體.
我們怎樣去參與在神的計劃當中.
其實同樣啟示錄也是一本這樣的書.
它不是要預告歷史將要怎樣去發生.
而是它要告訴你.
神對這個世界對歷史已經有一個完美的計劃.
這個計劃我們不需要知道細節是怎樣.
我們只需要知道.
神預了我們參與在其中.
我們要問的是.
當中經文教我們怎樣參與在神的計劃當中.
我也解釋一下.
原來有什麼問題.
舉手那個.
你是不是有問題.
你走前一點去咪高峰說.
我也解釋一下.
這些WhatsApp問題我會繼續說下去.
等榮恩姐問完之後.
我基本上是跟你的次序.
你先打進來我會先問.
不過如果我看不明白你的問題.
我就會skip你的問題.
有些人的問題都幾複雜.
所以你明白大概是這麼欣賞.
榮恩姐.
我問一問.
剛才說聖經說邪情和善念.
我問聖經有沒有被邪惡的觀念.
這是第一個問題.
還有剛才說了猶太學者的理解.
早期教會的猶太學者的理解.
我很想知道.
當然這個不應該在這個時候問.
不過我很有興趣知道.
等日再找一下.
二次大戰之後的猶太學者.

$^{161}$怎樣去理解這個邪惡.
我再多說一點.
當我們自己去理解邪惡.
去悔罪去改過之後.
我們對於社會的邪惡有沒有責任呢.
怎樣去做呢.
這是我現在很切身的問題.
我回答吧.
(笑聲).
是的.
白屠殺之後會怎樣.
這個我不認識.
我對各方面的了解都很少.
不過猶太人對邪惡的體會.
從民族的角度來說是非常深的.
也因為這個緣故.
我剛才提到的智慧的傳統和天啟的傳統.
其實都是要去處理.
有關於以色列猶太人所面對的苦難的問題.
究竟聖經裡面有沒有避邪惡這個觀念呢.
我猜聖經裡面.
起碼我所理解的.
當剛才主題是說.
人裡面有邪惡的傾向.
但其實他本身不一定是邪惡的.
其實問題只是在於.
當人不能控制自己的慾望的時候.
就會有各種不同的邪惡湧現出來.
這種湧現出來的邪惡.
其實可以歸結成為一個系統.
所以人的盲點.
不單止一個人的盲點.
可以是一個國家的盲點.
沒有這個盲點,哪裡有南京大屠殺呢.
你明白嗎?.
沒有這個盲點,哪裡有亞美尼亞的屠殺呢.
這是土耳其人說的.
沒有這個盲點,根本沒有這些事情發生過.
所以這種迷惑也好,自我欺騙也好.
不單止一個人會發生.

$^{201}$一個群體會發生,一個社會會發生.
一個國家會發生.
你可以說就是因為這種矯情.
人不單止自己犯罪.
而且還喜歡別人去行.
羅馬書第一章.
你會看到的確會集結成一個體系.
而這個體系會使我們蒙蔽.
使我們覺察不到原來我們在做錯的事情.
當然這個體系本身的確可以成為對我們的壓制.
這個絕對可以.
不過聖經裡面用所謂避罪的說法.
我反而覺得沒有什麼很直接的說法.
但是我們在這個制度底下.
會被蒙蔽也好,會被壓制也好.
是很明顯存在的.
(陳偉業) 順帶延伸這個問題.
(陳偉業) 抱歉,接著下一個到你.
(陳偉業) 延續剛才問的問題.
(陳偉業) 你在引述的資料裡.
(陳偉業) 好像有一句說到猶太人說.
(陳偉業) 神沒有准許誰犯罪.
(陳偉業) 但是事實上又犯了罪.
(陳偉業) 沒有准許和事實上犯了罪.
(陳偉業) 神准許犯罪和神引誘人犯罪.
(陳偉業) 有什麼分別呢?.
剛才引用的《變沙拉之訓》.
主要是說人犯罪是出自人的自由意志選擇.
其實我引述的部分可能跳得比較快.
其實我是說像《變沙拉之訓》這麼智慧.
我認為其中《第二聖殿時期》.
一本典型的智慧文學書.
講述到的邪惡傾向仍然還沒出來.
是到《死去的古今》才看到那些觀念更加清晰.
所以《變沙拉之訓》主要是說.
要為上帝辯護.
邪惡基本上不是上帝造成的.
犯罪不是上帝造成的.
犯罪是人選擇.
所以那段基本上是說這個觀念.

$^{241}$兩位博士你好.
想問兩位.
面對現在這個警方.
社會或者在信徒群體裡面.
對於神的道.
我自己有感是.
各人都有各人自己的詮釋.
甚至乎是一個宗派.
對著另一個宗派都有.
可能是很對立的神學詮釋.
我究竟看著所謂的.
每一個人的神學詮釋.
我應該憑著我自己對神的領受.
去做標準.
還是我有點不明白.
我應該如何去開放去聽.
這些多元的神學詮釋.
如果我自己要憑著自己的標準.
我就會很迷失.
那是不是又變成我自己的神學詮釋呢.
想問兩位博士.
會不會有出路.
讓我平衡一下我的心理狀況.
這個問題我交給第三位博士.
雷博士.
到底你再一句綜合你的問題是什麼.
太混亂了.
如果一句綜合就是.
我理解神的道是有多元性.
但是如何在現在很多元度.
我有些迷失的狀態下.
去找回我自己心裡的平衡或標準.
這是一個博士的建議.
接著可以將牧師繼續練習.
第一我會建議你.
作為一個基督徒一定有你的傳統.
我會建議如果你真的很迷失的話.
你的安全之地應該是跟隨自己的傳統去走.
然後你看有沒有其他傳統.
可以營造你自己的傳統.

$^{281}$我覺得現在的傾向太容易罵自己.
自己圈子罵自己圈子.
其實世界有這麼多條路.
我覺得神等你在某一個傳統.
可能都是神給你一種保護.
第二我覺得你在混亂之中.
你需要在神面前安靜.
好像剛才張牧師也說過.
第三點我覺得如果你能夠認出一點.
就是我自己需要走這條路.
但不代表其他人的路是錯的.
你安於走我這條路.
不需要為所有其他人的路做一條審判.
我覺得對我來說已經可以平靜很多.
有沒有補充?.
我想問問這些文字的問題.
然後再去問各位現場觀眾.
現場觀眾.
有一個人問.
在啟示錄裡面.
兩個建制人都很轟烈.
做了很多神蹟歧視.
然後被人殺了.
在今天的處境裡.
第一你應該怎樣去詮釋這些很轟烈的行為.
比如誇張地說.
怎樣衝入立法局.
我們應不應該做一些轟烈的行為.
或者反過去到另一個極端.
這兩個見證人最後死了.
我們在跟隨耶穌.
比如面對一些.
如果你要持守真理.
有些很危險的關卡.
我們是否做死士呢.
死就死吧.
怎樣去了解這兩個人的見證呢.
當然啟示錄裡面.
他用的那種意象.
那些圖像是比較.

$^{321}$你可說是比較極端.
或者比較鮮明一點.
不要說極端.
所以他擺出來不是一種.
完全我們有樣學樣的表達方式.
但他其實是在表達一種價值取態.
可以說是一種神學的取態.
首先回答第二個問題.
是否要做死士呢.
這個我覺得的確是.
視乎你當時的處境是怎樣.
不是說你.
怎樣說呢.
甚麼都不理會.
總之我衝去死.
我相信這個從來不是聖經的教導.
正如近日有些人.
因為國際的緣故他跳下來.
我相信這個不是聖經的教導.
我們不是自己去尋死.
但是從聖經裡面也好.
從教會歷史裡面我們都看到.
當死亡.
當殉道真的要迫到身上的時候.
尤其是在早期教會裡面.
我們看到那些先賢.
他們是義不容辭.
來面對這個死亡.
甚至有人說.
我們想方法救你.
他寫信跟那些人說.
不要了.
你不要救我.
我現在是向著榮耀的結局.
來出發.
所以我覺得.
首先一個很簡單的原則.
從來聖經沒有教我們要尋死.
但是聖經也教導我們.
不需要逃避.

$^{361}$當殉道要臨到的時候.
不需要逃避.
至於怎樣去詮釋.
那些轟烈的行為.
我會這樣看.
我會問.
其實你做這個行為的時候.
你用剛才說的見證了什麼.
你見證了什麼.
我們不是要用這個行為去表達我的理念.
我心目中的公義.
我剛才說我們守住耶穌基督的見證.
耶穌基督是怎樣見證的呢.
當然你可能馬上就拿耶穌潔淨聖殿.
進去反台.
的確聖經有記載耶穌反台.
但是耶穌的反台.
我經常問這個問題.
耶穌在聖殿裡面反台.
他是否藉著這件事去完成他的使命呢.
明顯地不是.
在那一刻.
他是用了一個很激烈的行動去表達.
或者或許在一些很特殊的情況.
或許.
不需要完全否定這個可能性.
但到最終聖經很清楚地告訴我們.
我們的見證不是靠這些激烈的行動.
來完成神交託給我們的使命.
我是什麼日報的記者.
開玩笑的.
我想問一下.
剛才你提到的以色列四書.
說人要自力克制.
承認自己裡面的邪惡傾向.
然後在最後亞各書的解決方法.
那三方面.
就是神的道.
信仰群體的提醒.
最後就是神的恩典.

$^{401}$我想問一個問題.
如果人要自力去克制自己的邪惡.
其實後面的解決方法.
似乎都是從神而來的方法.
就是神的道.
或者是恩典.
或者是頂尖媒體的提醒等等.
究竟其實那個克制是指人的啟動.
還是一個從神而來的.
那個我想是一個開始.
起始點在哪裡.
我的傳統是.
改革中的傳統.
我是長老會的牧師.
我有班會友在這裡.
永遠所有的事都是從上帝的恩典開始.
永遠所有的事都是因為上帝為我們所完成的.
透過耶穌基督所完成的.
祂誘發我們裡面對祂的信.
不是我有能力去信.
而是上帝的恩典呈現在我們面前.
祂透過祂的聖靈的工作.
祂所賜給我們的智慧.
引發我們來到祂的面前.
所以恩典永遠先行.
我們只是對恩典的回應.
就算我們有信心.
就算我們的自省.
永遠都是向恩典的回應.
所以不是我自己可以做到什麼.
如果我們了解整個修道的傳統.
我們知道早期有很多所謂的hermit.
他們是隱士.
他們走到沙漠.
特別是埃及.
很多走到沙漠等地方做修士.
自己找個地方躲起來.
當然最出名的就是第二世紀的安東尼.
後來有很多人去找他.
然後得到他的幫助等等.

$^{441}$這是修道運動的一個很重要的開始.
但如果你看Cassian他寫的.
有關整個修道運動.
他們要注意的事情的時候.
他就提出一樣很重要的事.
他說過往有很多.
這是第四世紀的.
Cassian是第四世紀的一個修道士.
他說過往有很多的隱士.
他們去到曠野.
結果怎樣呢.
結果他們神經錯亂.
精神崩潰.
頭頂都不少.
為什麼呢.
因為他們很多人以為.
他們自己去到曠野.
好像安東尼一樣.
可以在那裡自己搞定自己.
但他說安東尼是其中一個例外.
所以整個修道運動.
慢慢就是修道院組織起來.
雷博士好像錯了.
(雷博士:看我的書).
所以我們很喜歡.
我現在退休.
我在修煉我自己.
不知道煉上第幾層.
沒有這回事的.
特別我們對那些所謂內省.
我們特別喜歡的人.
我們要很小心.
上帝把我們放在教會群體裡.
就是要保護我們.
讓群體幫我們一起來做分辨.
這些很重要.
當然你有宿靈的道士等等.
生命的道士幫你是好的.
這也是教會裡面很重要的傳統.
所以回到上帝恩典的工作.

$^{481}$透過他的道,聖靈,智慧.
也透過群體.
(雷博士:我問到這裡的問題).
(其實這裡還有很多問題).
(不過我選了一些精簡的).
(時間有限).
(我這裡問到一個問題).
(然後又是在場觀眾問到問題).
(我想今晚就要停在這裡).
(這個問題就問你剛才說的).
(那個慾望的問題).
(首先有人問).
(為什麼神不可以做到我們).
(只有好的慾望而沒有壞的慾望).
(我想對女孩不禮貌).
(忽然間我的智慧就停了).
(這樣行不行).
(第二你引述一件事).
(如果沒有了慾望).
(連生兒育女都沒有了).
(所以我也是一個有慾望的人).
(我也是一個有罪的人).
(你講這件事).
(你引述是認同這件事).
(還是你覺得和基督教信仰).
(到底吻合不吻合).
(還是你怎麼看這種解釋).
猶太教裡面看邪惡的傾向.
是上帝所創造.
如果我們不用這個.
所謂邪惡的傾向這個名詞.
我們只看人是有慾望的.
而慾望本身.
我也完全同意.
慾望本身是好的.
人最大的問題不是沒有慾望.
而是人當有慾望的時候.
要麼是放在錯誤的位置.
我們欲求另外一些.
我們不應該欲求的東西.

$^{521}$另一方面就是.
我們控制不了我們裡面的慾望.
以至慾望控制了我們.
佛羅伊德所謂的"it".
那隻馬完全不受我們控制.
所以從這個角度來看.
我是同意上帝做人.
就是做人有慾望.
做人有選擇.
人可以選擇.
這是必須要的.
你可以問一個問題.
究竟神做石頭還是做你.
你媽媽生你還是機械人.
生不到的機械人.
還是做機械人.
做機械人你叫他做什麼.
最好沒有AI.
你叫他做什麼他做什麼.
你說哪一個好.
當然是生你出來.
你媽媽叫你做事我不肯.
當然這樣好.
不是生我叉燒.
一個人本身有自己的喜好慾望.
他可以在當中做選擇.
神看者是好的.
就這麼簡單.
不過人的確.
因為上帝給了人一種選擇的能力.
而人裡面有慾望.
如果人是控制不了他的時候.
的確會製造很多麻煩.
所有麻煩都要由他製造出來.
所以這個救恩.
不單是說洗淨我們的罪.
而且是說怎樣幫助我們.
重新調節我們整體的生活.
包括我們怎樣控制自己的慾望.
所以聖靈與情慾相爭.

$^{561}$其實就是說.
我們生活怎樣可以在聖靈的管制裡面.
讓我們裡面的慾望能夠產生出.
人愛喜樂和平.
忍害恩慈.
良善信仇.
溫柔節制.
沒錯,這是聖靈的果子.
但由我們裡面的渴求而產生出來.
所以慾望本身是完全可以是好的.
(MC)最後一個問題.
(傑)我叫傑瑞.
其實是問啟示錄.
我寫下了問題.
第一條可以不回答.
啟示錄應該抱什麼心態呢.
第一條可以不回答.
因為可能不太方便回答.
我其實問.
如果有些教會.
他們查一至三章是七教會.
第四至二十二章我知道是二元.
他們就不說了.
說其他的.
我覺得是愚民政策.
我覺得不應該這樣.
你們可以不回答.
如果是敏感的話.
第二條也是啟示錄.
四至二十二章的預言.
有目者.
即是數目師.
他會有解釋.
可能七年大災難.
或者是三期的災難.
以及以色列人的十四萬四千.
他就很簡單.
他說那個內容.
是什麼就是什麼.
神從來沒有說.

$^{601}$二十四小時就是一天.
或者一年千禧就是一千年.
我都會認同.
他沒理由刁鑽.
不過他其中一個總結.
我也是這樣看.
因為他50%的內容.
就有些表象.
就是這麼多.
另外那50%可能有其他預言家.
那種的.
那些給我參考.
譬如可能七頭十國秀.
可能今天西歐那些純粹參考.
因為沒有直接寫下去.
其實變成50%是啟示錄提的.
另外那50%.
我只是拿來做參考.
或者是一個想像空間給我們想.
第二就是說.
其實最重要最重要.
就是警醒傳福音.
這才是最重要的啟示錄.
不知道我是這樣去看.
不知道對不對.
就是這樣.
因為沒有100%去解釋.
啟示錄是最精彩的.
先回應最後一句.
我同意啟示錄最核心的訊息.
就是警醒.
正如剛才說的.
在這個世界對我們有很多軟硬兼施的攻擊裡面.
我們怎樣持守住.
我們對耶穌基督的忠誠.
無論你怎樣解釋那些內容意象也好.
都是回到這個核心的訊息.
就是我們的忠誠.
這個我絕對認同的.
稍稍回應你開頭的問題.

$^{641}$他說層一到三章四到二十二章不說.
潛水.
我自己覺得這個反應.
他沒有信心說.
他不夠膽說.
那你叫他來忠誠讀吧.
哈哈哈哈.
那什麼呢?是佔順位嗎?.
佔順吧.
好啦.
接著你說.
到底那些是.
到底他字面說是這樣就是這樣.
十四萬四千就是十四萬四千.
就是十二個支派.
每支派一萬二千就是這樣算.
還是怎樣呢?.
你剛才也說.
但又有些明顯地.
你不可能照字面解釋.
你不會看到第六章.
四隻馬你真的覺得.
某年某月某日天上會走一隻馬出來.
不可能的.
那個明顯地是一種象徵.
所以問題就是.
你怎樣決定哪些是象徵哪些是字面.
這個其實很主觀.
你說這段應該是字面的.
這段搞不定了.
其實很主觀.
所以到最後我還是回到.
一開始回答剛才的問題.
我所說的.
最根本的問題.
到底啟示錄是什麼呢?.
它是怎樣的書呢?.
當然不同的傳統.
回答這個問題.
有不同的答案.

$^{681}$但是.
我會認為啟示錄.
基本上主要是用一種.
圖像化的方式.
去表達一種世界觀.
一種神學觀.
所以我會認為.
裡面基本上是象徵.
多於字面要表達一個.
具體的東西.
我們再次給兩位講員掌聲.
我自己做一個結束的禱告.
我們低頭做一個禱告.
主耶穌我們知道這個世界有邪惡的事情.
我們知道在我們心裡也有邪惡的事情.
求你的恩典來臨到我們心中.
也來臨到這個世界.
正如張牧師所說.
恩典是永遠行先.
我們每天每時都需要你的恩典.
求你的恩典來臨到香港的社會.
讓罪惡不能戰勝光明.
也來臨到我們心中.
讓我們懂得在這個世代.
怎樣作你忠心的見證人.
為你打美好的一場戰爭.
求你這樣祝福保守我們.
禱告奉耶穌基督命.
多謝大家.
字幕:MG.
(字幕由 Amara.org 社群提供).
(音樂停止).
\newpage



\section{}
\label{sec:uudfjjhd4hA}
\textbf{不能逃避的戰爭:終末大戰 - 光明與黑暗之爭戰}
\newline
\newline
連結: \href{https://youtube.com/watch?v=uudfjjhd4hA}{\texttt{ https://youtube.com/watch?v=uudfjjhd4hA}} ~~~~ 語音日期: 2019-07-29 
\newline
\newline
\hyperref[sec:eN8W8le4E48]{\small{< < < PREV SERMON < < <}}
~
\hyperref[sec:index]{\small{[返主目錄]}}
~
\hyperref[sec:81BflBsZN_g]{\small{> > > NEXT SERMON > > >}}
\newline
\newline
$^{1}$(片頭).
張諒博士跟我們分享到.
在我們裡面有善念和惡念之間的衝突.
那個爭戰.
這個其實不只是我們裡面的事.
張諒博士剛才提到.
其實這個當我們展現出來的時候.
造成我們群體當中.
那種紛爭衝突.
在猶太傳統的觀念裡面.
或者甚至在後來教會的傳統裡面.
其實這種的爭戰.
牽連到一個更大的範圍.
就是一種宇宙性的.
光明和黑暗之間的爭戰.
這個就是今天後半部分.
我要跟大家分享的內容.
這個光明黑暗的爭戰.
今晚我會跟大家集中去看.
那個關於中末戰爭的觀念.
這個無論在猶太傳統.
在教會的傳統當中.
都是一個很重要的觀念.
在當中我們有興趣的問題.
我們同樣會問.
在猶太傳統當中.
對這個中末的戰爭.
到底有什麼想法.
有什麼觀念在背後.
當然剛才張諒博士也說了.
這個我們要問.
到底是什麼時期的猶太傳統.
是哪個群體.
哪個支派的猶太傳統.
是一個很大的問題.
同時除了猶太傳統之外.
我們會問的是.
我們知道早期的教會.
其實都是從猶太傳統的土壤裡面.
去發生出來.

$^{41}$我們會問早期的教會.
它怎樣繼承了中末戰爭的觀念.
同時它又怎樣去轉化了.
這個中末戰爭的觀念.
它不是一味地抄襲了猶太教的東西回來.
如果是這樣.
其實就是猶太教沒有分別.
它將猶太傳統一定有一個轉化.
在教會對耶穌基督的信仰下.
它怎樣去改變了.
這個中末戰爭的觀念.
當然兩個都是很大的問題.
一晚的講座.
不可能做一個很全面的探討.
所以今晚我選擇.
從兩份的文獻.
跟大家初步去看看.
這兩個的問題.
一份的文獻是來自兩日中間.
剛才張諒博士提過.
死海古館裡面一份的文獻.
另一份當然是我們熟悉的.
《新約聖經》裡面的啟示錄.
我們就從死海古館開始.
開始大家知道這是什麼.
混凝土來自第一洞.
也是在昆蘭曠野.
十一個山洞裡面.
第一個被人發現的山洞.
第一批被發現的死海古館.
這份文獻因為比較多學者去研究.
所以它不只是一個號碼.
剛才大家看過.
它只有一個號碼.
這次不是只有一個號碼.
學者給了它一個名字.
叫做Mikama.
這個字的意思是戰爭.
所以英文我們通常翻譯為.
War Scroll.

$^{81}$戰爭的書卷.
大家看到這幅照片.
昆蘭曠野的紅色箭咀指著.
第一洞的地方其實不太能看到.
但是你看到其實周圍有很多山洞.
如果你看一下這幅照片.
那裡就是一個這樣的地形.
所以能夠發現到山洞裡面的古卷.
可以說是神的恩典.
右邊的大一點的照片.
就能看到第一洞的入口.
這個連結不知道能不能打開.
在以色列博物館的網頁裡面.
他把其中幾卷.
保存得比較完整的死海古卷.
做了一個數碼化的影像.
你回家有興趣.
可以上以色列博物館的網頁看看.
你看下去.
這個就是War Scroll.
你看到它的保存狀況相對不差.
也挺完整的.
當然.
穿了洞就難免了.
還有最重要的是.
你會看到下面.
爛了.
沒有了.
這個當然是因為經過長年累月.
它受到水氣潮濕.
它的皮卷腐爛.
你可以一直拉過去.
還有你看到它是怎麼抄寫的呢.
希伯來文從右邊看過去.
他是一行一行地寫下來.
抄了一個座位.
然後就第二個座位.
第三個座位.
所以剛才張樂牧師提過.
他不是說章節.

$^{121}$而是說文字出自哪裡.
我們就說這本書是第幾個座位.
第幾行.
看完照片有圖有真相.
《One Q.M》.
我們通常用第一字母.
One Q.M.
它說什麼呢.
既然叫做《War Scroll》.
戰爭的書卷.
裡面當然是說一場戰爭.
這本書一開始的標題.
告訴我們這本書是說什麼.
它說這本書是關於光明之子.
和黑暗之子之間那場終末的戰爭.
到了歷史的終結的時候.
會發生一場戰爭.
連這個日子.
是神一早定了.
在古時已經定了.
藉著這場終末的戰爭.
讓黑暗之子完全被消滅.
換句話說.
你推翻的時候.
就是說在終末戰爭發生之前.
黑暗之子仍然會存在這個地上.
仍然會和光明之子在一個張力衝突當中.
另外有趣的是.
我們看到這份文獻裡面.
提到這場終末戰爭.
他提到這場戰爭有七個階段.
還沒打.
還沒到那一天.
已經說定了.
告訴你七個階段.
你慢慢等吧.
前六個階段.
互相有優勢.
不是一面倒的.
是說光明之子和黑暗之子.

$^{161}$光明之子站在神那邊.
當然是贏了.
不是的.
原來七個階段.
前六個階段是梅花間燭.
有時候光明之子佔優.
有時候是黑暗之子佔優.
光明之子會敗退.
當然到了最後的階段.
第七個階段.
神會親自介入.
讓他的子民.
光明之子.
獲得終極的勝利.
另外在這本書一開始的時候.
他提到.
這個終末戰爭的時期.
是神的子民磨難的時期.
甚至說到這是磨難.
比他們之前經歷的一切苦難.
剛才提過.
第一聖殿被毀.
第二聖殿被毀.
比這一切的苦難更甚.
但這個苦難卻是引向永恆的救贖.
他要說.
神的子民.
光明之子.
你要準備好面對這個磨難.
就好像保羅在《二忽所書》裡面說.
你要穿起我們的全副軍裝.
其實也是一個戰爭的意象.
這個書卷很長.
所以當然不可能跟大家慢慢去看.
這個書卷.
相信大家也比較陌生.
你不會無緣無故拿一個社區古卷來看.
所以我在這裡當中選了兩段文字.
讓大家稍微品嘗一下.
淺嘗一下.

$^{201}$這本書裡面說的東西.
到底說些什麼.
第一段的文字是利字.
懂得看了.
第五個階段.
第十六行.
到第六個階段.
第四行.
但你記得.
下面是破爛了的.
所以中間有一段文字是沒有了的.
這裡說到當.
光明之子上陣.
來到兩軍對壘的時候.
他們怎麼打這場仗.
他說這些軍隊要列成七條的戰線.
一行在另一行後面.
很有陣型的.
中間要有一個空間.
三十爪.
然後底部就開始破爛了.
到了後面就說了.
他們列好陣型之後.
三型的步兵要上前來.
站在戰線中間做什麼呢?.
開始攻擊了.
第一型的步兵要向敵人投擲七支槍.
這些槍不是隨便找來的.
要精心準備.
上面寫了一些文字.
這裡記載了三段這樣的文字.
第一支槍寫著.
"茅的光彩是為了神的權能".
或者"藉著神的權能"去發出.
第二支槍寫著.
"藉著神的憤怒去擊殺敵人".
一個是血戰.
第三支槍.
"藉神的審判吞食惡者的劍刃".
他說這些是步兵.

$^{241}$他們要投擲槍七次.
然後回到自己的位置.
在這些文字我們首先留意到.
第一.
"七"這個字也出現了好幾次.
我們很熟悉.
"七"是一個很有象徵性的數字.
在猶太傳統裡面.
"七"是象徵源泉.
這個我猜大家不陌生.
所以你看到這個戰爭的描述.
其實是在說這場是一場很源泉的戰爭.
藉著這場戰爭是要彰顯神的源泉.
所以七條的戰線.
七支的槍.
七次的投擲.
第二我們看到這些武器.
就是特意在上面要寫上這些文字.
我們看到這個戰爭的目的.
其實不是為了消滅敵人.
剛才說其實消滅敵人是到最後的階段.
前面六個階段打來打去.
互有優勢.
但其實最終都是到了第七個階段.
神親自介入.
那你問前面的六個階段為什麼要打呢?.
直接到第七個階段就好了.
這場戰爭.
其實不是為了.
用這些武器去擊殺敵人.
是為了藉著這個戰爭的行動.
去彰顯神的榮耀.
神的權能.
神的憤怒.
神的審判.
戰爭成為一個見證.
第二段的文字後一點點就到了第七個階段.
第三到第六行.
這裡提到關於軍隊的聖結.
說他們要離開耶路撒冷.

$^{281}$出去征戰直到他們回來這段期間.
孩童和婦女不可以進入營中.
今時今日可能會被人去評姬會告.
強子,弱子,癱子和身體有那些.
除去的瑕疵.
可能是永久的疤痕之類.
和那些不潔的人.
都不可以和他們一起出去征戰.
這裡很明顯是一個潔淨的要求.
如果你對比.
比如舊約,摩西五經對祭物.
獻給神的祭物是要無瑕疵的.
不可以跛,不可以盲.
對祭司的要求.
在會幕裡侍奉神的祭司同樣.
他不能夠有這些殘障.
不能夠有這些不潔.
所以這是一個潔淨的要求.
孩童和婦女一樣.
特別是如果你們記得.
在初一及二當神在西乃山.
要和以色列人納約的時候.
神吩咐摩西.
要以色列人準備自己要自潔.
其中一件事他們要做的就是.
他們不要親近婦女.
以這樣來保持他們群體的潔淨.
所以這裡是描述這些人不能夠.
和他們一起出去征戰.
這個其實是一種潔淨的要求.
律法裡面的潔淨的要求.
他們是自願參戰.
身體和心靈是純潔的.
這裡說得更加直接.
他們準備好出去打仗.
到了征戰的日子.
那些全員不潔的人.
不可以和他們一起出去打仗.
所以不僅是前面說的.
孩童和婦女殘障的.

$^{321}$他們不能夠參戰.
即使出去打仗的士兵.
到他們臨陣要上陣的時候.
都要保持他們的潔淨.
全員不潔的人.
不可以出去打仗.
什麼叫做全員不潔呢.
這個當然是一個隱喻.
全員是在說性器官.
我們知道律法說到.
男性有一些蒙遺等等.
他會沾染了不潔.
他要經過一個潔淨的儀式.
他才能夠恢復潔淨.
所以這裡說的是.
同樣引用了律法裡面.
那種潔淨的要求.
套用在這群的軍隊上.
最後這裡說到.
原因.
因為聖潔的天使.
臨在於他們的軍隊當中.
為什麼這麼重要.
就是要保持潔淨.
因為有天使.
有神的使者.
天使代表神.
臨在於他們當中.
他們的潔淨.
就是保持著.
天使能夠跟他們在一起.
因為在不潔的當中.
神或者聖潔的天使.
不能夠臨在當中.
所以這裡我們也看到.
這場戰爭不單止是一場.
地上的戰爭這麼簡單.
其實是跟天使有關.
天使在中間一起去打仗.
甚至某程度上來說.

$^{361}$天使才是這場戰爭.
主導的力量的來源.
很簡單看了兩段很簡單的文字.
我們稍微.
闊一點看.
整個《One QM》.
這個《Wall Scroll》裡面.
說這場是終末戰爭.
有什麼特色呢?.
其中一些我們從剛才那幾段文字.
有看過.
第一.
它是一個很理想化的戰爭過程.
我一開始說七個階段.
還沒打已經告訴你.
是會有七個階段.
七象徵完全.
那個很形式化的作戰過程.
我們剛才看.
那些武器寫上字的.
列隊.
然後扔了矛七次.
你想想真的打仗的時候.
軍隊的組織可以很有秩序.
行軍的過程可以很有秩序.
但是上到戰場.
真的兩軍對壘.
短兵相接的時候.
基本上是不可能這麼有秩序的.
真的短兵相接的時候.
兵荒馬亂.
你看那些警民衝突你就知道了.
不可能一早計劃好你怎麼做.
需要隨機應變的.
但是《One QM》裡面說的.
它是一個很形式化的作戰過程.
另外我們剛才沒有看到.
裡面說到戰爭在安息年是會停止的.
你說以色列人.
他們嚴守神的律法.

$^{401}$他們去到安息年不打仗.
這個還可以想像.
但是.
舉黑暗之子.
包括那些外邦人.
他們怎麼會這麼聽話.
跟你在安息年停戰.
不可能.
所以說到戰爭在安息年停止.
這個很明顯有一種很理想化的.
對中末戰爭的一種想像.
另外我們看到的就是.
整個的戰爭其實跟敬拜結合在一起.
我們剛才看到那些槍上面寫的文字.
去彰顯神的榮耀.
另外《One QM》裡面提到.
記載很多的禮文.
禮文,禱文.
在戰爭不同的時候.
出發之前.
臨上陣之前.
打仗的時候.
拜退的時候.
或者打勝仗回來慶祝的時候.
記載了很多這些禮文,禱文.
你看到整個的戰爭是跟敬拜結合在一起.
最後.
這裡提到.
這場戰爭根據《One QM》的說法.
是由祭司去領導.
不是軍王.
不是戰士.
不是將軍.
由祭司去帶領.
整場的爭戰.
這讓我們想起.
可能就是《約書亞記》裡面.
以色列人攻打耶利哥的那場戰爭.
以色列人跟隨著祭司.
圍繞耶利哥七天.

$^{441}$坐城就倒塌了.
很可能這個也是.
《One QM》中末戰爭的.
想像裡面一個很重要的根源.
由祭司帶領.
彰顯神榮耀的戰爭.
從這些理想化的描述.
我們就看到.
其實整個《One QM》.
整個《Wall Street Journal》.
雖然裡面有很多很仔細的.
關於戰爭的描述.
提到軍隊的陣型是怎樣.
武器是怎樣.
裝備是怎樣.
過程是怎樣.
很仔細,很具體.
但是.
其實整本書說的東西的意義.
不是教你.
怎樣去打這場仗.
其實這本書.
寫作的手法,體裁.
很像什麼呢?.
很特別.
很像當時羅馬的軍隊手冊.
羅馬軍隊手冊就真的教你.
怎樣去行軍,怎樣去打仗.
真的是一個很實用性.
給軍隊用的手冊.
Wall Street Journal.
他參照了這種寫作的方式.
寫得真的好像一個軍隊手冊.
但是其實.
他不是要教你打仗.
而是藉著想像.
去描述這場的終末戰爭.
來表達這個群體.
他的身份.
和他終末的盼望.

$^{481}$為了知道.
最終一定會贏.
一早就告訴你.
第七個階段神會介入.
就會贏.
那還需要教你怎樣打嗎?.
不用的.
他不是要教你怎樣打.
他告訴你.
既然你知道會贏.
既然你知道在勝利來臨之前.
你要經歷這段磨難的日子.
那你怎樣選擇?.
你怎樣去準備好自己.
進入這個磨難當中.
怎樣準備好自己.
迎向這個終末的勝利.
這是這本書要講的.
雖然《One QM》.
是一個好像戰爭手冊一樣.
但是其實裡面表達的.
那種觀念基本上就是一種.
張牧師剛才提過的.
一種天啟的觀念.
迎向神一早預定的終末結局.
今天神的指紋.
怎樣去預備好自己.
我們稍微再仔細看多一點.
裡面對這種終末戰爭的想像.
或者描述.
有什麼特色可以讓我們留意到呢?.
第一.
當然是一種很強烈的二元論.
二元的對立.
光明之子和黑暗之子.
是斷言而分.
是壁壘分明.
不能站在中間.
你不是站在我這邊.
你就是站在對方那邊.

$^{521}$而且.
在《One QM》裡面提到的黑暗之子.
不單止包括外邦人.
甚至包括以色列人當中.
那些違背神的約的人.
所以甚至不是每一個以色列人.
都是光明之子.
只有那些依從群體的教導.
他們對律法的持守的人.
才是真正的光明之子.
當然這種地上的光明黑暗的二分.
連於天上更加大的正邪勢力的對立.
正的一方當然就是上帝和他的天使.
邪的一方就是魔鬼.
裡面用的名字是比列.
在哥林多後書也有提到比列這個名字.
比列和他的使者.
天上的征戰其實.
才是真正的征戰.
地上的戰爭.
只不過是反映著.
天上真正發生的那場戰爭.
在右邊這幅照片.
這就是以色列博物館.
其中一個展館的地方.
如果你有去過.
可能也會有印象.
這個展館叫做The Shrine of Book.
The Shrine of Book.
就是展覽死海古卷的地方.
它的設計很有趣.
你看到白色的那塊東西.
其實它是參照那些.
裝死海古卷的瓦屏的蓋子.
是一個蓋子的造型.
展館其實是在下面的.
所以白色的那塊東西是展館的天花板.
接著對著它有一幅黑色的牆.
你看到整個建築的設計就是象徵著.
死海古卷.

$^{561}$One QM.
這個光明和黑暗的對立.
除了光明和黑暗的對立.
也有一種時間性的劃分.
現在是磨難的日子.
但是經歷磨難之後.
神的介入.
將歷史帶進去另一個階段.
進入終末的福樂.
在One QM裡面整件事是很斷然易分的.
無論從群體來說.
從時間來說.
都是很斬釘截鐵.
另外我們剛才提過的歷史預定論.
這個歷史進程.
起初已經由神預定了.
一早已經決定.
一早已經定了象象有七個時期.
梅花間矚.
最終神介入歷史.
當然我們相信.
或者猶太人都相信.
神不斷在歷史當中介入.
但是在終末的時候.
神的介入是更加戲劇性.
更加神跡性.
這個最終神跡性的介入.
扭轉歷史的流向.
引來終末的來臨.
很快,很簡短.
稍微,很初步地認識一下.
One QM寫開古卷.
困難裡面的終末戰爭的觀念.
我們來到新約聖經早期教會.
我們剛才問它如何承接了.
這種終末戰爭的觀念.
又如何轉化了這個觀念.
有一位學者.
Borgum.
張立牧師博士論文的老師.

$^{601}$他提出了一個看法.
他說新約裡面啟示錄.
The Book of Revelation.
可以看為一個基督教的.
或者教會的War School.
雖然題材很不同.
剛才說的War School.
基本上是一個戰爭手冊的寫法.
但其實裡面表達的觀念.
正正就是一種啟示.
天啟,終末的觀念.
這個跟啟示錄相當相似.
但同時當我們看啟示錄的時候.
我們看到有很相似的地方.
但同時啟示錄裡面提到.
這個終末戰爭也有一些很獨特的地方.
跟我們在其他的猶太傳統.
跟我們在One QM裡面所看到的.
很不同.
啟示錄裡面的終末戰爭.
又是怎樣的呢?.
裡面也是跟大家看其中幾處經文.
第一處要跟大家看的是十二章.
如果你是熟悉啟示錄這本書.
你就知道十二章是一個很特別的意象.
是一個相當有趣的意象.
一開始說到一個懷孕即將要生產的婦人.
但說到這個婦人是.
「新披日頭,腳踏月亮,頭戴十二星」.
很明顯是一個宇宙性的形象.
它代表宇宙裡面.
正在發生某些事情.
接著經文提到有大紅龍.
當然後面我們知道這個大紅龍.
就是魔鬼想攻擊這個婦人.
想攻擊她要生下來的兒子.
但當然不成功.
接著經文提到有天上的征戰.
米加勒帶著天君天使和大紅龍去征戰.
戰爭的描述很簡短.

$^{641}$幾節的經文已經說完了.
大紅龍徹底的失敗.
然後經文有一個勝利之歌.
是一個重讚.
最後提到大紅龍被摔在地上.
他作他垂死的掙扎.
去攻擊這個婦人其餘的兒女.
特別留意第七到第九節.
這裡提到天上的戰爭.
剛才的很簡短很快就說完了.
代表光明的一方.
米加勒和他的天使.
代表黑暗的一方.
龍和他的使者.
後面提到古蛇叫魔鬼.
又叫撒旦.
是迷惑全地的.
所以不會點錯相.
就是他了.
戰爭的結果很簡單.
龍抵擋不住.
天上再沒有他們的地方.
然後龍被摔在地上.
和他的使者一同被摔下來.
一個決定性的失敗.
再沒有他們的地方.
沒有地方站了.
不用想了.
不能反擊了.
所以短短的幾節.
已經將這個決定性的勝利帶了出來.
所以接著就有勝利之歌.
提到我們神的救恩能力.
國度基督的權柄.
現在已經來到了.
天上那場戰爭.
在啟示錄的描述裡面.
天上這場戰爭.
其實已經打了.
龍已經輸了.

$^{681}$基督的權柄神的國度.
已經來到了.
所以.
其實地上的聖徒.
已經在勝利當中.
他們都能夠勝過這條龍.
他們怎樣贏呢?.
這裡提到藉著羊羔的血.
當然是在說耶穌基督的救贖.
這是一個客觀的基礎.
不關我們做了什麼事.
耶穌基督已經幫我們贏了這場仗.
但同時.
是什麼讓弟兄能夠得勝.
也是他們所見證的道.
這就是提到弟兄.
聖徒的參與.
不是繞著手在等.
耶穌基督接我們上天堂.
我們仍然在地上.
我們藉著我們所見證的道.
參與在這個終末戰爭當中.
迎向最終的勝利.
但這仍然有一個過程.
就是提到龍既然被摔在地上的時候.
他最後提到.
他要攻擊那個婦人其餘的兒女.
所以當天和其中的快樂的時候.
這裡提到地和海有和.
就像剛才在《Q and A》裡面.
我們看到磨爛的日子.
在迎向終末之前.
有磨爛的日子.
然後是啟示錄.
接著就是十三章.
提到那兩隻獸.
帶來地上的聖徒.
那些是迫不還難.
所以雖然天上那場仗.
打贏了.

$^{721}$已經決定勝利贏了.
但是地上的爭戰.
仍然持續著.
因為龍現在在地上.
作他垂死的掙扎.
啟示錄裡面提到地上的爭戰.
和大家留意其中一個細節.
就是啟示錄裡面有兩段經文提到.
有十四萬四千人.
在這裡雖然學者之間有討論.
有人認為是同一班人.
有人認為是兩班人.
我自己認為.
其實兩段經文提到的十四萬四千人.
都是指同一個群體.
第七章裡面提到.
十四萬四千人的組成很簡單.
十二個支派.
每支派一萬二千人.
第七章裡面提到.
這十四萬四千人是一群受保護的人.
那個天使.
在他們的額上印上了一個印記.
免得他們受到傷害.
這個印記代表他們是屬於神.
同時.
經文說是保護.
一個保護的印記.
典故大概就是雨節.
以色列人在門尾上.
塗了羊膏的血.
羊滅命的天使.
跳過他們的家.
所以他們的額上受印.
也是同樣成為一種保護.
如果在第七章裡面提到.
這十四萬四千人是一群受保護的人.
到了第十四章再提這十四萬四千人.
他們的形象就有一點點不同.
那個角度.

$^{761}$是另一個角度.
十四章提到這十四萬四千人.
基本上是一個軍隊的形象.
裡面的描述有什麼跟軍隊有關呢?.
第三節裡面提到唱新歌.
他們是唱新歌.
什麼是新歌呢?.
現在很多人作歌.
你上網聽很多新歌.
不是那些新歌.
新歌其實是一個戰爭之後.
的頌歌.
在《啟示錄》前面的經文提過.
在第五章.
當介紹耶穌基督出場的時候.
那個天上.
天庭的景象.
耶穌基督出來的時候.
二十四長老.
就來唱新歌.
去歌頌耶穌基督的特性.
所以新歌是一個勝利之歌.
接著說到未曾沾染婦女.
我們剛才提過在《One QM》裡面.
我們也看到.
這個其實不是在說一種的.
守童心.
就好像後來中世紀的教會.
他們注重那種修道的主義.
要守童貞.
其實主要不是在說這件事情.
而是在說.
在征戰當中那種的.
軍.
軍隊的潔淨.
(笑).
大家先繼續聽著.
是一個軍營裡面的潔淨的要求.
到了第四節.
第四節提到他們是跟隨羔羊.

$^{801}$當然我們也是跟從耶穌基督的人.
但是這種跟隨羔羊去到哪裡.
他們就去到哪裡.
就好像一個軍隊跟隨將軍.
將軍去哪裡.
軍隊就跟隨他去哪裡.
當然到了最後第五節.
提到他們是沒有瑕疵的.
因為也剛才提過在《One QM》裡面.
我們也看到.
其實是將舊約.
律法的那種對祭物,對祭司的要求.
放在神的軍隊身上.
所以他們的潔淨.
其實是在說.
他們是一群準備好.
上場為神爭戰的軍隊.
當然這種沒有瑕疵.
展現在他們道德的聖潔上面.
所以.
講沒有瑕疵之前說到.
在他們口中.
擦不出謊言.
內在那種道德的潔淨.
和外在禮儀的潔淨.
是相輔相成的.
當然地上的戰爭.
這個過程.
它將會完結.
終局就是耶穌基督.
終末的降臨.
到了十九二十章.
講到騎白馬的.
耶穌基督從天而降.
戰勝了受童地上的君王.
龍被捆綁一千年.
然後有頭一次的復活.
聖徒和基督一同作王一千年.
我們熟悉的千禧年.
雖然.

$^{841}$怎麼解釋有很多爭論.
接著講到撒旦被釋放.
哥格馬國的征戰.
這裡用以西傑書的典故.
然後有白色大寶座的審判.
耶穌基督降臨.
將十二章.
天上.
天君天使.
已經打贏了勝利.
帶到地上.
將地上的征戰.
都引向終局.
剛才我們一直在看.
天上的征戰.
那些對軍隊潔淨的要求.
那個終末的勝利.
和我們前面在《OneQM》裡面看到的觀念.
都很相似.
但是剛才也講過.
在啟示錄裡面.
對這個終末戰爭的觀念.
有一種.
有一些很重要的轉化.
在《OneQM》裡面我們看到.
雖然他對戰爭的描述.
是一種很形式化.
很理想化的描述.
但是那個.
仍然真的是一場戰爭.
仍然是拿刀拿槍.
拿弓箭去打.
至少在他的描述裡面是這樣說的.
但是在啟示錄.
啟示錄裡面提到.
這場仗是怎樣的呢?.
在啟示錄裡面.
其實這場仗.
不是拿刀劍去打的.
首先.

$^{881}$在耶穌基督身上我們看到.
當第五章剛才講到.
天庭的景象.
耶穌基督出場的時候.
五章五節才講到.
耶穌基督是猶大之派的獅子.
獅子當然是一個很威武的形象.
一出來.
萬獸之王大叫一聲.
所有其他的動物都要.
退避三射.
獅子.
帶有那種爭戰的形象.
但是接著.
第六節就講.
他是那個.
曾被殺的羔羊.
獅子變成被殺的羔羊.
你想清楚一點.
其實也有一種違和感在裡面.
前一幕還看到一隻.
嘩一聲的獅子.
下一幕就看到.
那隻羔羊被人害燙.
變成一隻羊被人宰.
在耶穌基督身上.
他又將獅子和羔羊.
兩個形象結合在一起.
是.
他是充滿權能.
他是獅子.
但是同時他的權能.
是彰顯在.
被殺的羔羊身上.
其實同樣.
這種違和感.
在第十一章講的兩個見證人身上.
我們同樣看得到.
十一章兩個見證人.
用了虐魔西銅爾利亞.

$^{921}$很多典故.
講到他們走很多神跡.
叫天不下雨等等.
叫地上的人受很多苦.
這兩個人很厲害.
你想這麼厲害.
走出來就殺光了.
但是接著經文講什麼呢?.
講他們被人殺了.
這麼厲害也被人殺.
是的.
這麼厲害也被人殺.
沒錯.
他們的神跡.
走的神跡.
是他們見證的.
一個很重要的環節.
但是在十一章裡面.
其實這兩個見證人.
他們最重要的見證.
不是那些神跡.
他們最重要的見證.
是他們被殺.
然後.
神叫他們復活.
正正就好像耶穌基督一樣.
充滿權能.
卻被殺.
然後.
神叫他們復活.
這兩個見證人.
就是跟從耶穌基督.
一模一樣.
這樣來到.
在眾人面前發出見證.
所以到了聖徒.
到了教會.
其實同樣.
啟示錄裡面.
教會.

$^{961}$如何參與在這場爭戰當中.
十二章十一節.
我們剛才看過.
藉著楊高的血.
和他們所見證的道.
接著那句.
我們剛才沒有讀.
說即使面對死亡.
就算是死亡.
他們也不害怕.
他們也在所不惜.
勇於面對逼迫.
甚至殉道.
教會.
聖徒的爭戰.
是在於他們無懼苦難.
無懼死亡的威脅.
不是拿刀拿槍去打.
反而.
是在於.
我們敢於.
承擔我們的苦難.
當然這個世界對.
聖徒的攻擊.
不是一定是硬攻.
不是一定去.
殺你頭.
不是一定捉你去鬥獸場.
給獅子吃.
更多時候.
可能是一種軟攻.
誘惑你.
和這個世界妥協.
所以聖徒的爭戰.
同時也是拒絕.
這是引誘.
不和世界妥協.
軟攻.
硬攻.
軟硬兼施.

$^{1001}$我們也要抵擋得住.
這就是我們.
啟示錄裡面說到.
我們參與.
在這場戰爭當中的方式.
不是靠勇武.
不是靠.
口罩.
不是靠搜雷彈.
不是靠鐵籠車.
靠我們的堅持.
持守基督的見證.
(廢話).
什麼叫做基督的見證?.
啟示錄裡面.
說到耶穌基督的見證.
出現了六次.
有五節的經文.
不詳細去讀了.
留意到兩件事.
第一.
說到基督的見證.
他有三次的出現.
是和神的道.
並排在一起.
神的道.
當然是說神自己的道.
來自神的真道.
所以既然基督的見證.
和神的道並排的時候.
所以基督的見證.
就是耶穌基督.
祂自己的見證.
這其實是一章二節最清楚的.
約翰就將神的道和耶穌的見證.
凡是他所看見的.
都見證出來.
耶穌基督的見證.
是約翰看到的其中一件事.
然後.

$^{1041}$約翰將他看到的.
這個耶穌基督的見證.
去見證出來.
所以基督的見證.
先是說耶穌基督.
祂自己.
發出的見證.
第二,我們留意到.
有兩處的地方.
說到耶穌基督的見證的時候.
是我們需要去持守.
保存著祂.
就好像耶穌基督的見證.
祂完成了祂的見證.
祂將祂的見證給了我們.
給了教會.
然後教會.
聖徒就要守著.
好像一個看監一樣.
守著這個見證.
不要讓它丟失.
不要讓它受到損害.
守著這個見證.
我們就能夠參與在這場戰爭.
參與在耶穌基督的勝利當中.
所以說到底.
到底什麼是基督的見證.
當然最直接.
就是耶穌基督.
祂用自己的生命.
來作出的見證.
耶穌基督.
祂道成肉身來到地上三十多年.
在三十多歲的時候.
祂出來傳道.
祂宣講神的道.
將神的道去教導.
教導我們.
將神的道去教導百姓.
教導以色列人.

$^{1081}$不單止宣講.
祂也親身去遵行神的旨意.
甚至全心信服.
以至於死.
死在十字架上.
這個是耶穌基督的見證.
被殺的羔羊.
十字架上.
耶穌基督作出.
祂最大最終極的見證.
祂告訴這個世界.
在神裡面有一個德性的盼望.
勝過當前的苦難.
即使是十字架的苦難.
就像希伯來書裡面說到.
祂因為擺在前頭的盼望.
就興旱羞辱.
藉著十字架的見證.
祂告訴這個世界.
死亡的權勢.
不能夠勝過神的大能.
耶穌基督.
對於當時的猶太人來說.
祂可能沒有改變到什麼.
耶穌基督來了.
耶穌基督死在十字架上.
耶穌基督復活.
升天.
如果我們看當時整個羅馬帝國.
似乎不是很大的改變.
但是在歷史當中.
經過幾十年.
經過幾百年.
耶穌基督的見證.
教會持守著耶穌基督的見證.
確是顛覆了整個羅馬帝國.
這個是教會的爭戰.
這個是我們參與在最終的戰爭當中.
所以其實說到終末戰爭.
這裡當然是終末.

$^{1121}$我們仍然相信.
在最終的時候.
神會介入歷史當中.
基督再來.
這個是我們相信的.
但是同時終末戰爭.
不只是在說遙遠的將來.
終末戰爭也在說現在.
其實我們已經.
參與在這場終末的爭戰當中.
藉著我們的見證.
這個戰爭.
既是外在.
天使和魔鬼的戰爭.
光明之子和黑暗之子的戰爭.
但是同時也是.
一開始我們說的內在的戰爭.
是我們裡面善念和惡念之間的戰爭.
引用很多年前.
一個廣告.
的一句話.
好像是黎明說的 我忘記了.
可能大家還記得.
要贏人.
先要贏自己.
然後再贏別人.
謝謝大家.
字幕由 Amara.org 社群提供.
感謝大家收看.
(字幕由 Amara.org 社群提供).
\newpage



\section{}
\label{sec:81BflBsZN_g}
\textbf{中神40周年院慶培靈會:張智聰博士}
\newline
\newline
連結: \href{https://youtube.com/watch?v=81BflBsZN_g}{\texttt{ https://youtube.com/watch?v=81BflBsZN\_g}} ~~~~ 語音日期: 2023-11-19 
\newline
\newline
\hyperref[sec:uudfjjhd4hA]{\small{< < < PREV SERMON < < <}}
~
\hyperref[sec:index]{\small{[返主目錄]}}
~
\hyperref[sec:UWFUcTme2GY]{\small{> > > NEXT SERMON > > >}}
\newline
\newline
$^{1}$來到最後一位要講的時候.
當然其他事情已經.
之前那麼多位嘉賓已經講了.
所以都不太知道可以講什麼.
最慘的那件事是因為怕over run.
所以一定要準時.
所以你為我祈禱.
希望我可以準時講完.
剛才講過啟示錄也講了.
不過之後我們再看.
其實在啟示錄裡面.
講的上帝將一切都更新.
這個景象.
其實不只是在啟示錄裡開始講過.
其實當以色列人.
他們面對亡國的時候.
其實上帝已經將同樣的景象.
給他們看.
猶大國末年的時候.
當時有一個強大的帝國興起.
南征北討所向披靡.
那個就是巴比倫帝國.
當時巴比倫帝國.
令到很多古晉東的文化.
國家都一一的來向他屈膝臣服.
猶大人其實他們嘗試.
來對抗改變扭轉這個政治的現實.
他們用自己最大的軍事資源.
留守到最後一刻.
希望保衛他們的城市.
他們嘗試聯絡其他的政治勢力.
幫助他們一起對抗巴比倫帝國.
同樣的當中有更加激進的人.
他們行食巴比倫帝國.
安立或委派給他們的總督.
那個叫行政長官.
那個叫基大利.
結果他們就一一.
每一種的抵抗方法.
都不能夠扭轉當時政治的形勢.

$^{41}$猶大國最終亡在巴比倫之下.
當時那班亡國的人.
他們其實都有一個想法.
在《以察書》40章第27節這樣說.
「雅各,你為何說?.
以色列,你為何言?.
我的道路向耶和華隱藏.
我的冤情上帝並不查問」.
對於當時一班亡國的以色列人來說.
他們都覺得上帝是否已經離棄了他們.
上帝是否已經不會在這個時代裡.
再去和他們展現再一個新的工作.
不過就在這個時候.
以察先知就向他們介紹了一個人物.
42章有一個這樣的人.
「看那我的僕人,我所扶持,所揀選,心裡所喜悅的.
我以將我靈賜給他,他必將功利傳給外邦.
他不怨揚,不揚聲.
也不是街上聽見他的聲音.
壓傷的蘆葦,他不截斷.
將殘的燈火,他不催滅.
他憑信實將功利傳開.
他不灰心,也不喪膽.
直到他在地上設立功利.
眾海島都等候他的分會」.
四字經文裡面介紹了這個僕人.
他的身份,他行事的方法.
和他的意志堅定一定要完成的東西.
不過無論他的身份,他的行事方法.
或者他的意志堅定一定要完成的.
都離不開一件事.
那件事就是功利要傳於外邦.
直到他將功利設立在地上.
這個功利是在說一個.
對於社會,對於當時的世界.
資源可以分配得平均一點.
權力可以用得合宜一點.
人活著可以多一點的尊嚴.
究竟這個僕人是一個怎樣的僕人?.
哪一個是這個僕人?.

$^{81}$竟然可以帶來一個這麼大的改變?.
對上去上面一章第41章又這樣多.
「唯有你以色列,我的僕人雅各,我所揀選的.
我朋友亞伯拉罕侯睿,誰是那個僕人?」.
以色列人,雅各,雅各的侯睿.
不是啊,有沒有搞錯?.
當時那群以色列人,什麼身份?.
他們已經成為一群亡國的一群人.
一群已經淪為巴比倫帝國之下.
一群亡國的一群囚徒.
對於他們來說,究竟有什麼政治的籌碼.
他們可以帶來的改變?.
他們甚至乎在國際的政治舞台裡面.
連一隻棋子,人們都懶得當他是.
這群人有什麼能力.
可以帶來一個這麼大的改變?.
不單止是這樣,我想看回剛才那段經文.
形容這個僕人是上帝所扶持.
當時在巴比倫的神話裡.
他們敬奉的神叫做馬督.
馬督在其中一幅畫裡.
描述馬督如何參赴巴比倫王的手.
對於以色列人來說,巴比倫這麼厲害.
沒理由沒有聽過這個神話.
不過以賽亞仙之就正正將馬督這個神話扭轉過來.
是他們所信奉耶和華的上帝扶持著.
不過扶持著什麼呢?.
剛才在馬督的神話裡扶持的是他們的王.
馬督和耶和華的上帝同樣扶持著.
是一位有王者風範的僕人.
這段經文下面說,我已將我的靈賜給他.
舊約聖經裡有不同的人.
都有上帝的靈在他們身上.
不過其中一個是我們都很熟悉的.
撒姆爾記上帝十六章十三節.
當撒姆爾告立了耶西的兒子大衛的時候.
十三節說,從這一天起.
耶和華的靈就大大感動大衛.
哪一個領受了上帝的靈?.
皇帝.

$^{121}$以色列人或者以賽亞仙之所說的這個僕人.
不單只是僕人.
他甚至是一位有王者風範的僕人.
但我們有沒有搞錯?.
我們剛才有沒有聽清楚一件事?.
以色列人當時在一個什麼處境裡面?.
被擄亡國.
一班只是淪為了帝國權勢之下的失敗者.
但是上帝就在這時候.
藉著他的先知告訴他們.
你們的身份不單只是一個僕人.
更加是一個有王者風範的僕人.
走出來說都笑死人了.
不過問題就是.
原來上帝就正正想這班子民.
re-imagine 他們的身份.
時代揀選他們.
不是,是上帝在這個時代裡面.
揀選他這班亡了國的子民.
成為帶來社會世界改變的change agent.
這是上帝對於他們的心意.
不過在《成卷二冊書》裡面.
如果你看到這班僕人的時候.
你都知道.
經常先知形容他們都是一班又聾又盲的僕人.
所以去到五十章的時候.
二冊先知就忍不住了.
他就走出來說.
讓我教你怎樣做一個上帝的僕人.
第四節到第九節.
二冊先知用了四次.
他講主耶和華.
怎樣改變他成為一個合乎上帝心意的僕人.
第四節講主耶和華賜給他一個受教者的舌頭.
使他知道怎樣用言語去輔助那些疲乏的人.
上帝給他有信息.
可以向他當代的人來扶持他們.
讓他們在困惑當中.
可以知道上帝對他們的幫助和說話.
不過這個舌頭怎樣培養回來的?.

$^{161}$每一天不是他自己去聽上帝說話.
是上帝親自叫醒他的耳朵.
為什麼要叫醒?.
因為睡著了.
上帝親自叫醒他沉睡的耳朵.
每一天來教導.
讓他最終可以聽到上帝交託給他.
向著這個時代要宣講的信息.
耳朵叫醒之後.
他跟著形容上帝主耶和華開通了我的耳朵.
什麼叫開通了耳朵呢?.
我們每個耳朵都開通的.
不過之後他就解釋.
是我不會違背也不會退後.
是我不會再從此之後.
我不會離棄那個從天上而來給我的照明.
我就會忠心順服地走在這個照明上.
不單止是這樣.
就算人怎樣打我.
我由他打.
人要mung 我兩腮那些鬍鬚.
我都由他去拔.
對於當時來說.
這個就叫做剃他眼眉的意思.
即是侮辱他.
大膽地羞辱他.
這個僕人他願意去順從上帝給他的使命.
甚至乎縱然他身邊的人個個都不喜歡.
個個都不認同他的緣故.
但那個並不會阻止他.
來跟隨上帝給他的心意.
為何他可以至死不渝地跟隨呢?.
第七節到第八節解釋了.
因為主耶和華必幫助我.
所以人不能夠定我的罪.
第九節他再一次解釋.
因為主耶和華必幫助我.
兩次都說主耶和華因為他幫助我的緣故.
以至今日我面對的挑戰困難.
我不需要擔心不需要害怕.

$^{201}$如果這樣說的話.
對於這位先知來說.
在一個時代裡面.
其實最大的問題不是有沒有人.
或者他一生裡面拿了多少like.
或者好像Facebook將會加那個是dislike.
不是他一生裡面有多少like或者是dislike.
他一生裡面最重要的一件事.
是他聽不聽得到上帝託付給他.
以至他可以向這個時代宣講的一個訊息.
或者用另一句說話來說.
對於這位先知來說.
如果一個生命需要犧牲.
這個不是最可憐的生命.
最可憐的生命是甚麼呢?.
是一個不知道自己可以為甚麼犧牲的生命.
伍玉聽他有一首這樣的歌.
其實我真的不懂唱.
下一站是不是天堂.
就算失望不能絕望.
伍玉聽他說.
縱然下一步.
不用翻譯廣東話,對不起.
縱然下一步我不知道能不能去到我們想去的地方.
不過都不能夠.
縱使令到我很失望.
但是我都不能夠放棄那份盼望.
不過對於他們來說.
對於阿信來說.
為甚麼他不能夠絕望.
是因為他仍然有一個追尋他的夢想.
那份堅持和堅執.
但對於我們.
這班上帝的子民百姓.
究竟甚麼不能夠令到我們絕望.
我相信不能夠令到我們絕望的是.
因為我們有上帝的說話在我們當中.
因為上帝給我們的說話.
讓我們重新去看.
今天我們的身份.

$^{241}$不單止是一班完全沒有任何選擇.
任何思考去面對這個世界的人.
我們是一班有王者風範.
王者僕人的身份.
而上帝希望藉著我們這班人.
縱使我們有多不足.
縱使我們會看錯.
縱使我們有多軟弱的情況.
不過祂藉著我們.
希望成就祂對於這個世界的心意.
不過我想上帝的說話.
更加呼召我們去成為一班.
為祂給我們的使命而去承擔.
縱然我們要面對的是很多的對抗.
或者是很多人對我們的不認同.
但這不是最重要的.
最重要的是我們聆聽上帝.
交託給我們每一個人.
究竟有一個什麼心意.
如果你還記得在提比利亞海邊.
當耶穌和彼得說將來彼得會怎樣死的時候.
彼得下一個反應是什麼呢?.
他轉過來看著耶穌所愛的門徒說.
那他呢?.
你記得耶穌回答他什麼嗎?.
關你什麼事?.
今天我們每一個人.
聆聽上帝給我們每一個人的心意和使命.
無論環境如何.
我們按著這份使命.
忠心來走.
或者這麼說.
正正是因為我們願意這樣忠心順服去走.
我們的生命才可以成為一個訊息.
向這個時代的人傳講.
又或者正正因為我們這樣忠心去走的緣故.
我們的順服被人看得見.
原來我們所信的上帝的說話是真實,是可信.
我們的生命成為這一代的人.
他們見到上帝的見證.

$^{281}$(掌聲).
(字幕製作:貝爾).
\newpage



\section{}
\label{sec:UWFUcTme2GY}
\textbf{中神40周年院慶培靈會:李耀坤博士}
\newline
\newline
連結: \href{https://youtube.com/watch?v=UWFUcTme2GY}{\texttt{ https://youtube.com/watch?v=UWFUcTme2GY}} ~~~~ 語音日期: 2015-10-14 
\newline
\newline
\hyperref[sec:81BflBsZN_g]{\small{< < < PREV SERMON < < <}}
~
\hyperref[sec:index]{\small{[返主目錄]}}
~
\hyperref[sec:S_AT9Lsna28]{\small{> > > NEXT SERMON > > >}}
\newline
\newline
$^{1}$(廣東話).
(陳志全議員).
因為這樣在歷史歷代教會裡面.
其實他們也是很關心自由這個課題.
有很多的反思.
有很多的討論.
是關於什麼是自由.
而在眾多的討論裡面.
我們不倫不提的.
就是這一位仁兄.
大家知不知道他是誰.
肥佬一個.
他是十六世紀.
德國的一位改革家.
他在1517年的時候.
他對當時的贖罪券的做法提出異議.
他行使他良心的自由.
結果是掀起了一場轟轟烈烈的宗教改革運動.
在這場運動裡面.
你可以說他經歷大風大浪.
在1520年.
就是在這個改革運動.
你可以說風起雲湧的時候.
他寫了一系列很重要的神學論文.
影響了以後教會的思想很深遠.
其中有一篇就是在這段時間他寫的.
就是討論到基督徒的自由究竟是什麼.
在這篇神學論文裡面.
馬丁路德一開始.
他給了兩句很奇怪的句子出來.
那句子是這樣說的.
基督徒是傳言自由的眾人之主.
然後接著他又再說.
基督徒是傳言順服的眾人之僕.
你眨眼看這兩句話是完全相反的.
而最難明的應該是第一句.
究竟這個世界哪裡有人這麼大言不慚.
說自己是眾人之主.
其實馬丁路德他要說一句這麼奇怪的話.
或者用一個這麼強烈的字句來表達.

$^{41}$他是要告訴教會.
或者告訴我們每一個信徒.
每一個基督徒在你的生命裡面.
其實有一個很激烈的.
一個拆毀和重建的工程.
是不斷在發生中.
而這個拆毀和重建的工程.
不是人的能力可以發生可以去進行.
馬丁路德說這個工程.
其實是上帝自己在進行中.
他要表明的是甚麼呢.
他說上帝在耶穌基督裡面.
現在要為我們每一個信徒.
在做一個重建.
或者破碎和重建的工程.
為我們預備在基督裡面的真正的自由.
而這個自由是一種雙重的自由.
所以他首先第一部份就要說一個.
你可以說是一個拆毀一個破的情況.
在這個破的層面裡面.
他特別指出的就是一個.
上帝要我們放下自我放下自異的一個工程.
馬丁路德按照聖經一路的教導.
他說其實這個世界人類社會裡面.
各種各樣的問題歸根究底.
其實都是罪帶來的各種的合制.
各種的捆綁纏累.
但是罪最狡猾的地方在哪裡呢.
他說最狡猾的地方是他會令到我們自己去欺控我們自己.
自己會欺騙自己的.
情況就好像怎樣呢.
就是你一個人說謊話.
最初可能知道自己在說謊話.
但是說著說著的時候.
你就會覺得.
嘩.
說著說著你就會將那個謊話當成是真的.
越說越真.
甚至你會覺得.
是喎其實就是這樣了.

$^{81}$在這個情況裡面.
人會怎樣呢.
人本來生命裡面有各種的幽暗破碎.
但是罪就會一路壓來壓到一個地步.
就是本來的幽暗突然間變成好像光輝燦爛.
然後本來我們裡面有各種各樣的問題.
但是說著說著覺得自己是無限正義.
所以馬努奴德就說.
上帝其實現在在做一樣很極端的事情.
上帝在做一個很大的手術.
是你想都沒想過的一個手術.
他要怎樣做才能夠去.
你可以說擦穿人這個向自己發出的謊話呢.
不動的.
他說上帝派他自己的兒子.
進入到我們這個充滿黑暗的世界裡面.
將真正的光明帶到我們這個世界裡面.
然後以致到我們才能夠真真正正.
看到我們自己真正的那個本相.
在這裡馬努奴德引用耶穌基督他自己的說話.
他說耶穌基督這樣說.
真理是使你得自由.
然後接著下去耶穌更加清楚地說.
如果神的兒子使你得自由.
你就是真自由.
在這裡馬努奴德很快就明白到.
耶穌基督來到就是那個最真實最真正的光明.
他來到就將我們那些假的真理那些假的謊言完全擦穿.
以致到我們迫使我們真真正正去面對那個真正的自己.
我們馬上看到自己.
原來我生命裡面是真真正正充滿各種的破碎.
原來我自己的本相就是這樣.
不單止這樣.
透過耶穌基督不是要令到我們很羞愧.
他是要我們知道.
他完全知道我們就是一個這樣的人.
我們不需要在他面前再有任何的掩飾.
我們可以真正直視自己生命的真正的狀況.
不單止這樣.
馬努奴德還說.

$^{121}$在裡面我們才真真正正經歷由神而來的真正的釋放.
而這種釋放其中一個很重要的結果是什麼.
我們真的可以很真誠地去認錯.
其實很真誠地去認錯.
我們很多時候信主一段時間之後.
我們經常都說.
是啦我知道你都是要講悔罪.
是我都是講悔罪.
不是什麼你沒聽過的東西.
不過你知不知道可以很真誠地面對自己.
很真誠地去認自己的軟弱.
認自己的罪.
這個其實是一件很難得.
很了不起的一個上帝給我們的禮物.
這位老人家.
他的名字叫做韋利·勃蘭特.
他是在1969年到1974年出任西德的總理.
他出任西德總理的時候.
他在1970年的時候.
他就出訪當時波蘭的華沙.
去到那裡和當地的政府有一些會談等等.
大家知道在二次世界大戰的時候.
納粹德國是攻佔了波蘭.
並且在當時殺害了很多波蘭的人民.
不單止是這樣.
納粹德國還在波蘭那裡設立了一些集中營.
用這些集中營來有系統地去準備滅絕猶太的民族.
當勃蘭特出訪華沙的時候.
剛好是在當地紀念二戰的時候.
有一件事件.
那件事件就是猶太街那裡.
猶太人因為被納粹德國趕到那裡.
然後他們進行反抗.
然後納粹德國就去鎮壓那個情況.
結果死了很多很多猶太人.
他去到的時候.
就剛好是在紀念這件事的日子.
於是乎 勃蘭特就和那些紀念的隊伍.
一直一直去到紀念碑那裡.
一直走 走到紀念碑前面獻花.

$^{161}$但當他獻花的時候.
他做了一個你可以說是震驚全世界的一個動作.
他就在那裡跪在紀念碑的前面.
然後在那裡默哀.
他這個動作.
令到當時在場的波蘭的那些政府官員.
手足無措.
因為在他們最初的那個計劃裡面.
那個行程裡面.
完全是沒有這樣的事情.
後來很多年之後.
有記者訪問回勃蘭特.
勃蘭特就和他們說.
我當時是完全沒有想過.
完全沒有預先計劃.
是這樣去做這個跪下的動作.
他說他去到那裡.
就是當一直這樣去的時候.
他走到那個紀念碑的時候.
他一直想起德國之前.
對波蘭的人民所做過的一切.
對猶太人做過的一切.
然後在那一刻.
他就很真誠地這樣跪在那裡.
代表德國向當時所有被他們傷害過的人.
去表達那種悔罪.
他說他一直做這件事的時候.
他之前從來沒有想過.
究竟國內國外.
究竟會用什麼眼光來看待這個情況.
他完全沒有想過.
究竟之後會有什麼後果.
他就是很簡單.
就是在那裡真誠地跪在那裡去悔罪.
結果.
結果怎樣呢.
他這個下跪.
撫平了很多波蘭百姓.
他們心裡面那種的悲傷.
以至他們那種憤怒仇恨.

$^{201}$你可以說.
波蘭特他在這個訪問裡面.
他完成了一件很難以置信的事情.
他重新贏取波蘭的人民對德國的信任.
結果他在這個行程之後.
他成功地令到德國和波蘭重新簽訂協議.
你可以說兩國重新修補這個關係.
這件事是一點都不簡單.
要記住在1970年的時候.
是冷戰最厲害的那個時間.
在這段時候.
你可以說東西方的陣營那些非等是互相對著.
竟然一個西方的領袖.
可以在這樣的場合裡面.
做出一個這樣的動作.
並且能夠獲得波蘭的人民重新去信任.
這件事是很匪夷所思.
亦都是因為.
你可以說整個德國的態度.
整個歐洲的歷史的各方向.
都是為之改變.
如果我們看看亞洲這方面.
你就很明白當中的差別.
這幅圖大家都知道他是哪一位.
不用介紹.
安倍晉三.
剛剛過了的中戰70年的演講裡面.
他的演講裡面特別提及的.
就是一句這樣的說話.
我們不能夠讓戰爭毫無關係的子孫後代.
擔負起繼續道歉的宿命.
為甚麼70年過了.
仍然日本在掙扎中.
你可以說要終止這個宿命.
或者應該要想一想.
為甚麼亞洲其他人民.
仍然會咬住日本不放.
而在歐洲卻是完全另一番的景象.
當我們看見十架的時候.
我們才真真正正明白.

$^{241}$原來因為人的自欺.
因為人那種撕裂互相的傷害.
差異原來是造成多大多深遠的傷害.
而有十架我們也再一次明白到.
原來上帝為我們預備.
這個可以認錯的自由.
是一個多麼寶貴的自由.
我們從來都沒有這樣去想過.
這樣去欣賞過十架原來為我們預備的.
是一個怎樣的自由.
但路德不是停在這裡.
路德說當這個自由.
這個釋放了的自由之後.
你可以說接下來的是甚麼自由呢.
是我們重新可以得到一個.
雀躍服侍上帝的自由.
在這裡路德就會說.
當我們不再死撐.
當我們不再憂慮充滿恐懼.
這樣去證明自己的時候.
我們就可以重新有一種釋放.
在這個釋放裡面.
我們很簡單的.
就是將我們經歷過.
在上帝裡面所經歷過的各種的美善.
去跟我們身邊的鄰舍去分享.
路德有一段話是這樣說的.
他說由信就流出在主裡面的那種愛和喜樂.
又由愛那裡流出自由的心靈.
然後我們樂意侍奉我們的鄰舍.
不計較感恩或者忘恩.
不計較抵毀還是讚譽.
不計較得還是失.
在基督裡面的人.
他們不分敵友.
也不期望別人去感謝或者不感謝.
因為他們看到在基督裡面.
由基督裡面所展現出來的.
賜福的那位上帝.
就是一個這樣的上帝.

$^{281}$這位老人家.
他的名字叫做烏巴拿.
這個名字我最初聽的時候都覺得很特別.
他是緬甸一間孤兒院.
一間孤兒院叫做Full Moon的一個創辦人.
這裡是這個孤兒院裡面一班小朋友.
他們在敬拜的情況.
乍看之下跟你教會的兒童的崇拜.
或者兒童主義學都沒什麼分別.
都是這樣舉手唱歌.
不過原來這一班小朋友.
是因為山上的部族衝突.
然後遺留下來的一班孤兒.
意思是什麼呢.
跟他在身邊一起敬拜的那個.
分分鐘就是他殺父母仇人的下一代.
烏巴拿牧師將他們招聚在一起.
然後用基督的愛來關懷他們.
去照顧他們.
然後讓他們走在一起.
一起生活一起成長.
這一間孤兒院是在郊區的一個地方.
但是竟然吸引了另一位人的注意.
這張照片裡面在後排.
左起第四個.
有沒有認到一個很熟悉的面孔.
就是昂山素姬.
昂山素姬當被軟禁釋放之後.
她其中一個經常要處理.
或者思考的問題.
就是怎樣為緬甸帶來真正的和平.
怎樣能夠化解部族衝突造成的傷害.
昂山素姬就留意到這個孤兒院.
然後在2012年的時候.
昂山素姬就邀請了各個部族的領袖.
出席在緬甸的Union Day的慶祝活動.
然後他們也請了Full Moon這間孤兒院裡的孤兒.
來一起去表演一個Union Dance.
這張照片就是當中他們表演的兒童.
他們當中有七個部族.

$^{321}$然後每個部族派一男一女代表自己的族.
穿上他們自己部族的傳統服飾.
然後他們就一起去跳Union Dance.
這就是他們一起拿著彩帶在跳這支舞.
一邊跳一邊織一支很漂亮的旗桿.
在這裡他們表演給在場的部族領袖看.
你想一想這件事是一件很不簡單的事情.
他們這一班就是在成人世界的仇殺之下.
的一班受害人.
但是現在這一班受害人.
他們竟然能夠在一班成人的面前.
就是這些領袖的面前.
去教他們一個很寶貴的功課.
這就是在基督裡面的自由.
在基督裡面你可以說.
當我們去運用行使這個自由的時候.
我們就會將基督的愛和復活的大能.
你可以說是活化在人的面前.
在裡面不是什麼轟轟烈烈的事情.
也不是波瀾壯闊的場面.
但是一點一滴就寫在人的生命裡面.
也都寫在那個社群的歷史裡面.
這種的自由用勞德的說法.
不單止給了烏巴拉牧師.
不單止給了孤兒院裡面的兒童.
也給了我們在座每一個屬於基督的人.
你願不願意在你的生命裡面.
為主為你的鄰舍去行使.
上帝已經交給你的一份的自由.
(字幕製作:貝爾).
(字幕由 Amara.org 社群提供).
\newpage



\section{}
\label{sec:S_AT9Lsna28}
\textbf{中神40周年院慶培靈會:沈祖堯教授}
\newline
\newline
連結: \href{https://youtube.com/watch?v=S_AT9Lsna28}{\texttt{ https://youtube.com/watch?v=S\_AT9Lsna28}} ~~~~ 語音日期: 2018-10-21 
\newline
\newline
\hyperref[sec:UWFUcTme2GY]{\small{< < < PREV SERMON < < <}}
~
\hyperref[sec:index]{\small{[返主目錄]}}
~
\hyperref[sec:g45zbSUTL0c]{\small{> > > NEXT SERMON > > >}}
\newline
\newline
$^{1}$(字幕:陳志全).
大家好.
原來要大聲.
(笑聲).
我在進來這個禮堂的時候.
覺得很驚訝.
因為原來我以為宗神叫我去分享.
宗神有多大呢?.
應該都是一百多人.
原來來到一個這麼大的禮堂.
今晚在這裡.
我真的看不到你們.
今晚在這裡請問有多少人是三十歲以下.
或者三十六歲以下的請舉手.
(掌聲).
我們先給他們掌聲.
(掌聲).
你也是三十歲,你像不像?.
Roger你不要騙我.
三十歲以上的你們今晚可以休息一下.
因為我們今晚的對象是年輕人.
李思敬博士也想我在這裡和年輕人分享一些看法.
在開始的時候我想跟大家說.
第一,我也年輕過.
也知道過三十歲以下.
不過我今天已經不是了.
所以我今天如果我形容一群八十後,九十後,零零後.
可能會有些偏差.
因為我不是在零零後,九十後出生的.
但是我也可能會有一些看法.
其他人未必有.
因為我每天都在接觸八十後,九十後,零零後的年輕人.
最近香港很多人對於大學生,對於年輕人.
有一個比較負面的稱呼,廢青.
尤其是大學生.
那些人說你為什麼讀大學讀成這樣?.
其實這句話是由我媽媽或我媽媽的媽媽那時也這樣說.
你真的不像我了.
我那時像你那麼大我就怎樣怎樣.
現在真的是一代不如一代了.

$^{41}$各位,我絕對不同意這些說法.
我覺得今天的年輕人一點都不廢.
我覺得一代不如一代是不公平的說法.
我覺得應該說一代不同一代.
我說這些不是拆你鞋的.
我是真的有這樣的想法.
但是我也真的對年輕人有一些認識.
或者今晚也值得你去思考一下.
不知怎樣.
這幅畫是講述一個希臘的神話故事.
在希臘的神話裡面有一個男孩是很帥的.
他的名字叫Narcissus.
這個男孩可能是在眾多神裡面最帥的.
他有一個女朋友叫Echo.
Narcissus就常常說我很帥.
他的女朋友就說你很帥你很帥你很帥.
他是Echo.
因為他不停地Echo他說的話.
是真的我不是說笑.
故事就是Narcissus覺得自己帥到一個地步.
終於有一天闖了禍.
就好像這幅畫一樣.
有一天他去到一個池邊.
他看到池裡面的倒影.
這個人真的帥.
像劉德華一樣.
他覺得自己的形象帥到他走不開.
因此他就站在池邊走不了.
直到他被泡在池裡浸死了.
所以在英文裡面.
現在有一個英文的字叫Narcissism.
就是自戀.
自己愛慕自己到一個地步.
接近病態.
在一年前有一本.
又說按這個可以的.
在一年前有一本時代雜誌的封面.
就是這個女孩.
她拿著一個手機.
和自己在拍Selfie.

$^{81}$其實我現在也很熟悉了.
做這件事.
你知不知道拍Selfie.
不要看著鏡子中間.
要看著左上角的點.
不然你的眼睛就會歪了.
這個封面的題目叫.
This is the me me me generation.
這個是我我我的一代.
意思是說什麼呢.
其實那個女孩.
寫了這篇頂尖的文章在IMAGAZINE.
她說我是一個二十多三十歲的年輕人.
我絕對有資格寫這篇文章.
因為我正在描述我們這個世代.
她說這個年代的人是怎樣的呢.
不是我說的.
她說The millennials are lazy.
很煩的.
Entitled the narcissist.
和Living with their parents.
和住在母親那裡.
但是其實這又代表什麼呢.
這又有什麼問題呢.
其實她在文章裡面說了一些數字.
我想告訴大家.
讀一讀.
她說根據美國的一些調查裡面說.
她說現在的年輕人.
有40%的人.
相信他們每兩年是要.
如果正在工作的話.
相信每兩年是應該要升一級.
覺得升一級是一個entitlement.
很值得的.
因為我這麼靚.
他們覺得他們.
他們很容易很想找到工作.
但是他們又不太想.
take up那份工作的responsibility.

$^{121}$那個責任.
你不喜歡要什麼責任.
他們每天平均send30個text message出去.
接50個text message.
這個是較了一點.
應該不止的.
她說在這個世代裡面.
人是越來越喜歡自己.
所以我們經常對著不同的場合.
對著鏡頭拍自己的樣子.
又拍自己吃的牛扒.
又拍下甜品的樣子.
然後再拍自己的樣子.
然後放上facebook那裡.
然後大家like.
我又很開心.
其實我有少少做這些這樣的事.
但是這個代表著什麼呢.
就是其實真的和我.
小時候的時候.
我十幾二十歲的時候有點不同.
我們是比較花俏一點的.
我們不會.
我就不會站在前面拍照.
然後把照片放在拍照簿的玻璃窗裡.
我就不會做這些事.
但是這個self image.
這個self esteem是很高的.
這個世代.
因為爸爸媽媽都會告訴你.
你是他們的王子和公主.
因此慢慢慢慢我們覺得.
我們是這個世界的中心.
帶來的後果就是.
第一.
我們覺得如果這個世界有些東西.
我得不到的話.
為什麼?.
為什麼這樣?.
為什麼學位不夠?.

$^{161}$為什麼我宿舍那裡又要加一百元的贖費?.
為什麼今年學位派不到我?.
為什麼我出來找不到一份.
別人找到一份一萬八千元.
我找到一份一萬六千元的工作?.
為什麼我住的房間那麼小?.
我們會有一種.
很容易會問.
為什麼我沒有?.
為什麼你不給我呢?.
我entitled的.
因為可能在我們小時候.
父母都把所有最好的東西都給了我們.
當我們去到工作的時候.
更加有一個很大的危機出現.
甚至還沒有去到工作.
就有一個危機出現.
為什麼我的career.
我已經做到五點半了.
我還沒有走.
你還想我怎樣呢?.
我有自己的私人生活.
我的quality of life很重要.
我在醫院裡面做了很多年.
到我做最後那幾年的.
一個內科系的主任的時候.
竟然有個媽媽來找我.
她說你為什麼要我兒子.
一個月當四次更那麼多呢?.
我說不是很多.
我以前六天都當了四次更.
每隔六天.
現在你兒子三十天才當四次更.
我說不是啊.
他不用回家嗎?.
你不請多些人?.
為什麼你要我兒子當四次更那麼多呢?.
我都投了一行.
我轉頭想.
為什麼他兒子不來找我.

$^{201}$他媽媽來找我.
說他要當四次更呢?.
那個shocking的地方是.
我們不會去想.
是否我們值得去得到.
我們認為所得到的東西.
然後.
如果我們受到一些挫折的時候.
我們更加容易接受不到這個挫敗.
真的.
大家都知道.
由今年開學到現在.
已經有三個學生跳樓死了.
三個.
其實這些數字.
每一次我看到這些數字.
我都很不開心.
無論它是我的學校還是不是我的學校.
因為.
你知不知道.
心理學家告訴人.
人最傷心的情況.
不是.
unfortunately.
不是你爸爸媽媽的死.
不是你的妻子或者你丈夫的死.
人最難接受到的是你的兒女的死.
所以中國人說.
白頭人送黑頭人是一件很痛苦的事.
所以每一次當我看到.
無論是小學中學或者大學生.
因為某些原因.
受不到那種打擊.
而輕生的時候.
我都覺得是非常難過.
我還記得幾年前.
有一個我們學校的學生在.
這個.
開學沒多久.
就跳樓了.

$^{241}$我去到他家裡慰問他.
帶上我們的Dean of Students.
就是副導長.
一起去探望他.
爸爸還是很鎮定地.
請我們進去他家裡.
倒茶給我們喝.
坐在那裡不說話.
然後我們都不知道該說什麼.
當副導長拿出那張學生證.
他說.
某某先生.
對不起.
我們還是把這張學生證給你.
哇 那一刻.
整個人崩潰下來.
爸爸哭到沒辦法說話.
媽媽更加不再說話.
那種痛苦.
我到現在都非常深刻.
有一次有位內地生.
很多年前.
在內地來讀書的學生.
高材生.
狀元 是整個省的狀元.
進來之後.
所有東西都拿了A.
然後畢業.
然後他爸爸就說.
你不如在大學讀碩士學位吧.
他就讀碩士學位.
誰知道第一個學期.
有一科不合格了.
然後他在學校吊頸死了.
我們打過電話去他的家鄉.
打了三次.
他們都不相信.
以為有人惡作劇.
作弄他.
到第四次他終於相信.

$^{281}$於是他全家都下來.
我還記得那一幕.
坐在我的辦公室裡.
我們拉上幾張沙發.
爸爸坐在這裡 媽媽坐在這裡.
然後姨媽姑姑.
全部人轉圍著我們.
我簡直覺得那張椅子在震動.
真的在震動.
因為父母無法接受.
他一生的希望.
一生最大的驕傲.
竟然會在這樣的環境裡.
放棄了自己.
自我自信 自我形象.
以及我們的獲利.
似乎是這個世代的年輕人.
一個很重要的特色.
當在五六十年代.
一個家庭 一對夫婦.
可能有四五個子女.
六七個子女.
在家裡餵飯.
哪有空餵.
餵完都不知道餵了誰.
放了兩碟菜在這裡.
你吃就吃.
我收碟了.
爸爸就會說.
兒子 你能入讀中學就讀吧.
能入讀大學就讀吧.
不能入讀就去學師吧.
我們那時候是這樣的.
現在不是.
現在我們是非常之不同.
Time Magazine 真是很小心.
這些 Gender Issues.
他們製作了一個女的.
還要製作一個男的做封面.
男的都是 Me Me Me Generation.

$^{321}$我發覺原來第二版是一個男人.
他一樣都是說.
自己愛自己.
以致到一個地步.
就是我們覺得.
我們所看的東西是完全正確的.
我所說的東西都應該是正確的.
尤其是如果我的 Peers.
都是這樣看的話.
這個應該不會錯.
所以當挫敗來到.
當人生不如意的事來到的時候.
Why Me?.
為什麼是我呢?.
當我得不到一些東西的時候.
Why Me?.
但當得到那些東西的時候.
我們又忘記了.
Being Grateful.
應該要去感恩.
應該要有一種.
身體對我真是好.
爸爸媽媽對我真是好.
這個世界對我真是好.
可能是少一些的 Gratefulness.
多一些的 Grievances.
對於子女是這樣的.
父母又怎樣呢?.
我就很不喜歡人家說.
父母是怪獸家長.
因為我都是家長.
所以我就不太喜歡.
為什麼按來按去都不去.
原來外國人有一個比較文明的 Term.
叫做直升機家長.
Helicopter Parents.
我也查了一下 Wikipedia.
為什麼外國人叫家長做 Helicopters.
原來直升機就常常在上面.
高空盤旋看著.

$^{361}$有什麼事發生就馬上衝下去.
所以這個直升機家長這個名字.
是代表著那些父母.
是非常關注他的子女.
尤其是在教育方面.
所以我們開學.
譬如我們收一千人.
就會離出三千人坐在禮堂.
我們收三千人就離出一萬人.
因為父母都來了.
我們開學的 Open Day.
問問題的全部都是家長.
那些子女只是坐在那裡.
好像觀眾一樣.
問完問題之後.
媽媽就說我們去哪裡看.
那個兒子就說好啊好啊.
然後就走了.
全部都是由父母去作主.
這是不是一件壞事.
也不是一件壞事.
我也是父母.
我也知道為什麼父母要這麼關心.
因為我只有一個.
或者我只有兩個.
我當然要把最好的東西給他.
我就要把最關心.
所有的關心.
所有的能力.
所有的 attention 都放在他那裡.
但是我們做先生的.
我們做教育的.
就希望這些直升機降落之後.
都慢慢升上去.
不要整天盤旋在我們學校的天空那裡.
或者你久不時都飛到別處去看.
因為你整天盤旋在我們那裡.
我們是做不到事的.
所以 helicopter should come down and go.
就不要留在那裡.

$^{401}$其實父母的寵愛.
也都帶來了今天年輕人.
一個相當之 dependent 的.
相當之依賴.
相當之依靠父母的一個心態.
在大學裡面.
我們經常都會製造一些機會.
給學生有實習的時候.
我就跟一班學生說.
我們今年在內地找了幾百個實習的位置.
希望你有機會去工廠.
去銀行做實習.
他們說好啊好啊.
不過去哪裡呢.
我說到處都有的.
北京也有.
武漢也有.
西安也有.
他們說可不可以近一點.
不要去那麼遠.
因為我不想去那麼遠.
我說哪裡才叫遠呢.
他們說可不可以在東莞之內.
東莞不就真的很近.
其實一個小時車程.
我們連工作.
我們連嘗試去工廠.
我們都不太希望太遠.
因為離開了自己的 comfort zone.
離開了自己的舒服或者安全範圍之內.
這個也是今天一個現象.
我們常常都見得到.
我再說一次.
我絕對沒有貶義對於今天的年青人.
其實今天的年青人.
如果他們有的問題.
是我們製造出來的問題.
對不起所有爸爸媽媽.
是我們作為父母.
是有份去製造出來的問題.

$^{441}$但是我們也要想一想.
不要 condemn 他們.
因為我們小時候.
我們父母都是說我們沒用.
都是說我們一代不如一代.
剛才說過.
每天打三十個字.
我想一定不止.
我想在這裡澄清一下.
我是沒有 WhatsApp 的.
所以在 WhatsApp 裡面說我講了什麼.
都是假的.
那些文字都不是我寫的.
因為我是沒有 WhatsApp 這件事.
我也不懂得用.
各位.
說了這些問題之後.
我想想想.
作為一個基督徒.
作為一個稱為上帝的兒女的人.
我們又應該怎樣走前面的路呢.
剛才那個 video 一開始就說.
這是什麼世代.
你想二十年之後變成什麼.
你今天是二十歲.
你今天是三十歲.
你想四十歲五十歲的時候.
你會走到哪裡呢.
聖經裡面有一句.
令我震撼的話.
羅馬書第十二章第一節.
它這樣說.
所以弟兄們.
我以上帝的慈悲勸你們.
要將身體獻上.
當作活祭.
是聖潔的.
是神所喜悅的.
你們這樣侍奉乃是理所當然的.
我不太明白這句話說什麼.

$^{481}$怎樣有身體獻上.
當作活祭呢.
如果用 Eugene Peterson 的翻譯.
他說.
So here is what I want to do.
What I want you to do.
God helping you.
Take your everyday ordinary life.
不是轟轟烈烈那一天.
不是考試.
不是一些很重要的日子.
Everyday your ordinary life.
Your sleeping.
Your eating.
Going to work.
Walking around life.
And place it before God.
As an offering.
每一天.
由你起床到你睡覺.
擺它上去.
當作一個offering.
Embrace what God does for you.
Is the best thing you can do for Him.
擁抱著.
擁抱著他為你所做的一切.
究竟說什麼呢.
下一句很精彩.
他說不如效法這個世界.
要心意更生而變化.
察現什麼是神的善良.
純全可喜悅的旨意.
如果用他的譯本.
突然之間好像有很多個字跳了出來.
令我很吸引.
他說.
所以各位年輕的80後.
90後.
00後.
他說Don't be so well adjusted to your culture.

$^{521}$不要太過快速適應.
今天這個社會.
你身邊的culture.
that you fit into it.
without even thinking.
你想都沒想過.
你就跟著別人去做.
你想都沒想過.
你就剪了頭髮.
剪到只有頭頂一束草.
旁邊全部什麼都沒有.
你都沒想過.
那像什麼呢.
你要想一想.
Don't fit into the culture so readily.
Instead.
不過呢.
Fix your attention on God.
You'll be changed from inside out.
他說了一個很重要的東西.
就是你的眼光.
你的focus.
不要看自己.
剛才我說.
我們這個generation.
Me, me, me generation.
但是他說.
將你的鏡頭轉向上.
Don't fix it on yourself.
Fix your attention on God.
You'll be changed from the inside out.
你會整個人.
由裡面一直改變.
To readily recognize what he wants from you.
自自然然你就會知道.
他想你做什麼.
不用有聲音.
突然間一個聲音.
一個喇叭走出來.
不用突然之間.

$^{561}$那天看到聖經有句話跳出來.
你慢慢慢慢.
將你的focus.
不要放太多在你自己.
不要太過戀愛自己.
去想一想.
He is God.
他才是那個決定.
不單只是我的一生.
他才是那個決定香港的前途.
他才是那個決定.
這個世界明天會發生什麼事的上帝.
Readily recognize what he wants from you.
And quickly respond to it.
你要有行動.
你要respond to上帝.
對你那個的影響.
Unlike the culture around you.
不要和你旁邊的culture一樣.
不要對著facebook.
跟著人家說去吧.
那我就去.
不要看到whatsapp說.
我們今天就去做那件事.
就去罵那個人.
我們就一起去罵那個人.
Unlike the culture around you.
always dragging you down.
to its level of immaturity.
他說我們旁邊的culture.
好像帶著我們去越來越未成熟.
越來越不成熟.
God brings the best out of you.
上帝會將你最好的部分拿出來.
Develop well for maturity in you.
帶給你真正的成熟的人生.
這句話是不停在我的腦海裡面想了一年多.
時時我都.
當我面對年輕人.
我面對我自己.

$^{601}$我今年已經五十.
多一點點了.
我都還是覺得.
我是需要不要太過focus自己.
不要太過想我今天人生的得失.
別人讚我還是批評我.
別人對我是好的還是不好.
不要太過注重我自己個人的得失.
看我自己是否被欣賞.
我放了一張相片上去facebook.
有多少個like在裡面.
No.
Fix your attention on God.
And He will bring the best out of you.
And develop your well formed maturity.
如果你只是看著自己.
不停欣賞自己.
你就好像那個快要掉下去井的Narcissus.
你就會自戀.
帶給自己.
跟著的是一個自我的毀滅.
因為你會走不出你自己的世界.
It's important that you do not misinterpret yourself.
as people who are bringing this goodness to God.
不要覺得自己是因為我做了這些好事.
所以將榮耀帶給上帝.
No.
God brings it all to you.
他說所有的東西.
其實是他帶給我們的.
其實整段說話說來說去.
都是將主和客的位置調轉.
我們Me Me Me Generation.
我們就是主人了.
今天各位.
如果你想成熟.
你想在你的人生裡面找到.
上帝要你走的方向.
如果你想知道你將來要怎樣服侍這位神.
才是將身體獻上當作活祭.

$^{641}$將你的camera轉過來.
不要太看自己的得失.
不要太看自己是不是很靚.
不要太看自己有沒有多了一條皺紋.
太看自己賺了多少錢.
太看自己和你的朋友相比.
你的成功指數是多少.
No.
Fix your attention upon God.
最近我看.
說完這一句.
最近我看了一段聖經.
《若書也記》.
那裡很鼓勵到我的聖經.
因為我發覺在這一年裡面.
或者在這幾年的裡面.
香港變成了一個社會.
就是壁壘分明.
敵或者是友.
是要分得很清楚的.
你是幫我的呢.
還是你是對抗我的呢.
我是站在黃色那邊的.
你是站在藍色那邊的.
如果我穿一件綠色衣服就比較安全一點.
但是《若書也記》第三章裡面說到.
當他帶著一班以色列人過了約旦河之後.
他面臨很多的敵人.
知道快要打仗了.
因為在那個地方有很多又強大又好打的敵人.
那班就是一班烏合之眾.
更慘的是昨天還做了國禮.
心想都不知道怎麼打仗.
《若書也記》在那個晚上在河邊走來走去的時候.
他一眼看下去就知道是耶和華的使者.
因為他手裡面拿著一把劍.
耶和華的使者來到.
《若書也記》就馬上問他.
究竟你是幫我的呢.
還是幫我的敵人呢.

$^{681}$Are you on my side or on the side of my enemies.
究竟你是藍的還是黃的還是綠的呢.
上帝的使者竟然這樣回答他.
他說我來是掌管耶和華的軍隊.
I am here to command the army of the God.
我不是幫你的.
我又不是幫他的.
因為我才是掌管這個歷史.
我才是掌管每一天發生的事.
他說你站在的地方是聖地.
So take off your shoes.
脫下你的鞋.
意思就是說將你的主權.
將你自己的自我放下.
約書亞就馬上脫下他的鞋.
俯伏在他的面前.
承認主權是屬於上帝的.
各位如果我們太過著重我們自己個人的得失.
We totally miss the point.
而這個miss the point.
就成為了我們不能夠成長不能夠成熟.
這個就是今天我要給大家的小小的分享.
多謝.
(字幕製作:貝爾).
\newpage



\section{}
\label{sec:g45zbSUTL0c}
\textbf{中神40周年院慶培靈會:鍾氏兄弟}
\newline
\newline
連結: \href{https://youtube.com/watch?v=g45zbSUTL0c}{\texttt{ https://youtube.com/watch?v=g45zbSUTL0c}} ~~~~ 語音日期: 2015-10-14 
\newline
\newline
\hyperref[sec:S_AT9Lsna28]{\small{< < < PREV SERMON < < <}}
~
\hyperref[sec:index]{\small{[返主目錄]}}
~
\hyperref[sec:DregqwIedRk]{\small{> > > NEXT SERMON > > >}}
\newline
\newline
$^{1}$(廣東話).
Hello 大家好.
各位來賓.
各位弟兄姊妹.
沈校長.
大家好.
我們是忠士兄弟.
我是Henry.
我就.
有聲音了.
終於.
我Roger.
大家好.
今天主題很厲害.
是的.
就在問香港現在這樣.
我們怎樣呢.
我們作為年輕人.
八九十後.
還有00後.
是的.
我們應該怎樣.
很大的課題.
是的.
你有沒有想法呢 Henry.
其實我今天本來.
雷敬業博士邀請我的時候.
他應該告訴我今天是畢業禮.
所以我就準備了一篇.
我以為很偉大的講章.
加上沈校長也在.
又不可以太過失禮.
我又發現.
原來我準備了這篇所謂的講章.
今天也可以適用.
雖然不是畢業禮.
但我覺得.
跟大家分享一下也挺好的.
好的.
也希望可以跟之前.

$^{41}$兩位講者說的話.
可以有些呼應.
也不算是講章.
只有幾個重點.
是的.
對於我來說.
可以說是這樣說.
Roger.
我不問你.
我問大家.
其實大家可以放鬆一點.
不用太緊張.
或者我們都很.
專心而已.
很放鬆的.
不要緊.
可以動一下身體.
或者怎樣.
記不記得2005年.
Steve Jobs去史丹福大學.
給了一個畢業的演講.
他說了些甚麼.
最有名的一句.
就是.
Stay hungry.
和Stay foolish.
就是.
叫我們.
經常覺得自己不足.
和要渴求一些知識.
我們不知道的東西.
要經常存一個渴求的心.
其實也非常好.
我覺得有一點像耶穌在.
登山補訓的時候說的八福裡面.
饑寒無異的人有福了.
我作為一個聽眾.
我聽到這句話的時候.
當然也很振奮人心.
但是.

$^{81}$我小時候經常會.
質疑老師說的話.
老師說東.
我就覺得應該這樣是不對的.
應該是西才對.
我覺得我有一點質疑Steve Jobs的說法.
我就.
腦筋一轉.
不如這樣會不會更好一點呢.
我就反過來想.
Stay fool.
和Stay wise.
會好一點嗎.
Stay fool.
Fool就是Hungry的相反詞.
飽足.
Foolish的相反詞就是Wise.
我就在想.
我可不可以做一個論述.
可以很圓滿地解釋這個論述呢.
我就想了一下.
就覺得應該可以的.
因為Fool.
我們說Fool是.
我們要知足.
聖經裡面叫我們凡事要謝恩.
還有.
我們覺得耶和華的恩典.
是永遠夠我們用的.
我們也要經常存一個謙卑的心.
我們有一首歌叫做《行公義好憐憫》.
但很多人都忘記了.
其實最重要是後面那句.
全謙卑的心與我們的上帝同行.
還有在聖父,聖靈三權之上.
其實我們不是超然的.
只有上帝才是超然的那一位.
這就是Stay fool.
我們要經常覺得自己.
是要用謙卑的心去.

$^{121}$走我們前面的路.
那Stay wise 是什麼意思呢?.
Wise是指有智慧.
不是小聰明那些.
不是自己上載東西到Facebook上.
點讚那些小聰明.
而是真的在說大的智慧.
所羅門王在《針燕》中經常求上帝.
給他一些智慧.
這件事我自己覺得非常重要.
我有兩個我自己很佩服的超人.
(你指超人,Batman那些?).
對,一位就是沈祖堯校長.
他今天在這裡.
另外一位就是蜘蛛俠.
蜘蛛俠說了一句很重要的東西.
但其實不是他說的.
是他叔叔告訴他的.
就是With great power comes great responsibility.
就是能力越大,責任越大.
我們作為基督徒在這個社會裡.
我們背負著一個很重要的責任.
就是耶穌要我們作炎作光.
所以我們在社會裡要做一些這樣的事.
我們要別人認識我們為耶穌的門徒.
更加要求這個大的智慧.
所以Stay wise就是這樣的解法.
這就是我對Steve Jobs的演講.
另外一套我自己的想法.
一個人生的哲理.
(不錯,原來你真的有準備).
還可以的.
(但你這裡只有幾個重點).
(你怎麼說這麼多?).
對,其實我也是想帶回我們今天.
我們想準備的這兩首歌跟大家分享.
這一首不如讓你介紹一下.
好,我們在2012年已經寫了這首歌.
但這首歌是2014年5月才出品的.
就是《公諸於世》.

$^{161}$這首歌叫《時代的顛覆者》.
當時出這首歌的時候.
我們也沒有想過會有這麼大的迴響.
在坊間很多人.
不是騙讚.
我也沒有想過會有這麼多讚.
Henry和我寫的時候.
真的將我們覺得香港的一些.
我們看到的一些事情的一些改變.
我們甚至覺得很悲涼的一些事情.
寫在這首歌裡面.
其實只是很單純的一種想法.
就是希望跟香港人分享這首歌.
不知道大家聽這首歌會想到什麼.
我們自己當時也沒有什麼答案.
答案可能就是.
就像Bob Dylan說的.
"Blowing in the wind".
真的未必知道.
但原來出了來的時候.
這首歌原來安慰了很多人.
因為原來很多人也有這種想法.
而只是他可能覺得.
終於有人肯將這件事寫在一件藝術品裡面.
音樂本身其實是一種藝術品.
我會覺得.
他有那種共鳴.
原來這件事也可以撫摸到他的心靈.
可能他覺得受了傷的地方.
那麼這首歌.
今天也很想在這裡跟大家分享.
我覺得作為.
很多在在.
相信都是基督徒.
我們也在想這個問題.
究竟香港走到今時今日.
我們可以怎樣呢?.
首先我們不如先聽這首歌.
我們回顧一下.
其實香港這幾十年發生了什麼事.

$^{201}$之後我們再探討一下.
可以怎樣好不好.
《時代的顛覆者》送給大家.
謝謝.
《時代的顛覆者》 詞:陳家麗 曲:陳志遠.
村莊給清拆了 海港都縮窄了.
冷氣的商舖數目會增減.
這裡越繁榮 錢財越覺重要.
私家車的廢棄 對積蓄的怨氣.
到處都污染 只沒法透氣.
政客厚面皮 橫蠻沒有道理.
只是想壞你 又想講真理.
亦當初赤子心一方.
今天的社會慶漢安.
我理性分析 你卻話偏頗.
奮勇去抗爭 想不到換來自現實的煎熬.
如何能逃離這腐朽制度.
講不出的控訴 不聽媽媽勸告.
再次想解決你去宣告.
政策有問題 前途沒法預告.
我也不斷尋求 但始終找不到.
始終找不到.
(演奏).
亦當初赤子心一方.
今天的社會慶漢安.
我理性分析 你卻話偏頗.
奮勇去抗爭 想不到換來自現實的煎熬.
如何能逃離這腐朽制度.
啊.
思想給清潔了 顛覆這不見了.
街邊的黑蓋 暗地裡苦笑.
看透世途 文明是個玩笑.
這裡安定和諧 但聲音消失了.
(演奏).
《時代的顛覆者》.
這句《看透世途》文明是個玩笑.
這裡安定和諧 但聲音消失了.
我想可能我們經常都會有些myth.
中文不太好 迷思.
我覺得很多東西都是迷思.

$^{241}$究竟安定和諧是否都可以是迷思呢?.
和諧這個字又可以怎樣去解讀呢?.
和諧會不會是一百萬人說同一樣東西.
還是一百萬人說不同的東西.
但原來可以和而不同呢?.
在音樂裡面有一個概念叫和弦.
和諧和和弦的英文是同一個字.
叫harmony.
harmony就不是unison.
unison就是讀同一個音.
harmony就可以是五個音 六個音.
甚至十個音 不同的音.
甚至可以是撞音.
不過都可以是好聽的.
彩虹都有七色.
當我們活在這樣的社會裡面.
我們經常高舉和諧.
但原來有些聲音我們不想聽的時候.
這個又是否真正的和諧呢?.
我想這首歌其實想帶出的是這些東西.
而之後可能有很多question mark.
那又如何呢?.
我們可以怎樣呢?.
話說為何我們會作這首歌呢?.
就是有一個機構.
前年找我們作一個他們40周年的主題曲.
這個機構實不相瞞就是突破機構.
蔡元文醫生和梁永泰.
當時是總幹事.
和他們的同工找我們.
說不如你作一首歌.
叫《時代的顛覆者》.
我們說原來你想作一些突破的東西.
我們就作了剛才的那首歌給他們.
就是這個版本.
他們收到歌詞和音樂的時候.
他們就說.
中士兄弟我們不要這首歌.
為什麼呢?.
因為如果大合唱的時候.

$^{281}$唱不到.
我說也是的.
很悲.
有點悲涼.
因為這首歌是我們40周年的主題曲.
我說那怎麼辦呢?.
我們就想了一條計策.
因為《時代的顛覆者》最後是一個question mark.
我們覺得這首歌還未完.
不如我們給它一個續集.
變成現在我們會和你們唱的這首歌.
這首歌就是.
我們作了一首叫《Reimagine》.
給突破機構重新想像.
這首歌的內容其實就是.
《時代的顛覆者》的延續和回應.
因為《時代的顛覆者》是說香港多年的變化.
看透了世途,但文明是個玩笑.
作為一個基督徒或年輕人.
我們如何回應這個時代和世界.
重新想像.
不可以說是一個答案.
但可以給一些建議.
如何創造一個新的風格.
如何用一個不怕艱難的精神.
去面對我們看不到的未來.
和突破.
他們是serve很多青年人.
他們跟我們說.
當你作這首歌的時候.
很希望這首歌可以鼓勵到青年人.
我們小時候.
有一首歌很能鼓勵我們.
有青年的字在歌詞中.
我經常在想.
剛才問的問題很大.
我們作為青年人如何回應.
神要我們如何回應.
我覺得這首歌也能給一些答案.
你說哪一首歌呢?.

$^{321}$就是這首歌《青年向上歌》.
(笑).
《青年向上歌》.
但它的英文名字不是《青年向上歌》.
一會兒我會說英文.
你先說中文吧.
《青年向上歌》.
這首歌是怎樣說的呢?.
每一次我聽和在教會唱.
我不是輸我直言.
我平時去不同的教會.
他們唱最後之前.
其實我也不是很投入.
但我一去到《青年向上歌》.
你就投入了.
我簡直覺得.
嘩!什麼事呢?.
好像上帝在和我說話一樣.
因為你喜歡唱一些不對音的詞.
會不會呢?.
一定是.
就是我要真誠.
(笑).
就是我要真誠.
莫負人家信音深.
其實也不知道他在說什麼.
其實這首歌我小學時已經唱.
我也是.
我讀的是珍光小學.
有男生的.
不只是女生.
(笑).
我們有一本叫珍光歌集.
那時候我已經被這首歌打動了.
除了不合音之外.
我覺得他說得不錯.
我要真誠.
莫負人家信任深.
我要潔淨.
因為有人關心.

$^{361}$我要剛強.
人間痛苦才能當.
我要膽壯.
奮鬥才能得勝.
我要愛人.
愛敵也愛淪落人.
我要師爭.
心誠義重才興.
我要虛懷.
不忘我身多弱點.
我要向上學主榜樣助人.
我呢.
就是經常在想一件事.
究竟為什麼這首歌叫青年向上過.
(笑).
那我.
終於找到答案了.
對.
其實是兩個星期前.
我才發現這個答案.
因為你看了英文的歌詞.
沒錯.
我自己本身也是中文大學任教的.
我是崇基學院那裡.
不是崇基神學院.
我屬於昌志學院.
院長陳偉光院長.
他在我們開學禮的時候.
分享了青年向上歌的英文版.
那一次我很少唱歌.
尤其是唱聖詩.
會唱到哭.
這首歌是會的.
原來這首歌叫《I would be true》.
嗯.
它的歌詞其實是一首詩.
是一位H.A.Walter的人寫的.
寫給他媽媽的.
好像一種宣言.
他那時候快要.

$^{401}$因為他患上了一種病.
他知道他快要去世.
他用這首歌來紀念自己的生命.
是這樣的英文.
他說.
I would be true.
For there are those who trust me.
我覺得這個位置已經感動了.
因為我要去真誠.
不是因為我自我中心.
不是因為我覺得自己很帥.
因為有人原來是信任我的.
I would be pure.
For there are those who care.
有人關心我.
所以我要潔淨.
或者這個字.
其實翻譯出來做中文.
也奇奇怪怪的.
潔淨.
怎樣潔淨呢.
I would be pure.
很真誠.
也是在說一種.
信真.
信真赤子.
I would be strong.
For there is much to suffer.
這一句也是很奇怪.
翻譯成了人間痛苦才能當.
就好像.
有些consequentialist.
我覺得.
這個字有少少功能化了.
其實就是說.
因為這個世界.
或者我們自己的生命裡面.
其實有很多我們要去擔當的重擔.
那種suffer.
未必是一些.

$^{441}$你被人斬.
或者是.
有意外發生.
未必是肉體.
可能是心靈.
可能是.
原來有一些人.
在這個社會裡面.
他是被主流的社會藐視的.
這些人.
會有很多suffering.
我們可能.
作為主流的一群人的時候.
我們未必看到.
但是原來聖經是要我們去看見這些人.
看見這些人.
要覺得他們是.
有需要去幫助他們.
And I would be brave.
but there is much to dare.
德勝這個字.
我又覺得有時候.
中文的譯本.
變成好像要贏.
其實未必是要贏的.
對於我這些中文一般般的人.
我覺得譯得不錯.
I would be friend of the foe and friendless.
我覺得這個是非常難做到.
怎樣做到敵人和一些沒有朋友的人.
通常我們見到一些沒有朋友的人.
都覺得這個人怪怪的.
我覺得是遠離他比較好.
或者你的敵人.
我們怎樣去愛他們.
我們怎樣去做他們的朋友.
I would be giving and forget the gift.
要施捨.
或者不要用施捨這個字.
好像很高的字.

$^{481}$不是.
可能是share.
And I would be humble.
for I know my weakness.
我覺得這個是很觸動到我的一句.
因為我覺得.
看完整本聖經.
基督教有一樣東西很特別的.
我覺得在其他的哲學也好.
或者宗教也好.
比較少去講這一樣東西.
尤其是很多的哲學.
或者中國的哲學可能會講道德.
道德是教我們.
做這件事對.
做那件事不一定對.
那件事是很normative的.
但基督教講人性的本相是軟弱.
我覺得這件事是很重要.
它叫我們去humble.
去謙虛.
去謙卑.
即是說.
剛才也有講到一個很重要的訊息.
就是我們不可以自我中心.
甚至覺得自己未必是對的.
就算去到今時今日.
很多時候在學術也好.
或者在神學也好.
其實也有很多爭論.
但我覺得應該要抱一個attitude.
當我們去爭論的時候.
我們通常都會覺得自己的論點是最對的.
這個是正常的.
但可能.
我經常都是這樣覺得.
但還有一層的.
還有一個層次就是.
我可以是錯的.
我錯的時候.

$^{521}$是可以勇於去承認錯誤.
因為我們實在知道不夠多.
沒錯.
而這一句最後.
我想是sum up了基督教的一個理念.
一個情操.
終於我想回答.
為什麼叫做青年向上歌.
I would look up and laugh and love and lift.
嘩!我看到這一句的時候.
原來他說.
我會向上望.
只是不是向上望那麼簡單.
他會笑.
因為他覺得這個生命值得我去活出來.
我會愛.
因為最大的誡命是愛人如己.
其實很難做到的.
我知道這件事.
當你可以做到這件事的時候.
是一個星華.
英文有一句字叫做transcendence.
就是一種星華.
青年向上歌這首歌.
也感動了我們很多人.
雖然唱起來的時候不合音.
但其實唱英文這首歌是很好聽的.
你知道嗎?.
我知道.
不過.
哈哈.
那我有點感動.
不如唱一小段給大家聽.
不如大家一起唱吧.
英文.
英文就你唱吧.
哈哈.
我想告訴大家.
這首歌大家可以多點拿來唱.
不如給個音來.

$^{561}$你想用什麼key.
好啊.
I would be true.
For there are those who trust me.
And I would be pure.
For there are those who care.
I would be strong.
For there is much to suffer.
I would be brave.
For there is much to dare.
I would look up.
And laugh and love and live.
送給大家青年向上歌.
I would be true.
其實沒有準備過.
所以就變成這樣.
我們準備了另外一首歌.
這首歌是比較激動.
比較激勵的歌.
這首歌叫Reimagine.
重新想像.
香港未來可以怎樣.
我們可不可以用不同的角度去想.
會不會可以創一些新的東西出來.
會不會主流的東西.
未必是合時宜的呢.
我們可能要去顛覆.
顛覆什麼呢.
可能顛覆我們自己的想像.
這首歌送給在座每一位.
重新想像 Reimagine.
Thank you.
迷失 活在地獄過一餘年.
末日就將故事預言 城邦已死.
尋覓 默默望著價值侵倫.
靜靜地讓信念埋藏 仍不作聲.
扭曲思想 充斥社會威脅和.
分開變作抑壓日子都坎坷.
不安的時代殘酷地面向我.
這世漂泊歲月裡.

$^{601}$不知怎過.
如果 患難路上有你在旁.
一起分擔悲傷失意.
同街群友 能否伸出雙手多點關注.
偉動盪世界我願能尋找變改.
儘管身邊許多聲音嘲笑我.
我已決意衝破內心的枷鎖.
聽聽這時代無奈的吶喊過.
這世漂泊歲月裡 許多挑戰要面對.
今天開始不管得失 奮勇不息.
齊和現實對立.
再相著 創出新風格路向.
不怕輸 再起步 為了找緊這夢想.
(音樂).
儘管身邊許多聲音打壓我.
我已決意衝破內心的枷鎖.
聽聽這時代無奈的吶喊過.
這世漂泊歲月裡 許多挑戰要面對.
今天開始不管得失 奮勇不息.
齊和現實對立.
我要努力以負去面對.
即使遭壓迫會後退.
誰要找緊這夢想.
再次進入窄路裡.
磨練我鬥志仍不急退.
哪怕世事會絕對.
Re-imagine Re-imagine a whole new world.
誰要跨過這路障.
跟拍子 再起步.
去隔阻防釋疆.
(音樂).
送給大家的重新想像 Re-imagine.
希望我們都可以重新想像香港未來的路向.
活出精彩的人生.
(MC) 終於突破了 要了這首歌.
多謝各位 再見.
(掌聲).
\newpage



\section{}
\label{sec:DregqwIedRk}
\textbf{中神40周年院慶培靈會:陳關韻韶老師}
\newline
\newline
連結: \href{https://youtube.com/watch?v=DregqwIedRk}{\texttt{ https://youtube.com/watch?v=DregqwIedRk}} ~~~~ 語音日期: 2015-10-14 
\newline
\newline
\hyperref[sec:g45zbSUTL0c]{\small{< < < PREV SERMON < < <}}
~
\hyperref[sec:index]{\small{[返主目錄]}}
~
\hyperref[sec:yNpGxFfqF_k]{\small{> > > NEXT SERMON > > >}}
\newline
\newline
$^{1}$不知道大家聽到這裡.
收不收到上帝想你未來二十年做些什麼呢?.
沒錯,我們的信仰是要用行動實踐出來的.
不過當你去行動之前.
我很想給一個很重要的邀請大家.
就是這裡的邀請.
來看,come and see.
他將一切都更新了.
其實這個邀請是上帝發給你的.
這張卡已經發給你了.
你有沒有上他facebook查一下.
在他的timeline啟示錄.
21章5節那裡發的.
看啊,我將一切都更新了.
這裡是啟示錄裡描述新天新地.
是啊,說新天新地.
知不知道新天新地和這個立方體有什麼關係?.
你回去查啟示錄.
新成耶路撒冷的尺寸是什麼形狀?.
是一個立方體.
這個new heaven, new earth.
有很多信徒覺得是將來的事.
和今天世界議題無關.
什麼叫做傳福音?.
就是叫人信耶穌.
將來有永生的盼望.
所以我們今天要做好我們的見證.
吸引多些人信耶穌.
將來就可以多些人一起上天堂.
至於什麼政治,社關,扶貧,共融,倡議.
或者環保那些.
就留給別人去做吧.
為什麼?因為這個世界都要結束的.
是不是只有這樣呢?.
我邀請大家看清楚上帝post的東西.
就在同一段,說新天新地的來臨的時候.
21章,三節.
那裡說,神的帳幕在人間.
神的帳幕在人間,祂和我們同住從何時開始?.
記不記得馬太福音那裡說.

$^{41}$哪有同女懷孕生子.
人要給他起命叫二媽來.
Emmanuel.
神和我們同在,神的帳幕在人間.
新天新地已經來了,但是還沒到.
Already, but not yet.
已經全速地展開了,神的更新工作.
在我們的生命裡,和我們息息相關.
應該是見得到,摸得到.
因為耶穌怎麼跟我們說呢?.
他說,父怎麼猜我來,我也照樣猜來.
如果我們忘記了這個世界的話.
不理這個世界的話.
其實就是漠視了神猜我們進入世界服侍這個命令.
但問題出在這裡.
一進到世界的場景.
不同的陣營,或者相反意見的信徒.
都很可以拿到聖經來支持他們的想法.
每個人都說他們的理想就是天國體現在人間的理想.
他們的做法就是最合乎真理的做法.
我們何去何循呢?.
有時候我們就躲起來.
做一些不太激烈的事,不要有太多爭拗.
我們很想和諧.
是不是這種和諧是我們要的呢?.
看啊,我將一切都更新了.
看啊,神的帳幕在人間.
我相信我們要調教出一種雙視野.
這個雙視野讓我們可以看到這兩件事.
這個雙視野既有終末性,也有現時性的一種視覺.
那怎樣可以調教這些視覺呢?.
那我就請教眼科醫生吧.
他就說,其實人的眼不可以同時遠視和近視.
而且我們眼睛不可以同時對焦兩個距離.
不可以同時遠和近.
就算讓你遠又可以,你試想像一下.
遠又可以,近又可以.
雙視野的時候你真的會很暈.
結果兩邊都看不清楚,那還怎麼做事呢?.
我們其實這些活在已經有的,但不一樣的.

$^{81}$其實是不是真的覺得自己可以看得清呢?.
保羅怎麼說我們現在看東西呢?.
我們好像對著一塊蒙茶茶的鏡子那樣看.
猜猜而已,好像猜謎語那樣.
我不知道你喜不喜歡猜謎語.
不過如果你知道自己看得不太清楚的話.
或者你可以對自己和身邊的人寬容一點.
不過雖然說是無不清,但也有很多提示的.
就好像猜謎語有提示那樣.
所以今天我要跟大家分享四個提示.
給大家參考一下.
用來做什麼呢?.
就是幫你調教一種雙視野.
去回應上帝的邀請去看他在做什麼.
這個雙視野可能會令你暈暈的.
不過暈暈的也比盲中中好.
四個提示用四個英文字母去說.
第一個英文字母T,T for transformation.
它將一切都更新了.
我挑戰大家對這個transformation.
有一個超然的期待.
因為是我們眼未曾見過.
心未曾想過的一種transformation.
是在說一個metamorphosis,蛻變的過程.
你或者想像不到結果會是怎樣的.
出來的過程可能會令你有不安的.
當日的法尼塞人.
他們一直都在期盼尼塞的來臨.
但是他們對於尼塞亞走出來應該是什麼樣子.
用什麼方法去救贖呢.
有一套預設的框架.
所以他們怎樣看都認不出.
耶穌就是那位將要來的位.
如果說差謎語他們就猜錯了.
還錯得很慘.
我們今天又怎樣呢.
我們今天去實行我們福音的召命.
天國的使命.
我們有沒有一些框架框住我們.
令我們認不出神所做的更新的工作呢.

$^{121}$要記住.
a butterfly is a transformation.
not just a better caterpillar.
你心目中那條漂亮的毛毛蟲.
可能是一個公平的社會.
一個資源分配得均勻的制度.
一個人和大自然和諧共處的宇宙.
所以當你內心對於信仰的承擔.
要求你站出來為公義發聲的話.
那你就別無選擇.
你只可以全力以赴.
不過你要知道.
毛毛蟲的蛻變.
時間表在上主的手中.
而當蛻變發生的時候.
那種漂亮又何止是一條漂亮的毛毛蟲.
更加不止我們可以想像得出.
那種公義,仁愛和和平.
又或者你心目中那條漂亮的毛毛蟲.
是教會興旺,信徒人數倍增.
所以當你內心對信仰的承擔.
要求你走出你的安書區.
加入牧養和宣教的行列的時候.
那你也別無選擇.
你也只可以全力以赴.
哪怕要你上山下鄉.
但是你要知道.
毛毛蟲轉化蛻變的時間表.
在上主的手中.
而當蛻變發生的時候.
那種漂亮又何止是一條漂亮的毛毛蟲.
也遠遠超越我們心目中所想像的那種救贖.
那種敬拜.
Transformation.
New Expectation.
這個提示要求我們全力以赴.
但另一方面又要少一點執著.
多一點想像力.
Reimagine.
第二個提示是S字頭的.

$^{161}$S for Scripture.
大家都說自己跟聖經.
但是當立場不同的時候.
大家都會覺得自己對.
剛才他們也confess了.
我也一樣confess.
想大家重新思考一下.
其實我們自己以為.
這裡是展聖經的意思.
我們經常說別人斷章取義.
說別人只是看了一部分聖經.
你還沒看完.
我看的那部分你有沒有看.
我們經常以為自己明白那些部分.
是不是真的明白呢.
我認識一個小朋友.
他很喜歡唱獅子山下.
簡直是他的飲歌.
去到哪裡唱到哪裡.
但是他靠聽來學.
他不會看歌詞.
我問他.
其實你明不明白那首歌是說什麼的.
他說明.
人生總有驚喜.
就是說做人.
有時會害怕.
有時會開心.
這個就是他對驚喜的解讀.
第二句更精彩.
難免亦常有淚.
就是說吃辣麵的時候.
吃很辣的麵的時候.
會流眼淚.
對他來說很合理.
小朋友只可以從自己的經歷.
去理解他所聽的東西.
其實我們看聖經也是這樣.
所以如果我們有新的經歷.
新的經驗.

$^{201}$我們會有新的insight.
挑戰大家.
看聖經的時候.
我們期待上主給我們有新的insight.
又或者如果不同立場的人.
有多些耐性.
可以聽一下對方.
他的經歷.
如何去塑造他對聖經的解讀的時候.
我們會從中得到更多的new insight.
於是我們認得出.
上帝所成就的那種更新.
先在耶路撒冷.
長老和士徒們.
他們就這樣聽保羅.
去描述外邦人怎樣領受聖靈.
作為上帝接納他們的憑證.
耶路撒冷的長老和士徒們.
他們就已決不用再強迫外邦的信徒.
去跟隨.
好像猶太信徒那樣.
跟隨他們去守律法.
摩西的律法和行國禮.
這個決定.
你想想當時其實是很破格的.
一點都不簡單.
可以去到蛻變的層面.
就是因為他們肯聽.
scripture new insight.
這個提示提醒我們.
要有合乎聖經的double vision.
不要給自己有限的理解去框住自己.
要對神所做的新事有開放的心.
讓聖靈去帶我們去認識那位.
在聖經裡面啟示自己的神.
第三個提示.
G for globalization.
借用社會學家Roland Robertson.
創作的這個字.
去描述一下神的國度裡面.

$^{241}$的一個global new race.
你有沒有留意到聖經怎樣形容.
在新天新地裡面的新人類.
在啟示六七章裡面說.
我見到很多人.
多到數不到那麼多.
他們是從各國各族各民各方來的.
坐在寶座和高陽面前.
身穿白衣手拿中樹枝.
大聲地叫喊.
好像喊口號那樣說.
願救恩歸於座.
在寶座上我們的神也歸於高陽.
這班人認得出.
他們是從各國各族各民各方來的.
原來新天地的新人類.
是沒有被去種族化的.
上一幕他們還是民攻打緊民.
國攻打緊國.
下一幕.
一起身穿白衣手拿中樹枝.
同一個口號唱同一首新歌.
舊事已過了.
但是大家的不同仍然認得出來.
都變成新的了.
在基督裡面.
這個兩下合而為一的合一.
不是globalization.
直到新天新地神的子民.
都是一個global new race.
所以我們要think globally.
act locally.
這種視角對我們今天面對多樣性非常重要.
我相信我們要擁抱的多樣化.
不單止是關乎種族和文化.
也都關乎所有將我們隔開.
使我們鬥爭的意識形態.
或者是身份認同.
到時在新天新地的時候.
藍營綠營.

$^{281}$鷹派甲派.
黃絲帶藍絲帶.
什麼顏色都好.
黑色白色甚至粉紅色.
都仍然認得出來.
但是舊時已過.
在基督裡面這個多元而又合一的new race.
要同唱一首新歌.
今天教會又如何呢.
教會可否表徵到這種.
這樣的diversity.
又有unity.
我們如何面對我們現在的分歧呢.
或者我們可以向阿富汗人學習如何吵架.
他們吵架是很激烈的.
你見到他們以為下一刻會動手.
但是你有辦法看得出他們會不會出事.
教你看.
如果大家意見很不同.
但如果他們握著手來吵架.
很高難度.
你試下.
握著手來吵架.
即是他們仍然覺得大家都是自己人.
那就不會出事.
我覺得很有趣.
我覺得很有意思.
完全表達到那種double vision attention.
既撕裂但又聯合.
我們在今天這個年代.
我們考驗我們的眼力.
這個global new race.
我們能否認出眼前這個人.
這個完全不行的他者.
看不看得出按照上帝的形象.
做的這個人是自己人.
又或者.
這條臭毛蟲.
其實在神的手中.
有一天會和我一起蛻變成為蝴蝶.

$^{321}$最後一個提示.
C.
這四個英文字母其實很容易記.
TSGC.
你將CGSC將忠臣倒轉一句.
你就會記得.
C for Christ.
基督雖然是神的兒子.
但他也從苦難學了信服.
為我們開了一條新路.
a new way.
是愛忠信服社稷的道路.
耶穌在十字架上說成了.
其實他在向誰交代成了呢?.
其實我們身邊有很多聲音.
經常催迫我們.
你行了嗎?.
你搞定了嗎?.
鐘又在催迫我.
你講完了嗎?.
你做完功課了嗎?.
快點吧.
找到工作了嗎?.
你的計劃做好了嗎?.
存夠錢買房子了嗎?.
很不容易走到結婚.
很快又被人追.
有計劃生孩子了嗎?.
如果你一個不小心.
你未來二十年.
你也知道會怎樣.
就是不停地為了對這些要求有交代.
而耗盡你的青春去追deadline.
耶穌呢?.
當我們的成了是向老師說的.
老師成了.
媽媽成了.
老闆成了.
耶穌只有一個目標.
他要為那位愛他的天父去工作.

$^{361}$向他交差.
耶穌拒絕去行身邊的人.
逼他走道路.
快來坐人.
叫他收聲.
他就不收.
叫他為你顯個神跡來看看.
他說坐在這裡浪費力氣.
叫他你走出來.
做個判斷.
說說究竟這些人有沒有罪.
他就蹲下來畫公仔.
然後他見到權貴去奉獻.
他不單不讚他一句.
還要串他兩句.
身邊的人叫他上馬.
做人民領袖大力革命.
他就索性躲起來避一避.
世界人看他走的路很飄忽.
很不make sense.
不知道他想怎樣.
但其實耶穌很清楚自己想怎樣.
他很consistent.
他心裡面對神的愛.
引伸至他對天父的兒女的愛.
這種愛一直帶動他的喜怒哀樂.
甚至他的生死抉擇.
在人的角度看.
在十字架上耶穌生命.
凡凡都交不到功課.
但他可以無憾地面對生命講成了.
今天我們也有自己要背負的十字架.
耶穌為我們開了這條.
在愛中馴服的新路.
以致我們在地上.
在一些覺得很困迫的.
很被動的.
很失效的場面裡.
我們可以從容不迫地.
焦點面對將一切更新的神.

$^{401}$看神的裝模在人間.
聖經裡有另外一個很重要的立方體.
你知道在哪嗎.
尺寸是舊約的.
可能有些人猜到了.
就是致勝所.
Holy of Holies.
當耶穌說成了.
他將自己的靈魂交在天父的手中.
他走完他馴服的路程.
發生什麼事呢.
聖殿裡的萬子.
由上至下列為兩半.
上帝為我們開了一條新路.
進入這個奇妙的立方體.
神與人同在.
這個致勝所.
原來這條新路是相承的.
是互相呼應的.
啟示錄裡.
坐在寶座的那位說.
看啊,我將一切都更新了.
接著他在下一節說.
都成了.
成了,都成了.
相承的新路互相呼應.
成就在地,成就在天.
弟子們,當我們在地上跟從基督.
走這條愛忠,捨己馴服的道路的時候.
就等於用我們的生命去念主禱文.
願你的旨意成就在地.
如同成就在天.
其實,日光之下無新事.
我們今天如何去回應時代呢?.
都不會是前無古人,後無來者.
但是上帝邀請我們這些.
活在already but not yet的人.
在一些見得到摸得到.
但一定會摺在一起的一些東西上.
全力以赴.

$^{441}$以示我們學去看.
他在做的新事.
學去用這個double vision.
發現他為人類開的新路.
來看,他將一切都更新了.
這四個提示是我在使命實踐的過程中.
在很多的撞了個板.
很多的困壁中.
去得到的一些禮物.
今天跟大家分享.
相信有一天你會用得著.
謝謝大家.
(掌聲).
(字幕製作:貝爾).
\newpage



\section{}
\label{sec:yNpGxFfqF_k}
\textbf{中神40周年院慶感恩暨開學崇拜(證道:榮休院長余達心牧師)}
\newline
\newline
連結: \href{https://youtube.com/watch?v=yNpGxFfqF_k}{\texttt{ https://youtube.com/watch?v=yNpGxFfqF\_k}} ~~~~ 語音日期: 2015-09-16 
\newline
\newline
\hyperref[sec:DregqwIedRk]{\small{< < < PREV SERMON < < <}}
~
\hyperref[sec:index]{\small{[返主目錄]}}
~
\hyperref[sec:RU4oBv0Wasg]{\small{> > > NEXT SERMON > > >}}
\newline
\newline
$^{1}$(主席)我1974年加入忠臣,加入忠臣的時候,剛才跟珍樂華牧師說,我都是一個小孩子,1975年開校的時候,我26歲,是最年輕的一個..
一眨眼就過了40年,在這40年當中,我們看到神真的很祝福我們,而其中我們可以說,我們一班老師,一班同工,和我們一班同學,我們都是一班很願意將自己奉獻給神,.
給神使用一班這樣的人.我們在這裡慶祝忠臣的40週年,除了感恩,我們更加要向上帝仰望,願祂加倍感動忠臣新一代的領袖們,同工們,勇敢的迎向時代的挑戰,.
秉持意象,向更廣闊的領域進發.我今晚站在這裡是代表忠臣第一代意象的秉持者,Vision Bearers,我今晚和周牧師,自覺我們第一代退場的人,.
我們對承接忠臣侍奉的同工們,滿有期盼,今晚我在這裡說出那個期盼,所以很大膽地用烈王記下第二章的一句話,將它稍微改變,就是願感動我們的靈,加倍的感動你們..
請聽我讀出今晚我要和大家分享的一段經文..
烈王記下第二章,第一節至第十五節..
耶和華要用旋風接以利亞升天的時候,以利亞與以利沙從吉格前往..
以利亞對以利沙說,耶和華猜我往伯特利去,你可以在這裡等候..
以利沙說,我指著永生的耶和華,又敢在你面前起誓,我必不離開你..
於是二人下到伯特利,駐伯特利先知的門徒出來見以利沙,對他說,耶和華今日要接你的師父離開你,你知道不知道?他說,我知道,你們不要作聲..
以利亞對以利沙說,耶和華猜我往耶利哥去,你可以在這裡等候..
以利沙說,我指著永生的耶和華,又敢在你面前起誓,我必不離開你..
於是二人到了耶利哥,駐耶利哥的先知門徒就近以利沙,對他說,耶和華今日要接你的師父離開你,你知道不知道?他說,我知道,你們不要作聲..
以利亞對以利沙說,耶和華猜我往約旦河去,你可以在這裡等候..
以利沙說,我指著永生的耶和華,又敢在你面前起誓,我必不離開你..
於是二人一同前往,有先知門徒去了五十人,遠遠地站在他們對面..
二人在約旦河邊站住,以利亞將自己的外衣捲起來,用意打水,水就左右分開,二人走乾地而過..
過去以後,以利亞對以利沙說,我未曾被接去離開你,你要我為你做什麼,只管求我..
以利沙說,願感動你的靈,加倍地感動我..
以利亞說,你所求的難得..
雖然如此,我被接去離開你的時候,你若看見我就必得著,不然必得不著了..
他們正走著說話,忽然有火車火馬將二人分開,以利亞就乘船風升天去了..
以利沙看見就呼叫說,我父啊,我父,以色列的戰車馬兵啊,以後不再見他了..
於是以利沙把自己的衣服撕為兩片,他拾起以利亞身上掉下來的外衣,回到站在約旦河邊..
他用以利亞身上掉下來的外衣打水,說,耶路亞的神在哪裡呢?.
打水之後,水也左右分開,以利亞就過去了..
駐耶利哥的先知們徒從對面看見他,就說,感動以利亞的靈,感動了以利沙..
他們就來迎接他,在他面前俯伏於地..
作為忠臣第一代的同工,我們稱我們自己是異象的秉持者,vision bearer..
因為忠臣的創立,如果不是出於從天上來的異象,基本上是不會有今天的忠臣..
今天的忠臣是從異象與異象開拓出來的,因此思考,成全,感動,自然我們就要問我們成全什麼,我們盼望什麼的感動..
獵王記上下有關與利亞和以利沙的侍奉,以及他們使命的傳與承,特別是獵王記下第二章一至十五節,這一段講述以利沙承接以利亞先知積分的那段經文,可以給我們有不少的亮光..
獵王記上向我們展示一個驚心動魄的時代,一個極壞的時代,也可以說是一個極好的時代..
他是一個極壞的時代,是一個政治混亂,信仰破產,道德敗壞的時代,但同時也是一個極好的時代,因為先知運動在這個時候開展,改變了以色列日後的宗教面貌..
極壞的時代,我們看一看,政治混亂,我們看到一幕又一幕叛變篡位,王朝的家族被血洗,這些一幕一幕向我們展開..
信仰破產,巴力異教橫行,供奉亞瑟拉的墓柱遍佈境內,阿哈謝命危,他從閣樓掉下來,命危,他要求問巴力西卜,卻不去求問耶和華..
巴力先知數以百計,道德敗壞,你可以說是敗壞到極處,阿哈擁有一切,但他竟然垂涎那帕的葡萄園,為了得到葡萄園,他不惜誣陷那帕,說他褻瀆耶和華,.
將他用石頭打死,然後再取他的葡萄園,這是一個極好的時代,以尼亞,以尼沙帶動先知的運動,改變了以色列屬靈的面貌,.
從此一個又一個中心的宣講神話語的先知,前赴後繼,就好像阿摩斯,荷西亞,耶利米,以塞瓦,耶利米,.

$^{41}$他們都是被上帝說話,向他們說話,先知是誰呢?先知是the one spoken to,是被上帝向他們說話的那些人,.
他們被上帝向他們說話,然後宣講,他們這班人成了更新以色列文的一種新的動力,以尼亞之前難道沒有先知嗎?有,摩西就是先知的典範,.
但是當先知變成了建制的一員,當他們的使命變成了職能,他們就成為以色列屬靈衰落的催化者,在以尼亞出現之前,以色列當然有先知,.
但是他們在哪裡呢?當耶駛別追殺他們的時候,他們因為要躲避耶駛別的追殺,失縮地藏在洞穴裡面,.
單單是俄巴底一個這樣的人,就收藏了一百個先知在兩個洞穴裡面,還有一些像俄巴底這樣的人收藏這些先知,.
當耶駛別狂妄地推行異教的時候,這班人是噤若寒蟬,當巴力的先知肆虐,他們沒有為真理挺身而出,.
當以色列所立的國度與別國無異,具有效的管治機制,機器和軍事裝備,但是卻失去了他們的靈魂,.
失落了選民的歷史使命,失卻了方向,失卻了屬靈的動力,簡而言之就是失卻了意象,.
民無意象就會放肆,因而步向滅亡,這個正是以色列當時的處境,.
德國一位舊約學者Victor Mack寫了一篇很有趣的文章叫做Malcute Yahweh,The Kingship of Yahweh,.
這篇文章裡面指出,以色列宗教獨特的地方在哪裡呢?.
獨特的地方就是他們的vectorial kinetic character,他們有一個特性,這個特性就是有方向的動力,.
即是他們有強烈的歷史的動向感,以色列人的身份就是在他們自覺,.
他們不斷向著上帝所指定的那個未來來進發,becoming,他們的生命是在becoming當中,.
就是上帝為他們,也是為全人類所定的目標,他們向著那個方向來進發,.
因此他們進入了迦南,並不以此為最終極的目的,.
他們的角度觀和別國的角度觀是完全不同的,.
但可惜當時的以色列竟然是和別國看齊,立了王,建立了管治的機制,.
卻因此被困在其中,革命的動力消亡,那個vectorial kinetic character也因為這個緣故而失落,.
他們忘卻了上帝交付給他們的歷史使命和他們要走的方向,.
他們就停下來,止住在那裡,Victor Meek用一個字來形容他們當時的光景,.
就是sedentarization,他們這班人是sedentarized,停在那裡,.
他們固守著自己的小千世界,小小的天地,並且以之為他們最終極的目標,.
Victor Meek說,sedentarization是以色列人最嚴重的過犯,.
一切的建制就如王權,如果不是為了遠大的歷史的任命,.
這些建制,這些權位,這些一切所有的成就都會是阻礙上帝的工作,.
當以色列人將眼目放在維護,鞏固既有的國土和疆界,既有的成就,.
這個就是他們衰敗的時候,就是在這個衰敗的處境下,.
突然間,毫無預警的情況下,就好像平地一星雷一樣,.
以利亞和以利沙出現,並且佔據了烈王記上下相當重要的位置,.
他們兩個人都有一個共同的特點,就是他們願意將他們的生命完全獻上,.
完全聽命於上帝,任由他來猜險,其實在烈王記上下,.
聽命是一個非常重要的主題,或者不聽命是一個非常重要的主題,.
不聽命是可以有很嚴重的後果的,神要以利亞起來,往哪裡去?.
以利亞就往哪裡去?要他做甚麼?做一些令人覺得他古怪的事,他就去做,.
神要他躲起來,他就躲起來,神要他出來挑戰有王家參要的巴力,.
巴力先知就出來,從剪除巴力先知,以重現神的榮耀,到他逃亡,到他尋死,.
他都只知道一件事,就是他自己為耶和華大發熱心,甚至可以將他的生命豁出去,.
以利沙,一個不見經傳的少年人,當以利亞將外衣搭在他身上,表示上帝對他的選照,.
他就義無反顧的離開他的父家,跟從了以利亞,.

$^{81}$他是代表了那七千個未向巴力屈膝的神的僕人的其中一位,他們二人都被稱為先知,.
但是上帝給他們一個更加貼心的稱號,一個很特別的稱號,稱他們做什麼呢?.
稱他們做神人,Man of God,他們是屬神的人,就是全然委身於上帝這樣的人,.
忠臣創校的意象秉持者,特別是那四位的四君子,包括趙天恩,陳濟民,高澤樂,周永健,.
以及後來加入的幾位,好像馬洪昌,馮蔭坤,也都包括我自己,本來這句話我想了很久要不要寫下去,.
昨晚結果都寫了下去,應該都包括我自己,因為我都自覺自己都是全然將自己的生命獻上,.
將自己傳歸上帝一個這樣的人,我們當時非常強調全然獻上自己的一生,.
因此將自己結合成為一個委身團,決定為了神學教育這個運動,毫無保留,毫不計較地獻上自己,.
以尼沙緊緊的跟從以尼亞直至最後一刻,他們之間所傳所成的是什麼呢?.
其實正正就是這個全然的委身,全然屬於神的這種情操,我深盼忠臣的同工們,特別是那些作帶領的,.
能夠秉承忠臣創校的時候,那班異象的秉持者,全然委身的這種心志..
我們讀第二章一至十五節,我們不難看到有幾點,讓我在這裡跟大家分享..
第一點,那裡屢次提到先知的門徒,或者有翻譯他們是先知的群體,.
A company of prophets,是先知的群體,表示什麼呢?.
表示以尼亞再不是那個單打獨鬥的一個這樣的人,先知不是那些英雄式的個人,.
而是一場運動,是一個群體被一場屬靈復興運動所帶動,.
以致他們站出來,帶領七千個未向巴黎屈膝的人,來復興以色列..
以尼亞和以利沙雖然是領袖,但是他們畢竟只不過是群體裡的一員,.
沒有個人英雄,只有群體的生命,忠臣獨特的地方,.
也是我不斷為忠臣感恩的地方,正正就是我們非常強調群體生命和團隊精神..
我可以說,四十年來在藤牧師和周牧師的帶領下,.
我們大致上都能夠保持這種情操,.
強調彼此信任,生命彼此開放,意象共同對照,同心地去服事..
第二,在這段經文裡,我們可以看到兩師徒的緊密相連的關係,.
其中彼此信任是毋庸多說,.
但關鍵乃是他們一同的信任和服從上帝,.
上帝要以利亞從吉格去伯特利,再由伯特利去耶利哥,然後從耶利哥去約旦,.
約旦就是上帝接以利亞終極之地,.
熟悉以色列地理的人都知道,吉格其實就在耶利哥的附近,.
而約旦就在旁邊,為何要繞過一個大圈,.
由吉格去伯特利,然後再由伯特利折返去耶利哥,再去約旦呢?.
聖經沒有解釋,以利亞沒有解釋,.
只是耶華吩咐我從吉格去伯特利,.
在伯特利的時候,又說耶華吩咐我從伯特利去耶利哥,.
然後說耶華吩咐我去約旦,是耶華吩咐,.
以利亞聽命於耶華,.
以利沙信任他的領導,不離不棄地緊緊地跟著他走每一步,.
信任的同行是感人至深,.
也唯有這樣,同行者以利沙才能說出他心中所願的,.
就是願感動你的靈,加倍地感動我..

$^{121}$這句話其實要說一點都不容易說,.
如果真的要說出心中所願,.
他必定是出自謙卑和信任..
第三方面,以利沙很大膽,.
他要求正式受到確認,作為以利亞的繼承者,.
他要求加倍地感動,即是雙倍地祝福,.
是指向長子的名分,.
他膽大不是出於個人的野心,.
而是出於一個信念,.
就是耶華交付給他的責任和權柄是非常大,.
他要承擔所交付給他的責任,.
他是不可以卸駁,.
因此他需要受到確認,.
也需要有那個能力去承擔,.
他膽大是因為他不看自己的得失,.
而是看到擺在他面前的歷史責任,.
形勢非常凶險,.
連他的師父都灰心求死,.
而他要完成的事,.
其實比他師父做的更加大,.
不知道大家記不記得,.
在《列王記》上第十九章,.
上帝要以利亞完成三件事,.
就是高哈舍作亞蘭王,.
高耶護作以色列王,.
和高以利沙作先知來接續他,.
三件事是關乎以色列歷史命運的大事,.
以利亞只做了一件,.
就是高以利沙作先知來接續他,.
其他兩件朝代兄弟的大事是誰做的呢?.
是以利沙做的,是以利沙完成的,.
以利沙知道有更大的事等著他,.
他豈能不祈求更大的屬靈能力呢?.
第四點,.
以利沙可以求,.
但是這是以利亞沒有能力給他的,.
哪怕他多麼願意給他,.
所以以利亞說:.
「你所求的甚難.」.
以利亞雖然是先知,.

$^{161}$有神賦予他特殊的能力,.
但他不能保證他心愛的,信任的,親手高立的以利沙.
真的可以得到他所祈求的能力..
在他被接的時候,.
他說:「如果你看到我,你就可以得著,.
若見不到,就得不著.」.
重要的是漢見,.
是見,見到,.
不過見到什麼呢?.
以利亞以為,.
以利沙要見到他被提升天,.
不是,.
不是見到他,.
以利沙的確親眼見到以利亞被提上升,.
但這不是最關鍵的地方,.
最關鍵的所在,.
是他看見遠遠超越一個偉大先知被提的榮耀的景象,.
他看見的是天上的景象,.
他看見的時候,.
當他看見的時候,.
他驚叫:.
「我父啊,我父,以色列的火車,火馬,.
Chariot of fire and horses of fire.」.
他看到耶和華的大軍,.
他看到世上的勢力和權能不能阻擋的,.
他看到神讓他看到,.
和要他見到的,.
那就是從天上來的異象,.
沒有從天上來的異象,.
我們縱然有最高超的聰明才智和魄力,.
也不能為我們的主成就什麼大事,.
不能作出出人意外的事情,.
無權無勢的保羅面對著位高權重,.
可以定奪他生死的阿基帕王,.
他講出了一個很簡單的話,.
「阿基帕王啊,我固此從來沒有違背那從天上來的異象.」.
阿基帕王認為保羅瘋了,.
四十年前,四個神學初哥,.
為了忠臣十億美德之民,.
為了中國民族的命運,.

$^{201}$他們要開展神學教育新的運動,.
當時應該有不少人以為他們瘋了,.
但是如果他們真的瘋狂的話,.
他們是為主瘋狂,.
因為他們見到天上來的異象,.
有了異象,他們就可以有膽量來作事,.
才有異常的能力,.
本來不能夠出現,本來不可能做成的事,.
靠著神的能力,他們就做成..
講成全,到底成全什麼呢?.
成全最核心的,.
不是學術優良,設備完善,.
行政的效率,而是從天上來的異象..
第一代怎麼看見,這一代同樣地都要看見,.
並且要比上一代看見的是更廣,.
更闊,更大,這樣的異象..
這一代不能夠靠上一代的異象,.
繼續前進..
以尼亞自己看到異象,.
他也是一個滿有異象的人,.
但是他不能夠保證他要交棒的人,.
一定會見到異象,.
他只能告訴他,.
你若看見,就得著能力,.
若看不見,就得不著..
我深信,我深願忠臣這一代的領袖們和宮門,.
一同看見那從天上來的異象..
當周牧師榮休的時候,.
我作了一對不是很工整的對聯,.
代表著眾同工送給他,.
向他致意,.
也真是對他的生命,.
一個我認為是很貼切的描繪..
這對聯是這樣說的:.
「懇託全心,大捨無悔,.
耕耘用力,天象不畏.」.
什麼是大捨無悔呢?.
佛家語,.
小捨有求,.
就算你捐十億,.

$^{241}$你如果有求,.
你的心是掛著自己,.
你所捨的,.
只不過是小捨..
大捨是什麼呢?.
大捨是無求之捨,.
捨棄就是捨棄,.
不為自己求什麼,.
懇託全心,大捨無悔..
很感謝周牧師,.
他作為一個榜樣,.
用他一生的精力來懇託這一片土地..
他所捨棄的是大捨,.
是無求之捨,.
大捨無悔,.
耕耘用力,天象不畏..
用盡所有的精力來耕耘,.
為什麼呢?.
是為了不畏天象..
我心盼三十年之後,.
七十年了,.
在忠臣慶祝七十年周年的時候,.
這一對的對聯,.
依然能夠嚴重當時的領袖們,.
同工們的情操和胸懷..
(字幕製作:貝爾).
(字幕由 Amara.org 社群提供).
\newpage



\section{}
\label{sec:RU4oBv0Wasg}
\textbf{中神45周年北美培靈講座(西岸)}
\newline
\newline
連結: \href{https://youtube.com/watch?v=RU4oBv0Wasg}{\texttt{ https://youtube.com/watch?v=RU4oBv0Wasg}} ~~~~ 語音日期: 2021-01-20 
\newline
\newline
\hyperref[sec:yNpGxFfqF_k]{\small{< < < PREV SERMON < < <}}
~
\hyperref[sec:index]{\small{[返主目錄]}}
~
\hyperref[sec:_3eTxXlzPX0]{\small{> > > NEXT SERMON > > >}}
\newline
\newline
$^{1}$各位主內鼎姊妹,平安!.
歡迎你們來參加.
中國神學研究院今年所舉辦北美的培靈講座.
和李詠廷博士,我們舊約的老師.
詠廷老師會從路德基教授.
和我們探討困苦到恩典的心路歷程.
詠廷老師是我們忠臣儲備師之.
在學院受訓,他畢業之後加入.
成為我們45年來頭一位舊約師.
成為我們45年來頭一位舊約聖經科的女教授.
所以我們感謝神.
年輕一代的同工陸續加入.
各位弟兄姊妹,主內平安!.
在這裡也特別向身處北美的弟兄姊妹問安.
雖然我們相隔了一個海洋.
但感恩我們透過網絡仍然可以一起分享神的話.
在這個疫情,我們更加感覺到.
整個世界都在同一天空下.
因為我們都面對著這個疫情帶給我們各方面的危機和挑戰.
過去這一年,無論我們身處在什麼地方.
相信都體會到人生有很多事情我們無法掌控.
疫情說來就來,一浪接一浪.
生活大小的事都完全被打亂.
還有身心,社靈可能都面對不少衝擊.
在生活上,因為疫情帶來經濟的挑戰.
很多人都要面對失業,陷在困境.
生活突然失去保障.
而同人的關係上.
我想我們為了防疫,很努力保持社交距離.
但與此同時,我們與親友相聚的機會少之又少.
甚至教會的聚會,我們都很多時候要改在網上.
有時都會覺得突然間好像沒有群體,好像只有自己.
而我想最難受的,很多時候是心理的壓力.
在持續的困局中,我們都覺得做很多事都是徒勞無功.
有時都會想,到底上帝何時才出手相救呢?.
很多人心中都會覺得很愁苦,甚至憂鬱.
我們都好像沒有什麼盼望.
在這個時候,有時我們都會想,在這個大時代.
像你和我這樣的小人物,我們還可以做些什麼呢?.
當我自己預備這個培靈會訊息的時候,我都去祈禱.

$^{41}$我都問上帝,在這個話語裡面,在這個時勢.
可以怎樣給我們安慰和鼓勵呢?.
在這個時候,我讀到《路德記》.
我發現原來聖經裡面都有記載著.
在一個動盪的大時代裡面,一個小家庭的尋常故事.
在裡面沒有什麼驚心動魄的偉大神蹟奇事.
也沒有上帝的顯現.
甚至沒有一句是由上帝親自說的說話.
但是,卻是從路德的身上.
我們看見他怎樣從困苦去經歷蒙恩.
今天我們又是否可以這樣去經歷呢?.
我們看到其實路德身處的時代.
都可以說是一個充滿了動盪的時局.
我們看到路德記第一章一節這樣說.
當事事秉靜的時候,國中遭遇饑荒.
大家有讀過《事事記》應該都知道.
其實在事事時期是充滿了戰爭.
很多的外敵攻擊.
可以說描述裡面是充滿了暴力和混亂.
加上國中遭遇饑荒.
可以說在當時的大環境.
無論是政局,無論是民生都不是一個好景.
而路德,剛才我們說的.
就好像一個大時代裡面的小人物.
雖然我們知道路德將來會成為大胃王的祖先.
但是在那個時候,他的確是一個小人物.
在一個小家庭的裡面.
在一個大環境之下.
其實他面對的困境是更加雪上加霜.
路德的困境,首先他喪夫,他也無子.
在當時來說,一個寡婦跟著另一個寡婦拿餓米.
他的生活是完全沒有保障.
而且路德是一個魔壓女子.
他離鄉背井跟著拿餓米回到伯利恆的時候.
其實完全沒有屬於他自己的群體.
所以他也是一個很孤零零的一個人.
而更加他還需要去照顧身邊的人.
他要照顧他的婆婆拿餓米.
而拿餓米因為喪夫喪子.
其實他的心情經常處於一種極度仇苦的狀態.

$^{81}$如果用我們今天的說話來說.
其實是一個充滿負能量的人.
所以拿餓米有時也會跟別人說.
不要再叫我拿餓米了,拿餓米是甜的意思.
叫我瑪拉吧,瑪拉是苦的意思.
因為全能者使我受了大苦.
所以在這裡可以看到路德身邊的人都是充滿了苦情.
而他也只能夠跟婆婆一起去承擔這個苦情.
不過在這樣的情況下.
路德雖然面對各人的困境.
但在第二章他發現路德三次提到他蒙恩.
而今天我們看路德記第二章.
也會特別集中在這三段.
這三段分別在第二章第二節說.
摩訶女子路德對拿餓米說.
「用我往田間去,我蒙誰的恩,就在誰的身後去到拾取物遂」.
還有二章十節是路德跟波亞斯說的.
「我既是個愛邦人,怎能蒙你的恩,這樣去連戍我呢?」.
二章十三節也是路德跟波亞斯說.
「願我在你眼前蒙恩,我雖然不及你一個侍女.
你還用慈愛的話安慰我的心」.
蒙恩這個字,英文的翻譯可能更加貼近原文.
就是「to find favour in your eyes」.
在你眼前,我得到幫助或好處.
其實這句也是一句常用的說話.
不過在這麼近的經文裡面重複出現三次.
也很值得我們去看看.
到底路德是怎樣在這個困苦的境況裡面.
可以提說三次他去經歷蒙恩.
剛才我們說第一次,就是出現在第二節.
在這個段落,正正就是路德已經跟拿俄米回到伯利恆.
路德在第二節主動跟拿俄米提出.
「用我往田間去,我蒙誰的恩,就在誰身後拾墨水」.
這句話很簡單,反映在路德心目中.
蒙恩是什麼呢?就是一個可以在人身後拾墨水的機會.
而當時窮人,寄居者可以在修閣的人身後去拾墨水這個做法.
其實正正就是記載在《五經》的律法裡面.
我們可以去看看《利美記》十九章九至十節就是這樣說.
「在你們的地修閣莊稼的時候,不可以割盡田閣.
也不可以拾取所遺落的,不可以摘盡葡萄園的果子.

$^{121}$也不可以拾取葡萄園所釣的果子,就是要留給窮人和寄居的」.
所以在這裡簡單看到,就是你不可以收盡所有的莊稼.
不可以摘盡所有的果子.
而類似的描述在《生命記》二十四章十九至二十一節都有.
就是你在田間修閣莊稼,如果忘記了拿一捆的.
你不要回頭拿,就留給那些孤兒寡婦和寄居的.
甚至你打橄欖樹,都是你枝上打剩的.
你不要再打,要留給寄居和孤兒寡婦.
你摘葡萄園的葡萄,都是一樣.
剩下的不要再摘,留給寄居和孤兒寡婦.
所以我們可以看到,其實這條律法最主要就是要神的百姓留有餘地.
不要收盡所有的莊稼.
就是留給窮人,孤兒,寡婦,寄居的可以去撿.
可以去收割剩下的糧食裡面,去找到微小的幫助.
讓他們可以在生活裡面得到最基本的保障.
就不用一定要去借貸或者去行乞.
而剛才我們說的路德,他正正就是歸入這個裡面.
他是在貧窮裡面,他是寡婦,他是寄居的.
所以他能夠有一個蒙恩的機會,可以在人身後拾物水.
其實也是源於這個恩慈的律法.
當然這個恩慈的律法,就是源於那位定立律法,有恩慈的上帝.
當在困苦的時刻,當拿訛米還在自怨自艾說全能者死去受苦的時刻.
路德他就決定主動去養大這個恩慈的律法.
所以第二節他就是主動說用我去填耕.
不是拿訛米提出的,是路德自己提出.
而在第七節其實也都看見,他這樣出去.
他會問那裡田地的監督.
其實他是要冒著被拒絕的可能.
但是他都勇敢去提出要求.
而另一方面我們都看到,其實路德他不只是把握了一個機會.
他自己都很積極,辛勤去到工作.
監督說他從早晨到如今.
除了在屋子裡面坐了一會兒,就常常在田裡.
其實看見路德一直都很積極去到工作.
甚至乎以這十節去描述,他是十勿睡到晚上.
路德就是握著這個律法給他的保障.
同時路德他自己都主動去積極去工作.
而另一方面,在這段經文我們看到.
可以說路德都是放手交托的.
為甚麼這樣說呢?.

$^{161}$你見到他第二節說「我蒙誰的恩,就在誰的身後去十勿睡」.
看來這件事是沒有甚麼計劃的.
但是如果我們細心去想.
其實這件事可不可以很有計劃去進行呢?.
如果我們看二章二十節.
其實拿訛米是一早就知道他有個親戚叫波亞斯.
這個親戚是有田有地.
不過他沒有去計劃,只是想去投靠.
這樣整件事就好像充滿了一個巧合去發生.
而我們看經文第四節.
亦都是說了這種恰巧.
第三節亦都是說了這種恰巧的出現.
所以我們看到.
罪事者都好像刻意去突出這種不期而遇的情況.
去看到路德是恰巧去到以利米勒本族人波亞斯的那塊田.
這種不期而遇,無心插柳.
就讓我們感受到.
似乎背後有一個掌管萬事的上帝在安排.
的確,在路德記的描述裡面.
神母行甚麼神跡騎士.
就是藉著她恩慈所定的律法.
以及藉著她恰巧的安排.
就嫁入在路德的生活裡面.
而路德就是懷著那份交託的心.
同時主動積極去行動.
養育這個律法給他的保障.
他就可以從那個困苦的處境稍稍有些突破.
弟兄姊妹,當我們今天面對著各樣的困境.
我們會像拿餓米一樣自怨自艾.
會在愁雲慘霧裡面.
還是我們願意像路德一樣.
交託,同時積極去行動.
養育神為我們預備的資源.
我們願意踏前一步嗎?.
在這裡也要提一下.
路德之所以可以拾墨水.
以及養育這個恩慈的律法.
很重要,還有原因是.
有像波亞斯這些願意進行律法的人在其中.
波亞斯的田地沒有割盡農作物.

$^{201}$也願意開放給人去拾墨水.
所以可以看到在困苦的局面.
神未必透過超自然的神蹟介入.
他可以透過你和我平凡的人.
去承傳他的律法和吩咐.
也可以讓有需要的人生活是經歷神的恩慈.
我自己也想起疫情初期.
不知道大家有沒有類似的經驗.
口罩是極度短缺.
尤其是長者和基層很多時候都買不到口罩.
當時我們教會就有人去呼籲.
會不會大家可以將你家裡有餘的口罩拿出來.
我聽到這個呼籲,最初是有點兒女的.
因為當時我覺得口罩是難買過iPhone的.
排隊排很長時間,也要很早.
甚至凌晨就要去排隊.
不知道弟兄姊妹會不會去響應呢?.
但是很感恩.
我後來發現原來大家不是需要擁有很多.
才願意分享.
有些弟兄姊妹可能家裡只有兩三盒存貨.
但他們都願意拿著一盒出來給教會收集.
最後教會竟然收集到一定數量的口罩.
不僅我們可以分給教會裡面的長者.
而且我們還可以給一些服務基層的機構.
一解燕眉之急.
這個小小的例子我們可以看到.
即使我們未必像波亞斯一樣是大田主.
擁有很多的田產.
或者我們都未必可以做些什麼去即刻扭轉局面.
但原來我和你可以拿出一些有餘的.
也是在承傳上主的吩咐.
也可以叫到身邊的人去經歷神的恩典.
除了體會在物質上有恭敬之外.
當路德第二次提及他蒙恩.
就不單單是在說物質的生活.
我們去看的是二章十節.
路德再一次提及蒙恩.
就是他聽到波亞斯一番說話之後.
路德就俯伏在地上求拜.

$^{241}$對波亞斯說.
我幾時愛邦人.
怎麼蒙你的恩.
這樣連顧恤我呢?.
路德其實有很大的反應.
馬上俯伏求拜.
我想是因為他經歷了一個很大的顧恤.
在和修本這個字就譯為照顧.
在原文顧恤這個字就可以解作認識,認得.
其實是想說明彼此之間有那份關係.
到底波亞斯說了什麼.
令路德感動到這麼大的反應呢?.
這麼感覺到被顧恤呢?.
我們去看第八,第九節.
就是波亞斯去吩咐路德.
女兒聽我說.
你不要去別人的田裡灑墨水.
你不要離開這裡.
接著他有三個吩咐.
他說你要經常和我的侍女在一起.
你見到僕人去哪裡工作.
你就跟著我的僕人.
第三個比較特別的就是.
你頸渴嗎?.
你就可以去器皿那裡.
喝僕人打的水.
其實這三個吩咐.
可以說都有一個共通點指向同一樣東西.
就是他把路德當成自己人.
讓路德走進這個群體.
好像波亞斯其他的侍女僕人一樣.
讓他可以跟他們出入.
甚至這個題說.
他口渴的時候可以喝他僕人打來的水.
這個情景就有點像是什麼呢?.
可能你的公司有員工專享的區域.
或者員工專享的餐廳.
路德是一個外邦人.
是一個普通人.
他可以跟著波亞斯的侍女僕人.

$^{281}$進入這個地方.
其實傳統上打水這個工作.
通常由外邦人的女子去做.
在這裡作為外邦人女子的路德.
竟然可以飲用由以色列男丁打的水.
其實也很特別.
正正也在這裡反映了.
路德已經成為這個群體的一分子.
再者,波亞斯也再三叮囑他的僕人.
不可以欺負,不可以羞辱,不可以斥嚇路德.
在15,16節更加提到.
就算路德在禍群裡去倒墨水.
這件事不是律法讓他這樣做.
不過波亞斯也提醒了他的僕人.
你們也不可以去說他,去羞辱他.
所以在這裡看見.
波亞斯真的把路德當成自己人.
他不想路德因為外邦人的身份.
而受到排斥或者欺負.
所以對於被當成自己人.
路德是喜出望外.
所以他強調的,在第十節說.
我何時是外邦人.
他其實自己知道.
他要強調其實他的身份.
本來就是一個外邦人.
本來就是跟這個群體沒有關係.
所以他才這麼驚訝.
可以得到這一份的連述.
而在之後不單是上班工作的時候.
路德可以有這份固卒.
而到吃飯的時候.
在經文後部分.
波亞斯也是一樣.
去招呼路德可以坐在他修國的人旁邊.
在第十四節.
甚至說到他們空了的睡紙都遞給他吃.
所以在一起可以同台吃飯.
就是一份很大的捷.
而二章二十三節到最後.

$^{321}$其實也提到路德一直都會跟波亞斯的侍女去撿墨水.
直到收完所有大墨小墨.
所以看到整個第二章.
路德就是慢慢地融入這個群體.
無論工作,勞動的時候.
無論是休息,吃飯的時候.
都可以和大家在一起.
我相信對於一個離鄉別井.
真的覺得自己是異鄉人.
是非我族類.
是另一個族類的人的時候.
能夠在異地找到一個可以歸屬的群體.
是一件很蒙恩的事.
再者其實波亞斯願意接納路德這個外邦人進入他的群體.
甚至讓路德可以在禾菌裡撿墨水.
這種固率就不是律法明文的規定.
剛才我們讀了經文.
我們看到律法其實要求不要收盡裝甲.
所以可以說波亞斯的慷慨和對路德的接納.
其實已經超出了律法的要求.
當然我們也可以聯想.
不幫助波亞斯這樣做.
或者是出於對路德的欣賞和憐愛.
但同時可以說波亞斯這種固率正正反映.
他活出實踐律法的精神.
不單是律法的規定.
也活出律法的精神.
就是善待寄居的.
善待孤兒寡婦.
在《生命記》24章22節提到.
其實神要求以色列人.
他的子民這樣做.
也是要紀念以色列人自己.
也曾經是寄居.
並且是在埃及遺留.
所以我們看到.
這就是律法的精神.
能夠讓路德去經驗一個這麼大的固率.
不過能夠經歷這樣的折納和固率.
可以融入在另一個文化.

$^{361}$另一個群體裡面.
其實我想也不是理所當然.
因為我們知道人和人之間.
本身真的有很多事情.
可以令我們阻隔.
甚至可以充滿對立和差異.
所以有時在這樣的處境之下.
到底我們如何去實現路德所經歷的固率.
能夠讓不同的人.
不同的群體彼此結連呢?.
很想分享這對夫婦的見證.
在美國有機會認識了一位神學院的聖經教授.
Dr.Kinner和他的太太Madeen.
我在他們身上.
真的看到兩人如何跨越了各種不同的阻隔.
走在一起.
一起建立一個共融的群體.
他們都很喜歡分享自己的故事.
他們都會提到.
其實他們兩人真的很不同.
性格也很不同.
而且兩人的膚色,背景,文化,生活的經歷.
都有很大的差異.
一個在美國的城鎮過著穩定,小康的生活.
而另一位卻是在非洲的剛果.
在一個充滿種族仇恨,戰火貧窮的地方成長.
而他們之所以相遇.
是因為剛剛Madeen有機會到美國進修.
就認識了Kinner.
他們相遇之後互相對對方有好感.
不過最初Kinner說.
他因為一些侍奉的考慮和種種各樣的原因.
他婉拒了Madeen的好意.
所以當時Madeen帶著少許失望.
回到家鄉.
她就經歷了一段帶給她很多創傷的婚姻.
亦很不幸.
Madeen回到家鄉的時候又遇上了內戰.
她的一家在戰爭裡面成為了戰爭難民.
足足18個月.

$^{401}$這18個月她完全和外界失去聯絡.
她有親人被槍殺.
在聖誕節的時候被槍殺.
亦曾經面臨過一些敵人強攻入屋.
她要在槍林彈雨之下和家人逃命.
後來Madeen終於幾經辛苦.
在苦口餘生之後.
她寫了一封信.
找到一個機會寫一封信給Kinner.
這封信的第一句是.
I'm still alive.
我仍然活著.
這句很重要.
原來當Kinner收到這封信.
她很震撼.
因為Kinner和Madeen失聯了18個月.
其實她也每天掛念她.
她也很擔心Madeen到底還在不在世上.
18個月每天為Madeen祈禱.
所以當她終於收到.
I'm still alive.
這句說話.
Kinner很興奮.
亦是那一刻她發現.
原來真是生命.
最重要的是生命.
她也確定了自己的心意.
於是她放下過去很多種種的考慮.
她決定無論如何.
靠著神的愛.
靠著神的恩典.
也因著耶穌基督十字架的愛.
她決定要跨越一切的差異和阻隔.
於是她也很努力做了很多事情.
去協助Madeen離開南極家鄉.
這種困苦的情境.
長話短說.
他們經歷了重重的波折.
兩個人終於可以終成眷屬結婚.
Madeen也分享.

$^{441}$其實經歷了各樣很創傷的經歷之後.
她很體會.
丈夫對她的接納和顧恤.
帶給她很大的意志.
亦當然在裡面深深經歷神的愛.
她們覺得這一段的關係.
她們的經歷.
是一個不可能的愛.
所以她們也寫了這本書.
《不可能的愛》.
去分享她們的故事.
不過她裡面有一點也強調.
她們說她們的經歷.
是,也好像大團圓結局.
但是她說其實在非洲.
在世界很多的角落.
還是戰火瀕營.
仍然有很多因為差異而對立.
和仇恨的地方.
彼此去做很多傷害的事.
所以她們夫婦二人如今.
其實一直很努力去做一些.
種族復和的事工.
亦都很希望去到鼓勵更加多的人.
可以跨越那種阻隔.
重建關係.
活出真正有愛的群體.
我深深體會她們的生命.
那種感染力.
就是因為她們可以跨越一切的阻隔.
去到彼此結連.
在艱難的日子.
我們都會更加體會.
其實愛的關係.
愛的群體.
可能是比更多物質.
生活更加重要和珍貴.
頂姐妹.
可能在這一年.
是因為疫情.

$^{481}$因為政局.
我們人和人之間.
都經歷很多的疏離.
很多時候我們的關係有了阻隔.
有時真是物理的距離.
我們有社交距離.
但可能有時更多的.
是因為我們的背景.
種族身份甚至政見等等.
令到我們產生疏離.
在這樣的日子.
我們可以怎樣做呢.
我們願不願意都多走一步.
去到和一些我們可能體會.
是非我族類的人.
去到重新連繫上.
重新建立這份關係.
可能這一刻.
大家可以想想.
在我們身邊.
有沒有一些我們忽略了的人.
在我們身邊有沒有一些.
我們平時覺得和我們很不同的人.
我們可不可以透過言語.
或者簡單的行動.
甚至一兩句說話.
或者好像麥甸那樣寫一封信.
我們去重新連繫上.
突破那些物理上.
或者心理上的距離.
重新建立一個.
可以互相結連.
接納和固縮的群體呢.
最後第三點.
也是路德去第三次提到蒙恩的.
就是在二章十三節.
路德說.
向博亞斯說.
主一願我在你眼前蒙恩.
我雖然不及你一個侍女.

$^{521}$你還用慈愛的話.
安慰我的心.
在這一句.
蒙恩就以一個祈願的方式去表達.
原文其實是一個未完成式.
也可以理解為.
路德正正都是在感受.
博亞斯對他的那份恩情.
就好像英文聖經ESV或者NASB的翻譯.
也會說.
I have found favour in your eyes.
在這裡關於蒙恩的描述.
其實對於路德來說.
又是一件什麼事呢.
在這裡說.
他感受到博亞斯對他說出安慰的話.
原文的意思是.
因你安慰我.
你的話到我的心.
是很貼心的說話.
這個表達可以說比前兩次更加強烈.
到底博亞斯說了什麼.
令到路德的心靈得到這麼大的安慰呢.
我們去看第十一和第十二節.
其實博亞斯說了什麼呢.
就是說路德發生過的事.
他說路德.
你.
為什麼我會這麼固恤你.
就是因為自從你的丈夫死了之後.
凡你向你婆婆拿我米所做過的事.
是你離開了父母和本地.
來到猶大的伯尼罕.
來到一個你不認識的民族裡.
這些事他都知道.
在這番話裡.
你們都應該可以感受到.
博亞斯表達的那種肯定和欣賞.
我都會去想.
這番話只是平鋪直述.

$^{561}$可以有多安慰呢.
其實關於路德對拿我米的忠誠和照顧.
都不用多說了.
我們看經文第十八節.
往後他都很照顧婆婆.
他有得執了墨水.
有得吃吃飽.
他都會帶回去給婆婆.
所謂見微知著.
就知道路德絕對是一個好媳婦.
不過路德過去有沒有得到這麼大的肯定呢.
就著路德離鄉別井.
跟拿我米回伯尼罕的時候.
其實又真的.
拿我米都好像沒說過半句肯定的話.
就算之後路德執完墨水回家.
拿我米的反應是怎樣呢.
拿我米最大的感激.
主要是博亞斯.
願那個顧恤你的人得福.
願神賜福給那個人.
似乎都沒有特別提到路德.
或者對路德表達那份感激之情.
當然我也覺得.
路德也未必真的一定是.
很想聽到這些讚他的說話.
但是你明白.
當路德一直默默耕耘去做的事.
原來有人知道.
博亞斯他知道.
並且表達那個肯定和欣賞的時候.
我想心裡面自然都會感到很安慰.
而且他很感受到博亞斯體會.
他那個離鄉別井.
他那個來到不認識的民裡面的處境.
心裡面覺得安慰.
這番話說到他的心坎裡.
我相信更重要的是.
在第十二節中間的部分.
第十二節的部分.

$^{601}$願耶和華照你所行的賞賜你.
你來投靠耶和華以色列神的賜婆.
願你滿得祂的賞賜.
這番話並不是單單說.
博亞斯對路德的肯定.
而是博亞斯更加重複祈願.
耶和華要按路德所做的去賞賜他.
這番話對於路德來說.
可能是另一個的震撼.
因為其實對於一個摩訝女子.
耶和華對她來說有什麼意思.
有什麼意義.
她對耶和華有什麼認識呢.
我們回顧一下.
如果他從婆婆口中聽到關於耶和華的說話.
是什麼呢.
第十三節說.
耶和華伸手攻擊我.
第十節說.
傳靈者使我受了大苦.
耶和華使我空空的回來.
耶和華降禍.
傳靈者使我受苦.
路德聽了這麼多.
可能在他心中.
耶和華彷彿和受苦降禍有關聯.
所以就連路德自己發誓.
在一章十七節我們都很熟悉.
他的起誓都是.
除非死靈使你我相離.
他和萊奧米發誓的時候.
不然願耶和華重重降佛給我.
路德自己去許願.
都是說到耶和華要降佛給他.
所以在這一刻.
剛才我們讀的經文裡.
去說波亞斯說.
耶和華會按著他的善行賞賜他的時候.
我相信這個就是對路德來說.
很獨特的一個表達.

$^{641}$而且他邀請路德去投靠.
在耶和華以色列的神的賜旁下.
這句話正正夾在賞賜的中間.
很重要的一段經文.
賜旁.
看到賜旁.
賜旁這個意象在古晉東其實也常見.
在古晉東裡面.
賜旁很多時候是表達神明的庇護.
那種的隱蔽.
而投靠的意思.
就是說你去敬拜這個神.
也可以理解為你將自己交託給神明.
你願意跟神明的群體結連的意思.
而在《證券的路德記》裡面.
這一次可以說是第一次正式有人向.
愛邦女子路德發出邀請.
投靠以色列的神的賜旁下.
我可以想如果這一幕要拍成電影.
其實也會很感動.
因為我也代入路德.
默默的辛勞.
默默的付出.
原來邦亞斯知道.
原來耶和華會知道.
而且說會有賞賜.
原來即使路德是一個魔壓的女子.
也可以投靠在以色列神的賜旁下.
所以對一個困苦的路德來說.
他所經歷的蒙恩.
不再單單是物質上.
關係上.
而更加是在屬靈上.
他可以體會神的賜旁.
耶和華的賜旁.
以色列神的賜旁.
都要成為路德的隱蔽.
賜旁的隱蔽.
我不知道大家小時候有沒有玩過一個遊戲.
叫做「麻鷹捉雞仔」.

$^{681}$簡單來說.
有一個人扮麻鷹.
有一個人扮母雞.
而其他人扮雞仔.
通常躲在母雞的後面.
當麻鷹要捉雞仔的時候.
母雞就會在前面張開雙手.
像翅膀般掩護.
如果你是做小雞的話.
會玩的話會怎樣玩呢.
你會捉緊母雞.
因為你知道如果一脫手.
或者你嘗試靠自己能力.
到處走的時候.
你離開了母雞的賜旁保護.
你就很容易被麻鷹捉到.
這樣你就輸了.
在這個遊戲裡面我們看到的關鍵就是.
原來我們的雙眼不是注目在那隻麻鷹上面.
我們最重要的是注目在母雞和他的賜旁.
在他的隱蔽下面.
就好像我們有時候.
人生都會面對各樣的難處.
就好像面對著來勢洶洶的麻鷹一樣.
嚇得我們有時候會亂走一通.
但其實這樣反而更加危險.
因為最穩當的就是我們可以投靠.
我們投靠在神的賜旁之下.
關於神的賜旁.
有一位美國靈修學的導師David Muskins.
在一個著作裡面分享他的經歷.
他的經歷是怎樣呢?.
他說他的媳婦,他的媳婦患了敗血病.
一個星期就突然離世.
病患死亡.
我想這段日子我們都有很多的感受.
這位靈修學的導師.
他說其實他在過程裡面很希望什麼呢?.
很希望神蹟出現.
很希望他的媳婦可以神蹟地康復.

$^{721}$但是事與願違.
一個星期他的媳婦就死了.
當時他只有46歲.
他的媳婦就留下了丈夫和三個年幼的孩子.
他們一家都非常之傷痛.
就算這位靈修學的導師.
他和上帝都有很親密的關係.
但他那一刻都會問.
上帝你怎麼可以容許這樣的事情發生呢?.
他都找不到答案.
他覺得很難去解釋.
直到他在媳婦的追思禮拜裡面.
他聽到師班的獻詩.
一首詩歌叫做《成英翅膀》.
我在這裡找了中文的歌詞給大家聽.
裡面其中的歌詞就是說.
「投靠在主翅膀下,主信實是臀牌」.
最後他說.
「因他要為你吩咐他眾使者在你的道路上保護.
他們必定用手托著你」.
這位靈修學的導師Munskins.
他體會神的愛.
原來就是藉著眾多信徒的關懷.
去托著他一家.
在這個對他來說是悲情的時刻.
他描述他經歷的是基督的肢體.
用憐憫的蚌蜜擁抱著悲動的我們.
上帝就是藉著信徒群體.
去展示他的保護和承托.
去展示他的翅膀.
所以「投靠在神的翅膀下」.
弟兄姊妹學.
我想我們都會面對不同的危難和困苦.
或者這一刻我們已經在面對.
我們都未必真的看到神跡的出現.
我們可能會問.
「神為什麼你會容許這些事發生?」.
我們心裡面都有很多的不解.
驚惶.
驚惶 甚至疑惑.

$^{761}$但是就讓我們思想.
我們不要忘記.
神願意張開他的翅膀去保護我們.
他的翅膀是藉著他的教會.
藉著耶穌基督的群體來承托和安慰我們.
就好像神也藉著波亞斯成為路德的蔭鼻.
在這段經文第三章第九節.
其實也提到當路德再去找波亞斯的時候.
他請求波亞斯要以「依cum 」去遮蓋他.
「依cum 」這個字.
原文其實就是翅膀.
神就是讓路德投靠在他的翅膀之下.
在困苦裡面去經歷蒙恩.
頂姐妹.
我們從路德的經文去看見.
特別是第二章.
看到是如何由困苦到蒙恩.
在這個情況下.
真的未必大環境有什麼改變.
甚至困難的日子其實依然繼續.
路德仍然是經濟困難.
就好像我們仍然面對著一開始說的.
經濟上,關係上,心靈上的困局.
甚至現在我們經常面對著.
疫情,病患,死亡的陰影.
可能家人有病.
到現在都沒有好轉.
我們還可不可以去經歷蒙恩呢?.
這個時候我也想起.
我自己的經歷.
在八年前.
我丈夫的妹妹患上腦癌.
她當時只有29歲.
這個病其實是相當的折磨.
因為腦袋一直在大的時候.
身體的機能就會慢慢失去.
開始不懂得走,不懂得走,不懂得動.
不懂得吃東西.
到最後甚至連說話都不行.
不過意識還在.

$^{801}$只能夠睡在病床上.
我們都好像剛才那位靈修學的導師一樣.
我們都祈禱.
我們都希望神蹟可以出現.
都希望有神蹟性的康復.
但是最後都沒有發生.
18個月之後.
天父接走了妹妹.
很記得想起那18個月.
真是不容易的經歷.
因為我們家裡都面對著經濟上和時間上很多的拉扯.
但是在這個時候我回想.
神蹟藉著教會,藉著很多弟兄姊妹.
給我們各樣的支持.
有很實質的經濟上的支持.
有很具體的湯水各樣的支持.
而在裡面也有一個期間.
因為妹妹30天連續要去醫院往返去做電療.
我們都很困難每天去抽時間去陪她的時候.
弟兄姊妹在安排每天有不同的人輪流去陪伴妹妹.
去減輕家人照顧的壓力和奔波.
我們去想起過程是充滿了艱難.
而這個病其實也帶給妹妹很多的痛苦.
但是我想說的.
到了人生最後的階段.
妹妹還可以說話的時候.
她說她仍然是覺得感恩.
她的感恩甚至在她的遺願裡去表達.
她就跟我說.
她說如果你和哥哥有時候生小孩的話.
你記得記得記得改名.
名字裡要有個「恩」字.
是恩典的恩.
是蒙恩的恩.
我看到妹妹的體會.
雖然在病床上不能動.
在各樣的困苦裡.
原來心裡還可以這樣去體會蒙恩.
因為妹妹真的在整個過程.
在很多微小的事上去體會神的愛.

$^{841}$體會神的恩惠.
體會弟兄姊妹的愛.
弟兄姊妹今天我們都面對著疫情.
經濟,政局.
可能都面對著人生各樣的困境.
我們都會希望可以經歷上帝施恩的手.
太能的手.
可以去工作.
可以去扭轉乾坤.
不過可能更多的時候.
我們的經歷可能比較像是路德記的描述.
沒有什麼震撼超自然的神蹟.
又或者我們都看不到神直接的顯現和聲音.
但是神的恩慈.
祂是可以透過一個個的小人物.
很尋常,很微小的行動去展現.
上主的翅膀.
就是要藉著祂百姓的群體.
進行祂的吩咐.
來成為困苦者的蔭被.
所以弟兄姊妹.
今天我們不要自己靠自己去四處奔波.
我們好好地投靠耶和華.
以色列神的翅膀之下.
去經歷祂的保護和承托.
同樣都盼望我們今天可以在生活.
在關係或者心靈各樣的困苦裡面.
不單止我們自己經歷神的恩慈.
我們也可以同樣叫我們身邊的人去經歷蒙恩.
(音樂).
(字幕製作:貝爾).
\newpage



\section{}
\label{sec:_3eTxXlzPX0}
\textbf{中神45周年北美培靈講座(東岸)}
\newline
\newline
連結: \href{https://youtube.com/watch?v=-3eTxXlzPX0}{\texttt{ https://youtube.com/watch?v=-3eTxXlzPX0}} ~~~~ 語音日期: 2021-01-21 
\newline
\newline
\hyperref[sec:RU4oBv0Wasg]{\small{< < < PREV SERMON < < <}}
~
\hyperref[sec:index]{\small{[返主目錄]}}
~
\hyperref[sec:zyLpufeGBYs]{\small{> > > NEXT SERMON > > >}}
\newline
\newline
$^{1}$各位主內丁姐妹平安.
歡迎你們來參加.
中國神學研究院.
今年所舉辦北美的培靈講座.
今年我們的講員是余振寧博士.
我們新約的老師.
余振寧博士.
我們新約的老師.
Kelvin老師會跟我們討論.
耶穌在路加福音所講解的兩個重點.
在信靠當中如何競成.
振寧老師是我們跟愛丁堡大學.
聯合班授的博士課程.
是我們第一位畢業生.
他是第二位加入課程的同學.
爬了頭畢業.
我們感謝主.
丁姐妹主內平安.
很感恩有機會跟大家分享主的話語.
當忠臣安排我們講北美培靈會的時候.
李思敬院長告訴我們.
其實過去我們很少在一月份的時間.
在北美這個地方舉辦培靈的聚會.
原因很簡單.
因為一月份正值北美的嚴冬.
天氣相當惡劣.
除了很冷之外.
很多地方也有積雪.
出門很不方便.
我自己未在北美生活過.
所以這些情況.
我只能夠透過聽別人講.
或者有時候從電視的片段看到.
不過我也知道.
北美的冬天的確很不容易.
如果在這個時候我們安排聚會的話.
一個不幸在聚會的日期遇上暴風雪.
可能聚會也迫不得已要取消.
所以在過去一月份.
我們很少會在北美安排一些聚會.

$^{41}$冬天很不容易.
而且今年的冬天.
我們面對的不單止是惡劣的天氣.
我們也印象很深刻.
在過去差不多一整年的時間.
全世界正在面對新冠病毒的疫情.
北美也是重災區之一.
每天看新聞報道.
世界不同的地方確診的數字.
死亡的數字.
北美特別可能是美國這個地方.
那個數字也相當的高.
這個疫情讓我們原本已經很不容易的嚴冬.
變得更加困難.
疫情對我們的影響最直接.
當然是我們也會擔心自己會不會染病.
可能我們出門或平日生活.
也變得小心翼翼.
免得自己受到病毒的感染.
可能我們身邊也會有我們認識的人.
親人朋友甚至家人.
真的會感染了這個病.
甚至可能因為這個病而離世.
這是疫情對我們最直接的影響.
這些經歷當然會令到我們的心情很低落.
很沉重.
當然疫情的影響除了是.
讓我們或我們身邊的人染病之外.
疫情對經濟也帶來很重的打擊.
就算我們避得過這個病毒.
我們沒有受到感染.
但是經濟受到影響的時候.
可能我們也要面對失業.
做生意的可能要面對生意越來越差.
甚至生意要倒閉.
在這樣的環境當中.
我們再去想我們前面的日子.
有時候也難免覺得好像前路茫茫.
不知道該怎麼辦.
或者來到一月份的時候.

$^{81}$我們稍稍值得慶幸的是.
似乎疫苗的研發已經取得一定的成功.
甚至已經能夠開始讓人注射疫苗.
讓我們覺得似乎疫情的結束.
也有些是眉目有些是盼望.
我們也期待著當疫情過去.
我們的生活希望可以回復正常的部分.
不過或許要等到疫情真的完全過去.
還有一段日子.
這時候我們仍然是在一段很困難的時候.
當我們面對困難.
面對沮喪.
面對迷惘.
身為一個基督徒.
我們知道其實我們很需要聆聽神的話.
讓神的真道成為我們的幫助.
成為我們的力量.
所以我們有這次的北美培靈聚會.
其實這也是一個很特別的機會.
正正是因為疫情的原因.
很多實體的聚會都被迫取消.
包括我們眾神原本安排了一連串.
45週年遠慶的聚會都取消了.
其他很多的聚會我們都要改為網上進行.
而正正因為我們有網上這個平台.
所以我們可以很特別地.
在一月這個不適合舉辦聚會的時候.
舉辦到這次的培靈會.
如果用世人慣常的話去說.
我們會說有危必有機.
當環境有很多的限制.
很多的挑戰.
這個有時候會迫使我們有些突破.
有危必有機.
這是世人常說的話.
不過用聖經的話去說.
其實我們應該說.
在這些危機,困難當中.
是指引我們更加清楚看到.
神的應許的真實.

$^{121}$看到神的恩典是怎樣救我們用.
看到在試煉當中.
神怎樣親自為我們開出路.
當環境的轉變超過我們的預期.
超過我們所能夠控制.
這個是在提醒我們.
原來有一些事情我們過往習以為常.
覺得好像很理所當然.
但原來這些事情一點都不理所當然.
一切都是神的恩典.
如果神要我們離開一個我們過去熟悉的環境.
離開我們過去熟悉的生活方式.
祂必定會親自帶我們去到一個更豐盛的境地.
就正如昔日他呼召阿伯拉罕離開他本地本族父家.
其實神都讓阿伯拉罕失去了他習以為常的生活.
失去了他覺得最可靠的東西.
但神的心意是要帶領阿伯拉罕進入更豐富的英許之地.
今天我們都是一樣.
我們的環境可能有很多的改變.
我們會覺得不習慣.
會覺得很不舒服.
但如果當我們願意謙卑信服神的帶領.
我們都會經歷到.
其實神正是要帶領我們進入到更加豐盛的英許當中.
所以頂姐妹就用我們透過聖經的說話.
我們再一次回到神的面前.
去領受祂的英許.
讓聖經的說話幫助我們去察驗神的美善旨意.
好讓我們可以在神的恩典當中.
經過這個不一樣的嚴冬.
在當中我們去體會神在我們身上的美善旨意.
今天我選了一段經文和大家去看.
是路加夫音12章22-48節.
這段經文比較長.
所以在這裡我先不去讀整段的經文.
一會兒我一路說的時候.
我們會一起去看這些經文.
這段經文一開始說到耶穌向門徒作出一個教導.
這個教導是耶穌說不要為生命憂慮吃什麼.
為身體憂慮穿什麼.

$^{161}$其實這個教導這兩句經文我們都很熟悉.
可能很多頂姐妹都當為金句去背.
耶穌教導門徒不要憂慮.
在我們的生活當中.
其實我們難免會有叫我們憂慮的事情.
我們遇到很多事情.
有時候我們都會有忐忑不安的時候.
我們會有憂慮歸根究底.
原因就是很多事情的確不在我們的掌握當中.
就算我們多努力去做一件事情都好.
結果不到我們說了算.
我們的生活,工作,侍奉都是這樣.
很多事情不到我們控制.
又或者.
舉一個具體的例子.
如果當中有為人父母的.
我們大概都會很明白.
父母為子女去掛心去擔憂的那種心情.
我們中國人傳統一句說話.
養兒一百歲長憂九十九.
說到要憂慮.
其實很多事情可以讓我們憂慮.
長憂九十九.
所以耶穌這裡.
這個不要憂慮的教導.
或者我們說的吩咐.
其實對於我們每一個人來說.
其實都是一個很悉切的提醒.
我們在任何的環境.
任何的處境當中.
我們都要學會去交托.
學會去倚靠我們的主.
不過我們稍微停下來.
我們想一想.
是的.
憂慮是我們很經常會有的一種情況.
不過我們想一想.
我們過去通常叫我們憂慮的是什麼事情呢.
你過去的日子裡.
通常是什麼事情會讓你覺得有憂慮.

$^{201}$可能每個人憂慮的事情都有不同.
不過我相信其實我們大部分時間.
大部分時間那些叫我們憂慮的事情.
都不是一些會即時影響到我們生死的事情.
大部分時間我們憂慮的可能是關乎我們一些期望.
我們做的事情的結果.
是不是能夠符合我們的預期.
這個會叫我們憂慮.
但是有時候我們想深一層.
就算我們這些期望達不到.
其實這個是不是真的馬上就搞不定呢.
其實大部分時間都不是.
那些不是馬上會影響到我們生死的事情.
舉一個我自己的例子.
前幾個月我和太太忙於幫我們的兒子去報學校.
因為我兒子差不多兩歲了.
準備想幫他報一些學前班.
主要因為我和太太日間都要上班.
不能夠親自去照顧兒子.
平日是奶奶照顧他.
但是奶奶年紀大了.
照顧小朋友都辛苦.
所以希望如果他有學習的時候.
在照顧上就會方便很多.
但是在香港其實要報這些學前班不是那麼容易.
原因是學額其實不是太多.
競爭都大.
所以前幾個月我們一直在張羅去報這些學前班的時候.
我太太就很緊張.
她一直很擔心如果沒有學校收她怎麼辦呢.
她一直都很緊張.
不過坦白說.
我自己其實就不是很擔心.
我不擔心不是因為我不緊張兒子.
而是因為我很覺得.
就算他今年九月真的沒有學回來也好.
不是我們馬上就搞不定的.
我們的生活總有方法可以安排得到.
所以其實在整件事裡面.
我不是真的十分緊張.

$^{241}$因為我知道.
就算結果不符合我的預期.
都會有其他的方法.
不過最後感恩.
有學校收了我兒子.
他九月就有學回來了.
不過當然疫情的緣故.
九月是不是真的能開學.
還是未知之數.
說了這些例子.
其實我想說的是什麼呢.
耶穌這裡這個不要憂慮的教導.
是呀我們是很熟悉的.
我們過去的經歷裡面.
多或少總會對這個教導有些體會.
不過其實對於我們絕大部分的人來說.
可能在我們過去的日子裡面.
我們都未真正經歷到.
未真正體會到耶穌這裡這個教導的份量有多重.
因為我們過去憂慮的事.
其實往往都不是一些份量很重的事情.
不過在去年這個疫情的底下.
或許我們經歷一個我們以前未經歷過的困境.
而這個正正是一個機會讓我們更深刻.
去明白去經歷耶穌這裡的教導和應許.
我們要更深入去明白耶穌這裡的教導.
首先我們要記得.
當時耶穌對著群眾或對著門徒.
來講這些教導的時候.
那些聽耶穌教導的人.
那些百姓.
其實主要是一些低下階層的貧苦百姓.
用我們今天的話來說.
可以算得上是草根階層.
二千年前的社會沒有今天那麼富裕.
當然也不會有今天很多的社會的保障.
所以那些低下階層的百姓.
其實他們的生活是很艱難的.
他們當中有不少人.
他們過的是一些日薪的工作.

$^{281}$他們是每一天發工資的.
如果那天他有工作.
他就有工資發.
他就有飯吃.
用今天的說話來說.
就是手停口停.
而且當時的人工的水平其實是很低的.
通常他們每一天能夠賺到的工資.
大概只夠他那天的食用.
多一點也沒有.
所以他們要存錢其實也很困難.
很難說積谷房機.
因為每一天用完也沒得剩.
更困難的是.
他們這些人.
他們不單只是日薪每一天發工資.
有些人甚至是要每一天去找工作.
所以你記得.
耶穌曾經說過一個這樣的比喻.
就說到一個葡萄園主.
出去請一些工人回來.
幫他去收割葡萄.
早上出去請了一班工人回來.
跟他們說好了.
那天的人工是一錢的銀子.
其實一錢銀子在當時來說.
就是正常的工人一天的工資.
不會太多也不會太少.
大概就是足夠他或他的家庭那天的食用.
然後比喻繼續下去的時候.
耶穌說到那個葡萄園主.
差不多每兩個小時.
就出去再請一班工人回來幫忙.
如是者請了好幾班工人.
去到最後一次.
就是去到有初的時候.
大概是下午五時左右.
下午五時其實太陽差不多下山.
差不多下班了.
因為你記得以前沒有電燈.

$^{321}$太陽下山黑漆漆就做不到事了.
所以下午五時.
差不多是要下班的時候.
這個葡萄園主最後出去請最後一批工人回來.
如果你還記得這個比喻裡面.
這個葡萄園主見到.
最後這班工人在市場等人請.
葡萄園主就問他們.
為什麼你們整天在這裡閒站.
為什麼你們無所事事地站在這裡.
你可以想像得到.
其實這班人.
他們基本上是一班老弱殘兵.
沒什麼能力去工作.
所以沒人肯請他們.
事實上比喻裡面他們都是這樣回答的.
他們就回答這個葡萄園主.
因為沒人僱傭我們.
沒人請我們我們可以怎樣呢.
我們不站在這裡還有什麼方法呢.
當時其實他們基本上只有兩個選擇.
要不就去黑.
去行黑.
如果不想行黑的話.
就好像故事裡面說到.
他們繼續站在那裡.
用俗語說叫做等運到.
看看會不會有好心人來請他們.
就好像比喻裡面.
他們終於等到這個葡萄園主.
如果真的沒人請.
他們一下就去黑吃.
一下就要捱餓.
我們要記得當時耶穌身邊.
聽著他說道的那班群眾.
其實他們日常的生活就是這樣.
今天不知道明天的事.
存不到錢.
不能積穀防饑.
萬一有什麼不測.

$^{361}$萬一那天沒人請的話.
真的不知道怎麼辦.
你想想如果你的生活是這樣的時候.
你會不會憂慮呢.
我不知道你會不會憂慮.
不過我可以告訴你.
我一定會很憂慮.
今晚那頓飯吃完之後.
不知道明天有沒有飯吃.
怎麼可能不擔心.
當時的人他們就是這樣.
有一百個一千個一萬個的理由去憂慮.
但是耶穌卻是跟他們說.
不要為明天憂慮.
你能不能感受到在當時的人的耳中.
耶穌這個不要憂慮.
我們好像很熟悉的一個教導.
其實在他們的耳中.
這個教導有多麼的不合理.
其實耶穌這個教導是一個很不合理的教導.
怎麼可能不憂慮.
但是當他們經過了最初的這種震撼.
他們回想.
他們真的願意虛心去接受.
耶穌這個不要憂慮的教導.
其實他們就能夠經歷到.
耶穌這個教導.
這個吩咐所帶來的那種釋放.
耶穌說不要憂慮.
不是因為沒有東西需要我們去憂慮.
事實上剛剛相反.
是因為有太多的東西要讓我們去憂慮.
所以我們更加需要明白.
其實神是多麼的期望.
我們能夠在祂的裡面放下我們的憂慮.
在祂的裡面卸下我們一切的重擔.
這個是神.
是耶穌基督.
向我們這一群有太多理由需要憂慮的人.
發出的一個恩典的邀請.

$^{401}$耶穌吩咐我們不要憂慮.
當中一個很重要的原因.
耶穌說是因為我們的憂慮.
其實是幫不了我們.
這裡也是我們很熟悉的經文.
耶穌說你們哪一個能用思慮洗授數多加一克呢.
因為最少的事情你們尚且都做不到.
為什麼還要憂慮其餘的事情呢.
是的.
我們的能力就是這麼有限.
連讓我們受數多加一克這麼小的事情.
我們都做不到.
其他很多的事情.
其實更加不在我們控制的範圍裡面.
如果我們去憂慮的話.
其實這種憂慮背後正正反映的是.
我們希望可以掌握到我們的事情.
掌握到我們的將來.
我們希望事情能夠按照我們的期望去發生.
於是我們用盡方法嘗試去保證.
事情能夠按照我們的想法去實現.
如果我們能夠保證得到.
我們就不用憂慮了.
這是很多人或者我們自己都是.
我們很多時候會有這種想法.
如果我們能夠保證事情能夠成功.
就不用憂慮.
不過其實這樣只是自欺欺人.
因為事實上沒有什麼是我們能夠保證得到.
耶穌這裡這個不要憂慮的教導.
除了是一個安慰和一個應許之外.
其實也是一個當頭棒學.
祂要提醒我們.
不要以為你自己可以保證到什麼.
其實你沒有什麼是能夠保證得到的.
既然沒有什麼是能夠保證得到的時候.
你需要學的就是.
如何在神的裡面放下你的憂慮.
第二個原因.
為什麼耶穌說我們不應該憂慮.

$^{441}$這個原因可能我們更加熟悉了.
耶穌說.
「天上的飛鳥也不種也不收.
神尚且養活他,何況是我們?」.
之後耶穌再說.
「天野百合花,什麼都不做,.
但是神給他最美的裝飾.
甚至連所羅門最榮華的時候都比不上.」.
耶穌提醒我們.
為什麼我們不需要憂慮.
因為天父的恩典夠我們用.
不過我們要留心.
聖經說神的恩典夠我們用.
不是說我們想要多少就有多少.
不是說我們想要什麼樣的恩典.
神就會叫我們心想事成.
不是這樣的.
所以耶穌不是向貧苦大眾做一個保證.
不用擔心.
神會保守你.
明天一定有人請你.
耶穌不是做這個保證.
所以到了第二天.
如果當中有人沒有人請的時候.
他不能回來跟耶穌說.
「耶穌,不是啊,你昨天叫我不用憂慮.
為什麼我今天沒有人請?」.
耶穌說不要憂慮.
不是要保證你害怕的事不會發生.
耶穌要說的是.
無論我們環境有多困難.
無論我們有多少限制.
多少缺乏.
這無法阻礙神的能力.
神的恩典.
神的旨意在我們身上成就.
神的大能超過我們的缺乏.
就算事情不是按照我們期望發生.
就算白人第二天真的沒有人請.
神的恩典仍然不會離開他.

$^{481}$這是耶穌在這裡的教導.
是的,弟兄姊妹.
我們現在都在經歷很大的困難.
可能對很多人來說.
都是一個前所未有的困境.
今天可能那些叫我們憂慮的事.
比起我們過去的日子.
我們曾經有過的憂慮.
更加的嚴峻.
更加的迫切.
但是這正是一個更好的機會.
讓我們去經歷耶穌這裡的教導.
去明白這個教導的份量有多重.
當困境將我們迫到牆角.
無路可走的時候.
我們才真正能夠體會到.
我們平時說.
唯有神是我們的依靠.
而這句話是什麼意思.
弟兄姊妹.
我不是親身在你們當中.
所以我未必能夠很具體.
去明白你這一刻經歷的事.
是怎麼樣.
有什麼困難.
不過重要的不是我知不知道.
重要的是神知道.
你正在經歷什麼事情.
這一刻的困境.
還不至於說很困難.
這當然是值得感恩的事情.
但是如果你這一刻.
都在經歷一些很艱難的處境.
甚至可能去到一個朝不保夕的地步.
那我懇請你.
好好回到耶穌教導的裡面.
領受耶穌這裡的吩咐.
這裡的應許.
我們不要為我們的明天憂慮.
我們注目在神的恩典身上.

$^{521}$好讓我們能夠在主一裡面得釋放.
得平安.
不過同時我們又要留心另一邊.
耶穌這裡說我們不要憂慮.
他的意思也不是說.
我們過一個無憂無慮.
什麼都不用想的生活.
甚至是一個今朝有酒今朝醉的生活.
反正都不知道明天會怎樣.
不如今天我們就盡情去享受.
耶穌絕對不是這個意思.
我們不要忘記.
剛才我們說到當時耶穌身邊那群群眾.
他們實際生活的環境.
其實不容許他們為將來有什麼計劃.
剛才也提過.
其實他們要存錢也很困難.
更何況有其他的計劃.
在他們的處境當中.
這個很難.
所以耶穌說的是.
當你沒有能力去計劃的時候.
你就讓神帶領你走前面的路.
但耶穌不是否定.
如果我們可以的話.
其實我們也應該為我們的將來做好準備.
只不過重要的是.
我們為將來所做的準備.
不是出於憂慮.
不是出於我們想掌控我們的將來.
而是我們繼續在神面前.
存著一個謙卑信靠信服的心.
帶著這種謙卑信服.
我們好好去善用.
神今天讓我們所有的東西.
包括用這些東西.
為我們前段日子做準備.
其實某程度上來說.
這也是我們在神面前.
應當要盡的責任.

$^{561}$你記得在舊約創世紀.
說到約瑟在埃及地.
約瑟透過法羅夢.
他知道將會有七個的方年.
所以他在方年還沒到之前.
在那七個的豐年的時候.
他就儲備了足夠的糧食.
所以其實我們用神賜給我們的東西.
去為將來做準備.
這也是神的心意.
重要的是我們不是出於憂慮.
而是出於一種盡忠的心態.
不要憂慮.
不是說無牽無掛.
不是說今朝有酒今朝醉.
在這裡我們進一步去留心.
耶穌這裡說到不要憂慮.
其實祂說得更加具體.
耶穌說你不要為生命憂慮吃什麼.
為身體憂慮穿什麼.
因為生命勝於飲食.
身體勝於衣裳.
去到二十九節.
祂再說一次.
你們不要求吃什麼喝什麼.
也不要掛心.
所以我們在這裡看清楚耶穌的話.
其實耶穌不是很一般地說.
總之你完全不要憂慮.
其實耶穌說得很具體.
耶穌說我們不要憂慮.
我們不要為了什麼憂慮.
吃什麼 喝什麼 穿什麼.
耶穌說我們作為一個認識神.
作為一個跟從神的人.
這些不應該是我們憂慮的事.
耶穌說這些是外邦人所求.
世人會為了這些憂慮.
他們會擔心有沒有飯吃.
擔心穿什麼衣服.

$^{601}$但我們作為屬於神的人.
這些不是我們應該憂慮的事情.
我們應該為什麼憂慮.
應該為什麼掛心.
耶穌說生命勝於飲食.
身體勝於衣裳.
我們的生命.
我們的身體.
比起外在的飲食衣裳.
更加值得我們去掛心.
當然我們要留心.
這裡說的也不是我們要好好保養我們的身體.
保持身體健康.
青春常駐.
長命百歲.
不是現在說長命百二歲.
生命勝於飲食.
身體勝於衣裳.
其實這裡是用了希伯來詩歌.
傳統那種對句的方式.
就是說下半句.
身體勝於衣裳.
是重複上半句.
生命勝於飲食的意思.
這裡身體和生命.
其實說的都是同一件事.
就是神賜給我們整全的生命.
比起外在一切的東西.
其實神早就已經將最好的禮物賜給我們.
就是從神而來.
我們耶穌基督的救贖.
在聖靈帶領下的美善生命.
當然神也會將我們外在的需要賜給我們.
不過比起神已經賜給我們整全的生命.
其實外在的東西就變得不足掛齒.
神將這個生命賜給我們.
祂期望我們可以好好善用這個生命.
按著祂的心意去過一個蒙神喜悅榮耀神的生命.
這才是我們應該去掛心的事情.
所以到了31節.

$^{641}$耶穌就說得很清楚.
或者我們引用馬太福音的版本.
我們更加熟悉.
「你們要先求他的國和他的義」.
「這些東西都要加給你們了」.
所以其實重要的不是我們會不會有擔心.
事實上會有擔心.
這也是人之常情.
重要的不是我們有沒有掛慮有沒有掛心.
重要的是我們為著什麼而掛心.
我們懂不懂得去分辨.
什麼是對我們真正重要的東西.
其實這是耶穌這裡的教導更加重要的意思.
在論語裡面有一句說話我們熟悉的.
「人無遠慮必有近憂」.
我們傳統的智慧都告訴我們.
其實我們是需要有遠慮.
我們需要為我們將來的事去做好準備.
去思慮周全.
我們傳統的智慧都是這樣教導我們.
不過重要的是.
對於我們來說.
我們的遠慮是什麼.
我們為了什麼而去打算.
為了什麼而做準備.
這裡我想跟大家講一個例子.
一個故事.
關乎日本一個企業家.
叫做松下幸之助.
松下幸之助這個名字可能對你來說比較陌生.
你未必聽過.
他創立的公司松下電器.
可能你都不太知道這家公司.
不過其實這家公司的英文名字.
可能你會聽過.
其實松下電器就是Panasonic.
Panasonic在香港是一個很出名的品牌.
我不知道在北美有沒有這麼出名.
不過大概可能都聽過這個電器的牌子.
Panasonic這位創辦人松下幸之助.

$^{681}$他其中一件比較多人認識的事情.
就是當他創立這家公司的時候.
他為這家公司定下了一個計劃.
可能如果你在公司裡面工作.
你公司可能都會有一些的全年的計劃.
五年的計劃.
十年的計劃.
或者教會我們有時候都會定下這樣的計劃.
不過松下幸之助為他的公司定下的是什麼計劃呢?.
不是五年計劃.
不是十年計劃.
你猜猜他定下的計劃是一個多少年的計劃?.
說出來可能你會嚇一跳.
他為他的公司定下了一個250年的計劃.
250年.
你想想250年的計劃.
讓你想得出來都很難做得到.
讓你自己很有毅力很有堅持.
你堅持一生的人去到為這個計劃這個目標奮鬥.
但是250年你本人肯定過了生.
你的繼承人是不是真的可以繼續這麼有毅力去堅持.
其實很難保證的.
又或者就算你的繼承人都是這麼厲害.
但是你想一下Panasonic是電器公司.
是隨著科技的發展受到科技的影響.
今時今日科技日新月異.
幾十年前能夠想到的事.
來到今時今日都已經完全不同了.
所以250年的計劃不單是誇張這麼簡單.
甚至有些異想天開.
不過耶穌要我們關心的事情.
遠遠超過250年的計劃.
耶穌說我們要求神的國.
神的國是直到主在來的日子.
什麼時候會發生.
聖經說哪日子哪時辰沒有人知道.
可能很快也可能還要等一段時間.
因為聖經說主看千年如一日.
一日如千年.
我們不知道神的國什麼時候來到.

$^{721}$但是耶穌提醒我們.
這個神的國才是值得我們去掛心.
值得我們今天要去做好準備的事情.
所以再說.
耶穌這裡的教導不單止提醒我們.
我們可以將我們的憂慮去卸給神.
不單止提醒我們神的恩典夠我們用.
神會供應我們的需要.
當然這些都是真的.
但是耶穌這裡的教導其實更加是要我們好好去問問我們自己.
我們今天是為了什麼而生活.
為了什麼而做準備.
耶穌吩咐我們不要憂慮.
其實是在挑戰我們.
挑戰我們要將眼光放得更高更遠.
耶穌很期望我們能夠明白.
什麼才是對我們真正重要.
耶穌期望我們不要將生命浪費在一些.
其實不值得我們努力不值得我們憂慮的事情上.
耶穌期望我們為神的國去祈求.
為神的國去努力.
為神的國去掛心.
雖然沒錯我們的生活上仍然有很多事情.
很多問題我們未能解決.
甚至我們的生活可能仍然是朝不保夕.
但是你要記得這個不單止我們是這樣.
剛才說到耶穌當時身邊的百姓都是這樣.
你留意32字這裡耶穌稱呼他們為小群.
當時這群老百姓都是一群很弱勢的人.
他們沒有能力為自己做什麼.
沒有能力為自己爭取什麼保障.
但是耶穌對他們說.
你們這小群不要懼怕.
環境或者不能夠有很大的改變.
我們也未必有能力解決所有的問題.
但是我們的眼光.
我們的心卻是可以改變.
我們可以選擇.
選擇我們為什麼而掛心.
所以從這個不要憂慮的教導.

$^{761}$經文就帶到下一段.
關於警醒的渲染.
從《路加福音》這個鋪排.
我們看到其實我們不要憂慮.
我們對神的信靠.
和我們在神面前的警醒.
其實是分不開的.
當我們懂得不憂慮的時候.
我們才真正能夠過一個警醒的生活.
反過來.
其實我們的警醒.
就是我們在神裡面不憂慮.
一個很具體的方式.
《路加福音》來到33節.
而將33節到48節.
這裡記載了耶穌一連串的吩咐當中.
包括好幾個的比喻.
都是圍繞著警醒這個主題.
首先第一個我們看到的吩咐.
是33節這裡.
「你們要鞭賣所有的州制人.
為自己預備永不壞的前浪.
用不盡的財寶在天上」.
當時那些百姓.
剛才提過好幾次.
他們生活很艱苦.
今天不知道明天會是.
辛辛苦苦可能死慳死抵.
每一天剩下一點點的剩餘工資.
可能稍微儲到一點點的積蓄.
就好像你記得瑪利亞那瓶香膏.
可能也要儲很久才能儲到回來.
生活這麼艱苦.
說到要州制窮人.
排隊都還沒輪到他們.
可能他們自己都需要.
靠其他人去州制他們.
但是耶穌這裡的吩咐.
「你們要鞭賣所有的州制人」.
其實正正就是向著這群貧苦的百姓.

$^{801}$向他們去說.
耶穌說到鞭賣所有的州制人.
不只是跟有錢人說.
是跟所有人說.
跟這群生活很艱難的人說.
自己生活都這麼艱難.
為什麼還要州制其他人.
原因很簡單.
因為我們真正需要的財富.
不是我們能夠死省死抵儲到的有限積蓄.
而是耶穌說.
神在天上為我們預備那些用不盡的財寶.
其實這裡州制窮人的吩咐.
不只是耶穌說的.
從舊約到新約.
一貫都有州制貧窮人的教導.
其實在神原初創造的世界裡.
按著神的心意.
是不應該有貧窮人的.
因為神的供應.
神的恩典是絕對的豐足.
不過可惜的是.
當人犯罪墮落之後.
世界就被罪破壞.
世界裡充滿了很多不公義欺壓的事情.
帶來其中一個結果.
就是出現貧窮人.
很多人落在貧窮受欺壓的境況當中.
所以在舊約裡.
當神拯救以色列人.
跟他們納約的時候.
神也吩咐以色列人.
要關顧.
要幫助他們中間的貧窮人.
來到新約的教會.
神同樣吩咐我們.
要去進行這個救濟窮人的吩咐.
當我們去救濟我們身邊的貧窮人.
這其實就是我們向這個世界作見證.
一個很重要的方式.

$^{841}$耶穌基督的救贖恩典不單止告訴我們.
將來可以有永生可以上天堂.
這個救贖恩典更加是要.
讓這個世界脫離罪所帶來的敗壞.
其中就是因著罪的緣故.
讓人落在貧窮的境況當中.
我們透過救濟窮人告訴這個世界.
神的心意是要改變.
這些罪帶來的破壞.
所以在新命紀15章4節.
當摩西說到這個救濟窮人的吩咐的時候.
他也這樣跟以色列人說.
「你若留意聽從耶和華你神的話.
謹守進行我今日所吩咐你這一切的命令.
就必在你們中間沒有窮人了」.
這是神最終的心意.
我們也是透過這個方式.
向世界作見證.
見證福音的真實.
當然可能在我們今天這個社會環境當中.
我們未必可以完全按照字面去進行.
耶穌這裡的變賣所有的吩咐.
我想我們當中沒有誰是散盡家財.
在中世紀的時候.
有些修道的傳統.
他們真的會按照字面的方式去進行.
真的散盡家財.
進入修道院當中.
過一個類似行乞的生活.
今天的社會環境未必容許我們這樣去做.
也不必然需要這樣去做.
不過其實主的吩咐沒有改變.
我們要追求的.
對我們重要的財富.
不是我們地上能夠有的積蓄.
我們需要追求的是天上的財富.
用什麼方法追求.
就是盡我們所能去周濟我們身邊的窮人.
不過這裡又帶出另一個問題.
說到周濟窮人.

$^{881}$但是我們放眼世界.
世界貧窮的狀況.
是一個很大的問題.
或者你不要說世界這麼大.
就說北美.
只是在北美裡面.
可能已經有很多很多的貧窮人.
需要我們的周濟.
面對著一個這麼大的需要.
我們可能都會覺得.
我們能夠做到多少呢.
我們盡力去做.
都只不過是杯水車薪.
幫不了很多.
於是我們自然會問.
我們必須要去問.
在我們有限的資源底下.
我們怎樣去用這些資源.
用在哪裡幫助哪些人.
其實一個很類似的問題.
聖經裡面也有人問過.
這段經文我們又很熟悉了.
好撒瑪利亞人的比喻.
為什麼耶穌會說這個好撒瑪利亞人的比喻.
你記得經文的上文是說到.
當時有個律法師去問耶穌.
最大的誡命是什麼.
耶穌反問了他.
然後律法師說.
是愛神和愛靈社的誡命.
其實律法師也很清楚.
但接著這個律法師就問耶穌.
那誰是我的靈社.
這個律法師會問這個問題.
誰是我的靈社.
原因是他帶著當時猶太人.
一個很傳統的觀念.
就是一種親疏有別.
敵我分明的觀念.
在他們的觀念當中.

$^{921}$有些人是我們的靈社.
那些人我們當然要去愛.
但有另一些人.
他不單止不是靈社.
更加是我們的敵人.
甚至不只是我們的敵人.
更加是神的敵人.
神的敵人我們沒什麼理由去愛.
這是當時猶太人他們有的.
親疏有別.
敵我分明的心態.
所以很自然這個律法師就問.
誰是我的靈社.
換句話說.
哪些是我的敵人.
耶穌沒有很直接去回答這個問題.
耶穌說了這個很撒瑪利亞人的比喻.
這個比喻.
耶穌專用一個撒瑪利亞人.
成為了主角.
成為了英雄.
正正就是針對這個律法師.
這個問題背後的那種敵我分明的心態.
因為在當時的猶太人心目中.
一個撒瑪利亞人.
不可能是他們的靈社.
他們之間太多的矛盾.
太多的衝突.
太多的牙齒印.
但是耶穌要打破這種敵我分明的心態.
原來在神的眼中.
誰是我們的靈社.
這個不是受著種族.
文化.
歷史.
甚至宗教的因素去限制.
就算那個人是你的世仇.
就算你和他當中之間很多的牙齒印也好.
又或者不要說得那麼嚴重.
可能那個是一個陌生人.

$^{961}$你不認識他.
大罵也搞不定.
但是當你遇見這個有需要的人.
當神將他擺在你面前.
這個就是你的靈社.
就正如比爾裡面這個撒瑪利亞人.
遇見一個三不識七的受了傷的人.
撒瑪利亞人就成為這個受傷的人的靈社.
其實這樣的教導也不是耶穌第一個說的.
在舊約的律法裡面已經有這樣的吩咐.
塞俠的23章4-5字裡說.
若遇見你受敵的牛或驢失迷了路.
總要牽回來交給他.
若看見恨你的人的驢鴨在重駝之下.
不可以走開.
「勿要和驢主一同抬開重駝」.
就算是仇敵.
就算是恨你的人也好.
當神將這個有需要的人放在你的面前.
他就是你的靈社.
他就是神要你去愛的人.
你發覺從舊約摩西的律法.
到新約耶穌的教導.
都是很一致.
有時候我們讀到路加芬第十章.
那個殺瑪利亞人的比喻.
我們會對那個律法師的印象很差.
覺得這個律法師很律法主義.
雞蛋裡挑骨頭.
講愛靈社還要問誰是我的靈社.
但是我們回頭看看我們自己.
在我們華人的文化裡.
其實我們一樣有一種很強的.
親疏有別.
敵我分明的觀念.
特別當我們人在外地.
我們身邊如果能夠有一些同聲同氣的人.
互相去守望互相去幫助.
這個也很重要.
所以我們很自然會和.

$^{1001}$我們眼中的自己人比較親近.
我們會關心他們多一點.
可能一些跟我們關係沒那麼密切的人.
我們也很自然對他們不會那麼著緊.
其實這個也是很正常的事.
其實沒什麼大問題.
不過作為一個基督徒.
作為一個基督徒.
當我們實踐聖經裡的愛靈社的吩咐的時候.
我們想到的是什麼人.
我們是不是只想到.
那些跟我們同聲同氣.
跟我們關係密切的人.
其實老實說.
這些跟我們同聲同氣.
關係密切的人.
就算聖經不吩咐你.
你也會關心他們.
你也會很自然去幫助他們.
不過聖經的吩咐提醒我們.
我們的靈社.
我們要去愛.
我們要去幫助的.
不單只是我們覺得很自然.
我們會去關心這些人.
特別當我們身處一個.
很多元文化的社會當中.
大家身處北美.
也是很多元文化.
很多不同的種族.
當中可能華人也是佔少數.
但正如耶穌昔日對律法師的吩咐.
今天耶穌也是這樣吩咐我們.
你就照著比喻裡的.
這個撒瑪利亞人所做的去走.
不需要問對方是誰.
只要你遇到你身邊一個有需要的人.
你就盡你所能去幫助他.
去成為他的好靈社.
這是聖經給我們的提醒.

$^{1041}$我們回到路加福音12章.
說完33-34節.
這個「周濟窮人積財在天」的吩咐後.
路加一連記載了幾個關於警醒的比喻.
第一個是35-38節.
這是一個僕人的比喻.
裡面說到當一個僕人警醒.
他不會偷懶.
當主人從婚宴上回來.
那個僕人就立刻去開門的時候.
這個僕人就要得到賞賜.
第二個比喻是39-40節.
一個盜賊的比喻.
說到因為我們不知道盜賊什麼時候會來.
所以我們要經常隨時做好準備.
第三個比喻也是一個僕人的比喻.
在42-48節.
這個比喻說到當主人出門離開了一段時間.
那些僕人有沒有盡忠心去盡他當盡的責任.
這個就決定當主人回來的時候.
這個僕人會經歷什麼待遇.
他到底是要得到主人的獎賞.
還是要面對主人的責罰.
這三個比喻的共通點.
都是說到我們要警醒等候主.
在這裡我不跟大家詳細逐個逐個比喻去看.
反而我想跟大家特別留意一節經文.
就是41節.
這一節經文是彼得問耶穌一個問題.
「這比喻是為我們說的,還是為眾人?」.
彼得問耶穌這個問題.
然後耶穌回應彼得.
就是說了42-48節的比喻.
在這裡我們第一件留意到的是.
其實耶穌接著說的比喻.
似乎沒有直接回答彼得的問題.
因為彼得的問題很簡單.
只有兩個選擇.
A 這個比喻是為我們說的.
B 這個比喻是為眾人說的.

$^{1081}$要回答這個問題很簡單.
你一下選A 一下選B.
但是耶穌的回答卻沒有選擇.
是為我們說還是為眾人說.
甚至我們讀完整個耶穌的比喻.
可能我們還沒有弄得很清楚.
其實是為我們還是為眾人呢?.
還是耶穌根本不是在回答彼得呢?.
耶穌的答案似乎和彼得的問題不太相配.
有點像牛頭不對馬嘴.
但是我們回頭留意這個問題本身.
你想清楚一點.
其實彼得的問題是有點奇怪的.
其實當時耶穌身邊有很多人.
在聽耶穌的教導.
按著路加福音12章.
一直說的那個場景.
這裡不是說耶穌進了內室.
只是門徒在祂身邊.
是耶穌對著祂身邊的十二門徒說話.
如果是這樣的話.
其實彼得不需要問這個問題.
因為很清楚.
耶穌就是對著門徒說.
彼得要問這個問題.
其實耶穌身邊當時也有很多群眾.
但是為什麼彼得要問.
耶穌說的話是故意告訴我們.
還是告訴群眾呢?.
打個比喻就好像.
你們主日崇拜有個講員來講完編度.
接著有人上台問那個講員.
講員先生.
你講這篇道是講給所有會眾聽.
還是只是講給教務同工聽?.
你想想你也會覺得這個問題問得很奇怪.
我是正常不會這樣問的.
但是彼得就是問了這個問題.
彼得要在這麼多群眾當中去分開.
到底耶穌的教導是講給誰聽.

$^{1121}$為什麼彼得要這樣問.
我們對比彼得的問題和耶穌的回答.
其實這就揭露了彼得這個問題.
背後一些想法和動機.
我們留心兩個僕人的比喻.
42到48節一個僕人的比喻.
和前面35到38節都是一個僕人的比喻.
兩個比喻有些很相似的地方.
都是說到僕人要警醒.
要忠心去侍奉主人.
他們就能夠得到獎賞.
但是後面的比喻和前面的比喻.
也有一些很不同的地方.
第一個不同的地方是後面的比喻.
特別提到的是管家.
不是一般的僕人.
而是一個有些地位的管家.
第二個不同的地方是後面的比喻.
除了講到獎賞之外.
其實你留意到更多的篇幅.
是關乎到責罰.
而前面35到38節的比喻.
其實根本沒有提到責罰.
所以從這裡我們看到一些蛛絲馬跡.
到底41節彼得的問題.
其實彼得想問的是什麼呢?.
彼得這個問題反映了.
其實在他的心目中.
他作為耶穌身邊最貼身的門徒.
他覺得他理所當然.
是和眾人有分別的.
如果我們套用彼得在另一處地方.
曾經說過的話.
彼得曾經問耶穌.
我已經撇下所有的跟從利.
我們要得什麼獎賞?.
彼得和當時耶穌身邊的門徒.
他們都很著緊這件事.
他們有什麼獎賞?.
他們作為和耶穌關係最密切的人.

$^{1161}$他們覺得他們會比其他群眾更加優勝.
更加配得到更大的獎賞.
而前面35到38節的比喻.
正正就是講獎賞.
那個中生的僕人要得獎賞.
而且這個獎賞是很誇張的.
耶穌說.
主人必要他們坐直.
自己束上帶進前侍候他們.
你要明白.
在當時的文化裡.
這件事基本上是不能夠想像的.
當時因為是一個很階級分明的社會.
沒有人能夠想像得到主人親自去侍候那些僕人.
一個很誇張的獎賞.
彼得聽到這個.
他立刻眼睛都開了.
按照聖經一貫給我們對彼得的認識和印象.
我們也投入一下彼得當時的心境.
彼得聽到這個關於獎賞的比喻.
他立刻就會想.
按照我作為耶穌的大弟子.
應該是最配得這些獎賞的人.
接著可能他進一步就會想.
還有多少人可以得到這個獎賞呢.
如果太多人一起得到就不慶祥了.
如果老師頒獎.
全班同學都有.
那沒什麼特別.
如果只是頒給其中一個兩個同學.
這樣就很了不起了.
彼得這個問題背後是一種驕傲.
耶穌看穿了彼得心裡在想什麼.
所以耶穌用一個比喻去回答彼得.
耶穌再一次肯定.
沒錯 忠心的僕人是會得到獎賞.
這是一個很確實的應許.
但耶穌同時說出圖畫的另一面.
當忠心的僕人得獎賞的時候.
其實一個不忠心的僕人.

$^{1201}$是要面對責罰.
如果你真的覺得你比其他人更加優勝.
你真的覺得你是耶穌最貼身的門徒.
比其他人更明白主的心意.
那請你記得.
同時你要承擔更重的責任.
所以耶穌最後說.
「僕人知道主人的意思卻不預備.
又不信他的意思行.
那僕人必多受責打」.
反過來.
如果一個本身不知道主人意思的人.
做了當受責打的事.
反而是要少受責打.
所以沒錯.
我們是應該盼望主的獎賞.
這是主給我們一個很確實的應許.
這個獎賞的盼望.
讓我們無論在地上經歷多少艱苦.
為著主的緣故受到多少逼迫.
我們仍然能夠因為主的應許.
我們能夠堅持下去.
不過同時我們要小心.
不要像彼得一樣.
將神的獎賞.
將神的應許.
變成了一種我們互相比較的事情.
就好像聖經記載.
昔日耶穌身邊的門徒.
他們互相爭論誰偉大.
但給我們的應許.
這些是獎賞.
不是要讓我們沾沾自喜.
反而是提醒我們.
我們需要比其他人更加警醒.
更加忠心.
這就是路加福音提醒我們.
在信靠當中的警醒.
我們一方面.
完全將我們自己投靠在神的恩典裡.

$^{1241}$同時我們因著我們領受的恩典.
我們警醒過一個合神心意的生活.
當我們身處的環境充滿了挑戰.
我們需要信靠.
我們也需要警醒忠心.
這樣我們才能夠在困境當中.
繼續成為耶穌的見證.
警醒的吩咐.
不是要給我們很大的壓力.
他不會給我們很大的壓力.
因為在耶穌基督的裡面.
神早已將最大的恩典.
最大的倚靠賜給了我們.
還記得耶穌說.
「你們這些小群不要懼怕」.
「因為你們的父樂義將國賜給你們」.
你們只要求他的國.
這些東西就必加給你們了.
當我們信靠神的應許.
我們就能夠真真正正放下地上的憂慮.
我們不再問吃什麼.
學什麼.
穿什麼.
我們能夠去關心.
我們真正應該真正需要去關心的事.
就是我們為了神的國而掛心.
為了神的國而向神祈求.
就算我們面前仍然是風雪漫天.
當我們真的願意按照主的吩咐.
基於我們對主的信靠.
我們在主的裡面.
保持警醒 忠心 謙卑 信服.
我們必定能夠得到主所應許的賞賜.
願主繼續帶領我們走前面的路.
多謝您.
(字幕製作:貝爾).
未经许可,不得翻唱或使用.
\newpage



\section{}
\label{sec:zyLpufeGBYs}
\textbf{中神北美培靈講座}
\newline
\newline
連結: \href{https://youtube.com/watch?v=zyLpufeGBYs}{\texttt{ https://youtube.com/watch?v=zyLpufeGBYs}} ~~~~ 語音日期: 2016-05-20 
\newline
\newline
\hyperref[sec:_3eTxXlzPX0]{\small{< < < PREV SERMON < < <}}
~
\hyperref[sec:index]{\small{[返主目錄]}}
~
\hyperref[sec:2TEwldoXzT8]{\small{> > > NEXT SERMON > > >}}
\newline
\newline
$^{1}$(廣播中).
各位好.
歡迎大家來到我們北美.
忠臣的北美培靈講座.
合神心意的侍奉.
我們首先歡迎多倫多各位嘉賓.
和主內的弟兄姊妹.
但我亦歡迎在北美有19個城市.
27個聚會點的嘉賓和弟兄姊妹.
我們今晚的聚會是會直播和轉播.
去這27個聚會點的.
我也在這裡歡迎我們今晚的講員.
我們中國神學研究院的院長.
李思敬牧師博士和李師母.
他們遠道在香港來到.
李牧師是我們今晚的講員.
合神心意的侍奉.
在開始之前.
我請我們城北華人基督教會的.
副主任牧師林志輝牧師.
將我們今晚的聚會交託給天父上帝.
(主席).
請我們一起起來禱告.
我們一起低頭禱告.
親愛的天父我們感恩多謝你.
你帶領我們.
你亦與我們同在.
你將你的話語時常的.
來釋放給我們.
我們感恩.
每一次我們讀你的話語時.
我們心裡得著興奮.
得著歡喜.
因為你的話語充滿能力.
主啊求你亦教導我們.
將你的話藏在心裡.
免得我們得罪你.
主啊今天晚上我們感謝你.
因為讓我們在這裡有.
李思敬院長給我們的一個培靈會.

$^{41}$主啊求主你教導我們.
在今天晚上裡面.
聽到你的說話.
你的靈也在我們當中運行.
打開我們的心.
以致我們能夠好好的聆聽.
怎樣來土神的喜悅.
怎樣來敬拜主你.
怎樣來為主你服侍.
主啊求主你今天晚上.
帶領我們每一個人.
也給我們在這個地方裡面.
接受主你的祝福.
天父我們也求主你給李院長.
他有好的身體.
好的聲線.
我知道他從香港過來.
可能需要有少許的適應.
求主你堅強他.
給他力量.
我們求主也都今天晚上.
在這個地方裡面.
賜福給我們每一個人.
以致我們能夠更加知道.
怎樣來好好的聽你的話語.
土神的喜悅.
來侍奉主你.
求主你恩代.
祈禱奉主耶穌基督命求.
阿門.
各位請坐.
我們請我們的.
Worship Team帶領我們敬拜主.
各位多人多愛的弟兄姊妹平安.
各位海外的弟兄姊妹平安.
今天晚上我們能夠跨越時空.
在神的裡面我們一起來敬拜.
實在是神的恩典.
所以這一刻無論你是在哪一個地方.
讓我們一同的恭敬的來祈禱.

$^{81}$我們懷著感恩快樂的心.
一起來讚美我們的天父上帝.
祂就是那位聖潔尊貴的主.
我們一起讚美祂.
神美聖潔.
道德尊貴.
來生歡呼拍掌.
救我傳頌你.
我的主.
神你帶領.
因點經過.
來心感恩喝彩.
今天上位你.
屬於主.
我要每天.
每刻.
都贊同你施神.
你是有念.
你受上血念.
令我會有上身.
我要每天.
每刻.
都贊同你施神.
念出一生.
有與主你守中.
念出一生.
有與主你守中.
今次請姊妹.
神美聖潔.
道德尊貴.
來生歡呼拍掌.
救我傳頌你.
我的主.
神你帶領.
因點經過.
來心感恩喝彩.
今天上位你.
屬於主.
我要每天.
每刻.

$^{121}$都贊同你施神.
你是有念.
你受上血念.
令我會有上身.
我要每天.
每刻.
都贊同你施神.
念出一生.
有與主你守中.
念出一生.
有與主你守中.
有我們同心低求禱告.
我讚愛主.
我們多謝你.
因為你的恩典賞與我們同在.
主你的慈誠愛塑.
每一日都牽引著我們.
你的愛從來不會離開我們.
天王我們感謝你.
此刻你的獨生兒子.
耶穌基督寶貴的救贖.
我們透過耶穌的救贖.
我們這群原本不配的罪人.
我們能夠與你和好.
並且能夠被稱為.
你的寶貝兒女.
主啊我們實在感謝你.
讚美你.
天王上帝.
每當我們每一次去認真.
細想你對我們那份徹底的.
那份無條件的愛的時候.
我們豈能不去開口.
感謝讚美你.
我們豈能不把我們的身心.
都獻情給你呢.
天王就求你幫助我們.
引領我們每一個人.
能夠過一個謙卑自竭的生活.
求你的話語.

$^{161}$叫我們的腳步穩健.
不容許罪孽克制我們.
主啊又求你.
無助我們.
叫我們每一個人.
都是能夠成為.
合你心意的器皿.
能夠成為聖潔.
合乎主用.
叫我們的行事為人.
能夠令你的命.
得到榮耀.
讓你的心.
得到滿足.
主啊我們就是這樣.
恭敬的在你的面前祈求.
禱告奉耶穌基督.
得聖的名字而求.
Amen.
神美聖潔.
多德尊貴.
來生歡呼拍掌.
由我傳送你.
我的主.
神美大明.
點點營火.
來心感恩喝彩.
求現上為你.
贖天主.
我要每天每刻.
都贊助你是神.
你是恩典.
你手上應有的我沒有信心.
我要每天每刻.
都贊助你是神.
願出一生.
交與主們相助.
我要每天每刻.
都贊助你是神.
你是恩典.

$^{201}$你手上應有的我沒有信心.
我要每天每刻.
都贊助你是神.
願出一生.
交與主們相助.
願出一生.
交與主們相助.
願出一生.
交與主們相助.
Amen.
各位請坐.
今天晚上我們很高興能夠有中國神學研究院的院長李思敬博士在我們當中為我們分享神的說話.
用我們恭敬的一張時間交給李思敬院長.
請我們以熱烈的掌聲來歡迎李思敬博士.
各位弟兄姊妹主內平安.
不單止是多倫多.
準確一點是Richmond Hill.
其實今晚是中晨.
在加拿大和美國.
我們有不同的聚會點.
過去多數是生命之道.
今晚還有一兩個地方.
是在未曾舉行生命之道的聚會點.
我們有中晨北美的培靈聚會.
我們今晚的題目.
合神心意的侍奉.
如果我問大家.
在聖經裡面.
最集中地出現.
侍奉這兩個字的經文.
不知道你們會想到.
是記載在哪裡.
如果你以為是新約.
就不對了.
在舊約聖經.
你看到我站起來.
都應該是講舊約的.
生命之道的弟兄姊妹讀過的.
約書亞記二十四章.
我讀給大家聽.

$^{241}$如果你有聖經.
你都可以打開.
約書亞記二十四章.
我們要讀的是十四節到十五節.
短短兩節的經文.
你心水清.
讀的時候.
你數一下.
侍奉這兩個字.
總共出現了多少次.
約書亞對百姓說.
現在你們要敬畏耶和華.
誠心實意地侍奉他.
將你們列祖在大河那邊.
和在埃及所侍奉的神除掉.
去侍奉耶和華.
若是你們以侍奉耶和華為不好.
今天就可以選擇所要侍奉的.
是你們列祖在大河那邊.
所侍奉的神.
是你們所住這地阿摩利人的神.
至於我和我家.
我們必定侍奉耶和華.
可能你掛著數.
聽不到約書亞說什麼.
總共多少次.
七次.
不是每一次都說侍奉耶和華.
有兩次說侍奉其他的神.
不過在短短兩節.
約書亞所說的這一段說話的總結.
他不停提到的是侍奉.
當然他要說的是侍奉耶和華.
他自己也很清楚.
跟百姓說明他的立場.
至於我和我家.
我們必定侍奉耶和華.
As for me and my house.
We will serve the Lord.
英文的文法.

$^{281}$通常我們後面應該是.
We shall serve the Lord.
但是用了個will字.
就是必定.
這個語氣在希伯來文是很清晰.
約書亞的結論.
可能你家牆上都會不會有個牌.
英文的.
不是啊.
我小時候在我家的客廳.
有中文無筆字寫著.
我家必定侍奉神.
我小時候以為是我父親寫的.
長大之後才發現原來不是他的墨寶.
不過一直掛在客廳的中間.
在我腦海當中留下很深刻的印象.
重複不是偶然的.
約書亞的第一章.
耶和華上帝和約書亞三次重複.
剛強壯膽.
然後去到第一章總結.
百姓和約書亞說.
你只要剛強壯膽.
重複四次.
我們就很清楚看到.
約書亞的開宗明義.
第一章重點主題.
剛強壯膽.
同樣重複的技巧.
在最後這一章二十四章出現.
約書亞兩節短短的說話七次.
你現在不要數.
回家再讀.
接著由第十六節.
一直記載到百姓最後回答約書亞說二十四節.
我們必侍奉耶和華.
又是另外七次的重複.
還不夠.
去到這一章結束之前.
三十一節.

$^{321}$約書亞在世.
和他死後的長老還在的時候.
百姓以色列人侍奉耶和華.
短短半章的經文重複十五次.
這個很明顯是約書亞的二十四章.
或者說約書亞的這一卷聖經總結的時候.
他要強調他要突出的焦點.
侍奉原來早在舊約聖經.
早在入迦南約書亞林中.
對百姓所說的這一番說話.
壓軸的高潮.
結論的部分.
侍奉耶和華.
不過我們要指出一個很重要的事實.
就是昔日.
以色列人聽這句說話.
和今天我們坐在這裡聽這句說話.
是完全不同的.
為什麼呢?.
很容易的.
今天我們說侍奉是什麼意思?.
很自然.
我們所說的是在教會的侍奉.
不是我出去短宣.
仿宣離開教會四步牆.
但你還是在做傳福音.
宣教.
或者你在教會的群體弟兄姊妹一起.
你會說侍奉.
明天上班.
你不會和公司的同事說.
我今天侍奉.
雖然在神學上我們可以討論.
你的工作.
你朝九晚五.
那個都是侍奉.
但你不會用這兩個字.
這兩個字幾乎成為了信徒彼此之間.
才明白的術語.
cliché.

$^{361}$jargon.
好像我們時時說.
回到屯溪有很美好的交通.
你的意思是今天不塞車嗎?.
不是.
我剛才來塞了兩個半小時.
在Mississauga開出來走路.
交通.
一般人的印象一定是說traffic.
但是弟兄姊妹這個用詞是說屯溪.
所以有些詞彙.
有些術語.
有些觀念.
是我們用慣了不覺的.
侍奉.
就是你在教會有什麼侍奉.
敬拜隊的侍奉.
不是,門口招待的侍奉.
派聖餐的侍奉.
全部是說聖公.
是和上帝的教會直接有關係的.
這個是侍奉.
昔日.
以色列人聽見約書亞說.
侍奉耶和華.
希伯來文這個字其實很普通.
絕對不是一個只是在會幕裡面.
或者在聖殿裡面.
或者在會堂裡面所用的詞彙.
希伯來文這個動詞.
說的是服侍.
說的是作奴僕.
再說直接一點.
在以色列人的心目當中.
他一聽到這個動詞.
不是聽到在教會侍奉.
他是聽到在埃及做奴隸.
服侍法奴.
是同一個字.
埃及法奴怎樣奴役他們.

$^{401}$你在主教學長大.
你讀過出埃及記.
你知道.
人多.
那些男嬰.
弄些手腳.
就可以弄死他們.
抱怨.
說虐待.
不給材料.
自己去找草.
但是要交同樣數量的磚.
來給法奴建造兩座積火城.
你讀出埃及記.
你很熟悉.
法奴.
怎樣奴役百姓.
在約書亞說這番說話的時候.
大概印象猶新.
所以.
百姓聽到約書亞這樣說的時候.
大概浮現出來的第一個感覺.
不是回到禮拜堂.
帶茶經.
做早漲.
不是不是不是.
他想起的是以前.
給一個暴君.
怎樣壓制.
怎樣zar 乾zar 盡.
你怎樣用牙膏.
用到最後你捲不捲起它.
有一次.
有個弟兄告訴我.
我要剪開它.
還可以用多一次兩次.
侍奉上帝.
是不是都是這樣.
上班下班.
已經很辛苦了.

$^{441}$還有家裡開門七件事.
哎呀.
很多的壓力.
回到教會.
還要講侍奉.
上帝是不是好像法奴.
這樣對待他的百姓.
所以約書亞記第24章.
他不是一開始就講侍奉耶和華.
剛才我們所讀的是14節.
你知道第一到第十三節講什麼嗎?.
約書亞叫百姓來到.
不是第一句就跟他們講侍奉耶和華.
不是不是.
如果這樣講的話.
百姓浮上來.
想得到的.
他們的記憶充斥著的.
就是埃及遺奴的經驗.
不過現在換了一個主人.
都是這樣對待我們.
所以不知道你有沒有留意到.
約書亞講侍奉耶和華之前.
他用了大半的篇幅.
由第二節開始到第十三節.
他是首先跟以色列人重溫歷史的事實.
不是這樣講很沉悶.
講歷史?.
要記年份.
要背人名.
要背地名.
還要用英文考會考.
怎麼串的.
現在不用了.
DSE是吧?.
不是講歷史.
是介紹給百姓重新認定.
耶和華是怎樣的一位上帝.
你聽聽.
約書亞在這裡講的是.

$^{481}$以色列人三段.
他們所知道.
甚至他們有份經歷的事實.
第一部份是上帝呼召阿伯拉罕.
從大河那邊來到迦南地.
二十四章第三節.
上帝用第一身單數的代名詞.
「我」.
來講.
我將你們的祖宗阿伯拉罕.
從大河那邊帶到來.
領他走遍迦南全地.
是我將他的子孫洗他們眾多.
是我」.
耶和華說.
「將以撒賜給阿伯拉罕.
又將雅各和以素馬省兄弟賜給以撒.
又將西以山賜給以素遺孽」.
所以不是以素貪愛世界.
離開了應許地迦南.
自己移民去西以.
不要冤枉他.
是上帝賜給他的產業.
雅各和他的子孫落到埃及.
這是第二段的事實.
上帝說.
「是我差遣摩西亞倫.
照我在埃及所行的.
是我降災給埃及」.
然後.
是上帝說.
「我將你們領出來.
領你們的祖宗出埃及.
過紅海.
在紅海的邊緣.
你們的祖先埃及耶和華.
就使海水淹沒埃及人的軍隊.
車領馬兵」.
上帝說.
「我在埃及所行這一切的事情.

$^{521}$你們是親眼見過」.
然後第三部份.
就更加直接.
是眼前所發生的事.
「我領你們第八節.
來到約旦河東.
阿摩利人所住之地.
他們和你們爭戰.
是我將他們交在你們的手中.
是我在你們的面前.
將他們滅絕」.
說的是兩個迦南的王.
也說的是文素記.
大家也許會記得的.
巴蘭先知要就坐以色列民.
上帝不容許.
結果就坐成為祝福.
過了約旦河之後.
上帝說.
「是我將迦南人交在你們的手裡」.
然後.
不知你讀聖經的時候有沒有留意.
十二節上帝用了一個比喻.
是整本聖經從來沒有出現過的.
上帝說.
「我打發黃蜂飛在你們的前面.
將阿摩利人這兩個王.
從你們面前撚出去」.
剔手邊.
做一個好像車字上.
還有代替的剔字上半字.
是讀「撚」字的.
廣東話也有用.
不是撚死人的撚.
是撚一樣東西出去.
上帝說是我做的.
如果上帝打發黃蜂.
以色列人就不用打仗了.
不是.
以色列人要打仗.

$^{561}$不過這個比喻很清楚.
告訴百姓聽.
其實戰爭得勝.
是上帝所賜.
不是他們的努力.
不是他們的勇武.
然後.
你們很熟悉的.
最後十三節.
上帝說.
是我賜給你們地圖.
不是你們修治的.
是我賜給你們成衣.
不是你們所建造的.
你們住在裡面.
又吃那些不是你們栽種的葡萄園.
和橄欖園的果子.
從頭到尾.
這一段回顧歷史.
約書亞和百姓要溫習的.
就是告訴他們.
到底上帝是一個怎樣的上帝.
你就會發覺.
原來上帝不是法老王.
上帝不是那個要奴役百姓.
要找奴隸來幫他建造這個建造那個的上帝.
事實上剛剛相反.
以色列人的歷史從阿伯拉罕到入加南.
每一件事情.
是上帝在做事.
不是百姓在做事.
你看不看到這個事實.
今天我們想起時逢.
是想起我們做事.
我多辛苦.
開兩個半小時車來港島.
不是.
原來不是.
原來是上帝一早.
已經在你和我的生命當中.

$^{601}$做了很多事.
你是怎樣來加拿大的.
你是怎樣來北美的.
是我買飛機票的.
不是.
是上帝帶領我們.
離開本地本族.
附加.
往我們所不知道的地方去.
是這樣的嗎?.
我以為只有阿伯拉罕才是這樣的.
你當時沒有這樣的感覺嗎?.
不是的,我旅行看得很清楚的.
山明水秀的.
特別溫哥華.
暑假153天不下雨.
移民去到9月就開始下雨.
一直下到第二年的Victoria Day Long Weekend.
這個星期.
是不知道的.
來到了,誰帶領你來的?.
誰拯救你脫離罪惡的奴役?.
又是誰將今天美好的生活.
不是你蓋的屋.
不是你種的樹.
你明白的嗎?.
上帝不是那個需要我們去做他的奴隸.
去為他做事的上帝.
詩篇說得好.
千山上的牛羊是屬於他的.
然後上帝問一個問題.
萬一我真的肚餓了.
那用不用等你來獻祭呢?.
很幽默的.
不用了.
地和其中所充滿的.
世界和住在其間的.
什麼?.
都屬耶和華.
他是創造的主.

$^{641}$他是拯救的主.
他是帶領的主.
他是保守的主.
他是賜福的主.
他是恩代的主.
其實這一切都是他在做事.
做上帝的奴僕.
服侍上帝.
其實很值得的.
你有沒有想過.
可能你的感受沒有我們香港人那麼深.
我們很辛苦.
打份工.
工層樓.
無敵海景向南.
誰享受呢?.
家裡的姐姐.
請她回來.
你就回去.
照七晚十一.
她就一直拖地.
還是不用.
自己機械人在這吸塵.
開著電視.
開著中央冷氣.
對著海景.
打電話回鄉下.
這樣做僕人.
真是值得.
你明不明白.
侍奉耶和華就是這樣的.
要用到這樣的比喻你才明白.
因為你太被法老的奴役.
壟斷了你的想法.
這個世界是這樣對我們的.
渣幹渣靜.
Deadline is a deadline.
你交不到的.
份工沒了.
交了的.

$^{681}$健康就沒了.
人若賺得全世界.
是可以的.
不過賠上什麼.
付出什麼代價.
你心知肚明.
不過人在江湖.
身不由己.
以為移民了.
其實都一樣很淒涼.
一樣薄到盡.
是啊.
資本主義的經濟.
其實實行了不久.
Globalization.
全球一體化.
更加是一個很新鮮的實驗.
和民主政制差不多.
成功與否.
歷史還未落結論.
不過我們就在付這些代價.
因此我們很自然.
又很容易.
將我們在日常生活所經歷的奴役.
就放進來.
成為我們和上帝的關係.
唯一的指標.
侍奉還有其他可能嗎?.
不是能者多勞嗎?.
不是越做越多嗎?.
不是要病才可以請假嗎?.
怎麼說都沒用.
所以約書亞在這裡.
他要先和百姓.
從他們所經過.
但是很容易看不到的焦點.
再三強調.
是上帝在做事.
你未出生之前.
上帝已經在做事.

$^{721}$你說我不是三代基督徒.
我是自己走進教會信耶穌.
是我決定舉手決志的.
請問誰擺一間教會讓你走進去?.
他就是自己在這裡.
盤古初開.
不可能的.
哪位宣教士建立這間教會?.
哪位差會差的宣教士來的?.
這個差會是怎樣成立的?.
它的歷史在哪裡?.
早在你和我還是蒙昧無知的時候.
上帝的工作在歷史當中.
一直可以追上去.
可以追回到阿瑪拉罕.
甚至可以追回到起初.
上帝創造天地.
新約.
盧伽福音.
耶穌的加譜.
追上去亞當.
亞當是神的兒子.
不是說他是耶穌的大佬.
是上帝所創造.
歷代至上第一章.
以色列人亡國被擄.
回歸重建耶路撒冷.
他們數歷史.
也是從亞當開始.
上帝的工作.
早在你和我認識他之前.
他已經開始了他的救贖.
並且他的工作沒有停止.
並且他的賜福.
他的恩典.
不是只是讓你認識他.
只是要你讀聖經的時候.
他就與你同在.
你出你入.
耶和華保護你.

$^{761}$你所擁有的這一切.
上帝給你享受.
他不會眼紅.
他不會jealous.
他不會非要給你一些你不喜歡的東西.
有些人對上帝的印象是這樣的.
可能是來自家裡的背景也不奇怪.
不過約書亞說得很清楚.
其實你只要看回事實.
上帝不需要以色列人的服侍.
相反.
如果容許我們很大膽的說一句.
上帝正在服侍他的百姓.
不應該這樣說.
上帝塑造他的百姓.
上帝帶領他的百姓.
上帝在那裡賜福給他的百姓.
上帝在那裡恩待他的百姓.
這樣叫做侍奉耶和華.
至於我和我家.
We will serve the Lord.
一定的.
你認識了上帝這麼好.
你怎麼會打第二份工.
你怎麼會轉老闆.
浪子也會回頭.
我得罪了天.
得罪了你.
我不配做你的兒子.
讓我做一個僕人.
也比看著更好.
這個比喻背後的神學.
對上帝的認識和了解.
其實早就大衛在詩篇說過.
寧可在耶和華的殿看門口.
比住在惡人的帳篷吃大餐好.
你記不記得在哪裡.
你現在Google很容易找到.
吃大餐那三個字我加上去.
不過你明白我的意思.

$^{801}$其實聖經一直介紹的上帝.
不是沒有你就不行.
是靠你去侍奉的.
從來都不是.
約書亞說.
至於我和我家.
用什麼選擇.
不是沒得選擇.
聽不聽得出分別.
不是沒得選擇.
不是強迫他要服侍神.
你是領袖.
那沒得選擇.
天天都看著你.
不是.
約書亞說這句話的時候.
你記得是從第二節開始說到現在.
他認識上帝是這麼好的上帝.
他經歷的上帝是這麼好的上帝.
因此他的祈禱.
不單止是他作上帝的僕人.
他的家.
他的後代.
都要同樣侍奉耶和華.
我們再讀下去.
第十六節.
百姓回答說.
聰明的.
百姓聽得明白的.
他們沒有打瞌睡.
他們聽到約書亞剛才所說的重點.
你聽聽百姓怎麼說.
我們斷不敢離棄耶和華.
去侍奉別神.
有原因的.
第十七節說.
因耶和華我們的神.
曾將我們和我們列祖.
從埃及地遺留之家.
領出來.

$^{841}$講歷史.
講事實.
講上帝的拯救.
講上帝的作為.
在我們眼前.
行了大神蹟.
在我們所走的路上.
所經過的諸國都保護我們.
耶和華又把住此地阿摩利人.
從我們面前趕出去.
約書亞很詳細說的三段歷史.
現在百姓很簡單的重複.
撮要.
因為是這樣的緣故.
總結.
百姓說.
所以我們必侍奉耶和華.
因為祂是我們的神.
這樣就好了.
約書亞說完這麼長的一段說話.
沒有浪費.
大結局.
大團圓.
百姓堅定相信.
並且知道所信的是誰.
又知道上帝在他們民族的歷史事實上.
曾經做過這一連串的事情.
然後百姓懂得.
怎麼說呢?.
Draw the right conclusion.
他們來到一個很正確的立志.
所以我們必侍奉耶和華.
祂是我們的神.
這樣就完成了.
約書亞記來到.
其實十八節就可以The End.
不是Game Over.
完啦.
大功告成.
不是嗎?.

$^{881}$你讀過約書亞記二十四章.
你知道還未開始.
這一章的聖經來到.
百姓第一次的回答.
只不過是鋪了基礎.
這個基礎很重要.
因為如果這個基礎仍然是埃及.
仍然是奴隸.
仍然是奴役.
你沒辦法明白.
再讀下去.
約書亞所說的是什麼意思.
你只能夠勉為其難.
不能選擇.
有一點被迫的感覺.
這樣說.
好吧,我們繼續侍奉神吧.
完全不是這回事.
你首先要反轉了.
那個埃及的印象.
你反不轉這個.
你沒辦法明白什麼叫做侍奉神.
你還以為是搬過來.
好像以前拜偶像搬了過來.
以前保佑我的是觀世音菩薩.
現在保佑我的是耶穌基督.
不是這樣的.
聖經從來都不是這樣的.
把門神拔掉.
把十字架放在那裡.
沒有這回事.
上帝不是我們以前.
心目中以為的那些民間宗教.
所以不要搬了那些舊的壞習慣過來.
然後就說上帝是這樣的.
祂喜歡的.
怎樣侍奉神呢?.
百姓已經說得出.
又有正確的立志.
又有正確的原因.

$^{921}$來到第十九節.
約書亞石破天驚.
忽然之間說了一句莫名其妙的話.
約書亞對百姓說.
你們不能侍奉耶和華.
要明白約書亞說.
至於我和我家.
我們必定侍奉神.
這句話的內涵.
你必須從第一節開始讀到十三節.
你要明白約書亞.
所侍奉的神是怎樣的一位神.
祂不是法老.
祂不是奴役我們那個.
祂不需要我們服侍祂.
反過來是祂的恩手.
在我們的生命.
在我們還未認識祂之前.
祂已經開始祂的工作.
這是上半章.
要明白約書亞說.
至於我和我家.
我們必定侍奉神.
你也需要明白下半章.
約書亞忽然之間反過來說.
你們不能侍奉耶和華.
這個翻譯比較古老.
不能有兩個意思.
第一個意思就是不准.
你們不准侍奉耶和華.
不是.
約書亞當然不是這個意思.
前面剛剛才說.
你們要選擇.
我選擇了.
我願意服侍上帝.
我的下一代繼續服侍上帝.
你們怎樣選擇.
所以約書亞的意思一定不是.
你們不准選侍奉耶和華.

$^{961}$不是.
第二個意思就只剩下.
你們心有餘力不足.
不能夠.
希伯來文很清楚.
你們不是不准侍奉神.
而是你們想侍奉祂.
但是你想清楚.
你們多數不行.
多數會搞禍了.
你們想這樣做.
真的做起來的時候.
不是這麼簡單的.
說什麼呢?.
因為祂是聖潔的神.
是忌邪的神.
必不赦免你們的過犯罪惡.
你們若離希耶和華.
去侍奉愛邦神.
耶和華在降福之後.
必轉移降禍於你們.
把你們滅絕.
真的要懂得聽約書亞說什麼.
否則解釋錯了.
上帝好像雷公一樣.
一路盯著你.
行差踏錯.
一個雷閃電.
就砍死你.
還不是這樣.
你又把民間宗教信仰.
搬進來聖經讀了.
你要由第一節讀下來.
說到這裡.
約書亞說什麼呢?.
約書亞說的兩件事很清楚.
第一.
侍奉耶和華.
不是我們為上帝做什麼.
其實上帝不用我們做事.

$^{1001}$有時甚至我們會說.
我們做事阻礙上帝.
是的.
我們在神學院做老師.
我們上一任的院長.
余達新牧師提醒我們.
我們不要阻礙上帝.
在每一個同學生命當中的工作.
這句話原來不是他提醒我們.
是有人提醒他.
是的.
上帝正在做事.
他未進來神學院之前.
未曾夢召.
未曾奉獻.
未曾全時間侍奉.
上帝已經在他生命當中做事.
上帝的工作奇妙.
是他不是我們在做事.
這是約書亞斬釘截鐵說得很清楚.
不過有時我們讀聖經.
之前的先入為主的侍奉觀念.
太過橫梗勃逆.
難阻我們去明白聖經.
要麼我們只背今句.
至於我和我家.
我們必定侍奉神.
阿們忘記上文下理.
要麼讀完後都水過鴨背.
聽不明白.
約書亞說的不是法老.
很清楚.
第二.
百姓仍然要選擇.
不是發脾氣.
不是賭氣.
上帝做完了.
等上帝抬我上天堂.
很多人這樣信耶穌.
很大脾氣.

$^{1041}$很野蠻.
對不起.
信耶穌不是拍拖.
不到我們發脾氣.
上帝做這一切.
但你可以選擇.
選擇什麼.
約書亞很明顯.
來到中間十四十五節.
不斷重複.
侍奉侍奉侍奉.
重點是你要選擇.
如果你以為侍奉耶和華不好.
今天選擇你所要侍奉的.
你可以選擇.
但選擇了.
你要小心.
約書亞現在說的是選擇了.
百姓選擇了.
我們必定侍奉神.
學你吧約書亞.
跟著你吧.
你帶領我們這麼好.
我又見到有個板在這.
還不侍奉上帝.
侍奉上帝多好.
我小時候唱一首青年聖歌.
不知道你們會不會唱.
侍奉耶穌.
無福.
請敬拜隊.
待會帶我們唱吧.
臨時怎麼唱.
他們會背的.
很熟悉的詩歌.
還不選擇上帝.
一定選擇.
但你明不明.
難在哪裡.
你明不明.

$^{1081}$不是說我沒資格去服侍他.
我沒神學的degree去服侍他.
我沒經驗去服侍他.
我不夠背景.
我信耶穌日子淺.
我們還在想.
法老那條線.
那個奴隸比較厲害.
升他做總管吧.
不是啊.
完全不是這回事.
侍奉耶和華.
按照聖經的教導.
是上帝在我和你的生命當中.
完成他的工作.
他的傑作.
We are his masterpieces.
這個是Eugene Peterson.
翻譯.
不過不是翻譯.
約書亞記.
翻譯《已忽所書》第二章第十節.
我們熟悉的金句是第八節.
第九節.
不過你可能中間背一背就斷了.
還是打開聖經讀一次吧.
《已忽所書》第二章第八節.
我一開始讀.
你幾乎可以接著一起背.
你們得救是本乎因也因著信.
並不是出於自己乃是神所賜的.
也不是出於行為免得有人自誇.
我說搞亂的就是後面那兩段.
通常都背不出來.
不要緊.
不過背完之後.
馬上按就完了.
還沒完.
《已忽所書》第十節說什麼.
你讀下去.

$^{1121}$我們原是他的工作.
We are his masterpieces.
不是我們為神工作.
但聽不進去.
不是我們做事嗎.
做得很辛苦.
一早就來到了.
擺椅子也是我.
派詩歌也是我.
以前而已.
現在搞PowerPoint也是我.
做電腦也是我.
都是我.
那個我不是你.
那個我是上帝.
《若書雅記》第24章第二節開始.
若書雅用了上帝的口吻.
來講上帝所做的一切事情.
我們原是神的工作.
我們是那幅畫布.
上帝是那個畫家.
是他在你和我的生命當中.
完成他的傑作.
不是你和我的塗鴉.
分不分得到.
不過我們總是不信得過.
上帝所做的是傑作.
我們總是要自己加一些意見上去.
這裡紅色比較好.
那裡留一些白色.
有些意境.
我也會說兩句.
在講的不是那幅畫.
是我的生命.
上帝是我的生命.
你創造了我.
你給我自由.
你給我選擇.
我想這樣.
上帝你為我成就.

$^{1161}$我們每次祈禱都是這樣.
上帝我要這樣我要那樣.
求你幫助我.
耶穌不是這樣教我們祈禱.
我們經常讀.
你們未求已先.
你們的父早已知道了.
不用像外邦人那樣祈禱.
Peterson翻譯外邦人.
就是那些不認識真神的人.
他們會祈禱.
很多事情要求.
不過他們不認識神是誰.
所以搞錯了.
以為話多了.
必蒙誰聽.
以為人多了.
必蒙誰聽.
以為round the clock.
24小時.
又必蒙誰聽.
陰公.
聖經從來都不是這樣答應我們.
你們祈求.
就給你們.
敲門就開門.
尋找就尋見.
這麼簡單?.
這麼容易?.
不用祈禱了.
要怎樣祈禱?.
先求神的國和神的義.
不懂得怎樣祈禱.
耶穌教了你.
你懂得背.
我們在天上的父.
願人都尊你的名為聖.
再背下去.
願你的國降臨.
還有呢?.

$^{1201}$願你的旨意成就在地.
如同成就在天.
這樣都要日用的飲食.
這是另一件事.
針研第30章.
你回去看清楚.
不是求日用的飲食.
是求不要貧窮.
不要富足.
你讀清楚一點.
你讀那句就不出聲了.
主啊,我不要貧窮.
不過我要富足.
所以我們祈禱.
有時真的要讀聖經.
不是說得出就說.
求神的旨意成就.
馬丁路德說得好.
神的國自己來的.
用不著你求.
神的旨意一定成就的.
用不著你祈禱.
那為什麼要祈禱?.
路德說.
當神的國降臨的時候.
主啊,不要漏了我.
當神的旨意成就的時候.
主啊,成就在我的身上.
照你的意思.
不是照我的意思.
很實際的.
你病的時候怎樣祈禱?.
主啊,醫治我.
可以.
你心裡是這樣想.
天父早已知道.
祂不知道嗎?.
你騙不了祂的.
不過祈禱最後.
你效法主耶穌.

$^{1241}$父啊,倘若可行.
求你叫這個杯離開我.
祈願了沒有?.
沒有.
然而.
你讀下去.
不要照我的意思.
照你的意思.
這個叫做侍奉.
照神的意思.
讓祂成就祂的傑作.
在哪裡?.
在天上?.
在教會當中?.
在牧師的家庭?.
不是.
在我和我的家.
主啊.
求你的旨意成就.
求你的國度降臨.
求你的名得到榮耀.
不要加把口.
不要插手.
不要祈禱搖動神的手.
好不好?.
你搖錯了.
你仰望祂就行了.
明白嗎?.
差之毫釐.
謬之千里.
我們以為我們知道.
什麼是最好的?.
我們以為我們懂.
上帝就聽我的.
聖經從來都不是這樣教的.
你可以說.
不過你是祂的兒女.
兒女不用說了.
我們開玩笑.
動動尾巴就知道祂想什麼.

$^{1281}$長大了.
到女兒跟我說.
爸爸你動動尾巴.
我知道你做什麼了.
是啊, 祂厲害了.
不用說了.
眨一眼就知道了.
一個眼神.
大家就溝通了.
上帝是很好的.
祂一定將最好的給你.
不用你求的.
信得過嗎?.
有些時候.
你信不過.
你就要插手.
哎呀.
病了這麼久都不好.
搬一把金魚回來.
動動床.
對過另一個方向.
搞什麼鬼.
哪一科?.
你明白我的意思嗎?.
沒有的.
都有些學問的風水.
你兜得回的.
你聽聽上帝怎麼說.
不要去侍奉.
上帝以外.
任何的一個.
這個傑作.
是上帝.
單獨完成在你的身上.
你試一下教畫家怎麼畫畫.
我們都不是畫家.
你試一下教人怎麼抓象棋.
趕你走的.
你明白嗎?.
觀其不與真君子.

$^{1321}$看我怎麼抓.
三步殺他.
就是不是了.
你這個抓錯了.
你明白我們嗎?.
我們自作聰明.
在我們的人生當中.
在我們的生命裡面.
我們每一次都教上帝.
最好是這樣.
我想這樣.
行不行?.
還要扭一下上帝.
你不放手.
讓上帝去成就他的傑作.
什麼叫侍奉神?.
約書亞記二十四章.
解釋得很清楚.
你現在要選擇.
選擇Hands off yourself.
不要自己抓住上帝的手.
要他完成我的計劃.
或者我的兒子.
我想他這樣.
我的女兒我想他這樣.
我的下一代.
我的上一代.
我的家.
我的事業.
我的我的.
忘記了那個我是上帝.
不是我.
你再讀約書亞記.
很清楚.
到最後.
你要選擇.
很難的.
對人來說.
最難是放下自己.
越厲害越難.

$^{1361}$越懂得思考.
越難信服.
總覺得自己有些東西看到.
上帝看不到.
邏輯上沒有可能.
但這裡總是這樣覺得.
不合理.
服侍上帝.
你選擇.
所以保羅提醒我們.
提得好.
所以弟兄們.
包括姐妹們.
我以神的慈悲.
勸你們.
將身體獻上.
當作活祭.
又衝出去做事.
獻上身體.
全獻在壇上.
那祭生可以做些什麼.
死了.
放在那裡.
點火.
沒有事可以做.
保羅說活祭.
活的就可以衝出去做事.
不是.
活的就是願意信服上帝.
任憑主的旨意成就.
所以是聖潔的.
是神所喜悅的.
你們如此侍奉.
乃是理所當然的.
還有下面第二節.
不要效法這個世界.
只要心意更新變化.
叫你們察驗.
何謂神善良.
純全可喜悅的旨意.

$^{1401}$是神的旨意成就.
成就在我的生命當中.
這個叫做侍奉.
這個叫做作神的獨人.
不用你做事.
我以後回到教會動起腳.
多好啊.
聽完之後.
一身鬆了.
你回到家裡也不能動起腳.
家務就是家務.
五毛錢才倒垃圾的不是你家.
洗完碗要計數的不是你家.
見東西就要做的.
那些不是侍奉.
那些是本份.
明不明白.
那是家.
你家裡有很多東西.
看到就收拾.
不要說家裡.
家裡回家也不做事.
踢球.
要教練叫你才補位.
坐一邊.
自己補.
自己走.
一隊球十一個人.
你不去補那個位誰補啊.
很自然的.
什麼是侍奉啊.
什麼是作神的獨人啊.
是讓神成就他的心意.
合神心意的服事.
不是你做什麼.
討主喜悅.
是你來到主的面前.
你說我將身體獻上.
成為祭物.
但是我一生.

$^{1441}$不是等我死了.
是我一生每天.
我都願意.
你的意思成就在我身上.
你帶我去哪我去哪.
你叫我做什麼我做什麼.
你叫我脫離這件事情我脫離.
你帶我進入新的境界.
我遵命信服.
離開我的舒適區.
我又再走多一步.
感謝主.
神的榮耀.
就是這樣彰顯.
叫人見到我們.
有背錯聖經的.
不過翻譯是這樣譯的.
好行為.
不是啊.
你叫人見到上帝的心意.
成就在我的生命當中.
走出來的.
其實不是我能夠做的事情.
榮耀就歸給我們在天上的福.
就是這麼簡單.
肯還是不肯.
約書亞說.
想清楚.
不要加把口.
不要幫手.
不要差我上帝在你身上.
祂要成就最美好的心意.
我們同心一起祈禱.
多謝天父.
你的說話解開.
發出亮光.
使我們得釋放.
得自由.
得智慧.
我們恭敬將你的話藏在心裡.

$^{1481}$求聖靈在我們需要的時候.
提醒我們.
信服你.
跟從你.
願意見到的.
不是我們所求所想的.
夢理英雲.
而是你最高的心意.
成就在我的家.
在我的身上.
榮耀單單歸還給你.
奉耶穌基督的名祈禱.
阿們.
我們多謝院長給我們的教訓.
我們的訓練.
記得我們今晚回家想清楚.
在這裡我也多謝我們的敬拜隊.
剛才帶領我們一起敬拜天父.
我也代表我們忠臣.
多謝我們城北華人基督教會.
今晚給我們的地點.
款待我們.
也特別多謝我們城北堂的AV隊.
我們今晚的直播轉播.
是很多功夫的.
我們27個聚會點的工作人員.
和我們城北堂負責的弟兄姊妹.
做過多次的測試.
才敢今晚在這裡直播轉播.
大家都是很努力.
用了很多的時間.
很多的精神.
很忍耐.
忠誠的合作.
也多謝香港忠臣拓展部.
給我們有很多方面的提示和指點.
這一切.
最後我也很感謝我們.
很多的聚會點裡的弟兄姊妹.
為今晚的聚會.

$^{1521}$作出不停的禱告.
以致我們今晚可以很暢順的進行.
感謝神聽我們的禱告.
以下的時間.
我請忠臣加拿大董事會主席.
李偉廷先生.
為我們今晚有一個感恩禱告.
(音樂).
很開心從溫哥華來到當中.
見到這麼多的弟兄姊妹.
有機會能夠參加今晚的陪寧會.
請會眾一同的起立.
愛我們的主.
我們同心的感謝贊美利.
你用你自己無比的愛和慈誠愛誦.
招聚我們眾弟兄姊妹一起.
一同見證.
上帝賜給忠臣過去無比的恩典.
豐盛的恩典.
我們更加為忠臣獻上無限的感恩.
主人我們感謝你.
藉著中國神學研究院所舉辦的.
生命之道的查經.
更加激勵弟兄姊妹.
愛聖經.
愛讀經.
愛慕主你自己的話語.
我們祈求主更加使用中國神學研究院.
承擔神學教育的使命和意象.
訓練更多的傳道人.
訓練更多神的僕人.
主人我們再一次的感謝你.
藉著忠臣的院長.
李思敬牧師.
今天晚上.
講一個這麼切身的題目.
是合臣心意的侍奉.
我們今天能夠站在台上侍奉.
站在教會裡面侍奉.
傳示神的恩典.

$^{1561}$讓我們清清楚楚知道.
我們要尋求神的旨意.
成就失神的旨意.
主人求你繼續使用我們每一個.
在你面前的弟兄姊妹.
能夠更加投入教會的侍奉.
更加多領人歸主.
見美好的果子.
為你作美好的見證.
求神使用我們每一個.
跟隨你的人.
祝福每一個願意侍奉你的人.
讓我們一生跟從你.
讓我們尋求神的旨意.
榮耀主你的名.
我們誠心的禱告.
侍奉主耶穌基督得勝的名求.
阿們.
各位請坐.
請大家可以有一些時間默禱.
然後彼此安安.
謝謝大家.
我們的生命是與生俱來的.
\newpage



\section{}
\label{sec:2TEwldoXzT8}
\textbf{中神北美培靈講座 2016 「合神心意的事奉」-- 李思敬院長}
\newline
\newline
連結: \href{https://youtube.com/watch?v=2TEwldoXzT8}{\texttt{ https://youtube.com/watch?v=2TEwldoXzT8}} ~~~~ 語音日期: 2023-12-02 
\newline
\newline
\hyperref[sec:zyLpufeGBYs]{\small{< < < PREV SERMON < < <}}
~
\hyperref[sec:index]{\small{[返主目錄]}}
~
\hyperref[sec:JHQ2Beaggow]{\small{> > > NEXT SERMON > > >}}
\newline
\newline
$^{1}$(中文字幕:三馬兄).
各位好.
歡迎大家來到我們北美.
忠臣的北美培靈講座.
合神心意的侍奉.
我們首先歡迎多倫多各位嘉賓.
和主內的弟兄姊妹.
但我亦歡迎在北美有19個城市.
27個聚會點的嘉賓和弟兄姊妹.
我們今晚的聚會會直播和轉播.
去這27個聚會點.
我也在此歡迎我們今晚的講員.
我們中國神學研究院的院長.
李思敬牧師博士和李師母.
他們遠道在香港來到.
李牧師是我們今晚的講員.
合神心意的侍奉.
在開始之前.
我請我們城北華人基督教會的.
副主任牧師林志輝牧師.
將我們今晚的聚會交託給天父上帝.
請我們一起起來.
我們一起來禱告.
我們一起低頭禱告.
親愛的天父我們感恩.
多謝你.
你帶領我們.
你亦與我們同在.
你將你的話語時常的料釋放給我們.
我們感恩.
每一次我們讀你的話語時.
我們心裡得著興奮.
得著歡喜.
因為你的話語充滿能力.
主下求你亦教導我們.
將你的話藏在心裡.
免得我們得罪你.
主下今天晚上我們感謝你.
因為讓我們在這裡有.
李思敬院長給我們的一個陪靈會.

$^{41}$主下求主你教導我們.
在今天晚上裡聽到你的說話.
你的靈也在我們當中運行.
打開我們的心.
以致我們能夠好好的聆聽.
怎樣來土神的喜悅.
怎樣來敬拜主你.
怎樣來為主你服侍.
主一舊帶領我們每一個人.
亦都給我們在這個地方裡.
接受主你的祝福.
天父我們也求主你給李院長.
他有好的身體.
好的聲線.
我們知道他從香港過來.
可能需要有少許的適應.
求主你堅強他.
給他力量.
我們求主你今天晚上.
在這個地方裡.
賜福給我們每一個人.
以致我們能夠更加知道.
怎樣來好好的聽你的話語.
土神的喜悅.
來侍奉主你.
求主你恩代.
祈禱奉主耶穌基督命求.
阿們.
各位請坐.
我們請我們的.
Worship Team帶領我們敬拜主.
各位多人多愛的弟兄姊妹平安.
各位海外的弟兄姊妹平安.
今天晚上我們能夠跨越時空.
在神的裡面.
我們一起來敬拜.
實在是神的恩典.
所以這一刻.
無論你是在哪一個地方.
讓我們一同的.

$^{81}$恭敬的來到這裡喜立.
我們懷著感恩快樂的心.
一起來讚美我們的天父上帝.
祂就是那位聖潔尊貴的主.
我們一起讚美祂.
神美聖潔.
道德尊貴.
來生歡呼拍掌.
今我傳送你.
我的主.
神你帶領恩典經過.
來心感恩喝彩.
今現上為你.
屬於主.
我要每次每刻.
都贊頌你是神.
你是恩典.
你手上的.
給我滿有信心.
我要每天每刻.
都贊頌你是神.
獻出一生.
交於主你手中.
獻出一生.
交於主你手中.
神美聖潔.
道德尊貴.
來生歡呼拍掌.
今我傳送你.
我的主.
神你帶領恩典經過.
來心感恩喝彩.
今現上為你.
屬於主.
我要每天每刻.
都贊頌你是神.
你是恩典.
你手上的.
給我滿有信心.
我要每天每刻.

$^{121}$都贊頌你是神.
獻出一生.
交於主你手中.
獻出一生.
交於主你手中.
讓我們動心低頭禱告.
我讚愛主啊.
我們多謝你.
因為你的恩典賞與我們同在.
主啊你的慈情愛索.
每一天都牽引著我們.
你的愛從來不會離開我們.
天王我們感謝你.
此刻你的獨生兒子.
耶穌基督寶貴的救贖.
我們透過耶穌的救贖.
我們這班原本不配的罪人.
我們能夠與你和好.
並且能夠被稱為你的寶貝兒女.
主啊我們實在感謝你.
讚美你.
天父上帝.
每當我們每一次去認真.
細想你對我們那份徹底的.
那份無條件的愛的時候.
我們豈能不去開口感謝讚美你.
我們豈能不將我們的身心都獻情給過你呢.
天父我就求你幫助我們.
引領我們每一個人.
能夠過一個謙卑自竭的生活.
求你的話語叫我們的腳步穩健.
不容許罪孽克制我們.
主啊我又求你無阻我們.
叫我們每一個人都能夠成為.
合你心意的器皿.
能夠成為聖潔.
合乎主用.
叫我們的行事為人.
能夠令你的命得到榮耀.
讓你的心得到滿足.

$^{161}$主我們就是這樣.
恭敬的在你的面前祈求.
禱告奉耶穌基督得勝的名字而求.
Amen.
神里聖傑.
道德尊貴.
內心歡呼拍掌.
救我傳授你我的主.
神里大靈.
用點經過.
內心感恩喝彩.
救現上為你贖你主.
我要每天每刻都贊助.
你是神.
你是恩利.
你是信義.
令我滿有信心.
我要每天每刻都贊助.
你是神.
願出一生教育主你受眾.
我要每天每刻都贊助.
你是神.
你是恩利.
你是信義.
令我滿有信心.
我要每天每刻都贊助.
你是神.
願出一生教育主你受眾.
讓我們同聲對神說.
願出一生教育主你受眾.
Amen.
各位請坐.
今天晚上我們很高興.
能夠有中國神學研究院的院長.
李思敬博士.
在我們當中為我們分享神的說話.
用我們恭敬的一張時間.
交給李思敬院長.
請我們以熱烈的掌聲.
來歡迎李思敬博士.

$^{201}$各位弟兄姊妹主內平安.
不單止是多倫多.
準確一點是Richmond Hill.
其實今晚是中晨.
在加拿大和美國.
我們有不同的聚會點.
過去多數是生命之道.
今晚還有一兩個地方.
是在舉行生命之道的聚會點.
我們有中晨北美的培靈聚會.
我們今晚的題目是.
合神心意的侍奉.
如果我問大家.
在聖經裡面.
最集中地出現.
侍奉這兩個字的經文.
不知道你們會想到.
是記載在哪一處.
如果你以為是新約就不對了.
在舊約聖經.
你看到我站上來.
都應該是講舊約的.
生命之道的弟兄姊妹讀過的.
約書亞記二十四章.
我讀給大家聽.
如果你有聖經.
你都可以打開.
約書亞記二十四章.
我們要讀的是十四節到十五節.
短短兩節的經文.
你心水清.
讀的時候.
你數一下.
侍奉這兩個字.
總共出現了多少次.
約書亞對百姓說.
現在你們要敬畏耶和華.
誠心實意地侍奉他.
將你們列祖在大河那邊.
和在埃及所侍奉的神除掉.

$^{241}$去送耶和華.
若是你們以侍奉耶和華為不好.
今天就可以選擇.
所要侍奉的.
是列祖在大河那邊.
所侍奉的神.
是你們所住這地阿摩利人的神.
至於我和我家.
我們必定侍奉耶和華.
可能你掛著數.
聽不到約書亞說什麼.
總共多少次?.
七次.
每次都說侍奉耶和華.
有兩次說侍奉其他的神.
不過在短短兩節.
約書亞所說的這一段說話的總結.
他不停提到的是侍奉.
當然他要說的是侍奉耶和華.
他自己也很清楚.
跟百姓說明他的立場.
至於我和我家.
我們必定侍奉耶和華.
As for me and my house.
We will serve the Lord.
英文的文法通常我們後面應該是.
We shall serve the Lord.
但是用了個will字.
就是必.
這個語氣在希伯來文是很清晰.
約書亞的結論.
可能你家牆上都會不會有個牌.
是英文的.
不是的.
我小時候在我家的客廳.
有中文無筆字寫著.
我家必定侍奉神.
我小時候以為是我父親寫的.
原來不是他的墨寶.
不過一直掛在客廳的中間.

$^{281}$在我腦海當中留下很深刻的印象.
重複不是偶然.
約書亞的第一章.
耶和華上帝和約書亞.
重複「剛強壯膽」.
然後去到第一章總結.
百姓和約書亞說.
你只要剛強壯膽.
重複四次.
約書亞的開宗名義.
第一章重點主題.
剛強壯膽.
同樣重複的技巧.
在最後這24章出現.
約書亞兩節短短的說話七次.
你現在不要回家再讀.
接著由第十六節.
一直記載到百姓最後回答約書亞說.
24節.
我們必侍奉耶和華.
又是另外七次的重複.
還不夠.
去到這一結束之前.
31節.
約書亞在世.
和他死後的長老還在的時候.
百姓以色列人侍奉耶和華.
短短半章的經文.
重複十五次.
這個很明顯.
是約書亞的24章.
或者說約書亞的這一卷聖經總結的時候.
他要強調.
他要突出的焦點.
侍奉.
原來早在舊約聖經.
早在入家衙.
約書亞林中對百姓所說的這一番說話.
壓軸的高潮.
結論的部分.

$^{321}$侍奉耶和華.
不過我們要指出一個很重要的事實.
就是昔日.
以色列人聽這句話.
和今天我們坐在這裡聽這句話.
是完全不同的.
你說為什麼呢?.
很容易的.
今天我們說侍奉是什麼意思?.
很自然.
我們所說的是在教會的侍奉.
不是.
我出去短宣.
恍宣.
離開教會四步牆.
但是你還是在做傳福音.
宣教.
或者你是在教會的群體.
弟兄姊妹一起.
你會說侍奉.
明天上班.
你不會和公司的同事.
你說我今天侍奉.
雖然在神學上.
我們可以討論你的工作.
你朝九晚五.
那個也是侍奉.
但是你不會用這兩個字.
這兩個字幾乎成為了信徒彼此之間.
才明白的術語.
Cliché.
Jargon.
好像我們時時說.
回到屯契.
有很美好的交通.
你的意思是今天不塞車嗎?.
不是.
我剛才來塞了兩個半小時.
是Mr. Saga開出來走路.
交通.

$^{361}$一般人的印象.
一定是說traffic.
但是弟兄姊妹.
這個用詞.
是在說屯契.
所以有些詞彙.
有些術語.
有些觀念.
是我們用慣了不覺的.
侍奉.
你在教會有什麼侍奉?.
敬拜隊的侍奉.
不是.
門口招待.
派聖餐的侍奉.
全部是說聖公.
是和上帝的教會直接有關係的.
這個是侍奉.
昔日.
以色列人聽見.
約書亞說.
侍奉耶和華.
希伯來文這個字.
其實很普通.
絕對不是一個只是在會幕裡面.
或者在聖殿裡面.
或者在會堂裡面.
所用的詞彙.
希伯來文這個動詞.
說的是服事.
說的是作奴僕.
再說直接一點.
在以色列人的心目當中.
他一聽到這個動詞.
不是聽到在教會侍奉.
他是聽到.
在埃及.
做奴隸.
服事法老.
是同一個字.

$^{401}$埃及法老怎樣奴役他們.
你在主教學長大.
你讀過出埃及記.
你知道.
人多.
那不就是那些男嬰.
弄點手腳.
就可以弄死他們.
complain.
虐待.
不給材料.
自己去找草.
但是要交同樣數量的磚.
來給法老建造兩座積火城.
你讀出埃及記.
你很熟悉.
法老怎樣奴役百姓.
在約書亞說這番話的時候.
大概印象猶新.
所以百姓聽到約書亞這樣說的時候.
大概浮現出來的第一個感覺.
不是回去禮拜堂.
帶茶經.
做組長.
不是不是.
他想起的是以前.
被一個暴君.
怎樣壓制.
怎樣zar 乾zar 盡.
你怎樣用牙膏.
用到最後捲不捲起他.
有一次有個弟兄告訴我.
我要剪開他.
還可以用多一次兩次.
侍奉上帝.
是不是都是這樣.
上班下班.
已經很辛苦了.
還有家裡開門七件事.
很多的壓力.

$^{441}$回到教會還要講侍奉.
上帝是不是好像法老這樣對待他的百姓.
所以約書亞記第24章.
他不是一開始就講侍奉耶和華.
剛才我們所讀的是第14節.
你知道第1至第13節講什麼嗎.
約書亞叫百姓來到.
不是第一句就跟他們講侍奉耶和華.
不是不是.
如果這樣說的話.
百姓浮上來.
想得到的.
他們的記憶充斥著的.
就是埃及遺奴的經驗.
不過現在換了一個主人.
都是這樣對待我們.
所以不知道你有沒有留意到.
約書亞講侍奉耶和華之前.
他用了大半的篇幅.
由第2節開始到第13節.
他是首先跟以色列人重溫歷史的事實.
不是這樣說很沉悶.
講歷史.
要記年份.
要背人名.
要背地名.
還要用英文考會考.
怎麼串的.
現在不用了.
DSE.
不是不是講歷史.
是介紹給百姓重新認定.
耶和華是怎樣的一位上帝.
你聽聽.
約書亞在這裡說的.
是以色列人三段.
他們所知道.
甚至他們有份經歷的事實.
第一部份是上帝呼召亞伯拉罕.
從大河那邊來到迦南地.

$^{481}$24章第3節.
上帝用第一身單數的代名詞.
「我」.
來說「我將你們的祖宗亞伯拉罕.
從大河那邊帶到來.
領他走遍迦南全地.
是我將他的子孫洗他們眾多.
是我」.
耶和華說.
「將以撒賜給亞伯拉罕.
又將雅各和以素馬省兄弟賜給以撒.
又將西爾山賜給以素遺業」.
所以不是以素貪愛世界.
離開了應許地迦南.
自己移民去西爾.
不要冤枉他.
是上帝賜給他的產業.
雅各和他的子孫落到埃及.
這是第二段事實.
上帝說.
「是我差遣摩西亞倫.
照我在埃及所行的.
是我降災給埃及」.
然後上帝說.
「我將你們領出來.
領你們的祖宗出埃及.
過紅海.
在紅海的邊緣.
你們的祖先埃及耶和華.
他就使海水淹沒埃及人的軍隊.
車領馬兵」.
上帝說.
「我在埃及所行這一切的事情.
你們是親眼見過」.
然後第三部份.
就更加直接.
是眼前所發生的事.
「我領你們第八節.
來到約旦河東摩利人所住之地.
他們和你們爭戰.

$^{521}$是我將他們交在你們的手中.
是我在你們的面前將他們滅絕」.
說的是兩個迦南的王.
也說的是文素記.
大家也許會記得.
巴蘭先知要就坐以色列民.
上帝不容許.
結果就坐成為祝福.
過了約旦河之後.
上帝說是我將迦南人.
然後不知你讀聖經的時候有沒有留意.
十二節上帝用了一個比喻.
是整本聖經從來沒有出現過.
上帝說我打發黃蜂飛在你們的前面.
將阿摩利人這兩個王從你們面前掐出去.
手邊做過好像車字上面.
還有代替的替字上半的字.
是讀掐字的.
廣東話你也有用的.
不是掐死人的掐.
是掐一樣東西出去.
上帝說是我做的.
如果上帝打發黃蜂.
以色列人就不用打仗了.
不是.
以色列人要打仗.
不過這個比喻很清楚.
告訴百姓聽.
其實戰爭得勝是上帝所賜.
不是他們的努力.
不是他們的勇武.
然後你們很熟悉的.
最後十三節.
上帝說是我賜給你們地圖.
不是你們修置的.
是我賜給你們成衣.
不是你們所建造的.
你們住在裡面.
又吃那些不是你們栽種的葡萄園.
和橄欖園的果子.

$^{561}$從頭到尾.
這一段回顧歷史.
約書亞和百姓要溫習的.
就是告訴他們.
到底上帝是一個怎樣的上帝.
你就會發覺.
原來上帝不是法老王.
上帝不是那個要奴役百姓.
要找奴隸來幫他建造這樣建造那樣的上帝.
事實上剛剛相反.
以色列人的歷史從阿伯拉罕到入加南.
每一件事情.
是上帝在做事.
不是百姓在做事.
你看不看到這個事實.
今天我們想起時逢.
是想起我們做事.
我多辛苦.
開兩個半小時車來港島.
不是.
原來不是.
原來是上帝一早.
已經在你和我的生命當中.
做了很多事.
你是怎樣來加拿大.
你是怎樣來北美.
是我買飛機票的.
不是.
是上帝帶領我們.
離開本地.
本族.
附家.
往我們所不知道的地方去.
是這樣的嗎.
我以為只有阿伯拉罕才是這樣的.
你當時沒有這樣的感覺嗎.
不是的.
我旅行看得很清楚.
山明水秀.
特別溫哥華.

$^{601}$逢暑假153天不下雨.
移民去到九月就開始下雨.
一直下到第二年的Victoria Day Long Weekend.
這個星期.
是不知道的.
來到了.
誰帶領你來的.
誰拯救你脫離罪惡的奴役.
又是誰.
將今天美好的生活.
不是你蓋的屋.
不是你種的樹.
你明白嗎.
上帝不是那個需要我們去做他的奴隸.
去為他做事的上帝.
詩篇說得好.
千山上的牛羊是屬於他的.
然後上帝問一個問題.
萬一我真的肚餓了.
那要不要等你來獻祭呢.
很幽默的.
不用了.
地和其中所充滿的.
世界和住在其間的.
都屬耶和華.
創造的主.
他是拯救的主.
他是帶領的主.
他是保守的主.
他是賜福的主.
他是恩代的主.
其實這一切都是他在做事.
做上帝的奴僕.
服侍上帝.
其實很值得的.
你有沒有想過.
可能你的感受沒有我們香港人那麼深.
我們很辛苦.
打份工.
工層樓.

$^{641}$無敵海景向南.
感受吧.
家裡的姐姐.
她回來.
你就回去.
朝七晚十一.
她就一直拖地.
現在自己機械人在吸塵.
開著電視.
開著中央冷氣.
對著電話回鄉.
這樣做僕人.
真是值得.
你明不明白.
師奉耶和華就是這樣.
要用到這樣的比喻你才明白.
這樣你才明白.
因為你太被法老的奴役.
壟斷了你的想法.
這個世界是這樣對我們的.
渣幹.
Deadline is a deadline.
你交不到的.
份工沒了.
交了的.
健康就沒了.
人若賺得全世界.
是可以的.
不過賠上什麼.
不是.
付出什麼代價.
你心知肚明的.
不過人在江湖.
以為移民了.
其實都一樣很淒涼.
一樣薄到盡.
資本主義的經濟.
其實實行了不是很久.
Globalization.
全球一體化.

$^{681}$更加是一個很新鮮的實驗.
和民主政制差不多.
成功與否.
歷史還未落結論.
不過我們就在付這些代價.
因此我們很自然.
又很容易.
將我們在日常生活所經歷的奴役.
就放進來.
成為我們和上帝的關係.
唯一的指標.
侍奉還有別的可能嗎?.
不是能者多勞嗎?.
不是越做越多嗎?.
不是要病才可以請假嗎?.
怎麼說都沒用.
所以約書亞在這裡.
他要先從百姓.
從他們所經過.
但是很容易看不到的焦點.
再三強調.
是上帝在做事.
你未出生之前.
上帝已經在做事.
我不是三代基督徒.
我是自己走進教會信耶穌.
是否?.
請問誰把教會放在那裡讓你走進去?.
他就自己在那裡.
盤古初開.
不可能的.
哪位宣教士建立這教會?.
哪位差會.
差的宣教士來的?.
這個差會是怎樣成立的?.
它的歷史在哪裡?.
早在你和我.
還是蒙昧無知的時候.
上帝的工作.
在歷史當中.

$^{721}$一直可以追上去.
可以追回到亞馬拉罕.
甚至可以追回到.
起初上帝創造天地.
新約,盧伽福音,耶穌的加譜.
追到上去亞當.
亞當是神的兒子.
不是說他是耶穌的大佬.
是上帝所創造的.
歷代至上帝一章.
以色列人亡國被擄.
回歸重建耶路撒冷.
他們數歷史.
也是從亞當開始.
上帝的工作.
早在你和我認識祂之前.
祂已經開始了祂的救贖.
並且祂的工作沒有停止.
並且祂的賜福,祂的恩典.
不是只是讓你認識祂.
只是要你讀聖經的時候.
祂就與你同在.
你出,你入.
耶和華保護你.
你所擁有的這一切.
上帝給你享受.
祂不會眼紅.
祂不會憤怒.
祂不會隨便給你一些不喜歡的東西.
有些人對上帝的印象是這樣的.
可能是來自家裡的背景也不奇怪.
不過約書亞說得很清楚.
其實你只要看回事實.
上帝不需要以色列人的服侍.
相反,如果容許我們大膽地說一句.
上帝正在服侍祂的百姓.
不應該這樣說.
上帝塑造祂的百姓.
上帝帶領祂的百姓.
上帝在那裡賜福給祂的百姓.

$^{761}$上帝在那裡恩待祂的百姓.
這樣叫做侍奉耶和華.
至於我和我家.
We will serve the Lord.
一定的.
你認識了上帝這麼好.
你怎麼會打第二份工.
你怎麼會轉老闆.
浪子也會回頭.
我得罪了天,得罪了你.
我不配做你的兒子.
讓我做一個僕人.
也比看珠子好.
這個比喻背後的神學.
對上帝的認識和了解.
其實早就大衛在詩篇說過.
寧可在耶和華的殿看門口.
好過住在惡人的帳篷吃大餐.
記載在哪裡.
你現在Google很容易搜尋到.
吃大餐那三個字我加上去.
不過你明白我的意思.
其實聖經一直介紹的上帝.
不是沒有你就不行.
是靠你去侍奉的.
從來都不是.
約書亞說.
至於我和我家.
用什麼選擇.
不是沒得選擇.
聽得出分別嗎.
不是沒得選擇.
強迫他要服侍神.
你是領袖.
沒得選擇.
人人都看著你.
不是.
約書亞說這句話的時候.
你記得是從第二節開始說到現在.
他認識上帝.

$^{801}$是這麼好的上帝.
他經歷的上帝.
是這麼好的上帝.
因此他的祈禱.
不單是他作上帝的僕人.
他的家.
他的後代.
都要同樣侍奉耶和華.
我們再讀下去.
第十六節.
百姓回答說.
醒目的.
百姓聽得明的.
他們沒有黑眼訓的.
他們聽到約書亞剛才所說的重點.
你聽聽百姓怎麼說.
斷不敢離棄耶和華.
去侍奉別神.
有原因的.
十七節說.
因耶和華我們的神.
曾將我們和我們列祖.
從埃及地遺留之家.
領出來.
講歷史.
講事實.
講上帝的拯救.
講上帝的作為.
在我們眼前.
行了大神跡.
在我們所走的路上.
所經過的諸國.
都保護我們.
耶和華又把住此地.
阿摩利人從我們面前趕出去.
約書亞很詳細說的.
三段歷史.
現在百姓很簡單的.
重複.
撮要.

$^{841}$因為是這樣的緣故.
總結.
百姓說.
所以我們必侍奉耶和華.
因為祂是我們的神.
這樣就好了.
約書亞說完這麼長的一段話.
沒有浪費.
大結局.
大團圓.
百姓堅定相信.
並且知道所信的是誰.
又知道上帝在他們民族的歷史事實上.
曾經做過這一連串的事情.
然後百姓懂得.
怎麼說.
Draw the right conclusion.
他們來到一個很正確的立志.
所以我們必侍奉耶和華.
祂是我們的神.
這樣就完結了.
約書亞記來到.
其實十八節.
就可以The End.
完結了.
大功告成了.
不是嗎.
你讀過約書亞記二十四章.
你知道.
還沒開始.
這一章的聖經來到.
百姓第一次的回答.
那個只不過是鋪了基礎.
這個基礎很重要.
因為如果這個基礎仍然是埃及.
仍然是奴隸.
仍然是奴役.
你沒辦法能夠明白.
再讀下去.
約書亞所說的是什麼意思.

$^{881}$你為其難.
沒得選擇.
有一點被逼的感覺.
這樣說.
好吧.
我們繼續侍奉神.
完全不是這回事.
你首先要反轉了.
那個埃及的印象.
你反不轉這個.
你沒辦法能夠明白.
什麼叫做侍奉神.
你還以為是搬過來.
好像以前拜偶像.
搬了過來.
以前保佑我的是觀世音菩薩.
現在保佑我的是耶穌基督.
不是這樣的.
聖經從來都不是這樣的.
拔了門神.
放了十字架在那裡.
沒有這回事.
上帝不是我們以前.
心目中以為的那些民間宗教.
所以不要搬了那些慣過來.
然後就說上帝是這樣的.
祂喜歡的.
怎樣侍奉神?.
百姓已經說得出.
又有正確的立志.
又有正確的原因.
來到第十九.
約書亞石破天驚.
忽然之間.
說了一句莫名其妙的話.
約書亞對百姓說.
你們不能侍奉耶和華.
要明白約書亞說.
至於我和我家.
我們必定侍奉神.

$^{921}$這句話的內涵.
你必須從第一節開始.
讀到十三節.
你要明白約書亞.
所侍奉的神是怎樣的一位神.
祂不是法老.
祂不是奴役我們那個.
祂不需要我們服侍祂.
反過來是祂的恩手.
在我們的生命.
在我們還未認識祂之前.
祂已經開始祂的工作.
這是上半章.
要明白約書亞說.
至於我和我家.
我們必定侍奉神.
你也需要明白下半章.
約書亞忽然之間.
反過來說.
你們不能侍奉耶和華.
這個翻譯比較古老.
不能有兩個意思.
第一個意思是不准.
你們不准侍奉耶和華.
不是.
約書亞當然不是這個意思.
前面剛剛才說.
你們要選擇.
我選擇了.
我願意服侍上帝.
我的下一代繼續服侍上帝.
你們怎樣選擇.
所以約書亞的意思一定不是.
你們不准侍奉耶和華.
不是.
第二個意思就剩下唯有.
你們心有餘力不足.
不能夠.
希伯來文很清楚.
你們不是不准侍奉神.

$^{961}$而是你們想侍奉祂.
但是你想清楚.
你們多數不行.
多數會搞禍了.
你們想這樣做.
真的做起來的時候.
不是這麼簡單.
說什麼.
因為祂是聖潔的神.
是忌邪的神.
必不赦免你們的過犯罪惡.
你們若離棄耶和華.
去侍奉愛邦神.
耶和華在降福之後.
必轉移降禍於你們.
把你們滅絕.
真的要懂得聽約書亞說什麼.
否則解釋錯了.
上帝好像雷公一樣.
一路盯著你.
行差踏錯.
一個雷閃電就砍死你.
還不是這樣.
你又把民間宗教信仰.
搬進來聖經讀了.
你要由第一節讀下來.
說到這裡.
約書亞在說什麼.
約書亞在說兩件事很清楚.
第一.
侍奉耶和華.
不是我們為上帝做什麼.
其實上帝不用我們做事.
有時甚至我們會說.
我們做事阻礙上帝.
是的.
我們在神學院的老師.
我們上一任的院長.
余達新牧師提醒我們.
我們不要阻礙上帝.

$^{1001}$在每一個同學生命當中的工作.
這句話原來不是他提醒我們.
是有人提醒他.
是的.
上帝正在做事.
他未進來神學院之前.
未曾夢召.
未曾奉獻.
未曾全時間侍奉.
上帝已經在他的生命當中做事.
上帝的工作奇妙.
是他不是我們在做事.
這是約書亞斬釘截鐵說得很清楚.
不過有時我們讀聖經.
之前的先入為主的侍奉觀念.
太過.
怎麼說呢.
太過橫梗勃逆.
難阻我們去明白聖經.
要麼我們只背今句.
至於我和我家.
我們必定侍奉神.
阿們.
忘記上文下理.
要麼讀完之後.
都水過鴨背.
聽不明白.
約書亞說的不是法老.
很清楚.
第二.
百姓仍然要選擇.
不是發脾氣.
不是賭氣.
我上帝做了.
我翹起手.
等上帝抬我上天堂.
有很多人是這樣信耶穌.
很大脾氣.
很野蠻.
對不起.

$^{1041}$信耶穌不是拍拖.
不到我們發脾氣.
上帝做這一切.
但你可以選擇.
選擇什麼.
約書亞很明顯.
來到中間十四十五節.
不斷重複.
侍奉.
重點是你要選擇.
如果你以為侍奉耶和華為不好.
今天選擇你所要侍奉的.
你可以選擇.
但選擇了.
你要小心.
約書亞現在說的.
百姓選擇了.
我們必定侍奉神.
學你吧.
約書亞.
跟隨你吧.
你把我們帶得這麼好.
我們又看到有個板.
還不侍奉上帝.
侍奉上帝多好.
我小時候唱一首青年聖歌.
不知道你們會不會唱.
侍奉耶穌.
蒙福.
請敬拜隊待會帶我們唱.
臨時怎麼唱.
我們會背.
很熟悉的詩歌.
侍奉上帝一定選擇.
但你明不明.
難在哪裡.
你明不明.
不是說我沒有資格去服侍他.
我沒有神學的學位去服侍他.
我沒有經驗去服侍他.

$^{1081}$我不夠背景.
我信耶穌日子淺.
我們還在想法老那條線.
那個奴隸比較厲害.
升他做總管吧.
不是的.
完全不是這回事.
侍奉耶和華.
按照聖經的教導.
是上帝在我和你的生命當中.
完成他的工作.
他的傑作.
We are his masterpieces.
這是Eugene Peterson.
翻譯.
不過不是翻譯約書雅記.
翻譯《已忽所書》第二章第十節.
我們熟悉的金句是第八節.
第九節.
不過你可能中間部位就斷了.
都是打開聖經讀一次罷了.
《已忽所書》第二章第八節.
我一開始讀.
你幾乎可以接著一起背.
「你們得救是本乎因也因著信.
並不是出於自己乃是神所賜的.
也不是出於行為免得有人自誇」.
我說搞亂的就是後面那兩段.
通常都背不到.
不要緊.
我一開始就說完了.
還沒完.
保羅接著第十節說什麼.
你讀下去.
「我們原是他的工作」.
We are his masterpieces.
不是我們為神工作.
但聽不進去.
不是我們做事嗎.
做得很辛苦.

$^{1121}$一早就來到了.
擺椅子也是我.
派詩歌也是我.
以前搞PowerPoint也是我.
做電腦也是我.
都是我.
那個我不是你.
那個我是上帝.
《約書亞記》第24章第二節開始.
約書亞用了上帝的口吻.
來說上帝所做的一切事情.
是我們原是神的工作.
我們是那幅畫布.
上帝是那個畫家.
是他在你和我的生命當中.
完成他的傑作.
不是你和我的塗鴉.
分不分得到.
不過我們總是不信得過.
上帝所做的傑作.
我們總是要自己加一些意見上去.
這裡紅色比較好.
那裡留一些白色.
有些意境.
我也會說兩句.
說的不是那幅畫.
是我的生命.
上帝是我的生命.
你創造了我.
你給我有自由.
你給我有選擇.
我想這樣.
上帝你為我成就.
我們每次祈禱.
是這樣祈禱的.
上帝我要這樣.
我要那樣.
求你幫助我.
耶穌不是這樣教我們祈禱的.
我們經常讀.

$^{1161}$你們未求已先.
你們的父早已知道了.
不用像外邦人那樣祈禱.
Peterson翻譯外邦人.
就是那些不認識真神的人.
他們會祈禱.
很多事情要求.
不過他們不認識神是誰.
所以搞錯了.
以為話多了必蒙誰聽.
以為人多了必蒙誰聽.
以為說廿四小時.
又必蒙誰聽.
慘了.
聖經從來都不是這樣答應我們.
你們祈求就給你們.
敲門就開門.
尋找就尋見.
這麼簡單這麼容易.
不用祈禱了.
要怎樣祈禱.
先求神的光和神的義.
不懂怎樣祈禱.
耶穌教了你.
你會背.
我們在天上的父.
願人都尊你的名為聖.
再背下去.
願你的國降臨.
還有.
願你的旨意成就在地.
如同成就在天.
這樣都要日用的飲食.
這是另一件事.
針研第三十章.
你回去看清楚一點.
不是求日用的飲食.
是求不要貧窮.
不要富足.
你讀清楚一點.

$^{1201}$你讀那句就不出聲了.
主啊,我不要貧窮.
不過我要富足.
所以我們祈禱.
有時真的要讀聖經.
不是說得出就說.
求神的旨意成就.
馬丁路德說得好.
神的國自己來的.
用不著你求.
神的旨意一定成就.
用不著你祈禱.
為什麼要祈禱.
路德說.
當神的國降臨的時候.
主啊,不要漏了我.
當神的旨意成就的時候.
主啊,成就在我的身上.
照你的意思.
不是我的意思.
很實際的.
你病的時候怎麼祈禱.
主啊,醫治我.
可以.
你心裡是這樣想.
天父早已知道.
祂不知道嗎?.
你騙不了祂.
不過祈禱最後.
你效法主耶穌.
父啊,倘若可行.
求你叫這個杯離開我.
祈願了嗎?.
還沒有.
然而.
你讀下去.
要照我的意思.
照你的意思.
這個叫做侍奉.
照神的意思.

$^{1241}$讓祂成就祂的傑作.
在哪裡?.
在天上?.
在教會當中?.
在牧師的家庭?.
不是.
在我和我的家.
主啊.
求你的旨意成就.
求你的國度降臨.
求你的名得到榮耀.
不要張嘴.
不要插手.
不要祈禱搖動神的手.
好嗎?.
你搖錯了.
你仰望祂就行了.
明白嗎?.
差之毫釐.
謬之千里.
我們以為我們知道.
什麼是最好的.
我們以為我們懂.
上帝聽我.
聖經從來都不是這樣教的.
你可以說.
不過你是祂的兒女.
兒女不用說.
我們開玩笑.
動動尾巴就知道他想什麼.
長大了.
到女兒跟我說.
爹地.
你動動尾巴.
我知道你做什麼.
是啊.
祂聰明了.
不用說.
眨一隻眼就懂.
一個眼神.

$^{1281}$大家就溝通了.
上帝是很好的.
祂一定將最好的給你.
不用你求.
信得過嗎?.
有些時候.
你信不過.
你就要插手.
哎呀.
病了這麼久都不好.
搬一缸金魚回來吧.
搬一張床吧.
對過另一個方向.
搞什麼鬼啊?.
哪一科啊?.
你明白我的意思嗎?.
沒有啊.
都有些學問的風水.
你兜得回的了?.
你聽聽上帝怎麼說.
不要去侍奉上帝任何的一個.
這個傑作.
是上帝單獨完成在你的身上.
你試一下教一個畫家怎麼畫畫.
我們都不是畫家.
你試一下教人怎麼抓象棋.
會趕你走的.
觀其不與真君子.
看我怎麼抓.
三步殺他.
就是不是了.
你這個抓錯了.
你明白我們嗎?.
我們自作聰明.
在我們的生命裡.
我們每次都教上帝.
最好是這樣.
我想這樣.
行不行?.
還要掌摑上帝.

$^{1321}$你不放手.
讓上帝去成就他的傑作.
什麼叫侍奉神?.
約書亞記二十四章解釋得很清楚.
你現在要選擇.
選擇Hands off yourself.
不要自己抓住上帝的手.
要他完成我的計劃.
或者我的兒子.
我想他這樣.
我的女兒.
我想她這樣.
我的下一代.
我的上一代.
我的家.
我的事業.
我的我的.
忘記了那個我是上帝.
不是我.
你再讀約書亞記.
很清楚.
到最後.
你要選擇.
很難的.
對人來說.
最難是放下自己.
越聰明越難.
越懂得思考.
越難信服.
總覺得自己有些東西看到.
上帝看不到.
邏輯上沒可能.
但這裡總是這樣覺得.
不合理的.
服侍上帝.
你選擇.
所以保羅提醒我們.
提得好.
所以弟兄們.
包括姐妹.

$^{1361}$我以神的慈悲.
勸你們.
什麼?.
將身體獻上.
當作活祭.
可以衝出去工作.
獻上身體.
全獻在壇上.
那祭生可以做什麼?.
死了.
放在這裡.
點火.
沒什麼可以做.
保羅說活祭.
活的就可以衝出去工作.
不是.
活的就是願意信服上帝.
任憑主的旨意成就.
所以是聖潔的.
是神所喜悅的.
你們如此侍奉.
乃是理所當然的.
還有下面第二節.
不要效法這個世界.
只要心意更新變化.
叫你們測驗.
何謂神善良.
純全可喜悅的旨意.
是神的旨意成就.
成就在我的生命當中.
這個叫做侍奉.
這個叫做作神的僕人.
不用你做事的.
我以後回來教會動起腳.
多好啊.
聽完之後一身鬆.
你回到家裡也不能動起腳.
家務就是家務.
五毛錢才倒垃圾的不是你家.
洗完碗要算數的不是你家.

$^{1401}$見東西就要做.
那些不是侍奉.
那些是本份.
明不明.
那是家.
你打開這個家.
你很多東西見到.
執頭執尾.
不要說家.
回家也不做事.
踢球.
要教練叫你才補位.
坐一邊.
自己補的.
自己走的.
一隊球11個人.
你不去補位誰補.
很自然的.
什麼是侍奉.
什麼是作神的僕人.
是讓神成就他的心意.
合神心意的服侍.
不是你做什麼.
討主喜悅.
是你來到主的面前.
你說我將身體獻上.
成為祭物.
但是我一生.
不是等死了.
是我一生每天都願意.
你的意思成就在我身上.
你帶我去哪裡.
你叫我做什麼.
你叫我脫離這件事.
我脫離.
你帶我進到新的境界.
我遵命信服.
離開我的舒適圈.
我又再走多一步.
感謝主.

$^{1441}$叫人的榮耀.
就是這樣彰顯.
叫人見到我們.
有背錯聖經.
不過翻譯是這樣譯.
好行為.
不是.
叫人見到.
上帝的心意.
成就在我的生命當中.
走出來的.
其實不是我能夠做的事.
榮耀就歸給我們在天上的父.
就是這麼簡單.
肯還是不肯.
約書亞說.
想清楚.
不要加把口.
不要幫手.
不要掐我上帝在身上.
祂要成就最美好的心意.
我們同心一起祈禱.
多謝天父.
你的說話解開.
發出亮光.
使我們得釋放.
得自由.
得智慧.
我們恭敬.
將你的話藏在心裡.
求聖靈.
在我們需要的時候.
提醒我們.
信服你.
跟從你.
願意見到的.
不是我們所求所想的.
夢理英雲.
而是你最高的心意.
成就在我的家.

$^{1481}$在我的身上.
榮耀.
單單歸還給你.
奉耶穌基督的名祈禱.
Amen.
(掌聲).
我們多謝院長給我們的教訓.
我們的訓練.
記得我們今晚回家想清楚.
在這裡我也多謝敬拜隊.
剛才帶領我們一起敬拜天父.
我也代表忠臣.
多謝城北華人基督教會.
今晚給我們的地點.
款待我們.
也特別多謝我們城北堂的AV隊.
我們今晚的直播轉播.
是很多功夫的.
我們27個聚會點的工作人員.
和我們城北堂負責的弟兄姊妹.
做過多次的測試.
才敢今晚在這裡直播轉播.
大家都是很努力.
用了很多的時間.
很多的精神.
很忍耐.
忠誠的合作.
也多謝香港忠臣拓展部.
給我們有很多方面的提示和指點.
這一切.
最後我也很感謝.
我們很多的聚會點裡的弟兄姊妹.
為今晚的聚會作出不停的禱告.
以致我們今晚可以很暢順的進行.
感謝神聽我們的禱告.
以下的時間.
我們請忠臣加拿大董事會主席.
李偉廷先生.
為我們今晚有一個感恩禱告.
(感恩禱告).

$^{1521}$愛我們的主.
我們同心的感謝讚美你.
你用你自己無比的愛和慈性愛索.
招聚我們眾弟兄姊妹一起.
一同見證上帝賜給忠臣.
過去無比的恩典.
豐盛的恩典.
我們更加為忠臣獻上無限的感恩.
主人我們感謝你.
藉著中國神學研究院所舉辦的.
生命之道的查經.
更加激勵弟兄姊妹愛聖經.
愛讀經.
愛慕主你自己的話語.
我們祈求主更加的使用.
中國神學研究院.
承擔神學教育的使命和意象.
訓練更多的傳道人.
訓練更多神的僕人.
主人我們再一次的感謝你.
藉著忠臣的院長李思敬牧師.
今天晚上說一個這麼切身的題目.
是合神心意的侍奉.
我們今天能夠站在台上侍奉.
站在教會裡面侍奉.
全是神的恩典.
讓我們清清楚楚知道.
我們要尋求神的旨意.
成就神的旨意.
主人我們繼續使用.
我們每一個在你面前的兄弟姐妹.
能夠更加的投入教會的侍奉.
更加的多領人歸主.
結美好的果子.
為你作美好的見證.
求神使用我們每一個跟隨你的人.
祝福每一個願意侍奉你的人.
讓我們一生跟從你.
讓我們尋求神你自己的旨意.
榮耀我們誠心的禱告.

$^{1561}$是奉主耶穌基督得勝的名求.
各位請坐.
請大家有時間默禱.
然後彼此民安.
(掌聲).
\newpage



\section{}
\label{sec:JHQ2Beaggow}
\textbf{中神北美培靈講座 2017 「你們說,我是誰?- 作門徒的召喚」 - 余達心牧師}
\newline
\newline
連結: \href{https://youtube.com/watch?v=JHQ2Beaggow}{\texttt{ https://youtube.com/watch?v=JHQ2Beaggow}} ~~~~ 語音日期: 2017-06-22 
\newline
\newline
\hyperref[sec:2TEwldoXzT8]{\small{< < < PREV SERMON < < <}}
~
\hyperref[sec:index]{\small{[返主目錄]}}
~
\hyperref[sec:ml6Ww0ZQVk8]{\small{> > > NEXT SERMON > > >}}
\newline
\newline
$^{1}$(音樂).
其實我很喜歡別人叫我做漁牧師.
我從1982年開始就在牧會.
我不是做顧問.
我是親自去牧會.
1985年我建立自己的堂會.
由六個人開始.
我們看著教會一直成長.
我一直陪著這間教會成長.
之後我就交低.
然後就進入天水圍幫助我太太.
她在天水圍建立另一間教會.
我們在天水圍牧養最基層的教會.
牧養了十多年.
有人問我如果要給我一個稱呼.
我喜歡選擇什麼稱呼.
要叫我神學家.
Deologian 還是Pastor.
我不用想.
我最基本就是Pastor.
當Pastor帶領人更深的認識真理的時候.
我是用我的神學思考作為教導的工作.
十多年前我就說.
我有兩道走天涯.
有兩篇道走天涯.
第一篇道是C篇19篇.
另外一篇道是C篇119篇.
如果大家心水清都知道這兩篇道講什麼.
是講神的話語.
神的話語是怎樣的重要.
在我們生命成長裡面是怎樣的重要.
為什麼要講這兩篇道呢?.
因為我牧養的時候.
越來越覺得年輕的一代.
對神的話語的興趣越來越薄弱.
願意花時間去研讀神的話語.
這種熱切是越來越淡薄.
心裡很擔心.
我成長的過程當中.
有一個弟兄和一個姐妹.

$^{41}$在我一信主就帶領我去研讀神的話語.
每一個星期跟我查經.
查了兩年.
然後我就按門鈴.
自己來懂得怎樣讀經.
這在我整個屬靈生命的成長裡面.
成為一個很重要的基礎.
因此我很盼望弟兄姐妹.
都能夠建立這樣的基礎.
為此我就開始推動弟兄姐妹讀經.
生命之道之所以成為忠臣一個很重要的事工.
我都是在背後出了一些力.
在過去六七年.
我講道經常有一個主題.
剛才是兩道走天涯.
我現在常常都是一個主題.
這個主題是什麼呢?.
是重新將我們的焦點放在作主的門徒.
我們很容易在我們很忙碌.
在各樣事工的需要底下.
來團團轉,我們忙這個忙那個.
都是一些很好的教會發展的事工.
但是在這個忙碌當中.
我們很容易會忘記了.
我們的信仰一個最重要的核心.
就是我們要跟從我們的主耶穌基督.
作祂的門徒.
作祂的門徒做什麼呢?.
作祂的門徒就是我們整個人的生命被轉化.
我們過往有的那些價值.
完全被改造.
甚至有一些價值被革命.
被顛倒過來.
或者可以說是被顛覆.
我們很容易忘記我們自己是誰.
我們是主的門徒.
我們先做主的門徒.
我們先去想.
我們可以為我們的主做什麼.
所以今天我很想跟大家回到這個.

$^{81}$作為一個基督徒.
我們最基礎性的焦點.
這個主題的上面.
所以我今晚就跟大家分享.
《馬太福音》第十六章十三節至二十六節.
一段非常重要的經文.
在福音書裡面非常重要的經文.
但是亦不常聽到講解.
或者在講道上面宣講的一段經文.
請聽我讀出第十三節開始.
耶穌到了凱撒尼亞.
或者該撒尼亞菲勒比的境內.
就問門徒說.
人說我人子是誰.
他們說有人說是施洗約翰.
有人說是以利亞.
又有人說是耶利米.
或是先知裡的一位.
耶穌說:你們說我是誰.
西門彼得回答說.
你是基督.
是永生神的兒子.
耶穌對他說:西門巴約亞.
你是有福的.
因為這不是屬血氣的.
只是你的.
乃是我在天上的父.
只是你的.
我還告訴你.
你是彼得.
我要把我的教會.
建造在這盤石上.
陰間的權柄或者陰間的權勢.
不能勝過他.
我要把天國的約時給你.
凡你在地上所捆綁的.
在天上也要捆綁.
凡你們在地上所釋放的.
在天上也要釋放.
當下.

$^{121}$耶穌祝福門徒不要對人說.
他是基督.
21節.
從此.
耶穌才指示門徒.
他必須上耶路撒冷去.
受長老,祭司,文士許多的苦.
並且被殺.
第三日復活.
彼得就拉著他說.
勸他說:主啊!萬不可如此.
這事必不臨到你身上.
耶穌轉過來對彼得說:撒旦.
去我後面去吧.
你是伴我腳的.
因為你不體貼神的意思.
只體貼人的意思.
於是耶穌對門徒說.
若有人要跟從我.
就當捨己.
背起他的十字架來跟從我.
因為凡要救自己生命.
必喪掉生命.
凡為我喪掉生命的.
必得著生命.
人若賺得全世界.
賠上自己的生命.
有什麼益處呢?.
人說:什麼換生命呢?.
這段經文.
是在整卷的《馬太福音》裡面.
極為關鍵性的一段經文.
亦是整個舊恩歷史聚焦的所在.
新約學者Donald Heckner.
詮釋這段經文時.
他用了三個詞彙.
來標示這段經文的核心意義.
第一個字是高潮.
第二個字是轉捩點.
第三個字是途徑.

$^{161}$這段經文是整卷《馬太福音》前半部的高潮.
前半部一直說.
耶穌基督在加利利行走.
這個醫病趕鬼.
行各樣神蹟騎士.
並宣揚上帝國度的拉比.
他真正的身份是什麼呢?.
就在這段經文揭示出來.
這亦是全書的一個轉捩點.
他將舊屬所必須的.
必須的.
舊屬必須的.
宣示出來.
就是按上帝舊屬的旨意.
基督必須受苦.
耶穌基督自己強調是必須.
基督必須受苦.
是被釘十字架.
同時跟隨他的人.
亦都是必須背起十字架.
走十字架的道路.
要得到舊屬嗎?.
背十字架是必須的.
不單止是這樣.
要得到舊屬的人.
他一定要同時投入參與.
基督拯救人類.
這個使命.
而這條正正就是十字架的道路.
我剛才說必須.
必須不是我加上去的.
在聖經裡的原文.
或英文的翻譯.
是很清楚的.
It is necessary.
必須.
現在讓我們從.
耶穌基督.
在哥爾薩尼亞菲勒比的場景.
轉一轉過來.

$^{201}$轉到我們現在歷史的場景.
今日的教會.
可以說是站在教會歷史裡面.
上面的一個高潮.
賓夕凡尼亞大學.
歷史學的教授.
Philip Jenkins.
在他寫的一本書.
如果我沒有記錯.
應該是2000年出版的一本書.
叫做The Next Christendom.
The Coming of Global Christianity.
在這本書裡面.
他指出.
教會從未有.
從來都沒有像今天般興盛.
基督徒的人數的增長.
從來沒有過像今天般驚人.
他認為.
一個基督教國度的新紀元.
是慢慢地來開展.
但是.
我們同時也站在歷史的轉捩點.
是一個充滿危機的轉捩點.
我們縱觀過去.
從來沒有一個世代像今天般.
道德價值的崩潰.
是來得這麼徹底.
人心的迷失.
是如此普遍和嚴重.
否定神,敵黨神的氣勢.
沒有今天般囂張.
然而就在這個時候.
教會的內在生命.
也從未有過如此脆弱.
信徒的信仰.
你可以說是如此膚淺.
神學也是如此混亂.
約翰·斯托克在他那一本.
《The Living Church》裡面.

$^{241}$憂心忡忡地說.
今天的教會闊是三千里.
但深度只有一寸.
不少的信徒他說.
只是在基督裡面作鷹鞋.
教會身處極大的危機.
而不自知.
他在另一本書.
《The Radical Discipleship》裡面指出.
今天基督徒如果要經歷復興的話.
唯一的途徑.
就是回到追隨基督的路上.
也是十字架的道路上.
不少的人.
特別今天不少的人.
信徒也好教會也好.
是會因為教會人多勢眾.
而沾沾自喜.
殊不知.
沒有生命.
沒有使命感.
這樣的人多勢眾.
可能是叫教會潰敗得更快.
也更重.
當教會為增長.
為發展.
為各樣事工.
在這裡團團轉.
上下忙得透不過氣來.
他就真的很容易忘記一件事.
信仰最核心的那件事.
就是做基督的門徒.
就是我們的生命.
徹底地被福音轉化改變.
今晚我希望和弟兄姊妹.
回到這個起點上.
這段經文所記述的高潮.
所記述的轉捩點.
在哪裡發生?.
在一個很特別的地方發生.

$^{281}$其實耶穌基督可以隨時隨地.
都向他的門徒講明.
他是誰.
一班跟從他的人.
他們應該怎樣背起十字架來跟從他.
但耶穌基督偏偏選擇在這個地方.
向他們宣告他的身份.
這個地方是哪裡?.
這個地方就是剛才我們讀那段經文所講的.
蓋撒利亞菲勒比.
是一個極具象徵性的地方.
蓋撒利亞菲勒比是希律的兒子.
菲利所建造的.
本來他是用自己的名字來命名.
菲利普就是菲勒比.
這樣來命名.
但這個人是很有政治觸覺的.
當他要把自己的名字.
擺在這個城上.
為這個城命名的時候.
他突然間想起這樣做.
可能不是.
在政治上不是政治正確.
所以他在這個自己.
菲勒比這個名字前面.
換了另外一個名字.
就是蓋撒利亞Caesarea.
是來表示向蓋撒效忠.
這個城就叫做蓋撒利亞菲勒比.
這個城基本上就是一座.
你可以說是希臘化的城.
為什麼這樣說呢?.
因為它裡面有一個很大的廟.
是供奉著希臘的神Pan.
當然在這個城裡面.
為了政治正確.
建了一個很巨型的.
用白色的大理石建造的一座廟.
供奉誰呢?.
就是供奉蓋撒.

$^{321}$考古學家在這個城裡面.
也挖出了多座巴力的神廟.
他們很相信.
在當時的Fertility cult.
是崇拜生殖能力的.
這種宗教是非常盛行的.
而這個城也位於黑門山腳.
面對著一個很巨大的岩石懸崖.
在這個懸崖上.
作了不少的神像.
並且挖了很多洞.
這些洞就是供奉一些神.
有些神像放在裡面.
也因此可以稱為神殿.
因此這個石崖被稱為眾神之石.
蓋撒令亞菲勒比就是一座這樣的城.
而菲力是極盡奢華的建造這個城.
以它來象徵政治,宗教,文化,經濟的勢力.
這座城是一座極其世俗化的城.
因此猶太人看這座城是世俗罪惡的象徵.
一個好的猶太人是不應該去這座城的.
這個地方還有另一個特點.
就是在石崖的下面有一個巨大的洞穴.
洞穴裡面有一個深淵.
相傳這是約旦河的河源.
約旦河的河水從那裡湧出來.
亦相傳這是陰間的門.
是巴力出入的門戶.
象徵著罪惡勢力盤踞的地方.
耶穌基督就在一個這樣的地方.
一個政治,經濟,宗教,勢力非常強大的地方.
亦同時是幽暗勢力在那裡盤踞的地方.
宣告祂是誰.
讓門徒很清楚知道祂是誰.
祂在那裡宣告.
祂要將這個世界的價值徹底改變.
祂要將這個世界的命運徹底改變.
門徒一定要知道.
他們的信仰不是活在雲端.
他們的信仰一定要活在這個世俗的時代.

$^{361}$在這個充滿罪惡,幽暗勢力的世界.
而他們的門徒要用他們的生命.
來與這個幽暗,罪惡的世界對抗.
並且要改變這個世界.
就是在這個地方.
耶穌要門徒表白.
他們到底知不知道跟隨著的是誰.
耶穌先問他們.
人說人子是誰.
人家說我是誰.
人子你直譯的話.
亦都可以譯作是我.
人家說我是誰.
門徒就說有人說你是施洗約翰.
是以利亞,是耶利米.
或者是先知的一位.
說出來這些名號都是當時.
你可以說是猶太人非常尊崇的一些名號.
施洗約翰可以說是當代的先知.
有不少的猶太人當時跟了他.
他帶動了當時的以色列人.
認罪悔改.
回到神所要求的那種敬虔,聖潔.
很多人都因為這樣.
生命得到改變.
他是象徵以色列復興的盼望.
有人就說這是復活的施洗約翰.
這樣說是給耶穌一個很高的地位.
下去後,他的地位就更高.
有人說他是以利亞.
為何會說以利亞在這個時候出現?.
如果你看《馬拉基書》三章一節.
是這樣說的.
「萬軍的耶華如此說:.
我要差遣我的使者在我面前預備道路.
你們尋求的主必忽然進入他的殿堂」.
當神要來之前.
他要他的使者預備道路.
宣告他要來.
說他是以利亞的意思.

$^{401}$就是說耶穌就是宣告.
上帝要來的那位使者.
亦都是有一個很重要的歷史任務的使者.
有人說他是耶利米.
或者可以這樣說.
耶利米相對來說更加重要.
因為在這裡很多猶太人相信.
在末世神要審判這個世界之前.
耶利米先知,火同,以賽亞先知.
是會一同出現在這個世界.
於是末世將臨.
甚至有人相信.
耶利米是帶著失去的約衛回來.
以表示神的道,神的同在再一次臨到這些以色列人.
這些曾經背叛神的以色列人.
神的赦免,神的擁抱將會來到.
神將要復興他們.
有人說他是耶利米.
都是很高的榮譽.
世人真的很尊崇耶穌.
那時候是這樣的.
今日都有不少人尊崇耶穌.
認為他是先知.
認為他極具智慧.
一個這樣的智者.
他能夠預視人間的危機和困迫.
可以為他們指點迷津.
將生命的道點出來.
值得受至高的崇敬.
但如果他只是先知的話.
我們就可以遠遠地跟著他.
只是聽他的指點.
和我們的生命可以沒有很大的關係.
世界上很多人都想遠遠地跟從耶穌.
只是遠遠地聽他的智慧之言.
他們不願意跟從耶穌.
走他所走的路.
事實上在當時的門徒當中.
可能也有一些門徒是這樣看他.
耶穌聽完人說他是誰之後.

$^{441}$轉過來問他的門徒.
你們說.
應該直譯的話.
應該還要加上「自己」.
你們自己說.
Humans.
這裡有個希臘文字Humans.
就是自己.
你們自己告訴我我是誰.
耶穌這樣問表明了什麼.
表明人說我是誰不重要.
最重要的是你們這些跟從我的人.
你們說我是誰.
這才是關鍵的所在.
西門彼得.
Petros.
他的希臘文名是Petros.
回答說你是基督.
是永生神的兒子.
這是一個決定性的宣認.
耶穌基督說西門巴約那.
你這個宣告一定不是從你那裡出的.
一定是天上的父感動你.
才可以出這個宣告.
我想你記住.
他稱彼得為西門巴約那.
一會兒你就知道為什麼我要這樣強調.
接著.
耶穌說我要告訴你.
你是彼得Petros.
你是彼得.
不同的解經家對這句話有不同的理解.
有人說耶穌基督這樣說.
其實就好像當時耶和華和阿伯拉罕.
叫他改名為阿伯拉罕.
有異曲同工之妙.
在這裡耶穌基督極可能.
是向西門巴約那宣告.
他要有一個新的名字.
他的名字就是石頭.

$^{481}$或者他的名字就是盤石.
耶穌繼續說.
我要把我的教會建立在這盤石上.
Petra.
不少天主教的解經家.
認為耶穌基督是宣告彼得.
這個宣認耶穌基督是神的兒子.
宣認耶穌基督是基督的彼得.
就是教會的盤石.
耶穌基督是要宣告彼得是教會的盤石.
但奇怪的地方是.
他剛才叫彼得做什麼?.
Petros.
是一個男性的名稱.
他說我要把我的教會建立在這個盤石上.
這個盤石是Petra 是女性.
所以你可以說.
基督教的解經家.
我們覺得.
耶穌基督不是要把教會建立在彼得這個人上.
而是建立在彼得的宣認上.
而他叫彼得是Petros.
是石頭.
和建立教會的盤石.
就是彼得的宣認.
同一樣的東西都是石頭.
為什麼這樣說?.
有些解經家認為.
凡是認耶穌是基督的人.
他都屬於這個盤石.
是和這個盤石有份.
教會的基礎是相信宣認耶穌是基督.
是神的兒子.
亦都是神自己.
就是神自己.
這個宣認帶來了一個很深的含義.
祂是救主.
祂是我們的神.
但作為救主.
祂是親臨人間來救贖我們這樣的一位神.

$^{521}$亦都是那個真正能夠親自摧毀罪惡勢力的一個主.
是一個能夠將人從罪的捆綁中釋放出來的主.
耶穌基督在這段經文裡.
其實最重要的.
不是要宣告祂是一個怎樣偉大的主.
祂是要宣告.
凡是宣認祂是基督的.
祂要建立他們成為祂的教會.
建立在一個盤石上.
很穩固的盤石上.
而這個教會是甚麼呢?.
這個教會是大有能力的一個群體.
基督向他們宣告.
陰間的權勢不能勝過祂.
陰間的門不能阻擋祂.
陰間的門擋不住教會的力量.
教會因此可以進入到陰間最幽暗的地方.
將被捆綁的人釋放出來.
教會要進入到幽暗勢力操控的領域.
將那些勢力粉碎.
將被捆綁的人釋放.
這一段話.
在蓋薩利亞菲勒比.
在陰間的門的面前宣告出來.
含義是非常深.
基督將教會放在罪惡之城.
世俗潮流的當中.
要他們做甚麼呢?.
是要他們做生命的轉化者.
他們怎樣有能力這樣做呢?.
我們看看我們自己.
我們常常都覺得我們自己是軟弱無力.
耶穌基督說我們是.
敗有權柄的一個群體.
我們可以改變人的命運.
我們將人從幽暗的操控當中釋放出來.
我們怎樣有能力這樣做?.
我們有沒有能力這樣做?.
就是端視乎我們是否按照上帝的旨意去做.
上帝的旨意是怎樣?.

$^{561}$上帝的旨意就是要我們跟從耶穌基督.
上帝要我們走的路.
就是基督走過十字架的道路.
教會還有一個更重的權柄.
是耶穌將天國的約事交給教會.
約事是代表掌管這樣的權柄.
耶穌更進一步的說明.
這個是代表教會要釋放哪一個.
哪一個就會被釋放.
他要捆綁哪一個.
哪一個就會被捆綁.
這個是一個權柄.
陰間的勢力是阻不到教會這樣做的.
你說這是多麼的榮耀的宣告.
但是基督徒真的有這樣的能力嗎?.
印度的聖雄甘地對基督徒說了一句.
很不客氣的話.
他說:You Christians are very ordinary people.
making big claims.
你們這班人誇誇其談.
說自己是怎樣是怎樣.
但是你們只不過是很普通的人.
口出狂言.
這樣的取笑不是沒有道理的.
我們很多時候看看我們自己.
我們都不能夠不問.
我們真的有能力嗎?.
我們真的有能力來抵禦世俗的洪流.
這個能力是從哪裡來的?.
來到這裡耶穌基督宣告完之後.
話鋒一轉.
你可以說是進入了反高潮.
令人振奮的宣告.
令人覺得榮耀的宣告.
突然間洗門逃苦勞不堪.
甚至可以說是不知所措.
宣告了這個這麼震撼性的事實.
聖經告訴我們從那時起.
為什麼要說從那時起呢?.
因為耶穌不止一次.

$^{601}$而是兩次再重複.
來向他們宣告.
耶穌向門徒說明.
還怕我們不明白.
是說耶穌向門徒明說.
用英文翻譯是show.
祂是怎樣呢?.
必須.
原文day.
It is necessary.
第23節.
當彼得要阻擋耶穌的時候.
耶穌怎樣跟他說呢?.
你是撒旦.
你體貼人的意思.
你不明白是神的意思.
是神的意思.
耶穌基督要上耶路撒冷.
受長老,祭司和民事的苦.
並且被殺.
這是神的意思.
因此祂是必須.
這條路祂是必須要走.
關鍵的所在.
我們得能力關鍵的所在.
我們得權柄能力的所在.
在哪裡呢?.
就在十字架.
在死亡.
在捨棄.
這是答案的所在.
亦是能力的所在.
上帝的心意就是這樣.
要戰勝邪惡.
只能夠以上帝的美善.
以祂極端的人愛.
結合了祂的公義.
同時以祂的公義.
來結合祂的愛.
才能夠讓我們有力量.

$^{641}$戰勝邪惡的勢力.
這一切都一定要用我們的生命來活現出來.
基督首先就是這個活現的生命.
十字架就是一個最清晰的表述.
我們作祂的門徒.
我們吩咐被吩咐.
要出去改變這個世界.
我們必須要將世界高舉的價值.
倒轉過來.
革祂的命.
今日逐世的洪流.
以自我為絕對.
以自我為法律.
以滿足成全自我為最高的價值.
基督是怎樣呢?.
我們要向世界的價值說不.
反其道而行.
我們追求的是什麼?.
我們追求的.
我們操練的.
就是放下我們自己.
以否定自我來作為成全自我之路.
很吊詭的.
十字架的道路.
就是戰勝邪惡唯一的方法.
捨己之路就是剋性自我中心之道.
沒有第二條路.
所有跟從祂的人.
都要走這條十字架的路.
以捨己的生命來改變轉化這個世界.
所以耶穌基督說.
若有人要跟從我.
就必捨己.
這裡和合本聖經譯為.
當就當捨己.
強烈過當.
不是應該的意思.
是必須.
就必須捨己.
背起自己的十字架來跟從我.

$^{681}$教會是建基在宣認耶穌是基督這個盤石上.
但基督是誰呢?.
基督就是那個被釘十字架.
那位的神.
祂就是神.
但祂被釘十字架.
我們這位神.
是道成肉身的神.
祂所走的路就是十字架的路.
祂的標記也是十字架.
祂生命的實質是捨己.
祂最高的榮耀在哪裡彰顯出來?.
祂最高的榮耀就是.
倒空祂自己本來有的那種榮耀.
認耶穌為基督的教會.
存在的基礎就是背起十字架.
以十字架為生命之路.
以十字架作為身份的標記.
不背十字架.
就算你賺得全世界.
就算教會賺得全世界.
所有人都湧入來.
但如果這個教會是不背十字架的教會.
我們就會失去.
失去什麼呢?.
失去最寶貴,最重要,最核心的一件事.
就是生命.
耶穌基督說得很清楚.
不肯捨己的.
縱然你賺得全世界.
你始終會喪掉你的生命.
這是非同小可的結局.
今天我們的教會是怎樣?.
我們不能夠不反省.
當斯託德說今天的教會闊三千里.
深度只有一寸.
不少人在基督裡面只是鷹孩兒.
我們又怎麼看?.
我們同不同意這是今天教會真實的寫照?.
無關乎.

$^{721}$有人好像甘地的取笑我們.
我們說我們是一個大能的群體.
我們唱詩不斷地唱.
我們是大能的群體.
但我們的能力在哪裡?.
世界依然顧著我們.
自戀.
我們有什麼力量去改變這個世界?.
問題在哪裡?.
如果我們說我們沒有能力的話.
問題在哪裡?.
不少教會所宣講的.
很對不起.
我一定要在這裡說出來.
不少教會所宣講的是廉價的福音.
Cheap Grace.
潘博華在他那本《追隨基督》裡.
劈頭就說廉價恩典和重價恩典的對比.
廉價恩典是教會的死敵.
我們要為重價的恩典而戰.
重價的恩典就好像雜價攤上賤賣的平價貨.
我們將聖禮,赦免全部廉價來傾銷.
廉價恩典是將恩典化成一套教義.
一個原理,一套系統.
宣告罪得赦免.
作為一個補遍的真理.
將神的愛作為基督教神官來教導.
廉價恩典是指罪的清義.
而無需罪人的清義.
宣講赦免而不需要悔改.
受洗而不需要接受教會的紀律.
團契生活是沒有彼此認罪.
沒有彼此問責.
廉價恩典是沒有門徒生命的恩典.
是沒有十字架的恩典.
是沒有道成肉身的基督的恩典.
重價恩典正是我們今天必須得著的福音.
這個恩典重價.
因為它召喚我們追隨.
而它是恩典.

$^{761}$因為它召喚我們追隨的是我們的主耶穌基督.
我們本來沒有資格跟隨祂.
但祂召喚我們跟隨祂.
它是重價的.
因為祂叫人以生命作為代價.
它是重價的因為祂譴責罪.
它是恩典因為祂叫罪人清義.
重價恩典.
叫我們這位最榮耀的上帝.
竟然道成肉身並且走上十字架.
願意追隨基督的人.
一定要甘願的.
跟祂一樣背起十字架.
並且跟從祂走祂走十字架的路.
潘博華在追隨基督這本書一開頭.
就講一句非常重的話.
他說:When Jesus calls a man to follow Him,.
He bids him come and die..
當耶穌呼召一個人.
祂叫他來是要他死.
我相信今天我們很少聽到.
耶穌要我們死這個訊息.
但我告訴大家.
我信主我成長的年代.
常常聽到我們要預備釘十字架.
我們預備將我們的老娥釘死.
並且同時我們隨時預備為主的緣故.
放下自己的生命.
不少人信耶穌.
是因為受到上帝愛的感動.
或者是為了得到上帝的祝福.
從來沒有想過要追隨基督.
沒有想過要背起十字架來追隨祂.
各位弟兄姊妹.
我不知道你們決定信耶穌的時候.
心裡所想的是什麼.
你知不知道當你承諾自己.
要去信耶穌的時候.
將自己承諾是一個信仰的時候.
你有沒有想過要背起十字架.

$^{801}$跟從耶穌基督.
貫徹耶穌基督生命的形態.
這樣來做一個基督徒.
不少人願意跟從耶穌.
受他捨己愛人的情操所感動.
亦都願意仿效.
但是很少人想過會跟從祂上十字架.
所以有人說.
I am willing to follow Jesus to the cross.
to the cross, but not on it.
不是在上面.
教會必須在這個無神的世界當中.
分上神為世界所受的苦.
追隨基督.
甘負代價.
我們其實不時都見到.
有些認真的基督徒.
他們作了生命的示範.
2011年我和我太太在歐洲旅行.
是一個12天的旅行.
在我們團裡絕大部分都是老外.
只有一個香港來的家庭.
我們當時參加這個旅行時.
我們在愛丁堡.
當然一個香港的家庭.
我們當然每天吃飯.
三餐都跟他們在一起.
有談有說.
有一晚.
四人家庭的男人的丈夫.
大罵香港的醫生.
都不知道為什麼.
大罵香港的醫生.
沒用的.
一味懂得賺錢.
你有沒有聽過什麼叫星球人.
一球你知道是什麼嗎.
一百萬.
你人家是很會投資.
一百萬.

$^{841}$星球人就是一個星期.
賺一球就一百萬.
月球人就差一點.
一個月才賺一百萬.
只懂得賺錢.
我就忍不住.
我就反駁他.
我說不是.
我說香港其實有很多很優秀的醫生.
我就認識不少.
很願意服務的醫生.
我說我認識一個朋友.
他說他是一個很出色的醫生.
很年輕就做了顧問.
然後默默地在公立醫院服務.
訓練了很多人.
他訓練出來的醫生.
做了幾年之後就出去賺大錢.
他就繼續留在崗位.
繼續這樣做.
有一天.
有個私家醫院重金挖他過去.
給他幾倍的錢.
而且做手術有花紅分.
他可以賺很多錢.
他走來跟我說.
我問他.
你怎麼做決定.
他說當然不去了.
我說不去?.
多很多錢.
他說如果我去了這個私家醫院.
就只有那些有錢的人.
才可以用到我的服務.
那些沒錢的人怎麼辦?.
也因為這樣的緣故.
他就留在這個醫院裡面.
在公立醫院的系統裡面.
我跟這個人這樣說的時候.
他一說完.

$^{881}$他就說你是否在說某某?.
我們認識他嗎?.
他說這個人.
在我們醫療界.
香港的醫療界.
是一個很重要.
影響力很大的一個人.
我說你認識他嗎?.
我說你是醫生嗎?.
他說不是.
他是誰呢?.
他說他是美國Johnson  and  Johnson.
在亞太區的老總.
他認識很多醫療界的人.
他也認識這個人.
並且給了他很多錢.
做研究的工作.
希望他能夠買他的儀器.
買Johnson  and  Johnson的儀器.
但是這個人.
我認識的這位朋友.
從來都沒有買.
他說那些.
不要說了.
免得影響別人.
一個基督徒.
他整個的價值.
是和世界完全不同.
這是一個認真的基督徒.
他要將基督的生命活出來.
我另外.
我再說一個故事.
一個很感動我的故事.
是一個真的故事.
十年前.
有一個弟兄姐妹來找我.
談他們結婚的事.
談完之後.
男的問我可否再問一個問題.
我回答問多兩個都可以.

$^{921}$他就跟我說.
他信主大概六七個月.
第一信主就被ICAC.
即是廉署.
告他賄賂.
他在一間很大的公司裡做事.
是很高層.
告他來賄賂.
他做了一點點的事.
簽了一張信.
不是他去做賄賂的事.
只是簽了一張信.
他律師跟他說一定會打掉.
沒有問題的.
很容易的.
只要記住一件事.
就是死人,塌樓.
什麼都不要承認.
OK.
他說.
我現在是基督徒.
牧師你告訴我.
我應該怎麼做.
他問這個問題的代價很大.
我說要回答你這個問題.
我首先問一個問題.
你告訴我你信主.
你是不是真的信.
很少牧師會問得這麼直接.
和問得這麼粗暴.
你是不是真的信.
他說是的,我真的信.
如果你真的信的話.
我就再問你幾個問題.
第一,我們願不願意見到.
上帝的旨意.
行在地上,如同行在天上.
他說我願意.
我們願不願意見到.
上帝的旨意行在你身上.

$^{961}$如同行在天上.
他說我願意.
還有兩個問題.
我們願不願意見到.
上帝的公義行在地上.
如同行在天上.
他知道我來勢洶洶.
他說我願意.
我們願不願意見到.
上帝的公義行在你身上.
如同行在天上.
他猶豫了一會.
然後說牧師我願意.
我如果是這樣的話.
你不用回答我.
你只要回去想想.
你有沒有做過不公不義的事.
你不用回答我.
幾天之後.
他打電話來.
告訴我他做了一個決定.
就是他決定認罪.
當他做這個決定的時候.
他的律師就罵他.
你不單止害死自己.
因為你這樣認的話.
你極可能被判坐牢20個月.
不單止這樣.
你連累街坊.
連累一群牽涉在案件中的人.
他們都會被定罪.
結果所有的人都被定罪.
他不去上訴.
其他人就上訴.
宣佈坐牢.
刑期是遠遠超乎他想像的長.
不止20個月.
我一直都不敢去探望他.
直至到5個月後.
我去探望他.

$^{1001}$我去探望他的時候.
我看到一個連樣貌都改變的人.
我問他後悔嗎.
他說不後悔.
我應該這樣做.
然後他告訴我.
我一生人從來都沒有像現在這樣.
覺得這麼釋放.
真是感謝主.
他說現在有很多時間看聖經.
我現在看詩篇.
很享受.
並且他也有很多時間.
向人介紹這位耶穌.
向人講福音.
他就是那個生命的見證.
他就是願意應服於上帝的公義.
出來之後.
過了一年多.
同一個公司裡面的一個人.
我找他出來吃飯.
我希望他捐些錢給我教會辦學校.
找他出來吃最好的日本餐.
我問他要三十多萬.
他說很容易.
三十多萬就給我.
叫教會寫封信來.
自然就提到這位弟兄.
我說這位弟兄不應該回到公司工作.
他說不是的.
牧師你錯了.
他們知道嗎.
在過去一年多.
他回來了.
他說我們的公司文化.
我們的企業文化完全改變.
我們經常抱怨.
做基督徒付代價很難.
付代價的確很難.
但是上帝會托住我們.

$^{1041}$上帝會給力量我們這樣做.
而當我們這樣做的時候.
我們的生命發揮的.
可以是極大的影響.
剛才那位醫生.
Johnson  and  Johnson亞太區的老總問我.
你也是醫生嗎.
你怎會認識這個人.
你也是醫生.
我說我不是醫生.
我是牧師.
我說這個人是我會友.
他說他是基督徒嗎.
我說是的.
這個家庭是未信主的.
我說我一輩子都未曾感覺到.
牧師做得這麼榮幸.
這個就是一個真誠的基督徒.
他用他的生命發揮出來的影響.
讓上帝真正得到榮耀.
各位靈子們.
你願不願意用你的生命.
讓上帝得到他應該有的榮耀.
我們怎樣可以做.
怎樣可以令到我們的生命榮耀上帝.
一個途徑.
就是跟從耶穌.
學他的榜樣.
我要停在這裡.
後面打出一個靈字.
我們可否做一個簡單的祈禱.
親愛的主我們感謝你.
你的恩典深到我們無法想像.
無法察覺.
至高無上的上帝.
榮耀無比的上帝.
你竟然願意來到我們當中.
降卑.
放下你的榮耀.
來到我們這些不配的人當中.

$^{1081}$擁抱我們.
托住我們.
釋放我們.
而你自己為了這個緣故.
釘身十架.
主.
我們越了解我們的不配.
我們就越感激你的恩典.
就讓我們心中的感激.
引發一個很堅定的決心.
用我們的生命.
堅決不移的.
來跟從你.
甚至.
釘十架.
主讓我們願意背起我們的十架.
每天背起我們的十架.
來跟從你.
我們禱告,服主的名,阿們.
阿們.
字幕:J Chong.
(字幕由 Amara.org 社群提供).
\newpage



\section{}
\label{sec:ml6Ww0ZQVk8}
\textbf{中神北美培靈講座 2018 「我和我家必定事奉上主」 - 黃國維博士}
\newline
\newline
連結: \href{https://youtube.com/watch?v=ml6Ww0ZQVk8}{\texttt{ https://youtube.com/watch?v=ml6Ww0ZQVk8}} ~~~~ 語音日期: 2018-10-31 
\newline
\newline
\hyperref[sec:JHQ2Beaggow]{\small{< < < PREV SERMON < < <}}
~
\hyperref[sec:index]{\small{[返主目錄]}}
~
\hyperref[sec:XGZrzl_HY54]{\small{> > > NEXT SERMON > > >}}
\newline
\newline
$^{1}$恢復會議.
主席.
各位觀眾朋友,晚安!.
很高興我們能在這裡敬拜讚美我們的神.
在詩篇裡說你們要讚美耶和華.
在神的聖所讚美他.
在他顯能力的窮倉讚美他.
耶和華偉大,該受大讚美.
氣大,無法測度.
稍後會有我們一起起立.
用詩歌敬拜讚美我們的上帝.
回應祂對我們的愛.
高唱入魂.
同唱,主恩三口,大理六人.
崇拜,願被的主高聲高唱入魂.
多謝世界,仍是冷漠.
基督降世,這生,是我們從催淚得生.
我願縱容,我願縱容,歸拜你.
來承禱,崇入魂,舉起手來,祝贈你.
祝福主愛,大恩主恩.
你賜我師父主恩.
你賜我群漢主恩.
你賜我君心不變,主長,感謝你.
同唱,主恩三口,大理六人.
崇拜,願被的主高聲高唱入魂.
多謝世界,仍是冷漠.
基督降世,這生,是我們從催淚得生.
我願縱容,我願縱容,歸拜你.
來承禱,崇入魂,舉起手來,祝贈你.
祝福主愛,大恩主恩.
你賜我師父主恩.
你賜我群漢主恩.
你賜我君心不變,主長,感謝你.
我願縱容,我願縱容,歸拜你.
來承禱,崇入魂,舉起手來,祝贈你.
祝福主愛,大恩主恩.
你賜我師父主恩.
你賜我群漢主恩.
你賜我君心不變,主長,感謝你.
主恩,我們一起去回應你的愛.

$^{41}$主恩,我們站在你面前.
我們一起感恩,感恩你的救贖.
至今日我們能夠站在你的殿裡面.
我們一起敬拜,贊美你.
最後,但願我們每一個人站在你的殿裡面.
我們都能夠將我們最好的去回應你.
用我們的生命獻上,來處理你所有.
每一個人都來處理所有.
祈禱可以供諸彌陀.
(音樂).
祝你最好的年月出,迎來年青的力量.
祝你順濟熱情生力,衷心為真理打仗.
祝耶穌以友好榜樣,勇敢堅定不懼怕.
你要衷心敬敬愛主,唱最好的年月卡.
唱你最好的年月出,迎來年青的力量.
全相救,因全副本著,衷心為真理打仗.
唱你最好的年月出,儲來偉大怒火氣.
他將自己作地所教,天上榮耀上次利.
他捨生命毫無怨言,當你奪取百血神.
你要衷心敬敬愛主,唱最好的年月卡.
唱你最好的年月出,迎來年青的力量.
全相救,因全副本著,衷心為真理打仗.
衷心為真理打仗.
唱你最好的年月出,迎來年青的力量.
全相救,因全副本著,衷心為真理打仗.
陳婉嫻唱《神乎之心》.
神與之心機,無限生氣,豐富勝在因點力量.
神屬性,蓋我半性,得性越有餘.
神與之知名,神福呼應,日氣盛在艱苦任命.
團隊制奮鬥不息,堅決齊共和.
敬最親似手,燕笑,以盡一生給主容.
用盡你共我的真誠,何牽掛.
認最交會心,齊笑,以盡一生給主容.
用盡你共我的恩情,拼搏為主.
《神與之心》.
燕笑赤子頌.
神與之知名,神福呼應,日氣盛在艱苦任命.
團隊制奮鬥不息,堅決齊共和.
敬最親似手,燕笑,以盡一生給主容.
用盡你共我的真誠,何牽掛.

$^{81}$認最交會心,齊笑,以盡一生給主容.
用盡你共我的恩情,拼搏為主.
敬最親似手,燕笑,以盡一生給主容.
用盡你共我的真誠,何牽掛.
認最交會心,齊笑,以盡一生給主容.
用盡你共我的恩情,拼搏為主.
《創世紀》第22章1-19節.
經文是這麼說.
「這些事以後,神考驗阿伯拉罕,對他說:.
阿伯拉罕,他說我在這裡.
神說:你要帶你的兒子,就是你所愛的獨子爾薩.
往摩利亞地去,在我指示你的一座山上,把他獻為凡濟.
阿伯拉罕清早起來,預備了爐.
帶著跟他一起的兩個僕人和他兒子爾薩.
劈好了凡濟的柴,就起身往神指示他的地方去了.
到了第三天,阿伯拉罕舉目要望那地方.
阿伯拉罕對他的僕人說:你們和奴留在這裡.
我和孩子要去那裡敬拜,然後回來你們這裡來.
阿伯拉罕把凡濟的柴放在他兒子爾薩身上.
自己手裡拿著火與刀,於是二人同行.
爾薩對他父親阿伯拉罕說:我扶你.
阿伯拉罕說:我兒,我在這裡.
爾薩說:看,火與柴都有了,但凡濟的羔羊在那裡呢?.
阿伯拉罕說:我兒,神必自己預備凡濟的羔羊.
於是二人同行,他們到了神指示他的地方.
阿伯拉罕在那裡築壇把柴擺好.
綁了他兒子爾薩,放在壇的柴上.
阿伯拉罕就伸手拿刀,要殺他的兒子.
耶和華的使者從天上呼喚他說.
阿伯拉罕,阿伯拉罕,他說:我在這裡.
天使說:不可在這孩子身上下手,一點也不可傷害他.
現在我知道你是敬畏神的人了.
因為你沒有把你的兒子,就是你的獨生子,留下不及我.
阿伯拉罕舉目觀看,看那一隻公面羊兩角纏在灌木叢中.
阿伯拉罕就去牽了那隻公面羊,獻為凡濟,代替他的兒子.
阿伯拉罕給那地方起名叫耶和華爾了.
直到今天,人還說,在耶和華的山上必有預備.
耶和華的使者第二次從天上呼喚阿伯拉罕說.
耶和華說:你既行了這事,沒有留下你的兒子,就是你的獨子.
我指著自己起誓,我必多多賜福給你.

$^{121}$我必使你的後裔大大增多,如同天上的星,海邊的沙.
你的後裔必得受敵的城門,並且地上的萬國都必因你的後裔得福.
因為你聽從了我的話.
於是阿伯拉罕回到他僕人那裡,他們一同起身往別士巴去.
阿伯拉罕就住在別士巴.
各位弟兄姊妹平安,很開心能夠來到北美,來到Edmonton,愛城,華仁,在這間福音堂和大家見面.
也透過直播和北美洲其他的弟兄姊妹見面.
其實這次我來北美也算是回家了,因為我之前曾經在北美不同的城市住過.
其實二十多年前我也是在Calgary住了幾年,所以感謝上帝.
前兩天我回到Calgary和朋友們去敘舊.
開始的時候,我首先要代表忠臣,院長李思敬博士和其他同工和老師向大家問安.
也感謝大家一直參與支持,紀念我們的侍奉.
明天是父親節,所以今天早上已經是父親節,香港父親節早上.
所以我們也思想家庭,今天分享的題目也是圍繞家庭,圍繞做父母的訊息.
其實我們剛才讀了一段經文,是講到阿巴拉罕.
新教聖經說阿巴拉罕是我們的信心之父.
所以每次我們思想阿巴拉罕的時候,我們想想他信心給了我們什麼榜樣.
不過我們要留意,原來阿巴拉罕也是肉身的父親.
所以我們也可以從阿巴拉罕的身上學習如何做父親.
不過問題是這樣,剛才記載了經文說.
這位父親曾經嘗試殺他的兒子,獻為凡濟.
嘗試殺一個兒子的父親,如何是一個好的父親呢?.
不過你會說,不是的,這是上帝吩咐他做的.
最後也沒有殺,我想阿巴拉罕也是一個好父親.
不過聖經是怎麼說的呢?.
所以我們今天就重溫這個故事.
從一個角度去看這段經文,告訴我們父子的關係是什麼.
或者你會跟我說,我還沒有做父親,我還沒有做父母.
不過可能在生命裡面也有一些世侄,或者有一些徒弟.
你很關心他,你想牧養他.
我們也希望今天透過這個故事,這段經文可以提醒大家.
我們進入上帝的話之前,我們一起進行一個祈禱.
天主上,我們感謝你,因為你不斷透過聖經向我們說話.
求你今天繼續再一次打開你的話,讓我們能夠聽得明白.
我們求聖靈在我們的心裡面親自教導,感動.
祈禱奉耶穌基督的名,阿們.
初世紀第22章開始的時候,就說神吩咐阿巴拉罕要獻以殺.
在第二節,上帝的說話很準確.
祂說你要將你所愛的獨子以殺獻上.
那時候阿巴拉罕有兩個兒子.

$^{161}$從夏甲生的以殺瑪尼,和從薩拉生的以殺.
神說要他獻獨子,即是繼承他產業的那個.
還要加上一句,你所愛的以殺.
所以阿巴拉罕逃不掉.
上帝吩咐他獻以殺,不是以殺瑪尼.
經文其實我們都熟悉.
阿巴拉罕聽了神的吩咐.
第二天就帶了以殺,盧和兩個僕人出發去摩利亞地.
到第三天就到了.
或者上山的路太斜,走不下去.
阿巴拉罕就將路留給僕人看.
將柴放在以殺身上.
自己和以殺兩個人上山.
到達目的地之後.
阿巴拉罕就起了一座壇.
將兒子綁上去,要他獻為梵祭.
不過當他要下手的時候.
就有天使在天上呼喚他,叫他不要傷害以殺.
阿巴拉罕停手,舉目一看.
見到有隻公綿羊,角纏在樹叢裡面.
他走過去拿隻綿羊來,代替了以殺.
獻在壇上.
我們讀這段經文的時候.
通常將焦點放在阿巴拉罕身上.
見到他過程沒有遲疑.
看到他是多麼遵從上帝的吩咐.
不過我們很少去探索這件事.
如何影響阿巴拉罕和以殺兩個人的關係.
當我們看這段經文的時候.
特別的地方就是.
其實經文的情節是很緊張的.
一個父親要殺兒子.
不過聖經記載好像很平鋪直敘.
沒有講他們的心情,沒有講他們的想法.
平淡得像父子兩個人去旅行一樣.
我們很難從經文裡面去讀到.
究竟父子之間發生了什麼事.
如何影響他們的關係.
所以我們要從聖經的一些線索去解答問題.
第一個線索在第六至第八節的經文裡面.

$^{201}$這裡是講到阿巴拉罕和他的兒子一起上山.
原來這三節經文是整本聖經裡面.
記載了他們兩個人最多聊天的片段.
以撒問父親.
火和柴都有了,但是獻祭的羊在哪裡.
父親很細心地回答兒子.
上帝會預備的.
短短幾節經文記載了父子兩人四次的對話.
以撒叫阿巴拉罕做我父.
阿巴拉罕叫做我兒.
加上我們留意第六和第八節結尾的時候.
重覆了兩次.
於是二人同行.
可見聖經形容他們兩個的關係其實很親密和諧.
第二個線索到了第九節.
當他們兩個人到了目的地.
阿巴拉罕就起了個壇.
聖經說他將兒子以撒綁在壇上.
用刀去殺他.
聖經說阿巴拉罕要綁住以撒.
代表以撒可能會掙扎.
又或者阿巴拉罕恐怕他會掙扎.
總之綁這個字都表達了不願意.
或者反抗或者約束的意思.
我想我們都明白的.
阿巴拉罕要將他最愛的兒子獻上.
以撒要給感對待.
我想不願意抗拒都是很正常的事.
不過父親這樣綁兒子.
怎樣影響他們兩個的關係.
聖經又沒有很直接地說.
或者我們遲些回頭解答這個問題.
特別的是到了聖經第十節之後.
再沒有提到以撒這個人.
之後就只記載阿巴拉罕.
怎樣和天使對話.
怎樣用綿羊來獻祭.
一直都沒有提到以撒.
以撒沒有出現.
直到還有最後一個線索.

$^{241}$就是最後一節.
到了第十九節找到這個線索.
這裡說什麼呢.
當整件事完結的時候.
於是阿巴拉罕回到他僕人那裡.
這裡說他從山上下來.
回到僕人和牢裡.
線索就在回到這個字.
中文看不出來.
不過原來原文這個回到的動詞是單數.
就是說阿巴拉罕一個人從山上回到山下.
再下去那句.
他們一同起身往別墅巴去.
那個他們是說到阿巴拉罕和他的僕人.
不過再下去最後一句.
阿巴拉罕就住在別墅巴.
那個住在原來都是單數.
所以聖經說只有阿巴拉罕一個人.
在山上回到僕人那裡.
最後也都一個人住在別墅巴.
因為這兩節的單數.
令我們猜測究竟發生了什麼事呢.
為什麼聖經說只有阿巴拉罕一個人.
以撒在哪裡.
還是我們會懷疑.
是不是聖經的作者惡意寫了單數呢.
不過我們記得第六和第八節說了兩次.
於是二人同行.
到下山的時候.
用兩個單數的動詞去形容阿巴拉罕一個人回來.
所以我們可以相信聖經作者是故意說阿巴拉罕一個人下山.
還有我們發覺以撒在山上下來的時候.
沒有再出現沒有再聊天.
聖經對他隻字不提.
甚至我們看到原來之後的章節.
之後的聖經沒有再記載阿巴拉罕和以撒再有任何的對話.
這件事令我們猜想究竟在山上父子兩人發生了什麼事呢.
因為這兩個單數的動詞.
有猶太人的識經傳統.
就認為獻以撒這個故事其實是根據一個更加古老的版本編輯.

$^{281}$那個古老的版本說阿巴拉罕真的獻上了以撒.
真的在山上殺了以撒獻給了上帝.
所以傳統認為第十一至十五節天使介入的章節是後來加上去.
所以第十九節保留舊的版本說阿巴拉罕一個人下山.
因為他在山上真的獻上了以撒.
這個識經傳統雖然能夠解釋第十九節那兩個單數動詞.
但最大問題是以撒之後還是出現的.
又結婚又生孩子.
怎麼會在山上被殺呢.
這個傳統有解釋.
它說因為上帝後來叫以撒復活才出現.
不過如果神真的要以撒復活.
這件事這麼特別這麼重要.
為什麼不再提.
不只是這裡沒有提.
整本聖經都沒有線索.
所以我們覺得這個解釋不是太合理.
不過這個識經傳統強調了一件事.
就是阿巴拉罕真的把他的兒子獻給了上帝.
我想他們這樣的解釋也不是沒有原因.
當然我們不能相信這個解釋.
不過這個解釋傳統令我們問一個問題.
究竟上帝吩咐阿巴拉罕獻撒的意思是什麼呢.
是不是神真的看到他下手就取消.
不用你了不用你獻了.
我們再留心看這幾節經文的時候.
我們會看到還有其他經文.
都好像說阿巴拉罕真的獻了以撒.
我們看看第十二節.
第十二節的下半.
神就對阿巴拉罕說.
因為你沒有把你的兒子.
就算你的獨子留下不給我.
同樣的經文第十六節出現了一次.
祂說你既行了者是沒有留下你的兒子.
你的獨子.
神讚阿巴拉罕兩次沒有留下兒子不給我.
只是看這兩處地方.
字面來看.
其實阿巴拉罕真的獻了兒子給上帝.

$^{321}$我們怎樣去解釋這些經文呢.
如果以撒真的被獻上.
為什麼他沒有死呢.
我想我們看看第十二節的上半.
給我們一些線索去回答這個問題.
當阿巴拉罕要下手以撒的時候.
聖經記載天使這樣說.
不可在這孩子身上下手.
一點也不可傷害他.
留意天使叫他不要傷害他.
天使沒有跟他說.
我知道你現在要獻.
所以你不用獻他了.
你下山吧.
天使是吩咐阿巴拉罕不要傷害以撒.
意思是上帝要他繼續在地上活著.
所以其實上帝是要活生生地接收了以撒這件祭物.
神要接納以撒為祭.
不過要他活下去.
目的是什麼呢.
我們看看第十八節.
究竟上帝要他活著做什麼.
去到第十八節.
上帝跟阿巴拉罕說應許.
這裡他說地上的萬國都必因你的後裔得福.
誰是他的後裔呢.
以撒就是阿巴拉罕的後裔.
是上帝揀選的後裔.
所以神要以撒的生命存留.
要他和他的後裔叫到地上的萬國得到福.
所以看來天使的介入.
沒有將獻祭的吩咐變成不用獻.
只是改變了獻祭的方法.
從獻為凡祭.
改變為獻上一個活生生的活祭.
是能夠祝福萬國的活祭.
原來獻以撒這個故事不是說.
神見阿巴拉罕你肯獻就不用他獻.
而是上帝閱納了阿巴拉罕的獻祭.
不過他保存了以撒的生命.

$^{361}$我們就明白原來上帝要阿巴拉罕獻以撒的意思.
在整件事的開始.
上帝吩咐阿巴拉罕要獻以撒為凡祭.
其實上帝吩咐出沒有收回他的吩咐.
所以阿巴拉罕是要獻.
他要獻以撒也要獻凡祭.
天使的介入沒有取消上帝的吩咐.
只是改變了獻祭的方法.
以撒不需要被獻上作為凡祭.
他的名字要留在這裡.
改為獻上成為活祭.
祝福萬國.
所以到第十三節我們看到.
阿巴拉罕看到一隻公綿羊.
就用這隻綿羊來代替他的兒子獻為凡祭.
因為上帝也要他獻凡祭.
不過可以用公綿羊來代替.
所以在山上阿巴拉罕獻了以撒.
也獻了綿羊.
獻了他的兒子作為活祭.
獻了綿羊作為凡祭.
正如神所說.
阿巴拉罕沒有將他的兒子留下不給我.
所以當我們回到第十九節.
要解釋那兩個單數的動詞的時候.
不是說以撒死了.
只是象徵阿巴拉罕已經將他獻上給上帝.
對阿巴拉罕來說.
這個兒子已經失去了.
不再屬於他了.
屬於上帝.
所以這兩個單數的動詞就象徵阿巴拉罕只剩下他一個人.
如果我們要回去.
剛才我說阿巴拉罕在壇上綁著以撒.
那個意思.
第十節.
阿巴拉罕將以撒綁在獻祭的壇上.
這個動作是甚麼呢.
是將以撒一出生以來.
一直在自己身邊.

$^{401}$他很愛的兒子.
親手將他綁在神的壇上.
綁在上帝那裡.
從那一刻開始.
阿巴拉罕知道以撒不再屬於他了.
綁這個字代表他不願意.
又代表了約束.
象徵阿巴拉罕將以撒和他自己之間.
由出生一直的關係.
所謂那條磁帶.
從自己身上解開.
綁到上帝那裡.
阿巴拉罕願意不願意也好.
以撒已經不再屬於他了.
有一位猶太學者.
叫Leon Kass.
他認為上帝揀選阿巴拉罕的家族.
跟他們同行.
目的是要他們學習怎樣做神的子民.
而上帝透過不同的經歷去訓練他們.
獻以撒這個歷程.
對這位學者來說.
是上帝教導阿巴拉罕怎樣做父親.
這個經歷讓阿巴拉罕知道.
原來作為神揀選的人.
做父親的目標.
是要將自己的子女獻上給神.
成為活祭祝福萬國.
這個可謂叫做父母之道.
屬神的人需要是這樣.
這個父母之道也需要一代一代傳下去.
好讓上帝的道能夠在地上延續.
所以阿巴拉罕的後裔.
我們稱為阿巴拉罕後裔.
屬神的子民.
當我們有下一代的時候.
也需要學習將自己的子女.
綁在上帝的祭壇上.
這個是我們父母之道.
其實今天的世界裡.

$^{441}$也跟我們說做父母有很多要求.
應該怎樣去做.
很多時候世界跟我們說.
我們要確保子女要好好讀書.
找份好工作結婚生子.
簡單來說我們也覺得.
我們需要叫子女安居樂業.
不過特別的地方是.
原來上帝賜我們兒女.
他沒有叫我們要確保他們安居樂業.
反而期望我們將他們獻上給上帝.
我們想現在這個世界裡.
要子女安居樂業越來越不容易.
我不知道北美.
不過香港近年樓價升到二三萬元一呎.
父母很想安居樂業.
窮一生的力量也未必做得到.
近年如果大家有留意香港.
小朋友入學也是很難.
上幾個星期.
小學派位.
發現原來有家長為了.
兒子要入好學校.
用一萬元租了地址.
幸好他能夠用地址來報學校.
有時我聽到這些消息也很佩服.
這些父母真的很愛他們的子女.
為他們付出很多.
不過聖經沒有直接跟我們說.
我們要這樣去培育子女.
反而聖經跟我們說.
做父母容易很多.
就是把我們的子女獻上給上帝.
成為祝福萬國的活祭.
所以其實換句話說.
做父母的目的是脫身.
讓子女不再屬於自己.
屬於上帝.
不過同時我們又知道.
做基督徒父母一點也不容易.

$^{481}$因為我們獻上的兒女.
是要合神心意.
願意祝福萬國的.
這些錢買不到的.
而我覺得最困難的就是.
我們每一個都很愛惜我們的子女.
但上帝跟我們說.
其實到時候.
我們都要把他們綁在上帝的祭壇上.
把他們獻給上帝.
我想對父母來說.
這也是一個不容易做的任務.
我自己在七八十年代在香港長大.
那時候香港經濟起飛.
我自己也讀一所傳統好的學校.
我的同學畢業生都很喜歡選擇一些.
穩定受人尊重的專業商界工作.
所以對我的同學來說.
到今天其實很多都很成功.
上了位.
生活經濟可以說是無憂.
不過有時候我認識他們.
又不要覺得他們活得很開心.
其實我感覺到很多人對生活都很多不滿.
覺得自己形形形大半生.
錢都是不憂的.
不過覺得自己的工作沒有什麼意思.
生命看不到什麼目標.
上陣子我有個同學說.
要突破這個悶局.
要去做自己想做的事.
身邊很多同學鼓勵他.
快點快點.
每個人都聽到都很開心.
不過過了一段日子.
都發覺好像沒有什麼事發生.
我想都不難去理解.
穩定的生活之後.
其實要有勇氣作出大的改變不容易.
不過我想最大的困難就是.

$^{521}$很多人就算想突破這個所謂生命的悶局.
其實他都不知道應該做什麼.
不過最令我驚訝的就是.
我的同學雖然很不滿自己走的那條路.
但仍然很踴躍地安排自己的子女.
進回同一間學校.
走回同一條路.
有時我會想為什麼這麼特別.
他們的人生好像在世上很成功.
但他們活到這個時刻覺得自己其實很不滿足.
為什麼都要自己的子女走回同一條路.
另一方面我都明白.
對於不認識上帝的人其實沒有什麼可以選.
這個世界說是這樣就是這樣走.
不過原來基督徒很不同.
我們不需要跟著這個世界走.
而是上帝吩咐我們.
要將自己獻給上帝.
將我們的子女獻給上帝.
祝福萬國.
祝福萬國不是說一定要做牧師,宣教士.
或者在教會裡面有些很重要的侍奉.
而是上帝說我們不應該跟從世俗.
我們要開放自己.
在上帝那裡尋求人生的方向.
並且要祝福我們的鄰舍.
祝福萬國為目標.
上帝要求我們將子女獻上.
意思是我們子女的人生方向.
他們的呼召,他們的vocation是上帝決定.
不是我們決定的.
而做父母的甚至要鼓勵子女尋找上帝的心意.
活出祝福別人的生命.
所以我想如果是這樣的時間.
有一天你的兒子跟你說.
爸爸,我不做醫生,不做工程師了.
上帝感動我要透過藝術,作曲,煮飯去服侍人.
或者你的女兒跟你說.
爸爸,我不做會計,我不做pharmacist了.
上帝感動我要服侍社會上邊緣的人.

$^{561}$我也沒有很恭喜他們.
我更加恭喜你.
因為你已經做到上帝對你成為父母的一個期望.
你的子女被上帝接納成為祝福萬國的活兒.
所以弟兄姊妹.
當我們有自己的下一代的時候.
我們心底裡究竟期望他是一個怎樣的人.
我們會不會都好像阿伯拉罕的榜樣.
願意將他獻上給上帝去祝福萬國呢?.
之前提到我們讀獻耳撒這個記載.
焦點通常放在阿伯拉罕身上.
很少去想耳撒發生什麼事.
不過原來耳撒也是主角.
其實他們父子兩個人在山上經歷很不同.
一個是將兒子獻上.
有一個自己的生命要獻上給上帝.
其實我們想想山上的經歷.
都代表了阿伯拉罕和耳撒兩代的縮影.
阿伯拉罕是主動去接收或開啟信仰的人.
他75歲被上帝呼召.
要他離開本地本族父家.
他的信仰是在他家族以外上帝呼召他來.
他可以叫得做第一代的信徒.
自己摸索和認識上帝.
倚靠上帝.
到99歲才行割禮.
而上帝吩咐他獻耳撒.
也是跟他講的.
叫他主動地將兒子獻上.
但對耳撒來說很不同.
耳撒是一個承傳信仰的人.
他出生就已經在一個信仰的家庭裡面.
第八天BB仔的時候.
阿伯拉罕已經替他行了割禮.
他家境富裕.
他的信仰,家境,生活方式,文化都是繼承而來.
在山上他的經歷都是很被動.
被父親去獻上.
所以耳撒可以說是一個信義代.
阿伯拉罕和耳撒的經歷很不同.

$^{601}$我們留意到原來他們的觀點和需要都很不同.
或者我們今天都在經歷一個這樣的年代.
世界變得很快.
每一代人所遇到的,所想的東西都很不同.
其實聖經裡面當我們看到阿伯拉罕和耳撒的時候.
我們留意到他們兩個人的經歷和看法都很不同.
以前我記得小時候.
很多年前我們去北美洲讀書的時候.
要和家人溝通時怎樣呢?.
打電話,寫信.
今天電話不是用來講的.
是用來按的.
以前我在外國不懂煮飯.
打電話回去問媽媽.
今天上Youtube看.
一個世界很不同.
而我接觸新一代有一個很大的特色.
原來新一代的年青人.
已經不是很追求要一個很穩定做一輩子的工作.
我在自己的教會裡面.
有些年青人會工作幾個月.
然後辭職.
然後去旅行幾個月.
回來再找工作.
又或者去一兩年的工作假期.
對於這種工作的態度.
其實對上一代的人很難接受.
覺得他這麼不穩定,不承擔.
因為上一代的人覺得穩定是很重要的.
所以因為這樣的緣故.
可能很多上一代的人會覺得.
一代不如一代.
現在的人真的沒有我們以前那麼努力了.
不過有人說新一代是後物質的年代.
年青人不再追求經濟工作的穩定.
而是尋求物質以外的經驗.
所以其實我想不是一代不如一代.
可能更加對的是一代不同一代.
我們想想在這個例子裡面.
究竟哪一代人所想的東西.

$^{641}$是更加合乎基督教信仰的價值呢?.
想想其實新一代那些非物質的追求.
其實更加像聖經所說.
人活著不單靠食物的道理.
當然年青一代可能感覺上是比較自我中心.
經常喜歡做自己做的事.
所以他們都需要聖經下一句的教導.
不單靠食物.
但是要靠神口裡出一句話.
所以我們見到兩代之間其實各有特色.
各有需要.
我們相處的時候應該明白.
大家的不同.
父母其實不需要.
沒有可能叫子女跟自己的追求和想法一樣.
回到憲義法這個故事.
這個提醒我們同一件事.
兩代人的經驗看法可以很不同.
其實很簡單很自然的事.
不過對基督徒來說.
是每一代都要懂得獻上自己給上帝去祝福萬國.
不過每一代獻的方法可以很不同.
不過無論這兩代的人經驗有多不同.
當小朋友還小的時候.
家庭裡面的方向.
家庭裡面的氣氛和文化.
始終都是父母決定.
我們見到在憲義法這個記載裡面.
上帝只是跟阿伯拉罕吩咐.
阿伯拉罕是主動的人.
因為阿伯拉罕有帶領家庭的責任.
不過我們留意.
上帝怎樣叫阿伯拉罕帶領家庭.
不是要他指揮其他人做事.
只要他有口.
阿伯拉罕帶領家庭.
是自己先順從上帝.
用他的榜樣去帶領.
其實今日在社會裡培育下一代一點都不容易.
基督徒面對很多不同價值觀的挑戰.

$^{681}$做父母都很渴望知道.
用甚麼方法去教育子女.
可能參加很多不同的班.
不同的工作坊.
當然這些都是重要.
不過聖經提醒我們.
最重要的其實不是方法.
是父母的榜樣.
好像阿伯拉罕一樣.
如果我們盼望子女要順服上帝.
問題是究竟我有多順服上帝.
如果我們知道原來子女要獻上給神.
祝福萬國.
我們要問究竟我自己的生命.
是否祝福別人的生命.
在山上阿伯拉罕學到要獻上的兒子.
我們問另一個問題.
究竟以撒學到甚麼呢.
我想以撒從羊身上學到最大的東西.
我們知道以撒上山時問阿伯拉罕.
獻上羊在哪裡.
當時他不知道.
原來他自己就是.
當然阿伯拉罕不會對他說你就是.
不過當他在山上被父親綁在壇上時.
他就知道.
到後來阿伯拉罕找隻羊來代替以撒.
以撒就鬆綁下來.
我們可以想像一下他那時經歷了甚麼.
他站在祭壇邊.
看著祭壇.
可能心裡很慶幸自己脫身不用死.
不過原本自己綁著的地方.
那隻羊在那裡被殺被燒.
以撒就明白.
其實那個地方應該自己躺在那裡.
這個經歷告訴他其實他已經死了.
所以以撒明白.
現在活著不再是我.
我的生命不再屬於自己.

$^{721}$是屬於上帝.
我們就明白為甚麼上帝要阿伯拉罕不用獻.
也要獻羊呢.
因為原來這是對於以撒一個很重要的學習機會.
在這個祭壇上.
以撒學了幾樣東西.
第一.
原來這個信仰很危險.
阿伯拉罕的危險在於.
他要離開本地本族父家.
以撒的危險更大.
信仰危害生命.
要負上自己的生命.
很大的學習.
第二.
以撒明白到原來我的父親也要信奉上帝.
他很不赦賞也好.
他也要信奉神.
所以原來父親不是家中的主.
上帝才是.
第三他學到甚麼呢.
原來他知道從此他的生命是屬於神.
不再屬於父母.
中國人說身體發枯受諸父母.
但對基督徒來說.
原來我們的身體發枯是屬於上帝.
所以我們的生命要向上帝負責.
第四他學到的是.
原來父親要將自己綁在上帝那裡.
所以原來到時候.
他要自己將自己和上帝建立關係.
尋求自己信仰的路.
阿伯拉罕和以撒這兩代人有很不同的學習.
阿伯拉罕因為以撒的學習.
也要明白幾點.
如果信仰會危害兒子的生命.
其實做父母也要預備.
兒女有機會為了信仰的緣故.
落入生命的危險.
有時候兒女去很危險的地方短孫.

$^{761}$很緊張.
不過信仰帶領他們去的地方.
可以很危險.
如果上帝是家中的主.
做父母就不是.
所以在兒女面前.
其實自己也要在上帝面前謙卑.
我們要記住.
在神的審判之下.
其實我們做父母和兒女是平等的.
如果兒女的生命是屬於上帝.
不是屬於自己.
就不能要求兒女去回報.
或者滿足自己的期望.
他們要滿足上帝的期望.
如果兒女要和神建立關係.
就要在適當的時候.
讓他們和上帝建立關係.
尋找自己的路.
不過我們發覺.
阿伯拉和以撒這兩代人.
學習不同.
不過他們原來有共同的學習.
就是學習捨棄的功課.
在生命上.
父親學習捨棄他最愛的兒子.
兒子學習的是捨棄自己的生命.
謙以撒對父子兩人來說.
都是很大的考驗.
有時候代入他們的話.
會覺得上帝給這個考驗這個家庭.
很難 很殘忍.
所以有很多神學家討論這個事情.
就想了.
上帝要阿伯拉和謙以撒是否合理.
這個殺人論理上是否沒有問題.
不過如果我們將謙以撒這個故事.
成為論理的討論.
可能會觸錯了用神.
這個記載不是討論殺人是對還是錯.

$^{801}$而是讓我們明白上帝是怎樣的上帝.
讓我們明白上帝是一位捨棄的上帝.
我們新約的信徒就很明白.
以撒是預表了基督在十字架上獻上生命.
成就救贖 祝福萬民.
在神的計劃裡面.
天父為了救贖人.
將他的兒子去捨棄.
獻在十字架上.
原來阿伯拉和謙就啟示.
我們的天父是怎樣的父親.
而以撒啟示了我們的救主.
是怎樣的一位捨棄生命的救主.
捨棄所愛的 捨棄自己的生命.
是上帝愛的表達 是上帝的本性.
如果是這樣的話.
上帝吩咐阿伯拉和謙獻上以撒.
又要求以撒獻上自己的生命.
原來不是強人所難.
反而是一個很榮耀的邀請.
邀請他們參與在上帝捨棄的生命裡面.
所以我們明白.
神為何不叫阿伯拉和謙獻上以撒馬利.
因為那不是應許的.
而是以撒才是.
他才有機會去面對這個考驗.
接受這個邀請.
所以原來我們都是阿伯拉和謙的子孫.
我們都是被上帝揀選.
參與在上帝愛和捨棄的工作和行動裡面.
我們獻上自己就是效法基督.
我們獻上子女就是效法天父.
所以原以我們都甘心樂意.
將自己的子女獻上給天父.
原來這是天父邀請我們.
進入上帝捨棄的生命裡面的一個邀請.
看來上帝吩咐阿伯拉和謙獻以撒.
不單是訓練他的順從和信心.
而是邀請他們進入上帝的工作裡面.
亦都叫他們更加認識上帝.

$^{841}$經過這個經歷裡面.
我們發覺原來阿伯拉和以撒.
都對上帝有一個更加深的認識.
大家或許聽過有人會這樣解釋.
神為何要去吩咐阿伯拉和謙獻上以撒.
是因為阿伯拉和謙到某程度.
太愛他的兒子.
愛的程度比神更加多.
所以上帝就要以撒.
要他獻上以撒給上帝.
當然愛子女不能夠多於愛上帝.
這是很正確的.
不過這樣去解釋經文有一個問題.
就是假設上帝是一個很妒忌的神.
與以撒爭寵.
他發覺阿伯拉愛兒子多於愛自己.
就要他殺了兒子.
這個解釋其實真的有些問題.
聖經所說的上帝是否這樣呢.
就算我們真的愛兒子多於愛上帝.
神就要我們殺了兒子嗎.
難道.
我們想如果神為了拯救我們.
連他自己的兒子都能夠捨棄.
又怎會妒忌到要我們殺了自己的兒子呢.
怎樣去解決這個問題.
我想第十四節裡給我們一個提示.
當阿伯拉和謙見到神為他預備了一隻羊做祭物.
然後他就把地方起名為耶和華爾納.
透過謙與撒.
阿伯拉明白神是一個替人預備的上帝.
其實阿伯拉罕真的很愛惜與撒.
所以出生到開始已經為與撒預備最好了.
我們認識阿伯拉罕其實他是一個很有錢很有勢力.
打仗又打贏的人.
在那個年代裡.
與撒出生在這樣的家庭裡很幸福.
幸福過今天的富二代.
與撒出生後阿伯拉罕會以為.
我這麼有能力這麼有資源.

$^{881}$要保護他其實是很沒問題.
會以為與撒這個這麼愛的孩子.
會一世人無災無病地留在自己身邊.
不過當上帝吩咐他要獻上與撒.
他突然發現一件事.
原來與撒的生命沒有他想像中那麼穩固.
原來與撒的生命不是真的掌握在自己的手裡.
生命不是我們想像中那麼安全.
不過特別的地方就是.
當阿伯拉罕願意將他獻給上帝.
神就在山上為他預備那隻羊去代替與撒.
這件事令阿伯拉罕發現.
原來上帝會親自為與撒去預備.
他不用阿伯拉罕去為他預備.
而且上帝的預備比阿伯拉罕的預備更加好.
因為神能夠控制阿伯拉罕控制不到的事.
阿伯拉罕就發現原來自己將兒子獻給上帝.
放在神的手裡.
比留在自己身邊更加穩妥.
對兒子來說是更加好的預備.
原來將兒子獻給上帝是更加愛他的方法.
所以看來耶和華議立.
不單止是說耶和華會預備.
更加是說當我們獻上給上帝.
在耶和華的手裡是最安穩.
當我們的下一代在神的手裡是最安全.
我們發現原來上帝從頭到尾都沒有與以撒去爭寵.
沒有叫阿伯拉罕殺他的兒子.
只是愛上帝.
反而透過經歷去教阿伯拉罕.
什麼叫做真正愛他的兒子.
神對他說.
你將他放在我的手裡.
我為他預備.
我是以勒的上帝.
你這樣做.
是你能夠給他最大的愛.
這就是屬神的人的愛.
愛不是自私地為自己.
不是將所愛的人留在身邊.

$^{921}$而是懂得將他交在上帝的手裡祝福萬國.
愛不是只愛自己人.
愛家人 愛朋友.
而是將這個愛向外.
為我們的鄰舍.
甚至是我們的仇敵.
我們越愛一個人.
就越要將他獻上給上帝.
獻以撒的經歷讓阿伯拉罕明白到.
將所愛的人留在自己身邊為他預備.
是不能確保他幸福豐盛.
反而將他獻上.
讓神為他預備.
帶領他一生.
這才是對他最好.
最愛他的方法.
看來原諒我們愛子女和愛神是不應該有衝突的.
愛神的緣故將家人獻上.
反而是最愛家人的表現.
其實做基督徒我們很明白.
我們都要獻上自己成為活祭奉上帝.
每次去培靈會.
培靈會的講義都跟你講一次.
每次都說你要獻上自己成為活祭.
我們聽前年忠臣北美培靈會.
李院長講了合神心意的侍奉.
他說侍奉不是上帝奴役我們.
反而是我們能夠侍奉那位那麼愛我們.
又賜恩給我們的神.
是最有福的事.
我們都要問.
其實我們是否真的這麼相信.
是否真的覺得自己獻上給上帝是最有福.
最好的生命.
相信也好.
我們是否實踐著這樣的生命.
不過問題是.
我們生子女出來.
我們經常覺得要給最好的他.
我們是否同樣覺得.

$^{961}$原來他們的生命獻上給上帝.
就是最好呢.
是不是這樣對他們是最好的呢.
我想對我們來說.
要獻上自己給上帝都不容易.
不過有時候要獻上自己的家人給上帝.
是更加難的.
或者對一些自己都覺得.
自己毀身侍奉的人.
我覺得其實我無所謂.
我爛命一條.
上帝要我.
不過你不要搞我的家人.
有時候可能基督徒聽到上帝的呼召.
要你放下工作.
去服侍我要你服侍的人.
想到這樣做.
我自己無所謂.
不過打擾家人平穩的生活.
影響經濟.
不是太好.
可能我們會跟神說.
不如等孩子長大一點.
等他進了大學.
離開.
我有時間了.
到時才回應上帝好不好.
又或者跟神說.
神啊.
你要我好.
你放過我的子女.
不過如果你這樣想.
其實就將到神的呼召.
和我們子女的幸福對立.
因為愛神回應神.
就等於不愛子女.
耶和華以勒提醒我們.
原來將自己一家去獻上.
將整個家庭放在神的手裡.
原來是最愛家人的決定.

$^{1001}$看來今天的題目.
我和我家必定侍奉耶和華.
其實原來不單止是一個立志.
其實是一個對自己對家人最好的安排.
在初期教會.
信徒被很多的逼迫.
基督徒經常被人抓.
嚴峻的時間要信徒被人扔進鬥獸場.
行刑處決.
那時基督徒父母.
他們面對這些逼迫的時候.
他們沒有將子女收起來.
他們相信信徒要一家一齊.
這是最榮耀的見證.
我看過一個歷史記載.
說到有一次有一班基督徒被抓.
面對逼迫要殉道.
有一個母親用了獻耳撒的事來鼓勵她的兒子.
她說你要學耳撒在祭壇上.
勇敢地去面對祭壇的刀.
我們要效法他.
那時基督徒父母很明白.
子女和自己殉道是最好最榮耀的事.
將自己放在神的手中是最穩妥的.
他們明白耶和華爾勒的道理.
耶和華爾勒不是說上帝會解決我們一切的困難.
而是說原來我們願意放在上帝的手裡.
無論發生什麼事.
是最穩妥,最安全的.
我們今天在北美的基督徒.
未必要面對同樣的逼迫.
不過我們要知道.
其實我們和初期教會的信徒.
都是阿伯拉罕的子孫.
我們都要明白獻上給上帝.
將自己放在上帝的手中.
是一個最穩妥的生命.
在結束的時候我想分享一個.
香港基督徒家庭的故事.
有一對夫婦.

$^{1041}$他們有一個兒子.
也有一個兒子.
兒子出生的時候發覺.
患上了唐氏綜合症.
不單止這樣.
還有很多其他的問題.
兒子需要接受很多不同的治療.
其中很重要的是語言治療.
教他怎樣幫助他怎樣說話.
這對夫婦的家庭經濟狀況其實很好.
這對夫婦考慮到兒子長大了.
要人照顧.
所以經濟好的時候.
他們就買了一層樓.
預備了很多錢.
想著兒子將來.
可以靠這些錢好好生活.
很不幸.
兒子還小的時候.
有一次意外.
這個孩子離開了他們.
他們很傷心.
不久後有人邀請他們.
去代理一隻高級的朱古力.
他們沒有什麼經驗.
不過吃過朱古力很喜歡.
就做了這件事.
想著反正預付兒子的錢.
那間樓都沒用了.
買了一間樓.
開了朱古力店.
特別的地方當然不是開朱古力店.
原來他們決定.
用朱古力店賺到的錢.
去支持一個做語言治療的中心.
去幫助一些基層有同樣需要的人.
因為他們的孩子.
越小接受的治療越好.
政府資源未能做到.
他們就把握這個機會.

$^{1081}$他們就這樣支持很多孩子.
接受語言治療.
大家猜猜那個語言治療中心的名字是什麼.
用兒子的名字命名.
這對夫婦是很好的基督徒.
他們開朱古力店的時候.
也很好的見證.
他們請年輕人來培訓他們.
因為他們的工作.
叫到很多同事都信了主.
而那個語言治療中心.
同時間服侍了幾百個小孩子.
大家知道這對夫婦遭遇其他好不幸.
他們的兒子離開世界.
絕對不是一件好事.
其實他們可以選擇埋怨上帝封閉自己.
但他們沒有這樣做.
反而他們知道兒子原來在上帝的手裡.
而他們是願意.
用原來為兒子預備的東西.
去獻上給上帝.
就令到兒子仿佛仍然活著.
去祝福其他人.
其實獻上我們的兒女.
祝福萬國.
其實可以有很多不同的方法.
可以有很多不同的故事.
重要的地方.
是否我們聽到上帝的吩咐.
願意去到獻上.
人家知道生命其實可長可短.
生命也不在乎長短.
因為生命的長短在上帝的手裡.
我們能夠選擇的.
就是我們怎樣去到過我們有限的日子.
我們是決定將我們的生命.
捉在自己的手裡.
跟從世界 讚得世界.
還是願意將它放在上帝的手裡.
去祝福萬國呢.

$^{1121}$聖經提醒我們.
唯有將我們的生命獻上給上帝.
當作活祭.
生命才會得到價值.
這真真正正是阿伯拉罕的子孫.
我們有一個祈禱.
第一節目我想邀請大家.
你可以思想.
究竟你自己對於家人.
對於兒女有甚麼期望.
今天我們再看阿伯拉罕和以撒的故事.
對你有甚麼提醒.
或者你想和上帝有甚麼祈禱.
我給大家一點時間為自己祈禱.
然後我會帶領大家去結束禱告.
天父 我很感謝你.
因為你願意獻上你最愛的獨生子.
為了救贖我們的緣故.
你捨棄你的兒子.
主耶穌我們也很感謝你.
因為你願意為了我們的緣故獻上你的生命.
主耶穌我們看到你願意捨棄.
我們就知道沒有任何東西.
能夠將你對我們的愛去到隔絕.
主耶穌我們知道你邀請我們.
加入你這個捨棄的生命裡面.
天父 原來你也懂得去效法.
效法阿伯拉罕 效法你自己.
天父 我們聽到你說.
原來我和我家必定侍奉上帝.
我們將自己擺在你的手裡.
原來是一個最穩妥的決定.
天父 你也知道我們心裡面有很多困難.
我們可能不夠信心.
我們心裡面有很多仍然需要降服的地方.
主就求你降服我們.
願意叫我們捉緊你的吩咐.
將我們自己的生命.
我們兒女的生命.
我們家人的生命.

$^{1161}$都獻在你的前面.
祝福萬民.
聽我們祈禱.
祈禱奉耶穌基督的名.
阿門.
(字幕製作:貝爾).
(字幕由 Amara.org 社群提供).
\newpage



\section{}
\label{sec:XGZrzl_HY54}
\textbf{中神北美西岸培靈講座 2019}
\newline
\newline
連結: \href{https://youtube.com/watch?v=XGZrzl-HY54}{\texttt{ https://youtube.com/watch?v=XGZrzl-HY54}} ~~~~ 語音日期: 2019-06-28 
\newline
\newline
\hyperref[sec:ml6Ww0ZQVk8]{\small{< < < PREV SERMON < < <}}
~
\hyperref[sec:index]{\small{[返主目錄]}}
~
\hyperref[sec:jGn4gOBS1HA]{\small{> > > NEXT SERMON > > >}}
\newline
\newline
$^{1}$(字幕提供:MG).
各位大眼睫們晚安.
歡迎大家參加.
其實也不是很慢.
今天在西雅圖.
陽光普照.
風光明媚.
華氏五十多度.
早上比較涼.
華氏七十多度.
早上五十多度.
非常好的.
Yard Work的時間.
也是行山的時間.
但也是很好的時間.
參加.
忠臣北美的培靈講座.
我們非常歡迎.
張哲聰博士.
還有羅先生.
來到我們當中.
我們很高興.
大家一起仔細一課.
能夠一起來敬拜.
聽培靈的訊息.
我們一起祈禱.
親愛的主我們很高興讚美你.
今天這個培靈講座.
這個聖經講座.
的重點是讚美的抗爭.
到底在這個忙碌的社會.
我們能不能停下來思想你呢.
常常停下來紀念你呢.
常常停下來.
將我們的心和你能夠交接呢.
主我們何等渴望.
我們讚美你的時間.
不只是在崇拜的裡面.
甚至不只是在靈修生活裡面.
是在我們的侍奉裡面.

$^{41}$我們的工作裡面.
我們的家庭裡面.
是我們信敬的時候.
甚至是我們逆境的時候.
我們快樂的時候.
我們受壓力的時候.
或許我們受到痛苦的時候.
多謝你.
今晚一個這麼好的題目.
也都你帶領張博士和羅先生來.
關於忠臣的事工.
更加是一個培靈講座.
我們感謝你.
願你開啟我們的心.
對你的話語.
不要麻木.
對你的話語是學武.
我們將一會兒的敬拜.
一會兒的講道.
一會兒的介紹.
完全交在主的手中.
願你得到最高的榮耀.
我們交代禱告.
奉耶穌基督的聖名祈禱.
Amen.
以下我有敬拜的時間.
請.
在詩篇150篇.
你們要讚美耶和華.
要在神的聖所讚美他.
要在他顯能力的紅瘡下讚美他.
要因他大能的作為讚美他.
要因他無限的偉大讚美他.
凡有氣息的都要讚美耶和華.
你們要讚美耶和華.
靈姐妹讓我邀請大家一起起立.
我們用讚美的詩歌.
去開始我們今晚讚美的聚會.
(音樂).
是你一手創造掌管一切.

$^{81}$萬天聖獸是你所為.
不可猜測估計.
神動有智慧要雙眼萬世.
眾讚歸於 全能榮耀上帝.
是你一手創造掌管一切.
眾讚歸於 全能榮耀上帝.
還有主天 書上的恩惠.
聽於深海同馬王.
生命兩旁呼應.
神太愛恭維.
地與天共共振.
我的心 讚頌我的主.
斷腸出讚美感恩的華語.
在你恩典中 幾多的抱怨.
清早到晚上也未停止.
藏在你著墨的隱密處.
渴想一生盡力愛著.
若是一生多寒苦.
被世間天氣嚇唬.
但你縱看護 樂意參呼.
受滿海嘆念 作歡呼.
人海中 多份我維護.
但我的心始終堅固.
投靠你 便當人面豐富.
在此時 請師球妹繼續彈奏音樂.
希望大家有幾分鐘安靜的時間.
或許你可以閉上眼睛.
剛才我們唱的詩歌.
提及到神有很多大能.
祂有不同的屬性.
祂在我們身上的作為.
或許在這段安靜的當中.
你可以回想起.
你對神有什麼回應.
你可以怎樣去讚美祂.
又或許你在生命當中.
現在在低谷的時候.
你說心裡疲乏 無力.
不知道怎樣去讚美神.
但我希望在這個安靜的時間.

$^{121}$你嘗試去感受神的同在.
祂曾經應許過.
對我們不離不棄.
我們所經歷的.
神一樣都會感受到.
你會相信嗎?.
(音樂).
我們的主,我們的神.
就願一切的榮耀,尊貴,權柄.
都是貴於你.
我們讚美你.
因為你已經將最好的福氣賜給我們.
因著你的緣故.
我們有不一樣的生命.
祈禱交託奉耶穌名頭.
我們一起唱這首詩歌最後一次.
(音樂).
願我的心,讚頌我的主.
願唱出讚美感恩的話語.
在你恩典中,幾多的告知.
清早到晚上也未停止.
藏在你張望的隱密處.
渴想一生盡力愛著.
也許一生多困惑.
被世間天氣限制.
但你總刻骨放棄參苦.
守門台他便作歡呼.
人海中都奮鬥迷糊.
但我的心始終堅固.
投靠你便覺人顛峰富.
也許心中感痛你.
甚至失去了力氣.
讓我找得你,你的生命.
可以將我再生次高飛.
人海中誰點會有你.
但你仍在不捨不棄.
投靠你是最好的福氣.
全靠有著你,才令這生極美.
大家請坐.
願我們的生命都是一個充滿讚美的生命.

$^{161}$歡迎張智忠博士和羅先生來到我們當中.
羅先生是拓展主任.
張智忠博士是副教務長和聖經科的副教授.
遠道從人來,忠臣就不是我們陌生的了.
以往我們都有不同的忠臣院長,老師,講師來.
秋天都有梁國權先生來.
今天張智忠博士講到的題目是讚美的抗爭.
我們用拍掌歡迎.
(掌聲).
弟妹晚安.
多謝北美西岸不同的教會.
合力籌辦這次的培靈講座.
我其實來做的只是站在這裡說一會.
但他們事前做了很多準備.
心裡感到很感恩.
這次是我第一次來到西雅圖這個地方.
以前聽過一套戲叫做Sleepless in Seattle.
這幾晚經歷了什麼叫Sleepless in Seattle.
因為每一晚睡覺都是香港白天的時間.
不過很感恩今天暫時在這個地方.
暫時沒有很sleepy的感覺.
所以盼望今天晚上我們可以藉著上帝的說話.
一同彼此來分享.
不知道大家當你想到讚美的時候.
你會想起什麼.
讚美會想起很好聽的音樂.
剛才敬拜隊帶領我們所重唱的詩歌.
還是你會想起每個星期日回來.
我大概都知道那套套路你會做什麼.
先有些很強勁的節拍.
然後有些比較低沉的.
然後再帶起大家一個升級的.
一個循例的讚美.
還是你會覺得.
讚美其實我也不知道為什麼.
不過正如剛才所說.
這是我當盡之誼.
我姑且每個星期日回來.
我就要讚美他.
又或是你覺得讚美就是每一次.

$^{201}$你預備崇拜之前的熱身.
早上明天宣道會.
你們是八點半堂還沒睡醒.
所以就要熱身.
讓你可以來預備待會的訊息.
或是甚至讚美有這樣的用處.
就是等人齊.
因為大家還沒到.
所以就透過詩歌唱的時間.
剛剛好,然後大家就齊了.
不過其實如果你想清楚一點.
聖經裡面怎樣看讚美.
讚美,我想從另一個角度去看.
聖經是在說.
讚美是當你覺得那個人.
他為你做了一件很重要的好處.
或者幫助的時候.
你就去讚美他.
在這個世界上.
很多事情都在爭我們去讚美他.
因為很多事情都在爭我們.
覺得他為我們做的幫助.
他為我們的好處.
是對我們非常重要.
那些是什麼呢?.
我不知道你心裡數不數得出.
有不同的事情.
希望你來去讚美他.
那個可能是你身邊的配偶.
那個可能是你的上司.
那個可能是你的政府.
每一樣東西可能都在爭的.
希望你來去讚美他.
甚至乎那樣東西是一種產品.
某一個商品.
希望你去讚他.
其實很多事情都在爭我們.
去覺得他為我們做了一些很重要的好處.
帶來很重要很重要的幫助.
聖經怎樣幫我們可以更加明白.

$^{241}$更加正確.
或者更加清楚的來去讚美.
更加讓我們明白.
原來我們去讚美的時候.
其實是一場抗爭.
我們在抗爭.
原來我們唯一.
唯一一個真真正正應該覺得重要.
那個要讚美的對象.
只有我們耶和華的上帝.
我想今天藉著一首詩篇.
詩篇135篇.
和大家一起分享.
詩篇135篇不是大家很熟悉的詩篇.
所以我猜大家未必背得出.
如果可以的話.
我邀請你翻開你們的聖經.
詩篇135篇.
我們會用和合本修訂版的翻譯.
當你找到之後.
我想邀請你先聽我讀出第一至第三節.
然後我想請弟兄姊妹.
你說我們讀出第四至第七節.
然後之後我想再讀出.
由我讀出第八至第十四節.
然後弟兄姊妹請你又讀出十五至十八節.
最後我們一起讀出十九至二十一節.
我會再提醒大家.
我先開始讀第一至第三節.
Hallelujah 你們要讚美耶和華的名.
是立在耶和華的殿中.
耶和華的僕人.
是立在我們神殿中的.
要讚美他.
你們要讚美耶和華.
因耶和華本為善.
要歌頌他的名.
因為這是美好的.
請弟兄姊妹.
(念書).

$^{281}$第八節.
他將埃及投身的連人大新竹都擊殺了.
埃及啊.
他施行神蹟歧視.
在你中間.
在法老和他所有神僕身上.
他擊打許多國家.
殺戮大能的君王.
就是亞摩尼王西王巴沙王岳.
和迦南一齊的國度.
他嘗試他們的地位業.
作為自己百姓以色列的產業.
耶和華.
你的名字傳到永遠.
耶和華.
你的稱號傳到萬代.
耶和華要為自己的百姓伸冤.
為自己的僕人發憐憫.
請弟兄姊妹.
(念書).
我們一齊.
以色列家.
要稱頌耶和華.
亞倫加.
要稱頌耶和華.
尼米加.
要稱頌耶和華.
尼們敬畏耶和華的.
要稱頌耶和華.
住在耶路撒冷.
錫安的.
耶和華.
是應當稱頌的.
Hallelujah.
我們同心有祈禱.
天父上帝.
願你親自解開.
你自己的說話.
發出當中的亮光.
使到如幻的我們.

$^{321}$得到悟性.
求你讓我們聽.
又聽得到.
看又看得見.
有信靠的心.
去活出來.
奉耶穌基督.
明治而救.
阿們.
詩篇135篇.
可以說是這麼多首詩篇裡面.
其中一首最次等的詩篇.
用今天的角度說.
它是一首次等的詩篇.
為什麼它是一首次等的詩篇呢.
我給你看這幅圖.
你就會明白.
這幅圖裡面.
告訴我們.
凡是它有藍色.
深藍色或淺藍色的意思.
都是說那一節.
它是抄寫聖經裡面.
其中一段的經文.
如果是淺藍色的.
它就是抄寫得來.
比較多改動.
如果是深藍色的話.
就代表差不多半字過字.
這樣就抄了下來.
用今天的學術的角度來說.
這叫做什麼呢.
這叫做.
Pleasureism 抄襲.
這首詩應該失敗的.
應該被人篩出去的.
但偏偏在聖經裡面.
我們發現這首詩.
它在大部分的章節.
21節裡面.

$^{361}$它大部分的章節.
都是抄其他的經文而來.
毫不客氣地抄過來.
或者我們要問的問題.
未必是問.
為什麼這首詩.
這麼沒有道德.
這麼不顧版權.
或者我們應該問的是.
這首詩.
為什麼它要抄那些經文.
我們不妨又看一下.
剛才我們讀這段經文的時候.
我們看到第一至第三節.
還有最後.
由第十九到第二十一節.
兩個段落.
都是在呼籲那些人.
來讚美上帝.
由第一節到第三節.
我們會看到.
有五個動詞.
來呼籲那些人.
來讚美上帝.
Hallelujah.
這個字其實是.
你們要讚美耶和華.
所以第一個的要讚美.
在Hallelujah裡面看到.
然後再數下去.
你們要讚美耶和華的名.
這個是第二次.
第三節要讚美他.
是第一至第二節的最後.
然後第三節.
你們要讚美耶和華.
第四個呼籲人來讚美.
然後第三節的下半節.
要歌頌他的名.
五個的命令.

$^{401}$五個的呼籲.
我們回到第十九至二十一節.
以色列家要什麼.
稱頌耶和華.
第一個呼籲.
亞倫加要稱頌耶和華.
第二個呼籲.
第三個呼籲.
利美加要稱頌耶和華.
你們敬畏耶和華的.
要稱頌耶和華.
四個了.
最後的命令.
在哪裡發生呢.
Hallelujah.
Hallelujah是什麼.
你們要讚美耶和華.
所以一頭一尾.
兩個的呼籲.
很均勻的.
都是有五個叫人來讚美上帝.
都是要呼籲人.
你們要稱頌上帝.
這首詩歌.
叫人來稱頌上帝.
究竟稱頌上帝是什麼呢.
一開始的時候.
第一至第三節.
還有十九至二十一節.
好像一個首尾呼應.
來叫人來讚美上帝.
由第一節到第三節的時候.
他叫人要稱頌耶和華的名字.
因為耶和華的名字是好的.
耶和華本為善.
說完這句.
通常我們今天唱的詩歌.
就會收工了.
因為沒有那麼多時間.
可以繼續唱下去.

$^{441}$不過這首詩.
他就跟著用了第四節.
去到第十八節.
解釋耶和華的名字是好的.
究竟是什麼意思.
我們不如繼續看下去.
第四至第十八節.
我剛才說過.
在這首詩歌裡面.
他很多地方.
都是抄襲其他人.
我們姑且又看一下.
原來在第七節.
第十四節.
還有第十八節.
他這三節都是.
我們剛才說的是嚴重抄寫.
他差不多將別人的經文.
半字過字地抄過來.
我們從這個角度.
我們又姑且可以這樣去看.
由第四節到第十八節.
每一次嚴重抄寫.
都是那一段的結語.
都是那一段的結束.
如果從這個角度.
第四至第七節.
又是這個分段裡面的第一個分段.
然後第八到第十四節.
又是另一個分段.
十五至十八節.
又是之後的分段.
是人透過三個段落.
嘗試去跟我們說.
為什麼耶華的名字是好的.
如果我們再仔細去看的時候.
四至七節.
他有兩個字眼.
一個字是行和做.
第六節他說.

$^{481}$在天在地在海洋在各深淵.
耶華都隨自己的旨意而行.
第七節他做電隨雨而閃.
這個行這個做字.
其實在第十五至第十八節都有出現.
第十五節說.
外邦的偶像是人手所做的.
然後十八節說.
做他們的要像他.
除了這個字之外.
另一個字在這兩段都是重複出現的.
那個字就是第七節.
從倉庫中吹出風來這個字.
風這個字在第十五至第十八節.
就是第十七節裡面說氣息那個字眼.
如果從這個角度說.
四至七節.
十五至十八節.
是兩個彼此呼應著的段落.
經文序中間.
說到上帝他值得讚美的.
就是第八節到第十四節.
上帝在歷史當中的作為.
詩人他透過三個段落.
想告訴我們耶和華的名是好的.
他先告訴我們.
以色列人在列邦當中.
但是耶和華的名仍然是好的.
然後第十五至第十八節中.
透過說偶像在列邦之中.
跟著告訴我們耶和華的名是好的.
然後透過八至第十四節.
回顧上帝在以色列人當中的歷史.
也告訴我們耶和華的名是好的.
我們不如逐段逐段去看.
究竟耶和華的名為什麼是好的.
我們再看第四至第七節.
我們看第四至第七節的時候.
詩人一開始說.
耶和華揀選雅各歸自己.

$^{521}$揀選以色列作他寶貴的產業.
這句話最特別的字.
其實是雅各這個字.
通常我們說舊約裡面.
上帝的百姓上帝的子民.
他們的名字叫什麼.
通常叫做以色列人.
所以下半節我們看到.
揀選以色列作他寶貴的產業.
不過在這裡詩人.
他不是一開始提以色列的名字.
一開始他是提雅各這個名字.
而且他不單單提雅各這個名字.
更加他將它砌到另一個動詞.
耶和華揀選雅各.
揀選雅各這個字眼.
其實在聖經裡面很少出現.
其中一個另一次有出現的地方.
就是以賽亞書第41章第8節.
我邀請弟兄姊妹一起讀出這句經文.
為你以色列我的僕人雅各.
我所揀選的我朋友亞伯拉罕的後裔.
在這裡上帝說的雅各是他所揀選的.
不過如果我們知道以賽亞書第41章.
說的以色列是一個什麼時期的以色列.
它再不是一個獨立自主的國家.
當時以色列已經是一個被擄的國家.
它已經在別人的統治之下.
它成為一個沒有自己的身份.
沒有自己的能力.
沒有自己的政府的一個以色列.
這裡說的雅各.
它不是一個很強大很偉大的雅各.
這首詩歌說耶和華的名字好.
因為他揀選雅各.
正正反映當時他的百姓.
不是一個很強盛的時候.
不是一個很安穩的時候.
而是他已經成為了一個破碎.
一個沒有能力沒有自主權.

$^{561}$也都成為了別人所統治之下的一個雅各.
所以下半句就很有意思了.
下半句他說揀選以色列.
作為他寶貴的產業.
這句話源出於《新命記》第七章第七節.
我們再看第六節到第七節.
因為你是屬於耶和華的.
你上帝神聖的子民.
耶和華你上帝從地面上的萬民中揀選你.
作自己寶貴的子民.
這裡說到上帝對於以色列的揀選.
不過下半句很重要.
他讓我專愛你揀選你.
不是因為人數比其他民族還多.
其實你們的人數在各民族之中是最少的.
其實兩句話合起來就很有意思.
上帝揀選的是雅各.
上帝的名氣好.
因為他揀選了不是一個富強的以色列.
他不是說一個很偉大的以色列.
他揀選的以色列當時的雅各已經是一個.
已經淪為了別人所統治之下.
沒有自主權.
沒有自己能夠去說他可以做什麼.
就做什麼的一個雅各.
但是上帝告訴他.
我對你的揀選不會因為你的政治情況而有任何的改變.
因為從來一開始.
我對你們的揀選都是在於.
不是因為你強.
你在民眾你一向都是一個弱小的民族.
對於這班以色列人來說.
我們不知道當時他身陷什麼的處境.
但是他們知道.
上帝對他們的揀選.
不因為他們任何的處境而改變.
不因為他們處於一個風盛.
或處於一個低落的時候而有任何的改變.
接下來那三句話就很有意思.
接下來第五至第七節.

$^{601}$這三句話.
詩人都抄自三段不同的經文.
第五節.
我知道耶和華本偉大.
也知道我們的主超乎萬神之上.
這句話是抄自《釵及記》第十八章.
當時摩西的岳父葉特羅就帶著他的女兒.
去跟摩西團聚.
他知道當時上帝在以色列人當中.
做了什麼奇怪的事之後.
他就這樣說.
耶和華是應當稱重的.
他救了你們脫離埃及人和法老的手.
將百姓從埃及人的手裡救出來.
現在從埃及人狂傲地對待以色列人者事上.
我知道耶和華比萬神更大.
《新漢語》譯本也比較好.
他說現在我知道上主偉大超過萬神.
對到第五節.
詩篇一八三十五篇第五節.
我知道耶和華上主偉大.
他知道我們的主超過萬神.
不過有趣的地方是.
葉特羅是什麼人.
摩西的岳父.
他的國籍是什麼人.
他是一個米顛人.
也就是說他的國籍不是以色列人.
他是一個外邦的人.
走來說你們這一神厲害.
我們再看第二句.
你會發現第二句也是這樣.
第二句他在說第六節.
在天在地在海洋在各深淵.
耶和華都隨自己的旨意而行.
其實這句話也是抄自另一首詩篇.
詩篇一八一十五篇.
如果各位姐妹不介意.
我請你們一起讀出二至三節.
為何讓列國說他們的上帝在哪裡.

$^{641}$但是我們的上帝在天上.
萬事都隨自己的旨意而行.
你看到相同的地方.
不過這句話在一八一十五篇.
他在回應一句什麼話呢.
他在回應當時的列邦.
走去問那些以色列人.
你們的上帝在哪裡.
列邦以為他們贏過了以色列.
所以他們的上帝是大過以色列人的上帝.
然後就去勸那些以色列人.
你們的上帝沒用的.
他在哪裡.
然後詩篇一八一十五篇的那些人.
就告訴自己的子民.
不是我們的上帝隨己意而行.
他其實才是大過列邦的上帝.
又是再一次說.
我們的耶和華是超乎列邦的神.
那一位的耶和華.
第三節.
接下來的第七節.
他使雲霧從地極上騰.
造電隨雨而閃.
從倉庫中吹出風來.
這句話抄得很厲害.
我們來看看.
對不起.
沒寫到.
沒理由的.
在這裡.
耶利米蘇的第十章.
第十至十三節.
請你聽我讀出.
為耶和華是真上帝.
是活的上帝.
是永遠的王.
他一發怒.
大地震動.
他怒恨列國擔當不起.

$^{681}$你們要對他們說.
那些不是創造天地的神明.
必從地上從天下被除滅.
耶和華以能力創造大地.
以智慧建立世界.
以聰明鋪張窮倉.
十三節.
他一出聲.
天上就有眾水澎湃.
他使雲霧從地極上騰.
造電隨雨而閃.
從倉庫中吹出風來.
我們看到那三句相同的經文.
不過這三句相同的經文.
都是回應一件什麼事情.
都是回應.
耶和華的上帝.
是以智慧來創造天地.
以聰明來建立銅倉.
所以.
列邦的神在他面前.
一文不值.
全部都不是神.
唯有耶和華才是.
永活的上帝.
其實詩篇135篇.
第五節.
第六節.
第七節.
三節.
由葉特羅的口中.
告訴人.
一個外邦人口中告訴人.
耶和華才是那位.
有能力的上帝.
然後透過第六節.
透過一一五篇的詩人.
告訴人.
列邦對於耶和華上帝的質疑.
其實是完全不能夠站立得穩.

$^{721}$然後再透過第七節.
他請耶利米上來.
感覺好像不是很好.
請他上來.
然後告訴人.
其實唯有我們的耶和華上帝.
才是永活的上帝.
耶和華揀選雅各.
雅各沒錯.
今天.
你已經陷入一個不能夠自主.
不能夠有自己地位的處境.
不過不要緊.
詩人連續三句.
告訴以色列人的百姓聽.
其實.
縱然你在一個這樣的政治處境之下.
但是其實.
列邦不能夠動你分毫.
你今天面對最大的危機.
最大的挑戰.
並不是外面的勢力對你的影響.
縱然你覺得你自己沒有能力來招架.
縱然你覺得你已經沒有能力去應付.
這些列邦的強勢.
但是詩人一而再.
再而三.
告訴我們.
我們今天面對最大的挑戰.
最大的牛機.
不是我們覺得在外面.
很大很大的勢力.
這一幅圖.
不知道大家熟不熟悉.
如果你曾經去過一些天主教的靜修院的話.
你可能會熟悉這一幅圖畫.
在很多天主教的靜修院裡面.
它都會有一個一系列的圖像.
那個叫做苦路十四站.
這一站是第九站.

$^{761}$他們通常傳統會叫這一站.
做耶穌他第三次跌倒.
第一次跌倒是在第三站.
第二次跌倒是在第七站.
第三次跌倒是在第九站.
你想像到第一次在第三站.
第二次捱到去第七站.
但是接著呢.
捱得不是很久.
第九站他第三次跌倒.
我經常,我有一次看著這一幅畫.
我看見那位跌倒在那裡的基督.
你發現.
如果你比較那三站的圖畫的時候.
第三次的跌倒.
耶穌他跌得比前兩次更加低.
他差不多整個人他趴在地上.
你發現你不斷會問.
其實耶穌你這麼辛苦.
為什麼你還要爬起來.
為什麼在這個處境裡.
為什麼你還要爬起來.
你明知你爬了起來之後.
其實前面只會等待你的.
可能是第四次,第五次,第六次的跌倒.
甚至你知道最終你爬了起來之後.
你只能夠望見的是.
最終你就是要被釘上十字架.
為什麼你跌了第三次.
你還要起身.
等於會正如你和我一樣.
可能有時候你都在問你自己.
為什麼我還要爬起來.
我經歷著我的病患.
我受到每一天的痛的折磨.
其實我真的不太想再起身.
但是為什麼我還要爬起來.
可能你心裡面經歷著你的情緒.
經歷著你的哀傷和折磨.
其實每一天你都不知道你的情緒的困擾.

$^{801}$你明天起床你將會是一個怎樣的人.
你都不知道.
為什麼我還要繼續爬起來.
在我工作的地方.
我每一天望見那個折磨我的老闆.
我連押都不敢.
但是我每一天仍然要開車上班.
每一天都知道我今天會承受下一輪的折磨.
為什麼我要爬起來.
在香港的處境.
一輪又一輪.
我們發現我們今天所面對的.
香港所擁抱的價值.
香港所擁抱的所謂的法治.
一輪又一輪好像被侵蝕.
有時候我們會問.
為什麼你還要繼續起身.
來抗爭.
或者來發出我們的聲音.
我們每個人都曾經是亞國.
我們每個人都曾經發現.
原來在我們生命裡面.
我們沒有能力去自主.
我們沒有能力去對抗.
我們當前所要面對的困難.
正如耶穌一樣.
第三站.
祂只能夠再一次跌倒.
耶穌為什麼你要爬起來.
也許可能對於基督來說.
祂發現祂自己興翰祂自己的羞辱.
也知道未來祂面對的十字架.
不是祂的終局.
所以祂可以爬起來.
或者甚至祂明白那件事是.
十字架所帶給祂的影響.
那個外來壓力的影響.
不能夠注定了祂最終的結局.
以至祂再一次爬起來.
這首詩篇.

$^{841}$第一部祂告訴我們.
上帝揀選了我們這班亞國.
不是有能力的亞國.
我們是一班沒有能力的亞國.
但是祂想提醒我們.
透過讚美告訴我們.
耶和華的名是好的.
因為其實外來的勢力.
外來的影響不是最致命.
不是最決定我們未來的.
所謂的棋子那些因素.
外來的勢力一次又一次.
在這首詩歌告訴我們.
祂不能夠動我們分毫.
因為我們的上帝.
是一位遠遠大過這些勢力的上帝.
第十五至第十八節.
詩人講完.
外來的勢力不是最大的危機.
不是最致命的事.
那什麼是最致命的事.
十五至十八節.
祂就為我們講出來.
我們再一次讀出第十五至第十八節.
外邦的偶像是金的.
是銀的.
是人手所做的.
有口卻不能言.
有眼卻不能看.
有耳卻不能聽.
口中也沒有氣息.
做他們的要像他們一樣.
凡靠他們的也必如此.
這段經文其實來自詩篇的一百一十五篇.
中間詩篇一百一十五篇.
都同樣講那些外邦偶像.
不能做這樣.
不能做那樣.
也不能做這樣.
如果你看一看.

$^{881}$我們當中變了出來的筆字的時候.
你會發現詩篇一百一十五篇非常之好.
他講了那些偶像不能言.
不能看.
不能聽.
不能聞.
不能摸.
不能走.
不能說話.
多少個筆.
有七個筆.
大家會聽慣的.
在舊約聖經裡.
七代表一個完全.
完全什麼.
完全不能.
其實如果照抄下來給我.
我一定抄了七次筆字.
但是詩篇一百三十五篇.
你看他抄了多少次筆字.
三個筆字.
打了折扣.
不過最終他有一句話.
非常之特別的.
翻譯了出來.
最終這句話.
口中沒有氣息.
在希伯來文裡.
如果要翻譯出來.
是一句非常之巧口的說話.
其實沒有氣的存在.
在他的口中.
是不是很笨拙.
這句話為什麼要寫得這麼笨拙.
但是他正正想帶出一個.
很重要的字眼.
如果你有印象.
剛才我們說在這段經文裡.
除了做這個字之外.
另一個字和第四節到第七節.

$^{921}$重複的字眼.
那個字是氣息.
氣息這個字.
如果沒有氣息.
在那些偶像的口中.
代表著那些偶像是什麼.
死.
大家都說得出.
這群偶像.
他不能夠給任何的氣給自己.
不能夠給自己任何.
有生命的跡象.
相對來說.
上帝第七節他做什麼.
他使雲霧從地極上騰.
做電隨雨而閃.
接著從倉庫中.
做什麼.
吹出氣來.
大把氣的在哪裡.
在上帝的倉庫當中.
上帝也說上帝.
非常能夠帶給生命.
但是這些偶像.
要給自己口中有氣.
他都沒有能力.
他只是一個死的偶像.
但最恐怖的那件事.
他說了什麼.
下一句說話.
十八字.
做他們的.
就要像他們一樣.
凡靠他們的.
也必如此.
做偶像.
靠偶像的那些人.
他們都成為.
不能看.
不能聽.

$^{961}$不能聞.
也都怎樣.
好像偶像一樣.
毫無氣息.
真的嗎.
真的看不到嗎.
真的懵到.
他聽不到嗎.
如果我們聖經裡.
其中一個很出名的事件.
加密山上.
以利亞大戰.
450個巴力先知.
我最深印象的.
不是這個夜話.
降火下來那一下.
我最深印象的是.
如果你有印象.
當以利亞挑戰巴力先知.
叫你的神降下火來吧.
他們由早到晚.
他們就在跳舞.
跳著跳著的時候.
發現不對了.
沒有任何聲音.
於是他們開始做什麼.
他們開始用刀.
用劍出來.
自割.
自橫.
然後就流出.
渾身是血.
我經常看到這裡.
我都覺得很匪夷所思.
如果這班巴力先知.
都只是找一頓飯吃.
由早到晚.
在這裡跳舞.
我想大家都做得到.
不過出刀來.

$^{1001}$要傷害自己.
要刺下去那一下.
還要刺到渾身是血.
為什麼你會這樣做.
這筆錢很難賺.
但為什麼他們都會這樣做.
如果你想想.
如果在香港來說.
在一些殯儀行業當中.
破地獄的那些.
就叮叮噹噹地獄.
打爛那些東西.
沒問題.
大家都做得到.
不過破完之後.
然後那些符水.
要啪一聲.
喝下去.
我猜沒什麼人會從事這個行業.
為什麼巴力先知.
竟然會這樣做.
我後來突然間驚覺一件事.
原來巴力先知.
覺得他們信的東西.
是真的.
他們覺得循著這種方法.
他們的神明真的會聽他們的.
他們倚靠著那些神明.
令他們分辨不到.
再分辨不到.
哪一樣東西是真的.
哪一樣是假的.
他們不能看.
不能聽.
其實在二賽亞書44章.
很清晰地說了一次給我們聽.
一個做偶像的木匠.
他做偶像的過程裡.
他突然間覺得很冷.
又覺得很餓.

$^{1041}$於是就拿著一塊木.
剁了一半.
然後放進柴火裡取暖.
還有在燒雞.
燒完之後就說.
哇飽了.
現在暖了.
另外那半塊木.
就用來做什麼.
就做他的偶像.
但他們從來不會覺得.
我拿塊木出來.
來燒烤.
然後取暖.
但用另一塊木來.
就覺得他是一個神明.
是一件很奢侈.
或者很傻的一件事情.
二賽亞書已經很清楚地告訴我們.
他們不明白.
也看不見.
原來那些偶像.
當我們信靠的時候.
的而且確.
我們真真正正.
會成為不能看.
不能見.
不能聞.
也都好像他們一樣.
不再有任何的氣息.
對於你們來說.
今天我不再拜偶像了.
全都扔掉吧.
不過聖經往往說的偶像.
其實就是覺得.
你認為他能夠帶給你好處.
和救贖的.
那樣東西.
就是偶像.
對於香港的人來說.

$^{1081}$孩子的偶像.
是什麼呢.
孩子的偶像.
就是贏在.
起跑線上.
真是慘.
連你們都知道.
我們的風雲到來這個地方.
都讓你們知道.
大家為了爭取一個好的學校.
大家為了有一個好的將來.
無所不用其極.
希望孩子們.
可以有一個美好的將來.
因為贏在起跑線上.
是大家覺得帶來.
大家的好處.
大家的救贖.
那個偶像.
我想講另一個例子.
在1949年至到1951年間.
中共政府建國.
它建國之後.
它就嘗試去籠絡各方.
希望讓他們都去承認它的政權.
當時它其中一個要籠絡的對象.
就是基督教.
而當時基督教也有一個所謂的代表.
或者所謂的聯會.
我們當時叫基督教協進會.
協進會當時.
它就和中共政府.
有很多的所謂的交涉.
希望既能夠在這個過程當中.
它能夠明白新時代裡.
他們當時的新時代裡.
基督教應該怎樣可以繼續生存下來.
在這個過程裡.
我們的其中一位老師.
宋君老師.

$^{1121}$就讓我們看到.
在這個協進會.
它一步一步認為.
它唯有靠著這個權勢.
靠著這個政權.
以至這個政權.
才可以讓他們.
所謂當時基督教的事業.
有一個生存下去的空間.
所以在當時.
中共政府要求基督教界.
簽一些宣言.
某些人他們就願意去簽.
但是協進會在很多的思考之下.
最終也都因為這個政權的緣故.
它選擇了簽.
簽了什麼呢.
就是中國的基督教.
是和西方的基督教.
帝國主義之下的基督教.
要一刀兩斷.
彼此之間不可以再有任何的來往.
第二就是基督教界.
都要支持當時中共政府.
推行的土地改革的策略.
其實在這兩件事上.
宋君老師都讓我們看到一件事.
其實整個的大公教會.
怎麼能夠將它獨立而存呢.
第二件事.
土地改革這件事.
當時政府用的方法.
是將從別人身上搶走土地.
然後再平均分配.
這個其實也都不是在信仰裡面.
可以站得住腳的一個立場.
不過為了換回一個可以生存的空間.
協進會當時選擇了.
來信仰這個政權.
宋君老師和我在閒聊之中.

$^{1161}$他這樣和我說.
他說正正是因為這樣.
協進會他看不到.
他自己本身作為一個教會.
他自己的真相是一個什麼的身份.
同樣他都看不到.
和他交手的政權.
是一個怎樣的政權.
凡倚靠他的都別如此.
不能看不能問.
最終在五一年.
協進會在政府的手段之下.
一步一步被瓦解.
領事會私人想和我們說.
外部的勢力.
我們覺得外部很大的威脅.
不能夠置於瓦角於死地.
不過什麼會置於瓦角於死地.
就是他們靠了那些不應該倚靠的偶像.
將那些不應該視為.
會帶給他們有好處.
會帶給他們有終極的救贖.
那些事視為他們最終要倚靠的對象.
這件事就成為一個最致命的危機.
你翻開聖經.
由舊約開始說到新約.
都是在說我們要禁戒我們做什麼.
拜偶像.
弟兄姊妹.
今天你的偶像是什麼.
那些偶像才是置我們於死地的偶像.
很可惜.
我們知道往往我們都未能夠撕掉那套印.
這首詩中間那段.
它給了我們一個很特別很特別的提醒.
第八至第十四節.
他將埃及投生的連人帶身畜都擊殺了.
埃及啊.
他施行神蹟奇事在你們中間.
在法老和他所有神僕之上.

$^{1201}$他擊打很多國家.
殺戮大量的君王.
就使亞摩尼王西王巴薩王岳和迦南一切的國度.
他嘗試他們的地為業.
作為自己百姓以色列的產業.
耶和華你的名字傳到永遠.
耶和華你的稱號傳到萬代.
耶和華要為自己的百姓伸冤.
為自己的僕人發憐憫.
詩人他回顧上帝對於當代的以色列人的一些歷史.
在曠野時代.
是未成立立國之前的以色列人的歷史.
第八至第九節.
就說上帝如何在迦南當中.
施行他的奇事.
擊殺當中的君王.
接著第十至十一節.
就說他們未進入迦南地之前.
如何擊殺亞摩尼王西王巴薩王岳和迦南一切的國度.
所有的王上帝都能夠擊殺.
不過在這兩件事件的中間.
有一些很重要的事.
我們發現詩人他沒有提到.
那件事是甚麼事.
救他出埃及.
接著他們要進入迦南地之前.
又去擊殺那些敵對他們的君王.
中間有一段甚麼事件沒有提到呢.
大家都很熟悉.
中間隔了多少年.
四十年.
四十年以色列人沒有聽說這件事件.
詩人他沒有提到.
不過重點是他不是想說這件事.
他想說的是.
縱使當中四十年以色列人多次背叛上帝.
但上帝仍然好像當初救他們離開埃及一樣.
當他們要進入迦南地之前.
仍然為他們擊殺他們所有的手下.
縱使他們不是一班聽話的百姓.

$^{1241}$但上帝對他們忠誠的慈愛.
上帝對他們守約的慈愛.
仍然幫助他們來擊殺當中敵對他們的君王.
所以這樣說的話.
十四節就顯得很有意思.
十四節.
耶和華為自己的百姓申冤.
為自己的僕人發憐憫.
這句話其實是出自生命記第32章的第36節.
生命記第32章.
那段經文.
上帝對摩西說.
你寫下這首歌.
指正這班百姓.
將來他們進入迦南地之後.
他們一定會如此來搏擊我.
不聽我的話.
整首摩西之歌在第32章裡.
就說到這班百姓將來會怎樣選擇.
離棄耶和華他們的上帝.
二十八節他們就說.
因為他們是缺乏自貿的國家.
以色列人.
他們在裡面毫無聰明.
他們有智慧能明白這件事.
他們就想到自己的結局.
他們就不會選擇離棄上帝.
但是接著35節.
上帝就說.
因為他們離棄我.
申冤報應在我.
到時候他們會怎樣.
失敗.
因為他們遭難的日子近.
他們的厄運快要臨到.
上帝要按著他們所做的.
來加懲罰在他們身上.
但是36節說什麼.
36節說.
我們一起讀出來.

$^{1281}$耶和華見他的百姓毫無能力.
無論是遺老的.
自由的.
都沒有存留.
就必為他們申冤.
為自己的僕人發憐憫.
上帝什麼時候為他們僕人申冤.
什麼時候為他們僕人發憐憫.
見到他們毫無能力.
縱使上帝他發他的路.
來懲罰他的百姓.
但是當他懲罰的過程之中.
見到他們無力來抵擋.
無力來招架的時候.
上帝就發憐憫.
接著來為他們去申冤.
耶和華的上帝由始至終.
才告訴他們.
以色列人那段歷史.
由埃及到臨入迦南地之前.
見證的一件事.
縱使人多次的來到搏役.
但是上帝見著我們那份毫無能力.
他就手約師前來.
我們人生之中.
最決定我們未來的.
不是外邦的勢力.
甚至都不是我們搏役上帝.
去倚靠偶像那些的罪惡.
師人想告訴我們.
我們人生之中最為關鍵的那件事.
就是上帝對我們千載不變的.
那份手約的慈愛.
師人說.
當我們不能擺脫對偶像的倚靠的時候.
我們重新去紀念.
哪一個是真真正正.
對我們有那種扭轉性的慈愛.
縱使我們上上都不聽他.
縱使我們搞到自己一個這樣的天地.

$^{1321}$我想陳牧師很明白.
明明白白的跟弟兄姐妹說.
那個人你不要跟他結婚.
但你都一意孤行.
結婚之後就怎樣.
婚姻出現問題.
陳牧師會怎樣.
陳牧師說.
老祖跟你說了.
不理你了.
你不要再跟我說有任何的輔導.
不會.
陳牧師見大家什麼.
毫無能力.
就會怎樣.
就會發憐憫.
為他們來到深淵.
這個就是上帝對我們那份的慈愛.
而這份的慈愛.
就是我們人生最扭轉性.
最決定性的因素.
這首讚美的詩歌.
它提醒我們.
不要見到外面的勢力.
我們就會害怕.
我們就會向他們屈膝.
因為那些事不能夠最終影響我們.
這首詩歌提醒我們.
提醒我們什麼.
當人家爭著要我們稱他為.
所謂帶給我們最好的好處.
帶給我們最終的救贖.
那些偶像.
我們要懂得分辨.
以致我們不要將上帝應有的讚美.
歸給其他人.
但這首詩歌更加提醒我們.
這張桌子寫著.
In remembrance of him.
我們常常記住.

$^{1361}$上帝那個不變的慈愛.
所以這首詩.
去到最終的時候.
它再一次呼籲人.
來去稱頌上帝.
但這首稱頌的上帝.
其實我們如果看下去.
一百一十五篇的時候.
其實這個稱頌上帝.
在一百一十五篇.
它用的是信靠.
怎樣幫我們真真正正.
可以信靠到我們的上帝.
這首詩歌說.
透過我們真正的讚美.
我們明白什麼是讚美.
我們就能夠真真正正.
學習怎樣去信靠我們的上帝.
弟兄姊妹.
每個星期日.
我們會回來教會.
我們會在教會裡面去讚美.
讚美我們發現.
原來再不是為了熱身.
再不是為了令到你們心情好一點.
再不是崇拜程序裡面.
一個可有可無.
一個所謂的雞欄.
讚美其實是每一個星期.
我們回來.
我們在預備.
當我們在面對.
我們再不是一個.
所謂能夠有自主.
或者有外來很多壓力之前.
其實每一次.
我們都是一個預演.
地震前的一個預演.
我們幫助我們.
對準哪一個.

$^{1401}$才是我們值得讚美的對象.
以致當我們在挑戰.
危難面對之前.
我們不會失的.
我們不會忘記了.
我們真正忠誠.
信靠的應該是哪一個.
心願大家每個星期.
我們再一次去讚美的時候.
我們明白.
原來我們真是在讚美當中.
我們在打一場.
很重要的一場仗.
願上帝藉著祂的話.
祝福大家.
(字幕製作:貝爾).
(字幕由 Amara.org 社群提供).
\newpage



\section{}
\label{sec:jGn4gOBS1HA}
\textbf{中神墨爾本培靈會:用我一生}
\newline
\newline
連結: \href{https://youtube.com/watch?v=jGn4gOBS1HA}{\texttt{ https://youtube.com/watch?v=jGn4gOBS1HA}} ~~~~ 語音日期: 2018-11-12 
\newline
\newline
\hyperref[sec:XGZrzl_HY54]{\small{< < < PREV SERMON < < <}}
~
\hyperref[sec:index]{\small{[返主目錄]}}
~
\hyperref[sec:EtbCz7LXbWY]{\small{> > > NEXT SERMON > > >}}
\newline
\newline
$^{1}$(廣播).
我先代表香港中國神學研究院.
問候在 Melbourne 的弟兄姊妹平安.
今晚的時間會有兩個的部分.
講到之後我會提出有幾件事情.
大家紀念在禱告當中為忠臣前面.
我們面對的挑戰.
來一同警醒守望待土.
我先讀出今晚我們要看的聖經.
新約彼得後書第一章.
第五節到第七節.
彼得後書第一章第五節到第七節.
我為大家讀出.
正因者緣故,你們要分外的因勤.
有了信心,又要加上德行.
有了德行,又要加上知識.
有了知識,又要加上節制.
有了節制,又要加上忍耐.
有了忍耐,又要加上虔敬.
有了虔敬,又要加上愛弟兄的心.
有了愛弟兄的心,又要加上愛眾人的心.
使徒彼得,寫彼得後書.
其實他心裡知道.
他離開世界的時候快到.
如果你繼續讀下去.
在第十二節到第十五節.
他就重覆三次.
跟弟兄姊妹說.
我以為應該趁我還在這帳篷的時候提醒你們.
因為知道我脫離這帳篷的時候快到了.
這兩句是比喻.
第三句彼得就明講了.
在第十五節.
並且我要盡心竭力.
使你們在我去世以後.
時常紀念這些事.
我們知道彼得是為主在羅馬殉道.
所以大概他感受到壓力越來越重.
所以在他還有一點點的時候.
他把握這個機會.

$^{41}$寫了這封信給教會留下.
分享他的心聲.
剛才我們讀的第五節到第七節.
就是他一開始在這封信裡面.
他重覆提到這些事.
這些事,這些事.
講了很多次.
他很記掛弟兄姊妹.
但他更記掛的是弟兄姊妹能夠有以上他所提的這八件事情.
所以他不是客氣和大家說風花雪月.
新約聖經也很少掛念教會建堂成不成功.
人數有沒有增加.
事工是否順利.
不是,彼得心裡面很清晰的一個焦點.
就是弟兄姊妹的生命.
在主的恩典和真理當中.
怎樣成長.
他不是像保羅,彼得和保羅很不同.
保羅是一個學者.
在耶路撒冷受教於加瑪列的門下.
所以他寫羅馬書.
有層有次.
很清楚將信仰一點一畫.
每一個教義解釋得深入淺出.
有時也不是很淺.
要很心機才讀得明白.
彼得不是一個神學家.
他本來的職業是打魚.
打魚的人不讀書.
如果在船上下網,殺網,收網.
他只顧看書.
一來書本會弄濕了.
也沒有,哪有書看.
所以對打魚的人來說很實際.
是他的經歷告訴他.
信仰成長到底是一件甚麼事.
新約聖經裡面記載不同的人物.
除了主耶穌之外.
我們可能知道得最多的片段是彼得.
甚至可能還清楚過保羅.

$^{81}$當然保羅的書信幫助我們明白他內心.
不單止他的思想.
也是他裡面的掙扎.
但彼得的一生.
其實在新約聖經.
我們可以從他的心上去明白.
他在這裡所說的八件事.
有了信心又要加上德行.
有了德行又要加上知識.
有了知識又要加上節制.
有了節制又要加上忍耐.
有了忍耐又要加上虔敬.
最後那兩個容易記.
愛弟兄和愛眾人的心.
彼得是用他的生命的經歷.
來寫出他最後給弟兄姊妹的書信.
信心是一個基督徒靈性生命的起點.
你記得彼得是怎樣信耶穌的嗎?.
你說有一天在街邊走著走著.
有一個白色袍鬍鬚的男人迎面來.
跟你說一句Follow me.
那你就辭了工就跟著他.
但他不會,神經病嗎?.
彼得也不是這樣跟從主的.
你讀四福音.
彼得第一次不是在加利利湖邊.
不是兩兄弟四個人撇下魚網.
那個不是他第一次見主.
你有沒有印象?.
約翰福音第一章.
不記得了嗎?.
記得新約福音書的記載.
施洗約翰在約旦河邊有兩個門徒.
約翰介紹主耶穌.
看了神的羔羊.
這兩個門徒其中一個不是西門彼得.
是他的兄弟安德烈.
約翰福音第一章.
安德烈回去找到彼得.
帶彼得去見主耶穌.

$^{121}$耶穌跟他改名.
以後不叫西門了.
彼得有沒有在約旦河跟隨耶穌?.
沒有.
是回到加利利湖.
記載最詳細的不是馬太,馬可或約翰福音.
而是路加福音.
路加福音第五章.
那天耶穌借了西門的船.
坐在船上.
向百姓講道.
耶穌講道也挺長氣的.
否則就不會有五餅二魚那一次.
日頭平西.
門徒說再不叫他們吃飯回家.
大家就肚餓了.
所以耶穌講完道之後.
路加福音記載.
他對西門說.
路加福音第五章第四節.
把船開到水深之處.
下網打魚.
我們很喜歡這一句.
以前還有夏令會的時代.
我不知道現在的教會有沒有夏令會.
很喜歡這一句做主題.
開到水深之處.
靈性盡心.
你知道耶穌這句話的意思是什麼嗎?.
你不知道嗎?.
你用了別人的船大半天.
你要付租金.
你不可以走路.
所以耶穌是付租金給西門.
但他身上沒有錢.
所以他對西門說.
你開船出去.
下網打魚.
西門怎樣回答他?.
西門說.

$^{161}$夫子,我們整夜勞力.
並沒有打著什麼.
給很尊敬耶穌.
夫子.
老師,博士,院長.
講道理得.
打魚.
我在行.
我們整夜勞力.
打魚是什麼時候打的?.
不是天光白日.
我們昨晚沒有偷懶.
但沒有魚.
我們整夜勞力.
並沒有打著什麼.
給這一句一出口.
他知道.
講快了.
不客氣.
收不回.
收回.
但依從你的話.
我就下網.
好,這次聽你的.
不過打不到魚.
我有言在先.
不要怪我.
也不怪你.
你外行,我在行.
給兩句很簡單的話.
其實他心裡想什麼.
你會聽的.
你聽到的.
結果如何?.
你知道.
他們下了網.
圈著很多魚.
網險些裂開.
招呼另外那隻船.
一起裝滿了兩隻船.

$^{201}$甚至船要沉下去.
在這個時候.
西門彼得.
俯伏在耶穌的膝前說.
這次他不稱呼耶穌為夫子.
他叫耶穌為主.
西門跪在主的面前.
他說主啊.
離開我.
我是一個罪人.
沒有.
耶穌沒有數他有什麼罪.
為什麼彼得突然間.
會覺得他是一個罪人呢?.
我們太多時候以為.
犯罪就是做錯事.
說錯話.
想些不應該想的事.
看些不應該看的事.
不是.
彼得在這個時候.
他明白到.
有眼無珠.
不認識主.
就在他的面前.
所以他說主啊.
離開我.
我是一個罪人.
在這樣的情況之下.
耶穌對他說.
不要怕.
從今以後.
你要得人.
在這個時候.
耶穌呼召他.
來跟從我.
我要叫你得人.
如得魚一樣.
他們就撇下所有.
跟從耶穌.

$^{241}$Eugene Peterson.
The Message.
翻譯彼得後書.
信心.
一個字希臘文.
Pistes.
他用了兩個英文字.
Basic Faith.
基本的信心.
信心的基礎.
認識主耶穌.
跟從主耶穌.
這是彼得.
屬靈的生命.
開始的第一步.
但不可以停在這一步.
很多人停在這一步.
我信了主.
我得救了.
連水禮都不需要參加.
不是,有了信心.
彼得接著說.
又要加上.
得行.
Good Character.
Peterson 的翻譯.
好的品格.
西門彼得在耶穌十二使徒當中.
馬太,馬可,路加三卷福音書.
記載這個名單的時候.
頭一個.
一定是西門彼得先行.
意思就是說.
你可以想像.
他是老大.
他是眾門徒所尊敬.
他是大家都服的一個領袖.
西門彼得不是靠嘴巴.
令他的同伴佩服他.
欽佩他.

$^{281}$你試一下在教會裡有一個這樣的弟兄.
只是靠嘴巴.
大家避之則吉.
我們中國人說得好.
以德服人.
你從彼得在十二使徒中間.
他一定是走在前面.
是帶領的.
大家是佩服他的.
跟他的.
聽他的.
你從這一點就知道.
他是一個眾望所歸.
他是一個有德行.
有品格.
這樣的一個人.
有了信心.
基本的信心.
你還要有好的品格.
有了信心和品格.
彼得說第三件事.
是什麼.
知識.
知識這個我懂.
不是那麼難.
就是讀神學.
或者拿學位.
讀書.
知識.
Gnosis這個希臘文.
Peterson不是這樣翻譯.
Spiritual Understanding.
屬靈的洞察力.
我們大家都記得.
那次在蓋薩利亞菲律賓.
耶穌考他的門徒.
人說我人子是誰.
馬太福音第十六章.
那些門徒.
每個人都拿他們的民意調查出來.

$^{321}$有人說你是施洗約翰.
有人說你是先知.
有人說你是誰誰.
大家說得口沫橫飛.
原來耶穌這個問題.
好像我們以前在香港中學.
是Mock Exam.
是不是.
不是真正考試的題目.
耶穌接著問第二個問題.
你們說我是誰.
西門彼得回答說.
你應該會背.
你是基督是永生神的兒子.
耶穌怎樣讚彼得.
聰明.
聽到沒有打瞌睡.
有做筆記.
有溫書.
有讀書.
不是不是.
你聽主耶穌說.
馬太福音第十六章.
第十七節.
耶穌對他說西門巴約拿.
你是有福的.
因為這不是屬血肉的.
只是你的.
乃是我在天上的福.
只是你的.
西門彼得認識耶穌是基督.
是永生神的兒子.
不是讀神學回來的.
不是他有多少個學位.
不是他家裡有多少本書.
不是他過目不忘.
不是他努力慢慢在這裡混雜.
是上帝啟示給他.
Spiritual Understanding.
所以不是他自己賺回來.

$^{361}$努力能夠獲取的結果.
很快的你知道發生什麼事.
接著耶穌跟門徒說.
要上耶路撒冷.
受苦被殺.
然後復活.
二十二節.
彼得就拉住耶穌.
勸他說.
主啊萬不可如此.
這事必不臨到你身上.
你背下去.
我猜你也記得這兩個字.
耶穌轉過身來對彼得說.
什麼?.
撒旦.
嘩!.
一邊稱讚他天上有.
一邊說他是撒旦.
不是,你聽清楚一點.
耶穌不是說彼得是撒旦.
耶穌說的話全句是這樣說的.
撒旦退我後邊去吧.
你是伴我腳的.
因為你不體貼神的意思.
只體貼人的意思.
體貼神的意思.
不體貼人的意思.
其實反過來.
我們每一個人.
都一定是按自己的意思來祈禱.
你病的時候怎樣祈禱?.
主啊醫治我.
耶穌怎樣祈禱?.
父啊照你的意思.
不要照我的意思.
其實耶穌傳道.
魔鬼試探他三次.
都是這個選擇.
很正常的.

$^{401}$所以每一件事要做決定的時候.
跟我自己的看法.
跟我自己的意思.
有什麼不同?.
如果這個意思不是神的意思.
那這個就是來自撒旦的試探.
彼得很深印象.
這一邊耶穌稱讚他正在飄飄然.
接著忽然之間明白.
原來你要認識神.
不是頭腦的知識.
而是心靈的深處.
願意遵行神的旨意.
那以後怎樣祈禱?.
是,你仍然可以求你心裡最想的.
不過學學耶穌.
祈完之後不要這麼快奉主名求阿門.
在這句之前.
加多一句是什麼?.
然而不要照我的意思.
只要照你的意思.
Spiritual Understanding.
那個屬靈是來自聖靈.
來自上帝的啟示.
來自祂的幫助.
祂的提醒.
信心.
Basic Faith.
基本的.
入門的第一步.
德行.
Good Character.
我們需要努力的.
百尺竿頭.
還可以更進一步.
Spiritual Understanding.
不是我們自己擁有.
不是我們自己努力.
看到彼得一路分享.
他生命裡的成長.

$^{441}$他的經歷.
真的越來越深.
不是拿完這個拿那個.
有了這個又加上那個.
不是.
是上帝一步一步的帶領他.
讓他從信仰的起手.
從他的德行品格.
現在他明白到.
原來神在他身上.
一步一步的帶領.
讓他明白.
認識神.
又讓他願意.
順從神的意思.
信心.
德行.
知識.
懂得解釋嗎?.
有時要轉一轉.
第四.
節制.
這個沒錯了.
我知道節制.
不要吃朱古力.
不要吃這麼多沒益的東西.
我們發覺.
好吃的東西就沒益.
有益的東西不好吃.
節制.
我回到香港.
我的家庭醫生.
是一位教會的長老.
他是我父親的醫生.
也是我的醫生.
所以去見他.
一定要磅重量高量血壓.
有一次.
他對著我很認真地跟我說.
司徑.

$^{481}$你一定會背聖靈九個果子.
我看著他.
我點頭.
第九個是什麼?.
節制.
你知道他接著說什麼.
Peterson 怎麼翻譯這個字呢?.
Alert.
Discipline.
是Discipline.
但Discipline這個字不只是說.
看到好的東西不吃.
Discipline是紀律.
是操練.
每天要做運動.
現在Pokemon來到Melbourne.
你要孵蛋.
要走10K.
這是唯一令美國人肯走路的遊戲.
不要罵到他這麼慘.
Discipline.
是一個操練的經歷.
Peterson再加上前面.
Alert 警醒.
警醒的操練.
我們記得.
路加福音.
記載耶穌在上十字架之前.
他怎樣跟彼得說.
路加福音22章.
第31節.
我讀給大家聽.
主又說.
西門 西門.
撒旦想要得著你們.
好西你們像西墨紙一樣.
但我已經為你祈求.
叫你不至於失了信心.
你回頭以後.
要堅固你的弟兄.

$^{521}$耶穌又說撒旦.
不過這次不是罵彼得.
這次是對著彼得說.
你們將會面對很嚴峻的考驗.
他用的比喻我們不太明白.
什麼叫西墨紙.
不要緊.
略過了比喻.
你應該明白耶穌的意思.
耶穌接著說了一句.
其實很特別.
耶穌說我已經為你祈求.
叫你不至於失去信心.
如果耶穌這樣對我說.
你將要面對極大的考驗.
但是我已經為你祈禱.
叫你不至失去信心.
牧師說我都半信半疑.
耶穌這樣對我說.
西門彼得這次沒有死.
一定站得穩.
耶穌為他祈禱.
你聽下去.
最後主耶穌說.
你回頭以後.
要兼顧你的弟兄.
聽得懂嗎?.
說回頭.
即是已經跌倒了.
你一定會跌倒.
不過你要回頭.
所以彼得聽完之後受不了.
你懂不懂得這樣讀聖經?.
你不懂就很怪的.
靈修.
彼得說.
主啊.
我就是和你坐牢.
我和你受死.
也是甘心.

$^{561}$聽到彼得抗議嗎?.
怎麼會回頭?.
我和你坐牢.
他只是想坐牢.
不是.
死了就是甘心.
其他門徒在爭論誰偉大.
彼得願意跟隨主耶穌.
直到最後.
最終極的那一步.
這樣說完之後.
才逼耶穌連不說的那句也說出來.
耶穌說.
彼得.
我告訴你.
今日雞還沒有叫.
你要三次不認我.
其實是那個女孩.
那個婢女.
在大祭司的後院.
多多事幹.
走到彼得那裡.
你的口音和裡面那個一樣.
認不認識她?.
給你和我上班就算三年五年.
我們都有七個不同的方法可以把她騷擾出去.
你認不認識?.
認識?.
是啊是啊是啊.
今晚月光多好.
我們就換話題.
你的口音也像.
你認不認識她?.
你是她妹妹?.
反手為公.
用不著發毒誓不認識耶穌.
拍心口.
跟你一起死.
也是甘心.
反過來.

$^{601}$三次.
不敢認耶穌.
福音書記載.
耶穌在那個時候.
轉過身來.
看彼得.
有沒有印象?.
可能他發誓的聲音太大了.
耶穌在裡面聽不到.
你以為耶穌轉過身來看彼得.
是怎麼看他的?.
有火在眼裡燒死他.
沒有.
彼得出去.
同哭.
你回頭而後.
彼得沒有走.
彼得和猶太最大的分別.
不是一個賣主一個愛主.
其實兩個都出賣主耶穌.
不過一個不肯回頭.
出去吊頸蛇.
這是舊約聖經歷史典故.
你以為是中國文化典故.
民間宗教要報仇的.
就去吊頸蛇.
舊約聖經不是這樣看.
舊約聖經是說大衛的好朋友.
阿希多芬.
加入了亞沙隆的叛變.
到他知道失敗之後.
他沒有面子回去見他的好朋友大衛.
他就回到自己的家.
吊頸自殺.
彼得沒有.
彼得頭暈頭轉.
回到門徒當中.
和門徒一起.
直到耶穌復活.
你回頭以後.

$^{641}$堅固你的弟兄.
彼得這個經歷一生難忘.
Alert Discipline.
是失敗之後.
他能夠回頭.
然後他才明白.
原來人人都會軟弱跌倒.
有時你覺不覺得在教會裡面.
那些沒有遲到過的弟兄姊妹.
不明白為何其他人會遲到.
你早半個鐘頭出門就行了.
總是多多藉口.
你知道嗎.
我的寶寶很小.
早九個小時出門.
他都會裝大事.
你不明白.
你明白了.
你就懂得怎樣幫他.
來到.
趁現在不阻礙人家.
慢慢後面有位子坐.
請寶寶去寶寶房.
你去體貼你的溫柔.
你的鼓勵.
就能夠堅固軟弱的肢體.
Alert Discipline.
原諒我這樣說.
和吃朱古力無關.
是我們生命當中.
一個失敗跌倒.
然後重新起來.
成為身邊其他人的幫助.
有了節制.
現在明白了.
甚麼是節制.
有了Alert Discipline.
要加上忍耐.
Peterson翻譯是Passionate Patience.
一個滿有熱誠的忍耐.

$^{681}$忍耐甚麼.
在彼得的生命當中.
他要等的是主耶穌回來.
所以彼得後書說得很好.
你又記得.
我一讀你差不多都會背.
親愛的弟兄.
有一件事你們不要忘記.
就是主看一日如千年.
千年如一日.
彼得後書第三章第八節.
主所應許尚未成就.
有很多人以為是擔言.
其實不是.
彼得說是寬容.
不願一人沉淪.
乃願人人悔改.
所以彼得是怎樣等主耶穌回來.
他不是繞著手.
連你的身體語言都出來了.
Passive.
看看吧.
沒有事可以做了.
不是.
彼得是拼命傳福音.
信從神.
不信從人.
是應當的.
說的是傳福音.
保羅的說法.
務要傳到.
無論得時不得時.
Passionate Patience.
不是一個消極.
咬實牙關.
忍下去.
不知要等到何時.
有人以為擔言.
有人甚至以為沒有.
只是這樣說.

$^{721}$彼得說其實是神的寬容.
這個寬容只有一個目的.
人人悔改.
上帝不是想有任何一個人沉淪.
所以你等他這個日子.
你要充滿熱誠.
傳福音.
彼得學到.
所以他說有了Alert Discipline.
不只是回教會見顧弟兄.
還有Passionate Patience.
等候主回來.
傳福音.
然後有了Passionate Patience.
有了忍耐.
要加上乾敬.
Reverent Wonder.
Peterson的翻譯.
Reverent我們知道.
這個就是敬畏.
但他加上Wonder.
Wonder是什麼.
你認識那個形容詞.
Wonderful.
奇妙的乾敬.
是什麼意思.
你回到彼得的生平裡.
你就明白.
那天彼得在屋頂靈修.
祈禱.
他每天都祈禱.
天上有異象.
有塊布掉下來.
你會說的.
我隨便叫一個.
你上來都能說得出.
《少林寺傳》重複三次記載.
裡面所有的動物.
都是不潔淨的.
但天上卻有聲音說.

$^{761}$起來載了氣.
彼得回答.
主啊.
萬萬不可.
你叫的.
他是主.
你知道聲音是從天上來.
為什麼你say no.
其實彼得理直氣壯.
因為是你吩咐的.
上帝你忘記了.
利美記.
生命記.
聖經教導的.
怎麼可以出意反意.
其實.
上帝可以在異象當中.
捏住彼得的喉嚨.
捏開他的口.
塞一條蜥蜴進去.
吃吧.
上帝叫你都不吃.
沒有.
《史獨行傳》記載.
這個異象重複了三次.
那就沒有了.
彼得正在狐疑之間.
樓下就敲門.
誰來.
哥利樓.
白夫長的僕人.
彼得立刻醒水.
不是說他吃不吃.
是說這個外邦人.
這個外邦人是誰.
我們在香港現在就明白.
駐港解放軍的中將.
在昂船洲.
找一架軍車.
什麼Z字頭那些.

$^{801}$來到你教會門口.
恭請主任牧師.
去解放軍軍營開報道會.
嘩 你敢不敢去.
要被消失的.
彼得很厲害.
所以他帶了幾個弟兄一起去.
不是做保鏢.
是見證這件事.
結果回到教會.
大家就質問他.
為什麼要去外邦人那裡.
還要去羅馬人那裡.
還要去軍人那裡.
還要和他們私洗.
連串問題.
彼得沒得答.
主啊 萬萬不可.
他也是這樣說的.
但他見到的是聖靈奇妙的工作.
Referent Wonder.
上帝今天在我們當中所做的.
往往是令我們出乎意料之外.
不是我們計劃的.
沒有budget的.
不知道為什麼事情又會這樣發生.
你能不能夠跟從主的帶領.
來學習.
敬畏祂.
走一條你可能不知道的路.
Referent Wonder.
彼得很深刻的經歷.
不過還沒完.
還有有了乾淨.
要加上愛弟兄力甚.
這個字你和我也認識.
Philadelphia.
有沒有聽過美國費城.
新約聖經有譯出來的.
啟示錄七個教會.

$^{841}$譯做菲拉鐵菲.
譯音.
不過美國我們從來都沒有譯菲拉鐵菲.
我們費事譯這麼長就費城.
Philadelphia.
這個是希臘文.
兩個字加起來.
前面Phileo是愛.
後面Dolphus是兄弟.
兄弟之愛.
彼得在說誰.
如果你讀到彼得後書.
最後結束第三章.
我讀給你聽.
第十五節.
並且要以我主長久忍耐為得救的恩由.
就如我們所親愛的兄弟保羅.
親愛Phileo兄弟Dolphus.
彼得心裡想的是保羅.
兩兄弟當然是相親相愛.
士道行傳兩大男主角.
一至十二章是彼得.
十三到二十八章是保羅.
打死不離親兄弟.
錯了.
彼得是誰.
保羅又是誰.
你忘記了嗎.
五旬節.
彼得站起來高聲說.
十一個使徒站在他身邊.
三千人悔改.
五千人悔改.
教會由一百二十人祈禱.
突然變成八千人的教會.
請問保羅在哪裡.
保羅還在停車場扔石頭.
對不對.
他還反對基督教.
兩個不能比的.

$^{881}$是不同輩份.
對不對.
容許我們這樣說華人教會.
我們這一代還應該認識.
藤近輝牧師來貓奔開培靈會.
有個小孩還站在外面.
那裡罵髒話.
這裡是彼得.
那裡是保羅.
懂了嗎.
但保羅是一個怎樣的保羅.
如果你讀加拿大書.
保羅曾經記載.
發生過一件這樣的事.
你記得嗎.
加拿大書.
第二章.
十一節.
保羅說後來機發來到安提阿.
安提阿就是保羅和巴拿巴牧羊的教會.
保羅當時是.
Youth Pastor.
不是Outreach Pastor.
他不是Senior Pastor.
巴拿巴才是Senior Pastor.
懂不懂得讀聖經.
請了彼得來講道.
彼得是耶路撒冷教會領袖.
世界華福總幹事.
你懂得了.
保羅說我見到他有可責備之處.
我就當面抵擋他.
因為他們所做的不正確.
不符合福音真理.
不符合當面的抵擋.
保羅說我在眾人面前.
就對機發說.
容許我這樣做比喻.
你又會明白一點.
不過這樣說好像很驕傲.

$^{921}$請你原諒.
院長來到大家中間.
假如你看到我有不事之處.
你會怎樣做?.
你不會出聲.
你背後說一件事.
或者你找一個童工出來.
拉院長去一個密室.
細細聲去說.
你試試你當眾走出來.
站在這裡.
當著眾人面前.
這樣來說話.
保羅是一個這樣的人.
這口氣怎能接受?.
以後不來Malvern了.
不要來.
但彼得沒有這口氣.
彼得從來都沒有這口氣.
所以他寫信到最後.
他怎樣稱呼保羅?.
就如我們所親愛的弟兄保羅.
我很喜歡Eugene Peterson.
在他的翻譯中.
他介紹西門彼得.
Peterson.
彼得的兒子介紹彼得.
Peterson是這樣說.
我讀給你聽.
不需要翻譯.
他很自然走出來就是領袖.
彼得得到童工中間的尊敬.
因為他本身已經令你心服口服.
他當然是第一個.
華人教會排名不分先後.
不過總是藤近輝牧師走先.
大家都佩服他.
彼得就是初期教會裡.
大家都佩服的一個領袖.
真誠的禱告.

$^{961}$Bold healing and wise direction.
confirmed the trust placed in him.
The way Peter handled himself.
in that position of power.
is even more impressive.
than the power itself.
He stayed out of the center.
did not wield power.
maintains a scrupulous subordination to Jesus.
Given his charismatic personality.
and well-deserved position at the head.
he could easily have taken over.
using the prominence of his association with Jesus.
to promote himself.
that he did not do it.
Given the frequency with which spiritual leaders.
do exactly that.
is impressive.
Peter is a breath of fresh air.
彼得很容易.
可以將注意力.
眾人的愛戴.
放在他自己一個人的身上.
Peterson說.
其實很多所謂屬靈的領袖.
都很容易會這樣做.
但你讀新約聖經.
從來沒有見到彼得.
有這樣做過.
彼得真是一股清新.
Breath of fresh air.
應該怎麼翻譯呢.
清流又好像很文鑄鑄.
你明白我的意思了.
因為大家都明白英文.
愛 帝 興.
Philadelphia這個字.
不是一個抽象的教導.
而是彼得生命當中.
一個很深刻的弟兄.

$^{1001}$給他學習.
一個很深刻的功課.
還沒完.
還有最後彼得說.
有了愛弟兄的心.
又要加上什麼.
你本聖經.
眾人的心旁邊有沒有小點.
有的.
如果你的iPhone沒有.
就不要用iPhone看聖經了.
就要拿一本和合本出來.
因為很重要的.
和合本告訴我們.
旁邊加小點是什麼意思.
本來是沒有的.
是翻譯的人加上去.
有時候加錯了怎麼辦.
所以彼得寫到最後.
有了Philadelphia弟兄的愛.
最後要加上愛.
愛 阿加培歐.
你認識這個字嗎.
阿加培 認識的.
耶穌就是用這個字去問彼得.
在他復活之後.
在提比利亞海邊.
三次問.
約翰的兒子西門.
你愛我比這些更深.
彼得不敢用這個字回答耶穌.
彼得就是用Phileo兄弟的愛.
手足之情.
回答耶穌.
你知道我看你好像兄弟一樣.
耶穌第三次.
就用兄弟的愛這個字去問彼得.
約翰福音記載.
彼得憂愁.
但他仍然不敢.

$^{1041}$他只是深深記得.
耶穌是這樣問他.
耶穌用這個愛字問他.
到何時他回答主這個問題.
彼得憂愁.
來到最後.
他用這個字.
最後要加上的.
阿加培歐.
加上愛.
我估這裡彼得不是在說愛眾人.
所以我和彼得有不同的意見.
他說兄弟.
溫暖的友善.
之後要加上.
對人的慷慨愛.
我估彼得在說的是對主.
我們知道彼得是殉道而死.
如果你跟旅行團去羅馬.
導遊會帶你去看彼得.
本來可以逃脫.
走到一半路.
他回頭.
自投羅網.
怎樣證明他要回頭.
導遊會帶你看石頭上有個腳印.
那個不是彼得的腳印.
那個是主耶穌向他顯現.
完了後.
一陣一陣升上天堂.
留下的腳印.
你信不信.
你笑了.
我不用再說了.
不過另外一個傳統.
告訴我們.
彼得是怎樣死的.
倒轉頭釘十字架死的.
因為他說我不配.
我不是耶穌.

$^{1081}$他釘十字架.
我要倒轉頭釘.
你信不信.
羅馬的刑場.
有白夫長負責.
這個彼得就說要倒轉頭釘.
那個長老就說.
我打側左邊釘.
另一個執事就向右邊釘.
刑場上的十字架.
東歪西倒.
到最後長官來巡視的時候.
誰會被釘十字架.
No way.
在服刑的時候.
彼得是誰.
是一個號碼.
24601.
他完全沒有說他可以怎樣死.
所以教會是這樣.
有個傳說留下.
未必是真的.
但他殉道卻是事實.
他可以逃脫.
柴玲,王丹,吾爾開希.
全部都可以逃脫.
彼得為何不能逃脫.
你明白嗎.
但他沒有逃脫.
因為他記得.
耶穌跟他說了什麼.
耶穌問完他之後.
耶穌跟他說.
後來你最後要被人綁住.
帶到你不想去的地方.
彼得又立刻問問題.
想得很快.
旁邊的門徒.
你記得耶穌怎樣回答他嗎.
我要這個人等到我回來.

$^{1121}$與你何干.
你跟從我吧.
其實是同一件事.
第一次下網打魚.
兩隻船滿到要沉.
耶穌跟他說你跟從我.
這次又是153條大魚.
耶穌說你跟從我吧.
請問第一次容易還是這次容易.
年輕的時候.
呼召,決志.
撇下所有跟從主.
我什麼都沒有,舉手.
今天我什麼都有.
願不願意為主.
信道.
想過了.
跟神學院談過規則.
怎樣退休.
你明白我的意思嗎.
最後彼得說.
不是愛人如己.
是愛主.
唯有這樣你能夠跟從主.
直到你生命最後的一步.
不敢要你背.
不過你有印象嗎.
基本的信心.
良好的品格.
淑齡的洞察.
警醒的操練.
熱誠的忍耐等候.
還有呢.
敬畏神.
但是預備會有出乎意料的結果.
愛那個你最愛不下.
最啃不下的弟兄.
最後愛主.
這是彼得的一生.
然後他說.

$^{1161}$你們如果充充足足有了這幾樣.
就好了.
是他最後寫下.
給教會每一個弟兄姊妹.
心裡面的說話.
我們同心一起祈禱.
懇求真理的聖靈帶領我們.
不單止明白聖經裡的說話.
讓我們更加願意學習進行.
今日我們的生命來到哪一個階段.
求主光照鑑察.
又提醒我們繼續追求長進.
有了信心.
加上德行.
有了德行.
加上知識.
有了知識.
加上節制.
有了節制.
加上忍耐.
有了忍耐.
加上虔敬.
有了虔敬.
又再加上愛弟兄.
有了愛弟兄.
最後還要加上愛主.
求你幫助我們.
我們恭敬.
將你的說話藏在我們的心中.
誰聽我們的祈禱.
奉耶穌基督聖名救.
我們請敬拜隊帶領我們.
多謝各位.
弟兄姊妹請起來.
謝謝大家收看,下次見!.
(字幕製作:貝爾).
\newpage



\section{}
\label{sec:EtbCz7LXbWY}
\textbf{中神澳紐培靈講座 2019}
\newline
\newline
連結: \href{https://youtube.com/watch?v=EtbCz7LXbWY}{\texttt{ https://youtube.com/watch?v=EtbCz7LXbWY}} ~~~~ 語音日期: 2019-07-25 
\newline
\newline
\hyperref[sec:jGn4gOBS1HA]{\small{< < < PREV SERMON < < <}}
~
\hyperref[sec:index]{\small{[返主目錄]}}
~
\hyperref[sec:c1_Cz96z2yY]{\small{> > > NEXT SERMON > > >}}
\newline
\newline
$^{1}$非常歡迎大家來到香港中國神學研究院.
第一屆的澳紐培靈講座.
也很歡迎在澳洲六個城市.
應該攝影機在那裡 大家好.
十個的堂會的直播站.
我們一起來可以在今天中頂節目.
一起在主的恩典裡面.
去共享這個豐盛的數靈現象.
請我們恭敬的喜立.
我們為了今天的晚上聚會來祈禱.
愛我們的天父上帝.
我們實在要以我們心裡面最深的深處.
發出的感謝來獻上感恩給你.
天父感謝你 你賜下負擔,意象.
讓香港中國神學研究院.
他們有籌辦澳紐培靈講座這個方向.
經過了一定的時間的籌備,測試.
甚至帶著一些的限制.
主要是讓今天晚上這個聖功我們能夠舉行了.
我們仰望你的恩典.
是絕對真是由神你自己所掌管的.
是一個的恩典.
雖然在試播的過程有起起跌跌.
或許今天的天氣也都很大風.
但是這些真的就不是等於主你不聽禱告.
我們向好的天氣祈禱.
但是神你掌管一切.
你就將萬事萬物.
真是在你自己的裡面配搭得合宜.
我們今天晚上仍然有一個這麼安全的地方.
這麼好的地方來到聚會.
我們恭敬獻上感恩.
你又將你自己的中心的僕人帶到我們當中.
我們感謝你.
李院長他帶著身體的軟弱.
他依然是忠心的圍籬奔跑.
李院長他在生命的當中.
不停的用他自己的生命來見證神的恩典.
和他所出口的毒.
我們感謝你.

$^{41}$將今天晚上的時間恭恭敬敬的教堂.
我們思想到因為主耶穌的緣故.
我們眾肢體在不同的地方.
我們能夠合一.
因為主耶穌的救恩.
因為聖靈的同在.
我們能夠這樣同心.
我們真的實在只有學習.
將我們的生命獻給主.
來回應你的大愛.
我們感謝,讚美你.
今天我們靠著聖靈的引導.
帶著聖靈所賜的悟性.
我們來聽聖餐你自己寶貴的道.
我們又靠著神你自己的恩典.
我們要學習將聽到的走出來.
就求你繼續這樣帶領私恩.
將榮耀和讚都歸還給主耶穌.
奉耶穌基督特赦的名祈禱.
阿們 業禮.
我將以後的時間恭敬的交給李思敬博士.
各位弟兄姊妹主內平安.
(主內平安).
都要同時代表香港中國神學研究院.
向澳洲好幾個不同的城市的弟兄姊妹問候.
雖然我們見不到他們.
不過在主的愛當中.
今天晚上我們感謝主.
有這樣的機會.
可以一同在神的說話上面.
來彼此激勵.
忠臣舉辦海外的培靈講座.
今年是第四年了.
過去三年.
我們都只是在美國和加拿大.
選一個城市.
然後四個市區.
二十多個教會一起直播或者轉播.
第一年我們在多倫多.
第二年在休斯頓.

$^{81}$第三年就在艾滋賓頓.
今年其實是同一天.
不過北美洲的市區比我們遲很多.
奧克蘭是不是應該是最早慶祝元旦的?.
是不是?.
所以我們是最先.
其實同樣6月9號.
今年第四年.
我們在西雅圖.
就有第四次的北美忠臣的培靈講座.
感謝主同工提議.
不如在澳洲紐西蘭開始了.
院長打頭陣.
所以今天晚上的聚會.
真是我們倚靠神的恩典.
預備的時候.
其實沒有想過離開香港.
在這個週末.
我的心是很沉重的.
如果弟兄姊妹都關心香港.
就明白到其實一個很簡單的條例.
但是引起劍拔弩張的情況.
實在是非常之.
怎麼說呢?.
沒有想過.
明天不單止香港會有遊行.
你有沒有看到全世界.
除了奧克蘭.
澳洲,北美,歐洲都有遊行.
我們真是要在神的面前.
學習怎樣祈禱.
所以本來預備今晚的信息.
2月的時候已經祈禱準備.
沒有想過在這樣的處境當中.
來到宣講.
不過上帝有祂的心意.
所以今晚我們一同來讀的經文.
有好幾段的.
你有聖經可以翻開或者點擊.
很方便.

$^{121}$如果你沒有聖經不要緊.
我比較傳統.
我沒有預備PowerPoint.
2000年來教會信徒都是聽神的話.
所以比較傳統的禮儀.
在讀聖經之前.
讀的人會宣佈請聽神的話.
第一段我們會讀的經文.
是大家很熟悉的.
我一讀你差不多都會背.
《馬太福音》第七章第七節第八節.
如果你找到我們可以一起同聲讀出.
如果你沒有聖經請你聽我們讀出.
耶穌的教導.
《馬太福音》第七章第七第八節.
「你們祈求就給你們,尋找就尋見,叩門就給你們開門..
因為凡祈求的就得著,尋找的就尋見,叩門的就給他開門.」.
《馬太福音》從第五章到第七章是登山部分.
我們很熟悉的八福.
雖然很少人背得出.
不過講到祈禱.
主耶穌教導祈禱的經文.
主禱文.
我們差不多每一個信徒在教會裡面都背得出.
當然我們還記得耶穌提醒我們.
「未求已先,天父早已知道.」.
「早」字真是好.
不用我們說的.
好像父母親一樣.
小朋友有什麼需要.
不用開口的.
早已經預備好了.
主耶穌提醒我們.
不需要像外邦人.
Eugene Peterson 的翻譯.
就是不認識天父的世人.
他們以為說話要多.
這樣才蒙誰聽.
今天我們也是.
我們以為人多.

$^{161}$必蒙誰聽.
我們以為時間夠長.
就蒙誰聽.
不是.
主耶穌說「你們不可效法他們.」.
耶穌教導祈禱的經文.
我們刻骨銘心.
我們記得.
同拜偶像.
民間宗教很不同.
「心誠則靈.」.
這個是中國民間傳統教導.
所以讀書的時候.
臨到考試才靈修.
你有沒有錯誤的神學思想.
平時反而懶懶散散.
沒什麼所謂.
需要上帝幫助.
就滴起心肝.
不是的.
不過同樣在登山補訓的經文.
也是我們很熟悉的這一句金句.
「你們祈求就給你們.」.
「尋找就尋見.」.
「叩門就給你們開門.」.
「凡祈求的就得著.」.
「尋找就尋見.」.
「敲門就開門.」.
很熟悉的經文.
不過我不知道你們.
我自己每一次讀的時候.
我都會心裡面有一點問題.
有一個問號.
「真的這麼好?」.
「真的祈求就得著?」.
「那我就祈了.」.
「哎呀,又不行.」.
於是有人就提醒.
《馬國書》在哪裡說.
「你們祈求得不著是網球.」.

$^{201}$不是打網球那個網球.
是網稱「耶和華的名」的網球.
那我怎知道我何時才是「祈求就得著」.
何時是「網球就得不著」.
所以這句金句是很熟悉.
但是我要承認.
我始終都不明白.
耶穌真的給了一個這樣的應許.
給他的門徒嗎?.
「你們祈求就得著」.
直到我讀《紐約聖經》.
我才恍然大悟.
明白原來很多主耶穌的教訓.
甚至祂自己的言行.
其實真的好像祂說的.
祂所做的,祂所教導的.
早已經記載在聖經裡面.
祂說的聖經不是《紐約聖經》.
因為在耶穌在生的年代.
有沒有《紐約聖經》?.
沒有的.
那耶穌那本是甚麼聖經?.
那就是《紐約聖經》.
不過我們以為不是.
耶穌熟悉的,耶穌教導的.
耶穌提醒門徒.
祂所說的,祂所做的.
已經記載在聖經上.
譬如我們很熟悉的十加七言.
你記得嗎?.
耶穌第一句.
四本福音書分別記載不同的三句,一句.
次序是教會每年壽苦節自己定出來的.
但通常我們說一上十字架.
第一句是「父啊赦免他們.
因為他們所作的.
他們不曉得」.
你記得嗎?.
兩本福音書一起記載.
中間那句「我的神,我的神.

$^{241}$為甚麼離棄我」.
是《紐約聖經》.
是C篇22篇第一節.
所以耶穌是否真的在那裡叫喊.
上帝離開了他.
我們要回去讀C篇22篇.
即是「我們在天上的父」.
你知道其實是背著《主禱文》.
「起初神創造天地」.
你知道是背著《創世紀》第一章.
「耶和華是我的目者」.
你知道是C篇23篇.
宣道會的弟兄姊妹不知道知不知道.
「我信上帝全能的父.
創造天地的主」.
我們背著的是《使徒信經》.
所以你要整篇讀完.
你才知道第一句其實是甚麼意思.
C篇22篇不是說上帝離棄他的百姓.
第三節已經說上帝在以色列的讚美當中.
坐著為王.
另外耶穌在十字架上說過.
「我喝了」.
《約翰福音》記載說明.
「為要應驗經上所記的」.
是C篇的說話.
我聽過有牧師解釋.
耶穌為我們上進地獄的火.
怎樣證明?.
「我喝了」.
聽起來是頭頭是道.
但原來C篇所說的.
未有拉去到地獄那麼遠.
它只是說他的仇敵在旁邊戲弄他.
「我喝了」.
他們不是拿水來給主耶穌喝.
是拿甚麼?.
是拿不能解渴.
反而令他更加辛苦的情況來戲弄他.
還有一句我們很少知道.

$^{281}$留意原來耶穌也是用C篇來祈禱.
「父啊,我將我的靈魂交在你手裡」.
只是隨便舉例.
耶穌說的比喻.
很熟悉的.
很像馬利亞人的比喻.
原來是歷代寺的歷史記載的典故.
很多這些情況.
我們有時真的疏忽了.
「你們祈求,就給你們;.
尋找,就尋見」.
原來耶穌也是在背誦金句.
在哪裡?.
在哪裡的金句?.
耶利米書29章.
11節到第14節.
如果你有聖經我們一起讀.
如果沒有,你留心聽上帝的話.
「耶和說:我知道我向你們所懷的意念.
是賜平安的意念,不是降災禍的意念.
要叫你們沒後有事,要求你們要有心.
要求你們要有愛和的意念.
要叫你們沒後有子望.
你們要呼求我,禱告我,我就應允你們.
你們尋求我,若專心尋求我,就必尋見.
耶和說:我必被你們尋見.
我必將你們從各國中和我所趕你們到的各處招聚了來.
又將你們帶回我使你們被擄掠離開的地方.
這是耶和說的」.
我們要讀長一點.
因為先知記載上帝的說話.
在某個公式開始的時候.
祂說「耶和說」.
結束的時候,祂會說「這是耶和說的」.
這個在古代的世界,我們稱為一個「使者公式」.
就是做使者的.
祂傳達祂主人的信息.
祂開口就會說「我的主人」.
結束的時候,祂會說「這是某某某說的」.
這是舊約聖經研究先知書一個很清楚的結論.

$^{321}$在這一句上帝說的話,你看到吧?.
耶和說說什麼?.
中間那兩句「你們呼求我,禱告我.
我就應允你們,你們尋求我.
專心尋求我,就必尋見」.
你說還差敲門嗎?.
你讀聖經,你會捉這些字失嗎?.
會的,我沒有睡覺的.
第三句是耶穌補上去的.
然後啟示錄,引用.
「我在門外敲門,若有聽見我聲音就開門的.
我要進去與他坐直」.
是的,但前面兩句「祈求就得著.
尋找就尋見」.
聽到嗎?.
很清楚耶穌在背誦先知書.
上帝親口的應許.
於是我們就要問.
打爛紗盤問道督.
為什麼上帝要這樣應許?.
與誰應許?.
在什麼情況之下上帝會這樣說?.
就要從耶利米書第28章第一節開始讀起.
28,29章是一氣呵成的記載.
耶利米書第28章第一節說.
猶大王西底加登基第四年五月.
西底加是南國猶大最後的一個君王.
西底加王第十一年.
就是巴比倫軍隊入城耶路撒冷亡國.
被擄的時候.
公元前586年.
但這個時候還沒到.
耶利米書記載的是西底加登基第四年的五月.
西底加是怎樣登基的?.
如果你熟悉舊約的歷史.
你知道是因為巴比倫王來到耶路撒冷.
不過沒有滅到猶大國.
只是將當時在位的君王擄去.
並且帶走了一幫貴族,一幫領袖,一幫公將.
要耶路撒冷不可以造反.

$^{361}$西底加是我們中國歷史稱為「傀儡」皇帝的事實.
擄去的你記得岳飛的《滿江紅》.
靖康始猶未說.
臣子恨何時滅.
中文班背不背?.
不背的了.
是以前在香港讀書才會背的.
其實這些情況很相似.
第四年五月發生什麼事?.
欺騙人壓索的兒子先知哈那尼亞.
在耶和華的殿中當著祭司眾民.
對耶利米說.
「萬君知耶和華,如此說.
我已經截斷巴比倫王的鴨.
兩年之內.
我會將巴比倫王利布甲尼撒從這地掠到巴比倫的器皿.
就是耶和華殿中一切器皿都帶回此地.
我又要將猶大王約瓦敬的兒子耶哥尼亞.
和被擄到巴比倫去的一切猶太人.
都帶回此地.
因為我要截斷巴比倫王的鴨」.
說完了,這是耶和華說的.
先知哈那尼亞公開宣佈.
兩年之內被擄的猶太人就要被釋放.
就要回耶路撒冷.
開始要組織籌備小組了.
怎樣歡迎他們?.
西底階要退位.
讓本來被擄走的君王繼續統治.
還是怎樣?.
要開內閣談了.
好像說Teresa May今天辭職了.
要下台了.
要選舉了.
很多事情要準備.
兩年之內.
被擄到利布甲尼撒王到巴比倫的器皿.
為甚麼是器皿?.
大家知道耶路撒冷的聖殿是沒有神像的.
巴比倫的軍隊到達每一個民族.

$^{401}$努力的第一件事是將他的神帶回到巴比倫.
因為你的神輸給我的神.
所以我打贏你.
來到耶路撒冷沒有.
耶路撒冷的聖殿是沒有像的.
從《春愛及記》開始記載.
所以只能將所有的器皿擄掠了.
現在先知哈拿尼亞站起來說.
兩年之內.
嘩.
這個是好消息.
這是大件事.
先知耶利米如何回答哈拿尼亞.
你要讀聖經不要自己作.
假先知?不是.
耶利米第一句的回應記載在28章第6節.
他說甚麼?.
阿們.
懂不懂解希伯來文的音譯?.
誠心所願.
祈園禱之後有些教會說阿們.
有些教會說誠心所願.
我去港島經常都搞錯.
但願如此.
願你所說的.
真是這樣發生.
因為所有的猶大人聽到這個消息.
都必定興奮.
神的救贖臨到.
為甚麼不說阿們?.
大家心裡所渴想的.
為甚麼不回應?.
為甚麼不歡喜?.
所以耶利米首先說.
願耶和華如此行.
願耶和華成就你所預言的話.
將聖殿器皿.
一切被擄的人.
從巴比倫帶回此地.
耶利米的心的確.

$^{441}$和其他百姓一樣.
他很想這件事真是這樣發生.
哈拿尼亞是先知.
28章第一節介紹他的時候.
他的確是.
不知道是否忠臣畢業.
不過有畢業證書的.
不是路邊突然間.
一個無名氏這樣站出來的.
不過耶利米還沒說完.
所以第七節.
耶利米繼續.
他說:然而.
我向你和眾民以中所要說的話.
你應當聽.
從古以來.
在你我之前.
所有的先知.
他們說預言.
他們是說甚麼的?.
他們是說.
征戰,災禍,瘟疫.
都是不好聽的.
先知如果說平安.
是要到話語成就.
人才可以認識他.
真是耶和華所猜來的.
耶利米說了歷史.
我很希望你所說的是真的.
不過歷史上有這樣的記錄.
先知大概不是說平安的訊息.
他是說災難的.
耶和華派來的先知.
如果真是說平安的訊息.
要何時才可以肯定呢?.
要事情成就才可以這樣肯定.
所以耶利米的態度是怎樣的?.
不知道.
暫時我不知道.
所以第十節.

$^{481}$哈拿尼亞就做了一件事.
他把耶利米頸上的額拿下來.
截斷了.
當眾做這件事.
然後又當著眾民說.
耶和華如此說.
兩年之內.
我必照樣從列國人的頸上.
截斷巴比倫王尼布格尼薩的額.
宣佈完之後.
聖經記載.
於是才知道耶利米就走了.
沒有辯論.
沒有兩個人互相指責.
你是假的.
耶利米說完了.
耶利米說我不知道.
不過我也很想事情是這樣發生.
耶利米的意思是說等兩年.
你說兩年之內.
我們大家等.
結果如何?.
不用等兩年.
十二節.
哈拿尼亞這樣做完之後.
把耶利米頸上的額截斷.
耶和華的話就臨到耶利米.
說什麼?.
你去告訴哈拿尼亞說.
耶和華如此說.
你截斷木的額.
現在要換上的是鐵的額.
因為萬君之耶和華.
以色列的神如此說.
我已經將鐵的額加在這些國的頸上.
使他們服侍巴比倫王.
尼布甲尼撒.
他們總要服侍他.
我甚至將田野的走獸.
都賜給尼布甲尼撒王.

$^{521}$於是先知耶利米.
就對先知哈拿尼亞說.
哈拿尼亞.
你應當聽耶和華母差遣你.
你竟然使這百姓倚靠謊言.
所以耶和華如此說.
哈拿我要叫你去世.
你今年必死.
因為你向耶和華說了叛逆的話.
二十八章十七節最後的結束.
這樣先知哈拿尼亞當年七月間就死了.
你說每次都是這樣就清楚了.
是吧?.
你找我們這班小信徒來玩而已.
兩個大先知說的信息南轅北轍.
你叫我聽哪個好呀?.
你有沒有這樣的感覺?.
經常都有的.
經常都覺得為甚麼這麼混亂.
首先我們平心靜氣來問清楚.
西底加王第四年.
耶利米已經在耶路撒冷.
蒙召作先知三十三年.
約西亞王十三年蒙召.
一直到西底加王十一年.
耶路撒冷覆亡.
耶利米一直沒有離開過耶路撒冷.
三十年的牧師你認識他嗎?.
我這樣過境.
上次三年前來過一下.
你不認識我的.
我穿著西裝打著領帶.
每個人都是這樣的.
中國人說甚麼醫官.
我不說下去了.
你認識的這些成語.
知人口面.
你也會接駁下去的.
但一個牧者.
如果在弟兄姊妹當中.

$^{561}$服侍了三十年.
你說你不認識他?.
你怎麼會不認識他?.
你認識他的太太.
認識他的子女.
認識他的父母.
我們香港人這樣說.
你起了他的底很久了.
他騙不了你的.
他的人是怎樣的.
對不對?.
所以不只是他說甚麼.
我們不要這麼快就埋怨.
上帝呀.
你們兩個先知在這裡.
各有各說的.
我聽誰的好.
耶利米先知是耶路撒冷的百姓.
認識了三十年的先知.
今年是2019年.
你的牧者099989.
剛剛六四三十年了.
就來到奧克蘭的了.
你又剛剛來到的了.
大家一起三十年打天下.
甚麼不認識?.
他動一下尾巴.
我也知道他怎樣.
他不開心的時候會怎樣.
我就知道.
他開心的時候怎樣.
我又知道.
他也是這樣認識你.
怎會不知道耶利米是怎樣?.
耶利米是真的先知.
還是假的先知.
人人都知道的.
三十年的track record.
是不是?.
所以不要這麼快就去.

$^{601}$我們應該反過來感謝主.
不是每次兩個先知.
每人說一樣.
結果另一個兩個月就死了.
嚇死你.
如果教會這樣發生事情.
是不是?.
上帝有憐憫.
整本舊約的上帝才這樣.
兩個月就死了.
你忘記了新約時代行主發生甚麼事.
是不是?.
你也會說了.
即死.
上帝其實不是分身舊約的.
一個上帝來的.
是不是?.
是呀.
有些時候很嚴厲的.
先知不要隨便說話.
因為他是先知.
這是耶利米書28章.
不過事情沒有停在這裡.
消息因為發生在耶路撒冷的聖殿.
是當著全體的群眾.
哈拿尼亞這樣宣佈.
你可以想像.
聽見的人一定會做一件事.
就是他們會告訴被擄到巴比倫的親戚.
家人,朋友.
兩年了.
上帝開口.
透過他的先知哈拿尼亞宣佈.
兩年之內你們就回來了.
一定的.
這個消息不會只是停在耶路撒冷.
兩個月等的過程.
信息就是這樣傳過去巴比倫.
你按都按不住的.
當年沒有Facebook.

$^{641}$沒有WhatsApp.
也沒有Email.
也沒有微信.
沒有.
托人五個月送信去到巴比倫.
消息已經傳開.
所以29章接續記錄的.
就是耶利米寫信給巴比倫的猶太人.
先知耶利米奉上帝的名.
對一切被擄到巴比倫的人.
怎樣說呢?.
你有聖經在手.
29章第5節到第10節.
剛才我們讀的是11節到14節.
現在我們讀它的上文.
這樣就齊全清楚很多了.
大家找到聖經嗎?.
我們一起讀.
你們要蓋造房屋住在其中.
栽種田園吃其中所產的.
娶妻生兒女.
為你們的兒子娶妻.
使你們的女兒嫁人.
生兒養女.
在那裡生養眾多不至減少.
我所使你們被擄到的那城.
你們要為那城求平安.
為那城禱告耶和華.
因為那城得平安.
你們也隨著得平安.
萬君知耶和華以色列的神如此說.
不要被你們中間的先知和占卜的誘惑.
也不要聽信自己所作的夢.
因為他們託我的名對你們說假預言.
我並沒有差遣他們.
這是耶和華說的.
耶和華如此說.
為巴比倫所定的七十年滿了以後.
我要眷顧你們.
向你們成就我的恩賢.

$^{681}$使你們迎回此地.
上帝說什麼?.
上帝透過耶利米先知說了好幾樣東西.
第一.
引用《創世紀》.
你很熟悉的另一個典故.
什麼?.
生養眾多.
看到嗎?.
在第六節.
在那裡生養眾多.
為什麼他們不生養眾多?.
因為那裡是巴比倫.
那裡是外邦.
不知道留多久.
環境很惡劣.
其實自己的性命都朝不保夕.
怎麼會蓋造房屋.
栽種田園.
娶妻生兒.
又為自己的兒女婚嫁.
沒有了.
所以耶利米第一件事情.
提醒被擄在巴比倫的人質.
猶大人生養眾多.
第二.
耶利米告訴他們.
你們要為那城求平安.
第三.
不要被你們中間任何的先知.
占卜.
甚至自己發的夢誘惑.
因為他們甚至托我的名.
所說的.
都不是我猜險他們的.
這是第三.
第四.
上帝說得很清楚.
為巴比倫所定的是七十年.
要這個時間完了之後.

$^{721}$你們才可以回耶路撒冷.
說完這四件事.
才說你們祈求就給你們.
尋找就尋見.
而整句話.
之前發生的事情就是28章.
哈那尼亞宣佈.
兩年之內被擄的猶大人要回來了.
耶利米說阿們.
但願你所說的是真的.
不過結果上帝說.
他說的是假的.
沒有信.
兩個月後.
哈那尼亞就死了.
耶利米趕緊寫信.
提醒巴比倫的猶大人.
不要這麼快訂機票.
你們還沒有得回來.
一盤冷水淋下去.
是嗎?.
多興奮.
你猜去到巴比倫的猶大人.
這四年來有沒有開過行李箱.
我猜他沒有收拾出來.
明天就走.
是嗎?.
馬上回耶路撒冷.
今晚就馬上可以.
不用收拾的了.
東西都還沒有打開.
買房子.
分家.
從來沒有想過.
你想想在集中營裡面.
在難民營裡面.
說什麼這些.
哪有條件去說這些.
沒有的.
然後什麼.

$^{761}$叫他們不要信.
他們當中有任何的人.
說我昨晚做了一個夢.
我夢見自己回到耶路撒冷了.
日有所思.
那就是夜有所夢.
上帝早就說了.
沒有.
我沒有猜險.
我沒有人去你們中間.
70年.
是什麼意思.
香港來的弟兄姊妹.
很明白.
不過你們等不到.
50年不變.
是1997年.
香港回歸.
鄧小平.
答應.
是嗎.
還有幾句我們都不想在港台重複的說話.
什麼馬照.
你們都知道.
一切不變.
50年不變.
為什麼50年不變.
其實很簡單.
變的時候.
我們這一代都不在了.
2047年.
你幾多歲.
如果你是97的時候.
你會害怕.
所以這個是.
沒話沒所謂.
等50年.
現在上帝說.
等70年.
被老.

$^{801}$的猶大人.
不是嬰兒.
所以70年之後.
他不是剛剛70歲.
還可以走得回去.
70年之後.
如果他被老的時候.
是20歲.
90歲.
還可以回去.
大概他會提醒.
他的兒子.
你記得弱失的遺言.
將我的骸骨.
帶回去.
加拿大墨比拉洞.
阿伯拉罕所買的.
墳地安葬.
所以.
70年這句不是預言.
鄧伯伯抄聖經.
你明白我的意思吧.
有這樣的一天.
不過你等不到.
開行李箱.
unpack.
你的luggage.
好好地生養眾多.
還要為那.
成求平安.
請問.
被老的猶大人.
去到巴比倫.
祈禱嗎.
你怎麼知道.
你點頭點得這麼快.
你怎麼知道他祈禱.
因為聖經有記載.
聖經哪裡記載.
巴比倫的猶大人祈禱.

$^{841}$詩篇.
第幾篇.
是最出名的.
奏坐詩篇.
137篇.
很清楚.
我們曾在巴比倫的河邊坐下.
一追上石安就哭了.
他們在巴比倫的河邊.
他們想起石安耶路撒冷.
他們哭了.
我們不可以在那裡唱歌.
為什麼.
我們怎麼可以在外邦.
唱耶和華的歌.
不要誤會.
這不是祈禱的意思.
這是唱石安的歌.
敵人撩他們.
你不是有什麼保佑你的歌嗎.
唱來聽聽.
唱完之後.
不是的.
你的神都不保佑你的.
不上教會好好的.
一上教會就癌症.
你有沒有聽過這些說話.
你的神保佑你的.
為什麼.
這就是巴比倫的詩篇.
留下很清楚的記載.
猶大人被擄到巴比倫祈禱.
為什麼.
你讀下去.
結束.
我們很深印象.
那你的英鞋摔在盤石上的.
那人便為有福.
前面那句.
將要被滅的巴比倫城.

$^{881}$報復你將你待我們的那人.
便為有福.
前面那句.
耶路撒冷遭難的日子.
以東人說拆毀拆毀直拆到根基.
耶和華.
祈禱.
怎麼祈禱.
求你紀念這首.
所以香港有些基督徒說.
我們不用說粗口.
我們用這些詩篇祈禱就行了.
還有效.
神聽祈禱的.
神聽不聽這樣的祈禱.
沒有那麼多人敢說.
神聽不聽到祈禱.
聽到當然聽到.
祂是真又活的神.
祂的百姓.
向祂的祈禱.
怎麼會聽不到.
聽到.
不過請問.
上帝怎樣.
誰聽.
回聽.
回覆.
祂的子民這樣的祈禱.
這個我們要問.
你當然可以這樣祈禱.
沒有,我不生巴比倫的.
我來到現在.
奧克蘭,我都想買個農場.
退隱.
學徒冤命.
誤墮塵網中.
一去三十年,還好我這裡還有三十年.
可以喝下.
最健康的羊奶.

$^{921}$牛奶,吃下最健康的.
水果.
感謝主.
但你都會.
在你的生命.
生活當中會碰到.
人,令你很生氣.
身邊的好.
遠一點的好.
你來到.
上帝的面前.
你不說,他都知道.
都知道.
感謝讚美.
創天造地.
好像慈禧太后,十八個字.
說完之後.
從創造以來,你又創世紀開始.
直到出埃及過紅海.
經過抗野,進入迦南.
你都未背完.
你自己都睡了.
這樣祈禱.
上帝聽不聽到.
上帝說,不要以為話多了.
必蒙誰聽.
所以有很多人喜歡說話很長.
千萬不要在崇拜.
這樣領土,否則.
人們上完廁所回來,你都未啞門.
你不說.
上帝都知道,那不如誠實一點.
主啊.
生氣.
卻不要犯罪.
不可含怒.
道日樂.
保羅都是在引用詩篇.
都是舊約的金句.
不過我們又是不懂.

$^{961}$可以生氣,但不要犯罪.
猶大人在巴比倫.
有沒有拿起武器.
汽油彈.
走去巴比倫警察局門口.
扔.
沒有.
沒有這樣做.
祈禱而已.
他求什麼.
拿你的英鞋.
摔在盤石上的.
這句說話.
是什麼意思.
不是他們半夜三更.
偷了人家的嬰兒.
然後爬上十八樓.
扔下來.
沒有,這是戰爭的現實.
所以他們的祈禱.
是說什麼.
報復你.
將你待我們的.
你曾經.
來耶路撒冷.
怎樣對我們.
你的軍隊怎樣遊掠我們.
將有一日.
上帝是公義的.
必定追討巴比倫的罪.
於是巴比倫城.
亦都會成破人亡.
拿你的英鞋.
摔在盤石上.
報復你將那.
對待我們的.
那人便為由.
到時猶大人.
為哪邊祈禱.
為巴比倫的敵人.

$^{1001}$祈禱.
這個是.
CP137篇.
上帝聽見.
這樣的祈禱.
上帝.
怎樣回復.
從被擄的.
頭一天.
到西底加皇帝.
四年五月.
猶大人每晚.
都是這樣祈禱.
祈了四年半.
上帝都未聽.
聽啊.
先知耶利米寫信來.
耶和華.
如此說.
上帝回覆.
嘩!說甚麼.
上帝就會.
幫我們了.
拯救我們了.
上帝說.
你們要為.
那城.
求平安.
我讀了很久.
我都解錯了這句說話.
我以為是說.
為香港求平安.
為奧克蘭求平安.
為悉尼求平安.
為波斯求平安.
還有哪個城市.
今晚轉播中.
是吧.
Melbourne求平安.
你當然是Melbourne.

$^{1041}$這麼想Melbourne.
不是啊.
如果是這樣.
就應該是為耶路撒冷求平安.
是吧.
猶大人是從耶路撒冷來的.
猶大人.
每晚都為耶路撒冷求平安.
猶大人.
四年五個月以來.
絕對不會.
為巴比倫求平安.
對嗎.
因為這是敵人.
這是敵人.
這是令他們.
國破家亡的.
尼泊爾.
但上帝竟然.
回覆.
他自己的百姓.
四年半來的.
祈禱.
為巴比倫.
求平安.
這個平安.
的主題.
就是剛才.
我們讀29章.
上帝說.
我向你們所懷的意念.
是平安的意念.
重新提起平安.
然後上帝說.
你們呼求我.
禱告我.
我就應運你們.
你們尋求我.
專心尋求我.
話說我必被你們尋見.

$^{1081}$請問.
上帝要聽的.
祈禱.
是誰的祈禱.
這樣就很清楚了.
就不是.
就坐巴比倫的.
祈禱.
看到了嗎.
就是如果.
猶大人.
收到信開始那一晚.
我們的反應就是.
70年了.
要去建屋.
那就不回祈禱會了.
以前就很多人回的.
逢星期三晚.
大家回來求什麼.
回耶路撒冷.
盡早神聽祈禱.
不是,我們就坐巴比倫.
多些人就坐他.
我們同仇敵愾.
我們回來祈禱.
現在沒用了.
今晚祈禱第一件事.
為巴比倫求平安.
你做不做.
不回.
很正常.
但請問.
你知不知道.
猶大人.
從西底交往.
第四年.
如果說.
耶利米信要五個月才去到.
可能都.
大半年過去了.

$^{1121}$他們有沒有.
按照先知.
的教導.
這樣祈禱呢.
有沒有.
你很誠實.
你的頸動都不動.
你不敢動.
你不敢搖頭.
有沒有,他們有沒有這樣祈禱.
不知道.
聖經沒有記載.
連.
野史都沒有記載.
故事都沒有記載.
口頭傳述.
都沒有記載.
我們不知道.
不過我們知道一個事實.
我們知道巴比倫.
是怎樣亡國的.
知不知道.
聖經有記載的.
但以你書.
第五章.
白沙薩王.
那晚.
大宴群臣.
牆上有.
紙頭出來寫字.
Writing on the wall.
這個英文的.
諺語,大家都會解.
那晚.
白沙薩王.
還賞賜.
但以你國.
中位列第三.
那晚巴比倫就亡了.
嘩.

$^{1161}$真是夜夜笙歌.
真是走詞欲林.
真是亡國.
美美之音.
那晚打仗的.
都在大宴群臣.
看他亡國,你又誤會了歷史.
白沙薩王.
做夢都沒有想過.
那晚會亡國.
為什麼?.
因為按照歷史的記載.
馬代波斯的軍隊.
是兵不血刃.
這樣進了巴比倫城.
為什麼?.
因為有人大開城門.
迎接敵軍.
誰這麼大膽開城門?.
是巴比倫的祭司.
都是領袖.
宗教的領袖.
他們.
懷恨在心.
因為白沙薩王.
巴比倫的政府.
削減他們的待遇.
都是這件事.
又是預算的問題.
所以.
他們不敢開城門.
因為他們不敢開城門.
又是預算的問題.
你們很熟悉的.
所以.
他們寧願通敵賣國.
打開城門.
因為.
對不起.
古烈.

$^{1201}$答應了祭司.
會讓他們重建.
所有的.
馬道克廟宇.
恢復.
他們的薪酬.
厚待他們.
你怎麼知道這麼多?.
其實.
古烈的詔書.
在.
大英博物館.
展覽了出來.
Syracylinder.
你今天去你都會看到.
那個版本是寫給巴比倫的.
不是.
舊約聖經.
以斯拉記開頭.
寫給猶大人的版本.
所以.
巴比倫的亡國.
和亞述很不同.
和波斯很不同.
和希臘很不同.
竟然.
是這麼平安的.
怎麼證明很平安?.
但以你書有記載.
雖然改朝換代.
但以你.
卻服侍.
尼布甲尼撒.
又服侍白沙薩.
又服侍.
大利烏.
你讀聖經有沒有讀到?.
主要學以為講一下.
不是的,是歷史.
因為這麼平安.

$^{1241}$在元年.
猶大人.
就可以回歸耶路撒冷.
重建.
聖殿.
聖城.
聖文.
是不是因為.
猶大人的祈禱?.
不知道,你不要說.
沒有記載.
不知道.
不過上帝的心意是怎樣?.
不是愈亂愈好.
不是殺戮.
不是報仇.
看到嗎?主耶穌怎樣教導我們?.
路加福音.
第六章.
二十七節.
我讀給大家聽.
只是我告訴你們.
這聽道的人.
我們.
我們在聽道.
你們的仇敵.
要愛他.
恨你們的.
要待他好.
就坐你們的.
要為他祝福.
凌辱你們的.
要為他禱告.
凌辱你們的.
要為他禱告.
有弟兄姊妹聽到這句話就問.
怎樣為他祈禱?.
你還出咒助詩篇?.
不是.
你為他求平安.

$^{1281}$是吧?.
怎樣為他祝福?.
怎樣待他好?.
怎樣愛他?.
你不是同意他.
你不是附和他.
這裡講得很清楚.
是你的仇敵.
我沒有仇敵.
感謝主,做基督徒.
你不當他是仇敵.
行不行?.
你不贊同他對婚姻的看法.
行不行?.
他來拆你十字架.
可不可以?.
那你怎樣做?.
教會站起來.
我們宣告公義.
我們在互聯網錄一段講話.
咒助他.
對不起,聖經不是這樣教導我們.
聽清楚.
我們很想這樣做.
黑白分明.
敵我立見.
你對我不好.
孔子都教我們以直報怨.
我不用祝福你的.
不用以德報怨.
耶穌不是這樣教.
所以我們不是孔子的門徒.
我們是耶穌的門徒.
再讀一次.
你們的仇敵要愛他.
恨你們的要待他好.
咒助你們的要為他祝福.
凌辱你們的要為他禱告.
逼迫你們的要怎樣做?.
我們的反應和聖經的教導.

$^{1321}$巴比倫這個城對當時被擄的猶大人.
詩篇137篇是正常的反應.
你的英雄率在盤石上.
這人便為有福.
感謝主上帝沒有聽到這個祈禱.
先知提醒他們.
為巴比倫求平安.
我和我的同學說.
從內地來到中神讀書的同學.
我說如果中國教會今天為北京求平安.
你猜上帝聽不聽?.
你猜我們肯不肯?.
如果海外的華人教會.
同心合意.
每個主日崇拜.
為中國求平安.
你猜上帝會怎樣做?.
你猜上帝要怎樣聽?.
你們祈求就給你們.
尋找就尋見.
這個應許原來不是一張空白的支票.
不是空頭.
是空白的.
任你填你的要求上去.
原來他有一個歷史的典故.
在這個歷史的處境當中.
作為神的兒女.
他們原來不是這樣祈禱的.
他們是理直氣壯的.
拿你的英雄率在盤石上的巴比倫人.
他有福.
你受咒了.
幸好神的子民是祈禱.
沒有付諸行動.
上帝聽見.
上帝透過耶利米才寫信回覆巴比倫這一代被擄的猶大人.
什麼話?.
有沒有聽清楚?.
是的,耶利米的信是這樣說的.
你再讀一次給我聽.

$^{1361}$再讀一次都是這幾句話.
你們被擄到的那城.
你們要為那城求平安.
因為那城得平安.
你們也隨著得平安.
我向你們所懷的意念.
是賜平安的意念.
不是降災禍的意念.
主耶穌沒有答應我們.
有求必應.
這四個字是聖王的菩薩才這樣說的.
明白了,香港來的,不用說這麼清楚.
聖經沒有這樣答應我們.
但聖經卻答應我們.
你按著上帝的心意教導來祈禱.
上帝一定聽.
那什麼是上帝的心意?.
看這一書.
是逼迫猶大人的巴比倫.
才說為他們求平安.
上帝一定聽.
我們不相信.
不可能的.
上帝起碼要分是非黑白.
我們唯一知道上帝聽不聽的只有一個做法.
是怎樣?.
我們這樣祈禱.
我們樂意這樣祈禱.
我來了這裡,也不干涉以前的事情.
為特朗普祈禱.
我不認識你們的總理.
也要為他祈禱的,牧師說的.
為奧克蘭祈禱.
為你的城市祈禱.
為那些不是在神的旨意當中的權力政府祈禱.
怎樣祈禱?叫他們早點下台?.
為他們求平安.
你相信嗎?.
你願意這樣聽上帝的教導嗎?.
因為他們得平安.

$^{1401}$你們作為神的兒女也隨著得平安.
感謝主.
要不我們就不讀聖經.
按我們的心靈的感覺就坐.
很過癮的.
你選了一些就坐詩篇出來,逐一讀.
嘩!真是比打小人好.
上帝聽見,不過上帝不會這樣去做.
相反,很清楚,祂透過先知提醒我們.
學習合神心意的祈禱.
求主幫助我們.
我們同心低頭禱告.
真理的聖靈,我們多謝你.
因為你的說話解開,發出亮光.
使我們得到智慧.
叫我們不需要在黑暗當中摸索.
不知道怎樣去祈禱才是合你的心意.
因為你的說話一早就教導我們.
是呀,靈育我們.
原來我們要為祂們禱告.
就坐我們,原來我們要為祂們祝福.
那些對我們不好的,原來我們要這樣對他們.
因為你已經掌權得勝.
我們是你的兒女,我們是天父的兒女.
我們要學習的是你的mercy,是你的完全.
求你打開我們的心,裝載你的說話.
在今天我們祈禱的時候.
我們懂得說,不是照我們的意思.
乃是照你的意思.
就好像主耶穌昔日在赫西瑪尼祈禱一樣.
從我們個人的需要到我們的家.
我們的兒女,我們的父母.
我們的教會,我們的國家.
無論是寄居的,是我們出生的地方.
主呀,你已經教導我們怎樣禱告.
不是在乎祂怎樣對我們.
是在乎天父怎樣愛我們.
今天我們就是站在一個蒙恩的位置.
我們向你禱告,是按照你的吩咐,你的教導.
而我們很清楚知道這樣的祈求必定得著.

$^{1441}$尋找必定尋見,敲門必定開門.
多謝你,誰聽我們眾人心裡面向你發出的禱告.
幫助我們,不單止聽到,並且行到.
在真理的亮光當中,我們得著釋放自由.
可以跟從主的教導和吩咐.
我們為香港祈禱,明天,以後的日子.
我們為政府祈禱,我們為教會祈禱.
我們為牧者祈禱,在不同的政見當中.
怎樣去牧養群養.
我們為這個城市求平安.
我們都紀念北京,我們的國家.
我們求平安,求主恩待,求主憐憫.
求主的旨意聖咒在地,如同聖咒在天.
多謝你,多謝你.
我們恭敬,仰望,等候,祈求.
奉主耶穌基督得勝的名字,阿們.
阿們.
多謝您.
(字幕由 Amara.org 社群提供).
\newpage

\allsectionsfont{\centering}

\setlength\parindent{0pt}
\setlength{\columnsep}{1.25em}
\setlength{\parfillskip}{0pt}
\setlength{\tabcolsep}{1em}
\raggedbottom

\pagenumbering{gobble}


\newfontfamily\leftfont[Path=../fonts/fell_french_canon/, Ligatures=TeX, ItalicFont=IMFeFCit29C.otf, BoldFont=AveriaLibre-Bold.ttf]{IMFeFCrm29C.otf}
\newfontfamily\leftcitationfont[Path=../fonts/frankruehl/]{FrankRuehlCLM-Medium.ttf}
\newfontfamily\centerfont[Path=../fonts/garamond/, Ligatures=TeX, ItalicFont=EBGaramond-SemiBoldItalic.ttf]{EBGaramond-SemiBold.ttf}
\newfontfamily\rightfont[Path=../fonts/averia/, Ligatures=TeX, ItalicFont=AveriaLibre-RegularItalic.ttf, BoldFont=AveriaLibre-Bold.ttf, BoldItalicFont=AveriaLibre-BoldItalic.ttf]{AveriaLibre-Light.ttf}
\newfontfamily\rightcitationfont[Path=../fonts/rashi/]{Mekorot-Rashi.ttf}
\definecolor{hcolor}{HTML}{D3230C}
\definecolor{rcolor}{HTML}{D36F0C}
\newcommand{\chfont}[1]{\centerfont{\huge\textcolor{hcolor}{#1}}}
\newcommand{\leftcitation}[1]{\leftcitationfont{\Large\textcolor{hcolor}{#1}}}
\newcommand{\rightcitation}[1]{\rightcitationfont{\normalsize\textcolor{rcolor}{#1}}}
\newfontfamily\flowerfont[Path=../fonts/fell_flowers/]{IMFeFlow2.otf}

\begin{sloppypar}

\chapter*{\chfont{編按結語}}

\columnratio{0.5,0.5}\begin{paracol}{2}

\fontsize{11}{13}\leftfont \Large \leftcitation{א} \leftfont 余少好文.宏志博覽群書而不忘.善存藏經籍文獻備後時之用。\leftcitation{ב} \leftfont 歸主年時.受友所薦.聞道網海.\switchcolumn\fontsize{11}{13}\rightfont \Large \leftcitation{ח} \rightfont 有見粵道之危.國之封講道千言亦將就至.急之何則為?\leftcitation{ט} \rightfont 嘗聞猶太者之傳承.在其力守口述之

\end{paracol}


\columnratio{0.32,0.32,0.32}\begin{paracol}{3}

\fontsize{11}{13}\leftfont \Large 尤以吳約翰遜者 \switchcolumn[2]\fontsize{11}{13}\rightfont \Large 統.以煉千載不

\end{paracol}

\columnratio{0.32,0.32,0.32}
\begin{paracol}{3}\fontsize{11}{13}\leftfont \Large 為重.其載上之粵語講道緩緩入耳.收之藏其音頻.善妥整存.反復而嚼.受益無窮。\leftcitation{ג} \leftfont 我城我國既限.歷一四一九之不測.肺疫延年.信徒靈長屢受圍創.神州燈臺數盡指日可待.粵道之求與日俱增。\leftcitation{ד} \leftfont 觀乎社、經、法、媒、言、信、網之地.愈趨受鋤.自翔不果.授受壓力.粵道聖言亦愈漸艱難。\leftcitation{ה} \leftfont 況崇基例乎.學苑講道屢逆權勢者.其言末強受壓.舊章盡刪以存其身。講道釋數失傳.徒嘆奈何。

\switchcolumn

\fontsize{11}{13}\centerfont 
\begin{tikzpicture}
    \node (0,0) [xshift=-0.10cm, yshift=-1.0cm, opacity=0.10]{\includegraphics[width=0.30\textwidth]{../ot_frontcover.png}} ;
    \node (0,0) [xshift=+0.20cm, yshift=+2.0cm, opacity=0.10]{\includegraphics[width=0.20\textwidth]{../christ_on_cross.png}} ;
\end{tikzpicture}
\Large 

\leftcitation{ס} \centerfont 詩百又廿七載:
\leftcitation{ע} \centerfont 非耶和華建屋宇.則匠人之經營徒.
\leftcitation{פ} \centerfont 非耶和華衛城邑.則守者之儆醒徒.
\leftcitation{צ} \centerfont 余獻是卷予華人社區.願為福音流通之器.願獻斯微材為祭榮耀上帝.
\leftcitation{ק} \centerfont 阿門

\switchcolumn

\fontsize{11}{13}\rightfont \Large 滅.時越次聖殿期及當今。\leftcitation{י} \rightfont 猶太者力廣納之.筆錄以卷軸.便以傳、閱、頌、攜、守、鎖、抄、譯、釋、編,得書塔木德、密示拿等經傳.家喻戶曉.傳流若芳。\leftcitation{כ} \rightfont 猶太者文以載道.傳其口述.今我輩粵道之傳應當作如是.遂力行粵音識辨之法.載言載道.以盡忠傳粵道以待興。\leftcitation{ל} \rightfont 蒙下賜恩惠.無畏海量字音文書.既馭上帝之道.今廣及粵語講道.重駛編程之技.匯導粵音遂字稿.重塑講道現場.以傚猶太卷軸之舉便以傳流。\leftcitation{מ} \rightfont 是卷乃粵音口述傳之屬.莫通華文白話之語.

\end{paracol}

\columnratio{0.5,0.5}
\begin{paracol}{2}\fontsize{11}{13}\leftfont \Large \leftcitation{ו} \leftfont 斯殺一違儆百逆.既禁壓之.我輩聞風無奈.在所難免。\leftcitation{ז} \leftfont 另有異人例乎.以版權之名.脅網絡頻道之舉.同授礙予粵道之存流。

\switchcolumn

\fontsize{11}{13}\rightfont \Large 惟待後繼來者之傚.以譯釋傳之於神州華文地。\leftcitation{נ} \rightfont 今能排程驅馭圖靈以編彙文檔,其碼長共數千千亦無逢大礙.全蒙上帝保守。

\end{paracol}



\columnratio{1}\begin{paracol}{1}

\fontsize{11}{13}\rightfont \Large
~~~~~~~~~~~~~~~~~~~~~~~~~~~~~~~~~~~~~~~~~~~~~~~~~~~~~~~~~~~~~~~~~~~~~~~~~~~~~~~\leftcitation{ר} \rightfont 二零二三年二月一日

~~~~~~~~~~~~~~~~~~~~~~~~~~~~~~~~~~~~~~~~~~~~~~~~~~~~~~~~~~~~~~~~~~~~~~~~~~~~~~~\leftcitation{ש} \rightfont 米迦勒

~~~~~~~~~~~~~~~~~~~~~~~~~~~~~~~~~~~~~~~~~~~~~~~~~~~~~~~~~~~~~~~~~~~~~~~~~~~~~~~\leftcitation{ת} \rightfont 書於香港

\end{paracol}

\end{sloppypar}
\end{document}
