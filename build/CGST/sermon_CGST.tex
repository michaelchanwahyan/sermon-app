\documentclass{book}
%\usepackage[letterpaper, portrait, margin=1cm]{geometry}
%\usepackage[letterpaper, bindingoffset=0.2in, left=1in,right=1in,top=.5in,bottom=.5in,footskip=.25in,marginparwidth=5em]{geometry}
\usepackage[letterpaper, left=1in,right=1in,top=.5in,bottom=.5in,footskip=.25in,marginparwidth=1cm]{geometry}
% ---------------------
% mini-table-of-content
% ---------------------
\usepackage{minitoc}
\setcounter{minitocdepth}{1}
\setlength{\mtcindent}{24pt}
\setcounter{secnumdepth}{-2}
%\renewcommand{\mtcfont}{\small\rm}
%\renewcommand{\mtcSfont}{\small\bf}
%\usepackage{setspace}
%\usepackage{tocloft}
%\setlength\cftparskip{-1.2pt}
%\setlength\cftbeforesecskip{1.3pt}
%\setlength\cftaftertoctitleskip{2pt}
%\renewcommand{\cftsecafterpnum}{\hspace*{02.0em}}
%\renewcommand{\cftsubsecafterpnum}{\hspace*{02.0em}}

% ---------------------------
% Chinese Characters Packages
% ---------------------------
\usepackage{fontspec} 
\usepackage{xeCJK}
\setmainfont{Times}
\setCJKmainfont{BiauKai}
\newfontfamily\sblgoodhebrew{SBL BibLit}[Script=Hebrew,Contextuals=Alternate]
\newfontfamily\sblgoodgreek{SBL BibLit}[Script=Greek,Contextuals=Alternate]

\usepackage{ifpdf,cite,algorithmic,url,tikz}
\usepackage[cmex10]{amsmath}

% ---------------------------
% Hebrew Characters Packages
% ---------------------------
\usepackage{polyglossia}
\setmainfont{Times New Roman}

% -------
% General
% -------
\usepackage{multicol}
\usepackage{multirow}
\usepackage{color,colortbl}
\usepackage{xparse}
\usepackage{pbox}
\usepackage{stackengine}
\usepackage{titlesec}% http://ctan.org/pkg/titlesec
\usepackage{tabularx}
\usepackage{xltabular}
\usepackage{titlesec}
\usepackage{makecell}
\newcommand{\sectionbreak}{\clearpage}

\author{
  Editor, Michael Chan\\
  \texttt{michaelchan\_wahyan@yahoo.com.hk}
}
\usepackage{tocloft}

\usepackage{hyperref}
\hypersetup{
    colorlinks=true, % set true if you want colored links
    linktoc   =all , % set to all if you want both sections and subsections linked
    linkcolor =blue, % choose some color if you want links to stand out
}

% ----------
% Afterword
% ----------
\usepackage{marginnote}
\usepackage{sectsty}
\usepackage{ragged2e}
\usepackage{lineno}
\usepackage{xcolor}
\usepackage{paracol}

\begin{document}

\clearpage
%% temporary titles
% command to provide stretchy vertical space in proportion
\newcommand\nbvspace[1][3]{\vspace*{\stretch{#1}}}
% allow some slack to avoid under/overfull boxes
\newcommand\nbstretchyspace{\spaceskip0.5em plus 0.25em minus 0.25em}
% To improve spacing on titlepages
\newcommand{\nbtitlestretch}{\spaceskip0.6em}
\pagestyle{empty}
\begin{center}
\bfseries
\nbvspace[1]
\Huge
{%\nbtitlestretch
\Large
\textbf{中國神學研究院 粵語講道逐字稿 \\
       Youtube Channel: China Graduate School of Theology
       }}

\nbvspace[1]

{\large
Editor: Michael\\
\texttt{michaelchan\_wahyan@yahoo.com.hk}
}

\nbvspace[1]

{\large
Revision: \texttt{v1.1}\\
Last Update: \today
}


\vfill
\begin{tikzpicture}
    %remove comment for OT cover%\node (0,0) [opacity=0.03]{\includegraphics[width=15cm]{../bible_out/ot_frontcover.png}} ;
    %remove comment for NT cover%\node (0,0) [opacity=0.03]{\includegraphics[width=15cm]{../bible_out/christ_on_cross.png}} ;
    %remove comment for Bible cover%\node (0,0) [xshift=0.8cm, yshift=+2cm, opacity=0.03]{\includegraphics[width=10cm]{./christ_on_cross.png}} ;
    %remove comment for Bible cover%\node (0,0) [              yshift=-2cm, opacity=0.03]{\includegraphics[width=14cm]{./ot_frontcover.png}} ;
\end{tikzpicture}
\vfill

\end{center}

\newpage

\setcounter{tocdepth}{0}
\dominitoc
\begin{multicols}{3}
\addtocontents{toc}{\protect\hypertarget{toc}{}}
\tableofcontents
\end{multicols}

\large
%\twocolumn

% the color definition syntax is as follow:
% \definecolor{name}{system}{definition}
% example: a mono-channel color can be defined as
%          \definecolor{Gray}{gray}{0.9}
% example: an rgb-3-channel color can be defined as
%          \definecolor{LightCyan}{rgb}{0.88,1,1}
%          \definecolor{pink}{rgb}{0.68,0,0.68}

\definecolor{CUV1LightRed}{rgb}{1,0.75,0.75}     % for CUV1
\definecolor{LZZVLightGray}{rgb}{0.9,0.9,0.9}    % for LZZ
\definecolor{KJVVLightGreen}{rgb}{0.75,1,0.85}   % for KJV
\definecolor{CUV2LightYellow}{rgb}{1,1,0.75}     % for CUV2
\definecolor{CNVVLightBrown}{rgb}{1,0.85,0.7}    % for CNV
\definecolor{NRSVLightBlue}{rgb}{0.75,1,1}       % for NRSV
\definecolor{WENLLightPurple}{rgb}{0.95,0.85,0.9}% for WENL
\definecolor{TCV19PaleGreen}{rgb}{0.85,1,0.95}   % for TCV19
\definecolor{MSGVLightWhite}{rgb}{0.98,0.98,0.98}% for MSGV
\definecolor{NETSLightRed}{rgb}{1,0.75,0.75}     % for NETS
\definecolor{JPS1917LightYellow}{rgb}{1,1,0.75}  % for JPS1917
\definecolor{SBLGNTPaleRed}{rgb}{1,0.85,0.80}    % for SBLGNT

\section{目錄}
\label{sec:index}
{ \scriptsize


\begin{xltabular}{\textwidth}{|p{0.15\textwidth} p{0.6\textwidth}|p{0.07\textwidth} p{0.1\textwidth}|}
\hline
\end{xltabular}
}
\newpage

\allsectionsfont{\centering}

\setlength\parindent{0pt}
\setlength{\columnsep}{1.25em}
\setlength{\parfillskip}{0pt}
\setlength{\tabcolsep}{1em}
\raggedbottom

\pagenumbering{gobble}


\newfontfamily\leftfont[Path=../fonts/fell_french_canon/, Ligatures=TeX, ItalicFont=IMFeFCit29C.otf, BoldFont=AveriaLibre-Bold.ttf]{IMFeFCrm29C.otf}
\newfontfamily\leftcitationfont[Path=../fonts/frankruehl/]{FrankRuehlCLM-Medium.ttf}
\newfontfamily\centerfont[Path=../fonts/garamond/, Ligatures=TeX, ItalicFont=EBGaramond-SemiBoldItalic.ttf]{EBGaramond-SemiBold.ttf}
\newfontfamily\rightfont[Path=../fonts/averia/, Ligatures=TeX, ItalicFont=AveriaLibre-RegularItalic.ttf, BoldFont=AveriaLibre-Bold.ttf, BoldItalicFont=AveriaLibre-BoldItalic.ttf]{AveriaLibre-Light.ttf}
\newfontfamily\rightcitationfont[Path=../fonts/rashi/]{Mekorot-Rashi.ttf}
\definecolor{hcolor}{HTML}{D3230C}
\definecolor{rcolor}{HTML}{D36F0C}
\newcommand{\chfont}[1]{\centerfont{\huge\textcolor{hcolor}{#1}}}
\newcommand{\leftcitation}[1]{\leftcitationfont{\Large\textcolor{hcolor}{#1}}}
\newcommand{\rightcitation}[1]{\rightcitationfont{\normalsize\textcolor{rcolor}{#1}}}
\newfontfamily\flowerfont[Path=../fonts/fell_flowers/]{IMFeFlow2.otf}

\begin{sloppypar}

\chapter*{\chfont{編按結語}}

\columnratio{0.5,0.5}\begin{paracol}{2}

\fontsize{11}{13}\leftfont \Large \leftcitation{א} \leftfont 余少好文.宏志博覽群書而不忘.善存藏經籍文獻備後時之用。\leftcitation{ב} \leftfont 歸主年時.受友所薦.聞道網海.\switchcolumn\fontsize{11}{13}\rightfont \Large \leftcitation{ח} \rightfont 有見粵道之危.國之封講道千言亦將就至.急之何則為?\leftcitation{ט} \rightfont 嘗聞猶太者之傳承.在其力守口述之

\end{paracol}


\columnratio{0.32,0.32,0.32}\begin{paracol}{3}

\fontsize{11}{13}\leftfont \Large 尤以吳約翰遜者 \switchcolumn[2]\fontsize{11}{13}\rightfont \Large 統.以煉千載不

\end{paracol}

\columnratio{0.32,0.32,0.32}
\begin{paracol}{3}\fontsize{11}{13}\leftfont \Large 為重.其載上之粵語講道緩緩入耳.收之藏其音頻.善妥整存.反復而嚼.受益無窮。\leftcitation{ג} \leftfont 我城我國既限.歷一四一九之不測.肺疫延年.信徒靈長屢受圍創.神州燈臺數盡指日可待.粵道之求與日俱增。\leftcitation{ד} \leftfont 觀乎社、經、法、媒、言、信、網之地.愈趨受鋤.自翔不果.授受壓力.粵道聖言亦愈漸艱難。\leftcitation{ה} \leftfont 況崇基例乎.學苑講道屢逆權勢者.其言末強受壓.舊章盡刪以存其身。講道釋數失傳.徒嘆奈何。

\switchcolumn

\fontsize{11}{13}\centerfont 
\begin{tikzpicture}
    \node (0,0) [xshift=-0.10cm, yshift=-1.0cm, opacity=0.10]{\includegraphics[width=0.30\textwidth]{../ot_frontcover.png}} ;
    \node (0,0) [xshift=+0.20cm, yshift=+2.0cm, opacity=0.10]{\includegraphics[width=0.20\textwidth]{../christ_on_cross.png}} ;
\end{tikzpicture}
\Large 

\leftcitation{ס} \centerfont 詩百又廿七載:
\leftcitation{ע} \centerfont 非耶和華建屋宇.則匠人之經營徒.
\leftcitation{פ} \centerfont 非耶和華衛城邑.則守者之儆醒徒.
\leftcitation{צ} \centerfont 余獻是卷予華人社區.願為福音流通之器.願獻斯微材為祭榮耀上帝.
\leftcitation{ק} \centerfont 阿門

\switchcolumn

\fontsize{11}{13}\rightfont \Large 滅.時越次聖殿期及當今。\leftcitation{י} \rightfont 猶太者力廣納之.筆錄以卷軸.便以傳、閱、頌、攜、守、鎖、抄、譯、釋、編,得書塔木德、密示拿等經傳.家喻戶曉.傳流若芳。\leftcitation{כ} \rightfont 猶太者文以載道.傳其口述.今我輩粵道之傳應當作如是.遂力行粵音識辨之法.載言載道.以盡忠傳粵道以待興。\leftcitation{ל} \rightfont 蒙下賜恩惠.無畏海量字音文書.既馭上帝之道.今廣及粵語講道.重駛編程之技.匯導粵音遂字稿.重塑講道現場.以傚猶太卷軸之舉便以傳流。\leftcitation{מ} \rightfont 是卷乃粵音口述傳之屬.莫通華文白話之語.

\end{paracol}

\columnratio{0.5,0.5}
\begin{paracol}{2}\fontsize{11}{13}\leftfont \Large \leftcitation{ו} \leftfont 斯殺一違儆百逆.既禁壓之.我輩聞風無奈.在所難免。\leftcitation{ז} \leftfont 另有異人例乎.以版權之名.脅網絡頻道之舉.同授礙予粵道之存流。

\switchcolumn

\fontsize{11}{13}\rightfont \Large 惟待後繼來者之傚.以譯釋傳之於神州華文地。\leftcitation{נ} \rightfont 今能排程驅馭圖靈以編彙文檔,其碼長共數千千亦無逢大礙.全蒙上帝保守。

\end{paracol}



\columnratio{1}\begin{paracol}{1}

\fontsize{11}{13}\rightfont \Large
~~~~~~~~~~~~~~~~~~~~~~~~~~~~~~~~~~~~~~~~~~~~~~~~~~~~~~~~~~~~~~~~~~~~~~~~~~~~~~~\leftcitation{ר} \rightfont 二零二三年二月一日

~~~~~~~~~~~~~~~~~~~~~~~~~~~~~~~~~~~~~~~~~~~~~~~~~~~~~~~~~~~~~~~~~~~~~~~~~~~~~~~\leftcitation{ש} \rightfont 米迦勒

~~~~~~~~~~~~~~~~~~~~~~~~~~~~~~~~~~~~~~~~~~~~~~~~~~~~~~~~~~~~~~~~~~~~~~~~~~~~~~~\leftcitation{ת} \rightfont 書於香港

\end{paracol}

\end{sloppypar}
\end{document}
