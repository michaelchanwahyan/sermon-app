\documentclass{book}
%\usepackage[letterpaper, portrait, margin=1cm]{geometry}
%\usepackage[letterpaper, bindingoffset=0.2in, left=1in,right=1in,top=.5in,bottom=.5in,footskip=.25in,marginparwidth=5em]{geometry}
\usepackage[letterpaper, left=1in,right=1in,top=.5in,bottom=.5in,footskip=.25in,marginparwidth=1cm]{geometry}
% ---------------------
% mini-table-of-content
% ---------------------
\usepackage{minitoc}
\setcounter{minitocdepth}{1}
\setlength{\mtcindent}{24pt}
\setcounter{secnumdepth}{-2}
%\renewcommand{\mtcfont}{\small\rm}
%\renewcommand{\mtcSfont}{\small\bf}
%\usepackage{setspace}
%\usepackage{tocloft}
%\setlength\cftparskip{-1.2pt}
%\setlength\cftbeforesecskip{1.3pt}
%\setlength\cftaftertoctitleskip{2pt}
%\renewcommand{\cftsecafterpnum}{\hspace*{02.0em}}
%\renewcommand{\cftsubsecafterpnum}{\hspace*{02.0em}}

% ---------------------------
% Chinese Characters Packages
% ---------------------------
\usepackage{fontspec} 
\usepackage{xeCJK}
\setmainfont{Times}
\setCJKmainfont{BiauKai}
\newfontfamily\sblgoodhebrew{SBL BibLit}[Script=Hebrew,Contextuals=Alternate]
\newfontfamily\sblgoodgreek{SBL BibLit}[Script=Greek,Contextuals=Alternate]

\usepackage{ifpdf,cite,algorithmic,url,tikz}
\usepackage[cmex10]{amsmath}

% ---------------------------
% Hebrew Characters Packages
% ---------------------------
\usepackage{polyglossia}
\setmainfont{Times New Roman}

% -------
% General
% -------
\usepackage{multicol}
\usepackage{multirow}
\usepackage{color,colortbl}
\usepackage{xparse}
\usepackage{pbox}
\usepackage{stackengine}
\usepackage{titlesec}% http://ctan.org/pkg/titlesec
\usepackage{tabularx}
\usepackage{xltabular}
\usepackage{titlesec}
\usepackage{makecell}
\newcommand{\sectionbreak}{\clearpage}

\author{
  Editor, Michael Chan\\
  \texttt{michaelchan\_wahyan@yahoo.com.hk}
}
\usepackage{tocloft}

\usepackage{hyperref}
\hypersetup{
    colorlinks=true, % set true if you want colored links
    linktoc   =all , % set to all if you want both sections and subsections linked
    linkcolor =blue, % choose some color if you want links to stand out
}

% ----------
% Afterword
% ----------
\usepackage{marginnote}
\usepackage{sectsty}
\usepackage{ragged2e}
\usepackage{lineno}
\usepackage{xcolor}
\usepackage{paracol}

\begin{document}

\clearpage
%% temporary titles
% command to provide stretchy vertical space in proportion
\newcommand\nbvspace[1][3]{\vspace*{\stretch{#1}}}
% allow some slack to avoid under/overfull boxes
\newcommand\nbstretchyspace{\spaceskip0.5em plus 0.25em minus 0.25em}
% To improve spacing on titlepages
\newcommand{\nbtitlestretch}{\spaceskip0.6em}
\pagestyle{empty}
\begin{center}
\bfseries
\nbvspace[1]
\Huge
{%\nbtitlestretch
\Large
\textbf{J. Ng 頻道輯錄 粵語講道逐字稿 2025-26 \\
       Youtube Channel: JohnsonNg
       }}

\nbvspace[1]

{\large
Editor: Michael\\
\texttt{michaelchan\_wahyan@yahoo.com.hk}
}

\nbvspace[1]

{\large
Revision: \texttt{v1.2}\\
Last Update: \today
}


\vfill
\begin{tikzpicture}
    %remove comment for OT cover%\node (0,0) [opacity=0.03]{\includegraphics[width=15cm]{../bible_out/ot_frontcover.png}} ;
    %remove comment for NT cover%\node (0,0) [opacity=0.03]{\includegraphics[width=15cm]{../bible_out/christ_on_cross.png}} ;
    %remove comment for Bible cover%\node (0,0) [xshift=0.8cm, yshift=+2cm, opacity=0.03]{\includegraphics[width=10cm]{./christ_on_cross.png}} ;
    %remove comment for Bible cover%\node (0,0) [              yshift=-2cm, opacity=0.03]{\includegraphics[width=14cm]{./ot_frontcover.png}} ;
\end{tikzpicture}
\vfill

\end{center}

\newpage

\setcounter{tocdepth}{0}
\dominitoc
\begin{multicols}{3}
\addtocontents{toc}{\protect\hypertarget{toc}{}}
\tableofcontents
\end{multicols}

\large
%\twocolumn

% the color definition syntax is as follow:
% \definecolor{name}{system}{definition}
% example: a mono-channel color can be defined as
%          \definecolor{Gray}{gray}{0.9}
% example: an rgb-3-channel color can be defined as
%          \definecolor{LightCyan}{rgb}{0.88,1,1}
%          \definecolor{pink}{rgb}{0.68,0,0.68}

\definecolor{CUV1LightRed}{rgb}{1,0.75,0.75}     % for CUV1
\definecolor{LZZVLightGray}{rgb}{0.9,0.9,0.9}    % for LZZ
\definecolor{KJVVLightGreen}{rgb}{0.75,1,0.85}   % for KJV
\definecolor{CUV2LightYellow}{rgb}{1,1,0.75}     % for CUV2
\definecolor{CNVVLightBrown}{rgb}{1,0.85,0.7}    % for CNV
\definecolor{NRSVLightBlue}{rgb}{0.75,1,1}       % for NRSV
\definecolor{WENLLightPurple}{rgb}{0.95,0.85,0.9}% for WENL
\definecolor{TCV19PaleGreen}{rgb}{0.85,1,0.95}   % for TCV19
\definecolor{MSGVLightWhite}{rgb}{0.98,0.98,0.98}% for MSGV
\definecolor{NETSLightRed}{rgb}{1,0.75,0.75}     % for NETS
\definecolor{JPS1917LightYellow}{rgb}{1,1,0.75}  % for JPS1917
\definecolor{SBLGNTPaleRed}{rgb}{1,0.85,0.80}    % for SBLGNT

{ \scriptsize


\begin{xltabular}{\textwidth}{|p{0.15\textwidth} p{0.6\textwidth}|p{0.07\textwidth} p{0.1\textwidth}|}
\hline
\multicolumn{4}{c}{} \\
\multicolumn{4}{c}{\hyperref[ch:preacher1]{徐武豪}} \\
\multicolumn{4}{c}{} \\
\hline
約翰一書 2:29-3:10 & \hyperref[sec:Y_0n0vkhyDU]{愛的兒女 不辱主名 (約翰一書2\_29-3\_10) - 徐武豪博士} & 2025-01-11 & \href{https://youtube.com/watch?v=Y_0n0vkhyDU}{\texttt{ Y\_0n0vkhyDU}} \\
約翰一書 3:11-24 & \hyperref[sec:K8E95o7ZcvU]{愛的後果, 神前安穩 (約翰一書3\_11-24) - 徐武豪博士} & 2025-01-11 & \href{https://youtube.com/watch?v=K8E95o7ZcvU}{\texttt{ K8E95o7ZcvU}} \\
\multicolumn{4}{c}{} \\
\multicolumn{4}{c}{\hyperref[ch:preacher2]{李思敬}} \\
\multicolumn{4}{c}{} \\
\hline
創世記 1:26-28 & \hyperref[sec:cau3XNPSx68]{「形象」和「配偶」 (創世記1\_26-28) - 李思敬博士【繁簡字幕翻譯 by Johnson Ng】《五經》中的性別神學講道系列 - (第1講)} & 2025-02-16 & \href{https://youtube.com/watch?v=cau3XNPSx68}{\texttt{ cau3XNPSx68}} \\
出埃及記 40:34-35 & \hyperref[sec:YdX9gstJs1g]{上主榮光充滿帳幕 (出埃及記40\_34-35) - 李思敬博士【繁簡字幕 by Johnson Ng】「認識神的榮耀」講道系列 - (第1講)} & 2025-02-05 & \href{https://youtube.com/watch?v=YdX9gstJs1g}{\texttt{ YdX9gstJs1g}} \\
詩篇 148:1-14 & \hyperref[sec:l8BFPdIOADs]{合神心意的讚美 (詩篇148\_1-14) - 李思敬博士【繁簡字幕翻譯 by Johnson Ng】} & 2025-01-25 & \href{https://youtube.com/watch?v=l8BFPdIOADs}{\texttt{ l8BFPdIOADs}} \\
\multicolumn{4}{c}{} \\
\multicolumn{4}{c}{\hyperref[ch:preacher3]{楊慶球}} \\
\multicolumn{4}{c}{} \\
\hline
啟示錄 2:18-29 & \hyperref[sec:fJrsPMmDHtU]{紛亂中的堅守︰推雅推喇教會 (啟示錄2\_18-29) - 楊慶球博士} & 2025-02-09 & \href{https://youtube.com/watch?v=fJrsPMmDHtU}{\texttt{ fJrsPMmDHtU}} \\
\multicolumn{4}{c}{} \\
\multicolumn{4}{c}{\hyperref[ch:preacher4]{蕭壽華}} \\
\multicolumn{4}{c}{} \\
\hline
希伯來書 3:12-4:13 & \hyperref[sec:toZa1ewaUWE]{以信心與所聽見的道配合(希伯來書3\_12-4\_13) - 蕭壽華牧師} & 2025-01-08 & \href{https://youtube.com/watch?v=toZa1ewaUWE}{\texttt{ toZa1ewaUWE}} \\
希伯來書 5:11-14 & \hyperref[sec:8LlYAk0Xlok]{嘗過天恩 卻長蒺藜 (希伯來書5\_11-14;6\_1-12) - 蕭壽華牧師} & 2025-01-04 & \href{https://youtube.com/watch?v=8LlYAk0Xlok}{\texttt{ 8LlYAk0Xlok}} \\
\multicolumn{4}{c}{} \\
\multicolumn{4}{c}{\hyperref[ch:preacher5]{蘇穎睿}} \\
\multicolumn{4}{c}{} \\
\hline
約翰福音 15:11-17 & \hyperref[sec:prT7wwZLltI]{新與變 (約翰福音15\_11-17) - 蘇穎睿牧師 [約翰福音研讀 - 第61講]} & 2025-01-03 & \href{https://youtube.com/watch?v=prT7wwZLltI}{\texttt{ prT7wwZLltI}} \\
約翰福音 15:18-25 & \hyperref[sec:GDV7iT9TooA]{計算代價 (約翰福音15\_18-25) - 蘇穎睿牧師 [約翰福音研讀 - 第62講]} & 2025-01-08 & \href{https://youtube.com/watch?v=GDV7iT9TooA}{\texttt{ GDV7iT9TooA}} \\
約翰福音 15:26-27 & \hyperref[sec:pF1FrHKEPww]{認識聖靈 (約翰福音15\_26-27;16\_5-15) - 蘇穎睿牧師 [約翰福音研讀 - 第63講]} & 2025-01-17 & \href{https://youtube.com/watch?v=pF1FrHKEPww}{\texttt{ pF1FrHKEPww}} \\
約翰福音 16:12-16 & \hyperref[sec:M4alGuubf1o]{啟示與真理(約翰福音16\_12-16) - 蘇穎睿牧師 [約翰福音研讀 - 第64講]} & 2025-01-28 & \href{https://youtube.com/watch?v=M4alGuubf1o}{\texttt{ M4alGuubf1o}} \\
約翰福音 16:16-24 & \hyperref[sec:HaGDtN4u47U]{那等候的日子 (約翰福音16\_16-24) - 蘇穎睿牧師 [約翰福音研讀 - 第65講]} & 2025-02-02 & \href{https://youtube.com/watch?v=HaGDtN4u47U}{\texttt{ HaGDtN4u47U}} \\
約翰福音 16:25-33 & \hyperref[sec:fV_h6TniAkc]{信與覺 (約翰福音16\_25-33) - 蘇穎睿牧師 [約翰福音研讀 - 第66講]} & 2025-02-09 & \href{https://youtube.com/watch?v=fV_h6TniAkc}{\texttt{ fV\_h6TniAkc}} \\
約翰福音 17:1-5 & \hyperref[sec:wiDRWRXrtjM]{榮歸 (約翰福音17\_1-5) - 蘇穎睿牧師 [約翰福音研讀 - 第67講]} & 2025-02-13 & \href{https://youtube.com/watch?v=wiDRWRXrtjM}{\texttt{ wiDRWRXrtjM}} \\
\multicolumn{4}{c}{} \\
\multicolumn{4}{c}{\hyperref[ch:preacher6]{袁惠鈞}} \\
\multicolumn{4}{c}{} \\
\hline
撒母耳記上 16:1-17:58 & \hyperref[sec:OtTM_EdQEtA]{如何勝過生活中的巨敵 (撒母耳記上16\_1-17\_58) - 袁惠鈞牧師[大衛傳系列 - 第2講]} & 2025-01-15 & \href{https://youtube.com/watch?v=OtTM_EdQEtA}{\texttt{ OtTM\_EdQEtA}} \\
撒母耳記上 18:1-19:24 & \hyperref[sec:9t69tF6ci0k]{神啊!求你救我脫離惡人 (撒母耳記上18\_1-19\_24) - 袁惠鈞牧師[大衛傳系列 - 第3講]} & 2025-01-22 & \href{https://youtube.com/watch?v=9t69tF6ci0k}{\texttt{ 9t69tF6ci0k}} \\
撒母耳記上 20:1-21:15 & \hyperref[sec:rN0dS2BBBmc]{神祝福危難中的謊言嗎?  (撒母耳記上20\_1-21\_15;22\_6-19) - 袁惠鈞牧師[大衛傳系列 - 第4講]} & 2025-02-05 & \href{https://youtube.com/watch?v=rN0dS2BBBmc}{\texttt{ rN0dS2BBBmc}} \\
撒母耳記上 22:1-5-20-23 & \hyperref[sec:WCt7vYrgwVY]{走出憂鬱與黑暗的秘訣 (撒母耳記上22\_1-5,20-23) - 袁惠鈞牧師[大衛傳系列 - 第5講]} & 2025-02-12 & \href{https://youtube.com/watch?v=WCt7vYrgwVY}{\texttt{ WCt7vYrgwVY}} \\
撒母耳記上 23:1-24:22 & \hyperref[sec:GqTOPwqfjwM]{不可伸手害神的受膏者! (撒母耳記上23\_1-24\_22) - 袁惠鈞牧師[大衛傳系列 - 第6講]} & 2025-02-19 & \href{https://youtube.com/watch?v=GqTOPwqfjwM}{\texttt{ GqTOPwqfjwM}} \\
羅馬書 8:28-39 & \hyperref[sec:9ORA5941xxk]{永不能與主的愛隔絕 (羅馬書8\_28-39) - 袁惠鈞牧師[羅馬書系列 - 第22講]} & 2025-01-03 & \href{https://youtube.com/watch?v=9ORA5941xxk}{\texttt{ 9ORA5941xxk}} \\
撒母耳記上使徒行傳 13:14 & \hyperref[sec:w_ajWsBZ9eQ]{大衛:最合神心意的人 (撒母耳記上13\_14, 使徒行傳13\_22) - 袁惠鈞牧師[大衛傳系列 - 第1講]} & 2025-01-10 & \href{https://youtube.com/watch?v=w-ajWsBZ9eQ}{\texttt{ w-ajWsBZ9eQ}} \\
\multicolumn{4}{c}{} \\
\multicolumn{4}{c}{\hyperref[ch:preacher7]{許樹源}} \\
\multicolumn{4}{c}{} \\
\hline
馬太福音 11:2-19 & \hyperref[sec:yRzXvTTOZfM]{當公義被踐踏時  (馬太福音11\_2-19;14\_1-14) - 許樹源教授【繁簡字幕 by Johnson Ng】} & 2025-02-10 & \href{https://youtube.com/watch?v=yRzXvTTOZfM}{\texttt{ yRzXvTTOZfM}} \\
\multicolumn{4}{c}{} \\
\multicolumn{4}{c}{\hyperref[ch:preacher8]{賴若瀚}} \\
\multicolumn{4}{c}{} \\
\hline
出埃及記 32:1-14 & \hyperref[sec:srCkvhUNl9w]{扭曲的敬拜, 代求的更新 (出埃及記32\_1-14) -  賴若瀚牧師} & 2025-01-11 & \href{https://youtube.com/watch?v=srCkvhUNl9w}{\texttt{ srCkvhUNl9w}} \\
\multicolumn{4}{c}{} \\
\multicolumn{4}{c}{\hyperref[ch:preacher9]{陳恩明}} \\
\multicolumn{4}{c}{} \\
\hline
約翰福音 4:23 & \hyperref[sec:lTGVgidxHms]{尋人啟示 (約翰福音4\_23) - 陳恩明牧師} & 2025-01-28 & \href{https://youtube.com/watch?v=lTGVgidxHms}{\texttt{ lTGVgidxHms}} \\
\multicolumn{4}{c}{} \\
\multicolumn{4}{c}{\hyperref[ch:preacher10]{黃紹權}} \\
\multicolumn{4}{c}{} \\
\hline
馬太福音 27:1-10 & \hyperref[sec:499K9je19EI]{蛇鼠一窩的假敬虔 (馬太福音27\_1-10) -  黃紹權牧師 [馬太福音信息系列 - 第143講]} & 2025-01-08 & \href{https://youtube.com/watch?v=499K9je19EI}{\texttt{ 499K9je19EI}} \\
馬太福音 27:11-23 & \hyperref[sec:ZN4O4BAmHMA]{沒有"不關我們的事"(馬太福音27\_11-23) -  黃紹權牧師 [馬太福音信息系列 - 第144講]} & 2025-01-19 & \href{https://youtube.com/watch?v=ZN4O4BAmHMA}{\texttt{ ZN4O4BAmHMA}} \\
馬太福音 27:24-31 & \hyperref[sec:HaaLhKYBRSg]{一群全勝的輸家(馬太福音27\_24-31) - 黃紹權牧師  [馬太福音信息系列 - 第145講]} & 2025-01-21 & \href{https://youtube.com/watch?v=HaaLhKYBRSg}{\texttt{ HaaLhKYBRSg}} \\
馬太福音 27:33-44 & \hyperref[sec:oCpi7n8ictU]{我們/他們是怎樣的人? (馬太福音27\_33-44) - 黃紹權牧師  [馬太福音信息系列 - 第146講]} & 2025-02-02 & \href{https://youtube.com/watch?v=oCpi7n8ictU}{\texttt{ oCpi7n8ictU}} \\
馬太福音 27:45-56 & \hyperref[sec:7upP8JmD6zY]{愛護我們的比親人更知心 (馬太福音27\_45-56) - 黃紹權牧師  [馬太福音信息系列 - 第147講]} & 2025-02-09 & \href{https://youtube.com/watch?v=7upP8JmD6zY}{\texttt{ 7upP8JmD6zY}} \\
馬太福音 27:57-61 & \hyperref[sec:MfR5_HAo14I]{沉實的門徒  (馬太福音27\_57-61) - 黃紹權牧師  [馬太福音信息系列 - 第148講]} & 2025-02-13 & \href{https://youtube.com/watch?v=MfR5_HAo14I}{\texttt{ MfR5\_HAo14I}} \\
\end{xltabular}
}


\chapter{徐武豪}\label{ch:preacher1}
\begin{multicols}{3}
\minitoc
\end{multicols}
{ \scriptsize


\begin{xltabular}{\textwidth}{|p{0.15\textwidth} p{0.6\textwidth}|p{0.07\textwidth} p{0.1\textwidth}|}
\hline
約翰一書 2:29-3:10 & \hyperref[sec:Y_0n0vkhyDU]{愛的兒女 不辱主名 (約翰一書2\_29-3\_10) - 徐武豪博士} & 2025-01-11 & \href{https://youtube.com/watch?v=Y_0n0vkhyDU}{\texttt{ Y\_0n0vkhyDU}} \\
約翰一書 3:11-24 & \hyperref[sec:K8E95o7ZcvU]{愛的後果, 神前安穩 (約翰一書3\_11-24) - 徐武豪博士} & 2025-01-11 & \href{https://youtube.com/watch?v=K8E95o7ZcvU}{\texttt{ K8E95o7ZcvU}} \\
\hline
\end{xltabular}
}
\newpage



\section{約翰一書 2:29-3:10}
\label{sec:Y_0n0vkhyDU}
\textbf{愛的兒女 不辱主名 (約翰一書2\_29-3\_10) - 徐武豪博士}
\newline
\newline
連結: \href{https://youtube.com/watch?v=Y_0n0vkhyDU}{\texttt{ https://youtube.com/watch?v=Y\_0n0vkhyDU}} ~~~~ 語音日期: 2025-01-11 
\newline
\newline
\hyperref[sec:MfR5_HAo14I]{< < < PREV SERMON < < <}
~
\hyperlink{toc}{[返主目錄]}
~
\hyperref[ch:preacher1]{[返講員目錄]}
~
\hyperref[sec:K8E95o7ZcvU]{> > > NEXT SERMON > > >}
\newline
\newline
約翰一書 2:29-3:10
\newline
\begin{longtable}{cl}
\hline
\hline
章節 & 經文 (和合本修訂版)\\
\hline
2:29 & \begin{tabularx}{0.7\textwidth}{X} 你們若知道他是公義的,就知道凡行公義的人都是他所生的。 \end{tabularx} \\ \\ \relax
3:1 & \begin{tabularx}{0.7\textwidth}{X} 你們看父賜給我們的是何等的慈愛,讓我們得以稱為神的兒女;我們也真是他的兒女。世人不認識我們的理由,是因他們未曾認識父。 \end{tabularx} \\ \\ \relax
3:2 & \begin{tabularx}{0.7\textwidth}{X} 親愛的,我們現在是神的兒女,將來如何還未顯明。我們所知道的是:基督顯現的時候,我們會像他,因為我們將見到他的本相。 \end{tabularx} \\ \\ \relax
3:3 & \begin{tabularx}{0.7\textwidth}{X} 凡對他有這指望的,就潔淨自己,像他是潔淨的一樣。 \end{tabularx} \\ \\ \relax
3:4 & \begin{tabularx}{0.7\textwidth}{X} 凡犯罪的,就是做違背律法的事;違背律法就是罪。 \end{tabularx} \\ \\ \relax
3:5 & \begin{tabularx}{0.7\textwidth}{X} 你們知道,基督曾顯現是要除掉罪;在他並沒有罪。 \end{tabularx} \\ \\ \relax
3:6 & \begin{tabularx}{0.7\textwidth}{X} 凡住在他裡面的,不犯罪;凡犯罪的,未曾看見他,也未曾認識他。 \end{tabularx} \\ \\ \relax
3:7 & \begin{tabularx}{0.7\textwidth}{X} 孩子們哪,不要讓人迷惑了你們;行義的才是義人,正如基督是義的。 \end{tabularx} \\ \\ \relax
3:8 & \begin{tabularx}{0.7\textwidth}{X} 犯罪的是出於魔鬼,因為魔鬼從起初就犯罪。神的兒子顯現出來,是為了要毀滅魔鬼的作為。 \end{tabularx} \\ \\ \relax
3:9 & \begin{tabularx}{0.7\textwidth}{X} 凡從神生的,不犯罪,因神的道存在他裡面,他也不能犯罪,因為他是由神所生的。 \end{tabularx} \\ \\ \relax
3:10 & \begin{tabularx}{0.7\textwidth}{X} 這就顯明誰是神的兒女,誰是魔鬼的兒女了。凡不行義的,不是出於神,不愛他弟兄的,也是如此。 \end{tabularx} \\ \\
[1ex]
\hline
\hline
\end{longtable}
$^{1}$各位弟兄姊妹早晨.
沒有人說早晨的.
聽到了.
好感恩.
再一次和大家繼續講約翰一書.
今天是第三講.
今天這個題目叫愛的兒女不辱主名.
是二章二十九至三章十節.
我再提一提.
如果大家有聖經的話.
最好打開聖經.
如果有手機的話.
最好打開手機.
可以跟著看.
第一章和大家講的.
叫做愛的生活.
提到要以罪為罪.
要以罪為罪.
第二章和大家提到的.
就是愛的心智.
以神為神.
我想問一問.
有沒有辦法可以看到PPT.
看到了.
第一章.
第一章愛的生活.
愛的生活.
就是以罪為罪.
上一章.
提到.
三個認.
要對神對己.
對人認罪.
兩個不停講.
要告訴別人自己有什麼罪.
不停做.
不停做一些事.
幫助我們不要犯罪.
兩個定.
定時要見人.

$^{41}$定時要問人.
以致幫助我們離開罪.
兩個總.
總有一個人.
總有一些事.
可以改或者可以做.
第二和大家講.
愛的心智.
以神為神.
提到三個神.
安神.
靠神活出生命.
愛神而活在世上.
安神而活在主裡.
三個心.
就是要決心.
愛心 用心.
決心靠實.
愛心愛實.
用心安神.
今天講第三章.
愛的兒女.
不辱主名.
我先提一提.
大家信主到今天.
有很多都會聽過.
教會醜聞.
或者信徒失敗的見證.
我過去信主這麼多年.
都不是很多.
但不多不少.
陸陸續續都會聽到.
甚至每年都有.
都會聽到一些比較大的新聞.
有時在教會當中.
甚至在自己小組.
都會聽到.
有時在教會當中.
甚至在自己小組.
或者在自己團契.

$^{81}$或者在自己教會當中.
都聽到一些我們知道不好的事.
是不應該有的事.
有時當我聽到這些教會醜聞的時候.
或者信徒失敗的見證的時候.
我第一個反應是什麼呢?.
就是很羞決.
很難看.
我們一群基督徒.
我們稱為神的兒女.
稱為神的教會.
我們說有信仰的人.
居然會有這些醜聞.
居然會有這麼失敗的見證.
非常羞決.
但當我自己想到.
看到這些人.
想到這些人羞恥的時候.
我又想起自己都曾經很羞恥.
自己都曾經信主之後.
甚至到今天.
時不時都會犯錯.
都會犯罪.
都會做錯事.
有時當自己想起的時候.
我現在都有想起.
自己信主.
甚至想起今年.
做錯了.
做得很羞恥的事.
有時心中有愧.
有時人都會很憤怒.
為什麼我會這樣呢?.
為什麼我信了主.
我都會犯這些事.
我都會這麼無知.
甚至有時明知而犯錯呢?.
為什麼會犯錯?.
為什麼會犯罪?.
我同時會問自己.

$^{121}$為什麼可以不犯錯.
可以不犯罪?.
因為我真的不想收家.
《若行一書》二章二十九節到三章十節.
你看聖經裡面.
其實你看到.
我是特意由二十九節開始.
不是由三十一節開始.
分開有兩段.
其實在《若行一書》裡面有三段難解的經文.
在這段聖經裡面.
就有其中一段難解的.
就是後面說到.
主的就不犯罪.
但事實又不是.
所以要解釋這段經文.
為什麼我們會犯罪.
為什麼我們不犯罪呢?.
這段經文分為兩段.
大家看到PPD打出來.
二章二十九節到三章八節.
提到主的顯然.
二章二十九節到三章八節.
剛才那一章.
二章二十九節到三章十節.
分為兩段.
第一段是二章二十九節到三章八節.
主的顯然.
神的兒子陳顯然.
三章九至十節是人的表現.
神的兒女要顯出.
或者神的兒女有表現.
都可以.
主的顯然.
神的兒子陳顯然.
神的兒女應該有表現.
應該有顯出.
我們是屬於神的兒女.
第一段是主的顯然.
神的兒子陳顯然.

$^{161}$二章二十九節到三章八節.
大家看回經文.
螢幕不會顯示.
大家看回自己手上的經文.
第二節.
三章第二節.
提到未顯然.
第五節提到.
第八節提到顯然.
提到三次顯然.
主的顯然.
神的兒子的顯然.
為甚麼呢?.
提到主.
主耶穌有兩個.
未顯然.
和陳顯然.
我們現在是神的兒女.
將來如何?還未顯然.
還不知道.
但我們知道主如果顯然.
這裡提到顯然.
還未顯明.
即將來主會顯然.
但又提到.
祂曾經顯然.
祂曾經顯然也顯現出來.
祂顯然過但還要顯然.
耶穌來世上有兩次.
我們常說主再來.
主再來.
主來了一次 主還會再來.
兩次的顯然.
希伯來書第九章二十六節.
和二十八節.
提及這兩次的顯然.
提到在這個末世顯然一次.
將自己顯為祭.
將自己指主耶穌.
為了除掉罪.

$^{201}$將來要等候的.
第二次顯然.
並與罪無關.
乃是為拯救他們.
第一次來是要除掉罪.
第二次來與罪無關.
乃是為拯救他們.
為甚麼呢?第二次顯然前面說.
為拯救他們.
他們就是自己.
等候他們的人.
為甚麼要等候他們呢?.
因為他們已經信了耶穌.
罪已經除去.
所以與罪無關.
而是要拯救他們.
他們的身體會不會改變.
他們會得到最後.
最終的拯救.
所以這裡提及將來會改變.
所以兩次的顯然.
第一次顯然的目的是甚麼?.
有兩次.
第一次顯然的目的.
就是人可以成為.
神的兒女.
他要除去我們的罪.
所以在約翰福音第一章.
十二節.
那裡說凡接待他的人.
就是信他名的人.
他就賜他們權柄.
做神的兒女.
當然約翰福音第一章.
道成獨身來到我們當中.
說到我們如果信他.
神就讓我們成為他的兒女.
成為他的兒女.
在約翰福音一書第三章.
第一節第二節都提到.

$^{241}$他說到.
神給我們幾個詞.
稱為神的兒女.
我們真是他的兒女.
後面又說我們現在是神的兒女.
兒女的意思是甚麼呢?.
你是生出來.
神是我們的父.
我們是他的兒女.
所以耶穌第一次來的目的.
就是讓人可以接受他的話.
接受他的救恩.
接受他的待贖.
成為神的兒女.
我們與神恢復關係.
又受的成為兒女.
這個後果是甚麼呢?.
有兩個很重要的後果.
第一是人可以面對罪的審判.
第二是人可以脫離罪的權勢.
第一次演繹的目的.
前面有說.
希伯來書第九章.
為要拯救他們.
為要對付罪.
所以.
為要除掉罪.
除掉甚麼罪呢?.
第一是罪的審判.
當你信了耶穌.
成為神的兒女.
你將來不需要面對神的審判.
不需要有一個永遠的死.
不會有第二次的死.
你就可以與神一起.
第二就是可以.
除掉甚麼呢?.
除掉罪的權勢.
可以脫離罪的權勢.
第一就是不用面對罪的審判.

$^{281}$耶穌幫我們付了罪的代價.
我們不需要再付這個代價.
第二就是可以脫離罪的權勢.
講到有兩次的顯現.
耶穌兩次來到世上.
這裡提到的也是兩個的結果.
一個就是將來面對罪的審判.
我們無懼.
第二就是.
現今我們可以脫離罪的權勢.
我們不需要受到罪的影響.
所以同樣是有兩次.
過去和現在.
耶穌講到過去和將來.
耶穌過去來過.
將來會再回來.
我們將來可以面對罪的審判.
現今可以脫離罪的權勢.
這是耶穌第一次來的顯現目的.
所以脫離罪的權勢的意思就是.
人可以行義.
做對.
不需要做錯.
所以我們可以脫離罪的權勢.
我們可以脫離罪的權勢.
不需要做錯.
所以在約翰一書第二章29節提到.
犯行公義就是他所生的.
約翰福音第一章說.
接待他就是信哈明的人.
賜權並造神的兒女.
他所生的.
我們是神的兒女.
我們能夠行公義.
例如約翰一書第三章第六節有說.
他說凡住在他裡面就不犯罪.
第九節有說.
他說他也不能就不犯罪.
凡從神生的就不犯罪.
他也不能犯罪.

$^{321}$因為他是由神生的.
再說也是神的兒女.
當我們有神的生命.
我們就不犯罪.
我們就不可以犯罪.
所以你看到耶穌第一次來.
目的是什麼?.
就是讓我們能夠面對罪的審判.
讓我們脫離罪的權勢.
這裡也有提到.
第二次顯現的準備.
約翰一書第三章第二節提到.
他說他還未顯現.
他說他知道將來.
如果主若顯現.
或者主會顯現.
我們是會仗他.
必得見他的真體.
然後下面就說.
凡向他有這個指望就潔淨自己.
意思是.
我們如果是等候主耶穌回來.
我們相信主耶穌回來.
我們知道主耶穌回來了.
我們現在要準備.
第一個就是.
第一次顯現的目的.
我們可以脫離罪的權勢.
我們可以面對罪的審判.
第二次顯現的準備.
就是人要潔淨自己.
人要潔淨自己.
不犯罪.
所以約翰一書.
三章二至三節提到.
如果我們有這個指望.
耶穌再來的時候.
我們應該怎樣準備.
就是要潔淨自己.
所以順著耶穌不是說.

$^{361}$我們坐在這裡甚麼都不做.
我們不要有失敗的見證.
我們要有得勝的見證.
我們不要收家.
我們所謂光宗耀祖.
不要收家.
不要有辱師門.
但我們另一方面想.
大信徒會犯罪.
我們想想.
如果他說你要潔淨自己.
甚麼意思呢.
如果我說請你洗乾淨.
你就會髒.
如果我說你要小心.
不要染到.
有可能染到.
所以一方面說.
有神的心路的人不應該犯罪.
另一方面來說.
有神的人仍然會犯罪.
我們看現實中有沒有.
我剛才一開始說.
甚至到今天.
甚至到這個月.
我都有犯錯.
我都會生氣.
但老實說我又會發現.
我沒有以前那麼離譜.
我沒有以前那麼無知.
以前初初犯了錯.
自己都不知道.
還牙斬斬.
覺得自己很威風.
後來慢慢知道原來自己這樣做不對.
覺得很醜.
覺得很壞.
但我覺得比以前好很多.
最低限度不會無知無覺.
會有未知先知先覺.

$^{401}$最低限度有知有覺.
自己的知衰.
以前真的有時甚麼都不知道.
雖然我說我是不好.
但我慢慢有改變.
現實中我們發現.
我們真的會有這些問題.
真的會犯錯.
教會中都會有很多.
所以我剛才說教會醜聞.
大家都回教會.
或者不同的教會.
我去過那麼多教會.
沒有一間教會說沒有問題.
很多都有.
有時去不同地方講道.
很多團結之後.
跟弟兄姐妹聊天.
跟六姐吃飯.
他們都會跟我提到很多教會的問題.
這個不對那個不對.
總會有的.
一方面從神生的.
我們就不能犯罪.
不應該犯罪.
不可以犯罪.
另一方面他們又叫我們潔淨自己.
意思我們是會犯罪的.
所以要小心.
如果不會犯罪的話.
聖經裡面就不用寫那麼多東西.
新約書述裡面.
不用寫那麼多提醒我們.
提完又再提講完又再講.
勸完又再勸.
要小心多樣.
不要做多樣.
沒有可能做就不用提你不要做.
你早上起床.
我不用叫你起床.

$^{441}$如果你自己沒有問題.
我就不用告訴你有什麼問題.
為什麼我們還會犯罪呢.
留意第三章第一節.
他說.
你看父子今我們是何等的慈愛.
使我們得稱為神的兒女.
然後說我們也真是他的兒女.
世人所以不認識我們.
因為未曾認識他.
這裡提到.
認識我們.
認識他.
認識是有關係的.
兩樣事.
如果你認識這件事.
你就會認識這件事.
如果你明白這件事.
你就會了解這件事.
世人因為不認識神.
所以他不認識我們.
很多時候人們不欣賞基督徒.
因為他不認識神.
認識神他就不會這樣看我們.
因為他們不認識神.
所以不認識我們.
他不認識我們.
因為他們不認識神.
不單止世人有認識的問題.
基督徒也有.
一書第三章第六節提到.
凡住在他裡就不犯罪.
這裡提到不犯罪.
但後面這次提到.
凡犯罪.
即是住在他裡就不犯罪.
凡犯罪是為什麼呢.
未曾看見他.
未曾認識他.
他就是指主.

$^{481}$因為前面提到.
主若顯現.
第三節提到凡向他有者指無.
第五節又提到主曾顯現.
第六節提到凡住在他裡.
後面提到未曾看見他.
未曾認識他.
意思是.
為什麼人不認識自己呢.
因為他不認識神.
所以有些人.
他不知道自己犯罪.
他不知道自己可不可以犯罪.
為什麼會這樣呢.
因為他不認識主.
認識主的意思是.
主耶穌第一次來.
是要幫我們不犯罪.
要認識這件事.
上一堂.
我和大家分享裡面提到我的兒子.
他犯錯之後.
他出來和我說.
爸爸我知道為什麼我會犯罪.
因為我是罪人所以我要犯罪.
我說沒錯.
我們是罪人.
但某一個罪人是可以不犯罪的.
信主之前我們不可以不犯罪.
信主之後.
是可以不犯罪的.
電主耶穌的時候.
是不會再犯罪的.
所以你看第三章.
第六節提到.
他說.
凡住在他裡就不犯罪.
犯罪未曾看見他.
亦未曾認識他.
你看看下半節提到.

$^{521}$如果你認識他.
你就不會犯罪.
所以我們需要追求認識主.
認識主讓我們知道.
原來我們是可以不犯罪的.
在三章第三節.
又提到一樣東西.
凡是向他有展望就潔淨自己.
好像他潔淨一樣.
即是他告訴我們.
我們要潔淨自己.
剛才我有說.
如果要潔淨自己我們可能會不潔淨.
如果神說你要潔淨自己.
表示你是可以潔淨自己.
我再說一次.
當聖經這樣說.
凡是我們對主耶穌有展望.
即是我們相信主耶穌會再來.
是與罪無關.
我們可以脫離罪的審判.
我們相信這件事.
同時要知道我們今天可以脫離罪的權勢.
他說就潔淨自己.
凡是有這個展望.
就潔淨自己.
你可能會不潔淨.
所以你要潔淨自己.
表示你是可以潔淨自己.
所以你要潔淨自己.
希望我沒有搞亂大家.
總之意思是.
我們是潔淨.
或不潔淨.
就在於我們是否潔淨.
你是可以的.
你要認識主.
認識主第一次來幫我們.
可以不犯罪.
是脫離罪惡的權勢.

$^{561}$所以要知道.
而且我們要潔淨自己.
上次我提到.
有兩個說話.
神不會要我們做的.
也不會要我們不行的.
如果沒有的神會要我們給他.
如果不行的神會要我們做.
當時這樣說.
凡有這個展望就要潔淨自己.
表示你是可以潔淨自己.
怎樣可以呢?.
如剛才的經文說到.
要認識主.
世人不認識我們.
因為他們不認識神.
我們為何會犯罪.
因為我們不認識主.
在這裡說得很清楚.
耶穌說他既然被試探而受苦.
就能夠答應被試探的人.
其實耶穌是不會犯罪的.
為何他要被試探.
因為他可以勝過試探.
他也能夠救我們勝過試探.
他是能救的.
你要記住.
所以我們要知道.
如果神叫我們這樣做.
神不會要我們沒有的.
神不會要我們做些做不到的.
神不會叫我舉五百磅的東西.
我舉不到的.
神叫我做的事.
我做得到的.
所以當他要潔淨自己的時候.
我們是可以的.
所以他能夠答應被試探的人.
不要說沒有辦法.
有時你很怕聽人說.

$^{601}$人在江湖身不由己.
有些人甚至引用保羅說.
有誰軟弱我不軟弱呢.
我都是人而已.
經常給自己很多理由.
但聖經說得很清楚.
耶穌能夠答應.
意思是他能夠幫你.
能夠讓你做得到.
如果做不到他不會叫你做.
我們是可以不需要的.
有時我提醒自己.
有時會放棄.
有時認罪認罪.
又來到神面前不好意思又是我.
又是做這些錯.
但我自己記得神這樣說過.
我就不放棄.
我是可以贏的.
我不是不行的.
認識主.
第二節聖經在不萊書第四章十六節.
第四章十六節.
那裡怎樣說呢.
只管坦然無懼.
不用怕.
我們來到施恩保助前.
得廉恤蒙恩為.
做隨時的幫助.
有時有沒有試過.
有時遇到一些事情.
沒辦法控制自己.
這裡說隨時的幫助.
不需要等到上教會.
不需要等到祈禱會.
隨時隨地神都會幫我們.
這裡坦然無懼的意思.
不需要客氣.
我們求神幫不用客氣.
所以我記得有一次.

$^{641}$很多年前聽一個.
小朋友的詩歌集.
錄音帶.
那兩個小朋友在吵架.
吵架的時候.
有個叫做.
詩歌先生.
他就說.
你們祈禱求神吧.
兩個小朋友在吵架.
怎樣求啊.
他說教都可以求的.
隨時.
所以我們要知道.
主是能夠幫我們.
我們就不用客氣.
隨時求神幫.
神真的會幫.
我很感恩這間屋對付自己.
一些脾氣和耐性.
有時對某些人.
不是所有人.
都很溫柔.
對某些事很沒耐性.
我都要嘗試.
遇到同一類人.
我都會隨時求神.
你幫幫我.
讓我不要再犯錯.
我發現很多事都可以.
真的可以.
原來神說的話.
真的可以做到.
因為是祂的兒女.
神是會幫我們.
但要認識.
英文叫know.
認識不是單單知道.
也要有經歷.
要有經歷.

$^{681}$不是知道這件事.
而是經歷過有這經驗.
我自己就.
因為我有時和太太.
去一些地方.
我們都會看youtube.
有什麼地方好去.
有什麼好吃.
有時看了很多.
有時和人說.
對呀.
你這樣都知道.
很好吃呀.
他問我有沒有吃過.
你就不是真的認識.
你不是真的知道.
你只是風聞.
只是道聽途說.
認識神不是單單有神學知識.
三惟一體.
不是這些.
而是不單知道神是這樣的神.
而且你經歷到神是這樣的神.
就是神能夠幫我們.
不犯罪.
耶穌能夠救我們脫離斯坦.
我們能夠坦然無懼.
不用客氣的.
有隨時的幫助.
我們讀這些新聞讀了很多次.
你只是知道.
真的知道.
你不單知道.
而且是經歷有經驗的.
有經驗的.
很多時看youtuber的東西.
知道好吃但你沒吃過.
即是你不知道.
有多好用.
我很喜歡說.

$^{721}$有一句銀行的說話.
You are richer than you think.
比你想像中富有.
我們是stronger than we think.
你不要看自己這麼差.
我這件事性能過.
我這個脾氣趕不上.
幾十年了.
我這些人沒辦法接受.
我這些事沒辦法做到.
We are stronger than we think.
因為我們認識主.
他是能夠幫助我們.
他能夠做我們隨時的幫助.
所以我很盼望.
從這些經文看到.
我們要由我們以為自己不行.
到知道自己可以.
所以要對自己說一下.
我又不需要跟別人說.
你是可以的.
但你跟自己說.
有時今晚你回去.
今天你睡覺前.
或者你明天起床.
你跟神說多謝你讓我可以.
我是可以的.
由以為自己不行.
到知道自己是可以.
凡是等候他的人.
我們都要潔淨自己.
為什麼可以做到呢?.
給大家一個建議.
希伯來書十章二十四字.
十章二十四字.
彼此相顧激發愛心.
勉勵行事.
我再說一次.
彼此相顧激發愛心.
勉勵行事.

$^{761}$你背了它你知道.
但你還不知道.
所以今天大家在這裡.
我看到大概有二十多人.
你要問一問.
有誰要顧他呢?.
有誰不是激發他.
而是激發他的愛心呢?.
有誰要勉勵他行事呢?.
你要做出來才可以.
這些教導給了我們.
我們是可以做的.
你需要彼此相顧激發愛心.
勉勵行事.
我鼓勵你.
我們才有一個集體的見證.
我們才不會收價.
你提起BICFE與議會的時候.
這班人是很好的.
他們很團結的.
他們很關心的.
彼此支持.
而不是爭風合燥.
互相鬥毆.
這樣就不好了.
真的很收價.
所以要互勉同行.
有沒有見到或知道某些人.
是需要的.
你去關心他.
他需要鼓勵的你去鼓勵.
他需要勉勵的你去勉勵.
彼此的幫助.
不要互不相干.
有時問多一句.
都好像得罪你一樣.
大家不要這樣.
要互相鼓勵彼此的鼓勵.
是需要這樣做.
做一些事說一些事去幫對方.

$^{801}$最低限度為對方來代討.
我記得2014年.
有些學生.
有些牧師.
傳人來找我.
他說徐老師.
有些人叫我徐弟兄.
我們現在出來教會服事.
我們開始講道.
但有時講道的壓力超大.
有時講極都講不好.
好像力不從心.
徐老師有什麼可以幫我們呢.
所以2014年成立了.
港大學的教育委員會.
2014年成立了.
港道團契.
開始有很多傳道人.
剛出來港道.
或有些新學生.
都想學港道.
他們就來參加.
我們彼此幫.
不是我教他們的.
是這班人聚在一起.
星期一9點半至12點半.
有一個港道團契.
有14個人.
我們一邊看講章.
一邊討論.
分析.
有人會講.
有人會評.
大家互相幫助.
我們做了10年.
我發覺.
這樣來到.
大家一起可以成功.
團結就是力量.
不要被人捉過擊破.

$^{841}$我們一定要團結一起.
彼此相顧激發愛心.
勉勵行事.
我們要想想.
有什麼事我們可以做.
當年我開始到現在.
最多一年有五個港道團契.
後來覺得太多.
慢慢減少.
主的顯現.
提到神的兒子.
曾經顯現.
人可以不犯罪.
我們要彼此.
互勉同行.
雖然如果想行義利惡.
人要認識.
要知道.
第一我們要這樣.
第二我們可以這樣.
主耶穌可以幫我們.
主耶穌可以救我們.
我們大家之間.
要彼此勉勵.
所以.
三個問題給大家.
第一何人.
第二何事.
第三何事.
今天完了崇拜.
大家出來喝茶.
打電話.
發微博.
什麼都好.
有什麼人我們可以鼓勵他.
提醒他.
何時打何時找他.
例如我明天.
我這裡是晚上.
明天早上崇拜.

$^{881}$晚上六點約了幾位主教老師.
和一位傳道人.
傳道人.
下星期就安立.
我出了浮.
你們約他出來.
我也很想恭喜他.
很想祝福他.
定時間.
明天六點.
有時間做什麼事.
跟他說什麼.
我們要這樣做.
人與人之間.
如果沒有互動.
沒有互勉.
我們很弱.
人有互勉就強很多.
有試過行山嗎.
一起行.
可以彼此鼓勵.
想著行不到.
我試過行長城.
你們也應該行過長城.
想去那裡.
很遠的.
一起行會行到.
想想什麼人.
什麼時間.
能夠彼此互勉.
讓我們能夠潔淨自己.
不犯罪.
第二就是人的表現.
神的異類要顯出.
第九節至第十節.
二章二十九節至三章八節.
剛才說有三次顯現.
三章九節十節.
有一次在第十節.
顯出.

$^{921}$從此就顯出.
誰是神的異類.
從此就顯出.
在那裡.
在經文說.
凡從神身的就不犯罪.
神的存在於心裡.
他也不能犯罪.
由神身的人.
我們的生命不能犯罪.
但因為耶穌還未顯現.
我們裡面仍然有舊的生命.
我們神的生命不能犯罪.
但舊的生命是可以犯罪的.
是可以犯罪的.
由神身的.
神的生命是不能犯罪的.
不能犯罪的.
所以當你要顯出.
你是否屬神的呢.
還是不屬神呢.
就看你的表現.
所以他說.
從不犯罪能顯出誰是神的異類.
誰是魔鬼.
魔鬼的異類就會犯罪.
凡不行義的就不屬神.
不愛的已經廿四年前.
如果一個人不行義又沒有愛心.
神就是愛.
神就是光.
光的意思就是沒有黑暗.
愛就是有愛人的心.
愛人的表現.
這些愛的表現.
行義的表現.
就顯出我們是神的異類.
顯出我們是神的異類.
所以這裡說到.
神的兒子曾顯現.

$^{961}$但神的異類要顯出.
顯出誰是神的異類.
很多年前.
有一個.
我不記得是弟兄還是姐妹.
他跟我說.
我去聽了一個很好的.
是某一個神學院的前院長.
我說你自己去聽吧.
他介紹了給我.
我上去聽.
是真的很好.
題目很吸引.
這是誰家的孩子.
這是哪一個家的子女.
他說什麼呢.
他說的例子.
我大概記得的例子.
就是看到一些小孩很乖.
很乖.
很有禮貌.
很樂意助人.
看到的表現.
有些人看到這些小孩.
就好奇想知道.
他們的父母是誰.
怎麼可以教出這麼好的子女.
不知道有沒有試過.
有人會問.
有人問.
你是他們的父母嗎.
你們的子女真是很乖.
你怎麼教他們.
那位牧師說.
我們是神家的兒女.
我們要顯出我們是屬神.
不要讓人說.
你哪個家的子女.
這麼沒家教.
這麼沒禮貌.

$^{1001}$千萬不要這樣.
我們要顯出是神的兒女.
不是魔鬼的兒女.
不要有神的生命.
活出魔鬼的生活.
有時真是很恐怖.
有神的生命居然活出魔鬼的生活.
為什麼基督徒都會這樣.
可以說出這樣的話.
可以做出這樣的話.
真是很難相信.
剛才我說的.
神的生命是不可以犯罪.
人的生命是可以犯罪.
所以我們一定要活出神的生命.
顯出我們是屬神的人.
屬神的兒女.
馬克思第七章16-20節說.
憑著果實認出他們.
說了兩次.
認出他們.
有時我想問.
我們在外面有沒有顯出我們是神的兒女.
人認不出我們是神的兒女.
認出我們是神的兒女.
不要羞恥.
八月份.
我不知道大家有沒有看奧運會.
我看了.
很少看城牆比賽.
但奧運會.
我記得.
除了荃紅城.
鄭欽文.
潘子洛.
大家都知道.
你們知不知道.
你們在不在.
誰是潘子洛.
知道嗎.

$^{1041}$對不起.
我收不到聲.
我很記得訪問這些運動員.
有些是贏的.
有些是贏不到金牌銀牌.
贏不到獎牌.
有些是贏不到.
有句說話我聽到.
大家可能聽過.
其實我穿著這件衣服.
前面的國旗比背後的名字更重要.
我們會看到.
他們很看重國家榮辱.
不是說一定要贏.
不贏不要緊.
但體育精神有盡力而為.
表現出應該表現的.
應該有拼搏的鬥志.
不要羞辱自己的國家.
我記得有一次我看一個.
我記得在奧運會播放以前奧運的片段.
說到有一個田徑的長跑的運動員.
跑到最後他不知為何.
可能抽筋.
可能扭傷.
跌倒在地上.
但他沒有趴在地上.
他堅持站起來.
跑過去.
這是一樣的.
雖然他輸了是最後一個過終點.
但全場拍手.
當看到的時候.
他尊重他的國家.
尊重他成為一個運動員的身份.
他要跑過他的賽事.
不贏不要緊.
但他需要跑過他的賽事.
我們在這個世上是代表神.
在哥林多前書四章第九節說.

$^{1081}$我們成了一台戲.
人看著我們.
可能外人看不到你們.
當你離開這個地方.
走出這個地方.
人們看著我們基督徒.
你們是信教的.
信教的.
你是教徒.
你不要聽別人說.
不是吧.
你這樣的人也可以說是信教的.
你這樣的人也可以說是基督徒.
千萬不要這樣.
2015年差不多九年前.
2月6日.
浙江衛視有個綜藝節目.
叫做我看你有戲.
不知道有沒有人看過.
我看你有戲.
有戲的戲就是戲劇的戲.
我看你有戲.
幾個評委.
馮小剛 張國立.
成龍 李冰冰.
這個節目2月6日啟播.
但5月2日.
真的不到3個月就停播.
沒戲看.
真的很諷刺.
我看你有戲 最後你自己沒戲.
沒戲給別人看.
信徒不要有辱主名.
怎樣可以做到呢.
我們看回第三章第六節.
剛才其實也讀了一點.
犯罪在他裡面.
就不犯罪.
犯情神心就不犯罪.
為什麼不犯罪 因為神的道.

$^{1121}$存在他心裡.
所以這段經文的難解.
犯情神心都不犯罪.
為什麼我們會犯罪呢.
為什麼會犯罪呢.
這裡提到住在他裡面就不犯罪.
犯情神心就不犯罪.
因為神的道存在他心裡.
兩個裡面.
一個是住在主裡面.
一個是存在心裡面.
在上一堂.
我講了上一段經文.
有四次在裡面.
在第二章.
二十四節.
一直到第二十八節.
講到住在子裡面.
住在父裡面.
住在主裡面.
講了四次.
我上一次也提到.
裡面的意思.
我覺得在自由當中要馴服.
我講了兩句話.
就是沒有沒有自由的馴服.
也沒有沒有馴服的自由.
我記得上次我解釋過.
馴服裡面.
馴服不表示你沒有自由.
自由不表示你不用馴服.
用平和六燈做一個比喻.
所以第六節講到住在裡面.
大家說要留在裡面.
你要住在裡面.
就是要意志的操練.
要的是節制.
住在裡面.
什麼意思呢.
就是範圍.

$^{1161}$你不要過界.
範圍裡面.
不要出界.
要留在裡面就要有節制.
意志的操練.
所以我們說我們怎樣能夠不犯罪呢.
我就跟大家講.
就是要操練你的意志.
生活世界操練你的意志.
很簡單 衣食住行.
我今天和我大姐去吃東西.
和她說一個親戚.
她說這個親戚是不受控制的.
買東西不受控制的.
她這次過來.
和親戚一起過來.
我大姐拿了一個大行李箱.
這個親戚拿了五個大行李箱.
做什麼呢.
她說買東西.
我大姐問她.
她已經走了.
她說買這麼多東西有用嗎.
她說沒有用的.
為什麼買呢.
便宜啊 減價啊.
怎可以不買呢.
她說控制不了.
買了五個行李箱回去.
當然她有人和她一起.
生活上操練意志.
一個不犯罪的人.
是要有意志力.
是要有節制的.
所以可以操練.
例如開車.
我都很喜歡開快車.
我經常提醒我小心一點.
不要超速.
經常和自己說超速一點點不要緊.

$^{1201}$慢慢我老婆在那裡勸我不要.
她覺得這是一個很好的操練.
我不知道你們那裡是用公里還是英里.
我們這裡是用公里.
有些地區是40公里.
我真的很慢.
我最近看到一個是30公里.
我記得當我行的時候.
我說很浪費時間.
前面沒有車.
為什麼開這麼慢呢.
我提醒自己.
不如操練一下自己的意志.
一直保持著30公里.
或者40公里.
一般我們站在那裡.
我們都會有一個人在那裡.
我發現這裡有一個很好的操練.
我記得昨天.
今天是星期五.
我們去我七姐的家.
我們吃Potluck.
大家一起吃.
我看到很多菜.
十多種不同的菜.
有些我很喜歡吃.
操練節制.
你很喜歡吃.
一件起兩件事.
不要太狼.
我都是一種操練.
有時說話也是.
我都要操練說話.
我自己有學習.
有時說話越說越興奮.
說得語無倫次都可以.
懂得閉嘴就閉嘴.
應該開口就開口.
這些都是操練.
所以我要住在主裡面.

$^{1241}$我要順服神.
操練自己的意志.
在日常生活裡鼓勵大家操練意志.
第二就是留在裡面.
留在裡面.
留在甚麼裡面呢?.
神的道留在裡面.
所以我們要看重神的話.
我們才有希望.
神的話不是道理.
神的話有生命.
當你真的看重神的話.
神會給我們力量.
所以我們要逆性而行.
甚麼意思呢?逆性而行.
我們不聽神的話.
人的本性是不看重的.
有沒有發現?.
很多東西很看重.
看韓劇.
看奧運.
金晶火顏.
在教會聽道.
在教會查經.
真的污微吸碎.
我們要看重神的話才有機會.
神的話不留在裡面.
你又不留在主裡面.
那你有沒有辦法不犯罪呢?.
你不犯罪.
但你需要住在主裡面.
也需要讓神的話留在你裡面.
所以.
我們真的需要.
對弟兄姊妹.
我經常和教會說.
對教會有要求.
所以有時我做多一點.
例如今次和大家講讓律書.
我就給大家一個結構圖.

$^{1281}$希望大家看.
我已經不過份.
在神學院.
我已經有一個結構.
原來可以背.
整個月可以背下來.
要有要求才會成長.
才能站穩.
所以對聽道有要求.
對講道也有要求.
我對自己也是.
今年我講道有些改變.
我以前很少這麼詳細.
做這些PPT.
或者做這些大綱.
今年要求多一點.
為什麼這樣做?.
因為我們要站穩才有機會勝過罪.
所以個人要想想.
我們個人自己要想想.
我們怎樣能夠住在主裡面.
怎樣節制紀律.
不過份不放肆.
群體要想想.
我們怎樣整體.
整個禮堂.
怎樣看好神的話.
以致能夠留住神的話.
在我們心裡.
不要聽完走出去就忘記了.
所以整段聖經說.
信道是應該不犯罪.
亦提醒我們怎樣可以不犯罪.
個人要住.
群體要立.
主臣顯現.
神的兒子.
神的兒子曾經顯現.
人可以不犯罪.
要彼此互勉同行.

$^{1321}$剛才說過.
大家要想一想三件事.
何人何事何事.
想想要和什麼人.
什麼時候說什麼.
要問這三個問題.
人可以不犯罪.
要彼此互勉同行.
另外主臣顯現.
人要有表現.
人應該不犯罪.
個人要操練意志.
能夠住在主裡面.
群體要有所要求.
以致神的道能夠留在我們裡面.
所以作為一個教會.
整體來說都要想想.
怎樣能夠幫助基督徒.
群體也好.
可以謹記神的話.
怎樣幫助他們.
哪方面我們可以操練.
哪方面我們應該加重.
都要問這個問題.
我們怎樣操練意志.
為了看主臣的話.
而加重我們的要求.
讓病人才會記得.
我們聽了什麼.
我們要做什麼.
所以今天的題目叫「愛的兒女」.
愛的兒女就不辱主明.
如果你真的愛神的話.
不要羞辱主的明.
不要羞辱神的明.
要互勉同行.
能夠住能夠留.
不辱主明.
因為我們是神的兒女.
所以今天講完之後.

$^{1361}$大家回去看回大綱.
溫述今天所講的.
你可以上網聽.
你可以看回結構圖.
看回經文.
我們住在你裡面.
直接住在你裡面.
以致我們能夠住在主的裡面.
像楊方所講.
我們住在他裡面.
他的道也住在我們裡面.
兩樣的.
我們住在主的裡面.
神的道也住在我們裡面.
我們就能夠從神生而不凡進.
我們不辱主明.
我們不會成為一群羞家的兒女.
希望大家能夠記住.
有很多問題大家剛才提到.
什麼人什麼時間.
做什麼或說什麼.
我們有什麼需要操練.
有什麼可以加重.
以致我們能夠真正不辱主明.
好我們一起祈禱.
我們求你幫助我們剛才聽的.
能夠藏在心裡.
讓我們知道主神顯現.
我們要顯出.
我們要顯出主神顯現.
我們要有表現.
以致人認得出我們是屬於你的.
我們不辱主明.
因為我們是愛的兒女.
我們愛我們的神.
我們希望我們能夠榮耀我們的神.
幫助我們個人群體.
個人要操練群體要護面.
我們要將神的道存在心中.
以致我們能夠存在主的裡面.

$^{1401}$我們不犯罪.
我們能夠潔淨自己.
\newpage



\section{約翰一書 3:11-24}
\label{sec:K8E95o7ZcvU}
\textbf{愛的後果, 神前安穩 (約翰一書3\_11-24) - 徐武豪博士}
\newline
\newline
連結: \href{https://youtube.com/watch?v=K8E95o7ZcvU}{\texttt{ https://youtube.com/watch?v=K8E95o7ZcvU}} ~~~~ 語音日期: 2025-01-11 
\newline
\newline
\hyperref[sec:Y_0n0vkhyDU]{< < < PREV SERMON < < <}
~
\hyperlink{toc}{[返主目錄]}
~
\hyperref[ch:preacher1]{[返講員目錄]}
~
\hyperref[sec:cau3XNPSx68]{> > > NEXT SERMON > > >}
\newline
\newline
約翰一書 3:11-24
\newline
\begin{longtable}{cl}
\hline
\hline
章節 & 經文 (和合本修訂版)\\
\hline
3:11 & \begin{tabularx}{0.7\textwidth}{X} 我們要彼此相愛。這就是你們從起初所聽到的信息。 \end{tabularx} \\ \\ \relax
3:12 & \begin{tabularx}{0.7\textwidth}{X} 不要像該隱;他是屬那邪惡者,殺了自己的弟弟。為甚麼殺了他呢?因為自己的行為是邪惡的,而弟弟的行為是正直的。 \end{tabularx} \\ \\ \relax
3:13 & \begin{tabularx}{0.7\textwidth}{X} 弟兄們,世人若恨你們,不要驚訝。 \end{tabularx} \\ \\ \relax
3:14 & \begin{tabularx}{0.7\textwidth}{X} 我們知道,我們已經出死入生了,因為我們愛弟兄。沒有愛心的,仍住在死中。 \end{tabularx} \\ \\ \relax
3:15 & \begin{tabularx}{0.7\textwidth}{X} 凡恨自己弟兄的,就是殺人的;你們知道,凡殺人的,沒有永生住在他裡面。 \end{tabularx} \\ \\ \relax
3:16 & \begin{tabularx}{0.7\textwidth}{X} 基督為我們捨命,我們從此就知道何為愛;我們也當為弟兄捨命。 \end{tabularx} \\ \\ \relax
3:17 & \begin{tabularx}{0.7\textwidth}{X} 凡有世上財物的,看見弟兄缺乏,卻關閉了惻隱的心,神的愛怎能住在他裡面呢? \end{tabularx} \\ \\ \relax
3:18 & \begin{tabularx}{0.7\textwidth}{X} 孩子們哪,我們相愛,不要只在言語或舌頭上,總要以行為和真誠表現出來。 \end{tabularx} \\ \\ \relax
3:19 & \begin{tabularx}{0.7\textwidth}{X} 從這一點,我們會知道,我們是出於真理的,並且我們在神面前可以安心, \end{tabularx} \\ \\ \relax
3:20 & \begin{tabularx}{0.7\textwidth}{X} 即使我們的心責備自己,神比我們的心大,他知道一切。 \end{tabularx} \\ \\ \relax
3:21 & \begin{tabularx}{0.7\textwidth}{X} 親愛的,我們的心若不責備我們,在神面前就可以坦然無懼了。 \end{tabularx} \\ \\ \relax
3:22 & \begin{tabularx}{0.7\textwidth}{X} 我們一切所求的,就從他得著,因為我們遵守他的命令,行他所喜悅的事。 \end{tabularx} \\ \\ \relax
3:23 & \begin{tabularx}{0.7\textwidth}{X} 神的命令就是:我們要信他兒子耶穌基督的名,並且照他所賜給我們的命令彼此相愛。 \end{tabularx} \\ \\ \relax
3:24 & \begin{tabularx}{0.7\textwidth}{X} 遵守神命令的,住在神裡面,而神也住在他裡面。從這一點,我們知道神住在我們裡面,這是由於他所賜給我們的聖靈。 \end{tabularx} \\ \\
[1ex]
\hline
\hline
\end{longtable}
$^{1}$各位大兄姊妹平安.
我都聽到有人說.
今天是跟大家分享第四課.
今天的題目是愛的後果.
神前安穩.
我們先做個祈禱.
天父我們求主保守我們眾人.
我們來到你的面前.
來到電的當中.
我們來敬拜.
我們來聆聽你的話語.
求主你幫助我們.
在你面前不單止口唱心和.
而且真的要言行合一.
不單止我們唱出有關愛的詩歌.
而且能夠活出愛的生活.
幫助我們在愛的群體當中.
真的有愛的實質在當中.
讓我們過去不認識.
今天在主裡成為一家人.
能夠彼此切實的相愛.
幫助我們從你話語當中有提醒有學習.
也有鼓勵.
從你話語當中得到指引.
能夠有方向.
能夠知道該走的路該行的事.
帶領我們今天所要聽的道.
幫助我們能夠明白.
而且能夠遵行.
聽我們祈禱.
奉耶穌的名祈禱.
阿們.
剛才一路唱詩歌的時候.
很多都是關於愛的詩歌.
想起我自己過去.
和想起自己今天的改變.
真的覺得是一個很大的奇蹟.
愛的特別就是能夠顧及別人.
過去自己未信主之前.
可以說只顧自己不顧人.

$^{41}$可以選人利己.
今天學習如何為人捨己.
是不容易的.
但是有信仰的人能夠有愛的表現.
所以我也很感恩.
這次選擇說約翰書信.
能夠再一次重溫.
確定愛的重要性.
我發現人生當中有很多矛盾和掙扎.
我不知道大家有沒有.
很多時候有些事情不想做.
但又去做.
或者知道不應該做.
又去做.
我們叫做心得猶己.
有時候我記得有個姐妹跟我說.
每每看到有些零食.
她忍不住會買.
也忍不住會吃.
不但會買會吃.
而且會吃完.
她說沒辦法禁止自己.
她說人在江湖身不由己.
我說這樣也可以.
有些事情知道要做.
很想做但又做不到.
我們就叫力不從心.
很多事情是應該做.
但又做不到.
我記得有一次我去參加一個健身會.
健身館的健身會.
教練叫我做一些動作.
我也很想做.
但真的做不到.
很簡單的一個動作.
例如彎腰碰到地.
我盡力了.
最後也碰到.
為什麼碰到呢.
我屈膝了.

$^{81}$完全碰到.
手掌貼平了.
那些叫做出貓.
人生中很多時候我們知道很多事情.
但做不到.
我們知道要愛.
但有時做不到.
明知道愛是好的.
但沒去做.
而得不到應該有的結果.
我又見到有很多人.
有時跟別人說.
叫別人不要禁.
但又不聽.
常常在人生當中.
覺得很不開心.
其實教會裡面有說.
應該怎樣做人.
但很多時候人眼裡聽不進.
有時聽得進就做不出.
所以我記得在之前講道德.
我跟大家講過.
有很多人你講給他聽.
他又聽不明白.
他明白的時候又不做.
做的時候又會做錯.
做錯的時候又不認.
認的時候又不改.
改又改不久.
有見過這樣的人嗎.
令你很辛苦.
其實他自己也辛苦.
改來改去改不到.
身不由己力不從心.
人生當中很多事我們知道.
但沒做.
或是知道我們做不到.
這是人生中一個很大的掙扎.
有時對自己也很煩躁.
為何會這樣.

$^{121}$為何做不到.
為何記不住.
我記得剛參加一個婚禮.
是我的姪女.
她叫我幫她主持婚禮.
我幫她主持婚禮.
我跟她們說.
大家都知道婚禮有誓言.
雙方互相承諾.
講誓言.
每一次聽到.
每一次婚禮的誓言.
都聽到我很感動.
真的很感動.
兩個講道真的很偉大.
但我又發現.
誓言講到多偉大也好.
不表示婚姻就會美滿.
我就練習了.
我見現在在在有些人很年輕.
可能不認識一個人.
叫吳惠國.
你們應該不認識吧.
有人認識的.
那些年紀比較大.
有個電視片集.
叫無山盟.
裡面有首歌.
叫海誓山盟誰能守.
這句可能有些人有印象.
海誓山盟誰能守.
講而已.
講就講.
很多人都能夠講.
我要為你點上刀山下油鍋.
到真是發生.
第一個走就是他.
走都走不前.
講愛很容易.
真的要做出來.

$^{161}$真的不容易.
約翰一書有兩條主線.
神就是光.
神就是愛.
基督徒要行在光中.
活在愛中.
如果愛沒有光.
濫愛.
如果有光而無愛.
這叫殘酷.
只有光而無愛.
只有愛而無光都不行.
人生是兩條主線.
有光有愛.
有愛有光.
人生才是完美神所喜悅的人生.
前三童和大家講.
要以罪為罪.
以神為神.
不辱主名.
不辱主名可以改做.
不辱神家.
今天看的叫.
愛的後果.
前安穩.
今天的人很多時候不是很穩.
你發現.
特別是這幾年我發現.
資訊傳媒.
很多東西.
令我們.
不知道怎麼做決定.
誰是誰非.
誰真誰假.
什麼是黑什麼是白.
我們都分不清楚.
甚至不知道何去何從.
我們今天看一看.
在約翰的書.
神的命令.

$^{201}$告訴我們人的需要.
以致人可以在神的面前.
能夠安穩.
不慌也不亂.
今天很多人真的很慌.
現在的人很多憂慮.
今天的人很亂.
經常都不知道.
什麼是對應該怎麼做.
我們希望今天能夠看到.
人可以怎樣在神前安穩.
不慌也不亂.
三章一.
十一至二十四節.
大家如果有聖經可以看.
可以分成四段.
十一節講到愛的命令.
十二至十五節.
講到愛的問題.
十六至十八節講到愛的表現.
十九至二十四節.
就提到愛的結果.
或者愛的後果.
我們先看愛的命令.
先看愛的命令.
我發現我看不到.
powerpoint的.
你們看不看到的.
看到是嗎?我看不到.
anyway我們就看愛的命令.
多謝多謝.
看不到會中.
真的很矛盾.
愛的命令就是守不守.
不斷的成長.
你發現在.
十一節講.
我們應當彼此相愛.
這就是你們從起初.
所聽見的命令.

$^{241}$起初所聽見的命令.
起初是指什麼時.
就是指約翰福音.
十三章三十四節.
所講的.
九六起約翰福音.
十三章三十四節.
那裡說什麼呢.
耶穌說我賜給你們一條新的命令.
就是叫你們彼此相愛.
我怎樣愛你們.
你們都要怎樣相愛.
在約翰一書第二章.
七至八節都有提過.
這個命令.
他說我寫給你們不是一條新的命令.
是從起初.
所受的舊命令.
這個舊命令你們所聽見的道.
再者寫給你們一條新的命令.
在上一節的課程裡提到.
到底這個是新還是舊的呢.
其實命令.
是已經給了.
但如果沒有守的話.
仍然是一個新的命令.
是一個新的命令.
在第八節.
在第一章.
第八節.
有提到.
應該是約翰一書.
第一章第七至第八節.
第一章第八節.
有提到我寫給你們一條新的命令.
在主是真的.
在你們也是真的.
漸漸過去.
真光已經照耀.
什麼意思呢.

$^{281}$漸漸過去真光已經照耀.
是一個過程.
我們守這個命令.
不是一天.
不是一次就守完.
而是不斷慢慢來.
有很多這個命令裡.
有很多部分.
很多方面.
很多內容是我們沒有守的.
沒有守或未守好.
對我們來說.
都是一個新的命令.
都是一個新的嘗試.
可以說是新又可以說是舊.
有些你未做的.
是新的.
有些你做了的是舊.
但未做好的.
那你要繼續做.
仍然有可以做的地方.
所以這是一個過程.
守命令.
或者愛是一個過程.
沒有人夠膽說.
我已經愛盡了.
除非你已經死了.
大家坐在這裡當然是未死.
你一日活著.
仍然有愛的成長的可能.
仍然有愛的進步的空間.
所以是不斷在.
第一章第八節說.
黑暗漸漸過去.
真光已經照耀.
黑暗是漸漸過去.
好像我剛才說的.
我這個人以前是很自私.
是否完全不自私呢.
有時也會的.

$^{321}$有時出去吃飯.
見到我最喜歡的一碟菜.
或者我們吃魚的面豬燉.
有時我忍不住.
很快就夾了它.
但會留在另一邊給別人.
也有少少自私.
因為只有兩塊.
有時也會有某些方面.
自私的表現.
但我要慢慢改變.
漸漸過去.
我不知道大家家裡有沒有窗簾.
我家旁邊有一個窗簾.
那些叫做Sibra Line.
叫斑馬窗簾.
即是一格一格的.
我發現.
有時當我拉這個窗簾.
你可以關掉.
即是光不能進來.
但當你開少少.
有些光進來.
開多些又有更多光進來.
所以不是沒有光.
但不是那麼光.
有時我們不是沒有愛.
但不是那麼多愛.
我們的愛可以更多.
更大更廣更深.
所以這裡提到.
起初所聽的命令.
但我們還要.
約翰還要提我們.
我們還要彼此相愛.
因為愛不單止愛是恆久忍耐.
提到愛是永不止息.
愛也是無窮無盡的.
你想想如果我真的需要.
去愛的時候.

$^{361}$無論是愛人或愛神.
在內容方面在程度方面.
都可以不斷改變.
可以不斷增加.
所以我發現愛不單止是永恆.
而且是無盡的.
所以我很欣賞.
我經常記的一節聖經.
就是羅馬書十三章第八節.
保羅說.
凡事不要虧欠人.
唯有彼此相愛.
不必為虧欠.
什麼意思呢?我們要記住.
愛是愛不夠的.
永遠愛不夠的.
所以有一次我在婚禮裡.
都有勸勉提到.
結婚兩位新人.
要彼此相愛.
我們要彼此相愛.
愛永遠不會問.
到底我要做多少才夠呢?.
而是說我可不可以做多一點呢?.
就是.
Never ask how much I have to do.
而是How much more I can do.
常以為虧欠.
保羅的意思是.
我們不斷能夠愛多一點.
再愛多一點.
想想還有什麼人.
我們可以愛.
想想還有什麼事.
我們可以表現我們的愛.
當你這樣想的時候.
你會發現愛是無盡的.
無論是對象,內容,程度.
可以不斷增加.
你會發現.

$^{401}$愛的特質.
就像貪心.
但不是貪心.
貪心會損人利己.
愛不是貪心.
而是有心.
貪心是怎樣呢?.
什麼都想要多一點.
有心的愛.
什麼都想做多一點.
所以有時候我鼓勵弟兄姊妹.
不要點到即止.
要多走一步.
真正愛的時候.
有時候我們回教會.
開門後面有人.
不單用一隻手.
頂著門.
有時候會被人走過.
開門有很多方法.
大家都應該知道.
有不同的.
有些人頂著就放手.
有些人拉著門.
到那個人走過才放手.
愛和貪心.
有些相同.
相同就是想多一點.
一個損人利己.
一個捨己為人.
愛不是貪心.
而是有心.
希望能夠做多一點.
愛多一點.
這個心也是神的心.
神也是這樣.
希望我們可以這樣.
第一點就是.
提到愛的命令.
是一個過程.

$^{441}$第二點要提的就是愛的問題.
有還是沒有.
要不斷改變.
你會發現.
有些事情.
我仍然很有私心.
我要慢慢改變.
不要損人利己.
要顧及別人.
所以你看十二節講到.
十一節講完要彼此相愛.
十二節說不要像該人一樣.
為什麼呢.
他殺了他的弟兄.
因為他的行為是惡的.
他兄弟的行為是善的.
看不過眼.
不值得.
所以十二節提到.
不可.
不可以像該人一樣不愛.
為什麼不愛呢.
因為他的行為是惡的.
十三節他又提到.
世人如果恨你們不要以為稀奇.
不單止我們不要.
因為自己不好而不愛人.
但你也要明著.
有人他不好.
他會不愛我們.
不要覺得很稀奇.
所以我們在.
在約翰福音.
十五章.
十八至十九節.
有提到耶穌提醒門徒.
世人會恨你們.
但你們要知道.
恨你們的意思已經是恨我.
後面說為什麼.

$^{481}$因為我們不熟悉這個世界.
世界當然是愛屬自己人.
不熟悉世界而被神揀選出來.
所以世界就恨我們.
為什麼恨我們呢.
約翰福音三章十九至二十節又提到.
光來到這個世界.
世人因為自己行為是惡.
他就不愛光.
倒去黑暗.
下面又提到凡是作惡的就會恨光.
所以你發現.
人為什麼不可以愛呢.
因為自己有罪.
所以不能夠愛.
第二就是.
其他人有罪所以他不愛.
所以就不能彼此相愛.
我有罪我不愛.
人有罪人又不愛.
人就不能彼此相愛.
人為什麼不能夠愛.
是因為人有罪在他生命的內裡.
我住在這裡.
我住在一個大廈.
我住在17樓.
我經常都會見到.
不是見到日出.
我是見到日落.
但我用日出來講一個比喻.
當早上日出的時候.
你會發現.
光慢慢越來越多.
暗就越來越無.
或者日落的時候.
當光越來越少的時候.
黑暗.
即是天就越來越黑.
是這樣的.
即是當光越來越多.

$^{521}$暗就越來越少.
光越來越少.
暗就越來越光.
所以有時候我自己想起.
例如在家裡很簡單.
為什麼有時候對家人不能夠好一點.
為什麼對太太語氣又不能好一點.
態度又不能好一點.
做事又不能主動一點.
我發現到.
原來我經常問自己.
我怎樣可以好一點.
我發現如果我能夠離開.
一些我自己知道的罪.
無論大小都好.
如果我能夠發現.
我離開其他的罪.
我發現我能夠愛多一點.
我再講多一次.
可能我對我家人或太太.
某些事情不是很好.
但當我離開.
我這些罪.
我這些事情就會好.
我發現我能力可以好一點.
可以改變.
原來一個人有罪.
他愛的能力就會減少.
例如.
當我少看手機.
當我多睡覺.
當我少管閒事.
當我多精神.
少惹麻煩.
我心情就會好一點.
語氣又會好一點.
所以當我減少去做一些不應該做的事.
我發現我愛的能力.
就會增加.
我真的語氣又會好一點.

$^{561}$態度又會好一點.
表現又會好一點.
所以離開我們的罪.
擁有或擁抱我們的愛.
就像剛才說的開窗簾.
當我越開的時候.
就會越光.
當我越開的時候就會越光.
當我們越離開我們的罪的時候.
我們越能夠愛.
所以第一.
當你發現.
為什麼我不能對人好一點.
為什麼我不能說話好一點.
為什麼我不能幫多一些人.
為什麼很多時候.
很多事情我放不下.
離開一些我們知道明顯的罪.
無論大小都好.
你會發現你愛的能力.
就會越來越多.
所以不要小看小的罪.
不要開始.
大的罪.
當一個人你要愛.
你就要離開罪.
不要像戈癮一樣因為自己的惡.
不要像世人一樣因為他們的惡.
以致我們能夠去愛.
離罪的人.
是能夠愛的.
所以第一點我們提到.
愛的命令.
不是貪心是有心.
要多一些.
你有手還是沒有手.
你有手.
你要不斷的成長.
不斷的成長.
第二就是愛的問題.

$^{601}$我們不要灰心.
我們要小心.
有還是沒有.
要不斷的改進.
當你離開罪的時候.
你就能夠去愛.
第三.
就是愛的表現.
行不行.
不斷的付出.
對不起.
最近我感冒.
咳嗽.
所以還沒斷氣.
愛的表現.
行不行不斷的付出.
但很多時候我們都會有句說話說.
說就天下無敵.
做就無能為力.
無能為力.
另句說話.
說就天花瀧鳳.
做就無影無蹤.
這就是人.
有時人不是不知道.
人應該愛.
甚至他不是不知道愛有問題.
但他總是改不了.
好像開頭.
引言說身不由己.
力不從心.
這就是人的基本問題.
甚至一些掙扎.
一些煩惱.
瘦肉直提到.
主耶穌的榜樣.
為我們寫命.
主為我們寫命.
從此我們知道什麼叫愛.
主的榜樣.

$^{641}$就是為人寫命.
是否表示.
我們大家都要彼此寫命.
你為我寫命.
我為你寫命.
如果真的這樣的話.
我們就沒有人.
你為我寫命.
我為你寫命.
如果我們守命的話.
但要記住主的寫命是什麼意思.
主的寫命就是.
為了救命.
為何要寫命.
因為這是我們所需要.
所以意思是.
主的寫命是愛的表現.
愛的表現就是要滿足.
或者給予別人所需要.
別人有缺乏.
你可以給予別人.
所以十七節.
提到.
凡是有世上財物的.
在見弟兄窮乏.
塞著憐恤的心.
愛神的心.
怎麼能夠存在他內.
你有的你不給的話.
你就沒有愛.
什麼叫愛呢.
主的寫命.
以至我們得生命.
是我們需要的.
得的要幫那些不行的.
所以不是真的叫你.
明天撞牆為弟兄死.
為姐妹死.
撞牆不是愛的表現.
因為不需要你撞牆.

$^{681}$你死了他做什麼.
當你真的要寫命的時候.
你才寫命.
主耶穌一定要寫命我們才有生命.
所以主耶穌為我們寫命.
但這裡舉的例子是什麼呢.
當你見到自己有財物的時候.
別人沒有.
你要幫他.
有的你要幫那些沒有的.
得的你要幫那些不行的.
這是愛的表現.
所以我很喜歡一段經文.
亞國書二章十五到十七節.
那裡說到如果有人.
赤身露體.
缺了日用飲食.
中間的人跟他說平平安安去.
願你著得暖吃得飽.
又不被他們所需用.
有什麼益處呢.
信心如果沒有行為就是死.
弟兄姊妹信心沒有行為是死的.
但是愛心沒有行為.
都是死的.
今天很多人有信心.
但是沒有行為.
今天很多人說愛心.
但是沒有行為.
說就天下無敵.
唱就唱到天花龍鳳.
但是眼睛就無能為力.
無影無蹤.
所以提醒我們第三件事.
就是愛的表現.
要做出來.
有一次.
很諷刺的一次.
我還記得我還在讀書.
回教會完了崇拜.

$^{721}$離開教會.
外面在下微雨.
不是傾盆大雨.
我一直在走路要坐車.
坐車回家.
一直走路在下雨.
我又沒有帶傘.
不是很嚴重.
我發現旁邊有輛車經過.
另一個弟兄開著車.
在我旁邊停下來.
把窗戶關上.
跟我說弟兄回家了.
我說是的回家了.
平安平安.
把窗戶關上就走了.
他又不接我.
上車我開車回家.
我看著他開車.
我看著他眼光光.
車子空的只有他一個人.
不是坐滿人的.
是不是這樣.
我看著他說什麼平安祝福.
開車走了把窗戶關上.
愛心的行為.
是死的.
所以我們要知道.
我們需要.
彼此的幫忙.
其實你會發現很多事情.
我們都可以幫助.
我遲些會做一個小組組長訓練.
小組組長訓練.
我都會提到我們怎樣.
彼此的關心.
要秉定是人.
ABCD是範圍.
A可以代表工作.
B可以代表情緒.

$^{761}$C可以代表家庭.
D可以代表習慣.
打交叉的.
在這些方面是不行的.
有顆星的.
在這些方面是可以的.
甲的.
在A那邊是不行的.
但在B和C是很行的.
在D又不行.
乙呢.
A就不行.
但C就很行.
乙可以在C方面幫甲.
甲在A方面可以幫乙.
大家明白嗎?.
給點表示.
我看到你們的.
明白嗎?.
OK.
多謝.
比如丙呢.
B很不行.
但C很行.
很多都不行.
但D就非常行.
甲乙丙呢.
D都不行.
丁很行.
丁可以幫甲乙丙的D.
意思是什麼呢?.
只要你肯彼此相愛.
人總有長短強弱.
人總有愛的可能.
問題是你做不做.
你肯不肯.
你肯幫到的.
有些事你可以的.
有些事你有的.
你又給那些沒有的.

$^{801}$所以在十七節說.
如果你有那些.
看到弟兄缺窮乏.
無論是ABCD.
你有的有A就幫A.
你有C就幫C.
不要濕著你的憐恤心.
如果你這樣的話.
你要說愛神的心.
怎會在你裡面呢?.
所以大家.
我們要看我們自己有什麼.
有些人你支持的長處.
我這方面很行.
我這方面很強.
我這些我知道的比較多.
有些人沒有這些.
是很不行的.
你幫一下他.
當我.
早前我病的時候.
病到五顏六色.
又沒有氣又沒有力.
人又瘦又弱.
很多弟兄姊妹包括你們在內.
都常常問候我.
為我代禱支持我.
鼓勵我.
我常常提我的病好轉.
不單止是因為.
神的恩典.
也有人的關愛.
令我人很舒服.
可以休息到心裡很滿足.
很平安很開心.
因為很多人關心我.
當我自己好轉的時候.
教會就問.
徐弟兄什麼時候可以開始.
回來我們港島.

$^{841}$當我可以的時候.
我就馬上回來.
我答應你們說了一封書信.
我就開始說了一封.
我需要幫的.
我亦都嘗試我會去做.
我可以做的.
給你們所需要的.
大家彼此的幫助.
這就是彼此相愛的表現.
十八節提到.
不要只在言語和舌頭上.
總要在行為和誠實上.
不要只在言語和舌頭上.
不要只在.
總要在.
不要只在說.
不要只在.
總要在.
可以說的都要說.
譬如你們關心我當然要說.
不要默默無聲.
但都可以說出口.
徐弟兄我們都很擔心你.
你最近怎麼樣.
我前兩天去一個團體港島.
有個姊妹看到我.
有個老姊妹跟我說.
徐弟兄今天看到你.
真的很開心.
你知道我多擔心你.
很久沒見了.
因為我一年沒去港島.
她知道我生病.
看到她真的很開心.
這個姊妹說這兩句說話.
令我心裡很溫暖.
雖然我不是很認識她.
但她都很關心我.
她都說我們第一時間.

$^{881}$為你祈禱.
非常感恩.
說出口沒問題.
但都要有行動.
你真的問候我.
甚至有什麼你會提醒我.
我以前是做童子軍的.
香港做童子軍第十六旅.
童子軍有句口號.
就是日行一善.
在家裡在公司在教會.
總有人總有事.
你可以去愛他.
所以只要有心.
總有可行.
只要有愛.
必有所為.
這個世界.
愛是無敵.
愛是無限.
要愛的人愛不完.
要做的事做不完.
所以保羅說尚以為虧欠.
總有事還可以做.
所以你想想今天.
你和身邊的人說一聲.
神愛你.
當然我們說不出口.
很難說我也愛你.
但我們弟兄姊妹.
要彼此相愛.
想想身邊的人.
有沒有人你知道.
你最近知道你聽到.
或者他有提過的.
他有什麼需要.
或者你幫不了他.
但你都可以關心.
關注他.
讓他知道你肯聽他講.

$^{921}$你都顧念他.
想想對誰有愛的表現.
你一會去做.
愛的命令不是貪心.
是有心.
守定不守呢?.
愛的問題不是灰心.
而是小心.
有或沒有要不斷改進.
第三愛的表現.
有些人說很怕會做死你.
你不用擔心.
你要真心.
恨不恨.
做不做要不斷付出.
愛真的愛不完.
剛剛有一對夫婦.
傳短訊給我.
徐弟兄想找我聊聊天.
他說下星期一行不行.
我說不行.
再下星期一吧.
約了他.
八點半在哪.
能夠做的我們繼續去做.
不用擔心會做死.
當然你不要做到自己死.
不需要太擔心.
量力而為也盡力而為去做.
最後一點是愛的後果.
穩不穩.
不斷追求.
今日的世界不單止選人利己.
甚至有時還會選人選己.
愛是利人利己.
是雙贏的.
甚至三贏.
除了利人利己外.
還能夠利神或神喜悅.
神喜悅.

$^{961}$你想想家中多一個兒子.
多一個女兒.
有兩三個小孩.
兄弟姐妹之間相和.
相愛.
做父母很開心.
彼此相愛神也很開心.
這裡提到愛的後果.
有三個後果.
大綱中我用的.
安穩在神前.
好過蒙應穩.
神住在裡面.
剛才讀經最後兩節提到這幾方面.
第一安穩在神前.
千金難買真平安.
我們常說.
平生不做虧心事.
半夜敲門也不驚.
平日多做愛心事.
日日敲門也不驚.
我發現平安.
一個人心中無懼.
真的.
有人說勇者無懼.
我覺得愛者無懼.
真的我們.
如果願意去愛的時候.
真的心裡的平安.
那份安穩.
心很定.
無懼的人.
可以站直.
睡得好.
睡覺也睡得好.
這些真是千金難買.
我發現一個越有愛的人.
心越能夠平穩.
越能夠平穩.
所以第一.

$^{1001}$安穩在神前.
有神的平安.
第二就是禱告蒙應穩.
有神的同工.
剛才說有神的平安.
有神的同工.
在我以前的房間.
機構的房間.
掛了一個牌.
上面寫著兩個字.
大家看到.
聖諫這兩個字.
是中國一個弟兄寫的字.
送給我們機構.
我就表了它.
放在我的房間.
上上提醒自己.
要聖諫.
為什麼呢?.
因為我們才能有神的同工.
上上提醒同工.
要看到神的手在我們當中.
要有神的同工.
要有神的同工要先與神同行.
當你說要有聖諫.
其中一樣就是要有愛.
沒有愛的人.
其實是得罪神.
得罪人.
有愛.
有神的同工.
因為我們與神同行.
聖諫裡面記載什麼人與神同行呢?.
創世記第五章廿四節.
提到以諾與神同行.
神將他娶去.
他就不再逝了.
他與神同行.
但除他之外.
第六章有第二個人就是羅亞.

$^{1041}$聖經說羅亞是一個異人.
在當時的世代是一個原傳人.
羅亞與神同行.
六章九節之後發生什麼事呢?.
神要羅亞做方舟.
救了他全家.
羅亞與神同行.
神與他同工.
不單能夠建成方舟.
而且救了他全家的人.
先與神同行.
後有神同工.
如果我們要心中有平安.
手中有能力的時候.
我們就要去愛.
原來愛真的可以讓我們心安.
讓我們有能力.
所以在廿一節提到.
安穩在神前.
有神平安.
廿二至廿三節說到.
禱告蒙應允.
有神同工.
廿四節說到.
神住在裡面.
有神同坐.
不單有平安.
而且有同工.
而且有同坐.
耶穌有個名叫以馬來尼.
什麼意思呢?.
神與人同在.
我記得以前曾經有外國人的教會.
問我為什麼中國教會會成長.
當時我給他們一個英文字.
叫做Emmanuel.
就是我們以馬來尼.
我因為神與他們同在.
然後他們問我.
神怎樣與他們同在.

$^{1081}$神什麼與他們同在.
我說他們有神的話語.
有道可傳.
他們有神的能力.
有事可食.
當我們有神同在.
我們能夠明白神的心意.
可以和人宣講神的心意.
當我們有神同在的時候.
我們有神的能力.
我們可以成就神的工作.
完成神的使命.
當我發現.
在港獨當中.
我有一個港獨團體.
和弟兄姊妹分享.
港獨其中一樣很重要的.
我們要有愛在其中.
港獨不是表現.
不是表現你港獨的能力.
港獨是表現你愛的心態.
為什麼我們要用心去預備呢?.
剛才我.
在下午的時候.
我再看回.
powerpoint和大綱的內容.
我又再更改.
我怕不能夠講得更加清楚.
更加明白.
為什麼我相信神的道.
可以使人蒙福.
可以使人成長.
也可以直接講.
因為我愛神的教會.
也愛弟兄姊妹.
也愛你們.
所以我嘗試盡量講好它.
我們可以有道.
可以聽.
可以講.

$^{1121}$我常常在港獨團體提醒.
我們要愛聽我們港獨的人.
也要愛讓我們港獨的神.
當你愛人愛神的時候.
我們港獨才會好.
港獨.
如果你沒有愛的話.
沒有什麼道可以講.
就算講也沒有什麼力量.
港獨不單止合情合理.
而且要友情友誼.
但要有愛.
所以有神同在的人.
是可以愛.
而愛的人是可以有神同在.
到底是有神同在.
先然後才去愛.
還是愛.
然後才有神同在.
其實我覺得是一個無限循環.
你有神同在.
你去愛.
你愛的時候就會有神同在.
有神同在.
你就可以去愛.
你去愛的時候就會有神同在.
一直轉來轉去.
所以當你一路增加愛的時候.
神的同在就越明顯.
當神的同在越明顯.
你就越能夠去愛.
愛是無始無終無止境.
所以愛的後果.
就是愛.
一定要愛.
所以我用一個字.
愛不可以懶惰.
不可以有惰心.
不要懶惰.
專心.

$^{1161}$這個世界有很多東西可以令我們分心.
你這樣你那樣.
當時的閒事八八八八.
我們不要懶惰.
隨隨便便地愛.
我們要專心.
專注如何可以更愛.
我們教會裡面的弟兄姊妹.
如何可以更愛神家裡面.
其他教會的弟兄姊妹.
如何可以更愛我身邊的人.
或者我不認識的人.
如何可以做.
愛的時候我們可以穩.
穩不穩在於你愛不愛.
你想穩一點的話.
你就不斷追求.
又記得窗簾的例子.
當你開得越多光就越多.
你想穩一點.
你想心安一點.
有神的同在又有神的同工.
你要不斷追求去愛.
不斷追求去愛.
愛的後果就是.
不可以惰心.
要專心才能穩.
人就要不斷追求.
今天的題目叫做愛的後果.
神前安穩.
人要安定.
生活安定,收入安定,健康安定.
什麼都要安定.
我覺得最重要的是求神前.
能夠安穩.
神前很穩.
我盡力去愛.
我量力去愛但我已經盡力去愛.
我們要知道.
愛當中.

$^{1201}$一開始說是一個過程.
定但又不定.
定而不定是什麼意思呢?.
定是方向我們要愛.
而且要不斷去愛.
不定是過程.
越愛越多.
定是一個方向我們要去愛.
而且要不斷去愛.
方向是向前的.
不定就是過程.
向前的路上.
愛的路上.
是一步一步向前的.
所以方向是定的.
過程是不定的.
因為要不斷成長.
不斷改進,不斷付出,不斷追求.
追求神的平安.
追求神的同工.
追求神的同在.
不斷追求愛.
剛才那首歌叫活出愛.
不斷活出愛.
你已經活出很多了.
可以再多的.
你有沒有愛呢?.
不可能沒有的.
但又不可能什麼都有.
愛不完.
還有很多人需要愛.
有很多事可以愛.
由小到大,由淺到深,由窄到闊.
愛的成長.
這是神的心意.
這也是神的命令.
也是神的命令.
愛的命令.
就是要對準愛的目標.
鎖定方向.

$^{1241}$愛的結果就是要踏上愛的旅程.
向前邁進.
想想要愛什麼人.
想想要做什麼事.
愛的命令.
不是貪心.
要有心.
守定不守呢?.
我們要不斷的成長.
我們是有守的.
但要守更多.
愛的問題.
不要灰心.
要小心.
有些人說愛不到.
為什麼?.
因為愛的問題.
你只要愛就能夠愛.
第三,愛的表現.
不用擔心會愛死你.
不用擔心,要真心.
做還是不做呢?.
要不斷的付出.
第四,愛的後果.
不要墮心.
要專心.
穩定不穩定.
想安穩在神的裡.
要不斷的追求愛.
不斷的追求愛.
什麼心我們要死.
貪心,灰心,擔心,墮心.
這些要死.
什麼心要活呢?.
有心,小心,真心,專心.
這些要活起來.
大家都不同.
應該有的我們就要有.
不應該有的我們就要停.
以致我們能夠彼此相愛.

$^{1281}$不斷的成長.
起初引言有說.
有時我說我不知道你明不明白.
不要說你又不明白.
明你又不做,做你又做錯.
改你又不來,這樣又不好.
我們知道要愛的時候.
我們就去愛.
起初引言有提到海誓山盟誰能守.
但彼此相愛.
就必須守.
因為這是神的心意,也是神的命令.
如果神要我們守的話.
我們就能夠守,神不會要我們沒有的.
神不會要我們做些我們不能做的.
既然是神的命令.
就是你做得到.
不要賴,你是做得到.
所以神才叫我們彼此相愛.
彼此相愛.
大家要彼此互勉.
彼此相愛,以致能夠有神的平安.
有神的同工.
有神的同在.
在我們教會裡面.
一起靠神來成長,一起來討神的喜悅.
今天你聽了之後.
你明白到.
愛是很重要的,愛是不可以沒有的.
而且愛是要成長.
愛是要不斷的.
所以看看有沒有其他人.
更多人我們去愛,有沒有其他事.
我們可以去做.
以致能夠.
我們的我,活出我們的我.
求神幫助我們.
我們一起祈禱.
我們求你幫助我們.
不單止我們聽了.

$^{1321}$甚至我們明白了.
更開我們的眼.
讓我們看見我們可以做什麼事.
有什麼人有什麼需要.
我們有的.
我們要分,我們得的.
我們要幫.
以致我們在家庭.
甚至在我們工作的地方.
在我們社區的當中,在教會的當中.
我們能夠活出.
我們要活出的愛.
以致能夠安穩在神的面前.
求主幫助我們.
聽我們祈禱,奉耶穌的名祈禱.
阿們.
\newpage



\chapter{李思敬}\label{ch:preacher2}
\begin{multicols}{3}
\minitoc
\end{multicols}
{ \scriptsize


\begin{xltabular}{\textwidth}{|p{0.15\textwidth} p{0.6\textwidth}|p{0.07\textwidth} p{0.1\textwidth}|}
\hline
創世記 1:26-28 & \hyperref[sec:cau3XNPSx68]{「形象」和「配偶」 (創世記1\_26-28) - 李思敬博士【繁簡字幕翻譯 by Johnson Ng】《五經》中的性別神學講道系列 - (第1講)} & 2025-02-16 & \href{https://youtube.com/watch?v=cau3XNPSx68}{\texttt{ cau3XNPSx68}} \\
出埃及記 40:34-35 & \hyperref[sec:YdX9gstJs1g]{上主榮光充滿帳幕 (出埃及記40\_34-35) - 李思敬博士【繁簡字幕 by Johnson Ng】「認識神的榮耀」講道系列 - (第1講)} & 2025-02-05 & \href{https://youtube.com/watch?v=YdX9gstJs1g}{\texttt{ YdX9gstJs1g}} \\
詩篇 148:1-14 & \hyperref[sec:l8BFPdIOADs]{合神心意的讚美 (詩篇148\_1-14) - 李思敬博士【繁簡字幕翻譯 by Johnson Ng】} & 2025-01-25 & \href{https://youtube.com/watch?v=l8BFPdIOADs}{\texttt{ l8BFPdIOADs}} \\
\hline
\end{xltabular}
}
\newpage



\section{創世記 1:26-28}
\label{sec:cau3XNPSx68}
\textbf{「形象」和「配偶」 (創世記1\_26-28) - 李思敬博士【繁簡字幕翻譯 by Johnson Ng】《五經》中的性別神學講道系列 - (第1講)}
\newline
\newline
連結: \href{https://youtube.com/watch?v=cau3XNPSx68}{\texttt{ https://youtube.com/watch?v=cau3XNPSx68}} ~~~~ 語音日期: 2025-02-16 
\newline
\newline
\hyperref[sec:K8E95o7ZcvU]{< < < PREV SERMON < < <}
~
\hyperlink{toc}{[返主目錄]}
~
\hyperref[ch:preacher2]{[返講員目錄]}
~
\hyperref[sec:YdX9gstJs1g]{> > > NEXT SERMON > > >}
\newline
\newline
創世記 1:26-28
\newline
\begin{longtable}{cl}
\hline
\hline
章節 & 經文 (和合本修訂版)\\
\hline
1:26 & \begin{tabularx}{0.7\textwidth}{X} 神說:「我們要照著我們的形像,按著我們的樣式造人,使他們管理海裡的魚、天空的鳥、地上的牲畜和全地,以及地上爬的一切爬行動物。」 \end{tabularx} \\ \\ \relax
1:27 & \begin{tabularx}{0.7\textwidth}{X} 神就照著他的形像創造人,照著神的形像創造他們;他創造了他們,有男有女。 \end{tabularx} \\ \\ \relax
1:28 & \begin{tabularx}{0.7\textwidth}{X} 神賜福給他們,神對他們說:「要生養眾多,遍滿這地,治理它;要管理海裡的魚、天空的鳥和地上各樣活動的生物。」 \end{tabularx} \\ \\
[1ex]
\hline
\hline
\end{longtable}
$^{1}$請姐妹處內平安.
我應該講新年蒙恩 福輝滿人.
這是去年的 不用緊張.
去年我們一起讀了舊約聖經最後一卷《馬拉紀書》.
今年我想回到摩西五經 聖經最早的五卷.
不過我想用一個比較小眾的角度.
來選讀五經當中創出的紋身.
五段或平時我們不一定有機會接觸討論講解的經文.
所以我選擇的題目是五經中的性別神學.
很少講 我也很少講.
所以請大家為我祈禱.
不要說錯話 被人在網上罵.
我選擇的出發點是多謝忠臣院長黃國維牧師出版的一本書.
這本書是家庭神學.
他就任忠臣院長之後所出的一本書.
家庭神學其實是脫胎自他的博士論文.
這本深很多.
剛才那本我會推薦給沙孫的團契來做Book Study.
我們的使命宣言是建立家庭.
究竟聖經我們的信仰怎樣教導我們關於家庭這件事呢.
其實伯納院長的博士論文是在17年出版.
這是一個神學倫理的專題探討.
Beginning from Man and Woman.
如果說華人教會當中對於神學倫理.
我相信伯納院長可能不喜歡.
認了第二沒有人敢認第一.
所以這是一個權威的書籍.
不過不推薦你看免得你看不明白亂說.
這是一個很重要的基礎.
書名提醒我.
我讀大學的時候1975年.
美國Fuller神學院.
現在山火很厲害.
Pasadena那裡的神學院的神學教授Paul King Jewett.
出版了這本書.
Man as Male and Female.
長話短說我不是今天介紹書籍.
不過忍不住多說兩句.
這本書引起軒然大波.
在70年代末被神學院院董.

$^{41}$攻擊Fuller神學院不信聖經.
其實很不幸.
所以我也要為他說一句公道的話.
Professor Jewett在他病逝之後.
這本書是他的學生同事幫他出版的.
一本遺作.
這本書很深奧系統神學.
如果你對同性婚姻同性戀的運動.
你有很濃厚的興趣.
我推薦Professor Jewett這本書的其中一章.
討論這個問題.
是我讀過在福音信仰神學討論.
homosexuality這個課題.
討論得非常精彩.
並且他不是一面倒.
保守還是開放.
不是神學不是這樣說話的.
教會會這樣說話.
所以分得清楚不同層次不同的討論.
但這不是每個人都看得明白的.
如果這個問題煩擾你.
令你很多疑惑.
這是我第一個推薦.
這本書不容易看.
但那篇文章相當平易近人.
今天我們要主要集中讀聖經.
我從創世記開始.
從第一章開始.
我們說beginning from the very beginning.
真的從頭開始去問聖經.
到底怎樣說性別.
這回事.
創世記第一章.
告訴我們起初神創造天地.
這章的聖經其實是源於第二章第三節.
詳細我不會像上課那樣去解釋.
他停在「安息聖日,天地萬物都做齊了」.
所以是一個很完美的開場白.
我們留意到這章的經文.
去交代神創造天地.

$^{81}$是用了一個我們中文的說法.
「苦如皆懂」白居易的唐詩.
沒有歧視的性質.
人人都明白的一個框架.
就是六日的創造.
一個星期神創造天地.
我們沒有時間去討論六日創造的結構.
不是我們今天想說的.
我們想看的就是每一日結束.
你一定記得.
有晚上有早晨.
你填充了.
是第幾日.
這個格式很嚴格很嚴謹.
這樣的結束就提醒我們.
其實每一日的開頭.
第一句說話都是重複的.
不過我們時時都沒有留意.
我們只記得有晚上有早晨.
也好,起碼你記得一半.
更重要的上半就是六日的創造.
開頭總是用神說這句說話開始的.
你有聖經沒有聖經.
你可以查看這個大家很熟悉的經文.
並且如果你數一數.
重複的次數有時都提醒我們.
有晚上有早晨是六次.
因為是六日創造.
但是神說卻在這六日的創造裡面.
是出現過九次.
他在第三日出現了兩次.
然後在第六日出現了不止一次.
這個很簡單.
你數數吧,我們讀聖經不習慣這樣的方法.
所以有時會看不到也不奇怪.
不過更重要的是我們問神說有甚麼重要呢?.
當然是神不說話那做甚麼呢?.
在詩篇33篇第九節.
也是很熟悉的一節金句.
神說有就有.

$^{121}$有沒有印象?.
下面那句.
明立就立.
這句說話的意思.
在福音書有一件事解釋得很清楚.
或許你記得或不記得.
一個白夫長羅馬人.
他家中的僕人病了.
他來找耶穌請夫子醫好僕人.
耶穌說帶路吧.
我很願意去.
這個白夫長立即說.
不敢勞動大駕.
當然他不是說中文.
意思是這樣.
你不用來我家.
你只要說一句話.
我的僕人就好了.
最特別的是在馬太福音和路加福音記載.
白夫長解釋.
我也是一個有權柄的人.
我手下有一百個兵丁.
我說你去他就去.
我叫他來他就來.
白夫長做事不用動手動腳.
他說有就有.
明立就立.
所以我們讀回舊約聖經創世記第一章.
六日的創造.
九次神說.
我們一定要明白.
這裡說的是上帝的權柄.
很清楚.
他說話是就這樣成了.
我們看到第六天.
上帝又說什麼呢.
剛才多謝敬拜隊.
為我們讀了26至28節.
神說.
我們照我們的形象.

$^{161}$按我們的樣式去做人.
讀舊約聖經.
所有的讀者.
不只是讀創世記.
都一定會立即問.
神沒有形象.
你會不會這樣問.
哪裡教.
十誡.
上帝在西乃頒佈十誡命.
第二誡命已經說.
不可以做任何形象去跪拜.
神沒有形象.
以塞亞書40章.
這裡所說的形象.
如果我們由理解為.
他怎樣的.
要像他.
解錯了.
又是回到福音書.
更加記得清楚的另一件事.
猶太人拿一個羅馬的錢幣.
來問耶穌納粹給該殺.
可以不可以.
你明白這句問題.
是想迷惑耶穌.
可以.
你就是遊奸.
你就是叛國.
你就是羅馬的走狗.
不可以.
由帝國來通緝你.
所以左右怎樣回答.
其實都是很大件事.
你記得耶穌怎樣回答嗎.
耶穌說看看錢幣上.
是誰的象該殺.
於是耶穌的答案.
可惜中文聖經譯錯了.
該殺的物.

$^{201}$物件歸給該殺.
神的物.
不是說雀的那對物.
是物件歸給神.
教會以前奉獻商.
都喜歡刻了這句話在上面.
拿走那個物字.
該殺的.
給該殺.
上帝的.
給上帝.
不是說死物.
不是說金錢.
不是說我們所擁有.
就是問誰有神的形象.
懂不懂得這樣聽.
錢上面的形象是該殺的.
給他.
誰有神的形象.
不是另外一些錢.
十分一的錢是神的形象.
不是.
我們是有神的形象.
明白嗎.
我們整個人是屬於上帝的.
所以形象的意思.
不是樣貌.
再說最後一次.
神沒有形象.
所以你對著那幅相.
對著那幅畫.
對著十字架來祈禱.
不需要的.
聖經不是這樣教的.
我會專心一點.
不要當作偶像.
不會那些是神聖的.
希西加王改革聖殿.
把摩西的銅蛇都扔掉.
有沒有讀過這段歷史.

$^{241}$摩西的銅蛇.
醫好人在曠野.
火蛇咬的.
都要扔掉.
你明白背後的意思嗎.
不可以取代了.
神自己本身的因典.
神的形象.
在這一節的聖經裡.
上帝說我們按我們的形象.
記得了不是說樣貌.
屬於我們.
人是屬於神的.
做人使他管理海裡的魚.
空中的鳥.
地上的牲畜.
人地包括所爬的一切昆蟲.
於是神照自己的形象做人.
照他的形象做男做女.
神賜福給人.
對他們說要生養眾多.
遍滿地面.
要治理這地.
管理海裡的魚.
空中的鳥.
地上各樣行動的活物.
治理全地的責任.
從古到今.
甚至今日.
都是男人的.
你同意嗎.
所以武則天在中國歷史上.
是謀朝篡位的.
慈禧太后.
不要說她了.
皇帝一定是男性的.
你說生養眾多.
不是的.
所以上帝很公道的.
你忘記了家譜怎麼寫的嗎.

$^{281}$你背給我聽.
不懂背嗎.
阿伯拉罕生爾薩.
爾薩生那國.
你生個兒子給我看看.
弟兄們.
當然是太太做的.
不提太太的.
連生養眾多.
都是男人的尊利.
哎呀 聖經真是落後.
請問我們中國人的族譜.
息輔的名字怎麼記的.
沒有啊.
我們都沒有族譜的了.
你回去鄉下查.
息 吾 士.
沒有名字的.
是這樣記載的.
上帝做人.
聖經開宗明義.
說清楚一樣.
男女都有神的形象.
都是屬於上帝.
不只是男性.
擁有從神而來.
治理全地的責任.
或者傳宗接代的天職.
文化是這樣看.
對不起.
聖經開頭上帝說.
說什麼呢.
他照他的形象.
做男做女.
你要懂得解釋這句話.
不是像他.
上帝是男人還是女人.
不是.
就是權柄.
尊嚴.

$^{321}$不只是男性.
所有的文化都尊重男性.
走出來.
應該.
不是的.
聖經開頭一開始記載.
起初神創造天地.
就不是這樣說的.
行不行.
明白嗎.
聖經在這裡顛覆我們一貫的觀念.
你說是吧.
不過《創世記》還有第二章.
還有上帝做人更加詳盡的記載.
怎麼說呢.
在第二章那句.
神說那人獨居不好.
我要為他做一個配偶.
幫助他.
老婆你要幫我.
我做什麼你都要支持我.
我要不客氣地說.
這個翻譯.
都是帶了很重的文化觀念.
配偶是要來幫助弟兄的.
這是上帝的旨意.
容許我說一點點稀白來文.
不過不會打出來給你看.
直接的翻譯.
這句說話是說.
我要給他做一個幫助.
在他對面.
配偶原本那句是.
在他對面.
Opposite to him.
什麼意思呢.
聖經學者解釋得很好.
他說這個描述.
是一個戰爭當中的情況.
兩軍交戰.

$^{361}$交著打仗.
很辛苦.
大家各有勝負.
不知道難解難分的時候.
忽然間戰場的對面.
出現新的軍隊升旗.
大家看.
原來是幫我們這邊的援軍.
生力軍.
那個叫做在他對面.
懂不懂.
決定勝負.
不是你原本的力量.
是外來的幫助.
但當他出現的時候.
他就可以扭轉整個局面.
都是幫助啊.
我們喜歡爭辯.
讀聖經.
那你就真的要讀一下希伯來文了.
幫助這個名詞.
我們要列出詩篇121篇.
第一第二節.
沙川的弟兄姊妹懂不懂背.
我從小就要背.
我要向山舉目.
一起來.
我的幫助從何而來.
第二節.
我的幫助從造天地的耶和華而來.
這是聖經用幫助這個字眼.
是神的幫助.
不是普通人的幫助.
在他對面.
這個是打仗.
但幫助在舊約聖經裡面.
是用在神的幫助.
所以詩人說.
我向山舉目.
難道我的幫助在山上來嗎.

$^{401}$不是.
我的幫助是從創造天地.
在眾山之上的獨一真神而來.
請問上帝怎樣幫助你.
行雷閃電.
讓我見到異象.
不是.
你又走入民間宗教.
聖經的幫助.
很實際.
就是你家裡看不起的那個.
什麼絕經.
什麼賤內.
以前的人說.
不可以出來.
他主內.
我主外.
那個歧視不是聖經的表達.
雖然我們說很不幸.
在翻譯的時候.
還是譯成了做一個配偶幫助他.
然後我們就去爭辯.
辣骨.
應該是平等的.
其實不用爭辯的.
是比你高.
懂不懂這個意思.
姐妹聽到很高興.
不要驕傲.
神的幫助.
是呀.
在當時的文明.
所看不起的人的身上.
聖經開宗明義說.
他同樣有神的形象樣式.
上帝做的不只是男人.
次一等的才是女性.
不是.
做男做女.
同樣有神的形象.

$^{441}$那人獨居不好.
他需要神的幫助.
在他以外.
在他身邊.
但是不是.
怎麼說呢.
每樣東西都是幫他.
不是的.
上帝的幫助.
有些時候是改正我們的.
有些時候是教我們的.
所以上帝透過怎樣的幫助.
每一個人.
這個在上帝的手中.
其中一個就是在家庭神學.
在家庭婚姻的關係上.
顯出上帝創造的心意.
這是我們讀《創世記》開始.
神創造天地.
是.
經文的主題.
不只是人.
天地萬物.
六日創造.
所以正常我們解《創世記》第一第二章.
我們很少提到.
今天我們所提的.
性別神學.
但如果我們真的要集中看清楚.
男女在創造裡.
上帝是怎樣說的呢.
我們不要忘記.
神說.
我照我的形象.
做男做女.
那些人獨居不好.
我不是幫他找個盤.
和他一起玩.
陪陪他.
不是.

$^{481}$我要幫助他.
上帝說我怎樣幫助他.
夏娃.
就是亞當.
從神而來.
有別於他以外的幫助.
Opposite to him.
所以有些時候是Opposite的.
是不同的.
所以就吵架.
吵架是好的.
不是真理越辯越明.
看到自己的盲點.
原來我是這樣的.
是呀.
上帝的幫助.
出人意表.
更加出人意表的.
是我想跟著看下去的第三個重點.
創世記沒有停留在創造.
它進入到歷史的現實.
所以這個書名.
只是在中文聖經找到.
其實也是翻譯一個先入為主的概念.
Genesis不是創世記嗎.
那個不是英文.
希臘文是翻譯十一次創世記的標題句.
創造天地的來歷.
Toledot, Genesis.
所以創世記這本書.
其實百分之九十六.
五十章裡面的四十八章.
都是記載歷史的.
不是說神創造天地.
在歷史的記載當中.
我們會很記得.
神呼召阿伯蘭離開本地本族附加.
不要告訴我你不認識這段聖經的歷史.
很重要的.
這就是舊因歷史的起點.

$^{521}$創世記記載.
神揀選阿伯拉罕.
呼召他離開迦勒底的吾爾.
去到迦南應許之地.
就開始了以色列十二個支派的舊因.
直到新約直到今日.
沒錯.
這個起點很重要.
不過我們又回到創世記的記載.
我們發覺十二章之前.
其實還有很短的六節.
十一章結束.
就說阿伯蘭的父親.
他沒有離開本地本族附加.
所以他拉死在哈蘭.
這個可能我們不熟悉.
不過這是阿伯拉罕的生平記錄的頭一段.
然後第二段就記載他離開本地本族附加.
為什麼我要指出這兩個段落的開始呢?.
因為是很美的結構.
二十四章二十五章結束阿伯拉罕的生平.
最後一段就記載阿伯拉罕死在迦南.
去到沒有?去到了.
他真的離開了本地本族附加.
他不是死在哈蘭.
他死在迦南墨比拉洞買了一個墳地.
你有沒有印象?.
但更重要的卻是我們從來不說的一段.
就是二十四章很長的經文.
說什麼呢?.
就算我們說阿伯拉罕為兒子娶妻.
兩個大男人.
然後老僕人千里迢迢回到故鄉.
第三個男人.
其實整章二十四章的主角是誰呢?.
就是阿伯拉罕一開始跟老僕人說.
你回去找一個女孩來.
老僕人說有沒有?.
肯不肯離開本地本族附加.
這麼遠來嫁給你兒子.

$^{561}$雖然你很威風.
但家人是否這樣看呢?.
有沒有這個主角一開始就成為懸疑.
然後李百嘉出場了.
記得嗎?老僕人回到祈禱.
李百嘉出來幫忙打水.
讓所有的牲口都喝水.
能夠安頓下來.
然後問他的家庭背景.
老僕人知道上帝聽祈禱.
回到家裡說明原因.
飯都不肯吃.
有沒有印象?.
那段經文記載.
阿哥和家人最後說.
讓我們的妹妹自己決定.
她肯不肯去呢?.
傳統多數是不肯去的.
去緬甸還是去泰國呢?.
怎知道呢?.
你身光頸靚.
去到要捱牙.
住tong 房的.
就算肯去.
也拖他一年半載.
所以我們不明白這些文化背景.
等我回去埋葬我父親.
然後我來跟從主.
十年八載.
等於我退休提早來讀神學一樣.
尼伯加說了一個字.
中文也要翻譯兩個字.
說什麼?.
我去.
他貪慕虛榮.
你試想想.
你可以代入想想.
等於創世記記載.
約瑟十七歲離開父家.
如果上帝問他.

$^{601}$我讓你去埃及做宰相.
不過你以後不能回家.
你不去.
十七歲的你去不去?.
尼伯加現在也不是很大年紀.
他不去.
我去.
尼伯加離開本地本族父家.
跟阿伯蘭離開本地本族父家.
我們從來在港台上都沒有相提並論.
看到我們的盲點嗎?.
看到聖經不是我們這樣看的嗎?.
我們覺得你藉氣撞一個男人.
回應上帝的呼召.
就開始了舊恩歷史.
阿伯拉罕萬歲.
但是阿伯拉罕的人生到最後.
他發覺如果上帝不呼召多一個小女孩.
同樣離開本地本族父家.
以撒就要娶一個嘉楠女子為妻.
那就敬拜巴力了.
你明白那個大件事嗎?.
所以到今天猶太人的家譜.
不是計父親的.
是計母親的.
計父親是計母親的.
你問問他吧.
「利伯加在創世記的歷史幾乎不見經傳」.
Rebecca有沒有人叫這個名字?.
很普通的一個女孩.
但是在阿伯拉罕的生平當中.
她做的決定卻是和阿伯拉罕所做的決定不分輕重.
聽得明白嗎?.
懂得這樣看嗎?.
不懂的.
我經常聽到弟兄讀神學.
我太太跟著來的.
最近我見到一個弟兄跟她聊天.
她說「我太太夢照讀神學」.
「所以我又想讀神學」.

$^{641}$「不過我沒有夫照」.
「不過我想跟她聊天」.
「所以我又來讀神學」.
是呀!沒有分輕重的.
但是你看到聖經怎樣記載舊恩歷史.
下一課我們進入昌開及記.
我們會再看詳細一點.
但是創世記的開始.
無論是創造天地.
還是阿伯拉罕這麼重要的聖經人物列祖.
其實不只是女性主義的聖經學者抗議.
為何不提撒拉?.
為何不提創世記裡的女性?.
為何不提?.
聖經有提的.
不過是我們的文化遮蓋了我們雙眼.
到查經的時候.
我們還是喋喋不休地說阿伯拉罕多偉大.
但是說真一句.
只是阿伯拉罕離開本地本族父家.
沒有舊恩歷史.
看到嗎?.
這是創世記刻意記載.
在它的首尾呼應結構上提醒我們.
可惜我們教會從來都記得十二章.
第一到第三節.
我們從來都不記得二十四章.
利伯家離開本地本族父家.
沒錯.
古今中外文化.
重男輕女.
這是事實.
到今天都是.
我最受不了的是有些機構或政府規定.
要四分一是女性.
要規定就已經是歧視.
你明白我的意思嗎?.
弄色水出來.
今天好多了.
好多了.

$^{681}$但仍然根深蒂固.
讀聖經.
有時我們真的要放下我們的假設.
來問聖經如何教導我們.
今天我們假設男女平等.
這段聖經就沒有意思了.
有意思.
請問今天你覺得甚麼人.
跟我們香港人是不平等的?.
沒有神的形象.
這些人死刑就對了.
這些人趕他出去.
不要讓他進來就對了.
多了黃蟲.
聽到嗎?.
不是性別.
真真有詞.
甚至教會都是這樣說.
聖經如何教導?.
性別的神學.
不是矯前在特朗普總統簽署的行政命令.
將聯邦政府三十多個性別的選項.
簡化為男女兩個.
三十多個都很難想出來.
都要做到很複雜.
不是在那裡鑽牛角尖.
我們反而要問的是.
今天我們理直氣壯.
是不是聖經的教導?.
我們的盲點.
我們看我們周圍的人.
未必是做男做女.
但我們總會覺得.
我們應該上天堂.
他們應該落地獄.
上帝審判惡人.
我們是義人.
你看到嗎?.
聖經說做男做女.
阿伯拉罕.

$^{721}$尼伯加.
是完全當時的社會.
不會這樣說的.
因為這樣說是犯眾怒的.
今天也是.
教會有沒有這樣的勇氣.
在總統先生坐在下面.
主教會提議總統先生.
請你憐憫我們當中.
美國社會裡被看輕的人.
各個社會不同.
所以聖經的教導不是拍人.
我們搞定了男女平等.
看看沙孫.
我們同工教木.
姐妹多過弟兄.
牧師也是.
這不是我們的問題.
我們的問題是甚麼?.
我們要問下去.
執事會?.
不一定是性別的.
所以求主幫助我們.
昔日我們面對的.
根深蒂固理所當然的觀念.
今天可能不是相同的一件事.
但總有一些我們是理直氣壯.
我們需要謙卑.
回到神的話面前.
放下我們的想法.
學習明白上帝的教導.
所以過去的問題是男尊女卑.
今天香港誰是尊誰是卑.
非勇?人勇?面勇?還可以數下去.
還是公有?.
他不帶安全帶.
我們怎樣看我們身邊的人.
香港社會很自由,很西方,很理性,很開放.
這個問題不嚴重.
但另外呢?.

$^{761}$神的話怎樣幫助我們?.
所以讀聖經有時不是很舒服.
我們常說「安舒軀」.
聽一些很舒服的道.
不是的.
聖經怎樣說?.
我喜歡那個比喻是「兩刃的離劍」.
要很小心的.
一旦碰到就戒損了.
兩邊都鋒利的.
即是不安全的.
英國一位神學家.
更加說得好.
聖經就是上帝擺在教會的心臟旁邊.
一把兩刃的離劍.
John Webster很出名的一句說話.
求主幫助我們.
今年想跟大家從一個小眾的出發點.
去明白摩西五經.
一些我們平時很容易忽略的經文.
今天是創世記.
下一次到出埃及紀.
你喜歡可以先看經文.
或者你很熟悉.
出埃及紀還很熟悉.
尼米紀,文素紀,新明紀就未必熟悉.
求主幫助我們.
我們同心低頭祈禱.
懇求真理的聖靈開我們的心竅.
幫助我們明白聖經的說話.
並且樂意謹守進行.
不是照著字面搬字過紙.
影印這樣的墨守成規.
而是讓我們的心真的被光照.
讓我們看到今天我們需要回轉悔改是什麼事.
幫助我們能夠成為將你的說話藏在心裡.
免得我們得罪你這樣的子民.
多謝你垂聽我們不配的祈禱.
奉耶穌基督的聖名求.
阿們.

\newpage



\section{出埃及記 40:34-35}
\label{sec:YdX9gstJs1g}
\textbf{上主榮光充滿帳幕 (出埃及記40\_34-35) - 李思敬博士【繁簡字幕 by Johnson Ng】「認識神的榮耀」講道系列 - (第1講)}
\newline
\newline
連結: \href{https://youtube.com/watch?v=YdX9gstJs1g}{\texttt{ https://youtube.com/watch?v=YdX9gstJs1g}} ~~~~ 語音日期: 2025-02-05 
\newline
\newline
\hyperref[sec:cau3XNPSx68]{< < < PREV SERMON < < <}
~
\hyperlink{toc}{[返主目錄]}
~
\hyperref[ch:preacher2]{[返講員目錄]}
~
\hyperref[sec:l8BFPdIOADs]{> > > NEXT SERMON > > >}
\newline
\newline
出埃及記 40:34-35
\newline
\begin{longtable}{cl}
\hline
\hline
章節 & 經文 (和合本修訂版)\\
\hline
40:34 & \begin{tabularx}{0.7\textwidth}{X} 那時,雲彩遮蓋會幕,耶和華的榮光充滿了帳幕。 \end{tabularx} \\ \\ \relax
40:35 & \begin{tabularx}{0.7\textwidth}{X} 摩西不能進會幕,因為雲彩停在其上,耶和華的榮光充滿了帳幕。 \end{tabularx} \\ \\
[1ex]
\hline
\hline
\end{longtable}
$^{1}$請自妹樹來平安.
感恩去年廣恩二十週年.
這是去年 不是今年 不用怕.
我們一起讀了崔埃及的中間三章四章的經文.
《立約》的書卷.
今年我準備和大家進入到崔埃及記的主題.
崔埃及記的主題不是崔埃及.
不要給個名字令我們有一個先入為主的觀念.
是 崔埃及是很重要的一件歷史事實.
雖然有什麼證據 今天很喜歡講證據.
以色列人如果不是真的經歷過.
不需要講以前我們在埃及做奴隸.
不是打工 是做奴隸.
430年了 所以是很重要的一件事.
但是崔埃及記的書名是後來翻譯了希臘文才加上去.
Exodus 你認識這個字吧.
不是英文來的 是希臘文來的.
去年九月我們有機會去到希臘.
在公路上坐巴士唯一認識的一個希臘文.
就是Exodus 出口.
等於英文Exit 認識了.
因為聖經有這一卷書.
所以崔埃及記的主題是什麼呢.
我們會分開四段的經文.
來一起思想和學習.
我先讀今天的經文.
就是崔埃及記的四十章三十四,三十五節.
當時雲彩遮蓋回眸 耶和華的榮光就充滿了帳幕.
漠世不能進回眸 因為雲彩停在其上.
並且耶和華的榮光充滿了帳幕.
唯一的可惜就是和學本甚至和修版中文翻譯.
都翻譯成榮光.
其實原本cavote這個字.
好幾次出現在崔埃及記.
都是翻譯成榮耀.
這句其實都一樣.
耶和華的榮耀充滿帳幕.
結束的景象成為崔埃及記的高潮.
成為崔埃及記的焦點.
這是我們今天想和弟兄姊妹一起進入崔埃及記的主題.

$^{41}$認識神的榮耀.
我們閉上眼就想到神的榮耀是很殘硬的.
在天上 天時天國.
我們當然有這些概念.
有時候讀一節半節聖經.
但大部分是唱詩 詩歌告訴我們.
也有些背景是我們傳統民間宗教信仰.
聖經怎樣說.
關於神的榮耀 崔埃及記怎樣說.
我們要小心聆聽 一同學習.
崔埃及記可以分開三個主要的部分.
剛才我說不是主題崔埃及.
因為頭十八章早就走了.
剩下的就是崔埃及記的焦點核心.
中間的部分是西乃立約.
然後今天我們讀的結束經文.
是屬於第三部分經文.
用了十六章的篇幅去記載建造回望.
我猜二十年在廣恩有沒有讀過這部分經文呢?.
建造回望 郭良牧師點頭.
有啊 很難得.
多數我們不會有興趣讀.
這段經文很重複的 說甚麼呢?.
再分開每一個部分四個段落.
十二段 十二支派.
這個不是標準.
不過我們可以嘗試去理解.
二十五章到三十一章 上帝命令摩西做回望.
然後中間三十二到三十四章.
我們比較熟悉的金牛獨事件.
接著三十五到三十九章.
就真的落手落腳一起去建造回望的經過.
最後今天我們讀的經文四十章.
就是回望完成.
很短的時間前後不過十個月.
在西乃曠野建造回望完成.
所以去年我們讀的約書.
是屬於昌威及記中間西乃立約的部分.
你說十誡都很重要的 為甚麼不說呢?.
十多年前我寫過一輯文章在台北校園雜誌.

$^{81}$後來他們出版了這本《五經行》.
我說我不看書的.
好啊 聽吧.
多謝忠臣校友做了一個有聲版的書籍.
他們已經做了二十本.
其中一本你可以上網按五經行就已經版出來了.
所以如果你對十誡有興趣.
在約書的背景如何理解十誡呢?.
這個可以推薦大家參考.
不過今天我們要明白的焦點.
就離開了立約進入到回望的這一部分的經文.
上帝命令做回望.
二十五到三十一章幾乎所有的吩咐都再重覆一次.
三十五到三十九章.
只不過由命令句變成完成句.
如果你用文法去說.
上帝這樣吩咐 百姓就這樣做.
用不著重覆這麼長的篇幅吧.
《聖經》這一部分可以說是.
《新舊約》《聖經》六十六卷最長的重覆.
猶太的拉比錫經傳統都找不到一個大家同意的答案.
因為我們的討論很豐富.
如果我們比較在中間插入金牛讀的事件.
我們就或者可以發現原來敬畏上主.
敬拜上主的關鍵不是一個創意技巧.
今天我們很著重.
不是說他錯.
對的 有新鮮的方法.
有好的技巧難道出來大唱詩就走音嗎.
不可以的 彈琴都要練的.
這些是很基本的.
但是《聖經》提醒我們.
和金牛讀事件比較是什麼.
金牛讀是很有創意技巧的.
你有沒有留意到.
在曠野要造一隻金屬的牛讀.
不是我們主要學看的插圖.
很大隻的像維多利亞公園的女皇像.
那是插圖想像.
我見過真正的金牛讀在哈佛大學的博物館.

$^{121}$我們90年去到.
當年他們剛剛同學的考古隊伍.
在菲利士其中一個城的遺跡.
挖到一隻金牛讀.
用「挖」字可能你以為是用鏟.
用鋤頭 不是的.
考古隊伍是用牙刷的.
你知道吧 為什麼.
不要傷到那些這麼寶貴的古物.
所以早上不是挖的.
是用牙刷去掃走那些泥沙.
看到了 看到了 看到了.
就更加要小心.
找到 托在手板上而已.
這麼大隻.
不是很宏偉.
因為你想想 但是它很精緻的.
它四隻腳可以動的.
不知道是不是有AI.
會懂得自己走的呢.
總之是一個很精細的.
你開始明白了.
讀科學的 讀引擎的弟兄姊妹.
你要有一個模子.
你要倒模子.
然後你要用一個高溫的熔爐.
去熔掉那些金屬.
由固體變成液體.
倒進這個模子.
然後才會出到牛的不同身體.
頭,手,腳的部分.
沒有分手腳的 都是四隻腳.
你沒有吃過豬手.
懂不懂得分辨.
前的是手 後的是腳.
不同的 有不同味道的.
這個是實物唯一考古學找到的.
你發覺在曠野裡面.
要做一隻這樣的金牛獨.
都非常之不簡單.

$^{161}$以色列人聰明.
應該多謝他們在埃及做奴隸.
做奴隸就是做苦工.
經常被人打 錯了.
如果你對於古代世界有少許的理解.
就算去到希臘,羅馬.
古文明的世界 歐洲.
奴隸佔人口三分之一的比例.
所以作反是另一回事.
第二 奴隸原來就等於今天的專業人士.
今天的專業人士做什麼的.
醫生 做會計 做老師.
這些全部都是奴隸的職業.
所以今天你做了一隻職.
其實有根可尋.
自由人是做什麼的.
蘇格拉底帕拉圖 李士多德.
那些就是傾哲學.
讀哲學系的那些就是自由人.
社會的理解和我們今天的誤解.
其實有很大的分別.
金牛獨的事件.
以色列人用最好的拿出來服侍上帝.
他們得罪神的地方不是去拜偶像.
不要搞錯.
他們得罪神的地方就是.
上帝剛剛說完不要做任何的形象.
天上地下地底水中百物.
來代表我就發揮創意.
做了一隻在埃及文明古國.
大家都稱讚的傑作出來.
上帝應該要收穫 閱立.
不是啊 原來敬拜上帝是信服聽從.
這個是金牛獨事件.
給我們做一個對照.
我們可以明白起回眸的關鍵.
原來是領袖帶領弟兄姊妹信服聽從.
照著上帝的吩咐來做.
所以建造上主和百姓同在的回眸.
遵守十誡約書.

$^{201}$以至整個舊約聖經的五經分繪.
三者共同的基礎是一樣的.
不過知而行難.
頭腦知識很容易.
真的做起來各有各的看法.
上主怎樣吩咐 他們就怎樣作.
這是建造回眸的精髓焦點.
重覆不是無厘頭.
不是不重要.
不過我們可以再看.
如果你沒有聖經 你可以回去看.
在上帝命令做回眸的頭部份.
25章到31章.
去到31章的總結.
上帝最後的命令是做什麼.
知不知道.
多數是不知道的.
守安息人.
原來做回眸這麼大件事.
我們在西乃曠野十個月之後要起行.
快一天一天吧.
早點完工吧.
十個月計一計數.
每個星期有一天要休息.
很浪費時間播 不行的.
不可以有任何理由不聽上帝的吩咐.
所以守安息聖人的命令.
在建造回眸的總結最後一段很有意思.
聽得明白嗎.
今天我們不習慣了.
這件事很重要.
都是服侍上帝.
安息人放在一邊.
不准的 不可以的.
上帝的教導的關鍵不是快點完工.
不是做得好.
不是很有創意很有技巧.
是聽話.
聽從上帝的教導.
再看下去.

$^{241}$原來回眸已經在此.
你讀《昌埃及記》.
很少留意到這個事實.
33章在金牛毒事件當中.
經文記載交代了.
其實摩西有一個帳幕.
他放在百姓的地方外面.
要出到營外.
有一個帳幕.
這個帳幕是摩西在那裡.
與上帝會面,回眸的意思.
百姓有求問耶和華的事情.
就可以來到這個很簡單.
摩西在那裡.
所以你去露營.
上下大可能大一點.
但不是一個很宏偉的建築.
一個夠用.
摩西在那裡上班.
摩西在那裡安靜.
摩西在那裡祈禱.
這樣的回眸已經在此.
已經在此.
為何上帝又要吩咐.
再做另一個回眸呢?.
這是《昌埃及記》的記載提醒我們.
不是沒有.
現在從沒有做有.
上帝命令做回眸的經過.
郭良木斯說讀了.
不知道你來了沒有.
有沒有印象?.
第一件事情是.
吩咐百姓奉獻.
有沒有印象?奉獻.
拿實物出來奉獻.
做一個回眸需要的材料.
你想到十樣.
不要十樣那麼多.
想三樣.

$^{281}$想到什麼?.
不一定是很貴重的東西.
針線要不要?.
一定要.
每個人家裡都一定有針線.
出門沒有帶不行.
爛了褲子穿了.
都要補的.
奉獻的意義不是定一個目標.
一個預算.
一個我們大家要努力.
其實是大家參與.
你繼續看下去.
不同的材料.
百姓在家裡找到.
他就拿出來參與.
有些很簡單.
一眼就針.
需要的很重要的.
這些基本的材料.
物料.
工具.
不是比較奉獻黃金與香末藥的.
那些就很威辣.
就明明在門口有他的名字.
沒有的.
但是你能不能想像到一個事實.
來到會幕.
小朋友可以小聲跟媽媽說.
我認得這塊布.
是我家裡的.
大一點的會說.
亂的時候我有份幫忙的.
用我媽媽的眼睛針的.
我有份的.
聽不聽到這句話?.
整個會幕不是從天掉下來.
整個會幕不是天使造成.
這樣才配得上帝榮光充滿.
原來不是的.

$^{321}$在曠野.
大家合手合腳.
有的拿出來.
重要的是什麼?.
去到會幕敬拜的時候.
我有份的.
我幫過忙的.
很重要的.
所以其實上主的百姓信服進行的目的.
上帝的目的.
是讓每一個人都有份參與建造.
如果只是考創意的.
右腦的人就比左腦好.
明白我說什麼嗎?.
那些人特別有創意的.
如果要講技術的.
我學過彈琴的就比你不會音樂的厲害.
不是的.
不需要的.
不是這樣的.
會幕其實如果講深一層.
不是在講製成品有多華麗.
是你有沒有份.
神的榮耀降臨.
充滿的會幕.
原來是百姓.
每一個百姓都可以參與.
沒有強迫.
是不是?.
摩西沒有登記.
誰沒有奉獻.
抽他出來.
沒有沒有.
完全沒有.
今天我們收奉獻.
都沒有要求.
每一個人都要放東西進去.
不是這樣的意思.
甘心樂意.
但是可以參與.

$^{361}$如果你有機會.
或者以後有機會.
我們讀到利未記憲制.
第一章講梵制.
獻梵制.
很隆重.
抬一隻牛來.
不是每個人都可以獻到一隻牛.
家裡有沒有羊?.
拿一隻羊來.
你記得拿丹仙芝講的比喻.
窮人家家裡只有一隻.
怎樣拿出來獻制?.
你知不知道梵制還可以獻什麼?.
獻雀仔.
斑溝還講明不知道是什麼.
雛甲知不知道是什麼?.
我們就這樣叫.
我們叫什麼?.
乳甲.
馬上想到吃.
是不會飛的那些.
所以不用買的.
你去找就行了.
人人都可以獻梵制給耶和華上帝.
不是一定要有錢.
一定要有面子.
上帝閱立三段的記載.
獻牛獻羊獻雛甲斑溝.
都是興香的火祭.
都是耶和華所閱立.
明白這個真理嗎?.
上帝教導百姓參與在其中.
不要袖手旁觀.
當上帝的榮耀降臨的時候.
與你無關.
教會也是.
不是那個環境.
不是有多美有多宏偉.
是你有沒有份在當中.

$^{401}$當上帝在我們當中.
顯出他的榮耀的時候.
你可以心裡感恩.
我有份的.
我有參與的.
我不是旁觀的.
今天我們常以為.
崇拜來到朝見上帝.
瞻仰他的榮美.
就是聽道.
不是.
我做講道的.
講這句話都很大膽.
以後不用聽道了.
以後不用講道了.
馬丁路德說讀聖經都可以.
嚴格來說.
不要講神的話.
宣講.
但是重要的.
從《春安及記》開始.
就是百姓參與在當中.
然後當上主的榮光充滿.
充滿在哪裡?.
我有份幫忙建造的這個回望.
這是我們讀建造回望的經文.
我們需要留意.
第二件事.
回望不是聖殿.
它不是一座建築.
所以問第七.
第七章又告訴我們.
十二支派的首領.
他們商量奉獻禮物.
這一章的聖經很長.
一到八十八節.
又是一段重複很多的經文.
不過沒有《春安及記》重複得這麼細微.
都很詳細的.
那是奉獻徵信.

$^{441}$十二支派所奉獻的禮物.
完全一樣.
那用不用重複十二次?.
要啊.
因為這是交代.
沒有中間落格的奉獻徵信.
但是在十二支派奉獻.
每日一個支派之前.
首領主動.
十二個領袖.
他們奉獻了六輛篷車.
十二隻公牛.
做什麼呢?.
跟摩西說.
用來搬運回望的.
有沒有印象?.
不是每樣東西都放在車上.
你記得.
藥櫃不可以用牛車搬運的.
薩姆爾記大衛.
放了藥櫃在車上.
結果出事.
死人的.
哎呀 上帝這麼殘忍.
早就說了.
民數記已經說得很清楚.
摩西已經吩咐.
其中利美一個家族.
沒有分到一輛車.
沒有分到兩隻牛.
哎呀 奴役我們.
不是 你們是要抬藥櫃.
抬登台.
有些事不是亂來的.
聽上帝的吩咐.
但是這個車和牛的作用.
就是藥櫃也好.
會幕也好.
在西乃曠野十個月.
百姓一起聽從上帝的教導.

$^{481}$按著吩咐所建造出來的會幕.
從西乃經過三十八年.
曠野漂流.
進入了迦南.
我們不知道約書亞.
擺了會幕在哪裡.
沒有詳細 很清楚.
不過我們知道後來.
會幕在士羅薩姆爾的時候.
接著打仗.
被非利人擄走.
回到基列耶林.
最後未曾建好聖殿.
來到耶路撒冷.
這是一個歷史的事實.
哎呀 上帝漂泊.
很淒涼.
大衛也是這樣想.
我每晚有皇宮可以睡覺.
上帝還是住在帳幕裡.
如果我們記得.
列王記記載.
所羅門獻聖殿的時候.
他的祈禱.
聖殿不是神的居所.
你不是來這裡找上帝的.
上帝無所不在.
上帝和我們同在.
所以西乃不是一個聖地.
每年好像麥加一樣.
一個好的穆斯林一生.
要去一次朝聖.
你去過沒有.
香港有些弟兄告訴我.
我去約旦河洗禮.
我幾乎想問.
是不是天堂排隊.
有一個快一點的隊伍.
不用排隊了.
沒有聖地這回事.

$^{521}$西乃其實沒有人知道在哪裡.
我去過.
你知不知道你被人騙了.
他不是故意騙你.
不過如果你追問下去.
導遊會說我們去一個.
很像西乃的地方.
讓你感受一下.
是什麼環境.
沒有人知道西乃在哪裡.
西乃不是一個.
上帝在那裡.
我們去朝聖.
接近一些插頭柱香.
這是民間宗教信仰.
這不是獨一的真神.
教導他的百姓.
從頭一天開始.
上帝和他的百姓立約之後.
不是吩咐他們.
年年你們要組團.
回來西乃山.
在這裡敬拜我.
不是.
回望剛剛相反.
回望是陪同百姓.
一起從西乃入迦南.
直到王國的時期.
沒有聖地的.
就算去到耶路撒冷的聖殿.
焦點.
以塞亞書是這樣說.
主耶穌結證聖殿的時候.
也是這樣說.
我的殿是萬民禱告的殿.
開放的.
那也要去才上帝聽祈禱.
那你就連猶太會堂都不如了.
你知道猶太會堂到今天.
沒有聖殿.

$^{561}$你知道他們在會堂祈禱.
你知道他們.
是會堂安息日的祈禱.
就是等同在耶路撒冷聖殿獻祭一樣.
會堂不只是限在耶路撒冷.
全世界都有.
你知道香港也有嗎.
你不知道在哪裡嗎.
不關我的事.
我只認識廣恩.
諾便神道.
不止的.
其實有幾間會堂.
用不著這麼多嗎.
祈禱嘛.
大家來到向上帝祈禱的地方.
不是一個聖地.
只是去那裡才有用.
所以這很重要.
我帶了耶路撒冷教會的十字架.
這是我在百里行聖經學校買的.
一個紀念品.
很小的.
所以我要用powerpoint拍出來.
看不看得清楚.
我跌爛了.
所以有些膠紙痕在黏著.
它也是坎的.
不是一塊木雕刻出來的.
特別的地方是.
耶路撒冷教會的十字架.
是五個十字架.
不是一個十字架.
耶穌身邊最多只有兩個強盜.
三個十字架.
二端.
耶路撒冷教會.
這麼快說它是二端.
Mother Church.
所有教會都是耶路撒冷教會開始.

$^{601}$這個十字架很有意思.
沒錯.
它的十字架是用四方形做的.
十字架有很多形狀.
不只是我們熟悉的那種形狀.
但它特別的地方就是.
在四方形的四個空間.
它多放了四個十字架.
如果中間這個十字架代表耶路撒冷教會.
你明白其他四個十字架代表什麼嗎?.
不熟悉了.
我們極其量是讀新舊約聖經.
我懂得出埃及記已經很好了.
教會歷史我們沒有印象.
懷仁教會從《使徒行傳》.
從《保羅書信》就跳到今天.
中間二千年沒有事情發生過.
很多事情發生.
四個十字架就代表耶路撒冷教會.
四個宣教成立的教會.
第一個安提厄.
你在《使徒行傳》也讀過了.
敘利亞 今天的敘利亞.
第二個亞歷山大 埃及.
埃及有教會的 古教會.
Coptic Church.
到今天二千年來.
在阿拉伯的世界.
二千五百萬的基督徒.
從來沒有生活過在基督教的政權之下.
但是他們是很有活力的教會.
第二個十字架 埃及 亞歷山大.
第三 羅馬 保羅去到羅馬.
西方拉丁教會所熟悉的羅馬教會.
還有第四個.
我去年也去過了.
康士坦丁堡 當年的說法.
今天是土耳其的城市.
橫跨亞洲 歐洲 很有歷史的.
去到真是教會的歷史.

$^{641}$二千年來.
Constantinople東羅馬帝國的首都.
原來和西羅馬帝國的羅馬.
一樣有這麼多歷史.
很精彩.
四個教會都是耶路撒冷教會.
不是分裂出去.
是宣教.
碧白臨道 四行傳 八章四節.
分散的人往各處去傳道.
為什麼耶路撒冷教會要這樣設計十字架呢?.
很有創意.
其實這是歷史.
如果只有耶路撒冷教會.
大家也明白主後七十年.
耶路撒冷被羅馬軍隊毀滅.
聖殿又被拆毀.
又來亡國被擄.
沒有國家可以亡 也沒有被擄.
只是耶路撒冷變成一個荒涼的遺跡.
如果教會只是停在耶路撒冷.
每年信耶穌的外邦人都要去耶路撒冷.
七十年之後就再沒有耶路撒冷教會了.
為什麼今天有廣恩堂呢?.
因為有宣導會.
為什麼有宣導會呢?.
因為有長老會.
孫信博士是長老會的牧師.
為什麼有長老會呢?.
因為有加爾文.
有歐洲的改革運動.
為什麼有歐洲的教會呢?.
羅馬教會.
你追上去 記得嗎?.
這是教會歷史.
上帝的恩典 上帝的福音.
不是聚在一個地方.
全世界的人都只是來這裡才最接近上帝.
所以不要以為退休去到山上.
很寧靜的地方就接近上帝.

$^{681}$又是民間宗教迷信.
聖經不是這樣說的.
上帝是在祂的百姓中間.
祂的百姓從西乃到曠野.
進入了迦南回望.
一路用十二隻公牛.
六輛篷車載著一起去.
每一次搭起帳幕.
神的榮耀在祂的百姓中間.
不是上帝在不在.
是我們在不在.
還是我們隔著Zoom來看上帝的榮光.
袖手旁觀 保持安全距離.
小心啊 小心啊.
這是說明你要明白神的榮耀.
你要認識神的榮耀.
從聖經開始.
從出埃及記最早的歷史開始.
新約聖經也很清楚告訴我們.
道成為肉身.
耶穌基督住在我們中間.
讀過神學的人.
同學 校友都一定記得住字.
和出埃及記上帝吩咐.
你們為我做一個聖所.
使我可以住在你們中間.
是同一個字.
那個不是說住屋.
神沒有屋企的.
是天上的天上.
且不夠祂做寶座.
這個住字原來是搭一個將王的動詞.
Mishkan的動詞 很有意思的.
耶穌基督來到我們中間.
但是祂是有因典有真理.
而約翰說.
祂的榮光 負獨生子的榮光.
是我們見過的.
明白那個一脈相承的教導嗎?.
認識神的榮光.

$^{721}$不用看異象.
今天還要說.
我拍到一張照片回來.
很漂亮啊 長雲啊.
嘩 上帝出現了.
不是的 不是的 不是的.
我們仍然停留在民間宗教的迷信當中.
聖經的教導.
上帝吩咐他的百姓.
一起參與 聽話.
不要自作主張.
當我們信服上帝.
當我們一起去到哪一處.
耶穌都說過.
兩三個人祈禱.
我就在他們中間.
那個只是小組.
要翻崇拜.
還要二十週年坐在一起.
拍一張全體照片.
才是榮耀.
你又搞錯了.
行不行?.
要這樣說才明白.
不是說一切不重要.
重要.
不過小組.
今天有一半人病了.
只有兩三個人.
讀一下馬太福音.
主耶穌說.
我就在你們中間.
神的榮光 負獨生子的榮光.
這是《春海及記》開始的教導.
你說最後那一章.
是 《春海及記》開始.
我們才進入五經裡面第二卷書.
教導百姓.
怎樣認識上帝的榮耀.
下一堂我們再明白.

$^{761}$因點真理.
才是神的榮耀.
不是神話裡的情節.
上帝出現的情景.
要回到聖經裡看.
如果不回到聖經裡看.
我們總有自己創意的想像.
很多加鹽加醋的.
說到似層層.
我見異象你還不信?.
不是不信.
是不信你.
分不分得到?.
我信上帝.
全能的父.
創造天地的主.
我信我主耶穌基督.
我信的不是你.
我信的是聖經.
可以嗎?.
我們希望能夠謙卑.
來到主的根前.
從《春海及記》我們明白.
原來神的榮耀在我們中間.
不需要完美的.
曠野所造的回望有多完美.
但是一針一線.
一手一腳.
大家都有份的.
教會有多完美.
不完美的.
明白嗎?.
我要找一間完美的教會.
不吵架的.
沒有的.
保羅怎麼說.
我經常重複.
保羅說基督的身體.
有眼,有手,有腳.
不能眼對腳說.

$^{801}$我不需要你.
開會大家都是眼很快樂的.
十年計劃.
但腳呢?.
腳不是用來看東西的.
是用來踏實地的.
所以眼和腳一定吵架.
不吵架的.
全身都是眼.
那不是基督的身體.
那不是教會.
那是政黨.
那是同溫層.
那是圍爐取暖.
現在TikTok還是小紅書.
連在一起.
不是的.
教會不是這樣的.
教會就是一個.
大家來到神的面前.
學習,信服,進行.
你看看在起會幕的過程當中.
竟然會出現這麼大件事.
都可以的.
《昌海禽記》就是這樣記載.
四個段落.
記得嗎?.
25章到40章.
很有意思的經文.
幫助我們踏入新的一年.
我們怎樣來迎見神的榮耀.
在我們廣恩當中.
我們同心低頭祈禱.
多謝主耶穌.
你來到這個世界.
成為人.
並且成為奴隸.
且死在十字架上.
但是就是在十字架上.
你榮耀父神.

$^{841}$亦都是父神榮耀祂的兒子的時候.
多謝你告訴我們.
原來我們需要學習.
是怎樣去見到你的榮耀.
不是在我們自己的想像當中.
不是在我們的傳統當中.
而是在你的說話裡面.
多謝你留下聖經在我們的手.
使我們可以學習明白遵行.
並且得到自由.
可以快跑跟從我們的主.
誰聽我們的祈禱.
奉耶穌基督的聖名.
求阿們.
多謝主耶穌基督的聖名.
求阿們.
多謝主耶穌基督的聖名.
求阿們.
多謝主耶穌基督的聖名.
求阿們.
多謝主耶穌基督的聖名.
求阿們.
\newpage



\section{詩篇 148:1-14}
\label{sec:l8BFPdIOADs}
\textbf{合神心意的讚美 (詩篇148\_1-14) - 李思敬博士【繁簡字幕翻譯 by Johnson Ng】}
\newline
\newline
連結: \href{https://youtube.com/watch?v=l8BFPdIOADs}{\texttt{ https://youtube.com/watch?v=l8BFPdIOADs}} ~~~~ 語音日期: 2025-01-25 
\newline
\newline
\hyperref[sec:YdX9gstJs1g]{< < < PREV SERMON < < <}
~
\hyperlink{toc}{[返主目錄]}
~
\hyperref[ch:preacher2]{[返講員目錄]}
~
\hyperref[sec:fJrsPMmDHtU]{> > > NEXT SERMON > > >}
\newline
\newline
詩篇 148:1-14
\newline
\begin{longtable}{cl}
\hline
\hline
章節 & 經文 (和合本修訂版)\\
\hline
148:1 & \begin{tabularx}{0.7\textwidth}{X} 哈利路亞!你們要從天上讚美耶和華,在高處讚美他! \end{tabularx} \\ \\ \relax
148:2 & \begin{tabularx}{0.7\textwidth}{X} 他的眾使者啊,要讚美他!他的諸軍啊,都要讚美他! \end{tabularx} \\ \\ \relax
148:3 & \begin{tabularx}{0.7\textwidth}{X} 太陽月亮啊,要讚美他!放光的星宿啊,都要讚美他! \end{tabularx} \\ \\ \relax
148:4 & \begin{tabularx}{0.7\textwidth}{X} 天上的天和天上的水啊,你們都要讚美他! \end{tabularx} \\ \\ \relax
148:5 & \begin{tabularx}{0.7\textwidth}{X} 願這些都讚美耶和華的名!因他一吩咐就都造成。 \end{tabularx} \\ \\ \relax
148:6 & \begin{tabularx}{0.7\textwidth}{X} 他將這些設定,直到永永遠遠;他訂了律例,不能廢去。 \end{tabularx} \\ \\ \relax
148:7 & \begin{tabularx}{0.7\textwidth}{X} 你們哪,都當讚美耶和華:地上一切所有的,大魚和深洋, \end{tabularx} \\ \\ \relax
148:8 & \begin{tabularx}{0.7\textwidth}{X} 火和冰雹,雪和霧氣,成就他命令的狂風, \end{tabularx} \\ \\ \relax
148:9 & \begin{tabularx}{0.7\textwidth}{X} 大山和小山,結果子的樹木和一切香柏樹, \end{tabularx} \\ \\ \relax
148:10 & \begin{tabularx}{0.7\textwidth}{X} 野獸和一切牲畜,昆蟲和飛鳥, \end{tabularx} \\ \\ \relax
148:11 & \begin{tabularx}{0.7\textwidth}{X} 世上的君王和萬民,領袖和世上所有的審判官, \end{tabularx} \\ \\ \relax
148:12 & \begin{tabularx}{0.7\textwidth}{X} 少年和少女,老人和孩童, \end{tabularx} \\ \\ \relax
148:13 & \begin{tabularx}{0.7\textwidth}{X} 願這些都讚美耶和華的名!因為獨有他的名被尊崇,他的榮耀在天地之上。 \end{tabularx} \\ \\ \relax
148:14 & \begin{tabularx}{0.7\textwidth}{X} 他高舉自己百姓的角,使他的聖民以色列人,就是與他相近的百姓得榮耀。哈利路亞! \end{tabularx} \\ \\
[1ex]
\hline
\hline
\end{longtable}
$^{1}$剛才我們一起讀的詩篇148篇.
其實是舊約聖經當中150首詩篇這本書最後的5篇.
叫做Hallelujah詩篇.
為什麼呢?因為這5首詩篇開頭結束都是同一個字Hallelujah.
我們華人教會不是認識很多非白來文.
不過起碼認識兩個字.
一個是Hallelujah 另一個是阿們.
這個是非法來文 不是國道門.
這個是翼翼Hallelujah都是翼翼.
是什麼意思呢?.
CV148T的第一節和十四節.
其實翻譯的意思是你們要讚美主.
開始從這句話Hallelujah結束最後Hallelujah.
所以5首詩篇都是用這個格式.
我們一般會叫做Hallelujah詩篇.
作為詩篇這本書的壓軸,總結,高潮.
148篇就是這5篇詩篇的中間篇.
你心水清數一數.
它教我們怎樣讚美神.
每一首詩篇都很整齊.
好像我們唱聖詩一樣.
我當然在說剛才主席弟弟好好介紹了.
同年代的同的詩歌 最後那首粵詩.
是可以小時候唱的.
懂得唱的都是小時候才唱的.
現在不唱了 除了四部之外.
其實它每一節都很整齊.
有沒有留意到?.
比較新的 不是叫新潮.
新的一代的聖詩.
可能沒有了每一節齊齊整整的間式.
喜歡粗一點 喜歡短一點.
沒問題的 這是創作.
詩篇148篇其實很整齊.
5個段落 或者我們用今天唱詩的說法.
每一個段落裡面有三句六行.
很整齊的.
第一段 由第一節.
「你管要讚美主 哈利路亞」之後開始.
一直到第三節.

$^{41}$這個叫做第一段.
然後第四節到第六節.
又是三句六行.
很工整的格式和結構.
第三段七到九節.
第四段十到十二節.
到這裡為止.
《玄學聖經》的節數都沒有配合.
每一段三節.
所以四段十二節.
最後那兩節就要馬上分了.
因為它只是分了兩節.
其實都是三句六行的一段.
試過結束的最後一段.
所以如果我們就這樣用和合本去讀.
我們未必察覺它原本.
《非白萊文》原來是一首很整齊的詩歌.
好像有福的確區那樣.
有時我們崇拜長期港獨主席.
看一看標筆 過了中就已經指揮心急.
最後那首詩歌.
我們唱第一節和最後那一節.
閹了中間三節.
是呀 不行的 其實真的不行的.
因為作者每一節有他的重點.
你拿走中間的.
派出那些五節變成三段兩節.
要不就不要唱了.
你這樣部分的內容出來.
其實有時我們不知道他在說甚麼.
CP148篇五段說甚麼的.
他都有 起門要讚美主.
我們一定認為聖經是在叫我們讚美主.
我們這一班弟兄姊妹.
我們這一班肅神的兒女.
我們聚集在星期日聖餐的場前.
我們在崇拜 不如說敬拜讚美.
還不是國樂地.
有沒有留意到CP148篇.
你們要讚美主.

$^{81}$這個你們是說誰的.
我嘗試重新翻譯.
想把格式排出來讓你看到.
頭一段其實每一句每一行.
都是用你們開始的.
你們要讚美上主.
你們要讚美客.
不怕心虛 再讀多四次.
每一句都是你們.
你們是誰?浩瀚諸天 雲霓高處.
是你們家嗎?.
我住在十二樓.
都沒有去到浩瀚諸天 雲霓高處.
所有他的使者 主的僕人.
太陽 月亮 明亮的星宿.
第一段原來不關我們事.
完全不是跟我們說話.
不是叫我們來讚美.
不要覺得自己很緊張.
哎呀 差了四步 失去了三音.
還沒有排隊.
這是在說天上的讚美.
神的僕人 太陽 月亮 星宿.
這個窮瘡宇宙超過我們的氛圍.
我們有些見到 有些見不到.
原來是在跟他們說.
你們要讚美神.
是你們要讚美主.
所以每一段有它的重點.
第一段不是在說我們.
原來天上的讚美早在我們出現之前.
它已經在讚美著.
大家明白嗎?.
詩篇是這樣開始的第一段 很重要的.
重複北京說你們要讚美.
沒有一句你們要讚美是跟我們說.
我們忍不住了.
我們覺得自己很重要.
還怎麼不跟我們說話.
你小心點 有賴聖明 石經的教導.

$^{121}$教我們怎麼讚美 但是他先幫我們看看.
天上是怎樣讚美的.
我們又是想得不對.
他們的詩班很漂亮的.
他們七步合唱的.
他們是整個交響樂團的.
天使天君的讚美的詩歌.
很大聲的.
我們是將我們的敬意.
認為天上的讚美.
都一定好像我們這樣的讚美.
所以一定要懂得唱歌.
「阿彼日」你不要走音啊.
如果不是家教詩班讚美神.
在天上走音就不好了.
天群第二段解釋給我們聽.
他們讚美上主的名.
是有三個原因.
第一是命令 第二是實建立.
第三是地化.
三個動詞不是說我們.
是說上帝的.
天上的讚美的基礎是因為上帝命令.
於是他們就被創造.
上帝建立他們才無窮的世代.
從被創造到今天.
太陽,月亮,星宿,嬰兒,運作.
最後上帝定下界限.
他們會越過,會走歪了.
你看到嗎?這三件事情.
其實不需要醒悟,不需要唱歌.
是他們怎樣遵行上帝從創造開始的命令.
上帝建立他們.
他們遵行上帝的命令.
於是他們從古到今.
一直以來從來都沒有歪了.
太陽轉錯了.
月亮又撞到地球.
全部都是按照上帝的吩咐.
這是CP148P第二段告訴我們.

$^{161}$天上的讚美是以他們遵行上帝的吩咐.
這個事實就已經是一個對式的讚美.
所以不要介意你唱歌走音還是不走音.
記得第一首清唱,很棒.
沒有走音,怎麼唱到最後那句.
伴奏進來,沒有被壓迫.
我們讀音樂的,Absolute Pitch.
厲害啊,只是他可以讚美.
我們當中有沒有Arnold Tone的弟兄姊妹?.
唱音一定不準的,一定走音的.
有,有我就是了.
你可以讚美神的,你不用自卑的.
因為天上的讚美不是說你的聲音有多好.
中年沒聲音,不是的.
不是靠聲音的.
我要這樣說你才能印象到.
讚美上帝原來不是靠聲音的.
靠聲音就要人多勢眾.
就要音準,就要搏擊隊.
就要很多這些東西.
我們以為天上的讚美不是.
其實如果這樣看一個人.
一個太陽.
他上帝命令創造他.
上帝建立他從創造到今天.
他的運作沒有逾越上帝定下的誡法.
遵守上帝的命令.
這個已經是讚美了.
這是天上的讚美.
這是一個簡化.
可以說詩篇148篇的上半篇.
不是說我們.
是抗拒天上.
不過你明白了.
其實是一個釋放.
因為很多時候弟兄姊妹覺得.
要音樂很高水準.
才可以參加勁拜讚美.
原來不是計較這些的.
好,這個是上帝所創造.

$^{201}$獨一的上主.
所以整個宇宙都聽他天父的聖人.
下次我們唱.
這是天父的聖人.
懂不懂唱.
很多教會都懂得唱這首歌.
這麼老套的歌.
是很有意思的.
裡面有一句話.
罪惡雖然好像得勝.
每個字都很重要.
罪惡套帖.
不是,雖然罪惡套帖.
也不是,這首詩歌說.
雖然罪惡.
好像得勝.
是不是真的得勝.
其實不是.
不過很像他將那寶藏.
所以這些詩歌不要太聽它.
因為明白了.
基督的教導從創造到今天.
讚美上帝的是宇宙當中.
神的僕人.
太陽月亮星宿還有大水.
都是按照上帝的命定.
祂的建立.
然後祂定下的屆時運作到今天.
這個是他們的讚美.
再看下去.
第三段七到九節.
你們要讚美上帝.
你們說誰.
到我們了.
到沒有.
不要那麼快立口字.
來不及了.
一定是說我們.
.
從大地蒼茫.

$^{241}$你看看.
.
一眾海獸深淵火焰冰雹飄雪煙霧暴風.
.
那處香柏樹.
你聽到嗎.
.
幽靈無論山高低也好.
他們都讚美上帝.
這個是地上我們比較熟悉.
接觸見到摸到的市民經歷到.
所以昨晚呼呼聲北風.
你家的窗有沒有響.
如果不響就很感恩了.
因為暖.
家裡有個北風.
很大聲.
那些聽淺行鼠的說話的暴風.
都在讚美.
你不需要說服著.
他們說服著讚美讚美讚美.
不是我們有共鳴.
我們有私關.
他按照上帝的命令創造.
來到運行.
這個就是天地的讚美.
再到第四段.
.
終於有本地份了.
.
你屬於年輕的使英勳舉手.
有人舉手的.
.
我失婚了.
我還沒有拿過旅遊卡.
少女.
又有.
老年人拿了旅遊卡.
舉手.
.

$^{281}$當然有份的.
還有剛才離開的孩童.
有份的.
讚美的名單.
在地上兩段的詩篇.
說什麼呢.
如果你懂得對照一看.
其實這裡的次序.
是創世記的次序.
上帝創造天地.
在第一章創世記記載.
深淵在第二章.
圓明在第一章.
圓明在第二章.
圓明在第一章.
鳥和海獸在二十一節出現.
最特別的是.
為什麼我說是這個次序.
二十五節.
創世記野獸生畜昆蟲.
各蟲其類.
同樣的次序.
在一百四十八篇.
野獸生畜昆蟲.
無理由的這樣提法.
是不是從大到小都不像.
什麼意思呢.
因為這是提回創世記上主的創造.
所以這個大地是屬於他的.
讚美他.
因為上帝命令建立.
上帝幻定了界線.
不過第四段沒有停在創世記.
十一,十二節,一百四十八篇.
所列出來的烈王,百姓,君主,領袖,年輕的男女,老年人,小朋友.
這個沒有出現在創世記.
有啊,做人嘛,做男做女.
如果上半的名單次序是這麼清楚的.
這個後半就不是出自創世記.
這個是從創造進入到人類的歷史.

$^{321}$君王,百姓,領袖,男女,老年人,小朋友.
我們中文都有這個識字.
創造就是美神.
但是原來CP148篇提醒我們讚美不是停在創造.
第六節,剛才那裡說上帝.
劃出範圍和合本的翻譯.
就說他定了命,不留廢渠.
如果你有聖經在手.
記住和合本中文的翻譯者有發明詞.
加多了幾個小字.
註解廢渠或作越過.
再準確些譯或作愚愚.
懂不懂這兩個字?.
純粹,說什麼?愚之.
就是那個字,你現在明白了,讀聖經是讀的.
愚之說什麼?說創造?.
不是,愚之是說春愛求.
是說神的拯救.
是說十二支派離開遺牢之地.
上帝帶領他們經過曠野.
去到英舉的嘉納.
所以是一個很簡單的讀詞.
不是細閥,也不是越過.
是愚越.
說回上帝的拯救.
君和在十一節出現.
這個很明顯覺得是聖經的歷史.
人類的歷史.
還有十四節.
十四節和合本的翻譯.
就說那些一切的聖民,以色列人.
其實在十四節是說忠於上主的百姓.
忠於這個字眼是神的信實持耐的字眼.
這本來文是黑色的.
神守約斯持耐.
所以不只是國情道.
是說以色列以後所走的道路.
神和他們同在.
他們也要忠於耶和華.
他們的神.

$^{361}$這裡說的是從創造進入到歷史.
如果我們說創造的萬物讚美神.
大家都心裡會點頭.
我們去到退休,去到郊外,去到高山,神山.
我們特別親近神.
我們獨善見身而要默契相合.
創造的不是只躲在高山上.
要近一點,窗師又小聲,神明聽不到.
聖經說神是部署不在的.
不是咖啡檸檬油.
是的,不用創造到山頂.
這不是聖經的教導.
上帝不只是停在創造的完美世界.
人類歷史墮落了,淒涼了.
上帝不在了.
創世記開頭說.
神看著他所做的一切都甚好.
有晚上,有早晨,是第六天.
這個好字在創世記第五十章.
結束之前,若瑟跟他哥哥說.
從此你們的意思是要我蓋墓.
聖經背下的.
但神的意思,願是好的.
最有理由埋怨上帝不好的若瑟.
十七歲迫埋到埃及.
從此沒有機會回到家.
到他最後一句說話.
他說神的意思.
在我們那些不好的作惡多端.
我們彼此不是相愛,是互相仇恨.
在人類歷史的現實裡.
神的意思,願是好的.
在今天香港,神的旨意是好的.
你知不知道,只有在那些地方.
自由民主,可以不受阻礙的見拜.
我們都會回應.
那次是指控你,所有公共場所都不可以進會.
不是針對宗教的.
不要這樣看,不要這樣說.
不要這樣借題發揮.

$^{401}$以為迫不及待.
沒有,神的旨意是好的.
在外國是好的.
在西方是好的.
特朗普做總統的美國也是好的.
你要知道這樣看才行.
不只是我們喜歡的地方就好.
不喜歡的地方就不要上班.
不是這樣做的.
如月君王,忠於.
在埃及遺老之地,上帝請求.
在西亞抗疫,上帝立約.
然後在旺國時期,上帝和他的百姓同坐.
這離開了《創世記》第一章.
你在大字頁當中見到.
他們就是美神,沒錯他們就是美神.
但是在歷史的驗診裡.
在人類的罪惡當中.
在一塌糊塗,天三倒四.
人的那種,不懂得說,總之你明白他就不明白.
神的讚美在不在?.
在啊.
所以最後一段高潮總結壓軸.
讚美就算是一個這樣的世代.
這樣的邪惡的世界是屬於所有忠於他.
這句一定要說出來.
因為在每一個地方都有忠於上帝的子民.
有忠於上帝的百姓.
是天父的兒女,耶穌基督的末徒.
這句話在詩篇的時候.
跟著下半句解釋.
是讚美屬於以色列的眾子.
讚美屬於親近耶和華神的百姓.
是的,人人都遠離上帝.
人人都不認識上帝.
無神主義橫行.
東方西方,伏斯林,哎呀好多.
不是啊,不是看這些.
原來聖經現在來到訪我們了.
在現實的世界當中.

$^{441}$不是關上門,我們這裡聖潔,不是啊.
在整個大地上,上帝的讚美.
是透過那些懂得遵循他的旨意的百姓.
我們就好像天上神的僕人.
太陽月亮星宿.
他們按照上帝的命令來運轉.
從開始到今天.
這是我們讚美需要效法學習的一個典範.
但是這裡其實再很清楚地說下去.
目前只是以色列認識上帝.
外邦未曾認識過.
香港有多少基督徒都沒有計算過.
剛剛才數完一大堆.
大概還有百分之十有什麼移民走了一大堆.
你們教會可能很厲害.
我們教會走200個界線.
嘩,那就是那百分之十就跌到百分之八了.
所以不是數字.
從來2000年來.
從來在教會歷史上.
教會都不是大多數的.
有沒有留意這個事實?.
教會都不是每個人都信耶穌的.
在歐洲每個人出生不是拿出生紙.
去教會應驗洗禮.
但是他信不信的?.
他都不信的.
有些國家不用收貢獻的.
因為你生出來洗禮.
交稅的時候他們扣十分之一給教會的.
就吵了.
我不要,我不要交給教會.
加拿大我也不是讀教會學校的.
我不免稅.
很多這些的爭論.
不是在說真的忠於上帝的人.
但是有很多這樣的神的子民.
2000年來不是生活在基督教的國家.
我們香港教會的英子會認不認識他們?.
不認識的.

$^{481}$埃及我們只認識去那些金字塔下面.
燒烤,吃宵夜,打卡.
你知道埃及最古老的教會.
Coptic Church.
從司道行傳到今天.
從來沒有在任何埃及.
從來沒有出現過基督教的政府.
生活在2000年.
敘利亞的教會追上司道行傳的安提拉教會.
你認識嗎?.
KINGMA力豐基金新的總閣.
敘利亞的牧師Ria Cassis.
我們三年的時候認識她.
我們家的教會可以追上安提拉教會.
敘利亞的教會.
多懸廟啊.
又內戰,又恐怖分子,又伊斯蘭國.
2000年來他們的基督徒屬於上帝.
他們忠於上帝.
你知道在阿拉伯世界不只是回教徒.
你知道有多少徒兒.
你知道有多少基督徒.
今天1800.
不是生活在西方國家.
是生活在回教的社會.
所以我們要針對大地來看.
是啊是啊很淒涼啊.
受逼迫.
其實這個是屬於,忠於上帝的百姓.
上帝不是要我們走在一起快快樂樂.
他們要我們永遠快樂地生活.
他是把我們猜險出類.
從耶路撒冷,猶太地域滅絕下去.
撒瑪利亞直到地極.
做甚麼?.
作那些見證.
說教士就是這樣.
哎呀那裡不信耶穌教.
哎呀那裡迫不及待教會教.
哎呀那裡信耶穌會趕出家門的.

$^{521}$專教士說過他.
是啊.
這個事實.
其實親近神.
不是視乎你住在哪裡.
你親近上帝.
上帝就可以把你放在一群.
沒有親近神的人中間.
你家裡.
你的公司.
你教書也好.
上帝給一群學生一年.
當中有多少人信耶穌?.
哎呀.
是啊.
上帝就把你放在他們中間.
這裡不是說一個很理想.
百分之一百.
在人類的歷史上.
其實我們是要引領外邦萬民回傳.
以至到有一天.
譬如說地上的君和百姓.
君主領袖.
男女老幼都懂得讚美神.
但今天還不是.
今天還不是.
不是閉上眼睛.
這是我們的使力.
效法天地宇宙萬物.
記得上半篇的CP.
不是靠我的作法.
是我們的生命.
我們的生活.
遵行上帝的吩咐.
這樣的生命.
這樣的生活.
上帝放在我們這一群不認識祂的人當中.
這是我們的使命.
帶領他們和我們一起同心既拜讚美祂.
可不是說不用唱詩.

$^{561}$下次回來聽完這篇講道.
我們不再唱詩了.
不是.
唱詩是很快樂的.
你不記得很多金句.
你記得那首歌的.
你不記得那些字.
你都記得那個旋律的.
音樂是很特別的.
是值得的.
不過不只是靠聲音來讚美神.
你明天的生活是怎樣的.
更加重要的.
不只是我們這群人讚美神.
還有外面.
未曾認識神.
這是我們的使命.
讚美包括身份.
這個身份是遵從上主.
讚美也包括使命.
這個使命是為主作見證.
讀完這句詩.
你對讚美有沒有一個不同的了解.
這就是聖經的教導.
幫助我們.
是 我們還可以唱詩.
待會有回憶.
試過大聲唱.
不過不是因為樓頂很矮.
大家轉頭就聽不到.
所以在我們中間.
我們的生命.
是不是真的這樣來賺錢.
讚美神.
然後吸引未曾認識神的人.
特意要去找他們.
不用把他們扔在香港.
以前我們都說要去內地傳播音.
現在每月150個單程正下來.
已經下來20多年了.

$^{601}$是吧.
有大量人口下來.
還有什麼高財通,優財通.
還有什麼通.
全部在我們門口.
就在石門.
就在京穗.
你搭電梯的時候.
有沒有聽到那些口音.
有沒有聽到普通話.
這個上帝將未認識他的人.
放在我們身邊.
你記得你們要讚美耶和華.
包括君王領袖.
包括男女老幼.
在歷史當中.
我們的使命.
也是我們讚美主的時候.
我們承擔對上帝的重視.
我們唐十禮拜.
多謝天父的恩典.
因為你讓我們先認識你.
原來不是叫我們特別無恩.
特別不幸.
而是成為我們身邊的人的見證.
透過我們的信福.
透過我們生命的改變.
都將你的舊一介紹給我們身邊的人.
無論他們是我們的家人.
是我們的同事.
是我們的朋友.
還是我們學校,工作場所.
我們所服侍的群體.
原來都是你放我們在當中.
是你的子.
我們就學校化.
天地宇宙你的服役.
這樣去遵行你的吩咐.
行走在你所劃定的界限當中.
成就你美好的志意.

$^{641}$多謝你.
誰聽我們不背地祈禱.
共耶穌基督的名堂.
多謝你.
\newpage



\chapter{楊慶球}\label{ch:preacher3}
\begin{multicols}{3}
\minitoc
\end{multicols}
{ \scriptsize


\begin{xltabular}{\textwidth}{|p{0.15\textwidth} p{0.6\textwidth}|p{0.07\textwidth} p{0.1\textwidth}|}
\hline
啟示錄 2:18-29 & \hyperref[sec:fJrsPMmDHtU]{紛亂中的堅守︰推雅推喇教會 (啟示錄2\_18-29) - 楊慶球博士} & 2025-02-09 & \href{https://youtube.com/watch?v=fJrsPMmDHtU}{\texttt{ fJrsPMmDHtU}} \\
\hline
\end{xltabular}
}
\newpage



\section{啟示錄 2:18-29}
\label{sec:fJrsPMmDHtU}
\textbf{紛亂中的堅守︰推雅推喇教會 (啟示錄2\_18-29) - 楊慶球博士}
\newline
\newline
連結: \href{https://youtube.com/watch?v=fJrsPMmDHtU}{\texttt{ https://youtube.com/watch?v=fJrsPMmDHtU}} ~~~~ 語音日期: 2025-02-09 
\newline
\newline
\hyperref[sec:l8BFPdIOADs]{< < < PREV SERMON < < <}
~
\hyperlink{toc}{[返主目錄]}
~
\hyperref[ch:preacher3]{[返講員目錄]}
~
\hyperref[sec:toZa1ewaUWE]{> > > NEXT SERMON > > >}
\newline
\newline
啟示錄 2:18-29
\newline
\begin{longtable}{cl}
\hline
\hline
章節 & 經文 (和合本修訂版)\\
\hline
2:18 & \begin{tabularx}{0.7\textwidth}{X} 「你要寫信給推雅推喇教會的使者,說:『神的兒子,那位眼睛如火焰、雙腳像發亮的銅的這樣說: \end{tabularx} \\ \\ \relax
2:19 & \begin{tabularx}{0.7\textwidth}{X} 我知道你的行為:愛心、信心、勤勞、忍耐;又知道你末後所行的善事比起初所行的更多。 \end{tabularx} \\ \\ \relax
2:20 & \begin{tabularx}{0.7\textwidth}{X} 然而,有一件事我要責備你,就是你容忍那自稱是先知的婦人耶洗別教唆我的僕人,引誘他們犯淫亂,吃祭過偶像之物。 \end{tabularx} \\ \\ \relax
2:21 & \begin{tabularx}{0.7\textwidth}{X} 我曾給她悔改的機會,她卻不肯悔改她的淫行。 \end{tabularx} \\ \\ \relax
2:22 & \begin{tabularx}{0.7\textwidth}{X} 看吧,我要使她病倒在床上。那些與她犯姦淫的人若不悔改他們的行為,我也要使他們同受大患難。 \end{tabularx} \\ \\ \relax
2:23 & \begin{tabularx}{0.7\textwidth}{X} 我又要殺死她的兒女,眾教會就知道,我是那察看人肺腑心腸的,我要照你們的行為報應各人。 \end{tabularx} \\ \\ \relax
2:24 & \begin{tabularx}{0.7\textwidth}{X} 至於你們其餘的推雅推喇人,就是一切不隨從這教訓,不明白他們所謂撒但深奧之理的人,我告訴你們,我不會再把別的擔子放在你們身上。 \end{tabularx} \\ \\ \relax
2:25 & \begin{tabularx}{0.7\textwidth}{X} 你們只要持守那已經有的,直到我來。 \end{tabularx} \\ \\ \relax
2:26 & \begin{tabularx}{0.7\textwidth}{X} 那得勝又遵守我命令到底的,我要賜給他權柄制伏列國; \end{tabularx} \\ \\ \relax
2:27 & \begin{tabularx}{0.7\textwidth}{X} 他必用鐵杖管轄他們,如同打碎陶器, \end{tabularx} \\ \\ \relax
2:28 & \begin{tabularx}{0.7\textwidth}{X} 像我也從我父領受了權柄一樣。我又要把晨星賜給他。 \end{tabularx} \\ \\ \relax
2:29 & \begin{tabularx}{0.7\textwidth}{X} 凡有耳朵的都應當聽聖靈向眾教會所說的話。』」 \end{tabularx} \\ \\
[1ex]
\hline
\hline
\end{longtable}
$^{1}$各位早安 新年快樂.
每年的新年很多人都去求福.
特別是香港人如果不是返教會就去廟宇求福.
上頭炷香 你見很虛澄 香港台灣.
就是希望求一年的福氣或者一生的福氣.
就是財富 健康 求福其實無可厚非.
我們每個人都盼望有健康 有財富.
不過在過程裡面你會感覺到很多人.
就會將求福的對象當作提款機.
有求必應 求多少給多少.
在聖經裡面提到上帝是賜福的主.
上帝是不可以成為一個提款機.
所以真正賜福 上帝按他的心意將最好給我們.
未必一定要給夠錢 未必一定要給即時的健康.
因為神有計劃.
今日我講這段經文 區亞特拉教會裡面有一個教導.
耶駛別 在當時來說是一種西族文化.
求財富 求健康.
在教會很受歡迎 因為大家都很想.
但是在聖經裡面說.
這件事將財富代替了上帝.
人們的重心就放在金錢裡.
就是拜馬媒不要上帝.
於是責備他們說他們行淫.
淫亂的意思是他們心不喜歡上帝 喜歡金錢.
這是一個最大的問題.
今日我們先看這段經文 經文相當長.
我們先讀一次.
你要寫信給區亞特拉教會的使者說.
神的兒子那位眼睛如火焰 雙腳像發亮的銅的這樣說.
我知道你的行為 愛心 信心 勤勞 忍耐.
又知道你末後所行的善 比起初所行的更多.
然而有一件事我要責備你.
就是你容忍那自稱是先知的婦人 耶駛別教唆我的國人.
引誘他們犯淫亂 吃掣偶像之物.
我曾給他悔改的機會 他卻不肯悔改他的任何.
看吧 我要使他病倒在床上.
那些與他犯奸淫的人若不悔改他們的行為.
我又要使他們同受大患.
我又要殺死他的兒女.

$^{41}$宗教會就知道我是那察罕人 肺腑深長的.
我要照你們的行為報應各人.
至於你們其餘的推也推拉人.
就是一切不隨從者教訓.
不明白他們所謂 察但心無知理的人.
我告訴你們 我不會再把別的擔子放在你們身上.
你們只要持守那已經有的 直到我來.
下得聖又遵守我命令到底的.
我要賜給他權柄制服列國.
他必用鐵杖管轄他們 如同打碎陶器.
像我也從我父領受了權柄一樣.
我又要把神聖賜給他.
凡有耳朵的都應當聽聖靈向宗教會所說的話.
一封很長的信.
七封信裡這封是最長的.
但這教會不是最大的教會.
是一個比較小的教會.
還有這教會也不是很顯眼.
我們看地圖就看到了.
由開始以弗所的大港口那裡影響很大.
約翰也在那裡做過長老.
善美娜也是一個很繁榮的城市.
人數很多 教會影響也很大.
再上一座別加摩就是上次我們說的.
是羅馬大道經過的一個地方.
是一個交通樞紐.
並拜羅馬皇帝的最主的廟宇.
再過來就是推也推拉.
推也推拉就內陸在山那裡.
地位不顯要 鄉下地方來的.
所以教會也不大 人數也不算很多.
但這封信對推也推拉教會是非常稱讚的.
稱讚他甚麼呢?.
有愛心 有信心 有勤勞 有忍耐.
如數家珍.
教會不在乎大不在乎小.
最重要是教會對上帝對人有愛心.
對神有信心 很勤力的傳福音工作.
對迫不得已的教會很忍耐.
最寶貴的是幕後所行的善事.

$^{81}$比起初所行的更多.
當時羅馬大道的迫不得已教會.
很多教會已經倒塌了.
很多人離開了.
但他在迫不得已之後做的善事比之前更多.
所以當耶穌稱讚的時候.
我們看到教會真的值得成為我們的模範.
雖然不是很大的教會.
但值得我們效法.
但書信說有件事要責備.
但我們看完整封信.
責備並不重要.
因為不是很多人追隨.
如果看完後面的信.
耶穌說有很多人不聽祂的話.
在紛亂的世界是持守神的道.
耶穌說不會責備那些人.
只要堅守基本的道就行了.
即是在教會裡面有一些人.
而且是重要的人.
他們教導的東西違背了聖經所說.
也不是耶穌的心願.
那是甚麼教導呢?.
就是耶洗別.
自稱先知的婦人耶洗別教導我的僕人.
引誘他們行奸淫吃製偶像之物.
先知的婦人 先知當然很重要.
有人問是不是教會領袖的太太師母.
來教導那些人行奸淫呢?.
那你問耶洗別是誰呢?.
是不是先知 是不是教會領袖呢?.
我們發現不是.
所以不要想過去.
耶洗別不是一個人.
也不是教會領袖的妻子.
所以不要猜了.
不是.
這不是教會的牧師長問題.
也不是太太張的問題.
耶洗別和我上一堂課說的.

$^{121}$可能你忘記了.
別加摩巴蘭尼哥拉.
也不是指一個人.
是指文化 是指世俗文化.
世俗潮流.
如果你要詳細看就要看17,18章.
就說到大淫婦.
巴比倫大淫婦.
大淫婦不是指一個女人.
招攬很多男人.
不是這個意思.
是指巴比倫是指城市.
有很多商業活動.
這些商業活動是叫人離開神.
追隨世俗.
追隨金錢.
追隨世俗的生活方式.
好言樂 母親情.
執書行徒 慘我敗家.
以利益為主等等.
這些商業活動.
不依靠神.
夜洗別就是叫他們離開上帝.
敬拜金錢 敬拜馬門.
敬拜金錢.
淫行的意思是心不對上帝.
對著金錢.
好像丈夫的心不對太太.
對外面的女人.
或者太太的心不再對丈夫.
對其他男人.
淫行是指這樣.
就是心不再敬拜上帝.
以其他東西取代上帝.
當時的太太教會.
都有很多商業活動.
例如世麻報出名的紫色布匹.
有錢人要買的.
都是很興旺的.
和這些有沒有關係呢.

$^{161}$可能有關係.
不過我們就要問.
為什麼用夜洗別.
為什麼說它像夜洗別.
不是巴蘭.
不是尼哥拉夜洗別呢.
我們要回到舊約.
記不記得.
舊約以色列阿哈王的太太.
叫做夜洗別.
夜洗別在以色列做什麼呢.
它顛覆了整個以色列的宗教.
就是推行巴力敬拜和阿瑟拉敬拜.
事實上以色列人.
之前已經敬拜了.
其實巴力拜了很久.
阿瑟拉也拜了很久.
但夜洗別就將它成為國家的宗教.
巴力和夜洗別.
為什麼會在以色列得到以色列的敬拜呢.
如果你記得我講過雙兩次安息.
我就提到.
當以色列人出埃及的時候.
當時在埃及哥山地所住的以色列人.
他們生活相當.
雖然是奴隸.
但生活不錯.
因為他們經常說在那裡很好吃.
又吃火鍋.
又有黃魚.
又肥牛.
去抗疫沒得吃.
摩西就說將來去到迦南地.
就是流乃與物之地.
流乃與物之地.
很好.
好世界.
誰知道進入迦南之後.
發現貨不對辦.
迦南地又山又谷.

$^{201}$乾燥的.
跟埃及沒得比.
就埋怨.
經常埋怨又回頭.
當他們進入迦南之後.
你看回薩姆爾.
連王記的歷史.
你會看到.
他們信不過上帝.
他們覺得上帝怎麼搞的.
他們很怕沒有魚.
於是他們就拜巴力臣.
因為巴力臣是負責降雨的.
拜亞瑟拉.
他們生畜需要生育眾多.
就拜亞瑟拉.
所以由來已久.
但耶駒別將這個成為國家宗教.
就跟你說你要繁榮.
你要財富.
就拜巴力.
如果你要家道興旺.
個個有錢就拜亞瑟拉.
整個以色列就取代了耶和華上帝.
整個心不敬拜神.
就敬拜巴力和亞瑟拉.
所以耶駒別提出財富福音.
你要財富就拜巴力.
上帝應許給你財富.
這是不足夠的.
財富福音很吸引人.
當今天有人說.
你信耶穌.
就可以得到財富.
一樣很吸引人.
我今天要說.
當日耶駒別在金曜裡面.
有一個教導在教會很流行.
就是財富的福音.
我們叫做什麼呢.

$^{241}$就叫做成功神學.
如果你有看電視.
星期日明珠台的崇拜.
你就會聽到成功神學.
很吸引人.
很多人聽.
你要很小心聽.
他說什麼.
很好的.
你依靠神.
信主.
你有信心.
神就會給你.
你有信心.
所以成功神學.
有一個俗稱.
叫做健康與財富的福音.
你求神.
有沒有說錯.
神會賜福給我們.
有恩典,平安,財富.
那有什麼問題呢.
沒有問題.
問題是主從的問題.
主人的問題.
誰是主.
成功神學.
上帝變成提款機.
我求上帝要多少.
就按多少出來.
那個決定權.
那個主權在我那裡.
不是在上帝那裡.
問題就在這裡.
不是我們求上帝求財富.
求健康.
不對.
不是這個問題.
上帝按心意給我們.
我們需要的.

$^{281}$神會照顧我們.
向神要就必得著.
Name it and claim it.
多麼的強大.
Name it 要什麼.
一百萬.
神就給你.
開心嗎.
如果你回教會.
牧師會這樣跟你說.
你喜歡聽嗎.
當然喜歡.
教會有很多人.
動輒幾十萬人.
很多人.
很危險.
並且.
當你再看歷史時.
你會發現.
發生很多問題.
很多人受傷害.
很多人求不了什麼.
離開教會.
神也沒有了.
成功神學的起源.
是一種美國文化.
來自美國.
很出名的哲學家.
Permetism.
實用主義的哲學家.
William James.
這個哲學家.
大家可能不熟悉.
我以前讀哲學.
讀過他的書.
我記得當時買了本書.
Permetism.
找馬掌後幫我去Sweden買.
那本書很貴.
四十多元.

$^{321}$那時大家都很窮.
那本書是傑作.
其中一本書叫.
Various Religious Experiences.
不同宗教經驗.
這本書說.
思想可以改變現實.
你想什麼.
外面環境會被你改變.
比量子力學還厲害.
這個哲學.
成為了美國的積極思想.
就是你.
凡事向好處想.
對你身體好.
對很多事都好.
本來積極思想.
都不是錯的.
凡事不要太陰暗.
我們應該對自己有自信心.
對子女都這樣說.
有自信心我們對事正面去看.
這個都不是問題.
問題是什麼呢.
當你積極思想的時候.
你會控制整個環境.
這就成為了積極思想在美國.
你會覺得他們很進取.
他們要去控制很多很多的東西.
所以這種運動叫積極思考.
和同類相吸.
積極思想會帶來積極的後果.
同類相吸就是.
你想好的東西就會來.
引用聖經說.
祈求就給你們.
是濫用聖經.
祈求神就會給你的.
因為神不說謊的.
那你求什麼呢.

$^{361}$最好就是財富.
健康.
大家都需要的.
所以你求就給你.
上帝憐憫.
上帝可以給我們.
但誰去命令.
問題就在這裡.
有一個神學家說.
這種思想有四個錯誤.
第一個就是.
耶穌救贖了我們.
有永生.
也包括在地上有財富.
擺脫物質的貧困.
就是說耶穌救贖.
一定會給我們健康和財富.
第二.
基督奉獻.
我們會從神裡得到.
加倍的補償.
耶穌和彼得說.
他們在地上所放棄的.
天上是千倍百倍還給你.
成功的時候說.
不是天上.
是現在.
所以你奉獻一元給教會.
神會給你十元.
一百元有萬元.
一千元有十萬元.
所以牧師是大富大貴的.
但奉獻的人是否真的個個都有呢.
又不見得.
算不算是騙人呢.
是濫用了聖經.
信心是出於自己的力量.
可以改變世界.
可以迫使神.
祈禱是可以迫使神.

$^{401}$施行成功的工具.
禱告可以搖動神的手.
所以這首歌我不唱了.
神的手被你搖動.
你做上帝.
你指大煞.
你指示神.
你做什麼就做什麼.
這些就是成功神.
我要舉一個例子.
聖經裡有個很出名的.
他們的師父.
叫方達拜恩.
這本書有桌子.
師母問我.
可不可以買這本書.
他說很貴的.
是在成品買的.
為什麼買這本書.
他寫了很多書.
這本書是特別神奇的.
很多成功神學的牧師.
用這本書作為.
理論的基礎.
他說.
真正的我就是.
健康富貴權位.
不好的不是真正的我.
上帝是要.
恢復你的真我.
這個也不重要.
最重要的是.
你的思想.
是可以令環境改變.
是根據.
William James.
即是詹姆士的靈性哲學.
是將它深化.
擴大了.
你的思想改變了.

$^{441}$根據他的說法.
祈禱上帝一定答應.
並且已經答應.
一定答應.
有一個女士.
39歲.
都很成功.
工作也不錯.
她就問.
Brian.
她說.
我很想結婚.
很想有個丈夫.
她就問.
有沒有祈禱.
她說有啊.
但現在還沒有丈夫.
她說有問題.
你應該相信上帝.
已經給了你一個丈夫.
她說.
我現在沒有.
她說不是.
你錯了.
你應該相信.
上帝已經給了你.
她說.
你現在睡的是單人床.
還是雙人床.
她說沒有一個人.
睡單人床.
她說錯了.
你丈夫睡地下.
她說應該是丈夫已經給了你.
所以你要換張雙人床.
King size.
讓你丈夫可以睡.
她說沒有.
你錯了.
所以你要買雙人床.

$^{481}$於是她就換了雙人床.
她說你睡一邊.
你丈夫睡旁邊.
你一邊給你丈夫睡.
她問你衣櫃有沒有空位.
掛西裝.
她說沒有.
你知道女人有很多衣服.
錯了.
你應該有個空的衣櫃.
讓你丈夫可以掛西裝.
她說是啊.
因為上帝已經給了你一個丈夫.
於是她就買了一個空的衣櫃.
讓她丈夫可以掛西裝.
她問你廁所有多少支牙刷.
有多少個杯.
她說只有一個.
錯了.
應該有兩支牙刷.
兩個杯.
凡是雙份.
因為你丈夫早上要刷牙.
毛巾有兩條.
她說照做.
表示你相信上帝已經給了你一個丈夫.
你就要這樣做.
於是就做了.
這本書不是我作的.
三個月後.
她真的有一個丈夫.
躺在她旁邊.
真的有一個丈夫.
衣服掛在衣櫃.
她不用交代怎麼來的.
她只是說一件事.
她這樣做之後.
過了兩個星期.
有一晚她在餐廳裡.
她見到一個男人.

$^{521}$吃飯.
這樣猜測.
這應該是上帝賜給她的丈夫.
於是她走過去和他聊天.
事就如此成了.
後來就做了她的丈夫.
我不知道你們是否這樣找老婆.
或者這樣找丈夫.
結婚.
是兩個人.
很深入.
彼此了解的.
不是就這樣.
這本書叫.
思維可以使你.
事實.
現實.
因為這個緣故.
那些成空神學家.
按照這樣.
不斷宣傳.
歐斯汀就是這樣.
不斷宣傳.
只要相信上帝.
信是給你的好的.
例如你要驗身.
驗身不好就不要信這些.
你要信身體好.
我驗身有三高.
這是錯的.
你要信.
樣樣都好.
這種不知什麼信心.
如果你的兒子不聽話.
你要信他很乖.
就算他現在吸毒.
因為上帝給你的兒子是好的.
不面對現實.
一直這樣講.
成空神學.

$^{561}$我舉三個例子.
結束我今天要講.
實例.
這三個例子.
美國水晶大教堂.
蕭律白牧師.
韓國羽島順豐教會.
趙容基牧師.
新加坡城市豐收教會.
康熙牧師.
這三個不是壞人.
他們不是壞人.
是好人.
都是好用心目會.
但被成空神學害了.
首先.
水晶大教堂蕭律白牧師.
如果你以前喜歡看.
星期日.
明珠台的崇拜.
一定看到他.
因為他很多時候播放他的權能時間.
還沒播放他的講道.
不知道你有沒有聽.
他是一個很典型的成空神學.
他一直講.
思想可以改變環境.
可以改變你.
他說你幻想自己自然.
瀟灑有禮.
變成帥哥.
我沒什麼信心.
他說如果你假定自己是高招成功.
信箭出人頭地的.
就一定會發達.
他經常這樣講.
我是引他講的.
你相信.
你想像神會給你.
你要信神已經給了你.

$^{601}$他呢.
其實是來自William James.
就是心靈控制物質.
叫My all the matter.
My all the matter.
思維可以改變.
思維可以改變這個世界.
他當時教會發展.
他相信神會叫他更成功.
成功更成功.
水晶教會已經很靈驗.
再發展.
不斷地去借題.
不斷地叫教會更大更靈驗.
請更多人做更多的事.
一路無限的擴展.
一定成功.
神已經給了你.
當他一路發展的時候.
他相信.
他那班長老相信.
他那班執事都相信.
不過有一個人不相信.
銀行不相信.
2010年.
銀行突然跟他說.
資不抵債.
立即清盤.
當時會友完全不知道.
回家說雞九沒了.
就拍賣了那間教會.
資不抵債.
他說上帝給了我們.
是的 上帝給了你.
給了你什麼呢.
2010年10月18日.
結果破產.
整個教會.
找人接手.
天主教買了.

$^{641}$叫基督堂.
還在那裡.
LA叫基督堂.
Christ Church.
天主教就買了.
天主教都很厚道.
他在教會旁邊.
有個牌.
教會就是.
蕭律帕牧師建立的.
都給了他一點點的信心.
你知不知道.
教會之後.
那些人都散了.
信心全部失去.
教會會這樣.
會倒閉的.
會被人拍賣的.
你回到大埔堂.
一天都不能進去.
打個交叉.
教會賣了.
你會不會很失望.
你教什麼.
信心堅穩的.
就找其他教會.
信心不堅穩的.
去哪裡呢 去喝茶.
就是這樣.
多慘啊.
第二個.
趙鎔基.
他在港台上寫了一本書.
成功神學.
什麼叫成功神學.
他寫的書是神一定祝福的.
他在網上有的.
免費下載的.
其實很成功.
他教會全世界最大.

$^{681}$86萬人.
成功有很多因素.
他教會很傳福音.
還有他教會做很多社區關懷的工作.
照顧窮人.
又差派宣教士.
他教會的組織很好.
他是好牧師.
為什麼好牧師呢.
教會試過經濟危機.
他真的把自己所有的東西賣掉.
來幫助教會.
連他太太最後的珠鏈.
結婚時候的珠鏈.
都要賣掉.
是一個好牧師.
不過他相信成功神學.
他相信神一定會給他的.
神到最後.
都是叫他得到成功.
所以他不怕.
神會百倍千倍還給他.
他就這樣做.
因為這個緣故.
他後來犯了一個很大的錯誤.
他說上帝祝福我.
我成功.
我子女都成功.
他的兒子在2000年的時候.
是做清潔生意的.
很大上市公司.
那時候經濟不好.
股票都不夠.
他就說.
現在經濟不好.
股票就跌了.
甚至有經濟危機.
他就和執事說.
這是魔鬼的攻擊.
是幻覺.

$^{721}$上帝祝福我一定成功.
所以我們一定要幫他.
經歷這個困難.
就好像以前.
他賣掉東西幫助教會.
神一定會補償他.
他現在要幫他兒子.
怎樣幫呢?.
建銅基金.
七千多萬港幣.
幫他兒子買股票.
買完之後股票就會上升.
你就會拿回所有錢.
結果.
七千多萬投進去.
你知道.
買股票的東西很難說.
全都倒了.
不單止沒有了.
還一錢不值.
全都沒有了.
他那時候.
他退休.
教會給了他很多退休金.
他幫助他兒子.
都用了.
所以他沒有報稅.
結果賣掉.
他挪用教會的公款.
教會新的長老說.
不行一定要處理.
於是就告他.
證據確鑿.
判了他要坐三年監.
但法官都很好.
他是節目師.
他不想在老人家坐監.
於是就說.
我給你緩刑.
罪已經是罪成.

$^{761}$罪是沒有問題的.
真的判了案.
你犯了罪.
要坐監.
不過緩刑.
三年之後不用你坐.
給他面子.
之後他就說.
成功神學.
連累了他.
給了他一個錯誤的觀念.
第三個.
我覺得是最可憐的.
康熙牧師.
我不知道你認不認識.
新加坡很有名.
很年輕.
大學畢業和太太.
1989年創立了豐收教會.
兩萬多會友.
他講得很生動.
但他是名人派.
也是一路成功神學.
你要相信.
神有給你.
很多人聽.
我怎樣欣賞他呢.
因為他和太太何耀山說.
外面有很多年輕人.
未信耶穌.
如果我們要帶他們信耶穌.
就不能叫他們來聽我們的歌.
我們施班走得很好.
但你叫外面的街頭人來聽.
他們不喜歡聽.
要唱他們的歌.
Rock Music.
要將呼音化.
要帶他們信主.
要他們聽的歌.

$^{801}$何耀山就要成為一個很出名的歌星.
Rock Music的一個Star.
那些小朋友會跟他.
跟他唱.
這個project叫什麼呢.
叫Crossover Project.
何耀山三十多歲.
兩個小朋友.
願意變成一個Superstar的明星.
我覺得他很偉大.
你覺得他偉不偉大.
我師母一定不肯.
他走了.
我看他唱歌.
我聽他唱歌.
我看他跳舞.
我覺得他很偉大.
為了呼音緣故.
放下身段.
Rock Music很多錢.
要找導演來幫他拍戲.
要拍攝形象.
要住在荷里活.
要很出名.
他用了七千萬美金.
住在最豪華的地方.
錢從哪裡來.
教會一個新的建堂基金.
他怎樣跟那些執事說.
這個是福音.
神一定叫你成功.
他太太一定會成為Superstar.
將來他的MV一出.
一定會很多人去聽.
很多訂閱者一定會讚的.
他是不是梅艷芳.
不是梅艷芳.
很少梅艷芳.
結果他的銷路不行.
吸引不到人.

$^{841}$欠了一大筆債.
後來教會一翻賬.
不得了.
那些賬目這樣來用.
為了呼音緣故.
但是不正當的手段.
一個欺騙手段.
不誠實的虛偽建成.
於是最後新加坡法院判了他坐八年監.
他已經坐完監了.
他沒有坐八年.
他坐了五年多.
他出來的時候.
他就跟人說.
他成功神學誤導了他.
他知錯.
因為他過去一直相信.
一定成功.
神一定給你.
你思維可以改變.
你一定可以.
誰是上帝.
你是上帝.
天上的上帝.
所以.
在今天我們求神的福氣.
要知道.
上帝會按照最好的給我們.
不是真的給你一筆錢.
或者給你即時健康.
神可以.
但神給更好的東西.
是你想不到的.
在你的生命上所經歷的.
有時困苦都會成為一個祝福.
所以在我們人生上.
我們最重要的是聆聽神.
與神同行.
艱難的日子.
誰沒有呢?.

$^{881}$我自己都經歷過艱難的日子.
在我年輕的時候.
我都經歷過艱難的日子.
我讀大二年班的時候.
當時我妹妹有個病.
很疼我妹妹.
她的脊髓做不到血.
當時香港沒有得醫.
唯一要去美國.
根本沒有可能.
如花似玉十八歲.
就離開這個世界.
當時很難過.
不過在背後.
當時除了她這個病.
我媽媽心臟病.
經常入醫院.
我爸爸胃癌.
所以我親戚說.
你信耶穌做成這樣.
但神有祝福.
我父親因為我妹妹去世.
後來他信了主.
為什麼呢?.
我妹妹去世的時候.
他知道自己得救.
他很平安.
他在面前很清楚知道.
他在天上會等我們.
我父親見到妹妹很平安.
成為一個人.
無恐懼.
一個這麼年輕的少女.
他自己心裡怎樣想我不知道.
不過知道.
他後來因為這件事.
改變了他.
他本來很硬心.
不肯信主.
他又看見教會很有愛心.

$^{921}$教會的人無怨無悔.
很有愛心.
第一次見到他很感動.
所以那年的聖誕節.
他接了洗禮.
很奇妙.
人生的困難.
在當日好像很困難.
但神的帶領很奇妙.
人的一生最主要是與神同行.
順境逆境有神的同在.
不是即時的富貴.
也不是即時的健康.
神會給我們.
按祂最好的給我們.
而最好的是耶穌基督給我們的永生.
我們無懼這個世界.
因為永恆的生命已經有.
還有什麼比這個更大.
其他的東西.
神按祂的心意.
按祂的計劃.
不是按我的心意.
上帝不是提款機.
不要把上帝當提款機.
要求必然上帝不是王大仙.
耶穌不是菩薩.
祂是我們的主.
所以成功神學的問題.
就是把上帝當作提款機.
把耶穌當成菩薩.
問題就在這裡.
不是聖經有問題.
是那個思想的教導有問題.
這個平約室現在正在說.
經常在權衡時間裡說.
你聽聽他所說.
你可以掌握你的生命.
上帝呼召你成功.
享受健康.

$^{961}$並且得勝.
上帝指引你.
不是要你失敗.
貧窮一事無成.
好不好聽啊?.
當然好聽啊.
很多人聽啊.
Hallelujah! Hallelujah!.
人生不是這麼簡單.
上帝帶領我們人生很精彩.
很多事發生.
上帝說要使病毒在場.
上帝要打擊這種思想.
還有那些追隨者.
要清除他們.
不過推也推拉的教會.
不是很多人追隨他們.
所以上帝說.
不會把其他的擔子放在你身上.
那些持守真道的.
上帝會祝福他們.
幫助他們.
並且給他們權柄.
所以今天回歸聖經.
回歸神教.
與主同行.
無論順境還是逆境.
認定神的同在.
無怨無悔.
繼續前行.
有主的同在.
一起起來.
我們同心祈禱.
因為主你自己.
已經將最好的給我們.
就是我們的救贖.
主你給我們重振智慧.
彼此相愛.
讓我們教會遵守你的話語.
回歸聖經.

$^{1001}$讓我們教會在大埔這個地區.
成為你的見證.
服侍社群.
讓更多人得到你的福氣.
你的福氣是源遠流長.
並且是細水長流.
我們知道你的福氣.
是滿日幫助我們.
無論教會.
無論我們的家庭.
都蒙受你的恩典.
帶領我們.
求你祝福教會未來的發展.
發展是為了你的命.
為了你的福音的緣故.
按照你的計劃.
我們一步一步前行.
求你繼續帶領.
我們祈禱奉主名下.
尋心所願.
\newpage



\chapter{蕭壽華}\label{ch:preacher4}
\begin{multicols}{3}
\minitoc
\end{multicols}
{ \scriptsize


\begin{xltabular}{\textwidth}{|p{0.15\textwidth} p{0.6\textwidth}|p{0.07\textwidth} p{0.1\textwidth}|}
\hline
希伯來書 3:12-4:13 & \hyperref[sec:toZa1ewaUWE]{以信心與所聽見的道配合(希伯來書3\_12-4\_13) - 蕭壽華牧師} & 2025-01-08 & \href{https://youtube.com/watch?v=toZa1ewaUWE}{\texttt{ toZa1ewaUWE}} \\
希伯來書 5:11-14 & \hyperref[sec:8LlYAk0Xlok]{嘗過天恩 卻長蒺藜 (希伯來書5\_11-14;6\_1-12) - 蕭壽華牧師} & 2025-01-04 & \href{https://youtube.com/watch?v=8LlYAk0Xlok}{\texttt{ 8LlYAk0Xlok}} \\
\hline
\end{xltabular}
}
\newpage



\section{希伯來書 3:12-4:13}
\label{sec:toZa1ewaUWE}
\textbf{以信心與所聽見的道配合(希伯來書3\_12-4\_13) - 蕭壽華牧師}
\newline
\newline
連結: \href{https://youtube.com/watch?v=toZa1ewaUWE}{\texttt{ https://youtube.com/watch?v=toZa1ewaUWE}} ~~~~ 語音日期: 2025-01-08 
\newline
\newline
\hyperref[sec:fJrsPMmDHtU]{< < < PREV SERMON < < <}
~
\hyperlink{toc}{[返主目錄]}
~
\hyperref[ch:preacher4]{[返講員目錄]}
~
\hyperref[sec:8LlYAk0Xlok]{> > > NEXT SERMON > > >}
\newline
\newline
希伯來書 3:12-4:13
\newline
\begin{longtable}{cl}
\hline
\hline
章節 & 經文 (和合本修訂版)\\
\hline
3:12 & \begin{tabularx}{0.7\textwidth}{X} 弟兄們,你們要謹慎,免得你們中間有人存著邪惡不信的心,離棄了永生的神。 \end{tabularx} \\ \\ \relax
3:13 & \begin{tabularx}{0.7\textwidth}{X} 總要趁著還有今日,天天彼此相勸,免得你們中間有人被罪迷惑,心腸剛硬了。 \end{tabularx} \\ \\ \relax
3:14 & \begin{tabularx}{0.7\textwidth}{X} 只要我們將起初確實的信心堅持到底,就在基督裡有份了。 \end{tabularx} \\ \\ \relax
3:15 & \begin{tabularx}{0.7\textwidth}{X} 經上說:「今日,你們若聽他的話,就不可硬著心,像在背叛之時。」 \end{tabularx} \\ \\ \relax
3:16 & \begin{tabularx}{0.7\textwidth}{X} 聽見他而又背叛他的是誰呢?豈不是跟著摩西從埃及出來的眾人嗎? \end{tabularx} \\ \\ \relax
3:17 & \begin{tabularx}{0.7\textwidth}{X} 神向誰發怒四十年之久呢?豈不是那些犯罪而陳屍在曠野的人嗎? \end{tabularx} \\ \\ \relax
3:18 & \begin{tabularx}{0.7\textwidth}{X} 他向誰起誓,不容他們進入他的安息呢?豈不是向那些不信從的人嗎? \end{tabularx} \\ \\ \relax
3:19 & \begin{tabularx}{0.7\textwidth}{X} 這樣看來,他們不能進入安息是因為不信的緣故了。 \end{tabularx} \\ \\ \relax
4:1 & \begin{tabularx}{0.7\textwidth}{X} 所以,既然進入他安息的應許依舊存在,我們就該存畏懼的心,免得我們 中間有人似乎沒有得到安息。 \end{tabularx} \\ \\ \relax
4:2 & \begin{tabularx}{0.7\textwidth}{X} 因為的確有福音傳給我們像傳給他們一樣;只是所聽見的道對他們無益,因為他們沒有以信心與所聽見的道配合。 \end{tabularx} \\ \\ \relax
4:3 & \begin{tabularx}{0.7\textwidth}{X} 但我們已經信的人進入安息,正如神所說:「我在怒中起誓:他們斷不可進入我的安息!」其實造物之工,從創世以來已經完成了。 \end{tabularx} \\ \\ \relax
4:4 & \begin{tabularx}{0.7\textwidth}{X} 論到第七日,有一處說:「到第七日,神就歇了他一切工作。」 \end{tabularx} \\ \\ \relax
4:5 & \begin{tabularx}{0.7\textwidth}{X} 又有一處說:「他們斷不可進入我的安息!」 \end{tabularx} \\ \\ \relax
4:6 & \begin{tabularx}{0.7\textwidth}{X} 既有這安息保留著讓一些人進入,那些先前聽見福音的人,因不信從而不得進去, \end{tabularx} \\ \\ \relax
4:7 & \begin{tabularx}{0.7\textwidth}{X} 所以神多年後藉著大衛的書,又定了一天—「今日」,如以上所引的說:「今日,你們若聽他的話,就不可硬著心。」 \end{tabularx} \\ \\ \relax
4:8 & \begin{tabularx}{0.7\textwidth}{X} 若是約書亞已使他們享了安息,後來神就不會再提別的日子了。 \end{tabularx} \\ \\ \relax
4:9 & \begin{tabularx}{0.7\textwidth}{X} 這樣看來,另有一安息日的安息為神的子民保留著。 \end{tabularx} \\ \\ \relax
4:10 & \begin{tabularx}{0.7\textwidth}{X} 因為那些進入安息的,也是歇了自己的工作,正如神歇了他的工作一樣。 \end{tabularx} \\ \\ \relax
4:11 & \begin{tabularx}{0.7\textwidth}{X} 所以,我們務必竭力進入那安息,免得有人學了不順從而跌倒了。 \end{tabularx} \\ \\ \relax
4:12 & \begin{tabularx}{0.7\textwidth}{X} 神的道是活潑的,是有功效的,比一切兩刃的劍更鋒利,甚至魂與靈、骨節與骨髓,都能刺入、剖開,連心中的思念和主意都能辨明。 \end{tabularx} \\ \\ \relax
4:13 & \begin{tabularx}{0.7\textwidth}{X} 被造的,沒有一樣在他面前不是顯露的;萬物在他眼前都是赤露敞開的,我們必須向他交賬。 \end{tabularx} \\ \\
[1ex]
\hline
\hline
\end{longtable}
$^{1}$此錄音或錄影的版權是屬基督教宣導會不確同所有.
我們的屬靈傳統之一是以真理教導為本.
歡迎您將此錄音或錄影的連結傳給親友.
懇請注意未經授權請勿轉錄或剪輯本錄音或錄影.
各位聽眾平安.
剛才我在禮堂後面等待進堂時.
比較留心看周圍的弟兄姊妹.
看到有很多熟悉的面孔.
有些不太熟悉但知道都是習慣坐那個位置的.
在看的時候我深感一種很特別的感受.
所以當你每次來敬拜的時候.
你知道你身邊的是你的家人.
就算你不認識在靈裡面我們都是相通的.
所以我都想大家在今天一齊我們施上聖祭之前.
你和你身邊的家人打個招呼.
好弟兄好姊妹一齊敬拜.
我們一齊再祈禱.
讓群體中間一切的間隔.
一切的牆壁被你拆除.
在耶穌基督的和平裡面.
成就了人之間的和平.
祝我們今天到你的面前.
承認我們一切所能得到的.
都是出於你的恩惠.
都是你的作為.
一直到我們繼續跟隨你的時候.
求主保守我們的心.
不會走回舊路.
讓我們知道每一天依靠你的恩惠.
你都走成聖的路.
主啊帶領我們.
在我們跟隨你的路上.
有些時候會面對衝擊.
但是求主兼顧我們的信心.
讓我們憑著聖靈的幫助.
我們一生堅定信靠你跟隨你.
如今我們打開你自己的話.
一齊思想求親愛的聖靈.
指教我們引導我們的心思.
我們仰望祈禱奉主耶穌成名求.

$^{41}$阿們.
我們今天繼續思想聖經希伯來書第三章第四章的經文.
提到以信心與所聽見的道配合.
經文裡面提到有不少的人.
因為不能夠繼續信主.
所以離棄主.
我很快想到有過去見到一些曾經是基督徒的朋友.
問起他們的情況時.
都感到很詫異.
他們因為某些原因放棄信仰.
一位信主多年的姐妹.
她與弟弟關係密切.
大家很有親情.
後來弟弟患了癌症.
因此為姐妹迫切為弟弟祈禱.
呼求神來醫治她.
但一段時間後弟弟過世.
姐妹很失望.
她開始對人說.
我不能再相信這位神.
祂很冷漠,很殘忍.
我不能再相信.
因此她離開了上帝.
另一位少年的基督徒.
初移民到英國.
到英國時很想有朋友.
因為香港的舊朋友已經分隔開了.
但在學校,在家中都找不到可以聊天的對象.
後來出現一個朋友.
很願意與他聊天.
原來發現他是一個穆斯林.
過了一段時間這位基督徒.
與自己的父母說.
現在我不能再相信耶穌了.
因為我的好朋友穆斯林.
我也是穆斯林了.
他便繼續跟隨這位朋友.
去了當地的很多清真寺.
很多時候你會發覺.
人生出現一些事情衝擊我們的信心.

$^{81}$甚至我們到了一個地步.
會想到不如放棄了.
對信仰放棄了.
但你會發覺這件事情.
不單是因為環境或人際關係所產生的衝擊.
也明顯有惡者的攻擊.
如果你會記得剛剛過了壽苦節的時候.
我特別留意到聖經裡主耶穌說的話.
當時在馬利沿前.
主耶穌在如節晚餐時.
與彼得說.
與其他門徒說.
「西門西門,撒旦想要得著你們」.
撒旦想要得著你們.
要篩你們就像篩墨紙一樣.
大家知道篩.
在當時的農業社會裡.
他們用一些像收機的器具.
將墨紙放上去.
然後大力搖晃.
甚至撞擊它.
讓康丕和墨紙可以分開出來.
在過程中墨紙遭受很大的衝擊.
主耶穌說撒旦要篩你們.
就像篩墨紙一樣.
之後你會看到.
很多門徒有很多軟弱跌倒的地方.
你會發覺.
惡者會用很多不同的方式.
用各種苦難迷惑.
要去試探你.
要去引誘你.
叫你放棄你的信仰.
或者說他的攻擊最重要目的.
是要你攻擊你的信心.
叫你對神的信心就算沒有放棄.
也變得很冷淡.
算了,不用這麼認真了.
而你會發覺背後.
如果你不是警醒面對.

$^{121}$你不是有力量來抗拒這些衝擊.
你會發覺聖經裡.
很多地方提醒我們警覺著.
也提出很多在舊約經文裡.
有很多不同的以色列人.
特別在抗疫的時候.
他們埋怨上帝.
他們背棄上帝.
他們不信上帝.
但是曾經去到哥林多前書.
其中一節經文提到這些事的時候.
就特別解釋背後的原因是什麼.
背後的原因就是要警戒我們.
在這些末世裡的人.
免得我們會跌落同樣的試探.
哥林多前書第十章前面我們看到.
然後聖經就說.
所以自以為站得穩的人必須謹慎.
免得跌倒.
要謹慎.
不要以為順住幾十年了.
應該沒什麼事的.
但是仍然要謹慎.
免得跌倒.
但是聖經給我們一個很大的安慰.
經文下面說.
大家會留意到.
你們所受的考驗.
正如一個字眼描述這些的衝擊.
考驗無非是人所承受得了的.
甚至信實的.
它不會讓你們遭受無法承受的考驗.
意味著你所承受的衝擊.
是你能夠承受得到的.
在受考驗的時候.
總會給你開一條出路.
讓你們能忍受得了.
你會遲或早經歷不同的考驗.
但是在你面對考驗的時候.
你要記得.

$^{161}$這些考驗不是過於你能夠承受的.
你在當中繼續堅持你的信心.
繼續放心去忍耐面對.
這些考驗會成為你的生命的祝福.
考驗帶來的.
是更加成熟.
更加堅固的信心生命.
當醒一路提醒我們警覺的時候.
你知道神背後一切的目的.
是要你更清晰的.
更知道這位是上帝.
自然剛從我們所唱的詩歌.
在心魄的認識這位是偉大的上帝.
是可信可靠的上帝.
以致你有堅定的信心來跟隨他.
我們再讀到希伯來書的時候.
你會發覺希伯來書剛才我們讀的三章十二節開始.
一段經文特別提醒我們要小心.
不要讓我們的信心變得剛硬.
你有沒有想過信心有時候會變得剛硬.
聖經三章十二節說.
弟兄們你們要小心.
免得你們中間有人懷著.
邪惡不信的心.
以致離棄永活的神.
由聖經講到這種信心是一種邪惡的不信.
信心當離棄不再相信的時候.
聖經說是邪惡的事情.
當你讀聖經的時候.
如果大家都好請大家好不好.
都返回聖經.
你會發覺聖經希伯來書裡面的上下文.
當你讀的時候.
你會更加體會到那個警告是多麼的強烈.
警告是多麼的嚴肅.
多麼的重要地提出.
有時候我們都習慣了.
我們看外面投影的經文.
但我都鼓勵大家.
有時候可以拿回聖經出來.

$^{201}$你會更清楚看到經文的上下文.
你會更清楚知道經文的意思.
希伯來書第三章.
在第十二節之前.
你會看到聖經在描述.
以色列人在抗野怎樣去違背上帝的情況.
講完之後就提到你們要小心.
免得你們中間懷著邪惡不信的心.
這裡是描述以前的人的情況.
但同時會出現在我們今天.
聖經就提到當時的這些以色列人.
他們去到迦南地之前.
他們會派探子.
摩西派十字的探子去試探當地的情況.
大家記得嗎?.
其中十個探子回來的時候.
就向以色列人報惡信.
這個地方很凶險.
我們進不去的.
我們只會受死.
當時以色列人就很懷疑神的應許.
不相信神的保護.
不相信神的話是真實的.
他們就開始產生那種對神的不信.
開始埋怨上帝.
為什麼你帶我們出來.
要我們死在迦南地.
然後埋怨我們寧願死在埃及地.
不需要這樣走出來.
死在抗日的地方.
在這個時候.
聖經在當時的民數記十四章.
在外面我們看到.
「對摩西說:這百姓藐視我要到幾時呢?」.
神看人的不信是一種對他的藐視.
不認為他能夠保護我.
不認為他是按照他的應許會成全.
眼前的形勢這麼困難.
明顯神不會帶我們進入迦南地.
因此在這個不信過程當中.

$^{241}$產生對神的一種藐視.
這個藐視引發神的怒氣.
引發神公義的審判.
但回看剛才的經文的時候.
在第三章大家讀的經文的十二到十四節的時候.
聖經提到一個防範的機制.
這個防範的機制在十三節這樣說.
「你們親著還有稱為今日的時候.
天天彼此勸言.
免得你們中間有人被罪惡迷惑.
而變得心裡剛硬」.
聖經提到一個危機就是.
當你面對試探的時候.
其實是被罪迷惑.
被罪迷惑不去抗拒的時候.
因此產生的後果就會心裡剛硬.
各位觀眾你會留意到.
聖經常常講述一件事情.
就是罪有一種迷惑性.
以撒旦.
聖經說他是說方人之父.
他是在很多方話裡迷惑人.
而聖經裡在啟示六十二章提到.
撒旦的工作是迷惑普天下的.
迷惑這個字相關的字句.
在聖經的新舊約出現超過一百五十次.
常常提到惡者如何透過不同的迷惑方式.
要神的指紋跌倒.
迷惑如果再直接說.
就是惡者要透過一種假象.
一種謊言.
叫你相信你這樣做的時候.
你就得到這樣的利益.
你一定記得在伊甸園的時候.
魔鬼如何迷惑夏娃.
你放心吃吧.
你知不知道你吃完禁果之後.
你的眼睛就明亮.
你好像神一樣.
能夠知道善惡.

$^{281}$所以一個假象.
以為你做某件事情.
就得到某天你想得到的事情.
或者另一方面.
當你要做一件事情.
會得到一筆金錢的時候.
你知道其實都不是很應該做的.
但可能會有些假象出現.
你放心.
你得到這筆錢.
你就快樂無憂了.
你得到這筆錢.
很多人就會尊重你.
不會再輕視你了.
有時候那種迷惑.
透過一種你想不到的方式.
在你心思裡面出現.
或者是可以將惡念.
放在你的心裡面.
有時候突然間一件事情.
有一些試探.
有些人對你有些特別的要求.
你在掙扎的時候.
突然間有個意念.
這個都是你控制不了的.
順其自然吧.
做吧.
甚至你可能突然間想起這個廣告.
一個大的Tick在你面前.
Just do it.
突然間好像不顧一切.
我都是控制不了的.
突然間好像自己不肯再作主.
去拒絕這件事.
認為自己不能夠控制了.
就順其自然吧.
這頂芝麻在你人生的路上.
你會面對著迷惑.
而迷惑會帶來了剛硬的心.
是叫你不順的.

$^{321}$因此你會明白.
為什麼主耶穌在赫西瑪利園.
當祂自己面對神事間的時候.
祂自己都要小心的祈禱.
祂也要吩咐他的門徒.
要警醒禱告.
免得入了迷惑.
能夠勝過迷惑的是警醒禱告.
在禱告當中.
你的心靈比較清醒.
你會分辨是非.
也有力量.
不是你的力量.
是求神所賜的力量.
可以去勝過迷惑.
有些弟兄告訴我.
我常常祈禱.
我問你怎樣常常祈禱.
我有感動我就默禱.
默禱兩句繼續做事.
好 為他感恩.
但你知道當主耶穌面對衝擊的時候.
他也要跪下長時間的祈禱.
何況你.
你不要輕易兩三句祈禱.
這樣就有力量.
你需要幫助自己.
是真的讓你的意志.
你的心思能夠集中.
你可以全心全意向尋求恩典.
在集中的過程當中.
你的心靈能夠清晰.
能夠分辨迷惑.
你才能夠勝過一切的邪惡.
很多的教徒.
有些時候會有信心軟弱的時候.
但總不會一時間出現.
是一個過程慢慢疏忽.
放鬆放鬆.
而跌入陷阱當中.

$^{361}$所以請你不要留心.
免得你們中間有人被罪惡迷惑.
而變得心裡剛牙.
但當你讀前面的時候.
你會留意到聖經特別提及.
「趁著還有清為今日的時候.
天天彼此勸勉」.
聖經提及今天.
也就是說在救恩的時期.
你仍然可以求赦免的機會.
仍然有的時候.
你要把握這個機會.
你可以迴轉去尋求神的赦罪.
但過程中聖經特別提及.
「天天彼此勸勉」.
立場你看到.
聖經很清楚提及.
能夠勝過迷惑的一個關鍵.
是因為你有弟兄姊妹.
我們常常說.
團契對基督的生命很重要.
這不是一個口號.
是真正屬靈生命的需要.
你會發覺聖經提及.
「天天彼此勸勉」.
未必你能夠和你的弟兄姊妹天天見面.
但肯定是常常聯繫著.
在聯繫過程中.
當然有很多快樂,歡聚的時間.
很多不同的活動.
但當中必須包含一種彼此的勸勉.
「勸勉」這個字在聖經源頭的意思.
包含了一個將軍.
面對一隊軍人.
在出征之前.
將軍跟他們很清楚地說.
去勉勵他們放膽,剛強面對挑戰.
去幫助他們謹慎面對惡者.
面對敵軍的很多惡勢力.
這叫「勸勉」.

$^{401}$在弟兄姊妹的相處當中.
尋求一種誠實的彼此的勸勉.
我們就能夠勝過彼此的迷惑.
而能夠堅定地信教實.
在過程中你會發覺.
神恩使幫助我們.
在第十四節說.
因為我們若真的持守起初確實的信心.
沒錯,當初很清楚知道.
這位是真神.
他所持的真道是必然成全的.
但有些時候突然間慢慢鬆弛了.
灰心了.
因此放棄了.
但聖話說要持守著確實的信心.
就在基督裡有份.
這說的是永遠真實的關係.
我想起一件事.
在我教會在職服侍的時候.
有一段日子比較低沉.
在當時面對一些攻擊.
亦留意到其他在教會界裡面.
有一些人物的行為.
我感覺很詭詐的行為.
我很難接受這些領袖竟然是這樣.
在那段時間我發覺自己裡面也有些灰心.
有些想法覺得.
你怎樣努力都有人批評你.
你做得多好都有人批評.
在那時候我裡面產生一種灰心.
好像沒有了一種動力繼續去服侍.
在那段時間自己祈禱的時候.
有一次想起神一主的話.
神對以利亞先知所說.
當時以利亞不斷批評.
在以利亞當中很多人背棄上帝.
但神就對以利亞說.
我留下七千人未曾向巴黎屈膝.
這句說話一直鼓勵我的心.
神是信實的.

$^{441}$祂保守著祂的子民.
也讓很多不同的子民.
仍然堅定地侍奉祂跟隨祂.
因此我也要堅守我的良善.
無論見到什麼壞榜樣.
仍然堅定地信靠神.
在那時候你發覺.
在禱告裡面.
神會用不同的方法去兼顧你的信心.
以至你可以繼續走應走的路.
繼續堅持你的良善跟隨著.
我們再讀聖經的時候.
《下帝經文》裡面提及一個安息的問題.
有些人不能進入永恆神給我們的安息裡面.
是為什麼呢?.
在經文第四章的時候.
你會留意到中間已經提及到.
有很多關於以色列人在舊約的時候所經歷的事情.
來到第四章第三節的時候.
聖經就提及到我們這些是信了主的人.
是正進入安息.
但相對來說.
有些人神要對他們起誓說.
他們絕不可進入我的安息.
這個安息特別是強調屬於神的安息.
是神的安息.
但他們不能夠再有機會進入神給他們的安息.
然後聖經就提及到關於創世記裡面提及到的.
神創造天地提到神所賜的安息日.
當我整經講到這些事情的時候.
你會明白聖經提到神創造天地第七日安息.
然後他賜下安息日給他的子民.
作為在舊約裡面神與人納約的一個記號.
是神將安息賜給我們.
然後跟著聖經又提到.
當以色列人進入加南地的時候.
神允許他們進入安息地.
進入應許之地.
領受神給他們的安息.
但聖經又提及到.

$^{481}$其實他們未曾得到安息.
在四章第八節你會見到.
當時約書亞帶領他們進入加南地.
但其實他們未曾得到安息.
聖經在後來會提及「另有一日」.
所以你會留意到聖經無論是提及神賜的安息日.
或是神賜給加南地的應許的安息.
這些不是終極的安息.
只不過是聖經所提及的類比.
這種的typology.
是指向耶穌基督終極要給我們真正的安息.
人在其中未能得到真正生命永恆的安息.
而唯有耶穌基督的救恩帶給我們真正永恆的安息.
所以當你再讀的時候.
你要發覺聖經裡面提及「另有一日」.
然後再讀下去的時候.
第九至十一節就提到.
神有一個安息日的安息.
留給神的子民.
這個時候.
神的子民就揭了自己的弓.
好像神創造天地之後揭了他的弓一樣.
是在說什麼呢?.
是在說人當他明白自己無能為力.
他不可以憑自己來拯救自己.
憑自己來遵行神的話.
關鍵是他揭了自己的弓.
意味著他不再憑著自己去遵行神的律法.
而是完全憑著神.
因為神經完成了一切的功.
從福音的角度來說.
不是我們自己去承傳救恩的工作.
不是我們很偉大做某些事情.
或者我們能夠遵行律法因此得到救恩.
不是.
是完全在乎信靠耶穌基督所承傳的功.
是神自己所承傳的工作.
是他所作成的.
我們只是去領受.
所以你很記得聖經中的楊枯陰.

$^{521}$說到主耶穌用餅而餵飽了很多人.
很多人回來找耶穌繼續有飯吃.
耶穌就勸告他們.
你們不要為那些不能存到永遠的食物努力.
他們就問 如果這樣的話.
那怎樣才算是做神的工作呢?.
你記得主耶穌怎樣回答嗎?.
耶穌就說.
信神所差來的就是做神的工作.
我們自己無能為力去領受救恩.
無能為力能夠進入永恆的安息裡面.
是神自己完成一切.
耶穌基督完成整個拯救.
而我們所能夠作的.
或者說我們所能夠回應的.
是信靠祂.
相信祂的應許是真實的.
相信祂的道是確切的.
是不會改變的.
是必然會成全的.
因此你會明白.
為什麼在聖經裡面.
從創世記一直到新約啟示錄.
上上講信心.
因為唯有信心.
叫我們能夠得著神的恩典.
但是你要發覺.
基督在人生的歷程當中.
信心正正是被攻擊的對象.
或者說目標.
在過程裡面.
怎樣靠著神的恩典.
勝過這些攻擊.
而定信心.
領受神至終給你.
一定一切的恩惠.
一轉眼你會發覺.
聖經一直提醒我們.
要留心.
揭了自己的弓.

$^{561}$表示你完全在信心當中.
去跟隨.
依靠上帝.
但是聖經回覆前面一件事.
不要記得.
我們剛才四章開始時提到.
有些人已經聽過福音.
領受了神的道.
但是為什麼領受了之後.
完全與他們無益呢?.
聖經的道是奇妙的.
是滿有能力的真道.
但是為什麼你聽了之後.
會無益處呢?.
你接著要留心聖經的說話說.
但所聽見的道.
沒有使他們得益.
因為他們沒有用信心.
跟那些曾聽見的人聯合起來.
或者說.
我有用信心與所聽見的道.
配合著.
我講一件自己的事情.
我很多年前發現自己患了骨質疏鬆.
當時醫生都很奇怪.
男士這個年紀是沒有的.
再檢查一次可能檢查錯.
但後來真的發現.
是骨質疏鬆.
而且是一些遺傳性的骨質疏鬆問題.
我開始吃很多鈣片.
一直想補足.
吃了一段長時間之後.
再去看醫生再檢查.
嘩!變差了.
我這麼努力吃鈣片.
現在變差了.
他幫我再檢查.
抽血檢查維他命D的情況.
檢查後.

$^{601}$他說:哦!你嚴重缺乏維他命D.
你現在很少曬太陽了.
我說:是的.
他就說:如果你身體沒有維他命D.
你吃多少鈣都不會吸收.
都沒有用.
因此我後來很留意.
吃一些鈣片.
再加上維他命D.
再加上曬太陽.
經歷讓我體會一件事.
身體某些情況.
就算你加了很多維他命D.
但它沒有基本接收的能力.
你多少都沒有益處.
我想說.
你領受了很多真道.
你聽過很多真理.
明白神給你的很多應許.
但如果你沒有信心.
與所聽的道調和.
配合著.
於你無益.
你看見聖經裡很清楚說.
用信心去領受.
當然信心不是一種沒有基礎的相術.
而當你讀聖經時.
你看到聖經所描述的.
神的工作,神的偉大.
神在歷史中所承傳的事情.
你就知道今天他所說的.
一定會承傳.
因為他是信實的神.
在當中幫助自己.
堅固著信心來跟隨他.
過程中聖經有提醒我們.
要留心.
遇到衝擊時.
要竭力進入那安息.
第十一節所說.

$^{641}$免得有人因為不順從而跌倒.
聖經提到要竭力進入那安息.
當你看上下文時.
你就明白竭力.
不是說你自己憑著自己拼搏.
努力去爭取.
不是.
而是你的內心裡一種.
盡心去領受.
全心去接受神所賜的應許.
你聽得清楚.
是全心去領受神所賜的應許.
在裡面自己集中精神去信靠他.
全心全意的持定著.
堅固著你的信心.
你在過程中.
你就不至於因為不順從而跌倒.
因此聖經提到.
沒錯你已經進入安息.
但你仍要竭力.
保持著你的信心.
以至你得著神最終要給你的.
去到十二節的時候.
當你看聖經.
讀完十一節.
突然十二節.
為何在說神的話.
好像主題轉了.
但當你再讀的時候.
你會留意到.
這段跟著的經文.
其實正正提醒我們.
如果我們不小心聽從神的話.
我們就會跌倒.
因為神的話是活的.
神的道是活的.
是活著的.
神的道是活的.
意思是神的應許.
神的警告.

$^{681}$是永恆真實的.
就算你相信不相信.
仍然是會成全.
是永遠真實的.
因為神的話是活的生命.
是活的道.
神的話是有功效的.
意味著神的話是永遠有功效.
聖經二賽斯所說.
不會徒然返回.
必定會成全所應許的事情.
一定會成全.
你稍稍留意到聖經裡面.
說很多關於寓言的事情.
寓言將來會發生的事情.
你發覺到今天.
有人曾經做一些調查.
發覺聖經裡面很多的寓言.
有98\%的寓言.
都已經在歷史當中應驗.
而你知道剩下2\%的寓言.
大多數是關於將來主在臨的一些寓言.
也都將會應驗.
當你看的時候你發覺.
神的話是有功效的.
也就是兩人的利劍.
是能夠將你心中最隱藏的動機.
都能夠揭示出來.
你沒有什麼在神面前能夠隱藏得住.
不要以為可以做一些表面的敬虔.
可以呼運過去.
人家以為我有多好.
神都不會知道.
不會的.
你最心底隱藏的動機.
神都知道.
所以聖經期望我們.
以無偽的信心跟隨祂.
以誠實的信心去聽從祂.
因為祂已經將祂的真道賜給了我們.

$^{721}$如果你相信祂的道是真實的.
你就去聽從.
你就去敬畏神.
你就得到神最終給你的恩愛.
因此在過程當中.
你要察覺.
聖經裡面一直都在勉勵我們.
警覺著.
神的話是真實的.
你去堅定去信靠神的話.
得著神要給你的應許.
各位靈性朋友.
當我們讀到《殷天經文》的時候.
我心裡有一種擔心.
因為很多時候.
當我們讀到聖經對我們的警告的時候.
有時候我們會不知不覺地覺得.
我都搞不定了.
會產生一種自卑的情緒.
甚至會覺得我都放棄了.
但是回想聖經一直對我們提出警戒的時候.
目的不是要叫我們自卑.
是要叫我們警醒.
警告是要帶給我們警醒.
我們更加知道去全心去依靠神.
剛才我們讀到《亞瑟瑪利源》的時候.
主耶穌對神的全心仰望而依靠.
當他全心仰望上帝的時候.
他領受力量面對一切的挑戰.
你會記得當聖經裡面提到.
一個爸爸的兒子被鬼附.
門徒嘗試趕鬼.
但竟然不成功.
爸爸有點失望.
他對耶穌說.
若你能夠作神魔求你幫幫忙.
但耶穌說不是我能夠做什麼.
而是你能否相信.
在那時候似乎焦點放在爸爸身上.
但爸爸很明白.

$^{761}$他馬上對主耶穌說.
主耶穌說我信.
但我信不足.
他承認我信不足.
但他接著說.
求主幫助.
他立即將焦點放在主的身上.
主說我信不足不要緊.
但你能夠幫助我.
因此主耶穌施行大能.
叫他的兒子得釋放.
各位聽眾.
在你們思想希伯來書的經文的時候.
不斷提醒我們要信靠上帝.
但到了最終你仍然回頭.
知道都是神的恩典和能力.
你只是全心去依靠祂.
甚至依靠的力量都是祂給你的.
我信但信不足.
但求主幫助.
因此你讀到聖經.
最後在希伯來書說到第七章的時候.
就清楚跟我們說一句話.
所以凡藉著他進到神面前的人.
他都能夠拯救到底.
你留心.
凡藉著他進到神面前的人.
因為他長遠活著.
他為我們代求.
帶著一份平安的信心.
一路依靠上帝恩典.
去面對一切的衝擊.
你會領受神的保守.
你都能夠站穩在神的真道裡面.
我們一起祈禱.
一切都在乎你自己的恩典.
主要求你心中感動我們的心靈依靠你.
不以為我們可以靠自己解決問題.
不以為我們憑著自己能夠勝過事態.
主要保守我們的心知道向你禱告.

$^{801}$向你尋求.
以致我們雖然面對挑戰.
但這些挑戰反而成為了我們的考驗.
讓我們更加堅固.
主要求你保守我們今天在你面前的眾兒女.
求你堅固我們的信心.
一生堅定信靠你.
奉耶穌成名祈禱.
我們願以我們的真名.
保守我們的信心.
祈禱.
我們願以我們的真名.
保守我們的信心.
祈禱.
我們願以我們的真名.
保守我們的信心.
祈禱.
我們願以我們的真名.
保守我們的信心.
\newpage



\section{希伯來書 5:11-14}
\label{sec:8LlYAk0Xlok}
\textbf{嘗過天恩 卻長蒺藜 (希伯來書5\_11-14;6\_1-12) - 蕭壽華牧師}
\newline
\newline
連結: \href{https://youtube.com/watch?v=8LlYAk0Xlok}{\texttt{ https://youtube.com/watch?v=8LlYAk0Xlok}} ~~~~ 語音日期: 2025-01-04 
\newline
\newline
\hyperref[sec:toZa1ewaUWE]{< < < PREV SERMON < < <}
~
\hyperlink{toc}{[返主目錄]}
~
\hyperref[ch:preacher4]{[返講員目錄]}
~
\hyperref[sec:prT7wwZLltI]{> > > NEXT SERMON > > >}
\newline
\newline
希伯來書 5:11-14
\newline
\begin{longtable}{cl}
\hline
\hline
章節 & 經文 (和合本修訂版)\\
\hline
5:11 & \begin{tabularx}{0.7\textwidth}{X} 論到這事,我們有好些話要說,可是很難解釋,因為你們聽不進去。 \end{tabularx} \\ \\ \relax
5:12 & \begin{tabularx}{0.7\textwidth}{X} 按時間說,你們早該作教師了,誰知還需要有人再將神聖言基礎的要道教導你們;你們成了那需要吃奶、不能吃乾糧的人。 \end{tabularx} \\ \\ \relax
5:13 & \begin{tabularx}{0.7\textwidth}{X} 凡只能吃奶的,就不熟練仁義的道理,因為他是嬰孩。 \end{tabularx} \\ \\ \relax
5:14 & \begin{tabularx}{0.7\textwidth}{X} 惟獨長大成人的才能吃乾糧,他們的心竅因練習而靈活,能分辨善惡了。 \end{tabularx} \\ \\
[1ex]
\hline
\hline
\end{longtable}
$^{1}$此錄音或錄影的版權是屬基督教宣導會不確同所有.
我們的屬靈傳統之一是以真理教導為本.
歡迎你將此錄音或錄影的連結傳給親友.
懇請注意未經授權請勿轉錄或剪輯本錄音或錄影.
各位靈眾平安.
見到很多很熟悉的面孔.
也有機會跟大家一齊來敬拜.
每一次的敬拜我常常都感覺.
都不是必然的.
每次能夠到神面前可以同頂尖一齊同心敬拜.
是因天.
所以我帶著這樣的心.
一齊來到去思想神的話.
請大家和我再一齊同心祈禱.
我們仰望教託主.
我們是主我們的神.
主啊你是各位信實不變的主.
在我們人生的每一個階段.
主啊你仍然的看顧.
你的憐憫永不斷絕.
因為你的慈愛上達諸天.
你的信實直到窮滄.
主啊縱使我們自己環境身體有變化.
但是主啊你永不改變.
因此我們到你的面前來敬拜你.
讚美你.
也全心的依靠你.
因為你是我們的盆石.
我們站在你的上面.
我們就永不動搖.
如今我們到你的面前.
打開你的道.
懇求天上的主賜下你的恩賢.
賜下你寶貴的聖靈.
開啟我們的心思.
讓我們在你面前蒙恩領受.
我們仰望.
祈禱奉耶穌基督聖名.
阿們.
今天跟大家讀思想希伯來書第五章第六章的經文.

$^{41}$我想回憶一件事情.
就是好幾年之前.
我有機會和一群弟兄姊妹去到北美一個地方.
去旅遊.
在當地我們去到一個很特別的地方.
一個地質奇景.
一個叫大裂縫的地方.
我們停車之後.
看到地圖想著很近就能走過去.
所以大家很開心地一路走.
但我首先發覺那條路很崎嶇.
很難走.
上山下山.
走了不久我們經歷了一種感覺.
喂!不要看了.
回頭吧.
上到去搞不定.
大家就說不怕了繼續走.
走多一段路已經有更多人決心要回頭.
但那時候突然間就看到一些看完奇觀的旅客回頭.
看到他們就問還有多久.
一個小時.
哇!還有一個小時.
但他們就告訴我們很值得看.
奇觀很難得看到.
你們快點上去看吧.
因為他們的鼓勵我們就繼續忍耐那一個小時.
繼續在很崎嶇的路.
然後我們終於去到那個地方.
一去到的時候.
心裡有一份滿足.
幸好我們沒有回頭.
這個經歷讓我心裡常常想起一件事情.
在我們跟隨主的路上.
有時候會想回頭.
但何時想到前面神為我們預備的永遠無比的榮耀.
我們心裡就多一份力量去忍耐.
去繼續走.
走完主給我們的路.
去得到主給我們終極的恩典.

$^{81}$極大的恩典.
大家讀聖經的時候會記得聖經裡面在馬特科密十四章.
當主耶穌講到末世很多徵兆的時候.
一路講到末世裡面有很多艱難.
聖徒會被萬民來恨惡.
又提到很多的雕塗會跌倒.
一路講的時候講到最後.
主就說唯有忍耐到底的必然得救.
一個很重要的訊息就是忍耐到底.
這個忍耐不單單是講你忍耐面對的那些患難.
同時是講你忍耐著持守著你的信心.
和持守著你的良善.
因為忍耐包含了你對神的信心.
以至你繼續聽從神對你的吩咐在地上行事.
而聖經就說忍耐到底的必然得救.
而今天我們讀的聖經在不萊書五章六章這段經文.
就正是向當時很多面對患難的信徒發出的警告.
是一段警告的說話.
因為聖經裡面很清楚提醒我們.
凡是蒙恩得救的人都有機會背棄上帝.
但是何時人背棄上帝的時候.
要知道也會有被神棄絕的危險.
因此一個很清楚的訊息同樣的說.
要在一切情況之下忍耐到底.
因此我們今天思想這段經文.
去思想在我們人生的路上面對任何的困難.
如何忍耐到底.
剛才大家讀完經文的時候.
你會留意到一開始第五章十一十二節的時候.
很清楚提醒我們要留意去實踐你所信的.
不單單有一種透露上的了解.
要去警覺著.
他們常常都會提醒我們.
不要單單做一個表面的信徒.
而是真正在生活上行出來.
這本書同樣提醒我們要留心.
在經文裡面你會留意到.
當時提到一些信主的猶太人.
他們信主一段長時間.
本來應該可以做老師.

$^{121}$教人很多信仰的道理.
但是雖然信主這麼多年.
但仍然不能夠領受基本的道理.
十二節所說.
聖經就說不能夠去領受基本道理.
是為什麼呢?.
是因為聖經說到他們遲鈍了.
第十一節說聽覺已經遲鈍.
這裡不是說他們耳朵出現什麼問題.
而是說他們的屬靈的領悟力.
是遲鈍了,變弱了.
領悟不到屬神的道理.
但是你問為什麼他們的屬靈領悟力會遲鈍了?.
是聖經跟著他們的十三,十四節.
聖經就說到因為他們對公義的道理缺乏經歷.
這個經歷是說他們不是不認識公義的道理.
他們認識的,不過他們沒有去練習.
沒有去操練,運用出來.
他們認識,不過他們沒有去練習.
這是一個很重要的提醒.
我師母,我太太是教鋼琴的.
我都說了等一下會知道.
她有時候提醒學生.
有些學生很聰明.
很自然就會懂得看譜.
又會知道拍子是多快多慢.
甚至連指法都很清楚.
但問題是他們總是彈出一些粵語的音樂.
為什麼?因為沒有練習.
很簡單,就是因為沒有練習.
所以就算他們懂,也不能彈奏粵語的樂曲.
也很難再盡心明白很多關於彈琴的事.
也不能去領悟.
聖經就提到一件事情.
十四節提到.
聖經說他們的觀能通過操練而變得成熟.
這種熟練的領悟力不能夠成熟.
是因為不能夠去操練.
但唯有當他們去操練的時候.
就能夠成熟,而且能夠分辨事情的好壞.

$^{161}$你會體會到聖經所說的.
你明白熟練的道理.
不純粹因為你熟悉聖經.
不是因為你讀過很多關於聖經的事情.
而是當你將你所認識的公義道理走出來的時候.
你越走,你越能夠認識神.
越能夠明白熟練的道理.
是在實踐當中去明白.
因此你會知道.
今天你可能有很多機會聽到很多不同的講道.
你上網,全世界最著名的目者所說的訊息.
你有機會聽到.
甚至你可以在網上完成一個神學的學位.
你可以聽到很多不同的聖經的道理.
但如果你不留心.
在你的生活上去實踐.
在你的工作環境,在你的家庭當中.
常常留意我如何實踐.
神自己給我的公義道理.
你會發覺你仍然是熟齡的嬰孩.
你仍然是要靠著吃奶,聖經所說.
來去成長.
但在吃奶的過程中.
你會發覺不能真正地長大.
而且會有跌倒的危險.
這正是聖經提醒我們要警覺警覺.
以致能夠站穩.
在英國有一個作者.
他寫了一本書.
這本書裡說及一件事情.
英文是The Things Among Us.
在我們中間的成同.
英國願意承認自己有信仰的基督徒.
有57\%.
其實今天已經沒有這個數字.
但願意真正實踐信仰的只有7\%.
有人問你如何知道他有沒有實踐信仰.
他說從他回答一條問題就知道了.
回答哪條問題呢?.
有一條問題問他們.

$^{201}$如果你面對著一件事情.
你樂意付代價去做這件事情.
甚至就算你自己不太想做.
但也會去做.
因為你知道這件事情是神的旨意.
因為知道這件事情是神的旨意.
因此你就樂意去做一些.
甚至你不想做的事情.
當人這樣選擇的時候.
就發覺他是一個實踐信仰的人.
而在其他的選擇裡.
發覺他在信仰其他的層面.
同樣有一個很清晰的表現.
他各方面信仰的情況.
都是很明確地表達出來.
兄弟姐妹.
在神心教路上.
你知道實踐信仰有時是付代價的.
但你記得那是神的旨意.
如果你真的相信是神的旨意.
你付上代價去做.
你就真的可以實踐你所領受的真理.
你可以行出來.
當我們在讀的時候.
聖經跟著就再提一件事情.
當我們去保守自己.
去實踐所信的.
但你發覺有些時候.
如果你不去繼續成長.
有一個危險.
你會跌倒.
我們中國人常說.
逆水行舟.
接著不進則退.
不要以為你信主這麼多年.
都對信仰有一定的認識.
但你要幫助自己.
繼續追求成長成熟.
在聖經這裡的六章開始三節就提及到一件事情.
請聖徒離開這些初階.

$^{241}$走向完全的地步.
聖經提到離開初階.
不是叫他們放棄這些信仰的初階.
而是不要停留在初階的地方.
當時很多信了主的猶太人.
大家記得我們之前一直說.
他們面對著很多逼迫.
在當時寫這本書的時候.
聖經裡說到.
在聖徒的生活當中.
面對著很多很多的挑戰.
因為當時的尼祿王.
他焚燒了猶太神.
但他教和給了基督徒.
這些基督徒面對著很大的逼迫.
甚至有些基督徒轉回去相信猶太教.
因此他們不斷在討論猶太教和基督徒.
有些因此一腳踏兩船.
他們都想跳.
因此當時有很多的討論爭論.
究竟誰好誰差.
一直思考兩者之間的憂慮.
但保羅告訴他們.
聖經的作者告訴他們.
要留心離開道理的初階.
你們要進入真正的實踐.
走出神給你的教導.
聖經提到.
神會帶領我們走向成熟的路.
這裡第一節所說.
走向完全的地步.
這個走向原來是一個被動詞.
也意味著不是我們自己.
跟著自己去遵行神的律法.
而是神帶領我們.
神幫助我們走向完全的地步.
因此當我們提到聖經裡提到.
「若神許可」第三節.
我們必定這樣行.
也就是說.

$^{281}$當我們全心的依靠耶穌基督的時候.
我們必能夠在信仰的基礎上得到承傳.
神會幫助我們承傳祂對我們的應許.
重要的是.
你繼續向前走.
按著你所認識的走出來.
但何時我們不這樣做的時候.
在聖經裡提到一段我們很震驚的經文.
經文提到有些教徒.
他們真的經歷過救恩.
上過神自己給他們的天恩滋味.
但竟然他們背棄信仰.
聖經說不可能使他們重新悔改.
這是一個很大的提醒和警惕.
當然弟兄姊妹你們知道.
聖經裡很清楚.
有很多說話可以告訴你們.
神能夠拯救我們到底.
或者我們再看一看有兩節的經文.
我們常常都會讀到.
楊甫音第六章三十七節裡提到.
凡是父慈給我的人必到我這裡來.
到我這裡來的人我必不打他.
趕出去.
上帝會拯救我們.
凡是願意到他面前來的人.
他都會接納.
聖經希伯來書繼續說.
凡是願意藉著他到神面前的人.
他都能夠拯救到底.
這幾個經文給我們看見.
在神的層面.
我是說在神的層面.
在神的層面.
在他的應許裡面.
神能夠拯救我們到底.
就算我們軟弱.
我們有時會跌倒.
但在神的裡面.
他仍然能夠幫助,感動.

$^{321}$引導我們回頭.
回轉歸向他.
但聖經裡面同時提到一個情況.
就是在人的層面.
當他領受神給他的提醒.
三番四次的勸告.
要他回轉,要他歸回.
但人仍然驕傲自恥.
仍然明知道這是唯一得道的路.
但堅定刻意背棄,拒絕上帝.
才說他們.
將耶穌基督重新釘十字架.
明明羞辱他.
對這些人.
聖經會說.
神容讓他們走自己的路.
離開神的救恩.
因此我們見到聖經提及到.
在神裡面.
神對我們有完全的拯救.
他能夠拯救我們到底.
但當人堅持拒絕上帝的時候.
仍然會有一個跌倒的危險.
當我們讀剛才的希伯來書第六章經文時.
你會發覺經文裡.
一路提及一件事情.
聖經裡提及的事情就是.
如果我們一路留心.
保守自己.
持守著坦然無懼的心.
希伯來書三章六節.
三章十四節繼續說.
持守著起初確實的信心.
就在基督裡面有份.
這些經文一路提及的事情.
就是聖經裡提及的.
有跌倒的危險.
其實不是經文的重點.
經文的重點是提及.
我們要警覺.

$^{361}$我們要留心忍耐到底.
堅持到底.
正如這裡所說的.
持守著坦然無懼的心.
持守著起初確實的信心.
就能夠領受神給我們永恆的福分.
永遠屬於神.
是神家裡面的人.
這件事當聖經這樣提及的時候.
不是要嚇我們.
是要不斷鼓勵我們.
在信仰的路上要繼續尋求.
繼續能夠警覺著.
將你所認識的走出來.
你就能夠自終得著.
神給你的恩典.
在讀到的時候.
你會看到聖經裡說到一件事情.
就是在繼續仰望過程當中.
忍耐行善.
聖經裡提到當時的聖徒.
他們無論面對著當時很多迫不及待的環境.
但他們仍然是因為主的名義的緣故.
來保持他們的愛心.
以愛心去服侍人.
我們讀到第九節第十節.
聖經一直強調一件事情.
但法國聖經提到他們.
是能夠繼續堅持行善.
堅持愛心.
因為一個緣故.
然後以第十節提到.
聖經說.
因為神並非不公義.
以致忘記了你們的工作.
和你們為他的名所顯出的愛心.
是為什麼緣故呢?.
是為主的名.
是因為主的名.
不是因為對方可愛.

$^{401}$所以你愛他.
不是因為環境順利.
所以你過去來工作侍奉.
而是因為主吩咐你這樣做.
因此你願意這樣服侍.
在過程中.
聖經繼續來到第十一節的時候.
就很清楚提出一個最重要.
對我們的提醒.
我請大家一起幫我讀出第十一節.
我們渴望你們各人也顯出同樣的熱情.
一直到底.
使你們的盼望可以如廚實現.
聖經提到一直到底.
堅持到底.
保持那份熱誠.
什麼是熱誠呢?.
是繼續堅定信靠主.
堅定行出你的信仰.
剛才提到一群弟兄姊妹.
他們在逼迫當中.
願意活出那種愛心的行為.
我今年年初有機會去到一個宣教工場.
一個很貧困的國家.
遇到一些很資深的宣教士.
因為宣教士在工場裡面.
面對很多的困難.
他也受到當地人對他的傷害.
但後來他退休.
很多人都想著他快快回到香港.
但他卻決心以義工的身份.
繼續留在工場服侍.
很多其他的宣教士和他聊天的時候.
他常常說一件事.
他開玩笑地說.
他說不要緊的.
他就說.
忍耐到底必有東西看的.
放心忍耐著.
一定會見到神.

$^{441}$會承傳祂對我的應許.
我想起一件很有趣的事情.
我記得有一次我去做一個小手術.
去做一個心導管檢查.
去到醫院的時候.
醫院安排我進入一個小型手術室幫我做.
然後在我大腿的內側插一條管進去.
然後去到心臟 看心臟的情況.
插了之後再抽出那條導管.
導管的傷口稍為大.
所以他要很小心.
找護士姑娘幫我按壓傷口.
要起碼二十分鐘.
然後才讓我走.
過程中有一位護士.
過來幫我按壓住.
當然還是有一點痛.
一直痛著讓她按壓傷口.
突然間一位教徒.
一位姑娘 姐妹.
認得「施牧師是你嗎?」.
「對呀 我想問一些牧師關於苦難的問題」.
然後她就問我.
「為何教徒會有苦難呢?」.
當我一直被她按壓的時候.
我突然間發覺在苦難當中回答苦難問題.
都很不容易.
要感受不只是理性的討論.
但突然間我心裡有一種想法.
就是這個傷口會康復的.
這個傷口將來會不同.
我就帶著一個平安的心.
繼續和她聊天.
然後我發覺當聖經裡說到.
我們面對著很多挑戰的時候.
聖經說「你要記得」在十二節說.
「你們不會散 難散 反悔」.
「會效法那些」.
「藉著信心和忍耐承受應許的人」.
一個很特別的情況就是.

$^{481}$有些人當他帶著忍耐.
他就有更大的動力.
什麼動力呢?.
因為他裡面有一種盼望.
你記得前面十一節.
最後一句話說.
「使你們的盼望可以完全實現」.
「因為對前面主應許你的永遠的極重的榮耀」.
「他所應許給你的完全的拯救」.
「因此你就會有更大的力量」.
「來忍耐眼前的難處」.
「效法這些有信心又忍耐的人」.
「因為他們都是有盼望的人」.
「而你今天面對著你今天的挑戰」.
「你記得這些一切都只不過是暫時」.
「會是過去」.
「但是神告訴你」.
「你依靠祂 盼望祂給你的應許」.
「你能夠繼續忍耐去愛人」.
「繼續忍耐去行善」.
「行完神給你的路」.
「繼續祂要給你的豐盛的應許」.
「我們一起祈禱」.
「我們是主我們的神」.
「我們雖然軟弱」.
「面對著很多生活的挑戰」.
「主我們知道靠著你」.
「我們能夠成過」.
「我們懇求天上的主」.
「將永恆的盼望刻在我們心中」.
「讓我們心中知道」.
「你自己掌管天地」.
「你有應許」.
「至暫至輕的苦楚」.
「要為我們承傳下極重」.
「無比永遠的榮耀」.
「我們就放膽忍耐面對」.
「親愛主帶領我們走前路」.
「一生歡喜快樂的跟隨耶」.
「奉耶穌基督成名祈禱」.

$^{521}$「阿們」.
\newpage



\chapter{蘇穎睿}\label{ch:preacher5}
\begin{multicols}{3}
\minitoc
\end{multicols}
{ \scriptsize


\begin{xltabular}{\textwidth}{|p{0.15\textwidth} p{0.6\textwidth}|p{0.07\textwidth} p{0.1\textwidth}|}
\hline
約翰福音 15:11-17 & \hyperref[sec:prT7wwZLltI]{新與變 (約翰福音15\_11-17) - 蘇穎睿牧師 [約翰福音研讀 - 第61講]} & 2025-01-03 & \href{https://youtube.com/watch?v=prT7wwZLltI}{\texttt{ prT7wwZLltI}} \\
約翰福音 15:18-25 & \hyperref[sec:GDV7iT9TooA]{計算代價 (約翰福音15\_18-25) - 蘇穎睿牧師 [約翰福音研讀 - 第62講]} & 2025-01-08 & \href{https://youtube.com/watch?v=GDV7iT9TooA}{\texttt{ GDV7iT9TooA}} \\
約翰福音 15:26-27 & \hyperref[sec:pF1FrHKEPww]{認識聖靈 (約翰福音15\_26-27;16\_5-15) - 蘇穎睿牧師 [約翰福音研讀 - 第63講]} & 2025-01-17 & \href{https://youtube.com/watch?v=pF1FrHKEPww}{\texttt{ pF1FrHKEPww}} \\
約翰福音 16:12-16 & \hyperref[sec:M4alGuubf1o]{啟示與真理(約翰福音16\_12-16) - 蘇穎睿牧師 [約翰福音研讀 - 第64講]} & 2025-01-28 & \href{https://youtube.com/watch?v=M4alGuubf1o}{\texttt{ M4alGuubf1o}} \\
約翰福音 16:16-24 & \hyperref[sec:HaGDtN4u47U]{那等候的日子 (約翰福音16\_16-24) - 蘇穎睿牧師 [約翰福音研讀 - 第65講]} & 2025-02-02 & \href{https://youtube.com/watch?v=HaGDtN4u47U}{\texttt{ HaGDtN4u47U}} \\
約翰福音 16:25-33 & \hyperref[sec:fV_h6TniAkc]{信與覺 (約翰福音16\_25-33) - 蘇穎睿牧師 [約翰福音研讀 - 第66講]} & 2025-02-09 & \href{https://youtube.com/watch?v=fV_h6TniAkc}{\texttt{ fV\_h6TniAkc}} \\
約翰福音 17:1-5 & \hyperref[sec:wiDRWRXrtjM]{榮歸 (約翰福音17\_1-5) - 蘇穎睿牧師 [約翰福音研讀 - 第67講]} & 2025-02-13 & \href{https://youtube.com/watch?v=wiDRWRXrtjM}{\texttt{ wiDRWRXrtjM}} \\
\hline
\end{xltabular}
}
\newpage



\section{約翰福音 15:11-17}
\label{sec:prT7wwZLltI}
\textbf{新與變 (約翰福音15\_11-17) - 蘇穎睿牧師 [約翰福音研讀 - 第61講]}
\newline
\newline
連結: \href{https://youtube.com/watch?v=prT7wwZLltI}{\texttt{ https://youtube.com/watch?v=prT7wwZLltI}} ~~~~ 語音日期: 2025-01-03 
\newline
\newline
\hyperref[sec:8LlYAk0Xlok]{< < < PREV SERMON < < <}
~
\hyperlink{toc}{[返主目錄]}
~
\hyperref[ch:preacher5]{[返講員目錄]}
~
\hyperref[sec:GDV7iT9TooA]{> > > NEXT SERMON > > >}
\newline
\newline
約翰福音 15:11-17
\newline
\begin{longtable}{cl}
\hline
\hline
章節 & 經文 (和合本修訂版)\\
\hline
15:11 & \begin{tabularx}{0.7\textwidth}{X} 「我已對你們說了這些事,是要讓我的喜樂存在你們心裡,並讓你們的喜樂得以滿足。 \end{tabularx} \\ \\ \relax
15:12 & \begin{tabularx}{0.7\textwidth}{X} 你們要彼此相愛,像我愛你們一樣,這是我的命令。 \end{tabularx} \\ \\ \relax
15:13 & \begin{tabularx}{0.7\textwidth}{X} 人為朋友捨命,人的愛心沒有比這個更大的了。 \end{tabularx} \\ \\ \relax
15:14 & \begin{tabularx}{0.7\textwidth}{X} 你們若遵行我所命令的,就是我的朋友。 \end{tabularx} \\ \\ \relax
15:15 & \begin{tabularx}{0.7\textwidth}{X} 以後我不再稱你們為僕人,因為僕人不知道主人所做的事;但我稱你們為朋友,因為我從我父所聽見的一切都已經讓你們知道了。 \end{tabularx} \\ \\ \relax
15:16 & \begin{tabularx}{0.7\textwidth}{X} 不是你們揀選了我,而是我揀選了你們,並且派你們去結果子,讓你們的果子得以長存,好使你們奉我的名,無論向父求甚麼,他會賜給你們。 \end{tabularx} \\ \\ \relax
15:17 & \begin{tabularx}{0.7\textwidth}{X} 我這樣命令你們,是要你們彼此相愛。」 \end{tabularx} \\ \\
[1ex]
\hline
\hline
\end{longtable}
$^{1}$各位大兄姐妹 各位朋友 你們好.
歡迎大家參加我們三藩市播道會.
網上的午堂崇拜.
今天我們繼續思想約翰福音的信息.
我們來到約翰福音第十五章.
第一節至十七節.
題目是新與變.
在思想這段聖經之前.
請你同心低頭 我也同禱告.
雖然過去一年真的風風雨雨.
到處都是好像凹凹亂亂.
但我們相信你仍然保守我們.
我們為了這個世界局勢交在主的手中.
叫人都聽到你的福音.
以致能夠歸向你.
求聖靈也都向我們每個人講你所講的話.
因為你賜給我們一個新的命令.
彼此相愛.
就教我們能夠實踐彼此相愛.
你與我們同在.
祈禱奉耶穌基督名求.
阿們.
今天是十二月二十九日.
還有幾天呢.
就到二零二五年.
一個新的一年.
新的一年.
在希臘文裡有兩個字可以做新.
一個叫Kainos.
一個叫Neos.
雖然這兩個字可以做新.
但意思就有很大很大的分別.
首先我們講講Neos這個字.
Neos的意思是新.
或者叫年青.
幼嫩.
英文的Neo這個字.
是希臘文的.
換句話說.
新之所以為新.

$^{41}$是因為從時間方面來說.
有新就必有久.
有幼就必有老.
今天是老夫老妻.
曾幾何時.
是新婚夫婦.
今天是新丁.
沒多久就變成老首了.
當然.
新往往是陷入一些危機.
新入行.
什麼都不懂.
這是一個試煉期.
新移民初來步到.
什麼都不熟悉.
面對困難重重.
真的不容易.
如果我們說新是一個危機.
但我覺得舊更加是一個更大的危機.
所以做的所謂老化.
老油條.
正是表示這個危機之所在.
年長的時候.
很容易就會.
恩循仇舊.
一句就說.
舊時不是這樣的.
就騙了很多生機.
同樣.
在淑寧的生命裡.
如果我們恩循仇舊.
生命沒有新的經歷.
新的體驗.
新的知識.
我們所說的見證都是幾十年來.
陳年老土.
真的大有問題.
所以聖經時常提到.
我們要更生變化.
羅馬書第十二章第二節.

$^{81}$說不要效法這個世界.
只要心意更生而變化.
很有趣.
保羅是用更生而變化.
這個新.
不是用Neos的字.
是用另外一個字.
叫做Kinos.
Kinos的意思是常生.
日日生.
九日生.
不會老化不會變舊.
我們看看新約聖經.
永遠會變回舊約聖經.
聖經說若有人做基督女.
就是新做的人.
永遠是新人.
不會變成舊人.
同樣新耶路撒冷.
就是新耶路撒冷.
這是Kinos的新.
不是Neos的新.
今天我們需要做常生的人.
不住更生.
不住脫去舊人.
穿上新人的樣式.
不住禱告.
不住關懷.
不住學習.
天天經歷神的恩典.
這是重要的一件事.
聖經裡有另外一個字.
跟新字很有關係.
有個變字.
我們想起.
《諾貝爾書》第十二章第一至第二.
第輕門.
我以神的慈悲勸你們.
將生理獻上當作活祭.
是聖潔是神所起義.

$^{121}$我們如此是逢那時理所當然.
不要效法這個世界.
只要心以更生而變化.
叫你們參與何謂神的善良.
順傳可喜悅的旨意.
原來變這個字.
單單在.
《諾貝爾書》第十二章第一至第二.
有兩個不同的希臘文字.
都可以變.
第一個叫Skema.
第二個叫Morphe.
Skema是指.
外形的改變.
七歲的我和七十歲的我.
當然身材樣貌.
心智都很不同.
但我仍然是我的我.
同一個生命的我.
同一道理.
我講道時穿西裝.
睡覺時穿睡衣.
表面上很不同.
但仍然是我的我.
在《羅馬書》第十二章第二至第二.
中文本書將Skema效法.
不要隨意隨地.
去改變你的人.
就是看世界做人.
不要效法這個世界.
這個世界好像變形蟲.
隨波逐流.
誰有好處就跟誰.
這就叫Skema.
正經叫我們Morphe.
Morphe就不同Skema.
那個不是一個.
外表的改變.
是生命的褪變.
就好像一個蟲.

$^{161}$脫繭而出.
變成一個美麗的蝴蝶一樣.
英文的Metamorphosis.
就是由這個字組成.
這個字就是Morphe.
心而更生.
而變化.
Morphe叫我們生命不住的更生.
不住的成長.
不要像外濟不前死水一池.
我就想起一個故事.
是魯舍所說的.
一個非常有趣的故事.
叫做魯智豪.
沙合場一個賣絲綢緞的魯智豪.
他金面黑色的招牌.
放著舊的茶几.
掛著的燈籠.
一派武俠小說中的.
古老鋪店的格局.
他不單格局老.
經營手法也都陳舊.
全使用一切現代式的方式.
不賣廣告.
不做宣傳.
甚至不准夥計大聲說話.
他以魯智豪.
招來顧客.
魯夥計生得遲.
在沙合場工作了十幾二十年.
他一身南部長衫.
一舉手.
一頭竹.
甚至他的咳嗽.
本有沙合場的味道.
全是魯智豪的產品.
他覺得自豪.
這陣子前長鬼退休.
沙合場就請了一個姓周的.
新長鬼來.

$^{201}$在新一字的眼光.
周長鬼簡直是野雞.
破壞了沙合場的老規矩.
上任不到一兩日.
就換了一個七彩的招牌.
上面寫著.
大減價三個大字.
又安排一檔陽虎陽號.
從天罡就吹到三間.
還送傳單.
賣廣告.
客人一來就送茶送煙.
甚至買一送一.
又贈送洋娃娃.
這些噱頭.
看不過眼.
非常不滿.
以為他破壞了魯智豪的格局.
正被他辭職不幹的時候.
怎知周長鬼先讓他走一步.
他覺得沙合場的老規矩太深.
不能盡展所長.
索性辭職不幹.
前長鬼再出江湖.
一切恢復舊觀.
沒多久他們發現.
對面開了新的絲綢堡.
嘩!再看清楚.
不是誰.
竟然.
剛剛辭職周長鬼.
竟然.
打對台戲.
搶我們的生意.
不行.
三萬元.
不行.
沙合場為了節省開支.
不惜裁員.
這個身不自一人兼任四職.

$^{241}$堅持原則.
深信客人是懂得選擇的.
不會受這些小動作.
被騙.
但情況卻和他所想像.
完全不同.
半年後沙合場就沒生意.
倒閉了.
中國人是守舊怕改變.
往往是孤立封閉.
因秦苟且.
以至毫無生氣.
聖經教導我們不住更新.
深意改變.
在個人方面.
在教會方面.
都是如此.
為什麼我們有這樣的能力.
可以改變而不會因秦呢.
我們今天看這段聖經.
就非常有趣.
因為耶穌基督給我們有三樣新的東西.
新的命令.
是彼此相愛.
給我們有新的身份.
我們不是奴隸.
我們是他們的朋友.
給我們新的使命.
原來我們是耶穌基督.
被指派做大使.
我們詳細看看.
究竟今天.
所給我們這個新人.
新的命令.
新的身份是什麼意思.
這不是Henir的身.
這是一個Kainos的身.
是永遠的身.
因為我們已經.
生命不已改變成.

$^{281}$耶穌基督的朋友.
又成為他的大使.
所以我們這樣做.
我們詳細看看.
第一 神的命令彼此相愛.
12-13 字.
奇怪 這是耶穌臨別之言.
他並沒有答應門徒.
一個新的政府.
一個沒有剝削.
一個被人尊重.
一個自由.
一個有權利.
一個有民主的政府.
耶穌沒有答應門徒.
新的改革.
做出一個人間天堂.
不再有苦難.
不再有風雨.
沒有.
他沒有給我們這些.
他給我們什麼.
他給我們三樣東西.
新的命令 新的身份 新的使命.
我們逐一來看看.
思想探究一下.
彼此相愛.
這就是我的命令.
人為朋友寫命.
人的愛心.
沒有比這個更大的.
耶穌說你們要彼此相愛.
這是我的命令.
耶穌說這是法律的依歸.
法律的依歸是什麼.
你要盡心盡性.
盡意盡力愛你的神.
並且要愛人如己.
看來耶穌給我們一個新的命令.
就是彼此相愛.

$^{321}$我們奇怪.
愛怎麼可以是一個命令.
如果我不喜歡他.
見到他就討厭.
你叫我迫我去愛他.
尤其是他所作所為.
他的虛偽.
看見他已經火滾了.
你竟然叫我去愛他.
那我豈不是虛偽.
我不喜歡他.
我怎麼去愛他.
愛是一個感受.
情有獨鍾.
怎麼可以是一個命令.
如果我是被迫.
去愛某個人.
這個人就不是真正愛.
看來似乎很有道理.
但深深思考.
這些是似是非.
舉一個很簡單的例子.
我知道我自己也有很多.
不太好的事.
我也很不喜歡.
也很討厭自己的不好.
雖然我不喜歡.
自己的很多事.
性格上.
自卑等等.
但我仍然非常愛我自己.
世上沒有一個人說.
為什麼你這麼虛偽.
既然你知道你這麼多不好的事.
就扔掉它.
怎麼會喜歡這種不好的事.
怎麼會喜歡一個這樣的人.
你怎麼去愛你自己.
如果這樣說.
你一定心理有問題.

$^{361}$耶穌說你愛人如己.
就是這個意思.
雖然你不喜歡某人.
有很多事.
但你仍然可以去愛他.
正如你不喜歡你自己很多事.
你依然可以愛自己.
世上愛不是一種感受.
這麼簡單.
時刻來時刻去.
如果是這樣.
愛是毫無保障.
愛是一個意志多感情.
是內心的決定.
當你決定去愛的時候.
你會發覺.
這個人也有可愛之詞.
所以耶穌說這是我的命令.
換句話說.
這不是可有可無.
愛可以培植.
愛是可以培植的.
我們要學培植彼此相愛.
所以耶穌說.
這是我給予我們一個新的命令.
或者你會問.
愛這件事.
很抽象的.
是不是這麼抽象呢.
我們看看耶穌13次怎麼說.
人為朋友寫命.
人的愛心.
會有比這個大的.
因為愛是一個犧牲的愛.
甚至寫命的愛.
寫命這字叫Tethymy.
Tethymy是什麼意思呢.
放下寫下.
放下不吝嗇不吝嗇.
為了他人.

$^{401}$不惜寫下自己生命.
你說怎麼會有這樣的事.
就是抽象.
新雙城的Tale of Two Cities.
才有這樣的故事.
這個故事.
現實哪有這樣的事.
誰會這樣做.
不是耶穌就是這樣.
耶穌為了我們甘願寫下自己的生命.
以至我們得到生命.
或者會說寫命.
我都可以.
沒問題很容易.
因為沒什麼機會.
真的寫命.
你說你愛太太.
為她寫命.
多可惜.
你太太的命不夠你可以代她.
不是太多.
但你叫我犧牲我的金錢.
犧牲我的嗜好.
犧牲叫我不看49ers.
犧牲叫我做這個做那個.
那我就不願意.
犧牲我的命就願意.
犧牲其他就不願意.
這根本就是藉口.
耶穌叫我們彼此相愛.
這是一個命令.
我們愛.
但我們為何不可以這樣做.
我們告訴別人.
犧牲命無所謂.
但我不犧牲我的時間.
我不犧牲我的喜好.
我不犧牲一些罪惡的慣性的問題.
根本上.
你就不是愛.

$^{441}$人的問題在哪裡.
你不能彼此相愛.
相反你只是彼此踐踏.
彼此掙扎.
彼此憎恨.
彼此為了道氣.
一時的氣氛.
他們竟然做出傷天害理的事.
這是人的罪行.
人離開神的結果.
一個人在罪裡面.
得不到解決的時候.
這個世界是永無寧日.
我想到很久之前.
聯合國秘書長韓馬紹先生.
說了一句話.
我想我永遠無法見到.
安寧和平的日子.
在人生在家庭在世界在共事.
都是血淋淋的商場.
今天世界最需要的不是錢.
不是食物.
而是內心的平安.
彼此相愛的心.
我想說得很對.
今天你看看這個世界是怎樣.
你憎我奪.
你如我詐.
我們都是.
你打我我打你.
你欺負我我欺負你.
你利用我我利用你.
充滿了.
很多的憎恨.
很多的掙扎.
耶穌說你們要彼此相愛.
我們看到一個新的命令.
就是彼此相愛.
我們再來看看.
耶穌不單止給我們一個新的命令.

$^{481}$祂還給我們一個新的身份.
這個身份是什麼.
我們是祂的朋友.
14-15.
你們若遵守我所吩咐的.
就是我的朋友.
以後我不再稱你們為親友.
以後我不再稱你們為僕人.
因為僕人不知道主人所作的事.
我乃稱你們為朋友.
因為我從我父所聽見的.
已經都告訴你們了.
耶穌說.
我不再稱你們為僕人.
稱你們為朋友.
一個新的身份.
或者我們首先看看.
什麼叫做僕人.
你不再是僕人.
僕人是什麼意思.
英文叫做Doulos.
Doulos的意思是什麼.
奴隸是一個沒有自主權.
沒有自由.
沒有尊嚴沒有身份的奴隸.
摩羅蘇說.
我未曾信耶穌之前.
是罪的奴隸.
是賣給罪了.
以致納智行善由得我.
作出來由不得我.
好像失去了一個自控的能力.
任由罪惡擺佈驅使.
其實我想我們很多人都經歷過這個掙扎.
好像失去了一個自控的能力.
好像掙扎出不來.
但耶穌說.
我不稱你們為奴隸.
你們是我的朋友.
Doulos這個字是朋友.

$^{521}$是親愛的友人.
耶穌就稱.
神的朋友.
現在是成為耶穌基督的新朋友.
我想在香港大學讀書的時候.
羅漢牧師的爸爸.
羅香林.
是我們的老師.
羅香林教授是一個很有學問.
很謙虛的學者.
有一次上課的時候.
說蘇慧若你來辦公室.
我有事跟你談.
於是我不知道什麼事.
就去到羅教授的辦公室.
原來他剛剛出了一本新書.
新作來的.
送一本給我.
我真是開心到極.
第一頁看一看.
他寫了幾個字.
榮耀吾興旨政.
一拿這本書心花怒放.
教授叫我吾興.
我只是一個學生.
又不是一個特別出色的.
他竟然現在稱呼我.
吾興.
清興到底.
還叫我旨政.
我哪有資格旨政.
心裡面開心到極.
現在耶穌介紹我們是他的朋友.
你們不是我的僕人.
是我的朋友.
這是我的朋友.
心都涼了.
不單是我也是你也是他.
所有在基督裡的人.
在基督裡都成為了耶穌的朋友.

$^{561}$一個新的身份.
做朋友當然有權利.
也有義務.
在我們來說義務是第十四字.
你們遵行我所吩咐的.
就是我的朋友了.
如果你想做耶穌的門徒.
就有一件事一定要做.
是遵行他所吩咐的.
沒有第二個方法.
唯一的方法就是要知道他所吩咐的.
跟著他吩咐而行.
這個就是他的朋友的特徵.
有一件事他要做的.
所有看到的都已經告訴你們了.
耶穌沒有隱藏什麼.
他將他父親告訴他的事.
一一告訴門徒.
並且叫門徒遵守.
這個就是朋友了.
我記得在廣大讀書的時候.
何太太說.
在系裡有一個讀書小組.
幾個朋友.
組成了一個study group.
一起讀書.
有些人找到一些很好的資料.
考試很不錯.
但他不動聲色.
一樣是去讀書小組.
但將這些資料收藏起來.
不為人知.
一個人獨佔.
考試的時候.
他能夠單獨答出答案.
取得高分.
這些就是朋友了.
當然不是了.
耶穌是我們的朋友.
他將他的救恩之道全部告訴我們.

$^{601}$遵守,這就是真正的朋友.
今天人的問題在哪裡.
身份問題.
我們不是耶穌的朋友.
而是罪的奴隸.
林青園是台灣一個很出名的作家.
被當一個虔誠的佛教徒.
有一次他被邀請.
去到一個小學四年級的學生演講.
講題叫極樂世界.
首先黑板中間畫一條線.
將黑板分成兩半.
右半邊寫著天堂.
左半邊寫著地獄.
然後叫小學四年級的學生.
在天堂和地獄中寫一項文字.
天堂有什麼.
地獄有什麼.
那些學生很開心.
看看天堂的欄.
有些人寫著花.
笑.
樹木.
愛情.
水果.
光.
星星.
朋友.
蛋糕.
美女.
俊男.
銀紙.
天堂.
地獄.
黑暗.
骯髒.
灰色.
人間介乎天堂和地獄之間.
林青元說不是.
人是可以在天堂.

$^{641}$當他內心充滿愛的時候.
就在天堂.
但也可以在地獄.
內心充滿恨的時候就在地獄.
林青元只是說了一半.
當一個人陷在罪惡裡的時候.
他是不是真的能夠決定脫離罪惡呢.
中國人這句說話說得很好.
無論你多大也好.
你捧自己也捧不了.
溺者不能自救.
壯士不能自居其身.
所以保羅說得好.
立之行善由得我.
做出來由不得我.
我不想做那些惡我偏偏去做.
我想做的事我偏偏做不到.
不去做.
溺者不能自救.
壯士不能自居其身.
問題不是在乎我們想不想去愛.
想不想離開這些恨惡.
但很多時候身不由己.
問題在於我們的身份.
如果我們的身份成為罪的奴隸.
而不是耶穌的朋友.
一定有問題.
所以我們去到第三.
新的使命.
一個什麼象徵給我們使命.
是大事.
我們看看第十六至十七節.
不是你們揀選了我.
是我揀選了你們.
並且分派你們去見果子.
叫你們的果子上傳.
使你們奉我的名.
無論向父求什麼.
他就致給你們.
我這樣吩咐你們.

$^{681}$是要叫你們彼此相愛.
首先我們要了解.
我們之所以有新的命令.
新的身份.
不是我們有什麼好東西.
不是因為我們做了什麼.
豐功偉績.
不是完全是上帝的恩典.
是你們揀選了我.
是我揀選了你們.
揀選這個字很有意思.
這個Agnegomai這個字.
是什麼意思.
從眾人中呼召你出來.
揀選了你出來.
原來神是要.
將我們從為奴之地.
揀選了我們出來.
不是我們揀選神.
是神先揀選我們.
這是祂的恩典.
這是祂的愛.
祂揀選我們做什麼.
三個非常重要的動詞.
第一個字是.
派.
第二個字是去.
第三個字是結果子.
首先我們看看派.
派是什麼意思.
耶穌不單止揀選我們.
更派我們出去.
派這個字Tetheim.
是和十四字寫明這個字是同一個字根.
非常有趣.
換句話說耶穌的寫明.
犧牲.
指派我們去.
是耶穌為我們的死.
而激勵我們這樣做.

$^{721}$第二個字是去.
Opago.
去傳福音.
去宣傳耶穌基督為我們犧牲的福音.
作宣告的工作.
我們都是基督的大能.
被選派去做福音的工作.
第三個字就是.
結果子.
而結果子是常存的.
為什麼可以常存.
人的生命.
得救的生命.
耶穌也吩咐他們.
將福音傳開.
讓人得生命.
這才是永恆的投資.
耶穌知道我們的軟弱.
所以給我們一個定心丸.
凡你們奉我祈求的.
就給你們.
禱告的應許是何等寶貴.
使耶穌給我們的使命.
不是建立一個四化的中國.
一個民主的中國.
不是去改變我們外面的改革.
事實上很多時候.
外面是改不了.
真心剔過.
有很多時候.
我們發覺.
我們以為可以改變.
其實改變不了.
就好像傳道書說.
這些都是一代又來一代又去.
是呀沒錯呀.
神給我們使命.
我們要改變.
神給我們使命.
讓我們福音傳開.

$^{761}$讓人離開罪惡.
成為耶穌的朋友.
如果我們這樣去做的時候.
第一次就告訴你喜樂.
第十七次就告訴你彼此要相愛.
面對著種種的掙扎.
矛盾.
問題.
我們相信哪樣東西才是唯一的真理.
唯一的代求.
幫助我們.
讓我們能夠.
真正.
打開我們的心門.
讓我們能夠明白.
原來耶穌基督不是叫我們建立一個.
人間的天堂.
是改變我們的生命.
轉變這個世界.
成為一個新天新地的世界.
而不是我們現在的世界.
繼續延續.
耶穌叫我們離開罪惡.
成為祂的朋友.
給使命給我們.
這樣我們就可以有兩樣東西.
第十一次就說喜樂.
我們就很開心.
第十七次就彼此相愛.
面對著種種的掙扎.
矛盾問題.
我相信這是唯一的出路.
唯一的盼望.
主祝福我們.
享受.
祂所給我們的新的生命.
新的命令.
更加給我們.
有一個新的看法.
一個續生的看法.

$^{801}$有愛.
有意義.
有價值.
的一個人生.
我們同心一同禱告.
給你寶貴的話語提醒我們.
新的命令彼此相愛.
這個世界實在太多恨.
太多厭惡.
誰能夠懂得去.
享受你所賜給我們.
那個和睦.
你與我們同在.
賜給我們一個新的生命.
你不單賜給我們.
一個新的生命.
你給我們一個新的身份.
在耶穌基督裡面我們成為了一個新造的人.
舊時已過都變成.
新的你.
你更加幫助我們.
有新的使命.
做你的大事.
將你的福音傳開.
讓萬民聽到你的福音.
得救.
多謝你.
祈禱奉耶穌命求.
阿們.
\newpage



\section{約翰福音 15:18-25}
\label{sec:GDV7iT9TooA}
\textbf{計算代價 (約翰福音15\_18-25) - 蘇穎睿牧師 [約翰福音研讀 - 第62講]}
\newline
\newline
連結: \href{https://youtube.com/watch?v=GDV7iT9TooA}{\texttt{ https://youtube.com/watch?v=GDV7iT9TooA}} ~~~~ 語音日期: 2025-01-08 
\newline
\newline
\hyperref[sec:prT7wwZLltI]{< < < PREV SERMON < < <}
~
\hyperlink{toc}{[返主目錄]}
~
\hyperref[ch:preacher5]{[返講員目錄]}
~
\hyperref[sec:pF1FrHKEPww]{> > > NEXT SERMON > > >}
\newline
\newline
約翰福音 15:18-25
\newline
\begin{longtable}{cl}
\hline
\hline
章節 & 經文 (和合本修訂版)\\
\hline
15:18 & \begin{tabularx}{0.7\textwidth}{X} 「世人若恨你們,你們要知道,他們在恨你們以前已經恨我了。 \end{tabularx} \\ \\ \relax
15:19 & \begin{tabularx}{0.7\textwidth}{X} 你們若屬世界,世界會愛屬自己的;只因你們不屬世界,而是我從世界中揀選了你們,所以世界就恨你們。 \end{tabularx} \\ \\ \relax
15:20 & \begin{tabularx}{0.7\textwidth}{X} 你們要記得我對你們說過的話:『僕人不大於主人。』他們若迫害了我,也會迫害你們,他們若遵守了我的話,也會遵守你們的話。 \end{tabularx} \\ \\ \relax
15:21 & \begin{tabularx}{0.7\textwidth}{X} 但他們要因我的名向你們做這一切的事,因為他們不認識差我來的那位。 \end{tabularx} \\ \\ \relax
15:22 & \begin{tabularx}{0.7\textwidth}{X} 我若沒有來教導他們,他們就沒有罪;但如今他們的罪無可推諉了。 \end{tabularx} \\ \\ \relax
15:23 & \begin{tabularx}{0.7\textwidth}{X} 恨我的也恨我的父。 \end{tabularx} \\ \\ \relax
15:24 & \begin{tabularx}{0.7\textwidth}{X} 我若沒有在他們中間做過別人未曾做的事,他們就沒有罪;但如今連我與我的父,他們也看見了,也恨惡了。 \end{tabularx} \\ \\ \relax
15:25 & \begin{tabularx}{0.7\textwidth}{X} 這是要應驗他們律法上所寫的話:『他們無故地恨我。』 \end{tabularx} \\ \\
[1ex]
\hline
\hline
\end{longtable}
$^{1}$各位大型節目的各位朋友你們好.
歡迎大家參加我們三藩市報道會網上的午行崇拜.
今天我們繼續思想約翰福音的信息.
我們來到約翰福音第十五章十八節至二十五節.
題目叫計算代價.
在開始思想這段聖經之前.
請同心低朋或者同禱告.
慈悲讓我天父我們滿心多謝你.
多謝你賜給我們有生命有氣息.
更加給我們有機會學習你的話語.
求你藉著你的話語提醒怨念教導安慰督澤我們.
叫我們跟隨你的時候真的要認真計算好代價.
不單是這樣.
求你更加幫助我們計算不跟從你.
是要付上什麼代價.
求聖靈啟開我們的心.
叫我們明白你的話語.
並且遵守你的話語.
我們這樣禱告.
拉系奉耶穌基督名求.
阿們.
我曾經聽過一個這樣的故事.
從前有隻老蜘蛛.
在一間破舊的房屋.
建了一個蜘蛛網.
他不單有潔癖.
而且他深信他的顧客一定很喜歡那些.
又乾淨又美麗的環境.
所以每逢有昆蟲或者蒼蠅.
進入蜘蛛網的時候.
他就立即清除乾淨.
不留下任何痕跡.
以免引起新的顧客有任何懷疑.
那天來了一隻自認為是聰明絕頂的老蒼蠅.
他在蜘蛛網邊走來走去.
看清楚了.
不敢輕舉妄動.
因為他深信聰明人不可以亂來.
不可以衝動.
一定要計清計楚.

$^{41}$蜘蛛看到這隻老蒼蠅.
肥絲大隻.
應該是一頓美好的晚餐.
以為可以隨手可得.
誰知他不想當.
於是蜘蛛對這位蒼蠅大哥友善地說.
喂 大哥大哥.
請進來坐坐吧.
你這樣飛來飛去.
飛到累了.
為什麼不休息一下.
先喝杯茶.
享受一下吧.
那隻蒼蠅怎樣回答他.
我就不想你當.
我爸爸教下.
沒有人光顧的地方.
極有限.
我看你這隻蒼蠅.
靜鷹鷹.
一定不是好東西.
如果是好東西.
一定奇門弱鷲.
蜘蛛看到他不想當.
也無奈.
忽然之間.
他看到地上有一大堆蒼蠅.
在跳舞.
跳來跳去.
好像很開心.
開什麼大食會一樣.
再看清楚.
在一片很大的黃色地毯裡.
在打棍鬥.
在跳舞.
好像很享受.
蒼蠅看到很開心.
不知道是不是7月4日.
國慶日.
在開派對.

$^{81}$正想飛下去的時候.
和一班同志一起享受.
突然有個黃蜂擋住他.
對他說.
蠢材不要去.
是陷阱.
你知不知道.
那些是專用來捉蒼蠅的膠紙.
那個自以為聰明的蒼蠅.
我就不信你了.
如果是陷阱.
怎麼會有那麼多蒼蠅上當呢.
你看看.
那些蒼蠅全部在跳舞.
個個好像很開心.
一定沒有錯的.
好不理你那麼多.
我先去參加派對.
你就知道有什麼後果了.
耶穌告訴我們.
在我們前面有兩條路.
一條是寬的路.
一條是窄的路.
一條是寬的路.
路是寬的.
走的人也多.
一條是窄的路.
走的人很少.
但寬的路引致滅亡.
在今天這個世代的風氣.
就好像老蒼蠅一樣.
我們喜歡人多.
我們要排隊.
我們要跟風.
我們不喜歡獨善其身.
更不喜歡獨排眾議.
潮流興我們就做什麼.
變愛興我們就做什麼.
所謂逆流遇上.
無所謂與世相馳.

$^{121}$我以為.
人必須要多.
那你才會坐靶.
你去唐人埠哪間餐館吃東西呢.
你看看那間餐館.
人丁都沒有.
那你就知道不是好東西了.
你選那些旺丁旺財.
那些才是好的嘛.
於是.
我們就會發覺.
我們是跟大眾大隊走.
周圍的人.
喜歡穿什麼名牌的.
那我也穿名牌.
人人吸煙那我就拿支煙來吸.
人人講三字經.
我就講幾句.
人人喜歡吃大麻.
那我又吃大麻.
這個是時代興嘛.
文化.
跟風.
就好像老蒼蠅一樣.
自尋死路.
耶穌很誠實的.
整天到你和我都出奇.
他不是像那些蜘蛛那樣.
花言巧語來騙你.
他不需要很多引誘的說話.
他也不會避開.
他老實告訴你.
你想要跟隨我.
你要想清楚.
人若要跟隨我.
就要寫幾天天背起十字架.
跟隨我.
那你不就不想跟隨我了.
就要寫幾天天背起十字架.
跟隨我.

$^{161}$你看耶穌不是賣什麼貨.
什麼推銷那些.
信耶穌吧.
信耶穌有平安有喜樂.
凡事順利有永生.
有很多朋友.
有很多人會好心幫你的.
牧師也好.
甚至陪你看醫生.
沒錯這些都是事實.
但是耶穌要我們考慮.
要我們想清楚什麼.
你跟隨他的時候要付上代價.
而且是非常重要的代價.
如果你不願意付代價.
根本就不配做他的門徒.
究竟這是什麼代價呢.
這份福音第十五章.
十八至二十五節就告訴你.
如果你要跟隨耶穌.
很可能你會失去.
世界上人認為是最好的東西.
你也有可能被排斥.
被拒絕.
被迫離開.
被排斥被拒絕.
甚至被壓迫.
而且是無緣無故被人恨.
無緣無故被人憎.
我們要考慮.
我們不只是考慮.
跟隨耶穌的代價.
我們更加要考慮.
不跟隨耶穌有什麼代價.
耶穌說.
人如果賺得全世界.
卻賠償了你的生命.
有什麼好處呢.
人就用什麼可以換取你寶貴的生命呢.
你要付的代價.

$^{201}$就像蒼蠅一樣.
自尋死路.
我們詳細來看看.
這段聖經.
究竟我們要跟隨耶穌的時候.
你要付什麼代價.
首先我們來看.
第一個代價.
記在第十八節.
二十節.
這個代價是什麼呢.
當你要立志相信耶穌的時候.
有可能.
會被人恨.
聖經說.
世人若恨你們.
你們知道.
恨你們已先.
已經恨我了.
你們若屬世界.
世界必愛自己的.
只因你們不屬世界.
那是我從世界中.
揀選了你們.
所以世界就恨你們.
你們要紀念我從前對你們所說的話.
僕人不能大於主人.
他們若逼迫了我.
也要逼迫你們.
若遵了我的話.
也必遵守你們的話.
若將這個世界.
分為兩大陣營.
一個是屬耶穌的.
一個是屬世界的.
第十八節.
所說的世人和世界.
其實是同一個字.
希臘文都是cosmos這個字.
這兩個陣營.

$^{241}$屬耶穌的陣營.
和屬世界的陣營.
是水火不容.
若是屬這個世界.
就不屬耶穌.
若是屬耶穌.
就不屬這個世界.
沒有中間路線.
沒有半天調.
不能一隻腳站在耶穌身上.
另一隻腳站在世界.
不能星期日屬耶穌.
星期一至星期六.
屬世界.
不可以的.
可得兼.
一定要寫一取一.
究竟約翰所說的世界.
是什麼意思呢.
難道我們一信了耶穌.
就像那時的出家人.
就要去到深山野嶺.
獨處.
這樣走來走去都可以通.
很明顯.
約翰所說的世界.
並不只是物質的世界.
因為物質的世界都是神所創造的.
他認為最好的.
所以就說.
這時好的.
同時.
這個世界.
也不是很抽象.
很多時候這個世界.
是說世界上的人.
就像約翰福音第三章.
神愛世人.
同一個字來的.
神都愛這個世界.

$^{281}$這個愛是指世上的人.
因為Cosmos.
往往其實有一個道德的含義.
就是那些敵黨神.
不信神的系統.
不信神的人生觀.
不信神的真理.
也不願意跟隨神的人.
對他們來說.
世界和神是敵對的.
就像約翰一書.
告訴我們.
你們若愛這個世界.
愛父的心就不在你們裡面.
或者我們說回.
基督徒所負的代價.
十八字就告訴我們.
恨這個字.
在原文是一個.
Present Continuous Sense.
意思不是一時火滾.
昨晚上班睡晚了.
很累 今天火氣很重.
不是的.
見到有什麼事就大罵一輪.
不是的.
不是像在教會旁邊.
一早就唱詩嘈著.
不是這個恨 是不住的.
上上的 恨是恨基督徒.
但我們要留意.
這裡說到有兩個恨.
世人會恨你們.
但因為.
未恨你們之前.
已經恨了我.
其實在原文只有一個恨字.
不過.
在第二個恨字是一個.
Perfect Tense.

$^{321}$意思是不變.
兩個加起來我們就知道意思.
世人無論在心態上行為上.
都是不住的.
去憎恨基督徒.
由憎恨耶穌到憎恨基督徒.
什麼叫憎恨.
希臘文的字叫Misèl.
是無意無故的憎恨.
有時我都不明白.
為什麼那些人會憎恨基督徒.
在中國.
有不少的宗教士來到中國.
從遠處來到.
犧牲他們的家庭.
犧牲他們的職業 甚至犧牲他們的生命.
但為什麼我們中國人.
反而說他們.
是侵略者呢.
帝國主義者呢.
他們又不是造反的.
他們又對國家.
對中國有好處.
為什麼要將我們抓去坐牢.
受逼迫呢.
有些年青人信耶穌.
行為比以前好一些.
為什麼父母那麼兇.
甚至不准他們回教會呢.
真的很奇怪.
為什麼他們會.
無緣無故的去憎恨基督徒呢.
尤其是在今天.
在美國.
你一信耶穌.
你叫他聖誕快樂.
他已經很不喜歡了.
什麼聖誕快樂.
我不信聖誕快樂.
他以為你信耶穌.

$^{361}$就要用宗教的標準.
用反選的標準.
用各種的語言.
叛家歧親.
尤其是在大眾傳媒.
對教會更加是一個敵對.
今天.
為什麼.
世界上的人會那麼憎恨基督徒呢.
為什麼世界上的人會那麼憎恨基督徒呢.
耶穌說世人不但恨你們.
二十節更加告訴我們.
他們更加要迫害你們.
耶穌說的一句話一點都沒有錯.
你只要回到教會歷史看一看.
你就會看得出.
幾乎每一個世代基督徒都沒有一個人恨面的.
全部是被迫迫的.
早期羅馬尼祿王的時候.
燒了羅馬城.
陷害基督徒說是他放火的.
於是.
整個帝國的人都去憎恨基督徒.
在中國國大的時候.
不少虔誠的教徒被殺地迫迫.
北韓入侵南韓的時候.
不少基督徒被殺地迫迫.
中國共產黨入主中國的成千上萬的信徒.
都迫迫.
皇命倒無緣無故被捕下獄.
事實上我們發覺.
很多基督教領袖.
甚至在今天.
一樣被捕下獄.
好像王怡牧師.
他講福音.
就是被捕下獄.
在美國.
不要以為美國是一個自由國家.
你看今天在美國的社會.

$^{401}$有很多人對基督教.
對宗教.
有一個很強烈的反感.
很多年前.
Columbine High School的屠殺.
其中有兩個學生被槍手殺之前.
槍手問你信不信上帝.
你知不知道是基督徒.
他們說我信.
他就一槍將他打死.
成為一個殉道者.
在今天美國社會.
一樣無論是公司還是社會.
當我們表明我們的信仰.
我們的道德標準的時候.
我們就成為眾的之恥.
忽然之間.
這都是耶穌早已預約.
因為他們憎恨耶穌.
他們憎恨我們.
因為我們不屬於這個世界.
我們不是跟著這個世界路線走.
於是他們覺得我們是.
另外一族.
是敵對的一族.
所以我們看看.
為什麼會憎恨耶穌.
因為基督徒跟隨耶穌的人.
是屬於另外的一族.
我們詳細來看看第十九節.
你們若屬世界.
世界邊外屬自己的.
只是你們不屬世界.
為什麼呢.
很簡單.
因為基督徒不屬這個世界.
是另外一族人.
什麼叫屬這個世界.
原文是說.
從這個世界出來.

$^{441}$Not of the world.
是世界仔來的.
是同一族類的.
同一淵源的.
同一個價值觀的.
同一個老闆的.
同一個撒旦的.
人邊外屬自己的人.
不屬自己的人.
他們不接受就不喜歡.
我以為性關係.
無親無故.
基督徒偏偏說是罪.
我當然不喜歡.
我以為同性戀是我的選擇.
你又說我是罪.
當然不喜歡.
於是我就說你是一個.
Bigotry.
我以為墮胎.
是我的選擇.
是我的權利.
你偏偏說是罪是殺人.
當然不高興.
於是指斥你是迫不得已的婦女.
扼奪婦女的權益.
路不同不相為謀.
你的信仰.
很自然就成為了這個世界人的.
眼中釘.
因為大家的價值觀不同.
這是第一個原因.
第二個原因就是因為我們跟隨耶穌的時候.
他們是恨耶穌.
當然也恨基督徒.
他們迫害耶穌.
當然迫害跟隨耶穌的人.
因為耶穌是從世界上.
揀選了我們.
十九節說.

$^{481}$他揀選脫離這個世界.
不是說叫我們搬到深山野嶺住.
不是.
而是因為我們的價值觀.
我們不喜歡我們拒絕同流合污.
我們的價值觀.
我們的人生觀不同.
耶穌說過一句話.
僕人不能大於主人.
我們都是耶穌的僕人.
他們恨耶穌當然會恨我們.
因為主的名的緣故.
他們恨你.
不是因為你做了壞事.
做了不好的事.
他們這樣做很簡單.
就是因為你信奉主的名.
他們不信耶穌.
也不信任何跟隨耶穌的人.
這是第二個原因.
第三個原因.
因為主的話主的教訓.
他們要遵守我的話.
也要遵守你們的話.
但他們因我的名而向你們行這一切的事.
因為他們不認識那差安來的.
耶穌這樣說.
是說明給他們聽.
不遵守耶穌的話.
就是不遵守我們所傳的話.
這是告訴他們一件很重要的事.
基督徒之所以討人厭.
不是因為什麼.
就是因為他們所信的信仰.
因為他們信耶穌.
他們相信耶穌所給我們的教訓.
其實想深一層可以理解.
因為基督教所說的.
對他們來說很討人厭.
基督教說我們是罪人.

$^{521}$人心比萬物還要詭詐.
邪惡.
那誰喜歡這樣的道理呢.
基督教又告訴你.
人死後落地獄永遠沉淪.
那誰喜歡這樣的教訓呢.
你說我是罪人.
就咒我永遠沉淪.
你說對他們來說.
他們會不會生氣呢.
又怎會喜歡呢.
換句話說.
如果你想討好世人.
很簡單.
拋棄你的信仰.
如果你要講你的信仰.
你不遵守上帝的話.
世人就不喜歡你.
要記住.
你就和上帝無份.
值不值得呢.
今天這個世界就是這樣.
當我們認耶穌的時候.
認真是我們.
因為我們說你是罪人.
你不悔改的時候.
你會落地獄.
對很多人來說.
有沒有搞錯.
你嚇我啊.
不喜歡聽這些話.
於是就頂撞他.
基督徒是甚麼.
如果我們要考慮.
要接受耶穌基督的時候.
你要考慮一樣東西.
不是走世界的路線.
要頂天立地.
跟隨耶穌怎樣.
你看看.

$^{561}$世人真相的角度.
你就會知道.
22-24.
我若沒有來教訓他們.
他們就沒有罪.
但如今他們的罪是無可推諉.
恨我的也是我的負.
我若沒有在你們面前.
中間行過別人.
未當行的事.
他們也沒有罪.
但如今他們的罪是無可推諉.
恨我的也是我的負.
他們也看見也恨不了.
耶穌說甚麼.
如果我沒有來教訓你們.
你們就沒有罪.
甚麼意思.
假如政府從沒公佈.
公速公路.
時速限制是55米.
警察就不能控告你.
說是超速行駛.
因為都沒有說.
但一旦公佈了.
你不可以超過55米.
有理由.
你超過了時可以抄你的牌.
情況就是這樣.
交通警察可以抄你的牌.
主耶穌來了.
祂指出我們的罪性.
如果我們說.
我不知道.
那就可以推諉.
但耶穌未來之前.
先知已經來了.
已經告訴我們我們是罪人.
是沒有理由.
主耶穌不單揭開了我們的罪性.

$^{601}$更是走了別人未曾走的路.
「是」這個字.
在南文叫「erga」.
出現過二十多次.
大多數是指神蹟.
但也包括耶穌在世上所行一切的事.
包括死而復活.
藉著一切來除去我們的罪.
換言之.
耶穌不單揭開了我們的罪.
更為我們開了一條生路.
可惜世界上的人不接受耶穌的教訓.
也不接受祂為我們預備的救恩.
反而恨祂.
甚至將祂釘在十字架上.
又應驗了聖經裡面的寓言.
「他們無故地恨我」.
二十五字.
是沒有理由的.
是罪的問題.
你想想我們未信耶穌的時候.
我們何嘗不是這樣.
嘲笑耶穌.
恨耶穌.
無意無故地憎恨祂.
這只會帶來我們永遠的滅亡.
永遠的沉淪.
弟兄姊妹.
信耶穌是要付上代價的.
因你的信仰.
你可能會失去.
一些所謂的良貧.
你可能會失去一些.
所謂的好處.
甚至被人憎恨.
我認識一個朋友.
在香港做醫生.
每逢星期二.
他就要去地獄.
每逢星期二.

$^{641}$他就要去地方診所出產.
政府說明.
他來回的.
有錢的.
政府可以補助.
來回七里.
但有史以來.
所有同事都是寫上十里.
來賺多三里的苦頭.
基督徒不敢做.
寫十里就寫七里.
他毅然寫上七里.
他知道寫了七里的時候.
一定會被黑人震.
可能會遇到很大的困難.
但他內心充滿了平安.
充滿了喜樂.
我們可不可以這樣做.
在我們前面有兩條路.
一條是闊的路.
世界上人都是這樣走的.
但這條路.
是引致滅亡的.
在我們面前有一條窄的路.
這條路卻是引導榮生的.
今天我們眼光轉了一點.
不是看近時的事.
我們問一問.
當我們要選擇.
跟隨這個世界.
還是跟隨耶穌的時候.
我們千萬不要忘記.
這個世界的後果.
當我們跟隨這個世界的時候.
帶來什麼後果.
計算清楚你的代價.
然後在主耶穌的面前.
做一個毀身和決志.
弟兄姊妹.
你是不是已經決心跟隨耶穌.

$^{681}$還是你仍然在徘徊.
在考慮.
不能夠把心一還.
將你的生命交給他.
求主幫助我們.
讓我們有一個敏銳的眼光.
有聰明的抉擇.
人如果能夠站得直.
就能夠站得穩.
人如果能夠站得直.
就能夠站得穩.
人如果能夠站得直.
就能夠站得穩.
人如果能夠站得全世界.
給賠上他的生命有什麼好處.
人要用什麼東西.
換取他的寶貴的生命.
願上帝祝福我們.
祝我們同生低堂.
我們同土共.
我們知道在我們前面有兩條路.
一條是引導永生.
這一條是艱辛的道路.
一條是引導滅亡.
是很多人走的.
這一條路是引導滅亡.
教導我們.
叫我們從你的眼光來看人生.
你已經為我們釘上十字架.
又給我們生命有價值.
求主幫助我們.
能夠去服侍你.
讓我們能夠成為你的兒女.
帶領更多親朋戚友.
去歸向你.
祝福幫助我們.
祈禱奉耶穌基督名靠.
阿門.
\newpage



\section{約翰福音 15:26-27}
\label{sec:pF1FrHKEPww}
\textbf{認識聖靈 (約翰福音15\_26-27;16\_5-15) - 蘇穎睿牧師 [約翰福音研讀 - 第63講]}
\newline
\newline
連結: \href{https://youtube.com/watch?v=pF1FrHKEPww}{\texttt{ https://youtube.com/watch?v=pF1FrHKEPww}} ~~~~ 語音日期: 2025-01-17 
\newline
\newline
\hyperref[sec:GDV7iT9TooA]{< < < PREV SERMON < < <}
~
\hyperlink{toc}{[返主目錄]}
~
\hyperref[ch:preacher5]{[返講員目錄]}
~
\hyperref[sec:M4alGuubf1o]{> > > NEXT SERMON > > >}
\newline
\newline
約翰福音 15:26-27
\newline
\begin{longtable}{cl}
\hline
\hline
章節 & 經文 (和合本修訂版)\\
\hline
15:26 & \begin{tabularx}{0.7\textwidth}{X} 「但我要從父那裡差保惠師來,就是從父出來的那真理的靈,他來的時候要為我作見證。 \end{tabularx} \\ \\ \relax
15:27 & \begin{tabularx}{0.7\textwidth}{X} 你們也要作見證,因為你們從起初就與我同在。」 \end{tabularx} \\ \\
[1ex]
\hline
\hline
\end{longtable}
$^{1}$各位大英主媒各位朋友你們好.
歡迎大家參加我們三藩市報道會.
網上的五堂崇拜.
今天我們繼續思想.
約翰福音的信息.
我們來到約翰福音第十五章.
二十六節至二十七節.
和十六章五節至十五節.
題目叫做認識聖靈.
在開始思想這段聖經之前.
請你同心低頭和祂禱告.
你賜下聖靈給我們每一個.
束縛你的人.
讓我們明白真理.
教導我們如何跟著你的旨意行.
求你藉著你的話語提醒,勸勉,教導,安慰,福澤我們.
求聖靈向我們講你要所講的話.
我們今天禱告那是奉耶穌基督的名求.
阿們.
我記得還在香港大學讀書的時候.
有一次有機會和宿舍的同學聊天.
有一位非常出色的高材生.
我們無所不談.
不知為何說到要來大學讀書.
這位同學回答得很妙.
我讀書是貪過癮的.
我們問他.
未聽過有人說讀書是貪過癮的.
什麼叫過癮.
他說過癮就是忘記自己就是過癮.
每逢打風下雨.
我自己一個人不帶雨傘.
不帶雨衣就走出去上山.
任憑大雨打在我身上.
他說這是最過癮的.
天地與我為一.
萬化與我吻合.
他又說考試也很過癮.
什麼都忘記.
忘記得一乾二淨.

$^{41}$只顧著考試.
這個故事給我有很大的啟示.
現代人是很尋求經歷的一代.
真理不是重要.
最重要是經歷.
台灣有位很有名的作家.
王尚義說過.
二十世紀是追求經歷的時代.
是反叛理性的時代.
我曾經聽過一個這樣的故事.
一個日本遊客.
去中國旅行.
去到黃山.
看到雲海.
嘩!美啊美啊.
大聲說這個真的好美.
我終於找到這麼美的地方.
說完之後.
跳下懸崖.
連屍首也找不到.
這種尋求經歷的哲學.
充斥在我們整個社會.
有時.
看風景拿不到.
這個經歷.
吸毒,大麻,可樂因.
尋求刺激.
享受,醉酒.
會成為今天的風氣.
其實不單是微信的社會.
教會也如是.
所以我們看到.
二十世紀的明恩運動.
的發生.
絕對不是偶然.
在1901年.
Kansas的Bethel Bible College.
有位叫.
Edward Oxman的.
一位老師開始.

$^{81}$講所謂的謊言.
沉寂了19世紀.
這個古怪的現象.
叫做謊言又興起來了.
1906年在LA.
ASUSA的地方.
也開始這個運動.
五重節教會和.
靈恩運動.
到了1901年之前.
靈恩運動簡直是鳳毛麟角.
早期的靈恩運動.
有Montanus和Tertullian.
這兩位.
叫做早期的領導者.
Montanus本來是一個.
異教徒後來改信基督教.
變了自稱是聖靈的.
發言人.
天國已經臨近他的家鄉.
有兩個女弟子.
也可以講謊言.
教會認定他是異端邪說.
逐出教會.
經過1700年.
一個法國人.
而且能夠說謊言.
如果小朋友不教他講話.
他長大後.
可以講純正的法國話.
教會認為.
這是異端邪說.
到了20世紀.
靈恩運動又開始蓬勃.
而且在福音教會.
和主流教會.
也流行對聖靈的工作.
引起極大的興趣.
通常我們說.
20世紀的靈恩運動.

$^{121}$分為有三波.
三波.
第一波就是20世紀初年.
基督教復興.
例如Adnais Osmond.
就把靈恩運動.
引起了五神教和神教會.
在第一波的產品.
他們重視神意和方言.
第二波.
在60年代.
所謂耶穌運動.
即Jesus Movement.
不少的希臘人.
都是參與的.
主流教會和天主教人士.
一波風投入靈恩運動.
重視經歷.
以聖靈充滿代替大麻.
隨著Counter Cultural Revolution.
的運動蓬勃.
在越戰結束後.
這個運動就開始.
隨風而逝.
到了第三波.
80年代.
由John Wimber.
開始葡萄園.
第三波的特點.
大多數來自福音派的教會.
重視敬拜靈歌.
神意,方言,講鬼,權能報道.
和智慧言語.
所謂智慧言語.
能夠知過去和未來.
這個運動不單在美國.
在香港,南美,英國.
極之流行.
同時也引起不少教會的分裂.
最近有所謂Sling of the Holy Spirit.

$^{161}$即是.
聖靈充滿的時候.
跌倒,滾地,大笑.
甚至嘔吐.
或大哭.
其實這些重視經歷.
見證,醫治.
說到只是15分鐘.
不注意這些所謂經歷.
小組分享.
不重視言經.
以為聖經只是一個地圖.
一個菜擺.
你看著地圖不會去到目的地.
你看著菜擺不會吃飽.
看聖經也不會飽.
是經歷才飽.
聖靈充滿經歷才會吃飽.
和去到目的地.
在跟大家講約翰福音之前.
我一定要強調.
我們的信仰和道德.
最高的標準.
就是神的話語.
聖經.
聖經說.
聖經就是腳前一燈路上的光.
是撥亂反正.
叫我們能夠.
明白真理.
神的啟示.
所以我們信主.
我們覺得他靈.
所以信耶穌.
不是覺得靈.
不靈.
是因為他是我的主.
聖靈在我們心裡感動.
知道我是一個罪人.
回到耶穌基督裡.

$^{201}$狠頭去寬恕.
接受耶穌基督.
做我們個人救主和主宰.
所以我們從神的話語.
去認識上帝.
從神的話語.
去見悟第二條路.
個人的經歷.
不是我們信仰的標準.
所以.
我們今天要去看.
聖靈的工作.
有關聖靈的真理的時候.
不是聽某某人講.
方言又有什麼經歷.
那些不是我們信仰的標準.
最標準是神的話語.
我們試試從約翰福音.
第十五章十六章來看看.
至少我們看到聖靈三方面的工作.
他的見證.
他的審判.
和他的啟示.
我們進一步來看看.
首先我們看看見證的靈.
聖靈又稱為見證的靈.
約翰福音第十五章.
二十六至二十七節.
但我要從父那裡.
差保衛斯來.
就是從父出來的聖靈.
他來了.
就要為我作見證.
你們也要作見證.
因為你們從起頭.
就與我同在.
究竟聖靈是何方神聖?.
一直以來有很多人認為.
福音派的教會比較忽略了.
這一方面的教導.

$^{241}$聖經.
講述聖靈的工作的時候.
譬如在聖經.
耶穌用了兩個名字來形容聖靈.
第一.
聖靈是保衛斯.
希臘文叫Paraklēto.
這是一個幾乎不能翻譯的希臘文.
NIV就譯作Counselor.
輔導者.
RSV就譯作Comforter.
眷惠者.
有些就譯作Helper.
負納者.
中文也譯作保衛斯.
老實說.
保衛不是防衛.
衛是因為.
因為斯.
有時也會拗手一頭.
什麼是因為斯?.
到底是什麼意思呢?.
似乎在中文的字裡.
沒有這個詞句.
天主教的詩稿.
本來就譯作護衛者.
護就是保護.
衛就是勸衛.
中文的西語就譯作.
奉衛斯.
有教訓.
有勸衛.
到底這個字是什麼意思呢?.
原來這個字是由兩個希臘文組成的.
一個叫Para.
Para就是同行平行.
Para有Para這個字.
一個是Kratos.
Kratos是被召.
被呼召.

$^{281}$兩個字合起來就叫做被召的人.
Someone who is called in.
原來在古代的時候.
法庭審訊.
傳召證人作供.
這些證人就叫Para Kratos.
所以這個字可以譯作.
被傳召作證的人.
後來更引申為.
被召幫助別人的人.
特別是那些人有難的時候.
有麻煩的時候.
更加需要有人去.
扶立他們去安慰他們.
所以聖靈.
第一個名字叫保衛斯.
是由一個法庭的圖畫.
被召出來.
做證人.
第二個是真理的靈.
同時聖靈又稱為真理的靈.
所謂真理的靈.
有一個很重要的使命在裡面.
這個使命是什麼.
它是.
傳揚真理的靈.
什麼叫做真理.
耶穌說我道路真理和生命.
所以聖靈的工作.
就是去見證耶穌.
祂不是高舉某個人.
不是很厲害.
聖靈主要的工作.
就是去見證耶穌.
祂的救贖 祂是神的兒子.
祂怎樣救贖我們成為我們的救主.
主宰我們.
還有一件事我們要明白.
聖靈是從父神那裡出來的.
希臘文的字叫.

$^{321}$Apolliumi.
是指從哪裡出來的.
come out from someone.
就是說聖靈是從神那裡來的.
coming out from God.
是三位一體的神.
所以當.
要形容聖靈的時候.
他用一個希臘文叫.
Akhenos 或者Haw.
這個代名詞.
是用於人的身上.
不是物件的身上.
換言之聖靈不是一種精神.
不是一個東西.
不是一樣東西.
有位格的 有感情的.
是三位一體的精神.
three person.
聖靈是一個person.
我們要猜這個聖靈來.
耶穌說我要猜者來.
這個是future tense.
是未來的.
聖靈降臨的時刻.
聖靈降臨在.
正徒的身上.
所以我們看到有聖靈.
在聖經裡.
他是我們的保衛師.
他告訴我們.
他是我們的證人.
為我們作證 勸慰我們.
第二是真理的靈.
透過聖靈.
啟示我們什麼是真.
真理是什麼.
第三.
是從父神那裡來的.
不是自己造出來的.

$^{361}$不是人造出來的.
是從父神那裡來的.
我們又會問 聖靈的工作是怎樣的.
聖經說他來了.
就要為我作見證.
聖靈來是要.
為耶穌作見證.
作見證這個字很有趣.
希臘文叫material.
是作證人 見證一些.
的真理 見證耶穌就是真理.
你想一想耶穌死而復活.
升天.
剩下的門徒繼承耶穌.
將福音傳遍天下.
但是如果不是有.
聖靈幫助我們 有膽量.
有智慧在將福音傳開的時候.
很可能.
福音早已經停頓了.
就好像馬太福音第十章十九.
你們被教的時候.
不要思索怎樣說話.
說什麼話.
到這個時候必次給你們當說的話.
因為不是你們自己說的.
乃是你們父的靈在你們裡頭說的.
聖靈住在我們心裡的時候.
當我們遇到患難.
神就賜給我們應說的話.
這是很重要的.
很多時候我們.
跟別人傳福音.
或者做輔導.
不知道怎樣繼續下去.
不知道怎樣說話.
但聖靈會教導我們.
引領我們怎樣說.
而說出來的時候.
他們會明白.

$^{401}$很奇怪.
神往往給我們合而的說話.
一語道破.
這就是聖靈的工作.
但是聖靈.
不會自己單獨見證.
祂要你和我一起做見證.
二十七節說.
你們也要作見證.
因為你們從頭就與我同在.
所謂見證是見而證之.
所以國內人.
是見過是經歷過上綱之味.
譬如我去過北京.
我可以和你說天壇.
天安門十三陵.
義和團.
我經歷過.
這是很重要的.
一個經歷過癌症的人.
他經歷過的時候.
他可以安慰另一個受癌症的人.
他所做的見證更加有力.
因為他是國內人.
他經歷過.
我們一樣像門徒一樣.
我們經歷過主耶穌基督.
他怎樣為我們死為我們服.
我們怎樣享到平安.
我們可以講出自己親身的見證.
靠他的見證是不行的.
聖靈要在他心裡面.
做工.
一同見證.
這個見證就大有功效.
這叫神人合作.
所以我們看到聖靈第一個.
很重要很重要的.
工作.
見證的靈.

$^{441}$我們再來看看第十六章.
五至十一節審判的靈.
.
你們中間並沒有人問我你往那裡去.
只因我將這事告訴你們.
你們就滿心憂愁.
然而我將真情告訴你們.
我去懸事與你們有約定.
我若不去保衛司就不到你們這裡來.
我若去就差他來.
他既來了又要叫世人為罪為義.
為審判自己責備自己.
為罪是因他們不信我.
為義是因我往那裡去.
他們就不再見我.
為審判是因為這世界的旺壽而要審判.
我們要留意這段聖經.
耶穌在最後晚餐.
這個farewell dinner講這一番說話.
非常有意思.
那時候門徒內心充滿憂慮擔心.
耶穌走了.
我們怎麼辦.
沒有頭怎麼辦.
蛇無頭不幸.
還有他給我們一個重大的責任.
我們怎麼負得起.
所以耶穌對他們說.
我去是為你們好.
因為我在世在肉身上還有限制.
在肉身上在這個世界上.
我們有合有離.
有下有拜.
但我去了之後.
聖靈會在你們心中.
再沒有肉身上的限制.
所以是為你們好.
我去了之後.
就會猜見聖靈在你們心中.
你就再沒有這個限制.

$^{481}$如果我經常在你們心中.
你們就經常想著我.
經常要跟著我.
是有肉身上的限制.
但在聖靈在你們心中的時候.
就再沒有這個限制了.
我們再來看看.
聖靈的公約又是怎樣呢.
因為給我們一個很重要的啟示.
聖靈不單是見證主.
同時也是在人心裡面做責備的工作.
第八次說.
他既來了.
亦是要叫世人為罪.
為義.
為審判.
自己責備自己.
我們要留意.
自己責備自己這個字.
中文在這裡是意譯的.
原文的意思是甚麼呢.
原文的意思是.
當聖靈來到.
他就要Elecho世界.
有關他們的罪.
他們的義.
和他們的審判.
原文是這樣說的.
當聖靈來到的時候.
他就要Elecho這個世界.
Elecho是甚麼呢.
他們的罪.
他們的義.
和他們的審判.
當然了.
如果我們直譯Elecho這個字.
就不知所謂了.
所以中文就要自己責備自己.
亦都相當合理.
這樣可以發揚.

$^{521}$甚麼叫Elecho.
中文要自己責備自己.
在希臘文字裡面.
Elecho這個字.
幾乎是很難翻譯.
這個字有幅圖畫.
是昔日在古希臘的法庭裡面.
法官盤問證人或被告.
直至他不能不承認自己有錯.
所以英文可以做Convict.
Convict即是指定了他.
正因如此.
因為證據確鑿.
不容許否認.
於是他們只能承認.
自己錯了.
自己責備自己.
你想想在法庭上.
被告在播放.
在申辯.
但證據確鑿.
他不能不承認.
所以.
這些證據Elecho了他的罪.
為罪為義為審判.
然後自己責備自己.
為甚麼會責備自己呢.
耶穌提出三件事.
責備甚麼呢.
第一罪.
是因為他們不信王.
當聖靈在一個人心裡動弓時.
很奇怪.
他不再強咀.
不再死頂.
自以為是.
開始看到自己的罪.
自己的不信.
自己的高傲.
變成在神面前.

$^{561}$在人面前悔改形象.
這是一件不可思議的事.
我記得我帶一個查聽班.
有一個大約50多歲的女士來參加.
她是一個很成功的商人.
她第一次來的時候就說.
朋友叫我來.
那我就來吧.
沒所謂.
認識一下就好.
但我告訴你.
我一定不信.
我每樣都好.
我又好人.
我又行善事.
我對家庭又好.
我每樣都好.
我沒有這個需要去信耶穌.
誰知查聽查到第五次的時候.
我們正要第六次開始.
繼續查經之前.
她說我有些事想說.
她說查了這幾次聖經.
看完聖經之後.
我對我自己有很不同的看法.
當我苦心自問.
我發覺.
我內心是充滿了很多罪惡.
我對人很多時候是假的.
不知為何.
我發覺我有很大的改變.
我就說這就是聖靈的工作.
聖靈在我們心裡面動功的時候.
convict our sin.
是指證了我們的罪.
證據確鑿.
見我們誠實去面對自己的時候.
就看到自己的罪.
是的.
我見到很多很多的人.

$^{601}$他們來到耶穌基督面前.
本來是去死頂嘴的.
死覺得自己是很好的人一樣.
但是讀了聖經.
聖靈在我們心裡面工作.
去看到那個廬山真面目.
去看到自己的罪.
唯有聖靈的工作.
才叫人在神面前可以悔改認罪.
每一個基督徒都經歷這個奇異的因典.
是為罪自己責備自己.
因為自己是不信.
聖靈責備的時候.
就看到自己的廬山面目.
不單是為罪.
也為義自己責備自己.
是因為我們去負.
什麼意思呢?.
義者而也.
合語的意思.
希臘文的dikaiosuné.
在登山補品的這個字.
和福音和天國是相通的.
這個字的意思就是說.
這個義是指耶穌基督死和復活.
正如釘耶穌的十字架.
看到耶穌受死的時候.
這個真是一個義人.
耶穌是盡了諸般的義.
在十字架上受死埋葬復活.
回到天府更證明.
耶穌是一個義人.
所以為義受逼迫.
就是為耶穌受逼迫.
為義自己責備自己.
就看到耶穌的義行.
為我們釘死了十字架.
而去自己責備自己.
不單止我們看到自己的罪.
聲明convict of our sin.

$^{641}$更加看到原來耶穌基督的愛.
耶穌基督是一個義人.
為我們釘死了十字架.
就好像白夫長說的.
這個真是一個義人.
不單止知罪.
更加認識耶穌是為我們死.
為我們復活.
第三.
更加是為審判自己責備自己.
因為這個世界的皇守都審判.
世界的皇守是指撒旦.
也是一切不相信的人.
他們一切不相信的人都在頭頭.
有一天所有不信的都被定罪.
都要審判.
聲明是叫我們知道不信的時候.
就有這樣的結果.
或者說我就不怕了.
靠嚇.
我是不怕的.
你想一想.
我舉一個很簡單的例子.
你要看醫生.
你有兩種病.
一種就是胃痛.
一種就是香港腳.
又癢又痛.
醫生給了你兩種藥.
一種是為醫治胃痛.
紅色藥水.
白色藥水是用來塗腳的.
記住.
是外服的.
是用來塗香港腳的.
不可以用來吃的.
吃了有毒的.
會死的.
醫生為了小心起見.
特別叮囑.

$^{681}$並且在藥水面前.
弄一個骷髏頭在那裡.
寄食.
世上沒有一個病人說.
不如醫生靠一下.
我吃給你看看.
有什麼所謂.
你靠嚇.
我不聽.
我就不受嚇的.
不是的.
世上沒有一個這麼蠢的人.
你開車的時候.
前面寫著.
不要再前面.
那根尖頭斷了.
你不要再去.
靠一下.
我偏偏要去.
世上沒有一個這麼蠢的人.
如果神不告訴我們.
將來有審判的時候.
這個就完全沒有愛的神.
是不是.
上天是告訴我們.
將來是有審判.
你想一想.
當我們聽到呼應的時候.
我們看到我們自己個人的罪.
我們看到耶穌基督是十字架的義.
將來不信耶穌基督的審判.
於是我們只能在上帝面前.
謙卑自己.
願意降服在神面前.
這個就是信主.
這個就是神蹟.
好像是違背人的天性一樣.
但是聖靈的工作.
就是一個這麼大的神蹟.
聖靈叫人為罪.

$^{721}$為義.
為審判.
自己責備自己.
最後我們看看第三樣.
這個是啟示的靈.
十二節至十五節.
聖靈第三樣工作就是啟示.
第十三節.
只等聖真理的聖靈來了.
祂要引導你們明白一切的真理.
這是一個很重要的真理.
聖靈是幫助我們去明白聖經.
聖經不是一幅地圖.
聖經不是一些知識給你的.
不是的.
聖經是一把很鋒利的劍.
亦刺透人心.
當你看聖經.
當你讀聖經.
當你聽聖經的時候.
聖靈在你心裡動功的時候.
你會覺得紮心.
你會看到自己的廬山面目.
你會看到上帝的愛.
你會受感動.
然後你會明白這個真理貼在身上.
聖靈是啟動我們明白聖經.
明白這個真理.
讓我們能夠進入真理.
在下一次我們講.
第十六章十二至十五節的時候.
我們會詳細講一講.
聖靈怎樣去啟示這個真理給我們認識.
這是一個很重要的聖靈工作.
聖靈是叫我為罪為義為審判自己.
責備自己.
聖靈是一個見證的靈.
為耶穌基督做了見證.
看到他在十字架的犧牲.
以致我們能夠相信.

$^{761}$聖靈亦是一個啟示.
要我們明白真理.
求主幫助我們.
請我們再一次同心低頭.
不得同禱告.
天父再一次多謝你.
讓聖靈在我們心裡面動功.
讓我們明白到我們的廬山面目.
又讓我們明白到耶穌基督.
他怎樣在十字架成就了這個義.
更加讓我們明白到將來.
如果我們離棄你的時候.
會遭受這個審判.
你又啟示我們真理是甚麼.
求你繼續在我們心裡面動功.
讓我們能夠在你面前願意學福利.
祈禱奉耶穌基督名告.
阿門.
(影片即將開始,別忘了按讚分享喔).
(字幕由AI產生,別忘了按讚分享喔).
\newpage



\section{約翰福音 16:12-16}
\label{sec:M4alGuubf1o}
\textbf{啟示與真理(約翰福音16\_12-16) - 蘇穎睿牧師 [約翰福音研讀 - 第64講]}
\newline
\newline
連結: \href{https://youtube.com/watch?v=M4alGuubf1o}{\texttt{ https://youtube.com/watch?v=M4alGuubf1o}} ~~~~ 語音日期: 2025-01-28 
\newline
\newline
\hyperref[sec:pF1FrHKEPww]{< < < PREV SERMON < < <}
~
\hyperlink{toc}{[返主目錄]}
~
\hyperref[ch:preacher5]{[返講員目錄]}
~
\hyperref[sec:HaGDtN4u47U]{> > > NEXT SERMON > > >}
\newline
\newline
約翰福音 16:12-16
\newline
\begin{longtable}{cl}
\hline
\hline
章節 & 經文 (和合本修訂版)\\
\hline
16:12 & \begin{tabularx}{0.7\textwidth}{X} 「我還有好些事要告訴你們,但你們現在擔當不了。 \end{tabularx} \\ \\ \relax
16:13 & \begin{tabularx}{0.7\textwidth}{X} 但真理的靈來的時候,他要引導你們進入一切真理。因為他不是憑著自己說的,而是把他所聽見的都說出來,並且要把將要來的事向你們傳達。 \end{tabularx} \\ \\ \relax
16:14 & \begin{tabularx}{0.7\textwidth}{X} 他要榮耀我,因為他要把從我領受的向你們傳達。 \end{tabularx} \\ \\ \relax
16:15 & \begin{tabularx}{0.7\textwidth}{X} 凡父所有的都是我的,所以我說,他要把從我領受的向你們傳達。」 \end{tabularx} \\ \\ \relax
16:16 & \begin{tabularx}{0.7\textwidth}{X} 「不久,你們將不再見到我;再過不久,你們還要見到我。」 \end{tabularx} \\ \\
[1ex]
\hline
\hline
\end{longtable}
$^{1}$各位大型節目的各位朋友你們好.
歡迎大家參加我們三藩市報道會.
網上的午堂崇拜.
今天我們繼續思想約翰福音的信息.
我們來到約翰福音第十六章.
十二節至十五節.
題目叫做啟示與真理.
在開始思想這條聖經之前.
請我們同心低頭或同禱告.
你藉著聖經向我們啟示.
你自己的創造和你的救贖.
你更加拆顯聖靈在我們心裡.
引導我們進入真理.
求聖靈在我們個人心裡動功.
藉著你的話語去提醒,勸勉,教導.
安慰,督責我們.
我們這樣禱告.
是奉耶穌基督的名求.
葡萄園教會是一個靈恩派的教會.
創始人叫John Wimber.
他開始這個教會的時候.
是一個非常有趣的經歷.
在1978年.
在Uberlinda的Calvary Chapel教會裡面.
是一個牧會.
到了1981年母親節那天.
他們教會請了一個年輕的牧師來正道.
誰知這個牧師一上台.
就大聲地說出方言來.
台下的人竟然一邊應聲起舞.
一邊說方言.
一邊大聲呼喊.
用John Wimber他自己的話.
那時看上去就像戰場一樣.
很多人躺在地上滾來滾去.
好像很緊張很刺激那樣.
很不習慣.
很不舒服.
但見到會友很投入.
連他自己的太太也是一樣.

$^{41}$她當時是做私琴的.
她不知道在做什麼.
於是就繼續用猛烈的彈琴去作興.
過了一晚.
她沒辦法入睡.
整晚希望讀聖經.
從聖經裡面找到答案.
究竟這些是聖靈的工作還是邪靈的工作.
但看來看去都得不到答案.
到早上五點左右.
她已經筋疲力倦.
非常苦惱.
突然在這時候電話響了.
對方有一個陌生的男子.
只說了一句話.
是我.
就斷了線.
John Wimber覺得這個電話很古怪.
但他肯定這個是神打給他的.
向他特別的啟示.
證明這個是聖靈的工作.
是神打電話給他.
作為一個啟示.
就因為這個緣故.
他就開始了第三波的靈因運動.
開創了葡萄園教會.
我看完這個故事之後.
我不禁會問.
把整個信仰.
建基在一個這麼古怪的經歷裡面.
是不是太過牙醫呢.
這個不是我作出來的故事.
如果大家有機會可以看看.
John Wimber寫的書.
Power Encounter.
他親自所作的見證.
同葡萄園.
同一個信仰.
有密切關係的有所謂.
Kansas City的先知.

$^{81}$在1984年.
有一個叫Mike Bickel的人創立的.
除了Mike Bickel之外.
還有Paul King.
John Paul Jackson.
Bob Jones.
通常這些人稱為Kansas City Prophet.
在1990年加入了葡萄園的組織.
他們每個人都以為自己是先知.
神對他們有特別的啟示.
即是他們所說的東西.
他們所說的語言.
就是神的話語.
其中有些很古靈精怪.
比方來說.
如果你想分辨哪些是真正的聖靈的工作.
哪些是假的邪靈的工作.
你只要吃牛肉就能分辨出來.
吃牛肉可以分辨真假聖靈?.
而這些所謂的先知.
大多數都沒有受過神學訓練.
Vera的發言人Jack Deer.
曾經在Dallas神樂院教書.
他公開的說法.
這些先知所說的有些古怪.
但不等於他們所說的不是真的.
我想到其中一個叫Paul King的人.
他的故事更加古怪.
他母親在44歲時患上癌症.
有一天有天使向他啟示.
你不用怕.
你的病不止於死.
而且你還會懷孕生子.
當你生了兒子時.
就改名叫Paul.
他就像是士多保羅一樣.
他母親聽到這個啟示.
果然之間他的病就康復了.
又生了一個嬰兒.
所以就生了Paul King.

$^{121}$他一直活到60歲.
而Paul King本人屢次從天使口中得到特別啟示.
在1950年的時候.
有天使對他說.
喂 上帝說.
祂妒忌你這個美妻.
祂吩咐你跟她斷絕來往.
因為上帝覺得你太太很美.
他就決定跟每個美妻分手.
自己終生不娶.
這樣都可以?.
這位Paul King.
又從天使中得到很多啟示.
他說夏娃的罪是什麼呢?.
因為他跟蛇交配.
所謂三位一體.
不是真理來的.
是污鬼的道理.
總之他所說的話.
是非常古怪的.
其實.
特別是第三波的靈恩派.
最大的問題是有關啟示的問題.
他們以為上帝仍然是說話的.
是透過特別有恩賜的人向他說話.
特別有恩賜的人.
他們有一種恩賜叫智慧言語.
所謂智慧言語是說他們能夠知過去未來.
他們所說的話不是他們自己說的.
是神直接向他們說.
是聖靈直接向他們說.
Peter Wehner是其中一個代表.
他說過自己的一些經歷.
有一次他在芝加哥坐飛機的時候.
同機的有一個男人.
他突然間見到他額頭有一個Jane.
這個女人的名字.
於是他對這個男人說.
我有特別的啟示.
聖靈告訴你你犯了奸淫.

$^{161}$你的情婦叫做Jane.
這個男人聽了之後就惡言.
我想任何一個人聽到這樣說的時候.
如果是真的就嚇得魂不附體.
如果是假的就氣得爆炸.
但我想告訴大家.
這種理論這種神學是錯的.
而且是非常危險的.
如果有人告訴你.
神啟示你要嫁給我.
你嫁不嫁?.
有人告訴你神啟示你要把房子給我.
你給不給?.
如果每個人都說他得到信.
他啟示.
那麼這就是滿天神佛?.
究竟誰是真誰是假?.
聖經告訴你.
神已經在聖經向我們啟示了.
神今天不會直接向人啟示.
如果不是的話.
我那本聖經來幹什麼?.
那本聖經一直加...加....
加多不知多少萬章下去了.
是不是?.
那個啟示已經完結了.
聖靈是透過聖經向我們講說話.
而不是直接向我們講說話.
不是直接向我們啟示.
聖經是有正典已經完結了.
我們不能夠歪曲這個真理.
靈因派最大的問題.
以為是神不絕地向我們講說話.
不只是聖經.
你聽某個人所講的說話.
因為他有特別的啟示.
這個說法是大錯特錯.
今天我們試試從約翰福音第十六章.
十二至十六節來看看啟示和真理這個課題.
首先我們看看第十二節.

$^{201}$啟示是漸進的.
第十二節說.
我還有好些事要告訴你們.
但你們現在擔當不了.
耶穌對門徒說.
我還有很多重要的事情要告訴你們.
去教導你們.
但現在你們擔當不了.
什麼叫擔當不了呢?.
原來這是一個非常有趣的字.
希臘文叫Bastazon.
這個字是解作背負,抬,舉高.
或者擔東西的時候.
拿個肩膀來擔著東西.
為什麼門徒不能擔當呢?.
為什麼門徒不能夠明白呢?.
耶穌特別強調現在.
他們現在不能了解所有真理.
什麼時候他們會了解呢?.
就是當聖靈降臨到他們身上的時候.
聖靈幫助他們.
打開他們的心竅.
他們就會明白了.
因為他們看到了一個非常重要的真理.
神的啟示是漸進式的.
Progressive的.
神並不是一下子將所有真理都曉喻我們.
就好像希伯來書第一章第一節.
到第二節告訴你.
神既然在古時就在眾先知多慈多方的曉喻.
列祖.
就在這末世的時候就在他兒子曉喻我們.
因為告訴你.
兩個很重要的真理.
在舊約的時代.
神是藉著先知和祭司.
多慈多方向我們啟示.
但是到了新約末世的時候.
最後在末世的時候.
他就藉著耶穌降生在這個世界.

$^{241}$聖靈降臨的時候.
最後的曉喻我們.
這個才叫做完成了.
這個啟示是一個過程.
是一個漸進式的過程.
我們所謂的過程.
並不是指今天的啟示第一屆.
第二天明天就啟示第二屆.
後天就啟示第三屆.
或者新約的時候.
今天就啟示羅馬書第一章.
第二天就啟示第二章.
不是.
這個漸進是什麼意思.
Progressive是什麼意思.
神的啟示就好像一棵樹.
開始的時候是一棵小樹.
很小的樹.
但已經有葉有根有樹幹.
它一路長大一路長大.
長到一棵很成長的樹.
漸進式的啟示.
不是說先啟示給你有葉.
先啟示給你有根.
接著啟示給你有樹幹.
不是.
也不是像我們看電腦的時候.
先彈個頭出來.
接著彈個手出來.
接著彈個身出來.
接著彈個腳出來.
不是.
所謂漸進是.
一開始是一個BB.
BB有眼有口有鼻.
但他只是BB.
幾歲大了.
接著十幾歲大了.
廿幾歲漸漸成為一個成人.
這種所謂漸進式啟示.

$^{281}$就是聖經所說的接進式.
聖靈降臨的時候提醒我們真理.
這個真理在聖經裡面.
就完成了啟示的工作.
所以啟示是漸進的.
第二啟示是全部的.
第十三字.
只等真理的聖靈來了.
祂要引導你們明白進入一切的真理.
因為祂不是憑自己說的.
乃是把祂所聽見的都說出來.
並要把將來的事告訴你們.
第二點有關神向我們啟示的真理.
祂不單只是漸進式.
從舊約到新約.
有四十個不同的作者.
漸進的啟示我們.
第二個更加重要的.
是全部的.
是完全的.
不是一塊一塊的.
正如我剛才所說.
最初的時候BB都是完全的.
然後一路一路地長大.
到真理聖靈來的時候.
祂就會引導我們明白進入一切的真理.
引導這個字是一個很有趣的字.
叫Odegeo.
就好像一個導遊.
引導你去這裡那裡那裡.
讓你明白.
我記得很多年前.
我們有機會去倫敦的時候.
倫敦的地下鐵就四通八達.
好像八陣銅一樣.
亂了大龍.
再加上我們年紀大了.
眼睛不是很好.
那些地圖又小又小.
很難看.

$^{321}$看不到.
幸好我的女兒和我們同行.
她在倫敦住了幾天.
她知道了交通工具的情況.
她就做了我們的導遊.
引導我們左穿右插.
不會迷路.
這個就叫Odegeo這個字.
聖靈就引導我們進入真理.
讓我們在真理裡面明白.
上帝向我們啟示什麼.
而且這個啟示是一切的真理.
全部的真理.
不是片面的真理.
不是只是看到頭體的腳.
和像核子摸著一樣.
摸到這一塊就說什麼.
不是的.
是看到整個的真理.
是完全的.
是全部的.
是一切的.
包括以前的事.
現在的事.
將來的事.
聖靈都會引導我們去明白.
並且寫在聖經裡面.
是一切和全部的.
這個是我們基督教信仰.
特別是基督教信仰.
特別要強調的真理.
聖經.
全部的聖經.
是神的啟示.
是完全的.
不需要你加一點.
也不需要你減一點.
我們和一些新神學派說.
不是.
聖經是有些是真理.

$^{361}$聖經只是含有一些是真理.
錯.
我們相信全部的聖經都是真理.
我們也不是像天主教說.
聖經不是完全的.
有些東西是沒有涉及過的.
需要教皇告訴我們.
我們就要加多一些.
不是.
聖經很強調告訴我們.
它就是真理.
不是需要你解釋才是真理.
天主教又說.
教皇所解釋才是真理.
What the pope says, God says.
但我們相信.
What the bible says, God says.
聖經都是神所密使的.
於教訓督察使人歸正.
教導人學而都是有益的.
要熟成才能完全預備行各種善事.
所以我們看到一件事.
沒錯.
神的啟示是漸進的.
但這個漸進不是一分一分.
才有頭才有手才有腳.
是有一個嬰兒一直這樣.
然後有一個少年.
還有一個青年一直這樣.
這樣漸進.
第二.
神的啟示是全部份的.
但越來越清楚.
全部份的.
全部的真理不是片面的真理.
我們不需要去加多.
也不需要去減少.
這是聖經所說的.
全部份一切的真理在裡面.
我們看第三個.

$^{401}$啟示是從神那裡來的.
不是從人那裡來的.
第十三字.
啟示不單止是漸進.
不單止是全部.
而且是從神那裡來的.
不是人那裡來的.
不是亂來的.
第十三字.
因為他不是憑自己說的.
乃是把他所聽見的都說出來.
並要把將來的事都告訴我們.
神啟示門徒.
啟示門徒.
藉著這些寫聖經的人.
將真理寫在聖經裡面.
沒錯.
是人寫的.
是人的言語.
但聖靈一直引導他的時候.
以致他寫出來的每一句.
每一個都不會錯.
都是來自神的話.
神的啟示不是像Dictation那樣.
我那你就寫我.
一切就一切.
阿神.
愛世人.
不是這樣的.
當聖經的作者寫他的聖經的時候.
他用他自己的文筆.
他用他的文化.
用他自己的思想去表達出來.
寫的過程和寫成出來的時候.
聖靈一直控制著他們.
以致他成為了上帝的官管子.
將啟示告訴我們.
正因為這樣.
我們強調.
最終來說.

$^{441}$聖經裡面所講的真理.
是來自神.
所以聖經就說.
聖經都是神所默示的.
默示的意思不是默書.
是啟示我們.
沒錯.
是有作者的言語文化背景.
但另一方面.
聖靈一直引導他.
以致他所寫出來的都是真理.
這個真理是不容許我們去否定的.
是上帝告訴我們的.
是我們的信仰和行為的標準.
在這個世界裡面五花八門.
有人說同船聯好.
同船聯不好.
有人說度多對.
度多不對.
有人說這樣可以.
這樣可以.
究竟黑白是非的標準在哪裡.
聖經告訴你.
就是聖經的道理.
神就是善的標準.
神就是告訴你.
他給我們的標準.
那個才是唯一的標準.
聖經叫我們拒絕罪惡.
但是接納罪人.
好我們再來看看第四樣.
啟示的中心是耶穌.
我們看十四字至十五字.
他要榮耀我.
因為他要將授予我的告訴你們.
凡負所有的都是我的.
所以我說.
他要將授予我的告訴你們.
聖靈啟示.
我們的真理的中心和目的.

$^{481}$都是以耶穌為主的.
聖經不是說一些道德的道理.
聖經不是說一些歷史.
聖經的中心是指耶穌.
所以耶穌說.
凡是舊約的律法書,先知書,詩篇.
都是指著他來說的.
是要記載耶穌的救贖.
是要榮耀他.
我們留意耶穌的一句話.
他就是聖靈.
要將授予我的告訴你們.
什麼叫授予我的.
簡言之就是屬於耶穌的.
一切有關耶穌的真理.
他的降生,救贖,他的再來.
都告訴我們.
所以有人問聖經的主題是什麼.
是關於耶穌.
正如路加福24:44.
「這就是我從前與你們同在的時候.
所告訴你們的話.
摩西的律法,先知的書.
和詩篇上所記載.
都是指著我必須要應驗」.
聖經一切的話.
無論是新的聖經還是舊的聖經.
都是指著耶穌基督來說的.
所以我們可以做一個這樣的結論.
啟示的中心就是耶穌.
聖經所說的中心就是耶穌.
在引言的時候.
所提到的Paul King的啟示.
是亂七八糟,一片胡言.
就因為他所說的不是說聖定.
不是說耶穌.
最後我們看啟示.
一方面是已經完成了.
但另一方面又未曾完全的完成.
我們叫做Already but not yet.

$^{521}$十六字.
等不多時你們就不得見我.
再等不多時你們還要見我.
究竟是什麼意思呢.
一時說有時見到我.
一時說見不到我.
究竟是什麼意思呢.
解經間有兩種不同的看法.
第一種看法就是過了不多時.
指耶穌是死的.
因為耶穌死了你們不再見他.
但三日之後復活了.
所以他們又見到他.
因為已經預言了耶穌死和復活.
所以耶穌說等不多時.
也就是說不久.
你們不得見我.
因為我會被釘十字架.
再等不多時你們還要見我.
因為三日之後會復活.
所以意思就是說.
我會死會復活.
有另一種看法.
就是英國的學者.
C.K. Barrett所說的.
他說.
見到他.
不是指耶穌死.
而是指耶穌死.
復活和升天.
所以見不到他.
再見到他.
不是指復活.
是指耶穌基督的再來.
我個人贊成C.K. Barrett的看法.
今天神已經將他的救贖計劃.
救恩告訴我們.
神已經啟示給我們.
但不要忘記.
這個字是已經的.

$^{561}$還有還沒有的一部分.
所以保羅在.
哥林多前書13章說.
如今我們面對鏡子.
無惡不清.
但到那一天.
面對面見神的時候.
我們就看得明.
今天我們還有很多東西不明白.
今天還有很多奧秘.
我們不完全了解.
但到主耶穌再次.
面對面的時候.
那個Not Yet 的部分.
已經完成了.
所以神的啟示.
是分開.
Already but Not Yet.
今天我們所需要.
有關救贖的知識.
已經有了.
但還有很多東西我們不明白.
將來會是怎樣.
我們不是完全明白.
但是.
我們會有很多東西.
但是.
我們知道有一天.
我們見面的時候.
我們就會完全了解.
我們明白了啟示的真理.
我們就更加覺得.
我們需要去讀聖經.
今天有很多很多基督徒.
只是想著要經歷一些東西.
經歷一些東西.
尋求一些經歷.
是很危險的.
不是說經歷不重要.
但是你一定要有一個基礎.

$^{601}$是有神的話語做基礎.
沒有神的話語做基礎的時候.
很多時候會錯誤.
所以為什麼聖經.
是這麼重要.
為什麼讀聖經這麼重要.
為什麼教會裡面強調.
神的話語就是我們整個教會的基礎.
今天我們花了多少時間.
去研讀聖經.
教會裡面有主日學.
教會裡面有查劍班.
是讓我們明白到.
聖經是什麼東西.
很奇怪.
當有人看聖經.
天秘在聖經裡面.
找出真理的時候.
他的生命就會改變.
好我們今天講到這裡.
請我們再一次同心低頭.
我們同禱告.
正是你的話語提醒我們.
因為你的話語就是腳前的燈路上的光.
讓我們明白到你的救恩是什麼.
讓我們明白到我們在世上.
怎樣活得有意識.
讓我們知道.
還有很多東西我們不明白.
到過一天的時候.
你就將你的真理完全顯明.
我們今晚祈禱.
乃是奉耶穌基督命求.
阿們.
\newpage



\section{約翰福音 16:16-24}
\label{sec:HaGDtN4u47U}
\textbf{那等候的日子 (約翰福音16\_16-24) - 蘇穎睿牧師 [約翰福音研讀 - 第65講]}
\newline
\newline
連結: \href{https://youtube.com/watch?v=HaGDtN4u47U}{\texttt{ https://youtube.com/watch?v=HaGDtN4u47U}} ~~~~ 語音日期: 2025-02-02 
\newline
\newline
\hyperref[sec:M4alGuubf1o]{< < < PREV SERMON < < <}
~
\hyperlink{toc}{[返主目錄]}
~
\hyperref[ch:preacher5]{[返講員目錄]}
~
\hyperref[sec:fV_h6TniAkc]{> > > NEXT SERMON > > >}
\newline
\newline
約翰福音 16:16-24
\newline
\begin{longtable}{cl}
\hline
\hline
章節 & 經文 (和合本修訂版)\\
\hline
16:16 & \begin{tabularx}{0.7\textwidth}{X} 「不久,你們將不再見到我;再過不久,你們還要見到我。」 \end{tabularx} \\ \\ \relax
16:17 & \begin{tabularx}{0.7\textwidth}{X} 有幾個門徒彼此說:「他對我們說『不久,你們將不再見到我;再過不久,你們還要見到我』;又說『因我到父那裡去』。這是甚麼意思呢?」 \end{tabularx} \\ \\ \relax
16:18 & \begin{tabularx}{0.7\textwidth}{X} 於是門徒說:「他說『不久』到底是甚麼意思呢?我們不明白他說甚麼。」 \end{tabularx} \\ \\ \relax
16:19 & \begin{tabularx}{0.7\textwidth}{X} 耶穌看出他們要問他,就對他們說:「我說『不久,你們將不再見到我;再過不久,你們還要見到我』,你們為這話彼此詢問嗎? \end{tabularx} \\ \\ \relax
16:20 & \begin{tabularx}{0.7\textwidth}{X} 我實實在在地告訴你們,你們將要痛哭,哀號,世人反要歡喜。你們將要憂愁,然而你們的憂愁要變成喜樂。 \end{tabularx} \\ \\ \relax
16:21 & \begin{tabularx}{0.7\textwidth}{X} 婦人生產的時候會憂愁,因為她的時候到了;但孩子一生出來,就不再記得那痛苦了,因為歡喜有一個人生在世上了。 \end{tabularx} \\ \\ \relax
16:22 & \begin{tabularx}{0.7\textwidth}{X} 你們現在也是憂愁,但我要再見到你們,你們的心就會有喜樂了;這喜樂沒有人能奪去。 \end{tabularx} \\ \\ \relax
16:23 & \begin{tabularx}{0.7\textwidth}{X} 到那日,你們甚麼也不會問我了。我實實在在地告訴你們,你們奉我的名無論向父求甚麼,他會賜給你們。 \end{tabularx} \\ \\ \relax
16:24 & \begin{tabularx}{0.7\textwidth}{X} 直到現在,你們沒有奉我的名求甚麼,如今你們求就必得著,使你們的喜樂得以滿足。」 \end{tabularx} \\ \\
[1ex]
\hline
\hline
\end{longtable}
$^{1}$各位弟兄姊妹 各位朋友 你們好.
歡迎大家參加我們三藩市報道會.
網上五堂崇拜.
我們今天繼續思想要求福音的訊息.
我們來到要求福音第十六章.
十六節到二十四節.
題目叫做「那等何的日子」.
在思想這段聖經之前.
請同心低頭地讀草稿.
在這個世界裡.
雖然有很多我們不能預料的事.
會發生在我們身上.
但我們知道是誰帶領著我們.
在等候的日子裡.
很多時候我們驚惶恐懼.
但有一直引領著我們.
給我們平安.
求你對著你的話語提醒.
渲憐安慰 督責 改變我們.
我們禱告 奈何耶穌基督命求 阿們.
首先大家去想一想.
你發覺肚子痛.
醫生懷疑你患上肝癌.
要做抽血和切片檢查.
一個星期後才有報告.
你試想想.
這個星期後報告的時候.
你的心情會是怎樣呢?.
又或者.
舉另外一個例子.
你半夜大約一點多鐘就醒了.
你去看看你十八歲女兒的房間.
你發覺空了.
原來你女兒還沒回家.
她曾經告訴你.
她去同學家玩.
十一點鐘就會回來.
你因為累.
十點鐘就上床睡覺.
誰知到一點半起床的時候.

$^{41}$去女兒的房間看看.
你發覺她還沒回來.
你又不知道同學在哪裡.
又不知道她叫甚麼名字.
她又沒有手提電話.
你心裡想到最壞的事情.
會不會是交通意外呢?.
會不會是失蹤呢?.
會不會被人拐帶呢?.
你猜你還睡不著呢?.
在等候期間你的心情又如何呢?.
又或者你想一想.
你申請工作.
三個月都沒有回音.
你家裡呆呆等著.
看著電話聲音也不相下.
你的心情又會如何呢?.
等候我相信是近代人最難學的功課.
在美國有一個有趣的故事.
有一個人向上帝祈禱說.
神啊求你給我忍耐.
我實在很需要忍耐.
因為我太沒有忍耐了.
但我要立即得到.
我不能夠等了.
這個快餐的時代.
我們的字典裡已經沒有等.
或者忍耐這些字眼了.
我們講效率.
在電腦時代.
以前我們講什麼286.
接著是plenty.
現在越整越快.
市面上有很多快速的東西.
fast lane.
fast food, fast pass.
什麼都是立即的.
instant noodle, instant this, instant that.
即食麵,即食粥.
即食芝麻撲,即食紅豆冰.

$^{81}$什麼都即食的.
除此之外還有一個叫slim fast.
三天之內就可以減10磅.
如果你再試多幾次.
還有slim fast.
還有fast fast.
前陣子.
太太在西斐.
買了一些美味的葡萄籽.
但奇怪沒有人吃.
沒有人去吃.
直至發霉.
後來她再買一些顏色淺的葡萄籽回來.
一頓就吃光了.
分別何在.
不是顏色.
不是味道.
分別在.
之前買的有核.
後來買的沒有核.
我們都不喜歡吃有核的葡萄籽.
太麻煩.
太久沒有興致去剝.
沒有核的葡萄籽.
好得多.
一吃下去乾乾脆脆.
所以.
吃豆腐就好過吃骨頭.
事實上我們.
在這個世代裡.
是一個即食世界.
高速公路的文化.
根本不知道等是什麼.
而且我們已經不想等.
我們要求是立即滿足.
但另一方面.
等又不是容易.
我們求婚的時候寫了一封信.
給我現在的太太.
表示愛意.

$^{121}$等候回覆期間一生難忘.
什麼都不能做.
只能等.
因為主權不在你的手裡.
你只能等候.
我回覆.
多利山第一個小孩的時候.
太太懷孕十個星期.
一日吃飯樓下有一個太太.
就走來問.
Doris你以前有沒有患過德國疹.
我追問完.
為什麼你會問呢.
原來她說她的嬰兒染了德國疹.
前一日.
就嬰兒試過她.
她就知道她有了新寄.
如果一個有了新寄的爸媽.
如果染了德國疹.
對嬰兒影響很大.
我們就去看醫生.
怕會染到.
醫生說要等兩個星期才知道.
才檢驗.
這兩個星期真是度日如年.
永世難忘.
結果醫生說沒有染.
一生都送了.
德國疹就好像死跟著我們一樣.
1976年我們回香港侍奉.
太太又再懷孕.
八個星期之後.
這次糟糕了.
真的染了德國疹.
這次不等是兩個星期.
要等八個月.
醫生說沒有藥吃.
唯一能做的就是等.
看她生出來是怎樣.
生出來可能會.

$^{161}$心漏 盲 膿.
或者是腦障礙.
要等八個月.
這八個月什麼都不能做.
就在等 或者是焦慮.
真是度日如年.
回想起那段時間.
相對無言 唯有淚千行.
兩夫妻經常這樣.
那種無助焦慮的感受.
盡不在言中.
等候後似乎.
是一個沒有忍耐 沒有安全感.
沒有信心的人.
是一個很難學的功課.
等就好像空中飛人.
你甩了一個鞦韆.
但另一個鞦韆還沒到.
什麼都不能做.
唯一能做的就是等.
對一般來說.
等是不安全的.
等候期間不能做什麼.
增加自己的安全感.
也不能保護自己.
我們不喜歡等.
我們不喜歡.
要等才有的東西.
所以.
萬子說過鴨寮座長的故事.
有個農夫.
每天見到不生長的就把他拉高.
結果就全部死了.
沒有好結果.
當我們回到聖經來看.
在聖經裡等這個字是非常突出.
和重要的一個字.
在新約聖經裡.
這個字出現過95次.
而舊約呢.

$^{201}$特別在詩篇.
我猜是千年過一千次.
你想想一個人.
說95次或是說千次.
你覺得這個重要嗎.
我發現.
無論是希伯來文還是希臘文.
有很多字可以做等.
有時可以做持續.
有時可以做期望.
有時可以做堅持.
有時可以做忍耐.
在原來在聖經裡等.
絕對不是白等.
是有積極的.
不是負面的.
是積極和正面.
當耶穌升天的時候.
他沒有立即降下聖靈.
不是的.
他要門徒等40天.
聖靈才降臨.
在這40天裡.
等候聖靈降臨.
等候期間.
他首次說同心合意的禱告.
就像先知耳上說.
等候耶和華必從心得力.
絕對不是消極.
一個懂得等的人.
是一個智慧的人.
一個不懂得等的人是危險的人.
像開車一樣.
遇上紅燈不懂得等就衝過去.
這樣就失敗了.
人生裡有很多綠燈.
但人生裡有很多紅燈.
如果沒有優子乎的時候.
人生裡唱完就不是一個好的音樂.
有時要停頓一下.

$^{241}$等是得力的傳語.
詩篇第62篇.
是一篇等候的詩篇.
我們心下要默默無聲.
專心等候耶和華.
《約翰福音》第16章.
12至24節.
也是耶穌討論到等候的功課.
讓我們在《約翰福音》裡.
詳細看看.
等候的功課.
首先我們看16至18節.
耶穌是怎麼說的呢?.
等不多時,你們就不得見我.
再等不多時,你們還要見我.
有幾個門徒就比喻說.
他對我們說.
等不多時,你們就不得見我.
再等不多時,你們還要見我.
又說:因我往乎那裡去.
有什麼意思呢?.
門徒比喻說.
他說.
等,等不多時.
到底是什麼意思呢?.
我們不明白他說的話.
其實不單門徒不明白.
今天我們讀這段聖經時.
也摸不著頭腦.
說什麼呢?他說什麼呢?.
等不多時,你們就不得見我.
再等不多時,你們還要見我.
好像有些語無倫次.
更加不明白他所言.
更加說.
因我往乎那裡去.
如果要明白這段聖經.
我們一定要明白當時猶太人的背景和思想.
因為耶穌在裡面用了一些術語.
叫做technical terms.

$^{281}$在我們看來不覺得有什麼意思.
但對當時的猶太人來說.
是有特別的意思.
譬如到那日.
什麼叫那日呢?.
那日的富人生產又是什麼意思呢?.
是其中一個例子.
究竟猶太人的思想是怎麼看的呢?.
他們將歷史分為兩個大的時期.
一個稱為現今的世代.
一個稱為邪惡的世代.
這個時期是黑暗.
罪惡掌權的時代.
第二個時期就是將來的時代.
活世的時代.
光明的時代.
這兩個時期的交接.
就是彌賽亞內林的時候.
因著彌賽亞內林啟開一個新的紀元.
一個充滿平安喜樂的紀元.
但在交接期間.
在in between period.
卻是一個非常的時期.
我們留意舊約和龐經.
怎麼形容這個時期.
以上亞書第十三章第九節.
就這麼說.
耶和說日子臨到.
必有殘忍憤恨烈怒.
指者地荒涼從其中除滅罪人.
才知以上亞形容.
這個時間.
必有殘忍憤恨烈怒.
指者地荒涼.
是一個非常的時期.
約書第二章一至二節.
又說.
因為耶和日子將到.
已經臨近那日是黑暗幽冥.
密雲污黑的日子.

$^{321}$好像神光.
鋪滿生靈.
有一對蝗蟲又大又強.
從來沒有這樣的.
以後直到萬代也必沒有.
才能夠.
形容這個時期.
黑暗幽冥密雲污黑的時代.
我們又看看.
龐經的巴六書.
第27章.
主的日子來到.
好像賊一樣.
一大響聲.
一切被火焚燒.
歸於無頭.
所說的耶和說日子.
就是彌賽亞尼亞日子.
就是這樣的日子.
我們就會了解到.
究竟這個非常時期.
是何時開始的呢.
23節說.
到那一日.
耶穌說等不多時.
是等甚麼時候呢.
再等不多時又是指甚麼時候呢.
一向以來.
解經家有不同的意見.
首先我們看看等不多時.
你們就不得見我.
這裡說只要耶穌見到就死.
沒多久.
你們見不到我.
因為我要死.
被釘十字架.
再等不多時又是甚麼意思呢.
有兩種不同的解釋.
第一種是指耶穌的復活.
換句話說再等不多時.

$^{361}$就是等到生了.
你們還要見我.
因為我將會復活.
並向你們顯現.
所以你們就會見到我.
有些人就認為不是.
這不是指復活.
再等不多時.
不一定是指三天.
而是指比較長時間的事情.
因耶穌死後三天復活.
關鍵就在於.
因我往父那裡去.
門徒見不到耶穌.
因為他要往父那裡去.
耶穌死後三天.
並沒有往父那裡去.
他還在地上和門徒.
有一段時間才升天.
所以不是指復活的顯現.
而是指主在那一日.
主在那一日.
究竟是指復活的日子.
還是主在那一日呢?.
似乎兩種說法.
都有道理但也有漏洞.
英國的聖經家.
C.K.Barratt有一個非常精彩的見解.
他認為耶穌是故意說得含糊不清.
因為他所說的是一個非常時期.
也是一個頗長的時期.
由耶穌的死.
和復活開始.
一直到主在那一段時期.
就是從耶穌的第一次來.
和第二次來之間.
這就稱為那日子.
或者末世.
或者非常時期.
這包括了復活後的顯現.

$^{401}$同時也包括了主在來的時候.
所有兩者之間皆可.
所以當聖靈降臨的時候.
他就引用這個意思.
就說這末後只是開始.
也是一個非常時期的開始.
而在這個時候.
凡求主明的就必得救.
換言之.
日子過後就沒有這個機會.
神學家稱這個時間為.
Already but not yet 的時間.
也就像是婦人生產孩子的時候.
孩子在媽媽的肚子裡.
已經在了.
但Not yet.
還沒有出生.
生產充滿了痛楚.
但同樣充滿了希望.
耶穌如何讓孩子.
在這個時候.
感到非常希望.
耶穌如何形容這個非常時期.
也是我們所處.
現今所處的時期.
第十節就說.
我實實在在地告訴你們.
你們將要痛哭.
哀號,世人都要喜樂.
你們將要憂愁.
而你們的憂愁必變為喜樂.
耶穌提到痛哭.
哀號.
這個字是哀號.
這個字是Cryo.
還有另外一個字.
就是Phrenio.
是死人哭喪的形容詞.
表示非常傷心的意思.
很明顯.

$^{441}$只要耶穌快要死.
他們心裡憂愁哭泣.
這裡還有一個更廣的意思.
我們活在這個.
非常時期.
心裡有很多的哀號.
傷痛.
我們看到耶穌來到這個世間.
成就了救贖.
聖誕的日子.
也是平安的日子.
也是和君要和權衡的日子.
但是我們周圍所看到的.
並不是這些.
並不是平安.
並不是救贖.
而是看到剝削.
苦難,不平,災難.
所有這些日子.
痛苦,不平,戰爭.
我們看到周圍都是這樣.
我們又看看周圍的基督徒.
很多時候我們軟弱.
不信.
又看到世界不平.
我們不禁會問道.
神啊!你在哪裡?.
為什麼總是看不見呢?.
世人卻大喜大樂.
伊拉維紹.
說過一個非常有趣的故事.
一個拉比.
他終日辛勤工作.
看到太多的痛苦.
他就厭倦這種.
沐養的生活.
他不想看到人間的痛苦.
他實在太疲倦.
他索性脫下.
拉比袍.

$^{481}$去過著一個.
沒有負擔,沒有痛苦的世界.
第二天他看到.
一個老虎垂死.
這個老虎問他.
為什麼神叫我在這個世上.
拉比無言以對.
唯有說.
既然是這樣,你就接受吧.
這個婦人就死了.
拉比.
他覺得很慘.
他要走.
就是不想看到這樣的情形.
不想去安慰人.
現在又要去安慰.
心裡很難過.
他對自己說不行.
從今天開始.
我什麼都不說.
過了不久.
他去到另一條村.
看到一個少女.
抱著一個嬰兒.
嬰兒上來.
安慰她.
嬰兒死了.
他就合力把她撞了.
在這個時候.
這個年輕的母親就問拉比.
拉比.
神為什麼讓這個嬰兒.
生在這個世上.
那拉比.
無言以對.
唯有滴下.
同情的眼淚.
撞了嬰兒.
他就離開.
離開的時候.

$^{521}$他對自己說.
從今天開始.
我不單是啞的.
我更加是聾.
他走到深山.
在那裡.
不再有人.
跟他說這些痛苦.
不開心的事.
他那裡真是逍遙遊.
有一天他出來.
看到小鳥受了傷.
他就為他用山草藥.
用了一隻.
又啞又聾又啞的拉比.
看到這個小鳥就喜歡了.
有一天他出來.
忽然間有一個大石.
不知從哪裡掉下來.
打死了小鳥.
他看到的時候.
他很痛苦.
他對自己說.
從今天開始我不單是聾是啞.
我更加是慢的.
一個又聾又啞又慢的拉比.
他能夠做些什麼.
他想了沒多久.
他對自己說.
好,我再開眼.
我再開耳.
我再開口.
回到會堂裡面.
繼續做拉比.
耶穌說你們在世上有苦難.
在我裡面.
有平安.
耶穌說.
沒多久你就見不到我.
是說他的死.

$^{561}$他的死.
啟開一個新的紀元.
但這個紀元是一個.
already but not yet 的紀元.
所以還有下半場.
再等不多時.
是起和落的一個時間.
我們看看19-22節.
耶穌看出.
他們要問他的話.
我說等不多時.
你們就不得見我.
再等不多時你們還要見我.
你們以為這話.
彼此相聞嗎.
我實在告訴你們.
你們將要痛哭哀號.
世人都要喜樂.
你們將要憂愁.
然而你們的憂愁.
會變為喜樂.
富人生產的時候.
就憂愁.
因為他的時候到了.
既生了孩子.
就不再紀念苦楚.
因為歡喜世上生了一個人.
你們現在也是憂愁.
但我要再見你們.
你們的喜樂是沒有日子.
跌去的.
我雖然是一個男人.
沒有生產經歷.
我親眼見到我自己兩個小朋友出生.
你也看到太太.
熬了十幾個小時的勞動痛楚.
生孩子確實是.
一件非常傷痛的事.
但是.
生孩子和生孩子不一樣.

$^{601}$生孩子.
你奇怪.
母親.
她不會覺得.
算了.
不生了.
她歡迎這個痛楚.
來到.
當震動來到的時候.
她說.
孩子快出生了.
生癌症.
是沒有這個盼望.
關莫大於生死.
但基督徒來說.
就好像.
代產的母親生了孩子.
勞動痛楚.
不是癌症痛楚.
因為我們有盼望.
盼望新的孩子快要誕生.
每一個震動來.
都有一個記號.
記號就是孩子快要出生了.
所以耶穌說.
你們的憂愁要變為喜樂.
正如勞動痛楚之後.
會帶來歡欣.
因為你會見到你的孩子.
你看耶穌.
如何形容這個喜樂.
你們的心就喜樂.
喜樂是沒有人奪去.
世人的喜樂是隨環境而定.
英文happy.
這個字是來自一個字經.
叫做Hap.
意思就是機會.
你happy與否就在乎.
你的機會如何.

$^{641}$加薪 中樂合彩.
股票上升 有好處.
當然happy.
但加薪 加辛苦的薪.
被炒油.
子女不聽話.
事事不願意.
絕對不happy.
這就是耶穌所說的color.
喜樂的意思.
是超越環境.
沒有人可以奪去.
24節看到第二個特色.
叫你們的喜樂可以滿足.
滿足的意思是plurom.
就是滿足實踐了 充滿.
充滿 無缺陷.
耶穌賜給我們的喜樂.
是滿到寫的 完全無缺陷.
24節看到第三個特色.
生你的孩子.
就不再記念那副錯.
因為歡喜.
世上生了一個人.
這喜樂是將昔日的痛楚驅走.
就有一個媽媽生了一個BB.
她不會再想到生BB的時候.
痛楚.
她不會對BB說.
你現在是最壞的生 你弄得我這麼痛.
你不要親 她會捨去.
她會日喜樂.
是超越一切.
有什麼用呢.
等到主再來.
我們才有喜樂.
我們現在怎樣.
我先說耶穌復活之後.
這是一個新的時代開始.
同時也是舊的時代.

$^{681}$仍存在.
所謂的already but not yet.
在這個時期是一個非常時期.
我們可以嘗到喜樂.
此時刷.
將優秀變為喜樂.
為什麼.
將來不是說將來.
是將來影響現在.
如果我們知道將來.
影響現在.
就會很不同.
孔夫子說.
未知生 焉知死.
但美國神學家.
把這句話反過來說.
未知死 焉知生.
如果我們知道死亡是什麼.
我們在生命會更加有意.
如果你知道將來發生什麼事.
你現在有預備.
找到機會.
所以不是說將來.
是說現在.
你知道將來發生什麼事.
如果你知道灣區什麼時候會有地震.
你現在就有預備.
你知道什麼時候發生什麼事.
你現在就會有改變.
我們神學叫ascadological perspective.
一個末世的觀念來看.
你的人生就完全不同.
所以我們看看.
那等何日子.
23至24日.
到那日你們什麼也不問我了.
我實實在在告訴你們.
你們若向父求什麼.
他必因我的名賜給你們.
向來你們會有奉我的名求什麼.

$^{721}$如今.
求就必得著.
叫你們的喜樂可以滿足.
究竟我們怎樣去等待.
後的日子呢?不是白等.
不是死捱.
是歡歡喜喜的等待.
何解呢?.
教導我們怎樣去祈禱.
我們儘管來到主的私人寶座面前.
求連率.
蒙恩慰.
不住的祈禱.
凡事謝恩.
我們的祈禱就會帶給我們滿足的喜樂.
《書第五章》第七節教導我們.
弟兄們啊.
你們要忍耐直到主來.
當那農夫忍耐的時候.
地裡寶貴生產.
直到秋雨春雨.
你們也當忍耐.
因為主的日子近了.
忍耐這個詞是說來的.
就是生孩子的一個圖畫.
有盼望的.
媽媽生孩子的時候.
雖然痛了但是有盼望的.
就像一個農夫.
在等候.
他不是鴨寮座長.
我們需要這個忍耐的功夫.
但這是一個積極的等待.
今天我們或者仍然有疑問.
仍然有苦難.
仍然有不公不義.
到哪天你們什麼也不再問我了.
本來在哥林多前書.
第十三章也說過.
我們現在面對鏡子.

$^{761}$模糊不清.
到哪天必看得清清楚楚.
今天我們仍然會有疑問.
我們問神為什麼.
為什麼.
如果我們相信了祂.
如果我們明白了將來.
是一個什麼世界.
我們就勇往直前.
我們不再問why.
神在一個黃金的時代.
我們可以做什麼.
希望我們能看得清清楚楚.
今天我們.
是一個ready but not yet的時代.
但我們看.
今天這個世界.
有很多事令我們沮喪.
就過去.
2024年.
有61個國家轉換更替了他們的領袖.
轉了全球超過GDP的一半.
還有.
有39個國家.
正在打仗.
超過十萬人死亡.
有時令我們很心痛.
很悲痛.
為什麼會這樣.
雖然我們會在這個世代.
但我們不是躺平.
我們不是放棄.
我們不是逃跑.
我們更加看到.
這是黃金機會.
主祖來的日子緊了.
為什麼我們不好好抓緊這個機會.
將福音傳開.
讓人們直著耶穌基督.
有這個盼望在裡面.

$^{801}$以致我們有新的日子在這個世上.
都是過得精彩的.
因為我們知道.
將來是怎樣.
我們就用將來這個意識.
帶到今天世界裡面.
以致我們活得更加有意義.
等於怎樣.
你不就像一個拉比想逃跑.
扮龍扮啞.
扮蠻.
曾經說我們看到.
黃金的機會.
主啊你幫助我.
讓我看到.
你賜給我的機會.
願你的名.
得到榮耀.
我們同心一同祈禱.
天母我們再次多謝你.
讓你寶貴的話語.
提醒我們.
讓我們知道今天我們會在一個什麼世代裡.
我們多謝你.
祈禱奉耶穌基督的名銜.
阿們.
\newpage



\section{約翰福音 16:25-33}
\label{sec:fV_h6TniAkc}
\textbf{信與覺 (約翰福音16\_25-33) - 蘇穎睿牧師 [約翰福音研讀 - 第66講]}
\newline
\newline
連結: \href{https://youtube.com/watch?v=fV_h6TniAkc}{\texttt{ https://youtube.com/watch?v=fV\_h6TniAkc}} ~~~~ 語音日期: 2025-02-09 
\newline
\newline
\hyperref[sec:HaGDtN4u47U]{< < < PREV SERMON < < <}
~
\hyperlink{toc}{[返主目錄]}
~
\hyperref[ch:preacher5]{[返講員目錄]}
~
\hyperref[sec:wiDRWRXrtjM]{> > > NEXT SERMON > > >}
\newline
\newline
約翰福音 16:25-33
\newline
\begin{longtable}{cl}
\hline
\hline
章節 & 經文 (和合本修訂版)\\
\hline
16:25 & \begin{tabularx}{0.7\textwidth}{X} 「這些事,我是用比方對你們說的;時候將到,我不再用比方對你們說,而是要把父的事明白地告訴你們。 \end{tabularx} \\ \\ \relax
16:26 & \begin{tabularx}{0.7\textwidth}{X} 到那日,你們要奉我的名祈求;我並不對你們說,我要為你們向父祈求。 \end{tabularx} \\ \\ \relax
16:27 & \begin{tabularx}{0.7\textwidth}{X} 父自己愛你們,因為你們已經愛我,又信我是從神而來的。 \end{tabularx} \\ \\ \relax
16:28 & \begin{tabularx}{0.7\textwidth}{X} 我從父而來,到了世界,又離開世界,到父那裡去。」 \end{tabularx} \\ \\ \relax
16:29 & \begin{tabularx}{0.7\textwidth}{X} 門徒說:「你看,如今你是明說,不用比方了。 \end{tabularx} \\ \\ \relax
16:30 & \begin{tabularx}{0.7\textwidth}{X} 現在我們曉得你凡事都知道,也不需要有人問你;從此我們信你是從神而來的。」 \end{tabularx} \\ \\ \relax
16:31 & \begin{tabularx}{0.7\textwidth}{X} 耶穌回答他們:「現在你們信了嗎? \end{tabularx} \\ \\ \relax
16:32 & \begin{tabularx}{0.7\textwidth}{X} 看哪,時候將到,其實已經到了,你們要分散,各歸自己的地方,留下我獨自一人;然而我不是獨自一人,因為有父與我同在。 \end{tabularx} \\ \\ \relax
16:33 & \begin{tabularx}{0.7\textwidth}{X} 我對你們說了這些事,是要使你們在我裡面有平安。在世上你們有苦難,但你們要有勇氣,我已經勝過世界。」 \end{tabularx} \\ \\
[1ex]
\hline
\hline
\end{longtable}
$^{1}$各位電影主播 各位朋友 你們好.
歡迎大家參加我們三藩市報道會.
網上的午後崇拜.
今天我們繼續思想約翰福音的信息.
我們來到約翰福音第十六章.
二十五節至三十三節.
題目叫信與覺.
在開始思想這段聖經之前.
請我們同心低頭 或者同禱告.
前面的人我稱呼我們多謝你.
因為你測驗聖靈來到我們當中.
記著你的話語.
祂叫我們能夠有自省.
知道我們是一個罪人.
接受你做我們個人救主和主宰.
求聖靈在我們心裡面動功.
讓我們有醒覺的心.
不是夢察察做人.
求你真是祝福帶領我們.
祈禱奉耶穌基督命求 阿們.
在中國佛教當中有一個支派叫禪宗.
禪宗是處於六祖慧能.
大家可能聽過他所提及的一首詩.
菩提本無處 明鏡亦非台.
本來無一物 何處惹塵埃呢.
當其他人唱到說.
是時勤忽悉 莫處惹塵埃的時候.
他就反駁 本來無一物 何處惹塵埃呢.
究竟甚麼是禪宗呢.
禪其實是一種醒悟.
是一種覺醒.
所謂眾人皆醉我獨醒.
是一種醒覺 是一種醒悟.
而這種醒悟的途徑是不靠明智.
不靠言語 純是一種體驗.
所以禪宗是不納文智.
有他的理論 純是個人體驗去醒覺.
去醒悟的.
到唐朝晚年的時候.
流行一種所謂公案.

$^{41}$所謂公案是來自古代禪師的日語.
和問答 或者是提示和質問.
幫助那些信眾去醒覺 去悟 去體驗.
或者是參透一些公案.
就立即啟悟過來 明白道理.
這些就是頓悟.
耳祖慧能有一次向達摩禪師說.
我心很不安 惱惱亂.
可不可以請老師替我安心.
達摩反問他.
你把你的心拿出來 讓我替你安穩.
過了一會兒 慧可回答.
我找了很久 也找不到心.
達摩說 好 我已經把你的心安好了.
你聽了這個公案 你會醒悟些什麼呢?.
就抓著頭不知在想什麼呢?.
另外一個例子 有一個和尚問.
世界上有什麼是最貴的?.
這個禪師曹山回答.
死貓頭是最貴的.
和尚問死貓頭最貴是什麼?.
曹山說 沒人會出錢 沒人會出價.
嘩 這樣也行?.
你對這個公案又有什麼醒悟呢?.
又是另外一個例子.
有個和尚跟曹山說.
我終身都是病 請老師醫治一下.
曹山說 不醫.
和尚覺得很奇怪 問他為什麼不醫?.
曹山說 是叫你求生不能 求死不得.
嘩 真厲害.
這種醒悟是什麼意思呢?.
這個公案給你有什麼醒悟呢?.
還有一個更離譜的例子.
一個潮州禪師向弟子文遠說.
我們來個比賽.
看誰用比喻將自己說得最低最賤.
如果能說得最低最賤就贏了.
我就輸一塊餅給你吃.
師父先說 我是一隻驢.

$^{81}$徒弟說 我是驢的屁股.
師父說 我是驢的屁股出來的糞.
大便.
徒弟說 我是糞裡的蟲.
師父上了半天都接不下去.
很不高興問.
你在糞裡做什麼?.
徒弟說 在那裡度假.
師父嘆一口氣說.
你真厲害 你是可以的.
輸你贏 輸你贏.
你聽完這個話.
你又誤會到什麼真理呢?.
其實佛教和基督教主要分別在哪裡呢?.
佛教是說醒覺 自覺.
自己去醒悟 靠自己的力量去覺醒.
而基督教是說信.
或者我們用一個比喻.
假如我們要搭飛機或者要去香港.
佛教是說渡 背渡禪師.
什麼渡什麼渡.
如果去香港就渡過太平洋.
渡過生死海 去到彼岸.
靠自己的力量 靠自己游水渡過.
但對基督教來說.
你永遠游不過去.
你坐飛機.
你上飛機師 飛機就飛你過去.
所以佛教是人本主義.
人道主義 人民主義的宗教.
問題就是 匿者不能自救.
壯士不能自居其身.
我們是不是真的可以能夠度悟呢?.
是不是看了這些公案就可以醒悟過來呢?.
是不是葡萄本無樹 明鏡亦飛台呢?.
是不是世上無一物 何處惹塵埃呢?.
或者你去問問自己.
何處塵埃?.
不就是一樣是塵埃嗎?.
你有癌症就有癌症.

$^{121}$有塵埃就有塵埃.
這些是客觀的事實.
你不能夠說沒有.
那些癌症就會走了.
這只不過是自欺欺人.
就好像莊子在他太太死後跳舞.
他說去生死 去長短.
這就是自欺欺人.
是一個木頭人.
沒錯 基督教也是講覺醒.
也講醒悟過來.
但它不是靠自己的力量.
而是靠神的力量 靠信.
是因為人領悟到.
當我們醒悟的時候.
我們就能夠去做.
當我們醒悟的時候.
也是聖靈在我們心中動弓的時候.
開啟了我們的眼睛.
看到我們的廬山真面目.
我記得以前我在一間中學教書.
在香港的時候.
我教的科目其中一科是聖經科.
不少學生讀聖經.
只是為了應付考試.
考試完了什麼都不記得了.
後來他們有機會去參加教會.
參加查經班.
讀同一本聖經.
奇怪了.
讀到阿瑟突然心中領悟過來.
我相信在我們中間.
不少人都有過這個經歷.
今天就讓我們看看.
約翰福音這個課題.
信是什麼事.
首先我們來看看25至31節.
這些事我自用比喻對你們說的.
事後將到我不再用比喻對你們說.
那要將乎冥冥的告訴你們.

$^{161}$到那日你們要奉我的名祈求.
我並不對你們說.
我要為你們求符.
符自己愛你們.
因為你們已經愛我.
又信我是從符出來.
我從符出來.
到了世間.
我又離開世界.
往復那裡去.
夢到說.
如今你是明說.
並不用比喻了.
現在我們曉得你凡事都知道.
也不用人去問你.
因此我們信你是從神出來的.
耶穌說.
現在你們信了.
或者我們詳細來看看這段聖經.
首先我們看看時間上的問題.
耶穌說.
事後將到我不再用比喻對你們說.
那要將乎冥冥的告訴你們.
到那日你們要奉我的名祈求.
究竟事後是指什麼時候呢?.
那日是指哪一天呢?.
原文「事後」這個字.
其實直譯可以說.
那時辰.
那一個鐘頭.
那個鐘頭.
是指哪一天呢?.
在上回我們討論到約翰福音的時候.
我們已經提到猶太人當時.
是有些術語.
而耶穌所提到的就是當時的術語.
猶太人相信.
這個世間可以分為兩個階段.
現今階段.
現今的世代.

$^{201}$也可以稱為邪惡的世代.
另一個就是光明的世代.
也就是將來的世代.
這兩個世代交接期間.
就是一個非常時期.
而這個時期就是耶穌所說的.
那小時.
那一日.
在這一段時間.
就是指耶穌從受死到復活升天.
直至主再來這一段時間.
就是耶穌第一次來和第二次來之間.
這一段時間.
就是耶穌所說的那一段時間.
是一個非常時期.
一方面邪惡的世代仍然存在.
但另一方面.
光明的時期也開始.
神學家就稱為.
Already but not yet.
的非常時期.
這個時期的特色.
Already已經有盼望有光明在我們裡面.
但同時還有Not yet方面.
仍然有痛苦仍然有邪惡.
但在這個時期.
有另一個特色.
這個特色就是.
耶穌已經成就了救恩.
他已經死而復活.
聖靈又降臨在我們裡面.
聖靈教導我們去明白真理.
進入真理.
並且為我們禱告.
所以到那一天.
是指那一段時間.
那到那一天有什麼特色呢?.
耶穌就說.
我心想我不再用比喻對你們說.
要將父明明地告訴你們.

$^{241}$這是什麼意思呢?.
首先我要看看比喻這個字.
Paramea.
這個字是由兩個希臘文組成.
一個叫Para.
Para就是並列同行.
Alongside的意思.
另一個是Amox.
Amox就是路,道.
兩個字湊在一起就是排列之道.
意思是比喻,比方,真言.
通常這些比喻是比較隱晦一點的.
是比較深奧一點的.
是比較難明一點的.
這個字的相反詞就是明明.
希臘文叫Paraisea.
是直言的,坦言的.
比較容易明白的.
或者你覺得奇怪.
為什麼耶穌先前.
是用難明的言語.
難明的方法.
就是比喻.
現在我坦白說.
說到明.
其實明明是有兩個因素的.
第一個是說者的問題.
說不明白.
我說什麼都不明白.
我嘴巴頓一點.
我說什麼你們都不明白.
我說什麼我自己都不明白.
很多時候不懂說.
一個不是說者的問題.
是聽者的問題.
就像中國人所謂.
牛皮燈籠.
怎麼說都不明白.
不是因為說者無能.
而是聽者無心.

$^{281}$而耶穌所說的.
就是屬於後者.
不是說者的問題.
而是聽者的問題.
可以看到.
我們首先看看.
「時候將到,我不再用比喻對你們說.」.
事實上這是耶穌最後的教訓.
除非耶穌.
在復活之後的教導.
不需要用比喻.
但這是不可能的.
因為29節佛陀說.
「如今你是明說,並不用比喻了.」.
換句話說.
在未復活之前.
未死之前.
門徒已經明白了.
所以這不是說者的問題.
而是聽者的問題.
再者.
這個PARAMIL這個字.
可以做難明,不明,深奧.
也可以指出聽者的問題.
而不是說者的問題.
所以25節的意思就是.
「時候將到,你們的心開了.」.
在聖靈引導下.
你們不再覺得我所說的教訓.
是那麼難明,那麼隱晦.
29節就說.
「門徒說:如今你所說的,我們明白了.」.
我們還有另一個證據.
在這裡之前.
耶穌有不少的教訓.
都是坦然教導的.
並沒有用比喻的.
更加證明了.
當時他們不明白.
現在為何會明白呢?.

$^{321}$因為以前.
他們聽有問題.
現在聖靈啟開了他們的眼睛.
讓他們進入真理.
他們就醒過來了.
不再醒過來.
究竟門徒明白了甚麼呢?.
他們認識到甚麼呢?.
30節就告訴你.
說得很清楚.
「現在我們曉得,你凡事都知道,也不用人問你,因此我們信你是從神來的.」.
28節.
耶穌說:我從父出來,到了世界,我又離開世界,往父那裡去..
在這幾節聖經.
我們很明顯看到有下列的幾個真理.
第一,耶穌是從神那裡來的.
祂是神.
門徒開始完全明白.
耶穌是從神的角度來的.
祂就是神.
第二,祂是道成肉身.
從神那裡來到這個世界.
祂是從神的世界來到人的世界.
第三,祂是無所不知.
不需要人問祂的問題.
祂也知道.
祂不需要人去證明祂.
因為祂已經是神了.
當祂返回天府的時候.
祂會離開世界.
返神那裡.
以現代的神學術語來說.
他們信耶穌的神聖.
信耶穌的道成肉身.
信耶穌的救贖.
信耶穌超越性無所不知.
更加信耶穌愛他.
為我們死復活升天.
為什麼門徒會相信呢?.
第一節說:耶穌說:現在你們信嗎?.

$^{361}$從經文的上文下理來看.
我們至少看到有兩個原因.
第一,十三節說:是聖靈的工作.
是聖靈來到的時候.
引導他們明白.
第二,是信.
三十節和三十節是信的因素.
因為他們相信,他們就明白了.
明白了之後.
因為他們明白了之後.
他們就相信了.
是相輔相成.
他們有一個初信.
在第二節做見證的時候.
他說:我初信的時候.
有很多東西都不明白.
尤其是未信之後更加不明白.
但是現在我信了主.
奇怪了,那些問題不翼而飛.
不是在理性上我解答了嗎?.
那些問題已經不再困擾我了.
這是聖靈的工作.
也是信心的問題.
以致他們能夠醒悟過來.
弟兄姊妹.
普羅說:我們如今面對鏡子無不清.
有很多東西我們都未必明白.
但是那些已經不再困擾我們的問題.
聖靈會幫助我們了解明白.
我們需要知道什麼.
有什麼我們未必需要知道.
更加因為信,信得過耶穌基督.
既然有信心的時候.
我們就拿出信心來跟隨.
正如你坐飛機.
你不信得過飛機師.
從頭到尾在那裡顫抖.
怕飛機會出事.
你就慘了.
你信得過飛機師的時候.

$^{401}$你可以大教訓.
但是信的問題要醒覺.
我們再看第二個問題.
叛徒的苦難.
32至33節.
我們千萬不要以為.
信了耶穌之後就一帆風順.
事事如意.
生意興隆.
有子女就聽話.
你看耶穌怎麼說.
「多看吧!時候將到,且已經到了.
你們要分散.
各歸自己地方去.
留下我獨自一人.
其實我不是獨自一人.
因為有父與我同在.
我將這事告訴你們.
是要叫你們在我裡面有平安.
在世上你們有苦難.
但你們可以放心.
我已經戰勝了世界」.
一開始的時候.
又是那一句.
「時候將到,且已經到了」.
這個時候就是指耶穌.
就是指那一個鐘頭.
指非常時期的開始.
是從耶穌基督受死的時候起計.
他們會怎樣呢?.
他們會遭受苦難.
壓迫.
而且是各有各非.
世上有苦難.
但在耶穌基督裡面.
卻有平安.
如果你以為你要明白.
信主就會一帆風順.
還財就手.
萬事勝意.

$^{441}$對不起,你找錯了.
那個不是基督教.
是黃大仙.
他不是黃大仙.
他是耶穌.
耶穌從沒這樣回答過我們.
他提醒我們你們在世上有苦難.
事實上當我們信主的時候.
還會遭遇到一些排斥.
或者被一些人的藐視.
甚至覺得很孤單.
非常痛苦.
但耶穌給我們一個很大的應許.
你會覺得孤單嗎?.
耶穌說你們樓下.
我獨自一人.
其實不是獨自一人.
因為我有父與我同在.
同樣的我們也不是獨自一人.
是有主耶穌與我們同在.
是有神與我們同在.
你會覺得世上有很多壓力嗎?.
太多將領了.
太多苦難了.
耶穌說你們在世上有苦難.
但在我裡面有平安.
你們可以放心.
因為我已經勝了這個世界.
何等寶貴的應許.
我就用一個無名氏.
所寫的一首詩作結束.
我求神給我力量.
以致我可以做事成功.
但神給我軟弱.
以致我懂得謙卑.
和述說.
我求神給我康年.
以致我能夠做大事.
神給我軟弱的身軀.
以致我懂得欣賞祂的保守.

$^{481}$我求神給我有財富.
以致我要開開心心活著.
但神卻叫我貧窮.
以致我更有智慧.
我求神給我權能.
以致能夠得人稱許.
但神卻給我成為一個弱者.
以致我懂得去議教祂.
我求神給我豐豐足足.
以致我可以享受人生.
但神卻給我生命.
以致我可以享受一切.
我所求的.
我並不得著.
但我所望的.
一個也沒有落空.
其實我是世上最幸福的一個人.
所以我們看到基督徒在這個世代的矛盾.
一方面你們在世上有苦難.
是有苦楚.
這個世界仍然是凹凸不平.
有很多事情我們看不順.
但我們不是因為這個原因.
我們就喪志.
不會因為這個原因.
我們覺得上帝不理我們.
不是的.
相反來說.
耶穌說你們在世上有苦難.
在我裡面有平安.
我所指的平安不是世人所指的平安.
世人所指的平安.
是看機會.
但耶穌基督所指下的平安.
是超越環境.
超越所有這些.
我們今天能夠信耶穌.
很寶貴一樣.
耶穌說我必不撇下你做故意.
祂仍然看守著我們.

$^{521}$祂仍然帶領著我們.
在崎嶇路.
過風雨.
常與我們同在.
有聖靈在我們身邊.
導引我們的腳步.
讓我們能夠明白.
我們在世上有苦難.
但在耶穌基督裡面.
卻有平安.
我們更有耶穌基督給我們的使命.
以致在我們生活裡面.
是沒有盼望.
沒有競力.
好我們今天講到這裡.
下次再和大家講另外一個課題.
請我們再一次同心禱告.
您的話語提醒.
讓我們知道.
在今天的世代裡.
我們仍有苦難.
但我們也知道在您裡面有平安.
您也有盼望.
求你讓我們看得準.
也活得精彩.
求主讓我們能夠跟隨您.
與我們同在.
祈禱奉耶穌基督的名.
求你讓我們能夠.
在這次的苦難中.
能夠和祂同在.
祈禱奉耶穌基督的名.
阿們.
\newpage



\section{約翰福音 17:1-5}
\label{sec:wiDRWRXrtjM}
\textbf{榮歸 (約翰福音17\_1-5) - 蘇穎睿牧師 [約翰福音研讀 - 第67講]}
\newline
\newline
連結: \href{https://youtube.com/watch?v=wiDRWRXrtjM}{\texttt{ https://youtube.com/watch?v=wiDRWRXrtjM}} ~~~~ 語音日期: 2025-02-13 
\newline
\newline
\hyperref[sec:fV_h6TniAkc]{< < < PREV SERMON < < <}
~
\hyperlink{toc}{[返主目錄]}
~
\hyperref[ch:preacher5]{[返講員目錄]}
~
\hyperref[sec:OtTM_EdQEtA]{> > > NEXT SERMON > > >}
\newline
\newline
約翰福音 17:1-5
\newline
\begin{longtable}{cl}
\hline
\hline
章節 & 經文 (和合本修訂版)\\
\hline
17:1 & \begin{tabularx}{0.7\textwidth}{X} 耶穌說了這些話,就舉目望天,說:「父啊,時候到了,願你榮耀你的兒子,使兒子也榮耀你; \end{tabularx} \\ \\ \relax
17:2 & \begin{tabularx}{0.7\textwidth}{X} 因為你曾賜給他權柄掌管凡血肉之軀的,使他把永生賜給你所賜給他的人。 \end{tabularx} \\ \\ \relax
17:3 & \begin{tabularx}{0.7\textwidth}{X} 認識你—獨一的真神,並且認識你所差來的耶穌基督,這就是永生。 \end{tabularx} \\ \\ \relax
17:4 & \begin{tabularx}{0.7\textwidth}{X} 我在地上已經榮耀你,你交給我做的工作,我已完成了。 \end{tabularx} \\ \\ \relax
17:5 & \begin{tabularx}{0.7\textwidth}{X} 父啊,現在求你使我在你面前得榮耀,就是在未有世界以前,我同你享有的榮耀。 \end{tabularx} \\ \\
[1ex]
\hline
\hline
\end{longtable}
$^{1}$各位大兄姐妹各位朋友你們好.
歡迎大家參加我們三藩市播道會.
網上的五堂崇拜.
今天我們繼續思想約翰福音的信息.
我們來到約翰福音第十七章一節至五節.
題目叫做榮歸.
未開始思想這段聖經之前請我們同心低頭.
我們同禱告.
因為你降生為人.
甘願為我們的緣故釘身在十字架.
成就了這個救恩.
因為你的死我們得醫治.
我們有盼望.
得享榮生.
求你藉著你的話語去提醒怨念教導.
安慰督教我們.
求你幫助我們.
讓我們打開我們的心.
讓你的話語去改變我們的生命.
我們今天禱告.
拉下鳳爺的紀律名.
阿們.
中國人有這樣說法.
死有輕於鴻毛.
重於泰山.
沒錯.
死本來就很醜惡.
很傷痛.
很難過的事.
但萬古歷史.
死卻製造了無數英雄的熱誓.
我想到我們中國歷史的文天祥.
為了救國救軍.
被蒙古兵囚在獄中.
在獄中寫著著名詩句.
人生自古誰無死.
留得丹心照寒青.
他所寫的正喜歌.
是lum 然萬古傳.
令人肅然起敬.

$^{41}$雖然隔了幾百年.
但讀起來.
心裡有一股浩然之氣.
猶然而生.
千地有正氣.
澤言呼柳營.
下則為河岳.
上則為日升.
於人若好怨.
背負蔡蒼茗.
士氣所磅礡.
lum 然萬古情.
當其故日月.
生死惡誤逐輪.
被為賴以立.
天處賴以轉.
沒錯.
真是一首非常令人感動的一篇文章.
我又想到一首.
由猶太人史碧高所寫的一首詩.
叫做《最後的審判》.
這首詩記述了一個非常動人的故事.
公元69年.
以色列人造反.
要推翻羅馬政府.
爭取他們的民族自由.
很多以愛國家.
以愛他們民族.
以愛他們信仰的猶太人.
都捐軀在這次大屠殺當中.
這次造反.
差不多超過100萬猶太人被殺.
其中有一個婦女.
帶著七個兒子投入戰圈.
結果被羅馬兵重重圍困.
最後她拿出自己的刀.
將七個兒子一個一個砍死.
最後自殺身亡.
當史碧高記述這件事的時候.
他就這樣寫.

$^{81}$當那婦人以她心愛的兒子.
和自己的獻上的時候.
她覺得這個獻上.
教主阿巴拉罕憲以實.
來得更轟烈.
更勇敢.
更光明.
更徹底.
他們成了民族的英雄.
我又想到我媽媽.
在1951年.
家父接到從大陸親戚寄來的字條.
即是簡單幾個字.
你的太太已經上吊自殺身亡.
當然她的死是一個極之不幸的消息.
是我們家的悲劇.
但人的瘦身的結果是怎樣呢?.
她的死亡.
沒錯是很不幸.
對我們家庭來說是一個很大的悲劇.
亦都是人瘦身的結果.
但我們看到人的瘦身.
自私和貪婪.
但對我們家人來說.
我們有另外一方面的看法和感受.
對我們來說.
這是一個轟轟烈烈.
既光明又有價值的一個悲劇.
我媽媽被迫寫信來香港.
引誘我父親回大陸.
以致可以被捉拿.
被審判被迫害.
她為了她丈夫.
為了她兒女.
為言果敢的原諒自盡.
她成為了代罪的羔羊.
為她所愛的來死.
我可以這樣說.
她為我們一家犧牲了寶貴的生命.
是我們一家的殉道者.

$^{121}$當我們看看耶穌的死的時候.
我們覺得很奇怪.
好像一幅完全不同的圖畫一樣.
在耶穌的死的故事和情景裡.
我們找不到半點英雄光明的跡象.
在聖經記載.
她心裡甚是憂傷.
幾乎要死.
她肚子裡吐血.
又說:阿爸父啊.
在你凡事都能求你將這杯撤去.
然而不要從我的意思.
只是從你的意思.
這樣看來.
好像一種很懦弱很怕死的態度.
比起文天祥那種慷慨赴死.
人生自古誰無死.
留得丹心照汗青.
似乎耶穌的死訊息得多了.
另外馬克福音更加記載.
當耶穌被釘十字架的時候.
大聲喊著說:我神啊我神.
為什麼離棄我呢?.
我們奇怪了.
為什麼耶穌會這樣叫呢?.
難道是因為搞革命失敗.
在患烈失望不平之際.
提出一個控訴.
神啊神啊為什麼你在這時候離棄我呢?.
耶穌基督的死.
似乎觀點光榮都沒有.
但事情又不是那麼簡單.
令人感到困惑.
當這個毫無光彩的死.
呈現在白夫長一個釘十字架的專業士兵.
看到耶穌大聲呼喊.
好像很懦弱很軟弱.
很沒有尊嚴的時候.
竟然說:這真是神的兒子.
這個專業釘死人的士兵.

$^{161}$在過無數次的死亡.
有些是轟轟烈烈的死.
有些是害怕恐懼的死.
但奇怪的是當他看到這個.
似乎毫無光彩的死亡的時候.
竟然見證說:這真是神的兒子.
為什麼呢?.
而約翰福音更將耶穌的死視為榮耀.
何榮耀之有?.
我想到當日戰打得最轟烈的時候.
尼克森總統對全國人民廣播.
他說:我們要撤退.
但是光榮的撤退.
當時他派遣了更多B-52型重型轟炸機.
在河內和海防轟炸迫使北越談判.
然後促使美國公明撤退的計劃.
所謂光榮是負上無數人的生命.
我們中國人有一句話說.
一丈功成萬骨枯.
一個光榮的英雄出現.
背後流了不知多少人的血.
所以所謂英雄的光榮.
是建築在很多人的痛苦身上.
失敗和內疚.
令我們不禁詫異.
為什麼聖經說這是光榮?.
為什麼這樣的死亡.
對聖經來說是征服者,勝利者,成功者的死亡?.
但耶穌一出生.
你看看.
就是被人遺棄在馬槽裡.
不夠兩天.
又有人想殺祂.
祂就要逃到夏拉及.
以撒拿更加形容祂無佳容美貌.
被藐視,被人嫌棄.
多愁痛苦,常經憂患.
有人尊重祂.
當祂收治人生.
差不多去到最後一程的時候.

$^{201}$又為門徒出賣.
繼後各散東西.
所謂忠心的門徒.
竟然只能三次不認祂.
最後忍受人間最殘酷的死刑十字架.
又在十字架說神啊神啊.
為什麼在這個時候離棄我?.
在聖經裡面的記載.
由耶穌在赫山瑪利縣祈禱.
一直被審,被釘.
在十字架.
我們看不到半點光明.
我們看不到半點光彩.
但耶穌說正是人子得榮耀的時候.
何解?.
我們要從約翰福音第十七章.
來看奧秘.
十字架的榮耀.
通常稱約翰福音第十七章為大祭司的禱告.
因為是耶穌臨死的時候.
被捉拿的時候的一篇禱文.
第一節就這樣說.
耶穌說了者說就舉目望天.
所謂舉目望天.
那就是祈禱.
在十一章第四十一節所說.
這篇禱文大約分為三部分.
一至五節是耶穌的死與榮耀.
六至十九節是耶穌為門徒禱告.
二十節至二十六節.
為那些因門徒的侍奉而信的人禱告.
我們要去研讀第一至五節.
耶穌為自己特別是祂的死禱告.
耶穌一開始的時候.
祂就這樣祈禱說.
「伯父啊!時候到了.
願你的榮耀為你的兒子使兒子得榮耀」.
這個所謂的時候到就是一個非常時期開始了.
這個非常時期就是祂的死.
祂的瘋瘟.

$^{241}$祂的升天.
聖靈降臨一直到祂再來.
這一段時間.
我們就叫做非常時期.
祂的死對耶穌來說.
是神的榮耀.
神榮耀祂的兒子.
但同時也是兒子榮耀神.
為什麼會有這樣的想法呢?.
我們要看看榮耀這個字是什麼意思.
猶太人一提起「Boksha」這個榮耀這個字.
很自然就想起舊約的會幕.
這個會幕就是神的榮光.
因為會幕是象徵神的同在.
The presence of God.
如果你比較一下出埃及的33章.
所描繪的會幕.
和耀廷第一章.
說耶穌出生「道成肉身」.
你就不難發覺到.
這兩章聖經有很多相似的地方.
會幕是象徵神的同在.
道成肉身.
也是象徵神來到世上與我們同在.
但是不要忘記會幕是建築.
不單是象徵神的同在.
同時也是用作建制之用.
為人民贖罪.
同樣耶穌來到世上.
不單是象徵神來到我們這個世界.
同時也是要走到十字架道路.
為我們贖罪.
就是說舊約的會幕.
只不過是預表了耶穌.
如果猶太人明白這個會幕是榮耀.
他們就會了解耶穌基督的死.
也是神的榮耀.
是神榮耀人子的時候.
也是耶穌榮耀神的時候.
難怪耶穌用比喻說.

$^{281}$聖殿插毀了.
三天又建造起來.
所謂插毀了是說自己的死.
他生的復活起來.
就是說他的復活.
主要是耶穌基督受死和復活.
他就用聖殿來比較.
不單是這樣.
耶穌在這段聖經說得很明白.
為什麼他要死呢.
為什麼他的死也是榮耀呢.
第四節.
我在地上已經榮耀你.
你所託付我的事.
我已經完成了.
原來耶穌的死是神託付他的.
其之所以榮耀.
就是因為耶穌已經完成了託付.
Mission accomplished.
所以我們要了解.
耶穌的死不是人為的錯.
不是不夠運.
不是行雖運.
不是一時之錯.
而是他的使命.
原來世上就是要死.
是父神給他的託付.
耶穌一出生就要走上十字架道路.
為什麼要走上十字架道路呢.
他一生看到.
人間的痛苦被他忘國.
逃亡飢餓.
他看到人的壽星自私.
人最大的痛苦是什麼呢.
就是與神分離.
與神隔絕.
當耶穌上十字架.
為我們最釘心的在十字架上.
因而被父神欺絕.
以至在十字架上大聲說.

$^{321}$神啊神啊為什麼這個時候離棄我.
原來.
這不是失敗的託付.
是勝利的託付.
他連最後一步也走上了.
這一步是什麼呢.
與神父神相離.
背了我們的罪.
在十字架上被神欺絕.
他承擔了我們的罪.
背負了我們的重擔.
與那些受苦難的弟兄姊妹.
完全完全的應當過來.
這就是耶穌基督的榮耀.
他完全的降服.
就是他的榮耀.
就是神榮耀他.
也是他榮耀神.
一而為一.
就發生在這件事上.
第二我們看看十字架的大難.
十字架不單止是榮耀.
它更是神的大能.
我們發現一個非常有趣的真理.
我被釘死的耶穌.
確是滿有尊嚴.
滿有權柄的耶穌.
正如你曾賜給我權柄管理凡事有趣的.
這個中文可能有些令人誤解之嫌.
原文是Educus.
Educus是過去認識的一個Aries.
就是說早已經第一次給了我們.
這個不是現在他才滿有權柄.
他早已經滿有權柄.
如果形式是滿有權柄.
但在釘十字架的那一刻.
雖然他仍然擁有權柄.
絕對的權柄.
為甚麼他會死呢?.
為甚麼這個權柄.

$^{361}$這個管理的權柄.
凡有血氣的管理這個字.
在中文加上去的.
原文是沒有這個字.
只是用一個Alternative case.
是表明說耶穌的權柄.
是過於所有有血氣的.
包括人.
包括皇帝.
原來他不是一個受害者.
他是一個滿有權柄的神.
這十字架給我們的圖畫.
表面上是無能.
是沒有尊嚴.
是被釘死.
是一個囚犯被釘死.
被處死.
但我們讀耶穌受死的故事.
聽到十字架之言.
苦阿赦免他們.
因為他們所做他們不知道.
我實實在在的告訴你.
今晚我要和你到樂園.
這是甚麼說話.
Powerful.
他是滿有權柄的神.
是最有尊嚴的一個.
所以十字架是一幅很弔詭性的圖畫.
表面上是最沒有尊嚴.
最軟弱.
被人釘死.
真是很慘.
但你看回十字架裡所說的話.
他就告訴你.
其實當時最有權柄的.
最有尊嚴的就是耶穌.
以至他能夠對著.
跟他死的囚犯說.
我實實在在告訴你們.
今晚我要和你到樂園去.

$^{401}$不過.
他不是用他的權柄去統治.
去征服.
去控制.
去報仇.
不是.
而是去賜下永生.
第二支.
要他將永生賜給你所賜給他的人.
漁民有一個非常有趣的字.
叫Hinacross.
Hinacross是說明目的.
耶穌擁有權柄的目的是甚麼.
就是賜永生給屬於他的人.
人子來不是受人服侍.
乃是要服侍人.
並且要捨己.
作多人的贖價.
這是多麼奇妙的訊息.
這個擁有絕對權柄的神.
卻以他的權柄為我們死.
以至我們有永生.
我想到年前有一套電影.
Schindler's List.
那個德國納粹黨的士兵.
虐待一個猶太人.
Schindler對他說.
為甚麼你這樣做.
真正的權柄不是殺戮.
而是寬恕.
雖然你有權柄可以治這個人死命.
但你有更大的權柄.
是寬恕他.
寬恕是一個更大的權柄.
這個就是耶穌基督的權柄.
因為他的寬恕.
因為他的死.
以至我們得永生.
但十字架不是整個故事.
它只不過是整個故事的一半.

$^{441}$耶穌第一次就說.
願你榮耀你的兒子.
使兒子也榮耀你.
願問.
這個不是一個時間的次序.
完全不是.
他不是說.
首先神就榮耀耶穌.
然後耶穌就榮耀回神.
不是.
願問是一個Hindercross.
就是說.
父榮耀耶穌.
以至耶穌可以榮耀父神.
是一個Logical Relationship.
不是一個Chronological Relationship.
不是一個時間性的次序.
第一個榮耀是指耶穌的死.
記著耶穌的死.
神得榮耀.
第二個榮耀是指耶穌的復活.
復活就等於耶穌基督的榮耀.
神的榮耀.
所以十字架不是整個故事.
高音不是只說耶穌的死.
還有後半段.
重要的是耶穌的復活.
正因為耶穌的復活.
所以我們可以得到永生.
因為他受的鞭傷.
我們得以醫治.
因為他受了復活.
以致我們有復活的盼望.
但永生不是繞著手就送上門.
第三個就是告訴你.
獨一的真神.
並且認識你所猜來的耶穌基督.
這個就是永生.
原文認識的字是Gnostic.
不是頭腦的認知.

$^{481}$而是一種親密的關係.
你必須信耶穌.
求祂赦免你.
要你的罪得以赦免.
這樣就和祂有一個親密的關係.
父子的關係.
有這個關係才有永生.
你可能有頭腦的認知.
你可能有很多宗教的活動.
但你沒有認識耶穌.
沒有耶穌基督給你的生命.
你沒有接受過祂.
做你的個人救主和主宰.
建立這個關係.
一切都是謊言.
所以我們看到十字架的吊詭.
在這段聖經裡面.
我們可以看到聖經的真理.
十字架的奧秘.
保羅在羅馬書裡.
將十字架的吊詭.
這個paradox說得非常精彩.
哥林多前書第一章十八節.
十字架的道理.
在滅亡的人為預註.
但在我們得救的人卻是神的大能.
猶太人要求神職.
希臘人要求智慧.
但我們卻存十字架的基督.
在猶太人看來是半個石.
在外邦人看來是預註.
但對蒙召的人來說.
無論是猶太人希臘人.
基督總是神的大能.
神的榮耀.
沒錯.
人以為預註的.
無能的.
卻成為了大能.
其實我們會問.

$^{521}$世上的英雄豪傑.
以為槍桿子就是權柄的領導.
今天去了哪裡.
希特勒去了哪裡.
拿破崙去了哪裡.
毛澤東去了哪裡.
白鋼動去浪淘盡.
千古風流人.
希伯來被打不還頭.
被罵不還口的耶穌.
今天征服了多少的心腸.
以致有千千萬萬的人.
生命不得改變.
苦苦在耶穌的面前.
求祂憐憫.
求祂寬恕.
也求祂大靈.
這就是十字架的吊詭.
今天在這個世界裡.
一個凹凹亂亂的世界.
很多人都會問.
我們怎樣可以有和平.
怎樣可以有平安.
是不是制度的改變.
就可以解決這些問題.
是不是科技的進步.
就可以解決這些問題.
是不是節省多點錢.
讓我們富強的時候.
就可以解決問題.
不是.
是我們的罪.
除非我們願意謙卑.
苦服在耶穌的面前.
承認自己的罪.
有學校耶穌在十字架下.
呼喊要恕他們.
這種寬恕的心.
一種和神有密切的關係.
這種關係.

$^{561}$就叫我們的生命完全改變過來.
以致我們的生命.
能夠影響千千萬萬人的生命.
求主憐憫我們.
我們再次向心底同學和禱告.
是因為你的死的緣故.
要我們得生命.
十字架看來好像很羞辱很軟弱.
但它是十字架.
它是我們的大能.
要相信它的人.
求主真是聽我們的禱告.
繼續在我們心裡動功.
明白你呼應的權能和奧秘.
我們今天祈禱.
乃是奉耶穌基督命強.
阿們.
\newpage



\chapter{袁惠鈞}\label{ch:preacher6}
\begin{multicols}{3}
\minitoc
\end{multicols}
{ \scriptsize


\begin{xltabular}{\textwidth}{|p{0.15\textwidth} p{0.6\textwidth}|p{0.07\textwidth} p{0.1\textwidth}|}
\hline
撒母耳記上 16:1-17:58 & \hyperref[sec:OtTM_EdQEtA]{如何勝過生活中的巨敵 (撒母耳記上16\_1-17\_58) - 袁惠鈞牧師[大衛傳系列 - 第2講]} & 2025-01-15 & \href{https://youtube.com/watch?v=OtTM_EdQEtA}{\texttt{ OtTM\_EdQEtA}} \\
撒母耳記上 18:1-19:24 & \hyperref[sec:9t69tF6ci0k]{神啊!求你救我脫離惡人 (撒母耳記上18\_1-19\_24) - 袁惠鈞牧師[大衛傳系列 - 第3講]} & 2025-01-22 & \href{https://youtube.com/watch?v=9t69tF6ci0k}{\texttt{ 9t69tF6ci0k}} \\
撒母耳記上 20:1-21:15 & \hyperref[sec:rN0dS2BBBmc]{神祝福危難中的謊言嗎?  (撒母耳記上20\_1-21\_15;22\_6-19) - 袁惠鈞牧師[大衛傳系列 - 第4講]} & 2025-02-05 & \href{https://youtube.com/watch?v=rN0dS2BBBmc}{\texttt{ rN0dS2BBBmc}} \\
撒母耳記上 22:1-5-20-23 & \hyperref[sec:WCt7vYrgwVY]{走出憂鬱與黑暗的秘訣 (撒母耳記上22\_1-5,20-23) - 袁惠鈞牧師[大衛傳系列 - 第5講]} & 2025-02-12 & \href{https://youtube.com/watch?v=WCt7vYrgwVY}{\texttt{ WCt7vYrgwVY}} \\
撒母耳記上 23:1-24:22 & \hyperref[sec:GqTOPwqfjwM]{不可伸手害神的受膏者! (撒母耳記上23\_1-24\_22) - 袁惠鈞牧師[大衛傳系列 - 第6講]} & 2025-02-19 & \href{https://youtube.com/watch?v=GqTOPwqfjwM}{\texttt{ GqTOPwqfjwM}} \\
羅馬書 8:28-39 & \hyperref[sec:9ORA5941xxk]{永不能與主的愛隔絕 (羅馬書8\_28-39) - 袁惠鈞牧師[羅馬書系列 - 第22講]} & 2025-01-03 & \href{https://youtube.com/watch?v=9ORA5941xxk}{\texttt{ 9ORA5941xxk}} \\
撒母耳記上使徒行傳 13:14 & \hyperref[sec:w_ajWsBZ9eQ]{大衛:最合神心意的人 (撒母耳記上13\_14, 使徒行傳13\_22) - 袁惠鈞牧師[大衛傳系列 - 第1講]} & 2025-01-10 & \href{https://youtube.com/watch?v=w-ajWsBZ9eQ}{\texttt{ w-ajWsBZ9eQ}} \\
\hline
\end{xltabular}
}
\newpage



\section{撒母耳記上 16:1-17:58}
\label{sec:OtTM_EdQEtA}
\textbf{如何勝過生活中的巨敵 (撒母耳記上16\_1-17\_58) - 袁惠鈞牧師[大衛傳系列 - 第2講]}
\newline
\newline
連結: \href{https://youtube.com/watch?v=OtTM_EdQEtA}{\texttt{ https://youtube.com/watch?v=OtTM\_EdQEtA}} ~~~~ 語音日期: 2025-01-15 
\newline
\newline
\hyperref[sec:wiDRWRXrtjM]{< < < PREV SERMON < < <}
~
\hyperlink{toc}{[返主目錄]}
~
\hyperref[ch:preacher6]{[返講員目錄]}
~
\hyperref[sec:9t69tF6ci0k]{> > > NEXT SERMON > > >}
\newline
\newline
撒母耳記上 16:1-17:58
\newline
\begin{longtable}{cl}
\hline
\hline
章節 & 經文 (和合本修訂版)\\
\hline
16:1 & \begin{tabularx}{0.7\textwidth}{X} 耶和華對撒母耳說:「我既厭棄掃羅作以色列的王,你為他悲傷要到幾時呢?你將膏油盛滿了角;來,我差遣你到伯利恆人耶西那裡去,因為我在他兒子中已看中了一個為我作王的。」 \end{tabularx} \\ \\ \relax
16:2 & \begin{tabularx}{0.7\textwidth}{X} 撒母耳說:「我怎麼能去呢?掃羅一聽見,就會殺我。」耶和華說:「你可以手裡牽一頭小母牛去,說:『我來是要向耶和華獻祭。』 \end{tabularx} \\ \\ \relax
16:3 & \begin{tabularx}{0.7\textwidth}{X} 你要請耶西來一同獻祭,我會指示你當做的事。我對你說的那個人,你要為我膏他。」 \end{tabularx} \\ \\ \relax
16:4 & \begin{tabularx}{0.7\textwidth}{X} 撒母耳遵照耶和華的話去做,來到伯利恆,城裡的長老都戰戰兢兢出來迎接他,有人問他說:「你是為平安來的嗎?」 \end{tabularx} \\ \\ \relax
16:5 & \begin{tabularx}{0.7\textwidth}{X} 他說:「為平安來的,我來是要向耶和華獻祭。你們要使自己分別為聖,來跟我一同獻祭。」撒母耳把耶西和他眾兒子分別為聖,請他們來一同獻祭。 \end{tabularx} \\ \\ \relax
16:6 & \begin{tabularx}{0.7\textwidth}{X} 他們來的時候,撒母耳看見以利押,就心裡說,耶和華的受膏者一定在耶和華面前了。 \end{tabularx} \\ \\ \relax
16:7 & \begin{tabularx}{0.7\textwidth}{X} 耶和華卻對撒母耳說:「不要只看他的外貌和他身材高大,我不揀選他。因為耶和華不像人看人,人是看外貌,耶和華是看內心。」 \end{tabularx} \\ \\ \relax
16:8 & \begin{tabularx}{0.7\textwidth}{X} 耶西叫亞比拿達從撒母耳面前經過,撒母耳說:「耶和華也不揀選他。」 \end{tabularx} \\ \\ \relax
16:9 & \begin{tabularx}{0.7\textwidth}{X} 耶西又叫沙瑪經過,撒母耳說:「耶和華也不揀選他。」 \end{tabularx} \\ \\ \relax
16:10 & \begin{tabularx}{0.7\textwidth}{X} 耶西叫他七個兒子都從撒母耳面前經過,撒母耳對耶西說:「這些都不是耶和華所揀選的。」 \end{tabularx} \\ \\ \relax
16:11 & \begin{tabularx}{0.7\textwidth}{X} 撒母耳對耶西說:「你的兒子都在這裡了嗎?」他說:「還有一個最小的,看哪,他正在放羊。」撒母耳對耶西說:「你派人去叫他來;他若不來這裡,我們必不坐席。」 \end{tabularx} \\ \\ \relax
16:12 & \begin{tabularx}{0.7\textwidth}{X} 耶西就派人去叫他來。他面色紅潤,雙目清秀,容貌俊美。耶和華說:「起來,膏他,因為這就是他了。」 \end{tabularx} \\ \\ \relax
16:13 & \begin{tabularx}{0.7\textwidth}{X} 撒母耳就用角裡的膏油,在他的兄長中膏了他。從這日起,耶和華的靈就大大感動大衛。撒母耳起身回拉瑪去了。 \end{tabularx} \\ \\ \relax
16:14 & \begin{tabularx}{0.7\textwidth}{X} 耶和華的靈離開掃羅,有邪靈從耶和華那裡來擾亂他。 \end{tabularx} \\ \\ \relax
16:15 & \begin{tabularx}{0.7\textwidth}{X} 掃羅的臣僕對他說:「看哪,有邪靈從神那裡來擾亂你。 \end{tabularx} \\ \\ \relax
16:16 & \begin{tabularx}{0.7\textwidth}{X} 我們的主可以吩咐你面前的臣僕,去找一個善於彈琴的來。神那裡來的邪靈臨到你身上的時候,他用手彈琴,你就會感覺爽快。」 \end{tabularx} \\ \\ \relax
16:17 & \begin{tabularx}{0.7\textwidth}{X} 掃羅對臣僕說:「你們給我找一個善於彈琴的,帶到我這裡來。」 \end{tabularx} \\ \\ \relax
16:18 & \begin{tabularx}{0.7\textwidth}{X} 僕人中有一個回答說:「看哪,我曾見伯利恆人耶西的一個兒子善於彈琴,是大能的勇士,說話合宜,容貌俊美,耶和華也與他同在。」 \end{tabularx} \\ \\ \relax
16:19 & \begin{tabularx}{0.7\textwidth}{X} 於是掃羅差遣使者到耶西那裡,說:「叫你放羊的兒子大衛到我這裡來。」 \end{tabularx} \\ \\ \relax
16:20 & \begin{tabularx}{0.7\textwidth}{X} 耶西把幾個餅和一皮袋酒,以及一隻小山羊,馱在驢上,由兒子大衛的手送給掃羅。 \end{tabularx} \\ \\ \relax
16:21 & \begin{tabularx}{0.7\textwidth}{X} 大衛到了掃羅那裡,就侍立在掃羅面前。掃羅很喜歡他,他就作了掃羅拿兵器的人。 \end{tabularx} \\ \\ \relax
16:22 & \begin{tabularx}{0.7\textwidth}{X} 掃羅派人到耶西那裡,說:「讓大衛侍立在我面前,因為他在我眼前蒙了恩寵。」 \end{tabularx} \\ \\ \relax
16:23 & \begin{tabularx}{0.7\textwidth}{X} 從神那裡來的邪靈臨到掃羅身上的時候,大衛就拿琴,用手彈奏,使掃羅舒暢,感覺爽快,那邪靈就離開他了。 \end{tabularx} \\ \\ \relax
17:1 & \begin{tabularx}{0.7\textwidth}{X} 非利士人召集他們的軍隊來爭戰。他們聚集在猶大的梭哥,在梭哥和亞西加中間的以弗‧大憫安營。 \end{tabularx} \\ \\ \relax
17:2 & \begin{tabularx}{0.7\textwidth}{X} 掃羅和以色列人也聚集,在以拉谷安營,擺陣迎戰,要與非利士人打仗。 \end{tabularx} \\ \\ \relax
17:3 & \begin{tabularx}{0.7\textwidth}{X} 非利士人站在這邊的山上,以色列人站在那邊的山上,當中有谷。 \end{tabularx} \\ \\ \relax
17:4 & \begin{tabularx}{0.7\textwidth}{X} 從非利士營中出來一個挑戰的人,名叫歌利亞,是迦特人,身高六肘一虎口。 \end{tabularx} \\ \\ \relax
17:5 & \begin{tabularx}{0.7\textwidth}{X} 他頭戴銅盔,身穿鎧甲,甲重五千舍客勒銅。 \end{tabularx} \\ \\ \relax
17:6 & \begin{tabularx}{0.7\textwidth}{X} 他腿上有銅護膝,兩肩之中背負銅矛。 \end{tabularx} \\ \\ \relax
17:7 & \begin{tabularx}{0.7\textwidth}{X} 他的槍桿粗如織布機的軸,槍頭的鐵重六百舍客勒。有一個拿盾牌的人走在他前面。 \end{tabularx} \\ \\ \relax
17:8 & \begin{tabularx}{0.7\textwidth}{X} 歌利亞站著,對以色列的軍隊喊叫,對他們說:「你們出來擺陣作戰是為了甚麼呢?我不是非利士人嗎?你們不是掃羅的僕人嗎?你們選一個人出來,叫他下來到我這裡吧。 \end{tabularx} \\ \\ \relax
17:9 & \begin{tabularx}{0.7\textwidth}{X} 他若能與我決鬥,把我殺死,我們就作你們的奴隸;我若勝了他,把他殺死,你們就作我們的奴隸,服事我們。」 \end{tabularx} \\ \\ \relax
17:10 & \begin{tabularx}{0.7\textwidth}{X} 那非利士人又說:「我今日向以色列的軍隊罵陣。你們叫一個人出來,跟我決鬥吧。」 \end{tabularx} \\ \\ \relax
17:11 & \begin{tabularx}{0.7\textwidth}{X} 掃羅和以色列眾人聽見非利士人這些話就驚惶,非常害怕。 \end{tabularx} \\ \\ \relax
17:12 & \begin{tabularx}{0.7\textwidth}{X} 大衛是猶大伯利恆的以法他人耶西的兒子,耶西有八個兒子。在掃羅的時候,這人年老,在眾人中受敬重。 \end{tabularx} \\ \\ \relax
17:13 & \begin{tabularx}{0.7\textwidth}{X} 耶西最大的三個兒子跟隨掃羅出征。出征的三個兒子名字是:長子以利押,次子亞比拿達,三子沙瑪。 \end{tabularx} \\ \\ \relax
17:14 & \begin{tabularx}{0.7\textwidth}{X} 大衛是最小的,最大的三個兒子跟隨掃羅。 \end{tabularx} \\ \\ \relax
17:15 & \begin{tabularx}{0.7\textwidth}{X} 大衛有時離開掃羅,回伯利恆為他父親放羊。 \end{tabularx} \\ \\ \relax
17:16 & \begin{tabularx}{0.7\textwidth}{X} 那非利士人早晚都出來站著,共四十日。 \end{tabularx} \\ \\ \relax
17:17 & \begin{tabularx}{0.7\textwidth}{X} 耶西對他兒子大衛說:「你拿一伊法烘了的穗子和十個餅,跑到營裡去,交給你的哥哥, \end{tabularx} \\ \\ \relax
17:18 & \begin{tabularx}{0.7\textwidth}{X} 再拿這十塊奶餅,送給他們的千夫長,並要問你哥哥好,向他們要個憑據回來。」 \end{tabularx} \\ \\ \relax
17:19 & \begin{tabularx}{0.7\textwidth}{X} 掃羅和大衛的三個哥哥,以及以色列眾人,都在以拉谷與非利士人打仗。 \end{tabularx} \\ \\ \relax
17:20 & \begin{tabularx}{0.7\textwidth}{X} 大衛早晨起來,把羊交託一個看守的人,照耶西所吩咐的帶著食物去了。到了軍營,軍隊剛出到戰場,吶喊叫陣。 \end{tabularx} \\ \\ \relax
17:21 & \begin{tabularx}{0.7\textwidth}{X} 以色列人和非利士人都擺列陣勢,彼此相對。 \end{tabularx} \\ \\ \relax
17:22 & \begin{tabularx}{0.7\textwidth}{X} 大衛把東西留在看守物件的人手中,跑到戰場,問他哥哥好。 \end{tabularx} \\ \\ \relax
17:23 & \begin{tabularx}{0.7\textwidth}{X} 他與他們說話的時候,看哪,那挑戰的人,就是迦特的非利士人歌利亞,從非利士隊伍中上來,說了同樣的話,大衛聽見了。 \end{tabularx} \\ \\ \relax
17:24 & \begin{tabularx}{0.7\textwidth}{X} 以色列眾人看見那人就非常害怕,從他面前逃跑。 \end{tabularx} \\ \\ \relax
17:25 & \begin{tabularx}{0.7\textwidth}{X} 以色列人說:「這上來的人你看見了嗎?他上來是要向以色列人罵陣。若有人能殺他,王必賞賜他大財,將自己的女兒嫁給他,並在以色列人中免除他父家納糧服役。」 \end{tabularx} \\ \\ \relax
17:26 & \begin{tabularx}{0.7\textwidth}{X} 大衛對站在旁邊的人說:「若有人殺這非利士人,除掉以色列人的羞辱,他會怎樣呢?這未受割禮的非利士人是誰,竟敢向永生神的軍隊罵陣!」 \end{tabularx} \\ \\ \relax
17:27 & \begin{tabularx}{0.7\textwidth}{X} 百姓照同樣的話對他說:「若有人殺了那人,必這樣待他。」 \end{tabularx} \\ \\ \relax
17:28 & \begin{tabularx}{0.7\textwidth}{X} 大衛的長兄以利押聽見大衛與他們所說的話,就向他發怒,說:「你下來做甚麼呢?在曠野的那幾隻羊,你交託誰了呢?我知道你的驕傲和你心裡的惡意,你下來只是為了看戰爭!」 \end{tabularx} \\ \\ \relax
17:29 & \begin{tabularx}{0.7\textwidth}{X} 大衛說:「我現在做了甚麼呢?只是問一句話也不可以嗎?」 \end{tabularx} \\ \\ \relax
17:30 & \begin{tabularx}{0.7\textwidth}{X} 大衛離開他轉向別人,問了同樣的事,百姓也照先前的話回答他。 \end{tabularx} \\ \\ \relax
17:31 & \begin{tabularx}{0.7\textwidth}{X} 有人聽見大衛所說的話,就在掃羅面前報告;掃羅就派人叫他來。 \end{tabularx} \\ \\ \relax
17:32 & \begin{tabularx}{0.7\textwidth}{X} 大衛對掃羅說:「人不必因那非利士人灰心。你的僕人要去與他決鬥。」 \end{tabularx} \\ \\ \relax
17:33 & \begin{tabularx}{0.7\textwidth}{X} 掃羅對大衛說:「你不能去與那非利士人決鬥,因為你年紀太輕,他從小就是戰士。」 \end{tabularx} \\ \\ \relax
17:34 & \begin{tabularx}{0.7\textwidth}{X} 大衛對掃羅說:「你僕人為父親放羊,有時獅子來了,有時熊來了,從群中抓走一隻羔羊。 \end{tabularx} \\ \\ \relax
17:35 & \begin{tabularx}{0.7\textwidth}{X} 我就追趕牠,擊打牠,把羔羊從牠口中救出來。牠起來攻擊我,我就揪牠的鬍子,打死牠。 \end{tabularx} \\ \\ \relax
17:36 & \begin{tabularx}{0.7\textwidth}{X} 你僕人曾打死獅子和熊,這未受割禮的非利士人必像獅子和熊一樣,因為他向永生神的軍隊罵陣。」 \end{tabularx} \\ \\ \relax
17:37 & \begin{tabularx}{0.7\textwidth}{X} 大衛又說:「耶和華救我脫離獅子和熊的爪,他必救我脫離這非利士人的手。」掃羅對大衛說:「你去吧!耶和華必與你同在。」 \end{tabularx} \\ \\ \relax
17:38 & \begin{tabularx}{0.7\textwidth}{X} 掃羅把自己的戰衣給大衛穿上,將銅盔戴在他頭上,又給他穿上鎧甲。 \end{tabularx} \\ \\ \relax
17:39 & \begin{tabularx}{0.7\textwidth}{X} 大衛佩刀在戰衣上,試著走走看。因大衛沒有試過,就對掃羅說:「我穿戴這些不能走路,因為我沒有試過。」於是他脫下身上的這些軍裝。 \end{tabularx} \\ \\ \relax
17:40 & \begin{tabularx}{0.7\textwidth}{X} 他手中拿杖,又在溪中挑選了五塊光滑的石子,放在袋裡,就是牧人帶的囊裡,手裡拿著甩石的機弦,迎向那非利士人。 \end{tabularx} \\ \\ \relax
17:41 & \begin{tabularx}{0.7\textwidth}{X} 那非利士人漸漸走近大衛,拿盾牌的人在他前面。 \end{tabularx} \\ \\ \relax
17:42 & \begin{tabularx}{0.7\textwidth}{X} 非利士人觀看,見了大衛,就藐視他,因為他年輕,面色紅潤,容貌俊美。 \end{tabularx} \\ \\ \relax
17:43 & \begin{tabularx}{0.7\textwidth}{X} 非利士人對大衛說:「你拿著杖到我這裡來,我豈是狗嗎?」非利士人就指著自己的神明詛咒大衛。 \end{tabularx} \\ \\ \relax
17:44 & \begin{tabularx}{0.7\textwidth}{X} 非利士人又對大衛說:「來吧!我要把你的肉給空中的飛鳥和田野的走獸。」 \end{tabularx} \\ \\ \relax
17:45 & \begin{tabularx}{0.7\textwidth}{X} 大衛對非利士人說:「你來攻擊我,是靠著刀槍和銅矛,但我來攻擊你,是靠著萬軍之耶和華的名,就是你所辱罵、帶領以色列軍隊的神。 \end{tabularx} \\ \\ \relax
17:46 & \begin{tabularx}{0.7\textwidth}{X} 今日耶和華必將你交在我手裡。我必殺你,砍下你的頭,今日我要把非利士軍兵的屍體給空中的飛鳥和地上的野獸,使全地的人都知道以色列中有神, \end{tabularx} \\ \\ \relax
17:47 & \begin{tabularx}{0.7\textwidth}{X} 又使這裡的全會眾知道,耶和華使人得勝,不是用刀用槍,因為戰爭全在乎耶和華。他必將你們交在我們手裡。」 \end{tabularx} \\ \\ \relax
17:48 & \begin{tabularx}{0.7\textwidth}{X} 那非利士人起來,迎向大衛,走近前來。大衛急忙往戰場,迎向非利士人跑去。 \end{tabularx} \\ \\ \relax
17:49 & \begin{tabularx}{0.7\textwidth}{X} 大衛伸手入囊中,從裡面掏出一塊石子來,用機弦甩去,擊中非利士人的前額,石子進入額內,他就仆倒,面伏於地。 \end{tabularx} \\ \\ \relax
17:50 & \begin{tabularx}{0.7\textwidth}{X} 這樣,大衛用機弦和石子勝了那非利士人,擊中了他,把他殺死;大衛手中沒有刀。 \end{tabularx} \\ \\ \relax
17:51 & \begin{tabularx}{0.7\textwidth}{X} 大衛跑去,站在那非利士人身旁,把他的刀從鞘中拔出來,殺死他,用刀割下他的頭。非利士眾人看見他們的勇士死了,就都逃跑。 \end{tabularx} \\ \\ \relax
17:52 & \begin{tabularx}{0.7\textwidth}{X} 以色列人和猶大人就起來吶喊,追趕非利士人,直到該和以革倫的城門。被殺的非利士人倒在路上,從沙拉音直到迦特和以革倫。 \end{tabularx} \\ \\ \relax
17:53 & \begin{tabularx}{0.7\textwidth}{X} 以色列人追趕非利士人回來,搶奪了他們的軍營。 \end{tabularx} \\ \\ \relax
17:54 & \begin{tabularx}{0.7\textwidth}{X} 大衛拿著那非利士人的頭帶到耶路撒冷,卻把那非利士人的軍裝放在自己的帳棚裡。 \end{tabularx} \\ \\ \relax
17:55 & \begin{tabularx}{0.7\textwidth}{X} 掃羅看見大衛去迎戰非利士人,就問押尼珥元帥說:「押尼珥,那年輕人是誰的兒子?」押尼珥說:「王啊,我在你面前起誓,我不知道。」 \end{tabularx} \\ \\ \relax
17:56 & \begin{tabularx}{0.7\textwidth}{X} 王說:「你可以問問那孩子是誰的兒子。」 \end{tabularx} \\ \\ \relax
17:57 & \begin{tabularx}{0.7\textwidth}{X} 大衛打死那非利士人回來,押尼珥領他到掃羅面前,大衛手中拿著非利士人的頭。 \end{tabularx} \\ \\ \relax
17:58 & \begin{tabularx}{0.7\textwidth}{X} 掃羅問他說:「年輕人,你是誰的兒子?」大衛說:「我是你僕人伯利恆人耶西的兒子。」 \end{tabularx} \\ \\
[1ex]
\hline
\hline
\end{longtable}
$^{1}$我們上次已經開始了大衛轉系列.
一個新的系列.
我們上次是講了大衛的介紹.
今天是要講三號以上的十六章和第十七章.
我給的題目是如何勝過我們生活中的巨敵.
我們每個人的生活當中都有些強敵有些巨敵.
我們需要克服.
今天就是透過大衛的故事讓我們看到.
我們如何戰勝我們生活中的敵人.
首先我們需要重溫一下薩姆爾上.
因為我們很久沒有講薩姆爾上記.
上次我們講到以色列人已經拒絕了薩姆爾.
成為他們的領袖.
說薩姆爾的年紀已經很大.
他們兩個兒子又不爭氣.
人民是希望蘇羅能夠成為他們的王.
但很可惜蘇羅屢次違反了神的命令.
所以他失去了從神而來的王權.
他雖然仍然坐在他的王位.
但神已經棄絕了他.
薩姆爾也跟他斷絕了關係.
薩姆爾為了這件事非常悲傷.
非常哀動.
聖經裡告訴我們.
哀動有時是傳道書的三章四節.
但行動也有時是約書亞記的七章第十節.
神給薩姆爾的工作還未完結.
神要薩姆爾高立新王.
這就是耶西的兒子大衛.
蘇羅是人民揀選的王.
但大衛是神所揀選.
在這兩章的經文我們清楚看到.
大衛真是神所預定的君王.
所以第一點就是神所預定的大衛.
在薩姆爾上的十六章第一節.
耶和華對薩姆爾說.
我既厭棄蘇羅作以色列的王.
你為他悲傷要到幾時呢?.
你將高遊盛滿了國.
我差遣你往伯尼行人耶西那裡去.

$^{41}$因為我在他眾子之內.
預定一個作王的.
有沒有看到「預定」這個字?.
「預定」這個字在舊約一共出現了1300次.
但只有四次被翻譯為「預定」.
其餘的1290多次.
大部分時間都是翻譯為「漢見」這個字.
在今天的經文.
薩姆爾上的十六章和十七章.
我們看到「漢見」這個字重複出現11次.
很清楚作者要讓我們看到有兩方面.
神是靠著祂的主權來預定一切.
人是靠我們的眼所看見的.
這就是作者有心在這兩章給我們看到的對比.
在創世記的二十二章13至14節.
說阿伯拉罕叫神做耶和華以納.
翻譯出來就是耶和華必有預備.
這個「預備」這個字就是我們今天的「預定」這個字.
可以解釋為「意象」.
也可以解釋為「漢見」.
用在神身上.
神就是預定這一切.
神預定一個人來接替掃羅.
這個就是大衛.
其實這個和我們剛剛說完羅馬書第八章很相似.
如果大家記得的話.
我剛剛才說完在羅馬書第八章.
神預知的就是祂預定的.
神預定的就是祂預知的.
這個就是我們今天所說的.
薩姆爾要去高納大衛.
但他有一個顧慮.
就是掃羅是一個很多疑的人.
薩姆爾要從他的家鄉拉馬去到伯利恆的地方.
要經過掃羅的家鄉基比亞.
如果掃羅知道神差派薩姆爾去高納新王.
掃羅一定會殺死薩姆爾.
所以神教.
薩姆爾說你帶一隻公牛犢.
去到伯利恆獻給神.

$^{81}$然後你請耶西來一起吃祭肉.
耶西就會帶他的兒子來.
我就會告訴你.
哪一個兒子我將會高納他成為王.
有人說.
神是不是教薩姆爾說謊.
聖經說不可以說謊.
薩姆爾有沒有說謊?沒有.
你不說出來的.
沒有人說你啞.
但如果你說出來不是屬實的話.
那你就是說謊.
所以神只是告訴薩姆爾.
你去到伯利恆獻祭.
請耶西和他的兒子來吃祭肉.
完全沒有違反聖經的教導.
我剛才說神預定.
但人是漢見.
人看的就是看外表.
在獻祭的時候.
耶西的六個兒子.
首先來到.
薩姆爾看到他們.
就在想誰才是神所揀選的呢?.
見到長子以利亞高大威猛.
薩姆爾又來了.
他說這個一定是神所揀選的.
他是憑眼見不是憑信心行事.
記不記得他見到蘇羅就犯了同樣的錯誤.
蘇羅高大威猛英俊.
高過別人一個頭.
這個還不是以色列人的王.
他現在又來了.
神尋求的是合他心意的人.
我們上次說大衛的介紹的時候.
我也說過什麼叫做合神心意的人.
我們說大衛做了很多事.
威神做了很多事.
其實我們今天可以給合神心意的人一個定義.
就是神所喜愛的東西.

$^{121}$這個人也喜愛.
神所憎恨的東西.
這個人也會憎恨.
就是這個人將神的優先次序.
成為自己的優先次序.
就好像歷代志下十六章九節所說.
神不是看外表.
神是揀選內心.
薩姆爾上的六章七節.
耶和華不像人看人.
人是看外貌.
耶和華是看內心.
神是監察人心.
神是知道人的動機.
在想什麼.
雖然聖經告訴我們.
外貌對神來說並不重要.
但是大衛的外貌的確很吸引人.
在聖經告訴我們.
他是面色光紅.
面色光紅就像耳塑.
舊約的耳塑.
他的皮膚也很紅.
耳塑還有紅頭髮.
所以很多人認為.
大衛也是紅頭髮.
但這裡不是說他的頭髮.
這裡是說他的面色.
說他比一般猶太人的面色更紅潤.
的確大衛是有迷人的個性.
在薩姆爾上的十六章十八節.
他也是很得人心.
所以有很多人會為他賣力.
大衛就是耶西第八個兒子.
一共有八兄弟.
但聖經裡只說到他有六個兄弟的名字.
如果有八兄弟的話.
應該有七個兄弟的名字.
聖經只有六個兄弟的名字.
除了長子以利亞.

$^{161}$其他的兄弟都不太重要.
所以我不會讀出他的兄弟的名字.
很有趣.
為什麼七個兄弟會只有六個名字呢?.
相信其中一個已經死了.
因為他沒有後裔.
所以從家譜中除了他的名字.
大衛還有兩個姐妹.
就是洗老亞和阿比蓋.
洗老亞就是亞比西,約納和阿薩克的媽媽.
記不記得我上次和你們說過.
大衛和約納的關係很特別.
約納是大衛的元帥.
但他有時就好像忠心大衛.
但有時就好像不顧大衛的利益一樣.
這個就是約納.
他就是洗老亞的兒子.
還有亞比蓋就是亞瑪薩克的媽媽.
這些人名可能對你們來說.
現在覺得不太熟悉.
但到我們說大衛的故事的時候.
你就會看到這些人都在大衛王朝中.
是很重要的角色.
當薩姆爾在想.
哪個耶西的兒子才是將來以色列人的王.
他不知道原來大衛還在目緊仰.
沒有在場出現.
是後來大衛才來到.
我們說神是揀選內心.
不是看外表.
神是揀選怎樣內心的人.
首先神是揀選有牧者心腸的人.
在舊約君王和官吏.
很多時都被比喻為牧羊人.
大衛就是有牧羊人的心.
今日的教會都是一樣.
屬靈的領袖應該有牧羊人的心.
不是用重價,公價請回來的牧羊人.
而是一個真正愛羊.
願意為他的羊付出的人.

$^{201}$牛是可以被鞭策來被趕的.
你可以在牛的後面用一條鞭來鞭牛.
那你就可以叫牛向一個方向走.
但羊是不能夠鞭策的.
羊是要在牠們的前面帶領牠們.
不可以在後面.
羊是認識牠們的牧羊人.
知道牧羊人愛牠們.
牠們就會跟著牧羊人.
羊不懂得自慰.
也看東西不清楚.
我時時都說為何馬戲班裡.
從來不會見到有隻羊在做馬戲.
就是羊真的不是那麼聰明.
一個真正的牧羊人就是會帶領牠的羊.
大衛就是一個真正的牧羊人.
當牧羊做得好的時候.
神就給整個國家牧羊.
詩篇的23篇就是有關羊和牧羊人的關係.
詩篇23篇我們見到不是一個年青的牧羊人所寫的.
是大衛年老的時候回顧一生.
他在詩篇23篇第一節.
耶和說:是我的牧者,我必不自缺乏.
大衛正是神所揀選的以色列人的領袖.
他能夠彌補掃羅所做出來的一切傷害.
除了一個被神揀選的人.
他有牧者的心腸.
他也有牧人的心腸.
在馬太福音25:21.
神說:你這又良善又忠心的牧人.
你在不多的事上有忠心.
我要把許多的事派你管理.
可以盡來享受你主人的快樂.
大衛作為神的牧人.
他在小事上忠心.
你說什麼小事上忠心?.
神叫他做一個牧羊人.
他就做一個很好的牧羊人.
當他在小事上忠心.
神就給整個以色列國來管理.

$^{241}$他和掃羅不同的地方是什麼?.
大衛曾經在掃羅的權力之下.
他珍惜權力.
他知道真正掌權的是神.
他等待神.
但掃羅一得勢之後.
他就不可一世.
他好像變了一個人一樣.
他以前好像頗為淑女.
但當他成為王之後.
他就改變了 變了一個人.
這就是大衛和掃羅不同的地方.
大衛這個名字的意思.
我上次和大家說過.
就是被愛的那一個.
他的確是被神所愛.
他也是第八個兒子.
在聖經裡面.
八代表一個新的開始.
因為一個星期有多少天?.
七天.
第八天就是一個新的開始.
所以的確神運用大衛.
無論在政府或人民的屬靈上.
都是叫大衛幫他們帶來一個新的開始.
在聖經裡面.
除了君王,先知和祭司.
都是被高納的.
神差派的人來高納.
這些先知,祭司,君王.
都是神差派來的.
在聖經裡面.
他們是會用油來高納.
油代表什麼?.
就是代表聖靈.
或者代表從聖靈而得到的力量.
有一位很出名被高納的那一位是誰?.
是耶穌基督.
在希伯來文.
在舊約.

$^{281}$彌賽亞就是受高者.
但同一個字在希臘文.
在新約就是基督這個字.
所以舊約的彌賽亞.
新約的基督都是受高者.
神大大地差派他的靈.
臨到大衛的身上.
當聖靈臨到大衛的身上.
聖靈也同時離開了掃羅.
我和大家說過.
在舊約和新約不同的地方.
就是舊約聖靈能夠離開一個人.
大衛也知道.
所以日後當他和拔士巴犯了奸淫.
得罪了神之後.
他在詩篇的51篇第11節.
他說不要刁棄我.
使我離開你的面.
不要從我收回你的聖靈.
大衛知道神可以收回他的聖靈.
就好像在掃羅的身上.
神就是將掃羅身上的聖靈收回.
在新約就不同了.
聖靈是入住一個信徒.
是永遠不會離開.
這個也是我和大家.
新近說羅馬書的時候所說過.
所以我們見到.
新約和舊約很一致.
說的話其實都很相似.
因為同樣有一位聖靈在後面.
這個也是原因我著重用聖經解釋.
因為這才是唯一正確解經的方法.
在羅馬書第八章我說過.
一個真正得救的人.
是有聖靈在他們的身上.
而聖靈會帶領他們完成神的救贖.
剛剛羅馬書第八章我們才說過.
如果不是一個真正得救的人.
是沒有聖靈在他們的身上.

$^{321}$一個人如果沒有聖靈的能力.
是不能夠遵守神的旨意.
不能夠榮耀神.
這就是耶穌基督在約翰福音15:5說.
「離開了我,你們就不能做什麼」.
離開了耶穌基督的靈.
我們不能做任何的事.
最後在罰尼恆大衛出現.
撒慕爾知道這是神所揀選的.
就高立了大衛.
這次的高立不是公開的.
是私底下.
在私底下撒慕爾告訴大衛.
他將成為下一任以色列人的王.
如果是這樣的話就非常難得.
因為我們知道後來大衛還要侍奉訴羅.
如果大衛知道自己將來是以色列人的王.
和訴羅是平起平坐.
為什麼我還要侍奉你呢?.
但是我們看到大衛是怎樣呢?.
他是一個非常服從謙卑的人.
他仍然願意服從.
願意侍奉訴羅.
神不單止預備揀選大衛.
神也準備大衛.
大衛在他媽媽的肚子裡.
神已經認識他.
詩篇139.
這和新約的保羅很相似.
保羅在他媽媽的肚子裡.
神已經將他分別為聖.
能成為神的特別器皿.
就是加拉太書的一章15節.
大衛就是神所預備帶領以色列人.
帶領神的子民.
我們剛才說過.
在《撒母耳上》16章17章.
看見和預定這個字是同一個字.
現在又有第三個翻譯.
都是同一個字.

$^{361}$這個字這次被翻譯為「找」.
尋找.
你說尋找什麼呢?.
因為聖靈離開了訴羅.
就有惡魔從神那裡擾亂訴羅.
每次擾亂他的時候.
他都很不安.
所以他希望找一個人.
彈琴給自己聽.
能夠幫助他舒服一點.
能夠安慰他.
這個就是《撒母耳上》16章17節.
訴羅對神說.
你們可以為我找一個善於彈琴的.
帶到我這裡來.
訴羅叫他的僕人去找一個.
懂得彈琴的人.
他的僕人就靠自己的能力去找.
所以我說16章和17章.
給我們看見神的主權和人的無能.
我們只憑眼見.
只憑我們認為是好的.
但是神已經預定了.
雖然是訴羅的僕人去找到大衛.
但其實是神早就預定大衛去幫助訴羅.
因為訴羅叛逆違背神.
所以神就容許他有精神的困擾.
也是作為他管教的一部分.
訴羅為什麼會有邪靈上他的身?.
因為他是一個妒忌的人.
他的報復心很強.
所以才有機會被邪靈進入他的心.
有人會問.
基督徒會不會有邪靈進入我們的心上我們的身?.
沒有的.
為什麼?.
因為一個真正的基督徒是有聖靈在我們裡面.
如果邪靈要進來的話.
要住在我們裡面.
聖靈會不會容許?.

$^{401}$聖靈絕對不會容許.
而聖靈一定能夠勝過邪靈.
邪靈只不過是墮落的天使.
但聖靈是神自己.
所以沒有這個可能.
這個也是有很多牧師.
尤其是中國人的牧師經常都搞錯.
他們經常說某弟兄姊妹鬼上身了.
我要去驅魔.
真正的基督徒不會鬼上身.
你說不是的牧師.
這個人真的好像鬼上身一樣.
有兩個可能性.
一個可能性就是這個人根本不是基督徒.
他有可能會鬼上身.
還有一個可能性就是他不是真正鬼上身.
可能他精神上的困擾有些問題.
不一定是鬼上身.
所以我們要很小心.
一個真正的基督徒絕對不會被鬼上身.
因為我們有聖靈在我們裡面.
在這個時候.
撒慕爾上的十六章十八節.
其中有一個少年說.
我曾見伯利恆人耶西的一個兒子.
善於彈琴.
是大有勇敢的戰士.
說話俠義.
容貌俊美.
耶和華也與他同在.
有沒有看到這裡說大衛容貌俊美.
大衛我們上次介紹的時候說過.
他是一個詩人.
是一個音樂家.
他擅長彈樹琴.
他還能夠作很多的詩歌出來.
他被譽為是以色列的美歌者.
撒慕爾下的二十三章第一節.
但是經文說他什麼.
同時他是一個勇敢的將領.

$^{441}$很難得.
文武相傳.
又可以作詩彈琴.
又可以在戰場上作戰.
他為什麼能夠兩者俱備.
經文有告訴我們.
因為耶和華與他同在.
撒慕爾上的十六章十八節.
十八章十二十四和二十八節都有說到.
耶和華與他同在.
亦都是約瑟.
舊約的約瑟.
還有約書亞.
撒慕爾的秘訣.
他們都有神與他們同在.
以致他們能夠很成功.
有人說我是基督徒.
我時時都有聖靈與我同在.
我天下無敵.
不是的.
所以在聖經裡時時都教我們.
要被聖靈來充滿.
要被聖靈來帶領.
是我們基督徒有聖靈在我們身上.
但很多基督徒不聽聖靈的話.
不讓路給聖靈.
這就是聖經裡所說的.
令到聖靈擔憂.
甚至令到聖靈消滅.
聖靈是不會離開我們.
但如果你不給他掌權的話.
你就不能得到從聖靈而來的力量.
大衛就是被聖靈來引領他.
他每做一個決定.
他都希望符合神的旨意.
他知道自己的恩賜是什麼.
他就用他的恩賜來榮耀神.
他服從神.
他內心等候的那個.
神已經允許他.

$^{481}$但未得到的王位.
就好像詩篇的二十三篇.
以及以弗所書的三章都說.
神是為自己的名來引導他.
一步一步行而路.
大衛最初不是住在.
素羅的軍營裡.
最初他是住在軍營的外面.
但因為素羅時時都會有邪靈上身.
需要大衛彈琴安慰他.
所以慢慢大衛就住進軍營裡.
而後來當他殺死了哥利亞之後.
他就成為素羅拿兵器的人.
拿兵器的人.
只是凱里爾菲 林紀.
其實拿兵器的人都很重要.
你想想一個將軍.
將他的兵器給你保管.
用什麼突然用他的武器殺死將軍.
有些像是以前的王.
都有酒精和善精.
這些人都很重要.
你不要以為他們只管住王的飲食.
他們也要試過王的食物有沒有毒.
最容易毒死王的就是這兩個人.
酒精和善精.
所以他們都很需要王信不過他.
拿兵器的人都是一樣.
就在這個時候.
在撒母耳上第十七章.
看見這個字又再出現.
這次看見什麼呢.
是非利士人來攻擊以色列人.
他們害怕到看見非利士人就逃跑.
又一次我們看到.
神的主權和人的軟弱和無奈.
人憑眼見就怎樣.
首先看外表.
然後看到巨大的敵人就會怎樣.
害怕到不得了 逃跑.

$^{521}$這令我想起在列王記下的第六章.
你記不記得 以利沙先知.
他和他的僕人在城裡.
城是被敵人圍住.
他的僕人很害怕.
以利沙說你不用害怕.
求神打開僕人的眼睛.
結果怎樣.
看見萬軍的耶和華的天兵天將.
保護著城 害怕都不用害怕.
這就是這張書告訴我們.
我們每一個人都要從神的眼光來看事.
而不是靠我們自己無能的眼光.
那些非利士人來攻打以色列人.
在什麼地方呢 是一個山谷.
我們知道非利士人是著名鑄鐵的.
做了很多戰車.
但他們的戰車在山谷有什麼用.
無用武之地.
所以這班人就想出一個計謀.
你們選一個人出來.
我們又選一個人出來.
兩個人來搏鬥.
誰贏了就勝利.
對方就要投降.
他們為什麼會想出這樣的計謀.
因為他們有個巨人哥利亞.
有他們的好處.
所以想出這樣的方法.
這些對陣方式都是舊弱 新弱.
這個又是羅馬書.
我最近和你們說過.
新弱都有的.
有什麼好處出現呢.
記得羅馬書的五章.
我們要不就屬耶穌基督.
要不就屬亞當.
屬耶穌基督就是屬靈的.
屬亞當就是屬血氣屬肉體的.
你要選誰做你的領袖.

$^{561}$當然選耶穌基督.
就好像菲利士人.
他們選擇哥利亞.
結果就選錯了.
在經文告訴我們.
哥利亞是六爪高又有一虎口.
一虎口就是一隻手那麼長.
一爪就是一尺半.
你聰明一點.
大概是九尺九寸.
巨人來的.
他的盔甲是125磅.
我最近去買一些重量.
綁在我的腳上.
打算做運動.
幫我練一下兩隻腳.
我看到亞馬遜.
那些重量五磅.
小事一樁.
我還在想有沒有十磅.
五磅那麼輕.
五磅買回來.
一隻腳綁五磅.
居然走不了.
我要由一磅一磅來訓練自己.
你說背著一件125磅的盔甲.
我想我會跌在地上.
不需要打都已經輸了.
他的長矛重是多少.
17磅那麼重.
哥利亞是什麼.
是巨人的後裔.
巨人是什麼.
在創世記第六章叫做偉人.
而在七十字譯本.
是翻譯成巨人.
原文這個字的意思是什麼.
是墮落者.
很有意思.
所以我相信.

$^{601}$這些都是墮落的天使.
和人所長出來.
才有那麼高大.
我講過一篇道.
叫做從聖經看UFO和外星人.
這一篇道裡面.
如果真的有外星人的話.
絕對他們就是墮落的天使.
偉人這個字在聖經裡面.
一共出現兩次.
第一次就是創世記第六章.
第二次就是民數記十三章33節.
記不記得十二個探子進入迦南地.
來窺探那個地方.
他們就見到有很多巨人.
他們很害怕.
後來約書亞帶領以色列人進入迦南地.
戰勝了當地的人.
毀滅了那些巨人.
尤其是那些巨人族叫做阿納族人.
再沒有阿納族人了.
告訴我們.
但是約書亞記裡面的十一章.
二十一至二十二節.
告訴我們.
在迦塔和阿薩達.
仍然有巨人留下.
而這個哥利亞在哪裡的人呢?.
就是在迦塔這個地方.
哥利亞來到以色列人那裡.
罵陣四十天.
在聖經裡面四十代表什麼?.
是一個試探的期.
是一個試驗的期.
記不記得耶穌基督在曠野被撒旦.
試探多少天?.
四十天.
現在就是有哥利亞來試探以色列人.
而大衛去到的時候是最後的一天.
是耶西派他的兒子去到戰場.

$^{641}$來給食物給三個哥哥和軍隊的元帥.
我們看到有兩件事.
第一大衛非常服從他爸爸.
雖然他已經知道自己是將來以色列人的王.
他仍然服從他爸爸.
叫他送飯他就去送飯.
第二我們看到什麼?.
是神的時間.
在這個時間猜派大衛去到戰場.
就在這個時候.
蘇羅發了一個命令.
重賞之下必有勇夫.
任何人能夠殺死哥利亞的話.
我就將我一個女兒嫁給他.
他還能夠得到很大的財富.
還能夠免稅免疫.
就在這個時候大衛就出現.
神不單止預定揀選大衛.
不單止準備大衛.
首先叫他做一個牧羊人.
還有神是使用大衛來戰勝以色列人的敵人.
在這個時候.
以利亞就是大衛的哥哥長子.
他怎麼說?.
在撒姆以上的十七章二十八節.
你下來作什麼呢?.
在曠野的那幾隻羊.
你交託了誰呢?.
我知道你的驕傲和你心裡的惡意.
你下來特為要看真戰.
看這個字又出現了.
我知道你就是想來看熱鬧.
有沒有看到他哥哥多麼藐視他的弟弟.
在曠野的那幾隻羊.
你都看不懂.
你還走來這裡看熱鬧.
你快點回來看你的羊吧.
我們看到每當我們為主作供的時候.
有很多人會敵對我們.
會反對我們.

$^{681}$有什麼人會敵對我們反對我們呢?.
首先就是我們的家人.
大衛的哥哥以利亞嘲笑他.
還有什麼例子在聖經裡.
舊約的約瑟.
記不記得他的兄弟很憎恨他.
甚至賣了他去埃及.
約瑟做了什麼呢?.
他只是把他的夢告訴他哥哥.
他的夢是從神而來的啟示.
神給他啟示.
他只是把這些啟示告訴他哥哥.
他哥哥就非常生氣.
在說創世記的三十七章.
還有摩西的姐姐米利安.
她的哥哥亞倫.
都是不忿摩西妒忌他.
你以為只有神用理論傳達他的話嗎?.
我們一樣可以.
只是毀謗摩西.
耶穌基督自己的家人都不理解他.
都反對他的事工.
耶穌基督在《馬太福音》的十章三十六節.
他說人的仇敵就是自己家裡的人.
很多時候當我們要為主作供的時候.
我們的敵人就是我們自己家裡的人.
還有呢?.
領袖都會輕視他.
大衛的領袖是誰呢?.
是蘇羅.
蘇羅當他有聖靈降臨在他身上.
他曾經是北聖.
但是因為他叛逆神.
所以聖靈離開了.
他失去能力.
甚至失去信心.
所以他不鼓勵大衛去挑戰哥利亞.
他說你這麼年輕.
你不可以做的.
你做不到的.

$^{721}$蘇羅的反應和當年十二個探子當中.
十個探子是否很相似呢?.
都是見到巨人.
十個探子說.
喂!不可以進去的.
加南地有巨人的.
我們打不贏.
蘇羅也和大衛說.
你不可以的.
你眼前的是巨人.
你這麼年輕.
你沒用的.
但是我們注意大衛怎樣呢?.
有沒有去辯駁?.
大衛沒有辯駁.
大衛知道真正掌權的是神.
他不需要去辯駁.
他知道他真正的敵人是誰.
除了家人.
除了領袖輕視.
敵人也都藐視他.
哥利亞指著自己的神.
詛咒大衛.
又對大衛說.
萊巴,我將你的肉吸空中的飛鳥.
填飲的酒受吃.
這就是在《述十武》以上的十六章.
十七章告訴我們.
當我們只是憑眼見.
從人的角度來看的話.
一切都很灰心.
但我們需要憑著信心.
從神屬靈的眼光來看.
當我們有信心依靠神的時候.
神就會介入.
就會改變我們的處境.
大衛在他自己的生命當中.
他曾經經歷過神的大能.
你說他牧羊而已.
怎樣經歷神的大能呢?.

$^{761}$牧羊的時候見到有獅子.
有紅人.
他就殺死了獅子和紅人.
現在他見到眼前的哥利亞.
他覺得多一隻野獸而已.
他根本不把牠放在眼內.
他知道神會用微小的來戰勝我們的敵人.
大衛用什麼方法來戰勝他的敵人呢?.
一共有三點.
第一是榮耀神的名字.
這個很重要.
因為如果我們要戰勝我們生活當中的巨人.
我們都要學習這三點.
首先就是榮耀神的名字.
大衛對斐理士人說.
你來攻擊我是靠著刀槍銅擊.
我來攻擊你是靠著萬軍之耶和華的名.
就是你怒罵帶領以色列軍隊的神.
你們來攻擊我們是用武器而已.
我來攻擊你們是耶和華的名.
比你們厲害.
大衛奉耶和華的名來攻擊哥利亞.
大衛要哥利亞和斐理士人的軍隊都知道.
耶和華就是以色列人的神.
哥利亞嘲笑以色列人的神.
褻瀆他的名.
他很快就要被平反.
大衛認為這場的真戰是什麼呢?.
是以色列人的真神和斐理士人假神的一場真戰.
所以大衛絕對一點都不怕.
除了他榮耀神的名.
第二點是什麼呢?.
他是要彰顯神的事蹟.
在撒母耳上的十七章四十六節.
今日耶和華必將你交在我的手裡.
我必殺你.
斬你的頭.
將斐理士軍兵的屍首.
吸空中的飛鳥.
地上的野獸吃.

$^{801}$使普天下的人都知道.
以色列中有神.
有沒有看到.
他就是要普天下的人都知道.
以色列中有神.
神不斷要他的子民來傳揚他的名.
為什麼他和阿伯拉罕納約.
在樹上誓幾個十二章.
就是要透過阿伯拉罕使萬國得福.
在以塞俄書的四十二章六節.
他要以色列人成為外邦人的光.
當以色列人寄居在埃及的時候.
神要摩西帶他們出來.
法老王不肯.
於是神就降十災.
為了什麼呢?.
要傳揚神的名.
要傳揚神的榮耀.
出埃及記的九章十六節.
甚至神是打開紅海.
讓以色列人出埃及.
打開約旦河.
讓以色列人進入迦南地.
為了什麼呢?.
要見證以色列人的神是真神.
約書亞記的四章二十三至二十四節.
就算建聖殿.
都是要為了見證以色列人的神.
使外邦人能夠認識他.
能夠敬畏他.
在聶皇紀上的八章四十二至四十三節.
所以神要大衛戰勝哥利亞.
為了什麼呢?.
要彰顯神的名.
彰顯神的事蹟.
你說.
怎樣用大衛來彰顯神的事蹟呢?.
今時今日.
很多非基督徒都知道.
有大衛戰勝哥利亞的故事.

$^{841}$就算你不是基督徒都知道.
為什麼呢?.
就是神要世人知道他的事蹟.
透過大衛.
這其實已經是聖經的預言.
現在很多非基督徒都知道這件事.
還有第三呢?.
大衛要顯明神的力量.
大衛的武器是什麼呢?.
只有幾顆石頭和頭石器.
非利士人的武器呢?.
刀槍劍擊.
光看武器都輸了.
大衛選了五顆光滑的石頭去對抗哥利亞.
有人問.
哥利亞只有一個.
如果大衛真的信靠神的話.
一顆石頭就夠了.
為什麼需要五顆?.
是不是他不夠信心?.
不是的.
遲些你會看到.
在撒母耳夏的第21章.
歷代至上的20章.
原來哥利亞有四個兄弟.
大衛說如果哥利亞全部兄弟出齊.
我也照樣殺死他們.
所以他預備了五顆石頭.
神幫助大衛戰勝哥利亞.
第一顆石頭就打中哥利亞的額頭.
哥利亞就掉到地上.
大衛就拔出他的劍.
將哥利亞的頭割下來.
以色列人得到全面的勝利.
在很多年後.
大衛在他的詩篇第18篇說.
唯有那以力量束我腰.
使我行為完全的.
他是神.
他教導我的手能以真戰.

$^{881}$甚至我的臂膀能開同弓.
大衛仍然紀念他戰勝哥利亞的事蹟.
在這個時候漢建這個字又再出現.
這次是甚麼呢.
是非理士人看到哥利亞死了.
他們就驚惶逃跑.
至少他們跑了有十里.
死了很多人.
跑到哪裡呢.
去到加特就是哥利亞的家鄉和以格倫兩個城市.
可惜非理士人有沒有遵守他們的諾言呢.
沒有遵守.
他們的領袖是打輸了.
但他們沒有投降.
他們只是逃跑.
蘇羅的軍隊得到大大的勝利.
拿了非理士人很多的戰利品.
大衛也是從這個時候開始.
得到勇士的名字.
他就是追趕非理士人.
為甚麼我說從這個時候開始.
他得到勇士的名銜.
記不記得以後我們會看到.
那些人民怎麼說.
蘇羅殺死千千.
大衛呢.
殺死萬萬.
他就是得到他勇士的名銜.
他將哥利亞的劍.
偷偷放在自己的帳幕上.
後來在那帕的回眸.
這把劍出現.
為甚麼偷偷放在帳幕上.
遲一點就去了回眸.
我相信大衛是將這把劍獻給神.
所以出現在那帕的回眸.
不是因為蘇羅得到這場戰爭的勝利.
當年約拿丹的兒子.
戰勝了非理士人.
蘇羅只不過是冷眼旁觀.

$^{921}$他作出了錯誤的決定.
差點連勝利都失去了.
蘇羅這次又冷眼旁觀.
白白看著大衛單槍匹馬.
戰勝了敵人.
蘇羅問亞利爾.
這個年青人的爸爸是誰.
有很多的解經家都拗頭.
為甚麼蘇羅不認識.
大衛時時都彈琴給他聽.
還是他拿兵器的人.
為甚麼蘇羅會不認識他.
其實這裡很簡單.
因為蘇羅曾經下了諾言.
會將自己的女兒嫁給他.
另外這個人會得到財富免稅免疫.
所以我相信蘇羅要知道大衛是屬於哪個家族.
所以才問這個年青人的爸爸是誰.
今天結束的時候.
我們今天的題目是甚麼.
如何戰勝我們生活中的巨敵.
今天的經文.
兩章的經文讓我們看到漢建這個字.
或者神預定這個字.
首先我們要知道的就是.
我們不要靠人的眼光去看事.
我們要靠神屬靈的眼光來看事.
神是有主權的.
神是預定的.
我們看到的會怎樣.
只是看外表我們看到的東西會很害怕.
但你不用害怕.
神是掌權.
撒母耳金屬靈的人仍然看外表.
以為大衛的哥哥就是神所揀選.
但神是揀選大衛成為他合神心意的人.
神說我不像人看人.
是看外貌.
耶和華是看內心的.
這就是第一點.

$^{961}$我們遇見強敵的時候.
我們要從神屬靈的眼光來看事.
第二我們要做好準備.
大衛願意被神用他作為牧羊人來訓練他.
他在小事上忠心.
神就給他更大的事來掌管.
所以我們都很需要做好準備.
第二點是有個願意服從的心.
撒母耳已經高納大衛成為下一任的王.
但大衛仍然願意侍奉蘇羅.
當他的父親叫他送食物給他哥哥.
他也服從.
他完全沒有價值.
就是因為這樣他能戰勝哥利亞.
有沒有看到神的計劃在當中.
所以他們絕對需要服從.
還有就是要勝過我們身邊的挑戰.
當我們為主作供的時候.
有很多人會反對我們.
甚至我們的家人.
甚至可能是教會裡面的領袖.
但我們不用怕.
我們只需要以大局為重.
用神屬靈的眼光來處事.
還有第四點.
是要記得神的恩典.
以往的恩典.
大衛記得神怎樣帶領他戰勝獅子,紅人.
他就是用一樣的手法來戰勝哥利亞.
還有就是依靠耶和華的命.
大衛說我是靠著萬軍耶和華來攻擊你.
第六點是要將榮耀歸給神.
大衛說使普天下的人都知道.
以色列中有神.
最後一點是很重要.
是需要忍耐地等候.
大衛雖然被安納成為以色列人的王.
但神要他繼續訓練他.
所以他要等候.
我們遲些會見到.

$^{1001}$講大衛的故事的時候.
大衛有很多機會可以除去蘇羅.
可以殺了他.
但大衛都沒有做.
大衛願意等候神的時候.
他是有忍耐的等待.
這是第七點.
願意我們能夠像大衛一樣.
能夠戰勝我們生活中的巨人,巨敵.
將榮耀贊成歸給神.
讓我們低頭禱告.
讓我們見到大衛怎樣戰勝哥利亞.
願意我們能夠學習他.
首先我們要知道.
我們要從神屬靈的眼光處事.
而不是靠我們自己無能的眼光.
會使我們懼怕驚惶.
我們不用怕.
因為有你在我們那邊.
我們絕對能夠為你作弓.
我們不用怕我們眼前的巨敵.
願意我們為了你的事工.
作好準備.
有一個服從的心.
勝過我們身邊挑戰我們的人.
記得你的恩典.
倚靠你的名將榮耀歸給你.
而最後我們都需要有一個忍耐的心來等候.
等候你的旨意成就在我們身上.
以致我們能夠像大衛一樣.
將榮耀頌讚歸給你.
我們禱告.
奉旨耶穌基督的名而求.
阿們.
\newpage



\section{撒母耳記上 18:1-19:24}
\label{sec:9t69tF6ci0k}
\textbf{神啊!求你救我脫離惡人 (撒母耳記上18\_1-19\_24) - 袁惠鈞牧師[大衛傳系列 - 第3講]}
\newline
\newline
連結: \href{https://youtube.com/watch?v=9t69tF6ci0k}{\texttt{ https://youtube.com/watch?v=9t69tF6ci0k}} ~~~~ 語音日期: 2025-01-22 
\newline
\newline
\hyperref[sec:OtTM_EdQEtA]{< < < PREV SERMON < < <}
~
\hyperlink{toc}{[返主目錄]}
~
\hyperref[ch:preacher6]{[返講員目錄]}
~
\hyperref[sec:rN0dS2BBBmc]{> > > NEXT SERMON > > >}
\newline
\newline
撒母耳記上 18:1-19:24
\newline
\begin{longtable}{cl}
\hline
\hline
章節 & 經文 (和合本修訂版)\\
\hline
18:1 & \begin{tabularx}{0.7\textwidth}{X} 大衛對掃羅說完了話,約拿單的心與大衛的心深相契合。約拿單愛大衛,如同愛自己的性命。 \end{tabularx} \\ \\ \relax
18:2 & \begin{tabularx}{0.7\textwidth}{X} 那日掃羅留住大衛,不讓他回父家。 \end{tabularx} \\ \\ \relax
18:3 & \begin{tabularx}{0.7\textwidth}{X} 約拿單愛大衛如同愛自己的性命,就與他立約。 \end{tabularx} \\ \\ \relax
18:4 & \begin{tabularx}{0.7\textwidth}{X} 約拿單從身上脫下外袍,給了大衛,又把戰衣、刀、弓、腰帶都給了他。 \end{tabularx} \\ \\ \relax
18:5 & \begin{tabularx}{0.7\textwidth}{X} 掃羅無論差遣大衛往何處去,他都做事精明。掃羅立他作軍隊的指揮官,眾百姓和掃羅的臣僕都看為美。 \end{tabularx} \\ \\ \relax
18:6 & \begin{tabularx}{0.7\textwidth}{X} 大衛打死了那非利士人,同眾人回來的時候,婦女們從以色列各城裡出來,歡歡喜喜,打鼓奏樂,唱歌跳舞,迎接掃羅王。 \end{tabularx} \\ \\ \relax
18:7 & \begin{tabularx}{0.7\textwidth}{X} 眾婦女歡樂唱和,說:「掃羅殺死千千,大衛殺死萬萬。」 \end{tabularx} \\ \\ \relax
18:8 & \begin{tabularx}{0.7\textwidth}{X} 掃羅非常憤怒,不喜歡這話。他說:「將萬萬歸給大衛,千千歸給我,只剩下王國沒有給他!」 \end{tabularx} \\ \\ \relax
18:9 & \begin{tabularx}{0.7\textwidth}{X} 從這日起,掃羅就敵視大衛。 \end{tabularx} \\ \\ \relax
18:10 & \begin{tabularx}{0.7\textwidth}{X} 次日,從神來的邪靈緊抓住掃羅,他就在家中胡言亂語。大衛照常彈琴,掃羅手裡拿著槍。 \end{tabularx} \\ \\ \relax
18:11 & \begin{tabularx}{0.7\textwidth}{X} 掃羅把槍一擲,心裡說:「我要將大衛刺透,釘在牆上。」大衛閃避了他兩次。 \end{tabularx} \\ \\ \relax
18:12 & \begin{tabularx}{0.7\textwidth}{X} 掃羅懼怕大衛,因為耶和華離開自己,與大衛同在。 \end{tabularx} \\ \\ \relax
18:13 & \begin{tabularx}{0.7\textwidth}{X} 所以掃羅叫大衛離開自己,立他為千夫長,他就領兵出入。 \end{tabularx} \\ \\ \relax
18:14 & \begin{tabularx}{0.7\textwidth}{X} 大衛所做的每一件事都精明,耶和華也與他同在。 \end{tabularx} \\ \\ \relax
18:15 & \begin{tabularx}{0.7\textwidth}{X} 掃羅見大衛做事精明,就更怕他。 \end{tabularx} \\ \\ \relax
18:16 & \begin{tabularx}{0.7\textwidth}{X} 但以色列和猶大眾人都愛大衛,因為他領他們出入。 \end{tabularx} \\ \\ \relax
18:17 & \begin{tabularx}{0.7\textwidth}{X} 掃羅對大衛說:「看哪,我將大女兒米拉嫁給你,只要你作我的勇士,為耶和華爭戰。」掃羅心裡說:「我不好親手害他,要藉非利士人的手害他。」 \end{tabularx} \\ \\ \relax
18:18 & \begin{tabularx}{0.7\textwidth}{X} 大衛對掃羅說:「我是誰,我是甚麼出身,我父家在以色列中算甚麼,豈敢作王的女婿呢?」 \end{tabularx} \\ \\ \relax
18:19 & \begin{tabularx}{0.7\textwidth}{X} 掃羅的女兒米拉到了當嫁給大衛的時候,掃羅卻將她嫁給了米何拉人亞得列。 \end{tabularx} \\ \\ \relax
18:20 & \begin{tabularx}{0.7\textwidth}{X} 掃羅的女兒米甲愛大衛。有人告訴掃羅,這件事在掃羅眼中看為合宜。 \end{tabularx} \\ \\ \relax
18:21 & \begin{tabularx}{0.7\textwidth}{X} 掃羅心裡說:「我把這女兒嫁給大衛,作他的圈套,好藉非利士人的手害他。」所以掃羅第二次對大衛說:「你今日可以作我的女婿。」 \end{tabularx} \\ \\ \relax
18:22 & \begin{tabularx}{0.7\textwidth}{X} 掃羅吩咐臣僕:「你們暗中對大衛說:『看哪,王喜歡你,王的臣僕也都愛戴你,現在你就作王的女婿吧。』」 \end{tabularx} \\ \\ \relax
18:23 & \begin{tabularx}{0.7\textwidth}{X} 掃羅的臣僕照這話說給大衛聽。大衛說:「你們把作王的女婿看為小事嗎?我是貧窮卑微的人。」 \end{tabularx} \\ \\ \relax
18:24 & \begin{tabularx}{0.7\textwidth}{X} 掃羅的臣僕回奏說,大衛說了這樣的話。 \end{tabularx} \\ \\ \relax
18:25 & \begin{tabularx}{0.7\textwidth}{X} 掃羅說:「你們要對大衛這樣說:『王不要甚麼聘禮,只要一百非利士人的包皮,好在王的仇敵身上報仇。』」掃羅的意圖是要大衛落在非利士人的手中。 \end{tabularx} \\ \\ \relax
18:26 & \begin{tabularx}{0.7\textwidth}{X} 掃羅的臣僕把這話告訴大衛,大衛就歡喜作王的女婿。日期還沒有到, \end{tabularx} \\ \\ \relax
18:27 & \begin{tabularx}{0.7\textwidth}{X} 大衛和跟隨他的人起來前往,殺了二百非利士人,將包皮足數交給王,為要作王的女婿。於是掃羅將女兒米甲嫁給大衛。 \end{tabularx} \\ \\ \relax
18:28 & \begin{tabularx}{0.7\textwidth}{X} 掃羅見耶和華與大衛同在,女兒米甲又愛大衛, \end{tabularx} \\ \\ \relax
18:29 & \begin{tabularx}{0.7\textwidth}{X} 就更怕大衛,常常與大衛為敵。 \end{tabularx} \\ \\ \relax
18:30 & \begin{tabularx}{0.7\textwidth}{X} 每逢非利士的軍官出來打仗,大衛做事比掃羅任何臣僕更精明,因此他的名極受尊重。 \end{tabularx} \\ \\ \relax
19:1 & \begin{tabularx}{0.7\textwidth}{X} 掃羅吩咐他兒子約拿單和眾臣僕要殺大衛,但掃羅的兒子約拿單卻很喜愛大衛。 \end{tabularx} \\ \\ \relax
19:2 & \begin{tabularx}{0.7\textwidth}{X} 約拿單告訴大衛說:「我父掃羅想要殺你,現在你要小心,明日早晨留在一個僻靜的地方藏起來。 \end{tabularx} \\ \\ \relax
19:3 & \begin{tabularx}{0.7\textwidth}{X} 我會出去,到你所藏的田裡,站在我父親旁邊,與父親談論到你。我看情形怎樣,會告訴你。」 \end{tabularx} \\ \\ \relax
19:4 & \begin{tabularx}{0.7\textwidth}{X} 約拿單向他父親掃羅說大衛的好話,對他說:「王不可得罪王的僕人大衛,因為他未曾得罪你,他所行的對你都很有益處。 \end{tabularx} \\ \\ \relax
19:5 & \begin{tabularx}{0.7\textwidth}{X} 他拚了命殺那非利士人,並且耶和華為全以色列大施拯救。那時你看見,也很歡喜,現在為何要犯罪,流無辜人的血,無緣無故殺大衛呢?」 \end{tabularx} \\ \\ \relax
19:6 & \begin{tabularx}{0.7\textwidth}{X} 掃羅聽了約拿單的話,就指著永生的耶和華起誓:「我絕不殺他。」 \end{tabularx} \\ \\ \relax
19:7 & \begin{tabularx}{0.7\textwidth}{X} 約拿單叫大衛來,把這一切事告訴他。約拿單帶他去見掃羅,他就像以前一樣侍立在掃羅面前。 \end{tabularx} \\ \\ \relax
19:8 & \begin{tabularx}{0.7\textwidth}{X} 此後又有戰爭,大衛出去與非利士人打仗。他大大擊敗他們,他們就在他面前逃跑。 \end{tabularx} \\ \\ \relax
19:9 & \begin{tabularx}{0.7\textwidth}{X} 從耶和華來的邪靈又降在掃羅身上,掃羅手裡拿槍坐在屋裡,大衛正用手彈琴。 \end{tabularx} \\ \\ \relax
19:10 & \begin{tabularx}{0.7\textwidth}{X} 掃羅想要用槍刺透大衛,把他釘在牆上,他卻躲開掃羅,掃羅的槍刺入牆內。當夜大衛逃走,躲起來了。 \end{tabularx} \\ \\ \relax
19:11 & \begin{tabularx}{0.7\textwidth}{X} 掃羅派一些使者到大衛的房屋那裡守著他,等到天亮要殺他。大衛的妻子米甲對大衛說:「你今夜若不逃命,明日就要被殺。」 \end{tabularx} \\ \\ \relax
19:12 & \begin{tabularx}{0.7\textwidth}{X} 於是米甲將大衛從窗戶縋下去,讓他走;大衛就逃走,躲起來了。 \end{tabularx} \\ \\ \relax
19:13 & \begin{tabularx}{0.7\textwidth}{X} 米甲把家中的神像放在床上,頭枕在山羊毛的枕頭上,用衣服蓋起來。 \end{tabularx} \\ \\ \relax
19:14 & \begin{tabularx}{0.7\textwidth}{X} 掃羅派一些使者去捉拿大衛,米甲說:「他病了。」 \end{tabularx} \\ \\ \relax
19:15 & \begin{tabularx}{0.7\textwidth}{X} 掃羅又派一些使者去看大衛,說:「把他連床一起抬到我這裡,我好殺他。」 \end{tabularx} \\ \\ \relax
19:16 & \begin{tabularx}{0.7\textwidth}{X} 使者進去,看哪,神像在床上,頭枕在山羊毛的枕頭上。 \end{tabularx} \\ \\ \relax
19:17 & \begin{tabularx}{0.7\textwidth}{X} 掃羅對米甲說:「你為甚麼這樣欺騙我,放我仇敵逃走呢?」米甲對掃羅說:「他對我說:『你放我走吧,我何必要殺你呢?』」 \end{tabularx} \\ \\ \relax
19:18 & \begin{tabularx}{0.7\textwidth}{X} 大衛逃跑躲避,來到拉瑪的撒母耳那裡,把掃羅向他所行的事全告訴他。他和撒母耳就去,住在拿約。 \end{tabularx} \\ \\ \relax
19:19 & \begin{tabularx}{0.7\textwidth}{X} 有人告訴掃羅說:「看哪,大衛在拉瑪的拿約」。 \end{tabularx} \\ \\ \relax
19:20 & \begin{tabularx}{0.7\textwidth}{X} 掃羅派一些使者去捉拿大衛。去的人見一隊先知受感說話,撒母耳站在當中領導他們,掃羅派去的使者也受神的靈感動說話。 \end{tabularx} \\ \\ \relax
19:21 & \begin{tabularx}{0.7\textwidth}{X} 有人把這事告訴掃羅,他又派另一些使者去,他們也受感說話。掃羅第三次派使者去,他們也受感說話。 \end{tabularx} \\ \\ \relax
19:22 & \begin{tabularx}{0.7\textwidth}{X} 然後掃羅親自往拉瑪去,到了西沽的大井,問人說:「撒母耳和大衛在哪裡?」有人說:「看哪,在拉瑪的拿約。」 \end{tabularx} \\ \\ \relax
19:23 & \begin{tabularx}{0.7\textwidth}{X} 他就往那裡去,到了拉瑪的拿約。神的靈也臨到他,他一面走一面受感說話,直到拉瑪的拿約。 \end{tabularx} \\ \\ \relax
19:24 & \begin{tabularx}{0.7\textwidth}{X} 他也脫了衣服,也在撒母耳面前受感說話,一日一夜赤身躺臥。因此有人說:「掃羅也在先知中嗎?」 \end{tabularx} \\ \\
[1ex]
\hline
\hline
\end{longtable}
$^{1}$大衛傳的第三講.
今天我們要講撒母耳上記的十八章和十九章.
上一次我們已經講了大衛戰勝巨人哥利亞的故事.
每當神的僕人成功的時候.
撒旦就很多時候會來試探我們.
會來攻擊我們.
就好像耶穌基督在馬太福音第三章.
耶穌為了我們做一個很好的榜樣受洗.
耶穌基督根本不需要受洗因為他是無罪.
他是為了我們來做這個榜樣.
天就打開有聲音在天上說.
「這是我的愛子,我所喜愛的」.
他是被天父所喜悅.
做了這件事之後就怎樣.
立刻在馬太福音第四章.
耶穌基督被帶領到曠野被撒旦試探四十天.
很多時候當我們做了一件好事,成功的事.
撒旦就會來攻擊.
大衛戰勝哥利亞之後.
撒旦就煽動蘇羅.
來憎恨大衛要殺死他.
我今天的這篇道是叫做.
「神啊,求你救我脫離惡人」.
為什麼會有這麼奇怪的主題.
其實很簡單.
這正是在詩篇的59篇.
今天的故事說到大衛被蘇羅追殺.
就是這個時候大衛寫詩篇的59篇.
是他說的.
「神啊,求你救我脫離惡人」.
讓我們在大衛的身上學習.
當有惡人追趕我們,逼迫我們的時候.
我們應該怎樣來回應.
根據猶太人的律法.
一個人要20歲.
起碼20歲才能當兵.
大衛是很特別.
當他在蘇羅的軍隊裡.
成為高級的官長的時候.
他可能只有18歲.

$^{41}$大衛發現自己和蘇羅有很嚴重的衝突.
問題不是在大衛的身上.
他是一個很誠實的人.
問題是在蘇羅的身上.
他是一個很妒忌,很奸險的人.
大衛的成功沒有令他驕傲.
他打贏了哥利亞之後.
他就回到自己的工作單位繼續工作.
他甚至願意很謙卑地服從蘇羅.
幫助他治治心理的病.
換了別人又會怎樣?.
如果換了是我.
我想真的不會這樣了.
神已經廢除了蘇羅.
又高納了大衛.
有沒有搞錯啊?.
你都已經被人廢了.
為什麼我還要侍奉你,服從你?.
這個就會變成我的心態.
但大衛不是這樣.
大衛仍然願意服從來侍奉蘇羅.
當神的靈離開了蘇羅.
臨到大衛的身上.
大衛就能帶領蘇羅的軍隊所向無敵.
得到很多的勝利.
他漸漸得到很多人的喜愛.
首先在《撒母耳上》的十八章五節.
說眾百姓和蘇羅的神僕都喜悅大衛.
在《撒母耳上》的十八章十六節.
說以色列和猶大眾人都愛大衛.
去到《撒母耳上》的十八章三十節.
說大衛的名被人尊重.
這令到蘇羅非常妒忌.
還動了殺機要殺死大衛.
就當大衛不斷上升被人喜悅.
蘇羅的情形每況愈下.
當神的靈離開了他之後.
他就變得害怕.
害怕這個字可以翻譯成擔心.
在《撒母耳上》的十八章第十二節.

$^{81}$他就用擔心的態度來觀察大衛.
變得再次害怕.
這個害怕又可以翻譯成驚訝.
在《撒母耳上》的十八章十五節.
然後他不能理解.
為什麼這麼多人在他身邊都這麼喜歡大衛.
他感覺到恐懼.
恐懼這個字更好的翻譯就是更加害怕.
所以有沒有看到他越來越差.
一開始是擔心變成驚訝.
再由驚訝變成更加害怕.
令他非常妒忌大衛.
以致日後追趕大衛.
令大衛要逃亡.
大衛逃亡的生涯有十多年.
但神就利用這個逆境來塗鑿大衛.
幫助他.
給他更大的勇氣.
給他更有信心.
慢慢慢慢成為一個好的領袖.
蘇羅也曾經愛大衛.
在撒母耳上的十六章二十一節.
經文有告訴我們.
但很快這個愛就變成了妒忌,仇恨.
甚至在撒母耳上的十八章二十九節.
說他常常與大衛為敵.
當大衛與蘇羅的關係出現問題.
大衛與蘇羅的兒子約拿丹的感情就越來越好了.
有很多解經家都認為.
大衛與約拿丹的年紀.
大概是十多歲的青年人.
是teenagers.
大家彼此相愛.
其實這個想法不是很正確.
因為我們剛才說過.
一個人起碼要二十歲才可以當兵.
在約拿丹認識大衛的時候.
他已經是掌管蘇羅三分之一的軍隊.
而且已經得到了兩次很大的勝利.
記載在撒母耳上的十三章.

$^{121}$所以不可能他是一個teenager.
其實他已經是一個很有經驗的官長.
也有學者認為.
其實約拿丹可能比大衛大二十歲.
約拿丹是蘇羅的長子.
他要注定繼承蘇羅的王位.
所以如果有一個人妒忌大衛的話.
應該是約拿丹.
但反過來是蘇羅妒忌大衛.
約拿丹還將自己的盔甲.
和自己的武器送給大衛.
其實在古代這是一個象徵.
象徵他願意接受大衛成為他的王.
所以我相信大衛有告訴約拿丹.
撒母耳高立了自己.
而大衛和約拿丹立約.
立什麼約啊?.
約拿丹願意接受大衛成為他的王.
而大衛保證日後一定會照顧保護約拿丹的家人.
這些事後我們就會發現.
大衛真的遵守他的諾言.
當大衛殺死斐理士人回到以色列人的營地.
以色列的婦女就走出來歡迎他們.
他們歡歡喜喜打鼓激興.
他們歌唱跳舞迎接大衛和蘇羅.
在撒母耳上的十八章第七節他們說什麼.
蘇羅殺死千千.
大衛殺死萬萬.
在箴言的二十七章第二十一節說.
鼎為煉銀.
勞為煉金.
人的稱讚也是煉人.
這句話很有意思.
我們看到神的試煉很多時候有雙方面.
有負面和正面.
我們說羅馬書第八章尾的時候.
我們說過神可以用苦難來試煉他的僕人.
萬事互相效力.
要愛神的人得到好處.
這是負面的.

$^{161}$但也有正面的試煉.
別人給你的讚美.
別人給你的讚美會否飄飄然覺得自己很厲害.
還是你真的將榮耀歸神.
不單是別人給你的讚美.
可能是別人給你身邊的人的讚美.
你又怎麼回應呢.
你會不會妒忌呢.
為什麼他讚美我身邊的人不讚美我.
就好像素羅一樣.
為什麼素羅這麼憎恨大衛.
就是那些婦女身邊的人.
都將他們的注意力和讚美給了大衛.
令他非常妒忌.
那些婦女說素羅殺死千千.
大衛殺死萬萬.
我相信也有些言過其實.
大衛雖然殺死很多菲利士人.
但我相信很難一個人殺死幾萬人.
但我相信那些婦女心想.
如果他能夠殺死哥利亞.
就等於殺死了萬萬個菲利士人.
所以他們將榮耀歸給大衛.
素羅就妒忌了.
但如果真的有一位被忽視了.
是不是素羅?不是素羅.
是誰被忽視?是神自己.
那些婦女將榮耀歸給素羅.
歸給大衛.
但神在哪裡?.
為什麼他們不讚美神?.
記不記得出埃及第十五章.
當以色列人過了紅海.
他們唱摩西之歌.
將榮耀歸給誰?.
是歸給萬軍的耶和華.
在述事司機的第五章.
底波拉打贏了他的敵人之後.
在底波拉之歌榮耀歸給誰?.
是歸給神.

$^{201}$為什麼那些婦女不將榮耀歸給神.
反而歸給大衛和素羅?.
是和他們的領袖有關.
一切由上至下.
當他們的領袖自己都不將神放在首位.
他手下的人一樣不會將神放在首位.
所以他們都沒有將榮耀歸給神.
素羅就是將榮耀歸給自己.
這就是他的問題.
為什麼他會妒忌大衛.
想將榮耀歸給自己.
他甚至沒有將榮耀歸給神.
那些婦女的歌唱有沒有影響大衛?.
沒有影響到.
但是激怒了素羅.
素羅心想我已經失去了皇權.
神已經將我拉下來.
連你們這些婦女都不尊敬我.
還說大衛殺死萬萬.
你們唯一沒有給他的.
就是連我的王國都給了他.
所以素羅非常妒忌大衛.
素羅對大衛的反應和在新約斯洗約翰對耶穌基督的反應是很不同的.
斯洗約翰是怎樣對耶穌基督的反應?.
他的回應.
他說他必興旺我必衰微.
在約翰福音的三章三十節.
現在素羅他說我必興旺他必衰微.
我要殺死大衛.
他就不會要脅我.
其實在舊約在希伯來文有好幾個字都可以翻譯成妒忌.
這個字在舊約出現八十次.
就是妒忌.
可以翻譯成妒忌熱心.
你說熱心是什麼?.
熱心想得到別人的東西.
還有呢.
疑恨.
這個字用在神的身上就是妒忌的神.
我們所敬拜的神是妒忌的神.

$^{241}$是妒忌的神.
有人說不是很好.
妒忌不好.
但用在神的身上有沒有問題?.
沒有問題.
為什麼?.
我是牧師我去妒忌另外一個牧師對不對?.
當然不對.
世上那麼多牧師.
一個體育家去妒忌另外一個體育家.
應不應該?.
不應該那麼多體育家.
但神呢.
世上有多少位唯一的真神?.
只有耶和華萬有的真神.
只有一位.
所以神絕對是有權去妒忌.
但我們每一個人呢.
我們是沒有權去妒忌.
所以這個字用在神的身上絕對是正確.
祂就是妒忌的神.
這個字在新約也有兩個字翻譯成妒忌.
第一個就是希臘文的"Jealous"這個字.
"Jealous"這個字英文翻譯做"Jealous".
中文是忌恨或者妒忌.
這個字在新約出現了45次.
是"Jealous".
另外一個字呢.
是"Fathomless"這個字.
這個字是被翻譯做"Envious".
中文也是妒忌.
很多時候我們誤會了.
中文的"Envious"翻譯成羨慕.
很多時候.
你聽起來羨慕怎樣都比妒忌好.
但在希臘文的原文和英文是剛剛相反.
其實"Jealous"是比"Envious"好.
"Jealous"是什麼呢.
就是希望得到別人的東西.
眼紅別人.

$^{281}$貪戀別人的東西.
就是"Jealous".
"Envious"呢.
是自己得不到的東西.
連你我也想你得不到.
最多一拍兩散.
一個好的例子是什麼.
你看到你的鄰居有輛很漂亮的跑車.
你"Jealous"的時候又怎樣.
發奮圖強.
努力工作.
你也希望有輛跑車.
這樣就"Jealous".
但是你"Envious"呢.
你自己得不到又怎樣.
你刮花他的車.
讓他也沒有.
所以我們看到.
其實"Envious"遠遠比"Jealous"好.
"Solo"不單是"Jealous".
他也是"Envious".
在中文來說.
就是一個字.
就是妒忌.
其實這個就是十屆最後的一屆.
不可貪戀別人的房屋.
不可貪戀別人的妻子.
那你說.
"Solo"貪戀大衛的什麼.
貪戀大衛被人讚賞.
不是物質上的東西.
物質上的東西他也有多過大衛.
但是他貪戀別人被大衛的讚賞.
這個就是他的問題.
所以他就想殺死大衛.
上次說到"Solo"有些精神的問題.
他需要別人彈琴給他聽.
他就請了大衛彈琴給他聽.
當大衛彈琴給他聽的時候.
"Solo"就在想著怎樣殺死你.

$^{321}$拿著長矛就扔在大衛身上.
希望射死他.
這件事發生了兩次.
在撒姆以上的十八章第十一節.
兩次都逃脫了.
這兩件事可能發生在.
大衛戰勝哥利亞之後.
變成"Solo"的官長之前.
雖然"Solo"想殺死大衛.
但大衛仍然忠於"Solo".
仍然願意來服侍他.
我們今天的主題就是.
被敵人逼迫的時候.
被惡人逼迫的時候.
怎樣來回應.
第一點就是.
不需要取悅人來證明自己.
我們要取悅的是誰?.
是神自己.
一個人如果是靠著信心.
他就不會靠自己的謀算.
但"Solo"剛剛相反.
他就是靠自己的謀算.
他絕對不信靠神.
這就是他的問題.
"Solo"違背神.
每次都有很多藉口.
這也值得我們尷尬.
我們自己做錯了事.
或者被人指出我們錯的時候.
我們是不是有很多藉口呢?.
基督徒應該怎樣做?.
應該是面對我們的罪.
勇於認罪.
這樣才是基督徒.
我們得救的時候.
就是因為我們向神認罪悔改.
以致我們能夠得救.
如果我們發覺自己.
別人指出我們錯的時候.

$^{361}$我們有很多藉口.
就是不肯認錯.
認為自己是對的.
我做這件事是有原因的.
這是很值得我們尷尬.
大衛就是一個勇於認罪的人.
"Solo"就是剛剛相反.
不單止這樣.
"Solo"還是會想殺死.
或者除去挑戰他的人.
當他發覺大衛對他有挑戰的時候.
他就妒忌大衛.
希望將大衛除去.
他認為自己要保住自己的王位.
大衛一定要死.
"Solo"還記得嗎?.
當哥利亞來叫陣的時候.
他曾經下了諾言.
誰殺死哥利亞就能夠娶自己的女兒.
事後他真的願意將大女兒米拉嫁給大衛.
但大衛怎樣回答?.
我不配的.
我這麼卑微.
"Solo"怎麼說?.
你自己說的.
你不要怪我.
於是將大女兒嫁給了另外一個人.
但事後呢?.
有人走來告訴"Solo".
他的小女兒米甲愛上了大衛.
"Solo"說好啊.
我給你多一次機會.
於是"Solo"就派僕人去大衛那裡說謊.
告訴他.
"Solo"很喜歡你.
很想你做他的女婿.
娶他的女兒米甲.
大衛怎樣?.
又說不行的.
我很卑微的.

$^{401}$我又沒有聘禮.
這句話令到"Solo"很開心.
你沒有金銀財寶.
我可以要別的.
請你將100個非利士人.
未受國禮的非利士人殺死.
將他們的包皮拿來獻給我.
我要100個.
"Solo"心想.
婦女說你殺死萬萬.
你不是真的這麼厲害.
一個打100個.
你死定了.
我就要這樣.
結果怎樣?.
大衛去殺死了200個非利士人.
拿了他們的包皮回來.
獻給"Solo".
其實我們可能都很欣賞大衛的勇氣.
但是想想他需不需要這樣做?.
他不需要這樣做.
他已經依靠神的能力.
戰勝了哥尼亞.
這是他應該得到的.
成為"Solo"的女婿.
他不需要再證明自己.
不需要再做這些事.
現在他做這些事.
是不是取悅神?.
不是取悅神.
他是取悅誰?.
"Solo".
他是取悅人.
但也有可能.
他真的有自我形象的問題.
他真的覺得自己不配.
除非殺死很多非利士人.
才配來娶"Solo"的女兒米格為妻.
難怪世上這麼多人都說.
自我的形象.

$^{441}$Self-image.
自我的尊嚴.
自尊.
Self-esteem.
我們基督徒需不需要這樣做?.
不需要.
我們已經被神接受了.
就好像《耳窟所書》前面那三章.
《耳窟所書》六章的經文.
前面那三章說什麼?.
我們已經被神接受了.
我們在主內有很多福氣.
然後去到後面那三章.
我們只需要將這些福氣活出來.
我們只需要活出與蒙召的恩相稱的行為.
這個是我們所需要做.
我們取悅神.
但我們不是取悅人.
雖然大衛做錯了.
但神仍然保守他.
在《說三母》以上的十八章第三十節.
經文說.
每逢非利士的軍長出來打仗.
大衛給掃羅的神復作是證明.
因此他的命被人尊重.
除了當我們被惡人逼迫追趕的時候.
我們不要取悅人.
不需要證實自己.
我們要取悅神.
第二點.
我們要靈巧仗舍.
順良將甲子.
掃羅對大衛充滿仇恨.
他吩咐若拿丹和他手下的神僕.
要將大衛殺死.
但神偏偏透過大衛.
征服了以色列人的敵人.
鞏固了以色列的國.
還給他機會得到大量的財富.
將來來建聖殿.

$^{481}$還透過大衛帶領彌賽亞.
就是耶穌基督來到世上.
難怪撒旦那麼想殺死大衛.
所以撒旦激動掃羅來殺死大衛.
就是因為這樣的原因.
若拿丹立刻向大衛通風報信.
說掃羅想殺他.
但另一方面他又去勸掃羅.
大衛根本沒有犯罪.
不至於死.
為什麼要處死他.
還有他又說大衛忠心來侍奉你.
也為了以色列國擊退我們的敵人.
為什麼要殺死他.
於是掃羅發誓.
我不會殺死大衛.
但掃羅是一個偽君子.
他發誓當吃生菜.
但因為他的發誓.
若拿丹和大衛真的回到掃羅身邊侍奉他.
大衛一邊對抗斐尼斯人.
另一邊繼續彈奏音樂來幫助掃羅.
而掃羅繼續想殺死大衛.
在《撒母耳上》第十九章第十節.
第三次用長矛釘死大衛.
在《約翰福音》第八章第四十四節.
說撒旦是說謊的.
是殺人的.
當掃羅被撒旦控制的時候.
掃羅也是一樣說謊.
出爾反爾.
發誓不殺死大衛.
但他不遵守他的諾言.
他仍然想殺死大衛.
當大衛回到自己的家.
他仍然沒有安寧.
掃羅派使者到他的家殺死他.
當他的太太米甲知道.
就叫大衛跳窗出去.
走到安全的地方躲起來.

$^{521}$然後用一個偶像.
用羊毛包著他.
做了一個假人放在床上.
這個可能是一個半身的偶像.
米甲用枕頭做下身.
做了一個假人躺在床上.
很難明白.
大衛是一個合神心意的人.
居然他的太太是拜偶像.
有偶像在家.
其實這個我們不應該覺得很奇怪.
在以色列一直有這個問題.
記不記得亞國.
舊約的亞國.
他叫家人在事件這個地方.
把偶像埋在橡樹下.
在創世記的35章2-4節.
另外到約書亞的時候.
他叫以色列人要除去他們的偶像.
當約書亞臨死的時候.
在約書亞記的24章23節.
這個都是我們很多基督徒的問題.
現在我們基督徒.
我都說過很多次.
不會再拿個公仔出來拜.
現在我們的偶像是什麼.
任何東西.
你用的時間在這些東西身上.
多過你用在神的身上.
無論是時間.
金錢.
這東西就變成了你的偶像.
我們都需要很小心.
另外我們也看到米甲是怎樣的一個人.
除了他拜偶像.
他還做了一個假人出來騙人.
其實他和蘇羅真的很相似.
都是詭計多端.
就在這個時候.
大衛寫了詩篇的59篇.

$^{561}$這個都成為了我們今天的主題.
大衛求神的神.
求你救我脫離惡人.
這個就是詩篇的59篇第一節.
在詩篇的59篇的序言.
我們能夠知道.
他真的在這個時候寫了詩篇.
在這個序言說.
蘇羅打發人窺探大衛的房屋.
要殺他.
那時大衛用者金絲.
交予靈掌.
調用幽曜毀壞.
當我們讀這篇詩篇的時候.
我們能夠感覺到.
蘇羅派他的探子在大衛的家外.
跑來跑去.
等待大衛出來捉拿他.
也能夠聽見大衛.
把這些蘇羅的僕人.
比喻為咆哮的狗.
在這裡我們真的看到.
大衛很信靠神.
神就是他的保障.
就是他的避難所.
他絕對不是將他的信心.
放在自己的計謀.
甚至是他妻子米甲的計謀.
而是放在萬軍的耶和華的身上.
我們這次讀大衛傳.
我們說我們會欣賞很多大衛的詩歌.
詩篇的59篇就是其中的一篇.
我們能夠見到大衛的信心.
在詩篇的59篇第十節.
大衛說.
神要叫我看見我仇敵遭報.
在11節他說.
主啊,你是我們的盾牌.
求你用你的能力使他們.
不再被人欺騙.

$^{601}$你是我們的盾牌.
求你用你的能力使他們.
四散且降為悲.
在12節他說.
願他們在驕傲之中被纏住了.
在第13節他說.
求你發怒使他們.
消滅以至歸於無憂.
有沒有看到大衛.
真的很大信心.
他認為神一定會幫他.
為什麼你說.
我們讀.
《撒姆意上》第19章的時候.
大衛就害怕得落荒而逃.
為什麼他要逃跑.
如果他這麼大信心的話.
他應該說我就靠著神的信心.
面對蘇羅的使者.
你們能夠將我怎樣.
但這不是一個很容易回答的問題.
在《馬太福音》第十章第十六節.
耶穌基督怎麼教我們.
我猜你們如同羊進入狼群.
所以你們要靈巧像蛇.
純良像甲子.
耶穌基督是不是說.
我猜派你們出去.
你們要憑著信心.
逆來順受.
耶穌基督是不是這麼說.
不是這麼說的.
你們要靈巧像蛇.
有很多基督徒都會說.
誰說耶穌基督沒有教我們逆來順受.
耶穌基督沒有教我們逆來順受.
耶穌基督想十字架.
就是逆來順受.
他有沒有反抗?沒有反抗.
這裡有不同的.

$^{641}$因為在聖經裡面的寓言.
耶穌基督就是要為我們釘十字架.
他來到世上.
主要的目的就是什麼?.
為我們釘十字架.
所以他才是逆來順受.
因為這個是聖經的寓言.
但是耶穌基督真的教我們要靈巧像蛇.
所以我相信大衛的圖譜絕對是正確的.
在《說與哥林多前書》的十六章十三節.
都有很相似的經文.
神會用大衛所做的事來完成他的旨意.
直到最後帶領耶穌基督來到世上.
但是我們要小心.
不要只記得靈巧像蛇.
就忘記了純良像甲子.
有很多基督徒會怎樣呢?.
你做初一我就做十五.
你不仁 時就我不義.
你不要怪我.
聖經怎麼說呢?.
要純良像甲子.
不是要我們不擇手段.
就算我們靈巧像蛇的時候.
我們的手段都要榮耀神.
這就是聖經教我們的.
第二天早上.
那些使者就來求米甲交出大衛.
米甲就說:我的上公病了.
那些使者就去回答.
掃羅說:你這臭丫頭.
你動動尾巴我就知道你在想什麼.
好 你說你的上公病了.
就叫使者連床帶人.
搬回來給我檢查.
帶到掃羅那裡就真相大白.
看到那個原來是一個偶像.
掃羅就罵女兒.
你手指咬出不咬人.
老爸你都不幫老公.

$^{681}$米甲看到爸爸罵她.
她就說:不關我的事.
爸爸是大衛威脅我的.
如果我不是這樣做.
他會殺死我.
有沒有看到米甲.
真的有她爸爸的風範.
真的和她爸爸一樣詭計多端.
除了當我們被敵人追趕迫迫.
我們不需要證實自己來取悅人.
第二我們要靈巧匠舍.
要純良像甲子.
還有第三呢.
當我們眾叛親離的時候.
要信靠神.
大衛跑到拉馬去找薩姆爾.
因為他知道薩姆爾是他信得過的朋友.
可以信靠的.
薩姆爾帶領大衛去拿約.
拿約是什麼意思呢.
就是住所的意思.
那你說是誰的住所呢.
拉馬是先知聚集的地方.
是先知的學校.
所以相信是先知所住的地方.
薩姆爾和大衛都在那裡.
一齊來敬拜.
一齊來禱告.
尋求神的智慧.
他們也和其他的先知一齊來禱告.
但是很快.
素羅的追兵就會來到.
一共有三批追兵.
不單止一批.
所以你看到素羅真的要殺死大衛.
但是當這些士兵來到的時候.
去到拉馬又怎樣呢.
他們都有聖靈臨到他們身上.
開始被神的靈感動來說預言.
預言這個字怎麼解釋呢.

$^{721}$可以解釋成預知將來會發生什麼事.
但是這個字也可以解釋成.
向神來唱歌讚美上帝.
我不相信素羅的士兵.
個個都變成先知在說預言.
我相信神將他們變成什麼呢.
變成敬拜者而不是追殺者.
他們都被聖靈感動說出敬拜的話.
神保護大衛和撒母耳.
不需要派天兵天將.
只需要將這些人變成敬拜者.
這樣就夠了大衛和撒母耳.
這個正如哥林多後書第十章第四節所說.
我們真正的兵器.
本不是屬血器的.
乃是在神面前有能力.
可以攻破堅固的營雷.
士兵受辱就怎樣呢.
皇上自己出馬.
三隊的兵馬都不能夠成功.
於是素羅說你們都沒有用.
我要親自出馬.
於是素羅就趕到拉瑪的地方.
發生什麼事呢.
他也有聖靈臨到他的身上.
連他也開始敬拜神讚美神.
和合本很有趣.
素羅脫下衣服赤裸在撒母耳面前.
我相信這隻驚魂.
除了他黃袍外衣.
他和其他人一樣.
在撒母耳面前敬拜神.
這將會是素羅和撒母耳最後一次.
在生的時候的見面.
下一次素羅見到撒母耳是何時呢.
在撒母耳上的28章.
當時是一個月黑風高的晚上.
素羅去到一個膠鬼的女巫那裡.
靠著行巫術的婦人.
叫撒母耳的靈魂上來.

$^{761}$和她見最後的一面.
這個在撒母耳上的28章我們會說.
這裡也讓我們看到.
說到有聖靈感動說話說預言.
其實這件事發生了兩次.
第一次在素羅身上.
是當撒母耳高納他為王.
他也有被聖靈感動說預言.
當時還有人問.
難道素羅也被列在先知的行列裡嗎.
現在他去到拉瑪又說預言.
所以又有人問.
難道素羅也被列在先知的行列裡嗎.
這兩件事讓我們看到.
一個人可能有非凡信仰的經驗.
有屬靈的經歷.
但如果他沒有生命的改變的話.
對這個人絕對是沒有益處.
不代表這個人是得救.
所以我們要很小心.
不是說一個人有屬靈的經驗.
就一定是一個得救的人.
猶大和素羅都很相似.
猶大也有很多屬靈的經驗.
你說有什麼屬靈的經驗呢.
在馬太福音第十章一至八節.
說門徒都是講道和行神蹟.
這個也包括猶大.
他有講道和行神蹟.
還有他有很多恩典和福氣.
他能夠聽到耶穌基督親自的講道.
他能夠享受最後的晚餐.
就好像我們守聖餐一樣.
你可能有很多屬靈的經驗.
你可能有很多信仰上的經歷.
但不等於一個人是得救.
所以這個我們要很小心.
大衛就開始了他流亡的生活.
他的生涯.
當素羅在先知學校讚美敬拜神的時候.

$^{801}$大衛就偷偷離開了拉瑪.
去到基比亞附近和約拿丹見面.
他們會作出最後的努力.
希望和解大衛和素羅當中的問題.
但也幾乎使約拿丹的命都沒了.
素羅連自己的兒子都不放過.
一槍就想刺死約拿丹.
為了大衛的事.
在《樹亞國書》的一章八節.
經文所描述的和素羅很相似.
心懷異議的人.
他所行的一切都不穩定.
這個就是素羅.
素羅試圖一邊靠自己的能力.
維護國土擊敗菲利士人.
擊敗他的敵人.
另一方面繼續追殺大衛.
當大衛越是逃跑.
素羅就變得越是癲狂.
最後他是死在戰場上.
為什麼呢?.
因為他趕走了能夠幫助他戰勝敵人的大衛.
大衛知道現在離開素羅的時候.
這就開始了大衛有十多年逃亡的生涯.
下一次開始我們會說大衛逃亡的生涯.
但神則是透過這些逆境培養大衛成為一個傑出的領袖.
在這個時候大衛真的經歷眾叛親離.
他離開了他的上司.
離開了他的王就是素羅.
他也離開了他的家人他的妻子米甲.
遲些我們讀薩姆爾夏記的時候.
我們會見到米甲會再次出現.
他會再次和大衛復合.
但那個時候的米甲已經變得很藐視大衛.
大衛也和約拿丹失去聯絡.
約拿丹就是他唯一最好的朋友也失去聯絡.
這個時候可能就是他寫詩篇的63篇的時候.
因為詩篇的63篇說大衛在猶大的曠野寫了詩篇.
但也有學者說這篇詩篇可能是他日後被他自己的兒子亞沙隆追殺的時候所寫.
究竟是亞沙隆追趕他還是素羅追趕他來寫的呢?.

$^{841}$我們已經沒有辦法知道.
但我們可以欣賞一下詩篇的63篇第一節.
大衛說:神啊!你是我的神,我要切切地尋求你.
在乾旱疲乏無水之地,我渴想你,我的心切無離.
什麼叫做乾旱疲乏無水之地?.
就是神從大衛的生命當中拿走一切的東西.
令他山窮水盡,沒有依靠.
神就是要教導他單單依靠自己.
大衛在這個時候可以說是他一生當中最陷鑊的時候.
是最可怕的時候.
很多時候我們都是一樣.
有很多時候我們覺得神好像將我們所有的資源.
將我們所有的親朋戚友都拿走.
要我們單單依靠他,就好像大衛一樣.
我們應該怎樣學習大衛呢?.
你看看詩篇的63篇第二節.
大衛說:我要瞻仰你,為要見你的能力和你的榮耀.
在第七節他說:因為你幫助我,我就在你翅膀的任下歡呼.
第八節他說:我心緊緊地跟隨你,你的右手扶持我.
在第十一節他說:凡指著他發誓的,必要誇口.
誇口什麼?誇口自己有多厲害嗎?.
不是,是誇口神的榮耀,將榮耀歸給神.
我們今天說的這個訊息.
就是神求你使我脫離惡人.
我們說當惡人追趕逼迫我們的時候,我們應該怎樣回應?.
我們看到大衛真的被訴羅追趕到不知所措.
三次要用長矛刺死他.
派使者去他家殺死他.
再派三批士兵去拉馬追趕大衛.
甚至訴羅自己親自出馬.
我們真的看到訴羅不肯放過大衛.
在這樣的情形下我們應該怎樣做?.
如果發生在我們身上,我們今天學到三點.
第一點就是我們不需要取悅人.
我們不需要證明什麼.
我們已經被神接受,我們所取悅的就是神.
第二,我們要靈巧匠邪.
但不要忘記要純良匠甲子,不可以不擇手段.
第三,當我們眾叛親離,在乾旱疲乏無水之地,我們要信靠神.
其實給我們有很大的鼓勵和勇氣的經文就是詩篇.

$^{881}$例如今天的詩篇59篇和63篇都是大衛在被追殺的時候所寫.
我們都很需要學習詩篇的教導.
今天也讓我們看到我們需要警醒的一點.
我們有沒有嫉妒,妒忌的心呢?.
你說我怎樣知道我有妒忌的心呢?.
當有人把注意力,讚美給了你身邊的人而不是給你.
你會怎樣呢?你會不會不開心呢?會不會妒忌呢?.
例如舊約的雅各,當他把注意力給了自己的兒子約瑟的時候.
其他的兄弟都妒忌.
例如今天的經文當約拿丹和米甲,素羅的子女都把注意力給了大衛.
大衛就妒忌得不得了.
當我們妒忌的時候我們就不能和身邊的人一起慶祝,歡喜快樂.
你說有什麼值得慶祝,歡喜快樂呢?.
我們身邊的人被人讚美我們應該要替他開心.
和他一起慶祝,歡喜快樂.
但如果我們妒忌的時候我們就剝削了這一切.
我們就不能做得到,其實是很糟糕的.
我們應該和身邊的人一起慶祝.
素羅就是因為他的妒忌而知道他身敗名裂,最後死在戰場.
如果我發現我自己真的有妒忌的問題,我應該怎樣做呢?.
最好的辦法就是用更多的時候來侍奉神,來將榮耀歸給神.
為什麼我有妒忌的時候你要我侍奉神將榮耀歸給神呢?.
素羅的問題在哪裡呢?.
他就是將榮耀歸給自己,所以才會妒忌.
所以他手下的婦女就怎樣呢?有沒有將榮耀歸給神呢?.
沒有將榮耀歸給神,只是將榮耀歸給首領.
因為素羅就是將榮耀歸給自己.
當我們有妒忌人的問題的時候,最好的方法就是將榮耀歸給神.
就好像大衛一樣,我剛才也說過.
只有神才是世上唯一的神,只有神才真正配受我們的讚美.
他就是那位忌邪的神.
除了我們妒忌的時候將榮耀歸給神,還有第二點呢?.
我們要愛人若忌,就好像若拿丹一樣.
如果你說應該要妒忌大衛的是若拿丹,不是素羅.
但他就能夠將榮耀歸給大衛,他願意接受大衛成為他的王.
他是和大衛一起慶祝大衛的成功,這是值得我們學習的.
其實如果我們再想清楚一點,這個其實就是律法,就是十誡所教我們.
十誡頭四誡是和神的關係,十誡後面的六誡是和人的關係.
這就是為什麼有人問耶穌基督,誡命當中最大的是什麼?.
他怎麼說呢?盡心盡義盡聖愛神,其餘的就是愛人若忌.

$^{921}$有沒有看到?盡心盡義盡聖愛神就是十誡前面的四誡.
愛人若忌就是十誡當中後面的六誡.
所以我們要做到將榮耀歸給神,也要將屬於人的榮耀歸給人.
與人同樂,這就是對付我們妒忌之心最好的辦法.
願神祝福他自己的話語,讓我們低頭禱告.
讓我們看到大衛被掃羅追殺,從他的身上學習.
我們應該怎樣回應當我們被人逼迫追殺的時候應該怎樣.
首先我們要學習,我們不需要取悅人.
我們不需要證明自己,因為我們已經被你接受.
我們所需要的就是取悅於你.
我們也需要靈巧像蛇,但我們也需要純良像甲子.
千萬不要不擇手段.
當我們在眾叛親離,乾旱疲乏無水之地的時候.
我們要學習大衛來信靠你.
就好像詩篇的五十九篇和六十三篇一樣.
然而當我們發覺自己有妒忌問題的時候.
我們願意將榮耀歸給你.
願意將屬於人的榮耀歸給人.
以致我們能夠與他們同樂,一起慶祝他們的成功.
願意我們能夠成為一個忠心良善的僕人.
將一切的榮耀贈贈歸給你.
我們這樣禱告,奉主耶穌基督的名而求,阿門.
我們將會在眾叛親離時,在眾叛親離時.
\newpage



\section{撒母耳記上 20:1-21:15}
\label{sec:rN0dS2BBBmc}
\textbf{神祝福危難中的謊言嗎?  (撒母耳記上20\_1-21\_15;22\_6-19) - 袁惠鈞牧師[大衛傳系列 - 第4講]}
\newline
\newline
連結: \href{https://youtube.com/watch?v=rN0dS2BBBmc}{\texttt{ https://youtube.com/watch?v=rN0dS2BBBmc}} ~~~~ 語音日期: 2025-02-05 
\newline
\newline
\hyperref[sec:9t69tF6ci0k]{< < < PREV SERMON < < <}
~
\hyperlink{toc}{[返主目錄]}
~
\hyperref[ch:preacher6]{[返講員目錄]}
~
\hyperref[sec:WCt7vYrgwVY]{> > > NEXT SERMON > > >}
\newline
\newline
撒母耳記上 20:1-21:15
\newline
\begin{longtable}{cl}
\hline
\hline
章節 & 經文 (和合本修訂版)\\
\hline
20:1 & \begin{tabularx}{0.7\textwidth}{X} 大衛從拉瑪的拿約逃跑,來到約拿單面前,對他說:「我做了甚麼,有甚麼罪孽,在你父親面前犯了甚麼罪,他竟要尋索我的性命呢?」 \end{tabularx} \\ \\ \relax
20:2 & \begin{tabularx}{0.7\textwidth}{X} 約拿單對他說:「絕無此事!你必不至於死。看哪,我父做事,無論大小,沒有不告訴我的。我父親為甚麼要隱瞞我這件事呢?不會這樣的!」 \end{tabularx} \\ \\ \relax
20:3 & \begin{tabularx}{0.7\textwidth}{X} 大衛又起誓說:「你父親確實知道我在你眼前蒙恩。所以他說,『這事不要讓約拿單知道,免得他愁煩。』我指著永生的耶和華起誓,又指著你的性命起誓,我離死只差一步而已。」 \end{tabularx} \\ \\ \relax
20:4 & \begin{tabularx}{0.7\textwidth}{X} 約拿單對大衛說:「你心裡所求的,我必為你成就。」 \end{tabularx} \\ \\ \relax
20:5 & \begin{tabularx}{0.7\textwidth}{X} 大衛對約拿單說:「看哪,明日是初一,我必須與王同席用餐,求你讓我去藏在田野,直到第三日傍晚。 \end{tabularx} \\ \\ \relax
20:6 & \begin{tabularx}{0.7\textwidth}{X} 你父親若見我不在席上,你就說:『大衛懇求我允許他趕回本城伯利恆去,因為他全家在那裡獻年祭。』 \end{tabularx} \\ \\ \relax
20:7 & \begin{tabularx}{0.7\textwidth}{X} 你父親若說好,你的僕人就平安了;他若大怒,你就知道他決意行惡。 \end{tabularx} \\ \\ \relax
20:8 & \begin{tabularx}{0.7\textwidth}{X} 求你施恩於僕人,因你在耶和華面前曾與僕人立約。我若有罪孽,你就親自殺死我,何必把我交給你父親呢?」 \end{tabularx} \\ \\ \relax
20:9 & \begin{tabularx}{0.7\textwidth}{X} 約拿單說:「絕無此事!我若確實知道我父親決意害你,怎麼會不告訴你呢?」 \end{tabularx} \\ \\ \relax
20:10 & \begin{tabularx}{0.7\textwidth}{X} 大衛對約拿單說:「你父親若嚴厲回答你,誰來告訴我呢?」 \end{tabularx} \\ \\ \relax
20:11 & \begin{tabularx}{0.7\textwidth}{X} 約拿單對大衛說:「來,讓我們到田野去。」二人就往田野去了。 \end{tabularx} \\ \\ \relax
20:12 & \begin{tabularx}{0.7\textwidth}{X} 約拿單對大衛說:「願耶和華-以色列的神作證。明日約在這時候,或第三日,我一探出我父親的心意,看哪,若對大衛是好意,我怎麼會不派人來告訴你呢? \end{tabularx} \\ \\ \relax
20:13 & \begin{tabularx}{0.7\textwidth}{X} 我父親若有意害你,而我不告訴你,送你平安地離開,願耶和華重重懲罰約拿單。願耶和華與你同在,如同從前與我父親同在一樣。 \end{tabularx} \\ \\ \relax
20:14 & \begin{tabularx}{0.7\textwidth}{X} 你要照耶和華的慈愛恩待我,不但我活著的時候免我死亡, \end{tabularx} \\ \\ \relax
20:15 & \begin{tabularx}{0.7\textwidth}{X} 就是耶和華從地面逐一剪除大衛仇敵的時候,你也永不可向我家斷絕恩惠。」 \end{tabularx} \\ \\ \relax
20:16 & \begin{tabularx}{0.7\textwidth}{X} 於是約拿單與大衛家立約:「願耶和華從大衛仇敵的手來追討。」 \end{tabularx} \\ \\ \relax
20:17 & \begin{tabularx}{0.7\textwidth}{X} 約拿單因愛大衛如同愛自己的性命,就叫他再起誓。 \end{tabularx} \\ \\ \relax
20:18 & \begin{tabularx}{0.7\textwidth}{X} 約拿單對他說:「明日是初一,你的座位空著,人必察覺你不在。 \end{tabularx} \\ \\ \relax
20:19 & \begin{tabularx}{0.7\textwidth}{X} 到第三日,就要走一段長路下去,去到你遇事那天所藏的地方,在以色磐石的旁邊等候。 \end{tabularx} \\ \\ \relax
20:20 & \begin{tabularx}{0.7\textwidth}{X} 我會向磐石旁邊射三箭,如同射箭靶一樣。 \end{tabularx} \\ \\ \relax
20:21 & \begin{tabularx}{0.7\textwidth}{X} 看哪,我會派僮僕,說:『去把箭找來。』我若對僮僕喊說:『看哪,箭在你的這邊,把箭拿來』,你就可以平安回來;我指著永生的耶和華起誓,你一定沒有事。 \end{tabularx} \\ \\ \relax
20:22 & \begin{tabularx}{0.7\textwidth}{X} 我若對孩子說:『看哪,箭在你的前方』,你就要離開,因為是耶和華差你去的。 \end{tabularx} \\ \\ \relax
20:23 & \begin{tabularx}{0.7\textwidth}{X} 至於你和我,我們所說的話,看哪,耶和華在你我中間作證,直到永遠。」 \end{tabularx} \\ \\ \relax
20:24 & \begin{tabularx}{0.7\textwidth}{X} 大衛就去藏在田野。到了初一,王要坐席用餐。 \end{tabularx} \\ \\ \relax
20:25 & \begin{tabularx}{0.7\textwidth}{X} 王照常坐在靠牆的位子上,約拿單在對面,押尼珥坐在掃羅旁邊,大衛的座位卻是空的。 \end{tabularx} \\ \\ \relax
20:26 & \begin{tabularx}{0.7\textwidth}{X} 這日掃羅沒有說甚麼,因為他說:「大衛或許有事,偶染不潔,還未得潔淨。」 \end{tabularx} \\ \\ \relax
20:27 & \begin{tabularx}{0.7\textwidth}{X} 初二,大衛的座位還空著。掃羅對他兒子約拿單說:「耶西的兒子為何昨日、今日都沒有來用餐呢?」 \end{tabularx} \\ \\ \relax
20:28 & \begin{tabularx}{0.7\textwidth}{X} 約拿單回答掃羅說:「大衛懇求我允許他回伯利恆去, \end{tabularx} \\ \\ \relax
20:29 & \begin{tabularx}{0.7\textwidth}{X} 說:『求你讓我去,因為我家在城裡有獻祭的事,我哥哥吩咐我去。如今我若在你眼前蒙恩,求你讓我去見我的兄弟。』所以大衛沒有來赴王的筵席。」 \end{tabularx} \\ \\ \relax
20:30 & \begin{tabularx}{0.7\textwidth}{X} 掃羅向約拿單怒氣大發,對他說:「你這頑梗悖逆之婦人所生的,我怎麼會不知道你選擇耶西的兒子,自取羞辱,也使你母親露體蒙羞呢? \end{tabularx} \\ \\ \relax
20:31 & \begin{tabularx}{0.7\textwidth}{X} 只要耶西的兒子還活在世上一天,你和你的國必保不住。現在你要派人去,把他帶到我這裡來,因為他是該死的。」 \end{tabularx} \\ \\ \relax
20:32 & \begin{tabularx}{0.7\textwidth}{X} 約拿單回答父親掃羅說:「他為甚麼該死呢?他做了甚麼呢?」 \end{tabularx} \\ \\ \relax
20:33 & \begin{tabularx}{0.7\textwidth}{X} 掃羅向約拿單擲槍要刺他,約拿單就知道他父親決意要殺死大衛。 \end{tabularx} \\ \\ \relax
20:34 & \begin{tabularx}{0.7\textwidth}{X} 於是約拿單氣憤憤地從席上起來。他在初二這天沒有吃飯,因為他為大衛愁煩,又因為他父親羞辱了他。 \end{tabularx} \\ \\ \relax
20:35 & \begin{tabularx}{0.7\textwidth}{X} 次日早晨,約拿單按著與大衛約定的時候到田野去,有一個小僮僕跟隨他。 \end{tabularx} \\ \\ \relax
20:36 & \begin{tabularx}{0.7\textwidth}{X} 約拿單對僮僕說:「你跑去把我所射的箭找來。」僮僕跑去,約拿單就把箭射在僮僕的前方。 \end{tabularx} \\ \\ \relax
20:37 & \begin{tabularx}{0.7\textwidth}{X} 僮僕到了約拿單落箭之地,約拿單呼叫僮僕說:「箭不是在你的前方嗎?」 \end{tabularx} \\ \\ \relax
20:38 & \begin{tabularx}{0.7\textwidth}{X} 約拿單又呼叫僮僕說:「快去,不要站在那裡!」僮僕就撿起箭來,回到主人那裡。 \end{tabularx} \\ \\ \relax
20:39 & \begin{tabularx}{0.7\textwidth}{X} 僮僕不知道這是甚麼意思,只有約拿單和大衛知道這事。 \end{tabularx} \\ \\ \relax
20:40 & \begin{tabularx}{0.7\textwidth}{X} 約拿單把他的弓箭交給僮僕,吩咐他說:「你拿到城裡去。」 \end{tabularx} \\ \\ \relax
20:41 & \begin{tabularx}{0.7\textwidth}{X} 僮僕一去,大衛就從南邊出來,俯伏在地,拜了三拜。他們彼此親吻,一起哭泣,大衛哭得更悲哀。 \end{tabularx} \\ \\ \relax
20:42 & \begin{tabularx}{0.7\textwidth}{X} 約拿單對大衛說:「你平平安安地去吧!因為我們二人曾指著耶和華的名起誓說:『願耶和華在你我中間,以及你我後裔中間作證,直到永遠。』」大衛就起身走了,約拿單也回城裡去了。 \end{tabularx} \\ \\ \relax
21:1 & \begin{tabularx}{0.7\textwidth}{X} 大衛到了挪伯的亞希米勒祭司那裡,亞希米勒戰戰兢兢地出來迎接他,對他說:「你為甚麼獨自一人,沒有人跟隨你呢?」 \end{tabularx} \\ \\ \relax
21:2 & \begin{tabularx}{0.7\textwidth}{X} 大衛對亞希米勒祭司說:「王吩咐我一件事,對我說:『我差遣你,吩咐你的這件事,不可讓任何人知道。』因此我已告訴一些僕人到某處去。 \end{tabularx} \\ \\ \relax
21:3 & \begin{tabularx}{0.7\textwidth}{X} 現在你手中有甚麼?請你給我五個餅或是可以找到的食物。」 \end{tabularx} \\ \\ \relax
21:4 & \begin{tabularx}{0.7\textwidth}{X} 祭司對大衛說:「我手中沒有普通的餅,只有聖餅,只能給沒有親近婦人的年輕人。」 \end{tabularx} \\ \\ \relax
21:5 & \begin{tabularx}{0.7\textwidth}{X} 大衛回答祭司說:「我們確實沒有親近婦人,如同往常我出征的時候一樣。平常行路,僕人的身體都還分別為聖,何況今日豈不更使自己分別為聖嗎?」 \end{tabularx} \\ \\ \relax
21:6 & \begin{tabularx}{0.7\textwidth}{X} 祭司拿聖餅給他,因為在那裡沒有別的餅,只有那從耶和華面前撤下的供餅,就是換上熱餅的日子取下來的。 \end{tabularx} \\ \\ \relax
21:7 & \begin{tabularx}{0.7\textwidth}{X} 當日,有掃羅的一個臣僕在那裡,他留在耶和華的面前,名叫多益,是以東人,作掃羅的畜牧長。 \end{tabularx} \\ \\ \relax
21:8 & \begin{tabularx}{0.7\textwidth}{X} 大衛對亞希米勒說:「你手中有沒有槍或刀?因為王的事緊急,連刀劍兵器我都沒有帶。」 \end{tabularx} \\ \\ \relax
21:9 & \begin{tabularx}{0.7\textwidth}{X} 祭司說:「你在以拉谷所殺的非利士人歌利亞的那刀,看哪,裹在布中,放在以弗得後邊。你若要可以拿去,除此以外,再沒有別的了。」大衛說:「沒有甚麼可以跟它比的了!請你給我。」 \end{tabularx} \\ \\ \relax
21:10 & \begin{tabularx}{0.7\textwidth}{X} 那日大衛起來,躲避掃羅,逃到迦特王亞吉那裡。 \end{tabularx} \\ \\ \relax
21:11 & \begin{tabularx}{0.7\textwidth}{X} 亞吉的臣僕對他說:「這不是那地的國王大衛嗎?那裡的人跳舞唱和:『掃羅殺死千千,大衛殺死萬萬』,不是指著他說的嗎?」 \end{tabularx} \\ \\ \relax
21:12 & \begin{tabularx}{0.7\textwidth}{X} 大衛把這些話放在心裡,就很懼怕迦特王亞吉。 \end{tabularx} \\ \\ \relax
21:13 & \begin{tabularx}{0.7\textwidth}{X} 於是他在眾人眼前一反常態,在他們中間裝瘋作癲,在城門的門扇上胡寫亂畫,任由唾沫流在鬍子上。 \end{tabularx} \\ \\ \relax
21:14 & \begin{tabularx}{0.7\textwidth}{X} 亞吉對臣僕說:「看哪,你們看這人瘋了,為甚麼帶他到我這裡來呢? \end{tabularx} \\ \\ \relax
21:15 & \begin{tabularx}{0.7\textwidth}{X} 我豈缺少瘋子,你們竟然帶這人到我面前瘋癲嗎?這個人可以進我的家嗎?」 \end{tabularx} \\ \\
[1ex]
\hline
\hline
\end{longtable}
$^{1}$大胃傳的第四講.
我們已經講到第四講了.
其實舊約真的很精彩.
我很喜歡舊約.
新約很多的教導很直接地告訴我們應該要怎樣做.
但是舊約是透過很多的故事.
其實反映出來的教導都是和新約很相似.
例如好像今天這樣.
這三張書我們要講.
撒母耳上的二十章至二十二章.
講出由三個大胃所講的方言.
讓我們看到和羅馬書的教導其實是非常相似.
今天的題目是什麼?.
神祝福在危難中的方言嗎?.
很多時候我們都會有一個概念就是.
白色的方言不怕講.
白色的方言是什麼?.
是不傷害人的方言.
白色的方言不傷害人不怕講.
幫到人的方言有人說更加應該講.
幫到人.
好像今天這樣.
能夠救命的謊言救命的方言我們又應不應該講呢?.
透過今天的經文我們就會有答案.
首先一些背景.
我們講到大胃被蘇羅追殺.
在撒母耳上的第十九章.
他去到拉瑪這個地方.
躲起來.
但蘇羅窮追不捨.
但神就打救了他.
於是他就離開了拉瑪.
去到基比亞附近.
約拿丹就是蘇羅的兒子來會面.
其實約拿丹都很尷尬.
有這個場面.
手背也是肉手掌也是肉.
一邊是他的爸爸蘇羅.
是爸爸來的.
另外一邊是他最好的朋友大胃.

$^{41}$很難做.
如果蘇羅要殺大胃怎麼辦呢?.
這個其實和耶穌基督在馬太福音第十章的教導都很相似.
當我們願意歸向神跟隨神的時候.
很多時候有人會反對我們.
而其中可能是我們的家人.
大胃就在基比亞附近和約拿丹見面.
約拿丹不相信蘇羅要殺死大胃.
因為蘇羅和約拿丹發誓.
我不會殺死大胃.
而大胃就說你這樣的話你都會相信.
你沒見到你爸爸怎麼做嗎?.
他曾經三次用長矛扔向我那裡想殺死我.
後來又派軍隊來我的家捉拿我.
是米鴿叫他跳窗而逃跑.
這個我們上次都說過.
然後蘇羅再派三隊軍隊來追殺大胃.
然後自己親自出馬也是一樣.
來到拉馬追殺大胃.
但是神將這班人全部變成敬拜者.
你變成敬拜者的時候.
你忘於敬拜神.
你就不可以追殺大胃了.
其實大胃如果躲在拉馬的話.
他應該很安全.
因為見到神能夠用神跡騎士來幫助他.
能夠打救他.
但是他缺少信心.
於是他離開了拉馬去到基比亞.
他和約拿丹一起會面.
一起商量應該怎麼做.
他們就想到一個辦法出來.
這個就是大胃第一個謊言.
這個謊言是怎麼樣的?.
他們要測試蘇羅究竟是不是真的想殺死大胃.
於是他們就想到原來猶太人在新月的時候.
一個月的開始都會大排筵席.
和家人來吃飯.
就好像我們今天有愛宴.
一個月第一個星期我們就有愛宴.

$^{81}$很相似.
在這些這麼大的家庭聚會.
所有的家人都會出席.
大胃身為蘇羅的女婿.
又是蘇羅的侍衛長.
他一定要出席.
而就在這個時候.
他們想到就是將大胃藏在.
在儀式的田野裡面.
然後就和蘇羅說謊.
說大胃回到伯利恆.
和家人聚會獻祭給神.
這個就是第一個謊言.
如果蘇羅見到大胃不在場.
沒事原諒了大胃.
那就是大胃的顧慮是多餘的.
但是如果蘇羅因為這件事發大脾氣.
要追殺大胃的話.
這樣就是說他真的對大胃有殺機.
有心想殺死大胃.
這樣的建議其實是挺好的.
但問題就是要向蘇羅說謊.
還有一個問題.
就是當約拿丹發現了蘇羅的回應是怎樣.
反應是怎樣.
怎樣將這個信息告訴大胃呢.
他們不能夠相信任何的人.
因為任何的人都可能是蘇羅的心腹.
都會將這件事告訴蘇羅.
所以他們就想到一個方法.
就是約拿丹會向田野射三支箭.
然後叫童子去撿這些箭的時候.
就會大聲暗示給大胃聽應該要怎樣做.
究竟他應該回來還是應該逃跑.
就是用這樣的方法來做暗喻.
大胃和約拿丹約定在三天之內.
約拿丹就會將這個消息.
究竟蘇羅的回應是怎樣.
是是敵是友告訴大胃.
約拿丹做這件事不單單是為了眼前的事.

$^{121}$他也是預備他自己的將來.
因為他知道有一天.
大胃將會成為以色列的國王.
記不記得他在說撒母耳上的十八章.
已經和大胃納約.
這個約是怎樣的.
他現在就要大胃重申這個約.
這個約就是他會接受大胃成為他的王.
但是大胃也要保證.
他不會傷害約拿丹的後人.
你說為什麼大胃會傷害約拿丹的後人.
其實以色列和中國都很相似.
一朝天子就是怎樣.
一朝臣.
很多時候一個新的皇上任的時候.
舊皇上的家屬就遭殃.
在中國還會怎樣.
豬狗族.
所以約拿丹要大胃保證不會傷害他的後人.
而大胃真是日後以耶和華的恩慈.
來對待約拿丹的後人.
就是約拿丹會有一個兒子.
這個殘廢的人叫米菲波切.
大胃真的會好好地對待他.
甚至和他在一張桌子裡吃飯.
說回來就是那個筵席.
在第一天筵席開始的時候.
大胃就躲在以色列的田野裡.
來等待約拿丹的訊號.
素羅坐在牆前.
有很多解經家都說素羅很怕人暗殺他.
所以他永遠是用背對著牆.
怕後面有人會來襲擊他.
坐他旁邊的就是亞彌爾.
是他的元帥.
也可以說是他的保鑣.
保護他.
對面就是約拿丹.
而約拿丹旁邊就是大胃的座位.
是空了.

$^{161}$大胃當然是沒有來過.
第一天素羅怎麼回應呢.
他一句話都沒有說.
可能他認為大胃是有不潔淨的禮儀.
不能參加這個聚會.
為什麼不潔淨就不能參加.
原來他們在這些聚會會分食滯肉.
不潔淨的人不可以分食滯肉.
所以素羅可能認為大胃摸了不潔淨的東西.
例如死了的動物.
或者甚至跟他太太行房.
都認為是不潔.
但如果這些不潔的人需要做什麼.
將自己隔離.
然後沐浴更衣.
第二天就沒事了.
就可以回到社區.
和其他人在一起.
但問題是過了第二天.
素羅見到大胃還沒有出現.
他就覺得很奇怪.
他懷疑不可能是大胃不潔.
因為不潔第二天應該可以出現.
為什麼他還沒有來呢.
他就去問約拿丹.
喂 你的死黨去了哪裡.
為什麼還沒有出現.
約拿丹就和素羅說謊.
說大胃回到自己家裡百里行.
來探望她的哥哥.
她想念她的幾個哥哥.
所以她沒有來.
她還用了一個動詞.
非常動聽.
大胃說容我去吧.
容我去在原文的希伯來文.
就是速速地去.
她會速速地回來.
這個謊話天衣無縫.
但素羅相信嗎.

$^{201}$素羅一點都不相信.
他還衝口而出.
破口大罵.
這個就好像耶穌基督在世的時候.
在馬太福音十二章.
他罵的法利塞人.
他說法利塞人心裡充滿的惡毒.
就從口裡說出來.
我們口裡說出來的.
就是我們心裡充滿的.
素羅他不是罵大胃.
而是他罵約拿丹.
具體來說他甚至不是罵約拿丹.
是罵約拿丹的媽媽.
罵他自己的老婆.
很多時候我們可能都會這樣做.
當我們這些子女做些很好的事出來的時候.
這就是我的兒子.
這是我的女兒.
是我的.
但子女做些沒那麼好的事.
你會怎樣.
這是你的兒子.
這是你的女兒.
會和我們的配偶這樣說.
我們自己都會.
現在素羅又怎樣.
這個根本不是我的兒子.
我都不知道你和誰生出來的.
怎會這樣.
簡直是壞人.
手指咬出不咬入.
他怎可能幫耶西的兒子才可以.
來違背他自己的爸爸.
我是他的爸爸.
這個王位是傳給他的.
他怎可以雙手讓給別人才可以.
所以其實素羅.
為什麼罵他自己的老婆.
其實他就說.

$^{241}$這個根本不是我的兒子.
如果他是我的兒子.
他就不會這樣做.
素羅其實所擔心的是什麼.
是他自己的王位.
他希望保住自己的王位.
他應該記得在撒姆以上的十三章和十五章.
神已經說他的王位.
他的國度不會長久.
而且在撒姆以上的十六章.
神已經高立了大位來代替他.
現在他和誰作對.
其實他是和神作對.
和神的旨意作對.
他還叫約拿丹自己的兒子來幫助自己.
他覺得約拿丹一定知道大位躲在哪裡.
所以他叫他自己的兒子供出大位.
當約拿丹不肯的時候.
還勸素羅的時候.
素羅怎樣呢.
一支長槍就扔向自己的兒子.
曾經三次他將長槍扔向大位.
現在連自己的兒子都不放過.
中國人說什麼.
苦毒不吃子.
但他連自己的兒子約拿丹都想殺死.
而這次已經是第二次.
你記不記得在撒姆以上的十四章.
他曾經想殺死約拿丹.
等到第二天.
於是約拿丹就去到以色列的田野.
向著田野射三支箭.
因為他要告訴大位.
素羅真的想殺死他.
其中一支箭是射到.
超過了執箭的童子很遠的地方.
當童子去執箭的時候.
約拿丹就大聲跟童子講話.
其實他根本在暗示大位.
他說箭在前頭.

$^{281}$速速的去不要前移.
叫大位快點躲起來.
不要回來.
於是當童子把箭交回約拿丹的時候.
他就把弓給童子拿回營地.
自己去見大位.
這次的會面和大位.
不是他們最後一次.
但非常感人.
兩個人都抱頭痛哭.
大位更加哭得傷心.
為什麼.
因為他知道自己前路茫茫.
開始了他要逃命的生涯.
他不知道何時可以回到自己的家.
更加不知道何時可以再和約拿丹見面.
於是大位就離開了基比亞.
去了祭司城那柏.
而約拿丹就回到基比亞.
繼續侍奉他的父親.
這樣就開始了大位逃亡的生涯.
大位用了很多時間在曠野生活.
因為有很多詩篇記載了大位在曠野的生涯.
所以我們知道他這段日子是怎樣過.
有很多詩篇都是大位在曠野的時候所作.
很多這些詩篇我都不會一一讀出來.
我會留在powerpoint那裡.
如果大家有興趣的話.
可以自己download來看.
其實我們上一次和你們說大位的故事.
記不記得我們已經說了詩篇的59篇和63篇.
今天我們會和大家看的就是詩篇的56篇和34篇.
就是大位在薩姆以上的20章至22章所發生的事情所寫的詩篇.
很多這些詩篇給我們都有很多的幫助.
尤其是當我們遇見試驗的時候.
我們讀這些詩篇能夠給我們很多的力量和勇氣.
透過大位自己的經歷.
其中一篇我很喜歡的就是詩篇18篇.
是大位當神幫助他勝過他的敵人的時候所寫的讚美詩.
這首詩歌對我自己來說是有很大的鼓勵.

$^{321}$耶穌基督在十字架的上方.
他都是引用過兩個詩篇.
就是詩篇的22篇第一節和詩篇的31篇第五節.
很多時候當我們說謊的時候.
不可能停留在一個謊言.
說了一個謊言的時候我們就會說第二個謊言.
大位一口氣離開了基比亞.
去到基比亞南邊三里拿柏這個地方.
就是會幕所在的地方.
你說會幕所在約瑰是不是都在那裡?.
不是.
約瑰在基列耶林的阿比拿達的家中.
撒姆以上的七章一節.
是遲些大位成為了王之後.
他才會將約瑰搬回耶路撒冷.
那你說大位為什麼要走到拿柏這個地方會幕?.
相信他和撒母耳的交情.
他知道如果他走到會幕的地方.
大祭司會幫他.
能夠成為他一個避難所.
所以他就去到會幕.
會幕裡面的大祭司是阿希米蘭.
阿希米蘭是以利的曾孫.
記不記得以利?.
就是撒母耳代替了以利成為以色列人的大祭司.
當阿希米蘭見到大位的時候.
她覺得非常驚訝.
為什麼?.
因為大位是蘇羅的女婿.
又是蘇羅的侍衛長.
為什麼可能單身一個人來到會幕?.
所以她覺得很奇怪.
就在這個時候.
大位就說了她第二個謊言.
她說是蘇羅差遣她.
有件事委託她.
叫她不要讓人知道.
她說蘇羅說.
她會告訴一些人.
在某個地方來接應大位.

$^{361}$這個就顯明了第二個謊言.
因為蘇羅根本沒有說過這些話.
大位他去到會幕.
其實一個很重要他需要的是什麼?.
是食物.
因為他已經很餓.
他問大祭司有什麼吃.
大祭司說什麼都沒有了.
只剩下陳設餅.
但是陳設餅只有祭司才可以吃.
如果當時的以色列人的百姓.
他們都是照著神的律例典章來做的話.
應該沒有問題.
應該在會幕有很多東西可以給大位吃.
因為大家都將十一奉獻帶到會幕.
就是因為當時的人都是.
靈命低落,靈命倒退.
所以在聖殿裡只有陳設餅.
很多人會說.
大位吃陳設餅是錯的.
但是記不記得.
在馬太福音十二章.
耶穌基督就是用這件事來教訓法利賽人.
當時耶穌基督和門徒經過麥田.
他們很餓.
於是摘了些墨水搓來吃.
一搓就出現問題了.
因為法利賽人說什麼.
你們在安息日動工做事.
你們違反了律法.
所以耶穌基督就用大位這件事來罵他們.
法利賽人其實有兩個問題.
第一個問題是什麼.
他們只求外表服從神的律例典章.
服從禮儀.
內心完全沒有服從神的律例典章.
他們只不過是虛有外表假冒為善.
所以這是第一點.
耶穌基督說他們沒有真正的服從.
第二點.

$^{401}$他們沒有屬靈的辨別力.
就是要知道律法的精義是什麼.
律法的精義就是要我們愛神愛人.
我時時都說十屆.
頭四屆是同神的關係.
是要我們愛神.
後面六屆是同人的關係.
是要我們愛人.
這就是為什麼耶穌基督在馬太福音22章.
當有人問他哪條律法是最重要.
耶穌基督怎樣回答.
要盡心盡意盡性愛神.
其次就是愛人若己.
這就是律法的精義.
是要我們愛神愛人.
現在有人肚餓.
你說他不吃什麼.
因為是安息日.
有人病了.
你說不可以跟他醫病.
因為又是安息日.
其實你搬個安息日出來幹什麼.
好像你自己很守律法.
但是你沒有了愛心.
你根本不愛人.
你也不愛神.
因為你對人的態度.
就是反映出你對神的態度.
所以耶穌基督是用大胃吃陳設餅.
來鬧發理財人.
所以大胃能不能吃陳設餅.
絕對沒有問題.
但是當時的阿希米納.
也是有些像發理財人一樣.
問大胃你結不結症.
你吃那些陳設餅.
大胃說不要緊.
我沒有近類色.
我可以吃這些餅.
於是大胃拿了那些陳設餅.

$^{441}$還問大祭司你有沒有武器.
大祭司心想有沒有搞錯.
我這裡是聖所.
你以為是武器房.
問我有沒有武器.
大祭司說有有有.
哥利亞的劍就在回眸.
哥利亞的劍為什麼會在回眸.
相信大胃戰勝哥利亞之後.
把這把劍獻給神.
所以在回眸.
而大祭司是把這把劍.
和以弗德放在一起.
以弗德是什麼.
是祭司所著的.
裡面有污靈和土名.
可以尋求神的旨意.
這個就出現了很大的問題.
遲些我們會見到.
當大胃拿到食物拿到劍.
他就可以上路.
但他回頭一看.
大件事.
有一個以東人多益高在回眸裡面.
哇這個以東人多益.
為什麼會在回眸.
有兩個可能性.
一個可能性就是.
因為他要加入素羅的士工.
而他自己是以東人.
所以他需要加入猶太教.
有很多禮儀他需要做.
所以他在回眸進行這些禮儀.
另外也有一個可能性.
就是他玷污了自己.
所以他需要在神的面前.
來潔淨自己.
但無論如何.
以東人多益看到大胃.
絕對不是一件好事.

$^{481}$等下我們就會見到.
他會向素羅告發.
一個謊話就帶至第二個謊話.
第二個謊話就帶至第三個謊話.
當大胃見到多益他就很害怕.
他知道遲早素羅都會追殺他.
所以他一口氣跑到距離.
那泊23里的加特的地方.
喂你說加特在哪裡.
加特是斐理士人的地方.
記不記得是巨人哥利亞的家鄉.
你說他為什麼會跑到那裡.
首先他缺乏信心.
當他缺乏信心他懼怕的時候.
就靠他自己的智慧.
他認為可能最危險的地方.
就是最安全.
敵人的地方可能最安全.
為什麼.
離開了以色列人的領域.
素羅在以色列做王.
他認為去到加特應該沒問題.
但是很可惜.
加特人都認識大胃.
剛才我們讀經文.
他們都知道大胃是以色列人的王.
並且素羅殺死千千.
大胃殺死萬萬.
你說那些斐理士人未必認得他.
但是他現在身上有什麼.
哥利亞的劍.
巨人的劍很多.
你不容易把他收起來.
那些斐理士人想.
你殺死了我們的英雄.
還拿著劍來耀武揚威.
招搖過市.
你都大膽.
所以他們每個對大胃虎視眈眈.
就在這個時候.

$^{521}$大胃在薩姆以上的21章13節經文說.
大胃就在眾人的面前.
改變了尋常的舉動.
在他們手下假裝瘋癲.
在城門的門扇上胡寫亂畫.
使兔末留在鬍子上.
開始裝瘋了.
為什麼呢?.
他認為每個人都想殺他.
如果他裝瘋的話.
反而可能會憐憫他.
這個瘋子都不用理他.
於是就放他走.
我剛才也說過.
如果大胃有信心的話.
留在拉瑪.
他需要接二連三的說謊嗎?.
他不需要.
他就是缺乏信心.
他已經看到神怎樣幫助他.
所有追殺他的人.
神都令他們變成敬拜者.
不能追殺他.
他就是不能信靠神.
所以他逃離拉瑪.
繼續說謊.
這三個謊話有什麼後果?.
多益真的去告發大胃.
一個詭詐的領袖.
他身邊有很多詭詐的跟隨者.
多益就是其中一個.
這些詭詐的跟隨者.
都很希望從他們的領袖拿到獎賞.
所以他們都很希望有些消息.
是他們的領袖想知道.
現在多益看到大胃.
大做文章告發大胃.
多益沒有因為自己逼迫.
神已經受高的君王而感到羞愧.
他也沒有因為自己誣告大祭司阿希米納而感到慚愧.

$^{561}$難怪大胃在詩篇的52篇.
罵多益這個人.
你說我怎麼知道.
詩篇的52篇是有關多益.
詩篇有說.
以東人多益來告訴素羅說.
大胃到了阿希米納家.
那時大胃作者奮沛思交與靈象.
所以這篇詩歌是大胃罵多益.
詩篇的52篇.
素羅於是拿著他的長矛.
在基比亞附近的小山上開庭審訊.
那支長矛是他權柄的象徵.
曾經三次扔向大胃.
一次是扔向他自己的兒子.
就在這個時候有人告訴他.
大胃可能躲在死海附近曠野.
他就大罵他的手下.
這班人都沒用的.
沒有人告訴我大胃和我兒子約拿擔納約的事.
他身為爸爸都不知道.
他去罵其他人沒有告訴他.
還有他開始利誘他的手下.
我才是皇帝.
有獎賞的都是由我給你們.
你以為大胃可以給你們獎賞嗎.
有沒有看到素羅是怎樣的人.
是威逼利誘.
一方面他罵手下.
另一方面利誘他們.
大胃需要這樣做嗎.
不需要.
大胃是有很迷人很吸引人的性格.
有很多人都願意為大胃來效忠.
但素羅是靠他自己的手段.
來用威逼利誘.
波益就開始誣告大胃.
首先他說.
大祭司阿希米納給了大胃食物.
給了他武器.

$^{601}$這兩件事都是真的.
阿希米納真的給了大胃武器和食物.
但第三件事他就是說謊.
他說阿希米納是用污淩和土茗.
來為大胃尋求神的旨意.
這就是我剛才所說的.
為什麼他會這樣說.
因為可能他看到大祭司的手上.
有哥利亞的劍和伊芬德.
而伊芬德的裡面.
就是有污淩和土茗.
所以他就很有想像力.
拿到伊芬德出來.
他一定是為大胃尋求神的旨意.
雖然這是謊言.
但素羅是很想聽的.
為什麼.
因為他要誣告大胃生豬肉.
他要說大胃謀反.
現在大祭司幫你尋求神的旨意.
來謀反來敵對素羅.
基比亞其實距離那帕很近.
所以素羅就叫人去接大祭司.
和他的家人來到基比亞這個地方.
素羅這個人真的很狡猾.
他根本不想給大祭司有任何榮耀.
他都不稱呼阿希米納為大祭司.
他也不叫阿希米納的名字.
為什麼.
阿希米納希伯來文的意思就是王的弟兄.
如果叫他做阿希米納.
是什麼意思.
就是和他稱兄道弟.
叫他做自己的好兄弟.
所以他叫他做什麼.
叫他做阿希特的兒子.
阿希米納的爸爸就是阿希特.
阿希特是什麼意思.
好兄弟.
你的爸爸和我是好兄弟.

$^{641}$你是阿希特的兒子.
你是低我一級.
有沒有看到素羅盡量在羞辱大祭司.
其實他很應該在大祭司面前.
來認罪悔改來懺悔.
因為他追殺神所高納的君王大位.
於是素羅就提出了三項罪名.
說阿希米納做了什麼.
第一 給逃犯有食物.
第二 給逃犯武器.
第三 為逃犯尋求神的旨意.
來謀反朝廷.
阿希米納首先沒有為自己辯護.
他首先是為大位辯護.
大位是一個好人.
是一個忠心的僕人.
而且是一個勇士.
其實這些東西約拿丹也有和素羅說過.
但素羅沒有聽到.
然後阿希米納說什麼.
大位是你的女婿.
是你的自己人.
又是你的侍衛長.
你這樣也要對付他.
要殺死他.
然後阿希米納否定.
我沒有用伊忽德.
我沒有用污名套命.
來尋求神的旨意.
我沒有謀反.
素羅根本沒有證據.
但他仍然決定阿希米納和他全家都要死.
其實這件事是很荒謬.
在使神的律例典章.
就算大祭司有罪.
罪不及他的家人.
不可能連他的家人都要處死.
唯一你說阿希米納做得不對的地方.
就是知情不報.
他見到大位.

$^{681}$他沒有告訴素羅.
但這也可以說是大位的謊言.
他和阿希米納說.
我有要事在身.
所以是個秘密.
所以阿希米納沒有告訴別人.
所以知情不報可能是大位誤導了他.
是因為大位的謊言.
素羅的衛兵不肯執行死刑.
他們知道阿希米納是無辜的.
這就像在撒母耳上的十四章.
素羅起了一個誓.
全軍隊都要禁食.
誰不禁食就要處死.
當時他的兒子約拿丹不知道有這個誓言.
吃了一些蜂蜜.
以致素羅要將他處死.
當時的士兵都不肯將約拿丹處死.
就好像現在一樣.
但這次不同了.
因為素羅手下有一個人蠢蠢欲動.
很想執行這個命令.
是誰呢?是多益.
他一口氣將阿希米納全家殺了.
然後去祭司城拿柏.
將85個祭司全部殺了.
他其實是想殺掉所有的祭司.
但走漏了一個.
遲些我們會說這件事.
他去殺死這些祭司.
連畜生都不肯放過.
難怪大衛在詩篇的52篇.
罵多益這個人.
他這個人真的非常詭詐.
這個血案.
拿柏的血案真的令我們很不安.
看到這麼多人死亡.
而且都是無辜.
但我們看到神是容許這件事發生.
拿柏的祭司被屠殺.

$^{721}$其實部分是應驗了神的預言.
這個預言是什麼呢?.
大祭司以來的後人.
將不會再成為大祭司.
殺了這些人.
其中所有都是以來的後人.
所以大祭司的職位.
以後就由殺毒家來代替.
這個也是神的預言.
所以雖然世上可能有些不好的事發生.
但神仍然可以透過他.
來成就他自己的旨意.
不是說神要殺死阿希米納一家.
但神就是能夠用這件事.
來成就他自己的旨意.
也讓我們看到.
當我們犯罪的時候會怎樣.
都是有後果.
今天的題目就是.
神會不會祝福危難中的大話.
我們看到每一個罪都有後果.
當我們說謊的時候.
是會有代價和後果.
很多時候我們認為.
無傷大雅.
讓我這裡說一句謊話.
又或者說這句謊話可以幫助到人.
不怕說了.
又或者我要救急.
所以我要說這句謊話.
但我們要小心.
不是說後面都有後果.
都有代價.
神會容許.
但也會用這些事來成就他自己的旨意.
大衛正式開始逃亡的生活.
但神仍然與他同在.
透過逆境幫助他成為以色列下一個君王.
有沒有看到.
神仍然掌權.

$^{761}$雖然大衛被素羅逼北.
但神仍然能夠透過這件事來鍛鍊他.
成為一個偉大的君王.
這就是我們相信的神.
是永遠掌權的神.
我們也可以透過兩個不同的角度來看這三個謊言.
第一個角度就是一個首尾結構.
你們都知道我很喜歡首尾結構.
時時都說首尾結構是什麼.
就是頭和尾的經文都很相似.
中間的經文是最重要的.
這三個謊言有什麼首尾結構.
第一和第三個謊言都是和一位王說謊.
第一個謊言是和素羅王說謊.
第三個謊言是和加特王亞吉說謊.
都是王.
中間是對著大祭司亞希米納.
在哪裡說的呢?.
在會幕說的.
在神的代言人面前你都敢說謊.
和王說謊已經很離譜.
你還要在神的面前說謊.
這就是我們看到事情的嚴重性.
我也知道有些朋友會說.
我平時都會說兩句謊言.
星期天回到教會就不會了.
現在大衛在神的面前仍然說謊.
這就是他的問題.
第二個就是漸進式的結構.
三個謊言越來越嚴重.
第一個謊言和素羅王說的已經不簡單.
第二個謊言在神的面前我們都說過.
更加離譜.
但這兩個謊言都在以色列本地.
第三個謊言在哪裡說的呢?.
在外邦人的地方說的.
你說多好啊.
不要騙自己人.
你說謊也要騙外邦人.
行不行?當然不行.

$^{801}$你想想我們基督徒.
讓教外的人知道這個人也說謊.
基督徒來的.
他也說謊.
我們是不是會羞辱神的名?.
絕對是羞辱神的名.
這就是事情的嚴重性.
大衛昔日在斐理世人的面前多威風.
打敗了哥利亞.
他還說我是靠神的名來戰勝哥利亞.
將榮耀歸神.
現在同樣在斐理世人面前.
他在做什麼?扮癲立福.
人們說這是以色列人的王.
有沒有搞錯?昔日打贏哥利亞.
現在變成這樣.
還不是羞辱神?.
所以我們都要很小心.
我們的謊言很多時候都會越來越嚴重.
就好像大衛所說的一樣.
罪的可怕.
大衛我們知道是合神心意的人.
但他真的犯了很多罪.
由一個層面的罪到另一個層面.
我們對罪要很敏感.
否則我們都會犯同一個錯.
這也是我開始的時候和大家說.
透過這些故事我們就會見到神學.
羅馬書第七章說什麼?.
我們要對自己的罪很敏感.
現在舊約一樣透過故事我們會見到一樣的教導.
很多時候我們認為大衛和拔士巴犯奸淫.
殺死烏尼亞只是一時的衝動.
大衛根本平時都不犯罪.
不是的.
你看看他在薩姆以上的十八章.
我們說十八章的時候都說過.
他首先做什麼?討好訴羅.
本來他可以娶米甲.
沒問題的.

$^{841}$因為是訴羅自己作出的誓言.
有誰戰勝哥利亞我將我的女兒許配給他.
但是大衛仍然希望戰勝斐尼斯人來討好訴羅.
這就是他第一個問題.
他敬畏人多過敬畏神.
第二個問題.
神已經幫助他避過訴羅的追殺.
有神跡騎士陪伴他.
他仍然不夠信心離開拉瑪爾.
靠自己的智慧說了三個謊言.
而且是越來越嚴重.
最後他犯奸淫犯謀殺.
你有沒有看到一步一步越來越嚴重.
這就是我們所說的.
我們真的需要處理我們細小的罪.
否則慢慢就會變得一發不可收拾.
另外一件事我們看到的.
雖然大衛是有罪.
但神仍然透過大衛帶出無罪的彌賽亞.
帶出耶穌基督.
耶穌基督就是大衛的後人.
透過大衛的軟弱.
讓我們學習到我們應該學習的功課.
而且也讓我們看到他帶出耶穌基督.
成為我們的救主.
雖然大衛犯很多罪.
但我們看到他不失為一個合神心意的人.
有一樣東西我們很需要學習的.
不是說大衛不犯罪.
但他犯罪之後也會很痛心.
也會懺悔.
這是很重要的.
詩篇的51篇就是當他和拔士巴犯了奸淫之後.
他所寫的.
你能夠看到他真的有很痛心,懺悔的心.
在神的面前認罪悔改.
起碼有8首詩篇是大衛逃離蘇羅的時候所寫.
其實我們上次也說了兩篇.
59篇和63篇.
今天也看到詩篇的52篇.

$^{881}$就是大衛鬧多益.
其中有兩篇是和我們今天的經文.
撒姆以上的20章至22章.
有直接連帶的關係.
就是56篇是說到大衛的思念.
和34篇是說到大衛的感恩.
我們先看詩篇的56篇.
是當時他在斐理士人當中.
斐理士人的領袖都誹謗他,毀謗他.
用言語來攻擊他.
希望加特王會殺死大衛.
所以在詩篇的56篇第5節至第6節.
他說「他們終日顛倒我的話.
他們一切的心思都是要害我.
他們聚集,埋伏,窺探我的腳zoung .
等候要害我的命」.
然後大衛就懇求神的憐憫.
他說「神啊,求你憐憫我.
因為人要把我吞了.
終日攻擊欺壓我」.
然後他懇求神紀念他.
他說「我幾次流離你都記數.
求你把我的眼淚裝在你的皮袋裡.
這不都記在你的冊子上嗎?」.
然後他就記起神的大能.
兩次他說「血氣之輩能把我怎樣呢?.
人能把我怎樣呢?」第4節和第11節.
如果他早點記得這兩句話.
他會不會逃離拉瑪?.
不會逃離拉瑪.
「人能把我怎樣呢?有神來幫助我」.
就是因為他缺少信心.
以致逃離拉瑪形成他說了三個謊言.
詩篇的34篇是另外一個詩篇.
是說到他讚美神.
詩篇裡34篇說到他的懼怕.
但他說「就算我懼怕.
神仍然可以從患難中來救贖我」.
詩篇34篇也讓我們看到.
他在迦牟尼作出很多禱告.

$^{921}$例如34篇的4-6節.
17-22節.
他知道敬畏耶和華能勝過一切的懼怕.
他懇求神,他知道神會憐憫他.
會將他帶回自己的領土,自己的土地.
詩篇34篇12-13節說.
「有何人喜愛存活,愛慕長壽,得享美福.
就要禁止洩頭不出惡言,嘴唇不說詭詐的話」.
有沒有看到?.
很明顯他因為自己說了三個謊言.
所以他在這裡說.
一個真正向著神的人嘴唇不說詭詐的話.
他開始後悔.
他知道自己的謊言造成很大的後果.
詩篇34篇第22節.
他說「耶和華救贖他僕人的靈魂.
凡投靠他的必不自定罪」.
在今天結束的時候.
我想和大家再回溫一下.
大衛就是因為他信心不足.
離開拉瑪爾至說了三個謊言.
一個比一個的嚴重.
「類至大祭司一家人被殺.
祭司乘85個祭司被屠殺」.
有人說「多好啊!大衛說謊話和不及己.
那些懲罰去了別人那裡.
不怕說了」.
當然不是.
你知道你說謊話會害到別人也不想.
但是當大衛沒有學到他的功課的時候.
最後的審判真的臨到他自己身上.
當他和拔士巴藩奸淫殺死烏尼亞的時候.
拿丹仙芝和他說「刀劍將不離你家」.
甚至大衛會四倍償還.
你說為什麼?.
這是大衛自己說的.
當拿丹仙芝用一個比喻告訴大衛他的罪.
說有一個富有人家家裡很多羊.
他都沒有殺死來招待他的客人.
當他有客人的時候.

$^{961}$他偷了他的鄰居的羊.
那個鄰居只有一隻羊.
就是拿這隻羊來招待客人.
大衛很生氣.
大衛說什麼?衝口而出.
這個人要四倍償還.
他真的要四倍償還.
刀劍不離他的家.
第一個是什麼?.
他和拔士巴的小孩死了.
這是第一個.
第二個呢.
他的大兒子暗戀姦污了妹妹.
被亞沙隆殺死.
這是第二個.
第三個呢.
亞沙隆自己反叛大衛.
以致被約克殺死.
第四呢.
是他第四個兒子大衛.
第四個兒子亞多尼亞.
因為他和所羅門爭奪皇位.
以致被所羅門殺死.
有沒有看到?.
真的四倍償還.
所以我們真的要很小心.
仍然帶領耶穌基督來世上.
來成為我們的救主.
一切都是按照他的旨意來進行.
我們看到神就是掌權的神.
我不知道今天聽我這篇道的.
有沒有沒有接受耶穌基督.
你會看到神是掌權的神.
你願不願意接受這位神成為你個人的救主.
我希望如果你沒有接受耶穌基督的話.
你今天就會接受他.
但我們已經接受了耶穌基督.
我們要很小心.
我們千萬不要在小事上妥協.
就算我們身上有小的罪.

$^{1001}$我們都要處理.
不要以為小的罪就沒有問題.
在大衛的身上我們看到.
小的罪會變得越來越大.
我們要對我們自己身上的罪有敏感.
應不應該說謊.
白色的謊言.
能夠幫助人的謊言.
救人的謊言.
應不應該說?.
我相信這篇道給我們看到.
我們都不應該說.
願神祝福他自己的懷孕.
讓我們低頭禱告.
三號以上的第20章至第22章.
我們看到你就是那位掌權的神.
一切都是照著你的旨意在進行.
願意在我們當中未接受你的.
今天就能夠作出決定成為你的兒女.
願意我們已經接受你的人.
都能夠監察我們自己的身上.
有沒有一些小小的罪是我們沒有處理的.
願意我們就像羅馬書第七章一樣.
教導我們要對我們自己的罪敏感.
願意我們在我們成聖的路上.
能夠越來越像耶穌基督.
能夠處理我們身上越來越小的罪.
以致我們能夠成為一個夢理喜悅的人.
將一切的榮耀頌讚歸給你.
我們今天的禱告奉主耶穌基督的名而投.
阿門.
\newpage



\section{撒母耳記上 22:1-5-20-23}
\label{sec:WCt7vYrgwVY}
\textbf{走出憂鬱與黑暗的秘訣 (撒母耳記上22\_1-5,20-23) - 袁惠鈞牧師[大衛傳系列 - 第5講]}
\newline
\newline
連結: \href{https://youtube.com/watch?v=WCt7vYrgwVY}{\texttt{ https://youtube.com/watch?v=WCt7vYrgwVY}} ~~~~ 語音日期: 2025-02-12 
\newline
\newline
\hyperref[sec:rN0dS2BBBmc]{< < < PREV SERMON < < <}
~
\hyperlink{toc}{[返主目錄]}
~
\hyperref[ch:preacher6]{[返講員目錄]}
~
\hyperref[sec:GqTOPwqfjwM]{> > > NEXT SERMON > > >}
\newline
\newline
撒母耳記上 22:1-5-20-23
\newline
\begin{longtable}{cl}
\hline
\hline
章節 & 經文 (和合本修訂版)\\
\hline
22:1 & \begin{tabularx}{0.7\textwidth}{X} 大衛離開那裡,逃到亞杜蘭洞。他的兄弟和他父親全家聽見了,都下到他那裡去。 \end{tabularx} \\ \\ \relax
22:2 & \begin{tabularx}{0.7\textwidth}{X} 凡生活窘迫的、欠債的、心裡苦惱的都聚集到大衛那裡,他就作他們的領袖,跟隨他的約有四百人。 \end{tabularx} \\ \\ \relax
22:3 & \begin{tabularx}{0.7\textwidth}{X} 大衛從那裡往摩押的米斯巴去,對摩押王說:「請你讓我父母搬來,跟你們在一起,等我知道神要為我怎樣做。」 \end{tabularx} \\ \\ \relax
22:4 & \begin{tabularx}{0.7\textwidth}{X} 大衛領他父母到摩押王面前。大衛住山寨一切的日子,他父母也都住在摩押王那裡。 \end{tabularx} \\ \\ \relax
22:5 & \begin{tabularx}{0.7\textwidth}{X} 先知迦得對大衛說:「你不要住在山寨,要到猶大地去。」大衛就去,來到哈列的樹林。 \end{tabularx} \\ \\ \relax
22:6 & \begin{tabularx}{0.7\textwidth}{X} 掃羅聽見大衛和跟隨他之人的下落,掃羅正在基比亞,坐在山頂的柳樹下,手裡拿著槍,眾臣僕侍立在左右。 \end{tabularx} \\ \\ \relax
22:7 & \begin{tabularx}{0.7\textwidth}{X} 掃羅對左右侍立的臣僕說:「便雅憫人哪,聽著!耶西的兒子也能把田地和葡萄園賜給你們各人嗎?他能立你們各人作千夫長和百夫長嗎? \end{tabularx} \\ \\ \relax
22:8 & \begin{tabularx}{0.7\textwidth}{X} 你們竟都結黨害我!我兒子與耶西的兒子立約的時候,無人告訴我;我兒子挑唆我的臣僕謀害我,像今日這樣,也無人告訴我,為我憂慮。」 \end{tabularx} \\ \\ \relax
22:9 & \begin{tabularx}{0.7\textwidth}{X} 那時以東人多益站在掃羅的臣僕中,回答說:「我曾看見耶西的兒子往挪伯去,到了亞希突的兒子亞希米勒那裡。 \end{tabularx} \\ \\ \relax
22:10 & \begin{tabularx}{0.7\textwidth}{X} 亞希米勒為他求問耶和華,給他食物,又把非利士人歌利亞的刀給了他。」 \end{tabularx} \\ \\ \relax
22:11 & \begin{tabularx}{0.7\textwidth}{X} 王就派人把亞希突的兒子亞希米勒祭司和他父親的全家,就是在挪伯的祭司都召了來,他們都來到王那裡。 \end{tabularx} \\ \\ \relax
22:12 & \begin{tabularx}{0.7\textwidth}{X} 掃羅說:「亞希突的兒子,聽著!」他說:「我主,我在這裡。」 \end{tabularx} \\ \\ \relax
22:13 & \begin{tabularx}{0.7\textwidth}{X} 掃羅對他說:「你為甚麼與耶西的兒子結黨害我,把食物和刀給他,又為他求問神,使他起來謀害我,像今日這樣?」 \end{tabularx} \\ \\ \relax
22:14 & \begin{tabularx}{0.7\textwidth}{X} 亞希米勒回答王說:「王的眾臣僕中有誰比大衛忠心呢?他是王的女婿,又是你的侍衛長,並且是你宮中受敬重的人。 \end{tabularx} \\ \\ \relax
22:15 & \begin{tabularx}{0.7\textwidth}{X} 我今日才開始為他求問神嗎?絕非如此!王不要歸罪於我和我父全家,因為這事,無論大小,僕人都不知情。」 \end{tabularx} \\ \\ \relax
22:16 & \begin{tabularx}{0.7\textwidth}{X} 王說:「亞希米勒,你和你父全家都是該死的!」 \end{tabularx} \\ \\ \relax
22:17 & \begin{tabularx}{0.7\textwidth}{X} 王吩咐左右的侍衛說:「你們轉身去殺耶和華的祭司吧!因為他們幫助大衛,知道大衛逃跑卻不告訴我。」但王的臣僕都不願動手殺耶和華的祭司。 \end{tabularx} \\ \\ \relax
22:18 & \begin{tabularx}{0.7\textwidth}{X} 王吩咐多益說:「你轉身去殺祭司吧!」以東人多益就轉身去殺祭司,那日殺了穿細麻布以弗得的,共八十五人, \end{tabularx} \\ \\ \relax
22:19 & \begin{tabularx}{0.7\textwidth}{X} 又用刀把祭司城挪伯中的男女、孩童和吃奶的都殺了,又用刀殺了牛、羊和驢子。 \end{tabularx} \\ \\ \relax
22:20 & \begin{tabularx}{0.7\textwidth}{X} 亞希突的兒子亞希米勒有一個兒子逃脫了;他名叫亞比亞他,逃到大衛那裡。 \end{tabularx} \\ \\ \relax
22:21 & \begin{tabularx}{0.7\textwidth}{X} 亞比亞他把掃羅殺耶和華祭司的事告訴大衛。 \end{tabularx} \\ \\ \relax
22:22 & \begin{tabularx}{0.7\textwidth}{X} 大衛對亞比亞他說:「那日我見以東人多益在那裡,就知道他一定會告訴掃羅。你父的全家喪命,都是因我的緣故。 \end{tabularx} \\ \\ \relax
22:23 & \begin{tabularx}{0.7\textwidth}{X} 你可以住在我這裡,不要懼怕。因為尋索你命的也要尋索我的命,你在我這裡可得保護。」 \end{tabularx} \\ \\
[1ex]
\hline
\hline
\end{longtable}
$^{1}$大胃傳的第五講.
我們上次說了三號以上的20章至22章.
其實我們還沒說完22章.
還有22章的1至5節和20至23節.
我們是要今天來講.
而我們今天也不單止會講22章.
也會涉及到23章25,27和30.
因為有些經文說有些人加入了大胃的隊伍.
我們也會說這些經文.
我們說我們讀大胃傳.
一個很重要的目的就是要欣賞大胃的詩歌.
我們其實已經讀了有五篇大胃的詩歌.
34篇,52篇,56篇,59篇和63篇.
今天我們還要看57篇和142篇.
就是和今天的經文有關.
我們說我們要在大胃身上學習.
怎樣做一個合神心意的人.
其實在詩篇裡我們學到很多東西.
記不記得上次在詩篇的56篇.
大胃最後怎麼說.
他說人能打我怎樣呢.
他最後發現原來神才是他的保守.
才是他的避難所.
如果他早點發現的話.
他就不會離開拉瑪.
不會離開撒慕爾.
離開神的保佑離開祝福.
他就是因為信心不足所以離開了.
而在34篇他又教到我們一些東西.
他離開拉瑪之後.
一連串的說了三個謊言.
而且他的謊言是越來越嚴重.
而去到詩篇的34篇.
他說一個屬靈的人應該怎樣呢.
嘴唇不說鬼詐的話.
所以我們真的從這些詩篇能夠學習到.
怎樣能夠成為一個合神心意的人.
今天的主題是什麼.
走出幽屈黑暗的秘訣.
是說到大胃去到一個山洞裡面.

$^{41}$這個就是亞渡難動.
有很多時候我們都好像走進了山洞裡面.
漆黑一片前路茫茫.
這個就是幽屈黑暗.
很多人有幽屈黑暗就會面臨一樣的處境.
我們今天看看透過詩篇的142篇和57篇.
教我們怎樣能夠走出幽屈黑暗.
我們說雖然聖經叫大胃做合神心意的人.
他不是一個無罪的人.
聖經從來不會遮掩人的罪.
表揚人的功績.
聖經不會這樣做.
聖經是照事論事.
我們每一個讀撒母耳記的人.
都能夠認同大胃身上所發生的一切事.
上一次我們說大胃去到他人生最低點.
這個是暫時的最低點.
將來他會和拔士巴梵奸人殺死烏里亞.
使他的人生又去到另一個低點.
但是現時來說是他人生的最低點.
他說謊.
他和約拿丹來夾定.
和索羅說謊.
自己躲在以色列的田野裡面.
但是就騙索羅說去了伯利恆來獻祭.
其實他就是想看看索羅對他的反應是怎樣.
有人可能會說大胃說謊沒有錯.
因為索羅想殺死他.
他應該保護自己的.
他應該說謊.
又有人說索羅是個小人.
是個詭詐的人.
你應該是以其人之道還施其身.
跟他說謊撕詭詐天公地道.
但是聖經怎樣教我們.
教我們要靈巧仗羞.
要保護自己.
但是不要忘記還有一句.
順樑杖夾子.
就是就算遇見危難.

$^{81}$就算有很大的危險.
我們都不可以說謊.
不可以詭詐.
大胃說了謊.
而且他的謊言是越來越嚴重.
以至濾到祭司阿希米納全家被殺.
祭司城裡面85個祭司被屠殺.
我上一次還說漏了一樣東西.
那個多益去到祭司城.
男女老幼都殺了.
連畜牲都不放過.
這就是大胃說謊的後果.
於是他就躲到加特.
在加特王面前扮瘋癲.
認錯神 羞辱了神.
你看看詩篇34篇17節怎樣說.
義人呼求耶和華聽見了便救他們脫離一切患難.
神救什麼人啊.
耶和華告誰啊.
是義人.
所以當我們遇見困難.
遇見有危難的時候.
仍然要純良像甲子.
千萬不要撕鬼砸.
大胃知道神一定會救他.
但不是立刻.
我們生活在一個節奏很快的社會.
樣樣你都希望快點得到.
我記得我在香港的時候.
想買一樣東西.
就放一些毫子到豬仔錢罐裡.
儲錢 儲夠了就.
不是說打破的 豬仔錢罐下面有個開口.
拿了錢出來就買東西.
現在的人會不會還是用豬仔錢罐.
不會的 去看信用卡.
先洗未來錢急不及待.
你想吃東西怎樣.
去快餐店夠快.
快餐店都不夠快.

$^{121}$你還要去快餐店的外賣.
樣樣你都希望快點得到.
飲咖啡就是即食咖啡.
你食麵又怎樣 即食麵.
樣樣都是立即想得到.
但神不是這樣的.
神有他的時間表.
他是會打救大胃.
但不是立刻.
神是會最後將他擺上以色列人的王位.
但現在來說他仍然要逃避蘇羅的追殺.
而且他將來登上王位.
他首先要登上猶大國的王位.
七年半然後才會成為整個以色列的王.
這個就是神的時間表.
那你說為何神一次過打救大胃.
因為大胃還有很多東西需要學習.
需要神來磨練他鍛鍊他.
他需要經過很多的試驗試探.
然後才會成為一個合神心意的人來統治以色列.
我今天的這一篇是有三個段落.
我是用三個地點來分段.
三個不同的地方.
第一個地方就是亞都蘭洞.
大胃為何會去到亞都蘭洞.
你記不記得上一次說他去了加特的阿吉王那裡.
阿吉的隨從希望阿吉王會殺死大胃.
於是將大胃帶到阿吉王的面前.
阿吉王怎麼說呢.
我這裡有很多瘋子 很多瘋子.
你還加多一個大胃.
趕他走 趕他走.
大胃就逃出了 逃到以拉這個地方.
以拉就是他當年打勝哥利亞的地方.
以拉在哪裡呢.
以拉的亞都蘭洞其實是在加特 伯利恆和希伯倫的中間.
距離加特大概十里 距離伯利恆大概十五里.
在那裡有很多山洞.
那你說誰才是亞都蘭洞呢.
其中有一個很大的山洞.

$^{161}$可以容納四百至六百人.
有很多學者說這個就是亞都蘭洞.
如果你現在去以拉這個地方.
你都會見到這個洞 就是亞都蘭洞.
我們遲些會見到大胃會有四百至六百人跟隨他.
不妨看看這個地理.
我們上次說大胃因為不夠信心離開了拉瑪.
拉瑪就在那裡很不便.
然後他去到基比亞和約拿丹商議.
如何騙訴羅.
然後發現訴羅真的要殺自己.
他就跑去拿柏.
然後祭司城被殺的慘劇.
他就去了加特.
然後由加特去到亞都蘭洞.
遲些還有一個地方.
如果大家見到就在死海旁邊有個叫馬薩達的地方.
馬薩達翻譯出來希伯來文就是山寨.
就是堡壘.
很多人說大胃曾經躲在那裡.
等下我們會說這個事.
現在我們要見到的就是大胃走進了亞都蘭洞.
很多時候我們都是一樣.
好像大胃一樣.
在一個黑漆一片的大洞裡面.
伸手不見五指.
好像沒有出路一樣.
很多人有憂鬱有情緒病就是一樣.
有很多人說基督徒不會有憂鬱病.
不是的.
基督徒一樣會有憂鬱症.
我們怎麼知道.
記不記得聖經裡面的以利亞.
他被耶齊別追殺.
他逃到曠野.
他是尋死.
甚至他低落到希望尋死.
還有耶利米先知被喚為哭泣的先知.
今天還有大胃.
他一樣被掃羅追殺.

$^{201}$我相信他一樣有很低落的情緒.
就在這個時候.
當大胃走投無路的時候.
神就帶領了四百個勇士來投靠大胃.
去到亞都蘭東.
在撒母耳上的二十二章第二節.
凡受困迫的欠債的心裡苦惱的.
都聚集到大胃那裡.
大胃就作他們的頭目.
跟隨他的約有四百人.
這些人都知道大胃就是神所高納的王.
他們在大胃身上都看到有希望來投靠大胃.
這班是什麼人呢.
是陷入困境的人.
是負債的人.
有沒有見到經文說是欠債的人.
有很多這些人相信是掃羅使他們欠債.
使他們陷入困境.
我們怎麼知道呢.
在撒母耳上的十四章二十九節.
約拿丹說對不起.
他跟士兵說是我的爸爸幫你們添麻煩.
我相信在撒母耳上的十四章.
不是唯一次掃羅替以色列人添麻煩.
所以我們見到有很多人都是民不聊生.
以至有四百人去投靠大胃.
這四百人很快就會變成六百人.
部分這些勇士的名單.
你都可以在撒母耳下的二十三章和歷代史上的十一章.
你可以看到.
有很多人說現在大胃有六百個勇士跟著他.
他不用怕掃羅追殺他.
不是的.
你看看撒母耳上的二十六章.
掃羅的軍隊有多少人呢.
三千人.
而且這些是選民所組成的.
是被選舉出來的士兵.
是精兵.
大胃那些是流氓跟隨他.

$^{241}$所以大胃仍然寡不敵眾.
仍然強弱懸殊.
這班人都是被拒絕的人.
但他們將來就會成為以色列國家的棟樑.
這班人成為了以色列人的希望.
成為將來以色列人的祝福.
當他們被大胃訓練的時候.
當大胃建立了他的國.
他們在大胃的朝廷裡成為了很重要的官員.
一個真正的領袖.
是會吸引很多優秀的人跟隨自己.
這班人都很佩服大胃的品格.
很佩服大胃的質素.
若非大胃跟他們交往的話.
相信這班人的名字不會名流青史.
不會流傳在聖經裡.
大胃就得到這麼多人跟隨自己.
因為他很有質素很有品格.
上次我也說過.
掃羅是否一樣有這麼多人願意跟隨他.
不是的.
掃羅是要威逼利誘人才會跟隨他.
其實耶穌基督揀選門徒都是一樣.
耶穌是否揀選那些有智慧有能力有權勢的人.
不是的.
他是揀選那些卑賤的被社會所遺棄的.
被社會所厭棄的人.
這就像哥林多前書的一章二十六至三十一節所說.
神是揀選那些軟弱的愚蠢的.
以至使有智慧的羞愧.
那為什麼神要揀選軟弱的愚蠢呢.
很簡單.
因為如果神揀選有智慧有能力.
如果成功了.
榮耀歸給誰呢.
榮耀歸給這些人.
因為他們有能力.
但如果神揀選軟弱愚蠢.
而成功了的話.
榮耀歸給誰呢.

$^{281}$榮耀歸給神.
所以我們看到耶穌基督的時候.
他揀選那些愚蠢的.
瑞利就是當時被以色列人所藐視的人.
很多時候當我們憂鬱的時候.
會心煩意亂.
不想做任何事.
很多憂鬱的人會怎樣呢.
沒有心情做事.
這就是憂鬱.
當你發現自己什麼都不想做的時候.
你就要小心.
你可能已經開始有憂鬱症.
在這個時候.
神就帶領了四百個人來到大衛.
你說好啊.
大衛有憂鬱症.
什麼都不想做.
現在神派四百個人來幫他做事.
不是的.
你如果是做領袖的話.
四百個人跟著你不是一件小事.
只是你想找每天給他們吃什麼.
四百個人你都夠頭痛.
所以其實是給大衛有機會來侍奉神.
所以當我們有憂鬱症.
當我們什麼都不想做.
其實我們應該用多點時間來侍奉神.
才能夠走出來.
憂鬱症是什麼呢.
是將注意力放在自己身上.
但是當你有事要做.
將你的腦拿開.
從你自己想著自己的困境.
而去想其他事的時候.
你就會走出憂鬱症.
這個也是上次我和大家說過.
記不記得我說素羅他有妒忌.
一個人為什麼會有妒忌呢.
又是想將榮耀歸給自己.

$^{321}$又是有個自我在.
我說過如果有妒忌的話.
最好的辦法就是侍奉神.
將這個目標轉移了.
不是將榮耀歸給自己.
而是將榮耀歸給神.
你就會醫到你的妒忌病.
如果你有憂鬱的話是一模一樣.
不要將注意力放在自己身上.
將注意力放在神的身上.
你去侍奉祂.
你就能夠一步一步走出來.
神揀選人來侍奉祂.
也是揀選人最快樂的時候.
才給你侍奉.
為什麼呢.
因為一個人情緒最低落的時候.
就是倚靠神的時候.
不是倚靠自己的能力.
中國人其實也是很相似的.
萬子說什麼.
天將降大任於斯人也.
必先苦其心志.
勞其筋骨.
餓其體膚.
空乏其身.
當你是快有大的任務.
任務上你的身之前.
你就會有這樣的經歷.
其實聖經裡面也是一樣的教導.
在希伯來書第十一章.
我們看到有信心的龍虎榜.
希伯來書記載了.
很多有信心的人.
在希伯來書第十一章.
但是作者怎麼說.
在希伯來書十一章三十八節.
這些人都在曠野山嶺山洞.
歷月漂流無定.
管事世界不配有的人.

$^{361}$這班人要成為有信心的人.
他們首先要像中國人說的.
勞其筋骨.
以至神才能夠將重要的任務給他們.
在聖經裡面很多屬靈的領袖.
都是一樣的.
記不記得摩西.
他在米田的曠野多少年.
四十年來訓練他.
曠野的生活不容易捱.
約瑟被他的兄弟扔進坑裡.
後來又賣去埃及.
還坐牢坐很久.
以利亞我剛才說過.
被耶駛別追殺.
甚至是求死去到曠野.
以利沙就是以利亞的徒弟.
在約旦河旁邊.
突然間見到以利亞被神接去.
那種心情你可以想像.
沒有依靠.
很失落.
恩師都被神接去了.
我怎麼辦.
但以利亞在獅子坑裡.
耶利米在泥坑裡.
所以我們見到.
很多時候我們都要經歷這些事.
才能夠被神所用.
當我們在憂鬱.
在黑暗裡.
我們就要記得.
我們要順靠神.
慢慢慢慢神就會提升我們.
脫離我們的處境.
我們剛才說了.
我說金編道有三個地方.
第一個就是亞都蘭洞.
大衛去到這個洞裡.
他情緒很低落.

$^{401}$很困擾.
還有第二個地方.
跟著他去到摩亞.
他去摩亞只不過是暫時.
為何他會去到摩亞.
因為當時他的家人.
來投靠自己.
其實本來他家人對大衛.
是沒有好感的.
記不記得當薩姆爾.
去高納素羅的承繼人.
耶西差點忘記了.
自己還有個孫子.
根本就忘記了他.
跟著大衛去到以拉.
面對巨人哥利亞.
他的兄弟跟他說.
你苛管閒事.
有所好閒.
你看管的那幾隻羊.
幾隻羊你都看不見.
你為何不回去.
為何在這裡.
但現在他們開始.
來投靠大衛.
其實大衛的兄弟.
本來是在素羅的軍隊裡.
現在都棄暗投明.
來到大衛那裡.
其實這個和耶穌基督.
很相似.
耶穌基督在世的時候.
他本來的兄弟都不信他.
約翰福音七章五節.
他們根本不信耶穌基督.
但後來他們都成為.
耶穌基督的跟隨者.
瓦國 猶大.
兩個在新約裡.
都有書信.

$^{441}$大衛的家人.
來投靠自己.
又有一個問題.
大衛是通緝犯.
現在他的家人.
都變成通緝犯.
很危險的.
他父母都來投靠自己.
所以他要將他的父母.
帶到摩鴉.
給摩鴉王照顧他們.
有人會問.
那麼多地方不去.
為什麼大衛要去摩鴉.
摩鴉給以色列人的印象.
不是那麼好.
首先他是羅德和大女.
亂倫所生的後裔.
以色列人都看不起他們.
還有在摩西時代.
摩鴉的女子.
還來引誘以色列人.
犯姦淫.
民數記的二十五章.
所以不是很好的印象.
但記不記得.
大衛的真祖母是誰.
是路德.
路德是什麼人.
是摩鴉女子.
所以大衛認為.
將他的父母帶到摩鴉.
摩鴉王會照顧他們.
當摩鴉王.
答應照顧大衛的父母.
大衛就回到亞都蘭東.
我們看到大衛.
真的很孝順.
他先照顧了自己的父母.
雖然大衛現在有家人.

$^{481}$他的兄弟.
和六百個勇士.
和他在一起.
但他仍然很孤單.
你說為什麼.
一個人有憂鬱症的時候.
雖然有家人.
有朋友.
護著你.
你都會覺得很孤單.
很多時候.
人來說.
依賴朋友.
家人.
但當你有憂鬱症的時候.
你發覺只有神.
你才可以依靠.
你才是真正的避難所.
有很多人說.
大衛是領袖.
領袖怎會孤單.
其實我可以告訴你.
領袖其實是非常孤單.
你不要以為.
有那麼多人跟隨他.
但是領袖的生活.
是非常非常孤單.
在這個時候.
大衛學到什麼.
在這個時候他就寫了.
詩篇的57篇.
和詩篇的142篇.
我叫亞道蘭洞.
做洞中的書院.
因為他在那裡.
學到很多東西.
從詩篇的142篇和57篇.
我們可以明白.
大衛內心的處境.
在這兩篇的詩歌.

$^{521}$大衛都叫神做他的避難所.
尤其是.
詩篇的142篇.
是一篇的訓誨詩.
訓誨詩是什麼.
就是有教訓的詩篇.
在這詩篇裡.
一共有13篇訓誨詩.
就是詩篇的.
32篇42.
44 45.
52至55.
74 78.
88 89.
和142篇.
是給我們有教訓的詩篇.
就是當我們遇見.
有很相似的情形的時候.
這些詩篇都成為我們的力量.
當我們讀這些詩篇的時候.
我們就懂得怎樣做.
例如今天.
142篇.
給我們看到大衛進入了黑洞.
情緒低落.
當我們有憂鬱症.
當我們發覺沒有前途.
當我們發覺沒有希望的時候.
這一篇的詩篇.
就成為我們的座右銘.
要走出憂鬱黑暗.
今天我們看到.
一個人的心.
今天我們看到.
一共有六點.
就是由詩篇的142篇.
和詩篇的57篇.
我們所見到.
我本來想叫這篇.
道做從傾訴患難.

$^{561}$到崇洋主名.
為什麼有這樣的題目.
因為有六個步驟.
首先是要向神.
傾訴患難.
告訴神.
你的患難你的苦處.
然後去到最後第六點.
崇洋主名.
當你由傾訴患難.
去到讚美神.
你的憂鬱症就會慢慢好起來.
所以第一點.
是什麼?.
我們要向神哀求.
神是准許我們.
向祂發洩我們的情感.
我們可以向神.
大大地來呼求.
就好像詩篇的142篇.
一至二節.
大衛說.
我發聲宣告耶和華.
然後再重複.
發聲懇求耶和華.
我在祂面前套路.
我的苦情.
陳說我的患難.
有沒有看到套路我的苦情.
神就是要我們在祂的面前.
來套路我們的苦情.
發聲這個字.
原文的希伯來文.
的意思是什麼?.
大聲叫.
在山洞裡大聲叫.
有回音的.
很恐怖的.
大聲叫有回音.
山洞裡還有600個人.

$^{601}$當大衛在叫的時候.
你猜他們會覺得怎樣?.
我們的領袖瘋了.
為什麼會這樣?.
大衛已經放下了自己.
他已經不在乎.
他向神來呼叫.
有很多人.
在大衛面前.
有很多人.
在患難當中都不肯.
向神呼求.
為什麼?有很多人會說.
神都知道我需要什麼.
何必向神呼求.
神是知道一切的.
但是.
神和我們的關係.
就好像父母和子女的關係.
一樣.
當我們的子女需要什麼.
你猜父母不知道嗎?.
父母絕對知道.
但是當你的子女.
來跟你說.
爸爸,我需要這東西.
或者去跟媽媽說.
媽媽,你可不可以.
幫我做這件事.
是不是特別的甜蜜.
特別的親切.
所以神都是一樣.
他當然知道我們需要什麼.
但當我們向他呼求.
向他哀求的時候.
他覺得特別的親切.
特別的甜蜜.
在大衛面前有很多例子.
我們看到都是.
向神呼求.

$^{641}$第一個例子就是在耶利米哀歌.
裡面.
我們都知道耶利米哀歌.
是怎樣成書的.
就是以色列人亡國.
耶路撒冷被淪陷.
聖殿被毀.
人被屠殺.
所以耶利米先知.
叫以色列人起來呼喊.
在主的面前.
傾心如水.
向主舉手禱告.
耶利米哀歌的二章十九節.
這個和大衛所說.
向神吐露苦情.
是不是一模一樣.
傾心如水.
向神說出你的苦處.
還有.
在《說撒姆》以上的一章十五節.
記不記得哈娜.
哈娜的故事.
她沒有兒子生.
二奶又有兒子又有女兒.
在她的面前.
經常嘲笑哈娜.
於是哈娜在耶和華的面前.
傾心吐意.
《說撒姆》以上的一章十五節.
傾心如水.
還有大衛向神吐露苦情.
都是一模一樣.
大衛已經走投無路.
他離開了自己的太太米甲.
他離開了自己最好的朋友約拿丹.
他說了謊話.
令到整個祭司城的人全部被屠殺.
現在還被訴羅繼續追殺.
你想想他會覺得怎樣.

$^{681}$他真的情緒低落.
他真的憂鬱.
所以他向神來呼求.
很多時神就會像大衛一樣.
將我們身邊的東西全部拿走.
當我們身邊還有很多東西的時候.
我們將注意力放在哪裡呢.
就是這些東西的身上.
親朋戚友.
我們所有的東西的身上.
但當神將這些東西全部拿走.
迫使我們歸向神.
很多時我們都是將注意力放在親朋戚友的身上.
一有事的時候我們會怎樣呢.
馬上拿起電話.
打電話給我的朋友.
看看他們教我怎樣做.
他們有什麼意見.
但我們所應該追求的是神.
不是我們的親朋戚友.
有沒有看到大衛在詩篇的142篇.
他怎樣說呢.
等一會我們會解釋這一節的經文.
他說我曾向你哀求.
求你質疑聽我的呼求.
求你救我脫離.
逼迫我的人.
求你領我出來.
被困之地.
有沒有看到求你求你求你.
不斷出現.
大衛就是向神呼求.
你可以想像他就是在山洞裡.
不斷向神呼求.
為什麼有些人遇見萬難.
還不向神呼求呢.
他們認為他們仍然可以.
處理他們眼前的事.
仍然可以面對他們眼前的事.
他們就不向神呼求.

$^{721}$這就帶領我們.
去到第二點.
我們要承認自己的軟弱.
很多時候當我們不承認.
自己的軟弱.
我們不會向神呼求.
在詩篇的142篇.
第三節大衛說.
我的靈在我面前.
發奔的時候.
你知道我的道路.
發奔這個字是什麼意思.
是軟弱的意思.
神啊你知道我發奔.
你知道我軟弱.
你知道我筋疲力遠.
你知道我山窮水盡.
我不能夠再支撐下去.
我不能夠再依靠自己.
這個就是原因.
神將我們身邊的東西.
全部拿走.
你說.
神這樣做不愛我.
不是的.
神就是愛你才將你的資源.
全部拿走.
你是可以依靠祂.
以至你是可以將注意力.
放在祂的身上.
大衛說我筋疲力遠.
我發奔.
神甚至會容許.
有病淋到你的身上.
甚至會容許.
有問題出現在你的精神.
在你的心理上.
甚至會容許將你物質上的東西.
一切拿走.
不是祂不愛你.

$^{761}$剛剛相反.
是祂把一切拿走.
你才會承認自己的軟弱.
你才會依靠神.
在述詩篇的142篇第四節.
大衛說求你.
因為沒有人認識我.
我無處避難.
也沒有人眷顧我.
以色列人.
是他們的傳統.
最重要的人站在他右邊.
就像我們拍照.
他們會將最重要的人.
放在他們的右邊.
現在大衛說.
沒有人.
沒有人站在我右邊.
沒有人能照顧他.
沒有人能幫助我.
沒有人眷顧我.
不是有家人.
大衛有很多家人和勇士.
在他四周嗎.
我剛才也說過.
當你有憂鬱症.
當你有情緒病的時候.
大把人污著你也沒有用.
你會覺得非常孤單.
在詩篇的142篇第六節.
大衛說我落到極悲之地.
求你救我脫離迫迫我的人.
因為他們比我強.
有沒有看到大衛認輸.
他們比我強啊神.
我發瘋啊.
我孤立啊.
我已經認命了.
他完全放下了他自己的承認.
他自己的怨言.

$^{801}$這個是誰來的.
這個是大衛啊.
是以色列人將來的王.
都會是這樣.
所以你感到憂鬱.
你感到情緒低落.
你不要自卑.
不要緊的.
當你走出來的時候.
你可能像大衛那樣.
大大地被神使用.
當我們有憂鬱的時候.
第一步就是向神來呼求.
但不要停在那裡.
要承認自己的軟弱.
還有我們要倚靠神.
在詩篇的142篇第五節.
耶和華啊.
我曾向你哀求.
我說你是我的避難所.
在活人之地你是我的福分.
這節經文的文法很有趣.
是現代式.
不是過去式.
大衛不是說你以前是我的避難所.
是我的福分.
大衛也不是說你將來會是我的避難所.
會是我的福分.
大衛說你現在就是我的避難所.
你現在就是我的福分.
大衛在哪裡啊.
在山洞裡面漆黑一片.
但他覺得神與他同在.
他覺得神的福分與他同在.
這就是他有倚靠神的心.
詩篇的142篇第三節.
你知道我的道路.
在我行的路上.
敵人為我暗設網羅.
雖然敵人暗設網羅.

$^{841}$但大衛說不怕.
你知道我的道路.
我倚靠你.
你會打救我.
本來大衛覺得.
躲在山洞裡面像監獄一樣.
你看回詩篇的142篇第七節.
他就覺得好像監獄一樣.
但他仍然倚靠神.
他知道神會帶領他離開黑洞.
他知道神一定會遵守他的諾言.
將他放上以色列人的王位.
將所應許給他的國賜給他.
大衛說你知道我的道路.
這與約伯記的23章8至10節的經文很相似.
約伯說.
指示我往前行.
他不在那裡.
往後退也不能看見他.
他在左邊行事我卻不看見.
在右邊隱藏我也不能見他.
然而他知道我所行的路.
他思念我之後.
我必如精金.
約伯說我見不到神.
左右前後我都見不到神.
但他信靠神.
他知道神與他同在.
鍛鍊了之後.
他就會成為精金.
他會成為一個更加成熟的人.
所以我們見到.
當我們有憂鬱.
第一向神呼求.
可以大聲向神呼求.
但不要停留在那裡.
要承認自己的軟弱.
第三要依靠神.
但我們要知道.
神不是立刻來打救你的.

$^{881}$就算大衛.
神是容許掃羅繼續追殺他.
有一段日子才讓他走出來.
所以當我們有憂鬱症的時候.
很多人希望快點康復.
神快點救贖我.
但神有他的時間性.
所以我們要耐心.
當你看見不是這麼快康復.
你不要害怕.
神有他的旨意.
除了我們哀求神.
承認軟弱.
信靠主.
他知道我們的道路.
我們還要將信心放在神的身上.
142篇第七節.
大衛說.
求你領我出來避困之地.
我好稱讚你的名.
義人必圍繞我.
因為你是容口恩待我.
有沒有看見大衛向神說什麼.
是不是說你快點救我.
將我擺上王位.
不是.
他說求你使義人圍繞我.
他是不是求神快點使他脫離他的災難.
不是.
他說讓我稱讚你的名.
有沒有看見.
這個是信心.
我們來說.
在危難中.
哈神啊你救我脫離這個災難.
你救我脫離這個危難.
是不是.
但不是大衛.
你讓義人圍繞我.
你讓我有機會稱讚你.

$^{921}$這個就是大衛.
我們在這裡能夠看見他的信心.
這個也帶領我們去到詩篇的57篇.
大衛的禱告.
第五點就是我們要向神禱告.
在詩篇的57篇第一節.
大衛說神啊你憐憫我.
然後他再重複憐憫我.
因為我的心投靠你.
我要投靠在你翅膀的任下.
等到災害過去.
哇 大衛在山洞裡面.
但這個山洞變成了聖殿.
甚至變成了神的致聖所.
我們知道致聖所裡面有什麼.
有藥櫃.
藥櫃的蓋是什麼.
是神的私人寶藏.
上方有基督派的天使.
那對翼是遮蓋著藥櫃.
大衛說我好像在致聖所裡面.
你的天使的翅膀就是遮蓋著我一樣.
雖然他在黑洞裡面.
他向神禱告.
他覺得自己就在聖所裡面.
向神祈求.
最後一點就是讚美神.
所以我們如果有憂鬱的時候.
我們要由第一點向神哀求.
一直到讚美神.
我們才能夠一步一步走出來.
你記不記得在詩篇的142篇.
大衛已經說.
主啊 願你救贖我.
我好稱讚你的名.
神救贖了大衛沒有.
還沒有救贖.
但是大衛已經稱讚神.
在詩篇的57篇第二節.
大衛說我要求告至高的神.

$^{961}$就是為我成全諸事的神.
在第七節他說.
神啊 我心堅定 我心堅定.
我要唱詩 我要歌頌.
有沒有看到我要歌頌.
在第八節他說.
我的靈啊 你當醒起琴室啊.
你們當醒起 我自己要極早醒起.
醒起做什麼啊.
這麼早起床做什麼啊.
讚美神.
第九節主啊.
我要在萬民中稱謝你.
在列邦中歌頌你.
有沒有看到啊 稱謝歌頌.
在第十節呢.
因為你的慈愛高及諸天.
你的誠實達到窮倉.
大衛一直由哀求到讚美.
這樣才能夠走出憂鬱黑暗.
最後大衛就是將榮耀歸給神.
當他讚美神的時候.
他就將榮耀歸給神.
我剛才說過憂鬱症的問題是什麼.
就是將注意力放在自己身上.
所以當我們將注意力放在神的身上.
榮耀祂的時候.
我們就能夠走出憂鬱症.
很多時候當我們憂鬱的時候.
你會發覺你說的話.
全部都是圍繞你自己.
一個人憂鬱的時候他會說什麼啊.
你看看我現在多麼的瘦.
我以前不是這樣的.
你看看我現在不比從前了.
現在我很差了.
或者你看看我有多慘.
當你憂鬱的時候.
注意力全部放在你自己的身上.
所以你要榮耀神將榮耀歸給他.

$^{1001}$這個方法不是我紙上談兵.
我自己試過.
我自己曾經有憂鬱症.
就是靠著這個辦法.
一步一步走出來.
所以我知道是有效.
我不單止自己走出來.
我還曾經帶領我媽媽.
她也有憂鬱症.
都是一步一步照著這個方法走出來.
當你有憂鬱症的時候.
你不是去看心理學醫生.
沒有用的.
心理學是照人的方法來醫治你.
是你的身體出現問題.
你是去看醫生.
但當你的心靈出現問題.
你不是去看醫生.
你不是去看心理學醫生.
你是要用神的話語.
來幫助你一步一步走出來.
當大衛讚美神的時候.
他將榮耀歸給神.
在詩篇的57篇第5節和第11節.
兩節的經文一模一樣.
大衛怎麼說.
神啊 願你崇高過於主天.
願你的榮耀高過全地.
兩次同樣的經文出現.
有沒有看到.
願你的榮耀高過全地.
這就是大衛將榮耀歸給神.
我說我這一篇是用三個地方來分段.
我們說了亞渡蘭洞 摩亞.
第三個地方是哈列的樹林.
大衛曾經將他的跟隨者帶到一個山寨.
叫做堡壘的地方.
這個地方很多學者認為.
是在亞渡蘭洞東南大概35里的地方.
在死海旁邊叫做馬薩達.

$^{1041}$馬薩達翻出來就是山寨堡壘.
為什麼會有這樣的名稱.
因為天然形成像堡壘一樣.
人可以躲在那裡逃避敵人的追擊.
這個我相信是詩篇第18篇第2節.
大衛不斷提到山寨堡壘.
他叫神做山寨堡壘.
但有很多學者認為.
他可能就在馬薩達這個地方.
在這個時候.
在撒母耳上的22章第5節.
先知迦德跟他說.
你不要住在山寨 要往猶大地去.
大衛就離開那裡進入哈列的樹林.
因為迦德先知跟他說.
這裡不安全 是礦野.
你不如走回猶大地比較安全.
所以大衛走回亞道蘭洞附近.
有個地方叫做哈列的森林.
哈列的森林這個字翻譯出來.
就是灌木叢 就是矮樹林.
先知迦德會再次出現在大衛的生命當中.
但當他再出現的時候.
大衛就有很大的問題.
因為他是數點士兵 不被神所喜悅.
迦德先知就來告訴他.
在撒母耳下的24章.
而迦德也曾經幫助大衛在聖所建立音樂的事工.
在歷代治下的29章第25節.
他也寫了一本關於大衛統治的書.
在歷代治上的29章第29節.
迦德這個人是很重要.
我講撒母耳記介紹的時候.
我也曾經說過.
迦德可能是撒母耳記其中的一個作者.
大衛不單止有迦德先知幫助他自己.
他也有祭司亞比亞他.
亞比亞他是誰呢?.
亞比亞他是亞希米納的兒子.
記不記得我說過在樹懸那柏這個地方.

$^{1081}$整個祭司城男女老幼都被殺.
我那時候不是和大家說過.
有一個人逃脫了.
這個就是亞比亞他.
他是亞希米納的兒子.
所以現在不單止大衛.
有600個勇士來幫助他.
還有先知迦德和亞比亞他.
而大衛也和亞比亞他說.
很對不起是我令到你全家被殺.
大衛也願意來保護亞比亞他.
所以你會看到亞比亞他一直跟隨大衛.
直到大衛死了之後.
因為亞比亞他幫助大衛第四個兒子亞多尼亞成為王.
和所羅門來對敵.
後來被所羅門將亞多尼亞殺死.
然後就將亞比亞他來夾擊.
所羅門沒有殺亞比亞他.
因為他知道大衛曾經許下諾言.
但這也完成了神對以利加的預言.
以利的後人不會再成為大祭司.
亞比亞他就是最後一個成為大祭司的以利後人.
這件事很重要.
在大衛身邊不單止有一個代表神向人發言的先知.
這就是迦德.
也有一位代表人向神獻祭的祭司.
就是亞比亞他.
在結束的時候我們不妨來看看.
今天我們看到神帶領大衛走出亞都蘭洞.
神也能帶領我們走出我們的憂鬱黑暗.
就好像從洞裡走出來一樣.
但我們要怎麼做呢.
我剛才說過有六點.
第一點就是向神傾心吐意.
說出你的苦情.
向神哀求.
但千萬不要停在那裡.
很多人停在那裡永遠向神哀求.
繼續要做什麼呢.
第一點就是要承認自己的軟弱.

$^{1121}$你才會依靠神.
第三點.
然後你要信任他.
然後向神禱告.
最後就是讚美他.
將榮耀歸給他.
才可以一步一步走出來.
第二點.
就算我們這樣做.
神可能不會立刻幫到我們.
因為我們要知道.
神在我們每個人身上都有功課.
我們每個人都有學習的餘地.
神會鍛鍊我們磨練我們.
以致我們能夠成為一個合他心意的人.
還有第三點.
我們看到大衛不是一個沒有罪的人.
雖然他合神心意.
他犯很多罪.
他妥協.
他說謊.
遲些還會犯姦淫犯謀殺.
但我們看到神仍然可以用他.
神仍然寬恕他.
所以我們看到給我們很大的鼓勵.
很多人認為我犯了很大的罪.
神一定不會饒恕我寬恕我.
你看看大衛就算殺了人.
神都能夠寬恕他.
不是說沒有後果.
是有後果.
對他的罪來說是有後果.
但神仍然寬恕了他.
所以任何人只要我們願意接受神的話.
不論你犯了什麼罪.
神都會赦免你.
唯一不能夠赦免的罪是什麼.
聖靈感動你去接受的時候.
你不接受.
願神祝福他自己的話語.

$^{1161}$讓我們低頭禱告.
詩篇的一百四十二和五十七篇.
願意我們能夠學習大衛.
當他在曠野的時候.
他是經過很大的苦難.
我們人生當中.
都是有苦難都是有憂鬱的時候.
願意我們能夠將注意力放在你的身上.
而不是放在我們自己的身上.
學習大衛一步一步這樣走出來.
至於我們當中.
如果是有未接受你的.
願意他們都能夠接受你.
因為世上沒有罪大到你不能夠赦免的.
只要我們來到你的面前認罪悔改.
願意接受你成為我們的救主.
你是會得救贖我們.
願意我們都能夠將榮耀頌讚歸給你.
我們這樣禱告.
奉主耶穌基督的名義求.
阿們.
\newpage



\section{撒母耳記上 23:1-24:22}
\label{sec:GqTOPwqfjwM}
\textbf{不可伸手害神的受膏者! (撒母耳記上23\_1-24\_22) - 袁惠鈞牧師[大衛傳系列 - 第6講]}
\newline
\newline
連結: \href{https://youtube.com/watch?v=GqTOPwqfjwM}{\texttt{ https://youtube.com/watch?v=GqTOPwqfjwM}} ~~~~ 語音日期: 2025-02-19 
\newline
\newline
\hyperref[sec:WCt7vYrgwVY]{< < < PREV SERMON < < <}
~
\hyperlink{toc}{[返主目錄]}
~
\hyperref[ch:preacher6]{[返講員目錄]}
~
\hyperref[sec:9ORA5941xxk]{> > > NEXT SERMON > > >}
\newline
\newline
撒母耳記上 23:1-24:22
\newline
\begin{longtable}{cl}
\hline
\hline
章節 & 經文 (和合本修訂版)\\
\hline
23:1 & \begin{tabularx}{0.7\textwidth}{X} 有人告訴大衛說:「看哪,非利士人攻擊基伊拉,搶奪禾場。」 \end{tabularx} \\ \\ \relax
23:2 & \begin{tabularx}{0.7\textwidth}{X} 大衛求問耶和華說:「我可以去嗎?我可以去攻打那些非利士人嗎?」耶和華對大衛說:「你可以去攻打非利士人,拯救基伊拉。」 \end{tabularx} \\ \\ \relax
23:3 & \begin{tabularx}{0.7\textwidth}{X} 大衛的人對他說:「看哪,我們在猶大這裡尚且懼怕,何況到基伊拉去攻打非利士人的軍隊呢?」 \end{tabularx} \\ \\ \relax
23:4 & \begin{tabularx}{0.7\textwidth}{X} 大衛又再求問耶和華,耶和華回答說:「你起身下基伊拉去,我必將非利士人交在你手裡。」 \end{tabularx} \\ \\ \relax
23:5 & \begin{tabularx}{0.7\textwidth}{X} 於是大衛和他的人往基伊拉去,與非利士人打仗,大大擊敗他們,奪取他們的牲畜。這樣,大衛救了基伊拉的居民。 \end{tabularx} \\ \\ \relax
23:6 & \begin{tabularx}{0.7\textwidth}{X} 亞希米勒的兒子亞比亞他逃往基伊拉到大衛那裡的時候,手裡拿著以弗得。 \end{tabularx} \\ \\ \relax
23:7 & \begin{tabularx}{0.7\textwidth}{X} 有人告訴掃羅,大衛到了基伊拉。掃羅說:「神將他交在我手裡了,因為他進了有門有閂的城,把自己關起來了。」 \end{tabularx} \\ \\ \relax
23:8 & \begin{tabularx}{0.7\textwidth}{X} 於是掃羅召集眾百姓,要下去攻打基伊拉,圍困大衛和他的人。 \end{tabularx} \\ \\ \relax
23:9 & \begin{tabularx}{0.7\textwidth}{X} 大衛知道掃羅設計陷害他,就對亞比亞他祭司說:「把以弗得拿過來。」 \end{tabularx} \\ \\ \relax
23:10 & \begin{tabularx}{0.7\textwidth}{X} 大衛說:「耶和華-以色列的神啊,你僕人確實聽見掃羅設法要到基伊拉來,為我的緣故毀滅這城。 \end{tabularx} \\ \\ \relax
23:11 & \begin{tabularx}{0.7\textwidth}{X} 基伊拉人會把我交在掃羅手裡嗎?掃羅會下來,正如你僕人所聽見的嗎?耶和華-以色列的神啊,求你指示僕人!」耶和華說:「他會下來。」 \end{tabularx} \\ \\ \relax
23:12 & \begin{tabularx}{0.7\textwidth}{X} 大衛又說:「基伊拉人會把我和我的人交在掃羅手裡嗎?」耶和華說:「他們會交出來。」 \end{tabularx} \\ \\ \relax
23:13 & \begin{tabularx}{0.7\textwidth}{X} 於是大衛和他的人約有六百名起身離開基伊拉,往他們所能去的地方去。有人告訴掃羅,大衛離開基伊拉逃走了,掃羅就停止出發了。 \end{tabularx} \\ \\ \relax
23:14 & \begin{tabularx}{0.7\textwidth}{X} 大衛住在曠野的山寨裡,在西弗曠野的山區。掃羅天天尋索大衛,神卻不將大衛交在他手裡。 \end{tabularx} \\ \\ \relax
23:15 & \begin{tabularx}{0.7\textwidth}{X} 大衛看到掃羅出來尋索他的命。那時,他住在西弗曠野的樹林裡; \end{tabularx} \\ \\ \relax
23:16 & \begin{tabularx}{0.7\textwidth}{X} 掃羅的兒子約拿單起身,到樹林裡去見大衛,使他的手倚靠神得以堅固, \end{tabularx} \\ \\ \relax
23:17 & \begin{tabularx}{0.7\textwidth}{X} 對他說:「不要懼怕!我父掃羅的手無法害你 ,你必作以色列的王,我必作你的宰相。我父掃羅也知道這事。」 \end{tabularx} \\ \\ \relax
23:18 & \begin{tabularx}{0.7\textwidth}{X} 於是二人在耶和華面前立約。大衛仍住在樹林裡,約拿單就回家去了。 \end{tabularx} \\ \\ \relax
23:19 & \begin{tabularx}{0.7\textwidth}{X} 西弗人上基比亞到掃羅那裡,說:「大衛不是在我們那裡,在樹林裡的山寨中,在荒野南邊的哈基拉山藏著嗎? \end{tabularx} \\ \\ \relax
23:20 & \begin{tabularx}{0.7\textwidth}{X} 現在,王啊,請隨你的心願要下來,就請下來;至於我們,一定會把他交在王的手裡。」 \end{tabularx} \\ \\ \relax
23:21 & \begin{tabularx}{0.7\textwidth}{X} 掃羅說:「願耶和華賜福給你們,因為你們體恤我。 \end{tabularx} \\ \\ \relax
23:22 & \begin{tabularx}{0.7\textwidth}{X} 請你們回去,再確定一下,調查並看清楚他落腳的地方,是誰看見他在那裡,因為有人告訴我他很狡猾。 \end{tabularx} \\ \\ \relax
23:23 & \begin{tabularx}{0.7\textwidth}{X} 你們要看清楚,調查他藏匿的每一個地方,回來給我確實的報告,我就與你們同去。他若在境內,我必從猶大的千門萬戶中搜出他來。」 \end{tabularx} \\ \\ \relax
23:24 & \begin{tabularx}{0.7\textwidth}{X} 西弗人動身,在掃羅以先往西弗去。大衛和他的人卻在瑪雲曠野,在荒野南邊的亞拉巴。 \end{tabularx} \\ \\ \relax
23:25 & \begin{tabularx}{0.7\textwidth}{X} 掃羅和他的人去尋索大衛。有人告訴大衛,他就下到巖石那裡,留在瑪雲的曠野。掃羅聽見了,就在瑪雲的曠野追趕大衛。 \end{tabularx} \\ \\ \relax
23:26 & \begin{tabularx}{0.7\textwidth}{X} 掃羅在山的這一邊走,大衛和他的人在山的那一邊。大衛急忙躲避掃羅,掃羅和他的人正四面圍住大衛和他的人,要捉拿他們。 \end{tabularx} \\ \\ \relax
23:27 & \begin{tabularx}{0.7\textwidth}{X} 有使者來對掃羅說:「非利士人入境搶掠,請快快回去!」 \end{tabularx} \\ \\ \relax
23:28 & \begin{tabularx}{0.7\textwidth}{X} 於是掃羅不再追趕大衛,回去迎擊非利士人。因此那地方名叫西拉‧哈瑪希羅結。 \end{tabularx} \\ \\ \relax
23:29 & \begin{tabularx}{0.7\textwidth}{X} 大衛從那裡上去,住在隱‧基底的山寨裡。 \end{tabularx} \\ \\ \relax
24:1 & \begin{tabularx}{0.7\textwidth}{X} 掃羅追趕非利士人回來,有人告訴他說:「看哪,大衛在隱‧基底的曠野。」 \end{tabularx} \\ \\ \relax
24:2 & \begin{tabularx}{0.7\textwidth}{X} 掃羅就從全以色列中挑選三千精兵,往野山羊磐石的東邊去,尋索大衛和他的人。 \end{tabularx} \\ \\ \relax
24:3 & \begin{tabularx}{0.7\textwidth}{X} 到了路旁的羊圈,在那裡有個洞,掃羅進去大解。大衛和他的人正藏在洞裡的深處。 \end{tabularx} \\ \\ \relax
24:4 & \begin{tabularx}{0.7\textwidth}{X} 大衛的人對大衛說:「看哪,這日子到了!耶和華曾對你說:『看哪,我要將你的仇敵交在你手裡,你可以照你看為好的對待他。』」大衛就起來,悄悄地割下掃羅外袍的衣角。 \end{tabularx} \\ \\ \relax
24:5 & \begin{tabularx}{0.7\textwidth}{X} 隨後大衛心中自責,因為他割下了掃羅的衣角。 \end{tabularx} \\ \\ \relax
24:6 & \begin{tabularx}{0.7\textwidth}{X} 他對他的人說:「耶和華絕不允許我對我的主,耶和華的受膏者做這事,伸手害他,因為他是耶和華的受膏者。」 \end{tabularx} \\ \\ \relax
24:7 & \begin{tabularx}{0.7\textwidth}{X} 大衛用這話勸阻他的人,不許他們起來害掃羅。掃羅起來,從洞裡出去,預備上路。 \end{tabularx} \\ \\ \relax
24:8 & \begin{tabularx}{0.7\textwidth}{X} 然後大衛也起來,從洞裡出去,呼喚掃羅說:「我主,我王!」掃羅回頭觀看,大衛就屈身,臉伏於地下拜。 \end{tabularx} \\ \\ \relax
24:9 & \begin{tabularx}{0.7\textwidth}{X} 大衛對掃羅說:「你為何聽信人的讒言,說『看哪,大衛想要害你』呢? \end{tabularx} \\ \\ \relax
24:10 & \begin{tabularx}{0.7\textwidth}{X} 看哪,今日你親眼看見,在洞中耶和華將你交在我手裡。有人要我殺你,我卻愛惜你,說:『我不敢伸手害我的主,因為他是耶和華的受膏者。』 \end{tabularx} \\ \\ \relax
24:11 & \begin{tabularx}{0.7\textwidth}{X} 我父啊,請看,看你外袍的衣角在我手中。我割下你外袍的衣角,卻沒有殺你。你知道,並且看見我沒有惡意要悖逆你。你雖然要獵取我的命,我卻沒有得罪你。 \end{tabularx} \\ \\ \relax
24:12 & \begin{tabularx}{0.7\textwidth}{X} 願耶和華在你我中間判斷,願耶和華在你身上為我伸冤,我卻不親手加害於你。 \end{tabularx} \\ \\ \relax
24:13 & \begin{tabularx}{0.7\textwidth}{X} 古人有句俗語說:『惡事出於惡人。』我卻不親手加害於你。 \end{tabularx} \\ \\ \relax
24:14 & \begin{tabularx}{0.7\textwidth}{X} 以色列王出來要尋找誰呢?你要追趕誰呢?不過是一條死狗,一隻跳蚤而已。 \end{tabularx} \\ \\ \relax
24:15 & \begin{tabularx}{0.7\textwidth}{X} 願耶和華作仲裁者,在你我中間判斷。願他鑒察,為我伸冤,救我脫離你的手。」 \end{tabularx} \\ \\ \relax
24:16 & \begin{tabularx}{0.7\textwidth}{X} 大衛向掃羅說完了這些話,掃羅說:「我兒大衛,這是你的聲音嗎?」於是掃羅放聲大哭, \end{tabularx} \\ \\ \relax
24:17 & \begin{tabularx}{0.7\textwidth}{X} 對大衛說:「你比我公義,因為你以善待我,我卻以惡待你。 \end{tabularx} \\ \\ \relax
24:18 & \begin{tabularx}{0.7\textwidth}{X} 今日你已顯明是以善待我,因為耶和華將我交在你手裡,你卻沒有殺我。 \end{tabularx} \\ \\ \relax
24:19 & \begin{tabularx}{0.7\textwidth}{X} 人若遇見仇敵,豈肯放他平安上路呢?願耶和華因你今日向我所做的,以善回報你。 \end{tabularx} \\ \\ \relax
24:20 & \begin{tabularx}{0.7\textwidth}{X} 現在,看哪,我知道你一定會作王,以色列的國必要堅立在你手裡。 \end{tabularx} \\ \\ \relax
24:21 & \begin{tabularx}{0.7\textwidth}{X} 現在你要指著耶和華向我起誓,你必不剪除我的後裔,必不從我父家除去我的名。」 \end{tabularx} \\ \\ \relax
24:22 & \begin{tabularx}{0.7\textwidth}{X} 於是大衛向掃羅起誓,掃羅就回家去,大衛和他的人也上山寨去了。 \end{tabularx} \\ \\
[1ex]
\hline
\hline
\end{longtable}
$^{1}$大胃傳第六講.
今日要講十五以上的23章至24章.
繼續探討如何成為合神心意的人.
今日的題目是甚麼.
不可伸手害神的受高者.
我是信徒來的.
我的受高者只有一位就是耶穌基督.
但在我們的生命當中.
在世上神是否放了一些上司領袖.
一些老闆在你的上層呢.
有時這些老闆上司很苛刻.
甚至很無理.
真的像蘇羅追殺大胃.
你心想這個老闆真是想要我的命.
我們應該如何回應呢.
我們應該如何做呢.
今日這篇道有給我們一個很好的榜樣.
一個教導.
這個背景我們見到蘇羅不斷追殺大胃.
大胃是陷於很悲慘很艱難的一段日子.
很多時我們衡量一個人成功與否是如何衡量.
看他做那件事是有多成功.
但其實我們很應該衡量一個人.
在他未成功之前如何克服困難.
如何克服障礙.
這是一個比較好衡量人的方法.
如果我們用這個方法衡量大胃.
我們會發現大胃是一個非常成功的人.
因為他未成功之前要克服很多困難.
很多困境.
十年來大胃不斷被蘇羅追殺.
他一方面要與蘇羅對抗.
另一方面他仍然不斷打救以色列人.
從敵人手中將以色列人打救出來.
不是容易的.
他與他的跟隨者到處在曠野沙漠動月漂流.
最後神才救贖他成為以色列人的王.
將他擺上寶座.
大胃成為以色列人的王.
不單對以色列人很重要.

$^{41}$對我們也很重要.
因為他也是我們的王.
將來他會回到世上掌權.
成為世上萬國的王.
所以這件事對我們也很重要.
耶穌基督就是大胃的子孫.
就是神的兒子.
今天我這篇道和上次的道.
唯一相似的地方是什麼?.
上次我是用三個地方來分段.
今天我是用四個不同的地方來分段.
就是大胃被掃羅追殺.
由一個地方去到另一個地方.
然後去到第四個地方.
神給他一個機會來平反.
給他一個機會來表白他自己.
第一個地方是什麼?.
是基伊拉.
基伊拉是以色列人邊境的一個城市.
是很接近斐理士人的地方.
距離斐理士人的加特城只有十二里.
這個地方也在哈列森林的西邊大概十里.
你說哈列森林好像似曾相識.
你真的記得了.
我上星期和大家說過.
記不記得在撒母耳上的二十二章第五節.
加特先知和大胃說.
你不要躲在山寨堡壘.
你要去哈列森林猶太人的地方.
你要回到猶太人的地方.
所以大胃就在哈列森林來扎營.
距離基伊拉大概十里.
基伊拉這個地方很危險.
因為太近斐理士人.
在說史詩記的時候我們也說過.
每逢有收穫的季節.
敵人就會來攻打以色列人.
為什麼?.
可以搶他們的糧食.
所以斐理士人蠢蠢欲動.

$^{81}$會來基伊拉攻打他們.
蘇羅若是關心他們的子民.
應該派士兵來保護基伊拉.
但他沒有這樣做.
他唯一希望得到的是什麼?.
殺死大胃是最重要.
反而敵對敵人他完全不關心.
這和大胃有很大分別.
大胃和蘇羅在這片土地有很多探子.
蘇羅有他的探子.
大胃也有.
大胃的探子回來回報.
斐理士人要來攻擊基伊拉.
於是大胃立刻停下來尋問神的旨意.
這次不是第一次.
在薩姆以上的23章.
是三次求問神的旨意.
遲些會在25-26章繼續尋求神的旨意.
這就是大胃.
沒有問題首先求問神的旨意.
這和我們的做法很不同.
很多時我們會尋求自己的利益.
有利益我就做.
沒有利益我就不做.
很少尋求神的旨意.
在這樣的情況下我們尋求利益.
我們會怎樣說呢.
喂 我們已經被蘇羅追殺了.
還要我去打基伊拉.
開玩笑嗎.
用什麼腹背受敵.
你不要搞我.
我們就會這樣做.
另外有些人會怎樣.
有些事情發生就很害怕.
驚惶失措.
然後定下了心理準備.
讓我來籌算一下.
計劃一下應該怎樣做.
然後才禱告.

$^{121}$神啊祝福我的計劃.
此末倒置.
你將禱告放在最後.
其實一有問題發生.
你就應該立刻來禱告.
而不是像我們很多人一樣.
大胃有很多藉口.
當他被蘇羅追殺.
他可以說我自身難保.
我泥菩薩過江.
但他沒有這樣做.
他首先尋求神的旨意.
禱告是大胃的生活習慣.
當蘇羅最後死了.
神幫助他成為猶大的王的時候.
他首先做什麼.
求神的旨意.
在殺墓以下的第二章第一節.
然後經過七年半.
神將他擺上全以色列人的王位.
他又首先做什麼.
又是求神的旨意.
在殺墓以下的五章十九和二十三節.
這就是原因.
為什麼大胃成為一個合神心意的人.
你想想.
如果你每做一件事.
你都首先求神的旨意.
你是否會成為一個合神心意的人.
這就是大胃的秘訣.
有人可能會覺得很奇怪.
如果大胃都是求神的旨意.
當亞比亞他這個祭司.
跟隨大胃之前.
大胃是怎樣尋求神的旨意呢.
可能有人會問.
亞比亞他有以弗德.
裡面有屋靈和土名可以尋求神的旨意.
但當亞比亞他還未跟隨大胃.
他怎樣做呢.

$^{161}$相信加德已經加入了大胃的隊伍.
所以我相信在亞比亞他之前.
大胃就是求問先知加德.
因為先知是神的發言人.
所以我們看到大胃在有重要的時刻.
很多時候都會求問神.
就好像撒慕爾上的23章.
25章和26章.
都是這樣告訴我們.
神就叫大胃去幫助基拉的人.
來敵對斐理士人.
大胃的手下很不滿意.
因為他們覺得平時去打斐理士人.
無話可說.
但現在被掃羅追殺.
還要去救基拉的人.
用什麼腹背受敵.
所以他們不肯.
大胃再一次尋求神.
第二次尋求神.
神跟他說一樣的話.
你不用怕 我會將斐理士人交在你的手裡.
我們看到即使大胃逃避掃羅.
他仍然為他的子民著想.
這個和掃羅有很大的分別.
其實如果他去打救基拉的話.
對他是有好處的.
為什麼.
因為基拉是有城牆包圍的.
這個比大胃在礦野飄流好得多.
如果他幫助基拉進入城.
現在有城牆可以保護他.
所以對他來說是一個好處.
很多時候我們做事會怎樣.
看到對我們有利益有好處的.
我們就會說.
哎呀 你看.
神打開道門給我了.
立刻就反應.
立刻就湧到去.

$^{201}$想都不用想.
大胃怎麼做.
尋求神的旨意.
尋問神.
這個是不是真的你的旨意.
這很重要.
於是神就遵守了他的諾言.
幫他打敗了斐理士人.
還幫他得到很多很多的戰利品.
於是大胃就搬到基拉.
有城牆來保護他.
當年阿比亞他逃離那帕的時候.
也是躲在這個城裡.
但是掃羅的探子非常靈通.
知道大胃去了基拉.
法兵來攻打基拉.
而且還威脅.
如果基拉的人不交出大胃的話.
他就會滅城.
你猜他是不是真的有心滅城.
我相信他真的是.
因為曾經以前我們看到他.
屠殺了那帕整個城.
就是為了要殺死大胃.
所以他絕對有可能是這樣做.
他被仇恨控制了.
他失去了他的利子.
而且濫用他的權威.
這就帶領我們去到第二個地方.
第二個地方就是西伐的曠野.
我們看到掃羅已經在自己的王宮當中.
三次用長矛想扔死大胃.
又派了三隊軍馬來追殺大胃.
現在大胃帶著600人.
本來以為來到基拉可以有安身之地.
但掃羅就來攻打基拉.
於是大胃又怎樣做.
又再次尋問神的旨意.
又再去叫亞比亞他來尋求神.
在撒姆以上的23章第10節.

$^{241}$大胃禱告說.
耶護說以色列的神.
你的僕人聽真了.
掃羅要往基拉來為我的緣故滅城.
有沒有看到大胃很信服神.
叫自己做僕人.
這就是大胃的作風.
在詩篇的78篇和89篇.
他都叫自己做僕人.
他非常謙卑.
而且我們看到他很擔心基拉的人.
所以他尋問神的旨意.
他很關心他的子民.
本來大胃救了基拉的人.
以為這群人會幫他.
他是救命恩人.
基拉的人是不是應該知恩圖報.
但最後基拉的人願意交出大胃.
所以神警告大胃.
你快離開這個城.
他們已經準備將你交給掃羅.
基拉的人都很怕掃羅.
他們都聽過有拿柏的事情.
所以他們願意和掃羅合作.
而掃羅心狠手辣.
言出必行.
所以大胃就帶著他600個勇士.
離開了基拉去到西伐的曠野.
這就是第二個地方.
西伐的曠野距離基拉大約15里.
在希伯倫東南的四里的一個城市.
當掃羅知道大胃離開了基拉.
他就立即撤兵.
你看到他最重要的就是想殺死大胃.
其他的事情他都不理.
他放下了國家很重要的事情.
總之就是要追殺大胃.
這就是他唯一的目的.
他很怕大胃威脅他的王位.
他非常非常擔心.

$^{281}$其實西伐的曠野是屬於猶大曠野的一部分.
在死海旁邊是一個非常貧乏的地方.
在這裡大胃有很大的考驗.
因為你心想去到一個這麼貧乏的地方.
實在不是容易的事.
如果你現在去聖地的話.
去到西伐的地方你會很驚訝.
為什麼大胃居然能夠在這個地方生存.
什麼都沒有的.
只有山溝只有洞穴.
實在不是一件容易的事.
猶太人的粉瑞黨曾經兩次對抗羅馬政府.
都是躲在西伐的曠野.
第一次是公元66年至73年.
那次是羅馬政府軍臨城下.
戰勝了以色列人.
攻陷了耶路撒冷.
然後連聖殿都毀滅了.
在這個時候粉瑞黨就躲在西伐的曠野.
還有一次是公元132年至135年.
也是躲在西伐的曠野.
約拿丹就是塑羅的兒子.
他來到西伐的曠野和大胃見面.
約拿丹真的很愛大胃.
他來安慰他 堅顧他 叫他不要害怕.
再一次和他重新索納的約.
就是在《撒母耳上》的18章20章所納的約.
他將會接受大胃成為他的王.
但他希望大胃不會消滅他的後裔.
他向大胃保證.
神一定會幫助你脫離塑羅.
會使你登上以色列人的王位.
有沒有看到約拿丹真的很愛大胃.
在《箴言》的18章24節說什麼.
朋友比兄弟更親近.
是誰寫箴言呢.
是所羅門王寫箴言.
我相信他是透過大胃的口.
知道大胃和約拿丹的關係.
所以他寫了這句箴言出來.

$^{321}$18章24節 朋友比兄弟更親近.
這次是約拿丹和大胃最後一次的見面.
再下一次提到約拿丹.
在《撒母耳上》的31章第二節.
約拿丹是戰死在沙場.
所以這次是最後一次的見面.
這樣就帶領我們去到第三個地方.
就是馬雲的曠野.
西弗人和大胃本來是同一個支派.
西弗人都是猶大人.
他們居然怕死.
本來應該是同聲同氣自己人.
但他們居然怕死.
願意把大胃交出來.
蘇羅根本沒有辦法在西弗曠野找到大胃.
如果你去西弗曠野.
你會發現全部都是岩石和山洞.
到處都是一模一樣.
你進去就像迷宮一樣.
所以蘇羅一定要知道具體大胃躲在哪裡.
而西弗人知道.
也願意供出大胃.
大胃也有探子.
探子回報說.
蘇羅已經呼召了西弗人.
要供出大胃所在之地.
所以大胃再一次逃亡.
這次是去到西弗南邊三里.
有個叫馬雲的曠野.
他以為去到那裡就沒問題了.
為什麼呢.
馬雲的曠野是大胃以前放羊的地方.
我告訴過你們.
大胃是一個牧羊人.
他以前就在馬雲的曠野放羊.
對於馬雲的曠野的地理.
鳥語子長他全部都認識.
所以他認為在那裡應該沒問題.
就在這個時候.
蘇羅也帶著軍隊進入馬雲的曠野.

$^{361}$就在一座非常出名的岩石山那裡.
很快就會和大胃的士兵相遇.
蘇羅很聰明.
他是兩面包抄.
一面軍隊向山的一邊經過.
另一部分軍隊在另一邊.
兩面包抄.
看起來大胃一定逃不掉.
這次一定會失敗.
就在這個時候.
神給我們看到神仍然是掌權.
他激動了斐理士人來攻擊猶大.
所以蘇羅只可以收兵.
那你說喂.
又是你說的.
蘇羅唯一的目的就是要打敗大胃.
就算那些斐理士人來攻擊.
為什麼他要收兵.
你想想.
如果他一邊和大胃來征戰.
另一邊斐理士人來攻擊.
他會否腹背就敵.
他一定會輸.
所以他很聰明的立刻就收兵.
猶太人叫這個地方做什麼.
西拉哈瑪希羅傑.
這麼奇怪的名字.
什麼意思.
就是光滑的盤石.
滑溜溜的盤石.
為什麼叫這個地方做滑溜溜的盤石.
就好像我們廣東人說什麼.
滑過泥鰍.
滑溜溜的滑溜就被大胃逃脫了.
就是因為神救他.
大胃於是就從馬雲的曠野.
去到隱基底的山寨.
這個就是第四個地方.
隱基底在耶路撒冷南邊大概20里.
在死海的西面.

$^{401}$在希伯倫的東南.
在撒母耳上的二十四章第二節.
叫這個地方做野羊的盤石.
為什麼叫做野羊的盤石.
在那裡有很多野生的動物.
有很多的動物.
是一個非常好藏身的地方.
而且水源充足.
所以才有這麼多的動物.
於是大胃就去到隱基底.
你說你這邊道理奇怪了.
四個地方有什麼意思才行.
其實有意思的.
你看看大胃.
雖然他是有尋求神.
但他仍然是有靠他自己的智慧.
他心想第一.
如果我救了基伊拉.
成為他們的恩人.
他們一定會幫我.
結果基伊拉的人反而出賣他.
去到第二個地方.
去到西伐.
自己人.
一定會幫我.
結果自己人都出賣他.
然後他去到馬雲.
這個地方鳥語紙掌.
我熟悉這個地方.
結果掃羅來到兩面包抄.
又是不割吉.
然後去到隱基底這個地方.
他終於走在神的旨意裡面.
以致神給他有機會平反.
給他有機會表白.
我遲些就會說.
在詩篇的54篇.
大胃有個禱告.
他求神為他伸冤.
給他有機會向掃羅表達.

$^{441}$他不是一個忘恩負義.
他不是一個謀反的人.
而掃羅是大胃的王.
還是大胃的岳父.
岳父來的.
大胃從來不會叫掃羅做敵人.
無論掃羅怎樣差.
怎樣邪惡.
怎樣追殺他.
他從來不叫掃羅做他的敵人.
當掃羅的軍隊到達隱基底的時候.
神就給大胃一個機會平反.
來表白.
如果我們今天去聖地的時候.
我們會發現.
隱基底有很多很多的山洞.
大胃和他的手下.
就找了一個很大的洞.
躲在裡面.
而掃羅就走到這個洞裡面來大解.
為什麼這麼多洞不去.
走去那個洞.
為什麼他要離開自己的營地.
找地方來大解.
這是摩西的律法.
摩西的律法.
神的律法對大解是非常嚴格的.
因為是有關衛生的.
在新命記的23章.
12-14節講得很清楚.
逢士兵要大解的話.
他們必須離開營地.
不會影響到其他的軍人.
離開營地.
還要帶一個鏟子.
然後大解完之後.
要將你的排泄物賣掉.
這樣才合乎衛生.
那你說.
掃羅就算大解.

$^{481}$也應該帶一些跟隨者.
為什麼這麼危險.
一個人走到山洞裡面.
我相信掃羅也希望有些隱私.
你不是去派對.
你去大解.
還帶一群人跟你去.
自己一個人就去.
但是他想不到.
進了山洞裡面.
他希望有隱私.
結果有六百雙眼看著他大解.
這是他完全想不到的.
不是掃羅的探子無能.
是什麼呢.
是神完全掌權.
給大衛一個機會平反.
於是大衛就有很大的誘惑.
是不是真的應該殺死掃羅呢.
很難得的機會.
掃羅一個人摸進山洞.
大衛有六百人在.
一人向著掃羅.
吐一口水就淹死他.
天災難逢的好機會.
應不應該殺死他呢.
天時地利人和.
讓我們想也不用想.
如果他殺死掃羅的話.
有什麼好處.
不再需要漂流在曠野.
立刻能夠一步踏上他的王位.
能夠過平靜的生活.
從以色列人角度的眼光來看.
都是一件好事.
除去了一個腐敗敗壞的君王.
可以由一個合神心意的人來代替他.
從這個角度來看.
應該殺死他.
不管你怎麼看.

$^{521}$橫看橫看都是正確的做法.
大衛面對一個考驗.
有一個捷徑他可以登上王位.
這個和撒旦在馬太福音第四章.
引誘耶穌基督一模一樣.
撒旦和耶穌基督說.
有一個捷徑.
只要你跪在地上敬拜我.
我立刻給你這個世界.
你立刻可以做這個世界的王.
你為什麼要釘十字架.
這麼痛苦.
死了復活.
升天.
然後過很長的日子才回來世上做王.
為什麼你要這麼笨.
我現在給你一個機會.
向我來敬拜.
你立刻就可以成為王.
喂 撒旦有沒有這樣的權柄.
有 他是世界的王.
聖經也是這樣告訴我們.
這個就是捷徑.
耶穌基督有沒有走這個捷徑.
沒有走.
除了有很大的誘惑.
大衛的手下也有誘惑他.
千載難逢的好機會啊大衛.
六百個人都和大衛說.
蘇州過後就沒有艇搭了.
你要把握時機.
這個是大好時機.
他手下來引誘大衛.
和耶穌基督的門徒引誘耶穌基督.
都是一樣.
你不記得耶穌基督和他的門徒說.
我要釘十字架.
彼得怎麼說.
彼得立刻說.
主啊 萬不可如此.

$^{561}$這事必不能到你的身上.
馬太福音16:22.
耶穌基督怎麼說.
撒旦退到我身後.
為什麼耶穌基督叫彼得到撒旦.
因為彼得被撒旦利用來引誘耶穌基督.
耶穌基督也沒有被他的手下.
被門徒引誘到他.
大衛也是一樣.
大衛沒有被他的手下引誘到他.
為什麼.
大衛是服從神的話語.
詩篇的105篇的15節.
撒武以上的24章10節.
和26章的9節都有說.
不可伸手害神的受高者.
這就是我們今天的題目.
為什麼大衛的手下.
認為這是神的應許.
要大衛殺死訴羅.
不是說空穴來風.
他們是有根據的.
我相信有兩節經文.
第一句就是神和訴羅說的話.
已經將這個國交給別人.
撒武以上的第15章.
已經要交給一個合神心意的人.
這是第一句話.
還有是神對撒武以上的另一句話.
神說我已經厭棄訴羅作以色列的王.
我差遣你往伯尼行人耶西那裡去.
因為我在他眾子之內.
預定一個作王的撒武以上的第16章.
這就是大衛的手下所希望看到.
但是大衛不會這樣做.
他知道神有他自己的時間性.
有他自己的方法來裸奏訴羅.
不需要他自己下手.
大衛在山洞裡打了一場很漂亮的仗.
得到勝利.

$^{601}$這個勝利不是因為他殺死了訴羅.
而是他能夠平息600個人.
並不是他的手下要他殺死訴羅.
他能夠平息600個人的怒氣.
為什麼這件事這麼重要.
因為他將來要帶領這600個人.
如果他今天殺死了訴羅的話.
日後有什麼事他們不喜歡大衛.
他們就會怎樣.
大衛也殺死了訴羅.
我們也可以殺死大衛.
大衛就是讓他的手下看到.
你不要這樣做.
有沒有看到.
他是能夠服眾.
以至他以後能夠帶領這600個人.
他讓他們看到.
他是服從神的話語.
剛才我們說了.
不可伸手害神的受告者.
還有第二點.
他是等待神的時間.
等待神的旨意.
還有第三點.
當他的手下認為是一個好的機會來報仇.
大衛怎麼說.
這個應該是思念的時間.
這個應該是我們有一個正確心態的時候.
他就是使他的手下能夠服從他.
以至他日後能夠帶領這班人.
當他這樣做的時候.
神就應驗了他給大衛的應許.
給他一個機會伸張正義.
就好像詩篇54篇所說的.
給他一個機會平反.
大衛知道不應該殺人.
因為神的話語說不可殺人.
你說基督徒不可殺人.
但是在戰場上可以的.
如果正義之師在戰場上.

$^{641}$你是可以殺死對方的.
另外當有人來襲擊你.
你為了智慧殺人.
這個是可以的.
但是如果你去殺一個手無寸鐵.
正在大解的人.
這個怎麼看都是不對的一件事.
所以大衛提醒他的手下.
不要傷害神的受告者.
不單止不要殺他.
連詛咒領袖都是不對的.
根據猶太人的律法.
出埃及記22章第28節.
尚且神的律法說.
不可以詛咒你的領袖.
和褻瀆神的名同罪.
所以我們要很小心.
於是在這個時候.
大衛摸了去素羅的身後.
素羅剛剛脫了他的外衣在大解.
於是大衛就把素羅的衣襟切下來.
在撒母耳上的第25章第5節.
就是因為大衛做了這件事.
他非常非常後悔.
非常內疚.
你問為什麼.
因為聖經裡面.
衣襟是非常重要.
是一個人的身份.
是代表一個人的權柄.
你說經文出於哪裡.
首先開始這個傳統的.
在民數記的第15章38至39節.
神吩咐以色列人.
世世代代在他們的衣襟或水紙.
釘一根藍繫帶子.
他們佩戴這水紙.
就代表他們紀念.
遵行耶和華一切的命令.
是代表神的話語.

$^{681}$是權柄.
是身份的象徵.
水紙.
在古代的以色列人.
他們的水紙.
他們的衣襟.
都是有不同的花紋.
有時甚至他們會怎樣做.
在他們的水紙衣襟.
塗上一些物質.
然後在合同的上面打上印.
這個就代表他們的身份.
是有這樣的代表性.
在以色列的16章第8節.
神說我從你的旁邊經過.
看見你的時候正動愛情.
便用衣襟搭在你的身上.
遮蓋你的赤體.
又向你起誓.
與你結盟.
你就歸於我.
這是主耶和華說的.
有沒有看到.
一個象徵就是一個人.
將衣襟搭著另一個人.
就是說這個人是屬於我.
我願意今世都照顧你.
這個就是原因.
為什麼在路德記.
記不記得在波亞斯的和場.
路德和波亞斯說.
你願不願意將你的衣襟遮蓋我.
其實他是向波亞斯示愛.
當波亞斯將衣襟搭在路德身上.
即是說他願意娶路德為妻.
你說這些都是舊約.
新約沒有這支歌唱.
都有的.
你看看在耶穌基督的時候.
在馬太福音第9章.

$^{721}$和馬可福音第5章.
有一個血流婦人.
她有血流病.
血流病當時是令人羞恥羞愧.
你不可以跟別人說.
這些是不潔的.
你不說都不可以跟別人說.
這個婦人就偷偷走到.
耶穌基督的背後.
然後去摸耶穌基督的衣襟.
摸他的水指.
一摸就有能力從衣襟水指.
到這個婦人的身上.
幫她治好她的病.
所以我們看到.
衣襟真的有非常重要的地位性.
所以大衛挖掘了.
素羅的衣襟下來.
他非常內疚.
他認為他自己冒犯了素羅.
當素羅離開了山洞.
走了幾遠.
大衛就向他呼喚.
叫他做什麼.
我的主啊.
我的王啊.
你是耶和華的受高者.
是我的父親啊.
為什麼叫他做父親.
他是大衛的岳父.
他對素羅非常尊重.
甚至屈身臉伏於地下拜.
在我開始講這篇信職的時候.
我跟大家講過.
在我們的生命當中.
有沒有一些上司.
有沒有一些老闆.
可能對我們很苛刻.
甚至好像想要我們的命一樣.
但是聖經怎麼教我們.

$^{761}$是要服從.
是要尊敬羅馬書第十三章.
和彼得前書第二章.
十三至十七節.
大衛開始為他自己辯護.
其實大衛的邏輯是無懈可擊的.
大衛說什麼.
我是有機會殺死你的.
但是我沒有這樣做.
我的手下都說應該殺死你.
但我反而斥責他們.
我沒有殺你.
我對你是完全沒有惡意.
大衛用了一句真言.
是當時的人耳熟能詳.
是什麼.
他說惡事出於惡人.
惡事出於惡人.
是什麼意思.
就是說一個人的行為.
表現出他的品格.
大衛說你對我這樣.
我有機會殺死你.
我也不殺死你.
你可以看到我的品格.
但其實大衛在暗示什麼.
反過來看看你.
我又沒有得罪你.
我還是你的女婿.
還是你軍隊裡的統領.
你反而窮追不捨地殺死我.
你的品格又怎麼樣.
好像在暗示給訴羅聽.
跟著大衛說什麼.
冤耶和華在你我中間判斷是非.
在你身上為我伸冤.
我卻不親手加害於你.
大衛還說.
為什麼你要聽人的殘言.
為什麼你要聽人說.

$^{801}$我是謀反.
我根本沒有這樣的心意.
我根本沒有傷害你的心.
大衛還說什麼.
我被你追殺.
好像一隻死狗.
好像一隻跳蚤一樣.
在那時候的以色列的地方.
這些話語是用來罵人的.
你呀 你這隻死狗.
你這隻跳蚤.
你這隻死狗.
這麼小的.
不是用來罵自己的.
大衛怎麼說.
我是死狗被你追到.
好像一隻死狗.
好像一隻跳蚤一樣.
可以看到大衛在訴羅面前有多謙卑.
他再次跟訴羅說.
耶和華是公義的審判者.
會為他忠心的僕人辯護.
他說我將一切交給神.
神會主持公道.
於是訴羅就說.
哇 情緒很混亂.
訴羅的情緒就好像有些問題.
現在我們的術語會說他.
有些問題.
有時就好像窮追不捨對大衛.
有時就對著他在哭.
這次就是了.
他居然在大衛的面前哭.
不是第一次.
當約拿丹來勸訴羅的時候.
訴羅就向約拿丹發誓.
我不會殺死大衛的.
現在情緒又是非常不穩定.
這就是訴羅.
西相其實對人來說有三種回應.

$^{841}$第一種回應是什麼.
以善報惡.
這就是大衛.
是屬靈的回應.
人家對你不好.
你對人好.
還有第二種.
人家對你好 你就對人好.
以善報善.
這種是什麼.
屬血氣 屬肉體.
你說有沒有搞錯啊牧師.
人人都是這樣的.
人對你好 你就對人好.
這句話不是我說的.
屬肉體 屬血氣.
是耶穌基督說的.
福音六章三十二節.
耶穌基督說.
善待那善待你們的人.
罪人也是這樣做.
罪人都會.
對你好 你又對人好.
所以這個屬血氣 屬肉體.
還有第三種人.
就是以惡報惡.
甚至是以惡報善.
人家對你好 你對人家不好.
這個就像素羅一樣.
但素羅現在情緒不穩定.
居然承認大衛比他更加公義.
他承認大衛是以善報惡.
沒有殺到自己.
但他反而到處和大衛作對.
素羅甚至公開承認.
大衛將會成為下一任以色列人的王.
三號以上的二十三章第十七節.
他還說以色列國會在大衛的手上被建立.
有人會說素羅痛改前非.
是不是這樣?.

$^{881}$不是的.
我和大家說過.
真正認罪悔改有沒有條件?.
沒有條件的真正認罪悔改.
有條件就有問題.
素羅的條件是什麼?.
素羅永遠關心自己的名譽.
關心他的後裔.
所以他和大衛說.
我要你發誓你成為王之後.
你不會對待我的後人.
你不會滅了我的後人.
素羅永遠關心他的名譽.
他的後裔.
但他從來不為他的人民關心.
其實大衛已經答應了約拿丹.
和約拿丹所立的約.
在撒姆以上的十八章和二十章.
大衛已經答應了不會傷害素羅的後裔.
其實也不需要大衛去殺死素羅的後裔.
大衛沒有追殺素羅的後裔.
但如果我們看到在《說聖經》裡.
記載很多他的後裔都沒有了.
我相信是什麼?.
就是因為素羅很多人憎恨他.
所以當他失去勢力的時候.
他的後裔首先遭殃.
死剩一個是誰?.
《聖經》告訴我們.
就是米菲波切.
為什麼會死剩他?.
因為他是一個殘廢的人.
所以我相信這個反而救了他.
沒有被人殺死.
在撒姆以下的第九章.
大衛會收養了米菲波切.
而且和他同桌吃飯.
於是素羅就回到基比亞的地方.
而大衛呢?.
繼續過著他流亡的生活.

$^{921}$為什麼?.
你以為素羅痛哭流淚.
言語激動.
你以為他就放了大衛.
不是的,他繼續追殺大衛.
在結束的時候.
我們看看.
大衛一生打了很多場仗.
無數的仗.
但他有一場仗打得最漂亮的是哪一場?.
在《引機底山洞》裡的這場仗.
是打得最漂亮的.
不動一兵一卒.
也沒有殺死素羅.
但他可以平伏六百人.
而這六百人在日後會對他死心塌地.
大衛讓他的手下看到.
一個仁義的領袖應該是怎樣做.
在《箴言》第十六章第三十二節.
說不輕而發怒的勝過強者.
自理己深的勝過功成.
這個也很值得我們學習.
我剛才也說過.
神可能在我們的生命當中.
也有領袖上司老闆在我們身上.
我們應該怎樣對待他們呢?.
就算他們對我們很差.
我們應該學習大衛的做法.
就是知道身冤在神.
身冤在神在《說煞暮異常》的二十三章至二十五章是一個主題.
今天我們說了二十三章到二十四章.
下一次我們會說二十五章.
都會看到身冤在神是一個主題.
當我們將事情用自己的方法去處理的話.
我們做出來的事會比神的處理好.
有沒有這樣的可能性呢?.
沒有這樣的可能性.
所以如果我們身冤在神.
將事情放在神的手中的話.
將會得到一個更好的後果.

$^{961}$大衛就是這樣做.
但也有人會跟我說.
你今天這篇道叫做不可伸手害神的受高者.
你說我的老闆上司不是我的受高者.
但你不要忘記.
素羅雖然曾經被神高納.
但已經被神廢了.
嚴格上來說他不是大衛的受高者.
但大衛仍然尊敬他.
大衛仍然來服從他.
這就是羅馬書第十三章和希伯來書十三章.
都告訴我們我們要服從掌權.
雖然我說要服從掌權.
但我還是要說清楚.
我們不是盲目服從.
當我們的上司我們的老闆.
甚至我們的國家.
叫我們做一些事是違背聖經的教導.
服不服從?絕對不會服從.
一個例子就是在疫情的時候.
政府說什麼?不可室內敬拜.
當時我就說我不會服從.
而我們粵語堂是我們教會三堂裡.
最早在室外開始敬拜.
當時要向理事會申請.
英文部說你們粵語部這麼小.
為什麼要先過我們敬拜?.
理由是什麼?是神的話語.
神的話語說不可停止聚會.
所以我們首先在室外敬拜.
然後很快就搬到室內來敬拜.
另外一個例子.
剛才Larry也提到.
有很多國家現在已經沒有了禱告的自由.
有一天如果我們的政府說.
你不准禱告.
或者說你不可以說同性戀是罪.
又或者說你不可以說變性是罪.
你猜我在台上會不會繼續說呢?.
我可以告訴你我會繼續說.

$^{1001}$坐牢?沒問題.
只要服從神的話語.
提養東西什麼都不用怕.
願神祝福他自己的話語.
讓我們低頭禱告.
讓我們看到大衛的生平當中.
他的確有很多困難.
被訴羅追殺.
非常危險的時候.
但他仍然抱著新冤在身.
就算他有機會殺死訴羅.
他也不願意伸手害神所高納.
願意我們能夠學習.
在我們的生命當中.
有些上司領袖老闆.
很不近人情.
很可能他們非常苛刻.
甚至我們覺得好像想要我們的命一樣.
但願意你能夠讓我們學大衛.
知道新冤在身.
知道你的處理怎樣都比我們好.
願意我們能夠將一切權柄歸給你.
由你來處理.
但我們也要學會.
雖然我們要服從我們的上司.
服從我們的掌權.
但如果他們叫我們做的事.
是違背你的話語的話.
我們也應該不要服從.
因為你的話語才是最重要.
你的話語就是你自己.
願意我們能夠有這個心.
榮耀你的名 榮耀你的話語.
將一切的榮耀贊歸給你.
我們禱告.
奉主耶穌基督的名義而求 阿們.
阿們.
多謝各位收聽.
\newpage



\section{羅馬書 8:28-39}
\label{sec:9ORA5941xxk}
\textbf{永不能與主的愛隔絕 (羅馬書8\_28-39) - 袁惠鈞牧師[羅馬書系列 - 第22講]}
\newline
\newline
連結: \href{https://youtube.com/watch?v=9ORA5941xxk}{\texttt{ https://youtube.com/watch?v=9ORA5941xxk}} ~~~~ 語音日期: 2025-01-03 
\newline
\newline
\hyperref[sec:GqTOPwqfjwM]{< < < PREV SERMON < < <}
~
\hyperlink{toc}{[返主目錄]}
~
\hyperref[ch:preacher6]{[返講員目錄]}
~
\hyperref[sec:w_ajWsBZ9eQ]{> > > NEXT SERMON > > >}
\newline
\newline
羅馬書 8:28-39
\newline
\begin{longtable}{cl}
\hline
\hline
章節 & 經文 (和合本修訂版)\\
\hline
8:28 & \begin{tabularx}{0.7\textwidth}{X} 我們知道,萬事都互相效力,叫愛神的人得益處,就是按他旨意被召的人。 \end{tabularx} \\ \\ \relax
8:29 & \begin{tabularx}{0.7\textwidth}{X} 因為他所預知的人,他也預定他們效法他兒子的榜樣,使他兒子在許多弟兄中作長子。 \end{tabularx} \\ \\ \relax
8:30 & \begin{tabularx}{0.7\textwidth}{X} 他所預定的人,他又召他們來;所召來的人,他又稱他們為義;所稱為義的人,他又叫他們得榮耀。 \end{tabularx} \\ \\ \relax
8:31 & \begin{tabularx}{0.7\textwidth}{X} 既是這樣,我們對這些事還要怎麼說呢?神若幫助我們,誰能抵擋我們呢? \end{tabularx} \\ \\ \relax
8:32 & \begin{tabularx}{0.7\textwidth}{X} 神既不顧惜自己的兒子,為我們眾人捨了他,豈不也把萬物和他一同白白地賜給我們嗎? \end{tabularx} \\ \\ \relax
8:33 & \begin{tabularx}{0.7\textwidth}{X} 誰能控告神所揀選的人呢?有神稱他們為義了。 \end{tabularx} \\ \\ \relax
8:34 & \begin{tabularx}{0.7\textwidth}{X} 誰能定他們的罪呢?有基督耶穌 已經死了,而且復活了,現今在神的右邊,也替我們祈求。 \end{tabularx} \\ \\ \relax
8:35 & \begin{tabularx}{0.7\textwidth}{X} 誰能使我們與基督的愛隔絕呢?難道是患難嗎?是困苦嗎?是迫害嗎?是飢餓嗎?是赤身露體嗎?是危險嗎?是刀劍嗎? \end{tabularx} \\ \\ \relax
8:36 & \begin{tabularx}{0.7\textwidth}{X} 如經上所記:「我們為你的緣故終日被殺;人看我們如將宰的羊。」 \end{tabularx} \\ \\ \relax
8:37 & \begin{tabularx}{0.7\textwidth}{X} 然而,靠著愛我們的主,在這一切的事上,我們已經得勝有餘了。 \end{tabularx} \\ \\ \relax
8:38 & \begin{tabularx}{0.7\textwidth}{X} 因為我深信,無論是死,是活,是天使,是掌權的,是有權能的 ,是現在的事,是將來的事, \end{tabularx} \\ \\ \relax
8:39 & \begin{tabularx}{0.7\textwidth}{X} 是高處的,是深處的,是別的受造之物,都不能使我們與神的愛隔絕,這愛是在我們的主基督耶穌裡的。 \end{tabularx} \\ \\
[1ex]
\hline
\hline
\end{longtable}
$^{1}$今天是聖誕節後第一個主日.
其實我很想講一個有關聖誕節的訊息.
但是我隨即想到現在在講羅馬書.
今天的主題是什麼?.
永不能與主的愛隔絕.
聖誕節是耶穌基督到城肉身來到世上與我們同在.
有什麼訊息能夠使我們與耶穌基督的愛隔絕.
這也是一個非常好的訊息.
今年最後一個主日的訊息.
就是下一年我們都能夠有這個盼望.
主的愛永遠不會與我們隔絕.
其實上一次我已經開始講這一節的經文.
就是萬事互相效力.
要愛神的人不道益處.
有人說如果羅馬書是一個筵席的話.
羅馬書的第八章就好像是主菜一樣.
我們粵語堂敬拜完之後都很喜歡去同一間餐館.
每個星期都去那裡聚餐.
老闆覺得我們去得多就送一些甜品給我們.
送一些煎堆仔給我們吃.
我時時都說這裡才是主菜.
因為我很喜歡吃這些煎堆仔.
如果羅馬書真的是一個筵席的話.
這些煎堆仔就是羅馬書的第八章.
羅馬書其實很重要給我們看到的.
就是我們得到救贖的保障.
早在羅馬書的第五章我們已經看到.
是耶穌基督為我們釘十字架救贖了我們.
這個是非常可靠的.
每一個接受他的人都能夠被他救贖.
現在去到羅馬書的第八章.
是聖靈為我們工作.
三位一體的神友兩位都幫助我們.
聖子耶穌基督救贖我們.
聖靈保障我們能夠完成我們的救贖.
羅馬書真的可以說是一氣呵成.
羅馬書的第八章就更加有連貫性.
一開始就說.
如今那些在基督耶穌裡的就不定罪了.
然後去到羅馬書的最後一章.

$^{41}$第八章34節.
保羅問有誰能定他們的罪呢?.
答案是什麼呢?.
沒有人能夠定我們的罪.
為什麼呢?.
因為耶穌基督救贖我們.
因為聖靈繼續幫助我們.
羅馬書的第八章28節.
其實可以說是羅馬書第八章的一個總結.
我們得救的原因是什麼呢?.
神要將榮耀歸給自己.
但是在這個過程當中.
萬事互相效力.
要愛神的人不一處.
就是按著他旨意被召的人.
有一個清教徒說了一句話很有意義.
神就像一位名醫.
一位很厲害的藥劑師一樣.
他開的方子裡開的藥丹裡.
可能甚至獨立來說有些藥是有毒的.
但是就是透過全部這些藥加在一起.
就能夠醫治人.
所以今天保羅所說的也是一樣.
萬事互相效力.
是要愛神的人得到益處.
就算我們生命當中有逆境的話.
神都能夠幫助我們完成我們的救贖.
在聖經的歷史裡真的有很多這些例子.
尤其是在以色列人身上.
在新命紀的八章16節.
神帶領以色列人40年漂流在曠野的生活.
為了什麼呢?.
就是要磨練他們.
幫助他們進入英許之地.
能夠得到英許之地的祝福.
在耶利米蘇的二十四章第五節.
甚至神將以色列人趕出英許之地.
又是為了什麼呢?.
都是為了他們的好處.
就像修理無花果樹一樣.

$^{81}$要他們得到最後的益處.
在哥林多後書的四章十五節.
保羅也有說:凡事都是為你們.
好叫因為因人多 越發加真.
這節的經文跟我們今天的經文很相似.
都是為了你們要將榮耀歸給神.
其實在人類的歷史當中.
有一個非常好的例子.
神透過一件不好的事使人得益.
是什麼呢?.
就是耶穌基督在十字架上為我們所做.
無論你怎麼看.
將一個義人無罪的人釘在十字架上.
都是一件壞事.
但神就是透過這件事使萬人得救.
這就是一個很好的例子.
我上次也跟大家說過.
羅馬書的八章二十八節.
是一節很被人濫用的經文.
很多時候我們自己做錯了事就會怎樣?.
哎呀不要緊.
萬事互相效力都是要我得益處.
很對不起.
這節的經文是在說救贖.
不是說你做錯事都是有益處.
是救贖萬事互相效力.
要愛神的人就是按他旨意備召的人得益處.
就是要完成神給我們的救贖.
這個益處這個益字或者是好處.
原文希臘文的Agathon這個字.
不是物質上的好.
不是外在的好.
是內在的好和本質上的好.
也可以說是道德上的好.
所以不是說神顛倒是非黑白.
明明是壞事他說是好的.
明明是好事他又說是壞的.
而是要完成他在我們身上的救贖.
然後將榮耀歸給自己.
你可能會問.

$^{121}$有什麼壞事發生在我們身上?.
可能是一些苦難.
所有的苦難都和罪有關.
未必是你自己的罪.
但所有的苦難都是和罪有關.
是和亞當夏娃的罪有關.
因為當他們未墮落.
未犯罪之前.
這個世上是沒有罪.
是沒有苦難.
就是因為他們犯罪之後.
苦難就淋到這個世界.
但不是所有的苦難都和你自己的罪有關.
和你的罪有關的是什麼?.
神的管教.
當你犯罪之後.
如果你是神的兒女的話.
神會管教你.
就好像他管教以色列人一樣.
當神管教你的時候.
你就知道你自己真的是神的兒女.
希伯來書的十二章第八節.
約伯他的朋友也有對約伯說.
神所懲治的人是有福的.
所以你不可輕看全能者的管教.
千萬不要輕看神的管教.
如果是他的兒女犯罪的話.
他會管教我們.
還有一種的苦難.
是和我們的罪無關.
這個就是磨練.
神會容許苦難淋到我們的身上來磨練我們.
就好像約伯.
記不記得約伯有苦難淋到他和他家人的身上.
是不是因為他犯罪?.
不是的 是因為神要磨練他.
另外在舊約有一個好的例子就是約瑟.
神容許苦難淋到約瑟的身上.
為了什麼呢?.
是要磨練他.

$^{161}$以至他將來能夠救贖他的家人.
當以色列有饑荒的時候.
神已經差派他去埃及成為宰相.
能夠透過他在埃及的苦難.
慢慢成為埃及人的宰相.
以至他將來能夠救贖他的家人.
新約另外有個例子就是保羅自己.
他身上有一斤刺.
求神三次.
拿開這斤刺神都沒有拿開.
因為神知道這斤刺對他有益處.
可以幫助他信靠神 倚靠神.
聖經裡經常用煉銀來代表.
比喻一個人被神磨練.
你想想真的很貼切.
你怎樣來煉銀呢?.
你是要用火來燒銀.
火代表那些苦難的磨練.
慢慢燒到銀紅的時候.
除去雜質.
就好像一個人被神磨練.
除去他身上的雜質.
越來越清潔 越來越像耶穌基督.
所以詩篇第66篇第10節.
神啊 你曾試驗我們 磨練我們.
與磨練銀紙一樣.
那你說神容許發生在我們身上的苦難.
會有什麼後果呢?.
一共有四個後果.
第一個後果就是會幫助我們憎恨罪.
這就像大衛被敵人追殺的時候.
他懇求神懲罰敵人.
因為他憎恨敵人的罪.
在詩篇第109篇我們看到.
在這裡要做一個廣告.
下一年我會說大衛的故事.
會說很多大衛的詩歌.
今年我們知道英文課是講詩篇.
國語部也是講詩篇.
但我就先不說了.

$^{201}$因為我知道下一年會說很多大衛的詩歌.
所以我就留待下一年再說.
另外在新約拉薩路的事我們也看到.
耶穌基督去到拉薩路的家裡.
祂是哭泣.
那你說為什麼這麼奇怪呢?.
耶穌基督知道很快會使拉薩路起死回生.
使祂復活.
那為什麼還要為拉薩路哭泣呢?.
其實祂不是為拉薩路的死而哭泣.
祂是為人的罪而哭泣.
因為亞當夏娃還沒有犯罪之前.
還沒有墮落之前.
這個世上是沒有死亡的.
就是因為人的罪以至死亡進入了這個世上.
所以耶穌基督是為了這件事而哭泣.
因為祂是真恨罪.
除了苦難會幫助我們真恨罪.
苦難也會幫助我們監察我們自己.
歷代志下的33章11至12節.
所以耶和華使亞述王的將帥來攻擊他們.
用撓鉤勾住馬拿西.
用銅鏈鎖住他.
帶到巴比倫去.
他在急難的時候就懇求耶和華他的神.
並在他列祖的神面前極其自卑.
當我們遇見苦難的時候.
我們就會謙卑自己.
我們就會接近神 依靠神.
我們知道這裡所說的就是南國的王馬拿西的事.
我們知道北國是被亞述國征服滅國.
但是南國沒有被亞述國滅國.
但是曾經南國的王馬拿西是被亞述王擄去.
當我們遇見苦難的時候.
不單止幫助我們真恨罪.
也會幫助我們監察自己.
還有第三點.
苦難也會幫助我們親近神.
神警告以色列人.
當他們進入迦蘭地安居樂業的時候.

$^{241}$他們就會忘記了神.
在《申命記》的六章十二節.
那時你要謹慎.
免得你忘記將你從埃及地遺牢之家領出來的耶和華.
如果一個人安居樂業就會離棄神的話.
遇見苦難就剛剛相反.
就會使我們更加親近神.
所以在哥林多後書的一章八至九節.
保羅提醒我們.
弟兄們,我們不要你們不曉得.
我們從前在亞細亞遭遇苦難.
被壓太重,力不能勝.
甚至連活命的指望都絕了.
自己心裡也斷定是必死的.
我們不靠自己,只靠叫死人復活的神.
這裡所說的事跡.
究竟保羅遇見什麼大困難.
我們已經不知道,聖經沒有說.
但相信是在他寫哥林多前書後所發生的事.
這裡讓我們看到.
當我們遇見苦難的時候.
不靠自己,只靠叫死人復活的神.
我們就會親近神.
最後一點,第四點.
當我們遇見苦難會怎樣呢?.
不單與神親近,也會與人和好.
這就是舊約的故事.
剛才我也說過舊約的約瑟.
他的兄弟害他.
結果神的意思是好的.
就好像創世記的五十章第二十節.
約瑟說從前你們的意思是要害我.
但神的意思原是好的.
我們與人的關係反映出我們與神的關係.
當我們與神和好,與神親近的時候.
我們也會與人和好.
這是聖經教導我們.
透過苦難,約瑟是與人和好.
因為他是與神和好的人.
所以我們看到神容許我們遇見苦難.

$^{281}$有四點的後果.
第一,幫助我們真行罪.
第二,使我們鑒賊自機.
第三,幫助我們親近神.
最後使我們與人和好.
除了苦難,還有什麼壞事.
臨到我們身上是為了我們的益處.
就是試驗.
你說苦難和試驗有什麼分別呢?.
苦難就是一些逆境,一些困境.
但試驗未必是逆境.
就好像耶穌基督在曠野四十天.
被撒旦引誘.
甚至撒旦說如果你跪在我面前敬拜.
我就將全世界都給你.
所以可能是一些很大的引誘.
亞各書的一章三至四節.
因為知道你們的信心.
經過試驗就身忍耐.
但忍耐也當成功.
使你們成全完備,毫無缺陷.
當耶穌基督經歷試探的時候.
他沒有犯罪.
聖經告訴我們耶穌基督經歷了世上所有的試探.
希伯來書第四章,但他沒有犯罪.
所以當我們經歷試探.
而我們能夠不犯罪.
我們能夠經得起試探的時候.
就等於什麼呢?.
就是等於我們能夠越來越像耶穌基督.
所以亞各書的一章二十七節說.
忍受試探的人是有福的.
因為他經過試驗以後.
必得生命的冠冕.
或者是主應許給那些愛他之人.
其實苦難和試驗都很相似.
兩樣神容許發生在我們身上.
都是為了我們好.
兩樣東西都能使我們親近神.
當一個人遇見試驗的時候.

$^{321}$我們就會怎樣呢?.
禱告也多了.
就好像耶穌基督在船馬太福音的六章十三節.
教我們主禱文.
教我們不遇見試探.
但脫離凶惡.
就是這個意思.
當我們遇見試探的時候.
很多時候我們就會禱告多了.
會更加親近神.
萬事互相效力就是要愛神的人得到益處.
那你說是什麼益處呢?.
保羅有告訴我們在羅馬書的八章二十九節.
因為他預先所知道的人.
就預先定下效法他兒子的模樣.
使他兒子在許多弟兄中作長子.
是什麼益處呢?.
是要我們最終能夠像耶穌基督一樣.
我剛才說過.
這個不是外在的益處.
不是物質上的益處.
而是內在本質上的益處.
所有神容許發生在我們身上的事.
都是為了達到這個目的.
就是要效法他兒子的模樣.
羅馬書的八章三十節很有趣.
是很有系統性的.
一樣一樣來發生在我們身上.
有什麼呢?.
首先他預先所定下的.
他就會召他來.
所召他來的人又稱他們為義.
所稱為義的人又叫他們得到榮耀.
有沒有看到這裡有四點.
一樣跟著一樣.
首先是他預定預先知道的人.
第二點他就會召他們來.
第三點他就會稱他們為義.
最後他就會使他們得到榮耀.
我們逐樣逐樣來看.

$^{361}$第一樣就是他預先所定下.
神預先要揀選你.
要預先定下.
他需要拿出他的水晶球嗎?.
看看誰會接受我.
我就揀選他.
他需要這樣做嗎?.
他不需要這樣做.
他是預先知道.
他不單止預先知道.
他還預定.
使徒行傳的十三章四十八節.
外邦人聽見這話就歡喜了.
讚美神的道.
凡預定得永生的人都信了.
有沒有看到這裡預定這個字.
就是原希臘文的Boule這個字.
就是預先定了你會得到永生.
那段經文很有趣.
如果你留意的話.
他預先知道的人.
就預定下效法他兒子的模樣.
有沒有看到有兩樣東西.
預知和預定.
希臘文的文法如果有兩樣東西.
中間有個連帶詞.
這個連帶詞是什麼呢?.
英文就是also.
中文就是就.
兩件事被連接詞連在一起.
前面那個字有個定貫詞.
有個definite article.
後面那個如果沒有的話.
這兩樣東西都是相同.
都是說一樣的東西.
這個就是希臘文的文法.
所以神所預知的就等於他預定.
他所預定的就等於他預知.
希臘文的pronosis這個字.
是由兩個字加在一起.

$^{401}$pro就是預先.
nosis就是知道.
這個字我和大家說過很多次了.
你們應該都認得.
知道或者認識是很深入的.
是有愛的成分在裡面.
不是單單知道有一件事.
而是很深入的了解.
聖經裡面說.
當知道夏娃或者認識夏娃.
就生了該人出來.
所以你看到這個很親密.
在亞摩斯書的三章二節.
聖經說在地上萬族中我只認識你們.
神說我只認識以色列人.
你說神那麼孤陋寡聞.
地上有萬族他只知道有以色列人.
這裡不是這個意思.
這個知道這個認識是很深入的.
是有愛的成分在裡面.
神說我只愛以色列人.
現在來說有很多人都很憎恨以色列.
為什麼呢?可能是撒旦的工作.
很憎恨以色列.
但我不管你憎恨以色列也好.
你喜歡以色列也好.
神是喜歡以色列.
神是鍾愛以色列人.
他說以色列人是他眼中的同人.
這一點我們要很注意.
在提摩太后書的二章十九節.
經文說主認識誰是他的人.
神不單止預先知道他揀選的人.
他也愛這些人.
這節的經文就是這樣的解釋.
除了他所預知的他愛他們.
還有什麼呢?.
神就召他們來.
呼召這個字我們要很注意.
因為在新約的聖經有兩個不同的用途.

$^{441}$一個用途主要在福音書裡面.
另一個用途主要在福音書以外.
新約的書信裡面.
我們先看第一個用途.
就是在福音書裡面的呼召.
是什麼意思呢?.
是大眾化的呼召.
是一般性的呼召.
是外在的呼召.
這個就像馬太福音的二十二章十四節.
因為被召的人多.
選上的人少.
有沒有看到被召這個字.
是用在一般性大眾化的身上.
很多人被召.
全世界的人都被召.
但是得救的有多少呢?.
被選上的很少.
這個就帶領我們去到第二個用途.
就是在新約的書信裡面的用途.
在新約的書信裡面的用途.
呼召這個字就變得是個人化.
變得是內在的呼召.
就是羅馬書八章三十節所說的.
神所呼召的人就是他所揀選的人.
有沒有看到是個人化.
是內在的.
當我們回應神的呼召的時候.
神就會稱我們成為一個義人.
這個就帶領我們去到第三點.
神稱我們為義人.
這個義字有什麼解釋呢?.
就是做對的事.
一個義人就會做對的事.
神怎樣稱我們為義人與神和好呢?.
就是將我們的罪放在耶穌基督的身上.
然後將耶穌基督的義放在我們的身上.
交換了.
這個就是我們在說羅馬書第五章的時候說的.
為什麼不可以靠我自己的義呢?.

$^{481}$神是無限聖潔的.
我不管你做到怎樣好.
你都不能夠達到無限的聖潔.
只有耶穌基督的聖潔才合乎神的標準.
所以被神呼召的人就稱你為義人.
這個呼召是內在的呼召.
我剛才說過.
有人就會問.
神內在的呼召.
去到你被神稱為義人有多久呢?.
沒有人知道.
每個人都不同.
但我相信這個時間性是很短的.
還有呢.
不單止神預先知道你認識你愛你.
然後他呼召你.
然後他稱你為義人.
最後你會得到你的榮耀.
這是第四點.
但是注意原文希臘文的文法很有趣.
這四件事完全是一樣的文法.
不是說神以前認識你.
以前預先定了你不救.
然後現在呼召你.
稱你為義人.
將來你會得到榮耀.
雖然我們的榮耀.
真的要等到耶穌基督再次來臨.
我們才能夠得到.
但這節的經文文法完全一樣.
所以可以說我們現在已經在主內北度榮耀.
這就是約翰福音17章22節所說.
所以聖經的教導.
是神揀選我們.
不是我們揀選神.
預定我們能夠得救.
但如果是這樣.
神是否預定人們落地獄呢?.
有很多人都會這樣問.
聖經絕對沒有告訴我們.

$^{521}$神是預定人落地獄.
你說不合理.
如果神預定某些人上天堂.
其他人就落地獄.
但這裡最好的解釋.
就是用神是光來解釋.
如果神是光.
接受祂的人有沒有光?.
有光.
但不接受祂的人呢?.
沒有光.
他們是去到地獄.
所以是他們自己的作為.
不是神一早預定他們落地獄.
所以我們看到.
這永遠都是雙方面.
連今天的經文我們都看到是雙方面.
怎樣看到是雙方面呢?.
首先我們看到.
愛神的人得益處.
愛神的人從誰的角度來看呢?.
是由我們的角度來看.
我們要接受神 要愛神.
但還有另外一邊.
是照著祂的旨意被照的人.
有沒有看到是雙方面的?.
我們有我們要做的.
就是接受祂 就是愛祂.
但神也有祂的方面.
就是祂已經預定了接受祂的人.
所以永遠都是雙方面.
當我們不接受祂的時候.
我們在黑暗當中不是因為神的作為.
而是我們自己選擇這樣做.
有人會問.
究竟得救的目的是什麼?.
神為什麼要我們得救?.
很多人說.
就是要我們的罪被赦免.
也有很多人說.

$^{561}$得救就是得到神的愛.
得到和平 得到喜樂.
神賜一些恩賜給你.
這些都是我們傳福音的時候.
喜歡跟別人說的.
神賜你很多東西.
你會得到這樣 你會得到那樣.
是就是的.
但這只不過是你得救的一部分.
就算你能夠上天堂.
都是你得救的一部分.
你得救最重要的目的是什麼?.
是要像耶穌基督.
像耶穌基督也有兩個部分.
也有外在和內在.
外在的是什麼?.
就是將來我們每個人都會有復活的身體.
有榮耀的身體.
就好像耶穌基督復活的身體一樣.
無瑕疵 無疾病 不會死.
這就是外在.
我們將來是會像祂.
但現在很多人忽視了.
就是現在內在我們都要像耶穌基督.
我們要越來越聖潔.
越來越像耶穌基督.
這就像哥林多後書的三章十八節說.
就像我們照鏡一樣.
榮上加榮.
是要從一個榮耀的水平.
去到另外一個榮耀的水平.
就是要我們越來越像耶穌基督.
越來越聖潔.
我們要像耶穌基督.
不單是外在 也是內在.
羅馬書第八章二十九節說.
要在許多弟兄中作長子.
Pro total cause這個字就是長子.
猶太人的長子是能夠得到雙倍的產業.
好像我們中國人.

$^{601}$我聽過台山人都是這樣.
兒子都不是雙倍的.
拿了所有的東西.
但是在猶太人的律法.
長子是得到雙倍的產業.
我們每個人都要像耶穌基督.
但長子只有一個.
所以我們要做什麼.
我們要越來越聖潔.
將榮耀歸給他.
要將長子襯托出來.
我們就是六葉.
耶穌基督就是花.
我們就是要將耶穌基督襯托出來.
將榮耀歸給他.
所以神救贖我們.
最終的目的.
不是要你脫離地獄.
不是要你上天堂.
不是要你得到這一切的好處.
最終的目的是要你像他的獨生子.
要你像耶穌基督.
不單是外在也是內在.
所以我們看到.
是什麼人得到益處.
就是愛神的人得到益處.
有人會問.
哪些是愛神的人.
有人說是被神救贖的那些.
是被神揀選的人.
但聖經怎樣告訴我們.
愛神的人就是被神赦免罪的人.
我再說一次.
愛神的人就是被神赦免罪的人.
你說從哪裡看到這個教導.
在路加福音第七章.
記不記得有一個法利賽人叫西門.
這個西門曾經請耶穌基督.
去他的家吃飯.
當時有一個女人.

$^{641}$有一瓶香膏.
很名貴的香膏.
打碎之後抹在耶穌基督的腳上.
西門這個法利賽人見到之後.
心裡立刻想.
如果耶穌基督真是先知的話.
怎會不知道這個女人是什麼人.
五三五四的女人會被他摸自己的腳.
有沒有這個可能.
耶穌基督知道他在想什麼.
就跟他說了一個比喻.
他說就好像一個主人有兩個僕人.
兩個僕人都欠了這個主人的債.
一個欠了五十兩.
一個只欠了五兩.
兩個人都無力償還.
而這個主人就赦免了他們的債.
耶穌基督問.
哪個僕人會更加愛這個主人.
西門答得很正確.
他說當然是欠錢多的那個.
會更加愛這個主人.
耶穌基督說你說對了.
這個女人就是知道她自己有多大罪.
所以她才用這麼多錢買了一瓶這麼名貴的香膏.
表示她的愛塗在耶穌基督的腳上.
耶穌基督說西門你答對了.
所以一交道是什麼.
我們如果是被神赦免得更多的人.
我們就會更愛神.
被神赦免得更多的人就會更愛神.
你可能會說.
牧師我都沒有什麼罪.
我又沒有殺人又沒有放火.
你跟我說這些.
有沒有搞錯.
你忘記了你自己那單誤殺罪.
你說更離譜.
你這個牧師血口噴人.
我哪裡有殺人.

$^{681}$那單誤殺罪你還跟我一起合謀.
我們一起合謀.
你說哪裡有這樣的事.
你知不知道.
把耶穌基督推上十字架.
為我們釘十字架.
是你和我的罪.
是我們的罪.
你說為什麼會是誤殺罪.
記不記得耶穌基督在十字架上怎麼說.
他說父啊赦免他們.
他們所作的他們不曉得.
我們做了我們都不知道.
所以我們都是犯了誤殺罪.
每一個人被神赦免的程度都是一樣.
我們都是犯了誤殺罪.
但不是每一個人愛神的程度都是一樣.
為什麼.
就是有很多人我們不承認我們的罪.
或者我們是不認識我們的罪.
如果我們認識我們的罪.
我們承認我們的罪的話.
我們真的應該很愛神.
愛神的人就是對神的回應.
你說我們應該怎樣愛神.
詩篇的十八篇告訴我們.
我們要思念他的榮耀.
有人說我有時也會思念的.
神的榮耀.
你不單止思念.
如果你真的思念他的榮耀.
你就會走出來.
你會去做大使命.
你會為主作功.
你不單止坐著思考就算了.
還有詩篇的三十一篇.
你會依靠神的能力.
你說我有很多事都做不到.
我時時都要依靠神的能力.
不是依靠神的能力來做你要做的事.

$^{721}$你有沒有依靠神的能力來為他作功.
為他做他的事功.
又好像詩篇的六十三篇和八十四篇說.
你有沒有渴望神的同在.
你說有時我也會的.
但你是不是更渴望留戀在這個世界.
你會不會像保羅所說.
我活著就是基督.
我死了就有益處.
你會不會是喜愛神的公義.
詩篇的八十九篇.
你會不會是遵守神的律例.
來填詩篇的一百十九篇.
這個世界只有兩種人.
沒有第三種.
就是要麼你愛神.
要麼你憎恨神.
你說我都不是那麼憎恨神.
我雖然不愛神.
但我沒有憎恨他.
但耶穌基督怎麼說的.
不與我相合的就是敵我的.
這是萬太福音的十二章三十節.
路加福音的十一章二十三節.
愛神的人就會怎樣.
就會得到救贖的保障.
保羅告訴我們.
如果你真是一個得救的人.
你真是愛神的話.
你這個救贖打風也打不掉.
保羅給了我們五個論點告訴我們.
為什麼我們的救贖有那麼大的保障.
第一就是神幫助我們.
很多時候我們都會像舊約的雅各一樣.
我剛才說了約瑟的故事.
雅各就是很愛約瑟.
但因為他的兄弟謀害他.
賣了他去埃及.
他就喜歡變雅文.
因為約瑟和變雅文都是拉傑生的.

$^{761}$他最喜歡拉傑.
但兄弟現在又來找雅各.
在埃及的宰相想見我們的小弟弟.
雅各說有沒有搞錯.
我已經不見了約瑟.
你現在連我的小弟弟也要拿走.
所以雅各就認為.
最壞的事就是發生在他的身上.
但很多時候我們有神幫助我們.
我們應該好像大衛的想法一樣.
在詩篇的二十七篇第一節.
他說耶和華是我的亮光是我的拯救.
我還怕誰呢.
耶和華是我性命的保障.
我還懼怕誰呢.
這裡就像羅馬書的八章三十一節所說.
神若幫助我們.
誰能抵擋我們呢.
有誰比神大得過.
神作出來的決定.
有誰能否定.
神說這個人是義人.
有誰能推翻這件事.
我們是神的兒女.
神一定將最好的給我們.
雖然現在我們有些苦難.
有些困境.
但我們要知道萬事互相效力.
是要愛神的人得一處.
就像耶利米蘇的二十九章十一節.
這一節的經文就是舊約裡面的萬事互相效力.
在耶利米蘇的九章十一節.
耶和華說我知道我向你們所懷的意念.
是賜平安的意念.
不是降災禍的意念.
要叫你們末後有子往.
這一節的經文就是說.
不是我的旨意要降災.
就算你們身上有災難的話.
你們要知道末後是有子往.

$^{801}$這就是舊約的羅馬書八章二十八節.
除了神幫助我們.
還有第二點保障我們的救贖.
是耶穌基督為我們死.
羅馬書八章三十二節.
神既不愛惜自己的兒子.
為我們眾人捨了.
豈不也把萬物和他一同伯伯的賜給我們.
這裡保羅是用了大事和小事的比較.
神連他的獨生子.
他最愛的都給了我們.
還有什麼其他他不會給我們.
這就是大和小的比較.
耶穌基督在世的時候.
他的教導都是一樣.
都是用大和小的比較.
就算麻雀就算百合花.
神都眷顧.
何況你們是神的兒女.
又是大和小的比較.
還有另外一個角度.
我們可以看這件事是怎樣.
當我們還是罪人的時候.
微不足道.
神已經給了他的獨生子給我們.
赦免我們的罪.
現在我們成為神的兒女.
他是不是會更加幫助我們完成救贖.
有沒有看到這兩樣東西.
我們以前是罪人的時候.
他已經這樣幫我們.
何況我們現在是他的兒女.
還有第三樣.
第三個保障.
不單止神幫助我們.
耶穌基督為我們死.
還稱我們為異人.
我們知道在世上.
有一個人專門指控我們.
是誰呢?.

$^{841}$是撒旦.
在約伯記的第一章第二章.
我們看得很清楚.
還有啟示錄的十二章第十節.
都是這樣告訴我們.
但是在聖經裡有個例子.
讓我們清清楚楚看到這件事.
是什麼呢?.
是在撒加利亞書的第三章.
撒旦指控大祭司約書亞.
大祭司約書亞站在使者面前.
但是撒旦站在他右邊.
不斷地和他作對指控他.
但是已經被神稱為異人的人.
有誰能夠說他是一個罪人.
沒有人能夠改變這個事實.
在撒加利亞書的第三章第二節.
耶和華向撒旦說.
撒旦啊 耶和華責備你.
就是揀選耶路撒冷的耶和華責備你.
這不是從火中抽出來的一根柴嗎?.
什麼叫做從火中抽出來的一根柴?.
火是什麼?.
火是審判.
在火裡抽出來的柴是再沒有審判的.
就是羅馬書第八章一節所說的.
有誰能定他的罪?.
我們在耶穌基督裡就不被定罪.
在火裡抽出來的柴被神稱為異人的人.
有誰能夠再定他的罪?.
注意這裡的經文說.
約書亞大祭司是穿著污穢的衣服.
他不是穿著大祭司的袍.
解釋什麼呢?.
約書亞是一個罪人.
他不是說他聖潔被神稱為一個異人.
而是他真的接受神.
和我們一樣.
我們都是有罪.
但我們被神稱為異人.

$^{881}$我們這些信徒.
我們每天的感受都不同.
我今天就比較熟悉.
我明天就沒有那麼熟悉.
如果我們的得救.
是在乎我們自己熟悉的程度.
不用看了.
你一定不能夠得救.
甚至很多時候我們會控告我們自己.
我們失敗的時候.
我們身邊的人都會控告我們.
撒旦肯定不斷地在指控我們.
但是神會不會控告我們呢?.
不會控告我們.
因為我們的罪已經被耶穌基督承擔了.
神已經稱我們為異人.
這就是第三點.
其實已經很足夠了.
但是我們看到保羅再給我們印證.
我們真的有很大的保障.
第四個保障是什麼呢?.
耶穌基督親自為我們祈求.
羅馬書的八章34節.
誰能定他們的罪呢?.
有基督耶穌已經死了.
而且從死裡復活.
現今在神的右邊也替我們祈求.
祂不單止是我們代罪的羔羊.
為我們的罪而死.
祂現在是怎樣呢?.
成為大祭司在神的右邊.
不斷地為我們祈求.
三位一體的神友.
第二位的衛教耶穌基督親自為我們祈求.
三位一體第三位的衛教聖靈也為我們祈求.
這是我們之前曾經說過.
聖靈甚至用說不出的話語來幫我們祈求.
當我們不懂得禱告的時候.
聖靈為我們禱告.
所以我們這個救贖真的有很大的保障.

$^{921}$就好像羅馬書的五章第十節.
因為我們作仇敵的時候.
且藉著神兒子的死.
不與神和好.
既已和好.
就更要因他的身得救了.
耶穌基督的死救贖了我們.
祂不是說就這樣就算了.
祂復活了現在在父的右邊.
幫我們來祈求.
一個活著的耶穌基督.
更加能夠保障我們的救贖.
這是羅馬書裡面很清楚的主題.
還有第五點最後一點.
我們的保障因為耶穌基督愛我們.
羅馬書的八章三十五至三十九節.
在羅馬書的八章三十一至三十四節保羅.
讓我們看到神是不會放棄我們.
神一定會使我們幫助我們完成我們的救贖.
但是在八章三十五至三十九節.
是說沒有東西能夠令我們和耶穌基督的愛分開.
沒有東西能夠.
我見過有很多人在他們的生命當中.
有很多的考驗.
他們失敗了.
他們覺得很大的苦難淋在他們的身上.
他們不再相信.
但是經文說什麼呢?.
經文說沒有東西能夠令你和耶穌基督的愛分開.
這些人說我不再相信了.
我要實踐我的自由.
我有主權.
是呀!你說神不會放棄我.
我放棄我自己行不行?.
經文說什麼呢?.
沒有東西能夠令你和耶穌基督的愛分開.
沒有東西是包括什麼呢?.
是包括你自己.
耶穌基督的愛是能夠吸引著你.
令你繼續走完這條路.

$^{961}$但是你說我決定不相信了.
我決定放棄了.
那是不是這個人失去了他的救贖呢?.
不是.
我也說過很多次.
約翰一書的二章十九節.
我們當中出去的人是不是屬於我們呢?.
不是屬於我們.
如果真的屬於我們.
是不會由我們當中出去的.
所以這些人自己以為相信.
其實他們根本沒有真正的得救.
我們是絕對不會失去我們的救贖.
耶穌基督的愛就能夠保證吸引我們.
以致我們不會失去我們的救贖.
保羅接著說了什麼呢?.
他說了自己的例子.
在羅馬書的八章三十五節.
誰能使我們與基督的愛隔離呢?.
難道是患難麼?.
是困苦麼?.
是迫不得已麼?.
是飢餓麼?.
是赤身怒體麼?.
是危險麼?.
是刀劍麼?.
這一切都是保羅自己經歷的.
如果你讀哥林多後書的十一章.
二十三至二十七節.
你就會看到他經歷這一切.
神不一定會拿走你身上的苦難.
為什麼呢?.
因為這些苦難能夠幫助我們成長.
但是神保證我們.
當我們在苦難當中萬事互相效力.
要愛神的人不渡亦處.
我們經歷這些苦難.
最終要做什麼呢?.
將榮耀歸給神.
所以我們也可以說.

$^{1001}$當我們經歷苦難的時候.
就是為了神而經歷這些苦難.
如果我們真的為了神經歷這些苦難.
神會不會放棄我們?.
不理會我們嗎?.
絕對不會吧.
所以當我們經歷苦難的時候.
反而會幫助我們更加親近神.
我相信保羅寫羅馬書第八章的時候.
他可能腦海裡在想著一些人.
是什麼人呢?.
在羅馬的教堂裡.
可能有很多基督徒將要面對國鬥士.
Gladiators.
可能他們將要面對萬壽.
保羅在鼓勵他們.
要堅持他們的信心.
最後不單止保羅說出這一切.
在羅馬書第八章37節.
他還說神已經將德性的能力給了我們.
保羅叫我們做超級的征服者.
Hubert Nicholson這個字的意思就是超級的征服者.
我相信這個字是被Nike用來做他們公司的名字.
但他們只不過是Nike.
不是超級的Nicholson.
他們只不過是征服者.
所以他們的口號是什麼呢?.
Just do it.
征服者你們去征服吧.
保羅告訴我們我們是超級的征服者.
所以無論是生還是死.
無論是現在的事.
無論是將來的事.
沒有東西能使我們和耶穌基督的愛分開.
甚至神賜給我們勝利的力量已經給了我們.
這是無條件的.
神不是說你這樣做我就會這樣做.
是神先愛我們.
不是我們先愛神.
我們愛神是我們的回應.

$^{1041}$但是我們是否能夠完成我們的救贖.
不在乎我們有多愛神.
是在乎神有多愛我們.
我們應該要做的是什麼呢?.
就是我們要接受祂 要相信.
享受神的愛.
這就是我們所應該要做的.
在結束的時候.
我們看看羅馬書第八章.
真是一氣呵成.
一開始經文說在耶穌基督裡就不被定罪.
然後到最後.
沒有東西能使我們和耶穌基督的愛分開.
一氣呵成.
有很多基督徒都想知道.
他們都會問我是否真正的基督徒呢?.
我是否真的得救呢?.
他們都會問這個問題.
但是問問你自己吧.
問問你自己是否真的很憎恨罪.
你是否願意越來越像耶穌基督.
將榮耀歸給神.
你也可以問問你自己.
你是否很愛神.
你是否願意思念神的榮耀.
倚靠祂的能力.
渴望祂的同在.
喜愛祂的公義.
遵守祂的律法.
如果你是的話.
我恭喜你.
你是一個真正得救的人.
你是一個真正被神呼召的人.
Martin Lloyd-Jones.
他曾經說過一句話.
一個沒有生命改變的基督徒.
只不過是一個哲學上的基督徒.
就是我們中國人所說的紙上談兵的基督徒.
他說他們根本沒有真正得到神的真理的信徒.
所以我們都要很小心.

$^{1081}$我們都要真真正正越來越像耶穌基督.
有生命的改變.
將榮耀歸給神.
如果你現在正在苦難當中.
我也要鼓勵你.
萬事互相效力.
要愛神的人得到益處.
願意神給你力量度過你每一天.
使你能夠越來越像基督.
將榮耀歸給祂.
讓我們一起低頭禱告.
你讓我們有羅馬書第八章.
讓我們看到萬事互相效力.
要愛神的人得到益處.
願意我們有愛神的心.
願意思念祂的榮耀.
信靠,倚靠祂的能力.
看望祂的同在.
喜愛祂的公義.
遵守祂的律法.
以致我們能夠越來越像耶穌基督.
將一切的榮耀和頌讚歸給祂.
我們禱告.
奉主耶穌基督的名義而求.
阿們.
\newpage



\section{撒母耳記上使徒行傳 13:14}
\label{sec:w_ajWsBZ9eQ}
\textbf{大衛:最合神心意的人 (撒母耳記上13\_14, 使徒行傳13\_22) - 袁惠鈞牧師[大衛傳系列 - 第1講]}
\newline
\newline
連結: \href{https://youtube.com/watch?v=w-ajWsBZ9eQ}{\texttt{ https://youtube.com/watch?v=w-ajWsBZ9eQ}} ~~~~ 語音日期: 2025-01-10 
\newline
\newline
\hyperref[sec:9ORA5941xxk]{< < < PREV SERMON < < <}
~
\hyperlink{toc}{[返主目錄]}
~
\hyperref[ch:preacher6]{[返講員目錄]}
~
\hyperref[sec:yRzXvTTOZfM]{> > > NEXT SERMON > > >}
\newline
\newline
$^{1}$今天是2025年第一個主日.
我們有一個新的系列.
就是要講大圍轉.
聖經真的非常精彩.
我們剛剛才講完羅馬書第八章.
有很多神學.
但是現在又回來講大圍轉.
有很多很多的故事.
我們不會一口氣全部講大圍轉.
因為很長很長.
我們首先會講薩姆爾常記有關大圍的事蹟.
然後去到復活節的時候.
我們就會轉台.
我有一個復活節的訊息.
講完之後我們就會講羅馬書九至十一章.
然後講完大概在七月的時候.
我們又會回來講薩姆爾夏有關大圍的故事.
今天我們要講的主題就是大圍最合神心意的人.
我們今年的主題.
我們的口號就是怎樣做一個合神心意的人.
我們都需要學習大圍.
今天我要講的經文是由歷代至上的二十二章.
講到二十九章.
哇 你說牧師是新的一年.
你這麼大野心講百章的經文.
我都試試能不能講完.
可能要比較簡化一點.
還有呢.
你說為什麼將大圍新平的總結.
就是歷代至上的二十二至二十九章.
在介紹大圍的時候講.
其實有原因的.
當我們知道大圍的新平.
知道大圍的功績的時候.
我們就能夠更加認識這個人.
聖經裡面有六十二章的經文.
都是有關大圍的事跡.
在聖經裡面除了耶穌基督以外.
最多篇幅的人就是大圍.
大圍真的有很多的功績.

$^{41}$他統一了以色列.
為人民帶來了平安.
他擴張了以色列國的領土.
他還建立了大圍的王朝.
就是神跟他納約.
說他的王朝會直到永遠.
我們知道這個就是透過耶穌基督.
他有一天會回來掌權.
大圍也都是預備了建造聖殿.
一切的材料,人力,物力.
在大圍之後我們知道.
所有猶大王的功績.
都是跟大圍來比較.
你就知道大圍在聖經裡面有多重要.
這些經文跟其他王的比較.
都在《樹列王記》上和《列王記》下.
有記載.
早在《創世記》的49章第一節.
我們就知道神預言.
以色列的王是會出於猶大之派.
大圍就是出於猶大之派.
那為什麼以色列第一位王.
掃羅是出於變雅文之派.
這個我講撒母耳記的時候也說過.
是因為以色列人急不及待.
希望有一位王.
所以神就將掃羅給了他們.
掃羅不是真正神所喜悅的王.
路德記裡面也有大圍的家譜.
是由法勒斯到大圍.
你說法勒斯是誰?.
是猶大和他的媳婦.
他媽戀倫所生出來的.
是一個私生子.
根據猶太人的律法.
私生子十代不能進入聖所.
如果我們計算一下.
大圍就剛剛是第十一代.
所以神早就預備好大圍成為以色列人的王.
大圍的意思 明治的意思是什麼?.

$^{81}$就是被愛的那一位.
大圍真是神所愛的那一位以色列的王.
我們今天的系列.
大圍傳的系列.
我們會由撒母耳上講到撒母耳下.
如果你記得的話.
我們講撒母耳傳的時候.
已經講到撒母耳上的第十五章.
現在我們又要由十六章開始.
就是由大圍的崛起.
撒母耳上的十六章到第三十章.
之後我剛才說.
我們會去看羅馬書九至到第十一章.
然後七月份我們會回來看大圍的事蹟.
撒母耳下的一至十章就是大圍的勝利.
撒母耳下的十章至二十章就是大圍的沒落.
很有趣在聖經裡面.
大圍是三次被安納為以色列人的王.
第一次在撒母耳上的第十六章.
是撒母耳安納他為王.
這個是私底下.
第二次他被安納為猶大王.
在撒母耳下的第二章.
第三次他被安納成為全以色列人的王.
在撒母耳下的第五章.
說到大圍我們立刻想到他和拔士巴的姦情.
但誰贖無過.
每個人都會做錯事.
問題就是我們做錯事之後.
我們怎樣來認罪悔改.
大圍就是一個勇於認罪悔改的人.
所以聖經對他的評價是很高很高.
叫他做一個合神心意的人.
剛才我們讀過經文.
就是在撒母耳上的十三章十四節.
這是舊約.
在新約也有記載.
在史徒行傳的十三章第二十二節.
你問怎樣能夠成為一個合神心意的人.
是不是大圍當他被敵人追趕.

$^{121}$他好像一隻野狗到處奔跑.
但他最後仍然歡恕他的敵人倚靠神.
是不是這樣呢.
還是當他成功的時候得勢的時候.
懂得怎樣來處理他的成功.
還是他被他親生兒子背叛追殺他的時候.
他仍然這麼愛他的兒子.
還是他犯罪之後.
他懂得怎樣認罪悔改.
向神來懺悔.
我相信這一切都是以至他能夠被神稱為合神心意的人.
我相信還有其他很多的事蹟.
能夠幫助我們知道怎樣才是合神心意的人.
在這個系列我們會來看看.
大圍的信心是很大.
以至以色列人的神被稱為大圍的神.
這是很重要.
讓我們看到他是一個很有信心的人.
在署列王記下的20章第5節.
和以塞亞書的38章第5節.
都是這麼說.
神還說大圍是被他揀選來統治他的民.
統治以色列人的.
列王記上的8章第16節.
有很多的詩篇都說大圍就是神的僕人.
在詩篇的78篇70節和89篇的20節都是這麼說.
在使徒行傳的13章36節告訴我們.
大圍不單止遵行了神的旨意.
他還服事了那一世代的人.
一個忠心的僕人.
不單止服事了他自己那一世代的人.
也服事下一代的人.
為什麼我會這麼說.
大圍的功績在幾個世紀仍然是祝福以色列人.
就好像約翰一書的2章17節所說.
唯獨遵行神旨意的是永遠常存.
大圍不單止籌備了聖殿一切所需要的材料,人力,物力.
他還創造了敬拜的歌曲.
他還設計了敬拜的樂器.
更重要的是耶穌基督透過大圍的家來到世上.

$^{161}$耶穌基督在喀什錄的22章16節說.
「我是大圍的根,又是他的後裔」.
其實我今天所說的是歷代至上的22章到29章.
我剛才說有8章的經文.
是很大野心希望能夠說完.
主要我要說的一共有5點.
第一是聖殿的策劃.
第二是聖殿的組織.
第三是軍事的管理.
第四是大圍的奉獻.
第五是大圍與民同樂.
來慶祝神的恩典.
我們首先看看聖殿的策劃.
就是歷代至上的22章.
大圍說他自己居住在宮殿.
神居住在帳棚.
所以他很有心要為神來建聖殿.
但神不准他建聖殿.
因為他殺戮太多.
因為他滿手都是鮮血.
所以神說你不要為我建聖殿.
但你的兒子可以替我建聖殿.
有一句古老的話.
一個好的開始就是成功的一半.
所以大圍幫助所羅門.
預備好所有聖殿的需要.
他買了建聖殿的那塊地.
就是阿奴那的打鞠場.
另外他預備了很多建聖殿的材料.
還有人力物力.
使得所羅門建聖殿的時候能夠事半功倍.
大圍籌備聖殿的資源是第一點.
我們不能確定神何時感動大圍.
幫他建聖殿.
但當他購買阿奴那的打鞠場的時候.
相信是一個很大的信號.
記不記得大圍在阿奴那的打鞠場.
捉了一個壇獻祭給神.
當時從天上降下火.
將他所有的祭物都燒了.

$^{201}$大圍就知道神已經赦免了他的罪.
我這裡所說的罪不是他和拔士巴的罪.
而是他在殺母以下的24章.
他數點士兵的罪.
當他數點士兵不為神所喜悅.
神就降下很大的災害.
但當大圍的獻祭被神閱立的時候.
他知道自己的罪被赦免.
而他繼續在那裡獻祭.
而不是去到機片的回望.
因為他知道那塊地對神來說是很重要.
那塊是什麼地.
是摩尼亞山.
他知道神要將他自己的殿建在摩尼亞山.
你記不記得摩尼亞山在哪裡.
在創世記的22章.
就是阿伯拉罕獻耳撒的地方.
但是建聖殿的地方不是阿伯拉罕獻耳撒的地方.
阿伯拉罕獻耳撒的地方.
是在北面有一個山峰.
在2000年後在同一座山.
耶穌基督為我們釘十字架.
所以我們見到舊約的經文告訴我們.
獻祭是要在北面的地方獻.
就是在聖殿的北面.
完全符合聖經的教導.
大衛可能在這個時候寫詩篇的第三十篇.
儘管當時的聖殿還沒有建在這塊地.
大衛已經將這塊地和將來的聖殿獻給神.
這就是詩篇的第三十篇.
多年來大衛一直在累積建聖殿的材料.
我們現在沒有辦法計算到.
究竟這個聖殿的價值有多大.
大部分這些資源都是來自大衛的戰利品.
就是他打勝敵人的時候.
拿了敵人的財物.
大部分建聖殿的資源都是出自敵人的戰利品.
在聖經裡的舊約告訴我們.
以色列人如何建聖殿的經文.
其實和我們今天建立教會的教導息息相關.

$^{241}$當我們比較哥林多前書的三章9至23節.
或者以弗所書的2章19至22節.
我們看到和今天的經文.
歷代至上的22章28章和29章有很多相似的地方.
大衛知道建造聖殿是要用金銀寶石來建聖殿.
這是記載在歷代至上的22章和29章.
而在《說針嚴》我們知道.
金銀寶石就是用來代表神話語的智慧.
在《說針嚴》的第二章,第三章和第八章都是這樣說.
而保羅在新約這些材料是用來反映出教會屬靈的情況.
保羅說草木禍皆隨手可得.
但如果我們要做到像金銀寶石一樣.
我們一定要下很大的功夫.
我們不是用人的智慧或者模仿這個世界來建立教會.
有很多人都跟我說.
牧師呀你不如用這個辦法吧.
這個辦法使某某公司變成一間大公司.
很實用值得我們去學習.
我時時都會說我不會用俗世的方法來學習.
怎樣建立教會.
建立教會是應該用聖經的教導.
是應該用神的話語.
就像哥林多前書的三章18-20節所教我們.
大衛除了準備了聖殿的材料.
他還準備了建立聖殿的人力.
他招募了很多猶太人和外邦人來建立聖殿.
他還建立了一個部門叫做建立聖殿的部門.
是由亞多蘭來管理.
亞多蘭的人在聖經上的四章六節.
也叫做亞多尼蘭.
當時他揀選了有三萬名猶太工人.
去到尼巴嫩來斬樹.
我們知道尼巴嫩的香柏木是非常名貴.
這些工人用了一個月的時間來斬樹.
他們就能夠回到他們的家鄉休息兩個月.
另外還有十五萬的外邦人.
他們是得到猶太人公投的監督之下.
去到山上挖掘很大的石頭.
然後搬運到耶路撒冷來建立聖殿.
我們千萬不要以為猶太人虎待這些外邦人.

$^{281}$住在以色列的外邦人.
其實在摩西的律法裡面.
有很多的教導都是禁止這些行為.
就是虎待居住在以色列的外邦人.
例如出埃及記的二十二章.
二十三章和利美記的十九章.
外邦人和猶太人一起來建立聖殿.
其實我們看到一幅很美麗的圖畫.
就是一個預言將來這個聖殿.
會是外邦人和猶太人一起敬拜萬國禱告的殿.
就好像以塞瓦書的五十六章第七節所預言.
大衛不單止預備了建立聖殿的人力物力.
他還勸告所羅門.
我們不知道大衛籌備聖殿的時候多少歲.
但是有學者認為大概他應該是六十歲左右.
那所羅門究竟是多少歲呢?.
根據幾節的經文我們就可以知道.
就是歷代至上的二十二章和二十九章.
都說所羅門是年幼脆弱.
在聶王紀上的三章七節說他是幼小的孩子.
這就是原因大衛多次勸告鼓勵所羅門.
要他遵行神的道 要他完成建立聖殿的工作.
大衛也勸告他的手下要幫助所羅門建立聖殿.
大衛在他死之前要預備好一切建立聖殿的事.
以至所羅門建立聖殿的時候就會比較容易.
在我們的印象當中大衛不是一個好的爸爸.
我們在說大衛轉的時候就會看到.
他不是一個好的爸爸.
但是在他教導所羅門的教導裡面.
我們看到他不失是一個慈父.
大衛就鼓勵所羅門一共有三點.
他怎樣鼓勵所羅門呢?.
第一要他執行神的旨意.
歷代至上的二十二章六至十節.
大衛告訴所羅門現在以色列已經得到和平盛世.
這是神給他們的恩典.
這是最好建立聖殿的時候.
大衛鼓勵所羅門要走在神的旨意裡面.
來完成神給他的事工.
第二點大衛鼓勵所羅門要將榮耀歸給神.

$^{321}$大衛重複告訴所羅門.
建立聖殿不是為了大衛的榮耀.
更加不是為了所羅門的榮耀.
一切都是將榮耀歸給耶和華.
這個和塑羅和亞沙隆他們的行為.
他們所表現的很不同.
因為亞沙隆和塑羅.
他們都是建自己的紀念碑來紀念自己.
他們將榮耀歸給自己.
但大衛提醒所羅門要將榮耀歸給神.
第三點大衛勸所羅門不要懼怕不要驚惶.
大衛說只要所羅門服從神.
神一定會保護以色列賜給他們平安和和平.
在歷代至上的二十八章第二十節.
經文說你當剛強壯膽去行.
不要懼怕也不要驚惶.
因為耶和華就是我的神.
與你同在他必不撇下你也不丟棄你.
直到耶和華殿的工作都完不了.
其實這一節的經文令我們想起.
摩西勸約書亞很相似的經文.
摩西臨死的時候說同樣的話.
在《申命記》的三十日章.
其實摩西死了之後.
神也重複鼓勵約書亞在約書亞記的一章六節和九節.
摩西和約書亞都是神中心的僕人.
只要他們是走在神的旨意裡.
神會幫助他們完成他們的任務.
這也是大衛如何教會所羅門.
這也是給我們教會.
只要我們繼續走在神的旨意裡.
神一定會幫助我們完成他的工作.
大衛不單只勸告所羅門.
他也吩咐他的領袖要幫所羅門合作.
繼續建聖建.
大衛提醒這些領袖.
他們所享有的平安.
是因為神賜福給他們.
給他們有平安.
擴展了以色列的領土.

$^{361}$所以作為領袖.
他們一定要立心來侍奉神.
尋求神.
有沒有看到大衛.
他所著重的就是人民和領袖.
他們要關心神的名.
他們要高舉神的名.
這是大衛所著重.
講完聖殿的策劃.
就是聖殿的組織.
大衛知道如果要榮耀神的話.
一定要組織聖殿的人員侍奉神.
很多時候我們都是將注意力放在建築物.
很多時候我們就說.
建新堂了.
就將新堂獻給神.
聽起來好像很屬靈.
但我們有沒有注意到教會屬靈的情況.
很多時候就是一班長輩和會眾.
站在一起.
將新堂獻給神.
但這班長輩和會眾.
可能靈命非常倒退.
但他們關心的是將堂獻給神.
這不是神所喜悅.
大衛知道建聖殿是重要.
聖殿裡的組織更加重要.
所以他揀選了利美人.
祭司.
聖殿裡的歌唱者.
還有聖殿的官員.
一起來侍奉神.
大衛希望在神的家裡.
一切都能夠規規矩矩地.
按照次序侍奉神.
就好像哥林多前書14章40節所說.
大衛在他作出決定的時候.
他和兩個祭司一起來祭籤.
尋求神的旨意.
其實這個方法.

$^{401}$和約書雅在約書雅記裡.
將應許之地分給十二支派很相似.
都是透過祭籤.
有人問.
現在教會你們決定事情.
是不是拿個祭籤筒出來求問神.
祭籤嗎?.
當然不是.
我說過很多次.
大衛和約書雅的時候.
他們沒有全部神的話語在他們手上.
所以他們需要祭籤來尋求神的旨意.
現在我們有所有神的話語在我們手上.
所以當我們要作出決定的時候.
是不是祭籤呢?.
不是.
甚至不是投票.
我再說一次.
不是投票.
很多教會就是投票.
少數服從多數.
這些就是依靠人的智慧.
當教會要決定一件事情的時候.
我們應該一班長輩坐在一起.
來看看聖經究竟怎麼說.
而不是很倉促的來投票.
組織本身不重要.
但是為什麼要有這個組織的目的.
是很重要.
就是要侍奉神.
所以大衛提醒我們.
我們每一個神家裡的僕人.
我們最主要的責任就是侍奉神.
就是透過神的話語來侍奉他.
首先我們看看他簡傳出來的.
就是由利未人在歷代至上的23章.
在聖經裡的教導.
利未人出來侍奉神.
在聖殿裡侍奉神.
是起碼要30歲.

$^{441}$很多的經文都是這樣告訴我們.
但是後來是減到20歲.
可能當時符合30歲標準的人並不多.
負責聖殿的利未人有38000人.
分為四組來侍奉神.
其中24000人在聖殿裡幫助祭司.
此外還有6000人是官長和審判官.
有4000個是守城門的人.
還有4000個是歌手.
在聖殿裡敬拜神.
當時有一座的聖殿.
有一位大祭司.
有一本神的律法書.
有一位需要被侍奉的神.
但是恩賜和侍奉有很多很多.
就好像今天教會一樣.
當時有祭司在壇上侍奉神.
有利未人在聖所服侍主.
和現在一樣.
有牧師在台上講道.
有一群童工來服侍神.
但我們要注意.
不是牧師掌針.
不是童工所做的事最重要.
每一個在教會侍奉的人.
他們所做的職責都是一樣重要.
大衛不單組織了音樂師.
他還設計了敬拜的音樂來敬拜神.
祭司和利未人在聖殿裡做的一切.
侍奉的禮儀不是人作出來的.
是根據神所命定.
記不記得在舊約.
大祭司亞倫有第一第二的兒子.
兩個兒子記不記得.
就是拿達和阿比胡.
當他們用自己的方法來敬拜神的時候.
出現什麼事.
一把火從祭壇出來將他們吞沒.
所以我們在聖殿.
無論是聖殿還是教會侍奉.

$^{481}$都要跟著神的話語.
這是很重要的.
除了利未人.
還有被揀選在聖殿侍奉的就是祭司.
大衛揀選了24個班次的祭司來侍奉神.
透過抽籤決定他們侍奉的時間.
除了在聖殿侍奉這班祭司.
還要教導百姓.
他們不在聖殿侍奉的時候.
他們就去教導百姓.
這個傳統到了撒加利亞的時候.
到了耶穌基督的時候.
仍然都有實行.
撒加利亞就是施洗約翰的父親.
他正在聖殿侍奉神的時候.
天使就跟他展現.
告訴他以利沙白的妻子將會懷孕.
會生施洗約翰出來.
當時撒加利亞就是按著24個班次.
正在聖殿侍奉.
撒加利亞就是屬於亞比亞家族的後人.
記載在歷代志上的24章第十節.
除了利未人和祭司還有詩歌班.
在文躁記的第十章.
吹角的儀式是照著聖經的教導.
除此以外.
舊約沒有說到如何用音樂侍奉神.
我們相信這個都是大衛精心所組織.
他組織了24個班次的詩歌班.
不單有24個班次的祭司.
還有24個班次的詩歌班.
大衛就是一個偉大的詩人.
他作了起碼有73篇詩篇.
這些詩篇都成為我們現在很多的鼓勵.
我們讀詩篇的時候.
見到大衛當年被敵人追趕追殺.
成為我們現在有苦難的時候的鼓勵.
這次我們說大衛的系列.
我們也會說到很多大衛所作的詩歌.
大衛是一個非常有才華的音樂家.

$^{521}$在聖經裡他被稱為以色列的美歌者.
在撒母耳夏的23章一至二節.
昔日有什麼音樂.
有什麼樂器來敬拜神.
有彈琴,鼓瑟,敲筆.
歷代志下的29章25節.
還有我剛才說過的吹號或吹角的敬拜.
大衛善於用人.
他揀選了三個利美人.
來管理樂器和音樂的敬拜.
第一個就是我們熟悉的阿薩.
記不記得詩篇裡有阿薩這個人.
他至少寫了有12篇詩篇.
詩篇的第50篇和73至83篇.
他是敲筆的.
在歷代志上的16章第5節有記載.
還有第二個就是希曼.
希曼是吹角或吹號.
他也被稱為王的先見.
為什麼呢.
因為這個人可能能夠分別神的旨意.
他是有特別的恩賜.
在聖經裡面神有應許他將會有一個大家庭.
我相信希曼的子孫眾多.
而且在他的子孫當中還有很多很好的音樂家.
還有第三個就是耶穌頓.
耶穌頓的名字的意思就是讚美.
這個名字的意思和猶大名字的意思很相似.
這個人是詩歌班的指揮.
就像我們認識的William 弟兄.
就是我們詩歌班的指揮.
這個耶穌頓他也是和詩篇的39篇.
62篇和77篇有連帶的關係.
除了有利未人.
有祭詩有音樂詩.
大衛也揀選了聖殿的官員.
有什麼官員呢.
有守門口的人.
守門口的人不是那麼重要.
但是還有師父.

$^{561}$師父就很重要.
是管理聖殿的財務.
我們今年粵語堂也有一個新的師父.
就是Sherman 姐妹.
她答應了成為我們粵語堂的師父.
還有什麼聖殿的官員呢.
還有一班執行任務的官員.
這班人是做什麼的呢.
他們是提醒以色列人要遵守安息日.
要遵守節期.
他們會派到不同的地方.
甚至去到約旦河的西面.
來進行這些職責.
在歷代至上的26章第32節.
說要辦理神和王一切的事務.
這班人就是利美人祭師.
音樂師和聖殿的官員.
都是在聖殿為神服務.
接著我們要說的就是軍事的管理.
在歷代至上的27章.
以色列人要建聖殿.
他們需要有軍隊來保護他們.
當這個國家多數的藍丁.
都在建聖殿的時候.
突然間敵人攻打.
就會很危險.
所以他們需要有一支很強大的軍隊.
所羅門不像他爸爸一樣.
是一個軍事的天才.
所以大衛需要組織軍隊.
組織首領管理的人員和謀士.
來幫助所羅門.
大衛有多少個軍隊.
有28萬8千的士兵.
由十二師所組成.
十二支派.
每一支都有24000人在軍隊裡面.
一年有12個月.
剛剛就有12支.
所以每一支都負責一個月.

$^{601}$但是當有緊急的情形.
當他們需要打仗的時候.
全部的軍隊都會出動.
十二師每一支都是由大衛的一個勇士來帶領.
而這些勇士的名單.
記載在歷代至上的第十一章.
在這裡也需要稍微解釋一下.
有很多人經常說聖經裡面充滿矛盾.
尤其是數字上的矛盾.
都不一樣的 數字亂七八糟.
我也說過在聖經裡面的確有文士抄寫錯數字.
但是機會不是那麼大.
有人注意到在撒姆以下的二十四章九節.
和歷代至上的二十一章第五節的數字.
士兵人數的數字是有很大的差別.
那你說為什麼呢.
這裡是有解釋的.
因為大衛數點士兵不為神所喜悅.
我剛才說過.
所以神降災給以色列人.
所以約阿都未數完士兵已經神降災.
這件事已經終止了.
所以在撒姆以下的二十四章和歷代至上二十一章.
所公佈士兵的人數是有出入.
就是這樣的原因.
大衛勇士的名單有幾個.
有一個是殺巨人的名單.
在撒姆以下的二十一章.
也有大衛身邊勇士的名單.
在撒姆以下的二十三章.
大衛軍事的戰績.
就是記載在撒姆以下的第八章.
我這次說大衛的事蹟.
有一張書我不會說.
就是撒姆以下的第八章.
因為是說大衛的戰績.
打贏了什麼國家.
拿到什麼領土.
這些事蹟.
沒有什麼屬靈的功課在裡面.

$^{641}$所以我不會說撒姆以下的第八章.
也是唯一一章我不會說的經文.
除了軍事的管理.
也有產業的管理.
在蘇俄統治的期間.
似乎是有些收稅的結構.
記載在撒姆以上的十七章二十五節.
去到大衛的時候.
好像完全沒有提到收稅.
但我相信是有的.
只不過是沒有記載.
但去到所羅門的時候.
很多的經文都告訴我們.
是有很重的稅.
以至民不聊生.
大衛是擁有很多的農場.
有果園 有牛群和羊群.
這些都成為滿足王宮裡面一切的需要.
大衛也有倉庫可以儲蓄很多農作物.
大衛我們相信也是一個比所羅門更加接近的領袖.
為什麼我會這樣說.
起碼他沒有所羅門那麼多的妃嬪.
你要後宮三千那麼多的妃嬪.
你以為是小事嗎.
需要很多的資源.
所以任何的資源財物去到大衛.
我相信是能夠發揮更大的作用.
還有就是除了軍事產業.
還有王的謀事.
謀事你千萬不要看輕.
是很重要的.
一個王是需要有一個謀事在他的旁邊.
來監察他.
看他作出來的決定.
他的動機是否正確.
所以是很需要有好的謀事.
給我們看到其中一個例子就是拿丹.
拿丹才知道當大衛犯罪的時候.
他就去勸告他.
所以王身邊的謀事是很重要.

$^{681}$你說還有什麼其他的謀事.
其中一個就是約拿丹.
他是大衛的叔叔.
他是一個非常好的謀事.
另外有個叫做耶傑.
他是皇宮中的眾王子的老師.
還有一個叫做阿希多佛.
阿希多佛是大衛的好朋友.
也是一個名智的顧問.
阿希多佛是誰呢?.
是拔士巴的爸爸.
我們現在就知道.
這個阿希多佛和大衛的關係非常特別.
雖然是他的好朋友.
但我相信就是因為拔士巴的事件.
令到他和大衛的關係不好.
因為我相信阿希多佛.
氣大衛使他的女兒夢想了淫婦的名字.
所以當亞沙隆反叛大衛的時候.
阿希多佛就投靠亞沙隆.
但亞沙隆沒有聽阿希多佛的勸告.
反而聽了一個臥底.
叫做霧西的勸告.
以致他失敗.
而阿希多佛最後就自殺死了.
跟著阿希多佛的就是耶和耶大.
耶和耶大就是皇家侍衛長比那雅的兒子.
還有就是亞比亞他.
他是一個大祭司.
等下你就會看到.
亞比亞他也是投靠錯人.
以致被革職.
還有一個在書經文裡面提到的就是約阿.
你說約阿是元帥.
為何會在謀士的名單裡面有約阿.
大衛和約阿的關係也是很特別.
約阿有時就好像很忠大衛.
但有時對大衛的利益好像完全不放在心上.
我相信大衛時時跟他聊天的緣故.
就是想知道他的心究竟在想什麼.

$^{721}$第四點就是大衛的奉獻.
大衛知道有最好的人選.
有最好的籌備組織.
有人力物力都不能代替神的作為.
大衛有一天是會離開的.
所羅門是沒有經驗.
所以在起聖殿一個這麼艱難的任務.
大衛是要做好所有的準備.
而最重要的就是得到神的祝福.
所以他願意獻祭給神.
這個很重要.
也是很需要我們學習.
不單他獻祭給神.
他也挑戰他的領袖來服從神.
他召集了他的首領.
告訴他們是神高納他自己.
現在是神揀選所羅門成為他的繼承人.
他再一次提到神和他自己納的約.
所以他說如果他的首領繼續遵守著神的約.
繼續走在神的旨意裡.
神一定會賜給他們.
在這塊地上仍然有太平盛世.
大衛也再一次勸告所羅門.
為什麼這麼奇怪.
剛才我們在22章已經說了.
大衛勸告所羅門.
現在到了28章又一次勸告所羅門.
其實我相信在歷代至上有一個結構.
所以看到22章和28章.
都是說大衛勸告所羅門.
他勸他什麼呢.
他首先要誠心樂意來侍奉神.
什麼叫誠心樂意.
就是有個專一的心.
這正是所羅門日後所犯的問題.
到日後所羅門能不能有專一的心.
不能再有專一的心.
為什麼.
因為他有很多妻妾都是拜偶像.
以至使他離棄神.

$^{761}$大衛一早就勸他要誠心樂意來侍奉神.
還有大衛勸他要剛強壯膽.
這個其實我們剛才已經說過.
就是和摩西勸約書亞在新明紀31章很相似.
大衛於是獻上他的禮物.
我們看到一共有三點.
有什麼禮物.
第一就是記載在歷代至上的28章.
11至19節.
我相信就是聖殿的藍圖.
建聖殿的計劃.
第二就是建造聖殿的人選.
這些人才物力.
記載在歷代至上的28章21節.
還有第三就是獻出所有的財富.
包括他自己的財富.
他自己有多少財富.
110噸的黃金.
260噸的白銀.
國家的財富有3860噸的黃金.
和37760噸的白銀.
有沒有看到大衛自己的財富和國家的財富比起來.
微不足道.
這個就是一個屬靈的領袖應有的質素.
現在我們看看很多貪官在大陸在中國.
他們的財富真的不得了.
嚇死人了.
一個地方官微不足道的.
都說幾百億的身家.
但是你看看一個一國之君.
大衛他自己的財富比起國家真是微不足道.
大衛還呼籲他所有的領袖都要慷慨解囊.
來獻出他們的財富.
這是大衛禱告的開始.
他知道是神祝福他.
他讚美神稱他為尊大的有能力的有榮耀的強盛的威嚴的.
他知道神是有一切的主權.
他將一切的榮耀都歸給神.
大衛願意將最貴重的禮物獻給神.
原因是什麼呢.

$^{801}$他知道一切都是來自神.
這也是我們很需要學習的.
很多時候我們認為.
我們的錢財我們的物業就是我們自己的.
是我們賺回來的.
但這一切都是神給你的.
所以我們都很需要學習.
昔日在舊約那些人民將自己所有的獻出來.
你不需要像大衛那樣將所有的獻出來.
但起碼你都需要獻一部分歸還給神.
然後大衛與民同樂一起慶祝.
這就是最後的第五點.
他獻了有兩種祭.
第一個是繁祭.
第二個是平安祭.
繁祭的意思是什麼呢.
繁祭就是將全部所有的獻給神.
但平安祭你可以將平安祭的一部分拿出來.
大家一起來享受.
這就是平安祭與繁祭不同.
繁祭是要全部燒給神.
所以當時大衛與人民一起來分享.
一起來慶祝.
也在這個時候所羅門被加冕.
所有在場的人他們都同意神揀選所羅門.
如果他們不同意的話.
不同意他們有同一個領袖.
所羅門不能帶領他們來建聖殿.
在此同時舉行的就是薩督被任命為拜祭司.
這件事看起來微不足道.
但很重要.
應驗了有兩個聖經的預言.
你說哪兩個聖經的預言呢.
首先薩督就是亞倫第三個兒子.
以利亞薩的後人.
他代替的亞比亞他.
剛剛我說過亞比亞他被廢除.
因為他是大衛反叛所羅門.
他支持亞多尼亞.
以致失敗失去了他大祭司的職位.

$^{841}$他是誰呢.
他就是亞倫第四個兒子.
以他瑪的後人.
所以這裡我們看到應驗了第一個聖經的預言.
就是以利亞薩的後人將會成為大祭司.
這是記載在民數記的二十五章第十三節.
記不記得當時那些摩訶女子引誘以色列人來行刑.
以利亞薩的兒子.
斐尼哈就是一槍將兩個正在行刑的以色列人.
和摩訶女子刺死.
所以神說以利亞薩的後人一定會成為大祭司.
還有一個聖經的應驗.
就是對以利加的審判.
記不記得在撒母耳上的第二章.
神曾經審判以利加族.
亞比亞他就是以利的後人.
所以他失去了職責.
這裡就是撒母耳上二章三十三至三十五節.
和列王記上二章二十七節都有提及.
所以神的話語真的不簡單.
神的預言都是會實驗.
今天講完了八章的經文.
歷代至上的二十二章至二十九章.
在結束的時候我們不妨來看一下.
我們沒辦法像大衛一樣.
給神起聖殿.
沒辦法做到這些.
但是有什麼我們可以做到的呢.
我們一起來看一下.
學大衛做一個合神心意的人.
第一就是我們看到大衛真的有愛神的心.
他說我住在皇宮.
神就住在帳篷.
我要為神來起聖殿.
他在歷代至上的十六章七至八節.
和歷代至上的二十九章十至二十節.
他都是以詩歌來讚美耶和華.
在詩篇的第九篇.
第八十六篇.
第一百篇和一百三十八篇.

$^{881}$所以我們看到大衛真的是一個很愛神的人.
我們需要學習大衛有愛神的心.
第二點我們需要服從神.
當大衛說要起聖殿給神.
神說你不可以起聖殿給我.
大衛就發脾氣.
不是.
他就為所羅門預備所有起聖殿的東西.
他將這個責任交給所羅門.
他是絕對服從神.
在詩篇的四十篇第八節.
大衛說要照神的旨意而行.
在詩篇的一百十九篇.
大衛遵行神的道.
遵行神的命令.
思想神的律例.
還有他也是一個倚靠神的人.
在詩篇的三十四篇第四節.
他是尋求耶和華.
在詩篇的第三十七篇第五節.
他是交託耶和華並倚靠他.
在詩篇裡面.
大衛屢次提到神就是他的避難所.
就是他的盆石.
還有第四點.
他是向神坦白.
否開心在神的面前.
這個真的很需要我們學習.
在詩篇的二十六篇第二節.
耶和華求你察看我.
試驗我.
熬煉我的肺腑心腸.
在詩篇的一百三十九篇.
二十三至二十四節.
他說神求你監察我.
知道我的心思.
思念我.
知道我的意念.
看在我裡面有什麼罪恨沒有.
我們夠不夠膽去到神的面前.

$^{921}$否開我們的心給神看.
你看看我裡面有沒有邪惡的思念.
有沒有任何惡行.
我們夠不夠膽這樣做.
我相信我們不是很夠膽這樣做.
另外還有第五.
好像大衛一樣有悔改的心.
他悔改不是因為他的罪被發現.
而是他一直有一個知罪的心.
他為自己的罪來優秀.
在詩篇的第五十一篇.
當他和拔士巴犯了奸淫之後.
他就衷心向神悔改認罪.
在詩篇的一百三十一篇第一節.
也說他有謙卑的心.
最後大衛有禱告的心.
在聖經裡沒有一個人像大衛那麼多禱告.
他就是在聖經裡最多禱告的人.
我們都需要學習大衛做一個不著禱告的人.
在新的一年希望我們都能夠學習.
怎樣做一個合神心意的人.
來愛神服從神信靠他.
在他面前坦白有悔改的心.
有向他禱告的心.
將一切榮耀贈給他.
讓我們低頭禱告.
在三無以上和三無以下.
願意我們每一個人在新的一年都能夠學習大衛.
以致我們能夠愛你 服從你.
信靠你 向你坦白.
有一個悔改和禱告的心.
並將一切的榮耀贈給你.
我們這樣禱告.
奉主耶穌基督的名義求.
阿們.
\newpage



\chapter{許樹源}\label{ch:preacher7}
\begin{multicols}{3}
\minitoc
\end{multicols}
{ \scriptsize


\begin{xltabular}{\textwidth}{|p{0.15\textwidth} p{0.6\textwidth}|p{0.07\textwidth} p{0.1\textwidth}|}
\hline
馬太福音 11:2-19 & \hyperref[sec:yRzXvTTOZfM]{當公義被踐踏時  (馬太福音11\_2-19;14\_1-14) - 許樹源教授【繁簡字幕 by Johnson Ng】} & 2025-02-10 & \href{https://youtube.com/watch?v=yRzXvTTOZfM}{\texttt{ yRzXvTTOZfM}} \\
\hline
\end{xltabular}
}
\newpage



\section{馬太福音 11:2-19}
\label{sec:yRzXvTTOZfM}
\textbf{當公義被踐踏時  (馬太福音11\_2-19;14\_1-14) - 許樹源教授【繁簡字幕 by Johnson Ng】}
\newline
\newline
連結: \href{https://youtube.com/watch?v=yRzXvTTOZfM}{\texttt{ https://youtube.com/watch?v=yRzXvTTOZfM}} ~~~~ 語音日期: 2025-02-10 
\newline
\newline
\hyperref[sec:w_ajWsBZ9eQ]{< < < PREV SERMON < < <}
~
\hyperlink{toc}{[返主目錄]}
~
\hyperref[ch:preacher7]{[返講員目錄]}
~
\hyperref[sec:srCkvhUNl9w]{> > > NEXT SERMON > > >}
\newline
\newline
馬太福音 11:2-19
\newline
\begin{longtable}{cl}
\hline
\hline
章節 & 經文 (和合本修訂版)\\
\hline
11:2 & \begin{tabularx}{0.7\textwidth}{X} 約翰在監獄裡聽見基督所做的事,就派他的門徒去, \end{tabularx} \\ \\ \relax
11:3 & \begin{tabularx}{0.7\textwidth}{X} 問耶穌:「將要來的那位就是你嗎?還是我們要等候另一位呢?」 \end{tabularx} \\ \\ \relax
11:4 & \begin{tabularx}{0.7\textwidth}{X} 耶穌回答他們:「你們去,把所聽見、所看見的告訴約翰: \end{tabularx} \\ \\ \relax
11:5 & \begin{tabularx}{0.7\textwidth}{X} 就是盲人看見,瘸子行走,痲瘋病人得潔淨,聾子聽見,死人復活,窮人聽到福音。 \end{tabularx} \\ \\ \relax
11:6 & \begin{tabularx}{0.7\textwidth}{X} 凡不因我跌倒的有福了!」 \end{tabularx} \\ \\ \relax
11:7 & \begin{tabularx}{0.7\textwidth}{X} 他們一走,耶穌就對眾人談到約翰,說:「你們從前到曠野去,是要看甚麼呢?看風吹動的蘆葦嗎? \end{tabularx} \\ \\ \relax
11:8 & \begin{tabularx}{0.7\textwidth}{X} 你們出去到底是要看甚麼?看穿細軟衣服的人嗎?那穿細軟衣服的人是在王宮裡。 \end{tabularx} \\ \\ \relax
11:9 & \begin{tabularx}{0.7\textwidth}{X} 你們出去究竟是要看甚麼?是先知嗎?是的,我告訴你們,他比先知大多了。 \end{tabularx} \\ \\ \relax
11:10 & \begin{tabularx}{0.7\textwidth}{X} 這個人就是經上所說的:『看哪,我要差遣我的使者在你面前,他要在你前面為你預備道路。』 \end{tabularx} \\ \\ \relax
11:11 & \begin{tabularx}{0.7\textwidth}{X} 我實在告訴你們,凡女子所生的,沒有一個比施洗約翰大;但在天國裡,最小的比他還大。 \end{tabularx} \\ \\ \relax
11:12 & \begin{tabularx}{0.7\textwidth}{X} 從施洗約翰的日子到今天,天國受到強烈的攻擊,強者奪取它。 \end{tabularx} \\ \\ \relax
11:13 & \begin{tabularx}{0.7\textwidth}{X} 眾先知和律法,直到約翰為止,都說了預言。 \end{tabularx} \\ \\ \relax
11:14 & \begin{tabularx}{0.7\textwidth}{X} 如果你們願意接受,這人就是那要來的以利亞。 \end{tabularx} \\ \\ \relax
11:15 & \begin{tabularx}{0.7\textwidth}{X} 有耳的,就應當聽! \end{tabularx} \\ \\ \relax
11:16 & \begin{tabularx}{0.7\textwidth}{X} 「我該用甚麼來比這世代呢?這正像孩童坐在街市上向同伴呼喊: \end{tabularx} \\ \\ \relax
11:17 & \begin{tabularx}{0.7\textwidth}{X} 『我們為你們吹笛,你們不跳舞;我們唱哀歌,你們不捶胸。』 \end{tabularx} \\ \\ \relax
11:18 & \begin{tabularx}{0.7\textwidth}{X} 約翰來了,既不吃也不喝,人們就說他是被鬼附的; \end{tabularx} \\ \\ \relax
11:19 & \begin{tabularx}{0.7\textwidth}{X} 人子來了,也吃也喝,他們又說這人貪食好酒,是稅吏和罪人的朋友。而智慧是由它的果子來證實的。」 \end{tabularx} \\ \\
[1ex]
\hline
\hline
\end{longtable}
$^{1}$各位早晨.
後面聽不聽到麥克風.
我讀出兩段經文.
今日想和大家分享的題目.
是和大家有關的.
如果宣揚神的話.
會引致殺身之禍.
不單止是牧師傳道人的問題.
而是每一個基督徒傳福音的時候.
都會面對的一個危機.
所以我今天想和施洗約翰被殺這件事.
去看看這件事公義在不在.
和耶穌基督如何面對這麼不公義的事.
今日我們選擇的主題.
當社會上如果我們留意到很多人會覺得.
如果指鹿為馬的事發生.
或者受害人被告甚至被判罪.
我們就會問一個問題.
公義何在?.
這些不是今時今日的事.
歷史上經常發生.
其實當亞當夏娃離開了上帝.
離開了聖潔公義的本體.
整個人類就落在一個不公義不聖潔的環境裡.
所以我們犯罪得罪上帝.
就是我們在不聖潔不公義的環境的生活表現.
今日我特別是講公義這個題目.
一般我們講公義的時候.
我們會想就是一個社會的公義.
或者我們首先看看聖經裡面的公義的意思.
在耶穌基督登山補訓裡.
有兩段經文是講福.
第四段是愛慕公義如飢如渴的人.
因為貪玩必得飽足.
第八段是為異而受迫害的人.
天國是屬於他們的.
亦都在登山裡要求所有跟從他的人.
先求神的國神的義.
首先我解釋聖經裡面的義和公義是甚麼意思.
其實聖潔公義就是神的本體.

$^{41}$神本身就是聖潔公義的本體.
在聖經裡面講到的義.
根據斯托德牧師的解釋有三個層面.
第一個層面是最重要的層面就是法理上的義.
法理上的義就是指人已經得罪了上帝.
要被稱義即被當為無罪.
唯一的方法就是與神和好.
即認罪悔改與神和好.
這部分我們稱為恩順清義.
是清義的過程.
清義的過程與人的努力沒有關係.
完全是神的恩典.
不是我們做任何事都可以賺取到神的救恩.
當我們無罪是神接住聖靈.
三位一體的神感動我們.
叫我們知罪悔改接受耶穌基督.
我們就被當為義人.
我們不是義人.
但因為耶穌為我們付了罪的代價.
我們就被當為義人.
所以叫恩順清義.
與我們人的功勞無關.
當一個人被聖靈重生之後.
聖靈就會藉著聖經引導我們去到誠信的階段.
這就是第二部分道德上的義.
我們看聖經學習神的話.
就是得救之後去學習聖潔.
我們不完全.
我們死去一天都要繼續走這條誠信的路.
但為什麼我們要走呢?.
因為神的話語不是給不得救的人.
因為不得救的人不能遵守神的律法.
得救的人為什麼要守律法呢?.
就是我們要學習聖潔公義.
很叫我們每天去面對聖潔公義的上帝.
不是功勞.
我們守律法不是功勞.
我們學習聖潔公義.
很叫我們每天面對面見聖潔的上帝.
所以頭兩部分就看到.

$^{81}$清義和誠信就是整個救恩的意思.
所以我們講很多術語救恩.
救恩就是被神清義.
是神的恩典.
到誠信是聖靈的帶領.
我們要努力.
我們有份參與的.
要努力去學習聖潔的生活.
學習聖潔的生活不是功勞.
是應份的.
本應如此.
當我們心思意念改變的時候.
我們就會由愛上帝開始表達出來.
在生活裡面.
愛鄰舍.
變成社會上的公義.
我們要去維護社會的公平.
照顧弱小社群等等.
這個就是聖潔裡面講義.
由裡面愛神開始.
變成愛鄰舍表達出來.
這個就是整個聖經裡面講義的三個層次.
今天說出了出生在約翰被殺這件事.
不單止在聖經裡面有講.
在歷史書裡面也有講.
在第一世紀的猶太歷史學家約瑟夫的書裡有記載.
不單止記載出生在約翰被殺.
當時他寫到約翰被殺是因為小希律王擔心很多人跟隨他會作亂.
在萌芽的時候就要先捉住他.
讓他不能有那麼多人跟隨他.
甚至他講得很仔細.
連希羅底的女兒的名字Salamie的名字都講出來.
很仔細.
當然在聖經裡面是更加詳細去解釋.
現在土耳其的考古學家在約旦的馬卡斯路的地方.
如果大家看地圖.
死海的右邊紅色的位置.
可以看到死海的左上角的位置.
土耳其的考古學家就做了這個模型出來.
在右下角那邊.

$^{121}$我們相信死海被囚在皇宮的地牢監獄裡面.
這個監獄不是一個正常的監獄.
是判了罪之後關人的地方.
因為大家留意死海被捕不是經審訊.
他被捕是因為他宣告神的律法.
告訴小希律王你娶你太太是犯法的.
他是宣告神的律法.
作為先知做他應有的事.
宣告神的說話.
被捕了.
因為希律和希羅底要捉住他.
禁聲滅聲.
讓他不能再自出他們的錯誤.
如果我們看回今天的經文.
我想用一個圖表去講前後的背景.
由施洗約翰到約旦河和人施洗傳福音.
為耶穌受洗到他被殺.
如果我們看回跟隨耶穌的圖表.
大約是兩年時間.
由馬太福音第三章到第十四章.
大約是兩年時間.
但施洗約翰何時被希律拘捕.
我們不知道.
所以我們只知道他在兩年之內.
他坐了多久牢我們不知道.
一般神學研究估計他可能六個月至少兩年.
不是短的時間.
不是幾天的時間.
這件事其實三個福音書都有提到.
從第三章開始施洗約翰傳福音.
到第十一章他派門徒去見耶穌.
因為他被捉後只能聽耶穌的事跡.
看不到耶穌實際所做的事.
到第十四章他被殺.
我們今天看十一章和第十四章.
施洗約翰的死在馬克福音和路加福音都有記載.
而耶穌的回應三個福音書都有記載.
特別是路加的醫生比較仔細.
他告訴我們耶穌聽到施洗約翰死後.
他獨自離開.

$^{161}$他要安靜.
可能祈禱可能悲傷.
然後民眾發現了他跟隨他.
耶穌憐憫他們.
跟他們醫病傳福音.
他們沒有東西吃.
就做五頂耳瑜的神蹟餵飽他們.
到約翰福音再加多些資料.
他餵飽他們後那些人想逼他做皇帝.
所以他叫門徒先離開.
然後他就在海上跟門徒會合.
他不想被人推他做皇帝.
這不是他們的目的.
這就是整個故事前後的圖表.
為什麼希律要捉住施洗約翰呢?.
這個希律不是大希律王.
當耶穌出世的時候.
大希律王就要下令殺死那些小的嬰兒.
這個希律安提伯斯就是他的兒子.
這個故事就是安提伯斯有一個兄弟叫做肥力.
這個肥力娶了一個叫希羅底的女士.
希羅底是什麼人呢?.
就是大希律王的孫女.
這個希羅底嫁了一個叔叔叫肥力.
跟著跟他離婚.
嫁了第二個叔叔就是現在的希律王.
很混亂吧.
所以作為神的先知.
施洗約翰按照他的職份.
就要跟管轄猶太人地方.
應該要遵守猶太律法的王.
宣告你做這件事是不合法的.
因為他宣告神的話.
而引致殺身之禍.
就是今天的故事內容.
今天我和大家思想四個重點.
第一個是為什麼施洗約翰要派門徒去問耶穌是否彌賽亞呢?.
因為我看到很多解經或一些人的講道.
有很多的推敲.
有些人估計他會不會長時間單獨囚禁.

$^{201}$開始抑鬱.
開始沒有信心.
很多這樣的講道.
有些人說會不會開始信心動搖.
我覺得這些都是很多的推敲.
仔細看聖經.
聖經有答案給我們了解多一點.
今天我會從聖經裡嘗試理解.
為什麼施洗約翰會問耶穌.
你是否彌賽亞.
需不需要等第二個呢?.
第二個重點就是.
當施洗約翰整件事如此不公義.
公義何在呢?.
第三個問題就是.
為什麼耶穌聽到施洗約翰被囚禁時不去救他呢?.
最後就是.
如何從施洗約翰的死和耶穌的回應.
去明白神的心意.
和我們如何裝備好自己.
我們首先看施洗約翰的事跡.
他在約翰篤音第一章開始.
說到他見到耶穌.
他說看啊!神的羔羊.
除去世人的罪.
他很清晰知道彌賽亞的目的.
是羔羊為人贖罪.
他說我看見聖靈.
彷彿甲子從天上下降.
停留在耶穌基督的身上.
我先前不認識他.
但是我猜他來用水施洗的.
即天父和父神.
跟他說.
你看見聖靈降在那個人身上.
那個人就是會用聖靈施洗的.
而我看見了.
所以我作證.
他是一個神的兒子.
大家留意一下我括著的字.

$^{241}$施洗約翰對他的使命是很認真.
很審慎的.
他看見看見看見.
留意到他要很清楚知道.
他才推論耶穌是神的羔羊.
神的兒子.
他不是聽出來的.
他真是有認證到.
他見到聖靈.
好像甲子降臨.
他才認定耶穌基督是神的兒子.
他不是亂作出來的.
另外你留意到.
第31至33節.
他說我先前不認識他.
他們本來是表兄弟.
但他們不是一起長大的.
留意他過的生活是在曠野.
吃黃蟲野物.
神學家推敲.
他應該是從小就被帶去曠野修道院受訓練.
所以他過的生活不是耶穌基督在城市裡的生活.
他很清楚說我不認識這個人.
因為他們不是一起長大的.
他過的另一個生活.
但他是按著神的印證.
天父直接跟他說聖靈降臨在誰那裡.
那個就是彌賽亞.
他是完全聽到天父的父神的吩咐.
按著他看見的才認定這個人.
為什麼他會開始坐牢的時候.
去問耶穌是不是他呢.
我們看回麻煩福音第十一章.
因為他已經單獨囚禁.
他唯一的渠道就是聽見耶穌所說的話.
因為他看不到.
他沒有機會親身跟著耶穌.
看耶穌做的事.
所以他很自然問這個問題.
他聽見了.

$^{281}$耶穌就回答他一個很短的答案.
第五節第六節.
一會兒我會詳細看第五節第六節.
因為我們要重溫舊約聖經.
才明白耶穌想說什麼.
我們聽見之後.
到底聽見什麼才會問耶穌這個問題呢.
我們看回聽見那件事.
在馬太福音第十一章頭那部分.
沒有提到聽見什麼事.
有些人在推敲.
後來我發現.
我們看到第十八節十九節.
原來耶穌基督給了我們一個答案.
耶穌基督說完書仔約翰.
是婦人所生最大的那個之後.
跟著他說了一段話.
我們吹笛.
那些人不跳舞.
我們唱哀歌那些人又不去悲傷.
跟著他就說.
約翰來了.
他也不吃不喝.
人就說他是被鬼附的.
人子來了.
也吃也喝.
留意人又說.
他是貪食好酒的.
是瑞利和罪人的朋友.
這句話就是他聽見的那些.
就不用推敲猜他聽見了什麼.
因為耶穌基督原來有說出來.
原來當時那些人.
看到耶穌和罪人一起吃飯.
就說耶穌是貪食好酒的人.
是瑞利和罪人的朋友.
所以當那些約翰的門徒.
回到監獄裡.
告訴施一郎.
那些人說你這個彌賽亞.

$^{321}$是貪食好酒的人.
所以他就問耶穌基督這個問題.
因為他看不到.
他只是聽別人傳給他的消息.
跟他看見的是不同的.
他看見認定的是他親眼見證的.
但他聽的和他想的不一致.
所以他問這個問題.
阿偉要來的是不是你呢.
我想大家想想一個問題.
如果你為耶穌準備.
耶穌的使命.
你又被人抓去坐牢.
你第一個訊息.
想送給耶穌.
叫他問的問題是什麼呢.
我想可能是耶穌找個律師來救我.
我是不經審訊被捕入獄.
出了約翰沒有.
所以有些人說他的信心動搖.
其實不是的.
他是對他的使命很認真.
而他聽到的訊息.
跟他認為的使命是不同的.
他沒有要求自己的安全.
他沒有要求耶穌叫人去示威.
拯救他出來.
他想認定我一生天賦給我的使命.
我是否做得到.
是否有搞錯.
他很審慎去看他的使命.
他沒有問任何要求自己的好處.
所以我們很多時候聽到.
他信心又動搖.
其實很不一致.
你聽到你明白耶穌所聽到的事.
在一日的房間聽到耶穌基督所做的事.
而耶穌基督說了出來.
就是那些人說他貪食好酒.
是稅理罪人的朋友.

$^{361}$他聽到這個傳言.
所以他再問耶穌.
是不是你.
所以我們留意到.
他是很忠心.
對神的使命.
他在監獄裡面.
看重的不是自己人生的安全.
不是人生的自由.
是上帝的使命.
是否有忠心去完成.
這才是他問這個問題的原因.
如果我們明白到.
就不會誤解到.
他有抑鬱的推敲.
但是.
斯德約翰也有限制.
除了他聽到那些傳言.
影響了他要發文之外.
他對舊約的認知.
也有一個限制.
這不是他的問題.
當時他還不知道我們新約.
所以在天國裡面.
最少的一個也比他強.
就是因為我們知道耶穌死而復活的事跡.
和耶穌有兩次降臨的事跡.
當時他不知道.
所以如果我們看回.
耶穌基督的答案.
他說盲人看見.
去死行走.
病人得到醫治.
聾人聽見死人復活.
窮人聽有福音.
這些經文在.
以賽亞書有三段提及過.
我們看回那三段經文.
耶穌回答得很短.
我們看回那三段經文.

$^{401}$這是在以賽亞書第二段.
第一段是29章.
然後是35章.
然後是61章.
我們看回這段.
35章.
這段是神.
藉著以賽亞書.
和神的子民說.
你們軟弱的人要剛強.
不要膽怯.
你們的神會為你們報仇.
將來會施行.
極大的報應.
也會來拯救你們.
然後就說.
那時候核子會開眼.
聾子會聽到.
核子會走路.
阿爸會說話.
將來他們會帶他一條聖路.
專給被救贖的子民走.
留意到.
說到核子.
聾子等.
這些事的囚.
有兩個元素.
一個是有審判.
一個是有拯救.
所以對施洗約翰的認知.
將來來到賽亞地時.
他會施行拯救.
也會進行審判.
所以.
這個認知也是.
藉著施洗約翰的說道.
表達他的認知.
在《馬賽福音》第三章.
說到我是用水和你們施洗.
叫你們悔改.

$^{441}$但我之後來的.
比我能力更大的.
我連和他搭鞋都不配的.
他用聖靈和火和你們施洗.
而這個人.
會將墨紙收在倉裡.
把罡用不滅的火.
燒聚.
他的理解也知道.
這個尼塞瓦尼會有拯救.
會有審判.
如果留意耶穌基督.
那裡.
他只說了醫治.
拯救的部分.
沒有提到審判的部分.
所以施洗約翰可能會有一個疑問.
為什麼還沒有開始審判.
如果我們再看.
其餘兩段.
比較耶穌的答案.
如果在《以賽亞書》第61章提到.
耶和華的靈在我身上.
在救主身上.
耶和華用高高我.
叫我報好信息給窮人.
窮人有福音聽到有救.
醫好傷心的人.
有盲人可以見到.
騎子可以走路.
但要留意耶穌沒有說.
被擄的不釋放.
被捆綁的不自由.
耶穌基督沒有說.
免得施洗約翰有誤解.
似乎耶穌基督沒有說這句話.
但我們再看.
《以賽亞書》第29章.
我們就明白.
耶穌基督.

$^{481}$醫病.
趕鬼行神蹟.
的目的.
很多時候我們祈禱.
我們有病求神醫治.
就停在這裡.
這不是目的.
看《以賽亞書》第40章.
看《以賽亞書》第29章第18節.
那時聾子必聽見.
聾的可以聽得到.
聽見什麼.
是這書上的話.
是神的話.
盲人可以見到.
見到什麼.
第19節要說.
以色列聖者.
原來上帝做醫治.
叫盲人可以開眼.
聾子聽見.
治了你的病.
要你去認識上帝.
救你的靈魂.
身體會死.
上帝耶穌基督.
來的目的是拯救靈魂.
因為靈魂是永恆.
肉身是會過去.
所以你留意到.
第29章.
是第一次提及.
聾的聽見.
盲的可以開眼.
目的不是治好你的病.
是看到上帝明白神的話.
才是目的.
如果我們有.
親人有病.
為他健康祈禱.

$^{521}$我們要繼續站在一起.
不要神求你醫治.
求你醫治.
好叫這位朋友.
明白你的愛.
得而不拯救.
但你留意到.
耶穌基督在答案.
20節21節.
殘暴的人的結局.
是沒有提及的.
因為對施洗日行來說.
他不明白.
拯救和審判.
耶穌是經兩次去做.
第一次.
耶穌降臨是拯救.
第二次來才是審判.
所以當施洗日行說.
他知道有審判有拯救.
他知道耶穌基督會將墨紙修好.
將沒有用的炕燒.
就是審判.
而耶穌基督在.
《馬化福音》.
庇子的庇譽解釋了.
當時庇子的庇譽提到.
主人在田裡.
殺了好種子.
但晚上敵人.
殺了壞的種子.
一起生長.
毒人問主人.
可否動手清除.
壞的種子.
主人說不要.
免得連好的都受影響.
神是容許.
好的不好的一切.
在《馬化福音》.

$^{561}$提到人子將來.
和天使再回來.
那次是第二次.
那次才分開綿羊山羊.
才審判.
所以我們新約時代.
恩典時代的門徒.
天國最小的那個.
我們都知道原來.
有前後的分別.
但施洗日行當時不知道.
所以有這個誤解.
所以我們明白.
耶穌基督說的.
你去到曠野.
見誰呢.
他就說.
這就是先知.
先知是最大的.
為什麼最大的.
因為所有先知都是預言.
救主來到.
但施洗日行親眼見到救主.
施洗日行有兩本.
舊約的書.
預言.
這個人會降生.
聖經沒有預言.
只有施洗日行.
神的聲音在曠野.
為救主預備.
所以在.
女子所生.
在舊約.
律法時代.
沒有一個比施洗日行.
更加大.
因為她親眼見到.
創世以來.
沒有預言救主.

$^{601}$親自為她施洗.
但天國裡.
最少的像我們這樣.
我們知道很多.
的救恩.
例如耶穌基督.
會死而復活.
會升天.
聖靈降臨.
耶穌基督第二次來.
才審判.
所以施洗日行在監獄裡.
問這些問題.
沒有審慎地看.
神交託給他的使命.
我們看第二個重點.
施洗日行.
被無辜殺害的時候.
神的真理和公義.
在哪裡呢.
我們會由第十一章開始看第十四章.
這段經文.
在馬太福音和馬福音.
都有很詳細的記載.
我們嘗試將.
這兩段的故事.
合而為一.
綠色的部分.
是馬克福音.
其餘的部分是馬福音.
將所有的細節.
加在一起.
大家可以看得很清楚.
這段經文.
那時.
是發生之後的事.
第三節是原來.
是說之前的事.
那時這一段開始.
是說耶穌基督.

$^{641}$傳福音.
很多人跟隨他.
施洗日行已經被殺害.
希律王聽到.
耶穌的聲明之後.
就開始不安.
留意一下.
施洗日行的使命.
沒有做過神蹟.
耶穌有做過神蹟.
但為什麼會搞亂.
施洗日行.
會是耶穌基督.
復活了的耶穌基督.
對不起.
施洗日行復活了的耶穌基督.
因為他害怕.
他無故殺害.
一個他知道是聖人的人.
他知道是先知的人.
但其實這件事.
本身就跳回第二節.
講回之前的故事.
之前的故事.
我剛才解釋過.
施洗日行.
宣告神的律法.
告訴他你娶你的侄女.
是別人的太太.
你的太太還在世.
你要和她離婚是犯法的.
宣告神的律法.
所以整件事的開頭.
就是神的公義.
開始.
宣告神的公義.
神的光照到人的黑暗裡.
人有不同的反應.
有些人會認罪悔改.
有些人會更加抗拒.

$^{681}$抗拒罪惡.
有些人更加抗拒上帝.
甚至起殺機.
將宣揚神說話的人滅聲.
這就是原因.
所以一開始.
神的公義開始.
神的公義從來沒有缺席過.
整件事就是神的公義.
神的光照到人的黑暗裡.
指出你們這樣做是錯的.
不合法的.
令到神的光.
令到那些人情願要黑暗不要光.
所以神的公義沒有離開過.
甚至一開始就是神的公義.
引發這件事.
施洗日漢可以不出聲.
作為神的僕人.
是他的責任.
宣揚神的話.
你留意到.
施洗日漢希律.
他想殺.
第五節他想殺.
施洗日漢.
而希律懷恨他也想殺.
說到被人想殺.
也很特別.
所以我們牧師前道人.
我們也要留意.
我們要面對一個危機.
但這是每個基督徒.
不單止牧師前道人.
你傳福音也會遇到的.
一個反應.
但他們兩個都殺不了.
因為希律怕民眾.
而希律怕民眾.
所以希律就想了一個方法.

$^{721}$借刀殺人.
直至希律生日的時候.
安排一個計劃.
讓他不殺人.
我們看到這件事.
真理從來沒有缺席過.
因為神的光.
來到這個世界.
人知道自己行為是惡的.
他不愛光 倒害厄林.
這樣神將來會定他們的罪.
但大家留意到.
很奇怪的.
歷代也有一個規則.
很多時候.
有權的權貴.
他要去監禁一些人.
是因為他怕那些人.
很特別.
被監禁的是被害怕的.
希律去監禁斯大林.
因為他怕他.
斯大林沒有怕希律.
留意到嗎.
歷世歷代也有這樣的事發生.
被監禁的人是被害怕的人.
監禁的人是怕.
才抓人坐牢.
抓人坐牢.
我們看到.
這件事.
希律怕民眾.
怕約翰.
他怕同籍的人.
可以沒有面子.
要殺約翰.
他唯獨不怕上帝.
基督徒說.
要做的對和相反.
我們只要敬畏上帝.

$^{761}$不要怕人.
但你看到希律王.
他怕民眾.
怕約翰.
他怕坐牢的人.
他怕沒有面子.
但不怕上帝.
這就是他的問題.
聖經記載.
約翰的門徒不怕.
他去拿他的身體.
埋葬去見耶穌.
下一個問題.
為何施洗約翰.
死亡或被捕時.
耶穌不會拯救他.
我們看回耶穌做了什麼事.
剛才我提及.
在《馬太福音》第十四章.
耶穌的回應.
當他聽到施洗約翰被殺時.
他獨自離開.
聖經沒有提及.
他獨自離開時做了什麼事.
但我們看回.
當他的朋友死時.
他有時會哭泣悲哀.
所以我們相信.
他可能那時是祈禱.
甚至有悲傷的時候.
但很快被人發現.
聖經有提及.
《馬太福音》和《盧格福音》.
耶穌出來見到一大群人.
他憐憫他們.
耶穌接待他們.
治好他們的病.
耶穌基督.
在公義被踐踏時.
當施西約翰無故被殺害時.

$^{801}$他的回應是.
繼續傳福音.
繼續用神的光照人的黑暗.
這就是耶穌基督的回應.
為什麼呢?.
我們再看回耶穌基督的使命.
他要避開要做王.
因為他的國不在這個世界.
他的國在我們心裡.
他要改變人心.
拯救靈魂.
所以我們看回.
他治病.
憐憫人.
是希望聆聽.
聆聽的人.
可以聽見神的話.
盲的人可以看見.
可以見到上帝.
這才是目的.
耶穌基督的使命.
不是拯救我們的肉身.
是拯救我們的靈魂.
所以在啟示錄.
第二章.
上帝藉著約翰比斯梅娜教會的信.
提到.
那首先抹後曾經死去而活過來的.
這就是講主耶穌基督的說話.
你們不要怕受苦.
你們有些人會被下監.
會受到試煉.
你們要至死忠心.
我就把那生命的冠冕.
賜給你.
聖靈向宗教會所說的話.
有耳的就應聽聽.
得勝的.
決不會受第二次死的害.
耶穌基督來的目的.

$^{841}$就是將你和我避開.
第二次死的害.
第一次我們會死.
就算耶穌基督.
他治好拉撒路.
他治好其他兩個年輕人.
這些人都會死.
只是你傷心一次.
和傷心兩次的分別.
明白嗎.
耶穌基督最大的使命.
就是要我們避開第二次的死.
第二次的死就是永遠與神隔絕的死.
所以.
在第20章.
白色大補助的審判.
所有死的人.
都要站在神的寶座面前.
各人都會照著.
所行的受審判.
如果是生命冊裡.
沒有名字的.
就是被調入火狐.
這個火狐就是第二次的死.
所以耶穌基督來.
不是要拯救我們.
肉身的生命.
是拯救我們靈魂.
所以耶穌基督.
沒有提醒.
被困的會得自由.
因為他已經預定.
他容許.
施一諾汗.
殉道.
不只.
施一諾汗會殉道.
還會更多人會殉道.
我們聖經裡有說到.
所以最後我們看.

$^{881}$如何去看耶穌基督的回應.
和明白他的心意.
耶穌基督.
走的路.
比施一諾汗更大.
施一諾汗的死.
是無辜的.
最大的不公義.
是耶穌基督的死.
因為他背負起.
所有人的罪.
一個無辜無罪的人.
為有罪的人.
去負責.
這個才是最大的不公義.
這個不公義.
你和我也有份.
因為你和我.
曾經羞辱耶穌基督.
你和我的罪.
釘在十字架上.
所以耶穌基督要走的路.
施一諾汗都要走.
因為耶穌基督的教訓.
學生不能高過先生.
人也不能高過子人.
但耶穌基督.
也提醒我們.
那殺身體不能殺靈魂的.
不要怕他們.
我有把.
那身體和靈魂都滅在地獄裡的.
正要怕他.
所以耶穌基督給我們一個鼓勵.
我們每一個.
跟從神的人.
都可能要走上這條路.
還有很多人.
會殉道流血.
在舊約裡.

$^{921}$我們看到很多先知.
例如猶爾塞亞 以利米等.
的先知.
都是被殺害的.
從十個門徒殉道.
司祭梵.
講完最偉大的講章.
被人用石頭扔死.
保羅當時在場.
到保羅悔改認罪之後.
都是殉道.
彼得神曾經在監獄.
拯救過他.
最後都是殉道死.
所以很多人.
還會為神去殉道.
為什麼呢.
啟示錄告訴我們.
在第五印的時候.
在這裡講到.
日漢見到.
祭壇底下.
有為神的道.
並作見證被殺的人.
的靈魂.
大聲的喊著.
聖潔真實的主啊.
你不審判住在地上的人.
給我們身流血的冤.
要等到何時呢.
於是有白衣人賜給他們各人.
又有話對他們說.
還要安息便是.
等著一同做僕人的.
和他們的弟兄.
也讓他們被殺.
滿足了數目.
主要再回來.
除了得救的人數滿.
還會有.

$^{961}$為主流血的人數滿.
才是.
主再回來的時候.
如果主還沒回來.
還會有人為上帝要殉道.
耶穌基督給我們一個.
很清晰的教訓.
這些教訓是幫助我們.
改變我們很多時候.
錯誤的思想.
因為我們祈禱的時候.
都希望上帝像.
入廟拜神的神靈一樣給我們好處.
祈福不會賜禍.
如有白浩天跟我們說.
這個真實的上帝.
是絕對主權的.
祂可以賜可以收.
可以賜福可以降禍.
所以在.
登山補訓第八幅.
第四幅講愛慕公義.
第八幅講為義而受迫害.
的人是有福的.
特別是在第八幅裡面.
加多一句.
其他七個福氣沒有說.
第八幅裡面加多一句.
因我的緣故辱罵你們.
迫害你們並且捏造各樣壞話.
毀謗你們.
你們就有福了.
你們就有歡喜快樂.
你們在天上的賞賜是大的.
在你們以前的先知.
他們也曾這樣迫害.
這裡特別加多一句.
雙重的福分.
給為義而受迫害的人.
天國是他們的.

$^{1001}$天上的賞賜是大的.
所以我們不需要問.
為什麼神不拯救.
施洗約翰.
因為耶穌基督已經講得很清楚.
天國是他的.
他的賞賜是大的.
但這裡有個範圍.
大家要留意.
為義受迫害.
是因耶穌基督.
因上帝的緣故受迫害.
不是因為你和人的政見不同.
受迫害.
不是因為你和人的看法不同.
受迫害.
兩個人不是基督徒都可以政見不同.
互相迫害.
這裡講的為義受迫害.
是因為你願意過一個基督徒生活.
在一個分別為性生活的範圍內.
受迫害.
明白嗎.
在這個範圍你要做基督徒.
受迫害.
是因為耶穌基督的緣故.
跟隨耶穌基督的緣故受迫害.
這才是範圍.
所以我們不要有個錯覺.
做基督徒會讓凡事.
很舒服.
聖經不是這樣講.
聖經提醒我們.
我們是用心靈.
按真理.
用心靈.
按真理去敬拜上帝.
這個真理.
上帝跟我們講.
是跟從神的人會受苦.

$^{1041}$這個就是真理.
所以我們祈禱的時候要記住.
求神給我們能力去面對這些困難.
不是求神讓我.
避開了吹吉避凶.
吹吉避凶的祈禱不是合乎神心意的祈禱.
真正合乎神心意的祈禱.
就是.
你既然給我有困難.
我求你與我同在.
與我同行一起去面對.
這個才是合乎神心意的祈禱.
不是吹吉避凶的祈禱.
我們要和別人不同.
分別為性的生活.
包括我們祈禱的生活.
你既然.
給我承擔這個困難.
我就求你給我能力與我同行.
我不是要求神.
吹吉避凶.
所以耶穌基督在克西瑪利人的禱告.
最後他說.
父神是照你的主意行.
不是照我想怎樣去行.
就是這個意思.
神亦都很清楚.
與那些不公義的人.
有個宣告.
許二上亞書第十章.
和哉.
那些設立不義之律例的.
和記錄奸詐判詞的.
有和 為什麼呢.
神問他們三個問題.
到.
降佛的日子來的時候.
即審判來的時候.
你要怎樣做呢.
你可以怎樣做呢.

$^{1081}$你去哪個奔走求救呢.
你的財寶要放在哪裡呢.
神已經有個宣告.
給他們.
你將來怎樣辦呢.
你現在做這些事.
你將來怎樣辦呢.
千萬不要.
這裡有兩個含義.
立不義的.
立法的.
記錄奸詐判詞的.
參與執行的人.
神都會審判.
神問他們三個問題.
他們要記住.
你將來要怎樣辦.
你走到哪裡.
你的財寶放在哪裡.
全部沒有用.
神說了.
我以說出就必成就.
我以謀定也必做成.
神說了一定會做.
我們不用擔心.
再看斯西約翰.
和耶穌一樣.
他有33歲的生命.
所以其實生命.
不在乎長短.
我這次準備查經的時候.
再想想.
斯西約翰.
在座當中.
他經濟不是很好.
一生人可能吃一兩餐.
是好的.
但他享受過.
斯西約翰.
住在曠野大食.

$^{1121}$他真的沒有享受過.
他可能沒有.
一個家庭的幸福.
沒有兒女的幸福.
他去到坐牢的時候.
都沒有要求自己的自身安全.
希望拯救.
他一生.
為主耶穌基督預備.
將神交託的使命行出來.
至死忠心.
不在乎長短.
生命在乎果效.
今天我有兩個例子.
一個是短的.
一個是長的.
和大家分享.
拜德森創立內地會的時候.
在英國呼召人們.
為中國人祈禱.
去那裡住福音.
當時有一個.
牛津大學的醫學生.
是獲獎的.
第一名.
叫做費爾德醫生.
他是出身富有家庭.
也是醫生世家的人.
他回應呼召.
1878年加入中國內地會.
當時他被派去太原的地方.
建立當時第一間醫院.
當時內地會已經看到.
很多人吸鴉片.
所以他去那裡成立一個醫院.
去幫那些人戒毒.
他去了第三年的時候.
染了藥疾.
死了.
他死之前.

$^{1161}$最後一句話.
告訴拜德森先生和警察會.
在中國這三年.
是我一生最快樂的三年.
一個在牛津.
在英國的醫學生.
他去到那裡.
一個在牛津.
醫學第一名的人.
去中國裡面做福音.
三年不是很長.
如果不明上帝的人就會問.
為何這麼早死呢?很可惜.
但其實不是.
在神的國裡面是有意思.
耶穌基督說過.
我實際上告訴你們.
一粒墨紙不落在地裡.
仍舊是一粒.
若是死了就結出許多粒子.
許多子粒來.
史佩德醫生死了兩年後.
劍橋有七個人回應.
劍橋七子 劍橋七傑.
七個劍橋的畢業生.
回應拜德森的呼召.
去中國做傳道.
這七個人.
如果坐在前面.
左手邊第二位.
可能大家未聽過.
叫Dixon Host.
他最後.
86歲生命.
25歲去中國.
85歲才離開.
在中國60年.
這是長的.
有什麼特別呢?.
可能大家沒人聽過.

$^{1201}$他是第二任總幹事.
德森.
他知道自己病.
開始找一個.
接班人.
但因為義和團.
殺外國人.
他接任的被殺害.
Dixon Host.
接任第二任總幹事.
大家看看.
中間的照片.
Dixon Host.
是一個富有的家庭.
他父親.
是英國高級的軍官.
他也可以做高級的軍人.
他本身家裡很富裕.
但他選了60年.
在中國生活.
如果大家有機會.
去到新加坡.
現在叫OMF.
以前叫內地會.
49年後總部去了新加坡.
右手邊.
馮浩留牧師.
上一任總幹事.
他做了19年.
現在是寧豐基金的主席.
馮浩留醫生.
也是博士.
因為他在中國.
是歷史系的博士.
他的博士研究.
就是研究Dixon Host.
為什麼中國人.
做了60年.
沒有人認識這個人呢.
馮醫生的研究.

$^{1241}$追查了檔案.
在倫敦的檔案.
原來很多Dixon Host的同事.
在他死後.
說了一句話.
一樣的.
他Live to be forgotten.
原來Dixon Host一生.
侍奉.
是很低調.
不邀功.
低調到人們記不住他.
為什麼呢.
因為他只想人們記得耶穌.
不是記住他.
這個就是他一生的目的.
Dixon Host做了第二任的時候.
本來.
殺洋人是需要賠償的.
但內地會拒絕賠償.
他不要求任何賠償.
當時有58個.
內地會的前導人被殺害.
另外有21個.
他們的孩童被殺害.
但內地會沒有要求任何賠償.
最後結束的時候.
我想和大家分享一句說話.
忠心侍奉.
靜靜地死去.
這句說話是怎麼來的呢.
我讀英國讀書的時候.
是去英國的.
英國的教育學院.
我讀英國的教育學院.
我讀的教會是斯托德牧師.
當時還有港島的教會.
他在港島的時候.
很多人在排隊去聽.
當時有個很年輕的.

$^{1281}$傳導人.
就是左邊這個.
Nicole Tice.
有一次港島.
他進去教會的時候.
有一個很出名的牧師在.
他想一想.
我一生在這個教會可以做什麼呢.
他說斯托德牧師.
四點鐘起床為人祈禱.
他說斯托德牧師.
由四點鐘到九點鐘所做的事.
已經多過他一天的事.
他可以做什麼呢.
如果大家聽過.
Alpha Course.
有兩個全世界影響力最大的.
福音的查經.
一個是Alpha Course.
一個是Christianity Explore.
他就是發起Christianity Explore.
或者叫生命歷期.
的那個Program.
那個Program的人.
當他思想.
我侍奉.
究竟崗位應該怎樣做的時候.
有一個很出名的牧師在.
我怎樣侍奉呢.
有個當時的主教.
跟他說.
你研究一下斯特耶罕一生.
你就會明白.
所以他後來在2021年寫了一本書.
他將斯特耶罕.
侍奉的一生.
他立成一句.
Serve faithfully.
Die quietly.
即是說忠心的侍奉.

$^{1321}$靜靜的死去.
什麼意思呢.
我們一切的侍奉.
不要奪取上帝的榮耀.
當.
如果你在教會侍奉.
你覺得不受尊重.
而不高興.
是錯的.
我們要侍奉不是要我們.
是否受尊重.
我們在教會的會議裡.
你覺得你的提議.
雖然會不接受.
你又不高興.
這又是錯的.
因為我們不是要我.
在乎我的想法是否受接納.
如果你參與侍奉.
做報告的人.
漏了你的名字.
你又不高興.
這又是錯的.
因為我們不是要別人去讚揚我的侍奉.
我們不可以用.
我們的侍奉去高舉自己.
因為整個的侍奉.
是高舉耶穌基督.
所以迪生浩斯.
施西約翰.
都是做同一件事.
是侍奉.
高舉耶穌的名字.
因為一切榮耀.
頌讚都是屬於神.
是全是.
因為你的國度,權柄,榮耀.
全部是屬於上帝.
神的國的事.
神的聖靈的大能.

$^{1361}$所以當時的西約翰說.
他必興旺,我必衰微.
這就是我們.
每一位基督徒侍奉.
應有的態度.
我們不是要自己的榮耀.
不要奪取上帝的榮耀.
我們侍奉神是理所當然的.
因為國度,權柄,榮耀.
全是你的.
直到永遠.
多謝.
\newpage



\chapter{賴若瀚}\label{ch:preacher8}
\begin{multicols}{3}
\minitoc
\end{multicols}
{ \scriptsize


\begin{xltabular}{\textwidth}{|p{0.15\textwidth} p{0.6\textwidth}|p{0.07\textwidth} p{0.1\textwidth}|}
\hline
出埃及記 32:1-14 & \hyperref[sec:srCkvhUNl9w]{扭曲的敬拜, 代求的更新 (出埃及記32\_1-14) -  賴若瀚牧師} & 2025-01-11 & \href{https://youtube.com/watch?v=srCkvhUNl9w}{\texttt{ srCkvhUNl9w}} \\
\hline
\end{xltabular}
}
\newpage



\section{出埃及記 32:1-14}
\label{sec:srCkvhUNl9w}
\textbf{扭曲的敬拜, 代求的更新 (出埃及記32\_1-14) -  賴若瀚牧師}
\newline
\newline
連結: \href{https://youtube.com/watch?v=srCkvhUNl9w}{\texttt{ https://youtube.com/watch?v=srCkvhUNl9w}} ~~~~ 語音日期: 2025-01-11 
\newline
\newline
\hyperref[sec:yRzXvTTOZfM]{< < < PREV SERMON < < <}
~
\hyperlink{toc}{[返主目錄]}
~
\hyperref[ch:preacher8]{[返講員目錄]}
~
\hyperref[sec:lTGVgidxHms]{> > > NEXT SERMON > > >}
\newline
\newline
出埃及記 32:1-14
\newline
\begin{longtable}{cl}
\hline
\hline
章節 & 經文 (和合本修訂版)\\
\hline
32:1 & \begin{tabularx}{0.7\textwidth}{X} 百姓見摩西遲遲不下山,就聚集到亞倫那裡,對他說:「起來!為我們造神明,在我們前面引路,因為領我們出埃及地的那個摩西,我們不知道他遭遇了甚麼事。」 \end{tabularx} \\ \\ \relax
32:2 & \begin{tabularx}{0.7\textwidth}{X} 亞倫對他們說:「你們去摘下你們妻子、兒女耳上的金環,拿來給我。」 \end{tabularx} \\ \\ \relax
32:3 & \begin{tabularx}{0.7\textwidth}{X} 眾百姓就摘下他們耳上的金環,拿來給亞倫。 \end{tabularx} \\ \\ \relax
32:4 & \begin{tabularx}{0.7\textwidth}{X} 亞倫從他們手裡接過來,用模子塑造它,把它鑄成一頭牛犢。他們就說:「以色列啊,這是領你出埃及地的神明!」 \end{tabularx} \\ \\ \relax
32:5 & \begin{tabularx}{0.7\textwidth}{X} 亞倫看見,就在牛犢面前築壇。亞倫宣告說:「明日要向耶和華守節。」 \end{tabularx} \\ \\ \relax
32:6 & \begin{tabularx}{0.7\textwidth}{X} 次日清早,百姓起來獻燔祭和平安祭,就坐下吃喝,起來玩樂。 \end{tabularx} \\ \\ \relax
32:7 & \begin{tabularx}{0.7\textwidth}{X} 耶和華吩咐摩西:「下去吧,因為你從埃及領上來的百姓已經敗壞了。 \end{tabularx} \\ \\ \relax
32:8 & \begin{tabularx}{0.7\textwidth}{X} 他們這麼快偏離了我所吩咐的道,為自己鑄了一頭牛犢,向它跪拜,向它獻祭,說:『以色列啊,這就是領你出埃及地的神明。』」 \end{tabularx} \\ \\ \relax
32:9 & \begin{tabularx}{0.7\textwidth}{X} 耶和華對摩西說:「我看這百姓,看哪,他們真是硬著頸項的百姓。 \end{tabularx} \\ \\ \relax
32:10 & \begin{tabularx}{0.7\textwidth}{X} 現在,你且由著我,我要向他們發烈怒,滅絕他們,但我要使你成為大國。」 \end{tabularx} \\ \\ \relax
32:11 & \begin{tabularx}{0.7\textwidth}{X} 摩西就懇求耶和華-他的神,說:「耶和華啊,你為甚麼向你的百姓發烈怒呢?這百姓是你用大能大力的手從埃及地領出來的! \end{tabularx} \\ \\ \relax
32:12 & \begin{tabularx}{0.7\textwidth}{X} 為甚麼讓埃及人說:『他領他們出去,是要降災禍給他們,在山中把他們殺了,將他們從地上除滅』呢?求你回心轉意,不發你的烈怒,不降災禍給你的百姓。 \end{tabularx} \\ \\ \relax
32:13 & \begin{tabularx}{0.7\textwidth}{X} 求你記念你的僕人亞伯拉罕、以撒、以色列。你曾向他們指著自己起誓說:『我必使你們的後裔像天上的星那樣多,並且我要將所應許的這全地賜給你們的後裔,讓他們永遠承受為業。』」 \end{tabularx} \\ \\ \relax
32:14 & \begin{tabularx}{0.7\textwidth}{X} 於是耶和華改變心意,不把所說的災禍降給他的百姓。 \end{tabularx} \\ \\
[1ex]
\hline
\hline
\end{longtable}
$^{1}$各位電影節目大家好.
在2020年亞利桑那基督教大學.
請了美國著名的調查機構George Bonner為主.
調查美國整個情況.
他發現調查得來的結果是美國現在流行的一個世界觀就是混合主義.
混合主義是什麼呢?混合主義就是混合一些東西.
混合一些東西.
混合主義其實是看你混合什麼在裡面.
他發現現在的世界觀主要可以用一個名稱去稱呼它.
名稱叫道德治療自然神論.
你看它的英文可能覺得很複雜很難理解.
簡單來說它有三樣東西.
第一樣就是自然神論.
自然神論就是他相信一個很遠的神.
而不是很近的.
這個世界有沒有神呢?有神.
不過這個神也不是我們聖經所相信的那種有位格的.
它是一個自然神論.
而這個神是一股力量可以幫到我.
但是他又不來管我.
這個其實很矛盾.
今天的世界其實很多人都需要一種感覺好.
自我舒適.
感受良好的宗教.
這個叫自然神論.
第二樣就是治療.
這個世界其實很破碎.
很多問題.
所以人在這個世上他知道自己是有限.
但是他要掌管自己的人生.
在遇到困難的時候.
他很想這個神.
雖然平時是遠遠不可及.
但是當他需要的時候.
最好他就過來幫忙.
這個叫治療.
就是一種叫自我舒適.
自我感受良好的宗教.
第三當然.
如果你說到有宗教信仰.

$^{41}$或者有人基本的道德.
他就一定要有道德的因素在裡面.
所以一些道德.
加上很多治療.
加上一個可以幫到他的神明.
但是不要管他.
只是幫他.
大家覺不覺得這個就是現在整個世界的趨勢呢?.
我覺得是的.
雖然我們在美國.
但是我們也說了.
其實很多東南亞或其他地區.
人慢慢漸趨專愛自己.
提摩太后書第三章所說.
就是專門愛自己.
希望自己能夠感覺好.
這個其實是放置於四海而皆盡.
這個世界其實越來越複雜.
越來越危險.
也越來越更具挑戰.
其實道德治療自然神論.
在出埃及時在摩西時就已經有.
這種混合主義在摩西時代就已經有.
今天我要看的就是出埃及記第32章.
第1到第14節.
我們的題目叫做扭曲的敬拜.
代求的更新.
如果要說出埃及記第32章.
我們要看出埃及記整本書的結構.
整本書的結構很簡單.
就是先是1到第18章.
講到蒙救贖.
大家如果看出埃及記.
最主要的事件就是.
余捷和埃及長子被殺.
然後他們過紅海離開維羅茲瓦.
第19章他們來到神的山.
19章開始就是要訓誨.
在山上摩西領受誡命律例典章之後.
就下到山下來吩咐這些百姓.

$^{81}$這五章聖經其實很多都是一些教導.
一些律例要他們怎樣去生活.
怎樣去敬拜.
怎樣去遵守屬靈的規矩.
蒙救贖的人需要受訓誨.
而受訓誨的人需要學敬拜.
在最後25章到第40章.
我們看到會幕的建造.
這個大段落可以分為三個段落.
就是一個三文治結構.
三文治結構就是兩塊麵包.
兩塊麵包就是會幕的敬拜.
會幕的敬拜先有會幕的藍圖.
25到第30章.
然後就有會幕實際的建造.
就是35到第40章.
而這兩個會幕正統的正確的敬拜.
中間就夾雜了一個叫做金牛獨的敬拜.
如果你用麵包和那塊肉.
這兩塊麵包是好的.
就是那塊肉就壞了.
那塊肉就是扭曲的敬拜.
今天我們來看看32章.
32章第一到第十四節.
我們來看看它的劇情是怎樣的呢.
基本上以色列人因為摩西上了山.
遲遲沒有下來.
所以以色列人請亞倫和他們立一個金牛獨.
以致他們犯下彌天大罪.
耶和神譴責以色列百姓.
說他們叛亂.
然後神宣佈要滅絕他們.
但摩西不肯.
摩西不答應.
神本來要以摩西重新開始.
但摩西求神赦免要恕他的百姓.
最終耶和神改變了他的心意.
以致他的怒氣被轉少.
大家要注意32章其實是在山下發生的事情.
摩西和耶和神有三十多天.

$^{121}$有四十天但現在還未到.
有三十多天在山上彼此溝通.
我們來看看這段經文.
先講第一節.
百姓見摩西前伯下山.
就大家聚集到亞倫那裡.
對他說:起來為我們做神像.
可以在我們面前引路.
因為領我們出埃及地的那個摩西.
我們不知道他做了什麼事.
你看這裡講話的語氣很不尊重.
領我們出埃及地的那個摩西.
如果你的會友跟你說那個牧師.
其實是很不尊重.
摩西站在山上接受神的誡命.
神的吩咐.
在民眾消失了三十多人.
而失去主心骨摩西領袖.
這群會眾陷入了恐慌.
他們前路茫茫.
不知道如何面對前路.
大家知道群眾就是群眾.
群眾需要一個領袖.
特別是在曠野的地方.
所以這群群眾失去領袖時.
他們缺乏信心.
他們自我抬頭.
所以最終他們沒有辦法面對.
他們就起哄.
其實這群百姓很可憐.
也很無知.
為什麼這樣說呢.
其實他們在埃及地出來的時候.
他們經歷了很多神跡奇事.
埃及的十災.
逾越節.
然後過了紅海.
過了紅海之後.
他們在曠野裡吃馬拿.
馬拿其實是神蹟.

$^{161}$然後摩西擊盤出水.
而他們在曠野裡.
經受到雲柱和火柱的帶領.
這些全部都是神的大能的彰顯.
但是他們過了一會兒.
他們就忘記了這位納約的神.
以致他們要自己另立一個金牛獨.
人有時看東西看不清楚.
如果大家以前看過我的視頻.
就知道我是戴眼鏡的.
我最近做了兩隻眼的白內障手術.
現在還未配新的眼鏡.
所以醫生說.
你現在看遠的東西應該是OK的.
看近的東西可能需要再做一隻眼鏡.
我現在還未去做.
所以我現在看東西.
近的東西是有點模糊的.
其實人很多時候都有熟齡的白內障.
久而久之.
就有一個很模糊的鏡片.
好像蓋住你的視覺.
所以我們需要經常去體會.
經常去檢視.
甚至求神來做一個熟齡白內障的手術.
這班匯眾其實是被自我蒙蔽了.
他們看不到前面真正的警方.
以致他們做出一些很愚昧的要求.
百姓無知.
他們來到亞倫的面前.
亞倫就是第二把交椅.
他叫亞倫來為他做一個神像.
但是亞倫沒有去拒絕這個很荒謬的要求.
反而說你們去摘下你們妻子兒女以上的金環拿來給我.
百姓就都拿摘下他們以上的金環拿來給亞倫.
這些金器當然是從埃及人那裡拿回來的.
他們這麼多年受奴役被壓榨.
所以神就說你們向埃及人討你們過去的補身.
所以就拿了這些金銀出來.
現在亞倫就說你們拿這些金器給我.

$^{201}$然後我替你做什麼呢.
我替你煮一個金牛肚.
我們看到這個領袖的無能.
其實亞倫是第二把交椅.
大家應該都知道亞倫是夢神的選照.
他是當時的祭司.
亞倫是當時的大祭司.
所以摩西就管行政.
亞倫應該是管屬靈的事.
誰知道這一次他糊里糊塗陷入屬靈的危機.
是他的無能.
因為他的無能帶來了妥協.
又因為妥協而帶來了危機.
亞倫在第四節說.
他從他們手裡接過這些金器.
然後就鑄成了一隻牛肚.
用雕刻器具造成的.
他們就說以色列是使你出埃及地的神.
領袖無能做出一個妥協的行動.
亞倫在第五節在牛肚的面前捉壇.
宣告明日要向耶和華首節.
明日清早百姓起來獻梵祭和平安祭.
就坐下吃喝起來喚傘.
大家如果要看這段經文.
要先看看金牛毒是什麼.
金牛毒應該是一些仿效周邊.
可能是埃及.
又可能是周邊的國家.
他們異教徒的風俗.
用一些動物作為偶像.
當作神明去敬拜.
古代經常都有.
特別是在異教的國家.
這是什麼呢?.
這似乎是在供奉神.
是在敬拜神.
因為如果你看這群以色列民.
他們說什麼呢?.
以色列這是領你們出埃及地的神.
是否在暗示耶和華神呢?.

$^{241}$是的.
亞倫在第五節說明日要向耶和華首節.
在牛肚的面前捉壇.
向耶和華首節.
他們是否當牛肚是耶和華神的代表呢?.
是的.
所以我們可以說.
這是仿效耶和華神而做出的一個像.
亞倫用Yahweh這個名稱.
其實是神納約的名字.
這些百姓沒有用Yahweh這個名稱.
他們說這是領你出埃及地的神.
這個神就是Elohim.
無論你用什麼神的名稱.
其實這不是真的代表神.
只不過是樣貌.
我們中國人有一句話.
是掛羊頭賣狗肉.
掛著的名字.
其實裡面所說的內涵根本不是.
詩篇106篇第十九到第二十節說.
他們在河列山做了牛肚.
叩拜著神的像.
如此將他們榮耀的主換成為吃草之牛的像.
很明顯這是拜牛像.
雖然口中稱祂為神.
但實際上他們是在拜牛像.
和異教徒沒兩樣.
耶羅波安後來重套覆轍.
在丹和伯特利兩個地方都設立了金流獨.
讓他們去敬拜.
代替耶路撒冷首節去敬拜真神.
如果你看這段經文.
你會發現他們在神明的面前.
在牛毒的面前歡呼.
在那裡喚灑.
在雷歷聖經研讀本裡.
喚灑暗示男女之間的玩樂.
很可能他們有種酒,狂歡,跳舞.
圍繞著金牛獨的野性自我情慾的表達.

$^{281}$不知不覺離棄了真神.
如果你看金牛獨事件違背了第二屆.
第二屆不可以雕刻牛像.
第二屆已經違反了.
也違反了第三屆.
不可以妄稱耶和華神的名.
他們現在稱牛毒為耶和華神.
是否妄稱神的名?.
他們也違反了基本的第一屆.
不可以敬拜別的神.
剛才我們說了.
金牛獨是仿效耶和華神的信仰.
但他們現在是敬拜耶和華神.
不是.
他們看的是牛毒.
雖然他們口中說是耶和華神.
但他們敬拜的是牛毒.
神明確說不可以雕刻牛像.
不可以有另外的神.
所以他們是敬拜另外的神.
違反了第一屆.
真神的信仰其實是啟示的信仰.
不是你造出來的.
牛毒是他們自己造出來的.
所以有一句說話.
真神造人假神人造.
真神找人假神人找.
人找回來的不是啟示的信仰.
是人找回來的.
他們將耶和華神的信仰和牛像的敬拜融合在一起.
其實不是純正的.
這就是混合主義.
我們要問一個基本的問題.
亞倫為何答應百姓無理的要求呢?.
他其實是一個屬靈的領袖.
他應該知道的.
亞倫知不知道自己做的事情是錯的呢?.
我相信他知道的.
因為後來他和摩西解釋的時候.
辯解他為什麼會這樣做的時候.

$^{321}$他也知道是推了責任給匯眾.
所以他知道這件事是惡的.
他說百姓專注作惡.
你知道的.
他們對我說要做一個神像.
後來他拿了金銀在二十四節說.
他們就吸了我.
然後我就把金銀拿在火裡.
然後這牛毒便出來了.
大家去看這段經文也覺得很搞笑.
你把金氣放進去.
那隻牛毒就出來.
這麼簡單這麼神奇.
所以亞倫知道是錯的.
為什麼他仍然要繼續做呢?.
因為群眾的壓力.
後來因這件事被殺的有三千人.
我相信這三千人是始作俑者.
這三千人都是搞手.
如果這三千人面對著你.
要求你要做一件事.
你敢不做嗎?.
你不做的話.
你會有生命危險.
所以我相信亞倫是因為群眾的壓力妥協.
也因為自己的生命會受到危險.
很可惜是吧?.
亞倫沒有擔當.
他在最需要的時候沒有站起來.
所以約翰·馬克斯威爾曾經說過一句話.
Everything rises and falls on leadership.
教會的存亡存在於領袖是好還是壞.
簡單來說就是領袖定乾坤.
大家同意嗎?.
其實我覺得有很多教會之所以分裂.
很多教會之所以沒落.
很多教會之所以冷淡死氣沉沉.
很多時候是因為缺乏了領袖.
好的領袖,熟領的領袖,有擔當的領袖.
知道如何帶領弟兄姊妹向前走的領袖.

$^{361}$如果沒有好的領袖就像核子領核子.
亦如無頭蒼蠅.
四處轉來轉去.
原地打滾.
所以我們做主的牧人.
領袖可以有很多層面.
可以做教會領袖帶領全教會.
也可以是團契領袖.
也可以是小組領袖.
可以是師班領袖.
總之我們做領袖要有擔當.
最重要是有熟齡的體見.
而又肯面對壓力時仍然堅持到底.
因為真正的領袖是要向神負責.
金牛道事件讓我們看到我們今天需要有什麼注意.
我們回到開始時說George Bonner的2020年的調查.
混合主義.
混合主義的問題我們要小心.
有太多時候我們會接觸一些怪羊求賣狗肉.
有時他們說的是屬靈的說話.
但他們說的屬靈的說話是有偏差的.
從神的話語去驗證時會發現他們是出錯的.
所以我們自己也要在神的話語裡好好去紮根.
否則我們被這些假師傅以假論真.
有少少真有很多都假的.
令自己一頭煙.
小心今天的金牛道事件其實是混合主義的危機.
我們在世上遇到很多世俗的事物.
在教會裡有時這些屬世的東西會衝進來.
混在一起我們要小心.
也要小心不要被道德治療自然神論影響我們.
金牛毒這東西現在很多人把它變成拜金主義.
成功神學成功病毒.
我覺得也對的.
其實很多時候人.
如果你看出乃及記的百姓.
其實他們是為了自己的安全.
為了自己的保障.
為了自己的穩妥.
然後找金牛毒出來做他們的帶領者.

$^{401}$今天人找什麼來代替神呢.
很多時候就是找金錢.
找成功.
找一些我們可以靠的靠山.
而金錢財富可能是其中一個很重要的.
大家也知道香港最大面額的鈔票是一千元.
一千元他們叫做金牛.
也有書寫金牛毒.
其實是說一些成功或者拜金主義的危機.
毒害在影響我們.
所以小心.
我們不是不喜歡金錢.
也不是不喜歡成功.
但我們要注意在神的旨意和應許的底下.
在神的喜悅底下的成功.
小心要陷入無知的妥協裡面.
我們也求神興起屬靈的領袖.
有見識的領袖.
有擔當的領袖.
而帶領我們的教會.
今天太多時候是一些有雄心的領袖建立教會.
但是受到一些野心或者私心的領袖.
將他搞砸了.
小心啊.
自我的私慾.
有時不能成就神的心意.
我們看下面的段落.
這個題目叫扭曲的敬拜代求的更新.
基本上是兩個的東西.
而後面的就是帶來轉機.
在扭曲的敬拜底下.
我們看到耶和神和摩西在山上和他說.
耶和神吩咐摩西.
因為你的百姓就是你從埃及地領出來的已經敗壞了.
你有沒有覺得這句話有些奇怪呢?.
之前在第十九章.
神說他將這班百姓像鷹一樣背著他們出來.
離開埃及.
是神的百姓.
是神的拯救.

$^{441}$是神的帶領.
現在口氣好像轉了.
就是你的百姓.
你從埃及地領出來的已經敗壞了.
他們偏離我所吩咐的道.
為自己鑄了一隻牛犢.
然後向他下敗獻祭.
第九節.
神對摩西說.
我看這百姓真是硬著頸看的百姓.
你且柔著我.
將他們發烈怒.
將他們滅絕.
使你的後裔成為大國.
哇.
神說要滅絕這班百姓.
從摩西開始.
不是再一次是阿巴拉罕爾薩雅國的神.
如果你是摩西.
或者你是一個功力不夠深厚.
聽到這句說話你會怎樣呢.
如果一個功力不夠深厚.
有自己的議程.
聽到這句說話.
我覺得應該暗暗高興.
機會來了.
輪到我當家.
這班背信棄義的百姓.
搞得我一頭煙.
帶他們出來一點感激都沒有.
整天發怨言.
是時候和他們做一個了結.
現在不是我要求神要滅絕他們.
神說要滅絕他們.
而且要從我重新開始.
哇.
你說多好.
從我身上建立.
是唯一的門派.
已經沒有阿巴拉罕派.

$^{481}$現在傳下來的只有摩西派.
簡稱摩派.
你說多好.
我相信一個功力不夠深的人.
一定會和神說.
神啊,這是一個好主意.
阿們,願你的旨意成就吧.
是不是這樣呢.
這是一個相當大的考驗.
很大的考驗.
但是與神親密了這麼多天的摩西.
他沒有辜負神的心意.
他認識神.
他知道神的心意.
他明白神在這個時候的心情.
他明白.
神是不是動了真路呢?.
是.
而神是不是要威脅.
毀滅以色列人呢?.
是不是來真的呢?.
大家覺得呢?.
神是不是真的要毀滅這些以色列人呢?.
看上去似乎是.
大家覺得呢?.
有兩派的人.
我看一些書和文章.
發現有人說.
神真的要滅絕這些人.
但是我看Don Carson.
Don Carson是一個很出名的新約學者.
福音派的學者.
他寫了一篇文章.
關於舊約事件.
很難得.
一個新約學者來寫舊約.
他分析得非常清晰.
他說從表面看.
神好像要威脅.
要毀滅以色列.

$^{521}$但如果神真的要毀滅以色列人.
他就會取消對阿布拉罕的神聖的承諾.
他說如果是這樣.
神的承諾.
他說過的話就會失去效用.
神就會違背自己.
我們看哥林多後書第一章.
說神不會違背自己.
他的應許.
他說的事就是事.
他不會違背自己.
如果是這樣.
要怎樣解釋呢.
即是說神這次的怒氣.
是發作得非常猛烈.
以至於他說了一堆話.
是用我們人的感情來表達.
即是說神很憤怒.
憤怒到一個地步.
就好像如此.
不選你們.
將你們毀滅算了.
我們要解釋.
我們廣東人.
經常小朋友一言九鼎.
很生氣.
做了壞事.
爸爸媽媽很生氣.
生氣到一個地步.
生個叉燒都比生你好.
其實這是他極度憤怒的時候.
說的一句話.
有少少疑人法.
我就覺得.
如果像Carlson所說.
如果神真的要毀滅以色列百姓.
他的信實是受到大的挑戰.
所以應該不是.
不是是甚麼呢.
剛才我說是疑人法.

$^{561}$來表達一個人的怒氣到了一個極限.
而我們看《撒基里撒》四章.
後面兩章說甚麼呢.
這段經文經常被引用.
來講到神的屬性.
就是耶和華在摩西的面前宣告.
耶和華是有恩典有憐憫的神.
不輕易發怒.
且有豐盛的慈愛和誠實.
為千萬人存留慈愛.
赦免罪孽,過分和罪惡.
如果配合這節聖經.
剛剛兩章之後.
都是同一個場景.
你會發現.
其實在這裡講出.
神原本的屬性.
是有豐盛的慈愛和誠實.
而他是有恩典不輕易發怒.
這幾個原因已經夠解釋了.
我再加多一個.
有沒有可能神要試驗摩西呢.
試驗他合不合格呢.
試驗他是否真正的領袖呢.
是否真正無私的領袖呢.
到底摩西的心在想甚麼呢.
這個我補充下去.
剛剛這幾點已經夠解釋了.
但這個補充下去可能真的有.
但驗證了摩西.
pass with flying colors.
他拿到很高分.
他根本沒有私心沒有野心.
他完全是以神的心意為主.
所以我們看下去.
摩西祈求耶和華.
神發烈露的時候.
摩西祈求耶和華.
耶和華你為甚麼向百姓發烈露呢.
百姓是用你大力和大能的手.

$^{601}$從埃及地領出來的.
第一是他appeal to神的拯救.
神的大能神的能力.
百姓是你用能力拯救出來的.
第二是甚麼呢.
為甚麼是埃及人議論說.
他領他們出來.
就是要將他們殺在山上.
將他們從地上除滅呢.
這是神的明星.
神的聲譽的問題.
如果這樣做.
埃及人他們怎樣看呢.
外邦人他們怎樣看呢.
他們會覺得神出爾反爾.
你帶他們出來.
然後將他們殺了.
這是給人覺得是怎樣的一位神呢.
這是第二.
他appeal to 第一是神的拯救大能.
第二是神的明星和榮耀.
第三是求你轉移不法你的烈露.
後悔不降災禍給你的百姓.
第十三是祈求你紀念你僕人阿伯拉罕以撒以色列.
神指著自己起誓說.
我必使你們的後裔像天上的星那樣多.
並且我所應許的全地必給你們的後裔.
你們要永遠承受為義.
所以摩西他很認識神的約言.
神對阿伯拉罕的約.
不單止阿伯拉罕以撒和雅各.
他稱雅各為以色列.
即是摩西非常熟悉神在過去的救贖歷史.
他亦知道神給阿伯拉罕的約是永遠的約.
你注意最後那句.
他們要永遠承受為義.
阿伯拉罕的約是永遠的約.
所以為何摩西不能接受神給他的建言.
這是不可能的.
因為這是神的心意.

$^{641}$所以第三.
摩西的懇求.
第一是以神的救贖.
第二是以神的榮耀和名聲.
第三是以神的應許和話語.
這個代求我覺得是很高層次的.
我們太多人的代求是以人為本.
例如有弟兄病了.
你的祈禱多數是求神早日醫治他.
如果有弟兄姐妹失業.
你的祈禱是求神早日找到合適的工作.
我們帶出很多這樣的需要.
就是希望神有一個快速的解決.
但我們沒有預留空間去問神.
到底在他的身上有什麼心意呢.
你是不是在他的身上有一個更大的計劃呢.
藉著這次的困難.
能否彰顯到你在他的身上的榮耀呢.
我們很多時候不去注意這一塊.
其實為弟兄姐妹的病得醫治是沒錯的.
為他們找到工作是沒錯的.
為一些破裂的關係能夠復合也是沒錯的.
但這不是最優先的.
大家知道主禱文.
為日用飲食祈禱可不可以呢.
是可以的.
為不要遇見試探祈禱可不可以呢.
是可以的.
但這些其實是跟著前面優先的.
可不在天上的父.
願人尊你的名為聖.
願你的國降臨.
願你的子行在地上如同行在天上.
所以我們要認識神的屬性.
我們要以神的屬性來做我們代求的基礎.
如果是這樣的話.
我們比較能夠摸到神基本的心意.
在詩篇106篇23節.
所以祂說要滅絕他們.
若非有祂所揀選的魔勢站在破口.

$^{681}$使祂的怒氣轉燒.
恐怕祂就要滅絕他們.
今天需要什麼呢.
需要有人做代禱的勇士站在破口來防禦.
站在破口其實就是用禱告來保住.
我們的代禱可以讓人回歸.
也讓神的榮耀得到恢復.
頂住了.
最後我們看到神後悔.
不將所說的禍降給他的百姓.
怒氣被挽回.
就是因為有人站在中保的角色.
在現在的混合主義衝擊.
拜金主義的衝擊下.
我們要警醒.
我們要知道如何分辨撒旦的伎倆.
我們求神興起更多有擔當見識的中保人.
我們的領袖們.
他們不僅是站在破口做代禱的勇士.
而在黑縫中求神施恩.
不要做核子.
要做一個好的領袖.
我們帶頂姐妹回歸到真神的境界.
頂姐妹教會遇到危機.
其實這個世代也在遇到危機.
我錄製這節節目時.
面臨美國大選還有四天.
我發覺現在整個美國.
其實是陷入一個很大的深淵.
我經常為美國這個國家擔憂.
我真是很希望在四天後能夠選到的領袖.
神立美國國家原先的義師.
回歸真神的境地.
我發覺現在美國.
如果用16個字來形容.
第一不敬真理.
將神開始在立國先父的信仰.
真理的基礎完全要排在學校和國家門外.
第二不講道理.
經常以兇惡.

$^{721}$惡止能登.
大聲夾惡.
如果不服他有時會告上法庭.
用這種方式來解決問題.
第三不合情理.
不論你是什麼黨派.
最重要是你的政策是否符合常識.
常識是什麼呢.
常識是你要明白.
一直以來男就是男女就是女.
現在男的可以做女的.
甚至你不需要經過什麼變性手術.
你可以說你是男的就是男的.
你可以說你是女的就是女的.
你說是不是真是不合情理.
No common sense.
第四不顧倫理.
弟弟在父母養育下.
長大後可以聽學校的老師建議.
去變性而不需要父母知道.
你說是不是不顧倫理.
不講道理不合情理不顧倫理.
如果再加四個字就豈有此理.
我們面對一個很複雜的世代.
我相信在其他地區.
我亦體會到同樣的情況.
屬世主義妥協主義拜金主義.
混合主義其實衝擊我們的信仰和教會.
求神保守我們興起一些有見識的領袖.
求神帶領我們不要陷在撒旦的伎倆地下.
我們一起低頭祈禱.
天父上帝我們獻上感恩.
我們知道你的話語是安寧在天.
我們看見人性的敗壞.
不單單是今日在舊約時代已經很清楚的看見.
我們今日亦從舊約的一位擔當的領袖.
我們看見我們應該如何力挽狂瀾.
站在破口.
興起更多屬靈的領袖.
屬靈的弟兄姊妹有見識.

$^{761}$亦有擔當在破口防堵.
主啊帶領美國這個國家.
也帶領我們其他地區的國家.
求主都讓我們在你的面前.
教會能夠做一個明亮的燈台.
照亮我們周邊的人.
我們感恩祈禱奉耶穌基督的聖名.
阿們.
\newpage



\chapter{陳恩明}\label{ch:preacher9}
\begin{multicols}{3}
\minitoc
\end{multicols}
{ \scriptsize


\begin{xltabular}{\textwidth}{|p{0.15\textwidth} p{0.6\textwidth}|p{0.07\textwidth} p{0.1\textwidth}|}
\hline
約翰福音 4:23 & \hyperref[sec:lTGVgidxHms]{尋人啟示 (約翰福音4\_23) - 陳恩明牧師} & 2025-01-28 & \href{https://youtube.com/watch?v=lTGVgidxHms}{\texttt{ lTGVgidxHms}} \\
\hline
\end{xltabular}
}
\newpage



\section{約翰福音 4:23}
\label{sec:lTGVgidxHms}
\textbf{尋人啟示 (約翰福音4\_23) - 陳恩明牧師}
\newline
\newline
連結: \href{https://youtube.com/watch?v=lTGVgidxHms}{\texttt{ https://youtube.com/watch?v=lTGVgidxHms}} ~~~~ 語音日期: 2025-01-28 
\newline
\newline
\hyperref[sec:srCkvhUNl9w]{< < < PREV SERMON < < <}
~
\hyperlink{toc}{[返主目錄]}
~
\hyperref[ch:preacher9]{[返講員目錄]}
~
\hyperref[sec:499K9je19EI]{> > > NEXT SERMON > > >}
\newline
\newline
約翰福音 4:23
\newline
\begin{longtable}{cl}
\hline
\hline
章節 & 經文 (和合本修訂版)\\
\hline
4:23 & \begin{tabularx}{0.7\textwidth}{X} 時候將到,現在就是了,那真正敬拜父的,要用心靈和誠實敬拜他,因為父要這樣的人敬拜他。 \end{tabularx} \\ \\
[1ex]
\hline
\hline
\end{longtable}
$^{1}$請姐妹好.
今日的講題.
大家看到吧 是甚麼話.
為甚麼用這個啟示呢.
有中文老師在嗎 是不是應該用另外一個.
視講的視 是不是.
是不是寫錯了題目 是誰打錯了.
其實這個題目是我給的 我是故意的.
一般都會將尋人啟示.
有時真的寫錯了 成為上天啟示的啟示.
今日我要跟大家思想的.
今年度第一篇跟大家分享的訊息 是關於敬拜.
原來一個人能夠敬拜 是因為有.
上主來尋找你 你才能夠.
成為一個真正的敬拜者.
這是我們基督信仰獨一無二的地方.
與其他信仰有分別的地方.
另一方面 亦都想告訴大家.
這個從天上而來的啟示.
是真的需要讓世人都知道.
所以有尋人啟示.
為何要敬拜.
我們豐盛生命堂有一個五年計劃.
是怎樣的話 第一句.
不錯.
是.
向下紮根 往上結果.
向外拓展.
如果沒有向下紮根 甚麼都假.
而我們向下紮根 就是想講與神之間的關係.
敬拜.
有人說信仰在人類歷史上.
宗教在人類歷史上造成最大禍害的因由.
所以到今時今日.
不應該再有任何的宗教 任何的信仰.
寫這本書的人是史丹福大學.
修讀哲學 研究哲學和人類腦神經的科學.
所以很有邏輯 很有思想的人.
他告訴你不應該敬拜任何的真神.
任何的神 任何的Personal God.

$^{41}$不需要的 不應該的.
不過我看他的說話.
就發覺有些地方似乎是穿幫.
他說在這個受造界.
在這個宇宙的大大小小的創造中.
我們不會失去那種驚訝之情.
但我們驚訝就好了 千萬不要敬拜.
因為亞伯拉罕的上帝不配受敬拜.
不值得敬拜.
我看完這句說話再看下去.
就發覺他連人是怎樣.
他都說我都不知道怎樣定義何為人.
因為人是一個不斷改變的東西.
不斷有新的科學的發現.
不斷有新的東西來影響我們整個人生.
整個人甚至那個精神狀態 結構等等.
人是甚麼 不知道.
不過只知道一件事 人一定會死.
但未死之前既然有這麼多精彩的東西.
我們就應該加倍珍惜.
他們所傳的福音就是.
你未死你就好好地珍惜.
你珍惜我 我珍惜你.
就天下幸福了.
事實是不是呢 大家就再想想.
1月25日大家知不知道是甚麼日子.
不知道 多數不知道.
晚上6點03分的時候.
你看這個天空就發覺.
是金木水火土月亮.
是裂成一條直線.
你那晚有沒有看到.
當然沒有.
6點多 天又不光不黑.
而且全部都是雲層.
不過就這個天文現象.
你說奇不奇妙呢.
奇妙吧 是不是要讚美.
是不是創造很奇妙.
創造主很奇妙.

$^{81}$Sam Harris就說 創造是奇妙.
不過不需要在創造裡面.
你不需要敬畏創造者.
而且他這班人會有一個措辭.
可能你都聽過.
他說你們相信這些東西.
只不過是the god of the gaps.
即是你的理性 科學研究還未到位的時候.
你就說是上帝做的.
你這班懶鬼 甚麼都推向上帝.
其實又如何呢.
事實是不是這樣呢.
我就聽他說你在這個奇妙的創造界裡.
不需要敬拜的時候.
你會發覺他真的受西方基督教文化影響太深.
以致他用的詞令自己出現邏輯上的謬誤.
在創造上即是有創造者.
不過他是抗拒敬拜.
但我們今天要聽的是主耶穌基督.
和教會傳統裡給我們的一些智慧.
在教會傳統裡.
在十七世紀英國教會含羽肉非常紛爭混亂的時候.
有一班是敬虔的人.
是尊重聖經 敬畏上主的人.
叫做Puritans 清教徒.
他們聚在一起.
要將最重要的思想.
來濃縮成為一個基本的教義.
要你問答 叫做西敏斯答.
要你問答 有大的 有長的和短的.
但第一題是這樣的問題.
第一題就是.
What is the chief end of man?.
到底人之所以為人.
最重要的所為何事?.
答案就是敬拜.
人是活著為要榮耀上主.
並且以上主為樂.
我們就聽著這幾句話.
來思考我們怎樣可以成為一個土主喜悅的敬拜者.

$^{121}$我們豐盛生命堂所求的五年的東西.
若不是根植於對上主的愛.
對上主的敬畏.
對上主的榮耀的渴求.
我們所做的一切都會歸於土地.
主耶穌和撒瑪利亞夫人的一席說話.
給我們很多關於敬拜的啟發.
其實那位女士.
按人性來說.
按她的水平來說.
似乎沒有資格說一個這麼深奧的敬拜問題.
她要孤伶伶.
在很熱的時候.
避開所有人的目光.
走去辛辛苦苦地取水來用.
但主耶穌特意去到她面前.
主耶穌雖然是疲乏.
但在她的井旁仍然跟她說話.
問她可不可以給我水.
這句話其實包含了一個很豐富的憐憫.
很大的謙卑.
亦是很重要的.
向一個從來有蛇孔的女士.
打開她的心扉.
給我水喝.
原來猶太人和撒瑪利亞人.
男性和女性.
當時根本不會有任何交往.
但主耶穌不讓這一切向她取水.
然後引導她說.
你需要的是活水.
如果你有活水.
即是你能夠自來水.
就不需要出來打水.
那位女士說.
那給我吧.
主耶穌說 且慢.
你要嗎?.
叫你的老公來吧.
為何會這樣?.

$^{161}$她馬上說.
大家很熟悉的.
我怎樣?.
有還是沒有?.
沒有.
主耶穌有沒有直斥其非?.
沒有.
主耶穌有時很幽默的.
他說 你說得沒錯.
你說得對.
你曾經有五個.
現在有的都不是老公.
這位女士就無所遁形.
難道你是?.
然後她就轉話題.
當人面對自己的處境時.
自己一些黑材料時.
人總想轉話題.
不要說這些.
我們的祖先說在這裡拜.
你們就說在那裡拜.
那怎麼辦?.
耶穌沒有因為這個緣故.
繼續窮追猛打.
將他的道德生活問題.
耶穌就繼續和他談敬拜的事.
主耶穌亦老實不客氣.
就像我們今天這段.
《約翰福音》第四章十六節開始的經文.
你們拜的你們不知道.
你們全錯了.
撒瑪利亞的種族.
他們的祖先所傳下來的.
認為是在基里山山上來敬拜的.
這是錯的.
而且他們敬拜的對象.
亦都可能搞錯了.
全盤否定了他們所說的話.
主耶穌說我們拜的我們知道.
因為救恩是由猶太人出來的.

$^{201}$有根有據十足把握.
有先知預言的傳統.
有三百多處彌賽亞的預言.
一直在猶太人.
特別是阿伯拉罕大衛的族系.
一直傳下來.
這位創天造地的上主.
人以為不需要敬拜的Personal God.
這位活的上主從來沒有停止他的工作.
他在滄西的時候.
一直成就這個宇宙奇妙的.
到今時今日.
仍然憑著他權能的話語托住萬有.
上帝是這個God of the Gaps.
還是God of the Sources.
還是God of the Origin.
是這位萬有之源.
不是因為我們的知識達不到.
所以就把他拉下來填淡.
而是因為他根本就是萬有的本源.
離開了他.
你有多少的知識.
你都會變成一個無知的人.
而且必定離開了上主.
人的生命就陷入這個黑暗裡.
企圖將上主排走在你的思想系統.
排走在你的生活裡.
其實你想排走那一位.
可能不是你真的想排走那一位上主.
可能你是在排走你被錯誤來闡述的.
被錯誤來描述的.
或者你錯誤地幻想的那一位上主.
真正的那位上主.
是阿伯拉罕以撒瓦國.
我主耶穌基督的神.
救恩由猶太人那裡出來.
我們所拜的我們知道.
現在我們這個時候就到了.
要敬拜他的.
是以靈以誠來拜天父.

$^{241}$拜父.
在靈裡和真理裡面拜父.
這是一個豁然開朗的啟示.
我們可能不是很感覺到.
但是當時主耶穌是和這位撒瑪利亞婦人.
很清楚地說你們全錯了.
然後他就說不在你們那座山.
也不在我們那座山.
突然間一個爆發性的啟示.
告訴他們是以靈以誠.
破除了種族的限制.
破除了地理的限制.
破除了禮儀的限制.
然後就說拜父.
很清晰的對象.
那一位至高的上主要拜他.
然後拜他是以靈以誠來拜他.
因為為何要這樣拜呢.
父要這樣的人拜他.
是父的心意.
如果你要敬拜.
你要這樣敬拜.
那你說以前說這座山那座山.
那是錯的嗎.
那些只不過是短暫過渡時期.
臨時措施.
真正的敬拜.
父要這樣的人來敬拜.
是父要.
我看到這個父要的時候.
我很大的感觸.
很大的感動.
其實主耶穌為何會去到.
撒瑪利亞的井旁.
主耶穌為何會這麼有耐性.
和一位這樣的女士.
不避嫌的.
你和她這樣聊天.
不就是來尋找嗎.
她找我.

$^{281}$她找你.
無論你的處境.
是多麼的惡俗.
多麼的卑微.
多麼的黑暗.
多麼的不堪.
你多麼想逃避所有的人.
你不單止寫恐.
你是恐懼面對她.
其實亞當的後裔.
全部都是害怕的.
亞當的後裔.
全部都是黑材料.
每個人都想逃避.
但是那一位愛我們的上主.
父要這樣的人拜他.
原來的文字翻譯出來.
不見了一個很重要的元素.
我希望我沒有說錯.
因為我也認識希臘文.
我們讀希臘文第一課的時候.
就知道那個尋找是.
Seiteo.
英文是The Father Seeks.
Seek.
我問一下明威主席.
這個Seek的過去式是什麼.
是不是Seeked.
我有一個朋友英文很複雜.
Go When Gone這麼複雜.
應該是Go Goedet.
Seek and sought.
父是尋找.
這麼實際的字在那裡說.
爸爸是在尋找.
找誰.
不是找薩瑪利亞.
不是找那些什麼.
法利塞.
彭斯.

$^{321}$祭司.
不是找那些.
不是.
是在找薩瑪利亞夫人.
這個是愛的尋找.
這個是對迷失混亂沉淪人的尋找.
他尋找我們.
我們才有盼望.
他尋找我們.
我們才懂得去敬拜他.
是主耶穌主動去找他.
廢寢忘餐.
不辭勞苦.
輸尊降貴.
我們看見這位女士的經歷.
五個丈夫曾經.
第六個都不是丈夫.
似乎如果他未喝活水的時候.
他應該會繼續沉淪下去.
其實他在尋找什麼.
人都是不斷地尋找.
能夠滿足自己心靈的東西.
但是若果沒有活水.
是永遠都會繼續乾涸下去.
好像奧古斯丁這位曾經淪落的浪子.
他說上主啊.
你是為著你自己創造我們.
叫我們若果不回歸於你.
我們不得安息.
敬拜原來是上主來尋找.
而我們知道唯一真正.
配受敬拜的對象.
就是這位的天父.
然後主耶穌說.
上帝是靈.
God is spirit.
沒有A的.
不是上主是一個靈.
上主是靈.
是在說什麼呢.

$^{361}$我相信是在說.
祂是超乎萬有的.
在萬有之上的.
不是由人手所做的.
更加不需要藉著人手供奉的.
就好像仕途保羅說.
那一位永恆至尊的永世君王.
不惱壞不看見.
獨一的神.
直到永永遠遠.
我讀基督教中學的時候.
雖然自己不是很認識上主.
但很記得其中一首.
我們的讚美詩歌.
就是immortal invisible.
God only wise.
這位不能惱壞.
不能看見獨一的真神.
是的.
神是靈.
所以那些很愚昧的.
拿神給我看.
你的神是怎樣的.
其實你要知道.
祂是超越一切.
是看不到.
是看不到永恆.
永耀的那一位.
但是祂是靈.
靈是什麼意思呢.
靈是虛無飄渺.
什麼都是神.
有些人說到好像神就是.
虛無飄渺.
我相信主耶穌在說.
我的父是靈的時候.
是在告訴我們.
祂是監察一切.
超越一切.
至高至盛至尊的一位.

$^{401}$而那些拜祂的人.
是要以靈以誠來拜祂.
用少少時間來思想.
我們大家都非常熟悉的.
以靈以誠來拜.
是什麼意思呢.
在我們今天.
我們很容易.
用幾樣東西來說.
以靈來拜.
以人的最高的那一部分.
因為人是分三元的.
是靈.
是魂.
體.
體就是做這些勞動.
活在世上.
魂就是做一些美術.
思考.
哲學.
感情.
靈就是敬拜.
靈修.
這些人.
好像今天還是喜歡這樣.
將人分為三份.
我不是很喜歡.
因為我分不清我的靈.
和我的魂到底有什麼分別.
怎麼分呢.
英文翻譯希臘文和希伯來文.
這個Numa.
這個Pshuke.
這個Rua的時候.
你會發覺很多時候.
靈與魂是交替來用的.
其實都很難分的.
其實真真正正的.
要怎樣以靈來敬拜呢.
等一下再說.

$^{441}$有些人就說.
不是.
因為主耶穌在說.
聖靈.
是在聖靈裡面.
來敬拜.
有些所謂靈因的弟兄姊妹.
會說.
你一定在聖靈裡.
有些超自然的經歷.
有些超自然的敬拜.
譬如你能夠有.
謊言的敬拜.
有神蹟奇事的出現.
有很亢奮的敬拜.
那才是.
到底主耶穌是否在說這些呢.
那才是.
才是證明.
如果說到這個靈.
就是人要以自己最深.
最深的最重要的來敬拜.
我絕對同意.
如果你說人需要聖靈.
然後有真正的敬拜.
我絕對同意.
因為如果沒有基督沒有聖靈.
你真的不能成為.
一個真正的敬拜者.
但到底這個經文.
當主耶穌和撒瑪利亞婦人.
在談這個事的時候.
他有沒有放這麼多東西進去呢.
我相信這個就沒有.
就不用.
不過我們今天補充一兩句.
我們當然要盡自己.
最深處的.
來敬拜.
然後聖靈運行在聚會當中.

$^{481}$感動我們的時候.
我們當然也要.
開放讓聖靈來帶領.
我們的敬拜.
不過我時常喜歡說這句話.
靈因運動所為.
一件衣服.
你會全躺下.
有些人說這是聖靈工作.
我不知道.
我不知道是不是.
我會很謹慎.
而且當看這些人.
噴射飛機的時候.
我就不太想跟著.
而且這樣的敬拜.
一定要是.
很失常地.
很Ecstasy.
靈魂出竅的敬拜.
我們就會成為一個.
進來都不知道會發生什麼事的群體.
我不相信.
那個是.
我尊重聖靈.
但我拒絕.
將所有東西.
放進去掃.
明白我想說什麼嗎.
寧願.
按著聖靈的感動.
繼續在生命當中.
有結出聖靈的果子.
有理性.
有神性.
有尊嚴地.
來敬拜它.
不過.
靈因運動過去幾十年.
而且現在還沒停.

$^{521}$是繼續對教會.
都有一種振奮的作用.
明白嗎.
給我一點反應吧.
不知道.
如果我們敬拜死咕咕的.
有些比較靈因的弟兄姊妹.
就會快樂一點.
Hallelujah.
起身舉手唱歌跳舞.
有沒有錯.
我覺得沒有.
因為一群人坐在那裡.
就像一碗飯.
就像你的信仰很無趣.
很多精彩的東西.
就要表達出來.
有什麼問題嗎.
不過不要被人.
拉到錯謬的裡面.
其實我一向以為.
是靈與真.
兩樣是平衡的去.
但我發覺.
可能未必是這樣.
可能重點.
都是在靈與真理.
因為在靈裡.
在靈裡.
在真理裡.
兩個在文句上.
似乎不是平衡的.
而是一個在字.
包了靈與真.
會不會最簡單.
來說就是.
面對想逃避自己真相的人.
主耶穌告訴你.
來到天父面前你不能離開.
你是全人的.

$^{561}$敞開的.
來按照著這位父.
你是怎樣.
來敬拜他.
按照他啟示你的真理.
你就這樣來敬拜他.
如果論到我們自己.
有什麼實際的作用.
我覺得大有可為.
大有作用.
因為我們成為基督徒之後.
或者未成為基督徒的時候.
我們其實時常想將自己的黑材料.
收起來.
但天父告訴我們.
你不能離開.
他想告訴你.
你不用離開.
你最重要的是認識我.
按照我當德的榮耀.
來侍奉我.
按照我的榮美.
來敬拜我.
這樣你說好不好.
這樣是不是簡單直接.
清脆理諾.
這樣我就多謝天父.
連我這樣的人你都找我來敬拜你.
你都教我認識你.
你都教我認識你是一位怎樣的主.
然後你讓我可以.
以我的真.
來坦然無懼的.
進到你的面前.
當人真正面對上主.
真正敬拜他的時候.
You are what you worship.
你真正認識上主的時候.
你真正敬拜了上主的時候.
你會欣賞這個敬拜.

$^{601}$而你的生命.
全然的繼續更新改變.
Hallelujah.
我是如此相信.
但願我們都如此.
去到學習.
神是零.
爸爸是零.
敬拜他的.
以零與真.
來敬拜他.
但願我們真的可以.
盡心盡性.
盡意盡力.
按照基督所啟示.
按照聖經所教導的.
來敬拜上主.
我很喜歡我老師J.I.Packer所教導的.
他說.
我們是再生靈女.
透過基督.
親近天父.
In the spirit.
Through the Lord.
To the Father.
In the spirit.
Through Jesus.
藉著基督進到父面前.
我覺得我們應該.
整個生命都被.
三一神全然的擁有.
然後在敬拜裡面.
不需要好像.
某一些偏差了.
只是單單說.
基督什麼都不說.
單單說天父什麼都不說.
單單說聖靈.
什麼都不說.
這三樣東西在教會歷史文化裡面.

$^{641}$都可以出現問題.
包括神體一元論.
包括唯尊基督.
獨尊基督論.
獨尊聖靈論.
全部繼續這樣走的時候.
都會出現錯誤的.
唯獨在基督的啟示裡面.
主耶穌教我們的.
是要以靈.
以聖來敬拜父.
所以多些.
學習這樣的敬拜.
侍奉.
主耶穌這番說話.
對於.
撒瑪利亞婦人.
有什麼幫助呢.
幫助她脫離.
祖先傳下來的限制.
幫助她脫離.
自己生命裡面的黑材料.
對她的克制.
包括帶領她.
進到父面前.
其實就是給她一個好的offer.
你可以成為一個.
真正拜父的人.
結果這位女士.
主都不要走回到聖靈裡面.
連自己的黑暗材料.
都不需要再遮蔽.
都不需要再逃避.
我相信她的生命.
自此以後會轉化改變.
什麼叫做敬拜.
我們要敬拜.
是怎樣敬拜.
敬拜就是五體投地.
敬拜就是.

$^{681}$耳嘴親近他.
敬拜.
其實上主一直.
在聖經裡面.
由會幕到聖殿到主耶穌那裡.
都是教導我們.
怎樣敬拜.
他是在找.
能夠敬拜他的人.
敬拜其實是.
生命的回歸.
整個人的.
真正的.
回到你生命的根源.
面前.
五體投地的.
來敬拜他.
讚美來敬拜他.
然後獻上自己.
盡心盡性.
盡意盡力.
Body Mind and Spirit.
身心靈.
全然的.
來敬拜他.
願你和我都能夠繼續學習.
做一個真正的敬拜者.
即是什麼.
什麼叫做真正的敬拜者.
要學習.
繼續認識.
聖經的真理.
認識聖經所啟示的上主.
是怎樣的.
認識我們可以怎樣坦然無懼.
不再虛偽.
不再抱著錯誤.
不再塑造.
自己的偶像.
做出一個假神.

$^{721}$雖然他的名字.
繼續叫做耶穌.
但已經扭曲了.
繼續說我是敬拜三一真神.
裡面的餡已經被你.
叫做掉了包.
是偷龍轉鳳.
變走了.
親愛的兄弟姐妹.
在今天這個世代.
很多似是而非的道理.
出現了.
我們需要憐憫.
需要謙卑.
在主耶穌基督的面前.
在聖靈的感召之下.
在聖賢的光照之下.
親近天父.
活在真理之下.
是很重要.
順便一說.
法利塞人.
認識很多東西.
但主耶穌不喜歡他們.
為什麼?他們搞很多敬拜.
是敬拜專家.
但主耶穌說.
這百姓用嘴唇.
親近我.
心腳遠離我.
我好憐憫.
勝於你們的現實.
你們穿著.
光彩的外袍.
但心裡藏污納垢.
醜陋過.
一個死人的墳墓.
到底我是一座聖殿.
還是一座墳墓?.
真的敬拜.

$^{761}$可能告訴你.
不是聚會.
不是唱歌.
不是奉獻.
不是激情.
那我以後不回來了.
不用了.
那又不要想大了.
真正的敬拜.
是要唱歌.
真心的去歌頌.
真正的敬拜.
當然要奉獻.
為著上主的榮耀.
但更重要的是.
敬畏.
懺悔.
離開罪惡.
信服.
以主為尊為王.
恭傾上主的聖言.
並身體力行的回應.
這是敬拜.
What is the chief end of man?.
人之所以為人.
最重要的是什麼?.
願榮耀尊貴.
恩著你和我.
這一座聖靈的殿.
無時無刻的信服.
依靠.
尊榮他.
生活就是敬拜.
將身體現象當作活祭.
是聖潔的事實.
所起的.
你們如此敬拜.
是理所當然的.
你們如此鼎禮侍奉.
是理所當然的.

$^{801}$赴尋找.
敬拜者.
真的敬拜者.
但願我們都成為一個.
這樣的敬拜者.
唱一句歌給你聽.
是仰望奇妙十家.
其中一節的歌.
那個英文的歌詞是這樣的.
Love so amazing.
So divine.
Demands my soul.
My life.
My all.
愛既如此.
神聖奧妙.
配得我心.
我命.
我一切.
但願我們都成為一個.
繼續去敬拜的人.
一齊祈禱.
多謝天父.
藉基督來尋找.
失喪的人可以回歸.
沉淪的人可以得救.
願你繼續施恩給我們.
教我們藉著聖靈.
藉著基督.
來到父面前成為一個.
真誠敬拜理事奉利的人.
如此都好高奉耶穌基督成名.
阿門.
\newpage



\chapter{黃紹權}\label{ch:preacher10}
\begin{multicols}{3}
\minitoc
\end{multicols}
{ \scriptsize


\begin{xltabular}{\textwidth}{|p{0.15\textwidth} p{0.6\textwidth}|p{0.07\textwidth} p{0.1\textwidth}|}
\hline
馬太福音 27:1-10 & \hyperref[sec:499K9je19EI]{蛇鼠一窩的假敬虔 (馬太福音27\_1-10) -  黃紹權牧師 [馬太福音信息系列 - 第143講]} & 2025-01-08 & \href{https://youtube.com/watch?v=499K9je19EI}{\texttt{ 499K9je19EI}} \\
馬太福音 27:11-23 & \hyperref[sec:ZN4O4BAmHMA]{沒有"不關我們的事"(馬太福音27\_11-23) -  黃紹權牧師 [馬太福音信息系列 - 第144講]} & 2025-01-19 & \href{https://youtube.com/watch?v=ZN4O4BAmHMA}{\texttt{ ZN4O4BAmHMA}} \\
馬太福音 27:24-31 & \hyperref[sec:HaaLhKYBRSg]{一群全勝的輸家(馬太福音27\_24-31) - 黃紹權牧師  [馬太福音信息系列 - 第145講]} & 2025-01-21 & \href{https://youtube.com/watch?v=HaaLhKYBRSg}{\texttt{ HaaLhKYBRSg}} \\
馬太福音 27:33-44 & \hyperref[sec:oCpi7n8ictU]{我們/他們是怎樣的人? (馬太福音27\_33-44) - 黃紹權牧師  [馬太福音信息系列 - 第146講]} & 2025-02-02 & \href{https://youtube.com/watch?v=oCpi7n8ictU}{\texttt{ oCpi7n8ictU}} \\
馬太福音 27:45-56 & \hyperref[sec:7upP8JmD6zY]{愛護我們的比親人更知心 (馬太福音27\_45-56) - 黃紹權牧師  [馬太福音信息系列 - 第147講]} & 2025-02-09 & \href{https://youtube.com/watch?v=7upP8JmD6zY}{\texttt{ 7upP8JmD6zY}} \\
馬太福音 27:57-61 & \hyperref[sec:MfR5_HAo14I]{沉實的門徒  (馬太福音27\_57-61) - 黃紹權牧師  [馬太福音信息系列 - 第148講]} & 2025-02-13 & \href{https://youtube.com/watch?v=MfR5_HAo14I}{\texttt{ MfR5\_HAo14I}} \\
\hline
\end{xltabular}
}
\newpage



\section{馬太福音 27:1-10}
\label{sec:499K9je19EI}
\textbf{蛇鼠一窩的假敬虔 (馬太福音27\_1-10) -  黃紹權牧師 [馬太福音信息系列 - 第143講]}
\newline
\newline
連結: \href{https://youtube.com/watch?v=499K9je19EI}{\texttt{ https://youtube.com/watch?v=499K9je19EI}} ~~~~ 語音日期: 2025-01-08 
\newline
\newline
\hyperref[sec:lTGVgidxHms]{< < < PREV SERMON < < <}
~
\hyperlink{toc}{[返主目錄]}
~
\hyperref[ch:preacher10]{[返講員目錄]}
~
\hyperref[sec:ZN4O4BAmHMA]{> > > NEXT SERMON > > >}
\newline
\newline
馬太福音 27:1-10
\newline
\begin{longtable}{cl}
\hline
\hline
章節 & 經文 (和合本修訂版)\\
\hline
27:1 & \begin{tabularx}{0.7\textwidth}{X} 到了早晨,眾祭司長和百姓的長老商議要處死耶穌, \end{tabularx} \\ \\ \relax
27:2 & \begin{tabularx}{0.7\textwidth}{X} 就把他綁著,解去,交給彼拉多總督。 \end{tabularx} \\ \\ \relax
27:3 & \begin{tabularx}{0.7\textwidth}{X} 這時,出賣耶穌的猶大看見耶穌已經定了罪,就後悔,把那三十塊銀錢拿回來給祭司長和長老, \end{tabularx} \\ \\ \relax
27:4 & \begin{tabularx}{0.7\textwidth}{X} 說:「我出賣了無辜人的血有罪了。」他們說:「那跟我們有甚麼相干?你自己承當吧!」 \end{tabularx} \\ \\ \relax
27:5 & \begin{tabularx}{0.7\textwidth}{X} 猶大就把那銀錢丟在殿裡,出去吊死了。 \end{tabularx} \\ \\ \relax
27:6 & \begin{tabularx}{0.7\textwidth}{X} 祭司長拾起銀錢來,說:「這是血價,不可放在聖殿的銀庫裡。」 \end{tabularx} \\ \\ \relax
27:7 & \begin{tabularx}{0.7\textwidth}{X} 他們商議,就用那銀錢買了窯戶的一塊田,用來埋葬外鄉人。 \end{tabularx} \\ \\ \relax
27:8 & \begin{tabularx}{0.7\textwidth}{X} 所以,那塊田直到今日還叫做「血田」。 \end{tabularx} \\ \\ \relax
27:9 & \begin{tabularx}{0.7\textwidth}{X} 這就應驗了先知耶利米所說的話:「他們用那三十塊銀錢,就是以色列人給那被估定的人所估定的價錢, \end{tabularx} \\ \\ \relax
27:10 & \begin{tabularx}{0.7\textwidth}{X} 買了窯戶的一塊田;這是照著主所吩咐我的。」 \end{tabularx} \\ \\
[1ex]
\hline
\hline
\end{longtable}
$^{1}$好 第二次煤早晨了.
我原本想今天空多吉少.
因為我病了一個星期.
聲音也沒什麼氣.
但是昨天開始煤溶.
聲音回來了.
和大家一起在主的道理上學習.
如果大家留意到今天崇拜的程序表.
我們有一個新教會的主題.
叫做「不用玷污自己的神國子民」.
出自於但以理書第一章八節.
這是今年的主題.
這八節大概還有八節就講完了.
講完八節後就會進入一卷舊約書卷.
但以理書.
其實我計劃了一年.
但未有機會和大家詳細介紹.
今年會花多一點時間和大家一起看一卷.
可以說是舊約聖經和新約啟示錄.
可以比的一卷書卷.
希望我們能夠一起.
在神的話裡一起看看.
上帝的旨意是怎樣.
在這個世界上一步一步的陳明和剖白.
在今天講到這裡.
讓我們一起同心禱告.
仰望主我們的神.
賜給我們天賦.
多謝你讓我們能夠有生命氣息.
來到新的一年裡.
打開新一年的時縫.
讓我們每一個屬主的子民.
神國的公民.
我們每一個都能夠在新的一年裡.
做好一個更全面準確的準備.
能夠拿捏到上主的心思意念.
來到往後的日子當中.
繼續作主你所交託的時縫.
求你的帶領.
讓我們在今天講道當中.

$^{41}$透過聖經給我們一個很重要的提醒.
好使我們能夠謹慎度日.
免得我們只有一個虛名的教徒的名稱.
而失去了真正基督徒本身應有的生命是怎樣.
求你的帶領幫助.
我們為疾病裡的弟兄姊妹交在主的一手.
讓追求主道真理的生命.
在他們裡面仍然火旺地燃燒著.
為弟兄姊妹身邊的病患或憂傷.
懇求主一一的施安慰.
讓我們能夠在得時不得時.
都同樣學會順靠順服依靠主.
我們將這段時間完全交上.
祈禱奉靠主耶穌基督.
北聖的名次而求.
阿們.
在2024年11月6日.
剛剛過了不久.
大家都會留意到.
我們南邊的國家美國.
終於選了一個新的總統出來.
這次選舉的結果.
可以說是在美國歷史上一個很特別的發展.
因為無論是總統的位置.
或者總票數的數目.
即是全國的國家的國民投票.
或者是美國的眾議院.
或者是參議院.
以及州長.
都完全可以說是被共和黨的候選人包攬.
今天我不是和大家一起討論美國的政治.
不過我卻留意到一件.
在這個選舉之後有很多的評議.
在當中指出.
這個共和黨的候選人勝出之後.
無獨有偶.
很多時候我們看到輸的一方理論上是承認自己的落敗.
但是在他們宣佈自己輸了選舉之後.
他們沒有直接說他們真的輸了選舉.
反而他們提出了一件令我非常擔心的事.

$^{81}$甚至我們作為信徒應該更加留心的事.
他們反而出來說要繼續進行爭戰.
甚至對方落敗的一方.
這個民主黨的女候選人.
用神的名字出來說要繼續爭取很多不同的自由.
才是真正美國民主的意義.
而且當中有很多人非常和應.
他們覺得他們要繼續爭取的是要有墮胎權.
變性權.
吸毒權.
女性的女權.
以至於指出如果他們不繼續爭取的話.
整個美國就沒有了大愛.
當他們說到這一點的時候.
我就更加擔心.
因為基督教一向都經常說愛.
在我自信主80年代的時候.
以至於現在的時代.
基督教仍然廣泛地說要有大愛.
上帝就是愛.
其實這個陳述在我很早信主的時候.
我已經發覺它有它本身的問題.
上帝是不是只說愛呢.
而什麼叫做愛呢.
所以在聖經裡面.
我很多時候很賣力去研讀的就是.
找回聖經裡面所說的愛是什麼定義.
如果我們不在聖經裡面找回愛的定義的話.
我們很泛濫地去用愛字眼的時候.
就很容易跌入了一個人為所定的愛.
就好像我所看到的這些輸了的民主黨的候選人.
他們所說的愛是指原來支持很多上帝不起悅的事情.
那份權利的所謂愛.
於是造成了什麼呢.
造成了過去在美國這麼多年來.
甚至現在在加拿大都有這個現象.
就是很多非法移民甚至罪案激增的情況.
在這個星期真的嚇我一跳.
在多倫多除了打劫金舖.
好像已經是常規一樣.

$^{121}$大家如果一會兒完了崇拜.
如果去Markville Mall Fairfield Mall.
你小心點.
因為那裡有兩間金行劫完又劫.
好像劫不停一樣.
現在的賊大膽到可以用.
剷泥車剷銀行的提款機.
可以放火燒銀行.
為什麼會落到這種現象呢.
我自己所看到的反而是所謂一路高舉的.
所謂大愛包容這種思想.
原來根本就不是出於聖經裡面的思想.
而很可惜今天我們的基督徒.
很大部分人自己不知道原來所主張的東西.
其實不是上帝所主張的東西.
反而我們看到的.
那些支持什麼權利什麼墮胎權等等這些事情.
反而是魔鬼撒旦的手段.
怎樣去討好人心.
大兄姐妹.
看到這個世界的發展.
看到今天的教會的神學的墮落.
就讓我想到今天這段聖經裡面.
裡面所說到的主人翁.
包括眾祭司長.
長老們.
以及一個我們經常提及的人物.
叫加略人尤大.
他們的靈性可以說是一個非常虔誠的表現.
我一會兒去仔細看看.
他們很虔誠的.
但他們的虔誠很可怕.
使我們覺得原來這種虔誠.
漸漸在今天的教會裡面.
甚至今天流露在很多基督徒裡面.
我相信今天這段聖經.
可以成為我們今天很多基督徒的一個.
很重要的靈性上的反面教材.
能夠幫助我們去逃避一個.
很容易墮入的陷阱.

$^{161}$就是一個蛇鼠一窩的假敬虔的狀態.
今天的教會實在太多假敬虔的人.
聚在一起.
蛇鼠一窩.
在做著很多好像很虔誠的事情.
但其實卻是上帝不喜悅的事.
我們一起回到這段聖經看看吧.
馬太福音第27章.
第一至第十節.
剛才Sandy姐妹已經幫我們讀了一次.
不如我們自己再打開.
再自己親身讀多一次.
因為自己讀就會深入很多.
入心很多.
我們一起打開聖經讀讀吧.
馬太福音第27章.
第一至第十節.
找到之後一起讀吧.
「到了早晨,眾祭司長和民間的長老.
大家相議要致死耶穌.
就把他捆綁,戒去.
交給巡撫比拉多.
這時候,賣耶穌的猶大看見耶穌已經定了罪.
就後悔.
把那三十塊錢拿回來給祭司長和長老.
說我賣了無辜之人的血是有罪了.
他們說那與我們有什麼相干.
你自己承擔吧.
猶大就把那銀錢吊在殿裡出去吊死了.
祭司長拾起銀錢來.
說這是血價.
不可放在婦女.
他們相議就用那銀錢買了瑤湖的一塊田.
為要埋葬愛鄉人.
所以那塊田直到今天還叫做血田.
這就應驗了先知耶利米的話.
說他們用那三十塊錢.
就是被古定之人的價錢.
是以斯列人中所古定的.
買了瑤湖的一塊田.

$^{201}$這是照著主所吩咐我的.
這段聖經先前說了什麼呢.
我們看一看吧.
很簡單的看一看吧.
原來前文剛剛才說完.
彼得在這裡三次不認主這段經文.
上次我已經和大家說到.
其實已經是三個星期前的講道了.
如果不記得可以回去看.
精斷的馬太福音來到最後的地步.
其實是介紹一個人物.
叫做西門彼得.
也是耶穌基督的門徒.
這位門徒的靈性是怎樣的狀態.
這是馬太故意提出的主旨.
也是我們每一個讀這段聖經的時候.
應該集中看的人物.
而不是純粹看耶穌基督怎樣受釘死.
人們怎樣去誣衊耶穌等等.
其實是在透露一個墮落的信徒.
馬太的主旨其實是要告訴我們.
每一個信徒即使你多麼的聰明.
但你也會有墮落的時間.
基督徒不是長期都是靈性很高超.
會有靈性低潮的時間.
那你說牧師就不會了.
我可以很肯定地告訴你.
我作為牧師也可以告訴你.
我有靈性低落的情況.
只不過我靈性低落的情況.
我知道怎樣可以回到上帝的面前.
重新去鞏固.
重新去認罪去建立起來.
這個工作因為我明白到.
上帝是給人有意思的.
最近不知道發生什麼事.
在我家有很多熟齡書籍.
周到這樣浮游在我身邊.
其中一本書浮游在我身邊.
可能因為我女兒在看書.

$^{241}$標題叫做.
The God Always Has Second Chance.
一位永遠讓人有機會的神.
基督徒不是屬靈超人.
基督徒會跌倒的.
但基督徒在跌倒過後.
仍然給我們機會可以回轉.
如果我們作為信徒知道這一點.
我們的生命就會知道怎樣繼續掙扎.
不然的話.
我們的狀態就好像今天我們所看的.
其中一個人物就是加勒人猶大.
所以整段聖經的主旨.
其實落在一個失落的信徒.
怎樣被挽回.
上文我有說過.
彼得的失落其實很諷刺地.
或者帶出一種諷刺的意味.
就是他三次不認主的裡面.
都是被侍女所帶到墮落.
這位失落的信徒怎樣回轉起來呢.
很特別的也是用侍女.
用一些女士把他挽回上來.
不過在這個過程當中.
就加插了今天這段經文.
所以這段經文很有趣地.
夾在彼得的靈性.
由走向下坡到完全散散的狀態之下.
再重整起來的過程當中.
一個片段.
這個片段有一個很重要的目的.
就是要將一個真正在神的裡面.
會跌倒的信徒.
但也會成長翻身的信徒的生命.
和一些根本不會翻身的信徒.
他們作一個對比.
我們一起去看看這段聖經說了些什麼.
我會分為三個部份和大家一起去看.
第一部份第一和第二節.
第一第二節說了些什麼.

$^{281}$到了早晨眾祭師長和民間的長老.
大家相議要致死耶穌.
就把他捆綁戒去.
交給巡撫比拉多.
在這段聖經裡面.
我給了一個副題.
叫做盡情腐爛的聖品人.
英文我給一個題目.
叫做Total Depravity of the Saints.
這個聖品人.
這是天主教裡面的一個常用的字眼.
天主教很喜歡封聖.
大家知不知道為什麼要封聖.
因為要用人版去讓你去景仰.
甚至將這些聖人變成和上帝差不多.
你可以景仰這些聖人.
甚至模仿這些聖人.
就等同於你和上帝一樣.
所以天主教有這樣的神學背景.
但原來這些聖品人.
或者用今天說的傳道人.
再說得闊一點.
基督徒.
上帝完全將他們的腐爛程度.
盡情發揮出來.
我們看看這段聖經說什麼.
一開始記載到.
當彼得還在掙扎.
為什麼自己三次不認主的情況下.
另一邊廂馬上一大清早.
眾祭司長和民間的長老.
就出來相要至死耶穌.
這句話有很多地方我們讀就很快跳過.
但如果我們仔細去看的時候.
第一個時間的標記.
一大清早.
如果這本聖經是你的.
或者你用電子聖經的話.
一大清早這個字眼.
我覺得很值得你勸起.

$^{321}$因為遲些之後.
當耶穌被釘死.
沒多久.
這些祭司長又來騷擾秦府.
又是一大清早走來.
大家可以在後一點點已經看到.
第27章的後一點點.
第62節開始.
你就會看到他們的發展.
他們一大清早就要去找秦府.
比拉多.
做什麼事要這麼大清早就要去找.
正代表他們心急如焚.
他們怕耶穌死的時間不夠快.
所以一逮捕了耶穌之後.
他們一大清早就去見比拉多.
希望盡快.
怎麼樣.
致死耶穌.
致死耶穌這個說話.
如果我們稍稍回顧第26章第59節.
我們已經知道耶穌老早預言了這件事情發生.
而在馬太福音第10章第21節裡.
耶穌老早已經說了一個事實.
他說要把兄弟們交到致死的地步.
如果單純在第10章第21節裡.
我們就會覺得耶穌無情地說了一句.
我們都不明白為什麼耶穌會這樣說.
來到這裡我們明白了.
耶穌作為猶太人的其中一份子.
現在就被猶太人的兄弟致死.
怕他死得不夠快.
所以一大清早就聯合了一個外邦人.
羅馬人.
買官買回來的比拉多.
然後盡快把他弄死.
廣東話好說.
盡快把他弄死.
所以聖經的記載是怎樣的.
就把他捆綁.

$^{361}$交給巡撫比拉多.
因為他們希望盡快把耶穌交到比拉多的手上.
以比拉多的權力斬殺耶穌基督.
或者處決了耶穌基督.
來到這一點.
我想重提一個人物.
這個人物叫做仕途堡羅.
在《使徒行傳》第21章第33節裡.
曾經他這樣在《使徒行傳》被記載下來.
當他去耶路撒冷的時候.
他同樣被人用兩條鐵鏈捆綁.
然後押到千夫長裡面.
然後在第23章第23節裡.
就介紹到仕途堡羅被押送到巡撫菲力斯進行審判.
為什麼我要提仕途堡羅給大家聽呢.
其實就是想告訴大家弟兄姊妹.
信徒被逼迫是被命定的.
很多人喜歡信徒是依著上帝的心思意念而活的.
如果我們真的依照上帝的心思意念而活的話.
我們的遭遇和耶穌是沒有什麼分別的.
只不過今天不一定有一個巡撫的人來審判你.
而是你的上司.
不過從這段聖經裡面我們發現到一個事實.
上帝在這個過程裡面沒有像耶穌先前所說的.
可以派天兵天將來完成保護他的工作.
他沒有這樣做.
上帝也沒有進行不義行為.
於是激殺這些人物.
沒有這樣去做.
同樣道理在過去的日子.
如果大家看到在這個世界裡面.
容許同性戀 容許咳大麻.
容許那些偷東西不用坐牢的人.
等等這些罪案的人.
其實在我們心裡面有一種很不公義的反感在他當中.
我認為上帝你最好馬上將這些人去激殺.
趕他們下台.
不用說哪一個我們在加拿大最好說的.
我不是說政治不過我想說給大家知道.
我們加拿大的總理主張這些同性戀變性人.

$^{401}$主張人怎樣去選擇自己的性別.
在小學的時候就已經去做.
亦都怎樣去鼓勵人去咳大麻.
很多人以為他很好.
他滿足我們的慾望.
很多華人都是這樣想的.
因為他讓我移民過來加拿大.
所以我就投票支持他.
其實是自己擺一個陷阱在自己裡面.
將越來越多的人的腐爛暴露出來.
我們作為基督徒.
我曾經試過有些同工這樣說.
祈禱 祈求神將這個總理馬上拉下台.
最好這個總理不得好死.
請姐妹 這些禱告不合上帝的心意.
是 我們可以對罪惡有反感.
我們可以對不公義的事情表達不同意.
但我們同時需要看一件事.
上帝容許這些人出現是甚麼原因呢.
其中一個原因就是這段聖經出現.
讓祂盡情將罪惡暴露在人的面前.
當我們看見罪惡越來越在這個世界張牙舞爪的時候.
就告訴你和我.
這個世界根本就是在罪惡裡面.
如果我們以為基督徒在這個世界.
用神的道理 用行事.
就可以扭轉這個世界的罪惡可怕.
我可以告訴你是不可能的.
這個世界始終要廢去.
我們只是在這個世界廢去的過程當中.
如何仍然將它屬神的子民召喚回來.
以馬里環浸信會就是這個目的.
我們雖然不是很多人.
但我們在網上收聽的人為何會這麼多.
自從我們做了字幕之後.
我發覺還有翻譯.
有人連我們港獨的翻譯都翻出來.
為甚麼.
因為有些追求主導的信徒.
能夠體現真正的道理.

$^{441}$能夠讓他們得到堅固在他們的位置上.
而我們的角色不是在那個位置上去搞抗爭.
所以我從來不認同基督徒要搞抗爭.
基督徒要搞的活出真理講道理.
但這個世界是否可以讓你說道理.
這個世界不是聽道理的世界.
基督徒會遇見被迫罷的事情.
可能因為你站得正 行得正.
按照主的道理.
你可能升職不會找你.
在這樣的情況之下.
你會不會仍然站穩那個位置呢.
還是你反而在這個時候去妥協呢.
這班聖品人 文士 祭司.
而且他們不是一般.
是祭司長 現在不是一個.
眾多的祭司長 眾多的長老.
一齊去夾謀害耶穌.
在這樣的情況之下.
我們會不會說他們是主流.
他們當正 我們要順應他們呢.
還是我們在這個時候仍然站穩在上帝的說話裡.
或許我們的站穩改變不了這個世界.
但至少讓這個世界的人看見.
罪惡在這個世界彰顯橫行的時候.
仍然有什麼叫做真理在當中.
而這個分別就是馬太福音的作者.
馬太希望藉著彼得和其他.
尤其是加里人猶大和猶太人公會.
兩者最大的分別.
這是第一點.
我們看見上帝容許這個世界的腐爛聖品人.
盡情的腐爛.
第二件事 從第三節到第五節.
我們一起看見一個.
死也捆綁於自己唯我獨尊的人的思想是怎樣.
我們一起打開聖經.
我們再讀一下.
第三節到第五節.
我們一起讀.

$^{481}$(看見耶穌已經定了罪就後悔).
(那三十塊銀子拿回來給祭司長和長老).
(說我賣了無辜之人的血是有罪了).
(他們說那與我們有什麼相干).
(你自己承擔吧).
(猶大就把那銀錢丟在店裡出去弄死了).
我給了一個英文的副題.
The Almighty I.
聽清楚 不是The Almighty God.
是Almighty I.
今天的人.
尤其是罪人.
那個我.
是放到至高無上的位置.
當這群祭司長及猶太人公會的長老正如火如荼地陷害耶穌之際.
另一邊雙馬上又出現了一個人物.
這個人物正在後悔.
首先我要說後悔的字眼.
在希臘文裡叫meta-menorme.
和認罪悔改希臘文的字眼metonomeo是不同的字.
為什麼要在這裡放兩個希臘文呢.
不是拿出來包裝聖經說得好看一點.
不是 如果是這樣的話動機是錯的.
這兩個字截然不同是因為這兩個字帶來不同的意義.
如果指到這段聖經裡的後悔meta-menorme這個字眼.
是指到一個人改變了他的心意.
或者他有一種另外的想法恨錯難返.
他覺得有些事情做錯了.
但又不能回頭.
加雷耶猶大就是這個人物.
現在他發覺出了事情.
出了什麼事情呢.
他在這個時候賣耶穌的事情已經定了.
罪也定了.
誰的罪呢.
耶穌被賣的耶穌的罪.
還是加雷耶猶大自己賣耶穌的罪呢.
聖經在這裡用了一個比較含糊的陳述.
當然看來好像是耶穌被祭司們定了是死罪.
前文已經說過了.

$^{521}$他們怎樣去誣陷耶穌.
用一些假見證來誣陷耶穌的罪.
說不用說了 時間到了.
不用了 見證人都不用了.
好像是罪定了.
但其實這句話有一種含義.
含義同時說到這件事成就了之後.
有一種罪形成了出來.
他見到這個情況後他後悔.
哎呀 糟了 他這個時候.
他覺得這個情況有點不妥.
怎樣不妥呢.
他很熟悉聖經.
你看看他怎樣說.
他馬上把這三十塊錢.
拿回去給祭司長和長老.
然後跟他們說.
我賣了無辜之人的血.
是有罪了.
這句話精彩了.
原來我們一直不知道.
這位加雷耶猶大.
他對聖經這麼熟悉.
他走到這個時候.
內疚然後說我犯了錯.
我害死了無辜人的血.
這完全是他自己的內疚感.
他不是認罪.
他覺得不好意思.
用今天說的.
有些人做錯事不承認自己.
通常就這樣說.
對不起.
其實他不是認錯.
不好意思.
這視乎怎樣.
但他也沒有嘗試去修復這個錯誤.
他可不可以做修復這個錯誤.
如果他認罪.
他可以的.

$^{561}$他可以自己走上去.
跟他們說你不要害耶穌.
他沒有罪的.
他有沒有這樣做.
聖經完全沒有記載.
因為他自己只是想將自己的罪疚感脫離.
他覺得賣耶穌這件事.
不好意思.
這句話.
於是乎他就向他們給一個理由.
這班世師長和長老.
第四節.
我賣了無辜之人的血是有罪了.
這句話是有聖經根據.
哪裡有聖經根據.
申命記第27章第25節.
在申命記第27章第25節裡面這樣說.
受賄賂害死無辜之人的.
必受咒詛.
百姓都要說阿們.
申命記是舊約聖經裡面第二卷.
重新聖經裡面的教導律法書卷.
也可以這樣說.
昔日在以色列人歷史裡面.
上帝容許在王國分裂期間.
讓以色列人重新歸回上帝的一個很重要的書卷.
是一個不知名的聖經手抄本.
隱藏在聖殿當中.
後來被王發掘了出來.
這句聖經說明.
受賄賂害死無辜之人的必受咒詛.
於是乎.
加里猶大就會覺得死火了.
這次我就是這樣.
我收了賄賂.
多少錢?.
三十塊錢.
於是乎怎樣?.
在這樣的情況下.
我便受咒了.

$^{601}$糟了.
但你也猜不到.
加里猶大這麼有聖經基礎.
我告訴你.
在教會裡犯罪的人.
絕大多數.
對聖經有某程度的熟悉.
不少疑端也是從錯誤的聖經解讀而出現.
甚至我熟悉的.
中派.
橫教會.
我不怕老實說.
就算ABCM神.
他的神學都是錯誤的.
如果你看過他的書.
他自己自述.
而且宣道會我曾經試過.
有一個弟兄親口承認.
ABCM是怎樣扭曲了聖經的介紹.
而成立了一個中派.
他不是不熟悉的.
不過扭曲了.
加里猶大在這個時候.
他的靈性原來和這班祭司長和文士長老們很相似.
相似到一個甚麼地步呢.
大家都是很熟悉神的話.
不用說了.
大祭司,文士,祭司長老他們不懂聖經的話.
我想我這樣說出來也笑大人的嘴巴.
他們可以說是神學院的院長.
神學院最高的教授級人士.
但是他們的回應聽見.
在懊悔的.
在覺得有些後悔的.
在掙扎的.
猶大怎樣處理呢.
我們看看第四節下半節.
他們說那與我們有甚麼相干.
你自己承當吧.
靈姐妹.

$^{641}$如果你是一個真正認識追求主導的信徒.
你聽見牧師有一天講道.
你聽不明白.
有些地方你不了解.
你主動去問牧師.
你做字幕有時不明白.
你也會問我牧師那裡講甚麼.
你想問清楚.
牧師不單不回應你的問題.
牧師還多說了一句.
那與我有甚麼相干你自己承當吧.
用廣東話翻譯一次給大家聽.
關我們屁事.
你自己搞定它.
很俗吧.
文士祭司法尼賽人就是這樣說話.
關我甚麼事.
你自己搞定它.
自己搞定它這句說話.
滾開吧.
走吧.
於是加拿大怎樣做.
第五節就告訴大家.
猶大就把錢丟在電裡.
就出去吊死了.
在這裡我不想詳細解釋.
究竟是吊死還是爆肚而死.
因為這是另一個課題.
馬太福音告訴我.
他就出去吊死了.
當一個信徒或教徒有問題來問.
一個作教導指導幫助他能夠成長的人士.
即是這裡所說的文士祭司法尼賽人.
他不單不教他還叫他走出去.
形同推他死有甚麼分別呢.
弟兄姊妹.
你知不知道由去年開始.
我們有一個主要學叫聖經任你問.
是怎樣來的.
牧師沒有工作.

$^{681}$找些事情來做.
我告訴你聖經任你問這個課程.
北約兒童已經有兩個傳道人跟我說.
你想死嗎.
他跟我說.
很傷神的.
你知不知道信徒問甚麼問題.
你很熟悉聖經嗎.
弟兄姊妹.
我不是很熟悉聖經.
我不是想死.
就是因為我也不想你死.
如果你有問題.
我作為牧者.
我竭盡自己能力去回應你的問題.
或許我的回應未必能完全解答你的問題.
不過至少我作為牧者.
我竭盡自己能力.
這班祭司長和長老們.
竟然說自己出事幹.
在靈性上知道聖經說了些不對的.
想改.
你不給他們出路.
也不給他們好的指導.
你叫他們走.
他們就去死.
弟兄姊妹.
如果我們作為牧者.
不去幫助信徒的靈命成長.
只是在教會和他們嘻嘻哈哈玩玩.
笑笑的話.
甚至是飲飲食食.
去旅行的話.
我不是否定那些活動.
不過我只是想說.
你只是沉迷在那些不幫助信徒的靈命成長的時候.
你和推他去宿命永死有什麼分別呢.
你是不是想找一個這樣的牧者呢.
是不是需要找一間這樣的教會去參與呢.
弟兄姊妹.

$^{721}$今天在這個世界裡面有很多這樣的教會.
可以負責來娛樂.
讓你可以享受娛樂.
在這些滿足你個人喜好的事情上.
特別是一些屬靈的事情.
使你覺得自己很屬靈.
但其實他們個個都是各自為政.
家鄰人猶大.
只關心自己的內疚.
想脫離自己的內疚.
脫離不了.
祭司 法尼賽人 甚至是一些文士 長老們.
他們只是為了自己想鞏固他們的權力.
不想被耶穌威脅到他們的權柄.
於是想盡快整死耶穌.
他們將自己放在第一位.
他們哪有空理你呢.
你以為他們跟你同座.
是你的好朋友嗎.
你以為那些教會的牧者.
經常給你一些靈性上的娛樂.
你們以為我的靈性很好嗎.
其實到有問題的時候.
他們不是你的朋友.
他們只是曾經跟你一起的共犯.
你走在死亡的道路上.
他們是不會理會的.
而這段聖經更加反映了.
在舊約聖經裡的一個人物.
在撒慕爾記下第十七章第二十三節.
有一個人物叫阿希多弗.
阿希多弗曾經在亞沙隆的半邊.
大衛王的兒子亞沙隆的手下.
獻祭給亞沙隆.
如何去謀反他的父親大衛王.
但後來當亞沙隆開始轉移.
聽一個合心意的說明.
叫做胡西或胡斯.
阿希多弗就知道自己前途.
再沒有任何可以走的路.

$^{761}$於是他同樣回到故鄉.
然後在自己的故鄉裡自殺而死.
為什麼會有這樣的人.
他們全部都好像活在上帝國度裡的人.
為什麼他們會自殺而死.
因為他們的自我太重要.
當自我太重要的時候沒有出路.
他只能夠走絕路.
這群人蛇鼠一窩.
最後賠上自己的生命.
為了三十塊銀錢和耶穌基督切割.
但結局更諷刺的是.
因為這三十塊銀錢完全和他切割.
換來的只是死亡的下場.
各位姐妹.
不要以為今天在教會裡就很安全.
一定要去到天國裡.
在天國裡的路上.
魔鬼有很多古靈精怪的人士在你身邊.
隨時帶你離開上帝.
他只需要稍微扭曲一點點.
使你覺得開心.
你就很容易離開主的道理.
跟隨錯誤的神學教導.
多奉獻 多侍奉 多蒙祝福.
這些空泛的陳述.
很多時候成為了信徒的陷阱.
所以今天有很多教會的信徒.
他們真的帶著一份擔心赤子之心.
來侍奉主 完全沒有保留地去參與.
但卻換來的結果是當你侍奉再無價值的時候.
你就被否決.
在這樣的情況下.
他能不能趕緊回轉呢.
還是走向絕路 離開主呢.
以馬來華浸信會的網絡事工這麼多年.
就遇過不少這樣的信徒.
離開了教會很長一段時間.
因為在教會裡面他很努力地侍奉.
最後竟然發現出賣他的就是他們的牧者.

$^{801}$是他們教會的領袖.
在這樣的情況下 丁子沫離開主.
上帝帶他們來到以馬來的時候.
聽到上帝的教訓 重新振作起來.
他不需要永遠留在以馬來華浸信會這個群體.
但只需要他靈命得到喚醒的時候.
上帝的工作繼續在他身邊.
上面繼續的工作.
這個就是我們作為教會應有的角色.
不只是牧師要做的事.
不是牧師說到的事.
而是你要積極參與.
你在你的工作環境裡面有沒有見到這樣的信徒.
然後你在當中幫助他們搞清楚他們的信徒.
還是他們回來教會仍然傻傻地坐在那裡.
每天以為自己每個星期回來就很虔誠.
成為一個好信徒.
他們有沒有真正認罪悔改呢.
還是仍然對上帝若即若離呢.
我們的角色有沒有去幫助他們呢.
這個就是我們以馬來華浸信會的目的.
最後第六至十一節.
這是一個叫做「瑤護田地的啟示」.
我們一起讀一下這段聖經.
我們從第六節讀到第十一節.
「濟師長拾起銀錢來說:.
『這是血的,不可放在庫裡』.
他們相已就用那銀錢買了瑤護的一塊田.
為要埋葬我鄉人.
所以那塊田直到今日還叫做血田.
這就應了先知耶利米的話說.
他們用那三十塊錢.
就是被古定之人的價錢.
是以色列人中所古定的.
買了瑤護的一塊田.
這是照著主所吩咐我的』」.
在這裡這班濟師長再一次商議.
他們很喜歡開會的.
經常商議的.
如果大家看回聖經裡面.

$^{841}$十二章十四節.
二十二章十五節.
二十七章第一節.
第七節.
二十八章十二節.
都很喜歡商議的.
他們甚麼都商議的.
他們開會多過吃飯的.
所以說真的.
教會是否真的要經常開會呢.
這是值得深思的題目.
不過他們開會這次討論甚麼呢.
這三十塊銀錢怎樣搞呢.
他們為這三十塊銀錢立了一個名字.
叫做血價.
血價是甚麼呢.
這些錢是謀財害命.
所有關的金錢.
這些錢不能放進神的墊褲裡面.
神的墊褲裡面.
其實有十三個儲存現金的箱.
這些箱子是用來收取人們.
去奉獻金錢.
放進去電稅等等.
全部放進去.
他們說不行的.
上帝的聖殿收錢的地方.
這十三個銅很聖潔的.
因為上帝的殿很聖潔的.
所以這些和血死亡有關的錢.
不能和血死亡有關的錢一起放.
於是他們要分開.
他們很虔誠.
他們假裝出來的虔誠.
實在很勁虔.
他們就把這筆錢.
指出不如這樣.
為加勒人猶大買一幅田.
這幅田是向瑤戶買的.
為何要向瑤戶買呢.

$^{881}$如果大家熟悉巴勒斯坦耶路撒冷地勢.
就會明白.
耶路撒冷是一個山城.
山城在山上有些山谷.
在山谷裡往往成為了工業地區的發源地.
因為有水.
發現在耶路撒冷的東北邊.
有一條河谷.
那條河谷昔日曾經很熱門地被瑤戶.
用來生產碗碟 瓦器等等的地方.
那裡因為是一個山谷的地方.
於是就挖掘很多泥土.
而那些泥土就很適合做瑤戶的產品.
大家知不知道中國人有個地方叫景德鎮.
景德鎮有甚麼出名呢.
陶瓷.
為何要在景德鎮做陶瓷呢.
又不是在多倫多市中心做.
又不是在廣東省做.
又不是在上海做.
又不是在新疆做.
為何要在景德鎮做呢.
因為一個原因.
因為那裡的泥土是獨特.
用來做陶瓷最好的.
因此我曾經去過一個陶瓷工藝博物館研究過.
他刻意展示一個角落是景德鎮的陶瓷.
在多倫多市中心.
他說在全世界裡面做陶瓷最好的泥土不是很多地方.
其中一個地方就是奧地利.
另一個地方就是英國.
所以你會留意到在這個世界裡面.
出產最好的陶瓷都是在這三個國家裡面.
英國的陶瓷真的非常好.
泥土的組合也很漂亮.
但當陶瓷的工匠窯戶發展到某個地步.
他就在那裡挖掘出很多洞穴.
挖掘出來的泥土就變成了陶瓷.
那些洞穴用來做什麼呢.
最好用來做現成的墳場.

$^{921}$而聖經裡面記載得很清楚.
這些窯戶的前地有很多洞.
用來讓人埋葬愛鄉人.
在這段聖經裡面就清楚告訴我們一件事.
耶路撒冷的人對於加勒人猶大.
視之為一個外鄉來的猶太人.
猶太人也有疑問.
不過當他們是外鄉人.
不是自己人.
潮州話叫做不是自己人.
不是自己人.
於是怎樣呢.
專門把外鄉人埋葬在耶路撒冷.
不是用墳墓去埋葬.
而是用窯戶找出他們以前挖掘過的洞.
然後怎樣呢.
把他們埋葬了.
簡單來說.
猶太人從頭到尾想黏著猶太人公會.
想從此升官發財.
至少我幫過你們.
出賣耶穌.
你們也要看著我.
但原來結局是.
他們只當他們是外鄉人.
頂智末.
你一生用了很多努力.
想黏著一些人.
你以為和他一起同謀.
可以共得利益.
但很多時候.
只要他不在真理裡.
結局是怎樣呢.
他就是出賣你的人.
甚至和你完全一刀兩斷的人.
在這個情況下.
買了一塊這樣的地.
而講完這段聖經的時候.
就有這樣的備註.
那塊田地第八節.

$^{961}$那塊田直到今日還叫做血田.
(這就應了先知耶利米的話說).
(他們用三十塊錢).
(就是被古定之人的價錢).
(是以色列人中所古定的).
(買了瑤湖的一塊田).
(這是照著主所吩咐我的).
頂智末請留意這段聖經.
這是馬太為了這件事的註腳.
馬太為我們解釋這件事的獨特之處.
亦是這段聖經的高潮所在.
這段聖經不只是講買一塊田.
用三十塊錢的問題.
而是用三十塊錢.
用古定的價錢.
而且是以色列人古定的價錢.
來買這塊田地.
是主的吩咐.
主吩咐了什麼.
馬太要在這裡強調.
告訴我.
這就是這段聖經我們要集中看的地方.
先前已經看到一群.
大家一起同謀合污.
盡力對犯罪的文士.
祭司法利塞人 祭司長.
還有一個和他們同謀合污的嘉麗耶和大.
他們的罪盡情發揮了.
但結果他們完全沒有弟兄的關係.
完全是分開各自為政.
不過是私心滿佈他們的生命.
來到這段聖經.
馬太要告訴我們.
這群人的最可怕之處就是.
他們做所有的事情.
都是合情合理合法.
不過是出於假虔誠.
但上主卻要一個人做一件事.
是不合情理.
但真正的真虔敬.

$^{1001}$我解釋給大家聽.
這段聖經我們看到最後.
第九和第十節的時候.
聖經學者在研究.
究竟這段聖經出自哪裡呢?.
有學者指出.
這段聖經應該出自.
撒加利亞書第十一章第十一至十二節的經文.
我們現在馬上打去撒加利亞先知書.
是舊約聖經尾二的一卷.
十一章.
我們先看第十二和第十三節.
大家一起讀吧.
第十二和第十三節.
我對他們說.
你們若以為美就給我功架.
不宜就罷了.
於是他們給了三十塊錢作我的功架.
耶和說.
要把眾人所估定的美好的價值.
標級而護.
我便將這三十塊錢.
在耶和的殿中標級而護了.
像不像?.
可以告訴你.
這句是馬太所記下來的說話.
有九成的說話都是從撒加利亞先知書.
第十一章第十二和第十三節而出.
不過在這裡出又怎樣呢?.
有什麼特別意思呢?.
我們要看完第四節至第十三節說什麼.
我讀出來你們慢慢細細去看內容.
耶和說.
我的神如此說.
你撒瑪利亞要牧養者將宰的群羊.
買他們的宰了他們.
以自己為無罪.
賣他們的雪.
耶和說是應當稱俗的.
因我成為父宗.

$^{1041}$牧養他們的並不連戍他們.
耶和說我不再連戍這地的居民.
必將這民交給各人的鄰舍和他們王的手中.
他們必毀滅這地.
我也不救這民脫離他們的手.
第七節.
於是我牧養者將宰的群羊.
就是群中最困苦的羊.
我拿著兩斤杖.
一斤我稱為榮美.
一斤我稱為連戍.
這樣我牧養了群羊.
一月之內我除滅三個牧人.
因為我的心厭煩他們.
他們心也必憎厭我.
我就說我不牧養你們.
要死的由他死.
要喪命的由他喪亡.
剩餘的由他們彼此相食.
我折斷那稱為榮美的杖.
表明我廢棄與萬民所立的約.
當日就廢棄了.
這樣那些仰望我的困苦羊.
就知道所說的是耶和華的話.
我對他們說.
你們若以為美就給我功架.
不然就罷了.
於是他們給了三十塊錢.
作為我的功架.
耶和華吩咐我說.
要把眾人所固定美好的價值.
都再饒護.
我要將這三十塊錢.
在耶和華的殿中都給饒護了.
我又折斷稱為連戍的那根杖.
表明我廢棄猶大與以色列弟兄的情誼.
你知道這段聖經說什麼嗎?.
我用一個很簡單的說話.
描述給你聽.
先知撒加利亞在這裡.

$^{1081}$要被上帝指示他去玩魔鬼.
要作撒旦魔鬼的角色.
你沒聽錯嗎?.
上帝叫一個先知要做到魔鬼的角色.
What is the reason?.
什麼原因?.
因為這群人實在不能再去牧羊.
這群人一開始說明是一群什麼羊.
是一群張鼻仔的群羊.
這群羊根本不稀罕自己的生命.
只是付出出力.
希望從中得到好處.
於是耶和華在這裡說.
既然要被人陷害.
不如我抄一個先知撒馬利亞先知.
你現在去做這件事.
做什麼事?.
你去牧羊這群張鼻仔的羊.
我將牠交在你們的手腳上.
然後你要好好告訴牠們.
耶和華在這個時候.
讓你們看到你們的罪有多可怕.
罪可怕到很虔誠地可怕.
虔誠到一個什麼地步.
他們認為自己是無罪的.
所以你看一開始的五節說.
他們以為自己是無罪的羊.
但他們卻將自己出賣.
既然要被人出賣.
被自己出賣.
不如我們親手將牠出賣出來.
先知於是從第七節裡.
牧羊著這群羊的時候.
他做了兩根杖.
這兩根杖分別叫做榮美和聯索.
榮美和聯索是象徵著他們和上帝的關係.
但這兩個和上帝的關係.
都逐漸被截斷.
在截斷的時候.
撒加尼亞先知給了他們一個機會.

$^{1121}$讓他們顯出自己虔誠的可怕.
因為他們的虔誠是假的虔誠.
於是他們說.
我作為牧羊你們的牧羊人.
你給我一個工價.
你們定價錢給我.
他們就定了.
最合理的三十塊錢就OK了.
於是定了這個工價.
把這三十塊錢交給牧羊.
然後耶和華神和撒加尼亞先知說.
你把這三十塊錢扔進去.
然後告訴牧羊.
我和你的關係一刀兩斷.
等一會.
一群假勁乾的人.
他們做的一切都很符合上帝的律法.
就像我們剛剛看到的.
殺耶穌的那群眾祭師.
文士.
包括加勒人猶大.
他們全部都熟悉神的律例典章.
他們做的一切都很符合律法規限.
甚至加勒人猶大最後都認為自己殺了無辜的人.
他認為自己有罪.
他想脫身.
全部都是假虔誠.
而活在猶太人公會裡面.
但真正虔誠的是誰呢.
撒加尼亞先知書第十一章裡面.
就是被耶和華所猜派要扮演魔鬼角色的撒加尼亞先知.
先知有一個共通的特點.
愛上帝的子民.
上帝你要我帶著一份愛你的子民.
使他們和你斷絕關係.
我是不是很痛苦.
但最愛以色列人的.
正正是這位先知.
他帶著乾勁.
但他做的事是最不能夠令人明白的.

$^{1161}$你害我?.
你斷絕那些聯繫和名美?.
有沒有搞錯啊?.
靜姐妹.
在新約聖經裡面.
真正帶我們去得救的.
那位主耶穌.
他做的事情.
何嘗不是一般人認為不合情理的事情嗎?.
死在十字架上.
靜姐妹.
最愛我們的.
是這位上主耶穌基督.
但他卻做了一件.
是這個世界連魔鬼都想不出最不合情理的事情.
死在十字架上.
因為他的不合情理.
打通了拯救的門路.
馬太福音的作者馬太.
就要告訴我們這位彼得.
西門彼得的生命.
要從跌倒的裡面再次走上來.
是被誰去激動他們.
是一位做出了不合情理.
但卻是最驚奇的主耶穌基督.
因此馬太福音的作者.
將這個訊息交給你和我.
我們是否仍然沉迷在一些.
很開心很快樂.
以為自己都後悔了.
我MADDOWMANEL的事情.
而真正未做到MADDOW.
凝聚悔改.
當一個人知道自己做錯了事之後.
內心會繼續掙扎.
繼續從中找出一條出路.
如何和我們得罪了的神重新修和.
這個人就是一個有凝聚悔改生命的人.
而這個人雖然最後會面懵懵.
但他要在上主面前說.

$^{1201}$因主是我自己不好.
連累到你要寫在十字架上.
但我有主.
他會這樣對我說.
不要緊的.
我知道你會跌倒.
我知道你會有些時候失信.
只要你肯回來的時候.
知罪悔改.
我一定會繼續護養你.
弟兄姊妹.
你是那一種凝聚悔改的人呢?.
還是只是一點點後悔內疚的人呢?.
我們一起和心低頭禱告.
馬太將一個重要的信息交給我們.
我希望在今天這個信息裡.
藉著這段聖經的教導.
讓我和追求主道的每一個屬神的子民.
在今年為納治成為一個不玷污自己的屬神的神國的公民.
求你的幫助.
我們會不會有挑戰.
但願主你的聖靈陪著我們.
我們禱告.
奉靠主耶穌的德性命之求.
阿門.
\newpage



\section{馬太福音 27:11-23}
\label{sec:ZN4O4BAmHMA}
\textbf{沒有"不關我們的事"(馬太福音27\_11-23) -  黃紹權牧師 [馬太福音信息系列 - 第144講]}
\newline
\newline
連結: \href{https://youtube.com/watch?v=ZN4O4BAmHMA}{\texttt{ https://youtube.com/watch?v=ZN4O4BAmHMA}} ~~~~ 語音日期: 2025-01-19 
\newline
\newline
\hyperref[sec:499K9je19EI]{< < < PREV SERMON < < <}
~
\hyperlink{toc}{[返主目錄]}
~
\hyperref[ch:preacher10]{[返講員目錄]}
~
\hyperref[sec:HaaLhKYBRSg]{> > > NEXT SERMON > > >}
\newline
\newline
馬太福音 27:11-23
\newline
\begin{longtable}{cl}
\hline
\hline
章節 & 經文 (和合本修訂版)\\
\hline
27:11 & \begin{tabularx}{0.7\textwidth}{X} 耶穌站在總督面前,總督問他:「你是猶太人的王嗎?」耶穌說:「是你說的。」 \end{tabularx} \\ \\ \relax
27:12 & \begin{tabularx}{0.7\textwidth}{X} 他被祭司長和長老控告的時候,甚麼都不回答。 \end{tabularx} \\ \\ \relax
27:13 & \begin{tabularx}{0.7\textwidth}{X} 彼拉多就對他說:「他們作證告你這麼多的事,你沒有聽見嗎?」 \end{tabularx} \\ \\ \relax
27:14 & \begin{tabularx}{0.7\textwidth}{X} 耶穌仍不回答,連一句話也不說,以致總督覺得非常驚訝。 \end{tabularx} \\ \\ \relax
27:15 & \begin{tabularx}{0.7\textwidth}{X} 總督有一個常例,每逢這節期,隨眾人的意願釋放一個囚犯給他們。 \end{tabularx} \\ \\ \relax
27:16 & \begin{tabularx}{0.7\textwidth}{X} 當時有一個出名的囚犯叫巴拉巴。 \end{tabularx} \\ \\ \relax
27:17 & \begin{tabularx}{0.7\textwidth}{X} 眾人聚集的時候,彼拉多就對他們說:「你們要我釋放哪一個給你們?是巴拉巴呢?是稱為基督的耶穌呢?」 \end{tabularx} \\ \\ \relax
27:18 & \begin{tabularx}{0.7\textwidth}{X} 總督原知道他們是因為嫉妒才把他解了來。 \end{tabularx} \\ \\ \relax
27:19 & \begin{tabularx}{0.7\textwidth}{X} 正坐堂的時候,他的夫人打發人來說:「這義人的事,你一點不可管,因為我今天在夢中因他受了許多的苦。」 \end{tabularx} \\ \\ \relax
27:20 & \begin{tabularx}{0.7\textwidth}{X} 祭司長和長老挑唆眾人,要求釋放巴拉巴,除掉耶穌。 \end{tabularx} \\ \\ \relax
27:21 & \begin{tabularx}{0.7\textwidth}{X} 總督回答他們說:「這兩個人,你們要我釋放哪一個給你們呢?」他們說:「巴拉巴。」 \end{tabularx} \\ \\ \relax
27:22 & \begin{tabularx}{0.7\textwidth}{X} 彼拉多說:「這樣,那稱為基督的耶穌我怎麼辦他呢?」他們都說:「把他釘十字架!」 \end{tabularx} \\ \\ \relax
27:23 & \begin{tabularx}{0.7\textwidth}{X} 總督說:「為甚麼?他做了甚麼惡事呢?」他們更加喊著說:「把他釘十字架!」 \end{tabularx} \\ \\
[1ex]
\hline
\hline
\end{longtable}
$^{1}$各位姐妹早晨.
今天凌聽主導議先.
再次和心低頭禱告仰望主愛的神.
慈悲的天父多謝你.
讓我們每一個能夠學慕主理道的人.
都能夠有生命再次來到你的庚前.
打開你的啟示.
來明白當中的教導.
並且讓我們從當中學會.
如何作為一個合神心意的信徒.
為此獻上感恩.
讓我們不要糟蹋這段時間和機會.
因為我們知道在這個世界上.
有很多地方同樣想追求主導的人.
我們的弟兄姊妹.
連一本聖經都未能夠得到.
但願主理讓我們能夠為這些弟兄姊妹.
繼續加上禱告.
讓他們在有限的資源之下.
仍然學會如何秉行主理的教訓.
我們知道我們作為信徒.
在地上活著的時間.
雖然我們未必完全明白.
長大騙上帝你的啟示.
但就讓我們每一個.
學到多少就實踐多少.
求你帶領我們同心的禱告.
奉靠主耶穌基督得勝的名字.
阿們.
在上星期的講道裡.
有一點我特意留在這星期.
和大家一起思想一下.
就是有關於我們談論到.
猶大他自己的死的問題.
如果大家去看第27章.
馬太福音第27章的時候.
我們去到最後的時候.
第10節談論到有關猶大死的時候.
有一句話在最後說.
是照著主所吩咐我的去說.

$^{41}$說了些什麼呢.
原來就在第9節裡.
記載到原來馬太記載到.
有關於加勒人猶大之死.
是源自於先知耶利米的說話.
說他們用那三十塊錢.
就是被古定之人的價錢.
是以色列人中古定的.
買了而戶的一塊錢.
在上星期的時候.
我們討論到這段經文.
原來是本身和.
撒加利亞先知書第11章裡的.
是完全可以說是九成幾是吻合的.
上次我們也談論到有關於.
在撒加利亞先知書裡的記載.
主要針對的核心.
是有關於以色列民.
他們對上帝的牧養的鄙視.
也借助撒加利亞先知.
在當時被以色列民的看待.
他們要求在牧養裡.
要給他們一份工資.
但他們估計的估值只是三十塊錢.
而這個數目.
在聖經學者的研究裡有兩派.
其中一派認為這是一個勞瀑的價值.
但也可以反映出.
只不過是四個星期的一般人的收入.
簡單來說.
一個勞瀑人來的.
本身很珍貴.
但價值也被貶低只是值四個月的工資.
以色列民根本對上帝.
藉著先知的牧養和引導.
是處於一種鄙視的態度.
於是唯有上主.
既然你們鄙視我.
我就用撒加利亞先知.
以一個惡毒的態度.

$^{81}$用相等對待的態度.
來將我和你的關係表明切斷.
這是我們上星期看到的聖經解釋.
但為什麼聖經會在馬太的口中.
被陳述為耶利米先知的話如此說呢.
在聖經的研究裡.
這一門學問是很普遍的.
就是嘗試去看聖經的記載.
似乎顯出了一些錯誤.
尤其是一些對基督教否定的信徒或教徒.
我指其他教徒.
他們都會用這些來否定了聖經的真實和準確性.
這也引來很多信徒在權度或靈性成長的過程中.
有很大的猶疑和憂慮.
所以我特意要去剔出一段時間.
雖然今天的講道一定會很長.
我老實地和大家說.
但是為什麼我要編出來.
在這裡所說呢.
馬太是否像一些信徒甚至聖經學者.
來指出是一個錯誤的引述呢.
還是好像有另一類別的聖經研究.
說這只不過是抄寫員抄聖經抄本的時候.
他的腦海中跳出耶利米先知.
而他寫下來是一個誤抄呢.
還是他們本身有一個固定的神學.
認為所有先知當中以耶利米為首位的一位很重要的先知呢.
於是引用了他的名字呢.
在我的研究裡以上所提出的理由都不充分.
因為首先我們沒有辦法知道.
首抄的首抄員是不是真的腦海中有這個錯誤或前設在裡面.
而耶利米先知是不是眾先知之首呢.
這也是一個很值得懷疑的問題.
為什麼馬太要用耶利米先知的話說來引述.
是一段我們很肯定是從撒加利亞先知而寫的內容呢.
這一點其實涉及於引用耶利米先知的時候.
和這段經文有什麼關係.
如果我們細心去查考.
在耶利米書裡面有好幾段聖經所談及的內容.
都很接近這裡有關於買一幅地.

$^{121}$特別是由瑤湖買一幅地的問題.
這個我們可以參考在耶利米先知書第十八章第二至第六節裡面.
談論到當時耶利米先知曾經向阿拿達地區的人查探.
想去買一幅地.
然後去到第十九章第一至十三節裡面.
更確切的談論到他向瑤湖買了一幅地.
在變雅文境內正正就是所談及到的阿拿達地區.
我們再仔細去看耶利米先知書第32章第六至第十五節的經文裡面.
這段經文更加精彩的提到一個事實.
就是當時耶利米先知被猶大國的王西底加.
將他捆綁或監禁在這個殿的囚室裡面.
並且怎樣去指控耶利米先知指出.
要發預言耶利米所處的城市耶路撒冷.
將會被尼布甲尼撒王所.
吞併.
而西底加王很不高興.
不喜歡聽這個消息.
即使當時尼布甲尼撒王已經軍臨到耶路撒冷城.
圍攻了耶路撒冷.
西底加王仍然不接受耶利米先知的陳述.
以上這三段經文就顯示出.
當馬太寫道有關於撒加利亞先知買地埋葬等等.
或者撒加利亞先知想凸顯出以色列民那種對於上帝的否定.
不想再在上帝的牧養之下的環境.
但又要裝作自己是虔誠的猶太子民或希伯來人.
來到耶利米先知書.
就由一個王作為代表去顯示出.
整個以色列民單從平民以至於統治階層.
都要表現出自己好像很敬虔在上帝裡面.
但實質上是不想在上帝的管治之下而活.
這正正可以反映和銜接到在馬太福音裡面.
馬太要顯出猶太的領袖將要臨盪的情況.
即是他們將要失去他們的影響力.
而這亦成為我們今天說這段聖經的前設和背景.
因為來到這段聖經裡面.
今天我們所看的這段經文正正就要凸顯一個題目.
就是很多當時以色列的領袖.
在巴勒斯坦地裡面的領袖和人群.
他們個個都表面很敬虔.
但實質上他們的生命是完全不想在上帝的指導和請求當中.

$^{161}$而亦成為我們今天說的主題.
叫做沒有不關我們的事.
個個都覺得殺耶穌不關我們的事.
包裝到不關他們的事.
但原來沒有一個人是不關這件事.
讓我們一起仔細去看這段經文.
剛才彭玲幫我們讀了一節.
我們再讀多一次讓我們更加入心去了解這段經文.
從第十一節至第二十三節.
我們一起去看一看.
一起打開一起讀.
耶穌站在巡撫面前.
巡撫問祂說:你是猶太人的王嗎?.
耶穌說:你說的是.
他被祭司長和長老控告的時候.
什麼都不回答.
比拉多對祂說:他們作見證告你這麼多的事.
你沒有聽見嗎?.
耶穌仍不回答.
連一句話也不說.
以至這巡撫甚麼稀奇?.
巡撫有一個常例.
每逢這節期.
隨眾人所要的釋放一個囚犯給他們.
當時有一個出名的囚犯叫巴拉巴.
眾人聚集的時候.
比拉多就對他們說.
你們要我釋放哪一個給你們?.
是巴拉巴甚麼呢?.
是稱為基督的耶穌呢?.
巡撫原知道他們是因為嫉妒才把他戒了裡.
正在坐堂的時候.
他的夫人打發人來說.
這二人的事你一點不可管.
因為我今天在夢中為他受了許多的苦.
祭司長和長老挑唆眾人.
求釋放巴拉巴.
除滅耶穌.
巡撫對眾人說.
這兩個人你們要我釋放哪一個給你們呢?.

$^{201}$他們說巴拉巴.
比拉多說這樣那稱為基督的耶穌.
我怎樣扮他呢?.
他們說把他釘十字架.
巡撫說為甚麼呢?.
他做了甚麼惡事呢?.
他們便極力地喊著說.
把他釘十字架.
在上文我們剛才已經說過.
基本上是說甚麼事情.
不過在這段聖經裡.
我們會看到耶穌真真正正被壓戒到巡撫的地方.
所以第一部分我們去看的.
在這段聖經裡.
會看到耶穌怎樣去回應巡撫比拉多的回答.
我給他一個副題.
從第十一至第十四節.
就是神的智慧是叫人詫異.
在第十一至第十四節裡.
我相信這段聖經應該包括以下這幾個群體.
一羅馬所委派的巡撫比拉多.
一個熟悉世界的領袖.
另外猶太人公會的群眾.
尤其是我們所看到的大祭司和長老.
而最後的一個群體.
我相信應該毫無疑問的就是耶穌基督.
簡單來說這個壓戒過程.
可以說是將耶穌壓戒在一個私人的審問過程.
這個場景應該是屬於羅馬巡撫比拉多的內院當中.
進行一連串的私下對話.
為什麼呢?.
因為我們會看到這件事討論完之後.
才會再搬到另一個公開的場合進行更進一步的討論.
在聖經當中我們會發現到.
有一個很特別的事件.
除了在他審問的過程當中.
還夾雜了一段有關比拉多的妻子在夢中發現的事情.
所以簡單來說我們可以說.
在這段聖經當中.
其實是緊接一環而一環.

$^{241}$而上帝仍然在運作在其中.
上帝完全不顧耶穌基督.
在整件事上耶穌和父神其實是並肩一起走這條道路.
但是讓我們看到.
從這段聖經開始我們會有一個更肯定主題.
突顯在這段聖經.
其實之前已經說過.
原來這群祭司和長老們.
他們一直都認為這件事與我們無關.
如果大家去看前一點.
在第27章當猶大發現自己陷害了一個無辜的耶穌.
或者這裡所說無辜之人的血.
其實原文寫得很有趣.
如果用廣東話翻譯.
因為我對於廣東話有很多規限.
因為要趕時間等等.
如果用茶經這段聖經就很好說.
猶大其實都說.
哎呀死了我這次又害了一個無辜的人的血.
文雅就這樣說.
如果廣東話說法就是.
我害了一個無辜的人.
連耶穌的名字都不想說.
在這樣的情況下.
他就跟這群祭司說.
哎呀,我不要了,我得罪了.
這群祭司怎麼說呢?.
拉與我們有什麼相關的事情擔辦.
這段聖經已經打開了第27章.
這個與我無關的.
用廣東話說法.
或者用廣東話更俗套的說.
與我無關的.
你自己搞定他吧.
這個主題重複重複在這裡出現.
不過重複發生在不同人身上.
我們先看看第一段.
這裡所說的11-14節.
什麼與我們無關.
這節先前說猶太人公會無關.

$^{281}$現在來到耶穌面前面對的這個巡撫.
他又怎麼去面對耶穌.
他一開始.
其實羅馬巡撫無法置身於道外.
因為他要管治這個地方.
所以大家知道.
有很多人要擁有權力.
有權力其實要有相應的靈性才能擁有權力.
否則很容易濫權.
有時權力不是你有什麼可以做得到.
有時是因為你有權力.
事情不能避開就是因為你的權力.
這個巡撫因為權責問題.
所以必須要知道這件事情.
於是第一件事.
見到耶穌.
他就問耶穌.
你是猶太人的皇嗎.
在這段聖經裡.
我們會去問一個問題.
何來比拉多第一個要問這條問題呢.
如果從上文我們一路去看的時候.
耶穌和文殊長老們去問指責他是什麼的時候.
都很強調問一個問題.
你究竟是不是神的兒子.
你是不是基督.
比拉多在哪裡來的這個說法.
你是不是猶太人的皇呢.
我相信在這個過程當中.
兩者由神的兒子基督.
去到猶太人的皇這個陳述的改變.
不多不少涉及兩方面.
第一方面是因為巡撫是一個政治的體制.
所以他去想的就是.
你是不是稱自己為皇.
而去對於羅馬政權有威脅.
這是其中一個可能的想法.
但另一個更可能的情況就是.
在轉介的過程當中.
相信猶太人的長老和祭司們.

$^{321}$開始要傳遞一個更新了的.
或者一個由猶太人的神學基督這個觀念.
將它轉化成一個政治觀念的陳述.
去解說給比拉多聽.
希望藉此比拉多可以和他們同一條陣線.
於是比拉多唯有一開口就去問耶穌.
你是不是猶太人的皇呢.
這個絕對是在聖經裡有可能發生的事情.
不然怎麼會在一個這麼私人的空間去審問耶穌呢.
無論怎樣的原因也好.
在這個時候猶太人的皇這個身份.
也再一次凸顯了在耶穌受洗的過程.
他不單在宗教層次上.
或者在傳統猶太人的宗教層次上.
他是一個基督尼西亞神的兒子.
而且現在更加賦予上一個權力.
在政權上管治猶太人的皇.
當這個巡撫問耶穌的時候.
問完他就想得到一個答案.
在聖經研究裡.
或者很多讀聖經的弟兄姊妹.
讀到這裡就大作文章了.
我們看看耶穌怎樣.
因為耶穌在這裡給了一個回答.
耶穌說「你說的事」.
「你說的事」是什麼意思呢.
「你說的事」如果我們按照希臘文的翻譯.
就是「You say so」翻譯為英文.
如果用中文翻譯就是「你現在正正說出事實」.
「你說了個事實」.
「你問我,我是不是猶太人的王」.
耶穌的回答是「你說了個事實」.
某程度上耶穌承認了他是猶太人的王.
但這個說話的見證和答案不是由耶穌提供的.
但他卻將一個鐵一般的.
毋庸置疑的真相提出來.
就好像我們所說的太陽從東方升起的說法.
這是你說出來的事實.
不是我鋪陳出來的.
當我們看到這段聖經的時候.

$^{361}$很多信徒就說耶穌嘗試逃避.
用他的口說答案.
其實不是.
耶穌是承認的.
他承認我是猶太人的王.
但這個答案是你總結出來的.
不是我去勸服你說出來的.
當我們明白這一點的時候.
我們就會看回今天我們去和人傳講福音.
甚至介紹有關於上帝的事情.
其實你不需要用很多的神說去辯論上帝是怎樣的.
不用說什麼.
大家知道這個禮拜直至這一刻最熱門的新聞是什麼.
很熱的.
這個都不夠.
加州大火.
真的很熱.
杜魯多辭職是另一回事嗎.
他說辭職都沒有辭職.
還坐在那裡.
都還沒執行禮.
不要說他我們不是說政治.
加州大火.
有很多人考究加州大火究竟是什麼.
歸咎是否政治的問題呢.
是否涉及窮人抵不住脖子.
見到有錢人就放火燒他們的家呢.
我不想作這些推論.
可能是可能不是.
但是我很肯定一件事.
這場火的發展是完全在上帝的允許和他的手下運作.
牧師你這麼肯定.
不是我肯不肯定.
你有眼睇.
整場火的情況根本沒有人能夠用人手去撲滅.
你那邊撲滅一個火種.
另一邊的火種就已經起.
我昨晚看到一點鐘.
其實不想的不過睡不著.
看著從遠遠的地方突然間中間又會起火.

$^{401}$遠遠的地方突然間還要燒不燒.
專門燒荷里活山.
最多明星最多有錢人住的地方.
很美的我去過兩次.
很美.
我們就要問為何上帝要這樣容許這件事.
現在估計這個火災的燒的程度.
明天會更加厲害.
到星期三會更加全面地燒.
我見到很難堪.
但我見到上帝的手在運作.
上帝想說什麼給這個世界的人知道.
你越以為自己掌握到的東西.
可以不理會上帝的時候.
上帝就來找你.
是恩典的時間.
不是懲罰的時間.
好像以色列人在撒加利亞先知書裡一樣.
他不要上帝了.
上帝仍然找一位先知來作為一個醜人.
但上帝仍然把兩個重要的訊息告訴了他們.
就是那兩支杖.
上帝仍然想關係存在.
但你們要破壞.
既然你用你的方法想破壞.
上帝說不如用我的方法和你破壞.
保證你想破壞的一定破壞到.
然後我再次將他收獲.
在聖經裡我們看到.
秦府當看到耶穌這樣回答的時候.
他沒有停下來.
怎麼樣.
第十二節.
他被祭司和長老控告的時候.
什麼都不答.
馬太就留意到這一點.
一路耶穌在被質問.
被這幫人冤枉.
用假見證去誣告他.
他一句都不出聲.

$^{441}$但為什麼你突然出聲回答我呢.
於是在這個情況下.
比拉多忍不住.
就和耶穌說多一句.
他們作見證告你這麼多的事.
你沒有聽見嗎.
比拉多這個回答更加特別的是.
他不是問你為什麼不回答他們.
我現在問你一條問題.
你馬上回答我說的事實.
為什麼他們說的時候.
你不回答.
不回答什麼.
耶穌根本沒什麼回答.
比拉多反而在這裡強調.
你沒有聽見嗎.
這句話不是耶穌要回答的.
是馬太記載給我們看.
比拉多在想什麼.
比拉多覺得很奇怪.
先前冤枉你的你又不回答.
現在我說的你又回答.
既然是這樣.
我再問你.
你沒有聽見嗎.
馬太福音記載.
耶穌仍然不回答.
連一句話也不說.
以至巡撫甚覺稀奇.
丁子妹.
在聖經想說什麼呢.
耶穌想做什麼呢.
耶穌做了什麼.
耶穌只說了你說的事一句.
但是在其他人的腦海裡.
產生了一大堆疑問.
在這個疑問裡.
耶穌出了一個心化.
心化什麼.
你問我有沒有聽到他們說什麼去誣告我.

$^{481}$耶穌不回答.
為什麼不回答.
丁子妹你知道當一些事情不值得你回答的時候.
你不需要去回答.
因為你回答多說多錯.
反而你不回答.
這就造成了上帝在人生命裡的反思.
在反思的過程中.
他覺得很難明白.
越想想越想理解.
你看耶穌的做事.
我們就侍奉.
耶穌就做事.
耶穌做的事給我一個很好的介紹.
和一個很好的例子.
有時神的智慧.
不是在乎你做了什麼.
反而是在乎你不做什麼.
今天很多信徒以為.
為什麼才顯出我們基督徒有神的智慧在我們裡面.
於是他們出盡努力去做一些侍奉.
讓人看到上帝做事.
但是有沒有想過.
如果我們用這樣的看法.
我們太小看上帝的智慧是怎麼運作.
智慧在聖經裡.
如果大家記得耶穌曾經說過.
如果有些人不抵擋我們的時候.
他就是我們的朋友.
但也有可能是什麼呢.
當那個人幫助我們的時候.
也是我們的朋友.
究竟幫不幫好.
朋友是怎麼斷定的.
弟兄姊妹.
上帝做事不是單面運作.
沒錯.
你可以做些侍奉.
來顯出上帝的智慧.
上帝的啟示.

$^{521}$但是同樣上帝的智慧.
也可以在你不做任何事的時候.
都能引發出來.
這段聖經就做了出來.
耶穌不回答那些誣衊他的指控.
其實就已經做了事了.
今天我們信徒很多時候.
我們不明白這一點.
於是我們就說要找些事來做.
要搞事工.
並且當我們要面對一些困難.
或者要變道的時候.
我們就會求上帝給我一些智慧.
使我能夠和別人變道的時候.
或者做一些事工的時候.
做了出來很燦爛.
讓人感覺到很驚訝.
不是的.
當時你不做一些事.
尤其是針對那些罪惡的時候.
你不出聲.
反而可能誘發了一些人.
認為為什麼基督徒在這些情況.
不去評論.
是不是你對公義不再理會.
耶穌是不是對公義不理解.
所以不想回答他們.
不是.
耶穌在針對的是他能夠看到的比拉多.
他就將比拉多引發到一個點.
讓他去稀奇.
如果我們看其他福音書.
特別是路加福音的時候.
記載到瑪利亞.
由他開始知道自己要去聖靈懷孕.
到他真的懷孕.
整個過程當中.
以至沙漠耶穌之後.
耶穌上了聖殿等等.
所做的事情.

$^{561}$一直有一個很重要的字眼.
稀奇.
馬太用這個方法去介紹給我們知道.
如何去顯出神的智慧.
侍奉主真的不需要很多大動作.
很多時候搞一些大動作.
搞一些事工出來.
其實只不過是對於我們一些信徒.
他的靈性的一種安慰劑.
英文好說.
Spiritual Placebo.
安慰自己覺得自己靈性上很好.
但實質上我們遠遠離開上帝.
耶穌這個做法引致下一個發展.
就是從第十五節.
跟著往後的經文.
我們先看兩節.
第十五和第十六節.
我們可以看到.
以假換真的教徒心態.
我們讀一讀.
這兩節經文很有意思.
第十五節.
秦府有一個常例.
每逢這節期.
隨眾人所要的.
釋放一個囚犯給他們.
當時有一個出名的囚犯叫巴拉巴.
到這裡為止.
這兩節經文.
很多時候我們讀中文聖經.
會水過鴨背.
或者.
看上去有一個人叫巴拉巴.
但其實這段聖經是一個中長的解話時間.
是為了之後.
接下來從第十七節開始的記載.
如何顯露出這班.
以為自己是很虔誠的.
猶太人工會的領袖.

$^{601}$眾祭司長.
眾長老.
還有一批叫做猶太人.
他們都以為自己很虔誠.
但他們的虔誠卻.
讓我們看到.
他們寧願換一個假的耶穌.
來取代一個真的耶穌.
你問我什麼牧師我怎麼知道.
我們細心看看.
第十五節到十六節裡.
在這裡記載了當時有一個.
羅馬人政府的一個行政手段.
我稱為一種懷柔手段.
就是每逢大時大節的時候.
我們都要去釋放一個囚犯給這班人.
釋放囚犯的目的是什麼.
大家知不知道.
是因應你們要求釋放誰.
那就是滿足你的想法.
是一種討好人的做法.
釋放一個囚犯.
其實是羅馬人統治猶太地.
或者巴勒斯坦地的一種慣常手段.
而且在大時大節去做的時候.
就更加討人喜悅.
不是美國總統拜登去赦免他的兒子亨特.
不是的.
那個擺明是討好他自己.
其實也不是的.
我們不在這裡說了.
當他運作的時候.
這班人要求什麼.
我們看看.
巡撫留意到.
他們心目中有人選.
巡撫於是在第十六節提出.
馬太福音作者提出.
其實他們暗中心已經有一個人物.
就是一個很出名的囚犯巴拉巴.

$^{641}$出名的囚犯原文翻譯為秋明遠播.
秋明遠播的囚犯巴拉巴.
奇妙在巴拉巴.
在古抄本裡有很多字.
尤其是在巴拉巴之前.
巴拉巴這個字眼是什麼意思呢.
Son of the father.
那父親的兒子.
巴拉巴.
但也有一個正名.
正名原來是耶穌.
所以原名叫做耶穌巴拉巴.
在很多古抄本裡都有這樣的記載.
我們大兄姐妹就問.
牧師為什麼我們和合本這麼短的記載.
就是巴拉巴.
是不是和合本又錯了.
不是.
這次和合本是按照一貫聖經的抄本特色.
就是盡量把最短的版本寫下來.
因為按照聖經的意思.
或者研究的裡面.
每逢聖經越來越詳細記載的.
就是越後期的寫作.
因為後期的人才會加一些備註解釋.
在原本的聖經裡.
原本的是沒有的.
簡單來說.
原來巴拉巴的原名是耶穌巴拉巴.
我們就開始有些眉目了.
原來這麼有趣.
原來這群人心目中都想找另一個耶穌來取代.
真正的耶穌.
不過那位耶穌叫基督的耶穌.
他們現在更多的名堂給了他.
叫猶太人的王的耶穌.
現在他們心目中已經有個人選.
都是叫耶穌.
甚至叫那父親的耶穌.
對於廣大的聽眾.

$^{681}$或者對於言語模糊的人.
聽起來就像是父的兒子耶穌.
你這樣聽起來.
父的兒子是誰呢?.
父啊.
你不會怎麼理會的.
聽起來父的兒子耶穌.
那父當然是天父了.
這種手法.
馬太特意用了這兩節記載出來.
告訴我們原來這群以色列民.
尤其是領袖.
在他們心目中已經有一個很重要的盤算.
將一個假的耶穌.
一個救世主的耶穌.
將他用來作為一個真正的彌賽亞的耶穌.
也知道真正的基督耶穌.
他過去所作的.
他的能力是怎樣.
但是他們要決意將一個假的.
換取一個真正的.
不是換取.
取代了一個真正的耶穌基督.
對於我們有什麼警戒呢?.
這個我們要避開.
領子們.
今天其實很多人.
我現在講得更加準確一點.
很多基督徒都很喜歡找一個朱二省的耶穌來取代.
而在教會當中.
今天有很多傳道人.
所傳的耶穌也是一個朱二省的耶穌.
知道什麼是朱二省嗎?.
假手式.
朱二省要不要解釋給大家聽.
朱二省其實是一個民間真正的手式的人名.
他原本是在中國大陸做手式的.
但是做些什麼呢?.
全部都是假的.
因為他知道平民都要戴珠寶的.

$^{721}$要做些漂亮的東西.
但是買不起金色.
於是他就想起.
用銅的金屬.
做了些手式出來.
然後倒了些金進去.
是不是真金都不敢講.
不過當是吧.
然後賣給很多人.
很多婦女買了.
不是婦女.
很多男士都喜歡戴珠寶的.
戴了錢就漂亮了.
但其實全部都是假的.
充得到出來.
領珠妹.
今天很多信徒其實是在信朱二省耶穌.
在他心目中.
有很多人在傳一個朱二省的耶穌.
朱二省的耶穌有什麼特點呢?.
便宜啊.
朱二省是便宜嗎?.
不便宜你怎麼會買朱二省呢?.
有錢當然買真金啊.
便宜.
容易買到.
容易.
朱二省的耶穌很容易信的.
還有.
朱二省的耶穌不需要你負什麼責任.
有一次我在香港的時候.
我回家探望我父母的時候.
在家裡.
無端端在茶几面前.
坐在沙發上.
看到茶几.
有幾隻金戒指放在那裡.
我叫他快點收起來.
我媽媽走過.
我叫他快點收起來.

$^{761}$我媽媽說扔掉吧.
扔掉?.
假的.
沒有責任的.
你不需要負責任.
不見了不可惜.
今天很多基督徒就是這樣的態度.
要容易信.
信了上天堂.
不需要負責任感.
不見了不要緊.
所以他們隨時可以否定耶穌基督.
隨時否定自己相信耶穌.
這就是馬太提出來給我們看的假耶穌.
但很多人喜歡.
人人都買得到.
教會更多.
奉獻一點錢.
又可以做多一點.
靈性好.
人家又會來吹捧你.
你真有愛心.
尤其是那些專賣假意.
朱二省的耶穌.
推介給你的那些目者.
你真是愛主了.
奉獻多.
用奉獻來量度你的靈性.
這種信仰方式.
我有一個形容詞.
叫舒服死的信徒.
舒服死.
廣東人喜歡用什麼去到最盡.
用死來形容.
很盡的.
舒服到死的信徒.
這件事繼續發展下去.
原來我們發現.
每個人都說不關他們事.
但是不是真的不關他們事呢.

$^{801}$其實在這段聖經裡面.
馬太提說全部都關你們事.
我們看看.
從第十七至二十三節.
我讀給大家聽.
你們細心去聽.
(中).
祭司長和長老挑唆眾人求釋放巴拉巴,除滅耶穌..
巡撫對眾人說:「這兩個人你們要我釋放哪一個給你們呢?」.
他們說:「巴拉巴」..
彼拉多說:「這樣,稱為基督的耶穌,我怎麼凡他呢?」.
他們都說:「把他釘十字架.」.
巡撫說:「為什麼呢?他做了什麼惡事呢?」.
他們便極力地喊著說:「把他釘十字架.」.
在聖經裡就印定了我之前所說的所有的東西..
第一,這幫猶太人早在馬太的理解裡.
其實他們已經預定了一定要找耶穌巴拉巴..
他們已經想好了,他們已經計劃了..
所以為什麼在這段聖經裡他們如何教唆人們來求釋放巴拉巴?.
早就洗了人去做了前期工作..
第二,有沒有留意這段聖經刻意記載了.
其實巴拉巴本身也叫耶穌..
你問:是什麼牧師?我仍然看到只有巴拉巴這個字..
不過我們仔細去看一下..
當這幫人要求要去釋放耶穌的時候,.
他們要求要巴拉巴第二十二節,彼拉多這樣說:.
「這樣那稱為基督的耶穌,我怎樣扮他呢?」.
為什麼要用這個記載?.
因為兩個都叫耶穌,一個叫The Son of the Father,.
那父親的兒子的耶穌,叫巴拉巴..
一個叫基督的耶穌..
所以讀聖經要很仔細,不是隨便說,.
要學習給你學習..
其實兩個都叫耶穌,馬太早就告訴我們了,.
彼拉多也知道兩個都叫耶穌,.
這個巴拉巴耶穌你要我釋放,.
這個基督耶穌你要我怎樣做呢?.
他們就把他打倒,如何打倒?.
釘他上十字架..
第三樣東西,從這段聖經裡面,.

$^{841}$很奇妙地記載了這個秦府的夫人很趕急,.
在中間的時間,要找一個人突然來報信給她知道..
即是類似的情況是什麼呢?.
牧師在說道,說著說著的時候,.
突然有個人要走進來,打斷了我訪道的時間,.
然後跟我說,喂喂喂,.
誰誰有什麼事發生,牧師怎樣搞?.
試過了,以前不是用直播,.
電話放在身上,通常用來錄音的時候,.
突然有個人說牧師我有事,我怎樣怎樣,.
我怎樣怎樣,我不會說道的時候停下來等的,.
不過我會看一下究竟發生什麼事,.
於是他發生什麼事?.
他的夫人來到,趕緊派了一個人來跟比拉多說,.
一說什麼?第一句,.
「這二人的事」,這是第三點,.
從一個外邦人的政治系統當中的太太,.
都看得出這是二人,誰呀?.
基督的耶穌是二人,.
這個說法不是這裡第一次說的,.
第一次說是誰呀?.
是我們很憎恨那個人,叫做加略人猶大說的,.
不過他不肯說那個耶穌的名字而已,.
他說我害了一個二人的命,.
你看看連那個惡人,加略人猶大都看到耶穌是二人,.
現在這個二人的身份顯露在另一個人身上,.
而這個提醒也更加令比拉多更進一步在希期之後再思想,.
所以弟兄姊妹,.
耶穌的死不是純粹說一個耶穌死的帳戶那麼簡單,.
原來在裡面上帝有很多東西,.
藉著馬太想告訴我們,.
說些什麼讓我們知道進一步,.
他說我在夢中受了,因為怎樣呀?.
為他受了許多的苦,是什麼苦呢?.
我又不知道,我又不想作,.
我亦都不應該作,.
上帝說透過這一段啟示告訴他知道,.
這個秦府的夫人受了許多苦,.
在夢中,夢是上帝在人生命上常用的工作,.
尤其是在舊約裡面,.

$^{881}$他不是今天很多靈因派的信徒,.
突然間有一天走來跟牧師說,.
牧師你知不知道我昨晚做了一個夢,.
我做夢你應該把你的房子賣了,.
然後捐給我,讓我做多點事工,.
不是這些亂說廿四,憑幾義而行的行為,.
不是,我做牧者這麼多年,.
靈因派的信徒來跟我說過很多這些東西,.
牧師我昨晚又做夢,.
我做夢你應該怎樣將那個弟兄開除去會籍,.
然後我就說你怎知道你的夢是這樣,.
我沒有這個夢,.
他說你沒有這個夢,.
你的靈性低而已,.
我高啊,.
應該大家弟兄姐妹都有這樣的夢吧,.
弟兄姐妹不是那一類,.
這裡是上帝父神和耶穌同工的不明證,.
這個做法可以說在馬太福音長期出現,.
你有沒有留意在馬太福音裡面,.
我說過很多次,.
是誰經常宣告耶穌是神的兒子,.
給鬼父那些人,.
所以弟兄姐妹,.
上帝在這裡工作,.
上帝用不同的人來工作,.
而且在這段聖經裡面更加特色的地方是,.
一個外邦人的權貴家屬受到上帝的夢的啟示,.
在舊約聖經裡面也有類似的運作,.
在哪裡呢?.
以斯帖記,.
上帝如何去感動當時波斯的王,.
來重新鑒賞當時以斯帖的叔叔,.
或者養父,.
或者他的cousin,.
即是他的堂兄,.
的昔日如何拯救波斯王的歷史,.
上帝在這裡工作,.
在這個工作的過程當中,.
我們就開始看到,.

$^{921}$這裡全部的人想將自己包裝成怎樣呢?.
秦府的比拉多,.
在這個時候,.
他第一樣東西很清楚在馬太福音裡面記載,.
當他宣告要他們選擇釋放哪一個,.
他清楚說了這兩個選擇給他們知道,.
巴拉巴呢?.
還是基督的耶穌呢?.
這個秦府,.
馬上在馬太福音的記載裡面說,.
他原知道他們是因為嫉妒才將他戒了,.
將這幫人的惡毒心態暴露了出來,.
秦府不是不知道的,.
他們清楚知道為什麼他們帶耶穌來當中,.
秦府不是純粹聽你們說話,.
他也有他自己的線眼,.
線報的情報網的,.
你帶一個人來,.
我怎麼會不想清查你們究竟發生什麼事才帶他來呢?.
再加上他這個太太給他的提醒,.
於是這個比拉多雖然知道自己脫不了身,.
太太說這件事情千萬不要管,.
但是他不管不行,.
於是他不管不行之下,.
他要怎麼處理?.
他唯有將這個球扔給他們決定,.
既然你們要求的時候,.
那我就問你們要求哪一個呢?.
於是這幫人就馬上說,.
要求釋放巴拉巴,.
比拉多再說多一個問題,.
第22節,.
比拉多再說多一句,.
究竟要怎麼處理基督的耶穌,.
已經是第二次說了,.
第二次說基督的耶穌,.
怎麼搞這個基督的耶穌呢?.
比拉多在這裡提醒這幫猶太人,.
這是基督的耶穌,.
這件事是你們在宗教猶太教裡面,.

$^{961}$一個應該由你們去決定的問題,.
你們怎麼處理這件事?.
你們要我放一個賊沒問題,.
放一個假的耶穌給你們沒問題,.
但這個真的怎麼樣?.
比拉多這樣說完之後,.
這幫人就一錘定音,.
一起說,打他釘十字靶,.
巡撫再說多一句,.
我為什麼要這樣做呢?.
我找不出他有什麼惡事,.
比拉多是第三個宣告耶穌是義人,.
第一個是加里人猶太,.
第二個是巡撫的太太,.
第三個巡撫比拉多,.
我找不出他有惡事,.
他表現出這件事完全不關我的事,.
要做什麼決定,.
你們說了,.
於是這幫人全部一起極力地喊再說,.
把他釘死,.
各位姐妹,.
很多人以為自己活在聖靈引導之下,.
於是就變得瘋狂,.
這幫人看不到上帝仍然在運作,.
提醒他們,.
基督的耶穌是一位義人,.
由一個犯罪出賣耶穌的猶太,.
到現在的比拉多,.
可不可笑?.
他們是不是一幫沒有聖經的人?.
他是祭司長,.
是長老們,.
長老們正式翻譯是猶太人公會的顧問,.
顧問牧師,.
很多顧問牧師,.
知道嗎?.
橫教會有很多顧問牧師,.
一請不到牧師就找顧問牧師,.
顧問牧師,.

$^{1001}$有些顧問牧師滿口講聖經,.
其實根本不按照聖經行事,.
還要將包裝成不關我們的事,.
連這幫猶太人公會的領袖,.
都將球從比拉多扔到猶太人群眾裡面,.
這幫猶太人群眾以為自己很靚,.
用今天的例子,.
一幫合心不良的信教會領袖,.
一幫信徒會眾,.
以為自己很虔誠,.
做了很多事,.
以為自己很靈性,.
其實他們正在侍奉的是.
主義性耶穌,.
當你聽完我這樣介紹,.
不是我作的,.
這些全部是聖經裡面講的,.
我們就會看見,.
雖然拉多無論是他的太太,.
都看到耶穌基督是二人,.
但他們雖然不能脫身,.
但他盡量能夠提醒以色列人民,.
有關真正的耶穌基督,.
神的兒子這位基督耶穌,.
他的真正二人身份,.
反而屢勸不改之下的,.
卻是猶太人公會,.
以及那幫被他們煽動而仍然.
懵然不知的猶太人群眾,.
這種行徑是什麼意思呢?.
四個字可以形容,.
自掘墳墓,.
不是沒有神的話啟示,.
不是沒有上帝的工作在其中,.
而是人根本沒有一對.
淑寧的眼睛去看到之後,.
然後進入淑寧的生命裡面,.
接受這個真理..
今天我作為一個牧者,.
我看到信徒最可惜的一件事,.

$^{1041}$不是他們不想追求真理,.
而是給了他們真理之後,.
他們很難接受,.
因為要他們有責任,.
有重擔去更新改變,.
他們不肯,.
為什麼他們不肯呢?.
因為他們沒有真正凝聚悔改,.
因為聖靈不能在他們生命裡工作,.
不是聖靈無能,.
而是他們對聖靈的拒絕,.
誰告訴他們呢?.
撒加利亞先知書第十一章,.
耶利米先知書第十八章,.
十九章,二十三章,.
所以馬太將這些歸納起來,.
在一個瑤護的比喻裡,.
用耶利米先知的名字,.
顯示給我們知道,.
一班自掘墳墓的教徒的表現..
回到弟兄姊妹,.
今天很多信徒回教會,.
都為了把自己包裝成為一個好人,.
包裝成為一個聖人,.
很多牧者在教會裡面傳的道,.
其實都是把自己包裝成為一個.
很忠信上帝說話的人,.
但其實自己也不遵從上帝說話而行的人,.
這種靈性的精神分裂症,.
是充滿摧毀性,.
但他們完全不覺得是一個問題,.
因為他們覺得在這一刻,.
我feel happy,.
我覺得很舒服,.
我為甚麼要去以馬來還朕信會,.
這些教會裡面經常被黃牧師,.
經常說我們這樣有罪,那樣有罪,.
就算我沒有罪,他們也找到我的罪,.
不是我找,我只是提醒你有沒有這樣的罪,.
你有這樣的罪,你就向上帝認罪,.

$^{1081}$我自己也這樣做,.
我每天都這樣跟自己說,.
我有沒有做錯了甚麼,.
我是不是做了甚麼得罪了甚麼人,.
或是別人使我這樣,.
我又如何避免別人這樣牽動我呢,.
有時不說話比說話好,.
你跟他說多兩句,他還有位置進去,.
你不要跟他說,.
不是我對罪沒有了解,.
是那個人對罪沒有了解,.
你做基督徒不提醒他,.
你做牧師不提醒他,.
不是我提醒他一定會醒,.
留下上帝的靈在他裡面運作,.
這種就真真正正是與聖靈同工,.
領主們,一班人都包裝到不關他們事,.
其實全部都關他們事,.
關比拉多事,.
關這班猶太人著名的事,.
關這班猶太人群眾的事,.
今天有甚麼關我們事,.
耶穌死在十字架上,.
怎會不關你和我事,.
不過怎樣關你和我事,.
就是因為你和我有沒有真真正正,.
是被聖靈光照之後而願意悔改,.
還是我們繼續見到這班猶太人工會,.
他們是掌握權柄的人,.
我們依附他,順應他,.
就好像很多信徒,.
去大教會那裡,.
不要理那些牧師陰謀,.
總之他討好我,大事一定好,.
就算不好,一起死都一起死一大堆,.
我曾經以前有人傳道給我聽,.
如果這麼多信徒都錯,.
你只是其中一個,.
如果每個都對,.
你就逃脫了機會不信耶穌,.

$^{1121}$我覺得這句說話應該反轉來看,.
這麼多個都錯,.
你是否想陪他們一起錯,.
還是你是那個仍然看得到,.
上帝真正的真理,.
你都透過我可以做到的事,.
或者我不需要做都可以感動到人的事,.
不讓人回轉回來,.
去到神的面前,.
那就是和心對抗,.
慈悲父神多謝你.
讓我們在這段聖經裡學習,.
馬太福音給我們對於在靈性上的反思,.
遠遠比我們只是單純看耶穌受死,.
那種表面的陳述深而又深,.
一主求你繼續幫助我們,.
讓我們在你的聖靈引導之下,.
能夠作更深沉的反思,.
從而讓我們往這條永恆的天路奔走的過程裡,.
站穩在你的說話當中,.
託讓我們在過程當中若有偏差,失足,.
求聖靈的鞭策,.
讓我們可以回轉過來,.
求主你的大靈的手,.
在這個世界上觸動那麼多人的生命,.
可以讓他們凝聚悔改,.
求你的大靈,.
禱告奉靠主耶穌的德性命治,.
安排..
\newpage



\section{馬太福音 27:24-31}
\label{sec:HaaLhKYBRSg}
\textbf{一群全勝的輸家(馬太福音27\_24-31) - 黃紹權牧師  [馬太福音信息系列 - 第145講]}
\newline
\newline
連結: \href{https://youtube.com/watch?v=HaaLhKYBRSg}{\texttt{ https://youtube.com/watch?v=HaaLhKYBRSg}} ~~~~ 語音日期: 2025-01-21 
\newline
\newline
\hyperref[sec:ZN4O4BAmHMA]{< < < PREV SERMON < < <}
~
\hyperlink{toc}{[返主目錄]}
~
\hyperref[ch:preacher10]{[返講員目錄]}
~
\hyperref[sec:oCpi7n8ictU]{> > > NEXT SERMON > > >}
\newline
\newline
馬太福音 27:24-31
\newline
\begin{longtable}{cl}
\hline
\hline
章節 & 經文 (和合本修訂版)\\
\hline
27:24 & \begin{tabularx}{0.7\textwidth}{X} 彼拉多見說也無濟於事,反要生亂,就拿水在眾人面前洗手,說:「流這人的血,罪不在我,你們承當吧。」 \end{tabularx} \\ \\ \relax
27:25 & \begin{tabularx}{0.7\textwidth}{X} 眾人都回答:「他的血歸到我們和我們的子孫身上!」 \end{tabularx} \\ \\ \relax
27:26 & \begin{tabularx}{0.7\textwidth}{X} 於是彼拉多釋放巴拉巴給他們,把耶穌鞭打後交給人釘十字架。 \end{tabularx} \\ \\ \relax
27:27 & \begin{tabularx}{0.7\textwidth}{X} 總督的兵把耶穌帶進總督府,把全營的兵都聚集在耶穌那裡。 \end{tabularx} \\ \\ \relax
27:28 & \begin{tabularx}{0.7\textwidth}{X} 他們脫了他的衣服,穿上一件朱紅色的袍子, \end{tabularx} \\ \\ \relax
27:29 & \begin{tabularx}{0.7\textwidth}{X} 用荊棘編了冠冕,戴在他頭上,拿一根蘆葦稈放在他右手裡,跪在他面前,戲弄他,說:「萬歲,猶太人的王!」 \end{tabularx} \\ \\ \relax
27:30 & \begin{tabularx}{0.7\textwidth}{X} 他們又向他吐唾沫,拿蘆葦稈打他的頭。 \end{tabularx} \\ \\ \relax
27:31 & \begin{tabularx}{0.7\textwidth}{X} 他們戲弄完了,就給他脫了袍子,又穿上他自己的衣服,帶他出去,要釘十字架。 \end{tabularx} \\ \\
[1ex]
\hline
\hline
\end{longtable}
$^{1}$好 各位姐妹早晨.
在我凌定主道的意思.
讓我們先一齊心低頭禱告.
讓我愛主愛主神.
天父我多謝你.
讓我們能夠再次有生命有氣息.
來到你的身前.
在敬拜的裡面.
不單歐哥頌讚.
並且有你的算話的啟示.
讓我們能夠更加準確地掌握.
上主你的心思意念.
讓我們可以看到.
我們願意求恩主你怎樣帶領.
每一個屬你的子民.
能夠在真理的上.
能夠去怎樣生根建造.
我們懇求恩主你的帶領.
無論在上線下線的.
每一個弟兄姐妹.
都讓我們能夠在.
我們所處的環境裡面.
作合神心意的見證.
禱告奉靠.
主耶穌也得勝為名至其合.
Amen.
我認識有一對夫婦.
他們經常吵架.
一天到晚都在吵架.
於是怎樣呢.
其中一方.
他就去.
每一次嘗試.
跟他的配偶說.
你要提出的理由.
或者你提出的事情.
其實都沒有什麼理由.
你可不可以去.
真心思想想.
但是不說又是可.

$^{41}$說了之後.
火上加油.
對方就再罵兩聲.
其中一方.
勸導的他看到.
在這樣的情況下.
既然再說都沒有什麼改變.
既然對方.
每次都要爭取勝利.
於是他就下了.
一個這樣的決定.
以後每次對方.
說什麼.
他都任由他說.
希望.
即使.
勉得雙和氣.
但在這個做法裡.
最後造成了一個現象.
就是.
其中一方.
經常都要爭拗.
他反而就認為.
他自己贏了所有.
這個關係.
但他卻不知道.
他輸了他的婚姻.
和配偶對他的愛意.
認識一個員工.
在一個公司裡.
可以說是公司的.
靈魂人物.
因為他所認識的.
技術可以說是.
那間公司裡.
沒有別的可以認識.
於是這個人.
就有了這樣的想法.
在公司越做得久的時候.
他就以為.

$^{81}$這間公司沒有他.
就搞不定.
而他的上司.
也要聽他的說話.
如果他不做好他的工作.
可能連他的.
上司也可能.
連工作也沒有.
當他很得意.
忘形的時候.
誰不知.
當經濟轉差.
公司真的要去到.
做裁員的工作的時候.
所有的.
管理層第一樣.
想到的.
就是要將這個技術員.
辭退.
因為他在公司.
說話太多.
也以為沒有他.
不行.
於是整間公司裁員.
當中只裁這個人.
沒有裁其他人.
因為他是公司.
最高的技術人員.
所以被裁所節省的.
那份工資.
沒有給其他同事.
可以繼續和這間公司.
共赴難關.
上面我所提出的.
這兩個耳熟能詳的.
例子.
我相信.
每一天在這個世界裡.
不斷上演.
接下來.

$^{121}$下星期.
明天.
美國有一個新的總統.
現在大家都很擔心.
在美國總統.
新上任之後.
發生的.
一件事.
不是加拿大被加關稅.
最擔心的.
就是.
會不會有一個.
現在的新任.
總統.
身邊的一個紅人.
功高蓋主.
而造成了一個.
更大的破壞.
在這個世界.
這個正正被我們看到.
在這個世界上.
很多時候有很多人.
以為自己很了不得.
但其實.
最大的問題.
可能就是.
因為你的存在.
導致有更大的危機.
所以廣東人.
有一句說話.
人怕熊.
豬怕肥.
為什麼?.
(怕出名).
一樣.
人怕出名 豬怕肥.
為什麼?.
因為當人越來越覺得.
自己很了不起的時候.
要宰割.

$^{161}$那個對象.
正正就是你.
而這個情況.
卻是發生在.
整個世界人類的歷史裡.
包括.
這群我們今天會看的.
經文去陷害.
耶穌的人.
其實他們都以為自己.
很了不起.
甚至他們是贏了.
但卻是在.
最後我們所了解到.
輸的.
正正就是這群人.
所以今天和大家一起分享的一個.
講道題目叫做.
一群全盛的輸家.
他們以為自己全勝.
但卻是最後.
輸的.
就是他們這群人.
不如我們一起去看這段聖經.
在馬太福音第27章.
第24節.
至第31節.
一起打開之後我們一起讀.
這段聖經不是太長.
一起讀.
使我們更加認識這段經文的內容.
一起打開.
馬太福音27章24至31節.
比拉多見說也無濟於事.
反要生亂.
就拿水在眾人面前洗手.
說.
流者二人的血.
罪不在我.
你們成當吧.

$^{201}$眾人都回答說.
他的血歸到我們和我們子孫身上.
於是比拉多釋放巴拉巴及他們.
把耶穌鞭打臉.
然後把他們的血.
流到他們的身上.
然後把他們的血.
把耶穌鞭打了.
教級人釘十字架.
巡撫的兵.
就把耶穌帶來衙門.
調全營的兵.
都聚集在他那裡.
他們給他脫了衣服.
穿上一件.
珠紅色袍子.
用荊棘編造冠冕.
戴在他頭上.
那根位子放在他右手裡.
跪在他面前.
戲弄他.
說:恭喜.
猶太人的王.
猶兔躺沒在他臉上.
拿位子打他的頭.
戲弄完了.
就給他脫了袍子.
仍穿上他自己的衣服.
帶他出去.
要釘十字架.
他們出來的時候.
遇見一個古里內人.
名叫西門.
就勉強他同去.
但西門.
就勉強他同去.
抱背著耶穌的十字架.
我讀完第32節.
因為今天是第32節.
西門我們說什麼呢?.

$^{241}$西門我們說到.
人一方面要犯罪.
但一方面又要.
努力地.
把自己包裝成一個好人.
這種靈性的分裂症狀.
其實是充滿摧毀性的.
沒有一個真正相信耶穌基督的人.
會這樣做.
甚至這班要把自己裝作漂亮的屬靈人.
不過是假屬靈人.
他們會做到更加盡的.
就是把一個真正的耶穌.
用一個假的耶穌去取代.
上星期我們看過.
他們用了一個稱為.
耶穌巴拉巴的人.
去取代一個真正來到這個世界.
成就救恩的神的兒子耶穌基督.
加勒人就是其中一個典型的例子.
除此之外.
我們今天會更進一步去看.
原來火拍著這班文士.
祭司長和發利賽人的人民.
猶太人.
他們怎樣以為自己可以得到他們的心願成就.
但最後反而成為了魔鬼陰謀的火拍伴侶.
我們一起去看這段聖經.
分為兩大部分去看.
第一部分從第24節至第26節.
我給一個副題.
這段聖經叫做.
似乎都是贏家的輸家.
The winning losers.
我們一起去看這段聖經說什麼.
我們已經說過.
這班猶太人的領袖.
或者猶太人公會的領袖.
他們去煽動了一班猶太人的民眾.
而這班民眾他們非常之熱情.

$^{281}$熱情到一個地步.
他們要將耶穌釘死.
而這件事比拉多是多次提醒他們的.
比拉多問他們.
你們想在這個時間放一個囚犯.
你們想放誰呢?.
其實耶穌當時已經被列為其中一個囚犯.
但他們卻選擇了假的耶穌.
叫做耶穌巴拉巴.
當他們選擇了這個決定之後.
比拉多再問多一句.
如果你要選擇這個.
這個另一個稱為基督的耶穌.
那怎麼處理呢?.
這班人群情非常之洶湧.
要將他釘死在十字架上.
連怎樣殺他都已經決定了.
比拉多再問多一句.
他犯過什麼事.
你們要這樣做呢?.
再提供他.
可以說比拉多進行了兩至三次提醒.
一班陷在罪狀當中的猶太子民.
你們想清楚吧.
給我一個理由.
我嘗試在這個過程當中.
辯白你們是否真真正正看到他有罪.
是配去釘死十字架.
但誰不知在第23節裡.
這班人什麼答案都沒有.
只是極力地喊著.
將耶穌釘他十字架.
在這一連串群情洶湧的情況.
比拉多深刻明白到.
這班猶太人的群眾.
他們要用國語說.
割掉他.
廣東話就是殺掉他.
既然他們到了這個地步.
比拉多很清楚.

$^{321}$再說話都沒有意義.
我們要留意.
比拉多為什麼要有這樣的處理方法.
因為比拉多是一個按照羅馬政府懷柔政策的手段去管治.
最重要的是不要亂.
這句說話很熟悉.
不要亂.
讓一讓不要亂.
於是怎麼樣呢.
如果再說.
連我都搞不定.
可能這班群眾會對比拉多有反感.
所以比拉多在和諧本說.
他知道見說也無濟於事.
反要生亂.
他不想在這種情況下.
再挑起民眾有任何的不高興.
於是他就做了一個行為.
他做了一個什麼行為呢.
在聖經裡說.
他唯有任由他們去做.
但任由他們去做.
變成了我自己是一個行兇者.
有份參與在其中.
比拉多在這裡.
你可以說他嘗試將自己抽離.
但也可以告訴我們.
比拉多知道這件事不要尖手下去.
因為這件事有很長的尾巴.
我們之後會知道這件事有很長的尾巴.
他現在這樣做.
我們之後會知道他怎樣做.
他都很長的手臂.
但比拉多明白到.
他太太 夫人老了就叫他.
千萬不要在這件事上有任何的參與.
千萬不要用這段聖經來說.
所以一個成功的男人背後.
女人或聽老婆的話.
你一定會發達.

$^{361}$聖經不是說這些.
在這個時候他做了很多盤算.
繼續這樣做下去.
只會使自己的官職受到威脅.
於是他就叫人拿出一盤水去洗手.
有些人就會在這裡.
牧師 這些就是我們廣東人.
或中國人常說的金盤洗手.
以後世界事我都不處理.
特別是看黑社會片.
很多都是這樣.
通常江湖大佬收生之前.
找個金盤.
我不知道是不是金盤.
就出來洗手.
以後江湖之事我不再理了.
是不是這個意思.
聖經是否在這裡教金盤洗手.
可以潔身於道外呢.
當我們仔細看回.
這群民眾的回答的時候.
我們就會明白多一點.
這個所謂洗手是什麼意思.
比拉多拿了盤水出來.
在眾人面前洗手之後.
他就這樣說.
流者二人的血 罪不在我.
你們承當吧.
比拉多在這裡說了這番話.
很清楚地告訴群眾.
這個判處不是出於我的決定.
是你們的決定.
所以從這個上下文.
連貫看這個所謂洗手這件事.
其實就代表著.
比拉多明白這件事.
是和「陷二人於不義」這件事.
是一個直接的連結.
不是不理江湖的事.
比拉多不可以不理江湖的事.

$^{401}$因為之後我們發覺到.
比拉多仍然在管治裡.
有一個很重要的角色.
但是這件事所謂洗手.
其實是強調一件事.
是和流罪人的血這件事分離.
因此這段聖經.
比拉多這個行動.
是將他自己在判詞上.
表明不關我的事.
不是不想再理會.
他會繼續理會.
不過這判詞罪不在我.
最有趣的是你們承當吧.
和合本就寫你們承當吧.
英文聖經NET的翻譯.
You take care of it yourselves.
當我們讀到這句聖經的時候.
原來在原文聖經裡.
是說到你們自舉去Orao.
Orao是什麼意思呢.
Orao這個字指的是看.
看東西.
看清楚了.
謹慎看路走.
明白嗎?弟兄姊妹.
原來正正在之前第27章第4節裡.
有一群人也這樣說過.
在第27章第4節裡.
我們看到當交流人猶大.
他知道自己賣無辜人的血.
是有罪這件事的時候.
這群濟士長和長老們.
也說了這句話.
你自己承當吧.
這次很聰明.
用同一個字眼來翻譯.
你們自己承當吧.
你自己看吧.
不關我的事.

$^{441}$以後的路怎麼走.
你自己看路走吧.
大家讀到這裡.
就會發覺.
其實無論是濟士長.
以至於中長老們.
和現在的比拉多.
他們都叫他們自己看路走.
在這樣的情況下.
比拉多也明白.
他可以正正式式的.
脫離這個.
陷義人於不義的行為.
這份罪不在他身上.
因為這個審判.
不是比拉多所同意的.
不過分別在於什麼呢.
和這個眾濟師長.
比拉多知道自己要離開這個罪.
但是眾濟師長.
他們無論怎麼說.
都仍然在這個罪裡面.
為什麼呢.
整件事的策劃.
以至於煽動.
都是在眾濟師長裡面.
一起去做.
這個所謂洗手的行為.
其實在聖經裡面.
申命記第21章第6至7節裡面.
曾經有這樣記載過.
在申命記第21章第6至7節裡面這樣說.
那誠的中長老.
就是離被殺的人最近的.
要在那山谷中.
在所打折頸項的母牛.
獨以上洗手.
禱告說.
我們的手未曾流這人的血.
我們的眼也未曾看見這事.

$^{481}$如果我們讀到這裡來算.
更加肯定多一件事.
讓我們知道.
比拉多不是一個.
盲中去管治巴勒斯坦的人.
他是一個可以說是猶太通.
或者可以這樣說.
他身邊有一些猶太通.
來幫助他管治.
他做每一個行為.
每一個決定.
都嘗試用當地那班人的民情.
來作出管治的行動.
等同於那些人.
譬如我要去和英國政府.
有些溝通的.
摸清他們的思路.
英國通.
如果現在美國要加加拿大.
不知多少關稅.
或者不加.
這些不是我們討論的.
不過你要找個美國通來問問.
你不是自己飛下去.
和人吃飯就搞得定.
尤其是你和人都不通.
找一個通的了解了.
比拉多原來是這麼小心地運作.
就算他要將這件事置身於道外.
他仍然要按照猶太人的規矩.
原來有些兇殺案的罪案.
如果有一個無辜者.
是被參與在其中的.
被受害的.
原來要用這種行為.
在你們傳統有這樣做的.
並且當每次這些免責聲明.
要去講出的時候.
都有一個洗手的行動.
除了剛才的申命記第21章6-7節之外.

$^{521}$詩篇第26篇第6-7節裡.
也都同樣有一個洗手的行為.
往往在位的人需要宣判一些事情.
而這個宣判的判詞不是他同意的.
不過群情需要了.
民主政制下蹈.
Honestly 民主政制不一定是好事.
按照亞里士多德.
民主政制不是一件好事.
不過民情要這樣.
大家要玩TikTok.
就是這樣.
這樣民情洶湧就不行了.
你逼著要這樣做.
但事情與我無關.
詩篇第26篇第6-7節同樣說.
耶和華我要洗手表明無辜.
才還擾你的祭壇.
我好法清謝的聲音.
也要述說你一切奇妙的作為.
詩人都這樣看.
有些時候要陷無辜人的血.
我都害怕.
在這種情況我有份.
所以我要用洗手這個行動.
在上帝面前表明.
我不想參與這些事.
然後才去到上帝面前.
發出敬拜讚頌的聲音.
比拉多做了這樣的判決.
他完全按照猶太人的傳統去做.
這班猶太人又如何回應呢.
看完了.
眾人都回答說.
他的血歸到我們和我們的子孫身上.
這班人我自己這樣說.
你自己壞就好了.
你這個人要株連自己的後裔.
其實這種說法.
我們叫做說得大一點.

$^{561}$你知道有些人.
叫做吹牛.
你知道為什麼要吹牛嗎.
其實他自己都知道.
他說出來的東西是有虛假成份的.
不過說大一點的時候.
讓人以為這件事是你肯定的行為.
我小時候.
我媽媽經常跟某人說輸賭.
說某人做了什麼事.
輸賭的時候.
有句話出現.
叫做我找個人頭搏芋頭.
有沒有聽過這句話.
什麼叫做找人頭搏芋頭.
我用我的頭擔保.
換你的芋頭.
我說的話我看的東西一定沒錯.
這種說法.
其實就是這班猶太人在這一刻所說的.
所以大家明白到他們說話說大了.
就是這個意思.
殺耶穌這件事.
這件血殺人.
他的血有什麼後果.
歸到我們身上.
不單歸到我的子孫身上.
聽起來就像連害自己的子孫.
其實他是想說.
沒錯我是被人攬上身了.
所以這句話大家也了解到.
為什麼在出埃及記裡面.
當上帝說到追討人的罪的時候.
致父及子直到三四代.
是不是只是追三四代這麼少.
不是.
就是說這件事上帝追到底.
明白嗎大英小輩.
這班人他們犯罪犯到底.
所以在這樣的情況下.

$^{601}$既然你們要人頭搏斧頭.
比拉多就知道這次成功了.
完全不關我的事.
你們說的.
不單我禮儀也做了.
你們的口也說了.
於是比拉多就按照他們所說的行動去執行.
所以二十六節.
比拉多釋放巴拉巴.
即是耶穌巴拉巴.
假耶穌救不了你的.
沒有救恩的假耶穌.
給了他們.
然後就將真正的神的兒子基督.
稱為基督的耶穌.
就在這裡把他鞭打.
然後交給人去進行釘十字架的行動.
大英小輩有什麼學習呢.
在這裡.
比拉多好像贏了這一仗.
這件事他老婆又安心.
他夫人又臨急臨忙.
要找個人來中途提醒他.
這件事一定要不要碰.
老公就很開心了.
回家後就不會被老婆罵了.
他自己又安心.
因為保住官職.
能夠防止任何動亂發生.
更重要的是什麼呢.
比拉多為自己鋪上一層.
很漂亮而猶太教的人.
覺得很神聖的面紗在身上.
我為什麼要提醒大家這件事呢.
你知道很多時候.
傳假道的人.
往往把自己包裝得很高超卓越.
他盡量把自己表現在眾人面前.
我是完全屬於上帝的.
親近我就等於去到上帝面前.

$^{641}$把自己包裝得很漂亮.
比拉多好像贏了.
老婆在自己家裡又開心.
對外的工作又穩固.
在管治猶太人的宗教層面上.
也鋪上一個很美麗的外表.
這種情況簡直是.
Win Win Win Position.
他就全贏了.
眾祭司長和眾長老.
他們都以為全贏了.
為什麼全贏了.
因為他們找到一群替死鬼.
這群替死鬼是哪一個.
就是眾群眾.
這群眾猶太人.
這群眾猶太人被祭司長和長老們.
搞得他們昏昏沉沉的情況下.
他們終於求仁得仁了.
不過人就是人.
不是仁義的仁.
而是不仁不義.
他們在這個情況下.
覺得自己好像很偉大.
好像實踐了猶太教裡面.
那份很虔誠的宗教投入.
甚至他們以為.
他們保住猶太教.
但他們以為護教成功也好.
但他們卻是失去了.
那位私行拯救的主耶穌.
他們陷在流無辜人的血的罪.
而他們親口說.
這血歸到我們及我們子孫裡.
在這些各種的舊約裡面.
講了一件事實.
一種只能夠讓自己有宗教狂熱.
甚至宗教感滿足的屬靈生活.
其實是很危險的.
危險在你以為自己已經得著了.

$^{681}$但原來你卻是完全輸了.
這點形給我們看到.
他們做了代罪羔羊.
他們自己也覺得沒問題.
他們覺得自己在裡面好像很健全.
反過來看今天的教會的信徒.
我們說過信假的耶穌有很多好處.
今天有很多信徒很喜歡.
什麼好處呢?.
假信徒很喜歡.
容易.
不用有什麼責任感.
不用追求.
去教會就可以了.
有時做一點點事.
很多人哄你.
你很健全.
還有在這種情況.
你要把自己包裝得很漂亮.
在別人面前覺得自己很誠信.
今天很多信徒回教會就求這些.
有很多賣假信仰的傳道人目的.
就是推銷這件事的做法.
所以就搞很多事工給你搞.
喜歡搞話劇.
喜歡唱詩.
喜歡唱粵曲.
就做粵曲.
粵曲福音.
盡量滿足你喜歡怎樣搞.
當你以為每樣都滿足你的時候.
你靈性就以為很好.
其實表面很好.
但你卻是走到一個完全被出賣的狀態.
很多使你陷在溫溫凍凍的傳道人.
其實就好像昔日的祭司長和長老們一樣.
他們不是你的盟友.
他們不是盟友.
他們只是希望你能夠被他們利用.
因為他們要把你變成和他們的命運共同體.

$^{721}$和你一起走在一個欺絕耶穌基督的狀態.
犯罪的人就是這樣.
你越犯罪就越想更多人和你犯罪.
在這樣的狀態下.
你就不會覺得自己犯罪是很突兀.
反而你會覺得自己很正常.
那些不想犯罪的人反而被人針對.
你不想.
比拉多也是其中一份子.
不過他做得巧妙一點.
比拉多其實也是不得救的.
他嘗試說很多話去啟明猶太人.
但他的動機從來不是想他們脫罪.
是想自己脫身.
所以在這種情況下.
我們可以看到一群似乎全部都贏的人.
但其實這群人全部都輸了.
而這種狀況不是說到了這一點完結.
不是的 而是更加深陷在裡面.
這就是馬太福音記載的特色.
犯罪的人不是滿足了就不犯罪.
他們滿足了之後.
他們的犯罪會更加深入.
接下來我們來看下一個部分.
從第27章第27節至第32節的經文.
我們一起去看這部分的經文.
為什麼32節在最初頭我沒有叫大家去讀呢.
因為這節經文在釋經上可以歸類到下一段解釋.
但在釋經上我發覺貼在上文就更加貼切.
我們一起去讀多一次.
從第27節至第32節.
巡撫的兵就把耶穌帶進衙門.
叫全營的兵都聚集在他那裡.
他們給他脫了衣服.
穿上一件珠紅色袍子.
用荊棘編造冠冕.
戴在他頭上.
拿一根圍紙放在他右手裡.
跪在他面前.
戲弄他說.

$^{761}$恭喜猶太人的王啊.
又套襯在他臉上.
拿圍紙打他的頭.
戲弄完了就給他脫了袍子.
還穿上他自己的衣服.
帶他出去要釘十字架.
他們出來的時候.
遇見一個古利內人.
名叫西門.
就勉強他同去.
好背著耶穌的十字架.
在這段聖經裡.
我給他一個副題.
有怎樣的領袖.
就會有怎樣的隨從.
英文叫.
Like father, like son.
或者Like leader, like follower.
在這段聖經裡.
解釋.
在我自己所認識的.
過去聽過的目者的解釋.
都嘗試將這段聖經.
將他以耶穌成為猶太人的.
君王的登基典禮.
來介紹.
不過我覺得這個說法.
不太合常文.
沒錯.
似乎這段聖經.
好像說到耶穌.
真的被封為王一樣.
所以這段聖經.
我們看看他們會將耶穌怎樣.
他又要穿珠紅色袍子.
又要有荊棘的冠冕.
又有靚袍.
又有皇冠冠冕.
還有大家不要漏了.
在他右手裡.

$^{801}$有一支權杖給他.
是不是這樣解釋.
這段聖經.
如果是這樣解釋.
這段聖經完全拿不到.
主旨想說什麼.
我們上文已經說過.
這群人一直想陷在罪當中.
又不想將自己.
表露出.
自己是一個.
陷人於不義的行徑.
他們要做壞人.
但同時又要裝作包裝成為好人.
在這個大前提之下.
我們看看.
正式怎樣解釋呢.
比拉多在這個時候.
他真真正正使出他的縮骨功.
怎樣縮骨.
他在這個時候.
他嘗試將自己抽離.
所以一開始.
27節.
巡撫兵.
把耶穌帶去衙門.
首先留意.
26節裡.
比拉多釋放巴拉巴之後.
把耶穌鞭打.
交給人處理.
這件事他完全沒有尖銳在其中.
他又找了第二批.
第二梯隊的受罪羔羊.
就是他的兵士.
這群巡撫的兵士.
是近身的羅馬兵士.
或者用錢收買的.
其他民族.
包括猶太人的人.

$^{841}$來幫他做兵.
不過照我們所看.
這應該是巡撫羅馬人的兵士.
他把交代交給他們處理.
聰明了.
怎樣處理呢.
他們將他帶去衙門.
衙門就是中國人的傳統說法.
其實正式來說.
這就是他們的軍事大本營.
按照耶路撒冷的設計.
應該是當時將耶穌壓戒到.
一個所謂安東尼亞城堡.
Antonio Fortress的地方.
這個地方在哪裡呢.
這個地方我去過.
就在耶路撒冷東北角落口.
一個出口的地方.
這個地方的門.
原本叫做Stephen Gate.
聖史提芬的出口.
後來改名叫做Lion Gate.
獅子出口.
那個出口上面也有隻獅子頭.
為什麼有隻獅子頭呢.
這個象徵著羅馬人的統治.
這班士兵將耶穌帶到他們的大本營.
在安東尼亞城堡.
然後比拉多容許他們全員的士兵聚集在那裡.
你沒有看錯.
現在他們抓了多少人呢.
一個人.
壓戒一個人.
要用全營的士兵.
一個羅馬人的一營的士兵.
大約是由.
這裡說的中文.
中文說的看不到.
全營的士兵.
大約精調是五百至六百人.

$^{881}$如果在這樣的結構裡.
我們會問一個問題.
你壓戒一個已經被你綑綁.
也看到他不會走的人.
叫做耶穌基督.
你用五六百人來壓戒他.
想搞什麼呢.
這也是為什麼有人將聖經說成.
耶穌基督登基大典的解釋.
因為五六百人的士兵群體.
擁護著你.
這是典型的登基大典的狀況.
不過在這件事上.
這五六百人想做什麼呢.
在第28至31節就記載了給我們看.
這群人在做什麼呢.
一個不成比例的狀態.
為什麼要這樣做呢.
因為在這段聖經裡.
要反映出秦府比拉多有多詭詐.
或者有多低賤.
他們要包裝得漂亮.
雖然想盡量離開.
不關這件事.
但也要表現出.
我是擁護你們的.
我們支持你們.
於是用五六百的陣容.
將整個狀態做到聲勢非常燦爛.
但抓了五六百人做什麼呢.
首先將耶穌的衣服脫了.
請大家留心.
第31節.
當戲弄完之後.
就吸他脫了袍子.
然後再凝穿上他自己的衣服.
為什麼要大家看那一段呢.
整段經文所做的事情.
除了脫衣服.
但之後又穿上原本的衣服.

$^{921}$想搞什麼呢.
為什麼要這樣做呢.
我們先看下去吧.
脫了衣服幹什麼呢.
然後為他穿上一件珠紅色的袍子.
一般的軍王.
穿的袍子不是紅色的.
不是英女皇.
不是查理斯三世.
不是.
他們那些是紅色的.
一般以前的軍王家是穿紫色的袍子.
為什麼要穿上珠紅色的袍子呢.
有些人不懂正經的.
珠紅色就是保血.
靈異解經很容易說的.
其實不是保血.
不是象徵耶穌基督的皇權.
不是.
不是在說這個.
珠紅色的袍子.
其實是典型的羅馬士兵所披的外袍.
即是說這班士兵隨手拿一件在他們當中.
拿出來給耶穌披上去.
披完之後怎麼樣呢.
披完之後再用荊棘為他做一個冠冕.
你做過皇冠就做過皇冠.
為什麼要用荊棘去做冠冕呢.
很多人不知道.
原來在耶路撒冷聖安東尼城堡一走出去.
那個地區有一個河谷.
那個河谷我們很熟悉.
為什麼熟悉呢.
因為河谷的頂部就是昔日窯戶掘地.
用來採取材料做瓦器等等的地方.
很熟悉.
什麼人在那裡呢.
伽略人猶大.
剛剛才記載他埋葬在那個地方.
是一個荒蕪之地.

$^{961}$換句話說.
他們在一個荒蕪之地隨時都撿到荊棘.
大家知不知道什麼叫荊棘.
有時去看西部牛仔片.
特別那個叫做什麼.
擔住一支磨乾草的那個.
是不是奇形異事活.
有首歌就來了.
然後呢.
你會看到地上有東西在滾動.
有風吹著它滾過來.
那些就是荊棘.
是有刺的野生植物.
在荒蕪之地.
在聖經所說的.
就是在那裡隨便撿東西出來.
當成荊棘.
不用擠的.
我給你一條荊棘你擠給我看.
我想你摸不到那支荊棘.
牧師你不要搞了.
不是用藤來擠的.
藤可以擠東西.
荊棘不是用來擠東西的.
隨便拿一件來套在耶穌身上.
當皇冠.
還有什麼呢.
戴了它在頭上之後.
拿一條蕙子放在他右手裡.
蕙子就是蘆葦草.
隨便拿一條蘆葦草.
為什麼要隨便拿一條蘆葦草呢.
大家有沒有見過蘆葦草在加拿大.
有沒有見過.
有啊 加拿大很多地方都有蘆葦草.
蘆葦草呢.
通常上面有一球東西.
特別高的.
硬硬的.
有一次我和師母去一個島.

$^{1001}$叫蛇島.
安德里有得去的.
坐船 免費船.
坐過去.
坐過去之後.
一直在岸邊蘆葦草.
硬硬的.
就那一條.
整件事是告訴我們.
這班兵兵.
五六百人一起搞一場.
大龍鳳.
一場戲.
這場戲是做什麼的呢.
把耶穌包裝成.
先前巡撫比拉多所說的.
他是猶太人的王.
那我就把你包裝成.
像一個王一樣.
當我包裝你像一個王之後.
怎樣對待他呢.
我們看清楚.
他怎樣對待他呢.
當這一切做好之後.
這班人就在耶穌基督面前.
戲弄他.
說一句說話.
恭喜你猶太人的王.
其實這句說話.
和合本寫得太平鋪直敘了.
我們將《斷聖經》.
立體化一點去看.
他們的說法就是.
恭喜你猶太人的王.
類似我小時候.
跟我爸爸農曆年初一.
年初二才去拜年.
在行家之間.
年初一年初二年初三收市.
年初二開始.

$^{1041}$看到行家出來.
逛街的時候.
或者喝茶.
我爸爸就是這樣.
恭喜發財.
還是真的恭喜他發財呢.
我不知道.
我想是吧.
我爸爸也不想人壞.
氣派就是這樣.
他越想你恭喜發財.
其實最好財來我那裡.
不是來你那裡.
你明白嗎.
一世人的思想.
就是餅這麼大.
我給了你分了.
我就少了.
這班猶太人.
把耶穌交給羅馬的兵丁.
羅馬的兵丁.
都用猶太人的思想.
來串耶穌.
這不是耶穌登基典禮.
這段聖經是說.
這班羅馬的兵丁.
如何將耶穌恥笑.
他越說你是猶太人的王.
他就越不相信你是猶太人的王.
三十節.
然後就向他吐口水.
吐口水不用解釋.
這是中國人的文化一部分.
看不起他們.
吐口水在他們身上.
然後就拿著蘿蔔.
有沒有留意.
然後用蘿蔔來打他們的頭.
等一等.
他們這些行為.

$^{1081}$有好像跪在那裡拜耶穌.
有好像在那裡高捧他.
稱呼他為王.
但實際上夾雜著中間的.
就是鄙視.
戲弄,吐口水.
鄙視耶穌這位.
所稱為猶太人的王.
並且用蘿蔔來打他的頭.
蘿蔔有什麼作用呢.
這裡我想多加一點.
每個君王都有一支杖.
大家知不知道.
這叫做shaft.
如果大家記得聖經裡.
有一個王.
在以斯帖記裡.
以斯帖沒有被召見的時候.
會死的.
主動不能見王.
但他說死就死吧.
見到王.
見到以斯帖皇后.
就伸出權杖.
指著給他.
施恩給他.
這群猶太人.
把耶穌交給一群外邦人.
也用外邦人的神官.
或者君王官.
拿著一支杖.
用這支杖來打耶穌.
完全不是施恩.
而是一種侮辱.
你的皇權.
我隨時可以拿走.
當他們戲弄完之後.
就脫下衣服.
穿上耶穌的衣服.
然後就開始帶耶穌出去.

$^{1121}$釘十字架.
這聖經和26節.
剛剛好包住.
整段聖經想說什麼呢.
整段聖經想說給我們知道.
其實這群兵兵下流的行為.
其實是反映了領袖秦撫比拉多.
表現什麼呢.
內心有多下流.
但表面就好像不下流一樣.
怎麼不下流一樣呢.
把耶穌抓進衙門裡.
玩了一輪之後.
按照他們的喜好.
怎樣去奚落耶穌.
但他們不想讓人知道.
因此他們先脫下耶穌的衣服.
玩完一輪之後.
穿上耶穌的衣服.
人們帶出去的時候.
怎麼會覺得他們虐待耶穌.
戲弄耶穌.
這種包裝.
其實反映了秦撫比拉多.
同樣是這樣的心態.
有這樣的領袖.
就有這樣的跟隨者.
而這些跟隨者的行為.
反映了領袖.
不多不少的狀況.
尤其是他們的靈性狀況.
我們看回這個行為.
一直延續到往後.
繼續在耶路撒冷教會裡.
不斷發酵.
不過在這個圖畫裡.
更加要將它推到一個極點.
但在這個極點也帶出了一個轉捩點.
第32節.
我為大家讀出第32節.

$^{1161}$大家請留意.
第32節這樣說.
他們出來的時候.
遇見一個古利內人.
名叫西門.
勉強他和她背著耶穌的十字架.
在整段經文裡.
最奇怪的就是第32節.
為什麼在耶穌前往處決路程上.
特別要馬太記載這個古利內人西門呢?.
留意.
在這群士兵上路的過程中.
遇見了這個古利內人.
首先古利內是哪個地方呢?.
古利內不是在巴勒斯坦地區.
古利內是北非的一個重要城市.
今天來算是利比亞的首都尼泊爾.
好像改了名字.
迪尼泊爾.
原來在耶穌上十字架的路程上.
有一個從外面世界而來的人物.
這個人叫西門.
不要小看這個記載.
馬太刻意放在這裡.
西門這個名字.
不是中國的西門吹雪.
西門是典型的猶太人名字.
一個從古利內而來的西門.
即是用猶太人名字的人.
來到耶路撒冷巴勒斯坦地.
在路程上竟然被人看見他的存在.
所以從這段聖經中的記載.
也反映出一件事.
西門其實要作一個很重要的見證.
或是一個很重要的意義.
第一.
可以看見.
西門後來要去.
勉強地去背耶穌基督的十字架.
他必須勉強地去.

$^{1201}$而不是自願地去.
換句話說.
西門有很大可能性是一個猶太人.
不過他本身長期活在外邦世界的古利內.
他是因為節期的緣故來到耶路撒冷.
在這個節期.
正正是逾越節的節期.
在這個節期來過節.
可以說是一個虔誠人.
其實我們怎知道他是虔誠人呢.
因為後來在聖經其他地方裡.
有記載過古利內人西門.
他是誰呢.
原來是兩個稱為亞歷山大和魯夫的人物的父親.
大家可以參考羅馬書第十六章第十三節.
在這段聖經裡.
甚至保羅向當時羅馬宗教會的同工問安的時候.
聲明要為這兩個人去問安.
如果這個西門.
即是古利內西門.
正正是保羅後來在羅馬書裡記載的西門呢.
我們可以告訴大家.
保羅看到這兩個人物亞歷山大和魯夫.
其實是一個虔誠人.
所以他問安.
而這兩個虔誠的人的父親就是西門.
很共通的.
他是虔誠的猶太人來過節.
但是他是被迫去徵用.
羅馬人不講理由的.
他用權力去滿足自己所希望做的事情.
當然有些聖經學者猜測.
耶穌當時已經被氣壟鞭傷.
可能很慘.
所以沒有力搬十字架.
於是才找人來幫他搬.
這不是馬太夫的說法.
馬太夫在這裡說的.
很清楚地告訴你.
這個人物是被迫去幫耶穌搬十字架.

$^{1241}$一群沒有理由的人.
或者叫士兵.
來作一些沒有理由的要求.
隨便找個人徵用你.
你現在就幫我搬十字架.
一條大木.
第二.
馬太特意記載這個人物.
也提醒了一個同名的人物.
這個人物我們很長時間沒有見過.
也叫西門.
不過叫西門彼得.
在這段聖經裡.
我們看到一個正處於矛盾失敗.
內疚的西門彼得.
和一個在這個時候虔誠來到耶路撒冷.
但他被迫都肯去幫.
搬十字架的這位古里內人.
做了這個對比.
就告訴我們.
一個真正虔誠的人.
是怎樣面對這個世界.
順主不是在這個世界順了主之後.
就會一帆風順.
也不是因為我順了主.
去參加一間大教會.
所以在大教會裡有很多人可以幫助我.
解決我很多問題.
我的人生就會更加亨通更加順利.
但卿子無好相愛.
所以有問題就互相解決.
我就很時時亨通.
是不是這樣?不是.
就因為你正正是虔敬的.
這個世界隨時對你都有一些不合理的要求.
在這些不合理的要求的時候.
你不是去反抗.
不是去和他掙扎.
掙扎.
死過.

$^{1281}$不是.
而是怎樣?.
幫一把能做到的事.
就去做.
只要這件事不是不合真理.
就去做.
領子妹.
馬太寫到這裡其實有一個隱藏的意思.
這位古里內而來的西門.
他其實願意在這樣的情況下.
仍然去成就參與耶穌基督的人生的其中一個片段.
我們基督徒不是很厲害.
不過我們基督徒有一樣東西.
就是無論任何時間.
我們都會盡量竭力.
憑我們自己的力量去參與在耶穌的生命生活當中.
所以從這段聖經裡.
給我們看見.
這個古里內人的西門.
就是一個最貼近耶穌.
陪著耶穌一同上各國他.
足留地被處決的地方.
他成就了耶穌基督.
這個藉著死亡的一個救恩.
他在耶穌在地上.
這一段最後的時間.
陪著耶穌走.
我們作為信徒.
有沒有這種生命.
我經常都想.
如果在我的人生裡.
去到最後的時候.
我最想見的是什麼人呢?.
或者我最後想做的一件事是什麼呢?.
我給自己一個答案.
我其實想在我人生最後的階段裡.
見到所有我認識過的人.
然後和他們認真地去談一談信耶穌的事情.
希望他們能夠在這個聊天裡.
聖靈可以用得著去感動他們明白.

$^{1321}$並且接受耶穌基督的救恩.
我也想接觸每一個我曾經養過的信徒.
或者教徒.
希望他們真的坦然無懼地.
活在上帝的面前.
而不是再次匿藏在教會當中.
以為自己已經得救.
但其實還沒得救.
更重要的是.
我最希望的是.
每一個我所認識的人.
他們不是一個以為自己贏了的輸家.
今天有很多信徒.
不知道什麼原因.
寧願做一個在命行裡傳書的人.
但他們卻要在這個世代裡.
做一個好像全部得勝的人.
這種想法.
我覺得在這段聖經裡.
就是馬太要帶出來的訊息.
耶穌的受死.
其實是反照著你和我.
究竟對於上帝的拯救.
是否真的完全的毀身.
還是我只是一個包裝得漂亮一點的.
不得救的人.
求主幫助我們.
讓我們一起和心底人道過.
讓我在這段聖經裡學習.
我們在這個世界上有很多信徒.
口稱自己相信耶穌基督.
心裡有另外一個判選的人.
我們為他們祈禱.
甚至是為自己.
求你幫助我們.
我們會成為一個表面好像全勝.
但實際在救恩裡是傳書的人.
求主你的幫助.
讓我們得著的救恩是真真實實的.
而不是貧民家母一柳輕煙一樣的騙局.

$^{1361}$求你的幫助.
特別當我們看到一些使我們激動的宿命經驗的時候.
求恩主你的聖靈提醒.
讓我們深思究竟我們敬拜的是自己的宗教狂熱.
還是真正忠誠於主的生命.
求你的大力幫助.
禱告奉靠主耶穌基督得勝的名字而求.
阿們.
\newpage



\section{馬太福音 27:33-44}
\label{sec:oCpi7n8ictU}
\textbf{我們/他們是怎樣的人? (馬太福音27\_33-44) - 黃紹權牧師  [馬太福音信息系列 - 第146講]}
\newline
\newline
連結: \href{https://youtube.com/watch?v=oCpi7n8ictU}{\texttt{ https://youtube.com/watch?v=oCpi7n8ictU}} ~~~~ 語音日期: 2025-02-02 
\newline
\newline
\hyperref[sec:HaaLhKYBRSg]{< < < PREV SERMON < < <}
~
\hyperlink{toc}{[返主目錄]}
~
\hyperref[ch:preacher10]{[返講員目錄]}
~
\hyperref[sec:7upP8JmD6zY]{> > > NEXT SERMON > > >}
\newline
\newline
馬太福音 27:33-44
\newline
\begin{longtable}{cl}
\hline
\hline
章節 & 經文 (和合本修訂版)\\
\hline
27:33 & \begin{tabularx}{0.7\textwidth}{X} 他們到了一個地方,名叫各各他,就是「髑髏地」。 \end{tabularx} \\ \\ \relax
27:34 & \begin{tabularx}{0.7\textwidth}{X} 士兵拿苦膽調和的酒給耶穌喝。他嘗了,不肯喝。 \end{tabularx} \\ \\ \relax
27:35 & \begin{tabularx}{0.7\textwidth}{X} 他們把他釘在十字架上,然後抽籤分了他的衣服, \end{tabularx} \\ \\ \relax
27:36 & \begin{tabularx}{0.7\textwidth}{X} 又坐在那裡看守他。 \end{tabularx} \\ \\ \relax
27:37 & \begin{tabularx}{0.7\textwidth}{X} 他們在他頭上方安了一個罪狀牌,寫著:「這是猶太人的王耶穌。」 \end{tabularx} \\ \\ \relax
27:38 & \begin{tabularx}{0.7\textwidth}{X} 當時,有兩個強盜和他同釘十字架,一個在右邊,一個在左邊。 \end{tabularx} \\ \\ \relax
27:39 & \begin{tabularx}{0.7\textwidth}{X} 從那裡經過的人譏笑他,搖著頭, \end{tabularx} \\ \\ \relax
27:40 & \begin{tabularx}{0.7\textwidth}{X} 說:「你這拆毀殿、三日又建造起來的,救救你自己吧!如果你是神的兒子,就從十字架上下來呀!」 \end{tabularx} \\ \\ \relax
27:41 & \begin{tabularx}{0.7\textwidth}{X} 眾祭司長、文士和長老也同樣嘲笑他,說: \end{tabularx} \\ \\ \relax
27:42 & \begin{tabularx}{0.7\textwidth}{X} 「他救了別人,不能救自己。他是以色列的王,現在從十字架上下來,我們就信他。 \end{tabularx} \\ \\ \relax
27:43 & \begin{tabularx}{0.7\textwidth}{X} 他倚靠神,神若願意,現在就來救他,因為他曾說『我是神的兒子』。」 \end{tabularx} \\ \\ \relax
27:44 & \begin{tabularx}{0.7\textwidth}{X} 和他同釘的強盜也這樣譏諷他。 \end{tabularx} \\ \\
[1ex]
\hline
\hline
\end{longtable}
$^{1}$好 各位大明星 各位早晨.
讓我們先聽道義.
我們一起同心低頭.
將我們的心思念教在我們的主.
我們的神人面前.
直接禱告仰望在他當中.
父神多謝你讓我們可以回到你的啟示當中.
都一一被他的門徒你所使用的僕人馬太將他寄下來.
求主你的帶領.
讓我們每一個能夠得知上主你的作為.
亦都同當中更加去明白上主你的心思意念.
讓我們亦都更加活得像你.
求你的帶領.
讓每一個無論在上或下的弟兄姊妹.
都祈求主你將渴望追求跟隨主的心志.
常傳在我們每一天的身上.
多謝你.
我們同心的禱告.
奉靠主耶穌基督得聖名至求.
阿們.
在開始之前我想問大家一條問題.
你有沒有試過被人冤枉過呢?.
廣東人說你有沒有被人冤賴過呢?.
如果你曾經試過被人冤枉的時候.
你回想一下你的回應態度是怎樣呢?.
我曾經試過被人冤枉過.
於是乎很不服氣.
跟人進行了一場辯論.
其實都不是什麼辯論.
只不過是將我的看法或認識.
重新用不同的角度去展示一次.
但對方仍然很不服氣.
於是發出一個擂台挑戰的宣告.
當時我都不知道.
直至在教會裡面.
有位教會長輩來跟我說.
Alan 你知不知道有人設擂台跟你進行辯論.
你知不知道?.
我不知道 還沒有收到邀請函.
但我都不會再去討論.

$^{41}$因為有時我發覺.
你跟一些人去討論的時候.
如果大家有理智有理性地去討論.
是沒有問題的.
但如果當一個人只是將自己的前設.
硬套在你身上.
甚至將他的觀點套在你身上.
你跟他再討論是沒有意義的.
如果你在香港來的.
應該會聽過一句說話.
如果看到這些人你繼續跟他討論的話.
真的教聰明了他.
不要將錢放在他的口袋.
留在自己的內裡.
這不是自私.
而是一種智慧.
如何去面對一些人以為自己有理由的.
但其實往往所提出的只是個人的偏見.
為何我會說到這個題目呢.
因為當我們來到馬太福音.
今天的記載的時候.
可以說已經步入了決定性的階段.
耶穌就被審判了.
又被誣告.
給了一些不是祂的罪名的罪.
最後都要被釘上十字架.
在這個時候.
我們看看耶穌的回應是怎樣.
在上次我們已經說到.
有時候你不用說話.
都已經可以作成上帝的工作.
你記不記得.
比拉多問耶穌.
為何那些人不斷去.
尤其是祭司長和長老們.
去無顧你的一些罪狀.
控訴.
為何你一言不發呢.
但我問你是否猶太人的王.
你就說你講出這個事實.

$^{81}$為何你要回答我呢.
造成了提醒了比拉多的機會.
比拉多有沒有改變了他的生命.
這是另一回事.
但至少我們可以從耶穌基督的反應裡面.
看到原來默然不語的主耶穌基督.
是有他本身的意思.
所以今天我們一起去看看.
這個問題.
甚麼原因耶穌不再回答這些人.
甚至耶穌不想再發一言一語呢.
在今天的廣道第27章33至第44節.
我們一起看看這個題目.
叫做我們或他們是怎樣的人.
我們一起去打開這段聖經.
我們看看這段經文裡面說些甚麼.
如果大家有聖經的話.
也請大家一起打開馬太福音.
第27章33至34節.
我們一起讀讀這段經文.
讀多一次吧.
我們熟悉一下這段經文.
我們一起讀吧.
到了一個地方名叫葛國他.
意思就是足留地.
冰冰拿著苦膽調和的酒.
給耶穌喝.
他嘗了就不肯喝.
他們既將他釘在十字架上.
就拎軍分他的衣服.
又坐在那裡看守他.
在他頭上安一個牌子.
寫著他的罪狀.
說這是猶太人的王耶穌.
當時有兩個強盜和他同釘十字架.
一個在右邊.
一個在左邊.
從那裡經過的人.
譏笑他搖著頭說.
你這冊位聖殿.

$^{121}$三日又建造起來的.
可以救自己吧.
你如果是神的兒子.
就從十字架上下來吧.
祭司長和民事.
並長老也是這樣戲弄他.
說他救了別人.
不能救自己.
他是以色列的王.
現在可以從十字架上下來.
他們就信他.
他依靠神.
神若喜悅他.
現在可以救他.
因為他曾說.
我是神的兒子.
那和他同釘的強盜.
也是這樣地譏笑他.
在上文說什麼呢.
上文說比拉多一群士兵.
和比拉多都是這麼壞的性情.
大家又肅骨又濫權.
又要在眾人面前.
包裝自己多漂亮.
但又要用盡自己的權力.
來怎樣奚落.
或者玩弄耶穌.
這些人正正反映出.
一些沒有信任的人的表現.
也是今天很多不信耶穌的人.
或者不相信上帝的人的心思.
因為他們只是流離朗蕩.
在自己喜悅的活動裡.
其實到時碰到什麼.
他們就做些事.
用一種衝動的回應.
好像一條船沒有撈.
浮到哪裡飄到哪裡.
隨遇而安.
所謂的安只是指.

$^{161}$將自己的快樂建造在其他人身上.
今天我們會看到.
這群人開始進行行刑的時間.
他和耶穌基督一起.
上到一個地方叫葛國他.
聖經知道.
但怕大家不知道.
這個地方有一個俗稱的名稱.
叫做築留地.
在上路的過程當中.
有一個古利內人叫西門.
在路遠的地方.
在今天的的黎波利的地方.
特意來到耶路撒冷.
可能過節.
在圍觀的過程當中.
被徵用.
他有幸能夠和耶穌一起.
走上耶穌最後的一程的路.
在這個過程當中.
我們可以看到.
有信的人和沒有信的人.
同時出現.
但最可怕或最擔心的一件事就是.
會不會在人生最後的時間.
我們可能突然間.
最後.
我們爛尾了.
不再跟從主.
成為一個好像贏了.
但他是真正的贖名書家呢.
這是我們上文已經說過的題目.
來到今天我們看這段經文開始.
我們會仔細看.
這個行刑的過程.
這個行刑的過程當中.
將很多的人性.
將他完全表達出來.
特別是那些要謀害耶穌基督的人.
或者群體.

$^{201}$他們集體的一些表現.
以至於從他們的表現當中去反映.
究竟今天我們會不會同樣跌在.
他們這樣的局面.
或者這樣的靈性的狀態呢.
所以這段聖經可以說是將.
人性或者特別我們說的比較技術性的罪人.
他的特性.
一一將他展露在你和我的面前.
而馬太寫這段經文.
就是要將這些人的罪性.
將他充分地.
並且透徹地去描寫.
鋪排在我們面前.
讓我們看到他們的罪有多深.
而他們所陷在罪當中的那種捆綁.
有多緊迫在他們裡面.
我們分幾個部分去看.
如果大家在程序表上.
我們將港獨的大綱寫出來.
第一個部分是從第33至34節.
我們一起看看耶穌如何以善意去接待一群.
蓄意惡待他的人.
我們一起去看這兩節經文.
我再讀一次這兩節經文很短的.
他說到了一個地方名叫葛國他.
意思就是足留地.
冰冰拿苦膽調和的酒.
給耶穌喝.
他嘗了就不肯喝.
在第33節所說到的.
這群冰冰和古利內人.
一行人我相信還包括了耶穌基督的媽媽.
一群跟隨著耶穌多年的婦女.
和一大群的群眾一起走到這個刑場裡面.
這個刑場有個名叫足留地.
葛國他.
這個字眼是什麼意思呢.
這個字眼本身是源自於一個亞蘭文.
在這裡我首先介紹一下什麼是亞蘭文.

$^{241}$因為在遲些講但以理書的時候.
我會再詳細講.
亞蘭文可以說是古晉東的一個通用語言.
就好像今天這個世界裡面所看到的英語.
這個世界你可以不懂中文.
不懂日文.
不懂法文.
但你不可以不懂英文.
為什麼.
原來很多東西基本上只有英文能夠說得通.
其實我很多年前才知道原來是一件這樣的事情.
當時我去做短宣去到法國的時候.
有一群弟兄姊妹就說去法國要帶一些電腦去法國的一個教會.
我就覺得很奇怪.
為什麼要帶電腦過去呢.
那些電腦不是現在的notebook.
是一座好像你家裡的抽屜那些.
家裡現在還有四五台那些.
就帶過去.
我就覺得奇怪.
我說你們帶電腦過去做什麼呢.
電腦全部是英文的.
法國當然是用法文的.
誰不知原來他們這樣跟我說.
你不知道的.
原來讀電腦的人.
所有人用英文去讀的.
是嗎.
我不知道.
他說無論從電腦的語言.
以至於那個英文.
以至於電腦的運作系統.
全部都是用英語的.
我說不是啊.
我看過有些畫面都是法文的.
他說那個是英語之後再翻譯出來的畫面給你看的.
背後所有的東西都是用英語.
亞蘭文其實就好像我們現在所看的英語一樣.
你不懂亞蘭文.
基本上你就不知道很多的事情.

$^{281}$這個足留地這個字眼本來就源自於亞蘭文的名稱.
但是後來就把它轉化為希臘文.
而後來把聖經翻譯的時候.
就把足留地這個字眼翻譯成拉丁文.
而拉丁文的翻譯就叫做Calvaryia.
就是後來我們所熟悉的Calvary這個陳述.
這個地點在哪裡呢.
為什麼要把耶穌押到這個地方裡面呢.
就有很多聖經的研究.
特別是這個所謂聖地研究.
是一個專門的考古學.
在聖經學也有的聖地研究.
就提到了好幾個地方.
如果大家今天去到耶路撒冷旅遊的時候.
你一定會進去耶路撒冷舊城.
大家知道耶路撒冷其實不是很大.
不過不大也好.
它分為舊城區和新城區.
舊城區就是我們經常讀聖經.
要講的圍牆圍著的聖殿山為中心的地點.
這個地方有個教會.
那個教會就叫做The Church of the Holy Sepulcher.
有些人就說是主誕堂.
很多翻譯我不知道.
我曾經去過.
這個教堂很特別.
因為它本身的歷史很久了.
長達到一個教堂的地步.
所以如果大家想看YouTube或紀錄片.
耶路撒冷原來猶太人的新年會走進這個教堂.
但這個教堂裡面又有一間教堂.
相傳這個地方是耶穌被釘死的地方.
當時是一個小小的山丘.
後來被移平建了一個教堂.
原本在山丘的教堂.
後來在外面又再建了一個教堂.
把教堂包成教堂.
就是這個情況.
另一個地方建議的是.
耶路撒冷城的舊城.

$^{321}$正正北面的出口對著的地方.
叫做花園塚.
為甚麼叫做花園塚呢.
因為當然耶穌基督埋葬在那個地方附近.
附近有一個山坡.
如果大家有看聖地課程.
我曾經出過一幅圖片給大家看.
那個山坡其實是一個像峭壁的地方.
我去過的.
山上有兩個凹陷的地方.
有一個凸出來的石頭.
形成了一個像骷髏頭的地區.
有些人說為甚麼我們更肯定.
這個地方是耶穌真正被釘死的地方呢.
因為釘死的地方正正像骷髏頭一樣.
所以就叫做足留地.
在足留地過沒多遠.
就有一個新的墓.
那座墓就是耶穌埋葬的地方.
如果你去的話.
我勸你去一下就算了.
不要相信了.
因為在當地我去的時候.
公元2000年.
25年前.
當時有一個浸信會的美國牧師.
在那裡做負責導遊.
因為在聖地所有要說話.
要說話去介紹聖地都要拿牌子的.
他就拿了牌子在那裡.
長期做介紹.
他很有趣的.
介紹完所有東西之後.
他最後就這樣說.
你們不要相信了.
我說你說甚麼.
他說耶穌根本不是葬在那裡.
這個只是做出來給遊客看的.
給你的信徒朝聖的.
其實在聖地裡面有很多這樣的東西.

$^{361}$但是釘死耶穌的地方.
究竟是不是那個地方呢.
我們都不知道.
另外一個建議就是.
在耶路撒冷這個所謂.
The Church of the Holy Sepulchre.
西南邊的一個175米的地方.
同樣有一個地方.
亦都是有個小山丘.
估計亦都有可能是耶穌被釘死的地方.
講到這裡.
牧師你講了這麼多.
究竟釘死在哪裡呢.
答案就是不知道釘死在哪裡.
但重點是想說甚麼呢.
不是你想知道釘死在哪裡.
然後在那裡摸摸石頭.
嗨嗨那個十字架.
那裡有十字架的景點.
那你會神聖一點嗎.
不是.
我想告訴大家.
這個地方在當時耶路撒冷城裡面.
是一個皆知巷聞的地方.
就好像多倫多有誰不知道.
Eaton Centre在哪裡.
有沒有人不知道Eaton Centre.
可能真的有人不知道.
有沒有人不知道我們這裡.
沒有人不知道嗎.
有沒有人不知道CN Tower在哪裡呢.
我想除了你是瞎的.
看不到這件事.
但你沒理由不知道.
看不到你都聽過.
在市中心近湖邊.
那裡有個打球的地方.
那個球場的頂會開.
以前叫Sky Dome.
現在叫Roger Centre東西.

$^{401}$在那裡隔壁.
你都會聽過.
個個都知道.
其實當時築樓地是一個人所皆知的地方.
而來到這裡.
為何馬太要刻意提出這個地方呢.
即是說很多人都會去經過這個地方.
就等同於你如果去香港.
你如果未去過中環.
未去過旺角.
那你怎會去過呢.
來到多倫多.
你告訴別人你未去過.
Young guy真是笑死人.
人人皆去過的地方.
個個人都去.
就是一個個個人都會去的地方.
來將耶穌處決.
這個顯然易見.
是一個公開的宣示.
這個宣示是給比拉多和他的同黨.
包括耶路撒冷公會的所有領袖.
包括大祭司 文士 長老 法利塞人.
你想得出的人.
全部在這裡.
殺都該人.
全部都在這裡.
他們希望在這個地方裡面.
彰顯他們得勝的情況.
當我們去看見這班人的動機是這樣的時候.
我們就看見34節.
34節 於是乎這班兵兵就拿著苦膽調和的酒給耶穌喝.
他喝了不嘗了就不肯喝.
耶穌只是試試味就沒有喝了.
這是什麼來的呢.
為什麼要在這一刻調和苦膽調和酒給耶穌喝呢.
苦膽是什麼呢.
有很多研究的.
當然啦.
如果你看其他的福音書裡面.

$^{441}$也有其他的陳述.
說這是一種活躍的東西.
但是我再重複講一次.
不要將遞卷福音書的東西搬進來這裡講.
活躍的東西是貴的.
聽說好像有紙吸法壇功效.
你去藥材舖買一下很貴的.
不關事.
馬太是不記載這些是什麼.
但是告訴我們是苦膽.
簡單來說.
苦膽就是一些很苦的東西.
有些聖經學者研究說.
這可能是一些動物的膽汁的東西.
是什麼呢?.
沒有人知道.
或者是蛇膽穿背.
最重要的是.
這東西是不會正常地拿來喝的.
他就將它和酒混合.
有些所謂的學者.
包括信徒.
就嘗試在這裡作個解釋.
喝了這東西就能夠止痛.
這些純粹猜測的東西.
全部丟掉.
因為教會有很多這些猜測的東西.
因為我們有很多這些.
聽聞的傳聞.
又不去印證.
為什麼要止痛呢?.
有些人說洗腦耶穌死得久一點.
殘忍一點.
這些是你作的.
有沒有這樣說呢?.
沒有.
就麻煩將它丟掉.
但是在這段聖經裡面.
很肯定的一件事.
就是這些苦膽.

$^{481}$混合酒來給耶穌喝.
是不好喝的.
我怎麼知道不好喝呢?.
耶穌試過了沒有呢?.
雖然他不是去米芝蓮.
一試就知道是什麼東西.
一喝了.
耶穌不肯喝.
在這裡.
在《釋經》裡面有一個難處.
如果你讀聖經.
如果就這樣讀和合本.
你就會覺得.
有什麼難處呢?.
耶穌給了一群兵去傷害他.
但是如果你仔細去看英文聖經的時候.
第34節所說到的是什麼呢?.
是一群人帶著耶穌上去.
然後這群人帶到一個足留地.
然後給了一群人苦膽和酒混合的東西.
給耶穌喝.
喝了之後.
耶穌就不再喝了.
和合本就說是冰的.
但是在原文聖經裡面說.
是一起在耶路撒冷帶耶穌去各國他地方的群眾.
問題就回來了.
究竟是誰給耶穌喝這些苦東西呢?.
是如和合本所說的冰的.
還是群眾當中其他的人呢?.
馬太福音在這裡特意將這個無伶兩可的引用或描述.
擺在我們面前.
就要我們嘗試用兩個角度去想.
如果這些用苦膽和酒混合的東西.
是其他的群眾不時冰冰給耶穌喝的話.
或許可能是耶穌基督的跟隨者.
尤其是一群婦女的話.
我們就會看見.
這群婦女給耶穌喝這些東西有什麼作用呢?.
打開了一個問題.

$^{521}$是不是真的有藥效呢?.
是不是真的可以幫到耶穌呢?.
還是越幫越忙呢?.
但如果不是的話.
如和合本所說的冰冰連接上文之前的經文.
不只是33節.
33節之前的經文.
所謂的day是指壓榨耶穌基督的士兵.
包括古利內人西門的話.
會不會可能是西門甚至是士兵所做的呢?.
當我們去從這個考量的時候.
我們會從不同的角度去想.
如果是西門的話.
西門會不會做一些這樣的事情呢?.
他是來過節的.
沒有預備這些東西來.
難道這些東西真的隨身帶嗎?.
好像我們去旅行.
每次去旅行什麼都不敢帶.
只帶一件東西去正路苑.
怕肚子餓.
因為去旅行吃很多好東西.
吃到肚子餓.
我相信不像是他.
如果不是他.
剩下一個去考慮的就是冰釘.
冰釘為什麼要這樣做呢?.
要拌一杯這樣的東西.
在現場給耶穌去嘗試呢?.
我相信這些冰釘.
如果是他們做的話.
他們一定有一個動機.
就是在先前.
把耶穌裝作好像君王的情景.
蓋過的事情.
現在逐步在一個公眾的場合裡.
將它暴露出來.
記不記得前文他們是將耶穌.
困在比拉多的軍營裡.
在軍營裡將耶穌.

$^{561}$像君王一樣的奚落.
脫光衣服.
氣動完之後.
再穿上耶穌的衣服.
好像沒有人知道.
這些冰釘全部都是好人.
是不是這樣.
來到這裡.
這些冰釘的惡行.
就彰顯在眾人面前.
惡人的表現.
無論你怎樣爆都好.
終有一日會暴露.
這班人為何要在這裡暴露呢.
因為他們要盡情在眾人的面前.
去羞辱耶穌.
希望耶穌在這個時間.
所受的痛苦是最大的.
希望使很多人看見這件事.
滿足他們要殺耶穌的動機.
所以簡單地來說.
冰釘你猜他有沒有自己的盤算.
有.
因為他既然要在這件事上.
將來還要繼續面對耶路撒冷的居民.
怎樣去管治他們.
最好的方法就是討好他們.
比拉多爺一樣.
為何要放一個囚犯在這個大時大節.
討好他們.
上的要永遠做得好看.
下的有時就要扛一些難受的東西.
所以有個總統經常說.
I am a good friend with so many people.
我和很多人都很老友.
不過底下那班人就要做事.
是不是真的對他們那麼好.
我們看看這班人.
其實將一個苦膽撈走.
來給耶穌喝.

$^{601}$其實是為了討好這班人.
當這班人越見到耶穌基督越受苦.
他們就越開心.
耶穌在這個時候有什麼回應.
一句回應都沒有.
不單止先前他們不斷誣告耶穌的假見證的控罪.
耶穌不回答.
錯的事情你不需要回答.
因為錯的事情你回答都浪費力氣.
但來到這裡他給你喝這樣的東西.
你會怎樣.
我不知道.
如果你是在這樣的狀態下.
我想你不會那麼簡單.
我們曾經說過.
彼得自己不認主.
他都會說髒話.
記不記得.
他發奏了.
很多中國人都會這樣.
做錯了事就說髒話.
他說髒話不是想罵人.
他只是對自己不滿.
耶穌這樣給他喝苦膽加酒.
聽清楚了.
苦膽不是老虎的膽.
在這樣的情況下.
耶穌一句都沒有回應.
沒有說任何咒詛他們的說話.
沒有說為什麼你們給我喝這樣的東西.
你們想什麼意思.
但充分反映出這群人的惡意.
見風洗盡利.
人能用各式各樣的方法去刺激耶穌.
越痛苦的人就越高興的時候.
耶穌不要上他們當.
反而耶穌仍然走在一種不出聲的情況.
這是什麼態度.
在聖經裡有記載.
在馬太福音第九章第十三節裡記載.

$^{641}$經上說我喜愛憐憫不喜愛濟滋.
這句話的意思是你們去揣摩吧.
我來本不是召義人乃是召罪人.
馬太福音第九章十三節所說的.
耶穌不是不知道他面對一群罪人.
只不過他不用罪人的態度去對待這群人.
用善待的態度去召他們.
希望幫助這群罪人走離罪.
脫離罪的困所.
所以耶穌既然見到你們這群人來者不善的時候.
不得寸進尺.
隨便你們不得寸進尺.
不過當你們的邪惡越暴露的時候.
其實你也更加清楚看到.
這群人自己的罪是什麼.
換句話來說.
你不需要看到自己犯錯才懂得改過.
你看見一群人這樣虐待耶穌.
你的心會不會說不用這樣了.
那些黑道是你的囊中之物 是你的獵物.
釘在十字架上 仍然要戲弄他做什麼呢.
會不會激動了一些人回轉過來.
亞太福音在這裡給了每一個罪人一副透視鏡.
包括你和我.
去看清楚我們的罪有多深.
也同時間看我們上主是怎樣為我們受苦的時候.
我們可以動到慈心在上帝的名前.
敬罪悔改.
來到聖經的時候可以看到耶穌面對著這些人.
也和昔日先知的遭遇很類似.
如果大家看耶利米書第八章十四節.
九章十五節.
二十三章第十五節.
我們也會留意到一個作為神的僕人.
他們的遭遇.
他們的遭遇是什麼呢.
都不好的.
尤其是做先知.
中文受苦的時候.
先知要首當其衝去受苦.

$^{681}$有時先知要受的苦甚至不是自己的民.
是其他的民.
記不記得有位先知叫若拿先知.
他跌在被拋進海漁福的海裡三日三夜.
在這樣的情況下.
他受的苦是什麼呢.
一方面是他自己不信服上帝的呼召.
但同時也要讓將要傳道的人看到.
一個不聽話的所謂屬靈人的下場.
上帝如何仍然保守.
從這段聖經裡我們看到.
耶利米先知的遭遇和耶穌基督的遭遇很相似.
因為為了激勵很多仍然在罪裡面的人去反思.
但這群人有沒有反思呢.
35至37節我們看到.
這群無聊的罪人仍然做出無聊的事情.
去回應耶穌基督的善待.
我們一起去看看.
35至37節我們一起讀.
他們既將他釘在十字架上.
就拈君分他的衣服.
又坐在那裡看守他.
在他頭上以上安一個牌子.
寫著他的罪狀.
說:這是猶太人的王耶穌.
如果大家再細心去看這段經文的時候.
我們會發現.
從35至44節的經文.
是很平衡地去記載.
好像在第28至31節的經文.
大家可以看一看.
經文的內容是說什麼的.
第28至31節27章.
第28至31節.
在那段經文裡記載了.
這群士兵盡情地虐待.
欺弄耶穌.
但又要將自己包裝.
好像沒有事發生過一樣.
來到這段經文.

$^{721}$35至37節.
同樣這群人都是一樣的表現.
當這群人帶耶穌去刑場之後.
做了什麼呢.
將他釘上十字架.
然後就「捏弓分他的衣服」.
什麼叫捏弓分他的衣服呢.
好像文化上有些都奇怪.
中文原來「捏弓」和「分」是兩個動詞.
「捏弓」就是指導他們去做一種遊戲.
這遊戲有研究過.
當時說是一種紮飾的遊戲.
現在很多人上網玩遊戲時.
按一個雙數或單數.
雙數代表什麼.
單數代表什麼.
一號代表什麼.
二號代表什麼.
「捏弓」原來是一種遊戲.
一種紮飾的遊戲.
當是紮飾吧.
怎麼玩呢.
很多考究.
我以色列的時候去過現場.
看過有塊石頭畫好一個圈.
分開格.
然後就釘一塊石頭進去.
不是釘一顆色.
一顆色有多少面呢.
六面吧.
一至六.
太久沒讀過probability.
忘記了.
不是這樣.
是在地上畫了一個圈.
或者是一格格.
分了數字.
然後釘一塊石頭.
看釘中哪一格.
然後就做什麼.

$^{761}$有這樣的遊戲.
分開衣服.
就把衣服扯爛.
或者脫掉.
我們可以看見衣服的作用又再次出現.
所以大家看節目有沒有留意.
為什麼主要學要講一課.
要講聖經裡面的衣服的意義.
就是這個原因.
其實是我神學院其中一個study.
那時候我神學院的教授也說.
我教授的教授是一個很好的人.
我神學院的教授也說這個study很厲害.
後來我找了很多人做.
不只是我做.
衣服是代表什麼呢.
衣服是代表一個人的身份.
先前在第28至31節裡面.
我們看見有穿小紅袍.
用荊棘做皇冠.
用蘆葦草做他的手杖.
其實是反映出一個社會的角色.
你說你是皇嗎?.
是皇啊.
弄得你好像皇一樣.
但其實你是一個落難而沒有實權的皇.
這個就是兵兵在秦府的堡壘之下的行為.
來到這段聖經.
這班人要繼續他們的諷刺耶穌基的行為.
於是怎麼樣.
扔石頭或是白色的.
不管是什麼方式.
總之每次選一件.
耶穌的衣服就把它脫下來.
分給大家用.
其實耶穌還有多少件衣服呢.
沒有多少件.
如果要脫.
扔兩下就扔完了.
外袍內袍.

$^{801}$最多出一對拖鞋.
但有沒有穿鞋我們不知道.
如果是將耶穌所剩的一兩件衣服.
逐件扯爛.
現在扔一號號碼就扯爛左邊的手袖.
二號號碼就扯爛腰帶.
第三就扯爛外袍後面.
如果是這樣.
這班人就用撕裂人的衣服.
一個社會角色的身份.
來滿足圍觀者的喜悅.
所以這班人的行為.
不單要將耶穌扔在十字架上.
並且要將他的衣服.
逐件分裝或撕裂.
當然扔在十字架上就沒有穿衣服.
衣服在下面是沒有需要的.
就是這個原因.
他所做的行為根本是沒有需要的行為.
簡單來說是一個多餘的行為.
但為什麼一班乒乒在公眾要做這麼多餘的事呢.
剛才我說過很多次.
收買人心.
既可餘己又可以餘人.
表面上眾人喜歡做的事.
我為你們做了出來.
你們就開心了.
你們要扔十字架嘛.
我現在就扔上十字架.
不夠呀 做多點.
將根本不需要的衣服.
徹底的去將它捏光.
和分開.
在這種做法裡.
無非是要討好所有人.
這種動作.
我們可以看到.
到最後.
這班人.
做什麼呀.

$^{841}$做完這些事之後.
又坐在那裡看守塔.
你們的工作就是將釘子坐在那裡.
看守就算了.
為什麼要做這些事呢.
還有坐在那裡看守塔.
就表現出這班士兵.
好像又不關我們事.
stationary 英文叫.
他們在那裡.
station.
停在那裡.
好像沒事發生.
不關我事的.
不過你們喜歡我就做了給他們.
但在耶穌的頭上.
又動了一些功夫.
記不記得在先前第28至31字裡.
在耶穌的頭上動了什麼功夫.
他做了一個.
荊棘的棺面.
不是做的.
是撿回來的.
在山邊滾了一個出來.
隨便的.
這班人怎樣.
既然你說自己是王.
好.
我就將你的罪狀.
整個牌子寫在上面.
這個牌子正正寫著.
這是猶太人的王.
耶穌.
各位聽眾.
來到這裡的時候.
我們看到這班罪人.
其實一直都隱藏著.
他們心裡面.
那份無聊的事情.
做一些害人又不利己的事.

$^{881}$以這種情況.
上帝有沒有改變他們.
藉著耶穌回答他們的虐待.
似乎沒有.
他們只會有更變本加厲的情況.
並且他們很沉溺在自己對罪人.
那種義人.
無辜的人.
那種施害.
那種快感.
在這種情況之下.
這班人有沒有真的回轉呢.
來到這段聖經.
我們可以看到.
這班人.
他們是值得他們不得救的.
信仰的結果.
他們值得不相信上帝的下場.
不是上帝不想在這個過程當中.
讓他們看到一個人賜的主.
如何去回應一個惡待他們的人.
這班人完全沒有改變.
等同於.
你對一個人好.
你對他關心.
當他做了一些事.
有沒有試過.
他回應你一句.
沒有叫你做.
你做什麼.
做那麼多事.
有沒有試過.
你怎麼想.
由他死吧.
耶穌其實仍然不想這班人死.
他們要死.
還要變本加厲.
沒有辦法.
頂智慕來到最後的部分.
這部分其實應該分一堂課講的.

$^{921}$不過我用很快的方法去講給大家聽.
這段聖經從第38至44節.
就是.
講給我們知道.
基紹的粉墨登場.
基紹是什麼意思.
Blessedly.
Blessedly就是誣蔑.
或者恥笑.
鄙視人.
Look down.
英文很容易說.
我看不起你.
這段聖經38至44節.
全部的內容都以基紹.
包含在裡面.
我們一起看看.
我讀你們可以聽聽.
從裡面你能畫出多少個基紹的動作.
或者行為甚至動詞.
38節.
當時有兩個強盜和他同釘十字架.
一個在右邊一個在左邊.
從那裡經過的人.
基紹他.
搖著頭說.
你這拆毀聖殿.
三日又建造起來的.
可以救自己吧.
你如果是神的兒子.
就從十字架上下來吧.
41節.
祭司長和民事並長老也.
是這樣戲弄他.
說.
他救了別人不能救自己.
他是以色列的王.
現在可以從十字架上下來.
我們就信他.
他依靠神.

$^{961}$神若喜悅他.
現在可以救他.
因為他曾說.
我是神的兒子.
44節.
那和他同釘的強盜也是這樣地.
基紹他.
畫到嗎.
畫得很尖.
那些字.
直到用到基紹這個字眼.
在和合本裡.
分別有第38節.
也有第44節.
如果我們再仔細看.
第41節裡.
祭司長和民事並長老也.
是這樣戲弄他.
英文是怎樣說.
In the same way even the chief priests.
together with the experts.
in the law and elders.
were mocking him.
基本上在這段聖經裡.
這種.
基紹的行動.
重複重複重複地出現.
不過就以這個祭司長和民事.
或者大祭司長老們.
推到最極點.
mocking.
不單止.
串兩句.
在聖經裡.
我順帶說一說.
褻瀆.
和blasphemy.
是同義詞.
所以如果你去基紹上帝.
你等同於褻瀆.

$^{1001}$這裡不詳細說了.
在聖經裡我們看到.
馬太沒有記載過.
任何一句耶穌在十字架上.
說任何話.
有沒有留意.
那些牧者很喜歡說.
十加七言.
耶穌在十字架上說了什麼.
這裡完全一句都沒有.
馬太刻意地記載.
不是她不知道耶穌在十字架上說話.
而是那不是主題.
在十字架上不說話的耶穌.
其實讓我們看到.
耶穌真是作為一個羔羊的角色.
來對承擔別人的罪.
祂沒有掙扎過.
在聖經的對話過程當中.
集中反而要希望我們看到.
人有多醜陋.
這群人水平很低.
耶穌和丁在十字架上有兩個強度.
一個在右邊一個在左邊.
在這裡要說.
左邊的怎樣 右邊的怎樣.
這裡不是聖經的主旨.
聖經想說.
這些強度仍然在上面.
都是譏笑耶穌.
在這裡我想說一句.
好像清文那篇卷福音書.
說十字架上有兩個強度.
和丁其中一個後來認罪悔改的.
沒有譏笑.
回去看清楚那篇福音書的記載.
他是譏笑完之後.
有一個聽完之後才認自己的罪.
聽清楚.
十字架上的傳福音.

$^{1041}$是一個仍然是.
原本是譏笑耶穌的罪犯.
聽完別人的使用之後.
發出持心悔改的心靈.
因為其中一個囚犯還說.
該死了耶穌.
那人聽到這樣說的時候.
立刻說我們的罪.
其實我應該承擔.
他?我真的看不到.
立時悔改.
那個是大卷福音書我們不說.
馬太福音書很強調的.
沒有任何一個人不譏笑耶穌.
罪性裡面就是死剩一張嘴.
我一會再詳細說.
當這兩個人被與耶穌同釘的時候.
第一批經過的人.
就在那裡走過笑耶穌.
剛才記不記得我說過.
這是街知巷聞的地方.
人人都經過.
每個人走過都在那裡搭啤酒.
知不知道什麼是搭啤酒.
等同街上的人看見人家下棋.
說你為什麼不上馬.
搭啤酒不需要的.
不關你的事.
你去搞一輪.
說什麼呢.
還要有表現.
搖頭.
好像自己是判官.
你這個這麼厲害.
說拆毀聖殿三天又建造起來.
可以救自己吧.
如果是神的兒子就從十字架上下來吧.
這番話和諧本寫得很平鋪直敘.
這番話其實是一段挑釁性的說話.
英文叫Taunting Words.

$^{1081}$這麼厲害.
拆毀聖殿三天建造.
其實他沒有說過.
聽傳聞.
用fact check.
所以不要DEI了.
取消了.
不要fact check了.
是不是這樣.
不要fact check了.
耶穌沒有這樣說過.
祂說你自己救自己吧.
很囂張.
還要囂張到什麼地步呢.
這麼厲害.
你說自己是神的兒子.
如果耶穌真的從十字架上下來.
他們會不會相信他是神的兒子.
有沒有這樣想過.
有沒有這樣問過.
這班人根本從頭到尾都不相信耶穌是神的兒子.
更加他不相信耶穌是有拯救的能力.
因為祂說你可以救自己.
你下不來就是救不了自己.
這班人是什麼人.
我給你一個形容詞.
這班口賤不人憎的人.
不關你的事.
路人甲乙丙你在說話.
第二批.
關他們事.
祭司長和民事.
並掌路也.
使者用戲弄他說.
我們看看怎樣戲弄.
他救了別人不能救自己.
多囂張.
他救了別人不能救自己.
多囂張.
他救了別人不救自己.

$^{1121}$好像常說.
醫生能醫不治醫.
這樣的說法.
接著說.
他是以色列人的王.
現在可以從十字架上下來.
我就相信了.
這句話很熟悉.
剛剛那班路人甲乙丙說的話.
原來祭司長和民事.
掌路的水平.
和路人甲乙丙說的一樣.
原來祭司長和民事.
掌路的水平.
和路人甲乙丙說的一樣.
第三.
四十三節.
他依靠神.
神若喜悅他.
現在可以救他.
因為他曾說.
他是神的兒子.
哇.
他在這裡更加說了一件事.
這句話你明白嗎.
充滿煽動性.
你們現在看這個人.
被釘死的死不足惜.
因為他是神喜悅的人.
如果是喜悅的.
上帝就已經救了他.
明不明白這句話.
假設陳述.
所以為什麼要讀希臘文.
假設陳述.
If有四種不同的情況.
就是這樣.
每一句說出來.
都是一種否定.
假設就越不相信.

$^{1161}$丁小妹.
這群人全部都關他事.
因為整件事.
是耶穌由他們設計的.
但這群人.
到了這一刻有沒有悔改.
完全一點悔改的心智都沒有.
也沒有覺得自己.
有任何的問題.
並且在這個時候.
作出一個判處.
耶穌不是神的兒子.
所以陳述在馬太方先前第26章裡.
不斷提出耶穌是神的兒子.
這個身份.
他們根本不相信.
最後第44節.
.
這群人.
死到臨頭.
這兩個.
仍然牙斬斬.
這是一個罪人的典型特色.
罪人沒有什麼厲害.
最厲害的就是嘴巴.
嘴巴沒有什麼好說的.
每句話都充滿恃意.
有時你會問.
牧師.
現在這個世界的人.
你說話不囂張一點.
你會被人囂張.
在這個生活裡.
丁小妹.
是不是我們.
要活在這個世界.
要用囂張的說話.
去對待一些專講.
羈笑說話的人.
耶穌有沒有說過.

$^{1201}$一句都沒有說過.
從聖經裡面.
可以看見.
這群人.
全部都是.
根根計較的人.
他們不懂.
根根計較.
他們計較的是自己開心嗎.
自己是否得到.
上帝的喜悅.
其他人是否喜悅.
他們不在乎.
最重要的是我是否得到喜悅.
而且這群人.
那種信仰的方式.
完全是.
一群以訛傳訛的信仰.
看到嗎.
連那群祭司長和長老.
都是這樣的水平.
我讀到這裡.
我會問一個問題.
讀甚麼神學.
好像最近.
我和一個目者說.
聊天.
他說的東西.
我真的很難接受.
好像和一個沒有讀過神學的人一樣.
人啊.
我不說信徒.
和沒有讀過神學的人一樣.
浪費力氣.
和他們繼續說.
但耶穌在這種情況下.
在我們看見這群人.
縱使他們的信仰.
是這麼低.
耶穌沒有.

$^{1241}$發任何一言.
和他們辯論.
因為耶穌知道.
你再和他們辯論也好.
他們的生命上.
都是一個樹架.
他們不要拯救.
他們的生命.
他們寧願要一個.
叫耶穌巴拿巴的人.
取代一個.
真正施拯救的耶穌.
來到這段聖經.
馬太寫完這堆東西之後.
讀了這段聖經之後.
我們應該問一下.
我們像不像這些人.
我平日的生活.
是不是像這些人.
見到別人落難.
被人打.
被人殺.
見到別人落難.
你就多踩兩腳.
好像路過那些.
搖頭扮ye .
還計謀.
還說耶穌根本.
不是神的兒子.
而那個.
在十字架上.
在叮叮那兩個囚犯.
仍然嘴嘴嘴.
在那裡.
譏笑耶穌.
今天這段經文充分地描繪了一幅.
由多種人物.
都是罪人的人物的描寫掃描.
讓我們清楚量度一下.
我們是不是一個罪人.

$^{1281}$馬太為什麼來到這段聖經.
要說一段這樣的經文給我們聽呢.
是不是純粹要為了記載耶穌死而死.
不要忘記還有一個人物.
叫彼得.
在第27章後面至28章.
就說這個人物.
他的改變.
什麼使他要改變.
我相信這段經文.
當馬太寫給當時受眾的時候.
他看見.
你可以作為一個判官.
你看看這些人多壞.
你看看這些人多虛偽.
你看看這些士兵.
做的事很無聊.
你看看這些死到臨頭都是嘴巴張開的人.
你看看這些人說話很囂張.
你可以這樣批評.
還是你回來看完這段經文之後.
我是不是這樣的人.
我是不是像他們這樣的心態.
我是作為信徒.
我在教會裡是不是活了.
像他們這些.
甚至像祭司長和文士長老們的態度呢.
說信其實不信.
靈姐妹.
這是不是馬太寫到這裡.
發出的問題.
提醒我們.
我們不是完美.
我們可能會錯.
我們可能會在迷路.
但是會不會回來.
回轉.
我做牧者的.
我自己也知道.
我不是完美的.

$^{1321}$我一定會有錯的事.
有沒有承認自己的錯.
承認自己的不足.
甚至因為回到教會多的時候.
有些人以為自己.
漸漸已經比其他那些初信的好.
其實不是的.
可能你越回到教會.
你的信越污穢.
越受污染.
你有沒有想想.
這段聖經.
馬太寫給他的讀者.
就是要做一面鏡.
去看自己.
定時定後來讀.
所以為什麼聖經要定時定後來讀.
看這面鏡.
我是不是這樣的人.
嘴唇張張.
經常串爆.
電影的角色.
黑社會.
串爆.
為什麼要串人.
是不是叫你串.
耶穌一句不出聲.
為什麼要串人.
信徒是不是要串人.
求主幫助我們.
讓我們透過這段聖經.
得以提醒.
讓我們誠心低頭.
多謝你讓我們在這段聖經裡.
藉著馬太的記載.
讓我們看見人性的可怕.
求你讓我們藉著.
他們的表現.
成為借鑒.
定時定刻地.

$^{1361}$再次反思我們是一個怎樣的人.
我們是不是像他們一樣.
還是像.
真正屬神子民一樣.
求你的幫助.
求你的聖靈.
讓我們有勇氣.
去面對自己.
真正的屬靈狀態.
並且.
可以回歸.
常主你的真理的童從.
禱告奉靠.
主耶穌基督.
得聖名至已救.
阿們.
\newpage



\section{馬太福音 27:45-56}
\label{sec:7upP8JmD6zY}
\textbf{愛護我們的比親人更知心 (馬太福音27\_45-56) - 黃紹權牧師  [馬太福音信息系列 - 第147講]}
\newline
\newline
連結: \href{https://youtube.com/watch?v=7upP8JmD6zY}{\texttt{ https://youtube.com/watch?v=7upP8JmD6zY}} ~~~~ 語音日期: 2025-02-09 
\newline
\newline
\hyperref[sec:oCpi7n8ictU]{< < < PREV SERMON < < <}
~
\hyperlink{toc}{[返主目錄]}
~
\hyperref[ch:preacher10]{[返講員目錄]}
~
\hyperref[sec:MfR5_HAo14I]{> > > NEXT SERMON > > >}
\newline
\newline
馬太福音 27:45-56
\newline
\begin{longtable}{cl}
\hline
\hline
章節 & 經文 (和合本修訂版)\\
\hline
27:45 & \begin{tabularx}{0.7\textwidth}{X} 從正午到下午三點鐘,遍地都黑暗了。 \end{tabularx} \\ \\ \relax
27:46 & \begin{tabularx}{0.7\textwidth}{X} 約在下午三點鐘,耶穌大聲高呼,說:「以利!以利!拉馬撒巴各大尼?」就是說:「我的神!我的神!為甚麼離棄我?」 \end{tabularx} \\ \\ \relax
27:47 & \begin{tabularx}{0.7\textwidth}{X} 站在那裡的人,有的聽見就說:「這個人呼叫以利亞呢!」 \end{tabularx} \\ \\ \relax
27:48 & \begin{tabularx}{0.7\textwidth}{X} 其中有一個人立刻跑去,拿海綿蘸滿了醋,綁在蘆葦稈上,送給他喝。 \end{tabularx} \\ \\ \relax
27:49 & \begin{tabularx}{0.7\textwidth}{X} 其餘的人說:「且等著,看以利亞來不來救他。」 \end{tabularx} \\ \\ \relax
27:50 & \begin{tabularx}{0.7\textwidth}{X} 耶穌又大喊一聲,氣就斷了。 \end{tabularx} \\ \\ \relax
27:51 & \begin{tabularx}{0.7\textwidth}{X} 忽然,殿的幔子從上到下裂為兩半,地震動,磐石崩裂, \end{tabularx} \\ \\ \relax
27:52 & \begin{tabularx}{0.7\textwidth}{X} 墳墓也開了,有許多已睡了的聖徒的身體也復活了。 \end{tabularx} \\ \\ \relax
27:53 & \begin{tabularx}{0.7\textwidth}{X} 耶穌復活以後,他們從墳墓裡出來,進了聖城,向許多人顯現。 \end{tabularx} \\ \\ \relax
27:54 & \begin{tabularx}{0.7\textwidth}{X} 百夫長和跟他一同看守耶穌的人看見地震和所經歷的事,非常害怕,說:「他真是神的兒子!」 \end{tabularx} \\ \\ \relax
27:55 & \begin{tabularx}{0.7\textwidth}{X} 有好些婦女在那裡,遠遠地觀看,她們是從加利利跟隨耶穌,來服事他的; \end{tabularx} \\ \\ \relax
27:56 & \begin{tabularx}{0.7\textwidth}{X} 其中有抹大拉的馬利亞,又有雅各和約瑟的母親馬利亞,並有西庇太兩個兒子的母親。 \end{tabularx} \\ \\
[1ex]
\hline
\hline
\end{longtable}
$^{1}$父神多謝你讓我們有生命來打開你的說話.
今天有多少人仍然有一本聖經在他手上.
今天有多少人有聖經仍然沒有讀.
但感恩你給我們一班追求主導的人.
回到上帝的面前 不論在實體崇拜還是在線上.
都要一起去等候和學習主你的說話.
我們為此獻上感恩.
你有一班弟兄姊妹能夠和我們一起.
堅守在追求真道的這條路上.
就讓我們再一次在這個時候.
懇求聖靈帶領我們去明白主你的教訓.
為同心的禱告奉靠主耶穌基督得勝的明智求.
阿們.
在聖經第十八章第二十四節.
Proverbs Chapter 18 Verse 24.
話本聖經裡這樣記載.
他說「男交朋友的智取敗壞.
但有一朋友比弟兄更親密」.
有一次我和一個姐妹一起去喝茶.
其實是和她先生一起喝茶.
聊聊天 很久沒見.
她多數會問我.
她說「你的子女怎麼樣呀」.
我說「他們都很好 各自有自己的忙碌」.
她說「是的 當長大了之後.
每個人都有自己的朋友.
回想起自己的朋友 他的生活.
他都有說過.
對於他的人生都有很大的影響.
希望你的子女都同樣.
有一班好的朋友在身邊.
一起走他人生的道路」.
當她這樣說的時候.
就讓我想起這節經文.
這節經文我已經聽了很多年.
但究竟聖經裡所指的朋友是甚麼人呢.
特別那句「但有一朋友比弟兄更親密」.
我真的想了很多年.
弟兄即是你自己的Brother and Sister.
應該是和你玩到長大的人.

$^{41}$為甚麼朋友比你的家人更重要.
我很不明白這句說話.
但很多時候很多目者.
都說過有說數.
又不解釋.
你表面覺得他怎樣解釋就怎樣解釋.
於是有一天我就仔細去找回.
希伯來文時去看究竟這兩次所提及的朋友.
所指的是甚麼呢.
有一個學習在他的原文聖經裡的理解.
當我再查這節經文的時候.
原來發現這節經文所談到的朋友.
並不是一般的朋友.
原來是在和合本句裡所指的朋友.
在原文希伯來文是用了一個分詞.
甚麼是分詞呢.
分詞是永遠將一個動詞.
變成一個名詞或形容詞的詞態.
即是將一個動作變成一個名詞.
而這個名詞代表這個人或這件事.
是一個動作是要去做的.
於是我就更仔細去看.
究竟這個Proverbs 18 verse 24.
箴言十八章二十四節的朋友是怎樣定義.
他就有一個英語的翻譯.
用希伯來文的原意翻譯出來.
就是One who is loving.
一個一直願意愛護著你的人.
當我查回希伯來文了解這個字眼的時候.
就真真正正讓我看到.
這個朋友並不是你在街邊認識一個.
和你做生意的朋友.
我爸爸做生意的.
不過他讓我看到所謂的朋友.
那些人來叫他做朋友的.
其實不一定是朋友.
他來只是想看你身上得到甚麼利益.
那些不是對你有愛的.
那些是剝削你的人.
如果用英語說.

$^{81}$就是One who is abusing you.
那些是在你身上獲取利益的人.
但希伯來文裡面說.
是一個人持續地關愛你的.
為你的好處提供愛護.
不論甚麼情況.
只需要你在任何情況面對他的時候.
他都會為你的最大的益處去考量.
當我查到這節聖經的時候.
就讓我想起耶穌基督.
在十字架上死的時候.
曾經和他稱兄道弟的一班門徒們.
又如何對待他呢.
這班門徒是否仍然這麼愛著耶穌.
關心著耶穌.
還是這班門徒完全各自走自己自保的道路呢.
當我讀到耶穌基督臨死的片段.
或者我們稱為Passion narrative.
我不想說太多技術性的東西.
因為在研究聖經裡面有很多技術的術語.
當我介紹到耶穌受難蒙難這段記載的時候.
就讓我看見一件事.
一直陪伴著耶穌的不是一班門徒.
而是一班沒有對白.
沒有表現.
甚至沒有太遲的婦女.
而這班婦女卻緊緊地跟隨著耶穌.
即使她未能在現場身邊.
但仍然在可見的距離之下.
一起留心守望著耶穌.
希望最後還有什麼可以為耶穌做呢.
我相信這班婦女也是馬太福音裡有一個很特別的記載.
要告訴我們一班不喜歡出風頭的人.
不喜歡出來玩玩神蹟.
傳道 說得精精彩彩的人.
而是一班很黑勤地低調地去守望著耶穌的婦女.
而這班婦女是一直關懷著耶穌每一個的片段.
還有什麼可以做呢.
古近東的文化婦女是一班無權無勢的人物.
一般人對於婦女的看法是覺得她們是軟弱無能.

$^{121}$但其實我們看聖經裡面.
婦女的角色就是正正發揮到婦女這種安靜柔弱.
但也愛護的心態.
這反而是今天基督徒很少留心去看的事情.
所以今天當我們去看回耶穌死在十字架上的時候.
一方面透視到周遭的人的靈性是怎樣.
我們上文已經看過.
即使是那班兵兵 周遭走過的路人.
以及那班祭司 民事都好.
將他們的靈性暴露出來.
但不是單純說一個不好的一面.
來到今天聖經我們更加看見一班怎樣愛護著耶穌基督的人.
他們怎樣在這個時候彼此守望.
並且能夠一直發揮出舊約聖經.
箴言十八章二十四節裡面那種.
真是持續地愛耶穌的一班信徒.
所以今天我和大家一起看的題目就是.
愛護我們比親人更至深.
不代表親人不好.
不代表親人能夠做到信徒彼此守望的更漂亮.
但這段聖經告訴我們.
原來當一個持續地愛護著你的人.
原來他對於你的貼心.
是比你的親人更加的為之親密.
我們一會再去看一看.
讓我再次打開這段聖經.
不是很長的.
我們自己再讀一次.
第27章第45至56節.
你可以隨便用自己的語言去讀就行了.
我就用廣東話帶領大家一起去讀.
我們一起打開馬太福音27章45至56節.
我們一起讀.
從五聖道新初遍地都黑暗了.
若在新初.
耶穌大聲喊著說.
以利以利拉瑪撒巴各大尼.
就是說我的神我的神為甚麼離棄我.
站在那裡的人有的聽見就說.
這個人呼叫以利亞.

$^{161}$內中有一個人趕緊跑去.
拿海俑也滿了槽.
綁在帷子上送給他喝.
其餘的人說且等著.
看以利亞救他不來.
耶穌又大聲喊叫.
氣就斷了.
忽然電裡的萬紫從上到下裂為兩半.
地也震動.
盤石也崩裂.
墳墓也開了.
以遂聖徒的身體多有起來的.
到那耶穌復活以後.
他們從墳墓裡出來.
進了聖城.
向許多人顯現.
伯夫長和一同看守耶穌的人.
看見地震並所經歷的事.
就極其害怕.
說這是神的兒子了.
有些好些婦女在那裡.
他們就從加利利跟隨耶穌來服侍他的.
內中有莫大拉的瑪利亞.
又有雅各和約西的母親瑪利亞.
並有西比泰兩個兒子的母親.
今天我們會從三方面一起去看這段聖經.
第一方面就是靈性和感官的關係.
可以從第45至第50節一起去看.
人有五種感官.
味覺,聽覺,觸覺等等.
大家慢慢數.
人的五種感官是直接連結著人的靈性.
如果只有五種感官的話.
沒有靈性就等同與動物畜生沒分別.
譬如吃東西可以吃到甜酸苦辣.
我家裡有幾隻動物.
我女兒養的.
其中一隻叫瑪利亞.
一隻叫什麼鼠?.
我都不知道叫什麼鼠.

$^{201}$豬呀.
土撥鼠呀.
不是土撥鼠.
是千竹鼠.
很有趣的.
我每一早都會跟他們打招呼.
因為我女兒還沒出生.
通常她們聽到我腳步就叫我.
我就給她們吃東西.
到達的時候.
她們最喜歡的東西是不同的.
土又不同.
千竹鼠又不同.
按照她們喜歡的東西給她們吃.
她們都很開心.
有時候我會特意做一些東西.
希望大家可以一起吃.
其中一樣就是吃橙.
我很喜歡吃橙.
有時候吃橙千萬不要買橙皮.
買橙皮的橙皮.
下面有一大塊東西.
黏著皮.
浪費了不值得吃.
我買橙皮也很棒.
不要買那塊東西.
那裡有一個.
但有時候看不到.
然後就撕一片片肉給她們吃.
土很喜歡吃.
土很喜歡吃橙.
給了她們吃.
她們很開心.
另外一隻又給她們吃橙.
那隻叫做Loaf.
什麼都吃.
就像垃圾槽一樣.
另外一隻Molly就很特別.
當我放一塊橙在她們面前的時候.
她們很聰明的在籠子裡看著我.

$^{241}$我就把一塊橙伸進去給她們.
你猜她們有什麼樣相.
現在看影片好一點.
可以做一個影片給你看.
她有兩隻門牙.
她就首先把兩隻門牙治好.
然後叫她們走掉.
她們不吃.
我叫她們不吃.
我就會說一句.
我給另外一隻.
那隻會吃掉.
其實如果動物只有五種感官.
是沒有什麼意思.
但如果有靈性.
那種感官就有意義出來.
當耶穌死在十字架上的時候.
其實本來沒有需要死在十字架上.
為什麼要死在十字架上呢.
因為在十字架上受死的耶穌.
彰顯了很多人性在下面.
甚至在他身邊的兩個銅釘十字架的人.
他們真正的靈性是怎麼樣呢.
上星期我已經說過.
和耶穌銅釘十字架的兩個人.
一起和其他人都去奚落耶穌.
當然大卷書有介紹到.
其中一個有悔改.
但很明顯馬太在這裡先記載了.
他們最初的時候的反應.
哇 每個人都奚落.
不要執書了.
我都奚落了一份.
在今天這段聖經裡.
要進一步將這些人的人性.
全部暴露出來.
但也同時在最後的時候.
將一些截然不同的人.
我們稱為婦女.
而且這些人有名有姓.

$^{281}$我一會再詳細去看.
將他們逐一記載.
做了一個很大的對比.
給我們去看看.
但有一部分我們所看的.
當那些人的靈性是多麼的腐爛墮落的時候.
就算他們有齊感官也好.
他們看到有視覺.
可能有嗅覺.
但他們所做的事.
其實不是帶著愛和照顧恩慈.
而是希望在當中.
可以用我現在喜歡的字眼.
抽水 過一下心癮.
可以看看這段聖經發生了什麼事.
在一開始第45節.
宗教從五正到新初.
偏偏都突然黑暗.
為什麼黑暗呢.
一會再解釋.
但在這個時候.
就讓我們看到耶穌在新初的時候.
五正到新初有一段時間.
按照猶太人的曆法.
每一個時間大約有兩個小時左右.
五正到新初.
如果你查看七十四譯本的時候.
就會知道是從第六小時到第九小時.
首先我再重複說多一點.
猶太人的時間觀念.
不是像我們這樣硬梆梆.
將一天24小時都斬開.
轉個小時是一樣的.
不是這樣的.
每個時節不同的時候.
鐘頭的長短都不同.
所以這裡說到第六小時到第九小時.
一個小時的時間可能是一個小時多一點.
有些可能接近兩個小時多一點.
無論怎樣.

$^{321}$白天分十二個小時.
晚上分十二個小時.
這是一般和我們現在很類似的情況.
但在這個時候.
到這個大約第九小時的時候.
耶穌在這個時候大聲喊著說.
說了一句話.
以利以利拉瑪撒巴國大利.
我做傳道人.
我自己有個心願.
有些人經常說.
牧師你講得這麼長.
我就在這裡透露心聲給大家聽.
因為我相信我處理那段經文.
一生可能只跟大家說一次.
也處理一次.
這麼多卷書.
我一生人不敢說.
如果說退休前.
六十五歲來的.
我還有九年.
九年怎麼說.
我很多卷書還沒說.
你又這麼長.
又不知道什麼時候說.
但以利以利可能要三月才說.
現在就算了.
三月才說.
馬太夫人說了兩年多.
為什麼這麼長.
因為我希望我每一次處理好這段經文.
將重點清楚說完之後.
我或者沒機會再跟大家說.
但很難得這段經文.
經常都說.
因為每年都要說復活節.
當然要說受苦節受難節.
這段經文一定有份.
但這段經文所說的.
以利以利拉瑪撒巴國大利.

$^{361}$是什麼意思.
馬太福音的讀者.
我相信他們當時用的是.
希臘文為主.
或者甚至是亞蘭文.
因此他們對於這句.
這麼傳統的希伯來文的說話.
未必是這麼全面掌握.
耶穌傳道的時候.
我們相信主要說的語言是亞蘭文.
大家請留意.
耶穌應該說的是亞蘭文.
亞蘭文是當時普及的語言.
就像今天的英語一樣.
你不懂英文.
很多事情都做不到.
讀電腦的都一樣.
你不懂英文.
你不會寫那個指令.
有些人.
我記得我三十多年前.
去學電腦的時候.
我讀電腦學習的時候.
有個人說我需要學英文去寫程式嗎.
現在我們很先進.
用中文寫程式.
我說是不是這麼厲害.
阿偉東是不是這麼厲害.
用中文寫程式.
他說可以的.
我說你怎麼可以.
原來他說那些指令.
全部可以轉成中文.
我打中文就可以了.
什麼return.
我不用再打return了.
回車.
什麼回車.
我就不懂得怎麼回車.
return.

$^{401}$倒轉我懂得怎麼回車.
他做了翻譯.
後來我發覺那個人.
也是用英文寫.
我相信你打return.
比打回車快.
所以你不懂英文沒有用.
很多事情都做不到.
耶穌應該是用亞蘭文說.
或者亞蘭文是一種什麼語言.
很近似希伯來文的一種語言.
所以在這裡.
馬太提出.
以利拉瑪撒巴國大利的時候.
他都翻譯給當時一班.
已經習慣用希列文的讀者.
說My God.
My God.
Why have you forsaken me.
翻譯為我的神.
我的神為什麼離棄我.
這個是馬太福音.
為什麼有這麼多翻譯出現.
而當聖經去說的時候.
就反映出原來當時的人民.
他們根本不太理解這個語言.
為什麼呢.
第47節.
站在那裡的人.
有的聽見就說.
這個人呼叫以利亞.
內中有一個人還趕緊跑去拿海洋.
海洋沾滿了醋.
綁在帷幅上.
送給他人.
其餘的人說.
且等著看.
以利亞來救他不來.
第50節.
耶穌又大聲喊叫說.

$^{441}$氣就斷了.
這段聖經說給很多信徒聽都很熟.
甚至不用說了.
目的我們都知道說什麼.
但這段聖經說這麼多的圖畫.
原來是在說給我們知道.
有關於舊約聖經有一個很重要的原則.
就叫做耶和華的日子.
The Day of the Lord的來臨.
我不詳細說了.
因為在約珥書的時候.
我講道有說過.
如果大家可以重溫.
可以重溫約珥書的時候的講道.
耶和華的日子.
The Day of the Lord是什麼意思呢.
一方面是上帝的審判日.
審判日子.
上帝和你清算你所有的事.
但是耶和華的日子.
也是一個思憐的日子.
It's a time or day to grant the grace of the Lord.
為什麼呢.
因為上帝在這個時候.
要向一群認罪悔改的人.
提供一個機會.
相信他們在這個時候認錯的話.
上帝會接納他們.
而這幅圖畫往往就像.
今天這段聖經裡面的記載.
包括了在經文裡面所談論到的.
有黑暗在遍地的當中.
又有些大聲的喊叫在裡面去呼喚.
還有很多的我不詳細說了.
這幅圖畫其實告訴我們.
上帝在這個時候.
正要向世人作一個最終的分界線的時間.
人究竟在這一刻.
只是嘗試了審判.
還是得到上帝恩典的開始呢.

$^{481}$所以當這段聖經.
提到這段經文的時候.
馬太介紹就告訴我們.
其實上帝在這一刻.
透過耶穌基督的受死.
一方面是向罪後的重重的攻擊.
一個決定性的攻擊.
但同時間也是讓那些願意真心悔改的人.
可以得以進入到恩典的裡面.
這位全能的神來到他們當中.
希望帶來的不只是災難的審判.
而是希望帶來一份扭轉罪惡的一種時刻.
恩典的時代的開始.
插句話說.
現在加拿大受25\%的關稅.
有些人覺得死了就沒工作了.
死了很多公司會倒閉.
為什麼不想想這是上帝給加拿大人的一個很重要的警號.
沒錯可能你會受傷害.
沒錯可能你會減薪甚至失去工作.
沒錯可能裁員我都快要被裁了.
在這樣的情況下.
上帝給我們看到什麼.
有很多人勸我牧師不要再在稅務局做事了.
做什麼呢.
他以為我為錢.
不是.
向很多偷,騙,拐騙的人.
給他們一個正確的觀念.
他們覺得很奇怪.
為什麼我要這樣做呢.
我可以得個且個算數就算了.
不是的.
讓我看清楚人的罪惡同時.
上帝在這個空間有什麼角色去做.
當這些人在這個時候面對上帝的審判來臨.
他們能不能夠準確收到訊息呢.
這也反映出人有沒有神的靈.
你知道有神的靈的人是怎樣嗎.
他們遇見每一件事都在裡面去問上帝.

$^{521}$上帝有什麼意思在裡面.
包括病.
教會主席問我.
牧師你對我病的看法是怎樣.
有什麼觀點呢.
我記得我去探望他的時候說了一句話.
不要浪費這個病.
Don't waste your illness.
上帝給你每一樣遭遇.
是呼喚你回去而不是叫你更遠離上帝.
有很多人說上帝你不保守我.
於是使我失業.
使我失去女朋友.
失去婚姻.
失去健康.
這些上帝我不會再相信你.
沒有靈性的人.
有靈性的人.
在遇見這些事的時候.
更緊密地走到上帝面前.
上帝你究竟是什麼意思呢.
所以我們做牧者很簡單.
如果你看到他遭遇患難的時候.
你為他祈求的是不是解決問題.
不是解決的問題.
而是問題引發出你對生命對上帝的看法.
你怎樣對待這位上主.
他是否一位持久地愛護著你的上帝呢.
還是你覺得他只不過信他就這槍.
逆他就這亡.
很多信徒是這樣的情況.
這班人坐在耶穌下邊的人.
當耶穌呼叫 以利 以利拉瑪撒巴葛大尼的時候.
我懶得說希伯來文了.
人們說牧師你不要說那麼多希伯來文.
他們聽到的東西是什麼呢.
我想問大家一個問題.
以利 以利是什麼意思呢.
你用不著懂得希伯來文.
那不就解釋給你聽了.

$^{561}$我的神 我的神.
但這班在下面聽到耶穌叫.
以利 以利拉瑪撒巴葛大尼的人.
他們聽出了什麼呢.
他們聽出了以利亞的名字.
不知道你有沒有試過.
有些人說話說得很清楚.
但那個人聽不清楚的時候抱怨甚至反擊.
你為什麼說這些話.
然後你都瞎了頭.
為什麼呢.
在這段聖經裡我們很清楚看到.
這班人當耶穌在這個時候呼喚出.
原來神在這一刻離棄他的時候.
是將他送入到罪的承擔的位置上.
我簡單地說 上帝和耶穌在同宮.
耶穌把他呼喚出來 告訴下面的人知道.
下面的人聽不聽到.
以利是聽覺的其中一樣東西.
今天很多人有聽覺是聽不見的.
眼是有視覺 看不清楚.
但最可怕的是因為他們的心由蒙知由.
這些經文是從哪裡來的呢.
我剛才說的不是我說的東西.
是以塞瓦書.
弟兄姊妹 在以塞瓦書裡.
清楚介紹出來這班人士.
其實他們的問題癥結是有耳聽不見.
有眼看不到.
這些感官對於他們幫不了他們的心.
能夠更貼近上帝.
反而使他們的心更遠離上帝.
這段經文引用於詩篇第21篇第一節.
其實是原本用來提醒當時的人.
究竟他們有沒有真正在上帝的裡面.
這班人聽見耶穌這樣叫的時候.
他們聽覺可以說是有還聽.
聽清楚一句話.
其實講了兩次耶穌.
以利 以利 講了兩次.

$^{601}$你都聽出來是以利.
為什麼會聽出來是以利.
因為他們的心從頭到尾就覺得脫離困難.
就找先知以利亞.
這就是他們的前設.
很多時候人為什麼會有五個觸覺.
他們不能夠明白上帝的心.
為什麼他們有五個觸覺都不能夠明白上帝的心.
因為他們有前設.
他們有宗教框架.
因為在猶太教裡面.
他們覺得有任何問題的時候.
找最厲害的先知.
那個先知是誰.
就是以利亞.
在這樣的情況下.
這班人不但聽得清楚耶穌的宣告.
而且他們那種自私和奚落耶穌的伎倆.
越來越不放過.
我們再看聖經說什麼.
這班人當聽見耶穌的時候.
第四十八節.
內宗有一個人趕著跑去拿可用飲料或是斟料.
然後放在路位子上.
給耶穌喝.
馬太福音沒有說耶穌有沒有喝這些醋.
專注點是要讓我們看回這個人.
在原文聖經裡面.
希臘文說內宗有一個人趕著跑去.
其實是說馬上跑去.
用英文翻譯.
immediately went somewhere.
馬上去做.
在聖經裡面.
馬太福音有一個慣常的運作.
我已經說過好幾次.
每逢一個詞彙用在耶穌基督身上的時候.
都是正面的.
但當用在人的身上都是負面的.
這個就是典型例子.

$^{641}$refuse.
希臘文這個字眼.
當用在耶穌馬上行動的時候.
馬上就去拯救.
馬上去呼召門徒.
馬上去醫治.
但每次用在人的身上的時候.
每次馬上都沒有好事做出來.
馬太福音就用了一個助動詞.
或者動詞的形容詞.
來介紹出耶穌基督和身邊的同丁.
以及在下面的人.
這群人的靈性狀態是多麼邪惡.
這些醋喝了有什麼作用呢.
是否用來使他們更痛苦.
或者死得久一點拖長一點才死呢.
醋還是解作酒呢.
醋是酒來的.
可以翻譯為酒.
很多人在做這些研究.
看清楚聖經.
聖經不是說酒是醋.
不是說功能.
而是說這些人喝了這些東西.
放在蘆葦子上.
然後送給耶穌.
又蘆葦?.
穆斯林已經說了第三次了.
第一次說蘆葦子是哪一次呢.
那群兵兵在城堡裡脫下耶穌的衣服.
裝成耶穌的軍王.
然後給他弄一支蘆葦子.
當作是權杖用的.
羞辱耶穌好像軍王.
你說你是軍王.
我扮成軍王.
不過你沒有用的.
第二次就是當耶穌拖到十字架的刑場的時候.
他們就拿這支蘆葦子來打耶穌的臉.
記不記得.

$^{681}$現在這支蘆葦子又大派用場了.
用你的權杖.
給你喝酒.
這是什麼態度.
現在不是你做王.
現在是我做王.
我給你喝酒.
完全充滿著.
我是你老闆.
這不是髒話.
我是你的老闆.
I boss you.
I am the king.
你說你是king.
你就是王.
浪費力氣了.
我才是王.
這群人的醜行.
充分反映出在這段經文身上.
所以大兄姐妹.
當我們看回這段經文的時候.
我們會發現這群人以為自己可以為所欲為.
多麼開心.
I have freedom.
I can do whatever I like.
我喜歡做什麼就做什麼.
你以為你是王.
我才是王.
但在這個世界裡.
你越以為自己是王.
你越以為自己是自由.
原來你正是捆綁在自己的罪惡裡面.
在這段聖經裡面.
清楚讓我們看見.
這群人不是更自由.
而是他們在心靈的傷殘裡面走不出來.
They are fully handicapped in spiritual life.
兄弟姐妹.
這種情況就好像你買一個最先進的手提電話.
現在最先進的手提電話是iPhone16.

$^{721}$或是華為三摺.
我不管你買什麼電話.
如果你買一個電話.
原來它不能接駁WiFi和Data.
你認為這個電話有沒有用.
今天很多信徒就是這樣的電話.
接不上網.
They cannot connect back to God.
他以為自己和上帝連網.
原來根本不連網.
他的網是內網.
Internet.
Internet就是自己.
很可怕.
也很可憐.
弟兄姐妹.
你是否這樣的人.
在這樣的情況下.
耶穌受死.
51至53節.
在這裡確實是比我看見.
耶穌雖然受死.
死在十字架上.
但他的呼喚卻將原本塞了的屬靈道路打通了.
我們一起看看第51至第53節.
我讀出你們留意聽.
忽然電裡的萬子從上到下裂為兩半.
地也震動.
盤石也崩裂.
墳墓也開了.
以歲聖徒的身體多有起來的.
到耶穌復活以後.
他們從墳墓裡出來.
進了聖城.
向許多人顯現.
這段聖經可以說是聖經裡面其中一段.
很嚇人的情況.
怎樣嚇人呢.
第一件事我們來看看.
事件突然發生.

$^{761}$電裡的萬子從上而下裂為兩半.
升電的萬子分內裂和外裂兩重裂腳.
在這兩條裂腳裡面分開打破.
可以說是史無前例的情況.
而這件事是一件很明顯非自然的行動.
你沒有看到有任何人知道.
因為這件事是忽然發生.
突然發生.
在耶穌的呼喊裡面發生.
當這裂腳被打開後.
我們就看看.
一連串的圖畫繼續發展耶和華的日子.
The Day of the Lord的典型記載.
地震山崩.
在傳統的耶和華審判日子裡面.
就是這樣的情況.
使人驚嚇並且無能為力的情況.
有沒有人遇見地震.
我真真正正試過地震一次.
去年我經過日本的時候.
在酒店裡面突然間浮起浮起的東西.
在衣服上搖晃搖晃.
接著電話就出來了.
說有地震了不能乘升降機.
沒有辦法.
其實我也害怕的.
如果那層樓塌了我就死了.
幸好我住在十樓.
如果塌了就塌在下面.
還有時間浪一浪.
地震完全是出了人所能控制的情況.
是一種完全失去控制的情況.
不再說我讀了三個PhD.
四個Master.
然後可以保證你不用地震不怕.
沒有的.
等同於南亞海嘯.
大家記得嗎.
2004年南亞海嘯.
那裡的基督徒沒有死嗎.

$^{801}$不是的.
一樣死.
基督徒不是一樣死.
一個超出自己的控制的情況.
但這時候發生了什麼事呢.
竟然那些睡了的信徒.
信徒的身體都有起來.
從那裡起來呢.
從墳墓那裡起來.
為什麼墳墓會起來.
因為墳墓都開門了.
大家看看第53節說.
墳墓也開了.
以睡的信徒的身體就多有起來.
起來做什麼呢.
原來預備和耶穌的復活迎接.
當耶穌的復活來到的時候.
他們就跟著從墳墓裡出來.
走入聖城裡面.
向很多人展現.
哇 嚇死你.
那一刻 無端端死了的外祖母.
為什麼又出來呢.
搞什麼事呢.
原來她信了耶穌.
不是信上帝的.
在這幅圖畫裡面.
讓我們看回一件事.
原來有信的人.
是和耶穌的生命真真正正成為有連結.
這些就是真正命運共同體.
你和耶穌同一體的時候.
耶穌復活的時候你也會復活.
但是你復活是為了什麼原因呢.
為見證的原因而復活.
現在不是討論這群人之後還會不會再死.
這群人當然最後會死.
和今天我們所說的永恆的死.
和永恆的復活是有點不同的.
不在講道講了.

$^{841}$因為這些很複雜的神學問題.
不過很明顯馬太也要告訴我們.
巴巴肯相信主的人.
他就要在上帝超自然的能力之下.
復活起來 作見證.
今天每一個認罪悔改的信徒.
他們最重要的目的是什麼呢.
作主的見證.
見證一個有神生命的人.
而且不是口說 是有行動.
這群人怎麼行動呢.
他們走進了聖城.
證明了墳場在外面.
然後他們走進了聖城.
作聖城裡面的人的見證.
各位姐妹.
馬太在這裡將耶穌死後的第一手資料.
在這裡說出來給我們知道.
很明顯馬太福音已經是過了耶穌復活之後的事情.
但是要重新勾起所有收到馬太福音的讀者.
再次明白到復活這件事情.
不是空口說白話.
而是一件真真正正發生過的事.
而這些復活從墳墓走出來的信徒.
就成為見證.
今天你在地上是在做上帝耶穌基督的見證.
還是在做你自己想做的見證.
甚至你不是做見證.
而是為自己謀取利益.
最後一部分第54至第56節.
讓我們看見信耶穌基督的人.
他那份愛護人的情操的彰顯.
我們一起讀吧.
這段聖經很漂亮.
所以我想大家一起讀.
第54至第56節.
白夫長和一同看守耶穌的人.
看見地震並所經歷的事.
就極其害怕.
說:這真是神的兒子了.

$^{881}$有好些婦女在那裡.
她們是從加利利跟隨耶穌來服侍他的.
內中有莫大拉的瑪利亞.
又有雅各和約西的母親瑪利亞.
並有西比泰兩個兒子的瑪雅路母親.
在聖經裡給我們看到.
當地動山移的情況.
展現在眾人面前的時候.
這群人當中有一個白夫長.
和一同看守耶穌的人.
看見這件事.
他們產生震驚.
然後宣告了一個事實.
這真是神的兒子了.
NRSV的翻譯是.
Now when the centurion and those with him.
who were keeping watch over Jesus.
saw the earthquake and what took place.
they were terrified and said.
Truly this man was God's son.
當你讀到這條聖經就.
哇 有些人很驚訝 所以他這樣說.
讓占培仔細看.
為什麼要讀英文聖經給大家聽.
When the centurion and his people.
誰是白夫長呢.
為什麼要指明有一個白夫長.
和他的人在驚嚇之下宣告.
耶穌是神真正的兒子呢.
誰呢.
在馬太福音裡有一個白夫長.
在馬太福音第八章第五至十三節.
Matthew 8:5-13.
有一個在加伯隆求耶穌.
醫治他家中害癱患病的僕人的白夫長.
曾遠道而來找耶穌救他家中跌傷的家丁.
或者他心愛的家中的僕人.
如果不心愛他怎麼會來照顧他呢.
為什麼要來找耶穌呢.
因為太關心家丁或他的僕人.

$^{921}$會不會就是這個白夫長呢.
頂住門 我沒有答案.
不過很明顯.
馬太在寫這本書的時候.
在她的腦海裡就有這個白夫長.
如果我根據這樣的推論.
很可能她是指這個白夫長.
他和這些婦女們有一樣特性.
是什麼呢.
不是因為那個人的身份有多高或低.
不是因為他是僕人 下人.
可以虐待他.
因為我同樣是愛護他.
不離不棄服侍耶穌的人.
雖然我作為白夫長.
這個僕人是來服侍我 幫我做事的.
但我仍視為我自己所愛護的人.
這個見證在保羅講神學講到有關於.
一個家庭裡面夫妻之間的關係.
僕人和主人的關係都有介紹.
我不詳細講了.
自己去靈修看經文.
其實都告訴我們.
在主裡面的人.
即是你是屬於僕人還是主人都好.
主人會愛護僕人.
僕人也不會因為我是主內弟兄.
我可以和你同起同坐.
我仍然做僕人盡心盡力服侍你.
這班婦女就將自己的身份.
以一個能夠服侍耶穌的能力.
顯露在整個耶穌人生的旅程.
甚至是救恩成就的過程.
所以當這班人能夠有這份相信耶穌的人.
他看事物就不同了.
他看事物就是認到耶穌.
沒有神靈的人.
就算對他多好都好.
他看不到耶穌在他身上怎樣做事.
他只不過覺得你蠢姐對我好.

$^{961}$或者有人對我好何樂而不為.
但是有神靈的人.
就會看見這一切都是神所作的.
因為他是一位愛護我們的神.
而這個愛護的心志.
不單止在白虎掌和他的僕人身上.
也在這班婦女的身上.
所以第五十五節說.
有好些婦女在那裡遠遠地觀望.
一直觀察著耶穌.
「耶穌死了」.
「死了應該散場吧」.
「看甚麼呢」.
不是的.
仍然遠遠地看著耶穌.
而這班人他們的愛護耶穌的心志.
不是一時一刻.
不是感情的事.
是從加利利開始.
由他們和耶穌接觸開始.
就一直去服侍他.
而且有名有姓.
聖經不想我們當這班人.
只不過是路人甲乙丙.
他們是甚麼人呢?我們看看.
「莫大拉的瑪利亞」.
是亞國和約西的母親.
「瑪利亞」和西比泰.
兩個兒子的母親.
這幾個女士.
不只是在這裡才出現.
加利利已經出現.
接下來繼續出現.
他故意將這幾個女士.
擺在他們面前.
不是叫我們去做變性人.
做女人.
不是的.
不要這樣解釋.
亂說廿四傳道人有這樣說嗎?.

$^{1001}$不.
而是學校這班婦女的愛護上主之情.
而主耶穌基督又怎樣對待他們每一個呢?.
「莫大拉的瑪利亞」.
和他們有甚麼交往呢?.
讀到這裡.
馬太福音的作者提到你和我.
耶穌怎樣對待他們每一個.
上主耶穌基督是一位.
真正一直愛護著我們的人.
不是一般的朋友.
不是一般合作的partner.
而是一個願意付出.
一直照顧我們最需要的地方.
當這班人有這樣的靈性的時候.
他們就能夠體貼到耶穌的心腸.
用同樣的生命回應.
馬太在這裡不是叫我們做甚麼.
而是要讓我們看到一個有神明的人是怎樣.
不是拿著爐子去灑醋給耶穌喝的人.
不是那些兵丁.
不是那些文士,祭司,法利塞人,長老們.
而是像這位白夫長和他一同看守著耶穌基督的人.
一起看到的.
他是真正神明.
這次沒有人,耶穌已經死了.
可以說是在耶穌整個的預言.
誘因的發展的第二部分完成了.
第一部分是甚麼呢?.
耶穌被捉拿之後被鞭傷蜥蜴.
過完了.
現在第二部分死了.
還有第三步.
第三步就是耶穌的復活.
聽了耶穌講了這個預言和計劃的門徒.
總共十一個人.
他們現在是甚麼情況呢?.
我相信他們現在的情況是既驚慌又糾結.
這班人在這樣的情況下會不會回到耶穌基督面前呢?.
記不記得有一個人.

$^{1041}$那個天使走來跟這班婦女們說.
你去將這個信息報告.
這個遲些會講的.
不過這班門徒在失落的情況下.
丙姐妹,今天你可能同樣在失落的情況下.
可能你在這個時候仍然不知道自己在做甚麼.
為甚麼我會落到這樣的情況下.
下一步我應該怎樣做.
我完全沒有方向.
可能你的努力怎樣發奮都改變不了事實.
事實上我們要努力.
但努力不一定和你期望的成果是成正比的.
但是如果在這個時候你能夠看清楚.
認清楚誰是真正神的兒子的時候.
你的生命就會扭轉過來.
誰做這個角色.
幫他們扭轉過來.
婦女們.
一班很多人看不起的人.
沒有特別作為的人.
在家煮飯照顧孩子.
摘衣服.
有甚麼用.
不是的.
你試試在家裡沒有人洗廁所沒有人煮飯.
你試試看怎樣.
沒有人洗衣服會怎樣.
你家裡沒有人洗衣服會怎樣.
拿衣服出來反轉.
或者養兩養再穿.
不是的.
不行的.
你出去不見人.
身體會臭的.
靈姐妹.
一個人的生命能不能在這個時候得到迴轉.
準確看到焦點.
來到這段聖經.
馬太就要指出這一點.
用箴言第十八章二十四節做今天的結語.

$^{1081}$濫交朋友的自取敗壞.
但有一朋比弟兄更親密.
今天很多人以為濫交朋友.
搞大自己的網絡.
然後你就會成功.
我告訴你.
你搞大網絡.
可能你更加失敗得更快.
而且你的心靈會更加混亂.
最後結語的時候.
我講一句.
講一個見證給大家聽.
我認識一個人.
一個基督徒.
他說他信了主很多年.
不過他有很多事情想.
有很多計劃.
他經常說要做.
他的子女要栽培他.
他自己的事要這樣搞.
結果很多事情他都要插手下去.
結果有一天他病了.
他的病使他突然昏倒.
送到醫院的時候.
醫生查來查去.
身體沒有任何地方有任何問題.
他估計是他的腺體特別是甲狀腺出事.
醫生說甲狀腺的問題.
我們查到只能給你吃藥.
按壓一下.
於是他吃藥.
吃了一段時間.
仍然長期這樣昏倒.
他不明白.
結果有一天.
他開始看到一點.
什麼使我甲狀腺有問題呢.
遺傳?.
似乎不一定是遺傳.
不是每個人都有這樣的問題.

$^{1121}$於是他就在想.
什麼使我這樣.
吃不安 睡不著 長期失眠.
後來有一次遇到我.
我跟他說了一句話.
你是不是想得太多.
我跟他說完之後.
他突然醒了.
人如果沒有了上主為中心.
即使你自己叫基督徒.
沒有上主為中心.
想很多自己想做的事.
現代人叫做stress.
你為自己加添重擔.
背在身上的時候.
後來這個人怎樣好呢.
他改變了他真正的想法.
既然我很多事想了都做不到.
想做出力都做不到的時候.
為什麼我不留意我現在可以做什麼呢.
每一次人輕鬆了 病好了.
頂姐妹 我們基督徒有時都會掉落這種境況.
想得太多 想得太多.
還要in the name of God 去說上帝你快點給我.
上帝偏偏不給你.
想你在種種情況下.
想出誰是真正神的兒子.
聖經提醒我們.
馬太福音的讀者不是純粹想看耶穌怎樣死.
每個人都知道他怎樣死.
但馬太這樣的記載.
是馬太關心那些修讀馬太福音的人.
我不想你們死.
你找回上主 開除靈心低頭禱告.
天父多謝你讓我們在這段聖經裡學習.
讓我們看到馬太怎樣愛護他的讀者.
也看到耶穌昔日一群跟隨你的婦女.
她怎樣愛護上主耶穌基督呢.
很多時候我們今天跌入一個境況.
我們可能只是做一個掛名的基督徒.

$^{1161}$但我們的心其實只是走自己的道路.
環顧今天的教會當中有很多這樣的人被吹捧.
甚至以為自己是走在上主的道路裡.
但最後一次又一次的失望和失落裡.
反而傷害的是自己.
被困在罪的當中.
求主你的幫助讓我們能夠走出來.
能夠再次準確地向人生的標竿.
主耶穌基督靠著你的引導.
靠著你在我們生命周遭所安排的事情.
讓我們去偷竊認識你.
更委身於你裡面.
多謝主你給我們這段聖經.
禱告奉靠主耶穌基督得世命之教.
阿們.
多謝各位收聽.
\newpage



\section{馬太福音 27:57-61}
\label{sec:MfR5_HAo14I}
\textbf{沉實的門徒  (馬太福音27\_57-61) - 黃紹權牧師  [馬太福音信息系列 - 第148講]}
\newline
\newline
連結: \href{https://youtube.com/watch?v=MfR5_HAo14I}{\texttt{ https://youtube.com/watch?v=MfR5\_HAo14I}} ~~~~ 語音日期: 2025-02-13 
\newline
\newline
\hyperref[sec:7upP8JmD6zY]{< < < PREV SERMON < < <}
~
\hyperlink{toc}{[返主目錄]}
~
\hyperref[ch:preacher10]{[返講員目錄]}
~
\hyperref[sec:Y_0n0vkhyDU]{> > > NEXT SERMON > > >}
\newline
\newline
馬太福音 27:57-61
\newline
\begin{longtable}{cl}
\hline
\hline
章節 & 經文 (和合本修訂版)\\
\hline
27:57 & \begin{tabularx}{0.7\textwidth}{X} 到了晚上,有一個財主,名叫約瑟,是亞利馬太來的,他也是耶穌的門徒。 \end{tabularx} \\ \\ \relax
27:58 & \begin{tabularx}{0.7\textwidth}{X} 這人去見彼拉多,請求要耶穌的身體,彼拉多就吩咐給他。 \end{tabularx} \\ \\ \relax
27:59 & \begin{tabularx}{0.7\textwidth}{X} 約瑟取了身體,用乾淨的細麻布裹好, \end{tabularx} \\ \\ \relax
27:60 & \begin{tabularx}{0.7\textwidth}{X} 然後把他安放在自己的新墓穴裡,就是他鑿在巖石裡的。他又把大石頭滾到墓門口,然後離開。 \end{tabularx} \\ \\ \relax
27:61 & \begin{tabularx}{0.7\textwidth}{X} 有抹大拉的馬利亞和另一個馬利亞在那裡,對著墳墓坐著。 \end{tabularx} \\ \\
[1ex]
\hline
\hline
\end{longtable}
$^{1}$.
各位大家好,我是Tim.
剛才我們唱的那首詩歌叫做榮耀的一天.
英語的歌詞特別是副歌.
是這樣說的.
.
跟中文的歌詞有些很不同的地方.
中文的歌詞唱到好像主不救我們,不可以.
但當我們看回這首詩歌的歌詞的英文原著.
主活著是因為祂愛護我們.
主為我們死,祂的死亡是為了帶來我們的拯救.
祂的埋葬其實是將我們帶離罪的綑綁.
將整個歌詞的核心帶出來.
主的活著,死亡以及祂的埋葬.
讓我們看到我們是在上主的恩典裡.
不是主不救我們不行.
而是主提供了一個讓我們再活.
能夠離開罪惡以至於得救的境況.
這首詩歌也是我特意選擇來預備今天講道的時間.
因為今天所講的經文正正就是耶穌.
在祂死了之後所帶來的拯救或是人的迷失.
如何能夠走出來呢?.
讓我們聆聽主導演先和心不提頭討論.
慈悲的天父多謝你.
讓我們能夠在你的啟示當中明白你的救恩的意義.
讓我們更清楚明白主你的教導和你的作為.
從中讓我們帶著感恩的心再次領受這個救恩.
並且活在這個得救的道路上.
直奔永恆直奔永遠與主同住的盼望.
求你的帶領讓我們能夠一同在今天講道裡.
看你的說話讓我們抹開很多我們自己過往對於主你的說話的一些前設或偏見.
好讓我們歸回在主你的啟示當中.
禱告奉靠主耶穌基督得勝靈至求.
阿們.
我信主的年日是上世紀的八十年代.
在那個時候我記得有一個很大的人生經歷.
由香港來到加拿大開始步入在北美洲的生活.
我信主是1989年.
我還記得是4月26日.
當時為什麼這麼深刻呢.

$^{41}$因為自己一個人來到加拿大當中作為一個國際留學生.
在這裡開始讀書開始信主.
我師母邀請我回教會.
回到教會的時候就看到能夠認識上帝是一件美事.
同時我也發覺教會有很多東西令我很難理解.
甚至超出我的邏輯和慣性的思想.
為什麼呢.
我在1989年信主不久後.
到1991年開始讀大學.
1991年我由瓦斯特大學轉到多倫多大學.
由讀電腦科學轉讀會計學和環境規劃.
碰巧開始加拿大經濟衰退.
我記得在1991,92,93年畢業.
很多人都希望在外國讀完書.
可以在北美洲找到工作.
然後留在北美洲生活.
甚至可能在加拿大移民等等.
我一直沒有想過這件事.
但身邊很多人,特別在教會裡.
身為留學生,不少都經常討論這個問題.
怎樣可以留在加拿大.
最重要是找到工作.
如果大家在那段時間生活過在加拿大或北美洲.
就會發現那時代經濟衰退得很厲害.
每天由我學校宿舍走到校園.
大約十分鐘的路程.
有時五分鐘也有,因為課堂比較近我的宿舍.
走到去的時候經過很多報紙商.
現在已經沒有報紙商了.
現在還有沒有報紙賣呢?.
有沒有?我不知道,大家研究一下吧.
那時唯一可以知道資訊的就是報紙商.
報紙商在加拿大很特別.
在報紙商上輸入一元.
我忘記了,因為我通常都是用報紙來看.
例如去飯堂吃東西,有人丟了一份報紙就拿來看.
那時經過的報紙商,前面一定有今天的頭條新聞.
每天經過的報紙商,就算不買報紙也好.
你只會看到一個新聞.
就是今天哪間大企業有裁員多少千人.

$^{81}$你只能看到幾個字,就是Layoff和一個number.
然後下一個就是一間公司的名字.
在那期間,在教會當中我留意到.
很多大英姐妹突然間要蒙神呼召,要做傳道人.
所以當時可以說是教會一個傳職奉獻的熱門時間.
差不多每個星期都會有大英姐妹說.
我有上帝的感動,我要做傳職的傳道人.
因為上帝呼召我要做傳道人,我要準備入神學院.
但是真正要做一個傳職的傳道人.
在我對聖經裡面有一個很重要的因素.
也是我可以說是唯一一個去決定你是否做傳道人的.
就是你有沒有上帝的呼召.
你從所有被上帝所呼召作為一個傳主說話的人.
無論你可以看到昔日的阿伯拉罕或者是摩西.
至於後來早期的先知,好像薩普爾.
後來有一連串的時師,都真是有一大堆被稱為先知.
Navin的一班先知出來.
每一個都有一個很重要的因素.
由他開始那天到他進行傳主說話的過程當中.
必然有上帝呼召他去做.
所以這個就叫做CALL.
什麼叫呼召呢?.
呼召就是一個人被委任的意思.
在那時候有很多的弟兄姊妹在教會當中都說要做傳職的傳道人.
我就會有一個問題出來了.
為什麼有一個現象,這些弟兄姊妹要做傳職傳道人的絕大多數.
都是因為剛剛被裁員之後去做傳職傳道人呢?.
有些甚至這樣說,我被裁員就是因為上帝呼召我的原因.
我不覺得一定沒有可能上帝可以這樣做.
但是是不是每一個被裁員的.
就等同於上帝給了一個呼召他來做傳道人呢?.
我不想做一個批評.
不過後來當我在神學院的時候.
有一位很出名的老牧者.
可以說他已經去世好幾年了.
德高望重.
甚至可以說在多倫多的華人基督教會裡面.
沒有人景仰他.
於是就請他去神學院教書,教華人侍工.
當時我在華人侍工部不是讀華人侍工.

$^{121}$但是有很多同學都是讀華人侍工.
他們上完那堂之後就出來在大堂討論這句話.
大家有沒有聽到剛才那位牧者說.
他德高望重,我們沒理由不聽.
因為他是權威.
他就說教會是不是真的這麼多人需要全時間去侍奉呢?.
他認為不應該是每個基督徒都應該去奉獻做全職傳道人.
也不應該有這麼多.
他說如果這麼多人都來做全職傳道人的話.
教會還有誰賺錢回來奉獻給教會運作呢?.
他們說了這番話給我聽的時候.
我都很詫異.
已經這麼多人傳道.
又在說要去做全職傳道奉獻傳道.
再加上有些弟兄姊妹回來說.
因為老牧者都這樣說.
讓我想到基督教會的事公是和錢有直接關連掛勾嗎?.
原來個人沒了錢之後就等於奉獻.
等於呼召.
教會如果沒錢就沒人繼續供養運作不了教會.
教會就不用傳道了.
當然我自己明白一件事.
傳道這個功夫是每一個信主的基督徒.
不論有沒有奉獻或全職都應該做的事.
但如果連這麼不高望重的牧者都這樣宣揚.
教會是和錢直接掛勾.
而沒錢差不多可以做不了事的話.
我就會問原來教會是一個流行歌曲的歌詞.
No money no talk.
有沒有聽過.
No money no talk.
來吧.
有錢就有得來.
沒錢就沒得來.
是不是這樣呢.
我講了一大堆東西.
今天教會有很多信徒.
有很多東西勾綁著他們的思維.
認為教會只有一種模式.
一種信仰 一種教導 一種思維.

$^{161}$沒有了這種思維 單一的思想.
這就不是教會.
甚至有很多人認為.
信主的人一定要怎樣.
要很高調地傳主的福音.
甚至要很高調地表揚給其他人知道.
我是一個信徒.
這種思想其實是對上帝傳道的功夫的一種侮辱.
而今天我們去看這段聖經.
我相信馬太來到這裡.
是要向很多人去將我們去「刮醒」.
讓我們重新看一看信主的人.
真正信主的人的多樣化.
打破很多我們稱之為信仰的框框.
走出了框框不是代表不信主.
不代表不傳道.
而是讓我們重新看一看真真正正的實質信主的人.
他的生命是怎樣.
所以今天我和大家一起思想的這段經文.
馬太福音第27章第57至61節.
我給了一個標題 它叫做「沉實的門徒」.
有個英文標題叫做「Solid Disciple」.
我們一起再次打開這段聖經去讀一次.
不是很長的.
很短的經文.
很多經文越短.
弟兄姊妹們以為自己都知道.
但原來有很多東西我們在裡面是錯誤的.
請大家一起打開馬太福音第27章第57至61節.
我先拿一張大字聖經.
大神啊 到了晚上有一個財主名叫約瑟.
是阿里馬太來的 他也是耶穌的門徒.
這人去見彼拉多求耶穌的身體.
彼拉多吩咐給他 約瑟取了身體.
用乾淨細麻布裹好 安放在自己的新墳墓裡.
就是他坐在盤石裡的.
他又把大石滾到墓門口就去了.
有莫達拉的瑪利亞和那個瑪利亞在哪裡.
對著墳墓坐著.
在上文當我們去看之前我們所說到的經文.

$^{201}$耶穌釘死了在十字架上.
我們也說過這已經是完成了耶穌基督的預言的第二部分.
要進入第三部分的開始 叫復活的片段.
當我們聽到這群門徒.
縱使聽了耶穌三次的預言.
說耶穌在整個過程中被捆綁 被戲弄 被虐待 被釘死.
以至於復活的時候.
這群門徒個個都雞飛狗走.
但是卻剩下來的最後只有一群婦女仍然留守看著耶穌.
這段經文我們今天去看的時候.
很多時候信徒會很快跳過去.
並且他們會覺得是很順理成章的事.
但是當我們仔細想清楚的時候.
我們就會問 為什麼馬太在這個時候.
既有一個財主叫做約瑟呢.
第二 這個約瑟為什麼可以輕易走到比拉多的面前呢.
而且他走到比拉多的面前還可以要耶穌的身體.
而比拉多也很順利地把身體給了他.
你有沒有想過一連串的問題.
為什麼會這樣呢.
就算是死囚也好 要領他的身體.
都還沒有輪到一個我們都不知道是誰的人 叫做財主約瑟.
這個人很特別 從很遠的地方來的.
這個人從阿里米太來的 哪裡來的呢.
我們覺得好像很順理成章.
然後他就把耶穌的身體包好 跌正好.
然後放在自己墳墓的安放.
那個宮還可以把約瑟唱好.
基督教很多時候是這樣的信仰.
但是為什麼我們要馬太在這裡有一連串的介紹呢.
我們逐一去看一看.
而這個過程的記載也讓我們今天作為信徒.
應該打開很多我們固有的信仰的框框.
上個星期我談及到一個經文.
就是說耶穌被釘死在十字架上.
我們看到第54節.
有一個伯父長和他的同仇士兵.
看著耶穌的 看到地震.
然後有一連串的 怎樣打開等等.
他極其害怕 於是就說了.

$^{241}$「這真是神的兒子」.
有很多人很順理成章.
說這段聖經說的一定是看守著耶穌的兵精.
我不否定 這裡也寫明了.
這群人是看守著耶穌基督的人.
不過這個人在聖經裡.
是刻意馬太說是哪一位伯父長.
於是大家在上個星期聽到我說.
哪一位伯父長是指哪一位伯父長.
在馬太福音裡最有可能記載到.
應該是第八章提及的.
要救他家中有一個年輕的僕人.
甚至是他手下的一位伯父長.
於是有些人就討論到.
黃牧師你這樣說有沒有理據.
我理據就在聖經裡.
聖經說在馬太福音對上講的伯父長只有那位.
加上馬太福音的結構.
有一個所謂交叉式的結構.
有些人聽了很多年都認為.
這只是兵精其中一個.
然後他就在悔改認罪.
不能夠接受第八章所說的.
這不是錯的.
因為我們經常有很多傳統的思維綁死了我們.
像今天這段聖經.
我剛才一連串開始說到財主約瑟.
為什麼他可以來領耶穌的身體.
我們覺得很自然不過.
有人死有人領.
但他是不是配受呢.
而馬太寫這連串是什麼目的呢.
我們仔細看一看.
第一部分我們看第57節.
我們看到這位是一位很難得的信徒.
我再重申多一次.
這位是一位很難得的信徒.
英文我甚至給標題.
An Extraordinary Christian.
我們看看這位信徒有多難得.

$^{281}$第57節.
到了晚上.
有一個財主名叫約瑟.
是阿利米泰來的.
他也是耶穌的門徒.
精彩了.
在這段聖經裡面.
在馬太福音裡面經常說錢.
特別是有錢人.
我們雖然不是很有錢的人.
但至少我們是中產階級.
我們在加拿大來算.
馬太福音裡面說很多錢.
因為馬太本身就是一個稅吏.
稅吏最精明的是什麼呢.
就是錢.
收錢的.
每天都要用錢.
所以當誰有錢誰沒有錢.
他們信仰的狀態對於馬太來算是很敏感的.
而且他們也可以從這個角度裡面.
去否白一些我們一般信徒.
一直有些誤會.
譬如在馬太福音第十九章二十三節.
如果大家有聖經我們一起打開.
馬太福音第十九章二十三節裡面.
耶穌曾經怎樣對門徒講道.
財主的經歷是怎樣.
我們打開馬太福音.
第十九章二十三節.
我們一起讀讀.
馬太福音第十九章二十三節.
說到一個少年型的財主.
我們一起讀讀.
耶穌對門徒說.
我實在告訴你們.
財主進天國是難的.
我又告訴你們.
穿過針的眼比財主進神的國還容易呢.
我們讀到這裡為止.

$^{321}$在我們讀到聖經的時候.
我們就說.
耶穌告訴當時的門徒.
很清楚知道原來財主進入天國.
是很難的一件事.
難 是難.
難到進到一個什麼地步.
好像駱駝穿針眼.
真的不可能.
神學就在這裡說.
基督徒女不要有錢.
有錢是很大問題的.
馬太福音甚至說到.
你的財寶在那裡.
你的心也在那裡.
這個說話是什麼意思.
Follow the money.
這個世界人個個都跟著錢走.
加拿大也是這樣.
很多人來到加拿大.
帶著一大筆錢.
來到加拿大生活.
退休.
我不是啊.
我和你都不是.
我們要努力工作.
有不少是這樣的情況.
一聽到加什麼稅就很害怕.
害怕自己的財產走都走不及.
特別在上個星期.
當美國說要加加拿大25\%關稅的時候.
最多人提到的問題是什麼.
大家記不記得上個星期發生了什麼事.
星期二的時候.
美國準備加關稅的時候.
之前一天.
加拿大還在撐.
其他的報導.
說到加拿大的有錢人.
怎樣計劃將加拿大的財富移離加拿大.

$^{361}$怎樣保障自己不會受這些關稅影響而貶值.
上個星期二.
加拿大的roller coaster的一天.
直到下午四點鐘才知道.
明天不用加25\%.
我個人覺得.
美國應該加加拿大關稅.
讓加拿大的人不要再懶惰.
上帝要用加關稅這一招去割你的肉.
割醒你就好了.
財主是難進入天國.
不過不代表財主進不了天國.
在聖經裡我們馬上看到.
在耶穌死的那天晚上.
有一個財主叫做約瑟.
他有名有姓的.
弟兄姊妹.
他的名字叫做約瑟.
很明顯這是一個典型的猶太人.
他有錢.
但他不是住在耶路撒冷.
他住在阿里米太.
阿里米太是一個什麼地方呢.
如果大家知道阿里米太這個名字.
原本在希伯來文裡.
有一個地方叫做Rama.
是同一個地方.
這個地方我們可以看回史詩記的時候.
第四至第五節的時候.
我們知道原來在阿里米太的地方.
有一個女仙姿叫做底波拉.
她曾經坐在拉瑪和伯特利中間的一處中樹下.
向當時的以色列民作很多啟示和宣告.
我再重申一點.
我不是歧視女性.
不過在聖經裡每次用到女仙姿的時候.
當時以色列民當中的男人已經進入了很窩瓊的地步.
靈性已經跌到底了.
要找女性的女仙姿來提醒他.
馬太福音在這裡做這個活動.

$^{401}$記不記得馬太福音怎樣介紹.
當耶穌釘死了之後.
後來被埋葬.
是什麼人來接受上帝的啟示來喚醒門徒呢?.
牧師還沒說到.
還沒說到 之後就要來了.
很明顯我們知道是一群婦女.
所以為什麼馬太福音從第27章54節開始介紹這群婦女.
今天又介紹這群婦女.
馬太福音有一個脈絡在這裡寫.
是一群不行的門徒.
現在要找一群婦女來提醒他.
好像昔日底波拉怎樣在史詩的時代去提醒以色列民.
而這位財主約瑟典型猶太人.
從阿里米太 阿里米太是什麼地方呢?.
如果大家知道阿里米太不是耶路撒冷.
是耶路撒冷正正以北的一個大城市.
它是位於伯特利和耶路撒冷的正中間.
換句話來說 阿里米太是一個重要的補給站.
補給站.
而這位財主就在那裡居住.
他的發跡從呂仙芝坐在棕樹下.
我們可以簡單推論到.
他應該是靠種植棕樹類植物而發跡的.
我有機會去到以色列讀書.
全費讀書 讀了一個月.
我去到以色列的時候.
除了在修道院讀書之外.
大部分時間讀書.
有三四天我們就去到外面.
去一些我們稱為聖地的遺址去參觀.
在過程當中有很多人很著重看聖地.
或者我天生比較怪異.
看電影呢 我不知道你們有沒有興趣.
看電影我不是看主角在做什麼.
我是看主角後面的人在做什麼.
俗稱臨記 看他們做得好不好.
背景是很重要的.
做任何事如果只是看主角.
而看不到背景.

$^{441}$我們會完全喪失主角想做的帶來的訊息.
我雖然在旅程當中.
我經常坐在旅遊巴的後面.
看著外面的風景.
我看到到處都有椰子樹 棗樹.
很多很多種得很漂亮.
如果你去以色列旅行.
去死海旅行.
死海的海水不沉下去.
在死海附近一幅又一幅的棕樹林.
做得很漂亮 一排一排.
好像一個公園一樣漂亮.
原來在巴勒斯坦地.
種這些棕樹類植物.
是很容易得到果實的.
有椰子 有棗 有蜜棗 有桃.
有巴西莓 還有很多果實.
你們女士喜歡買死海的化妝品.
全部在那些地方種植的植物體現出來.
所以我有理由相信這位的約瑟.
是因為種植這些橡樹類植物而發家.
不過很奇妙的是.
在這段聖經裡.
卻記載了他不是十二位門徒之一.
但他卻是耶穌的門徒.
領之妹 你說可以這樣嗎.
信耶穌的人不是直接跟著耶穌的都叫門徒嗎.
信徒也好 或者現在再低級一點叫教徒也好.
這段聖經很強調他是耶穌基督的門徒.
即是說他是一個深信上主的人.
為什麼我會介紹他是一位很難得的信徒呢.
或者用英文說是一位非常奇妙的徒弟呢.
為什麼呢.
因為他打破了很多我們的想法.
馬太這麼著重看錢的人.
因為他做稅吏.
他有說過財主很難入天國.
但他在這裡記載了財主不僅是信主.
而且是一個門徒.
馬太福音記載.

$^{481}$如果你們去比較其他福音書的時候.
同樣記載了這位埋葬耶穌基督的藥室.
你就會發現一件事.
馬太是故意不介紹藥室的一個特殊身份.
這是什麼特殊身份呢.
他就是一個猶太人在耶路撒冷公會裡的其中一個會員.
換言之耶穌基督被判死刑.
被釘死在十字架上的那幫劊子手.
就是在耶路撒冷的猶太人公會裡.
這位藥室就是其中一員.
他不禁記載.
為什麼馬太福音不記載呢.
因為他故意不記載.
就是要我們不要留意到他有什麼特權.
他不是用特權在耶路撒冷裡面贖耶穌的身體.
反而在聖經要讓我們留意到.
他偏偏不是和這幫人同一夥的人.
他是和耶穌一夥的人.
因為他是耶穌基督的門徒.
而一個有錢.
我們在讀經上也知道他是一個特殊身份的人.
但他沒有用這個身份.
而他仍然在馬太裡面.
馬太高舉他是一個耶穌基督的門徒.
就集中讓我們留意到這個人的特色在哪裡.
他的信不是因為他有錢沒錢的問題.
他的參與不是因為他在哪一個權力之下.
而是因為他的身份是耶穌基督的門徒.
他的信帶領他走這條.
侍奉耶穌的路.
我一會兒看看他怎麼侍奉.
所以弟兄姐妹.
我們很多時候讀經有很多綁住我們.
有錢人很難入天國的.
所以引申一些奇奇怪怪的神婆.
不要那麼有錢.
不要做有錢人.
做有錢人信不了主.
我個人認識有些有錢的信徒.
為什麼他們做了信徒.

$^{521}$不是錢的問題.
因為他們沒有follow the money.
而是他們跟隨著耶穌.
follow Jesus Christ.
弟兄姐妹.
所以這一點就是他最珍貴的地方.
而我們作為一個信徒.
在這段聖經裡面去看.
如果你要作為一個全職侍奉主的人.
或者你是一個平信徒.
你不全職侍奉那些叫平信徒.
我不想用這個字眼.
不過很多信徒是這樣說.
我平的.
我很差的.
我信主很隨便的.
我不侍奉的.
你不推我.
我不讀聖經的.
你不提醒我.
我不會信主的.
你不叫我.
我不回教會.
弟兄姐妹.
這段聖經所說給我聽.
一個人能夠真正信主.
就是因為他願意毀身在耶穌基督裡面.
而且他侍奉.
不是靠著有個什麼身份.
不是耶穌是十二門徒之一.
也不是因為我在那裡有什麼影響力.
而是全心全意跟隨主的目的去做.
打開這一點的時候.
我們就去看看.
這位難得的信徒.
他有什麼侍奉.
他的侍奉有多沉實.
有多堅固.
我們去看第58至60節.
我們一起讀.

$^{561}$58至60節.
「這人去見彼拉多.
求耶穌的身體.
彼拉多就吩咐給他.
若失取了身體.
用乾淨細麻布裹好.
安放在自己的身份墓裡.
就是他作在盤石裡的.
他又把大石頭.
找到墓門口就去了」.
我做到這裡為止.
在這裡若失於是.
在耶穌將他的靈魂交給上帝之後.
很快在同一晚裡.
他就走到彼拉多裡.
去求提取耶穌的身體.
彼拉多又下令.
把耶穌的身體的屍體交給約瑟.
我們在這個時候.
應該就要問一個問題.
為何彼拉多會允許約瑟.
去取耶穌的身體呢.
但要排隊都未排到約瑟.
要排隊都不排到約瑟.
要排隊的都排到.
耶穌基督的媽媽或兄弟.
但我們在這裡有沒有見到.
耶穌基督的媽媽兄弟做過這件事.
我沒有見到.
他們不是不知道耶穌死了.
看著耶穌被釘死十字架上.
為何兄弟都不做這件事.
為何做媽媽的都不去做這件事.
而約瑟又憑甚麼理由.
可以說服彼拉多批准他呢.
還有一個在這些節期被釘死的人.
原來在聖經裡面是有規矩的.
在律法上很清楚記載.
要被釘死的囚犯不能夠過夜.
在哪裡我們知道.

$^{601}$新明記第21章第22至第23節.
我們打開看看.
新明記第21章第22至第23節.
找到之後我們一起讀.
人若犯該死的罪被致死了.
你將他掛在木頭上.
他的屍首不可留在木頭上過夜.
必要當日將他埋葬.
免得玷污了耶和華.
你們神所賜你遺孽之地.
因為被掛的人是在神面前受咒助的.
讀到這段聖經的時候.
我們馬上會有一個很大的詫異.
既然耶穌被釘死在十字架上.
他的兄弟姊妹不來領他的屍首埋葬.
我也算了.
當他不認識神的教訓.
或者他害怕不敢做.
或者他不知道流程.
不懂得找比拉多的官去拿屍體.
當這一切是這樣.
不過既然他們不知道也好.
有一群人一定知道.
就是耶路撒冷公會的祭司長和民事長老們.
為什麼呢.
因為這是律法的規定.
他們沒理由不做這件事.
因為你們一手一腳策劃釘死耶穌基督.
我們來到這裡就反映到馬太想說的另一件事.
馬太其實想告訴我們.
即使這群民事祭司法尼賽人長老祭司長.
整個猶太人公會策劃釘死耶穌基督.
他們從來不覺得自己有問題.
釘死不是我的事.
不關我的事.
這件事是比拉多釘死在十字架上.
或者比拉多把鑊卸給猶太人民眾在耶路撒冷的居民.
要執屍也是他們執的.
要守這條律法也是他們的.
但是律法的話事傳釋者.

$^{641}$當時在耶路撒冷的.
正正就是祭司長和民事長老的猶太人公會.
馬太在這裡清楚告訴我們.
這群人完全不覺得自己有問題.
因為他們覺得殺耶穌也不是他們的問題.
但實際上整個計劃就是由他們去草擬.
由他們一手一角去締造出來.
很可惜有群民眾被背了黑鑊.
由這群祭司卸了黑鑊給他們.
這群民眾又不懂得去處理.
但在這個時候.
有一個人叫做約瑟.
我仍然知道.
馬太沒有說約瑟是一個什麼背景的人.
但有一個人願意去領耶穌的屍體.
讓他不過夜.
不會掛在木頭上.
換句話來說.
約瑟在實踐聖經的教導.
一群本來應該熟悉聖經的人.
他不做.
要由另一個約瑟來做.
用今天的例子就是.
牧師不講清楚唐道給人聽.
並且不帶領信徒去執行.
要找一個其他的信徒來執行的時候.
比拉多看到這樣的狀況.
他就尋找了他.
尋找的其中一個原因.
我相信是連比拉多都知道.
有這條論例點像存在.
請大家留意.
比拉多不是一個渾渾噩噩地.
來管理耶路撒冷巴勒斯坦地的官員.
他是做足了解.
猶太通 巴勒斯坦通.
來這裡做的.
所以既然能夠有人來實踐猶太人的律法.
何樂而不為呢.
於是比拉多就在這樣的情況下.

$^{681}$允許了一個不是耶穌的親屬.
來處理司祭.
明白嗎 弟子姐妹.
不是一件偶然的事.
不是隨便做的事情.
不過亦都馬太在這裡.
介紹了另一個意思給我們知道.
連熟悉神話的人都不做.
反而外邦的巡撫比拉多.
他都知道的時候.
有人肯去做他都願意配合.
我們來到這裡.
我們開始就看見一件事.
馬太福音寫到這一點.
其實是向一班以為自己熟悉聖經.
甚至掌握聖經全息的人.
將他掩蓋了一大把.
這班是甚麼人呢.
外邦的巡撫.
以及另一個我們不知道.
不過肯定知道他的名字叫約瑟的人.
他們反而願意去做.
補了這班人的不足.
這班人羞不羞恥呢.
鄧小慕我再重申多一次.
在這幅圖畫馬太表達出來的是.
一班這麼有神學研究.
PhD postdoc博士級的猶太人工會領袖.
他們的神學竟然糟糕到這樣.
他們羞不羞恥呢.
馬太就想將這個羞不羞恥的訊息傳了出來.
告訴馬太福音的讀者知道.
不要以為自己讀過神學.
你就一定是靈性很好.
我在神學中見過很多神學生.
神學畢業生.
甚至今天我的所謂同工.
那些神學簡直錯到不堪.
連道德都沒有.
羞不羞恥呢.

$^{721}$羞恥是一種讓人迴轉的動力.
我重申多一次.
羞恥是一種讓人迴轉的能力.
仍然是上帝的一份祝福.
有些鄧小妹經常跟我說.
牧師 我覺得我犯了罪.
我好羞恥.
我怎樣好 怎樣算好.
真是好 上帝真的在祝福你.
你不覺得羞恥才可罷.
就是因為這份羞恥.
上帝帶你迴轉.
而這班祭司長 文士長老們.
有沒有羞恥.
往後的日子講道理.
你就會知道.
他們沒有一個覺得羞恥.
而且變本加厲.
約瑟做了什麼呢.
當他拿了耶穌基督的屍體之後.
這個身體.
他用乾淨的細麻布裹好.
然後安放在自己的新墳墓裡.
我重新重點說出這兩個重要的形容詞.
他用乾淨的細麻布裹好.
然後將耶穌的身體安放在自己的新墳墓裡.
這兩個字在聖經裡面很獨特.
乾淨這個字眼.
希臘文.
在聖經當中只用了三次.
這三次分別在五章第八節裡面的八福裡面.
講到清心的人.
清心就是這個字.
Katharos.
另外一次就是23章第26節裡面.
要指到發彌賽人.
要先潔淨自己的裡面.
潔淨就是這個字眼.
Katharos.
然後就是今天這一節.

$^{761}$馬太福音的寫作.
我重覆再說一次.
有一個叫做交叉對疊式的結構.
是什麼意思呢.
最開頭和最後的說話是有回應的.
第二部分和尾二說的話是有對稱的.
第三部分和尾三說話是有對稱的.
最後中間的部分也有對稱.
既然如此.
來到第27章尾部分.
對稱點就在第五章第八節.
清心的人有福了.
馬太在這裡很強調.
告訴我們.
這位的弱勢來做這件事.
潔淨耶穌基督的身體.
並且用乾淨的細麻布去裹好.
其實同時反映這個人.
他做這一連串的行動.
其實是沒有什麼異心的.
清心的人是什麼.
沒有異心的.
當做的事就做.
作為信徒.
以往的事錯了不要緊.
以後應做的事專心去做就行了.
這位有錢的弱勢就是這樣心智的人.
為什麼他有這種心智的人.
因為他就是八福裡面提到的被祝福的人.
Blessed的人.
Makarios的人.
是什麼人呢.
基督徒.
真正屬神的子民.
耶穌基督的門徒.
新的那日子在馬太福音裡面.
最近提及到新的.
就在第26章第29節.
那裡耶穌談論到的就是.
在新的那日子的時候.

$^{801}$約定和我們在當中喝聖宴.
聖餐經常提及.
不需要重複再說.
簡單地說.
這位弱勢其實是毫無保留.
沒有任何異心.
沒有任何異議.
沒有任何奇怪的動機.
不是說因為我有錢.
所以我就要操控猶太人教會.
或者以後基督徒要跟隨我.
我有錢.
他沒有.
他一條心.
我作為一個耶穌基督的門徒.
我就去埋葬耶穌基督.
將最好的放在上面.
包好它.
然後期待新的日子會來到.
所以他將它放在自己身上.
新的那份露天也不介意.
他用完就不新了.
耶穌用完就舊了.
You use the old tomb.
他不介意.
最重要是有上主的同在.
然後怎樣呢.
留意一下.
這個墓中是剛剛他用時間.
花金錢在那裡作石造出來.
很多弟兄姐妹不留意這一點.
為何弱勢可以這麼輕易.
在耶路撒冷裡面.
隨便找座山.
然後在那裡自己作墳墓呢.
你有沒有想過.
你讀了這段聖經很多年了.
作就作.
有錢就喜歡作.
有錢可以作.

$^{841}$你試一下.
耶路撒冷是誰的地頭.
耶路撒冷是猶太人公會的地頭.
你在他的地頭隨便作個墳墓.
有錢就行嗎.
大哂呀.
弟兄姐妹.
亞太在這裡不告訴我們.
這個弱勢的身份.
但也告訴我們.
這位弱勢的身份是非比尋常的.
他不需要告訴我們.
他是耶路撒冷猶太人公會的一份子.
但他能夠在耶路撒冷裡面.
一個這麼特別的地方.
不是一般人可以作到的墳墓.
記不記得被埋葬的加勒人猶太是怎樣的.
隨便找個兒戶的洞.
用三十塊銀錢買的.
在山頭荒山野嶺裡面作的.
原本是用來做陶具的地方.
是棄置的礦坑.
但這個弱勢不是.
而且他用自己的金錢去作.
作完之後怎樣.
成為一個新的墳墓.
原本是為自己去用的.
類似情況是怎樣的呢.
在香港我曾經認識一個朋友.
他的外祖母臨死前說要去教會受洗.
受洗後成為教會的會友.
死後便可以葬在香港的華人基督教永遠墳場裡.
原來香港華人基督教的墳場有這樣的規矩.
你和我不行的.
就算你死在香港也不能有我的想像.
你要是香港華人基督教聯會的會員堂的會友.
你才可以葬在那裡.
還要有位置.
你想想是多麼困難的事.
這個弱勢在這個獨特的地方裡.

$^{881}$有一個權利可以作自己的墳墓.
原本自用現在送給耶穌.
做完之後還把整個墳墓關上石門.
把石門滾動.
封住墓口.
然後聖經記載就離開了.
就離開了.
梅姐妹你剛才沒有聽錯.
他是負責實踐聖經第21章第22至23節的教導.
這些事應該由誰來做呢.
要不就是親人.
要不就是耶路撒冷公會的人.
這個財主大可以弱勢開一條單給他們.
收回錢也好.
新墳墓是我自己的.
有沒有開單呢.
聖經沒有記載.
沒有記載就沒有了.
你不要說沒有記載就有了.
我解釋不了.
如果沒有記載就有了.
我沒有記載.
甚至是誰允許他做呢.
比拉多.
他可不可以開一張單給比拉多.
比拉多現在我的新墓給了耶穌用了.
幫你搞定了一些手尾.
等於黑社會殺了人之後清理了現場.
收回安家費也應該吧.
我說沒有.
他就去了.
希臘文寫得很漂亮.
他就悄然離開.
一個真真正正侍奉主的人.
不是和上帝計較.
不是和世人計較.
而是看上帝給了你有什麼能力.
就用在上帝的事情上.
從這段聖經裡.
我們更加看到約瑟是一個不張揚的人.

$^{921}$他做了這一連串的事情.
他沒有大字標題說.
作天的大囚犯.
耶穌被釘死之後.
由大財主約瑟將他埋葬.
沒有啊 沒有出這些號外.
很低調地關上墓門就離開.
好像沒有人知道一樣.
但在這件事上是否真的沒有人知道呢.
我們真的要去看.
不過從這段聖經裡.
有法宗可以去想一件事.
今天的教會很多時候.
就像葉織一樣.
有多少人回教會.
招了多少人回來.
曾經有一間教會和我談.
黃牧師我們教會很欣賞你.
有幾個執事都很想你成為我們教會的牧者.
你來和我談一談吧.
談不談我無所謂.
看上帝是否有這樣的工作.
他談的時候.
我就問他.
我們很滿意.
每個人都很開心.
你講一下你有什麼工作.
我就這樣和他說.
我平日一星期差不多每天都開工.
不是像以前那樣.
每天都開工.
通常我預備主日學和港渡.
至少要花三至四天.
當我星期日不休息也好.
星期六也要休息.
我工作六天.
我用三至四天做港渡和主日學.
我覺得比一般牧者快很多.
你知道他怎樣回答我嗎.
他說如果是這樣我不能容許你這樣做.

$^{961}$如果你成為我們的牧者.
我不能讓你有三至四天.
預備港章和主日學.
我說為什麼.
他說要出去找多些人回來.
招攬多些人回教會.
讓更多人聽到你港渡.
你港渡這麼好.
我回心想想.
你不給我時間預備一堂好的港渡.
但你要我出去找些人回來聽我講好的港渡.
這兩件事好像自相矛盾.
然後我和他說.
我沒有三至四天去預備港渡和主日學.
我講不到一堂好的渡.
他說你要自己研究.
我的研究很簡單.
即時有的就是這間教會沒有上帝的呼召.
不用再談了.
各位先輩.
今天很多牧者為了滿足所謂KPI.
Key Performance Index.
來招攬人進入教會.
在教會裡面.
在馬來西亞沒有這件事.
你們來到應該感覺到.
在馬來西亞沒有人捉著你.
催逼你來教會.
你願意追隨主的.
很自然你會追隨主.
我不需要有多高調去唱.
教會是怎樣.
來吸引人回來.
真正要吸引的是屬神的子民.
不是一群來混吉的人.
我希望大家不是混吉的人.
一個這麼有錢的弱勢.
他完全可以高調.
但他沒有高調.
超然離開.

$^{1001}$一句聖經的說話都沒有記載他.
找不到.
馬太福音沒有.
這個就是馬太想告訴我們.
一個沉實的基督徒.
如何去侍奉.
我曾經和大家分享過一位牧者.
叫做D.A. Carson.
他爸爸叫Thomas Carson.
現在患了帕金遜病.
停止所有的侍奉.
很可惜.
但我相信上帝有他的美意.
他是一個很好的牧者.
他爸爸在Quebec侍奉在一間小教會.
侍奉到死的那天為止.
幾十年.
他從來不是要做一間大教會的牧者.
他只希望上帝的子民能夠得到做集人.
到最後一節.
一種信心的守望.
我們一起看吧.
我們一起讀吧.
第61節.
有沒大拉的瑪利亞和那瑪利亞在那裡對著墳墓坐著.
這段聖經來到這一節的時候.
其實就更加清楚讓我們看到一個結構.
原來27章第45至56節.
以及今天我們所看的第57至61節.
有一個共通的地方.
就是同樣以一群婦女作為結束.
而馬太福音第61章在這裡.
馬太就做上這個總結.
在這個總結裡再次提及到.
沒大拉瑪利亞和幾個瑪利亞.
他們這次是怎樣呢?.
他們在這裡同樣是看.
但看的是什麼呢?.
看清楚了嗎?.
是對著墳墓坐著.

$^{1041}$他們還在這裡守候.
不單是在這裡看.
而是守候.
有些人就說.
這群婦女因為女性比較感性.
所以他們不捨得耶穌死.
每天都還去耶穌基督的墳墓看.
我真的很欣賞這種心智.
我真的很欣賞你對你過去和你同甘共苦的人.
甚至相識的人.
你有這份情懷.
我真的很欣賞.
因為你沒有一種過橋抽板.
事過境遷的態度.
而是你讓我們看到.
你的心仍然在那個人身上.
縱使那個人死了.
那群婦女就是這樣的情況.
但更重要的是.
她在這裡守候.
坐在她旁邊.
做什麼呢?.
是做一個和當時馬太福音記載了.
十一個已經散開的門徒的最大比較.
後來我們會知道.
這群門徒離開了.
不想和耶穌有任何關係.
讓人知道.
是由這群女士看到一個異象.
然後被警察叫回這群門徒.
簡單來說.
同樣知道耶穌基督的預言的人.
不單是十二個門徒.
這群婦女也同樣知道.
但他們一直沒有表現出.
有什麼要做.
有什麼要看.
他們可以怎樣參與其中.
好像是弱勢的一群.
但這群絕對在靈性上是不弱的人.

$^{1081}$因為他們不想錯失任何一個機會.
當耶穌真的按他的說話.
進入第三步復活的時候.
他們不會錯過.
馬太福音就是要將這個結局告訴我們.
最後不錯過耶穌復活的人.
就是這群婦人.
回轉的人.
都是這群婦女的工作所締造出來的.
這就是馬太福音的主旨.
梁靜茹.
你有沒有一路守候在耶穌基督的裡面.
看他的復活.
面見這位主.
這份心志.
他們不單是哀哭.
也在等.
而他們的等不是白等.
不是一種盲目的等.
不是去掃墓那麼簡單.
而是一種期望.
我和你再會的期盼.
所以當你對自己死去的人.
你的親朋.
他如果信了主.
你帶著這份期盼與他再相會.
這份生命就是主放在你裡面.
他們看主耶穌基督曾經做的預言.
何時兌現.
要活出一個能夠見證上主耶穌說的話的真實.
先前已經有一個已經做了.
就是白夫長.
他說真是的.
這就是真理.
耶穌就是神的兒子.
他宣告了這個題目出來.
現在下一步就是這班女士.
要印證出這個真相.
聽著.
真正信主的人.

$^{1121}$沉實的信徒.
不是搞大鑼大鼓.
不是在教會搞KPI.
帶領多少人回教會.
有多少人能夠在今年和去年的次數.
回教會多了.
多少人在受浸.
不是這些數目.
而是你有沒有一份沉實的生命.
在主的裡面.
不輕忽地去侍奉主.
付出你能夠付出的.
去做吧.
教會大姐姐妹經常說.
教會的財政是怎樣搞的.
牧師你們也有十多人.
能夠和我同心.
一起將你有的能力.
一起付出去參與.
在這個全道的工作.
以馬來尼亞有多少人.
可以借助我們去得救.
這在上帝的手上.
但上帝讓我可以在他的說話上得到助紂.
我就盡力將他介紹出來.
弟兄姐妹.
你不需要一定要大鑼大鼓.
但上帝很肯定.
讓你在你的工作環境所處的空間裡轉動.
延除波濁流.
求主幫助我們.
即使你是一個財主.
不方便告訴別人你是一個信徒.
但他的行動好像是一位藥師.
他是一個不折不扣的耶穌基督的門徒.
今天我們在加拿大.
有些人說在政府機構工作.
有人在公營機構工作.
有人在銀行工作.
有人在私人公司工作.

$^{1161}$不能說耶穌.
不能說上帝.
歌也差不多不能說.
聖誕節也不准說聖誕快樂.
你是不是要妥協.
牧師我不妥協不行的.
我不妥協違反公司規例.
那你試試吧.
為什麼你不能做.
求主給我們勇氣.
關於財同心的就到了.
慈悲天父多謝你讓我們在聖經裡看到.
馬太的陳述記載.
讓我們明白一件事.
我們有很多規限.
固有的規矩.
例如教會要增長多少人才成長.
教會有多少人參加才是教會.
甚至在聖經的真理上.
我們只要向耶穌祈禱都可以.
但聖經沒有這樣介紹.
我們降伏在哪裡呢.
降伏在權威裡.
降伏在教會的規模裡.
還是降伏在上主的啟示裡.
求你的幫助.
藉著這位我們可以說在馬太福音的記載裡.
只有一個名字的財主約瑟.
他的侍奉與一群穆達拉瑪利亞.
與幾個瑪利亞的侍奉.
我們看到他在位置崗位上.
活出一個對主有信靠的生命.
來見證出他是一個不折不扣的信徒.
求主幫助我們.
同樣有這份智慧.
在當中活出有屬神的生命.
禱告奉靠主耶穌的德性的明智.
對不對.
我們現在來看看.
我們看到的這個圖.

$^{1201}$是一個圖像.
我們看到的這個圖像.
是一個圖像.
我們看到的這個圖像.
是一個圖像.
我們看到的這個圖像.
是一個圖像.
\newpage

\allsectionsfont{\centering}

\setlength\parindent{0pt}
\setlength{\columnsep}{1.25em}
\setlength{\parfillskip}{0pt}
\setlength{\tabcolsep}{1em}
\raggedbottom

\pagenumbering{gobble}


\newfontfamily\leftfont[Path=../fonts/fell_french_canon/, Ligatures=TeX, ItalicFont=IMFeFCit29C.otf, BoldFont=AveriaLibre-Bold.ttf]{IMFeFCrm29C.otf}
\newfontfamily\leftcitationfont[Path=../fonts/frankruehl/]{FrankRuehlCLM-Medium.ttf}
\newfontfamily\centerfont[Path=../fonts/garamond/, Ligatures=TeX, ItalicFont=EBGaramond-SemiBoldItalic.ttf]{EBGaramond-SemiBold.ttf}
\newfontfamily\rightfont[Path=../fonts/averia/, Ligatures=TeX, ItalicFont=AveriaLibre-RegularItalic.ttf, BoldFont=AveriaLibre-Bold.ttf, BoldItalicFont=AveriaLibre-BoldItalic.ttf]{AveriaLibre-Light.ttf}
\newfontfamily\rightcitationfont[Path=../fonts/rashi/]{Mekorot-Rashi.ttf}
\definecolor{hcolor}{HTML}{D3230C}
\definecolor{rcolor}{HTML}{D36F0C}
\newcommand{\chfont}[1]{\centerfont{\huge\textcolor{hcolor}{#1}}}
\newcommand{\leftcitation}[1]{\leftcitationfont{\Large\textcolor{hcolor}{#1}}}
\newcommand{\rightcitation}[1]{\rightcitationfont{\normalsize\textcolor{rcolor}{#1}}}
\newfontfamily\flowerfont[Path=../fonts/fell_flowers/]{IMFeFlow2.otf}

\begin{sloppypar}

\chapter*{\chfont{編按結語}}

\columnratio{0.5,0.5}\begin{paracol}{2}

\fontsize{11}{13}\leftfont \Large \leftcitation{א} \leftfont 余少好文.宏志博覽群書而不忘.善存藏經籍文獻備後時之用。\leftcitation{ב} \leftfont 歸主年時.受友所薦.聞道網海.\switchcolumn\fontsize{11}{13}\rightfont \Large \leftcitation{ח} \rightfont 有見粵道之危.國之封講道千言亦將就至.急之何則為?\leftcitation{ט} \rightfont 嘗聞猶太者之傳承.在其力守口述之

\end{paracol}


\columnratio{0.32,0.32,0.32}\begin{paracol}{3}

\fontsize{11}{13}\leftfont \Large 尤以吳約翰遜者 \switchcolumn[2]\fontsize{11}{13}\rightfont \Large 統.以煉千載不

\end{paracol}

\columnratio{0.32,0.32,0.32}
\begin{paracol}{3}\fontsize{11}{13}\leftfont \Large 為重.其載上之粵語講道緩緩入耳.收之藏其音頻.善妥整存.反復而嚼.受益無窮。\leftcitation{ג} \leftfont 我城我國既限.歷一四一九之不測.肺疫延年.信徒靈長屢受圍創.神州燈臺數盡指日可待.粵道之求與日俱增。\leftcitation{ד} \leftfont 觀乎社、經、法、媒、言、信、網之地.愈趨受鋤.自翔不果.授受壓力.粵道聖言亦愈漸艱難。\leftcitation{ה} \leftfont 況崇基例乎.學苑講道屢逆權勢者.其言末強受壓.舊章盡刪以存其身。講道釋數失傳.徒嘆奈何。

\switchcolumn

\fontsize{11}{13}\centerfont 
\begin{tikzpicture}
    \node (0,0) [xshift=-0.10cm, yshift=-1.0cm, opacity=0.10]{\includegraphics[width=0.30\textwidth]{../ot_frontcover.png}} ;
    \node (0,0) [xshift=+0.20cm, yshift=+2.0cm, opacity=0.10]{\includegraphics[width=0.20\textwidth]{../christ_on_cross.png}} ;
\end{tikzpicture}
\Large 

\leftcitation{ס} \centerfont 詩百又廿七載:
\leftcitation{ע} \centerfont 非耶和華建屋宇.則匠人之經營徒.
\leftcitation{פ} \centerfont 非耶和華衛城邑.則守者之儆醒徒.
\leftcitation{צ} \centerfont 余獻是卷予華人社區.願為福音流通之器.願獻斯微材為祭榮耀上帝.
\leftcitation{ק} \centerfont 阿門

\switchcolumn

\fontsize{11}{13}\rightfont \Large 滅.時越次聖殿期及當今。\leftcitation{י} \rightfont 猶太者力廣納之.筆錄以卷軸.便以傳、閱、頌、攜、守、鎖、抄、譯、釋、編,得書塔木德、密示拿等經傳.家喻戶曉.傳流若芳。\leftcitation{כ} \rightfont 猶太者文以載道.傳其口述.今我輩粵道之傳應當作如是.遂力行粵音識辨之法.載言載道.以盡忠傳粵道以待興。\leftcitation{ל} \rightfont 蒙下賜恩惠.無畏海量字音文書.既馭上帝之道.今廣及粵語講道.重駛編程之技.匯導粵音遂字稿.重塑講道現場.以傚猶太卷軸之舉便以傳流。\leftcitation{מ} \rightfont 是卷乃粵音口述傳之屬.莫通華文白話之語.

\end{paracol}

\columnratio{0.5,0.5}
\begin{paracol}{2}\fontsize{11}{13}\leftfont \Large \leftcitation{ו} \leftfont 斯殺一違儆百逆.既禁壓之.我輩聞風無奈.在所難免。\leftcitation{ז} \leftfont 另有異人例乎.以版權之名.脅網絡頻道之舉.同授礙予粵道之存流。

\switchcolumn

\fontsize{11}{13}\rightfont \Large 惟待後繼來者之傚.以譯釋傳之於神州華文地。\leftcitation{נ} \rightfont 今能排程驅馭圖靈以編彙文檔,其碼長共數千千亦無逢大礙.全蒙上帝保守。

\end{paracol}



\columnratio{1}\begin{paracol}{1}

\fontsize{11}{13}\rightfont \Large
~~~~~~~~~~~~~~~~~~~~~~~~~~~~~~~~~~~~~~~~~~~~~~~~~~~~~~~~~~~~~~~~~~~~~~~~~~~~~~~\leftcitation{ר} \rightfont 二零二三年二月一日

~~~~~~~~~~~~~~~~~~~~~~~~~~~~~~~~~~~~~~~~~~~~~~~~~~~~~~~~~~~~~~~~~~~~~~~~~~~~~~~\leftcitation{ש} \rightfont 米迦勒

~~~~~~~~~~~~~~~~~~~~~~~~~~~~~~~~~~~~~~~~~~~~~~~~~~~~~~~~~~~~~~~~~~~~~~~~~~~~~~~\leftcitation{ת} \rightfont 書於香港

\end{paracol}

\end{sloppypar}
\end{document}
