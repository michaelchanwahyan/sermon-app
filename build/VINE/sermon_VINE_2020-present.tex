\documentclass{book}
%\usepackage[letterpaper, portrait, margin=1cm]{geometry}
%\usepackage[letterpaper, bindingoffset=0.2in, left=1in,right=1in,top=.5in,bottom=.5in,footskip=.25in,marginparwidth=5em]{geometry}
\usepackage[letterpaper, left=1in,right=1in,top=.5in,bottom=.5in,footskip=.25in,marginparwidth=1cm]{geometry}
% ---------------------
% mini-table-of-content
% ---------------------
\usepackage{minitoc}
\setcounter{minitocdepth}{1}
\setlength{\mtcindent}{24pt}
\setcounter{secnumdepth}{-2}
%\renewcommand{\mtcfont}{\small\rm}
%\renewcommand{\mtcSfont}{\small\bf}
%\usepackage{setspace}
%\usepackage{tocloft}
%\setlength\cftparskip{-1.2pt}
%\setlength\cftbeforesecskip{1.3pt}
%\setlength\cftaftertoctitleskip{2pt}
%\renewcommand{\cftsecafterpnum}{\hspace*{02.0em}}
%\renewcommand{\cftsubsecafterpnum}{\hspace*{02.0em}}

% ---------------------------
% Chinese Characters Packages
% ---------------------------
\usepackage{fontspec} 
\usepackage{xeCJK}
\setmainfont{Times}
\setCJKmainfont{BiauKai}
\newfontfamily\sblgoodhebrew{SBL BibLit}[Script=Hebrew,Contextuals=Alternate]
\newfontfamily\sblgoodgreek{SBL BibLit}[Script=Greek,Contextuals=Alternate]

\usepackage{ifpdf,cite,algorithmic,url,tikz}
\usepackage[cmex10]{amsmath}

% ---------------------------
% Hebrew Characters Packages
% ---------------------------
\usepackage{polyglossia}
\setmainfont{Times New Roman}

% -------
% General
% -------
\usepackage{multicol}
\usepackage{multirow}
\usepackage{color,colortbl}
\usepackage{xparse}
\usepackage{pbox}
\usepackage{stackengine}
\usepackage{titlesec}% http://ctan.org/pkg/titlesec
\usepackage{tabularx}
\usepackage{xltabular}
\usepackage{titlesec}
\usepackage{makecell}
\newcommand{\sectionbreak}{\clearpage}

\author{
  Editor, Michael Chan\\
  \texttt{michaelchan\_wahyan@yahoo.com.hk}
}
\usepackage{tocloft}

\usepackage{hyperref}
\hypersetup{
    colorlinks=true, % set true if you want colored links
    linktoc   =all , % set to all if you want both sections and subsections linked
    linkcolor =blue, % choose some color if you want links to stand out
}

% ----------
% Afterword
% ----------
\usepackage{marginnote}
\usepackage{sectsty}
\usepackage{ragged2e}
\usepackage{lineno}
\usepackage{xcolor}
\usepackage{paracol}

\begin{document}

\clearpage
%% temporary titles
% command to provide stretchy vertical space in proportion
\newcommand\nbvspace[1][3]{\vspace*{\stretch{#1}}}
% allow some slack to avoid under/overfull boxes
\newcommand\nbstretchyspace{\spaceskip0.5em plus 0.25em minus 0.25em}
% To improve spacing on titlepages
\newcommand{\nbtitlestretch}{\spaceskip0.6em}
\pagestyle{empty}
\begin{center}
\bfseries
\nbvspace[1]
\Huge
{%\nbtitlestretch
\Large
\textbf{The Vine Church Sermon \\
       Youtube Channel: thevine hk, thevine yl
       }}

\nbvspace[1]

{\large
Editor: Michael\\
\texttt{michaelchan\_wahyan@yahoo.com.hk}
}

\nbvspace[1]

{\large
Revision: \texttt{v1.1}\\
Last Update: \today
}


\vfill
\begin{tikzpicture}
    %remove comment for OT cover%\node (0,0) [opacity=0.03]{\includegraphics[width=15cm]{../bible_out/ot_frontcover.png}} ;
    %remove comment for NT cover%\node (0,0) [opacity=0.03]{\includegraphics[width=15cm]{../bible_out/christ_on_cross.png}} ;
    %remove comment for Bible cover%\node (0,0) [xshift=0.8cm, yshift=+2cm, opacity=0.03]{\includegraphics[width=10cm]{./christ_on_cross.png}} ;
    %remove comment for Bible cover%\node (0,0) [              yshift=-2cm, opacity=0.03]{\includegraphics[width=14cm]{./ot_frontcover.png}} ;
\end{tikzpicture}
\vfill

\end{center}

\newpage

\setcounter{tocdepth}{0}
\dominitoc
\begin{multicols}{3}
\addtocontents{toc}{\protect\hypertarget{toc}{}}
\tableofcontents
\end{multicols}

\large
%\twocolumn

% the color definition syntax is as follow:
% \definecolor{name}{system}{definition}
% example: a mono-channel color can be defined as
%          \definecolor{Gray}{gray}{0.9}
% example: an rgb-3-channel color can be defined as
%          \definecolor{LightCyan}{rgb}{0.88,1,1}
%          \definecolor{pink}{rgb}{0.68,0,0.68}

\definecolor{CUV1LightRed}{rgb}{1,0.75,0.75}     % for CUV1
\definecolor{LZZVLightGray}{rgb}{0.9,0.9,0.9}    % for LZZ
\definecolor{KJVVLightGreen}{rgb}{0.75,1,0.85}   % for KJV
\definecolor{CUV2LightYellow}{rgb}{1,1,0.75}     % for CUV2
\definecolor{CNVVLightBrown}{rgb}{1,0.85,0.7}    % for CNV
\definecolor{NRSVLightBlue}{rgb}{0.75,1,1}       % for NRSV
\definecolor{WENLLightPurple}{rgb}{0.95,0.85,0.9}% for WENL
\definecolor{TCV19PaleGreen}{rgb}{0.85,1,0.95}   % for TCV19
\definecolor{MSGVLightWhite}{rgb}{0.98,0.98,0.98}% for MSGV
\definecolor{NETSLightRed}{rgb}{1,0.75,0.75}     % for NETS
\definecolor{JPS1917LightYellow}{rgb}{1,1,0.75}  % for JPS1917
\definecolor{SBLGNTPaleRed}{rgb}{1,0.85,0.80}    % for SBLGNT

\section{目錄}
\label{sec:index}
{ \scriptsize


\begin{xltabular}{\textwidth}{|p{0.75\textwidth}|p{0.07\textwidth} p{0.1\textwidth}|}
\hline
\hyperref[sec:uJkE_tmNC8Y]{2020-12-01 11AM Service Live: Advent 2020 “The Call” [uJkE-tmNC8Y].mp3} & 2020-12-01 & \href{https://youtube.com/watch?v=uJkE-tmNC8Y}{\texttt{ uJkE-tmNC8Y}} \\
\hyperref[sec:zigfpAW3XVI]{2020-12-07 Church Everywhere Live: Advent 2020 - Prepare the Way “The Desert” [zigfpAW3XVI].mp3} & 2020-12-07 & \href{https://youtube.com/watch?v=zigfpAW3XVI}{\texttt{ zigfpAW3XVI}} \\
\hyperref[sec:yAN1zxPhVpI]{2020-12-20 Church Everywhere Live: Advent 2020 - Prepare The Way - “The Release” [yAN1zxPhVpI].mp3} & 2020-12-20 & \href{https://youtube.com/watch?v=yAN1zxPhVpI}{\texttt{ yAN1zxPhVpI}} \\
\hyperref[sec:KS3UcRDETW4]{2020-12-27 Church Everywhere Live: New Years Message [KS3UcRDETW4].mp3} & 2020-12-27 & \href{https://youtube.com/watch?v=KS3UcRDETW4}{\texttt{ KS3UcRDETW4}} \\
\hyperref[sec:LHgF2voP5ys]{2021-01-11 Church Everywhere Live: Post Traumatic Growth - New Possibilities [LHgF2voP5ys].mp3} & 2021-01-11 & \href{https://youtube.com/watch?v=LHgF2voP5ys}{\texttt{ LHgF2voP5ys}} \\
\hyperref[sec:hd9p2DZoYtI]{2021-01-18 Church Everywhere Live: Post Traumatic Growth - Relationships With Others [hd9p2DZoYtI].mp3} & 2021-01-18 & \href{https://youtube.com/watch?v=hd9p2DZoYtI}{\texttt{ hd9p2DZoYtI}} \\
\hyperref[sec:d8XMdofd39c]{2021-01-25 Church Everywhere Live: Post Traumatic Growth - Appreciation of Life [d8XMdofd39c].mp3} & 2021-01-25 & \href{https://youtube.com/watch?v=d8XMdofd39c}{\texttt{ d8XMdofd39c}} \\
\hyperref[sec:bs6i66xQaTI]{2021-01-31 Church Everywhere Live: Post Traumatic Growth -  Personal Strength [bs6i66xQaTI].mp3} & 2021-01-31 & \href{https://youtube.com/watch?v=bs6i66xQaTI}{\texttt{ bs6i66xQaTI}} \\
\hyperref[sec:PMXFFRS8kEs]{2021-02-08 Church Everywhere Live: Post Traumatic Growth -  Spiritual Change [PMXFFRS8kEs].mp3} & 2021-02-08 & \href{https://youtube.com/watch?v=PMXFFRS8kEs}{\texttt{ PMXFFRS8kEs}} \\
\hyperref[sec:lzQ5tfRY0rI]{2021-02-14 Church Everywhere Live: Reunion \ and  Reconnect [lzQ5tfRY0rI].mp3} & 2021-02-14 & \href{https://youtube.com/watch?v=lzQ5tfRY0rI}{\texttt{ lzQ5tfRY0rI}} \\
\hyperref[sec:TWnjFhuOiys]{2021-02-22 Church Everywhere Live: Flourish - Scripture as One Story [TWnjFhuOiys].mp3} & 2021-02-22 & \href{https://youtube.com/watch?v=TWnjFhuOiys}{\texttt{ TWnjFhuOiys}} \\
\hyperref[sec:ODvUnLlR0pQ]{2021-03-08 Church Everywhere Live: Flourish - Shaped By Love [ODvUnLlR0pQ].mp3} & 2021-03-08 & \href{https://youtube.com/watch?v=ODvUnLlR0pQ}{\texttt{ ODvUnLlR0pQ}} \\
\hyperref[sec:DS7k2k5IRoo]{2021-03-15 Church Everywhere Live: Flourish - Refreshing Friendships [DS7k2k5IRoo].mp3} & 2021-03-15 & \href{https://youtube.com/watch?v=DS7k2k5IRoo}{\texttt{ DS7k2k5IRoo}} \\
\hyperref[sec:M1r4SUj2xRk]{2021-03-22 Church Everywhere Live: Flourish - Work as Worship [M1r4SUj2xRk].mp3} & 2021-03-22 & \href{https://youtube.com/watch?v=M1r4SUj2xRk}{\texttt{ M1r4SUj2xRk}} \\
\hyperref[sec:q7Cbfs_x0GI]{2021-03-30 Church Everywhere Live: Flourish - The Triumph of your Testimony [q7Cbfs\_x0GI].mp3} & 2021-03-30 & \href{https://youtube.com/watch?v=q7Cbfs_x0GI}{\texttt{ q7Cbfs\_x0GI}} \\
\hyperref[sec:P6yQmGrZNeA]{2021-04-19 11AM Service Live: A Different Spirit - Introduction [P6yQmGrZNeA].mp3} & 2021-04-19 & \href{https://youtube.com/watch?v=P6yQmGrZNeA}{\texttt{ P6yQmGrZNeA}} \\
\hyperref[sec:kH_AzSSAmbg]{2021-04-25 11AM Service Live: A Different Spirit - What do you see? [kH-AzSSAmbg].mp3} & 2021-04-25 & \href{https://youtube.com/watch?v=kH-AzSSAmbg}{\texttt{ kH-AzSSAmbg}} \\
\hyperref[sec:vtinqnnv5SU]{2021-06-22 11AM Service Live: World Refugee Day - Home Together [vtinqnnv5SU].mp3} & 2021-06-22 & \href{https://youtube.com/watch?v=vtinqnnv5SU}{\texttt{ vtinqnnv5SU}} \\
\hyperref[sec:q7HuGRvEWFw]{2021-06-29 11AM Service Live: From The Heart - The Gospel Afresh [q7HuGRvEWFw].mp3} & 2021-06-29 & \href{https://youtube.com/watch?v=q7HuGRvEWFw}{\texttt{ q7HuGRvEWFw}} \\
\hyperref[sec:yD3_LszW5Bw]{2021-07-07 11AM Service Live: From The Heart - Cry of My Heart [yD3\_LszW5Bw].mp3} & 2021-07-07 & \href{https://youtube.com/watch?v=yD3_LszW5Bw}{\texttt{ yD3\_LszW5Bw}} \\
\hyperref[sec:sQhUi8jUN2g]{2023-04-18 EXODUS - 01 In The Beginning [sQhUi8jUN2g].mp3} & 2023-04-18 & \href{https://youtube.com/watch?v=sQhUi8jUN2g}{\texttt{ sQhUi8jUN2g}} \\
\hyperref[sec:Bq85h5xoYT4]{2023-04-18 It starts here.. [Bq85h5xoYT4].mp3} & 2023-04-18 & \href{https://youtube.com/watch?v=Bq85h5xoYT4}{\texttt{ Bq85h5xoYT4}} \\
\hyperref[sec:cz4a6_st6To]{2023-04-24 EXODUS - 02 Midwives and the Power of No [cz4a6-st6To].mp3} & 2023-04-24 & \href{https://youtube.com/watch?v=cz4a6-st6To}{\texttt{ cz4a6-st6To}} \\
\hyperref[sec:LXiCHoxSu6Y]{2023-04-30 EXODUS - 03 Drawn Out [LXiCHoxSu6Y].mp3} & 2023-04-30 & \href{https://youtube.com/watch?v=LXiCHoxSu6Y}{\texttt{ LXiCHoxSu6Y}} \\
\hyperref[sec:gUw91uNiAAg]{2023-05-08 EXODUS - 04 Murder + Marriage [gUw91uNiAAg].mp3} & 2023-05-08 & \href{https://youtube.com/watch?v=gUw91uNiAAg}{\texttt{ gUw91uNiAAg}} \\
\hyperref[sec:1GHdbHuySwQ]{2023-05-15 EXODUS - 05 A God Who Hears and Remembers [1GHdbHuySwQ].mp3} & 2023-05-15 & \href{https://youtube.com/watch?v=1GHdbHuySwQ}{\texttt{ 1GHdbHuySwQ}} \\
\hyperref[sec:j3pdS8tSSoA]{2023-05-22 EXODUS - 06 The Burning Bush [j3pdS8tSSoA].mp3} & 2023-05-22 & \href{https://youtube.com/watch?v=j3pdS8tSSoA}{\texttt{ j3pdS8tSSoA}} \\
\hyperref[sec:GwiSx82TY4o]{2023-05-29 EXODUS - 07 The God of Freedom [GwiSx82TY4o].mp3} & 2023-05-29 & \href{https://youtube.com/watch?v=GwiSx82TY4o}{\texttt{ GwiSx82TY4o}} \\
\hyperref[sec:_gagT3D9Two]{2023-06-05 EXODUS - 08 What is in Your Hand? [\_gagT3D9Two].mp3} & 2023-06-05 & \href{https://youtube.com/watch?v=_gagT3D9Two}{\texttt{ \_gagT3D9Two}} \\
\hyperref[sec:pxaoLPqgKJE]{2023-06-12 EXODUS - 09 It Gets Worse Before It Gets Better [pxaoLPqgKJE].mp3} & 2023-06-12 & \href{https://youtube.com/watch?v=pxaoLPqgKJE}{\texttt{ pxaoLPqgKJE}} \\
\hyperref[sec:d2vwib6oxcU]{2023-06-26 EXODUS - 10 The Coming Of A Promise [d2vwib6oxcU].mp3} & 2023-06-26 & \href{https://youtube.com/watch?v=d2vwib6oxcU}{\texttt{ d2vwib6oxcU}} \\
\hyperref[sec:2hwTmUlFH_A]{2023-07-03 Conversations of Love [2hwTmUlFH-A].mp3} & 2023-07-03 & \href{https://youtube.com/watch?v=2hwTmUlFH-A}{\texttt{ 2hwTmUlFH-A}} \\
\hyperref[sec:OCz8LrBOC28]{2023-07-09 Conversations of Testimony [OCz8LrBOC28].mp3} & 2023-07-09 & \href{https://youtube.com/watch?v=OCz8LrBOC28}{\texttt{ OCz8LrBOC28}} \\
\hyperref[sec:Sc_q0QFc6ec]{2023-07-19 BOH Sunday - Seeds of Growth [Sc\_q0QFc6ec].mp3} & 2023-07-19 & \href{https://youtube.com/watch?v=Sc_q0QFc6ec}{\texttt{ Sc\_q0QFc6ec}} \\
\hyperref[sec:_QaFNgOx4_0]{2023-07-24 Conversations of Hope [\_QaFNgOx4-0].mp3} & 2023-07-24 & \href{https://youtube.com/watch?v=_QaFNgOx4-0}{\texttt{ \_QaFNgOx4-0}} \\
\hyperref[sec:n5lVginVaao]{2023-08-01 Conversations of Authority [n5lVginVaao].mp3} & 2023-08-01 & \href{https://youtube.com/watch?v=n5lVginVaao}{\texttt{ n5lVginVaao}} \\
\hyperref[sec:DHGUJmtAIQI]{2023-08-13 Conversations of Truth [DHGUJmtAIQI].mp3} & 2023-08-13 & \href{https://youtube.com/watch?v=DHGUJmtAIQI}{\texttt{ DHGUJmtAIQI}} \\
\hyperref[sec:6uo9Hz2W3WU]{2023-08-21 EXODUS - 11 Hard Hearts and Sacred Arts [6uo9Hz2W3WU].mp3} & 2023-08-21 & \href{https://youtube.com/watch?v=6uo9Hz2W3WU}{\texttt{ 6uo9Hz2W3WU}} \\
\hyperref[sec:OFYlvA1AkKg]{2023-09-03 EXODUS - 12 The Plagues [OFYlvA1AkKg].mp3} & 2023-09-03 & \href{https://youtube.com/watch?v=OFYlvA1AkKg}{\texttt{ OFYlvA1AkKg}} \\
\hyperref[sec:ezohhaWO5XQ]{2023-09-04 EXODUS - 13  The Passover [ezohhaWO5XQ].mp3} & 2023-09-04 & \href{https://youtube.com/watch?v=ezohhaWO5XQ}{\texttt{ ezohhaWO5XQ}} \\
\hyperref[sec:PHwKp7Ievsk]{2023-09-10 EXODUS - 14 And Also Bless Me [PHwKp7Ievsk].mp3} & 2023-09-10 & \href{https://youtube.com/watch?v=PHwKp7Ievsk}{\texttt{ PHwKp7Ievsk}} \\
\hyperref[sec:lsj62DXXjZ4]{2023-09-15 EXODUS - 15 He Can Part The Sea [lsj62DXXjZ4].mp3} & 2023-09-15 & \href{https://youtube.com/watch?v=lsj62DXXjZ4}{\texttt{ lsj62DXXjZ4}} \\
\hyperref[sec:JVjucZ_U4Bw]{2023-09-25 EXODUS - 16 Bread \ and  Water [JVjucZ\_U4Bw].mp3} & 2023-09-25 & \href{https://youtube.com/watch?v=JVjucZ_U4Bw}{\texttt{ JVjucZ\_U4Bw}} \\
\hyperref[sec:AG5PdzOFde8]{2023-10-03 EXODUS - 17 Generational Wisdom [AG5PdzOFde8].mp3} & 2023-10-03 & \href{https://youtube.com/watch?v=AG5PdzOFde8}{\texttt{ AG5PdzOFde8}} \\
\hyperref[sec:CKM0h5f9Oa4]{2023-10-16 Exodus - 18 The God of Yesterday, Today and Tomorrow [CKM0h5f9Oa4].mp3} & 2023-10-16 & \href{https://youtube.com/watch?v=CKM0h5f9Oa4}{\texttt{ CKM0h5f9Oa4}} \\
\hyperref[sec:FFFAigIMmRc]{2023-10-24 Exodus - 19 The Path of Life [FFFAigIMmRc].mp3} & 2023-10-24 & \href{https://youtube.com/watch?v=FFFAigIMmRc}{\texttt{ FFFAigIMmRc}} \\
\hyperref[sec:5UVcdfy_Atk]{2023-10-30 Exodus - 20 Falling Into Forgetfulness [5UVcdfy\_Atk].mp3} & 2023-10-30 & \href{https://youtube.com/watch?v=5UVcdfy_Atk}{\texttt{ 5UVcdfy\_Atk}} \\
\hyperref[sec:QuRsgZhXkqs]{2023-11-06 EXODUS - 21 Drawn In [QuRsgZhXkqs].mp3} & 2023-11-06 & \href{https://youtube.com/watch?v=QuRsgZhXkqs}{\texttt{ QuRsgZhXkqs}} \\
\hyperref[sec:eOywIjV2U2g]{2023-11-13 EXODUS 22 - Be Strong and Courageous [eOywIjV2U2g].mp3} & 2023-11-13 & \href{https://youtube.com/watch?v=eOywIjV2U2g}{\texttt{ eOywIjV2U2g}} \\
\hyperref[sec:O_UT_ubK9BE]{2023-11-20 God's Work in the World (Gary Huegen) [O\_UT-ubK9BE].mp3} & 2023-11-20 & \href{https://youtube.com/watch?v=O_UT-ubK9BE}{\texttt{ O\_UT-ubK9BE}} \\
\hyperref[sec:of7A7Q1wg7I]{2023-11-27 Exodus Finale 23 - The Death of Moses [of7A7Q1wg7I].mp3} & 2023-11-27 & \href{https://youtube.com/watch?v=of7A7Q1wg7I}{\texttt{ of7A7Q1wg7I}} \\
\hyperref[sec:lCE_pxD4_D4]{2023-12-03 Of Prophets and Prostitutes [lCE-pxD4-D4].mp3} & 2023-12-03 & \href{https://youtube.com/watch?v=lCE-pxD4-D4}{\texttt{ lCE-pxD4-D4}} \\
\hyperref[sec:wQE5SgaxsPo]{2023-12-11 Of Silence and Shame [wQE5SgaxsPo].mp3} & 2023-12-11 & \href{https://youtube.com/watch?v=wQE5SgaxsPo}{\texttt{ wQE5SgaxsPo}} \\
\hyperref[sec:UL6nYF5cG_Y]{2023-12-18 Of Relatives and Rejection [UL6nYF5cG-Y].mp3} & 2023-12-18 & \href{https://youtube.com/watch?v=UL6nYF5cG-Y}{\texttt{ UL6nYF5cG-Y}} \\
\hyperref[sec:Ini7uoDvO7A]{2023-12-25 聖誕節崇拜 | 廣東話崇拜 [Ini7uoDvO7A].mp3} & 2023-12-25 & \href{https://youtube.com/watch?v=Ini7uoDvO7A}{\texttt{ Ini7uoDvO7A}} \\
\hyperref[sec:9el2lQ77_AE]{2023-12-26 Christmas Message | English Service Live [9el2lQ77\_AE].mp3} & 2023-12-26 & \href{https://youtube.com/watch?v=9el2lQ77_AE}{\texttt{ 9el2lQ77\_AE}} \\
\hyperref[sec:t5Cfd_ii5kM]{2023-12-31 Of Signs and Swords [t5Cfd-ii5kM].mp3} & 2023-12-31 & \href{https://youtube.com/watch?v=t5Cfd-ii5kM}{\texttt{ t5Cfd-ii5kM}} \\
\hyperref[sec:eCZ2lV0ycAg]{2024-01-01 Press On [eCZ2lV0ycAg].mp3} & 2024-01-01 & \href{https://youtube.com/watch?v=eCZ2lV0ycAg}{\texttt{ eCZ2lV0ycAg}} \\
\hyperref[sec:xU27bdsfcJo]{2024-01-03 A Sign To You [xU27bdsfcJo].mp3} & 2024-01-03 & \href{https://youtube.com/watch?v=xU27bdsfcJo}{\texttt{ xU27bdsfcJo}} \\
\hyperref[sec:u3L5pvcvlOI]{2024-01-08 A Call To Intimacy: Introduction [u3L5pvcvlOI].mp3} & 2024-01-08 & \href{https://youtube.com/watch?v=u3L5pvcvlOI}{\texttt{ u3L5pvcvlOI}} \\
\hyperref[sec:fBOsothuSTk]{2024-01-15 What Stops Us: Rumors of God [fBOsothuSTk].mp3} & 2024-01-15 & \href{https://youtube.com/watch?v=fBOsothuSTk}{\texttt{ fBOsothuSTk}} \\
\hyperref[sec:VvUHDD1RwPI]{2024-01-22 What Stops Us: The Things of God [VvUHDD1RwPI].mp3} & 2024-01-22 & \href{https://youtube.com/watch?v=VvUHDD1RwPI}{\texttt{ VvUHDD1RwPI}} \\
\hyperref[sec:YW6pONcYjiU]{2024-01-29 What We Turn To: The World [YW6pONcYjiU].mp3} & 2024-01-29 & \href{https://youtube.com/watch?v=YW6pONcYjiU}{\texttt{ YW6pONcYjiU}} \\
\hyperref[sec:SARdI1NGpaQ]{2024-02-05 What We Turn To: Sex [SARdI1NGpaQ].mp3} & 2024-02-05 & \href{https://youtube.com/watch?v=SARdI1NGpaQ}{\texttt{ SARdI1NGpaQ}} \\
\hyperref[sec:m34r3U0C7b0]{2024-02-14 Putting On New Clothes [m34r3U0C7b0].mp3} & 2024-02-14 & \href{https://youtube.com/watch?v=m34r3U0C7b0}{\texttt{ m34r3U0C7b0}} \\
\hyperref[sec:plgL9V4r1NI]{2024-02-19 The Subversive Act of Generosity [plgL9V4r1NI].mp3} & 2024-02-19 & \href{https://youtube.com/watch?v=plgL9V4r1NI}{\texttt{ plgL9V4r1NI}} \\
\hyperref[sec:GMknFcIDczo]{2024-03-03 Vision Sunday 2024 [GMknFcIDczo].mp3} & 2024-03-03 & \href{https://youtube.com/watch?v=GMknFcIDczo}{\texttt{ GMknFcIDczo}} \\
\hyperref[sec:xe7hssHqQdQ]{2024-03-20 Conversations of Life [xe7hssHqQdQ].mp3} & 2024-03-20 & \href{https://youtube.com/watch?v=xe7hssHqQdQ}{\texttt{ xe7hssHqQdQ}} \\
\hyperref[sec:NdzV_39RUA4]{2024-03-25 Intimacy: Remain In Love [NdzV-39RUA4].mp3} & 2024-03-25 & \href{https://youtube.com/watch?v=NdzV-39RUA4}{\texttt{ NdzV-39RUA4}} \\
\hyperref[sec:bacTl9XHWhY]{2024-04-08 Ephesians: Grace \ and  Peace [bacTl9XHWhY].mp3} & 2024-04-08 & \href{https://youtube.com/watch?v=bacTl9XHWhY}{\texttt{ bacTl9XHWhY}} \\
\hyperref[sec:5M17mPHN4_U]{2024-04-15 Ephesians: The Will of God – Unity in Christ [5M17mPHN4\_U].mp3} & 2024-04-15 & \href{https://youtube.com/watch?v=5M17mPHN4_U}{\texttt{ 5M17mPHN4\_U}} \\
\hyperref[sec:PkcSALXT_yw]{2024-04-22 Ephesians: Incomparably Great Power [PkcSALXT\_yw].mp3} & 2024-04-22 & \href{https://youtube.com/watch?v=PkcSALXT_yw}{\texttt{ PkcSALXT\_yw}} \\
\hyperref[sec:AKMiqQYiQTY]{2024-04-29 Ephesians: The Work of Grace - God's Poem [AKMiqQYiQTY].mp3} & 2024-04-29 & \href{https://youtube.com/watch?v=AKMiqQYiQTY}{\texttt{ AKMiqQYiQTY}} \\
\hyperref[sec:Ps22iab4JXU]{2024-05-06 Ephesians: Making Peace [Ps22iab4JXU].mp3} & 2024-05-06 & \href{https://youtube.com/watch?v=Ps22iab4JXU}{\texttt{ Ps22iab4JXU}} \\
\end{xltabular}
}
\newpage



\section{}
\label{sec:uJkE_tmNC8Y}
\textbf{2020-12-01 11AM Service Live: Advent 2020 “The Call” [uJkE-tmNC8Y].mp3}
\newline
\newline
連結: \href{https://youtube.com/watch?v=uJkE-tmNC8Y}{\texttt{ https://youtube.com/watch?v=uJkE-tmNC8Y}} ~~~~ 語音日期: 2020-12-01 
\newline
\newline
\hyperref[sec:code]{\small{< < < PREV SERMON < < <}}
~
\hyperref[sec:index]{\small{[返主目錄]}}
~
\hyperref[sec:zigfpAW3XVI]{\small{> > > NEXT SERMON > > >}}
\newline
\newline
$^{1}$400 years of silence..
400 years of a people waiting in anticipation.
and expectation that God would speak,.
and yet 400 years of nothingness,.
a dryness, a prophetic wilderness, if you will..
400 years of the absence of something.
when something was so needed..
This 400 years of silence is the backdrop.
to the narrative of the New Testament,.
to the beginning of the story about the birth of Jesus..
This idea that there was 400 years.
since God had last spoken..
He had last spoken through his prophet Malachi..
400 years previous, Malachi had gotten the word of God,.
and Malachi had stood before all of God's people.
in that day and said, "Don't worry, a new time is coming.".
Malachi began to say, "Hey, a prophet will come.
"in the future, one who will look like Elijah,.
"and this prophet, he will bring God's word,.
"and it'll be like God will turn the hearts.
"of the children to the father,.
"and the hearts of the father to the children,.
"and there'll be this reconciliation and redemption.
"that God will move like he's never moved before.".
And for a group of people who had just come out.
of 70 years of exile in Babylon,.
who had come back into the promised land.
and had realized that their sin had taken them into exile,.
a group of people who were longing and realizing.
that redemption would never be found in themselves,.
but would only be found if God acted again..
Here's Malachi telling them that God is on the move,.
that God is about to bring a word.
that will literally change the hearts.
of everyone in this world..
You can almost sense at the end of the Old Testament.
that all of Israel is on its tippy toes,.
leaning forward in excitement and anticipation.
for God to speak..
And then this..

$^{41}$(crickets chirping).
Crickets..
Nothing for 400 years..
400 years of silence..
I don't know what you're like.
when you're waiting for God to speak..
I wonder how you are.
when your prayers don't seem to go answered..
I wonder what it's like for you.
when you wrestle with the reality.
that you've been asking God for something.
and it doesn't seem to be like God is present..
Today we start Advent..
And Advent is the idea of a season.
of four Sundays before Christmas..
Advent is the Latin word that means coming..
And literally what Advent is,.
it's a time to prepare ourselves for the coming of Jesus..
Note that I did not say,.
prepare ourselves for the coming of Christmas..
Because this is what we often turn Advent into.
in the global church..
We turn it into four weeks.
to be able to get ready for the celebration of Christmas Day..
We cover our buildings in Christmas decorations..
In fact, I wonder if you've been coming to the Vine.
for a while,.
whether you'll notice the distinct lack.
of Christmas decorations today..
Normally on this first Sunday of Advent,.
we cover this building in fairy lights.
and tinsels and Christmas trees..
We start to sing carols.
and we celebrate the reality.
that Jesus was born in a stable..
Now don't worry, we have not canceled Christmas..
COVID will not take Christmas from us..
Can I have an amen?.
(congregation cheering).
We will be open in Jesus name,.

$^{81}$prophetically for Christmas Day..
We'll see..
But Advent,.
Advent isn't about preparing for tinsels.
and Christmas trees and fairy lights..
Advent is about preparing your life for more of Jesus..
Advent sticks itself right in there.
and disrupts us in the most commercial of seasons..
Advent cries out for us to slow down..
Advent is like driving a wedge.
between the anticipation we have in our heart.
for all those great gifts on Christmas Day.
and the reality that our hearts might need to change..
Advent is disruptive and difficult,.
hard to grapple with..
I wonder if you could imagine what it would have been like.
for the Israelites in that moment,.
trying to grapple themselves with the reality.
that they so desperately want God to act.
and yet he seems to be so far away..
Advent gives us a chance.
to do some work on our hearts.
and to ask ourselves some deep and important questions.
even in the middle of the commercialism of Christmas..
Questions such as these..
Am I truly ready for Jesus to come again to this world?.
I mean, am I truly desiring for him to return?.
And with this, am I ready to hear what he has to say to me,.
to us?.
Am I ready to hear what God wants to say to me, to us?.
Israel was not ready to hear what God was about to say..
I mean, could you imagine it?.
400 years of silence, 400 years of waiting.
really can harden your heart, can't it?.
400 years of being then embedded.
in a most oppressive Greco-Roman empire..
By the time of the start of the narrative.
of the birth of Jesus in the New Testament,.
God's people had been waiting a long time.
and they had changed as a people..

$^{121}$In fact, in the opening page of the New Testament,.
what you see in Israel is a divided.
and fractured people of God..
There are essentially four types of people.
at the moment of Jesus's birth in Israel..
The first were called Essence..
And the Essence believed that the way to get God.
to speak again was to get themselves.
out of the ugliness of society,.
to retreat into the wilderness and simply pray..
And if they could pray enough, purify them enough.
outside of the horrors of society, God might speak again..
Then there were the Zealots..
The Zealots believed that the way that you get God to speak.
is you blow stuff up..
I mean, you fight the Romans, you bring revolution..
That revolution is the thing.
that will get God to say something again..
The Pharisees were embedded into the heart of society,.
but they believed that when God was to speak,.
it would come only if they could adhere.
to the Mosaic law as best as possible,.
because they realized that it was the absence of the law,.
it was disobedience to the law.
that brought them into exile in Babylon..
So now if they could just be as perfect.
with the law as possible, maybe God would speak again..
The Sadducees decided that the best way.
to get God to speak would be to integrate.
with the Greek Roman Empire..
I mean, this empire is here,.
it doesn't seem to be going anywhere..
If we could just partner with them,.
if we could come alongside of them,.
do some deals with them,.
maybe we'll get better freedoms on ourselves,.
and maybe if we're so nice to everybody,.
God might speak again..
I wonder what you're like.
when you're waiting for God to speak..

$^{161}$I wonder if you think the best way.
is just to remove yourself from everybody and pray..
I wonder if you think it's to fight..
I wonder if you think it's just.
to try to be a perfect Christian,.
maybe God will answer that prayer..
I wonder whether it might be some compromise in your life.
when you're waiting for God to speak..
Here's Israel, crying out, waiting,.
and here's the reality..
They were not prepared for Jesus..
They did not expect a pregnant virgin..
They did not expect the manger in a stable.
in a place called Bethlehem..
They had no idea what was ahead of them..
With the ministry of Jesus Christ,.
the healings He would do, the miracles He would do,.
they had no concept of the coming kingdom of God..
They could not understand the reality of a death on a cross.
and the profundity of a resurrection just three days later..
I mean, Israel was so steeped.
in the hardness of their hearts.
that they could not ever understand.
what was about to happen in the person of Jesus..
And so what Israel needed was a disruption..
What Israel needed was a change of heart..
They needed God to come and do something,.
to drive a wedge in them,.
to rip them out of their complacency.
and help them to begin to think again afresh on God..
What Israel needed is what I think.
we also need here in 2020, and that is Advent..
Advent sits for us in scripture around a particular person..
This person is the epitome of what Advent is all about..
His name is John the Baptist..
Now, I don't know what you think about.
when you think of John the Baptist..
It's probably not this image on the screen right now,.
although this is a beautiful painting.
of John the Baptist from church history..

$^{201}$But the John the Baptist you're probably aware of.
is the Sunday school version of John the Baptist,.
the one who lives in the desert,.
who has unkempt and long hair, the one who eats locusts..
I have done all three of those things in my life..
I just want you to know..
I've never done them at the same time,.
but I have done all three..
And here's John the Baptist,.
the most unlikely of sources who is about to step in.
and bring the disruption that Israel so needed..
And what we want to do here at the Vine.
over these four Sundays leading up to Christmas.
is we want to show you the life.
and the ministry of John the Baptist..
In fact, what we want to do is show you.
the four movements of his ministry..
John actually has four movements as he journeys Israel.
into a place of being prepared to receive Jesus..
And so as we think about what is it for us.
to be able to receive Jesus from his birth.
and also be excited about his return in the second coming,.
I think we need to follow these movements once again.
of John's ministry..
John issues a call to the people to respond..
He welcomes and invites them into the desert..
And in the desert place, he baptizes them..
And then he releases them as followers of Jesus..
A cool, a desert, the waters, the release..
The four movements of John's ministry.
will become the backdrop to these four Sundays for us.
as we get closer and closer to that moment.
where we will celebrate the arrival of Jesus..
But before we get there,.
may we find ourselves in his footsteps.
and may we step, step by step in getting prepared.
for the coming of Jesus..
Is everybody okay?.
So I wanna take you on this journey today.
with the call that he brings to Israel.

$^{241}$because I believe it is a call.
that resounds afresh for us today..
I'm gonna start by showing you how Luke presents.
the opening elements of John's ministry.
from Luke chapter three, starting in verse one..
By the way, I'm having so much fun right now..
I just want you to know, I really love you all..
And I really enjoy doing this..
And I'm so glad you're here..
And I'm so glad you're online..
We're one big happy family..
Amen?.
- Amen..
- Okay..
Luke chapter three, starting in verse one..
In the 15th year of the reign of Tiberius Caesar,.
when Pontius Pilate was governor of Judea,.
Herod, Tetrarch of Galilee,.
his brother Philip, Tetrarch of Etyria,.
and Trachytonus, and Lysnasus,.
I'm gonna murder all these words, by the way,.
Lynasus, Tetrarch of Abilene,.
during the high priesthood of Annas and Caiaphas,.
the word of God, after 400 years,.
came to John, son of Zechariah, in the desert..
He went into all the country around the Jordan,.
preaching a baptism of repentance.
for the forgiveness of sins..
For as it is written in the book of the words.
of Isaiah the prophet,.
a voice of one calling in the desert,.
prepare the way for the Lord,.
make straight paths for him..
Every valley should be filled in,.
every mountain and hill made low..
The crooked roads will be made straight,.
and the rough ways smooth,.
so that all people will see God's salvation..
Luke does a fascinating thing here.
when he wants to introduce the ministry of John the Baptist..

$^{281}$He starts by naming no less than seven rulers..
Now, Luke is a doctor and a historian,.
and he's passionate about making sure.
he puts all of his words into the context of history..
But normally what Luke would do.
is name maybe one leader,.
or one moment of that history.
to help you to understand the time.
that that event took place..
Nowhere in all of Luke's writings.
does he mention seven leaders back to back.
to help you to understand what Advent,.
what John the Baptist,.
and what this breaking in of God's word.
after 400 years is all about..
These seven rulers are important to understand..
Now, before we jump into that,.
I think if Luke was writing today,.
he would mention all the important rulers in our lives..
If you're here from China,.
maybe Xi Jinping would be at the top of the list..
If you're from the US,.
maybe it would be Donald Trump or Joe Biden,.
depending on whether it's December or January..
Maybe if you're from the UK,.
it would be Boris Johnson's name at the top of that list..
If you're a Hong Konger here,.
maybe Carrie Lam's name would be on top of that list..
Luke would put this moment into the context of its history,.
and he's doing so because he wants to teach something.
specifically about what Advent is all about..
So I wanna break down these seven people real quick for you.
to help you to understand what God's doing here..
Is that okay?.
So the first one is Tiberius Caesar..
This is the guy right at the top..
This is the guy who's right at the head.
of the whole Greco-Roman Empire..
That's where Luke starts..
And the interesting thing about Tiberius Caesar.

$^{321}$was he was one of the harshest, strongest,.
most oppressive Caesars that there was.
in all of the Greco-Roman Empire..
And Tiberius Caesar had done something..
He had made Pontius Pilate the governor over Judea..
In other words, the ruler over all of the Jewish people..
Now, this name, Pontius Pilate,.
will be relatively familiar to you.
if you know the Christmas story.
or if you know Jesus' story..
This is the guy who will eventually preside over the trial.
that Jesus is in at the end of his life..
This is the one who will eventually release Jesus.
to the crowds so that Jesus could be crucified..
Luke then tells us about two brothers..
He tells us about Herod, the Tetrarch of Galilee,.
and his brother Philip,.
the Tetrarch of Aetiria and Trachonitis..
Now, these two brothers are really important.
in the story of Jesus and John's life..
Herod here is not the Herod the Great.
who was there at the birth of Jesus,.
the one who decreed that every male Jewish son.
under the age of two needed to be killed,.
the one who caused Jesus and Joseph and Mary.
to flee to Egypt..
Not that Herod..
That Herod was the father of this Herod..
This is Herod Antipas..
And Herod Antipas is the one.
who would eventually arrest John and put him in prison..
This is the one who would eventually write.
the decree that would cause the beheading of John..
Now that happens because of the next guy on the list, Philip..
Philip is Herod's brother..
By the way, stick with it..
This is like "Game of Thrones" in a moment, okay?.
So just stick with this..
You've got Herod, you've got Philip, his brother..
Now, Philip's wife is Herodias..

$^{361}$Herodias has an affair with Herod Antipas,.
his brother's, yeah, whatever, yeah, with that guy..
They have an affair and it's not good..
And John the Baptist is willing to speak truth to power..
John the Baptist knows about this affair.
and he confronts Herod about it..
And he says, "What you're doing with Herodias is wrong..
"That's your brother's wife.".
And his willingness to speak truth to power.
ends up putting John in prison..
And Herod will be the one who would decree.
that he needs to be beheaded,.
that he needs to die for the reality.
that he was willing to speak up.
for what was wrong in the leadership around him..
Are you still with me?.
Now, Luke could have just stopped.
with the nasty, horrible Romans, but he doesn't..
He then mentions two Jewish leaders..
He mentions Annas and Caiaphas..
These were both high priests.
in the time of John and Jesus' ministry..
Both of them represent the power.
of the Jewish people at the time..
They represent the rich Jewish elite of the days..
And both Annas and Caiaphas had become incredibly wealthy.
off of the worship that was done in the temple..
They had personally pocketed that money..
Not only that,.
but they would eventually hold private trials against Jesus.
and they would be there mocking Jesus.
as he sent to the cross..
Now, Annas and Caiaphas started the marketplace.
in the temple where the poor people would come.
to sell whatever they could.
in order to buy a sacrifice worthy.
to take into the temple to repent of their sins..
In fact, that whole marketplace.
was known as Annas' Marketplace..
And Annas and Caiaphas made a lot of money.

$^{401}$out of the profits that the poor people.
were bringing to buy sacrifices..
This is the backdrop to when Jesus,.
in that last week of his life,.
triumphal entry into Jerusalem,.
goes straight to the temple..
He makes a whip of cords and he turns over the tables.
and he whips the people..
He spreads out and he busts out that marketplace.
because of the injustice that it stood for..
Are you with me?.
So Luke wants you to know,.
right at the story of Advent,.
here are all the leaders that were in place..
It's like Luke is laying out for us.
what the world's leadership looks like..
Broken, corrupt, oppressive,.
profiting off the poor and the vulnerable..
I mean, he just puts it straight out there..
All the ones that are gonna have an impact on the death,.
both of John the Baptist and Jesus..
It's like Luke is saying,.
this is the reality in which God begins to bring his word..
Look at where humanity has gotten to,.
all of the oppression, all of the injustice,.
all of the corruption,.
both with the Gentiles and with the Jews..
And into that, God brings his word..
Now, this is the other fascinating thing..
Notice in verse two..
During the high priesthood of Annas and Caiaphas,.
the word of God came to Tiberius Caesar..
No..
Sorry, the word of God came to Herod Antipas..
Okay, maybe the word of God.
would never have come to a Roman..
Sure, I get that..
So the word of God comes to Annas and Caiaphas,.
the high priest of the temple..
The word of God comes to John,.

$^{441}$son of Zechariah in the desert..
The word of God, in other words,.
having not spoken for 400 years,.
doesn't come into the corridors of power,.
but comes into the one that society had rejected..
That the word of God falls upon the very one.
who was living in the wilderness,.
living in the wasteland,.
who had been rejected by his own people..
The one who had the unkept hair.
and the eating of the locusts..
The one who had privilege himself.
because he was the son of a priest,.
but who had given up that privilege,.
turned it aside, reversed it,.
was willing to become a servant of his people.
because he had a word to bring them..
And in that place of wilderness, he would speak..
And what Luke wants you to see.
is something that's so profoundly important about Advent..
That true power and influence and leadership.
does not reside in the corridors of prestige.
and privilege and patronage,.
but in the less traveled pathways of humility.
and holiness and honor..
That's where the Lord's word comes..
And maybe that's some good news for all of us..
As we look at all the things.
that have been happening in the world this year,.
we look at the global reality that we have,.
it's a crisis of global leadership..
Maybe we shouldn't be looking.
for those in the corridors of power.
to speak a word of God to us..
Maybe that word resides in this room..
Maybe Advent tells us that God speaks to those.
that have been rejected, those that are not loved,.
those that struggle with the realities of life..
Advent is a welcomed gift to bring us closer to him..
Now, what is the message that John brings.

$^{481}$from the margins to the people?.
Well, it says here in verse three,.
it says, "He went into all the country.
around the Jordan, preaching a baptism of repentance.
for the forgiveness of sins.".
Stick with this church..
400 years of silence..
I don't know if I was God..
I don't think I would start with this topic..
Like if I was God, I would wanna come and go,.
yep, I've been silent for 400 years,.
but hey, I'm back and I wanna give you some hope..
I wanna give you some life..
I wanna tell you how awesome you are..
I wanna speak encouragement into you..
I wanna bring you into the flourishing of life..
Oh, I've got so much to say to you..
It's been 400 years, you're awesome..
Like that's what I would do..
What is it that God does?.
Here comes John from the margins..
Somebody, everybody had cast out and he was forgotten..
And he says, "Repent.".
He says, "There's sin in you,.
and that sin will block you.
from being able to see the arrival of Jesus..
That sin will cause you not to be able to see.
the beautiful kingdom of God.
that is about to be released in the Messiah..
That sin will hold you on the outside..
Repent.".
This word repent in the Greek is the word metanoia..
Metanoia means to turn around.
or it means to reorient yourself in a new direction..
I love that..
See, the repentance that John was bringing to his people.
was not to say sorry to God..
He wasn't coming to his people and saying,.
"Hey, just say sorry to God..
Just maybe feel a little bit bad about your sin.".

$^{521}$You know, or even he's not even just saying,.
"Confess your sin.".
And we've come as Christians to make those things.
what we think repentance is about..
All of those things are involved in repentance,.
but the sum of them is not what repentance is..
Because if you ask God to forgive you for your sin.
and make no effort to try to change your life.
away from that sin, in John's eyes, you've not repented..
'Cause repentance is a reorientation into another direction..
Repentance comes from the lips,.
but it is made manifest in the fist,.
in the hand, in the working out,.
in the walking out of our lives..
That's the repentance we're called to..
It's not for the faint of heart..
It means that we want to leave behind our life of sin..
It means that we wanna actually live in a different way..
And as Christmas comes in 2020,.
I wonder if the cry of Advent is relevant again for you..
I wonder if you could hear this word coming to you.
that maybe there's an option right now, a time, a period,.
whereas we long for the coming of Jesus again,.
that we make sure that there's nothing in our lives.
that would cause us to miss Him..
Maybe it's time for us to repent..
Not just to say sorry,.
but actually to live a different way..
But something I've noticed about this..
See, I've noticed that John the Baptist's call.
of this idea that a Messiah is coming.
who will put an end to all sin forever,.
who will judge injustice and unrighteousness.
and call it out for what it will.
and actually change it and turn it and put an end to it..
That message is not so frightening for the poor.
and the vulnerable and the oppressed of the earth.
as it is for those of us who have so much to lose..
Come on, church..
I put it to you that if you're super excited.

$^{561}$about that really nice present.
that your spouse is gonna give you for Christmas,.
that one that you've been asking about for ages.
and dropping hints and maybe writing things on the fridge.
and hoping that they might pay attention,.
that one that you've been talking about for so long.
and you know that they got the money and it's expensive..
Yes, it's expensive,.
but it will bring you so much joy and you're so excited.
that that little present's gonna be under the tree.
on Christmas day..
I put it to you that you're probably not that excited.
about the idea of praying for Jesus' return.
before Santa returns..
Or maybe bring it home this way..
Maybe you're in a business and you're running a business.
and your business is profiting.
off of some not so legal activities..
I don't think you're gonna be as excited to hear a message.
about repentance of sin if you're profiting.
from that very thing we think of as sin..
But what about this?.
What about the Christian who's in prison in China?.
What do you think the idea of repentance might mean for them?.
What about for that refugee or asylum seeker.
who's struggling to make ends meet in Yamate?.
What about for the one who's being oppressed or hurt.
or pulled down or discriminated.
because of the color of their skin.
or because of their gender?.
Maybe for them, the idea of a coming Messiah.
who will put all things right,.
maybe for them it's like honey on the lips..
The idea of the call of Advent, Maranatha, come Lord Jesus,.
might be the difference of life and death for them..
Maybe it's the poor and the vulnerable and the oppressed.
who need this the most..
I put it to you that those of us who struggle.
with the idea of repentance are probably more in love.
with what it is that we have to lose.

$^{601}$than it is to the one that we might lose it to..
Come on, church..
That the call of Advent might remind us.
that we need to put our lives right.
because He is coming again..
Because yeah, we look forward to that moment in the manger.
and the celebration of the reality.
of Him coming into this world..
But the scriptures also tell us.
that He's gonna return again..
And as He returns again, He's gonna put everything right..
All injustice is changed..
Everything's structurally moved..
The whole kingdom of God seen for the fullness of its glory..
And we stand in the middle.
between the first arrival in the manger.
and the coming again in the new Jerusalem..
And what is on our hearts?.
Are we crying out and saying,.
I wanna live in a way that honors the arrival.
of the second coming of Christ?.
Or am I going to play myself more towards the very thing.
that might cause me to miss Him completely?.
This is why Luke then gives us the idea.
of what this change of heart looks like,.
of what the renewal of Advent really is.
with this passage from Isaiah, verse four..
It says, "As it is written in the book.
"of the words of Isaiah the prophet,.
"a voice of one calling in the desert.".
This is John..
"Prepare the way for the Lord..
"Make straight paths for Him..
"Every valley shall be filled in..
"Every mountain will be made low..
"The crooked roads become straight..
"The rough ways made smooth..
"All people will see God's salvation.".
Isaiah crying out some 700 years.
before John's beginning of his ministry..

$^{641}$Foretells a time where one will come.
who will prepare the way..
Prepare the way because the people needed.
to be disrupted from the hardness of their hearts..
Prepare the way because the people needed.
to be changed to be able to see Jesus..
And this one will come and here's what he will do..
The way he will prepare the way.
is by making the path straight..
Making those crooked ways now straight again..
Removing the roughness from the path.
so that the King might come..
See, this imagery is important for Isaiah.
because this imagery is some bunch of years.
before the Greco-Roman Empire.
and all the amazing kind of avenues of infrastructure.
that they created..
In Isaiah's time, if a dignitary,.
if a king came to a town,.
they would send somebody in front of them.
to actually literally clear the path out the way.
'cause the roads weren't particularly good in those days..
And so they would literally take the stones,.
the big stones that might knock the chariot out of the way.
and they'd pick them up and throw them away..
If there were any ridges or kind of mounds like this,.
they would flatten down the mounds.
to make a straight, clear path.
so when the dignitary came in his chariot,.
he would be able to get to his destination.
where he was supposed to go..
That's the imagery in Isaiah's mind.
and John picks up on that imagery.
and he says, "This is my calling..
"This is my ministry..
"This is what I'm here to do..
"I'm here to make those crooked paths straight,.
"to take those stones away,.
"not from the physical roads anymore,.
"but from the crookedness,.

$^{681}$"the hardness of our hearts.".
And this really is John's call..
This is what Advent is all about..
You see, Advent is a call to personal renewal..
It's a call for you to examine the paths of your heart.
that lead to Jesus..
It's a call to you to take a look.
at what injustices sit within you,.
of what sin is there that you've wrestled with recently..
It's a call for you, an invitation for you.
to make the crooked paths of your heart straight again,.
to be able to remove whatever obstacles might be there.
for receiving more and more of Jesus at this Christmas time.
so that you can come and celebrate him.
with a refresh renewal of your heart..
Advent is a call for you.
to make the paths of your life straight..
Are you with me?.
Now, I want you to do that,.
not by sitting here and listening to this message today..
I want you to do that.
by putting that into practice in your life..
I want you to take what we're talking about today.
and actually reflect in your own spiritual disciplines.
throughout the week on exactly what we're talking about..
So here's what I want you to do right now..
I want you to get your camera out, your phone out,.
which it has a camera, most of them..
So get your phone out..
And I'm gonna show two slides here.
that I want you to take a photo of..
The reason being is I want you to sit.
with these questions in the week ahead..
I want you to reflect in your own personal time.
around these challenging questions..
I'll let you take slide one..
Five, four, three, two, one, slide two..
This is Advent for me, Advent for you..
It's about the renewal of your person..
And I believe as you sit with those questions this week,.

$^{721}$the Holy Spirit will challenge you.
about putting some of those crooked ways.
in your life straight..
Is that helpful?.
It's helpful for promise..
Is that helpful for anyone else?.
All right, now here's the important thing though..
Don't ever make Advent only about you..
Advent is not just about your own personal renewal..
When John the Baptist began to preach.
a message of repentance for the forgiveness of sin,.
it was not just directed at personal renewal..
He was also standing before all of the structures.
that Luke had laid out, both the Roman.
and the Jewish structures, and all of the injustice.
and the corruption and all the stuff that was happening..
And he was standing before that and he was saying,.
God's kingdom is going to reverse so much.
of what we see here..
In other words, there is social reversal.
that the kingdom of God brings..
See, Advent is also about social reversal..
It's not just about your own personal renewal..
As important as that is, that's the first step, absolutely..
But after that, then the cool comes..
What is it that's broken around us?.
What are the systems and the structures.
that are keeping the people impressed?.
What are some of the things that we see around us.
that as Christians we can stand up against?.
We can speak truth to power..
Truth to power like John the Baptist does.
in front of Herod that even cost him his head..
Where might we be able to speak truth to power.
and act in a proper righteousness.
to stand against injustice like Jesus does in the temple.
with those whips and the cords and the overturning.
of all of the profit that was being made from the poor?.
Where might we be able to partner with God this Advent.
in standing for this social reversal.

$^{761}$that we see so often in God's kingdom?.
This theme is in the whole of the Christmas story..
Let me just give you one example from Luke 1,.
starting in verse 51..
This is Mary's song, pregnant with Jesus..
And she turns before God, filled with the Holy Spirit..
And she sings these words..
She says, "He has performed mighty deeds with His arm..
"He has scattered those who are proud.
"in their inmost thoughts..
"He has brought down rulers from their thrones.".
Notice the reversal..
"But He has lifted up the humble..
"He has filled the hungry with good things,.
"but He sent away the rich empty.".
The social reversal that takes place.
in the coming of the kingdom of God..
This is prefiguring Jesus there in the Sermon of the Mount.
with the Beatitudes..
"Blessed are those who mourn, for they will be comforted..
"Blessed are those who hunger and thirst for righteousness,.
"for they will be filled.".
The kingdom of God that we get invited into,.
the story of Christmas, what Advent is all about,.
is not just a change within me,.
but a change within me for the change in the world..
That actually Christ comes to change the world.
to the kingdom of God..
It's subversive and it's powerful,.
and it doesn't shout from the corridors of power,.
but it comes from the broken and the vulnerable,.
and it turns everything upside down..
So get your phones one more time,.
and I want you to take a photo of these two slides,.
'cause this is Advent for us, the reversal..
I want you to reflect this week in your quiet time.
and in your own personal way on this question,.
five, four, three, two, one, and this one..
I want you to sit with these questions,.
because here at the Vine,.

$^{801}$God's speaking is not 90 minutes on a Sunday, amen?.
But what we're talking about here is important,.
and we wanna invite you to take your time.
to sit in the reality of the call of Advent,.
the call of John the Baptist, renewal and reversal..
That actually every act of God's redemptive power.
and activity in this world sits under these two things.
of renewal and reversal, renewal and reversal..
I believe this Christmas, God wants to renew your heart,.
and I believe He wants to show you.
the things that you could be praying for.
to see reversal come in our city and our society..
This is the heart of God..
The great question that Advent leads us to.
is are we willing to clear the paths?.
Are we actually willing to make the way straight?.
John the Baptist stood before a people.
who hadn't heard God for 400 years..
I stand before you today,.
not knowing when the last time God spoke to you..
And I issue a call to you this Christmas..
Repent..
By that I don't mean just say sorry..
Live a different way..
I wonder if you could stand with me, I wanna pray..
(footsteps).
If you're comfortable to, you can open your hands,.
you don't need to, but I'm gonna pray for us..
Father, we thank you that we stand in this moment.
in the call of Advent as we prepare our hearts.
for your arrival at Christmas in just a few weeks' time..
But in the bigger picture of things,.
you're coming again in your second coming..
And as we stand between those two world-changing events,.
forgive us, Lord, where we love what it is.
that we might lose more than the one.
that we are to lose it to..
Forgive us, Lord, where we've hardened our hearts.
off the back of such a hard year..
Forgive us, Lord, where our waiting for your voice.

$^{841}$in our unanswered prayers has challenged us.
to give up on you..
Father, we stand here in this moment.
as a community of faith..
We thank you for your presence and your word..
We thank you that after 400 years, you brought a word.
that was not easy to hear and not easy to swallow..
And we stand together now as a community of faith.
with that same word sitting over us..
We need that disruption of Advent as well..
All of us can think of areas in our lives.
that are crooked and not straight,.
areas of our lives that are waiting to be invited.
by your Holy Spirit to you,.
so that yes, we can ask for your forgiveness..
Yes, we can confess that sin,.
but also that we might begin to live in a new way,.
that we would turn around,.
reorient ourselves in a new direction..
Holy Spirit, I pray for anyone in this room.
where that is the cry of their hearts today,.
that over this week and the next coming weeks.
as we unpack the life of John the Baptist,.
they would feel themselves coming alive again,.
that they would hear the call and they would respond..
And we thank you for this in your name..
\newpage



\section{}
\label{sec:zigfpAW3XVI}
\textbf{2020-12-07 Church Everywhere Live: Advent 2020 - Prepare the Way “The Desert” [zigfpAW3XVI].mp3}
\newline
\newline
連結: \href{https://youtube.com/watch?v=zigfpAW3XVI}{\texttt{ https://youtube.com/watch?v=zigfpAW3XVI}} ~~~~ 語音日期: 2020-12-07 
\newline
\newline
\hyperref[sec:uJkE_tmNC8Y]{\small{< < < PREV SERMON < < <}}
~
\hyperref[sec:index]{\small{[返主目錄]}}
~
\hyperref[sec:yAN1zxPhVpI]{\small{> > > NEXT SERMON > > >}}
\newline
\newline
$^{1}$So if I was to ask you in your life,.
when you have felt closest to God,.
when you felt the most profound encounter with his presence,.
I wonder today what you would say to me..
Would you tell me about those best moments of your life?.
Those moments where everything was going great,.
where life seemed so amazing, where all the stars aligned,.
where everything was perfect for you?.
Or I wonder whether you would tell me about the hard times..
You would tell me about the times.
where you actually realized.
that you had come to an end in yourself..
The times where you were so desperately in need of God..
The times where you were literally.
in a situation of life and death,.
that if God didn't step in,.
things would disrupt and fail for you..
I wonder if it would be in the hardest moments.
that you have actually felt the closest to him..
As I was reflecting on that question for myself this week,.
I was surprised by my answer..
As I look back over this first 10 years.
of my time on staff here at the Vine,.
I know that the times that God has been closest to me,.
the times where he's manifested his presence most to me,.
have been in the hardest moments of ministry.
and the times where my skills and my talents and my gifts.
were not adequate for that moment.
that I was facing in ministry..
I remember the time where God's presence came upon me.
in the power of his peace,.
when I walked into a room to a crying mother.
who was holding her son.
who had just passed away in her arms..
I remember feeling this incredible gift of words of wisdom.
when I was sitting with a family.
who were dealing with the travesty of the reality.
that their daughter had just passed away..
I remember this moment.
where I was actually woken up at three in the morning.

$^{41}$by a phone call from one of our congregation members.
and they had just discovered.
that their spouse had cheated on them..
And I felt at the time that I had nothing to offer..
And then a gift of the word of hope came upon me..
I was to be able to share a scripture.
I hadn't thought of for years with her.
that really brought to life something in her spirit..
I remember the strength,.
the supernatural strength that came upon me.
when I sat next to my father at his deathbed in the hospital.
when cancer was taking his life from him..
I remember that moment of God's intimacy with me.
when I stepped up that mountain in Santa Fe,.
when I was at such war with myself and with him..
And I walked into this moment of fog on the mountain.
and I heard God speak to me again..
Yeah, it's in the hardest times of my life..
It's in the times where I felt least adequate,.
least with the amount of ability.
where I desperately needed God to speak.
or I've actually encountered his presence the closest..
God has been most real when I have been most barren..
- And we're stepping now into our second week of Advent.
and we're continuing our series.
on the ministry of John the Baptist..
And we move today from the call of repentance from last week.
to the place of repentance this week, the desert..
And when we think about John's ministry,.
it might surprise us that after 400 years of silence,.
that God would decide to speak again.
in a place like the wilderness, in a place like the desert..
It seems surprising and strange to us.
that that's where God's word would come from,.
the Judean wilderness, a place of hostility,.
of dark expansion, a place with really nothingness,.
of hills and valleys, of rocks and sand,.
that that would be the place that God would choose.
to manifest his presence again with his people..
Last week, we looked at the story of John the Baptist.

$^{81}$from the gospel of Luke..
Today, I wanna open up at you that story.
from the gospel of Matthew..
And I wanna show you particularly how Matthew introduces us.
to this place of repentance.
and the role that the desert plays.
in our understanding of Advent.
as well as our preparation for Jesus..
So this is from Matthew chapter three..
I'm gonna start in verse one..
In those days, John the Baptist came preaching.
in the desert of Judea and saying,.
"Repent, for the kingdom of heaven is near.".
This is he who was spoken of through the prophet Isaiah,.
a voice of one calling in the desert..
"Prepare the way for the Lord,.
make straight paths for him.".
John's clothes were made of camel's hair.
and he had a leather belt around his waist..
His food was locusts and wild honey..
People went out to him from Jerusalem and all of Judea,.
the whole region of the Jordan, confessing their sins,.
they were baptized by him in the Jordan River..
But when he saw many of the Pharisees and the Sadducees.
coming to where he was baptizing,.
he said, "You brood of vipers,.
who warned you to flee from the coming wrath,.
produce fruit in keeping with repentance..
And do not think that you can say to yourselves,.
we have Abraham as our father..
Let me tell you that even out of these stones,.
God can raise up children of Abraham..
For the ax is already at the root of the trees.
and every tree that does not produce fruit.
will be cut down and thrown into the fire.".
Last week, I shared with us this moment.
where Luke opens up his account of John the Baptist.
and Luke does this weird thing..
He mentioned seven political and religious leaders.
'cause he wanted to say something important.

$^{121}$about what Advent meant for the people..
Well, Matthew in a very similar way.
does something similar here..
He actually mentions the place of John's ministry.
three times, he repeats it three times..
Right there at the start, he says,.
he comes preaching in the desert place..
Just a little bit on, he says,.
he's a voice of one calling from the desert..
Further on, he says that all the people.
would come out from where they were and come out to him,.
inferring that they were coming out to him in the desert..
Matthew wants you to know.
that one of the core characters of Advent.
and of John's ministry is this character of the desert,.
of the wilderness..
He wants us to understand why the wilderness was important,.
why Israel had to come out of the comfort of their cities,.
the comfort of where they lived.
and had to move to a place of great barrenness.
in the Judean wilderness..
A few years ago, I had the great privilege actually.
of traveling to the Holy Land..
I spent a couple of weeks actually in Israel.
and in Palestine..
I journeyed through all of the area of Galilee..
And one place that I really wanted to go to.
and spend some time in was the Judean wilderness..
I wanted to put myself in that place.
and experience a little bit of where it was.
that John the Baptist did his ministry..
I wanted to put myself in the place.
where Jesus would be called by the Holy Spirit to go to,.
to be tempted by the enemy for 40 days and 40 nights.
in the same way that Israel had been brought out of Egypt.
into a 40 year journey in their wilderness..
I wanted to sit in the starkness and the barrenness.
of the Judean landscape to learn something of what it is.
that places such importance in the heart of God.
to such barrenness..

$^{161}$And I thought one of the great things to do today.
as we move into this idea of the place of the wilderness.
is actually to take you to the Judean hills today..
And so we filmed a little film whilst we were there.
to help all of us understand what Advent truly is.
and what the ministry of John the Baptist.
was actually like..
So I wonder whether you would come with me now.
as I take you to the desert place of the Judean wilderness..
The wilderness is the mark of the land of Israel.
and it continues to surround much of its major cities today..
Barren, expansive, hostile and stark..
It has largely been uninhabited throughout their history..
Consequently, this has meant that this land here.
is an ideal place for seeking refuge from their enemies.
or retreat from the world around them..
From the call of Abraham to the journey of Moses,.
to the release from Egypt, to the return from exile,.
the story of Israel has always been a story.
of God moving his people in and out of the wilderness..
(gentle music).
(birds chirping).
While the wilderness was never an easy place.
for the nation of Israel,.
it was a place where they could come to expect to meet God..
See, it took people out of the ordinary of their lives.
and brought them into a place.
where everything got stripped away..
A place where they were able to look at their lives afresh..
The wilderness was God's way.
of getting his people's attention..
Well, it was in this place.
that God began to form Israel's core identity..
And it always was a process that began.
with a strong call to repentance..
You see, the Old Testament had finished on a clear note.
that it was Israel's sin that had led them into exile.
and that their repentance of sin.
released them back into Jerusalem..
Now, some 400 years later,.

$^{201}$God renews the prophetic call for Israel.
to move into the wilderness.
through the ministry of John the Baptist..
And that would have communicated one clear thing to them,.
that God was about to redeem his people again,.
but it needed to begin once more.
with a call to repentance of sin..
John knew that his water baptism.
came before a more significant baptism.
of fire and the Holy Spirit that Jesus would perform..
This fire is Jesus's judgment on the world's sin..
He's come to liberate the world from this sin,.
once and for all,.
and that requires a sobering examination of our hearts..
But John also knew something else..
You see, God brought his people into the wilderness,.
not because he was angry with them,.
but because he actually wanted to go deeper.
in relationship with them..
You see, far from being a place of punishment,.
the wilderness was actually a gift for God's people..
It was a place where God could clear out.
all of the stuff that had accumulated in people's lives..
You know, the fact is when we go through life,.
we pick up things all the time,.
things that begin to kind of weigh us down,.
the daily struggles of life..
And God brings us into a wilderness season.
so that he can lay these things down.
in the barrenness of the place,.
where he can open us up.
so that he can deal with these things..
And God's heart is to basically burn them up.
in the wilderness place so that new life can begin..
- Luke links John the Baptist.
with the prophetic hope of Isaiah,.
where Isaiah declared that one would come.
to prepare the way for the Lord..
So as Luke introduces us to John the Baptist,.
he does so with great fanfare..

$^{241}$You can almost sense the excitement that Luke has.
as he writes, "This is the one who would prepare the ground.
"for our coming savior.".
John uses two metaphors to speak of Jesus..
He speaks of him as a fork and as fire..
And then he uses this picture of wheat and chaff..
And he says that Jesus is going to be like a winnowing fork.
that separates wheat and chaff..
And then he's gonna be like a fire that burns up that chaff..
And what John does essentially right at the beginning.
of the story of Jesus is introduce us.
to one element of Jesus's life.
that we don't often talk about,.
the fact that he's not just a savior, he's also a judge..
(gentle music).
- A fire in the wilderness is a welcome sight..
It's light breaks into the blanket of darkness around you..
You know, in our current culture of constant access.
and nonstop media, a time in the wilderness.
can actually feel a little bit like a curse from God,.
but it's actually anything but..
When we are invited into the wilderness by God,.
it's because he wants to bring us near.
to the burning presence of his Holy Spirit..
And it's in that Holy Spirit that our pain and our hurt.
and our brokenness is burnt up and dealt with..
In the wilderness, we come to see God.
gloriously, beautifully, holy..
And in the midst of that,.
get to see ourselves as humans fully alive..
(gentle music).
- Us, humanity, fully alive..
I don't know about you,.
but I don't know if that's how I feel.
in the wilderness experiences that I've had in my life..
We all have them, don't we?.
These times of desert experience,.
where it feels like everything of comfort.
is removed from our lives..
It feels like that presence of God is taken away from us..

$^{281}$It feels like life is just hard work every single day..
I think the reality is,.
is that 2020 has felt like that for us as a year..
It feels like this whole year.
has been like a wilderness journey,.
where it feels like all the things.
that we've relied on before,.
the freedoms that we've had and the comforts we've had,.
the normality that we've had, it's been taken from us..
I know for many of us,.
this is gonna be a different Christmas.
than we've experienced in the past..
Perhaps your loved ones are not able to travel.
to be with you here in Hong Kong this Christmas,.
or you're not able to travel to go be with them..
2020 has felt like a changing year.
where things are taken away from us.
that we would rather have,.
like we've been placed in this position of wilderness..
Could it be though,.
that just like Israel in the time of John the Baptist,.
so for us, God is actually calling us this year.
into a place of stark discomfort.
in order that we might discover something.
that we would never find in our place of security..
Could it be that the wilderness is actually designed.
to give us the very thing that we would miss.
in the normality of life?.
See, there was a reason why God called Israel.
out of the comfort of their day to day.
into the wilderness again,.
to hear from John at that time in the first century..
I spoke about it a little bit in the film.
that you just watched,.
the reality that actually in the Old Testament,.
the wilderness was a picture of the reality of something.
that we often don't think about it in our lives..
We think of the wilderness or the desert.
as this isolation experience,.
as an experience where we're actually separated.

$^{321}$or far away from God,.
but actually Israel's history tells them.
that the wilderness was a place.
where they would go to meet with God..
That the wilderness became a turning and a changing point,.
both in their understanding of the presence of God,.
as well as for the understanding of themselves..
I mean, you just need to open up the pages.
of the Old Testament and begin to see the centrality.
that the wilderness, the desert plays.
in the formation and the growth and the change of Israel..
There's Abram who's comfortable in the father's household.
in a place called Ur,.
and God shows up and pulls him out of that.
and moves him into the desert of the Negev..
And it's in that desert place that he looks up.
and he sees those stars and God gives him a vision..
God says, I'm gonna make through you a nation.
that'll be more numerous than the stars.
that you can see in this sky right now..
And God comes and transforms the person of Abram.
and makes him into Abraham,.
who comes and makes him a father of the multitudes..
It's in the desert place where actually Abraham.
gets a vision of what it is that God wants to do,.
that God's gonna create a covenant with this group of people.
to make them a blessing to all nations..
It's actually in the wilderness that Abraham begins.
to understand the purpose and the promises.
that he has for his life..
Or I have a thing about Hagar..
Hagar, who was the maid servant of Sarah.
and the concubine of Abraham..
The one who was under oppression from Sarah at the time.
and was forced herself to flee into the wilderness..
And it was in the desert place.
that an angel of the Lord came to her.
in her most desperate moment.
when she thought her son was gonna die,.
where she had given him up to die..

$^{361}$The angel of the Lord comes and says,.
I'm a God who sees you and shows her an oasis.
where she can get refreshment and provision.
and be able to bring her child back to life..
Or think about Jacob..
He wrestled with God in the desert place, in the wilderness..
And it was there that God changed his name.
and said that you're gonna be.
the biological father of Israel..
And your sons will be the 12 tribes.
that would start the journey of this nation..
Think about Moses, who had fled out of the home.
as an Egyptian prince, who had murdered an Egyptian.
and was now in the desert as a shepherd,.
running from all his pain, running from all his hurt..
And it's in the desert place that God in his holiness.
and his power shows up with the burning bush.
and welcomes him into a new time.
where he's restored in his identity,.
where he comes to understand and meet God..
And it's there at the burning bush.
that God declares himself as the I Am for the first time..
Yahweh, defining his reality for the people of Israel..
Or think about Israel themselves,.
taken out of slavery in Egypt,.
brought into the wilderness to Mount Sinai..
And it's in that desert place.
where the pillar of fire comes..
The law is opened up to them.
with the 10 commandments on Sinai,.
where they get this intimacy with God.
that they'd never had before..
And where God would speak to shape them into the nation.
he desired them to be..
Think about David..
David wrote the most profound Psalms that we have..
Almost all the Psalms that he wrote,.
he wrote in exile in the desert place..
Think about Israel themselves.
when they were taken into Babylonian exile.

$^{401}$through the desert..
And in a place of stark barrenness,.
had to wrestle with the reality.
that their sin had brought them there..
And that they were beginning to turn.
to a new understanding of God.
in the place of their wilderness..
What you see throughout the scriptures.
is God moving his people in and out of the desert place..
And it's there that he manifests the power.
of who he is to them..
You see, there's something critical.
we need to know about the wilderness..
See, the wilderness has always been the place for God.
of our accelerated growth and transformation..
God's always used the wilderness.
to accelerate growth and transformation in his people..
And you see throughout the story of the Old Testament,.
I think this transformation happening.
in three profound ways in the desert place..
First of all, the desert was always.
about a place of intimacy..
It was always where God would draw near.
to his people in their most desperate time..
It's always where you see this intimacy.
where almost like the distractions of our lives.
are stripped away from us..
The desert place in its barrenness.
means that there's nowhere else for us to go.
other than into this profound picture of who God is..
See, the desert place is the invitation to intimacy..
We often think that actually the desert is a lonely place,.
but actually the desert is a place.
of intimacy and connection..
The desert is a place of revelation.
where we can no longer hide God.
behind the distractions of our lives..
In fact, what the desert does is it puts God in plain sight..
He's right there, right there before us..
And we get to see something of his majesty and his beauty.

$^{441}$that would be distracted and covered up.
by the normal busyness of life..
It's about intimacy..
But the desert was also about this second process,.
that of teaching..
You see, God always brought his people in the desert,.
not just to draw near by his presence,.
but to give them a message..
He had something on his heart..
He wanted to communicate to them..
So Hagar experiences profoundness of God's presence.
by the water of that oasis..
And she understands that God is one who sees her,.
that God is one who provides for her..
And it begins to shape a teaching for her.
around the power of God's provision..
Moses with the mountain and with Sinai,.
God shows up and says,.
"This is what I want you to do to worship me..
You're gonna worship me with all your heart, mind,.
soul, and strength, and you're gonna worship.
and love your neighbor as yourself..
And as you do these two things,.
there is a commandment and a structure of the law.
to enable you to connect with me.
and to connect to one another.".
There's a teaching you need to know.
about what it is to walk in a place of holiness.
before each other..
When Israel was in that wilderness place of exile,.
God spoke to them about their sin.
and challenged them that they needed to change,.
the desert has always been a place,.
not just of the intimacy of his presence,.
but also of the power of his word to speak,.
to change, to challenge, and to make new..
And the third thing is that the desert.
is a place of great healing..
Because once you are intimate with the Lord,.
once he's met you in your own barrenness,.

$^{481}$once he's spoken his word and told you what it is.
that's on his heart for both you and for his purposes,.
then this place of healing comes,.
where Hagar is able to stand up and go and collect her son.
and know that she is not the person.
that Sarah had told her she was..
Where Moses would be able to be there.
before the holiness of the burning bush.
and be able to stand and say,.
"Even despite the fact that I'm not particularly articulate,.
I will go back to Pharaoh,.
the one I've actually run away from,.
and I'm gonna ask him to let my people go.".
Where Israel in Mount Sinai would begin to understand.
that they are not to worship a plethora of gods.
like the Egyptians had,.
but now they are to worship that one God..
And as they worship the one sole God,.
they are formed in their identity as a nation,.
no longer slaves,.
but released to the promised land for freedom..
You see, the desert is a place of intimacy,.
of teaching and of healing..
And it is a place that moves us on that journey..
And all of that would have been in the minds.
and the hearts of Israel,.
when they hear that some 400 years after God being silent,.
a prophet was speaking again in the wilderness..
The wilderness would have communicated.
to the Jewish people of the first century,.
the things that they had seen in the Old Testament..
When they heard a prophet was now calling them.
into the wilderness,.
they would have stepped into the wilderness.
with this thought in their minds..
They would have heard the idea.
that God is calling them into intimacy again..
They would have heard the idea.
that God has got something to say to them,.
a message to bring to them..

$^{521}$What they would have heard.
is that now would be the time for them to be renewed,.
for them to be healed again..
This would have been on their hearts and their minds..
They wouldn't have thought,.
oh, I'm being called into the wilderness to be told off,.
to be separated from God, to be in a barren place..
No, they would have felt this call of God to intimacy,.
to teaching, to healing, to restoration, to new..
When they would heard.
that there is an invitation to the wilderness,.
what would have been communicated to them.
above everything else was this,.
that God was about to act again to redeem His people..
That's what they would have heard..
That's not what we hear though, is it?.
I mean, how do we feel when we're in a wilderness place?.
How do we come to think of the desert place for us?.
I think so often when we think of the wilderness,.
we think of it as a place of judgment and punishment..
We think the wilderness is a place that we go to.
when we've done wrong with God..
The wilderness is a place that we go to.
to be told off by God,.
almost like it's the naughty corner of the Christian life..
And because we come to think of the wilderness.
as an absence of God, rather than an intimacy with Him,.
which is how the Jewish people saw it,.
as we see it as this absence of God,.
we do everything we can to avoid the wilderness..
Or if we think we're about to head into the wilderness,.
we wonder what's wrong with ourselves..
When actually, could it be that the wilderness.
is a place of our greatest freedom?.
Could it be that what we see.
in the ministry of John the Baptist.
and what we see in the season of Advent.
is actually an invitation for us.
to actually redefine our wilderness?.
Maybe here, as we journey to the end of 2020.

$^{561}$and we head to the beginning of 2021,.
maybe God wants you to redefine your wilderness..
That time that you thought was a sort of an abstraction.
from the presence of God.
might actually be the greatest welcome into intimacy.
and His word and your healing.
than you ever have experienced before..
That's what John the Baptist thought..
That's why he stood in that place of the wilderness.
and called Israel out to him again..
And I wanna show you actually through the scripture.
that we're looking at today,.
the way in which John the Baptist creates this idea of hope,.
even in a place as barren as the Judean landscape..
Let me read verses seven to 10 to you once again..
But when he saw many of the Pharisees and Sadducees.
coming to where he was baptizing,.
he said to them, "You brood of vipers,.
who warned you to flee from the coming wrath,.
produce fruit now in keeping with repentance..
And do not think that you can say to yourselves,.
oh, we have Abraham as our father..
I tell you that out of these stones,.
God can raise up children for Abraham.".
See, the ax is already at the root of the trees.
and every tree that does not produce good fruit.
will be cut down and thrown into the fire..
Now there's some pretty strong words that John has here.
for the Sadducees and the Pharisees..
I mean, he's speaking some things pretty directly.
to them, brood of vipers, ax at the trees,.
cutting things down..
Yeah, it sounds pretty strong, but in the midst of that,.
I want you to see what pops here..
Two things that John repeats..
He mentions the idea of fruit twice..
He mentions the second time that it's actually good fruit..
In other words, what he's challenging Israel with.
is yeah, they need to change..
Yeah, there is this repentance that,.

$^{601}$as I talked about last week,.
is that 180 shifting and reorienting into a new direction..
But John wants them to focus on the reality that is fruit..
Hey, live in the fruit, expect the fruit..
Don't hold anything back.
that might try to take away the fruit..
See, the reality is the desert.
is actually the place of fruitfulness..
The desert isn't this place.
where everything's supposed to go wrong..
The desert isn't a place where you're supposed to go to.
so you have to wait 40 years for God to speak..
No, now the desert becomes a place of great fruitfulness..
I like to think of it like this,.
the desert is like the great potter's wheel of God..
And through it, he actually creates us.
into new works of art, beautiful new works of art,.
ways in which we can become more fully alive,.
more fully true in him..
And yes, the desert is a harsh place..
Yes, there are words that God wants to bring us.
in that place where we bring our sin to him,.
where we repent of the things that we need to repent of,.
but there's a movement through the desert..
You see, one of the things that's important to know.
is that Israel never made their home forever in the desert..
The desert was a journey for them..
It was a movement of that intimacy teaching and that healing.
so that they would be restored to then be released.
back into what it is that God has for them..
And you actually see that in the verses here..
You see John actually bringing his people.
in this journey, in this movement..
First of all, calling them out of the civilization.
of their sin, welcoming them into the desert place.
of repentance, the wilderness of that intimacy.
and teaching and healing that God wanted.
to bring his people so that they could then go.
into the waters, the river of life..
And Justin's gonna come next week and talk to us.

$^{641}$about the beauty of the refreshing waters.
of the washing of our sins through this picture.
of baptism, but I want you to see this week.
the journey that takes place..
We have to get out of the civilization of our sin..
We have to walk into the desert wilderness place.
so we can be intimate, we can be taught,.
we can be healed so that we are then released.
into the cleansing and refreshing.
and the fullness of life that he has for us..
I wonder whether Christ might wanna come right now.
and begin to redefine for you your wilderness..
Some of you are in a wilderness time right now.
and you've been beating yourself up,.
you've been finding it hard, but maybe it's.
because you've been thinking about this time.
in the wrong way..
Maybe it might be a shift for you in your spirit today.
to begin to actually celebrate the very thing.
that God's got you in..
Doesn't mean it's gonna suddenly be easy,.
but a shift in your heart towards what the desert.
is a gift to you, the fruitfulness that God wants.
to bring to you out of the season that you're in.
in your life right now..
And that fruitfulness comes, and I love this in verse 10..
Let me read this to us..
The axe is already at the root of the trees.
and every tree that does not produce good fruit.
will be cut down and thrown into the fire..
This is John standing before Israel.
and telling them what's about to happen next..
He's like, God will do anything, absolutely anything.
to be able to fight back the enemy,.
where the enemy would try to keep people from him..
You see, what he's saying here is he's saying anything.
that will come against the good fruit,.
anything that stands in the way of what I wanna do.
through my intimacy and my instruction with my word.
and my healing of the people,.

$^{681}$anything that's gonna stand in the way..
Well, guess what?.
The axe is at the root of that tree,.
and I will cut down anything that will stand.
in the good fruit that this wilderness experience.
wants to bring..
And as I was preparing this message,.
I felt that in my spirit.
for some of you watching here today..
I felt like there's some things going on in your lives.
and there's some things that are happening,.
and you know that there is obstructions between you and God..
And I want you to know that no weapon.
that has been formed against you will ever prosper..
I want you to know that nothing that the enemy can do.
to hold you back from that healing and for that restoration,.
from the teaching of his word, for what he wants to do,.
nothing is gonna hold you back permanently.
from the good fruit that God desires to bring..
See, the axe is at the root of the tree and the strategies.
and the things that the enemy has,.
and anything that will hold that good fruit back,.
God wants to change..
See, this is the work, the act, the work of the wilderness..
It is actually to bring us into that place.
where God will do everything,.
that anything that would ever stop us receiving.
that intimacy and that teaching and that healing.
will be cut down and thrown into the fire..
I mean, isn't that an encouragement?.
I mean, for some of you who think.
that this wilderness journey right now.
is the worst thing that could ever happen to you,.
actually God sees it as the place.
of your greatest transformation..
That in him and through him,.
he might begin to shift and change you..
I think there are many things that we actually end up.
carrying around with us that end up weighing us down.
that God doesn't want us to have..

$^{721}$And you saw this in the video,.
and I wanted to remind us of this again today..
And I think there are some things.
that the enemy places upon us,.
that the enemy wants us to carry.
so that we feel weighed down in the process of our lives..
Things like the deep hurts that we have in our lives,.
the hurts that come to us,.
maybe from the words of people around us.
who have tried to define us in a certain way.
that's different to how God would define us,.
just like in Hagar's experience..
Or maybe it might be the abandonment we feel at times,.
the abandonment we feel when we haven't heard words of love.
and affirmation from those that are closest to us in life..
Or maybe in this season, it's the depression.
and the anxiety, those dual things.
that seem to come against us.
in the moments of our wilderness experiences,.
in the moments of the desert.
where we struggle with our depression and anxiety,.
and that stresses us.
and brings those additional stresses of life upon us..
Perhaps it's actually in the place of wound,.
those wounds that come upon us.
because of our broken relationships in our lives,.
maybe with our loved ones,.
like we were talking about earlier,.
or those problems in our families.
that just seem to weigh us down..
Or perhaps finally, it might be the idea of our habitual sin,.
that sin that we can't seem to change.
or the sin that we can't seem to shake from us.
that just seems to be placed upon us.
and places this weight and this burden on us..
And here's the reality..
We walk into our wilderness carrying all of this stuff,.
all of the things that the enemy would place upon us.
to hold us back and to weigh us down.
and to tell us that this is what defines us.

$^{761}$rather than the liberation and the presence.
and the intimacy and the teaching.
and the healing of the word of God..
But the problem we find is it's hard for us.
to pick up anything else when we're so burdened.
by the stuff that the enemy has given us..
You know, Jesus says in Mark chapter eight,.
he says it quite profoundly..
He says, "If you wanna follow me,.
if you wanna come after me,.
here's what you are to pick up..
You're not to pick up those wounds and the hurts..
You're not to pick up the things.
that you think bad about yourself..
You're not gonna pick up that abandonment..
Here's the thing to pick up..
Pick up your cross..
Pick up the very symbol of what the ultimate experience.
of the wilderness is..
For us, the cross is often celebrated by the cup.
and by the bread, isn't it?.
These two symbols of that communion experience given to us.
because of the cross of Christ..
And if the cross is anything,.
it's God's ultimate wilderness journey..
It's the time where Jesus goes to the cross.
and he goes into that place of separation from his father..
He goes into that place of abject loneliness,.
of being naked and unashamed, stripped down,.
whipped and beaten..
The cross, this great symbol of death.
is transformed in the wilderness.
to become this place of great life.
through the teaching and through that healing.
and through that intimacy that's found in the cross.
in the resurrection..
We now find ourselves in a place.
where we get to pick up our cross.
rather than the things that we've found.
gathered around us in the journey in life..

$^{801}$But how do we pick up what Christ has afforded for us.
in his death and resurrection?.
Well, it can only come when we realize.
that the wilderness of the cross provides for us.
the newness of our life..
And because of the forgiveness of sin,.
we are able to lay down the things.
that the enemy had placed on us.
and to pick up the new symbols of our life..
The bread of the body of Christ that was broken.
so that we would know the wholeness and the renewal of him..
The blood that was shed.
so that we would experience forgiveness of sin..
The wilderness of the cross becomes for us.
a journey into our wilderness,.
a journey into the place.
where we are stripped naked ourselves,.
where we realize that there is nothing.
that we have to offer,.
where we realize that we are not worthy,.
where we realize that all of those sticks.
that we're carrying do hold us back..
And yet in the invitation of the cross of Jesus,.
we come to a new place where we get to lay our crowns down.
and we get to receive from him.
the things that he paid a great price for,.
the things that truly set us free..
The invitation to communion is the invitation to us.
moving into that place of the river of life..
And as you are thinking.
about your wilderness experience today,.
let me challenge you with this..
Let me challenge you to redefine your wilderness..
Let me challenge you to have this expectation.
that in the dryness right now,.
you can expect the intimacy of God.
like you've never experienced before..
Let me encourage you.
that it's actually in the place of wilderness.
where God wants to speak to you,.

$^{841}$where his voice wants to come and begin to shape you.
and teach you and show you his ways..
May you have a change in your perspective.
this season of Advent where you begin to say,.
no, I'm gonna ask for my healing..
I'm gonna ask my restoration as I receive that intimacy,.
as I listen to his voice,.
I can expect the transformation of myself..
May you know that in your redefined wilderness,.
you will find what Israel found in that moment,.
a reorientation of life..
Just like I said in the film, us humanity fully alive..
So may I welcome you to the communion table today..
And if you've prepared communion,.
this is the time just to gather that around you..
If you haven't, you have about 30 seconds.
to rush off to your kitchen or wherever it is.
to grab something that you can use for communion..
It could be some grape juice, it could be some water,.
it could be maybe some fruit or a piece of bread.
if you have it at home, whatever works for you,.
maybe just gather up now that communion thing.
and we're gonna share in this together..
You know, they gathered in that upper room right there.
at the end of that final week of Jesus's life..
And Jesus having prepared himself.
for the coming wilderness of the cross,.
takes the bread in that moment.
that would have been shared in the normal Passover feast..
And he takes that Passover celebration.
and brings it onto himself..
And in taking the bread, he rips it open,.
symbolizing of course,.
the coming tearing apart of his body on that cross..
And as he rips it open, he takes it, he breaks it,.
he blesses it, and then he shares it.
with those that are around him..
And in sharing it, he says,.
"This is my body broken for you.".
Paul later would pick up on that to his church in Corinth..

$^{881}$And he says, "This is what we get to do.
as we celebrate the release into the fullness.
of what God has for us in our lives..
We take the bread that has been prepared.".
And so let me invite you now to hold that bread.
in your hands and let me pray for us for the bread..
Father, we thank you for this symbol of your life.
that has been broken for us..
We thank you that this life has been prepared for us,.
that whatever wilderness experience.
we might be in right now,.
we know that there is fruitfulness in this place,.
that we know that there is this redefining.
of our expectations as we now come to you,.
that your body broken on the wilderness of the cross.
invites us now to intimacy in you..
And as we take the bread, we do so knowing.
that your life forms and shapes for us.
the fullness of life that we are now welcomed into..
And so we take this bread in Jesus' name..
Let us eat this together now..
And as we take the bread, we now come to the cup..
And the cup is this great symbol, of course,.
of the shed blood of Jesus for us..
And it's a blood that's shed for the forgiveness of our sin..
And we know that as Israel was called into the wilderness.
by John the Baptist, it was with this message of repentance..
And we walk into that wilderness of the cross,.
knowing that it is also our place of repentance,.
our place to bring our biggest worries and our stresses,.
the war that I was speaking about earlier,.
those logs that we so easily carry around with us..
It's time for us to bring them before God.
and to feel Him release us from them..
So I want you just to take a moment,.
maybe just a few seconds right now,.
as you hold the cup in your hand,.
just to bring yourself before Him in repentance..
Allow the Holy Spirit to speak to you..
And is it wounds for you?.

$^{921}$They're hurt for you?.
Is it things that you're carrying around.
that is abandonment?.
Maybe you haven't heard the words from your loved ones.
that you are hoping to hear..
Maybe there's disappointment in yourself..
Maybe there's habitual sin that's at work..
Whatever it might be, take a moment now.
to lay down those sticks before Him..
So to visualize yourself unloading that heavy weight.
before the cross today..
I'm just gonna give you a few seconds.
to do that in the quietness of your moment..
Father, we thank you for this time.
where the desert reminds us of the barrenness.
of what sin does to us..
And yet the hope that there is in the meeting of you.
in that place, where you redefined identity for Israel.
and identity for all of the people that you encountered.
in the desert place..
So as we come to the crossing communion,.
you redefine our identity from sinners.
to those that have been forgiven,.
from those that have been at war.
to those that are now at peace..
And we celebrate this through the shedding of your blood..
And we take communion now in that spirit in Jesus name,.
we share together in the cup..
We're grateful that we get to do that in celebration.
and in release..
And now as we do so,.
we're gonna enter into a moment of worship.
as we come before God and celebrate the reality..
And wherever it is that you are here at the end of 2020,.
no matter what's been going on in your life,.
no matter how hard it's felt in that wilderness place,.
know that Christ is bringing his intimacy,.
his word to you and his healing to set you free..
Why don't we join together in this moment of worship?.
(gentle music).

\newpage



\section{}
\label{sec:yAN1zxPhVpI}
\textbf{2020-12-20 Church Everywhere Live: Advent 2020 - Prepare The Way - “The Release” [yAN1zxPhVpI].mp3}
\newline
\newline
連結: \href{https://youtube.com/watch?v=yAN1zxPhVpI}{\texttt{ https://youtube.com/watch?v=yAN1zxPhVpI}} ~~~~ 語音日期: 2020-12-20 
\newline
\newline
\hyperref[sec:zigfpAW3XVI]{\small{< < < PREV SERMON < < <}}
~
\hyperref[sec:index]{\small{[返主目錄]}}
~
\hyperref[sec:KS3UcRDETW4]{\small{> > > NEXT SERMON > > >}}
\newline
\newline
$^{1}$So recently I was invited to speak at the annual retreat of a large international nonprofit organization..
I was honored to be invited..
I was so excited to gather with this group of people overseas and take them through what I sense God had put on my heart..
I was invited to be the only speaker of the retreat..
It was over four days..
There were six main sessions that I was to bring content into..
And I remember being pretty nervous..
It was one of my first times to be overseas and speak..
And I was just really so prepared to bring what I sense God had put on my heart and felt so deeply honored to be able to do this for this group of people..
And I remember sitting there in the very first session..
I'm sitting on the front row..
I'm going through my notes..
I'm drastically trying to build the confidence in my heart to stand in front of this room that's packed full of all these people..
And the CEO of the organization, he gets up onto the platform to introduce me, to welcome me to the conference..
And here's what he says..
He goes, "Unfortunately, the person that I really wanted to be here to speak at this retreat couldn't make it this year..
So I'm grateful that Pastor Andrew Gardner from the Vine Church in Hong Kong can be with us today.".
I mean, can you imagine it, right?.
I'm sitting there and I've been introduced in lots of different ways throughout some 20 years of preaching, but I've never been introduced as the clear second choice..
And so I'm sitting there and all that inspiration and all that hope and encouragement is just drained from me..
I'm suddenly feeling like I'm weighing like 400 kilograms..
I like trudge up to the platform for my opening speech and I get in front of all these people..
And here's what comes out of my mouth..
I say, "Hey, everyone, I'm plan B..
No one likes to be second A..
I mean, none of us train hard for the silver medal, right?.
Like second is the great amnesia of history..
No one ever remembers who didn't win the prize, who didn't get the cup, who wasn't the one who actually succeeded..
The reason why we don't like second is because second reminds us that somebody else did better..
Somebody else performed more..
Somebody else was more desired than us..
When we're in second, it hurts our ego..
It hurts who we are..
And it's an uncomfortable place to be..
And so I think as humanity, we do everything we can to try to avoid the nastiness of being in second place..
The reality is, I think we do this in our churches..
I think so much of the Christian faith is so often from the pulpit shaped around this idea that as Christians, we never have to worry about second place..
No, as Christians, we get to be first, don't we?.
Yes, we do..
We're victorious..

$^{41}$We're the overcomers..
Oh, we use words like we're the chosen and we get to go to heaven, the ultimate first place prize..
And the reality is when we push ourselves into this idea that we should be in first place, we actually have to stop for a second and realize that we completely missed the point of Christianity..
But yes, it's true..
As Christians, we will be victorious..
We will rise again..
We will be at some point with our heavenly father..
Those things are amazing..
They're the great promises of our faith..
But here's the irony and the beauty of the Christian faith..
We actually win by losing..
We actually win by actually being way second place, by giving up ourselves, by laying our crowns down, by actually picking up our cross and denying ourselves..
You will never get to where God wants you with a desire to be in first place..
Here's the reality..
First place has always been and will always be reserved for Jesus..
He is the one who is victorious..
He is first..
We are second..
See, this desire to be first has always been the antithesis of the gospel..
It's a desire that kicked Satan out of heaven..
It's one of the temptations that Satan brings Jesus in the wilderness..
The desire to be first overcame the disciples as Jesus gained in popularity..
And they said to him, "Who's going to sit at your right hand and sit at your left hand in glory?".
The desire to be first, I think, has crushed the global church over almost 2000 years of history..
And it's created so many problems in leadership..
And here's the reality..
Here's the reason why..
The attributes of first are things like glory and power and authority and worship..
Things that are rightly for Jesus..
And here's the thing..
As humanity places itself in a position of first, when we try to grab a hold of attributes like power and authority and glory, we get so easily seduced by them..
They corrupt us..
They corrupt our hearts because these are attributes that are reserved for Jesus..
The desire for first is the antithesis of the gospel..
The desire to be second is fitting and deserving of it..
Because the idea of being second is the idea of some of the great themes that we see in the gospels..
The idea of humility..
The idea of honesty..
The idea of serving the other..
The idea of loving your neighbor as yourself..

$^{81}$The idea of realizing that there is somebody else who is as important, more important, more needful, more anointed..
That somebody else gets my life..
I can pour myself out towards them..
Christmas is the celebration that humanity gets to be second..
Because Christmas is all about the idea that Jesus comes..
That Jesus himself comes to model and to demonstrate what the Christian life looks like..
As we look at the beauty of the birth of Jesus..
I love how the apostle Paul brings it to us in Philippians chapter 2..
Paul says this, that even Jesus, the rightful one to be first, didn't hold being equal with God something to be grasped..
But instead, Jesus humbled himself..
That Jesus came in the incarnation as a human..
That he was willing to take off that power and that authority and lay it down..
Why?.
So he could serve us..
And what Jesus does in his life, in his death, in his resurrection, is models for us what it is to be in that place of second..
Everything Jesus did was to honor the reality of where we are to be..
And it blows my mind that the one who should be first chose the posture of being second to lead us out of the desire to be first into a place of second..
So that he could be truly glorified..
I hope you're following that..
Let me say it this way..
One of the beautiful ways that I think Jesus models to us what it is to be in second place..
Is that Jesus shows us what it is to be a follower..
A follower is the beauty of second place..
Jesus, throughout his ministry, would say this..
He says, "I can't say anything unless my father tells me to say it..
I won't go anywhere unless my father leads me there.".
Jesus's whole ministry was about listening to his father and being obedient to what his father was saying..
Because he was humbling, modeling something of what it is to be a follower..
And this is so beautiful..
It's Jesus consciously and deliberately demonstrated to us the art of what it is to be a follower..
And I call it an art because it's not a science..
Following after something is an art..
It's an expression of creativity..
It's individual..
It fills our own experience..
Following someone takes practice..
It's a sacrifice in order to do so..
We have to constantly fight that ego and that pride that would want to put us in front..
Being a follower enables us to put Jesus in that rightful place..
The beauty of being able to do that establishes for us something of the centrality of the kingdom of God..

$^{121}$That the foundation of everything we do comes from the idea that we get to follow Jesus..
You know, I always think of it this way..
I think following is like the undercoat of the kingdom of God..
It's like that first piece that's put on that canvas by an artist..
And he's going to paint over it many, many times..
But the ultimate color at the end only shines so deep and so beautiful because of the undercoat that was first placed on the canvas..
Our following, our welcome into what it is to follow Jesus is the undercoat of God's kingdom..
It's the foundation of everything..
And it's exactly what John the Baptist does in the final movement of his ministry..
We trace these movements that he's been doing throughout our Advent time..
We've looked at that first movement of calling people to repentance..
The second movement of welcoming them to the desert place where they could meet with God..
The third movement, as we saw last week, of being invited into the waters of baptism..
And the ultimate experience of the washing of the Holy Spirit in the baptism of the Holy Spirit..
And now, right at the end of his life, right towards the end of his ministry, John does the final thing..
He releases the crowds to follow Jesus..
John ultimately says, "I must decrease, He must increase.".
Advent, if it's anything, is about the idea of preparing ourselves to truly follow Jesus..
And I want to show you some ways in which I think John invites us into being followers of Jesus today..
And you may have been a follower of Christ for many years..
Maybe like me, you've been a follower of Christ since you were pretty young..
Or maybe this is a new experience for you..
Wherever we are, I think so often we end up finding ourselves slipping back into the centrality of the world..
Where we become the focus..
And Jesus is like this deity on the shelf that we turn to in times of need..
May I challenge you today that we need to learn what it is to follow Jesus again..
Let me read this to us from John chapter 1..
I'm going to start in verse 29..
"The next day John saw Jesus coming towards him..
And he said this, 'Look, the Lamb of God who takes away the sin of the world..
This is the one I meant when I said,.
'A man who comes after me has surpassed me because he was before me..
I myself did not know him..
But the reason I came baptizing with water was that he might be revealed to Israel.'.
Then John gave this testimony..
'I saw the Spirit come down from heaven as a dove and remain on him..
I would not have known him except for the fact that the one who sent me to baptize with water told me this,.
'The man on whom you see the Spirit come down and remain is he who will baptize with the Holy Spirit.'.
So I have seen and I testify that this is the Son of God.".
Take a moment to think about the context here..

$^{161}$This is just probably a day or two after John has actually baptized Jesus in the Jordan River..
That baptism has taken place..
And now John realizes that the final movement of his ministry.
is to try to release all the disciples that had gathered around him..
Because John had become popular..
John's message had gone out to the people..
They were excited that the Messiah was about to come..
And here's John now saying, "Okay, I don't need all of this..
I'm not to be in that place of first..
If you're following me, you're in trouble..
No, no, I need to release you.".
And so as he sees Jesus walk past on this one day,.
he shouts out with his whole voice in front of all these people..
"Look, look, there's the Lamb of God..
Like right here..
This is the one who's going to take away the sins of the world..
I'm not that one..
This is the one.".
I love this..
He says, "Look, there is the Lamb of God.".
Now that phrase, the Lamb of God, it's actually only used here in John's ministry.
and also in the book of Revelation..
But it's actually not very prominent in the New Testament as a whole..
And the question we should ask ourselves is,.
why is John using that particular phrase in this moment?.
Well, he's drawing from beautiful Old Testament imagery..
The whole imagery of the Passover..
In the Passover, where they slaughtered the lambs.
and the Jewish people who were enslaved in Egypt.
took the blood from the lambs and put it on their doorposts on their walls.
so that the angel of death would not come across their door.
so that anybody in that household would be protected..
And here's John, how many years later,.
and he's pointing at Jesus and he's saying,.
"This is the new Passover right here..
This is the one whose blood will need to be shed.".
By calling him the Lamb of God, right at the start of Jesus's ministry,.
John is declaring that he's going to die..
I mean, that's what lambs were for..
They were for sacrifice to be slaughtered..

$^{201}$And here's John saying, "This is the one who's going to be slaughtered.
so that all of the sins of the world will be forgiven..
No longer would it just be blood over the walls and the doors for the Jewish people.".
Notice, he says, "For all the world.".
Like everybody, Jews and Gentiles alike will come to have their sins forgiven.
because of this one..
Look, there is the Lamb of God..
And I think that's so powerful and so beautiful.
that we can remember that he forgives our sins,.
that he can release these things from us,.
that we get to follow him..
And John says in verse 30, he says it this way..
He goes, "This is the one that I meant when I said,.
'A man comes after me who has surpassed me because he was before me.'".
This is classic John..
I mean, imagine if you said that,.
"This is the one who's after me before me because he was before me.".
Like, this doesn't make sense..
Like, what are you talking about, John?.
This is like, you can imagine his disciples going like,.
"Did he eat too many locusts today?".
Like, dude, chill out..
Like, what you, I don't get what's going on right now, right?.
Here's what John's saying..
He's saying, "Jesus is first always..
He was before me..
He's going to be after me..
He surpasses me..
I get to be a follower of the Lamb of God,.
the one who forgives my sin.".
This one right here..
Jesus' ministry was to point out the Lamb of God,.
to show people the way to Jesus..
And I don't know about you,.
but I find that incredibly challenging at this time of the year..
As I see Christmas just coming up on Friday of this week,.
I'm thinking to myself,.
"Am I pointing people towards Jesus?".
Like, does my life, the things I say, the way I act, who I am,.
does it make space for the reality.

$^{241}$where I'm trying to lead my sphere of influence.
to the place where their sins can be forgiven?.
Am I an evangelist?.
Am I pointing out to people that this is the Lamb of God?.
This is deeply challenging,.
'cause I want to be honest about it..
I think so often as Christians,.
we might have this idea that Jesus is the Lamb of God,.
but we're not really pointing towards him..
In fact, most of our lives are actually pointing.
in the opposite direction..
Rather than saying, "Look, the Lamb of God.".
Here's what I think we often end up saying,.
"Look, the servant of God.".
And we put ourselves once again in that first place..
If Advent is anything,.
if Advent is this call to us.
to prepare ourselves for the coming of Jesus,.
both individually and as a community,.
we must ask ourselves, "Who are we pointing towards?.
Who is getting my focus?".
And I wanna invite you to reflect on that right here.
in the middle of this message..
I want you to take a moment now to ask yourself,.
"Is my life really pointing to Jesus?.
I mean, is he really my focus point right now?.
Or am I actually kind of pointing towards myself?.
Or I'm pointing towards my career?.
Or I'm pointing towards this job that I have?.
Or I'm pointing towards something else in my life?.
Or I'm pointing towards my social media that's about me?.
I'm pointing, whatever it might be,.
we find ourselves pointing in lots of different directions..
But are we releasing people through our lifestyle,.
through our words and our actions and our voice?.
Are we releasing people to follow the Lamb of God?.
Or are we inviting people to celebrate the servant of God?.
May I open your hearts to this thought right now..
And the way we're gonna do that is in a moment,.
I'm just gonna invite the worship team..

$^{281}$They're gonna come and actually lead us in a song..
And this song won't be too familiar to you..
It'll probably be new..
And I wanna encourage you to read the lyrics..
It's a song that celebrates this power.
that is in the name of Jesus..
This idea of the King being kings,.
that he is one that's seated on the throne..
It's a song that celebrates pointing to the Lamb of God..
And so as you reflect on this,.
as you listen to the song,.
as you allow the lyrics to take heart in yourself,.
allow the Holy Spirit to speak to you..
And if there's any way that he shows you.
that you haven't been pointing in the right direction,.
just in the beautiful kindness of his presence,.
bring it to him..
And after we finish this song,.
I'm gonna come back to us.
and I'm gonna show us what happens.
in the rest of the story..
♪ There is a King seated among us ♪.
♪ Let every heart receive him now ♪.
♪ Where there is praise, he will inhabit ♪.
♪ And there will be grace and mercy all around ♪.
♪ Every burden will be lifted in his presence ♪.
♪ Every trophy will be laid down at his feet ♪.
♪ There is a name that reigns above all others ♪.
♪ Jesus Christ, the King above all kings ♪.
♪ Unto the lamb, honor and glory ♪.
♪ Worthy is he who overcame ♪.
♪ Buried in shame, risen in power ♪.
♪ He is alive and the stone has rolled away ♪.
♪ And all our worship will belong to him forever ♪.
♪ Death is conquered and our savior holds the keys ♪.
♪ There is a name that reigns above all others ♪.
♪ Jesus Christ, the King above all kings ♪.
♪ And it won't be long, we will behold him ♪.
♪ And every tear he'll wipe away ♪.
♪ We'll be at home, the war will be over ♪.

$^{321}$♪ And soon we will meet our savior face to face ♪.
♪ And all our worship will belong to you forever ♪.
♪ Holy, holy for all eternity ♪.
♪ Yours is the name that reigns above all others ♪.
♪ Jesus Christ, the King above all kings ♪.
Let's take a look at what happens next in our story.
as John continues to release people to Jesus..
I'm gonna pick it up in verse 35..
"The next day, John was there again with two of his disciples..
When he saw Jesus passing by, he said, 'Look, the Lamb of God!'.
When the two disciples heard him say this, they followed Jesus..
Turning around, Jesus saw them following and asked, 'What is it that you want?'.
They said, 'Rabbi,'" which means teacher,.
"'Where are you staying?'.
'Come,' he replied, 'and you will see.'.
So they went and saw where he was staying and spent that day with him..
It was about the 10th hour.".
I love this..
This is the next moment in the narrative..
And it says right again, "The next day,".
so immediately after what we had just been talking about before,.
here's John and he sees Jesus again in the crowds..
And he once again shouts out and points out, "Here is the Lamb of God!".
But something different takes place this time..
Two, just two, but two of John's disciples.
who'd been with John the whole time,.
who had seen John and admired John.
and modeled their lives after John,.
who had followed him,.
are then aware that this one is the Lamb of God,.
that this one is the one who is about to take the sin from the world..
And so they leave John and they begin to walk in his direction..
They begin to come towards him, to follow after him..
And it seems like such a small thing in the text,.
but it actually is so much..
It's actually so powerful..
These are the first two disciples of Jesus..
And it challenges me because here's the thought..
See, I think we can very easily be people.
who can point out that Jesus is the Lamb of God.

$^{361}$and yet fail to actually follow him ourselves..
I think one of the dangers of the church,.
I'm gonna be really honest,.
I think one of the dangers of the church.
is that we can be filled with people who sing about the Lamb of God.
and yet fail to truly follow him..
And what can end up happening,.
particularly I would say in a church like the Vine sometimes,.
is that people can end up following a brand,.
following a church,.
following a speaker or a leader,.
and not actually following Jesus..
If there's something we are completely about here at the Vine.
is that we must decrease, he must increase..
That we must be less, he must be more..
That is the whole beauty of the Christmas story,.
that in the incarnation one comes,.
who takes that power and authority.
and all the thing that can seduce and corrupt us.
and brings it to someone else,.
where we can stand back and go,.
"There's the Lamb of God,".
and actually follow him,.
be released to give our lives to him..
I mean, what a joy that is..
What a gift that, my friends, is the heart of the gospel..
Notice what happens when we follow after Jesus..
It says in verse 38,.
"Turning around, Jesus saw them following him,.
and he asked them, 'What do you want?'".
I love this..
This is classic Jesus, right?.
So there's Jesus just going about his normal day, right?.
And he's got a couple of dudes following him now..
And he's like, "Have you ever been in like Ikea?".
I've done this many times..
It drives me nuts..
You're in Ikea, there's hundreds of people around you,.
and you're walking and you turn left,.
and then some people turn left with you,.

$^{401}$and you turn right,.
and some people turn right with you,.
and you want to turn around and go,.
"What do you want?".
Like, "What's going on?".
This is like the classic Hong Kong crowd thing..
Here's Jesus..
Got two people for the first time following him..
And he turns around..
When you read it in English, it's like almost Jesus is like,.
"What do you want? Why are you following me?".
It's beautiful..
Actually, in the Greek, if you read it, it says it more like this..
"What is on your heart?".
Isn't that cool?.
Like in the English, it's like, "What do you want?".
But in the Greek, it's more, "What is on your heart?".
In other words, Jesus turns to these two people,.
and he says, "Why are you following me?.
What is your desire?.
What are you trying to seek?.
What is ultimately in your heart?".
It's a powerful question..
It welcomes us into a different place..
It reminds us that following Jesus.
is not about Jesus walking ahead of us,.
and us just trying to diligently kind of walk in his footsteps,.
like we're trying to be perfect like Jesus..
Following Jesus is a relationship..
Following Jesus is a dialogue..
It's a community..
It's a chance to actually commune with the creator of all things..
When we follow Jesus, we're not distant to him,.
trying to keep up with him..
We're walking alongside with him..
In fact, actually, the Greek word for following here.
means actually to journey on a path.
or to go on a path with someone..
So you're walking almost alongside of Jesus..
And there's this kind of dialogue that you're having with him..

$^{441}$That's the beauty and the simplicity of what it is to follow Jesus..
If you want to know just basically what does it mean to follow Jesus,.
it means to live your life in dialogue with the creator of the universe,.
to have a conversation with him,.
to listen to him, to hear him, to ask of him..
And here are these two disciples that Jesus turns to in this moment,.
and he says, "What is on your heart?".
Imagine if Jesus turned to you right now and said those words to you..
What is on your heart?.
How would you respond?.
Maybe you might say, "All right, yeah, I've got a question for you, God..
I want to ask you, why is it that bad people always seem to get the good stuff?".
Or maybe you might ask, "Yeah, God, why is it that when I pray sometimes,.
it feels like my prayers don't get unanswered?".
Or maybe you'd want to ask, "Why does bad stuff happen to good people?".
You know, like there's a whole bunch of questions that we might have on our hearts..
If God turned to us and said, "What is on your heart?".
And I want you to hear what these two disciples respond with..
Could have asked anything of the creator of the universe..
Here's what they say..
In verse 38, they say this, they say, "Rabbi," which means teacher, "Where are you staying?".
Like, where are you staying today?.
Because we want to come and be with you..
We want to just spend our time with you..
We want to sit in your presence..
We want to be with you..
If you want to know what following Jesus essentially is, it is exactly that..
It is discovering where Jesus is in your life and spending time with him..
Just desiring to be present with him, wanting to walk side by side with him,.
wanting to be in the very places in our city where he is..
The invitation of their hearts is to say, "You're the lamb of God..
You're the one who's going to take the sin away from the world..
You're the one who's rightfully first..
All power and authority and majesty and glory is yours..
We just want to be with you.".
That heart cry of being a follower of Jesus is as simple as that..
You don't have to be learned..
You don't have to be excellent..
You don't have to be perfect..
You just have to carry that desire that says, "God, I just want to be with you..

$^{481}$I want to know you..
I want to hear you..
I want to sense your presence with me..
I want to see what it is that you're up to..
Would you lead?.
I want to follow..
Would you reveal?.
I want to see..
Would you tell me?.
Because I want to learn.".
It's a dialogue and a community with the creator of the world..
I mean, what could be better than that?.
And I love how Jesus responds to them..
He says this..
He says, "Come," He replied, "and you will see.".
Think about this for a sec..
What is on your heart?.
We want to know where you're staying..
You want to be with me?.
Come then, and you will see..
Jesus invites them to not only be with Him, but to see the things that He is about to do..
And I think this is so profound..
If there is a sentence that sums up all of Advent, where Advent is about the idea that.
I get to come to be with Jesus, that I get to come to prepare myself for the coming of Jesus..
That's what Advent is all about..
Here's Jesus turning Advent around, and He's saying, "You know what Advent is for Him?.
It's for Him to say, 'Come to me..
You come to me, and I will show you what I'm up to.'".
Jesus stands over your workplace, and He says, "Come, and you will see.".
He stands over your marriage, and He says, "Come, and you will see.".
He stands over those relationships that are broken in your life right now, and He's saying,.
"Come, and you will see.".
He stands over your pain and your hurt of 2020, and He says, "Come, and you will see.".
He's present in the worst places, in the places that are the hardest to find..
He's present in every aspect of our city, and He's crying out to us, and He's saying,.
"Come, and you will see..
I will show you what I'm doing in all the places where you thought I was absent..
Emmanuel, God with us.".
This is the Christmas message church..
This is the shepherds in the fields hearing the angels say, "Come, and you will see.".

$^{521}$This is the magi wondering what is ahead of them, following a star that's telling them,.
"Come, and you will see.".
Christmas is the manger that says to the world, "Come, and you will see.".
Where is it in your life that you want to see the activity of Jesus?.
You're called not for first place..
You're called not to increase, but decrease..
You're called to that great and glorious gift of following after Jesus,.
and in doing so, being in dialogue and community with Him,.
wondering in your heart where He is, and then in the miracle of faith, seeing Him show it to you..
Him saying those words, "Come, and you will see.".
Where for you do those words resonate most today?.
Where as you prepare your heart for Christmas on Friday,.
are you longing to see Jesus the most?.
That's what I encourage you to bring your prayer to today..
As we finish our time in this series, as we finish Advent,.
as we get ready to celebrate Jesus's birth on Christmas day,.
we do so from that posture of following after Him, hungry to see Him,.
revealed more in Christmas in 2020 than He has ever been in any time of history..
Maranatha, come Lord Jesus..
Let me pray for us..
Let's pray..
Father, for whoever is listening to this right now,.
where this is a word in season for them,.
maybe some who are listening have put themselves in the place of first,.
where they've been saying more, "Look the servant of God,".
rather than, "Look the Lamb of God.".
Lord, if that's the reality for any of us,.
thank you that we can bring that to you in this moment..
Thank you that you love us, you forgive us, and you release us, Lord..
And so we come in that place of repentance as we've spoken about every week..
We ask you now to wash us clean again, Lord..
If we've allowed ego and pride to center the universe around ourselves,.
would you forgive us, Lord?.
And may we take on that mantle of John the Baptist today..
And may our lives point people to seeing Jesus..
May we decrease, may Jesus increase..
Or maybe you've been listening to this today,.
and for you, in all of the things that have happened this year,.
and all the stuff,.
maybe you need to get that fire again in your heart that says to the Lord,.

$^{561}$"God, I want to follow you..
I want to just be where you are..
And I realize I've been distracted by so many things, but it's simple..
I just want to follow you, Lord..
I want to be in conversation with you..
I want to have dialogue with you again..
I want to hear your words spoken after me..
Show me where you are, Lord.".
And for some of you today, those words are a great encouragement to you..
Come and you will see..
And wherever it might be that you're looking for a breakthrough,.
wherever it might be that you're desiring for more of Him,.
hear the word spoken by the Spirit over you..
Come and you will see..
Lord Jesus, for anyone here who's listening to this right now,.
I want to pray that that would be a reality for them this Christmas..
That you would show them where you're staying in their businesses,.
in their marriages, in their families, in their relationships,.
in the places of healing they need,.
and all of the things that are going on that are stressing them..
Would you reveal the power of your glory to them?.
And would you show yourself new and afresh this Christmas?.
And we pray this together in Jesus' name, Amen..
\newpage



\section{}
\label{sec:KS3UcRDETW4}
\textbf{2020-12-27 Church Everywhere Live: New Years Message [KS3UcRDETW4].mp3}
\newline
\newline
連結: \href{https://youtube.com/watch?v=KS3UcRDETW4}{\texttt{ https://youtube.com/watch?v=KS3UcRDETW4}} ~~~~ 語音日期: 2020-12-27 
\newline
\newline
\hyperref[sec:yAN1zxPhVpI]{\small{< < < PREV SERMON < < <}}
~
\hyperref[sec:index]{\small{[返主目錄]}}
~
\hyperref[sec:LHgF2voP5ys]{\small{> > > NEXT SERMON > > >}}
\newline
\newline
$^{1}$Well, good morning, everyone..
Thank you so much for joining us here today..
2020 is coming to an end..
Can I hear an amen?.
And some of you are saying, finally..
It was about time..
Hey, to start things off in a bit of a lighter way,.
I want to do a review that I call Remember When..
Now, remember when the pandemic first hit.
and we became experts on sourcing masks?.
Now, I want to share with you about my moment of glory..
I feel really good about this..
I was at a friend's New Year's party,.
and word came out that HKTV Mall was.
going to be selling some masks..
And so we all got online on our phones,.
and we got ready to the strike..
And when those masks became available,.
I managed to secure 90..
And I was so excited..
I felt like I finally arrived in Hong Kong..
I know how to take care of stuff here..
You know, that's not that easy..
But then I also realized that I actually.
had an oven sitting in my HKTV Mall cart.
that I not only bought masks, but I also bought an oven..
So I got masks and an oven..
But hey, you know, in the pandemic,.
that wasn't that bad of a thing..
But I just was not as cool as I thought I was..
Now, we managed to source masks from all over the world..
You know, some of us on our own..
Hong Kong people are really resourceful..
They know how to get stuff..
Some of us collaborated with friends, family, co-workers,.
and they sent us masks here, only for us.
to send them back to them as the pandemic spread.
and when they needed masks and didn't have them..
Remember when we could go on vacations.
and didn't have to quarantine?.

$^{41}$Yeah, me neither..
Man, I miss traveling..
You know, that's one of the things I was really excited.
about coming back to Asia was the travel..
Remember when we became Zoom experts?.
We're like, hey, virus, we're going to bypass you..
You have nothing on us..
We're going to go digital online..
And so we learned about things like a virtual background,.
which works for some of us..
We have good enough computers and chips.
in our computer..
For others, not so much..
We learned about Zoom fatigue..
Anyone experience Zoom fatigue?.
Hey, you can put a comment in the feed.
if you know what Zoom fatigue's like..
I certainly have experienced that..
But we also learned about Zoom etiquette..
Well, that's at least some of us..
There's always that one person on the call.
who does not mute themselves..
Hit the mute button..
Come on, you can do it..
Hey, remember when-- and by the way,.
this is the height of my review right here..
This is the zenith, the pinnacle..
You know what I want to share with you?.
Remember when we were scared of running--.
you guessed it-- out of toilet paper,.
and we got in touch with our inner hoarder,.
that little man, that little woman living inside of us?.
And we went into the stores and into drugstores,.
and we grabbed as much toilet paper and tissue.
as we could to make sure that we're secure..
Now, a neighbor from me, him and I,.
we can look into each other's windows..
You know, we pretend like we don't see each other, but we do..
And so during that time, he had like stashes and stashes.
of like toilet paper and tissue there..

$^{81}$And I started feeling insecure about what I had..
But he was like, that's not cool, man..
Why you got to take all the toilet paper?.
But remember also when we actually slowed down,.
and we took time with God, reading his word,.
spending time in prayer, experiencing him speaking to us..
Remember when we got to spend a lot of time at home,.
work from home, and even though there's challenges in that,.
but nonetheless, we got to experience these precious moments.
with people that we love, with our parents,.
with our children, with our spouses, with our friends..
Remember when we called each other,.
and we actually checked in to make sure the other person was okay..
And we went out on the streets, and we handed out hand sanitizer..
We handed out masks for those who couldn't afford them.
or wasn't able to source them for themselves..
Hey, what's your highlight of 2020?.
Let us know about it..
Put it in the comment..
We would love to hear your highlight..
Love to celebrate with you the good things that God has done this year..
But even though a lot of good things happened,.
and they certainly have, and praise God for them,.
2020 has brought so many of us to our breaking points..
And to one degree or another, we've experienced loss,.
we've experienced pain, exhaustion..
Traveling has been difficult..
Some of us have missed funerals, weddings of loved ones..
I got married this year to my lovely wife, Chris..
Woo-hoo!.
And we started with a civil union,.
which happened during the peak of the second wave..
So we had to uninvite 30 people, which is awkward.
if you ever have to uninvite people to something..
Not fun..
Then we had our church wedding during the peak of the third wave,.
so we had to shift everyone from actually being here in person.
to now joining us online, and we had to delay our banquet..
But by God's grace, we were able to have our banquet.
between the third and the fourth wave, and I'm really grateful for that..

$^{121}$But this year has been tough..
So many difficult things have happened..
Some of us have lost loved ones..
And if that's you, I just right now just want to send you the love.
from myself but also from the whole Vine family..
We love you and we care deeply about what has been happening in your life,.
and we just want to send you our love and hugs to you and to your family..
Many other things have happened..
You know, we've been scared of the virus..
We've been living in fear..
We've lost jobs..
We had to postpone our college plans..
Graduations were online instead of in person..
Church everywhere online..
Hey, thanks so much, you know, for joining us in this space,.
but we sure miss you here in person..
But we're still glad we get to spend this time with you in this way..
Parents became teachers at home, having to figure out how to navigate.
this whole working with their kids and collaborating with the class teachers.
and getting all the lessons done so that the kids don't fall behind..
Friends have lost friends who've moved..
So there's grieving people who have moved..
And then there's a lot of uncertainty about the future..
What's going to happen in 2021?.
Is there going to be a recession?.
Am I going to have my job or is the vaccine going to address all the issues?.
So I don't know what your 2020 has been like,.
but to one degree or another, we've all experienced pain and loss,.
and there's been challenges this year..
And so as I was thinking about what do I want to share with you today,.
you know, what is God putting on my heart,.
this passage came to mind that has been with me throughout the whole year,.
that has come up multiple times..
And that's a passage from the Gospel of Matthew,.
chapter 9, verse 35 through chapter 10, verse 5..
And the reason why I want to choose this passage to share with you right now.
is because I believe that it's important for us to hear today,.
you know, that Jesus cares deeply about us,.
and he's aware of our circumstances..
He's aware of what's going on in your life..

$^{161}$But some of us were struggling to see that.
because this year has just blindsided us..
It's been so difficult..
And we don't feel like that Jesus is with us..
We don't feel like that Jesus cares, but he does..
And so in this message, I'm going to speak into the challenges of 2020,.
but I'm also going to be looking ahead.
and how Jesus invites us into very important work in 2021,.
the River Model Vision,.
you know, something that Pastor Andrew introduced two years ago in a Vision Sunday..
And so we're going to see how Jesus wants to speak to us now here in our challenges,.
but we're also going to be looking ahead..
So that's the plan for our time together..
So let me start by reading this passage in Matthew, chapter 9, verse 35..
And it says, "Jesus went through all the towns and villages,.
teaching in their synagogues,.
proclaiming the good news of the kingdom,.
and healing every disease and illness..
When he saw the crowds, he had compassion on them,.
because they were harassed and helpless,.
like sheep without a shepherd.".
So first, we hear a summary of Jesus's ministry up until this point..
He's been going from town to town..
He's proclaiming the good news of the kingdom,.
but he's also demonstrating the kingdom by bringing healing.
and bringing wholeness to people..
But then Matthew, he mentions a metaphor of sheep and shepherd.
in order to help us to understand how Jesus sees the crowd..
That's actually a metaphor that Jesus himself uses elsewhere.
about himself, himself being the shepherd,.
but also Israel being the sheep..
So for example, right after the passage that we're going to be looking at today,.
Jesus sends the disciples out to the lost sheep of Israel..
So Israel are the lost sheep..
And then in the Gospel of John, he calls himself,.
he says about himself, "I am the good shepherd.".
So not only did Jesus talk about this,.
but this was something that Jesus's contemporaries,.
they were familiar with..
It was a metaphor they knew,.

$^{201}$because throughout the Bible,.
there's this metaphor of sheep and shepherd..
Most of the time, it refers to sheep being Israel.
and shepherd actually being the leaders of Israel..
And a lot of times, it's a critique of their lack of leading Israel..
So Matthew, in our passage, he's using this metaphor.
to help us understand how Jesus sees the crowds.
and what's going on with them..
And so as he's been walking around,.
he's been making these observations,.
and he sees that the people, they're without protection..
They're vulnerable. They're harassed..
They're helpless..
They find themselves in a place of destitute..
So during that time, there was Roman occupation..
There was an oppressor that really turned up the screws.
or turned down the screws tight,.
you know, taking money..
So a lot of people were struggling economically..
But then also just not having the freedoms.
that the people desired to have..
But then there was also things like spiritual oppression..
And then just normal life, economic concerns, heartbreaks..
And so what Jesus sees when he looks at the crowd of people.
is he sees people that are barely making it..
That's what Jesus sees..
And he sees their struggle, and he sees their pain..
And he sees their lack of resources.
and their place of destitute.
and their need for somebody to come and lead them.
out of that place, to guide them..
But it does not just stop there..
Jesus not only sees and observes the struggles.
and the pain of the people..
You know, that'd be like, just imagine that..
Like he's like, "Oh, there's a lot of pain..
Oh, well, that sucks.".
You know, and then he just moves on..
But that's not what he does..
We're told in verse 36 that Jesus had compassion on them..

$^{241}$He allows himself to be deeply moved and touched.
by what's happening in people's lives..
He cares deeply about every single person.
and what's going on in their lives..
Every single person..
Tim Keller, a well-known pastor from New York City,.
he says this, "Compassion means to get yourself involved.
and make yourself vulnerable.".
So it's two things..
Get yourself involved, make yourself vulnerable..
Well, Jesus, he made himself vulnerable..
We just celebrated Christmas..
It's all about Jesus making himself vulnerable.
and coming to us, coming into our lives to rescue us,.
but also to see and identify with our pain..
And so he makes himself vulnerable..
He allows himself to be impacted by that,.
but it does not stop there..
He also gets involved..
He gets involved in people's physical pain,.
and he heals them..
He gets involved in people's emotional pain.
by taking away shame, restoring people's dignity,.
and comforting them,.
and telling them about their true identity..
And he gets involved in people's spiritual pain,.
whether that's spiritual oppression or sinfulness.
that is just working havoc in our lives..
And so he lays down his life.
and all those who place their trust in him.
can experience his deliverance.
and this newness of life that he gives..
So Jesus is not a passive bystander..
He gets involved, and this love and compassion.
that a lot of us and a lot of you.
that are watching here today have experienced,.
you know, it flows through Jesus.
because he deeply cares about every one of us..
And Jesus, he sees your 2020,.
and he cares deeply about what you are going through..

$^{281}$He's deeply touched and moved.
by what is going on in your life,.
and he wants you to know that today.
because we need to hear that..
Because just like those people back there,.
there's a lot of us barely hanging on..
You know, we're barely making it..
And we need to hear this word today.
that Jesus, he knows, he sees, and he is involved..
But there's a problem..
When things don't turn out the way we hope for,.
or when we walk through a lot of painful experiences,.
we tend to withdraw from Jesus.
and we withhold our struggles from him..
Now, in our Bible passage today,.
we see that even when we are in that place of withdrawal.
or pain or hurt, Jesus draws near to us..
He has compassion for us..
He cares about our struggles, your worry, your fear,.
your grief, your pain, your anger,.
your sadness, your frustration, your helplessness..
He cares about you and he wants you to know that,.
and he wants you to know that right now..
Jesus says about himself, "I am the good shepherd.".
And the good shepherd lays down his life for the sheep..
Jesus doesn't withhold anything from us, nothing..
He has given everything for you, everything for me..
He's not a bad shepherd..
He's the good shepherd..
Bad shepherds abandon their sheep..
They leave them vulnerable and exposed,.
but that's not Jesus..
Jesus doesn't leave you vulnerable and exposed..
He's with you every step of the way..
And he's also the one and the only one who is powerful.
and able to lead us to our destination..
And so right now, I want to invite you to pour out your heart.
in prayer to him and to experience his comfort,.
his care, his healing presence as he meets you right now..
So I want to encourage you at home..

$^{321}$You know, it doesn't matter who's around you,.
even if you're on the go and watching this..
I want to encourage you right now just to do some real talk.
with Jesus..
Just tell him what you're feeling and what's going on..
You know, even if you're barely struggling to do that,.
you know, that's actually a step of faith..
And even if it's so hard, and when we do that,.
Jesus, you know, he reveals himself to us.
and we get to experience him working..
So right now we're going to take like 20 seconds or so,.
and I just want to invite you to pray..
So if you close your eyes with me right now,.
wherever you are, and just pour out your heart to Jesus..
Jesus wants to meet you right now..
He wants you to know that he is with you..
He has not abandoned you..
And he wants you to know his love,.
and he wants you to have hope..
I'm with you..
You're going to make it..
I'm going to carry you through..
So pour out your heart to him right now..
(soft music).
Jesus, you've heard right now all the prayers..
Even if it was just like some stuttering or some,.
I'm in pain, some very simple words..
And Jesus, I pray right now that your presence,.
wherever we are, that you would just fill that space.
with your presence, that you would make yourself known,.
that you would start the healing process,.
that you lift burdens off people,.
that we could feel that release that you can only bring..
So Holy Spirit, I just pray that you would move.
powerfully right now..
Lord, we desire that we pray for that,.
and we know you're able,.
and we know you are the good shepherd.
who doesn't abandon us, who is still with us,.
and who will see us through..

$^{361}$So I pray that you would release faith, hope, and courage.
right now, in your name, Jesus..
Amen..
(soft music).
Now, besides Jesus meeting us in this difficult space.
that a lot of us are finding ourselves in,.
and by the way, I do wanna encourage you.
to keep just pouring out your heart to Jesus.
and experiencing him meeting you..
Even if you don't know him, you can talk to him,.
and he'll meet you, he'll make himself known to you..
But besides Jesus showing us his compassion,.
I want us to think about what does Jesus see.
when he looks at Hong Kong in 2020?.
What does he see?.
What does Jesus want his church to see and do?.
And I believe that the river model vision.
that we as a church have been called to.
is about seeing our neighbors' circumstances.
and allowing ourselves to be filled with compassion.
for our neighbors, and to get involved in people's lives..
Jesus wants his compassion in 2021 to flow through us.
to our neighbors, to the people here in the city.
that need his love, his compassion,.
that need to be touched,.
that need to know that there is a good shepherd..
So as we step into 2021,.
we need to see our neighbors struggle..
So as you have been looking around your neighborhood,.
what are you seeing?.
What is the spiritual, emotional, and physical condition.
of the people around you?.
Who is Jesus highlighting to you?.
Actually, I want to ask you to comment right now.
on the feed below..
Let us know what has been Jesus highlighting to you..
What are you seeing?.
What are you observing?.
Now, I believe that Jesus, he highlights specific people.
to us, that we are uniquely positioned.

$^{401}$to love, to serve, and to impact..
So what are you seeing?.
What are you noticing?.
Do you see the elderly neighbor that's barely making it,.
who needs some help?.
Do you see asylum seekers,.
our brothers and sisters who are struggling,.
especially now even more so than before during COVID?.
Do you see people's spiritual condition?.
Maybe they have the financial resources.
and they're doing it from that perspective, fine,.
but they're just lacking that joy.
or that purpose in their lives..
And you see that and it pains you..
What is Jesus highlighting to you?.
Do you see the single parent.
that needs others to come around him or her.
and just walk with that parent?.
And with the child?.
You know, what do you see?.
Whom is Jesus highlighting to you?.
Where do you find yourself filled with compassion,.
deeply moved and stirred?.
Pay attention to that..
Could it be that Jesus wants to work uniquely in that area.
that you see and where you are moved to compassion?.
What do you think?.
So as we make those observations about the people around us.
and we become aware of the needs,.
you know, there's usually two ways that we respond to that..
One of them is, I just wanna get involved right away.
and I jump right into action..
And a lot of times we burn out when we do that.
or it's just a short lift, you know,.
we do it for a little bit and then we move on..
The other reaction, I think this is the,.
probably the more common one,.
is that we see our neighbors,.
we see their need,.
we start to understand the complexities.

$^{441}$of the issues that they're facing.
and we just get so overwhelmed..
And we feel so small and we're like,.
I don't know what I can do or where to even start..
And then even though we're moved,.
we don't know what to do with it.
and then after a while we just,.
in order to not have this cognitive dissonance,.
we set it on the back shelf.
and we just move on with our lives..
And I think that's the way that Satan snatches away.
those things that he has, the Holy Spirit has sown.
and that he wants to move through us this year in 2021..
But Jesus, he actually provides us some perspective on that.
and so I wanna share it with you,.
continuing in Matthew 9, verse 37, it says,.
then he said to his disciples,.
the harvest is plentiful, but the workers are few..
Ask the Lord of the harvest,.
therefore to send out workers into the harvest field..
And I love this here..
Jesus acknowledges that the difficult reality,.
the harvest is plentiful, there's so much to do,.
but the workers are few..
But he also tells us what to do.
and it's not what we tend to do,.
it's not jumping right into.
or disengaging and just feeling overwhelmed.
and not doing anything, it's neither of just two things..
He tells us in this overwhelming need to start with prayer..
And it's prayer that propels us forward.
in responding in a powerful way.
to the needs that we see around us..
It's prayer that leads us and propels us.
into living out the river model vision in 2021..
Besides the Lord's Prayer, by the way,.
this is the only time where Jesus tells us.
specifically what to pray for..
Where he actually gives us the content, pray this..
So we should pay attention to that..

$^{481}$And Jesus is saying, don't get overwhelmed,.
don't get stuck in that place,.
don't just rush also into it,.
but pray and remember who I am..
I'm the Lord of the harvest, I'm already,.
I'm the good shepherd, I'm already working in people's lives..
I'm the one who gives divine appointments,.
I'm the one who transforms and who changes lives..
I'm the one who has all the resources that are needed.
to love this city..
Start with prayer..
And when his church prays,.
and when his church prays, guess what happens?.
The river starts flowing.
'cause Jesus starts sending out his river workers..
Now, the word for being sent out, and I love this,.
this is actually my favorite little nugget.
from this passage, is ekbalo..
And it actually can be translated as cast out,.
flung out, drive out, thrust out..
Now, just think about this..
Jesus is telling us to pray that we, his people, the church,.
is gonna be cast out..
This is the same word that Jesus used to cast out a demon..
So he's telling us to pray for workers to be cast out..
Now, why is that?.
Well, I think the answer is obvious.
'cause a lot of times we're living in our comfort place.
and we're just dealing with our own lives..
Or we are living in this place of just being overwhelmed.
and we need the powerful move of the Holy Spirit.
to light a fire in our belly to go out.
and to do the work that Jesus is inviting us into,.
his healing, his redeeming, his restoring work..
So I wanna challenge you this year to pray.
that Jesus will cast us out as his river workers.
all over Hong Kong..
And in January, we are having our fast,.
so that's a perfect time, you have no excuse..
That's a perfect time to be praying..

$^{521}$So let's see what happens as we do that..
Now, in Matthew 10, one through five,.
so we're gonna continue looking here at this passage,.
and I'm not gonna read all of this..
I'm not gonna butcher all the disciples' names,.
but I'm gonna just read a couple of verses..
He says, "Jesus called his 12 disciples to him.
"and gave them authority to drive out impure spirits.
"and to heal every disease and illness.".
And then there's a name of the 12 apostles,.
and I continue in verse five..
He says, "These 12 Jesus sent out..
"The disciples prayed for workers to be sent out,.
"and then Jesus sends them out..
"They're the answer to their prayer..
"When Jesus commissions them as apostles,.
"as sent out ones, he gives them authority.
"to cast out evil spirits and to heal.
"every disease and affliction.".
And in other words, he provides for them.
the gifts of ministry that they need to do his work..
And guess what?.
When we pray about being cast out,.
now Jesus, he also, when Jesus sends us out.
to carry out the river model vision,.
he empowers us for the task at hand,.
and he also gives us the specific gifts of ministry.
that we need to carry out his work..
This is something that we can be encouraged by,.
because Jesus, he sets us up for success, not for failure..
He calls us into something that we can actually do..
Now, in this passage, there's a significant shift.
and transformation..
The disciples of Jesus, they've been sort of tagging along.
with Jesus and watching him,.
and it must have been pretty cool to watch him..
I wish I could have been one of the disciples.
during that time and seen Jesus in that way in action..
But they've been watching him..
And so now, through this prayer,.

$^{561}$there's a transformation..
They become active participants in Jesus's work..
They're no longer just observers..
They're active participants..
Now, I believe that Jesus is calling us,.
he wants to do a shift in us as a church.
from watching maybe ministry leaders,.
or there's always a certain group of people.
who do a lot of watching them and just being observers..
He wants to shift us to being his river workers.
and active participants in what he wants to do.
in Hong Kong in 2021..
Now, this shift, it's a supernatural thing..
We can't force that on our own..
It's the Holy Spirit working in us..
And so let's pray for that..
So how can we step into 2021.
and start living out the river model vision?.
How do we do that?.
Well, first, start with prayer..
Pray for workers to be cast out into the city.
as his river workers..
Next, realize that Jesus, he uses the word workers..
The Greek word for workers,.
here used here is ergates,.
which simply means an agricultural worker or laborer..
So Jesus, he tells us to pray for workers..
Not for expert community developers, theologians,.
particular personality types..
If someone has those skills or that personality,.
that's awesome..
But he tells us to pray for ordinary people.
like you and like me to go out and do his work..
But we struggle with that, don't we?.
A lot of times we feel inadequate.
or we don't feel qualified.
or we feel like we're lacking something.
or we personally become 50 times more mature..
And so we don't think that God can use us.
and work through us..

$^{601}$However, even small acts of service can be used by God.
to achieve his purposes.
because we are not acting in our own strength,.
but out of God's strength.
and we're sent by his authority.
and we're equipped with his giftings..
Now let me share with you a true story..
This is a story from the US..
There was a high school student and he was walking home.
and then he was looking over and he saw this other student.
and the student was dropping his books..
He was carrying a lot of books,.
so he decided, you know what, I'm gonna go over..
And so he helped the student pick up the books..
And then he was thinking to himself,.
hey, if I don't help this person carry home,.
I'm just gonna keep dropping them,.
so I'm just gonna walk home with them..
So he helped him carry the books home..
And they started talking and they had a great conversation.
and a friendship grew out of that..
So several months later,.
the student that had dropped the books,.
he told his friend,.
hey, do you remember the day when we first met?.
Yeah, yeah, of course I do..
He's like, do you know why I had all these books.
with me that day?.
He's like, no..
Well, I had gone to school,.
I had cleared out my locker.
'cause nobody talked to me, nobody cared for me,.
and I was gonna take my life..
That day, you saved my life..
Never underestimate small acts of kindness and compassion.
and how God can use them..
He wants to work powerfully through them..
He wants to touch people's lives.
through you here in Hong Kong..
He wants his love and compassion to flow through you..

$^{641}$Now, the river model vision,.
it's more about regularly practicing small acts of kindness.
and mercy than one-off grand gestures..
So it's about that daily dying to self.
and going that little bit out of our way.
to help somebody pick up books.
or to pay attention to somebody, listen to someone,.
to ask somebody out for lunch..
And don't underestimate how God wants to work through you.
and touch people's lives..
Now, I wanna leave you with two, what I call,.
and I'm really excited about this term, by the way,.
river habits that you can practice..
The first one is reach out to two people each week.
to see how they're doing..
It's as simple, every one of us can do that..
And I wanna encourage you to go above and beyond.
the people you usually reach out to or talk to..
So don't talk to your best friend,.
you're already talking to them anyhow..
And just give them a call,.
take some time to talk to your neighbor,.
talk to somebody at your workplace,.
but also talk to somebody at church.
that maybe you don't know that well..
So talk to two people every week and see how they're doing..
And put that in your calendar, by the way,.
if you don't put it in your calendar, it doesn't happen..
Second, the second river habit,.
bless someone this week..
Go again, above and beyond what you normally would do..
Buy somebody some awesome coffee,.
buy a family that's struggling with groceries,.
buy some groceries and get to know them better..
Encourage someone with a note.
or with some really thoughtful words.
that you're gonna share with them..
So I wanna challenge you to choose.
one of these two habits this week and to practice them.
and trust that God will work through them..

$^{681}$Tell someone that you're gonna do that,.
like invite your community group and talk to your spouse,.
talk to your friend and challenge them to do it as well.
and see what happens as Christ's compassion flows through you.
to the people around you..
The city that we love,.
the people that we love are in pain..
In 2021, will we make ourselves available.
to be God's river workers?.
Let me pray for us..
Jesus, you have seen our 2020,.
you care deeply about it,.
but not only ours, but also for this whole city..
And you want everyone to know you as the good shepherd.
and you want your love to flow through the city.
where there's hope and new life and transformation..
And Jesus, I pray that you would cast us out.
as your river workers in 2021,.
that we would say yes to that,.
that we would allow ourselves to be used by you.
for your kingdom purpose..
Now pray right now also for those who just maybe,.
we don't feel confident that you would give them,.
that you would release faith and encourage to do this,.
that even this week as we practice these habits,.
that we already would experience you working through that..
And also wanna pray for those.
who are still struggling and in pain,.
that you would bring people around them to walk with them,.
that we know that you are with them,.
that you would continue to reveal yourself.
and just lift those burdens.
and that you would speak to them, give them specific words.
that will guide them and give them direction and hope..
And so we pray that you would help us.
to wrap up this 2020 well,.
but also that you would bless our 2021..
And so into your hands, we commit that..
In your name, Jesus, amen..
\newpage



\section{}
\label{sec:LHgF2voP5ys}
\textbf{2021-01-11 Church Everywhere Live: Post Traumatic Growth - New Possibilities [LHgF2voP5ys].mp3}
\newline
\newline
連結: \href{https://youtube.com/watch?v=LHgF2voP5ys}{\texttt{ https://youtube.com/watch?v=LHgF2voP5ys}} ~~~~ 語音日期: 2021-01-11 
\newline
\newline
\hyperref[sec:KS3UcRDETW4]{\small{< < < PREV SERMON < < <}}
~
\hyperref[sec:index]{\small{[返主目錄]}}
~
\hyperref[sec:hd9p2DZoYtI]{\small{> > > NEXT SERMON > > >}}
\newline
\newline
$^{1}$Well, Church, I'm really excited. We are in week two of our Post-Traumatic Growth series..
It's a series where we wanted to start 2021 focusing on this amazing idea that's been discovered in psychology recently,.
that in the worst moments of our lives, we can actually experience the greatest moments of our humanity,.
that we can actually change and transform and grow and be renewed in ways that would not have been possible unless the trauma had happened itself..
That actually that trauma, that hardship, that pain, that suffering is actually the catalyst to our greatest moments of who we are,.
that we don't actually get defined by that trauma, that it doesn't become the place where we settle and put our roots down,.
but the trauma actually meets us in a place where we can find the hope of Christ rising us out of it in such a way.
that we couldn't have experienced the hope of Christ in that way unless we had been in the place of that trauma to begin with..
You know, the research suggests, and I spoke about this last week, that that growth and that transformation happens in us in predominantly five main ways..
It takes place in the idea of our appreciation for life, in our relationships with others, in new possibilities for life, in our own personal strength,.
and finally, in our own spiritual transformation, our own spiritual change..
And in this series, we're going to take a look at each one of those this week and over the next four weeks ahead, and we're going to unpack each one for you..
But as I was thinking about and praying about, what's the one that actually is the foundation of this series?.
What's the one that I think is most pertinent that we can speak to right up front to help prepare you for that change in your life?.
For me, it was new possibilities..
And right here at the start of the message, I want to speak this over some of you..
Some of you are primed today by the Spirit of God to be woken into new possibilities in your life..
When I was praying for this series, and particularly praying for this Sunday,.
I saw a picture for us as a church and for us individually that I want to share..
I saw a picture of a red traffic light..
It was just a very clear, very bright red traffic light..
And right in front of this traffic light, there was this bus..
And the bus was filled with people, and the bus was stopped, of course, at the traffic light..
But what was interesting in the picture was that the bus had been there for a very long time..
You could see dust on the top of the bus..
You could see that it was worn out and getting a bit rusty..
The tires of the bus were beginning to deflate..
You could see in the picture that the red light had been there for a long period of time,.
and the bus had been stuck in that place..
And what was amazing to me in this picture is I could see the people on the bus,.
and the people themselves looked almost like they had resigned themselves to this place..
That because perhaps they'd been stuck at the red light for so long,.
that the bus and the journey and the destination didn't matter anymore..
They were stuck right there..
And because they'd been stuck for so long, they had begun to be comfortable in that place..
Almost as if that was the place that they were supposed to be,.
and they had long forgotten the destination that they were actually journeying towards..
I want to just speak on this right now,.
because I think that will resonate with some of you watching this..
Because of all the stuff that had taken place in 2020,.

$^{41}$maybe some of the personal trauma that you're actually carrying yourself,.
maybe some of the hardship that you've been facing..
I know many of us have lost jobs during this pandemic..
We've actually lost family members and loved ones..
Many of us have struggled with some of the hardest moments in life..
And when we're like that, when we're in a place of trauma,.
it can almost feel like a red light is in front of us..
A stop sign is right there, and it sticks to us..
And we become almost weighted in that place..
Almost like we put roots down in the place that we're stuck in..
And we think that we can't go forward. We can't move on..
And the idea of post-traumatic growth almost seems funny to us..
It's like, how can I grow in the mix of the place that I'm stuck in?.
I want to speak to you right now and tell you this right up front..
You were not created for the red light..
Listen to this. You're going to experience red lights in your life..
We're all going to experience them,.
and we've experienced some of those in the past 12 months..
But you were not created to put roots down in the place where you're stuck..
And in fact, actually the Spirit of God is at work,.
I believe, in this series and everything that we want to do through it.
to begin to take you from that place of being stuck and move you forward..
I believe the word that sits over the vine this year is to move forward..
And if you're feeling stuck today,.
I pray that what we're about to unpack in this message about new possibilities for you.
will be an encouragement and a powerful prophetic change for you.
to actually begin to move towards the destination that God had always planned for you..
I want to inspire you about new possibilities today by sharing with you a story..
It's a story that's famous to us..
It's a story from Scripture, the story of Simon Peter..
And to help us to understand that story, I want to actually show you something..
So just come with me for a sec..
I want to show you the four primary elements that are found in the story of Peter's life..
And these four things are a net, they're a sword here, they're a fire and a sheep..
Let me show you those again..
Net, sword, fire and a sheep..
And those four things are what actually comes to define the person of Peter..
And those four things become actually the thing that we begin to see.
to shape the transformation of Peter into the new possibilities that Christ had for them..
So let me tell you the story of Peter's life..

$^{81}$We first meet Peter actually on the shores of the Sea of Galilee..
This is right towards the beginning of Jesus's teaching ministry..
And Jesus at this point in his life hadn't called any of the disciples to him..
And he's teaching one day right there on the shores of the Sea of Galilee..
And in fact the crowd around him is so large that he begins to get swamped by that crowd..
And he's so hemmed in by it that he looks around to see how he can kind of move away from them..
And he sees that there are two fishing boats right there..
And he calls one of the fishing boats over and that boat belongs to a fisherman by the name of Simon..
And Jesus stands in the boat and he pushes a little bit out from shore..
And he in that space can begin to teach the crowds..
Now the Simon who's the fisherman who owns that boat is kind of a bit overwhelmed by what's taking place..
He's just been out all night fishing and him and his friends have not caught anything..
And they've just come back right here at the early part of the day..
They're tired. They're worn out. They've just come back..
And suddenly this rabbi has gone in his boat told him to push out from shore..
You can almost sense that there's Simon kind of sitting in the boat going like,.
"Okay, come on buddy, you know, like let's get this going. Let's get this done. Let's move on.".
Well, Jesus finishes teaching the crowds and then he turns to Simon.
and perhaps understanding the frustration that Simon's feeling..
He says, "Hey, why don't you head out back into the heart of the lake and put your nets down one more time.".
Now Simon's an expert fisherman and he's not been able to catch anything all night..
He knows that this is just one of those days where the fish are not biting..
But perhaps out of respect to the rabbi, perhaps out of the fact of the crowds that were around,.
Simon obeys him, goes out into the middle of the lake and I can almost sense that he throws the nets out..
He's probably thinking, "Look, I'll prove to you that there's no fish here.".
And of course, we know the story. Peter catches the largest amount of fish that he has ever done..
I mean, it's so hard that they can't even seem to pull the nets in..
And Simon suddenly realizes that this person sitting in his boat is not just some traveling speaker..
He's not just some rabbi. This person is a holy man..
What's taken place is a miracle of God..
And so Simon falls on his knees before the rabbi and he says, "Lord," he says, "go away from me, for I am a sinner.".
Simon realizes in the light of this miracle and in the power that was obviously of God in this person,.
he was ashamed of the things that were sticking in him, the things in his life, the way that he was living his life..
And he said, "I'm a sinner. You must leave me.".
And Jesus, in a moment of compassion that we'll see take place throughout the gospel stories,.
leans over and places his hand on Simon's shoulder..
And he says this, he says, "You do not need to be afraid, for from now on, you'll catch people.".
You're not going to be a fisher of fish. You're going to be a fisher of people..
That's now your destiny..
And here's Peter, thinking that his life revolves around this lake..

$^{121}$And suddenly this rabbi, this holy man of God, calls him into a new possibility,.
calls him into a new time to leave his nets behind and begin to minister and walk out.
and try to reach the Jewish people, beginning to be understood that a rabbi,.
if a rabbi calls you and asks you to follow after him,.
it was one of the most precious, important things that you could hope for as a young Jewish person..
And so Peter and his colleagues, they move up to the shore..
They leave the boats behind. They store up the nets..
They walk away from them and they start to follow Jesus..
And over the next three years, they encounter some of the most unbelievable events..
They see miracle after miracle..
They see this one do things that they thought would never be possible before..
Turning water into wine, feeding five thousands, walking on water..
And Simon begins to wrestle with the idea that this is more than just a prophet,.
more than just a rabbi..
Perhaps this one might be the long-awaited for Messiah that we were hoping would come..
And Jesus, eventually sensing this in Simon, turns to him one day and says,.
"Who do you actually think I am?".
I mean, not what the crowds think I am, not what disciples think I am..
Who do you personally think I am?.
And Simon turns to him and he says, "You are the Christ of God.".
I mean, you are the Messiah. He names it. He sees it..
And in faith, he proclaims it..
And Jesus turns to him and says, "Okay, up until now, you've been called Simon,.
which in Hebrew means listener..
I'm now going to call you Peter, which in the Hebrew means rock..
And this new name now classifies who you are, another new possibility for you..
You are now going to be the foundation of my church..
Upon this rock, I will build my church..
The revelation of me as a Messiah and you now, Peter,.
as the one that I call to found and to build my church.".
The Bible tells us that as soon as that proclamation of the Messiah of Christ happens,.
and soon as Jesus changes his name,.
Jesus and the disciples begin to walk specifically towards Jerusalem..
And you can imagine for Peter how excited he would be..
He's heard the stories..
He knows what's going to take place when the Messiah gets to Jerusalem..
He knows it's going to be a moment of great celebration..
He knows it'll be the time when the Messiah will take his throne.
at the evil Greco-Roman Empire that had been so oppressing the people would be pushed back..
He understood that this would be the moment where actually sin would be once and for all dealt with,.

$^{161}$that there would be literally heaven on earth..
It was a time of victory, of celebration..
You can imagine how excited as they crest up that hill and they glimpse on Jerusalem for the first time..
And then they walk into the city and nothing is as Peter had predicted..
I mean, Jesus, the conquering victorious Messiah, doesn't do anything that Peter thought he was going to do..
He goes into the temple and he annoys everybody and pushes out the people that were trading there..
He begins to teach the crowd about his death..
He begins to speak in ways that people are claiming he's blasphemous..
I mean, here's Peter thinking that they've come into the place where finally the Romans are going to be pushed aside..
And it actually seems like Jesus is getting pushed aside..
Rumors are beginning to be said that actually Jesus might get arrested and killed..
And Peter is thinking, "You cannot kill the Messiah..
I mean, that doesn't make sense. Like none of this is adding up.".
And then one evening, they're in this upper room and they're sharing in the meal of the Passover..
And it's just Jesus and his 12 closest disciples..
And there's Peter right next to him..
And in the meal, Jesus begins to speak of the reality that in just a few days,.
most of them in that room is going to deny and betray him..
And they're all sitting there and they're thinking, "That is not going to happen.".
And Peter being the kind of boisterous one he was, he kind of jumps up and he says,.
"Lord, I'm prepared to go to prison. I'm prepared to die for you..
Even if everybody here fails you, I will not fail you..
I mean, I am willing to even die with you..
I will do that before I will ever disown you.".
Jesus, perhaps a little bit like that time on the boat,.
leans over and puts his hand there on Peter's shoulders and he looks him in the eye..
And he says, "Peter, before even this day breaks, before a rooster crows,.
you're going to deny me three times.".
And Peter's like, "There is no way that it's ever going to happen..
I will die for you.".
That evening, they're out on the Mount of Olives.
and they're sleeping out there in the shadow of the temple..
And as they're sleeping, a large army arrives,.
soldiers from the servants of the high priest..
And they're carrying weapons and they all wake up and there's Peter..
And you can see that suddenly there's a confrontation.
that Rome and the Jewish authorities have come to arrest Jesus..
And Peter does what he had said he was going to do in that upper room..
He finds a sword that was lying nearby and he jumps forward..
And one of the servants of the high priest is right there, also armed..

$^{201}$And Peter swipes and he cuts off the ear of the servant..
And Jesus sees this and he's so overwhelmed..
He jumps forward and he rebukes Peter..
Not the first time he's done that, by the way..
He rebukes Peter and he says,.
"No, we will not do it this way.".
And he picks up the ear and he prays and he restores it on the servant's head..
And then he does something Peter never would have expected..
He allows himself to be arrested..
No fighting, no violence..
He actually gives himself up to the authorities.
and the authorities take him and bundle him up.
and walk him out away from the Mount of Olives..
And there's Peter trying to comprehend everything..
This is the Messiah..
This is the Christ..
This is the one that we thought would change everything..
And the ruling oppressive power has now arrested him and taken him away..
And almost like Peter's in this daze, in this place of shock..
He follows after Jesus..
He follows after the crowd that's leading Jesus away..
And they take him to the high priest's house in Jerusalem..
And they enter into this outer large courtyard..
And then beyond the courtyard, a small crowd takes Jesus into the house..
And Peter finds himself standing there on the outer courtyard.
trying to comprehend everything that's just happened..
Now it's really early in the morning..
It's super cold that night..
And he sees a charcoal fire at work just in the corner of the courtyard..
And Peter finds himself walking over to the fire.
that is surrounded by a bunch of the servants of the high priest..
And just standing there trying to warm himself on this fire..
Peter's trying to comprehend and work it all through in his head..
And he suddenly finds himself in a place of his greatest trauma..
That everything he'd hoped for, everything he thought Jesus was going to be.
doesn't seem to be working out how he expected..
He realized that he's just been arrested and no doubt is going to go to trial..
And no doubt just a few days later is going to be executed..
And Peter's standing around that fire thinking to himself,.
this Messiah is about to be killed in the worst way that's possible..

$^{241}$That the Romans had actually engineered this execution, crucifixion on the cross.
to be the most painful thing that it could be..
He's going to be crucified for a heretic..
And Peter's there thinking, hang on, if they're going to do that to the Messiah,.
imagine what they would do to one of his little followers..
Right about that moment, a servant girl who's standing around the fire.
looks at Peter and recognizes him..
And she says to everybody else there, she says, this was one who was with him..
And Peter immediately says, woman, I do not know him..
Denying him the first time..
The woman, the servant girl persists..
No, she said, no, I can tell in your accent that you're not from here..
You're a Galilean. You're part of his tribe..
And again, Peter denies having ever known him..
By this time, the crowd had gathered around..
The other servants were there..
They could see in the firelight his face..
And one of the relatives of the servant of the high priest who had had his ear cut off.
comes forward and recognizes that it was Peter who had done that..
And again, he says, no, you are the one..
You are the one who cut off the servant's ear..
And Peter stands back from that moment and he says with a loud voice,.
I tell you, I never knew the man..
The third time that he denies him..
And immediately in the distance, a rooster crows for the break of a new day..
Peter immediately remembers the words that Jesus had spoken on that day in the upper room..
The Bible tells us that he ran out of the courtyard and he wept bitterly..
That he realized for all that fighting words he'd had in the upper room,.
for all that bravado, I'm willing to go to prison..
I'm willing to die for you..
There it came to the very moment where he could have actually lived up to that.
and he denies even knowing his best friend..
He turns his back on Jesus in a moment where perhaps Jesus needed him the most.
and Jesus had never done that to him..
And here's Peter weeping bitterly..
The Bible tells us the word there means he was in anguish,.
crying out, could not understand what he had just done..
He felt such shame and embarrassment..
That he would be so feeble in the moment when he should have stood up to be counted..
He folded like a deck of cards..

$^{281}$The inevitable takes place, the trial and then the crucifixion on the cross..
And then three days after that, people start to say that this Jesus is not dead..
That he's been resurrected, that he's come back from life, that he's alive..
And Peter hears this and you can imagine how that might have felt for him..
Well, you might have thought that that would be filled with joy..
Could you imagine Peter going, I don't know if I can face him..
I mean, how can I stand before my best friend who I turn my back on on his darkest hour?.
How could I ever bring my, I mean, I'm as bad as Judas..
I betrayed him. I can't ever face him..
I'm not worthy to be his disciple. I'm not worthy to be his friend..
I am disqualified from everything that I ever thought would ever take place..
In the trauma that he's feeling, in the anguish and the shame..
Peter doesn't want to ever, ever put himself in the place.
where he might have to encounter the one that he had betrayed and denied..
And it's there that I want to pick up the story..
No one ever said the Bible was boring. That's not a boring story, is it?.
Don't ever let anyone tell you the Bible is a boring document, right?.
Let me pick up now the point of the story right here.
when Peter is overwhelmed and wondering what is next.
and wondering how he could ever face Jesus..
This is from John chapter 21 starting in verse 1..
"Afterwards, Jesus appeared again to his disciples, this time by the Sea of Galilee.".
"It happened this way, Simon Peter, Thomas called Didymus,.
Nathaniel from Cana in Galilee, and the sons of Zebedee,.
and two other disciples were there together..
'I'm going out to fish,' Simon Peter told them..
And they said, 'Well, we'll go with you.'.
So they went out, got into the boat, but that night they caught nothing.".
Notice the words here. Peter in front of his friends says,.
"I'm going to go out and fish.".
In other words, everything I thought was the thing that I was called into,.
everything thought that I was going to be, none of that I deserve anymore..
Peter has gone right back to where he began..
He's retreated right back to his old days,.
right back to the place where he was before Jesus called him..
He's defaulted himself, got himself stuck right back at the very beginning,.
feeling that he is not worthy to do anything else.
other than go back out and be a fisherman..
I might have left those nets behind before,.
but now those nets will become my life again..

$^{321}$I don't even know how I could ever lift my head up in pride.
and even look Jesus in the eyes..
I'm going to go out and fish..
It's the only thing that I deserve to do..
And guess what? Once again, he goes out on that lake..
He slaves all night and he catches nothing..
This is what trauma does to us..
Trauma and the stress of it will always cause us.
to want to fall back to what we've always known.
rather than press on into the thing that we may have been caught into..
Trauma and grief will always cause us to root ourselves backwards,.
will always cause us to think that we don't deserve..
The stress creates what's talked about in psychology as fight or flight..
And we've seen Peter do both in the story..
The time where he picked up that sword to fight.
and then the time when he's gathered around that charcoal fire.
in the outer courtyard and he flees from knowing Jesus..
And now here again, just days into the resurrection,.
he's fled again onto the shores of the Sea of Galilee.
to the place where he thinks he's the most comfortable.
doing what he thinks he should do.
because he feels he's disqualified from the thing that Jesus called him to do..
I want to speak to some of you watching this right now.
because I get a sense that I was praying for this message.
that there's a bunch of us who are acting exactly like Peter right now.
where maybe something's taking place in your life..
Maybe like Peter, you've done something that you're not happy about..
Maybe you've done something that you're ashamed of..
Maybe there's a sin in your life..
Maybe you've hurt somebody that you're close to..
Maybe you've acted in a certain way or done something.
and you are ashamed of it..
You're embarrassed about it and you've actually had to pay a price for it..
Maybe it's hurt the loved ones around you and you're really struggling.
and maybe there's been some consequences for that for you.
and because of those actions in the past,.
you feel like you are no longer qualified to do the things.
that you had on your heart that God had called you in to do..
Maybe you've actually self-condemned yourself back to the old habits,.
the old ways because you don't think you're worthy of the new.

$^{361}$or perhaps something's happened to you..
Maybe not by your own fault,.
but something outside external to you has taken place.
that's caused you to feel like you're stuck,.
to feel like you're at that red light,.
to feel like this is the only place now that you deserve to be,.
that all the hopes and the dreams and the promises that you had are now gone for you..
Perhaps that's the place where you feel like you're standing..
If that's you, I want to say this..
God stands over you in this moment and he says,.
"You need to remember I am in control..
You need to understand that I have not distanced myself from you..
I have not left you,.
that the place where you are right now is not your permanent home..
You are not disqualified," says the Spirit of God to you..
You are not disqualified ever from the kingdom of God..
Peter felt like he deserved nothing other than to go back and be a fisherman.
and Jesus is about to show up and tell him,.
"You are never disqualified from my kingdom.".
Never disqualified..
I want you to notice this..
What happens next is that this figure shows up on the beach.
and there's Peter having done his fishing and caught nothing.
and they see this figure and the figure cries out and says,.
"Hey, did you catch anything?".
Probably a common thing that was said in those days..
Peter honestly replies back, "No, we caught nothing.".
And this one says something quite simple..
He says, "Hey, why don't you throw the net on the other side of the boat?".
And Peter, perhaps going, "Well, we tried everything else,".
throws his net on the other side of the boat.
and once again, there is a catch that they can't even pull in..
The Bible says 153 large fish almost breaking the nets.
and in that moment, everything comes back for Peter..
He's taken right back to that very first moment where Jesus met him,.
took him out on the boat, caught that miraculous set of fish..
Now here he is all that time and he sees this person and he goes,.
"This is Jesus.".
I mean, the only person who can ever fill my nets like this is the Son of God.
and there he is and I love this..

$^{401}$This is so Peter..
Peter literally jumps out of the boat and begins to swim as fast as he can.
to get to the shore to meet with his Messiah and his friend..
And I want to pick up now the moment in the passage where they actually get to the shore..
It's found in verse 9..
It says this, "When they landed, they saw a fire of burning coals.
where they had fish, where Jesus had put fish on it and some bread..
Jesus said to them, 'Bring some of the fish that you have caught.'.
Simon Peter climbed aboard and dragged the net ashore..
It was full of large fish, about 153, but even with so many the net was not torn..
Jesus said to them, 'Come now and have breakfast.'.
None of the disciples dared ask him, 'Who are you?'.
For they knew that it was the Lord.".
I want you to see what Jesus does here..
Here he is on the beach and he's created what the scriptures say to us here.
as a charcoal fire..
That phrase, a fire of charcoal, is only ever repeated one other time in the New Testament..
Do you know when?.
When Peter was in the courtyard of the high priest, cold on that night.
and sees a charcoal fire and goes over and warms himself..
Jesus creates the same environment for Peter.
and welcomes him back into the worst moment of his life..
Now that might seem on the surface like a pretty harsh thing to do,.
but I want you to catch what Jesus is doing here..
He's actually reframing the narrative of Peter's trauma..
In creating a charcoal fire before him and actually inviting Jesus into that space,.
he's taking Peter right into the place where he denied him three times..
But in that moment, Jesus is now saying,.
"Let me take you back and actually reframe the narrative of your trauma..
What you thought was the place of your worst moment.
is about to become the place of your greatest moment of humanity..
I'm about to actually reframe everything you thought around the very charcoal fire.
where you denied me is about to become the place of your greatest redemption.".
I want you to know that this is very important in the idea of post-traumatic growth..
We said last week that our growth from trauma does not take place.
by ignoring our trauma, by trying to walk away from it,.
by trying to bury it or by trying to kind of bury the feelings that it creates in us..
Jesus doesn't try to ignore what's taking place..
At the same time though, Jesus doesn't beat it over his head..
Jesus gently, graciously brings him into the place of owning the reality of the moment,.

$^{441}$but not so he can feel shame, but instead so he can be redeemed,.
so that there can be a changing point for him..
And that's what takes place in this conversation..
Let me read it to you from verses 15 onwards..
"When they had finished eating, Jesus said to Simon Peter,.
'Simon, son of John, do you truly love me more than these?'.
'Yes, Lord,' he said, 'you know that I love you.'.
Jesus said, 'well then feed my lambs.'.
Again, Jesus said, 'Simon, son of John, do you truly love me?'.
He answered, 'Yes, Lord, you know that I love you.'.
Jesus said, 'well, take care of my sheep.'.
The third time he said to him, 'Simon, son of John, do you love me?'.
Peter was hurt because Jesus had asked him a third time, 'do you love me?'.
He said, 'Lord, you know all things, you know that I love you.'.
Jesus said to him, 'feed my sheep.'.
Notice what takes place in that place of trauma..
Three questions about whether he loves him.
for the three denials that took place in that courtyard..
Three times where Peter is able to say out loud in that moment,.
'I do love you.'.
See, this is the powerful thing about Jesus..
And this is the powerful pivot point for post-traumatic growth..
See, Jesus is not interested in shaming Peter for his past mistakes..
Instead, he wants to know what's on his heart right in that moment..
'Do you love me right now?'.
Like Jesus is not saying that was the worst thing..
He's saying, 'what is in your heart right in this moment?'.
And Peter is there going, 'you know I love you.'.
I love actually what he says right at the end..
He says, 'Jesus, you know all things.'.
This is Peter's way of saying to Jesus,.
'you knew everything that took place in that courtyard..
You know the trauma that I've been through..
You know the fact that I denied you three times..
You know the fact that that has torn me up inside..
You know the fact that I have retreated back here to the shores of the Sea of Galilee,.
doing the thing that I thought I was only responsible for..
You know how terrible I feel about myself..
You want to know what I think of you?.
I love you..

$^{481}$I love you because I know that even in how terrible I feel,.
you're here right now, reaching out to me.'.
Peter understood that Jesus had not given up on him..
And in saying those three times, 'I love you..
I love you. I love you.'.
Peter was prophetically standing against the enemy's rule.
to try to keep him locked down in the red light of his life,.
to keep him stuck there in the place that was his past,.
rather than move forward into the thing that was his future..
And Jesus knows that he needs to reframe the narrative of his trauma.
by speaking to him about his personhood, his identity..
I want you to see here in this passage, it's quite a beautiful thing..
Jesus refers to him as Simon, son of John..
He doesn't mention Peter here..
He actually goes back to his original name..
He says Simon, which means listener..
And then he says son of John..
See, Jesus is actually centering Peter in his true identity..
He wants him to know who he truly is..
He wants him to get right to the place of his identity.
and say the trauma has not defined who you are..
You want to know who you are?.
You're Simon, son of John,.
who I changed the name to be Peter, which means rock..
You are not what your trauma says you are..
You are this one right here, the one I love,.
the one who's sitting with me around this fire,.
restoring and redeeming the narrative of your trauma..
You and your identity is not linked to the worst moment of your life..
I need to preach this to someone right now..
The sum total of who you are is not the worst moment of your life..
And Jesus is getting into the very heart of the identity for Peter..
And he's saying, Peter, do you want to know who you truly are?.
Simon, listener..
That's who you are. You listen..
You've heard my word. You know who I am..
You just declared that you love me..
And guess what?.
I am now releasing you out of a place of a restored identity into new possibilities..
Note this. Every single time that Jesus speaks to him,.

$^{521}$he says, feed my sheep, feed my lambs, feed my sheep..
Remember what took place in the very first time they met?.
Jesus says, oh, you don't need to be afraid..
From now on you will catch people..
Now Jesus takes it a step further..
He says, hey, your identity is not just to catch people..
You're going to be the shepherd of my people..
I want you to go from here and I want you to love my people.
with the same way that you love me in this moment..
I want you to go from here out of the place of your trauma..
And guess what? The fact that you felt shame,.
the fact that you were embarrassed, the fact that you thought you were nothing..
That's the people you're about to encounter out there..
And as you go and start my church, as you found my church,.
I don't want you to go there just simply as a fisher of people..
You need to go there as a pastor and you're going to be able to pastor my people.
because you felt brokenness yourself..
Your trauma is the very thing that's going to be the root and the power of your ministry, Peter..
This is Jesus redeeming him out of the place and saying nothing is going to be wasted..
You're going to be able to go and pastor a group of people in such a powerful way.
that you would not have been able to do if you had not experienced the trauma in the first place..
This is Jesus releasing Peter to the new possibilities of his life..
And Peter from this moment will go and we see in church history what he does..
He goes and preaches to the crowds of boldness and fire..
Over 3,000 coming to Jesus in one day..
He's a wounded healer..
He goes and heals people in Jerusalem, just like Jesus healed people in the shores of Galilee..
He goes and starts and founds the early church, the first church in Jerusalem,.
and passes it for many years..
He writes letters, pastoral letters to his people.
that explains the beauty of forgiveness and the transformation of life..
Why can he do all that? Because he's been forgiven himself..
And Jesus is releasing him to new possibilities because of the trauma that he had been through..
Now is the time, Peter, for you to become the one that I had always created you to be..
So, a net, a sword, a fire, and then sheep..
Peter's story is also our story..
The same God who met Peter is the same God that's alive in my story and your story..
And I want you to know you are never distant from God..
I want you to know that that red light that you feel like you're sitting in front of.
does not define who you are..

$^{561}$I want you to know that as real as trauma is, as hard as it can be,.
it does not shape the fullness of your identity..
I believe that Jesus wants to show up at each one of our beaches..
He wants to show up at your beach right now..
After you have struggled over your long, dark night of trauma.
and felt like you have received nothing,.
Jesus will come and stand on the shore of that beach and He will tell you He is in control..
He will actually assert His authority over your life again..
He will actually invite you to that place of your moment of identity and who you are..
He will begin to speak new possibilities over you,.
begin to take you from a place where you were to a place where you're going to be..
And He knows that the scars that you carry from the trauma that you're in right now.
is going to be the very thing that's going to bring healing to the many that He wants to bring into your life..
What happens to Peter also now becomes a part of our journey..
That's true post-traumatic growth..
New possibilities for you..
Some of you watching this right now,.
there are new possibilities for you in the workplace that you have never allowed yourself to think about..
Maybe you've never been willing to take the risk in a new adventure.
or a new investment or a new opportunity at work..
And maybe some of you have wanted to change jobs for a long time and just haven't had the courage.
and God stands over you in this time of trauma of pandemic.
and He says, "You want to know what's ahead for you? Open your heart to me again..
I want to speak to you and give you courage and boldness for new possibilities.".
Some of you, those new possibilities are in relationships..
Some of you, you've got fractured relationships in your life,.
maybe because of some things you've done or because of what others have done to you..
And I feel like Jesus stands over you right now.
and welcomes, invites you into new possibilities,.
maybe to repair relationships that are broken.
or it might be to launch you into some new relationships.
that you thought you would never want it once again, ever receive..
Actually, as I'm saying this right now, I feel like some of you feel like you cannot be loved..
I feel like right now, some of you feel that you can't be loved..
Maybe you've done something wrong..
Maybe someone's done something to you.
and it's left you in a place where you think, "I can't be loved anymore.".
And I want you to know that there are relationships ahead of you.
that are filled with new love for you,.
new possibilities of relationships that you've not allowed yourself to even consider.

$^{601}$because you feel like you're not worthy to be loved by someone else again..
I feel like that's really important for some people who are divorced watching this right now..
You feel like you're on plan B..
And I feel like the Holy Spirit's saying right now, there's new possibilities for you..
New possibilities for you to be loved in a way that you perhaps never anticipated or expected..
And perhaps for some of you, that new possibility is in your faith..
Maybe this year is the year for you to take that step forward in faith..
Maybe you know that God's been calling you into something.
and you've been holding back, thinking maybe you're not worthy or deserving of it.
or maybe just afraid of it..
I pray that the Spirit of God would fall on you right now.
and fill you with something that would enable you to have the courage to step out..
Be strong and courageous, I hear the Lord say..
I hear the Lord say, "I'm not worthy of you, but I'm calling you to step out.".
And I pray that the Spirit of God would fall on you right now.
and fill you with something that would enable you to have the courage to step out..
Be strong and courageous, I hear the Lord say..
And I pray that the Spirit of God would fall on you right now.
and fill you with something that would enable you to have the courage to step out..
Be strong and courageous, I hear the Lord say..
And I pray that the Spirit of God would fall on you right now.
and fill you with something that would enable you to have the courage to step out..
And I pray that the Spirit of God would fall on you right now.
and fill you with something that would enable you to have the courage to step out..
And I pray that the Spirit of God would fall on you right now.
and fill you with something that would enable you to have the courage to step out..
And I pray that the Spirit of God would fall on you right now.
and fill you with something that would enable you to have the courage to step out..
And I pray that the Spirit of God is here for you to move you forward..
Can we pray together? Let's pray..
Father, we just thank you for this moment..
Lord, I want to pray for everybody in this room..
I want to pray for everybody who's watching online..
I want to ask that your Holy Spirit would come and minister to us new possibilities, Lord..
Some of us, like Peter, we've run backwards..
That picture of being stopped at a red light resonates with us deeply..
Would you hear the Holy Spirit today saying, "This is not the place I've designed you to be..
I've not designed you to put roots down at the red light.".
Some of you have gone back to fishing..
And I feel like the Lord is saying over you, "Drop the nets..

$^{641}$It's time for you to move forward again..
Don't let the enemy define who you are through your trauma,.
but allow the trauma to actually be the deposit of the things that the Holy Spirit is going to use most for his glory ahead..
For those of you that feel like you're no longer lovable,.
the love of Christ speaks over you against the lie of the enemy to hold you back..
For some of you who feel like you can't move forward in a calling that God has for you,.
I come against the enemy's strategy to put you on the boat on the lake rather than on the shore moving forward..
For some of you who feel like you're at an end point,.
I pray new possibilities in the name of Jesus over you..
Father, we don't do this in our own strength..
We don't do this because we try to be better people..
We do this because of a work of your Spirit..
So Spirit, break out. Spirit, break out..
Heaven, come down..
Lord, would your presence and your Spirit fill me and change me, renew me and send me forward.
into the person that you've always created me to be..
Not in my own effort, but in the grace of the power of the Spirit of God..
And we pray that over every person at the Vine, over every person watching this,.
and we thank you that you release us into this..
In Jesus' name. In Jesus' name..
Everyone says, "Amen.".
\newpage



\section{}
\label{sec:hd9p2DZoYtI}
\textbf{2021-01-18 Church Everywhere Live: Post Traumatic Growth - Relationships With Others [hd9p2DZoYtI].mp3}
\newline
\newline
連結: \href{https://youtube.com/watch?v=hd9p2DZoYtI}{\texttt{ https://youtube.com/watch?v=hd9p2DZoYtI}} ~~~~ 語音日期: 2021-01-18 
\newline
\newline
\hyperref[sec:LHgF2voP5ys]{\small{< < < PREV SERMON < < <}}
~
\hyperref[sec:index]{\small{[返主目錄]}}
~
\hyperref[sec:d8XMdofd39c]{\small{> > > NEXT SERMON > > >}}
\newline
\newline
$^{1}$So, about six months into being senior pastor here at The Vine in 2013, I had what could.
only be described as a full on panic attack..
I woke up one morning and immediately as I opened my eyes, it felt like a tsunami of.
fear and of anxiety was just flooding over me..
I was lying there in bed and the whole room around me was just beginning to spin..
I was so hot..
My heart rate was pounding and I was sweating all over despite the fact that the air con.
in our room was just blasting out..
I felt completely out of control..
I wondered whether I was going to be able to catch a breath and what was going to happen..
I managed to just get out of my bed and literally just stumbled towards the bathroom, whereupon.
I kind of opened up the toilet and I was sick in the toilet..
I threw up literally in the toilet..
In that moment, I felt like this overwhelming sense of fear, almost to a degree like paranoia,.
like everything that I was trusting, everything that I was hoping in, everything seemed to.
be shattered in that moment..
I didn't know what was going on..
I didn't know how to find my feet again, how to find that ground again..
I was there on the bathroom floor..
I began to understand what was taking place..
This panic attack wasn't just a random thing, but it was the result of so many months of.
bottling up how I was feeling..
I realized that this was connected to my role as senior pastor here at the Vine..
I had been struggling for months up until that point with a huge amount of self-doubt.
in my life, a huge amount of anxiety, self-doubt, and insecurity in who I was..
I knew that God had called me to be senior pastor here at the church..
I had been so wonderfully prepared by our elders, by John and Tony, our founding senior.
pastors here..
I was ready on paper to step into this role..
When I stepped in on that day in November 2013, I had all those prayers, just like we.
prayed for Elizabeth just a minute ago..
I felt this wave of just God's Spirit in me..
But it's so fascinating how quickly that was drained from me as I began to focus on my.
lack or my inability to do the thing that God had called me to do..
I was putting so much of my energy in those few months to pretending on the surface in.
front of everybody, standing in front of you every week that I'm this great pastor, that.
I've got it all together, that everything's going to be great..
Whereas inside, I was dying..
Inside I thought I was a failure..
Here was the main thought that went through my head in that first six months..

$^{41}$John and Tony had done such an amazing job of building such a strong, well-established,.
great church..
And here I was now with that whole church on my shoulders, all that expectations and.
all that pressures..
And I thought, "What if I'm the guy who 12 months down the line, the church goes from.
the size it is and it reduces down to like a hundred or so people?.
What if I'm the one who actually fails in leading this church?.
They've done such a great job and now it's on my shoulders.".
And I felt like I wasn't up to the task..
And that wave of tension in me between living two different lives, a secret life of anxiety.
and insecurity that no one knew about, and a public persona that I had everything all.
right and that I was doing fine and that I could lead this church..
Those two things created in me such a tension that on that morning, it just exploded in.
me..
Well, for the next number of days and the weeks ahead, I found myself thinking to myself,.
if people spend enough time with me, they'll realize that I'm completely out of my depth..
And so what that made me do was begin to slowly at first and then quickly withdraw myself.
from people..
I began in my trauma of the insecurity and anxiety that I was feeling to step away from.
people, kind of wall myself off..
I thought if I can just create my own little private sense of some false sense of security.
and comfort by walling others away from me, maybe I can get control again..
And so I began to distance myself from close relationships that had a strain on my marriage..
I began to distance myself from people that I was supposed to be ministering to, from.
social settings..
I didn't want to spend time with people because I thought if people would spend time with.
me, they would actually know that I am the failure, that I am the one that I thought.
deep down inside I was..
And I was carrying that tension with me, isolating myself from my community, and it eventually.
just kind of boiled over..
There was one evening where the tension was so strong that I was like, I have to tell.
my wife, Chris..
Like, I have to let her in to this pain..
I can't carry the pressure anymore..
I took her out on a date night..
It wasn't the kind of date nights that normally you would expect..
I took her out for this meal and we sat down..
And it was this public restaurant..
And as soon as we sat down, I just started to weep..
You know those moments where you don't plan to cry, but you suddenly just start weeping.

$^{81}$in this public space?.
And I'm just, I'm like a grown man weeping..
And I can't control myself..
And Chris is sitting there kind of going, "What is going on?".
And I got the courage to say to her, "Honey, I'm really suffering right now..
I'm sinking in the pressure of leading this church..
I don't think I can do it..
I don't think I've got the strength or the skills..
I don't think I'm a very good pastor.".
I began to give her a list of all the things..
And the biggest thing that was scaring me was that if I failed, everybody would leave.
and desert me..
And I said to her, I said, "Honey, if I fail with leading the vine, if the church dies.
and disintegrates under my leadership, will you still love me?".
That was how insecure I was..
I thought that her love for me was based on my success and my achievements..
And Chris, in the beautiful way that only she can, leant forward, put her hand on my.
hands..
And she says, "No matter what happens in your life, Andrew, whether you have great success.
or great failure, I will never stop loving you.".
And then she said, "But you cannot do this alone..
You cannot do it in isolation from others..
The thing you actually need the most is the thing that you're walking away from, that.
you need to come out and be known..
You need to let people know how you're feeling..
The church will support you..
We're here for you..
And that half of what you think is going to happen is not going to happen anyway.".
She started to comfort me with a challenge to not isolate, but to embrace the people.
around me..
Well, that evening was a turning point for me..
And I've been on a journey of healing ever since..
It's seven years in..
I think I'm still on that journey of healing, although I've learned so much and I've grown.
so much..
And you may wonder, one of the things that I think I'm known for in leadership is my.
authenticity and my vulnerability..
That's not come just because I've tried to do it..
It's come because I've had to reach deep inside of me and be courageous in being known in.
front of you..

$^{121}$I realized that I cannot carry the burdens of this ministry on my own..
And so I bring them to you regularly as a church, as a way of me saying, "I will not.
isolate myself.".
That actually I think it's in community..
It's when we're courageous enough to truly be known in our brokenness and our hurt and.
all the things that we're struggling with, that we actually can find the greatest moments.
of our healing..
Some of you watching this right now, you've isolated yourselves from people around you..
You've walled yourself off thinking that that's the place to find your comfort and security..
And I want to speak a message to you today..
If that's you, because it's been me and it still is me at some times, if that resonates.
with you, I want to call you once again into a place of being known by the people around.
you..
Ask any psychologist, ask any sociologist, anthropologist, they'll tell you the same.
thing..
They'll say that no one ever grows, transforms, changes, or develops for the positive in isolation.
from others..
In fact, like I experienced in my own life, the opposite is true..
Some of the greatest moments of breakdown in our physiology and our psychology happens.
because we isolate ourselves from a community around us..
If you're willing to put yourself into a place where you're open and vulnerable with people,.
where you can truly be known by them, where you can grow in the safety and security of.
their friendship and love, some of the greatest moments of growth and flourishing will happen.
to you..
But cut those things away..
Wall yourself off..
Retreat because maybe you're embarrassed or ashamed, or maybe something's happening and.
trauma in your life to such a degree that it makes you shut down..
Do that and you'll find yourselves moving in the opposite direction of growth, change,.
and transformation..
Here's the great irony of trauma..
Trauma does its very best to remove, to kind of take away community from you..
And the tension and irony is this..
When you need community the most is when you actually feel like you desire it the least..
When we're so overwhelmed with trauma, when we're so overwhelmed by the pressures of life,.
that's when we actually need to reach out towards others..
But it's the time where we desire it the least..
We want to walk in the opposite way..
I think this is one of the greatest strategies of the enemy over us..
We isolate ourselves when times are tough..

$^{161}$I think over 2020, we all did this to a certain degree..
Look, we all had to physically distance each other..
And that was an important thing..
It was the right thing as part of the pandemic that we've been going through..
But alongside of physical distancing came something else..
I call it social distancing, not the way that people used to promote social distancing..
I think it's okay to physically distance with this pandemic..
But what's come alongside of that is a desire in us to actually begin to distance ourselves.
emotionally from those around us..
Beginning to distance ourselves, not just physically, but emotionally and spiritually.
from the ones that mean the most to us in our lives..
We found ourselves with this growing sense of social anxiety during this time, where.
we don't want to catch this disease..
And if someone's caught the disease, there's suddenly like a little bit of an outcast in.
society..
Don't go near that person because you might get infected..
And here's a phenomenon that's happened during the time of 2020..
Did you know statistics show that divorce has arisen much higher in number than at any.
time prior to the pandemic?.
Because here's the reality..
People know what it's like to live in their marriage, just basically doing surface things.
with their partner throughout the week..
But you throw them 24/7 in a house together where they can't leave, and tensions have.
begun to emerge..
This is a challenge to us because I think when times are tough, when things are stressful,.
when things are hard, we begin to grate against some of those relationships in our lives..
We're going to find it hard to actually connect with people..
We find it hard to express the emotions and be honest about what we're going through..
And so we actually put up a front and we get angry and irritable and begin to rub against.
each other in the wrong way..
And so we naturally distance ourselves or we begin to upset..
Some of you watching this, you know this..
Your marriage is in maybe the hardest place it's been in for many years..
Some of you know this because of your workplace..
You found you've got breakdowns in relationships in your work..
I think for us here in Hong Kong, it's magnified because of everything we've been through in.
the last two years politically in the city..
Not only have we had this pandemic, which has caused us at times to have fractions and.
frictions with our closest people that we're in relationship with, we've also got all this.
political anxiety and all the change that's happening here politically and everybody's.

$^{201}$opinions and thinking about that..
And the way that that has also deeply divided our families, deeply divided us from our relationships.
deeply created issues for us, perhaps with those that we've come to love the most..
I want to speak this over us as a church..
I believe as we enter into 2021, we have a great opportunity right before us..
It's been prepared by the Holy Spirit for this moment in this hour..
And the opportunity is this, that we would actually see that trauma could be a catalyst.
towards a deepening of the most important relationships in our lives rather than a fraction.
and a division in them..
That maybe actually post-traumatic growth might say to us that actually it's in traumatic.
experiences where we can have a wake up call and realize that we need to actually reach.
out and beyond, even though it feels like the last thing we want to do, that actually.
maybe the right thing to do is to actually open ourselves up and be known and honest.
about all the things that are in our lives, because we need to have that healing that.
can only come through being deeply in relationship with others..
The research through post-traumatic growth has suggested that often people who have gone.
through crazy traumatic times prior to the trauma, they put relationships pretty low.
down on the list of things that were important..
Maybe at the top was work, maybe it was career, maybe it was making money, maybe it was doing.
projects and putting in success, but right in the middle of their trauma, they realized.
that actually the relationship that they suddenly needed the most were distant from them because.
they had pushed those relationships away..
And that actually trauma was an awakening time for them to realize that as they came.
out of that trauma, they needed to double down on the most important relationship in.
their lives..
One of the people that was interviewed recently said it this way, he said, "I came out of.
trauma realizing that it's people over projects.".
My good friend Josh, that we've been talking about a little bit in this series, I think.
manifested this so powerfully..
Josh had always been someone who put people over projects, but his trauma of the two years.
of his journey with cancer put himself in a realization that actually the relationships.
around him were the most important thing..
And he invested so beautifully and so heavily in those important relationships, even ones.
that were quite strained became renewed and rejuvenated through the work of the Spirit.
in the midst of his trauma..
And I know that just before he passed away, one of the things that he was so proud of.
was how deep his marriage was, how deep his relationships was with his parents, how deep.
he was in terms of reconciling some of the broken relationships he'd had in his life..
It was a joy to him to find community in the midst of trauma..
Jesus understood this better than anyone..

$^{241}$Jesus understood that there was a powerful connection between trauma and healing that.
could only come through community..
And I want to actually just share a story from Scripture today that draws out the beauty.
of what it is to find community in the midst of our trauma..
And that community could actually be a catalyst for us for our greatest amount of healing.
and growth as human beings..
I want to take this story from Luke chapter eight, and I'm going to start in verse 40..
Let me read this to us..
Now when Jesus returned, a crowd welcomed him for they were all expecting him..
Then a man named Jairus, the ruler of the synagogue, came and fell at Jesus's feet,.
pleading with him to come to his house because his only daughter, a girl of about 12, was.
dying..
As Jesus was on his way, the crowds almost crushed him..
And a woman was there who had been subject to bleeding for 12 years..
No one could heal her..
She came up behind him and touched the edge of his cloak and immediately her bleeding.
stopped..
"Who touched me?".
Jesus asked..
When they all denied it, Peter said, "Master, the people are crowding and pressing around.
you.".
But Jesus said, "Someone touched me..
I know that power has gone out from me.".
Then the woman, seeing that she could not go unnoticed, came trembling and fell at his.
feet..
In the presence of all the people, she told why she had touched him and how she was now.
instantly healed..
And he said to her, "Daughter, your faith has healed you..
Go in peace.".
Luke recounts for us this moment..
It's kind of just a random moment in Jesus's ministry..
He's just walking and ministering in the crowds of Galilee..
And I think every day he had requests and people coming to him, asking him for all sorts.
of things..
And here's Jairus saying, "Hey, would you come to my house and heal my daughter?".
Jesus walking on the way to do that..
And a woman who has been subject to bleeding for 12 years, she comes forward, she pushes.
her way through the crowd..
She reaches out and just grabs the hem of Jesus's garment..
And immediately, instantly, she's healed..

$^{281}$Luke describes a few things about this woman..
He says that she had been subject to bleeding for 12 years..
Now it's not specifically stated in the text, but this is likely to be a uterine hemorrhage..
It was a relatively embarrassing, relatively shameful thing to have..
And she had spent a lot of money trying to heal herself over that period of time, but.
nothing had helped her..
Now the thing that's really important to know about this that's implicit in the text and.
not explained directly here in the words Luke uses, is that because of her bleeding, she.
was actually deemed to be unclean by society, both the Jewish and social society of the.
day..
Her bleeding meant that she was ceremonially unclean, that anybody she touched and anybody.
who touched her would also automatically then be unclean..
Now the way you get yourself from being unclean to clean again was actually a long process.
of religious ritual..
It required the cleansing of you..
It required all these prayers..
It required you going before a priest and a priest anointing you and declaring that.
you are clean again..
It wasn't an easy thing..
And so if you were deemed to be unclean, all of society would stay away from you..
They would distance themselves from you..
They would say that this person out here is one that if we touch, we'd have to go through.
all that..
So they would push that person away..
You can imagine for this lady herself, she's been subject to this for 12 years..
And not only is there shame in her own physical brokenness, the fact that she can't heal herself.
and stop the bleeding, but there's an even deeper and greater shame about the distancing.
that everybody's done around her..
She's alone..
She's cast out..
She's pushed aside and her trauma is both the community leaving her, but also I believe.
her then distancing herself from the community..
You can almost imagine her walling herself off and saying, "Well, if they're going to.
treat me like that, well then I'm just going to distance myself.".
And a hardness coming over her in that way..
I mean, this is a woman who's had to carry this kind of trauma for over 12 years..
And there's habits that have formed in her where she will never go out in a crowd..
She will never find herself in public..
She would retreat and live her own life separate and isolated from everyone else, except for.
this day..

$^{321}$Because on this day, she hears that Jesus is coming past..
And she's heard stories about Jesus..
She's heard stories about his healing and his power..
She's heard conversations that he's not just a rabbi, but a prophet..
And despite the reality that she lives her life distant from others and avoids crowds,.
she fights her way through the crowd in order to receive the healing that she believed that.
she could receive..
Notice, if she touched someone, they would become unclean..
She would have had to fight her way, touching many people to get towards where Jesus was..
Not only that, but she reaches out to touch Jesus, believing that someone so pure and.
so beautiful and so holy and so powerful as this rabbi, if she could touch him, he might.
be able to touch her and cleanse her and renew her and heal her and establish her again as.
a true person in her community..
The courage to do this is unbelievable..
I mean, the risk she took that day to fight through the crowd just blows my mind..
This is like if somebody was like a COVID positive person and they were getting on the.
MTR in rush hour and they let everybody know around them that they were COVID positive..
Can you imagine what the crowd would do to that person here in Hong Kong in this time?.
They would scream, there would be panic, everybody would move out of the way, nobody would want.
to get anywhere near that person..
That's the woman here..
She was risking the crowd, understanding that she was unclean and all the hostility that.
would happen..
She was willing to risk that in order to touch the hem of Jesus's garments..
The Bible tells us here that as she touched him, immediately she was healed..
Look at this, immediately she was healed..
Now this is really important for what's about to happen next..
There's not a progress in her healing here..
This is not like another time where Jesus spat on the mud and put it on the eyes and.
the person couldn't see properly and then he prayed again and suddenly they were healed..
There's no journey in healing for this woman..
She's immediately, the emphasis in the Greek is on the moment that it happened..
It happened straight away..
Twelve years of shame and trauma and ridicule immediately removed in the physical healing.
that she suddenly experienced in that moment..
But this is important because of what happens next..
Jesus stops..
He knows that power has gone out from him and he says these words..
He says, "Who touched me?".
It seems like a pretty kind of like not that big deal of a statement, right?.

$^{361}$Like he's just like, "Who touched me?".
But there were so many crowds around him, so many people pushing in on him that Peter.
goes, "What are you talking about, man?.
What do you mean?.
Look at everybody..
Everybody's touching you..
Everybody's pushing in on you.".
And he's like, "No, no, no..
Someone touched me..
I know power has gone out..
Who touched me?".
Here's what I think Jesus is doing in this moment..
He realizes about this woman..
He's the son of God..
He knows exactly what's taking place..
He knows everything about this woman..
He knows how she's fought through the crowds and he knows how she's come out and touched.
him..
And he's standing there and before the crowds, he's intentionally creating space for her.
to have the courage to come forward and be known..
You see, Jesus could have turned around and said, "Hey, you lady, you who touched me,.
come forward.".
But he doesn't do that..
He instead says, "Who touched me?".
It's his way of saying to her directly, "Will you have the courage to make yourself known?".
Now the woman didn't need to do it..
She didn't need to come forward for her physical healing..
It wasn't like she was still wondering whether she had actually been restored and maybe I.
should go forward just to get the fullness of my healing..
The Bible had told us immediately she'd been healed..
Just as anonymously as she had touched the garment and received the thing that she had.
come from, she could have anonymously then just walked away..
She could have just escaped the crowd and gone in the other direction..
And no one would have known other than Jesus that that would happen..
There was no reason for her to step forward unless there was another piece of healing.
she needed to receive..
Get this church, Jesus calls her out, not just because he wants her to receive her physical.
healing, but because he wants her to see and feel her social healing as well..
Because it was her bleeding that had caused her to be cut out from the very crowd that.
she had just fought through..

$^{401}$And in him saying, "Hey, come forward now," what he's doing is creating space for her.
to actually walk into a place where she can stand in front of that community and testify.
to the reality that God has actually cleaned her..
That she has been restored by him into community..
The Bible tells us that rather than walking in the other direction anonymously, she steps.
forward before Jesus..
She comes and she falls at his feet..
And in front of all the crowd, she tells everybody why she pushed through the crowd, why she.
grabbed the hem of the garment, the healing she had received..
And she confesses it before everybody..
She takes the great courage to be known..
I think church, this is one of the greatest examples of courage that we have in the New.
Testament..
Not only did she have the faith originally to touch the hem of the garment, but now she's.
decided to push through the crowd once more and stand before everybody and confess the.
ugliness of her trauma and the reality that she has been so hurt by everybody around her..
And yet now the one who's healed her is in her midst..
I want you to see how Jesus responds to her in front of everybody..
First of all, he says, "Daughter.".
He addresses her daughter..
There is no other time in the gospels where Jesus speaks to a woman and uses this title..
This is the only example in all of the gospel stories that Jesus addresses a woman and calls.
her daughter..
The word actually was an incredibly intimate word..
It was a word that was only reserved and used for a father speaking to his biological child..
Jesus stands in front of everybody and says, "You're not just somebody I know..
You're not just a friend of mine..
You're not a disciple even..
You're intimately part of my family..
I am so here with you..
You're my daughter whom I love..
With you I'm well pleased.".
A similar kind of thing that God said over Jesus at his baptism..
Here's Jesus addressing this woman who has been so isolated by community for 12 years.
and he's saying, "You are in the middle of the most intimate community you could ever.
have..
Restoration in your relationship with your father, with me, with your God..
You're my daughter.".
And then in front of everybody, he says, "Your faith has healed you.".
I think Jesus is speaking there of both her faith to push through that crowd right at.

$^{441}$the beginning, but also her faith to push through the crowd on the second time and stand.
before everybody and testify to her brokenness and her healing..
The woman had the courage to be known and that faith healed her..
And Jesus uses those words on purpose because he wants all the crowd to know that she has.
received both her physical and her social healing..
That where they had to be distant from her before, they can now draw close..
And where she, if she had slipped away from the crowd anonymously that day, would have.
continued to live a habit of isolation, she now was being challenged to stand in front.
of everyone and welcome people into her journey so that she could be healed, not just physically,.
but emotionally and spiritually in her community..
Jesus was passionate about relationships and Jesus understood that healing was so much.
more than just physical..
I want to speak to you right now in this moment..
Where has it been in this last little while that you've been experiencing, all that we've.
been experiencing in this pandemic?.
Where have you begun to isolate and distance yourself from others?.
Where have you allowed your emotions that you're carrying at this time to wall you off.
from the people that you love the most?.
For those of you who are struggling in your marriages right now, where have you put up.
walls with your partner because of the intensity of what you guys have been going through together?.
Maybe I want to say this as well..
I believe that there are some people watching this where there's a secret sin that's been.
going on in your life..
A little bit like for me where there's the secretness of my insecurity and on the surface.
in front of the church, I was the best pastor I could be, but inside I was dying and that.
tension boiled over to a panic attack..
I think there's some people watching this right now and that resonates for you..
There's a secret sin..
There's something that you're doing that nobody else knows about and it's hidden..
And because of that, you've been ashamed of who you are..
You're ashamed of what you've done..
And yet I believe God's standing before you today like he did that woman..
And he's saying, "Who touched me?".
Are you willing to come out of the shadows and make yourself known?.
That you're healing, that you actually long for..
You don't want to feel the way you do inside..
You don't want to feel caught up and in battle in your spirit inside..
You don't want to feel the panic and the anxiety and the shame anymore..
No, you actually do want to be known..
But remember what I said at the beginning, trauma fights to rip community from us..

$^{481}$The irony is when we need it the most, we want it the least..
That's how you're feeling right now..
But I want to say this, God creates a space for you in his grace..
It's a space for you..
He's not going to force you, but he invites you to step into that space and say, "You.
know what?.
Not everything is right inside..
You know what?.
I've been carrying some stuff that I am proud of..
You know what?.
It's been hard lately..
You know what?.
I've been struggling..
You know what?.
I need help..
I need to respond in some way.".
Today this message is a grace space for you..
A chance for you to come out of the shadows and come into the light..
To begin to repair some of the broken relationships around you by being honest and humble, being.
real and taking courage..
It's going to require a lot of courage for you to do this..
It's not going to be an easy thing, but we've got a lot of ways here at the Vine where you.
can receive the help that you need, where you can be reestablished and reconnected in.
the community..
At five o'clock today, we have a group of pastors that every Sunday are prepared to.
pray and minister to you..
They're available today..
And maybe for some of you, that's a step forward..
That's like you coming out of the crowd and going, "You know what?.
I need prayer..
You know what?.
I need help..
I can't get through this on my own.".
For some of you, maybe it's the need for your community group to come around you..
If you're in community group, maybe there's a community group leader who can stand with.
you and pray for you..
I want to encourage you to reach out to your community group leader this week..
If you're not in a community, but you want someone to come around you and pray for you.
here at the Vine and you're not available today to be able to do that, email pastoralcare@thevine.org.hk..
That will open you up for one of our pastors to reach out to you, to connect with you and.

$^{521}$to pray for you..
We have an amazing team of counselors here at the Vine called Oasis, who are loving and.
willing to stand with you and pray for you and to actually walk with you in counseling..
When I eventually confessed to Chris how I was feeling and began to move into a place.
of that healing, I went into counseling for almost a two-year period, walking with a counselor.
on a weekly basis to receive the healing that I needed and the growth that I needed..
Maybe that's part of your journey too..
I want to pull all this together and I want to encourage you and challenge you by speaking.
something that I think is important for each one of us to understand as we draw this to.
a close..
Let me say this over us..
As we learn to walk together in open, broken and vulnerable relationships, caring for one.
another as we discuss our struggles and issues and rejoicing together in our victories and.
our healing, we are going to see personal growth in each one of us like never before..
That's my heart for you, church, that you would grow out of the trauma that you're feeling.
in a way that you connect in a deeper level with relationships around you like never before..
Like the woman in the story that we looked at today, will you hear God calling you out.
of the midst of the trauma that you're in right now and inviting you forward to be known?.
And will you, like the woman, step forward?.
That's my challenge..
That's my offer to you today, to step forward into the arms of God's grace again and to.
be reestablished in those deepest relationships and to know that out of them will come the.
greatest amount of growth and healing from your trauma..
Let me pray for us, church..
Let's pray..
Father, we just are so grateful for this moment..
So grateful that we get to stand together as a fellowship and as a community..
And Lord, we recognize that so often we do retreat from the amazing people that you've.
put around us, that so often in our anxiety and our stress and our trauma, we turn inwards.
rather than outwards..
We recognize that we put up barriers and walls to keep people out at times..
And Father, some of us have done that consciously and some of us do that subconsciously..
But Lord, it creates a barrier from the thing that actually is going to provide the greatest.
amount of healing for us, the men and women that you've put in our lives to love us and.
care for us..
And so Lord, we come to you in this moment, humble before you, realizing that in the midst.
of the crowd, you make a space and you don't force us..
But in your grace, you say, "Will you step forward and be known?".
And I know from my own journey, being known before, first of all, my wife and then being.
known before the church in the years ahead has been the thing that's really helped to.

$^{561}$strengthen my healing..
And Father, I believe that's going to be the same for many listening to this today..
Lord, where the enemy's had a strategy to keep things in the dark..
Father, I want to pray over those particularly that are struggling with secret sin, with.
stuff that's going on in their lives that no one knows about, but is tearing them up.
inside..
Lord, the enemy will create a strategy of shame and fear..
Oh, what if people knew that they wouldn't be your friend anymore?.
And like that prayer I said to Chris, I said, "If I fail, will you still love me?".
Those are the tricks that the enemy so often plays in our minds when we're dealing with.
secret sin..
Oh, if somebody knew this about me, they would never accept me ever again..
Father, we stand against that in the name of Jesus..
And we pray the love of Christ of every person watching this right now..
And Lord, where the enemy would try to cut us off from one another..
Father, we pray for a release of your Spirit upon your people, that anyone who's walking.
in darkness today would look up towards the light and would like that woman have the courage.
to take the risk to step forward this week..
So that's where the pastor here at the Vine, their community group leader, maybe it's their.
spouse, maybe it's a prayer partner that they're comfortable with, but they're willing to step.
forward and reach out and say, "I will be known.".
So Father, we pray for the power of your Spirit to be at work for each person on that journey.
this week..
And we thank you for the relationships you've placed around us..
Lord, may they be great wisdom and strength and encouragement and power to us..
And we pray this in Jesus' name..
Everyone says..
\newpage



\section{}
\label{sec:d8XMdofd39c}
\textbf{2021-01-25 Church Everywhere Live: Post Traumatic Growth - Appreciation of Life [d8XMdofd39c].mp3}
\newline
\newline
連結: \href{https://youtube.com/watch?v=d8XMdofd39c}{\texttt{ https://youtube.com/watch?v=d8XMdofd39c}} ~~~~ 語音日期: 2021-01-25 
\newline
\newline
\hyperref[sec:hd9p2DZoYtI]{\small{< < < PREV SERMON < < <}}
~
\hyperref[sec:index]{\small{[返主目錄]}}
~
\hyperref[sec:bs6i66xQaTI]{\small{> > > NEXT SERMON > > >}}
\newline
\newline
$^{1}$Church, I'm very excited that we get to have Pastor Ellison with us today..
Hello, Ellison..
Hi, Carla..
It is so good that you are here..
I'm glad to be here..
I wonder, can we pray for you?.
Yes, please. I think I need that..
Do you want to pray for him? Yeah? Yeah?.
Yes..
I think they do..
Okay, great..
Father, we thank you for Ellison and we thank you for the Word that you have given him..
And I pray, Father, that our hearts would be ready to receive..
Soften our hearts..
Let us be open to your Spirit at work in us..
And I pray for Ellison..
May you anoint him, the words that he speaks, not from himself, but from you..
In Jesus' name..
Amen..
Thank you, Carla..
Thank you, team..
And good morning, Church..
Welcome..
So glad to be with you guys this morning..
So, when I was a young boy, when I was a kid, my mom worked around the central area..
And so quite often she would have me meet her during lunchtime or after work or something like that..
The only problem is I didn't like meeting her near her work..
And the reason was because it was quite scary..
See, she would always ask me to meet her behind the area in St. John's Cathedral..
And it was scary for a couple of reasons..
Partly because, you know, that cathedral as a young kid, sometimes it's quite dark there, right?.
So it looks kind of scary..
There's also a grave site there, which made it a bit scary..
But the one thing that I was most scared of was that quite often, hanging out in that area, there would be two old ladies..
They looked like twin sisters..
And they would hang out there a lot of the time..
Now, I wasn't scared of them because they were old..
But what I found quite scary at that point was that they looked quite disheveled, almost homeless..
But the really, really scary thing was every time I would walk past them, they would just look at me, glare at me, not in a friendly way..
In this really sort of scary eyes, they would look at me..

$^{41}$And as I would walk past or as I would be waiting for my mum, they would just stare at me..
And they were muttering things under their breath..
You couldn't quite make out what they were saying..
And so, you know, I talked to my friends about some of this..
Have you ever seen those two ladies?.
And that's when the rumors really started flying, right?.
They were like, "Oh, yeah, you know, those two ladies, they are homeless witches..
They are possessed by the devil..
And they used to be rich, but someone put a curse on them..
And so now they're poor and homeless..
And because they're witches, they try to abduct little children and do the same thing to them.".
Now, of course, as a young, naive, gullible kid, I believed every word these people said to me..
And the thing that furthered the intrigue was that they were white ladies..
They were foreigners..
They were westerners, which was--and they looked disheveled and homeless, which was not a usual sight in Hong Kong..
And so in my wild, childish little imagination, I had it in my head that these ladies really were witches..
And if I hung around them too long, if I got too close to them or something like that,.
they would turn me into lizards and I would never see my mum again or something like that..
They would cast a spell on me..
But the truth is, actually, I don't really know what the situation for these two ladies was..
They were obviously two people--now I think back about it--that were distressed and in need of help and support..
And I also think it's fair to say that, you know, this wasn't the life these ladies had planned to live..
I'm sure at one point they had thriving lives, lives that were full and full of fun and family and all that good stuff..
But something must have happened along the way,.
which meant that at this moment they weren't able to appreciate life as much as they used to..
We're continuing our series on post-traumatic growth today, and we're talking about appreciation of life..
And the passage that we're looking at today is a bit like what I experienced as a kid,.
only probably like a thousand times more intense, okay?.
And as we unpack this passage, we're going to be reading about a man who was literally possessed by demons..
And as a result, he too didn't have very much to look forward to or appreciate about life..
But as we're about to discover, church, a true encounter with Jesus turns that around..
And so if you're following on with your Bibles or your apps or something, if you want to put it on the screen,.
you can close down Facebook for a second, pull up the Bible app, okay?.
Mark chapter 5 is where we're at, okay? And that's where we're going to be reading from today..
So read along with me, all right? Here we go. Mark chapter 5..
"They went across the lake to the region of Gerasimus..
When Jesus got out of the boat, a man with an impure spirit came from the tombs to meet him..
This man lived in the tombs and no one could bind him, not even with a chain,.
for he had often been chained hand and foot, but he tore the chains apart and broke the irons on his feet..
No one was strong enough to subdue him..

$^{81}$Night and day among the tombs and in the hills, he would cry out and cut himself with stones.".
Now, before we get into this passage, let me give you a little bit of some background, okay?.
In chapter 5 of the book of Mark, it tells us that the disciples had just crossed the lake to reach this destination..
But along the way on the lake, they encountered a really fierce storm..
A storm so fierce that Jesus and his disciples, the disciples especially, were all scared..
Now, the Sea of Galilee is quite famous for its sudden storms, very stormy conditions..
And Jesus and his disciples find themselves in the middle of one of these storms..
A storm so big that, like I said, seasoned fishermen even find themselves scared in it, feared for their safety..
And you also have to understand one thing, that the sea for the Jewish people carried with it this sense of evil and darkness..
This is where scary things came from. This was the place where monsters came from..
The sea was a dark and scary, evil place..
Now, I'm not sure if many of you have been in a stormy boat..
I have been on the Star Ferry on a T3, okay?.
And it's not quite the same, okay, but that's the closest I've been..
But even that was scary enough..
People flying about in the sea, you know, pumping sliced lattes being spilled all over the place..
You know, people screaming and shouting..
This was probably like a million times worse than that..
But in the middle of this storm, it tells us that Jesus was chilled..
He's relaxed. He's asleep even..
And his disciples wake him up screaming, "What? We're about to die, Jesus. Don't you care we're about to die?".
And they think they're about to be swallowed up and drowned by this big evil sea..
Now, Jesus wakes up and with a short phrase, he just says, "Quiet! Be still!".
And everything is calmed..
I always find that amazing as a kid, you know..
Have you ever, as a kid, stepped out in a storm and said, "Quiet! Be still!".
And nothing ever happens, right?.
Those words never work..
Especially if you're having an argument with your spouse..
Never say, "Calm down," okay?.
Because it just doesn't work..
Never, ever, ever works..
But it does for Jesus..
He rebukes the sea and it calms down..
And afterwards, he turns and he sort of rebukes his disciples at the same time..
He says to them, "Why are you so afraid? Do you still have no faith?".
And it says that his disciples reacted in fear..
They were terrified..
And I'm paraphrasing a little bit here, but they turn and say to each other, "Who is this guy?".
"What power does he have? Even the wind and the seas obey him. Who does that?".

$^{121}$You know, Church, I don't think we'd be far off wrong if we likened our past 20 months or so.
to being on a boat with Jesus, tossed around in the sea..
Sure, Jesus has been with us, right?.
But at times, hasn't it feel like that he's been asleep?.
How many times have we cried out individually and as a community, calling out to him, saying those exact words?.
"Jesus, don't you care what happens to this city?".
"Don't you see the things that this church, your people are going through?".
And we've been desperate for Jesus just to come and say a word and to calm everything.
and to make these big, scary things disappear..
And in the waiting for things to get better, it often feels like we lose a little bit of how to appreciate life at this moment, doesn't it?.
We were also looking forward to saying goodbye to 2020 and welcoming 2021.
to bring some peace and stability into our lives..
But it seems as though we've just been thrust into another storm, right?.
It seems like COVID is still all over the place and death rates are still rising in a lot of countries,.
even as vaccines are being rolled out..
In the US, recently, we saw some very disturbing political scenes unraveling..
In our own city, political tensions remain high..
Frustration over confusing COVID lockdowns and restrictions and all this kind of stuff..
We barely caught our breath out of 2020 and now 2021, we're facing storm after storm after storm..
And the same thing is happening to the disciples..
They get to the other side of the lake and just like the waves that rush towards the boat,.
a demon-possessed man comes rushing towards Jesus and the disciples..
And this man has been tossed and turned by the evil spirits he's been tormented with for some time now..
But Jesus, after calming the sea, is about to calm the storm in this man's life too..
And now, before we go any further, I just want to acknowledge something here, okay?.
And this is it, church..
Spiritual warfare is a real thing..
Alright, yes, I know we live in a messed up and broken world.
and a lot of the things that come our way is because of our own stupidity and foolishness and our own sin..
Okay, there's consequences and we have to take responsibility for those things..
But at the same time, there's also a very real enemy, Satan and his minions,.
these demons that are actively trying to disrupt you and harm you in any number of ways..
We're not going to go into too much detail about this, only to say two things, right?.
Demons can either affect us by oppression or possession..
Now, as Christians, if you're a Christian listening to this,.
if you receive the Holy Spirit into your life, we don't need to worry about possession..
I don't believe that Satan and his demons can possess a Christian..
But he can certainly use things to try and harm you and to oppress you..
However, what Mark describes clearly here is someone who's been possessed by demons..
We know this because he goes into some pretty graphic details about the state of this man..

$^{161}$It tells us that he lived in the tombs, he had some sort of supernatural strength.
and not even the chains could hold him down..
And I'm not talking about the good charismatic way that we sing about in church sometimes..
These were literal chains that people tried to put on him and he broke them off..
He would cry out at night like a wild animal..
His body was covered in wounds and scars because he had used stones to cut and hurt himself..
It's almost like Mark is not describing a man here, but more like some kind of monster or an animal..
Even the word Mark uses to describe him, the way people would treat him, it says here "subdue".
but a more direct translation of that word would be "to tame.".
You don't tame human beings, you tame wild animals..
And the community had outcast him..
The community that he had once been a part of had thrown him out there into the night..
They cast him up, they tried to chain him up, tie him down, abandoning him to the outskirts of town..
As I was reading this passage in Mark 5, I often wondered to myself,.
I wonder if that's what those two old ladies felt like sometimes..
Sure, they looked a bit scary, maybe they even acted a bit scary..
And it was clear that there's not just me, other people were afraid to approach them too..
But did the community around them ever really truly try to reach out to them or help them?.
Or did they just give up on them, abandoning them to the outskirts of society?.
Or maybe perhaps you listening right now are relating on a personal level..
As Pastor Andrew said at the beginning of this series,.
like this man, perhaps you found yourself outcast and abandoned, cut by stones..
You have these wounds and scars that you're not too sure what to do about..
And in talking about post-traumatic growth and this appreciation for life,.
we have to realize that this isn't something we can do our own..
This is a call we've been saying time and time again..
We can't do this by ourselves, but rather we need each other to help us move forward.
from mourning to healing and into a space of thriving again..
You ask anyone who's walked through some stuff in their lives,.
and they will tell you that they probably didn't do it alone..
I can almost guarantee you that there was people surrounding them,.
journeying with them through that place..
Vine Church, are we the kind of community that says to these kind of people,.
"Stay with us. We love you. We welcome you. We're not scared of you..
We want to walk through the pain with you.".
Or are we going to be the kind of community that says,.
"You know what? Out of sight, out of mind. We can't reach those people anyway..
They can go out there and scream and do whatever they want to do,.
and we'll just carry on our lives as normal.".
The community had given up on this man, but Jesus wasn't about to do that..

$^{201}$We carry on reading. Verse 6, it says this,.
"When he saw Jesus from a distance, he ran, fell on his knees in front of him..
He shouted at the top of his voice, 'What do you want with me, Jesus, son of the Most High God?.
In God's name, don't torture me.'.
For Jesus had said to him, 'Come out of this man, you evil spirit.'".
This man possessed by the evil spirits rushes towards Jesus,.
and he knows exactly who Jesus is..
He knows his name, he knows his identity, and he knows his power..
But even though he knows who Jesus is, he really doesn't want anything to do with him..
Now this is all happening in the context of a demon-possessed man, of course,.
but if we look at our own lives, do we ever react in the same way, perhaps?.
A lot of us will know, will say we know at least, who Jesus is..
A lot of us, we say, we acknowledge his power, we know his name..
But many times, for whatever reasons, we might truly fail to worship him..
And when I say worship here, I don't just mean singing along with the beautiful worship team..
I don't even mean just reading a Bible, or even spending time in prayer..
True worship is letting God invade every single part of your life..
Every single part of your life..
And I know that in my own life, it's often hard to take control of that..
I want autonomy, I want to make my own decisions,.
I don't want God's opinion in how to use my finances, my time, my energies..
Sometimes, to me, obedience feels more like resentment..
And perhaps, one of the reasons we lose the appreciation of life,.
when it gets stormy, is because we tend to lose the appreciation of the giver of life..
We stop Jesus from working in us..
We stop being thankful, we put up all these barriers to the Lord,.
and then we suddenly start to think to ourselves,.
where's all the hope gone? Where's all the joy gone?.
This demon-possessed man clearly has many things, many barriers,.
that are stopping him from truly connecting with Jesus..
And Jesus sees his pain, and he reaches out to him..
And of course, this is very typical of the way Jesus lived his life..
Where others would point and stare and stay away, Jesus invites in..
Jesus reaches out to these outcasts, the lepers, the sinners, prostitutes, tax selectors, demon-possessed people..
Jesus reaches out to them, and each time Jesus reaches out to these people in love and healing,.
he does so without condition..
This is an important thing for us to remember..
Notice that Jesus doesn't first of all question this man before casting out these demons..
He doesn't give him a lecture, "Tus, tus, tus, you naughty boy, you know you shouldn't have done that..
This is why you ended up like this in the first place.".

$^{241}$He doesn't force him to repent from his sins before he heals..
He simply heals..
He sees a distorted image of God..
A human being made in God's image has been completely distorted right now,.
and he reaches out in love..
This is another important thing for those of us who might be walking alongside people in post-traumatic growth..
Often it's tempting, isn't it, to play the role of judge before we move on to helping..
We might even do this subconsciously..
We blame the victim instead of addressing the source of the pain..
But a person who truly appreciates the beauty of life that Jesus wants us to have.
will also help others do the same..
And when we blame the victim, it does the opposite of this..
What Jesus realizes is that this was a man being tormented by these demons..
It didn't matter to Jesus how he got into this state..
All he sees is someone who needs help, and he reaches out..
And our challenge is this, church, are we willing to do the same?.
Are we willing to reach into the places where evil is in the name of Jesus.
and rescue these people out of the darkness by his power, by his grace?.
It tells us very specifically as we read on,.
the source for this man's pain was the fact, like we said, he's been possessed by many demons..
In fact, this demon has a name..
It says, "Legion." Mark tells us this, okay?.
Verse 9, "Then Jesus asked him, 'What is your name?'.
'My name is Legion,' he replied, 'for we are many.'.
And he begged Jesus again and again not to send them out of the area.".
Here's an interesting thing I found about this..
You know, often when we go through trauma, right?.
When we go through trauma, it's not always as simple as just one thing..
Legion means many things..
And for example, if we walk alongside people with depression,.
depression isn't just simple as feeling low, right?.
It carries with it things like anxiety, lack of sleep, lack of appetite, risky behavior, mood swings, those kind of things..
Or if you look at someone walking through grief,.
grief is this journey where there's ups and downs, there's twists and turns we don't see..
It's a long, thorny, and winding road..
And so as we look at our own trauma and the trauma of those around us,.
we must be careful sometimes not to oversimplify things..
We must be patient with ourselves and the people that we're walking alongside..
But there's also a bigger picture here I think the scripture wants us to capture as well..
When the name Legion was mentioned,.

$^{281}$it would have immediately struck a chord with the original readers of the book of Mark..
And what we have to remember is at this time, Roman soldiers overran the territory..
Legion was used to describe 6,000 soldiers..
And these soldiers would go around inflicting pain on people, overtaking everything they wanted, everything they saw..
And so in some way, what was happening to this man is a reflection of what was happening on the wider scale in the society around them..
And just as this man was longing to be freed from this legion of demons,.
society at that point was longing to be freed from the clutches of Rome..
And Rome was seen by the Jews at this time as this evil..
Some people even thought that Rome was a manifestation of Satan himself..
And Jesus in entering the scene though, his plan was even greater than just freeing this man..
His plan was greater than even just freeing the nation of Israel from the clutches of Rome..
His plan was to free the entire world from darkness..
Jesus realizes that there's a deep spiritual battle here that needs to be won..
Not just Rome, but the entire world..
Romans 8.22 tells us this, "For we know that the whole creation groans and suffers together.".
And the ultimate way Jesus is going to defeat evil once and for all is of course on the cross..
You see, as we look at the story right here in this moment, we look at Jesus and this demon possessed man..
That doesn't seem like there's very much in common with the two of them, right?.
Jesus has community, he's dressed, he's healthy..
This man is naked, alone, cut up and wounded..
But actually, it's not long after meeting this man that Jesus would too end up abandoned by his disciples..
Screaming in pain, put to death, left on a hill, on a cross, amongst the tombs, half naked..
His body cut up and tortured by the whips of the Roman soldiers..
And the sins of the world and the evils of the world upon his shoulders..
And it's three days later where the resurrection comes..
That Satan and evil and the demonic powers of this world are defeated once and for all..
So what's happening in this encounter with this demon possessed man,.
it's simply a smaller scale demonstration of what Jesus has ultimately come to do..
And this is the truth, church..
Jesus is and will be victorious over all sin and evil in this world..
And this legion of demons inside this man already knows what's up..
This is how the story ends..
A large herd of pigs were feeding on a hillside nearby..
The demons begged Jesus, "Send us among the pigs, allow us to go into them.".
He gave them permission and the in-power spirits came out and went into the pigs..
And the demons heard about 2,000 in numbers rushed down the steep bank into the lake and were drowned..
It's a pretty dramatic scene..
Imagine the commotions..
But what this is really proclaiming is this..
One day, the fate of all evil, Satan and his demons will ultimately be cast out and destroyed..

$^{321}$The fate of evil is always going to be defeat in the name of Jesus..
Because this is what Jesus comes to do..
This is what Jesus is all about..
Defeating evil, overcoming trauma, and restoring people..
And it says this..
When they came to Jesus, they saw the man who had been possessed by the legion of demons sitting there,.
dressed and in his right mind, and the community was afraid..
This is what happens when Jesus heals..
The change is dramatic..
Blind can see, the lame can walk, dead come back to life, rich give to poor,.
and the demon possessed are set free from their torment..
I want us to realize this, church..
Healing is possible..
If you're waiting for God's healing right now, healing is possible..
Overcoming trauma is possible..
There will be coming a time where you will not be tormented by whatever it is that you're wrestling with anymore..
And surely this is something that we can all celebrate, right?.
But this community reacts in fear..
Why?.
What were they afraid of?.
Maybe they were afraid in the same way that the disciples were afraid that we talked about just now..
This man has just suddenly controlled this guy we've been trying to control all this time..
How can that be?.
Maybe they were scared about the fact that they just lost 2,000 pigs,.
which would have been a huge financial loss for them..
If Jesus stuck around, maybe the financial losses would be even worse..
But how about us, church?.
This is a question we need to ask ourselves..
Do we want to see stories like this man become real in our community?.
Do we really want to see people break free from the traumatic things that have been plaguing their life?.
Because if we do, we need to prepare ourselves to walk in some of the messy things that come along with that..
It's going to cost us something..
But we don't need to be afraid of the enemy and his schemes..
We need to overcome our fear of welcoming those who are coming in with the wounds and scars and demons,.
broken and in need of help, and simply be a place where we can allow the presence of God to be at work in their lives..
But here's something more amazing that I want us to see..
In healing this man, Jesus didn't just cast out his demons,.
but more importantly, Jesus gave him something new to live for..
This is how the story really ends..
Pastors always do that..

$^{361}$They say, "I'm coming to an end," and then they don't come to an end..
I'm really coming to an end now..
As Jesus was getting into the boat, the man who had been demon-possessed begged to go with him..
Jesus did not let him, but said, "Go home to your own people..
Tell them how much the Lord has done for you and how he has had mercy on you.".
So the man went away and began to tell the Decapolis how much Jesus had done for him..
And all the people were amazed..
You see, when this man was demon-possessed, there was nothing he had to appreciate about life..
He lived in the tombs. He lived among darkness..
There was nothing that he could appreciate..
But one of the most crucial steps in our journey of post-traumatic growth is that once we are made well,.
once we have been healed, our task is not just to keep Jesus to ourselves..
Our task, like this man, is to go and tell others..
Jesus has something even better for him than staying with him..
He says, "I've got something better for you..
Go and tell everyone what the Lord has done for you..
The demons are gone. New life is here.".
And this new life doesn't mean all memory about his past trauma has been erased..
Like I said, rather it tells him to go..
Share the pain that you've been through..
And in a very strange way, isn't it, church,.
that one of the gifts we get out of coming out of traumatic experience is an even more deeper appreciation of the life that we've been given?.
As I say this, I'm thinking about people like Alfred here..
If you read the devotional last Sunday, you would have read about Alfred.
and how despite his ongoing battles to get healthy, he continues to appreciate each day..
I was just on a Zoom call with him yesterday, Alfred..
And church, he's having surgery on Tuesday..
So if you would join me in praying for him, that would be amazing..
But believe me when I tell you that Alfred is truly a guy who lives appreciating life every single day..
He preaches hope to the young adult community that he's part of way better than a sermon ever could..
Alfred, if you're watching this, we want you to know that we love you and we're praying for you..
Of course, I'm thinking about my friend Chelsea here,.
who in the midst of a pain and life looking totally different now, still continues to be a wonderful mom to Emery,.
an amazing friend to those around her and a voice of raw and real honesty about how God is walking with her in life..
And Chelsea, if you're watching this, know that we continue to love you and walk alongside you too..
And this formerly possessed man, despite all the pain and shame that would have come of being that guy,.
imagine being that guy for that long, he goes around now to travel and proclaim the freedom Jesus has given to him..
I found a really good quote as I was scrolling through Instagram..
Some good stuff does come from Instagram..
And it was from an account by a psychologist called Dr..

$^{401}$Caroline Leaf..
And she said this..
She says, don't be ashamed of your trauma or history..
One day, your story will become someone else's survival guide and inspiration..
By sharing your story of tragedy and triumph, someone may find the hope and courage to continue waiting..
And I would add to that by sharing what God has done in your life, someone too will be encouraged by the presence of God..
Now, I have no idea about the things, most of the things that you guys are wrestling with right now or what you've been through..
But what I do know is this..
What I do know is the words, the story that we've just read are true..
And I hope that in walking past this passage today, you can see that..
Firstly, yes, there is healing in Jesus name..
Every single thing that we wrestle with can and will be brought under the healing power, the healing power of Jesus..
But after healing church, there is life..
And when Jesus comes to storm in our life, when he frees us from the things that torment us, we are to live life appreciating each and every moment..
And what Jesus did for this demon possessed man, he can do for you, for your family, for society and for the world..
Our task then is simply to be proclaimers and believers of this wonderful news that in the name of Jesus is life and life in the fullest..
So church, please join me in prayer right now, as I would just spend a little bit of time praying and listening to God and seeing how he wants to bring you out into a journey of trauma, into appreciating life again..
Pray with me church..
Jesus, I thank you that there is nothing under your sovereign control, the wind and the waves, the universe, everything we see around us, including our own selves, including the forces of evil have been brought under your power..
And so Lord, I pray for those, those of us who are listening, who are watching right now, who are still feel like themselves are being tossed and turned by the sea, who are still being plagued by sickness and evil..
Lord, we pray for a release of that right now in Jesus name..
We pray for calm in Jesus name..
We pray for demons to cease their work in Jesus name..
And Father, I pray that you would bring us out of this darkness into a place of healing, into a place where we can truly appreciate this wonderful gift of life that you've given to each and every single one of us..
And as we do that, Lord, I pray that our lives would be a proclamation of who you are, a demonstration of your love..
We would reach out in love and kindness..
We would use the pain that we've been through to speak hope and love and joy in your name to those around us..
So Jesus, we thank you..
We love you..
Come rescue us..
Come heal us..
Come give us life..
In your beautiful name..
Amen..
\newpage



\section{}
\label{sec:bs6i66xQaTI}
\textbf{2021-01-31 Church Everywhere Live: Post Traumatic Growth -  Personal Strength [bs6i66xQaTI].mp3}
\newline
\newline
連結: \href{https://youtube.com/watch?v=bs6i66xQaTI}{\texttt{ https://youtube.com/watch?v=bs6i66xQaTI}} ~~~~ 語音日期: 2021-01-31 
\newline
\newline
\hyperref[sec:d8XMdofd39c]{\small{< < < PREV SERMON < < <}}
~
\hyperref[sec:index]{\small{[返主目錄]}}
~
\hyperref[sec:PMXFFRS8kEs]{\small{> > > NEXT SERMON > > >}}
\newline
\newline
$^{1}$So a couple of years ago,.
I felt God challenged me deeply to fast food for 21 days..
I felt like God was saying,.
"Okay, Andrew, I want you to physically sacrifice,.
to physically suffer in order to.
press into obedience with me.".
And so for 21 days,.
I allowed nothing in my body other than water,.
no food at all..
And I have to say, I struggled and wrestled.
and went back and forth as to whether I should actually.
share this story with you today,.
because I'm concerned, first of all,.
that you might think that in sharing this,.
that I'm kind of putting myself up there.
as some super holy person,.
but I trust if you're part of the Vine Church,.
you know that I'm not the super holy person..
Okay, so that's good..
But the second thing that I was concerned about.
is that you might hear me tell this story,.
and then you might think that this is what you should do,.
that you should copy me and try to do something like that..
And I just wanna say that actually,.
when you do something like this,.
it has to be a very specific call from God..
It is incredibly difficult..
It is actually physically quite dangerous..
And what God calls us to in those moments.
is quite a radical thing..
And so when I did it,.
I made sure I had a community of support around me..
I made sure I was accountable to people..
I wasn't just doing it on my own..
And I knew that I had the calling and the grace of God.
to step into that moment..
So make sure that that's for you,.
if that's what God ever calls you into..
Most of the time, I tell people not to do it..
But I share with you today about that experience.

$^{41}$because of what it birthed in me,.
what it actually forged in me that I think is central.
to what God wants to say to us today in his scriptures..
Fasting for 21 days, having no food in my body.
was the hardest thing.
that I have probably physically ever done..
Every single day, of course,.
my body was crying out for calories, for nutrition,.
for nutrients, for whatever it needed.
to have that energy to keep going..
But actually the hardest part of the journey of that 21 days.
wasn't the physical pain,.
it was the mental suffering and anguish that I went through..
It's incredible what your mind does.
when your body is crying out for food..
And I had constant days.
where there was just this mental anguish and suffering.
where I was fighting the temptations..
The irony is in this 21 days,.
my family was watching actually a cooking show..
I kid you not, we were watching every night a cooking show..
And I had to sit there and kind of go,.
I am not gonna be tempted to eat food in any way..
And the only way I got through these 21 days.
was through the work of the Holy Spirit,.
was through his grace and his empowerment of me,.
the way that he strengthened me.
and moved in my heart and my life..
That was the only way I was able.
to actually get through that time..
And I realized that what God was asking,.
the challenge he was requiring of me was,.
will I put him first?.
No matter how much I suffer,.
no matter how much I'm sacrificing,.
will I actually put him first?.
Will I rely upon him and his help and his spirit.
above anything else?.
Even when my body and my mind is trying to fight against it,.
will I stand for him?.

$^{81}$I learned out of those 21 days,.
that actually if God calls me to serve,.
and if that serving comes with suffering and hardship,.
then by his grace, I am actually able.
to adhere to that service and to follow him..
That personal strength was forged in me in that time..
And it's not lost on me that actually.
that lesson I had to learn happened just one year.
before 2019 and 2020..
That God knew all of the stuff.
that Hong Kong was about to enter into..
And he knew what the vine was about to face.
as it was dealing with a political upheaval in its city,.
as it was dealing then eventually with a pandemic.
that was going global..
As we were dealing with all that,.
God knew the stress and the pressure and the suffering.
that would come on me in my leadership.
in the midst of that..
And prior to that, he's like,.
I'm gonna take you through something.
where you're gonna discover a personal strength in you.
that's going to be there when you need it the most..
See, if post-traumatic growth teaches us anything,.
it teaches us this, that actually trauma can be the place.
where we discover a newness in life,.
a discovery of a personal strength,.
where we find ourselves a conviction of character.
that would not have been there.
unless we had been through that trauma in the first place..
That we might be able to emerge out of our hard times.
going, you know what?.
I know myself better..
I may not be perfect,.
but I know that there is a conviction of the gospel.
that sits inside of me,.
a conviction of what Christ is calling me to.
that wasn't there, that I couldn't rely on.
prior to that trauma, but now it is..
You see, the survival of a highly traumatic experience.

$^{121}$forges in us a character and a focus of mind.
that rarely exists in those that have not faced.
a great trial or a great adversity..
And you see, this is the beautiful thing, I think,.
about what we've been looking at in this series,.
this great invitation for us to think about our traumas,.
not as the stress they create,.
but actually as something that brings us.
into a newness with God..
And here in my years of pastoring at the Vine,.
I've seen some of the worst things..
I've seen people lose jobs..
I've seen people lose family..
I've seen people really struggle in their marriages..
I've seen divorce and infidelity..
I've seen depression and anxiety grip people,.
mental health issues..
I mean, you name it..
And almost every single case, when you look at it,.
these places are fertile ground for God to do something.
in those people that would not have occurred otherwise..
Here's the thing, our trauma can actually create a boldness,.
a courage, and a conviction of character.
that is found in us because we've experienced a hardship.
and we have actually overcome it..
You know, the apostles are a case in point.
in the death and resurrection of Jesus..
The apostles show to us this beautiful picture.
of what it is to stand with a new.
and renewed personal strength in God,.
despite what they have been through..
We've been sharing a lot about the apostles' stories..
I shared a couple of weeks ago about Peter.
and what we see in Peter's life.
and how Peter in the arrest of Jesus,.
and he follows the crowd into the courtyards.
of the high priest and there warming himself by the fire,.
realizes that he's in trouble,.
realizing that actually he might,.
in his own life, might also be at risk here..

$^{161}$And as he's standing around that fireplace,.
he hears the call of that on his spirit.
and the trauma and the fear of it..
And he denies Jesus three times right there..
And then in the shame and the guilt of that,.
he retreats to the beach,.
goes back to the very thing that he used to do,.
a fisherman before, not a fisher of men anymore,.
but a fisher of fish, literally,.
into the comfort and security of what he's always known..
And Jesus has to go there, meet him,.
create a fire for him and release him back.
into the restoration of who he is.
by asking him three times, "Do you love me?".
And that journey that Peter goes on.
is part of the trauma he needed.
to form what we eventually see in him..
And it's not just Peter..
All the apostles suffer Good Friday.
and hide themselves in fear on Easter Saturday,.
but it's in the resurrection of Jesus.
and in the outpouring of the Holy Spirit at Pentecost.
that there is such a personal strength of character.
and conviction that falls upon the apostles.
that we see them do incredible feats for the church..
We see them do for the church.
what the gates of hell cannot come against it..
And here's the thing,.
the book of Acts is the picture of a group of people.
who have such profound conviction and personal strength.
simply because of the trauma they had been through.
and the resurrection of Jesus.
and the power of the Spirit that's been birthed upon them..
And you see throughout the book of Acts,.
amazing moments where these apostles,.
the ones that had hid, the ones that had fled,.
the ones that were insecure,.
were able to stand in front of the worst powers.
that were trying to persecute them.
and not falter in their conviction of the gospel to say,.

$^{201}$"I will not be ashamed of the thing.
that has been formed in me by the resurrection of Jesus.
and the outpouring of the Holy Spirit.".
I wanna encourage you today with this,.
that I believe that same personal strength.
and that same sense of conviction can also be in you..
Actually, let me say it this way, it needs to be in you..
I wanna say today that I believe.
that the last one and a half years here in Hong Kong.
and all the trauma that it's created for all of us.
is actually the very thing that we have needed.
to be able to have a personal strength and conviction.
of the gospel in us for everything that is about to come..
I believe that God is gonna meet you in this hour.
and in this time, and He's gonna so strengthen your heart.
that no matter what might take place in the future,.
you will stand up and be counted for the gospel..
Could you imagine a church like that?.
Acts gives us a picture of that church..
And today I wanna share a story actually.
out of Acts chapter five..
It's right early on in the story of the early church..
It's literally probably just a few months.
from the death and resurrection of Jesus..
And in this story, the apostles have begun the church.
by regularly meeting every day on the temple courts.
just right outside the temple in Jerusalem..
And they would gather in those courts.
and they would share boldly the message of Jesus..
They would say that Jesus is the Messiah,.
that Jesus is the one that God has raised from the dead,.
ascended to the heavenly realms,.
that He's the one with all the power and authority.
to forgive and restore..
And they preached the gospel and many people,.
many Jewish people were coming to faith..
Now, perhaps not surprisingly,.
when this small little sect of Judaism was beginning.
and beginning to say that there is this carpenter.
from Nazareth who was actually the son of God,.

$^{241}$the powers to be, the power of the Jewish authorities.
of the day absolutely hated the guys for doing this..
I mean, they were so angry and so upset..
They were spreading this rumor about Jesus being the Messiah.
and they were doing everything in their power.
to stamp it out..
Anytime that you speak out the gospel.
and that gospel stands against people who are in power,.
that power is going to fight back..
And what you see in Acts chapter five.
is this power asserting itself onto the apostles..
The power was called the Sanhedrin..
The Sanhedrin was 71 men who actually were responsible.
for the control of the worship that happened in the temple..
They essentially did three things..
They set the temple service..
They actually controlled all of the bartering system.
that took place for the sacrifices..
So when people came to bring a sacrifice.
and they purchased the sacrifice at the temple,.
that was controlled by the 71 in the Sanhedrin..
They actually profited off of that sacrificial system..
And the third thing they did was actually set.
the common laws for the Jewish people.
and how they should interact with one another..
So they were both the religious authority,.
but also the political authority.
for how the Jewish people worked.
and lived together in that time..
And these people drag the apostles before them.
to hold them to court, to basically say,.
we want you to stop preaching the gospel.
because what you're doing in our eyes.
is blasphemous..
I wanna pick up the story actually from that point..
And I'm gonna start in verse 27 of Acts chapter five..
Is everybody okay?.
You guys all right?.
You're with me?.
Okay, cool..

$^{281}$I want to read this to you..
Here it says this..
Having brought the apostles, that's the Sanhedrin,.
they made them appear before the Sanhedrin.
to be questioned by the high priests..
We gave you strict orders not to teach in his name,.
they said, yet you have filled Jerusalem.
with your teaching and are determined.
to make us guilty of this man's blood..
I want you to notice what's happening here..
They come and they call the whole of the apostles to account.
and they actually, funnily enough,.
kind of reveal all their cards at once..
They say, look, we know that you're going around preaching,.
even though we told you not to..
And we know that you're filling the whole land.
with this idea of what it is to see Jesus as the Messiah..
And not only that, notice what they say here..
He says, you are determined to make us guilty.
of this man's blood..
You're determined to put his death on our shoulders..
So they're basically charging the apostles.
with heresy, with blasphemy,.
and with kind of charging the Sanhedrin.
with the guilt of the murder of this Jesus from Nazareth..
I want you to see how Peter responds, starting in verse 29..
Peter and the other apostles replied this..
We must obey God rather than human beings..
The God of our father raised Jesus from the dead,.
whom you had killed by hanging him on a tree..
God exalted him to his own right hand as Prince and Savior,.
that he might give repentance and forgiveness.
to the sins of Israel..
We are witnesses of these things..
And so it is the Holy Spirit whom God has now given.
to those who obey him..
Check this church..
This is the same Peter who just months prior.
is standing in the courtyard of the high priest's house,.
warming himself at the fire..

$^{321}$And when they say, you knew Jesus, he denies it three times..
Now, just a few months later,.
he's standing in front of the high priest,.
in front of all of the Sanhedrin,.
and he basically says three things to them..
First of all, you're a bunch of murderers..
That you've actually hung this Jesus on the cross..
That it was you, that the blood is actually on your hands..
And then he says, not only that,.
but you're actually a bunch of sinners..
He actually says, here's the crazy thing..
God has raised Jesus from the death,.
and he's put him now at the right hand of the father..
The right hand always symbolized the place of power.
and authority, and Peter preaches the gospel to them..
He says, now, because of the Savior and Prince, Lord Jesus,.
at the right hand of the father,.
anyone who asks for his forgiveness,.
anyone who repents will be forgiven by God..
He's placing it before the Sanhedrin.
and before the high priest..
He says, I know you killed Jesus, but guess what?.
There is still grace, even for you..
There is still the ability for you.
to come under the grace of God,.
'cause Jesus is now seated at the right hand of the father..
And if you confess, he will respond and forgive you..
And if that wasn't enough,.
then Peter throws in an extra piece of hot sauce..
He basically goes, hey, you know this Holy Spirit thing?.
This Holy Spirit thing that you try to control.
as the Sanhedrin in the temple,.
that you've kind of made all about the temple..
Well, guess what?.
That Holy Spirit right now is poured out.
on anyone who's obedient to Jesus..
Not just related anymore to what's happening.
in this building, but to anyone who's obedient with Jesus..
Peter stands in front of the Sanhedrin,.
and he says, you're murderers, you're sinners,.

$^{361}$and guess what?.
The Holy Spirit is not contained by your laws.
or by your power anymore..
And you have to ask yourself, how does Peter do this?.
I mean, what has happened in him?.
What is driving him to have such fortitude.
of his convictions that he's willing to say these things.
to the most powerful people around him in the region?.
Well, we find that answer right at the start of verse 29..
He says this, we must obey God rather than human beings..
We must obey God rather than human beings..
No, this is Peter..
And Peter, it's almost like he's saying this,.
I know what it is not to obey God..
I know what the trauma has done to me..
I know what it was to stand by that fire on that day.
and deny even knowing Jesus..
I know what it's like to allow fear to hold back the gospel..
I know what it's like to confess that I got it wrong,.
that I screwed up, but guess what?.
Jesus came and met me and he met me and restored me.
and renewed me and turned me into the person I am today..
And through all of that trauma,.
a conviction of spirit has come..
I will never deny him again..
You know what?.
I will always obey God, even if it comes at the cost.
of disobeying humanity, human beings, the human law..
Now, I want you to hear this..
Peter is not totally disregarding the law..
He's not trying to stir up people to be rebellious.
against the human law..
Peter lived most of his life in conviction and obedience.
to the Jewish laws around him..
But what he is saying is this,.
when those laws come into conflict with the gospel,.
you will always find me on the side of the gospel..
When those two things conflict,.
I will always stand up and be counseled.
for the blood of Jesus..

$^{401}$Maybe before in my story,.
I found myself on the wrong side of that equation,.
but through the fortitude that is now birthed.
and forged like diamonds is the crushing of coal.
and the crushing of my trauma..
I stand before you now and I say,.
I am on the side of the gospel..
I will obey God, even if it means having to disobey you..
I mean, this is Peter saying, I put God first..
My 21 days of fasting was me learning what it was,.
despite everything my body and my mind was saying.
to put God first..
And I know what you're thinking..
You're thinking, oh, Andrew, I put God first..
That's what I do..
Yeah, I'm a God first person..
Me and my family, we put God first..
Here's something that I've discovered.
in all my years of pastoring..
It's easy to put God first when it comes.
with little personal costs associated with living it out..
But when we put in God first,.
when it actually convicts us,.
when it actually comes with great cost.
and great personal suffering,.
when that happens, we so easily find ourselves.
on the side of what is safe rather than what is faithful..
When personal costs, when putting Jesus first.
actually impacts us, when actually he's gonna have.
a problem, a suffering for us,.
we so often side with the safety net.
rather than being willing to stand up.
and be counted as faithful..
That's the lesson the apostles had discovered..
That's the thing that God had showed them..
And they're willing to stand in front of the Sanhedrin.
no matter what the Sanhedrin might do.
and declare the reality of that..
And the Sanhedrin wants to kill them,.
literally wants to kill them..

$^{441}$And Gamaliel stands up, one of the Sanhedrin,.
and says, no, no, don't kill them..
Actually, here's the thing..
If we let them go and it's not from God, it's gonna fail..
If we let them go and it's from God and we persecute them,.
then that is on us..
And I wanna jump right to the end of this part.
of the story, and I wanna read you what happens.
right at the end..
This is starting in verse 40..
His speech, that's Gamaliel's, persuaded them..
They called the apostle in and they had them flogged..
Then ordered them not to speak of the name of Jesus.
and let them go..
The apostles left the Sanhedrin rejoicing.
that they had been counted worthy.
of suffering disgrace for the name..
They brought the apostles in..
They told them no more preaching in the name of Jesus..
And then they flogged them, they beat them..
This is not some light little punishment..
I mean, they went away with probably broken ribs,.
bruised faces, closed eyes..
They were beaten up severely..
There would have been weeks of need for recovery.
from the pain of what they had just gone through..
And they leave there rejoicing.
that they had been counted worthy of suffering.
in the name of Jesus..
From hiding and fleeing persecution,.
for rejoicing in the name of God.
and the suffering that that brings,.
what change has happened for these men?.
And I love this idea that they were able to rejoice..
They weren't happy about their suffering..
They're not finding joy in the act of suffering itself,.
but they're able to find rejoicing.
because actually the suffering was proof to them.
that they were living out the plans of Jesus..
So they realized that if we're gonna follow Jesus,.

$^{481}$it doesn't mean just following him into the great moments,.
the miracles, the walking on water,.
the feeding of the 5,000..
Yeah, Jesus did those things and he conquered that..
But Jesus also went to the cross..
Jesus also was ostracized..
Jesus was ridiculed and beaten and abused.
and eventually killed on the cross..
And so if we're gonna follow Jesus,.
we want the miracles, yes,.
but we also need to realize that in following Jesus.
and living out our convictions,.
it might also bring suffering to us..
Why?.
Because actually to follow Jesus means.
we will encounter both..
Oh, we will rejoice that we're not just.
receiving the miracles,.
but we have a strength of conviction in us.
that we can find solidarity in the sufferings of Jesus,.
that actually maybe it's in the suffering.
that I might become to know him most..
Over these 21 days of our fast.
and over these devotions we've been reading,.
I don't know about you,.
but I've been blown away by the stories.
of the men and women from the vine.
who have suffered real trauma..
We've heard stories of those who have lost loved ones,.
stories of those that have had great conviction.
in their lives, stories of those.
that were convicted of sin,.
stories of those who lost their jobs or everything,.
everything you can imagine has been in the last 21 days..
And I've been so encouraged that in every single story,.
I see glimmers of the awakening of a personal conviction..
I see a growth in the people who are able to then testify.
that God has done something in them.
that would not have happened.
unless that trauma had been there in the first place..

$^{521}$Today's the last day.
and my mom wrote the devotional today..
If you haven't read it today,.
your senior pastor, the son of the person who wrote it,.
is telling you to go read it..
And my mom would be super embarrassed about this.
and she wouldn't want me to admit this,.
but I'm actually gonna quote her in one of my sermons..
And my mom wrote this amazing devotional today.
about losing her husband two years ago,.
which of course is my father who died two years ago..
I wanna read you something that my mom wrote..
She said this, "So often I know God has carried me.
"in times of trouble, grief, and pain and sorrow,.
"even though I thought I was walking alone..
"God is where you can be safe when life is hard.
"and he never leaves those who are looking for his help.".
Isn't that beautiful?.
Such encouraging words from my mom.
to say that God is right there.
in the midst of whatever it is.
that we might be challenged with..
Those are some of the final words.
of the 21 days of devotions..
But let me read to you some words that came to us.
at the very first day of devotions..
And these are written by Joshua Wong..
Let me read these to us..
"God's intention is to leverage all of our experiences,.
"especially the painful ones,.
"so that our growth and progress.
"are beyond anything that we can imagine..
"That God would take even the worst things.
"and in those forge in us something.
"out of those painful experiences.
"that are the very thing that we would need next..
"Here's what the apostles did..
"They allowed their past trauma.
"and the subsequent healing of the Holy Spirit.
"to prophetically forge in them a personal strength.

$^{561}$"exactly for the moment of launching the first church.
"in the first century..
"God knew what he needed the apostles to do ahead of them..
"And he understood better than anyone.
"the persecution that was about to come..
"So God takes them through the fire of their trauma.
"to prophetically create in them a resolve strong enough.
"to stand up to the Sanhedrin.
"and say, God first.".
And I wanna say this over us as a church..
And this is really important.
and actually quite strong for us..
I believe that the last year and a half here in Hong Kong,.
with everything we've seen politically.
and everything we've gone through with this pandemic,.
all of it was God's fire.
in which he is forging prophetically in us.
a strength of character and a conviction of the faith.
so that when the time comes in the future,.
the church of Hong Kong can stand up.
and be counted for the gospel..
So that the church in our city can stand up.
against whatever persecution might come.
and not be ashamed of the gospel,.
that we might be able to stand in front.
of whatever Sanhedrins might be in front of us.
in the future and boldly proclaim the gospel.
regardless of the cost..
I think this is what God's doing in us, guys..
I think he knows what's ahead..
I'm not trying to preach doom and gloom on us..
I'm not trying to say all this bad stuff is gonna happen..
What I am saying is this, will we be found faithful?.
And I think the answer to that question.
comes from this trauma we're experiencing right now..
Will we allow it to forge in us.
something that wasn't there before?.
Are we actually willing, like the disciples.
and like people we've seen in our city recently,.
to go to prison because of the conviction of the gospel?.

$^{601}$Are we willing to stand for the gospel.
no matter what might actually come our way?.
Not because we glory in the suffering,.
not because we want to be victims.
under the wheels of injustice,.
but because we have a conviction in our hearts.
that we will obey God first and foremost,.
that we will not be ashamed of the gospel.
no matter what might happen to us personally..
Stephen refused to be silent in front of his critics.
and he was stoned to death..
Peter refused to be silent in front of the people.
that wanted to pull him down.
and he was in prison many times..
Paul and Silas refused to be silent.
with their conviction of the faith.
and they were in prison and eventually martyred..
And I wonder whether the vine will add its name.
to that list..
I wonder whether we as Christians here in Hong Kong.
would be willing to add our name to that list..
That's not a small thing to say..
But will we, like the apostles,.
be able to stand before whatever is in front of us.
and say that I consider it great joy.
that I've been counted worthy of suffering.
for the name of Jesus..
Peter, at the end of his life, he wrote this..
Writing to a church that was struggling and suffering,.
and Peter just a few moments before his own death says this..
This is 1 Peter 4, starting in verse 12..
Dear friends, do not be surprised at the painful trial.
that you are suffering,.
as though something strange were happening to you,.
but rejoice that you participate.
in the sufferings of Christ.
so that you may be overjoyed when his glory is revealed..
If you suffer as a Christian, do not be ashamed,.
but praise God that you bear that name..
Church, we stand together.

$^{641}$in an important time of our city's history..
We stand together, I think,.
in an important time of the vine's history..
And I believe everything that we're going through right now,.
if we allow the Holy Spirit to speak to us,.
if we allow the Holy Spirit to come and have its way in us.
like it did in Pentecost,.
there will be forged in us exactly what it is that we need.
for whatever might be ahead..
I believe that that personal strength.
is being raised up in the church in our city right now..
And I believe that that personal strength.
is being raised up in your spirit.
in this moment right now..
I wanna stand with you, I wanna pray for you,.
and I wanna believe that together.
we can be those on the roof, a community of faith,.
standing with the ones who are persecuted and oppressed.
and are pulled down, and Christ would look up and see us.
and say, "Because of their faith,.
I will do what only I can do.".
Let's pray, church..
Father, we stand with you now..
And Father, we wanna have such a conviction of faith, Lord..
Father, we're humbled by Peter's story..
A man who fled, hid, denied, was ashamed,.
and yet restored and renewed by you,.
filled with the power of the Holy Spirit,.
and then with such conviction.
could stand before the powers of his day.
and say, "I'm gonna obey God, no matter the cost.".
Lord, it was that faith and that personal strength.
and that conviction of spirit that formed a church..
And Lord, we, the church today,.
we've so often come to define the blessings of God.
as the absence of pain,.
that we've come to define and think.
that God's blessings are health and having a good family.
and making sure I'm safe and comfortable.
when the first church defined blessing.

$^{681}$as suffering for the name of Jesus..
Father, I pray that, Father, in our hour, in our moment,.
you would fortify your church, Lord..
I pray that you would do prophetically in us right now.
what is needed for the years to come..
I believe that you're returning for a spotless bride..
And I believe your spirit, like a Pentecost,.
is at work right now in powerful, beautiful, amazing ways.
that you are able to do what we cannot do..
And that ultimately, we know that our own personal strength.
will never be enough..
And so we say, come Holy Spirit,.
would you fall on us as a church like you did that day.
for the apostles and the disciples?.
Would you come and fill us with such power and conviction.
and faith and hope.
that we would never be ashamed of the gospel?.
Father, would you release us, Lord, to love our neighbor,.
to stand for you,.
and to be your church on its knees in prayer?.
Move, Holy Spirit, move..
Come, Lord Jesus, now, as we worship and respond..
Speak to us..
(gentle music).
\newpage



\section{}
\label{sec:PMXFFRS8kEs}
\textbf{2021-02-08 Church Everywhere Live: Post Traumatic Growth -  Spiritual Change [PMXFFRS8kEs].mp3}
\newline
\newline
連結: \href{https://youtube.com/watch?v=PMXFFRS8kEs}{\texttt{ https://youtube.com/watch?v=PMXFFRS8kEs}} ~~~~ 語音日期: 2021-02-08 
\newline
\newline
\hyperref[sec:bs6i66xQaTI]{\small{< < < PREV SERMON < < <}}
~
\hyperref[sec:index]{\small{[返主目錄]}}
~
\hyperref[sec:lzQ5tfRY0rI]{\small{> > > NEXT SERMON > > >}}
\newline
\newline
$^{1}$- Wow, so good to be with you today..
And I just want to welcome you whenever,.
wherever you're watching..
We just would like you to know that we love you..
And we're praying for you every single day.
during this pandemic..
And if there's anything we can do to help,.
please reach out to us. Or if you're in Hong Kong,.
why don't we just meet up for a chat over coffee or tea?.
I love that..
Now I know that many of you are going through stressful,.
even traumatic times..
If not you personally, then someone you know..
In the 21 day devotional series on post-traumatic growth.
that we finished last weekend,.
we followed stories of trauma and growth.
that have come from those of us at the Vine.
who've experienced this themselves.
and who've courageously shared their stories..
These stories have really challenged me.
and helped me to grow.
because I had the honor of collecting them..
Along with the sermons in the last month,.
these stories have helped all of us look at the pandemic,.
at trauma in new ways..
Now, if you haven't had time to read all of these,.
please go to our website.
and finish reading the ones you missed..
Now today, this is the final in our seven week series.
on post-traumatic growth,.
how God uses the most stressful and painful times.
in our lives to draw us closer to him and to each other..
A quick review, week one, wounds and scars..
The first week, Andrew set the stage for this series.
with Thomas meeting Jesus.
after witnessing the incredible trauma of the crucifixion..
Jesus shows traumatized Thomas and us.
how deep and painful wounds can become scars.
of healing and growth..
Week two, trauma can help us see new possibilities..

$^{41}$After Peter's massive meltdown at the crucifixion,.
Jesus helped Peter over breakfast on the beach.
to find forgiveness, to have his career transformed..
We too, with Jesus, we can move forward.
after massive setbacks and failures.
to have our lives and careers transformed..
And trauma can change our relationships with others..
Remember week three, a few years back,.
Andrew was having panic attacks, was ready to resign..
He shared the story of the woman who was healed.
after 12 long years of trauma.
when she had the courage to reach out.
and touch the hem of Jesus' cloak.
and to step out of the crowd and tell her story..
The same can happen to us when we reach out to Jesus.
and acknowledge our suffering and trauma.
to those closest to us..
Two weeks ago, Ellison helped us to see.
how with Jesus at our side,.
even demons and monsters can be defeated..
We can begin to experience joy,.
maybe we can even laugh in the midst of this pandemic.
that never seems to go away..
And we can share the joy of the healing and growth.
Jesus brings with the people around us,.
the people who see little reason for hope..
Last Sunday, Andrew tried to convince us.
that we can grow in our personal strength.
through even the worst kind of trauma,.
persecution, torture, even imprisonment,.
and in other kinds of trauma as well..
And through God's spirit in us,.
we can gain personal strength.
that not even death or the gates of hell can destroy..
Wow..
Did Andrew convince you?.
Now, if you missed any of these messages,.
you can watch them on YouTube, Facebook,.
or you can listen to them on our website in a podcast..
Now, that brings us today to the seventh and final.

$^{81}$in the Post-Traumatic Growth Series,.
how trauma forces us to ask the most fundamental questions.
about the meaning of life and faith.
and often leads to significant,.
sometimes even dramatic spiritual change..
We're looking today at trauma and spiritual change.
with a familiar character, Saul or Paul,.
depending whether you want the Hebrew.
or Greek version of his name..
Paul was sure Jesus was not the Messiah..
That was clear..
God's chosen person to fulfill the vision of the prophets,.
to fulfill God's promises, no..
Jesus could not be the Messiah..
He did not show the proper respect for the law of Moses..
He didn't even begin to break the Jewish people free.
from Roman enslavement and humiliation..
In fact, the hated Roman colonizers.
supervised the crucifixion of Jesus,.
along with a couple of thieves, thugs, no, no..
Jesus could not be the long-awaited Messiah..
And believe me, Paul would know..
This is how he described his credentials some years later..
The book of Philippians, he said,.
"I was circumcised when I was eight days old..
"I'm a pure-blooded citizen of Israel..
"I'm a member of the tribe of Benjamin,.
"a real Hebrew, if there ever was one..
"I was a member of the Pharisees,.
"who demand the strictest obedience to the Jewish law..
"I was so zealous that I harshly persecuted the church..
"And as for righteousness, hey,.
"I obeyed the law without fault.".
Wow..
So Paul, along with other opponents of Jesus.
who helped to get him killed, was on the defensive..
"Crucifixion should have ended it.
"because Jesus was clearly a heretic who deserved to die..
"Jesus was on the wrong side of Jewish law.
"and God needed Paul as a defender of the faith.".

$^{121}$But how can people hatch the ridiculous conspiracy theory.
that Jesus has come back to life since the crucifixion?.
That's ridiculous..
So why are thousands of people becoming followers of Jesus.
after he is very clearly dead?.
Now, put yourself in Paul's shoes..
Maybe he's feeling a little bit of jealousy.
at Jesus' continuing popularity..
I'm wondering if Paul was experiencing stress, anxiety,.
in spite of his rock-hard conviction,.
Paul could be increasingly torn up inside..
He cannot ignore all the talk.
about the fantastic miracles Jesus did..
And just recently, Paul watched as Stephen,.
a young disciple of Jesus, was stoned to death..
It was bloody mob violence.
and Paul was right there applauding..
Paul knew it was the right thing to do..
It was what the law of Moses required..
But it was a gut-wrenching, God-awful thing to watch..
A man dying under a hail of stones..
And yet Stephen was peaceful as he died..
He claimed to have a vision of Jesus in heaven.
being in a place of highest honor..
And Stephen had a look of joy on his face.
as he said his last forgiving words..
"Lord, do not hold this sin against them.".
Paul is haunted by those last words..
Forgiveness, God, don't blame these men.
for what they're doing..
Surely Stephen was the one to blame, not him..
Paul was doing the will of God..
And Jesus in a place of highest honor in heaven?.
Unthinkable..
But Stephen seemed so confident.
of what he was seeing as he died..
And how could he be so forgiving?.
What if Stephen was right?.
No way..
For Paul, this was an impossible thought..

$^{161}$Now, in spite of the reality,.
the Jesus movement is spreading like wildfire.
since Stephen's death..
And Paul now is headed from Jerusalem to Damascus.
to arrest and imprison Jesus followers.
who escaped Jerusalem and are now sharing.
this poisonous message about Jesus..
This must be stopped..
Let's pick up the story in Acts chapter nine..
Beginning at the beginning of the chapter..
Meanwhile, Saul was still breathing out murderous threats.
against the Lord's disciples..
He went to the high priest and asked him for letters.
to the synagogues in Damascus.
so that if he found any there who belonged to the way,.
whether men or women,.
he might take them as prisoners to Jerusalem..
As Paul neared Damascus on his journey,.
suddenly a light from heaven flashed around him..
He fell to the ground..
He heard a voice say to him, "Saul, Saul,.
"why do you persecute me?.
"Who are you, Lord?".
Saul asked, "I am Jesus whom you are persecuting,".
he replied..
"Now get up and go into the city.
"and you will be told what you must do.".
So Saul got up from the ground,.
but when he opened his eyes, he could see nothing..
So they led him by the hand into Damascus.
for three days..
He was blind..
He did not eat or drink anything..
Paul can see nothing,.
but he is beginning to see everything..
His mind is filled with the vision of what he saw,.
and it was the same vision Stephen saw..
Jesus is now in a place of highest honor.
with God in heaven..
That glorious vision blinded Saul physically,.

$^{201}$but began to open his spiritual eyes..
"You are persecuting me, Paul.".
Paul is stunned..
He has been doing God's will,.
but Jesus is telling him that he, Paul,.
is totally, absolutely wrong..
Everything Paul believes about Messiah is wrong..
Paul's passion to purify the people of God.
is causing him to go against God..
In an instant, it dawns on Paul,.
"I am torturing God's very own son.".
Last Sunday, Andrew talked about persecution..
When people attack us because of our faith in Jesus,.
he takes that real personally..
They're not just attacking us,.
they are attacking Jesus,.
and he's right there with us in our discomfort and pain..
Jesus died for you and me,.
and he continues to suffer with us.
even in this pandemic..
Well, back to our story..
What Stephen saw as he was dying,.
Paul saw in a flash of light,.
he saw Jesus, and Jesus spoke his name..
Jesus invited Paul in the worst trauma of his life.
to rethink his core beliefs.
and his prejudices about Jesus..
Paul now realizes Stephen didn't deserve to die..
Paul had helped to execute an innocent man..
What an awful thing..
Peter and John and the other leaders that Paul hated,.
they are right, he is wrong..
The people in Damascus he is ready to arrest..
They are innocent, and he, Paul, is guilty..
Paul should be arrested,.
not the Jesus followers in Damascus..
This is like finding out all at once.
that your politics is wrong,.
your theology is screwed up and needs a total makeover,.
and that your current career is like a total disaster..

$^{241}$You're not just a failure..
You are abusing people.
and dishonoring the God you serve..
Well, let's pick up the story in Damascus..
Paul is blind..
He's very hungry and very thirsty..
Three days..
God has asked a man named Ananias to go see Paul..
Acts 9, 13, I'm reading from the message..
Ananias protested, naturally,.
"Master, you can't be serious..
"Everybody's talking about this man.
"and the terrible things he's been doing..
"His reign of terror against your people in Jerusalem,.
"and now he's shown up here with papers.
"from the chief priest that give him a license.
"to do the very same thing to us in Damascus.".
But the master said, "Don't argue, Ananias, go..
"I have picked Paul as my personal representative.
"to Gentiles and kings and Jews,.
"and now I'm about to show him what he's in for,.
"the hard suffering that goes with this job.".
So Ananias went, found the house,.
placed his hands on blind Saul and said,.
"Brother Saul, the master sent me,.
"the same Jesus you saw on your way here..
"He sent me so you could see again.
"and be filled with the Holy Spirit.".
No sooner were the words out of his mouth.
than something like scales fell from Saul's eyes..
He could see again..
He got to his feet..
He was baptized and sat down with them for a hearty meal..
Well, perhaps you know the rest of the story..
Paul, the zealous persecutor of Jesus' followers,.
became a passionate Jesus promoter and church planter..
His trauma led to an incredible transformation.
of Paul's life and career..
You know trauma does this to people today..
Some years back, before I met my wife, Chris,.

$^{281}$before she was a Christian,.
her mom's family was traumatized.
by a tragic bus accident right here in Hong Kong..
Her mom's coat caught in the door of the bus.
as she was getting off,.
and she was dragged under the bus..
The bus ran over her leg..
Doctors weren't sure Chris's mom would live,.
and if she did, she would certainly lose her leg..
Traumatized by the likelihood.
that her mom would never perhaps walk again,.
Chris cried out to Jesus,.
"I don't believe in you, Jesus,.
"but if you heal my mother, I'll follow you.".
After two years in the hospital,.
11 surgeries and skin transplants,.
Chris's mom walked out of the hospital,.
and Chris became a follower of Jesus..
I'm hugely glad that Chris experienced.
spiritual change through trauma,.
because otherwise, she wouldn't be my wife..
We won't discuss the occasional trauma.
she still experiences being married to me..
And you've been reading the stories.
in our January devotionals.
about how many of us Vine people.
have experienced deep spiritual change through trauma..
Whether we're Christians or not,.
this pandemic has caused many of us.
to ask big questions..
Where is God in all this turmoil.
and confusion and suffering?.
Why doesn't a God of love step in.
to stop the pandemic?.
Our core beliefs have been challenged,.
sometimes shaken..
With church buildings closed,.
many Christians are asking fundamental questions..
What is the church?.
How does the church survive and thrive.

$^{321}$when we can't gather in a building?.
How will we do church differently after this pandemic?.
We may be asking even deeper questions.
about our relationships, our marriage,.
the family, about how we do our work,.
about schools and education,.
about how we run our businesses..
How can we manage our careers.
with everything so messed up?.
Some of us are probing even deeper,.
why am I even here?.
Why do I get up in the morning?.
Every day seems the same..
Why do I do what I do?.
What's it all about?.
After Paul met Jesus,.
he had to spend three years in the desert.
figuring out what this meant for him personally,.
for his understanding of the Hebrew Bible,.
what it meant for his life and career..
But Jesus was with Paul in that journey.
every step of the way.
through the presence and power of the Spirit..
Paul was not on his own..
Throughout his teaching and writing ministry,.
the presence and work of God's Spirit.
was always front and center,.
I should say, deep within Paul..
Paul reflects back on this experience.
as some years later in one of his letters..
This is Galatians chapter one,.
starting with verse 13..
I'm reading again from the message..
I'm sure that you've heard the story of my earlier life.
when I lived in the Jewish way..
In those days, I went all out persecuting God's church..
I was systematically destroying it..
I was so enthusiastic about the traditions of my ancestors.
that I advanced head and shoulders.
above my peers in my career..

$^{361}$Even then, God had designs on me..
Why, when I was still in my mother's womb,.
he chose and called me out of sheer generosity.
and now God has intervened and revealed his Son to me.
so that I might joyfully tell none Jews about him..
Paul's post-traumatic growth.
helped him understand God's purpose for his birth,.
his life, in a new way..
He was reading the same Hebrew Bible.
but seeing it in a totally different light..
He developed a new narrative,.
a new way of understanding his past,.
a new way of describing his way of life and career,.
a new way of teaching the scriptures.
he had learned to love when he was growing up as a boy..
Now, Paul saw everything in the light of Jesus,.
the Messiah, the Christ, who is now everything to Paul..
And for Paul, there's an amazing future awaiting him..
He says, "For me to live is Christ and to die,".
why, that's even better..
I believe God wants to bring deep spiritual change..
Spiritual change to you and to me,.
to us as a church during this pandemic..
It won't be the same as it was for Paul..
It's different for every person.
but God wants us to meet Jesus in a new way,.
to be filled with his spirit.
so that as we examine our core beliefs and values and habits,.
we will reconnect with God's purpose for our lives,.
for our families, for our careers,.
God's purpose for our educational systems and businesses,.
God's purpose for our city, for our nation..
If that happens, like Paul,.
we will never be the same again.
and we will bring change to many others around us..
We will bring change to our city and to our world..
Now, this won't happen in an instant..
It took Paul at least three years, indeed,.
the rest of his life to work out the full meaning.
of this profound spiritual change..

$^{401}$But this pandemic is the place for us to begin..
Everything has been turned upside down..
This is our Damascus road.
and there are three simple but challenging steps.
we need to take..
First of all, we need to meet Jesus..
It begins for us, as it did for Paul,.
with a new encounter with Jesus and his spirit..
To fall in love with Jesus,.
invite him to change you deeply from the inside out.
and to keep on doing it..
Our deepest pleasure begins and ends with Jesus..
Secondly, let's walk with Jesus to make our world better..
Invite Jesus to walk with you,.
to make you a force for good in your relationships,.
in your work, in your leisure time activities..
For me, recently, I've been singing the words.
of an African-American spiritual.
as I walk the streets in Hong Kong..
It was written by American slaves.
who may have suffered deeper and longer trauma.
than any group of people in human history..
I want Jesus to walk with me, the spiritual goes..
Oh, I want Jesus to walk with me.
all along my pilgrim journey..
I want you, Jesus, to walk with me..
Three, share the gift God's spirit has placed in you..
You are a gifted child of God..
You are endowed with a special talent.
and passion from God's spirit living in you..
There's no one like you..
Discover and use that gift from the spirit.
in new ways every day, every year..
Your greatest joy will come when your deepest passion.
connects with the greatest hunger of people around you..
You will truly feel God's pleasure..
Now, friends, don't do this by yourself..
Do it with family members, with friends,.
with your community group..
If you want a mentor, a spiritual friend to walk with you,.

$^{441}$let us know..
We're here to help..
Now, this brings us to the end of this series.
on growing through stress and trauma,.
but it's just the beginning of a new.
and challenging chapter in your life..
Moving forward, we're going to keep on growing.
through this pandemic..
As Paul discovered with Jesus, the best is yet to come..
Not always the easiest, but the best..
Well, I would like to invite you, if you would,.
to pray this prayer with me..
Jesus, I'm finding this pandemic really tough..
There's so much fear and confusion,.
so much conflict and suffering..
I'm asking hard questions about you, Jesus,.
about me, about my family, my career, about this city..
I want to meet you, Jesus, in a fresh way..
I want to discover again why you placed me here..
I want to walk with you to bring positive change.
in my world, and I want to share by your spirit.
living in me..
I want to share the gift and passion..
You placed in me to bring joy and blessing.
to my family, my friends, my colleagues..
I want to feel your pleasure..
Help me find at least one person to guide and support me.
on this exciting journey..
I pray this, Jesus, in your name, amen..
\newpage



\section{}
\label{sec:lzQ5tfRY0rI}
\textbf{2021-02-14 Church Everywhere Live: Reunion \ and  Reconnect [lzQ5tfRY0rI].mp3}
\newline
\newline
連結: \href{https://youtube.com/watch?v=lzQ5tfRY0rI}{\texttt{ https://youtube.com/watch?v=lzQ5tfRY0rI}} ~~~~ 語音日期: 2021-02-14 
\newline
\newline
\hyperref[sec:PMXFFRS8kEs]{\small{< < < PREV SERMON < < <}}
~
\hyperref[sec:index]{\small{[返主目錄]}}
~
\hyperref[sec:TWnjFhuOiys]{\small{> > > NEXT SERMON > > >}}
\newline
\newline
$^{1}$you know, during Chinese New Year, we love to go to different places, different homes to visit each other and say our blessings..
So today, we're going to do something special for our message..
We're going to visit Pastor Vivian and Elder Sidney at their house as they're delivering today's message..
[Music].
Hey dear, it's Chinese New Year. Why don't we make some Tong Yuen as we often do to celebrate the Chinese New Year?.
Yes, sure. The Tong Yuen is often made sweet in your family, whereas in my family, it is made salty with a lot of other ingredients..
Yes, true. It could represent that our family longs for something sweet in life..
Whereas for your family, you look for a rich and abundant life..
Well, the Chinese Tong Yuen are usually rambish regardless of various traditions. They may signify wholeness, perfection and many others..
Wow, that is prophetic. I just read from the Message Bible the other day..
The famous verse in Philippians 4 actually tells us that let's petitions and praises shape our worries into prayers, letting God know our concerns..
Before we know it, a sense of God's wholeness, everything coming together for good, will come and settle us down..
[Music].
Well, just now we are making the dumpling rambish. I think it actually releases a desire to see that there is a restoration to wholeness, reunion and reconciliation into the whole matter..
Well, actually, as a surgeon who stitches up a bounced wound and brings along healing, I guess I have my own perspective of Yuen..
As in recovery, restoration in Fu Yuan..
Every time I perform surgery, we're actually inflicting some degree of hurt, which we call necessary pain..
But it is warranted because we can achieve the healing that is desirable..
For instance, during the Chinese New Year, I actually did a little bit of artwork..
We cut into the buds of this water lily and in time, when nourished in water and nutrients, given the right season, then it will blossom for the new year..
In that sense, even for a very traumatic year for the society, the new year also spells a new season..
It will take time, but together with the right measures, we shall recover or even blossom..
That crucial time just can't be rushed..
But then pain is a gift as it points to practical reality, help us recognize where we stand and how far we may stretch ourselves literally when we recover from some of the lost functional abilities..
I love the book, The Gift of Pain, written by an orthopedic surgeon, Dr. Paul Grant, who spent decades working for leprosy patients in India..
Through his dedicated work, he observed that the patients lost their fingers, hand or foot due to neglect to repeated hurts of their body just because they lost their sensation of pain..
That is, pain actually alerts us to that we are being hurt. Something sinister needs to be tackled..
On the other hand, as the proper treatment has been conducted, progressive decrease in pain reassures us of the ongoing healing in good progress..
Well, as a mom, I would say most mothers would agree giving birth to a child is one of the worst pain experiences..
However, there is much to celebrate for new life and the natural longing for family to come together to celebrate..
Well, on the other hand, even as a physician, I recognize that there is some spiritual dimension of pain..
Didn't Paul plead for his thorn in the eye to be removed?.
But then Christ's grace is sufficient and His strength is made perfect in weakness..
He must have been painful for Paul. At the end, he learns to be humble before Christ as he experiences His sufficient grace..
Coming back to your Yuan metaphor, I do see Tang Yuan as more signifying wholeness, be it physical or emotional healing from wounds..
It's just part of the journey to wholeness..
The R for Rundish metaphor reminded me of what I learned in Wholeness and Inner Healing Prayer Ministry that its ultimate goal is reconciliation..
In particular, there will be a process of reconciliation to God, to others, and to oneself..
Let's have a look at these five R's in the process of reconciliation..
First, recognize our sinful reactions to hurts and the sinful patterns that affect self and others..

$^{41}$Second, confess and repent for ways we judged those tempted us to form judgments and expectancies, to believe the lies, and to make inner vows..
Third, release forgiveness to others and oneself..
Fourth, reckon as dead our sinful reactions and patterns on the cross..
Fifth, allow God to resurrect us, to bring new life..
It is also very important that while we long for bearing good fruits for the new season, we must get rid of all the bad roots..
So the physical closure of the year is only a timely reminder that we have to literally chop off the bad roots in our lives before we can be renewed to bear fruits different from before..
Only a proper closure sets us ready for something anew..
Yes, indeed, proper closure and then a new chapter..
Since we have the testimony from our own community member, whose relative had a very traumatic experience of going through a very difficult medical condition..
And yet no definite plan can be formulated because of the complexities in making the diagnosis..
With the COVID restrictions in the hospital throughout those few months of agony, he was nursed in isolated facilities and no visitation was allowed..
All the discussion of the management of the patients was just with the relatives over the phone..
Yet the risk of surgery was so high that no firm decision could be made, despite a long stretch and every hope seemed to be fading..
As a surgeon myself, logically speaking, I understand that it is almost a daunting gamble to change gear, to switch facilities and to proceed to surgery..
Yet the authentic faith of the family member, entirely entrusting to God, emboldened the decision..
Against all odds, he made it through and almost miraculously..
As critical as the matter of life and death, the family members cried out to God desperately and sought prayer support and medical advice from friends and the spiritual community..
The patient himself, having gone through every miraculous piece, confessed Christ as his personal Savior..
Again, his physical healing is only part of his healing journey..
There is much more than this, other than his reconciliation to God, there is reconciliation between himself and his family..
I am very convinced that the family being together to lend support to him makes a tremendous difference,.
so that he gathered the courage to go through the painful surgery and challenging recovery..
In a year which is as difficult as this immediate past year, I would say community, even in the form of household or spiritual communities,.
makes tremendous impact on how we cope with such uncharted and phantom difficulties..
Instead of looking into our own interests or finding comfort in our sphere of supposed safety,.
I would say we should make every outreach to others so that no one suffers in isolation..
Yeah, but then it is almost ironic that while Tuen Yuen, Tong Yuen, means reunion, Tuen Yuen,.
but then we can't have that reunion for our very own family..
Oh my dear..
I understand that as a family, we miss our son..
He is in lockdown mode, now going through his third wave in Melbourne..
Yet we have to entrust God that he will look after him and provide for him..
He is alone, but he is not lonely..
He is being watched by him, and in days to come, we shall reunite..
For now, maybe we should just share the pain and distress of others who are isolated from their family and loved ones,.
thousands of miles away, and many others who are stranded here in Hong Kong while this place may not be their home..
(Bells ringing).
Hey Ivan!.
Happy New Year!.
Thank you!.

$^{81}$Come in!.
Thank you!.
Please have a seat, Ivan!.
Okay!.
Happy New Year!.
I am very happy to be able to celebrate the New Year with you today..
I know that you have a family tradition,.
which is to pour tea for you from the back of the glass as a sign of respect..
Thank you!.
I know that your son, Christopher, your beloved son, is not in Hong Kong now..
I hope that today, we can take this opportunity to send him our best wishes from the future generations..
Thank you!.
May God continue to bless and protect you..
May God also bless and protect your family members, so that they are not restricted by the geographical boundaries..
Let's have some tea together!.
Okay!.
Let's have some tea!.
It's time to make a wish!.
Oh, yes!.
Me and Vivian wish you all the best!.
Thank you! I wish you all good health!.
I hope that I can give you a New Year hug..
Oh, that's great!.
Happy New Year!.
Thank you!.
Ivan, thank you for your blessing..
You are our beloved descendant..
In the past seven years, we have been very happy to know you..
We are even happier to witness your relationship with God..
We are even happier to witness your life's transformation..
We are proud of you for facing many difficulties and suffering..
Vivian, you have shared many special experiences..
Why don't you share your personal stories with me?.
It was indeed a difficult experience..
I remember that when I was still a firefighter more than ten years ago,.
I had a serious accident..
I had multiple fractures on my face..
I had to undergo many surgeries..
I had to undergo surgery for four years before I could recover..
Just like what Sydney said,.

$^{121}$I had to experience physical trauma and how to get treatment..
I agree that physical pain can really help us to know.
where our body is in a bad condition,.
where we need to get treatment,.
and where we can recover..
But physical trauma can take more time,.
mental effort to heal and recover..
This time, I had a post-traumatic stress disorder..
I was not in a good condition..
I had a hangover, a bad temper, and even had a relationship with my ex-girlfriend..
The person who was under the greatest pressure was usually the one who loved me the most..
Her heart was in more pain than mine..
Would you mind sharing with us who you are hurting?.
I am talking about my mother..
She even had a post-traumatic stress disorder..
I love my mother,.
but the trauma of this relationship made me unconsciously withdraw from the close relationship between us..
I realized that the more nervous I was, the more I would withdraw..
I couldn't even call her "mommy"..
Would you mind sharing with us your father?.
Did he help you in this situation?.
My father was another challenge for me..
He was a mental gambler..
I remember when I was in my teens,.
he moved to another house more than 10 times because of financial problems..
Since I was a kid, my father gambled and he brought a lot of damage to my family..
I swore not to gamble like my father..
But it was ironic that I gambled more when I was a gold minister..
I was so happy every day, but I ended up in debt..
The only difference is that I didn't use thousands of lies to deceive my family..
I lied to my family..
Wow, I heard you had a very painful experience..
But what we saw today was a very different Ivan..
What was the motivation and faith that led you to change so much?.
You just shared that pain is a gift..
The pain in my body and in my relationship made my soul extremely empty..
But the pain brought a precious gift that changed my soul, which is the Christian faith..
There is a scripture in the Bible about reconciliation..
It is the 5th chapter of the Gospels, chapter 18-20..
"All things are in the hand of God, who has made it possible for us to reconcile with Him,.

$^{161}$and has given us the discipline of reconciliation..
It is God who has made it possible for the world to reconcile with Him,.
and has not condemned their transgressions,.
and has entrusted us with the reason for reconciliation..
For this reason, we are the messengers of Christ,.
and God has made us to preach to the world..
We also ask Christ to reconcile with Him.".
This scripture has a deep impact on me..
This is a spiritual gift that affects my life and my daily life..
If it were not for God's grace and the people who love me,.
I would not be able to overcome my pain and to pay off my debt..
But more importantly, I can reconcile with my parents..
In this process, I saw my life being transformed by Christ..
I can help my father to understand that.
he has to follow Christ and have a new life..
Of course, he still has his own struggles in life..
But I believe that for everyone,.
we all have our own struggles..
But for me, I have already forgiven him..
I can still imagine and count his sacrifice for the family,.
his concern for the family..
He has dedicated everything he can to the family..
Ivan, you are sharing a wonderful journey..
You accepted your father and forgave him..
You even embraced him and thanked him..
This is a wonderful change..
We are very happy for you..
I can love my father as my son..
At the same time, I would like to take this opportunity to thank you..
Thank you and Vivian..
You have built our faith community with the love of your parents..
You have taught us how to build a Christian-centered family..
I remember you held a mass in 2014 at the Divine Center..
This mass allowed us to express our love to our parents..
In the past few years, you have been training us on the faith..
You have shown us the true identity of a father..
You have also made us realize.
the role of a father on earth..
You have shown us the love of a father..
You have made us realize the love of a father..

$^{201}$Now, the love between my parents is more direct and natural..
We can celebrate the New Year together..
Ivan, thank you for sharing your experience of mental trauma..
I understand that the heart of Christ.
needs to heal our mental trauma..
You may swear to be a different person..
In the New Year, others may swear to be a different person..
Only God's power of transformation.
can make our lives better and change us..
Yes, only God's power of transformation.
can make us change..
This is what the Bible says in the 4th chapter of the Old Testament..
God's Father's Messenger.
helps the hearts of the fathers to turn to their children..
He also helps the hearts of the children to turn to their parents..
What we have heard from you.
is the work of God in your life and family..
In addition to Christ, you mentioned that your wife has always been with you..
Yes, I am very grateful..
My girlfriend, my beloved wife, Michelle,.
her love, care, and her devotion..
I would like to share my second life change,.
which is the commitment to our relationship..
My wife has accompanied me.
through the most difficult time of my life..
We have committed to dedicate ourselves wholeheartedly.
to each other as a lifelong partner..
We also hope to build a family that is faithful to Christ..
We also have a great blessing.
to become a loving and lively parent..
Family is not only about the family of a family of four,.
but also the family of the church..
It has a great impact on my life..
We need companions to accompany us,.
to care and encourage us..
We also need them to share our pain,.
struggle, and setbacks..
At the same time, I have found that.
when I am willing to try to care for my siblings.
who are also facing challenges and failures,.

$^{241}$we are actually encouraging and helping each other..
Although my father has his own limitations and challenges,.
he has always been teaching me..
At the same time, I have realized that.
I need other people's wisdom to help me..
Just like you, my famous predecessors,.
you have taught me in my life..
Thank you for witnessing.
the relationship between me and Richie,.
our marriage, and our family..
You have also helped us to understand.
the beauty of God and the meaning of God to us..
Actually, Ivan, we have witnessed.
the precious legacy of God..
Just like each of us is a descendant of God..
In the past year,.
we have faced many challenges in different families..
Our family is also a family..
Our children have come from far away..
But we have seen that.
God's grace, no matter where they are,.
God will protect, guide, and care for them..
I also believe that in the future,.
God's grace will also be with your family,.
especially your two adorable children..
Ivan, as your spiritual teacher,.
we can bless you and your family..
It is our blessing..
In fact, in the past few years,.
we have been able to share the blessings of the spiritual.
with many of the brothers and sisters in the Chinese community..
It is really our great blessing..
I like the scripture you just mentioned..
It says that you are in the right path..
Christ allows you to be in harmony with God..
And you believe that God has entrusted you with the right path..
God allows you to be in harmony with your father..
God is in the right path.
to prepare you to become a better father..
Finally, I want to say that.

$^{281}$only our loving God.
can make a good work of life..
Very good..
Only God's work can make a good work of life..
Sunny, as a father,.
we would like to hear your blessings.
and wishes for your family..
I would like to thank the scripture for blessing my two sons..
I hope to see their wisdom and kindness..
And I hope that God's love for them will continue to grow..
I also hope that my marriage with my wife.
will be more stable..
Having a Christian child.
is like the three-pieced ring of the ring..
It is not easy to break..
In my opinion,.
the scripture says that.
love and blessings can be passed on from generation to generation..
I hope that Christ will continue to transform me.
and transform me to become a better father..
The relationship between father and son.
or the blessing between generation to generation.
is not only about how to deal with the tension between father and son.
or how to guide them..
More importantly,.
God's blessing can be a family legacy..
It is not only about the blessing on material or health..
We can use our next generation, our children,.
to know that we have a father in heaven..
I hope that God will entrust him to me as their father.
to show how much God's father loves them..
I have prepared some dumplings for everyone to eat..
I will go to the kitchen now to prepare them..
Let's eat together, okay?.
Okay, very good..
Let's eat..
Okay..
(glasses clinking).
\newpage



\section{}
\label{sec:TWnjFhuOiys}
\textbf{2021-02-22 Church Everywhere Live: Flourish - Scripture as One Story [TWnjFhuOiys].mp3}
\newline
\newline
連結: \href{https://youtube.com/watch?v=TWnjFhuOiys}{\texttt{ https://youtube.com/watch?v=TWnjFhuOiys}} ~~~~ 語音日期: 2021-02-22 
\newline
\newline
\hyperref[sec:lzQ5tfRY0rI]{\small{< < < PREV SERMON < < <}}
~
\hyperref[sec:index]{\small{[返主目錄]}}
~
\hyperref[sec:ODvUnLlR0pQ]{\small{> > > NEXT SERMON > > >}}
\newline
\newline
$^{1}$So, if you've been a part of the Vine for any period of time, you'll hopefully know.
that our vision here at the church is a simple one..
It is to grow big people..
You know, whether you're with us for two weeks, two months, two years, two decades, we want.
you to grow in your relationship with Jesus..
And really for us, everything we do at the Vine centers around that passion..
And growing big people is really the idea of two things..
First, it's the primary idea of your own personal spiritual growth..
Your way of learning about yourself, bringing your needs to Jesus, having that restoration.
and that healing, the anointing upon you, growing and deepening your roots in your spiritual.
faith journey..
All of the things that you would expect about your own personal maturity in Him..
But growing big people is also about becoming a big person in the community and society.
that you're placed in..
It's not just about growing for yourself, but it's also then about believing how the.
gospel grows in your community, about being an individual who stands on behalf of justice,.
of peace, of reconciliation in the brokenness that you see around yourself and your neighborhoods..
Growing big people is our passion for you personally, but it's also the passion that.
we believe God has for our city..
And we want to see us come together to transform not just ourselves personally, but the nation.
in which we live..
This was the Apostle Paul's passion when he began the early church..
I love his letter to the church in Ephesus..
This was a church that had just been saved out of a Greco-Roman empire..
And Paul writes to them, passionate for them to understand what it is to grow in their.
faith, to learn that that faith is given to them by grace..
They haven't earned it..
And in that maturity of faith, that they can go into their cities and their societies and.
change and transform those places for Jesus..
In chapter four, Paul gets really focused with them around this idea of their maturity.
in Christ..
And he says, look, if we can all come together, all giving one another these acts of service,.
we can be built up in the unity of the faith and in the knowledge of the Son of God, so.
we could become mature, attaining to the whole measure of the fullness of Christ..
I love that idea..
The whole measure of the fullness of Christ..
For us at the Vine, that's the central passage that sits around everything we do..
We want you to know the whole measure of the fullness of Christ for you, for your family,.
and for this city in which God has planted us in..
And really all the sermon series that we do here at the church, everything we teach, everything.

$^{41}$we do in our communities, in our ministries, it's all designed to move that idea forward..
The development of our faith personally and the development and implementation of that.
faith in the world around us..
And this series that we're starting today is no exception..
Back at Vision Sunday, I brought to us this idea that each one of us is a single drop.
of water in the river that God is unfolding here in our city in Hong Kong..
That God has called us as the Vine to think about ourselves as this kind of flowing river..
And I said to us that if we're going to flourish as a church in the future, if we're going.
to grow and change, if we're going to impact our city, it's going to happen because we're.
going to find ourselves deepened in our spiritual disciplines..
That our faith, both as a community and individually, is rooted in something far deeper than just.
our experience of God together on a Sunday in those 90 minutes when we gather..
I actually said in that series that one of the things that is challenging to us in this.
time of COVID is the reality that we often realize that so much of the sum total of our.
faith is found and dedicated to those 90 minutes on a Sunday..
So much so that when that's removed from us, as it has been during the COVID time, it actually.
brings a great sort of disruption to our faith..
And many of us have struggled to maintain our faith, to deepen our faith in a time when.
we've not been able to gather regularly together..
And I said that Sunday, I challenged us with this question, is our faith too dependent.
on our Sunday gatherings?.
Now of course, there's nothing wrong with Sunday gatherings..
We love it..
This church everywhere experience is fantastic..
We can't wait to get back into this building, gathering in person together..
There's something important for our spiritual growth about gathering as a community, worshipping,.
listening to the word..
We're passionate about it..
But the reality is, if that becomes the sum total of your faith, if the wholeness of your.
faith is built on that, I don't think you're living the Christian walk that Paul was calling.
the church to in Ephesians 4..
I don't think you're actually going to find yourself strengthening and deepening your.
relationship with Jesus if you're relying fully on just that Sunday experience..
Instead, it has to be much deeper and broader than that..
Because here's the reality, in the hardest moments of life, somebody else's faith is.
not going to cut it for you..
Somebody else's passion for Jesus is not going to cut it for you..
Some preacher or pastor or worship band is not going to come through for you..
You're going to have to have your own deep roots, your own spiritual walk, your depth.
in prayer, your reading of Scripture, the spiritual disciplines that God has given us.

$^{81}$as a gift..
Those are the things that are going to anchor us for our spiritual growth, no matter what.
might happen in circumstances and situations around us..
And that's exactly what this new sermon series is about..
I said on Vision Sunday that in 2021, we're going to help you to enrich and deepen your.
spiritual foundations..
And that's why we're doing Flourish this week and over the next six..
And really, Flourish is exciting because I get to introduce you today to something that.
we've been working as and working on as Apostle Team throughout 2020 and something we can.
now release to you as a church..
It's called our Spiritual Formation Toolkit..
And this is an online resource that we've created to help you to journey deeper in your.
daily spiritual disciplines..
The Spiritual Formation Toolkit is designed purely for those who want to say, "You know.
I want to invest in my reality of my faith outside of just what we experienced together.
on those Sundays..
I want to have deep roots so that no matter what comes in my life, I can flourish.".
And so we've designed this tool as an online resource to both welcome you and introduce.
to you what those spiritual disciplines are..
For some of you, you might be new to the idea of having daily disciplines..
But also for those of us that maybe have been doing those disciplines for years, we've also.
designed this resource to help to take you deeper in your journey with those disciplines..
And we want you to consider the Spiritual Formation Toolkit really as like different.
ingredients that goes into forming and creating your faith, to strengthening the roots that.
you have with Jesus..
And each of those individual ingredients are important..
In the toolkit, we actually give you six of those ingredients..
So now I want to kind of show you how this works today..
Let me show you the first of those ingredients..
It's the ingredients of Scripture..
And Scripture is so important because Scripture provides for us the foundation of how we come.
to understand God's story and everything that is in there..
But it's not, of course, just about the Scriptures..
It's also about this other ingredient that I think is really important to bring alongside.
of it..
The ingredients of prayer and how prayer becomes so central to our relationship with God, both.
again personally, but also in community with one another..
And that raises another ingredient that I think we need to focus on if we're going to.
deepen our faith..
And that is to understand that we have community, that we're not supposed to be isolated, that.

$^{121}$monastery on a hill, but we're supposed to be embedded in doing our faith with others..
You know, and it's not just about community on our own, but it's also this idea of deepening.
our spiritual friendships, finding those one or two or three individuals that we can go.
deeper in our community with them on a personal basis..
And I think when we do that, the fifth of these important ingredients comes to mind.
for me too..
Yeah, the idea of work and faith, that our Christian experience is not just this private.
thing that we have, but it's actually something that is to be found in all areas of our lives,.
particularly in the places where we invest a lot of our time in our workplaces..
And all of this works towards the final ingredients, which I think is this one right here, the.
idea of evangelism..
The idea of taking our faith and our growth in Jesus and seeing it lived out through the.
hands and feet of our community in the neighborhoods around us and living out the gospel story.
so others can be welcomed into it..
So these six elements are the elements that we've captured in the Spiritual Formation.
Toolkit..
And what we're going to do over the next bunch of weeks is actually unpack these for you.
to kind of give you the idea of, "Hey, this is what it's like to be a Christian.".
You know, you're going to take little bits of elements of each of these areas and you.
bring them into your life, concentrating on the things that are helpful to you and where.
you are in this moment..
And then together, as you bring these together into your life, you actually create the holistic.
strength of what it means to be someone who grows in Jesus..
I think this is what Paul was talking about, being mature in the faith..
And so over these next six weeks, we're going to take one of these each week..
We're going to unpack it..
We're going to let you understand it so you get to know all about it..
And this week, I get the great privilege of telling you all about Scripture..
In fact, what I'm holding in my hands right now, for those of you who are millennials.
and below, this is actually called a physical Bible..
This is the real, actual thing that you've got on your phone..
I hold in my hands what I think is the most profound and most powerful story that is ever.
being told..
This book, I think, is the greatest piece of literature in all of antiquity..
There is no other book like this one..
This has been banned, actually, internationally from international bestseller lists because.
it would be number one every single year..
I mean, just consider some of the facts about the Bible..
This Bible has been put together by over 39 different authors..
39 different authors contribute to the story..

$^{161}$And some of those authors were some of the most famous and powerful people of their day..
And some of them were the ones that were on the margins, who were forgotten by society,.
the ones that were the poorest people of their day..
Inside here are 66 separate individual books, each book that contributes something to the.
overarching story..
And those books have been written over a 1500 year period, from when Genesis 1 and 2 was.
first captured down to when John the Apostle finishes Revelations 21 and 22, a 1500 year.
time span of writing down the words of our Scriptures..
And inside those 66 books are 10 different genres, prophecy and legal writings, narrative.
and parables, different forms of commuting what is on God's heart for us, and all of.
it so that it can tell one simple story..
The story of how God is passionate to restore everything that has been broken..
A story about a God who moves powerfully in this world to redeem and restore everything.
back to how He intended it to be..
The story of a God who has the power to remove sin and bring life again..
The story of a God who sentenced it in the beginning of the bringing of His Son, Jesus..
This book tells one beautiful story, and that story is the Father, the Son, and the Holy.
Spirit, and it's about how you can find true life..
Let me tell you a couple of things about reading the Bible..
The first thing you need to know is that the Bible is not a self-help book..
The Bible is not designed for you just to use it as some self-help book..
Now of course, we can go to our Scriptures with our hurts and our pains and our struggles.
and the things that we're facing, and we can open its pages and we can learn and grow from.
it, of course, but it's not actually designed for you to use it to dip in and out of to.
get that catchphrase for your day..
I want to also tell you this, you are not the center of its story..
The Bible is not predominantly about you..
The Bible is actually about God..
He is the center of the story..
It's manifested through the Father, the Son, and the Holy Spirit..
The Bible sings the story of the beauty of who God is and what God has done..
It is Jesus..
It is about Jesus..
His story found in history..
And the most compelling and captivating thing about that is that the more that you come.
to understand His story, the more you're able to find your story within it..
See here's the thing, I think we actually have a Bible reading problem in the global.
church right now..
See, I think we actually read the Bible in the wrong direction..
Let me explain this..

$^{201}$I think so often we actually read the Bible from the position and the direction of starting.
with ourselves..
So we start with the lens of our problems and our issues and the things that we're struggling.
with and the things that we're not sure how to handle in our lives..
And we read down towards the Scriptures, hoping and asking and needing the Scriptures to inform.
our own story..
So we're reading in the direction of ourselves to Scripture and we become the lens so that.
everything we read in Scripture is filtered through the lens of my hurt, my pain, my struggle,.
my needs..
Everything becomes about that..
I think if that's predominantly how you're reading Scripture, you're reading it in the.
wrong direction..
Instead, I think the Bible is designed for us to read from the direction of the story.
towards ourselves..
That we actually start with what God is saying about Himself in the Word..
That we actually find that in the pages of this book, we get this captivating idea that.
God has not given up on a rebellious humanity..
That God is willing to pursue that humanity from every position and every angle..
That God is willing to come in the Exodus and take a people who are enslaved and bring.
them into the Promised Land..
A God who is willing to forgive those people time and time again as they turn from Him.
and begin to worship other idols..
A God that brings punishment to them in that and yet has this grace to redeem them every.
step of the way..
A God who so loves everything in this world that He's willing to send His Son, Jesus..
That in the life, death and resurrection of Jesus, we come to understand our restoration.
as people..
And in that restoration, we get to understand why it is that we not only worship Him with.
everything we have, but we want to see our communities change and transform..
Why when we see injustice happening in our city, we stand up and say, "That is not right.".
Not because we're trying to be some social activist, but because we believe in the Gospel.
and what the Gospel says about how God is at work in this world..
When you start with that story, it brings your story alive..
Maybe we should be reading the Bible in that direction..
You see, for me, all 66 books in this Bible draw that story together..
Each one of them uniquely telling us something, giving us a picture or a window into the person.
of Jesus..
Each one of them important for us to know and understand so that we can both understand.
God and then understand ourselves..
I want to give you a little illustration of what I think reading the Bible as one story.

$^{241}$is all about..
I want you to imagine with me for a moment if I was going to perhaps record a band, a.
large band, and maybe that band has 66 different instruments in the band..
And if you were going to record a band like this, if you were like the sound engineer,.
you would have lots of different mics to be able to pick up all those different instruments..
And all of that sound would come into like a sound console like this one..
And you'd have all these different tracks, every single unique individual instrument.
found in one of those tracks..
And here's the great thing..
If you're the sound engineer, if you want to hear like, say, the electric guitar, you.
just find that particular track, you'd solo it..
And as the band's playing, all you get to hear is that one electric guitar..
You can make sure it's in tune and sounds good..
Or then maybe you might want to solo the drum kit to make sure the drums sound good..
Or however it might be, you're soloing the individual tracks so that you can listen specifically.
to what's important for you..
Now, this is the interesting point..
The band doesn't want you just to hear the electric guitar..
They don't want you just to hear the one vocal or the one piece of drums..
The band records the song because it wants you to hear the whole melody..
It wants you to hear everything that is playing..
And if all you ever do is hear the electric guitar, then you're not actually hearing the.
song that is being created..
Are you following the example?.
See, I think reading scripture is very much like that..
In fact, I want to give you an actual example together today..
And I want to show you what I think kind of some of us do when it comes to reading scripture..
So for example, I can control this sound desk from this little thing here..
And so I'm going to play a track that I think sounds a little bit like maybe the gospel.
of Mark..
I think it might sound something like this..
So there it is..
There's like that entry into the person of Jesus..
Mark is this gospel that gives us the action of Jesus, kind of who he is and what he does..
And maybe you're somebody who loves the gospel of Mark and that's what you often listen to,.
just that sound..
But I want to challenge you that actually maybe it's not just about Mark..
Maybe you also like the gospel of John..
Gospel of John is a little bit more kind of there than Mark..
See, the gospel of John doesn't just tell us about what Jesus did, but it also tells.

$^{281}$us about what he taught..
So the gospel of John has a little bit more sound, a little bit more fullness to it..
Or what about maybe you're an Old Testament person, a little bit like my wife, Christine..
She loves the Old Testament..
One of our favorite books, perhaps like many of us, is the Psalms..
I think the Psalms might sound something like this, right?.
Uplifting and kind of beautiful and the Psalms, we love to go to, don't we?.
Just for that personal devotion and just for that calming influence that they have..
I remember many years, John Snowgrove used to tell us that his favorite book in the Old.
Testament was Joshua, the conquering of the promised land..
I think Joshua would sound a bit like this, you know, the drum set going, the kind of.
pounding of the rhythms of the drums as Israel goes into the promised land and defeats the.
nations there..
For me personally, my favorite book of the Bible is the book of Acts..
And I think Acts would sound something like this..
You know, there's that excitement to it..
It's kind of upbeat..
It's kind of happening..
There's movement to it..
And maybe if you are a reader of the Bible, you read Acts and maybe every once in a while.
you add in the Psalms and you get kind of that sound as well..
And there's a little bit of those two things because you're reading both of those books..
And then maybe if you have time, you throw in the gospel..
So let's just add Mark back in there as well..
So you've got Mark and the Psalms and Acts, and you're beginning to hear something more,.
but you're still not actually hearing the song that the artist wants you to hear..
That song actually sounds more like this..
That's all the tracks, all the tracks that are playing, all 66 of them helping you to.
hear..
But here's the reality, so many of us settle with just hearing this..
We're happy just to hear that one book of Scripture that we focus on perhaps predominantly.
when actually God wants us to hear all 66 of them playing together..
Every single one creating this incredible sound for us..
And we settle for just the electric guitar..
And God has us celebrating the whole melody..
See I think that's what Scripture reading is actually all about..
I think that's why God gives us 66 books..
He wants us to hear the whole symphony play..
I think you should read the Bible as in each book gives us something of the melody of Jesus..
Here it is the symphony of Scripture..

$^{321}$And perhaps right now you've found that your predominant place is reading one book, maybe.
two books..
And you think that that's been the totality..
You think you've got everything you need to know about God and Jesus from just those books..
And I want to challenge you and what this spiritual formation toolkit and this element.
of exploring the Bible is designed to help you to do is to begin to read the Bible like.
I've been describing it here today as one story together..
See the actual toolkit for this particular discipline of exploring the Bible is broken.
into two sections..
There is the kind of action steps and then there's going deeper..
And I want to tell you a little bit practically about that toolkit to help you to begin to.
think about, okay, I get you, Andrew..
I want to read the Bible as one story..
What are some practical ways that I can begin to do that?.
Well, in those action steps of that particular part of the toolkit, we actually mentioned.
a couple of things that are good resources for you to get..
The first is a book that's by Gordon Fee and Douglas Stewart..
It's called How to Read the Bible for All its Worth..
And that's a fantastic book that just, I don't get a commission for it, by the way..
It's a fantastic book that just opens up for us so much about what it is to read the Bible.
as one story..
How can you interpret those different genres?.
How do you understand the flow of Scripture?.
How do you bring it all together?.
So you're hearing all those individual tracks playing in one song..
It's a great book to get..
They actually released a sequel to that book called How to Read the Bible Book by Book..
And that's a great way for you to understand perhaps those individual tracks on the sound.
console that helps you to really understand what that particular book of the Bible is.
about and how that book pushes along the story of Jesus..
So I can't recommend those two books more highly..
Also in the toolkit, we talk about in the Action Steps segment about the Bible Project..
You might have already heard about the Bible Project..
It's a fantastic resource..
Lots of videos that are online..
You can go to their website..
We have the links to it in the toolkit..
And the Bible Project will help you to, again, see the totality of Scripture as that one.
story..
And you can watch lots of different videos from the Old Testament and the New Testament.

$^{361}$to begin to really get your understanding of what the books of the Bible are all about..
And then for those that want to go a bit deeper, we have a further, a kind of going deeper.
section in the part of the toolkit..
And there's a couple of things in there I wanted to just tell you about..
First of all, there is a biblical timeline that shows you all 66 books of the Bible and.
how they fit into a chronological timeline..
Now this is really interesting and really important because the books found in our canon.
as they occur in the actual Bible are not necessarily in chronological order..
But what we can do in that timeline is give a map for you to understand the chronological.
development of the story of God..
That way, when you're reading a particular book of Scripture, you can link it to that.
timeline and that will help you to understand the moment in history of that part of the.
book that you're reading..
I want to recommend to print out that timeline and put it actually in your Bible..
It's a great resource for you to have..
One of the other sections in going deeper is this idea of Lectio Divina..
You may have heard about Lectio Divina before..
It's a practice that's happened in the church over thousands of years..
It's really a way to prayerfully work through Scripture..
And there's a great app that we tell you about in the toolkit called Lectio 365..
And it's an app that you can download..
I believe it's free to use..
And in that app, there's a daily reflection and devotion, as well as a guide to help you.
to daily pray through a certain segment of Scripture..
And it's nothing more important than just reading Scripture, but also than making it.
a prayer, internalizing it, bringing your faith deeply into it..
And I want to encourage you to think about doing that..
The final thing that I want to say is really helpful as you begin to read the Bible as.
one story is begin to read it from different perspectives..
There's nothing more valuable for you than to get a group of people together who are.
from different cultures and backgrounds to you and to read the Bible from each other's.
different culture..
In this day and age where we see such importance in multiculturalism, where we need to understand.
and listen to the stories of other cultures and backgrounds, reading Scripture from different.
people's perspectives can really help..
That can be different gender perspectives..
It can be different race perspectives or culture perspectives..
That's a fantastic way for you to open up the glory of God's story in new and fresh.
ways..
So I want to encourage you to think about what it might be if you only ever read the.

$^{401}$Bible with people that look like you, dress like you, sound like you, and believe exactly.
the same things you do..
You might want to broaden your reading of Scripture by inviting in some people from.
different cultures and backgrounds..
I believe that will be an amazing investment that will bring great riches in your life..
That's the practical elements of the toolkit..
As I draw today to a close, I want to read you a passage of Scripture where I think all.
of this idea of reading the Bible as one story comes together..
It comes from a moment in Jesus' life..
This actually takes place after Jesus has died and rose again..
It's during His resurrection moments..
There are two disciples walking on the road back to their home in Emmaus..
They just experienced all the things that have happened in Jerusalem with the arrest.
of Jesus and His death on the cross..
As they're walking along the Emmaus road, they're talking together about everything.
they just witnessed..
Luke tells us in chapter 24 that Jesus comes up alongside them and He holds Himself from.
being recognized by them..
I don't know how He did that, but they weren't able to recognize Him straight away..
He asked them, "What are you talking about as you walk on the road?".
They're downcast..
They're sad..
They're really overwhelmed by trauma of what they just experienced..
They turn to one another and they say to Jesus, "Are you not aware of everything that's just.
happened in Jerusalem?".
And Jesus says, "Well, explain it to me.".
And they said, "Well, there's this Jesus that we believe was the Messiah, but we guess that.
He must just have been a prophet because they've just killed Him on the cross..
And what's more, the women have amazed us..
They now tell us that this Jesus has risen from the dead, but we're still not sure..
Maybe He's just this prophet.".
And in that moment, Jesus turns to them..
I want to pick up this passage..
This is Luke 24, just reading from verse 25 and 27..
Jesus said to them, "How foolish you are and how slow of heart to believe all that the.
prophets have spoken..
Do not the Christ have to suffer these things and then enter His glory?".
And beginning with Moses and all the prophets, He explained to them what was said in all.
the scriptures concerning Himself..
"If you could transport me to any moment in history, that's the moment I would want to.

$^{441}$go to..
Would you imagine what it would be like to stand next to those disciples with the resurrected.
Jesus and have Jesus open up all of the scriptures for them, the whole of the Old Testament,.
opening up all of the Old Testament and showing them how those passages prophesied and spoke.
about Him?".
I don't think what Jesus did was kind of gather it up and go like, "Okay, let's go back to.
Genesis 1 and 2..
Okay, where it talks about He spoke and they came into being..
That was me..
I'm the Logos..
I'm the Word..
I was right there at the beginning..
Okay, let's go forward to Psalm 10..
That's all about me.".
I don't think that's what Jesus was doing..
I think instead what Jesus was doing was retelling them the story of Israel..
I think He was going back into the scriptures and telling them the story of Israel and showing.
them how all of that story now culminates, not just in the birth and the incarnation.
of Jesus, but in His death on the cross and the resurrection after that..
And all of that is pushing forward the narrative..
Jesus is saying, "How slow you are!.
Could you not see in the beauty of all 66 tracks the symphony of the life, death, and.
resurrection of me, of your Savior?".
Jesus basically challenges them to read the Bible as one story..
And it's this amazing thing that the disciples say just after that..
Let me read this to you in verse 32..
They asked each other, "Were not our hearts burning within us while He talked with us.
on the road and He opened the scriptures to us?".
If there's one prayer that I have for us as a community of faith, if there's one thing.
that I've dedicated the last 20 years of my life to, if there's one thing as your senior.
pastor that I am the most passionate about, it's that your hearts would burn in you as.
we as a community of faith open the scriptures together..
As you see the Bible come alive, as the words of God don't become this boring thing that.
you kind of feel like you're forced to read, but you open the scripture starting with the.
story itself and then find your story within it..
That as you see Jesus come alive in the narrative of the gospels and in all of those 66 books,.
all playing the one track that sings the melody of Jesus, as you hear that, your heart would.
burn within you..
And there would be change and transformation so you would grow as a big person, deeply.
rooted in the things of the gospel and Jesus and transformative in your outreach to bring.

$^{481}$justice to the nations..
That's what I believe the reading of scripture can do..
May your hearts burn in you as God brings alive His Word..
Jesus before He ascends, gathers the disciples in the upper room..
The Bible tells us He does this amazing thing..
It says that He opened their minds so they could understand the scriptures..
I want to pray for you right now..
The Holy Spirit would come wherever you are and open up your mind and give you a new passion.
for God's Word that it would burn in you like never before..
So you can understand the fullness of the story of God..
Let me pray for us..
Father, we stand in this moment as we start this new series of Flourish..
As we turn our hearts to this spiritual formation toolkit, this online resource available for.
us so that we can have deep roots in our spirituality on a daily basis..
Father, we turn to your scriptures today..
Lord, forgive us where we've maximized just one channel on that sound console..
Where we've made our bread and butter and the depth of our faith simply an electric.
guitar rather than the wholeness of the sound of the melody of the band..
Lord, would you bring alive once again your scriptures to us..
Lord, like those disciples in Emmaus, we want our hearts burning within us..
And it's because you open the Word to us..
It's because your Spirit comes and helps us to understand what these 66 books have to.
say about you and how therefore it has to say about us..
Lord, as we open your Word this week, I want to pray for a fresh anointing on the Vine.
Church..
Lord, I want to pray that you would come like you did for the disciples in the upper room..
You would open our minds so we could understand your Word at deeper levels than we've ever.
experienced before..
Father, forgive us where we've made ourselves the center of scripture..
Forgive us where we treated it like some flimsy self-help book..
Lord, root us in your story..
May we find depth and truth and justice and renewal and restoration and hope in the dark.
times that we're facing all around us in this moment..
May we find that in a robust understanding of your story..
Lord, would your story come alive to us again..
And as it does so, Lord, we pray for changed lives, changed communities, and we pray for.
a changed city..
And we thank you for this..
In Jesus' name, everyone says, amen..
Amen..

$^{521}$[Music].
\newpage



\section{}
\label{sec:ODvUnLlR0pQ}
\textbf{2021-03-08 Church Everywhere Live: Flourish - Shaped By Love [ODvUnLlR0pQ].mp3}
\newline
\newline
連結: \href{https://youtube.com/watch?v=ODvUnLlR0pQ}{\texttt{ https://youtube.com/watch?v=ODvUnLlR0pQ}} ~~~~ 語音日期: 2021-03-08 
\newline
\newline
\hyperref[sec:TWnjFhuOiys]{\small{< < < PREV SERMON < < <}}
~
\hyperref[sec:index]{\small{[返主目錄]}}
~
\hyperref[sec:DS7k2k5IRoo]{\small{> > > NEXT SERMON > > >}}
\newline
\newline
$^{1}$Thank you, Pastor Ellison..
As he shared, we know each other for a long time..
He's actually the one that also brought me to the Vine Church..
But it's so good to be with you this morning..
Let me start with what God has put on my heart for you all..
When I was a freshman in university, they announced in chapel that you could sign up.
for community groups..
And it sounded like a good thing to me, so I signed up for one..
They told us our group leader will be in touch and give us all the information about our.
group..
So I got the details about our first meeting..
And when I show up, there's John from my hallway that I really struggled with..
Our personalities, they were so different..
And I had difficulty listening to him and being patient with him..
And quite frankly, loving him as a Christian brother, as my "die-hing.".
Now I knew right then that God placed John in my group because he wanted me to learn.
to love and appreciate this brother who's made in the image of God, whom Jesus died.
for and gave everything for..
But truth be told, I very much struggled to have patience with and love..
And perhaps you have a John in your life right now, maybe a neighbor you don't like, a colleague.
whom you can't stand, a family member that drives you crazy..
I feel like we all have that..
A church member that rubs you the wrong way, or perhaps even a pastor that bugs you..
So someone that you struggle to love and to care for and to show God's patience and compassion.
to..
Or maybe there are people in your sphere of influence that you already love, but you're.
very well aware that there are times where you miss the mark and you hurt and disappoint.
those you love..
We all have those moments where we realize we are not as loving of a person as we hope.
to be..
Or think of your relationship with God..
There probably isn't a week that goes by where we love God in the way that we wish and desire.
to..
Now this series that we are in is called Flourish, and it's all about deepening your spiritual.
foundation and disciplines in order for you to have a more mature relationship with Jesus..
And this week we are taking a look at the ingredient of community..
And in particular the idea that community is essentially our experience and exercise.
of love towards God and towards our neighbor..
Now one of the major ways that we as Christians are called to grow in love according to Jesus.
is to love our neighbor and to love him..

$^{41}$And that's our highest calling..
And Jesus, he put it this way..
He said, "Love the Lord your God with all your heart and with all your soul and with.
all your mind.".
And this is the first and greatest commandment..
And the second is like it..
Love your neighbor as yourself..
So we are meant to love God and reflect God's love in our relationships with those inside.
but also outside the Christian community..
But we so often miss the mark of loving God and loving others..
So today we're talking about being shaped by God's love through community..
And God wants to form us into more loving, less critical, and more gracious people..
And so today I want to explore with you in this message how do we become more loving?.
How do we grow in this Christ-like love?.
And what role does community play in all this?.
We're going to get really practical today..
And so towards the end of the message I'm going to be talking about our spiritual formation.
toolkit where we'll share some concrete steps that you can take as you seek to grow in love.
and faith through community..
So does that sound good?.
Great..
So let's turn to an episode in Jesus' life that will provide some answers to these questions.
that we're exploring..
We're going to be looking at a biblical account of Jesus being anointed for burial by a woman..
And let me provide you with a little context..
Jesus is in his last days before the crucifixion..
Just prior to our passage we learn that the religious leaders, they plan to kill Jesus..
And right after our passage we learn that Judas, one of Jesus' own disciples, will turn.
on him and aid the religious leaders in betraying Jesus..
So the story of the woman anointing Jesus and expressing her devotion and love stands.
in strong contrast to the hostility that Jesus is facing at this time..
So this is not the typical Bible passage that people draw on when they talk about growing.
in love through community..
But I believe that there's an angle from which we can look at this story that will shed some.
light on how we can grow in love through community..
So we're in Mark chapter 14 and verse 3..
It says, "And while he was at Bethany in the house of Simon the leper, as he was reclining.
at a table, a woman came with an alabaster flask of ointment of pure nard, very costly..
And she broke the flask and poured it over his head..
There were some who said to themselves, 'Why was the ointment wasted like that?.

$^{81}$For this ointment could have been sold for more than 300 denarii and given to the poor.'.
And they scolded her..
But Jesus said, 'Leave her alone..
Why do you trouble her?.
She has done a beautiful thing to me..
For you always have the poor with you..
And whenever you want, you can do good for them, but you will not always have me..
She has done what she could..
She has anointed my body beforehand for burial..
And truly I say to you, wherever the gospel is proclaimed in the whole world, what she.
has done will be told in memory of her.".
Now what we're going to do is to look at the story and see what different characters in.
the story teach us about how God works through community in our lives, to shape us into people.
who reflect his love in a more consistent way..
So first there's the woman that anoints Jesus..
She walks into Simon the leper's house where there are men hanging out, and she's disregarding.
cultural conventions by stepping in that space..
I mean, it's quite a courageous thing to do..
And there's something that moves her to do this..
The one thing that is clear is that her actions of walking into the house and anointing Jesus.
is an act of her love and devotion to Jesus..
Even though the gospel of Mark, it does not tell us, but it's safe to assume that Jesus.
through this ministry deeply touched this woman's life..
So in response to who Jesus is and what he has done, she expresses her love and devotion.
by anointing him in this beautiful and lavish way..
She breaks the neck of the jar, which actually was technically not necessary, and she pours.
that ointment over him..
And it was a dramatic gesture which demonstrated the woman's unreserved devotion to Jesus,.
holding nothing back whenever we respond to what God is doing in our lives with acts of.
love and devotion to him..
It is a beautiful thing..
Now I don't know if you ever have been touched and inspired in your love for Jesus and neighbor.
by other people in your community..
I certainly have..
When I moved from Germany to the US as a 17 year old, it was the first time that I experienced.
an intentional Christian community..
I was inspired especially by my host brother Dave's passion and love for Jesus..
I mean his love for Jesus, it awoken me to desire..
I wanted to know Jesus..
I wanted to spend more time with him..

$^{121}$So community is a place where our faith and love for Jesus and for others is strengthened.
and can even be set on fire..
Seeing this love in others often makes us draw near to Jesus because we want to know.
Jesus in that same way too..
We want to love Jesus in that same way..
So whose love for Jesus have you been touched and inspired by?.
Now being inspired and touched by this woman's act of love and devotion to Jesus could have.
been a response by the people watching the woman anoint Jesus..
But that's not how they respond..
You have this beautiful act of love and devotion and immediately you have the haters line up.
to hit on her..
New Testament professor Tom Wright, he kind of comments on this this way..
He says, "It always happens when people decide to worship Jesus without an ambition, to pour.
out their valuables, their stories, their dancing, their music before him, just the.
way they feel like doing that others looking on find the spectacle embarrassing and distasteful..
Whenever there are acts of love and devotion done for Jesus, criticism often follows.".
Now it's interesting because the criticism against the woman is actually couched in spiritual.
language..
So in verse 4 and 5 it says, "There were some who said to themselves indignantly, 'Why was.
the ointment wasted like that?.
For this ointment could have been sold for more than 300 denarii, a lot of money, and.
given to the poor.'.
And they scolded her.".
Now it's easy to use pious excuses to condemn and judge other people..
You know it sounds very spiritual to talk about the money that could have been given.
to the poor..
Talking about not wasting resources, I mean it sounds like good economics..
But what is being missed in this context is what the woman actually sees..
And it's about that Jesus is about to lay his life down at the cross for all humanity's.
sake..
What is being missed here is the heart and giving and expressing her devotion to Jesus..
But in the sacred moment of all of this is not seen..
Now what was the last time you were critical or judgmental of someone?.
When was the last time you spiritually judged another person's action using your language.
or spiritual language to critique their intentions?.
What did you fail to see when you were judging the person?.
Because there's always a broader context..
Now I love how Jesus reacts to those who judge the woman..
He responds by guarding the sacred moment of love and devotion expressed to him..
But he does it in such a way that it does not cut off the critics from community and.

$^{161}$relationship with him, but rather invites them to examine their hearts..
Now there's four ways that Jesus guards the sacred moment..
First, Jesus defends the woman and tells those who are scolding her, "Leave her alone and.
stop troubling her.".
Second, where is the man Jesus is sharing a meal with that they're criticizing the woman.
for anointing him?.
Jesus affirms what she has done as a beautiful thing done to him..
And he does not let the critics' negative judgment stand, but he affirms her act of.
love and devotion..
And third, Jesus does not just tell those judging the woman to stop, but he also explains.
what the critics have missed..
Jesus exposes this false spirituality and lack of love when he says, "For you always.
have the poor with you..
And whenever you want, you can do good for them, but you will not always have me.".
So Jesus is saying that the poor are always the Christian community's responsibility,.
and that no one is stopping them from giving money to the poor or helping them..
And so it's a little too convenient to talk about what others should be doing for those.
living in poverty instead of taking actions themselves..
So Jesus explains what the critics have missed..
They have missed their own lack of action and caring for the poor, and they have missed.
the significance of this moment, which served as his anointing for burial before his death.
that Jesus was about to die on the cross..
That did not miss the significance of this, but the critics did..
And Jesus did not have to explain all of this, but he chose to because he wanted to invite.
the critics to let go of judgment and to surrender to the love and grace of Jesus..
Finally, Jesus does not only defend the woman, but he celebrates her..
He celebrates what she has done..
She will be known..
Her act and devotion of love and willingness to courageously anoint Jesus will be forever.
memorialized and celebrated..
They're part of the gospel..
So Jesus does not let the sacred moment to be destroyed..
Instead he makes sure that the beauty of this love and devotion of this woman will be remembered.
forever..
And in fact, we are remembering it right now..
Isn't that awesome?.
So what do the woman, the critics and Jesus teach us about how we can grow more through.
community?.
First, we learned from the woman that a person's love and devotion to Jesus has the potential.
to inspire love and devotion towards Jesus and others..

$^{201}$My host brother David's love for Jesus and others inspired me to grow into more of a.
loving person and to love my neighbors more..
Who in your community is inspiring you to love Jesus and others?.
Who are you inspiring with your love and devotion to Jesus?.
Second, living in community brings out the best and the worst in us..
In the critics' case, the woman's extraordinary love and devotion brought out their judgment,.
criticalness and lack of love..
Now, truth be told, we all to one degree or another judge and condemn others..
And it's especially in community where a lack of love and grace surfaces..
Sometimes we're made aware of our lack of love by members of the community..
They might tell us that just like Jesus did with the critics..
And other times it might be the Holy Spirit convicting us..
And maybe the Holy Spirit is convicting you right now..
It is these moments of failure to love that provide an unexpected opportunity to grow.
in love..
These moments are an opportunity for healing and growth when we come to terms with them.
and own them..
You know, we think that we grow in love by working harder..
I'm going to try harder, but that's actually not how it works..
If we try to be less critical and judgmental out of our own strength, you know, we're going.
to fail..
But instead of trying to love more out of our own strength, my question is, will you.
surrender to God's love and spend time with him and allow yourself to be loved by God?.
Take some time to ask God to give you the grace to love those whom you struggle to love..
Richard Rohr, a well-known Christian author, he puts it like this..
Most of us were taught that God would love us if and when we change..
In fact, God loves you so that you can change..
What empowers change, what makes you desire is of change is the experience of love..
It is that inherent experience of love that becomes the engine for change..
You know, in Romans it says, "For it's your kindness that leads us to repentance.".
It's God's love that is the engine of change..
So surrendering to love is the comeback into the arms of our heavenly father..
He does not scold us for failing to love, but he welcomes you..
He restores you..
He embraces you..
And surrendering to love is to receive God's love and forgiveness, his grace..
It is his love and grace that transforms, you know, as we sit in God's presence and.
commune with him and just spend time with him..
This is what transforms us, not our trying harder..
No, God wants to release us from our own efforts of conjuring up or manufacturing this love.

$^{241}$out of our own strength..
And as a community of faith, we actually get to remind each other of our heavenly father's.
love that is so radical and unlike human love..
We remind ourselves of the truth that we love not out of our own strength, but relying on.
God's strength and our experience of being loved by God..
As Apostle John in 1 John 4, 19 put it, "We love because he first loved us.".
When we run out of love, we turn to God who never runs out of love..
Our failures to love and along the way, our opportunities again to surrender to Jesus,.
who is working and shaping and forming us through this grace into more loving people,.
people that are more consistently reflect his love and grace..
That is what happened with me when I was in community group with John in university..
Throughout a two-year period of time, God changed my heart and gave me the grace and.
the strength to learn to love John who had such a different personality from me and came.
from a different cultural background..
And it was a process of surrendering to God's love and grace on a regular basis that enabled.
me to do so..
Now I've experienced how God has grown me spiritually through community..
And right now I want to share with you about this week's Spiritual Formation Toolkit, Live.
in Community..
So like we've seen in the first two weeks, the Community Toolkit is broken into action.
steps and going deeper..
Let me just highlight a few things in each to give you a sense of what is in this week's.
tool..
For action steps, we have a document that's called "Why Invest Yourself in Christian Community.".
We also have a link to our sermon series, "A Call to Community" that can help you to.
understand the importance of Christian community..
So maybe you have questions around that..
That's a good place to start..
We also have a link to our community group map where you can explore groups and sign.
up for them and join them..
And just as a reminder, some of the pins on the map, they actually have multiple groups..
So make sure you scroll on them and see if there's other groups there..
Also if you want to start a group in your school, work, or a parent's group, or maybe.
in your home, you can email us at cg@divine.org.hk..
You know, we want to explore that possibility with you and support you to get those communities.
started..
For going deeper, we have a workshop on how you can become a river connector who helps.
other people connect to community..
You can check out the QR code and find out more information there..
If bringing people together brings you joy, I want to encourage you to sign up for this.

$^{281}$practical workshop on how you can use your gift to connect people to various expressions.
of community here at the Vine..
So this workshop is happening on the 11th of March, 8 to 9 p.m..
So check that out if that's for you..
We also suggest resources for those of you who want to go deeper into community, maybe.
with a friend or a group of people..
And I'm just going to highlight one of those resources..
It's by the author James Bryan Smith, and he's written a beautiful trilogy that's called.
The Good and Beautiful God, The Good and Beautiful Life, and The Good and Beautiful Community..
Each book has a small group discussion guide in the back of the book and has practical.
exercises on how you can grow in your relationship with God, how you can become a more Christlike.
person, and how you can grow in loving your neighbor..
So I want to encourage you to take a look at the Spiritual Formation Toolkit, Live in.
Community, and invest in your own and other spiritual growth by taking one suggested step,.
and then just see what God does, how he works through that as you take that step..
So as I draw today to a close, I want to leave you with a personal story that I believe illustrates.
how Jesus modeled love and grace to both the woman that anointed him, but also her critics..
Broken God makes his love tangible to us through community so that we can experience it and.
receive it..
In 2017, I went through the toughest time of my life, loss of relationship, loss of.
work, and loss of my home..
And when this happened, it really made me question God's love for me..
And in my brokenness and shame and loss, I wondered if God still loved me..
And the signs, they seemed to point in another direction..
But it was through community that God showed me his love..
It was the people around me that made God's love believable and tangible in my life..
Let me tell you about Akbar, Regina, and Kailani, their daughter..
When I had seemingly lost everything, they took me into their home..
They did not just give me a room to sleep in, but they welcomed me into their family..
And when I had lost a family, they were family for me..
Now Regina, she was highly pregnant during that time..
And when it came close to giving birth, I asked her, "Are you sure you want me to stay.
here during that time?".
And she assured me that they wanted me there..
And I was a mess..
Tears were frequent..
I didn't even know how I could possibly move forward with my life..
I didn't know what to do..
But in the midst of all of that, they showed me hints of God's unconditional love by welcoming.
me to be part of their family..

$^{321}$It's this family, my community, that made God's love tangible for me in a season where.
I had every reason to doubt God's love..
And they contributed in a major way to God healing deep wounds and hurts in my life..
And I honestly don't know where I would be today without them..
When I moved to Hong Kong in 2018 and moved out of their house, it was the first time.
in my life that I cried when I said goodbye to someone..
I usually don't do that, even though I'm crying right now..
But God, I know why I cried..
It was because their love was a reflection of God's love for me..
And God is fiercely in love with you..
And he wants you to know that..
And he uses community to show you that..
And I have a sense that some of you, the Holy Spirit is nudging to take the step of joining.
a community..
Don't wait any longer..
Just do it..
Let God work through community in your life..
But there's also a sense that some of you are called, like Akbar and Regina, to be community.
to others and to let God's love and grace flow through you to people who are struggling.
to believe God's love right now..
You know, and the awesome thing about that is you can do that anywhere..
It doesn't have to be in a traditional small group..
You can do that in your workplace..
You can be community for people, you can be community for people in your family, your.
extended family..
You can be community for people in your neighborhood..
Will you step into community and be known?.
Will you be community for others so that they might know God's unconditional love through.
you?.
Let me pray for us..
Jesus, I thank you so much that you call us into community and that you want us to give.
us a community where we can experience just your love..
And sometimes we're the ones that make that love tangible to others and sometimes we're.
the ones who need that love made tangible to us..
And so I just ask, Lord, that you know everyone who's watching and what they need, will you.
just speak to them right now and just show them what you're asking them to do, whether.
it's just to be community for others or take that step of stepping into community..
Now also pray, Lord, that you will make your love tangible to them..
Those who are right now are struggling and doubting God's love for them or even wonder.
if you are even there..

$^{361}$I pray that you will show them your love through us, the Vine family, that we can be community.
for others..
And I pray that in your name, Jesus..
Amen..
[BLANK AUDIO].
\newpage



\section{}
\label{sec:DS7k2k5IRoo}
\textbf{2021-03-15 Church Everywhere Live: Flourish - Refreshing Friendships [DS7k2k5IRoo].mp3}
\newline
\newline
連結: \href{https://youtube.com/watch?v=DS7k2k5IRoo}{\texttt{ https://youtube.com/watch?v=DS7k2k5IRoo}} ~~~~ 語音日期: 2021-03-15 
\newline
\newline
\hyperref[sec:ODvUnLlR0pQ]{\small{< < < PREV SERMON < < <}}
~
\hyperref[sec:index]{\small{[返主目錄]}}
~
\hyperref[sec:M1r4SUj2xRk]{\small{> > > NEXT SERMON > > >}}
\newline
\newline
$^{1}$We are so glad that you are here in this moment and a part of what we're doing..
And I'm really excited. We're doing something a little bit different today..
And we're going to have a conversation on this topic of spiritual friendships..
And, you know, this is something that's always been integral to our vision of growing big people here at The Vine..
We always say that actually our most amount of profound maturity and growth happens,.
not just in our own individual moments with God,.
but actually as we live out life with one another in that community..
Spiritual friendships are a part of that community..
It's where one or two people gather together and share life at a deeper level,.
pray for one another, love one another, walk with one another..
And that's what we get to do today. That's what we're going to talk about..
And we thought one of the great ways to do that would be to actually model to you.
a little bit of what those kind of relationships and friendships are like.
by doing an intimate conversation with myself and Sarah O..
And welcome, Sarah. It's great to have you with us..
It's great to be here. Thank you for having me..
You know, we're so blessed because Sarah has really over a number of years now.
with her family been embedded here at The Vine and just shared so much about our relationships.
and what life is like doing life together..
She's been very formative in our growing big people vision as well..
And I just want to thank you so much for the way that you haven't just sat in a pew on a Sunday,.
but you've really kind of really integrated your heart and your philosophy and spirituality into The Vine..
So, thank you. And what a pleasure to have you today..
It's great. It's so much nicer than sitting in my living room..
Yeah, and you're not in your pajamas either. You're looking very nice. That's great..
So, we're talking about spiritual friendships and in particular,.
you know, how mentoring and spiritual friendships kind of works together..
And it might sound a little obvious, but, you know, relationships obviously sits in the center of that..
And I know that you've thought and reflected quite a bit on what relationship is..
Tell me a little bit about how you see relationship being integral to a kind of spiritual relationship.
or spiritual kind of friendship environment..
Right. You know, you'd think it'd be obvious, but actually, sometimes it's easy to overlook the obvious thing.
and get busy with many, many things..
And that's what happened in Jesus' day..
You know, the Jews were really busy doing a lot of things to worship..
And when asked, Jesus summed up their entire law by saying,.
"It's really about loving God, loving others as yourself.".
And we at The Vine also, we have so many great ways to connect,.
but it's really worthwhile to pause and think about what's at the heart of those connections,.
which is relationship. And, you know, especially in this time of COVID, you know..

$^{41}$Yeah, you know, this has been such an interesting thing for all of us, right?.
COVID, how that's been a challenge to some of our relationships,.
how that's actually kind of thrown things a little bit out of the normal mix..
You and I were reflecting on this the other day..
Tell me a little bit more about what COVID has done for you in terms of your relationships.
and how that might impact us either in positive or negative ways.
as it comes into kind of thinking about spiritual friendships in this time..
Yeah, you know, before COVID, we could just, most of us could just turn up to places and events.
and have really nicely organized ways of connecting..
You know, you could go to work, you could go to church, you could go out to dinner with friends.
and just have this automatic ways of connecting without doing a lot of work..
Well, COVID took so much of that away..
And so on the one hand, we lost our usual ways of connecting..
And then on the other hand, we all found ourselves thrown into really spending time with a few people..
And then all of a sudden we had to deal with relationships..
And so, you know, one of the gifts of COVID is that it really leveled the playing field..
You know, we all experienced the same trauma..
We all went into this survival mode..
We all kind of felt like we lost control over our environment..
And so it did give us ways to connect on a human level that we really hadn't had before..
And so really, you know, though we had to deal with, and we still have to deal with relationships,.
the good, the bad, and the ugly, all of it..
But now we have this tremendous opportunity to show up with one another in more vulnerable and deep ways..
You know, showing up means being willing to be vulnerable and be present with people in their joys and sorrows..
I think this is such an important thing you're saying,.
because I think vulnerability is being something that this whole COVID season has opened to us..
And not only that, but the social unrest season we were in before,.
you know, and one of the interesting things with that time was that social unrest was such a divisive time..
It was like, which side are you on?.
And everybody was trying to work out like that kind of sort of,.
that sort of divided a lot of families and a lot of people in society politically at the time..
But COVID has sort of had the opposite effect in a way, right?.
It's kind of drawn us together..
It's kind of made us sort of say, look, we're all in this against this kind of disease, and how do we do that?.
But I think this idea of vulnerability is critical..
You know, one of the things that I'm sure many of you at home are sensing and feeling in this time is,.
is that sense of vulnerability, maybe not being around family as much as perhaps you have been able to in the past,.
or there's been disruptions in family relationships during this time..
And I think what sits at the heart of spiritual friendships is the idea of counterbalancing that disruption with love..
And the thought of like, you know, these relationships are a place where we can express and experience and find love..

$^{81}$And I know that you've thought a lot about this in relation to Jesus,.
and the ways in which Jesus, both with His disciples and others,.
model this kind of spiritual friendship and love in the heart of that..
Can you tell us a little bit more about what you see in Jesus and how Jesus models this for us in our relationships?.
Right, yeah..
You know, Jesus spent three years with His disciples..
They did everything together..
And on the eve of His crucifixion near the end of His life, He wanted to have dinner with them one last time..
And, you know, it's funny, all the ministry they did, all the traveling they did, and the teaching,.
John describes the relationship between these men in this way..
He says that Jesus loved His disciples during His ministry on earth, and now He loved them to the end..
And another version says that He showed them the full extent of His love..
And of course we know at this dinner, the way that Jesus demonstrated His love.
was by taking off His robe and washing the disciples' feet..
And then He told them to do the same..
And then as they talked and Jesus was sharing more, you know, really what was pressing on His heart,.
this is the eve of His crucifixion, you know, and He tells them He's going away..
And, you know, the disciples are all like, "Wait, where are you going?.
Show us the way.".
And Jesus does say something really interesting..
He says, "You do know the way.".
And He said, "I am the way.".
You know, the disciples wanted instructions..
They wanted directions to follow Jesus, but Jesus actually gave them Himself..
And so in the end, you know, it came down to love..
It came down to relationship..
And of course we know He went to the cross the next day..
And when we think of the cross, you know, we think about Jesus dying..
We think about how that brought us salvation..
But actually, the way that John says in that really well-known verse that most of us know,.
"For God so loved the world that He gave us salvation.".
No, it doesn't say that..
It says that God demonstrated His love for us..
He loved the world, so He gave us His one and only Son..
So the cross isn't primarily about salvation..
It's really about love and Jesus, or Jesus giving Himself, and the Father giving us Jesus in love..
I think this is kind of an interesting thing to bring up that perhaps is something we don't reflect on a lot,.
which is the idea that actually we have this God who offers Himself,.
like not offering a ministry or offering even the church..
The church was a beautiful thing that came, but Jesus gave us Himself.

$^{121}$and still continues to give of Himself today..
How do you think Jesus was able to do that?.
What was it about—now, we know Jesus was the Son of God,.
but what was it about Jesus that enabled Him to kind of offer Himself.
in this kind of self-sacrificial way to those that He truly loved?.
And what can we see in that that might encourage us as we think about doing that for one another?.
Yeah, you know, He was able to do that because He knew who He was..
But first I want to talk about how practicing life-giving relationships with one another,.
it really flows out of practicing life-giving relationship with Jesus..
You know, in keeping the greatest commandment to love God and to love others as we love ourselves,.
you know, the action is love, but that action comes from us..
You know, we love God, we love others, and we love ourselves..
And like the rest of the law, you know, even though God tells us and requires it of us,.
we will fall short because our love is flawed..
Our love will fall short..
So after Jesus demonstrated His love for His disciples, He gave them a new commandment..
And the new commandment says, "Love one another," the same, you know, action is love,.
but He said, "As I have loved you, so you must love one another.".
So the new commandment gives us the power to obey the greatest commandment.
because Jesus shows us the way of love..
You know, we don't have to find that love from within ourselves.
because that love will already be in us if we receive it from Jesus..
So our ability to love is really rooted in Jesus..
And then, like I said, He was able to do that because He knew who He was..
You know, John says that before Jesus washed the disciples' feet,.
he says a little bit earlier in the passage that Jesus knew the Father had put all things.
under His power and that He had come from God and that He was returning to God..
And so He got up and He was able to wash the disciples' feet..
You know, Jesus was able to humble Himself because He knew who He was..
He was rooted deeply in the Father's love for Him and the relationships that they had together..
Yeah, I think it's really interesting..
Pastor Tim spoke to us last week on love and love in community..
And now you're kind of picking up on that same sort of idea of,.
you know, because we have received His love, we're able to love one another..
And really, in many ways, spiritual friendships is this kind of sense, isn't it,.
of actually walking in love with one another,.
but doing so in a way that actually sort of gives ourselves to that person..
And I think this is also then ironically one of the challenges of spiritual friendships,.
because it is about me giving myself to you or whoever it is that I'm walking in relationship with..
But you know what? That sounds exhausting..

$^{161}$You know, it sounds tiring..
Like we're already feeling it, aren't we, from this whole kind of COVID situation..
We're all a little bit tired and burnt out..
And now you're asking us to enter into a relationship where I have to give myself to somebody else..
What are some of the reflections you have around that?.
For those that might be watching this right now and kind of freaking out a little bit,.
that we're asking them to give even more of themselves in a time when they're already quite tired..
Yeah, no, I agree. It can sound challenging,.
especially if we feel like we have to dig deep and find that, you know, find what we need to give away..
You know, let me tell you a little bit about my story..
You know, I am an introvert, naturally..
And so I often feel awkward around people..
And I sometimes in relationship, I don't even know where to begin..
And just like everybody else, you know, the past couple of years,.
I really did experience a lot of anxiety and stress..
And, you know, the collective trauma that we experience,.
I really brought up for me the trauma from my childhood..
And so last fall, when I was already feeling so overwhelmed and frankly, I was really just physically exhausted..
And around that time, you know, my husband, Sam, and I were wrestling with a decision to move out of Hong Kong..
And it wasn't moving to go back home to where our family is..
It was really to go to a new country..
And just that whole that whole situation just really pushed me close to burnout..
And so I reached out, you know, I knew I needed help..
I reached out to professionals, but really it wasn't enough for me..
Because I felt that I wanted someone who loved me..
I wanted someone who could see me..
And so I reached out to a really close friend that I meet with regularly..
And I said, look, this is where I'm at right now..
I just laid it all out..
And I said, how would you feel if we took our friendship to the next level?.
And, you know, would you look into my life?.
And I gave her permission to speak honestly, to speak truth..
And really in return, I offered really more access to my heart..
But also that I would do the same in return for her..
And so, you know, it's interesting..
I really want to make the point that even though I was seeking healing at that time, I knew it wasn't about my healing..
Because I knew then I would really be tempted to make it about myself..
Right, yeah, exactly..
So my goal really was to go deeper into love..
Because I knew the deeper I experienced love, that the more healing I would experience..

$^{201}$And I also knew that the more healing I experienced, that I would be better at loving myself, loving God and loving others..
I think you pick up on something there that I see as key, and that is courage..
You know, everything you just spoke about with your own story, and thanks for sharing your personal story with us today..
That takes a huge amount of courage to say to someone, "Hey, we've been friends for a while, but I actually need you..
I need you at a deeper level..
I need you to love me..
I need to open my... and I want to take that courageous step to open my life towards you.".
That's not an easy thing, but I think that is such an important catalyst to spiritual friendship..
It reminds me of the story in Mark chapter 2, and it's quite a well-known story of Jesus returning to Capernaum..
And there He is in a house, doing a house meeting, and these friends make a hole in the roof..
And they kind of let down their buddy who's on a stretcher down before Jesus' feet..
And when you read that story, it's such a beautiful picture, really, of the care and the concern that these four friends on the roof have for the person on the stretcher..
And Jesus looks up and sees their incredible faith, and because of that faith, He responds..
But what I think about that story is, so often we actually think of ourselves as the guys on the roof..
It's like, "I want to be the person on the roof. Like, who? I've got this great faith that's like healing my friend.".
When actually, a lot of times, we're the person on the stretcher..
And it's the courage to say—and I think in that Bible story, you see both elements of what spiritual friendships is about..
And this is one of the beautiful things about it, right?.
Let's say you and I are in a spiritual friendship relationship..
There's going to be seasons where you're like you've just described..
You're facing a really tough life choice, a decision. You're feeling burnout..
I'm going to be the person on the roof, letting you down before Jesus, and speaking life over you, and praying for you..
But in just another month's time or so, I know I'm going to be the person on the stretcher..
I'm going to need to then have you step in from where you're at..
And I think it's that kind of partnership together of being able to really honor each other..
You're not always—spiritual friendships is not always about giving out towards the other..
It's also about that place of receiving..
And sometimes, when we're like the paralytic, when we're on the stretcher itself,.
we desperately need those people to reach out to us..
And maybe ironically, I know that you've had your own physical kind of story recently of kind of being on the stretcher..
Do you want to tell us a bit about that?.
So, about a month ago, I had surgery on both of my feet, which, you know, I don't advise it..
Both at the same time. That's smart, Sarah. Nice one..
But one recovery..
So, I, you know, literally, I still have trouble walking right now, and I need lots of help..
But especially in that first week when my feet were just in so much pain,.
and I literally just had to sit all day with my feet propped up, and just really aware of the pain..
And at that time, I was thinking a lot about Jesus washing the disciples' feet..
And it was just the most intimate experience..
I can't even really describe it, but I experienced Jesus holding my wounded feet in his hands,.

$^{241}$like, all day long..
And it really was life-changing for me..
And I think I just realized in those days that I need, I don't just want to experience Jesus' love for me, I need it..
I need to walk in that kind of connection with Jesus..
Because without him, without receiving that from him, I really couldn't do anything..
And then, as I continue to heal, you know, I'm still depending a lot on my friends and family..
And I'm still, you know, quite frankly, I'm still healing from my burnout..
I've just been realizing that I need to walk..
I don't just want to walk. I need to walk with others in humble, vulnerable interdependence..
And it's kind of ironic because in this time when I can do so little for myself, I can do so little for other people,.
I feel like I'm learning so much about what it means to experience love and give love in return..
And I have these crutches I have to use..
And the doctor said, "Even if you don't feel like you need it, always use them when you go in public.".
Because they're like a physical, visual reminder that I'm delicate, I'm vulnerable,.
and that, you know, please be gentle around me..
So, it's really like wearing my vulnerability on the outside, you know, which has been really a blessing..
It's humbling, but it's a blessing to do that..
Well, let me pull together some of the threads actually of this time.
and what we've been talking about just to help us to kind of track along, I think,.
with some of what Sarah is sharing..
You know, we're talking about the reality that spiritual friendships are about relationship..
They're about love being at the center of those relationships..
We talked about how that love is given to us by God..
There's a storehouse, a reservoir of that love that we receive in order to give..
We talked about how spiritual friendships are about self-sacrifice towards the other,.
and also about courage and vulnerability..
They're about us being courageous enough to say, "You know what? I need help. I need some support in this time.".
And us also realizing that when we have those trusted relationships with others,.
that they're also going to be there for us in those moments where we need them as well..
And so, it's not just all about giving, but it is also about receiving..
And in that sort of cooperative relationship, we're able to then really give to one another.
that kind of love and support that really the Scriptures speak about..
So, we covered quite a lot of ground, I think, in terms of the theory of what spiritual friendships is.
and some of the important elements of that and how we see that in Jesus..
What about some practical tips?.
What are some ways that you would suggest we can practically maybe start some of these relationships?.
Perhaps some who are watching this right now, they already have these relationships in their lives,.
or maybe they don't..
So, what's really some practical tips you'd share?.
Well, the most important tip I can give is just to show up..

$^{281}$Show up and begin..
And, you know, we've lost so much control over many things in our life,.
but there are things within our control that we can begin with..
So, I would say begin with yourself..
Begin where you are, not where you're not..
Don't worry about that..
And begin today..
Don't put it off..
You know, every one of us has the power to cultivate deeper friendships.
instead of waiting for them to happen..
So, you know, maybe you feel like you need help yourself..
You know, you can begin with your friendship with Jesus..
He's already passionate about having a relationship with you..
He's already given himself to you..
So, you know, invite him into your need..
Let him come to you and let him fill your love tank..
And maybe some people watching are ready to take action..
You know, I would say think about how you can be the kind of friend that you want to have..
And, you know, for example, if you want to have more open, vulnerable sharing with your friend,.
think about like how can I, how can you cultivate that vulnerability within yourself?.
You know, in my own life, you know, right now in this area,.
I'm working on admitting my mistakes..
So….
I don't have any of those..
Not you, but me..
Okay, you. You're talking about you..
Carry on..
Yeah, so, you know, not only admitting them to myself and to other people,.
but letting Jesus again come into it and walk with me in it..
That's something that's been really precious to me these days..
So, and like me, like what I did with my friend,.
maybe you already have a friendship that you really want to take to the next level..
Maybe you can have a conversation with that friend to, you know, talk about what can you do?.
Can you meet more regularly?.
Can you share certain things that you haven't opened up until now and reserve judgment?.
Don't give advice, those kinds of things..
I think it's a great thought about not giving judgment, you know,.
and I think if you've got a trusted friend like that, then that's really a possibility..
You can do that and know that they're going to hold that judgment back.
and just pray for you and love on you..

$^{321}$So, hey, you know, as we've been doing each week,.
we also want to tell you about the toolkit that is available.
because everything we've been talking about today,.
we've captured the essence of this in our Spiritual Formation Toolkit..
Again, you can go to vineschurch.live/resources and you can find it there..
But in this particular section under Spiritual Friendships that we're releasing today,.
there are two segments..
There's the action steps and going deeper..
And we just want to tell you about a couple of the things that you can use this toolkit for.
immediately to deepen these friendships in your life..
In the action steps, there's this one called sharing your life story..
And this is something actually that I've made a practice with some of my friends over the years..
And I call it the one-hour walk..
So, what you do is you start in some location with your friend.
and then for half an hour, you walk in one direction..
It doesn't matter where, but you walk forward in a direction for half an hour..
And only one person speaks during that half an hour..
And that's the person sharing their life story..
And they're talking about all the things that they might want to talk about,.
about what's happened in their lives, how they've grown and developed as a human being..
And the other person is just listening..
And when you get to the 30-minute mark,.
you turn around and you walk back to where you started..
And it's the other person's time to speak..
And they get to share their life story and you get to listen..
And again, it's all about practicing active listening..
It's not about judging or making comment..
It's just about listening..
And when you get to the end of that hour, you find a coffee shop, you sit down,.
and you tell each other what you heard in their story..
And trust me, I've done this a number of times, as I've said..
And it's such a powerful thing to be able to get to know somebody deeper..
But you'll also be amazed about what people hear in your story..
And they're able often to point out things that perhaps you never would have thought of.
that God has been doing in your life..
And they're like, "Hey, did you recognize that you shared this?.
And I was so encouraged by that.".
So, the one-hour walk, I can't recommend it enough..
What else is in that action steps segment?.
Yeah, you know, we talked about being vulnerable and open and honest,.

$^{361}$and that it takes courage, right?.
So, in the toolkit, there's a link to this book called Daring Greatly by Brené Brown..
And it really shares stories and just really heartfelt stories,.
even personal stories from the author and research about openness and vulnerability..
And I can't recommend this book enough, you know..
It's really important to learn to be vulnerable, to live lives,.
and also to have relationships of significance..
We also have a teaching series that we did at the Vine a number of years ago.
on spiritual friendships..
I think it's a four or five-week series..
That's also linked in the action steps segment..
And if you haven't listened to those talks,.
there's a lot of deeper teaching, as well as practical advice.
contained in that teaching series..
So, that's something I definitely encourage you guys to check out..
Yeah, and also in the action steps,.
there's some information about how to be a safe person for another person..
And you mentioned active listening..
You know, there's some information and tips about active listening..
You know, most of us don't listen very well to other people..
And so, look at the resources to put into practice.
some of the things we've been talking about this morning..
Yeah, and finally, there's a going deeper section..
And the going deeper section really just highlights something.
that I think is so important in spiritual friendships..
Not something we've been able to talk much about today,.
but something we definitely want to encourage you to read about..
And that's boundaries..
You know, we talked a lot about self-sacrifice and giving of yourself,.
but there's also the appropriate boundaries in which you do that..
And sometimes you've got to create those boundaries.
and make sure that those boundaries are clearly communicated.
and to respect one another's boundaries and confidentiality.
in your spiritual friendship relationship..
So, in the going deeper section, we talk a little bit more about that..
Sarah, it's been so great to be able to spend this time together today..
And I know that many of you guys watching from home,.
you've been welcomed into this kind of conversation..
Sarah's really opened up a lot about her own personal life.
and how she's had to journey through this..

$^{401}$We've talked about those relationships, about love,.
about the reality of the self-sacrifice, about the courage and vulnerability..
We've shared some of those practical tips..
We've pointed you forward into the toolkit..
So, our heart is really that all of that has really spurred some inspiration in you..
You know, that's the point of doing this Flourish series here.
at the beginning of the year..
We want to inspire you to really think about.
how you're deepening your spiritual foundations in 2021.
and who you are deepening those spiritual foundations with..
And that's what spiritual friendships are..
They're choosing a few people to go deeper in your maturity with Christ,.
getting the support that you need and the love that can be encountered.
through another human being, being Christ to you.
in some of the hardest moments of your life..
So, thanks, Sarah, so much again for being here..
And I know it would be a blessing for all of us if you just spent some time.
maybe just praying over us as a church and praying into us.
as we all seek to step out a little bit deeper in these relationships..
Sure. Yeah..
Father, Abba, we are so grateful that, first of all, you love us,.
but you didn't love us from afar..
You demonstrated your love for us..
And you gave up to us the most precious thing to yourself, which was your son..
So, Lord, we really just want to open ourselves to receive from you.
this incredible love that you have for us..
And we just ask that you would, just every person in this room,.
but also watching, Lord, that you would just pour out this love.
that we can feel it tangibly in our lives, Lord..
We really can't do things without you, Lord..
We can't be the kind of friend..
We can't be the kind of spouse or mom or daughter or son.
that we want to be from within ourself, Lord..
We need you..
And so thank you for giving yourself..
We receive you..
We love you..
Lord, would you just unleash compassion and freedom in us.
as we receive from you, Lord..
Bless our relationships..

$^{441}$Bless our families, Lord..
We thank you, Lord..
In Jesus name..
\newpage



\section{}
\label{sec:M1r4SUj2xRk}
\textbf{2021-03-22 Church Everywhere Live: Flourish - Work as Worship [M1r4SUj2xRk].mp3}
\newline
\newline
連結: \href{https://youtube.com/watch?v=M1r4SUj2xRk}{\texttt{ https://youtube.com/watch?v=M1r4SUj2xRk}} ~~~~ 語音日期: 2021-03-22 
\newline
\newline
\hyperref[sec:DS7k2k5IRoo]{\small{< < < PREV SERMON < < <}}
~
\hyperref[sec:index]{\small{[返主目錄]}}
~
\hyperref[sec:q7Cbfs_x0GI]{\small{> > > NEXT SERMON > > >}}
\newline
\newline
$^{1}$A couple of Saturdays ago, I was having lunch with some friends,.
and we were talking about our weekend plans..
One of my friends said, "I get really bad Sunday blues,".
and explained that each week, as Sunday goes on,.
she gets more and more anxious about the work week ahead..
She asked, "Have you heard of Sunday blues? It's a real thing.".
I hadn't, so I looked into it,.
and I learned that it's also called the "Sunday Scaries.".
My Google search took me to the Urban Dictionary,.
which is this crowdsourced online dictionary for slang or cool words,.
and it defines Sunday blues as,.
"When it is Sunday, you have school the next day,".
presumably defined by a student..
There was a hyperlink on the word "Sunday,".
and I thought, "Why does Sunday need defining?".
So I clicked on it, and it defines Sunday this way..
"The last day of the weekend, and is usually ruined.
because of the thought of another dreaded Monday.".
Then Monday was hyperlinked,.
and it calls Monday the most dreaded day of the week..
A study done in the USA revealed that out of 1,000 people,.
more than 8 in 10 people feel more anxious on Sundays.
in anticipation of Mondays..
And of these people, 95\%, which is pretty much everyone,.
says it is related to work stress..
So this means about 8 in 10 of you who are watching this live.
or watching it on Sunday,.
you are feeling some level of anxiety about tomorrow..
And the stress isn't just felt by those who don't like work..
Even people who love their work reported a sense of anxiety..
Some said it's caused by expectations on their performance.
or caused by the high workload..
Some are stressed by a relationship with their coworkers..
Even worse, some worry whether they will lose their jobs..
And we know that this worry has just been worse by the pandemic..
Maybe you can relate to some of this..
We know as well that for some of you,.
you want to be employed, but you aren't..
And that is a huge source of stress in itself..
It's important for you to know that God cares about this so much..

$^{41}$The Bible has a lot to say about God's heart towards work.
and the purpose and meaning of work..
You know, we spend so much of our time at work..
We spend actually about half our waking hours at work..
And God wants to be part of your work..
Because if God is only relevant to you in the 10\% of your life.
when you engage with church everywhere or community group,.
but then absent from the 90\% of what you do,.
then we live a very limited and compartmentalized faith..
God is active in everything, in all areas of our life..
And work is one key area where God wants to meet you and empower you for..
What's really exciting for us here at The Vine.
is that this really is what the river is all about..
The River Vision is about everyone being empowered to encounter God.
in all aspects of our life, in everything that we do,.
every day of our lives, Monday through to Sunday..
River is about growing our personal relationship with God.
and being missional in the places where God puts us,.
where we are pouring out our talents, our gifts, our energy and time..
This is why I am excited to announce to you today.
that we are launching a faith and work ministry called the Work Life Ministry..
I'll tell you in a moment some of the plans we have for this ministry..
But first, let's take a kind of high level overview of what God thinks about work..
Is work even God's idea?.
If it is, why do we have these Sunday blues, these Sunday scaries?.
And what's the purpose of work?.
So let's define what we mean by work..
When I talk about work, what I mean is energy that is purposefully used,.
whether it is physical or mental, whether it is paid or unpaid,.
to meet the needs of yourself, others or creation..
So that applies to those of us who have a paid job,.
but it also applies to those who are involved in any unpaid or volunteer work..
The energy spent here at the Vine volunteering,.
that is indeed included in what we're talking about today, too..
It also applies to parenting, whether you're a full time stay at home mom or dad,.
or you have a job on top of parenting..
Studying is also work, so it applies to students, too..
And to understand what God says about work,.
let's start at the very first verse of the Bible..
Genesis 1.1 says, "In the beginning, God created the heavens and the earth.".

$^{81}$So the Bible begins with God at work,.
with him using his energy purposefully to create, to bring about the whole creation..
And we know that God works to create humankind,.
create us and to create relationship with us..
He works to give us everything so that we can flourish..
Later on the sixth day, Genesis 1.31 says this,.
"God saw all that he had made, and it was very good.".
It was like on the sixth day, God took a step back,.
reviewed his own work and thought, "This is not bad. Good job.".
There's a sense that his work brought him contentment,.
fulfillment, joy and satisfaction even..
Then on the seventh day, God stopped and rested..
So God himself limits himself to six days of work and one day of rest..
So there's a limit and a rhythm to working and resting that we are called to observe..
We can't just keep going, Alison..
And because God created us in his image, imago dei,.
which means we are called to reflect God, we are therefore called to work..
We are actually designed to work..
Work is in God's original design and we aren't meant to be doing nothing..
God designed work so that as we work in partnership with him,.
we feel that sense of fulfillment, satisfaction and we flourish..
So what is our work meant to achieve? What is it meant to be for?.
Genesis 1.28 says this, "Be fruitful and increase in number..
Fill the earth and subdue it..
Rule over the fish in the sea and the birds in the sky.
and over every creature that moves on the ground.".
So we are called to have dominion, to govern under God's authority..
These are words actually that describe how kings in the Old Testament are meant to govern..
So we are mandated to act as kings over creation..
We are expected to manage creation, to develop, to progress it,.
to care for it and to sustain creation..
Then in Genesis 2.15 it says that,.
"The Lord God took the man and put him in the garden of Eden to work it and take care of it.".
So we are also called to serve humbly, to work with diligence,.
to work hard so that we protect, we guard and we keep what God has already started..
The word work in Hebrew is related to the worship of God..
So if we put these two things together,.
what that means is that we are called to define the world we find ourselves in,.
our culture, our society, to take the goodness of it, to keep it, sustain it.
and to also shape it into what God would want it to be..

$^{121}$We are called to be diligent in our work, in what we do and also to care for people..
And so God invites us to work as a steward of his creation, to bring about God's shalom..
The shalom that means God's goodness, his peace, his wholeness,.
his flourishing in all areas of our lives..
That includes ourselves, our relationship with others and in also what we do..
So in work terms, what God has charged us to do through our products and services that we produce.
is that we serve others and we contribute towards the flourishing of his people, of his creation..
And we do that out of a worshipful heart towards God..
So God speaks about both the nature of work and the action of work..
And what this means is that what we do matters, how we work matters..
The way we do it matters, in the task that we do and how we treat others matter.
because it is meant to contribute to that flourishing..
In this way, good work has both extrinsic and intrinsic value..
It has extrinsic value because it brings some value to something or someone else.
and it has intrinsic value because work is good in and of itself..
That sounds amazing, doesn't it?.
It's important to know that that is God's design and his heart for work..
But we know in reality, work isn't quite like that..
Work isn't always a complete source of fulfillment and joy.
because if it is, we wouldn't have those Sunday scaries..
So why is that?.
Well, we know that the fall happened..
Adam and Eve wanted to be like God and to know both good and evil..
So they took things into their own hands, rebelled and sinned..
And what this means is that Adam and Eve's sin brings punishment from God.
and this punishment is felt everywhere and has an impact.
on the most fundamental areas of their lives, including their work..
The ground that they were called to take care of and steward.
will now produce thorns and thistles..
Work after the fall becomes more difficult, more challenging..
It becomes a place of struggle as well as a place of joy.
and that's where the stress, where the anxiety comes from..
And it's not only the work itself that becomes hard..
Our own sinfulness means that we are prone to making work.
into something that God doesn't intend for it to be..
Instead of seeing work as an avenue of service, of worship to God,.
we sometimes turn work into something that is purely for our own selfish good..
Perhaps we wish that we can avoid work altogether.
or we can crave the wealth, the power, the status that a job gives us.
and we run after those things..

$^{161}$The thing is those things in itself are not a bad thing..
A good job brings money for us..
It provides for our needs and for the needs of those who are dependent on us..
It is a way actually that God provides for us..
It's good and it's God ordained,.
but it becomes an idol if we use those things for our own advantage,.
maybe to fill our own worth or we base our identity on the wealth,.
on the power, on the status..
Or if we choose a profession that we think is far more respectable..
If as a parent you forced your children to choose certain professions over others,.
not because you think it is better suited to them,.
but because you think that that profession brings better stability.
or brings the family a higher status, it brings in more money,.
then we've taken God out of the equation of work..
Idols in our heart ultimately don't fulfill and can leave us feeling hopeless..
So it's very important to know that work itself is not the curse..
It is the curse that came from the disobedience that caused work to be difficult..
But God didn't leave us in that mess..
God's intention all along is that his kingdom would come..
We pray that in the Lord's prayer that Pastor Louise talked about a few weeks ago..
The Lord's prayer teaches us to pray these words..
"Your kingdom come, your will be done on earth as it is in heaven.".
This kingdom of God is the claim that God is the world's one true God,.
the one God who would reign and rule in a new way,.
that he would undo all the sin and the mess that we are in,.
that he would save us, rescue us, and undo all injustice in the world..
It would bring peace to every part of our world..
There would be no more hardship in what we do..
That would be gone and there will be peace in our hearts..
Shalom would be restored..
And we see in the Old Testament that God chose a special group of people,.
the Israelites, to bring about this shalom..
He chose them that through them the whole creation would experience this shalom..
However, we know that they rebelled and sinned against God too,.
just like Adam and Eve, and this shalom didn't fully return to earth through them..
So Jesus came to us and Jesus himself says that this kingdom of God had begun in him..
Jesus says this shalom would come back through him..
And what's really fascinating is that Jesus says that through him,.
life would be restored to what it was like before the fall..
There would be an ease to life, to all that we do, to our work,.

$^{201}$just like the times before the ground was cursed..
But Jesus didn't leave us with a finished kingdom..
He didn't come and fix everything, did he?.
He died on the cross and his followers were left thinking,.
"Well, what about this kingdom? We don't quite see it.".
But what Jesus did through his death and resurrection.
was that Jesus opened up a way for us to partner with God.
so that us and God together, we would bring this shalom..
And just as the nation Israel was chosen, we are all now the chosen people..
We are chosen to bring about the kingdom of God..
And one practical way God invites us to bring this about.
is through whatever area of influence he places us in..
And work is one of those areas..
In a crazy way, as we are empowered by the Holy Spirit,.
God is saying, God says to you, "You know, when you go to work,.
when you do your work, what I want to do is that through you,.
through your work, you bring my peace, you bring my shalom,.
my mercy, my justice to earth, to your team, to your company,.
to Hong Kong or wherever you're watching this,.
so that as you do that, you flourish, those around you flourish,.
your company flourishes and our city flourishes, creation flourishes again.".
That's a tremendous calling that every Christian has,.
that God would trust us and task each of us to do that,.
to bring his kingdom, his flourishing to the world..
The thing is, this isn't going to happen if we go to work.
and we keep God out of our work box..
If we leave God at church on a Sunday,.
if when you finish watching today, you say, "See you next Sunday, God.".
This can only happen if we allow him space to be with us on a Monday.
when we go to work..
When we make room for God in our work and ask for him to be with us at work,.
he will absolutely do that and meet us there..
And whenever God meets us, he molds us, he shapes us.
so that we look more and more like him..
And when we do that, we flourish and then we can bring that flourishing about..
Work then is a significant context for communion with God.
in that God is with us when we work.
and work is a vehicle for growing our relationship,.
growing in our intimacy with God, growing in our obedience and faith..
What I mean by growing in obedience is that if we take Adam and Eve.

$^{241}$and their first test in the Garden of Eden as an example,.
they were tested with whether they will trust God,.
trust what God said to them..
The serpent tested them and said, "Did God really say,.
'This is not good for you.'".
You know, the challenges that you face at work are actually challenges.
for growth in your relationship with God..
Because if you are tested at work, if you're at work and you're thinking,.
"Should I do this? Is this okay? Is this ethical?.
How should I respond to this difficult person?".
Ask for God's help and God will absolutely show you the way..
What that means is that we then have a choice..
We then have to choose whether we will choose his way or our own way,.
whether we will choose obedience or rebellion..
And God wants to guide you and empower you in your work..
He also wants to show you how he has gifted you,.
what talents and gifts he's given you,.
and he wants to strengthen you in your weakness..
In this sense, work can contribute to our sanctification..
And when seen in this way, work actually becomes an essential component.
of our spiritual growth and formation..
You may not have thought about work in this way before, but it is true..
Work is spiritual formation..
Work is a place where we get to learn to partner with God's will and purposes,.
where we get to work out our faith with fear and trembling..
Cardinal Wyszynski, a Catholic bishop, said this,.
"Work is a sacrificial surrender to God at his bidding and direction..
To accept work in all readiness, to guide it toward the heavenly Father,.
is to express oneself with one's whole soul,.
a soul that is full of trust and loving prayer.".
So just like we saw from Genesis 1,.
he says that the central purpose of work is to cooperate with God..
Christians who work with this purpose in mind.
are expressing their love and trust in God,.
using all that God has given them as a means of worshipping him..
Our workstation then becomes our worship station..
And Eugene Peterson, the late American author.
who wrote the Message version of the Bibles, says this,.
"I'm prepared to contend that the primary location.
for spiritual formation is the workplace.".

$^{281}$So what does this mean for us now?.
This is an invitation from God to us..
God is inviting us to offer our work with a heart full of trust and love.
and offer it back up to him as a sacrificial surrender..
This is about an all-of-life discipleship..
Our faith is meant to be lived out in all areas of our lives,.
outside of these walls, especially in our work..
And we shouldn't be surprised if God takes us to places.
where there is hopelessness or even chaos,.
whether that means helping someone in our jobs,.
helping someone with their finances, in their education, in their job search,.
or if someone needs legal assistance..
We can bring hope to these people through our work..
When we each do that, we become that single drop..
We usher in the kingdom of God..
We are the ones who bring hope, bring love, bring this renewal,.
flourishing and transformation to our city..
This is exciting and gives meaning and purpose to our lives..
This is why we are launching the Work Life Ministry..
So let me tell you about a few things that we have planned..
We would love for you to join us in some of these things..
First, you've already heard it mentioned earlier today.
that we are launching monthly online Lunch and Meet calls..
It's starting next Wednesday, the 24th of March..
The aim of these times is to explore together,.
how can we practically apply what God says,.
everything we've heard about today and more,.
how do we apply that in our lives?.
What does that really look like in real life?.
So once a month over a lunch break,.
we will hear work-related testimony,.
a little bit about what the Bible says about that topic,.
and then we'll have some time to talk about.
how we can do that in our lives..
Next Wednesday, our launch of our Lunch and Meet,.
we have one of our congregation members, Alison Chan,.
who is the leader in the e-commerce and retail industry,.
sharing her journey of how she discovered God's purpose for her in her work.
and how she seeks to honor God in her work..
So that is this coming Wednesday, join us..

$^{321}$We also care hugely about those who want to work but are not employed..
So we are hosting a one-off unemployment support group.
on Thursday, the 29th of April..
If this is you, this is an opportunity for you to share.
any challenges or struggles you are facing,.
be it on a practical or emotional level..
We know that it is a loss, and this will also give us an opportunity.
to listen, to come alongside you.
and hear how we can support you through this time..
So these are the two things that are happening now..
To tell you a few things that are happening later in the year,.
just very quickly, we are organizing a one-day joint church event.
on faith and work in October..
We're also looking at running a short-term community group on work..
So stay tuned for more details on these things..
It's also on my heart to gather people from the same industry,.
say we can get together those people in finance,.
or in creative, in whatever industry, we can get together..
And this time, we really want to hear from you what you think is best..
This could be a chance where you meet other Christians in the same profession,.
or it could be a chance to talk about industry-specific things,.
and a chance to pray for you and anoint you..
So I welcome your input..
Please tell me what you think would be best and most beneficial.
for fellow Christians..
I would also love to hear your stories..
What's your experience as a Christian in your workplace?.
I'd love to hear from you too, especially if you are already engaged.
in some faith and work activity in your workplace..
Is there anything we can do, that I can do to support you?.
The way to get in touch is on the toolkit..
My contact details are on the toolkit..
So let me tell you about the Connect Faith and Work toolkit.
that we are launching today..
These are some immediate first steps that you can take..
Under the action step, I point you towards a website..
We've only covered a tiny bit about what the Bible says about work..
So if that is something you would like to learn more about,.
this website, theologyofwork.org, would be a great place to start..
We would also love for you to start building spiritual communities.

$^{361}$or spiritual friendships at work..
Even if you don't meet at work physically,.
but you meet somewhere else, spiritual friendships.
with whom you can discuss your experiences at work.
is a great way to think through how you can follow Jesus in your workplace,.
especially if you're feeling isolated or lonely in your work,.
as Pastor Andrew prayed today..
Do consider starting that..
You can also dive into some daily devotions about work..
There are some suggestions on the toolkit,.
some devotions on the YouVersionBible app or on bible.com..
Then in the Going Deeper section,.
specifically for those who do want to host a group.
that explores the intersection between faith and work,.
be it at work or in a community group,.
there are five different resources that we have mentioned there..
So that's what's in the toolkit..
That's gone live..
Check it out..
Lastly, I'd love to share a little bit about my own journey with work..
In my own work over the years, I've been in Christian ministry..
I've worked in the business sector..
I grew up in Hong Kong, so my roots are here in Hong Kong..
I love Hong Kong..
And then I went to the UK for high school, university,.
and started my working life in London..
I was really settled and had no plans of moving back to Hong Kong..
And then in 2016, as I was working on setting up my own business in London,.
I heard God say completely unexpectedly,.
"Lay down London life. I'm calling you back to Hong Kong.".
Not only did I feel called to lay down my work,.
a business that I thought God had directed me towards,.
but the call was to lay down my friends, my church, my community,.
my whole adult life that I had known..
It was honestly one of the few things that God had asked me to do in my life.
that I didn't know how to say yes..
It felt so big and so overwhelming.
that I didn't know if I had the strength to do it..
I did manage to move back, and I moved back with no plans whatsoever..
I didn't know why God asked me to move back,.

$^{401}$apart from helping to take care of my parents, who are in Hong Kong..
It was terrifying..
But fast forward to five years to today,.
I am honestly humbled and honoured that I am part of this amazing church,.
an amazing team, and never did I dream that I would have the opportunity.
to be launching our work-life ministry.
and to be talking to you about something I am so passionate about..
What I'm really looking forward to is to wrestle.
through those work challenges together,.
to celebrate the victories together..
And even though work can be challenging,.
I know God is with you in your work..
God will be found in your work if you make room for Him..
My heart is to come alongside you.
so that you can be that powerful single drop.
in your office, in your workstation, in your home, in your school..
God has a purpose for you, and I can't wait.
to see how each of us will flourish.
and what God will do in our city when we work this way..
That's the hope that God has for us.
and the hope that I have for all of us..
This is audacious. It's exciting..
So yes, even as I invite you to join in on this ministry,.
I really would like to pray for us that we would each in our hearts.
individually respond to this call of God.
to make room for Him, to surrender our whole lives,.
our whole work, and to partner with Him in our work..
So let me pray..
Father, we thank you for this tremendous call,.
this tremendous invitation that you would trust us.
to bring about your kingdom, to bring about your flourishing.
to the people around us, to our work, to our city..
And God, I pray that we would respond now to you..
We would respond with that heart full of love and trust..
But I pray for those especially perhaps who do have a fear,.
who do worry that, "Can I really let go? Can I really trust God?".
I pray, God, you will show them your goodness..
We pray again for those who are feeling isolated..
If that's in the work, we pray, God, that you will bring.
those communities, those special people to them..

$^{441}$We pray for anyone, you know, if they do need to come to you.
and repent of making work something that you didn't mean it to be..
And we pray for anyone who perhaps does feel really burnt out at work,.
does feel tired..
God, may your spirit come, rest on them, and refresh them..
And lastly, we pray a blessing on each of your work..
We pray that God would anoint you as you go to work..
And so now let's take some time to respond to God..
Just really take some time to connect with God during this worship song..
\newpage



\section{}
\label{sec:q7Cbfs_x0GI}
\textbf{2021-03-30 Church Everywhere Live: Flourish - The Triumph of your Testimony [q7Cbfs\_x0GI].mp3}
\newline
\newline
連結: \href{https://youtube.com/watch?v=q7Cbfs_x0GI}{\texttt{ https://youtube.com/watch?v=q7Cbfs\_x0GI}} ~~~~ 語音日期: 2021-03-30 
\newline
\newline
\hyperref[sec:M1r4SUj2xRk]{\small{< < < PREV SERMON < < <}}
~
\hyperref[sec:index]{\small{[返主目錄]}}
~
\hyperref[sec:P6yQmGrZNeA]{\small{> > > NEXT SERMON > > >}}
\newline
\newline
$^{1}$- Thank you, Carla..
I've loved this series..
Week by week, we've received tips,.
biblical examples,.
different voices,.
but with the same agenda,.
to help us grow spiritually,.
to flourish..
I do hope you've had the chance to check out.
the Spiritual Foundation Toolkit..
It's an excellent resource.
and takes over where these sermons leave off,.
so to speak..
And I get the honor to close this series..
But I must warn you,.
I am gonna talk today,.
I'm gonna talk to you today.
about a dirty word,.
shock horror, yes..
A dirty word..
And it's this,.
evangelism..
I know you've been put off by tele-evangelists,.
over-hyped big stadium meetings.
and over-zealous Christians wearing witness t-shirts..
I saw one this week, very topical,.
given the number of mosquitoes here in Hong Kong..
And he said this,.
"I want to be so full of Christ.
that if a mosquito bites me,.
it flies away singing,.
'There is power in the blood.'".
The worldwide church has been confused..
Jesus had risen from the grave,.
was about to go to the Father.
and giving his disciples instruction.
as to how the work of the gospel would continue..
But what we've done is we've taken this verse, Mark 16, 15,.
I'll read it to you..
He said to them,.

$^{41}$"Go into all the world.
and preach the gospel to all creation.".
But what the church has done,.
has taken this verse,.
which describes the call to evangelism.
and made it seem like it only applies.
to those who we discern have a calling for it,.
those we deem to be evangelists..
And I'm here today to tell you.
that Mark 16, 15 is not a command just for evangelists,.
but for all believers..
In simple terms, you..
Sir Irankine evangelist DT Niles simply puts it this,.
"Christianity is one beggar telling another beggar.
where he found bread.".
And to demonstrate that,.
I wanna look at a passage in 2 Kings 7..
Some background..
In that story, Jerusalem had been under siege.
and the people were about to die of starvation..
Meanwhile, we read this, verse three..
They said to each other, "Why stay here until we die?.
If we say we will go into the city,.
the famine is there and we will die.".
"And if we stay here, we will die.".
"So let's go over to the camp of the Arameans and surrender.".
"If they spare us, we live.".
"If they kill us, then we die.".
Anyway, they went over to the camp.
and discovered that God had done something.
to frighten the attackers away,.
leaving food and precious possessions behind..
Let's move on to verse eight..
They ate, they drank,.
and they carried away silver, gold, and clothes,.
and went off and hid them..
They returned and entered another tent.
and took some things from it and they hid them also..
Then they said to each other, "We're not doing right.".
"This is a day of good news.

$^{81}$and we're keeping it to ourselves.".
"If we wait until the daylight, punishment will overtake us.".
"Let's go at once and report this to the royal palace.".
"This is a day of good news.
and we're keeping it to ourselves.".
We are like these beggars..
Only in our case, we are spiritual beggars.
and we have found spiritual food..
Then like the lepers, we then decide.
not to keep this good news to ourselves,.
but to share it with our starving neighbours..
And that is the simplicity of witness..
Christianity is one beggar telling another beggar.
where he found bread..
Now to demonstrate, I have chosen an unlikely assistant.
to demonstrate how to witness..
It's "Wall-E.".
"Wall-E.".
"Wall-E" was a 2008 film, an animated film,.
starring a robot who is designed to clean up.
an abandoned waste-covered earth..
"Now you know me and acronyms, so I'm gonna start with W.".
"W" is for words..
"1 Peter 3:15,.
'But in your hearts revere Christ as Lord..
Always be prepared to give an answer to everyone.
who asks you to give the reason to the hope you have.'.
This does not mean stand up at your desk and work.
and preach a sermon, unless God specifically.
asks you to do that.".
Probably unlikely..
But I wanna share with you some tips which have helped me..
I was in the business world longer than I've been a pastor..
Actually, I was a businessman almost as long.
as I've been a Christian..
I love Jesse's talk on faith and the workplace..
If you haven't heard it yet,.
I recommend you check it online..
Now the trouble with being a pastor,.
most of my time these days is spent mainly with Christians..

$^{121}$The famous inventor, Thomas Edison says this,.
"We often miss opportunity because it is dressed in overalls.
and looks like work.".
Now my first tip is this, fly your flag early..
It is more difficult to do it later on..
Now I'm gonna be honest with you..
I used to cheat a little bit..
I used to have a small cross on my jacket..
Now that's not for everyone,.
but I wanna suggest a few quick ways.
that business people can fly their flag..
To give you some practical ways you could do this.
in the office without making your colleagues feel.
they're getting bashed over their head with the Bible..
Try a conversation..
What did you do at the weekend?.
It gives you a chance to say something.
about being with Christian friends.
or a church event without being over religious..
When someone is sick or has a problem,.
how can I pray for you?.
No one says no to prayer..
And I had everyone from the tea lady to senior management.
asking for prayer for all sorts of things..
Be an encourager..
Don't get drawn into office politics,.
backbiting or negative talk..
People will soon notice a difference in you and ask you why..
Again, be gracious in sharing..
Delay in flying your flag could lead to compromise..
And then the realisation,.
oh, I didn't realise you were one of them..
Secondly, make a connection..
Find some common ground..
People like to talk about things they are interested in..
If we show an interest in the things.
that they are interested in,.
they may be more interested in what we have to say..
When I was in sales,.
the first thing I did was look at someone's office,.

$^{161}$their desk, their bookshelf for clues..
People like to talk about their hobbies, their families..
Invite them out for a coffee or a better still happy hour.
to get to know them rather than just WhatsApping them.
with the latest church invitation..
If I take someone out for lunch,.
I always start a conversation with the waiter or waitress..
Learn a few simple words of Chinese, Tagalog, Nepalese..
I particularly like Nepalese..
I go into a restaurant, I go, "Namaste.".
Oh, "Namaste.".
And then I go, "Jai Masi," which is,.
they say, "Oh, that's a Christian greeting.".
I said, "Jai, are you a Christian?".
'Cause we have a lot of Nepalese Christians at our church..
The person who you are with is by this time engaged..
I'd love to go on, but I wanna emphasise thirdly,.
look for opportunities..
The Holy Spirit will give you windows..
20 years ago, one of my team at work knocked on my door.
and asked me a question..
I didn't know at the time she was thinking of leaving..
She said, "How have you been able to stay.
at the same company for 25 years?".
I gave an answer I would not normally recommend..
I just said, "Jesus.".
And that prompted an hour or so of conversation.
at the end of which she prayed a prayer to receive Christ..
And I prayed with her to be filled with the Holy Spirit..
She came to the Vine..
Her name is Wendy Lee..
She is still very much part of the Vine..
I baptised her in Repulse Bay,.
married her in Union Church.
and dedicated her son here at the Vine..
I find aeroplane journeys a great opportunity..
Do you remember aeroplanes?.
Yeah..
You have a person's company for maybe 12 hours..
I let it slip around takeoff that I'm a pastor.

$^{201}$or maybe I'm on my way to a Christian conference.
or mission trip..
The result is either a long and interesting conversation..
Otherwise, they soon pretend to fall asleep.
and ignore me for 12 hours..
Fourthly, know your story..
No story is better than yours..
People can argue about theology.
but not your personal experience..
What I mean is this, what I was like before I met Christ,.
how I met Christ and the difference He has made..
In mission trip preparation, we do a one minute testimony..
Find a friend somewhere and practise..
We could call it the elevator pitch..
But sometimes we don't need to say anything.
as A is for actions..
Actions speak louder than words is a popular idiom..
People say, "I don't believe what you say,.
"I believe what you do.".
All talk, no action..
There is a danger when our words and actions.
don't match up, are inconsistent..
The reality of our media driven age.
is where people are saying seems to have such a big impact..
And oftentimes it's those who are keyboard warriors.
who are shaping the narrative of our cities and culture..
I think there is a good reminder for all of us.
to live less online and more in the real world..
To be more of a Christian in our everyday moments.
than pretending to be one online..
There's a common Christian expression, go the extra mile..
Jesus talked about it in Matthew 5:41..
Jackie Pullinger says this,.
"God wants us to have soft hearts and hard feet.".
The trouble is so many of us have hard hearts and soft feet..
We can be the extra mile people..
And that follows very nicely into L is for life..
People are put off Christianity,.
not by the person of Christ, but by his followers..
That is so sad..

$^{241}$Gandhi said this, "I like your Christ..
"I do not like your Christians..
"Your Christians are so unlike your Christ.".
But on the other hand, let me tell you about Joanna Chow..
Joanna Chow at the time was our Vine administrator..
More than 10 years ago was diagnosed with serious cancer..
I wish I could report that she had been radically healed..
And I know she had and has many struggles..
But the way she has conducted herself.
has led to many of her family members.
to give their life to Jesus..
And I know she is praying for the rest..
The famous American evangelist D.L. Mooney says,.
"Out of a hundred people, one will read the Bible..
"The other 99 will read the Christian.".
Our witness, our witness, our love, our actions,.
our words must be always with love..
We cannot give what we haven't received..
We receive to give away..
We're not to be like the Dead Sea,.
which feeds nothing..
But the River Jordan bringing life,.
it reflects our river model vision..
Those we witness to can never be notches on our belts.
or KPIs on our scorecards when they respond..
Paul says that without love,.
our witness is a noisy gong, a clanging cymbal..
Ask God to pour out His love..
Romans 5:5, "For those in Jerusalem, Judea, Samaria,.
"and the ends of the earth.".
After all, loving people is important,.
regardless of whether they end up.
saying the sinner's prayer or not..
There is great value in loving people,.
regardless of the surface result..
W plus A plus L plus L leads to E, evangelism..
And I repeat, evangelism is one beggar.
telling another beggar where to find bread..
And I'm gonna give you all an opportunity.
to put this into practise in a simple way..

$^{281}$I wanna introduce the alpha opportunity,.
one beggar telling another beggar where to find bread..
And my aim is this, for everyone.
who calls the Vyandale Home Church.
to invite at least one friend to Online Alpha..
And I want you right now in your mind,.
or maybe write it down, think of that friend..
And we pray in Jesus' Name, all of us,.
prayers across the city, across the world,.
for that one friend..
We pray, Lord, that You'll give us the grace.
and the opportunity to invite them to Online Alpha..
And so I've reached half time..
At this stage, I wanna recommend a stage.
of going even further, the Spiritual Foundation Toolkit..
It starts off with learning how to prepare.
your story and testimony..
It follows some resources in how to start conversations.
about your faith..
And there are some simple resources.
that you can help share with friends.
and a section on going deeper,.
a section on how to run Alpha.
and deal with difficult questions..
But as we move into the second half today,.
you, particularly those of you that know me well,.
would not expect me to give a talk on evangelism.
without at least offering some sort of challenge..
So what I wanna do is attempt to give you.
one of the best biblical examples.
on how to make a connection,.
but to use it to issue you with a personal challenge..
And the connection point is in six short words..
Will you give me a drink?.
Now I might be in trouble if I started going to stop.
to strangers in Hong Kong saying that,.
will you give me a drink?.
But in this instance, it led to one of history's.
most well-known conversion stories,.
which ended up with the words, come see a man..

$^{321}$Those words were uttered almost 2000 years ago by a woman..
She said, come and see a man.
who told me everything I ever did..
Could this be the Lord?.
Could this be the Messiah?.
John 4, 29..
She was not just any woman..
We know her today as the woman at the well..
We have no idea of a name,.
but what we do know is firstly, she was an outcast..
The Samaritans were hated by the Jews.
regarded as half breeds, pigs..
What we also know is secondly, she had a questionable past..
She had had five husbands.
and was living with a man, not a husband..
What we also know is thirdly,.
she had her own ideas about how to worship God..
In fact, for her, it wasn't about how, but where..
Samaritans believed that God had to be worshipped.
on a particular mountain..
Now I know that many of us here today.
may have felt like that in our lives,.
believing that to worship God,.
we had to go to this church, go to this place of worship,.
or behave in this way, even dress the right way..
We can maybe all identify with a Samaritan woman,.
but this lady met a man, not just any man,.
not like the men who had abused her,.
seen her as an object for lust..
This man's name was Jesus..
He had taken a detour just to be there in Sychar.
at the well to meet her..
We read that Jesus had to go through Samaria,.
but Jewish people would actually take a long detour.
to avoid going through Samaria..
Such was their hatred of each other..
So why did he have to go?.
He had to go because this woman was important to him..
And I need to tell you that Jesus is here on Vine Church.
everywhere today, specifically to meet you..

$^{361}$He is here because you are important to him..
Now back to the well,.
Jesus' disciples had gone out to buy lunch..
We read that Jesus was tired..
He was fully human and sat down by a well..
We read when a Samaritan woman came to draw water,.
Jesus said to her, "Will you give me a drink?.
Will you give me a drink?".
Now this was wrong..
Jesus, a Jewish man, talking to a Samaritan,.
talking to a woman, talking to a woman with a past.
who was probably there at midday.
to avoid being seen by the village gossips..
But you know, it was not about a drink he needed..
It was all about what she needed..
That was the connection point..
That was my Nepalese waitress conversation..
That was my time with Wendy Lee..
And we witnessed a most bizarre conversation.
about a drink that Jesus didn't really want.
from a woman who didn't think he should be asking her,.
leading to a discussion which interplays.
between the water from the well.
and this thing called living water..
Jesus said, she said to Jesus,.
"I don't know what this living water is,.
but it sounds like I should have some.".
Here at Jacob's well, after five husbands.
and living with a man, after a tragedy,.
a lifetime of tragedy and emptiness,.
disappointment and discouragement,.
sinfulness and spiritual thirst,.
she finally met a man..
Not a man who took from her, but a man who gave to her..
And what did he give to her?.
Life, eternal life, living water..
Jesus said, if she drank the water from the well,.
she would be thirsty again in no time,.
especially in the midday humidity..
I can understand that..

$^{401}$Then Jesus said, if she drank this living water,.
she would never go thirsty again..
No wonder she ran to the city and proclaimed to the men,.
the women wouldn't listen to her,.
"Come and see a man.".
Almost 2000 years later, on 28th of March, 2021,.
I say to you today, come and see a man..
I wanna introduce to you my best friend, Jesus..
You see, when this woman came to the well,.
she was a woman with a past..
As she left the well and went back to the town,.
she was a woman with a future..
You may have logged into church everywhere today.
as someone with a past..
To be honest, we've all done things.
that we know to be wrong..
The Bible tells us this very clearly..
But there is good news..
The Jesus who met the woman at the well is here..
His Holy Spirit is present and wants to meet you..
He wants to fill you..
And He wants to send you away from this meeting.
on Facebook Live or YouTube,.
not as someone with a past, but as someone with a future..
He has a plan for your life,.
no matter how good or bad you've been,.
whether you've been going to church all of your life,.
or this is the first time you've ever logged in.
to a Christian website..
He says one word..
Come..
Come as you are..
Don't wait until you think you're good enough..
You will never be good enough..
But there is just one thing..
And it's a big thing..
He wants all of you..
The Bible says, if you cling onto your life,.
you will lose it..
But if you give up your life for me, you will find it..

$^{441}$My friend, you have a choice..
Hold on to your old life..
I understand it..
You love it so much, you don't want to change..
We become comfortable..
Even in our sinful ways, we become comfortable..
But the Bible says, if we want to cling onto our past,.
our guilt, our shame, it's a no-win..
The woman of the world could have done that..
She could have resisted the free gift.
of what Jesus described as living water..
But Jesus says today, come..
If you want to be like the woman at the well,.
be a person who arrived online with a past,.
and you want to log off as a person with a future,.
I'm gonna ask you to take that small step..
Come and see a man..
I'm not gonna promise you a life.
that is free from trouble, a bed of roses..
But what I will promise you is freedom from your past,.
your wrongdoings forgiven, the love of the Father,.
the presence and the filling of the Holy Spirit.
to lead you and the gift of eternal life..
And a relationship with a God who says,.
I will never leave you..
I will never forsake you..
The next few minutes could determine.
where you spend eternity, and that's a long time,.
and how you spend the rest of your life here on earth..
And maybe not just you, but also the friends.
that each one of us is going to invite to Alpha..
If I may, can I ask you just to close your eyes.
wherever you are?.
And I'm gonna give you that opportunity now,.
the opportunity that Jesus gave to that lady at the well..
Come and see a man..
Today on this Palm Sunday,.
what a day to say goodbye to your past.
and hello to my future..
Let's pray together..

$^{481}$Father God, I come to You now as somebody who has a past..
I have done things in my life I am ashamed of..
I have done things in my life I know are wrong..
I believe that You sent Jesus to die for me upon the cross,.
to forgive my sins, to give me a hope and a future..
Lord Jesus, I give You my life now..
I'm not perfect, Lord..
I give You what I have and ask You, Lord, to take my life,.
forgive my sins, fill me with Your Spirit.
that I might receive this living water..
Help me today, Lord, not just to receive You for myself,.
but to share You with those around me..
I pray now for those I work with,.
those I travel on the bus or the MTR with,.
for my family, for my friends..
I pray one day, Lord,.
we will see them all face to face in heaven,.
looking back to the day when we said to them,.
come and see a man..
I pray this in Jesus' name..
Amen..
I realise we're not all face to face today..
I'm gonna ask you just a couple of things.
for those who responded..
If you prayed that prayer,.
particularly if you prayed it for the first time,.
please tell someone..
It doesn't matter who that person is to an extent,.
but please tell someone that you prayed that prayer..
Secondly, for those who are watching online,.
we have a chat box..
And if you just type the word, yes, Y-E-S,.
into the chat box,.
one of our pastors will spend that time with you,.
will come back and help you on the next steps.
that you have to make..
So firstly, tell someone..
Secondly, if you're online, put Y-E-S in the chat box..
And thirdly, if you're watching the recording,.
we have these connect cards.

$^{521}$and you'll see a QR code appearing.
on the screen at the moment..
Fill in that connect card.
and we would love to contact with you,.
love to invite you on Alpha..
There's lots of things in store for you..
But for all of you who prayed that prayer,.
God bless you..
This is the first day of the rest of your life..
And you leave here today,.
not as a person with a past,.
but a person with a future..
And for the rest of us,.
let's go into the whole world.
to share the good news of Jesus Christ..
Amen..
\newpage



\section{}
\label{sec:P6yQmGrZNeA}
\textbf{2021-04-19 11AM Service Live: A Different Spirit - Introduction [P6yQmGrZNeA].mp3}
\newline
\newline
連結: \href{https://youtube.com/watch?v=P6yQmGrZNeA}{\texttt{ https://youtube.com/watch?v=P6yQmGrZNeA}} ~~~~ 語音日期: 2021-04-19 
\newline
\newline
\hyperref[sec:q7Cbfs_x0GI]{\small{< < < PREV SERMON < < <}}
~
\hyperref[sec:index]{\small{[返主目錄]}}
~
\hyperref[sec:kH_AzSSAmbg]{\small{> > > NEXT SERMON > > >}}
\newline
\newline
$^{1}$Grateful for this moment to gather both here in the room and online, Lord..
Grateful that your presence is at work and moving amongst us..
Lord, we pray for every home right now that is watching this live feed..
Lord, we thank you so much for the people that are gathered in communities around the.
world and here in Hong Kong..
Lord, we thank you that our church is not just contained to these four walls..
And so we just lift up prayer for each home, for each story, for each person..
Father we pray that you would reach them right now..
Lord, that we are so grateful that we get to be a community in person, online, together,.
one unified body of Christ..
And Father, I thank you for each person in this room, for their story, for their journey,.
for what's happening in their lives right now..
Lord, we've loved your presence with us..
But as we open your word, we know that we are therefore opening your presence at a deeper.
level..
Because you say that you are the logos, you are the word..
So as we open your scriptures, as we open story, as we open up principles of theology,.
Lord, would you be even more present with us..
And Lord, we celebrate that, we're excited about that, and we pray this in Jesus' name..
Everyone says?.
Amen..
Let's give it up for the worship team for, oh man..
Have a seat, have a seat..
Well what an amazing time..
I'm so excited..
Real human people in the room..
Hi..
How are you all?.
Are you guys good?.
It's so good to see you..
Welcome to the Upper House..
How are you guys?.
Everybody doing all right up there?.
Beautiful..
You guys look amazing..
Socially spread out and amazing..
I love that..
That's beautiful..
I'm carrying in my spirit something that I'm really excited to preach to you guys today.
about..

$^{41}$I've been saying this so many times, I'm going to keep saying it..
There is never, I think in the whole of the history of Hong Kong, never a more important.
time to be Christian in our city than right now..
I'm so excited about where we are as the church and what is ahead of us in the years ahead.
for the gospel in our city..
No more important time for us to lean into the reality of what it is that we've been.
saved for..
What it is that we get to gather in this room and sing praises for..
We have to ask ourselves, is it just for us?.
Is what we're doing here just for this moment of 90 minutes?.
Or does it have something to say to everything that's taking place in not just our city in.
Hong Kong, but around the world right now?.
I believe that the Christian faith is the most profoundly important thing for this world.
in this hour..
And I want to open up for us today as we start this new series and over the next seven weeks,.
what I think it is to live the Christian faith in this moment..
I want to tell you a little bit about your story..
I want to talk to you about the gospel and about what it is that you've been saved for..
I want to challenge you also over the next seven weeks to rise up, to begin to be a voice.
of God's gospel in everything that is happening around us..
I want to call you to be different..
Are you ready?.
Let me start with a story..
Twenty years ago, twenty years ago..
I'm old..
It's always dangerous when you start a story with twenty years ago..
Twenty years ago, I led a small group of teenagers of youth from the Vine on a missions trip.
to the slums of Manila..
We were partnering with YWAM, visiting this particular part of metro Manila..
It's not there anymore, but in those days, twenty years ago, there was this place called.
Smoky Mountain..
I don't know if you've ever heard of Smoky Mountain..
It was famous around the world as being the world's largest rubbish dump..
It was right there in the middle of metro Manila..
It looked exactly like this..
This huge rubbish tip had four thousand people that lived on it and made their lives from.
it, scavenging on the rubbish tip day in, day out, gathering things up so they could.
take it and sell it and make a livelihood..
Our mission for a week, partnering with YWAM, was to go and do outreach to the four thousand.
people that were on this rubbish tip..

$^{81}$There was this basketball court, a couple of basketball courts right by the side of.
Smoky Mountain..
Now, you know in the Philippines, everything starts around basketball, right?.
There was these basketball courts, and we would go and we would hold outreach on these.
basketball courts for the people that were working and living, the children, the families.
on Smoky Mountain..
Well, on this one particular day, we thought that traffic in Manila was going to be really.
bad..
Who's faced traffic in Manila?.
You know it's bad, the worst in the world..
We thought it was going to be really bad, and so we went super early, but I guess God.
liked the Red Sea, parted the traffic for us..
And we got there 45 minutes early..
And we all got out of the van, there's me and like 10 kids, and I'm like the youth pastor.
going, "What are we going to do for 45 minutes with all these youth that are kind of just.
like, you know, wondering what's next?".
And I look up and I notice that around Smoky Mountain, right where we were by the basketball.
courts was this temporary housing unit..
And in this temporary housing unit, there were five or six homes that were on these.
rickety kind of bamboo structures..
And so I got this idea, "Okay, why don't we go up and we'll knock on the doors of these.
families and we'll go in there and we'll introduce ourselves and we'll share a little bit and.
we'll pray for them as families.".
We thought this was a great idea, so I got all the youth and they were super excited..
We walked up these rickety staircases and we knocked on the first door and we went in.
and we're like, "Hi, we're from Hong Kong.".
You know, and this family was like, "What is going on?".
You know, and then we would tell them a little bit about Jesus and we would pray for them.
as a family..
We'd go out and we'd knock on the next door and we kept on doing this..
And by the time we got to the last door, we had about seven minutes left before we were.
supposed to be on the basketball court to do our outreach..
And I said to the team, "Okay, guys, we don't have a lot of time with this family..
We're just going to knock on the door..
We're going to go in..
We're going to say hi..
We're going to pray real quick and then we're going to funnel out and get on with the outreach.
downstairs.".
And everybody was ready for it..
We opened the door, we went in..

$^{121}$As soon as we stepped into this last room, we recognized that this was different from.
all the other ones..
It was immediately just darker and dingier..
There was like this curtain that was pulled across the window, the only window in the.
hut..
And as we walked up, we found and discovered this old couple, probably in their mid-80s,.
sitting there on a bed..
And we noticed the man straight away because he had these two milky white type cataracts.
over his eyes..
And we could tell that he was blind..
And so we said through an interpreter, "Can you tell us your story?".
And he said, "Well, I've lived here on Smoky Mountain for 15 years..
And during that time, because of the way that the sun comes and combusts all the things.
on Smoky Mountain, all of the smoke has ruined my eyesight and I can no longer see..
I've got two full cataracts on my eyes from all of that smoke for all of those years and.
I can't see anymore.".
And we had this real compassion for him..
And I said, "Okay, guys, well, let's just pray for this family..
Let's pray for God's blessing.".
And so we kind of stretched our hands out..
I was doing the kind of pray, check my watch prayer because I needed to get down for this.
outreach..
And we finished the prayer and I started to funnel the kids out real quick..
And as all the kids go out, the last girl in our youth ministry, she was 14 years old..
And she stands in the door as I'm trying to usher her out of the house..
And she says to me, "I believe God wants to heal this man.".
Uh-oh..
This is like every youth pastor's worst nightmare..
This is like a lose-lose situation, right?.
Because if I don't pretend to have as much faith as she's got, I look like the loser,.
right?.
Equally, if we go in there, I know that on the whole, I've experienced, I know that God's.
probably not going to heal this person..
I know that's probably not going to happen..
And so therefore, I'm going to have to spend the rest of the afternoon trying to describe.
theologically to these kids why God doesn't always heal when we have faith..
And I'm going through all of this and she's standing there with these big doe eyes going,.
"I believe God can heal this man.".
I'm looking at my watch and I'm going, "All right, everybody back in the room, back in.
the room..

$^{161}$Come on, come on quickly..
Okay, okay, just say a prayer for him..
Come on, let's go.".
You know?.
So she gathers around and we all gather around and she puts her little hand on this guy's.
shoulder and she offers the simplest prayer..
I don't remember exactly what it was..
It was something like, "God, would you just heal this man?".
And what happened next is kind of hard to describe..
The best way I could describe it is, it's a little bit like if you've got a car and.
your windshield of your car is completely dirty and you turn on the windscreen wipers.
and it goes like this and cleans all that dirt away..
Like literally as I'm staring at this man, the cataracts in his eyes, the milky whiteness.
just is wiped away..
And I'm like, "Oh, I don't believe what has just happened.".
And this is a real gritty, real life miracle..
Not like ones that we read in the Bible..
This is really happening in this moment..
And because it's a real miracle, this man has not had light enter his eyes for about.
15 years..
So suddenly all this light is flooding into his system..
It freaks him out..
He goes, "Ahhhh!".
And he grabs his eyes and he starts screaming as loud as he can..
And his wife starts screaming right next to him..
The kids start screaming..
I'm screaming as well..
It's a scream fest..
And this man's like, "Ahhhh!".
Like this..
And then slowly, he's got his hands over his eyes and then he lets them down slowly and.
he's squinting and he's in like a little bit of pain and he's singing..
And he starts to count us in the room..
And his wife starts to weep and weep and we're all weeping and weeping..
And I'm like, "Forget the outreach.".
Like God's kingdom has come on earth as it is in heaven..
It's literally invaded this moment and in this life..
And then God has turned and changed and transformed this man..
And I'm there and I realize it's all because some 14-year-old girl had something different.
in her..

$^{201}$Something different from her youth pastor..
That she was carrying something in her that set her apart..
That she could stare deformity in its face and go, "I think I know a God who can do something.
about that.".
That she might carry something in her that was so different from the rest of us that.
God honors the reality that she was courageous enough to declare the promises of God when.
all I wanted to do was get to a basketball court..
This series that we're starting today, I believe is the central word for what God has for the.
Vine in this hour that we're in..
There is no more important time for us to begin to realize, to remember, to celebrate.
that we have something in us..
That means that we are different to the things that are around us in this world..
If you think about all the things that are happening right now in our city, I believe.
that we are in an inflection point for the gospel right now in Hong Kong..
That I think we're literally on a precipice where we're either going to move forward as.
the church in this city, holding the gospel loud and proud, or we're going to move back.
and we're going to lose a generation to Jesus..
I think we're at an inflection point because of everything that's taking place pandemic.
and politically right now..
This city is still hurting..
This city is still trying to work out what its identity is moving forward..
There are people in this city that are scared and fearful..
There are people in this city who are wondering, "Is it time to pack up the bags and go?".
Perhaps some of us are wrestling with that in this room right now..
There are people in this city that are hurting and still confused and wondering what's next..
I think the church has to ask itself a really important question..
Will we, the church in Hong Kong, become a bold and central voice of hope and faith and.
life and love in this time for the years ahead?.
Or will we feebly shrink back into the shadows of our own self-concern and self-preservation?.
Will the church in Hong Kong stand up and stand out?.
Will it shrink back and fade out?.
Will the church begin to actually live out the gospel the way that it was always designed.
to be?.
Or will we settle for a gospel that keeps the lights on in here but shuts down our light.
out there?.
Come on, church..
There's an inflection point right now in our city..
Let me tell you this..
Jesus didn't die on the cross so that we could enjoy a nice Christian holy huddle..
Jesus didn't die on the cross so that Christians would just know comfort and safety and security..

$^{241}$I want to misquote C.S. Lewis here..
I don't know if that's a thing, but I'm going to do it..
I'm going to misquote him by saying this..
The Christian faith is good, but it's not safe..
Jesus died on the cross not for our comfort..
Jesus died on the cross to create a new humanity, to begin the starting point of a new story,.
to create something within the mess and the brokenness that is different from it, that.
brings a hope, that brings a story and a narrative that begins to say something to that brokenness.
around..
Jesus' death and resurrection was a declaration that God's not getting rid of all of the.
mess, but he's starting afresh within it..
Paul would speak of Jesus as the new Adam, the starting point of a new story, of a new.
invitation, and that Jesus himself would say in all of his teachings before he got to the.
cross, "I am here to start, to bring, to declare the kingdom of God.".
In other words, my death on the cross is going to create what I like to call an alternative.
society, something about a new group of people that live within the rest of society, don't.
separate from it, don't go to be a monastery on a hill or birth within it, but are different.
to it..
I believe God is going to heal this man..
Jesus spoke about this alternative society a lot, the kingdom of God, the reign of God.
on earth..
In Matthew chapter 13, he tells us a bunch of stories about it..
My favorite is this, the kingdom of God, this thing that Jesus died and rose again for,.
the thing that we get to call ourselves Christian around, it's like this..
It's like a little piece of yeast being placed in a large amount of dough..
It's very small, but because it's in this large amount of dough and because it's different.
from that dough, when it gets kneaded over time, over a long period of time, guess what?.
Something so small can actually influence the whole dough and literally transform it.
into bread..
Jesus is saying, "Do you want to know what this kingdom is like?.
Why I'm dying and rising again?.
Just so that there would be something different in you..
So you might be like a piece of yeast, placed in your school or in your family or in your.
workplace or in your city..
And over a period of time, God can take our differentness and begin to work and influence.
in society..
The problem is, I think the church for so long, and particularly the modern church,.
has misinterpreted that parable and they said, "Oh, the church is like dough in dough.".
I mean, if we want to be relevant in this world, well, we just need to look a little.
bit more like the world..

$^{281}$We need to do things a little bit like them, but even better than them..
It means we need to imitate and copy what's happening in the world, but slap a Christian.
label on it..
We've come to think that our faith is dough in dough..
But Jesus is like, "Be different..
That I'm starting in you a different story..
Not be weird.".
Us Christians have struggled with this one as well..
He didn't say, "Be weird.".
We've all got the uncle, haven't we, in our family, right?.
You know, the weird uncle, right?.
That like no one knows what to do with and comes to the annual family barbecue and stands.
in the corner and smells a little bit, right?.
That's not us, okay?.
We're not supposed to be weird, but we are supposed to carry something different..
What is the different thing that we carry?.
Jesus' hope, Jesus' forgiveness, Jesus' love..
As we begin to love one another, we begin to show the world what that love looks like..
As we carry the hope that we've been redeemed, we realize it's not just for us, but it's.
for everyone else..
And we begin to speak of that hope boldly and courageously so that others might know.
the character and the person of God..
The difference is not us..
The difference is Jesus..
And we get to make Jesus known..
I don't want to do anything else with my life but that..
The great powerful audacity of the Christian faith is that through a relationship with.
Christ Jesus, we get to declare to the world that we are yeast in dough, a light in the.
darkness, a city on a hill..
And we get to live out the realities of what that looks like..
What does it mean for us to be different in Hong Kong in this time where there is fear,.
we might move in courage..
Where there is uncertainty and concern, we might declare faith..
Where there is slavery, we might break chains..
That we might become the church that God died for..
I think we're at an inflection point where as a church we get to decide if that's what.
we want or not..
And whether we're going to have our voice be heard..
There's a lot of stories in Scripture around holding this different type of approach..
I call it a different spirit..

$^{321}$What I want to challenge us in over the next seven weeks is what it is to hold a different.
spirit..
Again, not to be weird or strange or different for different sake, but to realize that in.
the story of the gospel and in the power of the spirit alive in us, there is something.
different that we carry that has something to say of hope in the world around us..
This different spirit is throughout Scripture, Old and New Testament..
What we're going to do over these seven weeks is we're going to park ourselves in one story.
in the Old Testament..
It's one of the most famous stories of the Old Testament..
It's found in Numbers chapter 13 and 14..
It's the story of the 12 spies being sent out to spy Canaan, the promised land that.
had been promised to them..
And what we're going to do is we're going to go week by week through this story together,.
and I'm going to show you what a different spirit looks like because I want to empower.
us and release us to live that different spirit in our city today..
Let me give you some context to this story as an introduction this morning..
We know that God has released Israel from their slavery in Egypt, and He's done it through.
compassion..
He's done it miraculously..
He's parted the waters..
He's brought them out, and they're now in their freedom..
But freedom is not an easy thing..
Freedom has responsibilities..
And in that whole journey of slavery into freedom, God's been giving them one story.
the whole time..
There's an inheritance for you..
There's a land that I'm leading you to..
There's a place where I'm going to be moving you into, a place that flows with milk and.
honey, a land that is a gift of God's people, that it's going to be a place for you to grow.
in, to learn in, to flourish, a safe environment where you get to show the world what God looks.
like..
There's a land for you..
And this dream and this vision and this promise of an inheritance of a land keeps Israel going..
And now in Numbers 13 and 14, they're right at the Jordan River, and they can see the.
land just right there..
And Moses does something really smart..
Rather than just suddenly invading the land, he gets 12 of the best men from the different.
12 tribes of Israel..
And he says to them, "I want you to go into the land and spy it out and come back and.
give us a report..

$^{361}$Is this actually the land that God's promised to us?.
What is it like?".
And so these 12 go out, and for 40 days, they spy out the land, they come back, and they.
give a report to Israel..
And the report essentially is twofold..
Number one, it is an amazing land..
It does flow with milk and honey..
There's incredible fruit in this land, and that fruit is good..
Look, we've even brought some of that fruit back to show you this is an amazing place..
What an inheritance..
And then the second part of their report is, there's a but, right?.
You know you're in trouble when there's that..
And they say, "But there's giants in that land..
I mean, there's obstacles..
There's fortified cities..
There's stuff that's there in that land..
We have to get to outreach in just a few moments of basketball court..
There's distractions..
There's things there that will stop us from moving in.".
And they begin to give this report that this is the end, that we've come all this way..
Why has God let us out of slavery in Egypt and brought us here to be killed by the giants.
in the promised land?.
And fear begins to spread in the camp..
In this time of our society and culture, in our world right now, fear spreads in the camp.
because of bad reports..
I shudder to use the word fake news, but you're with me, right?.
Opinions..
Everybody with a keyboard telling us what we should believe and think..
Everybody telling us whether this virus is good or not, the vaccine is good or not..
All these obstacles in communication and information, and we get swamped by it, and we get information.
fatigue, and it can lead us very easily into a place where we're no longer listening to.
the promises of God, and we're absorbing all of the commentary of the world..
Two spies, though, are willing to say to Moses, "I believe God could heal this man.".
Two spies stand up and say, "Hm, hang on a sec..
I mean, if this is a promise of God, if this is His land that He's giving to us, then I.
think we can take this land..
Yeah, we see the obstacles, but you know what?.
Our God is bigger than that.".
There are two that begin to give a different voice to the situation, begin to stand up.
and say, "No, actually, we cannot forget that God is at work..

$^{401}$We cannot forget..
We see the obstacles..
We see the giants, but there's something bigger than the giants..
There's the God behind them, and that God is way more powerful, and if we could just.
believe that God could do this, then everything could...".
And they begin to say, two voices out of the 12, that God is still with us..
God sees all this happening, and He's like, "Okay.".
And He separates the 10 from the 2..
And God says to the 10, "Because you are unfaithful, because you fail to believe, none of your.
descendants will ever enter the inheritance and the promise that I've given you.".
And He turns to the 2, and He says, "Because of you two, your descendants, your people.
will inherit the promises of God.".
I believe that Hong Kong as a city is God's promised land..
I believe God has Hong Kong as an inheritance, that the gospel is still ripe for work in.
this city..
Are we going to be the 10, or are we going to be the 2?.
Here's how God puts it in Numbers 14..
Is everybody okay?.
You still like me?.
Good..
Verse 21, it says this, "Surely as I live," this is God speaking, "and as surely as the.
glory of the Lord fills the whole earth," notice that, the whole earth, right?.
Including the promised land where the giants are..
"Not one of the men who saw my glory and the miraculous signs I performed in Egypt, in.
the desert, but who disobeyed me and tested me 10 times, not one of them will ever see.
the land I promised on the oath to their forefathers..
No one has treated me with contempt will ever see it..
But because my servant Caleb has a different spirit and follows me wholeheartedly, I will.
bring him into the land that he went to and his descendants will inherit it because this.
one has a different spirit..
This one carries something of my character..
This one believes in my promises above what he sees in the world..
This one and all of his descendants and his family will walk into that inheritance..
This one.".
Why?.
Because he carried a different spirit..
Caleb allowed his incredible vision of God and his understanding of God's power and his.
promises to filter into his thoughts and actions and his public expression of his faith..
And God saw that and believed it was very, very good..
Over the next seven weeks, I'm going to teach you about what I think this different spirit.

$^{441}$is all about..
But I want to start today by telling you what the different spirit is not..
Are you ready for this?.
A different spirit is not manufactured by the flesh..
A different spirit is not something that we work up in us with some nice music or some.
rah-rah or some shouting..
A different spirit is not something created in us..
It is not our inner skills or our inner strength or our personal abilities..
It is not the thing that I do that I do really well..
My different spirit is not the strength that I have, even if that strength has come from.
my years of walking with Jesus..
It's got nothing to do with the person or the gifts or the abilities of me..
If that was the case, then when God showed up to these 12 people, he would have said.
they all had a different spirit..
Because the 10 spies were the strongest, most courageous, bravest, best leaders of their.
tribes..
That was why they were chosen to go into the promised land..
But God doesn't say that..
He says over here, these two have something different to the rest..
So a different spirit has nothing to do with our abilities..
Can I have an amen?.
A different spirit also is not a type A personality..
A different spirit is not being extroverted..
Can I have an amen, introverts?.
Now I might be an extrovert myself, but the reality is a different spirit is not about.
your ability to communicate, your ability to be passionate, how loud you raise your.
voice, how comfortable you are in social settings, whether you're a good leader or not..
It's not about a personality profile..
The Bible is filled with introverts that change the world..
As hard as that is for me to say, it is true..
Introverts everywhere in Scripture do amazing things..
Moses, I can't go before Pharaoh because I stutter..
I have no voice..
Daniel, introvert..
Amos, Barnabas, Timothy, Jude, John, ones that would prefer the quieter life are used.
so profoundly and powerfully in God's story..
In fact, I would argue there are more introverts at work in Scripture than extroverts..
So a different spirit has absolutely nothing to do with your personality profile..
Are you with me?.
Here's the third thing..

$^{481}$A different spirit is not about moral perfection or your amazing holiness..
Now this is really important because I think as a church we can get foiled into this trap.
quite a lot..
You're hearing me today and you're hearing what I'm saying and you're thinking, "Okay,.
I get it..
I get that God can use other people of different abilities..
I get that God can use other people of different kind of personality traits, but you don't.
know my story, Andrew..
I'm an alcoholic..
I struggle with lust issues..
I'm cheating on my spouse right now..
I have anger issues that I can't control..
I lie on a regular basis.".
I mean, if you knew all the things that were happening in my life, you would realize that.
there's no different spirit in me..
You'd realize that that's for the other people in church..
I'm just going to come on a Sunday and just kind of slip in and slip out because I don't.
think I'm worthy of being a part of this grand idea of church that you're talking about..
I want to speak to you if you're feeling a little bit like that today..
I want you to know that there's nothing in this story..
Yes, God says to Caleb, "Well, he followed me wholeheartedly.".
There was a sense of Caleb's devotion and worship to God, but there's nothing in this.
story about how perfectly moral and ethical Caleb was..
In fact, I can tell you that probably Caleb was a man who struggled with stuff..
Caleb probably wasn't perfect..
Joshua definitely, as we will find out in his journey, was not perfect..
David, a man after God's heart, was a murderer, a rapist, and an adulterer..
When we see all of this in Scripture, it's there to remind us that even despite our brokenness,.
despite our sin, we can still be a part of God's alternative society..
In fact, actually, maybe our struggles are a part of the jars of clay, this broken and.
brittleness that actually when we admit the reality of our sin and our struggle, we release.
something of the power of God's grace..
Maybe a different spirit is at work despite the church's sin..
Now, listen to me..
I'm not soft on sin..
The Holy Spirit wants us to operate and to walk in a sanctification process..
I praise God that I'm not the sinner I was 10 years ago, 5 years ago, last month..
But I also recognize that I still carry sin in my life, and I'm on a journey of sanctification..
My heart is supple to the works of the Spirit..
Here's the reality..

$^{521}$Your struggle and your sin does not give you an excuse to opt out of God's alternative.
society from holding a different spirit..
It's not about your moral perfection..
It's about God's glory despite it..
See a different spirit in the New Testament, if we center in on Jesus for a moment, it.
was all about this idea that He is everything, that the early church would be able to live.
the way they did in the empire around them by simply understanding that they have a God.
who is like this, one who was willing to come to the earth and die with it, one who was.
willing to resurrect around it, one who then returned to show us and gave us a spirit of.
life that that one is what they now stood up for..
So Peter radically transformed, could stand before his people of his day and be a different.
spirit..
Why?.
Just because suddenly, you know, Peter's got all these new gifts and abilities because.
Peter is suddenly not a sinner?.
No, because Peter realizes it's not about me..
It's about Jesus..
Because it's about Jesus, even in my brokenness, I can be a voice..
I believe God can heal this man..
I wonder whether we would be willing over the next seven weeks to stand over our city,.
join hands together and declare, we believe God could heal this city..
That's a different spirit..
Could you stand with me?.
I want to pray for you..
I wonder whether you just open your hands with me as we pray..
Father, we come before you as a church, grateful for this invitation in this series to reflecting.
and thinking about what it is to live for Christ in this hour and in this time..
Father, we stand before you now, open hearts, open hands..
Father, some of us in this room, we rely far too much on our strengths and our abilities..
Some of us in this room think we're disqualified because we're not a certain personality type.
or profile..
Some of us in this room think we're disqualified because there's that sin that we continue.
to wrestle with..
Some of us think that if we could just try a bit harder, embrace some work a bit more.
and be driven by the law, we think we might get there, but the law will enslave us..
It sets us free..
And so we come, Lord, at the start of this series with our hands open before you..
Lord, I believe the Vine wants to be a church with a different spirit now..
I believe in this inflection point of the gospel you're calling this church to stand.
up and stand out, to be a church that's willing to declare the gospel, the saving knowledge.

$^{561}$of Jesus Christ to a hurting and broken world..
And Father, in all of our brokenness and all of our hope and strength, we now come to you..
We ask that your spirit would light a fire in us, and that as we journey in this series.
together, we would learn through the examples of Caleb and Joshua, all of the things that.
you're saying to us in this hour..
Make us, Lord, ones with a different spirit..
We pray this in Jesus' name..
Amen..
(gentle music).
[MUSIC].
\newpage



\section{}
\label{sec:kH_AzSSAmbg}
\textbf{2021-04-25 11AM Service Live: A Different Spirit - What do you see? [kH-AzSSAmbg].mp3}
\newline
\newline
連結: \href{https://youtube.com/watch?v=kH-AzSSAmbg}{\texttt{ https://youtube.com/watch?v=kH-AzSSAmbg}} ~~~~ 語音日期: 2021-04-25 
\newline
\newline
\hyperref[sec:P6yQmGrZNeA]{\small{< < < PREV SERMON < < <}}
~
\hyperref[sec:index]{\small{[返主目錄]}}
~
\hyperref[sec:vtinqnnv5SU]{\small{> > > NEXT SERMON > > >}}
\newline
\newline
$^{1}$Grateful that we don't just sing songs, but we encounter the risen Lord..
Grateful that in this room, in this moment, you're here by your presence and everyone.
watching right now, wherever they are in their homes, your presence is at work..
But we recognize with all the things that are happening in our world right now,.
you still count us worthy to visit us with your presence..
You are Emmanuel, the incarnation at work in this moment..
And we're so grateful. Lord, I want to pray as we open your word. Lord, we need your word, Lord..
Church, we just declare, don't we, that we're hungry for his presence.
and that we need his word, that we're thirsty. Lord, we're tired of hearing what the world has.
to say. Lord, we're tired of hearing what our social media feeds are telling us..
Father, we want to know what you have to say..
Lord, I want to pray as we open your word today, Father, you would open our eyes..
Father, we don't want to be blind to the realities of what you're doing in this hour and in this.
moment. Lord, we want to carry a different spirit in us. So Father, would you bring your word.
freshly to us? And Father, as we do so, I want to pray for liberation, freedom..
I feel just before I preach today, I feel some of you are carrying chains of fear..
And the shadow, it's almost like for some of you, you feel like you're living in a bit of a shadow.
right now. You've known what it is to live in the sunlight, but right now it feels like you're in.
a bit of a shadow. For some of us, the enemy is casting a big, broad shadow over us. I want to.
pray, Holy Spirit, that as we open the word today, that my words would fall away and your word.
would stick. I want to pray for the ministry of the Holy Spirit, for breaking chains, for bringing.
us out of shadows and bringing us into the light. For you, revive and refresh and renew your people..
So for every person in this room, everyone watching online, if anyone needs to be renewed.
by the refreshing of your word, would it come in this moment? We thank you for this in Jesus' name..
Everyone says, amen. Amen. Can we thank the worship team? Thanks guys so much..
Why don't you have a seat, have a seat. Welcome to everyone online. We're so glad that you guys.
are joining us from wherever you are. We know every single week we've got people from all over.
the world that are tuning in and connecting with us, and we are grateful for you. We know that you.
are a part of the Vine family, and at 30\% right now, we can't get everybody in this room. You are.
with us, and so welcome. We're so glad. And welcome everyone else in the room right now. It is so good.
to see you. This is 30\% of us. Have a look around the room. This is what 30\% kind of feels.
and looks like. I'm longing for 50\%, 75\%, 100\%, 130\%. But we do not serve a 30\% God in the name.
of Jesus. Can I have an amen? Like we are not expecting a 30\% spirit in this moment, right?.
I'm not going to preach a 30\% word to you. All right? You guys look like you're asleep already..
That's okay. That's all right. That's all right. It's going to happen. Hey, so last week, we started.
a brand new series here at the Vine, and I started last week by saying that I think this is the.
central word that I think God has for us as a church in the season that we're in. I said last.
week boldly that I feel like we're in an inflection point for the gospel in this season of our city,.
in this hour that we're in right now, that I think the church in the next couple of years has a.
decision to make in Hong Kong. Are we going to move forward with the gospel? Are we going to hold.

$^{41}$forward the hope of Jesus? Are we going to find ourselves finding life and boldness in the story.
and the narrative of the death and resurrection of Jesus? Or are we going to find ourselves.
shrinking back and fading out into the sidelines? The inflection point is the reality that if we.
don't step forward, I feel like we could lose a generation for Jesus in this hour, in this moment,.
in our city. I feel in my spirit a passion about that. I don't want to see us lose a generation.
in Hong Kong. And I challenged us last week that we need to think about how we as Christians are.
living the gospel in this important time. I mean, are we going to live the gospel as I think it's.
truly designed to be lived? Or are we going to settle for a gospel that keeps the lights on in.
here but shuts down the light out there? Like, are we going to stand forward and stand out? Are we.
going to shrink back? And are we going to fade out? And I argued last week that the difference for the.
church is that we would carry in us what we're calling a different spirit. That there might be.
something different in us than what we see in the world around us. That we wouldn't be weird..
Hear me, Christian. I'm seeing you. That we wouldn't be weird and strange, but that we would be.
connected to the beauty of our city and its society, but to do so by offering something.
different from what's around us. We talked about last week that we are yeast and dough, not dough.
and dough. That in our workplaces, people might look at us and go, "How come that person has joy.
despite the fears we're all sharing about our future? How come that one understands what.
forgiveness is when I can't even forgive my wife? How can this one have so much hope when there.
seems to be so much bad news? What is it about this one that attracts me to them?" That the church.
will become a place where people might be drawn towards it. Not because we're trying to be relevant.
and cool to everybody else around us, but because we're carrying something in us of the death and.
resurrection of Jesus. And we know that this different spirit means that we're invited to.
living a new humanity. That actually the Christian faith is the only faith that declares that we are.
made new. And therefore, because we're made new, we're released into the beauty of being the human.
that we had always been designed to be. And that maybe the call of the church is to stand up in.
this hour and say, "It should look a little bit like forgiveness and love and mercy and grace.
and power and transformation and healing and restoration and reconciliation and all of these.
things." That's the way we were designed to be. So last week, I took us to a story in the Old.
Testament, Numbers 13 and 14, where we see Moses send out the 12 spies into the promised land..
And the spies go out and they see the land, the land that God had promised them to receive. And.
I believe Hong Kong is our promised land. And they go out into that promised land, all the.
inheritances that God had for them. And they come back and they bring a report and 10 of them say,.
"We cannot conquer the inheritance God has provided for us." And two of them say, "No, we should,.
for we can." And Jesus looks at those two, Caleb and Joshua, and he looks at the 10 and he says,.
"These two carry in them a different spirit. And because they carry in them a different spirit,.
I'm going to release them and their generations to conquer the inheritance before them. But these 10,.
because they went into the promised land and did not see what I had put aside for them to see,.
and these 10 were so overshadowed by the work of the enemy, these 10 will not enter." And I think.
we as a church are standing in a moment where will we have a different spirit and inherit the land.
given to us, or will we find ourselves shrinking and dying in the wilderness? That's what this.

$^{81}$series is all about. We're going to take the next, this week and the next five to tell you about what.
a different spirit is all about through the life of Caleb and Joshua. And I want to start today.
with a question for you, and it's this, what do you see? Like, what do you really see?.
Like right now here in our culture and society and our city, right now here and what's going on in.
your life personally? What is it that you actually see? Because here's the reality,.
what you see and how you interpret it will shape your reality in the world that you live in..
What you see and how you interpret it, how you come to perceive it and think about it will shape.
your reality and therefore the world in which you live. And I think the Holy Spirit is asking us.
as a church, will we look out into Hong Kong in this time, in this hour, and will we see.
what everybody else is seeing, or will we see something different? So let me ask you again,.
what is it that you see? I want to open up the thought around what we should be seeing.
in our promised land by taking us to the beginning of this story in Numbers 13 this morning. Let me.
read this to you. It says, "The Lord said to Moses, 'Send some men to explore the land of Canaan,.
which I'm giving to the Israelites. From each of the ancestral tribes, send one of its leaders.'".
And then from verse 3 onwards, it has a whole list of a bunch of leaders whose names I cannot.
pronounce and I will skip over. Then in verse 17, "When Moses sent them to explore Canaan,.
he said this," listen to this, "Go up through the Negev and onto the hill country. See what.
the land is like and whether the people who live there are strong or weak, or few or many. What.
kind of land do they live in? Is it good or bad? What kind of towns do they live in? Are they.
unwalled or fortified? How is the soil? Is it fertile or is it poor? Are there trees on it or.
not? Do your best to bring back some of the fruit from that land, for it was the season for the.
first ripe grapes. So they went up and they explored the land from the desert of Zin as far as Rohab,.
to the Leob Hamath. They went up through the Negev and came to the Hebron, where Achamim and Shassai.
and Talmai is, the descendants of Anak who lived there. Hebron had been built seven years before.
Zohan in Egypt." Getting some history this morning? Yeah, I have no idea either, don't worry..
"When they reached the valley of Escol, they cut off a branch bearing a single cluster of grapes..
Two of them carried it on a pole between them, along with some pomine gramets and some figs..
That place was called the valley of Escol because there a cluster of grapes the Israelites cut off.
was from there. At the end of 40 days, they returned from exploring the land." Moses is really.
wise. I mean, they're camped right here on the corner of the promised land. But rather than just.
go into that land with the armies and just go and suddenly kind of attack, he sends these 12 leaders.
from the tribes and he says, "Go and see what it is in the land." Note that it says here, "I want you to.
see what is in the land before you." I think this act of seeing what is in our land, particularly.
the land in which we are planted in at this time, is perhaps one of the most important things.
that Christians should be doing in this time and in this hour. I believe actually one of the great.
spiritual disciplines we're invited into is the ability to take a look at what is happening around.
us. That we are invited with spiritual eyes to have a look at our culture, have a look at society,.
have a look at what's taking and see what it is that is taking place around us. That we might.
actually rise up as a people and take an account, see the land. I think we say this at the Vine.
quite a lot. Christians should be intelligent spiritual beings. That actually how we see the.

$^{121}$place is important. I think Christians should know what's going on in current affairs. I think we.
should be reading the newspaper. I think we should be educating ourselves about what is happening.
socially and politically. I think it's important for us to understand what social justice issues.
are at work in our neighborhood. We must see, not so that we can stand back from a distance and.
judge the reality of what we're seeing, so that we can actually discern what the Holy Spirit is doing.
amongst it. That we might see the realities of what's happening in our land and align ourselves.
to it. One of the great gifts of being a Christ follower, as Christ followers, we are invited to.
evaluate the fruitfulness of the land in which we are inheriting. And if Hong Kong is that land for.
us, what do you see? Are you with me? The problem is the church for many years has gotten really.
good at burying its head in the sand when it comes to the things that are happening around it..
We kind of label some things as spiritual and some things as not. And the things that we don't.
think are spiritual, we leave them aside. And the things that we think are spiritual, we magnify..
And I think God invites the church to open their eyes to see the wonder and the challenge of what.
is happening in the location they are planted in and to make a difference. So let me keep reading..
It says here that Moses says to them, "Do your best to bring back some fruit of the land.".
See, I think this is beautiful. Moses tells them to go and see. And the question is,.
what is it that they're supposed to be seeing? He does mention, "Check out the fortified cities,.
check out those things, have a look at that." Yes. But actually the emphasis is on the land itself..
He says, "I want you to see whether that land is fertile. I want you to see whether there is good.
fruit in that land. I want you to see whether we can actually build our crops there. I want you to.
see whether there are trees there." I mean, the emphasis of God's people going into the land.
is not to actually look at the cities and see how fortified they are. The emphasis is,.
is the land as God promised us? Is it fertile, flowing with milk and honey? And so to prove that,.
I want you, if you can, when you're in that land, to bring back to us some of its fruits..
Bring back to us a cluster of its grapes that are in season so that we can understand what it is.
that we're about to inherit. Notice this, church. This is so important. I love what Moses is doing.
here. Notice he doesn't say this. He doesn't say, "Bring me back a schematic map of Jericho.".
Have you ever noticed that? That would be what I would have asked him to do..
Hey, spies, find out how fortified it is, how high it is. Get all the measurements and stuff.
so we know how to overcome the obstacles. He doesn't do that. He says, "Here's what I want.
you to do. I don't want you to bring back a map of the towns. I want you to bring back some of its.
fruits." Why? Because I want my people to taste the sweetness of what they're about to inherit..
I want you to pass around the fruit to Israel so Israel can take it and bite those grapes and feel.
that sweetness in their mouths and that sweetness, oh, that the taste of the grapes would become the.
fuel for their fight, that they would realize that the land is exactly as God has called them to do,.
that it is the one that is flowing with milk and honey. And because it is, oh, we taste and see.
that the Lord is good. And that gives us the confidence to step into that land knowing that.
it is ours. He wants them to taste and know that what they're about to conquer is just the thing.
that he has provided for them. And church, I think this is what church is all about. I think the.
reason why we gather together like this on a Sunday, whether it's in this room or watching.

$^{161}$right now online, the reason why we gather like this is so that we can taste and know that the.
Lord is good. So that we can come together and bite something of the fruitfulness of Jesus..
That we can remind ourselves of the taste, the sweetness of forgiveness, of his love and his.
mercy for each one of us together. That we can sing songs that declare his victory and his goodness..
Why? So that we can get full? No. So that the taste of his love and his sweetness might become.
the fuel for our conquering of our promised land. That it might actually be the reason when we leave.
here at the end of 90 minutes with our hearts dripping and our mouths sweet with the taste of.
Jesus, that we say, I want others in the city to know that taste. I want other people in this place.
to experience the beauty and the wonder of forgiveness and love and mercy. I have tasted.
and seen that the Lord is good with vine on a Sunday. I go on Monday with the offer to people..
Maranatha, come and see Lord Jesus. I listened to an interview with Jackie Pullinger this week. I.
love Jackie. And she said in her early twenties, she came to Christ and she came to realize that.
she had eternal life. And she says on this podcast, she's like, I was filled with such a fire.
because I knew what eternity was like. And I wanted others to experience it..
Do you see the grapes around you? Are you tasting them? The beauty is that when we gather,.
we can feast together the sweetness of God, but not just so that we can consume in a holy huddle,.
but so that it fuels our desire to see others come to Jesus. Are you with me?.
So then notice what happens when they come back, having seen what they needed to see in the land..
Verse 26 says this, they came back to Moses and Aaron and the whole Israelite community at Kadesh.
in the desert of Paran. There they reported to them and the whole assembly and showed them the.
fruit of the land. They gave Moses this account. We went into the land to which you sent us,.
and it does flow with milk and honey. Here it is some of the fruit, but the people who live there.
are powerful and the cities are fortified and very large. We even saw descendants of the Anakhtar,.
the Amalekites live in the Negev, the Hittites, the Jebusites, the Amorites live in the hill.
country. The Canaanites live near the sea along the Jordan. Listen to the report..
They come back from their 40 days and they give two sides to the report. First of all,.
there is this beautiful fruit that's in the land. It is the way God had described it to it. It flows.
with milk and honey. Here is some of its fruit. In other words, come and taste and see that the.
Lord is good. But there's this other thing in the land. It's not just the reality of God's promises..
There's this other thing in the land. The people there are really, really big and scary. I mean,.
the people there are descendants from the Anakhtar. These are huge people. And so they come back and.
they provide this report. They say there are grapes and there are giants. In other words,.
there are the promises of God in the land and there is the problem in the land. And I want you.
to see something really important here. All 12 of them give that same report. It's easy for us to.
think that Caleb and Joshua didn't give that report. That Caleb and Joshua gave a separate.
report. That maybe actually Caleb and Joshua were sleeping in the camp the day that they went and.
spied out Jericho. Maybe they didn't see the fortified people and they didn't see the great.
giants in the land. No, no. All 12 of them, Caleb and Joshua included, come back and say,.
"Grapes and giants, promises and a problem." So the difference wasn't in what they saw..
The difference was in how they perceived what they saw..

$^{201}$Keep with me, church. I want to talk to you a little bit about grapes and giants..
I went to Welcome and I got the largest grapes that I could find. These are pretty big grapes..
But I want you to notice something. Imagine if you've gone into the promised land and you come.
back with that. And this is the grapes. And the grapes represent the promise of the fertility of.
the land for generations and generations. I don't know about you, but the grapes look pretty feeble..
They look pretty small, don't they? But Caleb and Joshua understood what these grapes represented..
That although these grapes appear small and feeble and relatively insignificant,.
what was behind them was massive and huge. That these were proof that the promises of God.
were good. And they were saying, "Hey, come and let us....
Wow, that's really good actually..
Come and taste and know that the Lord is good.".
But there were giants in the land. And these giants were massive. I did the study of the.
Hebrew and I did some exegetical work this past week on this passage. And I noticed this, that.
giants by their very nature are giant. That took me hours of study. That should blow your mind. If.
you're taking notes, write that one down. Giants by their very nature are giant. They are the very.
opposite of what a grape is all about. You cannot get a greater contrast between the smallness.
and the relatively feebleness of a grape and these massive giants. The giants are large for a reason..
They're designed to be so big that they block out the light of hope for you. The giants represent.
the problems that are found in the land of God's promises. And they're big and they're scary and.
they're massive and they're designed that way to kind of make you think that the grapes are nothing..
You see, you need to understand that there are always going to be giants in the land of God's.
promises. And then the giants are designed by the enemy so that they would overshadow the fruitfulness,.
the promise of that land so much that we're tempted to give up on the inheritance altogether..
Come on, church. This is a strategy of the enemy. There's always going to be giants in the land.
of God's promises. And those giants will so overshadow what might appear on the surface,.
something that is small and easy to give up on. Some of you right now, in your land,.
there are really big giants there for you. And those giants for us sometimes are personal and private..
Things that are going on in our lives, maybe sin we're grappling with, maybe stuff that's just.
happened, things that we've done, mistakes we've made, massive. And they look and they seem so.
overwhelming to us. And then think that giants are not just this personal private thing, but giants.
are also this massive thing in society as well. And that collectively, not just personally and.
privately, but collectively, we're also dealing with some pretty major giants in our city in this.
time. And those things can be so loud and so overpowering to us. And we need to realize that.
giants are designed to distract us from the small, but more powerful promises of God..
And some of us are getting too distracted by those giants. Let me talk to you for a moment.
about what I call giant tactics. Hands up if you're on social media. The rest of you are lying..
Shouldn't lie in church. It's really bad. When I ask a question, you answer. No, just kidding..
We're all on social media, aren't we? Here's the reality of social media. Social media is giant.
tactics. It's somebody's opinion magnified as loudly and as big as possible, driven to you on.
your feed by algorithms on Facebook and Instagram and YouTube that pick up what your personal likes.

$^{241}$are and tries to create an echo chamber of opinions that all go to either try to teach you.
or to drive you to get you to spend differently, get you to think differently. It's all designed.
with these algorithms. It's giant tactics at work so that you would come to believe that the thing.
with the loudest voice that's repeated the most must be the thing that's true. Come on, church..
Elections are won by this. One of my favorite social commentators, Seth Godin, he put it this.
way. Let me read this to you. "These days, it is irrelevant if something is actually true or not..
What is relevant is whether it's believed or not. And that belief is directly related to the size.
in which the state of fact or opinion is repeated and consumed." Giant tactics. The bigger,.
the bolder, the louder, the more often, the more easily consumed, that becomes the things that.
people believe. Here's the reality. Those with a different spirit have the audacious belief that.
sometimes the truth of God comes to us in the size of grapes and not the size of giants. That usually.
the truth of God is revealed to us in small, imperceptible ways that the world might say,.
"Plah, that's nothing. Look how feeble the church is." And we might say, "No, we're holding onto.
something that actually anchors us in the promises of a reality of a new creation.".
Jesus would say this, "You want to know what my kingdom's like? A mustard seed." It's not like a.
giant at all. It's a little bit like a mustard seed, the smallest seed in the whole of the.
garden. You want to know what my kingdom is really like? It's like a little piece of yeast in the.
giantness of dough. You want to know who I am and what I'm like? I'm the stone the builders rejected.
because it was too feeble and small, but on me will be the cornerstone of a new creation. You.
want to know how feeble the Christian faith is, how small compared to the giants? It's founded on a.
cross, an execution device that actually happened to the person that we believe then rose from the.
dead and conquered all things. Our faith in its very nature is grapes, not giants. And yet here's.
the problem. We're so overwhelmed by the giants. See, here's the great challenge for us. The great.
challenge for us as the church in this time in the city of Hong Kong is this, that we would claim.
to believe in grapes, but we'd actually end up listening more to the giants..
That we would allow the louder voices, the bigger voices to be the ones that determine our future.
rather than the small, beautiful, still voice. I love the how the scripture says,.
"Would you hear the still small voice of God in amongst the howling wind of worldly opinion?".
The 10 spies refused to listen to the small voice. They all came back, all 12 of them,.
two of them talking about grapes, 10 of them talking about giants. And here's what happens.
when we focus on giants in our lives. Let me show you this from verse 30. "Then Caleb silenced the.
people before Moses, and he said this, 'We should go up and take possession of the land, for we can.
certainly do it.'" A different spirit. "But the men who had gone up with him said,.
'We can't attack these people. They're stronger than we are.' And they spread a report amongst.
the Israelites, a bad report about the land that they explored. And they said, 'The land we explored.
will devour all those living in it. All the people we saw there were of great size. Can you not see.
the giants? We are too feeble. We're not strong enough.'" Notice this, "We are not strong enough..
They are stronger than we are." Moses never sent the spies into the promised land to ask them to.
give an account of whether they feel strong enough to take the land. He never once says,.
"Go into the land, and I want you to see how you compare to the people of the land." He doesn't.

$^{281}$say, "Go into that place and give an account of yourself." No, he says, "Go into the land,.
give an account of the situation of the land. Like, tell me about the promises of God. Have.
the promises of God come true? Are the promises of God going to be there for us? Tell me about.
His promises. I didn't ask you to compare yourself to the giants. When we compare ourselves to the.
giants, we're always going to come off feeling small, weak, and that we can't do it. They are.
stronger than us." I want to bring a reality check. The giants in your life, they're stronger than you..
This is why it was important that Caleb and Joshua saw the giants..
I think as Christians, we think that we are guaranteed a giant-free life..
Just ask the early church whether they had a giant-free life. We're not guaranteed a giant-free.
life. In fact, we should see the giants. The difference is what we perceive about the giants..
And Joshua and Caleb decided to perceive the grapes and the power of the promises over the.
fact of the reality of these scary giants. And that changed everything for them. And some of you in.
this room, you're allowing the giants to determine what your future is going to look like. And if.
that's true, then you are not living with a different spirit. And trust me, I know the giants.
are scary. I've experienced the giants myself. They are scary things. But I wonder whether together.
as a community, we might be able to stand with one another and magnify the sweetness of the grapes..
I want to taste and know that the Lord is good. When was the last time.
you bit the grape? Who would like a grape right now? Anyone want a grape?.
They're really good by the way. I washed my hands. I did actually wash my hands..
Anyone else want a grape? You're having one..
Okay, online, ready? This grape's coming for you. Watch this. Oh, oh no. Okay, all right, all right..
Okay, I'm going to stop throwing grapes around. Don't worry. But I wonder what it would be like.
if the church was a place where you came to know the sweetness and the goodness of God..
God never said to you, He never asked you whether or not that you're strong enough to do the thing.
that He's called you to do. God's never asked you whether or not you're strong enough to do.
the thing He's called you to do. See, it's not about your strength. It's about His promises..
And if the church can rise up on that, I believe we will walk in a different spirit right now in.
this time, in this season. Now, here's the second thing that happens. And I want you to see this.
because it's really important. They come and they don't just first say, "Hey, we're not strong.
enough." They then say this. And they spread among the Israelites a bad report about what they had.
explored. Here's the reality. When you let the giants speak to you, you will always turn good.
news into a bad report. If the giants are the ones that have the dominant voice, you will always take.
the good news, the inheritance of God, His promises, what He's doing and how He's alive, even.
the message of the gospel. And we'll have a way, we'll find a way of taking good news and turning.
it into a bad report. And so here's the 10. And I think this is what annoys God so much. He realizes.
that they're turning this good news, the promised land, into this bad report. And He says, "That's.
not the way it should be, that I invite you as my people to actually bring a good report about the.
good news." The challenge is, so often I think as Christians, we go into our workplaces and we bring.
a bad report of the good news because we're not that different from everyone else. And we don't.
really believe in forgiveness. And we're actually bitter and angry ourselves. And we're cheating on.

$^{321}$our wives and we're doing all this thing. And we're just like living a normal life. But we come.
on Sunday and we celebrate the grapes, but we don't eat them the rest of the week. And the Lord.
sees that and He says, "You're like the 10, not the two. Is there a different spirit in you?".
And would we hear, would we have ears to hear? Not like some condemnation, but actually an.
invitation into this reality of what life could truly be like, a different spirit. Would we taste.
the grapes again? And although that won't make those giants disappear, maybe we'll be shaping.
our lives on the thing that is an invitation. Look how Joshua and Caleb put it in verse 30..
When Caleb signed to the people before Moses, he said this, "We should go up and take possession.
of the land, for we certainly can do it." See, their attitude was, "We should, for we can.".
Right? We can because this is God's promises. So we should, for we can. But the 10 spies were.
basically, "We can't, so we won't." There's a completely different mindset based on whether.
they're perceiving the reality of the grace before them. We should, for we can, versus we can't,.
so we won't. And I wonder if I was honest with you today, which of these two are you living out?.
What kind of report is your life bringing about the good news of Jesus? Is it a,.
"We should, for we can," report? Or is it a, "We can, so we won't," kind of report?.
And this invitation is so beautiful. "We should go up and take possession of the land,.
for we certainly can do it." See, here's the reality. The report that you believe.
will determine the future you experience. Come on, church. I'm almost finished. Relax..
The report you believe determines the future you will experience..
And some of you here, here's the reality. You have a perception issue. And because of that,.
you have a report issue. And that report issue is determining a future issue. It all starts with,.
"What do you see?" Are you just seeing the giants and their ferocity, their size and their expanse?.
Are you carrying in the reality of the shadow they create? Or are you grabbing a hold, even if it.
looks so small and imperceptible, that still small voice, that little mustard seed? Are you grabbing.
a hold of something and saying, "I know this looks so small compared to that, but I'm going to take.
faith here and believe that this is the thing that I'm called to carry in this time." That we might.
see that, encourage that amongst us, taste and know that the Lord is good, and walk with a.
different spirit in this hour. So I stand before you today, and I simply say this again. In your.
context, in your situation, with everything that's going on for you, what do you see? You'll see both.
grapes and giants. Which one are you going to perceive over the other? Could you stand with me?.
I want to pray for you..
Father, we are grateful for this moment..
Grateful that we stand before you, a church, called by your name..
And in this important moment of history,.
in this important hour of our city,.
you're asking us, "What do we see?".
You have not given up on this place, Lord..
This is our inheritance. What a gift. We're so grateful to be living in this city in this time..
So grateful to be called Christians in this moment, in this hour. Thank you, Lord..
And we don't want to be a church defeated..

$^{361}$We want to be a church alive with a different spirit of hope..
Offering to our city the reality of what it tastes like,.
to being forgiven and redeemed and loved, reconciled, restored and renewed..
The Holy Spirit would say to you this morning, "Would you taste and know that I'm good once again?".
And would that taste be in your mouths as you stand before the giants?.
Whether you just open your hands just where you are, if you feel comfortable to do so. You don't have to..
It's just part of our posture that we do here at the Vine, just to welcome the Holy Spirit in,.
allow Him to speak to us again..
We stand together just, Lord, before your presence in this moment..
Lord, we recognize the giants in our lives. I want you to take a moment,.
just allow the Holy Spirit to show you. Maybe you already know what your giants are..
But whatever it is that you would just experience, just see again those giants..
But in seeing them, you wouldn't be overwhelmed..
You would realize that they're not going to get the final say..
Now, despite those giants that you're aware of, God is here for you..
His promises are new every morning. Great is His faithfulness to you,.
for He is able to change every circumstance and situation..
And so in faith this morning, you come before the Lord with all of the issues and the problems.
and the challenges. And He says, "My son, my daughter, what do you see?.
Do you see all those giants? And are you listening to their voice?.
Are you allowing them to define your future? Or instead, do you see me?.
Do you see what I'm doing? Do you see how I'm restoring and renewing you?.
Do you see the hope that is still there? Do you see the love that you once lost or forgot about?.
Do you see it again? For some of you, your giants are things that you've done in the past..
Those giants loom over you, trying to define who you are in the present and your future..
I feel like the Holy Spirit is saying to anyone in this room, "That's the case..
That's not... Your past does not determine your future.".
I feel like the Holy Spirit is saying, "There's freshness for you this morning.".
For some of you, it's been a while since you've tasted and known that the Lord is good..
I want you to take the time as we respond now just to invite Him to be with you,.
just to taste Him again. The sweetness of salvation, the beauty of forgiveness and mercy..
Allow that just to wash over you, forgiveness of your sin,.
strengthening of your hands for the work that He's called you to do..
Lord, we pray that grapes would burst forth in this room.
as we respond now to all that your Spirit is doing in this place..
Thank you for this..
\newpage



\section{}
\label{sec:vtinqnnv5SU}
\textbf{2021-06-22 11AM Service Live: World Refugee Day - Home Together [vtinqnnv5SU].mp3}
\newline
\newline
連結: \href{https://youtube.com/watch?v=vtinqnnv5SU}{\texttt{ https://youtube.com/watch?v=vtinqnnv5SU}} ~~~~ 語音日期: 2021-06-22 
\newline
\newline
\hyperref[sec:kH_AzSSAmbg]{\small{< < < PREV SERMON < < <}}
~
\hyperref[sec:index]{\small{[返主目錄]}}
~
\hyperref[sec:q7HuGRvEWFw]{\small{> > > NEXT SERMON > > >}}
\newline
\newline
$^{1}$Amen. Hey, can we thank our worship team today?.
Members from our Arise community on the drums. So appreciate that. Thank you so much..
Have a seat, everyone. Welcome. If you're just joining us online, we're so glad that you're joining us today, too..
We're celebrating World Refugee Day across our services. It's not just something that happens here in Hong Kong..
It's a global day that's celebrated by churches, NGOs and organizations around the world..
And it's our great privilege to be a part of that and a part of that story today..
And I want to tell you a little bit about how we're going to be doing that in a moment..
We've got the great privilege of having and hearing from the head of our refugee and asylum seeker development program with branches of hope, Roy Mjwabe..
And Roy is going to be here with us today, just sharing his heart for what it is for us to create a home together for asylum seekers at this time..
And then during Roy's message with us, we're actually going to take the opportunity to interview and have a conversation with two people that have made Hong Kong their home..
One being an asylum seeker and one being a local Chinese born congregation member..
We're going to hear what their story has been like experiencing home together here in Hong Kong..
So we're going to hear from Roy, but we're also going to hear from some of our asylum seekers and some of our community members and our congregation members here at the Vine as well..
But before we do any of that, we wanted to help frame what a day in the life can look like for one of our asylum seekers..
And particularly to see how our asylum seekers go out of their way to do their best to integrate themselves into Hong Kong society so that they can be not just people that are here in need, but here as contributors..
And we think here at the Vine and at branches of hope that our asylum seekers have so much to offer to Hong Kong during their time that they're with us..
And so this little video will give you a sense of what it is to be an asylum seeker and also to see how asylum seekers are integrating into Hong Kong society..
So would you join me in checking out this short film?.
[VIDEO PLAYBACK].
[NON-ENGLISH SPEECH].
[MUSIC PLAYING].
- Hi, I'm KK..
I'm from Pakistan..
I come in Hong Kong in 2001 because of some situation in my countries..
When I arrived, everything is so different to me..
I talk to myself..
I cannot change my home, but I get the new place..
This is my new home..
I get in the Kowloon Park..
And almost I going to the--.
passing my time for the exercise..
So that time, they are mostly, of course, in the Hong Kong people there..
So I learned from them the Cantonese language..
When I start to speaking in the Cantonese language here, so that time, first of all, I feel happy too because those people is treating me very well and very good and respectful to me..
I meet my wife..
It's very short in time to I come in Hong Kong..
So I have four children..
They are going to all in the school..
They grow up here in Hong Kong..
Every day, mostly, especially my job for the exercise also, about my health, I bring my children in the early morning to the school..

$^{41}$And after, I going to the market for buying the vegetables and the things for using at home..
[NON-ENGLISH SPEECH].
Home to me is where you are part of the society..
I don't know what will be happening in the tomorrow and in the future life..
But right now, I'm in Hong Kong..
Hong Kong is my home countries..
And this is my country..
And only God know what will be happen..
And I'm always looking God's hand..
God will be help..
And God will be done..
Everything is perfect for me..
[MUSIC PLAYING].
Hello, and welcome to this World Refugee Day 2021..
I am very excited to celebrate this important day with you all together with our refugees and asylum seekers in Hong Kong and all over the world..
My name is Roy Njuabe..
I'm the head of the Refugee Opportunity and Development Program at Branches of Hope..
And I'm very honored to share with you this message today..
It's a day that was set aside by United Nations to commemorate the strength and resilience of refugees..
And this year, the theme for the United Nations is that we should consider the power of inclusion..
The power of inclusion..
That is to include refugees and asylum seekers into our health system, in our education system, as well as our support system..
Here at Branches of Hope and Divine, we join this global celebration with the theme Home Together..
The video you just watched, if we remove the title "Refugee" out of KK, he's just like any one of us..
Going his normal day, take the kid to school, cook for the children at home, as well as trying to make Hong Kong a very beautiful home for everyone..
Whenever I think of the word "home," it often draws me back to my childhood memories..
I am home. That's a common word for my dad when he comes back from work..
And we all rush to him and say, "Hello, Daddy. Welcome back home.".
Me and my sibling, we're very excited to see Daddy every time at home after work..
And today is Father's Day as well..
I remember my father, too. He passed away a long, long time ago..
And we all want to honor the refugees, fathers in this hall and also all over the world,.
who have to take care of their children even though they don't have the right to work..
You can work, but you have to take care of your little ones..
Whenever we picture home, we often have this perception that, oh, for example, the Vine is my home church..
I am going home for holidays. I'm going home after work..
Hello, bye, everyone. Bye, my colleagues. I'm going home. The day is done..
Home can be that very wonderful place that we have this special connection to..
To some of us, this place can be a very good and fresh memories..
And to others, home can be a very bad place to be..

$^{81}$And I'm sure we all strive to make home a very good place to be..
Every society is striving to create that beautiful space with good memories..
Every society wants to create a special home for themselves,.
to make it remarkable, comfortable, culturally unique, with some special people inside..
People, they want to invite people that will come on with wealth to make it in a very rich place..
They want to invite people who come on with some kind of positive vibe, positive energy and some values..
They want to invite people that will come in with all the good things that they could ever list in their lives..
And so, they often select those that want to be in their home space..
And with our current world system, refugees are often considered towards the bottom of the list of those that we want them to be in our home..
The majority of countries these days don't want refugees..
They often push them away like rejected stones that they build and don't want..
The process to find a home for refugees is not easy at all..
It comes with a lot of rejections and pains and sacrifices..
Sometimes when the refugees are struggling to find a home, it could lead them to death..
Without a choice, refugees have to flee a broken and unstable home in search for a new and peaceful home..
And on this journey, they are often stuck between a broken home and an unknown home..
Life can be very confusing when you are between, especially when you don't have the right at that temporal home..
I want us to move into our Bible to get some biblical narrative of what to be in a temporal home looks like..
The Bible is full with so many stories about people who are searching for a new home while they are at a temporal space..
And one of these stories about the Israelites when they flee from Egypt, God promised them a permanent home..
The promised land..
The land was a geographical location that God the Father swore to his chosen people..
The descendants of Abraham in the book of Genesis 15, you can go and read that there..
The Israelites were on a journey from a painful, unstable home to a peaceful and stable one..
And on this journey, they were trapped in between and they had to live in multiple, several locations temporarily..
As the book of Exodus, we're going to highlight for us how this temporal home was like..
In Exodus chapter 13, verse 20, it reads, "After living so-called, they camped at Ethan on the edge of the desert.".
Exodus 14, 2, it says, "Then the Lord said to Moses, 'Tell the Israelites to turn back and encamp near Pihah Herod between the Magdol and the sea..
They are to encamp by the sea directly opposite Beelzebub.'".
This biblical narrative, we understand that God commanded the Israelites to move from one place to another..
Isn't that frustrating? You have to carry your kids, your animals, everything..
You move from one place to another..
The Israelites also find it so difficult to take care of themselves..
The experience living in this temporal home was so hard for the Israelites..
They have to move from one place to another and they could not even feed themselves..
They had to depend on God from manners from heaven where they were so frustrated, so angry, so confused that they cried out to Moses and said,.
"If only we could have died by the Lord's hands in Egypt..
There we sat around pots of meat and ate all the food we wanted, but you have brought us out into this desert to starve this entire assembly to death.".
Confusion in a temporal home without a right must be very frustrating..
For the refugees, they were great people in their countries..

$^{121}$Businessmen and women, teachers, nurses, veterinarians, farmers, engineers, pharmacists, doctors..
But in Hong Kong, they had to depend on others to survive..
They had to tell them where to buy food, that particular shop..
They had to tell them how much money they're going to use for food each week or each month..
When you are stuck in between at a temporal place without the opportunity to care for yourself,.
like the refugees who lack the right to work in the city, but you are forced to build a community, you don't feel that..
You want to be in a place where you have a sense of belonging..
You want to be in a place where you are under control of what you can do..
If you don't have this sense of belonging, you become frustrated, and sometimes you start to develop certain kind of mental challenges within yourself..
The frustration in a temporal home are so deep that the Israelites could not see the fruits in the promised land by the giant..
The frustration was so deep that they were afraid to even move forward. They wanted to go back..
For the refugees, sometimes this frustration can go so deep that they could not even see the talent in them,.
but the pain of depending on others for survival..
When people are stuck in temporal homes, they could be very frustrated, especially when the process is very slow..
Forty years in a desert. Oh, very slow..
The refugees are here for how many years? Decades. Some of them 10 and 15 years, and the process is very slow..
They wait for so long to know whether they have been accepted or rejected before they can move on to a permanent home in the future..
Sometimes the cases can take so long for it to be processed, and they don't know what to do..
And that waiting time, some of them will feel like, can I just go back home and face the persecution,.
or should I wait to move to a temporal place or to a permanent place somewhere in a third country?.
Just imagine during this season, I want to bring us to your own context right now..
Imagine during this COVID season, how many people have been unable to go to their home country to visit their family members..
At the beginning of the pandemic, it was okay..
You thought, oh, okay, the virus just disappear after one month or two months, maybe three months, it will be fine, I can travel..
But then it take months and then years, and they don't even know when you're going to stop..
You start feeling anxiety and frustrated because you cannot travel to your home country to visit your families and your friends..
You're afraid to travel because of travel restrictions and quarantine or maybe be infected by the virus..
Just imagine how frustrated you are in this season, and try to imagine how refugees and asylum seekers can be frustrated living in a temporal place for a very, very long time..
The refugees are struggling to live with the reality that they cannot travel back to their home country..
This reality contributed to the anxiety, sleepless night, frustration, confusion in life,.
whether to press on to try to move to the permanent space or to stay or to go back and face all the human rights abuses in their home country..
By understanding the pain and frustration and anger of the Israelites, we can start to understand how refugees and asylum seekers are struggling living in a temporal place..
Now the question is, why Hong Kong a temporal place for the refugees and asylum seekers?.
Well, Hong Kong is not a signatory of the 1951 Geneva Convention relating to the status of refugees..
Therefore, the Hong Kong government has a very firm and strong policy not to grant asylum, which means a refugee or an asylum seeker in this city is temporal..
They have no path to become citizens in Hong Kong..
So it means that if they are granted asylum, they will remain asylum or a refugee for the rest of their life once they leave Hong Kong..
They have just two options..
When your country is better, you go back, or you have another option for a third country, you can move on..
So what can we do when we live with people who are temporal in our city?.

$^{161}$The truth is we cannot change the situation, we cannot change the policy of making Hong Kong a permanent home for refugees..
That is in the hands of the authority..
But there's something we can do..
We can do something very special for our refugees..
We can make this temporal home a very comfortable place, a joyful place to stay with good memories..
We can open up opportunity for our refugees to enjoy our culture, our society, our spiritual family..
We can make this temporal home a better place for both the refugees and our children and our families..
With wisdom, we can work together to build a loving, caring home for our refugees and ourselves..
Proverbs 24, verse 3 to 4 mention that by wisdom a house is built, and through understanding it is established..
Through knowledge, rooms are filled with rare and beautiful treasure..
In my father's house, there's a place for me..
I'm a child of God, yes I am..
Now when we sing that song, you can understand that in the house of God, there is a place for everyone..
And how often do we create space in our own home for others?.
How often do we consider ourselves as that brother and sister that want to create a place, even though it's temporal,.
how do we create that space for our refugees and asylum seekers?.
Each person comes here with a treasure, with a want to contribute to the society..
This treasure could be as simple as a culture to share with you..
When we talk about treasure, it's not about a bag of money, or somebody come with millions of dollars to invest in your country..
It's not about bringing all the wealth and all the beautiful things..
It might be as simple as a culture to share with you..
Maybe a simple talent to contribute..
Maybe an experience that Kushtere could share with you and it would inspire you to think something beautiful..
We need knowledge to be able to make this treasure and this value an integral part of our community..
Some of our refugees who are here have been resettled to a third country..
Part of the work of BOH is to help refugees find a better place, a stable place, a permanent home..
That promised land, because we know they are living in this country, in this very city, temporarily..
And there's no way that will change anytime soon..
That anxiety is there, that fear is there..
So we want them to move on to a third country where they can have a stable life,.
where they can contribute to the development of those countries..
And today, I would love to share with you a short video about one of our refugees who went to Canada..
He was here for more than a decade, struggling, and he was rejected by the immigration system,.
but he was accepted in Canada..
Two countries, same story, same situation, two countries..
This one said, "No, I don't need. Your story is not correct.".
This one said, "Oh, come on, your story is correct.".
But anyway, let's see what this guy, how he's contributing right now in Canada..
Let's watch this short video, and I will come back in a second..
[Video playing].

$^{201}$[Speaking in French].
He was a refugee in Hong Kong..
Now he's building homes in Canada that are going to sell to Canadians to live in..
Isn't that a beautiful way that we could all work together to build our society?.
According to United Nations this year, the power of inclusion means when we learn together,.
we build a stronger community..
When we get the care that we all need, we heal together..
When we play as a team, we shine together..
And as followers of Christ, I want to add these few words to that list..
When we love each other, we live together..
When we worship together, we build a stronger faith community..
When we work together, we build a beautiful and harmonious society..
Whether we live here temporarily or permanently, you all have something to contribute.
to make this home a beautiful place..
At Branches of Hope, the R.O.A.D. program is what we are trying to do every day.
to create this learning home for our refugees when they are here..
We want them to learn something and also to create a caring home for our refugees..
All what you are doing and all what we are doing together is to make this space.
a little bit caring for our refugees..
Our secret angel program that you guys all supported goes toward helping our refugees.
find a home that they can live into peacefully..
We want our place to be a loving home as well as a faithful home..
Our Arise community, I've been sharing the gospel to hundreds of refugees..
Some of them came here, they are non-Christians, but they found faith right here.
in our Arise community..
We equipped many of our refugees who are now seminary students, pastors, church leaders,.
prison ministers, community leaders, you guys name it,.
because we created this temporal home for them to learn and experience love..
Whatever we teach our asylum seekers or we share with them,.
the truth is that when they leave Hong Kong, they will not carry this temporal home with them,.
but they will take alongside the faith that was developed, the knowledge gained,.
the friendship that they built will all last with them for a lifetime..
They will remember your love, your care, and your compassion towards them.
and give glory to God..
Each one of you have played a very important role in the lives of our refugees,.
and that is what we are celebrating here today..
I would like to invite two people who will come on stage right now with me.
to share with you guys how they have been contributing to make this home.
a very wonderful place for everyone..
One of them is living here temporarily, and the other, Hong Kong is a permanent home,.

$^{241}$but they believe that we all working together, we can make this home a beautiful place..
So let me like to invite to you Mr. Nas and Mr. Benson..
Please, can we all welcome them together on stage?.
[applause].
Wow, I'm really excited to meet these two gentlemen..
They have really been part of our community for a very long time..
And I'll just dive into the discussion that we have today,.
and I really want you guys to share with our congregation,.
and also for those who are watching online,.
to understand how impactful this journey might be to everyone..
I'm going to start with you, Mr. Nas..
I know that you've been here for quite some time now,.
and you believe that, and you still hold that notion that Hong Kong is a temporal home for you..
But can you share a little bit about yourself,.
and how do you see this temporal space that you are living in in Hong Kong right now?.
My name is Nasra. I'm from Egypt..
I have been here five years..
I came in 2016..
Really, I learned from my life that I learned how to be in love with where I live..
Because this is very important..
If I feel negative feelings like frustration or depressed or disappointed,.
what can I benefit from? What can I gain?.
I can change my situation..
But if I treat these difficulties, these obstacles, and stand on my way,.
with the mentality of courage and determination,.
I can overcome all these challenges..
I can do something positive in my life..
So I know that when I come here, I will face many difficulties,.
like different language, like tradition,.
like the mentality of people here is different..
I came also from a different country..
But I decided to overcome all these things..
I decided to do something positive in my community..
Thank you..
Wow, thank you..
Can we all give a very good hand for this beautiful word that we've heard from Mr. Nas here?.
And I will now turn to Benson..
I know that you've really engaged a lot with the refugee program at the Branches of Hope.
and other organizations in Hong Kong..
Maybe you give us a little bit about yourself,.

$^{281}$and also how do you see this permanent home, because Hong Kong is your home,.
and how do you engage with other people in this permanent home that you live in?.
Hello, good morning everyone..
I'm Benson..
I was born and raised in Hong Kong..
And I pursued my higher education in the United Kingdom..
So after spending 10 years in the UK, Singapore, I finally came back to my home in 2011..
So to me, luckily I was involved in the Branches of Hope Chinese tuition class..
Nas's son is one of my students..
A very smart kid, very brilliant..
I can see the future of Nas's kid..
And I think in Hong Kong I'm always very careful..
I feel that in Hong Kong, there's a lot of misconceptions about the refugee community in Hong Kong..
And I feel that they're under-resourced..
So that's why I hope to be part of the advocate or the one who can help this community to give them support,.
to give them emotional support..
Sometimes I think apart from financial support, there's also some emotional support as well..
And I think that having that companionship role, I feel that I can add some value to this community in Hong Kong..
Wow, thank you so much..
Can we all give a very good hand to Nas?.
That is brilliant..
Coming to our centre, teaching our refugees..
That is so lovely..
Thank you for doing that for the refugees in Hong Kong..
And so Nas, I'll come back to you again..
I know that you've been here for like five years..
And you've been doing a lot of things in Hong Kong that maybe we don't even know about..
Maybe you've been engaging, trying to share your culture..
Can you give us a little bit about how you've been contributing in Hong Kong?.
Though in your mind, I'm not living here temporarily, I have to leave someday..
But you still have that courage and that strength to keep on contributing in Hong Kong society..
Can you give us a little bit about that?.
Yes, before the COVID-19 pandemic, I have been teaching some Chinese group Arabic..
These people who intend to travel to the Middle East to work as missionaries and social workers..
And also I made some presentations in some churches and some secondary schools.
to explain Hong Kong and explain to the Hong Kong people and students a lot of the situation of refugees..
Because actually most of Hong Kong people don't know the situation of refugees very well..
Some of them don't know that we are not allowed to work..
Some of them think that we have an account bank..
Yes, some of them told me one time, "Do you have an account bank? I can give you money..

$^{321}$I can put in your account bank.".
I told them, "No, it's not allowed for us. It's not allowed to work here.".
But I think that you have to be positive in your community..
I try to help my community, my Egyptian people who come newly here, who don't know anything about Hong Kong..
I try to go with them to the governor building..
I try to translate for them..
I try to tell them about the procedures they have to do..
So you have to be effective..
If you are being effective in your community, I think you will be effective also in the society..
Wow, that's great. One more time again, let's hear from Nas..
Thank you very much for all what you are doing to the community in Hong Kong..
I'll come back to you, Mr. Benz..
I know that you may have done a lot of things already..
Maybe you can really share with the church and also those who are watching online,.
in the whole Hong Kong community, what you have been doing to care for this community..
Thank you. Since March this year, I'm one of the tutors in the Branches of Hope,.
teaching kids Chinese and of course sometimes more than about Chinese..
I'm having fun with them, playing games with them..
I go every week, every weekend, once a week, about one to two hours a week..
I'm looking forward to continue this commitment..
I feel that this is something that is really a hands-on experience to understand them..
I think in Hong Kong, a lot of people will be helping through donations..
But I feel that it's more impactful to really have a chance to interact with Nas and his kids,.
really understand what are the challenges..
I think through the interactions, we can give them hope, we can give them caring..
That is something that money can't really buy, that kind of caring..
I encourage everyone, if you have time, do volunteer with Branches of Hope..
I'm sure there are also organizations in Hong Kong that do helping refugees..
I hope that everyone can play a part, no matter how big, how small..
I really hope to continue being part of this community to help the refugees in Hong Kong..
Wonderful. Let's hear one more time for thank you very much. That's really awesome..
Now, I just want you guys to listen to them..
Each of you can have one minute just to look at the camera, also the community,.
those who are listening at home and those who are here with us today..
Just tell them in one minute what is on your heart to see how to build a home together..
First of all, I would like to thank Viney Church and thank Branches of Hope..
And also thank Sacred Angels, who support a lot of refugees here..
And also, I would like to tell you that diversity and difference can create a good climate.
for showing talents and skills that can benefit the society in which we live..
This is very important. If you support refugees here, if you give them motivation,.

$^{361}$if you give them love, they can get out their skills from inside,.
and this will benefit the society in Hong Kong..
I would like you to give them more love, more respect, and also more motivation and encouragement..
Thank you..
Thank you very much. That's wonderful..
I'm just feeling so glad to hear this..
So, Mr. Edmondson, maybe what would be your own last words to the Hong Kong people,.
and for those who are watching online and those who are in here today?.
Yes. For me, I feel that, first of all, thanks to Branches of Hope..
I think they're doing so much amazing initiative, very practical..
I can really feel it as a participant..
So, to those who live in Hong Kong, I think, first of all, don't label refugees..
As they mentioned, there's a lot of value that they can add to the community..
They can benefit us as well..
So, don't label them. Try to understand them through your own research..
Speak to them. Work with them. Be friends with them..
Secondly, I think, try to, of course, volunteering..
And sometimes, remember, small gestures make a difference..
Small gestures, you know, in a handshake, having a smile, just to greet them..
Just little things, right? Don't ignore them..
Try to be friends. Smile..
I think that can give them the sense of, you know, sense of belongings and, you know, welcoming in Hong Kong..
I think, all in all, I hope that everyone can, you know, together,.
we can make Hong Kong a much more inclusive society for all our, you know, refugees community here in Hong Kong..
Thank you..
Wow. Let's all give a very big hand..
Thank you guys very much. Please, thank you. And thank you again..
Wow. This really moves us, all of us..
Hong Kong might be a permanent home to some of us and maybe to others a temporal home..
Whether permanent or temporal, we could all play our part to make Hong Kong society a very important and beautiful place for everyone..
We can all join our hands together to make this home a beautiful one to welcome and to care for the strangers..
God understood what it takes to be in a temporal home, and he gave the Israelites this one commandment at the end of the day..
He said to them, in Leviticus 14, 34, "The foreigners residing among you must be treated as your native born..
Love them as you love yourself, for you were foreigners in Egypt..
I, the Lord, your God.".
God used the word "native born.".
It reminded the Israelites to treat strangers as members of their family..
It is our duties as followers of Christ to welcome our refugees as native born, even though it's for a period of time..
As you saw in KK's video, you saw Chester, and you see Naz and Benson, all of them played their own role to make this Hong Kong,.
even though a temporal place for some people or a permanent place for others, to make it more inclusive and diverse..

$^{401}$They have played their part..
The question is, what is yours?.
Can we all pray together?.
Heavenly Father, we thank you so much for being with us today..
Your love and your care and compassion have driven deep in our hearts, and we love to be together as a family..
In your home, there is a place for everyone, big, small, young, old, refugee or non-refugee..
And here in Hong Kong, we want to create that space for everyone, that we can all live together in harmony, love, joy, and peace..
And with that, oh Lord, speak to us individually, that we could do what you've called us to do..
We pray in Jesus' name. Can we all say?.
\newpage



\section{}
\label{sec:q7HuGRvEWFw}
\textbf{2021-06-29 11AM Service Live: From The Heart - The Gospel Afresh [q7HuGRvEWFw].mp3}
\newline
\newline
連結: \href{https://youtube.com/watch?v=q7HuGRvEWFw}{\texttt{ https://youtube.com/watch?v=q7HuGRvEWFw}} ~~~~ 語音日期: 2021-06-29 
\newline
\newline
\hyperref[sec:vtinqnnv5SU]{\small{< < < PREV SERMON < < <}}
~
\hyperref[sec:index]{\small{[返主目錄]}}
~
\hyperref[sec:yD3_LszW5Bw]{\small{> > > NEXT SERMON > > >}}
\newline
\newline
$^{1}$This beautiful thing that we're doing here is we're singing this song about that your.
love is greater than anything I've ever tasted..
Just earlier in the service, I saw a picture of somebody who had walked a long way to get.
to a restaurant..
And they get to this restaurant and they sit down and they open the food menu on the table.
and they're really, really hungry..
And they're so desiring to eat..
And they open up this food menu and it's blank from start to finish..
There's nothing on the menu at all..
And I felt like that was speaking to some of us in this room where you've got this real.
hunger and this desire for maybe an answer from God..
Maybe you've been praying for a while about something..
Or maybe there's a circumstance or situation that you're dealing with in your life and.
you're looking for direction..
You want an answer..
You want to know what is the Holy Spirit calling you to do in that particular thing..
And it feels like you're not getting any answer..
It feels like there's a hunger in you for the next step, a hunger in you for the achievement.
of something maybe, for a dream, for a breakthrough in a relationship..
But when you open the menu, it's just blank..
There's nothing there at all..
And there's a frustration in you..
And then I saw in this picture, the waiter approached the table and the waiter says,.
"Would you like the wine menu?".
And the person sitting there goes, "Sure.".
And the waiter gives the wine menu and he opens the wine menu and it's bottle after.
bottle after bottle of the best vintage wine that could ever be drunk..
And I felt like the Holy Spirit was saying that for some of us in this room, like, we.
have this hunger for an answer and we're looking for the answer in a certain place..
And yet actually the answer is a love that is greater than anything we've ever tasted..
That maybe the thing that you're searching for in that menu, it's blank because that.
thing's never actually going to satisfy..
It's never actually going to achieve the thing that you really want..
And the Holy Spirit's handing you a menu that He's written..
It's like the wine menu..
The New Testament talks about the new wine of Christ..
This new wine that has been poured out into new wineskin..
That we can receive a refreshment from Him..
Jesus does His first miracle at a wedding when the wine's run out..
And He takes those six ceremonial jars and He fills them with water and He, when the.

$^{41}$master of the banquet tasted it, He said, "This is the best wine I've ever had..
You saved the best for last.".
And I feel like Jesus is saying to some of us today, like, you've tried every other option,.
but guess what?.
The best thing is right here at the end..
I'm going to give you the best thing..
Some of you need to stop searching for what doesn't satisfy..
Just open your heart to the new wine of the Holy Spirit..
That will always satisfy..
You'll never run dry..
If that's for you today, I want to pray over us..
Before we do anything else in this service, I want to speak that new wine of Jesus into.
you..
If there's a question you're looking for, an answer you're desiring, there's some frustration.
there for you..
I want to pray for you..
If that's you, just open your hands where you are..
We're all got our eyes closed..
We're just doing this privately between us and God, but it's a way of saying, yeah, if.
you're watching online right now and this is speaking to you too, just open your hands.
wherever you are..
And I want to pray, Lord, right now, here at the vine, both in this room and those that.
are watching online..
Father, we pray that we are desiring this new wine..
We've looked in the menu for long enough, staring at blank pages..
We've looked for satisfaction in other things that are ultimately not Christ..
We've desired to find answers within ourselves or within others..
But Lord, it is you and only you that can satisfy..
You are this new wine that has been poured out in the new covenants..
And Lord, I want to pray for each person here, whether that's relevant or important for them.
this morning, that your new wine would come by the power of your Holy Spirit right now..
That you would fill and anoint and appoint, Lord..
That we would be like that firefly that Susanna saw in the darkness around us, holding out.
a light that is holding forth and holding fast to the gospel, Lord..
Lord, we pray that you would pour out your wine on your dry and your weary people..
It would be like an Ezekiel 39 moment, Lord, where the dry bones are brought together and.
your breath comes upon them and brings them to life..
I pray that the wine of Jesus, the wine that never runs dry, the cup that always runs.
over would be upon you and through you and in you for his glory..
And we pray this in Jesus' name, everyone says, amen..

$^{81}$Amen..
Awesome..
Can we thank our worship team?.
Just amazing as always..
So blessed by our worship..
Have a seat, have a seat..
My name is Andrew..
I'm one of the drummers here at the Vine..
It's great to be with you..
I know if you're new, I'm relatively new, I'm Andrew, one of the pastors..
And it's really nice to have you here..
Welcome online..
If you're tuning in with us, if you're just joining us, it's great to have you too with.
us..
We're considering our series From the Heart today where we're really moving off the back.
of a sermon series through April and May into June called A Different Spirit..
If you've been at the Vine for some time, for a season, you know we were doing that.
series on Numbers 13 and 14, looking at the spies that went into the promised land, asking.
the question, maybe it's a Caleb spirit that we need in this season and in this hour in.
Hong Kong with all the changes that we're facing, July 1 coming up this coming week,.
the one-year anniversary of really a significant change in our city..
And we're all still wondering how do we live now and navigate now as Christians within.
this new context..
And we've been saying that maybe there's this different spirit for us to carry in this new.
season..
And the From the Heart series is really a chance for us to hear from some of our leaders.
that have been here at the Vine over many, many years, some of the founding pastors,.
some of the current leaders, and ask them a simple question..
What's on your heart for the Vine?.
What's on your heart for us as Christians in Hong Kong in this hour?.
How we might begin to live out this different spirit that we feel the Holy Spirit is calling.
us to..
And today I couldn't be more excited to invite up someone who's been a long-term mentor to.
me, a friend, a father in the spirit..
Would you put your hands together as we welcome Tony Reid..
Let's put our hands..
Welcome, Tony..
Have a seat, my friend..
Welcome to my living room..
Thank you..

$^{121}$We have 400 people..
Music background as well..
Yeah, yeah..
Do you like my drumming?.
That's good, isn't it?.
Very good..
Anyway, it is a real privilege to have you with us today, Tony..
We don't say that lightly..
You've been a part of this church for 25-something years..
It doesn't show, don't worry..
But we are so grateful..
And I recognize that there's probably a bunch of people maybe in this room, and particularly.
maybe those online as well, that don't know you as well..
Maybe they're relatively new to the Vine in the last year or two..
Maybe you could just tell us a bit, first of all, who you are, how you came to Hong.
Kong, your journey with this church..
Thank you, Andrew..
It's a great privilege to be able to be with you and to just speak to you today..
Well, I arrived in Hong Kong in 1986 with my wife and three children..
We've since adopted a young Chinese girl, so we have four children..
But I came not as a pastor, I came as an engineer..
And that has always been my training to earn my living..
But all my life, I have been involved in the church..
The church has been on my heart..
I've done many, many jobs in different churches..
So as I came to Hong Kong, what was on my heart was thinking about why was God bringing.
me here?.
What was that purpose?.
How could I serve in the church in some way?.
And so as we began looking around churches, we eventually ended up in Repulse Bay Baptist.
Church..
I think you might be familiar with it..
I was there..
I was like 14..
14, yeah..
A while ago..
I don't think I was your Sunday school teacher, but never mind..
No, you weren't..
And that church, of course, then evolved by a number of stages into the Vine Christian.
Fellowship and then the Vine Church itself..

$^{161}$And that church had a certain character to it which attracted a number of people..
And it grew and it was run by a group of about four or five elders who ran the church as.
a sort of lay ministry, all doing different jobs..
And as the church grew, it became apparent that we really needed someone to be able to.
look after and to minister to the church, to be the pastor of the church..
And so we advertised..
And at the same time as that was going around, both John Snellgrove and myself really felt.
God speaking to us..
And I felt speaking to my heart quite late in life that actually, Tony, yeah, perhaps.
you should be a bit brave and step up and maybe this is a role that you should take..
And so we both put our names forward to the elders and said we both feel that God's calling.
us..
What a choice..
And so to their great credit, they appointed both of us..
And I must say, I treasure the faith which those pastors put in both of us at that time..
And yeah, as the church has grown and as we've seen all that's been happening in the Vine.
and just sensing that God had asked us to grow a next generation church, the realization.
of course dawned that we were not next generation pastors..
So we better start looking for one..
And very fortunately, there was this young man in the congregation who was a brilliant.
speaker and able to really express the heart of the Vine being with it from the beginning..
And so it was our great privilege to be able to hand over that leadership to Andrew..
And at the same time for us, a huge privilege to be able to still serve in the church..
And so that's my story..
You know, can we put our hands together just to honor this man?.
I just think it's amazing..
There's not a lot of churches that would allow two people to be senior pastor at the same.
time..
There's not a lot of people that would be able to be a senior pastor alongside of another.
equal senior pastor at the same time..
And there's not a lot of people that would stick around at the church after they've handed.
it over and continue to cheer it on and to believe for its best..
You know, the Vine is incredibly blessed with Tony and John Snellgrove..
You're going to hear from John in a couple of weeks in this series..
I think it's an amazing thing that we should never take for granted that God has done..
It's passing through into the next generation..
And I admire you guys..
All right, I'm going to stop loving on you, baby..
It's going to be a love fest for like 30 minutes, so we'll stop that..
You and I have been talking quite a bit about this different spirit series..

$^{201}$And you know, the role that Tony plays at the Vine is our justice advocate..
So Tony has a role here to help keep me and the other pastors and the church up to kind.
of date with all the stuff that's happening in terms of justice, social justice, God's.
heart for justice, the social political context of Hong Kong in the city, and much of what.
you perhaps have heard during a different spirit series coming from my mouth has really.
been birthed out of conversations that Tony and I have had around what we're sensing the.
Holy Spirit is saying about Hong Kong and the social political context..
And so that series was quite personal for both of us..
And I know you wanted to share a few things..
Yeah, I think it was..
I think it's been really very significant for the church..
You know, to me, this is not just been another series, another series, although, of course,.
all of these things are important for us..
But I feel this is another chapter for the church..
And I feel it's not only been significant for us..
I think it's a significant word for the whole of the Church of Hong Kong, because it's allowed.
people to be able to reflect, not just to accept things that are happening around in.
a passive way, but to be able to, as you know, as Andrew has shown us how to live our lives.
with that different spirit..
And so I think it's opened up for us a lot of possibilities, you know, thinking about.
the social and political circumstances that surround us and just appreciating and understanding.
what that means..
Considering that in that context, what the gospel means for us today in those circumstances..
And then lastly, I think, trying to work out what sort of community that we need to be.
in response to that..
And I guess it's partly that that I want to talk about today..
Well, so Tony and I have talked a lot about the gospel, and you might have heard in the.
different spirit series me talk a lot about the gospel and how this is a significant inflection.
point for the gospel..
And one of the things that I think has really resonated for us as we've conversed is the.
idea of, well, what is the gospel?.
And perhaps we don't embrace or grasp the beauty and the power of the gospel as much.
as it was originally intended to be seen and grasped..
So I know you want to share a few things around that..
So tell us, what is it for you that's really buzzing about the gospel at this time?.
Yeah, well, I think the important thing is that because of these circumstances, we're.
forced to sort of re-evaluate the value, the importance, the centrality of the gospel to.
everything that we're doing..
Because if we're going to live in this different spirit, then we've got to be able to have.
some solid ground, we've got to have some solid things in which we are basing our lives.

$^{241}$and living our lives as well..
And of course, we're called to be disciples of Jesus..
And so I think there is a challenge to us in this, a challenge to me to look again,.
to see what it was that Jesus was saying and how he was living, and how this affects the.
global church as well..
And so I want to start by focusing on two imperatives that I think are important as.
foundations of how we view what our perspective of the gospel is about..
And the first one really is understanding and appreciating the initiative of God to.
save the world from destruction..
You know, put yourself in the place of God when he created the world, you know, and all.
this process, all of his love, hard work and creative process goes into creating this amazing.
place, this cosmos, this planet..
And then thinking about all the possibilities of what that might mean in terms of the glory.
of this place..
But then afterwards, when seeing things going downhill, seeing the sin, seeing all the destruction,.
seeing idolatry, seeing all the desperate things that were happening in the world, imagine.
what his heart must have been thinking about..
And he must have thought, either I'm going to get rid of this and we'll start again..
Let's do something different..
Let's do something new..
But God didn't do that..
God was so intent..
It was his imperative to save the world, to redeem it, to make out of the mess that we.
have made it something that would bring him glory..
And of course, the key verses that we read in scripture that tell us about God's heart.
for this are really in John chapter 3, in the most well-known verses in the Bible..
John chapter 3, verse 16..
Let me read them to you again..
It says, "For God so loved the world that he gave his one and only son, that whoever.
believes in him shall not perish, but have eternal life..
For God did not send his son into the world to condemn the world, but to save the world.
through him..
Whoever believes through him, or in him, is not condemned, but whoever does not believe.
stands condemned already, because he has not believed in the name of God's one and only.
son.".
And of course, normally when we read these verses, we think about the love of God, you.
know, in sending Jesus, but there's that little word "gave" that God gave his son..
And in this giving, I think we get a sort of little picture into God's heart, a little.
sense of what God was doing here, that there is a sense of almost of desperation..
You see, I don't think we see the risk..

$^{281}$I don't think we see the commitment of God..
I don't think we see the danger that was involved in what God was doing here in sending his.
son..
I don't think we see the sort of drastic action that was required..
And there is within this, I think, a reflection of God's, the necessity that he felt to save.
the world..
That was imperative..
That was his heart, and that he will go to any extent in order to redeem and save this.
world..
And we also have to see the way God does this now..
You know, he's not now condemning the world..
He's not now thinking about destruction, but now we see a different face of God..
We see the love of God..
We see the embracing, the open arms of God, just like in that parable, that story of the.
lost son..
We see the sacrifice and the love of God welcoming, embracing us, wanting us to have that relationship.
with him that would save us..
I think what's really important for us to reflect on here is the reality of this dangerous.
gospel that in God giving, you know, we've made John 3, 16 such a bumper sticker, right?.
Like it's on the fridge magnet thing, and we kind of know that verse so well, and yet.
it actually contains this beautiful risk that you're talking about, this danger to it..
And that risk really comes out in that second half of the verse, which I think is the second.
imperative you wanted to talk about, yeah?.
Yeah, it does certainly, because of course, you know, John 3, 16 goes on to John 3, 17,.
which says, "God did not send his son into the world to condemn the world.".
And so it's important for us to see Jesus' mission, not as a condemner, not as a judge,.
but Jesus was inviting us the opportunity to have this new relationship with him and.
with God..
But there is a choice to be made..
And this choice is significant..
Whoever does not believe is self-condemned..
And so we see the imperative of God, the urgency of God to save the world..
There's a possibility that there may be an outcome that God does not want to see from.
him..
And I think we see a great tension here..
There is a tension between the immensity and the initiative and love of God on one hand,.
and on the uncertainty of the response on the other..
And it's this tension that I see, which I find so interesting, that God is in a sense.
pleading with the world that he will do anything to save the world from itself, even to the.
extent of giving himself in this dangerous manner on the cross, this demonstration of.

$^{321}$God's love..
You see, he's not commanding obedience, but he's inviting us a response of love, a response.
over which he has no control..
You see, God places such high value on saving the world, redeeming the world, bringing about.
his creation to what he intended, that he will risk anything, even to the extent of.
sending his son, even to the extent of giving us the free will to choose whether we will.
follow him..
And the interesting thing is that we are planted in the middle of that tension as well..
I think it really is a tension..
I think my daughter Mia, she's 10 years old, and for most of her life I've been commanding.
obedience from that little girl..
And she's been giving it by and large..
But she's now at the age where obviously she's wanting more independence..
She's wanting to express life and love, and she's wanting to make decisions..
And I want her to make good decisions and good independent decisions..
And so there's this stage, isn't there, in parenting where you have to kind of like let.
your children go a little bit, and you have that anxiety and that will they agree, will.
they continue on and follow well, or will they make mistakes?.
And if they make mistakes, I have to let them do that, learn from that..
But my love is so strong..
And I kind of feel like that with Mia..
Now that analogy breaks down between me and God, by the way..
But in some ways that's what God does in giving his Son..
He's giving us the kind of freedom to choose to respond to the invitation of grace or not.
to, and that's incredibly scary..
It is scary..
And the fact that God has entrusted us with this gospel, you know, I think we have to.
somehow take stock of what he has given us because there is a tendency, I believe, for.
us to undervalue it..
And maybe it's a challenge that we don't always see..
And so we get used to and very comfortable with our Christianity..
You know, we come to church on Sundays..
We learn the stories of Jesus..
We read the Bible..
We understand everything that happens there and all the community that we get praying.
together and the support and encouragement and the amazing sermons that we listen to..
And somehow if we're not careful, this can be so comfortable to us that we can tend to.
mold it to the shape of our own lives instead of necessarily perhaps seeing the challenge.
to us..
And I think that we don't always tend to see the radical transformation that God has made.

$^{361}$in our lives..
You know, that he's given us this new life, this new opportunity..
He's changed who we are..
And sometimes I feel for myself personally that, you know, have I really devalued, have.
I really honored and understood the value of that thing which God has taken such trouble.
to give and make available..
And I like the way that the Apostle Paul talks about the gospel..
You know, he doesn't talk about it as a message..
He doesn't talk about it as, you know, an inspirational word or something that is important.
for us to hear..
This is what he says about this gospel..
He says, "I'm not ashamed of the gospel.".
Now that's a word for us, I think..
"I'm not ashamed of the gospel," he says, "because it is the power of God to salvation.".
The power of God..
And so what I think we don't so easily and always appreciate that what God has given.
us in the gospel is the power of God..
It's a mighty thing..
It's a powerful word..
And we somehow, you know, think that we've got so little to offer and that we, you know,.
it's so difficult for us to do or we will not be able to do this thing right or live.
right..
But I think we have to understand that within the gospel, there is the power of God..
And you know, God is so mighty and so powerful and so for us..
You know, the Bible calls him the Lion of Judah..
And sometimes I feel that I denigrate him..
I treat him more like my domesticated cat that sits on my lap, you know..
And he's the power of God, and that's so important for us to understand..
So the gospel is perhaps more radical than we think it might be..
It's perhaps more dangerous..
It perhaps invites us into understanding life in fresh and new ways, maybe more than we.
even appreciate..
And I think there is this power in the gospel that, you know, when we think about what it.
is to live with a different spirit, I think it's this reliance and trust on the power.
of the gospel to be at work, not just in us and through us, but in our city and through.
our city..
And this is, I think, what's really on your heart, Tony..
My question out of that, though, would be how do you hold all of this together?.
What sort of metaphor, like how do you actually live out and deal with this kind of type of.
gospel?.

$^{401}$How do you remember it in your day-to-day lives?.
Well, to do this, I tend to think more about, and I have recently been thinking more about.
the gospel as the kingdom of God..
You know, that when Jesus came, he said, "The kingdom of God is near you," and all of his.
teaching was about the kingdom of God..
It's not just a message..
It's not just something that is spoken, but it's a kingdom that we live in..
And I think that there are a couple of things over the past 50 years or so within the sort.
of life and history of the church that have tended to, in some ways, limit the scope,.
the intensity, and the broadness of the gospel..
I think one of those is the sort of traditional message that much of the church has spoken.
out that, you know, Jesus came to die on the cross, to save us for our sins so that we.
could go to heaven when we die..
And of course, there is a central truth about that message..
But when we just limit it to that, then we miss, I think, something of the broader, the.
grander aspect of the gospel or the kingdom of God that he spoke about..
And another of the trends has been that we have, over the last 50 years, we've modernized.
the gospel, which is good..
You know, we've realized that we've got to use new language..
We've got to be able to speak and connect with many people in different ways..
And I think this communication has been good, and it has made the reality of the gospel.
more real in people's lives..
But at the same time, because it has been molded around us, there's a tendency for it.
to become more about me and my life rather than the kingdom of God..
And so here is the result of this..
We have a tendency to personalize and privatize the core emphasis of Christianity at the expense.
of a broader and richer understanding of its implications, really, on so many aspects of.
our life, our work, our community, and the government, and how we live together..
I think this is really important, Church, that we kind of really wrestle with this idea.
because I think the reality is, you know, when we think about Jesus and we think about.
His involvement in preaching and teaching in Galilee and the starting of the spread.
of Christianity, when we look at Christ, we think He's, you know, it's tempting for us.
to think that He's just in the world to give a new moral philosophy, or He's kind of in.
the world to tell you how to live your life better and live it in a way that pleases God..
That's not what Jesus is doing in the gospels..
He's not trying to tickle your ears and help you to live slightly more morally good or.
ethically better..
He's actually inaugurating a new reign of the power of God on earth..
It's the kingdom of God..
He's bringing that..

$^{441}$That's what all of His teaching was about..
It was all of His parables was about..
He was trying to help people to understand, you're now living in a new time, a new age..
The kingdom of God has come, and in Jesus' teaching, His death, and His resurrection,.
that kingdom of God is now here on earth, the now, but not yet..
We strive and desire for the fullness of that kingdom to come as Jesus returns for a second.
time, but we live in this moment now where we are a part of this radical, dangerous,.
incredibly subversive, beautiful, hope-bringing story of a new kingdom, a new way of living,.
and that's really what the gospel is..
It is, and we have to remember that all of this, Jesus' teaching and Jesus' life, was.
lived out in a very tense social and political situation..
Sometimes we like to view Jesus' life like we read in our Sunday school books and our.
picture books of green hills and sunshine and rainbows and the nice words of Jesus,.
but really it wasn't like that at all..
When Jesus came, there was really quite a tense situation that was there, that firstly,.
in the Jewish social background and political background that He lived in, there was a great.
divide between the rich and the poor and the wealthy..
Those who had power were those who were able to exercise their authority..
But many of the people that Jesus went to first were those who were marginalised, they.
were the poor, they were the people in the villages around who were in desperate need..
So there was a sudden sense of repression that was going on at that time in Judea..
The other thing I think we have to remember is that when Jesus was having conversations.
with the rabbis and the people around him, he was not just speaking to the religious.
people, he was talking to the political leaders at that time..
Because society, the leaders constituted the Sanhedrin, which was like their sort of government,.
their parliament by which they made decisions for the city and for the people, how they.
would live..
And then of course there was the temple worship itself, which again had great restrictions.
on how people lived and on people's lives as well..
And we see that there is that sort of quite difficult socio-political situation in which.
Jesus lived..
And then if you put on top of all of that the Roman Empire, the Romans who had conquered.
them, then you see that they were intent on maintaining peace and stability at any cost,.
including brutal suppression and crucifixion..
And all around them the sign of their conquest was the thousands of people that they just.
crucified..
And so there was this domination that was around them..
And so when Jesus spoke about the kingdom of God, it was already in a more immediately.
tense situation than perhaps we tend to think about or appreciate when we read the Gospels..
I think a lot of that social-political context is lost..

$^{481}$It's assumed because it's written to people of that day and of that era..
They didn't need to explain it..
It was already a part of the fabric of how people lived their lives..
And we have to keep it in mind when we read the stories of Jesus..
There's a moment where Jesus heals, for example, on the Sabbath..
And on one level it's an amazing story of the miracle transformative power of God, and.
that's often how we only think of it..
But actually Jesus does it on the Sabbath because He's also making a message about the.
controlling kind of situation of the day and how all the laws had restricted actually the.
work of this kingdom of God..
And a lot of people were going to miss the power of the kingdom, the true power, because.
they're more conscious of the structures than they were of actually the liberating power.
of God..
His parable ministry was very much like this as well..
Much of his teaching was designed to say something to the Pharisees, to the rabbis, to the rulers,.
the Jewish rulers and religious leaders of the day, to help them to begin to think differently.
about what this Messiah and this kingdom is going to be all about..
Yeah, that's true..
And later on, of course, as Christianity developed, you know, we mustn't make the mistake of thinking.
that Christianity developed in this neutral, very appealing environment that allowed it.
to spread and to work amongst people..
We have to appreciate that there were restrictions..
And the truth is this, that God nearly always reveals himself in times of trouble or starts.
something when there is hardship..
And Christianity is often most vigorous when the circumstances are difficult..
And this is an interesting thing that we need to take on board, I think perhaps in our own.
circumstances as well..
So when Jesus spoke about the kingdom of God, the people who would have listened to him.
would have immediately started connecting the dots to that sort of environment that.
they lived in and not to the religious ones or piety..
And the gospel, of course, would have resonated most strongly with those who were marginalized,.
those who were at the lower end of the socioeconomic spectrum, and with those who were outcasts.
and sinners..
It was them that would have received the freeing message of the gospel..
And so it's not long, you can imagine, before tension started developing within the Jewish.
society with what Jesus was preaching and teaching..
And so I think the challenge for us today and the challenge for many people around the.
world is, are we prepared to live in this tension?.
How do we deal with this sort of tension where we have Christians in May conflicting with.
different voices, different pressures around the world?.

$^{521}$And I think the challenge for us also here in Hong Kong is how do we respond to this,.
the political, social and political situations that we live in here in Hong Kong?.
I think this is a real challenge that I think Tony's bringing from his heart today, which.
is around how serious are we going to take the gospel?.
First of all, do we understand the gospel in the right context, in the right framework?.
And then how serious are we actually going to take that gospel?.
Are we going to embrace it?.
And I know for you, you see that seriousness really sitting in two primary areas, one about.
how we actually live our lives and one about how we actually see ourselves..
Can you tell us a little bit more about that?.
Yeah, it is..
And I think that there's a key word here is that we have to reposition ourselves..
I think we have to look at, recreate the sort of mental and spiritual maps of ourselves.
and realise that our primary allegiance is to Jesus..
That if we are going to be able to make any sense of what's going on, if we're going to.
be able to respond or think about how we live in this environment, then we've got to understand.
that our allegiance is firstly to Jesus..
And you know, it was God's imperative to save the world, but it's our imperative, I think,.
to maintain our strong allegiance to Jesus, that he takes first priority in any questions.
or calls of our loyalty or on the things that we do..
And secondly, it's not just about allegiance, but it's also how we see ourselves, our identity..
And this, of course, we understand and know that the core of our Christianity is that.
God has, there's been that new birth..
We've been reborn into his kingdom..
We have a new life..
We have Jesus, but we have the blood of the kingdom running through our veins, those thoughts,.
that spiritual life that we have..
And I think we need to take it a step further and think about our identity, that we are.
not just children of God..
We have to know which tribe we belong to..
We are Christians..
That is our tribe..
And it's interesting that this word, this identity that we so often use to describe.
and think about ourselves was not coined by the followers, the disciples of Jesus themselves.
at all..
But it was a name that was given to the followers of Christ in Antioch because of the way they.
lived their lives..
It was in contrast to the society that was around them, that people were pointing and.
seeing them because they were not living like everyone else..
They could see that their allegiance was to Jesus..

$^{561}$They could see that their identity as a people was different..
And so they were a people who were living in contrast to the people around them..
And so the church in Antioch were creating, if you like, an alternative society because.
they believed an alternative narrative to the one everyone else believed and because.
they worshipped a different Lord to the rest of their world..
And this meant that they were doing crazy things like they were taking babies, caring.
for babies off the street that were abandoned..
They were looking after their widows..
They were feeding the poor..
They were treating their slaves like they were their brothers and sisters..
Now that may not seem like much to us, but in those times in the Greek or Roman world,.
that was just like, "Whoa, you're doing that?.
How can you possibly do that?".
So they stood out in the society in which they lived..
And so in a way, their life was like a sort of criticism of the politics and lifestyle.
of AD40..
And because of that, they became known..
You are the little Christs..
You are the people who worship Christ..
You are the people who worship different gods, different God to us..
And I think this is going to be so important for us as we grapple with what Christianity.
is going to look like, not just here in Hong Kong, but around the world with all of the.
changes that we're seeing globally happen..
And I want you to think strongly about where is your allegiance?.
I mean, seriously, like is Jesus Christ your primary allegiance?.
And what is that going to mean for you in the years ahead?.
And do you identify yourself primarily as a Christian more than anything else?.
In other words, like you don't identify yourself primarily as Chinese or primarily as English.
or primarily as American or any culture as great as cultures are..
You don't primarily identify yourself as a certain political persuasion, whether you're.
yellow or blue..
Though you don't primarily identify yourself in various ways, your primarily identity is.
a Christ follower, is Christian..
And that becomes more important to you than anything else..
And that drives who you are becoming in the world today..
Fully committed to Christ, fully believing that you're a part of the tribe of God, this.
kingdom of God, you're a child of God..
And that builds unity and strength and purpose amongst all the people around you..
That's a different spirit, right?.
That's the gospel at work..

$^{601}$And it's interesting that you speak of Antioch as an alternative society..
One thing we're going to be doing in the second half of this year is a book study on Philippians..
And we're going to be looking at primarily how the church in Philippi lived as an alternative.
society..
Now, importantly, not in rebellion to everything around them..
And it's really important you hear this..
Not in rebellion..
They weren't trying to be different to annoy the authorities..
They were living as Christians, as Christ-like as they can, knowing that in doing that, their.
allegiance was now ultimately to Jesus..
Knowing that in doing that, they're identifying themselves primarily as Christian before anything.
else..
That brings them to helping those babies..
It brings them to serving the poor..
It brings them to living in a new way, to loving enemies and all these things..
And in doing that, they create this alternative environment, this alternative society..
Does that make sense to you guys?.
And I think that's the beauty of the gospel..
And that's what we're going to look at in more detail in the second half of the year..
Yeah..
And really, this, I think, is such an exciting time, Church..
I want to encourage every one of you to really try to think about and assess your own life..
You know, try to challenge yourself..
How do I live?.
And what's important to me in my life?.
And what are the things I should be doing?.
And how do I react?.
And how do I behave?.
And so this different spirit, as I said, I think is a new chapter that we're entering.
as a church..
And so I really encourage you to follow what we are doing..
Listen to these series and things that Andrew is preaching..
They are really important to us..
And so there's a lot of work to be done..
There's a lot of work to be done, Church..
And so together, I think, as we sort of recalibrate, reevaluate what it is God is calling us to.
do, then please join us in this adventure and this exciting journey that the church.
is on together..
God bless you as you do that..
Tony, you are awesome..

$^{641}$Could we give Tony a round of applause?.
What a man..
What a ledge..
So grateful..
I wonder whether we could all just stand together..
And Tony, I'm going to invite you just to pray for us and pray much of what you've been.
sharing and speaking over us as we respond as a church..
So why don't we just bow our heads and allow Tony to speak again..
Thank you so much for all that you have brought us..
Thank you so much for that imperative that you have set upon to save the world, to redeem.
it, to restore the amazing creation..
And Lord, thank you so much for allowing us to be part of this through Jesus..
Thank you for your saving grace..
And Lord, as we step into that place now, help us, Lord God, to step up..
We ask you, Lord, by your Holy Spirit to encourage us, to give us the strength..
Lord, as we examine our own lives, we see the opportunities around us..
Lord, that you would make us brave servants, brave bearers of your gospel, that as we bring.
the good news, you will pour your blessing and the power of your Spirit upon us..
In Jesus' name we pray, amen..
Amen..
I want you just to take a moment..
We're going to go into a time just to respond with the Holy Spirit..
I wonder whether you just close your eyes on me again and just let's allow the Holy.
Spirit just to continue to speak..
I think Tony shared some real wisdom with us today..
And perhaps there is something for you to sit in here about your allegiance right now..
Maybe you recognize that there is a split in your allegiance and that you're hearing.
Christ calling you back to the centrality of Him to be primary in your life..
Maybe you're wrestling with that identity as what it means to be a Christian..
Maybe being a Christ follower is kind of number eight or nine on your list of how you would.
speak about your identity and who you are..
Maybe the Holy Spirit wants to just challenge you a little bit and speak to you a little.
bit about that and remind you of that thing that Paul wrote, "I am not ashamed of the.
gospel to be called a Christ follower.".
Perhaps for some of you it's to just remember the radical nature of God who would send His.
only Son so that we would be invited in grace to respond, not by being coerced or forced,.
but in freedom to choose Him and to choose Him and to choose Him..
Perhaps you want to just honor Him and worship Him and glorify Him for making that choice.
today..
Whatever it might be and how you are to respond, allow the Holy Spirit to lead you in this.

$^{681}$time..
Amen..
\newpage



\section{}
\label{sec:yD3_LszW5Bw}
\textbf{2021-07-07 11AM Service Live: From The Heart - Cry of My Heart [yD3\_LszW5Bw].mp3}
\newline
\newline
連結: \href{https://youtube.com/watch?v=yD3_LszW5Bw}{\texttt{ https://youtube.com/watch?v=yD3\_LszW5Bw}} ~~~~ 語音日期: 2021-07-07 
\newline
\newline
\hyperref[sec:q7HuGRvEWFw]{\small{< < < PREV SERMON < < <}}
~
\hyperref[sec:index]{\small{[返主目錄]}}
~
\hyperref[sec:sQhUi8jUN2g]{\small{> > > NEXT SERMON > > >}}
\newline
\newline
$^{1}$Once I was traveling with dear friend to a certain place,.
to be away from a most troubled context,.
reflecting on the significance.
and the puzzling circumstances of things..
At the time, I and my company.
could only identify discouragement and unbelief..
Then we met someone on the road, a stranger to us..
Somehow we caught up and chatted..
And rather unusual to my introvert's inclination,.
I actually expressed some of my profound discouragement..
At the long faith journey in Christ,.
in a lato of hopes of how the kingdom of God may manifest,.
and me as a believer in Christ,.
together with my fellow believers in the city,.
could be part of this..
Yes, the circumstances and happenings.
seemingly dashed all such hopes..
Well, I least expected that this gentleman responded.
by rebuking us for our hardness of heart..
He even pointed us to the Bible.
concerning the manifestation of the kingdom..
We did another extraordinary thing..
We invite him to join us for dinner,.
and there was a great time of fellowship..
Then it dawned on us, well, that this person.
is no stranger..
We used to know him..
That's why our hearts burned,.
even as we shared bread and the cries of our hearts..
Somehow we find renewed hope,.
and actually decided to call off our departure,.
and return to where we came from,.
because we find new hope, new purposes..
You may by now guess this stranger's name is Jesus..
Luke chapter 24 recorded for us.
the account of two disciples.
going down the road to Emmaus..
Cleopas and his company represented discouragement.
and unbelief of Jesus' followers..
When they met Jesus on the road,.

$^{41}$his identity was somehow hidden from them,.
and they expressed their profound disappointment.
at the tragic events in the city of Jerusalem..
They had hoped that he might be the Messiah.
who would redeem Israel,.
yet his death on the cross dashed all such hopes..
Jesus responded by rebuking them.
for their hardness of heart..
Did not the scriptures predict.
the suffering of the Messiah?.
Even though now equipped with the truths of the scripture,.
the disciples did not recognize Jesus,.
until they invited him into fellowship.
around the dining table..
Jesus broke bread,.
Jesus commune with them,.
and now they recognize Jesus..
Jesus disappeared from their presence,.
and they joyfully returned to Jerusalem.
to report their experience..
There they find the other disciples,.
announcing the same good news..
Jesus is indeed a rise..
I'm thankful for Andrew's invitation.
to share something from my heart..
Please take me as a fellow brother in this congregation.
and from the internet..
I'm hopeful what is laid in my heart.
also finds resonance in you..
What I want to share from the scripture today.
is about my own reflections in church life.
and how we can best express it..
It just happens that I have only attended.
three churches in my lifetime..
10 years in a very local traditional Chinese church,.
and 10 years in the Assemblies of God Church in Singapore,.
and of course, 14 years here in the Vine..
We got a bit more involved in church life.
as we participated in the first Chinese Alpha.
from 10 years ago..

$^{81}$I want to reflect some of my concerns.
and my faith commitments, rooted, reach out, and renewal..
Let's return to the Bible first..
For Luke's readers, including us,.
basically this is a call to come out of the fog.
of disappointment that these two disciples left in two,.
and instead, dare to believe the promises of his word.
as we fellowship with Christ..
Interestingly, there's an epilogue to this event..
Jesus appeared again and fellowshiped with his disciples.
over the dining table in Jerusalem..
Why are you troubled, and why do doubts rise in your minds?.
This is what I told you while I was still with you..
Everything must be fulfilled that is written about me..
And he opened their minds.
so that they could understand the scriptures..
You are witnesses to all these things,.
and you will be closed with power from high above..
Jesus is talking about the Holy Spirit and empowerment..
And Jesus reveals himself.
while in table fellowship with his disciples..
He is indeed at home in our midst, in our everyday activity..
So how shall we apply all these scriptures into our life?.
What is my cry of the heart for my fellow brothers.
and sisters here in the vine?.
So I've mentioned the three R's..
To be rooted deeply in his word,.
to reach out with fervor, and that is the mandate,.
and to renew our hearts, be connected with him..
During the early days while we were still dating,.
and of course we are still dating,.
I loved writing letters to my then girlfriend.
and now my wife Vivian..
Yes, love letters,.
so that I may express my love for her in writing..
When we had a committed relationship,.
I actually gave her my diary..
I was open to her,.
and my story was passed on to her in writing..
I would hope that she would read my love letters.

$^{121}$in my diary time and again,.
so that she would know me, and she would know my love for her..
With some degree of similarity,.
God conveyed his love message to us in writing,.
so that we can read time and again..
So David, the psalmist, encouraged us..
I'll hide your word in my heart.
that I would not sin against you..
The biblical sense of knowing.
is actually to have an intimate relationship..
It is just inconsistent if we say.
we want to know his plan for our lives,.
and yet somehow we remain complacent.
in trying and getting to know him..
The yearning to know him more could differ.
depending on your own unique situation..
I'm encouraged to know that many fellow brothers and sisters.
are committed to studies offered by,.
say for example, the Bible Study Fellowship,.
or other Bible study groups in your own community groups..
But on the other hand, I'm sometimes frustrated.
that some of us hardly make an effort.
to allow his word to speak to us on a daily basis..
Anyway, for me, at the age of 50,.
I did enroll for seminary studies.
as I desired to know him more..
Yet, one of my greatest understanding of his word.
is when my own circumstances become so conflicting..
The time point was at the second year.
of my five-year part-time study program..
I was enthusiastic to be equipped for service..
It was then that I was diagnosed with leukemia,.
a blood cancer..
I struggled to reconcile the disruption to my plan..
I was serving him through my second half of my life..
And studying to know God's word.
become almost like a disillusion..
What is the point of knowing him more.
when ultimately we shall know him in perfection?.
For now, we see only a reflection as in a mirror..

$^{161}$Then we shall see him face to face..
There is a dilemma..
So much needs to be done..
So many need to be served..
And so much relationship to be treasured..
And so did the time..
It was at this point that his word brings clarity..
There's new light as I read..
For to me, to live is Christ..
If I am to go on living in this body,.
in this body, this will mean fruitful labor for me..
Somehow I find comfort that he knows best..
There's a perfect timing in his plans..
And by knowing him more, I actually learned.
to trust him more..
I actually decided to bite the bullet.
and carried on with my studies.
and my own clinical practice..
And at the same time, went through my chemotherapy.
treatment, which was spread over two years..
Today I'm glad to report that I've graduated.
from the seminary..
[audience applauding].
And I have finished my treatment..
[audience applauding].
Now you don't have to go through seminary.
or chemotherapy to help you make a tough choice.
whether to study his word..
You just need to make a commitment and do it..
That daily application of God's word in our ordinary lives.
is exactly what God desires for all of us.
in the secular world..
How about you?.
Knowing God's word through diligence studies.
will anchor you to face whatever lies ahead of you..
It can be COVID, it can be losing a job,.
a relationship, or even a loved one..
But his word will help us navigate.
the worst moments in life..
Isn't that good news?.

$^{201}$Our knowing him should actually find a way.
of expressing through testifying..
When we know him, we naturally want to share the good news..
On the road to Emmaus, the two despondent.
disciples were revived and they were given.
new purposes in life..
They were summoned to bear witness for the resurrection..
By the way, the Greek word to bear testimony.
is the same as to be martyrs..
Well, we can at least imagine the seriousness.
of the mandate to bear testimony, witness for Christ..
Well, lest we get too nervous about sharing,.
let's remember the Holy Spirit is always there..
He dwells in us to help us make sense of our testimony..
Around the world, that testimony is shared.
one message at a time..
It can be shared by a missionary or by a preacher..
But it can also be by a housewife with his neighbor,.
by a business person with his colleague..
And for the 11 o'clock service with a lot of families,.
it can be by a father or a mother.
with his precious son or daughter..
Or even by a child with his beloved parent or grandparent..
Testifying can come in moments of joy or despair,.
or even at death's door..
I was probably like any one of you,.
considering that sharing the gospel is just an option.
and not something to be pursued for fear of rejection..
Many years back while I was practicing in Singapore,.
in the line of work I came across this young man.
with very aggressive cancer..
As his disease progressed, I actually contemplated.
to share the gospel with him..
That opportunity never came..
He left into a coma..
By then nothing very much could be offered medically..
But I did offer to the devastated wife,.
well, that maybe I could connect her to a pastor.
from the church to pray for her husband.
and maybe for the family..

$^{241}$It was a Sunday..
Everyone else had their own commitments..
And to my disbelief, my pastor challenged me.
to be the one to share the gospel with her.
and to pray for the man..
I felt so inadequate, so unprepared,.
so lacking in faith, and so clumsy in my mentoring..
By asking my wife to come along,.
we did share the gospel and prayed with her..
And we led her to Christ..
This encounter had a tremendous impact.
on how I view myself as that person chosen.
at the right time to be empowered by the Holy Spirit,.
like any one of you, to be the right person.
at the right time, at the right place,.
to share a testimony..
Once that commitment was made, doors just opened..
I had this encounter with a lady.
who scolded me so badly about my care.
for his father's cancer..
We only find out that subsequently.
we were from the same church..
Well, it happens for a big church.
with more than 3,000 members..
And I could understand that she was really stressed.
by his diagnosis..
Anyway, along the way, we actually worked together.
and led his father to Christ..
And even community groups can come together.
to be a blessing..
A lady was stuck in the hospital for months..
My group, my home group came around.
and visited her in the hospital,.
reached out for her family with four young kids,.
and even brought her out to Ritz-Carlton.
to celebrate her birthday..
She used to be a Buddhist,.
but she found tremendous joy in knowing Christ.
before she departed..
We should just seize the opportunities.

$^{281}$once we are prompted by the Holy Spirit..
And I'm thankful to be brought up in a church culture.
where testifying, sharing the gospel.
anywhere, anytime, to anyone, is just natural..
We just embrace that simple commitment.
to share the gospel and bring one person to Christ.
every year..
There was also a rich Pentecostal heritage..
We just believe in the power of the Holy Spirit..
Where doors are seemingly closed, it can be opened..
We can take down spiritual strongholds..
We can speak prophetically when we hear from God,.
and we can bring hope where there seems to be no hope..
An invitation is all that really matters..
It is not so much because death is looming..
Rather, it is an invitation to new life..
My late parents came to know the Lord.
when they're in their late 70s..
Someone visited them week after week,.
drive them to church while they were stuck in Toronto.
with no means of transportation and no social life..
My wife came to know the Lord.
as some stranger housewife befriended her.
in the swimming pool where all the moms.
took the kids to learn to swim..
She experienced the work of the Holy Spirit.
the very first time she stepped into the church..
She was just stirred to weep,.
even though she never cried in adulthood..
Indeed, the church is not a members club for believers,.
but a place whereby seekers should be welcomed and embraced..
Could this church do that?.
I am eternally grateful for the different ones.
who invited me and my family to know the Lord..
I want to be the one who invites someone.
to receive that eternal blessing..
So my encouragement, be upfront in sharing your faith..
Don't be an undercover agent..
Be ready..
You should not be surprised when your colleague.

$^{321}$come forth to ask for your counsel and your prayers..
Be intentional to build relationship..
Be authentic in caring..
Invite a friend to Chinese offer,.
even though you may not be a Chinese speaker..
We do our best, God will do the rest..
Embrace your work..
That is the most powerful testimony in the workplace..
So far I've shared about rootedness in this word,.
as in the head..
Reaching out to testify as in the hand..
I wonder whether you notice that between the knowing.
and the doing, there's still something missing..
The heart matter..
And probably for many of us,.
the most difficult is the transformation of the heart..
The Emmaus encounter reminded us.
that Jesus spent time with the disciples.
who were so disappointed, disillusioned..
And yet, we witness that there's a major reversal.
of emotion, transforming despair into hope and action..
All is not lost..
The resurrected Jesus is in the midst of us right now,.
even when we are not aware..
All it takes is our willingness.
to have our hearts transformed..
The journey of the heart,.
despite the short distance from the head,.
is just not easy for many of us..
My wife is heavily involved in prayer ministry.
and inner healing..
She often asks me, "How do you feel?".
I would rather she ask, "What do you think?.
"What shall we do?".
Well, of course I knew some of the dark side.
from my own childhood..
It was very difficult economically in the '60s..
My wife conceived, my mom, I always say my wife, sorry..
[audience laughing].
My mom..

$^{361}$I loved my wife as much as I loved my mom..
[audience laughing].
Anyway, in the '60s, so you can guess my age..
My '60s, in the '60s, my mom conceived me.
and she was then at the age of 40..
And somehow, as she anticipated a lot of hardship ahead,.
she actually tried, and she tried very hard to abort me..
Well, by God's grace, I do not have too many physical.
or mental defects, as far as I know..
Yeah, but surely there were a lot of racial defects.
from those emotional trauma while still inside the womb..
She was in great distress, and to some extent,.
she passed on a lot of those fears.
and pessimistic outlook into my own life..
She went through her own transformation.
as she subsequently managed to share this opening.
with myself..
But in turn, I need my own renewal of heart.
to change my generally negative outlook in life..
In my own practice, I used to spend a lot of time.
on organ transplantation..
It is often and never taken, it is not often,.
and really, we should never take it for granted.
that organs becomes available..
Someone needs to be willing to give..
And the dilemma for heart transplant is even more intense..
Someone needs to die to make a heart available.
for a person in need..
But Jesus has already died that death,.
and he's ready to give that new heart to us..
I will give you a new heart and put a new spirit in you..
I will remove from you your heart of stone.
and give you a heart of flesh..
It would then be up to us to receive that gift.
from our donor, Jesus..
If there's a moment that I literally cried from my heart,.
that was five years ago, when I wept, I cried,.
or even yelled, so much so that my wife,.
who was sound asleep at three o'clock in the morning.
in the upper room, heard my cry and rushed to comfort me..

$^{401}$I was in great despair when I read through.
the medical literature to find out that.
the predicted median survival,.
the years to live for my type of leukemia.
would be eight years..
I did not go down a literal road to Emmaus,.
but I was distraught..
My wife came around to keep me company..
Then somehow and knowingly, Jesus,.
a stranger to me, since I hardly recognized him,.
Jesus also came along..
He helped me to know how foolish I was.
and how slow of heart to believe.
in all that he has spoken..
But most significantly, he embraced us..
The three of us had a time of fellowship.
with prayers and commitments..
My heart was transformed..
Late completion of a heart transplant surgery,.
it was pumping again..
So allow your heart to cry..
Cry out to Jesus..
Cry out to your loved ones..
And my desire for this church community.
is such that we find trusted fellow brothers and sisters.
whom we can cry out to to journey together.
and to receive healing together..
The disciples who were once so discouraged.
now become a community of believers.
with transformed hearts,.
coming together to testify his goodness.
while welcoming others..
In fact, every day they continue to meet together.
in the temple court..
They broke bread in their homes.
and ate together with glad and sincere hearts,.
praising God and enjoying the favor of all the people..
And the Lord added to their number daily.
those who were being saved..
So my encouragement is be committed to a community..

$^{441}$Invite your pre-believer friends to your community.
to share the goodness of Christ..
And for yourself, be open to inner healing.
and prayer ministry..
Begin your own journey of the heart today..
As we draw this sharing to a close,.
hear the cry of my heart for this congregation..
Let us be committed to be rooted in his word..
Be passionate in testifying.
and to continue our journey of the heart..
I do want to have a chance for all of us.
to respond to this message..
Can I invite the band.
and I also invite all of you to stand with us..
(audience murmuring).
There will be some of you.
who want to renew your commitment.
to know him through diligent studies.
so that you can trust him more..
Come, make that commitment today..
You will hear him speaking to you through his word..
Some of you are stirred..
You want to be that good friend.
who invites his friend to come to know Jesus..
Be bold to ask today for the privilege.
to be that right person at the right place.
at the right time for that person you love..
And some of you are crying from your heart..
You yearn for a renewal,.
but you don't know where to start..
Come, have fellowship with Christ..
The altar is open..
Our prayer team wants to partner with you,.
to pray alongside with you,.
to voice out, to shout out to the Lord.
your commitments to be rooted,.
to reach out and to renew..
So this is a time for you to come forth,.
to open your hearts.
and to stop false,.

$^{481}$to be prayerful in a community of believers..
Even in the upper house,.
our pastoral staff is ready to pray.
with whoever wants to have someone to pray alongside..
Come..
And for some of you who already made that commitment,.
I want also that you can invite.
someone standing next to you.
as long as you are comfortable..
You may not even know that person,.
but yes, you invite him to pray with you.
so that you would be the one.
who invites someone to Jesus..
You can also invite yourself.
to come and meet up with Jesus..
So spend that time as the one standing next to you..
Commit in prayers that you will indeed,.
you will indeed invite someone to Jesus.
and that someone can also be yourself..
Okay, shall we go for a time of prayer,.
mutual ministry, pray with each other.
and we spend good time praying with.
and praying for each other..
(gentle music).
\newpage



\section{}
\label{sec:sQhUi8jUN2g}
\textbf{2023-04-18 EXODUS - 01 In The Beginning [sQhUi8jUN2g].mp3}
\newline
\newline
連結: \href{https://youtube.com/watch?v=sQhUi8jUN2g}{\texttt{ https://youtube.com/watch?v=sQhUi8jUN2g}} ~~~~ 語音日期: 2023-04-18 
\newline
\newline
\hyperref[sec:yD3_LszW5Bw]{\small{< < < PREV SERMON < < <}}
~
\hyperref[sec:index]{\small{[返主目錄]}}
~
\hyperref[sec:Bq85h5xoYT4]{\small{> > > NEXT SERMON > > >}}
\newline
\newline
$^{1}$Amen. Church, would you have a seat? Thank you, worship team..
And today we kick off our long-awaited,.
[BREATHING].
I remember when I first heard about the Exodus. I was maybe like six or seven, sitting in.
some Sunday school classroom, hearing about a God who does miracles. The parting of seas,.
the dropping of food from heaven, the turning rivers into blood. I mean, it was the stuff.
of legend and fairy tale, and my imagination just soared. So I kind of grew up believing.
that when God delivers His people, He does so by stepping into moments of human history.
in earth-shattering ways, that He steps into the worst moments of human suffering to change.
it in an instant. Well, I've come to learn, of course, that actually the greatest miracles.
God does is not the parting of the seas, the dropping of the food, the turning rivers red..
The greatest miracles that God does are the small, almost imperceptible changes that happen.
inside of us. The things that on the surface appear almost inconsequential or meaningless,.
but when seen all together, create a story that actually is even greater than we could.
ever have first imagined..
[Music].
The Exodus story, as revealed to us in Scripture, tells us that when it comes to God delivering.
His people out of their slavery and bondage and bringing them into freedom, it's not actually.
in those breathtaking moments that God does the deliverance, it's actually in the journey.
itself. In fact, it's in the journey, in all of its highs and its beauty, as well as its.
struggles and trials and hardship, that actually the deliverance itself occurs. And this tells.
us something very interesting about the Exodus narrative, that if it teaches us anything,.
it teaches us this, that God indeed is in the business of delivering His people. That.
deliverance, however, may come in a form that we never first would have expected..
[Music].
In this series, I want to invite us on a journey of deliverance together, as we trace the historical.
and spiritual footsteps of Israel as they journey from their slavery to their newfound.
freedom in the Promised Land, some 400 miles away..
[Music].
It's a journey with many twists and turns, a journey of both joy and sorrow, of great.
hope and painful tragedy. It's a journey that will take us from these crowded streets of.
Cairo to the sacred tombs of Minya, to the shores of the River Nile, to the wide expanse.
of Sinai Desert, to mountain peaks of stunning beauty, to valleys barren and endless. It's.
a journey that will take us across the breadth of Egypt, right up through the land of Jordan,.
eventually into the heartland of Israel itself..
And it's a journey that starts with you. You see, we all have a story of oppression. We.
all long to be free from the things that enslave us. So could it be that this ancient story.
of tragedy, slavery, and oppression, and ultimately freedom and hope and love, is actually far.
more about you and your story than you ever realize? Our God is a God who longs to bring.
people out of slavery into freedom. He's a God of hope and goodness and promise. He's.

$^{41}$the God, in other words, of the one thing you need most, Exodus..
[Music].
Ah, there we go. So, there is nothing more powerful than truly being free. And we had.
originally planned to bring this series to you in the fall of 2020, and then this little.
thing called COVID got in the way. And I've always long thought that God has a wicked,.
ironic sense of humor, because a series that is designed to teach us about the joys of.
freedom got delayed by a virus that strongly restricted our freedoms. Perhaps we are now.
in a much better place today than we were, say, three years ago, to really begin to reflect.
about what it is like to be moving from a place of restriction to a place of freedom..
I mean, I think it would be wrong to grossly say that we've been in slavery in the last.
couple of years, but I think it would be right to say that all of us have desired a greater.
sense of personal freedom in our lives than perhaps ever before. In many ways, I think.
in the last four years or so, we've experienced maybe a microcosm of what the Exodus narrative.
is all about. We've all experienced what it's like to live under painful restrictions. We've.
all experienced how those restrictions have changed our identity, the way we think and.
the way that we feel. We've all experienced in the last few years, a longing and a desire.
for a change in the future. We've all realized that we don't have the ability to make that.
change ourselves. We all have been in need of something outside of ourselves to bring.
that change. We've all experienced that when we begun to see a glimmer of hope that there.
is a change coming, we felt that hope rise up inside of us. We've all found ourselves.
taking those tentative first steps into the new freedom, not sure whether that freedom.
is going to last and then get stuffed back into lockdown once again. We've all experienced,.
haven't we, walking into freedom again and finding ourselves in the joy of what that.
freedom has brought us and yet still carrying the baggage of the trauma of the years that.
have gone. In many ways, our journey in the last couple of years has very much been the.
journey that Israel went through in the book of Exodus. And one thing we learned from the.
book of Exodus, perhaps more than any other, is that God's timing is perfect. And so I.
think delivering the series now is the right time that God wants us to go on this journey.
together. Now, as much as COVID, I think has created us a new perspective for understanding.
the book of Exodus like never before. Today, I want to invite you actually to a much deeper.
level of a journey in Exodus that perhaps you might've thought of before. You see, the.
book of Exodus at its very core is essentially an invitation into two of humanity's greatest.
journeys. And that is the journey of discovering who is the creator of the universe and the.
journey of discovering who we are in response to that. I want to say that again, the journey.
that Exodus really invites us into is a journey of understanding who really sits at the center.
of the universe and then understanding who we are in response to that person. And really.
what the book of Exodus does right at the beginning of the biblical narrative coming.
right as it does after Genesis is invite us to ask some of the biggest questions of our.
shared human experience. Questions like, is there really a God? And if there is a God,.
does he care about our experience, our life, our suffering? And if this God does care about.

$^{81}$our suffering, then will he come to deliver us from that suffering? And if he does come.
and deliver us from our suffering, what is it that we learn about this God in the process.
of that deliverance? And not only what do we learn about him, but once we are finding.
ourselves in new freedom by his deliverance, then how do we come to understand how we live.
now in relation to him and importantly, in relation to one another? Those are the central.
questions of Exodus. And because of that, when we look at the book of Exodus, we're.
not just studying a book that helps us to understand something about God in the old.
Testament. What we're really looking at is a book that helps us to understand who God.
is in general. And therefore we're looking at a book that really helps us to understand.
the life, death, and resurrection of Jesus. When you think about the journey of Exodus.
and you think about those questions behind me on the screen right now, you can understand.
why this book begins to point towards Jesus Christ. And you can begin to understand why.
Jesus Christ in his ministry uses so much of the imagery of Exodus. His baptism, going.
into the desert for 40 days to be tempted by the enemy. The way he teaches his disciples.
to pray, which is essentially a prayer that takes them through the journey of Exodus..
How he invokes the imagery of the Exodus on the cross and in his resurrection. How the.
early church captured all of that in the communion time, the breaking of bread, the giving of.
that, the sharing in the wine, how all of that began to speak to the early church about.
the power of God who brings us through Exodus. The Exodus is not just a book in the old.
Testament. It is a blueprint for understanding the work of God in the world today. And this.
is the beauty of it, the power of it, the profundity of this book. And the book is shaped.
around two essential processes. Those are the process of going into freedom communally.
and individually. If Exodus is about anything, it's about a communal journey of a nation.
discovering freedom from oppression, discovering hope out of slavery. God comes to a nation..
He is turning compassion for a nation. He raises up people to lead a nation. He draws.
that nation through the most miraculous, powerful things that he has done, perhaps in all the.
scripture. And he does it because he wants the nation to be shaped by those shared experiences..
So Israel together experiences the parting of the sea. Israel together experiences the.
pillar of cloud on Mount Sinai. Israel together hears the law being given on the stone tablets..
Israel together sees what it's like to walk into the promised land and receive the long.
held promise that God had for them. God does it in a nation because there's something powerful.
about a community of people experiencing God together. But it's not just communal. Importantly,.
the Exodus is also personal and individual. As we'll see next week, it all begins in the.
calling of four women, individual women who have the courage to stand up against an empire.
and say enough is enough. It then focuses on an individual man, Moses, a man who's deeply.
broken, a man who's shamed of his past, a man who needs Exodus himself and how he feels.
about the darkness that sits in him. And then a man who is then used to lead a nation to.
freedom. And then throughout the journey, there are people like Hur and Aaron, Jethro.
and others that come to the fore. God uses individuals to also lead them on a journey.
of salvation. And over the next 25 weeks, this is what we're going to do. And I want.

$^{121}$to encourage you to set both a communal and an individual expectation. I believe that.
God's going to do something powerful amongst us over these 25 weeks communally together..
I want to invite you to be here every single week as we journey through this together,.
because I believe God's going to reshape something in our culture, reshape something in who we.
are. If God has come and done that for a nation in the past, can he not come and do that in.
Hong Kong and China today? If God has come and shaped and changed the culture of a people.
before, can he not come and shape the change the culture of a people now? It's something.
that he longs to do together. But if your expectations are only communal, then I think.
you're actually going to miss out a lot of what this is all about. Because at the very.
heart of it, Exodus is for you. There is a journey of Exodus that God is calling you.
on. He's calling me on. There is a movement, a need for a movement away from the brokenness.
that is in me towards more freedom in Christ Jesus. And if you come here expecting to look.
at some visually engaging films and be entertained and learn something about the Exodus, you're.
actually going to miss out on the whole point of this, which is for your Exodus..
This whole series was birthed five years ago. And as so many things are in my life, it was.
birthed whilst I was walking and praying. And I saw a picture, and I was praying for.
us as a church five years ago, and I saw a picture of a crowd of people worshipping..
And in this crowd of people worshipping, their hands were all lifted in the air. It's a beautiful.
picture. But as I looked closely, I could see that there were shackles around the wrists.
of everybody, and there was a chain between their hands. They weren't chained together..
They're all individual. But each of the hands that were in the air had shackles around them.
and chain between them. And I felt like God was communicating that this group of people.
were worshipping and had a desire in them to live the life that Christ had always called.
them to live, a desire in them to be Christian, to live Jesus in the world, and a loving desire.
to worship Him. And yet something was weighing them down. Something was holding them back..
They weren't truly free to worship Him. And these chains were heavy. And as I looked at.
this picture over time, I could see people's hands. They're trying desperately to keep.
them in the air, but the chains were wearing them down. And as I look closely in the chains,.
they were not new chains. They were brown, tinted, rusty. These were old chains, chains.
that had been hanging around for a long period of time. And I noticed that the chains, the.
links in the chains between the two hands of the people that were worshipping, weren't.
like normal links, but they were words. Some of them were names of people. Some of them.
were just specifically sins. Some of them were emotions. Some of them were places, the.
names of cities and places. Some of them were events that have happened. And I could see.
that what was weighing people down was these individual things, maybe one thing that they.
had not been able to shake, despite how desiring they were to worship. They had been carrying.
this thing around and around with them. And I realized that God was speaking to me about.
my own habitual sin, about those one or two sins that over 40 years of my life, I've struggled.
to shake, how I have seasons of victory. And yet I find myself then falling back into that.
sin once again, and knowing that I'm not really truly free. But I know that the picture wasn't.

$^{161}$just for me. I think the picture was for you. I think the picture was for everybody here..
Because each one of us, I believe, have a desire to worship, and yet we all have things.
that weigh us down. We all have sin that we struggle with, sin that we struggle to shake..
We all have experiences that have hurt us in the past that we carry around with us like.
baggage. We all have emotions that have run deep in us, and that kind of have this tendency.
to come up in us and hold us back. There is stuff that each one of us carry. And if that.
resonates with you, then this series is for you. God had you on His heart five years ago,.
and He was desiring for you to truly be free. Because what you need to understand when you.
read the book of Exodus, and when we teach a series like this, is that when we open the.
book of Exodus, we're not opening a story about something that God has done in the past.
for one group of people, written down for all eternity, so we can read something about.
what He did then. What we're actually reading is God's personality, His character, who He.
is, determining it for all people at all time, and saying, "What I did for this group of.
people, I will continue to do, because I'm the same yesterday, today, and forever.".
In other words, what we have is not a God who's the God of the Exodus of the past, we.
have a God who's the God of the Exodus today. Your Exodus, my Exodus, the thing that I need.
to break the chains that hold me back in my worship, a God who steps in and does what.
only He can do. There is no sin that Jesus Christ cannot help you to overcome. I'm going.
to say that again. There is no sin that Jesus Christ cannot help you to overcome. There.
is no slavery more powerful than the name of Jesus. There is no Pharaoh that God cannot.
bring to His knees. There is no empire that God cannot overcome. There are no seas that.
God cannot part. There are no mountains that God cannot shake. There are no deserts that.
God cannot provide for. There are no promised lands that God cannot stop you from entering..
That's our God. We have a God of the Exodus. You have not come here to watch some visually.
engaging films and to be entertained for 25 weeks and learn something about the biblical.
book of Exodus. You've come here for biblical Exodus yourself..
All right. I love the golf clap. It's beautiful. It's like he got a bogey. It should have been.
a par, but he got a bogey. Thanks, man. That was a birdie, wasn't it? You can tell he got.
a golfer in the second row. I love you guys. All right. I want to show you God's heart.
for you because God comes to the Israelites right at the beginning of the story. He says,.
"I want you to know what I'm about to do." As we start this series, I feel in my spirit.
this is what God wants to communicate to you. It's Him saying, "This is what I want to do.
in you. This is what I'm about to do in you." In Exodus chapter 6, the context is God has.
called Moses crazy, burning bush moment. A lot of stuff's happened. God comes in this.
moment and Moses is now back in Egypt. He's already spoken to Pharaoh a couple of times..
Pharaoh's rejected him. God is reiterating to him time and time again what he's about.
to do. In this moment, Moses says to the Israelites through God what God is going to do in their.
Exodus. I want to read this to you from verse 6. "Therefore say to the Israelites, 'I am.
the Lord, and I will bring you out from under the yoke of the Egyptians. I will free you.
from being slaves to them, and I will redeem you with an outstretched arm and with mighty.

$^{201}$acts of judgment. I will take you as my own people, and I will be your God. Then you will.
know that I am the Lord God who brought you out from under the yoke of the Egyptians,.
and I will bring you to the land that I swore with uplifted hand to give to Abraham, to.
Isaac, and to Jacob, and I will give it to you as a possession, for I am the Lord.'".
God says here, and He does this in a number of places, but I think it's more beautifully.
put here. He says five things. I'm going to do five things. These are the five critical.
movements of Exodus. These are the five critical movements that the whole book of Exodus is.
shaped on. These are the five movements that I'm going to preach through as we go through.
this series, and these are the five things that will always be at work in your freedom.
from sin. And here they are. The first one is this. God acknowledges their slavery. "I.
will bring you out from under the yoke of the Egyptians." He knew they were in slavery..
The second thing is He gives them a promise. He says, "I will free you from being slaves.
to them." There's a promise. This is what I will do for you. I will free you from being.
slaves to them. He then tells them about the process of His liberation. He says, "I will.
redeem you with an outstretched arm which always spoke about miracles and power, and.
with mighty acts of judgment." That's how He's going to liberate them. He then speaks.
about the reason why, because He wants to shape a new identity in them. He says, "I.
will take you as my own people, and I will be your God." And then He tells them the result..
He says, "I will bring you to the land that I swore with uplifted hand to give to Abraham,.
Isaac, and Jacob. I will give it to you as a possession, for I am the Lord." Five things..
Slavery, promise, liberation, identity, home. Those are your five steps to freedom. It starts.
with you saying, "I know that there is some slavery in me, and I've been carrying it around.
for a long time. There are chains at times between my lifted hands and worship." It's.
then believing the promise that God has given, that in the life, death, and resurrection.
of Jesus, your sins have been paid for, and the price has been paid, and in the blood.
of Christ, you have been set free. Then it comes in trusting the journey of liberation.
that He leads us on through the power of His Word and the work of His Spirit in our lives..
And then it comes in receiving the new identity that we have in Christ Jesus. For anyone who.
is in Christ Jesus is a new creation. The old is gone, the new has come. And as we walk.
in the new identity that we've been given in Him, we know then that we don't fall back.
into the old habits, but we find ourselves liberated into the new. And as we walk into.
the new, we are those new people who carry the hope of Christ to the world, inviting.
people out there into a new home themselves, into an experience of relationship with Jesus..
And in this process of slavery, promise, liberation, identity, and home, we journey from our sin,.
into greater freedom to worship Him. And that's the powerful thing in all of this, that this.
movement, that this journey is to lead us towards something new. See, the Exodus narrative.
is about disruption. It's about being subversive in order to create something in order that.
might settle. I want you to see the journey here. There is disruption from our sin, disruption.
from our slavery, subversion of oppression and empire, so that God can create a new nation,.
a new identity in His people, a new people that would go forward carrying His character.

$^{241}$in the world, so He could settle them in a home where they could worship Him without.
rest. And this is the really important thing. You see, Exodus is not actually about freedom..
Really important you hear this, church. Exodus is not actually about freedom. Exodus is about.
intimacy. Because what God says in chapter three of the book of Exodus, He says, "I want.
to liberate my people." Why? "So they might worship Me." So they may not have the chains.
between their hands anymore. See, the beauty of Exodus is that there is a drawing out to.
draw in. God is not just liberating us from our sin so that we can experience some great.
freedom and then in our freedom fall back into slavery again. Paul would write to the.
church, he would say, "You are free, now be free." See, we are taken out of slavery so.
that we are then taken into something. What is it that we're taking into? We're taken.
into greater depth of intimacy with Jesus Christ. You are free in order to worship..
You are free in order to find intimacy with Him. And this drawing out to drawing in is.
a metaphor throughout the whole book of Exodus. Moses is drawn out of the weeds, out of the.
reeds of the river, now taken into Pharaoh's home. Israel are drawn out of their slavery.
into the desert, because the desert was to refine them before they got into their home..
And the same will be for you. God through His power is drawing you out of sin, not just.
so that you can enjoy freedom, but in your freedom, truly be free. To worship, to be.
in intimacy, to walk with Him, to experience life and love like it has never been before..
That's what Exodus is all about. And so there are these two metaphors that run throughout.
the whole thing. The first I've already mentioned to you, it's the individual and communal..
The individual journey to the communal. When you become Christian, you are not just saved.
out of your sin, you are saved out of your sin into the body of Christ. So in other words,.
you are individually saved. Christ comes and redeems and renews your brokenness and your.
sin, your sin that put Jesus on the cross. You're free and liberated from that sin through.
relationship and forgiveness and repentance in Christ Jesus. It's a beautiful thing. But.
then you're put into a new family, a new community, the body of Christ. And when we come to see.
this, when we see what happens in the Exodus story, where they're taken out of Egypt and.
put into their own land, as they individually experience their intimacy with Christ, they.
then come into intimacy with one another as a new nation. This drawing out to draw in.
together is a big part of what Exodus is all about. This is why we talk about community.
all the time here at the Vine. It's not just something that we kind of want you to do..
It's what you've been saved for. You've been saved to be in intimate relationship with.
the people that you're sitting around right now. And you can't know everybody in this.
church personally and intimately, but you should know some of them because the church.
is not 90 minutes on a Sunday, it's the people of God. You've been saved out of your sin.
into the body of Christ. And my prayer for you through this series is that you would.
feel the encouragement and the love of a growing community around you. It's a little bit like.
all of these little mosaic tiles that I'm holding in my hand here, which you probably.
won't be able to see very well because they're really, really, really small, but I'm going.
to hold one up right here. I got all these tiles from a little shop in Amman, the capital.

$^{281}$of Jordan, just some two and a half hour drive away from the desert where Israel wandered.
for 40 years. And as you came in today, perhaps, especially if you're sitting here in the lower.
house, you may have received one of these little mosaic tiles and you're probably wondering.
why was I given a little Lego piece. It's not a Lego piece. It's a little mosaic tile..
Let me tell you about why this is important. Because in many ways, the Exodus is like this.
journey where we go from being this one little thing, this one little person who's carrying.
all of the things that they carry, and we experience a liberation from God, which doesn't.
keep us alone, but that liberation brings us together with others. And the powerful.
thing about these little mosaic tiles is, of course, on their own, they're just like.
this little, small, little thing. But when you put it next to lots of other little, small.
things, the shop where I brought this from, they make these incredible mosaic art pieces,.
the most incredible things you've ever seen, all made up of thousands, literally thousands.
of little, little pieces of tile, just like this one. And as you came in, you were given.
one, and it represents so much who you are, but it actually represents a much bigger story.
that you're swept up in, a story where your little tile, if it's held back, keeps the.
mosaic from being completed. But when your little tile is placed in to the mosaic that.
God is creating here at the vine, that mosaic takes on a picture and an image that it had.
never experienced before. That's a drawing out to draw in..
The other thing that is the metaphor that runs through the whole of the Exodus story.
is the metaphor of new creation. In fact, much of the language of Exodus is mirroring,.
echoing the language from Genesis 1 and 2, the language of creation. And the Hebrew in.
Exodus links back to that first moment in Genesis, where when God moves and He acts,.
it's like He's deciding to start again. He's deciding to step into, like He did in Genesis.
1 and 2, where He stepped into darkness. And out of that formlessness of darkness, He brought.
light. He shaped and formed and created something. So God in the Exodus steps into a new darkness,.
the darkness of sin, the darkness of slavery, the darkness of oppression. And He steps in.
with a light of hope and the light of compassion and the light of new life. And He moves His.
people from their darkness of slavery into a new light. And light becomes this main theme.
throughout the book. God first appears to Moses in a burning bush, the light of that.
bush drawing him towards it. God appears before the whole of Israel in a burning cloud of.
fire on Mount Sinai. He then becomes a pillar of fire that leads them through the desert.
day in and day out. He then becomes this power of fire that's met in the tabernacle across.
the Ark of the Covenant as they erect the tabernacle when they stop on that journey..
And light becomes this idea that God is at work, present with His people, moving them.
from their slavery into something better, from darkness into light. And that's the journey.
He's bringing you on as well. The word Exodus is a Latin word that derives from a Greek.
word which is exodus with an O at the end. And exodus means this, a departure. I think.
that's such a beautifully profound thought. Because what we're doing today is we're drawing.
a line in the sand and we're saying today is the day of our departure. See, no journey.
to freedom begins unless you take that first step. I love that moment in the Lord of the.

$^{321}$Rings where Samwise and Frodo are moving out finally after being given the mission by Gandalf..
And there's that moment where Samwise stops Frodo and Frodo's like, "What are you doing?".
And he's like, "I've never been this far away from home than right now." And he takes that.
next step towards his adventure, a departure. See, your exodus begins when you're willing.
to make a departure, when you're willing to say, "I'm tired of the darkness," when you're.
willing to say, "I'm so over this thing that continues to hold me back," when you're willing.
to say, "I admit that there's a habitual sin in my life that I can't seem to break," when.
you're willing to say, "I can't fix this myself, but I need God to come and do something that.
only He can do," when you're willing to say, "This is not actually always about me. This.
is about my family. It's about the people around me. It's about my community. It's about.
me becoming a better version of myself for them. It's about you saying, 'This is my moment.
where I can move into a greater sense of freedom than I have ever felt before.'".
So as we start this series on exodus, here's my key question to you. What is it that you.
need exodus from? What is it that you need to depart from? What is your exodus? That's.
the question. That's the question that has driven this whole series for me. It's the.
reason why we've taken four years to pull this together. It's the reason why our church.
has invested the money that it has into this series. It's the reason why we're taking 25.
weeks to walk through this series together. It's the reason why I'm going to get up here.
and preach my heart out to you as much as I can week in, week out, is because I believe.
this is your moment. I believe that that picture of us with chains between our hands as we.
worship will not be the picture of this church moving forward. I believe in your freedom..
I believe in your release, but it will require a departure. You need to choose to move from.
darkness to light. To help you to reflect a little bit more about this journey from.
darkness to light, as well as all the teaching films that we've created for this series,.
and a number of times, critical times throughout the series, we've also created some creative.
videos. These videos are not so much about teaching, but more about an experience. One.
of these videos, the one we're about to watch now, is a poem that we wrote as a team that.
put this together. We wrote this poem to help us to understand the profundity of what it.
is to be trapped in darkness, and yet the beauty that there is in a God who steps forward.
to bring light. As we watch this creative film now, I want to pray that you would do.
so in a prayerful spirit, asking that God would come and speak to you about what your.
departure is today. Then after that, I'll come and pray for us. Let's watch this..
It begins in the emptiness of nothingness, where darkness has extinguished even the memory.
of light, where nothingness and emptiness tear space into fragments and everything is.
nothing at all. It starts here, where fear shackles and failure persists, where slavery.
binds and oppression resists. It starts here, in the place you least expect, suddenly a.
light, dancing to be noticed within the empire of darkness, crackling with beginnings, restless.
with life, as if the emptiness had simply been waking for divine direction. It starts.
here. Shadows now play against the walls of death, movement and memory and miracles and.
might breaking into nothingness with the heresy of possibility, shaming the darkness with.

$^{361}$the creation of hope. It starts here. A drawing out, a calling forth, a voice in the wilderness,.
a wilderness of voices, whispers of freedom, bushes on fire, a journey beyond, ascending.
to glory. It starts here, in the emptiness of nothingness, in the slavery and oppression,.
in the fear and the failure and the pharaohs in us all. May you see what heaven is stirring..
For a light has come and a light is coming, death now dead, bushes now burning. Yes, a.
light has come and a light is coming, death now dead, bushes now burning..
[Music].
[BLANK AUDIO].
\newpage



\section{}
\label{sec:Bq85h5xoYT4}
\textbf{2023-04-18 It starts here.. [Bq85h5xoYT4].mp3}
\newline
\newline
連結: \href{https://youtube.com/watch?v=Bq85h5xoYT4}{\texttt{ https://youtube.com/watch?v=Bq85h5xoYT4}} ~~~~ 語音日期: 2023-04-18 
\newline
\newline
\hyperref[sec:sQhUi8jUN2g]{\small{< < < PREV SERMON < < <}}
~
\hyperref[sec:index]{\small{[返主目錄]}}
~
\hyperref[sec:cz4a6_st6To]{\small{> > > NEXT SERMON > > >}}
\newline
\newline
$^{1}$It begins in the emptiness of nothingness, where darkness has extinguished even the memory.
of light, where nothingness and emptiness tear space into fragments and everything is.
nothing at all..
It starts here, where fear shackles and failure persists, where slavery binds and oppression.
persists, it starts here, in the place you least expect, suddenly, a light..
Dancing to be noticed, within the empire of darkness, crackling with beginnings, restless.
with life, as if the emptiness had simply been waiting for divine direction..
It starts here..
Shadows now play against the walls of death, movement and memory and miracles and might,.
breaking into nothingness with the heresy of possibility, shaming the darkness with.
the creation of hope..
It starts here..
A drawing out, a calling forth, a voice in the wilderness, a wilderness of voices, whispers.
of freedom, bushes on fire, a journey beyond, ascending to glory..
It starts here, in the emptiness of nothingness, in the slavery and oppression, in the fear.
and the failure and the pharaohs in us all..
May you see what heaven is stirring..
For a light has come and a light is coming, death now dead, bushes now burning, yes a.
light has come and a light is coming, death now dead, bushes now burning..
[Music].
(music fades).
\newpage



\section{}
\label{sec:cz4a6_st6To}
\textbf{2023-04-24 EXODUS - 02 Midwives and the Power of No [cz4a6-st6To].mp3}
\newline
\newline
連結: \href{https://youtube.com/watch?v=cz4a6-st6To}{\texttt{ https://youtube.com/watch?v=cz4a6-st6To}} ~~~~ 語音日期: 2023-04-24 
\newline
\newline
\hyperref[sec:Bq85h5xoYT4]{\small{< < < PREV SERMON < < <}}
~
\hyperref[sec:index]{\small{[返主目錄]}}
~
\hyperref[sec:LXiCHoxSu6Y]{\small{> > > NEXT SERMON > > >}}
\newline
\newline
$^{1}$Lord, we thank you for this in Jesus' name..
Everybody says, amen..
Can we thank our team?.
Wonderful..
Have a seat, have a seat..
Say hi to somebody, say hi to someone..
Hi to everybody in the overflow..
It's great to have you with us as well..
Welcome to us..
And so good to see you..
Anybody online, it's great to have you two with us..
We are so glad you're here..
How's everybody doing?.
You're right?.
We are in week two of our Exodus series here at the Vine,.
and we're studying the book over Exodus over the next.
six months or so..
And this week, we actually get into the text itself..
This week, we find ourselves right at the beginning.
of Exodus chapter one..
And this week, we're also unveiling for you.
a couple of tools that we've created in this series.
so that you can really dive deep into Exodus..
You see, for us, our heart is is that Exodus.
is not just a sermon series that you come to..
Exodus is a book that you study yourself.
over these next six weeks, next six months..
And we want you to find yourself opening the pages.
of scripture during the week..
We want you to be thinking and reflecting.
about the things that we speak about on Sunday..
But more than that, we want you in the text itself..
And I feel like if all you do in the series.
is come on Sunday and listen to me or other preachers.
unpack a text, that's one thing..
But if you're then diving into the scriptures yourself,.
if you're then asking God for Exodus yourself,.
if you're opening the word during the week.
and saying, God, I need this, this is a reality of my life,.
I can tell that I've got some slavery that I'm a sin to,.

$^{41}$that I'm broken and holding back from,.
and Lord, I want freedom myself, and I'm opening the pages,.
and I'm grateful that I get what I get on Sundays,.
but that's not enough for me..
I want the scriptures to come alive for me..
- Yeah..
(laughing).
- Oh, we got some ways to go..
That's good..
So we've created a couple of tools to help you.
to get into the word..
First of all, I want you to know this series is structured,.
I told you last week is 24 weeks..
We're gonna structure it in two segments of 12 weeks each..
So between now and the end of June is part one..
Part two will begin at the end of August,.
and we'll go all the way to the middle of November..
There'll be a break for the summer.
for about six weeks or so in July and most of August..
But we're breaking it up in two big blocks..
Now, one thing that I also said last week.
is that the whole book of Exodus.
is structured around five movements..
That's the way it's written..
It's written in these five movements..
It's also the way that we believe that all Exodus.
from any slavery into freedom happens in our lives.
through these five movements..
God uses them today,.
just like he did for the Israelites back then..
And those five movements, as we talked about last week,.
are slavery, promise, liberation, identity, and home..
And so we've structured this whole sermon series.
around those five movements..
So we want you to be able to track every single week,.
where are we in the book of Exodus?.
Like what passage are we in?.
We also want you to be able to track with these five things.
to help you to know how we're progressing.
through that journey..

$^{81}$So in order to do that, we've created a tool..
We've actually made a bookmark for you.
that you can either put in your physical Bible,.
or you can put in another book,.
or you can stick it on your fridge..
The bookmark looks like this..
That's the front and the back of it..
And what you'll see, every single week,.
it has the date of that Sunday..
It has the topic that we're speaking about,.
but most importantly, it's got the passage of scripture.
we're preaching from or looking at as a church.
on that Sunday, so that in preparation, you can pray,.
you can read that passage for yourself..
You can come on a Sunday already soaking in the word of God..
And what you'll notice here is we've got the five movements,.
slavery, promise, liberation, identity, and home..
So again, you're able to track through the whole series..
So if you miss a couple of weeks.
and you're coming back to the vine,.
maybe you're away for business or whatever, or holiday,.
you come back, you might be like,.
"Where are we in the story?".
Well, you've got this..
It'll tell you exactly where you are..
It'll tell you what passage we're in,.
and it will tell you which movement we're a part of..
Is that helpful?.
You will get that next week..
Yeah, babies..
You gotta come back for that piece..
You gotta come back for that one..
We're gonna hand that one out to you guys this week..
But we do actually have something.
for you to take home this week,.
and I'm really excited about this..
Again, our heart in this series.
is not just that you hear something on a Sunday,.
but you'll begin to study Scripture yourself..
And so one of the ways that we've done that,.

$^{121}$our creative team have done an amazing job..
We have created a coffee table book for you on Exodus,.
and particularly Exodus part one,.
between now and the end of June..
What this is is a beautiful book.
filled with incredible images and resource..
Some of the stuff that we took ourselves.
as part of the filming, others that we have licensed..
But what's most important in this.
is that every chapter is a devotion that we've written.
that connects to a specific passage.
that we're preaching that week..
So the whole point of this.
is that we want you to come on Sunday,.
hear what's said on Sunday,.
but then we want you to go away.
and read a deeper reflection.
on what we spoke about on Sunday in your own time..
These are written in a way.
that you can't do this in five minutes..
This is not a brush your teeth devotion..
This is a 15 to 20 minute,.
read it, pray about it, soak in it,.
which is why we've made it a book..
It's something that you can keep at home,.
put in your living room, put by your bed,.
whatever you wanna do, and utilize it that way..
These devotions have been written.
by one of our congregation members..
His name is Chris Webster..
He comes to our 2 p.m. service..
He is a phenomenal writer, theologian, and thinker..
And his reflections, he wrote,.
I actually reached out to him four years ago..
I'm like, "We're gonna do this crazy thing.
"called the Exodus project..
"We're gonna do this whole thing..
"We're gonna make films..
"I'd love you to write a devotion.
"for every single week of the series.".

$^{161}$And so he went away..
He was very grateful that we delayed, by the way,.
so he could keep writing..
And he went away and he wrote all the devotions,.
and he wrote them without knowing.
what I was gonna say on a Sunday..
So it sits completely separate to what I'm gonna teach you,.
which is why it's a beautiful compliment for you..
And here's the amazing thing..
You get to take one of these home today, okay?.
We have one of these for every single person here,.
and it's completely free..
Yes, completely free for you..
You know, one of the things that we realize.
here at The Vine,.
we're a broad socioeconomic demographic here,.
and we don't want anyone to miss out on this resource.
if you can't afford it..
So we decided to give it away for free..
There is a catch, though, two catches..
Number one, we only have enough.
for physically everybody across our three services..
So what I do not wanna see after the service.
is you walking away with 20 copies like this,.
'cause you wanna give it to your friends and family.
who don't come to The Vine, okay?.
This is for us here at The Vine..
We got enough copies for every single one of you.
to take a copy, but please do not take more.
than just your own personal copy this week..
If we have any left over at the end of this week,.
we will make them available next week,.
and you can take as many as you want next week, okay?.
But for a starting point, just one today, is that okay?.
Here's the second little catch..
Although I say it's free, and it is,.
if you would like to make a donation towards its cost,.
we would be very helpful,.
and that would be very good for us..
So we're suggesting \$100 donation for this.

$^{201}$if you would like to..
You don't have to, again, hear my words..
Please take it home for free, okay?.
But if you are able to, and you'd like to give 100,.
or maybe more than that,.
to help with the cover of the printing of it,.
we would love for you to do that..
You're gonna get this..
It's gonna be on the second floor after the service.
in the upper house..
Hey, everybody in the upper house, you get a copy as well..
Just come down the stairs after the service,.
and you can grab it from the lobby here..
You don't need to rush..
There's enough available for everybody..
Is that awesome?.
- Yes. - All right, all right..
So, now, there's another reason.
why we've made you this amazing resource.
that we wanted to generously give to you today,.
because it actually connects.
to what the starting point of Exodus,.
the book of Exodus, is all about,.
because the cultural and contextual backdrop.
to the first verses in Exodus.
is the idea of abundant blessings.
that have been handed and given out..
The book of Exodus begins right off the back.
of the narrative from the book of Genesis..
There's no break..
They flow from one to the next..
And what we see right at the end of Genesis.
is something that has happened to Joseph,.
Joseph, the 11th son of Jacob,.
Jacob, who's the grandson of Abraham..
And we see right at the end of Genesis.
that Joseph is sold into slavery by his own brothers,.
and he is trafficked to Egypt..
And although his circumstance is a terrible beginning,.
God's favor and honor is on Joseph..

$^{241}$And whilst Joseph is there, Pharaoh has a dream..
And Pharaoh's dream is very disturbing to him..
And he asked for his own court,.
magicians and astrologers to interpret the dream..
No one can interpret the dream..
And somebody says, "There's a Hebrew in our land.
who has been able to interpret other dreams.
for other people..
He might be able to interpret for you.".
So Pharaoh calls Joseph to him,.
and Joseph interprets the dream through the power of God..
And in the dream, Joseph sees.
that what God is saying to Pharaoh.
is that there is a time coming of abundance.
and a time coming of famine..
And Joseph basically says to Pharaoh,.
"Hey, we need to start storing up food now.
in the time of abundance,.
because the time of famine is coming..
And if we don't have those resources available,.
then we're gonna really suffer in famine.".
Well, Pharaoh agrees, believes in Joseph..
In fact, he actually empowers him.
to become the number two person in power over all of Egypt,.
in particular, over the land of Egypt..
And so Joseph goes and he builds storehouses..
He begins to take the grain..
He stores it there..
And when the famine years come,.
the nations around Egypt are really suffering..
Famine hits hard, and it hits the people of Egypt hard too..
But because of Joseph's insight, his wisdom,.
and his hard work,.
there is the abundance available for Egypt..
Now, at this time, Joseph then goes to Pharaoh..
And he says, "Look, my family,".
talking about his father, Jacob, and his remaining family..
At that time, there was about 70 people in Jacob's family..
Joseph goes to Pharaoh and says,.
"Can I bring Jacob to Egypt?.

$^{281}$Can I bring him here as a refuge?".
'Cause this land has an abundance right now..
And basically, Pharaoh,.
because of all the incredible things.
that Joseph had done for him, Pharaoh agrees..
And at that point,.
Jacob and his family are able to travel to Egypt..
Now, I'm gonna show you a map..
This is a modern map of Egypt..
We're gonna use this map a lot throughout our series..
It is a modern map rather than one in the ancient times,.
but that's because it connects.
to a lot of what we're gonna be teaching for us..
This area up here roughly was Canaan.
at the time of the beginning of the book of Exodus..
The modern day Israel wasn't yet boundaried like this..
They lived in this area up here..
And so it was from there that Jacob.
and his 70 people would travel..
Now, this is Cairo, the modern day city of Cairo..
This is the Sinai Peninsula.
where a lot of the Exodus narrative takes place..
This is Mount Sinai right about down in here..
Now, very importantly, when Jacob and his family comes,.
right around about here is a land called Goshen..
In fact, we've got another map here.
that makes that clear to you..
This area right here became known in the Bible as Goshen..
And Goshen was the area that Pharaoh gave to the Israelites.
to be able to settle in..
And this is where Jacob came..
This is where Joseph would have spent time as well..
And this is where the Israelites.
were able to take refuge during famine..
They were able to receive the food.
that had been prepared for them..
And this is the starting point of Exodus chapter one..
We're now gonna start our journey..
Are you excited?.
All right, here we go..

$^{321}$Exodus one, verse one..
These are the names of the sons of Israel.
who went to Egypt with Jacob, each with his family..
Reuben, Simeon, Levi, and Judah,.
Iscah, Zebulun, and Benjamin,.
Dan, Naphtali, Gad, and Esher..
The descendants of Jacob numbered 70 in all,.
and Joseph was already in Egypt..
Now Joseph and all his brothers.
and all the generations died..
But the Israelites were fruitful and multiplied greatly.
and became exceedingly numerous.
so that the land was filled with them..
Now I said last week that one of the things.
the author of Exodus, Moses,.
who also authored Genesis, the same person,.
he's writing the book of Exodus using language and imagery.
that connects to the Genesis creation event.
of Genesis one, two, and three..
And right here at the start of the book of Exodus,.
we see a great example of this..
I wanna read to you verse seven once again..
It says, "But the Israelites were fruitful.
"and multiplied greatly and became exceedingly numerous.
"so that the land was filled with them.".
The words that Moses is choosing here are very specific.
'cause he's linking what the starting point is for Israel.
at the beginning of Exodus to the call that humanity had.
in Genesis chapter one and the call and the blessing.
that God gave humanity..
Let me show you Genesis 1, 28, which also Moses wrote..
"God blessed them and said to them,.
"be fruitful and increase in number,.
"fill the earth and subdue it,.
"rule over the fish of the sea, the birds of the air,.
"over every living creature that moves on the ground.".
So exactly what Moses is doing here.
is he's saying at the beginning of the Exodus narrative,.
there is this connection to the beginning.
of God's call on humanity where God said,.

$^{361}$"I'm gonna bless you..
"I'm gonna bless humanity, male, female,.
"I've created you, you're made in my image.
"and you're gonna be a blessing..
"And here's the blessing,.
"you're gonna be able to multiply,.
"you're gonna be prosperous,.
"you're gonna be able to fill the earth.".
And so when Moses starts Exodus, he says,.
"You need to know that my people are in a blessing..
"You need to know where they are..
"They're in this place where they're receiving.
"a great blessing from Egypt..
"This place is abundant blessing to us..
"And in that, we're living out the call of humanity.
"right in this land.".
Now we know what happens next in the story..
We know that slavery comes..
But it's really interesting that Moses,.
before he tells us about that,.
he starts with the Genesis 1 and 2 picture..
In other words, it's like, almost like Moses is saying,.
"Our time in Egypt for 380 years before slavery happened.
"was like our garden of Eden..
"It was an abundant place..
"We had favor there..
"Pharaoh liked us..
"There was a beautiful environment.".
Now, yes, that turns into Genesis 3..
That turns into the slavery that's to come..
But the starting point is good..
And some of you resonate with this in this room..
The journey that there is.
between sometimes something starting well.
and ending up really bad..
Some of you in here, it's like a relationship..
It started so well, but it's turned bitter,.
and it's painful now..
For some of you here, it's like a job..
The job started great..

$^{401}$You loved it at the start..
It was a great blessing to you..
But now you've become bitter, and it's annoying,.
and you hate it, and you wanna do something else..
It's interesting how fickle the human experience is..
And Moses wants to pick up on that right at the start.
and say, "Know that there was blessing..
"Know that there was favor..
"But it's so easy for that favor to change in a heartbeat.".
Now, at this point, all we have is Moses' word.
that this happened to Israel and happened to Egypt..
But what actual evidence is there?.
I mean, before we look at anything else in the Exodus story,.
surely we have to start by saying,.
"This is actual historical truth..
"We need to know that this is not just.
"some fairy tale myth, but it actually sits.
"in some actual historical truth.".
We're not gonna just take Moses' word for it..
What is there in Egypt itself today.
that helps us to understand that this is the reality.
for God's people, that this actually took place?.
Well, here's an amazing thing..
When I started researching this whole project for Exodus,.
and I was researching about the evidence that exists,.
I discovered a really interesting thing..
There's hardly any evidence in Egypt.
that Israel even lived there,.
let alone that the Exodus happened..
There's very, very, very little evidence at all..
And that should concern us..
Because what is this thing then?.
Is this book just some story that's been made up?.
Or does it root itself in actual archeological evidence.
and actual historical events?.
Well, to answer that question,.
let me take you now back to Egypt..
(dramatic music).
The first thing you need to know about the Exodus.
is that it's more than just an event..

$^{441}$In fact, it's perhaps the most significant demonstration.
that we have in all of our scriptures.
of God's dealing on a daily basis with humanity,.
of his power and control over nations,.
and of course, his ability to literally change.
the course of history..
But the Exodus is more than just old history..
It's also alive still today..
I mean, it's celebrated regularly by Jews.
at the Passover every year, and by Christians at Easter..
But perhaps more than that,.
the Exodus has become a metaphor for revolution,.
for those of oppression.
to the social injustices of the world..
I mean, from Mahatma Gandhi, from Martin Luther King Jr.,.
from Bob Marley, all of them have evoked.
the imagery of the Exodus,.
and have spoken about Moses' cry, "To let my people go.".
All of which is fascinating.
when you consider one sobering fact..
There's actually very little archeological evidence.
that Israel was ever in Egypt,.
let alone that the Exodus actually happened..
I mean, could it be that the Exodus is some fairy tale myth.
invented by religious leaders to teach us a moral.
about good versus evil, oppression versus freedom?.
I mean, could it be that we've been believing a lie,.
albeit a very inspiring and motivating one?.
Well, to begin to answer that question,.
I've come here to Beni Hassan,.
a small village just south of modern-day Minya.
in Middle Egypt, to a series of remote tombs.
that are cut into the hillside.
to uncover perhaps the only physical proof.
there is in this whole country of Semitic people.
arriving in the land of Egypt itself..
This is a breathtaking place to bury your dead..
The people buried here are actually.
from the elite class of nomachs..
They were provisional governors here.

$^{481}$in the Middle Kingdom period,.
about 4,500 years ago..
Now, there are 39 tombs here in Beni Hassan,.
and each one of them has an outer court,.
an inner pillared room, and then a shaft.
that leads down to the burial chambers..
But we haven't come all this way.
for me to show you things that are dead and buried..
Today, we've come here so I can show you something.
that is still very much alive..
These tombs are covered from floor to ceiling.
in vibrant, beautiful, living art..
The paintings here depict everyday scenes.
of the common life of this area at the time,.
as well as stories about their history and livelihood,.
their homes, and their culture..
Standing in the tomb here, I really get a sense.
of the sheer scale of this place,.
which is why it's a little bit boomy and echoey..
But when you look at the beauty in the art here,.
it amazes me 'cause it tells me something.
about the culture of this people,.
that at the end of their lives, when they bury themselves,.
they want to be surrounded with story..
They want the future generations to come here.
and not learn about how they died, but how they lived..
They wanted to tell them about their experiences.
and help them to get to know them personally..
I mean, this blows my mind..
I mean, us in the West, when we die,.
we just erect a tombstone with a couple of words on it..
But in this culture, they wanted to really communicate.
the heart of who they were to the future generations..
And as a storyteller myself, this truly just inspires me..
And it's actually amongst all of this storytelling.
that our Exodus journey truly begins..
This might be a small section.
of the overall paintings in these tombs,.
but its significance to us cannot be overstated..
What you see here are figures wearing a fashion.

$^{521}$that is completely different to any other people.
depicted in these tombs..
Their upper bodies are covered,.
and their clothes are of a striped red and spotted nature..
This was the clothing of the Semitic people of the time..
And the scene here depicts a story of Asiatic migrants.
arriving in Egypt with the hopes of starting a new life,.
making this the only recorded history we have.
in the whole of Egypt of the arriving presence.
of a Semitic people in this land..
From religious fairy tale myth.
to recorded historical evidence,.
these tombs tell us a story we all need to hear..
God's people did indeed come to this land..
But the question remains, did they actually settle?.
Well, to answer that question,.
I need to take you about 200 miles north.
to a land that the Bible calls Goshen..
(dramatic music).
These fields behind me here may not look like much,.
but hidden underneath them.
is some of the most significant archeological evidence.
for the existence of God's people here in Egypt..
This area around me is known as Goshen..
It's found actually in Genesis 47,.
as well as in Exodus chapters eight and nine..
And Goshen is the very place where God's people lived.
for the 480 years that they were here present in Egypt..
Now, unfortunately, much of that evidence.
is still under the ground here..
But there is one place,.
not too far from where I'm standing right now,.
where they've been able to excavate..
And what they found there is truly significant..
Unfortunately, it's not a public place,.
so I wasn't able to get in there and film,.
but we were given some footage..
And that footage reveals to us.
what it would have been like for God's people.
to live in this land..

$^{561}$These pictures and artist renderings.
show what is essentially the site.
of an ancient home here in Avaris..
In the excavation, you can clearly see various rooms,.
as well as the walls.
and the general structure of the house as a whole..
Now, the form of the house is not in the style.
of Egyptian homes in the time of the Exodus,.
but it's actually North Syrian in its form and construct,.
the area that the patriarchs were originally from..
This particular house has become known.
as the House of Jacob,.
and archeologists have found something fascinating.
in the garden surrounding it,.
a portico with 12 pillars..
Is this coincidence, or is it further evidence.
of the presence of the settlement of a Jewish people.
celebrating in their architecture.
the 12 tribes of their generation?.
Either way, the home that is being discovered.
not too far from where I'm standing right now.
proves to us something beyond the shadow of a doubt..
You see, Beni Hassan showed us.
that a Semitic people arrived in Egypt..
The House of Jacob shows us that they settled here..
(dramatic music).
- It's a really important thing.
that we can base the story of Exodus.
in actual recorded history for us..
But as I said earlier,.
sometimes the fate and the blessings of somebody.
can change in a heartbeat..
And as is so often the case when a immigrant group.
grows to prominence and power within a nation,.
the majority group of that nation can often fear.
and respond in tragedy and evil..
And that's exactly what we see happen here..
Let me read on from verse eight..
Then a new king, and by king, when it's mentioned here,.
it's always referring to a pharaoh..

$^{601}$Then a new pharaoh, who did not know about Joseph,.
came to power in Egypt..
"Look," he said to his people,.
"the Israelites have become much too numerous for us..
"Come, we must deal shrewdly with them,.
"or they will become even more numerous..
"And if war breaks out, we'll join our enemies.
"and fight against us and leave the country.".
The pharaoh begins to freak out,.
that this pharaoh has no connection to the Jewish people..
He has no connection to the pharaohs of the past.
who were influenced by Joseph and his wise decision-making..
He has no connection..
It's been over three generations have passed.
as this person now comes into power..
And because he's got none of that connection,.
he just sees this Israelite group.
as a massive group of people.
that he knows are not Egyptian, and he fears them..
Now, it's really interesting,.
'cause Moses uses a specific word in verse 10.
to describe this..
He says, "Come, we must deal," this is pharaoh speaking,.
"Come, we must deal shrewdly with them.".
That's exactly the same word that's found in Genesis three,.
when Moses writes about Satan as a snake..
He says the snake was more shrewd or crafty.
than any of the other animals..
Remember, this is still Moses writing..
He takes that same word that he applies to Satan,.
and he puts it right on pharaoh straight away..
And basically what he's saying here is pharaoh.
is in this narrative going to be like for us the snake..
He's going to be the one.
who is gonna have influence over us,.
and he's the one that's gonna make decisions.
that are of the enemy, not of God..
And Moses wants us to understand the satanic influence,.
the evil that sits behind all the things.
that pharaoh does next..

$^{641}$Let me read it to you from verse 11..
So they put slave masses over them.
to oppress them with forced labor..
They built Pithom and Ramesses as store cities for pharaoh..
But the more that they were oppressed,.
the more they multiplied and spread..
So the Egyptians came to dread the Israelites.
and worked them ruthlessly..
They made their lives bitter with hard labor.
in brick and mortar, and with all kinds of work in the fields.
and in all their hard labor,.
the Egyptians used them ruthlessly..
There's this power that comes from Egypt on Israel..
They're thrown into slavery,.
and that slavery is painful, it's hard,.
and it's in the totality of who they are..
Now, I want you to see something really important here..
What Moses is trying to communicate to us as we read this.
is a simple starting point of slavery..
He's trying to help us to understand.
that there is a key starting point.
that happens so often in slavery,.
and that's the starting point of insecurity..
You see, Pharaoh here acts insecurely before his people..
He looks at this migrant group who've grown to a big size,.
and they've never done anything to make Pharaoh.
think that they're gonna rebel against him,.
but he's so afraid of that,.
and in his own insecurities of losing control.
and losing his own power,.
he decides to suppress them.
in order to make sure that he would increase..
And what you need to know is that insecurity.
is so often the primary brokenness.
that actually enslaves and oppresses other people..
Say that again..
Insecurity is primarily, often,.
so often the primary brokenness.
that is the driving factor.
that would enable people to be enslaved and oppressed..

$^{681}$That's what insecurity does..
And Moses is trying to get us to know this.
because actually throughout the whole Exodus journey,.
you're gonna see time and time again.
the issue of insecurity rising up.
and then the issue of needing to deal with it before God..
This is why one of the movements.
of the five movements of Exodus is all about identity,.
because when our identity is broken,.
we operate out of a place of insecurity..
And insecurity happens because we look at ourselves.
and we look at others and we make comparisons between them.
and we make assumptions between them..
And you'll recognize in your own life.
that your insecurities will always tempt you.
to suppress others in order to elevate yourself..
Are you guys with me?.
And if that's the case,.
here's the key that Moses is giving us.
right at the start of the story..
He's saying, if there's one thing that you could do.
to align yourself to the process of Exodus,.
if there's one thing you could do to position yourself.
for more freedom in your future,.
it's deal with your insecurities..
And how do we deal with our insecurities?.
We don't deal with them.
by trying to become more secure in ourselves..
We don't try to convince ourselves.
that we're actually not as bad as we think..
We don't try to convince ourselves.
that we've got power ourselves..
No, what we do, this is what the Bible says,.
the only way you free yourself from insecurity.
is not finding security in yourself,.
but finding more security in God..
By actually looking out of yourself and looking to Him.
and saying, I will follow Him,.
I will trust Him, I will put my promises,.
my life, my faith in Him..

$^{721}$As I find my security in Him,.
I'm able to deal with the insecurities that sit in me..
See, as we become more secure in Christ.
and who we are as His children made in the image of God,.
we become less a vehicle of slavery and injustice.
and more a vessel of freedom and grace..
That's the journey of Exodus..
And Moses is saying, get that right here,.
right at the start..
And I wanna speak that over every person in this room..
If there's one thing I invite you to deal with.
over the next 24 weeks is to bring your insecurities to God..
He's got you..
And He, let me say it this way..
There is no other name more powerful than the name of Jesus..
We just sang it, but we sing it..
Do we believe it?.
And if we believe it,.
then we're able to move away from our tendency.
to enslave and oppress others.
and become more like Christ.
wanting to bring freedom and grace in this world..
And there's an army rising up.
of people that are willing to do that..
That, my friends, is an echo of Exodus..
Now, what does this slavery feel like?.
What was it like to be found in slavery in those days?.
What was it like to actually be oppressed?.
Was it just a physical thing or was it much more than that?.
And were men and women enslaved in the same way?.
Well, to answer those questions,.
let me take you back once again to Egypt..
We are introduced to the starting point.
of the Exodus narrative in Exodus 1:8-11,.
where we learn of a new king who comes to power in Egypt.
and immediately places Israel into slavery..
I mean, literally overnight, their freedoms are gone.
and they're suddenly thrust into forced labor,.
constructing storehouses.
as well as temples of Pharaoh across the land..

$^{761}$One such temple is here at Tel Basta,.
some 80 kilometers northeast of modern-day Cairo..
Built for the worship of the feline goddess Bastet.
some 3,500 years ago,.
the temple now lays here in ruin.
but gives us a unique insight.
into the kind of slavery the Jewish people were placed under..
Standing here today and being surrounded by all of this.
is an incredibly rare privilege..
I mean, these are 3,500-year-old antiquities..
And being here, I get a great sense of the grandeur.
and the size that this temple would have been..
It would have taken thousands of slaves years to construct it..
And take, for example, the granite here..
This granite is not from around here..
It's probably from a quarry about 1,000 kilometers away..
So the slaves would have had to have dragged this here.
and then shaped and constructed it on site..
I mean, that's incredibly backbreaking work.
in the harsh climate of this land..
Not only this, but the slaves also would have had.
to spend their days carving out idols of gods.
that they refused to worship.
and worldviews that they refused to accept..
Case in point is seen here..
This is the image of Ka..
It's two arms pointed to the heavens,.
which in ancient Egyptian religion symbolized the soul..
But notice something..
There's no head and no real body to it..
And that was on purpose,.
'cause this was communicating that our connection.
of our soul to the gods is devoid.
of any of our personal identity..
Now, think about the Jewish slaves and their worldview..
They worshiped a god that they saw as a good father,.
who they understood created them in the image of their god..
So the idea of kind of taking away the identity.
from our worship of God.
would have been actually deeply offensive to them..

$^{801}$All of which actually raises a really important point.
for each of us to reflect on.
as we think about our own journeys from slavery to freedom..
You see, slavery is never just merely physical..
It involves the whole of who we are..
The mental, emotional, and social toll on the Israelites.
in constructing this temple was just as cruel.
as the physical toll that was required to do it..
And I think actually that that was the main point..
You see, in constructing this temple,.
it was designed not just to break.
the Jewish people's physical backs,.
but their spiritual ones as well..
There was one further aspect of slavery.
that was unique to the Israelites here in Egypt.
and actually profound for shaping.
the whole of the Exodus itself..
And to tell you about that,.
I now need to take us from the temple here in Tel Basta.
to the remote turquoise mines of Sinai Peninsula..
The Sinai Peninsula is nicknamed the land of turquoise.
by the local Egyptians that live here..
The mining of it dates back to the first dynasty.
when Egyptians sourced it from the coastal mines.
and mountains of this area..
And while much like today,.
it was originally used for jewelry and ornaments,.
its adoption by many pharaohs created a mythology around it.
centered on the idea that the stone was a protector.
and bringer of good fortune..
So many Egyptian women wore the stone.
in hopes of becoming pregnant.
that the goddess of fertility, Hathor,.
was eventually called the mistress of turquoise..
(gentle music).
The turquoise that is mined in this area.
is some of the purest in the world.
and it really is so vivid and so beautiful..
But as is often the case in life,.
sometimes the things that are most beautiful on the outside.

$^{841}$are actually distracting us.
from the things that are dark within..
This is a small example of the mines.
that are found scattered in the mountains in this area..
Now, the actual turquoise itself is mined from the veins.
that you can see here in the ceiling and on the walls.
and then sent for processing at the plants in this region..
Now, the earliest evidence suggests.
that the mining was done here.
by local Sinai inhabitants at the start..
But as the turquoise gained in its popularity and demand,.
they needed to ship in some additional labor force..
So it was here in the hills of Sinai.
that many Jewish slaves were forced to come to work,.
spending countless hours in cramp and awkward conditions,.
often in the dark, digging and excavating.
for the precious stone that would go on.
to adorn the rich and powerful..
And here is what is important about all of that..
You see, unlike the muscle and brawn.
that was needed to construct the temple in Tel Basta,.
the work here in these caves was intricate and detailed.
and very repetitive..
It was work that was done largely by women..
I want you to think on this..
Imagine a thousand enslaved Hebrew women in these caves,.
working day in, day out to mine a precious stone.
that would go on to be adorned by Egyptian women.
to symbolize their freedom and power..
And if that wasn't kind of ironic enough,.
the actual ancient Egyptian goddess of justice.
came to be symbolized by turquoise..
Yeah, the goddess of justice..
I think for the Jewish women enslaved in these caves.
for hours upon hours, surely turquoise was anything but..
So slavery impacted both men and women alike,.
whether it was in temples or in caves,.
in cities or in deserts,.
the Jewish people were placed under a brutal regime.
that impacted their mental, physical, social,.

$^{881}$and spiritual lives..
Indeed, the oppression was so strong.
that in order to stop it,.
it would require an equal but opposite force.
that was so powerful that actually the whole of history.
itself turned on its axis..
And how did that power begin?.
Well, Exodus chapter one introduces us.
to two Hebrew unnamed midwives.
who have the courage and the audacity.
to finally stand up against an empire of injustice.
and draw the line in the sand and say enough is enough..
Deliverance so often begins in the most unlikely of places..
Would you like to meet these two women?.
- Yes. - Yes..
- Yes..
- Exodus one, starting in verse 15..
The king of Egypt said to the Hebrew midwives.
whose name was Shifra and Pua,.
when you help the Hebrew women in childbirth.
and observe them on the delivery stool,.
if it's a boy, kill him,.
but if it's a girl, let her live..
The midwives, however, feared God.
and did not do what the king of Egypt had told them to do..
They let the boys live..
Just a few verses here that Moses includes,.
but this is the most incredible turning point.
in this whole story,.
because the very first people that stand up.
against an evil empire and an evil regime are two women,.
two women who have the courage.
when Pharaoh has told them to do something,.
and remember, they're slaves,.
so they actually have no choice whether they do it or not..
They're expected to carry it out..
They decide that enough is enough.
and it's time to make a change,.
and Moses, who's writing this,.
it's fascinating to me because Moses was steeped.

$^{921}$in the culture of patriarchy..
He understood the culture of patriarchy.
and the way that men were seen as the movers.
and the shakers and the change agents,.
and he says, "You need to understand,.
"before I'm even on the scene,.
"the people that really change history are two women,.
"two women that are willing to stand up.
"at a time when they so easily.
"could have just turned a blind eye.
"and say, 'This slavery is wrong.'".
The Bible describes them here as Hebrew midwives..
Now, the profession of being a midwife,.
midwifery in those days, was incredibly important..
In fact, in the ancient Near Eastern time,.
particularly in Egypt,.
infant mortality rates were incredibly high,.
so the profession of being a midwife was very respected,.
and often family would put their trust.
in a particular midwife to ensure.
that their child would be delivered alive or not..
In other words, they saw that the midwives.
held in their hands literally the power of life and death,.
and in Egypt in particular,.
these midwives were deeply revered..
They were revered because of the superstitious.
and supernatural belief that the Egyptians had.
around life and death and birth and that sort of thing,.
and so they would see these midwives.
as actually very important people within society..
Now, we know that these particular midwives are actually,.
'cause they're Hebrew, they would have been slaves..
Now, because Pharaoh deals with them,.
it shows both the respect to the role they had,.
but it was also likely that these two midwives.
were midwives within Pharaoh's wider harem,.
probably within his house in particular,.
and he says to these two Hebrew women,.
"Hey, when the men are delivered,.
"when you deliver a boy, kill him right there.

$^{961}$"and right there.".
Now, you might think,.
"Why would a Hebrew kill another Hebrew in those moments?".
Again, you got to remember, this is slavery..
This is a time when when you're given an order,.
you never doubt it, and you do what it says..
Not only that, though, history will tell you.
that so often when a majority culture.
wants to do a genocide on a minority culture,.
so often that power will get members of the minority culture.
to carry out the genocide itself..
That has happened throughout human history,.
and here we see another example of it..
And so Pharaoh expects these two women to act..
Now, the other interesting thing.
about being a midwife in those days.
is that midwives themselves were chosen for that role.
largely because they were infertile and barren..
These are women that weren't able.
to bring their own children into the world,.
and because of that, they were then selected.
to bring the children of other people into the world..
Now, you can imagine that could either be viewed.
as a bit of an embarrassment, a bit of a shame,.
or that could be viewed as a great gift..
And we don't know what it is.
for these two particular women,.
but we do know that they are childless..
We do know that they have no families of their own,.
and we know that there are slaves within Pharaoh's harem,.
and they're responsible for now committing genocide..
And they decide that this is not what they're gonna do..
And I want you to see why..
It says here in verse 17,.
"The midwives, however, feared God.".
There it is..
Very simple thing, feared God..
And this fear of God was greater for them.
than the fear of Pharaoh..
And by fearing God, they're not,.

$^{1001}$this word is actually repeated a lot.
throughout the Old Testament..
It's not about this idea of afraid or scared,.
but feared God is the idea of reverence and awe.
and worship and respect and humility of saying,.
"God is all-powerful..
"He is all-holy..
"We're not, and we need to worship Him,.
"respect Him, be in awe of Him.".
That's the idea of fearing..
And what Moses tells us here is that these two women.
who are named Shiphrah and Puah,.
these two women are actually fearing God..
And because they fear God,.
they're able to take a stand.
against what's happening to their people,.
and they're able to take a stand.
against what they've been told to do,.
which is to do genocide..
Now, here's the important thing..
All of this is Moses trying to communicate to us.
one really critical piece.
right at the start of the Exodus story..
He's trying to say,.
no matter how powerful slavery might be,.
no matter how powerful the force of slavery might be,.
there exists in the world a power that is greater than it..
And that power is the courage.
that comes upon us as humans.
that is fueled and driven by our fear of God..
And that courage enables these two women.
to realize that despite how strong slavery was,.
it did not remove the power of their choice..
They still had a choice..
And their choice was obey Pharaoh's order and live.
or disobey Pharaoh's order and die..
And they choose death to themselves.
so that others might live..
Now, I want you to see why this is really important..
Note this..

$^{1041}$The Hebrew midwives choose death for themselves.
so that others might live..
That is the exact opposite of what insecurity does..
Insecurity ensures a life for yourself.
through the death of others..
Are you with me, church?.
Do you follow this?.
And so what Moses is saying is.
there's Pharaoh in insecurity.
who's enslaving the Israelites.
and calling out for genocide..
Here's two women who stand up against it.
because they decide that they have the courage.
in their fear of God,.
that they're saying he is more powerful,.
he is more honorable,.
I'm gonna listen to him over anything else..
And they're willing to even sacrifice their own lives.
so that others might live..
That's the exact opposite of insecurity..
And Moses is saying, check out the power of these women..
That when you're willing to stand in the gap.
against injustice,.
when you're willing to stand in the gap against enslavery,.
when you realize that you still have the power to choose,.
then Exodus happens..
The beginning of our Exodus story.
is the incredible story of two women.
who make an overtly political decision.
to stand against the political authority of their day.
and fear God first..
And that changes everything..
That literally is the starting point of everything..
The courage to choose life over death..
And what Moses is asking of everybody reading this,.
right at the start of the story,.
it's essentially this,.
whom ultimately will you serve?.
Come on church..
Whom ultimately will you serve?.

$^{1081}$Will you serve me?.
He's saying, this is God,.
will you serve me or will you serve her?.
Will you serve life or will you serve death?.
Will you serve the things of the flesh in your soul.
or the flesh in your sinful nature.
or will you serve and fear me and want to please me?.
Which will you choose?.
He's basically saying..
Because here's two women.
who are under the oppression of slavery from Pharaoh.
and they make a choice for life..
And then there's us..
And we have to think,.
are we in such a situation that they're in?.
Like are the choices that we're making ones.
where we could literally die?.
And yet how often do we choose death over life?.
Yet how often do we choose to remain enslaved.
rather than move into life?.
You see, Shifra and Pua teach us.
that we don't have to be passive in the face of evil..
Shifra and Pua actually teach us.
that we don't have to adopt the role of victim in a story..
Shifra and Pua actually teach us that when we fear God,.
we can find the courage inside of us.
to actually stand against the darkness of our day.
and say enough is enough.
and flex the one muscle that always breaks through.
and that's the muscle that is no..
This is enough..
This is done..
We are not gonna do what you say..
And in the beauty of this, Exodus begins..
And I want you to see what God is doing here..
And this is so beautiful to me..
You see, note this..
The very instruments that Pharaoh had chosen.
to carry out his plans.
became the very instruments that God uses.

$^{1121}$to thwart those plans..
Isn't that powerful?.
Because Pharaoh wanted to use the women to kill the kids..
That was his desire..
He wanted to use these women to kill the kids,.
to kill the men that were being born..
And God steps in and says,.
I'm gonna use those very instruments to change the world..
I'm gonna use those very instruments.
to be the starting point of no..
I'm gonna use those very instruments.
to be the beginning point of all that I'm gonna do.
in the rest of the story..
You see, what Pharaoh wanted to do.
was bring women under subjugation..
Notice this, women who are the carers.
and the givers of life are chosen by Pharaoh.
to be the destroyers of it..
And these women say, no, that's not our role..
That's not who we are..
We are not the destroyers of life..
We're the carers and the givers of life..
And these two midwives say, we're the deliverers of life..
And I want you to see the powerful prophetic thing.
that's happening here..
These two women are basically saying,.
our role is to deliver life and not death..
And so right at the start of the story,.
you see two women delivering Hebrew children to life..
And this is a prophetic announcement.
of what God's gonna do next in the story,.
where he delivers a whole nation from death to life..
But in the starting point,.
it's a microcosm on these two women..
And I think essentially what they're saying is this,.
hey, we're all midwives..
We're all midwives..
And we might not be delivering children,.
but we are delivering choices..
The most spiritual act you will ever do.

$^{1161}$is the act of a choice..
Right at the end of the Exodus,.
and I'm jumping forward a little bit,.
Moses stands before his people and he says,.
today I give you life and death..
Choose life..
He says, me and my household, we're gonna choose life..
And I wanna stand before you.
at the beginning of your own Exodus..
As you thought about your departure point last week,.
and you think about Exodus that's ahead here,.
and I know the stories,.
oh, Angie, if only you understood.
the slavery that you're under..
And here's two women,.
already under the slavery.
of a patriarchal society and culture,.
but equally under the slavery of Pharaoh and Egypt,.
who say we still have the power to choose life..
My prayer for you as you start this journey.
is that you would find in the fear of God,.
in your relationship with Him,.
the courage you need to say no.
to whatever it is that is enslaving you..
The courage that you need to say,.
I'm gonna make better choices..
The courage that you need to say,.
I'm not gonna be tempted by that sin anymore..
The courage that you need to say no.
where you have been saying yes..
And as you take that courage,.
you align yourself to the freedom that comes,.
not with the insecurity that wants to enslave,.
but the fear of God that wants to bring life..
Can I pray for you guys?.
Would you stand with me?.
Let me pray..
Father, I thank you so much for Shifra and Pu'er.
and for the legacy they leave behind to us..
I thank you so much, Lord,.

$^{1201}$for the way in which they live their lives..
I thank you that we have,.
in Moses' recounting of their story, their names..
It's fascinating to me that we never hear the name of Pharaoh.
but we are given the name of these two women..
And so for 3,500 years, they've been known..
And we know their names because they decided.
that they still had a power of choice..
And they were willing to go to their death.
so that others might live..
They were willing to sacrifice their freedom.
so that others might find freedom..
And as Moses brings us to the picture of these women,.
he's really bringing us to the picture of ourselves..
And he's saying, who it is.
that you're ultimately gonna follow, me or Pharaoh?.
Are you ultimately gonna think you're a victim?.
Or are you gonna realize that because my spirit is in you,.
same spirit that raised Jesus from the dead in you,.
that together we can work towards freedom.
and the chains can break..
There is no greater name than the name of Jesus..
And as we put our security in him,.
we find ourselves becoming vessels of freedom and grace..
Father, I wanna pray that you would move.
over each person here now..
And Father, I ask that you would give them courage, Lord..
Courage for the journey ahead in Exodus..
Courage to say no to the things that they need to say no to,.
yes to the things they need to say yes to..
Courage to stand against the work of sin in their lives..
Courage in the fear of you to know that they don't need.
to be giving in all the time to the temptations around them..
Courage to start making better choices,.
better decisions for their families,.
for their marriage, for their children..
Courage for their marriages, for their relationships..
Lord, I pray you'd fill us with the courage.
that you filled those two women with..
That in our fear of you, the awe, the worship,.

$^{1241}$and the majesty of you,.
your spirit would work to set us free..
I encourage you just to bring that before the Lord.
in your own way right now..
Ask for his courage..
Commit yourself in fear of him, in that awe,.
in the respect and the wonder of him..
And through that,.
put yourself right at the starting line.
of everything that he wants to bring to you..
(gentle music).
\newpage



\section{}
\label{sec:LXiCHoxSu6Y}
\textbf{2023-04-30 EXODUS - 03 Drawn Out [LXiCHoxSu6Y].mp3}
\newline
\newline
連結: \href{https://youtube.com/watch?v=LXiCHoxSu6Y}{\texttt{ https://youtube.com/watch?v=LXiCHoxSu6Y}} ~~~~ 語音日期: 2023-04-30 
\newline
\newline
\hyperref[sec:cz4a6_st6To]{\small{< < < PREV SERMON < < <}}
~
\hyperref[sec:index]{\small{[返主目錄]}}
~
\hyperref[sec:gUw91uNiAAg]{\small{> > > NEXT SERMON > > >}}
\newline
\newline
$^{1}$(gentle music).
Everyone says, hey, can we thank our worship team as always?.
And as you sit down,.
please try to move into the middle of the row.
as much as possible..
We are totally, totally overflown..
We have a whole group of people out in the overflow..
Hey guys, great to have you with us in the overflow today..
We know that there are people standing on the stairs.
down the Cape of Sea..
There are people upstairs as well standing on the stairs..
So if you have a spare seat next to you,.
can you please just, yeah, raise your hand..
There's three seats right here, hosts..
Three seats right here..
There's a couple of seats over here..
Come on in if you are outside and in need of a seat..
And we are so grateful..
We know people are sitting on stairs and stuff as well..
So it's a good problem when church is full,.
but we wanna make sure that everybody feels comfortable.
and that we make sure that we have a room for everybody.
that's possible..
Just a reminder, we have three services on a Sunday,.
9.15, 11 o'clock and two o'clock..
I know 11 is the very convenient service.
'cause it means you get to sleep in.
and then you can go out for lunch..
But if you want to, you can visit the two o'clock.
at any time..
We would love to have you at the two o'clock or the 9.15..
But welcome to church..
Welcome to the Vine..
My name's Andrew, I'm one of the pastors here..
And we're continuing our series in Exodus..
There's a couple of chairs here..
I think there's two chairs right there, possibly..
There's a chair, you can sit in my chair..
I don't own my chair, my chairs..
There's two chairs here..

$^{41}$There's a chair down here..
Awesome..
All right, I'm gonna stop speaking about chairs.
and start speaking about Jesus..
So,.
are you okay?.
You got a chair?.
You got a chair, awesome..
Good to see you..
So on Tuesday, April 17th, 2017,.
at exactly 11.15 in the morning,.
I saved somebody's life..
And by that, I mean I literally saved their life..
Like I'm convinced that if I had not been.
in that exact place at that exact time,.
the other person most definitely,.
without a shadow of doubt, have died..
I was in China at the time..
I was in the South of China at a resort,.
having a holiday with my wife, just the two of us..
We decided to go on like a second honeymoon..
We left our daughter Mia at home..
We went with two friends of ours..
And it just so happened that the husband.
that we were traveling with, with these other friends,.
he was a senior member of the hotel group.
that we were staying at and the place.
that we were staying in..
And on April 11th, we had this incredible breakfast,.
late breakfast..
We'd slept in, had this great breakfast..
And after breakfast, we decided that we wanted to go.
to the one pool in the complex..
There was about four pools in the complex..
We wanted to go to the one where kids were not allowed..
'Cause you know, when you're a parent.
and you travel without your kids,.
you wanna be without kids..
You know what I'm saying?.
So we found the one pool in the resort.

$^{81}$where kids are banned, not allowed to go to..
We go over there, we arrived there,.
and there are no other people in this pool..
No one is around the pool at all..
No one's swimming there or anything like that..
So we think this is fantastic..
We're all on our own..
So we basically find these two lounge chairs.
on the side of the pool..
We sit down, we start to get ourselves nice and relaxed.
and everything like that..
Within about three minutes or so,.
two people arrive in the pool..
Now, there are three things you need to know.
about these two people..
The first is, it was pretty obvious.
that this was a daughter and her mother..
The daughter was probably in her late thirties..
The mother, I would say maybe late sixties,.
something like that..
It was a mother and a daughter..
The second thing you need to know about them.
was that they were fully clothed, okay?.
They were not dressed for the beach or the pool..
They were fully clothed completely..
And by the time, while they were walking around the pool.
and talking to each other,.
they were talking in Mandarin Chinese..
I don't speak Mandarin,.
so didn't know what they were saying,.
but I could tell by their gestures.
and the way they're talking.
that neither of them could swim..
The third thing you need to know about this couple.
was that the mother was carrying with her.
one of those massive inflatable unicorn things.
that you see in the pool..
You know, those really annoying unicorn things.
that people like to lie on for the gram, right?.
This almost 70 year old mother.

$^{121}$was carrying one under her arm,.
which I thought was bizarre.
'cause they were wearing clothes.
and it was very obvious that they cannot swim..
So I was kind of keeping half an eye on them over my book..
You know, anyone else people watch when you're on holiday?.
Come on, you know you do..
You know you do..
I'm sitting there and I'm like, this is bizarre..
Sure enough, the daughter goes down by the stairs.
of the shallow end and she calls her mother over..
She puts the big unicorn floatie in the pool.
and she gestures for her mother to sit on the floatie..
As she's doing this, she gets out her camera..
And by this point, I'm like, this is not going to end well..
And as her mother sits on the floatie,.
she then pushes her mother out to the middle of the pool..
Like just pushes the floatie out.
and gets her camera ready for the photo of her lifetime..
At this point, I know disaster is looming..
And this mother floats out to the middle of the pool.
and she tries to adjust her balance.
on this awkward floatie thing..
And she leans forward and she flips it.
and she goes straight into the middle of the pool..
I don't even think..
I jump up from my chair..
I run as fast as I can, 'cause I'm gonna save this woman..
As I'm running, I realized something..
I am also fully clothed..
I have also just come from breakfast.
and I did not get changed at all..
And I'm fully clothed..
So as I'm running, I realized I'm wearing my glasses..
I'm wearing my hat..
So I whip off my glasses..
I throw off my hat whilst I'm running..
I get to the edge of the pool..
It was literally a Superman moment,.
you know, taking the clothes off..

$^{161}$I dive into the swimming pool..
I swim under the water and I come up right under her.
as she's drowning..
She's drowning there..
Her daughter is screaming her head off,.
standing at the side of the pool..
Now, the thing that saves my life and hers.
is that as I got to her,.
I realized I could stand on the bottom of the pool.
and just keep my head above water..
Just keep my head above water..
So I'm standing there just with my head above water..
I flip her around 'cause she was facing down to the water..
I flip her around and push her head up.
so that she can kind of see properly..
I then get her in like a fireman's kind of lift thing.
like this..
And I like as best I can kind of maneuver over.
to the stairs in the shallow end..
Her daughter is there..
She grabs her by the end..
She yanks her up and they walk away..
And I, I come up out of the water..
I literally collapse on the deck of the pool,.
absolutely exhausted, but feeling like a hero..
Now, yeah..
Now my friend, who's a senior member of this hotel group,.
hears about this heroic action from this white guy.
at the swimming pool..
And because of his connections,.
he was able to get the CCTV footage of this moment..
Do you want to watch?.
Yeah, all right, here we go..
Here it is, okay..
Here she is..
Okay, no, this is my wife, my wife here..
Look at her..
She's like not sure..
She's trying to get balanced..
Oh, get balanced..

$^{201}$Oh, oh, oh, oh, oh, oh..
Okay, here we go..
Look down here, look down here..
Yeah..
Yes..
(congregation applauding).
Now I'm like, I got her like, like I'm trying to work out..
I'm trying to get her breathing..
Finally, the pool guy shows up..
Thank you..
Like I got her in a fireman..
Look, there she is..
I got her in like this fireman's posy thing right there..
I'm helping her and I save her life in the name of Jesus..
Yeah..
Not just a pastor..
Real life superhero..
I love my dive..
My dive's like a really awkward dive..
Anyway, you're probably wondering.
what has this got to do with the Exodus, right?.
Today we come to the story of Moses,.
a story which is birthed in this idea.
of somebody who reaches into the waters of death.
and brings somebody back to life..
Last week when we were looking at the Exodus journey,.
we saw that this new Pharaoh comes to power in Egypt.
and out of both his own insecurities.
and the fear that he has.
for this large immigrant community in his land,.
he decides to throw them under slavery..
He decides to oppress them, to disempower them..
And he starts by enslaving them as we looked at last week,.
but then he decides very quickly.
to move from slavery to genocide..
And he's worried about this rising army.
that could rise up amongst this group..
And so he decides that he's going to kill every male boy..
Now to do that, he calls on the midwives,.
the Israelite midwives,.

$^{241}$to actually do the process of killing..
So as they're delivering the children,.
if they deliver a boy, they kill it..
If they deliver a girl, they let it live..
Now there are two incredibly brave midwives,.
as we saw last week,.
who decide that no, they're not gonna do this..
They decide that they're not gonna bring into genocide.
those of their own flesh and blood,.
and they defy the order of Pharaoh..
And when we pick up the story here this morning.
is right at this moment after the two women.
have defied Pharaoh,.
and Pharaoh realizes that his plans had been thwarted..
Let me read this to you from Exodus 1, verse 22..
Then Pharaoh gave this order to all of his people..
Every boy that is born, you must throw into the Nile,.
but every girl can live..
Pharaoh, knowing that he's been thwarted.
by these two incredibly brave Hebrew midwives,.
then turns to his people,.
rather than getting the Israelites to try to kill Israelites,.
he turns to his own people,.
and he says, "Here's what I want you to do..
If you come across an Israel, a Hebrew boy,.
who's young, a child, I want you to grab that boy,.
and I want you to throw it into the Nile.".
Now, this is not a suggestion..
This is a decree and a command of Pharaoh..
And every Egyptian who is loyal to Pharaoh,.
which they all had to be,.
anytime they saw a Hebrew boy, this is what they should do..
Pick up that child and literally throw him into the water..
The word here, throw him, is really important..
Throw them or throw into is really important..
That word, remember, Moses is writing this,.
that word is a Hebrew word.
that means literally to abandon..
And I want you to see what Moses is doing.
right at the start of this story..

$^{281}$He's wanting us to understand what evil.
and what the enemy is always trying to do through slavery..
He's always trying to abandon us from what is around us..
Pharaoh decides that the best way that he can do this.
is abandon this child..
And to abandon this child,.
to do this to the Israelites themselves,.
is to pick up the child and throw it in the water,.
to abandon it to the family,.
abandon it to their community,.
and abandon it, of course, to death..
And this power of abandonment, as Moses is writing this,.
is wanting us to understand that this is not just a moment.
that Pharaoh is doing,.
this is something that's seeped into the reality.
of what all of our sin does..
See, in our slavery to sin,.
sin is always at work to try to abandon us.
to our families, our careers, our loved ones,.
the things that are around us..
You talk to anybody who has had the travesty.
of the slavery of addiction,.
and they'll tell you that story..
You meet anyone who's been addicted to gambling,.
they will tell you that they've abandoned their possessions.
and were willing to abandon their possessions.
in order to feed that addiction..
You talk to anybody who's been addicted to pornography.
or addicted to drugs,.
you'll find that they will do whatever it takes.
to abandon their family and their friends and their time.
to the things that they're addicted to..
But it's not just addiction, it's any sin..
Any sin that you are held captive to.
is at work within you to abandon you to what is around you..
That's the work of sin..
That's what the enemy is trying to do in your life..
And it's important as you're on your own journey of Exodus,.
as you're thinking about what it means for you.
to go from slavery to greater freedom in your life,.

$^{321}$you need to realize that what sin is trying to do.
is abandon you to all that is good in your life,.
to strip that away from you..
And one of the starting points.
of all of our journey in Exodus is to refuse to be abandoned,.
to refuse to move with an abandoned spirit..
I want you to know this,.
that the greatest spiritual work that the enemy has.
over the global church is to fill the church with orphans,.
to fill the church with orphans in the spirit,.
those that feel alone, those that feel abandoned,.
those that feel given up on,.
those that feel separated from everybody else..
And not just with an orphan spirit about themselves.
and the people around them, but most importantly,.
this is the work of the enemy.
to make us feel like we're an orphan to God,.
to make us feel like we're abandoned by him,.
that he has left us alone..
And right at the start of this story,.
and Moses is about to unpack for us something incredible,.
he wants you to understand that the gateway into this story.
is the work that the enemy does.
through the reality of sin to abandon us..
Now, he then says they're being abandoned into the Nile..
Now, this is really important because the Nile,.
this is the first mention of the Nile.
in the whole of the book of Exodus..
And the Nile is gonna go on to dominate.
actually some of the integrable moments.
of the early part of the Exodus story..
You know, the five movements I've been teaching you.
throughout the series so far,.
slavery, promise, liberation, identity, and home,.
the first three of those, slavery, promise, and liberation,.
all are integral to the work of the Nile..
The Nile features in critical moments.
of those three movements..
And this is the first time we hear about the Nile..
And you need to understand why the Nile.

$^{361}$was super important to the Egyptian people.
and how the Nile is then used by God to flip the story,.
flip the script, and use the Nile.
as a liberating place for God's people..
And to help you to understand that,.
I wanna take you today to the banks.
of the River Nile in Egypt..
Let's take a look..
"Flowing some 4,180 miles.
"and covering one-sixth of the Earth's circumference,.
"the Nile River is almost twice as long.
"as the Mississippi, the Tigris, the Euphrates,.
"and the Colorado Rivers combined..
"The river is formed from two sources..
"The white Nile begins in Central Africa,.
"and the blue Nile begins in the Ethiopian highlands..
"They merge together in Khartoum,.
"where they flow some further 1,600 miles.
"to the Mediterranean.".
Historians believe that Egyptian civilization.
began about 6,000 years before Christ,.
as early settlements began to pop up around the Nile,.
not too far from where I'm standing right now..
And not surprisingly, the Nile became a source.
of all life for Egyptians,.
and became the center imagery.
in so much of their mythology..
But I think it was Napoleon who put it best.
when describing Egypt to his contemporaries..
He said, "In the Nile, Egypt has the spirit of good..
"In the desert, the spirit of evil.".
And I think Israel would not disagree,.
because the Nile would be the good.
that begins their journey,.
but the desert, the struggle that would come to define it,.
and to understand all of that,.
I wanna take us on a journey down the Nile ourselves..
(gentle music).
(water splashing).
(fire crackling).

$^{401}$All slavery begins with a simple step,.
the devaluing of human life..
Once we devalue the life of another,.
we open ourselves up to perhaps the most sobering reality.
of the human condition,.
that regardless of how good or moral we might think we are,.
all of us are capable of committing.
the worst kinds of atrocities imaginable..
Enslaving the other is the potential of us all..
This is why our story starts here on the Nile,.
for it's in these very waters.
that Pharaoh orders all newborn Israelite boys.
to be thrown in..
Innocent children are murdered here.
simply because some people have decided.
that they no longer have an intrinsic value of life..
Long before God turns these waters red in the plagues,.
Pharaoh does so with the blood of a thousand Hebrew boys..
The source of all life has now become a graveyard of death..
But stories of redemption and freedom.
so often are birthed out of the very same places.
where the worst acts of humanity actually take place,.
and the Nile is no exception..
For it will be in and out of these waters.
that would come two acts from incredibly courageous women.
that demonstrate to us the very first step.
that is required in any fight against evil.
and its enslavement,.
the resolute conviction of the gift of a human life..
In the midst of mass genocide,.
a Levite family bears a son..
Their culture taught them.
that he was special and valuable from birth,.
a gift from God..
This belief gives the mother the courage.
to resist the orders of an empire.
and do all that she can to try to save him,.
even if it would mean her own death in doing so..
So she takes a wicker basket,.
and in a decision of selfless courage,.

$^{441}$places her son amongst the reeds in the waters of the Nile..
It was a pure act of suffering love..
The only way that she could save him.
was through the sacrifice of letting him go..
And so the boy is placed into the very waters.
his peers had been drowned in..
And as destiny would have it,.
the very daughter of the man who decreed their deaths.
is the very one who would bring him back to life..
Redemption has a way of ensuring no irony is lost..
This is the Bible's first story of adoption..
And like all stories of adoption,.
it's only made possible through the courageous acts.
of those that value the intrinsic nature of life..
So a Hebrew mother moved to a selfless act of sacrifice.
in order to save her son..
Or take a Pharaoh's daughter,.
moved to listen to her own heart.
as she stands against her father's decrees..
And behind them both,.
a God who's always moved to use ordinary people.
to do extraordinary things..
This is how any act of Exodus from hopelessness begins..
And may it also then be for us..
May you know a God who is able to draw you.
out of the darkest of waters.
so he might then adopt you into his family..
- Genocide, abandonment, sacrifice, release,.
the idea of being cut off,.
and yet then salvation and redemption.
within the very same context that brought that abandonment..
This story right here has incredible power to it..
And there's so much here.
that I wanna hopefully communicate to you.
so that you can really begin what it is that,.
really understand what it is that God is doing.
in this beginning part of Exodus for you..
Let me read on from Exodus chapter two, verse one..
It says, "Now a man of the house of Levi.
married a Levite woman,.

$^{481}$and she became pregnant and gave birth to a son..
When she saw that he was a good child,.
she hid him for three months..
But when she could hide him no longer,.
she got a papyrus basket for him.
and coated it with tar and pitch..
Then she placed the child in it.
and put it among the reeds along the bank of the Nile..
His sister stood at a distance.
to see what would happen to him.".
It's really interesting what Moses does here.
as he's telling the story..
He's trying to bring a big contrast for you as the reader..
First of all, he's just started with this idea.
that Pharaoh has changed tactics now.
and is now abandoning all of these children.
to their death in the Nile..
But as soon as he said that,.
he then opens up a very ordinary story to you..
He tells you a story about an ordinary Levite family.
who are just people that fell in love,.
got married and decided to have a child..
And when we read the story,.
we think it's just kind of this light idea..
But if you think about the context of the moment,.
a couple who were Israelites at this time having a child,.
that was a very courageous decision..
It was a very brave thing to do.
because they knew the edict.
that was under the world and Egypt at this time..
They knew that Pharaoh had said that if they had a boy,.
that that boy would be murdered, boy would be slaughtered..
And so as you became pregnant in that time,.
you would have been feeling all this anxiety..
If I give birth to a girl,.
well, I guess the girl will live..
If I give birth to a boy,.
then sadness and travesty will come under us..
So just the idea of becoming pregnant.
was an incredibly risky thing to do at that time..

$^{521}$But this couple decided it was a risk worth taking..
And we're told right here that she has a boy..
And it's really interesting what it says..
And this is one of the critical bits in this whole passage..
It says, "When she saw that he was a good child.".
I love this idea of the idea of him being a good child..
Now, you gotta be careful here..
This is not talking about some moral statement..
It wasn't that this child.
was particularly a good behaving child..
It wasn't like he was just well-behaved and never cried.
and was just so nice that she decided to keep him..
As if the implication was if it was a naughty child,.
she'd throw him in the river herself..
That's not what it's saying here..
The word good here, remember, I've been saying each week.
that Moses, as he's writing this story,.
he's wanting to connect it to his other writings in Genesis..
That's what's happening here..
The word good here is exactly the same word.
that Moses writes in Genesis 1 and 2.
to describe what God thinks,.
what God feels at the end of every day of creation..
He stands back from that day of creation.
and he says, "This is good.".
The idea being that this is just as I've designed it to be..
This is just as it should be..
It's not a moral statement from God..
It's a statement that the world in that moment.
is just in his will..
It looks and it is as he desires it to be..
To say that it is good is to say that it is under my control,.
it is in my hands, it is just as I want it to be,.
and everything is all set..
The idea of shalom and peace.
is captured in this word of good..
So here's a Hebrew mother who's just given birth to a boy.
who should be killed, but she sees that this is good..
She stands back from creation like God does.
and she describes it as good..

$^{561}$What she's saying is, even though this is a boy.
and even though we're not supposed to be having boys.
at this time, I know that this is in line with God's will,.
that God has willed me to bring a son into this world.
so this child has value, this child has intrinsic worth,.
this child is good in God's eyes..
It is well with him, it is well with his soul,.
it is in line with his purposes and his promises,.
and she sees that and because of that,.
it drives everything she does next..
It's like Moses is wanting you to understand.
the theological thinking of this woman.
'cause we look at her and we think,.
wow, she's incredibly brave, and Moses is saying,.
you wanna know where that bravery comes from?.
It's because she can see God's goodness,.
the goodness of God in her son..
She sees that this son is in the will of God,.
what the will is for him..
Even if this son is gonna be killed,.
it was still in the will of God and she can declare it good..
In other words, I don't judge the will of God.
by whether something goes well or not..
Come on, church..
I don't judge, she's saying, the will of God.
by whether I'm happy or not..
I judge the will of God by whether he declares it or not..
Does he declare this good?.
This child is good..
So that idea then drives what she does next..
She takes a papyrus basket.
and she coats it with tar and pitch..
Now, very importantly, what Moses is writing here,.
what he's saying is, he uses a word for basket here.
that he only uses in a few other times.
in the whole of his writings..
One other time that he uses it is in Genesis 6 to 8,.
chapter 6 to 8, where he's describing Noah and the ark..
The ark and the word basket is the same word..
What he's trying to communicate to us.

$^{601}$is that this mother who knows that this child.
is in the will of God, she then takes a basket,.
she takes a ark and she pitches,.
she puts the tar and the stuff, that waterproofs it..
Interestingly, those are exactly the same two words.
that Noah uses to make his ark.
and to make the ark waterproof..
Notice, the ark was designed to be a vessel.
that could carry humanity to safety,.
to life on the waters of death..
Are you with me?.
So Moses says, what the mother does here.
is because he sees this child is good,.
she creates her own ark for this child..
And she places the child in her ark.
and she sets it off amongst the reeds of the Nile..
In other words, you've probably read this story before.
and you think that she's abandoning her child.
because she can't dare to see the child be killed..
Remember, it is Pharaoh that's abandoning..
She's not abandoning, she's dedicating..
She's releasing..
She's having the audacity to say,.
because this child is good,.
this is God's child more than it is my child..
And I'm gonna place this child in an ark,.
in something that could be a vessel of true life for him.
and I'm going to release it in trust of God..
I'm gonna release the gift that God has given me..
I'm gonna give it back to God.
and I'm gonna set it out on this water..
The waters that should be killing him.
are gonna be waters that will bring him life..
I'm gonna put it in God's hands..
I'm gonna trust and release..
And what Moses in writing this is wanting you to understand.
is the dichotomy that there is.
between Pharaoh that wants to abandon.
and a mother that so trusts God.
that she's willing to release the God..

$^{641}$She's willing to give to God all the things that are there.
and this difference is gonna be what we're gonna see play out.
in the whole of the Exodus journey..
And this mother makes the decision.
that she's willing to release to God..
You need to understand that if you truly want freedom.
for yourself and for others,.
then you're gonna have to have the courage.
to release everything you have into the hands of God..
Come on church..
If you really wanna be on an Exodus journey,.
if you really wanna go from slavery to freedom,.
you're gonna have to have the courage.
to release all the things in your life.
into the hands of God..
You're gonna have to have the faith.
to believe that God can be at work in the things.
as you give them to him..
You're gonna have to say,.
even though I don't fully understand,.
even though I don't know what's actually gonna result,.
I believe that God has a victory.
that's far greater than the victory.
I could ever defy for myself..
I'm willing to release into his hands.
and trust for his provision..
That's gonna be the journey you're all gonna be caught on..
And here's the thing,.
we're all okay with releasing the stuff.
that we wanna get rid of, amen..
Like when you hear a story about releasing stuff,.
you're thinking to yourself like,.
sure I'll release, I'll release the bad relationships.
that are dragging me down..
I'll release the bad habits.
that I know are not good for me..
I'll release all the mucky stuff in my life.
'cause I want the best life I can have..
Here's the challenge..
This mother gives away the greatest thing.

$^{681}$she had ever received, the birth of a son..
And here's the really important point..
Even the blessings in your life,.
if they are not wholly given over to God,.
will actually become the very things.
that will eventually enslave you..
Is anybody here?.
Even the blessings, even the good things,.
if we hold onto them too tightly,.
will eventually become the things.
that the enemy use to enslave us..
See, the reality is that God gives us blessings.
and the blessings of God are good things,.
but so often what happens when God blesses us.
with a great career, with great finances,.
with a great relationship,.
so often we can take the blessings of God.
and then we can magnify those things.
and they can become an idol for us..
And the very thing that was a blessing from God.
becomes a thing that enslaves us in idolatry..
Some of you in this room are enslaved to your work,.
which was originally a gift from God to you,.
but you've now turned it into an idol.
and it's slaving you, it's driving you..
Some of you are enslavement to the finances..
God's blessed you financially and it's been great,.
but you've taken a hold of it too tightly.
and you're now controlling it.
and you're not giving God any room to use it.
and to move it through your life.
and you become enslaved by the very blessing.
that God has given you, this woman,.
in the courage that can only be found.
in understanding that God is good at all times..
She's willing to release even her greatest blessing,.
trusting that God has his will and his plan..
You will come into the greatest moments.
of freedom in your life when you're willing to let go,.
not just the bad stuff you carry,.

$^{721}$but even hold the good stuff lightly before him..
That doesn't mean that you shouldn't be a steward.
of his stuff, doesn't mean that you don't steward.
the resource of God well,.
but it does mean that we have to be careful.
that we don't allow our stewardship of something.
to become an idol..
And some of you need to just release,.
release what it is that you're holding in your hands,.
release what it is that you're experiencing in your life.
into a fullness of God's trust..
And that courage to trust him like that.
will be the starting point of your exodus..
It'll be the starting point of all that God can do..
I wanna show you that although this mother.
does not abandon her child, God also does not abandon her..
Notice what happens next..
Then Pharaoh's daughter went down to the Nile to bathe,.
and her attendants were walking along the riverbank..
She saw the basket among the reeds.
and sent her slave girl to get it..
She opened it and saw the baby..
He was crying and she felt sorry for him..
"This is one of the Hebrew babies," she said..
I want you to track with what happens in this moment..
So here's the mother who releases on an ark.
that she's created her child into the hands of God..
And here comes one of the oppressors,.
not just one of the oppressors,.
but the very daughter of the one who's doing the oppressing,.
the daughter of Pharaoh..
And she sees this basket in the reeds.
where she's gonna bathe..
She sends a slave girl to get it..
She brings the basket to her..
She opens it up, and immediately she sees.
that this is one of the Hebrew boys..
Now, she in her mind is thinking,.
"I should kill this child.".
This is Pharaoh's daughter..

$^{761}$So the decree is not just some Pharaoh in Egypt.
that I guess I have to do because he's Pharaoh..
This is her own father..
And there is no time in her life.
that this daughter has ever disobeyed her father..
Pharaoh's daughters were powerful, influential authorities..
They were an actual representation in their symbolicism.
of the power and the might of Egypt..
There was no way that this girl.
would ever have disobeyed her dad..
And here she is, and she says, "This is a Hebrew boy..
"That means she should abandon him..
"She should throw him in the water, and she doesn't do it.".
And the question Moses wants you to ask is why?.
What makes the difference here?.
Why does she decide not to abandon the child?.
He tells us right here, he says,.
"He was crying, and she felt sorry for him.".
The word sorry for there is a Hebrew word.
which actually literally translates as compassion..
She felt compassion for him..
This idea of compassion is gonna be a central theme.
in the Exodus journey..
In fact, in just a couple of weeks time,.
we're gonna see when God decides to act.
on behalf of the Israelites,.
he decides to do so because he's moved with compassion..
In fact, this compassion is what drives God.
to do all the things that he does.
to bring Israel out of their slavery into their freedom..
Compassion will always be the starting point.
of any Exodus journey..
But what's amazing here is that it's actually.
not God's compassion that starts Exodus..
It's a woman's compassion that starts Exodus..
And it's not just any woman,.
but it's a woman who is a part of the oppressors..
I want you to see what God is doing here..
He's taking one of the very oppressors of his own people,.
and he's using her to have the compassion.

$^{801}$that will begin an Exodus journey.
through the saving of Moses' life..
God takes the heart of the one.
who should have thrown the child into abandonment,.
and he changes that heart.
so that she would have compassion to act differently..
And the way that that word is written,.
it's written in such a way.
that it's actually describing God's power in that decision..
In other words, her compassion is not just her compassion..
God is at work in her to make her compassionate.
so that she will do this act.
in order to save a Hebrew child..
More than I think the two midwives that we saw last week.
who have incredible courage,.
this woman has even more courage.
because the back that she's turning,.
she's turning her back on her own father..
This is personal to her..
But she does it because there is this transformation.
that happens in her heart by the Spirit of God..
Even though she doesn't worship God,.
she doesn't believe in God,.
she never thinks about God,.
and God is still at work 'cause God can take anybody.
and change their hearts for his purposes..
And some of you need to hear that in this room.
in your Exodus journey..
God can take anyone,.
and he can use his Spirit to change their heart.
so he can sweep them up in his redemptive work.
that he does in this world..
It's not just us Christians that get to enjoy that..
God even uses anyone, anyone, even a non-Christian.
can be swept up in the power of God..
Their hearts can be changed,.
and they can actually maybe even inadvertently.
begin to work on behalf of God's will..
That's how powerful our God is..
That's the victory that he has,.

$^{841}$that he's willing to stand over a group of people.
and say, "I will work on every human heart.
"to bring my purposes in this world.".
And for some of you in this room,.
that's a word in season for you.
because you're in a relationship right now.
where you've been praying for a long time.
for God to change their heart about something..
And you think that that heart cannot be changed..
You think that there's no way that that heart can be changed..
You've kind of given up..
Some of you in this room, some of you are parents,.
and you've been praying for the salvation of your children..
And you look at your children and what they do.
and how they act, and you think to yourself,.
they're about as far away from Jesus.
as I could ever imagine..
I don't think they're ever gonna come to faith..
Some of you have been praying for your adult children,.
and you're desperate to see them come to Jesus,.
and you think they're never gonna change..
Some of you are in a difficult relationship and work,.
and you're like, God, if you only stepped in.
and changed this person's perspective,.
then my life would be a lot easier,.
but you don't think it's gonna happen..
And here's a story where the daughter of the oppressor.
has her heart changed by God so she can set people free..
And that should encourage you never to stop praying..
Never give up in prayer..
If you've been praying for your child to come to Jesus,.
never give up on your prayer..
He's at work..
If you're praying for God to liberate you.
from some situation that you're in.
where you're feeling oppressed,.
and you think that person's heart can never change.
to have a softer heart towards you, don't give up praying..
God can step in, and God can do the very things.
that we would never think he could do..

$^{881}$And what's really beautiful and profound about this.
is that as God's at work in this woman's heart to do this,.
his heart is still so much for the mother..
I wanna show you what happens next..
Then his sister asked Pharaoh's daughter,.
"Shall I go and get one of the Hebrew women.
"to nurse this baby for you?".
"Yes, go," she answered..
And the girl went and got the baby's mother..
Pharaoh's daughter said to her,.
"Take this baby and nurse him for me, and I will pay you.".
So the woman took the baby and nursed him..
I want you to follow what's happening here.
'cause we serve a God who's able to turn things around..
We serve a God that when you release, he will give back..
We serve a God that when we step out in faith,.
he will do things that we never thought he would do.
to come around and turn things around..
Here's a story of a mother who releases on an ark,.
believing that God can save her child,.
but was probably thinking,.
"I'm never gonna see him before again..
"I'm never gonna see him again..
"I'm releasing him..
"I'm giving him to the Lord..
"I don't know what's gonna happen.".
Here's a God who takes the oppressor.
and he breaks the oppressor's heart by his spirit,.
who takes this child out of the waters of death,.
decides to then have this child..
Oh, and then it just so happens that Miriam,.
the child's sister, is standing nearby..
She shows up and goes,.
"Oh, I know someone who could actually look.
"after that child for you and nurse that child.".
The child goes into that person's hand..
That person takes it back to the mother,.
and the mother gets to raise the child,.
and get this, she's paid for it..
(congregation laughing).

$^{921}$I mean, if you wanna know a story of redemption,.
a story of God's heart for his people,.
he's saying, "This is how Exodus begins..
"I changed the heart of the oppressor,.
"and I have not abandoned Moses,.
"but I also have not abandoned this mother,.
"and she's gonna raise this child,.
"and she's gonna be the only Hebrew mother.
"in the whole of Egypt who is allowed to raise a boy.
"under the very obedience of Pharaoh himself.".
Pharaoh's household provides her the protection.
to be able to raise the child.
that should have been abandoned in the waters..
I said in the film that so often,.
redemption and freedom is birthed out of the very places.
where the worst expressions of oppression have occurred..
You need to know on your journey of Exodus,.
the places that you think are the most broken in you,.
the places where you think God is the furthest away,.
the places where you think you're never going.
to get anything out of that,.
those are the very places God loves to birth his hope..
Those are the very places that God loves to turn around..
Those are the moments where God steps in,.
and goes, "You thought that this would never happen..
"Watch what I can do, 'cause I'm the God of the Exodus.".
I love this..
There are some of us in this room.
where what you think is absolutely incomprehensible.
could never happen, God already sees as having had happened..
For when this mother gives this child in the ark,.
and releases this child in the basket, God already knew..
You could almost see him standing over her going like,.
(laughing).
if only you knew what was about to happen, right?.
You can almost sense God's excitement for her,.
that in her obedience of her letting go in faith,.
God is like, "I'm gonna reward that faith..
"I'm gonna reward that faith.
"by giving this child back to you..

$^{961}$"I'm not only giving this child back to you,.
"I'm gonna protect you so you can raise him,.
"and not only that, I'm gonna pay you.
"so that you can then provide for your wider family..
"This is all happening because I reward faith.".
And you need to understand that that's what God does..
God rewards faith..
And sometimes we worry about saying that in church,.
because we're so afraid of the prosperity gospel..
But we also have to realize that.
in being afraid of the prosperity gospel,.
we gotta be careful that we don't become.
the impoverished gospel..
And there's a center line here, where God does reward faith..
And it's gonna require faith for you.
to step out in your own exodus..
It's gonna require courage and faith for you.
to let go of even the very things that are blessing you.
to ensure that they don't become idols..
It's gonna take faith in you.
to know that God is at work and that it is good,.
even though it might seem like it's overwhelmingly bad..
Faith is an important part of your exodus..
And here's this Pharaoh's daughter.
who does an incredible thing,.
brave thing to stand against her father.
so that the child would be set free..
Notice verse 10..
When the child grew older,.
she took him to Pharaoh's daughter and he became her son..
She named him Moses, saying, "I drew him out of the water.".
See, Pharaoh's daughter didn't just draw Moses.
out of the water and give it back to the Hebrew family.
and then forget about him..
She drew him out of the water.
because she wanted to take him into her family..
And as I said in the film,.
this is the Bible's first story of adoption..
And adoption from this point forward.
is gonna become the main metaphor.

$^{1001}$that the Bible uses to speak about the redemption work.
of God..
In the New Testament,.
we know that the early church would use the adoption metaphor.
to help speak about Gentiles..
And if you're in this room right now watching online.
and you're not of Jewish descent, then you are a Gentile..
And you're a part of the incredible story of God.
because you have now been swept up in his adoptive power,.
that he has adopted you out of the waters of death.
and brought you now into his family..
You are his child now..
And that whole metaphor, that whole journey starts here..
And you see, adoption is about the reversing.
of the curse of abandonment..
Notice this, guys..
Adoption is about the reversing of the curse of abandonment..
Adoption is the very opposite.
of what a Pharaoh is trying to do here..
He's trying to abandon these children,.
but his own daughter moved in compassion.
'cause the Spirit is at work in her heart,.
is willing to draw this child out of the waters.
and bring it into her own family..
And as she's doing that,.
she's standing against the very abandonment.
that her father has decreed,.
and she's saying, "No, this is family.".
In drawing Moses out of the water,.
what she's doing is drawing the boundaries of her family.
much broader than the confines of bloodline..
It's hard to describe to you the travesty.
of what it would have meant for Pharaoh and his family.
that now a Hebrew child was part of the family..
That was deeply, deeply embarrassing for them..
And this daughter was the one who was willing to say,.
"We're gonna draw our family broader than just bloodline.".
Notice what she calls him..
She calls him Moses, saying, "I drew him out of water.".
This is really interesting..

$^{1041}$I wanna just close on this..
There's no way that Pharaoh's daughter.
spoke Hebrew or understood Hebrew..
Just would not have been a thing..
So she doesn't call him Moses in the Hebrew sense..
She calls him Moses in the Egyptian sense..
Interestingly, Moses was actually a compound word.
that was used in Egypt at that time..
The word itself actually is this word here..
It's M-S/M-S-I..
I know that doesn't really kind of sound like anything,.
but it means to give birth, to have a child or a son..
And it's used, so these two people here.
are pharaohs of this time, Thutmose and Rameses..
I want you to notice the Moses on Thutmose.
and the Meses on Rameses..
That's both this compound word that means Moses..
So Thutmose means Thoth is born..
Rameses means Ra is born..
So it's always this person is born..
So something Moses, does that make sense?.
It's a little bit like in our modern day,.
we might say Johnson as the surname, son of John..
So this is something Moses means the son of,.
the child of that person..
Here's the powerful thing..
She decides to just call him Moses.
because she doesn't know who he's the son of..
This child's been abandoned..
She doesn't know that this mother that's been nursing him.
has been the real mother..
She doesn't know where he comes from..
She doesn't know the prefix to his name,.
but she's willing to call him Moses.
'cause what she's saying, I think, to her father.
and to everybody around her is that even though.
we don't know who this is a child of, he is a child..
He is here..
He has been born..
He has value, intrinsic worth..

$^{1081}$I am not abandoning this one..
I'm even gonna take this one into my own family.
for he is a Moses, a child of, a son of somebody..
And I value that..
And so he is now mine..
Powerful, right?.
That's right against her dad..
And that's her saying, this one has value..
But Moses, as he's writing this,.
he uses the Hebrew form of his name, of course..
And let me tell you what Moses means in Hebrew..
Moses is Moshe in Hebrew, M-S-H-H,.
which is there on the right-hand side..
You can see the Hebrew actually of it..
It literally means to be drawn out..
Now, there's two ways you can use the word Moses,.
active, the one who draws out,.
or passive, the one who was drawn out..
And Moses, as he's writing this,.
he decides to use the active verb of his name,.
which means the one who draws out..
Because what he's trying to communicate.
is not just the heart that Pharaoh's daughter had to say,.
even though I don't know the prefix,.
I'm gonna say that this is a son of,.
he's also then saying, I am the one who will draw out..
God called me to be the drawer-outerer..
I'm not just being drawn out..
So it's not the passive context, the one who was drawn out..
He says, I am the active word..
I am the one who draws out..
And he's saying that this is what God's about to do.
in the rest of the story..
He's gonna use me to go back to Israel.
to actually take them from their slavery and sin.
and draw them into the promised land..
I am the one who will draw out..
And God stands over you and says,.
I'm the one who draws out..
I'm the one who draws you out..

$^{1121}$And I draw you out so I can draw you in..
I have drawn you out of the waters of death.
and I have brought you into new life..
And this idea of drawing out and drawing in.
is what all of the Exodus story.
really essentially is about..
And in about two weeks time, I'm gonna show you.
the power of what it is to be drawn out, to be drawn in..
But right here, right now, here's what God wants you to know..
You yourself are Moses..
You are a Moses in the Egyptian and Hebrew sense..
In the Egyptian sense, you are not abandoned..
You are not an orphan..
God looks down on you and says, this is my child..
This one has been created and it is good..
Some of you in this room, you don't think you're good..
And again, I'm not talking about some moral statement..
I'm talking about the purposes and promises of God.
to create you and bring you in this world..
There is no one like you..
And God loves you so much that he sees you just as you are.
and you are good and you are valued to him..
You are his son and his daughter..
You are his child and he welcomes you into his family..
And you are a Moses in the Hebrew sense..
You, if you're a Christian in this room watching online,.
you have been drawn out of the waters of death.
and you have been brought into new life.
through the life, death, and resurrection of Jesus..
He's forgiven your sin, removed it from you..
You've gone on your ark through your relationship with God.
and you are now free to be who he has called you to be..
It's a beautiful thing..
But you are also a Moses in the sense.
that your primary will, the primary will of God.
over your life is that now having experienced salvation.
and life yourself, having been taken from the waters.
of death and brought into life itself,.
you now are one who can Moses with others..
You now are one who can stand before the people.

$^{1161}$in your workplace, the people in your family,.
the children that you have, whatever it might be,.
and say, "These things will not be abandoned either.
"and this will be something that I can partner with God.
"to bring out of darkness into light.".
That's Exodus..
That's the journey..
That's what you're invited into..
You are not abandoned..
You are a child of God..
He sees you and loves you..
And as you trust him and as you're willing.
to release control of your life to him,.
you will see a freedom and an Exodus in your life.
like you've never seen before..
Can I pray for you?.
Would you stand with me?.
And as we stand together,.
I wonder whether you close your eyes, open your hands..
I wanna minister now to you guys in the spirit.
by reading a Psalm over you..
Thousands of years later,.
or maybe about a thousand years later,.
David writes a Psalm, Psalm 18,.
and he pulls up this imagery of Moses.
being drawn out of the water..
I wanna pray this over you..
It says this, "The Lord has reached down from on high.
"and taken hold of me..
"He drew me out of the mighty waters..
"He has delivered me from a powerful enemy,.
"from my foes who were too strong for me..
"For they confronted me in the day of my disaster,.
"but the Lord was my support..
"For he brought me into a spacious place..
"He has delivered me because he's delighted in me.".
Father, I thank you that you delight.
in every person here and online right now..
And Father, you are at work with deliverance, Lord..
And Father, I pray that you would give us.

$^{1201}$the courage of this mother to let go,.
the courage of this mother to release to you.
everything in our lives,.
trusting that you are a good father..
And Father, when we carry an orphan spirit in us,.
it's such a thing that breaks us deeply..
And it causes us to be abandoned from the people around us..
I wonder whether you just take a moment to recognize.
whether that's something that's a reality for you..
Do you feel abandoned?.
Abandoned by God, abandoned by your family,.
abandoned from people around you?.
Do you feel looked over?.
Do you feel like nobody values you?.
Do you feel cast aside?.
Do you feel like no one sees in you.
what you see in yourself?.
Do you feel like you don't have the worth.
that others might have?.
That's the work of abandonment in you,.
to try to define you in any other way.
than how God truly sees you..
Father, we come against the spirit of abandonment.
over these people in the name of Jesus..
And Father, I wanna release.
the spirit of adoption right now..
I pray, Lord, that every person in this room.
would know that they are valued, created, a Moses..
That they are loved by you, a child of yours..
That they would know that they are a Moses,.
drawn out of the dark waters into salvation and life..
And as we're just praying with our eyes closed,.
I wanna just say this..
There's some of you in this room and your parents..
And I felt this strongly in the first service,.
and I feel it strongly now..
Your parents over children..
And you're holding on too tightly to your children..
And your children are a gift from God,.
a blessing from God..

$^{1241}$But you're holding onto them too tightly..
And you're afraid..
You're afraid that they might feel abandoned..
But you're holding on so tightly, you're crushing them..
And the word of the Lord to you today,.
he's saying, "Do you trust me with your kids?.
"Do you trust me with your children?".
Because if you do, release them to me..
That doesn't mean you give up.
all your responsibilities, of course..
But it means that you let go of control..
That you release that control over your children.
to the Lord..
And some of you here believe that..
Believe that those children can't change..
And you're so frustrated..
And out of that frustration, you've been controlling..
God's saying, "Watch what I will do through my spirit in the heart of your child.".
[BLANK AUDIO].
\newpage



\section{}
\label{sec:gUw91uNiAAg}
\textbf{2023-05-08 EXODUS - 04 Murder + Marriage [gUw91uNiAAg].mp3}
\newline
\newline
連結: \href{https://youtube.com/watch?v=gUw91uNiAAg}{\texttt{ https://youtube.com/watch?v=gUw91uNiAAg}} ~~~~ 語音日期: 2023-05-08 
\newline
\newline
\hyperref[sec:LXiCHoxSu6Y]{\small{< < < PREV SERMON < < <}}
~
\hyperref[sec:index]{\small{[返主目錄]}}
~
\hyperref[sec:1GHdbHuySwQ]{\small{> > > NEXT SERMON > > >}}
\newline
\newline
$^{1}$Welcome to The Vine if you're new here..
My name is Andrew..
I'm one of the pastors here..
Welcome if you're online with us right now..
We're so grateful that you join us..
Some of you join us each and every week from around the world online..
We're so glad you're here..
Welcome to everybody in the overflow outside..
We know we've got a packed overflow outside as well..
It's great to have you with us..
If you're new to The Vine, you may not know who I am..
You may not know where I'm from..
I have a weird accent..
And when I talk, people, if they don't know me, they're like, they don't listen to a word.
of what I say for 20 minutes..
They just try to work out, is he American?.
Is he Australian?.
Like, where is he from?.
Actually, that question, where am I from, is one that has caused me a lot of stress.
and anxiety over my life..
It's actually the question that I have struggled to answer probably the most in my life..
I wonder whether anyone else here might resonate with that idea..
I was actually born in the UK..
I lived there for seven years, but then I left the UK and I lived in the US..
And I lived in America for four years, some of the most formative years as a kid from.
seven to about the age of 11..
And at the age of 11, I came to Hong Kong and moved here with my family and have been.
here for 36 years..
And so the interesting thing for me is that I've been in Hong Kong for 36 years, so in.
a way, Hong Kong is home, but the reality is I actually don't speak Cantonese..
I know, shocking and shameful..
I don't speak Cantonese after 36 years..
The reality also is, if I'm honest with you, is that I'm not actually deeply immersed in.
super local Cantonese culture..
And part of that is because I was raised in an international schooling system here and.
I've actually, for whatever reason, have for the majority of my life lived in districts.
of Hong Kong that are generally favoured by expatriates or international people..
So some of my local friends, when I tell them I'm a Hong Konger, they just smile at me and.
pat me on the back and kind of go, "No, you're not..
No, you're not..

$^{41}$No, you're not.".
Whereas some of my other local friends who know my heart and have seen my life over the.
last 36 years are like, "You are absolutely a Hong Konger.".
And so the question I often grapple with is, "Who actually am I?.
Am I British?".
Well, my passport tells me so..
"Am I a Hong Konger?".
My heart tells me so..
The reality is I probably exist somewhere between the two of those realities..
I'm a classic third culture kid..
I would imagine there's a bunch of others in the room online right now who would resonate.
with that..
I was born in one culture..
I grew up in another culture..
And over that, I've kind of gotten lost between those two cultures..
And I've had to try to work out, "Well, who exactly am I?".
And to take the best of one culture and the best of another culture and try to bring those.
things together..
But if I'm honest with you, I think the way that a lot of third culture kids feel is you.
just feel kind of adrift on an ocean of cultural and national identity that you never really.
feel like you're settling into..
You never really feel like you've got an anchor point that you can kind of put down in any.
place where you can find some roots and you long to find a home..
You long to have a group of people where you generally feel like you fully fit into and.
are, interestingly and importantly, fully embraced by those people..
It's funny, my wife Christine, she'll often say to me, "Andrew, you're a chameleon when.
it comes to culture..
You have an amazing ability to adapt to the color of the majority culture that you're.
finding yourself operating in in that moment.".
If I'm around a bunch of Americans, I start sounding more American..
If I'm with Australians, I become....
You can't even understand what I'm saying..
If I'm with Australians, because it's Australians..
Had to do it..
But I have this amazing ability to sort of adapt to whatever majority culture I'm around.
because I'm desperately trying to find people that will accept me..
And in fact, one of the biggest questions, I think, of my adult life has always been.
who exactly are my people and will I ever fit in?.
And as we continue our Exodus journey and we get to the midway point of the second chapter,.
that's exactly what is going on for Moses..

$^{81}$Moses is in a growing tension of an identity crisis..
He's wondering, "Who really are my people and where do I fit in?".
We saw last week that Moses born to a Levite family, so Hebrew in his blood and origin.
in that sense..
But at a young age, he's put in a basket and sent off down the Nile River and Pharaoh's.
daughter from the Egyptian culture finds him and adopts him into her family..
Now God, because He's incredible and miraculous and does all these amazing things, manages.
to get Moses back to his biological mother who gets the joy of nursing him..
Scholars think that that took place over probably three to four years..
So for three or four years, Moses was back in his Hebrew family, biological family..
And I'm sure during that time, his mother and father and everybody else around immersed.
him in the Hebrew culture, helped him to understand what it meant to be a believer in a monotheistic.
God, the father of Abraham, Isaac, and Jacob, and Joseph, to help him to journey in the.
understanding of what it meant to be a person, a part of the people of God..
But then at the age of three or four, his mother takes him back and hands him back to.
Pharaoh's daughter, who from that point forward raises him as her own son..
And just as if Moses had been taken and drawn out of the waters of the Nile, he's now taken.
and drawn out of his Hebrew culture and he's placed into a different culture..
And he grows up now immersed around that culture as a prince within the Egyptian royal courts..
Now interestingly, fascinatingly, Moses, who's writing the book of Exodus, doesn't tell us.
anything about that time, almost as if maybe he's embarrassed to talk about it..
But he doesn't mention anything about it..
In fact, we have to look outside of the book of Exodus to find any commentary about what.
it was like for Moses that time..
There are some commentary in the rest of the scriptures and also external to scripture.
in history..
Let me just give you some examples..
Acts chapter seven is the story of Stephen standing before the Sanhedrin, his power,.
authority of his people in the day, the Jewish people..
And he's fighting for his life and he's fighting for his life by telling them the story of.
his people..
And he goes right back to Abraham and he works through..
And of course, he speaks about Moses..
Let me read you from Acts seven, verse 21..
He says, "When he was placed outside, Pharaoh's daughter took him and brought him up as her.
own son..
Moses was educated in the wisdom of the Egyptians and was powerful in speech and action.".
We get a little glimpse right there..
Moses understandably was raised in the educational system of Egypt, but not just the educational.
system..

$^{121}$He's remembering he's in the royal court..
And you can do some research here and find out what it was like to be educated in the.
royal court of that time..
Moses would have received the best education in the world at that time..
He would have been taught maths..
He would have been taught political discourse and political history..
He would have been taught military strategies..
He would have been taught languages, architecture, a whole bunch of different things..
Philosophy..
He would have been taught importantly about the Egyptian religious system and how all.
of the spiritual deities of Egypt work..
All of this would have been his education..
And Stephen tells us here, "He was powerful in speech and deed.".
So he was good at what he had learned..
He'd learn all this stuff and it had become a part of who he was..
Interestingly, the first century historian, Josephus, also writes about Moses in this.
time..
And he tells us two things..
One, that Moses was being groomed by Pharaoh to take over as Pharaoh in Egypt..
Now it's likely that the Pharaoh of the time of the Exodus did not have his own son..
And although Moses was adopted by Pharaoh's daughter, it looks like through history that.
actually Moses and Pharaoh almost became a father-son kind of relationship..
And this father figure over Moses was grooming him to take over the running of Egypt..
The second thing that Josephus tells us is that Moses actually had led a military conquest,.
a victorious military conquest in Ethiopia in that time..
So I want you to understand something really important here..
Moses being adopted into Pharaoh's family doesn't mean that Moses was like the embarrassing.
third uncle that you kind of wish never came to the Christmas party and you're kind of.
embarrassed about him..
Moses wasn't some Hebrew that was embarrassing to the family and they kind of shoved him.
off in the side during all the important events..
Moses was centered into the powerful, oppressive work of the Egyptian empire..
He was being groomed to be Pharaoh..
Not only that, but he had already led military conquest on behalf of Egypt's power and might..
And he had done all of this in absolute opulent prosperity, living in one of the wealthiest.
families of that time..
And Moses, now 40 years old, is struggling with exactly who he is..
He's basically going to have a midlife crisis in the passage we look at today..
And he's like, "Hang on, who am I?.
Am I a Hebrew?.

$^{161}$Well, by blood, yes..
Am I Egyptian?.
Well, by my literal and metaphorical clothes, my identity, yes I am..
So does that mean then that I'm one of the oppressed or am I actually the oppressor?".
Moses is wondering, "Am I really a liberator of my people or actually am I a part of the.
very problem?".
And this internal conflict that's happening for Moses quickly enables Moses to realize,.
and for us as we look at this passage to realize, that in the Exodus story, there isn't this.
clear cut idea of the good and the bad..
But as so often in all of our lives, there's a lot of nuance and gray and the coming together.
of a lot of different things..
And just like me, who had been born in one culture and raised in the other and was desperately.
trying to work out who I am, Moses, born in one culture, raised in another, and now at.
40 years old going, "What do I do next?".
And it's all of that tension that we now turn our attention to as I take you back to the.
land of Egypt..
Let's have a look..
[Music].
So the kind of TV shows or movies that I've always been drawn to are the ones where there.
are no cliché good guys versus bad guys, but where every character has the propensity.
for moments of both great goodness and great evil..
And I think one of the reasons why I'm drawn to this kind of storytelling is because it.
reveals to us something about our shared common humanity, that each one of us has the potentiality.
to be both oppressed and the oppressor, to be winners as well as losers..
Every single one of us has had moments that we have been deeply proud of, but we've also.
had moments that we've regretted and have deeply buried..
You know, if we deny this, we actually enter into one of the slaveries that I think so.
many of us become blind to, and that's the slavery of denial..
As the Exodus narrative continues, we quickly discover that there are no straight cut good.
guys and bad guys..
At the halfway mark of the second chapter, we discover Moses as a grown man, and not.
just any man, but a prince of Egypt, having been raised in Pharaoh's very household..
I mean, consider the privilege and prestige that had been afforded Moses..
Despite his Hebrew origin, he was raised by one of the wealthiest Egyptians in the land,.
and was surrounded day and night by its customs and culture..
He had been cared for, raised, and educated by Pharaoh and his family, all while his own.
people suffered at the hand of their oppression..
At this point in the story, Moses is not the Jewish people's hero, but actually a part.
of the problem..
Perhaps even worse, for Moses was a Jew adopted into the family of the oppressors, and he.

$^{201}$had not done anything at all to stop that oppression..
But suddenly, everything changes..
One day, you see, Moses is out, and he's just observing the hard labor that his people are.
under, and he sees an Egyptian beating up a Hebrew..
Now, that was probably an everyday occurrence in that time, but for whatever reason, on.
this particular day, Moses decides to act..
He attacks the Egyptian, and it ends up actually killing him..
And in his panic, he decides to bury the Egyptian in the sand, hoping that in burying that Egyptian,.
he'd be able to distance himself from the violence that he's created, hoping to cover.
up his crime, if you will..
Now, before we judge Moses a little bit too harshly, I think we have to realize that all.
of us have the tendency to do this..
We may not be out there murdering people, but we hurt people..
We say things..
We actually pull people down, and when we do so, we so often try to bury that hurt,.
try to bury that shame, try to distance ourselves from it, and in so doing, actually, try to.
provide for us a cover for the thing that we've done..
Maybe, perhaps, just like Moses, rather than burying these things, we need to actually.
expose them in order for them to be redeemed..
(Music).
In many ways, Moses had been burying things for years..
He had buried his Hebrew ancestry in the sand of Egyptian culture and customs, hiding his.
true identity in the power of Pharaoh's royalty and prestige, remaining blind to the oppression.
around him..
This is what we do when we're trapped in the slavery of denial..
We run from the realities of our own contribution to the oppression of others by burying ourselves.
in self-righteousness that placates our guilt and shame and keeps us aloof, uncaring, unfeeling,.
and disconnected..
Moses murders an Egyptian and buries him in the hope that he can remain disconnected from.
his crime and get away from his violence..
But he's about to discover something we always seem to forget when trapped in the slavery.
of denial..
Our mistakes rarely stay buried forever..
Edward gets back to Pharaoh about what Moses has done, and not surprisingly, Pharaoh is.
furious..
I mean, the very one that he had raised almost like his own son, the one who had been adopted.
by his daughter, the one who he had high hopes for in his own court, has actually killed.
an Egyptian subject..
And how does Pharaoh respond?.
Well, he decides he needs to kill Moses, that there needs to be blood for blood..

$^{241}$Well Moses hears about the scheme of Pharaoh against him, and so he flees to the far side.
of Midian and comes to a well that's not so dissimilar to this one right here..
Now, these wells were a lifeline in the harsh desert conditions like these around me right.
now, and no doubt Moses would have camped near this kind of well for a couple of days,.
perhaps reflecting back on the week that had just been..
How one random act of violence had turned his whole life upside down, and how now his.
own family was trying to kill him..
And this is an aspect of the Exodus story that I think it's easy for us all to miss..
When we think of the Exodus story, we think of Pharaoh and Moses like bitter enemies against.
each other, as if they're like a protagonist and an antagonist, a good guy and a bad guy..
But actually the story is far more nuanced than that..
Actually the start of the Exodus narrative is a story about a betrayal within a family,.
about a son who feels betrayed by his adopted father, how a father feels betrayed by his.
adopted son..
And all of that combines together to create such fury and animosity and oppression that.
the whole of the Exodus journey begins from that point..
And I think all of this teaches us something that's quite critical about our own journeys.
from slavery to freedom, and it's this..
That so often the most painful moments of oppression occur not by those who are strangers.
to us, but by those that we love the most..
The wound causes Moses to bury once again..
He's lost his Egyptian family and he's run from his own people who are in need..
He has a deep sense of shame attached to both..
And like we all so often do when faced with our failures and our brokenness, he flees.
into lies and self-denial, recreating a new identity with a new group of people, and in.
so doing hoping he can permanently bury his past..
But that's not the way it works with God..
God is in the business of revealing that which is hidden in darkness..
And for Moses, a new light is about to shine..
God is in the business of revealing the things that are hidden in darkness..
What we're about to look at as we open up this story in a bit more detail together is.
I think the most significant thing that you need to open your heart to..
If you duly want to go in an Exodus journey yourself, I can't stress how important what.
happens next for Moses is for us as we think about our own liberation from the things that.
we also want to hide..
Let me read this to you from chapter 2, verse 11..
Everybody okay still?.
How beautiful by the way..
That was Jordan, not Egypt by the way..
Isn't that cool?.

$^{281}$Yeah..
Every week everybody's coming up to me and they're like, "I want to go to Egypt now.".
And I'm like, "Yeah, that's great.".
But that was Jordan, just want to say..
One day after Moses had grown up, he went out to where his own people were and watched.
them at their hard labor..
He saw an Egyptian beating a Hebrew, one of his own people..
Glancing this way and that, he killed him and then took the Egyptian and hid him in.
the sand..
The next day he went out and saw two Hebrews fighting..
Let's just stop there..
Let me read 12 again..
"Glancing this way and that and seeing no one, he killed the Egyptian and hid him in.
the sand.".
This is a turning point in Moses' life, a complete and utter radical turning point..
He's about the age 40 here..
And like I said before, Moses doesn't give us any details about the life prior to this.
point..
Almost as if he wants to just kind of rush over that and get to this point..
He's 40 years old, dressed in the robes of his Egyptian culture and identity..
And yet today he goes out and probably does what he had done on a very regular basis,.
take a look at what was happening amongst his own flesh and blood, amongst the Hebrew.
people..
And he sees a scene that would have been part of the everyday life of that time, an Egyptian.
slave master being tough on a Hebrew slave..
And again, Moses would have seen that many, many times..
But this day, something different takes place..
Question is, why?.
Well, Moses helps us to understand that by a word he uses here..
It says in the English, "He watched them at their labor.".
This word "watch" that Moses writes here is the word "ra'ah.".
And ra'ah is a very important Hebrew word that actually is now repeated from this point.
forward about five times in the next three chapters..
The word ra'ah means this..
It means to see with emotion, to see something and be moved by it..
Not just to observe something, but now to be moved by it..
This is the difference you see between Moses having seen the oppression of his people for.
40 years and never done anything about it, to this moment now wanting, stirred to, something.
in his spirit that moves him to want to act..
This is the exact same word that we saw last week that was applied to Pharaoh's daughter.

$^{321}$when she sees Moses in the waters and she is moved with compassion to save him..
She ra'ahs, she sees him with emotion, draws him up and brings him into her family..
The same word is being used here..
And just like we said last week, the word holds with it the idea that God's spirit is.
involved in the emotive, the compassion, the desire to do something, the heart change..
God's spirit is at work in it..
So what we're seeing here in the start of our story today is that Moses' heart is being.
changed..
There's a significant shift that's happening for him as he now sees with emotion and compassion.
and he's beginning to understand, "Hang on, these are my people..
Hang on, there's something going on here..
These are not just the oppressed around us..
These are actually my people and my heart is being stirred to kind of do something.".
But this is a difficult journey for him because he's still caught between these two identities..
He's still struggling with, "Am I this or am I that?".
The writer in the book of Hebrews, very interestingly, actually pulls out this exact moment and he.
says something that's, I think, quite helpful for us to see..
This is Hebrews chapter 11, starting in verse 24..
It says, "By faith, Moses, when he had grown up, refused to be known as the son of Pharaoh's.
daughter..
He chose instead to be mistreated along with the people of God rather than to enjoy the.
pleasures of sin for a short time.".
Now I think the writer of Hebrews here is writing about what took place for Moses after.
this moment..
Most scholars would argue, "Is it before this moment?.
Is it after this moment?".
I think it's after..
I think God is the starting point of stirring something in Moses' heart as he sees his people.
on this particular day and he will go on to align himself more and more with his Israelite.
identity and move away from his Egyptian one..
But in this moment, he sees something, it stirs his heart, and he wants to respond..
But this tension is real and this tension is also real for us..
Moses is struggling with his identity and realizing that the identity that he's embraced.
is not now the identity that God is beginning to speak to him about..
Are you following that?.
And that's going to be every person in here's journey if you truly want to go on Exodus..
You're going to have to realize that the identity that you have embraced is not necessarily.
the one that God wants you to walk in..
Each one of us wear the clothes of identities that we have embraced throughout our lives.
that do not represent the full truth of who we are..

$^{361}$Come on church..
That's the first thing you need to grapple with here..
That you have identities that have been formed in you, shaped in you by preferences and choices,.
that have been formed and shaped in you by family dynamics, that have been formed and.
shaped in you by your marriages and by the people around you in your workplaces, that.
have been formed and shaped in you by your thinking and your attitude, formed and shaped.
in you by your sin and your brokenness..
And those identities are often the things that we wear to try to find a place in this.
world and you will never, never journey in Exodus until you have the courage to step.
away from some of those identities..
The starting point of all of our journeys in Exodus is to remove the clothes that we.
are wearing that cover up the person that we truly want to be in Christ Jesus..
We have to have the courage to remove those identities, to recognize that those identities.
are ours, to be honest about them and sober about them and say that this identity has.
come and been placed upon me and I'm in this middle ground..
I'm a third culture person right now and I know what Christ wants..
Christ is stirring something in my heart..
There's a call to liberation..
There's a call to freedom from my sin and my brokenness and yet I recognize that I'm.
still dressed in the clothes of the oppressing identity..
Something has to change..
Some of us in this room, here's the honest reality..
We have gotten comfortable in the clothes of Egypt..
I'm going to say that again because I just felt strongly the Spirit of God for you..
The honest truth is I'm like this as well..
We have become comfortable in the clothes of Egypt..
If we're going to really be honest and truthful and sober about this series that we're in,.
it's only going to happen if we're willing to recognize that we've become comfortable.
in identities that have come to shape and form us and make us think that we're someone.
when the reality is we're only someone in Christ Jesus..
That invitation is a bold one from God and it's not an easy one to wrestle with and Moses.
is wrestling with it and here's how he tries to respond with it..
He sees an Egyptian slave master beating on one of his own people and in kind of going,.
"Well, I think I am Hebrew..
I am Hebrew and I want my Hebrew people to identify with me and I want them to accept.
me," he steps in and does what only he knows because of how he's been raised to kind of.
sort of sort out a problem that's happening between two people..
He goes in and he kills the Egyptian..
I want you to see this church because it's super important..
Moses being stirred by God in his heart to stand on behalf of the oppression of his people,.

$^{401}$his first act to be a liberator of his people is violence and murder..
In other words, Moses, even though his heart is beginning to be stirred, his clothes are.
still very Egyptian and so he has a heart to move..
He has the right idea, the wrong method and the Egyptian clothes that he wears metaphorically.
and as well as literally are shaping him still and he thinks, "Oh, well, this is the way.
I liberate my people..
I need to go in with that mighty kind of military strategy..
Look what I did in Ethiopia..
I can do that now on behalf of my people.".
He's basically acting like Pharaoh in order to be a liberator of his people..
Are you following this?.
And when we act in the brokenness that we embrace, in the wrong identities that we have.
to try to do God's will, there's always going to be a problem with that..
Need to understand that you cannot bring about God's exodus through your own brokenness..
Hello?.
You guys okay?.
You cannot bring about God's exodus through your own brokenness..
Here's the reality..
It is not your exodus..
So many of us, as we start this series, we're wanting exodus from the things that are holding.
us, from the things that are enslaving us and we have an idea of what that exodus will.
look like..
We have to let go of our agenda of exodus to actually receive God's exodus that he has.
for us..
Your brokenness is never going to be the vehicle by which you're going to be able to act in.
exodus..
And here's Moses' brokenness as one of Pharaoh who's thinking that it's all about power,.
it's all about prestige, it's all about might..
That psalm that says, "Some trust in horses, some trust in chariots, but I trust in the.
name of the Lord.".
Moses has the mentality, "We trust in horses..
We trust in chariots..
It's our wisdom..
It's our strength.".
And I think Moses took it even further..
I think Moses was thinking, "This is maybe why God has raised me in Pharaoh's household,.
so that I would be well-educated, so I would be strong, so I would know all the strategies,.
so I can now go and stand in the gap and just in the way that we killed all of those Ethiopians,.
I can now kill this Egyptian in the hopes of setting my people free.".
And you get the sense that God is standing back from all that and going, "Hmm.".

$^{441}$Moses buries his sin..
He kills somebody, had no authority to do so..
And he knew it was wrong, that's why he buries the Egyptian..
And he buries him in sand because he wants to get away with the sin..
You can almost see the shame that's inside of him..
It says in the scripture that he looked left and he looked right..
The thing he failed to do was to look up..
And he takes this Egyptian and he buries him in the sand, hoping he's going to get away.
with it, thinking he's gotten away with it..
In fact, so much so that he goes back the next day to see how his people is doing, thinking.
that he's completely got away with everything..
And we have to be really honest with ourselves and realize we do this too..
Just like I said in the film, we have a habit of burying our sin..
We have a habit of covering up the things that we do that we know is not what God wants.
for us, what we know is not right..
And we cover these things up and we kind of pretend like they haven't happened..
You want to know what I think is the true global pandemic in the church today?.
It's secret sin..
In fact, when I was preparing this series four years ago, I felt the Holy Spirit say.
to me very strongly, "Andrew, there's a lot of secret sin in the vine..
There's a lot of sin that we're doing knowingly and we're covering up with the sand to avoid.
and we think we're getting away with it.".
Yeah, I was on that website, but I cleared the history..
My wife won't see it..
Yeah, I did that dodgy business deal, but I trust the person..
He's not going to tell anyone..
It's going to be fine..
When the church walks in the secrecy of sin, the enemy has won because the enemy's nature.
is to be secretive..
The Lord's nature is to be open and vulnerable and transparent and real..
And the more that we operate in the secretness of our sin, the more the enemy has won and.
the power of the church is reduced..
And God's standing over us and He's going, "You can bury all you want, but I know..
I see it..
I love you..
I'm not here to condemn you..
I'm not here to make you feel terrible..
I'm here to set you free..
But in order for you to be set free, you've got to stop burying stuff..
You've got to stop doing that..

$^{481}$You've got to start working out that actually your honesty, your transparency, your willingness.
to not be like Adam and Eve who were once naked and unashamed, who through their sin.
became naked and ashamed, covered themselves up..
And God shows up and says, "Who told you you were naked?".
In other words, I'm welcoming you back into a relationship with me where there doesn't.
need to be secrecy, there doesn't need to be hiddenness, there doesn't need to be brokenness..
I want to set you free..
And Moses, like us, has killed, has felt shame, has buried..
And any time you bury sin, any time you bury shame, any time you bury guilt, it is just.
festering and festering and festering, and it has the power to come up and ruin you unless.
the Spirit of God comes and enables you through His beautiful kindness and conviction that.
leads to repentance where you can say, "Lord, I am ashamed of this, but in your eyes, you.
love me..
Here it is..
Set me free.".
That's the heart of Exodus..
Are you with me still so far?.
Now I want you to see what happens next, verse 13 onwards..
I'm enjoying this..
Are you guys okay?.
Verse 13 says this, "The next day he went out and saw two Hebrews fighting.".
So the next day, thinking he's gotten away with everything, goes back out to be the savior.
of his people, right?.
He asked one that was wrong, "Why are you hitting your fellow Hebrew?".
The man said, "Who made you ruler and judge over us?.
Are you thinking of killing me just as you killed the Egyptian?".
Ouch!.
It's like a mic drop moment..
We know what you did..
Everybody knows what you did..
We're all talking about it..
We all know..
You can't hide this thing..
You killed an Egyptian, and that's really bad news for all of us..
So are you going to kill us now like you killed the Egyptian?.
Are you going to try to cover up all of your mistakes all the time?.
Then Moses was afraid, understandably, and thought, "What I did must have become known.".
I love it..
It's really interesting what they say here..
They say to Moses, "Who made you ruler and judge over us?".

$^{521}$In other words, we didn't ask for a deliverer..
We didn't ask for you to be our deliverer..
"Who made you ruler and judge over us?.
Because you're still dressed in your Egyptian identity..
You're still dressed in your Egyptian clothes..
You're still thinking like a pharaoh, and we see you, Moses, as one of the oppressors,.
not one of the oppressed..
Oh, you might have our blood, but that's all you have..
Who made you?.
Who decided that you could be our ruler and judge?".
What they were basically saying to him is this, "Don't you realize that we don't want.
you?.
We don't want your pharaoh power in our lives..
In fact, Moses, you're a part of the problem..
You killing the Egyptian yesterday has made our lives worse.".
He's seen this..
Now there's also something else going on here..
God is also speaking to Moses..
Because when these two Hebrews say to him, "Who made you ruler and judge over us?".
God is speaking to Moses, and God is saying, "Who made you ruler and judge over my people?.
Have I done that yet?.
No, no, no, I haven't done that yet..
Oh, I'm stirring your heart by my spirit..
I'm rehiring you..
You're beginning to see the injustice of my people..
I'm beginning to work, but there's so much, Moses, that has to happen in your life, so.
much that needs to change in your life before I am going to use you to be my deliverer..
Because if I used you now, your methods and your thinking is going to be the world's wisdom,.
the world's strength, your mighty army strength..
I have so much I need to unravel, so much I need to get out of you, so much I need to.
change in you..
I have not released you..
You're in fact going to live for 40 years in Midian in this horrible desert before I.
even show up as a burning bush and begin to officially call you to be a liberator..
You are not that person right now, and you need to realize there's work that needs to.
be done.".
Do you see it?.
And God is being pretty strong with Moses here..
"I haven't told you yet, and yet you're jumping the gun..
You're getting straight into this stuff, and that's not the way it works here.".

$^{561}$And Moses realizes that there is so much he needs to deal with..
You see, for Moses, he is going to quickly realize that in order for him to be his people's.
deliverer, he's first of all going to need deliverance himself..
Come on, church..
In order for him to eventually become his people's deliverer, he's first of all going.
to have to experience deliverance himself..
And I want to be bold and stand before you all today and say this..
If you are enjoying this series, if you're feeling like the Lord is calling you into.
Exodus, here's the bold, sobering reality..
You will need some deliverance..
You are going to need, like Moses, to get rid of some of your faulty identities..
You're going to need to be willing to allow God to come and stir some things in you..
The journey of Exodus is not all cupcakes and My Little Ponies..
The journey of Exodus is gritty and real..
And if it's going to have any lasting change, it starts with the reality that we are dressed.
in the wrong clothes some of the time..
And God, would you come and, hmm, would you come and help me?.
Notice what happens next, because this is tough..
Verse 15, "When Pharaoh heard of this, he tried to kill Moses..
But Moses fled from Pharaoh and went to live in Midian.".
Fascinating..
Remember what I said in the film, and this is so true..
This is not just a good guy and a bad guy who are suddenly fighting against each other.
now..
This is an adopted father who is now heartbroken because of what his adopted son has done..
This is an adopted son who's wrestling with the oppression of his adopted father, who's.
trying to work that through in his head..
This is Pharaoh who probably always harbored insecurities that maybe one day Moses might.
turn against him..
That maybe one day because of his Hebrew blood, maybe one day he might..
And so perhaps Moses, oh sorry, perhaps Pharaoh in hoping to get all of that out of Moses,.
puts him in the school, gets them all pharaohed up, tells him he's going to groom him to be.
the next leader, sends him on a conquest to Ethiopia, because all of that is designed.
by Pharaoh to get the Hebrew out of Moses..
And now we're at a point where Pharaoh realizes that all that hard work has achieved nothing,.
because when it came to it, Moses has now chosen his people..
And so Pharaoh's response is blood for blood..
You kill one of my subjects, I'm going to kill you..
And couldn't you imagine the depth of pain now for Moses?.
I want you to see this, really important, because not only has Moses been rejected by.

$^{601}$his own people, the Hebrews, who have said, "We don't want you as judge over us.".
He's now been rejected by his own family, his own people that loved him and raised him.
for 40 years..
They now want to kill him..
And Moses is now in a place where he's like, "Hang on a sec, literally a minute ago I was.
somebody and now I'm nobody.".
And God goes, "Now we can get to work.".
Come on church, now we can get to work..
There's a journey that you have to go on, from being a somebody in the world's eyes.
to being a nobody in the world's eyes, so that God can then turn you into a somebody.
in his eyes..
So that he can now come and begin to form and shape in you the person that he truly.
is calling you to be..
This is Moses' great moment of disruption and disorientation..
And if you're anything like me, seasons of disorientation are disorientating..
They're not easy..
They're hard..
They're painful..
But you need to understand something very central to the Exodus journey..
There is no process of liberation until there is first a process of disorientation..
There is no ability for you to be set free until you first wrestle deeply with the reality.
of your brokenness..
There is no liberation that can happen until you first face disorientation, disruption,.
the recognition of a brokenness, and to say, "This is not what I want.".
And Moses, as painful as it is, is now stripped from all of his identities..
And he flees to Midian, and he sits down at a well, and he begins to form a new identity..
And we don't have time to look at the story today, but he meets a woman, and he does this.
amazing thing, and he saves this family, and they suddenly bring him into their family,.
and he has kids, and he builds a whole new life as a shepherd..
His father-in-law is a shepherd..
He becomes a shepherd..
All of that for 40 years before God even shows up in the burning bush and does anything..
And it's like God is saying this, "What the world meant for evil, I'm going to use for.
good..
You just got rejected by your people and rejected by me, sorry, by Pharaoh's family..
I'm going to step in now, and what was meant for evil, I'm going to now turn for good.".
None of this was a surprise to God, and God steps into the circumstances of brokenness,.
and he meets Moses and says, "Now we can begin.".
And here's the question, are you willing to embrace the disorientation of the broken identities.
that sit over your life?.

$^{641}$Are you willing to soberly see them with emotion, to recognize them, and where you have up until.
now just buried them, be willing to come to the Lord with an open hand and say, "God,.
this is what's going on, and I cannot, will not, don't want to..
And Lord, it is breaking me..
It is warring in me..
I am disorientated..
I am disrupted, and God, I am desperately in need..
And in order for me to move into your spirit and with what you want, I need to now purge..
Help me, Lord, do that.".
The season of disorientation starts by asking three very important questions..
I want to share these questions, and then I'm going to pray with you..
Here's the first question..
What clothes are you wearing today that God needs to strip away from you?.
All of us are carrying identities that I've talked about earlier that God wants to deal.
with..
What are the ones that you're carrying with you?.
Number two, what sins have you committed that you have long since buried that God in his.
love and grace wants to expose so he can heal you?.
What is the sand that you've been placing over the box that you've put stuff in, hidden.
away from everybody so they can't see it, and you're covering it all up?.
What are the sins that you're getting away with or think you're getting away with right.
now that God actually wants to begin to speak to you about and put his finger on out of.
his grace and mercy so that he can truly set you free?.
Here's the third and final one..
What desert might God need to bring you to in order to remove from you the earthly power.
and privilege that only gets in the way of God's power being made manifest in your life?.
That's the place of disorientation..
Where does God need to bring you to so that he can shake away the comforts of this world.
so that you can finally find comfort in him?.
That is how we start Exodus..
And Moses, as painful as it was for him to sit down in that well and wonder, "How did.
my life just get turned upside down?".
It is now that God can meet him and begin to form in him the leader he would become.
so that he wouldn't stand before Pharaoh with Pharaoh's authority and power, but instead.
with a simple shepherd's staff and say plainly before him, "Let my people go.".
As I stand before you today as your pastor, that's what I feel for you..
I stand before you and I say to the enemy who is trying to hold you back, "Let my people.
go.".
I wonder whether you would pray with me..
Let's pray..

$^{681}$Father, we come before you here..
Father, just settling ourselves into the uncomfortableness of disorientation, being willing to be disrupted.
by you..
Father, I want to pray for us as a community of faith..
Father, for each person here, it's their own journey..
But Lord, I want to pray for the kindness of your presence to convict, particularly.
for us who have secret sin that we haven't shared with anyone, maybe out of shame, guilt,.
whatever it might be, that we are hiding from our loved ones, and most importantly, hiding.
from you..
We know we can't hide anything from you..
You see all things, but we bury nonetheless..
Father, I pray for each person here who can resonate with the reality of buried sin..
I pray in your gentleness and kindness you'd meet them now..
I pray for courage for each person in this room to do the right thing..
Father, I pray for those of us in this room who are trying to bring about Exodus on our.
own terms, just like Moses wearing the wrong identity, thinking that liberation could come.
by might and power..
Forgive us, Lord, where we think our own Exodus comes through might and power of our own strength..
Let's try harder, power of positive thinking, whatever it might be..
Father, would we submit ourselves to your Exodus and your Exodus alone?.
And even if that means an uncomfortable season of disorientation, Father, would you give.
us the courage to walk into it?.
And Father, anyone here who knows that there's an identity that they're holding, maybe something.
that they've made an idol of, maybe it's the work you do, maybe it's the status you have,.
maybe it's the followers you've got, maybe it's whatever it is, but there's something.
that is, there's an identity that works in you and works, and you know it's your weak.
place..
You know it's the place that you use to cover up insecurity..
It's the place that you use to cover up your brokenness..
It's the outward facing, I've got it all together person..
And I wonder whether over this series you would have the courage to allow God to change.
that identity, to trust him that he knows you better than you know yourself, and that.
you'd be identified as his child, as the greatest identity you could ever hold..
And it doesn't mean that you can't be a great banker or a great lawyer, or you can't have.
a great marriage, or you can't be influential..
All those things can be there, but they come through our identity in Christ..
Those things are not our identity..
And I say that over some of you in the name of Jesus, because you need to hear it, and.
God wants to deal with it in your life..
So whatever it might mean for you, just take a moment now to let the Holy Spirit speak.

$^{721}$to you..
(upbeat music).
\newpage



\section{}
\label{sec:1GHdbHuySwQ}
\textbf{2023-05-15 EXODUS - 05 A God Who Hears and Remembers [1GHdbHuySwQ].mp3}
\newline
\newline
連結: \href{https://youtube.com/watch?v=1GHdbHuySwQ}{\texttt{ https://youtube.com/watch?v=1GHdbHuySwQ}} ~~~~ 語音日期: 2023-05-15 
\newline
\newline
\hyperref[sec:gUw91uNiAAg]{\small{< < < PREV SERMON < < <}}
~
\hyperref[sec:index]{\small{[返主目錄]}}
~
\hyperref[sec:j3pdS8tSSoA]{\small{> > > NEXT SERMON > > >}}
\newline
\newline
$^{1}$(audience applauding).
- That's great..
As you sit down, why don't you say hi.
to the person next to you, say hey..
That was a really beautiful moment I just shared with you..
Who are you?.
So good..
Well, welcome to The Vine..
My name's Andrew..
I'm one of the pastors here,.
and we are so glad you are here with us..
And we're in week five of our Exodus series here,.
and in week five, at the end of chapter two,.
everything changes..
In fact, this is a turning point.
in the whole of the story of Exodus.
that we're looking at today..
This is a moment where God steps in.
and does what only God can do..
And just in that song that we've just been singing.
and that desire that we have for God.
to fight battles for us, God to come through for us,.
this is the moment in the story where God reveals himself..
And in revealing himself,.
he creates for his people renewed hope..
I told you in week one of this series.
that the book of Exodus is provided to us.
and is shaped for us in five movements..
And those movements are slavery,.
promise, liberation, identity, and home..
And over the last four weeks,.
we've been immersed in the slavery aspect.
of those five movements..
We've been looking at the reality of slavery,.
and we've been looking at it, yes,.
in terms of what the scriptures tell us about Egypt.
and about Israel and the slavery that they're going through..
But more than that, we've been focusing on stepping back.
and saying, what about our slavery?.
What about our brokenness?.

$^{41}$What about the stuff that we're dealing with?.
And how do these two things create and come together?.
And we've been seeing for Israel, so also for us,.
that slavery is a total thing..
Slavery is completely holistic..
We saw in week two how slavery.
was not just physical for the Israelites,.
but it was physical, emotional, social, spiritual..
And it's the same for us,.
that our slavery to sin.
doesn't just impact our spiritual lives,.
it impacts everything that we are, everything that we do,.
the totality of who we are, our relationships,.
what we do in our workplaces,.
how we are in our city and our neighborhoods..
All of that is impacted by the realities.
of the slavery of our sin..
And over the last four weeks,.
we've seen some pretty tough stuff..
Just last week, we saw Moses fall.
into a place of disruption and disorientation,.
having worn for far too long the wrong identities,.
the wrong clothes of who he truly was,.
wrestling with the reality of,.
am I an Egyptian or am I a Hebrew?.
Am I an oppressor or am I a part of the oppressed?.
And we saw last week that God comes.
and begins to stir something in his heart..
And all of our journeys from slavery to freedom.
starts because God begins to burn something,.
turn something, do something in our hearts..
And Moses' heart begins to change.
and he looks out on his people..
And although he had been looking out on them for 40 years,.
on this particular day, he looks out.
and something changes, he sees them differently..
And because he sees them differently,.
he decides he wants to react..
But we saw last week how he reacts..
He ends up murdering an Egyptian.

$^{81}$who's abusing one of his people..
And we look at this and we can see.
Moses has the right heart, but he has the wrong method..
He has the right kind of desire,.
but he has the wrong idea on how to put that together.
because slavery to sin has so immersed him.
that his identities are so broken.
that he doesn't know how to react and respond.
to what it is that God is doing in him..
And in this rightful heart to wanna be a deliverer,.
he ends up committing the sin of murder..
And we saw last week that Pharaoh finds out about this.
and Pharaoh is so angry, so upset..
We might think, well, why?.
It's because Moses had been adopted into Pharaoh's family..
This wasn't just a tiff between two leaders.
of different nations..
This was a brokenness within a family..
And the very one who was hoping to raise Moses.
to take over as Pharaoh, now that one wants to kill Moses..
And Moses is so broken by this,.
he's so torn up by all of this,.
that in fear of his life, he flees from Egypt.
and he goes all the way over as far as he can.
through the desert to a place called Midian..
And we're gonna talk a little bit more next week.
as we come to the burning bush..
I can't wait to come to the burning bush next week..
We're gonna look all about Midian then,.
but in Midian, Moses finds himself disjointed,.
disorientated, he's a wreck of a man..
He's dealing with all this stuff.
that's happening in his identity..
And it's almost as if we're traveling.
through the first two chapters of this movement of Exodus.
and we can see how completely oppressive slavery is..
And that oppression is so deep that Moses flees.
and finds himself completely turned upside down..
And he realizes something that we all have to realize.
in our own journeys of Exodus..

$^{121}$And that's that we can't save ourselves..
We're not the ones who are able to free us from our sin..
We're not the ones that can free our people..
We come to this realization that actually.
we need something outside of ourselves,.
something bigger than us, something more powerful than us.
to step in and change the situation that we cannot change..
And when we try to change it like Moses did,.
we only often fall more and more into sin..
Our own human effort can never create salvation for us..
And so Moses sitting at this well in Midian,.
the wreck of a man, realizes that there has.
to be something more..
There has to be something outside of himself.
and he's longing for a promise..
He's longing to get a promise that God will come through..
And I sense in this room this morning,.
and for those that are joining us online,.
there's a lot of people here who are longing for a promise..
Oh God, speak to me..
God, say something to me..
Give me a promise that everything's gonna be okay..
And you have to understand that the journey of Exodus.
always begins with the coming of a promise..
The beginning of Exodus always begins.
with a coming of a promise..
And as bad as slavery has been, a promise now comes..
And when God brings a promise,.
just like I said a few moments ago,.
nothing can hold it back..
And I wanna show you the beauty of the coming of a promise..
This is the last part of Exodus chapter two..
During that long period, remember Moses is writing,.
during that long period, the king of Egypt, Pharaoh, died..
The Israelites groaned in their slavery and cried out..
And their cry to help because of their slavery.
went up to God..
God heard their groaning and he remembered his covenant.
with Abraham, with Isaac, and with Jacob..
And so God looked on the Israelites.

$^{161}$and was concerned about them..
Just a few verses in the Bible here,.
but this is a turning point in everything.
that is about to happen next in Exodus..
And I want you to see how cheeky Moses is.
right at the start here..
He once again skips over 40 years of his life..
He starts with during that long period,.
and then he basically just goes through it..
Remember last week we saw that Moses basically skips.
from when he's a baby in the waters of the Nile.
to when he kills the Egyptian..
He just basically skips over 40 years of his life..
He does it again right here..
It's almost like he's saying, man, Midian was rough..
Midian was tough..
We're just gonna skip over all of that stuff..
So basically now Moses is like 80 years old.
as he's writing this..
And Pharaoh has died..
And you can almost get a sense as Moses is writing this.
that Israel themselves may have risen up in hope..
Pharaoh's died..
Maybe the next person that comes in.
will be more benevolent..
Maybe the next person that comes in might free us..
And before we get carried away.
that when there's a change in authority,.
we think that there's gonna be a change in slavery..
Notice what it says here..
It says the Israelites groaned in their slavery.
and they cried out..
In other words, there was no change here..
The slavery has continued..
And I want you to get your head around this..
Moses has left Egypt..
He's in Midian and he's actually started a new life there..
He's free..
He gets married..
He starts having kids..

$^{201}$This guy's living the life..
He learns to become a shepherd..
All the while for 40 years,.
his people are still suffering..
His people are still locked in slavery..
His people are still bound by oppression and injustice..
And Moses is over here and his people are here..
No doubt Moses skips over those 40 years..
Because he wants us to understand.
that despite this kind of conflict.
between Moses seemingly living a life here.
and his people still in slavery,.
God has not given up on them..
- Right..
- And the people need a promise more than ever before..
It says here that they are groaning and crying out..
These two words, very important,.
could not be more different..
The word for groaning, the Hebrew word,.
it literally means to sigh..
I wonder if any of you have ever done that.
about something in your life..
To groan, it means two things in the Hebrew..
It means something that was personal and private.
and something that was done off the back of great pain..
That classic, (groans).
is exactly what it means here..
And it's something that's personal and private,.
which is really interesting because Moses,.
in describing the slavery of his people.
and the pain that they're under at this time,.
he starts with the individual.
and the personal and the private..
He starts with the reality that slavery.
was actually hurting them personally..
A few years ago, I was playing golf, about 15 years ago,.
and I'm on the 15th tee, and my game's going great..
And I've got my driver in my hand,.
I'm looking down the fairway,.
and I'm thinking this is gonna go long and straight..

$^{241}$If you ever play golf with me,.
you know that that's not how I play golf..
But anyway, just roll with it..
And as I go back like this,.
I do something with the lower back..
I don't know what I did..
I've done that motion many, many times in my life..
Something tweaked in my lower back..
I literally had to go on the ground,.
and I had to lie there..
I couldn't finish the round..
For about two weeks,.
I had the most excruciating pain in my lower back..
And since that day, for 15 years,.
every day I still have a dull pain in my back..
I groan a lot about my pain in my back..
Almost constantly..
You guys probably won't know this,.
but in private, on my own, I'm like,.
I want you to resonate with this..
What is it in your life that you groan about,.
that you find so hard to shake,.
and that you have almost kind of given into.
with that eternal sigh?.
Now, this word is then contrasted.
with this other word, cry out..
Now, you could not get a whole completely different idea.
in the Hebrew than this groaning,.
which was private and individual,.
and ugh, and a cry out,.
which was public and communal and loud..
In fact, that word literally can be translated shriek..
So there's this personal reality.
for the Israelites in their slavery,.
where they're feeling something,.
and it's like, ugh, this burden of this pain..
And yet together, they are coming together,.
and they have not given up crying out for change..
Now, this is really important..
You see this because this is the reality.

$^{281}$for Israel right here..
They are shrieking..
They are crying out..
They're asking God to step in.
and change things and do things..
They are asking..
The word literally means to call out in hope of response..
So they're not just going like, ah, life sucks..
They're actually going, this is not right..
This is not how I want things to be..
This shouldn't be the people of God..
We're crying out, God..
We're asking you, God, to step in, to do something..
We will not give in privately..
We're groaning publicly..
We're joining our voices together.
and saying the world should be a different place..
We shouldn't be under this oppression..
We shouldn't be facing what it is that we're facing..
We will cry out..
Now, this is shocking.
because Israel has been in slavery,.
by this time, for 100 years..
This is not like me when I pray something for God.
and after a month, he doesn't answer and I just give up..
This is 100 years of slavery..
This is probably already four generations of Israelites.
that have come and gone in that time..
And even after 100 years, they have not given up..
Even after 100 years,.
they are still crying out in their slavery,.
asking God to come through..
And the reality for all of us.
is that this should shake us up..
This should change our thinking.
because it's so easy for us to get comfortable in our sin..
And when we get comfortable in our sin,.
we silence the voice that should be crying out for change..
Have you silenced a voice in you.
that should be crying out for change?.

$^{321}$One of the realities is that.
we become comfortable in our sin..
As we become comfortable in our sin,.
we become comfortable in how things are..
And that voice that was once inside of us.
has now come down..
And that urgent desire that we've had before,.
to be holy before God, to walk with him,.
to do as much as we can has gone..
And the reality is this..
Some of us have silenced the cry of our soul for freedom..
Come on, church..
I need you to lean into this a little bit..
'Cause I think this is really important for all of us here..
Some of us have silenced the cry of our soul.
for liberation from our sin..
And Israel are crying, shrieking,.
asking, pounding, believing,.
shouting as loud as they can all together.
for God to step in and do something..
And I believe for some of you in this room,.
you need to rediscover your voice..
You need to rediscover your cry..
You need to draw a line in the sand and say,.
enough is enough..
I'm gonna stand up..
I'm gonna take a stand..
I'm gonna ask God..
I'm not gonna suppress it..
Yes, maybe I haven't seen the freedom..
Yes, maybe it seems like God has been distant..
Yes, maybe the promise hasn't come through..
It feels like God hasn't answered..
He hasn't seen, but I will not silence my voice of hope..
I will not silence my voice of prayer..
I will not give in to the reality of my sin,.
but I will take a stand and say, enough is enough..
I will cry out..
Some of you need to start crying out again..
And we can see in the text here.

$^{361}$that Israel has remembered their God..
The question in the text though.
is has God remembered them?.
Is God present with his people when they cry out?.
Does God draw near to us when we have a voice.
in our soul that says, I want more freedom?.
I want you to see what happens in the text..
Just a very small thing that Moses writes here..
He says, the king has died..
The Israelites groaned in their slave and cried out,.
and their cry for help because of their slavery.
went up to God..
This was not some random cry..
This was not just lots of words in a vacuum..
These words went up to God,.
and this is the first introduction.
to a personal reality of God..
God's been mentioned in the Exodus so far,.
but this is the first time where God is about to step in.
and actually introduce himself, show us who he is..
And like we say, next week we're gonna see.
the personal intimacy of that God.
as he meets Moses in Midian,.
but in this moment, God begins to express himself.
through his heart, through his soul,.
through his character, through his emotions,.
through who he is, and mostly through his promise.
and his promise for his people..
And it's this idea of God's character.
that is now going to define the rest of the Exodus journey..
In fact, from this point forward,.
all the way through the next 38 chapters or so,.
God always refers back to this moment right here..
He always comes back and says, this is who I am..
I am the God of Abraham, Isaac, and Jacob..
I am this, I am that..
You need to know my character..
You need to know my heart,.
because you are a people who are shaped by my character..
And when we're shaped by the character of God,.

$^{401}$God's promises become our promises..
God's word becomes our word..
God's work and mission becomes our mission..
When he says, pray that heaven will come to earth,.
God was looking at the church..
And we align ourselves to the character of God..
That's the thing that enables us.
to journey into freedom from our Exodus..
And we're gonna see a lot of that as we go,.
but today I wanna open up this idea.
of the character of God with you..
I want you to see why God acts on behalf of his people..
What is it inside of him?.
What is it about his compassion.
that drives him to respond on behalf of his people?.
When we were in Egypt, that was a question.
that was going around my mind.
whilst we were filming this series..
I was always asking myself, what was it?.
What was it in God's heart?.
What was it that caused God.
to wanna have this response for his people?.
What makes God fight the battles.
that he fights for his people?.
And as we were there, we had a chance to connect.
with somebody in Egypt..
His name's Akram..
He's an NGO worker..
He runs an organization called Better for Kids..
And he works with orphans in Egypt..
And I wanted to sit down with him.
and find out a little bit more about his heart.
and why he serves and why he does what he does.
on behalf of others,.
hoping that I might encounter something.
about God's heart in what he does..
And so I sat down with him..
We did an interview..
We're gonna play that for you now..
I wanna just encourage you though to lean into this,.

$^{441}$to really open your ears of both your heart.
and your head to hear this..
Akram's English is not the strongest English..
And so you do need to kind of just listen in..
But he's got some incredible nuggets.
for us to understand about the heart of God in compassion..
Let's check this out..
I've come here to the bustling city of Cairo today.
to the local offices of an NGO.
called Better for Kids International.
to meet with one of their staff called Akram.
and find out a little bit more about the work they do..
But even more importantly,.
to discover the central element.
to the start of the Exodus narrative..
Akram. - Hi, Andrew..
- Nice to meet you. - How are you?.
- I'm well, thank you. - Are you okay?.
- Yes, really good. - Have a seat, please..
Thank you so much..
It's such a pleasure to meet you today..
- Thank you. - Yes..
- Thank you, you're welcome..
- Well, thanks so much for spending some time with us today..
Why don't we start by you introducing yourself?.
Tell us a little bit about who you are,.
your family, what you do..
- My name is Akram..
And I have 45 years..
- Mm-hmm..
Since 2015, you've been working for an organization.
called Better for Kids International, I believe..
- Yes..
- Can you tell us a little bit about.
the work you're specifically doing.
with the orphans that you're working with?.
- Yes, since 2015, I worked with a mission..
It's called Better for Kids..
In here in Egypt, a mission, Egyptian mission..
You can say that..

$^{481}$And I'm here in Egypt..
- For orphans..
- For orphans, yes..
It's called Good Samaritan..
Good Samaritan Orphans..
We have 40 kids, boy and girls,.
different age, since two years or one years,.
till married..
- Tell us about where the children are from.
that you're working with, that you're serving..
What sort of background do these children have?.
- Okay, we serve through the church,.
at the poor area..
Not directly with the family, but through the church..
So the church tell us about the poor family and the kids,.
and we go directly to the church.
and serve with the kids before the school..
The age from three years or two years.
till six or seven years..
- Right, right..
- We make an education curriculum..
- Yes..
- A, B, C, D, and alphabet, or something like that..
- Yeah..
- And curriculum for the Bible study..
I love the kids..
All of my life I serve the adults..
I sing with the adults..
I give the training with the adults..
But my heart with the kids..
The wonderful time and the best time.
I spend during the week with my Sunday school,.
with seven or 10 kids in my class..
- And so it's the kids and it's their situation,.
it's their, the background,.
and maybe the poverty they've experienced.
that does something in your heart..
- Yes..
The need..
- The need they have..

$^{521}$- Yes..
I search all the time what the kids need..
So I can see the kids for this situation,.
for this background, for our family,.
for the area that's very poor..
He can find anything easily..
So my heart, the kids' needs,.
need the first for the love..
Love without anything..
I love you, just I love you..
Without anything..
And really, my heart can reach for the kids directly..
So I like to work with the kids.
'cause if the kids don't care for you, it's okay..
- Yeah..
- But if you found really love for your heart,.
he come for you..
If you punish him or if you make anything,.
but still he love you..
- Yeah, yeah..
- Because you love the first.
and you have a really love for the kids..
- So your love really motivates.
your desire to help them as well..
- Yes, yes..
- When you can't help a kid,.
how does that feel for you personally?.
Is it frustrating?.
Are you upset?.
How do you handle the emotions when you can't help a kid?.
- Many time I cry with the kids..
(laughs).
When I pray with them, yes..
When I speak with the kids.
and he explain what's actually happened with him..
- Yeah..
- Because many time I found kids,.
his dad or mom punish him in the hand..
- Hurt him?.
- Something like that, yes..

$^{561}$Something like that..
- Abuse..
- Abuse, yes..
I'm poor, I found some kids,.
her mom because he do something wrong,.
he bring the spoon and put it in the fire..
- Wow..
- And put for the hands..
At this time I didn't make my eyes.
to cry with the kids.
'cause it's a very difficult situation..
- And that's your heart for them..
That's your compassion coming out of you..
- Yes..
- Emotionally..
- Yes, yes, yes, really..
- And that feeling in you,.
I'm assuming that that's what drives your response..
You act because your heart is broken..
- Yes, many time I pray for how can I help these kids?.
I want to bring this boy or girl with me at my home..
If I can't do that, okay, directly I will do that..
What's in here, we can do that..
Just pray for them..
- Pray for them, yeah..
You've mentioned a few times.
that one of the reasons why you do what you do.
is 'cause of your relationship with God..
- First..
- Yeah..
What is it about God that motivates you.
to serve these children?.
- Heart of Jesus for the kids..
You know the situation with Jesus.
when the children come to Jesus and serve Jesus,.
no, no, no, no, no, no, no, no, don't pray Jesus..
The different situation between heart Jesus for the kids.
and heart disciples for the kids..
Jesus look for the kids, it's a human,.
not just the kids..

$^{601}$He's big..
So each time I read this chapter for the Bible.
about the heart of Jesus through the kids,.
I thank God for, I pray each time,.
give me little part for your heart to serve the kids..
- I think what we see in that story.
is Jesus' compassion for those kids..
And so you pray for just a little bit.
of that compassion yourself for the kids..
- Yes, really, really..
- And that compassion has really defined.
your whole ministry, your whole life..
- Really, I do that..
The Bible used the expression as Jesus is very big, angry..
Two time, it's the same expression..
The Bible, the New Testament use it for the Jesus..
The first time, the first time,.
this is the second time with the kids..
The first time when the Jesus come to the temple..
- Temple, yes..
- Yeah, and he say that many people.
pray and say something, yeah..
The same expression, Jesus is very angry..
I can't remember what the expression in English actually,.
but in Arabic, the same words,.
the same letters for the words for the Jesus.
for all of the New Testament..
- Wow..
So the anger Jesus has in the temple,.
in John chapter two, I think..
- Yes, yes..
- It's the same anger he has against the disciples.
for stopping them coming to the kids..
I never knew that in Arabic..
That's beautiful..
- The same words, the same letters, and wow..
- Wow..
- He be angry for the 12 men,.
is very nearest for the Jesus..
- Yeah..

$^{641}$- And at the same, equal from,.
he say that the men pray something at the temple..
- Yeah, wow..
Wow..
So that drives what you do..
- Yes..
- That same kind of,.
like, 'cause I think what you're talking about.
is compassion comes with two elements to it..
It comes with the love..
There's this love that drives compassion..
But also, like in that example you just shared.
on the Bible story, it comes with a certain sense of anger,.
like a righteous anger..
And both the love and the anger can actually form.
the compassion that we have to act in the way that we do..
Well, thank you so much for spending so much time with us.
and telling us about the work you do,.
but also more importantly, perhaps, the heart you have.
to serve the vulnerable children that you're working with..
All the best..
- Thank you..
Thank you..
- I think it's really quite profound.
what Akram opened up for me,.
looking at the Arabic words there for that anger..
And it really is so true that God's compassion.
has this dual element to it, love and justice,.
mercy and the righting of wrongs..
And those two things come together to create.
the holistic way that God is compassionate..
And it's that that we see now.
as we open up this part of the story..
I wanna show you just very briefly here.
some of the ways in which we see this compassion at work..
It says here from verse 24,.
"God heard their groaning,.
"and he remembered his covenant with Abraham,.
"Isaac and with Jacob..
"So God looked on the Israelites.

$^{681}$"and was concerned about them.".
This is just, interestingly,.
this is just 15 words in Hebrew, those two verses..
They're just 15 words in Hebrew..
Four of those words is the word God, Elohim..
God would later show himself as the I am that I am Yahweh,.
who we've become to be known..
In this moment, though, he's just Elohim, God..
And God, I'm gonna draw on this..
Can everybody see this?.
Everybody okay?.
So I'm gonna draw God..
There you go..
You're welcome..
It says here four times God something,.
God something, God something, God something..
So first of all, it's God hears..
Then it says God remembers..
Then it says God sees..
And finally, it says God knows..
This is what the Hebrew basically tells us.
to help us to understand the heart of God for his people..
He is a God who hears, a God who remembers,.
a God who sees, and a God who knows..
It says, first of all, that he hears their groaning..
Now, this is really important..
The word in Hebrew here means basically.
not just to listen, not just to hear something,.
but to actually act on what is being heard..
In other words, in the Hebrew,.
to understand how God hears,.
it's not just to think that God has kind of like.
heard my prayer, but it's actually that God.
is gonna respond and act on what he's heard..
Give you an example..
It's like my daughter Mia..
Now, I say to my daughter Mia every morning,.
make your bed..
Pretty much every day I come home after work.
and the bed is not made..

$^{721}$And I know that she heard me, but she didn't hear me..
Are you with me?.
Did you get that, right?.
'Cause if she had really heard me,.
she would have made the bed, right?.
Okay, that's the same sort of thing here..
God is hearing in the sense that.
he doesn't just receive some sound,.
but he's actually going to respond..
He's actually going to act..
And it's interesting..
Here's what he's heard..
Not the crying, shouting, loud stuff,.
as important as that was..
He hears their groaning..
In other words, what he hears is their personal,.
oh man, this is hard..
You need to know here that that's your God..
That sure, he comes and fights for the injustices.
of this world..
Sure, he comes and does great, amazing things..
He parts seas, he's a fire on top of a mountain..
He's all those things..
But you need to know first and foremost,.
he's a personal, intimate God for you..
He hears your groans, those private, personal things.
that no one else hears or even knows..
Those internal, oh man, if only I could be free..
Those are the things that he hears..
And then it says that he remembers..
Now this word for remember is zakar in the Hebrew..
It doesn't mean that God has forgotten..
I know when we hear this in the English,.
it's very easy for us to think.
that remembering means forgetting..
It's not saying that God has forgotten his people.
and that in their crying out, he's gone like,.
oh yeah, they're in slavery..
Totally forgot..
It's not meaning that..

$^{761}$What it actually means is that when God remembers,.
the Hebrew word basically means to commit to a promise,.
to commit to something that you have promised to do..
So where hearing comes with an action,.
the action is this remembering of the covenant,.
the covenant with Abraham, Isaac, and Jacob..
God remembers means that he is going to stay committed.
to the promise that he's made..
Give an example in Chris and I's life..
We got married 25 years ago..
And when we got married, I said some vows..
We said some vows to each other..
And part of our vows to each other was.
we would not sleep with anybody else..
We would not have an affair..
We would stay committed to each other..
And that's the way things are gonna be..
We made those vows to one another..
We made those vows to God..
I, after 25 years, have remembered those vows,.
not because I forget them here and there,.
but because I've remained faithful to her for 25 years..
Do you see that?.
That's remembering the vows..
That's me standing and saying,.
I made a commitment to this woman.
and I'm gonna stand by with her and alongside of her..
And she's done the same for me, I think..
Yes, you have..
Good..
She's done the same..
(audience laughing).
For me..
So she has remembered the covenant..
I've remembered the covenant we've made..
And by remembering, what that means is,.
not that we've forgotten, it's that we're living it out..
We're committing to that priority..
We're committing to that promise..
When God says he remembers his covenant with Israel,.

$^{801}$he's saying, I haven't given up..
I haven't stepped away from that..
I'm still committed to that promise..
Genesis 12 is the heart of that promise.
where God shows up to Abraham and he says,.
I'm gonna take you and I'm gonna bless you..
I'm gonna make your name great..
I'm gonna make the nation great..
And all nations are gonna be blessed through you..
God comes in the Exodus and says,.
I haven't given up on that..
I am still that person..
I'm still that God..
That's still my promise..
And I'm working that promise through even to this day..
So when you think I'm distant, I'm not..
When you think I haven't answered, I'm still at work..
I'm a God who remembers..
I'm committed to you..
And I will prove my faithfulness and my commitment to you..
And he does this by seeing them..
He then looks upon them..
That word is the word that we've looked at.
over the last couple of weeks..
It's the word Rahah..
It's the same word that Pharaoh's daughter,.
when she saw Moses in the reeds, she Rahad him..
It means to see with emotion..
It's this idea of compassion..
It's this idea of something changing inside of our hearts.
and that change causing us to wanna be different,.
causing us to act in a different way..
This is the same thing that Moses had.
when he saw his own people under slavery and he Rahad..
He saw them with emotion, this compassion..
This is what Akram was talking about in the film..
And it's so important that you see before God does anything.
that he hears, he remembers, and then he has this compassion..
He sees the people for who they truly are.
and he sees them with this idea of justice and love,.

$^{841}$this idea of mercy and righting wrong..
And because he sees them, he's going to go.
and do all the things that he does in the Exodus..
And so often, God will refer back to this.
in the chapters ahead..
I see you, I see you, I see you..
And what's key to grab ahold of here.
is something that I think is very important.
'cause we forget this a lot..
You see, God does not act against the evil.
or the sin of Egypt..
He primarily acts on behalf of the freedom of his people..
Now, it's really important you get this..
See, compassion, not judgment, is the fuel to Exodus..
Now, did the sin need to be judged?.
Does our sin need to be judged.
and dealt with with the Lord?.
Of course it does, but you need to understand.
that the starting point is not an angry God.
who has wrath in his heart against his sinful people.
and he wants to judge them..
The starting part is he looks upon us, he sees us,.
he has compassion with us, he has grace and he has mercy.
and he has love..
And just like Akram sees those kids.
and he sees all the things happening in their lives,.
but he's first motivated by love,.
a love that he's found in God..
So God sees, which means compassion.
is always the driving force of why God comes.
and judges sin..
It's because he wants to see us free..
It's because he loves us..
And there is mercy and justice..
There is love and the righting of wrongs..
And those two things come together..
And the heart of it is all this idea of love, compassion..
Some of you, some of us, 'cause I do this too,.
some of us, we stay in our sin.
because we're afraid of the anger of God against us..

$^{881}$And we don't deal with it and we kind of live in it.
and we think God's angry at us, God's upset at us,.
God must be so angry at me..
Oh man, he knows all the brokenness in my life.
and he can't like me..
God's not gonna do anything for me..
Now the danger of that is it becomes a cycle.
'cause then we think we have to try to earn God's love..
We have to try to make him happy with us,.
try to appease him..
All of that goes out the window when we realize.
that it is compassion that turns the heart of God to us..
Not his anger, not his wrath, not all of that..
It's his compassion..
He looks on us with love and in his love,.
we are able to love 'cause he first loved us..
He sees you..
And that should set some of you free here.
to actually deal with some of the sin.
that you need to be dealing with..
Knowing that God is not some wrathful, angry guy.
who wants to smite you..
He's a father who runs in the field towards you,.
puts a robe on you and a ring on your finger.
and sandals on your feet and he says,.
"You're not gonna be one of my slaves,.
"you're my son, my daughter, my child.
"and we're gonna do this together.".
So he sees and then finally it says that he knows..
I love this..
He knows..
The Hebrew word literally means.
that he understands everything about everything..
It's this biblical sense of intimacy.
and knowing something fully and completely..
And I want you to get this..
God knows everything about you.
and he has not turned his back on you..
God knows all of your sin and brokenness.
and it does not offend him..

$^{921}$Sin hurts him, sin upsets him,.
sin he needs to deal with absolutely..
But he knows the intimate details of your life.
that no one else knows..
The stuff that you would never have the courage.
to share with anybody, he knows that.
and he still loves you..
So this security comes in knowing a God.
who knows us completely and still accepts us.
and embraces us..
He hears us, he remembers his covenant,.
he sees with compassion, he leads with compassion.
and he knows us completely..
And because of all of that, that is the God of Exodus..
That's the God who reveals himself to his people..
That's the God who says, "Before I do anything,.
"you need to know who I am.".
And if I was to kind of summarize it.
in ways in which you can easily remember, here it is..
He is basically declaring that he is close..
He hears..
He's not distant, he hasn't given up on you..
He's close to you..
What he's also saying is that he's committed to you..
He's committed to you..
He's walking with you..
He's right here, right now, fighting battles for you..
He has not given up on you..
He remembers his covenant..
He's committed..
He's compassionate over everything.
that's happening in your life,.
all the stuff that's going on,.
all the things that you thought he would be angry about.
or upset about, he loves you.
and in his love, he has mercy and justice..
He wants to be with you..
He is compassionate..
And then finally, of course, he is concerned.
because he knows you and he's concerned.

$^{961}$about the brokenness and the sin.
and what that does to you.
and his concern drives him to want to respond..
He is close, he is committed, he is compassionate,.
he is concerned..
That's the God of Exodus..
And that needs to be our framework that we have.
as we step towards the burning bush next week..
This is how God introduces himself.
to all of his people in their slavery to sin..
And this, my friends, if you remember this,.
if this becomes how you see God,.
you will actually have the courage to cry out.
and see the coming of a promise in your life..
May that fill you with joy as you take the serious steps.
to no longer ignore the realities.
of what is happening in your life.
and stand before a God who hears, remembers, sees, and knows.
and allow him to set you free..
Can I pray for you?.
When you're with me, stand with me and I'm gonna pray..
Father, we are standing before you today as your people.
in the same way that Israel stood before you.
and groaned and cried out..
And Lord, for those of us in this room.
who have silenced our voice,.
so those of us in this room who have silenced.
our cry of our soul for freedom,.
Father, I pray that you would come by your spirit.
in this moment and just begin to rebirth and refresh.
and renew that desire within us..
Lord, you hear our groaning,.
those private, intimate,.
where we are expressing the depth of our pain,.
brokenness, you hear and you know..
God, you have not given up on us..
Every person here, you remember your covenant.
that you have with them, the new covenant.
that is in the blood of Christ Jesus,.
saved, redeemed, sanctified, being made new in him,.

$^{1001}$a new creation as Paul would talk about it..
God looks at you and he remembers the blood.
that was shed on the cross of Calvary.
and he sees you through that blood..
He sees you and he loves you..
God knows everything about you.
and because he knows everything about you,.
there is nothing you need to hide from him..
The enemy will want you to feel shame,.
but Adam and Eve were naked and unashamed..
And I pray that as you find your cry again,.
I pray that as you recognize God is hearing you,.
remembers you and sees you,.
I pray you will find yourself unashamed before him..
You will find yourself with a cry in your heart.
that says, do whatever it takes, Lord..
That you would come and realize.
that those promises do stand,.
that he is faithful and he is at work..
And I pray that this would be a God..
This would be your God..
This would be the one that you would worship..
Now, whenever the enemy would come.
to try to tell you differently,.
God is like this, no, God is not like that,.
just like he did to Adam and Eve in the garden,.
you would say, no, no, I stand on what the scriptures say..
I stand on a God who hears, remembers, sees and knows..
That's my God..
Father, I pray that that would set us free..
And we thank you for this, in Jesus' name..
Amen..
[MUSIC PLAYING].
\newpage



\section{}
\label{sec:j3pdS8tSSoA}
\textbf{2023-05-22 EXODUS - 06 The Burning Bush [j3pdS8tSSoA].mp3}
\newline
\newline
連結: \href{https://youtube.com/watch?v=j3pdS8tSSoA}{\texttt{ https://youtube.com/watch?v=j3pdS8tSSoA}} ~~~~ 語音日期: 2023-05-22 
\newline
\newline
\hyperref[sec:1GHdbHuySwQ]{\small{< < < PREV SERMON < < <}}
~
\hyperref[sec:index]{\small{[返主目錄]}}
~
\hyperref[sec:GwiSx82TY4o]{\small{> > > NEXT SERMON > > >}}
\newline
\newline
$^{1}$Everyone says, "Amen." Can we thank our worship team as always? Wonderful. Why don't you have a seat,.
turn around, say hi to someone, say, "Hey, who are you? Nice to meet you here. My name is...".
Whatever your name is. I'm so glad you're here. We are so glad you're here. My name is Andrew..
I'm one of the pastors here at the Vine, and if you're joining us online right now, we're so glad.
that you're here. Welcome to the Vine. If you're joining us in this room, we're packed in this room..
If you're joining us in the overflow outside, we are also so grateful that you're with us in the overflow..
Man, what a joy it is to gather together, and something that we should never take for granted,.
the beauty and the reality of gathering together. In the autumn of 1996, in the middle of a pretty.
ordinary week, and at the end of a very mundane day, something happened that completely changed.
the trajectory of my life. I had just graduated from university from the UK, and I was back in.
Hong Kong, and I was young, I was single, I was ready to take on the world. I'd just been given a job..
I just had my first job. I was hired by a local recruitment company. I remember getting my first.
paycheck and thinking I was a millionaire, and then I realized that I wasn't because I was living at.
home with my parents because I couldn't afford to live anywhere else. My parents lived in this small.
little makeshift apartment in a local NGO here called Mother's Choice, an NGO that looks after.
orphans in our city, and that helps girls that are in crisis and the pregnancies of crisis under age..
And we were brought into this moment where we were living with them. My father was working at.
Mother's Choice, and they gave us this little makeshift apartment in one of their buildings,.
and it was a really interesting place to live. It was like living with 100 orphans around you.
all the time. It was always a pretty exciting thing. And my apartment had unusually two front.
doors to it. There was the main front door that when you entered into that front door, you would.
enter into the living room of our apartment, but there was a smaller side door that was also like.
a front door, and if you were to enter into that one, you would go into a smaller room, and then.
there'd be some stairs that would go up to my bedroom, and then to the end of that room, there.
was a corridor that sneaked around the outskirts of the apartment and then came back in at the.
place of the kitchen. It was kind of bizarre that it was set up that way, but it just was. And so.
every day after work, I would come home and I would go through the smaller second door. I would.
dump my bag, whatever, in my room. I would then go on that little corridor on the outside of our.
apartment all the way around, and then I would enter into the kitchen. I would open the fridge..
I would grab a beer, and then I would wonder what is it that I was going to eat that night..
Well, on this particular day, I'd had a very hard day at work. It was one of those things,.
you know, when you're like the most junior person in the company, you get like all the bad work,.
right? You get all the dumped work on you. So I had this terrible day. I was doing all these.
spreadsheets. It was super boring. I didn't get done all the things I needed to do. I realized.
I had to take the spreadsheets home with me. I was exhausted at the end of the day. I actually got.
home later than normal. I went into that little side door, went up to my bedroom, dumped my bag,.
got into that corridor, walked around the outside of our apartment, into the kitchen..
I opened the fridge. I grabbed a beer, and I was just about to open it when I heard a sound..
It was a sound that I had never heard before, like that particular sound..
It was the sound of a woman's voice, but it was a woman's voice that I did not recognize..

$^{41}$And I realized it was coming from my living room, that my parents had a guest over,.
which was unexpected, and they had not told me about. And the thing that grabbed my attention.
straight away was that this voice, I mean, if you could have heard this voice, I had never heard.
anything more beautiful in my life. It was like it floated to me on clouds to my ears..
Not only was I drawn by the sound, but I was also drawn in curiosity. The thing about the sound and.
the voice was, although it was beautiful, I couldn't understand a word that this person was.
saying, because they had a really heavy accent. They were from a country that I could not place,.
and they were speaking English, but I could not understand. And I was so drawn to this..
And I remember holding this beer in my hand. I hadn't opened at that point. I was literally.
frozen still, and I had a decision to make. Something extraordinary had just entered into my.
life. And I remember turning really slowly and walking very quietly to the edge of our living.
room and pouring around the door so I could see into the room. What I saw was the most beautiful.
girl that has ever been created. She was around about my age. She was blonde. She was beautiful..
And she was from New Zealand. Yes. I had no idea what she was talking about,.
but she looked just like this. And literally, I'm not kidding. I have not done anything yet..
I step into the living room holding my beer can, mesmerized by this woman. And as I step in,.
I hear the audible voice of God, "Andrew, remove your shoes, for the ground you're standing on.
is holy ground." Are you with me, church? And I walked in and said hello to the woman who had.
become my wife. Yes. Something extraordinary had broken into my very ordinary and completely.
changed my life from that point. And as we continue our story in Exodus, we're at a moment.
where we meet Moses in his most mundane and his most ordinary. A man who for 40 years has been.
living in a desert called Midian. He's been looking after the sheep of his father-in-law,.
who has fled from his status and who he was in Egypt, and was now desperately trying to search.
for the person that he was going to become in his life. And in his very mundane and his very.
ordinary, God shows up with the extraordinary. God breaks into his mundane and invites him.
to make a choice. If you will, it was Moses' beer-in-the-hand-fridge moment. Is he going.
to respond to this thing that is suddenly, rudely, if you will, interrupted his life?.
Or would he choose to just carry on in the same old, same old as things have always been?.
And as you're gathered in this moment, whether in this room or online, I believe that the Holy.
Spirit wants to bring you to that place this morning. If you've been here at the Vine over.
the last five weeks, we've been journeying in the story of Exodus, and today you come to a place of.
decision. Today, God is going to invade your ordinary with his extraordinary, and you will.
have a choice just like Moses did. Will you decide to respond, or will you decide to remain in the.
comfort of what you have always known? You see, when the Lord wants to bring transformation in.
our lives, it is always to come and change something that's going to require courage.
and faith. And if you're going to continue in your Exodus journey, this is the moment.
where God is going to call upon you for courage and for faith. I want to read this to you because.
this is how Moses, who's writing Exodus, writes of his own story and narrative. This is Exodus.
chapter 3, verse 1. "Now Moses was tending the flock of Jethro, his father-in-law, the priest.
of Midian, and he led the flock to the far side of the desert and came to Horeb, the mountain of God.".

$^{81}$Let's just stop there. The first thing he says is he's trying to give you a sense of the geography.
of this moment. Before the extraordinary enters into the ordinary, he gives you a description of.
his ordinary. He's a shepherd looking after sheep. He's so humble, he's so poor, that he doesn't own.
the sheep himself. They're the sheep of his father-in-law, Jethro. He's gone from being a prince.
of Egypt in a palace to not even be able to afford his own animals. And there he says, "I am in a.
place," notice this, he says, "Horeb, the mountain of God." He gives us a glimpse right at the start.
of the story of where everything in Exodus is about to go. Because although he doesn't explain.
it here, we come to learn later that Mount Horeb is the same mountain as Mount Sinai, where God will.
meet Israel and give Moses the Ten Commandments. It's the same place. And it's likely that Moses.
had come to this mountain many times as he's grazing the sheep of the shepherd that he's.
looking after. And he's probably circled around this mountain many times. It's his ordinary, but.
God's about to step in and do something extraordinary. And he goes on to tell us right here,.
"I now call this the mountain of God. We as the people understand this as the mountain of God,.
but before I brought you out of slavery to that mountain, God met me on that mountain first.".
And he said this Horeb, Horeb meant glowing or heat. It gives you a little sense of what's about.
to happen next in the narrative. But Mount Sinai, that it comes to be called Sinai, sounds in the.
Hebrew like the word hatred. And that's how the Israelites felt as they came out of their slavery.
in Egypt. And Egypt was still pursuing them. They felt like they were a hated group of people. And.
God met them in that place and began to restore their identity. But before the identity of Israel.
was restored, Moses had to first have his identity restored. So God comes and meets him on Mount.
Horeb. Now, I want to give you just really quickly a sense of the geography of where this is all.
happening. In week one, I showed you this modern day map of Egypt. Obviously, it doesn't contain.
the whole of Egypt. It carries on below the bottom of the screen there. But this is the important.
area. This is an area in the Bible is called Goshen. It's roughly where the Israelites were.
camped when they were enslaved by Egypt. This, of course, is what's known as the Sinai Peninsula..
Modern day Israel comes all the way down here. But in those days, this land here was not Canaan..
It wasn't the modern day Israel. That was more up here. And then, of course, you've got Saudi.
Arabia here, modern day Jordan here. Now, the mountain of Horeb is really debated by scholars.
as to exactly its location. There are really two main schools of thought. You remember in week two.
of this series, I told you that there's very little archaeological evidence in Egypt that.
actually Israel was ever in the land, let alone that the Exodus actually happened. And because.
there's a lack of archaeological evidence, people have debated exactly where all of this stuff took.
place. Now, two schools of thought, one that's a bit of a more modern understanding and one that's.
more traditional. In the modern understanding, the idea is that Mount Sinai is actually over here.
in modern day Saudi Arabia. And when Moses fled, he would have fled across the Sinai Peninsula..
He would have crossed over the Red Sea here. And then he would have found his new family, Jethro.
and the shepherd and everything over here in Midian. Poole, actually in the New Testament,.
refers to Mount Sinai as being in Arabia. That's the word in the New Testament that we have..
And so scholars believe that this is here. Now, interestingly, there's quite a lot of archaeological.
evidence for some of the things that are explained in the Exodus in this area. And Mount Sinai,.

$^{121}$roughly here, is actually a mountain in Saudi Arabia that is burnt at the top of it. Nobody.
knows how that mountain got burnt. And so, again, people believe that this may actually be.
the place. Now, if that is the case, Israel would have come out here. They would have come down.
along here. They would have crossed the Red Sea, either at the bottom here or somewhere in the.
middle here. And then they would gather at Mount Sinai here. And then later on, after receiving.
the law, they would have come up here. This is where they would send the spies into the promised.
land. They come back with a bad report. So they wander for 40 years in the desert. And then.
eventually they get released to go into the promised land here. That's a modern understanding.
of what may have happened. However, there's a traditional view that has been accepted by.
scholarship over many years. That one places Sinai right here in the heart of Egypt, in the Sinai.
Peninsula. In fact, if you were to search on Google Mount Sinai, it's likely that you'll be.
drawn to this place right here. There's a monastery right at the foothills here. That monastery is.
called St. Catherine's. It was founded in the fourth century. So Christians, right at the early.
beginning of our faith, some 1700 years ago, decided and believed that this was actually where.
Mount Sinai is located. So that is the place in Egypt where you can go to this day. So in this way,.
Moses would have come down. Midian would have been in this area. And then Mount Horeb, Sinai here..
When the Israelites were released from their slavery, they would have come down here. They.
would have crossed the Red Sea here. And then they would have met God here. And then again,.
they would have come up here, sent the promised spies in, done their 40 years of wandering around.
here, and then finally up and in at the end of the Exodus. Helpful? Now, which one do I believe?.
Well, I actually think that it's more likely that Sinai is probably in Saudi Arabia than it actually.
is in Egypt. And there's a bunch of different reasons for that, which I don't have time to.
get into today. But we, for this series, we have filmed this particular route for you..
And we are basing this series out of Mount Sinai, right here in the Egyptian Peninsula. Why? Because.
it's a lot easier to film here than it is to film here. Okay? And also, the whole point of this.
series is not to give you like a documentary of exactly where Israel went. The point of this series.
is to help you to understand the events of the Exodus. And those events can be told beautifully.
by following this here and by visiting Mount Sinai right here. And it's that event that I want to.
bring you to now. Is that helpful? All right. In verse 2 and 3, it says this, "There the angel of.
the Lord appeared to him, Moses, in flames of fire within a bush. Moses saw that though the bush was.
on fire, it did not burn up. So Moses thought, 'I'm going to go over and see this strange sight.
why the bush does not burn up.'" This is the moment where God brings an extraordinary interruption.
into Moses' very ordinary and mundane life. God's not finished with Moses. There's much for Moses.
to do. Moses is totally unaware at this point of what is about to be ahead of him. And God brings.
a disruption into his life through this incredible sight of a burning bush. Now, God has revealed.
himself to humanity many times before he now reveals himself to Moses. And he's revealed himself.
quite directly to humanity before, for example, with the call of Abraham. So the first question.
we should have is, why does he do it in a random burning bush? Well, it's fascinating because fire.
in Egyptian mythology meant two things. It meant something that destroys in order to purify..
Moses, having been raised in Egyptian mythology, having been taught for years in the court of.

$^{161}$Pharaoh, would have known and understood deeply what the idea and the symbol of fire was all about..
In Egyptian gods, they believe that if they burnt something, they burnt it up completely..
They destroyed it in order for it to be a sacrifice to the gods. So Moses, going about his.
normal, ordinary, everyday, mundane life, he looks over and he sees a bush on fire. That in and of.
itself is not a big deal, but he describes something to us here. He says there was a.
bush that was on fire, but it did not burn up. That's a change for Moses in his mythology,.
in his spirituality, and how he understands the symbol of fire. This fire is burning a bush,.
but it's not burning up the bush. It's not consuming the bush, and it's that that causes.
him to stop in his tracks. This is his beer fridge moment. Like me that day, I heard a sound. It was.
a woman's voice, but it was something in that voice that was different for me. There's something in.
this fire that's different, and he names it for us. This fire should have consumed it, but it didn't..
What Moses is pointing out is something very important at this point in the narrative..
He's pointing out the beauty and the power of God's grace. He's saying our God is different..
Our God can burn something and not consume it. Our God can bring his presence to something,.
and it's not completely wiped out. It's not completely destroyed. We don't have to be.
completely destroyed to be purified by God. There's a new narrative that's beginning to.
work its way into scripture here, and he says this is an incredible sight. It's burning,.
but it does not burn up. This would become a symbol for Israel of themselves. The rabbinic.
teachings often speak of Israel as a burning bush. If there's a nation that has suffered more than.
Israel over the years, I challenge you to name one. Israel has suffered so much, and yet they are still.
sustained. That is the grace of God. They are a burning bush. In the same way, Jesus on the cross.
is a burning bush. He took on the sin of all of humanity, all of our sin on his shoulders,.
and the wages of sin is death, and the sin should have wiped him out forever, and yet God in his.
sustaining grace reaches in, redeems and renews, resurrects Christ three days later. The cross.
is a burning bush. The church today is a burning bush. You are a burning bush,.
because all of our sin, all of our brokenness should have consumed us,.
and yet by God's grace we are here, and yet by God's grace his church is still growing,.
and yet by God's grace even over two and a half thousand years, no matter how much persecution,.
no matter what society might say, no matter how this thing might be tried to be contained,.
it is growing. It is thriving. The church is alive in the city of Hong Kong by the grace of God..
Are you with me? Here's something you need to know right at this point of the narrative,.
because this is so important. It is only ever God who is able to sustain us through the things in.
life that seek to destroy us. Only God, and it is by his grace that we are saved. Paul would write.
to the church, and he would say this, "I am persecuted, but I'm not overwhelmed. I am knocked.
down, but I will rise up again." He says, "There is a sustaining grace on me." And some of you in.
this room, you need to understand that you're facing some hard time. There's something going on,.
but if you're a child of God, you will not burn up. The hand of his grace is on you, and as a.
burning bush, it doesn't mean it's going to be easy, but God will sustain us and renew us. Amen?.
Now this burning bush, really interestingly for Moses, he sees all this, and then he says.
something. He says, "It's a strange sight. It's a strange sight, because I am not expecting it to.

$^{201}$be this way." And he decides, it says here, to go over to the bush. This is fascinating. He actually.
decides to go over to actually see the bush. Now the word in Hebrew here actually suggests this,.
to change his course, to set a new direction. That's what the word in Hebrew means. So I want.
you to see what's happening here, what Moses himself is trying to explain to you. He's walking.
a normal path. He's doing what he always does in his ordinary, and God sets a burning bush over in.
the distance somewhere. Now think about that for a moment. If God had really wanted to get Moses'.
attention, why didn't he put a burning bush right in front of his current path? Why didn't he put.
it right there in front of him, so that he couldn't do anything, but actually engage with God? Instead,.
God sets a burning bush over in the distance somewhere, and essentially goes, "Will you come?".
"Will you see? Will you respond?" The burning bush is an invitation to respond..
It is God asking Moses whether he will take responsibility for realizing that he needs healing..
It is a beautiful, extraordinary gift of God to say to Moses, "Would you be willing to change your.
path? Would you allow yourself to be interrupted? Are you willing to step out of the ordinary,.
so that you can come and encounter my extraordinary?" And Moses himself says, "I'm going to go over..
I'm going to change my path to understand what is happening here." And the reality for so many of us.
as Christians, and I'm like this too, is that God is burning bushes around us all the time, and we.
just keep walking on the path that we're on. And God is showing up and welcoming us, inviting us.
to a deeper place of healing. He's showing up and welcoming us to deal with some of the sin in our.
lives, but we have to make the choice to turn, the choice to place ourselves in a position where.
we can commune with him, engage with him. We have to see the burning bush and then go over to it..
And yet fear keeps us locked on the path that we always know. There's risk and danger of the.
unknown when we come off of our normal, everyday path and head towards that bush..
It takes courage to decide that you're going to change your path. It takes faith to believe that.
whatever there, despite how scary it might seem, might actually be the thing that could truly set.
you free. And out of this whole part of the scripture, what God is, I think, asking Moses,.
and I think what he's asking us, is something deeply profound. Will you have the courage,.
the courage to embrace uncertainty, to go over and discover new paths,.
or will you choose to submit yourself to your unknown comfort and carry on just like it always.
has been? Will you agree? Will you take the risk, the courage, to actually break your path and head.
over towards what God is doing over here and discover new things with him, or will you stay.
submitted to your current comfort, just follow on, carry on like it's always been? At this point in.
our Exodus journey, that's the decision that's before you. You can just carry on as things have.
always been, and God will continue to pursue you with burning bushes, but very rarely are they.
placed right in front of you. More often than not, they're placed around you so that you take the.
first step towards him. Are you with me? Now I want you to see what happens when Moses does that..
It's really quite beautiful. Verse 4, "When I looked and saw that he had gone over to look,.
when the Lord saw that he had gone over to look, God called to him from within the bush, 'Moses,.
Moses!'" You know you're in trouble when God says your name twice. This happens with me and my wife..
If she says Andrew once, I'm okay. Andrew twice, not good. "And Moses said, 'Here I am.' Do not.
come any closer, God said. Take off your sandals, for the place where you're standing is holy.

$^{241}$ground." What an incredibly profound and beautiful moment. God has set up an extraordinary bush.
in such a place where Moses has to decide to turn and go towards it, but as he goes towards it,.
God then stops him. But I want you to notice something. God only calls to him after he sees.
that Moses is coming towards him. Come on church. God only calls to him once he sees that Moses.
has made that decision, has said, "I'm going to give this a go. I'm going to go over and investigate..
I want to know more about this bush that's burning and doesn't burn up." And it's in that step of.
faith that Moses takes that God begins to speak, "Hey, Moses, Moses, here I am," Moses says. And he.
says these most famous words in the Old Testament, "Take off your sandals for the ground that you're.
standing on. It's holy ground." And the question we need to wrestle with is, why does he have to do this?.
Why does Moses need to remove those sandals? What is it about this moment? What is it about this.
extraordinary encounter at a bush? What is it about God's presence that is inviting Moses to.
remove his sandals? What is it about the holiness of God in this part of the story that is so critical.
for everything that's about to happen next? Well, to help you to understand that, we took some time.
on our trip around Egypt to actually go to St. Catherine's Monastery that's at the foothills.
of Mount Sinai. And when we were there, the monks introduced us to something that truly is profound..
And I want to now introduce you to it. It's the actual burning bush. Let's take a look..
Right in the heart of the Sinai Peninsula, a triangular wedge of some 24,000 square miles in.
Egypt that forms a bridge between Asia and Africa, you will find St. Catherine's Monastery..
Built between 548 and 565, this Greek Orthodox monastery sits at the foot of Mount Sinai,.
the traditional site where Christians believe God first appeared to Moses and where the Ten.
Commandments were given. The monastery is named after Catherine of Alexandria, and tradition has.
it that after being martyred in the early 4th century at the mere age of 18, her remains were.
carried by angels and laid in the mountains of this area. The monastery itself is the oldest.
continuously inhabited Christian monastery in the world, with a history that can be traced back over.
17 centuries. As we will see, it is the home of the oldest continually operated library in the world,.
to an order of monks that has never been broken since the beginning, and to a chapel that has.
never fully been destroyed in all of its history. And it is also home to something else unique,.
unparalleled, and also present since the very beginning of the Exodus story itself,.
something which at first glance you might not think is very special at all..
This is traditionally believed to be Moses's burning bush, the actual bush. Now the bush.
itself is a type of bramble known as Rubes Sanctus, and it's found across Asia and parts of Europe..
And it's famous for its ability to survive and thrive for thousands of years in the harshest.
environments possible. And what fascinates me about this bush is that it's actually not found.
in any other part of Egypt, except right here at the foothills of Mount Sinai. Now tradition has it.
that when the first monks appeared here to actually establish the monastery, they were so drawn to.
this bush that they created the monastery around it in order to preserve it in its natural environment..
So think about this for a sec. This bush is incredibly rare. It's also able to survive.
for thousands of years, and scientists have dated this exact bush to the time of Moses,.
all of which has convinced the monks here that this really is the burning bush..
And if that is true, then the ground I'm standing on right now, well, it's holy ground..

$^{281}$The story in our scriptures is familiar to us. Moses is tending his sheep in the desert when he.
encounters a burning bush and is amazed that it was burning and not yet consumed. And so he walks.
over to it in order to investigate. As he approaches, God calls out to him from the bush.
and commands him to remove his sandals as the ground that he was standing on was holy ground..
The reason for the need for Moses to remove his sandals has been much discussed in Christian.
history. The traditional view accepted by the monks here revolves around the idea that sandals.
in those days were made of dead animal skin and therefore filled with impurities. And because.
nothing impure or unholy could ever go into God's presence, Moses had to do his part in removing.
those impurities before he could meet with the holy spirit. And so he had to remove his sandals.
before he could meet with the holiness of God. Monks, priests and pastors alike have preached.
this message ever since and use this moment of the removing of Moses's sandals to speak.
metaphorically of all of our need to ensure that we do our best to remove our own impurities as we.
seek to live out a life of holiness to God. Metaphor has a rich history in Christian tradition.
and this is seen perhaps no more so than right here in the museum at St. Catherine's, just a.
stone throw away from the burning bush. Inside are some beautiful pieces of Greek Orthodox art.
and not surprisingly a number of them are about the burning bush. Let me take you inside..
[Music].
This piece in particular fascinates me, particularly as we are thinking about.
metaphor in the tradition of the burning bush. This is a picture of Mary, the mother of Jesus,.
holding Jesus in her arms. But what's really interesting is that she's covered in flames..
Now this is an important metaphor in the Greek Orthodox tradition. They think of Mary as like.
a burning bush, thinking about how the holy spirit came upon Mary as a virgin, enabling her to.
conceive Jesus. And the spirit came upon her but didn't consume her, didn't overwhelm her. Well,.
in that way she is like a burning bush. And for those in the Greek Orthodox church, they approach.
Mary like they might approach a burning bush, thinking that in drawing close to her,.
they actually will draw closer to the holiness of God..
So whether through metaphor or reality, the story of Moses at the burning bush reminds us to reflect.
deeply on the holiness of God and the sobering reality of our own impurities in comparison..
And that's an important lesson. But it's actually not the only lesson we should be drawing from this.
narrative. In fact, there's another reason why I think God called Moses to remove his sandals on.
that day. And it's a reason that opens up for us a whole brand new way of thinking about scripture..
And to understand that, I want to actually open up the breadth of scripture to you now..
Beautiful monastery, right?.
You can visit it. They have accommodation there. It's amazing. I can highly recommend it..
But I want to show you now what I actually think is happening right here in this burning bush.
moment and why God asked Moses to remove his sandals. And I think it has something to do.
with impurities, but it has something to do with something much more important. I've been saying to.
you every week of this series that actually when Moses presents the story of Exodus, he's doing it.
in the backdrop of Genesis, another book that he wrote to explain God's creation of all things..
Now, we see at the end point of chapter 2 of Genesis, Moses described Adam and Eve in a very.

$^{321}$specific way. He says that they are naked and unashamed. And this is absolutely an unashamed..
This is absolutely a beautiful statement, not just about Adam and Eve, but of all of humanity..
And the idea of being naked and unashamed is not actually just speaking about their physicality,.
but the Hebrew words actually speak much bigger than that, much broader than that. It speaks about.
their mental, emotional, physical, spiritual, their relationship with God as well as their.
relationship with one another. The word naked in particular is actually a Hebrew word, which is.
this very important word, aramin. Now, aramin literally means this. It means to be laid bare..
It means to be open. And importantly, nothing hidden. This is the idea that this particular....
I'm going to run out of space again. I did that in the last service as well. It's really annoying..
Anyway, there you go. To be laid bare, to be open, nothing hidden. This is the idea of what it is to.
be aramin. So it's not just talking about their physical stature with each other, though that was.
true, but it's also talking about this deeper meaning that humanity has this nature that's.
been created by God to be aramin, to be open, to be fully laid bare. No hiddenness, no secrets,.
no sin. Sin has not entered the picture yet. And so the way in which we've been created is to be.
naked and unashamed with one another. Yes, but most importantly with God. This is why you see.
the picture in Genesis 1 and 2 of God walking in the garden in the coolness of the night and.
Abba Neve communing with God in a way that was fully laid bare, fully open, nothing to hide..
Are you following so far? Now, the word aramin is the last Hebrew word of chapter 2 in Genesis..
The first Hebrew word in chapter 3 is another word, and it is the word that literally means.
crafty, crafty or shrewd, depending on your English translation. This is the word that is.
actually used to describe Satan. So the snake in the garden was more crafty or shrewd than any of.
the other animals in the garden. That word crafty and shrewd is the Hebrew word aram, which those.
of you who are quick on your feet will realize is very similar to the word aramin. Are you with me?.
Now, aram actually means this. It means to be closed. It means to be hidden,.
and it means to be secretive. This is the nature that Moses chooses to describe.
Satan. And all of the evil and brokenness that happens has the nature of aram. So I want you to.
see what's happening in this beautiful poetic moment of Genesis 2 and 3. The last word aramin,.
the first word aram. Aramin, statement about who humanity is. Aram, statement about who the enemy.
is, what happens through the enemy's nature. And what do we see happen when Adam and Eve take of.
that fruit of the tree of the knowledge of Geneva that they weren't supposed to eat from? Two things.
happen straight away. First of all, they cover themselves because they realize that they're now.
naked. And when they were naked and unashamed, now they're naked and ashamed. And so they cover.
themselves. Then the second thing they do when God shows up in the garden is they hide from God..
They're secretive from Him. They pull back from Him. They're afraid of Him. Whereas before they.
were naked and unashamed with God because they were fully open and laid bare, sin has now made.
them aram. They are now closed. They are now hidden. They are now secretive. And all of us.
who struggle with sin, all of us who carry around the brokenness that we have in us because of sin,.
we are constantly battling between these two natures. And the enemy wants us to be secretive.
and hidden. One of the greatest works of the enemy is to convince us that actually the very thing.
that we've been created for the most is the kind of relationship now that we least are comfortable.

$^{361}$in. We don't want to expose ourselves. We don't want to be honest with people. We want to hold.
back. We want to be closed. We want to be hidden. We've become so much more aram when we should have.
been aramin. Are you following this, Somi? Now, why is this all important? Moses, for the last two.
chapters of Exodus, has been trying to explain himself in the backdrop of Genesis 1 and 2. We've.
looked at a number of those examples over the last few weeks. But here it's critical because.
Moses in many ways has lived out how we are as humans. He has struggled deeply with his identity.
and who he truly is. And in trying to live out of his identity, he does something wrong in murdering.
the Egyptian. And what does he do when he murders the Egyptian? He covers him up. He hides him. He.
buries him, thinking he can get away with a secret. But that secret is revealed. He's filled with fear..
Pharaoh wants to kill him. He runs away to Midian. When he gets into Midian, he's hiding once again.
there. He doesn't tell anybody about the fact that he used to be groomed to become Pharaoh. He doesn't.
tell anybody about his Egyptian past. He even doesn't tell anybody about his Hebrew origins.
either. He hides from all of that. He's embarrassed. He's ashamed. He's broken. He's running for his life..
He's in fear. He becomes more and more Aram than how he was created to be in Aram. Tracking with this?.
And then God invades his ordinary with his extraordinary. And God places a burning bush.
just off of here to cause him to make a choice to turn and enter towards and walk towards that.
burning bush. And as he goes towards that burning bush, God says, "Okay, great. I see you. You're.
responding. You're drawing near. Now stop, because I need you to remove your sandals, because the.
ground you're standing on is holy ground. And does that mean I need to remove my impurities?" Yes,.
it does. But it means so much more than that when you see it in the backdrop of what Moses.
has already explained in Genesis 1 and 2. When God calls him to remove his shoes, what God is.
really doing is saying, "I want to commune with you again like it was back in the garden. I want to.
now meet with humanity in a way that it was before. In other words, I want the flesh, the naked flesh.
of your foot, Moses, to stand on my spirit and commune with me." Are you with me, church? The.
burning bush is God's beautiful invitation to humanity to say, "I want the purest form of you,.
the most naked and vulnerable, the way you've been created to be, where you and I are in a.
relationship where you have nothing to hide, where you don't need to cover yourself with false.
identities anymore. You don't need to cover yourself from the shame of your sin. I'm going.
to meet you in such a way where we get to commune face to face," which is eventually what he will do.
with Moses. In this moment, it's flesh to spirit. It's saying, "A rum is not who you are. And if.
you're willing to remove your shoes, you can rediscover who you are." Isn't that so beautiful?.
"Would you come," God is saying, "and would you commune with me again? Can we be naked and.
unashamed together? And can we from there rediscover what it truly means to be human?".
This is the starting point of everything that changes for Moses. And in his obedience to go.
and have a look, in his obedience to remove his sandals, he stands onto the ground,.
and he says this beautiful thing. He says to God, "Here I am." The Hebrew words mean,.
"I am in complete submission and in love. Here I am before you.".
And we don't take any step further in Exodus ourselves until we're willing to come naked and.
unashamed before God and say, "Yeah, there's brokenness. There's sin. There's stuff that.
I'm ashamed about. But I have a God who sustains by His grace. I have a God who cools from the.

$^{401}$burning bushes around my life. A God who pursues me and has not given up on me. And a God, when I.
turn to Him, lays out the beautiful invitation to be able to deal with my shame through His embrace.".
And His love and His holiness and His goodness. I'm not holy. He's holy..
And in His welcome to become adamim, as I take off my sandals and step to Him,.
we can commune once again. It's really important you understand something. The burning bush was.
this incredible attraction for Moses. It was the thing God used to attract him out of his.
normal life and draw him towards it. The attraction was important, but it's really.
important that you see there was no power in the bush itself. God uses the bush to disrupt.
the everyday of Moses, but the bush itself wasn't powerful. What was powerful was the presence of God.
that was coming from the bush. Does that make sense to you? And the reality is that the Vine.
Church, and you need to hear this, the Vine Church is a burning bush. And every single person who's.
here and online right now, you're here because something's attracted you to be here. You've been.
drawn to something here in this community that's attracted you to be a part of this community..
Maybe it's community. Maybe it's the friendships you've got here. Maybe that's what attracts you.
to come to our ministries on a Sunday. Maybe it's our worship. You love our worship and you love the.
musical worship and it's something for you and you enjoy it and you're attracted and you come for.
that. Maybe it's the preaching or the teaching that's drawn you in, that's attracted you as well..
Whatever it might be, you need to understand this. There's no power in the thing that attracts..
And if all you ever do is come to the Vine because you're attracted by something,.
but you never actually push beyond the thing that is attracting you to actually meet the God behind.
the thing that's attracting you, there'll be no power or transformation in your life..
A church cannot change you. I'm going to say that again. A church cannot change you..
Great music cannot change you. The only thing that will change and transform you and really.
start your journey of exodus is the presence of Jesus Christ and his invitation to you to remove.
your sandals and step on holy ground with him. That's what will change you..
And because there's that invitation, your invitation is not to stand and watch..
It is to remove your sandals and commune..
And as I close, everybody's favorite words in every sermon..
As I close, I want to invite you to reflect.
on the burning bush that is around you at the moment,.
and the beauty that there is behind a God who welcomes you into this moment..
And I want you just to quieten your heart. And this is what this whole message has been about,.
is this moment right here. I want you just to stable your heart, bring it before him,.
and just give you a chance to soak in some imagery and some words that we've created,.
that I'm hoping will help to just minister to you much of what God's word is for you here.
this morning. We're going to watch this together, and then after that, I'm going to come back,.
and we're going to pray together. Thanks, Kingsley, for that. Let's watch this..
It starts in the fading of light..
In the separation of presence and purpose, in the distance between holiness and shame,.
where we find ourselves humbled and hurting, and where the sacred has become the profane..

$^{441}$We tremble in this hurt and shame and profanity, and stumble between the shards of light..
For we desire to wander in what is hidden and distant, and wonder if in the darkness we could.
ever know the light. And then, when least expected, a spark, and what was ordinary comes alight with.
curiosity, and we are drawn forwards, tentative and tender, bruised from being buried, but burnt.
with new hope. And just there, right before us, crackling with longing and intent, the divine.
invades the common, and the sacred subverts the profane. And a voice flickers into our brokenness.
like a flame dancing in the night, and we are welcomed into a new story that begins with a.
simple embrace. And when we see it, we can no longer look away, for there is now glory amongst.
the graceless, and the shards of light have become rivers of gold, and we are alive again,.
naked and unashamed, nothing now covering the rawness of our humanity from the restoring.
presence of an intimate God. So let us remove, let us replace, let us repent, and let us recreate..
May we see the bushes on fire around us, and together step on holy ground..
I wonder if you'd stand with me..
I wonder whether you'd be comfortable just to open your hands as I pray..
Father, I thank you for whatever has attracted the people in this room to the vine today..
I thank you for the bushes that burn..
But Father, we want to meet you..
We want to remove our shoes and our sandals..
And Lord, we want our naked flesh to meet with your spirit..
I wonder if you'd be comfortable just as your hands are open, just maybe picture yourself.
just even doing that. Maybe reflecting on some of the imagery you just saw in that poem..
God invites you to commune with him..
For Moses, this was the starting point of his healing..
And it begins with that simple embrace, and that desire to be connected..
Some of you in this room, God's been placing burning bushes around you..
And yet you're just carrying on in the same old, same old..
And I pray today you would like Moses go over.
and meet with God again..
Some of you here,.
there's a fear in you in the removing of your sandals..
You're not sure what God might say to you..
You're not sure whether he'll demand something of you..
Perhaps you're afraid to let go of something that you've been holding on for long..
Know that his grace is here, and his love is so deep for you..
And he just wants to be with you..
He wants to bring you back to the garden, which only happens through the blood of Jesus Christ..
And the resurrection of Jesus, sin overcome, we made righteous through Christ's righteousness..
So for some of you today, it's the righteousness of Christ..
The blood of Jesus Christ..
The blood of Christ that the Lord wants you to embrace..

$^{481}$And some of you are wracked with shame..
There's no condemnation in Christ Jesus..
The enemy has placed shame upon you..
And in this moment, the spirit of God wants to come and just minister to you..
And just set you free from shame through his intimate presence..
Whatever it is for you, just take this time..
Try to quiet down the distractions..
Try to keep your heart centered with him..
Allow yourself to meet with him now..
[BLANK AUDIO].
\newpage



\section{}
\label{sec:GwiSx82TY4o}
\textbf{2023-05-29 EXODUS - 07 The God of Freedom [GwiSx82TY4o].mp3}
\newline
\newline
連結: \href{https://youtube.com/watch?v=GwiSx82TY4o}{\texttt{ https://youtube.com/watch?v=GwiSx82TY4o}} ~~~~ 語音日期: 2023-05-29 
\newline
\newline
\hyperref[sec:j3pdS8tSSoA]{\small{< < < PREV SERMON < < <}}
~
\hyperref[sec:index]{\small{[返主目錄]}}
~
\hyperref[sec:_gagT3D9Two]{\small{> > > NEXT SERMON > > >}}
\newline
\newline
$^{1}$in Jesus' name, everyone says, amen, amen, amen..
(congregation applauding).
You can have a seat..
So God has just met Moses at a burning bush..
God has appeared out of nowhere.
and into the ordinary everyday moment of Moses' life,.
God comes and creates this burning bush.
just off in the distance to his right,.
or maybe it was to his left,.
but just off there in the distance..
And God does it there and not right in front of him.
because God wants to call him to leave his everyday,.
to leave his ordinary and to go and see what it is.
that's burning just over there on the horizon..
And Moses has to make a choice..
He has to make a choice to decide.
to draw towards the extraordinary..
And I wanna encourage you to be open.
to making that choice in your life.
because there are burning bushes.
that God places around us all the time..
And maybe for us, they're not like some flame in a bush,.
but maybe it's somebody that we meet on the street..
Maybe it's a new colleague at work..
Maybe it's a moment in a conversation with someone..
Maybe we're at a bar or a nightclub or a restaurant.
or somewhere where we didn't even think God was present..
And there is a burning bush before us..
I wanna empower you as people here in Hong Kong.
to walk boldly towards the burning bushes,.
to recognize that God often breaks into our ordinary.
with his extraordinary..
And when he does, it's chance and time for us to turn.
and walk and listen and hear..
And as Moses goes over, God says these words to him,.
"Don't come any closer, Moses,.
"for the ground that you're standing on is holy ground.".
And he invites him to remove his sandals..
And Moses does so..
And we saw last week that that had something to do.

$^{41}$with the impurities that Moses had on his shoes.
and the holiness of God,.
but it actually had far more to do.
with God's desire to be an intimate community.
with Moses again,.
with God's desire to break through all the things.
that had happened in Moses's life.
and meet Moses tenderly, intimately in communion with him,.
taking Moses back to Genesis one and two,.
where Adam and Eve were able to be naked and unashamed.
in God's presence before the onset of sin..
And despite all the sin that had been in Moses's life.
and was continuing to be in his life,.
God says, "Remove your sandals.
"'cause I want you to be naked and unashamed with me again,.
"that nothing you have done, nothing you have said,.
"nothing you have lived would ever take you..
"Nothing is an obstacle.
"from my intimate relationship with you..
"I love you..
"I'm here for you..
"I wanna connect with you..
"I want your flesh and my spirit to be together.".
And God invites Moses into something.
that he invites us all into,.
to know that nothing we do can ever hold us back.
from the grace of God..
And that God wants to intimately commune with us too..
He wants you to be naked and unashamed..
And Moses's response to this incredible call of God.
for him to be naked and unashamed before him again.
is simple words..
He says, "Here I am.".
Not kind of like, "Here I am.".
But, "Here I am..
"All of me..
"The good, the bad, the ugly..
"The me that was for 40 years an oppressor in Egypt..
"The me that murdered an Egyptian.
"and thought he could bury it and get away with it..

$^{81}$"The me that fled because I was in fear of my life..
"Me that came to a new place.
"and pretended that I was somebody that I wasn't..
"All of me, I now step into you..
"I now bring to you.".
And it's in this amazing moment of intimacy.
that God ministers to Moses..
And it's here at this moment,.
when God finally calls Moses.
to what the true purpose of his life was..
And I need you to hear this..
Moses was 80 years old..
Adrian, I need you to hear this..
Come on, baby..
Moses was 80 years old at this point in his life,.
just like Adrian was standing here before you right now..
And God said, "Now is the time for your purpose..
"Now is the time.".
Everything that's happened in the last 80 years.
has shaped and formed you.
to be the one that I can now use, yes..
But now is the time.
that you're about to live out the calling in your life..
And there are some of you in this room.
who are older than I am,.
who think that you're at the end,.
when actually you're at the beginning..
You're actually at a moment now with all the wisdom.
and all the life experience.
and all the gifting and anointing.
that has been placed upon your life..
Now God's rolling up his sleeves.
and he's saying, "Are you ready?.
"Because your purpose is still here.".
And so for this 80-year-old man.
who's standing naked before God.
and for the first time in 80 years, unashamed,.
God gives him a call..
I wanna read this to you from Exodus 3, verse 7..
The Lord said,.

$^{121}$"I have indeed seen the misery of my people in Egypt..
"I've heard them crying out because of their slave drives.
"and I'm concerned about their suffering..
"So I've come down to rescue them.
"from the hand of the Egyptians,.
"to bring them up out of that land.
"into a good and spacious land,.
"a land flowing with milk and honey,.
"the home of the Canaanites, the Hittites,.
"the Amorites, the Perizzites, the Hivites,.
"the Jebusites, every ite you could ever imagine is there.".
I love this, God shows up and Moses is removed his sandals.
and he's communing with God,.
the naked and unashamed moment..
And then God begins to speak to him..
And the first thing he does is he reminds Moses of who he is..
He reminds him of his heart of compassion..
And we saw this just a couple of weeks ago.
at the end of chapter two, when God declares his character..
And he says back in chapter two,.
"I have heard, I have remembered, I have seen, and I know.".
And because of this, if you remember,.
I drew that out on the whiteboard two weeks ago..
And because of this, God is saying, this is who I am..
This is my character..
I'm a God who hears..
I'm a God who remembers..
I'm a God who sees..
I'm a God who knows..
I'm for you, he says..
And then right here, as he begins to minister to Moses,.
and as he begins to give Moses his call on his life,.
he reminds him of all of this..
Notice what it says here in chapter three, verse seven..
He says, "I've indeed seen the misery of my people..
"I've heard them crying out.
"because they're a slave drivers..
"I'm concerned about their suffering,.
"so I know about their suffering..
"And now I have come down to rescue them..

$^{161}$"I'm gonna be faithful to my covenant.
"of deliverance for my people.".
He lays it all out there again,.
because he wants Moses to know.
that everything God is doing.
is out of a place of compassion..
And although God is gonna do more of this later.
in the journey of Exodus, at this moment,.
he wants Moses to be reminded.
that he's not like the Egyptian gods..
The Egyptian gods who are a pantheon of gods.
who have a hierarchy of power,.
where the Egyptian gods enslave other gods,.
which therefore makes no sense.
why then humans would enslave people.
if their gods enslave people..
God shows up and says, "I don't enslave people..
"I actually am turned towards people..
"My heart is compassion for people..
"And I hear their cry, and I'm going to act.".
And Moses, you can almost sense in this part of the story,.
it's probably like, "Yeah, great, let's go.".
He's thinking God's gonna go do something..
He has no idea God's about to call him.
to the hardest thing..
He actually says here, he says, "I'm gonna now send you..
"And I'm gonna go and take the Egyptians.
"and to bring them up out of that land.
"into a new and spacious land.".
This is the first mention of this idea of the promised land.
that God is gonna do for his people..
And he mentions two things about this promised land..
He says, "First of all, it'll be a land.
"flowing with milk and honey.".
This is a phrase that will get repeated a lot.
throughout both Exodus and Deuteronomy and Numbers..
The idea that the promised land.
is not actually literally flowing with milk and honey..
What it is, is it's a place of fruitfulness,.
a place of great blessing, a place of abundance..

$^{201}$It's God saying, "I'm gonna take you.
"out of a place of slavery.
"and put you into a place of abundance.".
You need to know that as you come out of the slavery.
to your sin, your destination is abundance..
Not abundance financially per se,.
not abundance maybe even in the things.
that you might want abundance in,.
but the abundance in how God sees fruitfulness.
in this world..
The abundance of the kingdom of God at work in your life.
like it never been before..
Abundance in God's presence with you.
like you've perhaps never felt before..
There's always a place of abundance.
after a place of slavery..
Come on, church..
And so as he's journeying his people here,.
he's saying this land is gonna flow with milk and honey..
It's like Genesis 1 and 2..
It's like a new garden for you.
where you're gonna be able to live out.
what it means to truly be human,.
where you're gonna be able to go into the earth.
and subdue it and see it flourish.
and see it become all that it can be..
That's ahead of you..
But he says something really important..
He says, "This is the home of the,".
and then he says, "Canaanites, Hittites, Amorites,.
"Parasites," and they're all there again..
In other words, he's saying, "I have a promise for you..
"It's a land that is incredibly fruitful,.
"but right now it's occupied by other people..
"It's the home of all these other people,".
which is really fascinating.
'cause God is saying, "I've got a promise for you,.
"but there's a part that you're gonna have to play.
"in receiving of that promise..
"There's a whole bunch of people there that you're gonna,".

$^{241}$now this is right at the beginning of the Exodus,.
and God's gonna talk about this a lot more.
as they go, but even at the start,.
God's not giving empty promises..
He's not saying, "Hey, it's a great land..
"Don't worry, it's gonna be fine..
"I'll tell you about the rest later.".
Right up front, he's saying, "No, this is a land,.
"but it's the home of these people right now,".
which means it's a promise that is given to you,.
but it's a promise that's gonna need conquering..
And it's really important that you understand.
how the promises of God work in our lives..
It's very rarely that the promises of God.
are given to us on a silver platter..
In fact, the promises of God are actually,.
I think, and over my experience of a number of years.
of serving Christ, I think the promises of God.
are the primary way by which God disciples us as his people..
Because think of it this way..
So often, when God wants to give us something,.
something that is so good to us,.
when he wants to do something that's good for us,.
he will demand something good from us..
And this is because God sees his promises.
that work in our lives, and he wants to partner with us..
You know, in the promises of God,.
we are not silent passengers..
We are active partners..
Now, be careful here, because that doesn't mean.
that your strength and your great wisdom and your ability.
will bring about the promises of God in your life..
That's not what this is talking about..
What it's talking about, though,.
is that you're gonna have to align yourself.
to the promises that God has spoken over you..
And that's gonna require three things,.
obedience, trust, and faith..
And it wasn't gonna be Moses's strength.
and Moses's wisdom and all the things.

$^{281}$that Moses is great at that would get Israel.
out of Egypt and into the promised land..
And it wasn't gonna be all those things.
that Joshua would have that would enable them.
to conquer the promised land..
What would enable that was obedience, trust, and faith..
And some of you in this room, you're struggling with this,.
because you're asking for the promises of God,.
but you're not willing to play your part in those promises..
And every promise that he will speak over your life.
will always call you to a deeper level of discipleship..
It'll always call you to walk in obedience and faith.
and trust in the promise that God has spoken over you..
And God is doing this for Moses, 'cause he's saying,.
"I'm gonna do all this stuff, but Moses,.
"there's a role for you to play here.".
And God is inviting Moses to be willing to play his part..
Will you have the obedience and trust and faith, Moses,.
to walk with me?.
And I feel like the Holy Spirit is saying to us.
here in Hong Kong in this hour,.
will you, the Church of Hong Kong,.
have the obedience, trust, and faith.
to see the 90\% of this city who don't know me come to faith?.
So I think he has a promise over Hong Kong..
Maybe I'm the only one here.
that thinks God has a promise over Hong Kong..
Anyone else think that God maybe has.
a promise over Hong Kong, right?.
(congregation applauding).
But the church is sorely mistaken.
if we think God is gonna deliver.
that promise on a silver platter..
We need to rise up in obedience..
We need to rise up in trust..
We need to rise up in faith.
if we really wanna see our city changed for Jesus Christ..
Now, how does Moses reply to this?.
Notice what he says here..
He says, it says verse nine, he says,.

$^{321}$"Now the cry of the Israelites has reached me,.
"and I've seen the way that the Egyptians.
"are oppressing them, so now go..
"I am sending you to Pharaoh to bring my people,.
"the Israelites, out of Egypt..
"But Moses said to God, 'Who am I,.
"'that I should go to Pharaoh.
"'and bring the Israelites out of Egypt?'".
Moses' honest words, "Who am I?".
This portrays two things about Moses..
The first is he's come a long way..
'Cause 40 years ago, he was like, look at me..
40 years ago, he was the prince of Egypt..
40 years ago, he was clothed in power and prestige.
so much so that he thought he could kill an Egyptian.
and get away with it..
Now, 40 years later, he is so stripped down.
of all his identity, all his privilege, all of his power,.
that when God shows up and says,.
"I'm gonna take you back to Egypt,".
he's like, "You should've done that 40 years ago.".
Can you hear Moses' heart here?.
Like, I'm nothing now..
Why didn't you come 40 years ago and ask me to do it then?.
'Cause then I had power, then I had prestige,.
then I had influence, now I'm nothing..
You're 40 years late, God..
I wonder if anyone in this room feels like.
the promises of God are a bit late in their life..
Like if only, God, you had called me 10 years ago.
when I had that great job and all that money..
Come on, church..
So interesting that we put the timeframes of God.
in our hands and God's like, "No, no, no, no, no, Moses..
"You don't realize now is the time.
"because you saying, 'Who am I?'.
"is the qualification for me to about to use you.".
'Cause notice the, "Who am I?" though..
There's another thing that's going on here..
The, "Who am I?" is Moses still thinking it's all about him..

$^{361}$Whereas before in Egypt, it was, "Who am I?.
"I'm a power, I have authority, I have prestige.".
Now he's like, "Who am I?.
"I have nothing, I'm a shepherd with nothing at all.".
But he still thinks that the ability for him.
to live out the call of God in his life.
is still down to him..
He's just gone from, "Yeah, I've got all the power,".
to, "No, I've got no power," but he's still focused on this..
"Who am I?.
"How can I do this?".
Basically, Moses is saying, "I am not adequate, God,.
"to be able to do the call that you've placed on my life..
"I'm not adequate, I don't have the skills,.
"I don't have the ability,.
"I certainly don't have the power and prestige.
"that I used to have..
"I am not adequate, God,.
"for what it is that you're calling for me to do.".
Now, here's the reality..
I think so many of us in this room,.
that's so often been our conversation with God..
When God calls us to things,.
it's so easy for us to say, "I'm not adequate..
"I don't have the skills to do that.".
I remember when God called me.
to be senior pastor here at the Vine..
My great cry was, "I'm inadequate..
"I don't have the adequacy to live out this call.".
And you can ask my wife,.
"I still feel that way 10 years later.".
It's a miracle that you come every Sunday..
Literally, that's the grace of God over my life,.
that you guys still show up and come here every Sunday..
'Cause if you understood how inadequate I feel.
to do what I do, you may not come every Sunday..
The grace of God is that way..
I wanna say something really important,.
and I'm gonna put it very bluntly..
When it comes to the call of God on our lives,.

$^{401}$our adequacy has nothing to do with God's sufficiency..
Come on, church..
Has nothing to do with God's sufficiency..
See, God is sufficient,.
regardless of whether we feel we're adequate or not..
God's not asking Moses whether he's adequate..
When God calls you to something,.
he's not looking to see if you're adequate for it..
He's not looking at your resume and going,.
"Oh, this guy's qualified.".
In fact, I think God chooses people.
because they're not adequate..
God calls people because they're not strong enough..
They're not capable enough..
They don't have the greatest influence..
They've not got a great ministry..
God raises up those that no one else in the world.
would look at because he looks upon them and he says,.
"Now this person can be used.".
See, so often in the call of God on our lives,.
God calls us to something that we are not able.
to accomplish in our own strength..
He calls us to something where right now.
our current resources cannot achieve..
He calls us, in other words, church, to faith..
Because if you are adequate.
to live out the call of God in your life,.
then you don't need God for his call..
And this is one of the first things.
that Moses has to grapple with..
It's this idea that God is calling him and he is inadequate.
and that's part of the story..
Because notice what God says next..
God says this..
He says, verse 12, God said, "I will be with you..
"And this will be a sign to you.
"that it is I that has sent you..
"When you have brought the people out of Egypt,.
"you," plural, all of you,.
"will worship God on this mountain.".

$^{441}$I love God's reply to Moses' cry of inadequacy..
Who am I?.
God goes, "It doesn't matter who you are..
"What matters is who I am..
"And what matters is that I'm gonna be with you..
"And if I've chosen to be with you,.
"who can be against you?".
Some of you, that's a word in season in this room right now,.
watching online right now..
You're not adequate, but that's okay.
'cause God is more than sufficient and God is with you..
And when God is with us, who can be against us?.
And God says, "There'll be a sign to you.
"that I was with you,.
"a sign that you can take to the bank.
"that I journeyed with you..
"And the sign is this..
"When you gather around this mountain,.
"you will worship me.".
I love this idea..
God is basically saying,.
"You wanna know what the fruit of Exodus is?.
"You wanna know what the fruit of freedom is?.
"It's worship..
"When my people are set free,.
"they can't help but worship me.".
And he says, "This is a sign.
"that despite what might be happening in your life,.
"you will take it as truth,.
"as a joy, as a privilege to worship me,.
"that your circumstances will not define.
"whether you praise or not..
"When I have delivered you, you will worship.".
There's a church in the center of Cairo.
that dates back to the fourth century..
It's one of the oldest churches that there is in Egypt..
And this church, despite an incredibly hostile environment.
since the fourth century,.
continues to gather to worship today..
Because they have understood.

$^{481}$that what God has done through the Exodus.
has purchased a heavy price.
for them to have the freedom to declare the praises of God,.
even in a very strongly Islamic culture..
And their ability to worship,.
I find incredibly inspiring and deeply challenging..
And I want you to experience a little bit of that..
I want you to feel a little bit of that today..
And so I wanna take you to this particular church now..
I wanna show you a little bit more of them..
They're a very different church to us..
They have icons and rituals..
They're a Greek Orthodox church,.
very different to how we might worship here..
And yet in their worship,.
we can see something that perhaps might help.
to facilitate ours..
Let's go to Egypt..
(wind blowing).
We've come today to the busy, bustling center of Cairo City.
to actually explore perhaps what is the most central part.
of the Exodus story itself..
Behind me right here is the Hanging Church of Cairo..
It's a fourth century Coptic Orthodox church,.
making this perhaps the oldest worshiping community.
in Egypt today..
This church truly has a remarkable story.
and I can't wait to show you around..
The Hanging Church is named for its location.
above the gatehouse of an ancient Babylon fortress.
where its nave is actually suspended without trusts.
over the passage of the fortress,.
giving the structure the impression.
that it's hanging in the air..
The church itself is approached by 29 steps,.
leading early travelers to Cairo.
to dub it the Staircase Church..
The entrance from the street is through iron gates.
under a pointed stone arch..
The 19th century facade with twin bell towers.

$^{521}$is then seen beyond a narrow courtyard.
decorated with modern biblical art..
Up the steps and through the entrance is the main chapel,.
the entrance of which leads to an outer porch.
that dates back to the 11th century..
But as beautiful as the structure of the church is,.
its real power rests in the stories that are found within it..
One such story is found right here.
along the main wall of the church itself..
This inscription here is in Arabic.
so that anybody walking here could read it..
And it's actually a summary of that moment.
where Jesus meets the woman at the well.
and tells her that there is a water she could drink.
that will never run out..
And here's the great thing..
The church, right below that inscription,.
has created two taps..
Now, they're boarded up here,.
but in centuries before, these two taps.
would provide sanitized running water.
to the poor people of Cairo..
This was the church's social justice outreach..
And the people would come here,.
they would take the clean, sanitized water,.
and at the same time read about a God.
who can provide water that would never run out..
I mean, this is the gospel at work..
This is the word and works side by side..
One of the fascinating things about the Coptic church.
is their deep connection to Christianity's Jewish roots..
In fact, a great example of that is found.
right here at the entrance to this particular church..
I want you to see something really cool here..
You'll notice that the Star of David.
is worked into the wood of the door.
right here at the entrance..
And this has significant theological meaning..
What the Coptics are saying is,.
before you even go inside the church to worship Jesus,.

$^{561}$you have to first enter through the doorway.
of the Old Testament..
And when you think about us now understanding.
what Christ has done in liberating us.
out of the slavery of our sin.
and bringing us into the freedom of our promised land.
in grace and forgiveness,.
well, we only can really understand the beauty of that.
when we link it to the original Exodus story.
in the Old Testament itself..
Like the Coptics, so also for us..
We understand Jesus through the doorway.
of the Old Testament..
The storytelling doesn't stop at the doorway..
Right inside the main entrance.
is a beautiful, spacious courtyard.
leading to the steps that take worshippers.
to the main chapel..
The courtyard is covered in modern biblical art.
telling stories of various Old Testament.
and New Testament events,.
specifically as they relate to Egypt..
You know, just being here and being surrounded.
by all these mosaics and this beautiful iconography,.
I mean, you really get the sense.
of how important storytelling and remembering is.
to the Coptic practice of worship..
And to find out a little bit more about that,.
I thought it would be great to sit down.
with one of the priests here,.
actually the priest that leads the morning prayers,.
and talk with them a little bit more.
about how central all of this beautiful art is.
to the way they practice their faith..
Father Samuel, we're so grateful.
that you would host us in your beautiful church.
and show us a little bit about the Coptic worship..
Maybe you could tell me a little bit.
about some of the icons that are here..
Yes, it was a pleasure..

$^{601}$Thank you..
This church is the name of the Virgin Mary.
and Saint Demian..
It's called the Hanging Church..
Why we call it the Hanging Church?.
Because it was built above the two fortress,.
Roman fortress..
The ancient people put palm tree above the fortress.
and built the church above the fortress.
without any foundation..
It used palm tree and the stones.
as a pavilion to this church..
Because it's called the Hanging Church..
As we see, the shape of the roof is made by wood.
because the heavy in the engineering style..
Right, right, right, right..
This church was built as a basilican style..
It's famous for columns and arches..
And as we see, it's built as Noah's Ark..
As Noah's Ark?.
Yes..
Oh, I didn't, I can see that with the way the wood,.
the roof is on the wall..
Yes, and the windows..
As we see, if you see the column,.
we can found eight columns..
Okay..
Which represent the eight persons who was in Noah's Ark..
Noah, his wife, his three sons, and his wife..
Yes..
Oh, wow, okay..
And why is Noah's Ark so important?.
Yes, because the Noah's Ark is symbol of salvation..
And the church is a symbol of,.
is a salvation for people in the church..
And I know for the Coptic church,.
the emphasis of salvation is very important..
Yes..
Are there other elements here that show us that?.
Yes..

$^{641}$This icon says, symbol of the heaven..
We can found Jesus Christ in the middle,.
and the Virgin Mary, and St. John's Baptism,.
and Archangel Gabriel, and Archangel Michael,.
and St. Peter, St. Paul..
Wow..
Yes..
And I understand Paul is very central,.
and Peter is also very central to the Coptic church..
Yes..
Right, right, right, right..
Okay, very interesting..
This icon says, made of three items,.
ivory, ebony, and ebonos..
Okay..
Collected together without any glue, as a bezel..
Oh, wow..
Yes, we can see..
And those three elements,.
do they represent like a trinity?.
Yes, yes..
Okay, maybe show me again..
As you see, this is ivory..
This is ebony, and cedars..
Oh, okay..
Yes, it is collected together as a puzzle..
Gathered together as a puzzle..
As a puzzle?.
Yes..
Oh, it moves..
Oh, no way!.
We can see the ancient, this icon says,.
come back to the 11th century..
Right..
We can see the worker, how made the shape,.
is in ivory, without any machines,.
without any laser cut..
Handmade..
Yes, and the collected ivory was ebony.
with cedars together..

$^{681}$Now, I know that there is a famous staircase here..
Yeah..
Can you explain to me the staircase?.
Maybe take me there and show me..
Okay, yes, okay..
We can found here, it's two doors..
Oh, the doors..
I didn't notice the doors before..
Yes..
Okay, okay..
This is doors, when it's open,.
we found the two way to escape from this church..
It's used in the ancient time..
Okay..
Because if anyone come to the church to kill people,.
you have two way..
You have stay in the church to be a martyrs,.
like our martyrs here..
Yes..
Or to escape from this way,.
to go a new generation in Christianity..
Of course, both is important..
Yes..
So people would actually, they would be worshipping here..
Yes..
Maybe if some danger comes in,.
the doors would open and they would go and escape that way..
Yes..
Wow..
From here, you can see..
Yeah..
You can..
Yeah..
Amazing..
Yes, we have two escape way..
This is way to the people,.
and this escape way to a small room..
Okay..
Yes, from here..
Oh, wow..

$^{721}$From the stairs to a small room..
If we didn't finish the service,.
the priest and one of the deacons.
take communion and go in this church.
and close the door and continue the liturgy,.
continue the service..
Let me understand this,.
'cause I think this is very beautiful..
So if people are having a service here,.
and somebody comes in in the old days,.
and it's dangerous, they would flee for their lives,.
but you would then go and still finish the service..
Yes, yes..
That's how important the worship is..
Yes..
To keep the worship going..
Yes..
Wow, that is a powerful story..
Thank you so much..
I mean, this for me is just so powerful.
to see the commitment to worship in Egypt.
for so many centuries..
Yes, it's just an amazing thing..
So why is all of this important?.
I mean, why have we come to this church today.
and understood its iconography and art.
and had examples of the worship here?.
Well, when God first appears to Moses.
at the burning bush in Exodus chapter three,.
he tells Moses the reason for the Exodus in the first place..
He tells him why he's gonna send them back to Pharaoh.
to demand for the people to be let go.
and why he wants them to move from slavery to freedom..
And it's for one simple reason,.
and that is to worship..
God desires to deliver his people.
so they would be free to worship him..
And that's why this church matters.
because all the people that gather here today.
in central Cairo are proof.

$^{761}$that the fruit of the Exodus is still ongoing..
It's proof that people can still gather.
to be free to worship Jesus today..
And that should be an encouragement to all of us..
When you think about you and your freedom of worship,.
you should never take that freedom for granted..
People have sacrificed and paid the price.
for thousands of years..
And God is not just the God of the original Exodus..
He is also the God of your Exodus too..
And the challenge for you is to ask yourself this,.
are you truly free to worship?.
And are you doing it in your everyday life?.
Doing that will honor God.
and will honor the Exodus itself..
(congregation applauding).
This question of whether we're truly free to worship,.
I think is a question that is quite relevant.
for our church here in Hong Kong in this time..
Perhaps it's a relevant question for us.
as we think about our future here in this city..
And the story of the Hanging Church inspires me deeply.
'cause I want you to see the focus.
that they have at that church..
Their focus is not on the political system.
and the political situation that they're in,.
although that's a very tricky situation in Egypt..
Their focus isn't on the cultural realities.
of being a minority Christian community.
in a majority Islamic culture..
Their focus is not even on whether somebody.
might come into that building with a bomb.
and try to kill people in their services..
Their focus is on God and worshiping Him..
So much so that if somebody does come in to bomb them,.
the priest will not flee with everybody,.
but will take some icon and some communion,.
will go down into that passageway and finish the service.
before he would look after his own life..
I love it in the film where,.

$^{801}$when Father Samuel said that to me,.
I was like, kinda thought like he was sort of half joking..
I'm like, "Really, really?".
And he's like, "Yes, yes, of course.".
Almost like, "Would you not do that?".
And I'm like, "No.".
I would run really fast and throw Vine people behind me..
No, just kidding, I wouldn't do that..
But what a deep challenge, isn't it?.
Like we worry, and maybe rightly so,.
what's our freedoms gonna be like in Hong Kong to worship?.
We are so free to worship..
We do not have things on the bottom of our floors.
that we need to escape down if somebody comes in to kill us..
Are you with me, church?.
And they are so focused on worship.
because they've come to understand the beauty of the Exodus.
that God sets us free so we can worship,.
sets us free so that we might be a sign.
to the world of that freedom..
I mean, this church in the center of Cairo today,.
surrounded by all that it's surrounded by,.
is a sign that the Exodus actually happened..
And I love how God says it here..
He says, "This will be a sign to you..
You will gather to worship.".
This is always what worship is..
It is a sign to a tired and weary world that God is true,.
His promises are right, and He delivers His people..
And I love how God says here,.
"It'll be a sign to you after you come out of Egypt.".
I love this because so often when God calls us to things,.
we wanna sign today..
Give me a sign first, and then I'll do what you asked me to..
Give me a sign, Lord..
We're like, "Oh, I think God's calling this.".
God, just confirm it with a sign..
And God says to Moses, "I'll give you a sign,.
but it will come at the end..
I'll give you a sign after you've been obedient.

$^{841}$and trustful and faithful..
And you will gather around this mountain.
and it'll be a sign..
And from that point forward,.
it'll be a sign for all centuries and centuries.
and centuries of the power that there is.
for a God to deliver His people..
The sign will come at the end to tell you.
that I was with you the whole way.".
The filming, I'm gonna finish with a story..
I am finishing..
I'm gonna finish with a story..
The day that we filmed in the hanging church.
was day seven of 23 days of filming in Egypt..
And it was quite easily the hardest day of filming for me..
Not because it was technically difficult,.
but because of how I was feeling inside..
We were six days in or seven days in,.
and I have to say it was such a battle.
to even get to Egypt..
Four years of canceled visas,.
of not getting security clearances,.
of not knowing whether the company we were working with.
in Egypt was gonna come through for us..
Four years of ups and downs..
And every step of the way, pushing and pressing on.
and trying to do it and wondering, what are we doing?.
Is this really gonna happen for us or whatever?.
And then we finally get to Egypt and we start filming..
And the filming is going relatively well,.
but self-doubt and insecurity is kind of really there for me..
And I'm wrestling with this reality of,.
have I just tried to make this work in my own effort?.
Like, is God really in this?.
I mean, is this really gonna happen?.
Is this gonna go through?.
Like, is this gonna be a blessing.
to the church in Hong Kong?.
Is this actually gonna help anyone?.
And the day before we filmed at the hanging church,.

$^{881}$one of our crew members from the UK.
got a phone call from his wife..
His wife was pregnant and she was supposed to give birth.
seven weeks after the end of filming..
But she calls him and says, "My waters are just broken.
"and I'm heading to the hospital.".
So he, understandably, jumps on a plane that day.
and flies back to the UK and he manages to get there.
in time for the birth, amazing thing..
But we were a person down on our shoot as well..
And that was frustrating me as well..
Like, oh God, why have we come all this way?.
And now like all this stuff is happening.
and why did that person come?.
God, what are you up to?.
How you even end this?.
Well, at the end of the filming with Father Samuel,.
I pretended like I was happy..
(laughs).
"Oh, Father Samuel, oh.".
He's such a nice guy, right?.
Like, it's like, how can you not like be nice around him?.
But inside I was like, ah, my attitude was really bad..
And at the end of it, after we had finished filming,.
he said, "Andrew, I'd like you to come to my office..
"I want to share Egyptian tea with you..
"But you can't bring your cameras.
"'cause it's a sacred place.".
And I'm like, well, internally I was like,.
well, if I can't take my cameras,.
what's the point in going?.
(audience laughs).
Are you with me, church?.
Don't be with me, don't be with me..
(audience laughs).
But I, and so I'm like having this bad attitude..
I'm like, I'm not sure..
And we had more filming to do in other places that day..
So I was looking at my watch, looking at the schedule..
And I remember like our Egyptian crew,.

$^{921}$one of the Egyptian guys that we had hired.
to help us with all the locations,.
he took me aside into a private place..
It's just like, you know, when Jesus was bad.
and the disciples had to take Jesus to the side,.
you know, and so..
So he takes me to the side and he goes,.
"Andrew, you need to understand.
"when an Egyptian priest invites you for a tea,.
"you do not say no, it's offensive if you say no.".
So I'm like, are you telling me I have to do this?.
He's like, "You have to do this.".
So I'm like, okay, come on, let's go..
Okay, I don't even like Egyptian tea, but that's fine..
(audience laughs).
I'll go..
So myself on my own, no cameras, no nothing..
We go into his room and we sit down..
And Father Sam was such a nice guy..
He sits me down in a chair next to his desk..
He's got this big, massive, ornate desk..
I sit down, there's a Bible opened on his desk..
There's a Bible opened on my desk, by the way,.
most of the time..
There's a big Bible opened on his desk..
And on top of the Bible is an icon, one of the icons..
It's a wooden cross..
And immediately my eyes are drawn to this cross..
And I'm trying to think,.
what am I gonna talk about with Father Samuel?.
So I'm like, oh, nice cross..
Can I have a look at the cross?.
So Father Samuel takes the cross off his Bible..
He passes it over to me and I look at it..
Now, it's a cross where there are two images on both sides..
On one image, on one side, it's the ascension of Jesus,.
Jesus floating up into heaven..
On the other side, it's Jesus on the cross,.
bleeding on the cross..
It was quite gruesome..

$^{961}$It was kind of one of these icons.
where you can turn it like this..
But what struck me as soon as I saw it.
was that on the cross at the top and the bottom.
were massive pieces of grapes, pitches of grapes,.
and a little vine that was connected to the grapes..
And I looked at this cross and I said to him,.
what does this mean?.
What is this cross?.
What does it mean?.
And he said, oh, this was made by the nuns in our nunnery,.
just by the side of the church..
And this is the cross that says,.
I am the vine and you are the branches..
And I started to like well up with tears..
'Cause I mean, this guy didn't know all of the emotions.
that were going around in my heart that day..
But God knew..
And God had provided a sign..
Not at the start, not even in the middle,.
not even at the beginning of year four..
But when I had finally gone to Egypt,.
we'd finally started recording and there it was,.
I was with you the whole time..
And that just released worship..
It just released gratitude and thankfulness.
that my God was with me the whole time,.
even though so often I felt like he was distant..
And I was literally like welling up with tears..
And I think Father Samuel thought.
that maybe the icon was having an effect on me or something..
And he said, would you like to keep it?.
And so this is now one of my favorite possessions..
Here it is, this is it..
You can see it up here, you can see a big picture of it..
Look at that, there's that side.
and then there's the not so nice side.
where he's bleeding and stuff..
But can you see the grapes, top, top and bottom?.
You can have a look at this afterwards.

$^{1001}$if you'd like to look at this..
It's one of my, honestly, one of my great,.
I'm gonna place it on my Bible, one of my great treasures..
(audience laughing).
It released me to worship..
And I want you to know that God has a call on your life.
that you're not adequate for..
And if you're part of the church here in Hong Kong right now,.
there's an important call on your life..
And that is to be someone who can testify.
and be a sign to the world that God's promises are true,.
that he can bring freedom,.
and that he can truly set us free by his grace..
And as we sing, as we go about our workplace,.
as we meet God in various areas of our lives,.
may we constantly be a symbol of the freedom of God..
Amen..
I wonder whether you'd astound me, I wanna pray for us,.
and I wanna release us into a moment of worship together..
I invite you just to open your hands..
Lord, we come before you now,.
grateful for the example of the Hanging Church in Cairo,.
for Father Samuel and the other priests,.
for the worshipers in that community there..
Father, as we're here in Hong Kong.
and we have some questions.
about our religious freedoms in our city,.
and as those questions perhaps are over us at this time,.
and we're wondering what the future might hold,.
Father, thank you for this reminder.
that our focus isn't on those things..
Our focus is on you,.
and of the incredible privilege we have right now.
to worship you, to be free to gather in this room,.
to open our hearts and to declare with our voices,.
to be a sign in our city that your promises are true..
Father, I pray for anyone in this room.
that feels inadequate for the call of God on their lives..
I pray that, Lord, you would fill them today.
with your spirit again and tell them I am with you,.

$^{1041}$that I am sufficient for your inadequacy..
Lord, I pray that you would release worship in this place.
as we recognize that our freedom from sin.
releases us to a place of gratitude,.
to a place of offering our hearts to you.
and our worship to you..
And Lord, as we worship you now, come Holy Spirit..
On Pentecost Sunday, they gathered in that room,.
your spirit fell, and they worshipped you..
They worshipped you in multiple tongues and languages.
so that everybody who didn't know you in their city.
could come to know Jesus..
That was a sign of the power of the freedom of worship.
led by the spirit..
And Lord, as we worship you now,.
would you move in the spirit,.
and with all the cultures that are represented in this room,.
may we be your hands and feet.
to a tired and weary Hong Kong..
Can we bring the hope of the gospel in this time..
We pray this in Jesus' name..
\newpage



\section{}
\label{sec:_gagT3D9Two}
\textbf{2023-06-05 EXODUS - 08 What is in Your Hand? [\_gagT3D9Two].mp3}
\newline
\newline
連結: \href{https://youtube.com/watch?v=_gagT3D9Two}{\texttt{ https://youtube.com/watch?v=\_gagT3D9Two}} ~~~~ 語音日期: 2023-06-05 
\newline
\newline
\hyperref[sec:GwiSx82TY4o]{\small{< < < PREV SERMON < < <}}
~
\hyperref[sec:index]{\small{[返主目錄]}}
~
\hyperref[sec:pxaoLPqgKJE]{\small{> > > NEXT SERMON > > >}}
\newline
\newline
$^{1}$Look, I know what you are thinking..
This is not Andrew..
You are not wrong..
For those of you that don't know me, I'm John..
I'm one of the founding pastors of the Vine Church..
But I'm sure I speak for all of us when I say how greatly we have been blessed and more.
particularly personally challenged by this amazing series..
I feel humbled and honored to be here today and pray I can do the subject justice..
And a big shout to those of you who are watching this online as well..
As we see Andrew in Egypt in a moment, we will have a chance to view one of the only.
remaining of the seven wonders of the ancient world..
But what we will also get is a perspective of the power and the wisdom and the strength.
of the Egyptians..
So what we'll see is his worldly power and worldly wisdom and worldly strength..
But can I say impressive nonetheless..
So without further ado, let's go over to Pastor Andrew in Giza..
[Music].
There is nothing more iconic of ancient civilization in Egypt than the great pyramids of Giza..
These monumental tombs were constructed some four and a half thousand years ago, each taking.
around 20 years to complete using an estimated 100,000 skilled workers..
The project was begun by Pharaoh Khufu in 2550 BC and his great pyramid is the largest.
in Giza, towering some 481 feet above the plateau..
These pyramids are made of local limestone and the outsides were so highly polished that.
the stones would sparkle in the sunlight, making the pyramids shine like a huge jewel..
On a clear day, they can be seen from as far away as modern day Israel..
They were and still are a powerful declaration of ancient Egypt's brilliance, power, and beliefs..
Belief was indeed one of the primary driving forces behind why these pyramids were constructed.
in the first place..
You see, ancient Egyptian pharaohs believed that they would become gods in the afterlife.
and so they filled their burial tombs with the things that they would want in the afterlife,.
things like jewelry and art and goods and food and gold..
And so contrary to what people often think, these pyramids actually tell us more about.
Egyptian life than they do about Egyptian death..
And it's this brilliance of Egyptian life that the pyramids really reveal to us when.
we get up and close and personal to them..
I mean to be here and actually to be standing amongst these incredible structures, I mean.
it truly is a breathtaking experience..
The scale and the scope of these structures is unparalleled in the ancient world..
And when you begin to understand some of the facts behind how the ancient Egyptians actually.
constructed the pyramids, I mean you truly become in awe of what those ancient people.

$^{41}$actually achieved..
Take for example the engineering precision of these pyramids long before modern day technology..
Each of the pyramids' sides rise at an angle of 51 degrees 52 inches and are perfectly.
accurately orientated to the four cardinal points of the compass..
They are in fact the most accurately aligned structures in existence, facing true north.
with only a 360th of degree of error..
Not only this, but they are located at the intersection of the longest lines of longitude.
and latitude, meaning they sit at the exact center of the earth's landmass..
Just stop for a moment and reflect upon that achievement..
I mean in order for the ancient Egyptians to actually get the accuracy of the placement.
of the pyramids so correctly, they would have had to have had advanced knowledge of such.
things like the form and the shape and the weight of the earth, its relationship and.
distance to the sun, things like the actual length of the solar year, the number of years.
that are in the precessional cycle, the average temperature of the habitable world, and so.
many cosmical facts and mathematic formulas that actually weren't going to be invented.
for centuries to come..
I mean how the heck did they actually manage to do it?.
Well scientists and engineers are no closer to discovering today some four and a half.
thousand years later..
The brilliance is seen not just in the engineering or the placement of the pyramids, but actually.
in the very building blocks themselves..
All 2.3 million building blocks had to be individually prepared, individually cut, then.
transported, brought here, lifted, placed into the pyramid in exactly the right place.
where it needed to go..
And the thing that blows my mind is that actually when you look at the average distance between.
every single building block on a pyramid like this, the average distance is just 0.5 millimeters.
in width..
I mean the accuracy of that just blows my mind..
But to show us something that any of us who live in a hot climate would deeply appreciate,.
I actually need to take us inside the pyramid itself..
Come with me..
Say hello to the most effective air conditioning system in the world..
Here's the crazy thing, despite the amount of heat that's pounding down on the outside.
of these structures right now, inside the pyramids the average temperature has never.
gone above 20 degrees centigrade..
Never..
And here's the crazy thing, 20 degrees centigrade is the average temperature of the earth..
Now how the Egyptians ever knew that, I have no idea..
But even this is not the most amazing fact about the pyramids..
There's one more thing that really blows my mind..

$^{81}$And to tell you about that, unfortunately, I now need to take you back upside to the.
heat..
If I was to ask you how many sides the great pyramids of Giza has, you'd say four, just.
like me..
Well, actually we would be wrong..
There's not four sides to these pyramids, there's actually eight..
And here's how it works..
The ancient Egyptians, when they built these, they built them with slight, from bottom to.
tip, slight concave indentations in each side of the pyramid..
And those concaves are only viewed at certain times within certain lights of the year..
Now here's the thing that really, really is incredible..
They built something into these pyramids that no one in their lifetime would ever actually.
be able to see..
And that's because you can only view the eight sides when you're looking down directly from.
above..
And of course, no one in ancient Egypt would ever be able to be up and above the pyramids.
looking down on them..
So the question we have to ask is, why did they even bother to do that in the first place?.
Why did they create such a unique thing in the engineering and in the design that no.
one in their time would ever view?.
Well maybe the answer is, just simply because they could..
I think it's fair to say that these pyramids are a true human miracle, a testament to the.
brilliance of the Egyptian civilization..
And when you're sitting here and you're a part of it and you experience it, you can.
truly understand why Moses was so terrified to oppose them..
I mean, how could a shepherd with a lisp stand against a civilization that could do this?.
Well, the answer to that is found in what we see God do next..
Life is not always fair..
For his part of the talk, Andrew got to go to Egypt..
For my part, I got to go to Legoland..
My nine-year-old grandson, Angus, put this together for me..
But the fact that even this is not your run-of-the-mill Lego model, it's described as an adult architectural.
model..
On the cover, age 18 plus..
It gets me how they think that someone who's 19 can actually build it quite easily..
Someone who's 17 is obviously too young..
And it was over 1,500 pieces..
It just reinforces Andrew's observations that the amazing background to the pyramids and.
the incredible way in which they have been constructed and some fascinating facts about.
them all show how smart, how really smart the Egyptians were..

$^{121}$And how much fear that must have created in Moses..
But what we're going to see, I'm going to give you a spoiler alert now, is that despite.
all the worldly power and worldly wisdom and worldly strength of the Egyptians at this.
time, God's power is indisputably the greatest power in the world..
And nothing could ever compare..
But Moses didn't see this..
By now he was just a shepherd boy..
And a shepherd boy with a sistama to boot..
He had experienced the great power of the Egyptians when living in Pharaoh's palace..
But by now, this was over 40 years ago..
We saw last week that God shocked Moses by telling him that he, he was God's chosen agent.
to bring about the exodus..
Predictably Moses said, "Who me?".
As Andrew so clearly showed us..
God said, "Yes, you.".
In chapter 3 we read this, remember it..
So now go..
Go!.
I'm sending you to Pharaoh to bring my people, the Israelites, out of Egypt..
In the last two weeks Andrew has shown us Moses at the burning bush and God's call on.
his life to partner with him in returning to Egypt and letting his people go..
Moses has had a lot to say about this and feels very inadequate for the task..
And it is this conversation that we now pick up this week as we head into chapter 4..
As we begin chapter 4 Moses is still stammering..
Then Moses answered, "But behold, they will not believe me..
They will listen to my voice, for they will say, 'The Lord did not appear to you.'.
Then the Lord said to him, 'What is that in your hand?'".
Let's stop there..
I guess many of us have felt like Moses..
"Who Me" has been our theme song, probably being paid on repeat..
I know that certainly that was the case 20 years ago when I felt the call of God to pastor.
this church..
I had none of the usual qualifications..
I hadn't been to theological school..
I hadn't pastored a church and I certainly didn't resemble any of the pastors I knew..
Thank goodness..
I was just an insurance man..
I joined the insurance industry at 17 from school and that was all I knew..
Who me?.
I found myself in a similar position to Moses..

$^{161}$I would say to myself, "People won't listen to me..
I'm not a real pastor.".
But in the same way, I felt the same question from God..
"What is in your hand?".
And Vine Church, this is the same question that God is asking each one of you today..
"What is it that is in your hand?".
In my case, I hastily arranged to have a dessert buffet at the Marriott Hotel with Tony Reed,.
who by the way was just an engineer, with our wives..
And we resolved that together and with God's grace and provision, we might just be able.
to do this..
Who me?.
Yes, you..
God then shows Moses three signs to help him to be able to prove that God is at work in.
this world and is truly the most powerful thing that there is..
The first sign is a staff..
Verse two says, "The Lord says to him, 'What is it that's in your hands?'.
He said, 'A staff.'".
In the 40 chapters of Exodus, the word "hand" is used 64 times..
The hand is associated with a person's activity..
It represents what a person does..
So how were Moses' hands made ready for the Lord's work?.
For starters, humility..
You see Moses' staff was that of a shepherd..
The Bible tells us in Genesis 46, 34 that every shepherd was an abomination, was repugnant.
to the Egyptians..
As Moses stood before Pharaoh and all of Egypt, the staff in his hand would be a constant.
reminder of Egypt's total disdain for him because of his vocation..
This in turn would remind him that without God, he could do nothing..
Moses, without God's favor, could do nothing to please Egypt with his ever-present reminder.
of his lowly stature grasped in his hand..
In himself, he was an outcast, a person despised and rejected in the world's most dominant.
nation..
But, I love that word, the most dominant nation in the world would soon learn that there was.
one more powerful..
In Moses' might, the staff was simply that of a lowly shepherd..
However, as the staff of God was used in this man's surrendered hand, the wooden stick became.
a symbol of authority..
This represents a precious principle regarding how God uses people..
God used what Moses had in his hand..
Moses' years of tending sheep were not useless..

$^{201}$Those years had put into Moses' hand the thing he could use for God's glory..
I love this..
I'll put this up on the screen for you..
God did not use the scepter that was in Moses' hand, in his royal hand when he was in the.
palace, but he did use a simple shepherd's staff..
Church, God loves to use what is in our hands..
Just look throughout the Bible..
God used what was in Shamgar's hand, what we call a cow prodder, a cow prodder to kill.
600 Philistines..
God used what was in David's hands, five stones in a sling to kill Goliath..
God used the jawbone of a donkey in Samson's hand to slay a thousand men..
God used five loaves and two small fish in the hands of a little boy to feed 5,000 men,.
plus women and children..
Again, the words of the Psalmist ring true..
Psalm 62, 11 says this, "One thing God has spoken, two things have I heard..
Power belongs to you, God.".
So after Moses' humility came his obedience..
Verse three says this, "And he said, 'Throw it into the ground.'".
Look what happened next..
So he threw it in the ground and it became a serpent and Moses ran from it..
I'm not surprised, would you?.
We all run from it, right?.
It became a serpent and Moses ran from it..
But the Lord said to Moses, "Put out your hand and catch it by the tail.".
Yeah, I'm sure..
Church don't miss this..
This is very, very, very dangerous..
I don't know if we have any snake charmers here, but any trained expert snake charmer.
would take up the serpents by the neck so that they might not be able to bite him..
Moses was encouraged to show his trust in God by taking the serpent up from the tail..
Yeah, right..
His courage as well as his faith is shown in his ready obedience..
Now this ties in beautifully with what Andrew said last week about God's promises inviting.
us into being active partners in the three areas of obedience, trust, and faith..
So this is what happened..
So he put his hand out and caught it..
And it became a staff in his hand..
I'm not demonstrating this one, by the way..
Thus Moses' initial fear was compounded by the command to grab it by the tail, an action.
that could easily result in a fatal strike..

$^{241}$But Moses obeyed the word of the Lord, overcoming his fear, and the snake returned to a staff.
in his hand..
The second sight was Moses' hand..
Now Moses' hand wasn't bitten off by the snake, right?.
So it's still there..
It was the subject of the second sign..
Let's read it..
"Again the Lord said to him, 'Put your hand inside your cloak.'.
And he put his hand inside his cloak..
And when he took it out, behold, it was leprous like snow..
And the Lord said, 'Put your hand back inside your cloak.'.
So he put his hand back inside his cloak..
And when he took it out, behold, it was restored like the rest of his flesh.".
After Moses' humility and obedience came his righteousness..
Putting his hand inside his cloak was symbolic of Moses putting his hand over his heart..
The heart represents what a person is and what he speaks..
As I said earlier, the hand is associated with a person's activity..
It represents what the person does..
The Hebrew word for leprosy here covered a number of assorted diseases, much as our word.
cancer does today..
When Moses pulled his leprous hand out of the cloak, it showed him that because man's.
heart is evil, so too will be the work of his hands..
But when Moses returned the second time, returned his hand and it was clean, it signified that.
his hand that holds God's stuff must be cleansed..
A cleansing evidenced by a clean heart..
Over the past few weeks, we've been looking at the personal journey that Moses has been.
on the level of his identity and what he thought about himself..
The naked and unashamed invitation by God at the burning bush to come just as he is.
to receive healing and acceptance..
Now with a pure heart and a pure hand, the Lord is saying, "Take my stuff and do my work.".
Each of the first two signs had to do with transformation..
Something good and useful, a rod or a hand, was made into something evil, a serpent or.
a leprous hand..
And significantly, they were transformed back again..
There was a real message in these first two signs..
The first said, "Moses, if you obey me, your enemies will be powerless.".
The second said, "Moses, if you obey me, your pollution can be made pure.".
Doubts in each of these areas probably hindered Moses..
And before these signs spoke to anyone else, they spoke to Moses..
This is a pattern with all of us who call ourselves followers of God..

$^{281}$Let me remind you, Andrew has been saying repeatedly that this Exodus series has been.
primarily about your personal Exodus, perhaps even more than the wonderful passage to freedom.
of the Israelites..
What is saying to us today, right here, right now, obedience..
Obedience is the key to your personal Exodus..
Let me say that again..
Obedience is the key to your personal Exodus..
Obedience will nullify your enemies..
Obedience will purify your heart..
Obedience is your key to deliverance..
The characteristics of humility, obedience, and righteousness that were necessary for.
Moses are the same for us today..
However, the foolishness of a shepherd's staff used to deliver people has been transformed.
into the foolishness of a cross..
Paul tells us in 1 Corinthians 1.18, "For the message of the cross is foolishness to.
those who are perishing, but to us who are being saved, it is the power of God.".
On to the third sign, and the third sign was water into blood..
Let's read it..
If they will not believe even these two signs or listen to your voice, you should take some.
water from the Nile and pour it on the dry ground..
And the water that you should take from the Nile will become blood on the dry ground..
My friends, the third sign was of a different nature..
The third sign was simply a sign of judgment..
Good, pure waters were made foul and bloody by the word of God, and they did not turn.
back again..
Notice that..
This showed that if the miracles of transformation did not turn the hearts of the people, then.
perhaps the sign of judgment would..
The Bible tells us if they do not believe even these two signs or listen to your voice,.
shows that the sign of judgment was only given when unbelief persisted in the face of the.
miracles of transformation right before them..
Now the power of Moses to turn the water of the great Nile into blood should be understood.
in the light of the status held by that river in the Egyptian culture..
The river Nile was honored as divine..
It was a god..
Its waters were held to be the source of all that was good and desirable in Egyptian life..
In a word, it was an idol..
Through Moses, God showed his power and superiority over the pagan gods of Egypt..
As an aside, it demonstrates the need for us to remove all the idols from our life as.
we work through our personal exodus..

$^{321}$And our idols, as Andrew's told us, can include seemingly good things like our job, our family,.
our financial security..
It was also a forerunner to the first great plague that we will see later this year in.
Exodus 7..
Now the signs may have convinced Israel..
They didn't convince Moses..
Or at least Moses was not ready to yield yet..
Look at this in verse 10..
But Moses said to the Lord, "Oh my Lord, I am not eloquent, either in the past or since.
you have spoken to your service, but I am slow of speech and of tongue.".
God's answer to this is astonishing and in a way incomprehensible..
Verse 11, "The Lord said to him, 'Who has made man's mouth?.
Who makes him mute or deaf or seeing or blind?.
Is it not I, the Lord?'".
My friends, this is a dramatic statement, revealing the sovereignty of God and God revealed.
it in the context of an invitation to trust God and to work with him..
Now we might feel that Moses had every right to expect healing..
We believe that healing is included in the atonement..
Not only does God not give healing to Moses, but he actually seems to take responsibility.
for the defect..
God knew as well as Moses that sickness and death, including Moses' impediment, were a.
result of the fall..
What God seems to be saying to Moses is the same thing he said to Paul when the apostle.
prayed for the removal of his thorn in the flesh..
"My grace is sufficient for you, for my power is made perfect in weakness.".
But this only leads to another round of punches between God and Moses..
As you know, always wins, right?.
God says, "Therefore go, I will be with your mouth and teach you what you shall speak.".
Moses, don't you love this guy?.
"Oh Lord, please send someone else.".
Do you recognize those words?.
Yeah?.
I do..
Every time I get an incoming call from my funeral director friend, Car, from White Lily,.
my heart goes in my mouth, telling me that someone has passed away..
Maybe someone old, maybe someone very young..
Maybe dying after a long illness, maybe a suicide..
I know what's coming next..
Can you possibly take the funeral?.
And oh, can you go and visit the family?.

$^{361}$Who me?.
Is there no one else?.
M'siqw'an dawah..
Moses now shows his true colors..
His problem wasn't a lack of ability..
It was a lack of willingness..
If you want to know how to enrage God, practice these words, "Send someone else.".
We read verse 14, "Then the anger of the Lord was kindled against him.".
God wasn't angry when Moses said, "Who am I?".
God wasn't angry when Moses asked, "Who shall I say sent me?".
He was not angry when Moses disbelieved God's word and said, "They won't believe me or listen.
to my voice.".
He was not even angry when Moses falsely claimed that he was not and had never been eloquent..
But God was angry when Moses was just plain unwilling..
The basic truth is that Moses was unwilling and not unable..
This leads us to a fascinating few verses..
I want to read them as a chunk here..
I love this..
"Then the anger of the Lord was kindled against Moses and he said, 'Is not Aaron your brother,.
the Levite?.
I know that he can speak well..
Behold, he is coming out to meet you and when he sees you, he will be glad in his heart..
You shall speak to him and put the words in his mouth and I will be with your mouth and.
with his mouth and I will teach you both what to do..
He shall speak for you to the people and he shall be your mouth and you shall be as God.
to him.'".
Now at first sight, it seems that God is helping Moses out..
"Ah, don't worry bud..
I understand how you feel..
I'll get your bro, Aaron, to help you.".
Now he's a real talker..
But we've been missing the point..
When God brought Aaron to help lead with Moses, it was an expression of his chastening to.
Moses..
Not of his approval or giving in to Moses..
Aaron was often more of a problem to Moses than a help..
In Exodus 32, it was Aaron that instigated the worship of the golden calf..
He fashioned the calf himself and built the altar himself..
In Leviticus, Aaron's son blasphemed God with impure offerings..
In Numbers 12, Aaron led an open mutiny against Moses..

$^{401}$As these episodes unfolded, Moses surely looked back at why God gave Aaron to Moses as a partner..
Because God was angry at Moses' unwillingness..
Now Aaron might have been a smooth talker, but a man weak on character..
Moses had to put the words of God into the mouth of Aaron..
You shall speak to him and put the words in his mouth..
Now don't get me wrong here..
Let's put this in context..
We shall see that there were times when Aaron was an incredible help to Moses and was in.
his own way a great servant of God despite his faults..
When we get to Exodus 17, we'll see Aaron and her just faithfully lifting Moses' hands.
during the battle, the Israelites' battle against the Amalekites..
But let's go back to Exodus 4..
The revelation ends..
This is great, I love this..
With God reminding Moses about his staff..
I don't know if there's any irony in God's reminder..
As if he wants to say, "Don't forget your stick.".
Verse 17, end of the chapter, "And take in your hands this staff with which you will.
do the signs.".
As we conclude, I want us to see this story as an encouragement..
When God calls us in life, whether that's in a marketplace position, in a family, in.
relationships, in whatever sphere of influence we have, he always equips us through his miraculous.
and awesome power..
We don't see staffs turning into snakes these days..
But what we do see is God changing people's hearts..
We do see him anointing us beyond our gifting..
We do see him creating favour we don't deserve and opening doors that we never thought could.
open..
In the same way that God said to Moses, "What is in your hand?".
So God says to us, "What is it I've given you?.
What is it I've equipped you with?.
I've gifted to you?.
Those are the things that through my Spirit can become the sign to others that I am the.
one true God.".
What we're going to do in a moment is something special..
Very simple ministry time..
I'm going to be asking you to stand in a few moments..
Very simple, I'm going to ask you to hold your hands out..
Now before you do that, am I giving you everything, Lord?.
Am I holding something back?.

$^{441}$You say to me, "I'm not very gifted.".
Whatever you've got, put it in your hands..
We as a church are going to stand before God and we're going to ask for his Holy Spirit.
to anoint those things..
I believe that as we do so, God is going to do in your life more than you could even.
imagine..
In the life of this church, in this city, more than we could imagine if we would only.
offer him what is in our hands..
But before we do that, I'm going to ask everyone to close their eyes for a moment..
I'm very conscious there may be people in here, and there certainly were at the first.
service who haven't yet put your life in the hands of Jesus..
You haven't surrendered your life to him..
I'm going to give you an opportunity to say to him, "Jesus, I know I'm a sinner..
I believe you died upon the cross..
Would you forgive my sins and come into your life?.
Come into my life?".
I'm just going to allow a moment as God speaks to you now..
If you need to make that step this morning, I'm going to pray for you and we as a church.
are going to support you..
Or maybe once upon a time you were walking with God, maybe as a young person, you don't.
actually know why you're in church today, but something's brought you here..
Maybe a friend's brought you here..
You've wandered a long way away from God..
Today is the day when you say, "God, I'm coming back to you..
Jesus, come into my life again.".
Church, this is potentially the most important time of someone's life..
Let's respect this by keeping our eyes closed..
And if you're that person or those people and you want to use this opportunity to give.
your life to Jesus Christ right here, right now, I'd love to pray for you..
I'm going to ask you to do one very, very simple thing just so I know who I'm praying.
for..
Could you, Roswa, with every eye closed, could you just raise your hand so I can see?.
Do you want to give your life to Jesus?.
Thank you, sir..
Thank you..
Anyone up in the balcony?.
Thank you, I see that hand there..
I'll give you just a moment..
I see that hand at the back..
Don't be shy, just raise it a bit higher..

$^{481}$Come on, it's good..
It's a good time to do it because we're going to pray for you..
I'm not going to make you feel embarrassed..
I'm going to just pray for you..
I think it is upon us as a church to really spend just a moment for those people who've.
raised their hands..
We never know why someone is raising their hand, but that is a step of faith..
And for those people who've raised their hands, we have people here afterwards who will pray.
with you and encourage you..
But just for the benefit of those raising their hands, I just see another hand go up.
over there..
Thank you very much..
Would you all pray this prayer with me for the benefit of those people as we welcome.
in the kingdom?.
Repeat after me, "Dear Jesus.".
I think that was a little bit shy, everyone, by the way..
Dear Jesus, I thank you that you died for me..
I recognise I am a sinner..
I'm in need of forgiveness..
Jesus, I believe you're the Son of God..
Jesus, I ask you now to forgive my sins and come into my life and fill me with your Spirit..
Lord Jesus, I know now that I am yours and that I will spend eternity with you and my.
brothers and sisters..
I pray this in Jesus' name..
Amen..
Let's welcome the people who've prayed that prayer..
Let's all stand..
Just take a moment..
When I ask you the question, "What is in your hands?".
Don't tell me I'm not as gifted as that person over there..
It's not time to run around and Lord, send somebody else..
Just put your hands out in front of you..
All across the church, this is amazing..
This is an act of faith..
Saying Lord, all I have is yours..
I may not have very much, Lord, but what I have, will you use it?.
And I pray, Holy Spirit, you'll come upon your church..
Fill this church..
Lord, take these hands which are surrendered to you and multiply the gifts and use the.
gifts Lord, for the benefit of others, for the benefit of this city, for this nation..

$^{521}$Lord, you used a little boy who was a shepherd boy, he wasn't little at the time, he was.
lowly to bring about your exodus..
I'm going to pray that God will give you courage..
Courage is the word I have..
That God, by your Holy Spirit, will you just give us courage..
Will you give us courage tomorrow when we go into our workplace?.
Will you give us courage?.
When we go out to dim sum today with our unsaved family friends?.
Will you give us courage on the streets of Hong Kong?.
We are believing, Lord..
This is a significant time for our city..
Holy Spirit, fill me afresh..
I need you..
There's a tremendous sense of God's presence in this place..
Let's not move from there quickly..
His hand is upon you..
See His hand upon this lady over here..
This gentleman here..
God wants to take what is in your hand..
Change a city..
Change a nation..
Who, me?.
Yes, you..
Thank you, Lord..
Let's release ourselves in worship now, say..
Amen..
\newpage



\section{}
\label{sec:pxaoLPqgKJE}
\textbf{2023-06-12 EXODUS - 09 It Gets Worse Before It Gets Better [pxaoLPqgKJE].mp3}
\newline
\newline
連結: \href{https://youtube.com/watch?v=pxaoLPqgKJE}{\texttt{ https://youtube.com/watch?v=pxaoLPqgKJE}} ~~~~ 語音日期: 2023-06-12 
\newline
\newline
\hyperref[sec:_gagT3D9Two]{\small{< < < PREV SERMON < < <}}
~
\hyperref[sec:index]{\small{[返主目錄]}}
~
\hyperref[sec:d2vwib6oxcU]{\small{> > > NEXT SERMON > > >}}
\newline
\newline
$^{1}$Why don't you have a seat, why don't you have a seat..
And I want to start today by asking you a question..
And it's a really honest question..
And the question is this, I wonder if you've ever felt hopeful for like really serious.
positive change in your life and then things just got worse..
Anyone ever experienced that?.
Where you were really hopeful for positive change to happen in your life and things just.
kind of got worse rather than better?.
Have you ever felt an experience where God has spoken a word to you?.
Where God said something into your spirit or confirmed something in you and you felt.
the rush of the wind of the spirit and you know you're about to move into something that.
God has promised for you and there's an excitement that builds up in you and perhaps you've even.
taken that first tentative little step of faith knowing that because God is saying it,.
because God is in it, because God is for it, you know that it's going to go well and as.
you step out in faith, it actually ends up being worse than what life was like before.
God had even given you the word in the first place..
Anyone ever felt like that?.
Just me..
Have you ever felt like God fans something into flame and as soon as you step forward.
in faith, the flame just died?.
Back in the spring of 2006, God spoke a word to me that was perhaps one of the most life-changing.
things He's ever said to me..
The word was this..
God said, "Andrew, it's time to build a family.".
And it's hard to describe to you the literally life-changing events that took place because.
God has spoken those words to me..
I don't know if I've ever heard the audible voice of God, but this was a time in my life.
where I had never felt a word so sure, a word so real, a word that released a wind of the.
Spirit in me like nothing before..
"Andrew, it's time to build a family.".
My wife and I got married early..
I was 16, she was 15..
Just kidding, just kidding..
It wasn't like that at all..
Some of you here, like, you're a guest today..
Somebody invited you to church..
You're like, "What?".
But we did get married young..
I was 23, she was 22..
Now my wife's from a culture where people often get married young and when they get.

$^{41}$married young, they usually start to have kids young..
And usually by about the age of 25, the family has kids already..
The first child comes into the world at about the age of 25..
I grew up in Hong Kong..
I come from a very different culture to my wife..
In Hong Kong, you have to have a big successful career first before you have kids, right?.
Like things take a lot longer in Hong Kong..
People don't generally have kids at a young age..
It usually happens when you get into your 30s and you've kind of lived a life and things.
have happened and maybe you've had some success and now you're ready to settle down and have.
kids..
So my culture and my wife's culture, very different..
And so we get married 23, 22 and Chris is like, "Woohoo, time for kids.".
And I was like, "Woohoo, no.".
Because I wanted to enjoy my life..
I wanted to enjoy my wife..
I wanted to go and see the world and do things together and have great dreams and enjoy life.
without the inconvenience of little kids around all over the place..
I didn't want to be shackled down..
I wanted to be free..
So my wife and I, soon after we got married, had a conversation..
My conversation read, "Argument.".
And she was like, "But I want to have kids..
I'm excited for kids.".
We talked about having kids and we had actually talked about having kids before we got married,.
my bad..
And I said, "But I'm not ready, honey..
We can't do it..
I'm not ready..
I don't feel ready.".
And so I said, "No.".
And this conversation would come up every year..
Every year we'd get to the end of the year and Chris would be like, "Are you ready now?".
And I'd be like, "No..
I'm not ready..
I don't want to do this..
I don't feel equipped..
I don't feel like this is my time..
I feel like we've still got more time together and I don't want to get shackled down.".
And I was going through all of this..

$^{81}$And throughout the years of my 20s, she kept on going, "Now?".
I kept on going, "No.".
We got to the end of our 20s..
I turned 30..
And Chris is like, "Man, it's been eight years..
We've been married for a long time now..
And what's going on?.
When is this ever going to happen?".
And we were in New Zealand at this time..
And I remember the exact moment..
I'm sitting in a chair and I'm staring out through this beautiful large window at a beautiful.
New Zealand park, green everywhere..
And there's this family, husband, wife, and one child..
And they're pushing their child on the swing set in this park..
And as I looked upon that, I almost heard the audible voice of God..
"Andrew, it's time to build a family.".
And I can't begin to describe the 180 that happened in my spirit in a single moment..
Before that moment, I would look out on those people in the park and I'd be like, "Man,.
that's tough..
You know, bless them, but that's not what I want.".
And literally within just that phrase from God, I felt my spirit completely change..
I felt the rush of the wind of the Spirit in me like perhaps I've never felt before..
And I knew that this was our time..
I knew that God had ordained this time that God was calling us into perhaps the greatest.
journey and adventure my wife and I would ever have..
We would start to build a family together..
And I remember looking out on this family and I remember almost seeing Chris and I standing.
in this park pushing a child together..
And I longed for that moment..
I longed to be a father like I had never longed before..
It was just this instant change in my life..
I remember sharing with Chris, "Hey, Chris, I feel like God's given me a word today and.
I'm ready..
Let's go..
Like this is happening.".
And she starts crying and she's so excited..
She's been so long suffering, waiting for me to get to the place where I'm finally there..
And together we're so filled with joy because God had spoken..
And when God speaks, things happen..
Amen?.

$^{121}$Amen..
Yeah..
I told this story many times over the years..
If you've been here at the Vine, you'll know what happened next..
God did not roll out the red carpet of fertility for Chris and I..
In fact, we tried for around about a year..
And at the end of the year, we realized that this was not happening like we thought it.
was going to happen..
We had the question, "Did God really say that it was time?".
And we entered into a time of tests..
I figured my wife, there's lots of things that could go wrong in that area..
With a guy, it's only one thing..
So I kind of like thought the issue would be with her..
But we very quickly discovered that the issue was with me..
And that entered us into another year of procedures, of tests that culminated in a surgery to discover.
whether I was biologically able to father children or not..
And at the end of that one surgery, the next week we were called into the doctor's office.
at the fertility clinic that we were a part of..
And Chris and I sat down together holding each other's hands, looking across the desk.
at this doctor white coat clipboard..
And he turns to me and he says these words..
These were his exact words..
He says, "Andrew, you will never be a father.".
I want you to sit into the contrast of the paradox of these two words about a year and.
a bit apart..
"Andrew, it's time to build a family.".
"Andrew, you will never be a father.".
And I remember thinking to myself, what is going on here?.
Because more than anything in my life, I knew that God had spoken..
I had never felt the rush of the wind of the spirit like that and ever before..
I knew that God was in this..
I knew that this was our time..
And suddenly things have gotten a lot worse than better..
In fact, our marriage was better before God had given me that picture of that family and.
God had given me that word..
We had now just gone through all of this pain, all of this hurt, all of this frustration,.
and I was angry..
And I wonder whether this resonates with any of you here today..
Like when you see these two things here, maybe for you it's not that exact issue, but you.
know a time in your life where you felt like God say something and you took that step out.

$^{161}$and things did not go how you thought..
In fact, it got worse rather than better..
I remember writing in my journal something super honest..
I wrote this, "How can you call me into freedom and then take me into more slavery?".
I was wrestling with this idea..
How can you, God, call me into great freedom, but then bring me into a place of deeper slavery?.
That question and the paradox that it creates is exactly what we see happen next in our.
Exodus journey..
We're entering into chapter five today, and I want to tell you this..
Chapter five is one of the hardest chapters in the whole of the journey..
Chapter five is the moment where Israel have looked out that window and seen that swing.
set and gone, "God is for us.".
And then nothing happens quite how they were expecting it to happen..
In fact, things get a lot worse before they get better..
And perhaps some of you today, you're in a situation where your life feels like it's.
worse than it's ever been before..
Perhaps things have happened or things are going on for you right now where you're wrestling.
with those deep questions, "Where is God in this?.
I thought God was for me, and if God's for me, then who can be against me, but this is.
my reality.".
And you're sitting in the reality of what you're struggling with..
Perhaps during this series of Exodus, you've enjoyed the series and you're excited because.
you believe that God is calling you out of a slavery..
You believe that God has spoken to you over the last six, seven weeks, and there is something.
that you're trying to exit from..
There's something that you're trying to depart from, an exodus for yourself, and you're excited.
about it, and you're really looking forward to it, and you know that God is for you and.
God's going to do great things..
I want to just say something right up front that I think is really important to understand.
about the whole journey and the book of Exodus itself, and it's this..
Just because God calls us to freedom doesn't mean that the journey is going to be a pain-free.
one..
Come on, church..
And I know this is not going to be easy, but I want to present to you the heart of what.
Scripture reveals to us..
Just because God might be calling us to freedom doesn't mean that the journey is not going.
to be a pain-free one..
It doesn't mean that suddenly we're going to be questioned from all the bad stuff..
It doesn't mean that perhaps we're going to have to go to even deeper places of our brokenness.
and our darkness before we might see the journey of freedom that God has, that maybe there's.

$^{201}$a first thing that needs to happen..
Because one thing you've got to understand that when you're on any journey of deliverance,.
any journey of freedom, the darkness always fights back..
The darkness always tries to fight back..
And if the Exodus story tells us anything, it tells us this, that the journey of deliverance.
is always a journey that has highs and lows, a journey of great mountain peaks and dark.
valleys, a journey, yes, where there's straight moments and straight paths, but also a journey.
where things wind and twist and you wonder whether it will ever end..
And I'm so grateful that the sobering reality of deliverance is that it is not pain-free.
and that the Bible does not shy away from it..
And I want to show you how it doesn't shy away from it here..
And I want to open this up to you with that pastoral heart that we've been talking about..
Because I believe that there is some real nuggets of wisdom for you this afternoon,.
if you resonate at all with the kind of thing that I've already been sharing with you..
I want to show you the contrast that happens between chapter four and chapter five..
John did a fantastic job last week with chapter four, as he showed you the signs that happen,.
that God gives Moses, throwing down the stick and becoming a snake, the leprosy in the jacket.
and that kind of thing..
Those signs were a revelation that God was at work, that God was here to fight on behalf.
of His people..
And at the end of chapter four, you get this climax moment where the people of Israel begin.
to believe that God is going to deliver them..
I want to read to you just the final few verses of chapter four..
"Moses and Aaron brought together all the elders of the Israelites, and Aaron told them.
everything that the Lord had said to Moses..
He also performed the signs before the people, and they believed.".
They saw the signs and they believed..
They believed that their God was come..
They believed that their God cared for them, that their God was compassionate, that their.
God was going to deliver them..
"And when they heard that the Lord was concerned about them and had seen their misery, they.
bowed down and worshipped.".
I mean, what a moment..
The whole of Israel realizing that their 430 years of being trapped in Egypt is about to.
be released..
The whole of Israel understanding that God is now here for them, fighting battles for.
them, that their God is concerned for them..
And so they fall down and they worship Him..
This is everything that God had asked, that He was going to deliver them so that they.
might be free to worship Him..

$^{241}$Right here, before anything else, they start to worship..
And you can almost sense the faith that's in the camp of Israel in this moment..
The sense that God is here now for them..
And it's off the back of this faith and this confidence that Moses and Aaron now go to.
Pharaoh to confront him and ask him to let his people go..
And you can understand that Moses and Aaron are in this moment of great joy..
They're in this moment of great confidence and faith..
And you can understand that they're expecting that if God is for us, who can be against.
us?.
And so in that confidence, they stand before God..
And notice chapter 5, verse 1 onwards..
"Afterwards," so after this amazing moment on this mountaintop experience of worshipping.
God, "Moses and Aaron went to Pharaoh and said this, 'This is what the Lord, the God.
of Israel, says, 'Let my people go so that they may hold a festival to me in the desert.'".
Pharaoh said, "Who is the Lord that I should obey Him and let Israel go?.
I do not know this Lord and I will not let Israel go.".
Not quite what Moses and Aaron were expecting..
They then said to Pharaoh, "The God of the Hebrews has met with us..
Now let us take a three-day journey into the desert to offer sacrifices to the Lord our.
God, or He may strike us with plagues or with the sword..
But the king of Egypt," that's Pharaoh again, "Pharaoh says, 'Moses and Aaron, why are.
you taking the people away from their labor?.
Get back to work.'.
Then Pharaoh said, 'Look, the people of this land are now numerous and you are stopping.
them from working.'".
Notice this, "That same day, Pharaoh gave this order to the slave drivers and the foremen.
in charge of the people..
You are no longer to supply the people with straw for making bricks..
Let them go and gather their own straw, but require them to make the same number of bricks.
as before..
Don't reduce the quota for they are lazy.".
And this is why they are crying up and going, "Let us go and sacrifice to our God.".
Make them work harder for the men so that they keep working and pay no attention to.
the lives..
This was not expected..
They were expecting, "Hey, let us go.".
"Oh, okay, sure..
Yeah, you can go.".
And God is at work here, but this is not how people expected things to happen..
It has gotten not just normal, but worse than before..

$^{281}$Notice this, that there is now slavery taking place at a level that it had never happened.
before..
There's an oppression that's happening at a level never before..
Pharaoh takes the straw away from the Israelites..
The straw was an essential component to making these bricks that they had to make..
It held them and bound them together..
He removes it from them and tells them that they still need to make the same quota of.
the same quality bricks as before, which meant that the Israelites had to go source the straw.
themselves in order to try to make these bricks..
If they made bricks without straw, they wouldn't be strong enough to hold anything..
This is an incredibly difficult thing..
And they're in this moment where they're like, "Hang on..
This is not the way things were supposed to go.".
Because God had said, "Now's the time to build a family.".
And then I discover I'm infertile?.
Hang on a sec..
What's really going on here?.
This so upsets, in fact, the leaders of Israel..
They go before Pharaoh later in the chapter, and they say to Pharaoh, "Hang on..
You must have got this wrong, right?.
Why are you making it harder for us?.
We're already your slaves..
Now you're actually making it harder for us.".
And Pharaoh's response is, "You're lazy..
I'm not going to let these people go so they can worship some God that I don't believe.
in or know..
I'm not going to do that..
In fact, you're going to work harder.".
And then he beats them up..
So not only are they slaves, now they're having to make bricks without straw..
Now they're being beaten up..
This is getting really bad..
And understandably, God's leaders, these leaders of Israel, turn and come to Moses.
and Aaron..
And I want to read you what it says at the end of this chapter as they speak to Moses.
and Aaron..
This is verse 20..
"When they left Pharaoh, they found Moses and Aaron waiting to meet them..
And they said, 'May the Lord look upon you and judge you..
You have made us a stench to Pharaoh and his officials, and they have put a sword in their.

$^{321}$hands to kill us.'".
The leaders of Israel come before Moses and Aaron, where just before they're all worshipping.
up on a mountain, believing God and thinking it's all going to be great..
Now they come before Moses and Aaron and they say, "We hope that God judges you.".
In other words, they put a curse on them..
And they say, "We've become a stench before Pharaoh.".
And I wonder if you could see what's happening here..
They're looking for someone to blame..
And the people that they're going to blame is the leader that's in front of them..
The ones that had caused them to believe that good things were going to happen..
I want you to know, it is not easy to be a spiritual leader of a group of people..
Church, it's not easy to be a spiritual leader of a group of people..
And over the years, man, you guys are lovely, I love you dearly, but over the years, pass.
around you, "Why did you do this?.
Why did this happen?.
You said this would go on..
You prayed for me, but it didn't get better..
It got worse..
You put a curse on me.".
And when you see this heart that we have so often when things don't go our way, who is.
it that we can blame?.
Notice what Moses and Aaron do off the back of this, verses 22 to 23..
Moses returns to the Lord and says, "Oh Lord, why have you brought trouble upon this people?.
Is this why you sent me?.
Ever since I went to Pharaoh to speak in your name, he has brought more trouble upon his.
people and you have not rescued your people at all.".
Moses goes before God and cries out honestly before him and he says, "God, what are you.
doing?.
Because I thought you were in this..
I thought you said build a family, but this is not happening now..
Like every time I try to do this, it's just getting worse..
Why have you not rescued your people?".
These are honest, true, and real words from Moses..
And I'm so grateful that we have this passage in Scripture because it means that we can.
come before God with some of these words that we have, with some of this anger and frustration.
in life..
I mean, if I'm honest with you, I could so easily rewrite this passage myself..
"Andrew returned to the Lord and said, 'Why, oh Lord, have you brought this trouble on.
me?.
Why did you give me the excitement to build a family to only make me incapable of doing.

$^{361}$so?.
Ever since I stepped out in your name, you have brought me nothing but trouble and you.
have not delivered on your promise at all.'".
I wonder if you might have something similar yourself, where you would find yourself at.
times coming before God and saying, "Why?.
This is not right.".
Perhaps you felt God has released you into something like I did that time and it hasn't.
worked out the way you thought it was going to work out..
Perhaps for some of you in here, God said, "Hey, I'm going to make your marriage better.
and your marriage has just gotten worse.".
Maybe for some of you in here, it's your finances and you thought your finances would turn around.
and you've been praying for your finances, but you've just seen your bank account get.
lower and lower and lower..
Maybe for some of you here, it's a sickness that you're carrying, a disease that you can't.
seem to shake and, oh, you've been praying and your friends have been praying and everybody's.
been praying, but it just seems to be getting worse rather than it's getting better..
Or maybe there's some sin or some brokenness in your life and you've genuinely brought.
it before God and you've asked Him to forgive you and you felt God's forgiveness, but you're.
more tempted than you've ever been before..
Why is it that sometimes it gets worse before it gets better?.
How do we reconcile this reality with a good God?.
It's one of the most fundamental questions of our scriptures and one of the most fundamental.
things that is happening right here in chapter 5 of the Exodus journey..
How do we reconcile this paradox that is so often before us in our Christian lives?.
Well, the last thing I'm going to do today is give you some glib answer..
I mean, it would be great if I could stand before you and go, "Oh, okay, here's the one.
prayer that you can pray that's going to sort your whole life out..
If you just pray this prayer in three easy steps, every problem you've ever had will.
go away for the rest of your life.".
Like I wish that could happen..
I wish I had a theological magic wand to teach you something or tell you something that's.
going to immediately change your perspective on every bad thing that's ever happened in.
your life and everything's going to be great again..
I'm not going to stand in front of you today and give you some glib answer..
I have to say, during my infertility journey, although I was surrounded by really good-meaning.
people and people that had a great heart, I got a lot of glib answers to my problems..
And I wonder whether you've ever experienced something like that..
It seemed to me when I was going through my infertility journey that everybody had the.
answer except for me..
I felt like everybody knew exactly what I needed to do to change myself so that I could.

$^{401}$suddenly produce kids all over the place..
"Oh, Pastor Andrew, prayer of Jabez is the prayer to pray..
You just pray the prayer of Jabez, everything will be fine.".
Or perhaps here at the Vine, it was more like this, "Andrew, there are Chinese herbs and.
medicines that you can have that will totally change your life..
Are you with me, church?".
Everybody seemed to have a solution except me..
Everybody seemed to have an attitude towards what I needed to do to suddenly make myself.
better..
And I want to be honest about it, some of those things were more painful and hurtful.
to me than my actual infertility journey..
And so I'm not going to give you some answer today..
What I want to do though is invite you into Moses' journey in this chapter..
And I want to share two specific things about what happens for Moses..
And I want to share them again, not as answers, but as invitations to reflection, invitations.
for you to reflect a little bit deeper about the things that are happening in your life.
and your relationship with God..
Does that sound okay?.
Here's the first thing..
Moses falsely accuses God..
And I think this actually opens up for us a deep reflection around so much of what we.
do as humans when life doesn't seem to match up to what we thought about what God was going.
to do..
You see, we so often accuse God for not being concerned or compassionate towards us..
That's one of the primary things that happens when we're going through a moment like this.
in life..
We suddenly accuse God of not having enough love, not having enough compassion, not being.
concerned enough about us..
And when we do that, we're doing that because of this..
We're doing that because we're allowing our circumstances to define how much compassion.
and love we think God has..
I did this throughout my infertility journey..
There were many times where I looked at the situation I was in and I allowed that to define.
for me how much concern or love that I thought God had for me..
Because if God had this kind of love, surely it wouldn't be like this..
And I was allowing my circumstances to define God rather than God define my circumstances..
Are you with me?.
And this is the first thing that Moses is doing here..
He comes before God and he's like, "Why have you not come through?.
Why have you not delivered?.

$^{441}$Why have you not rescued the people?".
Now this is really interesting because just two weeks prior to this, in the burning bush,.
Moses is like, "I don't want to go.".
Moses has no compassion at all..
"I don't want to go..
I'm happy here in the desert..
I've got a new family..
Everything's great..
Why would I go back to the place of my greatest brokenness?".
And God reveals something to him..
"I'm a God of compassion..
I'm a God who is so compassionate that I look down upon my people and I see them in their.
slavery and it's moved me and now I've come and called you to go back to Egypt.".
God defines himself the first time that he meets Moses as a God of compassion..
Two weeks later, Moses is pointing the finger at God..
Where's your compassion?.
Are you not compassionate?.
And this is really dangerous because here's where this takes us as human beings..
It's dangerous for us to begin to think that we have more love and more compassion than.
God himself..
That's what happens in this kind of journey..
We begin to think that we are the ones who are actually more compassionate and more loving.
than God..
And because we begin to think that way, we begin to take our circumstances into our hands..
"Oh, I'm going to fix it.".
"Well, if you're not going to do it, if you're not loving, well then I guess I'll try and.
do it myself.".
And we suddenly place ourselves in the perspective of God and we become gods in our minds..
This is idolatry and pride..
And Moses is pointing the finger at God and saying, "You're not doing all this stuff.".
And I think that's a place that I know I get to..
And I need a deeper reflection around how I allow my circumstances to define God rather.
than God give me a perspective on my circumstances..
That's the journey that Moses has to go through between this chapter and the next about 20.
or 12 chapters..
I think it's a journey that all of us at times in our lives have to go on..
Here's the second thing that takes place for Moses..
And this is like mind-blowing..
At the one and the same time that Moses is basically throwing a whole bunch of blame.
at God, God is actually at work in Moses..

$^{481}$See, our God is so loving, so compassionate, and so kind that in the very same moment that.
Moses is essentially lashing out at God for not having enough compassion, God is actually.
working on Moses' heart to give him more compassion..
And in a human thinking, we think like, "I would never do that..
This is God.".
And God's like, "Right now I'm actually forming and shaping you in Moses the very compassion.
that you're going to need for the journey that's ahead of you..
Right now as you're in the place where you think things are getting a lot worse before.
they're getting better, and it is actually a lot worse than it was just a week ago for.
your people, I'm actually doing something in you that is forming and shaping and forging.
in you the kind of compassion you're going to need to be the true leader of my people.".
Because Moses has already gotten a lot more compassion than just two weeks previously.
when God had met him at a burning bush..
He's already got more compassion, but he hasn't got the kind of compassion yet that God is.
seeking for him to have, because Moses is still pointing fingers and blaming God for.
how God's not seeming to act..
And what you're going to see later on in the story of Exodus is in the moments when it.
feels like God is not going to act or God isn't doing what Moses thinks, Moses doesn't.
point his finger and blame God anymore..
Here's what Moses says..
He says, "God, you are merciful and compassionate, and I believe that you are all these things,.
and I will not go forward..
We will not take another step forward unless you go with us.".
See, something changes in Moses' heart from a, "Why are you not delivering us?" to a,.
"We will not do anything without you, because you are a merciful, loving, compassionate.
God.".
There is a change that happens in Moses that forms a love and compassion in him that would.
never be there unless it's been forged in the fire of when things get worse before they.
get better..
God is doing something in Moses here that is so deeply profound, shaping him through.
great disappointment, through the rejection of his own people, and through his own anger.
and frustration that is creating in him something that could never be there without it..
Now, I've told you my story of my infertility hundreds of times over the last...if you've.
been at the Vine for the last 10 years, you're sick of hearing that story..
I get it..
But up until now, you have never heard my wife tell you that story, to tell you her.
perspective on that story..
And in many ways, my wife is the perfect echo of everything that happens in chapter 5 for.
Moses..
My wife lives the same journey that Moses lives..

$^{521}$My wife waiting for years and years and years for breakthrough and for hope that things.
will change..
And then her husband comes to her and says, "Yes, things are changing..
It's time to start.".
And my wife, in tears of joy that she now gets to move into a phase of her life that.
she's been long suffering for, and then to discover that there is more long suffering.
to come, in fact, some suffering that neither of us ever experienced..
My wife having to wrestle through the anger and the frustration, through all of the realities.
of the fact that it's gotten worse than it was before..
And then in that, my wife having to wrestle with some of the biggest questions about her.
womanhood, what it means to be a woman and carry a child..
What it means to be a mother..
And having to wrestle through some of the deepest pains and fears that she has..
And in that, find herself forging a new perspective that would never have been in her unless that.
journey had taken place..
I want to invite my wife now to share this story with you..
What you're going to see on this film is raw..
It's really vulnerable for her to do this, particularly to do this for everybody over.
all these services and for the internet and everybody that can see it..
But I want you to lean in and open your heart to her side of the story..
And I pray that you might be able to see a glimpse of Moses' story in her..
Let's have a look..
- You all good?.
- Yeah, good..
Thank you..
- Would you like to introduce yourself a little bit?.
- Sure..
- Okay..
- So, I'm going to introduce myself..
- Would you like to introduce yourself a little bit?.
- Yeah..
I'm Christine..
I'm Andrew's wife..
And I'm going to have a chat with you about our journey of how our family grew, our infertility..
I think we started talking about having children quite young..
We were quite young..
And that conversation lasted for a long time..
And then kind of when we were ready to have children, we couldn't..
We thought that we would get pregnant quite quickly..
I suppose a lot of people think that, and it does take time..

$^{561}$So after a couple of years, we started to get some blood tests and different procedures.
done to see if there was an issue..
And we discovered that there was an issue and that it was primarily with Andrew..
So he had to go through a number of other tests to see what exactly was going on, where.
the problem exactly was..
So it took quite a while..
We saw a number of specialists..
Yeah, there were a lot of emotions, mainly, I think for me, mainly around frustration..
Frustration that we didn't start the process sooner..
And I have to admit that I was like, well, I wanted to start it sooner..
So the fact that it's happening all now, and it's taken such a long time, and we're getting.
older and older, I suppose I kind of put that at Andrew's feet that my frustration is his.
fault in a way..
And then just sadness, anxiety around what things were going to look like, particularly.
for me, anxiety around how Andrew is doing..
Because I could see that it's really hard to see other people struggle, right?.
So I can kind of deal with my own struggles, and I usually keep that pretty private..
But when you're seeing another person in pain and struggling with questions, it's quite.
difficult..
When you want to move into the next life stage, and it's difficult, and you see everyone else.
doing it, it's kind of a bit, what's the feeling?.
It's a bit lonely..
You can feel a bit lonely..
I think that loneliness, that fear of being lonely and kind of being left behind, I get.
angry..
So I tend to, I do withdraw myself from the people that I'm angry with..
And then I feel guilty that I'm withdrawing..
And then I'm going in that whole loop of, well, if this is about a fear of being lonely,.
I'm causing myself to be lonely through withdrawing..
So I think for people around me, for friends, it might have been a little bit confusing.
sometimes..
But of course, they were really good, and they just let me feel the way I felt..
So that was an impact..
So I do remember there was a time where I decided to go out for a walk on my own..
When I'm on my own, I can then think about my own feelings, because nobody else is around.
for me to be attending to..
And that was when I was thinking about not carrying a child..
Because if we couldn't have children biologically, I wouldn't be carrying a child..
I was thinking about having a child in your womb, and all the experiences that I'll be.
missing out on..

$^{601}$And I felt a loss, because right from an early age, little girls are given prams to push.
and babies to care for, and you play house and all that kind of stuff..
And society is like, this is the way that you grow a family..
So I felt sadness, I felt loss, I felt grief that I wouldn't know any of those experiences..
I think when I processed through the life, not being able to carry this life and grow.
this life inside of me, which was sad, disappointing, grief, and then thinking about a life that.
is already present in this world, and we have such an ability to impact that life..
I started thinking about life and how these children who are available for adoption have.
already entered into this world, they've already got life, and their life before them is a.
blank slate..
And it's dependent not on what they choose, but on other people making these big choices,.
which is going to affect their life in such a phenomenal way..
And I felt a great love for that, that Andrew and I could make a decision that will impact.
a life that is already here in such a powerful way..
I felt, because there's a person right, we have a child now, and that's my child, I don't.
want another child..
Even if I went back to that place, knowing what I know now, I'd be even more like, I.
don't want a care, I want this child..
Not long ago I was thinking, you know, I'm disappointed, I feel a loss that I haven't.
been able to carry my child..
And it will always be there, I'm sure as she gets older, every stage of life, you know..
I don't know, there's something that comes through, carrying a child, and I really miss.
that I wasn't able to carry my child..
It's amazing to me that out of a place of my wife's greatest pain, for not being able.
to carry a child, God would forge a love in her for Mia that would be so huge, that now.
her sense of loss is not having carried Mia in her womb..
That her bond as a mother to Mia is so strong that that's now where she feels the loss..
And Moses goes through this moment where he's wondering, what is God doing here?.
This is getting worse than we had ever imagined..
God is forging in Moses a love that perhaps could not be found in him in any other way,.
but would be the very thing that would enable him to lead his people to the promised land..
The same thing is happening for you, and whatever it is that you're struggling with, whatever.
journey it is that you are on, whatever brokenness it is that you might be carrying..
So you have to understand that God's economy is different from the world's economy, and.
that in God's economy, conflict often precedes triumph, that suffering often precedes victory,.
that suffering does not mean that God is not faithful, that God sometimes allows his people.
to make bricks without straw..
Your greatest moments of freedom will happen when you allow God to take you to the places.
of your greatest fears and brokenness..
You see, the deliverance of God is never cheap, and when it is cheap, we so quickly forget.

$^{641}$it..
But when we realize that there is a journey that he brings to us that is not guaranteed.
to be pain-free, but where the pain can forge in us something of him, never be able to be.
found in any other way, perhaps then we cry out, "Maranatha, come Lord Jesus.".
I believe that's actually what the Exodus is all about..
Can I pray for you?.
Let's pray..
Father, I'm so grateful for each person here, and I'm grateful for this pastoral moment.
to be together, and for us to bring ourselves raw before you with whatever it is that is.
overwhelming us in this time of our lives..
Just like we heard today, there are no easy answers..
There are no perfect prayers..
Lord, we look into Moses' life and we see a man that we can connect to emotionally,.
because we've had some of those same frustrations, and we've carried some of those same questions..
And Lord, we thank you for Chris and for her willingness to vulnerably share her story.
with us, and where we can see how things did get worse before they got better for her..
And it was out of that time that you forged a love in her that ultimately would bring.
both Chris and myself to our daughter Mia..
Father, I pray for each person here, for their journey, for wherever it is that they are.
on that journey..
Lord, I pray for us as a church in our Exodus, that Lord, if you need to take us to places.
where we have to face some of these fears, where we have to wrestle sometimes with the.
brokenness at a level that is not easy..
Lord, I would pray that you would help us to go there..
Father, you're delivering us, and we're thankful that in the book of Exodus we can see the.
challenge that there is in such a journey..
We know that the enemy will fight back, and we know that sometimes it gets worse before.
it gets better..
But in the book of Exodus, we also have the way in which we see God work in the midst.
of human affairs, in the midst of human brokenness..
We see a God who is compassionate, who is loving, who is the fullness of mercy, and.
who works in us and through us no matter where we are in life..
And Father, I just really release that over people here..
I want to encourage you, if this is an important word for you to, I want to just encourage.
you whilst our eyes are closed, just to open your hands before you..
That simple posture of humility and surrender to Him is a way of saying, "Lord, would you.
come and meet me right here?".
I don't have the answers, Lord..
I don't know quite what's going on..
I'm trying to trust you, but this is hard..

$^{681}$As you open your hands, you're just coming before Him..
And I believe He will meet you in profound ways, perhaps ways that you never would expect..
And I believe He is forging and shaping in you..
The person who's always longed and seen in you..
And that life, that picture of the girl with the merry-go-round, that life is for you..
And there's nothing that the enemy can do to stop you from the fullness of the life.
that God has for you..
Father, would you come?.
Would you come and pastor your church, Lord?.
Would you come and pastor your people, Lord?.
Come Lord Jesus, come..
Perhaps as you're just in this place, and maybe you're not sure how to pray, I'm just.
going to invite in a moment the team to just sing over you..
Maybe the lyrics and the melody is your prayer..
As God is forming and shaping something in you in this moment, you don't need to rush.
this moment..
Allow His love and His compassion to be with you..
Amen..
(gentle music).
[BLANK AUDIO].
\newpage



\section{}
\label{sec:d2vwib6oxcU}
\textbf{2023-06-26 EXODUS - 10 The Coming Of A Promise [d2vwib6oxcU].mp3}
\newline
\newline
連結: \href{https://youtube.com/watch?v=d2vwib6oxcU}{\texttt{ https://youtube.com/watch?v=d2vwib6oxcU}} ~~~~ 語音日期: 2023-06-26 
\newline
\newline
\hyperref[sec:pxaoLPqgKJE]{\small{< < < PREV SERMON < < <}}
~
\hyperref[sec:index]{\small{[返主目錄]}}
~
\hyperref[sec:2hwTmUlFH_A]{\small{> > > NEXT SERMON > > >}}
\newline
\newline
$^{1}$- Man, it's hard for me to express, two weeks ago,.
two weeks ago here when I last spoke.
as part of our Exodus series was probably the most intense,.
most vulnerable, most personal,.
most overwhelming and emotional,.
but also the most pastoral, I think,.
service that my wife, Chris, and I have ever experienced.
together here at The Vine..
And as we opened up the vulnerability of our journey,.
of infertility, and as Chris in particular,.
profoundly spoke of her experience with God.
through that journey, so many of you were so pastoral.
towards us and reached out to us,.
and we're so grateful for that..
And we kind of knew heading into that week,.
it was gonna be an intense week,.
because chapter five that we were looking at that week.
is an intense chapter..
It's perhaps the most intense chapter.
of the whole Exodus journey itself..
It's a chapter where Moses and Aaron,.
with the wind of the spirit behind them.
and all of the support of Israel,.
go before Pharaoh for the first time.
and say to Pharaoh with so much confidence,.
"Let our people go,".
expecting that Pharaoh is gonna go, "Good idea.".
And Pharaoh turns to them and says,.
"I don't know this, God..
"Why would I let you go?.
"You're my slaves..
"You're part of the economy here..
"We couldn't survive without you..
"That's not happening..
"In fact, I'm gonna punish you.
"for even suggesting that I'm gonna let you go.".
And he places them under even more slavery,.
even harsher conditions..
And not surprisingly, Israel revolts against Moses and Aaron..
"What are you doing?.

$^{41}$"Why have you done this to us?".
And then Moses, at the end of chapter five,.
goes before God and says,.
"God, why are you doing this to me?.
"Like I thought things were gonna go well now..
"You had called us, you had promised us,.
"and I thought things were gonna happen, and it hasn't..
"Why have things gotten worse.
"before they've gotten to get better?".
And we talked about that week,.
how so often that's the journey of so many of us,.
that God tells us something, gives us a promise,.
gives us a hope,.
and we enter into it with a whole bunch of confidence..
And yet sometimes things get worse before they get better..
And I'm so grateful that Chris was willing to share.
just those profound reflections she had had.
about how God had met her in the worst.
and began to reform and reshape a character in her.
that would enable her to be the mother she needs to be.
to our adopted daughter, Mia..
And you see in the chapter five,.
God beginning to take Moses.
and begin to shape in him a character,.
begin to shape and form out of the worst moments.
something that he would need to be.
in order to stand before all of the challenges.
that were gonna be ahead of him.
to be able to stand for God in those moments..
He had to go through what it was.
that he was going through in chapter five..
You know, one of the things I've known.
and experienced in my own pastoral career.
is that so often when God brings us through intense.
and difficult and hard times,.
so often he'll then bring us into a time of release.
and lightness and joy..
See, God's a good God as we've just been singing.
and he's faithful to us..
And the journey with God is not all mountaintop experiences.

$^{81}$but thankfully it's not all valley experiences either..
And God moves us in valleys and peaks and valleys and peaks,.
times where things are hard.
and then a season afterwards of joy.
and a sense of closeness with God.
and things are going well again..
And this is exactly what we see happen.
as we continue our study in the book of Exodus..
'Cause chapter five, as hard as it is,.
as the deepest and darkest valley that it is,.
it moves into chapter six..
And chapter six is like the opposite of chapter five..
Where chapter five was intense and where are you God.
and why are things getting worse?.
Chapter six, God speaks again to Moses..
And in chapter six,.
God begins to redefine his character to Moses once more.
and begins to tell Moses about who he is.
and what he's about to do and the victory.
and the good things and the character he has.
and the joy that is gonna be set before Moses..
And you couldn't get more contrasting things side by side..
When we were planning this series about four years ago,.
we were planning this chapter five week with you.
and we knew that Chris would be sharing her story.
and I knew that would be super intense..
So I was like, okay, the next week,.
how do we capture the joy of chapter six?.
How do we capture the fact that it's so contrasting.
and so different and kind of fun.
and a lot lighter than chapter five?.
And one of the things that you guys have been asking me most.
over the last 10 weeks of doing this series,.
the number one question I've been getting from you.
is how did you actually film the thing?.
Like, how did you actually pull it off?.
Like, what was the journey.
of actually trying to put this whole Exodus series.
and all the films together?.
And so when we were planning this four years ago,.

$^{121}$we're like, wouldn't it be great to follow Chris's intensity.
with the joy of a behind the scenes kind of episode?.
Like take people behind the scenes.
and show them how crazy it was.
to actually put this whole thing together..
Because the reality is if chapter five is hard.
and chapter six is good,.
there's a journey that you have to go through to get there..
And if there's anything I can testify.
through this whole Exodus experience.
and the films and everything we made.
is that we went through that journey ourselves..
That things got worse before they got better for us..
That the whole journey was not easy..
It wasn't fun all the time..
It was challenging..
And yet through that, we saw God do things.
that we never thought or expected He would ever do..
And so today, I wanna make things a little bit more fun..
And we're gonna actually go back to Egypt and Jordan,.
but I'm gonna take you behind the scenes..
We're gonna do this by,.
but we actually put together a Zoom call.
with four of the key members of the crew..
Toby Thomas, who was our director based in London..
Anthony Gibbs, who's our editor based in Sydney..
Riley Su, who's the producer of the whole series,.
who's based here in Hong Kong and myself..
We all got on a call together.
to talk about the craziness of this journey..
And so I wanna now take you behind the scenes with Exodus..
Let's check it out..
(upbeat music).
- Hi, I'm Andrew..
- Hey, I'm Toby..
- Hi, I'm Riley..
- I'm Anthony..
- Yeah, so I think the project was originally conceived.
back in 2018..
- I think I probably got like an Instagram message.

$^{161}$from Andrew..
- Yeah, 2018 being like,.
"Hey man, love to speak to you about something.".
And then when he'd pitched the project to me,.
I was like, "Oh, this sounds like it's gonna be.
really, really exciting.".
- Toby, myself, and one of our producers.
in the very early days, Lauren,.
got on Skype back in those days.
and read through the book of Exodus.
over about three months together..
That was sort of late 2018 into early 2019..
And that was how we began everything..
And I think soon after we finished reading that,.
we were expecting things to be boom, boom, boom,.
get the crew together, do some recces, start filming..
And of course, you know, the world fell out from itself..
- I mean, like just to give you an idea.
of how long this took, right?.
Like when this started, Toby was unmarried with no children.
and I didn't have a child..
And now we're not even finished.
and Toby's married with two kids and I have one..
I mean, that just goes to show how long this took..
- I had brown hair when we started..
So I guess the crew was firstly,.
the person who stayed on the project.
the whole time was Devin..
Secondly, there was Oliver and then Anthony..
Riley is strict on the budget..
And we also had Ben with us..
And then my role was the director..
That was the crew on the ground..
- We started off with Jordan first because of delays..
- How was lunch?.
- Oh, fantastic..
- So you'd expect from the first day of a game..
- Thanks, friends..
- And first location was the amazing Petra..
We rocked up, you know, with our camera gear.

$^{201}$and there you have it..
Oh, nope..
- We're supposed to be shooting in the Petra..
We wanted to get in super early whilst it was quiet.
and the light was good, but we have permits,.
but apparently now there's a new rule..
We're being told that we need to wait maybe 30, 45 minutes.
for a security personnel to come.
and accompany us on the shoot,.
which we probably will have to pay for,.
which is kind of part of the thing..
So a little frustrating right at the start..
- How are you feeling, Toby?.
- A bit gassy..
- And so that was our first day of shooting..
We're all pumped up and it's like, oh, okay, slow down..
Hang on, slow down..
(upbeat music).
(indistinct chatter).
- We just done the first sort of four or five takes.
for episode 30..
Yeah, I'm feeling good..
From what I can tell on the little monitor,.
the shots look incredible..
I feel like the content was strong..
So it's all coming together..
Crew have gone back to the hotel just to test everything,.
make sure we caught everything that we need..
Now we're going to get a few more B-roll shots..
Loving it..
- There'll be a few moments on the trip.
where someone would turn up,.
there'd be five to seven minutes of shouting..
I mean like guttural shouting,.
and then people would get in a car and drive away.
and then apparently it was fine..
- We were at a location that we had gotten permit for.
and our fixers have told us that,.
yes, you can film there..
And we rock up, we're in the middle of shooting.

$^{241}$and the owner of the location comes out.
and starts berating our fixers.
for allowing us to shoot there..
- We went down to start filming..
We got permits and everything, so everything's legit..
But then we got a call from the local intelligence bureau,.
this area, saying that we had to stop shooting straight away..
We came back to the car park.
and now there are Jordanian fixes.
and the security guys are having this massive argument.
and we're just praying that we'll be able to clear this up.
and get shooting again..
But yeah, it's pretty disruptive.
to suddenly have this sort of stuff happen..
- Definitely the most extreme shouting..
Definitely the most intimidating people..
- And so we got kicked out.
and we had to delete all of our footage from that location..
- I distinctly remember at one point during the Jordan.
that I was like taking the SD cards out of my camera,.
putting like another empty one in.
and like putting the other one down my sock,.
just in case, you know..
Just be like, "Oh no, I've got nothing..
I haven't shot anything at all..
What are you talking about?".
(upbeat music).
- Yeah, so Egypt is split by different kind of districts.
in some ways..
And each of the areas have their own political sensitivities.
to them..
And as you travel,.
and of course we were tracing the journey of Israel.
through Egypt..
So we had to cross quite a few of these.
sort of internal borders, if you will..
Every single time you would do that,.
you would have to go through a checkpoint..
And the checkpoints are everywhere..
- This is the fourth checkpoint of the day..

$^{281}$And one of the guys that will wait.
for the Egyptian company raising this..
I think they're all getting a bit frustrated..
You know, it's really slow going where,.
you know, you constantly have to stop at these checkpoints..
And sometimes they just kind of wait you through..
Sometimes you have to stop for an hour.
and you just never know.
till you get there of what's gonna happen..
And this one, I think they thought.
we were just gonna drive through.
and they've stopped us again..
So yeah, tensions a bit high..
- And I think, you know,.
so security in Sinai in particular,.
the Sinai Peninsula,.
that area, you know, can be quite,.
you know, again, politically sensitive..
There were certain areas where we'd have a police escort.
to go to the various places that we needed to get to..
There was this moment where we were driving down.
to Beni Hassan, which is in the South..
We're in two vans with two police escorts.
with six people with guns in both vans,.
either side of us..
And Devin is just like,.
"I am not gonna tell my wife.
this is happening right now, this moment..
She can know when I get back.".
Because he's just like, "Why is this necessary?".
So you're just kind of like watching over your shoulder,.
trying to figure out why all these people.
are escorting you there..
It's because they think Andrew's a major celebrity, I think..
- Yeah, I think that was the reason, yeah..
I think the other important thing to kind of weigh in here.
with is I think actually doing a project like this.
in Egypt itself is incredibly complex and difficult..
And it's funny, actually, Riley and I chatted about this.
early on in the project, and we did our research,.

$^{321}$like, "Who else?.
What other films are out there by churches on the Exodus?".
And it was like, we could find nothing..
I remember maybe like a year into the project,.
Riley was like, you know,.
Riley and I were having a conversation..
It was basically like, now we know.
why no one else has done this thing..
It's just incredibly complex..
Oh my God..
Yeah, so we went to the pyramids..
We show up on day one..
It was literally day one of the shoot..
Day one, we went to the pyramids, excited..
And we got all the cameras out..
We're out there, and-.
- There's no one here..
We've got the spot, let's go..
- Okay, so we show up here back in 2019..
They didn't have these flying lawnmowers.
around the pyramids, but now they're absolutely everywhere..
And when you're trying to record sound,.
it's an absolute nightmare..
So much of the last two days,.
it's been us standing ready to go,.
like, ready to shot action, and that happens..
And the noise was just like,.
like, we can't do audio with these things.
flying over our heads..
And it was just like, oh, the morale on day one.
was just like everyone looking at each other,.
feeling a bit awkward..
- And any time, like, one would sort of fly away,.
we'd be like, "Okay, go, go, go, go.".
And I'd be like, "Do my thing.".
And it'd be like, this next one would fly in..
We'd all have to stop all of a sudden..
And so it was quite stressful, yeah..
- The walk up to the top of St. Catherine's..
I've never been able to be like,.

$^{361}$"I've seen something break someone,.
"but now I can say that.".
- That's Anthony and Andrew..
So close..
There they are..
I'm falling behind..
- Weather update, it is cold..
Over there..
- He's got his hands in it, he's so cold..
He's got his hands in it..
- And I think he healed over time..
Probably when he returned back to Hong Kong,.
but he was a different man by the end of the walk up,.
let's put it that way..
- In terms of like, yeah, I'm very proud of this project..
It's been birthed through prayer and study and fellowship.
and just a deep relationship and trust for one another..
- This is what we go through to get these shots..
- Three hour drive..
Till we sleep..
And I have to put on a face mask..
- Oh, this is sick..
- For me, there were two highlights..
I think one was on the top of Mount Sinai..
We timed it a little bit wrong.
and we got up there two and a half hours.
before the sun came up..
So we had a very cold two and a half hours.
on the top of this mountain in the pitch blackness..
And we laid down and we looked up at the stars.
and the stars were just incredible..
And we saw like, I don't know how many shooting stars..
And I remember just having this prayer time,.
just myself praying with God on the top of Mount Sinai..
Like it was just, that was very,.
I think that was the most spiritually moving moment for me..
You know, when I thought about the whole project,.
the fact that we were finally there doing this dream.
that we'd had for so long.
and being on top of that mountain.

$^{401}$and feeling very connected to God..
- I think for me, the hardest bit.
was when we had to cancel Egypt.
like two days before flying out..
And I think internally I was kind of upset with God.
because I was just like, what is this?.
You know, if it's a no, we're not meant to go,.
then just tell us, then I wouldn't have to go through,.
you know, all the stress and like, you know,.
working like at three, 4 a.m. at night,.
trying to, you know, like cancel flights and tell everybody.
and then, you know, what next?.
And then like, oh, reconnecting them to Jordan instead..
And that was, yeah, that was very hard..
- It's interesting you share that, Raleigh,.
'cause that's exactly how the Israelites felt.
when they were there by the Red Sea,.
the armies coming at them..
And they said to Moses, like, why did we even bother?.
Like, it would have been better.
if we'd just not even done this kind of thing..
And I think it's really interesting that I think in a way,.
we did experience that same sort of feeling.
throughout the trip..
You know, why are we doing this?.
Like, why has God set us up to fail?.
It felt like that a few times, you know?.
And yeah, it's interesting how that,.
I think that connects into the actual Exodus story itself..
- You like that how Moses would have felt like,.
oh gosh, like, God, you better,.
you better like show up and pull through..
I've got like 2 million people here.
who's gonna like slaughter me if this doesn't work out..
- Yeah, and that's exactly how I felt on the flight over..
I'm like, not that I've got 2 million people,.
I've got like a crew of eight or whatever,.
but you know, like I'm flying over going like,.
God, you better, this better work..
So we're taking a step of faith here.

$^{441}$and that's very much what Exodus is all about..
- Yeah, that,.
(audience applauding).
so that idea of a step of faith.
is what the Exodus is all about.
and it's exactly what we see.
between chapter five and chapter six..
And so much of our journey through the filming of this,.
is it gonna happen?.
Is it gonna work?.
Is God gonna come through?.
Is exactly how Moses is feeling.
as he steps into this new chapter..
And finally God begins to speak..
And it's amazing what God does.
in the beginning of chapter six..
He gives Moses three reasons why Moses should trust him..
And no matter where you are right now.
in your relationship with God,.
no matter what low you might be facing at the moment,.
these are three things that you can take to the bank..
Three things that you always have a reason to trust God..
I wanna share these with you..
Starting in chapter six, verse one..
Then the Lord said to Moses,.
"Now you will see what I will do to Pharaoh..
"Because of my mighty hand, he will let him go..
"Because of my mighty hand,.
"he will drive them out of this country or of his country.".
The first thing Lord does to try to remind Moses.
about how he can trust him.
is he speaks about his mighty hand..
Now this phrase, a mighty hand,.
becomes a common phrase.
throughout the rest of the Old Testament narrative.
from this point from here..
But its roots are in the Exodus story..
It's mentioned twice in this passage alone,.
but it's mentioned multiple times.
throughout the rest of the book of Exodus..

$^{481}$And when it's mentioned, it's talking about two things..
It's talking about the power that God has.
to fight battles that we are not able to fight..
That's the first thing that's meant..
When God says, "My hand is mighty,.
"my hand is working in a mighty way.".
He's saying, "I'm fighting battles on your behalf.
"that you cannot fight.".
The second thing it means is that,.
"I am at work influencing human affairs.
"in a way that you could not influence or do yourself.".
So these are sovereign responses of God..
His mighty hand fights battles we cannot fight.
and changes human hearts and changes human affairs.
in ways that we could never do..
And God is wanting to speak to Moses and saying,.
"Right now, it seems like it's getting worse.
"before it gets better..
"Right now, it seems like things are really hard,.
"but you gotta remember, my mighty hand is on your behalf.".
He doesn't say to Moses, "Trust in your mighty hand.".
He doesn't say, "Trust in your power,.
"trust in your ability to influence Pharaoh,.
"trust in your speaking ability,.
"trust in all these things.".
He doesn't say that, he says, "Trust in me..
"Trust in who you know I am..
"I'm a mighty God with a mighty hand..
"I'm fighting battles and I can influence world affairs.
"like you never could.".
And some of you here, that's the call for you,.
to trust in a God who is able to do what you cannot do.
and stop fighting in your own strength.
because that only ever leaves the burnout.
and actually open your heart to the fact.
that God is fighting on your behalf..
He's got a mighty hand..
And I love what God says right at the beginning.
of this verse, he says, "Now you will see my mighty hand.".
In other words, he's not asking Moses.

$^{521}$to believe in blind faith..
He's not just saying, "Believe in my mighty power,.
"even though you're never gonna see it.".
He's like, "Now you will see my might and my power at work..
"I'm about to change Pharaoh's heart..
"His heart is hardened, but I'm gonna soften it..
"I'm gonna cause him to let you go..
"You're gonna see the power of my hand at work..
"I'm gonna fight battles for you.
"throughout the whole journey..
"It's about to take place..
"Now you will see.".
And I think this is really important.
because so often if we're in a low time,.
if we're in a valley,.
if it seems like things are getting worse.
before they get better, it's so easy for us,.
so quick for us to forget the powerful things.
God has done in the past..
And I believe for some of you in this room,.
that's exactly what God wants to be doing..
He wants you to have that faith again to say,.
"Now I'm going to see the power of God.".
Something changes in us in terms of our faith.
when we see the power of God at work..
I remember a few years ago,.
I was on a KMB bus in Kowloon.
and I'm traveling through Kowloon somewhere.
and I'm on this bus and this is kind of like.
a pastor's thing, but every once in a while.
when you're a pastor and you get into a place.
where nobody knows you,.
you kind of just breathe a sigh of relief..
It's like, I don't need to be a pastor now.
for like the next five minutes.
and I can enjoy my bus ride in Kowloon.
where no one knows me..
It's awesome..
So I'm on this bus, I'm relaxed, I'm chilled..
I kid you not, after about five minutes of going,.

$^{561}$this guy about four seats away from me.
starts to manifest demonically..
Now I've been involved and I've been around demonic work.
and I've been involved in deliverance of people.
that have demonic spirits over my pastoral years.
and so I knew exactly what was happening..
I knew that this was what was taking place in this person.
and it was so funny 'cause my immediate reaction was,.
"Is there any other Christian on this bus.
"who can step up right now.
"'cause I'm on a short holiday right now?".
You know what I mean?.
And I heard God say something to me straight away..
He said, "Andrew, do you wanna see my power?".
It's a great question..
'Cause you'd think immediately we'd say yes, right?.
But do you really wanna see my power?.
And I said, "Yeah, God, I wanna see your power.".
He's like, "Well, pray for him then..
"Deliver a demon and you'll see my power.".
And so mustering up all of the faith that I have,.
which wasn't very much at that time,.
I said out loud, I said, "In Jesus' name, be gone.".
And immediately he fell asleep..
Just fell asleep..
Just totally like passed out, fell asleep..
And I was like, "Whoa, that was cool.".
I just saw the power of God..
I'm like, "Is that gonna work again.
"next time I'm arguing with my wife?.
"In the name of Jesus!".
(congregation laughing).
(laughing).
David's like, "Teach me, teach me.".
(congregation laughing).
God says to Moses, "Now you will see.".
And I think sometimes there's a faith needed.
in the church around the world to say, "God, I wanna see.".
Would you move in power like you've never moved before?.
We need it now, Lord..

$^{601}$Would you come and would you show yourself faithful?.
So God says, "You can trust me.
"'cause I'm a mighty hand, I'm at work.".
Here's the second thing he says in verse two..
God also said to Moses, "I am the Lord.".
This is the word Yahweh..
"I appeared to Abraham, Isaac, and to Jacob as God Almighty,.
"but by my name, the Lord,.
"I did not make myself known to them.".
So it's an amazing thing what he says here,.
even though I can't read it..
What he's saying here is there's two names.
that you have to understand that I've revealed myself..
There's a name that I revealed myself.
to the generations in the past,.
and he says that's the name Lord Almighty..
And there's a name now that I'm generating.
and that I'm giving to you, which is Yahweh..
Now, here's what you need to understand..
The name that he had given to the generations of the past,.
to Abraham, Isaac, and Jacob was El Shaddai..
It means the God of might or the God of power..
So when he first reveals and he calls Abraham.
to come out and to start the new nation of Israel,.
he reveals himself as El Shaddai,.
the God of power, the God of might..
But when he shows up and reveals himself to Moses.
at the beginning of the Exodus at the burning bush,.
he doesn't use that name, he uses a new name..
The name is Yahweh, which is translated.
in the Old Testament Lord in capital letters,.
but it essentially means I am who I am,.
or he is who he is, or he will be what he will be..
Or more specifically, it means he will be with us..
So I want you to see something powerful that God does..
And this is the second reason why he says to Moses.
that he can trust him..
He says, it's because you need to understand.
that I'm revealing something about myself.
that I had not revealed in history past..

$^{641}$With your generations in the past,.
I reveal myself mighty and powerful,.
but for you, I'm revealing myself personally,.
intimately, the one who is with you..
In fact, this is what we've seen of God.
throughout the whole of the Exodus journey so far..
I have seen the misery of my people.
and I've come down to be with them..
I have compassion because I see their pain.
and their suffering, I wanna be here with them..
And so it's fascinating to me that God reveals himself.
differently at different seasons and different times..
And note this, 'cause I think God got it.
around the wrong way..
When you see the El Shaddai at the beginning,.
God of power, and the Yahweh, the personal,.
shouldn't it have been the other way around?.
Like when God first reveals himself to Abraham,.
you would think he would reveal himself personally.
and intimately with him..
Hey, it's us, it's me, I'm your God..
Let me tell you a little bit about what it is to worship me..
Let's journey together..
You know, that's the establishing of relationship..
You think God would reveal himself personally..
And when it comes to the Exodus,.
wouldn't you think that God would reveal himself mighty.
and powerful and like awesome and stuff?.
But he doesn't, he does it the other way around..
With Abraham and Isaac, it's power and might..
With Moses and Israel, in the moment of their greatest need,.
it's intimacy, care, and love..
And the question you should be asking yourself is why?.
The answer is this, the greatest amount of God's power.
that you will ever experience in your life.
will come through his intimacy and his care for you..
God breaks chains over you.
by deepening his relationship with you..
So his personal and intimate relationship with us,.
his desire to draw close.

$^{681}$so that we can know him personally and intimately,.
that is always going to be the place.
where you will find your greatest freedom..
You will find your greatest breakthroughs,.
your greatest freedoms,.
your greatest walking out of slavery.
when you know the intimacy of a God,.
when you have that intimate, personal,.
connected relationship with him..
If you only have the God of power,.
you might break some chains,.
but you will fall back into those chains again..
The thing that sustains you,.
the thing that's gonna enable you.
to walk out of those chains forever and move into freedom.
will be because you know there's a God.
who walks with you intimately and purposely.
every step of the way..
His care, his love, his intimacy with you.
is the defining thing that won't just break the chains,.
but will keep them off you..
Are you with me?.
So you can trust me, God says to Moses,.
because I am this mighty and powerful God.
and I'm in fighting battles.
and I'm involved in the hearts of man,.
but you can also trust me because I'm with you..
I know you and I love you and I'm caring for you..
And both of those two things are a reality at the same time.
and your freedom needs both of those things..
Here's the third thing he does in verse four..
He says, "I also established my covenant with them.
to give them the land of Canaan.
where they had lived as aliens..
Moreover, I have heard the groaning of the Israelites.
whom the Egyptians are enslaving.
and I have remembered my covenant.".
The third reason that he gives.
for the reason why Moses can now trust him.
is 'cause of the covenant..

$^{721}$And God has mentioned his covenant.
every step of the way of the Exodus so far to Moses.
'cause it's that important to God..
His covenant is his contract with his people.
that he will deliver them,.
that he will bring the promises.
that he has in his heart for him..
His covenant is the established reality.
that I have taken you, I have chosen you,.
you are my people..
And because I have said my word, I never break my word..
You can trust me because I'm covenanted to you..
I'm connected with you..
I have a contract with you..
I've chosen you..
I've done that in my own free will,.
not because you've earned it, not because you deserve it,.
but because I desire to be in relationship with you..
And because I have covenanted,.
I've contractually committed myself to you..
No matter what you do,.
it will not break the power of the coming of the promise.
that I have for you..
You can trust me, God says, because I stand on my word..
You can trust me 'cause I never forget my contracts..
You can trust me because unlike so many of us,.
you can take my word to the bank..
If I've said it, if I've declared it,.
it will come to pass, says the Lord Almighty..
So these three things.
is what God is trying to communicate to Moses..
You can trust me because I am mighty before you..
I am personal with you and I'm covenanted to you..
And those three things are what you can build your life on..
You can trust God,.
no matter what circumstance situation.
is going on in your life,.
because he is mighty and powerful to save,.
because he is personal and intimate with his people,.
because he has placed a covenant over us..

$^{761}$In the Old Testament, in Exodus,.
a covenant written on stone..
In the New Testament through Jesus,.
a covenant given by the blood of Christ..
Those things have never changed..
These three things, which are started and founded here.
at the beginning of chapter six,.
become the complete foundation.
of our relationship with God..
It is the foundation of our Christian faith.
that he is mighty, personal, and committed..
Are you with me still?.
Now, off the back of all three of those,.
I want you to see what then God says..
He said, "Because you can trust me for these three things,.
here are seven things I'm gonna do for you.".
And I think these seven things he does.
for Moses in Israel are seven things.
he will always do for you..
Let me show you this really quick..
Verse six, "Therefore say to the Israelites,.
I am the Lord," there it is, Yahweh, this personal God,.
"I will bring you up out from under the yoke.
of the Egyptians..
I will free you from being slaves to them..
I will redeem you with an outstretched arm.
and with mighty acts of judgment..
I will take you as my own people..
I will be your God..
Then you will know that I am the Lord God.
who brought you out from under the yoke of the Egyptians..
I will bring you to the land that I swore.
with uplifted hand to give to Abraham, to Isaac, and to Jacob..
I will give it to you as a possession,.
for I am the Lord.".
Seven I will do statements here..
Let me show you the seven here..
I will bring you out, I will free you,.
I will redeem you, I will take you,.
I will be your God, I will bring you, I will give it to you..

$^{801}$These are the seven things that he says,.
"Because I have a foundation of my mighty power,.
my personal intimacy, and my covenant with you,.
I will do these seven things for you, Moses.".
And I say, he's doing these seven things.
for the vine church, for you,.
and whatever journey of Exodus you're in,.
he will do these things for you..
Now, what's beautiful is when you read this in the Hebrew,.
rather than the English, when you read it in English,.
it just looks like a random bit of a list..
When you read it in the Hebrew,.
you can see that poetically the writer here.
is trying to group these under four different phases.
of who God is..
And I wanna show you this really quick..
The first is this idea of take..
God declares to Moses, "I will take you..
I will take you and be your God, because that's who I am..
I've chosen you of my own free will..
And because I will take you, here's what I will do for you..
The second thing, I will bless you..
If I take you to be mine, then my promises, my character,.
my love will be yours..
In taking you to be part of my family, in being your God,.
I will then give to you these promises.".
And it's capturing this idea of I will bring you in,.
as he says in this passage..
In other words, there is a time in the future.
where you will be in my promises..
The things that I have for you.
will always be a blessing for you..
I bless the things that I have chosen and taken..
Are you with me?.
So I will take you so that I would ultimately bless you..
Now, however, I cannot bless you until I, third,.
break some of the things that are holding you enslaved..
So this idea of breaking,.
or I will bring you out of your slavery,.
I will free you from the bonds of your slavery,.

$^{841}$I will redeem you or save you..
These are the ideas of God breaking the bonds.
that are holding Israel out of the promises.
that God had already declared for them..
So I'm gonna take you because I have a promise.
that I wanna bless you with,.
but to get you there,.
I gotta break some of that stuff off of you..
Once it's broken, then I can give you,.
that's the final phase,.
I can give you what it is that I long to give you..
And so this idea of I will actually give you,.
as it said here..
So those seven things actually become a part.
of a four-stage process.
that God is declaring to Israel in this moment.
and saying, "The rest of the Exodus is about this..
I'm gonna take you, I'm gonna pull my blessings on you,.
I'm gonna break off Pharaoh's hold from you,.
and then eventually I will give you the promised land.
that you'll move in.".
Are you with me still?.
This idea of taking, blessing, breaking, and giving.
becomes the foundational thinking for Jewish people.
when it comes to their understanding of God..
And it's based here in Exodus chapter six,.
the mountain after the valley of Exodus chapter five..
But it's not the only time in history.
when this concept becomes so important for God's people..
In fact, as I draw to a close,.
I wanna share with you one final passage.
that happens in the New Testament..
This is a moment where Jesus,.
just literally the night before He's about to be betrayed,.
He comes before His disciples.
and they find an upper room to celebrate Passover..
Passover, the meal that was celebrated.
and is still celebrated today.
by Jewish people around the world,.
that recognizes the power of God.

$^{881}$to bring people out of their slavery..
And so in this moment,.
Jesus is celebrating a Jewish festival with His disciples.
that's linked to the Exodus..
Does that make sense to you?.
Are you with me?.
Everybody still okay?.
Okay, you need to follow this..
So He's sitting in this upper room.
and He's doing a Jewish festival with His disciples..
But for the first time in history,.
Jesus will now embody Himself into the Exodus narrative..
He will now say that that whole journey.
that took place in Exodus was actually about me.
and the journey I'm about to bring you on..
I wanna show you this in what He says, verse 14 onwards..
This is Luke 22, verse 14..
When the hour had come,.
Jesus and His apostles reclined at the table..
And He said to them,.
"I've eagerly desired to eat this Passover with you.
before I suffer..
For I tell you that I will not eat of it again.
until it finds fulfillment in the kingdom of God..
After taking the cup,.
He gave thanks and said,.
"Take this and divide it among you..
For I tell you that I will not drink again.
of the fruit of the vine until the kingdom of God comes.".
Verse 19, He took the bread, gave thanks and broke it.
and gave it to them and said this,.
"This is my body given for you..
Do this in remembrance of me.".
In the same way, after the supper,.
He took the cup saying,.
"This cup is the new covenant in my blood,.
which is poured out for you.".
I want you to hear Exodus in this..
It's the Passover celebration..
And Jesus is talking about a new covenant.

$^{921}$He's going to bring, linking to Exodus chapter six..
But I want you to see where the link is most powerful..
Take a look at verse 19..
"And He took bread, He blessed it,.
He broke it and He gave it to them.".
Isn't that crazy?.
Jesus, in deciding to create.
what we now commonly call communion,.
what the church has been practicing for over 2000 years,.
Jesus begins that practice by literally doing.
the very four things that is rooted.
in chapter six of Exodus..
He takes the bread, He blesses it,.
He breaks it and He gives it,.
creating a tradition that we now do.
that not only aligns ourselves to Jesus,.
but aligns herself to the God of the Exodus,.
the one who is mighty, personal and covenanted to us..
And this is exactly what happens for Jesus Himself..
God has taken Christ Jesus..
He has blessed Him with His life on earth..
His body is broken on the cross.
so that that body broken and the blood shed.
would give those who would believe in Him eternal life..
In the same way, Christ, through His relationship with you,.
He has taken you from your bondage of slavery..
He has blessed you with His presence and with the gospel..
He has broken off the slavery of sin in your life.
and He has given you eternal life..
Take, bless, give..
I got it wrong..
Take, bless, break, give.
is the process that God has for His people..
That's the God of the Exodus, my friends..
Whatever it is that you're going through right now,.
those four things God does for you.
and you can trust Him 'cause He has the might to do it,.
He's personally with you, He's covenanted to you.
and He will take you, He will bless you,.
He will break off the things that need to be broken off.

$^{961}$so that He can give you all the promises.
that He's longed to give you, amen?.
\newpage



\section{}
\label{sec:2hwTmUlFH_A}
\textbf{2023-07-03 Conversations of Love [2hwTmUlFH-A].mp3}
\newline
\newline
連結: \href{https://youtube.com/watch?v=2hwTmUlFH-A}{\texttt{ https://youtube.com/watch?v=2hwTmUlFH-A}} ~~~~ 語音日期: 2023-07-03 
\newline
\newline
\hyperref[sec:d2vwib6oxcU]{\small{< < < PREV SERMON < < <}}
~
\hyperref[sec:index]{\small{[返主目錄]}}
~
\hyperref[sec:OCz8LrBOC28]{\small{> > > NEXT SERMON > > >}}
\newline
\newline
$^{1}$Everyone says, "Amen." Hey, can we thank our worship team as always?.
Why don't you have a seat? Have a seat..
[Music].
Right. Half of you in here know exactly what that's all about..
The other half of you have no idea what's going on..
I'm in that second group who has no idea what's going on..
We're starting an amazing new series here called Blowing Water..
It's a series that's been birthed and shaped by Pastor Ellison.
as he's prayed over us as a church and wondered what the summer is going to be like..
We're so excited that he's here today to open up the series for us..
Can we put our hands together as we welcome Ellison?.
Hi..
Hi, bud..
Let's stretch out our hands and let's pray..
Father, we thank you for Ellison, for Brittany, for Isaiah, for Malachi,.
for this family that you've blessed this church with for many years now..
Father, thank you for just Ellison's heart, his passion to communicate the gospel.
in ways that are challenging and yet hopeful,.
and for the way in which you've put this series on his heart for the summer.
where we look at the conversations that Jesus had with lots of different types of people..
Father, as he opens up this word, Lord, we pray for that fertile soil in our hearts,.
that we would hear your word, Lord, that you'd speak to us and shape us and mold us.
into the people that you desire us to be..
Father, we thank you so much for this, and we pray this in Jesus' name..
Everyone says amen..
Amen. Thank you, Andrew. Thank you, team..
Good morning, everybody..
Good morning to those of you joining us online..
So, as Andrew said, over the past couple of months, we've been walking through Exodus as a journey..
We've been visiting Egypt through film, seeing all the sites,.
and learned about how God has powerfully been moving his people in his rescue plan,.
out of slavery into freedom..
If you've missed some of those messages, I encourage you to go back to YouTube,.
look them up, get caught up so you can start back together with us.
when we start back later this summer..
But for the coming weeks, as Andrew said, we're going to be entering into a new series..
So we're taking a break from Exodus, so you can think of it as like Exodus season one is over..
There's a little hiatus. Season two is coming soon..
But today, we're stepping into our new summer sermon series..
We're calling this series Blowing Water, as you see in Chou Sui,.

$^{41}$probably the most Hong Kong title you could give a sermon series..
But let me explain to you why..
Blowing Water is a direct translation for the Cantonese slang phrase, "chou sui.".
To "chou sui" or to "blow water" means to have a chit-chat or conversation with someone..
For example, this morning, as I was stopped by at 7-Eleven, I got caught..
I was late to work because the guy was trying to talk to me about my YUU points..
And we got on this whole thing about, you know,.
"Li koi yi yong chun ba da tong," and all this kind of stuff..
Link your YUU account to your Octopus card, and you don't have to put out your....
It was like a 10-minute conversation about YUU..
But yeah, that was a conversation..
Or maybe when you ride in a taxi or go to the market,.
whoever it is that you encounter, if you stop for a minute to have a chat,.
you can say, "Lei tong ga dei chui ha sui," right?.
Imagine like saliva being blown at your mouth..
It's kind of gross, I know, okay?.
I think that's the idea..
But a little while ago, I got a random message from a unknown number..
We've all been getting these a lot recently, right?.
And the way this person started the conversation was,.
"Hi, hai wo, yong mo shi gan chui ha sui?".
"Hey, it's me. Do you have time for a chat?".
Now, my curiosity peaked..
Because most normal people, most normal people would have said to themselves,.
"Don't know this number, right? Spam, report, block, move on," okay?.
Most normal people..
I, however, was on a long bus ride home..
I had all the time in the world, right?.
So I decided to engage..
I decided to reply to see where this conversation would go..
"Oh, hi, sorry. Who are you?".
And then the guy replied, "Oh, it's me, Walson. Aunty Chan's son.".
"Oh, yeah, I remember, Walson. Yeah, yeah, yeah. How are you?".
Right, and so for the next 45 minutes, actually, in fact, for the next couple of weeks,.
it started to get a bit weird, so I stopped, okay?.
But I pretended to be Jenny, okay?.
A middle-aged sales lady who recently had her heart broken.
because my pet cat just died..
My boyfriend broke up with me. I didn't know what to do with my life, right?.
And it turned into this really fascinating conversation, okay?.

$^{81}$If you look through messages, it's weird, okay?.
But he was trying to be kind this whole time..
He was trying to be comforting, a little bit creepy..
But I knew where this was heading, right?.
This was heading towards him asking me for money, right?.
This was a scammer, right?.
They're trying to use this conversation to build trust,.
to build a relationship..
In the end, they probably asked for money or credit card, something like that..
Now, if you look in the news lately, right,.
so many stories about people being duped or tricked, swindled, and convinced.
out of large amounts of money precisely because of conversations.
like the one I had with Walson, right?.
Over \$1 billion have been lost through scams, apparently..
And make me think, why?.
Well, I think it's because conversations are powerful, right?.
We crave connection through conversation..
In fact, I would say conversations are essential in how we build relationships.
with each other..
Conversations have the ability to stir our emotions, to comfort us,.
to make us weep, to make us laugh, to motivate us to do good or to do bad..
And this is why these scams work so well, simply because these people.
are good at conversing, right?.
They're so compelling..
Think back to some conversations you might have had..
Maybe sadly you've been a victim of one of these scams and you've lost money.
and lost other resources because of this..
Maybe it's a tough conversation you had with your boss..
Maybe it was bringing bad news to someone that you loved..
Or maybe it's staying on the phone with your crush till 3 AM saying,.
"You hang up. No, you hang up. No, you hang up.".
I'm looking at you, Bernice..
My conversation is this, right?.
The right conversation can leave a deep impact and maybe even change.
our lives altogether..
Right?.
One of the most memorable, impactful conversations that comes to mind.
was when I called my wife, Brittany, when I called her mom to ask.
for her permission to ask if I could propose to her daughter, right?.
It was through the phone because she lives in the US, but my hands were shaky,.

$^{121}$my palms were setting..
I remember stumbling over my words, and in the end, she gave me the blessing..
Now I think about it really quickly for some reason, okay?.
But that conversation has had an impact on my life for the rest of my life..
So this made me think also, "Well, who is it that's the master of conversation?.
Who left people wondering where did this person receive their power, wisdom,.
and authority from every time after talking to them?.
Who was the person that could literally set people free from the suffering.
with just the power of his words?".
This, of course, is Jesus..
And so over the coming weeks, we're going to be looking at conversations.
Jesus had with different people..
And as we look at these conversations, hopefully what we will find is some.
new and fresh perspectives..
We draw out truths and insights you may not have noticed before..
And I hope that with each conversation we look at, we'll also hear Jesus speak.
to our doubts, our fears, our failings, our sorrows, our setbacks,.
our insecurities, and challenge us to live a life more in line with his purposes..
At the same time, by examining these dialogues, I hope that we can learn.
to engage in authentic life-altering conversations with those around us.
that lead people and point people to where Jesus is..
Because one thing's for sure, everyone who spoke with Jesus left forever changed..
And so when we too are motivated by the love of Jesus, the power of God,.
the inspiration of the Holy Spirit, our conversations too can have the same.
impact as those around us..
So we're going to start in John 8..
It says this, "But Jesus went to the Mount of Olives..
At dawn he appeared again in the temple courts where all the people gathered.
around him, and he sat down to teach them..
The teachers of the law and the Pharisees brought in a woman caught in adultery..
They made her stand before the group and said to Jesus, 'Teacher,.
this woman was caught in the act of adultery..
In the law, Moses commanded us to stone such woman..
Now what do you say?'".
They were using this as a trap in order to have a basis for accusing him..
"But Jesus went down and started to write on the ground with his finger..
When they kept on questioning him, he straightened up and said to them,.
'Let one who is without sin be the first to throw the stone at her.'.
Again he stooped down and wrote on the ground, 'And at this,.
those who heard began to walk away one at a time,.

$^{161}$the older ones first until only Jesus was left.'.
With the woman still standing there, Jesus straightened up and asked her,.
'Woman, where are they? Has no one condemned you?'.
'No one, sir,' she said. 'Then neither do I condemn you,' Jesus declared..
'Go now and leave your life of sin.'".
This is a very well-known story in the New Testament..
However, if you read it in your Bible, you might see there's a little side note..
The problem is that in some early manuscripts of what we have as the Scriptures right now,.
this story isn't present in there..
Scholars think it's a bit out of place. It doesn't fit in with the context before and after..
Without going into too much detail, though, the general consensus is that.
even if those problems exist, we can still trust that it's an accurate account,.
reliable enough for us to study and to preach on..
And if you really want to get into asking more questions about the scholarliness of the Ovilette,.
send me an email. I'll forward it to Andrew, and he can tell you all about it, okay?.
So let's go back to the passage and take a deeper look..
Verse 2, "At dawn, He appeared again in the temple court,.
where all the people were gathered around Him, and He sat down to teach them.".
And often when we think about Jesus' ministry, we think about the things He did..
The miracles, walking on water, feeding the 5,000, healing the sick, turning water into wine..
There's a lot going on in the life and activity of Jesus..
But when we look at the book of John specifically,.
the focus isn't so much on Jesus' actions and His activities, but rather it's Jesus' words..
Right? Not surprising, since the book of John starts by describing Jesus as "the Word becoming flesh.".
So here we have Jesus sitting in the court, using His words to teach a group of people around Him,.
a common occurrence through the Gospels..
It tells us all the people were gathered around Him..
The fact is that Jesus was attracting a crowd of all sorts of people, maybe a bit like this right now..
This is important to take a note of, because it meant that Jesus' popularity was rising,.
much to the anger, of course, of the religious authorities at the time..
In fact, by this time in His ministry, there were already people trying to kill Jesus.
because of the things He was saying and teaching..
He was saying things like, "I am the bread of life that came down from heaven..
He who comes to me will never go hungry, and he who believes in me will never be thirsty.".
So to them, this was blasphemy..
How dare Jesus say words like this?.
How dare He claim Himself to be God?.
Jesus, however, continued to meet and teach in public in order to declare the truth of who He was,.
who His Father was, what He came here to do, despite the threats that He was receiving..
One thing that hopefully you will notice throughout the series is that the reason why Jesus was so impactful.

$^{201}$in His conversations He has with people is because He is always fully confident.
and fully obedient in His calling and His identity..
He never wavers when He's challenged..
He doesn't back down even if people are threatening Him, even if they're threatening to kill Him..
At the same time, though, He isn't arrogant or self-righteous..
He simply stands firm on the foundation of truth..
So this made me think, "What about the conversations I have with people about Jesus?.
Am I always able to stand firm in my belief of Him as my Savior?".
It's hard to do sometimes because talking about Jesus and the things He did sounds weird, right?.
Some of the things He did are almost--I mean, they sound unbelievable..
And so either I avoid talking about Him because I don't want to face those sort of difficult and weird conversations,.
or when I get challenged about it, I start to be shy and I start to waver, change the topic, or something like that..
Ask yourself this question this morning..
Is my life, are my convictions built upon the foundation of Jesus?.
Because if they're not, then every time we're challenged about Him,.
every time we are asked a difficult question, we might buckle..
We might sway to the left or to the right..
We may give in to people's fears, and that never leaves much room for a good conversation..
Church, the first challenge this morning is to be confident in our faith.
because if not, our conversations will never display a living faith..
We cannot communicate something that's not first inside of us..
Therefore, if we want to have conversations about Jesus, we must be strongly rooted in our love and our belief of Him..
So Jesus is there..
He's teaching the crowds when He is dramatically and rudely interrupted..
It says, "The teachers of the law and the Pharisees brought in a woman caught in adultery..
We had her stand before the group and said to Jesus, 'Teacher, this woman was caught in the act of adultery..
In the law, Moses commanded us to stone such woman..
Now, what do you say?'".
Being interrupted is really annoying, right?.
If I was Jesus, I'd be mad..
Being interrupted, it's just--it's frustrating, it's annoying..
Maybe if you have young children or toddlers, you understand what I'm saying, right?.
My son, Isaiah, sweetest kid, will not shut up, okay?.
Especially when I'm doing something, okay?.
I could be in the middle of doing nothing, sitting down, you know, minding my own business..
The moment I try to talk on the phone or I want to talk with Brittany, he starts bombarding me with questions, right?.
"Hey, dada, can elephant roll down hills?".
"Hey, dada, can you build a long train track with me?".
"Hey, dada, hey, dada, hey, dada, why didn't you talk to me two minutes ago?".
Anyway, so here's Jesus teaching when He's suddenly interrupted..

$^{241}$What's His reaction going to be?.
I think the way Jesus handles Himself here proves some good groundwork for the way we can approach situations like this in our conversations..
As we're about to discover, even though the situation was loud and tense and complicated, very shocking,.
Jesus stays calm and confident..
They bring this woman caught in adultery..
They make her stand before the group..
Now, let's take a sidestep for a moment and just put yourself in the position of this woman..
Imagine what she must have been feeling..
Right, firstly, the fact that she was caught in the act of adultery would probably be embarrassing enough..
Then she's forcibly brought before a group of strangers, publicly shamed with a sin laid bare for everyone to know about what she just did..
It would have been terrifying..
Imagine that happening to you right now..
When I read this passage, though, it reminded me of another time when Jesus was teaching and He was interrupted..
This is almost the same situation but with some big differences..
A group of men drag a friend in front of Jesus..
Jesus is teaching in this house when suddenly the roof is ripped open and four friends lower a paralytic man down before the feet of Jesus..
And he's interrupted again in the middle of His teaching..
But in this situation, Jesus notices the faith of these men..
And He heals the paralytic man so he could walk again..
The motivations of these men was full of hope and full of faith..
Right, they knew of Jesus' reputation as someone who was kind and compassionate and forgiving..
And they thought, "If only we bring our friend before Jesus, then He'll be able to heal him.".
But in this situation, the circumstances are almost like same, same but different, right?.
The Pharisees may not have brought this woman, right?.
She might not have been physically sick, but she needed help nonetheless..
But their motivations in bringing her to Jesus were not to seek for her healing, for her forgiveness..
It's clear that these men were out for blood..
I mean, they already announced a death sentence over her..
They were not interested in redemption..
They were looking at an opportunity for condemnation..
But what's even more than this is that actually John tells us this situation was a double-edged sword.
because the Pharisees were really just using this woman to get to Jesus..
I mean, if this was really about the law and about being zealous about punishing this woman,.
this was something that could actually have been probably handled in private, right?.
They could have brought her in without anybody else around and just dealt with it themselves..
If this was really addressing about the sin of adultery, then where was the man in this situation?.
Shouldn't he have been brought before the people as well?.
It takes two people to do something like this..
And so the fact that the man is not presently doing this publicly.
is just further proof that this whole scene was simply a setup..

$^{281}$The woman was something they were willing to use and to abuse in order to catch Jesus..
It tells us this very clearly..
They were using this question as a trap in order to have a basis for accusing him..
Well, accuse him of what?.
Well, from their perspective, they have Jesus perfectly between a rock and a hard place..
It's because they know of Jesus'--like we said just now,.
he has a reputation for being kind and compassionate and forgiving..
They were wanting to use this against him..
They're probably thinking Jesus was going to say something like, "Hey, fellas, okay, okay, come on now..
You know what I'm about. I'm about peace and love and forgiveness..
Why don't we just let her go? Let her be free.".
If Jesus had said something like that, then they would have been able to accuse him.
of directly going against God's law, calling him a blasphemer,.
and that he too should be punished with death as well..
On the other hand, if Jesus had said something like, "Yes, you guys are right..
This is what the law says. She is condemned to death..
Come on, guys, let's go and stone her,".
then his reputation for being kind and loving and compassionate would have been ruined..
And in addition to this, the Pharisees could then accuse him of revolting against the Roman government.
because technically speaking, they were the only ones that were able to hand out death sentences to people..
But these religious leaders have got it twisted..
They've missed the point..
They've taken the ideas of justice and mercy and pitched them against each other as enemies..
And ironically, they failed to realize that the true definition of justice and mercy.
was standing right before them..
These religious leaders were carrying the attitude of what Jesus was talking about in Matthew 23,.
where they shut the door of the kingdom of heaven in people's faces..
They've neglected the more important matters of the law, justice, mercy, and faithfulness..
And church, this is a reminder this morning for us too..
The story of the woman probes our own reflexes towards people who do not fit in into our expectations..
I know, in fact, that there's probably a number of people here who've been through what this woman is going through..
Maybe not with adultery or something like that, right?.
But what I mean is that during a time of struggling,.
during a time when you were wrestling with your sin and your suffering and your shame,.
and the reaction you got when people found out was not to help you to heal,.
but rather it was shame and condemnation..
Maybe that's even happened to you here in this community at the Vine..
And if that's the case, it should break our hearts and it should lead us to do better..
And church, when we approach those who seem to be lost in sin,.
when we have conversations with those who don't fit our expectations,.

$^{321}$who are struggling with darkness,.
when we're trying to guide people in the journey of exodus out of their slavery into freedom,.
what does our language sound like?.
Are we judgmental? Are we self-righteous?.
Are we far too excited when we get an opportunity to point out the speck in someone else's eye.
without looking at the log that is in our own eye?.
Because if so, conversations we have with people in this space are never going to be effective..
So how does Jesus react?.
Well, in true Jesus fashion, it wasn't like anybody expected Him to..
John continues to tell us this..
As they were questioning Him and questioning Him and questioning Him,.
Jesus bent down and started to write on the ground with His finger..
Now, I like to think of myself as a good listener..
Brittany, my wife, however, would probably tell you I'm not..
One of the reasons she doesn't think I'm a good listener is because it never appears as though I'm listening..
She could be in the middle of telling me about something, weekend plans,.
something that happened at work, meal prepping, whatever..
I don't really listen to what she's saying, right?.
But, okay, in the middle of her talking to me, I will often randomly get up and try to go and do something else, right?.
Like maybe clean up a mess with the toys or fix the bookshelf or something like that, okay?.
And because I'm doing something else at the same time, it seems like I'm not listening..
I reassure, yes, I am actually listening..
It's just that my mind gets easily distracted and it helps if I'm doing something else whilst listening to her at the same time..
And it seems as though this is what's happening here..
They're berating Jesus with questions and it doesn't seem like Jesus is listening at all..
Instead, he starts doing something completely random..
They kept on questioning him..
"Jesus, hello, did you hear us? What are you doing? What do you say? What do you say about this situation?".
Jesus is unfazed..
He stays chill. He stays calm..
He concentrates on what he's writing on the ground..
And we don't really know. If you read commentaries, there's going to be a bunch of theories..
What was he actually writing on the ground?.
We don't know, right?.
But the effect it had was providing a pause in a tense moment..
Jesus stays calm. He pauses. He thinks. And then he speaks..
Right? In the middle of being berated by questions, being attacked, his reaction is not to shout or to yell, to try and put out fire with fire..
I know often this is my own reaction..
When the volume gets loud, you try to match it..
When tempers get flared, you feel like you have to stick up for yourself..

$^{361}$Many times this happens perhaps in less proud moments of parenting,.
maybe in a heated conversation between you and your spouse or your partner, your parents or work colleagues..
But think about it this way..
When was the last time a shouting match led to a good conversation for you?.
When was the last time someone opened up to you after you yelled at them?.
Jesus demonstrates a different model for us to follow..
He takes on the wisdom of Proverbs..
A fool gives full vent to a spirit, but a wise man or woman quietly holds it back..
Church, if we really want to be able to dialogue with each other in this space,.
we must learn to be able to hold our tongue, to keep cool, even when we might be bombarded with accusations and questions..
Pausing, thinking, and then speaking could be a fundamental way for us to prevent our conversations from going sideways..
And we learn to speak from a place of calm and peace..
That might mean practicing a bit of deep breathing..
In for four, hold for seven, out for eight..
That might mean asking yourself, you know, "How would I like to be talked to if I was in this situation?".
Sometimes this might even be walking away to clear your head, right, before coming back to the conversation..
Whatever it is, find the thing that works for you to pause, to think, and then speak..
Jesus does this perfectly..
And as a result, he gives the perfect answer to this situation..
Listen to what he says..
"When they kept on questioning him, he straightened up and said, 'Let any of you who is without sin be the first to throw a stone at her.'".
Again, he stooped down and kept writing on the ground..
If there was a list of top ten mic drop moments in Scripture, right, this would probably be one of them..
The perfect answer to this situation..
Why?.
Well, remember, this whole thing was actually a setup in order to catch Jesus, try and catch Jesus saying the wrong thing..
But this was the perfect answer because, first of all, it didn't deny the law, right?.
He doesn't say, "Don't throw the stone at her," but at the same time, he doesn't say, "What the woman done is okay. Just let her go.".
The fact is, this woman has sinned..
And there were indeed laws that say that people who have been caught in adultery should be put to death..
Leviticus 20.10, Deuteronomy 22.22..
So if we were to paraphrase Jesus' answer, he's saying, "Okay, yeah, go ahead. Judge this woman..
You can throw a stone at her, but before you do that, judge yourself first.".
What Jesus is able to do with his answer is see right through the motivations of these religious leaders..
He struck at their hearts, forces them to examine themselves..
And as self-righteous as these religious leaders were, none of them would have been bold enough to say that they were perfect and without sin..
And so they're stunned into silence..
You see, church, a strong answer doesn't have to be loud..
It doesn't have to be harsh or even forceful..
Like we said before, all Jesus does is stand firm on the truth..

$^{401}$All he does is point to the truth..
He doesn't yell. He doesn't shout. He doesn't bash people with Scripture..
It's not even a long conversation that Jesus has with these religious leaders..
He doesn't get drawn into a big, long debate about him..
It's not a loud, angry exchange, but with some gentle yet powerful words..
Just like when Jesus calmed the storm on the sea, Jesus brings a calm to this hectic situation..
And from being a group that were badgering this woman, ready to sentence her to death,.
suddenly they have nothing to say and begin walking away..
They walked away because they know that they too would be just as guilty of punishment were they to throw a stone at this woman..
So at this, those who heard began to walk away one at a time,.
the older ones first, until only Jesus was left with the woman standing still there..
Jesus had escaped the death, saved the woman from death and escaped the trap set for him..
Every single woman who came for this woman had a change of heart..
These people had come to shame the woman, to shame Jesus,.
but now they were the ones who had to hang their heads in shame and walk away..
And after everyone had left, Jesus can finally fully focus on this woman that was being brought before the Pharisees..
The word "left" here is quite interesting. It's not just "leave.".
Actually, in the Greek, it's actually got a strong connotation of being "abandoned.".
So this woman had now been left alone and abandoned..
She was useless to the Pharisees now, the plan didn't work, so they left her..
Everyone else had left her, but not Jesus..
See, Jesus refuses to abandon her..
In St. Augustine, he summarizes this scene beautifully when he talks about this passage in Scripture..
"There in the quiet of the mountain, two were left,.
she standing in the misery of her sin, and he standing in the glory of his mercy.".
You see, it's only when the environment becomes safe and calm.
that Jesus finally gets a chance to have a conversation with this woman..
And again, it's a short conversation, but it's beautiful and challenging..
With the threats and distractions out of the way, Jesus is able to commune with her..
Because he knew that before he needed to communicate anything with her,.
he first needed to commune with her, to look her face to face, to acknowledge her as a person..
Jesus straightened up and asked her, "Woman, where are they?.
Has no one condemned you?" "No one, sir," she said..
You see, where the religious leaders treated this woman with shame and condemnation,.
Jesus treats this woman with love and respect..
When Jesus says "woman," it's not like, "Woman!" You know, it's not like that, okay?.
But it's actually a very—gosh, I'm sorry—it's actually a very courteous way,.
the most kind way he could have addressed her..
This was Jesus' way of letting her know, "Hey, it's okay..
Everyone else is gone, but I'm still here..

$^{441}$You're safe now. I'm here for you.".
This is an important point for us to keep in mind when we're trying to have conversations with people.
about tough things, because if people don't feel safe, they're not going to open up..
And our role as people who point people towards Jesus, we must be able to create safe environments.
for people to truly able to talk about their struggles, their sin, their fears, their darkness,.
without them thinking that they might be condemned..
And one great way we can do this is share out of our own brokenness and our own vulnerability,.
because when we do this, we let people know we're just the same..
Everyone in this room, everyone in this world battles with sin,.
and we all need to bring our sin before Jesus..
But when we let people see our journey, that's truly one of the best gifts we can give to someone,.
to commune with someone, to get to know them before we try and communicate anything with them,.
to embrace their brokenness and vulnerability, to embrace your own brokenness and vulnerability.
as a point of connection, and as they open up to you, to let them know that they're loved,.
to let them know that they belong..
This is why Jesus' response is so powerful..
He says to her, "Then neither do I condemn you," Jesus declared..
"Go now and leave your life of sin.".
The only person who could have given her the death sentence,.
the only person who have technically picked up a stone to throw at her,.
the person who truly had the authority to condemn her has told her she is not condemned..
The Pharisees notice the sin and see it as an opportunity to judge and to trap..
Jesus notices the sin and sees it as an opportunity to love, to forgive, to show mercy,.
to give a new life..
Now, please don't get me wrong. Jesus is not excusing her sin at all..
When Jesus says, "Neither do I condemn you," He's not saying, "It's okay. Just go. I don't care.".
No, Jesus wants this woman to help this woman see that the situation was serious..
She has literally been rescued for imminent death..
But He also wants her to know that now you are free..
The sin that held you back, the sin that held you in bondage, no longer has any power over you..
And now you have a new lease, a new way to live your life..
You see, what's amazing is that this woman doesn't really know who Jesus is yet..
She doesn't know that Jesus is on this mission to the cross..
She doesn't know what this woman doesn't know in that moment that He loved her so much..
He was going to take on that condemnation that was due for her..
He was going to sacrifice Himself to death so that she wouldn't have to..
That's the only way that she could have been forgiven..
So, yes, Jesus offers forgiveness. Jesus forgives..
But we must also remember that forgiveness is a costly thing and that we can never treat it lightly..
Forgiveness came at the cost of Jesus' life, which means that if we are to receive that forgiveness,.

$^{481}$it should also come with a change in our lives..
Jesus lets this woman know she is loved, sinful yet forgiven..
Then and only then does the conversation turn towards asking her to change..
Go now and leave your life of sin..
It's not just a quaint word. It's not just a cute ending to the story..
But it's a deep challenge for the woman now to live by..
It demonstrates the forgiving spirit of Jesus and His firm call to live a transformed life..
Jesus is telling her that she can start a life anew..
But the motivation He's doing that is not through shame, guilt, and condemnation..
It's through the full depth of His love and forgiveness..
And Scripture doesn't tell us what happened next to this woman..
But what I like to think is that if she really had been impacted by Jesus in this moment,.
in this short little conversation,.
she probably went on to live a life that was marked forever by His love and His mercy..
So where does that leave us?.
Well, in this room right now, we too are not perfect people. Far from it, in fact..
And so therefore, this community needs to be a place where we can put down self-righteousness.
so that we can enter into conversations with each other when we're struggling with sin and darkness..
And we don't need to yell. We don't need to berate. We don't need to condemn..
This story proves that often it's the kind, the loving conversations,.
the ones that confidently but gently point people towards Jesus.
that become the most effective ones we can have with others..
It shows us that when we allow for Jesus to calm the noise and to lead the conversation,.
that's when lives are changed..
When we make space for God in the conversation, He always, always appears to us..
So this community needs to be a place where anyone who's struggling with sin.
can become part of the community and be led to the one who can truly heal and forgive..
And often we might find that as we walk and talk with God to others,.
they too begin to start their own conversations with God..
God begins to speak to them in ways, amazing ways, that you never could if it was just done by your own power..
So church, in the coming weeks, as we continue to look at the conversations that Jesus has,.
as we continue to discover how He engages and speaks with people,.
I hope that this would be our starting point..
May we be a people, may we be a church that speaks truth and love in the darkness.
with the love and mercy of Jesus..
Would you close your eyes, church? I'm just going to pray for us for a minute..
This is our hope, this is our goal, to be this kind of church,.
to be the kind of people that have these kind of conversations..
And perhaps you're in here this morning and you think, "Oh, actually, you know what?.
I've been quite angry lately.".

$^{521}$And even in my own zealousness to bring people to realize who Jesus is,.
maybe perhaps you've realized you've let anger dictate your words..
You haven't taken the time to pause, to think before speaking out..
And Jesus is here to remind you, to show you, "I appreciate your heart,.
but maybe it's time to let go of the anger and embrace my approach..
Take on my mercy and impart that to others..
Love as I did, and let others know that they are safe,.
that they don't need to fear your judgment when you open your mouth to talk to them.".
Maybe you're in this space and you have used your words to condemn others,.
and you're realizing, "Oh, yeah, those words were hurtful..
I shouldn't have said those things.".
Just allow the Spirit of God, the mercy of Jesus to wash over you too,.
to help you in your further conversations..
Maybe you're in this room and you're feeling like that woman,.
and you've been hurt so much..
Yes, there are things you know that you're ashamed of, perhaps..
An area of sin, of addiction, of something that had a grip on you,.
and you've tried to talk to people,.
but it's often been met with scorn and scolding..
You just don't feel safe. You don't feel like, "Oh, you know, if this is the case,.
why should I even bother talking to anybody?".
If you felt condemnation in this way,.
know this morning that Jesus doesn't condemn you either,.
but He wants to draw you in..
He too wants to give His mercy and forgiveness to you also..
And as you encounter the true love of Jesus, His gentle yet strong heart,.
I pray that you would be led out of your freedom,.
the chains that hold you, the bondage of darkness,.
and led into a life with Jesus guiding the way,.
a transformed, a renewed life..
And so, Lord, this is our prayer..
Our words are powerful..
Conversations are a gift that you've given to us,.
and we can either use it to hurt or to heal..
And so, Jesus, may the words that come out of our mouth.
be drenched by your Spirit, Lord..
May the conversations we have about you be full of grace and mercy and wisdom..
And as we do this, continue to shape and mold us to the people,.
to the community, to the families that we need to be.
to show your love to the world around us..

$^{561}$We pray in your beautiful name, Lord. Amen..
[music].
[BLANK AUDIO].
\newpage



\section{}
\label{sec:OCz8LrBOC28}
\textbf{2023-07-09 Conversations of Testimony [OCz8LrBOC28].mp3}
\newline
\newline
連結: \href{https://youtube.com/watch?v=OCz8LrBOC28}{\texttt{ https://youtube.com/watch?v=OCz8LrBOC28}} ~~~~ 語音日期: 2023-07-09 
\newline
\newline
\hyperref[sec:2hwTmUlFH_A]{\small{< < < PREV SERMON < < <}}
~
\hyperref[sec:index]{\small{[返主目錄]}}
~
\hyperref[sec:Sc_q0QFc6ec]{\small{> > > NEXT SERMON > > >}}
\newline
\newline
$^{1}$For what you've done in us, we give you praise..
In Jesus' name, everyone says, Amen..
Amen..
Hey, can we thank our worship team as always, as always..
Why don't you have a seat, have a seat..
Brilliant..
Well, we're excited to continue our sermon series, Blowing Water, which Pastor Ellis.
and I started last week, and this week we're looking at this idea of testimony and the.
testimonies that we hold..
And I'm so excited because we have Pastor Jess with us today..
Can we put our hands together as we welcome Pastor Jess?.
Yes..
I'm going to pray for you..
Let's stretch out our hands as we pray for Jess..
Father, I thank you that so much of what Jess is about to share really is her own personal.
testimony..
The passion that sits inside of her to share the good news of Jesus through Alfer and a.
number of other ways that she leads here at The Vine..
And Father, you've done so much in her, and I know that through her you are doing so much..
And Father, she opens up Scripture today..
Lord, we now pray for ourselves..
And Father, we are anxious in a good way to hear your word, to receive that little thing.
that you have for us today..
And Lord, I believe that you want to speak personally to every person in this room, in.
the overflow, online right now, a personal word through what Jess is about to share..
So Lord, we just pray that our hearts would be fertile soil for what it is that you want.
to do..
And we thank you so much for this..
In Jesus' name, everyone says..
Amen..
Thank you, Pastor Andrew..
Hi, everyone..
My name is Jess, as Andrew said..
So one of the recurring conversations I have when I go about town, one of the "chui sui".
things that happen as I go about Hong Kong is my Cantonese..
And sometimes when I'm just ordering something or buying something in a shop, and if someone,.
you know, is really friendly and I'm speaking in Chinese, they go, "Wow, your Cantonese.
is quite good.".
And if I'm in a good mood and I want to engage in conversation, I'll say, "I'm a proud.
Hong Konger.".

$^{41}$And then we continue to chat..
If I'm in a grumpy mood, which I confess maybe sometimes is more than me wanting to chat,.
I'll just offer them my life story on a plate because I know what's coming..
They want to know why I speak Cantonese..
I'll say, "I grew up here in Hong Kong..
I went to my mom's Hong Kong Chinese, my dad's English..
I went to a local school when I grew up.".
Now I've told you, so I don't have to tell, I have conversations with you again about.
that..
If I'm in a really, really don't want to engage in conversation, I just speak in English..
I can just not engage..
But that's not how we should have conversations..
That's not what this sermon series is about..
Because conversations now, perhaps they're even more important during this period as.
we come out of COVID, a time that was defined by distancing, by isolation..
So we're looking at some conversations that Jesus had in the Bible and how that can show.
us, teach us, mold us about some conversations that we can have..
And like Andrew said, my role here as a pastor is I have the privilege of overseeing community.
groups and Alpha..
And we just had another round of Alpha finish..
But before this round, we had Sharena come here and share her testimony..
And her testimony was that she came as a lifelong Hindu..
She didn't believe in Jesus..
And on the Alpha day, which was the day we look at the topic of the Holy Spirit, she.
gave her life to Christ..
She subsequently received a miraculous healing from some back pain that still remains..
And after that day, after the Alpha day, I went home and I was just excited, thankful.
for all that God had done..
I found myself saying to God, "God, I'm so grateful for what you did..
I'm so excited you did this in Sharena's life, but that's not enough..
I would like to see way more people come and experience your love.".
And he broke my heart for Hong Kong and I prayed, "I would love everyone in Hong Kong.
to see your love.".
And I was like, "God, what are you talking about?".
Like realizing this is not from me..
I felt actually God was a bit cheeky..
It's like, "This is not from me, but you're making me sound like this is from me..
But this is a desire from God, I believe in.".
I'm like, "I don't understand..
What are you talking about?.

$^{81}$My remit is the vine..
I'm a pastor of the vine..
It's not Hong Kong..
What are you saying?.
Are you asking me to go and work for Alpha Hong Kong?".
And the answer is, "No, I'm not leaving..
You're stuck with me because I do believe my calling here is to you.".
And I believe actually what God was saying was something for all of us here..
God is speaking to us about our call to be representatives of God's love outside of the.
four walls of this church..
And so today we're going to look at a passage in John 4 that is about a woman that Jesus.
chats with..
Jesus reveals who he is and she then go and tells others about her testimony..
Now you might be a bit like me, sometimes sharing about our faith with people outside.
of the church can be scary, but it also can be one of the most powerful ways we can have.
conversations..
So looking at this scripture in John 4, up until this point in Jesus' ministry, he was.
ministering in Judea, but he left Judea to head back up to Galilee..
And the most direct route goes through the middle, which is it goes through Samaria..
We've got a map there..
It's the white route..
And this would take three days on foot to get there..
And this appears to be the route that Jesus took..
However, many devout Jews wouldn't take that route..
They would take a longer route..
They would take the red route or the green route because they wanted to avoid Samaria..
And what happened then was that there was a lot of bitterness and hatred between them..
And in the biblical context, in 702 BC, Samaria was invaded by Assyria..
And a lot of the Jews were exiled, but those who stayed, many of them married Assyrians..
And they created this new race..
They were half Jewish, half Gentile..
And so the Jewish people called them half breeds..
And it was particularly unpleasant or even disgusting because the Jews understood that.
God sent the Assyrians to conquer them because of their sin, because they sinned against.
God..
And because the people in Samaria adopted the ways of the Assyrians, there was just.
huge discrimination..
People looked down on them..
So there was a lot of theological and historical reasons why the Jews didn't like the Samaritans..
And by the time Jesus was on earth, this had been built up for a thousand years..

$^{121}$And this is why John 4, 9 says, "Jews have no dealings with Samaria.".
But it appears that Jesus and his disciples didn't avoid the Samaritans..
They went through Samaria..
And by the time we come to John 4, this passage, we believe they'd been traveling on foot for.
six hours..
It was noon..
Jesus was thirsty and tired..
His disciples went into town to buy food..
So Jesus was alone and he needed a break..
So he sat by a well, and it's a famous well..
It's Jacob's Well, which is fed by an underground spring..
It was a very reliable source of water..
And so as Jesus was sitting down by himself by the well, this is what happens..
A Samaritan woman came to draw water..
And Jesus said to her, "Will you give me a drink?".
Now this is radical for several reasons..
We've just heard that there's bad blood between the Jews and the Samaritans..
They were not friends..
But secondly, for a Jewish man to initiate a conversation with a woman alone was actually.
frowned upon..
It was known that a man would go to a well if they wanted to look for a spouse..
It was a bit like maybe a man trying to chat up a woman at a sleazy bar here..
And then a Samaritan woman, we've just heard how Samaritans are considered unclean..
The women are considered even more unclean because of blood..
And they were considered so unclean that even the vessels that they've touched, if Jesus,.
if anyone drunk from it, they would be defiled and they would be unclean..
So no surprise, the Samaritan woman responded to Jesus asking her for water, saying, "Really?.
You're asking me for water?.
Don't you?".
This is not in Scripture..
This is how I'd imagine she's saying, "Really?.
How can you?.
Don't you know there's bad blood that you look down upon us?".
She doesn't know who this person is asking for water..
And so let's take a deeper look at Scripture..
The whole story is covered in John 4, which spans 42 verses, which we won't go through.
today, but I really encourage you to read through it this week..
So we'll pick it up in verse 10..
So she's saying, "Why would you ask me for water?".
And Jesus answered her, "If you knew the gift of God and who it is that asks you for a drink,.

$^{161}$you would have asked him and he would have given you living water..
Anyone who drinks of this water, the water from the spring, will be thirsty again, but.
whoever drinks the water I give them will never thirst..
Indeed, the water I give them will become in them a spring of water welling up to eternal.
life.".
The woman said to him, "Sir, give me this water so I won't be thirsty and have to keep.
coming here to draw water.".
She still doesn't quite get it..
She's like saying, "Jesus, what are you talking about?.
Give me this water.".
And he then told her, seemingly not answering anything she just asked, "Go, call your husband.
and come back.".
And she says, "I have no husband.".
Jesus said to her, "You are right when you say you have no husband..
The fact is you have had five husbands and the man you have now is not your husband..
What you have said is quite true.".
So we've seen all the reasons that Jesus might have not wanted to engage with her, but now.
we know even she's living an immoral life..
She's living with a man who's not her husband and that is considered immoral..
But Jesus reaches out to her, initiates the conversation with her, and then later even.
declares to her quite boldly, "Then Jesus declared," verse 26, "I, the one speaking.
to you, I am he..
I am the Messiah.".
Jesus tells her who she is..
So over a conversation, she starts to realize who Jesus is and so she immediately couldn't.
help and go back to her town..
Her response was brilliant..
She runs back to her town..
She leaves her water jar and he says this, let's look at verse 28, "Then leaving the.
water jar, the woman went back to the town and said to people, 'Come, see a man who told.
me everything I ever did..
Could this be the Messiah?'".
They came out to the town and they made their way towards town..
Many Samaritans from that town believed in him, believed in Jesus because of the woman's.
testimony, "He told me everything..
He told me everything I ever did.".
Everyone from her town came out because of what she said..
And again, this is not in scripture..
This is what I imagine she would say..
She would, "Do you know, I just went about today about my normal routine..

$^{201}$I was just going to get water..
It was in the middle of the day and this Jewish man, this Jewish man was by himself and he.
started a conversation with me..
Doesn't he know we shouldn't be talking alone?.
And he asked me for water and then he knew everything about my life and I think he is.
the Christ..
Come and see him.".
That's what I think her testimony would have been..
And she immediately shares that with her people because she's so excited..
But it says that they came out to the town and they made their way to meet him..
But it says that many of the Samaritans believed because of her testimony, because of what.
she said..
So testimonies are powerful..
So why are they powerful?.
What is a testimony really?.
A testimony here means to bear witness, to give evidence, to testify..
And witness is one of the key themes in scripture..
And in fact, in ancient Israel in the early church, it carried a kind of forensic sense..
It meant that it was someone who has seen the truth or heard the truth..
They can testify to the truth in a court of law and they can declare what they've seen.
or what they've heard and give evidence to what they've experienced..
And witness throughout the Bible can certify facts and it's often related..
It talks about how the church of God, the people of God bear witness..
They can give witness to who God is..
And witness is so important in John's gospel that it talks about seven ways that we can.
testify to who Jesus is..
So seven ways that testify, yeah, who Jesus is..
And the first is the Father..
The Father, it says in John 5, 37, "The Father who sent me, sent Jesus, he has testified.
concerning me.".
Jesus himself testifies who he is..
We just heard that..
He declared that to the Samaritan woman..
The Holy Spirit..
It says that the spirit of truth goes out from the Father..
He will testify about me..
The works of Jesus tell of who Jesus is..
The miracles, the power that he carries..
Scripture reveals Jesus..
John the Baptist reveals and testifies to who Jesus is..

$^{241}$And last but not least, it's our human witness..
Our testimony is one of the things that the gospel of John tells us speaks to and can.
certify who Jesus is..
It speaks to his power..
So why is this so powerful?.
The amazing thing is everyone in here has a testimony..
Everyone in here who has encountered Jesus has a unique story that they can share..
And your unique story is about your details of how you experienced Jesus..
How did you meet him?.
What are the amazing things that he's done for you?.
In a way, it showcases what you've personally seen..
And it really is you sharing your witness and how God is personal to you, how you met.
God..
So a testimony is personal and subjective..
It's your story..
In a way, it's your love story with God..
It's how you encountered the spirit of truth and how this truth took root in you..
And so you are sharing this with joy with other people..
A testimony is not a sermon..
It's not a theology lesson..
It's not a Bible study..
And it's not preachy..
It's light..
We share it, hopefully, without obligation..
We share it because we're so joyous of what Jesus has done in our lives..
And we share it..
And when we do, a testimony makes way for others to come and see who Jesus is..
So it opens up an opportunity for other people to witness, to experience what you have experienced..
So it's hugely powerful..
And it's not without theological cause either..
Because 1 Timothy 2.4 says God's heart is that he wants all people to be saved and to.
come into a knowledge of the truth..
And so not only is God's heart that everyone wants to hear his truth, none of us, we've.
already heard, we all have a unique story..
None of us is disqualified from sharing our testimony..
None of us have a boring testimony or an unimportant testimony..
All our testimonies are important..
And actually 1 Peter 3.15 tells us to prepare our testimony..
It says, "Always be prepared to give an answer to everyone who asks you to give the reason.
for the hope that you have.".

$^{281}$So if someone asks you, that's a brilliant way to share why you have a hope in Jesus..
Why do you love Jesus?.
And so I really want to encourage you this week to take time to think of your testimony..
Just now when we prayed, maybe something already came to mind..
But to take time to craft your testimony so you can share it..
And I love practical tips..
So I love to give you a bit of a practical tip..
You know, how do we put together a testimony?.
And it really consists of three simple parts..
The before, the during, which is really the what and the how, and after..
So the before, a testimony doesn't just have to be about how you came to faith..
Although if that is your testimony, do you share what was life like before you came to.
Christ?.
But for those of us who have been Christians, who have been walking in a relationship with.
Jesus for a while, it could be something recent that Jesus did for you..
A prayer that you've been praying, a prayer that Jesus answered, a need that you had that.
Jesus came through for you, that caused you maybe to realise to a greater level the intimacy,.
the deepness of God's love for you..
So it could be how you came to faith..
It could be a recent testimony..
And then so that could be the before..
And then you just share what happened..
What were you praying for?.
Or how did you came to faith?.
And then the after..
How did your relationship with Jesus grow?.
Or what differences Jesus made to your life?.
That's it..
That's a testimony..
Before, during, after..
Simple, yeah?.
We can all do it..
And so your testimony, it will be personal, personal to you, your unique story..
It's honest and it's true..
It's true to what happened..
It doesn't downplay anything..
It doesn't exaggerate anything, but it speaks to the truth..
Like I said, there's no bad testimony..
There's no boring testimony..
And we are all called to be prepared to share our testimony..

$^{321}$Like I said, Alpha here just finished..
And we know you love hearing the testimonies..
So right now, we'd love to invite two people who were involved in the last Alpha to share.
their testimony..
Should we do that?.
Yeah..
So let's first invite Eric..
Eric, come join us..
Hi, Eric..
Hi..
Just have a chat..
Sorry..
So Eric here used to work here..
Many of you might recognize him..
He used to be part of the team that takes care of this building, this facilities..
He was part of our facilities team..
But recently you just left, which is sad..
But tell us, before you came to the Vine, were you a Christian?.
And so how did you come to join us on Alpha?.
Because I work the night shifts..
I worked during the last menacos when Pastor Andrew preached on Genesis..
When I heard it, I want to understand who Jesus is..
It is a special chance that I was invited by Karen to join Alpha..
This was an opportunity to join Alpha, and I don't know why I say yes..
So I joined Alpha..
And then what happened on Alpha?.
On week three of Alpha, the Alpha talk was, "Why did Jesus die?".
It mentioned about some planes we carry inside..
I have carried the plane since my mother passed away..
I haven't been able to let the pain go..
I was busy working at that time and overlooked many signs of my mother getting worse body condition..
A series four accident happened that injured her brain, causing her death..
I felt so guilty for not paying close attention to her body..
And I thought I was responsible for her death..
I have been carrying this guilt for six years..
During Alpha, I heard Jesus took away my sins..
I don't know whether I sinned against my mother when she had her fall..
I hope Jesus can forgive me..
I believe He forgives me..
But I'm still trying to forgive myself..

$^{361}$I shared this to my group, and my group leader asked me, "Would you follow Jesus?".
So what did you say?.
Yes, I said yes..
I didn't know how to become a Christian..
I didn't know the steps..
I thought I have to be baptized first..
I didn't know that if God comes to find you, you just respond..
It's so simple..
God comes to find me and I say yes..
And so what difference has Jesus made to your life?.
This thing has set very deep inside my heart, and I never told anyone..
Now because I've told Jesus about this and my small group, I have peace and strength.
to face this..
I don't have to face this alone..
I felt a power for me to confirm this and confirm the things of the past..
I feel like I have a new start in life for the better..
I have a lot of peace..
Another strange thing is I feel like I have new creative ideas..
I don't know what's happening..
I have new talents and gifts in cooking..
And I don't have ideas and thoughts that just come to me..
Even my wife is surprised..
So after this experience, I share it with my family, with my wife and two daughters..
I told them I became a Christian and I tell them we can pray together..
My younger daughter asked me, "Where is Jesus?.
Where does Jesus live?".
She will put her hand together to pray..
When we are facing challenges, I say to them, "Why don't we pray together?.
Let's talk about it.".
When my wife can't sleep, I offer to pray her..
Wednesday is a big day for our family..
We are waiting to hear if my older daughter get into secondary school..
I know I can trust God, so I put this in his hand..
Thank you, Eric..
Thank you so much for sharing your testimony..
I would love to share another story, another testimony with you..
This is a testimony about Zoe..
So Zoe has given us permission to share..
She's not quite ready to stand here and share in person, but she has given us permission..
So Zoe is originally from Beijing..

$^{401}$She's from a country, a city where very little of Christianity or spirituality was spoken.
about..
When she came here, she found that she had a curiosity about the Bible..
She says this, "If the Bible is the most read book, how come no one in my circle back home.
read it or even talked about it?".
So Zoe wanted to learn about the Bible, and so she emailed quite a few of the international.
churches..
Guess what?.
The Vine was the first one that got back to her, so she came to The Vine..
I know it's not a competition, but we won..
Anyway, so it was at The Vine that she heard about Alpha, and so she joined Alpha..
And so Alpha is made up of small groups, and each small group has a small group leader,.
small group hosts..
And we've invited Jennifer to come and share Zoe's testimony and her own journey..
So let's give a round of applause for Jennifer..
How's she?.
Hi..
You've got a fan club over there..
Hi..
So tell us, tell us Zoe's journey on Alpha..
Yeah, absolutely..
It has been such a joy walking with Zoe..
She's always a special one with a very strong spiritual hunger for the truth..
But I remember six weeks into Alpha, Zoe was telling our group, "I won't be so fast in.
calling myself a Christian when I'm not ready.".
And at that time, I was a little bit let down by it and thought, "Okay, this girl, I should.
never push her.".
But fast forward to Alpha day, during prayer ministry time, I asked Zoe, "So how can I.
pray for you?".
And Zoe said, "You know what?.
I've actually been trying so hard to build a relationship with God, but to be honest,.
I feel like there's a huge disconnection..
There's something blocking.".
It was then that I felt prompted to say, "Oh, would you actually want that head knowledge.
to become a heart transformation?.
Do you want to invite Jesus into your life?".
And she said, "Yes.".
So we did the sinner prayer together..
And after that, Zoe told me, "You know, it was when I had to repeat after you that it.
was the first time that I realized actually all along I might not think that I'm a sinner,.

$^{441}$but I should have messed up here or there..
I've heard similar prayers in church before, but it was the first time that it dawned on.
me that I'm actually a sinner and Jesus has died on the cross for me and God really loves.
me.".
And after the sinner prayer, actually the first thing she told me was, "Jen, I really.
want an English Bible, a physical copy..
I feel like God has been wanting to talk to me, but I'm the one who kind of want to stop.
him shutting the door because I've not been reading the Bible consistently.".
And so right after Alpha Day, on that day, we went to a bookstore to get a Bible..
And after that, I told Zoe, "Okay, let's just read one chapter of the Bible and the gospel,.
just one chapter, because don't overburden yourself, don't kill yourself..
But remember to just pray before reading and invite the Holy Spirit to really open your.
eyes to the living Word of God..
Otherwise, it's just like any other book.".
And so it was at that time that I felt, "Ah, it's kind of mean to just ask her to do it.
herself.".
So I really felt the Spirit said, "Read it together with her.".
And so I offered that and just said, "Maybe just what's that one line of reflection?".
But guess what?.
Zoe said, "No, can I email you instead?.
I want to properly write down my thoughts.".
That's how we started the journey on reading the Word together..
And now it has been a month and we just finished the book of John and we're reading Matthew..
And I remember 10 days after Alpha Day, it was June 20th, it was then that I came across.
the baptism course..
And I was thinking, "Should I ask Zoe to join?".
But little do I know, on the exact same day when she emailed me on her reflection, she.
said, "I've already signed up the course.".
So praise God for that..
And Zoe also wants to take this opportunity to really thank everyone who has been so helpful.
in her journey..
And she also wants to encourage you all to join Alpha..
And she said, "Yeah, it would be nothing like what you've imagined until you actually participate.".
And can you share a bit about also your experience of walking with someone, sharing this conversation.
with Zoe?.
Yeah, sure..
It has been such a joy..
I've never thought that there would be such a joy experiencing firsthand..
And the reason why I joined as a small group host is because I really want to learn..
I'm really never the type that is good at evangelizing..

$^{481}$I just feel super awkward sharing my faith..
But at Alpha, it's such a relief because I don't have to speak much..
I just have to give a space for people to throw out their questions and just share..
And the beauty of Alpha is that I don't even have to answer any difficult questions because.
all the teaching is done through the video and also the life talks..
So all I have to do is be a friend, and that I can do, and explore life with them and trust.
that in time, it is the Holy Spirit who will guide them to the truth..
So I just have to give them a space for them to feel loved, welcomed, and heard..
And to be honest, at the beginning, I was a bit helpless, especially when people started.
dropping out in the first few sessions..
But I really felt really lucky that it's not just me, but there are great helpers in my.
group..
And God also reminded me, after all, it's not my ability that matters..
All I have to do is to purely rely on the Holy Spirit and to know that what God sees.
in me is just my willingness to partner with Him..
If the focus is to give glory to God instead of myself, then I shouldn't have any pressure..
My only role is to create a space for people to encounter Jesus directly..
- Oh, let's thank Jennifer..
Thank you so much..
Thank you..
And so, as we come to a close, there's just one thing that Jennifer said that I would.
like to point out..
And looking back at John 4, verse 14 says this..
It said, "Whoever drinks the water I give them will never thirst..
Indeed, the water I give them will become in them a spring of water welling up to eternal.
life.".
And this water that's spoken here, it speaks of the Holy Spirit, that Holy Spirit that.
Jennifer spoke about..
And really, it's the Holy Spirit that sustains our relationship with God, that draws us into.
relationship with God..
But it's also the Holy Spirit that empowers us as He fills us with this spring of water.
to share our testimony..
It's the Holy Spirit that empowers us and anoints us and goes before us as we do..
Yes, we're called to prepare our testimony, to craft our testimony, but when we do share.
it with people, after we've shared, it is over to the Holy Spirit to testify as to who.
Jesus is..
We heard just now, the Spirit of truth who goes out from the Father, He will testify.
about me..
So it's important for us to know we are encouraging each of us as we have a unique testimony to.
prepare that, but also to be released..

$^{521}$It's not our responsibility whether someone comes to Christ..
They may love your testimony, which is always great, and it's always disappointing when.
they don't respond..
They might even think you're crazy, and we do risk that, and it does take courage, but.
we have freedom to just share what God has done in our lives so that it gives them an.
opportunity to experience God's love..
And this really is what the River Vision is about..
On Vision Sunday, Pastor Andrews spoke to us and reminded us that love and love alone.
is the gravity that moves the river of God's kingdom forward in a city..
God has called us each to move the river of God's kingdom forward where He's placed you.
in your area, your sphere of influence..
Your life is important..
Your life in Jesus carries a weight that when you share speaks to the majesty, the greatness.
of Jesus..
And so we are, as a church, we are called to make an impact on the city, and sharing.
your testimony hopefully is a way that you can share out of joy, out of what you have.
experienced with God without obligation, and share it where God has placed you..
That's how the whole city can maybe start to impact, be impacted by Jesus' love..
We heard how Eric shared that he became a Christian with his family..
If we each share with one person, maybe in our workplace, in our school, in our gym,.
wherever we are, we share it with one person..
We just heard from the Samaritan woman's testimony..
Her one testimony caused that almost domino effect that brought the whole town to faith..
So I really would love to encourage you..
What is your testimony?.
What is your story?.
It's important..
No one in here has an insignificant or bad testimony, and it's something that you can.
share..
It shares your hope..
It shares the power of Jesus, and as we each go out and do this, may God bring revival.
just like we sang just now..
May God transform the city that we love..
So yeah, I encourage you to take time to craft your testimony..
And as we come to a close, I'd just love to invite us all to stand, and I'd love to pray.
for us..
God, we thank you for this story of the Samaritan woman..
We thank you for the power of one woman's words, one woman's story..
And we thank you that every person in here and every person watching online has a story..
And so God, empower them, anoint them, show them how they can craft their story..

$^{561}$And I pray that you'll empower them to share it with people..
You'll give them opportunities to share..
If anyone does feel your story is not important, that's not true..
That's not true..
I pray that Jesus, as they prepare, they'll draw close to you..
They'll encounter you again, that your Holy Spirit will draw them close into relationship..
And for anyone facing hardship here, we thank you, Jesus..
We thank you that we can entrust everything to you..
In Jesus' name we pray..
Amen..
[MUSIC PLAYING].
\newpage



\section{}
\label{sec:Sc_q0QFc6ec}
\textbf{2023-07-19 BOH Sunday - Seeds of Growth [Sc\_q0QFc6ec].mp3}
\newline
\newline
連結: \href{https://youtube.com/watch?v=Sc_q0QFc6ec}{\texttt{ https://youtube.com/watch?v=Sc\_q0QFc6ec}} ~~~~ 語音日期: 2023-07-19 
\newline
\newline
\hyperref[sec:OCz8LrBOC28]{\small{< < < PREV SERMON < < <}}
~
\hyperref[sec:index]{\small{[返主目錄]}}
~
\hyperref[sec:_QaFNgOx4_0]{\small{> > > NEXT SERMON > > >}}
\newline
\newline
$^{1}$Alright, so again, welcome to the church..
You can have a seat if there's some free seats down here if you want to scoot in, feel free.
to take a seat as well..
This is Branches of Hope Sunday..
You've heard us say that a few times, you might have seen some of the booths outside..
But if you're new to the church, we welcome you by the way, you might be wondering, you.
know, what exactly is Branches of Hope?.
And what do we mean when we talk about seeds of growth?.
Well, Branches of Hope is an amazing NGO that partners alongside the Vine..
We are sort of two in one, but they do amazing work in the area of giving assistance and.
help and relief in all different ways to our refugees and asylum seekers here in this city..
And it's had a long history, it's, you know, going through some things and there's an aim.
and a vision that we want to partner with BOH in as well..
So to give you a little bit more about the context and the history of Branches of Hope,.
please turn your attention to the screen and welcome our founding pastor, Pastor John Snellgrove..
Hi, I'm John Snellgrove and I'm one of the founding pastors of the Vine Church..
I expect you're asking yourself on this special BOH Sunday, what has the Vine got to do with.
BOH? And I'd like to give you a little history lesson..
It's time to take back almost 20 years..
The Vine Church was known for a number of things, things like worship, worship style..
We're also known as a church that stood up for justice..
And when we were confronted with asylum seekers, when we actually went to visit an asylum.
seeker in prison, in Victoria Prison, who was only in prison because he overstayed..
We suddenly realized there was a sense of injustice..
And then one Saturday afternoon, I'm going to tell you this story..
I was in the church and there was a group of African boys from Togo and they'd actually.
been sleeping on the star ferry because they had nowhere else to sleep..
And they came in and asked me, you know, could I provide them a room where they could go.
and pray? And I found them that..
And I went out and got them a McDonald's big Mac meal..
And I guess the rest is history..
You know, that was the beginning of our African fellowship..
At that time, asylum seekers, children couldn't go to school..
And next to my office was another office, which was actually a makeshift kindergarten..
And I had to work during the day with kids making all sorts of noise..
And I wasn't allowed to take part in the things that normal kids were taking part in..
And I guess the most poignant story for me was of a young man called Sidi..
He came here from the Congo and he came to us and he actually ran our African fellowship.
for a while, but he got sick and he needed a transplant..
He needed a new kidney..

$^{41}$And you know something? He wasn't even allowed to be placed on the waiting list..
And of course, being a asylum seeker, we couldn't send him anywhere else..
And to cut a very long story short, this man that a man used to call me dad,.
and I regarded him as my spiritual son, actually eventually passed away..
And all of this has caused us as the Vine Church to want to do more..
And so rather than just something that we did as a church, we decided to form our own NGO..
We called it Branches of Hope because we believe in hope..
We believe in justice. We believe in supporting those who cannot support themselves..
And what we've been doing these past few years is working with refugees,.
asylum seekers, those caught in trafficking, and really doing what we can with our prayers,.
with our help, and with our finance to make their life easy..
Their life in Hong Kong still is not easy..
But hopefully, through the help of people like you, supporting Branches of Hope,.
we can do something. We can do something to make them feel they belong to a community..
And I think that's what Branches of Hope has actually done..
We've created a community here where people feel respected, they feel loved,.
and they feel part of what we're doing here at the Vine..
In conclusion, people always say to me, "The Vine has done a lot for refugees and asylum seekers.".
I say, "Yes." But then I say, "I'll tell you something..
Refugees and asylum seekers have done much more for the Vine..
The Vine would not be the church it is today were it not for Branches of Hope.".
Really wonderful. So you've heard a bit about the history of BOH and where it's come from,.
but to give us a bit of an update about where BOH is now and some of the initiatives that really need our support,.
we're going to welcome Herr Porter. Hello, my friend..
So Alex is our Executive Director at BOH, and him and his team have been doing some really cool work.
in working alongside the refugee and asylum seeker community..
We at the Vine get the privilege of doing some quite intimate work with them in ministry together,.
in relief work together. So here's Alex to give you a bit more of an update..
Why don't we just pray for him quickly as he comes in to share..
So Father, we thank you for BOH. Lord, we thank you for the work that they're doing.
and for Alex's leadership over this community, over this organization..
So just bless him today, Lord, as he shares a bit about what BOH is,.
what they're trying to do, and help us to connect with the work, Lord. In Jesus' name we pray. Amen..
Thank you very much. And good morning. I think it's full morning. Yeah, I'm sorry..
15 minutes left in the morning. And don't worry, I'm not going to do this in German despite my heritage..
So thank you very much, Alison. Yeah, it's great to be here..
We're extraordinarily blessed and honored to once again be hosted by the Vine for Branches of Hope Sunday..
Alison already mentioned I'm Alex. I'm the Executive Director of Branches of Hope..
Our theme this year is Seeds of Growth, and this serves as a metaphor for how our work enables the community.
which we serve to reach more of their potential. And it's also a reflection of the Vine's motto, Growing Big People..

$^{81}$Before Pastor Andrew takes the stage to give the work of Branches of Hope a bit more of a biblical perspective,.
I wanted to take a moment to connect the message that we heard from Pastor John in the video to what we do today..
So to connect the past with the present. Pastor John called it a history lesson, which I think was quite apt,.
because the work of Branches of Hope actually long predates the existence of the organization.
and is embedded in the very identity of the Vine Church..
What we do at Branches of Hope is born out of a church ministry and remains the justice expression of the Vine..
Perhaps more importantly, we continue to provide the groups we serve with a sense of belonging and community..
Pastor John also mentioned the Vine would not be the same without Branches of Hope, and while that is certainly true,.
we and the work we do could not exist without the continued support of the Vine Church and congregation..
In other words, every one of you here. Even though we work in different ways, we remain inextricably linked..
That said, Branches of Hope continues to grow and evolve, and last year we were blessed to celebrate our 10th anniversary..
We continue the work that we do with the generous support of the Vine and others,.
and serve hundreds of families and individuals with our two programs,.
Refugee Opportunity and Development, ROAD, and Stop Trafficking of People, STOP..
Though the needs of the two communities we serve are somewhat different,.
we have a unified approach in our service, and we call this the three pillars of service..
Those three pillars are care, empowerment, and social change..
While empowerment remains our core mandate by ensuring the self-agency of our communities through various personal development initiatives,.
we recognize that without addressing the basic needs of the communities we serve, we would not be as impactful as we could be..
Hence, the care aspect of our work. Pastor Andrew will be reflecting on this in greater detail..
And without addressing society as a whole and working towards changing hearts and minds about the communities we serve,.
large-scale, long-term positive impact would be out of reach..
Hence, the three pillars of service, care, empowerment, and social reform, social change, continues to inform our work..
What does that look like today?.
The STOP team reaches out on a regular basis to sex workers, migrant domestic workers, and other migrant workers.
to let them know that we are here to assist them if they or someone that they know faces a situation of labor exploitation or trafficking..
Over the past year, the team has spent 66 hours on the streets, in public spaces on Sundays, and even in brothels to reach those vulnerable that we are seeking to serve..
We provide Know Your Rights workshops, along with rehabilitation assistance to those who have been abused in their places of work,.
and conduct policy research to highlight best practices in prevention and response..
Over 2,700 members of the public and the community members we serve alike benefited from trainings, workshops, and educational talks over the past year..
Similarly, the ROAD team works very hard to fill the many gaps refugees and asylum seekers face that are left by the insufficient subsidies they receive..
The team helps find a new home or to connect to medical services when needed..
In the last 12 months, we have helped around 2,300 people in this way..
Sometimes with something as small as, for example, an in-kind donation, for example, some clothes or a second-hand phone..
In other cases, something much more major, like monthly financial assistance..
We keep children in school by providing and covering the costs of books, uniforms, and the like..
We provide summer activities, and we help young adults pursue a degree..
It is with great pride that we were able to support 80 students this year alone..
And as much as we do to foster a sense of community and offer avenues for integration,.
we recognize the hard and sad truth that, unfortunately, for the ROAD clients that we serve, none of them are actually allowed to remain in Hong Kong indefinitely..

$^{121}$And so we assist in resettling community members through private sponsorships to third countries,.
and we work to ensure successful reintegration for those being repatriated..
And over the past year, 19 of our community members were able to start a new chapter of their lives in countries outside of Hong Kong..
As you may tell from this little window into our world, we have a lot on our plate..
And to help you digest all this information and perhaps give you the opportunity to dive a little bit further,.
we did bring a little present for you all today..
You'll find the pouches on your seats, which Allison already mentioned..
Inside, there's a prayer guide, there's a brochure, and there's also the seed pencil..
The seed pencil is a bit of a symbol for our Seeds of Growth theme this year,.
because at the base, there's a little capsule that has basil seeds inside, so you can take that home and you can plant it..
And of course, we hope that as that plant grows, it'll be a continuous reminder of the work that we do at Branches of Hope..
Much like a plant, we have been watered by God's grace and your and the Vine Church's generosity..
Allowing us to meet many different needs over the past 10 years..
And with the same divine grace and continued generosity, we will continue to work to provide dignity, justice, and hope.
for the next 10 years and beyond, to continue reaching those who would otherwise be forgotten..
Thank you for your support..
Thank you, thank you..
Yeah, we're so excited that we get to do this together as a family today,.
because really at the very heart of it, Branches of Hope's work is a spiritual work..
It is the work of Jesus Christ at work in communities, fostering unity, creating hope, bringing dignity and justice..
And we really do see that as a work of God's Spirit..
So I wonder whether we might be able to just do a spiritual work together as we pray over Alex and his team..
I want to invite you guys all to stand as we do that..
And would you stretch out your hands with me and we're going to pray over these guys together this morning..
Father, we recognize that the work of justice, the work of empowering vulnerable communities,.
all the work that Branches of Hope does is your work..
And it is a work that's infused by the heart of God through the power of the Spirit..
So will we stand together as a church community and pray over Alex and what he represents with the wider team and staff.
and the clients that Branches of Hope serve..
We pray now, Father, that you would infuse this work with a deep sense of what we were singing about in that song earlier,.
that nothing can stand against the power of God..
And we pray that whatever issue that these clients are facing, whatever struggles that are in front of them,.
whatever obstacles seem like they can never overcome, nothing can stand against the power of God..
And we pray, Lord, that as we lean into this idea of your power and the way in which your power shapes us.
to be people of generosity towards those that are vulnerable,.
we pray that that power would be what Alex and his team look to every single day..
That they would understand, as I know they deeply know, in the humility that they carry,.
that they cannot do this work without you..
And so we stand together as a community of faith and we cheer them on, we pray for them,.
and we release, Lord, everything you have in your Spirit over these guys..

$^{161}$And we thank you for that in Jesus' name..
Everyone says? Can we give Alex another round of applause? So good..
Have a seat, have a seat..
So 20 years ago, 20 years ago, I met a man who radically changed the way that I think about justice,.
about serving vulnerable communities, and about God's heart to reach out to those that desperately need help..
This man actually represents so much, I think, of the kind of work that Branches of Hope does..
His name's Joseph Fati. And Joseph Fati fled to Hong Kong from Burundi..
And in Burundi at that time, there was a massive kind of ethnic cleansing that was taking place.
amongst competing tribes, the Hutu and the Tutsi..
And these two tribes were going at each other all the time, and some of the tragedies were incredible..
And Joseph Fati witnessed firsthand the horrors of ethnic cleansing..
He was from a large farming community, and his family was one of the largest farmers in their particular region..
And one day, Joseph Fati was coming home from a long day out on the farther ends of the fields of his farm,.
and he was late getting home. And as he gets near to his farmhouse,.
he sees that the competing ethnic group have surrounded his house..
And they have brought out every single member of his family to the front area of the porch of their home..
And Joseph Fati hides in the cornfields that they were growing nearby..
He hides inside these cornfields, and whilst he's hiding there,.
he witnessed this ethnic group murder every single member of his family, one by one..
I don't know if you could imagine what it would be like to see the people that you love,.
the people that you've grown up with, the people that you live with, systematically murdered in front of you..
But this caused a huge amount of trauma and really disorientation for him..
He fled out of the fields that day as the only survivor of his family..
A wealthy friend of his enabled him to basically pay off bribes in the airport to get Joseph Fati on an airplane..
He didn't know where he was going. He didn't even have a passport..
But he gets smuggled onto this plane, and the plane flies from Burundi and eventually lands here in Hong Kong..
Joseph Fati comes off the plane, he goes to immigration, he's got no passport, he's got no money,.
he's got nothing, just the clothes that he's literally wearing that day..
He walks up to immigration, and unsurprisingly, immigration places him in a quarantine center, in a detention center..
And there he discovers that there is this thing called UNHCR, the United Nations Human Rights Organization..
And they give him advice as to how he can now apply for asylum because of what he's experienced and what has happened to him in his home country..
So Joseph Fati applies for asylum, and that enables him a couple of months later to be released from the detention center,.
but he's placed in the city of Hong Kong with literally nothing..
No finances, not given the ability or the right to work, no resources whatsoever, no family, community, or friends in this city..
I wonder if you could imagine how lonely that would have been..
And that's a good example of how many of those in vulnerable communities in Hong Kong feel..
After about two weeks, only just about two weeks, Joseph Fati's found this reality that life is no longer worth living..
So he tries to kill himself..
And in the process of trying to kill himself, he ends up in a hospital here in Hong Kong..
And it just so happens that his bed in the hospital in Hong Kong is right next to the bed of one of our Vine congregation members, Ailing Father..

$^{201}$And this congregation member was going in daily to see their Ailing Father, and one day when they go in, there's an African man on his own in the bed next door..
And because this congregation member is super friendly and a really nice, kind person, they notice this African man, and they wanted to know his story..
So they asked the African man, "Well, what's your story?" And Joseph Fati begins to share his experience of what he had experienced in Burundi, getting on the flight, getting to Hong Kong, detention center, the whole story..
And the congregation member is like torn up inside..
And so the congregation member then tells this person their story..
And as part of that story is their relationship with Jesus Christ..
And two hours later, Joseph Fati prays the prayer to receive Jesus as his golden Savior..
I meet Joseph Fati for the first time just two days later as he's released from hospital and he shows up here at the Vine..
Back in those days, some 20 years ago, the church was much smaller than the one that you see here today..
And we didn't really know what to do with vulnerable communities like this, but there was Joseph Fati standing in front of us and saying, "I need help.".
And so we rallied around him as best we could..
I mean, he was the very definition of vulnerability..
He was alone. He was without any resources in need of emergency care..
And so we were able to raise some funds to be able to provide some food for him and get some clothes for him..
Actually, one of our founding senior pastors, Tony Reed, opened up his whole home for Joseph Fati..
And Joseph Fati came and lived in his house for two and a half years..
I think that's a rather cool example of sacrificial giving in small apartments like we have here in Hong Kong..
I remember one day I took Joseph Fati to buy a pair of sneakers..
I went to Sneakers Street in Mong Kok..
And I brought him what I can admit before you today to be a reasonably cheap pair of sneakers..
But for Joseph Fati, these were like I had brought him an original pair of Jordan 1s..
I mean, these were the greatest things he had ever received..
You know what I'm saying?.
And we were able to meet Joseph Fati's felt needs, but in the process of meeting his felt needs, we realized that there was something else that we could never meet..
And that was the brokenness that was inside of him, the trauma of seeing what he had seen with his family,.
the anger and hatred he had towards that ethnic group..
And we knew that the only way that that would change would be a work of the Holy Spirit..
That had to be something that God was going to do inside of him..
That needed to be a transformative thing..
And we were praying and we were believing and we were standing with him..
But Joseph Fati was going through life now cared for, but not necessarily free..
Free from the pain and the hurt and the anger and the bitterness..
I remember one day, as clear as anything from 20 years ago,.
we decided to go up into the rooftop of Tony's apartment building..
We weren't supposed to be up there, but there was five of us from the youth ministry..
We broke into the rooftop..
One of them had an acoustic guitar..
It was all for Jesus, okay?.
One of them had an acoustic guitar..
And we're up on this rooftop and we start doing worship together..

$^{241}$Joseph Fati's with us..
And Joseph Fati wasn't a very communicative person..
He didn't talk a lot..
He was very non-emotional..
You never really kind of knew..
I don't think I ever saw him smile very much before that moment on the rooftop..
But when we're up there on the rooftop and we're singing some Christian songs.
and we're doing some worship together, suddenly he begins to weep..
And he weeps and he weeps and he weeps..
And you can tell that the Spirit of God is doing something inside of him..
And we just are standing around and we're praying for him and laying our hands on him..
And he's just pouring out, just crying after crying..
You could see it..
And there was this, I guess, just months of all this pent-up pain just coming now out of him..
It was actually truly beautiful to see..
And after he had managed to kind of recompose himself,.
he begins to tell us what was going on inside of him and what he was doing..
And I want to read to you what he said to us that night..
He said, "I forgive them, for I myself have found grace and forgiveness in Jesus..
I love them," speaking of this ethnic group that murdered his family,.
"for I myself have been loved when I was not worthy..
And I pray for them, for I myself have been prayed for by Jesus.".
In that moment, Joseph Addy was more close to the heart of God than I think I've ever been in my life..
He was able to express something of the deep work of the Spirit of God.
and transform somebody into becoming more and more like him..
So Joseph Addy would go on to get refugee status by New Zealand of all countries..
Gotta love New Zealand..
He went to Bible school there, which was partly funded by the Vine Church..
He graduated with honors at Bible school..
He met the woman of his life..
They got married..
My wife and I were able to be at their wedding..
It was an amazing affair..
He now lives in Australia..
He has a growing family of his own..
And he now works for an NGO that's actually serving people that have experienced trauma from war, just like him..
That's somebody who has experienced something of Christian grace,.
who has experienced something of the transformative power of Jesus Christ,.
and who has now become a flourishing member of society,.
so much so that he's able to then feed into the very people that have experienced what he's experienced..

$^{281}$That's, my friends, seeds of growth..
And that's why we gather here today..
Because whilst Joseph Addy's story is personal to him, it is not unique to the people that we're serving at Branches of Hope..
You heard in the video about the Togo Boys and Saidi..
And there are many others, many that are actually sitting in this room right now,.
that we at Branches of Hope are walking with and serving,.
who have similar stories of trauma, similar stories of needing to flee their countries,.
similar stories of needing to be ministered to, both with the hope of Jesus,.
but also the practical resources that come with beginning to get your life back together again..
And although now, 20 years later, Branches of Hope does so much more than just relieving people's felt needs,.
as Alex so wonderfully just shared with us, some of that work that now Branches of Hope is doing,.
the core of our work over those 20 years has not changed..
We, as an organization, boldly walk into vulnerable communities in our city.
with the hope of Jesus and with the reality of health..
And through working, through relief work, we're able to provide physical, mental, emotional, and social care to those that need it the most..
That's what Branches of Hope does..
Now, what is it that drives this idea of us to serve those around us?.
Well, it's not just because there are needy people in society, and that just seems to be a nice thing to do..
We serve in this way as a church community and through Branches of Hope.
because of what I like to call a "theo-anthropological imperative.".
Everybody take a deep breath..
"Theo-anthropological imperative.".
You might be wondering, "What is that?".
Let me explain..
The reason why we serve vulnerable communities in Hong Kong is because God, Theo,.
has created all human beings, anthropological, in His image..
And because all human beings are created in God's image,.
every single human being, regardless of their status, regardless of their culture,.
regardless of whether they're wealthy or they're poor,.
regardless of whether they have it all together or their world is falling apart,.
every single person reflects something of God's image..
Therefore, everybody has inherent dignity,.
inherent worth,.
inherent purpose,.
and inherent value..
Every single one is called to walk in the purposes and promises of a God who's created them..
And if that's the case, then we all have a role, a joy, a life,.
to be able to serve those who are like that because we see them as one of us..
We don't see them as people that are desperately in need and we better do some nice things because that actually makes us feel better..
We actually see these communities as people who are made in the image of God,.

$^{321}$equal and powerful under God's heart, and because of that, we connect with them..
In other words, the vulnerable should never be charity cases.
in which we kind of think of ourselves as a savior..
No, we need to see them as children of God in which they become part of our family..
See, when we see vulnerable communities and vulnerable people in that way,.
that's a steel anthropological imperative..
We have this idea that they're brothers and sisters, sons and daughters, children of God..
We have this idea that they're part of our family,.
and if they're part of our family, when we support them, it's not out of duty and obligation..
"Oh, I guess I better do this because this is what a good Christian does.".
We support them because they're a member of our family..
They're people just like us, made in the image of God, carrying dignity and worth,.
and so therefore we have a joy and honor..
We serve out of love and honor, not out of duty and obligation..
And in just a bit, I'm telling you at the end before the beginning,.
I'm going to ask you to give some money today,.
but you're not going to give out of duty and obligation..
You're going to give out of love and respect and honor.
because we're not giving to charity cases that need a savior..
We're giving to children of God who are part of our family..
Are you with me, people?.
Yeah, that's a serf-something..
Now, it's not just us that has a steel anthropological imperative,.
but Jesus embodies this..
It was actually that that Jesus would use to speak to those around him..
He would speak of the reality of being made in the image of God.
and how we are all children in his kingdom..
This is the language that Jesus would use for the people of his day.
to get them to turn towards the vulnerable communities of their day..
Now, here's the crazy thing..
It was the religious leaders and the religious communities,.
like the divine church in the first century,.
that actually did not serve, did not help,.
did not stand in the gap with vulnerable communities..
And it was because the Jewish leaders of the day,.
within that sort of understanding of Judaism at that time,.
those who were on the outsides, who were vulnerable,.
were seen as unclean, mostly..
And because they were unclean, their theology was they were unclean.
because of some sin or because of some bad choices..

$^{361}$That's why they're blind..
That's why they're crippled..
That's why they're out there..
They've done something..
Either their sin or poor choices has caused them to be on the outside of society..
And if we engage with them, we're in the risk of becoming unclean and defiled ourselves..
And so the prevailing winds of the day was that there wasn't this kind of Christian movement.
towards serving those vulnerable people..
Instead, there was this kind of at a distance, holding them away..
And for Jesus, this was an affront to the heart of God..
Jesus couldn't understand how his people had gone so far away from what actually drives God's heart..
That if these people are made in the image of God,.
then they carry that equal dignity and worth,.
and justice is to reach out to the vulnerable and broken communities.
and bring them in as much as possible because we're part of one family..
So Jesus begins to speak strongly to the religious authorities and the people around him of his day.
about the need for them to get over this concern of becoming unclean.
and actually begin to serve the very people that they put on the outside..
One of the places that you see Jesus do this the most is around the dinner table..
Interesting, Jesus, man, he loved to eat, Jesus..
In fact, a lot of scholars say that in the Gospels, Jesus was either on the way to a dinner,.
at a dinner, or leaving a dinner..
That was pretty much Jesus..
In this way, Jesus really was a true Hong Konger..
Let's be fair..
Like, you know what I'm saying, foodies in the house, you know it..
That's Jesus, right? He was always at a meal..
And Jesus was invited to meals because, of course, in the first century, that was the way things happened..
That was the social context of the day..
A lot of business was discussed that way..
A lot of teaching was done in that context..
Around the meal table was a very important place for relational connection..
Now, here's an interesting context about the meal table in Jesus' time..
The meal table was a symbol of power..
The meal table was where if you were a host and you had a table in your home,.
and you invited somebody to come for a meal,.
you were, as the host, the person in power,.
and you would invite somebody into your home who you knew, by being invited into your home,.
they were now obligated to invite you back to their home..
There was this reciprocal culture in the first century.

$^{401}$where if I open my home to you, then you are under now my debt,.
and you need to now open your home to me..
Are you with me?.
Now, because of that culture, what it meant was a lot of wealthy people would invite even more wealthy people to their home.
because they wanted to get invited to the really wealthy people's homes..
Are you with me?.
Isn't it great that we don't do this anymore?.
Come on, Hong Kong..
But we do this, and we invite people over to things, we do things,.
and we kind of think that now they're kind of obligated to invite us back,.
and that's exactly what was going on in the first century..
And so Jesus saw that the dinner table, that the hosting of it,.
that the resources of eating and drinking and community and fellowship.
was actually a power structure, a hierarchical power structure,.
and it was pushing the poor away and was only benefiting the rich and powerful..
And Jesus, being invited into a Pharisee's home,.
because the Pharisee was expecting Jesus to give him something back,.
whether pray for him, bless him, do something for his family, whatever it might be,.
Jesus begins to stand against this prevailing thought,.
and he flips everything upside down..
I want to read to you what Jesus actually says to the Pharisee and his host in Luke 14, verse 12..
Everybody okay?.
Listen to this..
Now Jesus said to his host,.
"When you give a luncheon or a dinner, do not invite your friends,.
your brothers or your relatives, or your rich neighbors,.
for if you do, they may invite you back so you will be repaid..
But when you give a banquet, invite the poor, the crippled, the lame, and the blind,.
and you will be blessed..
Although they cannot repay you, you will be repaid at the resurrection of the righteous.".
This is Jesus turning the table upside down, if you will..
He's saying to the Pharisee, "You've invited me here today because you're expecting me to give something back,.
but here's the kind of table fellowship that is the reality in my kingdom.".
And he names four communities for him to invite..
He says, "You've got to invite the poor, the crippled, the lame, and the blind.".
And those four communities were exactly the four communities that the Pharisees thought had sinned,.
and that's why they were ostracized..
Those were the communities that they didn't want to invite in because they thought they would become unclean if they invited them in..
Jesus names exactly those four..
And he says, "You want to know who should be at your dinner table?.

$^{441}$It's the ones that you're currently keeping out..
It's the ones that have nothing to give you back..
It's the ones who when they come, they're only going to receive.
because they've actually got nothing to give..
Those are the ones that should be at your table.".
What Jesus is saying here is, "Here's your mindset to vulnerable communities that don't have resources..
If you have resources, you should open your table.".
Come on, church..
If you have resources, if you have stuff, then you should open it towards those that don't..
Jesus' thinking is radical..
It's this..
"If some have food, all will eat.".
What does that mean?.
Not if some has food, everybody..
But if some have food, all will be able to eat..
Because the some that have will be generous in my kingdom,.
and they will open up towards the some that do not..
Jesus' vision for community was where everybody was invited..
And the only way that everybody could be invited and sustained.
is if those that had resources were generous towards those that would not..
Are you with me still?.
I don't know if you've noticed, but over the last month, we had a local artist..
Her name is Erin Hung..
And she came and she painted one of the windows in Pacific Coffee downstairs..
It's the window facing Chef's Blend just on the opposite side of our street..
I want to show you a photo of that window that she created..
She drew this..
I love this..
"We are all humankind.".
The idea is we are all made in the image of God, all carry value, equal dignity..
Notice the picture..
Lots of different cultures, lots of different people coming around a meal table,.
all sharing together in a meal, everybody bringing the resource they have.
so no one goes hungry, no one's left out, and everybody can connect together..
Do you see that picture?.
And that's exactly what Jesus is talking about here..
He's saying, "If you have a dinner table, this is what you should be doing..
If you've got resources and there are those without in your community,.
you should be thinking about how you can distribute those resources.
so those in the community can also be a part of this table experience.".

$^{481}$Now, I want you to see two things really quick here.
that Jesus is emphasizing through this teaching to the Pharisees..
The first is this..
Notice how he says in the passage here..
He says, "Invite the very people who cannot repay you.".
Ha! I love that..
"Invite the ones who are under no moral obligation to have to pay you back..
Invite the ones that even if they wanted to, they can't right now..
They've got nothing. There's nothing they can do to serve you..
You're literally selflessly, sacrificially serving them.".
He's saying, "Invite those people.".
This is radical economic grace..
I'll say that again..
It's radical economic grace..
It's Jesus saying, "Wouldn't it be amazing.
if we gave our resources without any expectation of return,.
without any expectation that they can give us anything back?".
This is the kind of table Jesus sees..
And I want you to know that at the branches of hope table,.
the people that are sitting around that table are asylum seekers, refugees,.
people caught in human trafficking,.
people that are dealing with the ravages of war, sexually broken,.
those that cannot get themselves up in Hong Kong culture and society,.
those are the ones around our table..
And the beautiful thing is every single donation that comes into branches of hope,.
what that donation does is it puts literal and metaphorical food on that table..
It provides literally food, food, water, sustenance in that way..
But it also provides finances to enable accommodation,.
to help stand in the gap where there are material needs but also emotional needs..
The money that's given helps some people to be able to get counseling support.
for their trauma or whatever it might be that they're going through..
Whatever is given into branches of hope, 100\% of it gets on that table.
so that those who could never pay us back.
can actually get a step up in society around us..
This is the joy of the work we do..
We get the great joy of partnering with Jesus.
in the radical economic grace that is found in his kingdom..
And whenever we do that, our reward is not with the person that we're giving to..
Our reward only will ever come at the end of all things.
when Christ brings us into his kingdom.

$^{521}$and we get to see everybody at that time,.
nobody without the differences of resources,.
but everybody coming together..
And we get to say that future should be a part of the world right now..
Because if in the future everybody is going to be equal,.
why can we not use our resources today to begin to build a picture of what that future does?.
And if we can do that, maybe people outside in the world today.
will actually look at the church and its generosity.
and see its radical economic grace and begin to think,.
"Hey, let's not exploit profits over people.".
I think I'm the only one here today. Hi guys. My name's Andrew..
Anybody else here?.
Wouldn't it be great if companies didn't exploit profits over people?.
Could it be that the church could be a place where radical economic grace is shown.
through how we open up our meal tables?.
Literal and metaphorical meal tables for the world.
as we step in the gap to help those in need..
Are you with me still?.
But here's the second thing..
And this is the one that, oh, this is so Jesus..
The second thing is he's telling the Pharisee,.
"Open up your home and invite these guests in for a meal.".
He's not talking about a takeaway restaurant..
He's not talking about a food pant or a delivery room..
He's not saying, "Write a check and just send it over to them.".
He's saying, "Invite them in and have a relationship with them.".
Because if they're going to sit at your table, you're going to be in proximity to them..
You're actually going to have to connect with them..
You're actually going to have to talk to them..
You're actually going to have to host them for a number of hours..
You're going to have to get to know their story..
A little bit like my friend who was standing at the hospital bed.
decided to get to know the person's story..
This is the intimacy of what Jesus is doing..
He's saying this is not just about some charity..
This is actually about the work of justice..
You see, charity provides crumbs from a table..
Justice allows a seat at that table..
Are you with me?.
See, for Jesus, radical economic grace was as much about relationship as it was about relief..

$^{561}$And relief was important because that needed to be there in order to bring them up into society..
But he was also trying to create an environment where the Pharisees would actually get to know the poor and the vulnerable and the marginalized.
and those that they weren't getting to know because they were staying away from them..
Jesus is saying you need proximity..
I love the fact that with Jesus, he actually is with the poor far more than he talks about the poor..
I wonder if you've ever noticed that in Scripture..
He's actually with the vulnerable and the marginalized far more than he actually talks about the vulnerable and the marginalized..
I find that deeply challenging..
As I was preparing my message today, that was deeply challenging..
I'm like, do I talk at the Vine more about justice than I do justice?.
Do I talk more about the vulnerable and the marginalized than I'm actually inviting them into my home, getting to know them?.
That's a challenge for me as your apostle..
I wonder if that's a challenge for you..
That was Jesus' challenge to the Pharisees of his time..
He was saying, are you really willing to open your table, not just so that you can provide, but you can get to know?.
See, Branches of Hope is committed to not just providing a relief to someone,.
but actually putting that person in the very place where they can flourish in society the most..
That's what drives what we do..
We're not just about providing..
We're also about positioning..
Come on, church..
Not just about providing, but also about positioning..
Providing and positioning together..
Now, this really is summarized for Branches of Hope in the area of what we call, within Branches of Hope, economic emergency relief fund..
It's our emergency relief fund..
Emergency relief fund is the idea of bringing together both relief and relationship and seeing those two things come together..
You see, there are so many people within Hong Kong society that are in need of emergency support..
Many of the clients that actually Branches of Hope supports are in need of emergency support..
See, we have finances already available to provide some of the programs that we run..
But so often when you're working with people in vulnerable communities, an emergency takes place..
Let's give you an example..
Recently, we see some of our asylum seekers are evicted all of a sudden from where they're living, and they're now homeless..
Well, in those times, Branches of Hope can pivot with their emergency relief fund and provide temporary accommodation,.
whilst longer-term accommodation is being sorted out..
Or for some of our vulnerable communities, they have a mental health issue..
A lot of them are dealing with deep traumas like Josephati that I spoke about earlier..
And they need counseling support to help them to get emotional balance again..
And again, sometimes there's an episode that happens, a psychological episode, that we at Branches of Hope,.
we can step in with those emergency relief funds and help provide the kind of care that they need..
There's also sometimes medical issues..

$^{601}$You know, we all go through life, don't we, when we get sick or something happens,.
and most of us either have insurance through work or finances that we can use to go to the doctor..
With the vulnerable people that we're serving through Branches of Hope, they don't have those funds available..
So when they get sick, they need someone to help step in and provide immediate medical relief for them..
And that's also something we do through our emergency relief fund..
Let me tell you this briefly about how this fund works..
It costs us right now about \$68,000 a month in emergency relief work..
That's an average of what it is that we pay out in this emergency relief area, about \$68,000..
Now, \$68,000 goes to serve asylum seekers, refugees, and those caught in human trafficking..
We serve about 20 clients a month with that \$68,000..
So that money goes a long way to serve 20 people..
Now, here are the ways in which we're actually serving them..
Housing, food, transport, medical expenses, legal costs, mental health support..
And over a one-year basis and an annual basis, that's roughly about \$800,000..
And here's what we're wanting to do at all of our services across today on Branch of Hope Sunday..
We want to raise that amount..
We want to raise \$800,000 today..
And if we can do that, what we're doing is we're taking a huge burden off Alex's shoulders..
And I work side by side with Alex pretty much every single day..
And I know the pressures that their team are under, the burdens that they carry,.
and the sacrificial ways in which they serve these communities..
And if we, as a community of faith, were able to say, "You know what? Our table is big enough..
Our table is a place where we can invite people in..
We have resources that we can give in a way that we don't expect anything back.".
I think if we could raise \$800,000, we're standing in the gap today..
And we're taking a huge burden off of Alex and his team's shoulders..
And we're saying that emergency money that you need on a monthly basis to pivot as soon as you need to pivot,.
that money's in the bank..
We're setting that aside for you so that you can keep your table open..
Because the worst thing that happens is when there's not enough resources, tables get closed..
But when we see the resources flowing across the kingdom, we can keep a seat open for people..
People like Matthew..
Let me tell you about Matthew..
Matthew came to Hong Kong from a South Asian country..
He was persecuted for his religion in that country..
And he fled to Hong Kong and he claimed asylum at the border here..
He was put in detention for a period of time..
And after coming out of detention, he found himself on the Kowloon side, the Chipsaw Choy side of Star Ferry..
And when he's standing there, he bumps into somebody from his country..
So he begins to chat with them..

$^{641}$Turns out that person also is here as an asylum seeker..
And that person's a client of Branches of Hope..
So Matthew gets invited to Branches of Hope..
He comes along and hears about our services..
And he's somebody that we use the emergency relief fund to help in that initial period..
Get him off the streets, in the temporary accommodation, and get him set up for his time in our city..
Matthew has become, over time, a member here of the Vy Turch..
He's with us every single Sunday..
And Matthew's also passionate about going back into the detention centers to serve the asylum seekers there..
Because he knew what it was to be there himself without anybody coming visiting him..
So he now goes on a regular basis to the detention centers in Hong Kong, visiting asylum seekers, so that he can bridge the gap for them..
Matthew is someone who was met through emergency relief fund, but now becomes a flourishing member of society..
Or I could tell you about Namala..
Namala, who comes from also a South Asian country near here..
And Namala's husband in that country was injured..
And so much so that Namala suddenly had to provide for her whole family..
And she knew that she couldn't do that by staying in her country..
So she heard about an agency that was hiring people to be domestic workers in Hong Kong..
She signs up with this agency, but the agency's a bit of a scam..
And she gets to Hong Kong and she finds herself in labor exploitation..
Her actual employer here abused her deeply..
And so much so that she makes the brave decision to flee out of the house and make her living on the street..
Well, she comes into contact with Stopper, Stop Trafficking of Persons work within Branches of Hope..
And there they're able to take her in and help her get her off the street and get her also operating again as a normal functioning person in society through the emergency relief fund that we have..
And through that process, she is now in a place where she's looking forward to actually going back to her home country and starting an NGO there that does work in the very areas that she found herself struggling with here in Hong Kong..
That's somebody who's a seed of growth..
This is the work that happens through Branches of Hope and through this idea of our emergency relief fund..
And so in a moment, I'm going to call you to be a host of a table, to be willing to dip into the resources God has given you..
And in this room, I know that there's a broad social economic demographic..
There's some people in here that are very wealthy and some of us that are not so wealthy..
Whatever it is, it's not about anything to do with that..
It's about whatever it is that God has blessed you with, regardless if it's a lot or a little, the call is the same..
You have a theoanthropological imperative to respond out of love and honor to those in the community around you who desperately need help..
And I believe that as we open up the tables, both within Branches of Hope and here at Divine Church, we will be able to stand together on behalf of these vulnerable communities to make sure that nobody feels like they are unclean or defiled, but everybody feels welcomed into a community that grows..
Amen?.
I wonder whether you would stand with me..
I'm going to invite you to do a couple of things as we do this together..
First of all, I want to get your little pouch that you had earlier that Ellison talked you through..
I want you to open up your pouch..
Inside that pouch, there is a prayer guide..

$^{681}$That prayer guide has been put together by John and Tony, our founding senior pastors here at Divine..
And that prayer guide, we're going to go through in just a moment..
And we're going to actually pray the first day of that prayer guide together..
But let me tell you about what's going to happen after that..
After that, we're going to have a time of worship together..
And I want you to understand that this is the most important part of our service..
I want everybody's attention right now..
Can everybody please look at me?.
Everybody stop moving and look at me..
Thank you..
Okay..
This is the most important part of today..
In a moment, we're going to have a time of worship..
And during that time of worship, I'm going to invite you to give..
And I believe across services, we're going to raise that \$800,000 together..
Now, as you give, you have the opportunity to do that in multiple ways..
There are baskets right here in the lower house and two in the front..
So two in the middle just here, two in the front..
In the upper house, there are two baskets at the top..
On your seat are envelopes, blue envelopes, okay?.
Those blue envelopes are a way for you to put any money in there, a check in there..
If you want to put in your credit card details and credit card amount, you can put that in there..
And in a moment when we're worshiping, you'll be able to drop that in the basket..
If you want to pay directly by credit card today,.
you can even do that with our credit card machines, our Branches of Hope team right out here on the second floor have credit card machines..
You can go to them, and you can make a one-off donation to Branches of Hope directly through your credit card, through the machines also..
And you'll be invited to do that during our worship time in a moment if that is what you would like to do..
We also have ways of giving digitally..
Those are going to be on the screen during our worship time..
So once again, you can also give digitally if you would prefer to do that..
However you might do it, we want to invite you to stand with us today..
Whatever God puts on your heart is exactly what it is that you are to give today..
But before we do that, let's go into this in a place of prayer..
And I'm going to invite Alex to lead us in day one of your prayer..
So if you get the prayer guide, open up to day one..
There's some places for you to respond in the guide as well..
So, Abdi, our daily theme for Sunday is finance..
Lord God, thank you that when you created humanity, you made us like you..
That we might show your goodness and grace to others..
But we have disregarded your plans and instead have become aggressive and look for ways to deceive and exploit others..

$^{721}$We ask you to look down on your people again, Lord God, and hear the cry of those who are enslaved, abused and downtrodden by others..
Lord, lift the veil of deceit which hides these crimes from being exposed and release those who are often trapped by their own shame..
Lord, hear my cry for justice..
We thank you for the team in Righteous of Hope and who have worked tirelessly to meet the needs of those who struggle with finance and the daily provision in life..
Thank you for the help that our government offers and for those who have generously given to help them..
We pray for the Vine Church who have generously gifted the administrative costs for Righteous of Hope, allowing for all donations to be given to help the recipients..
Lord, bless this gift..
Lord, we pray for the needs of Righteous of Hope..
We thank you for the programs that have been designed to serve those who are victims of trafficking and the asylum-seeker and refugee community..
We trust you to provide for our needs and touch the hearts of those who would be moved to support financially..
Lord, hear my prayer..
Stir up my compassion..
So I invite you just to take a moment to reflect, to pray, and when you're ready you can fill out the envelope, you can give, you can move out towards the buckets during the worship time, you can give during this time..
But let's do this in an attitude of prayer and worship..
[BLANK AUDIO].
\newpage



\section{}
\label{sec:_QaFNgOx4_0}
\textbf{2023-07-24 Conversations of Hope [\_QaFNgOx4-0].mp3}
\newline
\newline
連結: \href{https://youtube.com/watch?v=_QaFNgOx4-0}{\texttt{ https://youtube.com/watch?v=\_QaFNgOx4-0}} ~~~~ 語音日期: 2023-07-24 
\newline
\newline
\hyperref[sec:Sc_q0QFc6ec]{\small{< < < PREV SERMON < < <}}
~
\hyperref[sec:index]{\small{[返主目錄]}}
~
\hyperref[sec:n5lVginVaao]{\small{> > > NEXT SERMON > > >}}
\newline
\newline
$^{1}$[ Music ].
>> This is the first of its kind of series of art, and we're really looking forward to today,.
and I hope to be the last one to do it, because I'm not sure I'm the person I'm going to be doing it..
But I'm very excited to have all of you here with me..
I'm going to go ahead and start..
[ Applause ].
I'm going to start by saying thank you to the people who supported me..
I think some of you have known me for a long time, so I'm going to say a couple of words..
I'm going to say a couple of words..
\newpage



\section{}
\label{sec:n5lVginVaao}
\textbf{2023-08-01 Conversations of Authority [n5lVginVaao].mp3}
\newline
\newline
連結: \href{https://youtube.com/watch?v=n5lVginVaao}{\texttt{ https://youtube.com/watch?v=n5lVginVaao}} ~~~~ 語音日期: 2023-08-01 
\newline
\newline
\hyperref[sec:_QaFNgOx4_0]{\small{< < < PREV SERMON < < <}}
~
\hyperref[sec:index]{\small{[返主目錄]}}
~
\hyperref[sec:DHGUJmtAIQI]{\small{> > > NEXT SERMON > > >}}
\newline
\newline
$^{1}$[ Music ].
>> As we continue to finish the course of the four years of graduate school, we're going to be having some students come in and do a lot of reading and classroom..
And then we're going to be doing some readings, a lot of the times we do like a conference meeting, but we're going to be doing a lot of reading and classroom..
We're going to be doing a lot of reading and classroom..
So, I guess the only way I can explain a little bit about the course, the way I would say it, is I thought as long as you're reading well, you can do a lot of reading..
And I think that's the way I would say it..
So, to sum up,.
what I would like to conclude.
is that I don't know.
what I would like to conclude.
is that I don't know.
what I would like to conclude.
is that I don't know.
what I would like to conclude.
is that I don't know.
what I would like to conclude.
is that I don't know.
what I would like to conclude.
is that I don't know.
what I would like to conclude.
is that I don't know.
what I would like to conclude.
is that I don't know.
what I would like to conclude.
is that I don't know.
what I would like to conclude.
is that I don't know.
what I would like to conclude.
is that I don't know.
what I would like to conclude.
is that I don't know.
what I would like to conclude.
is that I don't know.
what I would like to conclude.
is that I don't know.
what I would like to conclude.
is that I don't know.
what I would like to conclude.
is that I don't know.
what I would like to conclude.

$^{41}$is that I don't know.
what I would like to conclude.
is that I don't know.
what I would like to conclude.
is that I don't know.
what I would like to conclude.
is that I don't know.
what I would like to conclude.
is that I don't know.
what I would like to conclude.
is that I don't know.
what I would like to conclude.
is that I don't know.
what I would like to conclude.
is that I don't know.
what I would like to conclude.
is that I don't know.
what I would like to conclude.
is that I don't know.
what I would like to conclude.
is that I don't know.
what I would like to conclude.
is that I don't know.
what I would like to conclude.
is that I don't know.
what I would like to conclude.
is that I don't know.
what I would like to conclude.
is that I don't know.
what I would like to conclude.
is that I don't know.
what I would like to conclude.
is that I don't know.
what I would like to conclude.
is that I don't know.
what I would like to conclude.
is that I don't know.
what I would like to conclude.
is that I don't know.
what I would like to conclude.

$^{81}$is that I don't know.
what I would like to conclude.
is that I don't know.
what I would like to conclude.
is that I don't know.
what I would like to conclude.
is that I don't know.
what I would like to conclude.
is that I don't know.
what I would like to conclude.
is that I don't know.
what I would like to conclude.
is that I don't know.
what I would like to conclude.
is that I don't know.
what I would like to conclude.
is that I don't know.
what I would like to conclude.
is that I don't know.
what I would like to conclude.
is that I don't know.
what I would like to conclude.
is that I don't know.
what I would like to conclude.
is that I don't know.
what I would like to conclude.
is that I don't know.
what I would like to conclude.
is that I don't know.
what I would like to conclude.
is that I don't know.
what I would like to conclude.
is that I don't know.
what I would like to conclude.
is that I don't know.
what I would like to conclude.
is that I don't know.
what I would like to conclude.
is that I don't know.
what I would like to conclude.

$^{121}$is that I don't know.
what I would like to conclude.
is that I don't know.
what I would like to conclude.
is that I don't know.
what I would like to conclude.
is that I don't know.
what I would like to conclude.
is that I don't know.
what I would like to conclude.
is that I don't know.
what I would like to conclude.
is that I don't know.
what I would like to conclude.
is that I don't know.
what I would like to conclude.
is that I don't know.
what I would like to conclude.
is that I don't know.
what I would like to conclude.
is that I don't know.
what I would like to conclude.
is that I don't know.
what I would like to conclude.
is that I don't know.
what I would like to conclude.
is that I don't know.
what I would like to conclude.
is that I don't know.
what I would like to conclude.
is that I don't know.
what I would like to conclude.
is that I don't know.
what I would like to conclude.
is that I don't know.
what I would like to conclude.
is that I don't know.
what I would like to conclude.
is that I don't know.
what I would like to conclude.

$^{161}$is that I don't know.
what I would like to conclude.
is that I don't know.
what I would like to conclude.
is that I don't know.
what I would like to conclude.
is that I don't know.
what I would like to conclude.
is that I don't know.
what I would like to conclude.
is that I don't know.
what I would like to conclude.
is that I don't know.
what I would like to conclude.
is that I don't know.
what I would like to conclude.
is that I don't know.
what I would like to conclude.
is that I don't know.
what I would like to conclude.
is that I don't know.
what I would like to conclude.
is that I don't know.
what I would like to conclude.
is that I don't know.
what I would like to conclude.
is that I don't know.
what I would like to conclude.
is that I don't know.
what I would like to conclude.
is that I don't know.
what I would like to conclude.
is that I don't know.
what I would like to conclude.
is that I don't know.
what I would like to conclude.
is that I don't know.
what I would like to conclude.
is that I don't know.
what I would like to conclude.

$^{201}$is that I don't know.
what I would like to conclude.
is that I don't know.
what I would like to conclude.
is that I don't know.
what I would like to conclude.
is that I don't know.
what I would like to conclude.
is that I don't know.
what I would like to conclude.
is that I don't know.
what I would like to conclude.
is that I don't know.
what I would like to conclude.
is that I don't know.
what I would like to conclude.
is that I don't know.
what I would like to conclude.
is that I don't know.
what I would like to conclude.
is that I don't know.
what I would like to conclude.
is that I don't know.
what I would like to conclude.
is that I don't know.
what I would like to conclude.
is that I don't know.
what I would like to conclude.
is that I don't know.
what I would like to conclude.
is that I don't know.
what I would like to conclude.
is that I don't know.
what I would like to conclude.
is that I don't know.
what I would like to conclude.
is that I don't know.
what I would like to conclude.
is that I don't know.
what I would like to conclude.

$^{241}$is that I don't know.
what I would like to conclude.
is that I don't know.
what I would like to conclude.
is that I don't know.
what I would like to conclude.
is that I don't know.
what I would like to conclude.
is that I don't know.
what I would like to conclude.
is that I don't know.
what I would like to conclude.
is that I don't know.
what I would like to conclude.
is that I don't know.
what I would like to conclude.
is that I don't know.
what I would like to conclude.
is that I don't know.
what I would like to conclude.
is that I don't know.
what I would like to conclude.
is that I don't know.
what I would like to conclude.
is that I don't know.
what I would like to conclude.
is that I don't know.
what I would like to conclude.
is that I don't know.
what I would like to conclude.
is that I don't know.
what I would like to conclude.
is that I don't know.
what I would like to conclude.
is that I don't know.
what I would like to conclude.
is that I don't know.
what I would like to conclude.
is that I don't know.
what I would like to conclude.

$^{281}$is that I don't know.
what I would like to conclude.
is that I don't know.
what I would like to conclude.
is that I don't know.
what I would like to conclude.
is that I don't know.
what I would like to conclude.
is that I don't know.
what I would like to conclude.
is that I don't know.
what I would like to conclude.
is that I don't know.
what I would like to conclude.
is that I don't know.
what I would like to conclude.
is that I don't know.
what I would like to conclude.
is that I don't know.
what I would like to conclude.
is that I don't know.
what I would like to conclude.
is that I don't know.
what I would like to conclude.
is that I don't know.
what I would like to conclude.
is that I don't know.
what I would like to conclude.
is that I don't know.
what I would like to conclude.
is that I don't know.
what I would like to conclude.
is that I don't know.
what I would like to conclude.
is that I don't know.
what I would like to conclude.
is that I don't know.
what I would like to conclude.
is that I don't know.
what I would like to conclude.

$^{321}$is that I don't know.
what I would like to conclude.
is that I don't know.
what I would like to conclude.
is that I don't know.
what I would like to conclude.
is that I don't know.
what I would like to conclude.
is that I don't know.
what I would like to conclude.
is that I don't know.
what I would like to conclude.
is that I don't know.
what I would like to conclude.
is that I don't know.
what I would like to conclude.
is that I don't know.
what I would like to conclude.
is that I don't know.
what I would like to conclude.
is that I don't know.
what I would like to conclude.
is that I don't know.
what I would like to conclude.
is that I don't know.
what I would like to conclude.
is that I don't know.
what I would like to conclude.
is that I don't know.
what I would like to conclude.
is that I don't know.
what I would like to conclude.
is that I don't know.
what I would like to conclude.
is that I don't know.
what I would like to conclude.
is that I don't know.
what I would like to conclude.
is that I don't know.
what I would like to conclude.

$^{361}$is that I don't know.
what I would like to conclude.
I'm going to say my own words..
I'm going to be discreet.
about what I'm going to say..
I'm going to be discreet.
about what I'm going to say..
I'm going to say my own words..
I've been here since June of '14..
I'm going to say my own words today..
I'm going to say my own words..
[Music].
I'm going to say my own words..
So, I'm going to say my own words..
I'm going to say my own words..
This exhibition is a work for me..
It's a work for me..
I'm not a work for me..
I'm a work for me..
So as long as there are no more people who are trying to kill you,.
you're going to be constantly fighting for what you believe..
And the confession is the best way to find someone..
And the confession is not easy to do, but it's worth it..
And the confession is where the hate is really at its worst..
And it's worth us to be there..
I think that's the best way to sort of organize the force of the transition to violence..
So we are trying to get the truth out of this..
We are trying to reach out to the people who are not going to be the best for them..
So we're trying to be a school for them..
And I tell you that we've got to get people to the right and the right is good..
And we have to get people to the right who are going to be the best for us..
We are the only people who are going to have the right to support someone who's expecting to screw them..
And we have to get people to the right who are going to be the best for us..
So I think it's really important that we focus on the best,.
not on the hardest thing we can do, but on the things that matter..
These things that matter, that cause you, that work for you,.
those things that make your heart sing,.
those things that make your body move, those things that change..
Because it's not just about the things that matter..
We've got to focus on the thing that is important,.

$^{401}$that the best is always the best..
That's why I don't like to watch the big things..
I like to watch the small things..
I like to watch the big things..
[Music].
\newpage



\section{}
\label{sec:DHGUJmtAIQI}
\textbf{2023-08-13 Conversations of Truth [DHGUJmtAIQI].mp3}
\newline
\newline
連結: \href{https://youtube.com/watch?v=DHGUJmtAIQI}{\texttt{ https://youtube.com/watch?v=DHGUJmtAIQI}} ~~~~ 語音日期: 2023-08-13 
\newline
\newline
\hyperref[sec:n5lVginVaao]{\small{< < < PREV SERMON < < <}}
~
\hyperref[sec:index]{\small{[返主目錄]}}
~
\hyperref[sec:6uo9Hz2W3WU]{\small{> > > NEXT SERMON > > >}}
\newline
\newline
$^{1}$[ Music ].
>> Well, we are in our final sermon for our series.
of Blowing Water..
Are you ready?.
Let's welcome Pastor Ellison..
[ Applause ].
So, Ellison started us off..
Ellison curated this sermon series.
and Ellison will now finish..
Can we pray for you?.
>> Yes, please..
>> Father, we thank you for Ellison and we thank you.
for his humility and for his prayerful approach.
to your throne room..
I pray, Father God, now as he releases this word.
that our hearts would be open to receive truth..
In Jesus' name, amen..
>> Amen. Thank you, Carla..
Good morning, church..
It's good to be with you guys..
We're celebrating my son's third birthday at home today..
So, this morning there was a lot of bubbles.
and balloons and everything..
Yeah, so it's been a good morning so far..
Isaiah turned three..
Yes, so we're very tired but very glad.
to be here this morning..
But it has, it's been a great honor to journey with you.
through this summer over this sermon series and I hope.
that it has led to you having better conversations.
with the people around you,.
especially conversations about Jesus..
The hope is that you've been reminded and inspired to know.
that a good conversation can be life-giving,.
can even be life-changing,.
especially if it's centered upon who Jesus is..
And so, we hope that this is an ongoing thing.
that you'll take with you as you go about..
But this series, you know, we have, like Carla says,.
it's been called Blowing Water as a play on the idea.

$^{41}$of having conversations and, you know,.
with chit-chatting with the people around you..
But actually, the term chozo, right,.
to blow water can actually also mean something else..
Right, because it can actually also mean to sort.
of tell a little bit of a lie, right,.
to exaggerate the story a little bit, to embellish the truth..
Now, that's a weird word though, truth..
It's such a problematic word, right..
What is truth?.
How would you define truth?.
Right, this is an answer that, this is a question.
that people have been attempting to answer for many,.
many years in many, many different ways..
Right, so one area where you might be able to find truth is.
in the area of mathematics, right..
Okay, don't switch off, okay..
One plus one equals two..
Simple mathematic, we've all learned this in school..
Simple arithmetic, even a toddler can figure.
out one plus one is two..
Now, you see on the screen, you know it, but is it really true?.
Does one plus one always equal two?.
Are there times when one plus one does not equal two?.
So, for example, let's say I had one glass of water,.
I had another glass of water, so two cups of water,.
but if I poured them into one cup,.
then one plus one doesn't equal two..
One plus one somehow became one again..
All right, so maybe math doesn't always give us the truth..
How about logical statements, right..
Those of you who studied a bit of logic at university.
or in high school, something like that, right..
One of the things that people say is that statements.
that are true should lead to a true conclusion..
For example, some Hong Kongers are women..
Andy Lau is a Hong Konger..
Therefore, Lau is a woman, right..
Now, I don't think that's not true either..
Side note, by the way, I have a very interesting encounter.

$^{81}$with Andy Lau, if you guys ever want to hear about that,.
take me out for coffee, I'll tell you all about it..
Very interesting, okay..
But, okay, you can see it's actually, you know,.
it just creates more confusion..
Okay, if we just put theory aside, you know,.
all these are just theories and mathematics, you know,.
you guys just started school,.
you don't want to talk about that yet..
What about things that really have an impact on our life?.
Like, for example, for me growing up,.
there were a few things, three things my grandma.
and my mom always drilled into my mind,.
and they swore they were true, okay..
You might be able to relate, okay..
They told me these three things. Number one,.
if you sleep with a fan blowing in your face,.
you will suffocate and you will die, okay..
So, don't turn the fan on when you're sleeping, okay,.
all the time..
Number two, every piece of rice that you left uneaten will.
become a scar on your future spouse's face, right,.
which 好豆皮婆, right..
So, if you don't finish your rice,.
you're going to have an ugly wife, okay..
That was the other thing, okay..
Thirdly, don't drink cold water..
Even if it's super hot outside, do not drink cold water..
Cold water is like the worst thing for you..
If you drink cold water, oh, Caleb's agreeing, okay..
If you drink cold water, you will die, okay..
So, like, all these things to try and control,.
to try and, you know, tell me..
Needless to say, my childhood was spent very full,.
but very hot and very thirsty all the time, okay..
So, here's the truth, right..
Because it's such an elusive thing,.
that we want to try and grab it,.
we want to try and understand it,.
but we can't use our logic, we can't trust mathematics,.

$^{121}$we can't even trust the people who love us..
So, what is, how do we get to know the truth?.
And as we come to the end of the series today,.
I want us to be sure, I want us to know that the conversations.
that we've looked at through the past few weeks have been.
more than just casual conversations..
It's more than just choice..
So, it's definitely hasn't been lying or exaggeration.
or embellishing, because each and every word.
that we've looked at has been the truth..
Because they're conversations surrounding who Jesus is..
And if the truth is what we're after,.
then we'll only find our answer.
when we turn our attention to Him..
So, let's turn our attention to Scripture.
and see what we can discover..
John chapter 20, 2018, I can't remember the chapter..
Anyway, you can follow along on the screen, okay?.
Then the Jewish leaders took Jesus from Caiaphas.
to the place of the Roman governor..
By now, it was early morning,.
and to avoid ceremonial uncleanness,.
they did not enter the palace,.
because they wanted to be able to eat the Passover..
So, Pilate came out and asked them,.
"What charges do you bring against this man?".
"If we were not a criminal," they replied,.
"we would not have brought him over to you.".
Pilate said, "Take him yourselves.
"and judge him by your own law.".
"But we have no right to execute anyone," they objected..
This took place to fulfill what Jesus had said.
about the kind of death he was going to die..
Pilate then went back inside the palace,.
summoned Jesus, and asked Him,.
"Are you the King of the Jews?".
"Is this your own idea?" Jesus asked..
"Or did others talk to you about me?".
"Am I a Jew?" Pilate replied..
"Your own people and chief priests handed you over to me..

$^{161}$"What is it that you have done?".
Jesus said, "My kingdom is not of this world..
"If it were, my servants would fight.
"to prevent my arrest by the Jewish leaders..
"But my kingdom is from another place.".
"You are a king, then," said Pilate..
Jesus replied, "You say that I am king?.
"In fact, the reason I was born and came into the world.
"is to testify to the truth..
"Everyone on the side of truth listens to me.".
"What is truth?" retorted Pilate..
And with this, he went out again..
Now, before we get into the details,.
let's just get a bit of background foundation.
and context to what the passage is talking about this morning..
This passage happens in the middle.
of what is sometimes known as the Passion of Jesus..
Right, he is in his last hours before going to the cross..
And in these last hours, a lot has happened..
Jesus has spent some deeply personal.
and intimate time with his disciples..
He's washed their feet..
He's had their last supper with them..
He's prayed prayers of comfort over them.
and explained to them what's about to happen,.
that he is about to die..
But he's also explained to them the dangers.
that lie ahead for them, the hatred,.
the risk they would face as followers in the days to come..
He's been betrayed by one of his own disciples,.
arrested by a group of soldiers,.
abandoned then by the rest of them,.
and Peter, one of his closest friends,.
has just denied him three times..
He's then been mocked, slapped, and judged by the high priest..
It's been a really difficult couple of hours..
And this is where the story picks back up..
And the Jewish leaders took him from the high priest,.
that's Caiaphas, to the place of the Roman governor..
By now, it was early morning,.

$^{201}$and to avoid ceremonial uncleanness,.
they did not enter the palace.
because they wanted to be able to eat the Passover..
So one thing we've seen throughout this series.
is consistently the tension.
between the religious leaders and Jesus..
Right, they're always trying to get him..
They're always out to attack him..
From the moment Jesus starts his ministry,.
many of the religious leaders are determined.
to do whatever they can to put a stop to him,.
right, to have him arrested, to ultimately kill him,.
stop this movement, you know, stop his followers,.
and just be rid of Jesus once and for all..
And in this moment, it seems like.
they're finally about to be successful..
If there was a newspaper around at the time,.
the headline would have probably read something like,.
"Trouble-making false teacher finally apprehended.".
Right, this was going to be their big moment..
By the way, sidestep,.
the slides for this series have been amazing..
I mean, Anshan has been our designer,.
and I've just really appreciated the creativity.
she's put behind the slides..
So Anshan, if you're out there, if you're listening,.
thank you for your creativity.
to help bring these messages alive..
But this would have been the title, right?.
They finally find Jesus where they want him..
And Jesus finds himself under custody,.
face-to-face with a Roman governor..
Pilate, this was the man.
who had the authority to decide his fate..
And so from the perspective of the religious leaders,.
this was perfect..
They have Jesus exactly where they want him..
Finally, he's going to get what he deserves..
Right, after all, this was the guy.
who flipped tables in the temple..

$^{241}$This was the guy who was constantly breaking Sabbath rules..
This was the guy who hung out with sinners and prostitutes,.
tax collectors, adulterers, Gentiles,.
all those kind of unclean people..
And worst of all, this was the man who lied.
because he claimed to be one with God..
They said he was out of his mind..
They said he was the prince of demons..
But this was far from the truth.
because the truth was that throughout his whole life,.
throughout his whole ministry, Jesus never sinned..
The truth was that these religious leaders.
were the ones who were unable to see the truth.
because they were so bound up and blinded.
by their own judgmental and hypocritical hearts..
You see, there's something really ironic happening here.
that John wants us to notice..
He tells us that the religious leaders.
wanted to avoid ceremonial uncleanness..
So that's why they didn't enter the palace.
because entering into a Gentile's home,.
which the Pilate was, would have made them unclean..
And because of that, they wouldn't have been able.
to eat the Passover..
So they didn't go in to meet with Pilate.
because they wanted to be able to eat the Passover..
But do you see what's happening?.
Right, this Passover was one of the biggest celebrations.
in the Jewish calendar..
Right, this is where the Israelites commemorate.
and celebrate God bringing them out of exodus,.
right, bringing them out of slavery into freedom..
And one of the key elements in the Passover in Egypt.
was when God asked them to sacrifice a lamb.
so that they would be kept safe.
and the angel of death would pass over the homes.
that had the mark of the blood of the lamb..
And since then, after every Passover,.
lambs were slaughtered and sacrificed.
in memory of what God had done for his people..

$^{281}$So it's interesting, right?.
They wanted to be ceremonially clean,.
but at the same time in their hearts,.
they were planning to execute an innocent man..
Right, in other words,.
they just wanted to look clean on the outside..
Inside though, they were planning for evil..
They were willing to lie,.
to throw out false accusations, to condemn,.
but at the same time claiming to want to worship God..
Right, they were doing exactly what Jesus saw in them.
in the first place..
Matthew 23, "Hypocrites, you are like whitewashed tombs,.
"which look beautiful on the outside,.
"but on the inside, full of the bones of the dead.
"and everything unclean..
"In the same way, on the outside,.
"you appear to people as righteous,.
"but inside, you are full of hypocrisy and wickedness.".
And here's the irony..
As the lambs were being prepared.
to be sacrificed in the temple for the Passover,.
they couldn't see that standing before them.
was the ultimate sacrifice,.
the most purest and innocent lamb without sin..
And this time, his sacrifice wasn't just.
gonna be leading the Israelites out of Egypt,.
but it was to lead the entire world.
out of sin and into freedom..
But they weren't able to see it..
See, hypocrisy is when we believe.
or claim to believe in the truth, but we live out a lie..
And this is something we all need to be careful of.
because it's a trap that any one of us.
can very easily fall into..
Right, we want to come to church,.
we want to worship God, enjoy Christian fellowship,.
serve in the community, and all these things.
are good and great, but as we do these things, church,.
we have to ask ourselves,.

$^{321}$are our hearts in the right place?.
Because it's really, really easy.
to look good on the outside..
Now, hopefully, most of us in this room.
aren't conspiring for murder.
and put innocent people to death, right?.
We're not going to that extreme..
But this doesn't mean it's something that we can ignore..
Because I think in our context,.
it works itself out in our everyday lives..
And if we're not careful, I think we can very easily.
begin to live what I call a sort of like a double life, right?.
Doing whatever we want during the week,.
getting a spiritual high on Sunday,.
feeling a little bit of remorse,.
only to forget about it all again on Monday..
Rinse, repeat..
I know this 'cause my own experience was like this..
For many years of my life, I really felt as though.
I was living two separate identities, right?.
And church on Sunday, around my church friends,.
everyone thought I was doing great, right?.
I grew up in the church..
My third language is Christianese, okay?.
So anytime someone asks you,.
I knew exactly what to say..
Oh, how you doing?.
Oh, yes, you know, really plugged in to my small group..
I really got good at doing my QTs.
and all this kind of stuff, all those language..
I knew what people wanted to hear.
and I was able to pretend really, really well..
But I know in my heart, on the inside,.
and how I live my life, it was far from God..
And it was probably some of the most confusing.
and tiring seasons of my life..
And I thought I could keep it up,.
but in truth, I was living a lie..
Because the truth is,.
the truth is what Jesus is seeking.

$^{361}$is for people to worship him in spirit and in truth..
And what Jesus is looking for.
is people who are willing to take a good look.
and do something about the log that is in their own eye..
What Jesus is looking for.
is people who are willing to cut off their hand.
if that's what it takes for them to stop sinning..
Not that we have to be perfect, right?.
It's not calling for that,.
but the only way scripture tells us.
that we can seek and find God.
is when we do so with all our heart..
So for some of us here,.
before we have a conversation with anyone else,.
maybe the first conversation you have to have today.
is with yourself..
Are you claiming to love Jesus.
with your words and appearance,.
but in reality, your heart is far from him?.
If this is you this morning, keep listening,.
because maybe this is a wake-up call,.
a reminder to find out what it means.
to truly follow Jesus..
So let's go back to the passage and see what it has to say..
So Pilate came out and asked them,.
"What charges are you bringing against this man?.
"If we were not a criminal," they said,.
"we would not have handed him over to you.".
And Pilate said then,.
"But take him yourselves and judge him by your own law..
"But we have no right to execute anyone," they objected..
This took place to fulfill what Jesus had said.
about the kind of death he was going to die..
Right, so since they're being stubborn.
and they won't go in, Pilate has to come out..
Now, Pilate, who was Pilate?.
Pilate was the governor of this area at the time,.
but he had a reputation for being a brute and a bully..
He had a reputation for being a man.
who wasn't afraid to use violence.

$^{401}$in order to bring order back into his jurisdiction..
And he was the one with the authority to execute someone..
It was the perfect opportunity..
But what exactly has Jesus done wrong?.
Pilate wants to know..
Right, what charges are you bringing against this man?.
Right, there's a legal procedure that needs to happen here.
before we sentence him to death..
And like we said, right, these people,.
the religious leaders believe Jesus as a blasphemer,.
one who dared to put himself with equal footing with God,.
breaker of rules, a false teacher, son of the devil,.
all that kind of stuff..
But Pilate doesn't care about these things, right?.
As far as he, in his mind,.
those things are just internal Jewish squabbles..
You just squash that yourself..
Don't bring me into this..
But they're insistent, and they say,.
"If he were not a criminal," they replied,.
"we would not have handed him over to you.".
So this is one of those classic answers.
that tries to answer the question.
without really answering the question..
You know, it's like when you ask,.
when I ask my wife, "Hey, how much did that dress cost?".
And she says something like, "Well, it was on sale.".
Right?.
Okay, that doesn't really tell me anything, okay?.
It's sort of what's happening here, right?.
They're trying to convince Pilate,.
hey, if we can convince Pilate.
that this guy really is a rebel rouser, right,.
this guy's a subvert of the Roman Empire, right,.
he's gonna threaten your rule,.
he's gonna bring instability to the province,.
then maybe we can get him to sign off with our evil plans..
Right, so this was the way they thought.
they were gonna convince Pilate,.
only they have no real evidence,.

$^{441}$so they just throw out something really general,.
like, oh yeah, this guy is a criminal..
Again, something very ironic is happening here..
See, ultimately, what the Pharisees.
are trying to convince Pilate of.
is that Jesus is treasonous, he's committing treason,.
he's going against Caesar himself,.
and he's gonna be a threat to this governor..
But as they were accusing Jesus of treason,.
they were also committing treason..
Only it wasn't the earthly government.
that they were going against,.
but it was a going against God himself..
But they're blind, they can't see it..
But there's also one other thing they're blind to..
You see, this whole time,.
religious leaders at the time think they're being sly..
They think they're being clever,.
they think that, you know, this is the perfect plan,.
they're carrying out..
But the truth is this..
The reality is that this, all this that's happening,.
took place to fulfill what Jesus had said.
about the kind of death he was going to die..
So, on the surface, it looks like Jesus is under arrest..
On the surface, it looks like his fate has been sealed,.
he's been stripped of his power..
On the surface, it looks as though Jesus is done for..
But the truth is, everyone involved in this story.
are just players living out,.
acting what God has already said..
They just don't realize it..
And the truth is, despite of what things look like,.
Jesus is in full control of the situation..
In fact, Jesus could have put a stop to this at any moment..
In Matthew, he says, in Matthew 26, he says,.
"Don't you think that I can't call on my Father.
"and at once, at my disposal,.
"more than 12 legions of angels would come to my side?.
"But then how would Scripture be fulfilled.

$^{481}$"to say that it must happen this way?".
In talking about death, also, Jesus had once said,.
just that Moses lifted up the snake in the wilderness,.
so the Son of Man must be lifted up..
This is what Jesus was talking about,.
that it had to happen in this way..
Now, to understand this, we need to, again,.
go back to the Exodus..
In the Exodus, after the Israelites left Egypt.
and they were wandering out in the desert,.
the Israelites were constantly failing.
into the habit of complaining against God,.
complaining against Moses, turning to idols,.
saying things like, "Oh, it was so much better.
"when we were back in Egypt..
"Why did you lead us out here to die in the wilderness?".
And as a result, every now and then,.
God would have to send punishment to remind them.
of their wrongdoing in turning to idols.
and complaining against Him..
And one of the punishments that God sent.
was quite weird and quite unique..
He sent poisonous snakes into the camp,.
and people started getting bitten by these poisonous snakes.
and they would die..
And as this was happening,.
the Israelites would realize their sin,.
and once again, they called on God for help..
And in order to save them, God told Moses to do this..
He told Moses, "Make a snake and put it up on a pole..
"Anyone who is bitten can look at it and live.".
So then Moses made a bronze snake and put it up on a pole..
Then when anyone was bitten by a snake.
and looked at the bronze snake, they lived..
Jesus is living out this prophecy here.
because in the same way, Jesus knows that in a little while,.
like the bronze snake was gonna be hoisted up,.
he too was gonna be hoisted up..
Only it wasn't on a bronze pole..
Jesus was gonna be hoisted up on the cross..

$^{521}$And just as the Israelites received life.
when they looked at the bronze snake in the desert,.
anyone who would look up toward Jesus on the cross.
and have faith in His healing would too receive life..
All this to say, none of this was happening by accident..
Pastor Promise talked about this last week,.
but I want it to really be dwelling in our minds, Church..
Remember this well..
The truth is that Jesus is always in control..
Jesus is always in control..
Even as these men worked together.
to plot injustice and evil,.
God was gonna work all of this.
to bring about the destruction of sin and darkness..
Even what people planned for evil,.
God was gonna use it for good..
Excuse me, this is why Jesus is so confident.
as He continues to be grilled and to be questioned..
He never wavers..
Pilate goes back inside the palace..
He summons Jesus and asks Him,.
"Are you the King of the Jews?.
"Is this your own idea?" asked Jesus..
"Or did others tell you about me?".
"I'm not a Jew," Pilate replied..
"Your own people and chief priests handed you over to me..
"What is it that you have done?".
You know, when I read the Bible,.
sometimes I like to insert myself into the stories..
In this section of Scripture, I always wonder,.
what was Pilate thinking as he looked at Jesus?.
'Cause as he sees this man standing before him, right,.
Jesus at this point probably looking kind of feeble,.
kind of weak, a man whose followers.
have left Him high and dry..
What possible, in what possible realm of imagination.
could this guy be a king?.
So what Pilate's really asking is this, right?.
Who are you?.
People say you're this, people say you're that..

$^{561}$Who are you?.
Are you here to make trouble for me?.
Are you really gonna be a threat to my power?.
Who are you?.
Who is this man that's standing in front of me?.
And in true fashion, Jesus answers his question.
with another question..
"Is this your own idea?" Jesus says..
"Or did others tell you to talk about me?".
You see, Jesus answers this way.
because I think he knows Pilate's heart..
In other words, as Pilate is asking Jesus, "Who are you?".
Jesus is asking Pilate, "Do you really want to know?".
Like, is that question really coming from your heart?.
Do you really want to know who I am?.
Because if you do, if Pilate is genuinely interested,.
like the woman at the well,.
like the woman caught in adultery,.
like the Canaanite woman,.
Jesus was ready to tell him who exactly he was..
And Pilate is being offered the same opportunity here..
But let me also turn this question to us this morning..
Maybe it's a question we have to ask ourselves..
Do you know who Jesus is?.
Have you really asked that question?.
Jesus, who are you?.
And in your thinking about Jesus,.
maybe you see Jesus as a threat to your lifestyle..
Maybe it seems like Jesus is here.
to dictate how you live your life, right?.
To derail your plans, to threaten the authority you have.
over your own life right now..
And I know this is a stumbling block for many,.
to fully give the life to Jesus,.
because maybe you're in a good place..
Things are going well,.
and to invite Jesus into the equation.
seems like it's gonna be a threat to how you live your life..
But I'm gonna ask you again,.
because how you answer this question.

$^{601}$will 100\% have an effect on how your life turns out,.
how you live your life..
Do you really want to know who Jesus is?.
Or are you seeking Jesus for some other reason?.
Maybe your parents force you to..
Maybe your spouse drags you to church every week.
and say, "Come on, let's just go.".
Okay, fine, I'll go..
Maybe there's someone you're interested in.
that's a Christian, and you're here just to,.
you know, trying to get to know them a little bit better..
Maybe you come to church 'cause you like the music,.
and we have a good kids program..
There could be a whole bunch of reasons.
why we come in the name to church and to seek Jesus..
And we love you, God loves you,.
and we're so glad you're here..
But you're being challenged here today.
to check your hearts and motives,.
and be honest before God..
And if you've never seriously asked yourself this question,.
or you're in a place where you need a good reminder,.
then perhaps today is the day you open up yourself to Him..
Let me ask us again..
Do you really want to know who Jesus is?.
Well, let me try and help us out a little bit..
Because knowing the truth to this question,.
like I said, is not only gonna change your life,.
it's gonna have the most amazing impact on your life,.
now and forevermore..
Jesus said, "My kingdom is not of this world..
"If it were, my servants would fight.
"to prevent my arrest by the Jewish leaders..
"But now my kingdom is from another place.".
Ah, you are a king then, said Pilate..
Jesus answered, "You say that I am a king..
"In fact, the reason I was born and came into the world.
"is to testify to the truth..
"Everyone on the side of truth listens to me.".
Jesus finally tells Pilate plainly who He is..

$^{641}$Yes, I am a king..
I am a king..
Jesus has a kingdom, but His kingdom.
isn't like the sort of kingdom that He has in mind..
Jesus' kingdom is not like the kingdoms of this world..
His kingdom isn't one that's been secured by war,.
violence, political maneuvering, like earthly kingdoms are..
See, in that time, this is how kingdoms were won, right?.
Through violence, through power, through control,.
through defeating the enemy with your army..
But Jesus has always been adamant.
that this is not what I'm about..
Not that He's not capable to fight..
We said just now He was very capable of that..
He could dominate easily in that way if He wanted to,.
but the kingship and kingdom of Jesus.
was gonna be like nothing this world has ever seen..
So how does this play in our context?.
We don't really live in land ruled by kings and queens,.
but other things do rule our lives and our culture,.
and other forces attempt to exert.
similar control over our lives..
So what do our kingdoms look like?.
Our kingdoms are ruled by our own comfort,.
love of money and power and popularity..
Our kingdoms are dictated by our own selfish desires..
Our kingdoms appeal to our senses and say it's okay.
to follow the lust of your eyes, the desires of your flesh..
Our kingdoms tell us you only live once,.
so just go and do whatever you want to do..
Our kingdoms tell us that there are many ways.
to know the truth, that you are the captain of your destiny..
But Jesus says my kingdom is from a different place..
My kingdom functions differently.
because it's not of this world, it's for this world..
And my kingdom is here to show you.
that we don't have to be enslaved.
by these things that we just talked about..
Jesus says my kingdom is the kingdom of truth..
And because I am king,.

$^{681}$and because my kingdom is represented by truth,.
if you want to know the truth, listen to me..
"What is truth?" retorted Pilate..
With this, he went out..
Unfortunately, it was only a half-hearted rhetorical question.
and every time I get to this part in passage,.
what is truth, I can't help but to feel.
like a twinge of sorrow for Pilate.
because here he is face to face with Jesus.
with every opportunity to find out the definition.
of what truth was and yet he fails to figure it out..
He fails to feel it..
He fails to experience it..
He fails to understand it..
He fails to believe in it and he walks away..
But like I said, today might be the day for you..
And if today you found yourself asking that question,.
if God's speaking to your heart, do not walk away.
because here's the truth..
The truth is this..
The truth is that you were not made for this kingdom,.
the kingdom of this world,.
but you were created for the kingdom of Jesus..
The truth is that Jesus loves you..
He doesn't condemn you for the things.
that you yourself are not proud of..
And the truth is Jesus wants to change your life.
for the better..
He wants to give you a testimony.
so that you can proclaim what he has done to other people..
The truth is that everyone, everyone is welcome.
in the family of God and God will never discriminate.
against you no matter who you are..
The truth is that God is with you..
Even as you have to go and fight through the most horrible.
and darkest of times, God is with you..
And the truth is God is also with us.
as we struggle through trials and temptations..
God is there to lead us and to guide us..
The truth is that God is with you through it all.

$^{721}$and that he works all things for your good..
And the truth is that Jesus came into this world.
because he loves you to die for your sins,.
to bear the punishment that we couldn't bear.
so we too can have a place in Jesus' kingdom..
The truth is church, there's only one way to live.
that will leave you truly fulfilled..
And that's through listening, obeying, and loving Jesus..
You see, when we come into a relationship with Jesus,.
you'll find the truth is not some sort.
of far-fetched philosophical concept..
It's not religious stories made up to scare you.
to behave better..
It's not the strongest opinion or the loudest voice..
It's not a method or a scientific formula..
The truth is Jesus is the way, the truth, and life..
And salvation and freedom are found in him and him alone..
There's a very precious gift that we have here, church..
If you know Jesus, you hold the truth within you..
And this is the truth that everyone is looking for..
Everyone needs to hear this..
And since we have it, we've been told to share it,.
to declare it, for conversations of truth.
is what this world truly needs..
So I pray if you know Jesus and you love Jesus.
and you're following the truth,.
continue to blow water in his name.
and let more people know about who he is..
But perhaps you're in this room today.
and his truth is something you once held near..
But for whatever reasons, you've lost sight of it..
And you started going on your own path..
And somewhere along the way, you've got a bit lost.
and it's hard for you to know what the truth is anymore..
Today might be the time for you to find it again,.
to believe again, to come back to the path.
that Jesus has set before you, to talk to Jesus again.
and hear that familiar voice of love calling out your name..
And maybe for some of us in this room,.
this is the first time you're hearing about this..

$^{761}$And maybe God's been speaking to you..
You felt something in your heart, in your mind,.
and you say, yes, actually, this is what.
I've been seeking for..
This is what I've been longing for..
This is what I've been needing in my whole life..
I've never heard it like this.
or I've never understood it like this,.
but I've heard Jesus speak to me.
and your heart is changing, your mind is changing..
I want you to know that that's not me speaking to you,.
that's Jesus..
Because he wants you to know the truth..
And he's been waiting, he's been ready,.
his love has been there for you all this time.
to bring you out of the darkness into the light,.
to truly show you what your life on this earth is about,.
to give meaning and purpose and hope to your life..
So I wonder, Church, if we could just.
close our eyes for a second and just take a moment.
to think, where are you in this journey?.
You know, there's so many different things.
in this world that distract us.
and it is easy to stop putting our faith in other things..
It is easy to go astray and follow another voice instead..
And it is a long journey sometimes.
of just seeking and seeking and trying to find the truth.
and the meaning of why am I on this earth?.
And is there a God that really loves me?.
(gentle music).
Why am I on this earth?.
And is there a God that really loves me?.
Wherever we are in this journey,.
the true destination for every person.
that's ever walked on this earth,.
what you were created for was a life-giving,.
life-changing relationship with Jesus..
That's the only true way to live..
And so, Father, wherever our hearts are,.
I pray that you would lead us towards you..

$^{801}$(gentle music).
And Jesus, I pray especially for those.
who are maybe making that decision for the first time today.
to lay everything else down,.
to lay down their own plans and things.
which they thought were right and they thought were true,.
that they would respond to your voice,.
they would listen to you and join the side of truth,.
join the side of life, join the side of light,.
join the side of love because that's where you are, Lord..
So, Jesus, I thank you that you are for us.
and not against us..
Help us to hear, to see, to follow the truth.
and that's you, Lord..
In your name we pray, amen..
(gentle music).
\newpage



\section{}
\label{sec:6uo9Hz2W3WU}
\textbf{2023-08-21 EXODUS - 11 Hard Hearts and Sacred Arts [6uo9Hz2W3WU].mp3}
\newline
\newline
連結: \href{https://youtube.com/watch?v=6uo9Hz2W3WU}{\texttt{ https://youtube.com/watch?v=6uo9Hz2W3WU}} ~~~~ 語音日期: 2023-08-21 
\newline
\newline
\hyperref[sec:DHGUJmtAIQI]{\small{< < < PREV SERMON < < <}}
~
\hyperref[sec:index]{\small{[返主目錄]}}
~
\hyperref[sec:OFYlvA1AkKg]{\small{> > > NEXT SERMON > > >}}
\newline
\newline
$^{1}$Man, my name's Andrew..
I'm one of the pastors here, and we are so grateful that you are here, whether you're.
in the room, whether you're online with us right now, whether you're in the overflow.
just outside or anywhere else in the building right now listening to this..
We're so glad that you are a part of this, that you are here with us..
I cannot be more excited about the journey that is ahead of us as we start the second.
half of our Exodus series here today..
And I honestly believe, and I'm not just saying this, but if this is your home church, if.
this is where you fellowship, if this is the place where you're connected in community.
and you come here, I believe that the next 14 weeks of your life will truly be transformative.
to you..
Because we're not here to play church, amen?.
Let me try that again one more time..
We're not here to play church, amen?.
We're here to open our hearts and our lives to the transformative work of God's power..
And you know, I said the very first words that I said at the beginning of our Exodus.
series back in the middle of April are no more important than right now as we start.
the second half today..
And the words are this, that there is nothing more powerful than truly being set free..
Nothing more powerful for you, for your family, for your marriage, for your work, for your.
sphere of influence than you truly being set free..
Set free from the things that hold you bound to the patterns of this world..
Set free from the things that hold you enslaved to addictions and brokenness and hurt..
Set free from the things that the enemy has done, is trying to do, and will do in your.
life..
Set free to live the kind of life that Christ has always created you to be..
Over the next 14 weeks, we are unpacking what I believe is the most central spiritual journey.
that there is in our scriptures..
It's the journey of Israel going from their slavery in Egypt to their freedom in the desert,.
receiving the formation of their national identity in the desert and wilderness..
And then some 40 years later, walking with their freedom into the promised land, into.
all the things that God has for them..
In the next 14 weeks, we're journeying with Israel through all of this..
But the next 14 weeks is not about Israel..
It's about you..
It's honestly about you..
It's about your journey..
The book of Exodus is merely a backdrop to the journey that God actually has on his heart.
for you..
And that journey, as we've been talking about in this series, has five movements to it..

$^{41}$I talked about this in the first half of the series..
Slavery, promise, liberation, identity, home..
And those five things are the ways that there are five kind of movements in the book of.
Exodus in those five things..
And what we're starting today and over the next 14 weeks is the final three of those..
Liberation, identity, and home..
And incorporated within the liberation, identity, and home, which is basically chapter 7 of.
Exodus onwards, what's capsulated in there are some of the deepest spiritual truths that.
there are for not only you getting freedom, but you being able to stay in your freedom..
Paul would write many years later, "You are free..
Now be free.".
In other words, we have a tendency as humans to receive freedom, but then fall back into.
our old habits..
You're going to see over the next 14 weeks that this was Israel's issue..
That they received freedom, and yet it was so hard for them to get Egypt out of themselves..
And then they would fall back into patterns of old thinking and patterns of ways..
But in the next 14 weeks, we're going to see some things that I believe will set you..
If you hear them, if you apply them, if you think about them, pray about them, receive.
them, they're going to set you on a path not just to be free, but to truly be free..
Are you with me, church?.
The word Exodus itself, as I said right at the start of this series, means a departure..
And one of the questions I asked right at the beginning of the series I wanted to ask.
you again as we start the second half of the series today, and that is, what is it that.
you need to depart from?.
What is it that is in your heart that you're wanting to depart from?.
And our departure point, the slavery that we have that we want to leave and move away.
from will be different for every single one of us here..
Some of us it might be an addiction, an addictive behavior to something..
Some of us it might be some emotional, unhealthy emotional attachment to things..
Some of us it might be mistakes that we've made in our businesses and in our marketplace.
work and that we've never really been able to forgive ourselves for..
For some it might be abusive relationships, whether in the past or currently, that you're.
struggling to be free from..
From some it's hurt..
It's hurt that has been spoken to you, done to you, whatever that might have been that.
you've carried around for you for years and you've not been able to shake the reality.
of that hurt..
It'll be different for every single one of us..
But if you're anything like me, there is a point of departure for you..
There is something where you're saying, "I want to be free from this thing..

$^{81}$I don't want this thing to hold around my heart, hold around my life anymore..
I don't want this thing to be the thing that's going to direct my future..
I want to be liberated into the identity that Christ has me in the home that he's called.
me to..
And as I move in the trajectory of the Spirit of God, I want to feel that liberation so.
that I can not just be free, but so that I can be an instrument of liberation for others.".
And if this is your home church, then honestly the next 14 weeks, I believe, will be life.
transformative for you..
I believe the whole….
I can't express….
It's so hard to express this in a way that really is my heart..
As your pastor here, my heart is that you would be free..
My heart is so much that you would take a hold of everything we're going to be doing.
over these weeks with all the different speakers and all the different ways, with the devotional.
book that we're giving you, all the resources we're providing for you, so that you would.
taste and know that the Lord is good..
Amen?.
That's Exodus..
That's what Exodus is all about..
It's not a story about a nation in the past that we can learn some historically nice things..
Exodus is a pattern of renewal and revival for today..
And God is the same yesterday, today, and forever..
And what we see Him do for the Israelites, He is still doing for His people today, and.
He's going to do for you..
So where are we?.
Where are we in the story?.
Where are we in the Exodus today as we step into the second half of what's ahead?.
Well, to help with that, let me give you a little bit of a recap of the journey that.
we've been on so far and all the things that we kind of sense that God has been teaching.
and informing us about in the first half of the series..
We started back in mid-April in chapter one, and in chapter one, we're introduced to a.
couple of realities..
There is this Egyptian authority structure around a person called Pharaoh..
And Pharaoh has seen that this immigration group, this group of immigrants that have.
come into his nation over the last hundred years or so, have now grown to such a large.
size that he's afraid for his power..
In fact, we're introduced to the reality of the feeling of inadequacy and the feeling.
of kind of like, you know, sort of insecurity..
Because Pharaoh is totally insecure, even though he's the most powerful figure in the.
world at that point with an empire that's stretched larger than any other empire in.

$^{121}$that moment..
He sees this immigrant group, and he's frozen in fear..
The insecurities wreck him..
And what we saw in week one is that insecurity is always the primary brokenness that drives.
the oppression of others..
Any oppression that you'll ever find in your life, anything that you ever find weighted.
down on, or any enslavery that ever comes to you, comes from this root of somebody's.
insecurity..
Somebody who felt threatened, somebody who felt jealous, somebody who felt like they.
needed to do something to someone else to make those persons smaller so that they could.
be bigger..
Insecurity is the primary brokenness that drives the oppressions of others..
But what we saw in week one was this incredible thing..
Two brave Hebrew women who were midwives, who had been ordered by Pharaoh to kill all.
of the male Hebrew boys, refused to do it, drew a line in the sand, and said, "The only.
authority that we're going to stand under is the authority of God.".
And in the bravery of putting their own lives on the line, they start the Exodus journey..
We see in the beginning of chapter two, two other women who come along..
First we see Moses' mother, who now that the decree has gone beyond just killing the.
baby boys, it's now taking young Hebrew boys and drowning them in the river of the Nile..
And so she takes her child, Moses, and places him in a basket, refusing to allow him to.
be drowned, and sets him off down the Nile River, letting him go in trust of God, knowing.
that if I place him in this ark, just like how Noah had placed the animals in the ark.
and let this go, God will do something amazing..
And so God then takes another brave woman, Pharaoh's daughter..
And Pharaoh's daughter sees this Hebrew baby in the basket in the Nile, and she knows that.
she should turn the basket upside down and drown the child, but instead she stands against.
the decree of her father, takes this child, and brings this child into her household in.
order to be raised as a prince of Egypt himself..
And we learn in the second week something quite profound..
Stories of redemption and freedom so often occur in the very same places where the worst.
acts of oppression have occurred..
And Moses grows up in Pharaoh's household, conflicted..
Oh man, he is so conflicted in his identity, because he knows that his skin and his blood.
is Hebrew, but his clothes and his household and his culture is Egyptian..
And he's being raised, potentially, to take over the leadership of Egypt by Pharaoh himself..
So he's being raised within a Pharaoh culture, knowing that he's wearing Hebrew skin..
And this conflict of identity is a difficult thing for Moses..
He doesn't know, am I an oppressor or am I one of the oppressed?.
And he's caught in the middle of that tension..

$^{161}$And one day he sees an Egyptian beating up one of his fellow Hebrews, and he's so incensed.
in anger that he murders that Egyptian, and in doing that realizes he's crossed the line..
And he needs to flee from Egypt, flee from all the identity and all the things he knew,.
and he has to run away..
And we learned that week something very important, that the starting point of all of our Exodus.
is when we realize that we're all carrying broken identities around with us that cover.
up the real person that we truly are..
And part of our freedom in Exodus is to deal with some of those broken identities, to enable.
them to shift and to change in our lives..
And we see the end of chapter 2 how God begins to do that..
God shows up for the first time in the story, and he presents himself and he says, "Here's.
what I've seen..
I've seen the slavery and the misery of my people, and my heart is turned towards them.".
And he says, "I am in my compassion coming down to help them.".
And we learned something really important that week, that it's compassion, not judgment,.
that is the fuel for Exodus..
And judgment is a reality in the process of Exodus, and we're going to see that actually.
today..
But compassion is the fuel for Exodus..
Compassion is the starting point..
God didn't show up and go, "I'm going to judge.".
He shows up and says, "My heart is broken.".
So in order to respond to that, he then calls Moses..
In the middle of the desert, burning bush..
And Moses draws over in chapter 3 towards this burning bush, and he recognizes that.
there's something behind the bush..
And God speaks and says, "Remove your sandals, for the ground that you're standing on is.
holy ground.".
And what God is doing is saying, "I've moved in compassion.".
But the first point is to meet with Moses intimately on that broken level that he's.
in, and invite him to take his sandals off so the nakedness of his foot would touch the.
Spirit of God on the ground..
God was calling his people back to intimacy with him..
And we learned that week that all of our Exoduses, God doesn't call us to stand back and watch..
He invites us to draw in and connect..
God wants us to be close and intimate with him..
Exodus is not something where we go, "Oh wow, that's cool.".
Exodus is an intimate, personal journey with God..
Well not surprisingly, Moses in the rest of chapter 3 kind of freaks out a little bit..
And he kind of comes before God and he says, "I don't know if I'm adequate to do this..

$^{201}$I don't know if I've got the skills to do this..
I can't talk properly..
I'm not very much a good leader..
I've actually killed somebody before..
My resume is pretty bad.".
And God begins to speak to him and says, "It doesn't really matter about your resume..
What actually matters is about what I'm about to do with you..
And here's the hard thing, Moses..
I'm going to call you to go back to the very place that you fled from..
I'm going to call you to go back to the very place of your brokenness..
Because unless you go back there, unless you go to the very root of the place of your brokenness,.
there will be no liberation..
There will be no freedom for you.".
And we learned something that week that I think is so important that we have to understand..
That our Exodus doesn't begin when we have it all sorted out..
Our Exodus actually begins when we take the very things that we have and allow those things.
to be before God..
When we actually go back to the very places of our greatest brokenness and have the strength.
to do so..
Our Exodus actually begins when we're willing to break away from our comfort and step towards.
the unknown..
That's how it starts..
You see, it would have been easy for them to stay in the place of comfort..
But there was no Exodus for Moses in the place of comfort..
His comfort was the desert where he had raised a new family..
And God says, "No, you need to go back to Egypt and get into a place where you're uncomfortable,.
where you're going to have to deal with some of the stuff that is buried deep inside of.
you..
Only then will you be truly free.".
And we learned that that has to be our journey too..
As hard as it might be, we have to allow the Holy Spirit to uproot the deepest things in.
us if we truly want to be free..
Well in chapter 4, Moses turns to God and says, "Well, how am I going to do this?".
God says to Moses, "Well, what is in your hand?".
And he realizes he's got a shepherd's staff in his hand and God begins to do all these.
crazy things with the shepherd's staff in his hands..
And he's trying to teach Moses something that we learned that week..
That it's not actually, again, about who we are having it all together..
It's actually about this reality that in taking the things that God has given us, even the.
broken things in us, and relinquishing control of them, that's when God can step in and begin.

$^{241}$to do it..
And we learned that week that some of us actually don't move into Exodus because we're trying.
to control, we're trying to establish that control for ourselves..
And it's in the relinquishing of control, giving it to God, that we actually enable.
Him to show us what control truly looks like, being free with Him..
Also in the confidence of this idea that God is with them..
In chapter 5, Moses and Aaron, they actually go before Pharaoh for the first time and they.
say, "Let my people go.".
It's a bold, declarative statement and Pharaoh's like, "Who are you?.
Who are you?.
Who is this God?.
I don't know this God..
I'm not going to let your people go..
In fact, just because of your arrogance and asking that, I'm going to double down on my.
slavery for them..
I'm going to make it even worse for them.".
And we learned that week that sometimes in our journeys of Exodus, it actually gets worse.
before it gets better..
It's not always a straight path, that there are curves and there are twists and sometimes.
it's harder before it gets better..
And at the end of chapter 5, it's really difficult for Moses and Aaron because not only does.
Pharaoh hate them, but their own people start to hate them as well..
"Oh, you've come here to try to help us, but now you're making it worse for us.".
At the beginning of chapter 6, there's this cry in Moses and Aaron..
They feel disillusioned and they feel hated and they feel hurt and they feel like they've.
been a failure..
And God shows up again with His compassion and says, "Just you wait.".
He says, "You have to understand my heart and my nature..
You have to understand who I am and how I stand with you..
I'm going to do some things..
And just because I call you into freedom, it doesn't mean the journey is going to be.
pain-free..
Just because I call you into a place of freedom, it doesn't mean it's going to be easy, pain-free,.
suddenly like this..
I'm not Santa Claus, I'm God..
And I'm going to come and work in such a way that over time you're going to be rebuilt.
to the person that I want you to be, but it's going to require a submission to the journey.
that I have you on.".
And in chapter 6, he tells them what that journey is going to be like..
He says, "I'm going to take you as my own children..

$^{281}$I'm going to bless you..
I'm going to break off the chains from you, and then I'm going to give you new life.".
And he lays out the process of Exodus, this idea of taking, blessing, breaking, and giving..
And we saw that week that when Jesus is in the upper room just before he's about to get.
arrested and he's doing the Passover meal with his disciples, he picks up from Exodus.
chapter 6 that process..
And he takes the bread in the cup, and he blesses it, and he breaks the bread, and he.
gives it to his disciples, saying, "This is the way that freedom and liberation always.
come..
This is what the cross of Jesus is going to be about..
The life, death, and resurrection of Jesus is an Exodus for everybody, because God takes,.
he blesses, he breaks, and he gives..
Amen.".
Wow, that was about 10 weeks of teaching in 5 minutes..
Thank you, thank you, thank you, thank you..
All right..
You're caught up on Exodus..
But the question is, what happens next?.
And as we step into chapter 7 today, we step into this moment where Aaron and Moses are.
called once again to go before Pharaoh and say, "Let my people go.".
Can you imagine how they must have felt?.
I mean, this must have been the worst thing for them..
Because last time when they went before Pharaoh and said, "Let my people go," Pharaoh doubled.
down on his slavery on the Israelites..
And now here they are, and God says, "Okay, it's time to go again..
You're going to go in front of Pharaoh again.".
And you're probably thinking, Moses and Aaron, if you're Moses and Aaron, you're probably.
thinking this, "I don't know if this is going to work.".
Are you with me?.
And perhaps, as we start this 14-week journey ahead of us, and as I've been standing here.
telling you that this is going to be the most transformative thing in your life and that.
God is going to move you into new freedoms that you've never felt before, perhaps some.
of you, maybe you're into the Exodus and you've been following the series and everything's.
been great so far, but there's something in your mind where you're like, "I don't know.
if this is going to work..
I don't know if God's really going to come through for me.".
And the same feeling that Moses and Aaron felt as they stepped before Pharaoh again.
is perhaps the same feeling that many of us in this room feel as we step into these 14.
weeks..
"I don't know if God's really going to do this..

$^{321}$I'd like to get freedom from my addiction..
I'd like to leave those pains behind me, but I'm not even sure if I can do that..
I'm not sure if God's going to do it..
I'm not sure if He's going to come through.".
And we're filled with this questioning..
And if that's you, I want you to know you're in a good place, because Moses and Aaron had.
no idea as they stepped towards Pharaoh in chapter 7 what was going to happen in 7, 8,.
9, and 10..
They had no idea that the plagues were going to come, that the Passover would happen, that.
God would redeem them and release them, that they would see the waters of the Red Sea parted,.
that they would go to Sinai and receive the law..
I mean, they had no idea about everything that we know is in the story ahead..
And I want you to understand that you have no idea what God has reserved for your story.
in the next 14 weeks..
He's got so much ahead..
It's almost like if God could take you and plug you into His brain to see everything.
that's about to happen for you in the next 14 weeks, you'd be, "Woohoo!".
But maybe right now you're like, "I'm not even sure if this is going to work.".
Anyone with me?.
So I want to pray as we open God's Word together today that God will make this work for you..
Father, I just thank you so much for the people in this room online, the overflow, and anywhere.
else that they're listening..
Father, we just are grateful that you are at work..
And Lord, we come to you like Moses and Aaron, maybe with that question, "I don't know..
I don't know if I can get free from this.".
But Lord, we come to you also in that faith, that little mustard seed of faith that Moses.
and Aaron had, knowing that you can do immeasurably more than we could ever ask or imagine..
We thank you for this..
In Jesus' name, amen..
Let me read this to you..
Everybody okay?.
All right..
This is Exodus chapter 6, starting 28 into chapter 7..
"Now when the Lord spoke to Moses in Egypt, he said to him, 'I am the Lord..
Tell Pharaoh, the king of Egypt, everything I'm about to tell you.'.
But Moses said to the Lord, 'Since I speak with faltering lips, why would Pharaoh l-l-l-l-listen.
to me?'".
I added a little bit there, by the way, for dramatic effect..
"Then the Lord said to Moses, 'See, I have made you like God to Pharaoh, and your brother.
Aaron will be your prophet..

$^{361}$You are to say to him, 'Everything I command you and your brother Aaron is to tell Pharaoh.
to let the Israelites go out of his country.'.
But I will harden Pharaoh's heart, and though I multiply my miraculous signs and wonders.
in Egypt, he will not listen to you..
Then I will lay my hand on Egypt, and with mighty acts of judgment I will bring out my.
divisions, my people, the Israelites, and the Egyptians will know that I am the Lord..
When I stretch out my hand against Egypt and bring the Israelites out of it, Moses and.
Aaron did just as the Lord commanded them.".
An amazing set of Scriptures here..
And in here is the starting point of the liberation phase of Exodus..
Because in here we see some pretty interesting things..
We see an idea of hardened hearts, and we see a man who has faltering lips and God telling.
him to do it anyway..
We see judgment and justice, and we see things that maybe when we read it we're like, "That.
sounds a little harsh..
How does that all work out?".
And here's the crazy thing..
The thing that actually unlocks this passage is not actually a deeper dive into God himself..
The thing that actually unlocks this passage is a deeper dive into Pharaoh..
Because so much of what God does in just these small verses right here is related to who.
Pharaoh was and the kind of culture in which Pharaoh operated..
And so to understand these passages here, we have to actually understand a bit more.
about Pharaoh..
And to help you with that, I want to take you back to Egypt now, back to the very land.
of the Pharaohs..
And I want to take you to one of the most famous places in all the world so that you.
can understand a bit more about the culture that surrounded Pharaoh himself..
Let's have a look..
Pharaoh..
It was a title that demanded respect, power, and fear..
Originally derived from the Egyptian compound, which means "great house," the title came.
to mean the "Great High One" or the ruler of the Great High House..
It was used as early as the first dynasty and designated to the one person in Egyptian.
society who had all ruling power and authority over everyone else..
He owned all the land in Egypt, enacted all the laws, collected all the taxes, and defended.
Egypt from invaders as the commander-in-chief of the army..
In other words, he was the central figure to all Egyptian life, culture, and society..
But importantly, he was more than that..
You see, in Egyptian life, religion was central to absolutely everything..
And one of the important roles that Pharaoh played was an intermediary between the gods.

$^{401}$and humans themselves..
So for example, in any religious ceremony in ancient Egypt, Pharaoh deputized for the.
gods..
And in this way, Pharaoh was not just in control of all of the practical life of Egypt..
He was actually in control of all of the spiritual life as well..
When Pharaoh acted, it was as if the gods themselves were acting..
And in this way, Pharaoh was in control of the whole universe..
So not surprisingly, the pharaohs lived in unparalleled grandeur and splendor..
In life, their palaces were designed to project power and abundance..
And in death, their pyramids were designed to do so for all eternity..
The pharaohs surrounded themselves with only what was fitting for one who was a god, and.
that was vast displays of glorious wealth and opulence..
A simple case in point can be found right here near the Sphinx in Giza..
Well, none of those ancient palaces are still intact today..
But throughout Egypt, you can actually discover elements of their wealth..
And I want to show you one of those examples right here..
This pillar is actually made of a red granite..
And this red granite is not found in any quarries in Egypt at all..
In fact, scholars believe that these pieces of granite were actually sourced from Southern.
Africa, placed on boats, sent up the Nile River, and then finally constructed here in.
Cairo..
Now, get your head around that, because all of that is designed to communicate one simple.
thing, that anyone who ever comes into this place would know, that the person who owns.
it is really, really, really powerful..
[MUSIC PLAYING].
So a pharaoh who was god, and a palace outrageous in its opulence and wealth, all designed to.
communicate power, prestige, and authority..
So stop for a moment and think what it would have been like for Aaron and Moses to receive.
the call of God to go to Pharaoh and command the Israelites to be let go..
And then stop and think what it would have been like for them to walk into the actual.
palace, to be surrounded by all that power and wealth, and actually only being there.
with the clothes on their backs and those two staffs in their hands..
Do you imagine how much fear they must have felt?.
But then here's the crazy thing..
They're in that palace, and they're confident..
They're putting their trust and their faith in God, that even in the place of Pharaoh's.
greatest comfort and authority, he would actually bend to their will..
I mean, none of that actually makes sense..
And I've come to think that faith often operates that way, often in a place where things don't.
make sense..

$^{441}$I mean, I think you could actually define faith like this, "radical obedience amidst.
crazy overwhelming odds.".
Or put it in Moses and Aaron's terms, "two feeble shepherd's staffs against red granite.
pillars.".
Now, the question you have to ask yourself is simply this, "Would you have gone?".
It really is staggering to me, the faith that they must have had..
Oh, thank you..
It's really staggering to me how they must have felt walking into that place, believing.
that Pharaoh might respond, dressed in their shepherd's clothes with their little staffs.
amongst all that opulence and power and wealth..
And in fact, Moses, when he's telling you the story, he's honest about it..
He says these things..
He says, "I'm a person of faltering lips..
How am I going to be able to do this?".
But then on the other side, God says, "I am the Lord..
Go to Pharaoh and tell him what I'm about to tell you.".
There's this tension in the passage between how Moses felt about himself and his own inadequacies.
and how God was, and how God was calling Moses not to focus on his resume, but to focus on.
God, to focus on who he is..
And he was saying, "When you go before Pharaoh, you don't go in the strength of your own power..
You don't go in the strength of how smart you are..
You don't go in the gifts and the talents that you have..
You go literally in obedience..
You go in faithfulness..
And when you go, I am the Lord..
I'm the one who's going to work..
I'm the one who's going to be powerful..
I'm the one who has all of that authority..
You don't need any of that..
You just need to be faithful.".
Which is crazy to me, because so often we don't think of it that way..
We think it is about how strong we are..
If I'm going to get my freedom, if God's going to give me Exodus, it's about how I deserve.
it..
I need to prove that I deserve it to him, or I need to work really hard, or do more.
quiet times, or whatever it might be..
We begin to think that it's defined by who we are..
Oh, I'm a person of faltering lips..
How could this ever happen?.
And God shows up and says, "It's not about you..

$^{481}$It's not about your inadequacies.".
God never calls you to be adequate for the tasks that he puts before you..
He calls you to be faithful..
And being faithful, by its very definition, means you're not adequate..
Because if you're adequate, you wouldn't need faith..
You'd just do it..
And you'd do it because you were adequate to do it..
See, God puts us in situations, circumstances, where our adequacy is not enough to achieve.
freedom..
You will not get lasting freedom in your life through your own adequacy..
It'll only come through your faithfulness to what God calls you to do, even when it's.
above and beyond what you think you can be able to do..
Are you with me, church?.
What really blows my mind about this whole scenario is that then God, later on in the.
passage, actually tells Moses and Aaron that although they're going to go and speak what.
God asked them to speak, they're going to fail..
He says, "I'm going to harden Pharaoh's heart, and he's not going to listen to you.".
Can you imagine what this is like for Moses and Aaron?.
It's like, "Okay, all right, we'll go..
We'll be obedient to what you're asking for us, and we're going to get there..
And you've already told us that it's going to fail..
What's the point in going?".
Are you with me?.
But it's because we humans define success so differently to how God does..
We define success through achievement, through outcomes, through whether something works.
out or not, or we achieve something, or we win or not..
And you can imagine Moses and Aaron going, "Well, if you told us this is going to fail,.
why would we even bother going in the first place?".
And God's like, "I define success differently.".
God does not define success by what you achieve, but how you obey..
That's the definition of success in God's eyes..
And the number one thing that will set you up in the next 14 weeks towards your freedom.
will be the small, simple moments of obedience that no one ever knows about..
I know firsthand that the greatest moments that I've ever had as a pastor have been moments.
that no one has ever seen, moments where I made a simple decision to close that website,.
a simple decision to write that check, a simple decision to whatever it might be in the workplace,.
whatever it might be here in church, whatever spiritual thing, those simple things that.
I did in my heart that no one else saw, no one else knew about, those are the places.
that when I get into heaven, God's going to go, "Wow, Andrew, you did that," not, "Look.
at how big of a church you had.".

$^{521}$Are you with me?.
Your success is defined by the obedience of your heart, not by the achievements of your.
hands..
I'm going to say that again because I just made that up and it was pretty good..
Your success is going to be defined by the obedience of your heart, not by the achievements.
of your hands in God's heart..
Now notice what happens next..
Chapter 7 verse 1, "Then the Lord said to Moses, 'See, I have made you like God to Pharaoh,.
and your brother Aaron will be your prophet..
You are to say everything I command you, and your brother Aaron is to tell Pharaoh to let.
the Israelites go out of his country.'".
Now this sounds like it's contradicting everything I just said because I just said it's not about.
Aaron and Moses' resumes and God's going to do it all, but then in the very next verse,.
God says, "I'm going to make you like God before Pharaoh," which kind of sounds like.
suddenly Moses is going to look like a god to Pharaoh..
It suddenly sounds like suddenly Moses is going to be mighty and powerful and have it.
all together and Pharaoh is going to see him and go, "Oh my gosh, there's a greater God.
here than me.".
That's not exactly what's happening at all..
In fact, the way to unpack this is to understand the Egyptian culture that sits behind it..
See, whenever anybody went into the palace of Pharaoh and went towards the throne room.
with Pharaoh and had an audience with Pharaoh, they would always address Pharaoh by the deity.
of the god that they worshipped because Pharaoh was the representation of all the deities.
on earth..
That's what we saw in the film, right?.
He represented all the deities..
So if an Egyptian servant went to Pharaoh in his court, he would say, "Oh, great and.
mighty Ra," which was the sun god..
"Oh, great and mighty Ra, I would love for you to do this for me," or, "This is what.
I say to you," or whatever..
So Pharaoh would always be addressed by the deity that was being worshipped..
Are you with me?.
So what God's doing here is really fascinating..
He says, "I'm going to send you in, and Moses, you're going to be like God in Pharaoh's presence,.
and Aaron is going to be your prophet.".
In other words, there's suddenly going to be another deity in Pharaoh's presence..
And he's not used to that because he's the only deity in the land..
But I'm going to make you like God, a deity before him..
And here's the crazy thing..
That's going to completely subvert his power..

$^{561}$That's going to completely change everything for him..
And he's not going to know how to respond or react to that..
And here's even the most beautiful thing..
You're going to go and do this dressed in smelly shepherd's clothes, and you're going.
to be like God to him..
And he's going to look at you, and he's going to go, "How can an impoverished nomadic shepherd.
be a representative of a god?".
That's what he's going to think..
How can this one be a representative of God?.
Have you ever been in a situation where you feel completely overwhelmed and completely.
inadequate?.
Recently, I was at this event..
It was a small private fundraising event, and I was sitting on a table with eight people,.
and I had the responsibility of trying to raise some funds for a charity organization.
in the city..
And around the table of eight people, three of them were US dollar billionaires..
US dollar billionaires..
I have never felt more insignificant in my life than sitting around a table with three.
US dollar billionaires who were incredibly powerful, incredibly successful, really nice.
people..
I was hoping that they would be mean and angry and horrible, but they were really nice people.
too..
Not only super wealthy, but nice..
And I'm sitting there around this table, and I'm listening to their stories, and I'm feeling.
like I'm kind of like sliding under the table a little bit, you know, like..
And as I'm sitting there, I honestly, I hear God's spirit whisper in my ear..
How could an impoverished Hong Kong church shepherd be a representative of a God?.
And I felt like what God was saying is, do you understand where true power lies?.
And these were really nice people and great people, very successful people, which I really.
admire..
But God was saying, true power does not lie in the things of this world..
True power lies in salvation, redemption, and hope, and grace, and love..
And I tell you, in this room, online, in the overflow, there are representatives of this.
God..
You need to understand that when you're called into a situation where you feel inadequate,.
overwhelmed, when you're in a spirit of influence, where you feel like you don't have any power,.
the tendency naturally is for you to shrink away, but you need to rise up in those moments..
You are a representative of the living God..
Jesus has died and shed His blood so that you would know new life..
The same spirit that was in Christ that raised Him from the dead is now in you..

$^{601}$Take your stand..
You're a representative of God's true hope..
And I think Hong Kong more than ever needs people to go, "I am a representative of the.
one true God.".
That's not arrogant on our part, but that's us putting ourselves into a place where we.
can administer the hope of Christ, knowing that we are in Christ in the world..
Are you with me?.
Now, very quickly, notice what happens next..
Verse 3 says this, "But I will harden Pharaoh's heart, and I will multiply my miraculous signs.
in wonders in Egypt..
He will not listen to you.".
This is weird because they've been called to bring freedom..
They've been called to ask for the Israelites to go, and then God says, "I'm going to harden.
Pharaoh's heart, and he's not going to listen to you.".
And scholars have debated for years and years and years around what does it mean for God.
to harden someone's heart?.
How do we get our heads around?.
Because on the surface, what it sounds like is how can a just God harden someone's heart.
and then punish them?.
That doesn't seem like a just God..
It doesn't kind of seem like a God we would want to worship..
What is happening here?.
Well, what's sitting behind all of this is the reality of judgment and the judgment that's.
going to come..
I said earlier that compassion was the fuel for Exodus, but that doesn't mean that there.
wasn't judgment within the Exodus story..
And we're going to see this week, next week, and the week after what that judgment turns.
to look like, tends to look like..
But in this moment, God says, "I'm going to harden Pharaoh's heart.".
It's important that you understand the word that Moses uses here..
It's a Hebrew word, of course..
It's the word k'cheh..
The word literally means, or k'cheh, the word literally means to fasten a hold of or.
to strengthen something..
Now what's really important about this is that you have to understand that for God to.
harden something with this word, what it means is that God strengthens the direction that.
something's already on..
In other words, that there's already been a decision, there's already been a turning.
of the heart, there's already been a posture of the heart, and God is going to come and.
fasten, strengthen that direction..

$^{641}$The modern day equivalent would be the idea of doubling down..
Somebody's made a decision, and now they're doubling down on that decision..
Does that make sense to you?.
So when it says that God hardens Pharaoh's heart, what it's really teaching us here is.
that God sees the direction of Pharaoh's heart..
He looks into Pharaoh's heart, and he sees that he's already decided never to let the.
Israelites go..
We know that because we've seen that in chapter 5..
In chapter 5 where he said, "I don't know this God..
I'm not going to bow to this..
This is not going to happen.".
He's already set the direction of his heart..
It's really important that you understand this..
When it says in the text that God hardens Pharaoh's heart, it's not saying that God.
is making Pharaoh do something he doesn't want to do..
It's not talking about God overtaking the free will of Pharaoh and suddenly taking a.
soft pliable loving heart and making it hard and nasty so that he can punish him..
What it's saying is the direction of Pharaoh's heart had already been made up..
The decision had already been made, and God's going to fasten that, strengthen that so that.
it'd be brought into a place of judgment..
I want you to see the judgment element of this..
Verse 4 onwards, "Then I will lay my hand on Egypt, and with mighty acts of judgment.
I will bring out my divisions, my people, the Israelites.".
Again, to understand this, you have to understand the Egyptian cultural context here..
At the time that Moses is standing in front of Pharaoh, there was already in circulation.
what was called the Book of the Dead in Egypt..
The Book of the Dead was a story of lots of different deities and how they interacted.
with humanity..
One of those deities was Anubis..
Anubis was the deity of the decision of the future, the eternal future of people..
In this story that's found in the Book of the Dead, this person called Ammi comes before.
Anubis, a little bit like in a throne room like Pharaoh..
As Ammi comes before Anubis, Anubis is going to make a judgment as to whether Ammi will.
have eternal good life or eternal bad life..
The way that he judges, as you can see in this beautiful hieroglyphic, is he takes his.
heart and he places his heart on a scale..
Then on the other side of the scale, there's a feather that's placed there..
It's the feather of righteousness..
That's what it's called in the Book of the Dead..
There's this heart of Ammi on one side of the scale and the feather of righteousness.

$^{681}$in the other..
The idea is that Anubis is going to judge the heaviness of Ammi's heart..
If that heart has any evil in it, if that heart has any evil actions in it, then the.
scales are going to tip in the wrong balance and then he'll be sentenced to a life of death..
That's basically the way it worked within the Egyptian thinking..
Now God knows all of this and he picks up on all of this and he says to Moses, "Aaron,.
you're going to go and I'm going to judge Pharaoh..
I'm going to harden his heart because he's already made a decision..
The scales already are imbalanced and I'm going to show that the scales reveal that.
Pharaoh's heart is wanting.".
By whether you look at it through the Hebrew culture where they know that Pharaoh's heart.
is bad because of their slavery for so many years or if you look at it through the Egyptian.
culture which already understands that there's a scale of righteousness that has to be balanced,.
Pharaoh's heart is wanting..
God is saying, "I'm going to judge that heart publicly in Egypt so that all the Egyptians.
will see exactly this, that there is an imbalance of the scales of Pharaoh's heart.".
Because actions are just, whether you look through the Hebrew lens or the Egyptian lens.
of the actions and thoughts that were in his heart..
Is that helpful?.
Now, I want you….
We're finally….
We're finishing here..
I know I'm almost done..
Are we okay for five more minutes?.
This is the best five minutes so that's why I want you to be here..
Says this, verse five, "And then the Egyptians will know that I am the Lord when I stretch.
out my hand against Egypt and bring the Israelites out of it.".
So why does God judge?.
Why is God going to judge the weight of Pharaoh's heart?.
Why does he harden the heart in order to show that judgment, to show the direction that.
Pharaoh had already decided in his own free will?.
It's so that Egypt would come to know that I'm God..
Now when you read it like you see it on the screen here, it almost sounds like God's.
like, "I'm going to judge Pharaoh and all of Egypt so they will know that I am the.
mighty God.".
You know, like it almost has this kind of resonance of like that evil laugh to it, right?.
It's not what God is doing here..
What's fascinating to me is that the word that Moses chooses for "will know" is.
the very same word that's used to speak of biblical marriage and the intimacy physically.
that's found in biblical marriage, to know someone..

$^{721}$In other words, it's an incredibly intimate word..
It's a word that is defined by relationship and personal invitation..
So God is not saying, "I'm going to judge Pharaoh in Egypt so that everybody will know.
how bad they are.".
He's saying, "I'm going to judge Pharaoh and I'm going to judge Egypt so through the.
judgment all will come to know, would know intimately, personally be available to come.
into relationship with me.".
This is Him offering grace..
This is Him saying that there is this grace element to it..
There is judgment, but mercy triumphs over judgment..
We come to see that in its fullness in the work of Jesus Christ, but even here in the.
first Exodus, we see mercy triumphing over judgment..
Yes, there will be judgment..
Yes, the heart is hard..
Yes, the heart needs to be dealt with, but there is mercy that is coming and that mercy.
is even for the Egyptians..
And here's the crazy thing..
In Exodus chapter 12, verse 38, a whole bunch of Egyptians go with Israel to the promised.
land of their own free choice because they had come to know, to see, and to taste that.
the Lord is good..
So what does all this have to say to you as we start this second phase of our Exodus.
series?.
You need to understand that God indeed has the prerogative and the ability to take a.
hard heart and harden it, but He also has the power to take a hard heart and soften.
it..
He also is a God who is able to offer to the Egyptians themselves, the ones who had.
enslaved His people for close to 400 years, and say, "If there is a kernel, a mustard.
seed of your desire to turn towards me, I will soften your heart.".
Pharaoh's heart had been made up..
His heart was already there..
God had already perceived that there was no mustard seed of faith or hope or turn or change.
for Pharaoh, so He hardens His heart towards that aim..
But for everybody else, everybody in Egypt, the offer was there..
Would you allow me to soften your heart so that you would know that I am your God?.
As we start this second phase of Exodus, that's the call on us too..
Because if you're anything like me, your heart has moments of great, beautiful softness and.
joy and it also has parts of hardness to it..
And perhaps you're here today and you recognize that there is some hardness in your heart.
too..
This whole passage has been about God examining the heart..

$^{761}$He doesn't look at the surface..
He doesn't look at the resume..
He doesn't look at the achievements..
He looks constantly at the heart of humanity..
And He looks at Pharaoh's heart and says, "Hmm.".
But He looks at your heart..
And He looks at even the hard parts of your heart..
And He looks at your hope and your faith and your desire for His mercy..
And He has the power to take even the hard things in you and soften them..
For some of you, your hearts are hard because it's been years of pain and hurt from something.
that's happened to you in the past..
For others, your heart's hard because you're continuing to make choices, habits and choices.
that bring you into sin and harden your heart..
Whatever it might be for all of us, God's grace, His goodness can soften even the hardest.
parts of who we are..
And the number one thing that you begin the liberation phase of Exodus in is to bring.
your heart before God and say, "I want to know you..
I really want to know you..
And I know that in the work of judgment and justice, you take all the things in my heart.
and you judge them fairly..
But your ultimate heart is to know..
It's for me to know you and you to know me.".
And that's how I want to pray for you as we start this journey together..
I wonder whether you'd stand with me and I'm going to pray for us..
And if this has been helpful for you today, through all the things that we've looked at,.
I want to invite you just to open your hands with me..
And Father, we come before you now in this moment..
And Father, we come before you with our hearts..
And Father, if the people in this room online, Overflow, or anything like me, there's parts.
of our hearts that are hard..
And there might be lots of different reasons why that's the case, but the reality is there's.
hardness there..
As we were praying before the service, we particularly had the idea that some people's.
hearts are hard because of the way that they feel about family members..
Maybe there's been an argument in your family or with your children or your spouse or with.
a relative in your wider circle of your family, and there's, maybe you haven't talked to them.
for years, maybe there's just something there and there's a hardness in your heart towards.
them..
And again, maybe for some very good reason for some of you, but the hardness is still.
there..

$^{801}$The Holy Spirit says, "If you really want to be free, allow me to come and weigh your.
heart.".
And you see, because of Jesus Christ, there's no feather of righteousness..
We don't follow the book of dead..
We follow Jesus Christ, his life, death, and resurrection..
So no matter how heavy, how hard our hearts might be, his scales always balance in our.
favor..
And the grace of Christ has been offered to you..
The blood of Jesus shed for you so that you would always know the softness that comes.
by his Spirit, his forgiveness, and his grace..
And so if you're here today and there's some hardness there, your starting point of Exodus.
in the second part of the series is just to invite God to come and soften you..
Half the battle is just owning the hardness, to say, "There is this part of me that I struggle.
to deal with..
I struggle to make soft.".
Holy Spirit, I come to you today, and I ask you, I confess, I recognize the hardness,.
and I ask you now to come and over the next 14 weeks, soften that hardness in me..
I can't do it in my own strength..
I'm not adequate enough for the task set before me, but you are God..
You are God..
You could do abundantly more than I could ever ask or imagine..
Soften me, Lord..
Soften me, Lord, because I want to be free..
Take a moment just to bring that prayer before the Lord, and then we're going to close our.
time in some worship together..
together..
\newpage



\section{}
\label{sec:OFYlvA1AkKg}
\textbf{2023-09-03 EXODUS - 12 The Plagues [OFYlvA1AkKg].mp3}
\newline
\newline
連結: \href{https://youtube.com/watch?v=OFYlvA1AkKg}{\texttt{ https://youtube.com/watch?v=OFYlvA1AkKg}} ~~~~ 語音日期: 2023-09-03 
\newline
\newline
\hyperref[sec:6uo9Hz2W3WU]{\small{< < < PREV SERMON < < <}}
~
\hyperref[sec:index]{\small{[返主目錄]}}
~
\hyperref[sec:ezohhaWO5XQ]{\small{> > > NEXT SERMON > > >}}
\newline
\newline
$^{1}$Well, good morning again. How's everyone?.
Now I get to be nice..
I am nice. I'm a nice person..
I hope you received that with the love and the grace that it was given to you guys..
My favorite movie when I was a kid was the 1985 Steven Spielberg classic, "The Goonies"..
Anyone ever watch "The Goonies" before? Hands in the air..
Alright, you all qualify for the elderly community group here at The Vine..
You'll get to meet with me later. I'm also a part of that group..
"The Goonies" came out in 1985 and it literally changed cinema for anybody who grew up in the 80s..
The reason why I think I love the film so much is because it was about a bunch of boys.
who were relatively my age when the film came out..
In fact, these boys right here. This is Chunk, that's Mikey, that's Mouth and Data on the other end..
And this film was fantastic..
These four boys discover a treasure map in the attic of their house.
and go on this grand adventure where they discover hidden treasure.
and they fall against ancient curses. It was cinematic glory..
Now when the film first came out, I was far too young to go to the cinema to actually watch the film..
But lo and behold, a few months later, TVB here in Hong Kong began to show "The Goonies".
as one of its movies on rotation..
But the problem was it came on at 9 o'clock at night and I was still too young..
My parents would not let me stay up late enough to watch the film..
This was very frustrating..
If you wanted to watch a film in those days and you were a kid and you weren't allowed to stay up,.
you would miss out on all of the cinematic glory that God wants you to have..
You couldn't have it..
All of that changed one day when my father came home from work.
carrying a black box under his arms which he placed under the television..
It looked like this..
I thought this was something out of space. It was so cool..
This, apparently according to my dad, had the power to record anything on television.
so you can watch it at a later time..
My dad called it a VCR recorder. I called it God..
This thing literally changed my childhood..
Suddenly all the things that I couldn't stay up late enough to watch,.
suddenly these were available to me at the fingertips, including the Goonies,.
which TVB, just a few months later after dad brought this home, actually aired again on television..
I said to my dad, "Would you please put a tape in the machine?.
Would you set the recording time?.
Would you capture the movie so I can come home the next day and watch it for the first time?".
So my dad got the tape out. He put it in. He set the recording up..

$^{41}$I cannot tell you how excited I was to come home from school that day..
I gathered all my mates from the neighborhood..
We all got together. We had a little bit of popcorn..
We put the tape in. We pushed play. And guess what?.
It was the best movie I have ever seen in my life..
It was amazing. I'd never seen anything quite like it..
The kids were my age. I was just like, "This was worth waiting for.".
My favorite scene, trust me, I wore that tape thin..
I watched it probably over a hundred different times..
My favorite scene was the very opening scene..
In the very opening scene of the movie, a black SUV meanders up a very quiet cul-de-sac of a small suburban neighborhood..
It drives past very slowly this beautiful house..
The camera pans from the black SUV up to the house,.
and there's the main star of the film, Mikey, leaning on the window sill of his bedroom..
He falls back on his bed and he says the opening words of the movie..
He says, "Oh, bummer. Nothing ever exciting happens around here anyway.".
It's really ironic because he's about to discover a treasure map in his attic.
and go on this incredible journey that will change his life,.
but he knows nothing about that in that moment..
So those opening words from the opening scene is this ironic statement about all that happens,.
and I thought Steven Spielberg is the greatest director and storyteller ever..
Well, many years later, I'm at university, and I'm living in a house with a bunch of guys,.
and one night we're bored, and so we said to each other, "What should we do?".
And I said, "I know. Let's rent the Goonies.".
And we had all watched The Goonies in our youth, but now it was years later,.
and we wanted to watch it again. We were so excited..
Now, at that time, it was on the futuristic technology of DVDs,.
so we rented the DVD of The Goonies..
By the way, if you've not watched this film, your homework this week is to watch The Goonies..
So we get the DVD, we put it in our little DVD player, we put it in, we push play..
You will never guess what happened..
The film had a beginning that I had never seen..
In fact, the film had five minutes in the beginning that I had never seen before..
I could not believe it, and it suddenly dawned on me.
that my dad must have set the VCR recorder five minutes later than the beginning of the film..
And so I had missed out on the first five minutes of the greatest film ever made..
And let me tell you, that's not just any five minutes..
It is the greatest five minutes in cinematic history,.
because the opening of the film of The Goonies is created and designed by Spielberg.
to set up everything that comes next..

$^{81}$I mean, it's crazy what happens in the first five minutes that I had missed..
There's a fake suicide in a jail cell. There's a breaking out of jail..
There's a setting fire to the perimeter of the jail..
And then the criminal dives into the back of a vehicle, and guess what that vehicle is?.
A black SUV, right?.
He dives into the back of this black SUV, and then they start to break out of jail..
The whole police in this little town in Washington in the US are trying to chase after them..
And they have this massive car chase all the way through the city..
And whilst the car is being chased by all the police,.
it introduces us to the main characters of the film..
And by the way that it speeds in and out of all of the town, it destroys so many things..
It eventually ends up, the black SUV on a beach, and it's trying to get on this beach,.
and all the police are there, and eventually they manage to get around all of that,.
and then they take a left-hand turn, and they go into a small, quiet, suburban neighborhood..
And they drift up the road and turn right, and the camera pans up, and there's Mikey..
And he falls back on his bed, and he says the opening words of the movie..
"Ah, bummer. I wish I'd seen the first five minutes originally. No.".
He said, "Ah, bummer. Nothing ever exciting happens here anyway.".
And what I thought were ironic words about what was ahead.
were actually ironic words about what had already happened..
And the depth of those words..
I mean, when he's leaning out on that windowsill, and he's looking down on that black SUV,.
the people in that black SUV are the very ones who are about to change his life forever..
And Spielberg is a master storyteller, and he weaves all of this together..
And I had missed every single bit of it..
And when I finally watched the film from the original beginning,.
the way that the storyteller wanted me to understand it,.
it completely changed how I appreciated that movie even more..
You see, the good stories, the really good stories,.
always have an intentional purpose to their beginnings..
And the storyteller, as they're trying to unpack these stories for you,.
begin in a way to shape and form a lens through which they want you to view the rest of the story..
Which means when we get the beginning wrong,.
when we start the story, if you will, from the wrong place,.
chances are everything else in that story is going to be distorted from its original intended meaning..
You have to view the story through the original lens that the storyteller wants you to approach it..
Does that make sense to you?.
And that whole process is never more important than the story we're looking at today,.
the story of the plagues..
The plagues are perhaps the most central element that happens in all of the Exodus story..

$^{121}$But they are also the most controversial elements of that story..
In the plagues, which are captured for us from chapter 7 through to chapter 12,.
we see an event after event of God's awesome power on the earth,.
of God's judgment on the earth..
And in that judgment and in that power,.
we see the way in which God uses various things and does various things to express his judgment..
And we see in that things that we're not comfortable with..
And we wrestle with these deep questions of how can a loving God be so violent,.
do things like that bring that kind of judgment?.
I thought God was peaceful and loving and good..
How would he and why would he do this?.
And if you just grab the Bible and you start in chapter 7 of Exodus,.
you start reading the plagues, you're going to be weighed down by those questions..
Because you're not actually starting in the place that the storyteller wants you to start..
You see, there's some very specific beginnings,.
intended purposes that God has created and put in the story of Scripture up to the point of the plagues.
that unless you see it through there, you're going to actually struggle deeply with the narrative..
Starting in the right place with the plagues is essential for you to understand what they're really trying to say and do..
Now, it doesn't necessarily suddenly magically wipe away some of those big, deep theological questions.
of the violent nature of God in the Old Testament..
Those things linger and remain..
But seeing it through the right lens enables us to get an understanding of what God's really trying to communicate about himself,.
about the world, and most importantly, about you..
About you and your story, you and your Exodus,.
you and the very thing that you long to depart from..
And so as we open up the story of the plagues today,.
I want to open up to you that story through its real beginning..
That beginning is found in three places..
Two in the book of Genesis and one actually in the book of Exodus itself..
In the book of Genesis, we see right at the beginning of all things God's creation of the world..
And God's creation of the world in Genesis 1 and 2 is a primary lens through which the storyteller,.
God and Moses in writing it down, is trying to help you to understand as you go through the plagues..
What you see in the creation, and I've been saying this over so many weeks of this series,.
everything that happens in Genesis 1, 2, and 3 is very much what Moses is drawing into his writing of the Exodus..
The language, the imagery, the way he communicates,.
so much of it pulls from what he had already written in Genesis 1, 2, and 3..
Now in Genesis 1 and 2, we see some critical things about how God creates..
Let me show you these things..
Light and darkness are separated..
Creatures of the sea and land are separated..

$^{161}$When we come to creation in the Bible, what we're seeing is God saying,.
"I'm in power and control on creation, and when I create, I put things in their right order..
There's an order to the universe, and I separate things so that they would be in their right place and in their right order.".
The pinnacle of this, of course, is the creation of humanity,.
formed out of the dust itself and breathed on by God's breath and brought into life..
And that humanity then is given the task to multiply and fill the earth..
It's like God is saying, "I have ordered creation in a way..
The pinnacle of that creation is humanity, and I want humanity to be a blessing to my creation.".
And humanity is to multiply and fill the earth and bless and steward the creation that I've given them,.
which is why the final point here, harmony is secure between humanity, God, and creation,.
because that harmony is God's intent and purpose..
What you see at the end of Genesis 1 and 2 is God standing back and saying, "This is very good..
Everything is ordered how I wanted it to be, and I've created humanity so that it can walk with me in bringing harmony and peace,".
the biblical word shalom, "to the world that I've created.".
And by the time you get to the end of Genesis 1 and 2, it's like there's a breathing out sigh that everything is just as it should be..
Make sense? That's your first lens..
Here's the second lens, the creation of Israel..
This is found in Genesis 12..
Let me actually read you verses 2 and 3..
"I will make you"—this is God speaking to Abram—"I will make you into a great nation, and I will bless you..
I will make your name great, and you will be a blessing..
I will bless those who bless you. Whoever curses you, I will curse, and all the people on earth will be blessed through you.".
This is God speaking out over Abram at the starting point of Israel as a nation..
And I want you to see the critical thing here..
God is focused on blessing..
He's like, "I'm going to bless you. I'm going to make your name great..
And I'm going to form and shape you in such a way that you're going to be my blessing in the world.".
In other words, the harmony that is found in Genesis 1 and 2,.
and the way in which God creates that harmony, that is going to be seen best through how a nation works together..
A little bit like how I stood before you a moment ago and said,.
"Here are some cultural things about our community that are important for the world to see the nature of God.".
God creates Israel as a community to represent His nature to the world..
"I'll bless you, and you'll be a blessing.".
But notice, "I will bless those who bless you, and whoever curses you, I will curse.".
Very important you understand this..
What God's saying is, "Out of this nation of Israel, I am going to form and shape them the way that harmony in the world should be,.
the way that my purposes and my promises should be.".
So those that bless you as a nation are aligning themselves to my purposes and promises,.
but those that curse you as a nation are standing against my purposes and promises..
So the reason why I judge the nations of the world when I judge them,.

$^{201}$is because they have moved themselves and standing against the purposes and promises I have..
How you treat my people will be a representation of what's in your heart..
Does that make sense?.
Now that's a very important lens for us to look through as we get to the plagues..
But notice one other thing here. It says, "All the people on earth will be blessed through you.".
This is so key, because even in the moment of the creation of Israel,.
where God talks about blessings and curses, relationship, and how there is judgment.
if things are not aligned to his purposes and promises,.
he then immediately provides grace..
He says, "All the people of the world will be blessed through this nation, even the ones who curse the nation..
If they curse my people, they will be cursed, but cursings can be turned into blessings with God.".
Cursings can be turned into blessings with God..
And so all the people in all the nations, everybody, will ultimately be blessed..
Now we know in the story that's through the life, death, and resurrection of Jesus..
But here in Genesis, the idea is that through Israel will come this blessing to all nations,.
and God has provided both judgment and grace..
Does that make sense to you? Anyone?.
Here's the third lens that the author wants you to look through before you get there..
It's the prologue to the plagues themselves in the book of Exodus..
It's actually found, we read it last week, in Exodus 7..
Let me read you verse 3..
This is how God declares the plagues..
He calls them, "My miraculous signs and wonders.".
I love this. Very important you get this..
God actually declares that the plagues, when he describes them at the very start to Moses and Aaron,.
he's like, "I'm about to do all these plagues, but they are," you need to understand this,.
"signs and wonders.".
That's how he describes them. They're signs and they're wonders..
And this idea of signs and wonders is a very important thread throughout the whole of the Old Testament..
A sign was always something that happens that imparted some kind of knowledge about God..
A wonder was always a supernatural event that was designed to disrupt the ordinary.
in order to bring a message from God..
So a sign imparted some kind of knowledge..
A wonder was a shaking up. We sung it in that song..
"Shake up the ground of my tradition. Shake up the ground of my religion.".
That was what wonders were given for..
To shake up the ground so that we might come into a re-understanding of the character and the nature of God..
Do you see that?.
So when God himself talks about the plagues and lays them up as signs and wonders,.
what he's essentially doing is, "There's something you can learn about me in these plagues,.

$^{241}$and these plagues are to disrupt the ordinary so that something fresh and new,.
a new message of my heart will be understood.".
Now, here's the final thing with the signs and wonders..
Signs and wonders were usually brought through prophets to the kings..
You'll see this later on in the Old Testament narrative once Israel settles itself into the promised land..
And signs and wonders were the tools of the prophets.
in order to get kings to change their ways, otherwise judgment will come..
So the plagues, yes, they are acts of judgment..
But they are also signs and wonders provided by God to offer grace to his people and to Egypt.
to turn from their ways before the true final judgment comes..
The plagues are actually God's way of saying, "Will you listen? Will you see?.
Will you understand the story behind the story, or will you begin the movie five minutes in?".
Now, the way that Egypt had set up their film is they didn't want the movie of Yahweh at all..
For them, they had come up with their own system of understanding, harmony in the world..
And what we see in Genesis 3 is really important because we see this in the nature of Egypt at that time..
Genesis 3, Satan comes to Adam and Eve and says, "You can take this fruit and you can eat it..
And in doing so, you'll discover that there's other ways of thinking about the world.
than the way that God has created you to think about it.".
And Adam and Eve are tempted to take that fruit because they believe that God is holding out on them,.
that there is an alternative way of worship..
And what you see, fast forward in all the years when you get to Egypt,.
is that Egypt is living out an empire of an alternative way of worshiping..
They have a pantheon of gods that they have constructed in order to keep control over the workings of humanity..
And that pantheon of gods has created an environment where absolutely everything in Egypt's society.
was all down to the worship of various different idols in order to retain harmony in the world..
Very interestingly, harmony becomes a central part to Egyptian religious belief..
So when God shows up with the plagues, he's actually trying to subvert this alternative idea of worship.
and bring his people back to the way things had always intended to be..
And he does that through some pretty dramatic things..
And to help you to understand and see all of that, let me take you once again to the land of Egypt..
The one central driving force of ancient Egyptian life was religion..
The Egyptians believed in a pantheon of gods, which were involved in all aspects of nature and human society..
The gods quite literally were everywhere..
And in order to have a successful life for yourself and for your family,.
you needed to expertly navigate a complex system of beliefs and rituals,.
using prayer and offerings to interact with the various gods in control of the world around you..
All of which meant ensuring a prosperous life for yourself and those around you..
Well, it was actually really exhausting..
I mean, could you imagine being surrounded every day by these deities and these gods.
constantly reminding yourself of your need to be in harmonious relationship with them?.

$^{281}$And it's this concept of harmony that was actually central to ancient Egyptian religion..
See, core to their beliefs was a concept known as Ma'at..
This was a cosmic eternal order, a little bit similar to our modern day concepts of truth and justice,.
but found in the cosmos as well as human society..
And ancient Egyptians believed that their sole purpose was actually to maintain Ma'at in the universe..
And they did this through their relationship with the gods, by their various offerings, their daily rituals,.
all of which were designed to starve off disorder and ensure harmony remained..
And here is what is key in all of that..
Any threat to the cosmic order was not just a threat to the person themselves, but a threat to the whole universe..
And a threat like that needed to be dealt with in one of two ways,.
either aggressively suppressed or ideally completely wiped out..
So imagine what it would have been like if you were a Jewish slave in the land at that time..
I mean, you would have been raised to believe that there was only one God in the universe, Yahweh,.
who could hold all of Ma'at in his hands..
And yet every single day you would have constantly been reminded that actually that is a stupid idea,.
that instead thousands of gods are needed to maintain harmony in the universe..
And in this way, Egypt's slavery of God's people was actually more than just a physical oppression..
It was a spiritual oppression as well..
Everything Egypt could do was designed to mock the one central question of the whole of the Jewish faith..
Could it really be that one God is able to hold the whole universe in power at one time?.
Well, that question and the freedom the answer brings is what the plagues are all about..
And to understand that, I need to take us now back to the waters of the Nile..
If maintaining Ma'at in the universe was the primary aim of Egyptian religion,.
no single source of such harmony was more powerfully symbolized than by the life-giving power of the Nile River..
The Nile was everything for ancient Egyptians,.
a source of human life through its fish for eating and its waters for drinking,.
and in its annual flooding a source of cosmic life to all nature around it..
The Nile was Egypt's strongest demonstration of the balance and order that existed between the gods and people..
If its waters remained healthy, so would its people..
It should not surprise us then that this is the very first place God calls Moses and Aaron to come.
in order to display his cosmic power..
See, God tells Moses to instruct Aaron to stretch out his hand and staff,.
touch the water, and turn it into blood,.
thereby killing all of the fish and creating such a stench.
that people won't come to drink from here from that point forward..
And we have to see the story behind the story here..
You see, God's first act of convincing Pharaoh to let the Israelites go.
is designed to strike at the very heart of Egyptian religious belief..
God disrupts the Ma'at of the universe in order to demonstrate that the only Ma'at is found in him..
The rest of the plagues follow suit..

$^{321}$Each one is a specific statement from Yahweh that he is far more powerful than any of the many gods of Egypt..
So the plague of frogs is a demonstration against Heket, the frog goddess of fertility..
The flies against Kepri, the fly god of Egypt..
The plague of hell against Nut, the god of the atmosphere..
The darkness against Ra, the Egyptian god of the sun..
And with each plague comes a consistent message..
This one single god, Yahweh, is the one and only true God,.
and all other pretenses the deity must bow down..
It would be this constant display of God's power over and above the power of the gods of Egypt.
that would eventually soften Pharaoh's heart and bring him to his decision to finally let the Israelites go..
And we see in this process and God's involvement in it,.
something very important for all of our journeys in Exodus..
You see, God understands that in order to break the chains of our physical oppression,.
we have to first of all break the chains of our spiritual oppression..
We have to know that there is a God who's got greater power than anything in the universe,.
and through that find faith rising up in us to stand against anything that attempts to chain us..
See, God reveals his true power to Israel and their journey in freedom begins..
May it also be so for you..
I want to help you to see that journey of freedom for you by mapping out very briefly the plagues..
There are 10 of them in total..
The 10th one is the one that's perhaps the most shocking,.
the one that's perhaps the most difficult to deal with,.
and that's the one that we'll deal with next week, thankfully..
And that's all on the Passover and what God does there..
But the other nine plagues that we see are actually broken into three different cycles..
Let me show you a diagram..
I know I told you earlier to put your phones away,.
but you can get them out for just a brief term and take a photo of this slide if you would like,.
because this is a very helpful—I'll get out of your way a little bit—.
but this is a very helpful slide if you want to study a little bit more around the cycles of the plagues..
So as I mentioned earlier, they're found from chapter 7 through to chapter 12..
What you'll see here is that there are three separate cycles of in total nine plagues..
There are the Scriptures they have,.
particularly the Egyptian gods that each of those plagues are against and bringing a polemic against are there,.
but also, very importantly, God's intention about what he's trying to communicate about himself in each of those cycles..
And I'll explain that a little bit more in a moment..
But I want you to see something, first of all..
One of the things that God is doing through the lens that we looked at with the creation of the world in Genesis 1 and 2,.
his judgment against what Egypt are doing through their pantheon of gods.
is a judgment which basically results in his de-creation of the world..

$^{361}$So actually, and this is fascinating, all of the plagues are a de-creation event.
that mirrors a creative event that happened in Genesis 1 and 2,.
because what God is trying to do is basically say, "I need to de-create in order to then free Israel to create again Israel in the way that they'll be going forward.".
So all of these are de-creative events..
Let me give you an example. The frogs..
Remember I said that there was a separation between the earth, the land, or sorry, the sea and the land..
There's a separation between those two things..
Well, with the plague of frogs, frogs are the ones that have the ability to go on both land and in both water..
And God brings a plague of frogs to cover the earth between the waters and the ground.
in order to de-construct what he had done in Genesis 1 and 2..
Likewise, the gnats. We're told that in the gnats plague, God takes dust and he creates the gnats out of the dust..
In the same way that he had taken dust in Genesis 1 and 2 and created humanity,.
now he creates gnats to show the chaos that comes when you try to do things in a different way..
These all follow suit. The flies, for example..
Actually, when Moses writes about the plague of flies, he says they multiply and fill the earth..
The same language he uses about God's call on humanity in Genesis 1 and 2..
Again, God is de-creating in order to futurely create..
The plague of darkness on cycle 3..
Again, God had separated light and darkness..
Now in the plagues, God brings darkness in the middle of the day to once again de-construct.
in order to say, "I'm going to do things differently.".
And his main communication is, "If you have any other form of worship,.
any other way of thinking about harmony, it is not true..
It will only bring more chaos, and I show you that chaos in judgment of the other form of worship that you have.".
Notice, though, on the right side here, God's intervention..
Remember that God describes his plagues as signs and wonders..
In each cycle of plagues, God is actually saying something about himself to Egypt and to Israel..
In the first cycle, he says, "By this you shall know that I am the Lord.".
In other words, I'm inviting you into this intimate, personal relationship with me,.
and you would know that I am a true God, the one and only true God..
And I do these things in order to shake up, remember, disrupt your ordinary.
to bring you into a knowledge that I am the one true God..
In cycle two, he says, "That you may know that I am the Lord, and I am in the midst of the land.".
This is really important, because Yahweh is basically saying,.
"I'm not some god, the pantheon of the Egyptian gods where deities removed.".
God is saying, "I'm in the land itself. I'm here with you..
I'm not distant. I'm not on some fluffy cloud trying to manipulate the world in some way..
I'm actually present and with you. I am a God who is in the land itself.".
And that differentiates himself from all of the pantheon of Egyptian gods..
And then in cycle three, he says, "So that you may know that there is none like me in all the earth.".

$^{401}$No other God. Everything else is a pretense to authority..
The only authority is me. You would know through this that I am the only God,.
that none of your magicians can create any of the plagues in the way that I can..
I am the one true, powerful God, and all other pretenses to authority must bow down..
The lens through which we approach the plagues enables us to see that God is actually offering to Egypt.
the fullness of who he is. You don't need those hundreds of gods..
You don't need to live in a stressed way where you're trying to please everything around you..
You just need me. And I'm not just distant, but I'm present..
I'm in the land. And everything else submits to me..
I'm the one who holds all my ite in my hands..
And you can find escape from the pressures of trying to chase everything else with just your worship of me..
And some of you in this room, that's a word for you..
Because God is saying the same thing to you today through the plague narrative..
You don't need all this other stuff. All these other little ways in which you worship..
All these other little idols that you're bowing down to..
All this other stuff of culture that you think you need to fit into in order to succeed..
The only thing you need for true power, for true authority, true harmony, true peace is me..
And for some of you, that's a call of God on you today..
So through the plagues and all of this, what is it that the people learn?.
Well, here's what Egypt learns..
They learn that there is one God who can hold all my ite in his hands..
And they better bow down to that one God. Otherwise, things won't go well for them..
Pharaoh, his heart is hardened and he hardens his heart with everything..
His pride gets in the way. And eventually, as we'll see next week,.
he finally lets Israel go through the work of the Passover and the death of the firstborn sons of Egypt..
But all of that process is building the pressure on Egypt to see that there is only one true God.
that holds my ite in all his hands..
But every single plague, there's an opportunity for them to turn..
Every single plague, there's an opportunity for them to respond to this God.
and give their lives and their authority and their worship to him..
Israel also learns a lot..
They learn that after having spent so many years in Egypt,.
they had naturally begun to worship some of those deities of Egypt..
I mean, can you imagine, like I said in the film, being surrounded by all those gods all the time?.
Well, you would have felt that pressure to align yourself to the culture of your time..
And the plagues is a wonder to them..
It's a thing that disrupts their ordinary and helps them to realize,.
"Hang on a sec, what are we doing? Why are we worshiping all these other gods.
when there is only one God, the God of our forefather, of Abraham, Isaac, and Jacob?.
That's the God we should worship.".

$^{441}$And the plagues actually work for Israel as a shaking up of their ground.
so that they can remember the true God that they were always called to worship..
But the plagues also do something else for Israel..
They enable them to realize that Moses and Aaron are actually not false prophets..
Remember what happened in Exodus chapter 5..
Moses and Aaron had gone before Pharaoh and said, "Let my people go.".
And Pharaoh had gone, "I'm not going to do that.".
It makes their slavery even worse..
And up until this point, all of Israel thinks Moses and Aaron are false prophets..
They don't really speak on behalf of Yahweh..
The plagues demonstrate to all the Israelites.
that Moses and Aaron are actually speaking on the mouthpiece of God,.
that they are true representatives of Yahweh, and that they should follow them..
So the plagues actually work as a uniting movement for Israel.
around the leadership of Moses and Aaron,.
something that is critical for everything that's going to happen next in the story..
But what about for Moses?.
What does Moses learn?.
Moses learns that despite the fact that he has been raised as a prince of Egypt,.
and despite the fact that he would have been incredibly talented.
in knowing how to worship the variety of gods in Egypt,.
and despite the fact that he'd had 40 years in the wilderness,.
he was still carrying some of the residue of that identity..
And the plagues was God's disruptive way to once and for all say to Moses,.
"You have to let that stuff go.".
See, one of the things that Moses discovers is, I think,.
a central story of the whole of the Exodus journey, and it's this,.
that the call to let my people go was not just a call to Egypt.
about releasing God's people from their slavery..
It was actually a call to Israel about releasing themselves from their idolatry..
Come on, church..
This is actually what the plagues are about for Israel..
God is saying, "There are idols that you're holding in your hearts,.
and I'm demonstrating to you that I'm only the one true God,.
and you getting out of Egypt is only half the battle..
The other part of the battle is you letting go of Egypt,.
you letting go of the idolatry that you have become accustomed to,.
because I just want you.".
And Moses sees in the plagues the disgusting reality of the idolatry of his people..
And in that, a longing is birthed within Israel.

$^{481}$to have a pure relationship with God again..
And all of that is critical for them to move into the wilderness.
and meet with God at Mount Sinai..
But as I close, everybody's favorite words in a sermon,.
what is it that you learn through all of this?.
What is it that you can take away and reflect around?.
What is it that changes in your life because of the plagues?.
Well, first of all, I think the plagues help us to understand.
that despite how it might look at times,.
God is at work behind the scenes in the corridors of power in this world..
That God is the one who's ultimately in control..
He is the one who ultimately has the authority on heaven and on earth..
And sometimes it's easy for us to look at the way things are.
in terms of the power structures of this world..
It's easier for us to look at the ones who govern us or lead us.
and think that there's no way that anything good could ever come of that..
And what the plague narrative reminds us is that ultimately,.
God is in control of everything that happens in the world..
And that God brings signs and wonders at times.
to bring his world back into relationship with him..
And that should encourage us..
We should see the plagues and have faith rise in us.
that God will always hold governments to account.
for how they govern their people..
That God will always work towards justice in this world..
And as his people, we get the opportunity to lean into that with our prayers,.
lean into that with our trust, lean into that and say,.
"Even though I don't see how this election is going to happen.
or how this is going to work or how this is going to happen.
or how this injustice is going to change,.
I believe that God is not blind to it..
I believe that God cares about it..
I believe that God is involved in it and he's doing stuff.
that I don't even understand, I don't even see..
And my role is just to come alongside, to pray, to believe, to stand, to speak,.
to be his voice and his mouthpiece because he's a God of justice and he's at work.".
And the cool should be on the church to rise up.
and say that we believe that God is at work,.
even when it feels like the opposite..
The second thing I think you can take away is the knowledge.

$^{521}$that God is at war against the idols that are in your life..
God is at war against anything that will take your heart away from him,.
anything that will distract you and hold you hostile.
to God's good purposes in your life..
God has an unrelenting passion to see freedom come for his people..
And I want you to hear this..
God is not just interested in your freedom..
He also wants your heart..
And for God, if he doesn't get your whole heart,.
then you will never truly be free..
See, the purpose of the Exodus is not to get you out of some bad stuff.
that's happened to you in the past..
It's not just about freeing you from some chains that are around you..
The point of the Exodus and the point brought to us through the power of the plagues.
is that God doesn't want to just draw you out..
He wants to draw you in..
He wants to engage with you, be in relationship with you, know you more..
He wants your heart not split between all these different things..
He wants your whole heart, because it's only in your wholehearted devotion of him,.
your wholehearted worship of him, that you'll ever find yourself truly free.
and therefore find yourself in harmony, in shalom,.
with the way that you've always been created to be..
Does that make sense to you?.
God is passionate not just about your freedom, but about where your heart is..
And any time that our hearts are connected to idols, we're not truly free..
Which is the third and final thing I think you should take away with you..
And this should encourage you deeply..
God stands powerfully against every demonic force in this world..
If there's anything you take away from today,.
may it be that whatever strategy that the enemy has over you,.
your family, your marriage, you yourself, whatever it might be,.
God is at work right here, right now,.
fighting against the demonic forces of this world..
And that's ultimately seen, of course, in the life, death, and the resurrection of Jesus..
It's ultimately seen in the blood that is shed in Jesus,.
which we're going to talk about next week, and you're going to see the power in that..
But before we get to next week, know in this moment,.
that God stands against every demonic realm and every demonic force,.
and he stands against it, and he says, "Let my people go.".
And I wonder whether you could hear the resonance of God's voice.

$^{561}$over you and over your life, where he stands against any principality,.
and any darkness that's trying to have any influence over you, your kids, your family,.
whatever it might be, and he's standing and he's saying to those things,.
"Let this person go. Let this one go, because this one is my child.".
And God has an unrelenting, unbridled fury of love for you and your freedom..
And he will bring that against any pretense to authority and power.
with such fury, a fury passioned by love, not by wrath, but by love,.
to stand with you and say, "I will do these things for you, over you,.
where you're not even realizing it, whilst you sleep, and behind the scenes,.
because I want you free, and there is no weapon formed against you that will prosper.".
The plagues are God's way of saying to his people,.
"There is no weapon formed against you that will prosper.".
And if there's anything you take from today,.
may it be a sense of the fighting power of God on your behalf..
And I want you to feel it. I want you to see it..
The plagues are not nice. They're not easy. They raise big questions..
They're a fireball of the fury of God's love for his creation..
A fireball that stands against any other pretense to worship,.
any other system that would say there's another way,.
any other Satan that speaks to Adam and Eve and says,.
"Take this fruit, because this is better.".
God's plagues are a picture to us of this unrelenting fire that God has.
to say, "The way that I have created, the way that I am, is the only way to true peace for you.".
And he wants you to know him that way,.
to know that you're protected, fought for, redeemed, saved, and set free..
And that passion, I want you to feel now..
To help you with that, take a look at this..
Let my people go..
These wild, wild fires,.
raving and burning, consuming, hardened by the fury of heaven..
Stars become snakes, snakes become slaves,.
dressed in the chains of tomorrow's grave..
The power of fear, blood in the water,.
the stench of injustice now torn from another..
Skies become flies, flies become hail, hammering the hardened with the holiest of nails..
A stretching of hands, a rage from the heavens..
Darkness descended, darkness descended..
Now blood on the doorpost, an angel of death..
The horror of genocide not far from our lips..
A heaven of fury, burning and breaking..

$^{601}$The hardened consumed with words still shaking..
Let my people go..
That's how he feels for you..
I wonder whether you'd just open your hands..
Come Holy Spirit..
You are loved, accepted, fought for..
God has sent his son to the cross,.
to stand in the place where you should have stood..
And with the fury of love, cast sin upon him,.
so you would know true life..
You now stand more than a conqueror..
And whatever idol or idols that are present in your heart,.
the forgiveness and the mercy of God is like what we just witnessed..
It can be uncomfortable at times..
And I know firsthand that God has sent me signs and wonders.
to bring me to repentance..
And God brings his justice..
And I pray that you would, through this understanding of the plagues today,.
leave this place knowing the price that has been paid,.
but also the power that there is that stands alongside of you..
That the same spirit that raised Jesus from the dead is now in you..
That same power, that same love, that same authority, it's in you..
And although the plagues are uncomfortable,.
and they raise lots of questions that we have to wrestle with,.
ultimately, seeing through the lens that the storyteller provides for us,.
they are an invitation to less of our idols and more of God..
And I pray, in this moment and in the moments ahead for you this week,.
that you would continue to make that exchange.
by simply coming before him.
and saying the one powerful prayer in scripture that the whole Bible ends with..
It's "Maranatha". It means "Come, Lord Jesus"..
And Father, I pray for anyone here who needs to lay down idols,.
you would show them how to do that..
And I pray every person here would invite more of you into their lives..
Come, Lord Jesus. Come..
And we thank you for this. In Jesus' name..
Amen..
We love you..
[BLANK AUDIO].
\newpage



\section{}
\label{sec:ezohhaWO5XQ}
\textbf{2023-09-04 EXODUS - 13  The Passover [ezohhaWO5XQ].mp3}
\newline
\newline
連結: \href{https://youtube.com/watch?v=ezohhaWO5XQ}{\texttt{ https://youtube.com/watch?v=ezohhaWO5XQ}} ~~~~ 語音日期: 2023-09-04 
\newline
\newline
\hyperref[sec:OFYlvA1AkKg]{\small{< < < PREV SERMON < < <}}
~
\hyperref[sec:index]{\small{[返主目錄]}}
~
\hyperref[sec:PHwKp7Ievsk]{\small{> > > NEXT SERMON > > >}}
\newline
\newline
$^{1}$Have a seat, have a seat, have a seat. It is so good to have you guys with us here today. If it's.
your first time to The Vine, my name is Andrew. I'm one of the pastors here, and we are so glad.
that you're present with us. And we are absolutely jam-packed today in the overflow as well. Welcome.
to everybody in the overflow. We're so glad that you're with us. As well as everybody that's joining.
us online, streaming from around the world, we're so glad that you're with us as well..
We come today to the actual turning point of the whole of the Exodus story. In fact, the event that.
we're looking at today is not just a turning point in the Exodus story, but it is probably the.
central narrative in the whole of our scriptures. It's certainly the central narrative that has.
shaped and formed both the Jewish faith, but also the Christian faith ever since. It's a central.
act of God where the axis of his redemptive love for humanity actually pivots. And a moment.
where we see both the power and judgment of God, and yet also his mercy and his grace come together..
Pharaoh, after nine plagues that have slowly built up to bring a condemnation against the.
pantheon of Egyptian gods and the idolatry that it has created in its people,.
Pharaoh is about to discover the brutal consequences of his hardened heart..
For God is about to act in Egypt like God has never acted before. And when he does,.
finally, Israel's long walk to freedom is about to begin. This is our 14th week of Exodus,.
but we've only just got to the start of the Exodus. Are you with me?.
The 10th plague that we're looking at today is the plague of the death of every firstborn in.
Egypt, every firstborn son. It's a plague of staggering contrasts, because on the one side,.
it does show this reality of God's justice and his holiness. But on the other side, it challenges us.
with deep poignancy. It's the kind of plague that you can't stand back from and just read in the.
Scriptures and think, "Oh, that's nice." It's a plague that grabs you visually, passionately,.
and speaks of things that's difficult for us to reconcile with a peaceful and loving God..
But just like we saw last week, where we need the appropriate lens through which to view the.
plagues themselves, so when it comes to the 10th plague, we need to have the right lens to approach.
and to understand it. Like last week, where we saw that there was an actual beginning to the.
plague narrative that the storyteller wanted you to look through. So here in the 10th plague,.
there is also a lens through which God wants us to look through as we look at both the brutality.
and the poignancy of this 10th plague. And to show you that beginning, the actual Exodus story.
of the 10th plague, the plague of the firstborns, is found in Exodus chapters 11 and 12. But the.
lens through which God wants you to see it is actually found in Exodus chapter 4. Actually,.
in a moment where Moses has just met with God at the burning bush. And off the back of that burning.
bush moment, Moses is having a conversation with God about how Moses and them feel capable of being.
able to go back to Egypt. He's still in the wilderness at this point. He's not gone back to.
Egypt. He's never spoken to Pharaoh. But God speaks these words to him in that moment. This.
is Exodus chapter 4, starting in verse 21. "The Lord said to Moses, 'When you return to Egypt,.
which you're going to do in just a little while, see that you perform before Pharaoh all the wonders.
that I've given you the power to do. But I will harden his heart so that he will not let the.
people go.' Then say to Pharaoh this, 'This is what the Lord says, 'Israel is my firstborn son..
And I told you, 'Let my son go so he may worship me.' But you refuse to let him go,.

$^{41}$so I will kill your firstborn son.'" I want you to get into your head what's happening here..
Before Moses has gone back to the land of Egypt and confronted Pharaoh for that first time that.
we see in chapter 5, here whilst he's still in the wilderness, God begins to unpack for him what's.
going to happen in the place. And he begins to say, "Look, I see you as my firstborn son." Here's.
the powerful thing. This is the first time in all our scriptures where God refers to his people as.
his children. This becomes a major metaphor in the rest of the Christian narrative. The rest of the.
Old Testament and the New Testament speaks about God's love for his children, God's love for his.
sons and daughters. All of that starts here. This is the very first mention in the whole of.
scripture where God identifies himself as a father over Israel, as a father over his people. And he.
says to them, "Don't you realize that Israel is my firstborn son?" We know in Chinese culture.
the importance of firstborn sons. Can I have an amen? I know that was a very quiet amen..
Raise your hand if you're a firstborn son in the room. Come on, put them up proudly. All right..
Some women raised their hands. That's a little worrying, but we'll come back to that later..
But we know in the culture here that firstborn sons is important, isn't it? And it was no.
different in ancient Near Eastern culture. Because back there, if you were a firstborn son,.
it came with privilege. It came with prestige, and it came with power. You were going to be.
the one that would carry on the generations of your family name. You were the one that actually.
held all of that prestige in front of the world. The rest of the family looked to you to provide..
The rest of the family looked to you to be the one who would carry forth the integrity and the.
honesty and the value of your name. The firstborn son had a big job to do. And God says, "These.
people are like my firstborn son. They're the ones that are going to carry forward my name into the.
generations that are to come. They're the ones that have the value and the integrity and the.
honor to show the world my character and my name because my firstborn son represents who I am.".
Do you follow that? So it's important. But I want you to see this. He then speaks to Pharaoh,.
and he says, "You have a firstborn son too. And I'm going to do something to your firstborn son..
That firstborn son is going to be killed." And when we read it like this, we think,.
"How can God do that? How can God allow that to happen?" Well, this tension between the firstborn.
son of Israel and the firstborn son of Pharaoh's family is the lens through which God wants you.
to look at as you now come to the 10th plague itself, which is found, as I said earlier,.
in chapters 11 and 12. Now, there's a lot of scripture here. I'm not going to go through it..
I just want to pull out a couple of key verses for you to really understand God's message in the.
10th plague. The first is found actually in Exodus chapter 11. I'm going to start in verse 1. It says.
this, "Now the Lord God said to Moses, 'I will bring one more plague on Pharaoh and on Egypt..
After that, he will let you go from here. And when he does, he will drive you out completely.'".
Can I get an amen? Finally, it's going to work. Finally, this plague is going to have the work.
that God has longed to want to do. Finally, Israel is going to be released from their slavery..
Starting from next week in our Exodus series, we're going to be talking about a free Israel,.
no longer bound in slavery to the ways that they've been treated for 430 years in Egypt..
Now, because of this event, God's people are going to be set free. Verse 4 says this, "So Moses said,.
'This is what the Lord says. About midnight, I will go throughout Egypt. Every firstborn son.

$^{81}$in Egypt will die. From the firstborn son of Pharaoh, who sits on the throne, to the firstborn.
son of the slave girl, who's at her hand mill, and all the firstborn of all the cattle as well..
There will be loud wailing throughout Egypt. Worse than there has ever been or ever will be again.'".
I want you to see something really important here. Because often when we think of the 10th plague,.
and we hear about the killing of the firstborn sons, we immediately think that this is God's.
polemic against Egypt and Egypt alone. But I want you to notice something which is really clear in.
the Hebrew, but it's also clear in the English. It says, "Every firstborn son in Egypt will die.".
What it doesn't say is, "Every firstborn son of Egypt will die." Are you with me, church?.
This is a very important distinction. Because God in chapter 4 had said, "Israel is my firstborn.".
You've got to bring that context now into here. And now God says, "There's this 10th plague coming,.
the killing of all firstborns in Egypt." And he actually gives a demonstration. He says,.
"Pharaoh's son will die, but also the son of the slave girl." Who are the slaves?.
This is something that's happening, God is saying, to both Egypt and Israel. And it's important that.
you understand this, God is saying. Because this is the plague, all the other nine plagues have.
impacted Egypt and have let Israel go. But this plague, this plague is for both of you..
That picks up even further in chapter 12. Let me read you chapter 12, verse 12, probably the most.
important verse in the whole of the Passover narrative. "On that same night, I will pass.
through Egypt and strike down in the land of Egypt every firstborn man and animal. And I will bring.
judgment on all the gods of Egypt, for I am the Lord." There it is, "In all of the land of Egypt,.
every firstborn will die." This is really important, because Israel had to understand.
that they also were not right before God. Now, this wasn't just about God coming and saying to.
Egypt, "Hey, you guys have been nasty for 430 years, and you've treated my people as slaves,.
so I'm now going to be nasty to you." This is a God of justice. And a God of justice is not playing.
favorites here. And he's coming before both Egypt and Israel and saying, "Hey, Israel, don't think.
that this is not about you." Remember I said last week that God's call to let my people go was not.
just a call to Pharaoh in Egypt about the slavery they had caused for the Israelites. It was actually.
a call also for Israel about the slavery that they had created for themselves and their idolatry,.
that they needed to deal with their idolatry. And so God shows up and says, "Here's what's going to.
happen now. You're going to have to deal with your idolatry, because both your idolatry, Israel,.
and your slavery of my firstborn son's Egypt, both of those things have to receive justice..
So I'm coming, and I'm going to kill every firstborn son in Egypt.".
Could you imagine how that would have felt for Israel? Because you'd be forgiven, wouldn't you,.
from thinking that maybe by this time in the narrative, Israel were kind of going, "Heh,.
this is all about those nasty Egyptians and what they've done to us for 430 years,.
and how they've been so nasty and mean to us. And look what God has done. The last nine plagues.
have given them that impression, because the nine plagues didn't touch Israel at all..
But now everything changes." And the reason why this is really important for you to grasp.
is because when it comes to the idea of judgment for the sin and brokenness in the land,.
it involved everybody. God is not showing distinction or favoritism in the concept of.
justice in the land. If anybody has broken his law and broken his heart and broken sin in the land,.

$^{121}$that'll be judged by the 10th plague. That's a sobering thing that we all have to wrestle with..
But the reason why that is powerful is because of what we see happen next. You see,.
God then says, "Israel, you're as guilty. And you need to understand and wrestle with the reality.
that you're just as guilty as the Egyptians. But in your guilt, Israel, you are my firstborn child..
You are my firstborn son. And I'm covenanted in love with you. And so despite the fact that,.
yes, there is justice and judgment needed, I'm going to actually pay the price for that judgment.
and justice. Even though you are guilty and you need to understand that, Israel, I'm going to make.
a way for you. I'm going to make a way in my covenantal relationship for you that you don't.
need to go into impending death like the Egyptians. You're going to get to experience revived life.".
And here's the powerful thing. The only difference that there is between Israel and Egypt is this,.
blood. The only thing that is going to separate these two nations is nothing Israel's done..
It's not their moral superiority. It is not the fact that they are just a special kind of people..
It is only going to be this that will separate them, blood. And the question you should ask.
yourself when you're reading the Exodus story, why blood? Why is blood going to be the separating.
thing between Israel and Egypt in this moment? One thing that you may not have realized is that.
actually blood has been the background character in the whole of the Exodus story so far..
Blood was right there at the start when the midwives bravely decided to stand.
against Pharaoh's decree to murder all the newborn Hebrew boys. Blood is there when.
Pharaoh decides to drown all of those Hebrew boys in the Nile. Blood is right there when Moses decides.
to take on the mantle of Savior himself and murder an Egyptian. Blood is there when he flees from.
Egypt and finds himself in the desert. And in the desert place, he actually defends his future wife.
Sapphora against an invading shepherd army. Blood is there. And then blood is there right in the.
first plague itself. The first plague is spoken about so profoundly as Moses stretching out his.
hand with his staff and touching the Nile River and the Nile turning to blood. Blood has been.
there the whole time. And because blood has been there, the great question we wrestle with is,.
what is it about blood, not just for the Israelites, but for the Egyptians that matters so much?.
Because when God chooses blood to be the differentiating factor between the two nations,.
it's because blood wasn't just important to Israel, it actually had something to say to Egypt too..
In ancient Near Eastern Egyptian mythology, blood played a very central role. And to help you to.
understand that, I want to take you now back to Egypt once again. And I want to take you to where.
the Israelites lived in Egypt in those moments. It's a place called Goshen,.
because it is that place where we see the importance of blood. Let's take a look together..
(birds chirping).
(music).
I've returned to these fields here behind me today because it is right here that perhaps the.
most significant moment of Jewish history took place. Behind me and buried in these fields is.
the ancient city of Avaris. It was the capital city of Egypt for the Hyksos, a Semitic group of.
people who invaded here during the 15th dynasty. But Avaris as a city was actually older than when.
the Hyksos was here. In fact, recent archaeological evidence has revealed the existence of an older,.
more sophisticated community that used to live here. And those people were from the Canaan Syria.

$^{161}$area. A Semitic group of people that moved in here, settled, grew to a really large number,.
and then all of a sudden left very quickly, right at about the time that Egypt's power was decreasing..
All of which tells us this. It is very likely that right behind me here is exactly where the Passover.
took place, where God actually brought the final judgment on Egypt to release God's people and.
send them to the promised land. It happened right here. And here's how it happened. God actually.
came to the Jewish people and he told them to take a lamb and at twilight to sacrifice that lamb,.
dip their fingers in the blood and put it over the walls and doors of their home. In so doing,.
creating protection for them and their families. So I want you to track with this. There was an.
animal sacrifice. It created some shed blood and that blood was then used to create protection..
Now, why is all of that important to our story? Well, what is it about blood that actually was.
the significant catalyst to releasing God's people into freedom? What is it about blood.
that was important, not just for the Israelites, but for the Egyptians as well? Well, to answer.
that question, I need to now take you back to the ancient temple in Talbasta and show you.
something there that will change your perception about what the Passover is all about..
Since antiquity, blood has been recognized as the essential component of life. Without knowledge of.
the circulatory system, ancient Egyptian religion recognized the heart as the seat of the soul.
and all eternal life. But the Egyptians also saw blood as something hidden, visible only when.
flowing from a wound or during childbirth, miscarriage or menstruation. As such, blood.
became a symbol not just of life, but also of death, something that gives, but also something.
that takes away. Because of this, blood was deeply revered and to control blood was to literally hold.
in one's hands the power of life and death. This power was what animal sacrifice was all about.
in ancient Egyptian religion. And as blood was the central image of this power, the sacrifice.
of an animal to the gods revolved around the collection of the blood that was shed. Here in.
the museum of Talbasta is a great example of a sacrificial stone that was found in the temple.
of this area. The animal would have been placed on the top end of the stone slab where its throat.
would have been cut. The blood that came forth would then flow down these rivets, across towards.
the middle section, and then drain through the various holes you can see here, eventually flowing.
out of the bottom onto whatever was chosen to receive the blessing of the sacrifice..
Interestingly, it was the flow of the blood that was important. The blood flowing out of the animal.
onto the stone signified the bringing of death. But the blood flowing out of the stone onto.
whatever icon or offering was below it signified the bringing of new life. This new life was.
understood as a form of protection, protecting the person from death through the sacrifice of another..
Think about that for a second. Egyptian animal sacrifice was about the exercise of power through.
death and life and the protection it brings through the shedding of blood. And this is something that.
can still be seen in the Egyptian culture today..
Around the time that the city of Cairo was first established, between 909 and 1171,.
the Fatimid Caliph ordered the slaughter of a large number of sheep and lamb as a declaration.
and protection to the city. After the slaughter took place, people were encouraged to leave traces.
of the blood as a sign of blessing on their property or brand new items that they owned..
To this day, traditional Egyptian families continue this practice by.

$^{201}$dipping their hands in blood and placing it on their material goods, things like.
brand new cars, in order to show humility to their neighbors and honor to the gods..
Now, let me pull together what we've been teaching about blood in Egyptian history and culture..
See, for the Egyptians, blood was critical for three things. The blood actually said, first of.
all, that it symbolized the power over life and death. That blood also brought blessings and honor.
to the people. And finally, blood was a way of actually bringing protection for life. Like,.
if you wanted to be protected, you would use blood as that symbol. So, is it a wonder, then,.
that God chose the Passover as the thing to bring the final judgment to Pharaoh and the Egyptians?.
The very thing of asking the Israelites to take blood and to place it on the door frames of their.
property to symbolize protection and honor in that place. This was basically God saying, "Hey,.
Egypt, do you want to know who has the power of life and death? It's me. Hey, Egypt, do you want.
to know who can really protect you? It's me." It was God establishing His authority upon the.
Egyptian people. There's this verse in Leviticus that I love. It says, "The life of the flesh is.
in the blood." The Passover begins the Exodus narrative, and it does so by laying down perhaps.
the most foundational truth of our whole Christian story, that it truly is in the shedding of blood.
that we are set free. It truly is in the shedding of blood that we are set free. I want to show you.
how God speaks of this bloodshed in the original Passover story. Verse 13 of chapter 12 says this,.
"This blood will be a sign for you on the houses where you are. And when I see the blood,.
I will pass over. No destructive plague will touch you when I strike Egypt." The thing that.
God mentions about the blood right up front is it will be a sign to you. The blood was certainly a.
sign to Egypt like I just explained in the film. It was a sign to Egypt that all of the things that.
they had thought about with animal sacrifice really was only truly powerful in the hands of.
Yahweh, this one God. That the sacrifice to all this pantheon of Egyptian gods was useless and.
nothing. That no protection, honor, or life, which is what blood symbolized for the Egyptians,.
could ever happen through the worship of their Egyptian gods. It could only take place.
through Yahweh. The Passover was a sign to Egypt that this blood, the blood that has been shed on.
behalf of God's people, was the only blood that could truly set them free. But it was also a sign.
to Israel. It was a sign to them that God can provide a substitution to their sin. It was a.
sign of God saying to Israel itself, "Hey, you are guilty. There is a judgment that should come if.
I'm a God of justice. I have to bring that judgment, but I will provide a blood, an animal,.
who will be sacrificed as a substitution to that." And Israel learns that their sin and their.
brokenness can be covered by the blood of this animal. And I think for many in Israel, they would.
have remembered what happened to Abraham and Isaac. And now Abraham had gone up in that hill.
and was about to sacrifice Isaac, and God shows up and says, "No, there's another one for you..
There's a goat in a thicket for you, and that one will be a substitution for you." And now,.
here's Israel realizing that they have every reason to be judged, and God has shown up and.
said, "There is a substitution for you." My grace offered now through the reality of blood..
So when Israel comes and they take that blood and they place it on their doorframes,.
here's what's going on in their minds. They're thinking to themselves, "I am guilty.".
They're thinking to themselves, "I should be blamed." They're thinking to themselves,.

$^{241}$"We didn't get away with it. We have broken things. We've broken our covenant with God.".
They've thought to themselves, "God has every right to come and take our firstborn. God has.
every right to come and do what he said he's going to do against us, as well as against Egypt. We're.
equally guilty, but this blood, this blood we're applying in hope, in faith, and in trust that if.
God sees this blood, if he understands that we are aligning ourselves to the reality that we know.
that we're guilty, but this blood can set us free, then maybe God will pass over us." This was a sign.
not just to Egypt and not just to Israel, but also a sign to God. The blood on the doorpost.
was a sign to God for God to say, "This family trusts in me. This family agrees with me.".
Because I want you to think about this for a second, because this is actually really important..
You see, the actual sacrifice of the lamb itself was not the main thing. The taking of the lamb.
and sacrificing it and eating it that night was not the thing that was going to protect them and.
bring salvation. In fact, what Israel had to do was not just kill the animal, but then had to take.
the blood of that animal and apply it to the doorposts. And it was in the application of the.
blood that they were showing a moment of faith, a moment of trust. And any family of Israel that.
had just killed the lamb and not applied the blood would not be spared from the judgment that was.
coming. This act of faith, this act of trust made this whole original Passover the main teaching.
place for Israel themselves. Will we align ourselves to the hope that God has provided for us.
in the substitution for our sin? And so every home that had blood on the doorposts and blood.
on the walls, the destroyer, it says in the passage, passes over that home because of their faith..
Are you with me? So there's a call on Israel to partake in the substitutional sacrifice.
that had been provided for it. And in so doing it, they found freedom and grace..
They found the mercy of God. So the message of the plague of the firstborns is that God is holy.
and just. But the message of the Passover is that God is merciful. And in this way,.
through his infinite love, God makes it possible to be both just and forgiving at the same time..
He actually paves the way for what our Christian faith completely sits upon..
Should it surprise us then that the primary metaphor that the gospel writers use to speak.
of what Jesus does in his life and his death and his resurrection is the Passover imagery..
In fact, the disciples at the time and the prophets at the time, they grab a hold of this.
really quite vicious type imagery and they begin to apply it to the reality of Jesus.
because they understand that no number of slain lambs would ever cover the reality of humanity's.
sin. And so they begin to look for God to provide a substitutional sacrifice that's different from.
just slain animals, which is why when Jesus walks past John the Baptist for the very first time.
at the very beginning of John chapter 1, John the Baptist says something out loud about Jesus. I want.
to read this to you from John chapter 1 verse 29. He says, "The next day John saw Jesus turning.
towards him or coming towards him and said, 'Look, the Lamb of God who takes away the sin of the world.'".
No one told John about this. This wasn't like Jesus has performed all these miracles and Jesus.
has done all this stuff and they're beginning to connect the dots. This is a prophetic statement.
from John. The Holy Spirit is revealed to this in his spirit and he looks at Jesus before Jesus.
has done any ministry and he says, "This one is the Lamb of God. This one is going to be slain for the.
sins of the world. This one is going to be killed and if we take this one's blood and we apply it.

$^{281}$to ourselves, we will know what it is to be forgiven even though we carry sin." And this.
beautiful idea that Jesus somehow in his death and in the shedding of his blood was going to.
set people free becomes one of the primary ideas of the New Testament. In fact, all of these.
scriptures here draw that out from across the New Testament narrative. Things like, "Without the.
shedding of blood there is no forgiveness, that he might make atonement for the sins of his people..
To him who loves us and has freed us from the sins by his blood." The whole of the New Testament is.
the blood of Jesus does this, but there's something that's still amiss. Should we take.
somehow the blood of Jesus and apply it to the doorposts of our homes now? How do we take the.
blood of Jesus and how do we make this a part of our faith, a part of who we are? How does actually.
the death and resurrection of Jesus pay the price for our sin? And it's almost as if God says,.
"You don't need to place blood on the doorposts of your home anymore, because I'm going to take.
the blood of my son Jesus and I'm going to place it on the doorframe of your heart.".
It's going to be like I literally kind of just reach out with my hand with the blood of Jesus.
and I cover your heart. I just do it right here. I just take it and cover your heart,.
your heart of darkness, the heart of brokenness, the heart of sin. I cover it with his blood..
And when I cover it with his blood, I'm saying to you that you don't have to do anything to be set.
free, that the work of Jesus on the cross is for you. It's for you and your heart is covered now.
because I love you and I will provide the sacrifice for you. Are you with this?.
But just like Israel, who in the original Passover had to realize that the death of the.
animal was not enough, that they had to take the death of that animal and the blood of that animal.
and place it on the doorframes, so it is also for us. You see, Jesus' death on the cross is not a.
cure-all for all. It is not like a moment that universally suddenly cleans everybody of all of.
their sin and, "Hey, you can just live however you want for the rest of your life and everything's.
going to be perfectly fine. It doesn't matter because Jesus has paid the price for you." It's.
not a cure-all for everything. You, like Israel in the Old Testament Passover, have to apply the.
blood by faith. You have to dip your fingers in that blood, if you will, and you have to apply it.
on your heart. You have to be the one that says, "I believe that Jesus died on the cross for me..
I believe that that is a substitutional sacrifice for me. I will take that blood that Jesus.
has shed on behalf of my sins and in taking it by faith, I apply it to myself through repentance of.
my sin and belief that Jesus is really the only Son of God. And in doing that, I align myself to.
the reality that Jesus died for me and for all of the world. And in faith, I apply that in belief.
and repentance and by saying it with my tongue, by baptism, by communion, by all the ways that I can,.
that I am aligned to the substitutional sacrifice of Jesus Christ. It sets me free. It sets me free..
I don't need to fiddle with any doorposts, but I do need to keep my heart soft. I do need to choose.
in faith to walk with the blood of Jesus, to embrace that blood for me, for me. It shouldn't.
shock us then that when Jesus wants to explain about this sacrifice he's going to do, he uses.
the imagery of the old Passover to speak about it. In fact, what Jesus does just the night he's.
betrayed and arrested, he gathers his disciples in an upper room and he takes them at the very.
moment that they're celebrating the Passover, the very moment where they're celebrating the.
reality that this is what has happened in the Old Testament for the Jewish people in the past..

$^{321}$He takes that moment and he says, "Now we're going to change it." And Jesus does something that has.
never been done in history before. He embodies himself into the story of the Passover. And he.
says, "I now am going to be the one who will be your sacrificial lamb, and my bloodshed will truly.
set you free." I want to read this to you, hopefully without getting blood on my Bible,.
from Luke chapter 22. It says...oh, it's so sticky. Are you with me, everybody? I can't do it. Lord.
Jesus. All right. Luke 22, starting in verse 7. "Then came the day of the unleavened bread on.
which the Passover lamb was to be sacrificed. Jesus sent Peter and John saying, 'Go make.
preparations for us to eat the Passover.' 'Where do you want us to prepare for it?' they asked..
Jump through to verse 14. 'When the hour came, Jesus and the apostles reclined at the table,.
and then he said to them, 'I've eagerly desired to eat this Passover with you before I suffer,.
for I tell you I will not eat of it again until it finds fulfillment in the kingdom of God.'".
There it is. "This Passover will find fulfillment in me. After taking the cup, he gave thanks and.
said, 'Take this, divide it among yourselves, for I tell you I will not drink again of the fruit of.
the vine until the kingdom of God comes.' Then he takes the bread and he gives thanks, and he broke.
it and said to them, 'This, this is my body given to you. Do this in remembrance of me.' In the same.
way, after supper, he took the cup saying, 'This cup is the new covenant in my blood, which is now.
poured out for you.'" Jesus takes the imagery of the original Passover and he begins to apply it.
to himself and he says, "I am now your new Passover, and you don't need to smear stuff on.
on any kind of thing anymore. All you need to do is come around this new meal. I create a new way..
Just take the bread and break it and share it amongst yourselves. It's my body. Take the cup,.
this cup, which is now the symbol of the shed blood of Jesus, and apply this to yourself..
Take this now and allow this to become your lifeblood. This is being poured out for you so.
that you would know the forgiveness of sin, so that you wouldn't be judged, so that the Passover.
would happen for you." And what incredible encouragement it is. And I want you to sit.
with this final thought. This blows my mind. God chose his firstborn son to die. Have you noticed.
that? He takes the tenth plague, the one that he had against Egypt and against Israel, and he says,.
"If anyone's going to suffer in this plague, it's me. I will take my firstborn son, Jesus Christ.".
And he's going to go to the cross, willingly choose himself to go to the cross. He's going.
to have that moment in the Garden of Gethsemane where he's like, "Really, Lord?" And then he's.
going to go, "But let your will be done." And he will go freely to the cross, and he will give up.
his life because it's the ultimate example of love. It's the ultimate invitation into the freedom that.
comes from the blood that sets us free. It's our way now of gathering around this thing that we.
call communion and taking the bread and taking the cup to align ourselves in faith. This is our faith.
statement. It's us saying, "I agree that Jesus was God's lamb. I agree that I'm guilty, that I should.
be judged, but I also, thank God, agree that that judgment passes over me because even though I.
deserve it, his blood sets me free. For it really is in the shedding of blood that we are set free.".
What we're going to do in a moment is I'm going to invite you to come and take communion. If you've.
come here with family and friends, you're welcome to come and take communion as family and friends..
I want to invite you to take the communion and pray together as family and friends. Here in the.
lower house, we're going to do the communion by inviting everybody here in the lower house to come.

$^{361}$forward to receive communion. We also have communion tables just here on the sides if it's easier for.
those in the fixed seating. For everybody here, you're invited to come around the doorway and take.
communion here. In the upper house, if we ask you to get out of your seats and come somewhere,.
it would be a nightmare. So for you guys, our team are now going to hand out communion to you as well..
But for you as well, I want you to take your time. When you get the bread or the cup, you can pray.
with the people next to you. You can share in communion together. You can allow yourselves to.
come under this faith step of celebrating what God has done in the blood of Jesus. Whether we're here.
in the lower house or in the upper house, can I pray for us? Then you're going to have time.
to come and do communion. Father, I thank you for each person here. I thank you for your blood..
I thank you that it really is your blood that sets us free. I thank you, Lord, that you.
are the Lamb of God, that you are the one who has sacrificed all everything so that we who might be.
guilty get to walk free. Father, that's the central thought of the Christian faith. And Lord, there.
might be people here in this room where they've never heard that before. Maybe some people in.
this room are online who don't know you. But Lord, they're here today and they've heard the most.
central gospel message there is, that God loves them. That God loves them so much that no matter.
what they've done, no matter what brokenness, what sin, whatever it is that they carry,.
by the blood of Jesus, they're forgiven. There's a Passover for you..
And if you're here in this room today, and this is the first time you've heard this message,.
this is the first time you've heard that there is a God who so loves you, that he would send.
his only son to die for you so that you would be released from anything that you've done,.
that's good news. That's the best news that there is. And if that's you, I want to encourage you.
just to say a prayer in your heart as we take communion. And that prayer is just to come before.
God and say, "God, thank you. Thank you for dying for me. Thank you for dying for the things that.
I've done in my life that I'm not proud of. I ask you to forgive me of what I've done..
And I ask you to set me free from it by your blood. And I receive your sacrifice.
on my behalf. Thank you, Lord." It's a very simple prayer. "Thank you. Please forgive me.".
And then thank you again. And then you do that, you're doing that step of faith that I talked.
about. You're taking the death of Jesus and then applying it to yourself. And that means that you.
become one of his children, that you become a firstborn son of God, a firstborn child of God..
And it's the most amazing, wonderful thing. So if that's you here today for the first time,.
we welcome you to pray that prayer and come and speak to us afterwards. And we'd love to tell.
you a little bit more about what it means to be a Christian. But for the rest of us,.
this is communion now. This is your time. You don't need to rush it. It's your time to engage.
in the bread and the cup and receive the grace that Christ's sacrifice gives for you..
[BLANK AUDIO].
\newpage



\section{}
\label{sec:PHwKp7Ievsk}
\textbf{2023-09-10 EXODUS - 14 And Also Bless Me [PHwKp7Ievsk].mp3}
\newline
\newline
連結: \href{https://youtube.com/watch?v=PHwKp7Ievsk}{\texttt{ https://youtube.com/watch?v=PHwKp7Ievsk}} ~~~~ 語音日期: 2023-09-10 
\newline
\newline
\hyperref[sec:ezohhaWO5XQ]{\small{< < < PREV SERMON < < <}}
~
\hyperref[sec:index]{\small{[返主目錄]}}
~
\hyperref[sec:lsj62DXXjZ4]{\small{> > > NEXT SERMON > > >}}
\newline
\newline
$^{1}$- Amen..
Amen..
Hey, can we thank our worship team as always..
Have a seat, have a seat..
Well, after about 400 weeks of this series,.
well, not quite that bad..
After though, 400 years of Israel being in Egypt,.
and after about roughly 200 of those years,.
them being in slavery in Egypt,.
and after 10 devastating plagues.
that have built up and built up upon themselves,.
Pharaoh's heart is finally softened,.
and he finally decides to let Israel go,.
to let Israel move into the freedom.
that God has from his compassion.
from chapter two onwards longed for them to have..
And after those 400 years of being in Egypt,.
those 200 years of being in slavery,.
the plagues that they were seeing,.
Pharaoh opens his mouth and he makes that declaration,.
and he says, "You can go, you need to go.".
And this is a nation-defining moment for Israel..
And this is a moment that's gonna set Israel up.
for the rest of their history..
To this day, they look back on this moment.
as a moment of significant breakthrough,.
as God shifts them from a place of slavery.
to a place of freedom..
And I wanna read this to you.
as we come into the word together today,.
because there's so much celebration and joy.
on the sense of God's heart.
as he sees his people released into their freedom..
Exodus 12, which is where we see this take place,.
starting in verse 31,.
and this happens on the night of the Passover,.
right where we were last week.
and where we left off last week, we continue right here..
During the night, that night of Passover,.
Pharaoh summoned Moses and Aaron and said,.

$^{41}$"Up, leave my people, you and the Israelites, go..
"Worship the Lord as you've requested..
"Take your flocks and herds as you have said and go..
"And also bless me.".
The Egyptians urged the people to hurry.
and leave the country,.
for otherwise, they said, "We will all die.".
So as the people took their dough.
before the yeast was added.
and carried it on their shoulders.
in kneading troughs wrapped in clothing,.
the Israelites did as Moses instructed.
and asked the Egyptians.
for articles of silver and gold for clothing..
And the Lord made the Egyptians.
favorably disposed towards the people.
and they gave them what they had asked for..
So they plundered the Egyptians..
The Israelites journeyed from Ramesses to Sukkot.
and there were about 600,000 men on foot.
besides the women and children..
And many other people went up with them as well.
and large droves of livestock and flocks and herds..
I wonder if you can sense the picture.
that Moses is trying to paint here for the reader,.
for this incredibly nation-defining moment..
He says 600,000 men..
He's using, unfortunately,.
the patriarchal culture of his day, discounting the men..
And then he adds women and children..
But scholars know that it was probably around.
about a million people that moved on that day..
A million people who were released from their slavery.
and were beginning their long walk to the promised land..
And this million people, it says here,.
there were also people from other places,.
many other types of people..
So it wasn't just the Israelites.
that were walking in this parade of freedom, if you will..
We know actually later on in this chapter.

$^{81}$that Egyptians come with them..
But we also know that there are other ethnic minority groups.
who were perhaps also enslaved in Egypt.
at the time that the Hebrews were enslaved there,.
who are also given their freedom.
and are also able to go with the Israelites..
So a million people walking together,.
carrying all that they owned,.
which as slaves, admittedly, wasn't very much,.
on their backs with their herds and their cattle,.
heading to a place of great freedom..
And you can almost hear the conversations.
that must've been happening in this moment..
You can imagine them saying, "Wow, can you believe it?.
Can you believe that this is actually happening?.
That actually change has come?.
That we're actually, God came through on his promises?".
(congregation laughing).
I'm like, "It's not that funny, but all right.".
What was happening?.
Is there a fly?.
I tell you what, that fly is from Satan..
We've seen that fly many times..
No, I'm not even joking..
Once when I was preaching, it was like buzzing,.
where it landed on my head like three times..
It literally is Beelzebub with wings..
There it is, in Jesus' name..
(congregation laughing).
I just prayed against distractions in the name of Jesus..
And everybody says?.
So many other people, flies included..
I've been to Egypt, lots of flies,.
are making their journey towards freedom..
And that conversation of hope is on their hearts..
But right in the midst of all of that,.
Moses records to us four words in this story..
Four words that actually seem completely out of place.
with everything else..
Four words that actually are going to set up for Israel.

$^{121}$much of what God is gonna do in and through them.
in the weeks and months ahead.
as they journey to their freedom..
Four simple words that were the last words.
that Moses hears from Pharaoh's lips..
Let me read these words to you..
It's this, "And also, bless me.".
Pharaoh turns to Moses and says, "And also, bless me.".
I wonder if you could get your head around this moment..
Pharaoh has the audacity to say to Moses, "Bless me.".
And you can imagine Moses going, "What?".
'Cause you've been the architect.
for around 80 years of your life.
of the oppression and slavery of my people..
You've been the very one that's decreed our hardship,.
that's caused my people to live in abject poverty,.
that's caused us to be under some of the worst slavery.
that's ever been in all of history..
And Pharaoh turns to Moses and says, "Would you bless me?".
And you get the sense in this moment.
that Moses is like, "What are you talking about?".
It defies all human logic to even consider.
that I would bless you for the years and the years.
and the years that your people have oppressed my people..
But Pharaoh says, "And also, bless me.".
When we sit in the amazement of those words,.
we have to wrestle with some pretty major things.
'cause those words actually put a blinding spotlight.
on what is perhaps one of the most challenging teachings.
of all of our scripture and even central.
to Jesus' Sermon on the Mount..
And it is this, the call that we all have.
to bless, forgive, and release those who have hurt us,.
shamed us, done evil against us, abused us,.
or situations in our lives that have happened to us.
that have caused us great pain and hurt,.
perhaps even failures that we've done.
that we still continue to carry around with us,.
whether it's people or circumstances or situations,.
the scriptures call us to this radical thought.

$^{161}$that we are to forgive those, to bless even those.
that have done harm to us..
I mean, get your head around this..
The very first act that Moses is called to do.
after he finally is set free is to bless the very person.
who has abused and caused trauma on his people..
And there are, for all of us in this room,.
people that have hurt us, things that have been done to us,.
things that have come against us in our lives..
And if you're anything like me,.
every single one of us in this room,.
we carry baggage of those moments with us..
We carry the baggage of the words, of the abuse,.
of the trauma, and of the hurt..
And when Pharaoh turns to Moses and says, "Bless me,".
you almost get the sense that God is standing right there.
in the moment alongside with Moses,.
and that God is almost over Moses in this moment,.
wondering and looking at what Moses will do..
Will Moses have the courage to walk in the opposite spirit.
to all that Pharaoh had done to him?.
Would Moses have the audacity to actually do something.
which doesn't seem to make any human sense?.
Would Moses choose to walk in the opposite spirit.
of a hardened heart and have a soft heart.
to the one who had so abused him?.
Or would Moses turn his back?.
Would Moses refuse?.
Would Moses harden his heart and walk away.
in a place of bitterness and resentment?.
It's important you understand.
that Pharaoh did not deserve forgiveness..
Pharaoh did not deserve forgiveness..
He had done what he had done..
And it almost seems like it would have been unjust.
for Moses to respond in this way,.
but God's ways are actually often so different to our ways..
God's ways and his call is often so much higher.
than our call..
And what God is about to do in and through this moment.

$^{201}$is something in Moses' life.
that would be not just a turning point.
for Israel towards their freedom,.
but a turning point for Moses himself..
What's interesting is that the text doesn't tell us.
whether Moses did it or not, which is really fascinating..
And although it's an argument from silence,.
scholars actually have analyzed all of the five books.
that Moses is writing, the first five books in the Bible..
And pretty much every single time.
that Moses is asked to do something that Moses does not do,.
he says he does not do it, and he explains why..
And in this moment, scholars say.
that because Moses is silent on the issue,.
as there are many other examples in those first five books,.
that when he's silent, it means he did do it,.
that the majority of scholarship believes.
that Moses didn't do it..
He did bless..
He did offer his forgiveness..
He did turn to the very person.
who had so hurt him and his people,.
and he forgave him for those actions..
And I think as we sit with this, as I sit with this,.
I realize that I would not have..
And I think if we're honest with one another.
here in this room today, if you're honest with yourself,.
I think perhaps all of us would have struggled to do that..
I mean, think about the weight of that moment,.
and all that it seems to hold..
But God is calling in Moses and in Israel.
something that is so deeply profound.
that we all have to wrestle with,.
that our ability to forgive and to bless.
those who have done harm to us.
is perhaps the single most important thing we can do.
to both embrace and retain the freedom.
that God has designed for us..
Come on, church..
Our ability to forgive and to bless,.

$^{241}$to release the people who have done harm to us.
is actually perhaps the most important thing.
that we will ever do to enable ourselves.
to truly be free in this world..
And God is standing with Moses in this moment,.
wondering what Moses will do..
God knowing that if Moses refuses to bless,.
then Moses is going to leave Egypt with a heavy heart..
Moses is gonna leave Egypt from a place.
where he's still carrying so much of that pain..
So you need to understand that it's perfectly possible.
to be physically free, but to be emotionally,.
spiritually, and mentally chained..
And God knows that if Moses doesn't do this,.
if Moses can't find the ability in himself.
to let Pharaoh go, to release him,.
to forgive him, to bless him,.
then he's gonna be emotionally, spiritually,.
and mentally chained, even in the place of freedom.
that God's given to him..
And so God is inviting this moment..
It's a terribly difficult one to get our head around..
It doesn't seem in our human thinking to be just or right..
And yet in it, in it,.
comes the greatest amount of freedom we will ever, ever know..
The freedom of being set free from the things.
that wanna continue to chain us..
In this way, this passage is actually not about Pharaoh.
releasing Israel..
This passage is about whether Israel will release Pharaoh..
And Pharaoh's request to Moses.
is actually God's invitation to Moses..
God's invitation for Moses to be able to finally,.
fully, and completely walk free from the pain of his past..
And I want you to know this today,.
whether you're in this room or online right now,.
it's the same invitation today to you..
This moment right here, as we unpack this idea.
of what it means to love your enemy,.
what it means to bless those that have hurt and harmed you,.

$^{281}$right here is an invitation for you.
to walk into the greatest moments of freedom.
you will ever experience..
Here's the reality..
In every single moment of your life,.
there are three paradigms at work,.
your past, your present, and your future..
And the great thing about being human.
is that all of us in every moment of life,.
we're moving towards our future..
We're constantly on the move towards our future..
But when somebody or a situation or a circumstance.
has hurt us, and when we carry that hurt.
and we hold on to that pain,.
when we feel that anxiety and that fear.
and that frustration, that sadness, that depression,.
whatever it might be, when we're holding onto that,.
here's what it's like..
It's like we're walking into our future,.
facing towards our past..
It's like the past is the primary thing we're focused on.
as we're moving to our future..
When you do that, you're walking into your future,.
facing backwards..
Now, if you're worried, I'm gonna fall off the stage,.
you probably should be worried..
Because when you move into your future, facing backwards,.
then you're in danger of not moving into the future.
that God really has for you..
And I can't tell you the number of people,.
myself included, many times,.
who are walking into our future, facing backwards,.
because we're allowing the pain and the hurt.
and the frustration and the disappointment,.
or whatever it was that happened in our past,.
to be the primary paradigm that's shaping our present.
and telling us about our future..
Now, don't get me wrong,.
there's nothing wrong with the past..
The past is actually a good thing..

$^{321}$God's done a lot of things in your life in the past, I hope..
And the past is a place where we get to see.
some of that stuff, we get to learn from our past..
Even some of the hard things that have happened in our past,.
it's a way of shaping us and making us mature.
and helping us to be more of the person.
God's called us to be..
So it's not like the past is wrong,.
but the problem comes when we allow the past.
to be the primary paradigm.
by which we think about the present.
and we work towards the future..
God does not want you to walk into your future.
facing backwards,.
he wants you to walk into your future facing forwards,.
facing towards all the promises.
and the goodness that he has..
But that's gonna require you.
and also bless me..
It's gonna require you to do the one thing.
that is the most powerful thing.
to release you from the pain of your past..
And that is forgiveness..
I was in Belfast a few years ago..
And if you've ever been to Belfast in Northern Ireland,.
fascinating city,.
a city that's gone through so much in terms of its history,.
it's the place of the troubles.
where Catholics and Protestants.
were basically at war between themselves.
on two streets of the city..
And when I was there, I paid for a walking tour..
I like to walk and I like cities, walking tours are great..
And you basically, you hook up with this local guy.
and the local guy takes you around for two hours.
and he shows you the pubs where the bombings have happened.
and the roads and the separation.
between the Catholics and the Protestants.
and you kind of get a sense of everything that's happening.
or had happened in the past in Belfast..

$^{361}$And as we're walking along,.
he pointed out all this graffiti that's in the city.
and he pointed out this one particular graffiti,.
I wanna read this to you, it said this..
A nation that keeps one eye on the past is wise..
A nation that keeps two eyes on the past is blind..
And I think you could take that.
and you could apply that right to this moment.
here in Exodus,.
as God has now released his people and he looks at Moses.
and he says, "Moses, keeping one eye on the past is wise,.
but keeping two eyes on the past is blind.
and also bless me.".
That the only way, Moses, that you can walk forward.
into the freedom that I have for you.
is to release all of the things that's holding you back,.
all the things that are bounding you backwards.
into the past that has hurt you,.
the past that has enslaved you..
The only way you can move forward in the way that you can.
is by keeping one eye on the past,.
learning and growing from the things.
that have experienced you,.
but keeping those eyes focused in the front,.
releasing your two eyes, finding that freedom..
And it will only happen.
if we can find in ourselves.
the ability to bless,.
the ability to forgive,.
even the ones that have done the worst things to us..
God invites us to recognize.
that our freedom only will come.
when we can let the past be the past..
The question we should ask ourselves then.
is how do we do it?.
'Cause Andrew, if you just understood the pain,.
I mean, you wouldn't ask this of me..
Jesus, if you understood what I've been through,.
you wouldn't command me to be one who forgives..
How do we even get ourselves to the point.

$^{401}$where we can do that?.
To release the ones that have so hurt us.
and shaped us and shamed us,.
or even release ourselves..
Sometimes the hardest person to forgive is yourself..
Some of you in this room watching online right now,.
you're living a certain way in your life today..
You're doing a certain job that you're doing.
because you've told yourself you don't deserve any better.
'cause of something that you did in the past..
And you're as much enslaved to that thing.
as if someone else had done something to you..
Some of you need to go and also bless me to your inner self.
and learn what it is to let yourself go.
from the mistakes and the brokenness of your past..
When I was in Egypt, I had a incredible chance.
to sit down with a gentleman by the name of Peter..
And Peter had experienced perhaps some of the most.
painful atrocity and trauma.
that you could perhaps even face in your life.
and had learned what it is to walk out of that.
in a call to forgiveness,.
to forgive the very ones that had brought so much pain.
into his own life..
And as he sat there and shared his story with me,.
I realized that there was so many profound insights.
into this idea of what it means to bless.
and to forgive and to release.
that I wanted to sit down with him and capture it on camera,.
which is what we did..
And I can't wait to show you this..
It's not easy in places to see what happened for Peter,.
but I believe that inside of his story,.
there are so many nuggets for how you might be able.
to walk into better freedom for yourself..
So let's take a look at Peter's story..
Peter, we are so very grateful.
that you're with us here today.
and that you're gonna be willing to share your story with us..
Thanks so much for being with us..

$^{441}$- Thank you so much, Andrew..
Thank you..
- Why don't we start by you just explaining.
a little bit about yourself,.
where you're from, where you grew up,.
your family a little bit..
- Okay..
So we are a very normal family of me, my father,.
my mother, and my sister, Maria..
And we all grew up in Egypt, in Cairo,.
and we are all a Coptic Orthodox people..
And we, like any normal family,.
we just go into Sundays into the church..
And then at some certain point,.
I had a turning point in my life.
after an incident happened in my life.
because of the bombing that happened because of ISIS..
- Yeah..
So my understanding is this took place in 2016.
in your church, essentially, here in Cairo..
- We came and we all thought that nothing would happen.
in such a day because it is a very peaceful day..
And we just go, me, my father, my sister,.
and my mother to the church..
And then in the middle of the Holy Mass,.
a guy came into the church,.
and actually he looked very weird..
Like even his face, even like you can Google him..
You would like, you would notice him without a doubt.
that he's not a Christian..
He's not full of peace..
He's not all of that..
And then after the church was all happy,.
was the church and everybody was going with the families,.
in a moment, I didn't hear anything..
The whole church went into darkness..
And then we saw like a dust everywhere..
And then the chairs, the wooden chairs.
were like flying everywhere..
And the glasses were throwing everywhere..

$^{481}$And then there was a moment of silence..
Everybody was understanding what is happening..
And then I heard like screams everywhere..
And I swear I never heard such screams.
like this in my whole life..
And it was a slaughter..
- A bombing at Cairo's largest Coptic cathedral.
has killed at least 25 people.
and wounded nearly 50 others..
Most of the victims are reported.
to have been women and children..
The explosion took place during a packed Sunday morning mass.
in a chapel adjoining St. Mark's..
The cathedral in the capital is the main headquarters.
of Coptic Christians in Egypt..
- And then I even forgot that my mother.
and my whole family was with me..
So I just went to the place where they were screaming..
I didn't know what happened..
I just started to carry the dead people..
I just thought that they are okay.
and maybe they are just injured..
And then I started to remember.
that my mother and my sister are with me..
And then when I came to my mother and sister,.
I just saw my father in front of me..
And he was asking me, "They are both totally injured..
"Whom should I pick to just go to the hospital?".
And then it was a very hard question..
It was a very hard decision..
At the time, we just decided to pick up my sister.
because we felt like her chances are more better.
than my mother..
And we even both, we just thought that my mother died,.
but we couldn't face each other at that point..
And then I picked up my sister.
and I went to the hospital..
And even when she was in the middle of her pain,.
she was asking about, "Where is Peter?.
"Where is my father?.

$^{521}$"Where is my mother?".
And after she felt like everybody is okay,.
she felt like she started to look another way..
And then she was keep calling the name of Jesus..
I felt like, "Jesus, Jesus.".
And I was asking her if she can see anything..
And she just said, "No, I just wanna see,.
"check that you're okay.".
And that's it..
And she passed away..
- And you must, I mean, how did you feel?.
Were you angry?.
- I was totally angry, Andrew..
I was not that good believer that he was saying,.
"No, this is what was written in the Bible.".
No, I was totally mad..
And to be honest, I didn't expect God,.
or I didn't expect Jesus to understand my feelings.
at this point..
I didn't even know how he is going to respond me..
- Yeah, it must be really quite incredible to think.
you're in a church, you're in God's house,.
and this tragedy happens right there..
And it's very understandable that suddenly you were like,.
"Why did God allow this to happen?".
And the anger that you felt towards him..
It must have been very strong in you..
- Yeah, and where is all of your promises.
that you are going to protect me,.
that no one will hurt me,.
that you are like in your hands, God?.
Where are these promises?.
Are these fake promises?.
What was all this book and all this Bible about then?.
And I started to pray,.
and I felt like I need to say everything in my heart,.
even if I feel that it's not right to say to God,.
but I need to be honest,.
because this is a turning point..
And then I started to say to him,.

$^{561}$"Since my childhood, my mother that you took from me.
"and my sister, they were encouraging me to pray.
"and to be and serving in your house for you.
"and for the glory of your name..
"How could you do this to me?.
"How do you leave all the bad people are living.
"and they are enjoying?".
By the way, they were even,.
when I was carrying my sister,.
there were people outside of the church.
that they were supporting the Islamic people..
They were just happy,.
and they were clapping while I was carrying my sister..
- What?.
- I told you I passed by a very difficult moment..
- Let me also ask you,.
so you were mad at God, very understandably..
How did you feel towards ISIS,.
towards those that actually detonated the bomb.
in the church?.
- If I'm standing in front of like a big hole, okay?.
- Hole..
- Hole, yeah..
And someone, a deep hole,.
and then someone pushed me,.
and then I became totally injured and seriously injured..
But by any somehow, I just got up.
and then I came out of the hole..
What would be like my reaction?.
I think to take revenge of that guy..
But what if I saw that this guy is a blind guy?.
What are you going to do to him?.
And he's totally blind..
He never saw Jesus..
He never knew about peace..
All that he knew is just he needs to go to heaven.
and he needs to be crucified somehow..
And this is--.
- Suicide bomber..
- Yeah, and this is how he's going to heaven..

$^{601}$He wants to kill the Christians.
because they are against God..
And maybe he was doing this because he loves God..
Although this is not what I do believe..
I believe like who tries to find God,.
God would find him as he's standing in front of door..
But he wants a certain things, and that's it..
- You're angry at God..
You're obviously feeling strong emotions.
towards the people that have killed your mother.
and your sister..
We also know that the Bible speaks about forgiveness..
And perhaps one of the most challenging things.
the Bible says about forgiveness is when Jesus says,.
"Well, if you don't forgive, I don't forgive you.".
So it's a very strong command..
And I think for a lot of people watching this,.
they're probably thinking,.
well, they have to forgive their spouse.
if their spouse says something mean to them,.
or maybe they have to forgive their child.
if their child does something naughty..
You're in a situation where you need to forgive.
somebody for killing your mother and your sister.
while you were in the room..
I mean, I don't think there's many of us.
who are watching this who have ever had.
that kind of level of need for forgiveness..
Can you talk us through a little bit of that journey for you?.
How did you ever even get yourself to the point.
where you could take that command of Jesus.
and apply it to this incident in your life?.
- To be honest, Andrew, at the first point.
or at the very beginning, I was even thinking.
that I want to get their revenge..
I want to get their revenge..
And I didn't know how..
I never used to killing..
I'm not used to killing..
I never killed a fly..

$^{641}$So what should I do as a revenge?.
I even started to look in the Bible..
How can I go to revenge?.
Just give me a single example in the Bible.
of someone who got revenge..
And actually I got that example..
I started to think, when Stephen died,.
for example, he was that guy who was just carrying.
the clauses for the people who are going to kill him..
- That's right..
- Right?.
When the soul of St. Stephen went to the skies.
and went to God, there is a part in the Bible.
who says that the souls of the martyrs came to the Lord.
and then said, "When you are our almighty God,.
"we are going to have a revenge for us.".
And he said he will get their revenge, but later..
Then I started to think, how did Stephen get his revenge?.
I think he got his revenge from the kingdom of darkness.
by getting Paul out of darkness to be the strongest guy.
who tells about Jesus..
And then I started, okay, this is my turning point..
I want to take that point..
I want to take that revenge..
I will take everybody who just blinded by the darkness.
to just get them from darkness into light..
And this will be my strongest revenge..
- Wow..
What was it that really enabled you to fully forgive?.
'Cause obviously you were carrying some anger to God.
and to ISIS and to that situation..
What was it that really allowed you to fully forgive.
and to release your forgiveness?.
- As I told you, it was always like a continuous way.
of thinking..
My head was, I couldn't stop thinking at any time..
But every time I was thinking, I was just, deep inside me,.
I was just telling Jesus, "Please don't leave me..
"Please don't leave me..
"Just guide me.".

$^{681}$And by every time I was just thinking in a bad way,.
or I'm just thinking to hurt someone,.
I knew deep inside that this was not right..
This was not what Jesus wanted me to do..
But at that point also, I totally knew.
that Jesus totally respects all our human feelings..
And it just needs to be guided..
And that's it..
- And it seems like in all that you've been sharing,.
that forgiveness really happens.
through a process of being led by Jesus,.
knowing Jesus is with us, knowing He's present with us..
And we're so grateful for that..
I wanna thank you so much for sharing your story with us..
I know it's not easy to relive emotions.
and feelings like that..
And it's obviously a very personal story.
that you've so graciously shared with all of us..
And I know that people watching this,.
they're gonna be able to connect to your story,.
even though perhaps they've never experienced.
anything like that in their life..
But all of us have been commanded to forgive..
All of us have to bring ourselves to the point.
where we even have to forgive those that we don't like,.
forgive those that are our enemies,.
even forgive those perhaps that have done.
the atrocious things that have happened to your family..
And the only way we can do it.
is if Jesus is with us in that process.
and if Jesus meets us in that process, as you've shared..
Thank you, Peter, deeply for being willing.
to share that story with us today..
And we're so grateful to you..
- Thank you so much, Andrew..
Thank you..
- Truly profound, right?.
What a beautiful man, what a beautiful story..
And there are so many things that he shared in there.
that I think are very helpful for us.

$^{721}$as we think about the journey that we have.
to release the pain and the hurt.
and the brokenness of our past,.
and particularly release those that may have done it to us..
I wanna share just a couple of quick ones.
just practically for you,.
because I want you to be able to practically.
take something away from what we've been talking about today..
Here's the first thing, always put safety first..
Always put safety first..
What I mean by that is we need to distinguish.
between moments where we've had an abuse.
or something that's happened to us in our past.
and that's finished and we still feel the effects of it,.
but the abuse is finished,.
to moments where we're still currently.
under a place of getting abused..
Obviously, the call to forgive.
when we're in a place of actual ongoing abuse.
doesn't make sense and isn't right..
Safety is always first..
We have to get ourselves to the place.
where we come out from under that abuse.
before I think Jesus leads us into a process.
of forgiving the person that's been abusing us..
You gotta note that Moses,.
Moses was actually in a process of his freedom.
when God asks him to bless Pharaoh..
God doesn't ask him to do that.
whilst Pharaoh is still abusing them.
and they're still under a context of slavery..
It's only once that slavery has finished.
that Moses is in a place where he can begin a process.
of understanding what it would be for him.
to bless and forgive Pharaoh..
It's very important you understand this.
'cause forgiveness does not mean.
that you should stay in a situation of abuse..
Forgiveness does not mean that you should go back.
to a person potentially.

$^{761}$who has been abusing you in the past..
Sometimes in my pastoral work.
that I've been doing with the church over so many years,.
people begin to think that forgiveness means,.
oh, well, I guess, Pastor Andrews called me to forgive..
I guess I just need to stay being abused.
by this person in this situation..
That's my way of showing forgiveness..
That's not what this passage is saying.
and it's not what we're preaching, safety first..
Everybody understand that?.
Okay, it's really important..
But here's the second thing, okay?.
Know that you're in a battle of power..
I thought one of the things that Peter shares there.
was so helpful for us to think about the reality of power..
And first of all, you've got to understand.
that you're in a battle of power.
by what the enemy has done to you.
through whatever circumstances,.
situations that's happened in the past..
And that power is held over us by manifesting itself.
in the emotions that we carry today.
when we think about that event in the past..
So when we feel today anxious and frightened or sad.
or angry or bitter or whatever it might be,.
shame or failure today because of something in the past,.
that's the power that that thing still has over us..
And forgiveness is God's way of disempowering.
the power that that event has taken hold over us..
And so the reason why God calls Moses to do this.
is because he wants that power to go..
He wants the power of all that slavery.
to be disempowered in Moses's heart and life.
by his ability to let it go..
So you need to understand that you're in a battle of power.
and the work of forgiveness is your way of disempowering.
what the enemy is trying to do for you.
to keep you locked in decisions.
about your present and your future.

$^{801}$based on the pain and the hurt.
of that situation in the past..
Does that make sense to you?.
And I love the way that Peter brings this up as well..
He talks about that idea of revenge..
And he says, "I wanted to get revenge.
"by physically doing something to that person.".
But then he said, "Revenge actually is this,.
"taking somebody out of the kingdom of darkness.
"and bringing them into the kingdom of light.".
Wasn't that beautiful?.
He said, "The best way that I can fight back ISIS.
"for what they've done for me.
"is not by picking up bombs and shooting them.
"and making violence on violence..
"The best way is to embrace a nonviolent response.
"in finding myself blessing them by sharing the gospel.
"because they're blind people knocking people into holes.".
I mean, the guy whose mother and sister was killed.
is able to say, "There's a gospel work.
"that could be done out of this.
"that is the best form of what it looks like.
"to make this thing right.".
And Peter's dedicated his life now.
to being somebody who menaces the gospel.
and shares the gospel with Muslims.
and Islamic people of that background,.
believing that if he can help some people.
come to see God for who he truly is,.
then something can really change in his city..
Something can really change in his culture..
And also bless me..
The third thing is this..
Confront the hurt that you are carrying..
Notice that I didn't say,.
confront the person who has hurt you..
Confront the hurt that you are carrying..
Sometimes we think that forgiveness and reconciliation.
are the same thing..
Forgiveness and reconciliation are not the same thing..

$^{841}$And sometimes people mistake the idea.
of the command to forgive,.
meaning that they have to reconcile with that individual..
And if they don't reconcile with that person,.
then they haven't really forgiven them..
You need to understand that forgiveness and reconciliation.
are two different things in scripture..
Forgiveness, a command of Christ that we can do,.
whether the person that has hurt us or not responds at all..
We can still do that process of forgiveness..
Reconciliation, requiring two people to come together.
and build a new relationship together,.
be committed to doing that..
And you can't always reconcile with the person that hurt you..
And in some cases, it's actually not safe for you.
to reconcile with the person who's abused you..
But what you can do is forgive them..
What you can do is bring yourself to a place.
of releasing that hold that they have..
And the way that you do that is you confront.
where that power comes from and the source of that power..
That's in your emotions..
That's in the things that you're carrying,.
the hurt and the shame and the pain and all of that..
As you confront the hurt you are carrying,.
you expose it in order for it to be healed..
My wife, who's a therapist, said to me last week,.
and when she said this,.
it was like one of those mic drop moments for me..
She said, "You can't heal what you don't feel.".
And so often I think what we think.
is the way we forgive somebody for hurting us.
is by burying the hurt, ignoring the hurt,.
trying to pretend like the hurt's not there..
And I love Peter in the film..
He's like, "Andrew, I was feeling all this stuff..
"And I had to say some things to God.
"that I wasn't even sure if you could say to God.".
Because he had to let his feelings out..
He had to find an expression of that..

$^{881}$He had to make sure that those feelings didn't consume him..
He didn't bury those feelings so deep inside of him.
that his whole life is shaped by those feelings.
in expressing them, in sharing them..
He was getting them out..
In confronting the hurt,.
he was placing himself in a position.
to be healed and restored..
Now, the reason why we can do this.
is 'cause it's not like we're serving some God.
who's never felt anything in his life..
Jesus, at the very core of the Christian faith,.
comes and takes on flesh,.
becomes one of us and feels our pain with us..
He's not up on some cloud going, "Feel your pain.".
He's like, "I have come to feel your pain..
"I have come to feel what's even worse.
"than you may ever feel in your life..
"I've come to take on the sin of the world on my shoulders,.
"and I will be on that cross.".
And you know the words that come out of his mouth.
on that cross?.
"Forgive them, Lord, for they know not what they do.".
Jesus, in taking on the pain of the world,.
means that when we follow Christ,.
we have somebody who knows and understands.
and advocate for the pain that we feel,.
and there is no pain that we will feel.
that Christ has not already felt and not overcome..
So confront the hurt you're carrying..
Here's a way you can do that..
Just get a journal, get your phone with Notes app.
or whatever, sit down and say,.
"Lord, here is what I'm holding..
"Here is what has happened to me in that situation,.
"and here's the fear, here's the anger,.
"here's the shame, here's the hurt.".
Whatever it is, and you write it down,.
just write it down, and writing it down,.
it's a prayer to him..

$^{921}$He's saying, "God, I'm not gonna hide from these things,.
"I'm not gonna ignore these things..
"They're right here in my life, and I recognize them,.
"and I recognize that these things are doing stuff.
"that I don't like in me, and so I bring them to you.".
And pray and release those feelings..
Ask the Holy Spirit to come and meet you in those feelings..
Ask him to free you from the ways in which you feel,.
and you will find yourself set free in a way.
that you've never been felt before..
I love the way Peter goes,.
there was such peace for him,.
as he was able to confront the feelings he had.
in his conversation with God..
Here's the fourth thing..
Have a community around you..
Rely on your community..
Share with your community..
The number one thing that the enemy will do.
with the hurts in your life is to cause you.
to make it private..
The enemy wants you to soak in the privacy of your pain..
And when you expose that pain,.
when you expose it and share it with the people.
that Christ has put around you in your life,.
that's what church is all about..
That's what our community groups are all about..
That's what spiritual friendships are all about..
It's about having people in your life.
that you can walk with and say, "This is how I'm feeling..
"This was a situation that happened to me..
"I need prayer, I need support.".
Paul, when he writes to the church in Galatia,.
he says this, in Galatians chapter six, verse two,.
he says, "Carry each other's burdens,.
"and by doing this, you'll fulfill the law of Christ.".
Share your burdens with one another..
And you'll actually live out the way.
that Christ wants you to live in this world..
It's not about a privacy of pain..

$^{961}$It's about saying, "I can't in and of myself carry this..
"I can't deal with this..
"I need someone to pray for me, stand with me,.
"encourage me, walk with me..
"Would you help me?".
None of us are adequate in and of ourselves.
to process and deal with some of the deep trauma.
that we carry..
And unless you process it, you'll carry it around with you.
for the rest of your life and never truly be free..
The church is a great gift to you..
Whether it's Oasis,.
having professional people supporting you..
Whether it's our prayer team here at the Vine.
after a service and having people.
who have prayerfully trained themselves.
in ministry of prayer to pray for you..
Or whether it's your community groups.
or people around you, get a community..
Get it a community..
That is an important part of the process.
of forgiving those that have hurt us..
And here's the final thing..
Turn to your future..
It's really important that you know.
that your past is not the model for your future..
You need to really understand that the brokenness.
and the pain and the hurt of your past.
is not the thing that determines.
the direction of your future..
All it does is determine your starting point..
And the whole of the Exodus,.
the word Exodus itself means, remember, a departure point..
The whole of it means to depart from something..
And when you find yourself in a place.
where you can do this, where you can turn to your future.
and say, I want the promises of God.
more than the pain of holding onto that..
I don't want two eyes on my past..
I wanna have my eyes focused looking forward..

$^{1001}$As we turn to the hope of our future,.
we're able then to release our past.
and not allow our past to determine.
the direction of our future,.
but instead allow God's promises,.
His word, His hope, His freedom.
to tell you who you should be today.
and who you will be tomorrow..
Too many of you in this room,.
you're allowing that pain and that hurt.
to define who you are..
God says, and also bless me,.
release the pain,.
break the power that it has in you.
through this subversive act of forgiveness..
And as you forgive that,.
you'll find yourself putting your hands on the plow.
and not looking back over your shoulder.
and be able to walk confidently.
into the future that God has you..
And I wanna close by saying this,.
this whole idea of and also bless me.
is perhaps the most significant thing.
that our city needs right now..
'Cause if you think about all that's happened.
over the last four years,.
with the protest time and the COVID time,.
and when you look at all the things.
that have happened through that,.
there are so many people in our city.
that are still holding on to the hurt.
and the anger and the bitterness.
and the brokenness towards one another..
And I tell you this,.
the thing that's really gonna help Hong Kong to flourish,.
it is not going to be the best economy..
It is not going to be the fact.
that we can all travel now and have great holidays..
It's not going to be the fact.
that we don't have to be in quarantine anymore..

$^{1041}$The thing that's truly gonna flourish this city.
is when this city is able to forgive one another..
When this city finally gets itself.
to the point where it can actually release one another.
from the hurt that we have caused one another..
And I don't know about you, but if I think about for me,.
there are many pharaohs that I can think of.
over the last four years for me..
And maybe that is the same for you..
And whether you're pharaohs of the government,.
the police, student protesters, COVID policies,.
restrictions that were placed upon you,.
laws that have been passed, whatever it might be for you,.
and also bless me..
'Cause the true hope for our city.
will be when the church can demonstrate.
the power that there is in letting go,.
the power that there is in forgiving and loving.
even those that have harmed us and hurt us..
So as you walk into your freedom,.
as Israel moves into their freedom,.
as Moses begins to journey them towards freedom,.
the first thing Moses has to do is say,.
"I bless you, I forgive you, I release you,.
'cause I don't want anything that you have done to me.
to be a part of my present or a part of my future..
And I am gonna disempower the pain and the hurt.
by releasing it through the precious blood of Jesus Christ,.
releasing it so that I can truly be free.".
Some of you today, that's the greatest thing.
that you will ever do,.
is find true freedom from your past.
so you can walk into your future..
Can I pray for you?.
Let's pray..
Father, I thank you for every person here.
in the room and online..
Father, we just take a moment now to sit.
in the need that we have for your grace and your spirit..
Father, I pray for every person.

$^{1081}$because we are all carrying some hurt,.
some trauma, some pain, some challenge, some circumstance..
Some unforgiveness that sits within us..
And though we might think that most of the time.
we can get away with it,.
we recognize that actually.
it's slowly disintegrating our soul..
It's slowly hardening our heart.
and it is holding us more captive than we ever realized..
And so, Father, we come to you today in hope..
We come to you with this profound idea.
that Moses had to bless Pharaoh.
as the very first thing he did in his freedom..
And Lord, for us as a church to be free,.
for us as a city to be free,.
and for us as individuals to be free,.
we're gonna have to learn what it is.
to journey in forgiveness..
Father, as we put safety first,.
as we understand that we're in this battle of power,.
that that ultimately sits in your hands,.
as we confront the hurt that we carry.
and don't allow that hurt.
to be the predominant thing that drives us,.
as we look to one another to support.
and find that community together,.
and as we decide to turn to our future,.
knowing that our past doesn't determine.
the direction of our lives,.
Father, in this, we ask that you would truly set us free..
I want you just to take a moment,.
just with you and the Holy Spirit,.
and just allow the Holy Spirit just to speak to you.
and meet you where you are at in this moment..
Come, Holy Spirit..
The Holy Spirit's gonna bring to mind.
for some of you people that he wants you to bless..
People that he wants you to forgive..
The Holy Spirit's gonna show you.
maybe even some of the pain,.

$^{1121}$maybe some of the hurt that he wants to release from you..
And as I said early in the message,.
some of you in this room,.
and I feel this particularly for this service,.
it's you who has hurt yourself,.
and you can't forgive yourself.
for something that you did in the past..
God wants to release you from that self-hatred.
that you're carrying..
Some of you, that's a word I sense in my spirit.
as I'm praying, self-hatred..
Some of you in this room,.
you really don't like who you are..
That's a work of the enemy over you,.
and we as a church stand with you,.
and we pray that you would come to see how God sees you,.
and that you will be set free from the self-hatred.
that the enemy has placed upon you,.
and that you'll realize that your past.
doesn't determine your future..
Come, Holy Spirit, come..
Father, we stand in the hope..
In fact, I just feel like we should do that..
Would you just all stand with me?.
(gentle music).
I wanna pray this over you..
Father, we stand in the hope that comes.
because of the work of the cross of Christ,.
and although this teaching today is challenging,.
Father, it will set us free,.
and so, Father, I wanna pray that you would bring freedom.
for every person here..
I wanna pray for the hope that there is.
in the cross of Christ..
I wanna pray for the blood that we talked about last week.
to truly set us free..
Lord, I wanna pray against the strategies.
that the enemy has over each family, over each marriage,.
over each parent, and over each child,.
and over every business,.

$^{1161}$that the enemy has these strategies.
to lock us to the brokenness of our past.
when God wants nothing more than the hope of our future..
Lord, I pray, releasing in Jesus' name.
to the hope and the glory of your future, Lord,.
and I pray, Lord, that people this week.
would feel a peace, a lightness, a release.
from emotions and feelings and pain.
that they've been carrying for a long time..
Lord, there are years that are being wiped away.
in the name of Jesus right now..
Lord, I pray for the glory that comes..
Lord, we thank you for Peter.
and the way that he can speak so positively now.
about the joy that is found.
in releasing those that have harmed his family..
Lord, I pray for the power that will come for the vine.
out of a group of people who can release the pain.
and the brokenness of their past,.
who can walk into their futures facing forwards..
Lord, I pray for the power that that would be.
for the city of Hong Kong..
And Lord, we ask that you would do it amongst us.
in Jesus' name..
Everyone says..
- Amen..
Can we worship together?.
Let's worship together..
Let's finish with a moment of worship.
before we do anything else,.
and then Kyla will come in a moment.
and pray over you as we depart..
But let's worship..
(music).
\newpage



\section{}
\label{sec:lsj62DXXjZ4}
\textbf{2023-09-15 EXODUS - 15 He Can Part The Sea [lsj62DXXjZ4].mp3}
\newline
\newline
連結: \href{https://youtube.com/watch?v=lsj62DXXjZ4}{\texttt{ https://youtube.com/watch?v=lsj62DXXjZ4}} ~~~~ 語音日期: 2023-09-15 
\newline
\newline
\hyperref[sec:PHwKp7Ievsk]{\small{< < < PREV SERMON < < <}}
~
\hyperref[sec:index]{\small{[返主目錄]}}
~
\hyperref[sec:JVjucZ_U4Bw]{\small{> > > NEXT SERMON > > >}}
\newline
\newline
$^{1}$In Jesus' name, we bless this..
Amen..
(congregation applauding).
Hey, before we do anything, before you sit down,.
before you sit down, I wanna lean into this sense of faith.
a little bit more..
Today's an exciting day in the history of the Vine..
We're having our first formal services in Yuen Long.
in the new location that you guys have given.
so faithfully towards..
Pastor Ivan and Pastor Tim,.
the Cantonese service started at 9.30, that's already done..
The English service is happening right now..
They're gathering there for the first time.
in their new building..
We went in on Wednesday night, a group of us,.
and we prayed and dedicated that space..
It's happening right now..
And let's pray in faith that God will do something.
that only God can do through that new community,.
that new church that's planted there..
So would you pray with me?.
Let's pray..
Father, we thank you so much for what you're doing.
in the life of Yuen Long and all the things.
that are on your heart for the people of Yuen Long..
And Lord, you've called the Vine to plant a church there..
To plant a church there, to express something.
of your gospel, of justice, of hope and life there..
And Father, as they're gathering in Ho Shin Lee building.
for the first time, their permanent home.
after all the renovations and all the fundraising.
that we've done, they're literally breaking ground..
They're there right now..
Father, we wanna pray that you would release.
a spirit of anointing upon that community..
Father, we pray that you would bring salvations.
in the house today..
Lord, we pray that people that don't know you.
would come into this church and discover a God.

$^{41}$who has set them free..
Lord, we pray that everything we've been talking about here.
in the Exodus series would overflow, Lord,.
into the life of that community.
and that they would lean forward towards society around them.
with the hope of the gospel..
Lord, we pray for Tim and Ivan as they lead that community..
We ask that you would fill them with your power..
And Lord, we pray you would speak to them.
in ways that they've never experienced before,.
that you would give them an extra anointing, Lord,.
for the new season that they're in..
And Father, we stand with this community..
They're a part of us..
It is our expression here in the city..
So we fan into flame everything that you're doing.
in that space..
And we are looking forward to hearing the testimonies.
of the goodness of God in the land of the living..
And we pray this in Jesus' name..
Everyone says?.
Amen..
Can we thank the worship team?.
They're gone..
(congregation applauding).
They're gone..
All right, have a seat, have a seat..
Hey, if I was to ask you this morning,.
what is the most spiritual act.
that a Christian can engage in?.
I wonder what you would say..
What is the most spiritual act.
that a Christian can engage in?.
I wonder what would be on your heart for that..
Some of you in here would probably say prayer..
Prayer is the most Christian thing that we can do..
When we pray, we're connecting ourselves to God.
and we're allowing God to speak to us..
Prayer is the most spiritual act a Christian could do..
Some of you here might say worship..

$^{81}$I mean, we just had a phenomenal worship experience here.
within our community in this moment..
Maybe for you, worship is the most spiritual thing.
that you think you can do..
For some of you here, it might be reading the Bible..
Maybe it's your daily devotions..
That's the place where you feel.
like you're most spiritually connected to God..
For some of you, maybe it's coming to church..
For a lot of you, I hope it's coming to church.
where you think that that is something that is spiritual..
The most spiritual thing you could do.
is gather together with people on a Sunday.
in a room like this or in a small community,.
a community group somewhere in the city.
and believe that God is gonna do things..
Maybe that is the most spiritual thing..
All of those things are absolutely spiritual.
and all of those things are incredibly valuable,.
but I want you to notice something..
Every single one of those things.
is dependent on one other thing first..
All of those things, as spiritual and as great as they are,.
are actually dependent on one other thing.
that not only defines those things,.
but drives those things forward in your life,.
and that is the power of choice..
You have to choose to do those things..
You have to choose to pray..
You have to choose to open your Bible and read the Word..
You have to choose to come to church..
Hopefully you chose to come to church..
Hopefully nobody forced you to come to church today..
Hopefully you chose to be here..
We have to choose to do those things..
And here's the thing, if we don't choose them,.
if we're forced to do them,.
then we're in slavery and not in freedom..
Are you with me?.
In fact, the very definition of slavery.

$^{121}$is that choice is taken away from you..
The very definition of freedom.
is that you have the freedom to choose..
And last week, as we unpacked the idea.
that God had finally set Israel free,.
that Pharaoh had finally allowed Israel.
to move into their freedom,.
we saw that the very first thing.
that they were challenged with was to make a choice..
They had to make the choice.
whether they were gonna move into their freedom,.
leaving Pharaoh and Egypt behind them,.
or whether they were gonna move into their freedom,.
carrying the pain and the hurt.
and the travesty of the oppression.
that had happened for them for over 200 years..
And God challenges them, doesn't He?.
He says, "Will you bless Pharaoh?.
"Will you let go of the very one.
"who has caused you so much pain.
"so that you can walk into your freedom,.
"into your future, facing forwards?".
I said last week that so often for many of us,.
when we're hold to the hurt and the pain of our past,.
it's like we're walking into our future facing backwards..
We're allowing the hurt and the pain of our past.
to be the primary thing that defines our present.
and moves us forward into our future..
And we saw last week that there was a tool.
that God gives us to release the pain of our past.
so that we can move into our future,.
not looking back to the past,.
but embracing the future..
And that is the power of choice,.
the choice to forgive,.
the choice to open our hearts.
and forgive and to bless the ones who have hurt us.
and done these things to us.
so that we might be free,.
free from the pain, free from the hurt,.

$^{161}$free from anything that would try to hold us back.
from the good future that we have..
You see, your ability to choose,.
to choose the future over your past,.
to choose faith over fear.
will be the primary defining ground.
of whether you will experience great freedom in your life.
or whether you will remain in slavery..
The most spiritual thing you can do is to choose God,.
to choose to worship, to choose to honor,.
to choose to open your heart,.
to choose to create your future.
rather than be consumed by your past..
And here's the amazing thing..
Pretty much every moment,.
every story of great faith in the Bible.
is a story and a moment that emerges.
out of the epic intersection in life.
where God has driven his people to a point of decision..
So many of the great stories in our scriptures of faith.
where men and women show great faith,.
it's because God has put them in a place.
where they have to make a decision..
Will we go forward?.
Decide for yourself today who you will serve..
Will you serve God or will you serve something else?.
But you've got to make that decision today..
Decide to move into your future facing forwards.
or decide to carry on with the hurt.
and the past of your past..
It's a choice to have faith or to walk in fear..
And it is exactly,.
exactly what the narrative.
of the parting of the Red Sea is all about..
Your ability to choose.
is the most spiritual activity you will ever do..
And as God has freed Israel from their slavery,.
Israel is now in a new season..
Israel's now in a season.
where no longer are choices being made for them..

$^{201}$And they now have to take responsibility.
for their own lives..
They have the grand responsibility.
to walk out into their freedom with hope and with desire..
The freedom to choose whether they'll serve God or not..
And in this freedom,.
God realizes that after 200 years of slavery,.
their choice muscles,.
their spiritual activity of choosing is weak..
They understand that they've had all their choices.
stripped away from them for so long.
that now that they're free,.
it's easy for them to be overwhelmed.
with the reality of the choices that are in front of them..
And God understands that they don't have the strength.
right now to be able to make the choices.
that they need to make.
to walk in the freedom that God has for them..
And so what God does right in the early part.
of Pharaoh releasing Israel into their new freedom journey,.
what God does is he leads them into a number of places.
where he brings them to a point of decision.
because he wants to strengthen that muscle in them..
He wants them, it's almost as if he creates a training ground.
like a bootcamp for them every step of the way.
early on in their walk of freedom.
because he wants to strengthen within them their desire.
to say, we are always gonna choose God,.
we're gonna choose faith, we're gonna choose hope,.
no matter how hard something looks,.
that's what we're gonna do..
And God actually directs his people into places.
where fear is gonna be a reality for them,.
where they're going to have to decide.
are they going to respond in fear or respond in faith?.
And you need to know that in every aspect of freedom.
that God has for you,.
there's always going to be the potentiality.
for you to choose fear,.
for you to choose to return to slavery,.

$^{241}$for you to choose to hold on to that past..
And so the most spiritual activity you will ever do.
is choose faith..
And I want to show you how God brings his people.
into situations that will strengthen this muscle.
of choosing faith..
Is this okay so far?.
Everybody all right?.
Exodus 13, verse 17..
When Pharaoh let the people go,.
God did not lead them, notice this,.
God did not lead them on the road.
through the Philistine country,.
even though that was shorter..
I love this..
God was like, okay, you're free now..
And as the crow flies, you should go on this path.
'cause it'll get you to the fullness.
of your freedom quickest,.
but I'm not gonna lead you down that path..
He doesn't choose to take them.
to the place that was shorter..
For God said, if they face war,.
they might change their minds and return to Egypt..
So God led the people around by the desert road.
towards the Red Sea..
The Israelites went up out of Egypt, armed for the battle..
I want you to see what God's doing here..
He understands that the Israelites actually need.
to strengthen their choice muscles..
That is very weak right now..
And so rather than take them in the direct line,.
the shortest route, the easiest route.
that would take them through the Philistine country,.
He actually then leads them on the desert road,.
a circular route that's actually,.
and we'll see this in a moment,.
is gonna lead them to a place where they get trapped..
Lead them to a place where they're gonna have.
the Red Sea in front of them,.

$^{281}$and behind them there are gonna be mountains.
and a flat plain, and on those mountains,.
they're gonna see the Egyptian armies.
rushing after them to kill them,.
and they're gonna feel trapped,.
and they're gonna feel fear..
And before any of that happens,.
scriptures tell us that God leads them to that point..
He leads them to that point 'cause He knows it's there.
that they're going to be able to make a choice..
And that choice is gonna so strengthen them.
for all the journey and all the battle that is ahead..
He actually says that if they face war,.
there might change their minds..
Their minds are weak right now..
And if they face war, they might change their minds.
and go back to Egypt..
I need to bring them to a place.
where they're gonna be scared,.
a place where they're gonna face fear,.
a place where they think they're going to die.
'cause I'm gonna lay before them a choice,.
and if they can choose faith over fear in that moment,.
it's gonna so strengthen them.
that they're gonna be able to walk strongly.
towards Mount Sinai, strongly to the promised land,.
and they're gonna be able to do all the things.
that I have for them to do in the future..
Are you following this?.
This is really important..
Because God is committed to you being free.
more than he's committed to you being comfortable..
Oh, nobody wants me to preach on that one..
God is committed to you being free.
more than he's committed to you being comfortable..
And he recognizes that sometimes.
in order to get you fully free,.
he needs to strengthen your decision muscles,.
your choice muscles,.
and that means you're gonna be put into a place.

$^{321}$where you're gonna have to make a decision yourself.
whether you're gonna respond in fear or respond in faith..
And God moves us in life sometimes.
towards things like a Red Sea.
in order to help us to understand.
that before we can get to the promised land,.
we have to actually face this thing right in front of us..
See, this is the thing I love about God..
God has all of your life in his mind,.
all of your life in his hands..
He sees your future,.
and he sees the battles that are ahead of you.
on the path that you are yet to walk on..
He sees all that..
And so right here and right now,.
he is preparing you, shaping you,.
strengthening you, helping you.
to be prepared as best you can.
for those battles that are ahead..
And if he doesn't do it now,.
you're not gonna have the fortitude,.
the hope, and the courage, and the bravery.
to really stand before the battle that's ahead..
And the battle ahead is probably gonna be more challenging.
than the one you're currently facing..
And if you're anything like me,.
the challenge that's right ahead of you now.
seems totally the biggest thing you're ever gonna face..
And God's like, "You want to go ahead.
"into the promised land,.
"but you're not even ready to face the Red Sea.".
It's so interesting..
They want their freedom,.
but they're not quite ready for the reality.
of what they need to strengthen..
They need to strengthen their faith..
Some of you here right now,.
you're facing an obstacle in your life..
And what you think is the obstacle.
that God is absent from.

$^{361}$is actually the very thing he's brought you to.
so that you would have the strength of resilience,.
and fortitude, and courage, and faith.
to face the other battle down the road.
that you're not even yet aware of..
'Cause he cares for your freedom.
more than whether you're comfortable,.
whether you're right, whether you feel good..
And he wants all those things for you too,.
don't hear me wrong..
But the primary thing he wants is you to be so strengthened.
that you will be able to experience.
all the freedom he has for you..
Are you following me?.
Now, notice here what happens with Israel..
I love this bit..
It says at the end here,.
"The Israelites went up out of Egypt armed for battle.".
I love this..
You see, Egypt is dressed for war..
They're dressed for battle,.
but they're not ready for battle..
God doesn't see them as being ready for battle..
They might be dressed in all the right things..
They might have the armor on..
They might have the sword in their hands,.
but God leads them away from the war.
because he knows that if they go there,.
they're gonna get defeated..
Their minds are not strong enough yet..
And so although they're dressed for the battle,.
God knows they're not ready for the battle..
And because they're not ready for the battle,.
he's gonna actually move them into a place.
where they can actually get prepared..
You see, they think that what they need is the armor.
and the sword for the battle that's ahead of them..
Actually, the battle that they really need to fight.
is the battle of choice..
It's actually a battle within them..

$^{401}$Are they going to choose faith.
or are they gonna choose fear?.
That's the true battle before them..
No amount of armor and no amount of physical sword.
is ever gonna help them to wield the one weapon,.
the one sword that will truly set them free,.
and that's their faith in God..
Some of you here right now.
are facing some huge battles in your life,.
massive obstacles that you're trying to overcome..
And I wanna say this..
You're facing a massive battle,.
but you're dressed for the wrong war..
You're currently dressed for the wrong battle..
You're dressed for a battle.
that God doesn't want you to fight.
and not one that he's asking you to be involved in..
Some of us in here, we're dressed.
with all the right arguments to prove to that colleague.
or that spouse or that friend or that family member.
that they are wrong and you are right..
Some of you in here are dressed with all the right skills.
to overcome whatever adversity you think it is.
that you're facing..
You're dressed with all the charm that you think you need.
to convince that person that you're dating.
to marry you for the rest of your life..
And you're dressed for the wrong battle.
because your battle is not external to you..
It is not outside of you..
Your battle ultimately is in your heart..
Your battle is what's inside here..
It's the decision-making powerhouse that you have..
Your battle is whether you're gonna respond in faith.
and trust God with your life,.
not whether you have all the right arguments.
to defeat the person that's against you..
It's whether you really trust that God has you,.
has you in His hands, wants the best for you,.
whether you can trust that no matter how difficult.

$^{441}$that thing is at work, God has got you, He protects you,.
He will help you, He will walk with you,.
and He'll fight battles for you..
That's the battle for you..
And the one that He wants you dressed for.
is not the external one with swords,.
but the internal one with humility and courage.
and bravery and faith..
That's the battle..
That's the battle that was ahead for Israel..
That's the battle that's ahead for you..
Will you choose Him when everything around you.
would tell you you're an idiot if you do it?.
This is exactly what the parting of the Red Sea.
is all about..
This is what is on God's heart for His people..
And I want you to understand the movement.
of the human heart that God is trying to do.
through this incredible miracle..
And to help you with that, I wanna take you back to Egypt,.
and I wanna take you to the actual place.
where we believe the miracle happened..
Let's have a look at this..
- Israel's journey had begun..
Finally released by Pharaoh,.
God leads Israel by the desert road towards the Red Sea..
You can imagine a mixture of excitement and trepidation.
as they begin those first few steps..
God has performed a series of miracles.
to get Israel to this point,.
and their faith and expectancy in God.
would have been very high..
And yet they were also launching out into the unknown.
with many uncertainties facing them..
Like the mountains that now rose up around them.
and the desert valleys that now surrounded them,.
they're no doubt expected to encounter great highs.
and great lows in their journey with God.
as He has released them from their captivity..
They just didn't know that those highs and lows.

$^{481}$would happen so quickly..
The Bible tells us that some 600,000 men and women,.
aside from children, set out for the promised land..
In fact, biblical scholars would say.
that the total number of God's people.
at the start of the Exodus was around about a million..
Could you imagine a million people walking through valleys.
just like the one behind me?.
And they were heading in the direction of a place.
that in the Hebrew was called Yom Suf,.
which literally translates as the Sea of the Reeds..
And in what is perhaps one of the most famous.
misinterpretations in all of history,.
the writers of the first Greek Old Testament in Alexandria.
in about the third century BC.
translated that phrase as the Red Sea,.
the name that we now commonly refer to it..
No one knows exactly where the crossing actually took place,.
but a number of sites have been suggested.
based on the closest biblical analysis of the text.
and the corresponding Egyptian geographical landmarks..
We know from Exodus 14 that the Israelites.
had just come through a system.
of narrow mountains and valleys.
that then had opened up into a large plain on the shore.
just before a big sea..
We also know that the crossing point itself.
had to be shallow enough for the Israelites.
to be able to walk through once God had parted the waters.
and smooth enough for the Egyptian chariots.
to attempt to follow..
All of which is why I've come here.
to a place called Neweber Beach on the Sinai Peninsula..
The mountains and valleys behind me here,.
this massive flat plain all around me,.
and the relatively shallow waters of the sea.
just over there all point to this as a possible location.
for the greatest miracle in our Exodus story itself,.
the parting of the Red Sea..
But did it really happen?.

$^{521}$I mean, is the parting of the Red Sea just mythology?.
Or is there any actual archeological evidence.
that it took place?.
Well, as I've been saying throughout this series,.
there's very little archeological evidence in Egypt.
of the actual Exodus events..
But the parting of the Red Sea.
is one miracle that keeps on giving.
because it is one of the very few things.
in the Exodus story that actually is recorded for us.
in Egyptian history..
And to show us that,.
I now need to take us back across the Sinai Peninsula.
to a place called Ismailia on the port of the Suez Canal.
and to an archeological evidence.
that is unlike any other in this land..
(gentle music).
This is the El Irish stone..
It's a slab of black granite that weighs about two tons.
and was discovered on a farm in El Irish in 1887..
Now, when it was actually found,.
it was lying on its side being used.
as a water trough for cattle,.
but it's actually a family shrine..
It dates to about 360 years before Christ..
But what's really fascinating for us.
is that it's covered in thousands of inscriptions..
And these inscriptions actually tell a story.
that's about 1200 years earlier.
than when the shrine was made..
Now, let me show you one particular inscription..
It's found here at the back..
Right here, what you'll see is three waves.
and two daggers or swords, three waves and two swords..
Now, Egyptologists are somewhat confused.
as to the best way to translate this,.
but a literal way seems the smartest..
In other words, it's the cutting of the sea,.
the cutting of the sea..
Now, also included here is an Egyptian name for a place..

$^{561}$It's Pehati in the Egyptian,.
but it's actually in the Hebrew, Pi-Ha-Choreath..
And that's actually the very place.
where Israel camped before God parted the Red Sea..
So what really fascinates me.
is that this is so rare in all of antiquities.
because this is the cutting of the sea.
shown from an Egyptian perspective..
The only thing we have in existence.
where we have a commentary on the Exodus.
from the perspective of Pharaoh,.
which actually makes this stone a miracle of a miracle..
(gentle music).
It actually fascinates me that we have both Jewish.
and Egyptian sources for the parting of the Red Sea..
But for me, a question still remains..
What exactly was God trying to teach his people.
through this miracle?.
Well, to find an answer for that,.
I wanna take us back now to the waters of the Red Sea itself..
(water sloshing).
(gentle music).
I haven't actually tried to part these waters just yet,.
but being here and standing in this place.
is actually really emotional for me..
I mean, this may well be the very place.
that Moses stretched out his hands and the waters parted,.
and they crossed to the mountains.
that are over here on the other side..
But there's another emotion.
that was there for Israel on that day..
I mean, prior to Moses stretching out his hands,.
they felt incredibly trapped in this place..
They had just gathered here.
and they'd come up against the waters.
and they knew that the Egyptians were chasing after them..
In fact, as I stand here with the waters behind me,.
I look back to the mountains just over there,.
I can picture what it would have been like.
to have 600 Egyptian chariots coming down those mountains,.

$^{601}$coming through the valleys, chasing towards them..
They would have felt hemmed in.
and that anxiety and fear.
would have been such a critical part of that moment..
In fact, it was so strong that they come to Moses.
and they complained bitterly to him..
They say, "Why are we gonna come to die on the beach here?.
"We should have stayed in Egypt if we were gonna die.".
And so despite the natural beauty of this place,.
actually the geography here.
just communicated to them death..
What we need to remember here.
is that none of this was a surprise to God..
In fact, in Exodus 14,.
we see that God hardens Pharaoh's heart..
God hardens his heart so that he would drive.
after the Israelites, chase them to this very place..
I mean, God wanted Israel to be here, to feel trapped,.
to feel that fear inside them,.
to know that it would be impossible for them to go forward.
unless God stepped in and did a miracle..
And he wanted this because his desire.
was to expose the content of their hearts..
He was kind of leaning into his people.
and saying, "What's really actually inside of you?".
And this teaches us a critical thing here.
in the Exodus story..
You see, the Exodus is not actually so much.
about a physical movement from one place to the next,.
but a different kind of movement,.
the movement that actually happens in the human heart..
- When I woke up this morning and got dressed,.
I didn't realize I put the same pants on.
that I was wearing in the film on that day..
The Lord works in mysterious ways..
I wanna show you this movement of the human heart.
because it's important for you..
It's important for your own movement.
in your story of Exodus and the freedom.
that God longs for you..

$^{641}$And we see that movement happen in the physical movement.
of God's people in the story..
So I wanna just unpack that briefly with you..
Verse 10 of Exodus 14 says this,.
"As Pharaoh approached, the Israelites looked up.
and there were the Egyptians marching after them.".
So again, the story is you got the mountains behind them,.
that big flat plain you saw in the film,.
the waters that you saw in the film,.
they're up against the waters,.
they're looking back to the mountains.
and they see the Egyptians marching after them..
Notice what happens, "They were terrified.
and cried out to the Lord..
They said to Moses, 'Was it because there were no graves.
in Egypt that you brought us to this desert to die?.
What have you done to us by bringing us out of Egypt?.
Didn't we say to you in Egypt, 'Leave us alone,.
let us serve the Egyptians?'.
It would have been better for us to serve the Egyptians.
than to die in the desert.".
This is a classic response.
and you can understand why Israel responds this way..
'Cause they think they've come to a dead end..
And they've got the sea behind them,.
they've got the mountains in front of them,.
they've got the Egyptian armies,.
could you imagine what it'd be like?.
600 chariots rushing down those valleys,.
rushing towards them and all of God's people.
are crying out to Moses and they're saying,.
"This was not what we had in mind..
This is not what we thought freedom was like..
We thought freedom was gonna be perfect,.
freedom was gonna be great..
Oh, we wanted the promised land,.
but we weren't prepared for this thing right here..
Why did you bring us out of our slavery.
to bring us to a point where we die in the desert?.
This doesn't make sense.".

$^{681}$Which I think is so easy for all of us to say.
when we're in a journey of freedom,.
when we're moving from one thing to the next.
and when we come up against an obstacle.
as big as the Red Sea.
and we come up with something that seems like.
we can't get around it, we can't get through it,.
we don't know what to do with it,.
we think this is the end for us..
And what they thought Israel was the end for them.
was actually the very thing.
that was gonna provide what they needed.
for the journey ahead, as I've been saying..
You see, so often in our journeys of freedom,.
we want God's outcome, but we don't want his process..
God's about to do this incredible miracle and split the sea.
and they're about to walk into their freedom.
and they have no idea right now.
and instead what they see is an obstacle.
that God sees as the process by which.
they're going to strengthen their muscle to decide faith..
But in this moment, they decide fear..
Notice actually what they say, they say quite clearly..
They say, "It would have been better for us.
"to serve the Egyptians than to die in the desert.".
In other words, they've already made a choice..
The first obstacle they come up against in their freedom,.
they make a choice and that choice is to go backwards..
That choice is to move back towards their slavery..
They're like, "We would rather have slavery.
"than this situation we're in right now..
"We're up against this water, we're up against this trap..
"We feel like we're trapped in here..
"We don't know what to do.".
The choice we're going to make is we're gonna go backwards.
because at least there's some comfort in slavery..
We don't know what's ahead of us..
This is really important, church,.
'cause when we're moving in the choice of faith,.
faith means you don't have certainty..

$^{721}$Come on, church, I'm gonna preach on this one..
Faith means you don't have certainty..
You don't know what's gonna happen..
You don't know what's gonna happen with that sea..
You don't know if it's gonna part..
You don't know if it's gonna change..
You have uncertainty..
That's why you have to choose faith..
If you want certainty, go back to slavery.
'cause there you know what it's like..
You know what the oppression is..
You know how bad it is..
And that's what Israel's saying..
They're saying, "I don't want the uncertainty.
"of an unknown future..
"I would rather take the safety and the comfort.
"of a bad situation even if it's terrible for me.".
Let me preach on this for a second..
Some of you wonder why you're still.
in that really bad relationship you're in..
It's because you're so afraid of being lonely.
in your future that you are happy.
to accept a terrible relationship today..
Some of you are wondering,.
why am I still in the job that I'm in.
when God for a long time has been calling you.
to something different?.
And perhaps it's because you're so afraid.
of the uncertainty of your future,.
whether you're gonna be able to provide.
for your family or not,.
but you'd rather stay in the comfort.
of a job that's killing you..
Come on, church..
And so here's Israel, and they're in a place of decision..
God has put them in that place, ordained it for them,.
and it's not comfortable..
It's not easy..
And they are deciding out of fear to go backwards..
That's the choice that they're making..

$^{761}$They wanna go backwards..
And so that's what they're saying..
They're saying, "We'd rather take this than that.".
I love one of my favorite pastors,.
Erwin McManus, a church called Mosaic in LA..
He said it this way, and I love this..
He says, "We so often would rather accept the comfort.
"and safety of a past that holds us captive.
"than the mystery and uncertainty.
"of a future that sets us free.".
That, my friends, is what the Exodus is all about..
That's the thing that we have to try to break.
through the muscle that we have to choose faith over fear,.
to choose the uncertain and the unknown,.
the belief that God will be there,.
that He will provide, that we'll lean into that..
We gotta choose that uncertain future.
over the comfort that comes from the brokenness.
of our past, and whether we will decide that.
will determine our freedom,.
and that's why God's brought them here..
Now, notice what Moses now responds with..
I love this in verse 15..
Moses answers the people, "Do not be afraid..
"Stand firm, and you will see the deliverance.
"that the Lord will bring you today.".
The Egyptians you see today will never be seen again..
The Lord will fight for you..
You only need to be still..
The Hebrew word be still here means literally.
to root yourself in, to dig yourself in..
Now, this is really interesting.
because when Scripture uses this phrase be still,.
we often think it's because we have too much activity,.
too much desire, too much things going on..
We use this a lot in Hong Kong..
One of the great calls for Hong Kong is to be still.
and know that God is God,.
because in Hong Kong we're busy, we're active,.
we're always doing stuff,.

$^{801}$and sometimes God needs to come and say,.
"Just chill out, rest, get to know me more.".
That's not what's going on here,.
because Israel's not all bustling with activity,.
wanting to drive ahead, and God's gotta go,.
"Slow down, otherwise you're gonna get ahead of me.".
That's not what's happening here..
What's happening is that Israel wants to go backwards..
They wanna go back to the pain of their past.
rather than move into the future,.
and because they're choosing to move backwards,.
Moses says, "Be still..
"Put your, dip yourself in..
"Create a resolve in you..
"Don't go backwards..
"Don't go back..
"I know it's scary, I know it's hard,.
"but don't be afraid..
"God's gonna fight this battle for us..
"You're gonna see God about to do something.
"that you never experienced..
"All you have to do right now is choose to not move..
"Choose to stay where you are,.
"even though you feel trapped,.
"even though you feel fear,.
"even though you think nothing is gonna happen..
"Some of you just need to be still..
"Every time that you wanna retreat, be still..
"Every time you wanna give up, be still..
"Every time that you're allowing your slavery.
"to be more comfortable to you than your future,.
"which seems uncertain, be still..
"Do not move backwards.".
Are you with me?.
You root yourself in, you say,.
"This is where I'm gonna be,.
"even though it's hard, even though it's difficult..
"The first choice I make is I stop my heart.
"from wanting to be tempted to go back that way..
"I will be still..

$^{841}$"I will be still.".
Some of you are on a trajectory right now.
that God doesn't have for you..
And he's just saying, "Just stop..
"Just take a moment to be still,.
"and you're gonna make much better.
"spiritual decisions in your life,.
"and practical ones, emotional ones,.
"financial ones, whatever they might be..
"You're gonna make much better decisions in your life.
"if you stop the panic of going backwards..
"Just be still.".
Are you with me?.
Now, here's the funny thing,.
'cause then God shows up and contradicts.
everything Moses has just done..
I love this..
Who knows that God sometimes shows up.
and contradicts everything you've just done?.
Okay, just me, all right..
Then the Lord said to Moses,.
"Why are you crying out to me?.
"Tell the Israelites to move on..
"Reach out your staff and stretch out your hand.
"over the sea to divide the water.
"so that the Israelites can go through the sea.
"to the other side on dry ground.".
I love this..
So here's Moses going, "Be still.".
And then God shows up and goes, "Move on.".
And Israel's probably like, "What do I do?.
"Do I be still or move on?.
"Be still or move on?.
"What am I supposed to do?".
And it feels like it's contradictory, but it's not..
Because it was absolutely right for Moses.
to arrest the backwards movement of the choice.
of the heart of his people by telling them to be still..
But then God shows up and says,.
"It's right for you to be still for a time,.

$^{881}$"but inertia is not your permanent destiny.".
Inertia is right for a time,.
but it is not where you are to make camp..
You are to move on..
You are to be moving forward..
You are to be moving into the future that I've got for you..
So yes, you had to stop..
Yes, you had to dig in..
Notice, by the way, the word for move on.
here in the Hebrew, it means to uproot your tent pegs..
It's almost like God is saying to Israel,.
"You gotta be ready..
"You had to dig in for a while,.
"but now I want you to uproot that.
"because I'm about to move..
"I'm about to do something and you gotta get ready to go..
"You gotta be the one who's moving forward..
"You gotta be the one who's believing.
"that you're gonna take faith over fear..
"You're gonna be the one who's gonna say no.
"to the consumption of your past.
"and say yes to creating your future..
"That's gotta be in your heart..
"So you need to undo those tent pegs.
"and you need to get yourself ready to move on,.
"to move forward.".
It's like God is saying,.
"Take the handbrake off your life.
"because there's a green light in front of you.".
Anyone here ever been in a car at a traffic light.
and the traffic light goes from red to green.
and the car in front of you doesn't move?.
Anyone been in that situation?.
(humming).
Here's the biggest truth I'll preach all day..
The amount of time that it takes you to honk your horn.
shows how Christian you are..
(audience laughing).
Preach it, preach it..
'Cause if you're Jesus, boy, you wait..

$^{921}$Oh man, they must be texting somebody..
Must be a really important text..
I'm just gonna wait, I'm gonna be patient,.
I'm gonna let them do their thing..
They'll move when they're ready..
Bless them, they must be having a really hard day..
Just Jesus, just, I get an opportunity to pray for them..
They're right in front of me right now..
Lord Jesus, I wanna bless that car, bless that family..
If you're anything like me, you're not like Jesus..
If you're anything like me, you're on that horn.
as quickly as you possibly can because it's green.
and the people are not going..
God in this moment is not very Christian.
because it's green but the people are not moving..
And so he shows up and says, honk, courtesy honk..
I'll give you a real honk later,.
but courtesy honk right now..
It's time to move on, it's time to get going..
Don't dig yourself in anymore..
Uproot your tent pegs, get ready to move..
I love this because this is so important, church..
God is inviting Israel to move forward into their future.
out of their choice of faith in him,.
not in a response to fear..
Church, this is so important..
Fear does not just drive you backwards..
Fear can also drive you forwards in life..
Fear can also be a very motivating factor.
for us to move forward in life..
Fear can even motivate you to do things of God..
But let me tell you, anytime you do something of God.
out of a place of fear, you are not going to be.
reaping the benefits that God has for you..
We can't serve God from a place of fear..
We have to serve God from a place of faith..
So it is not Pharaoh who is chasing them..
It is God who is inviting them..
God is saying, move forward in faith..
Don't move forward into that ocean.

$^{961}$because you're afraid of the people behind you..
Move forward into that ocean.
because you've made a decision,.
a choice in your heart today to follow me..
You have made a decision to say, you will provide,.
you will split the sea, you'll do something..
I don't know how you're gonna do it,.
but I'm gonna move forward in faith.
because I'm choosing faith over fear..
I'm gonna have the thing that moves me forward in life.
to be faith and not fear..
Notice that God asks Israel to do this.
before he splits the sea..
Oh, it would have been easy to walk forward.
once the sea was split..
You can imagine, like, God does this amazing miracle, right?.
And they're being chased by these Egyptians,.
and then God says, move on..
They'll be like, yep, yep, yep, yep, yep..
And like, off they would go, like, straight away, right?.
They would run away..
They'd be like, yeah, I'm going, man, I'm going..
God's like, I haven't even done anything in the ocean yet,.
but you need to move on..
You need to put your feet,.
get your feet wet before I do anything..
You need to make a decision in your heart..
Remember, this is all about God strengthening.
the heart muscles, the decision muscles,.
the choice muscles to take faith over fear..
He's like, go..
And yes, Moses is gonna stretch out his hand..
He's gonna stretch out that staff,.
and he's gonna touch that water,.
and it's gonna do this crazy, amazing miracle,.
and they're gonna go through,.
and then the Egyptians are gonna chase,.
and the water's gonna collapse on them..
All of that is about to happen..
And God says to Moses, stretch out your hand.

$^{1001}$and see what's about to happen..
There's a partnership that God's inviting his people into..
First of all, to have the decision in their hearts.
to move forward in faith,.
and then to back that decision with action..
Stretch out your staff and watch the obstacles part..
Do you wanna know today who Moses' staff is in this world?.
It's the church..
The church is God's staff that he stretches out.
before all of the obstacles.
that are stopping people from coming to faith..
And he's saying, watch the obstacles part..
You, the greatest expression of your spirituality.
will always be your choices..
And if you're gonna walk into the freedom.
that God has for you, that he's paid the price for,.
that he's already released you into,.
our decision to choose, to choose that uncertain future.
when we know what the past has been like,.
to take that step of faith and really believe.
that he's gonna come through, to see him..
And we're not making blind faith choices..
We have the scriptures..
We have the spirit of God in us..
We have all the times that we come together and worship..
We have the testimony of God working in people's lives..
We have the testimony of God working in our lives..
It's not like we're walking forward blindly..
We know the character of God..
We see the character of God..
We understand the character of God..
He's spoken to us..
He's filled us..
He's done the work of the cross..
It's not in our strength that we do this, it's in him..
But we still choose, choose to move forward.
in all that God is or choose to remain inert.
or go backwards in the fear that we hold..
Some of you in this room, this is a wording season for you.
because you are currently moving backwards and you know it..

$^{1041}$You're returning to the things that you know.
you don't want to be doing..
Be still today..
Dig yourself in, give yourself some time.
before you make a decision.
that might ruin you or your family..
Some of you in this room, you wanna move forward.
but you are tempted to move forward in fear..
God would say to you, move forward in faith..
Move forward in hope..
Take the chance of the uncertainty of your future.
'cause it is there that I will meet you..
I will split the sea..
That obstacle that's in front of you, I am gonna move it.
but you're gonna have to make the choice.
to partner with me and stretch out your staff.
and believe that the waters will part..
Greatest thing you'll ever do.
is make the spiritual choice to follow him..
Have faith, vine, not fear..
Amen?.
All right, could you stand with me?.
I wanna pray for you..
I don't wonder whether you just hold your hands open.
before you if you're comfortable to do so..
I believe that this is an important word for many of us..
And just at the start of the service,.
I invited us to think about our obstacles,.
the things that are in front of us now..
As you stand before him with your hands open,.
maybe you can bring to mind once again.
that red sea of yours..
And for those of you here.
where the temptation is to move backwards,.
maybe the fear is strong and I understand that..
Maybe the obstacle seems insurmountable..
You don't know if there's going to be a way..
And everything inside of you is telling you to go back,.
that you're at a dead end..
What the Israelites thought was a dead end.

$^{1081}$was actually about to become a highway..
And what you're thinking is a dead end right now.
is a highway..
And notice that God did not take Israel around the Red Sea..
He took them through it..
And some of you, what you think is your dead end,.
what you would like God to just remove from you.
is actually the very thing that he'll walk you through..
Though we walk through the valley of the shadow of death,.
he is with me..
For some of you, the call is to be still,.
to take the courage today to say,.
"I will stop moving backwards.".
Help me, Lord..
Help me to trust you..
Help me to just be still..
Give me the courage to say no..
Give me the courage to stand against what my heart is.
trying to tell me to do,.
what the fear in me is trying to tell me to do..
Lord, I root myself, dig myself into you..
When all else seems lost,.
when all else seems like it's gonna fail,.
I root and dig myself into you, Lord..
And then for some of us,.
there's a green light where you think it's red..
And God is giving you that courtesy honk..
And he's saying, "It's time to move.".
Pull up your tent pegs..
It was right for you to be just stable for a while,.
but now it's time to move..
Take a step of faith and move forward,.
not in fear, but in faith,.
knowing that I'm with you,.
that the wind of my spirit is with you..
And together, we're gonna do amazing things..
There's Mount Sinai ahead of you..
There's my presence,.
like you've never experienced it before, ahead of you..
There's the promised land that you're gonna go and conquer.

$^{1121}$that is ahead of you..
You could only see the victories that are ahead of you,.
but it's gonna come through the strengthening of your heart,.
the strengthening of your faith..
Father, come Holy Spirit..
I pray and release faith over these people.
in the name of Jesus..
And I pray that faith would rise up in the house,.
that faith would be what we are moving forward in.
as a community of God..
Thank you, Lord, that you taught the Israelites.
a place of great faith..
Would you teach us and show us and release in us great faith.
and we pray this in Jesus' name..
Everyone says..
Amen..
Let's worship together as we respond in great faith..
\newpage



\section{}
\label{sec:JVjucZ_U4Bw}
\textbf{2023-09-25 EXODUS - 16 Bread \ and  Water [JVjucZ\_U4Bw].mp3}
\newline
\newline
連結: \href{https://youtube.com/watch?v=JVjucZ_U4Bw}{\texttt{ https://youtube.com/watch?v=JVjucZ\_U4Bw}} ~~~~ 語音日期: 2023-09-25 
\newline
\newline
\hyperref[sec:lsj62DXXjZ4]{\small{< < < PREV SERMON < < <}}
~
\hyperref[sec:index]{\small{[返主目錄]}}
~
\hyperref[sec:AG5PdzOFde8]{\small{> > > NEXT SERMON > > >}}
\newline
\newline
$^{1}$(soft music).
You guys go ahead and have a seat..
(audience applauding).
Well, good morning..
I feel like it needs to be said..
I am Pastor Promise..
It's a very odd thing to say Pastor Promise.
in reference to myself..
I normally go by just,.
have you guys ever thought about this?.
Like pastors are like, maybe doctors sometimes as well,.
but like you don't like go around and say,.
hi teacher, this person or hi counselor,.
this like Pastor Promise or I don't know..
It's a weird thing..
Anyways, I'm Pastor Promise, not preacher promise.
as Pastor Carla might get you to believe..
And I'm super glad that I can be here.
with you guys this morning..
And carry us another step forward in our Exodus journey..
Just quick show of hands, if you were here last Sunday..
Yeah, great..
What did Pastor Andrew talk about?.
What was the theme?.
We were here, but were we here?.
All right, good, now we know..
Last week was the Red Sea..
Last week was incredible..
It was one of the coolest miracles..
One of the best sermons that I've ever heard on the Red Sea..
The Red Sea is a fascinating concept, right?.
Like I could never wrap my head around the reality.
that God parted the sea..
Like let's just not overlook that for a minute..
The fact that a sea went, it's crazy..
But Pastor Andrew took us to that very place.
in the Exodus journey..
And he showed us how we have to have the ability.
and the desire and maybe the step of faith.
to choose to follow God, even before he parts the seas,.

$^{41}$but he still parted the seas..
Now I think really interestingly,.
I have a weird way of processing things..
So as that sermon was going through,.
I was thinking about this movie, this show..
I think it's a movie..
Has anyone here seen "The Prince of Egypt"?.
Hey, all right, cool, we all know the song..
♪ There can be miracles ♪.
- Ah, just hits the soul, right?.
Rest in peace, Whitney Houston..
Anyways, so in that movie, if you've not seen it before,.
it's DreamWorks' rendition of the Exodus story..
And it takes you through all the plagues.
and Moses growing up in Egypt,.
and they take them to the Red Sea..
And even though it's animated,.
it's still like massive and awesome and cool..
And it gets me thinking around the idea.
of what would it have been like for the people of Israel.
when the sea literally split?.
Like, let's just zoom in on that for a second..
What would that have been like to watch the sea go crazy?.
And then to like walk across it..
That's like a crazy miracle, right?.
Like I often wonder, like,.
were they walking on the dry land?.
And then like, you know,.
there's like a giant whale that comes by,.
but because there's a wall of water, right?.
Like, you know, you see the, like in the movie,.
you see the reflection of the whale,.
but it doesn't actually reach you..
Or if you were a fish and it was Red Sea crossing day,.
and you wake up that morning.
and you're going to your buddy Bob's house.
or Nemo's house, more appropriate,.
and you're trying to swim over..
And all of a sudden you get to this part.
and you're like, "Whoa, humans..

$^{81}$"What am I seeing?.
"A bunch of dudes walking and ladies walking.
"and carrying things.".
I'm curious if they would have been like,.
what is happening?.
Regardless, I think just, I know it's silly,.
but like the Red Sea splitting is a crazy concept..
It's like a miracle..
It's like the miracle of miracles..
If you were to list out like the best miracles,.
like a top three, the Red Sea's up there..
Like if I could go back and see any miracle.
that would have happened in the past,.
I feel like that's the one,.
I would have been like, "Sign me up,.
"get me a ticket, front row seats..
"I wanna see what really happened.".
It's a really cool thing,.
but I think what we have to take away from the Red Sea.
and these miracle concepts is that it shows us.
something that's really, really clear..
God has no limit in his capability.
to provide for his people..
Like if he's gonna part a sea, what will he not do?.
Like he's just so willing, so capable..
He has so much of an ability to provide for his people.
that he parted the sea for them..
It's really the miracle of miracles..
And I bring that up because the entire story of Exodus.
has been quite filled with miracles,.
whether it's the sea turning, the Nile turning to blood,.
or the sun going dark for a day,.
or the livestock, or the frogs, or the flies,.
or the gnats, or the boils, or the, you just name it..
It's just miracle after miracle,.
after miracle, after miracle..
It's a bunch of miracles happening..
And it makes me wonder,.
what exactly are all these miracles about?.
What do miracles actually do?.

$^{121}$And so I wrote a pretty little definition here..
It says this, miracles are the inbreaking of the divine,.
working and providing what we could not.
naturally achieve ourselves..
So just sit on that for a minute..
Miracles are the inbreaking of the divine..
So something that is done by God,.
and it's only working, or a key word here,.
providing what we could not naturally.
achieve or do on our own..
So many miracles are based around.
the idea of God's provision, right?.
Like the Red Sea, the people were,.
hey, been freed from Egypt, they're on their way,.
and then they had this giant obstacle,.
and they need someone to provide them a way over,.
or provide some direction,.
and God shows up and parts the Red Sea..
It's about his provision..
He literally provided a way.
when there was an impossible situation.
in front of his people..
He's never unable to provide..
I feel like that's what we get from that..
No matter what happens, God will provide for his people..
It's who he is, it's what he does,.
it's part of his name, it's just who God is..
But the thing is this,.
miracles are just one of the ways that he provides..
It might be the most recognizable..
Like when we think of a miracle,.
we think of the supernatural, we think of the big thing,.
it might be the most recognizable way,.
but it's only one way that God.
actually provides for his people..
God provides all the time for his people..
And so if miracles are just one way,.
that must mean that there are other ways.
in which God provides..
And I think it's very important that we today.

$^{161}$wrap our heads around what are some of the other ways.
that God provides for his people.
beyond just the obvious supernatural miracle..
Is there another way that God chooses.
to emphatically declare to the world.
that I am God, that I provide for my people,.
and to declare to us that I provide for you..
Are there other ways in which that happens?.
And if there are, we've got to be able to recognize it..
So that's the key..
If there are other ways in which God.
does provide for you and for me in our journey,.
then we've got to be able to recognize it..
But if we don't realize that it's there,.
we might miss it, only worshiping the God.
of the supernatural..
So as we continue our Exodus story today,.
that's what we're gonna focus on..
We're gonna be looking at the challenges.
that the people faced when they departed from the Red Sea.
and in that journey..
And it's important that we really pay attention.
and grab hold of these truths,.
because similar to how God shows us.
that there's multiple ways that he provided.
for the people of Israel, beyond just the miracle,.
God also has multiple ways in which he's gonna provide.
for you and me as we too go through our Exodus,.
as we walk away from slavery, are set free,.
and walk into his promise, into his freedom..
He has multiple ways of providing,.
and we've got to be able to recognize that as well..
So when we look at Exodus chapter 15 through chapter 17,.
we are given a series of stories..
We're only gonna focus on two of them..
Two stories, two moments, two locations.
in which God provides in a way that is different.
from the big miracle..
So for a way to frame these two accounts,.
and in order to best provide you a great introduction.

$^{201}$into what these stories are all about,.
let's take a look at the film today..
- Israel have just witnessed.
one of the most incredible miracles recorded.
for us in scripture, with the parting of the Red Sea,.
and the subsequent removal of the threat against them.
by Egypt's army..
But as is so often the case in our journey to freedom,.
dramatic moments of God's power are often followed.
by the seemingly difficult normality of daily life..
Israel have a long journey ahead,.
and their days are now consumed.
with the tough and tedious work.
of walking through the desert of Shur,.
one of the most hostile lands on earth..
They find themselves stretched daily.
to their physical and emotional limits,.
and their water and their food supplies quickly run out..
This is not the sort of place you would wanna be.
without food and water..
Across this series, I've spent hours.
walking through deserts like this.
to get a sense of what it would have been like for Israel.
in those first few weeks.
as they journeyed towards the promised land..
And I can tell you, it is unbearably hot..
It is sort of stark and barren..
There's a wind blowing right now.
that's putting sand in my eyes all the time..
You might be able to hear.
some of the wild dogs in this area..
I mean, this is not the place.
you wanna spend a lot of time in..
I mean, for me, I can escape easy..
I can go over there to the van that's air conditioned.
and get out of the heat..
But the Israelites, this was their life,.
week in and week out..
They were carrying all of their possessions.
on their shoulders..

$^{241}$They were walking daily, young and old alike..
I mean, every day would have felt like a moment.
of a fight for survival..
You know, the parting of the Red Sea.
in this kind of environment.
would have felt a long time ago indeed..
(dramatic music).
(water splashing).
the Red Sea. (dramatic music).
(water splashing).
(dramatic music).
But God is about to change Israel's thinking.
when it comes to their understanding.
of His care and provision for them..
And it all starts with something as simple as this,.
a well dug here at Elam,.
right in the heart of the desert itself..
You see, Exodus chapter 15 gives us the story.
of Israel being desperately in need of water..
And they suddenly come across this oasis,.
some 12 wells and 70 palm trees..
And to be honest, being here,.
I can really sense how overjoyed Israel would have been.
to finally find the relief of this natural water source..
I mean, it's a natural spring at the right time.
and in the right place..
All of which teaches us something quite critical.
about God's provision in all of our journeys.
from slavery to freedom..
You see, the greatest miracles of God.
are often found in the most natural of circumstances..
Yes, He is the God who parts the Red Seas,.
but He's also the God who comes.
and holds all of the world's natural processes in His hands..
I mean, the rain that falls and collects in pools like this,.
the air that we breathe every day,.
all of them are provisions of God's blessings..
You see, the provision of God.
is quite literally everywhere..
And the question that God was asking Israel.

$^{281}$in this moment was this, do you actually see it?.
Sometimes the provisions of God are far closer to us.
than we ever would have expected..
Soon after, Israel has been provided the water.
at the wells of Elam..
They then come here and are attacked by the Amalekites.
right in the Valley of Rephidim..
Now, Moses scrambles up to the side of the hill here.
and he looks down on the valley..
And the Bible tells us that.
if he holds the staff of God in the air,.
well then the Israelites are having victory..
But if his arms drop, the Amalekites have the victory..
Well, for Moses, you can imagine the pressure.
this put him under..
I mean, I've just tried to do this myself..
If I hold my arms in the air,.
I can probably do it for about three minutes.
before they start to ache and they get a bit heavy..
Well, the same thing happens for Moses,.
his arms get tired..
And so he calls on two of his friends, Aaron and Hur,.
who come alongside of him, hold his arms up.
and keep his arms high until sunset..
And a victory comes for the Israelites..
Now, you've probably heard that story before,.
but I wonder whether you've ever stopped.
to understand how really strange it is..
I mean, think of it this way..
The Bible tells us that actually the Israelites.
were highly outnumbered by the Amalekites..
This was not a fair fight..
And God gives them a miracle in their victory..
And I've always wondered,.
why didn't God extend that miracle.
just a little bit more to Moses' arms?.
Why didn't he give him a supernatural ability.
to hold his arms in the air?.
Well, God doesn't..
And instead, he has to call on his friends.

$^{321}$to help him achieve the very thing.
that God had called him to do..
Without Aaron and Hur by his side,.
perhaps the victory would never have occurred..
And I think all of this tells us something.
really interesting about God's provision.
in any of our journeys in Exodus,.
that so often His greatest power is seen.
when actually He brings the community around us..
It's when we're working together as a team.
that we can achieve things far greater.
than we ever would have been able to on our own..
In these early moments of the Exodus,.
we see God's people learning to put their trust.
in God's provision,.
even at times when they don't appreciate it..
Whether that provision comes to them.
through supernatural, miraculous ways,.
or through the natural processes of the creative world,.
or even in the community that God's placed around them..
And I wanna say this to you about your own Exodus journey..
I want you to know that God is at work.
always providing for your freedom..
And my prayer is that He would open your eyes today.
so you can see it..
- And so the Elam story takes place.
off the back of one of the greatest miracles.
we've talked about..
Let me set the scene for us a little bit..
The people have experienced the sea parting.
and they're going to be walking through the desert..
You get the first, I think in Exodus 15,.
you have the first ever worship song that's recorded..
'Cause they've just seen God at work.
in such a crazy way providing for them..
They sing a song praising God for rescuing them,.
freeing them, destroying the Egyptians,.
and just liberating them into freedom..
And just not shortly after that,.
they're taken off of that mountaintop place..

$^{361}$And they're brought into a reality.
that I feel like most of us are regularly sitting in..
It's the reality of the everyday life..
So they're no longer at the place.
where it's like big miracle, whabam,.
God is for us, God is here..
Now they're having to do the nothing fancy,.
the nothing big, the ordinary, the pretty simple stuff..
They've been liberated and they've already had the time.
of being able to focus on the hyper,.
hyper spiritual, supernatural stuff..
They've seen it and now God's put them.
in the boring, everyday, ordinary, no fancy miracle time..
And their job is this, they are migrating out of Egypt..
So if you're migrating, they've got maybe some backpacks,.
some donkeys, some horses, some carts,.
they've got all their stuff,.
and they're going to go into the desert..
So they're carrying things, walking,.
they're tired, they're hungry, they're annoyed,.
they're with people so they're definitely annoyed,.
they're frustrated, they're sweaty..
It's daily life, it's what we normally go through.
pretty much every single day of our life, right?.
Just the normal stuff..
It's not a stage that's set for the supernatural.
breaking in for liberation, it's the stage that's set.
for meeting the basic necessities of humanity,.
of substance, of bread and of water, just the basic stuff..
The text tells us that they've traveled.
for about three days now, which doesn't seem.
like a very long time..
And in those three days, their water supply.
is starting to run low..
This is important because you're in a desert,.
it's hot and you need to drink..
So the people are starting to freak out a little bit.
and they get to this area and the place,.
they don't have a name for it, but they get there.
and they're thinking, okay, maybe we can rest here.

$^{401}$and get refreshed and replenished,.
but the water that they find is bitter..
So then they can't stay there, no one's gonna stay.
and build and replenish their animals.
and take care of themselves with bitter water..
So they actually name the place, the Hebrew word.
for bitter, which is more mora..
So they leave that area and they keep going.
and they're getting tired, they're wondering,.
you know, what's next, where do we find our place of rest?.
We just wanna chill out, we just wanna break..
We've been walking in the desert for a few days..
And I think it's right there in verse 27.
that God breaks through in a pretty unpredictable,.
unnoticeable way, something that I feel like.
for most of us, we would kind of just read over.
and not really see, it says this in verse 27..
And they came to Elam, where there were 12 springs of water.
and 70 palm trees, and they encamped there.
beside the waters..
For the record, Elam sounds like a great place..
70 palm trees, it sounds like a good time..
So they see that and they're like, all right,.
here's a place..
I read that verse because there's nothing really special.
about what's just happened..
They were just walking, there was no staff,.
there was no tornado of fire, they're just going,.
they're just doing their everyday migrating stuff.
and they come across the right thing at the right time..
They find the right place..
No seas are being parted, but now they have a place.
where they can relax, where they can rest for a little bit.
before they continue their journey..
And it's a point where I feel like it's a bit awkward.
because it's not a clear picture of God's provision, right?.
Like it doesn't say, and then God provided wells.
or God provided 70 palm trees to just grow overnight..
Like they just showed up and there's wells.
and water to drink..

$^{441}$Is that a clear picture of God's provision?.
But I feel like it's pretty intentional..
Think about where this falls..
Right after they've had like maybe the clearest picture.
of God's provision, Spartan Red Sea,.
the whole Nemo thing that I talked about, right?.
You have that clear picture and now you have this.
where I just found some wells and some palm trees..
What do I do with that?.
I feel like it occurs intentionally in scripture.
right at that time on purpose for us to understand.
and see that God is working..
God is providing, he's leading and he's guiding.
even in the moments that are not so obvious..
That placing of that..
Now I wonder if we would consider our Exodus journey.
and where we're going with everything..
I feel like for us a lot of days are anticlimactic, right?.
Not too much is happening all the time..
It's pretty basic, pretty mundane, pretty ordinary,.
just living life, waking up, getting on the train,.
going to work, being annoyed at my friends,.
drinking four coffees..
It's just part of the regular average day..
And it can be a bit challenging for us to see God in that..
Like where is God moving?.
There's no big miracle happening so is God even active?.
Is he still a part of this journey.
or am I just doing my everyday stuff by myself?.
I think that we have to realize that just because.
your Red Sea moment is behind you.
does not mean the best moments are in the past..
God, the miracle worker who parts Red Seas is the same God.
who just basically leads and guides us step by step.
to where we need to go..
He's the one that brings us to our elims,.
our places of rest, the things that we need..
His provision on our Exodus journey is expressed this way.
intentionally so that we can hold on to hope.
that when we're not seeing him in the big things.

$^{481}$doesn't mean that he's not there..
This story gives us a bit of hope that.
just because the seed didn't part for me today.
doesn't mean God is inactive in my world,.
doesn't mean that God doesn't care,.
doesn't mean that God is far away..
Please don't gauge God's provision or even his presence.
with you on whether or not big miracles.
are taking place around you..
He's still with you, he's still for you,.
he's still leading you into the very freedom.
that he has promised..
So can we choose to trust that he doesn't always.
have to do the supernatural to provide for us,.
that the everyday and the ordinary.
are his provision methods too..
Perhaps the ordinary days are God's workstation.
and though it may feel empty and dry.
and like we're literally in a desert,.
can we trust that even when we don't see it he's working?.
We're just saying that even when we don't feel it.
he's working, he's working on our behalf,.
he's liberating, he's still the God that provides for us.
even in the small moments..
He's the God of the big and he's the God of the small..
Can we recognize him in those small moments?.
That's what Elam is calling us to see.
in this biblical narrative..
And so Elam provides that for us, right?.
It gives us an understanding that God is active,.
God is there in the small moments..
The Red Sea which happened right before that.
gave us an understanding that God breaks into our world.
and provides for us in miraculous ways..
We've got miraculous provision, we've got natural provision.
but there's actually a third way in which God chooses.
to provide for his people and it's based off.
of what happens in the second half of Exodus 17..
So what happens here is the people are,.
they've left Elam, they've journeyed into the Desert of Sin,.

$^{521}$they've had the experience of the manna and the Sabbath.
and all these great things and now they're moving on.
forward again, right?.
So they're continuing to go forward..
And just here in verse eight, it says it really clear..
Amalek came and fought with Israel at Rephidim..
This happens, it seems pretty dramatic.
but it does happen just as abruptly as it is written..
There is indeed a nation that, an established nation.
that starts a battle, an unprovoked aggression.
against the people of Israel who had not really.
been established as a nation yet..
You have to think about what would have their thoughts.
been in that moment..
God has provided for us in big supernatural ways,.
God has provided for us in the small.
but now there's a people trying to end our existence..
If we lose here, our journey of liberation might be over..
It's this whole Exodus thing for a waste.
if he'd bring us to the desert truly so that we would die..
I think those are the questions going through their heads..
Another question that I think they might have been asking.
themselves is, we've been slaves for hundreds of years..
How do we fight against an established people?.
Against an army that probably had weapons.
and had strategies and had different levels of hierarchy..
Actually Moses is funny, he just tells Joshua,.
go get some people and go fight Amalek..
I feel like Joshua's probably a little overwhelmed.
at that moment..
And also how do you recognize who,.
like there's thousands, hundreds of thousands of them..
How do you know who's the best fighter?.
Is it the tallest person?.
Is it the dude with the cutoff tank?.
How do you know exactly who needs to be a part.
of this journey of this battle?.
And Joshua has to go and find these people..
Who's gonna be the great military mind.
to help them move forward?.

$^{561}$Who's going to be the one that gives them the best strategy.
to help them move forward?.
How do you fight a battle when you've never fought before.
and your entire people's existence is on the table?.
It's the definition of a crisis..
And I love this, I find it interesting that God,.
his methods are interesting..
He meets his people and matches their challenge..
So what we've seen so far, at the Red Sea,.
giant sea, giant obstacle, right?.
God shows up in a giant way, right?.
For the 19th time, parting of the Red Sea,.
fish, all the things..
The next victim that they face,.
they're in the desert and it's everyday life..
It's just a regular of sustaining and finding water.
and finding a place to rest and finding a place.
to make sure that my animals are taken care of..
Just the basic everyday needs..
And how does God show up then?.
He leads them to Elam, leads them to the place.
of all the palm trees so they can just breathe and rest..
He provides for them based off of what their situation is..
He matches it..
And it wouldn't be any different.
in this coming situation, right?.
Because now the people of God are being attacked.
by a community, by a group of people under Amalek..
And God is going to provide a miracle..
However, this miracle is not gonna be a miracle.
that is able to be sustained in isolation..
This is a miracle that's gonna require the community..
They're being attacked by a community.
and God is going to work in a way.
that where their community is gonna be necessary.
to sustain and to actually experience.
the provision of the miracle..
God would signify the importance of what a community is,.
how it operates, and why it's so vital.
to every single one of our existence..

$^{601}$The only way that they will move forward.
and that they will experience this miracle.
is through a functioning community..
So let's read what the passage has to say here in verse 10..
It reads like this..
So Joshua did as Moses told him and he fought Amalek..
While Moses, Aaron, and Hur went up to the top of the hill,.
then whenever Moses held up his hand, Israel prevailed..
Whenever he let down his hand, Amalek prevailed..
But Moses' hands grew heavy, so they took a stone.
and they put it under him and he sat on it..
While Aaron and Hur, one on each side, supported his hands..
Thus his hands remained steady until the sun set..
And Joshua overwhelmed the people of Amalek with the sword..
And they have their victory..
There are a few things I really need us to lean into.
and pull out of this passage for our understanding..
The first one is this, is sometimes God chooses.
to require the participation.
that's only found within our community..
Hear that again..
Sometimes God is going to choose to require.
the participation that can only be found.
within the community he's put you in..
This miracle is different from the other ones..
Moses, unlike other miracles, is not gonna be.
the only person that God is going to operate through..
God's gonna need all the people..
He's gonna need multiple people in that community..
God wants to use multiple instruments.
to provide for his people in this case..
The reality is that there will be parts.
of your liberation journey that God intends.
on us doing together in community with his people..
God uses Aaron, God uses Moses, God uses Hur,.
God uses Joshua to provide this miracle..
This miracle shows us that God provides for his people,.
yes, but sometimes in such a way that their community.
is the only way for them to walk out.
that expression of provision..

$^{641}$The community becomes essential..
It becomes irreplaceable, a real part.
of the Exodus journey into freedom..
Every role is important..
It doesn't matter which role you hold..
They're all important in the community..
It may be easy..
I read this story, I naturally put myself,.
'cause I'm a little self-centered sometimes,.
I naturally put myself as the main character, Moses..
So I read this, I'm like, okay, cool,.
I need to be in community..
I'll be like the Moses and I'll have the,.
I don't know, I'll be holding up the stick.
as God's told me to and I'll have Jethro and Emma.
and B'lanna and Ron and Carla and Andrew.
and they'll all be holding up my arms.
'cause I'm not that strong..
Maybe, maybe..
But there are other people in this story, right?.
What if God's actually not called you.
just to be the one who's lifting your hands.
but maybe he's called you to be in this moment.
the one who lifts the hands of the person.
that's next to you?.
What if the person next to you needs,.
hear the word, needs your support?.
Like I love the language of this text..
It doesn't just give us vague terminology..
It says that they are supporting his arms..
Like it couldn't be a more clear text on community, right?.
What if your Exodus journey requires the support.
of the people next to you?.
Let's just, I would say take a minute and look.
but let's actually take a minute and look to,.
we're gonna do it all together, ready?.
Look to your left..
Hey, good job, some people know their left.
and some people don't know that..
It's all right, a little confurring..

$^{681}$All right, now take a minute and look to your right..
These, unless you're on the end,.
then it doesn't work but it's okay..
[audience laughing].
These are the people God has put around you on purpose..
Right, if we believe that God is almighty, all powerful,.
that he sits above the world and that he orchestrates things.
to be a certain way, then he has designed.
your Exodus journey to include the people around you..
You can't just come and attend, right?.
We can't just come and see it..
We're called to come and be a part of it..
You're called to play a role in your neighbor's journey.
to liberation..
But do we recognize that?.
Do we know that that's actually how God plans on providing.
in certain parts of their journey for them..
You get to be a part of that..
The people to your left, the people to your right..
We are surrounded by people who will lift your arms..
I can name so many people here who have metaphorically.
held my arms up in some of my toughest moments.
as I waited for God to bring my provision..
And that has sustained me, that has been miraculous for me..
But I feel like some of us don't have the same story..
Some of us haven't stepped into that aspect.
of community yet..
I believe God is calling us to consider how we.
could actually do that..
God sets up miracles and he does set up certain miracles.
in our lives to necessitate the need for community..
So whether it's in a community group,.
whether it's on the worship team or the production team,.
whether you're a host, whether you're downstairs.
on the first floor, whether you're with K4C,.
whether you're on any, whether you're a mom.
and there's a moms group that talks about mom things..
These are communities that we have here at The Vine..
Whether it's Chinese community, Mandarin community,.
whether it's 180, there are so many places for you guys.

$^{721}$to step into and find people who will lift your arms.
as God takes you forward in your journey..
There are endless ways for you to be connected..
And not just be connected for the sake of being connected,.
but being connected so that you can have the provision.
that God has orchestrated for you..
God often works this way..
The question is would we be willing to,.
would we recognize and actually take a step.
in being a part of a community so that we can experience.
what it is that God has..
And if God often works in community through his people,.
then you will find that your own liberation,.
that much of the freedom that God has designed for you.
will come through the community that he's placed you in..
I'll say that again..
If this is how God works through community,.
then you will find that your own liberation journey,.
the freedom that he is leading you into,.
that journey will be designed for you to have a community.
where his provision comes through..
It will be that community where you experience.
the fullness of God's actual provision for you..
So let me just simplify all of these two stories.
into a couple of takeaways for us..
And a couple of things that we need to hold onto.
to really get this..
These two stories are just basic snapshots.
of how God provides..
Again, the whole thesis of this is that God provides.
in multiple ways..
So as we journey into the freedom that God has for us,.
the thing that we have to sit on,.
the thing that we have to build around is that God provides..
God provides for his people, right?.
That is what he does, that is who he is..
But he provides in multiple ways..
And if we don't recognize the multiple ways.
that God provides in, we find ourselves being on the out..
We find ourselves missing what he's actually doing for us..

$^{761}$If we don't recognize the ways that God provides,.
we are at risk..
We are at risk of falling back into our slavery,.
feeling that God can't be found 'cause he's abandoned us,.
or that he's lost, or that he's forgotten about us..
The enemy would love nothing more, please hear this,.
would love nothing more than for you to think.
that your ordinary days mean that God no longer cares,.
that God's no longer interested,.
God's done working with you..
And that's a powerful lie because you're going to have.
so many more ordinary days than miraculous days..
So we've got to understand, we've got to recognize.
that God is still God working in the small,.
working in the little moments of our lives..
That he's brought you out of Egypt.
and he's not left you alone..
Just because we don't see miracles.
doesn't mean that he's not providing..
It can discourage us, it can leave us doubting and confused..
But the truth is that God is truly.
in all of those moments as well..
He's providing in the big miracles and the small miracles..
The story of Elam sits as a reminder to us.
to look around and see that maybe it's not.
something spectacular that's happening this week..
Maybe it's just a regular week where you have like,.
you know that game, not that game,.
that like check-in where it's like,.
give me your highs and give me your lows, right?.
Like the peaks..
Maybe your week was just, no real highs, no real lows..
Can you look and see that God is working in that as well?.
Can you find God?.
'Cause again, we're just gonna have so many of those days..
And it would be a terrible thing for us.
to not be able to see God and see his activity.
in the basic things of every day..
The second thing is this..
We are at risk of pivoting our dependency from God.

$^{801}$to ourselves when we don't recognize.
God's provision in the ordinary..
This one is, this was the point in the entire preach.
when I was studying and reading that I just felt like,.
like God hit me with..
Like we, the Elam moments feel so random..
They just seem so, they're so natural, right?.
Nothing big and spectacular happened..
And if we're only used to seeing God.
in the big and spectacular,.
perhaps we're tempted to see that Elam.
is a by-product of me, right?.
The temptation easily can be, well, I worked hard..
Israel had a very good sense of geographical location..
That's why they found Elam..
Or I studied hard, that's why I got this..
Or I left on time, that's why I got this..
Or I went out of my way and I was extroverted,.
that's why I have these many friends..
If we don't recognize God's provision in the little,.
we become little gods ourselves..
We think that we become dependent on us to get us through..
And here's why it's problematic..
Because if you're gonna have so many days of ordinary,.
what ends up happening is that we build our faith on us,.
on us, on us, on us, on us, on us, on us,.
and we reach a moment where we need God.
and we don't even know how to find him..
And we don't even know that he's actually the one..
And we depend on ourselves and we end up more broken.
and more disappointed..
Could we choose to see that in those moments,.
God is the one providing in the small..
God is the one that is the sustainer of every breath..
Scripture says that every good and perfect gift.
comes from who?.
Not from me..
We're not the ones that do it..
We've got to be able to remind ourselves.
and root ourselves in the daily things,.

$^{841}$the daily bread, the endless grace,.
the mercies that are new every morning..
All of these are blessings from God..
They're his provision..
They may not be big..
They may not be glamorous..
Seas might not be parting, but it's still God..
Can we see it?.
Can we guard ourselves from the temptation.
to become overly confident in us?.
I think one of the best ways we can do this,.
'cause I don't wanna just leave you with theory,.
I wanna give you something practical..
One of the best ways, it's super simple, super ordinary,.
which fits the theme of today..
Thank you..
Saying thank you, being thankful,.
small thing, super small thing,.
but saying thank you a couple times a day..
What will that do for us?.
How will that change the way that we see.
what's happening around us?.
How will that shift our perspective.
or awaken our recognition of the things that God is doing?.
I think if you actually challenge yourself.
to say thank you for the things.
that you're thankful for in your life,.
you'll recognize that daily,.
there are so many things to say thank you for..
You'll recognize that daily, God is providing..
You'll recognize that daily,.
God is proving himself to still be with you and for you..
Simple as saying thank you..
It's a cute little thing, but man, it goes a long way..
Could you challenge yourself this week?.
Maybe in the moments where you're tempted.
to complain and be frustrated,.
could you ask yourself, what am I thankful for today?.
What has he given me?.
I know what I'm wanting, I know where I wanna go,.

$^{881}$but what has he done already?.
I'm still here..
Thank you..
Try that..
Genuinely try that this week..
And see if it doesn't change the posture of our awareness,.
of the activity of God around us..
Giving thanks is massive..
And it'll help us to see that he is the God.
in the small as well..
And here's the third point..
I believe the third danger is,.
we face in our Exodus journey.
is not recognizing the role of community..
I'm, for the record, let me, disclaimer,.
I work at the Vine, yes..
I work on the creative team..
I am not on the community group's team..
They did not pay me to say any of this stuff..
But this text, if you read the text for what it says,.
you can't help but see how essential community is..
Community is one of God's methods,.
is one of his strategies for providing for us..
God isn't always going to use small things,.
or big things, or ordinary things, or red seas..
Sometimes God's gonna use people..
That's good and bad, 'cause sometimes we don't like people..
But that ends up being the avenue.
in which God chooses to reveal himself..
That ends up being the avenue.
in which God chooses to provide for us..
And if we don't recognize that God's provision.
can often be found through his people,.
then we could find ourselves not prioritizing.
the very community that we're called to be a part of..
We can find ourselves on the outside.
of the very group that God has called us.
to be a part of to receive his provision..
The story of Moses, and Aaron, and her, and Joshua.
is so helpful, because it shows us that this,.

$^{921}$that, okay, so Andrew talks about it in the video, right?.
He goes, "I put my hands up for three minutes,.
"and that was too long.".
Now, some of you know this, some of you don't know this,.
but Pastor Andrew works out a lot..
He goes to a gym, no, he has a home,.
he built a gym in his house..
Like, he's pumping weights every Wednesday morning..
And he can only do three minutes..
(audience laughing).
We should check on his routine..
No, but, this is the 11 o'clock, he's gonna see this..
Hi, boss..
Anyways, please don't fire me..
Okay, cool, back to the point..
Here's what I'm saying, here's what I'm saying..
As great as Moses was, his hands weren't enough..
No, none of us can hold the things God has for us.
on our own, the weight of his provision is heavy..
And it requires those around you,.
the people that, when you look to your left and your right,.
the people that are here in this community.
to be a part of upholding your hand,.
supporting your arms so that you can see.
God's provision through..
Will you be bold enough to take a step into community?.
'Cause God doesn't just do it for,.
see, that blessing, that miracle,.
that provision wasn't just for Moses..
Remember, this is for all the people..
Right, so one person's ability to be in community.
and to allow the community to support them,.
that's a big one, that's something I'm,.
whoo, preaching to myself here..
Allowing your community to actually support you..
Moses could have said, "Nah, I've got it..
"I'll work out like Andrew.".
No, right?.
Allowing your community to actually be there for you.
is part of what God has designed community to be..

$^{961}$I know there's, I've had great experiences with community..
I have some of my best friends actually came to the Vine.
for the first time today, I won't embarrass you,.
but I was so excited they were here..
I have great community..
But I know that's not the case for everyone here..
I know for some of us, there's some legitimate reasons.
why we're hesitant to be in community..
Maybe we're really busy, I get that..
Maybe we were in community before.
and we were stabbed in the back..
Maybe someone gossiped about us..
Maybe people have said things..
Maybe we weren't accepted..
Maybe we felt like we were on the outs..
Maybe we never truly fit in..
I understand that..
I get that..
These obstacles are real..
But can I say that I believe that this community.
here at the Vine can be a community.
that accepts you just how you are..
This community can be a community that loves you,.
that lifts your arms, that sees you into the provision.
that God has for you..
I believe that we can be that kind of community..
And if we can be that kind of community,.
why wouldn't you want that around you?.
If we can be that kind of community,.
what's stopping us now from stepping in.
to actually being with one another,.
not just attending a Sunday?.
My prayer for us here at the Vine is that we can truly.
position ourselves to care for our neighbors,.
to recognize those whose arms are tired.
and who cannot be sustained on their own,.
and be proactive to ask to come alongside.
and to support them..
My prayer for us, for all of us, is that we will do that..
And if we're doing that for one another,.

$^{1001}$man, how much is God planning on providing for us?.
How many things is he gonna provide for that?.
I just can't wait to see through the people here..
Community is so essential..
I believe God will call us to consider.
truly being rooted in community,.
supporting one another,.
supporting one another from a place of love,.
a place of being close, of knowing each other,.
lifting the arms of those that are tired..
So I'm gonna close..
My friends, my teams, can we come on up?.
I genuinely believe this, guys..
I really believe we can be that kind of community,.
the kind of community that chooses to participate.
in what God is doing in the lives of those around us,.
the kind of community that lifts the arms.
and the hands of each other,.
the kind of community that's close and that's there,.
that works together..
I also believe we can be the kind of community.
that can recognize the activity of God,.
because it's so important.
to be able to recognize where God is moving,.
what he is doing, what he has done..
That is just so crucial..
From the bottom of my heart,.
it breaks my actual heart to think.
that some of us do carry this concept of,.
where is God, because I don't see the big,.
and I know that's probably a lot of us here..
I know it breaks God's heart.
that we can't see his activity..
And so today, as we enter into a time of prayer.
and into a chance to respond,.
I believe what the Holy Spirit wants to do.
is reposition us to become people who are alert,.
who are aware..
The Holy Spirit wants to help you see.
what you've not been able to see before,.

$^{1041}$that God cares about the details about your life..
God cares about the small..
There's a passage that says that he knows.
the number of hairs on your head..
That passage goes to show you.
that he's aware of the small, of the details..
We serve an intimate God.
who walks with us through all seasons of life,.
who has an arsenal of ways to provide for us,.
and I don't want us to miss out on that provision.
because we only saw the big.
or because we're outside of community..
What does it look like to take a step today?.
What does it look like to choose to say thank you,.
to actually see that, man, God has been constantly at work.
providing for me day in and day out?.
God has been working..
How can you become better at recognizing it?.
How can you become better and more aligned.
to his design of community in your life?.
Father, we thank you for your word..
We thank you for these examples.
that just show us, again, a clear picture.
of just who you are, of how faithful you are,.
a faithful God who works and provides.
even in the small things..
Faithful you are..
God, I pray for those of us who are having a hard time.
seeing your provision in the small..
Holy Spirit, would you begin to awaken that alertness?.
Would you begin to show?.
Would you begin to show us just how often,.
just how regularly, just how close you are.
and how often you're providing?.
Would you remind us of where these good things come from?.
Would you remind us of where the breath.
in our lungs come from?.
The source of all good and perfect things..
God, for those of us who carry hurt,.
carry brokenness over the idea of community,.

$^{1081}$those of us who are afraid to take a step into community.
because we're ashamed, because we've been hurt,.
because we've not recovered from past pains,.
Holy Spirit, would you begin to start working now.
in our hearts?.
Start mending like you do..
Start healing like you do..
Start making the old things new like you do..
May the idea of hurt in the past no longer hinder your people.
from being in the place of community.
that you've called them to be..
God, would you make us a community that is sensitive,.
that is patient, that is kind, that exhibits your fruit.
so that when people do come in,.
that we will lift their hands and not judge,.
that we will welcome and not stand back,.
that we will treat them the way that you've treated us,.
the way that you've welcomed us into your family, God..
Make us into that type of people..
Make the vine here in Luncheye the kind of church.
that's like that..
We give you glory for all these things..
God who provides in the big and the small,.
the God who calls us as a community.
to experience this provision together,.
we thank you for these things..
Why don't you stand, church?.
\newpage



\section{}
\label{sec:AG5PdzOFde8}
\textbf{2023-10-03 EXODUS - 17 Generational Wisdom [AG5PdzOFde8].mp3}
\newline
\newline
連結: \href{https://youtube.com/watch?v=AG5PdzOFde8}{\texttt{ https://youtube.com/watch?v=AG5PdzOFde8}} ~~~~ 語音日期: 2023-10-03 
\newline
\newline
\hyperref[sec:JVjucZ_U4Bw]{\small{< < < PREV SERMON < < <}}
~
\hyperref[sec:index]{\small{[返主目錄]}}
~
\hyperref[sec:CKM0h5f9Oa4]{\small{> > > NEXT SERMON > > >}}
\newline
\newline
$^{1}$In Jesus' name, amen..
Thank you, worship team..
Thank you, everyone..
Say hello to somebody as you sit down..
Well, we are really excited that we continue to be.
in our Exodus series, which has been just such an amazing.
journey for us as a church..
And today we get the privilege of having.
our two founding pastors share our message today..
So would you give a very warm welcome.
as I introduce firstly, John Snellgrove, Pastor John..
It's so good to have you here..
It's so good..
And Sandra was here earlier as well..
It's such a great honor to have you stand.
and share a message..
And I know that there's a new season going for you.
and Sandra as well, Priscilla and Aquila,.
I believe it's called..
And this really is an opportunity for John and Sandra.
to empower the global church, encourage the global church.
in their continuous faith journey.
and to really equip the younger generation,.
which is really exciting..
And if you wanna find out more,.
then the details are showing up on the screen..
Do contact John and Sandra..
I'm sure that they would love to tell you more about it..
So let us pray for John and Tony..
Father, we thank you..
We thank you for the opportunity we have to listen.
and hear the wisdom from John and Tony..
And I pray that you would bless John as he shares right now.
and that our hearts would really receive what it is.
that you have for us in Jesus name, amen..
- Amen, thank you, Carla..
And good morning, church..
- Good morning..
- Good..
You are not able to go it alone..

$^{41}$Can you turn to the person next to you and tell them also,.
you are not able to go it alone..
I noticed something here in Hong Kong..
When you ask someone the question, how are you?.
I invariably get the answer..
You know what I'm gonna say, don't you?.
I'm busy..
Busyness in a city of business is seen like a badge of merit..
I'm busy because I'm important..
And yet Moses was in many ways like that..
Yes, our hero of the Exodus.
and the miraculous crossing of the Red Sea..
And as we look today how he dealt with it,.
I am praying for us..
That we would each experience a personal Exodus.
from the need to be busy..
To be the person who has to do everything.
so that the world continues to take over..
There is a term which is more and more frequently employed.
in Christian circles, which depicts a problem.
that has been widespread amongst evangelicals,.
even epidemic..
And that term is burnout..
Burnout happens frequently to Christian leaders.
who strive to make impossible expectations and demands..
The achievement of which will show him or her.
to be both spiritual and successful..
Failure to accomplish these expectations and demands.
is believed to prove one lazy,.
unspiritually minded or a failure..
Burnout occurs when in sheer exhaustion and frustration,.
we lose all hope of meeting the standard.
which is imposed on us either by ourselves, others or both..
And we simply give up..
And burnout is not just a phenomenon found.
in Christian leaders or just among Christians.
for that matter..
Burnout is caused by the sheer volume.
of non-essential activities, which we foolishly,.
foolishly strive to maintain..

$^{81}$Are you guilty?.
I know I am..
And Moses' answer comes in the unlikely shape.
of a visit from his father-in-law..
I mean, there are lots of mother-in-law jokes,.
but there's very few father-in-law jokes..
Moses was dangerously close to burning himself out.
when of all people, his father-in-law came to his rescue..
What appears on the surface to be the insignificant visit.
of a relative is really a divine provision.
to deliver Moses..
Not from the wrath of Pharaoh,.
nor from the attack of the Egyptian army,.
but to deliver him from himself..
Now as Jethro, that's the name of the father-in-law.
by the way, it's nothing to do with the vine worship leader..
As Jethro himself puts it,.
Moses was weary himself and the Israelites out..
Look at this, verse 18..
You and these people who come to you.
will only wear yourselves out..
The work is too heavy for you..
You cannot handle it alone..
Thanks to the common sense of a wise father-in-law,.
Moses was delivered from his own destruction..
The burnout which resulted from a distorted perception.
and a too demanding ministry..
We're gonna be looking today at Exodus 18.
and the text of chapter 18 divides evenly.
into two simple and straightforward portions..
Verses one to 12, which I will summarize.
by the title Jethro's Arrival..
And verses 13 to 27, which will depict Jethro's advice..
The two portions are very much related..
As a spoiler alert, we will have a film from Egypt.
at the halftime mark and a special guest,.
the vine's own Jethro in part two..
The first half of the chapter reveals several symptoms.
of a serious problems in Moses' life,.
which prompted not only the arrival of Jethro.

$^{121}$at the Israelites camp, but also his advice..
We would do well to listen to the wise words.
of this Midianite, who has much to teach us.
about managing our lives and our ministries..
For those of us like me, who are predisposed.
to busyness and over involvement,.
they can spare us from the deadly disease of burnouts..
The first section verses one to 12,.
which I've called Jethro's Arrival,.
breaks evenly into two divisions..
The first division, verses one to six,.
might be entitled Focus on the Family..
They reveal the occasion,.
the reason for the arrival of Jethro..
Verse one informs us of the basis.
for Jethro's decision to visit Moses,.
whilst verses two to six tell us.
of the purpose of that visit..
The second division, verses seven to 12,.
focus on the faith of Jethro..
They depict the outcome of Jethro's arrival..
Moses gets the chance to report.
on God's good hand to the Israelites,.
and Jethro responds to God's goodness to Israel,.
rejoicing, proclaiming God's greatness,.
and worshiping him with Moses and the elders of Israel..
But this is not the 21st century..
It's difficult for me to envision.
just how Jethro gathered the information.
about the well-being of Moses..
There was no Facebook or IG, or even CNN at the time..
But the text tells us, verse one,.
he heard everything God had done for Moses and his people..
Perhaps Jethro had made a point to invite travelers,.
even caravans, to share a meal with him,.
or spend the night in his tent,.
enable him to learn the things about Egypt..
The point of the passage is not how Jethro learned.
of Moses' well-being, but of what he learned..
He learned that God had protected Moses,.

$^{161}$and that he had delivered the Israelites out of Egypt..
Jethro had learned enough to conclude.
that the circumstances were such.
that Moses and his family should be reunited..
And verses two to six indicate the purpose.
of Jethro's visit to Moses, to reunite Zipporah,.
his daughter, Moses' wife, and Gershom and Eliezer,.
the grandsons, Moses' sons, with Moses..
It was apparently a pleasant surprise for Moses..
But whilst we might have expected him.
to pay much more attention to his wife.
and children coming back, we read that Moses.
is reported to have immediately gone to meet Jethro,.
kissed him, and then go to Jethro's tent to meet with him..
Inside the tent, Moses and Jethro went through.
the formalities of a reception of an honored guest.
in Oriental culture, in Eastern culture..
Moses brought Jethro up to date with a detailed report.
of how the hand of God had delivered the Israelites.
and devastated the Egyptians..
Verse eight, Moses told his father-in-law.
about everything the Lord had done to Pharaoh.
and the Egyptians for Israel's sake,.
and about all the hardships they had met along the way,.
and how the Lord had saved them..
Jethro's response, described in verses nine to 12,.
seems to be more than just Oriental courtesy..
It appears that Jethro here professes a personal faith.
in the God of Israel, which as a pagan,.
he had not previously..
First, Jethro rejoiced with Moses, praising God.
for his grace manifested towards Israel.
as evidenced by Moses' report..
We read this in verse nine and 10..
Jethro was delighted to hear about all the good things.
the Lord had done for Israel in rescuing them.
from the hands of the Egyptian..
He said, "Praise be to the Lord, who has rescued you.
"from the hands of the Egyptian and of Pharaoh,.
"and who rescued the people from the hands of the Egyptian.".

$^{201}$Secondly, Jethro seems to acknowledge for the first time.
the superiority of God above all other gods,.
which one would have supposed would have included.
his previously worshiped pagan gods..
Jethro's faith is demonstrated in the offering.
of sacrifices to God and the sacrificial meal,.
which Jethro, Moses, and all the elders of Israel shared..
Verse 11 says this..
"Now I know that the Lord is greater.
"than all the other gods, for he did this to those.
"who had treated Israel arrogantly..
"Then Jethro, Moses' father-in-law, brought a burnt offering.
"and other sacrifices to God, and Aaron came.
"with all the elders of Israel to eat a meal.
"with Moses' father-in-law in the presence of God.".
Therefore, for Moses, this was a happy occasion,.
not only because of Israel's victory over Amalek,.
but because of the renewed fellowship.
and the renewed relationship with his wife and his family..
But it leads me to a but, and it's a big but..
Having briefly considered the arrival of Jethro.
and Moses' family and the affirmation of Jethro's faith,.
I'm left with a nagging question..
Why was it, why was it that Jethro had to initiate.
the reunion of Moses and his family?.
Put it differently, why didn't Moses send.
for his wife and sons, rather than have Jethro.
show up with them unexpectedly?.
Now, we have a tendency, don't we, to assume the best.
of our biblical heroes, like Moses, David, Abraham, et cetera..
I call it the pious bias..
It seeks to elevate biblical characters.
to a level far above our own performance.
and far above what we would expect, knowing man's nature..
By the way, not Jesus, I'm saying, okay?.
But when we come to chapter 18, it is not Moses' strengths,.
it is not his virtues, which are extolled in this chapter,.
but his weaknesses..
Now, here is a man, Moses, that I can identify with,.
a man with flaws like mine..

$^{241}$And these flaws are keenly observed by Jethro,.
whose advice in verses 13 to 27 is based upon.
his observation of Moses' unconscious blunders,.
both with regard to his family, and as you shall see later,.
his function as Israel's leader..
Let us press on to see how the events of verses one to 12.
serve as a clue to the failures which Jethro seeks to remedy.
by his wisdom and counsel..
To prepare us, let's go to Egypt,.
and to an interview that Pastor Andrew had.
with Father Justin, not you, by the way..
One of the head monks at St. Catherine's Monastery.
at the foothills of Mount Sinai..
(gentle music).
(water rushing).
(gentle music).
(water rushing).
(gentle music).
(water rushing).
(gentle music).
(water rushing).
(gentle music).
(water rushing).
- Father Justin, it is such a privilege.
to be with you today and to see this incredible library..
I can't wait to kind of look around.
and have you show me some of the things that are here..
- Well, there have been monks living here for 1,700 years..
The library we have has been slowly built up.
over the centuries, and people have said.
it's the memory of the monastery, so it's a great treasure..
(gentle music).
This is the most illuminated manuscript.
of the latter that we have..
The passage that I opened it to.
is when he discusses discretion..
He says the greatest gift is discretion.
because discretion is knowing.
how to make spiritual decisions,.
how to weigh things, how to know the way forward..

$^{281}$And he illustrates that by having three monks..
The one is still lying in bed, half asleep..
The other one is middle-aged,.
and he's standing in the center..
And then the other one has attained, under old age,.
you see his white hair and his beard,.
and he is being crowned by Christ.
who appears from the corner, and he's blessing him..
- Yeah, I was gonna ask you if that was Christ..
- So this is the person.
who's just beginning the spiritual life,.
the one who has reached the middle stage,.
and the one who has reached the summit..
- And when you're fathering the younger monks.
in the monastery here,.
this kind of process of spiritual discipleship, I suppose,.
is that in your mind.
as you're leading the younger monks.
through their life here?.
- One of the beautiful things, not just in monasteries,.
but in the whole Orthodox Church,.
is that we are not expected to do this by ourselves..
We look to others who have experienced grace,.
who have grown in the Spirit,.
and we follow their example,.
and we receive their instructions..
So it's an essential part,.
especially at the beginning of the spiritual life,.
to have direction, to have guidance..
Many times we think that we're doing very well,.
and we're not..
Many times we think that we're not doing very well at all,.
and we are..
So it's through experience.
that a person can set us on the right path.
and make sure that we're going in the right direction..
- Well, I'd love to actually sit down with you.
and talk a little bit more about that relationship.
between being a spiritual father.
and the monks here in the monastery..

$^{321}$So perhaps we can go and sit down together.
and have a chat. - Fine..
- Thank you so much for showing me this..
Father Justin, that manuscript you just showed us.
of the three figures.
kind of in their various different states, I suppose,.
of spiritual maturity,.
really opens up for me a topic.
that I'm very interested in exploring with you.
around that whole sort of sense of being a spiritual father.
within the Orthodox community..
And I'd love to just get a sense from you,.
what does that look like?.
What does that sort of spiritual father.
to the junior monk relationship look like.
within this community?.
- A monastery usually has one person.
who is very responsible for the spiritual formation,.
especially of those who have just arrived..
And in Sinai, the senior father.
under the archbishop is called the dikeos..
The dikeos is a superior..
And one of his primary obligations.
is the spiritual formation of the novices..
So that's how he conveys to them the spiritual heritage.
and the spiritual witness of the monastery..
So when you read the sayings of the desert fathers,.
these are just short anecdotes..
A person went to a monk and asked him this question,.
he gave this answer..
And it's profound, and it's just a short saying..
So they're very approachable,.
but those are some of the earliest monastic texts.
that I read..
And I was so enthused by everything that I read,.
I was wondering, are there still people.
that live like this?.
That was my own beginnings..
And I think that everyone goes back to that first experience.
when they think about the monastic life.

$^{361}$and the groundwork that they laid..
And they understood from experience.
that you have a person who is more mature,.
more experienced, who can guide a person.
who's just starting out in the spiritual life..
So that tradition of the elder and the disciple.
goes back to the very beginning..
The most important book written here at Sinai.
is called "The Ladder of Divine Ascent.".
And if you look carefully, you can see how he was drawing.
on that whole tradition of the Desert Fathers..
He describes the spiritual life like a ladder,.
when you start for the first rung.
and go to the second rung with Christ at the summit,.
summoning us to make the spiritual ascent..
And when we first start out, we need care..
- Right..
- He mentions that when you have been in a monastery.
for a short time, many times you are thinking,.
well, now I've reached perfection, I can go back home,.
I can become the great preacher, I can bring out a vacation..
So from the very beginning, you need discretion.
to know which thoughts that are coming to you.
are to your benefit and which ones are undermining your goal..
And you gain that by having an experienced elder.
who can guide you, not only theoretically,.
but based on his own experience..
One person said, "It's not a little stream.
"that we jump over, it's like a great ocean.".
And no matter how tall a person is,.
when he wades out into the ocean,.
he finds things that are over his head..
So it's a very profound understanding,.
and it's a process that continues his whole life..
(gentle music).
- Father Justin, I also wanted to ask you a little bit.
about what a day in the life is like.
for a monk in the monastery here..
Can you tell me a little bit more about that?.
- It's centered around the services..

$^{401}$The first service is at four in the morning..
Four in the morning, all the stars are out.
in the desert sky, and it's beautiful.
to have a service in the quietest part of the day..
The service lasts until seven o'clock in the morning,.
and by that time, the sun has come up.
and a new day has begun..
And it's beautiful to watch the light.
slowly increase in the church..
- And in the service, it's readings?.
- Reading from the Psalms,.
the hymns for the saints of the day..
We celebrate the liturgy here every day,.
and so that's the morning service..
We have a short service at noon..
We have vespers in the afternoon..
So the beginning, the middle, and the end of the day.
is spent in the services..
And I think because the community is so small,.
there's a sense that we're more like a family..
- And what does this family mean to you?.
I mean, you've obviously been a part of this family.
for many years, as you were sharing with us earlier..
What does it mean for you to actually have.
that kind of family so consistently in your life?.
- There's an amazing balance..
You have your personal space,.
but you also have this sense of community..
And the one doesn't intrude on the other..
I think they're both necessary.
'cause you must be yourself..
There's no, what you would call, institutionalization.
where a person conforms to something..
A person remains himself, but at the same time,.
he becomes a part of this larger whole..
And as I mentioned, there's this beautiful balance.
between personal space and being part of a community..
- Wonderful..
You're gonna show me some other places in the monastery..
Shall we go? - Yes..

$^{441}$- Father Justin, I wanted to also ask.
about the wisdom tradition and the role.
that maybe the wisdom books play in your order as well..
We've talked quite a bit about the Psalms.
and obviously the big influence they have,.
but how does wisdom traditions play within the order?.
- We read the entire book of Proverbs.
during Great Lent in the services..
And the wisdom that he imparts.
is very much a part of our tradition..
And there's such a similarity.
between the inherited wisdom.
that is enshrined in the book of Proverbs.
and the sayings of the Desert Fathers,.
which is lived experience, solutions,.
memorable sayings, memorable events.
that have all been collected.
and then passed down from one generation to another..
The beautiful thing about living at a place like this.
is that every day you have a greater appreciation.
and a greater sense of the significance of the tradition,.
the significance of the place.
and everything that you experience here..
It's constantly growing and knowledge and wisdom.
and a beautiful process that never has an end..
- And that's the thing that I think.
you've really impressed on me over our time together.
is that it's this wonderful combination.
between theology, between practice,.
the services of worship you have,.
but then the lived experience.
and doing that in community together..
And all of that working together.
to really create that spiritual formation as a community..
I think that's profound..
- Thanks..
While I'm waiting, that was quite some beard, wasn't it?.
(audience laughing).
So let's move on with our story..
The next morning, Moses and the people of Israel.

$^{481}$began their daily routine..
The people who sought to know God's will from Moses.
began to line up at the designated place,.
perhaps just outside Moses' tent..
With a nation composed, as Andrew told us in previous weeks,.
of nearly two million people,.
one could imagine the line was long.
and that it began to queue up very early in the morning..
I don't know if it's four o'clock or not..
Moses, we are told, seated himself.
sitting as Israel's sole judge..
Let's read this..
The next day, Moses took his seat.
to serve as judge for the people..
And they stood around him from morning till evening..
When his father-in-law saw all that Moses.
was doing for the people, he said,.
what is this that you're doing for the people?.
Why do you alone sit as judge?.
Why are all these people stand around you.
from morning until evening?.
You see, the people came to him with all those matters.
which needed a decision, an instruction, or counsel..
The people looked to Moses alone for a word from God.
for guidance in their lives..
At the end of the day,.
the long line of waiting Israelites was still there..
Jethro was able to quickly identify the problem.
to which it seems Moses was oblivious..
Jethro was baffled by the inefficiency.
of what had taken place during the day..
So perhaps over dinner that night,.
they sat around and Jethro began to inquire.
about Moses' rationale for administering justice,.
judging as he was doing..
It was apparent from Jethro's questioning.
that he did not agree with the way.
that Moses was doing things..
Even the way the questions are phrased in writing,.
you can imagine the tone of voice which he asked..

$^{521}$Father-in-law, as you know,.
can do a better job than others on this..
He usually starts with a question..
Are you a good enough man to be my daughter's husband?.
I believe Moses was completely caught off guard.
by Jethro's disapproval..
Moses was so engrossed in his work,.
so desperately trying to keep his head above water.
that he didn't have time to reflect on what he was doing..
Jethro, on the other hand,.
had already suspected a problem for some time..
Moses had not only sent his family home.
for Jethro to care for,.
but apparently he'd had little contact with them.
and he delayed in reuniting with his family..
That morning, Jethro began to see the pieces.
fall into place..
Moses had not sent for his family..
Why?.
Because he did not have time to care for them,.
even to think of them..
The response of Moses reveals his distorted perception,.
which was the root problem here..
Moses answered him,.
because the people come to me to seek God's will..
Whenever they have a dispute,.
it is brought to me and I decide between the parties.
and inform them of God's decrees and instruction..
While Jethro quickly sized up the situation,.
Moses wasn't thinking very carefully.
about what he was doing..
His response reveals several misconceptions.
regarding his role as a leader..
Consider them with me for a moment..
Firstly, Moses believed that every request for help.
made the matter his responsibility..
Secondly, Moses seemed to assume.
that because the people came to him personally for help,.
it was his responsibility to help them personally..
Thirdly, Moses wrongly reasoned that because his task.

$^{561}$was to lead the entire nation,.
he must do it by dealing with people one at a time..
Fourthly, Moses seemed to have assumed that no one else,.
no one else was able to do what he was doing..
And most importantly, Moses seems to have lost sight.
of his unique gifts and calling..
Let me repeat that..
Moses seems to have lost sight.
of his unique gifts and calling..
God had not called Moses to do everything,.
but to do some things..
Moses was given responsibility.
to lead the nation of Israel as a whole..
And therefore, his task was very different.
from that of others who could deal with people.
on a personal, intimate, one-on-one basis..
I believe that we can gather several important principles.
of leadership from the words of Jethro.
which are addressed to Moses..
And to help me, I want to introduce my very own Jethro..
Not my father-in-law..
But a wise, can I say older?.
Okay, you know we're 150 between us,.
you know, just telling you that..
You gotta work out how that's balanced out..
But God sent this man to me.
and to this church to impart common sense.
and generational wisdom..
Tony will address these principles.
and their practical information..
So let's welcome Tony..
[audience applauding].
>> Well, thank you..
It's good to be here and to be able to talk on this as well..
We need to hear now, not from Moses' problems,.
but to Jethro's advice..
And we're gonna do that firstly by reading.
exactly what it was that Jethro said..
So here in chapter 18, Jethro is speaking.
and he's speaking to Moses..

$^{601}$And he says this..
Listen now to me and I will give you some advice.
and may God be with you..
You, you must be the people's representative before God.
and bring their disputes to him..
Teach them his decrees and instructions.
and show them the way that they should live..
But select capable men from all the people,.
men who fear God, trustworthy men who hate dishonest gain,.
and appoint them as officials over thousands,.
hundreds, fifties, and tens..
Have them serve as judges for the people at all times..
But for the simple cases, they can decide themselves..
That will make your load lighter.
because they will share it with you..
If you do this and God so commands,.
you will be able to stand the strain.
and all these people will go home satisfied..
Now we need to just step back a moment here.
and see that what Moses was doing.
was hindering the fundamental reason.
that the people were on this journey..
They were on this journey to seek their freedom..
It was for freedom that they were in the desert.
and leading to that promised land..
Jethro is sensing, quite rightly,.
that their journey of freedom.
is turning into one of resentment and exhaustion..
I want to summarize the advice.
by bringing out four general principles of leadership..
The first one of these.
is the principle of control and release..
Now you might think from a quick reading of this passage.
that Jethro's advice was intended to release Moses.
and to maintain control of the people..
But actually, it was the other way around.
because the principle role of Christian leadership.
is to lead people out of slavery and into freedom..
In order to do this effectively,.
the leader must maintain control of themselves first.

$^{641}$and not the people..
And when the leader is in control of himself or herself,.
then the people will respond and follow..
If it's not done well,.
then the leaders will have difficulty in leading.
because of the lack of confidence.
that they perceive in the leader himself..
And Jethro's advice to Moses.
involved a specific sort of pyramid arrangement.
of organizational model..
And it was designed to relieve the burden on Moses, yes,.
but also to release the people into leadership..
And that model perhaps was appropriate.
for slaves who had only experienced slavery.
in their lifetime,.
but it was not an ordained model.
that was divinely appointed.
for all Christian assemblies thereafter..
The significant thing about Jethro's advice.
was in its outcome to restore the journey of freedom.
and not in the specific organizational model.
that he spoke about..
So the first principle of leadership is this,.
that leaders need to be in control of themselves first.
before they can release others..
The second principle I want to talk about.
is the principle of the servant and master..
You see, God has given each of us giftings.
which we can use in our leadership..
But leadership itself is not a specific gift from God.
because there are many different types of leadership..
But Jethro provides some solid advice to Moses.
on how to exercise his specific role.
and how he should release the people into their role..
But more importantly, he gave him some advice.
on the qualities of the people.
that he should be looking for..
He said, "But select capable men,.
"men from all the people who secondly fear God,.
"thirdly, who are trustworthy,.

$^{681}$"and men who hate dishonest gain.".
And that would be the sense of having.
a sense of justice within their heart..
And Jethro wanted to make sure that these men.
would be able to carry out their duties as servants.
and not as overlords or masters..
We might use different wording in our Christian context,.
but these qualities of servanthood.
that he outlined are important.
no matter what sort of leadership gifting that we have..
And it's in this context of servanthood.
that we need to work on the basic skills and gifts.
that God has given us so that we can master those gifts.
and use them well in leadership itself..
Some leaders are gifted organizational managers..
Some are inspirational speakers..
Others may be knowledgeable scholars..
Others may use authenticity.
and life experience to lead others..
There are many ways of types of leadership.
that we can experience and use..
But whatever type of leadership you have,.
then the second principle of leadership is this..
Leaders are called to lead as servants of Christ.
by mastering the gifts that they have been given..
And the third principle I want to talk to.
is that of the shepherd and the sheep..
In a church setting, the role of the shepherd.
is to lead the flock to safe pastures.
where they can be cared for and protected..
In doing this, she or he is providing.
a collective leadership for the flock..
But what happens if one of the sheep.
goes astray and gets lost?.
Well, we know from our New Testament parables.
that in that occasion, the shepherd goes.
and leaves the 99 to find the one..
But in a large church, it will be impossible.
for one leader to know each individual person.
by name and their circumstances..

$^{721}$And just as Jethro suggested,.
it will be important to train up other leaders.
who can provide that important element.
of personal engagement and shepherding..
And this is why the development of communities is important..
In Moses' day, it was possible for him.
to use the natural structure of the tribes.
and families that existed around them..
But in our highly mobile societies,.
we need to be much more intentional.
in developing and nurturing communities.
where people can find greater spiritual.
and relational support..
And so at the Vine, we place great importance.
to our community groups as places.
where people can experience and learn.
and how to engage with different personalities and cultures..
It's in that setting that we can learn.
how to support each other, sharing life together,.
and putting into practice the teachings that we receive..
These groups are great incubators.
to learn leadership skills and to exercise our gifts.
and to grow in spiritual maturity..
So the third principle of leadership is this..
Shepherding is the caring and nurturing of the sheep.
so that they can in turn become the shepherds of others..
And the fourth and final principle.
that I want to talk about.
is the principle of the one and many..
Because at whatever level of leadership we find ourselves,.
we need to learn how to manage ourselves.
and how to manage others..
And I'm not talking about having to get an MBA here,.
but I'm talking about the basic skills.
of organizing ourselves and our activities.
to achieve our goals..
As our responsibilities increase,.
we will undoubtedly need to interact with others.
in teamwork and shared leadership situations.
become the requirement of the day..

$^{761}$You see, leadership is rarely a solo activity.
because of the sheer complexity of the lives we lead.
and the situations that we find ourselves in..
And at the Vine, we have many levels.
of shared leadership responsibilities..
We practice shared leadership in our staff organization..
We practice it in our eldership.
and in all of our ministries where possible..
And at one time in the past,.
the church was even run by an eldership.
as a shared leadership responsibility.
without any full-time pastors at all..
And our first full-time pastoral leadership.
was a dual appointment of John and myself.
as co-senior pastors.
in which we led the church together.
and using our complimentary gifts..
And I say this only to emphasize.
that shared leadership is very much in our blood as a church.
and from it, we can learn so much about life and ourselves..
So the fourth and final principle of leadership is this..
Church leadership is seldom a solo activity..
Practicing shared leadership together.
will produce a healthy, growing community..
Thank you..
(audience applauding).
- The principles which Moses learned from Jethro.
are applicable to every Christian,.
whether he or she is a leader or not..
So as we conclude, I wanna consider some initial steps.
which may get us started on our way.
to becoming better leaders and managers..
Number one is simple..
Find a Jethro..
God gave Moses a Jethro to point out his problems..
Fortunately, God gave me a Jethro in Tony and others..
If you do not have such a friend, get one..
Number two, establish a plan for both your private world.
and your public ministry.
and determine not to neglect either..

$^{801}$This is big..
Determine the priorities which will govern those things.
that you will do and those you will turn down..
Of all the things you could do,.
seek to identify and achieve those things you should do..
Number three, determine to facilitate.
the ministry of others, especially by encouraging.
and equipping them to do what they do best..
In other words, begin to focus as a Jethro..
Now, I know I've asked you just to find a Jethro,.
but begin to focus as a Jethro.
to the many Moseses around you..
And lastly, desire to grow in faith.
and at the same time, humility..
Faith is required to trust God to enable you.
to do what he's called you to do..
Faith is also required to enable you to leave.
what you should not do to others..
Humility will keep you from self-trust,.
will prevent you from taking credit.
for what God has accomplished..
It will also enable you to resist.
the ego-flattering suggestions.
that only you are the solution to a particular problem..
My friends, you are not able to go it alone..
Talk to the person next to you..
Tell them, you are not able to go it alone..
As I ask our band to come back,.
our story has a happy ending..
Let's see it..
Verse 24..
Moses listened to his father-in-law.
and did everything he said..
Amen, let's stand up..
(congregation laughing).
What I wanna do now is to pray for us..
We're just gonna have some music playing in the background..
But this is really your time..
As we prayed before the service,.
we had this picture of an ice cube that was melting.

$^{841}$and felt that really all of us need today.
to just allow God to melt away on the extremes for us..
I'm gonna ask that God would give to you.
maybe a picture, a name of someone you need to be asked.
to be your Jephro, it may be more than one person..
Also that God would put in your mind.
people that you need to get alongside,.
you need to encourage..
Let's pray..
Holy Spirit,.
we pray that you do a work in each one of our lives today..
We thank you for this story.
of how a father-in-law.
was used by you.
to change the life of Moses.
and by definition change the life of a relation..
Come to me now,.
Lord, and start to show me.
maybe there are things in my life.
that I'm doing that I need to stop doing..
(gentle music).
Lord, help me to find a Jephro.
and help me, Lord, to be a Jephro.
to the next generation..
And as a church, Lord, we ask you.
that you would make us this church of cross generations.
(gentle music).
Lord, how you would allow us..
We thank you, Lord, for the young people in this church,.
the sharp end, the energy,.
the desire to get things done..
Help us to be ones who would get around them,.
to empower them, to encourage them..
(gentle music).
And if necessary, show them the way..
We thank you, Lord, for the many families.
you're raising up in this church,.
in some ways the backbone of the church..
I pray, Lord, you'd help us.
in this area of family priorities..

$^{881}$So often in Hong Kong, Lord, we let business,.
we let church, we let things that we deem to be important.
get in the way of our families..
And we pray, Lord, that you continue to raise up.
wise people, Lord, sometimes older, sometimes not,.
but people who can just give us direction in life..
(gentle music).
So Holy Spirit, I just wait on you now..
Would you speak to me?.
Would you speak to us as a church?.
And actually, would you speak to us.
as a city here in Hong Kong?.
And would you raise up Moses's and Jethro's in this city?.
Thank you that your plan for us is cross-generational..
We pray this in Jesus' name..
And the whole church said, amen..
Thank you..
(gentle music).
[BLANK AUDIO].
\newpage



\section{}
\label{sec:CKM0h5f9Oa4}
\textbf{2023-10-16 Exodus - 18 The God of Yesterday, Today and Tomorrow [CKM0h5f9Oa4].mp3}
\newline
\newline
連結: \href{https://youtube.com/watch?v=CKM0h5f9Oa4}{\texttt{ https://youtube.com/watch?v=CKM0h5f9Oa4}} ~~~~ 語音日期: 2023-10-16 
\newline
\newline
\hyperref[sec:AG5PdzOFde8]{\small{< < < PREV SERMON < < <}}
~
\hyperref[sec:index]{\small{[返主目錄]}}
~
\hyperref[sec:FFFAigIMmRc]{\small{> > > NEXT SERMON > > >}}
\newline
\newline
$^{1}$In Jesus' name..
Everyone says, hey, can we thank our worship team as always?.
Amazing..
Have a seat, have a seat..
And as you see it, I wonder whether you could imagine.
with me or think with me back to your childhood..
And I want you to think about the greatest memories.
that you have of your childhood..
The times in your heart and in your lives.
that you remember back and you think back to almost.
what I would call core memories of your childhood..
Those moments where you think about all the great things.
that happened for you..
And maybe some of you have great memories of your family.
or great memories of a holiday you took.
or whatever it might be, but those memories where you're.
like, that defines who I was as a child..
Those are memories that I cherish,.
memories that I celebrate..
For me, so many of my core childhood memories,.
they're located largely in one singular place..
And that's this house..
This is the house that I lived in from the age.
of eight years old to about 11 years old..
The house is located in a small town in Connecticut.
on the Northeastern side of the US called Wilton..
Wilson's a small town and we lived in a small cul-de-sac.
in a small neighborhood..
And so many of my great childhood memories.
are from this particular house..
I think one of the reasons why I love this house so much.
is 'cause my dad loved this house so much..
My dad worked for a bank in the UK.
and he received a promotion..
And when I was seven years old, the bank said,.
"We wanna move you to the US and you're gonna start working.
"for the bank in the US.".
And so for a year we lived in Charlotte.
and then we moved up to this house..
And I remember my dad loved this house..

$^{41}$In fact, my dad probably took this photo..
This photo is from the early '80s..
My dad took this photo because the house represented.
for him really the first time in his life.
that he was able to provide something significant.
for his family..
He had a family of five and he had provided this home.
and there was a certain sense of pride.
that his family could grow up.
in a beautiful, spacious place like this..
My greatest memories of my father are found in this home..
I remember he taught me how to throw a baseball.
and catch it in the driveway of this house..
We played football on the lawn that you can see here,.
both the American kind and the British kind right there..
At winter, my father taught me how to ski here..
It would snow in Connecticut and you can't see it,.
but on the left side of the building,.
there's a steep slope right there..
My father taught me to ski down that slope..
I remember the many hours that I spent with my father.
sitting in the living room here in like the little den area.
playing on an Atari Commodore 64..
I don't know if anyone is old enough to remember that..
That was a computer before they were computers basically..
So many of my favorite memories are of my father here.
and when I think about the best times of my childhood,.
this is the place I think of..
When my father passed away in March of 2019,.
I felt this overriding urge and motivation.
to return back to this house and visit it again..
This overriding sense of heart that I had,.
that I wanted to reconnect with my father.
in the very place where I had.
the greatest memories of my father..
The funny thing is when we left here,.
when I was age 11 and we moved here to Hong Kong,.
we never went back..
I hadn't been back for over 35 or so years.
and so I felt as my father passed.

$^{81}$that to honor his memory and to reconnect with him.
in a deep way, I needed to get back to this house.
and so we had a sabbatical in the summer of 2019.
and so my wife, my daughter, myself, my mom and my brother,.
we jumped on a plane from Hong Kong,.
we flew to New York, we hired a car,.
we drove the three hours up the freeway.
and we entered into Wilton.
and the very first place we stopped was this house..
I remember pulling up along the driveway,.
remember, quiet cul-de-sac, we parked the car,.
all of us got out of the car and there was this fence.
that lined up next to the house.
and we leaned on the fence and we looked at the house.
for the first time in 35 years..
We would have looked very strange..
That's what the house looks like today..
It's essentially the same house.
just with a different paint job.
and as we were staring at it, staring at it,.
if anybody had driven by, it would have been weird,.
this group of like seven people just going like this..
I remember the joy that I felt in my heart.
because Mia, my daughter, was with me.
as we were looking at this house.
and she was about the same age as I was.
when I lived there..
And as I was staring at the house,.
I realized this is not enough for me..
I have to be inside the house..
This is a very creepy thing, I know,.
but I had to work out how am I gonna get myself.
inside this house?.
So I devised a plan..
I decided that I would buy a really nice card.
and I would write a letter inside this card.
and in the letter I would say,.
I used to live in this house from the age of eight to 11.
and it was my father's favorite house.
and my father's just died of cancer.

$^{121}$and you need to let me come in.
and relive the memories of this house..
Please let me in..
Here's my mobile phone, call me..
I wrote the best letter anyone has ever written.
and I dropped it into the little mailbox..
You know, like in America, you put up the little flag,.
you know, to say that there was some mail there,.
did a little prayer and then we retreated to our hotel..
I got my mobile out and I was like,.
two hours later, the phone rings.
and it's the wife in the house..
She's wife, husband, two kids.
and she's like crying on the phone..
She's like, your letter so moved me..
We bought this house five years ago.
and we love this home and it's so good to hear.
that you love it as well and your father's just passed away.
and would you like to come over tomorrow.
at 10 o'clock for coffee?.
I would love to host you in the home.
and help you to relive your memories.
and I was like, yeah..
Four hours later, the phone rings again..
This time, it's the husband of the house..
This is a very different conversation.
to the one with the wife of the house..
The husband's like, who are you?.
Why do you wanna come into my house when I'm not around?.
Like, who are you?.
Tell me about, and literally, like he was kind of aggressive.
he was like, how do I know you're not a freak?.
How do I know that you're gonna come into this house.
and you've just written a nice letter.
to work your way into the home.
and you're gonna do something nasty in our house?.
Like, this is terrible, who are you?.
So basically, he's asking for like a LinkedIn live.
kind of like go through my whole history with him, right?.
So I'm like, well, I was born in the UK.

$^{161}$and I started to go through my history..
That wasn't good enough..
He wanted to know every place that I'd ever worked.
and what position I had held in the places where I worked..
It was like I was being interviewed for the CIA..
So I'm going like, okay, sure..
So, well, I originally worked at this headhunting company.
and I did that for a while.
and then I moved to an American investment bank.
called Morgan Stanley and I worked there for a while..
He's like, what did you just say?.
And I said, oh, I worked for a US investment bank.
called Morgan Stanley, I worked there..
He's like, you worked for Morgan Stanley?.
I said, yeah..
He said, I've worked for Morgan Stanley for 38 years..
And he said, if you've worked for Morgan Stanley,.
if you're good enough to work for Morgan Stanley,.
you're good enough to come into my house..
(audience laughing).
So the next day, 10 o'clock, knock on the door..
The wife welcomes us into my childhood home..
And I can't really describe the emotions I felt.
as I walked through that house..
We got this cheesy family photo outside..
The one on the top there is my mom.
when she was super young with her mother.
and there's my mom and the rest of our family.
and my daughter there outside the front..
As we walked through every single room,.
I just had wave after wave of emotion.
and memory of my father..
I remember connecting with him deeply as I was walking,.
but I also sensed something else.
that I didn't expect to sense..
And that was a sense of closure..
I actually felt like my father was with me.
walking through the house and saying,.
"This was what you had..
"And this is what we had..

$^{201}$"And it was a good thing..
"But now you have a daughter of your own.
"and you have future core memories.
"that you need to give to her.".
And I felt like my father was like passing a baton to me.
as I was walking through the home..
And I realized that in that moment,.
there was a shift that was happening within me..
And this house meant so much to me,.
so many of my core memories,.
so many of my memories as a child to a father..
And I felt like what my father was encouraging me to do.
as I walked through the house was to shift my mentality.
and my identity to thinking about myself.
not as a child to a father in this home,.
but now as a father to my child, Mia..
And that shifting point in identity.
was one of the key reasons why I was able.
to begin to move through the morning.
that I was feeling at this time..
I remember leaving that house and getting in the car.
and feeling very emotional, shedding a few tears,.
thinking to myself, "I'm a father and I have a daughter.
"and I get to create such great memories..
"And I wanna live my father's legacy.
"by now shifting away from just thinking about being his son.
"to being a father myself.".
As we come to the halfway point of Exodus.
and as we come to chapters 19 and 20.
that we're looking at this week and next week,.
this idea of a shift in identity.
is what these two chapters are all about..
These are the chapters that capture for us.
Moses and Israel returning to the very place.
where Moses had been some 35, 40 odd years earlier,.
a place that held for him so many great core memories,.
the very place where he had met with God 40 years before.
in the burning bush.
and had such a profound encounter with God..
He's now returned to the same place, Mount Sinai..

$^{241}$And this Mount Sinai was a little bit like me.
returning to that house in Connecticut..
It was a place of deep connection for him..
Could we put up the Mount Sinai picture, please?.
There you go..
It was a place of connection for Moses..
It was a place where he could find himself.
with God once again, because prior, 40 years ago,.
that's what had happened for him..
That time in the burning bush, Moses was a broken man,.
a broken man struggling with his identity,.
wondering, am I actually a prince of Egypt.
or am I an Israelite?.
And he's stuck between these two identities.
that were causing him a lot of tension..
He had just murdered an Egyptian.
and fled from that Egyptian..
And now here he is in this mountain.
and God appears in a fire on a bush.
and calls him towards him..
Says, "Take up your sandals.".
The ground, he's standing on this holy ground..
And God begins to speak to him of a new identity..
Begins to say that I've got a purpose.
and a value and something for your life..
And Moses begins to realize.
that his life has not come to an end here..
There's a new beginning..
And he begins to receive the call from God.
to go back to Egypt,.
the very place that he did not want to go to,.
and actually say to Pharaoh, "Let my people go,".
because God wants them to come and to be with him..
And as God's giving him this vision.
and this idea of his identity,.
Moses is pushing back on it..
And I think we all do this as Christians..
When God speaks to us of calling and vision and identity,.
it's very easy for us to push back and say,.
"Well, I can't do that..

$^{281}$"That can't be me..
"I'm not like that..
"I don't have those gifts..
"I don't have those skills.".
And Moses is doing this before God..
And God, because he's a father,.
gives a fatherly promise to Moses at this time..
This is in chapter three..
We've looked at this before,.
but let me just remind you of this..
In chapter three, verse 12,.
and God says this to Moses, "I will be with you..
"And this will be a sign to you.
"that it is I who has sent you..
"When you have brought the people out of Egypt,.
"you will worship God on this mountain.".
In other words, you're gonna return back.
to this very place with a free people..
And at that time, Moses was like,.
"I don't even know if that's ever gonna happen.".
And now, as we're in chapter 19,.
the very thing that Moses believed would never happen.
has happened..
He and Israel return to the mountain to worship God..
Let me read to you how actually Moses talks about it.
at the start of chapter 19..
"In the third month, after the Israelites left Egypt,.
"on the very day,".
notice this, Moses is being very personal here..
He's like, "I remember the very day.
"they came to the desert of Sinai..
"After they set out from Rephidim,.
"they entered the desert of Sinai,.
"and Israel camped there in the desert.
"in front of the mountain.".
I wonder if you could sense the emotions.
that Moses must've been feeling..
Like me, walking through the house.
and all those emotions of those connections.
to core memories..

$^{321}$Now, here's Moses standing at the mountain.
with all of Israel with him,.
and all those core memories come back..
And all that God had promised.
and all that God said he would do has actually happened..
And I can imagine that Moses is standing at the mountain.
full of joy, full of wonder, full of amazement.
that God has done what he said he was gonna do,.
that his people are free and his people have gathered..
And you could sense Moses welling up in worship..
The very thing that God has said,.
"You're gonna return to this mountain and worship me.".
There is Moses going, "I can't believe it, but we're here.".
This is personal for Moses,.
because he has gone from a place of a broken identity.
to a place of a leader of his people..
And as he's standing at that mountain,.
I'm sure he's saying, "Only because of you, God.".
This personal nature for Moses with Mount Sinai.
was what I wanted to try and capture for you.
as we actually did a film for this particular week..
And when I was talking to the crew some four years ago.
and we were planning out every film,.
I was like, "Wouldn't it be awesome.
if we actually did the hike up Mount Sinai ourself?".
Now, Moses, when he did it, there was no path there at all..
And these days, there's a very well-worn path..
And if you're fit, you can actually do.
the about four and a half hour hike.
from the base of Mount Sinai up to the top..
If you're not fit, you can get a camel.
to take you up three quarters of the way,.
but then you still gotta do about a quarter of it..
And we thought it'd be fantastic..
Why don't we film this?.
And unlike the other films in the series.
where we have scripts and we have it all planned out,.
this one, we're just gonna make it pretty raw..
Because for Moses, it would have been raw..
It would have been this kind of pilgrimage.

$^{361}$for him to go up a mountain..
I said, "Why don't I go on a pilgrimage up the mountain?.
Let's film it..
Let's not script it..
Let's just see what happens..
Let's capture it for everybody.".
And hopefully in that,.
help people to understand the personal nature.
of what chapters 19 and 20 are all about,.
but also the personal nature.
of what this part of the Exodus story has to say to you..
So let's take a look at my hike..
Exodus chapter 19 is perhaps the most pivotal chapter.
in the whole of the Exodus story..
God calls Moses up to the top of Mount Sinai.
so he can meet with him,.
commune face to face,.
and ultimately give him the 10 commandments.
that will shape God's people forever..
Now, ever since I was a kid.
and came to faith in my early 20s,.
I've always had this deep desire in me.
to follow in Moses' footsteps,.
to actually do the hike myself up Mount Sinai.
and kind of experience for myself.
what it would be like to meet with God on the mountain..
Well, today, actually, the dream comes into reality..
I'm about to hike up Mount Sinai here,.
and there are some camels right here.
that are gonna go right past me..
Watch this, watch this..
This is a real thing..
Go..
(gentle music).
There you go..
That's how it works here..
All right, anyway, so I just stepped in the camel poo too..
But anyway, the camels are gonna go ahead of us..
I'm gonna hike up to Mount Sinai right now,.
and I'm really excited..

$^{401}$It's about just before midnight..
It's super cold..
It's gonna be a little bit windy,.
but I'm really excited to do this,.
and I can't wait 'cause I'm gonna bring you along.
with me on the journey..
Let's go..
(gentle music).
As I take the first few steps here.
past St. Catherine's Monastery,.
yeah, I'm reflecting on kind of the journey already..
It's been four years of the Exodus Project.
to get to this point,.
and it's taken years of study and research and preparation,.
and then COVID delayed it,.
and then now I'm finally here..
I've dreamt about this hike for so long,.
and it makes me think about our Christian journey, actually,.
that while study and research of Scripture.
is, of course, important,.
it's actually the doing that really counts..
It's actually putting what you've learned into practice.
that matters the most,.
and every single step I'm taking.
is a little bit kind of a metaphor on my own Christian life.
in the sense that every single day.
I have to take that next step in my journey with Jesus,.
and it's a practice of faith..
It's not just a research of faith,.
and as I start out on the trip,.
I'm reflecting a little bit about Moses's journey himself..
I mean, he didn't have a nice path to walk on like I do..
He would have had to scramble up all the rocks.
to get to the summit,.
and I'm sure at the beginning point,.
Moses was thinking to himself,.
"Why doesn't God come and meet me down here?.
"Why do I have to go up there to meet with God?".
I mean, God had met Moses on a flat plain.
in a burning bush in the past..

$^{441}$Why not now?.
And I think God was trying to teach Moses something..
He was trying to let him know.
that actually the relationship he's forming in Israel.
is gonna require obedience, faithfulness..
It's gonna require them to sacrifice..
It's gonna be a hard, narrow path..
Jesus, many years, would reflect the same..
He would say, "If you wanna follow me,.
"you're gonna have to pick up your cross..
"There's gonna be sacrifice..
"There's gonna be the need for obedience.".
And I think the journey up the mountain.
was really part of the formation for Moses..
Here's another thought..
You know, it's so dark here as we've been walking.
that I've been tempted to pull out a flashlight.
and use some artificial light to help me to walk..
And I think we do that in our sin as well..
When we're faced with the darkness of our sin,.
we try to throw some artificial light over it.
to make ourselves feel better about it,.
to try to pretend that it's no big deal,.
or maybe convince ourselves.
that we're gonna deal with it and get over it easily,.
when the reality is it's deeply entangled in us..
See, when you hike in a place like this,.
it's actually best not to use artificial light..
It's best just to walk in the darkness,.
to actually get accustomed to the darkness,.
to allow your eyes to realize.
the sort of optical illusion that is there.
when you use artificial light..
And I think it's the same for our sin..
When we put away those artificial lights.
that try to make ourselves feel better about it,.
and actually just sit in the actual reality.
of the darkness of sin,.
it enables us to know where we're starting.
when it comes to repentance..

$^{481}$And I think that's such an important part.
of the Christian journey..
I wanna encourage you to put away your artificial lights,.
and actually allow yourself to realize.
where you are in the sin that you're struggling with..
And from that point, God can bring his real light..
It's not artificial light when it comes to Christ..
It's light that brings us into freedom..
So we've been walking for about 40, 45 minutes now..
That's St. Catherine's in the village down there..
Our path is up here..
All right, I'm here at the halfway house of the mountain..
I'm literally about halfway up now..
And this is such a nice place to rest..
I mean, it's so comfortable here..
There's a beautiful fire..
And I tell you what, it's quite cold on the mountain..
So I've been here about 10 minutes now,.
and it's really comfortable..
Had a cup of tea..
And I've been reflecting that so often,.
this is like our spiritual journey..
God calls us to a destination..
And it's a difficult journey up the hill..
And it takes a lot of courage and a lot of sacrifice..
And we get about halfway,.
and we're tempted to settle there in our spiritual lives,.
in our spiritual halfway house..
And we kind of convince ourselves,.
oh, yeah, God called me up there,.
but I've already sacrificed so much,.
and I'm happy where I am right now..
And we kind of settle into this kind of halfway house.
mediocrity of our spiritual lives,.
knowing that Christ is calling us deeper,.
but happy just to settle where we are..
And I think that's the great temptation.
of all of our spiritual journeys..
And for me right now, I'd be happy to tell you,.
oh yeah, I made it to the top of the mountain,.

$^{521}$but kind of fake it and just stay here.
and then go back down the hill..
But that's not really what it is..
God has called us to be at the top,.
and we need to keep pressing into our relationship with Him..
So I'm gonna rest a little bit more,.
and then I'm gonna pick myself up, get going again..
And I wanna encourage you to do that.
in your spiritual life too..
If you feel like you've kind of rested,.
if you feel like you've sort of settled.
into mediocrity in your spirituality,.
can I encourage you, get up, get going again..
There's so much more that's ahead..
(gentle music).
As they say in Egypt, yalla..
Yalla, all right, let's go..
All right, we're about like, I don't know,.
maybe like three quarters of the way up now..
And like this is Mount Sinai right here..
This is the rock, you know what I mean?.
Like it's really actually beginning to feel.
a little bit like I think what Moses would have seen..
I think the earlier part of the walk,.
it felt like I was in some national park.
anywhere in the world..
But now when you see the kind of color here,.
you may not be able to see the color 'cause it's dark,.
but the actual shape and the feel,.
it's like this is kind of what I had in mind..
And it's really special, really special..
Mount Sinai..
(laughs).
I'm about three hours into the hike now.
and I'm on a staircase..
It's a rock staircase..
It's about 750 stairs now to the summit..
So I'm not actually that far from the summit,.
but this is the hardest part of the hike..
I'm out of breath..

$^{561}$The stairs are tall and steep..
And it's got me to reflect a little bit.
about our spiritual lives again, right?.
So often when we get close to a goal.
that Christ has put in our hearts, things get harder..
You know, we often think in a climb up the mountain.
that the hardest part is at the start..
It's not true..
The steepest part of the mountain.
is always right before the summit..
And it's the same with us in our spiritual lives.
with the goals that Christ gives us..
The enemy loves to come in right at the end.
when we've almost achieved the goal.
and try to knock us back again,.
try to make it so hard that we'll quit.
right at the finish line..
And I'm kind of feeling like that right now..
I'm like, man, I don't know if I've got much more energy.
to get up there..
And that's when we need to press in again..
We need to push through that barrier and keep going..
And I imagine Moses,.
as he's at this part of the mountain as well,.
he's thinking like, it's not that far, but it's super hard..
I think the hardest part of the climb.
is ahead of us right now..
And yeah, so we've got to push in..
Actually, there are actually about 3,750 steps.
from the very base of the mountain to the summit..
And the monks that live here,.
they call it the staircase of repentance.
because of the pain involved in getting up to the top..
And I have to say, now that I'm almost there,.
these stairs are very well-named..
(gentle music).
Well, after nearly four hours of hiking,.
I finally made it to the summit..
You can't see much right now 'cause it's still night..
It's still dark..

$^{601}$Ah, man, what a trip..
That was incredible..
Super hard, but I'm so glad to be here..
I'm at 2,285 meters..
That's how high we are right now..
And in a few moments, the sun will rise up behind me..
And you know, as I get here in the darkness,.
it makes me think about Moses.
when he first arrived on the summit..
You know, he must've had this mixture.
of excitement and anticipation,.
but I would also think probably a little bit.
of anxiety and fear as he's about to walk into the fog,.
you know, the presence of God.
that was on the mountain at the time..
And it must've been just this awesome sense of God here..
And you know, the fear in him must've been very palpable..
And you know, it's a significant moment..
The courage of Moses to not just do the walk up here,.
but then to walk into the presence of God.
and literally receive the law.
that would come to reshape the Jewish people.
and who they are and really change all of history..
That happened right here..
And as I sit here in this moment,.
having finally achieved my dream,.
I'm really quite overwhelmed by the profundity.
of what happened right here..
(soft music).
There's a section in Exodus 19.
where in the Hebrew it reads this..
It says, "Moses comes up to the top of the mountain.
"and then is on the top of the mountain.".
Now in the Hebrew, it repeats top of the mountain twice.
in the same sentence, and it has confounded scholars..
But actually here is what it means..
It means that for Moses,.
this experience of meeting God required two things..
There was the coming up to the top of the mountain..
That was the doing..

$^{641}$But then when he was on the summit, like I am right now,.
he had to then be with God..
That was the second part of being.
on the top of the mountain..
So Moses was here and he needed to settle.
in the presence of God, be with him,.
stop his doing and just rest..
And now that I'm here on the summit with the sun coming up,.
that's exactly what I wanna do..
So I'm gonna wrap this up here..
I'm so grateful that you were on the journey with me.
up the mountain, but now I'm just gonna spend some time.
to be with God..
(soft music).
It is this idea of being and not just doing.
that Exodus 19 and 20 are all about..
Because it is now here where God is gonna reshape.
their identity as a people..
It's been again to help them to understand.
who they truly are..
And I wanna just unpack a little bit about that here,.
if you're willing to give me about another 10, 15 minutes..
Is that all right?.
Everybody handle that?.
You didn't do the hike, I did the hike, say yes..
(congregation laughing).
Good..
I did it for you..
No, it's good, okay..
Verse three says, "And Moses went up to God.
"and the Lord called to him from the mountain and said,.
"this is what you are to say to the house of Jacob.
"and what you are to tell the people of Israel..
"You yourselves have seen what I did to Egypt.
"and how I carried you on eagles wings.
"and brought you to myself.".
I love it..
The first thing God does is he basically does.
like previously on the Exodus, right?.
And he takes them on this kind of journey.

$^{681}$of what's already happened..
And he says, "You yourselves have seen what I did in Egypt..
"You understood all the things that I've done,.
"how I revealed myself to you,.
"how I showed you my power and my might,.
"how I was compassionate for you..
"You saw the plagues, you saw the parting of the Red Sea,.
"you saw the manifold from heaven..
"You saw all of that and you understand that.
"because you know me..
"I've revealed myself to you in all of that.".
He says, "I carried you on eagles wings,.
"the idea I'm intimately caring for you..
"I stepped in, you didn't do any of that.
"in your own strength..
"I did all that for you.".
God is saying, "I carried you.
"on these intimate eagles wings.".
And he said, "I brought you to myself.".
Notice this, he doesn't say, "I brought you to freedom.".
He doesn't say, "I brought you to success.".
He doesn't say, "I achieved all your dreams for you.".
He said, "What I've done for you is I freed you.
"in order that you would find a home in me,.
"that you would come to myself.".
You know, the greatest gift.
that God will ever give you is himself..
Come on church, it is himself..
It's not always the things that you're dreaming for..
It's not always the things you're praying for..
It's not even sometimes you're healing..
It's not even sometimes that restoration you want..
The greatest gift that he'll give.
and he wants to heal you, he wants to restore you,.
he wants to do all those things,.
but the greatest thing he will ever do for you.
is show himself to you, is welcome you into his presence.
so that you can have an intimate relationship with you..
And he said, "I've done this..
"I've carried you guys on my wings..

$^{721}$"I've brought you to myself.".
And he starts by saying,.
"You yourselves have experienced all this.".
This is the only time in the Hebrew.
in the whole of the Old Testament.
where the words you is repeated twice side by side..
So in English, we say you yourselves,.
but in the Hebrew, it's you, you..
What he's saying is you corporately know what's happened,.
but you individually also know what's happened..
In other words, you have a personal encounter.
and you've personally witnessed.
all the things that I've done.
from the time that you were in slavery in Egypt.
all the way up to now..
It's like God is basically saying this to Israel..
He's saying, "You know me, but do you know who you are?".
You know who I am 'cause I've done all these things..
I've revealed myself to you.
and you've seen it all personally, corporately in person..
You know who I am, but do you know who you are?.
He's challenging them..
Do you have an understanding of your identity?.
Do you know what it is that I've saved you for?.
Do you know what it is that I've rescued you for?.
Do you know what your purpose is, your value is,.
your mission is?.
Do you understand what it is that I've done in you?.
And the reality is that God had revealed himself.
to Moses at the burning bush, to Pharaoh,.
to Egypt and to Israel..
He's brought them all to this mountain,.
but the story doesn't finish here..
Chapter 19 would be a nice place.
to wrap up the story, wouldn't it?.
Maybe the Exodus could have been just chapters one to 19,.
and that would have been a nice bow on the whole story.
'cause we would have seen God's deliverance..
But if that had happened,.
all we would have learned is who God is..

$^{761}$And what God wants to do at this part of the story.
and for the next 20 chapters is to tell you who you are,.
to help Israel understand who they are.
in their identity with him.
and to help you understand who you are.
in your identity with Christ..
That's what the rest of the Exodus story is all about..
See, our journey in Exodus is not just a journey.
in a God who makes himself known,.
it's a journey in a God who makes ourselves known..
And some of you here,.
you've experienced a deliverance out of some sin.
that you've been perhaps enslaved to,.
but you're still struggling to know who you are..
You're still struggling to know what your value is,.
what your mission is, what your calling in life is..
The number one question I get as a pastor,.
"Pastor Andrew, what is my calling in life?".
Well, if you give me another 10 minutes,.
I'll tell you what your calling in life is.
because this is what Israel is crying out and wondering..
We're free now, but what are we gonna do?.
Who are we?.
And what is God gonna do in and through us?.
And so God begins to unpack the idea of what he's gonna do..
You need to understand that the number one thing.
that slavery does, it strips your identity from you..
The whole definition of slavery is designed.
to strip from you your value and who you are.
and how you understand yourself..
I've actually had, I wouldn't call it the privilege,.
but I've actually sat with traffickers in Bangkok..
I was working with an NGO at the time.
and we were connected with traffickers, human traffickers..
And I met with this trafficker and he explained to me.
that the number one priority they have.
when they take women is to so disempower.
and strip them of their identity.
that they basically think they have no worth.
and no value so they won't fight back..

$^{801}$And that should shock us..
But that's what the enemy does with every form of slavery..
Any sin that we are enslaved through.
is designed to strip you of your identity.
and how you've been created in the image of God.
and how God sees you and to strip your value.
and how you understand yourself..
And so when you get free from a sin,.
when God heals you and you bring your sin to him.
and you repent of it and he washes you white as snow,.
it's a wonderful thing,.
but there's another step that should happen after that..
'Cause not only is it one thing to be taken out of slavery,.
it's another thing to then discover who we now truly are..
Israel arrives at Mount Sinai.
and they have no idea who they are..
Are you with this?.
And when we come out of our sin,.
we have to remember that we have lost.
something of our identity because of that sin..
The enemy comes to kill and destroy,.
but Jesus comes to give life and life abundant..
See, one of the very core ideas of the whole Exodus.
is that you're not just saved from something,.
you're saved to something..
And this is where Israel is..
And God is saying, "Do you know why I've saved you?.
Would you like to know why?.
No? Okay, let's go..
Would you like to know why you're saved?.
Why you have a calling?.
Why God wants you free to himself.
so that you can live for him?.
Let me show you what God says next..
Verse five..
Now, if you obey me fully and keep my covenant,.
then out of all the nations,.
you will be my treasured possessions..
Although the whole earth is mine,.
you will be for me a kingdom of priests and a holy nation..

$^{841}$These are the words that you are to speak to the Israelites..
In a moment where Israel is going, "Who are we?".
Because we don't know who we are anymore.
because in slavery, our whole identity has been submerged..
We've been disempowered..
We've been stripped of who we are..
They're now standing at Mount Sinai.
and they're saying, "Who are we?".
And the very first thing God says to Moses on the mountain.
is, "This is who you are.".
And for some of you in the room,.
what I'm about to tell you.
is the most singular important thing for you.
to truly grow in Christ Jesus..
Because what God says in this moment.
helps to unpack so much for Israel.
and I think so much for you..
Let me break down what it is that God does..
Is this helpful for anyone?.
Okay, let me break down for you.
exactly what God is saying about....
So first of all, God has made himself known..
That's the first thing he tells Israel..
He reminds them..
Oh, hello..
(mimics horn).
This is not gonna go well, is it?.
Okay, here..
(sings).
All right, give me one second..
I'm gonna fix this properly..
Oh no, it's getting worse..
It's really worth it, trust me, just wait..
Hallelujah..
So God has made himself known, remember?.
Through the whole of the Exodus journey..
And in doing that, what he's done is he's delivered.
his people..
Now, what you need to understand.
is that in his delivery of his people,.

$^{881}$he makes himself known as well..
So first of all, God made himself known.
to Moses at burning bush..
He delivers his people and in delivering his people,.
he further makes himself known..
Does that make sense to you?.
So they come to understand his character, his nature,.
by the fact that he has delivered his people, right?.
Then what God says on Mount Sinai in this moment,.
when he says, he says, I'm creating for you.
what's called this covenantal relationship..
Now, covenantal relationship..
Now, I'm gonna come back to this more next week.
when we have a bit more time,.
but what covenantal relationship is.
is what's about to happen on the top of Mount Sinai.
with the giving of the law and all of that..
God's saying, I've made myself known, I delivered you..
Now, out of that, I'm gonna create.
a covenantal relationship with you.
so that you can truly understand who you are..
And in this passage, he says two things for them.
about who they are..
He says, first of all, you have to understand.
that you are what he calls priests..
And he says, you gotta understand.
that you are also now called to be a holy nation..
You are priests and you are holy..
These are the fundamental parts.
of who we are in Christ Jesus..
If you're wondering, who am I?.
What value do I have?.
What input can I make in the world?.
You are a priest and you are holy..
Let me break these down real quick..
Priests in the Old Testament,.
when God speaks to Israel about them being a priest,.
priests had three primary functions..
Those functions were intercession, so praying,.
they were invitation, helping people to come.

$^{921}$into the presence of God, and they were imitation,.
helping to show the people God's character.
by how they live their lives..
These three things, intercession, invitation,.
and imitation is what it meant to be a priest..
Priests were the mediators between God and the world..
They would help to connect the world.
and God together through prayer..
They would pray on behalf of the world..
What we just did a moment ago.
when we prayed for the situation in the Middle East,.
we were priests together..
We were connecting God and the world together.
and saying there are needs and making that connection..
We are to invite..
Priests set the stage for people.
to encounter their worship of God..
Christians are designed to invite people.
into the presence of God..
That's what we're called to do..
Imitation, the priest, we're called to show the world.
God's character by how they live their lives..
So we as Christians are called to reveal the character.
and nature of God to everybody through how we live our lives..
Intercession, invitation, imitation is who we're called to be.
but we're also called by God as holy..
And this speaks of two areas, transformation,.
I can't spell, transformation and being set apart..
It is the Holy Spirit that transforms us.
from our brokenness and our sin.
into the life that he wants us to be..
Without his spirit at work in us, in our culture,.
in our society, there is no transformation..
Transformation towards becoming more like Christ Jesus.
is always driven by his spirit..
So when he says, I call you holy, I bless you to be holy,.
he's inviting us into a transformative journey.
in who we are..
He sends you..
And he says, you are to be set apart..

$^{961}$Being holy is set apart,.
which means you should look different, be different,.
have different values to everything else.
you see in the world..
He doesn't mean set apart in the sense that.
go up into a hill somewhere, be in a monastery.
and just act weird, okay?.
We're not supposed to be weird Christians.
but we are supposed to be set apart..
We're all supposed to have different values.
and different purposes and different hope..
When people around us feel like they've got no hope,.
we stand up and go, there is hope..
We have a different culture, a different thinking,.
a different narrative..
And God's saying, I've created you to be set apart.
in the world, but not of the world..
I didn't call you to be out of the world.
and not of the world..
I called you to be in the world, but not of the world..
'Cause if you're in the world,.
but you're holding a different narrative,.
you're holding a different hope,.
you're holding a different lifestyle,.
you're holding different values,.
you begin to imitate my character to the world around you..
Now more than ever here in Hong Kong,.
the church has to be holy..
Not pious, close the doors, just worship Jesus on our own.
and not talk to anybody..
Holy in the sense that we're set apart.
in the city that we're planted in..
Carrying a different kind of narrative.
to the one that the world is currently hearing..
Are you with me?.
Anyone else passionate about that?.
And what God is saying is in these two things,.
being holy and being priests,.
we find our calling and our mission..
And next week, as we look at chapter 20,.

$^{1001}$we'll learn a lot more about what that calling.
and that mission looks like..
But here as God speaks to them about identity change,.
he's saying, "Know who you are.".
You know who I am, do you know who you are?.
Paul would write this, he would say,.
"For we have been set free, now be free.".
What he's saying is now live as God has called you to live..
Peter, as he's pastoring the first church,.
he's trying to think,.
"How do I help them to understand who they are?.
How do I help them understand.
that they're not just being saved and that's it?".
But there's so much more to the story..
And so interestingly, Peter writes to the church.
and he actually quotes from this very chapter in Exodus 19..
Let me read this to you..
He writes this to a Jew and Gentile church..
He says, "But you are a chosen people,.
a royal priesthood, a holy nation,.
a people belonging to God that you might declare.".
Notice this, "You are a holy people..
You've been chosen by God..
You are a holy people..
You are a nation of priests, so that you might declare.".
You have a calling and mission..
"So that you might declare," what he says here is,.
"The light of God..
You would declare the praises of him.
who called you out of darkness into his wonderful light..
That you would be a mouthpiece, a voice piece.
for the incredible work that God does.
in and through us as people.".
This, my friends, is who you are..
You wanna know what your calling is?.
To be a priest, to be holy,.
so that you can declare the praises of him.
who's called you out of darkness into his glory light..
You know what you should be dedicating your life to?.
So that you could be a priest and you could be holy.

$^{1041}$so that you could call out of the darkness.
the light of God and give him praise for who you are..
That's his invitation to you..
Are you with me?.
If that doesn't inspire you, I don't know what will..
Like if that doesn't set your heart on fire,.
if that doesn't recalibrate how you think about work,.
if that doesn't change the way that you think.
when you go into the office tomorrow.
or when you raise your kids.
or when you think about your marriage.
or you think about your future,.
if this idea of being God's mouthpiece.
and his voice and his heart to being in the world.
but not of it, to bring hope where there is no hope,.
if that doesn't inspire you, I don't know what will.
'cause this is the gospel..
If Exodus had finished at chapter 19,.
we wouldn't have gotten any of that..
We would've just come to know him..
But Exodus 19 onwards is God now saying, get to know you..
What we're gonna be doing over the next number of weeks.
here at the Vine is helping you to live.
in the freedom that God has given you, amen..
Could you stand?.
I'd love to pray for you..
Father, we're just so grateful for the men, women,.
children and families that are here in this moment..
And Father, we stand before you as a chosen people..
We stand before you as a royal priesthood,.
as a holy people..
Sometimes we don't feel that way..
If we're honest, sometimes we don't act that way..
It's very easy in our slavery to sin,.
to discount who it is that we've actually.
been created to be..
And so a free Israel comes to Mount Sinai.
and God appears and he says, now the real journey begins..
(gentle music).
It's so important that we understand.

$^{1081}$that in the story of Exodus,.
it's not just a story of God's power.
in delivering a people without freedom..
It's a story of God's power in reviving a people.
without identity..
And some of you in this room,.
you're a Christian, you love Jesus..
He saved you, you're free,.
but you struggle to know who you truly are..
And if that's you, I wanna encourage you this morning.
just to receive from him..
Receive his presence with you..
Receive his voice over you..
You have value..
And your sin and your brokenness is stripped.
so much of how you see yourself, but you have value..
I see you as a priest..
I see you as the very one that can mediate.
between the world and God..
Some of you need to know,.
I feel this for this service in particular,.
some of you in this room,.
because of the slavery of sin that you've been in,.
you felt anything but holy..
And we're gonna learn next week a lot more.
about what holiness really is..
And I'm unpack that for you next week,.
but right now in this moment,.
some of you just need to know you are holy..
You need to know that that's how God sees you..
That's the identity he brings to you..
That in his presence,.
we are washed clean, we are holy..
I feel like the enemy has stolen that thought.
from some of you in this room..
And because of that, you've held back in your decisions.
and your dreams and your hopes..
You felt like God could never really use me.
because of X, Y, and Z in my past..
Israel comes to Mount Sinai.

$^{1121}$with all of that baggage of their past,.
and God is about to tell them who they truly are..
And I pray in this moment.
and over the next number of weeks,.
you will hear who it is that you are..
And for some of you in this room,.
God is saying, it's time to go..
Perhaps you know who you are,.
but for whatever reason,.
you've kind of camped at Mount Sinai..
It's really important..
Israel was at Mount Sinai for one year,.
a long time, 'cause they needed to know who they were,.
but they weren't camped at Mount Sinai forever..
Some of you, you're always seeking.
the mountaintop experience,.
and that's important in seasons of our life..
But for some of you, it's time to come down the mountain,.
and it's time to get into the valley,.
and it's time to see the goodness of the Lord.
in the land of the living,.
not just on the top of the mountain..
Some of you are receiving a calling today to go..
Holy Spirit, whatever it might be for whoever here,.
I just pray that you would come..
Maybe just open your hands if you're comfortable..
I'm just gonna invite the team.
to come and just worship over us..
And as they worship over us,.
just allow yourself to be ministered to.
in whatever way that God wants to speak to you today..
(gentle music).
[MUSIC PLAYING].
\newpage



\section{}
\label{sec:FFFAigIMmRc}
\textbf{2023-10-24 Exodus - 19 The Path of Life [FFFAigIMmRc].mp3}
\newline
\newline
連結: \href{https://youtube.com/watch?v=FFFAigIMmRc}{\texttt{ https://youtube.com/watch?v=FFFAigIMmRc}} ~~~~ 語音日期: 2023-10-24 
\newline
\newline
\hyperref[sec:CKM0h5f9Oa4]{\small{< < < PREV SERMON < < <}}
~
\hyperref[sec:index]{\small{[返主目錄]}}
~
\hyperref[sec:5UVcdfy_Atk]{\small{> > > NEXT SERMON > > >}}
\newline
\newline
$^{1}$- Amen, hey, can we thank our worship team.
for just the way they serve..
Have a seat, have a seat..
Welcome to the Vine..
If you're relatively new with us,.
my name's Andrew, I'm one of the pastors here..
I'm so glad that you're a part of this..
And we've been in a long series in the book of Exodus..
And last week we got to the midpoint.
of the whole of the book of Exodus, chapters 19 and 20..
And in chapters 19 and 20, we see a very important moment.
that Israel, who have been now set free.
from the slavery of their past,.
they now come to God's presence at Mount Sinai..
And as they come to God's presence at Mount Sinai,.
they come into that presence and they realize.
that all that God has done for them.
is to bring them to himself..
And just like we've been experiencing in worship.
and just praying just a moment ago.
that God has done all of that work of liberation,.
who's come and done the plagues and provided for his people.
and brought them into his presence,.
because in his presence, he wants to now speak to them..
And what we saw last week was something fascinating.
that although Israel was now free.
and were at the presence of God in the mountain,.
at Mount Sinai, they had a serious issue..
And their serious issue was,.
despite the reality that they were free,.
they had no idea who they were..
And God comes to them and says, I've shown you who I am..
Now it's time for you to discover who you are..
And we saw last week that the primary thing.
that slavery does to all of us,.
whether that's slavery to sin, slavery to our brokenness,.
slavery to our past, whatever it is,.
whatever it is that is enslaving our heart,.
enslaving our minds, causing us anxiety and stress,.
whatever that is,.

$^{41}$it is designed to strip your identity from you..
The very definition of slavery,.
when people are trying to enslave people,.
what they try to do is strip the identity.
of that person away from them..
So they have no choice..
They have no freedom of expression,.
their freedom of thought..
They're forced to do just what the other person.
tells them to do..
And Israel has had that for 200 years..
And now they've come to Mount Sinai..
And they're like, we have no idea who we are.
because everything has been stripped from us..
And we said last week that that's the reality.
for so many of us in this room as well..
That just freeing ourselves from sin.
as Christ does on the cross.
means that we're now in a identity vacuum..
And we're now trying to go, okay, who am I now to be?.
I know I'm free, but what am I being created free to do?.
And everything in chapters 19 and 20 of Exodus.
is God coming to his people and going,.
this is what you're free for..
This is now who you are..
This is how I now want you to live..
And he speaks to them about his character.
that he's placed in them..
And the first thing that happens, as we saw last week.
with Moses on Mount Sinai in chapter 19,.
is God says, here are two things.
that you need to know about yourself..
You're a kingdom of priests and you're a holy nation..
And this dual idea of being a kingdom of priests.
and a holy nation forms the foundation of Israel's identity.
but also our identity as well..
When God speaks about being this idea.
of a kingdom of priests,.
the priest really had three functions..
It was to connect God and the world together..

$^{81}$They were the ones that interceded and stood in the gap.
between God and the world..
All of the things that they could see happening in the world.
and they know of God's heart,.
they wanted to see those two things come together..
The second thing that the priest would do.
is that they would invite people.
to come and experience the presence of God..
They would create an environment.
by which people could encounter his presence,.
knowing that if they could just encounter.
the presence of God, they would be changed..
And then the third thing that the priest did.
is to help them to understand,.
not only is it about being kind of invited.
into the presence of God,.
but now we're called to imitate God's character in the world..
The priests were imitators of the character of God.
so that they could show everyone in the world.
what God's character and his heart.
and his love actually looked like..
So when God says to Moses, here's who you are,.
you're a kingdom of priests,.
he's saying, you're a bunch of people.
that can connect people to God,.
invite people into his presence,.
and actually come and do that in a way.
that imitates the life and the glory.
and the love and the character of God..
And then he says, you're a holy nation..
We saw last week that that meant two primary things..
It's the idea of being transformed.
by the power of the spirit of God in us..
And when we're transformed,.
we are able to deal with some of that slavery.
that we carry around with us..
Some of the faulty thinking that is in our mind,.
some of the core false beliefs that we hold onto,.
we're transformed, as the scripture says,.
by the renewing of our mind..

$^{121}$The spirit of God transforms us.
both personally and communally together..
But being a holy nation was also about being set apart..
It was about being set apart for the world,.
not set apart like be weird in the world,.
but set apart for the world..
Not to be a monastery up on a hill that kind of goes,.
oh, I don't wanna be in that mucky thing called the world..
Actually, God saves us, creates the church,.
'cause he wants the church planted in the heart of society,.
in the heart of the world,.
and that we would be a community of hope.
wrapped around a community of pain..
That we would be ones who would say,.
yeah, we're set apart because we wanna offer the world.
an alternative narrative..
That where things are stressed and anxious,.
we can offer peace and hope..
When people are angry and violent, we can show another way..
Where we can live in such a way.
that we take the character of God,.
we bring it down to this world,.
and we say there is another way to live..
This is all chapter 19..
And God says to Israel,.
this is who you need to know who you are,.
because I've set you free for a purpose..
But the culmination of this comes in chapter 20.
that we're looking at today..
Because in chapter 20, God takes it a step further..
He's laid that foundation of priests and holiness,.
but in his presence, he takes it a step further..
And what he basically says is, look,.
I've revealed myself through how I've acted amongst you..
You've seen what I've done,.
and that helps you understand who I am..
Now I'm gonna speak to you about your actions..
I'm gonna speak to you about.
how you should interact with one another,.
how you should walk with each other,.

$^{161}$how you should be in relationship with me,.
relationship with yourself,.
relationship with others, and relationship with the world..
And God comes to Moses.
and does something on the top of Mount Sinai.
that up until that point,.
he had never done before for his people..
And the thing that he does changes them forever..
The Jewish people today, as well as us Christians,.
we take so much of who we are.
in this one moment right here in Exodus 20..
And to show you what happens in that moment,.
let me take you back once again.
to the top of Mount Sinai..
(gentle music).
Last episode, we journeyed together.
from the foothills of Mount Sinai.
right up here to the summit..
And we saw Moses's journey in a different light.
and how difficult it would have been.
for him to actually make that ascent,.
but also the call that God had on him.
to come up here and meet him face to face..
But something also happens on the top of the summit.
that actually reshapes not just Moses's life,.
but the life and the character of Israel itself..
And to help you to understand that a little bit more,.
I wanna introduce you to a chapel,.
a chapel that sits on the top of Mount Sinai here,.
and tell you a little bit.
about the thing that reshapes God's people..
In the fourth century, Julian of the Euphrates.
built a chapel at the peak of Mount Sinai,.
so worshipers would have a place to shelter.
and perform religious ceremonies at the very place.
where God's presence was thought to have been revealed..
Later in that same century,.
the Emperor Justinian ordered the construction.
of a three-aisled basilia here at the same site..
The present chapel of the Holy Trinity,.

$^{201}$a private Greek Orthodox church,.
was completed in 1935,.
and now rests at the eastern end of the original complex..
It amazes me that right here at the top of Mount Sinai,.
worshipers can actually gather in a church building.
around a priest who is able to perform religious services..
The chapel itself is truly breathtaking,.
and is a powerful reminder of something.
that we see right in the text of Exodus 19 and 20..
When God gives Moses the law here on the top of Mount Sinai,.
it's because he desires to be.
an intimate relationship with his people..
He wants his people to come into his presence,.
and he wants to bring his presence to them..
And that's what I love about this Greek Orthodox chapel.
right here on the summit,.
because people are still gathering.
to worship God right in this place..
This is still the intimate presence of God,.
and it truly is special just being in here..
You know, often we think of the Ten Commandments.
as rules, as boundaries, as a strict law.
that we're afraid to break, but it's not like that at all..
The Ten Commandments was God's way of saying,.
"I want you to know who I am, and I wanna be your God..
"I wanna be connected intimately with you.".
And the chapel here today is still a living legacy.
to that very moment..
And here is what is fascinating about all of that..
You see, while God had placed some stringent laws.
about who was allowed to ascend up into his presence,.
the giving of the law to Moses was God's way of saying,.
"I want to now descend down the mountain.
"and be with my people.".
It was God saying, "Here is my character,.
"here is my nature, and I don't want all of your life.
"to be about some mountaintop experience.".
You see, the Christian life isn't always about ascending.
up to the top of the mountain with God..
Actually, the beauty of the Christian life,.

$^{241}$seen through the story of the Exodus,.
is that God has descended the mountain..
God has come down with us, and he now meets us.
in the power of his presence in the everyday,.
even in the mundane things that we struggle with in life,.
God is there..
And that is why this moment on the top of Mount Sinai.
for Moses, receiving of the law, is so critical,.
not just for them in history, but right here now..
(gentle music).
It was a real privilege to be able to get into that chapel..
It's not actually generally open for tourists..
It literally is at the top of Mount Sinai..
And it's strictly reserved for those.
of the Greek Orthodox tradition,.
and they gave us permission to film in there,.
which I was very grateful for,.
'cause I wanted to show you something of that presence,.
that holiness, that awe of God in that place..
And it's in that moment where God draws Moses.
into his presence, that out of that place of awe.
and holiness, he speaks these words..
Let me read you Exodus 20..
And God spoke these words..
I am the Lord your God, who brought you out of Egypt,.
out of the land of slavery..
You shall have no other gods before me..
You shall not make yourself an idol.
in the form of anything in heaven above.
or on the earth beneath or in the waters below..
You should not bow down to them or worship them,.
for I, the Lord your God, am a jealous God,.
punishing the children for the sin of their fathers.
to the third and fourth generation of those who hate me,.
but showing love to a thousand generations.
of those who love me and keep my commands..
You should not misuse the name of your Lord, your God..
The Lord will not hold anyone guiltless.
who misuses his name..
Remember the Sabbath day by keeping it holy..

$^{281}$Six days you shall labor and do all your work,.
for on the seventh day it's a Sabbath to the Lord your God..
On it you shall do no work at all,.
neither you nor your son or your daughter,.
nor your manservant, your maidservant, nor your animals,.
nor the alien within your gates..
For in six days the Lord made the heaven and the earth.
and the sea and all that is in them,.
but he rested on the seventh day..
Therefore the Lord blessed the Sabbath day and made it holy..
Honor your father and your mother.
so that you may live long in the land.
that the Lord your God is giving you..
You shall not murder, you shall not commit adultery,.
and you should not steal..
You shall not give false testimony against your neighbor,.
and you shall not covet your neighbor's house..
You shall not covet your neighbor's wife.
or his manservant or maidservant, his ox or donkey,.
or really anything that belongs to your neighbor..
These words are perhaps some of the most famous.
that we have in scripture..
And whether you've been a Christian all your life.
or you're relatively new to faith,.
you've probably heard something of these words..
We call them the 10 commandments,.
but actually that word is not in the Bible at all..
The biblical context to these words in the Hebrew.
means literally the 10 words, the decalogue, the 10 words..
And what's fascinating about these words.
is that they are central to both the Jewish.
and the Christian faith..
But because they're so familiar to us,.
it's really easy for us when it comes to God's law,.
the 10 commandments, for us to assume.
that we understand what they're about.
and therefore take a certain posture towards them..
And what I wanna do with you today.
is actually open up a fresh perspective for you.
about what these laws are all about..

$^{321}$I want you to understand and to hear God's heart behind them.
'cause what's critical is not necessarily.
just what they're saying, but it's the character.
and the identity both of God and ourselves.
that God is trying to communicate through them..
Let me start by telling you.
what the 10 commandments are not..
They are not a bunch of legalistic rules.
that are designed to basically constrain you,.
hold you back, and make your life miserable..
These are not rules that are created.
so that you're suddenly in a whole bunch.
of red tape with God, okay?.
Remember that God has just set his people free from slavery..
The last thing God wants to do.
is put his people back in slavery..
But so often we look at these laws.
and we think they're designed to basically shackle us.
and hold us back and strip all the joy out of life..
These are not prison bars..
They're really just traffic laws..
And traffic laws are important..
See, when you drive a car,.
you have the freedom to drive wherever you want..
You can drive down the wrong side of the road.
if you want to..
No one's actually gonna stop you in the moment, right?.
You have freedom..
But we have traffic laws,.
and those traffic laws are created with lines on the road.
and lights and signs to help you.
to safely arrive at the destination that you want to go to..
And if everybody stays within their lane.
and within those traffic laws,.
we're able to operate as a successful society.
and get efficiently to where we want to go..
The law, these 10 commandments, they're kind of like that..
No one's gonna shackle you..
They're not designed to hold you back..
No one's gonna force you to do them either..

$^{361}$They're God's path of life..
It's like God saying, "I have a destination for you..
"I have somewhere where I want you to go.".
And within this structure, there is life..
Within this structure, there is flourishing,.
there's safety, there's health..
If we live this way, there is a certain flourishing.
that comes not just for us, but also for those around us..
Did you get that?.
The second thing is that the 10 commandments.
are not a bunch of rules that you follow.
in order to gain your salvation.
or to gain more love of God..
And I think so often we think of these in that way..
We think, "Oh man, if I can just be a better Christian,.
"if I could just try to live with the law.
"a little bit better, if I could try to adhere.
"to the 10 commandments more,.
"I'm going to either earn my salvation.
"or I'm gonna get more of God's blessings..
"I'm gonna get more of His favor..
"I'm gonna get more of His love.".
Salvation and God's love is not the reward of God's law..
It's the reason why we wanna live out God's law..
Are you following this, church?.
It's not the reward..
We don't follow the law and then check a box.
and God goes, "Well done.".
And He puts a medal over us and says,.
"Congratulations, I love you now.".
We do the law because it comes out of a place of grace,.
out of a place of love,.
out of an intimate relationship we have with God..
Remember, Moses comes into God's presence.
and God speaks to him about how he should live..
In fact, Jesus Himself said this..
He says, "If you love me, you'll obey my commandments.".
He doesn't say, "If you obey my commandments,.
"I will love you.".
He doesn't say, "If you obey my commandments,.

$^{401}$"I can tell that you love me.".
He's saying, "If you love me.".
If we're in a relationship together,.
if there's this relationship of love that we have together,.
out of that will flow a new way that you'll want to live..
You'll wanna do things that honor me..
You'll wanna do things that connect.
the world and me together..
You're gonna wanna live that character.
of the kingdom of priests and as a holy nation..
Those things will naturally desire to come from you.
because they come from a place of love..
This is why it was great in worship.
that we spent that time in that place.
of understanding our love of God and His love of us.
'cause from that flows everything we're about to do..
And there's many of us who think.
if we can just do this list better, God will love us more..
And I wanna break that slavery from you..
Anybody here?.
I wanna break that slavery from you..
You've been set free by Christ Jesus.
'cause He loves you so much..
And out of that love comes the glory of life..
Here's the third thing that the 10 commandments are not..
They're not irrelevant for today..
They're not to be cast aside..
Oh, that was God in the Old Testament..
And now that we have Christ.
and now that we are a kind of a regeneration,.
I have new covenant and this old stuff,.
it doesn't really matter anymore..
I can live how I want because I have the grace of Jesus.
and He forgives me..
That's not what's happening here..
In fact, Jesus in His own ministry, He said this,.
"I didn't come to abolish the law..
"I didn't come to wipe it away..
"I didn't come to just say,.
"Hey, this doesn't matter anymore..

$^{441}$"No, I've actually come to fulfill it..
"I've actually come to show you in my life.
"how it is actually a blessing..
"I wanna show you what the law does.
"and the heart behind the law..
"And I wanna show you how I am the only one.
"who could ever fulfill the law.".
This is the fascinating thing..
You see, the law itself was always designed.
to move people to Jesus..
And so when Jesus says, "I am the fulfillment of the law,".
what He's saying is the law is designed to draw you to me.
so that when you find me, you can then live out the law..
Are you seeing the flow?.
So let me tell you a little bit.
about what the commandments actually are.
and how you might be able to live them out..
And I wanna do that by breaking them down for you.
in a way that perhaps you've not heard before..
But I hope in doing this,.
it'll really help you to understand.
how God has created you to be..
Most people, when they look at the law,.
most scholars, when they talk about it,.
they break it into two sections..
The first four being about our relationship with God.
and the next six being about our relationship.
with one another..
And that's absolutely a fair way of looking at these..
In fact, Jesus Himself pretty much saw it that way.
when the lawyer asked Him in the Gospels,.
"Hey, how do you summarize the law?".
He says, "Love the Lord your God.
"with all your mind, heart, soul, and strength.
"and love your neighbor as yourself.".
There it is, the Shema, as it's known in the Torah.
in the Jewish faith..
The Shema points us to the culmination of the law..
We are to love God and out of that,.
we are to love one another..

$^{481}$So it breaks quite neatly down.
into this kind of two sections..
But I think what's really important to understand.
for the Jewish people as they've arrived at Mount Sinai,.
their context is not necessarily this kind of idea.
of the Shema, of pulling it all together in two easy ways..
Their concept is still slavery and brokenness..
And their concept is identity vacuum..
And I need to know who I am..
And it is the context of Egypt and slavery.
that God brings the law into that..
So I think actually the 10 commandments.
break down into three sections.
that are all about how we love..
'Cause God is calling His people now to live in a new way,.
to live in a place of love,.
not a place of slavery and fear..
The first three are about how we love God..
The fourth is about how we are to love ourselves..
And the final six then are about.
how we are to love one another..
And they flow in that order really well..
Now, as I unpack these for us,.
I also wanna do it in three specific categories.
to help you to understand the heart behind.
what the 10 commandments are all about..
I wanna talk to you about three identities.
because that's what God is trying to communicate..
The first identity is what I'm calling Pharaoh's identity..
I'm not speaking specifically about Pharaoh,.
the actual person and his identity..
I'm talking about a summary identity,.
this idea of what humanity is like.
in its brokenness and sin..
What we see in the whole of the Exodus journey.
is Pharaoh becoming this kind of picture,.
this metaphor, if you will,.
of what humanity is like apart from God,.
what humanity acts like.
when it isn't in relationship with God..

$^{521}$So that's Pharaoh's identity..
Does that make sense to you?.
Then I'm gonna tell you about God's identity,.
how God reveals Himself in the 10 commandments..
And then finally, I'm gonna tell you about your identity,.
our identity, how that relates..
So there's gonna be the identity.
of broken, sinful humanity,.
the identity of God, and then our identity..
Making sense?.
Okay, let me give you some examples..
Let's start with the very first one..
The very first one is, of course, when God says,.
"You will have no other God than Me.".
No other God but Me..
It's very important that that's the first one.
because, of course, Israel's come out of a pantheon.
of religious understanding when they were slaves..
The Egyptian cultural religion had over 60 different gods..
So God says, "No, we're leaving that behind..
"That's not the way things are..
"There is just one God, and you are to worship God..
"You worship Me..
"You're not to have any other gods.".
But actually, God is saying something else.
about identity here that's really important..
It's easy to miss this..
Remember, the context is slavery.
and coming out of that Pharaoh identity..
The reality is, so much for many of us,.
is that we actually put ourselves in the place of God..
It is so easy to make life all about yourself..
It's so easy for us to begin to think.
that actually we are the center of the world..
I remember raising my daughter..
There were many years where everything revolved around Mia..
Now I'm still wrapped around her finger.
a little bit too much, but she still knows now.
that the world does not revolve around her..
Are you with me?.

$^{561}$But there is something in our identity,.
in Pharaoh's identity, where we would basically express it.
like this, "I am God, and I rule.".
And I want you to be honest with yourself.
'cause you'll hear that, and you'll be like,.
"No, no, I don't believe that.".
But then our actions, Monday to Friday,.
suggest that we are God and we rule..
Are you with me?.
Am I preaching to anyone?.
Getting a little uncomfortable?.
Good, I'm just warming up..
But the Pharaoh identity of this idea,.
"I am God, and I am rule,".
and there's so much of that in us,.
and God knows He needs to break that..
So God, with His first commandment, is saying,.
"I am God alone..
"It is just me..
"I am the only one..
"I have all authority in my hands,.
"all control and authority sits me.".
Because when we declare that we are God and we rule,.
we're basically saying we're in control..
God's saying, "All authority is in me.".
Our response to that then,.
our identity now as free people,.
is to say, "We are worshipers..
"We are worshipers of God.".
And that's really important.
because when you're in that Pharaoh's space.
of saying, "I am God and I rule," you worship yourself..
And God is saying, "I'm the one who holds all authority.".
Our response is to say, "I will worship God.".
Are you with me?.
Here's the second one..
The second one is you will make no image..
You will make no other idol..
Now, this is really important again.
because they've come out of their slavery.

$^{601}$and out of that pantheon of 60 gods,.
there's lots of idols that are created.
that they can worship..
And here's the fascinating thing..
Pharaoh would use those idols as a way of controlling.
and manipulating his people..
He would oppress them under a system of image worship,.
idol worship..
And this is really important to understand.
because I think all of us struggle to some degree.
with a deep-seated Pharaoh identity.
of basically worship my image..
Now I'm really preaching..
Worship my image..
Follow my social media..
Follow my ideas..
Do what I say..
I wanna control..
I wanna influence..
I wanna be the focus..
Worship my image..
One of the things that we're seeing.
in the social media space right now.
is all these apps and all these AI tools.
that can take a really not attractive person.
and turn them into a stunning person..
You no longer had to be like incredibly hot..
You could be ugly and still hot today..
And all of this is creating for us.
this idea of like worship my image..
And God is saying, "Hang on,.
we need to break the back of that slavery.".
And he's saying, "You have me.".
He's basically saying, "I am all you need..
And if you have me,.
you don't need to worship anything else,.
whether that's anything in the world.
or whether that's yourself, you have me.".
Our response to that is to say,.
"My identity therefore is I am undivided.".

$^{641}$Because one thing that image worship always does,.
it divides your heart..
And you end up worshiping a whole bunch.
of different things all over the place..
Pharaoh's identity is come and worship me, I'm the best..
God is saying, "You don't need anything else other than me.".
We're going, "I am undivided in my heart.".
Are you seeing this still so far?.
Anyone?.
Are you alive?.
Okay, here's the third one..
By the way, we're not doing this for all 10, don't worry..
This would be about a four hour sermon..
But I'm gonna do it for a few..
The third one is this idea of not misusing my name..
He says, "Don't misuse my name.".
When I grew up, I thought this was all about swearing..
Don't swear..
If somebody beats you to the lift and the door's closed,.
don't use Jesus's name out loud, right?.
Now, that is very important..
And please don't misuse God's name in that context..
But I don't think that's actually what God is saying here.
in this moment..
He's drawing his people to him out of a place of slavery..
And he's saying, "I've saved you..
"I have redeemed you..
"I have set you free..
"Now don't misuse my name.".
Name in biblical context means character,.
your good name, my character..
He's saying, "Don't misuse my character.".
In other words, if you're gonna say that you're a Christian,.
but your life is leading a very different picture.
to my character and who I am, stop doing that..
Don't misuse my name..
Don't claim to be a Christ follower.
and do all this other stuff that everybody else is doing,.
because that's just not gonna do any priestly function.
in the world at all..

$^{681}$You're not helping people connect God together..
People just think you're just like them..
Don't misuse my name..
Are you with me?.
And then there's this other thought as well.
that I think is in God's heart here..
And that is for us as Christians amongst each other..
Because I tell you one of the things that's at work.
in the global church right now.
that we have to be really, really careful of,.
and we need to stand up against is spiritual abuse..
And spiritual abuse happens when somebody takes.
God's authority to manipulate people.
within a church context..
The number one, I'm just gonna be honest with you..
The number one thing that Chris and I constantly pray about.
is that I would do my utmost to avoid the temptation.
to use God's name to get you.
to do something I want you to do..
That's every pastor's greatest challenge..
But it's not just the pastor's challenge..
It's also your challenge too..
That we wouldn't misuse the name of God.
for our own personal gain..
Pharaoh's identity is basically this, God serves me..
Like I'm gonna use God to get what I need to get..
God serves me..
God is basically saying, I serve no one..
Our response is to say, we are servants..
We are servants of God..
You follow the flow?.
So I want you to see what God's doing.
just in this first section of loving God..
He's trying to uproot some of this broken Pharaoh identity.
that so many of us still hold onto,.
that I know I still hold onto,.
where I say, I am God and I rule..
Where I say, worship my image..
Where I say, God serves me..
God says, I alone have all that authority..

$^{721}$You don't need anything else other than me..
And I serve no one..
And our response then in the reality of that.
as free people is to say, I am a worshiper..
I am undivided..
I serve God..
Follow it?.
Isn't it great?.
This is the path to life..
This is not shackles to bind you..
It is something to set you free..
This is why the fourth one is the most, I think, important..
Keep the Sabbath..
Have a day off, God says, and we all ignore it..
And he binds this in creative history..
He says, in the same way that God himself.
rested on the seventh day, you should also rest..
And God is not doing this.
because he thinks everybody needs a yoga retreat..
He's doing this because he's speaking about their identity..
Notice this, when you're a slave,.
you are never resting or free to rest..
Slavery by its very definition is you go, go, go, go, go.
all the time..
They work seven days, 365 days a year..
That's what it meant to be a slave..
So when God comes and he says,.
this is not about how you love me..
This is also about how you love yourself..
And if you don't rest, you are not proclaiming the freedom.
I have paid the price for for you..
'Cause when you fail to rest,.
you're basically adapting the identity of a slave..
This is why this is so key for Hong Kong..
Because every single one of us, we all work too much..
We all drive ourselves so much..
That song we sang earlier, there's so many times.
where Chris says to me,.
"Even when I don't see it, you're still working..
"Even when I don't feel it, you're still working..

$^{761}$"You never stop, you never stop working.".
(congregation laughing).
She's right..
I need this more than anything..
And here's what the identity is..
The Pharaoh identity is, I will provide for myself..
I will consume and provide..
I'm in control..
And we drive ourselves because we're fearful.
of not being able to provide..
And God is saying, I'm your provider, rest..
Take a breath..
You can't do it all in your own strength..
You can't drive yourself to success like that..
You can try, but you're just gonna reach burnout..
You need to rest..
And in that place of rest, we say, I have faith in God..
I have faith and I have trust in Him that He will provide..
And that means that I will stop my work..
I will turn my email off..
I won't respond to WhatsApps or whatever it is..
I'm gonna break away..
I'm gonna spend time with my loved ones..
I'm gonna get away from all of that stuff.
that's draining all my energy..
I'm gonna refill it by being in relationship with God..
And that's the way I'm gonna find life..
That's the path to life for me..
It's really interesting because we think Sabbath.
is about coming to a church service..
Sabbath itself is worship..
Come on church..
I like it when you come to church services..
Please keep coming to church services..
But I need you to know that Sabbath itself is worship..
Because when you Sabbath,.
you're saying I'm no longer a slave..
I have freedom to rest..
And when you rest,.
you proclaim a different narrative to the world..

$^{801}$Rest my friends, rest..
Now, here's the other important thing about Sabbath.
and why I think it's here, right here, number four,.
and why it's about ourselves..
Because here's the reality..
When we fail to rest, we're really bad at loving others..
When you're exhausted and tired and worn out and burnout,.
your patience with people goes like this..
At least mine does..
I know when I need a Sabbath.
is when I don't like you guys anymore..
(congregation laughing).
That's when I need a break..
That's when I need to get away..
Because I love you guys..
I really love you guys..
But there are times when I don't,.
and that's because I've gone and I've gone.
and I've worked and I've worked.
and I'm exhausted and in pain..
This is why God says you have to keep the Sabbath..
And it's about loving yourself because out of that,.
then you can keep the next six..
'Cause the next six are really hard if you're exhausted..
The next six are really difficult if you have not rested..
But if you're resting in me, abiding in me,.
finding your presence in me,.
then you will flow to your love of others..
Now I'm not gonna break these down one by one for you.
'cause it will take ages, but let me put them up on a slide..
This is the next six of them..
You don't need to take a photo of this.
'cause I'm gonna give you one slide in a moment.
that I have all 10..
So, but you can look at this..
I want you to see this because here you see,.
God has a heart for family..
God has a heart for people..
God believes in life over death,.
that God is non-violent,.

$^{841}$that God wants us to treat each other with respect.
and honor and peace and love..
This is a manifesto of how we should be walking together.
with one another..
And yet that Pharaoh identity so easily strips us.
of the freedom that God has created us for..
Are you with me?.
And all of this is his invitation..
It's like, come to me and out of me,.
come and live this stuff..
Now, if you're anything like me, let me put up all 10.
and you can take a quick photo of that..
And we're actually gonna make a social media post.
about it later this week..
So you can keep an eye out on our Instagram.
and Facebook for that..
But you can take a photo of that..
You can put it up somewhere where you can see it regularly.
because this is the path of life..
Now, you're all taking a photo..
(congregation laughing).
Serious, I'm gonna finish with this..
I don't know about you, but when I look at that list,.
my time is spent looking at Pharaoh's identity.
because I relate to it..
I look in here and I can see so much of Pharaoh in me..
I can see so much of the broken is still there in me..
I can feel the temptation to make myself a God..
I can feel the temptation to make people worship my image..
I can feel the temptation to wanna steal.
other people's creative ideas..
I can feel that brokenness in me to look with my eyes.
upon others in ways that I shouldn't..
So much of Pharaoh I know still sits in me..
And part of that is absolutely right.
because the law is to drive us to Jesus..
And when we see this broken Pharaoh-ness in us,.
it should compel us, drive us towards Jesus..
But here's what usually happens..
When we find the Pharaoh stuff in us,.

$^{881}$it doesn't drive us to Jesus, it drives us to ourselves..
And we try to work harder to achieve everything.
on the right hand side here in our own strength..
The worst thing you can do off of a message like this.
is to go home and go,.
"Andrew told me to pull my socks up.
and try to be a better human being.".
It's not what I'm telling you today..
If you try this in your own human strength,.
you will be exhausted..
The most exhausting thing on earth.
is being religious and not saved..
Oh, the most exhausting thing is being religious.
but not having a saving relationship with Jesus..
You cannot achieve this on your own, in your own strength..
And that's what the law is all about..
The whole point of the law.
is to bring us to somebody who can do it..
This is why Jesus said, "I didn't abolish it,.
but I fulfilled it.".
Paul writing in Romans 10, he put it this way..
He said, "Christ is the culmination of the law.
so that there may be righteousness.
for everyone who believes.".
But notice this, who believes?.
Who has relationship?.
Who is in a relationship?.
Not righteousness for everybody who works hard,.
tries hard, strives hard, does stuff on their own strength..
Righteousness for those who believe.
'cause Christ is the culmination of the law..
And as the culmination of the law,.
our belief and relationship with him.
releases the law in and through us..
This is why Paul would write.
that we no longer have the law on stones or tablet,.
but they're written on the flesh of our heart..
'Cause the spirit of God now is in us.
to help us to live in the way that would bring God glory..
We don't do it in our striving, we do it in our abiding..

$^{921}$We do it in our relationship with Jesus..
As we're drawn into him.
and intimately in relationship with him,.
we are then drawn out to the world..
The last thing I want you to do.
is just try to work harder..
What I wanna invite you into.
is to be so inspired by the path of life.
that you just wanna be more with Jesus..
The very central passage of the vine is John 15, five..
I am the vine, you are the branches..
If a person remains in me and I in them,.
they will bear much fruit..
Apart from me, you can do nothing..
Apart from me, it is useless..
It is just religion without salvation and that's painful..
But in me and with me, it is the path of life..
Abide in me, he says..
Don't try to go through that list.
and make it all work for you..
Use that list to propel you to Jesus.
so that in relationship with him,.
you are propelled to the world..
That's the path of life..
And that is Exodus chapter 20..
Let me pray for you..
Father, I am so grateful for everybody here..
I'm so grateful for your presence with us.
and your call to us..
Father, we sit in the beauty of these 10 commandments,.
these 10 words..
And Father, they invite us into reflecting.
on our Pharaoh identities,.
on the brokenness that we carry so easily amongst us..
And the law reveals the Pharaohs in us,.
not so that we would feel bad about ourselves,.
so that we'd feel alone in that field,.
so that we would beat ourselves up..
The law reveals our brokenness.
so that we are drawn to Christ,.

$^{961}$that we are drawn out of ourselves to him,.
the one who can save,.
the only one who's the fulfillment,.
the culmination of the law..
So Father, for every person here.
who is recognizing some of the Pharaohs in their identity,.
I pray that in this moment, they would come to you..
I wanna give you space to be able to do that,.
just in the quietness of your heart..
We've seen today God's identity,.
who he is, what he stands for,.
and the way that he is the opposite.
of all that slavery that we experience..
And out of his identity and who he is.
and how he's made himself known, we then respond..
And so I wanna pray for each one of you.
as you abide in Christ,.
as you come to Jesus in a beautiful place of grace,.
I pray that you would feel him transforming and changing you.
as a priest and as someone who is holy..
And I feel that out of that, I pray that you would know.
the joy that it is to live in this world.
in a way where we love our neighbor,.
where we don't commit adultery, where we don't fight,.
where we don't seek brokenness and slavery,.
but a world where we long to love and honor and respect.
and see peace come on earth as it is in heaven..
Father, would you inspire us by you.
and by how you are all of the law.
in a person and in your spirit..
So I wanna pray for the release of the Holy Spirit.
of every person here..
I wanna pray, Lord, that you would,.
by your spirit, write on our hearts your path of life..
In Jesus' name we pray..
I'm just gonna respond together in worship..
Would you stand with me?.
\newpage



\section{}
\label{sec:5UVcdfy_Atk}
\textbf{2023-10-30 Exodus - 20 Falling Into Forgetfulness [5UVcdfy\_Atk].mp3}
\newline
\newline
連結: \href{https://youtube.com/watch?v=5UVcdfy_Atk}{\texttt{ https://youtube.com/watch?v=5UVcdfy\_Atk}} ~~~~ 語音日期: 2023-10-30 
\newline
\newline
\hyperref[sec:FFFAigIMmRc]{\small{< < < PREV SERMON < < <}}
~
\hyperref[sec:index]{\small{[返主目錄]}}
~
\hyperref[sec:QuRsgZhXkqs]{\small{> > > NEXT SERMON > > >}}
\newline
\newline
$^{1}$In Jesus' name, everyone says, amen, amen, amen..
Can we thank our worship team as always?.
And have a seat, have a seat..
You know, in the last two weeks here at the Vine.
in our Exodus series, we've been situated.
in two incredibly important chapters..
Chapters 19 and 20 of Exodus,.
which really are a pivotal turning point.
in the whole narrative of the story..
Israel is free..
God has brought them out of their slavery.
for over 200 years and has brought them to Mount Sinai..
And he's brought them to Mount Sinai.
because he wants to fellowship with them personally..
He wants to connect with them face-to-face, if you will..
And in Mount Sinai, chapters 19 and 20,.
God begins to redefine the identity of a broken people..
Because up until that point,.
Israel was carrying around still their Pharaoh identity..
Slavery has stripped them of who they truly are..
And God gathered them at Mount Sinai in chapters 19 and 20,.
and he speaks to them about who they really are..
Chapter 19, as we saw two weeks ago,.
he says, "You're a kingdom of priests.
"and you're a holy nation.".
Israel needed to understand that they had a function.
to connect God and the world together..
They had a function to walk on behalf.
of the imitation of God in the world,.
to be his hands and his feet around them..
But they understood they're called a separateness as well..
They're called a holiness,.
to represent something of the heart of God in this place..
And then in chapters 20, as we saw last week,.
we see that God begins to speak to them.
about how they are to interact with him,.
with themselves, and with one another,.
to form, if you will, a worshiping community..
And in chapter 20, we see he gives the 10 commandments,.
as we know of them, these 10 words in the Hebrew..

$^{41}$This idea of what it is to love God,.
to love ourselves, and to love one another..
And through all of this, what God's trying to do is say,.
"There's an identity that I want you to hold now.
"as a free worshiping community.".
And that identity is important.
because when you come to know yourself as children of God,.
you can then live out of that identity.
in the world around you..
God brings them to himself.
so that he can then send them back to the world..
He had to bring them and purge them.
out of the brokenness of the slavery of the world,.
bring them to himself, so that in relationship with him.
and in receiving their identity in him,.
they could then go and serve the world,.
bring the world into its fruition,.
by saying there's an alternative way to live,.
that there's a new hope that's found.
in this idea of who God is and his character in the world..
And what we've been seeing over the last two weeks.
is it's the same for us,.
that we, the church today, have the same call on us.
that God brings to Israel.
as they're gathered around Mount Sinai..
We ourselves have been taken out of a place of slavery,.
the slavery of our sin,.
and we've been brought to Christ Jesus..
And in relationship with Christ Jesus,.
our Mount Sinai, if you will,.
we receive the law not written on tablets of stone anymore,.
but by his spirit written on our hearts..
And out of that, we can now live as a worshiping community.
in the world around us, going out into Hong Kong.
and saying we believe a church can change a city..
We believe that God and who he is and his character.
and the narrative he creates for this world.
has something to say to the brokenness of this world..
And so we're a sent people..
And what I wanna talk today about with you,.

$^{81}$and I'm gonna talk really passionately today,.
so if you don't like passion, I apologize beforehand..
But I'm passionate because I'm talking today.
about the church..
It's the one thing I love the most, the church,.
and all of its frailty and all of its brokenness.
and all of its challenging diversity..
The church is God's hope in the world..
And as we see what happens.
in the next 11 chapters of the story,.
we see some important things that we need to understand.
about what it is to be Christ's hands and feet.
in the world that he's planted us in..
Chapters 21 to 31 are actually 10 chapters.
where God expands on this idea of what it means.
to be a worshiping community..
And in there, you see actually three critical things..
The first in chapters 21 to 24 is a continuation from God.
out of the 10 commandments of how Israel should form.
and work together, how they should love one another..
And he speaks to them about their social responsibility..
He speaks to them about their equity.
that they're supposed to have amongst each other,.
how they're supposed to handle property.
and laws amongst themselves.
so that they can truly be an equitable.
and just community in an unjust world..
And then chapters 25 to 27 and chapters 30 and 31,.
God speaks to them about a new thought..
He says, I want you to create a place of worship..
It comes to be known as the tabernacle.
in the rest of the Exodus story,.
in the rest of the Old Testament..
And the tabernacle essentially was a tent of meaning,.
but it was a complex tent..
And for a bunch of chapters,.
God speaks about how they are to create this tent.
and the artifacts that are gonna go in that tent.
so that it can become a place of worship..
So that whilst they're a nomadic community.

$^{121}$and before they go to establish the temple in Jerusalem.
many years later,.
they would have a place to come to to worship their God..
And then in chapters 28 and 29,.
the third thing God speaks to them about.
is the priestly class amongst them..
He says, set aside people who can serve in the tabernacle,.
the very people who would eventually go.
and serve in the temple,.
and these ones will serve the worship life.
of that community..
And in chapters 21 to 31,.
we see a picture really of God's perception.
of what a worshiping community on earth should look like,.
what the church, if you will, should look like,.
how we should be just and loving and kind and equitable.
amongst each other to model something in the world,.
how we are to know that our identity at its very heart.
is to be a worshiping community.
and what it means to be worshipers on earth..
And then out of that,.
what it means for us to be a mission focused.
community of people..
We do not gather every week for 90 minutes in this room.
just so that we can feel good..
I'm not interested in building a church.
where you come to feel good..
Oh, I'm gonna preach on this one for a moment..
I hope you enjoy coming to the Vine, okay?.
But my purpose is not to preach..
Our purpose to do what we do here at the Vine.
is not designed just so that you feel.
a little bit better about your life..
It's designed to change a city..
And in order to change a city,.
we have to understand what it is that we're called to.
as believers in Christ..
And so 21 to 31 shows us that..
But in 32,.
chapter 32 of Exodus,.

$^{161}$we see perhaps what is one of the most horrendous things.
happen in the whole of the Exodus story itself..
And in what happens to Israel in Exodus 32,.
we see what I would call.
are the three greatest temptations that exist.
for any church to weaken and disempower it.
for the way in which it wants to change.
the society around it..
Three things in that chapter.
that show us the way in which the enemy is at work.
to disempower and weaken the body of Christ.
for its purposes in the world..
And so this is why I'm gonna teach passionately today.
because I love the church..
I love you..
And I believe the church can change a society..
But these three things we're looking at today,.
I'm gonna say this upfront,.
are three things that I think we struggle with.
here at the Vine sometimes..
I think we struggle with these things as a community,.
as a congregation..
I know that I personally struggle with them.
as a person individually..
And I wonder as I unpack them,.
whether you might resonate with that a little bit..
And I think therefore God might wanna do some healing.
and some change and transformation in us.
where we can recognize what the enemy is trying to do.
to disempower us, stand against it,.
so we might truly be Christ's hands and feet in a city..
Does that sound interesting to anyone here?.
You're right..
Wonderful..
There's one thing that sits behind.
all three of those temptations..
And it's the one thing that we see happen in Exodus 32..
Introduce you to that one thing..
Let me take you back once again to Mount Sinai..
It is within the shadow of the great Mount Sinai itself.

$^{201}$that the darkest moment of the Exodus story occurs..
Exodus 32 is a brutal chapter,.
outlining idol worship, sin, burning anger,.
punishment, and even death..
For the reader of the Exodus narrative,.
it's actually shockingly jarring.
coming off the back of such great chapters.
as Moses' ascent up the mountain.
and the giving of the law.
and his ongoing presence with his people..
I mean, from the intimacy of God with them.
to his burning anger against their sin..
Now, all of this is a stark reminder to us.
of one of the most important messages.
of the whole Exodus journey..
And that's this,.
that so often the most damaging slavery.
that happens to us is the slavery of the heart..
Egypt had one of the largest.
and most complex pantheons of gods.
of any civilization in the ancient world..
Over the course of ancient Egyptian daily life,.
hundreds of gods and goddesses.
could be worshipped at any one time..
These gods were assumed to be present throughout the world,.
capable of influencing natural events.
and the course of human lives..
So connection with these gods.
was of paramount importance to ancient Egyptians..
And that connection largely came.
through physical representations of the gods themselves..
So carvings, statues, amulets, sculptures,.
idols were literally everywhere..
And the more idols you had around you,.
the better you would be able.
to make a connection to the gods..
And therefore the better your life.
and the lives of your loved ones would be..
In modern day Cairo today,.
you can still see so much of this kind.

$^{241}$of ancient idol worship culture..
These tourist markets in old Cairo are a clear example..
These various sculptures of Egyptian gods.
are done out of the most beautiful turquoise.
found native to this part of the world..
And while these are of course,.
not designed to be specifically used for worship today,.
they give you the idea of the kinds of gods.
that were constantly on display.
at the time that the Israelites were slaves in this land..
It should not surprise us then.
that Israel struggled to adapt.
to a new form of spirituality.
as they emerged out of their slavery.
and met with God on Mount Sinai.
and actually received the 10 commandments..
I mean, think about it from their perspective..
They had been immersed in an idol worship culture.
for over 400 years..
And that culture had seeped.
into the very essence of who they were..
So as Moses took his time on the top of the mountain,.
actually Israel grew itchy..
They took the gold that they had taken from Egypt.
and they asked Aaron to form for them a golden calf.
so they could worship that as a picture of Yahweh,.
just like they had seen the Egyptians.
form and worship their own gods in Egypt..
I think it's fascinating how we see in Exodus 32,.
something play out that is actually spiritually significant.
for the whole of the Exodus narrative..
You see, the Exodus is not so much.
about getting Israel out of Egypt.
as it is about getting Egypt out of Israel..
And it's the same, of course, for us..
You see, any journey from slavery to freedom.
is going to require actually the purging from us.
of any of the cultures or the values.
or the habits or the brokenness.
that we embraced in our time of slavery..

$^{281}$You see, in the formation of the golden calf,.
what actually was happening.
was that Israel was essentially saying.
that despite all the miracles of God.
and all the physical liberation that they had received,.
the chains of slavery and oppression.
were still tightly entangled around their hearts..
And here is what angered God the most about that..
They just didn't seem to care..
I mean, they were happy to live.
with this half-baked kind of freedom.
where they had a little bit of what God wanted,.
but they were still holding onto.
the brokenness of their past..
So when they fashioned that idol of Yahweh,.
what essentially they were doing unwittingly.
was remaining locked in their previous slavery..
And isn't it great that you and I.
are so free of all of those idols today?.
- I think what I say in that film.
is actually really important for us,.
that it's easy for us to settle with a half-baked freedom.
where we live divided between having elements.
of God's liberation on the one hand.
and yet still holding onto our slavery.
that we've been so accustomed to on the other..
And we find ourselves divided between these things..
And that division means.
that we live in this half-baked freedom..
We experience some of that great release of God,.
and yet we find ourselves so easily entangled still.
in the things that hold us back..
And what we see in Exodus 32.
is that the key to the things that hold us back.
is this thing called idolatry..
The second commandment, of course,.
that God gives Moses in chapter 20.
was to have no idols, no images..
And the very first thing that Israel does.
is that they take an image.

$^{321}$and they create it in order to worship..
And I wanna show you a few things in this.
because I think there's so much for us to learn.
in our hearts about what it is.
to be a believer, worshiper of God ourselves..
Let me read to you Exodus 32, verse one..
It says, "When the people saw that Moses was so long.
in coming down from the mountain,.
they gathered around Aaron and said,.
'Come, make us a God who will go before us.'.
And as for this fellow Moses,.
who brought us up out of Egypt,.
we don't know what has happened to him.".
I want you to notice the thing.
that catalyzes God's people into idol worship..
It's seen right at the start..
It says, "When the people saw that Moses was so long.
in coming down from the mountain.".
We actually learn in Exodus 24.
that Moses was for 40 days and 40 nights up on the mountain..
40 days, 40 nights..
That's a pretty long period of time..
And Israel are waiting..
They thought he was gonna go up, meet with God,.
and come back down, maybe even in the same day..
But it's been 40 days and 40 nights.
and their expectations have not been met..
They're anxious about what's happened to Moses,.
their leader..
They're wondering what's going to happen in their future..
And it fascinates me that the triggers.
that cause them to move into idolatry.
is essentially unmet expectations and anxiety.
for the future..
And if you wanna think about what are the things.
that are gonna be the most that will trigger you.
into any form of idolatry in your life,.
it will be impatience, being out of control,.
not knowing what's gonna happen in your future,.
anxiety for what's ahead..

$^{361}$Those are the key catalysts that draw us.
to try to find something tangible,.
something we can trust,.
something that we can build our hope on.
when the thing that we were building our hope on.
seems so intangible, so distant, so out there,.
and we can't quite control it..
Are you with me?.
See, the core of idolatry is this desire to control..
Humanity is really bad at being out of control..
Well, this hasn't happened on the timeframe.
that I thought it was gonna happen on..
He's been up there so long..
I don't know when he's coming back,.
if he's gonna come back at all..
So we're gonna take matters into our own hands..
If you want to get a sense of what the idols.
might be in your life,.
look at what you turn to for comfort.
when you're anxious or disappointed..
Come on, church..
I wanna encourage you to do that this week..
That's gonna be a great devotional time.
for you and Jesus this week, okay?.
Have a think about what are the things.
that you are naturally turning to.
to find comfort and control and security.
when you're feeling anxious, disappointed,.
or your expectations have not been met,.
whether that's from your spouse,.
whether it's from your boss,.
whether it's from God..
Those things that we turn to to find stability.
when our life is not stable.
is very often the things that we turn into idols..
You follow that?.
Now, notice what happens here next..
It says here in this passage,.
this is what they say to Aaron, Israel..
They say, "Come, make us a God who will go before us.".

$^{401}$Now, this is fascinating..
The phrase will go before us is the first thing.
that shows us the first temptation.
that I think we are susceptible with here.
that disempowers us as a church..
They say, "Make us a God who will go before us.".
That phrase, "who will go before us,".
is a phrase that's been used a lot.
in the Exodus journey so far..
God's used that phrase about himself so many times..
You are my people..
I will go before you and prepare the way..
I will go before you and prepare a promised land.
that is flowing with milk and honey..
I will go before you as a pillar of cloud.
and as a fire during the day..
I will go before you has been a constant phrase.
that God has been speaking about the future,.
the vision, the destiny,.
the purpose of where God wants to take Israel..
And it's fascinating to me.
that when Israel create a God to worship,.
they use this phrase, "We want this God to go before us.".
In other words, we want this God to now achieve.
what this other God had set up for us..
You notice this?.
Now, this is really subtle..
It's really important.
because what Israel is doing.
out of their unmet expectations.
and out of their anxiety for their future,.
they're basically saying, "We have God's promise,.
"but we're not happy that that promise is not coming through.
"how we were expecting it to come through..
"So we're now going to do it in our way.".
This is the idolatry of the self..
This is saying that I am better.
than what God is going to do..
But notice they still want what God wants..
So at its very,.

$^{441}$listen to this, church, this is really important..
At its very core, identity..
Identity is the decision in our lives.
to basically in the flesh accomplish something.
that God started in the spirit..
Come on, church..
That at its very heart is what idolatry is all about..
It is the desire and the decision.
to accomplish, to finish in our flesh.
something that God has started in His spirit..
And for some of us, it started well..
It started out, God spoke to us..
He gave us a call, He gave us a vision,.
He gave us some hope..
And we began out of that place of the spirit.
to believe and to pray and to align our priorities.
towards the thing that God had said..
But over time, when our expectations got unmet,.
when we became anxious about our future,.
where things didn't seem to happen.
how we quite thought they were gonna happen,.
it is so tempting for us to finish in the flesh.
the thing that God started in the spirit..
It is very tempting for us to essentially create.
an Ishmael process for an Isaac promise..
Are you with me?.
And this is exactly what they're doing here..
And they're doing it out of that place of fear,.
out of that place of not knowing,.
out of that uncomfortable, not being able to control,.
I'm gonna finish in my flesh.
what God has started in His spirit.
'cause I'm tired of waiting..
Some of you in this room,.
you're trying to finish in your flesh.
something that God has promised you.
'cause you're upset that God hasn't done it.
in the way you thought it was going to be done..
This is not a new problem for the church..
This was a problem that existed.

$^{481}$right at the start of the church..
Paul, writing to the church in Galatia,.
he said it this way in Galatians 3.3..
Let me read this to you..
"Are you so foolish?.
"After beginning by means of the spirit,.
"are you now trying to finish by means of the flesh?".
See, the idolatry of self is one of the most.
powerful idolatries that the enemy brings the church..
And I wanna, as your pastor, be honest with you about this..
This is the idolatry I see the most in me..
As your leader, the idolatry I see the most in me.
is me trying to finish in my flesh.
what God has begun here at the vine..
And I know, by the way, this week I'm celebrating,.
this week is the end of my 10th year as senior pastor here..
I became senior pastor November 1st, 2013..
And I know over those last 10 years.
that I have on a number of occasions.
created a golden calf and held it before you.
as I tried to finish in my flesh.
what God has started here in his spirit..
And I can tell you this from firsthand experience,.
it is exhausting to do that..
The irony is is that anxiety draws us.
into taking matters into our own hands..
But when you take the things of the spirit.
into your own hands and try to finish them in the flesh,.
it only creates more anxiety in you..
And see, the enemy's strategy over you is this..
He wants to disempower God's work in your life.
by turning the very good things of God.
that God has put in your life and started in your life.
and turn them into exhausting and tiring things.
through his work in you..
You know, it should sober us to realize.
that one of the great strategies of the enemy.
is actually to take the promises of God.
that are designed to set us free.
and through the worship of self,.

$^{521}$turn them into shackles that exhaust and bind us..
Listen to this, church..
This is what the enemy wants to do..
Take the very good promises of God,.
the things that God has put in us to set us free,.
and through the worship of self,.
through the driving of trying to achieve.
what only the spirit can achieve,.
we will essentially burn ourselves out..
And the very gifts and the promises of God.
become shackles that exhaust and bind us..
And if you feel a little bit like that today,.
right at the end of our time together,.
I'm gonna share something that I believe.
can help to set you free..
But I wanna show you the second great temptation..
Is this helpful still so far for some of you?.
Here's the second temptation..
Aaron then answered them..
Verse two, "Take off your gold earrings.
"that your wives, your sons, your daughters are wearing.
"and bring them to me.".
So all the people took off their earrings.
and brought them to Aaron..
He took what they had handed to him.
and he made it into an idol cast in the shape of a calf,.
fashioning it with a tool..
Fascinating..
So Israel says, "We want an idol.".
Aaron, who's the de facto leader in Moses' absence says,.
"All right, well, take off all the earrings,.
"all the gold that you had taken from Egypt..
"Take that off, give it to me..
"I'm gonna melt it down..
"I'm gonna put it in this iron cast..
"I'm gonna create a golden calf.".
Essentially a bull, a young bull..
"I'm gonna create that and that's gonna become our idol.".
Now, the context to this is really important.
because back in Exodus 24,.

$^{561}$God had spoken from 25 to 27 and then 30 and 31,.
God had spoken to him about the creation of the tabernacle..
And God had said to Moses,.
"We're gonna create this place of worship.
"called a tabernacle that the people are gonna come to.
"to be able to worship me.".
And he said, "In the construction of this tabernacle,.
"I want you to take the gold.
"that you had plundered from Egypt.
"and I want you to use that to shape a place of worship,.
"a tabernacle..
"And this tabernacle will be so glorious.
"because it will be filled with gold and jewels.
"and it will speak of my might and my power and my glory..
"It will be a reminder to you of the things.
"that I have done in Egypt.
"and the freedom I've brought you..
"This place of worship will lead you.
"into the ability to worship me..
"Take the gold and fashion a place of worship.".
Are you following that?.
Now, interestingly in the Hebrew,.
the very same word that Aaron uses here in 32.
for making and taking the gold and making the idol.
is the same word God uses in those chapters.
to talk about making a place of worship..
And in this, you see one of the most powerful,.
one of the most destructive idolatries.
that can happen in the church today..
And this is it..
It happens when we take our place of worship.
and turn it into our object of worship..
Oh, I'm gonna have to preach on this one for a moment..
In taking the gold that should have been put aside.
for the creation of a place of worship,.
Israel takes that gold and they fashion a calf.
as an object of worship..
And when we as a church today take the very thing.
that was created by God as a place of worship,.
a place that makes us come so that we can worship Yahweh,.

$^{601}$and we twist it and turn it.
and we make the place an object of worship,.
then we've entered into one of the most destructive forms.
of idolatry there is,.
and we have disempowered the church.
for its witness in the world..
So when we come in here on a Sunday.
and we come in with the intention.
of enjoying experience together,.
which is higher and more valuable to us.
than the God behind that experience,.
we've entered into idolatry..
When we come in here excited to sing songs.
and to worship together.
and to fill an experience of worship together.
that we think is better than anything else.
that we've ever experienced,.
and we come excited for that experience of worship,.
and we don't think about the God behind that worship,.
we have entered into idolatry..
When we come in here and we pay lip service to our sin,.
when we ignore the conviction of sin.
that God brings into our heart.
through our ability to gather into worship like this,.
when God speaks to us and we pay lip service to it,.
and we don't really react to it,.
we don't really respond to it,.
we don't really repent to it,.
we've entered into idolatry..
We've made the place of worship our object of worship..
I believe that the church is God's gift to change a world,.
but let me be really clear about this,.
there is no power in this building..
There is no power in the chairs you're sitting in,.
in this lights that you see here,.
there's no power in the skill of our musicians,.
there's no power in the words that come from my mouth..
The only power that there is, is the power of Jesus..
And if we come into church and we fail to worship.
where the power truly lies,.

$^{641}$and we worship a subset of that power,.
and we in our human sin make the subset of the power,.
the power that gets to be worshiped,.
then we need to repent..
Then we are so caught up in idolatry.
that we have disempowered the effectiveness of the church..
They take the gold and they fasten an object of worship.
when they should have been using that gold.
to create a place of worship..
Let me encourage you with this,.
take the gold that is inside of you.
and create a place of worship..
Here's the third thing..
They say at the end here in verse four,.
then they said this, "This is your God, O Israel,.
"who brought you up out of Egypt.".
This is really fascinating to me..
The third thing they do in this process of idolatry.
is that once the golden calf is then presented.
before the people, they say these words.
to the whole of Israel, "This is your God.".
And they actually then define who this God is..
They say, "This is the God who brought us up out of Egypt.".
Now this is really interesting.
because the God that brought them up out of Egypt.
is what God?.
Yahweh..
Yahweh has brought them out of Egypt..
So they present this calf and notice what they do..
They present this calf.
as the physical earthly representation of Yahweh,.
which is really interesting.
'cause we think idolatry by its definition.
is the exchange of worship of one God.
for the worship of a completely different God..
But actually the most destructive and subtle.
and most powerful form of idolatry.
is not exchanging the worship of Yahweh.
for the worship of something else..
It is reducing who Yahweh is into a golden calf..

$^{681}$They've said, "This is the God who brought us out of Egypt.".
Well, we're not comfortable..
We're not happy with God being distant.
and in a cloud on a mountain..
We want a God who we can see, who we can physically hold,.
who we can reduce down into something that we can understand..
We wanna domesticate God, Israel says..
And the greatest idolatry that happens in the church today.
is the domestication of God..
It is fashioning a God who fits within our comfort levels..
And when we fashion a God who fits within our comfort levels,.
we have basically exchanged God for a changed God..
For over 15 chapters,.
God has been showing himself to Israel.
in might and power, in the plagues,.
most of them which weren't pretty.
and were quite uncomfortable,.
in the reality of the parting of the sea,.
and a God who can hold all creation in his hands..
And then they come to Mount Sinai to personally meet with God.
and now they witness a God who's thunder and lightning,.
who shakes the ground around them,.
who Moses goes up and disappears into a cloud.
and is up there for 40 days and 40 nights..
And in all of that, Israel is thinking to themselves,.
"I'm not sure I can handle this kind of God..
I'm not sure if I fully understand this kind of God..
This God is not controllable..
This God is somewhat unpredictable..
This God is so powerful.
that it is beyond my understanding of what power is.
and I'm not comfortable with that..
So I will take the gold and I will fashion God down.
into something that I can comfortably understand,.
grasp, control.".
The greatest idolatry you will ever do.
is to recast God into something you can handle..
God is not to be handled, he's to be worshiped..
Come on, church..
And if we at the Vine, if we do our best.

$^{721}$to try to place God into some comfortable box.
that we've always known Him in,.
in the last 30 years of our existence as a church,.
God is constantly wanting to try and break out.
of the expectations and the molds that his people put Him in..
That doesn't mean that we can't know Christ..
Of course we can know Him and we can see His character.
and we can trust what the Word says about Him..
We can find ourselves rooted in comfortable ideas.
of theology and knowing who He is..
All that is absolutely true..
But the temptation the church has,.
what I think the enemy does in trying to push the church.
is to say, "We want a God who is always comfortable.".
One of my favorite theologians and writers.
in the 21st century is C.S. Lewis..
If you've never read any of C.S. Lewis's stuff,.
I can't recommend it more..
There's this incredible book that he wrote,.
one of his most famous,.
"The Lion, the Witch, and the Wardrobe,".
where he writes about these children.
who find this wardrobe in their uncle's house.
and they go through it into this incredible land.
called Narnia where there's a fight.
between good and evil happening..
C.S. Lewis writes it in such a way.
to present what the kingdom of God is like..
There's a lion in this particular Narnia.
who is the representation of Jesus Christ..
Then there's all these other animals who talk..
It's a fantasy thing..
There's all the animals that talk..
These three children go into that land.
and they discover a world that they didn't know..
There's this one moment where Susan, one of the children,.
is talking to a beaver..
The beaver's talking..
Yes, it's fantasy..
And they're talking together..

$^{761}$I want to read this part to you..
It says this..
This is the beaver speaking..
As then is a lion, the lion, the great lion..
Oh, said Susan, I thought he was a man..
Is he quite safe?.
I shall feel rather nervous about meeting a lion..
Safe, said Mr. Beaver..
Who said anything about safe?.
Of course he isn't safe, but he's good..
He's the king, I tell you..
When a church tries to domesticate God.
and make him safe, we've made a golden calf..
He's not safe, but he is good..
And his goodness, his goodness is what changes a city,.
not his safeness..
God is not Gandalf on a cloud..
He's a refining fire in our hearts..
That's who he is..
And so these three things, these three temptations.
towards idol worship that I think every church faces,.
the temptation to exchange God, to change him.
and to make him into a way that we understand him,.
the temptation to finish in the flesh.
what God has started in his spirit,.
the temptation to think that we are the ones.
that can accomplish it and not him,.
these temptations that sit behind us.
and with us all the time..
We have to purge these from us..
If we're truly going to be a church.
that can change our city..
And the question we have to ask is where's the hope?.
Where's the hope in this?.
Well, the hope in chapter 32 is that Moses.
sees what's happening and God shows him.
what's happening in the idolatry..
And Moses petitions and intercedes with God.
and God changes his heart and doesn't do.
what he said he's gonna do to his people..

$^{801}$We don't have time to look at that today,.
but we have one more powerful than Moses.
that is standing in the gap for us in our idolatry.
and that's Jesus..
And Jesus comes before the Lord on behalf.
of the church today and intercedes on our behalf.
and stands in the gap for us and says.
that I am the only image you will ever need..
That in Jesus Christ who has become flesh for us,.
he has shown us the fullness of who God is..
And while there is still mystery and wonder.
and power and the unimaginable that is in God,.
which is a beautiful thing, in Christ,.
we get to see what God is truly like..
God is saying, I understand that you need something.
that is tangible to you..
So the incarnation, fully God, fully human,.
is the most tangible expression.
that we could ever have of God..
We no longer need the idols of golden calves.
because we have the person of Jesus Christ..
And when Paul speaks of that hope to the church,.
he is trying to shift them from their idol.
worshiping temptations that are there to disempower them.
and set up for them a worshiping,.
Jesus-centric church community..
And I wanna finish by showing you his vision.
of such a community..
This is found from Colossians 1:15..
He, speaking of Jesus, is the image,.
notice how he talks about it,.
is the image of the invisible God,.
the firstborn over all creation..
For by him, all things were created,.
things in heaven and on earth, visible and invisible,.
whether thrones or powers or rulers or authorities,.
all things were created by him and for him..
He is before all things,.
and in him all the things hold together..
And he is the head of the body, that's the church..

$^{841}$He is the beginning and the firstborn from among the dead,.
and that in everything he might have the supremacy..
For God was pleased to have, notice this,.
his fullness dwell in him,.
and through him to reconcile to himself all things,.
whether things on earth or things in heaven,.
by making peace through the blood shed on the cross..
Once you were alienated from God.
and were enemies in your minds because of your evil behavior.
but now he has reconciled you by Christ's physical body.
through death to present you holy, notice this,.
in his sight without blemish and free from accusation..
That's our hope being freed from idolatry,.
that we can stand before Christ because of Christ,.
holy in his sight without blemish and free from accusation..
If you continue in your faith, established firm,.
not moved by the hope held out in the gospel..
This is the gospel that you heard.
and that has been proclaimed to you,.
every creature under heaven,.
and of which I, Paul, have become a servant..
In other words, Paul's saying, see Jesus..
If you want an idol, if you want an image,.
if you wanna look towards something that's tangible,.
see Christ because in Christ,.
he has reconciled all things to himself.
and if the church lays down its worship of worship,.
its worship of preaching,.
its worship of being relevant in the world.
and turns back to the worship of Jesus,.
we will find where the power truly lies.
to change the city that we so long for..
Oh, I love the church.
but I also recognize the way in which the enemy is at work,.
even against us here at the Vine..
May we, as individuals and as a community,.
deal with the idols that we're tempted to embrace.
and as we come to Christ to deal with those idols,.
may he empower us for the work that he has for us..
Would you stand with me?.

$^{881}$And I wanna pray over you.
and I wanna commission you into this work.
and maybe if you're comfortable,.
you could just open your hands as I pray for you..
Father, in this room,.
there are many things that we hold in our hearts.
in this moment..
Those things are on the individual level in every heart.
and those things are on the community level.
within the church body..
And Lord, we're grateful for this community of faith,.
for the Vine..
And we're grateful that you've called us.
to be community together here..
We're grateful for every person in this room.
that calls the Vine their home church in Hong Kong..
And Father, we believe in the power of the church,.
not in the power of a building,.
in the power of programs,.
in the power of Jesus Christ.
who is alive in the church today..
And Father, we want that power to be seen in Hong Kong.
in this time, in this hour like never before..
Lord, we pray that you would purge your church..
Lord, that's a powerful prayer..
We pray that you would purge the Vine.
of the idols that it holds,.
that you would show me and the other leaders here.
what those idols are.
and bring us to our knees in repentance..
You would show our elders what those idols are.
and you would bring the elders to their knees in repentance..
You would show individuals in this church.
what those idols are.
and that they would humbly submit them.
to the leadership here to repent and pray through..
Father, we come corporately to you in this moment,.
recognizing that we are sinners.
and therefore there is idolatry amongst us..
But recognizing your grace and your goodness,.

$^{921}$that we can stand before you, as the scriptures say,.
without blemish or accusation..
Thank you for the blood on the cross of Jesus.
that does that for us..
And Father, I pray for every individual here too..
And I want you to take a moment just between you and God.
to allow the Holy Spirit to speak to you.
of any personal idols that you may hold..
And as he shows you these things,.
it's not to condemn you,.
it's to invite you into great freedom..
It's perhaps to even pull back the curtain.
and reveal to you a way in which the enemy.
is trying to hold you back.
from all that God has promised for you..
And our individual idols.
are things that we now bring to God in knowledge.
that in Christ, in his death and resurrection,.
we find freedom, that we can repent..
I wanna invite you into a place of repentance this morning..
(gentle music).
Let's not pay lip service.
to what the Holy Spirit shows us individually today..
Take a moment..
All you have to do is say,.
"Lord, I bring this thing of my life to you,.
"and I don't know how to handle it..
"I don't know how to finish it..
"I don't know what to do, Lord, but I confess it,.
"and I ask you to take it from me..
"I ask that you forgive me for the idolatry.
"that I'm holding in this particular place.
"in my life right now.".
(gentle music).
Come, Holy Spirit, come..
Cleanse your church..
Purge us, Lord, of the things that shackle and bind us again,.
things that drag us back into slavery..
Come, Jesus, come..
Jesus, come..

$^{961}$(gentle music).
(music fades).
\newpage



\section{}
\label{sec:QuRsgZhXkqs}
\textbf{2023-11-06 EXODUS - 21 Drawn In [QuRsgZhXkqs].mp3}
\newline
\newline
連結: \href{https://youtube.com/watch?v=QuRsgZhXkqs}{\texttt{ https://youtube.com/watch?v=QuRsgZhXkqs}} ~~~~ 語音日期: 2023-11-06 
\newline
\newline
\hyperref[sec:5UVcdfy_Atk]{\small{< < < PREV SERMON < < <}}
~
\hyperref[sec:index]{\small{[返主目錄]}}
~
\hyperref[sec:eOywIjV2U2g]{\small{> > > NEXT SERMON > > >}}
\newline
\newline
$^{1}$Well, we only have three weeks left in our Exodus series..
Can you believe it?.
Only three more weeks to go,.
and it's been an epic journey..
This is week 22 of the series,.
and if you've been with us,.
even for just a portion of the series,.
I'm sure that you felt challenged by God.
in the areas that He wants to meet you,.
grow you, stretch you, challenge you,.
the way He wants to shape and form you.
to be more like Him..
And in particular, the way that He wants to continue.
to purge out of us some of the brokenness.
that it's so easy for us to carry around with us..
And as we enter into the final three weeks of the series,.
we're really entering into what I would consider.
the core teachings of what this whole series has been about..
So if this is your first time to the Exodus series,.
you've just saved yourselves 22 weeks, well done..
I encourage you to go back and have a listen..
But these final three weeks are really,.
God welcomed us into the very things.
that He's been preparing you for..
And so if you've been part of this journey.
for whatever amount of time that may be for you,.
these three weeks, I think, are the most fundamental,.
where all the threads that God has been doing.
throughout this story are gonna be tied together..
And this week begins it.
off the back of what we saw happen last week..
Last week, we were in chapter 32 of Exodus,.
perhaps the hardest, most brutal chapter of the whole story,.
where just off the back of this incredible experience.
with God on the mountain and God's presence with His people.
and the giving of the law.
and the very second commandment not to commit idolatry..
In chapter 32, we see Israel do that very thing,.
the very thing that God had called them not to do,.
they enter into..

$^{41}$And they take the gold that they had taken.
and plundered from Egypt, and they burn the gold down.
and they fashion a golden calf,.
and they set the golden calf up as something to worship.
because it was tangible to them,.
where God seemed intangible, God seemed far away,.
God seemed this scary thing on a mountain..
They wanted to create a golden animal.
so they could worship something that was more concrete.
and tangible for them..
And I said last week that what we see in that story.
are three temptations that every church community faces..
And I would say three temptations,.
particularly that we here at the Vine face..
And that is the temptation to finish in the flesh.
the very things that God starts in the spirit,.
the temptation to turn our place of worship.
into our object of worship,.
and the temptation to recast God.
into a version that we can handle,.
a version that fits, if you will, within our comfort levels..
And all three of those temptations.
are idolatries in and of themselves..
But I want you to see, and really today is a part two.
of the message from last week,.
I want you to see what these three idolatries.
will move every church into..
All three of these have the power to strip a church.
of the very one thing that gives that church the power.
to change and transform the society around it,.
and that is the presence of God..
You need to understand that the enemy strategies.
about the idolatries that he tempts us with in the church,.
all of them are designed to strip the church.
of the presence of God..
The enemy does not care about churches.
that are without the presence of God,.
couldn't care less about them..
The enemy is not afraid of theological social clubs..
Come on, church..

$^{81}$The thing that pushes back the gates of hell.
is nothing less than the presence of God..
And the presence of God within a church community.
and that presence of God within the people.
of that community and that presence of God.
going from them and through them and out of them.
and flowing into the corridors of power.
and the workplaces that they represent.
and the families that they have.
and God's presence operating out of that,.
bringing peace and love and kindness and goodness.
and gentleness and faithfulness and self-control,.
that changes a city..
And so the reason why God is so upset.
about Exodus 32 and the golden calf.
is not just because it reveals.
what really is truly in our hearts,.
not just because it shows us.
that we've broken our covenantal relationship with God..
The thing that really upsets him.
is that he knows that idolatry.
strips his presence from his people..
And when his presence is stripped from his people,.
his people are stripped from their power.
'cause the power is not in us..
Surely the power is in the spirit of God, amen..
This is why Exodus 33,.
the chapter that follows Exodus 32.
and all this stuff about idolatry,.
Exodus 32, 33, sorry,.
is all about the battle for the presence of God..
And if there's anything that Exodus.
and the whole story is brought us to.
is to the pinpoint, the crossroads, if you will,.
of this battle..
Will we fight for the presence of God in our midst.
or will we choose a different path?.
It's a question that I think God puts before us.
as a church community as we draw the series to a close..
Do we really want his presence?.

$^{121}$Something like really want his presence..
Do we really want his presence in our hearts,.
in our lives, in our families, in our marriages?.
Like do we really want his presence with us.
or do we just want the benefits of his presence?.
I wanna show you this by unpacking Exodus 33..
It's a bit more of a brighter chapter than Exodus 32.
so you can be grateful for that..
But there's some challenges here.
that are sobering also for us..
Exodus 33, verse one, look at this..
Then the Lord said to Moses, "Leave this place,".
talking about Mount Sinai,.
"you and the people who I have brought up out of Egypt.
"and go up to the land that I promised on oath.
"to Abraham and Isaac and Jacob,.
"saying, 'I will give it to your descendants..
"'I will send an angel before you.
"'and I will drive out the Canaanites, the Amorites,.
"'the Hittites, the Perizzites, the Hivites,.
"'the Jebusites, anyites you could ever imagine,.
"'I'm gonna drive them out..
"'Go up to the land flowing with milk and honey,.
"'but I will not go with you.
"'because you are a stiff-necked people.
"'and I might destroy you on the way.'".
God shows up to Moses and he says this incredible thing..
He says, "Despite everything that I've just encountered.
"in Exodus 32, everything about the golden calf,.
"everything about your idolatry, everything about your sin,.
"despite all of that, I am still going to give you the land..
"I'm still gonna come through with my promises for you.".
I mean, I want you to see the depths.
of the grace of God in this moment..
That even Israel in their worst moment of sin.
and brokenness before God and in their relationship with God,.
God does not remove his promises from them..
He says, "You will go up, leave this place.
"and go up to the place that I have provided for you..
"I still have given you that land, the land of Canaan,.

$^{161}$"the promised land that he has long been calling them towards.".
And not only that, but he gift wraps the present..
He says, "I'm gonna send an angel ahead of you.
"and that angel is gonna drive out.
"all the people that are currently living in that land..
"You're not gonna have to lift a finger..
"I'm gonna basically roll out the red carpet for you.
"and you're gonna move into the promised land.".
And you read that and you think, this is a God of grace..
Because this is a God who should come.
and remove everything from his people..
You guys have screwed up, it's all over..
And some of you in the room here,.
that's how you think of God..
You think if you've screwed up, it's all over.
when it comes to your relationship with God..
You think your brokenness and your sin.
or the stuff that you've done means.
that you are on plan D with God, not plan A..
And God says to his people,.
"You're not on some plan D here..
"The exodus is still happening..
"I've still drawn you out to place you in this land..
"The promise has not been removed.".
And that should be an encouragement to some of you here..
But I want you to see something really clearly..
He does say this, "But I will not go with you.".
(laughing).
He says, "I'm not gonna go with you.".
In other words, the cost of idolatry.
is not the promises of God, it's the presence of God..
Oh, I'm gonna preach on this one, stay with me..
Buckle yourselves in..
That the cost of idolatry is not the promises of God..
He's like, "The promise will still stand..
"I promised that to your forefathers..
"I promised that you'd ever wait..
"In fact, I'm gonna send an angel forward.
"to do all that stuff.".
The cost though is that I'm not gonna go with you..

$^{201}$'Cause if I went with you, man, I might..
I might do some things I don't wanna do..
So I'm not gonna go with you..
You can have the promises, but you cannot have me..
And in that, God is putting a challenge before Israel.
that I think he's also putting before us here.
at the Vine in 2023..
He's asking them a deep, profound question.
that he wants them to wrestle with..
He's basically saying this,.
what is ultimately more important to you?.
Having God's promises in your life.
or having God's presence in your life?.
What is ultimately more important to you?.
Having my promises in your life.
or having me in your life?.
Now on the surface, I know how we're all gonna respond..
I want God's presence..
It must be the presence, right?.
And Pastor Andrew's saying it should be the presence..
I'm like, well, take the presence..
I know the natural response for us is to say,.
presence of God, yes, we want the presence of God..
But if you're anything like me,.
which I hope you are sometimes,.
if you're anything like me,.
you'll probably realize that if you dig a little bit more.
below the surface from the cheesy answer.
that you should give and take a fresh look.
at your prayer life and your worshiping life.
and your daily life, you might realize that your answer.
in your heart is a bit more complex than you might like..
If you're anything like me,.
I think we are all susceptible to developing a relationship.
with God primarily for what God can do for us..
That we pursue God often primarily for his benefits..
Because my relationship with Jesus means.
that I will have a better marriage,.
that I'll have a better relationship with my kids,.
that I'll have a better work environment,.

$^{241}$that I'll be more successful,.
my life will generally be better..
I'll be a kinder, more naturally loving, a better person..
Jesus is my guru to help me to live a better life..
Are you with me?.
Now, don't get me wrong..
It is great that you want a better life.
and a better marriage..
And God can absolutely help you to have a better marriage,.
to be better with your kids,.
to work in your workplace better,.
to see flourishing happen..
All of those things, absolutely, God wants for you..
But I want you to see the subtle danger.
that idolatry brings..
And it is that we think that primarily our relationship.
with God is based on what God can do for us,.
the benefits of God, the promises of God,.
the blessings of God..
And if we build a relationship with him.
that is primarily based on what he can do for us,.
our relationship, our actual relationship.
is going to suffer..
Now, let me tell you why it's going to suffer..
That if the basis or the foundation.
of your relationship with God is because he blesses you,.
he's nice to you, he helps you to be a better person..
If that's why you worship him,.
you will ultimately struggle in your relationship with him..
Why?.
Because a lot of the benefits we want from God.
are not objective ones, they're subjective ones..
They're benefits that we desire, that we want..
Not the benefit of salvation.
or the benefit of objective things.
that change us in our relationship with God..
Those are amazing..
But we largely follow him because we want the benefits.
for us personally..
Now, here's the issue with that..

$^{281}$We humans are never satisfied..
And we humans get really upset at God.
when he doesn't give us the blessings.
that we thought he was gonna give us..
The benefits that we thought was going to come to us..
I thought I would be married by now..
I thought I would have a better job by now..
I thought my marriage would have been sorted out by now.
because I believe in Jesus..
And when our benefits of him are largely subjective,.
even if those are good subjective things,.
and when those things don't seem to happen in our lives,.
the thing that actually breaks down.
is our relationship with God..
'Cause if your foundation of that relationship.
is the benefits he brings you,.
and you perceive that those benefits are not coming,.
what's gonna be infected by that.
is your relationship with him..
Are you with me?.
We will drive out the presence of God from our lives.
if we find our presence with God.
based on what he does for us..
He says to Israel, "You can have the promised land,.
"but if your focus is the promised land,.
"you will lose me along the way.".
This is why in Exodus 32,.
we saw Israel enter into idolatry in the very first place..
They got into idolatry like we saw last week.
because Moses was so long on the mountain..
All those timing didn't work out for them..
The benefit didn't work out for them..
They weren't sure about their future..
It was all about them, so they built the golden car..
And when we find our relationship with God.
primarily about what he does for us.
rather than just about him,.
then we will find ourselves,.
we will find ourselves walking in his promises.
without his presence, and there is nothing worse than that..

$^{321}$When you have the promises of God without his presence,.
you have self-help psychology that makes you feel better.
rather than the transformative power.
of Jesus Christ in your life.
that changes you eternally for good..
And so this challenge comes to Israel here in chapter 33,.
and God is saying, "What is it that you are going to choose?".
Ultimately, he's saying this..
"Do you want me, or do you want the things of me?".
And that is what all of Exodus has been about..
It has all been about this one final question..
Do you really want me, or do you just want the things of me?.
Because every step of the Exodus journey so far.
has been God taking his people out of their brokenness.
and bringing them into a place,.
not just so that they would experience his benefits,.
not just so that they would know the promised land,.
not just because of those things..
God has been drawing them out so that he can draw them in..
God draws them out so he can draw them in to himself..
It's all been about him and his presence with them..
It's all been about the power of him being amongst them..
It's all been about him coming down to them.
and living with them and being with them.
and fostering a life with them..
It's been about taking all of these benefits.
so that you could ultimately know who I am..
But it's not about the benefits, the story..
The story is about him with his people..
Are you following me still?.
And the problem we have in the church.
is that we forget that story..
And so often it's easy for us in the church.
to make our experience of God all about the benefits.
and we miss completely that it's ultimately about him..
Whether he blesses me or he does good things for me,.
whether he achieves all of the subjective things.
that I want him to achieve in my life,.
he is still worth worshiping..
That is the heart of what Exodus is all about..

$^{361}$I think that's the heart that God calls his church to..
But if you're anything like me, it's easy to struggle..
This is why actually Moses writes an aside.
in this chapter, verse seven..
He actually steps back and he shows them.
what's happened during the journey..
He says, "Now Moses used to take a tent.
"and pitch it outside the camp some distance away,.
"calling it the tent of meeting..
"And then he, in inquiring of the Lord,.
"would go to the tent of meeting outside the camp..
"And whenever Moses went out to the tent,.
"all the people rose and stood at the entrances of the tent.
"watching Moses until he entered into the tent..
"So Moses went into the tent..
"The pillar of cloud would come down.
"and stay at the entrance while the Lord spoke with Moses..
"Whenever the people saw that the pillar of cloud.
"standing at the entrance to their tents,.
"they all stood and worshiped each.
"at the entrance to their tent..
"The Lord would speak to Moses face to face.
"as a man speaks to his friend.".
Isn't that beautiful, the intimacy of that relationship..
"Then Moses would return to the camp,.
"but his young aide Joshua, son of Nun,.
"did not leave the tent.".
Moses puts this in the story here in chapter 33.
as a bit of an aside to everything else.
that God is trying to say,.
'cause he's trying to remind his people,.
God has been with us..
God has always been with us..
Every single time we pitched that tent,.
he came and was present with us..
And you all came on the outside of the tents.
and you're worshiping him..
And you are now in danger of throwing all of that away.
simply because you're focused primarily.
on what he does for you rather than who he really is..

$^{401}$You need to understand why this is so important for God here,.
why this question is so critical..
And to understand that,.
you actually have to go back a little bit further.
than the Exodus story itself..
You have to go all the way back.
to the original creation of the world in Genesis..
'Cause what you see in Genesis 1 and 2.
is this beautiful picture right at the end of it,.
where God creates humanity and makes humanity in his image..
And what picture do we get at the end of Genesis 2?.
It's the picture of God coming and communing.
with humanity in the garden,.
his presence and humanity side by side..
And there are so many benefits of Eden..
There are so many things that are filled with blessings.
that are found in Eden..
But the most important reality.
of what we see at the end of Genesis 2.
is that God has been created to be in communion with God..
It's really important that you see this.
right at the beginning of the biblical story,.
that our whole being of what it means.
to be made in the image of God.
is to be brought into the presence of God..
And so at the end of Genesis 2,.
this is called shalom, it's called peace..
In other words, this is the way things should be..
We should be in this communion with God,.
his presence with us, our presence with him,.
living in such an intimacy with him..
It's like God is with us as a friend..
That's the way we have been created to be..
And ultimately our fullness of joy,.
our fullness of who we are is found in the presence of God..
That's Genesis 1 and 2..
Are you following that?.
Which is why in Genesis 3, something really bad happens..
See, when sin enters the picture in Genesis 3,.
the result of that at the end of that chapter.

$^{441}$is the exile of Adam and Eve from Eden..
It is the pushing out of Adam and Eve.
from the presence of God in the garden..
It's important that you track with this.
'cause we are being created.
to be in intimate relationship with God,.
but our sin banishes us into exile..
It pushes us out of the presence of God..
And when we're in exile,.
there's a longing that is inside of us..
See, you have to understand that Adam and Eve.
are separated from God's presence because of their sin..
And when that happens, it creates in all of us.
our greatest brokenness and our greatest longing..
See, we've been created for the presence of God,.
and it is only in his presence.
that that brokenness is healed..
So the biblical story is we're created for God's presence,.
but in our sin, we break the reality.
of that presence from us,.
and it creates in us this brokenness.
and this longing inside of us..
And every human being in the world.
has this longing inside of them..
Whether they realize where it comes from or not,.
or whether they realize how to satisfy it or not,.
that longing is in inside all of us..
We don't want to be in exile from God's presence..
We long as a human race to be back.
into a place of peace and shalom again..
This is why Augustine,.
one of the greatest writers in church history,.
St. Augustine, he put it this way..
He said, "You have made us for yourself, O Lord,.
"and our hearts are restless.
"until they find their rest in you.".
Isn't that beautiful?.
You've made us for yourself, but we're restless..
There's this unsatisfiedness in us.
until we find that rest in you,.

$^{481}$until we find that relationship with you..
Until that's rebirthed and renewed,.
we find ourselves in a place of great brokenness.
and restlessness..
It's like I'm not fully the person that I'm supposed to be..
This is why in Psalm 42, the psalmist writes,.
"As a deer pants for water, so my soul longs after you.".
What the writer is saying is not this..
He's not trying to give you a picture.
of a nice deer in a lush meadow.
with an incredibly flowing stream,.
and the deer is lapping it up all happily..
He's giving you a picture of a deer.
who's in a desert, in exile, in wilderness,.
who is so parched and so thirsty.
that if he doesn't find water soon, he's going to die..
And he says that just as a deer would pant.
desperately for water,.
so my soul is panting always for satisfaction..
And that can only be found, the psalmist says, in you..
This is why what we see happen in Exodus 33 is so key..
God puts a crossroads before his people,.
and he says, "Do you want me,.
"or do you want the things of me?".
Because if you decide to remain with just the things of me,.
you are still in exile..
You're still outside of Eden..
And the longing of every human soul.
is how do we get back to being in the presence of God?.
That's what Exodus has been about..
The whole of the Exodus journey.
has not just been about what God can do.
to overcome slavery and oppression..
It has not just been about how God wants to step in.
and bring evil to its knees.
and show his sovereignty over creation..
The whole point of the Exodus story.
leads us to this very point..
The Exodus is the story of Eden restored..
It's the story of God stepping in and saying,.

$^{521}$"I am now coming to you..
"Even in your brokenness and sin, my presence is here.".
God is changing the story..
And he's inviting his people to recognize.
that the story has changed,.
and they can now have the restoration.
of the very thing that they've longed for the most..
So I want you to track with this..
Is that all with you?.
Now when you get that background,.
you can understand why in Exodus chapter 32,.
God is so upset at their idolatry,.
and why in Exodus 33, he says,.
"The promises haven't gone, but I will not go with you.".
Because there has to be a choice here..
You're either gonna choose me or the things of me..
That's your choice..
But everything I've done in this point.
is to bring myself to you.
because I have drawn you out in order to draw you in..
Now notice how Moses responds to this..
Verse 12, "Moses says to the Lord,.
"You have been telling me, lead these people,.
"but you have not let me know whom you will send with me.".
Now God's just said, "I'm gonna send an angel with you.".
But Moses goes, "I'm not satisfied with an angel.".
Anyone ever been not satisfied.
with who you've had to go through life with?.
Actually, don't answer that..
That's a bad question..
(congregation laughing).
God's just said to Moses,.
"I'm gonna send an angel with you.".
And God says, "You asked me to lead these people,.
"but you haven't told me who's gonna go with me.".
In other words, I'm not happy with an angel..
"You have said, I know you by name,.
"and you have found favor with me..
"If you are pleased with me, teach me your ways.".
Another way to translate that, "Show me the way.".

$^{561}$In other words, I need you to go with me..
"Show me the way..
"I don't know how to do it..
"I can't lead these people..
"Some angel's not good enough for me..
"I only want you..
"You have drawn me out to draw me in..
"I am in with you..
"Please do not leave me," Moses is saying..
I wonder if some of us can get that in our hearts..
God, please don't leave us..
We recognize the idolatry that's so in us,.
and Lord, we know that that drives you away..
Lord, we don't wanna be in exile again..
Notice how God replies in verse 14..
The Lord replied, "My presence will go with you,.
"and I will give you rest.".
Isn't that so beautiful?.
Now, I want you to see something here..
The idea of God giving them rest points them back to Eden,.
to that great place of Shalom where they had rest with God,.
but it also points them forward to the Promised Land,.
which was always promised to be a place of rest..
But I want you to see something even more important..
Notice the order God says here..
"My presence will go with you,.
"and then you will know my benefits.".
In other words, if you seek my benefits without my presence,.
you will ultimately be in idolatry,.
'cause you're in the idolatry of self..
And although those promises may not be removed,.
you're settling for the second best thing..
But if my presence goes with you,.
then out of my presence will flow.
all of the things you will ever need..
And the greatest thing you will ever need is rest..
With me, your soul that longs and pants.
after water like a deer will find its rest in me..
Notice how Moses responds..
"Moses said to him, 'If your presence does not go with us,.

$^{601}$"'do not send us from here..
"'For how will anyone know that you are pleased with us.
"'and with your people unless you go with us?.
"'What else distinguishes me and your people.
"'from all the other people on the face of the earth?'".
Moses is doing amazing theology here..
He's saying, "We will not go anywhere without your presence,.
"because if you haven't restored Eden,.
"if your presence isn't with us,.
"then what message do we have to say to the world?.
"We're no different from the Egyptians.
"if we go to the promised land without your presence,.
"'cause it's your presence with us.
"that communicates to the world.
"that we are welcomed back into God's presence..
"It means that you're pleased with humanity again..
"It means that the sin and the brokenness.
"is being paid for and changed..
"You with us is a message for the world..
"You with us says that, hey, we have a God.
"who despite our brokenness wants to be with us..
"We have a God who wants to communicate to us as a friend..
"We have a God who is worthy to be worshiped..
"And if we go without you, we've got nothing to say.".
It's like Moses is saying, "I am nothing without you..
"We can't do this..
"There's no point in doing this..
"We don't wanna be a theological social club..
"We need you.".
And I wonder whether that might be the cry.
of this generation in Hong Kong.
and the church in Hong Kong at this time..
God, we refuse to do anything unless you're with us..
Because we have nothing to say to government..
We have nothing to say to society..
We have nothing to say to our marriages,.
our businesses, our families,.
what it means to raise children..
We have nothing to say if it doesn't come through you..
Your presence, out of your presence,.

$^{641}$flows all the power to transform and change everything..
We can't do it on our own..
There's no power in us, Moses is saying..
So we will not go unless you go with us..
And I wonder whether some of us in this room.
might get that passion fired up in us..
We will not do this without you, Lord..
Basically, what we're saying is.
we're having a holy posture before a broken world,.
and we're saying we are nothing.
without the presence of Jesus..
There is nothing in this world.
that can satisfy other than him..
It is him, or it is nothing..
Are you with me?.
Well, it's easy for us to say yes..
'Cause I wanna say yes to that too..
But if you really put that into practice in your life,.
it will change everything for you..
The sacrifice of the house saying,.
"Me and my house will serve the Lord,".
is a deep, big call..
Not an easy one to make,.
and not one we should ever make lightly..
But it comes out of this place of saying,.
"The presence first..
"The presence first.".
It's amazing what the New Testament writers do..
The New Testament writers pull all of this together,.
and they begin to make the threads of God coming.
and being with his people.
and restoring the world in Jesus Christ..
You see, Exodus is just a foretaste.
of what's gonna happen in Jesus..
Exodus is just like a foretelling of a better story..
The better story, of course, is the gospel,.
the life, death, and resurrection of Jesus..
This is why, in the New Testament,.
the writers look at Jesus,.
and they say, "He is Emmanuel, God with us,.

$^{681}$"the fullness of God now with us.".
Not God on a mountain, not God in some tent,.
not God even in a temple, not even God in a church,.
God with us fully,.
humanity and God together in the incarnation..
Emmanuel now, God with us..
This is why John, when he starts the gospel.
that he's writing, the story of Jesus,.
he says it this way..
He says, "God has now had the pleasure.
"of taking the word and making it flesh.
"and having it dwell with us.".
The word dwell there is the word tabernacle,.
which links back to the passages.
we're looking at in Exodus..
God has decided to come and tabernacle with us.
in his fullness..
This is why, when Jesus breathes out his last on the cross,.
the curtain in the temple is torn, too,.
to allow people to come into God's presence.
and allow God's presence to go out to them..
This is why Jesus stands in his resurrection.
before his disciples and says, "Don't leave Jerusalem.
"until you receive the helper that is to come,.
"that is gonna fill you and restore you.
"in the only way that that can happen,.
"the fullness of my presence with you.".
Which is why, in Acts chapter two,.
the disciples are sitting in that room.
celebrating Pentecost, by the way,.
Pentecost being the festival that celebrates.
the giving of the law at Mount Sinai..
They're celebrating that, and the Holy Spirit.
is poured out on them in such a big way.
because the Spirit is now being written.
on the flesh of their hearts..
This is why Paul, when he writes to the church,.
says, "Hey, here's the thing..
"You are now the temple of the Holy Spirit.
"because he now lives fully in you.".

$^{721}$Our answer is not that we're drawn out of our slavery.
towards the mountain..
We are drawn out of our slavery to Jesus,.
and in that, Jesus puts his fullness in us..
You don't need a needon anymore..
You don't need some mountain with fire and an earthquake..
You don't need a tent or a temple..
You don't, as much as I love it, need the church..
You do, but anyway, that's a whole 'nother story..
But what you do need.
is the fullness of his presence in you,.
and that is available to anyone who confesses Jesus Christ.
as their Lord and Savior..
The Spirit is in us..
The same Spirit that raised Jesus from the dead.
is now in me..
Eden restored..
Are you with me?.
So he has drawn you out so that he could draw you in..
He's met you in the darkest place,.
and his light has shone in the darkest place,.
and it's shone into the darkest place.
to welcome you out of the darkness on a journey towards him,.
and when you find him, to find yourself settling on him,.
not on his benefits, not on the things of him,.
but him himself..
When we were in Egypt, we made a film for you,.
a film that could express and symbolize.
this incredible journey that God does in each one of us,.
and a film that we've created,.
which I hope inspires you to come home..
Have a look at this..
(dramatic music).
(wind blowing).
(dramatic music).
(wind blowing).
(dramatic music).
(wind blowing).
(dramatic music).
(wind blowing).

$^{761}$(dramatic music).
(wind blowing).
(dramatic music).
(wind blowing).
(dramatic music).
(wind blowing).
(dramatic music).
(wind blowing).
(dramatic music).
(wind blowing).
(dramatic music).
(wind blowing).
(dramatic music).
(wind blowing).
(dramatic music).
(wind blowing).
(dramatic music).
(wind blowing).
- He's drawn you out to draw you in..
May his presence be what everything in you is about..
Let me pray for you..
Father, we come before you as a people,.
a long, long Lord..
To live lives that represent what the Exodus is about..
Lives that are given to your presence..
Lives that live out of your presence..
Lives that are drawn into you.
and that out of you are then sent into the world..
Father, I pray that your presence would be.
on all of our lips..
Lord, it would be the thing that we would desire the most..
The thing that would shape and form the vine.
to be the community that it is..
And Lord, just like last week,.
we take the idolatries that we know are in us.
that have the power to strip us of your presence..
And we thank you that in Jesus Christ, we have the answer..
We have Eden restored in its fullness..
That even in our brokenness, in our sin,.
we have one who is dead and we have one who is alive..

$^{801}$And one who has taken the price of that.
and has opened the way for us to be fully known..
And to be fully back in communion with you..
And Lord, I wanna pray now..
Maybe you'd be comfortable,.
you just open your hands with me as I pray this..
Father, I wanna pray now.
for the indwelling of the Holy Spirit.
over every single person here..
That Father, your spirit and your power and your presence.
be there everything, Lord..
I pray Lord for those in this room.
that have not felt your presence for a long time.
would know in this moment right here.
the fullness of your presence, come Holy Spirit..
Lord, I pray for those who feel like.
that dancer in a dark room right now..
Who can see the shards of light breaking in,.
but don't know what to do..
I pray Lord that they would feel your presence now,.
drawing them out of that darkness into you..
Lord, for those in this room.
that are like the dancer on the desert,.
who have found freedom, but haven't found home yet..
Father, for those of us in this room.
where we put your blessings.
and the things of you above you yourself..
Would you forgive us and help us to be home?.
And Lord for every single one of us,.
I pray that we would be like her,.
quiet and attentive in your presence,.
receiving from you the fullness of life..
Come Holy Spirit, come..
I wanna invite you just to take a moment.
before we close our service, to just pray..
Our most classic form of connecting with God.
is through prayer..
(gentle music).
So just take some time to pray for yourself.
and pray with yourself..

$^{841}$Or you might even wanna pray with the people.
that are around you, if you feel comfortable to..
Maybe pray with your spouse or pray with your friends..
Just take some time just to acknowledge God's presence..
Confess the desire that you have.
for his presence in your life..
And invite more of that presence..
And we'll do that in a place of worship..
(upbeat music).
\newpage



\section{}
\label{sec:eOywIjV2U2g}
\textbf{2023-11-13 EXODUS 22 - Be Strong and Courageous [eOywIjV2U2g].mp3}
\newline
\newline
連結: \href{https://youtube.com/watch?v=eOywIjV2U2g}{\texttt{ https://youtube.com/watch?v=eOywIjV2U2g}} ~~~~ 語音日期: 2023-11-13 
\newline
\newline
\hyperref[sec:QuRsgZhXkqs]{\small{< < < PREV SERMON < < <}}
~
\hyperref[sec:index]{\small{[返主目錄]}}
~
\hyperref[sec:O_UT_ubK9BE]{\small{> > > NEXT SERMON > > >}}
\newline
\newline
$^{1}$Have a seat..
So in this penultimate week of our Exodus series,.
this saga of a series that we've been in for 25 weeks or whatever it is,.
we come to the moment where Israel finally departs from Mount Sinai.
and begins to head towards the Promised Land,.
a place that has been promised to them for generations and generations..
Up to this point in the narrative, Israel has actually been camped at Mount Sinai for a year..
They've been there for almost a year..
And when they arrived at Mount Sinai,.
we've seen over the last number of weeks here at the Vine.
that they arrived free but stripped of their identity..
They didn't really know who they were now called to be in freedom..
And we've been seeing in this series that that's so often the case for us,.
that Christ frees us from our sin,.
but we find ourselves wondering who are we now in this freedom?.
How should we now live in our freedom?.
And we saw that God began to speak to Israel deeply about that journey.
that they're in now to discover who they truly are..
He spoke to them about being a holy nation,.
about being a kingdom of priests..
He spoke to them about the idea of a law that should be amongst them,.
a law that we come to understand as the Ten Commandments and the Mosaic Law,.
but it's a law that at its very core is structured to help us to have a relationship with God,.
a relationship with ourselves,.
and then a relationship with the people around us..
And out of all of this identity forming Israel,.
who are trying to be by God shaped into a people.
that could be his hands and feet in the world,.
Israel struggles straight away..
And we saw a few weeks ago that they fall into idolatry,.
the second commandment that God says,.
"Don't make an idol, don't make an image.".
They fall into that reality very quickly..
And in that place of idolatry,.
they begin to live out an identity that wasn't the identity that God had called them to be..
And so God shows up and he asks them a critical question,.
out of this travesty of idolatry..
He says, "Do you want my presence with you or not?".
And last week we wrestled with that question for ourselves..
Do we really want God's presence?.

$^{41}$Do we want his promises more than we want his presence in our lives?.
And so Israel are brought to a crossroads where they have to make a decision..
Are we going to go forward with God's presence,.
or will we walk into the promises but leave God behind?.
And Moses, on behalf of all of Israel, stands up in front of the people,.
and he says to God out loud,.
he says, "If you do not go with us, we will not move from this place.".
In other words, it is you or it is nothing..
That we have nothing to offer this world,.
we have nothing that differentiates us from everybody else in this world,.
unless you go with us..
Your presence, your forgiveness, your power, your grace, your love,.
is what this world needs..
The world doesn't need us as a people, Moses is saying..
The world needs you..
And if you're with us,.
then we can be that hands and feet that you've called us to be..
And we saw last week that that's the cry of our heart as the church too..
That we have nothing to offer Hong Kong,.
we've got nothing to offer our families that don't know Jesus..
We've got nothing unless God is with us..
It is His presence, His power, His grace, His love,.
that transforms us into a people that can then offer the world something,.
founded in the person of Jesus..
Amen?.
And so Israel, they depart from Mount Sinai with this hope on their hearts,.
that God indeed is going to go with them..
And if God is going to go with them,.
not only would they receive the promises,.
but they will be shaped into a community that can reveal His heart.
to the many, many people that so desperately need it..
What's interesting for the readers of the book of Exodus.
is that the rest of the story of Exodus actually exists outside of its pages..
In fact, as we saw last week in chapter 33,.
where God challenged them with His presence,.
from chapter 34 to 40, the remaining chapters of the book,.
God reemphasizes to them about the stone tablets that God had given Moses,.
about the construction of the tabernacle,.
about His sense of glory amongst them..
There's no other story, if you will, that happens in the book of Exodus..

$^{81}$That story is picked up now in the book of Numbers and in Deuteronomy..
And in two weeks' time, when we finish our series on November 26,.
we'll look at the book of Deuteronomy..
Today, we're going to look briefly at the book of Numbers..
Because Numbers picks up exactly from where Exodus finishes off..
And in Numbers, we begin to see the journey that moves forward for Israel..
They leave Mount Sinai and they travel up towards the promised land,.
but they find themselves camping in a place that the Bible describes as Kadesh Barnea..
And Kadesh Barnea is the next significant turning point in their story..
Now, to help you to understand all of this geography,.
let me bring back the map that I showed you right at the beginning of our series..
If you were with us some 20-something weeks ago, you would have remembered this map..
Now, back then, I actually explained to you that there are actually two ways.
of understanding Israel's journey that they took from slavery into Sinai.
and then towards the promised land..
Two routes, if you will..
What's known as the traditional route and what's known as the modern-day route..
The traditional route has Israel starting up here in the land of Goshen..
They then come down here and they cross over here where the Suez Canal is today..
By the way, this is a modern-day map, okay?.
They cross over here..
They then come down here and they stay at Mount Sinai,.
which is here around St. Catharines on the Sinai Peninsula..
And then they go from there up towards the promised land, which is up over here..
That's the traditional route..
The modern-day route sees them up here again in Goshen..
They cross into the land of Egypt, still go down here,.
but they come over here to this place called Neuba Beach.
and they cross over the Red Sea at this point..
And then Mount Sinai is somewhere here in modern-day Saudi Arabia..
And then from there, they journey up to the promised land..
Now, which one of those you believe, well, it depends on how you read the Bible..
It depends on archaeological evidence..
If you were to ask me, I would say that actually,.
I think Mount Sinai is more likely to be here around in Saudi Arabia..
But at the end of the day, it probably doesn't really matter..
What really matters is what's the message that Exodus is trying to bring us..
And it's this message of a people of freedom trying to get towards that promised land..
Now, whether you go with the traditional route or the modern-day route,.
both are in agreement of where Israel goes next from Mount Sinai..

$^{121}$Whether they're starting here or whether they're starting there,.
they journey up to this place right here, which is known as Kadesh Barnea..
So this land here, if you go to the next map, guys, you'll see it on there..
Boom, there you go..
This area here, biblically, is known as Kadesh Barnea..
Now, obviously, in modern days, that has a bit of the modern-day Jordan,.
a bit of modern-day Israel..
In those days, the promised land was up here in this area..
And so they camp here, and it's here where they famously send in the spies.
to go up to the promised land to spy what's up in the promised land.
and bring a report back down to them..
That is something that we actually talked a lot with you guys about in 2021..
If you've been a part of the Vine for a while,.
we did a whole sermon series in 2021 called "A Different Spirit,".
where we looked at this exact story of the 12 spies going to the promised land.
and then coming back with that report..
So today, I'm not going to go into a huge amount of detail on that story..
What I want to do is take you to what that story has implications for,.
for both Israel and for us, to understand that story,.
to get an idea of what God does in the whole movement of the 12 spies.
to the promised land and returning back to Kadesh Barnea..
I actually want to take you now to modern-day Jordan,.
and I want to show you the geography of Kadesh Barnea,.
because when you actually see where Israel ended up being for 40 years,.
then you'll begin to understand what God needs to say to them and to us..
Let's take a look at this..
The laws of geometry teach us that the shortest distance between two points.
is actually a straight line..
But as we've seen in the Exodus narrative,.
Israel's journey from their slavery in Egypt to their freedom in the promised land.
has been anything but..
I mean, much like this canyon around me right now,.
it's been a journey of a lot of twists and turns,.
where God has come and brought them on different routes and different pathways,.
and, you know, it's led them to kind of even wondering,.
are they going backwards more than they're actually going forwards?.
But move forward they must..
At this point in the Exodus narrative, Israel has left Egypt,.
and they're working their way up the stark but beautiful land of modern-day Jordan,.
which is actually why I've come to Jordan today myself,.

$^{161}$to visit one of the seven wonders of the world.
and perhaps the most amazing archaeological site I have ever laid eyes on,.
the lost city of Petra..
Known as the Rose City from the color of rocks from which it is carved,.
Petra was founded by the Nabateans in the 4th century BC.
as a critical city strategically positioned.
on some of the most important trading routes at the time..
By the time of Jesus in the 1st century,.
it had flourished to a thriving city of around about 20,000 people..
Today, only a few of the original structures remain,.
but they stand testament to the wonder of both natural and human achievement,.
and they point towards the breathtaking beauty that can be found in this land..
Of course, Petra as a city didn't even exist at the time that Israel moved through this land,.
but it's actually just up over the mountains here where Israel did make their passage,.
and just a short hike from here, I can actually take us to a place.
where we can see something that was very critical to the Exodus journey..
Petra, the Rose City.
So I've come up here on this mountain just up out of Petra,.
and it's pretty glary up here, but I've come here for a very particular purpose..
The valley right here below me is known as Wadi Arabah..
In the Exodus story, it was called Kadesh Barnea..
Now, this was the place that Israel came and settled in.
as they sent the spies out to the Promised Land,.
which is about roughly about a hundred miles northwest of here..
You know, being up here today gives me a bit of a thrill.
because I can almost imagine the Israelites settled down there,.
hundreds of thousands of them,.
and their emotions would have been kind of that sense of exhaustion.
after that arduous journey that they've taken up from Egypt,.
but also a sense of excitement..
You know, the Promised Land, that land flowing with milk and honey,.
it's just now almost within their reach..
Petra, the Rose City.
So the mountain that I was just standing at near Petra is right there behind me,.
and I've just come down here now to the valley floor itself,.
and this really is incredibly hot, incredibly dry and arid,.
but standing here right now, I can see the mass expanse of land.
that Israel had as they encamped here,.
and I can understand why it was a great place to send out those spies.
to the Promised Land just north up here in this direction..

$^{201}$Now, the narrative of the spies going into the Promised Land.
is found in Numbers chapters 13 and 14..
We know the story well..
Moses sends out the twelve spies to the land,.
and the twelve spies come back with a report..
And first of all, they say, "Hey, the land is exactly as God said..
It's filled with all of His promises..
It flows with milk and honey..
Here is the fruit.".
They bring some of the fruit back to show Moses and the rest of Israel..
But there's a second side of the report..
They also say the land is filled with giants, with fortified cities, with massive armies,.
and because of that, they're filled with fear..
Right at the start of Numbers 14, they cry out with this..
They say, "Why have we come all this way to die right here in this desert?.
Surely it would have been better for us to have gone right back there to Egypt.".
And they actually say, "We would rather have slavery in Egypt.
than come and die in this place, in this desert.".
They're so filled with fear that they no longer see with faith, but they see with sight..
And that sight tells them to go backwards rather than forwards..
So it would be right here, like literally in this place..
That Israel would end up wandering for 40 years..
One year for every day that the spies were in the promised land.
as judgment for their fear and their lack of faith and trust..
You see, God wanted a whole generation of Israel to pass away.
before the new generation would freshly encounter God's promises..
Much like Moses, as he stood before that burning bush,.
how he was commanded to remove his sandals.
so that nothing impure would touch the holiness of God's presence..
So God now wants to remove a whole generation.
so that nothing unholy in his eyes would enter into his promises..
And this begins to set up for us something that is super critical about the Exodus narrative..
You see, our journey to freedom is a journey that God does in and through us..
But it's not a journey where we're silent participants..
We actually get to partner with Him in that journey..
God looks at us and wants us to have faith and courage and hope in His promises..
Sure, it's God that does the miracles..
It's God that moves in power..
But we are to trust and have faith in that work in our lives..
I mean, this is something that Moses has faced throughout his journey..

$^{241}$And it's a call that's going to come on Joshua.
as in 40 years, he's about to take Israel into the promised land..
It's the call to be strong and courageous..
And it's a call that God also places on you in your journey of deliverance with Him..
To be strong and courageous, not within yourself with your own abilities and your strengths.
and the things that you think you do well..
But actually to find that strength and courage in a God that fights battles for you..
In a God who actually can do the miracles, who can move the mountains, who can remove the giants..
And so right here in the desert of Kardesh-Banir,.
we learn something so significant for our Exodus journey..
That it's actually a place where we come to recognize that the strength isn't in us, it's always in God..
And so can I encourage each one of us to lean into the reality.
that in our Exodus, it's strength and courage that is the call upon us..
(soft music).
In our Exodus, it is strength and courage that is the call upon us..
I can't think of any better phrase to begin to draw this series to a close around than that..
Because you see this journey over the last 25 weeks in the book of Exodus,.
it's not a journey just so that you can understand a little bit about that book.
and maybe see a little bit about what God did for a bunch of people many, many years ago..
It's actually a story for you and a story about you..
It's a story about your next steps and how you're now to live your life..
And as we draw the series to a close, my heart for us as a community.
is that we would now begin to live the story of Exodus out in our daily lives..
That strength and courage would be how we live our lives..
It's interesting as the spies went into the Promised Land,.
they come back with this report of giants and fortified city..
They are the two ways that they sum up the challenges that are ahead of them.
in the Promised Land that God has for them..
And you need to understand that there are challenges ahead for you.
for how God wants you to live your life as a Christian..
Challenges ahead for how you're gonna live with your family,.
how you're gonna live with your career, how you're gonna raise your children,.
how you're gonna see your marriage flourish..
Challenges are ahead of all of us..
The issue is not whether there are challenges ahead..
The issue is how are we gonna respond to the giants and the fortified cities.
when they come..
And I wanna show you how Israel responds because there's something here.
that I think we need to grasp as we bring the series to a close this month..
Let me read Numbers 14 verses 1 to 4..

$^{281}$"That night all the people of the community," this is Israel,.
"raised their voices and wept aloud..
All the Israelites grumbled against Moses and Aaron,.
and the whole assembly said to them, 'If only we had died in Egypt.
or in this desert..
Why is the Lord bringing us to a land only to let us fall by the sword?.
Our wives and our children will be taken as plunder..
Wouldn't it have been better for us to go back to Egypt?'.
And they said to each other, 'May we choose a leader and go back to Egypt.'".
I wonder if this reminds you of anything..
This is exactly the same thing that Israel said when their backs were.
against the Red Sea and they saw the armies of Egypt charging after them.
and they turned to Moses and said, "Why have you led us here to die?.
We would rather go back to Egypt than go forward here because we're trapped..
We don't think there's any way to go.".
And now so many years later, as God has done so much in them,.
they're still saying the same thing..
Haven't you seen the giants?.
Haven't you seen the fortified cities?.
Don't you see all the obstacles that are ahead of us?.
It would have been better for us to remain in our slavery..
At least we understood it. At least it was known to us..
Yes, we were suffering, but it was a suffering we had become comfortable with..
We're not comfortable with the uncertainty of a future..
We'd rather take the comfort of slavery..
It's amazing what fear does to us..
Fear will always cause you to choose decisions designed for your comfort.
rather than decisions that are designed for your freedom..
Oh, come on church..
Fear will always cause you to make decisions designed for your comfort,.
not necessarily decisions that are designed for your freedom..
Israel would rather hold on to what was known and comfortable to them,.
even if that meant slavery,.
than move forward to what was unknown and uncomfortable for them,.
even though that was their freedom..
And you have not been saved by Jesus for a safe, comfortable, easy life..
If you carry the name of Christ with you,.
if you say you are a Christian,.
that means that you are the bravest, most boldest, most courageous person that there is..
It means that God has saved you for a mission and a purpose..

$^{321}$And that mission and the purpose is to reveal his heart of justice,.
his heart of love, his heart of grace in this world,.
and Paul writing to the church to help them to soberly understand this call,.
would say to them, "If anybody is in Christ Jesus, they will be persecuted.".
And he says it with a smile..
He says, "It's going to be hard. It's going to be tough..
There's going to be giants and fortified cities,.
and the call is to be strong and courageous,.
because if you're strong and courageous, you'll face the giants,.
you'll face the fortified cities,.
and we'll see this thing called my kingdom move forward..
But if you're weak, and if you hold off of those things,.
and if you would rather have comfort and safety,.
well, you can have comfort and safety without me," God says..
"But if you want me, then strap yourself in for adventure,.
because what I need my church to be doing is to be my hands and feet in this world.".
It's going to take strength and courage for you to live as God has called you to live..
And one of the things we have to ask ourselves is when we get too comfortable and too easy,.
we have to ask ourselves whether we are still living out of the missional purposes.
that God has for us, or whether we've decided to stay in Kadesh-Baniya.
rather than move forward to the promised land..
It's interesting, in Kadesh-Baniya, there were no giants or fortified cities..
And you saw it in the film, there was nothing there..
And when nothing's there, it's easy..
It's dry, it's barren, it's hot, but at least there's no giants or fortified cities..
The giants and fortified cities were in the place of God's promises,.
and that was where He needed His people to be..
And it was going to take strength and courage for them to move forward in that way..
And you need to understand that that's the life that God has called you to do..
It will take strength and courage for you to fight for the gospel..
You're going to have to fight against the giants of a mind and the fortified cities of a heart..
If you want to stand for justice in this world, it's going to take you courage.
to fight against the giants that are found in corridors of power.
and the fortified cities that are found in systemic injustice and abuse..
If you truly want to stand for the poor and the vulnerable and the marginalized in a community,.
it's going to take courage for you to stand against the giants.
that are found in our selfish decision-making.
and the fortified cities of our bank accounts and resources..
See, to live the Christian life is the greatest adventure you'll ever live,.
but it is not for the weak-hearted..

$^{361}$And when Christ invites us into this life, he invites us in with this challenge..
Kadesh Barnea or the Promised Land?.
I don't know about you, but I know Kadesh Barnea really well..
A number of years ago, I was working in investment banking,.
and I traveled quite a lot to our Tokyo office at the time..
And on one of my trips up there, I was having coffee with one of my colleagues.
in the department I was working in, and she was sharing with me.
about how she felt she was passed over for promotion because of her gender..
And I was having a coffee with her and listening to her share that because she was a woman,.
she felt like she didn't get the same promotional opportunities in the bank..
She felt like there was a culture that was patriarchal,.
a culture that was against the promotion of women..
I had a lot of sympathy for her..
I actually felt her heart, and I knew that probably what she was saying was true..
And she was sharing this with me because I actually was really good friends with her boss,.
her direct boss, and she was hoping that I might be able to, over a beer or something,.
have a chat with her boss and maybe help him to understand that this is how some of the women.
in her department or in his department were acting,.
and that perhaps there is a little bit of a culture that we have at the bank.
that needs to be addressed..
And I realized in that moment that the courageous decision would have been.
to have that conversation with the boss..
The courageous decision would have been to go and say, "Hey, I'm a man myself,.
and I understand that there is this culture. I've benefited from that culture myself.".
I realized that I've probably been promoted when maybe some other women in my department.
should have been promoted..
I realized that all the senior leadership in our firm are men,.
and maybe that diversity is not right..
I should have had the courage to sit down with him and say that there are some people.
being overlooked simply because they are not the same gender as the ones that are in power..
The weak response, though, the weak response was to listen to her and tell her.
that I cared for her and heard her, and yet not had that conversation with the boss.
because I realized in having that conversation with the boss,.
my own career would be put in jeopardy..
That maybe if I had that conversation, I wouldn't be able to be as promoted as I had been..
That actually, as a man, I kind of enjoyed the reality of the culture that was in the firm..
I thought I was the one who benefited from it, and I didn't want to put that at jeopardy..
I'd like to stand before you and tell you that I was courageous, but I was weak,.
and I never had that conversation..
And out of all the 10 years that I was in the business world,.

$^{401}$that's the one decision that I regret the most,.
that I never stood up for the person that asked me to stand up for them.
in a culture that needed standing up against..
After that, I carried some guilt and shame around with me for quite a while,.
and I experienced Kadesh Banir..
I experienced what it was like to stay comfortable, but in a dryness of the soul..
And I think I left my colleague in her Kadesh Banir as well,.
where she had longed for something more in the company she was working for,.
and because I was not strong enough to stand up for her,.
she probably remained where she did for longer than she should have..
I think we all know Kadesh Banir well in our lives..
And the question I have to ask myself is, how do I do better?.
Or maybe the question we need to ask ourselves is, how do we do better?.
How do we live more Christ-like in our city?.
How do we have the strength and courage to have the conversations.
and stand up for the vulnerable and the marginalized more in our city?.
How do we find the courage to do the thing that so often we lack in ourselves?.
There are two spies, of course, in the story that have that courage in them..
They're the ones that hold the different spirit..
And I want you to see how they respond to the fear that is amongst their people..
This is starting in verse 6 of Numbers 14..
"Joshua son of Nun and Caleb son of Jephunneh,.
who were among those who had explored the land, tore their clothes.
and said to the entire Israelite assembly,.
'The land we pass through is exceedingly good..
If the Lord is pleased with us, He will lead us into that land,.
a land flowing with milk and honey, and will give it to us..
Only do not rebel against the Lord..
Do not be afraid of the people of that land, because we will swallow them up..
Their protection is gone, but the Lord is with us..
Do not be afraid of them.'".
There's something in Joshua and Caleb that enables them to stand.
against the majority opinion around them,.
against the culture of their day, if you will,.
and say, "We're not going to remain in Kadesh-Benir.".
That if we believe that God is with us, if we believe that He is for us,.
then we will go up and take this land, because He's promised us this land..
There's something in them that I know that so often is not in me..
I want you to see how Joshua and Caleb try to bring courage and strength to their people..
The first thing they say is, "If the Lord is pleased with us.".

$^{441}$I want you to think about this phrase,.
because I think this means so much to every single one of us here..
"If the Lord is pleased with us," which really what Joshua and Caleb are saying is,.
if we believe that God is for us, then we have nothing to fear..
If we really believe that God stands with us,.
then we know that we can go to whatever obstacles,.
whatever giants, and whatever fortified cities might be there..
It doesn't mean we're always going to be successful..
It doesn't mean that everything we put our hands to is suddenly going to be amazing..
But it does mean that we have one who has promised us something,.
and one who's going to fight on our behalf..
It means that we can find the strength and courage in us.
to face the giants and the fortified cities,.
to say that Kadesh-Benir is not where we're going to live,.
if the Lord is pleased with us..
And here's the reality for all of us..
We so often make weak and bad decisions in our lives.
because we ultimately think that God isn't pleased with us..
See, when you think that God is not really that pleased with you,.
you're always going to choose self-preservation in your life..
You're always going to choose a path that takes you towards comfort and security.
rather than the future that God has for you..
And we think so often that God is not pleased with us because of our sin..
Some of us in this room, you know you're in a season where you're struggling with some sin..
And you're not proud of what it is that you're doing,.
and you're struggling to get free from it..
And because of that, your perception is you think that God is not pleased with you..
How could God be happy with you, with the life that you're leading?.
And because you're leading that life, you think God has given up on you..
You think he's distant from you. You think he's angry at you..
And all of that causes you to make more and more weak decisions.
towards the sin rather than the freedom..
And what we have to understand is in the Old Testament,.
when they said, "If the Lord is pleased with us,".
they're trying to work out whether God is actually pleased with them or not..
But we're not people of the Old Testament..
We're people of the New Testament..
We're people of Jesus..
We're people of the ones who live now under a new covenant..
We're people who now live under one who has stood in the gap.

$^{481}$and taken all of our sin and brokenness on his shoulders..
In the Old Testament, they were going, "Is God going to really be pleased with us?".
And we've got to uphold all this law to make sure we get God on our side..
In the New Testament, we live now in a time where we know that Christ is for us..
We know that he stood in the gap and paid the price for our sin,.
so we have this freedom..
Our question shouldn't be, "Is God pleased with us?".
Our question should be, "Because God has chosen to be pleased with us,.
how can I now make decisions in my life that honor the pleasure of Christ in me?".
Let me put it this way..
It's actually not about whether God is pleased with you or not..
It's actually about whether he's pleased with Christ,.
and he is pleased with Christ..
And because he's pleased with Jesus, he is pleased with you..
Because even in your sin, he still went to the cross for you..
Even in the ongoing sin that you struggle with, he has not removed his love from you..
Even in the challenges that we have, his grace is still present with us..
Make no mistake, God hates sin, and that sin is judged by God..
So we don't have the excuse because of Jesus to go on sinning and not care..
We have the call under Christ to continue to submit our brokenness to him,.
ask for his forgiveness, and walk in the power of that forgiveness..
But our reality is we have strength and courage in us.
because we should never guess whether God is pleased with us or not..
He is pleased with me because of the blood of Jesus..
He is pleased with me because I've been made a child, an heir of his..
He is my high priest who stands in the gap and speaks before the Father.
and says, "This one, Andrew, is not perfect, but see him through the blood of Jesus.".
And when you see him through the blood of Jesus, he's forgiven, he's saved,.
he is filled, and he is released..
Where does my strength and courage come?.
Does it come because I think I'm good? No..
Does it come because even in my brokenness he reigns? Yes..
If you truly believe that God is pleased with you,.
you will face giants and fortified cities..
Are you with me, church?.
But there's another bit here..
There's another bit that's even more important..
It's a bit that Caleb and Joshua speak to that I think the church is uniquely gifted for.
more than any other institution in the world..
And to help you to understand this process that Joshua and Caleb speak to in this passage,.

$^{521}$I want to actually take you back to Egypt, to Cairo, actually,.
for the very final time in this whole series..
And I want to take you to a place where people make paper..
I know. Stay with me for a second..
The greatest way that you're going to find strength and courage.
doesn't just come from your relationship with God and the fact that He's pleased with you.
despite the realities of your brokenness and sin..
The place where you'll find your greatest strength and courage will come.
when you understand how paper was originally made..
And so when I went to Egypt, I got taught how to make original paper,.
and I want to show you that experience. Let's have a look at this..
[VIDEO PLAYBACK].
[MUSIC PLAYING].
- We're here today at the Golden Eagle, a papyrus shop right in the heart of Cairo,.
and I'm here with Asma, and today she's going to be teaching me.
how to actually make papyrus paper, which is something that's been done.
for thousands of years. I'm so grateful to be here. Thanks for hosting us..
- Oh, my name is Asma, and I'm going to explain to you everything about this plant.
and how the ancient Egyptian people made the first paper out of this plant..
This is the papyrus plant. It's an aquatic and a tropical plant.
because it needs a lot of water and high temperature to grow..
That's why it grows beside to the banks of the Nile River..
And it's also a holy plant for the ancient Egyptians for two reasons..
The first reason, the flower looks like the sun rays, which is a symbol of Amun-Ra,.
the god of sun for the ancient Egyptians..
The second reason, the stem has a triangle shape..
- Yeah, that looks like a pyramid. - Yeah, which is a symbol of eternity..
- That's great. - So make a paper like this one..
We cut the stem according to the size that we want..
Now I'm going to remove the green cover..
The green cover nowadays is useless, but in ancient times, they used this part.
to make hats, mats, wood baskets, wood sandals, even the small boots..
I would probably chop my finger off if I tried to do that..
You're very skilled at that..
We use this part to make the paper, but this part is very weak and so easily broken..
- Right. - It's like sugar cane, but we don't eat it..
- Right. - Now I'm going to use the hammer..
- I'm moving all the water out like that. - OK, I'm going to try..
Yeah..
- It flattens very quickly. - Yeah..

$^{561}$- And here. Do you see the water? - Yeah, yeah, yeah..
- And after that, we use the ruler pen. - OK..
To make it flexible. Do like you do pizza..
- It's like bread or pizza, right? - Yeah..
OK. And this is just to get the water out and to make it as dry as possible?.
- Flexibly. - Flexible, right..
- OK, it's enough. - OK..
OK. Now there's no water inside it. It's more flexible and strong than before..
- Can I feel? - Oh, yeah..
- It's sticky, right? - Yeah, it's kind of like sticky..
After that, we soak it into the water to reduce the amount of the sugar..
Six days to get this paper..
- After that, between two sheets of carpet... - Right..
I'm going to take the slices and put them in one vertical and horizontal line..
- Would you do that? - Sure..
- So put this vertical? - Yeah..
- OK. - And one in horizontal..
- And you put, like, one on top of it like that? - Yeah..
- So it's kind of like a crisscross pattern? - Yeah..
- OK. - After finishing all the paper, we cover it..
Like this one. And put it under the press machine..
OK, so cover it and then put it under here..
And put it under the press machine for another six days..
- Six days? - Yeah..
Oh, my goodness. OK. Can I just put this in now?.
- How tight? Really tight? - Yeah, yeah, yeah..
But in ancient times, they didn't have that machine, so they used....
What did they use? Guess what..
- Uh, maybe stones? - Yes, maybe stones..
OK, we have here six days in the fresh water and six days under the press machine..
After 12 days, we will have the first paper in the history..
OK..
- Ah! - The papyrus paper..
- Wow. - It's strong and flexible..
So when it's just, like, one strand on its own, it's actually, like, really weak..
But when you kind of create them and pack them all together.
and press them and dry them out, it actually becomes really strong..
Yes, this is because of the amount of the sugar inside..
Right. So the more the sugar there is,.
the more you cross it around and press it together, the stronger it'll be..

$^{601}$Yes, and wait until it's dry, and we can use it easily..
And can you tear it?.
If I do that, I can cut it into slices..
We will see the vertical and horizontal lines, right?.
Mm-hm..
And I can put a little water here to fix it again..
- And one day under the press machine... - You can fix it again?.
Yeah. One day under the press machine, and we can use it again..
- So go back. - OK, OK..
I just need to get my head around that one..
OK, so you can tear it up and kind of screw up the paper..
- Yeah. - But as long as you put the individual crisscrossing again,.
putting them all back together again, pressing it again, it'll be strong again..
- Yes. - That is super paper..
- Yeah. Super paper. - It is super paper. That's amazing..
Hey, well, thanks so much..
I mean, I think that's helped us to really understand.
how something that on the surface appears very weak,.
just a little plant like that,.
can really, through that process, become so strong as it works together.
with that, with the crisscrossing and the sugar..
That's been amazing. Thank you so much for taking time to show us this..
- It's been incredible. - Oh, nice to meet you..
Nice to meet you, too..
Asma is the fastest-speaking Egyptian I have ever met in my life..
[laughter].
I had no idea what she said the whole time..
[laughter].
I hope she's not watching. Bless you, Asma..
Let me show you what Caleb and Joshua say here.
and link this to what you've just seen..
Verse 9, "Only do not rebel against the Lord,.
and do not be afraid of the people of the land,.
because He will swallow them up. Their protection is gone,.
but the Lord is with us.".
Do not be afraid of them. I love what he says here..
He says, "The Lord is with us.".
First of all, he says, "The Lord is with us.".
That's a good thing, because if it's just us, it's bad..
"The Lord is with us.".

$^{641}$But then he says, "The Lord is with us.".
He doesn't say, "The Lord is with Moses.".
He doesn't say, "The Lord is with the good people.".
He doesn't say, "The Lord is with just those who come to church every Sunday.".
He said, "The Lord is with us.".
He's with all of us, in all of our brokenness,.
in all of our weakness, in all of our strength,.
in our giftings, in all of our bad mistakes..
He's with all of that..
And because He is interwoven with us,.
we now are strengthened in Him..
And I love this picture of the papyrus paper..
This is a piece that they gave us from the shop..
I don't know if you can see this in the light at all..
Maybe if I turn to the side, you might be able to see the crisscross pattern..
Can you see that in there?.
Yeah, the crisscross pattern in there..
That's a beautiful picture of what church is all about..
Because on your own, you are pretty weak and vulnerable..
But when we come together in the body of Christ,.
and we realize that on our own,.
we may not be able to make all the decisions that would bring glory to Jesus..
But when we work together,.
when we realize that we are cross in our lives and in our destinies together,.
when we realize that we need the water added to us like that paper needed,.
the Holy Spirit given to a bunch of believers,.
when we realize that at times God presses us,.
brings us into persecution,.
brings us into hard times to press us together.
and to bond our unity even more together as we face the challenges in life..
As we realize that, we realize we become as a community,.
both strong like this and incredibly flexible.
so that we can do the things that Christ has called us to do in the world..
You want to know where your strength and courage comes?.
It comes, yes, from a God who is pleased with us.
and knowing that He's pleased with us,.
but it also comes in knowing that we have one another..
It comes knowing that we're not saved for an individual relationship with God.
and to be on our own in the world..
We're saved so that we can be together as the body of Christ,.

$^{681}$the body of Christ..
For the Lord is with us in all of our brokenness..
He is with us in all of our strength..
He is with us in all of our frailty..
He is with us in all of the things that we're anointed to do by His Spirit..
He is with us in all of it..
And as we commit to one another,.
we grow and strengthen, not just ourselves, but us as a community..
I love it when she took the knife and she went and cut the piece of paper.
and started to pull the strands out..
And if we're honest with ourselves as the body of Christ, as the church,.
there have been many times in church history over the last 2,000 years.
where it feels like a knife has been brought to the church.
and been torn apart..
I would say even during our protest time here in Hong Kong,.
the church was torn apart at times, during COVID even..
And your perspectives on that issue.
and whether you should get the vaccine or not and all of that.
had the challenge of tearing apart the unity of the church..
But I love the fact that with papyrus paper,.
even if it's been torn apart a little bit,.
it just needs to get crisscrossed a bit..
You need to add a little bit more water to it..
You need to press down a little bit..
And then, hey, presto, it's back together again..
And isn't it great that no matter how battered the church gets,.
as the Holy Spirit falls upon us and brings us together again,.
we are reunited in that one prayer that Jesus has..
"My prayer for them, Lord, is that they would be united as one,.
like you and I are united as one.".
Your strength and courage comes, yes, from how God is pleased with you,.
but it also comes from the person sitting next to you,.
from what it is to be called to be community..
And if you came in today feeling isolated and alone,.
you're a part of the body of Christ here at The Vine..
And that should mean something..
It should mean something to us and it should mean something to you..
And when we share our stories and our vulnerabilities,.
when we ask for prayer and we recognize our weaknesses,.
when we know what the giants and the fortified cities are in our lives.

$^{721}$and we ask each other to come together,.
we as a church move from Kadesh Barnea to the promised land that God has for it..
Whatever that might mean for The Church of Jesus Christ here in Hong Kong,.
you are not alone and you have been saved for the most courageous life..
You have not been saved for comfortable, easy decisions..
You've been saved for justice and goodness and love and mercy and grace..
And those things will require a strength of courage in you to walk forward..
Find that strength in a God who is pleased with you,.
no matter what it might be that's going on in your life..
And find that strength in the community of the brothers and sisters around you.
who even in their weaknesses, as broken jars of clay,.
enable the gospel to still continue to be seen..
That is how the Exodus story draws to its end..
And that should inspire us and fill us with great hope. Amen?.
Can I pray for us? Let's pray..
Father, we are so incredibly grateful.
that you have called us to be strong and courageous in our lives..
And Father, like I shared today, we know Kadesh Barnea well..
We know what it is to make decisions that keep us in the dry and weary place,.
the isolated place that we become comfortable with..
It takes courage to face the giants and the fortified cities,.
to live a life that you truly called us to live..
And Father, as we see in the papyrus paper,.
you have brought us together not to live that life alone..
And Lord, we're thankful for our brothers and sisters in this room,.
the people around us. We are your children..
And Father, I want to pray that you would draw us together in a deeper way.
as we draw our series to a close in the next week..
That Father, you would continue to fill us with your spirit,.
the water that binds us together..
That we would recognize that in those moments where we are pressed,.
we are not crushed..
That actually the pressing draws us into a deeper place of unity than ever before..
Father, I want to pray that our strength and courage is found in you.
and is found in our fellowship together..
I want to encourage you in response today to invite some prayer for you..
Perhaps there are some things that are going on for you.
where you would just love and value someone to pray for you..
And so we're gonna take some time as the worship team.
just play gently in the background..

$^{761}$I'm gonna invite you into a space to just pray for one another..
Maybe you came in here with some people, some friends, some family..
I want to encourage you to take some time just to maybe share some of the giants.
or the fortified cities that are in front of you in this moment..
And just invite the person next to you to pray for you..
If you came here on your own, this is a great way..
I know it's a little bit uncomfortable..
But you are a part of the body of Christ..
And so you can reach out to the person sitting next to you, introduce yourself.
and say, "Hey, I'd love some prayer for this..
There's this going on in my life. I can't do this alone..
I would love someone to just encourage me and support me today.".
So I want to invite you to do that..
That means you need to open your eyes..
It means that you need to look at the person next to you and smile at them.
and say, "Hi, nice to meet you. I would like some prayer today..
Could you pray for this for me?".
And then they're going to pray for you..
And then you can pray for them..
And then in a moment, we'll be closing our service together..
So let's spend some time in prayer together..
together..
\newpage



\section{}
\label{sec:O_UT_ubK9BE}
\textbf{2023-11-20 God's Work in the World (Gary Huegen) [O\_UT-ubK9BE].mp3}
\newline
\newline
連結: \href{https://youtube.com/watch?v=O_UT-ubK9BE}{\texttt{ https://youtube.com/watch?v=O\_UT-ubK9BE}} ~~~~ 語音日期: 2023-11-20 
\newline
\newline
\hyperref[sec:eOywIjV2U2g]{\small{< < < PREV SERMON < < <}}
~
\hyperref[sec:index]{\small{[返主目錄]}}
~
\hyperref[sec:of7A7Q1wg7I]{\small{> > > NEXT SERMON > > >}}
\newline
\newline
$^{1}$- Well, thank you, Andrew, for that almost kind introduction there..
That was very nice..
Good morning..
It's a great privilege to be with you..
Thank you all for giving me this opportunity to share this morning with you..
One of the great privileges of the job that I have is that I get to travel.
around the world and be with the people of God in a variety of churches all.
around the world..
And this morning, I do want to begin by just affirming what social commentators.
and sociologists are observing about our era, which is that it's an extraordinary.
era of disorientation, and instability, and confusion..
All over the world, humans are experiencing by far the most accelerated.
pace of technological change, cultural upheaval, and data stimuli.
in human history..
And here in Hong Kong, these years have been especially difficult..
I was last here in 2019 as social unrest was mounting..
Then this shocking pandemic descended, followed by very sharp economic.
uncertainties, growing geopolitical tensions, and then a war in Eastern Europe.
and war in Middle East..
Globally, it's very easy to sense that an aching confusion, and fear,.
and grief is growing even more intense..
And especially in advanced economies like Hong Kong, where sociologists see an.
unprecedented rise in what they now call the statistics of sadness,.
anxiety, social isolation, depression, loneliness, addiction, and suicide..
So I think the pressing question is this, in such difficult and disorienting times,.
what clarity can the followers of Jesus stand on?.
Or are Christians just as confused and disoriented as everybody else?.
Or worse, is Jesus perhaps just as confused and caught off guard.
as everybody else?.
Personally, I don't think He is..
And so maybe the followers of Jesus don't need to be either..
Especially as you, I think, are emerging out of this conversation you've been.
having with the book of Exodus about our own journeys to freedom..
A freedom for what?.
Because this whole question of disorientation, it does seem to require a.
return to our most basic orientation as to our purpose as Christians..
Why are we here?.
Jesus has an answer to that question, but it's an answer that makes zero sense.
to the instincts of the world or to the intuitions of our human nature..
But here it is..

$^{41}$The reason that we are here on earth is to be agents of Christ's redemption in.
broken lives and in a fallen world..
Put another way, the people of God are most who they are meant to be precisely.
when the world is least as it was meant to be..
Matthew 5 tells us that Jesus calls His followers, you and I, the light of the.
world and the salt of the earth..
Which sounds great until we realize that light finds its purpose and power where?.
In darkness..
And salt, a first century preservative, finds its purpose and power in decay..
Here Jesus is simply trying to help us, I think, find our footing by giving us.
clarity about our redemptive purpose in what is a manifestly fallen world..
Because with clear purpose in mind, why we're here, we have clear expectations.
about the journey..
Christians are not meant to live grimly, wallowing in all that's dark and decaying.
in the world..
Quite the opposite..
For as I'm going to speak to in a moment, the only way to sustain our redemptive.
role in the world is by drinking deeply from life's beauty and goodness and joy.
and abundance..
But Jesus does invite us to experience the chaos, confusion, and heartache of our.
world very differently because of two things, an orientation of purpose and a.
supernatural spiritual preparation..
As to the orientation of purpose, it's meant to be singularly redemptive,.
which means God has to send us to where things are wrong..
The people of God are most who they are meant to be when the world is least as it.
was meant to be..
I was thinking about this the other day, and to me it came to mind as the.
difference between an airbag in a car and a cup holder in a car..
Cup holders serve their highest purpose, and it's a great purpose..
It's amazing it took humans so long to invent the cup holder, but we have..
And cup holders serve their highest purpose when everything is going just right.
in the car..
You and your best friends are on a holiday and you're driving in your rental car down.
the highway, and amazingly there's no traffic..
And the right song on your playlist has come on, and right there by your side is.
your perfect beverage, right there in the cup holder..
In that perfect moment, the cup holder is everything it was meant to be..
Now the airbag, by contrast, serves its highest and best purpose when everything.
is going desperately wrong in the car..
At the point of high-speed collision when steel is crushing in and chaos is hurling.

$^{81}$soft bodies into hard objects, this is the moment the airbag lives for,.
to rush into, not away from the chaos, to place itself between the violence.
and the vulnerable, and to save lives..
And I think in a fallen and broken world, Jesus says we're meant to be more the.
airbag than the cup holder..
Because if you think about it, in a fallen world, what is God's plan for making.
it believable that He is good?.
Jesus says to His people, "We're the plan.".
And He doesn't have another plan..
He says to us, "You are the light of the world.".
But how fallen should we expect the world to be?.
Well, when Jesus tells His iconic Good Samaritan story, His example of the.
neighbor in need, what does that look like?.
It's a man who's grabbed by robbers, stripped naked, beaten, and left for dead..
This is our neighbor..
The Good Samaritan story is our central calling to love others..
But notice it's not a cute story, actually..
Jesus is setting an expectation about the kind of world into which He's sending us.
to love..
Second, Jesus is expressing an expectation about the kind of love that we are.
actually capable of..
Jesus believes you and I are capable of supernatural love..
He thinks we're capable of heroic airbag love, love that moves into pain and.
into peril..
Now, that kind of sounds inspiring, but also kind of terrible..
But this is what makes the love of Jesus supernatural, supernatural..
It's a love that is above and beyond our human nature, a love that only comes.
through a spiritual transformation of us by the power of the Holy Spirit..
In the story of the Good Samaritan, Jesus gives us a really vivid example of.
natural love in the priest and the Levite..
What do they do?.
They walk on the other side of the road..
They're looking for their purpose in Jericho, where everything is going quite.
well..
The Samaritan, by contrast, manifests supernatural love that moves into the.
pain of the man who is left along the road..
But why is this supernatural love so important to Jesus?.
Because it is supernatural acts of love that end up pointing to God and giving.
Him glory..
Because the quality of love is so manifestly beyond what humans are.

$^{121}$naturally capable of..
So let's make this practical..
What does this kind of supernatural love look like?.
Well, as Andrew was saying, I work for International Justice Mission, and many.
of you have been wonderful supporters and encouragers of that work here at this.
church for many, many years..
And IJM now consists about 1,500 IJM staff who are local indigenous Christians who.
live and work in some of the poorest communities around the world..
And as the body of Christ, they lean into some of the darkest and most evil.
violence in the world..
In the Philippines, for example, our local teams are taking on the nasty scourge of.
online sexual exploitation of children..
Yes, there is such a thing, and it is as horrible as you could imagine..
And it's the sickness that's gone viral in the poorest neighborhoods since the.
pandemic..
Basically, sexual predators in London or New York or Moscow, they pay criminals in.
the Philippines to be able to direct live the sexual abuse of children online..
Children like this eight-year-old boy, Marco, who lives in a slum under the.
control of an abusive relative..
Now, this crime has exploded during the pandemic..
And Philippine authorities, they get thousands of case referrals every month.
from around the world about Filipino children who are being abused..
Another example of leaning into the darkness, at a time when many countries,.
including my own, are faced with problems of police abuse, Christians in Kenya are.
facing a particularly terrifying plague of police violence..
Police routinely just round up people who are poor and throw them in jail as a way.
of extracting bribes from them..
During the first few months of the COVID pandemic, the Kenyan police killed three.
times as many citizens from enforcing the COVID quarantine than actually died from.
the disease itself..
Some of you friends might remember that three of IJM's own were murdered by.
police in Kenya..
That's leaning into the darkness..
Third example of leaning into the darkness, in South Asia, Christians are.
trying to address the suffering of millions who are trafficked illegally.
into slavery..
Pachiama, for example, is a young woman who was trafficked into a rock quarry as a.
young teenager, and she's forced to work seven days a week, 12 to 14 hours,.
breaking rocks with a hammer..
The owners used terror and sexual assault to enforce their will upon scores of.

$^{161}$enslaved people..
Pachiama herself had to watch a young child die in the quarry because the owner.
wouldn't allow the child to be taken to a doctor..
And as the world knows, and as you probably know, millions of the world's poor are.
being crushed under the cruelty of modern slavery like Pachiama..
And of course, in advanced economies, the body of Christ is also facing challenges..
We all face them..
The global Me Too movement some years ago surfaced this reality of abusive sexual.
misconduct that women and girls face in all kinds of socioeconomic circles..
And human trafficking exploits immigrants, refugees, and vulnerable populations in.
every country..
Altogether, the pain and upheaval at home and abroad is daunting, and it can feel.
pretty overwhelming..
So where do we begin?.
I suggest that we can begin by returning to the clarity the scriptures provide.
about why we exist..
As the Apostle Paul wrote in Ephesians 2.10, we're God's handiwork,.
created in Christ Jesus to do good works, which God prepared in advance for us to.
do..
We know, of course, that these works don't save us, but we know those works bring.
glory to our Father who is in heaven when they manifest His supernatural love in.
the world..
This is why we're here, to manifest a supernatural love..
So where would that supernatural love come from?.
I have seen Christians in the poorest communities in the world battling.
incredible darkness and evil, and I see three qualities to their supernatural.
love..
The first is a supernatural courage in their compassion, a supernatural.
willingness to draw near to the suffering and to be actually touched by it..
When IJM's undercover investigations in the Philippines first exposed that.
children in the slums were being sexually abused online to paying customers around.
the world, no one wanted to look at that..
Who wants to watch that video and get close to that pain?.
It was too grotesque and shameful..
But soon, over time, IJM found some churches and church leaders in the.
Philippines who were willing to sound the alarm..
They spoke about it from their pulpits, began to pray together about it by the.
thousands..
And here at the Vine Church, you came alongside some of our Philippine and.
South Asian teams and provided prayers and partnership that then led to this.

$^{201}$second quality that I see, not just the courageous capacity for compassion,.
but also a supernatural generosity..
Because after IJM worked with the authorities to conduct operations,.
we've done almost 1,500 operations rescuing kids out of these places..
Here's the problem..
There's nowhere for the kids to go..
No one wanted to take care of the kids..
It's not that they didn't want to, it's just it was too messy..
Because many times, the victims are actually babies or boys,.
and those weren't usual categories for this kind of care..
And so no one knew who to deal with it..
But there were several church communities in the Philippines that stepped forward.
into the mess..
And they had meager resources, but they opened up assessment centers.
and care facilities for a totally unfamiliar form of abuse..
They took great risks in saying yes to the need before they had all the resources or.
all the know-how to meet it..
Reminds me of the story of getting to know the needs of asylum seekers here..
But now, due to the generosity of this community here in Hong Kong,.
and hundreds of children have been rescued by IJM and the authorities,.
and the Christian community in the Philippines is actually pioneering.
aftercare facilities and programs that are now a model in the rest of the world..
We have a lot of hope for the fight against this evil of online sexual.
exploitation in the Philippines, because we've already seen other forms.
of child commercial sexual abuse reduced by between 75\% and 85\% over the last.
decade in the Philippines..
Turns out that Jesus takes these very small offerings, like loaves and fish,.
and multiplies them to this extraordinary impact..
Likewise, in South Asia, where you've been supporting our work,.
Christians have brought hope by manifesting this super generosity in providing a pathway.
of dignity and independence for people who are otherwise in places of slavery..
When IJM and local authorities were able to actually help Pachiamma and dozens.
of other victims get out of that slavery in the rock quarry,.
Pachiamma and the others faced terrible destitution because floods had destroyed.
their mud homes and swept away all their meager possessions..
Then, coalition of local churches and Christian students rallied their own.
resources with community and government to support, to provide permanent housing,.
running water, electricity, road works..
And now, not only are Pachiamma and her family thriving, but Pachiamma has now.
become the general secretary of the local Released Bonded Laborers Association,.

$^{241}$which has more than 1,500 members..
And Pachiamma and her team, they rescue hundreds from slavery..
For Pachiamma and her community, this is the legacy of what?.
Of Christians responding to brokenness and pain with supernatural generosity..
The third quality of love I see in this is supernatural persistence..
These Christians manifest a love that doesn't go away..
Children, imagine this, children healing from online sexual abuse need love.
for a long time..
Typically for us, for five to 10 years..
Now, who can do that?.
The local church can do that..
They're a permanent presence in that community..
Churches in the Philippines commit to these children's care and healing for years,.
taking care of trauma therapy, spiritual nurture, schooling, medical needs,.
therapy for families, livelihood training..
And here's the real test of supernatural persistence in love..
They stick with it even when it's not appreciated, even when the traumatized.
children lash out at them, even when all the efforts seem to fail,.
because sometimes they absolutely do..
What does not fail is the persistence of the love..
This is supernatural..
Likewise in Kenya, churches in the slums have signed up to advocate for men.
who've been illegally detained..
They do the scary work of confronting the authorities about the abuse,.
but they also do the long-term work of supporting families who are thrown.
into destitution when the breadwinner is thrown into jail..
It's followers of Jesus who make sure that these families are seen,.
that they don't go hungry, that they're not alone,.
that they're prayed for for as long as it takes..
So where does this capacity for supernatural love come from?.
It's clearly the work of the Holy Spirit transforming humans..
But I also notice three things that Christians do to allow the Holy Spirit.
to get started in that transformation..
And when they do it over and over again, the transformation becomes powerful.
and unstoppable..
These believers do three things, I think..
Number one, they take scary baby steps of love..
They pray and they chase joy..
First, scary baby steps..
Because you and I were made for redemptive purposes,.

$^{281}$when you and I draw near to broken lives and the fallen world,.
what's going to happen?.
The Holy Spirit will suggest a scary baby step of love..
And the trick is to take it..
This is not easy, however, because you will have other voices as well..
Our fallen nature will always suggest all the things we can't do..
And our spiritual adversary will also suggest all the amazing things.
that we could do later when we're better prepared..
But the Holy Spirit will suggest some baby step of love you can do now..
But it will be a scary baby step because it's meant to be,.
because it takes you to a place where you actually will need God..
This is necessary for us to authentically experience His power.
and His trustworthiness because that experience of Him.
will then give us the courage for the next scary baby step..
What's important to understand about our brothers and sisters.
in the Philippines and Kenya and South Asia when I share their stories.
is how scary every little baby step of love was for them..
Overwhelmingly, these are actually churches in poor communities,.
serving with deep dignity on thin margins, thin resources,.
thin training, and a thin line of survival, actually..
But lots of consecutive scary baby steps have turned them into giants of faith..
And as they take their scary baby steps, they pray..
They talk to God about it..
Because the baby steps are scary, they actually need Him..
And instead of feeling like, "Oh, we are scared and need God,.
so we must be in the wrong place," they sense, "Oh, this is the right place.".
And then God shows up..
He pours out His presence and power..
And so then they pray more because they're thankful.
and because they need His help for the next scary step..
And then they pray a lot..
At IJM, we've learned a lot from this, which is why all of our teams.
all around the world stop all of our work twice a day, every day..
Stop the work, pray for a half hour..
Do a little more work, pray for a half hour every day..
We tell our God we're scared, that we need help with the latest baby step..
And we thank Him earnestly for how He carried us through the last one..
Finally, we've learned from these believers of supernatural love to chase joy..
This is sometimes the thing that gets missed..
Because these followers of Jesus are able to love for the long haul.

$^{321}$because they continually come up for air and to chase joy..
The Bible says the joy of the Lord is our strength..
And so in ways that might surprise you, I find that my IJM colleagues.
laugh louder and harder than just about any Christians I know..
They dance and sing in celebration of beauty and life..
They play and joke and they get rid of all pretension..
They worship with passion and authenticity the God who has proven Himself to be good..
Then they go back to work..
In chasing joy, you and I too can be refreshed with the Spirit.
for the next scary baby step of supernatural love..
I think we have so much to learn from these Christians.
who are serving in these very difficult places..
And I can tell you, they have loved and would continue to love.
your companionship with them..
Many of you have experienced that companionship as you've connected.
with IJM and these teams around the world..
And thanks to many of you, more than 88,000 individual victims of abuse.
and slavery have come to see freedom and wholeness..
And now, as I mentioned, outside experts are studying these programs.
and find that they consistently produce reductions in violence of slavery.
by between 50 and 85\%..
And now we've identified scores of jurisdictions around the world.
that are ripe for that kind of transformation..
This is a miracle of scale that Jesus seems to be preparing.
after all these decades of offering just small little offerings..
And if you'd like to join that fight, please know that your brothers.
and sisters around the world would love your companionship in it..
These teams around the world would love for you to join them.
in helping them protect some of the poorest communities.
in the world from violence..
And please just know that this work proceeds because of the companionship.
of the body of Christ around the world..
So take the time as you're able, IJM has a little table or booth outside.
to get to know the work and explore ways of connecting with us.
as a prayer partner, as a freedom partner,.
someone who regularly supports the work, to make this rescue.
and restoration possible..
Please feel very warmly invited into all of that..
These are difficult, disorienting times indeed..
I have not been in a church around the world that isn't experiencing that..

$^{361}$But Jesus is not disoriented or confused..
He's crystal clear and unyielding about His sovereign purposes..
And He's utterly determined to joyfully pursue those redemptive purposes.
through you and through me..
Because the people of God are most what they are meant to be.
precisely, precisely when the world is least as it was meant to be..
You are the light of the world..
Let your light so shine among men and women that they'll see your good works.
and then give glory to your Father who is in heaven..
Let's pray together..
Kind Father, thank you that you love us so much,.
that you've given us life, that you've rescued our lives..
And now you want to rescue your world through us..
Father, will you take some word of truth that may be from you this morning.
and allow it to enter our hearts, to take root and to bear fruit.
for the glory of your name, Jesus..
Amen..
(gentle music).
\newpage



\section{}
\label{sec:of7A7Q1wg7I}
\textbf{2023-11-27 Exodus Finale 23 - The Death of Moses [of7A7Q1wg7I].mp3}
\newline
\newline
連結: \href{https://youtube.com/watch?v=of7A7Q1wg7I}{\texttt{ https://youtube.com/watch?v=of7A7Q1wg7I}} ~~~~ 語音日期: 2023-11-27 
\newline
\newline
\hyperref[sec:O_UT_ubK9BE]{\small{< < < PREV SERMON < < <}}
~
\hyperref[sec:index]{\small{[返主目錄]}}
~
\hyperref[sec:lCE_pxD4_D4]{\small{> > > NEXT SERMON > > >}}
\newline
\newline
$^{1}$Have a seat, have a seat. We're in our final week of Exodus. This has been a 30.
week series. If you are new to the Vine, this is your first week with us. You've.
come right at the end, well done. You're here, we're glad you're here. The majority.
of us, I'm sure, we've been journeying, whether you've been here for the whole.
30 weeks or maybe you've been here for some of the weeks of this series, but.
we've been journeying together and what it is for a God to look down on a people.
who are scattered and enslaved and oppressed and have compassion for them.
and want them to go out of a place of slavery into a place of freedom. Not, as.
I've said, just for freedom's sake, but freedom to be able to then be planted in.
the land of God's promises and His blessings. And as we come to the end of.
our series today, we come to this moment where Israel steps into, prepares.
themselves to step into the promises and the blessings that God has and I've.
just so enjoyed this journey with you. But before we get into what I want to.
share with you in this final week, I want to pause just for a moment to honor and.
recognize some incredible people that have worked so hard to pull off a series.
like this. This is the third film series that we've done here at the Vine and.
this one was by far the most complex, the most crazy, the most delayed. One of the.
series that we thought at one point would never even happen. It's taken us.
about five years from conception to completion of the series. We shot this.
series in two Middle Eastern countries over the process of about 40 production.
days over a one-year period during a global pandemic. It was crazy and you can.
imagine that although I'm perhaps the one that you see regularly up here.
communicating this series, the series has really only been possible because of the.
dedication of so many men and women that you haven't seen regularly but have been.
working so hard behind the scenes to make this happen. So I want to honor just.
some of them here today. This is just a kind of a collage of a bunch of our.
production team and our filmmakers that journeyed with us to Egypt and to Jordan..
This person right here in the middle of this photo is Riley Sue. Riley was the.
producer on this series. She's actually been the producer that's worked with me.
over the last three film series that we've done. Riley is an incredible person.
and you can imagine what it would have been like over a five-year period to do.
something like this where you have to get visas and permits and you have to.
hire production companies and get permits and security clearances to film.
in some of the most hostile places in the Middle East and be able to do that.
over a global pandemic with all the delays, all the frustrations. Riley is.
incredible. Can we honor Riley just a bit? She lives here in Hong Kong.
but right now she's visiting her husband who's based in California. So thank you.
Riley for all your incredible hard work. This guy right here is Toby Thomas. He.
was the director of the whole series, the film director. Toby, a good friend of mine,.
he's actually also, as well as a filmmaker, he's an Anglican priest based.

$^{41}$in London. So five years ago in 2018 I called him up and I said, "Hey, I've got.
this crazy idea to do this crazy series." He was in it from day one and we spent.
the first three months just on Zoom together reading the book of Exodus.
together and plotting and scheming and dreaming and praying. A lot of the.
content that we've delivered in this series has come out of those times. His.
creativity throughout this whole series has been fantastic. This guy here is.
Oliver James. Oliver was our director of photography on the shoot in both Jordan.
and in Egypt. Oliver was the guy that was behind the camera when I was in front of.
it doing whatever I needed to do. He was an incredible person. Right here at the.
back is actually Anthony Gibbs. Anthony was the second camera operator, the.
director of photography for our B-roll. B-roll is basically like all the rocks.
and beautiful vistas and pyramids and all the stuff that I'm not in, all the.
beautiful shots. Those were the shots that Anthony did. All the shots that I.
were in were the shots that I did with Oliver and all the other shots.
were Anthony. So if you loved any of the vistas or those close slow-mo shots of.
ancient rocks, that was that man right there at the back. But he was also the.
editor of the whole series. He edited, there was in total about 35 films in.
total across this series. He edited every single one of them. I spoke to him this.
week. He said, "I am so sick of hearing your voice." And I understand that..
This guy down here, this is Devin. Devin was on our shoot in Egypt. He was.
supposed to be on our shoot in Egypt for the whole time, but his wife was pregnant.
and she was gonna give birth about seven weeks after we were finished the filming..
So we figured and Devin figured that everything would be fine. About six days.
into the shoot in Egypt, his wife's waters broke and he had to jump.
on a plane and emergency fly back. They delivered a very healthy child, which is.
fantastic. But he was with us for about six days. So as he was stepping off,.
we were like, "We need someone to step in." And Riley was also supposed to come with.
us on the shoot, but she fell pregnant in all the delays that happened. And so we.
invited this incredible guy right here. This is Benjamin Chase on our production.
staff here at The Vine. And he stepped in and literally at the last minute learned.
how to do all the sound stuff that we needed on the ground. And he's standing.
right here looking at me. Can we give him a lot of love? Yes. I love it. I love it. I.
love it. And as well as everybody who was involved in all the filming side of.
the project, there was an incredible team behind the scenes here at The Vine who.
was working on taking once the films were produced and made, turning those.
films into the reality of what you've seen. And so I really want to honor.
Promise Armstrong and his whole creative team and all the work that they've done.
over this time. Every single person in his team has put a lot of hours into.
this series. I particularly want to highlight Balana Soriano. She's the lady.
that's done all of the graphic design work for this series, whether that's the.

$^{81}$posters that you've seen throughout the series, whether it's the logo that was.
designed, the devotional books that you got, all of that was designed and made by.
Balana. So I really want to honor her. Onshun was somebody in our team who did.
all the translation work. Zoe Chan did the translation itself. She did a.
phenomenal job. And then Onshun took that translation and put it into the films.
so that we have English and Chinese there for every film. That was amazing..
And then finally, I really want to honor Chris Webster, one of our congregation.
members, who spent about a year and a half writing all those devotions that.
you've been reading, and those devotions have been phenomenal. So for all of those.
people, can we put our hands together and just honor them?.
And one of the things that, you know, as the leaders here and pastors here,.
often we get some of the feedback around the series and what God has done, but so.
often we don't hear of all the spiritual work that happens through a series like.
Exodus. And I know that so much has happened spiritually in your individual.
lives over this series. And so what I thought, different from what we've done.
in other series, I would love to actually hear from you about what God might have.
done in your life over this series. And then I'm going to collate together a lot.
of that, and then we'll be able to feedback some of that to us as a church.
at a later date. So if God has spoken to you in this series, if there's been a.
significant breakthrough for you or something spiritually that God has done,.
I want to invite you to email me this week. This is my email. You could just.
shoot me an email and say, "Hey, here's what God has done in my life. I just want.
to, you know, testify really about what God has done through this series." And then.
we'll bring some of that testimony. You know, Ephesians speaks about that.
actually the church is built up when the glory of Christ is made known within the.
church. And I think this will be one of the ways that we can make known the.
glory of what God has done amongst us over these 30 weeks. Does that sound.
all right? All right. Enough of all of that. We'll put that now to one side..
And here, as we enter into our final week of Exodus, we actually enter into a.
concluding moment in the story that sets up everything that is about to happen next..
Because the reality is, at the end of the Exodus, although it's the end of the.
Exodus narrative, it's actually the beginning of a whole new story..
It's the beginning of Israel moving into the promised land and establishing for.
themselves a nation within that land and establishing eventually the city of.
Jerusalem, eventually the temple that's built there. There are so many years of.
flourishing and growth in the promises and the blessings that are ahead..
But as we come to the ending point of this series, we have to now prepare.
ourselves for the beginning point of something new. That was the call on Israel..
Right at the end of this story, the call that God brings them is,.
"Are you prepared? Are you ready now to step into the land of blessings?".

$^{121}$They came to a decision point. And it might sound funny that there's this.
decision point about whether they're going to enter the promised land or not,.
because surely they're going to want to enter the promised land..
I mean, this has been something that's been like a carrot on the end of a stick.
for them for over 40-something years by this point. And they've longed to be in.
that place. So you would think that this isn't much of a decision,.
but God knew that there was one more thing that they needed to transition.
within themselves to enable them to now walk into the land of blessing..
Because remember, up until this point in their kind of existence as a nation,.
all they've known for so many years is slavery, then God's liberation,.
but then a journey without a home. They've been homeless in the wilderness..
But now they're about to move into a new season where they'll have a home,.
and they need to flourish, and grow, and put roots down, and take their place.
in the land of promise. And God understood that for that transition.
to happen, another transition needed to take place. And it's the same also for you..
If you've been a part of this journey at all through our Exodus time,.
you're now also now at a transition moment. Will you step into all of the blessings.
and promises that God has for you? Or will you decide to remain at the distance,.
perhaps free from some of that slavery in the past, but not yet walking in to the.
promises that God really has for you? This challenge is really a challenge,.
not just for the end of this series, but one we've seen almost every step.
of the series on its way. The challenge is essentially this,.
will we move forward in the new despite its uncertainties, or will we remain.
in the old despite its deficiencies? That's the challenge that Moses finds.
himself at as we come to the conclusion of the story. And the fascinating thing is,.
Moses doesn't actually have to answer this question for himself because Moses is.
told by God that he will not enter the land of promise and blessing..
In fact, Moses learns as they get ready to transition in that he is not going to be.
the one that would lead his people into that place of great blessing and purpose..
And I wonder if you could imagine how that must have felt for Moses..
And Moses realizing that he's going to have to hand over his whole leadership.
of Israel, having worked so hard to get Israel to this point,.
he now needs to hand over the leadership of Israel to one of his mentees of the next.
generation, a man called Joshua. And this whole process of him being willing.
and prepared and able to hand over is really what the book of Deuteronomy.
is all about. If you were here a couple of weeks ago,.
you'll remember that I said then that the Exodus story doesn't finish in the book.
of Exodus. When the book of Exodus comes to an end, there's two other books that.
carry on the story, the book of Numbers and the book of Deuteronomy..
Numbers picks up from where the book of Exodus leaves off..

$^{161}$As Israel is departing Mount Sinai, it's picked up in Numbers and they come.
to Kadesh Barnea. And it's where in Numbers that they send.
out the spies into the promised land. The spies come back with the report they.
have. Fear fills the nation of Israel and they decide not to move forward..
And because of that, God says, "You're going to have to wander now.
for 40 years in the wilderness, one year for every kind of day that they were.
in the promised land, spying it out because they didn't have the faith to move.
forward into the promise." That's the book of Numbers..
The book of Deuteronomy picks up right where Numbers finishes off..
The 40 years of the wandering in the wilderness is now complete and now Israel.
is about to move forward towards their first steps in the promised land..
Let me read this to you right at the start of Deuteronomy 1, verses 1 to 3..
These are the words that Moses spoke to all of Israel in the desert east of the.
Jordan, that is, in the Arabah, opposite Suf, between Paran and Tufel,.
Laban, Heresof, and Dischbab. It takes 11 days to go from Horeb to.
Kadesh Barnea by the Mount Seir Road. In the 40th year, on the first day of the.
11th month, Moses proclaimed to the Israelites all that the Lord had.
commanded him concerning them. Right here in this moment,.
Israel is moving. They're moving from Kadesh Barnea now up to a new place where.
Moses is going to draw them together and speak some final commands to them before.
they enter into the promised land. And so let me just remind you of the geography.
of this moment one last time with us. Last time that we did the series,.
two weeks ago, we looked at Kadesh Barnea and the wilderness..
This is where they wandered for 40 years. But now they come on this 11-day journey,.
and they come all the way up here to roughly what is this area here,.
which is the area of the modern city of Amman in Jordan today..
But in those days, it was known as Moab. Arabah was all up in this area..
And this is a mountain called Mount Nebo. That mountain is still there today..
It's still called Mount Nebo. You can go to Amman in Jordan,.
and you can visit that mountain. It is this mountain that Moses comes.
to with Israel, and they stay at that mountain. And God calls Moses to go.
up that mountain so he can see for the very first time the promised land.
for himself, which you can imagine was an exciting moment for Moses..
But as he's looking out, he understands that he's not the person.
that's going to carry Israel into that land. In fact, I want to read this to you..
I want to read this to you from Deuteronomy 34, starting in verse 1..
It says this, "Then Moses climbed Mount Nebo from the plains of Moab.
to the top of Pisgah, across from Jericho. There the Lord showed him the whole land,.
from Gilead to Dan, all of Nathalie, and the territory of Ephraim and Manasseh,.
all the land of Judah as far as the Western Sea, the Negev and the whole region.

$^{201}$from the valley of Jericho, the city of Palms, as far as Zohar..
Then the Lord said to him, 'This is the land that I promised on oath to Abraham.
and Isaac and Jacob when I said, "I will give it to your descendants..
I have let you see it with your eyes, but you will not cross over into it.".
And Moses, the servant of the Lord, died there in Moab, as the Lord had said..
He buried him in Moab in the valley opposite Beth Peor, but to this day,.
no one knows exactly where the grave is. Moses was 120 years old when he died,.
yet his eyes were not weak, nor his strength gone.".
Hands up if you'd like that to happen to you. Come on, come on..
"The Israelites grieved for Moses in the plains of Moab 30 days,.
until the time of weeping and mourning was over..
Now Joshua, son of Nun, was filled with the spirit of wisdom,.
because Moses had laid his hands on him..
So the Israelites listened to him and did what the Lord had commanded Moses.".
Here's that moment. They're on top of Mount Nebo..
Moses is looking out on the promised land, and God begins to speak to him.
and shows him all the land, and he begins to see that it's this land.
flowing with milk and honey, and God says, "You can see it,.
but you can't go in it." And Moses understands that he's now called.
to do one final act. He's going to have to transition his leadership.
from himself to Joshua. I wonder if you could imagine the humility.
that was needed, having for over 40 years led his people to the very prize that God.
had always said that they would have, and yet he can't grasp the prize..
Transition in any kind of leadership is perhaps one of the hardest things we.
as humans have to do. And some of you in this room are in great places.
of leadership. Every single one of us in this room is a leader..
We are people that have people around us, that we have an authority over,.
that we have a sphere of influence within. How is it possible for us.
to transition leadership? This is the significant thing that God.
wants Israel to wrestle with as they step into the promised land..
It's the significant thing I want us to wrestle with as we step out of this.
series into all that God has for us in the future. To help us with that,.
for the very last time, I want to take you to the land of Exodus..
♪ [music] ♪.
The journey for Israel from their slavery in the land of Egypt,.
their escape across the Red Sea, their journey into the Sinai wilderness.
and up and down Mount Sinai, onto the desert of Kadesh Barnea.
and the 40 years of wandering, all of it has led to them now beginning.
the final leg of their journey as they now travel east and then north.
up the Jordan Valley and camp in the plains of Moab, just a stone's throw away.

$^{241}$from the Jordan River and their eventual crossing point.
into the promised land itself..
♪ [music] ♪.
So Moab is surrounded by a series of elevator ridges known as Ebrim,.
the tallest of which is Mount Nebo, rising some 2,300 feet above sea level..
This is the mountain that God calls Moses to ascend right in that final chapter.
of Deuteronomy. So from its summit, he can survey the promised land..
And I'm sensing that God might be calling me to ascend that mountain.
as well today..
♪ [music] ♪.
So it would have been right here in roughly about 1400 BC that Moses would have stood.
and for the very first time looked out on the land of Canaan..
I mean, this had been a 40-year journey to get to this point,.
and finally they're standing here. And you can imagine for Moses what it would.
have felt like. God had always promised to him that he would be able to see.
the promised land. And now in this moment, he's able to gaze out at the coming.
of a promise..
♪ [music] ♪.
Having just been in Egypt myself and having just retraced the journey.
that the Israelites went on, I can imagine just how incredibly excited Moses.
would have been standing in this place. I mean, they've just seen incredible.
miracles take place to get them here. But they've also had to wrestle.
with the difficulties and the hardships of their identity and their sin..
And so getting here, seeing the promise, there would have been a sense of praise.
and worship for Moses. But I think there was also another emotion at play..
And we get a little hint of it from Deuteronomy 34.4. For there,.
God reminds Moses that although he can see the promised land,.
he'll never be able to actually enter it. We know actually back in Deuteronomy 3.
why this is the case. Moses had actually disobeyed God in that moment of the rock.
and the water. And he had actually not listened to the way in which God had asked.
him to do that miracle. And even worse than that, later on,.
Moses actually claims credit for the miracle himself. And because of all.
of this, God's anger burns against Moses, burns against his unfaithfulness,.
and the way in which his leadership has now pulled down what God was wanting to do.
with his people. And because of that, Moses would end up dying on this mountain..
And instead, it would be Joshua that would lead his people into the promised land..
So this mountain represents for Moses both exuberant joy and sober reflection..
I mean, he's right here at the end of his life. And I think he would have sat.
on the mountain and reflected back on all that has happened for him..
I mean, he's seen God face to face, seen some of the most incredible miracles.

$^{281}$that we have recorded for us in the Scriptures. And he's even had the word.
of God in such profound ways that it's shaped all of history ever since..
But Moses along the way has also encountered the frailty of the human heart..
He's seen the power of sin and what that can do to people,.
both the Israelites themselves and also himself personally. And I think all.
of that would have been in his heart and mind as he sat here on this mountain.
on that day. And he knew that there was one final act that he had to do,.
and that was to pass leadership from himself to Joshua, the one who would take.
the Israelites into the promised land. And to help us to think a little bit more.
about what that transition would have been like, I want to take you.
to another mountain. It's actually a mountain that means a lot to me..
It's in my city of Hong Kong. And I want to introduce you to one.
of my spiritual fathers and talk to you a little bit more about the profundity.
and power of transition in leadership..
Tony, what a privilege it is to stand here on this mountain to look out over our city.
of Hong Kong that we love so deeply. And I think both of us have lived here..
I've probably lived here over 30, maybe over 40 or so years for you now..
- Yeah, been there. - In this city..
And to talk about transition, leadership transition, and I'm so blessed that.
yourself and John Snellgrove, the former senior pastors at the Vine Church,.
handed the church over to me at a relatively young age, at a time when you guys were.
at the height of your ministry, you handed the church over, you transitioned it..
And I don't think a lot of churches do that. So I wanted to just ask,.
why did you guys believe so much in leadership transition?.
Yeah, I think to understand that, you have to understand a little bit of about the church.
and what its constitution was. And we had a lot of young people in the church..
And it wasn't just that they were attached as part of the church..
They were the powerhouse of the church. They were the energy..
They were the passion that tended to provide the energy for who we were as a church.
and almost give us that next generation idea of where we felt the church was going to go..
And clearly, you know, God speaks about taking the church through generations.
and his promises from one generation to another..
Right..
And so it was clear to us that, you know, neither John and I were really next generation people..
Here we were somewhat late in our age, having the incredible privilege of running a church.
and yet seeing that God's vision and God's projection was so much further than maybe where we could take it..
Right, right..
And so this idea of wanting to transition to a leader who was part of that generation was born in our minds..
It's a big thing to let go, you know, and I think one of the things I really honor and admire in you and John.
is just your ability to have done that, to be able to let go of something that you had birthed..

$^{321}$I mean, if you think of the church like a child, you'd birthed the child, you'd raised the child..
The child had grown up and now you were sort of handing that child on..
That's a lot of letting go..
And I think of Moses, you know, as he's standing there on Mount Nebo, right?.
And he's thinking about looking out on the promised land that he's never going to enter..
He's having to let go of his own dream of walking into that promised land..
What was that letting go like for you?.
In some ways, that letting go was, it wasn't particularly painful, but in that process of letting go,.
there's a little bit, you know, whatever we talk and say about leadership,.
there's a little bit of our identity in leadership..
And so you begin to lose that sense of, yeah, I'm the pastor of the church..
And you have to then start looking for other ways to serve the church, other ways to be involved..
Because unlike many processes, of course, we did not disappear off the scene..
We were still around..
And so there was that....
You didn't die on Mount Nebo, like Moses..
I did not die..
That's good..
Not my time yet..
That allowed us, I think, to be able to be present in a way which did not interfere in terms of your decisions.
and the way that you were going, but allowed us to let go and find other ways of providing input and involvement in the church..
So there was, in a sense, it was not a sharp cutoff,.
but it was a gradual way of finding other ways in which we could be involved..
And I think, you know, all due respect,.
encouraged to you to be able to take on a church while the two former senior pastors are still hanging around..
Yeah, exactly..
I think about, again, Moses on the top of Mount Nebo, right?.
He's looking out on the promised land..
He's going through all this stuff in his head around the mistakes he had made..
And actually, part of the reason why he wasn't entering into the promised land was the Lord's punishment to those mistakes that he had made..
What mistakes do you feel like when you look back on the process that you didn't quite get right?.
And how does that make you feel now?.
I think we probably assumed too much..
And when I look back on the process, there was a sense in which probably at your young age, you know, you could argue maybe we dropped you in it too soon..
But there was a sense. And so I think, you know, that there has been that sort of struggle of leadership and things that you've had to dealt with that maybe we could have thought about a little more carefully..
But for us, I think we had to be reminded of the fact that we needed to let go..
And for me, I believe that was the best decision that we ever made in terms of our leadership in the church..
And I think it's been the most fruitful decision that's been made in the church..
So one thing I wanted to ask you was also about the process of transition, because I mean, there's the decision, which we've touched on already and how God was involved in that decision..
But then there's the process. There's the actual walking out in obedience, the thing that God's called you..

$^{361}$What was the process like for John and yourself?.
Well, I think, you know, you have to understand that transition is, everyone thinks about it just as change..
But transition is both change and continuity..
And so if you want to look at it in sort of broad terms, you could say that Moses' job was to get Israel up to the River Jordan..
Yeah. And then Joshua's job was to take them through that and into the promised land..
Yeah. And so there is that continuity is the people..
It's the church. Right..
And this is, I think, the reflection of God's heart is for his people..
And the fact as leaders, it's in a sense, it's not about us..
It's about the people of God and it's about that continuity..
And of course, you see that, you know, going right through scripture..
Yeah. So that gives you not just an abrupt change, but a certain process that leads you through..
And each part needs to understand what their part in the leadership process is..
Yeah. And that's not always easy..
No, not at all..
Final question really would be around just your reflections on Moses, you know, that that moment of him on the on the top of Mount Nebo..
You know what? Given that you've been through the process yourself, obviously in a very different way,.
but you know, you've had to hand on something that you've nurtured and cared for for so long..
What do you think Moses was feeling when he kind of I mean, we're looking out on Hong Kong, which is our our passion..
But he was looking out on the promised land, his passion..
Yeah. What were some of the things that you think he was processing in those last moments of his life?.
You know, you could say he might be processing the what ifs, I shouldn't haves..
Right. But I think that in that process, you know, it says that God took him and, you know,.
there was that sense in which he must have sensed I've achieved what I was asked to do..
Yeah. And I need to be satisfied with that..
Yeah. Not just satisfied with it. I have to take joy in that fact..
And I have the sense that he would walk away from that, you know, at his 120 years old and wherever it was that,.
you know, God buried him. Yeah. Yeah. With a sense of, yeah, that was good. Yeah. I've done that..
Well done, my good and faithful servant. Yeah, exactly..
As the Joshua in our context, I can say that to you..
I've said that to you many times, but it's an honour to say that to you again,.
just like the incredible job you did in nurturing me, nurturing our church, handing it over,.
doing so so open handedly, despite some of the fears and the challenges of that..
You did an incredible job. And it gives me such a beautiful model for my future as I think about doing that in my context..
So thanks, Tony, so much. It's my pleasure. And thank you for doing what you're doing..
You know, I know the burdens of some of that. You know, the burdens of a lot of it..
Yes. You face those burdens. So thank you for what you're doing..
I appreciate it..
It's hard to describe to you what it's been like to have two people like John and Tony in my life.
who have been able to nurture not just the vine, but myself as well,.

$^{401}$to a place of being able to walk into what God has for it..
And the way that they did that so open handedly, so honestly and with so much humility, I think is truly profound..
And we as a church are very blessed to have former senior pastors like the two of them.
that have done that in such a beautiful way in our midst..
And I think what you see in Tony there, I think a lot of that is in Moses as he's standing there on Mount Nebo..
Moses understands that God had called him for a time and a season..
I love one of the things that Tony says in the film. He says, "We realized we could see the trajectory,.
the projection of where God wanted to take the vine, but we knew that we weren't the ones that could get the church there,.
and we needed to let it go." And I think Moses on Mount Nebo, he's experiencing that same kind of thing..
I think he understood that he was God's appointed person to lead Israel out of their slavery,.
out of that oppression, bring them into the wilderness, meet with God at Mount Sinai,.
receive the law and bring them to literally just steps away from the promised land..
But he also knows he's not God's person. He's not God's anointed person to lead Israel now into the flourishing that's ahead..
Because Moses understood that the whole journey of Exodus was really not just about getting Israel out of a place of slavery and oppression..
It was actually also about getting them in now into a place of blessing and promise and abundance..
It was those two things. And Moses had the great joy of seeing them come through that first stage..
And now Joshua is going to have the great joy of taking them on..
And I wonder if you could imagine the humility that needs to sit in Tony and John's heart.
to hand over a church when they were still relatively young and had a lot of energy in them..
And how that humility must have been on Moses as he now understands that in order for the people of God.
to get the greatest of what it is that God has for them in the future, he has to let go..
And it is this act of letting go that I want to finish our time on today speaking to you about..
Because in the very same way that Israel were called and destined to step into the promised land,.
but they had to have this letting go of one leadership and the taking up of another leadership..
So it is also true for us that as we move from one place to the next,.
as we think about what it is for us to move into a land where we can flourish and grow.
and the promises of God can be with us, we also have to recognize that there needs to be, if you will,.
a leadership transition that happens within us..
A transition that happens within us when we think about the person that we have been in the past.
and the person that we now are today..
And to realize that the person that we were before that did all of those things that got us to where we are today.
is perhaps not the person that's going to be able to sustain us in the future that we have..
We need to transition all the time within ourselves towards the person that God is calling us to be today..
Are you with me on this?.
It's almost as if, if you will, we need to let our inner Moses die so our inner Joshua can arise..
Because there is a season of blessing..
I know this in my spirit. Honestly, it's hard for me to communicate this..
I know this in my spirit that 2024 is going to be a year of incredible blessing for the vine..
If you think about it, this whole year has been about our journey of Exodus..
And today we're about to step towards 2024..

$^{441}$I believe as we step into 2024, it's going to be like our promised land..
But we have a transition we need to make in order to get in there..
We need to allow ourselves to realize that the last thing we want in the promised land.
is to be led by the person who brought us there, but is not the person that can enable us to flourish there..
See, it's interesting..
There is nothing more counterproductive in the Christian life.
than living in the land of God's blessings and promise.
while still being led by the season of slavery in the past..
There is nothing more counterproductive in the Christian life than that..
To be where God has called us to be, and yet our mindset and our thinking and who we are.
still being rooted in the season in the past..
And so God brings Moses up to Mount Nebo..
He shows him the promised land and he says, "It's not you that's going to lead these people forward..
It's got to be Joshua..
You need to hand over, let go, transition to the person who can truly help to get you there.".
And it's the same for us as we come out of our Exodus series.
and we want to walk into the promise of this incredible city that God has.
and all the things God wants to do in us..
We have to go..
There is a person inside of me that helped me to get where I am today..
But that person is not the one who can sustain me and flourish me in the season ahead..
I also need to transition my soul, if you will, towards the one who can truly set me free..
This is the thing, and we have to be really careful with this..
You don't want Moses to lead you where only Joshua can take you..
You don't want Moses to lead you where only Joshua has been anointed to take you..
This is why Paul, when he writes to the church, he says, "If anybody is in Christ Jesus,.
you have to understand one fundamental thing..
If you are in Christ Jesus, you are what?.
A new creation..
The old is gone, the new has come..
The person you were before is not the one who should be making decisions now that you're new..
The person that you were before and the story that you had before,.
that story of pain and sin and that story of liberation and what God has done,.
that's a great story..
It's been a part of who you are, but that is not the story that should be now telling you.
how to think, how to act whilst you're in the land of promise..
You need to let Moses go and you need to take up Joshua..
And it shouldn't surprise us that Joshua, the very name, is a derivative of Jesus..
That Jesus and Joshua essentially is the same name because Jesus becomes our great Joshua,.
the one who leads us into the promised land of eternal life with him,.

$^{481}$a promised land of his grace and his mercy and his forgiveness,.
a promised land where the blessings flow upon us from generation to generation.
as we sung about before..
It is Jesus that we have to, on a daily basis, transition ourselves to..
I don't want my inner Moses to lead me where only my inner Joshua can take me..
I don't want my previous story to be the primary thing that makes me think now,.
"Here, I want the mind of Christ..
I want Jesus, who is the only one who knows what the land of blessing is truly like..
I want him to lead me into all that he has for me.".
But it's so easy for us to hold on to who we were before and even to think that,.
even in the blessings, it really is me, the broken thing before,.
rather than I am a son or a daughter or a child of God..
I am one that God is forming and shaping..
I am more than a conquering Christ Jesus, that the same spirit that raised Jesus.
from the dead now sits in me..
I have a story to take..
I will take my place..
Come on, church..
I will now walk into the land of blessings and promises,.
not with the mindset of everything that's happened before,.
but with a mindset that has been renewed by Christ Jesus,.
which is why Paul writes to the church and says,.
"If you want to live a life as a living sacrifice, renew your mind,.
because it's now our Joshua, Jesus, that leads us forward.".
Now, here's why this is all really important..
This is all really important, because everything that Jesus is doing inside of you internally.
throughout this whole series has really actually not just been about you..
It's actually been about us..
See, what we've seen throughout the whole of the Exodus journey.
is not just God's work on individual people to change and transform them..
Obviously, the most notable of that, Moses, but also Aaron and Hur and Jethro..
Pharaoh himself goes through a transformation, Pharaoh's daughter..
There's Moses' mother..
There's the midwives that we saw right at the beginning of the series..
God has come and worked on individuals every single step of the way..
But with that, there's been a bigger story unfolding..
And that's been the story of God forming out of individuals.
that are being transformed by him into a worshipping community together..
That the point of Exodus for God is not just that individuals would experience freedom,.
but that they would be created a worshipping community.

$^{521}$within which they can be together far more powerful than they would ever be on their own..
Everything that we've seen at Mount Sinai with the law,.
everything that we see on Mount Nebo as Moses hands over leadership to Joshua,.
and as Joshua takes a nation, a powerful nation,.
that is now going to represent the purposes and promise of God in the world,.
it's about their individual stories of redemption and liberation.
now intertwined together to create an even greater story of the church..
That's what the church is all about..
We are a combined beautiful story of the grace of God..
And together we are always much more powerful than we would ever be on our own..
Yes, God has changed and transformed us personally,.
but he's drawn us together so that our stories of liberation.
are intertwined with one another to create a much more powerful voice in the world.
than we could ever, ever imagine..
Jesus calls it the kingdom of God..
Back on April 16th, the very first week of the Exodus series,.
we did something symbolically together..
I wonder if some of you were here on that particular Sunday..
We had a basket here in the front of the auditorium..
And on that Sunday I spoke about, as we introduced the idea of Exodus,.
that Exodus actually means in the Greek a departure..
And I challenged you as a church to think about.
what are you departing from in this moment as the series begins?.
What is it in terms of your brokenness or your sin.
or a relationship that you want changed or emotions or habits that you want to break?.
We all have a point of departure that we want to move out from..
That's what Exodus is..
And so as I challenged you on that very first Sunday.
to think about your point of departure,.
there was a little piece of mosaic tile under every single one of your chairs..
And I told you at the end of the service,.
get that mosaic tile, that mosaic tile represents for you.
that thing that you're departing from..
And as we were worshipping together,.
you came forward and you placed that little tile in the basket..
And that was your prophetic act of saying, "I want to change..
I need to depart..
There needs to be an Exodus happening inside of me.".
And it was really a beautiful thing we did individually on that day..
Well, what you don't know is that we took up all of those individual little pieces of tile.

$^{561}$that you had placed in the basket across our three services,.
and we did something really special with it..
And we did something special with it because in the act of doing that,.
we were actually communicating what the whole story of Exodus has been about..
And to help you to see that, I want to show you the very, very final film..
And in this film, you will see what Exodus really means..
[VIDEO PLAYBACK].
[MUSIC PLAYING].
We took every single piece that you placed in the baskets on that Sunday..
[APPLAUSE].
And a man in Amman, Jordan took every single one of our little bits..
And as you saw in the film, he took them and he cut them and shaped them and formed them..
And then he placed them individually, one by one, into this beautiful mosaic that he created..
And I think this speaks to us of what this whole 30 weeks has been about..
Because this is exactly what God does with Israel..
Because at the start of the story, there are scattered, broken people..
And God, out of his compassion, gathers them up..
And he speaks hope into them..
And he says, "I've seen your suffering and your slavery, and I have a good promise for you.".
And he picks them up and he takes them to Sinai,.
and he begins to shape them and mold them into the people that he needs them to be..
And he draws them together as a new family and a new community..
And then he releases them into the promised land, filled with the hope of something new..
And this is us..
Lots of individual broken things that, by the grace of God and by his Spirit and his love, he's brought us together..
And he is making here in Hong Kong something more powerful than we would ever be on our own..
You matter..
And your story of Exodus is actually part of a much bigger story of Exodus that God is doing amongst us..
It's a beautiful, profound thing to think that God orchestrates all of this for his glory..
But as Moses hands over his leadership to Joshua, recognizing that it is Joshua who will provide flourishing in the land of blessings..
As we close this series, may I encourage you to continue to hand over the transition of leadership in your life to Jesus, your Joshua..
The only one who could ever sustain and flourish you in the land of blessings..
That, my friends, for the very final time, is what Exodus is all about..
Can we pray together? Let's pray together..
Father, I'm just so grateful for the individual mosaic tiles of people that you have here in this room and online..
Incredible people that you love so deeply despite so often our frailties..
We thank you that you gather us up out of your compassion and grace and you shape and mold us into a new future..
And Father, we thank you that you sent your son Jesus to be our ultimate Joshua..
And that you provide a promised land of the kingdom of God for us to walk into..
So that we may know the flowing milk and honey that you promise to your people, but we know it by the flowing of your spirit in us..

$^{601}$Lord, would you help us to take up our place?.
Would you help us to be the very people that you've called us to be by your spirit in us?.
Would we take on the mantle of you leading our lives?.
And would we recognize the times and the moments where we try to lead ourselves?.
Would we understand that a transition of our soul is always needed to flourish in the land of God's blessings and purposes?.
And Father, I thank you that you've taken us individually on that journey..
But you've taken us corporately and communally on that journey as well..
And for that, we give you honor and glory and praise..
Lord, as we come out of Exodus as a series, we now step into the promised land of our city of Hong Kong..
And we do so with all the hope, with all the life, and with all the joy that your spirit places inside of us..
Jesus, would you lead us where only you can take us?.
And we pray this in Jesus' name..
Everyone says, "Amen.".
Would you stand with me and we're just going to close our time with a final moment of worship together..
[BLANK AUDIO].
\newpage



\section{}
\label{sec:lCE_pxD4_D4}
\textbf{2023-12-03 Of Prophets and Prostitutes [lCE-pxD4-D4].mp3}
\newline
\newline
連結: \href{https://youtube.com/watch?v=lCE-pxD4-D4}{\texttt{ https://youtube.com/watch?v=lCE-pxD4-D4}} ~~~~ 語音日期: 2023-12-03 
\newline
\newline
\hyperref[sec:of7A7Q1wg7I]{\small{< < < PREV SERMON < < <}}
~
\hyperref[sec:index]{\small{[返主目錄]}}
~
\hyperref[sec:c_wVd_9_9_Y]{\small{> > > NEXT SERMON > > >}}
\newline
\newline
$^{1}$the last number of weeks as well..
Can we give another round of applause.
to our youth and all that they've done?.
(audience applauding).
So great, welcome to the Vine..
If you're relatively new to the Vine, my name's Andrew..
I'm one of the pastors here, and it's just so great,.
oh, sorry, it's just so great to be with you..
Oh..
(audience gasping).
Sorry..
Sorry, yeah, thanks..
Thank you, yeah, just, that's fine, thank you..
(audience laughing).
Things don't always work out.
how we plan them in life, do they?.
I think if we're honest with ourselves,.
actually, so much of our lives is a little bit of a mess..
And no matter how perfect we might try to make stuff appear,.
the reality is we can often struggle, can't we,.
with keeping things perfect in our lives..
That's so often, if we're honest with ourselves,.
we would have to admit that we're pretty messy,.
and at times, we can be pretty messed up..
And I think us humans have become really good.
at actually projecting a version of ourselves.
into the world that looks perfect to everybody around us,.
to kind of mask or hide the reality.
that there's still some messiness.
and some messed upness in us..
Or maybe I'm the only one who does that..
Anyone else here maybe do that?.
You don't have to put your hand up..
Okay, put your hand up if you do that..
Yes, I wanna see those hands..
Boldly proclaiming that we can do this, can't we, in life?.
That we know that there's some stuff in us.
that we're not proud of, that we don't love,.
and so we project into the world something.
because what we ultimately wanna do.

$^{41}$is kind of distract people.
from seeing the true person that we are..
Magicians call this misdirection..
In fact, if I, let me just show you this..
If I was, for example, going to make this coin disappear,.
what I would do is I would show you this coin.
and I would tell you that this is a Hong Kong \$5 coin,.
that it's a real coin,.
and then I would put it on my thing like this.
and I would do this,.
pretending that I'm gonna make it disappear,.
but then I'd accidentally drop it on the table.
and then I would put it back in my hand.
and I would start doing this again.
and then suddenly it would disappear..
(audience laughing).
Wow!.
You're welcome, I'm here all week..
(audience laughing).
What you may not realize, though,.
is that the coin is actually up here..
There..
Misdirection..
Distracting people over here.
so they don't realize what's truly going on over here..
Are you with me?.
Here's why I share all that with you..
I think in the world's modern day celebration of Christmas,.
we have one of the most subversive misdirections.
that there has ever been created..
I don't know if you've realized this,.
but the world has secularized Christmas.
over the last number of years..
That what we have in the world is,.
"Hey, look at this thing over here.
"that we now define and call Christmas.
"and it has nothing actually to do often.
"with the reality of the true, original Christmas.".
Are you with me?.
It's kinda like this sort of thing of like,.

$^{81}$"Hey, let's even try and get all of Jesus out of Christmas.".
Have you seen X-mas written around the place?.
Let's just get rid of Christ from Christmas.
'cause that story's just a little bit uncomfortable.
and not everybody believes in Jesus,.
so we can't talk about Jesus at Christmas,.
even though the holiday is actually about Jesus..
And so we'll call it seasons greetings, happy holidays..
And we'll try to make it look as benign and as nice.
and really all of this about family and celebration..
And we'll try to project this image.
of what Christmas should be to distract the world.
from the reality of what Christmas truly is.
about somebody who was born into this life.
to give people great hope, who on the cross died,.
gave up his life so that we may have grace and forgiveness.
and we could know the fullness of life..
But we don't want anyone to know that story,.
so happy holidays to you all..
Are you with me?.
Now, even when Jesus is included in the Christmas story.
in the modern day celebration of it,.
it's normally a saccharine version of Jesus..
It's normally the perfect Jesus that you've ever seen,.
or maybe the Hallmark card that you buy for someone.
to give to them on Christmas day presents the nativity scene.
in the most perfect and beautiful way you've ever noticed..
Let me give you an example of it..
I found this one online this week..
Ah, isn't that just the perfect nativity scene?.
You got the beautiful star in the background..
It looks like a wonderful oasis.
that you might wanna go to holiday for..
Look at that sheep, the perfect sheep ever..
The sheep is staring gazily into Jesus's eyes..
There's Jesus lying down in a halo of sunbeams,.
even though he's just been born..
Are you with me, people?.
So we turn the Christmas story into this kind of perception.
that we're comfortable with, that we think, you know,.

$^{121}$makes us feel good, when the reality is.
that's as far away from the reality.
of that first Christmas as possible..
And then we do this in movies and we do this in stuff.
like whatever Christmas movie you're gonna watch.
over the Christmas period other than "Die Hard,".
it presents a version of the Christmas family.
a lot like this one, right?.
Like the perfect Christmas family..
There they are, they can't get whiter than that..
They're surrounding by the tree behind them..
There's beautiful lights everywhere..
I mean, it is a beautiful scene..
This is Christmas..
And that makes us all, here's the ironic thing,.
that makes us all want to have.
the perfect Christmas ourselves, don't we?.
Like we want Christmas to be perfect..
We want it to look just like that..
And so we decorate our homes and we call our friends over.
and we cook the food and we make it perfect..
And we want it to arrive on time..
And if it doesn't, we get really angry and upset.
and it's super stressful to make everything super perfect..
But we think we need a perfect Christmas.
because the world is telling us.
that Christmas has to be perfect..
The funny thing is we, even as Christians,.
spiritualize the perfectness of Christmas..
Oh, if I could just pull off the perfect Christmas.
for my family this year, Jesus, we'd be glorified..
Like if the food is perfect, the turkey's nice and moist,.
if all the presents are brought.
and put under the tree at the right time,.
if my kids are happy with what I brought them,.
Jesus will be glorified in our midst..
I wonder, in our desire to try to create.
the perfect Christmas, that we're actually in danger.
of missing out on what Christmas truly is..
Or to put it another way, is our striving.

$^{161}$for the perfect Christmas simply Christmas misdirection?.
Think about it for a moment..
Because the first Christmas was an absolute mess..
Nothing went right with that first Christmas..
I mean, if you want a perfect Christmas,.
do not open the pages of the Gospels..
I mean, just stop for a moment and think about.
what you see in that first Christmas..
Number one, a crisis teenage pregnancy, awkward..
Number two, an unexplained and unexpected road trip.
at the last minute..
An arrival and the accommodation is messed up..
A not very ideal birthing environment for a child..
And then, to top it off,.
three weird old astrolobers from the East.
come over and visit you..
And then, just to add more to that,.
there's a genocide that happens in the very place.
where the child is born..
That's a messed up Christmas..
That should make you feel better.
about some of your messed up Christmases over the years..
But that's the reality of the first Christmas..
It's so messed up and so chaotic.
and the people involved in it are so confused.
about what's happening that they have no idea.
that they're getting swept up.
in this redemptive moment of history..
The first Christmas is so messed up.
and it's just as God wanted it..
It was quite literally Christmas..
See what I did there?.
Yeah, yeah, yeah..
Thank you, thank you, thank you..
This Advent, we want to take you.
into the messiness of Christmas..
Because it's actually in the messiness.
where the message of Christmas sits..
And if you try to do some misdirection,.
try to ignore the reality.

$^{201}$that the first Christmas was a little bit messy,.
you actually kind of miss what the hope.
of the first Christmas is all about for you and for me..
I mean, think about the incarnation for a start..
God decides to send His perfect representation of Himself,.
the perfect manifestation of Himself.
in the incarnation of Jesus, fully God and fully human,.
born into that smelly, horrible,.
not like the Christmas card,.
that smelly, horrible, dark, damp cave.
that He was born into..
And He was there in the messiness to tell us something,.
that no matter how messy the world is,.
no matter how broken and chaotic this place can be,.
and you could probably think that the world today.
is just as messed up and chaotic and broken as it ever was,.
that Jesus is happy to be born in that place..
That He's happy to be born in a place that isn't all right,.
that He's not interested in some misdirection,.
that He actually wants to meet you, yourself,.
in your personal brokenness, in your messiness..
That the beauty of the incarnation.
is that Jesus did not come for a perfect world.
filled with perfect people..
If that was the case, He wouldn't have come at all..
He's come to an imperfect world filled with imperfect people.
because in meeting the imperfect people,.
He might bring us into the righteousness,.
not the perfection, the righteousness of Christ,.
so that we might know a saving relationship with Him..
And when He comes again, Advent,.
when He comes again at the end of all things,.
we are presented to Him holy and pure,.
presented to Him without blemish as a bride before Him.
because of the work He's done inside of us..
That's why Christmas is messy,.
'cause it says something to us about our messiness..
And so over Advent, we're gonna unpack.
the messy moments of Christmas together..
Next week, we're gonna look at that really messy time.

$^{241}$with Zachariah and the angel in the temple and his silence,.
and then the messiness of Mary having her pregnancy.
and everybody thinking that she had sex before marriage..
The week after that, we're gonna look at the family,.
Mary and Joseph arriving in Bethlehem,.
and nobody in their family welcomed them in,.
not because, as the carols sometimes tell us,.
that the whole place was filled,.
but because nobody in their family wanted anything to do.
with a pregnant, unmarried teenager..
And then the week after that,.
we're gonna look at Jesus when He's just eight days old.
in the temple, and the prophets turn up and speak to Him.
and say some pretty messy stuff to Mary and to Jesus,.
stuff that would come to define the things.
that would happen in their lives in the years ahead..
And in this, our hope for you and for us.
is that we would strip the misdirection of Christmas from us.
so that we would see Christmas.
as it was truly meant to be seen..
It might surprise you, but the gospel writers were intent.
in stripping the misdirection out of the story.
of the life of Jesus..
In fact, Matthew and Luke, who are the ones.
that tell us the birth story of Jesus,.
they do everything in that birth story.
to try to communicate to you.
that this was not a perfect time,.
that this was not just as everybody else.
thought it should be,.
although it was exactly as God had planned it to be..
And I think no more powerfully do they show us this.
than in what both of them do.
in presenting to us the genealogy of Jesus,.
genealogy, that's a fancy word to talk about the bloodline,.
the family before Him that came to make who Jesus is.
and was..
Now, that genealogy is presented to us.
in both Matthew and in Luke,.
Matthew chapter one, Luke chapter three..

$^{281}$I'm not gonna read you in detail those genealogies today..
You can do that during the week..
I encourage you to do that..
What I want to do today is tell you the stories.
about the people who are in the genealogies..
'Cause I think in understanding their story,.
you might understand Jesus's story better,.
and therefore you might understand your story better..
You see some incredible people in the genealogies.
in Matthew and Luke..
You see people like Abraham, the father of the faith,.
the one who began Israel itself,.
the one who was faithful to God.
and did so many things as an act of love and grace to Him..
You see characters like Isaac,.
Isaac who was incredibly obedient..
In fact, one of the best examples we have in scripture.
of somebody who's obedient to God's call on their life..
We see the prophets like there's Hezekiah and Amos.
in the genealogies,.
great prophets who held God's word.
and had the courage and the boldness.
to speak that word into the world..
And it was a beautiful thing..
We see great moral leaders in those genealogies..
People like Boaz and Jesse, people like Elam,.
ones who walked out this idea.
of what it meant to live for God,.
the kingdom of God in the world today..
And of course you see kings..
You see kings in the genealogy,.
David, his son Solomon, Rehoboam,.
all kings that led Israel at different periods of time..
And when you see the genealogy like that,.
you can begin to understand why Matthew and Luke.
include such amazing figures in their genealogy..
See, genealogies in the first century.
were designed to tell you something about the individual..
If you wanted to say to someone,.
"Hey, this individual is worth your time.".

$^{321}$You would tell them the family that they've come from..
We don't do this in Hong Kong society at all, do we?.
Oh, okay, sorry, too close..
All right, so, but you know what I'm saying?.
Like who you've come from,.
the people who've gone before you.
say something about who you are..
And so when Matthew and Luke are trying to present Jesus.
as the Messiah, they're trying to fill the genealogy.
with people that speak of the glory.
and the goodness of this person..
Matthew, who's writing predominantly to a Jewish audience,.
starts from the place of Abraham, which makes sense,.
'cause he's trying to convince Jewish people.
that Jesus is their long-awaited for Messiah..
Luke, though, Luke's writing to a Gentile audience,.
mostly a Gentile audience..
So he doesn't start with Abraham..
He goes all the way back to Adam..
Why does he go back to Adam?.
'Cause Adam's the father of all people,.
regardless of background or race or culture..
And so Luke takes us all the way back there to say,.
"This Jesus is for everybody.".
And in their presentations, they're basically saying,.
"Here's Abraham, here's the kings,.
"here's the miracle workers, here's the prophets,.
"here's all these people that make Jesus.".
And we do the same, don't we?.
When we tell people stories about our families,.
we tell them the good stuff, right?.
We tell them, I've already received.
a number of Christmas cards so far this year..
I love getting Christmas cards..
Christmas cards tell you all the great stuff.
that's happening with families..
Here's all the things that have happened..
And we do this, don't we?.
We talk to people like this, like,.
"Hey, let me tell you about little Bobby.

$^{361}$"and what he got in his latest test.".
If little Bobby didn't do good in the latest test,.
you don't talk about that, right?.
Or whatever it might be, whatever achievement and stuff,.
we don't talk about the messy stuff, do we?.
We don't present the failures or anything like that..
Like, we don't tell people,.
"Oh, let me tell you about Uncle Bob who's in jail.".
Let me tell you about Ted who cheated on his wife..
He's part of my family, a little awkward,.
but let's talk about that one..
You come over to my house and you'll see photos of my house..
Those photos are from good times that my family has..
I don't splash the photos of the arguments.
and the breakdowns and the discussions.
and the things that Chris and I disagree on..
I don't present that sort of stuff to the world..
We love to present to the world.
who we think the world should think us to be..
And you would think that Matthew and Luke,.
in presenting Jesus as the Messiah,.
would do exactly the same thing..
Here's the thing..
While they do include a few people.
like I've already mentioned who are amazing people,.
the majority of the genealogy, guess what?.
The people are really messed up..
That should shock you..
That's what Matthew and Luke should not have done..
But they don't shy away from the messy background to Jesus..
In fact, they want you to see it..
They shine a light on it.
so that you can begin to wrestle.
with the incredible reality.
that some really screwed up people went into the bloodline.
that formed and brought Jesus Christ..
So I wanna tell you some of the stories.
of some of the people that you'll find.
in Matthew and Luke's presentation.
of the genealogy of Jesus..

$^{401}$You'll find characters like this,.
like Jacob and Jeconiah,.
who were basically serial liars and manipulative deceivers..
There's people like David and Jehoram,.
who were murderers, adulterers, and rapists..
You'll find Judah..
Judah's an interesting one..
Judah is a human trafficker and sexual abuser..
You'll find Abijah and Manasseh,.
both who are idol worshipers, by the way..
Down the list a little bit further in Matthews,.
there is Rahab..
She's a brothel owner and a prostitute..
Tamar's always a fascinating one for me..
Tamar slept with her father-in-law,.
so basically entered into incest..
You've got Solomon, who was essentially a sex addict..
He had 300 sex slaves..
You have Ahaziah,.
and this one might be the most disgusting of all,.
Ahaziah, who actually was involved in child sacrifice..
All of these people, Matthew and Luke,.
don't shy away from,.
despite the difficulties of their stories..
But it's not just those that have abused.
that are in the bloodline..
There's also examples of those.
that have been abused in the bloodline..
So in here as well, it's Bathsheba..
Bathsheba, who was exploited by King David for her body..
There's Ruth, who was a Moabitess..
The Jewish people hated the Moabites..
She's included in the line..
And she was someone who was actually discriminated against.
time and time again..
You then got Mary in the list, of course..
Mary comes towards the end..
She was someone who was falsely accused..
People accused her of having sex before marriage.
when that wasn't the case..

$^{441}$So you've got those.
that have also been mistreated in the list..
Add to that then a whole bunch of names..
Let me read these to you..
People like Azor, there's Zadok, Achim, Eliud..
The reason why I read you these names.
is because we know nothing about them..
There's no story about them at all..
They're not mentioned anywhere else in the Bible..
They're completely forgotten to history..
You could actually argue that these are forgotten,.
unknown, and insignificant people..
They're also on the list..
So just take a second and look at Jesus's family tree..
These are the stories that go to feed.
into the person of Jesus..
And Matthew and Luke are like, look at this..
And some of us in this room,.
we have some pretty broken families that we're from,.
but maybe this is a comfort to you.
that your family's not quite like this..
This is pretty full on..
And Matthew and Luke, what they're doing is,.
they're like, no more misdirection here..
We're not gonna try and convince you.
that the Messiah is this perfect God,.
this perfect person for a perfect world..
He's a perfect God and a perfect person.
for a really broken, screwed up world..
And guess what?.
All that brokenness, all that screwed upness,.
it's all come up to form who he is..
His bloodline is not great..
His bloodline actually is a reflection of us,.
of humanity..
And this, my friends, is what Christmas is all about..
Matthew and Luke are trying to shine a spotlight.
right at the start and say,.
here's the purest message of Christmas.
that Jesus is willing to step into the muck.

$^{481}$and the brokenness and all the stuff that there is in life,.
all of the stuff that we would rather sweep under the carpet,.
we'd rather do some misdirection.
so people don't know that about us..
All of that stuff is why Jesus came..
This is the gospel at its purest form..
He didn't come for you when you got yourself cleaned up..
He didn't come for you when you were all so perfect and nice..
So many of us think that Jesus would come to us.
if we just stopped sinning..
The reality is he came to you.
whilst you were still a sinner..
He steps into the muckiest of our mucks and says,.
I am here..
I'm here with you..
That's the beauty of Christmas..
And anytime we try to turn it into something else,.
we're actually stripping Christmas from the gospel.
for which it was founded in the first place..
See, here's what Jesus' genealogy teaches us,.
that no one is so bad, so sinful, so mistreated,.
so forgotten, so cast out of the circle of faith.
that they are outside the story of Jesus..
No one..
And if that doesn't encourage you today,.
I don't know what will..
That no one is disqualified.
from the grace and the mercy of a God.
who was born in a damp, dark cave.
to give us the message that no matter how damp, dark,.
and cave-like our hearts might be,.
Jesus can be born in there too..
His life can come alive in us..
That in fact, I think actually Jesus likes to search out.
those that seem more messed up than others.
and say, you're my child..
You..
I love you..
And I am here for you..
And the beauty of the genealogy of Jesus,.

$^{521}$in fact, the beauty of every genealogy,.
is that it tells you the people that come to form someone..
But then you can flip it because that person then.
becomes the starting point of more.
that would come from them..
Do you follow that?.
So genealogy goes up towards the person,.
but then the genealogy goes beyond the person,.
outwards from them..
And every single person who's in this room right now.
who's a Christian, you're a part of now the family line,.
the bloodline, if you will, of Jesus..
Not the literal bloodline,.
but the blood that was shed on the cross.
to pay the price for your sin..
And as we come into saving recognition of that,.
and we ask God to forgive us of our sins,.
and we grow in that, then we are a part of his family..
He says, you are not the same person that you were before..
You're a new creation..
You're my son..
You're my daughter..
You're my child..
You're a child of God..
You are in this story now..
Isn't that beautiful?.
So the reason why Matthew and Luke start with this story.
of a broken, messy background to Jesus.
is 'cause they're trying to say something..
That the family that Jesus was from.
anticipates the family he's come for..
You follow that?.
The family that Jesus is from anticipates the family.
that he's come for..
You and I, who are equally as messed up and as broken.
as some of those people in the list right there,.
and yet who are not disqualified,.
and who are ones that God now turns to and says,.
I choose you,.
that you are gonna be a part of this thing.

$^{561}$called the church,.
this thing filled with kind of messy people.
who know that they have a God who has saved them.
and redeemed them and have hope now..
See, as Christians, we don't worship the mess..
We don't go, oh, look at all my brokenness..
Look at all my sin..
Paul says, in recognition of the grace of God,.
do I continue to just go on sinning?.
No..
We don't worship the mess,.
but we realize that we are part of the mess,.
and that Jesus comes and redeems us out of that mess,.
and as he pulls us out of it,.
we then have a story to tell..
Christmas should be the most powerful,.
gospel, evangelistic story that could ever be told.
to a broken and desperate world,.
but we've allowed the world to strip Jesus from it..
Happy holidays..
When actually Christmas should be the time.
where the church rises up and says,.
there's a better story to be told..
There's a story about the mess of the world.
and the mess of humanity.
and the God who is not afraid of that mess,.
a God who steps into that mess, does something about it,.
who invites us to experience his love even in the mess..
I love how the scholar Raymond Brown puts it..
He wrote this..
He said, "The God who wrote the beginnings.
"with crooked lines also writes the sequence.
"with crooked lines,.
"and some of those lines are our own lives in witness,.
"a God who did not hesitate to use the scheming.
"as well as the noble, the impure as well as the pure,.
"men to whom the world hearkened.
"and women upon whom the world frowned..
"This God continues to work.
"through the same glorious mess of people.".

$^{601}$Isn't that beautiful?.
This God continues to work.
through the same glorious mess of people,.
the Vine Church in Hong Kong..
That fills me with joy..
That's the gospel that actually is Christmas..
And on your way out in a moment,.
I wanna encourage you to go down the main stairs.
of the building,.
particularly if you didn't have time.
to look at them on the way up..
If you're in the upper house,.
I might wanna encourage you guys to come and do that.
even though you might escape through the lift.
or some other way at the end..
Not escape, that was the wrong word..
You go in the lift..
You're more than welcome to use the lift anytime..
I will not think you're escaping..
The lift is good..
But you might wanna come down the stairs.
and come down the main thing.
because what we've done.
is we've wanted to visually represent.
everything I've been talking about today.
and what this series is all about.
through what we call cabinets of curiosity.
or rooms of wonder..
They're famous in the 16th century.
to help people to experience the wonder of the world..
And we, our creative team,.
have made four of these that are down there on the stairs..
You can look at them on the way out..
And we've put specific things inside of them.
'cause they represent what I'm talking about..
They represent the reality that on the outside,.
we're structured and we have a certain way.
and we project a certain image..
But on the inside is all the stuff..
Some of it, which we're happy for people to see..

$^{641}$Some of it, we're not happy for people to see..
But in that, we have a gateway.
to the need for the gospel of Jesus.
and the need for the hope that he brings us.
by his love and in his power..
And so as we come to look at these.
little cabinets of curiosity,.
I wonder you might reflect a little bit.
on the curiousness of your mess in you..
And it's an invitation for you to bring that mess to him..
And what we're gonna do every step of this series.
is give you an opportunity to bring a certain type.
or kind of mess to God..
And I believe that as we get to the end of the series.
and as we get to Christmas Day,.
you will find yourself living out the Christmas story.
so much better than ever before..
It is not a misdirection..
Instead, what Matthew and Luke do is they're like,.
I'm not gonna do this, by the way,.
but if they took this and went, wha, like that,.
they basically rip it all off and they say,.
this is how it really is.
because in this, there is hope..
For you in this room online right now,.
everybody in the overflow,.
no matter how sinful you are, there is hope..
No matter how broken you are, there is hope..
No matter how forgotten you are, there is hope..
No matter how mistreated you have been in your life,.
there is hope..
That's the gospel message of Jesus..
That's what Christmas is all about..
So may we enjoy the mess this year..
One of the greatest messes you'll ever encounter.
is the mess of Christmas Day..
Do not be too quick to clean it up..
Can I pray for you?.
Let's pray..
Father, I am so grateful for the people in this room..

$^{681}$I'm so grateful for every person.
who holds a different story..
I'm grateful, Lord, that you have provided for us.
all a gateway, an entry point to you..
That your birth in Bethlehem some 2000 years ago.
in a sequence of messy events.
is really our opportunity.
to experience your welcome in our lives..
And so I wanna pray for a few kinds of people in this room.
as our eyes are closed and our heads are bowed..
Some of you in this room, you don't know Jesus..
And maybe when you came in here this morning,.
you wouldn't have said that you were a Christian..
Maybe someone's invited you here,.
or maybe you feel like you had a relationship.
with God in the past, but that's gone.
and you're wondering how to get it back..
The message that I brought today.
is a message of hope for you..
You don't need to get yourself cleaned up.
before you can meet Jesus..
And some of you, that's a real message for you today..
Some of you are thinking,.
well, once I get through that divorce.
or I stop doing that thing there,.
or once I just become a little bit of a better person,.
then I can be a Christian and come to church..
The message today, the message of Christmas is.
no matter how messed up you are right now,.
you are welcome here..
You're a part of our community..
But more importantly, Jesus loves you..
He's for you and He can help you..
And so maybe some of you in this room right now or online,.
you wanna give your life to Jesus for the very first time..
You've never done that before, but this is a moment for you..
So whilst our eyes are closed and our heads are bowed,.
I wanna give you an opportunity to respond today..
Perhaps right now your heart is just beating.
a little bit faster 'cause you know.

$^{721}$that God is with you in this,.
that He's inviting you to a relationship with Him..
And you may not have all the answers..
You may not know all the things,.
but you do know in this moment.
that you want to meet a God.
who can meet you right where you are..
So if you've never given your life to Jesus before,.
I wanna give you an opportunity to do that..
And I'm gonna pray for you in a moment..
Whilst our eyes are closed, heads are bowed,.
you have an opportunity to start this new Advent series.
and season in the right place..
And so if that's you,.
I wanna just encourage you just really quickly,.
just put your hand up so I can see you,.
so I can pray for you personally..
Anybody here in the lower house?.
In the upper house as well,.
just put your hand up so I can see it..
Great, see that hand, thank you..
Anyone else wanna join my sister?.
Yeah, I see that as well, thank you..
I can't really see the upper house,.
but we have pastors up there who can see you.
if you have raised your hand..
All right, I wanna pray for those two people.
that I've seen and maybe some people.
in the upper house as well..
Father, I thank you so much for these two, Lord,.
who Lord have brought themselves to a place today.
to meet you..
And we're all gonna pray as one church together now,.
but those two in the upper house, if it's you as well,.
I want you just to pray after me..
We're all gonna do it out loud,.
so you guys are a part of it,.
but it's particularly for you guys..
So we're all gonna pray this together, let's pray..
Lord Jesus, I wanna thank you for coming to earth,.

$^{761}$for being born 2000 years ago,.
so I could know you..
Thank you for being born into the mess,.
into my mess, thank you..
I ask that you would come now and meet me,.
forgive me and help me..
I ask for your Holy Spirit to be in me and with me.
and helping me..
And I thank you that my new life starts today.
in Jesus' name, amen, amen..
Well, if you've prayed that prayer here in the room.
and in the upper house, do come and talk to me afterwards.
at one of our passes, we'd love to pray with you..
Everybody else, I want you to bow your heads.
and close your eyes again, sorry, 'cause we're not done yet..
And so this is now for everybody else here.
who's already a Christian..
And I know some of you are from really broken homes..
I know that because I'm your pastor.
and our pastoral team spends time with you guys..
And some of you are from really difficult,.
challenging family backgrounds..
Some of you have been deeply hurt by those in your family..
And some of you, you've been really impacted.
and affected over the years.
by the brokenness in your family..
And I pray today as we've taken a bit of a look.
at the brokenness of Jesus' family,.
that you would find hope renewed in you today,.
that you would know that there is a God.
who knows what it is that you're going through,.
who knows how hard that it has been for you,.
and a God who wants to come and walk with you..
He's not a God who always necessarily.
suddenly makes everything better,.
doesn't always suddenly change.
our circumstances or situations..
But He does guarantee His presence with you.
in whatever it is that you are facing right now.
and will need to face..

$^{801}$And so if you're from a broken home.
or challenging family backgrounds,.
know that the love of God.
and the presence of God is with you..
And Lord, I wanna pray for my family here,.
whether that's their reality..
Father, I pray that they would feel hope renewed in them,.
that they are now swept up in a family.
that you're a part of, that you started,.
that as we carry the name of Christ,.
we do so as part of your family,.
your sons and daughters, your children..
And Father, I wanna pray lastly.
for anyone here who's hurt people..
In the list of Jesus' genealogy.
are those that have been mistreated..
And we know, Lord,.
that sometimes we have done that ourselves..
Maybe some of our family issues are at our doorstep..
We were the perpetrators.
of some of the broken relationships in our families.
and some of the things that we're not proud of..
Maybe we're the reason why there's been some brokenness.
amongst the people we love the most..
And I wanna pray for you, if that's you in this room,.
that you would know the grace and the forgiveness of a God.
who forgives you so that you might be able.
to have the strength to move forward.
and invite and ask for forgiveness.
for those that you have hurt..
The Lord's Prayer, forgive us our sins.
as we forgive those who have sinned against us..
And I think the Gospels would further add.
as we invite and ask for forgiveness.
for those that we've hurt..
And so I wanna pray for anyone here.
who recognizes that they've played.
that kind of role in their family..
That Lord, I just ask that you would give them.
the strength, Lord..

$^{841}$The humility, Lord, to seek reconciliation.
where it's possible, Lord..
May not be possible in every circumstance,.
but where it's possible, Lord,.
give them the courage to do that..
And Father, as we stand with this thought today,.
we're thankful that you ministered to us.
in multiple different ways..
And it's all for your glory and for your grace..
In Jesus' name..
Can we stand together and we're gonna finish our time.
just allowing the Spirit of God.
to continue to minister to us..
We're gonna sing some worship together..
And I wanna encourage you that you may wanna seek prayer.
after the service..
In a moment, Justin will close the service..
And when he does so, we're gonna invite you.
to come forward for prayer..
And some of the things that I've been sharing about today.
touch a nerve for you, touch your heart..
We've got a team that would love to pray for you..
And so we'll lead you in that in just a moment..
But let's worship for now..
for now..
\newpage



\section{}
\label{sec:wQE5SgaxsPo}
\textbf{2023-12-11 Of Silence and Shame [wQE5SgaxsPo].mp3}
\newline
\newline
連結: \href{https://youtube.com/watch?v=wQE5SgaxsPo}{\texttt{ https://youtube.com/watch?v=wQE5SgaxsPo}} ~~~~ 語音日期: 2023-12-11 
\newline
\newline
\hyperref[sec:c_wVd_9_9_Y]{\small{< < < PREV SERMON < < <}}
~
\hyperref[sec:index]{\small{[返主目錄]}}
~
\hyperref[sec:dNq7mC9pMMw]{\small{> > > NEXT SERMON > > >}}
\newline
\newline
$^{1}$Well, first of all, I just have to do a shout out for my husband,.
who's unfortunately not very well..
So, usually he's sat on the front row..
He's not here, so, love you..
(Laughter).
Hello, everyone, and welcome watching online..
So glad that we can pack everybody in..
It's so good..
Well, you know, Christine Gardner works here..
She leads our Psychotherapist and Counseling Center Oasis..
And she did a staff training for us a number of months ago,.
and it was on self-care..
And one of the things that she said to us was,.
"You will have pain.".
Not quite what we were expecting for self-care,.
but she's right..
Her point was that life is full of pain..
We cannot get away from it..
When we understand that there will be pain and we accept it,.
we may not like it, we certainly don't have to want it..
However, when we understand it will be with us,.
then our relationship with pain changes..
It then allows us, rather than to suppress and try and push away pain,.
it allows us to then think about how we can invite God.
into the pain that we are experiencing..
In the same way, life is messy..
You cannot have a mess-free life..
I'm sorry..
It just is the way it is..
And so we have to accept that there will be mess..
But we learn that God is not unfamiliar or surprised by mess either..
In fact, today we're going to look at how God can be actually found in our mess..
In our second week of Advent,.
we look at the messy priest story of the birth of Jesus..
We look at two characters in the narrative captured in the Gospel of Luke.
who find themselves in deeply messy situations.
and yet find ways to trust God despite the far from ideal circumstances..
We'll look at how their stories relate to us,.
how they teach us that we too can find God in the messiness of our own lives,.
embrace that mess, and position ourselves for a hopeful future..

$^{41}$So of the four Gospels,.
only Luke begins with the story of the birth of John the Baptist..
And he introduces us to his father, Zechariah..
Now Luke tells us some interesting facts about Zechariah.
to help us characterize the type of man that he was..
First of all, his profession..
He was very honorable..
He was a priest, and he belonged to the priestly division of Abijah..
His wife, called Elizabeth, was also of priestly descent,.
and they were both described as upright in the sight of God..
We also learn that they were childless..
Scripture tells us that Elizabeth was barren,.
and we knew culturally at that time there was a lot of shame and public disgrace.
around not being able to have children..
And then finally we're told that they were old..
Well, actually, Scripture says that they were well along in years,.
which I think is a much better way of saying they were old, right?.
So next Luke leads us into the story by setting the scene in the temple..
The priests bore the responsibility of serving in the temple,.
offering the sacrifice required by Mosaic law,.
acting as the mediators between the people and God..
And on this occasion we are told Zechariah had been chosen by Lot.
to enter into the holy place..
He's there to burn incense..
And the offering of incense represented the prayers of intercession.
for all of the people, so it was considered a very sacred task,.
and for Zechariah an opportunity of a lifetime..
So while Zechariah was in the holy place, an angel appeared to him..
Now, understandably, Zechariah was afraid..
I think I would be too..
Even though, as a priest, he would have a sense of awe and trepidation.
when he's serving, so that would be natural,.
but this was a fear much, much more..
He was probably not expecting to encounter an angel on that day..
But the angel goes on to declare a very positive and personal message to Zechariah,.
telling him that Elizabeth and his prayers to have a child were about to be answered..
So let's pick up the story in Luke 1, verses 13 to 17..
When the angel said to him, "Do not be afraid, Zechariah..
"Your prayer has been heard..
"Your wife, Elizabeth, will bear you a son..

$^{81}$"You are to call him John..
"He will be a joy and delight to you..
"Many will rejoice because of his birth..
"He will be great in the sight of the Lord..
"He is never to take wine or other fermented drink,.
"and he will be filled with the Holy Spirit even before he is born..
"He will bring back many of the people of Israel to the Lord their God..
"He will go on before the Lord in the spirit and power of Elijah.
"to turn the hearts of the parents to their children.
"and the disobedient to the wisdom of the righteousness,.
"to make readier people prepared for the Lord.".
Wow, what a message..
You can almost feel the excitement in the angel's declaration.
that will ultimately prepare the way for the long-awaited Messiah..
But unfortunately, Zechariah's response was not one of joy,.
but rather disbelief..
So let's just take a moment to consider this..
So here is Zechariah. He's a Jewish priest..
We are told he is a good, upright man..
He's standing in the holiest, holy place,.
and he's carrying out the role of offering the prayers on behalf of the people,.
including his own very personal prayer, and he's facing an angel..
Now, Zechariah knew the Torah,.
so he would have been very familiar with the story.
of how a heavenly visitation to another elderly couple.
many years before, Abraham and Sarah,.
who also received a message that they would bear a son.
who would play an important role in the story of God redeeming a fallen world..
Despite all of this, Zechariah lacked faith, and he doubted God..
So we have to ask the question, why?.
Had he become hardened over the years of praying for something.
and waiting and waiting for that prayer to be answered?.
Can we relate to that?.
Was he over-familiar with the routine of priestly duties?.
I wonder maybe sometimes we can become hardened to God.
as our prayers appear to go unanswered..
Of course, we don't stop praying,.
but do we allow a little bit of disbelief,.
hinder us from truly seeing God,.
so when he does answer, we possibly miss it?.

$^{121}$I think for some of us here that have been Christians for a long time,.
we can find ourselves appearing to trust God on the outside,.
carrying all the good duties of a good Christian,.
saying all the right things,.
and yet if we were really honest,.
we've allowed the years of unanswered prayers,.
and with that, a disappointment in God to take root..
And if an angel were to visit us today, how would we respond?.
Well, Zechariah's response was, "How can I be sure of this?.
I'm an old man. My wife is long in years.".
Judging by the reply,.
I don't think the angel was particularly happy about that response..
The angel said to him, "I am Gabriel..
I stand in the presence of God,.
and I have been sent to speak to you and to tell you this good news,.
and now you will be silent and not able to speak.
until the day this happens,.
because you did not believe my words,.
which will come true at their appointed time.".
Ouch..
The angel is making a point..
While Scripture records numerous visitations of angelic beings,.
there's only a couple of recorded references to their actual names,.
and here, the Hebrew name for Gabriel means "God is my strength.".
The angel is leaving no doubt in Zechariah's mind.
that he's been sent directly by God,.
reminding Zechariah of the power of God.
and his promises that will be fulfilled..
And the consequences of Zechariah's disbelief?.
He's struck dumb..
He's physically silenced..
As Zechariah eventually leaves the holy place,.
all the expectant people and worshippers outside,.
no doubt wondering what on earth happened,.
they're met with silence from Zechariah..
The greatest chance for Zechariah to do the job well,.
and he messes up..
And I imagine he went home with disappointment in himself,.
silenced not just by the angel, but with a deep sense of shame..
So let's talk about shame..

$^{161}$I think we underestimate the power of shame over our lives.
and how it easily can hold us captive..
According to Brené Brown,.
a respected professor, author, and researcher.
for 20 years on vulnerability and shame,.
she says, "Shame is the painful feeling or experience.
"of believing that we are flawed,.
"therefore unworthy of love and belonging..
"It's the competing, conflicting expectation.
"of who we are supposed to be.
"and the huge weight of expectation we put on ourselves.
"or are put on us by others.".
Shame tells us that we are not good enough.
and we will never be good enough..
Brown connects the destructive nature of shame.
to addiction, depression, aggression, bullying, suicide,.
self-harm, eating disorders, to name a few..
And the problem with shame is that it is often hidden..
We don't talk about it..
We keep it hidden and then we carry this weight.
of self-judgment and then because of that,.
we get stuck in this cycle of negative behavior.
that then leads us to more shame and guilt..
And as Christians, we are not immune to shame..
And I wonder if especially in church communities,.
the secret, silent, judgmental hold of shame.
keeps us from being truly open and honest with each other..
Are we safe communities that allow people to be vulnerable,.
to share their stories without fear of judgment?.
Well, scripture tells us very clearly,.
"Therefore there is now no condemnation.
"to those who are in Christ Jesus.
"because through Christ Jesus, the law of the Spirit.
"who gives life has set you free.
"from the law of sin and death and shame.".
So if you are feeling the weight of shame,.
then today I want to encourage you to break the secrecy,.
the silence, your own judgment..
Talk to a trusted friend, see a pastor, a counselor,.
a community group leader..

$^{201}$Believe that because of Jesus,.
we can confess our failures and sins,.
know that we are forgiven,.
and shame has absolutely no hold over us..
And as a community, as the Vine Church,.
should we let's be less judgmental,.
stop the criticism, stop the pretense of perfection,.
and instead be known for our empathy,.
our love, and our grace?.
Shame has no place in the body of Christ..
When Elizabeth confirms her pregnancy,.
the shame that Zachariah had doubted.
could have festered within him..
However, we see that he makes the important decision.
to choose to be vulnerable, to surrender and trust..
So he decides to obey the angel's instruction.
and he's surrounded by his community.
when he names his son John,.
therefore declaring the hope in a future salvation..
And that shame, along with his silence, is broken..
And sandwiched in this story is another story,.
and this is the story of Mary, the mother of Jesus..
I love this painting..
This is by the artist Henry of Sawatana,.
because while it's rich in symbolism,.
it captures the vulnerability and simplicity.
of Mary in her surroundings..
Mary was a young woman, a teenager,.
in the context of the culture of that time of marriageable age,.
betrothed to Joseph, who was himself a descendant of King David..
And we're told that the angel Gabriel was sent to Mary..
And when he appears, she too responds with fear,.
like Zachariah, but unlike Zachariah,.
her fear is followed by a surrender and a trust..
So let's read from Luke 1, verses 30..
"But the angel said to her, 'Do not be afraid, Mary..
"'You've found favor with God..
"'You will conceive and give birth to a son..
"'You are to call him Jesus..
"'He will be great and will be called the Son of the Most High..

$^{241}$"'The Lord God will give him the throne of his father David,.
"'and he will reign over Jacob's descendants forever..
"'His kingdom will never end.'.
"'How will this be?' Mary asked the angel..
"'Since I am a virgin.'.
"The angel answered, 'The Holy Spirit will come on you..
"'The power of the Most High will overshadow you..
"'So the Holy One to be born will be called the Son of God..
"'Even Elizabeth, your relative, is going to have a child in her old age..
"'She who was said to be unable to conceive is in her sixth month,.
"'for no word from God will ever fail.'.
"'I am the Lord's servant,' Mary answered..
"'May your word to me be fulfilled.'.
"Then the angel left her.".
The 15th century Franciscan priest Roberto Caratiolo.
describes Mary's response to her encounter with the angel Gabriel.
as this emotional journey from this sense of disquiet, of being troubled,.
to a reflection as she's listening,.
to an inquiry as she begins to question,.
and then to submission as she acknowledges she's the Lord's servant,.
and later as merit as she is blessed..
But I think it can be too easy to dismiss the intensity of that moment.
and the range of emotions for Mary,.
especially as we, we're all familiar with her story..
We're familiar with the outcome..
But Mary, in real time, probably had no idea what the future would bring..
She's described as being deeply troubled,.
even before the angel had told her what was happening..
Mary probably knew because of her ancestry.
that being singled out by God was a fearful blessing..
And for us, how often do we in the moment.
find God's ways sometimes really overwhelming?.
We too are on this emotional journey..
However, when we can look back, it becomes so much clearer..
So as the angel delivered his message,.
despite the enormity of what had been shared,.
her humility and acceptance demonstrated a level of faith and trust in God,.
even that of beyond a priest..
She was to be the mother of Israel's long-anticipated, long-awaited Messiah,.
and her life would be turned upside down..

$^{281}$God's plan for Mary was messy and risky..
The scandal for Mary to be pregnant and unmarried.
exposed her to the rejection of Joseph..
But as with all God's plans,.
he controlled the responses of those that mattered..
And in the case of Mary, Joseph became Mary's number one protector..
And we read that Joseph accepted Mary after he too was visited by an angel.
and trusted God despite the social stigma he would also face..
And I love that about God, how he works should be an encouragement to us..
We don't face our difficult, complex, messy situations alone,.
and God is in control..
I have heard so many stories of messy, difficult, complex, tragic situations,.
and I have heard how God has stepped in and people have encountered.
that God is with them and in control..
So what do these stories show us?.
That even in mess, God is at work and God is in control..
Mess for Zachariah was the failure to believe, his physical disability,.
the embarrassment from his fellow priests and worshippers,.
as well as the shame at having doubted..
Mess for Mary was being pregnant, unmarried, the social ostracism,.
the judgment, and possible rejection..
Sometimes our choices mean we are responsible for the mess..
If Zachariah had chosen to believe instead of doubt,.
his story would have been very different..
The angel was very clear in telling Zachariah that because he doubted,.
he would not be able to speak..
But sometimes we have no control over the mess we find ourselves in,.
like pregnant Mary who had no control over the cultural response of that day.
and the rejection that could follow..
Just like the characters in these stories, whether we are responsible or not,.
we can choose how we respond in the mess..
You see, Zachariah eventually chose to see God in his answer prayer,.
to surrender through obediently naming his son John.
and trusting God as he declared his praise and thankfulness..
Mary chose to see God despite not fully understanding,.
surrendering herself and trusting God with an uncertain future..
Through both stories, we can learn that how we choose to see God in our mess.
will lead us to surrender and trust and then hope in a future..
So how do these stories become relevant to us?.
A number of years ago, I worked for an organization.

$^{321}$that provided funding for new companies.
and injection of capital for those that wanted to diversify and grow..
My job was to interview prospective clients and to prepare a business plan..
And then those business plans would then result in the funding being rejected or approved..
So I worked for this company for about just under a year..
And during that time, got very close to all of my colleagues..
It was a family-run business and so even outside of work hours, we would meet up..
Then just a week before Christmas, I arrived into the office.
and I was told that the company had closed and I was to go home..
Then shortly, we found out that actually the company that had been operating.
had no funding, there was no money, and it was now under investigation..
And so I went home also being told that there was also no money to pay the staff..
And so a week before Christmas, we were told we would not get paid..
It was a messy Christmas..
I think one of the things that I really struggled with.
was that knowing I had worked for a company for almost a year and I had no idea..
And for me, I felt really embarrassed and I felt shame..
I'd worked in a company that was responsible for ultimately causing a lot of pain.
and difficulty for organisation and companies..
And I had no idea. People had been deceived..
There was pain and disruption in that..
And now I faced the harsh reality that also I was out of a job..
I had no money, my husband Richard was studying at the time,.
and so we really didn't know what we were going to do..
But even though I had a lot of questions,.
because after all, I thought God had provided this job for me,.
the reality was this situation was way out of my control..
But I could control how I responded to it..
So I chose to lean into God, to try to see him in all of this mess..
And I asked him a lot of questions too..
I chose to surrender my fear, my worry and my shame..
And then I just chose to trust God that he was at work.
and that he was the hope in my future..
Didn't make sense to me, didn't have all the questions answered,.
but I chose to trust God..
It took about nine months before I got another job..
During that time it wasn't easy..
And I remember we ate a lot of baked beans..
And if you're from the UK, you will know that is the staple diet.
if you have no money in the UK..

$^{361}$But even though it wasn't always easy,.
and sometimes fear would creep in,.
"What if I never get a job again?.
What if nobody ever hires me?".
All of that, but I had to push through that.
and keep on trusting that God was with me..
So eventually I got another job,.
and that job turned out to be so much better than the other job..
And eventually it would lead me to here in Hong Kong..
But I had to learn an important lesson,.
that it was not about me getting a job..
I had to learn that the security of my future.
was not tied up in my work..
The security of my future was in the redemptive work of the cross.
that reconciled me back into a relationship with God.
and then a hope in a future with Jesus.
where there will be no more pain, no more death, no more sorrow..
And this is the hopeful future that Zachariah and Mary saw.
in the messages from the angel.
that declared the ultimate fulfillment of the promised coming Messiah..
So what is your mess?.
And how do you see God?.
Because how you see God will determine how you surrender to Him,.
how you trust Him, and where your hope comes from..
A hope that is not based on the changes in circumstances,.
but a hope that is in the future,.
because this life is temporary..
Our future is secured forever beyond this life because of Jesus..
So the world may appear to be in a mess,.
complete disorder,.
but as the stories of Zachariah and Mary tell us.
and as Scripture shows us time and time again,.
God is no stranger to mess..
He actively uses it to demonstrate His power,.
His sovereignty, and His love..
And honestly, the world desperately needs to see God right now..
These needless wars raging around us.
with so much destruction and devastation.
and loss of life that is just not justifiable..
The economic decline with rising poverty levels,.

$^{401}$global increase in anxieties and mental health disorders,.
and here in our city of Hong Kong,.
the alarming rise of young people's suicide or attempted suicide..
The world needs hope..
Jesus is that hope..
Both Zachariah and Mary knew that the future was in the promised redemption.
of their people through the Messiah, Jesus..
And they were to play the most important roles.
in ushering in the kingdom of heaven..
As Zachariah's son prepared the way for Jesus.
and Mary gave birth to Jesus..
Jesus is the answer to our world's mess..
But I can't help also wonder at the risk that God took.
to have Jesus arrive as a baby,.
totally weak and vulnerable,.
dependent on his mother and earthly father,.
but also guarded by his heavenly father..
Because our God, our heavenly father, is in control..
So how we see, to surrender and trust God.
is how we live with hope..
So what does that practically mean for us today?.
Well, we may not be visited by any angels anytime soon,.
or you might, I mean I'm not saying you won't,.
but don't just rely on that..
We can still see God..
We see God through scripture..
We read a lot about who God is and his character..
As we learned in our Exodus series,.
God himself declares his character..
He says, "The Lord, the Lord, the compassionate.
"and gracious God, slow to anger,.
"abounding in love and faithfulness,.
"maintaining love to thousands.
"and forgiving wickedness, rebellion, and sin.".
We see him through the incarnation of Jesus.
and the life he recorded in the gospels..
We see God in the world around us,.
the beauty of nature, the wonder of creation..
Paul refers to this in Romans..
"For since the creation of the world,.

$^{441}$"God's invisible qualities, his eternal power.
"and divine nature have been clearly seen,.
"being understood from what has been made.
"so that people are without excuse.".
And David also in Psalm 19,.
"The heavens declare the glory of God..
"The skies proclaim the work of his hands.".
We can see in creation around us,.
in the visible world, all that has been made by God.
and his invisible power and divine nature..
So take time to watch the sunrise or the sunset..
Take a walk in the mountains of Hong Kong.
and see his splendor..
Stand by the sea and watch the beauty.
and the power of the waves..
See the beauty of the creatures..
See God all around you..
And then as we grow in our understanding of who God is,.
we truly see him, our natural response will be.
we will want to surrender our lives to him..
Now sometimes that's easier said than done..
So how can we practically surrender?.
What does that mean?.
Well first it has to take humility..
We need to recognize and accept.
we need God..
His ways are so much bigger than our own..
And one practical way that we can take time.
is to regularly identify areas in our life.
where we feel we're holding and struggling for control..
And then confess your need to surrender to God,.
to surrender whatever it is and allow him to control..
Maybe it's overspending, overeating, drinking too much,.
challenging family dynamics, difficult work relationships,.
your job performance..
Whatever it is that you're desperately trying to control,.
bring it to God..
Create a habit where you regularly come in prayer to God.
with confession and surrender..
Commit to growing your personal relationship with God..

$^{481}$Be committed to a more intimate relationship.
that involves spending time in prayer..
Hang out with Jesus..
Allow the Holy Spirit to guide you..
Set some time aside that is away from distractions..
Go for a walk and talk with Jesus..
Sit and meditate on Jesus..
Whatever works, find something that works for you..
It takes time and effort,.
but any meaningful relationship does..
But it's worthwhile the investment of your time.
because as you spend more time in relationship with Jesus,.
you will see God more and you will want to surrender more..
You will trust more..
So it's a healthy cycle..
Life is messy..
So this Christmas, embrace the mess..
Choose to see God in the mess..
Choose to surrender despite the mess..
Choose to trust God with the mess.
and you will know hope..
Can I pray for us?.
Please would you stand..
(congregation murmurs).
Father, I thank you for the honor it is to speak today..
I just ask that my words would take root..
The things that are from you,.
the things that you're wanting to speak to us,.
that they would not return void..
Anything that is not of you,.
then may it just disappear..
But Father, my heart is that we would be a people.
that chooses to have hope in you, God,.
because we desperately need to see you in the mess..
I thank you for Jesus..
I thank you that he is our hope..
He is our salvation..
And Father, I want to pray specifically.
for those that are really struggling to see Jesus..
And one of the things that I sensed as I was praying.

$^{521}$was that there is a spiritual blindness..
You know, as Jesus went in the gospels.
and many of his stories are around healing.
of those that were blind so they would see..
There's something significant about that..
For us here today, for some of us,.
there's a spiritual blindness..
Some of you know that, some of you don't..
And I believe that Jesus wants to heal us.
of that spiritual blindness if you are willing..
And so, Father, I pray in the power of the Holy Spirit.
that you would release people from this captivity.
of spiritual blindness,.
that they would know what it is to see you, God..
So much of this world overshadows us,.
puts obstacles in our way,.
but Jesus, would you remove all that.
so you would be clearly seen..
God, we need to see you.
because when we see you, we will respond in surrender..
And I pray for those here.
that are choosing to surrender more..
I ask that you would give them the courage.
to sit with you to identify those areas.
and trust you will be with them..
And I pray for anyone who has experienced.
that captive nature of shame..
And I break the hold that shame has over us.
as a church, as a community, as individuals..
Father, may we respond with empathy towards one another,.
love and grace towards one another..
And would we be a people that is free.
from the hidden shackles of shame..
Our future is in you, God..
And it is a hopeful, bright future..
In Jesus' name, amen..
Amen..
[BLANK AUDIO].
\newpage



\section{}
\label{sec:UL6nYF5cG_Y}
\textbf{2023-12-18 Of Relatives and Rejection [UL6nYF5cG-Y].mp3}
\newline
\newline
連結: \href{https://youtube.com/watch?v=UL6nYF5cG-Y}{\texttt{ https://youtube.com/watch?v=UL6nYF5cG-Y}} ~~~~ 語音日期: 2023-12-18 
\newline
\newline
\hyperref[sec:dNq7mC9pMMw]{\small{< < < PREV SERMON < < <}}
~
\hyperref[sec:index]{\small{[返主目錄]}}
~
\hyperref[sec:L37qvhsDiAc]{\small{> > > NEXT SERMON > > >}}
\newline
\newline
$^{1}$Everyone says, "Amen.".
Good morning..
Good morning to our online community as well who are joining us..
Before I get started, during worship, there was a line from a movie that came into my head..
Some of you may have seen the movie, it's called The Holiday..
It's a Christmas movie..
And there's a music composer in this movie who writes music for movies..
That's his job..
And there's a scene in the movie where he writes a song for a new friend that he's made..
And he tells her, "I only used the best notes.".
And it came into my mind today that God would say to you today that when he composed you,.
when he put you together, he only used the best notes..
And some of you here today, I think there might be some people here who've been told otherwise..
And maybe you don't know the truth of what I've just said to you..
But God would say to you, when he put you together, when he composed your life,.
he only used the best notes..
I get to share the third sermon in our Advent series called Of Relatives and Rejection..
Let's start with reading from the Word..
Luke 2, 1-7 says, "In those days, Caesar Augustus issued a decree that a census should be taken of.
the entire Roman world..
This was the first census that took place while Quirinius was governor of Syria..
And everyone went to their own town to register..
So Joseph also went up from the town of Nazareth in Galilee to Judea, to Bethlehem,.
the town of David, because he belonged to the house and line of David..
He went there to register with Mary, who was pledged to be married to him,.
and was expecting a child..
While they were there, the time came for the baby to be born,.
and she gave birth to her firstborn, a son..
She wrapped him in cloths and placed him in a manger,.
because there was no guest room available for them..
Now, hidden beneath these verses is a whole other story..
Joseph returns to the place he belongs, the place his family, his extended family, is from..
And amongst all those people, not one of them opened their doors to Joseph..
Not one of them found a small space in their dwelling.
where mats could be laid on the floor for this family..
Not one..
And the reason was Mary..
Mary was pledged to be married..
She was not yet married, but she's pregnant..
She's with child..

$^{41}$Their voices that this child was a miraculous gift from God had fallen on deaf ears..
Mary and Joseph have brought shame on their family,.
and nobody wanted to go near them..
Nobody wanted their reputation to be tainted..
Mary and Joseph were rejected, and no room was available for them..
And so they find themselves in a room made for animals..
And Jesus is born..
Mary wraps him up and places him in a food trough where the animals would normally eat from..
Their comfort level?.
Maybe two out of ten..
But what about the smell?.
Just stop for a minute..
Have you been in a room with livestock or where livestock were once inside?.
Not very pleasant..
Maybe zero out of ten?.
Isaiah 53 says,.
"He was despised and rejected by men, a man of suffering and acquainted with disease..
He was despised as one from whom men hide their face, and we did not respect him..
Even before Jesus was born, he was despised and rejected..
His own flesh and blood hid their face from him.".
Now juxtapose this with smiling rosy faces, a roaring fire, Christmas stockings all in a line,.
smell of turkey wafting from the oven, beautifully wrapped gifts meticulously chosen,.
piled high under the exquisitely decorated Christmas tree,.
everyone in there matching Christmas jumpers, just like what Damien's wearing down here,.
or maybe matching Christmas pyjamas or Christmas socks,.
Michael Buble crooning away in the background..
You see, putting the first Christmas experience beside our current expectations of what Christmas.
should be, there is a huge discrepancy..
The first Christmas was messy, and Joseph and Mary were feeling very real feelings of.
rejection and loss, estranged from their family, isolated from a community that could only see.
an unwed mother, to now when we do everything we can to create the perfect Christmas experience.
for ourselves, even so far as deliberately avoiding any relatives who may potentially.
mess up our perfect Christmas..
The problem with this is that we are not perfect..
Life is not perfect..
Rejection, loss, disappointment, imperfection, confusion is still as real and current today.
as it was at the very first Christmas..
Maybe your story is similar to Jesus's..
Maybe you know all too well what it's like to be rejected by your family..
Maybe Christmas does not bring the family welcome or connection that it should..

$^{81}$Maybe you're the only believer in your family..
Christmas is not celebrated, and maybe your faith leaves you feeling lonely and at times.
despised..
Rejection can close us off, but the first Christmas can give us hope..
Yes, Joseph, Mary and Jesus were rejected by their relatives and left to fend for themselves,.
but others came to witness and share in the joy of that first Christmas..
Wise men and shepherds..
In Luke 2 verses 16 to 20, it records the joy and the praise and the treasured moments.
that were brought into Mary and Joseph's lives by the shepherd..
Luke 2 verses 16 to 20 says, "So the shepherds hurried off and found Mary and Joseph and.
the baby who was lying in the manger..
When they had seen him, they spread the word concerning what had been told them about this.
child, and all who heard it were amazed at what the shepherd said to them..
And Mary treasured up all these things and pondered them in her heart..
The shepherds returned glorifying, praising God for all the things they had heard and.
seen, which were just as they'd been told.".
You see, the beauty of Jesus's birth is that it created the opportunity for us all.
to be drawn back into family..
You know, I have always loved Christmas..
And as I was reflecting on this and on this sermon that I was preparing, I realized that.
Christmas was the time of my personal rejection and also the time of my adoption into my forever.
family..
In 1973, some of you are working out how old I am right now..
Let me save you the trouble, I'm turning 50 very soon..
[Laughter].
In 1973, four days before Christmas, I was born to an unwed schoolgirl..
She brought shame on her family and she was sent away..
To have me in secret in another part of Australia..
I was put up for adoption and two weeks later, I was swept into my forever family..
Adoption to me is the epitome of love..
Where there is no blood connection and yet love comes and wraps you up as its own..
Christmas undoes rejection..
Jesus comes and he wraps you up as his own..
We are all given the opportunity to be adopted into his love..
I challenge you that if you struggle with rejection, let this Christmas be the one that.
begins to set you free..
Let it be your adoption story..
John 1 verses 12 to 14 says, "Yet to all who did receive him, to those who believed.
in his name, he gave the right to become children of God.".
Children born not of natural descent, not of human decision or a husband's will, but.

$^{121}$born of God..
The word became flesh..
Jesus made his dwelling among us..
We've seen his glory, the glory of the one and only son who came from the father full.
of grace and truth..
When you accept Jesus as your saviour, you are lovingly swept into his family..
And you are lovingly swept into this family..
And in a church of this size, no one should be alone this Christmas..
There is a challenge here that I am deliberately giving to us all..
A challenge for us to reach out, to be vulnerable, to make space in our homes for those who need.
to experience family this Christmas..
What else can we learn from the first Christmas?.
What was its purpose?.
What was the purpose of the first Christmas?.
Jesus came into the world to make sure every human being knew their full worth, that they.
were loved and that they were valuable..
What would our Christmas look like if our sole focus was to ensure that the people in.
our lives came away from Christmas fully loved, fully assured of their value?.
You see, the trouble is that Christmas comes with lots of traditions..
I mentioned some earlier..
I have fond memories from my childhood of a present opened the night before Christmas..
It always contained a new set of clothes that we would wear the next day to church..
We couldn't open the rest of our presents till we got home from church..
We were always envious of the families who let their kids open their presents on Christmas.
morning..
And once we got home, we'd have a big Christmas dinner with a Christmas ham, grandma's potato.
salad, and all the trimmings..
We often had people at our table that were not blood related..
In the early years, we often had students from Papua New Guinea that mum and dad would.
bring back with them to Australia..
My mum and dad were teachers in Papua New Guinea, and every Christmas they would pay.
for some senior students to accompany them back with our family to Australia to have.
Christmas..
And in the later years, I remember bringing loads of our friends from university back.
to my parents' place to share Christmas..
After a big Christmas lunch, we'd play a game of backyard cricket and then get stuck.
into the leftovers, because there's always leftovers on Christmas Day..
Here are some photos from my childhood of students that my mum and dad would bring back.
from Papua New Guinea to us to share Christmas with us..
That's a little Susanna up there with the blonde hair..

$^{161}$Sorry, they're so hard to see because obviously they're back in, you know, the time when cameras.
were not that great..
Over the years, I've tried to recreate some of these traditions for my own children, wanting.
them to experience some of the joy that I remember from my childhood, and I've tried.
to recreate some of the joy that I remember from my childhood..
And there's nothing inherently wrong with this..
But recently, God got my attention and shared something with me that I'd like to share.
with you today..
A month ago, I was at Bethany on Chungchow Island, having a retreat..
If you don't know what Bethany is, it's a retreat centre on Chungchow Island, very.
cheap, but a beautiful place that you can go if you just want to stop..
You can either take your family or go as an individual..
I highly recommend it..
And when I go there, there's a place I go to sit in the morning..
It's out on the rocks on a cliff overlooking the water..
Every time I go to Bethany, I visit this same spot and have a quiet time..
It's become a bit of a tradition..
Well, this time when I got there, it began to rain..
At first, it was a light shower, and then it became a downpour..
I didn't want to leave..
This was my tradition..
So I began searching for a dry place..
I eventually found one tucked under some trees..
However, as I was sitting there, I realised I could see none of the views, none of the.
beautiful views that I loved going to that place for..
So I began searching again..
And after 10 minutes, I stopped and I said, "God, you knew I was going to be here..
You prompted me to come..
What are you trying to tell me?".
And I felt God say to me, "You can continue to waste time trying to recapture an experience.
you've had in the past, or you can embrace the reality of this moment, this new moment.
in time, and create a new experience, a new memory.".
You see, when we try to recapture an experience or memory from the past, you might create.
a version of it, like sitting in the dry space under the tree..
But you'll miss out on the views..
You'll miss out on the beauty of the new things you could have experienced..
At that first Christmas, Joseph's family were trying to hold on to their reputation,.
their traditions..
But the reality was that life had changed..
Circumstances had changed..

$^{201}$Joseph and Mary's life had changed..
Yes, it was messy, but they were family..
Instead of embracing this, Joseph's family chose to try and keep up the appearance of.
the perfect family by distancing themselves from the mess..
So let me ask again, what would our Christmas look like if our sole focus was to ensure.
that the people in our lives came away from Christmas fully loved and fully assured of.
their value?.
Would it look a little more messy?.
Would there be more people at our table?.
Would there be moments of tears as we held someone who might be experiencing their first.
Christmas without their spouse or their friend or their grandparent or their child?.
If we relinquished our preconceptions of what Christmas should be and instead made it a.
time of welcome, a time of inclusion where rejection is undone and loss is acknowledged,.
a time of deep love, a time where we reaffirm value despite the chaos and despite the mess,.
maybe, just maybe, we will see wondrous new views and unexpected beauty in the midst of.
the mess..
Something like what the wise men experienced..
In the middle of all the mess of Herod and his scheming, the wise men go on their way.
and the star they'd seen in the east goes ahead of them and it stops over the place.
where the child is..
When they saw the star, they're overjoyed and on coming inside, they see the child with.
his mother Mary and they bow and worship him..
They open their treasures and they present him with gifts..
The wise men got to see this wondrous new view, the brand new baby born in the midst.
of the chaos of that time..
Upholding the traditions of Christmas, the food, the presents, the decorations, it can.
be stressful..
I'm a little bit stressed at the moment because my four grown children have all their.
presents wrapped under the tree and I don't have one..
Very stressful..
The first Christmas came with some gifts but in every other way, it was very ordinary..
Why did God bring his son into the world this way?.
Almost in the poorest, humblest way..
Why?.
When he could have arranged it so differently?.
I think there are two main reasons..
I think he wanted to identify with his creation, with us..
With the least..
To tell us this Christmas that he understands what it is to be without..
He understands what it is to be estranged from family..

$^{241}$To be lonely and in need..
He understands what it is to feel rejected, to feel unseen..
And to be labelled invaluable by society..
In our mess, our chaos, our confusion, our disappointment, our loneliness, our exclusion,.
our rejection, Jesus can look us straight in the eye and he can say, "I fully understand..
I've been there.".
Another reason I believe God chose to bring his son into the world in this way is to loudly.
proclaim that he would not be all about the bells and the whistles, all the external trappings.
of wealth and position and privilege and pride..
He would not be like any earthly king..
His kingdom would not focus on the outward..
But the inward..
Can we be reminded this Christmas that it is not about the bells and whistles?.
Can we lay aside any stress that is building?.
Can we lay aside any strivings for the perfect Christmas?.
Can we learn from Jesus what Christmas is truly all about?.
Christmas can be simple..
And Christmas can be messy..
Christmas can be different from what it has been before..
This Christmas, can we be challenged to embrace the new moment we are in?.
Can we be challenged to create some new views?.
Can we open up our family home to those who need it the most?.
Can we be challenged to reconcile and invite family members back in who've previously been.
shut out?.
Instead of lavishing expensive gifts on people, can we give them the gift of our time,.
our presence, our love?.
Can we make sure that the least in our worlds are remembered?.
That they're visited, that they're given gifts?.
I want you to stop and think, who are the people in your life,.
who are the least that God has surrounded your life with?.
Are they domestic workers?.
Are they elderly relatives?.
Are they refugee families?.
Are they people who have no one to spend Christmas with?.
There's a lot of people in Hong Kong who have no family here to spend Christmas with..
And in fact, when we came to Hong Kong, we were very lonely..
And when it came to Christmas time, we decided that we would open up our Christmas table to.
anyone who didn't have a place to go..
And we forged some of the most beautiful relationships from those times..
Some of the people we've shared Christmas with are in the services today..

$^{281}$Some of them will be joining us on Christmas Day still..
If in our loneliness we had not reached out, if we'd not been vulnerable, we would have.
missed out on all those beautiful new views, all those beautiful friends that we now have,.
that we've made family..
Jesus' purpose in coming was to make sure every human knows their full worth, that they.
are fully loved and fully valuable..
Can this become our purpose this Christmas?.
Can you stand with me?.
Father, we just come to you..
And in the midst of all this crowd, each of us is seen by you..
Because you don't see a crowd, God, you see individuals that you love, that you know,.
that you value..
Father, we just ask that this Christmas we will take the opportunity to reimagine it..
See what it could be..
Ask you if there's something new that you want us to do with it..
Someone you want us to draw near..
Someone you want us to love better..
Someone you want us to reconcile with and to give a second chance to..
Father, we just remember in this moment how loved we are..
Wrap us up afresh in your kindness and your love..
And Father, this Christmas,.
may it be a little bit messy..
May we be willing to open up ourselves to share it with a few more people..
To share it with someone maybe who's grieving..
And Father, I also pray for people standing here today..
And no one's ever told you that you were composed with all the beautiful notes..
You've never heard that you are so loved..
That God didn't even withhold his own son..
He wanted you to know that you are fully valuable..
Fully treasured..
And if you don't know that today, then I encourage you.
to pray with someone, to let someone know..
Because you are loved with a love that has no end to it..
Father, for those of us who've been blessed with much,.
show us ways to make the least be seen this Christmas..
That they might know that they are known and that you see them..
[Music].
Father, we just thank you for the example that you gave us of that first Christmas..
You didn't make it something that was so out of reach that none of us could reach it..
You made it something so simple..

$^{321}$Just relationships..
[Music].
No material things..
We can all do that, God..
Just great love..
Fill us all afresh with your great love..
Your capacity that goes beyond offences..
That goes beyond estrangement..
That goes beyond our limits..
And makes a way where there is no way..
This Christmas, God..
May the people in our lives leave Christmas fully assured of that they are loved..
Fully assured that they are valued and that they are known..
In your precious name..
Amen..
\newpage



\section{}
\label{sec:Ini7uoDvO7A}
\textbf{2023-12-25 聖誕節崇拜 | 廣東話崇拜 [Ini7uoDvO7A].mp3}
\newline
\newline
連結: \href{https://youtube.com/watch?v=Ini7uoDvO7A}{\texttt{ https://youtube.com/watch?v=Ini7uoDvO7A}} ~~~~ 語音日期: 2023-12-25 
\newline
\newline
\hyperref[sec:HxTFQWb_Guo]{\small{< < < PREV SERMON < < <}}
~
\hyperref[sec:index]{\small{[返主目錄]}}
~
\hyperref[sec:9el2lQ77_AE]{\small{> > > NEXT SERMON > > >}}
\newline
\newline
$^{1}$一生一世地建造神的王國.
這座教堂的核心是.
耶穌基督的悲傷故事.
這座教堂的悲傷故事.
是以自由和榮耀為主題.
這座教堂並不被建築的牆壁.
而是被建築在這座教堂的社區中.
希望神的王國能在地球上生長.
這座教堂認為救贖是一種合理的事實.
以神的工作為人類而改變.
以神的心為代表的一代.
對於負責和迫害的人們來說.
這座教堂的目標是.
讓家庭能夠成為.
它所要的一切.
包括意識的 專業的.
和真正的關係.
在每個層次之中.
這座教堂是.
為年輕人和老年人.
所需要的一種地方.
支持和訓練.
這座教堂的背景和傳統.
都能感覺到是家.
是文化的牆壁.
以寬容與富裕為榮耀.
以耶穌基督為主的教堂.
這座教堂是.
為人類和家庭的傳統.
而建立的.
以耶穌基督為主的.
一種建議和信仰.
以耶穌基督為主的.
一種信仰.
這座教堂是.
為人類和家庭的.
傳統.
以耶穌基督為主的.
一種信仰.
以耶穌基督為主的.

$^{41}$一種信仰.
以耶穌基督為主的.
一種信仰.
以耶穌基督為主的.
一種信仰.
以耶穌基督為主的.
一種信仰.
以耶穌基督為主的.
一種信仰.
以耶穌基督為主的.
一種信仰.
以耶穌基督為主的.
一種信仰.
以耶穌基督為主的.
一種信仰.
以耶穌基督為主的.
一種信仰.
以耶穌基督為主的.
一種信仰.
以耶穌基督為主的.
一種信仰.
以耶穌基督為主的.
一種信仰.
以耶穌基督為主的.
一種信仰.
以耶穌基督為主的.
一種信仰.
以耶穌基督為主的.
一種信仰.
以耶穌基督為主的.
一種信仰.
以耶穌基督為主的.
一種信仰.
以耶穌基督為主的.
一種信仰.
以耶穌基督為主的.
一種信仰.
以耶穌基督為主的.
一種信仰.
以耶穌基督為主的.

$^{81}$一種信仰.
以耶穌基督為主的.
一種信仰.
以耶穌基督為主的.
一種信仰.
以耶穌基督為主的.
一種信仰.
以耶穌基督為主的.
一種信仰.
以耶穌基督為主的.
一種信仰.
以耶穌基督為主的.
一種信仰.
以耶穌基督為主的.
一種信仰.
以耶穌基督為主的.
一種信仰.
以耶穌基督為主的.
一種信仰.
以耶穌基督為主的.
一種信仰.
以耶穌基督為主的.
一種信仰.
以耶穌基督為主的.
一種信仰.
以耶穌基督為主的.
一種信仰.
以耶穌基督為主的.
一種信仰.
以耶穌基督為主的.
一種信仰.
以耶穌基督為主的.
一種信仰.
以耶穌基督為主的.
一種信仰.
以耶穌基督為主的.
一種信仰.
你們是為了什麼而來的呢.
為了邀請你的朋友而來呢.
還是為了小朋友而來呢.

$^{121}$當然是聖誕節.
我們當然是為了耶穌而來.
但是耶穌在這裡當中.
更加重要的是要我們一起.
我們教會.
大家走在一起.
這個就是教會.
這個建築不是叫教會.
我們走在一起.
這群人叫教會.
所以今天我們教會.
我們一起就是來慶祝.
耶穌基督的誕生.
聖誕節.
就是一個這麼重要的意義.
所以雖然這麼冷的天氣.
也很歡迎大家來到這裡當中.
有冷天氣.
更加有節日氣氛.
所以大家也可以穿一下.
一年只能穿一兩次的衣服.
可以找出來.
不如我也邀請大家起來.
我們有一個祈禱.
然後我們由我們的敬拜開始.
今天我們聖誕節的崇拜.
好不好.
我們一起同心祈禱.
親愛的天父.
多謝讓我們可以去紀念這一天.
因為我們知道.
這一天臨在我們這裡當中的時候.
值得我們一起去慶祝.
因為你的救恩臨在我們裡面.
因為你的歡欣.
你的喜樂.
你的平安.
同樣臨在我們當中.
求你帶領我們.
求你的靈在我們當中.

$^{161}$讓我們一起去敬拜.
讓我們用一個喜樂.
用一個慶祝的心情.
一起去敬拜耶穌基督.
我們今天可以和我們的小朋友.
一起去敬拜.
絕對是你的恩典.
我們同心敬拜.
奉主耶穌基督的名.
Amen.
《天地作》 詞:陳汝佳 曲:陳汝佳.
天地作 詞:陳汝佳 曲:陳汝佳.
來聽天使的歌歌.
心心關注大榮耀.
來問遍地處事人.
天地作不得收報.
歡呼喜愛與萬壽.
一起歌頌放注天.
天官之名也中槍.
奇妙救主已降生.
青天骨靈中唱.
榮耀救主已降臨.
來聽天使全歌歌.
心心關注大榮耀.
來問遍地處事人.
天地作不得收報.
歡呼喜愛與萬壽.
一起歌頌放注天.
天官之名也中槍.
奇妙救主已降生.
青天骨靈中唱.
榮耀救主已降臨.
《青雲》 作詞:陳汝佳 作曲:陳汝佳.
《青雲》 作曲:陳汝佳.
聖誕來到主前.
快樂終身歡欣.
敬拜救贖.
萬世主於百里行乘.
天使降昌旺.
救主誕生萬壽.

$^{201}$敬拜忠贊主我亡.
敬拜忠贊主祈禱.
忠贊歌聲永在.
耶穌基督降世.
曠野擁有精神.
引導至這祖籍.
到處願沈溺救主.
將乃萬獻上.
天使降贊祖.
至高處歸我身.
敬拜忠贊主我亡.
敬拜忠贊主祈禱.
忠贊歌聲永在.
耶穌基督降世.
嘆美崇拜君王.
金剛洗於世上.
見證萬民屬次歡.
倦有應還.
主愛寄惠丹.
將顯救恩養.
敬拜忠贊主我亡.
敬拜忠贊主祈禱.
忠贊歌聲永在.
耶穌基督降世.
敬拜忠贊主我亡.
敬拜忠贊主祈禱.
忠贊歌聲永在.
耶穌基督降世.
敬拜忠贊主我亡.
敬拜忠贊主祈禱.
忠贊歌聲永在.
耶穌基督降世.
敬拜忠贊主我亡.
敬拜忠贊主祈禱.
忠贊歌聲永在.
耶穌基督降世.
天使升頭同雨.
報祖千行福.
將老舍下冠.
連同罪與主面前一起敬拜.

$^{241}$主陪這永夜.
因所有賜予你.
因所有忠於你.
主你陪照所有.
天使陪這永夜.
因所有賜予你.
因所有忠於你.
主你陪照所有.
天使升頭同雨.
報祖千行福.
將老舍下冠.
連同罪與主面前一起敬拜.
主陪這永夜.
因所有忠於你.
主你陪照所有.
主陪這永夜.
因所有賜予你.
因所有忠於你.
主你陪照所有.
讓每天用我心.
對眾贊為主.
讓每天用我心.
對眾贊為主.
讓每天用我心.
對眾贊為主.
讓每天用我心.
(我們再唱一次).
讓每天用我心.
對眾贊為主.
讓每天用我心.
對眾贊為主.
讓每天用我心.
對眾贊為主.
讓每天用我心.
(我們再唱一次).
祝祝主陪這永夜.
因所有賜予你.
因所有忠於你.
主你陪照所有.
祝祝主陪這永夜.

$^{281}$因所有賜予你.
因所有忠於你.
主你陪照所有.
主耶穌說你是配得的.
因始於你 終於你.
因創始成終 因你.
因你的降生 將你救恩.
臨在我們的世界當中.
在過去二千多年.
你將你的恩典.
不斷給你所愛的人.
到今天.
我們仍然能夠這麼去頌唱.
主耶穌基督你的降生.
你的救恩.
是在我們的生命當中.
主是你配得的.
這個不是純粹一個節日.
這個是實實在在.
你臨在我們的當中.
主是你配得.
我們所有一切的贊美.
因為主.
你就是道成肉身.
用愛將我們去拯救過來.
用愛將這個世界去轉化.
今天我們同心頌唱的是你的名字.
有這個家庭.
我們可以一起去頌唱.
你的美好 你的恩典.
我們禱告.
祈奉主耶穌基督得勝的名球.
Amen.
大家請坐.
曾經有個電視台.
他們在生日的時候.
就會有一年一度的台慶劇.
果然是一個特別的聖誕.
因為我們的主角.
都相當緊要和重要.

$^{321}$我們在聖誕節的主角是誰.
耶穌.
好像沒有肯定.
我們的主角是誰.
耶穌.
所以我們的台慶劇.
都是圍繞著耶穌的出生.
而我們真的重金禮聘.
當然是用了很多的糖果.
用了很多的愛心和鼓勵.
一眾全宇宙最出色的最佳男主角.
最佳女主角.
最佳的小羊和最佳的星.
我們今天的新星.
也是在今天的誕生.
所以很希望待會.
如果有任何的歡樂的時候.
他們都為了這件事排練了很多次.
兩次那麼多.
我們在這個時候.
都將我們的舞台.
交給我們的明日之星.
我們有請我們今天的台慶劇.
大家好.
請咒詛.
今天跟大家講一個故事.
唱一首歌.
聖誕節怎麼發生.
你有沒有聽過.
拍照之余.
記得為我們拍拍手.
音樂起先.
音樂起先為我們一起唱歌.
從前有一個女孩叫瑪麗亞.
有一天她聽到消息.
嚇得她叫了出來.
瑪麗亞不用怕.
我有好消息告訴你.
你會生一個寶寶.
你會叫他做耶穌.

$^{361}$希望耶穌想說的話會成真.
在羅馬的希律王出了條律例.
要大家都回到家鄉.
對約瑟來說.
就是要他回到伯利恆.
跟著就可以探望人.
大家去回到出生的地方.
哇.
當他們去到伯利恆.
瑪麗亞肚裡的嬰兒開始想出來了.
但是找不到落腳的地方.
店主店主店主.
你們還有沒有房.
我是約瑟.
我太太快要生了.
我們只剩下一個馬槽.
你今晚就跟他們睡吧.
於是那晚.
約瑟和瑪麗亞就在馬槽里生了耶穌.
圓圓在馬槽里.
我怎麼沒錯.
小小的豬耶穌.
水澆向岸河.
從明星都望著.
珠水的地方.
小小的豬耶穌.
水在崗出生.
在這個時候.
有一群牧羊人.
在草原上放羊.
突然有一群天使.
在他們面前出現.
我有好消息告訴你們.
今天在伯利恆.
我們的救世主出生了.
你們會在馬槽里看到他.
快點走.
沒有時間睡.
要快點去找他.
在最高之處.

$^{401}$榮耀歸於神.
在地上平安留到.
他所喜悅的人.
有三個使者在東邊來.
他們想去找新生王耶穌.
我們一起跟著流星去找耶穌.
他們帶了禮物.
和我們一起唱歌.
大家圍著耶穌不分離.
我一起分享耶穌降生的快樂.
祝你們聖誕快樂.
祝你們生日快樂.
不行.
我拍不到照片.
這些可愛的小朋友.
哇.
不是明日之星.
是今天的明星.
我們教會有盼望了.
好可愛.
接下來我們投票.
我最喜愛的男女演員.
得獎者可以有一塊曲奇.
等小朋友慢慢安頓.
有什麼比小朋友更好.
那麼美好的小朋友.
那麼可愛的小朋友.
去跟我們說聖誕的信息.
聖誕信息好像經常有聽.
經常耳熟能詳.
但是如何去領受.
其實是一個最重要的方式.
我們如何去心態.
我們如何去知道.
這個信息原來不是一個戲劇.
這個信息原來和我們每個人.
都是息息相關.
小朋友在我們當中.
整個地方都是毛.
面養的毛.

$^{441}$其實就好像面養.
他一路成長一路長大的時候.
其實他一路去經歷.
很多生命中的事情.
其實都是經歷很多的恩典.
剛才除了大家很熱烈地.
拍照和拍掌的時候.
有沒有聽到有一句很powerful的說話.
是由小天使去跟那些小牧羊人說.
小天使和小牧羊人說.
其實都是跟我們說的.
有沒有印象.
其實是最難說的那句.
在至高之處 榮耀歸於神.
原來出來了.
在至高之處 榮耀歸於神.
我們不如一起說.
在至高之處 榮耀歸於神.
在地上平安歸於他所喜悅的人.
哇.
這個是硬是讓他們說的.
但這句話是很重要的.
這句話是當時天使上帝去跟眾人所說.
他是一個宣告耶穌基督的誕生.
不如我們整段經文看看.
這是在《路加福音》第二章出來的.
第二章記載到.
那天天使對他們說.
不要懼怕 看啊 因為我報給你們大喜的信息.
是關乎萬民的.
因今天在大衛的城裡.
為你們伸了救主 就是主基督.
你們要看見一個英雄.
包著布 餓在馬槽里.
那就是給你們的記號.
忽然有一大隊天兵和那天使贊美神說.
在至高之處 永耀歸於神.
在地上平安歸於他所喜悅的人.
沒錯這番話是要告訴大家.
神的兒子來到了.

$^{481}$他在那個很混亂的世界和世人的惶恐當中.
上帝要借著耶穌帶來一個很重要的一件事.
平安.
我們看見經文當中.
平安.
這份平安是非常非常重要的聖誕信息.
這份平安是非常重要的我們基督教信仰的信息.
希利文原文叫做Salom.
Salom有不同的解釋在字根里.
Peace, wholeness, completeness.
即和平 完全 完整的意思.
但同時帶有well being.
harmony security.
幸福 幸福 和諧 和睦.
安全 有保障.
我們經常說平安 祝你平安.
其實這份平安不單止在我們內心裡面.
是一個感受.
沒錯 這個很重要.
我們在混亂的世界當中.
世界可能還沒改變的時候.
我們才有這份從基督而來的平安.
臨在我們的心裡面.
但不止這樣.
我們的信仰從來不是這樣.
我們在聖經舊約新約.
到現在過去幾千年的歷史裡面.
平安不是一種感覺.
平安是實實在在.
直到耶穌要臨到我們當中這裡.
祂不只是說一個幸福感.
祂是改變生命的時候.
導致幸福呈現在世人當中.
祂那份和諧平安.
正正是世人的改變.
導致事情就這樣發生.
所以我們才會見到基督的真實.
耶穌的臨在是實實在在.
在我們每一個生命家庭和社會當中.
所以這個平安也是我們在10月1號.

$^{521}$我們在這裡有我們的Grand Opening.
第一天正式開始我們.
元朗葡萄藤教會的事工.
我們當天的主題就是.
我們在說平安的福音.
平安的福音同樣地因為主耶穌基督.
臨在這裡當中.
所以我們元朗葡萄藤教會的核心價值.
都是這份平安的福音.
我們懷著這份Calling.
我們有一份使命是借助耶穌基督.
耶穌基督是一個.
祂自己本身是和平之子.
祂自己本身是和平的締造者.
我們一起被呼召來這裡.
將這個福音.
這個和平的福音宣揚出去.
在過去四星期.
我們四個主日.
四個禮拜日.
我們看到耶穌如何在一個很混亂的世界.
在祂自己仍然很混亂的世界當中的時候.
發生.
這個的呈現是導致肉身.
將這份盼望.
用祂自己很實實在在地呈現出來.
導致我們所信的.
不只是一個信念.
而是一個歷史.
而是真真正正.
在我們生命裡面都可以發生的事情.
我們每一個基督徒.
我們每一個耶穌基督的跟隨者.
其實我們都能夠去見證耶穌的誕生.
因為耶穌的誕生.
不只是在二千多年前.
耶穌的誕生.
其實是分在我們的生命當中.
祂是再一次誕生.
上帝創造天地.

$^{561}$但是人的墮落.
但同時換來的是主耶穌基督的降生和救贖.
我們每一個都在活在的世界當中.
我們每一個都有我們自己的事情.
但這就是耶穌的臨在在我們的生命當中.
讓我們和祂一起.
我們叫做新的創造.
就是重新活過來的盼望.
所以聖誕節是很重要的一日子.
很重要的一件事情.
我們不只是去慶祝.
不只是吃大餐.
不只是喝好酒.
喝好的紅酒.
昨天都喝了幾瓶.
所以不是我一個人.
和其他人一起.
但今天都精神奮鬥.
很影響我們崇拜.
一起敬拜主的心.
但這個節日值得慶祝的.
最值得慶祝的.
是那份平安.
實實在在.
臨在我們當中.
我們再看同一個章節後面有兩段經文.
十五節和二十節.
眾天使離開他們.
升天去了.
牧羊人彼此說.
我們往百里行去.
漢漢所成的事.
就是主所告訴我們的.
二十節說到.
那群牧羊人回到那裡.
因聽見所看見的一切事.
正如天使向他們所說的.
就歸榮耀於神.
贊美他.
今天是聖誕節.

$^{601}$2023年.
2023年12月25日.
是嗎?聖誕節.
我們不僅是回顧過去.
剛才June用台慶來形容.
我們也不是2023大事回顧.
那麼簡單.
我們是要展望.
我們是2024展望.
我們是2025,2026.
我們是對將來展望.
因為耶穌給我們這份盼望.
就好像那些牧羊人一樣.
他們見到.
見到耶穌的誕生.
是一份喜訊.
今天在座的.
如果朋友邀請到你們來.
其實是一份喜訊.
剛才我問了大家的問題.
你們是為了什麼而來的呢?.
其實更加重要的一個問題.
我們要知道.
耶穌是為了什麼而來的?.
耶穌是為了大家而來的.
耶穌是為了大家在座每一個人而來的.
耶穌是為了你們所愛的人而來的.
耶穌也是為了我們這個地區.
這個城市.
這個國家.
這個世界而來的.
因為耶穌是為了愛而來的.
剛才所說的每一個基督徒.
為什麼我們說能夠見證基督的誕生呢?.
今天我們在這裡給的信息.
其實不是要.
啊!聖誕節.
大家要信主了.
舉手然後Hallelujah.
然後記下.

$^{641}$然後下一次抓你回來教會.
是嗎?.
耶!那當然是最好.
但不是這樣.
我們不是要你去信.
我們是要你去見到.
因為真的見到耶穌在這裡當中.
我們這群人.
我們小朋友.
我們每一個星期的崇拜.
是因為我們見到主耶穌在我們當中.
不是一個感覺來的.
只是一份經歷來的.
所以耶穌基督是在我們裡面.
所以好像剛才所說.
牧羊人所見到.
我們要告訴別人.
這個這麼大的喜訊.
所以我們在這裡當中每一個.
你見到耶穌.
所以我們都要將這份.
大好的訊息像牧羊人一樣.
我們要將它告訴其他人.
好不好.
那不如我們今天.
不再說20分鐘的信息了.
今天是一個大家很希望一起.
去慶祝的一個聖誕崇拜的日子.
不如我也邀請大家再一次起身.
我們用一個贊美的歌聲.
贊美的心.
喜悅的心.
我們一同將我們自己.
我們自己先去和耶穌主基督相遇.
好不好.
我們也請worship team.
可以上到我們當中.
為我們一起帶我們去敬拜.
邀請大家可以閉上眼睛.
我想為大家先有個祈禱.

$^{681}$昌爾天父我們感恩.
多謝你.
多謝你讓我們看到一個這麼美好的訊息.
這個美好訊息其實連小朋友都知道.
連小朋友都知道.
連幾歲的小朋友都知道.
這個不是一個要去.
好像世間說.
要去用很多力氣.
要讀很多神學.
或者甚至說被洗腦.
不是.
這個是很實在.
可以是很簡單的一個訊息.
就是主耶穌基督的降生.
主耶穌基督的臨在.
就是要在我們的生命當中.
主耶穌就是為了我們而來.
因為這份是基督的救恩.
祂說祂來是帶給我們每一個.
這份是救恩.
這份是喜樂.
這份是平安.
今天直著在這裡我們一起去領受.
我們同心一起去敬拜.
我們讓到這個歌聲.
在我們當中的時候.
我們知道我們一群弟兄姐妹.
同樣相信的.
同樣看到的.
就是主耶穌基督的臨在.
我們一起來聽聽.
《來聽天使靜歌歌》.
來聽天使靜歌歌.
聲聲關注大明人.
明問遍地諸斯人.
天祈祖家得壽母.
歡呼喜樂與滿足.
一起歌頌放注至.
天官天靈也中槍.

$^{721}$奇妙救主已降生.
青青尖骨正中槍.
榮耀救主已降臨.
來聽天使靜歌歌.
聲聲關注大明人.
明問遍地諸斯人.
天祈祖家得壽母.
歡呼喜樂與滿足.
一起歌頌放注至.
天官天靈也中槍.
奇妙救主已降生.
青青尖骨正中槍.
榮耀救主已降臨.
Hallelujah, Hallelujah.
我們主耶穌.
Worship Team.
小朋友在不在.
我們小朋友.
我們今天紀念耶穌基督的誕生.
但其實我們都希望藉著這個機會.
展示一個欣賞.
給一位在我們當中的.
好好的姐妹同工.
今天其實是June的生日.
Come on, come on, come on.
你完全不知道的,對,對不起.
對不起.
耶穌基督降生的時候.
都沒有人知道的,對.
多謝June為我們安排這麼好的.
一個話劇.
還有一直以來幫助我們的小朋友.
在K-Fonts裡面.
不如我們唱一首生日歌,好嗎.
也可以的,對嗎.
好.
(生日禮物).
這個我也不知道.
原來有我的份.
Happy Birthday, Happy Birthday.

$^{761}$Happy Birthday to you.
Happy Birthday to June.
Happy Birthday to you.
Happy Birthday, Happy Birthday.
謝謝June.
我們說曲奇派對其實不只是曲奇.
但你盡量去享受那個時間.
希望我們有一個很好的,愉快的.
一個大概.
我們還有一個多小時可以.
因為11點半我們才有英文崇拜.
所以我們可以一起.
在這裡有我們的.
我們叫做Community Time.
我們一起可以慶祝聖誕節,好嗎.
我們的英文小朋友.
都會在Playroom那裡準備.
所以如果你有小朋友.
剛才有在劇裡面的話.
你也帶他們去吃曲奇餅.
他們也會為英文小朋友.
去預備Play的時候用的.
對,所以家長可以接回小朋友.
那邊就會繼續去準備.
下一個崇拜的話劇,好嗎.
聖誕快樂.
待會見,或者下個星期日.
31號,我們再見.
謝謝.
\newpage



\section{}
\label{sec:9el2lQ77_AE}
\textbf{2023-12-26 Christmas Message | English Service Live [9el2lQ77\_AE].mp3}
\newline
\newline
連結: \href{https://youtube.com/watch?v=9el2lQ77_AE}{\texttt{ https://youtube.com/watch?v=9el2lQ77\_AE}} ~~~~ 語音日期: 2023-12-26 
\newline
\newline
\hyperref[sec:Ini7uoDvO7A]{\small{< < < PREV SERMON < < <}}
~
\hyperref[sec:index]{\small{[返主目錄]}}
~
\hyperref[sec:t5Cfd_ii5kM]{\small{> > > NEXT SERMON > > >}}
\newline
\newline
$^{1}$Imagine with me for a moment a church..
This is not a church in the singular, on its own and distinct, but a church in the plural,.
united across a nation, joined in purpose, focused on building up God's kingdom one.
life at a time..
It is a church that has at the heart of its DNA the grand narrative of the redemptive.
story of Jesus Christ, and that celebrates in the freedom and grace that such a story.
brings..
This church is not constricted by the walls of a building, but actively engaged in the.
community it is planted in, desiring for God's kingdom to take root and bear fruit in local.
soil..
A church that sees salvation as holistic, transforming God's people for God's work.
and equipping a generation to represent God's heart to the poor, the weak, the oppressed.
and the persecuted..
Imagine this church committed to seeing the family as integral to all that it's called.
to be and do, involving intentional, committed and authentic relationships across all levels..
It is the sort of place where both the young and the old feel cared for, supported and.
equipped..
A church where people of all backgrounds and traditions come to feel at home, a cultural.
mosaic that honors and embraces a richness in diversity that marks us as the body of.
Christ..
It is a church that is willing to innovate and express faith creatively, reflecting the.
boundless limits of God's creativity as glory to Him..
A place where people gather, hungry for the presence of God and passionate to see this.
invade all aspects of their lives, where they seek to develop an intelligent, spiritual.
faith rooted in God's word and driven with prayer..
A church that does not make its Sunday services the exclamation mark of its existence, but.
instead the comma in its ongoing sentence, a time of service, love and engagement that.
inspires and emboldens all equally for the conversation that lies ahead..
A church that cares, disciples, rebuilds, renews..
This is our city..
This is our church..
This is our home..
Good morning everyone and welcome to the Christmas Day service..
It's so good to have you..
We have lots of reason to be thankful and joyful..
We have a lot of exciting things planned, but before we do that, I want to invite you.
to turn to your neighbor and say Merry Christmas..
Yes, Merry Christmas everyone..
Well, I want to invite everyone to stand and I'll pray for us and then we'll sing joyful.
Christmas songs..

$^{41}$Let me pray..
Heavenly Father, we thank you so much for sending Jesus..
Jesus, we thank you for coming to us because you love us to rescue us, to show us your.
love, to lay down your life for us so that we can have joy, that we can know you..
And today we just want to lift up these songs and give you praise and say thank you Jesus.
for coming..
Thank you that you love us..
Thank you that you give us new life and that we can know that..
Thank you that you give us peace that we can experience..
So we pray all this in your name, Jesus. Amen..
The herald angels sing, glory to the newborn king..
Peace on earth and mercy wild..
God and sinners red with blood..
The herald angels sing, glory to the newborn king..
Joy the triumph of the skies with the jubilant hosts proclaim..
Christ is born in Bethlehem..
Hark the herald angels sing, glory to the newborn king..
Christ my highest heaven adored..
Christ the everlasting Lord..
All on high behold him come..
Offspring of the Virgin's womb..
Held in flesh the Godhead see..
Held in mind a deity..
Christ the Spirit meant to dwell..
Jesus our Emmanuel..
Hark the herald angels sing, glory to the newborn king..
[MUSIC PLAYING].
[MUSIC - "THE STAR SPANGLED BANNER"].
Oh, come all ye faithful, joyful and triumphant..
Oh, come ye..
Oh, come ye to Bethlehem..
Come and behold him, Lord, the King of angels..
Oh, come let us adore him..
Oh, come let us adore him, Christ the Lord..
Sing choir..
Sing choirs of angels..
Sing in exultation..
Oh, sing all ye citizens of heaven above..
Glory to God..
Glory in the highest..

$^{81}$Oh, come let us adore him..
Oh, come let us adore him, Christ the Lord..
Oh, Lord, we greet thee, born this happy morning..
Jesus, to thee be all glory given..
Word of the Father, now in flesh appearing..
Oh, come let us adore him..
Oh, come let us adore him, Christ the Lord..
Sing it again..
Oh, come..
Oh, come let us adore him..
Oh, come let us adore him, Christ the Lord..
[MUSIC - "WE'LL PRAISE YOUR NAME FOREVER"].
Sing, we'll praise..
We'll praise your name forever..
Christ the Lord..
Just some voices..
We'll praise your name forever..
Christ the Lord..
Let's give him a shout of praise..
Hallelujah..
God, we praise you..
[APPLAUSE].
[MUSIC - "GOD, YOU ARE WORTHY OF ALL PRAISE"].
God, you are worthy of all praise..
Jesus..
Sing, all the saints..
All the saints and angels bow before your throne..
All the elders cast their crowns before the Lamb of God..
Let's sing, all the saints..
All the saints and angels bow before your throne..
All the elders cast their crowns before the Lamb of God..
Let's sing..
Lift it up..
You are worthy of it all..
Yes, you are..
For from you are all things..
And to you are all things..
You deserve the glory..
We're singing..
Oh..

$^{121}$All the saints..
All the saints and angels bow before your throne..
All the elders cast their crowns before the Lamb of God..
You are worthy..
You are worthy of it all..
Yes, you are..
You are worthy of it all..
Worthy..
Oh..
For from you are all things..
And to you are all things..
You deserve the glory..
Sing again..
You are worthy..
Oh..
You are worthy of it all..
Oh..
You are worthy of it all..
For from you are all things..
And to you are all things..
You deserve the glory..
Oh..
You are worthy, Jesus..
Yes, you are..
You are worthy, Jesus..
Let's sing it out..
Day and night..
Night and day..
Let it sense to rise..
Day and night..
Night and day..
Let it sense to rise..
Day and night..
Night and day..
Let it sense to rise..
Day and night..
Night and day..
Let it-- let's sing it again..
Day and night..
Night and day..

$^{161}$Let it sense to rise..
Day and night..
Night and day..
Let it sense to rise..
Day and night..
Night and day..
Let it sense to rise..
Day and night..
Night and day..
You are worthy of it all..
Oh, you are worthy of it all..
For from you are all things..
And to you are all things..
You deserve the glory..
Let's sing it out..
Just the voices..
You are worthy of it all..
Worthy God..
For from you are all things..
And to you are all things..
You deserve the glory..
Yes, Jesus, you are worthy of it all..
We praise you and we thank you that you have come because you.
love us, to rescue us, to give us new life,.
to put us on a different trajectory,.
to give peace where there was no peace,.
to give hope where we had lost hope,.
to give comfort where we felt like nobody could comfort us,.
and also to remove any shame that we are carrying..
So we thank you for your love..
We celebrate you today on Christmas that you have come,.
which is the greatest sign of love..
So we pray all this in your name, Jesus..
Amen..
[applause].
Good morning..
You may all have a seat..
So I know that last time when Pastor Tim preached,.
he gave you guys a pop quiz..
And I'm just going to follow up that tradition and give you guys.

$^{201}$a pop quiz as well, okay?.
I know that last time Pastor Tim gave you guys a pop quiz,.
and so here it is..
How many sheep were there when the angel showed up to the.
shepherds?.
How many were there?.
Take a guess, take a guess..
99, I heard someone say..
Three, three..
Any other guesses?.
How many?.
Five, five, five..
Okay, so in a minute you are going to hear very biblically.
correct -- no, I'm joking, no..
You are going to watch a really excellent performance by our.
fantastic kids at K4C..
Let's give them a round of applause first..
[applause].
They have worked really hard, so during the show,.
please sing along with them, please clap along with them,.
give them the encouragement that they deserve..
And I hope that whatever happens,.
we're all going to have a great time..
And now, without further ado, let's introduce our amazing kids..
[applause].
Greetings all..
Now please take a seat..
We'd like to give you a small Christmas treat..
This play is about Christmas..
We'll do it in a rhyme..
Some people have lines while others will rhyme..
So don't forget cameras..
Please take a snap and always remember to sing with us and clap..
Our play will begin with a girl who's called Mary,.
a lovely girl who heard something scary..
[baby crying].
Mary, you are going to have a baby boy,.
and you will call him Jesus..
Wow, what an exciting day..
[laughter].

$^{241}$Meanwhile, Caesar and Rome made a decree..
You must all go home..
For Joseph this month off to Bethlehem town,.
so they rode on their donkey all the way down..
Everyone has to go back to their hometown..
[applause].
Where's the star?.
There's supposed to be a star, right?.
When they got to Bethlehem, the birth day was looming,.
but they couldn't find any hotel with a room in..
[background chatter].
Do you have any room?.
My brother has a room..
You can stay in my stable..
He's here, Joseph said, and Jesus was born in a cold,.
cattled shed..
[singing].
[applause].
So now we see heavenly angels appear to shepherds.
who were worried and shaking with fear..
The angel said, hurry, I bring you good news..
Quickly, get moving..
No time for a snooze..
I bring you good news..
Today in Bethlehem Jesus is born..
You will find him in a manger..
Glory to God..
Glory to God and peace on earth..
[applause].
Three wise men lived far in the east..
They followed the star to get to the feast..
Let's visit the newborn king..
We need to get him gifts..
Let's follow the star..
It will take us to the baby..
They all gathered around..
The new baby boy..
Everyone welcome to share in the joy..
From angels to donkeys, from shepherds to kings,.
share in joy..

$^{281}$Everyone shares the joy that he brings..
No one snapped out no matter how small..
Happy Christmas to one and to all..
[applause].
We wish you a merry Christmas..
We wish you a merry Christmas and a happy new year..
Good tidings we bring to you and your kin..
We wish you a merry Christmas and a happy new year..
A great gift of something important..
Good tidings we bring to you and your kin..
We wish you a merry Christmas and a happy new year..
We all sin against God, so our relationship with God is broken down..
We say hurtful things to each other,.
so there's a lot of breakdown in relationships with other people..
And even our relationship with creation is broken down,.
because we were called to be stewards of creation,.
but yet we treat God's beautiful creation as just something we can do whatever we want to with..
And so what I want to do is I want to actually set things in the larger story of the Bible..
So first, this is a story in four acts..
It's a mini overview..
The first one is God says he created the world, creation,.
and that it was good what he created..
And he also said that when he created humans, that it was very good..
It was the pinnacle of God's creation, because they're made in God's image..
But then we're told that sin enters the world..
It's known as the fall, where people rebel against God,.
and they struggle to trust God, that God really means it well with them..
And so they did what God asked them not to do,.
and that's what we know as the fall, the sin fall..
Now, as already shared, this sin has impacted everything, every fabric of society..
And it's not just that we ourselves sin,.
it's also that there are sinful structures that cause people to be oppressed and suffer and injustices..
And so when we talk about Jesus bringing good news and being the Savior,.
he has come to rescue us from our brokenness, from our sin..
And God's Word also says that we ourselves cannot do it..
We can't do it out of our own strength..
There's nothing within us..
We can't try harder to be a better person..
It's not about putting in that little bit more effort..
God's Word says you cannot do it..

$^{321}$You cannot fix your sin issue yourself..
Now, some of us might wonder, am I even a sinner?.
What is that?.
So I want to say a little bit more about that..
Sin means missing the mark of what God desires of people..
And that means it can be even things we do in our heart..
It's not just that we murdered someone or something like that,.
though that also, of course, is a sin..
But it can be like having judgmental thoughts, envying other people,.
acting selfishly when we know what is good..
And so Jesus has come to rescue us from our sin, to forgive us our sins,.
and to invite us into his family..
And that's why we celebrate Christmas and we're like, "Wow, this is amazing.".
The King of Kings humbled himself, became a little child to start his rescue mission.
so that we can move from being estranged from God to become family of God.
and be in that relationship and experience this peace, this joy, this life.
that Jesus only gives..
Jesus really puts us on a different trajectory..
Before we know Jesus, we're enslaved to sin..
That's what God's Word says..
But when Jesus comes, he rescues us from sin..
He breaks us free..
And then we can learn to live a different way.
because of Jesus being at work in our lives..
And we can rely on him..
He gives us the grace and the strength to become more loving people,.
become the people he created us in the first place to be..
And so this is the incredible news of Christmas..
So there is room in God's family for all people..
It's for everyone..
Nobody is too far away..
Nobody is too broken or too messed up that they are outside the scope.
of Jesus' rescue and love..
Nobody..
So that means that Christmas is also good news for every single person.
because we can receive this gift of salvation, this gift of new life.
that Jesus gives when we place our trust in him and believe in him.
and turn away from our way of living apart from him..
Now, I was watching Instagram..
Anyone?.

$^{361}$Scrolling..
Sometimes you get stuck in that endless scrolling to another thing..
And there was an image that I want to leave us with..
There was a family that asked this little boy to unpack a gift..
And so he unpacks it..
He opens up the package, and he gets out a picture frame..
And on the picture frame was a family picture, but he was not in it..
And then there was a note that said, "Next year, we want you to be part.
of this family picture.".
And so the kid was trying to figure out what this meant, and it was like,.
his name was Connor..
"Connor, we want to adopt you..
We want you to be part of our family.".
Now, on Christmas, what Jesus is saying is, "I want you to be part of my family..
I want to adopt you..
I want to welcome you in because he loves us, and there's room for every.
person in God's family.".
And so that same gift is extended to us, and it's just up to us, you know,.
whether or not we receive it..
So this, indeed, is the great news of Christmas, that Jesus has come,.
that he has addressed our sin issue, that he forgives us, gives us new life,.
and that he invites us to be part of his family because he wants you to be.
part of his family picture as well..
And so as we celebrate Christmas today, as you celebrate with friends and.
family, I want to encourage you to take some time to just express thanks,.
if you know Jesus, to what Jesus has done, or to talk about Jesus if you.
don't know Jesus yet and what Jesus means..
And to, yeah, thank God for the greatest gift of all..
And so what I'd love to do is pray for you, but I also want to invite the.
worship team to come back up as well..
So let me pray for us..
Jesus, we thank you so much that you didn't leave us stuck in our sin,.
but you chose to come in a way that we can receive you as a child, as a.
little baby, vulnerable, though you are the King of kings..
And you came to ultimately to lay down your life on the cross for us and to.
rise from the dead so that we can be forgiven and become new creations in you..
And so we thank you, Jesus..
We worship you, and we are just humbled and moved and touched by you,.
that you love us so much..
And we thank you that you make us part of your family, that you give us a new.

$^{401}$family to belong to, and that we can know your love, your acceptance, and.
also the new life that we have in you with peace and joy and hope that we do.
not know without you in the same way..
So we give you all the glory, and I pray this in your name, Jesus..
Amen..
[Piano music].
So what better way would it be to wrap this Christmas up, this Christmas.
service with another carol?.
And I think as we were singing "O Come All Ye Faithful" earlier, I was just.
really moved when we were singing "O Come Let Us Adore Him," and we'll.
praise his name forever..
So let's all stand..
[Piano music].
Let's sing..
♪ O come let us adore him, O come let us adore him, O come let us adore him, Christ the Lord. ♪.
Sing it again..
♪ O come let us adore him, O come let us adore him, O come let us adore him, Christ the Lord. ♪.
Let's sing that one more time..
O come..
♪ O come let us adore him, O come let us adore him, O come let us adore him, Christ the Lord. ♪.
Sing one more praise..
♪ We'll praise his name forever, we'll praise your name forever, we'll praise your name forever, Christ the Lord. ♪.
Sing it again..
♪ We'll praise your name forever, we'll praise your name forever, we'll praise your name forever, Christ the Lord. ♪.
Yes, Jesus, we praise your name forever..
We thank you for the day..
May you be with us as we're going to enter into the cookie feast and enjoy cookies with each other..
And just wish each other a Merry Christmas..
We pray this in your name, Jesus, amen..
Merry Christmas, everyone..
Hit it..
I'm ready..
[music].
Hey, go over there..
[music].
You can go..
[music].
Go buy the cookies over there..
♪ I want to wish you a Merry Christmas, I want to wish you a Merry Christmas from the bottom of my heart. ♪.
♪ I want to wish you a Merry Christmas, I want to wish you a Merry Christmas, I want to wish you a Merry Christmas from the bottom of my heart. ♪.

$^{441}$♪ Feliz Navidad, Feliz Navidad, Feliz Navidad, prospero año y felicidad. ♪.
♪ I want to wish you a Merry Christmas, I want to wish you a Merry Christmas, I want to wish you a Merry Christmas from the bottom of my heart. ♪.
♪ Feliz Navidad, Feliz Navidad, Feliz Navidad, prospero año y felicidad. ♪.
\newpage



\section{}
\label{sec:t5Cfd_ii5kM}
\textbf{2023-12-31 Of Signs and Swords [t5Cfd-ii5kM].mp3}
\newline
\newline
連結: \href{https://youtube.com/watch?v=t5Cfd-ii5kM}{\texttt{ https://youtube.com/watch?v=t5Cfd-ii5kM}} ~~~~ 語音日期: 2023-12-31 
\newline
\newline
\hyperref[sec:9el2lQ77_AE]{\small{< < < PREV SERMON < < <}}
~
\hyperref[sec:index]{\small{[返主目錄]}}
~
\hyperref[sec:L0I0Bu9dmvE]{\small{> > > NEXT SERMON > > >}}
\newline
\newline
$^{1}$- Thank you Carla for just being amazing as always..
So good..
Well, welcome to The Vine..
My name's Andrew, one of the pastors here,.
and there's probably a lot of you.
that are visiting with us this weekend,.
and we are so, so glad that you are with us and amongst us..
And I know there's loads of people.
in the overflow right now,.
so I wanted to say hi to everybody in the overflow..
Thank you guys for being willing to sit in the overflow,.
and then everybody that's sitting on the floor.
and everything crazy..
Anyway, we're so glad you guys are here..
A number of years ago,.
I was invited to speak at a church overseas,.
and this happens a number of times for me,.
and it's always a great privilege to travel overseas,.
to speak in other churches and other contexts..
But every time I get invited to speak overseas,.
I know that I, as well as being excited,.
I know that I'm also gonna be exhausted,.
because these trips are like full on..
From the moment that the pastor from that church.
picks you up at the airport,.
to the moment that they drop you back at the airport,.
like three days later, you are on 24/7..
It is crazy..
They expect you not just to speak in their church,.
but they expect you to sort of pastor all their people,.
pray for everybody, solve all the problems of the church,.
work out why their finances are not right,.
visit with their eldership,.
and solve all the problems there,.
and strategically believe for everything..
You're supposed to have a prophetic word.
for every single person in the church..
The whole thing is crazy, and it is a great privilege..
Honestly, one of the great privileges to go there..
I love what I do, and when I get invited to go do that,.

$^{41}$I'm very excited about it,.
but I recognize that I have to manage my energy..
If I don't manage my energy,.
I know that about a day and a half into the trip,.
I will hate everybody in that church..
I will be exhausted..
Anyone here not like people when you're tired?.
You know when you get tired, right?.
You get a little prickly with people..
And I realize that I'm going overseas to this church,.
I'm representing the vine, right?.
I don't wanna be that guy like two days in.
who really doesn't like the people around them,.
and is getting a little prickly with people..
So I have to like make sure that my energy.
is well managed on these trips..
Well, on this particular trip I went to,.
the church was larger than the vine..
They had five services on a Sunday..
I was speaking at all five services,.
pretty much back to back..
After the third service, I was tired..
After the fourth service, I was a little bit wobbly..
After the fifth service, I couldn't even remember my name..
Okay, I was really exhausted..
And of course it was after every service,.
they expected me to be on the prayer line..
So I'd go down the front, I'd be on the prayer line..
And it was after the fifth service,.
I'm on the prayer line after I've just preached,.
there's a crowd of people that are waiting.
to get prayed for me, that the most awkward thing.
that perhaps has ever happened to me, happens..
There was this guy on the prayer line,.
waiting for me to pray with him..
He comes forward and as usual with these sort of things,.
you know, we introduce ourselves and stuff..
And I say to him, the thing that I say to pretty much.
everybody when I'm ever on a prayer line,.
I say to him, what is it that you would like me.

$^{81}$to pray for you for today?.
You know, what is it that I can pray for you about?.
This was his response..
He said this, "You're a man of God..
"I don't need to tell you what my needs are..
"God knows, just ask God and he will tell you..
"I wanna see what you will say.".
(congregation laughing).
Five services, okay, I'm exhausted..
And I don't like people at this point, all right?.
And here's this guy, and I tell you what,.
this happens more than you might think,.
more than you expect..
This guy comes up to me, he's basically like,.
"I wanna see whether you're really a pastor..
"I wanna see if you really hear from God, okay?.
"I know what my needs are, God knows, but do you know?.
"Ha ha ha ha, you're gonna have to pray about it..
"And when you pray about it, we'll find out.
"if you hear from the Spirit of God.".
(congregation laughing).
By the way, if you're taking notes, write this down..
Us pastors do not like people like this, okay?.
This feels like you're put under a spiritual test..
It feels like they're testing you.
to see whether you're truly a person of the Spirit..
And so this guy is speaking like this..
By the way, this is on the same part.
as the other pet peeve that us pastors have,.
and that's this, "Pastor, you prayed for me seven years ago.
"and I've never seen you ever since..
"Do you remember what you prayed for?".
(congregation laughing).
Here's the answer, "No, we don't remember..
"We're like vessels, the Spirit comes in.
"and out of us, we are not spiritual data cards.
"that are storing everybody's spiritual issues..
"For decades, that doesn't happen.".
Okay, anyway..
So this guy is like testing me..

$^{121}$I'm, my attitude, like I am using all the remaining reserves.
of energy that I have to try to be calm.
and nice and polite to this guy..
So I said, "Okay, yeah, sure, let's pray.".
So I put my hand on his shoulder, we start to pray..
And I say to God, "God, you know I'm tired..
"And you know my attitude really is not good right now..
"But I come to you and if you have something.
"you wanna say to this guy, speak to me.
"and I'll pray over him.".
Well, as God often does with me,.
he gave me a visual, a visual something..
I often see things when I'm praying for people.
and I saw something..
And I begin to pray this over this guy..
About a minute into praying it,.
I realized that what I'm praying is not very positive..
It's not very sort of happy..
It's actually kind of a hard thing.
that I was praying for him about..
And I kept praying and I kind of lent in.
to what I sensed God was saying..
And as I prayed on, I could tell that the prayer.
was just a little bit heavy..
It was a little bit, it wasn't mean,.
but it was just like the things that I felt.
like God was saying to him.
were not necessarily easy things to hear..
And at the end of the prayer, I said, "Amen.".
And I opened my eyes and I was met with this face..
(congregation laughing).
And the guy says these words to me, he goes,.
"Uh, okay, thank you.".
And then he just walks off..
And I'm left there like,.
"That is the most awkward experience I have ever had.".
Well, six months later, I get an email..
And the email is from this guy..
And he's managed to track down my email..
And in the opening line of the email,.

$^{161}$I wonder if you can imagine what it said..
It said this, "Pastor, you prayed for me six months ago..
"I wonder if you remember what you prayed for me about.".
(congregation laughing).
The irony is in this case,.
I did actually remember what I prayed for this guy about,.
'cause it wasn't very nice, right?.
And he starts to tell me in this email, he's like,.
"You know, you prayed for me and you said these things..
"And I was actually really offended in the moment..
"And I didn't like what you said..
"And I walked away and out of the church that day..
"And I was angry..
"I was angry at you..
"I was angry at God.".
And anyway, this was not a pleasant email.
up to this point, right?.
And I was like, "Okay, great.".
And then he said, "But God kept speaking to me.
"about those words that you said over the coming weeks..
"And I kept having to bring myself back to prayer.
"with those words..
"And I realized that those words.
"that God was saying to me on that day.
"were designed to uproot.
"some of the deepest stuff that was in my life.".
And he went on to tell me some of the deep sins.
and the brokenness, addictions,.
and some things that he had in his life..
And as the email went on, he said,.
"You know, I was able, it took me about six months,.
"but I've been able to reach out to my wife.".
It turned out that him and his wife.
had been separated for some time..
And he reached out to his wife and he repented with her.
and apologized for some things..
And he said, "We've gotten back together.
"and we're trying to make marriage work again..
"And I just wanted to let you know.
"that although I treated you rudely in the moment,.

$^{201}$"that those things that you said disrupted stuff.".
He goes, he actually said this, he said,.
"I didn't wanna hear what you said,.
"but it disrupted some things in me.
"that needed to get disrupted.".
And he said, "I just want you to know.
"that I'm in a better place now.".
I wonder if you've ever experienced God.
saying something to you that is not easy for you to hear,.
that is not comfortable to hear..
I wonder if any of us have ever experienced.
when God speaks to us in a way that kind of,.
just comes in and disrupts some things in us..
It's not easy, is it?.
I think as Christians, we often think.
that when God speaks to his people,.
he's always gonna speak super positive things..
That when God speaks to his people,.
it's always gonna be kind of a feel good thing..
It's always gonna make me feel good..
It's always gonna be great..
It's always gonna be wonderful..
And I think us Christians get so conditioned.
into thinking that that's always how God speaks,.
that when God shows up and says something.
that's slightly different to what we expected,.
or says something that we weren't quite.
kind of wanting to hear or whatever it might be,.
we often, as Christians, I think, reject that..
We think, "Oh, that's probably not God..
"That can't be God..
"God wouldn't say something like that.".
And we kind of push away from it..
And I think we have this problem as Christians sometimes,.
'cause we're always expecting God to show up a certain way,.
but when he shows up in a different way,.
we really struggle with that..
Am I the only person here who struggles with that?.
I am, I guess, the only person here who struggles with that..
I'm assuming that more of you also do..

$^{241}$Here's the reality that I think.
I've come to learn about God,.
that I think is actually a very important thing.
for us to grasp, and it's this,.
that sometimes when God speaks,.
the very thing you need to hear the most.
is often the very thing you want to hear the least..
Come on, church..
So often when God speaks to us,.
the very thing we need to hear the most.
is often the very thing we want to hear the least..
That actually is a really good summary.
of the first Christmas that happened some 2,000 years ago..
See, when Matthew and Luke present the birth story of Jesus,.
they present it in a raw and very real way..
And yes, it's a story of incredible moments,.
a story of choirs of angels and God's birth,.
the incarnation and the glory and the peace.
that is exclaimed during that time..
But if the story of Jesus' birth is anything,.
it's a story of a God who speaks.
and a God who longs to speak to his people..
But if you trace through the story,.
you'll see very quickly that so often the things.
that God speaks to the characters in the Christmas story.
is not the things that they were expecting to hear..
It's not the things that they even perhaps wanted to hear..
Take Zachariah, for example..
God shows up in an angel through Gabriel and says to him,.
"Hey, because of your unfaithfulness,.
"you're now not gonna be able to talk.
"until your wife Elizabeth gives birth.".
Take Joseph, for example..
Joseph, having discovered that his unmarried,.
pregnant fiancee is now pregnant with a child.
that came from the Holy Spirit,.
and he understands the shame that will come upon him.
and upon Mary, and so he's making plans in his mind.
to do the right thing for her and for him,.
and which is to separate their relationship..

$^{281}$And God steps in and says,.
"No, you can't separate that relationship..
"You're actually gonna have to walk into that shame,.
"walk into that pain, 'cause I've got a bigger thing.
"that's happening here.".
Or think about the shepherds..
The shepherds who were faithfully looking after their sheep,.
which was their livelihood..
God shows up in the form of angels and says,.
"Leave your livelihood behind and go into Bethlehem..
"Just drop everything that you've worked hard for,.
"because we've got a new thing.
"that we want to happen in your life.".
Isn't it so interesting for us that we have a plan A.
that we think is our plan A,.
but so often God comes and says,.
"There's actually a B, C, and a D.
"that is really the thing that's gonna set you alight,.
"really the thing that's gonna make you as a person,.
"really the thing that's gonna flourish you.
"and bring you into the fullness of who you are.".
I mean, just think about Mary..
God shows up to Mary and says some pretty crazy things,.
things that she must've been so confused by,.
so overwhelmed by,.
not things that she was expecting God to say to her..
In fact, I think if Mary had the choice,.
she would have chosen not to be in a part.
of the whole Christmas story in the first place..
Imagine the shame that she had to hold..
Imagine the months that she had to go through,.
where people thought all this stuff about her..
If she had the choice, she probably would have said,.
"I didn't even wanna go anywhere near that story.
"or that thing whatsoever.".
And I think this is fascinating,.
that in the heart of the Christmas story,.
you see something that is so important.
for all of us to understand about how God works in our lives..
And it's this, that your personal preferences.

$^{321}$and your personal choice is not always.
the most important things in God's mind.
when it comes to his leadership of your life..
I know this is not the Christmas message.
you were hoping to hear..
(congregation laughing).
But I wanna encourage you to stay in with me.
because I think there's something important.
that you need to hear..
Our personal choices and our personal preferences.
are not always God's most forefront things in his mind..
When he thinks about you and his leadership of his life,.
and he sees you, he sees your future,.
he sees where he's leading you in his life..
And he says, "I know what's ahead..
"And I know all the good things..
"I know the bad things..
"I know the ugly things..
"And you want plan A,.
"but I've got this other plan over here.".
Your personal preference is to go down this path..
But if you go down that path,.
I tell you in five years time,.
you're gonna hit rock bottom..
But if you were to listen to me,.
if you were just to come with me,.
I've got a new path that I'm gonna take you down..
And this is the way that will bring flourishing.
and growth and life to you..
Your personal choices and preferences,.
they're important to God,.
but they're not the most important thing to him..
He sees stuff that you will never see..
And in faithfulness and trust,.
we are to turn to his word..
Are you following this?.
But sometimes the word that God brings is challenging..
It's hard to hear..
And it comes and disrupts some things in us..
Case in point is exactly what happens.

$^{361}$at the end of the Christmas story..
The story that Luke is presenting to us.
of the birth of Jesus,.
he adds a very important story right at the end..
It's almost like an epilogue to the story, if you will..
It's the moment where Mary and Joseph,.
having had the child Jesus,.
Jesus is now eight days old..
And from Bethlehem to Jerusalem,.
they make the journey from Bethlehem to Jerusalem,.
and they go into the temple when Jesus is eight days old..
And they're carrying him in.
because the ritual at the time,.
the tradition at the time,.
was that faithful Jewish families.
would take their firstborn son.
and bring him into the temple to be dedicated to the Lord..
And that dedication ceremony.
involved the act of circumcision..
And so this was what Mary and Joseph were doing..
They're faithful Jewish people going to the temple.
on the eighth day with Jesus,.
taking him into the temple.
in order for him to be dedicated.
and for the circumcision ceremony to take place..
And when they're in the house of God,.
something awkward happens..
A little bit like me standing in that prayer line.
and that man coming forward..
For Mary and Joseph,.
something significantly awkward takes place..
Let me actually break this down for you.
from Luke chapter two, starting in verse 25..
Everybody still okay?.
Do you still like me?.
I'm really, if you're visiting,.
I'm really a nice person..
Trust me, I am, I am..
But hang with us..
Verse 25 of chapter two of the Christmas story..

$^{401}$Now there was a man in Jerusalem called Simeon.
who was righteous and devout..
He was waiting for the consolation of Israel.
and the Holy Spirit was upon him..
It had been revealed to him by the Holy Spirit.
that he would not die before he had seen the Lord's Christ..
That's the Messiah..
Moved by the Spirit, he went into the temple courts.
when the parents, that's Mary and Joseph,.
brought in the child Jesus to do for him.
what was custom of the Lord, that the law required,.
that's the circumcision..
Simeon took the baby into his arms.
and praised God saying this,.
"Sovereign Lord, as you have promised,.
"now dismiss your servant in peace..
"For my eyes have seen your salvation,.
"which you have prepared in the sight of all people,.
"a light for revelation to the Gentiles.
"and for glory to your people Israel.".
Up until this point in the story,.
everything is going perfectly right..
This is wonderful..
And Luke is creating and writing this story.
to help you to see how amazing this is..
There's the temple in all of its glory and beauty.
in the presence of God..
For the very first time, the Son of God.
is being brought into the temple itself..
There's Mary and Joseph, the faithful family.
bringing their child for that dedication service.
and then we're introduced to the sort of creme de la creme,.
Simeon..
And Simeon is described by Luke in the best ways.
that you could ever hope to be described in the Bible..
Listen to this, righteous and devout,.
waiting for the consolation of Israel,.
the Holy Spirit upon him..
Luke is saying, this man is righteous..
He hasn't put a foot wrong..

$^{441}$This man loves the Lord..
He's devout in his devotion to God..
He is waiting for the Messiah to come..
He's so excited that one day God would step into the world.
with the Messiah..
Not only that, but he's filled with the Holy Spirit..
I mean, you can't paint somebody in a better light than that..
Righteous, devout, waiting for God,.
filled with the Holy Spirit..
And the Holy Spirit takes Simeon.
and he walks into the temple courts.
and he sees the baby Jesus.
and he sweeps up Jesus in his arms..
And notice what he says..
He says, "This is a light for revelation to the Gentiles.
"and glory to your people, Israel.".
You could not say a better thing about baby Jesus,.
that he's gonna be the light of revelation to the Gentiles..
I love that the Christmas story brings us the picture.
of Jesus's mission to the world..
That anyone who was not Jewish,.
that's the majority of us in this room,.
we can come to the saving knowledge of God.
through Jesus Christ..
He's a light of revelation to all of us here in Hong Kong..
But he's also the glory to Israel..
He's also gonna do all the things.
that the scriptures had prophesied.
about what was gonna happen to Israel..
He was gonna be their Messiah and their Christ..
And Simeon sweeps up baby Jesus in his arms.
and says all these incredible stuff..
And you can imagine Joseph and Mary standing there.
and thinking, "This is the most amazing thing.
"I have ever heard..
"This is the best thing I could ever hear about my child.".
I mean, think about this from your perspective..
Imagine if you had a baby.
and you come into the vine on a Sunday,.
it would be like one of the pastors from the vine.

$^{481}$sweeping up your child in their arms.
and going, "This one, this one is destined to go to Harvard.
"and graduate from the top of his class..
"This one will be a millionaire by the age of 30,.
"for he will become a doctor or a lawyer or a banker..
"This one will master the piano and the violin.
"and will speak 13 languages..
"And even in your old age, you will be blessed.
"'cause this one will look after you.".
Every Hong Kong family would say, "Praise the Lord..
"Are you with me?".
That's what it was like for Mary and Joseph, okay?.
This was their Harvard monument, all right?.
The child will be a light of revelation to Gentiles..
He's going to be the glory to Israel..
It could not get better than this,.
which is why it says in the next verse,.
it says, "The child's father and mother marveled.
"at what was said about him.".
The word marveled there means they rejoiced..
They were in wonder, they were in awe..
This was the most amazing thing they have ever heard..
And right now, Luke is basically saying,.
"It can't get better..
"Isn't God great?.
"Isn't he saying nice things?".
And then Simeon opens his mouth again..
And then he puts Jesus to one side and he turns to Mary..
And I want you to hear what he says to Mary..
"Then Simeon blessed them and said to Mary, his mother,.
"'This child is destined to cause the falling.
"'and the rising of many in Israel.
"'and to be a sign that will be spoken against,.
"'so that the thoughts of many hearts will be revealed.
"'and a sword will pierce your own soul too.'".
This was not in the Christmas script.
that was supposed to be the Christmas script..
This wasn't quite as nice.
as what had just been spoken about Jesus..
Simeon turns to Mary and he says some prophetic words.

$^{521}$over her that she did not want to hear,.
but she needed to hear..
He says to her, first of all,.
that Jesus is gonna be an absolute nightmare for you..
He said, "He's gonna cause the rising and the falling.
"of many in Israel and will be a sign.
"that will be spoken against.".
In other words, Jesus is gonna grow up.
and he's gonna be a controversial figure,.
he's saying to Mary..
He's gonna cause the falling and the rising.
that the contents of people's hearts.
are gonna be revealed.
and no one likes the contents of their heart to be revealed..
And some people will rise with Jesus..
So some people will find their greatest life.
in relationship with him,.
but there are others who will reject him,.
who will not believe him,.
who will turn their backs on him,.
others who will actually speak against him..
And this Jesus, he's gonna say some things..
I mean, he's gonna come and he's gonna say some things.
that you, Mary, as his mother,.
you're gonna be like, "Oh my goodness, I can't believe.
"he just said that.".
This is not gonna be an easy future for you, Mary..
And God so loves Mary in this moment..
So, I want you to see this,.
so loves Mary, but sees her future.
and knows it's gonna be hard,.
knows that raising the son of God is not gonna be easy..
And so out of his love for her,.
he wants to prepare her and mold her and shape her.
into the woman that she needs to become.
so that she can be the mother to the son of God..
And so he brings a word that she may not want to have heard,.
but she needed to hear in order to become the woman.
God wanted her to become..
I mean, Jesus, can you imagine what Jesus is about to do?.

$^{561}$He's about to say things like,.
"Unless you eat my flesh and drink my blood,.
"you have no part of my kingdom.".
That one got most of his followers to leave him..
He would say to the religious authorities of the day,.
"You brood of vipers, you whitewashed tombs.".
He would say to his disciples, the followers,.
he would say, "Unless you pick up your cross,.
"unless you die, basically, you can't follow me.".
He would actually say to his best friend,.
he would call his best friend Satan..
This is not how to win friends and influence people..
And then Simeon says this..
He says, "A sword will pierce your own soul too.".
He's not speaking, of course, of a literal sword..
He's not speaking of her death in the future..
He's speaking about her emotions..
He's speaking about how hard it's going to be for her..
And that if she's not prepared for the reality.
of the chance and the choices and the things,.
there's one point Jesus is going to,.
in front of his mother, say,.
"My mother and father is not biological..
"My mother and father is anyone.
"who follows after my father.".
Mary is gonna have a hard path,.
and God doesn't hold back from helping her to see.
that there's some tough times ahead..
There's a sword that's gonna feel like it pierces your soul..
It's gonna be difficult..
And God is saying this because he's like,.
"I love you so much that I'm gonna say the very things.
"to you that you need to hear,.
"even as you're holding an eight-day-old child.
"in your hands, the future ahead for him..
"He is gonna be the savior of the world..
"He is the light of revelation to the Gentiles..
"He is the glory to Israel..
"You are gonna have the joy of holding him,.
"but you're also gonna feel the pain.

$^{601}$"that no parent wants to feel,.
"and that is you're gonna watch your child die.".
And God has so much love for Mary.
that he's willing to speak a hard word..
God has so much love for you,.
and God sees everything that is ahead of you..
God loves you so much that he's also willing, at times,.
to come and speak over you.
and to help, perhaps, you to see that there are some things.
that might need to change in your life,.
or perhaps tell you that there are some things coming up.
that you need to get ready for and prepare for,.
'cause if you're not, it's gonna be overwhelming,.
and it will feel like a sword has pierced your soul, too..
I think sometimes, as Christians,.
we think that when we come to Christ Jesus,.
then everything's gonna be absolutely perfect..
The Christmas story shows us a different thing..
And of course, Jesus is our life..
And the wonderful thing that we celebrate at Christmastime.
is the arrival of our Savior and our Messiah,.
that when we're in relationship with him,.
our sin is taken care of, forgiven..
We're released and filled with the fullness of God,.
that the same Spirit that raised Jesus is found in us,.
and that we are found into flourishing, fullness of life,.
that we can know the fullness of peace, shalom,.
and that we'll have eternal life with him..
There is so much good news in the gospel, so much there,.
and God so wants all of that for you.
that he's sometimes willing to speak something.
that is not gonna be easy for you to hear.
to ensure that you remain on the path.
that he has for you to be on..
See, Jesus is not afraid to speak a disruptive word.
to your perfect world if your perfect world.
is perfectly misaligned to his perfect will..
He's not afraid to do that..
And it seems weird, doesn't it, on Christmas Eve.
that this would be a part of the Christmas story,.

$^{641}$but I think it's a part of the Christmas story.
because God is revealing his heart to his people..
I'm so passionately in love with you,.
and I so passionately want the best for you,.
that if there is anything in your life that is misaligned,.
I have the grace and the courage.
to lead you on a different path..
I don't know about you, but I've had the disruptive word.
of God into my life many times over my life..
I remember when I first quit my job,.
I was in banking here..
I was one of those kids that was the banker..
I wasn't a millionaire by 30, but I was a banker..
And I felt God calling me out of that marketplace world.
into being a pastor, and so I remember quitting my job.
and moving with my family to New Zealand.
to go to Bible school, and I remember the first month.
in Bible school, I felt so excited..
I felt so excited about what my future was..
I could not wait to become a pastor..
Could not wait to become a preacher..
I knew that God had called me to preach,.
and I was really excited to have a life.
where I could communicate God's word to people.
and speak life over people..
I mean, it was everything I wanted to do,.
and I remember in the first month of Bible school,.
I was setting myself ready for all the studies..
I wanted to learn Greek..
I wanted to learn Hebrew..
I wanted to understand the Bible as best I could.
because I wanted to be the best preacher.
of God's word that I could,.
and in the first month, I said to God,.
I said, "God, would you speak to me a word.
"about my preaching gift?".
And I was expecting that God would say,.
"You're awesome, Andrew..
"You're the best ever..
"You're the best..

$^{681}$"I can't wait to see your future..
"You're gonna be the city pastor.
"of a church called The Vine in Hong Kong,.
"and it's gonna be amazing,.
"and there's gonna be so many people on Christmas Eve.
"that not only are you gonna be able to fit.
"into the building, and it's gonna be great.".
(congregation laughing).
I was expecting a word along those kind of lines..
But God said to me,.
"You really want me to speak to you.
"about your preaching gift?".
I was like, "Yeah.".
He's like, "Go to Amos chapter eight.".
So I went to Amos chapter eight, and here's what I read..
"The days are coming, declares the sovereign Lord,.
"when I will send a famine through the land,.
"not a famine of food or of thirst for water,.
"but a famine of hearing the words of the Lord..
"People will stagger from sea to sea.
"and wander from north to east,.
"searching for the word of the Lord,.
"but they will not find it.".
I had asked God to speak to me about my preaching gift,.
and he brings me to this passage where he's like,.
"Andrew, you need to understand that in your life,.
"if you don't make some changes, a famine is coming on you,.
"and that famine will be on hearing the word of God..
"And you need to understand that what makes a preacher.
"is not how they speak the words of God,.
"but how they hear the words of God..
"I am nothing if I can't hear the words of God..
"I am the worst pastor ever..
"I have nothing to say on this stage,.
"week in and week out, if I can't hear the words of God..
"If I can't receive and hear something.
"and do my best to package it and bring it to you,.
"if I can't do that, then this is absolutely useless.".
And I sat there under this idea that God's saying,.
"You really want me to speak about your preaching gift?.

$^{721}$"Here's what you need to hear..
"It's not what you want to hear, Andrew,.
"but here's what you need to hear..
"A famine is coming on your life.
"where you will be completely useless,.
"no matter how much Bible knowledge you have,.
"no matter how much Greek and Hebrew you might know,.
"completely useless unless you change.
"some things in your life..
"There's some stuff in your heart, Andrew,.
"that you need to deal with..
"And if those things don't get dealt with,.
"I will block your ears and you won't hear my word.".
And this led me on a difficult six-month journey.
of dealing with some really important things.
in my soul and my heart that needed to be dealt with,.
things like pride,.
things like arrogance in my cool in life,.
things like thinking I'm better.
because I'm in Bible school than in the marketplace,.
things that were ugly in me,.
other things that I don't want to tell you about..
And I went through a six-month journey.
of having to bring those things to the Lord.
and have Him deal with me in.
so that I could even think about being.
in the place that I'm in right now.
and doing what I'm doing with you right now..
And every single time that I prepare a sermon.
here at The Vine, God breathes over me and says,.
"Don't forget Amos 8.".
This has been a disruptive word for me.
that has messed me up inside for years and years and years..
And by God's grace has enabled me.
to hopefully stand in front of you.
and say some things at times that are helpful to you..
Mary, Mary had this future ahead of her that God could see..
And so he came in the gentleness of this amazing man, Simeon,.
and said some things to her that were not easy to hear,.
some things that messed some things up inside of her..

$^{761}$God could see what my future was.
as a young 30-year-old wondering what was going on in life..
He could see what was gonna happen for me..
And he said, "You will be a disaster.
"for those people in Hong Kong.
"unless you first deal with this.".
He so loved me and you that he was willing to disrupt.
and make a mess of some things in my life..
And what you need to understand.
is that God doesn't just come into mess,.
sometimes he creates mess..
And he creates a mess sometimes by bringing a word.
that is designed to disrupt some faulty old ways.
of thinking in you so that you would be brought.
into new and transformative ways of being..
That's actually the Christmas story,.
that Jesus was willing to disrupt this world,.
willing in the incarnation to come and stir some things up.
so that the world would get a vision of a life.
that is so much greater and so much better.
than the one that was currently being experienced..
The point of Christmas, the point that Jesus has come.
is so that the world would know.
that there is an alternative way and is the Savior,.
one that can set us free, one that forgives sins,.
one that changes us..
And we need Jesus to come in to our perfect little worlds.
and mess some things up sometimes.
so that we would come to a place.
where we would truly worship him..
My heart for you is that you would think about it this way,.
and this is perhaps the more positive way to think about it..
If you really want to grow as a person in 2024,.
the greatest Christmas gift you can give yourself.
is an open and receptive heart.
to the correction and discipline of God in your life..
And I know that's not the Christmas Eve message.
you were hoping to hear,.
but I do believe it's the Christmas Eve message.
that God wants you to hear..

$^{801}$If you took that seriously in 2024,.
you will have your greatest maturing growth.
that you have ever experienced..
And it's not that God is gonna say those kinds of things.
all the time to us..
Of course, God so often and normally and usually speaks.
words of life and hope..
But let's not close our hearts off.
to the truly transformative moments.
where he puts his little finger on something.
and says, "Let's go on a journey here for a little while..
"And let's actually allow this thing to get healed.
"and renewed and redeemed for you.".
My prayer for us as a church as we head into the new year.
is that we would have a heart like Samuel.
in the Old Testament who walked into the presence of God.
and said this, "Speak, Lord, for your servant is listening.".
Tomorrow, we're going to give you a really positive,.
happy message on Christmas Day..
Today, my prayer is that you listened..
Should we pray?.
Let's pray..
Father, I am so grateful for all these amazing people.
in this room and the overflow around the building right now..
Father, I'm thankful that they are so hungry.
to come to worship you on Christmas Eve,.
that they're willing to stand and sit in the aisles,.
sit on the stairs..
Lord, we honor everybody in this room..
We're so grateful for the community.
that you've built here at the Vine..
And Father, we thank you for Christmas.
and the real Christmas story,.
the original one 2,000 years ago..
We thank you that Luke included a story like this,.
a story that perhaps is different.
from some of the other things we see in the Christmas story,.
a story where Mary sits under.
a challenging prophetic word from God..
And Father, you speak so often words of life,.

$^{841}$the hope and truth..
And sometimes you speak those words in ways.
that is designed to mess up our perfect worlds,.
particularly when we're misaligned to your will..
(coughing).
Father, I pray..
I pray by your Holy Spirit.
that people here would know the love of God.
seen in the beauty and the wonder of a child.
born in Bethlehem,.
that people in this room would know the call that's upon them.
that through relationship with Jesus,.
they can experience the greatest flourishing.
that they will ever experience..
And that everybody in this room would also know.
the openness of their hearts to whatever it is.
that God wants to say to them..
Father, you know our best far better than we do..
Father, we lay down our perceived perceptions and hopes.
and positive things that we want..
Father, we ultimately want you..
We want your leadership..
We want your love..
We wanna become the people that you need us to be.
so that the gospel could be heard in Hong Kong.
like it's never been heard before..
I pray this Christmas Eve, Lord,.
would be a turning point for some of us here..
And as we go into the new year,.
we would go in with open and receptive hearts..
And that maybe there's some words.
that you've been saying to us this year.
that we've gently put to one side..
Perhaps we've been like, I don't wanna deal with that..
And maybe as we've unpacked this story today,.
the Holy Spirit has reminded you right now.
of some of those words that God has said to you in the past.
that you've either ignored or put to one side..
And perhaps your Christmas gift this year.
is God just gently saying,.

$^{881}$I wanna bring you back to that word..
I wanna bring you back to that thing..
Let's talk about it..
I wanna heal you and restore you and redeem you..
The years of the locust have eaten from you,.
as Joel chapter two would say..
I wanna restore to you, the Lord says..
Let me speak..
Father, I pray that you would speak,.
that Lord, your church would hear,.
and that Lord, together we would glorify you,.
just like those shepherds who left behind the sheep.
and saw the child Jesus in the manger.
and went back glorifying and sharing the news of you..
Lord, I pray that that would be your church.
at Christmas time..
And Father, we thank you for this, in Jesus' name..
Everyone says..
- Amen. - Amen, amen..
God bless..
\newpage



\section{}
\label{sec:eCZ2lV0ycAg}
\textbf{2024-01-01 Press On [eCZ2lV0ycAg].mp3}
\newline
\newline
連結: \href{https://youtube.com/watch?v=eCZ2lV0ycAg}{\texttt{ https://youtube.com/watch?v=eCZ2lV0ycAg}} ~~~~ 語音日期: 2024-01-01 
\newline
\newline
\hyperref[sec:FDWw19jgJvM]{\small{< < < PREV SERMON < < <}}
~
\hyperref[sec:index]{\small{[返主目錄]}}
~
\hyperref[sec:xU27bdsfcJo]{\small{> > > NEXT SERMON > > >}}
\newline
\newline
$^{1}$Good morning, everybody..
Good morning..
Happy New Year..
Here we are at the start of a new year..
Thank you, Carla..
Beautiful..
All right..
It's good to be back..
My family and I, we've been away on sabbatical for a little while, and so it wasn't very.
restful when you're traveling with two little kids, but we had a lot of fun family time.
together..
But I'm so glad to be back with everybody and getting into the swing of things, Christmas,.
and of course, looking towards 2024..
We're going to be in Philippians chapter three today, so if you want to follow on the screen,.
it's going to come up on there, or if you have a Bible or a device that you can read.
from, I'm going to invite you to listen as I read Philippians chapter three, starting.
in verse seven..
Word of the Lord says this, "But whatever were gains to me, I now consider loss for.
the sake of Christ..
What is more, I consider everything a loss because of the surpassing worth of knowing.
Christ Jesus my Lord, for whose sake I have lost all things..
I consider them garbage that I may gain Christ and be found in him, not having a righteousness.
of my own that comes from the law, but that which is through faith in Christ, the righteousness.
that comes from God on the basis of faith..
I want to know Christ, yes, to know the power of his resurrection and participation in his.
sufferings, becoming like him in his death, and so somehow attaining to the resurrection.
from the dead..
Not that I have already attained all of this, or already have arrived at my goal, but I.
press on to take hold of that which Christ Jesus took hold of me..
Brothers and sisters, I do not consider myself yet to have taken a hold of it, but one thing.
I do, forgetting what is behind, straining toward what is ahead, I press on toward the.
goal to win the prize for which God has called me heavenward in Christ Jesus.".
So here we are, church, on the cusp of a new year..
Goodbye 2023, hello 2024..
For some of us, New Year's Eve is very exciting..
Let me tell you how it's going to go down at my house tonight..
After services today, I'm going to call Dobro's, I'm going to walk over to that little shop.
in Wan Chai, pick up my pizzas, we're going to go home and have pizza for dinner because.
it's super easy, put the kids to bed, I'm going to open a bottle of wine and then doom.
scroll Instagram for a little bit until I get a bit tired, then I'm going to put on.

$^{41}$a movie, I'll fall asleep on the sofa inevitably, wake up at 1am or the baby will wake me up,.
take a look at my watch, oh it's 2024 and go back to sleep..
I know how to party, that's how it's going to go down at my house..
But for you, for you out there, maybe you have champagne in the fridge right now, I.
know Hudson does, definitely..
Fancy dress party to attend later..
Perhaps some of you in this room or watching online, you're planning to ask the big question.
tonight, okay?.
Some of you today are going to be ringing in 2024, okay?.
And it's all going to be good, it's going to be a celebration..
Maybe that's you tonight, I don't know..
For some of us, 2023 has been a great year..
You've hit your goals, you've landed that dream job, you've met the person that you've.
been wanting to meet..
And as you look ahead, it's looking good..
There is so much to look forward to..
So tonight, as Prince tells us to do, we're going to party like it's 1999..
However, dates like New Year's Eve are difficult for some of us..
Because instead of looking forward to celebration and the excitement that's to come, tonight.
you're going to be reminded of how tough the past year has been..
And as you look ahead, the forecast is just more disappointment..
And 2023 hasn't been kind, and it's hard to see how things are going to change..
And so Christmas Eve, New Year's Eve, dates like this, it's just a painful reminder of.
pain and sadness..
So as we look towards 2024, I'm not here to tell you today it's just going to be a best.
year ever, okay?.
I'm not going to say that because I have no idea, we have no idea how this year is going.
to turn out..
Like I said, for some it's going to be awesome..
All your hopes and dreams and prayers are going to come through..
But for some it's going to be a storm..
And I don't know if things are going to get better..
But wherever we are in this journey, my hope for today, as we end this year, is going to.
be an encouragement and a challenge for everybody..
And yes, I understand we don't have to wait till New Year's Eve to do something like this,.
but today is Sunday..
We are gathered..
It does happen to be New Year's Eve..
So we have an opportunity to hear from the Lord about the things that have passed and.
the things that are yet to come..

$^{81}$And what I do know is this, what I do know is no matter what life looks like for you.
right now, in this moment, in this season, there is an attitude, there's a frame of mind.
and a faith that Jesus wants us to step into as we look towards 2024..
A foundation for us to build the next 365 days on..
Some framework for you to think about as we step into the fast..
And so whether or not 2024 brings you to the top of the mountains where you're celebrating.
or leaves you struggling in the valleys, we can be sure about our goals and our purposes.
for this year..
So let's go back to the passage we're looking at today..
Let's see what it has to say..
Verse 7, "Wherever were gains to me, I now consider loss for the sake of Christ..
What is more, I consider everything a loss because of the surpassing worth of knowing.
Christ Jesus my Lord, for whose sake I have lost all things..
I consider them garbage that I may gain Christ and be found in Him, not having a righteousness.
of my own that comes from the law, but that which is through faith in Christ, the righteousness.
that comes from God on the basis of faith.".
So let me ask you this..
If someone were to ask you, "How do you measure how successful your year has been?".
What would you say?.
I'm preaching for this iPad and it messes up on me..
Hang on a second..
Normally I use paper..
New year, new me..
It's not working out so far..
Hang on..
Okay, here we go..
Okay..
What would you say?.
How would you measure how successful your year has been?.
What would be your answer?.
Well, if you're a fan of musical theater, you might say this..
In daylights, in sunsets, in midnights, in cups of coffee, in inches, in miles, in laughter,.
and in strife, 525,000..
Many times when we look at life, we measure it by achievements, things we have done..
So as you look back towards 2023 and how successful it was, you might start by asking yourself.
the question, "Did I achieve all the things I set out to achieve?.
Did I hit all my goals?".
Right?.
The internet, social media is going to be full of memes about this..
You know the ones that say, right, "How it started, how it's going," right?.

$^{121}$And so as the year comes to an end, okay, not great for these guys, okay?.
You promise..
As the year comes to an end, you might have been on this process yourself, seeing how.
well you've hit your 2023 goals..
Now goals can be a good thing..
In fact, here at the Vine, as a church, as a ministry, we have a process for setting.
goals for ourselves personally and professionally, right?.
We do this as a way to keep focus on the things that are important..
It gives us a way to measure whether or not we're heading in the right direction, keeps.
us accountable to the things that we said we were going to do at the beginning of the.
year..
And when we hit those goals, it's awesome, right?.
Setting that new personal best for a fitness goal and finally reaching that, that's great..
Buying the dream car, getting that promotion, paying off that debt, whatever it is, when.
we reach our goals, there's an undeniable sense of achievement..
It feels good..
And so now as you look towards 2024, you might have already made a list of goals, resolutions.
to start the new year with..
We make these declarations of the things we want to achieve at the beginning of a new.
year..
And like I said, it can be a good thing..
In fact, for some of us in this room and some of us watching online, you'd be doing yourself.
a great favor by setting some real goals this year so that the next 365 days don't just.
go aimlessly sailing by..
At the same time though, if we make reaching these earthly goals the only thing, if hitting.
goals is a standard that we focus on, I don't think that's very healthy either..
And I don't think that's the way Jesus calls us to live our lives..
And by the way, statistics show us that only 9\% of people who make resolutions manage to.
see them through till the end of the year, which means that for most of us this year,.
we've already failed before it started..
Which is why when Paul says, "But whatever were gains to me, I now consider loss for.
the sake of Christ," it's absolutely ridiculous..
Because this was a man who had hit all his goals..
This was a man who achieved everything you could imagine to achieve and then some more..
Listen to how he describes himself in Philippians chapter three a bit earlier, just the verses.
before this, 4 to 6..
"If someone else thinks they have reason to put confidence in the flesh, I have more..
Circumcised on the eighth day of the people of Israel, of the tribe of Benjamin, a Hebrew.
of Hebrews in regard to the law, a Pharisee..
As for zeal, persecuting the church..

$^{161}$As for righteousness based upon the law, faultless.".
In 2023 translation, this is what he's saying..
"If you think you've done a lot with your life, trust me, it's nothing compared to what.
I've done..
I'm the best of the best..
I'm the creme de la creme..
If you look at my resume, it's perfect..
I'm the guy you aspire to be..
I've done and achieved things you can only dream of doing yourself.".
But there's something important to note here..
As much as he's achieved, what Paul is saying here, it's not the things that he achieved.
are of no value at all..
He recognizes that these goals, these achievements he has are of great value..
They are important..
These things were precious to him, but yet he's still willing to throw it all away..
In other words, what Paul is saying, Paul is not throwing away rubbish in order to gain.
Christ..
He is throwing the things away that were extremely important to him in order to gain Christ..
This is why his testimony is so remarkable..
What Paul is saying that the best achievements he's gained in his life have become meaningless.
to him, that all the goals that he's reached are rubbish..
Why?.
The answer is powerful but simple..
He has a new goal, and that's Jesus..
So in a world and a culture that puts so much focus on achievement, it seems like such a.
ridiculous thing to do, especially if you've achieved so much..
But this is exactly the challenge that lies before us today..
So let me ask us this, church, before or ahead of all the goals you want to see happen in.
2024, are we willing to put Jesus first?.
Paul's not talking about giving everything for the sake of nothing..
There's a reason and a motivation..
He hasn't gone crazy, although some might say he has lost his mind, but there's a motivation.
to do what he is doing..
He is willing to give up everything because he has found something that's worth more than.
all these things put together..
And Paul is giving up everything because of the surpassing worth of knowing Christ Jesus,.
my Lord..
Paul is giving up everything so he can walk in true righteousness..
You see, he recognizes he had a righteousness once..
He was perfect, he says, in following the law..

$^{201}$But his righteousness wasn't one from God..
It was one that was done by his own power..
But now he recognizes there's a better righteousness, one that he could never achieve on his own,.
one that can only be given through life in Jesus..
Paul is giving up everything so that he could have a life that is not defined by personal.
achievements, but is a life that is lived, defined through faith in Christ..
He's choosing to give up even the best things so he could receive something better..
At the risk of repeating myself, we have to understand Paul gave up everything, everything,.
his comfort, his reputation, the trajectory that was his, his money, his power, his identity,.
so that he could have a deeper and more intimate relationship with Jesus..
And where do you think he got this example from?.
Because by the way, isn't this what Jesus did for us?.
Isn't this what we just celebrated at Christmas?.
Jesus who gave up his position in heaven, who made himself nothing, taking on the nature.
of a servant, humbled himself, becoming obedient to death, even death on a cross..
Why?.
So then he could pave the way for us to have an intimate relationship with him..
So let me ask this again..
I'm going to phrase it more directly towards ourselves this morning..
Are you willing to let go of all things, even your best things, even your most precious.
things in order to gain Jesus?.
Could this be the one goal you give yourself above all other goals?.
If the answer is yes, and I hope it is, we're going to talk about how to do this a little.
bit..
Because after, like I said, after giving up our most precious things, we gain something..
This passage gives us new goals to aim towards..
Paul says this, "I want to know Christ, to know the power of his resurrection and participations.
in his sufferings, becoming like him in his death, and so somehow attaining to the resurrection.
of the dead.".
So like I said, if you want some new goals for 2024, if you're looking for some framework.
to walk towards in your fast, this might be helpful for you right now..
First thing is this, to know Christ..
To know Christ..
When Paul speaks of knowing Christ, this of course is more than just a general knowledge..
It's more than just an intellectual awareness..
But to know Christ is to be in relationship with him..
Many of us in this room, we might know a lot about who Christ is..
In fact, we might know a lot about a lot of people..
I follow Dwayne "The Rock" Johnson on Instagram, right?.
And I know sometimes he posts what he's doing..

$^{241}$I wouldn't say I know him..
I mean, I know we have a similar physique, but apart from that, I don't know him that.
well, right?.
So I wouldn't say I was friends with him..
So I'll say this again..
To know Christ means we have to be in relationship with him..
Not just being a part of a church, not just coming to Sunday services, attending Bible.
studies..
The call here is not to know about Jesus, but to know him in a way that actually shapes.
and directs your life..
My family in Christ..
Do you truly know Jesus?.
Now, in order to truly know Jesus, you must also participate in this next step, to know.
this thing, to know the power of his resurrection..
The most powerful display of God's love was shown to us when Jesus sacrificed himself.
on the cross..
But the most powerful display of God's power was shown to us when Jesus was resurrected.
from the dead..
And so if we truly want to walk in the life-changing relationship with Jesus in a way that actually.
shapes and directs our life, if we really want that, we have to understand that we do.
that by the power, the same power that raised Jesus from the dead lives in you..
That's what empowers you to live a life-changing relationship with Jesus..
But again, I think this is something we sometimes underestimate or forget about..
We do..
We underestimate what we can do in the name of Jesus..
So we forget about this resurrection power..
It makes us timid in our faith..
We get scared..
We start to think that there's certain things that God can't do..
But let's not forget, church, with God, what things are possible?.
All things are possible..
The biggest mountains of struggle in our lives that seem so insurmountable, God can move.
that mountain..
All things..
But here's the great mystery..
At the same time, the power of resurrection in our lives also works in more subtle ways..
Resurrection power might work in your way, in your life, like giving you encouragement.
in time of need..
It might give you the courage to have that conversation with the person that you know.
is struggling in their faith..

$^{281}$It might give you the wisdom and the patience to sit next to someone who's going through.
pain..
It might give you a generous spirit to give away things, to help someone without anyone.
ever knowing a thing about it..
It might not be as loud, but it's definitely as powerful..
So my question for us in this, do you want to know the power of resurrection in your.
life?.
Do you want to know Jesus, and do you want to know the power of his resurrection in your.
life?.
All this, of course, is gonna ask something for us..
And it's going to require you to, and Paul says, participate in suffering..
Now why would we want to participate in suffering?.
If we truly want to know Jesus and his power, then it's not just all about warm, fuzzy feelings,.
walking in his blessings, being encouraged, gathering community, as beautiful and as necessary.
as these things are..
There is also, church, a call to suffer..
To know Jesus fully means we must be able to relate what it means to suffer..
Jesus himself says this, "Don't be surprised if people hate you..
Take up your cross, then follow me.".
To the rich young ruler, he says, "If you want to follow me for real, sell everything,.
then you can come and follow me.".
Suffering is one of those Christian characteristics, one of those things of Christian life that.
we would like to conveniently ignore..
But scripture tells us clearly we must be willing to embrace it..
In other letters of Paul, in other books and parts of the Bible, it tells us that we have.
to go through things like endurance, refining, and perseverance, things that are not pleasant.
at the time when we go through them..
They are painful, but is necessary for our righteousness..
Now don't hear me wrong..
I know, okay?.
We need to look after ourselves..
Practicing self-care is an essential skill for our lives..
I'm not calling us to look, search for suffering and just go through suffering for the sake.
of it..
However, I do think there comes a point sometimes when self-care turns into selfishness or even.
self-indulgence, which means that every time something hard comes up, everything, you know,.
a tough road is up ahead, our natural inclination might be to avoid it..
And we start to think, "Well, surely God's not calling me to step into this direction..
That's gonna be hard..
That's gonna cost me something..

$^{321}$It's gonna be difficult.".
But church, let's not forget, we follow a Savior who was always willing to step right.
into the middle of suffering, who wasn't afraid of the storms..
We follow a Savior who never shrunk back from doing the difficult thing..
Therefore, as His followers, we need to learn how to embrace suffering as a way of getting.
to know Jesus better..
We need to learn to be comfortable in navigating through the difficult things in life because.
the truth is the world around us is in a constant state of groaning..
And the effects of sin in this world are still very much at play..
Therefore, learning how to go through suffering means learning how to see Jesus in the midst.
of pain and inviting others to do the same..
And the key to all of this, Paul tells us, it's becoming like Him in death..
Again, it doesn't seem to make sense..
If we want to gain something, we have to die..
If we want to gain Jesus, we have to die..
How does death lead to gain?.
Well think about it this way..
What was Jesus' biggest achievement when He was here on this earth?.
What would you say that was?.
He did a lot of miracles, fed 5,000 people, walked on water, healed people, healed the.
sick, even resurrected the dead..
But the biggest achievement, I might put to you, that Jesus achieved on this earth was.
in fact His death..
Because Jesus dying was a shock to everyone..
Nobody expected Jesus to die..
Not even the enemy expected things would work out that way for Jesus on this earth..
But He had to die..
He had to die so that we could live..
He had to die so that sin could be defeated..
And so brothers and sisters, in the same way, in order for us to truly know Christ, we too.
have to die to the sins in our lives..
To be in a true relationship with Jesus is to die to the things that hold us back from.
Him..
Why is death so important?.
Because without death, there could never be resurrection..
Without dying to the sin in your life, you will never be able to experience what a changed.
life looks like..
You want to break free from the sins that were holding you back in 2023?.
The only solution is to die to them..
And we do this by renouncing and repenting of our own selfish desires so that we can.

$^{361}$say yes to the path that Jesus wants us to walk down..
And unless we do, we will never know freedom..
You will never know what the new and resurrected life feels like..
You will never know what it feels like to live life and life in the full as Jesus has.
promised us..
We embrace these things..
We embrace suffering..
We embrace even death because we know that death does not have the final word..
And resurrection is a reality, a promise that is to come for those who are in Christ..
We can embrace death..
We don't have to be afraid because death has been defeated..
Now, if all that sounds challenging, it is..
These things are challenging because these are things we'll never be able to obtain,.
to embrace, to fully grasp in our lifetime..
You see, what we've been talking about is not just goals for 2024..
These are goals for life..
This is how Paul describes it..
Not that I've already obtained all of this or have already arrived at my goal, but I.
press on to take hold of that which Christ Jesus took hold of me..
Brothers and sisters, I do not consider myself yet to have taken hold of it, but one thing.
I do, forgetting what is behind, straining toward what is ahead, I press on toward the.
goal to win the prize which God has called me heavenward in Christ Jesus..
The journey to intimacy with Christ, the path of discipleship, a relationship with Jesus,.
the goal of knowing Him is going to be a lifelong pursuit..
It's something that we have to be willing to walk in daily, which is why Paul tells.
us three times, "Press on, strain toward, press on, keep going.".
And in order for us to press on in this journey, we cannot let the things of the past hold.
us back..
Forgetting what is behind, we strain towards what is ahead..
This is why this is a message and a challenge, like I said, hopefully for everyone..
Because if 2023 was the most amazing year for you, good for you, and what a blessing.
that was..
And I don't mean that in a condescending way, really..
But even so, even if 2023 was the most amazing thing for you, press on..
Don't sit back and bask in the glory of the past, but be reminded that there is still.
so much more ahead, that God can do things that you haven't even seen or imagined..
Be thankful for the things God has done, but keep looking forward to what is yet to come..
And if 2023 was a tough season for you, the encouragement is the same..
Press on..
And I know the pain must have been real, and the loss you suffered was unimaginable, and.

$^{401}$it cut so deep, and you don't really even know how to take the next step forward..
But know this, Christ Jesus has taken a hold of you, and He will never let you go..
And I hope that you are walking in a community around you, a community that truly is able.
to walk with you through pain, so you are not alone in this..
And as you look ahead, as fuzzy and dark as the future seems, as painful as you might.
anticipate it to be, be encouraged too, to know that there's a prize that awaits you..
And in the great mystery of God's love and grace, I pray that you will find peace and.
healing in Jesus..
So this is the challenge for us as we walk towards 2024..
This is what God is calling us to, not just for this year, but for our lives as we continue.
on in this journey..
So as we end, actually, I want to open up some space for God to speak to you, right?.
This is just a message from me..
I'm just a person here speaking to you guys..
I don't know what you've been through, but God does..
And there's something that God wants to speak to every single person here tonight, here.
this morning, watching online, wherever we are..
So we need to open up some space for God to be able to do that..
I have no idea, but I do know that God wants to speak to you..
So if you have something you can take notes with now, maybe it's on your phone or something.
that you write with, there's going to be some questions, reflections that we can use to.
guide our time in this..
We're going to take some time actually right now to do this, but you can also take a picture.
if you want some more time throughout the rest of the day as we look toward 2024 to.
do this..
And the team are going to play some music, but let's just open up our hearts right now.
and put distractions aside and ask God to speak to us..
So Lord, here we are..
Lord, we want to put aside some time right now..
Every single person in this room is precious to you..
Every single person in this room is a child of yours, whether they've realized it yet.
or not..
And you are holding out your hand..
That's the picture we got this morning, church, of Jesus holding out his hand, offering it.
to us..
Father, I pray that you would give us the strength and the courage now to take you by.
the hand..
Holy Spirit, I pray that you would come speak to us as we look ahead, as we strain towards.
the remaining time we have on this earth..
Help us to put you first..

$^{441}$Help us to know what a true and intimate relationship with you looks like..
Help us to die to the things that have held us back from you..
Walk with us through our suffering and help us to be a comfort to those around us as they.
are suffering..
Speak to us in this time, Lord..
We open up space for you to guide us and to lead us..
Let's just spend some time in prayer on your own, church..
(gentle music).
(gentle music continues).
Jesus, I thank you for speaking to us..
I thank you that as we gather together as family, Lord,.
you are in our midst..
The most amazing thing about you, Jesus,.
is that even as we leave,.
you continue to be with every single one of us.
and our families and the people we love.
and the things that we carry in our hearts..
And so Lord, for the things of 2023,.
we leave them in your hands..
And I pray comfort where comfort is needed..
I pray strength where strength is needed..
I pray thankfulness where thankfulness is needed..
I pray hope where there's been hopelessness..
And as we look ahead, Lord,.
as uncertain, as unknown as it might seem,.
I pray that you would give us the faith to know that.
if you are guiding us,.
even though if we go through difficult times,.
through difficult things, we are not abandoned..
It may be part of your plan,.
but I pray that every decision we make,.
each and every day as Jethro was praying just now,.
as Justin's been praying,.
would be a decision to walk hand in hand with you,.
to get to know you better..
(gentle music).
So Jesus, thank you that you never give up on us..
No matter what comes ahead, Lord, we trust you..
We trust you fully..
Lead us, guide us, protect us with your love..

$^{481}$And we pray these things in Jesus' name, amen..
Let's respond by worshiping Jesus together, church..
You can stand or you can remain seated..
And the worship team's gonna continue to lead us.
in some songs of worship..
[MUSIC].
\newpage



\section{}
\label{sec:xU27bdsfcJo}
\textbf{2024-01-03 A Sign To You [xU27bdsfcJo].mp3}
\newline
\newline
連結: \href{https://youtube.com/watch?v=xU27bdsfcJo}{\texttt{ https://youtube.com/watch?v=xU27bdsfcJo}} ~~~~ 語音日期: 2024-01-03 
\newline
\newline
\hyperref[sec:eCZ2lV0ycAg]{\small{< < < PREV SERMON < < <}}
~
\hyperref[sec:index]{\small{[返主目錄]}}
~
\hyperref[sec:u3L5pvcvlOI]{\small{> > > NEXT SERMON > > >}}
\newline
\newline
$^{1}$Merry Christmas, everyone..
Alright, alright..
You didn't have to. That's so nice..
So, a week before Christmas in 2014,.
a friend of mine who works for Nike reached out to me and she said,.
"We're having an event, a special lunch event,.
to say thank you to all our clients, our customers,.
our friends of the brand.".
She said, "I'd love you to come to the lunch.".
And I said to her, "Look," you know,.
"my thinking was, look, Christmas is a busy time for us pastors..
I don't always can fit in lots of social stuff around Christmas..
This lunch was really not something that really was a priority for me.".
So I reached back out to her and I said, "Look, I thank you so much for your invite..
I'm not really sure if this is going to work for me.".
And she's like, "Andrew," she's like,.
"there's a very, very famous person coming to this lunch.".
She says, "This person is so famous.
and I will arrange it that you will get to personally meet with this famous person.".
I was like, "Who is this famous person?".
She's like, "I'm not going to tell you who the famous person is..
All you need to know is that they're very, very famous.
and you will want to meet them.".
This is like a marketing technique of Nike, okay?.
This is how they get people to their lunches, right?.
So I was intrigued. It worked, right?.
I was intrigued. I'm like, "This priority has now gone up in my priority list.".
So I signed up for the event and a week or so later, I went along to the event..
I got there early and it was one of those really fancy events.
where there's lots of tables and you're assigned a specific table..
When you get to your table, there's a little name tag for you on the table..
So I got to my table, I walked around, I found my name tag.
and I looked at the name tag next to me and I could not believe what I saw..
The name tag next to me read "David Beckham.".
David Beckham, people!.
David Beckham is going to be sitting next to me for an--.
He's so lucky. Can you imagine how lucky David Beckham is?.
To sit next to me? Wow!.
What a day for Beckham..
But I couldn't believe that David Beckham was going to sit next to me..

$^{41}$My heart rate started to go up. My heart rate was pounding..
I was like, "What am I going to talk to David Beckham about?".
And I thought about all my sporting hero moments of my life..
Very short list..
But I did actually play international football..
I actually played international football for Hong Kong and I scored a goal..
I'm like, "I'm going to tell David Beckham about my international football experience.".
And then I realized that's the dumbest thing to ever say.
because he's just going to come back with, "Well, the time that I played in the World Cup.".
So I was super nervous..
I was like, "I'm not sure what I'm going to say to this guy,.
but how exciting that I'm finally going to meet David Beckham.".
I've been a fan of David Beckham for a long, long time..
He's a British sporting icon, of course..
I don't support Manchester United..
But I do support Newcastle United, and they are the greatest football club ever..
But even though he wasn't from my team, he was a sporting icon..
Couldn't wait to meet him..
So anyway, everybody was coming into this lunch..
I'm feeling like I'm the guy who's sitting next to the famous guy, so I'm feeling like all cool..
Everybody's coming in, taking their seats..
Obviously, the seat next to me is staying empty..
I figured David would just come up on stage, they'd introduce him, maybe take some photos..
He might share some words..
And then he was going to come and sit down next to me..
So we all sit down for lunch..
Everybody else on my table sat down..
The space is next to me..
And this guy rushes in. He's kind of late..
He's a little bit overweight, kind of a white guy..
And he comes in, and he's looking for his place..
And he comes to our table, and he's walking around the table trying to find his name tag..
And he comes to exactly where I am and right next to me..
And he looks at it, and he turns to me..
He goes, "Oh, hello. I'm sorry I'm late. My name is David Beckham.".
I'm like, "No, it is not! Your name cannot be David Beckham!".
It turns out that David Beckham is not an international football star..
He's a chartered accountant..
That was the David Beckham that sat down next to me for lunch..
I won't even tell you who the famous person was because they weren't really that famous at all..

$^{81}$And I got to sit next to a chartered accountant who told me over lunch about his three, what I thought were quite spoiled kids..
And it really was an absolute nightmare..
It was probably the most biggest disappointing lunch that I've ever had..
It was a huge disappointment..
Why am I telling you a story about a huge disappointment on Christmas morning?.
Because Christmas morning surely is about the most amazing thing, right?.
It's about the fact that Jesus was born..
It's about the fact that our Savior's come..
That it's a time to glorify and celebrate and worship as we are gathered here today..
Why would I talk about the disappointment of meeting somebody that you thought you were going to meet,.
but it didn't turn out to be the one that you were hoping to meet?.
It's because actually on the original Christmas day, some 2,000 years ago,.
a sense of disappointment was actually how a lot of people in Israel felt when Jesus was born..
So you've got to remember that they were waiting for a very long time for their Messiah..
They were so excited about meeting their Messiah..
They had a David Beckham, if you will..
The one who would come in and take over the world..
The one that they believed would actually finally get rid of the Romans once and for all..
In fact, their prophets from old had spoken about the day when the Messiah would come..
They had a term for it. It was called the Day of Yahweh..
And when the Messiah came, the way they spoke about it was,.
He's going to be like a mighty warrior, like a warrior king who will ride into Jerusalem on a chariot filled with flames..
And as He comes in, He will bring an everlasting peace to the world..
So the evil Greco-Roman Empire will be banished forever..
And it will be like this Messiah will come with an army of angels,.
and they will fight a battle against the Romans..
The Romans will be defeated..
And what will be established will be the Jewish and Israel Empire forever and ever. Amen..
That's what they were looking for..
That's what they were expecting..
They were looking for a Messiah who would be a warring king..
Now, you can understand why they were expecting this..
Imagine what it would be like to be a minority group in a majority culture.
that is constantly telling you that power is held in those who are the strongest,.
who have the armies, and who are oppressive in how they do that power on everybody else..
In the first century, the Israels were under the oppressive regime of the Roman Empire..
They were suffering under that regime..
They didn't have the freedoms that they wanted to have,.
and they wanted their Messiah to come and overthrow an empire that was around them..
And because that empire was all built on war, on violence, on oppression,.

$^{121}$and the use of power to oppress,.
it's understandable that the Jewish people were looking for a Messiah.
who would be able to war himself, defeat the battles and the armies of the Romans,.
and establish Israel forever..
So if that was the David Beckham you were expecting to meet,.
imagine how surprised you were on Christmas Day..
When there wasn't a warring king that broke from the clouds on a chariot.
to overthrow the Roman Empire,.
and instead there was a child, a baby,.
but not just any baby, not a baby born in a castle.
or a baby born in an incredible temple somewhere,.
but a baby born in a smelly, stinky cave under a house..
That's apparently the Messiah..
So the Jewish people struggled to understand this..
Now interestingly, the way that Luke presents the arrival of the Messiah,.
he plays on this tension that there is.
between what the Jewish people's expectations were.
and what the reality was in the coming person of Jesus as a child..
Let me pick up the story for you from Luke 2,.
part of the story that Elizabeth just read to us a moment ago,.
starting in verse 10..
"But the angels said to them, 'Do not be afraid, speaking to the shepherds..
I bring you good news of great joy that will be for all the people..
For today in the town of David a Saviour has been born to you..
He is the Christ, the Messiah, the Anointed One, the Lord.'.
The angels do come on that day, just as the Jewish people were expecting,.
but not an army of angels to overthrow the Romans,.
a choir of angels singing songs..
And these angels speak to a group of shepherds,.
not again the famous people that you'd expect God to speak to,.
but outcasts, nomads, those that were almost rejected by their own people..
It is these outcasts that God decides to make the first announcement.
of the arrival of His Son to..
And the angels show up and say all the right things about the Messiah..
First of all, they say, 'This is good news.'.
They declare good news because the news is of hope,.
and the ancient prophets of old had spoken about the news.
being a good news for all people..
They mention that He's born in the town of David, the city of David,.
which was their belief, that the Messiah would come from the Davidic dynasty,.

$^{161}$the line of King David..
Eventually the Messiah would come from the bloodline of David,.
and here they are telling the shepherds that the Messiah has come.
in the town of David..
Then they say the Messiah has arrived, has come, this anointed one, the Lord..
So you can imagine that the shepherds are super excited at this point,.
and they're still expecting to go into Bethlehem, the town of David,.
and meet some mighty warring king who's about to kick off some violent war.
so that the Romans would be defeated..
Notice what then, in the midst of that tension and expectation,.
the shepherds say..
Verse 12, "This will be a sign to you..
You will find a baby wrapped in cloths, lying in a manger.".
You could almost imagine the shepherds going,.
'Oh my gosh, the Messiah has come. I can't wait.'.
'This will be a sign to you. You will see a baby wrapped in cloths.
and lying in the manger.'.
I'm sorry?.
Sorry, can you just say that last bit again?.
We got you up to that point, but what was that last part?.
The Messiah is going to be a baby?.
A baby wrapped in cloths, lying in a poor manger?.
That's not quite what we ordered..
You ever been to a restaurant and received something that you have not ordered?.
I can almost imagine the shepherds going,.
'Could we speak to the manager, please?'.
Like this is not what they expected..
Now, I want you to see that actually this is everything.
that God is trying to communicate on Christmas Day..
And as we gather to celebrate Christmas Day today,.
it's actually this that we are gathering to celebrate..
Notice what the angels say right at the start..
It says, "This will be a sign to you.".
"This will be a sign to you.".
Now, how we so often interpret that normally.
is that the angels are basically saying,.
"Here are some divine directions to get to Bethlehem.
so that you can find the right person who's the Messiah.".
"This will be the directions for you. You'll find a baby in wraps in cloths.".
That's not what they're saying..

$^{201}$They're not trying to give them some like directions to Bethlehem..
In fact, the word that's in the Greek, the original language for sign here,.
literally translates to something that is beyond,.
something that points to something beyond..
It's the same word that Luke's going to go on to use.
to speak about Jesus' miracles..
He'll say, "These are a sign.".
He does these miracles as a sign.
because they point beyond themselves to the coming power of the kingdom.
and the nature and the character of God..
Well, the angels use the same word here..
They say, "Here is a sign to you.".
Something that points beyond the reality of what it is..
In other words, the Messiah hasn't come as a warring king as you expected..
He's actually come as a baby,.
wrapped in cloths, lying in a manger,.
and that is a sign to you..
It's a message to you that points beyond what it is..
And the question we should ask ourselves on Christmas Day is,.
what's the message?.
What's the sign?.
Well, think about it for a moment..
What is the sign of a child being born?.
I mean, a child comes into this world completely defenseless,.
completely vulnerable..
A child is born into this world completely at the mercy of its parents.
to be fed, clothed, cared for, looked after,.
completely vulnerable and exposed..
A child comes into this world completely at the need of others..
You could not get a more different picture.
between a warring Caesar on a chariot.
and a poor, defenseless child..
And that, the angels are saying, is a sign to you..
A sign to you that God's ways are different from the ways of this world..
A sign to you that actually power is different in God's kingdom.
than the power that is seen in this world..
In fact, I would put it like this,.
that the Christmas story is that true power, victory and hope.
is not found in the expected strengths of the powers of this world,.
but the unexpected fragility of the innocence of a child..

$^{241}$That, my friends, is the heart of the Christmas story..
That if you want to know where true power lies,.
if you want to know where actually hope and salvation can come from,.
it doesn't come from the worldly structures and definitions of power..
It doesn't come from political context..
It doesn't come from armies and bombs and guns and drones and war..
That's not where power is found..
The true power that sets people free, that raises from the dead,.
that resurrects people, that gives people hope,.
that pays the price for sin,.
that power is found in the innocence of the weak,.
and it's the weak who become strong in God's kingdom..
That's a sign to you, the angels say..
A sign to you that you need to shift your thinking about where power lies..
A sign for you to understand that if you're a Christian,.
if you hold the name of Christ,.
then you have to be on the side of the innocent,.
on the side of the vulnerable, on the side of the marginalized,.
on the side of the poor,.
on the side of the ones that are rejected by everybody else,.
because that's how Christ came into the world..
And if Christ comes into the world as a sign to us,.
we have to lean into that side of the world..
We are the ones who should stand against.
the ways in which power is manipulated in the systems of this world.
and instead glorify in the subversive, table-flipping nature.
of God's kingdom and the trueness of power..
Think of it this way..
God's way of saving the world is a child wrapped in cloth,.
not a warrior wrapped in armor..
And that, my friends, is a very important message for our world today..
If you think about over the things that are happening in 2023.
as we draw this year to a close,.
if you have a look around the world at the wars and the political infighting,.
if you look at the divisiveness of communities,.
if you look at the way that hate is being used more and more,.
if we look at some of the darkness that there still is here in the world,.
this message of Christmas should ring true for us more now.
than perhaps ever before..
And although the expectations of the Jewish people in that day.

$^{281}$was to find something that fitted more into the world's perspective,.
God shows up and says, "No, we Christians do things slightly differently..
There's an alternative way of thinking, an alternative way of being,.
an alternative way of living.".
And that should still be a sign to the world that Jesus has come..
And so in a way, there's a fire that gets placed within us.
if we're Christians at Christmas time,.
that any time we actually reach out to a vulnerable, marginalized community.
and we help them and serve them and love them,.
it's a sign to the world that Jesus has come..
Any time we help the oppressed get freedom and new life,.
any time we work with the innocent.
and help them to be found that they are not guilty,.
we are a sign that Jesus is here..
Any time war ceases,.
any time power that is being abused is stripped from someone,.
any time the vulnerable find safety and home,.
it's a sign that Jesus is still here..
I love the way that the Apostle Paul speaks of the incarnation of Jesus at Christmas time,.
writing to the church, and a church that was heavily oppressed,.
and actually a church that was under a huge amount of persecution..
Here's how Paul writes to his church..
He says, "Your attitude," in Philippians 2, "should be the same as Christ Jesus,.
who being in the very nature of God, did not consider equality with God something to be grasped,.
but instead made himself," notice this, "nothing,.
taking on the very nature of a servant, being made in human likeness,.
and being found in the appearance of a man," notice, "he humbled himself.
and became obedient to death, even death on a cross,.
the very weapon that the Greco-Roman Empire used to keep themselves in power,.
would become the very thing that Jesus would humbly submit himself to,.
so that he could reveal true power to the world..
Therefore God exalted him to the highest place,.
and gave him the name that is above every name,.
that at the name of Jesus every knee should bow,.
in heaven and on earth and under the earth,.
and every tongue confess that Jesus Christ is Lord,.
to the glory of the Father.".
Isn't that so beautiful?.
He's saying, "This is the incarnation..
This is a sign to you," he's saying, "Church,.

$^{321}$that God decided not to hold power like the world holds power,.
but to humble himself, strip himself of power, if you will,.
come down to this world as a defenseless child,.
to be a man amongst us, so through that he would submit himself.
to the will of his Father, be raised from the dead,.
take the sins of the world on his shoulders,.
pay the price so that we would be free.".
This is the story of the gospel,.
and it starts not with a warring king in a fiery chariot,.
but with an innocent child in a dirty stable..
It's a profound invitation for Christians on Christmas.
to declare a different spirit in this world,.
a different way of understanding power..
I wonder if you could think about it this way..
Imagine if the angels came today,.
maybe here's some of the sort of things they might say..
"This will be a sign to you..
You will find a church wrapped in the pain of its community,.
lying in the poorest of the poorest communities, bringing hope.".
Or maybe the angel would say, "And this will be a sign to you..
You will find believers in Christ Jesus.
wrapped in the hope of the gospel, lying in the marketplace.
and the workplaces and their families.
and their spheres of influencing, offering hope.".
Or maybe I would say, "And you will find me,.
and this will be a sign to you..
You'll find me wrapped too often in the cares.
and the worries of this world,.
but lying in the nurturing arms of my Father.
and finding my shalom and peace there.".
This, my friends, is the story of Christmas..
And Paul says, "Your attitude, your mindset,.
the way you think should be like this,.
not the warring, powerful, mighty thing that the world would say,.
but the innocent, weak thing that through God is made strong.".
This is why Jesus's beatitudes are so important to us..
Those who mourn will be comforted..
Those who've been rejected will be welcomed..
Those who are poor will be found rich in Him,.
but those who are rich will be stripped down..

$^{361}$The reversal of the heart of God.
to take the broken things of this world.
and make them true and strong..
That's what we gather together today to celebrate..
And so my encouragement to you on Christmas Day,.
and I pray that this would inspire you and fill you with hope,.
is that if you are to live out the Christmas message,.
it is to live out that message,.
that we are people who see power in a different way..
We are people who see love in a different way..
And we are people who carry a message that is truly good news..
May we walk towards the vulnerable, marginalize the poor..
We walk towards those who need the message the most.
and offer them true hope..
That's the mission of Christmas..
It's what we gather to celebrate today..
And as we worship Him and sing,.
and as we open the presents later on.
and have the incredible food for lunch that we'll have,.
may we do so in a spirit of true gratefulness.
for the humility of a God who shows us a different way..
That, my friends, is Christmas..
Amen..
Would you stand with me?.
I would love to pray with you..
Let's stand together..
Father, I'm just so grateful for each person here today,.
on Christmas Day..
And Father, while of course we long.
to meet the David Beckhams of this world,.
we're grateful..
We're grateful for every person..
The people who are standing around us right now,.
the people in our neighborhoods,.
our apartment buildings, our communities,.
the people in our offices and our workplaces,.
the people that you've surrounded us with,.
we are grateful for them..
Father, I pray that you would put a fire in the vine.
and a fire amongst us as your people..

$^{401}$To be people who love justice,.
people who fight for the vulnerable,.
people who stand for the nonviolent gospel.
that you have given to us in the infancy of a child..
Lord, I pray that this would be our Christmas message,.
that there is the great hope in the world,.
that the darkness is dark,.
but it has not overcome us..
That there is, because of Christmas,.
a light that has come into this world.
in the form of an innocent, vulnerable child.
to show us that despite everything happening for us,.
God can meet us too..
In our own vulnerability,.
in our own brokenness,.
he can be there..
Father, I pray for anyone here in this room.
that maybe doesn't know you,.
but maybe today they've been invited.
by friends and family and they've come here.
and they've heard a story and a message.
about a God who does things differently.
to what we see in this world..
Lord, I pray that that message.
would be an encouragement to them..
And I pray that you would bring many.
in this room today towards you..
Father, we thank you that at Christmas,.
we have this joyous sign.
that you have not left the world.
to spiral out of control,.
but you have stepped in to bring hope..
That Jesus himself would go on to say,.
I've not come to kill and destroy,.
but I've come to bring life and life abundant..
I wanna pray, Lord, for every person.
here at the Vine this year.
that they would know and feel life and life abundant..
And Lord, if there is anyone in here.
who's feeling lowly,.

$^{441}$who's feeling an outcast,.
who's feeling vulnerable,.
who's feeling overwhelmed,.
I pray that they would know.
that Jesus has humbled himself.
and has stepped into the lowliest of places.
so that he would know exactly.
how they're feeling right now..
And so that he would bring a comfort to say,.
I've been there, I know what that's like..
And if you come to me, I will give you peace..
Thank you that the angels would declare.
over the shepherds, peace on earth..
And Lord, that's what we desire..
With all the wars that are happening,.
with the political angst that is amongst us,.
with so many divisive perspectives.
that spew hate rather than love,.
we pray peace on earth..
And we pray this in the name of Jesus..
Everyone says, amen, amen..
Can we worship together?.
Let's continue to worship as a community..
[ Music ].
\newpage



\section{}
\label{sec:u3L5pvcvlOI}
\textbf{2024-01-08 A Call To Intimacy: Introduction [u3L5pvcvlOI].mp3}
\newline
\newline
連結: \href{https://youtube.com/watch?v=u3L5pvcvlOI}{\texttt{ https://youtube.com/watch?v=u3L5pvcvlOI}} ~~~~ 語音日期: 2024-01-08 
\newline
\newline
\hyperref[sec:xU27bdsfcJo]{\small{< < < PREV SERMON < < <}}
~
\hyperref[sec:index]{\small{[返主目錄]}}
~
\hyperref[sec:mXhI72JZGGA]{\small{> > > NEXT SERMON > > >}}
\newline
\newline
$^{1}$Everybody says, amen..
Hey, can we thank our worship team as always?.
Amazing, amazing..
Have a seat, have a seat..
Come on in..
There are some people standing..
We do have a few seats..
Here in the front, there's about three or four seats..
So if anybody's standing, come, come, come..
No one's looking at you, don't worry..
Actually, everybody's looking at you..
It's okay, just come..
Come, come, come, come..
I need about four people, three people, four people..
Come grab a seat, please..
My man, God is dope..
Come with me, come here, come here..
What's up, man?.
God is dope, baby..
God is dope..
There we go..
All right, all right, cool..
My seat is also available..
Anybody else need a seat?.
Lovely lady here..
Hi, how are you?.
Come on in..
This is the anointed seat..
It's because you're next to Jess..
There you go, have a seat..
All right, welcome to the Vine..
My name's Andrew..
If you're new here, we're so glad you're here..
I want to address the elephant on my face, which is the massive lip that I currently have,.
and the big blister on my lip..
I thought if I don't address this at the start, you'll all wonder what happened to me..
So let me tell you the story..
I've just come back from New Zealand..
I just had a short holiday in New Zealand..
And whilst I was in New Zealand, I went swimming with dolphins..

$^{41}$And swimming with dolphins is amazing..
And I'm out there in the ocean..
I'm swimming with these incredible dolphins..
They're baby dolphins with us..
And the baby dolphins are super cute..
And then a stingray comes out of nowhere and starts to attack the baby dolphins..
And I was the nearest, and I thought, "I'm going to put myself between the baby dolphins and the stingray.".
So I swam and put myself between the baby dolphins and the stingray..
The stingray didn't like that..
It stung me in the lip..
And if that story was true, that would be really cool..
But....
That's not what happened..
Thank you. Thank you..
I actually just got sunburned..
And I got so badly sunburned on my lip that it swelled up..
And the way it looks right now actually is kind of a lot better..
It really swelled up..
And I got this blister, and it got infected..
And I had to go to the hospital last night..
I landed last night, went straight to the hospital..
Anyway, all that to say, here at The Vine, we say, "Come as you are.".
And so I did think about putting on a mask today, but I'm coming as I am..
All the good, bad, and ugly of me in this moment..
Don't worry. If you're in the front row, it's not contagious..
Okay? You guys are all good..
One thing that is... I would appreciate prayer..
It's actually very painful, as you can probably imagine..
And so it actually hurts to talk..
And I was praying yesterday, and I was like, "God, I'm sharing a word for the church at the start of the year.".
And it's actually painful to talk..
But I want to do this because what we want to share today with you guys.
is something that our eldership, our pastoral team, our staff have discerned, we've prayed into..
And I want to share something that is a correction to us as a church.
as well as an invitation to us..
And you know when God speaks, and He brings a correcting word,.
it's important that a church community like ours, or like any church community, listens to what God says..
But it's also important that we don't just hear it privately as a leadership..
We want to make sure that that's publicly aware and publicly understood..
But I also want to share it for this reason..

$^{81}$I believe what we're talking about today is actually fundamentally.
what God wants to do and shape in your life in the year ahead..
You might have come into the Vine today, 1st Sunday of 2024,.
praying, wondering what is it that God wants to do in your life?.
How is it that God wants to speak to you today?.
And if the Vine is your community of faith, if this is where you regularly worship,.
then I want to put it to you that what we're going to talk about today and over the next five weeks,.
and actually what we're going to talk about today, we're going to be talking about the whole year, all of 2024,.
is actually not just a word for us as a corporate church,.
it's also a word for you personally, individually..
And I feel like God's got a few things He wants to say..
So I wonder whether we could pray as we come into this time together..
Father, what a joy it is to gather on a day like today,.
and to be a community together, and to come under the ministry of your word,.
not my words, not any pastor or leader's words, but under your word..
Lord, I want to pray that your word would speak today..
Father, I thank you that every single person in this room you love,.
you find is the joy of your heart..
And Lord, you have something you want to say..
And Lord, I pray that whatever words that I'm going to say,.
and not of you would fall away from their ears, your words would stick and remain true..
And Lord, we open our hearts to your voice..
And I just want to pray, maybe you could just take a moment, just five seconds, to pray in your own heart..
Lord, I'm open to what you want to say to me this year..
Maybe just make that a prayer right now, just in the quietness of your heart..
Lord, I'm open to what it is that you want to say to me this year..
And Father, we are excited by that. And we thank you for this in Jesus' name..
Everyone says..
Back in September, end of September, I received a dream from God..
And it was whilst I was on holiday in Koh Samui in Thailand..
And if you've been at the Vine for any period of time,.
you'll probably know that often God speaks to me in dreams..
That's often how I hear from the Lord..
Weirdly though, I often receive these dreams when I'm actually on holiday..
I think this is kind of bizarre because as a pastor, as a full-time pastor,.
I carve time out regularly during my work schedule to hear and discern and to get dreams and vision from God..
But the time that he often chooses to download stuff to me is whilst I'm on holiday..
So I actually have said to my elders, "You should just keep me on holiday the whole time for the benefit of the church.".
But they didn't buy that..
But anyway, oftentimes God speaks to me when I'm on holiday..

$^{121}$And here it was, the same thing, actually the last night of this holiday that Chris and I were having for our 25th anniversary..
And God gave me a dream..
And I know when the dreams are from God, they're incredibly vivid..
They're incredibly detailed..
And usually they have something to say to us as a church community..
And so let me tell you about this dream..
In the dream, I was in a massive large room, a room not too dissimilar to this one..
Actually, it was probably a bit bigger than this one, but it was a big large room..
And the room was completely empty..
There was nothing in this room except for two armchairs in the middle of the room..
And in the dream, I found myself sitting in one of the armchairs, and opposite me was another armchair..
And that armchair was... nobody was there. It was empty..
And as I'm sitting there staring at this empty armchair in this massive empty room, God's presence fell in the room..
And it was the most incredible thing..
The glory of God was all around the room..
His presence was in the atmosphere..
And actually, it reminded me straight away of those times when we gathered to worship here at the Vine..
And we've just experienced an incredible time of worship here together this morning..
And it's those moments that my mind went to as I'm sitting in this chair,.
and I'm experiencing this incredible sense of God being in the atmosphere, being in the room..
You know what it's like when you come to worship here, and it feels like there's this kind of tangible reality of God in the air, if you will..
And that's what it felt like..
But as I was sitting there in this dream, a thought went straight through my mind..
And the thought was this..
I'm not satisfied with God just being in the atmosphere around me in the room..
I want God to come and sit down opposite me..
My heart was, as much as I was enjoying the reality of His presence around us,.
I wanted Jesus to come and sit with me and be tangibly present with me.
so that we could be face to face together,.
so that we could commune in a way that is far deeper and far more intimate.
than just the reality of His presence in the room..
You following this?.
And so as I thought this, as I was like, "God, I don't want you just around me..
I want you with me. I want you here. I want you physically here.".
I saw all these shards of light and all this kind of light that sort of flashed around the chair itself..
And then I saw the cushion of the chair depress, as if somebody with some weight had just sat down,.
although I didn't see anybody, but I knew that Jesus had just sat down in front of me..
And this dream was incredibly real..
And I knew that Jesus had sat down in front of me, but I couldn't see Him..
I didn't see His face or anything, but I knew He was there, not just in the room,.

$^{161}$but now fully present with me..
And I have to tell you, it was the most at peace in this dream..
By the way, when I dream and God speaks to me in dreams, I can kind of think about it.
in the moment that it's happening, and I kind of reflect as it's going..
And it was like, this is the most tangible sense of the presence and the power and the peace.
and the shalom of God that I've ever felt..
And it almost was like in Jesus sitting right opposite me, even in my sin, in my brokenness,.
in the stuff that's ugly about me, the things that I don't talk about, all the stuff that I try to avoid,.
all of that was present to Him and none of it mattered at all..
I had complete wholeness in Him..
It's basically what the Bible speaks about as shalom, completeness..
I was completely how I was designed to be as I was sitting opposite Jesus..
It was the most beautiful kind of experience that I've ever had..
Now, right as I had kind of got my head around the reality of this beautiful experience,.
suddenly it changed..
And rather than Jesus being sat next to me or opposite me, suddenly a young adult from here at the Vine.
that I know, she was sitting suddenly opposite me, which was a bit of a jarring experience.
because it was Jesus and then it became this young lady..
And this young lady I know here at the Vine, and she's single..
And as she's sitting there, and before I can communicate with her or anything,.
she stands up and she turns sideways and she's pregnant..
And I thought, "This is bizarre.".
Because she's single, I know her, she's single, and she stands up and she turns to the side.
to show me that she's pregnant..
Now, in all of this, I know that God's speaking to me about worship and our community as a church..
I know that all of this dream, because of the way it started and the presence of God in the room.
and it reminded me of the Vine, I know that all of this is speaking about our worship..
So this lady was representing, in a way, worship at the church, and she's pregnant..
And I think immediately this is good news..
What God's saying is, "In 2024, we should be pregnant with expectation in our worship at the Vine.".
That God is going to birth some incredible things in our church this year..
But immediately, God was like, "Nah.".
Because I knew she was single, but she was pregnant..
And something didn't feel right..
The picture of it, the reality of it, I wasn't expecting it..
It didn't feel right. It wasn't quite the way things should be..
And I felt immediately that God was saying, "Andrew, in the worship at the Vine...".
Now, when I talk about worship at the Vine, I'm not talking about band or songs or music or that..
I'm talking about your adoration, our adoration of Jesus..
That's worship, our thanksgiving and adoration and praise, our lifestyle, the way we are..

$^{201}$Our worship at the Vine, God was saying, "Andrew, it's not quite right..
There's something wrong..
Sometimes it looks right. Sometimes it doesn't look right..
There's some problem with the worship.".
And as I was sitting in the reality of this lady who was pregnant, it didn't feel right..
I knew that there was something wrong here..
I felt like God said this, "Here's the sense, Andrew..
The church has settled for the general atmosphere of God,.
and they are not pushing beyond that to the actual person of God.".
And I want to say this as strongly to us as I can..
That I think one of the corrections God is bringing to us at the Vine is that at times here at the Vine,.
we have settled simply for the general atmosphere of God..
Oh, we come in on a Sunday and we sing our songs and we enjoy a nice feeling together,.
and we think that that's enough..
And we've settled for the general atmosphere of God,.
and we haven't pushed beyond that to meet with the actual person of God..
As I was sitting in that reality, suddenly my dream changed and gone were the two chairs..
And here's what happened next..
There was a great big bed..
There was a massive bed, a bed as big as you could ever imagine in this room,.
filled with this whole room..
It was one big bed, one massive mattress..
And on this bed, I could see all these people sitting..
There were some people sitting on their own..
There were some people sitting as couples..
There were some people sitting in groups, like as friends in groups..
And all of them were covered in bed sheets on this massive bed..
Now, it's important that you understand there wasn't anything overtly sexual in this,.
but I knew that God was speaking to me about intimacy..
And He was saying, "Here's the reality..
I have created people to be intimate with me,.
but at the vine, you have replaced your desire to be intimate with me.
with only a desire to be intimate with one another.".
Come on, church..
I know this is strong, but I'm putting it out there..
That we have replaced our rightful desire to be intimate with God,.
and we are only interested in being intimate with one another..
Whether that's groups of friends in the intimacy of friendship,.
whether that was couples in the intimacy of sexual intimacy,.
whether it was people who were on their own trying to find intimacy in various other ways,.

$^{241}$I think the bed represented a lot of different things,.
but it was the reality that there are alternative forms of intimacy.
that you can experience..
As soon as I saw that, the bed disappeared again, and, woom,.
I was straight back to the two chairs, and once again, I was sitting in the armchair..
And right opposite me, the armchair decompressed once again.
as the presence of Jesus sat down..
And as soon as the presence of Jesus sat down in front of me again,.
I felt this incredible, powerful sense of peace and calm..
Immediately, I knew that every longing I had in my heart,.
every thing that I desired, all the intimacy that I could ever want for.
was satisfied in the reality of the presence of God..
Not the atmosphere of God, but the person of God who was right here with me..
And this time, I jumped up in my dream..
And I jumped up, and I'm like, "Jesus, please stay there. Don't go anywhere.".
But I was like, "I want others to experience this, right?".
Like, I don't want to be selfish here..
I don't want to be the only one who ever gets to feel complete and whole..
And so in the dream, I started to look around the room,.
and I tried to find some other people. Remember, it was a big, empty room..
But I wanted to try and find some other people to come and sit down with me.
and experience the intimacy that I was experiencing with Jesus..
And that was my longed for and my wish for,.
and that was basically everything that I could feel in every ounce of me was,.
"Where are the people who will come and experience,.
who will be willing to sit for a moment and experience this intimacy with Jesus?".
And on that thought, unfortunately, I woke up..
And there I was, in my bed in Koh Samui, Thailand, on holiday,.
and I felt like I was in the worst place in the world.
because I had just experienced the best place in the world,.
that there is nothing that can ever replace the intimacy of Jesus Christ..
And when you've experienced it like that,.
everything else is sort of this other kind of experience..
And although I was in paradise and in this beautiful place,.
I longed to get back to that..
I grabbed my phone from the side of my bed,.
and I started to make all these notes about the dream.
so that I could tell you on the 7th of January..
But I wrote all the details about the dream down as much as I could,.
and I started to reflect a little bit about intimacy.

$^{281}$because I realized that what God's really speaking to us about is intimacy in this year..
In fact, everything we're going to talk about here at The Vine is going to be that one thing..
We're going to talk a lot about intimacy..
Here's the first thing you need to know about intimacy..
Every human being has been created with a desire, a longing for intimacy..
I believe that this is a God-created desire in all of us..
God has created every human being with a desire to find intimacy,.
and that intimacy is both with Him and also with one another..
That it is a good and a holy thing to want to be intimate..
That we long and desire for intimacy..
And that desire that we have is so deep in us, so visceral in us,.
that it only ever fulfills its fullness in our relationship with God..
I realized as I was literally sitting there writing all this down on my phone.
at like 3 in the morning or whatever it was,.
I remembered this is the reason why God created in the first place..
I mean think about it for a moment. Think about it theologically..
We're going to get theological. We've gone from dreams to theology..
Okay, with me. Think about it theologically..
Because in the Godhead at the beginning of all things,.
in the Father, the Son, and the Holy Spirit,.
there is an eternal intimacy that they have that is the fullness and the completeness of Shalom..
God did not need anything other than the Father, the Son, and the Holy Spirit together..
In other words, creation is not to satisfy something that God lacks..
Are you with me?.
God has the fullness of intimacy..
If you want to know and understand what intimacy is,.
the Father, the Son, and the Holy Spirit in the fullness together is the picture of intimacy..
The reason why God creates is invitation..
He creates because He wants to invite others into that intimacy..
In the same way that I jumped out of my chair and I was like, "Jesus, stay there.".
But who else can experience this with me?.
I could almost imagine the Trinity getting together..
"Hey, guys, this is amazing, but let's create.
so that others can come and be a part of this, and they can experience this,.
and they can know what the fullness is of this incredible intimacy..
And that intimacy will create the fullness in them, and it will be like this incredible thing.".
The first thing you need to know about intimacy is that it's founded on three things,.
creation, love, and invitation..
Every form of intimacy makes space for others..
Intimacy creates a place for others to come..

$^{321}$Intimacy invites and loves and creates..
And here is the really scary thing..
Every human being has this innate desire for that kind of intimacy,.
but in our fallen nature, in our brokenness,.
and in the struggles that we all hold in us,.
we more often than not seek the satisfaction of that desire for intimacy.
in things that can never satisfy it..
One of the great stories of Scripture is human beings trying to find intimacy.
in other things other than in God, because they have this desire in them,.
and that desire drives them to try to satisfy it..
But in order to try to satisfy that desire,.
they begin to look to one another for satisfaction that only God can bring..
Not that again, connecting with one another and being intimate with one another is a bad thing..
It's not. It's a good and holy thing, but we have to get the order right..
"Love the Lord your God with all your mind, heart, soul, and strength..
Love your neighbor as yourself.".
Know what it is to be intimately connected with God,.
so that out of the overflow of the intimacy, the fulfillment, the connection,.
we can jump up and invite others to it..
Our intimacy with one another is a response to the intimacy with God..
But so often in our brokenness, we make our intimacy with one another the true intimacy..
We think that's where we will get connected,.
when that so often just leads to further brokenness..
Let me put it in terms of the dream for you..
There is an invitation to a deep and personal armchair experience with God..
But so often we choose the more shallow, the more immediate bed experience in intimacy with one another..
That's just the reality of us as human beings..
And it's also, I think, the reality for us here at the Vine..
And as God began to unpack some of this to us, I began to see and sense His correction..
I came back from the holiday to the staff here,.
and I began to talk to our staff and our elders and our pastors about this idea of intimacy..
And as God so often does, God doesn't ever just speak to one person..
He never just speaks to the senior pastor..
He had been speaking to many people in our staff, many people in our intercession team,.
for over a year about the same sort of ideas, the same threads,.
that there's something not quite right in our worship,.
that there's something not quite right within our community,.
that we're not engaging in intimacy with God in the way that He designed it for me..
And as we began to talk about it, and as we began to pray about it,.
as we began to discern what it is that God is saying,.

$^{361}$I think ultimately what He's saying is this,.
that so often we show up on a Sunday,.
and we experience this incredible sense of worship like we had this morning,.
and we stop there. We think that is enough..
Or we think that that's what it's all about..
That's not what it's about..
Intimacy with God is not fully about coming to church on a Sunday..
And if we ever make the reality of our intimacy with God.
based on the atmosphere we feel at the vine on a Sunday,.
oh, we are so missing out on the completeness of who we've been designed to be..
You have not been designed for the atmosphere of God..
You've been designed for the person of God..
Come on, church. Anyone here? Yeah?.
There's a passage in Scripture that I love..
It's found in Psalm 100, verse 4, and it says this,.
"We enter His gates with thanksgiving in His hearts.".
We go into the courts with praise..
The psalmist is being beautiful here because the psalmist is creating an imagery of the temple..
He's talking literally about the temple..
And if you know anything about the temple,.
it was designed with the Holy of Holies right in the center of the temple,.
where the fullness of God's presence was..
And in the Holy of Holies, only one priest could go in once a year,.
and only after a purification rite of a number of days..
And it was always a thing..
But outside of the Holy of Holies was where the sacrifices were made..
And that was where the priest stood for the atonement of sin..
It was one of the most intimate expressions and connections.
that humanity could ever have with God..
And outside of the inner courts were the outer courts..
And on the outer courts is where they paid for the animals to be sacrificed,.
where they did business, where they connected communally..
And then outside of that were gates that led people into all of that..
The idea in the temple was that you went through the gates into a deeper place of intimacy..
And the psalmist says, "Do you want to know what it is that brings you as a human being.
into a deeper place of intimacy?.
It's worship..
It's worship. It's adoration. It's thanksgiving. It's praise..
You enter the gates with thanksgiving on your heart..
You go into the courts with praise..

$^{401}$And your worship, your worship is the vehicle that moves you.
from outside of God's presence into the heart of His presence..
The problem is, he did not write, "You camp at the gates with thanksgiving in your heart.".
He did not say, "You stand on the outside of the courts.
and you just stay there with praise.".
He says, "There's movement. You're going from the outside into the inner place.".
The problem I think we have at the vine sometimes.
is that we're camping at the gates with our worship..
And every time we think that when we gather on a Sunday.
and we experience this incredible sense of God's presence,.
as important and as good as that is, but when we think that that's enough,.
or when we think that that's all there is, oh, we are so missing out..
We're camping at the gates..
I want you to have a mind shift..
I want you to think every Sunday when you come in.
and we experience the power of God together,.
which is a beautiful, precious thing..
You should be thinking, "That's a gate..
This is a gate right now. I'm entering through this into something else..
And that something else is a deeper, more intimate, more personal relationship with God.
where God will tell me things that will shape and change me..
Not always things that I can handle. Not always things that I like..
But God, I want that deepened relationship..
I don't want you in the atmosphere. I want you right here.".
We enter His gates with thanksgiving on our hearts..
And God is saying that this is a correction, but it's also an invitation..
It's an invitation for us as a church to go,.
"This is what I want to make 2024 about..
If there's anything I could start the new year with,.
surely it's this, that actually I get the great joy of being a human being.
that experiences incredible intimacy with others,.
but that intimacy comes first and foremost through my intimacy with God.".
See, here's the thing you need to realize..
Our vision here at The Vine is growing big people..
It's all about your maturity in Jesus..
We do everything at The Vine to help you to grow in your maturity with Jesus..
Hear this, you will always remain spiritually immature.
if all you ever do is connect with the atmosphere of God..
I'm going to say that again..
You will always stay spiritually immature.

$^{441}$if all you ever do is connect with just the atmosphere of God,.
rather than pressing through that to the actual true and living person of Jesus..
And as wonderful as His atmosphere is,.
as wonderful as that presence is on a Sunday,.
it's an invitation to you to intimacy..
Are you with me?.
It's not an invitation for you to feel good on a Sunday..
It's an invitation to a life that is changed.
by the intimacy of your relationship with Jesus Christ..
So what we're going to do over this series here at the start of the year,.
over the next four weeks, we're going to unpack this idea.
of how we can enter through the gates..
Over the next two weeks, I'm going to talk to you about.
some of the ways that we struggle in our intimacy with God..
I'm actually going to talk to you about two of the very key ways.
that we actually stop ourselves from being intimate with Jesus..
And then we're going to have two weeks.
where we're going to look at alternative forms of intimacy,.
the things that we turn to other than in our direct intimacy with God..
And I believe through that, we'll be able to paint a picture for you.
of what it is that we sense God is both correcting us in,.
but also inviting us lovingly to in the year ahead..
Is this helping anyone so far?.
Are you excited for your year?.
Yeah. Okay..
Now, what I want to do is I bring things to a bit of a close..
I want to tell you a little bit about what this vision for intimacy is like..
I want to give you a taste of what this intimacy can feel like, can be like..
The Bible is filled with stories about this intimacy with God,.
this closeness with Him..
And rather than sort of say, "Hey, here are the five steps.
that you can do in your life to be intimate with God.".
The reality is intimacy is a very personal thing..
I mean, if you talk about me and my wife, Chris, our intimacy together,.
we don't share in intimacy with anyone else..
It's a personal thing. It's a unique thing..
And I'm not just talking about sexual intimacy..
I'm talking about communication and what we do as friends.
and how we enjoy life, the shows that we like or whatever it is..
There's a uniqueness to our intimacy..

$^{481}$There's going to be a uniqueness to your intimacy with God..
So we're not going to say, "This is exactly what you have to do.
in order to be intimate," because that's for you to discover..
What we will do, though, is paint for you a broad and beautiful picture.
of what intimacy with Jesus looks like..
And our prayer is, as you see that intimacy, you're going to go,.
"What can I do in my life to get like that?".
Are you with me?.
All right. So let me paint for you this intimacy..
Now, before I read this, by the way, I'm about to read to you a psalm of David..
You need to understand David was not a perfect person..
David had many struggles..
David was broken, and David had brokenness in intimacy..
He had brokenness of sexual intimacy..
He had brokenness with his friends..
David is a complicated person..
And yet, even in the complications of his brokenness,.
he can express this desire and this need to be deeply intimate with God..
And I think that's an encouragement for some of you here,.
because you might be hearing this, and you might be going,.
"Well, I'm not that kind of Christian..
I'm not even sure if I am a Christian..
I'm not even sure if God can accept me intimately..
I don't even know if He wants me to be with Him.".
And you might be thinking that your sin or your brokenness or your background.
disqualifies you from intimacy with Jesus..
If David is qualified, you are qualified..
Are you with me?.
Now listen to this. Psalm 63 says this..
Listen to how David expresses his intimacy with God..
"O God, You are my God..
Earnestly I seek You. My soul thirsts for You..
My body longs for You..
In a dry and weary land where there is no water,.
I have seen You in the sanctuary and beheld Your power and Your glory..
Because Your love is better than life, my lips will glorify You..
I will praise You as long as I live, and in Your name I will lift up my hands..
My soul will be satisfied as with the richest of foods..
My singing lips, my mouth will praise You..
On my bed I remember You..

$^{521}$I think of You through the watches of the night, because You are my help..
I sing in the shadow of Your wings..
My soul clings to You. Your right hand upholds me..
Those who seek my life will be destroyed..
They will go down to the depths of the earth..
They will be given over to the sword and become food for jackals..
But the King will rejoice in God..
All who swear by God's name will praise Him, while the mouths of liars will be silenced..
Can you hear David's passion?.
He's like, "Let me tell you about what this intimate relationship that I have with God is like..
I mean, it's my everything. Everything in me longs and yearns to be connected to Him.".
Notice just some of the things that he says..
He actually paints for us a beautiful picture of biblical immanency..
He says, first of all, "You are my God.".
I love this. "You are my God," he says..
In other words, there's no other God..
There's no other thing that I can be intimate with that could ever satisfy me..
You are the only thing that can satisfy me..
And notice, he says, "You are my God.".
He doesn't say, "You are Israel's God.".
He doesn't say, "You're a God of some nation.".
He didn't say, "You're the God of the vine at 11 a.m. on a Sunday.".
He says, "You're my God.".
This is personal..
I know you. I want you..
We're together. We're connected and committed..
You are my God..
Some of you, the best prayer you can do in 2024, right at the start,.
is simply to say that prayer after the service..
"God, you are my God.".
Because some of you, maybe that will be a struggle..
But for some of you, that's your soul cry..
That's your first response to recognize that maybe you haven't been able to say that,.
but that's the desire of your heart..
"You are my God.".
Notice the passion he brings..
Notice things like, "My soul thirsts for you..
My body longs for you because your love is better than life..
My soul will be satisfied. It's with the richest of foods..
My soul clings to you.".

$^{561}$He talks about this idea of his soul, the innermost part of who he is..
Not just his being or his thinking,.
but everything about him longs for this intimacy with God,.
and he's crying out for that intimacy..
That's like me sitting there and going,.
"I'm not satisfied anymore with you in the atmosphere..
I want you right here..
My soul clings to you," says David..
Now notice he talks about two very important things..
He says, "I have seen you in the sanctuary and beheld your power and your glory.".
There it is..
"I have seen you at 11 o'clock at the vine in Wanjai,.
and I beheld your power and your glory.".
Amen?.
I want you to hear this, everything I've been saying today..
We're not changing anything in terms of what we do here on a Sunday..
We love the reality of God's presence with us..
We'll continue to worship and praise him and do all the things that we do..
What we're calling for is a bit of a mindset change,.
because what we do on a Sunday is not necessarily wrong,.
but how we're reacting to it is what I think God is trying to create..
Because David says, "I've seen you in the sanctuary..
I beheld your power in that place.".
It's an amazing thing he's saying..
"I love the sanctuary..
I love the power that I feel and experience there.".
But notice what he also says..
He also says, "On my bed I remember you..
Through the watches of the night you are with me.".
In other words, I have felt your glory in church, but guess what?.
Your glory is not just in the church..
I've been intimate with you on a Sunday at 11 o'clock, but guess what?.
God, you're also intimate with me at work on a Monday morning at 9..
Guess what? You're there with me all the time..
Because my intimacy with you is not defined by a location..
It's defined by a relationship..
My intimacy with you is not defined by some band who plays well..
My intimacy with you is defined by the Word of God, my relationship with you,.
my prayer life, how I open Scripture..
It's defined by so many other things..

$^{601}$I have seen you in the sanctuary, and I beheld your power,.
but I have not made that my all and all..
I've also found you on my bed late at night when I'm anxious and overwhelmed..
I've found you in the darkest watches of night..
I have experienced your intimacy in all of my life..
That's what it means to commune with God..
Then he says all these things..
Now, I'm not going to call these benefits..
I'm not going to call these benefits of intimacy, because if I call them that,.
some of you in this room will want to seek him for his intimacy for the benefits..
And as soon as we seek God intimately for what God can do for us,.
we've missed the whole point, and I'm going to talk about that in a couple of weeks' time..
These are not benefits. Let me talk about these as fruits of intimacy..
These are things that are born in our lives as we find ourselves intimate with God..
Very quickly, here they are..
Things like, because you are my help, God helps us..
God comes and be with us..
He says, "I've hid in the shadow of your wings.".
There's protection. That's one of the fruits of intimacy, the protection of God..
Your right hand upholds me, feeling God is with us in the darkest moments..
When life is overwhelmed, it's a fruit of that intimate relationship with him..
Those who seek my life will be destroyed, knowing that God fights battles for you,.
that you cannot fight, that you'd be overwhelmed..
Those are the fruits. It's like confidence. It's like assurance..
It's like knowing God is with you..
Let me pull this together with a quick definition of biblical intimacy..
I would say it's this, an orientation of all of life that pushes beyond just the feeling of God.
to an actual real and deep relationship with him that produces in us the fruit of assurance, confidence, and joy..
That's Psalm 63. That's, I think, how so much of Scripture speaks about the intimacy of God..
That's worth taking a photo of if you have a mobile phone..
Because we're going to be digging into that a lot over not just these weeks, but this year..
As you're taking a photo of that, I wonder if the PowerPoint team, if you could jump back to that section earlier with the three questions..
This is also something I want you to take a photo of, and I'm going to finish with this..
Will we continue to simply enjoy the atmosphere of God, or will we desire to move beyond this to deepen our relationship with the person of God?.
Will we continue to settle for lesser forms of intimacy to satisfy a longing that can only be fully satisfied by God?.
And once we have found true intimacy with God, will we only long for ourselves to experience it, or will we….
Is there another slide? It's okay if there's not..
Or will we invite others to experience it?.
Here's how I'll close, and we're going to pray. I've gone a bit long. Is that okay? Everybody all right?.
Here's how we're going to close..

$^{641}$This is why we're doing it in 2024, why God's calling us to intimacy..
2023 was about the Exodus. We spent a whole year pretty much on the Exodus..
And the Exodus is all about that idea of God calling us out of slavery into freedom..
Why? We said it almost every week..
Why does God do the Exodus with Israel?.
So he can draw them together in intimacy, so they can be their God, and they can be his people, they can become a worshipping community..
That's what he said to Pharaoh, "Release my people, let them go, so they can worship me.".
2023 was our Exodus. It makes sense, therefore, that 2024 God is calling you to intimacy..
You've gone through Exodus. Now is the time to draw near to God..
My prayer for you is that you will find your satisfaction fully in him..
My prayer for you is that he will challenge you to let go of some false versions of intimacy that you're holding on to..
My prayer for you is that you'll want to push beyond the great experience of Sunday morning here at the Vine,.
and actually find a deeper relationship with him in your everyday life..
And through all of that, he will be glorified. Amen?.
Could you stand with me? I would love to pray with you..
And maybe if you're comfortable, you can open your hands..
And I'm going to pray, and then the team are going to play..
And we're going to finish with a song of worship, which is designed to move us through the gates into God's presence..
Lord, I thank you so much for each person here..
Lord, I thank you that as we start the fast tomorrow, we start the fast with a passion for intimacy..
And Lord, we thank you that for the 21 days, we're going to get a devotion every day written by a congregation member.
talking about their own intimate relationship with Jesus..
And Lord, I pray that the fast, I want to just pray over you right now..
And if you've not yet decided what it is for you to fast, that's fine..
You've got about 15 hours to do so..
Lord, I want to pray that you would bless each person in this room with what it is that you're asking them to sacrifice,.
to let go of, to lay down, so that they can move from the atmosphere of God to the person of God,.
so that they can commune with you intimately..
Lord, fasting is putting something aside to commune with you deeper..
Lord, I pray as we move into the fast tomorrow, that each person in this room watching online and our overflow.
will be filled Lord, with your intimate connection as they sacrifice and lay something down.
to start 2024 responding to your word, Lord..
Father, there are some people in this room that are going to need to repent in the next number of weeks..
Lord, I know I need to..
And so Lord, I want to pray for your kindness that leads us to repentance..
And Lord, if there are people in this room that have been convicted by this word this morning,.
I pray that they would bring that to you and respond and ask for your forgiveness..
And that they would start this year with an invitation to a deeper relationship with you..
Lord, as we talk about how we stop ourselves from being intimate, as we talk about alternative forms of intimacy,.
would you speak to us Lord, each one of us..

$^{681}$And Father, ultimately, that we would grow to be the people that you have created us to be..
We're excited about this journey, Lord..
When you speak a word like this, it is always because you are doing it in love.
to see your people become the very people you've created them to be..
I pray you would release this joy, release this excitement and expectation,.
release your conviction and repentance over the Vine Church,.
so that 2024 could be truly a year where we find you in new and fresh ways..
And we thank you for this in Jesus' name..
Everyone says amen..
\newpage



\section{}
\label{sec:fBOsothuSTk}
\textbf{2024-01-15 What Stops Us: Rumors of God [fBOsothuSTk].mp3}
\newline
\newline
連結: \href{https://youtube.com/watch?v=fBOsothuSTk}{\texttt{ https://youtube.com/watch?v=fBOsothuSTk}} ~~~~ 語音日期: 2024-01-15 
\newline
\newline
\hyperref[sec:60uvff9Awtw]{\small{< < < PREV SERMON < < <}}
~
\hyperref[sec:index]{\small{[返主目錄]}}
~
\hyperref[sec:8XDQRPVQTPs]{\small{> > > NEXT SERMON > > >}}
\newline
\newline
$^{1}$In Jesus name..
Everyone says, amen..
Hey, can we thank our worship team.
and these guys, grateful for them..
Have a seat, have a seat..
My name's Andrew, I'm one of the pastors here.
and it's so great to have you with us..
If this is your first Sunday of 2024,.
your first service with us this year,.
we're so glad you are here..
Last week, I shared with us all a dream.
that I sense God had given me at the end of last year.
or towards the end of last year,.
that was really a call to intimacy..
All the things that I've just kind of been talking about.
in that prayer moment with us, this call to intimacy..
And I said last week that we as a senior leadership.
here at the Vine, we've been wrestling with all the things.
that God has been saying to us as a church..
And God at the beginning of this year.
is bringing both a correction and an invitation to us..
And we talked about this a little bit in detail last week,.
that God is bringing a correction to us.
here at the Vine around our worship..
And it's really important that when you're a church,.
that you keep your ears open and your heart open.
to all the things that God wants to say to you..
And whilst we worship and we honor and we walk.
with a God of love and peace and mercy and glory and grace,.
He's also a God that sometimes has things to say to us.
that aren't always easy to hear..
And we've heard here as a leadership.
that we have to reflect a bit deeper.
around our worship experiences with God..
And one of the things that we've heard.
at the beginning of this year.
is something that needs a correction in us..
And that is so often we can come here on a Sunday.
and experience what we experience..
And we kind of exist, if you will,.

$^{41}$off the atmosphere of God, as great as that is,.
and what a privilege it is to have the presence of God.
here in the room with us..
But we think that that's the all,.
that's the be all and end all..
And we don't push beyond the atmosphere of God.
to the actual person of God,.
to the person of Christ with us..
And that we kind of settle, if you will,.
for a feeling of God in the room,.
rather than a personal, active, and living relation.
with Him every single day..
And so last week, we unpacked this idea.
of how so often we can find ourselves stopping short..
That the Scriptures would say that it is our praise.
that is leading us through the gates..
We go through the gates with thanksgiving on our hearts,.
but we go into a deeper relationship with Him..
Our worship is not us camping at the gates,.
our worship leads us through the gates.
into a deeper reality with Him..
Does that make sense to you?.
And I said last week that all of us have been created.
with this innate desire for intimacy with God..
It's a wonderful, beautiful thing,.
that God has created you with a desire.
for intimacy with Him..
And the reality is, is when we don't satisfy.
that innate desire for intimacy with God, with God,.
we'll look for lesser and other forms of intimacy.
to satisfy what actually can only be satisfied in God..
And so often, we look for that intimacy.
in broken ways with one another..
And in our brokenness and intimacy together,.
we think that will plug the intimacy desire.
that God has given in us..
And we talked about last week how it's only in the fullness.
of that intimacy with Christ that we're then able to walk.
in healing and good intimacy with one another..
It flows in that way..

$^{81}$That was the word of correction last week,.
but there was also a word of invitation for us last week..
And it was an invitation to orientate.
all of our lives towards God,.
to recognize that church, this moment that we have together,.
this incredible 90 minutes that we share on a Sunday.
is a wonderful, beautiful God gift,.
and we're so grateful for it,.
but it's a catalyst into a deeper and greater relationship.
with Jesus outside of those 90 minutes,.
and that our relationship with God,.
our intimacy with Jesus is not dependent on the church..
The church catalyzes that relationship,.
but we're not dependent on the church.
for the fullness of that relationship..
My heart for you is that you would come in here on a Sunday.
and you would encounter incredible intimacy with Jesus,.
but my greater heart for you is that you would find yourself.
more intimate with Jesus outside of these services.
than you are even here in these services,.
that it's in your workplaces and your relationships.
and your family and you yourself.
when you wake up on a Monday morning.
and you open God's word, that in those moments.
you find the richest of intimacy with Christ,.
and that you're not coming into church.
so drained and exhausted after a crazy week.
seeking an injection of Jesus.
to get you through the week ahead,.
but instead the whole week has been a living,.
embracing faith for you,.
and you come into service on a Sunday.
with this overflow of the intimacy of Christ.
throughout the week, and you can't wait.
to be with the people of God in the church of God.
experiencing intimacy together..
It's a catalyst to your faith..
Your faith should not be dependent on it..
Are you with me, church?.
So there's a correction, but there's an invitation..

$^{121}$And last week I spoke about a biblical definition.
of this kind of intimacy..
I wanna start by reading this to you once again..
An orientation of all of life.
that pushes beyond just a feeling of God.
to an actual real and deep relationship with him.
that produces in us the fruit of assurance,.
confidence, and joy..
That's what I think God is welcoming.
and inviting the vine into this year,.
inviting you personally into this year,.
an orientation of all of your life.
where you're not just existing off of some feeling of him,.
but you have a tangible, real daily relationship with him.
that creates a fruit in you of confidence and assurance.
and joy that you feel the fullness of life..
And I tell you what, Hong Kong right now,.
our world right now needs churches.
filled with individual Christians.
who have assurance in their faith,.
confidence in the gifts of the Holy Spirit,.
and a joy for life so that they could go.
into the workplaces, into the marketplace,.
into the corridors of power of that city,.
and speak of that assurance and that confidence.
and that hope and bring a new narrative to this world..
Surely we need that..
Our intimacy with God produces an overflow in us.
towards this world that God so deeply loves..
What I wanna do this week and next week.
is share two critical traps.
that I think all of us Christians fall into.
that have the power to actually restrict our ability.
to be intimate with Jesus..
I wanna share two critical things this week, next week,.
that are traps that I think we fall into.
that restrict our ability to move.
into deeper intimacy with him..
And I think if you can open your heart.
to these two things that we're talking about.

$^{161}$this week and next,.
I think it could be a wonderful way.
for you to start your new year..
If you can deal with some of this in your own life,.
and what I'm teaching you today has come out.
of a deep wrestling myself with this very issue..
If we can wrestle with it,.
I think you'll set yourself up for a freedom in intimacy.
with Jesus in the year ahead.
like you've never experienced before..
Two critical traps..
To introduce you to the first one this afternoon,.
I wanna tell you a little bit about the story.
of Ruth in the Bible..
The book of Ruth in the Old Testament,.
one of my favorite books in the Old Testament,.
one of my favorite books in the whole Bible..
It's the shortest book actually in the Old Testament..
So it's an easy one to read..
I recommend it in the week ahead..
It's a awesome story about a family..
And I love stories about families.
because family is something that's really important to me..
In this story, there is a father, his name is Elimelech,.
and he has a wife named Naomi, and they have two sons..
They live in Bethlehem..
Yes, the actual same Bethlehem as the Christmas story..
The name Bethlehem means the house of bread.
because Bethlehem was famous for the fields around it.
that were filled with corn and they made bread in that land..
But in this story, a famine happens.
right at the beginning of the story..
And the famine is so hard that Elimelech decides.
to actually uproot his family and move to a new nation.
where they can live and survive..
They move to a nation known as Moab..
As they go into Moab, they settle as a family,.
and it so happens that his two sons marry Moabite women,.
Orpah and Ruth..
Now, as the story goes, as they marry these women,.

$^{201}$and a short time after that,.
a tragedy happens to the family,.
and Elimelech and his two sons die..
And suddenly, Naomi is left as a widow.
with two daughters-in-law, two daughters-in-law.
that are not from her own bloodline or from her own land..
And around about this time,.
Naomi hears that the famine has lifted in Bethlehem.
and that she can now return home..
And she says to her daughters-in-law,.
"Look, I release you from the family bond, if you will..
"You're better off, your futures are better off.
"to remain here in Moab than to come back with me.".
And she tries to release her daughters-in-law..
Orpah decides that for her and her journey,.
that's the best thing for her to remain in Moab..
But Ruth replies in some of the most beautiful language.
and words that we have in the whole of the Bible..
She says, "No.".
She's like, "Our family is more than just this bloodline..
"We're connected together, and your land will be my land..
"Your God will be my God.".
And so Ruth makes the decision to not leave Naomi,.
not only a widow, but on her own,.
but instead come back with Naomi and settle in her land,.
in the land of Bethlehem..
As they get back to Bethlehem,.
the harvest is about to happen,.
which is a wonderful thing in that land..
But Naomi and Ruth find themselves on really hard times..
In the patriarchal culture of that day,.
without a husband for Naomi or a husband for Ruth,.
it meant that they were under an impoverished situation..
In fact, they were so poor,.
they were struggling to make ends meet and survive..
And Naomi, in the midst of that poverty,.
realizes that there is something that they can do..
She says to Ruth, "I want you to go into the fields.
"during the harvest time.".
And she says, "I want you to glean in the harvest.".

$^{241}$Ruth decides to go and glean in the harvest,.
and their story changes at that point..
But it's this idea of gleaning in the harvest.
that I actually wanna unpack and talk with you.
a little bit more about today..
The concept of gleaning in that time,.
about a thousand years before Christ,.
it was almost like the Hebrew social welfare structure.
of the day, okay?.
It was something that was written in the law.
so that those who were poor or the impoverished.
or the vulnerable or the outcasts of society,.
they would be able to come alongside in the harvest.
and actually receive something from the harvest..
You can imagine in a town like Bethlehem,.
the harvest was significant for that town,.
and the landowners were the wealthy people,.
and the harvesters were hired to come.
and harvest the crop of that land..
Well, gleaning was something that was constructed.
to help the poor people also partake in the harvest..
Let me explain how this works..
So the harvesters would take a basket.
like this wick of basket I'm about to put on my back..
They would put it on their back just like this..
And what they would do is they would stand.
at the beginning of a massive grain field..
Remember, they're harvesting corn..
And they would stand in this massive grain field..
There'd be a whole line of harvesters.
all with baskets on their backs..
And what they would do right at the same time,.
all of them in one go, they would step forward..
Now, remember, there's a massive field in front of them..
There's a lot of grain to harvest,.
and they have a short period of time in daylight to do so..
So this was a fast thing that they would do..
They would step forward..
They would grab the first stalk of grain in front of them..
They would uproot it from the ground..

$^{281}$They would sheaf off all of the leaves from it..
They would manage to get the husk of the corn from it,.
and they would throw the husk of the corn into the basket,.
drop the stalk, and step forward once again..
Pull up the next stalk, sheaf it all off,.
take that husk of corn,.
throw it into the back of the basket, move on to the next..
Do you understand that?.
And the faster that they could do this, the better it was..
So for those over here, they would be like,.
(imitates fast-forwarding).
like this, right?.
They were like ninjas when it came to harvesting corn..
It was awesome..
Now, every once in a while, though,.
a harvester would sheaf off the corn husk.
and throw the corn husk into the basket..
But because of the speed or maybe the movement,.
the corn husk would miss the basket and fall on the ground..
Now, when the corn fell on the ground,.
by law, the harvester was not allowed.
to pick up that grain anymore..
When the corn husk would fall onto the ground,.
you had behind the line of harvesters another line..
This was the poor people..
This was Ruth..
And they were behind the harvesters,.
and they were desperately watching the harvesters.
to see which one is likely to drop some corn on the ground..
Now, you can imagine when a harvester drops a corn.
on the ground by mistake and realizes it.
and realizes they can't pick it up,.
they move on, and suddenly,.
you can almost hear all of the gleaners go,.
(gasps).
and there's like this fight for that one piece of corn.
that's fallen on the ground..
And all the gleaners would run forward,.
and somebody would grab that precious bit of corn..
And as soon as they grabbed that corn,.

$^{321}$that was their property now,.
and they could take that home,.
and they could make some bread for themselves.
and hopefully survive for another week..
Gleaning was the process of taking the tidbits,.
the crumbs of the harvest,.
and making a meal out of it so you could actually survive..
Are you following this?.
So if you think about it this way,.
gleaning was like living off of the leftovers.
of the others who were harvesting..
Now, take this back off, put it back over here..
What has this all got to do with us and intimacy?.
I wanna put it to you that I think for the majority of us,.
when it comes to our intimacy with Jesus.
and when it comes to our spiritual life with Him,.
our posture is primarily one of gleaning.
rather than harvesting..
And what do I mean by that?.
What I mean is that I think so often.
in the Christian life,.
and this is the big trap that we fall into..
We primarily base our intimacy off of the throwaways.
of somebody else's intimacy..
I think we so often construct our faith,.
our knowledge of Jesus and who Jesus is.
off of somebody else who's done the work,.
the harvesting of understanding and knowing Jesus,.
and we glean from their insight,.
and we take the gleaning of their insight,.
and we build our own intimacy,.
not actually off of the person of Jesus Himself,.
but off of the gleaning that's come.
from somebody else's relationship with Jesus..
Anyone tracking with this?.
This process happens in the spiritual life..
I think it's so easy for us.
to glean other people's information..
And we can find ourselves actually beginning to believe.
in a Jesus that other people have told us about.

$^{361}$rather than a Jesus that we personally.
encountered for ourselves..
So in other words, our experience of Jesus.
becomes somebody else's experience of Jesus..
Our belief of Jesus is really somebody else's belief.
of Jesus..
Our relationship with Jesus has predominantly.
been somebody else's relationship with Jesus,.
and here's what happens..
When your primary posture in your relationship with Jesus.
is to glean from other people's information about Jesus,.
here's what you will do..
You will construct a version of Jesus.
that I like to call secondhand Jesus..
You're gonna make a secondhand Jesus..
And here's what you need to hear..
The single greatest danger to intimacy with God.
is the worship of a secondhand Jesus..
The single greatest danger to a church.
is the worship of a secondhand Jesus..
The Jesus somebody else told me about.
rather than the one that I actually know..
The Jesus of somebody else's information.
rather than the Jesus of my revelation..
The Jesus that has been constructed by something else.
that I've managed to glean myself from.
and I've aligned myself to,.
and the reality is we all do this..
Now, I want you to hear me really carefully..
Can everybody lean in?.
Hear me really carefully..
Gleaning is good..
Gleaning is actually an important part.
of your relationship with Jesus..
In fact, I tell you what,.
God has raised up some incredible men and women in the church.
through church history.
that have so much to say that you can learn from, grow from..
There is absolutely nothing wrong.
with downloading a sermon from some preacher.

$^{401}$that you love overseas.
and get some insight into God from that..
There is absolutely nothing wrong.
with reading devotionals every day this week..
Hopefully if you're a part of the Vine,.
you've been reading a devotional.
that somebody else has read..
That's the process of gleaning..
I want you to hear my heart..
There is nothing inherently wrong with gleaning..
That's not what I'm saying here..
Here is what I am saying..
When the posture of your relationship with God.
is built primarily.
on what other people's information is about God,.
if your approach to your relationship with God.
is primarily structured.
by what other people have told you about God,.
you are essentially worshiping a secondhand Jesus.
and you have what I would call a borrowed faith..
Does that make sense to you?.
Now again, gleaning is not bad..
Here's the reality..
I have made my whole life purpose to read the scriptures,.
learn the scriptures, to harvest.
so that other people in my church could receive and grow..
Every single one of us in this room right now.
is gleaning right now..
Yes, you are..
You're listening to me..
And my prayer and hope is that.
you'll glean something good today..
I pray that you'll hear something that'll be helpful to you..
I pray that there's some sort of impartation.
that will happen through what I'm sharing today.
into your life..
But hear me, I would more and much more rather have me.
as a catalyst to your personal faith.
than you basing your faith out of everything I say..
If you have structured your faith.

$^{441}$around what Andrew says at the Vine.
or any pastor at the Vine or any church.
or any sort of ministry,.
if that's the primary way.
that you're coming to understand your faith,.
you are worshiping a second-hand Jesus.
and you have a borrowed faith..
And that is not intimacy..
That's information that may or may not be helpful.
to catalyze you towards intimacy,.
but we can't make the mistake of thinking.
that that is intimacy itself..
Are you following this?.
There are two things that will happen, I guarantee you,.
if you focus in on gleaning.
more than harvesting in your relationship with Jesus..
Two things guarantee what happened..
Here's the first one..
Not only will you construct a second-hand Jesus,.
but you will construct a second-hand Jesus of your choosing..
'Cause the whole process of gleaning.
is picking up stuff that we like..
When our primary posture in building an intimacy with God.
is based off of what other people have said,.
what we do is we align ourselves.
to the things we like that other people have said..
Oh, that was a good quote from social media..
Oh, that was a bomb message there..
Oh, that was good..
And then we like those things.
because they sit well with our already confirmed worldview..
They sit well with how we already think we know about Jesus..
And when your life, your relationship,.
is primarily structured by gleaning rather than harvesting,.
what it means is that you will cherry-pick.
the things about Jesus that are comfortable for you to have..
This is the sifting that I talked about.
in the picture I got during my worship..
I want you to know this..
If that happens, here's what's gonna happen..

$^{481}$You will have a Jesus that you like,.
but not a Jesus that saves..
You will have a Jesus that you're comfortable with,.
but not a Jesus necessarily of the Scriptures..
And when your life gets tough,.
when things are hard and challenges come,.
you'll find yourself not being able to lean.
on your constructed version of second-hand Jesus..
Here's the second thing that will happen, guarantee you..
The second thing that will happen is this..
You will begin to mistake rumor of God.
for revelation of God..
This is really an important thing,.
and I want you to track with this..
When our posture, again, is primarily gleaning.
rather than harvesting ourselves,.
what we're really doing is we're building our lives.
off of other people's opinions about God,.
other people's ideas about God,.
other people's maybe even experiences of God,.
and we're gleaning from that..
What you need to realize is that.
when you glean from someone,.
the chances are what you're gleaning,.
they've gleaned from someone else..
And the chances are that that other person.
that they gleaned from gleaned from someone else..
And so on and so forth..
Have you ever played that game where maybe if we started.
with a whisper here, and I whispered something in your ear.
and you pass it all the way down over here,.
when it got down to here, it would not be the same..
Are you with me?.
Do you ever play that game?.
When your posture and your intimacy with God.
and relationship with Jesus is built primarily on gleaning.
rather than harvesting, you're gleaning something.
that's been passed down, passed down,.
and suddenly you're gonna have the mistake,.
there's the challenge, there's the temptation,.

$^{521}$to believe that the rumor of God.
is actually revelation of God..
But actually that rumor of God much more likely.
is something that's been filtered over so many things..
And the Bible's very clear,.
the truth that God is Jesus himself..
Jesus is the way, the truth, and the life..
At my Bible school in seminary,.
I went to Bible school and gleaned a lot, it was great..
Over the top of the door of my Bible school,.
when you walk into it, for the first time it says this,.
it says, "God has spoken, the rest is commentary.".
And the commentary is helpful and good..
The commentary can catalyze and strengthen our faith,.
absolutely, but we need to know that God speaks..
The truth is alive..
And the danger we have if we're posturing,.
mostly we're gleaning rather than harvesting,.
is that we'll end up turning rumor of God.
into revelation of God..
If you want to truly deepen your intimacy with God in 2024,.
what you need to do is trade rumor of God.
for a firsthand faith..
Come on church, everybody okay?.
You gotta start trading rumor of God for firsthand faith..
You need to be serious.
and take your own spiritual basket.
and realize that because of Christ's life,.
death and resurrection on the cross,.
there is a harvest of his relationship with you.
that is right in front of you,.
that there's a harvest of goodness and life.
and glory in him..
And you are to step forward.
and grab a hold of the spiritual stalks.
that he places in your life and sheath them off.
and grab the kernel of the word of God for you.
that changes and transforms you.
and place it in your spiritual basket.
and then step forward to the next thing.

$^{561}$that God has for you and unpack that.
and harvest the things that God has for you.
rather than stand back and wait for that person.
that you think hears more from God than you do.
to drop a little morsel of idea that you can suck up..
Instead, you are designed and created for the harvest..
You're a child of God..
You've been given the ability and the privilege.
to strap a basket of spirit on your back.
and harvest in intimacy with him..
Now,.
the thing that we have to sit in,.
thing that we have to wrestle with.
is that harvesting is work..
One of the people that I've followed.
that has written a lot about this,.
his name's Glenn Packiam, he leads a church in California..
He's written around this idea and this topic..
I wanna read you something that he said..
He said, "God didn't want me leaning completely.
"on someone else's knowledge of him..
"He didn't want me coasting,.
"finding shortcuts for knowing him..
"He wanted me to do the carrying, the heavy lifting,.
"the hard and delightfully glorious work of knowing him.".
I love how he puts that..
The hard but delightfully glorious work of knowing him..
In other words, there's work to be done..
Hear it this way, gleaning is easy..
Gleaning is simple..
You don't need any special gifts to glean in God..
You can just pick up what other people.
have done the hard work of uncovering for themselves..
It's much easier for you to go on a run.
and listen to a spiritual podcast than it is to pray..
It's much easier for you to listen to Nicki Gumbel.
with a Bible in a year.
than for you to actually open the Bible and read it..
I love Nicki, by the way..
Don't think I'm criticizing Nicki..

$^{601}$I love my man, Nicki, okay?.
But I'm making a point..
You get the point?.
See, here's the issue..
We live in Hong Kong,.
and Hong Kong is one of the busiest places.
and cultures to live in..
In other words, your time is the biggest resource.
that you have, the most important thing in your hand..
And if your approach to intimacy with Jesus this year.
is gonna be based on whether you're busy or not,.
let me just tell you, you're gonna be busy this year..
And when you're busy, you will naturally default.
to what is easiest in your spiritual life..
And what is easiest in your spiritual life.
is gleaning rather than harvesting,.
'cause harvesting's gonna take some work..
Harvesting's gonna require you to put some time aside.
and open the Bible..
Harvesting's gonna require you to pray..
Harvesting's gonna require you.
to roll up your sleeves a little bit.
and prioritize what it is.
to take the spiritual things in your life.
and wrestle with them and pray with them with God,.
share them with others, get their input and help..
That's the gleaning part that can help within it..
But the harvesting is work..
If you don't wanna work, then just glean..
But if you're gonna glean predominantly,.
you're gonna struggle in intimacy with Jesus..
When we harvest, what we're doing.
is we're posturing ourselves for His revelation..
And gleaning, although it's so easy,.
only ever leads us to a place.
where we'll fall shallow and empty..
You were designed for the harvest..
Jesus died for that for you..
Don't let the enemy get you so busy.
that you find yourself constantly defaulting.

$^{641}$to somebody else's revelation rather than getting your own..
Don't allow yourself to be so lazy.
that you're just surviving off the throwaways.
of your pastor rather than actually investing.
in the scriptures for yourself..
You need to move from rumors of God to a firsthand faith..
Here's something you need to know about rumors..
In the absence of revelation, rumors grow..
In the absence of revelation and truth, rumors grow..
And when we keep God at an arm's length,.
when we keep Him at a distance to us,.
when we hold Him away from us.
and we just consume what other people.
are experiencing with Him.
and not walking into that for ourselves,.
we're much more vulnerable to rumors of God,.
much more vulnerable as people.
to what everybody else seems to have a perspective.
and opinion, and we end up shaping and making life decisions.
on somebody else's opinion of God..
This was really important to Jesus..
Jesus, throughout the Gospels,.
wrestled with this with His disciples..
And there's this beautiful moment in Mark chapter eight..
Is this helping someone?.
You guys okay?.
In Mark chapter eight,.
Jesus begins to shift the focus to His disciples.
because He wants them to move from rumor of God.
to revelation of who He is..
And so let me pick this up from verse 27..
Jesus and His disciples went on to the villages.
around Caesarea Philippi..
On the way, He asked them, "Who do people say I am?".
What a great question..
He says to the disciples, "What are the rumors about me?.
"What does everybody else think?.
"What does everybody else think about me?.
"Those that are not like you,.
"who are walking day in, day out with me,.

$^{681}$"those that are in the crowds at a distance,.
"what do they think that I am?".
And the disciples knew, they said this,.
"Well, some say you're John the Baptist..
"Others say Elijah, and still others, one of the prophets..
"They say, 'Here's what the rumors are about you..
"'Here's what people are convinced that you are..
"'I mean, they've heard things about you.'.
"And so they've just said,.
"'Well, that must be John the Baptist.'.
"Or they think you're Elijah,.
"or they just think you're one of the great prophets.
"because they've heard of what you did.
"and the healing that happened over here..
"They just heard about it.".
They didn't witness it, by the way..
"They just heard about it..
"And so they defined you based on the rumor.
"of what they've heard..
"So they say to Jesus, 'Here are all the rumors.'".
And guess what?.
All the rumors were wrong..
And so then Jesus says what I think.
is one of the most critical things..
He says, "But what about you?.
"Who do you say I am?".
For some of you at the beginning of 2024,.
that is the most critical question.
that Jesus is asking you this year..
In a year of intimacy,.
he's starting with saying, "Who do you think I am?".
Forget what Andrew thinks..
I mean, don't fully forget what I think..
I hope you come here and learn some things, okay?.
But yeah, forget what Andrew thinks..
Forget what the vine says..
I'm more interested, Jesus says,.
about you and your heart..
Who do you think I am?.
And here's Peter's response,.

$^{721}$perhaps one of the most profound things.
in the New Testament..
Peter answered, "You are the Messiah..
"You're not John the Baptist..
"You're not some Elijah..
"You're not just simply a prophet..
"You're the son of God..
"You're the Messiah..
"You're the one that Israel has been longing for..
"We have been with you day in and day out..
"We're not existing off the fumes of some rumors.
"and trying to make some meal of faith.
"off of the tidbits and throwaways.
"and the leftovers of somebody else..
"We've walked with you..
"We've been with you..
"We've seen the miracles..
"We've heard your teaching..
"We've not always been comfortable with it..
"We've struggled with it..
"We've been confused at times,.
"if I'm honest with you, Jesus..
"But I cannot say, I cannot deny.
"that after all of that relationship with you,.
"you are not Elijah..
"You're not John the Baptist..
"You are the Messiah..
"You're Jesus..
"That's who I know..
"And that doesn't come to me.
"because someone else has told me," Peter is saying,.
"that is my revelation from my intimacy with you..
"You are the Messiah.".
Peter profoundly moves from rumor of God.
to firsthand revelation and declaration..
And on the back of that revelation,.
Jesus changes everything in his ministry..
He starts heading towards Jerusalem.
until the death on the cross.
because somebody now has got it..

$^{761}$Somebody has connected with him..
And you need to understand that when we exist on gleaning.
rather than harvesting,.
when we make our diet predominantly rumor.
rather than actually revelation with him,.
here's what rumor will do..
Rumor will tell you the what of God,.
but it can never tell you the who of God..
It'll tell you what about God,.
but it can't tell you the who of God..
And here's the thing I really want you to understand..
If you really base your relationship with God.
only on rumor of him, you will find yourself,.
this is really key,.
you will find yourself knowing what God is,.
but not who God is..
You will find yourself with other people's information,.
but not your own revelation..
You will find yourself with knowledge, but not intimacy..
You will find yourself with knowledge,.
but not necessarily intimacy..
The work of intimacy is moving from the what of God.
to the who of God..
It's moving away from rumor of him and information of him.
to firsthand faith, firsthand encounter..
And I wanna just say.
that one of the greatest fallacies in the church,.
and I think we struggle with this here in Hong Kong.
quite a bit, is that we have this misperception.
that those who are really knowledgeable of God.
are also really intimate with God..
Let me preach on this one for a moment..
We have this misperception, it's a fallacy actually,.
that those who appear really knowledgeable of God.
are therefore very intimate with God..
And trust me, that is not always the case..
But the problem that we have.
is that we see somebody who's super knowledgeable of God,.
and we know that they're knowledgeable,.
and we've seen them speak things.

$^{801}$that have really changed our lives..
And we look at this person and we put them on a pedestal.
and we say, "This person really knows God..
And if I can just follow this person,.
if I could just follow this pastor.
or this church or this ministry,.
if I could align myself to that,.
and if I could just pick up the scraps.
of their revelation with God, I tell you what,.
my gleaning of that person.
because of their knowledge of God.
is way better than the harvesting that I could ever have..
So I would rather align myself to glean from them.
than harvest myself.".
We think this way in the church..
And Jesus said, "Follow me.".
He didn't say, "Follow that pastor, follow that church.".
Now again, okay, it's good to follow churches,.
good to follow pastors, okay?.
Everybody with me?.
It's a preaching technique called taking you to the extreme.
to help bring you back to the middle, okay?.
(congregation laughing).
Don't misinterpret what I'm saying,.
but I hope you hear what I'm saying, right?.
That Jesus says, "Follow me.".
And we follow Christ because we are in the harvest field.
as a harvester, not a gleaner..
And some of you here,.
if you really want to move in intimacy with Jesus this year,.
it's gonna have to be a shift for you.
from a life that is predominantly building.
a relationship with Jesus.
off of somebody else's relationship with Jesus.
to a firsthand faith..
We have this thing where we think this,.
we think the one with the most knowledge.
has the most access..
But you need to understand that all of us have equal access.
when it comes to the person of Jesus..

$^{841}$We think that the one with knowledge has more access..
Well, that must be how they got all their knowledge.
because they got this great access to Jesus..
And if only I had that access,.
the reality is every single one of us,.
whether you're a day old Christian.
or you've been a Christian for 50 years,.
has the same access to the person of Jesus Christ..
There is this beautiful part in the book of Hebrews..
Let me just read this to you quickly..
In the book of Hebrews, it says this, the writer,.
he says, "Therefore, brothers and sisters,.
"since we have confidence to enter the most holy place.
"by the blood of Jesus," notice this,.
"by the blood of Jesus," not by how much you know,.
"but simply by the blood of Jesus,.
"by a new and living way open to us through the curtain,".
that is his body,.
"and since we have a great priest over the house of God,.
"let us draw near to God with a sincere heart.
"in full assurance of faith.".
Let every single one of us draw near to God.
out of full assurance of faith..
Why? Because you're smart? No..
Why? Because you know more than someone else? No..
Why? Because you've got a better experience of God.
than someone else? No..
Simply because the blood of Jesus.
has been shed on the cross,.
and you now, because of the life, death,.
and resurrection of Jesus, have your sins forgiven,.
which means you're redeemed and you're renewed,.
you now have Jesus standing at the right hand of the Father.
interceding on your behalf, you have access..
You're not a gleaner..
You haven't been, put it this way,.
Jesus did not go to the cross.
so that you would glean somebody else's intimacy..
Jesus went to the cross to pay a very high price.
so that you would have the joy and the wonder.

$^{881}$of a relationship with him..
Never forfeit the privilege of your relationship with God.
by outsourcing it to someone else..
Never forfeit the privilege of your relationship with Jesus.
by outsourcing it to someone else..
Can you learn from others? Absolutely..
Is it good to glean? 100\%..
Are there amazing people in church history.
that have a lot to teach us?.
Yes, yes, yes..
But intimacy comes from trading rumors of God.
to firsthand faith..
There's a book in the Old Testament.
that preaches what I'm telling you today.
over the whole book, it's the book of Job..
Job also, like Ruth, a story of a family,.
a story of a family that begins and ends,.
or begins and they go into immediate,.
complete travesty as a family..
Job is left on his own..
And interestingly, from chapters two.
all the way through to chapter 37,.
Job is surrounded by three friends.
that have a lot of rumors of God..
Three friends that show up and say,.
"Maybe this is why this is happening..
"Have you thought about this?.
"Did you know this about God?.
"Have you repented of this?.
"Has this happened to you?".
Three really well-meaning friends.
with lots of ideas and opinions and thoughts of God..
And every single step of the way through the chapters,.
Job is more and more confused about who God is,.
why this has happened, and the rumors are not helping him..
He's starving despite an influx of rumors of God..
But in chapter 38, God breaks through it all and speaks.
and pushes the three friends to one side.
and he speaks to Job..
And he takes four chapters and he says,.

$^{921}$"You wanna know who I am?.
"This is who I am..
"This is really who I am..
"And this is how I operate..
"And these are the things that I do..
"And this is my glory..
"And this is who you are in relation to me.".
And he lays it all out for four chapters..
And in chapter 42, finally, Job now responds to God..
I wanna read you how Job responds..
I'm gonna read from Eugene Peterson's version.
of the message..
It says this, "I admit I once lived by rumors of you,.
"but now I have it all firsthand.
"from my own eyes and ears..
"I'm sorry, forgive me..
"I'll never do that again..
"I promise I'll never again live on crusts or hearsay,.
"crumbs or rumors.".
The NIV writes it this way in verse five..
"My ears had heard of you, but now my eyes have seen you..
"I had the rumor I could hear things about you,.
"but now I've encountered you for myself..
"I am not a gleaner, I'm a harvester..
"And that has changed everything in me.".
Intimacy is found when we shift from rumor to revelation..
Intimacy is found when we use the gleaning in our life.
to catalyze a deeper, more effective harvesting of Him..
That's what He's opened up for us..
It's what He's opened up for you..
And as you start this year with a call on you.
to intimacy with Him,.
it has to start with an honest reflection..
Am I worshiping a secondhand Jesus?.
Do I have a borrowed faith?.
And while I'm not gonna get rid of all of that stuff,.
'cause it is all helpful,.
am I predominantly primarily building my relationship.
with Jesus off of my relationship with Jesus?.
Am I a harvester or am I a gleaner?.

$^{961}$My prayer for you is that you would hear Jesus say,.
"The harvest is plentiful, but the workers are few.".
And I wonder whether we would grab a hold.
of the harvest this year in our intimacy with Him..
Amen?.
Can I pray for you?.
Can we stand together?.
Let's pray..
And as you stand, I want you just to.
quieten down your heart..
Maybe let's just open our hands.
if you're comfortable to do that..
Let's come into a posture of.
preparing ourselves to hear from Him..
I've done a lot of talking..
I'm gonna shut up in a moment.
because really I don't want any more gleaning here..
I want some harvesting..
And if this message has been helpful to you,.
there's been some good gleaning for you.
out of our time together this afternoon..
Now's the time for you to move towards harvest with Him..
And I wanna create some space for you to pray with Him,.
to spend time with Him,.
to listen to Him, to His voice for you..
And some of you, it might be this idea.
of a secondhand Jesus,.
and realizing that it's been a while.
since you've actually spent just time with Him yourself..
Put away the helpful things,.
but come just rawly in prayer..
Open the scriptures for yourself to read.
and ask the Holy Spirit to give you revelation and word..
For some of you, it might be that you've.
taken room for revelation..
For some of you, it might be.
that you found yourself worshiping knowledge over intimacy,.
that you've been quite proud of what you know of God,.
and that maybe you've held onto that pride,.
but you realize that you actually,.

$^{1001}$if you're honest with yourself,.
you don't really know Him personally..
Maybe you've got some great theology,.
but you don't know Jesus truly..
It'll be different for all of us..
Take a time, take some time to listen to His voice for you..
He loves you so much..
He draws near to you..
(gentle music).
[MUSIC].
\newpage



\section{}
\label{sec:VvUHDD1RwPI}
\textbf{2024-01-22 What Stops Us: The Things of God [VvUHDD1RwPI].mp3}
\newline
\newline
連結: \href{https://youtube.com/watch?v=VvUHDD1RwPI}{\texttt{ https://youtube.com/watch?v=VvUHDD1RwPI}} ~~~~ 語音日期: 2024-01-22 
\newline
\newline
\hyperref[sec:8XDQRPVQTPs]{\small{< < < PREV SERMON < < <}}
~
\hyperref[sec:index]{\small{[返主目錄]}}
~
\hyperref[sec:J4wOHcthlVk]{\small{> > > NEXT SERMON > > >}}
\newline
\newline
$^{1}$Everyone says, "Amen." Hey, can we thank our worship team as always? Thank you guys..
Have a seat, have a seat. We are absolutely over full. We've had to close.
the service and not let anyone more in, but we do have one or two space. So if.
you've now sat down and there's a space next to you, could you just put your hand.
up? There's a whole bunch of people standing around here that would like to.
get a seat. There's one seat right here. This is the upgrade seat. It's like.
first-class for you. Yes, there's a seat there. Come on in, come on in, have a seat.
right here. I'm trying to sell you my seat. There you go. It's a nice seat. There.
you go. Any other seats around, please put your hand up if you have a seat. We want.
to make sure that we can get as many people in as possible. Just a reminder,.
once again, welcome to our 11 o'clock service. If you're relatively new to the.
Vine, we're so glad you're here. We do have a 9/15 service and a 2 o'clock.
service. There's a bit of room in those services, so we encourage you, if that.
works for your timing, to come to those services. We understand that there was a.
marathon today. Anyone run in the marathon today? Come on, yes! What did you.
do? Did you do like a 10k? 10k? Awesome. Anyone do a half marathon in the room?.
No, you're dead. You do the half marathon? Oh, you did one before? No, that doesn't.
count. Well done to you. All right. Well, my name is Andrew. Like I said, I'm one of.
the pastors, and it's great to have you with us. We have sent, at the start of.
this year, a call from God towards greater intimacy with Him. The sort of.
idea that, as much as we love gathering like this on a Sunday and coming into.
God's atmosphere and His presence here, and we'll continue to do this on a.
Sunday, that God has actually so much more for us. That coming into a community.
like this, into the atmosphere of God in a room like this, is not the end point of.
what God wants for you. It's actually the catalyst for you to push beyond just a.
feeling of God into a deeper and more personal intimacy with Him..
And that's everything that's on our heart for the Vine this year. And so this.
is your home church. If this is what you're a part of, we want to help.
facilitate in our times together this movement from just the atmosphere of God.
to a living and active faith with the very person of God. And last week, in.
week two, I mentioned that there are two critical traps. Really important you hear.
this. Two critical traps that Christians fall into when it comes to the intimacy.
of God, and that actually limits their ability to be in intimacy with God. Last.
week I spoke about the first of those traps, and it's the trap of building our.
relationship with God primarily on other people's opinions, other people's.
perspectives, other people's ideas, even other people's intimacy with God, that we.
end up building our relationship with Jesus not on a direct relationship with.
Him, but on everybody else's relationship with Him. Does that make sense? And what.
happens is we end up building this relationship with Jesus off of everybody.
else's perspectives and opinions by what we read, what we hear, what we download,.

$^{41}$what we whatever, and those things are good, of course. There's nothing wrong.
with learning from others. So much of our Christian faith is about learning from.
others, but when that is the primary posture of our relationship with Jesus,.
that we're only building a relationship with Him based on what everybody else.
says, then what ends up happening is what I said last week. You will end up.
worshiping a second-hand Jesus. You will end up having a borrowed faith. You're.
worshiping somebody else's experience of Jesus. And if we truly want to move into.
deeper intimacy with Him this year, what I said last week is resonant for us this.
week. We have to trade rumor of God for firsthand faith. See, rumor of God, other.
people's perspectives and opinions, is not intimacy, and we need to move from.
just rumor of God, as helpful as that can be for us, and as helpful as it can be.
for our theology, and great that it can be for us to learn and grow in Christ..
All of that is good, but we need to move from just rumor to firsthand faith. That.
was last week. That was the first critical trap. I want to talk today about.
the second critical trap that has the power to limit your ability to be in.
intimacy with Christ. And to introduce that to you, I need to talk to you about.
a funeral. I came on staff here at the Vine 15 years ago, and I came straight.
out of Bible school. I'd studied in Bible school. I'd had a career in banking prior.
to going to Bible school. At the age of 30, I went to Bible school for four years,.
and then off the back of Bible school, I came here to the Vine and became a part.
of the pastoral team. And of course, back in those days, there was other senior.
pastors. Senior pastors John and Tony were leading the church at that time. In.
about the first month that I am on staff, an emergency funeral comes up. There is a.
shocking death within our community. Somebody passed away that nobody.
expected would happen, and it meant that we had to do this sudden funeral for.
this family. And it just so happened that John and Tony were not available to.
attend or to run this funeral, so the funeral came to me. And to tell you that.
I was terrified is an understatement, okay? Bible school is great. Bible school.
gives you a lot of information about God. It helps you to study the Word of God. I.
love my time in Bible school. What Bible school doesn't do is help you to prepare.
for a funeral, okay? Some people feel like it's their funeral at Bible school, but.
anyway, that's a whole different story. But it doesn't help prepare you to help a.
family walk through some of the hardest times of their lives. And so I was.
terrified. I was so green in my experience in this area, but I did all.
the right things, right? So I prayed about it. I met with the family, and we grieved.
together. We talked about how they wanted the memorial to go, the flow of the.
memorial. On the day of the memorial itself, my heart was beating really hard..
I was super sweaty and nervous. I was praying that only 20 people would show.
up, so it'd be nice and small and easy to handle. Over 300 people came to this.
particular memorial. This guy was well known in his company. He was.

$^{81}$well known in Hong Kong. And so there was a lot of people there at the memorial,.
and I did a terrible job through the memorial. But it got to the point of the.
eulogies, and we had arranged that four people were going to give a eulogy about.
this person. And those four people were friends and colleagues of him and family.
as well. And so each person got up and did what a eulogy is supposed to do,.
right? They get up, and they talked about their relationship with this person. And.
as people were talking, and they were so beautiful, these eulogies, so articulate..
Every single person just did a phenomenal job in doing these eulogies..
But I noticed along the way that there was a pattern to what I was hearing in.
the eulogies. And the pattern was that everybody was speaking about all of the.
things that this person had done for them. And they were just so overwhelmed.
with gratitude for what this person had done in their lives. For some, it had been.
that he had been their mentor in the office and had helped them to get.
promotions through the company. For others, it was generosity. This person had.
been super generous to them, and they had received that generosity and were so.
grateful for it. For others, it was just the benefits of receiving hospitality in.
their home and welcoming them into the home. Every single person talked about.
all the amazing things that this guy had done for them, had benefited them with,.
had blessed them with, or some of the crazy great achievements that he had.
done in the world. And I sat there, and I thought to myself, "This is what I want at.
my funeral." I thought, "How amazing would it be if whenever it is that I die, that.
a whole bunch of people gather in a room and talk about all the.
things that I had done for them, all the things that I had achieved for them, all.
the things that I had blessed them with, that I had done for them on their.
behalf, my generosity, my love, my support, my prayers, whatever it might be. I was.
like, "This is what surely you would want at a funeral." Well, after the four eulogies.
had finished, the wife gets up. And as she gets up and walks towards the.
microphone, the whole room was just totally silent. And there was this huge.
amount of respect in the room for what this woman was going through. And she.
gets and stands there in front of everybody, and for ages she doesn't say.
anything. She's just overwhelmed with emotion, which you can understand. And you.
can see the emotion in her face. And this went on for quite some time, and I was.
at the point where I was thinking, "Do I need to get up and come alongside.
of her and help her?" And just as I'm kind of thinking that maybe that's what I.
need to do, she moves her hand up towards the microphone, and her hand is shaking.
because she's so emotional. And the sense I had was she had a prepared speech, but.
she doesn't want to give the prepared speech. And she grabbed the microphone,.
and she said this. She said, "I just miss him. I just miss him." And it was like the.
perfect thing to say, because everybody else had said that they had missed what.
he had done for them. They missed his hospitality, or his generosity, or the.

$^{121}$fact that he was a mentor for them. And no doubt that this woman had received.
more from her husband than anyone else. She had received more benefit, more.
provision. He had done some amazing things for her and her family, I'm.
absolutely sure. But she stood before everybody, and she said, "You want to know.
the one thing? I just miss him. Not the things of him, not the activities.
that he did, not all the great achievements that he made in his life. I just miss.
being with him." And in that moment, I saw a picture of what true intimacy is,.
because everybody else in that room had a relationship with this person, but that.
room, that woman in the room, she was the one who was intimate with him. And she.
just missed him. And as we start this year of intimacy here at the Vine, I.
think there's an absolutely critical question that we all need to sit with.
and wrestle with, and it's this, that in your relationship with God, is it built.
on the things of God, or on the person of God? Say that again. Is your relationship.
with God, as you start this year, largely, primarily built on the things of God, or.
on the person of God? Because let's be honest with ourselves, the wonderful.
thing about being a Christian is that we get to worship our God. That he's a God.
who's a good father, who loves to pour out his blessings on his people. He loves.
to fill us with his gifts, and his spirit, and his love, and his goodness. We just.
sung that song. He's been good to us. We love the reality that we worship a God.
who, out of the kindness and the generosity of his heart, wants you to.
flourish as best you can in your life. He wants to come along and bless you,.
and secure you, and grow you, and do all the things that he can do to make your.
life the best thing that it can ever be. But here's the reality, when that is so.
much the focus of our relationship, that we build our relationship with God on.
the things that he does for us, you put yourself at the center of that.
relationship. It's then becoming a little bit more about you than it is about him,.
and secondly, your emotions are actually now linked, not to the person of God, but.
to what he does for you. Follow this. I found this so much in my life. There was.
a season where I realized that if God was blessing me, if God was answering my.
prayers, if he was coming through for me in the ways that I was hoping he would.
come through for me, then I felt loved and accepted by him. But in the seasons.
where it goes a bit silent, in the seasons where the blessings weren't.
there, in the seasons where it felt like God was quite distant from me, I was.
disappointed and angry, and I felt lack of intimacy with him. Are you with me? And.
the reality is, for all of us, when we're in this kind of approach in our.
relationship with God, we have to wrestle with the reality that so often we.
actually build the foundation of whether we're loved or accepted by God by what.
he does for us, rather than just purely who he is. And what ends up happening for.
all of us is that our intimacy with God is actually not an intimacy with him, but.
it's an intimacy with the things of him. When your intimacy is built on the.

$^{161}$things of God and not God himself, that is idol worship. God had to come to me a.
few years ago in one of my quiet times and say this to me. He said, "Would you.
rather have me or my blessings?" I think that's a deeply challenging question..
Would you rather have me or my blessings? And I find this deeply challenging.
because I know from my own personal life that I am attracted to the burning bush,.
but often I don't see beyond the flames to the God behind the bush. I'm attracted.
to the benefits of God. I'm attracted to the things of God, what God can do for me,.
the blessings that come upon me, the things that I know that his.
mighty hand can provide for me. I'm attracted to those things, and I find.
myself drawing near to God because of my attraction to the benefits of being in.
relationship with God, but I find myself rarely taking off my sandals and stepping.
on the holy ground and communing with him. And when that's the case, we've got.
an issue. And I think this is not just my problem, I think it's our problem. And I.
think in a year of intimacy with him, we have to wrestle with this reality deep.
inside each one of us. It's easy for us to be attracted to the benefits of God,.
but when your primary intimacy with him is based out of what he does or doesn't.
do for you, your relationship with him will be like this all the time. And Paul.
would write to the church, and he would say, "I know what it is to be hungry, and I.
know what it is to be fed. I know what it is to be shipwrecked, and I know what it.
is to be safe. I know what it is to be wealthy, I know what it is to be poor. I've.
experienced the blessings, and I've experienced the seasons without the.
blessings, but I have learned the secret of being content. No matter what might be.
happening, no matter what God might be doing, no matter what blessings or not.
blessings might happen in my life, I've learned what it is to be intimate with.
him because I'm not basing the relationship that I have on Jesus by.
what Jesus does for me." I think we all have to recognize that the reality for.
many of us is that we want more from God, but not necessarily more of God. Come on.
church. Hi. Everybody okay? We have this thing where we want more from God, what.
he can do for us, but we don't necessarily want more of God, our.
relationship with him, and we're happy to have all the benefits of relationship.
with Jesus without any of the responsibility. A few years ago, my.
absolute singular obsession in life was to get to silver class on Cathay Pacific..
It was my absolute obsession to get from the green card that you get when you.
just buy a ticket to the silver card. Oh sweet Jesus on high, if you could just.
get me lounge access, I would love Cathay for the rest of my life. That would be.
amazing. Now I recognize some of you in this room watching online in the.
overflow, you have diamond status with Cathay Pacific. I recognize that. You have.
diamond status, yes, because you fly a lot, but you also have diamond status because.
you probably work for a large multinational company that buys your.
tickets in business class at the highest amount of price possible so that you can.

$^{201}$have the greatest amount of flexibility possible, which means you earn the most.
miles possible and the biggest sector points possible, and it's the sector.
points that actually help you to get the status. However, I work for a church, and.
you, if you tithe here at the Vine, you'll be very happy to know that if I could.
fly in cargo class, the church would put me in cargo class, okay? I am so far back.
in economy, they basically should just put me in a suitcase, okay? My secretary,.
Veronica, she is phenomenal at finding the cheapest economy class fares on Cathay.
Pacific that is absolutely possible, and from a governance perspective, for all of.
you, that's a really good thing, but what it means for me is that it's really hard.
to earn sector points and miles on Cathay Pacific when you pay peanuts, you.
get monkeys, okay? And so for two years, I had this obsession that I was gonna get.
myself to silver status, you know? And so every flight I took, whether personal or.
with work, I was on Cathay Pacific. I actually got one of the American Express.
Cathay Pacific cards so that every time I went to a restaurant, I earned more.
points. If you could have flown Cathay from Kowloon to Central, I would have.
flown Cathay Pacific. I didn't particularly like Cathay Pacific more.
than any other airline. Sorry, Ken. Ken's in the overflow. I love you, Ken. It was.
just, it was our national carrier. It was the easiest airline and it was having.
the best benefits, and I cannot describe to you what the moment was like when I.
got the letter in the mail with the silver card inside it. Oh, ladies and.
gentlemen, you know it comes with a little tag that you're supposed to.
put on your luggage. Nobody puts the tag. I put the tag on my luggage. I wanted.
everybody in the airport to know that I had earned my silver status on Cathay.
Pacific. I remember the first flight, the next flight I was on Cathay, I remember.
going to Kai Tak, no it was Chet Blackcock, I went to Chet Blackcock. I was there.
four hours early because I'm gonna make that buffet pay in the lounge. I went.
into that lounge with my silver card out like super proud. I'm like, where is the.
food? Give me the food. I dominated that food. I basically paid back for two years.
of flights on that buffet alone. Well, you can probably guess what happens. A year.
later, I haven't flown enough to keep my status on Cathay Pacific. A year later, I.
get a mail and the letter says, "Hey, congratulations, you are green card now.".
And I'm like, hang on a sec. After spending all those years and time and.
energy and everything I did on Cathay Pacific to get myself into the lounge.
with my silver card, then I didn't fly enough in that year post that to earn.
enough miles to stay in that status. And suddenly I dropped down to green and.
basically Cathay was dead to me. I wasn't interested in Cathay after that. I.
couldn't care less, whatever. They're not gonna treat me well, then I am not.
bothered. Here's the point. Some of you in this room, you are on a frequent flyer.
program with God and you're wondering why you're struggling to find intimacy.
with Him. Because your whole relationship with Him is based on the benefits of.

$^{241}$being in relationship with Him. And when those benefits are good, you're super.
happy. You're happy to fly God Airlines when things are going well for you. But.
when things don't go well, particularly when you thought you earned the miles,.
but the miles were not actually enough. When you thought you put on all that.
spiritual work to get yourself into a better relationship with Him. When you.
thought you had prayed harder for that thing than you had prayed for anything.
else and yet God didn't answer that prayer, you want a discount. You struggle.
and we struggle with this up-and-down relationship with God based surely on.
all of this process. And we have to recognize that when we base our.
relationship with God on the benefits of God, we've stepped into idolatry. I.
would say it this way. The greatest idolatry in the church today is the.
idol worship of the benefits of God. It is seeking God or pursuing Him primarily.
for what He can do for us rather than simply for who He is. It is mistaking.
blessing for intimacy. There's nothing wrong with the blessings of God. Don't.
hear me wrong today. There's nothing wrong with the fact that God loves to.
pour out His love. He wants to pour out Himself to you. His blessings are.
abundant for you. But when we get seduced by the blessings of God and we allow.
that to become the primary focus for us, the primary thing that helps us to feel.
whether we're in relationship with God or not, helps us to feel whether God is.
pleased with us or not. I'm gonna preach on this one for a second..
There's some of us here, and I put myself in this camp sometimes, where we.
actually put our whether God is pleased with us or not based on.
whether we receive His blessings or not. You need to understand that the blessings.
of God are not the marker of whether God is pleased with you, nor is it actually.
the proof that you're in a good relationship with Him. In fact, there are.
many examples in the Bible where because of God's generosity and because of God's.
love and because of how God sees His people, He's pouring out His blessings.
when His people are far from Him. When His people couldn't care less. When His.
people aren't in some intimate relationship with Him, God is still blessing them. We.
saw this recently in our Exodus series. If you're a part of the Vine over the.
last year, we went through the book of Exodus, and we saw there that there was.
this moment where because of their unfaithfulness, Israel's unfaithfulness,.
because they didn't trust God with the promised land, God made them wander in.
the desert for 40 years. God wasn't pleased with them in that moment. There.
wasn't this deep relationship that God had freed them out of.
their slavery for in that moment. Israel was walking in the opposite spirit. They.
were actually worshipping other idols in the land at that time. But every single.
day of those 40 years, God in His miracle working heart blessed them with manna.
from heaven that kept them alive. Every single day. Every single day He poured out the.
blessings of God on His people, even when their relationship was not in a healthy.

$^{281}$place. And some of us here, we are basing the health of our relationship with God.
on whether we're receiving His blessings or not. And if you do that, you're gonna.
have a very difficult time of being intimate with Him. There's gonna be some.
challenges for you, because there will be seasons where you will feel those.
blessings, but there will also be seasons when you don't. And your intimacy, and.
whether God is pleased with you, or whether even you're in a right.
relationship with Him, is not because He's answering your prayers, or He's.
giving you that promotion at work, or He's helping solve that thing in your.
family. It's simply because He loves you. He loves you. And your intimacy with Him.
cannot be based on what He does or doesn't do for you in your perspective..
It has to be based in who He is, regardless of what might happen for you..
This is not a new issue for the church. So if we are struggling with this, well,.
it's not a new issue. In fact, in the Gospel of John, in roundabout chapter 6,.
there is this amazing moment where Jesus does this incredible miracle. He's up on.
a mountain, and He's just taught the people, and the Sea of Galilee is just in.
front. And the people have been there all day, listening to Him teach. And the.
disciples come to Him and say, "Hey, the people are hungry, and we can't send them.
away without any food." And Jesus says, "Well, what have you got?" And they say, "Well,.
we've only got a couple of loaves and fishes." Jesus takes them, He prays for.
them, He blesses them. The disciples start handing them out, and they feed the.
whole 5,000 people. Just these little bit of resource in God's hands can achieve so.
much. And God, through Jesus, sends out His blessing to all of these 5,000 people..
Some of them who were poor people that were living in the Galilee area at that.
time, who probably hadn't had a big meal that whole week. People who are.
vulnerable and poor, they're now being fed by just a few loaves and a few fishes..
And God's heart for blessing is so abundant and so outpouring that the Bible.
tells us that leftovers were there after everybody had said, "Well, hobo, I can't.
take any more," right? Everybody said, "I am totally stuffed." There's still.
leftovers there because God is saying, "This is my heart for pouring out my.
blessings. I will pour them out so much of my people that there will be even an.
abundance and leftover from those blessings." It's a beautiful thing. The.
next day, perhaps not surprisingly, the people who had been fed the day before,.
they show up back where Jesus had been the day before because they want to get.
fed again. They want to experience that again. But interestingly, Jesus is not.
there. Let me pick up the story at that point. Is this helping some people here?.
No. Okay. Matthew, John chapter 6, starting in verse 24. So when the crowd saw that.
neither Jesus nor his disciples were there, they got into the boats and went.
across the Capernaum to look for him there. They found him on that other side.
of the lake and they asked, "Rabbi, when did you get here? Like, what's going on?.
We're trying to find you. When did you get here?" Verse 26, Jesus replied, "I tell you.

$^{321}$the truth. You want to be with me because I fed you, not because you understood.
the miraculous signs." Jesus sometimes didn't win friends or influence people..
Sometimes Jesus got right to the heart of the matter. And all these crowds showed.
up again and they're like, "Where have you been, Jesus? We've been looking for you..
We want to be with you." And he's like, "The only reason we want to be with me is.
because I fed you yesterday, but you didn't see beyond the burning bush to.
the God that's behind the flames. You didn't understand the miraculous sign.
that was there, that this is actually not about the fact that I can feed your.
stomachs for a few hours. It's about the fact that God is present with you, that.
the Messiah has come, that you're being welcomed into a new relationship that.
will change your heart and change your life." Jesus is saying, "You got to see.
behind the blessings, but what's happened is you've gotten so focused on the.
blessing that you've forgotten about the blesser. You're so into the gift that.
you're not even thinking about the giver. And you just want to come to me because.
I'm giving to you and you're coming to me dependent on what I'm giving to you.
and he's saying, "You've got to see bigger than that." This is really important.
because I think it picks up on a point that resonates for so many of us and.
it's this. When we're obsessed with the blessings of God and we're building.
our intimacy with him primarily on what he does or doesn't do for us, that is.
primarily a self-centered way of being. That's about what God can do for me..
Let me make this really clear. God does not exist so you can use him. Jesus is not a.
good luck charm that you wear around your neck to get a couple of blessings.
this week. Selfishness sits at the very heart of why we want more from God than.
more of God. Selfishness, self-centeredness. If you want to deepen.
your intimacy with God this year, that's something to pray about. That's something.
to come around and say, "Where is the selfishness, self-centeredness in me when.
it comes to my relationship with him?" It's really important you understand that.
Jesus died so that he would deliver us from a self-centered life and bring us.
into a God-centered life. That's why Jesus went to the cross. That's the.
biggest blessing that there could ever be. The reality of our salvation in him,.
the forgiveness of our sins, so that we would reorientate our lives away from.
"Give me, give me, give me" to "How do I serve? How do I honor? How do I worship? How do.
I adore? How do I give my whole self to the fact that now my life revolves.
around God, whereas before God was revolving around me?" And if I'm honest.
with you here at the Vine, I would say for some of us in this room watching.
online right now, in the overflow, for some of us we're in the center and God.
is revolving around us. That's not intimacy. That's self-centered idolatry..
Jesus says this to them. He says, "Don't be so concerned about perishable things like.
food. Spend your energy seeking the eternal life that the Son of Man can.
give you. For God the Father has given me the seal of his approval." In other words,.

$^{361}$if we're going to talk about blessings, Jesus is saying to the crowd, if we're.
going to talk about blessings, let's actually not focus on temporal blessings.
that help for a moment. Let's talk about eternal blessing. Let's talk about the.
real blessing that matters. The one about you and your soul. The one about your.
relationship for eternity. The one that's going to form and shape you forever..
Let's get our hearts focused on the eternal and off of just the temporal..
Again, not that God doesn't care about your temporal situation. Not that he.
doesn't want to pour out his heart in you. Not that he doesn't want to bless.
you. Of course he does, but he doesn't want your focus to be so much on the.
immediacy of a blessing that you miss the longevity of your relationship with.
him. Notice how the people reply. The crowds reply. This was not what they were.
expecting to hear. They replied in verse 28, "We want to perform God's works too, so.
what should we do?" This is always the inevitable result that happens when we.
have a self-centered and mostly a benefits of God relationship with him..
Here's what will always happen. We will always want to take control of those.
benefits. We will always want to shape our lives around those benefits, and in.
order to do that, we will try to grasp a hold of them. We will try to get those.
benefits in our hands. They turn to Jesus and they say, "Tell us what we need to do.
to be able to do the things that you've done for us. We want to take the.
blessings of God, and we want to be the ones that are in control of those.
blessings. We want to be the ones that can feed ourselves and our families.
whenever we want to. We want to be able to bless our community." And there was.
probably some good in their hearts when they were saying this to Jesus, but their.
thinking is faulty because they're basing it off of this kind of.
reciprocal sort of relationship with God, rather than a relationship with him which.
just puts him first. And they're saying, "Okay, how do we get the ability to.
control these things?" In other words, they're basically saying, "How do we earn.
the miles? What do we need to do? What's the credit card we need to get? What's.
the flights that we need to book? How do we get ourselves up on status with you,.
God, so that we can do the things that you do?" Jesus says this in response. Verse.
29, "Jesus told them, 'This is the only work God wants for you.'" I'm gonna read that.
again. "This is the only work that God wants from you. Believe in the one he has.
sent." In other words, the only thing he wants from you is to believe in him, to.
connect with him, to have relationship with him. Does he want to bless you?.
Absolutely he does. Does he want to pour out his gifts into your life? A hundred.
percent. Are those gifts good for you? Yes, they are. Do they change our relationship.
with him in some ways? Do we feel his closeness to us when we receive those.
gifts? Of course we do. But what Jesus is trying to do for the crowds, I think what.
he's trying to do for us, is elevate us into a different plane where our.
relationship is not dependent on whether we feel like we're being blessed by.

$^{401}$God, but our relationship is simply in the belief, in the confession, in the.
repentance, in the living out of an honest and real relationship with Jesus..
When we can echo with Paul, "I know what it is to be this, and I know what it is.
to be that, but I have found the secret of being content," that we could echo with.
that woman at the funeral and say, "I just miss him. I just miss him. Not all the.
things he did for me, him." If my wife, Chris, gave me a gift, which she does every.
once in a while, she gave me a gift and I took that gift and I threw it away just.
to focus on her, I wouldn't be honoring her in the giving of that gift. And so.
when God blesses you, he's not asking you to throw it away and ignore it. His.
blessings are good for you. Does that make sense? And you receiving those blessings.
and walking in those blessings in your life is a way of honoring and worshiping.
him. Equally though, if my wife gave me a gift and I was like, "I am so in love with.
this gift that I walk away from her, that I make all my time and my effort, my.
passion on the gift, and I don't even think about the person who gave me that.
gift," then that is also equally wrong. Are you with me? Both of those extremes are.
wrong, and the reality is, as Christians, we so live often between one of those.
two extremes. But what if my wife gave me a gift and I was so honored by that and.
I loved the gift and I enjoyed the gift, but then I allowed that gift not to make.
me self-centered in who I am and walking off and ignoring her, but actually using.
that gift to fill me with a greater love, a greater desire, a greater need to be.
intimate with her, and to realize that the gift is not the sum total of who she is.
for me, but it's just the catalyst that can bring me out of myself back to her.
in relationship with her. Are you with me? It's about getting the direction of God's.
blessings. Some of you, in action from this message, all you need to do is.
reorientate your thinking about the blessings of God. They are not a sign of.
whether He's pleased with you or not. They're an ability for you to connect.
better with Him, to not take it all into the gift and think the gift is everything,.
but to allow the gift to move... Another way of saying it is, God's gifts are not.
gods, but they are gods. All right, that's tricky in English. Have a look at it on.
the screen here. God's gifts in your life are not deities. They're not gods to be.
worshipped, but they are gods. They are an expression of His character, of His.
goodness, and how much He loves you, and when you get that in the right balance,.
when you stop worshipping the benefits of God and actually start to worship Him,.
realizing that those benefits and those gifts are a trigger for you to be in.
deeper and more personal relationship with Him, you come to realize that the.
gifts are actually not bad, not something you should throw away. They're triggers.
for you to connect with Him more. They're invitations for you to find Him, and in.
the seasons where you don't have those gifts, it's not like He's not inviting.
you to be intimate with Him. He's just letting you know that His intimacy and.
your intimacy with Him is not dependent on what He does for you every single day..

$^{441}$I love the way that Psalm 37 verse 4 puts it. It says this, it says, "Delight.
yourself in the Lord, and He will give you the desires of your heart." Notice.
the order there. It doesn't say, "Delight in the desires of your heart and all.
that God has done for you, and then try to find Him." It's saying, "Delight in the.
Lord with all of your heart, mind, soul, and strength, and then everything else.
will be there." It's like Jesus in Matthew 6. He says those amazing things. He says,.
"Seek first the kingdom of God, and everything else will be added to you..
Everything else will be given to you. I want to give you all those things, but.
seek first me." Here's the reality. When our intimacy is based primarily with the.
benefits of God, here's what happens. We end up seeking the blessings of God in.
hope that we would, through receiving the blessings, actually find God. But it's the.
other way around. We need to seek God, and in finding God, we will come to receive.
His blessings. Do you follow that? I think so often we seek the blessing as a.
defining characteristic of who God is. We think we're going to find God out of.
the blessing. God is saying, "Seek me first, and all these other things will be.
added unto you. I just miss Him." There's another beautiful psalm. I'll close with.
this. This is a beautiful psalm. Psalm 131, verse 2. I want to read this to you,.
because this is really important. "Instead, I have calmed and quieted myself," the.
psalmist writes, "like a weaned child who no longer cries for its mother's milk..
Yes, like a weaned child is my soul within me." He's talking about his soul,.
and he's saying, "I've matured. I've grown up, and my soul is now weaned before you,.
and in that place, I find myself at rest. I find myself with you like I've never.
found you before." It's a beautiful picture here of a child, and in the.
picture, there's a child nursing with its mother. If you're a mother in.
this room, or if you've been a mother in this room, you'll know what this is like..
When you've got a child who's nursing at your breast, the child knows that that's.
where it goes to for its milk. If the child is near you, it starts.
screaming. It starts crying out. It wants its milk. It wants its provision. Its.
primary relationship with its mother at that point when it's nursing is, "This.
mother can do something for me." When the mother doesn't do that, it screams,.
it cries, it lashes out. Nesamus is using that as a picture, and he's saying,.
"I want to be a nursed child. I want to be a weaned child, sorry, not a nursing child.
when it comes to God. I want to be one who finds myself still and alive in Him.
because I'm not kicking and screaming when God doesn't do what I want.
God to do. I want to find myself quiet and at rest, knowing that just like a.
weaned child, I can crawl up to my mother." This is what a weaned child does. "Comes to.
its mother now," and the child trusts the mother. It's not kicking and screaming.
anymore. It trusts that the mother will provide at the right time. It wants to.
just be with its mother for the comfort and security of lying in its mother's.
arms, no longer screaming for the milk from the breast. Nesamus is pulling all.

$^{481}$this picture together, and he's saying this, "You need to wean yourself." Last week I.
spoke about gleaning. This week I'm finishing talking about weaning. Some of.
you in this room, you need to wean yourself off of your codependency for.
the blessings of God. You need to go from a nursing child with God to a weaned.
child who can say, "It is all right with my soul, whether God does anything else.
for me ever again, whether He answers a prayer or not, whether He comes through.
or not, and I hope He does, and I really hope He does, but the reality is my whole.
relationship with Him is not dependent on what He's going to do for me. It is.
dependent on the fact that He is just who He is, and I trust Him. I trust that.
He will nurture me, care for me, provide for me, answer the prayers. I trust that,.
but I'm not gonna kick and scream, get angry when it doesn't feel like it's.
happening, because I am weaned from Him and that dependency of being on those.
things. I am not connected to them in a codependency state. Does that make sense.
to you?" And so I want to invite you, if this is a year of intimacy for you, this.
is a year of freedom, and the blessings of God are good and great, and we should.
hunger for them and desire them, absolutely, but may we not build our.
relationship with God based on what we do or do not receive from Him. May we.
build our relationship with Him, finding ourselves quietened in our soul before.
Him, because we just want Him. That, my friends, is true intimacy. Shall we pray?.
Let's pray. Father, I thank you for each person here. Father, I thank you for your.
love, and I thank you that you see them, and I thank you, Lord, that your heart is.
a good father, and that you long to pour out your blessings, your gifts, and there.
are so many great benefits of being in relationship with you. Lord, we are so.
grateful that we worship you and not some other idol in our lives. We thank.
you for the blessings that you pour out so abundantly that there are leftovers.
at a feast. We thank you that you pour out those blessings on us abundantly.
with a generous wrist. But Lord, we ask that you would forgive us. Forgive us.
where we've taken the good things of God and we've turned them into God's.
themselves. Forgive us where at times we've become codependent on the.
blessings. Forgive us where we've made the blessings the signal to us, the.
barometer of our health, of our relationship with you. Forgive us where.
those blessings have become more important to us than simply you. Lord, if.
we're in this room watching online in the overflow and we know that our.
intimacy with you is primarily based on what we receive from you, Lord, we pray for.
maturity for us this year. We pray that you would mature this church, you would.
mature each person here. You would help us to find Paul's words echoing in our.
hearts. I have learned the secret of being content whether there are.
blessings or not. I just want him. Take a moment for yourself now. Just a moment.
between you and God. Off the back of everything I've been sharing, how does he.
want to speak to you today? What is it that you might need to confess? How is.

$^{521}$your relationship built with him? I believe he wants to draw near to you now.
and just have a conversation directly with you. And so we're gonna make some.
space for that before we respond together in worship. So take some time..
Just quiet in your heart. Bring your soul to him. Ask him to speak to you..
[music].
[BLANK AUDIO].
\newpage



\section{}
\label{sec:YW6pONcYjiU}
\textbf{2024-01-29 What We Turn To: The World [YW6pONcYjiU].mp3}
\newline
\newline
連結: \href{https://youtube.com/watch?v=YW6pONcYjiU}{\texttt{ https://youtube.com/watch?v=YW6pONcYjiU}} ~~~~ 語音日期: 2024-01-29 
\newline
\newline
\hyperref[sec:J4wOHcthlVk]{\small{< < < PREV SERMON < < <}}
~
\hyperref[sec:index]{\small{[返主目錄]}}
~
\hyperref[sec:UXFDi7xyYD0]{\small{> > > NEXT SERMON > > >}}
\newline
\newline
$^{1}$Amen, amen, amen..
We wanna thank the worship team..
Thank you guys..
(congregation applauding).
Why don't you go ahead and grab a seat, everyone..
Welcome to the Vine..
Welcome to the 11 o'clock as per youge..
It is full..
It is real full..
There's like no room in here at all..
Well, guys, I am Promise..
I am the creative pastor here at the Vine..
Welcome, so glad you're here..
And to those who are watching online, also welcome..
We're so glad that you're tuned in..
You're not here, but you're kind of here..
So thanks for that connection..
We've been going through this series over the past few,.
actually since the beginning of the year on intimacy..
All right, if you've been here,.
you would have heard Pastor Andrew talk about this idea,.
this concept of intimacy..
And it's a strange word trying to figure out.
what does that mean?.
So we've been exploring it..
We've been peeling back the layers and understanding.
what does it mean for you and for me.
to find our intimacy with God?.
Pastor Andrew would have shared with you.
that we as a leadership here at the Vine.
have been really wrestling with this idea..
God spoke to us and has challenged us.
to lead the charge in helping us all understand.
that God has called us not just to come into the gates.
and stand at the entrance..
Or another way of putting it,.
he's not called us to settle for the atmosphere of God,.
but he actually has called us to press in.
to the person of who God is..
Intimacy is when we press beyond the atmosphere..

$^{41}$We don't just settle for an idea of God,.
but pressing in to the person of who God himself is..
To not just be satisfied with the atmosphere..
This kind of call is an invitation to a real journey..
Intimacy is not a one-time thing..
It doesn't just snap and happen..
It's a whole journey..
And so far we've looked at some of the obstacles.
that present themselves in our journey.
towards intimacy with God..
The first thing we talked about with this concept.
of rumors of God, how if we're not careful,.
we can settle for secondhand accounts.
of what someone else says about God.
to be our primary intimacy..
Pastor Andrew would have had this little basket.
and he would have shown you what it looks like to harvest.
and what it looks like to glean,.
to pick up the leftovers of what somebody else said..
And in order for us to actually have intimacy with Christ,.
we cannot allow the substance of our spiritual life.
to just be on someone else's account..
Intimacy means you and me have to do the hard work.
of actually being relational with God, coming close to him..
We can't just let it be someone else's story.
or someone else's sermon..
That's predominantly why you hear us regularly say.
that church cannot just be 90 minutes on a Sunday..
That's not just a cool saying..
It's 'cause for you to be intimate with God,.
you've gotta press through..
It's gotta be your time as well..
The idea of secondhand rumors of God,.
what other people says can be a challenge to our intimacy..
And if you were here last week,.
you also heard Pastor Andrew talk.
about this idea of blessings, right?.
And blessings are a good thing..
We all love blessings..
I want blessings, you want blessings,.

$^{81}$we all want blessings..
Blessings aren't bad..
However, our relationship to blessings,.
our posture to blessings is worth questioning..
We have the temptation sometimes if we're not careful.
to evaluate our intimacy with God.
based off of how blessed we are,.
how many things he's given us..
And if we determine our intimacy with God.
based off of our blessings,.
we all know that blessings come and blessings go..
Some days we are extremely blessed.
and it's everywhere and it's evident and it's tangible..
Some days we're in the desert,.
but that doesn't mean God has left you..
It doesn't mean God is absent..
That's why we can't build our intimacy on blessings.
because we end up building our intimacies with blessings..
We end up worshiping the gift over the giver,.
finding ourselves in a form of idolatry..
That's the second thing that ends up being an obstacle,.
a challenge to our relationship of intimacy with God..
See, intimacy is such a powerful thing..
Intimacy is vast..
It's going to shape the way that you see yourself,.
the way that you see life,.
the way that you are aligned,.
the things that you understand..
And if we're intimate with God, He shapes us..
He aligns us to Him and to His will..
He shapes our motivations, our preferences,.
our thought process, the way that we think..
He shapes our philosophies and our worldviews.
and our behaviors and our practices..
Intimacy and this closeness with God aligns us to Him.
so that we can know and be close to the heart of God..
We can hear the creator speak.
and know that when we speak, He listens and hears us..
Now, if we actually pull it back a little further,.
yeah, I dare say that intimacy is the gift of God Himself..

$^{121}$That's why we're talking about it..
That's why it's so important..
Now, achieving intimacy is one thing, right?.
Getting there, I'm intimate, awesome, cool..
Maintaining intimacy is a whole 'nother struggle..
Yes, there are things that will prevent us.
in our pursuits of intimacy,.
but once we have achieved.
and once we have become intimate with God,.
there are also other things..
There are also false gods, lesser goods,.
alternatives, faulty substitutes that position themselves.
and present themselves as viable options to God,.
looking to take our worship and our attention and our time..
There are substitutes that are all around us,.
not just obstacles blocking us,.
things that we can literally turn to accidentally.
and find our intimacy with them..
And that's what we're gonna be looking at.
over the next two weeks..
Two major things that we, if we're not careful,.
will turn to and find intimacy with,.
things that literally end up taking the place.
of God in our lives..
So with that said, let's talk about these things.
that we turn to..
It's good for us to understand this,.
that we have this tendency that when we fail.
to find intimacy in Christ,.
we are then tempted to turn somewhere else..
Now this is not like some big spiritual concept..
This is an everyday thing..
If you try at something and you fail to find the success.
you need at it, by nature,.
you're gonna look somewhere else for it, right?.
Like if I try hard to turn this light on.
and it's not working, I'm gonna go find another light..
That just makes sense..
That's just common sense..
But for us, we have to be aware that.

$^{161}$if we fail to find intimacy.
or fail to find satisfaction in Christ,.
then these other things become seemingly more desirable,.
seemingly like they'll fulfill more,.
like they will give us the validation.
and the significance that we think that we need,.
that they would satisfy..
And if we're not aware, other things,.
good things, end up taking the place in our hearts.
that only God was ever actually meant to occupy..
And that's why we have to wrap our heads around this.
and think around it and be careful..
Perhaps one of the most common,.
one of the easiest forms,.
one of the most inconspicuous things that wrestles.
and seeks and pulls us into intimacy..
And if we're not aware of it,.
we'll find ourselves turning to it time and time again..
It's called the world..
Now, I grew up in church.
and I've heard the phrase the world a lot..
It speaks, it's really big, it's kind of a big concept..
It doesn't really say a lot, is the world,.
what does that even mean, right?.
Like the world is a lot,.
but the world is also nothing in this explanation..
But we do know that scripture as a whole.
speaks about this concept of the world..
And my hope is that as we look into scripture today,.
we'll be able to help you understand.
what does it mean for the world to vie for our affections.
and what does it look like for our response.
to be correct in that..
Specifically, the author John in 1 John.
has a lot to say about the world..
So if you have your Bibles, if you have your smartphones,.
if you like the screen, we have it right here..
So let's read this together, it says this..
Do not love the world or anything in the world..
If anyone loves the world,.

$^{201}$love for the Father is not in them..
For everything in the world, the lust of the flesh.
and the lust of the eyes and the pride of life.
comes not from the Father, but from the world..
The world and its desires pass away,.
but whoever does the will of God lives forever..
Let me pray for us..
God, we come before your word knowing that it's speak,.
that it's living and active,.
that it's meant to speak to our hearts,.
that it's meant to shape us and change us..
So Holy Spirit, would you open us this morning?.
Would you allow us to position ourselves.
to hear from you, God?.
What your word has to say,.
will you allow us to position ourselves.
to be shaped by the truth found in your word, God?.
May it speak and may we be a people who hear and respond..
So you name me pray, amen..
So the first thing that we see in this passage.
is that John is gonna create somewhat of a duality..
He's gonna create two options..
There's the way of the world,.
then there's the way of the Father..
He's not saying that they're bad or good,.
he's just saying there's the world and there is the Father..
Two different ways, two paths,.
if you would have it that way,.
two alternative options for how your intimacy can be placed.
and how you can live..
Actually, if we zoom in even further,.
we'll see that across these two passages,.
John is going to emphasize one phrase a lot..
He's gonna look at the word the world.
six times in two verses..
Do not love the world, anything in the world,.
if anyone loves the world, in the world,.
from the world, the world and its desires..
John is speaking predominantly to a concept of the world.
and when something, a good little tip for you.

$^{241}$in your Bible studying, in your scripture reading,.
when you read a passage and you see a phrase.
show up over and over and over and over again,.
the author is intending for us to pick something up..
He's showing us what the key concept truly is..
And what John is doing here is showing us.
that there's something about the world.
that we need to understand before we decide.
where our allegiance will be..
So the word that he uses to represent the world.
is the Greek word that we get our English word,.
cosmos, from..
Now, the cosmos is something that might be quite familiar.
to all of us, right?.
Cosmos speaks of the world..
We think of the physical planet, the physical space..
That's one of the biggest definitions.
of what that word means,.
speaking of the earth and matter and substance.
and things that are tangible that we can feel and touch..
However, when John is talking about that,.
he's talking about the world,.
it would be really odd to utilize this definition,.
to not love the physical space,.
to not love the earth and the ground.
and the sphere that we live in..
So actually, in Greek, there's two definitions,.
or two primary definitions to the word cosmos..
Yes, there is the idea of the earth and the physical space..
However, there's a secondary definition.
which speaks of this..
It speaks to systems and structural ways of doing things.
that are different from God..
That's what the word itself means..
The world can speak of the sphere,.
or it can speak of the systems that are different than God.
in how we do things,.
speaking to the intangible elements of the world..
So for us to understand what he's saying.
about these systems and its impact,.

$^{281}$we're gonna dive really deeply into the idea of the world..
I'm gonna give us an easy definition to remember.
so that we can hold onto it..
So when I say the phrase the world,.
we won't think so general,.
but we'll have a clear position of what that means, okay?.
Everyone with me still?.
All right, so when we speak of the world,.
John is primarily saying this,.
the societal or cultural systems and attitudes.
that govern our values, beliefs, and practices..
Say that one more time..
When John is saying do not love the world.
and the world this and the world that,.
he's speaking of the societal or cultural systems.
and attitudes that govern our values,.
beliefs, and our practices..
See, culture can shape things around us..
Culture is powerful..
It can shape the way that we process information..
It can govern our understanding.
of our feelings and our behavior..
It can define for us what is right,.
and it can define for us what is wrong..
It becomes the lens that we wear.
to evaluate the experiences that we have.
as either good or worthy or not enjoyable..
Culture is powerful because it sets for us.
a set of general norms that we live within,.
and sometimes we don't even question it..
It's just the way things are, right?.
Culture shapes and molds the way that we think..
So when John is speaking of the world,.
he's speaking of the authority that we give culture.
to decide for us what is true and what is right,.
what is good, what is important,.
what our priorities should be,.
what we should believe, and what we should do..
And if he's saying that six times across two verses,.
let's actually ask ourselves,.

$^{321}$what does culture actually say?.
We all live here..
What does culture say to us, right?.
Culture directs the way that we live..
It directs our lives..
It tells us, study hard, work harder,.
get a good job, save a lot of money..
Don't get married so you can afford.
to take care of your spouse and feed your kids, right?.
Nothing bad about that..
Those are actually some good statements, good ways to live..
But the culture that we live in values that..
It tells us this is how we ought to be..
Culture gives us formula..
It says if you work hard or work hard enough,.
you'll earn a lot and you'll be able.
to take care of yourself..
And give the money to things that are promising,.
to things of value..
Save a lot for yourself..
You never know when you're gonna need.
some extra change..
Culture can define our behavior..
An eye for an eye..
Do the practical thing that makes sense..
Don't loan money to people who can't pay you back..
Eh, it kinda makes sense..
You get what you deserve..
Take care of yourself first..
Worry about your own problems..
Mind your own business before you worry.
about the other person's issues..
Seek the road that gives you the highest return..
Perhaps seek pleasure and avoid pain..
Love your neighbor..
Treat your neighbor good..
Maybe not hate your enemy,.
but don't got time for haters, right?.
Haters are gonna hate, I think is the famous phrase.
that goes around these days..

$^{361}$Culture validates significance..
You have meaning because you're popular..
You have value because what you're saying.
and what you're doing and what you're producing is relevant..
The more followers you have,.
the higher your worth somehow becomes..
That's a weird system..
The more friends you have online,.
that makes you into a more likable person..
Maybe, but that's just kind of what we've accepted..
That's kind of how things are..
And perhaps the most interesting one.
is that culture shapes the how..
Manifest it and it will happen..
Have you ever heard that idea?.
Manifest it and it'll happen, just will it into existence..
Life is what you make of it..
Live your truth..
You're the captain of your soul..
Self-care and repeat, just follow your heart..
Just live..
I think YOLO used to be a cool thing for millennials..
Where I come from, it's risk it for the biscuit..
Or if you grew up in black culture,.
it's you do you, boo-boo..
All right, like these are the things that we hear..
These are the things that ends up shaping us..
The reality is that these sayings and these beliefs.
are a direct byproduct of the culture we live in..
They are what the world says..
So when John speaks of the world,.
he's speaking of the systems and attitudes.
predominantly brought on by our culture and our society..
It's the way that our society sees and perceives things..
It's not evil, per se..
It's not bad, even..
It's the way it is..
It becomes the norm..
It's our default way of operating..
It's just the way that we go..

$^{401}$In this whole passage,.
John gives us all these ideas of the world,.
but then he interjects,.
and he's actually changing the direction of everything.
by introducing the Father..
He's saying, yes, there's a way, there's a default system..
There are norms, there's culture,.
there's society that tells you how to live.
and what to do and what to believe.
and what your priorities should be,.
but there's also another way..
There's also the way of the Father..
There's also a different way..
Much like when he speaks of the world,.
John refers to the Father..
He's primarily speaking about how God and kingdom values.
can shape the way that we see the world..
And all that happens because Christ comes into the world.
and reveals to us both who God is,.
the heart of God, the nature of God, who we are..
It's based off of Christ's revelation..
So in Christ, we have a tangible and clear picture.
of the Father, of the way of the Father..
By this, Christ ends up shaping how we see things,.
how we process information..
He defines for us what is right and what is good.
and what is wrong..
Christ becomes the lens by which we evaluate.
the experiences in our world.
if we are subscribed to this way of the Father..
Christ shapes and molds our thinking..
So when John speaks of the Father,.
he's speaking of the authority that we have given Christ.
to decide for us what is true and what is right,.
what is important, what is priority, and what is good..
So what does the option of Christ say?.
Christ shows us how to live..
He tells us to love God with all your heart.
and all your mind and all your soul and all your strength.
and to love your neighbor as yourself..

$^{441}$Really good things, nobody's gonna say no to that..
Christ says he welcomes those who are vulnerable..
He says, "Blessed are the poor in spirit.".
This was revolutionary when it was spoken..
"Blessed are those who hunger and thirst for righteousness.".
Perhaps my favorite one..
"Blessed are you when you are persecuted falsely.".
Blessed are the victims?.
Ooh..
Christ redefines our strength..
He says, "Blessed are the meek..
"Seek justice for the marginalized.
"and take care of the widow and the orphan..
"Care for the alien and the stranger in your land.".
He says, "Give without expecting a return.".
I think the parable he says,.
"When your right hand is giving,.
"make sure your left hand doesn't know what's happening.".
Right?.
He says, "Yeah, love your neighbor,.
"but get this, love your enemy.".
Christ says, "Turn the other cheek.
"and maybe go the extra mile.".
Not just if someone requires you to go a mile with them,.
go a mile further..
Christ defines, redefines our strength..
Christ purifies our heart..
He says, "Blessed are the pure in hearts..
"Do good deeds so that your Father will be glorified..
"Beware of practicing your righteousness.
"in front of other people..
"Store up treasures not here on earth, but in heaven..
"Humble yourself before the Lord.".
And lastly, Christ is our source..
He will tell us things like this..
"Do not be anxious..
"Ask God for what you need..
"Thy kingdom come, thy will be done.".
If God clothed the lilies of the field,.
how much more would He clothe you?.

$^{481}$If God would feed the birds of the air,.
how much more would He feed you?.
He tells us that if you have to ask, it'll be given..
"Every good and perfect gift comes from the Father above..
"The heart is deceitful above all else..
"Man might plan his way,.
"but it's the Lord that directs his steps.".
See, these sayings and beliefs,.
this is the way of the Father..
They reflect the heart of God..
In the intimate journey of transformation,.
He takes His followers on, those who are submitted to Him..
We become shaped and molded by the way of the Father..
So when John speaks of the Father, it's these systems..
It's these attitudes predominantly brought on by Christ.
at the center of our lives..
So with an understanding of the way of the world,.
culture, society, things that are just okay,.
not bad, not good, just are, and the way of the Father,.
John now turns our attention to understand.
the relationship between the two,.
and what we, who we are to be,.
relating to the world and to the Father..
John would say this,.
"Do not love the world or anything in the world.".
I'll say that again, 'cause that's a really big statement..
"Do not love the world or anything in the world.".
If anyone loves the world,.
love for the Father is not in them..
Let me put this in sermon series language for us.
so we stay quite relevant..
"Do not find your intimacy with the world.".
You cannot afford to find your intimacy with the world.
if the love of the Father is meant to be evident in you..
All right?.
[coughing].
John divides us two different ways, two different paths..
He separates them, and I think the temptation for us.
is a lot more subtle than this text..
I think whenever we hear about the world or idols,.

$^{521}$we're like, "Okay, I'm not bound on idols..
"I'm not worshiping the world, I'm okay.".
And I think for most of us, most mornings,.
we don't wake up, pop out of bed, alarm clock goes off,.
and we're like, "All right, what am I gonna do today?.
"Let's worship the world.".
I don't think that's on anyone's agenda, right?.
No one wakes up like, "How can I love the world?.
"How can I put God on the shelf?.
"How can I forget about Christ?.
"How can I just worship the world.
"and myself and myself and the world?".
I think it's actually significantly more subtle,.
therefore more dangerous than that..
I think for a lot of us, and please don't miss this,.
for a lot of us, we wanna love Jesus..
We wanna do the right thing..
We wanna be intimate with Jesus..
We wanna have Jesus..
But if we're honest, we also wanna sprinkle.
a little bit of the world on there..
We also want some of the things that the world offers,.
some of society's values, some of society's preferences..
So perhaps we subscribe quite a lot.
to the way that the world operates..
And that's the tricky thing..
We want the, "I buy Jesus and get a side of the world,".
the combo meal, the buy one, get one free concept..
And John is saying, "It doesn't work that way..
"It cannot work that way.".
Those two things, you can't have 'em both..
You can't be walking towards Jesus,.
but also walking towards the world, right?.
It's like this lamp..
It's either on or it's off..
It can't be on/off..
Doesn't exist, right?.
So I can, if I pull harder, it'll be, no, no, it's still on..
If I pull it again, it's still off..
There's no middle ground..

$^{561}$There's no fence to sit on..
And if we're honest with ourselves,.
most of us, if we evaluate where we are, are fence sitters..
We're trying to get some Jesus.
and we're trying to come to church,.
but also when we go through Monday to Friday,.
where some of our practices might be a bit dodgy.
and some of the preferences that we have.
may be a bit cutthroat.
and we might not really love our neighbor..
We might tolerate them.
and we definitely don't love our enemies..
The two things kind of don't work..
They can't go together..
If your intimacy, I'll put it this way,.
it's a dichotomous scale, right?.
So the more you choose to be intimate.
with the things of the Father,.
if you walk towards the Father in your intimacy,.
you're actually walking away from the things of the world..
Same vice versa..
If we walk towards the world in our intimacy,.
we're walking away from the things of the Father..
You can't do them both..
Now here's the thing..
I don't want to come across as someone who is condoning,.
you know, just be holy, live in your holy house.
on top of the hill and talk to no one.
and don't have Instagram and don't have friends.
and don't drive a car and don't have electricity..
And I'm not saying that..
That's actually not what the scripture is saying, right?.
To love God and to choose to walk closer to God,.
what it means is that these things control you less..
These things own you less..
They serve you, you don't serve them..
There's a freedom that we step into.
when we choose to walk in intimacy with God.
and allow the things of the world,.
we're just saying it, the cross before me,.

$^{601}$the what behind me, the world behind me, right?.
It's this way of engaging with God.
where we are choosing to walk closer to him.
and leaving certain things behind by choice, by choice..
The reality is if we aren't careful,.
we can allow culture and present philosophies.
and pseudo psychology and social norms.
to govern the way that we see things.
and practice life..
If our aim and our chief satisfaction.
is in satisfying ourselves,.
if we give society the ultimate authority.
to define what is right and what is wrong, what is good,.
then we are in fact substituting intimacy with Christ.
for intimacy with the lust of the flesh,.
intimacy with the lust of the eyes,.
intimacy with the pride of life..
And John makes it very clear..
He goes, "These things do not come from the Father,.
"but from the world..
"And the world and its desires will pass away..
"But whosoever will do the will of the Father.
"lives forever.".
Incompatible, the two things don't mix..
Think of oil and water..
The values of love others first.
and humble yourself before God.
and blessed are the poor and the meek.
don't really work that well in the marketplace, right?.
It's kind of a hard thing to do..
God's ways are different..
His people therefore must be different..
The values and methods of Christ.
when compared to the world, they cannot coexist..
There is no blend of them..
I know I'm beating that to a pulp,.
but I just gotta make that really clear..
And I guess the question we have to ask ourselves is why?.
Why can't I have a little bit of Jesus.
and a little bit of world?.

$^{641}$Why do they not work out that way?.
Well, here's the thing..
God is a jealous God..
There's a place in your heart for God..
There's a place you were created for him,.
by him, through him..
You have life because of him..
And so if he moves into your life, he takes up all of it..
There is no space for anyone else..
Jesus is not hoping to have a roommate in your heart.
that he shares with..
It's either Jesus or it's the other person, right?.
He's too big for that..
James chapter four, the message version.
reads it really, really well..
I'm just gonna read it to you..
So take a minute..
It's not gonna be on the screen..
You have to really listen..
And it says like this..
You are cheating on God if all you want is your way,.
flirting with the world every chance you get..
You end up as enemies of God and his ways..
And do you suppose God doesn't care?.
The proverb has it that he is a fiercely jealous lover..
And this is the best part..
And what he gives, what he offers,.
the way of the father is far better.
than anything else you'll ever find..
That's the kicker..
Our tendency is pairing..
It's pairing..
It's Christ and something..
Christ and intimacy with this..
Christ and fill in the blanks..
So Christ and success maybe for some of us..
Christ and validation..
Christ and relevance..
Christ and creativity..
Christ and security..

$^{681}$Christ and blessings..
Christ and, but I think it's supposed to be Christ alone..
Christ alone, not Christ and stuff..
Christ and things..
We will find ourselves settling for the idea of Jesus.
if we are trying to pair him with something,.
never really getting to him itself..
And so I think when you look at a message like this,.
I'm oftentimes cautious..
I don't ever wanna come across as,.
Promise has read all this stuff.
and his life is like perfect now,.
so he's telling you guys, get your act together..
That's not the case..
Couldn't be further from the truth..
I am just as broken as everyone else..
So I wanted to share because Christ and fill in the blank.
can be maybe too broad for you..
So let me share what it is for me..
What is my Christ and that I struggle with every single day?.
For me, it's Christ and control..
I wanna control my life..
I wanna have a say in what I do..
I wanna, I was telling Tony,.
I like to plan out what I'm gonna do every decade..
At 30, I'm gonna have a master's and have my first kid..
In my 40s, I'm gonna start my PhD..
In my 50s, I'm gonna be a professor..
In my 60s, I'm gonna write a book..
Like I have it all planned out..
I've written it out..
It's a beautiful chart, it's an Excel sheet..
I know what to do..
Christ and controls for me..
But this is the reason.
because I don't wanna depend on nobody else..
Right, I have a sad story too..
I came from a messed up situation too..
I know what it's like to not have anything..
I wanna make sure I earn enough..

$^{721}$So it's me having to give myself the independence.
to make sure I earn enough,.
to make sure I have the right job,.
to make sure that my kids are fed.
and make sure my wife goes without need..
I struggle with letting God have control..
And some of my closest friends will tell you.
that's the one thing they pray for me all the time..
I just can't let it go..
I really, really struggle..
For me, it's always pairing the idea of Christ..
I wanna be a good Christian..
I wanna be intimate with God..
But also I really wanna have control.
and have a say in how things turn out.
and make sure my life isn't a complete mess..
I pair them together and Christ has been convicting me.
that rather than pairing my control with Christ,.
I'm supposed to be submitting,.
submitting my control to Christ..
This is the thing, these objects,.
these concepts aren't bad in and of themselves..
God didn't just create these things.
and said it's just a bunch of bad things in the world..
No, there's a lot of good out here..
But the problem that we potentially have,.
the tendency that we have is to pair together things.
when things are supposed to be submitted..
And so I ask you, but that's my story, that's my reality..
And maybe it's same for some of you, but what is it?.
What are the things, if you were to really ask yourself,.
what am I pairing with Jesus.
that I'm supposed to be submitting to Jesus?.
What good is there out there that I like.
that I've actually accidentally put at the same level.
or made an idol of or become more intimate with?.
And it's pushing out Jesus day in and day out..
It is wearing me out..
It is exhausting to try and juggle both.
because it's not possible..

$^{761}$For some of us, not on scripts,.
for some of us, it's within relationships..
And I really, I prayed about it this morning..
I really felt I was supposed to share it at the 11 o'clock,.
so bear with me..
If there is a relationship in your life.
where you are putting that person.
at the same level of Jesus,.
let me explain why it doesn't work..
Because only God can satisfy fully..
Only God can fulfill..
So what we end up doing isn't just hurting ourselves..
We actually end up hurting that person..
We put God-like expectations on them to satisfy.
and make us feel or make us be a certain way.
when that can only be found in God..
And that's not a rebuke..
That's clarity..
All right?.
The Father reveals these things that we might change,.
that we might turn away..
So if that is you, I ask you to wrestle with it for a bit..
You know, what is it?.
Maybe that's not you..
Maybe it's something else..
But we all have this temptation,.
and that ends up thwarting our intimacy.
away from walking closely with Christ.
and instead being intimate with other things..
So I recognize this is a challenging,.
not even challenging,.
it's a very black and white sermon..
Very strong worded, maybe..
So I never wanna end on a negative..
So I wanna offer us some thoughts, some ways,.
because the scriptures are so good..
Like, God doesn't just tell us,.
"Oh, don't do this," and then like disappears..
He actually leads us to a place.
of greater intimacy with him..

$^{801}$So I wanna provide us two, I won't say simple,.
but hopefully simple to remember, steps..
Some two things that we can do.
that will help us better choose.
where we're gonna be intimate,.
who we're gonna be intimate with..
Is that okay?.
First thing is this..
It begins by recognizing transformation over conformity..
There's a passage in Romans 12.
that tells us not to conform.
to the patterns of this world..
The patterns, the ideas, the systems,.
the structures, the way that the world operates..
To not conform to that,.
but through the renewing of our minds.
to then be transformed..
The renewing of our minds..
Do not conform to the world..
Live in the world, be in the world..
Don't be of the world..
That's what that verse is saying..
My second thing is this..
I share a quick story before I tell you this point..
Last Sunday, I was listening to Pastor Andrew preach,.
and I was like, he got to his,.
I don't know, his deliverables or his action points, right?.
'Cause every good pastor has like an action point..
And one of his action points was,.
my favorite passage in scripture from Matthew chapter six..
It's the idea of seeking first the kingdom..
And I was so annoyed knowing that's what I'm after..
I'm like, Andrew, you stole my point for next week..
He laughed and he hugged me..
He goes, preach it anyways..
Here's the thing..
We're not saying it twice because we just want to repeat it..
We're saying it twice because we want you to actually do it..
Like this is super important..
The idea of God-centered priorities.

$^{841}$will change everything for us..
It will change everything for us..
Matthew tells us to seek first, first, the first thing,.
the most important thing, the thing of most value,.
the thing of highest worth..
Seek first his kingdom..
He's defining for us what we should think.
and how we should believe and how we should live..
Seek first the kingdom..
And he doesn't say ignore everything else..
He doesn't say, oh, well, we'll see what happens..
Seek first the kingdom of God and his righteousness..
And all these things..
Where promises kids go to school,.
how much money is in promises account,.
promises wife and family being in America,.
but me living in Hong Kong and missing Christmas.
for 10 years in a row..
All the things that are hard..
He says, seek first my kingdom..
Seek first my kingdom promise.
and I will take care of everything else..
He doesn't tell me how he's gonna do it..
He doesn't say here are my four steps..
Here's a PowerPoint, go look at it..
But he invites me to trust him..
And in that invitation of trust is perhaps.
a beautiful invitation to intimacy with God..
Seek first his kingdom..
I know there's a lot to process..
I know there's a lot of things in here,.
but I really want us to, I guess,.
leave today encouraged and hopeful..
Because if we grasp what intimacy is,.
if we can fall in love with the idea of being connected,.
being aligned, being in a God who speaks,.
who listens, who walks with you through every trial.
and who does not give you temptations,.
you cannot handle, who is your ever present help,.
who is your rock and your strong tower,.

$^{881}$who is the rock of your salvation,.
if we can understand intimacy with that,.
I believe it'll make us rethink our allegiances..
If we can know that intimacy with God.
means I can choose peace..
I don't have to worry about the things..
He actually tells me don't worry about it at all..
Don't be anxious for anything..
Two different places, Matthew and James,.
made it real clear, don't be anxious..
Just leave it up to him..
Intimacy with God, choosing intimacy with God..
Because the world will say worry about it,.
plan, plan, plan, get your spreadsheets,.
do your homework, prepare yourself, prepare yourself..
God's saying don't be anxious..
To be a Christian means to walk through this.
without anxiety, right?.
To live and there's things happening,.
but to not be shaken..
Because our house isn't built on the sand..
It's not built on what is important and popular in 2024,.
that'll probably be different in 2025.
and different in 2028..
It's not based off of popular psychology.
or pseudo psychology or whatever magazine we're reading,.
whatever the news tells us,.
because our hope is built on Christ.
and he's an unshakable rock..
That position is available for you..
This is a,.
(sighs).
this passage is a call for us to come back..
It's a call for us to recognize that we're human,.
we made mistakes, it's okay, but there's a way back..
So as the band comes up, I want to end with the reality.
of one of my favorite passages,.
one of my favorite parables in scripture..
It's the parable of the two sons, right?.
I think it was somewhat familiar to some of us,.

$^{921}$but the idea is that these two sons have a father..
One son gets his inheritance early, dips out,.
and he runs to the world..
How fitting, fits right in..
It's almost like I planned it..
He goes out to the world and he squanders his wealth.
and he does his thing and he lives his life..
And let's say he makes quite a few mistakes.
to where he's dirt poor, can't eat, has nowhere to sleep..
He's messed it up and the thought goes.
through his head one day,.
rather than living here in the pigsty.
and just being cold and dirty and hungry and not clothed,.
maybe I can go back to my father.
and maybe he will at least allow me to be a servant.
in his house and take care of like the cattle or something..
And so he musters up the courage to go back to his father..
And on his way to his father, his father sees him,.
and perhaps one of the most beautiful things,.
he disrobes so that there is nothing holding him back,.
nothing resisting him from sprinting.
to his son that he loves..
It doesn't matter that he's messed up..
It doesn't matter that he's chosen the world over the father..
The father drops everything..
He runs and he hugs him and he welcomes him..
He doesn't reject him..
Doesn't say you dirty sinner..
He welcomes him in..
He loves him..
It's such a story of love..
He takes his coat and he puts it on him..
He takes his ring and puts it back on him.
to let him know that you're not just gonna be.
someone taking care of my cattle..
No, you're my family..
You're with me..
I still wanna be intimate with you..
And perhaps there are some of us who are curious.
about what that road looks like back to the father..

$^{961}$Can I just encourage you that the father's arms.
are open wide, that he's a father that welcomes us back..
He welcomes the vulnerable..
Says that if we are faithful enough to confess our sins,.
he is faithful and just to forgive us and restore us..
We kept singing he makes all things new..
He makes, we wrote that on purpose, y'all..
Like that is the truth of who he is..
He will make you new, but will you be open?.
Will you be open?.
And so I wanna pray for us..
So if we can all just go ahead and stand..
We'll just take a moment.
and we will just come to the father together..
If this message hits anything for you,.
whether it's about priorities or about allegiances.
or about your intimacy or about the pairing,.
can I just ask you to open your hands?.
As a symbol, not for me, not for your neighbor,.
but this is a moment between you and God..
If there's something here that you feel like.
I need to do with God, I need to speak with God,.
just open your hands..
And Holy Spirit, as we sit here under your word,.
under your truth and your clarity,.
but also under your grace and forgiveness,.
God, we open our hands symbolically to say.
that we've messed up, that there's been moments,.
there are things that vie for our affections.
and our intentions..
We open our hands to say, God, we're open..
Could you please come?.
Would you please enter?.
Would you please come to meet us?.
We know that your arms are open, Father..
We know that the road back to you is one.
where you welcome us and that you love us,.
that you call us your own,.
that you call us sons and daughters, your family..
God, so I just pray against any lie.

$^{1001}$that's in anyone's head now..
Any voice that says, no, no, no, no, no..
Will God actually accept you?.
Will God actually respond to you?.
Any seeds of doubt that are flowing through this place,.
we rebuke them..
But instead we speak your truth, we speak your grace,.
we speak your mercies that are new..
Every single morning, we speak your steadfast love.
over your people that leads us back.
to a place of intimacy with you..
And God, Holy Spirit, would you begin to remind us.
and show us what it means to be intimate.
with the creator of the heavens and the earth,.
the creators of us?.
God, would you draw us in?.
Would you hold us close?.
Would you give us the fortitude and the courage.
to continue taking a step towards you,.
forsaking everything else?.
Because you are better, because you are better..
We give you glory for this in Jesus' name, amen..
(soft music).
(gentle music).
\newpage



\section{}
\label{sec:SARdI1NGpaQ}
\textbf{2024-02-05 What We Turn To: Sex [SARdI1NGpaQ].mp3}
\newline
\newline
連結: \href{https://youtube.com/watch?v=SARdI1NGpaQ}{\texttt{ https://youtube.com/watch?v=SARdI1NGpaQ}} ~~~~ 語音日期: 2024-02-05 
\newline
\newline
\hyperref[sec:UXFDi7xyYD0]{\small{< < < PREV SERMON < < <}}
~
\hyperref[sec:index]{\small{[返主目錄]}}
~
\hyperref[sec:m34r3U0C7b0]{\small{> > > NEXT SERMON > > >}}
\newline
\newline
$^{1}$All right, now I've got to preach about sex..
Oh, Lord Jesus!.
Well, welcome to the Vine..
Give me a moment, give me a moment..
Oh, that was emotional..
That wasn't supposed to be so emotional..
Oh, Lord Jesus..
All right..
Well, welcome to the Vine..
If you're new, we're emotional here..
Man, we're in this series of intimacy, aren't we?.
Talking about intimacy this year..
And we're talking about how God is calling us in 2024 to be as intimate with Him as possible..
To draw into deeper relationship and community with Him..
And we've been talking about how God has created us with an innate desire for intimacy..
All of us, as humans, created in God's image,.
we have an innate desire for intimacy with God and with one another..
And we've been teaching in this series that when we fail to find the fullness of our satisfaction in intimacy with God, first and foremost,.
when we fail in that connection,.
then what will happen for all human beings is that we will naturally turn to other alternative forms of intimacy.
to try to satisfy the hole that is there where we haven't been satisfied in that relationship with God..
Does that make sense?.
So when that one's not satisfied, we very naturally turn to other alternative forms of intimacy..
And I would say the most seductive and the most destructive alternative form of intimacy that we turn to is sexual intimacy..
And so today I want to talk about sex..
I know that is a full-on topic..
Some of you have got salt and pepper in your head right now..
We're going to talk a little bit about this idea of sex..
Now, I want to say something that is a personal conviction for me right up front on this topic, and it is this..
I do not think that the global church is talking about sex enough..
In fact, I would put it this way..
I would say the global church is actually in this time largely silent about sex..
And when the church is silent about a topic that is central to human flourishing and what it means to be made in the image of God,.
then basically what we're doing is we're allowing others to dictate to us in the church what we should believe about this topic..
Because you've got to understand that the world is talking about this topic..
In fact, they're talking about it all the time..
You just need to turn on your social media..
You need to look at the newspapers..
You need to watch films and TV shows..
You'll know that everybody is talking about sex..

$^{41}$In fact, I did a poll on my Instagram this week, and of course that is the ultimate source of everything..
I did a poll on my Instagram this week, and I asked people how regularly are people talking about sex?.
And over 65\% of the people that responded said that they are in kind of friendship groups where people are regularly talking about sex..
So this is a conversation that's happening out there..
Here's the reality. When the church outsources its responsibility to talk about sex, two things happen..
First of all, everybody else will define for us what sex is, and second of all, the church will be upset about that and will point fingers and judge..
Isn't that funny?.
Like when we outsource our responsibility to talk about sex, the world then talks about it, and then we judge that and point fingers at it..
You can't have your cake and eat it, church..
We have to take up the responsibility to begin to speak about a glorious, biblical vision of sex and sexuality,.
bring that back into the heart of the church, because when we don't, we shouldn't therefore be surprised when we see a lot of sexual brokenness,.
both in the world but also within the church..
Hello, everybody..
The problem is when we have talked about sex as the church, we've talked about it really badly..
If you grew up here in church, no doubt you've probably heard some stuff be talked about sex over the years..
Normally, here is the rhetoric. Normally, here's what church will do..
Church will just say, "Look, sex is bad. Sex is to be avoided. Sex is something that you should not do..
Sex is something that's kind of mucky and disgusting and is a little bit messy, and sex is not right, okay?.
And sex is just bad, so you should only reserve it for the one that you love.".
Like, that's a good argument..
Like, sex is terrible, bad, wrong, never do it..
Oh, by the way, but when you get married, suddenly everything will be perfect, and you'll have this amazing sex life..
This is how we as a church have often talked about it, and I have to say there is a brokenness within the purity culture..
Because if you have been fed a diet that sex is essentially bad and wrong,.
and when you get married, hey, presto, everything's going to be perfect, we end up actually letting a lot of people down in the church..
Because the reality is, what actually happens is, I can't tell you the number of times that I've pastorally sat with a young married couple.
who are struggling with guilt and shame in the early years of their sex life in marriage,.
because they've been fed a constant rhetoric in the church about how bad sex is,.
and they brought that into their marriage, and it's a wonder that they're now struggling with trying to find enjoyment and pleasure.
in something that God has created good..
Do you see that?.
So I'm convinced that there has to be a better way for the church to be talking about sex. Amen?.
And I want to do that a little bit today..
That's my hope. That's my goal..
It's to maybe paint a broader, bigger, and maybe a better way for the church to be talking about sex,.
and to understand the sex act, and also understand how that fits into a way that we've been created,.
and ultimately how that fits into marriage, and what all that is about..
And I want to say up front, I've only got about 40 minutes with you, I can't cover everything..
And I bet some of you have come in today and you're like, "Oh, I've got my question on sex..
Is Andrew going to talk about my topic of, you know, my question on sex?".

$^{81}$The reality is, I can't talk about everything today, but what I want to do is point a picture broad and glorious and beautiful,.
and invite you into it..
And I want to say a couple of things before I start. Here's the first thing..
The first thing that I think we have to wrestle with is the reality that this topic is for all of us..
No matter who you are in this room right now, this is an important topic for you..
If you're here and you're single and you're happy and you're celibate, it's also for you..
If you're here single and searching, if you're here single and dating,.
if you're engaged and preparing yourself for marriage, if you're married and you're flourishing,.
if you're married and you're struggling, if you're divorced, if you're a widow,.
no matter who it is, no matter where you're from, this topic is for you,.
because this topic sits at the very center of what it means to be created human..
And so this is a topic also for you..
Here's the second thing. I recognize that there is a lot of brokenness around this topic for all of us..
Every single one of us in this room, we bring a story into this topic,.
a journey that we've had perhaps in this area..
And there's hurt involved in that..
There's brokenness and there's some pain and some suffering..
And the reality is we're all in a journey in this area,.
and I just want to recognize and acknowledge that there are feelings about this topic, strong feelings..
Latest statistics would suggest that one in five people have been sexually abused or harassed at some point in their life..
One in five..
And so there are many of us in this room where this is a personal and painful topic,.
and I want to just acknowledge that..
And I want to let you know that I'm not standing here as an expert on this topic,.
nor am I standing here as someone who has complete victory in this area of my life..
I'm here also as a fellow sojourner, if you will,.
in the challenges and the difficulties that this topic presents in our lives..
And so we're in this together..
And one of the problems is that when the church has tried to address this topic,.
they've so often done it from a name and shame perspective..
And some of you in this room, I want to acknowledge this,.
some of you in this room have only ever been felt shame from the church.
because of your sexual history or your sexual background or your sexuality,.
and you've only ever felt hurt or pain from the church..
Perhaps even, if I can be honest, this church..
And I'm convinced of a couple of things..
One is that the church has done a lot of damage in this area over the years,.
perhaps even so the Vine..
I'm also convinced of this, that shame is never the way to help people walk through this topic..
And there have been way too many sermons that are based on trying to shame people into sexual purity..

$^{121}$I hope today is not one of those..
In fact, I'm convinced of this, that I think shame is one of the main reasons.
why there is so much brokenness, and if that's the case,.
then shame surely cannot be the solution to it..
Are you with me, church?.
Here's the third and final thing..
I'm not here today, my assignment here today is not to give you my opinions about sex or relationships..
If you want opinions, you can go to Instagram and hear lots of people give you their opinions about it..
Today is about hearing what Scripture has to say about it,.
what God's glorious vision is for this,.
and for us to sit within that glorious vision in our hearts and our lives..
Is that okay?.
So I hope that my opinion is limited,.
and I hope that the Scriptures are the things that you hear and take away with you today..
And so to do that, we have to start in Scripture..
And I want to start right at the very beginning,.
because the very first thing you need to understand about sex in God's perspective.
is that it is interwoven into the creation story itself..
In fact, sex takes one of the prominent positions in the opening pages of the Bible..
I want to take you to Genesis 1, verse 27 and 28..
Is this okay still so far?.
All right..
It says this, "So God created humanity or man in His own image..
In the image of God, He created him. Male and female, He created them..
God blessed them and said to them, 'Be fruitful and increase in number..
Fill the earth and subdue it..
Rule over the fish of the sea and the birds of the air,.
over every living creature that moves on the ground.'".
These are the fundamental things that God says to His created beings, humanity,.
His gendered beings, He speaks over them, these very critical things..
And what God is doing in these passages, these are famous words for you,.
but I want to break this down in a way that perhaps you've not seen before or understood before..
I want you to really get a grasp of what God is trying to create.
when He's created you and me and humanity..
And I want to do that by showing you the four specific things that God creates.
right at the start that have something to say about who you are..
The first thing He says is that He blessed them..
Very simple words, He blessed them..
Before there was original sin, that happens in Genesis 3,.
there's an original blessing that happens here in Genesis 1..

$^{161}$The very first thing He does after creating humanity is that He blesses humanity..
He pours out His heart towards humanity..
He creates pretty, this is the way of thinking about it,.
He brings Himself into relationship with humanity.
and invites humanity to be in relationship with Him..
That's the first primary thing that happens..
Before He says anything, before He does anything, before anything else happens,.
He actually says, "I bless these people and they are to be in relationship with Me.".
So what basically God does right at the start is He creates this worship element.
of Him with humanity and humanity with Him..
We are created fundamentally to worship,.
to be in relationship with God and in community with Him..
And we receive blessing in that and we give blessing to that as well..
You follow that?.
Then He says, "I have made them in the image of Me, in God.".
God makes us in His image..
This speaks primarily about the idea of identity..
Not only identity, okay?.
Not only are we created to be worshipers, first and foremost,.
in our blessed relationship with God,.
but out of that He says, "You are made in My image.".
He gives us an identity right at the start..
And here's what you have to hear..
That identity is central to who you are before you've done anything..
As soon as you are conceived in your mother's womb,.
you are given identity as one who is created in the image of God..
That means that you have, listen, you have inherent value, worth, love, and goodness..
It's inherent in you..
That's what it means to be made in God's image..
No one is born into this world without any worth, any dignity, any value..
No matter who they are, no matter what they go on to do,.
no matter whatever actions they might have,.
no matter what sexual history they have,.
in the beginning point, you are created with inherent value..
That's the gift of being made human in the image of God..
Are you with me?.
That identity is really important that you understand,.
and that you understand that that identity flows from your relationship with God..
And I'll come back to that in a bit more..
Then He says these words..

$^{201}$He says, "Be fruitful and multiply," or in other translations,.
"Be fruitful and increase in number.".
Here's the thing you've never seen before. I love this..
The first words of God over humanity are not, "Go and pray.".
Come on, church..
The first words that God speaks over humanity are not, "Go and do a Bible study.".
It's, "Go and have sex.".
I'm going to say that again..
Somebody's going to edit this, and it's going to be really bad..
"Go and have sex.".
That's what He says right up front..
He says, "Go and have sex together,".
because this is the way that you're going to express what it is to be in this relationship with me,.
made in the image of God..
And so there is this sexuality that is right there at the beginning of all things..
Now, I want you to hear this..
If you've never had sex in this room, if you're a virgin,.
or if you've made a decision to be celibate in your life for the rest of your life,.
you are still a sexual being who has sexuality..
So this is not talking specifically only about the sex act..
So I want to make sure that those in this room who have never had sex feel included in God's picture..
Sometimes we think if we haven't had sex, we're not fully who we are created to be..
I don't want you to hear that..
You, whether you've had the physical act of sex or not, you are a sexual being with sexuality..
You hold a sexualness to you, because that is what is being created in you in the image of God..
Are you following this?.
Okay, that's really important that you understand that there is sexuality that is created in us as human beings..
He then says, "Fill the earth and subdue it.".
This is God basically saying, "Not only are you created to be in a relationship with me,.
to know that you have inherent worth, value, and dignity, to be a sexual being,.
but you are also created with purpose..
Go and fill the earth, subdue it, bring it under flourishing.".
I don't have time to unpack this theologically, but essentially what God is saying is,.
"Go into the world and co-partner with me in its shalom building, its shalom making..
Go and flourish this creation that I have created..
You humans are to partner with me in the flourishing of creation.".
Now, I want you to see something really important here..
Not only do we have inherent self-worth, we also have active worth..
Come on, church..
We have inherent worth..

$^{241}$So, in other words, if you never picked up any piece of work, if you never bothered to do anything in life,.
you still have inherent worth, love, and dignity from God. Amen?.
But he's also invited you to active worth, active dignity, active love in how we relate with one another,.
how we do our work in the world, how we go and bring the purpose of the gospel into the world..
All of these things are relational things that we get to do and are invited into..
Are you still tracking with this?.
Okay, so this is what it means to be human..
You are somebody who's created to worship..
You have an inherent dignity and worth..
You are a sexual being, and you have purpose, active worth, in what you are doing in this world..
This is all of us, and when you stand back from this, you begin to see a picture of what intimacy is..
Because intimacy flows from all four of these central things about who you are..
So, there is a spiritual intimacy, which connects to the worship of who you are..
There is spiritual intimacy..
There is also here emotional intimacy, because you have this inherent self-worth and dignity,.
which gives you an emotional connection to who you are, to who God is, and to who creation is, and everybody around..
You also have a sexual intimacy, which again, as I said earlier, is not just about the idea of sex,.
the physical act of sex itself, but you are given a sexual intimacy in terms of who you are made in the image of God..
And that actually is really important, because what it says is that our immeasurable worth in who we are,.
created in God's image, and our sexual identity have always been linked from the very beginning..
It's really important you see this, because this actually has a lot to say later when we get into other areas..
Our identity, who we are in the image of God, and our sexuality, those two things are inherently linked from the very beginning..
And then purpose is the idea of relational intimacy..
How we relate to one another, how we relate to the world around us, how we relate to creation..
These four areas are the areas of our intimacy, and what you need to see is that all four of these are interlinked together,.
and none of them can exist alone, separate from the other..
We are created to be unified in these things..
So we have a spiritual intimacy, which impacts our emotional, sexual, and our relational intimacy..
Does that make sense to you?.
But you also have a relational intimacy that also impacts your emotional, spiritual, and sexual intimacy as well..
You can't separate any of these together..
They're all together, all interlinked, and this is the way that it was created..
And God stands back from that and says, "That's very good. That's really good..
That's the way the things ought to be.".
And then, before anything else happens in Genesis 3, something else significant happens in Genesis 2..
God says, "Here's how I've created every human being to be.".
And when this gets expressed in the idea of family, in the way that God sees this being expressed in the idea of marriage and family,.
this is where this takes on a fullness of itself..
So if I take you to the end of Genesis 2, it says, "For this reason, a man will leave his father and his mother and be united to his wife,.
and they will become one flesh.".

$^{281}$The man and his wife were both naked, and they felt no shame..
There it is, no shame..
So this is what happens, and this is why the Bible focuses in on the idea of marriage..
That a couple can come together and be what's called here, "one flesh.".
Let me unpack for you what this idea of one flesh is, because it has everything to link into what these four areas are..
The word that the writer here of Genesis, for oneness, the word he's using is the Hebrew word, "echad," E-C-H-A-D..
And "echad" means oneness formed out of many parts..
Now, here's the important thing for you to know..
The word "echad" is primarily used in the Old Testament to speak about God..
It's the primary word that's used in relation to who God is..
In fact, the central scripture of the Old Testament to speak about God's heart is the Shema..
"Love the Lord your God with all your mind, heart, soul, and strength. Love your neighbor as yourself.".
That, the Shema, it begins with this..
It begins with, "You, O Lord, O God, are one.".
It's the primary statement of the Old Testament position of who God is..
"You, Lord, are one." That's the word "echad.".
What it's saying is the Trinity, the Father, the Son, and the Holy Spirit,.
three somewhat different kind of elements, if you will,.
come together to form a oneness in God that no one can break..
Do you see this?.
And this is the important thing..
"Echad" is used in this relationship with God as the primary picture of who God is..
He is the Lord our God is one..
Three in one..
And there is a bond in that that cannot be shaken..
It cannot be broken..
They're in an intimacy and a relationship together that is the ultimate picture of oneness..
So when the writer of Genesis begins to think about,.
"How do I describe the coming together of a couple in marriage?".
He goes, "I'm going to link it to who God is.".
So what you've got to understand is that this is not just speaking about marriage..
What actually is happening here in Genesis 2 is speaking about God as well..
It's trying to bring marriage and God and the two of those things together as this one kind of idea..
When a couple comes together in marriage,.
they become "echad" in the same kind of way that God is "echad.".
And when God comes together in a oneness that no one can separate,.
so these two can come together into a oneness that no one should separate..
So marriage becomes the primary picture in the world of the "echad" of God..
You still following this?.
And this is a beautiful thing..

$^{321}$Because what it means to the couple getting married is that their relationship together.
is lots of separate parts coming together to form one whole.
that is a picture to the world of who God is..
So marriage is not just about the sex act..
But the sex act is included in this idea of coming together of the spirit, of the body,.
of the emotions, of our relationship, of everything that we are comes together with the other person.
that we're married to and in the sexual union, in the act of sex,.
it is the ultimate outcome of the "echad" that is seen in God..
This is why the Bible makes a really strong point that sex is reserved for marriage..
Because if sex is the coming together, remember for the Bible, sex is never just a physical act..
That's one of the greatest lies of the world, that sex can just be physical..
The Bible stands against that picture..
Because the Bible says sex is part of this one flesh, "echad.".
So it is mind and spirit and emotions and body and everything else all coming together in one..
And if that's the way it is, no one should be able to break that because that's never broken in God..
And if no one should break that, then you should not go do that with lots and lots and lots and lots of people..
Because if you do that, what you're actually showing the world is a separateness, not a oneness..
This is why sex is reserved for marriage in the Christian tradition and faith..
Because we get this beautiful gift to show the world oneness..
When done in the right way, sex and marriage stands boldly against the tendency in this world to see more separateness take place..
Now does that mean that Christian marriages are perfect?.
No..
Does that mean that we don't still struggle with brokenness in marriage?.
Of course it means all those things..
But that doesn't negate the reality that God has created this institution as the only place for sex to take place..
Because our physical act of sex is a part of our echadding in the world,.
showing the world what the oneness of God is like..
So when this all comes together, I want you to see how this then all flows together..
Because here's the thing, sexual brokenness is essentially the breakdown of the oneness, the echad that God has created for it..
You need to understand if this thing here gets broken first,.
when our relationship with God and our intimacy with Him, when that breaks, that always impacts this..
I need you to see the flow..
When this breaks, that impacts this..
Suddenly we don't feel loved anymore..
Suddenly we don't have the solid sense of inherent worth and value..
Because that solid sense of inherent worth and value only ever flows from intimacy and relationship with God..
God has imputed that to us, made in creation with Him, image of Him..
But when we break our relationship with Him, the first thing that takes place is us going, "I don't feel loved anymore..
I don't feel like I've got a value anymore.".
And not only that, but I don't think others have the same value I thought they had before..

$^{361}$And here's the inevitability, when this breaks, this breaks, and then that will always break..
One of the reasons why we have such great sexual brokenness in the church,.
there is so much sexual brokenness, I would say, in the church..
Primarily because there has been a breakdown in the fundamental relationship with God here,.
which then has changed our belief and our acceptance of who we are and who others are..
When that happens, we will find ourselves pursuing and seeking sex as a way to get back how we feel about ourselves..
We will then use sex as a tool to try to feel more love, try to feel more acceptance,.
try to feel more value in the world around us,.
which inevitably is going to break our relationships more with the people around us..
Because when you have a selfish desire to use this to get that back, you're going to break this as well..
Following?.
And so the way that God has created this is He's saying, "Can you just see the vision that I've created?".
When this is strong, you will have a strong sense of who you are..
And when you know that you're loved and you have a strong sense of who you are,.
then you can reserve the sexual act for marriage,.
believing that you want to show the world the oneness that there can be,.
that two have been brought together..
Let no one break that apart..
And in that, in the security of that, you can raise children, have children,.
flourish children, and have relational connections with those around us..
And you're not using relationships or identity and emotional brokenness and manipulation.
to try to satisfy something here that can only truly be satisfied from here to there to there..
That's the beautiful invitational vision..
And here's the brokenness..
We so often turn people into objects..
The majority of sexual brokenness that happens in the world happens because we objectify people..
This is really powerful..
I want you to track this with all the theology I've been giving you this morning..
If Echad is about oneness, about separate parts coming together,.
and if that oneness is best seen in marriage and in the way that that happens within marriage,.
then here's what objectifying other people is all about..
It's about separating the parts again..
It's the opposite of Echad..
It is basically saying, "I'm not interested in you for your emotions, for your relationship, for who you are..
I'm only interested in your body..
I'm going to separate all the other elements of who you are,.
and I'm just going to focus on your body for my own sexual gratification..
I think your body is attractive, and because I'm attracted to your body,.
I'm going to now use that to gain pleasure and satisfaction for myself.
because ultimately I'm broken here, and when I'm broken here, I think everybody else is broken,.

$^{401}$and I will start doing the opposite of Echad..
I will separate people down to just their physicality,.
and I will use that and abuse that to try to make myself feel better.
when this is the thing that's truly broken.".
Is this helping anyone?.
That's the glorious vision, and that's why there is so much brokenness..
We've gotten our intimacy in the wrong way and in the wrong order..
Let me finish just by showing you a little demonstration of this..
I got this demonstration from another pastor..
Pastor is Robert Madu. He's based in the U.S., although lots of other pastors have done it since,.
so I just wanted to tell you up front I'm gleaning something today..
Remember a few weeks ago? This is a glean, but I've changed it a little bit.
to make it more applicable to what we're talking about today..
I want you to see these boxes of intimacy..
The largest box is always the spiritual one because that is the primary one.
that has been created for us..
That is the top left-hand corner here..
That is the thing that is always for us..
Now, there are other intimacies that we have..
We have, for example, the emotional intimacy..
Emotional intimacy is a big one because there is also so much for us to say and do.
and understand about ourselves..
As I've talked about earlier, we have this idea of relational intimacy..
That's a little bit smaller box because our relational intimacy is not just dependent on us..
It's dependent on other people..
There are boundaries sometimes to that..
We're not able to always be in relationship with the people we want to be in relationship with..
That one's a little bit smaller..
Then over here, we have the sexual intimacy box, which is the smallest of all.
because this is the one that has the greatest boundaries around it in Scripture,.
the one that has the context around it, and also it's the one that--yeah,.
just is small because--yeah..
Anyway, it doesn't always last very long..
That's probably another way of saying it..
Okay, so--now, let me explain to you--sorry, sorry..
I should have said that..
There was bound to be something I was going to say that would come out wrong..
Okay, so I want you to see this..
Here's how I think we often approach relationships..
I would say in the world particularly but also, unfortunately, in the church,.

$^{441}$we start with a sexual premise to begin with..
That person's hot..
I'm attracted to that person..
There's a desire for that body or there's a desire for that satisfaction..
Right off the back of that, we throw on some emotional intimacy.
because what happens when you have physical sex with someone.
is there's an incredible, powerful link of emotional insecurity..
But what you do is you connect your sex to the emotion,.
and then that way, those things become unhealthily connected now.
so that if you're having sex, you feel emotionally connected to that person..
When you're not or that's broken, then you struggle with your emotions.
about that relationship..
In the world, often, we then try and throw relational intimacy on top of that.
because it's like we try to buy first..
We get emotionally connected..
We then go, "Well, I guess maybe we should start dating or hanging out more.
or maybe we should try and be a little bit more committed.".
I think in the church, sometimes this is the driving factor..
We then find this, and then we decide that we need to get engaged.
or something like that..
Then ultimately, if you're not a Christian, then that's where you'd stop..
But us Christians, we then want to slap some spiritual intimacy on top,.
and that's what happens..
Are you with me?.
Now, here's the biblical vision that God gives,.
which is the one we need to understand..
It starts with the worship..
It's got to start with the spiritual intimacy first and foremost,.
which then tells us who we are and who others are..
That's the identity piece that's there, which then, of course,.
enables us to be in healthy relationship with those around us.
as well as potentially a life partner..
If marriage does eventually come, then we can enjoy.
the incredible sexual intimacy that has been provided out of it..
That's the biblical picture..
That's what I want to inspire you with today..
For those that are single here and perhaps never going to get married.
or maybe have decided in celibacy,.
it doesn't mean you're one-third or three-fourths a human being, remember?.
Because you are a sexual person,.

$^{481}$whether you're actually engaged in that activity or not,.
you're a whole person, and you can still have healthy sexuality.
even if you're celibate or not ever married.
or if you're a virgin for life..
Does that make sense to you still?.
But for those that do get married and can engage in that,.
you then put that on top as a way of expressing this oneness,.
the echad of who God is in marriage..
That, my friends, is the biblical vision..
I think as we lean into this and get inspired by this,.
again, it doesn't mean that every Christian marriage is perfect,.
but it does mean that we submit ourselves under the grace of God.
and invite Him to help us..
Christianity is not behavior modification..
Christianity is inviting the vision of Christ and the gospel.
into our hearts and our lives and saying,.
"I want to live like that. God, give me the grace to do so..
And forgive me where I so know that I so often fall short,.
but that does not negate the beautiful picture you've given me..
Help me. Help me to be one who can honor the echad.
of who you are in this world.".
So my encouragement to you today is not to feel bad..
Hopefully, you haven't felt shamed..
Instead, I hope you felt inspired..
I hope you've been welcomed into a beautiful, bigger vision.
of who you are yourself..
And regardless of what state you are in life right now,.
what status you might have in relationships,.
that you would say, "This is what I want for me and my family.".
And as we come together in that, we pray for each other.
and we ask God to heal us of all the brokenness that is there..
Because if you're anything like me, there's brokenness there..
And we come around each other and encourage each other.
and love on each other and pray for each other..
And we live because the world needs this narrative..
This is hope. This is how families are created..
This is how children are to be nurtured and brought into the world..
This is the way it should be..
And God stands back from that and says, "It is very good.".
Can I pray for you? Let's pray..

$^{521}$Father, I thank you so much for every person here..
And Father, I thank you for the fact that you've made them human..
And as we've seen this morning, that's such a glorious picture..
And you've created us sexual beings..
And you've given us this gift of sex and all of its beauty and wonder and mystery..
And Father, we pray that you would help us to be good stewards.
of the intimacy you have blessed us with..
Spiritual, emotional, relational, and sexual..
And that in our stewardship of that intimacy, we would remind ourselves.
that it is our relationship with you, out of which flows everything else..
And when that relationship breaks, other things also begin to fall..
And I wonder whether you would just take a moment here this morning,.
just to allow God to speak to you about you and Him..
That spiritual intimacy with Him..
Because how you think about yourself and how you think about others,.
and therefore your sexuality, and therefore your relationships,.
all flow from that point..
You're blessed. You have inherent worth and dignity and value..
He loves you..
But when that breaks down, everything else breaks down behind it..
So maybe today's response is just for you and Him..
Just to take a moment now..
Maybe for some of you, you want to worship Him..
You want to just offer your life and heart to Him again..
For some, it might be you want to listen to what He has to say to you today.
about this area of sexual identity and sexuality and sex itself..
Perhaps for some of you, it's opening your heart.
to what the Holy Spirit might want to challenge you about..
But it's you and God always, first and foremost..
And out of that might come the need for other things..
But right now, I want to encourage you to connect with Him..
And in that intimacy, you would hear His voice..
[MUSIC PLAYING].
\newpage



\section{}
\label{sec:m34r3U0C7b0}
\textbf{2024-02-14 Putting On New Clothes [m34r3U0C7b0].mp3}
\newline
\newline
連結: \href{https://youtube.com/watch?v=m34r3U0C7b0}{\texttt{ https://youtube.com/watch?v=m34r3U0C7b0}} ~~~~ 語音日期: 2024-02-14 
\newline
\newline
\hyperref[sec:SARdI1NGpaQ]{\small{< < < PREV SERMON < < <}}
~
\hyperref[sec:index]{\small{[返主目錄]}}
~
\hyperref[sec:qcPJWbnjxGQ]{\small{> > > NEXT SERMON > > >}}
\newline
\newline
$^{1}$Thank you, worship team, and (speaks in foreign language).
God's blessings be with you,.
and may you be filled with the energy.
of horses and dragons, something like that, right?.
Anyway, it's happy new year,.
and I'm glad to see it's an honor to be here this morning..
We're gonna be in Colossians chapter three,.
so if you have a Bible,.
if you wanna turn your attention to the screen,.
Colossians chapter three is where.
we'll be talking about today..
I'm gonna read it first,.
and then we're gonna carry on with the message..
It says this, this is the word of the Lord..
"Therefore, as God's chosen people,.
"holy and dearly loved,.
"clothe yourselves with compassion,.
"kindness, humility, gentleness, and patience..
"Bear with each other and forgive one another.
"if you have any grievance against someone..
"Forgive as the Lord forgave you,.
"and over all these virtues, put on love,.
"which binds them all together in perfect unity..
"Let the peace of Christ rule in your hearts.
"since as members of one body,.
"you are called to peace, and be thankful.".
So it is new year,.
and if you've been celebrating Chinese New Year,.
it's probably been quite a busy time in the past few days,.
and the next few days coming up, right?.
Lots of visiting families, visiting friends,.
all that kind of stuff..
Certain things just have to be done, right?.
You maybe had to do a good spring cleaning of your house,.
lots of dinners..
We had our family over last night for dinner,.
and great, we had some hot pots,.
but it isn't Chinese New Year.
until your mom starts singing karaoke really badly..
Actually, my mom's a decent singer,.

$^{41}$but she was singing William Fong,.
and then I turned up the volume..
What was she singing?.
♪ Da da da dee dee new fong ♪.
♪ Da da da da da da ♪.
You know, all those kind of old Cantonese pop songs.
that she likes, so we had a good time last night..
But you know, you probably got your hair cut,.
stuffing Lai Si, passing out Lai Si,.
all these traditional aspects of Chinese New Year.
that we celebrate in this time..
And one of the quintessential things we do.
at Chinese New Year is wear new clothes..
Buy new clothes and wear new clothes, right?.
Especially new red clothing..
It's a sign of blessing, it's a sign of the festive season..
We love doing these things, okay?.
So, what am I wearing today?.
This is Uniqlo, okay?.
Oh, yeah..
It's not that new even, okay?.
But it is Uniqlo, okay?.
I'm a Uniqlo kind of guy..
They make good clothing for short, muscular Asian guys, okay?.
(audience laughing).
So I like shopping there, okay?.
But it's also kind of ironic..
Uniqlo, it's supposed to be unique clothing,.
but because they make a thousand.
or a million of the same thing, right?.
You buy new clothes at Chinese New Year,.
you walk out the street, everybody looks the same, right?.
We're all wearing Uniqlo..
There's nothing quite unique about it..
But they are new clothes, right?.
We buy new clothes at Chinese New Year.
as a symbol of the new blessings,.
the new things that we expect to come, right?.
This is why we get new clothes..
So it got me thinking about clothing,.

$^{81}$and not just Chinese New Year..
Like, you know, today we're all dressed in our best,.
wearing red and all this kind of stuff..
But clothing, actually, we have a fascinating relationship.
with clothing and fashion, right?.
It's become a very inextricable part of our lives..
Interestingly, though, clothing was never.
a part of God's original plan, all right?.
Genesis tells us that Adam and Eve.
were naked and unashamed..
The word in Greek, naked, means naked, okay?.
Naked and unashamed, right?.
It was only after they sinned that they realized.
that they were naked, and God clothed them, right?.
The Lord made garments of sin for Adam and his wife,.
and he clothed them..
And just like that, God created the first fashion line ever..
Garments of sin, the fall edition..
(audience groans).
You see what I did there, okay?.
And this is why, yeah, okay, thank you, Ken, thank you..
Every morning now, we have to go through.
the painful decision of picking out what clothes to wear..
But clothes are significant, right?.
Certain places we go have certain dress clothes,.
different seasons has different clothing, right?.
Clothing is very much a part of our cultural identity,.
like this I'm wearing today, right?.
Something a bit more traditional, a bit old school,.
but it's part of our cultural identity..
Helps us to represent our ethnic roots, right?.
Or think about going to a prom or an awards ceremony.
or a bride on a wedding day,.
the exciting process of finding the perfect dress..
Right, clothing can also be sentimental..
A lot of us have the one T-shirt or that hoodie.
we've been holding on for like 25 years..
We just can't get rid of it because it means so much to us.
and it's so soft and comfy..
Clothing can even help to define history..

$^{121}$Victorian England, right, small waist, big shoulders,.
that was what it was all about..
'60s and '70s, all about peace signs and ponchos,.
that was the style back then..
'80s and '90s with the neon colors.
and the high-waisted jeans,.
which are sort of making a comeback these days, I see..
In the 2000s, I went through a phase.
where I thought I was a gangster rapper, okay?.
And so I would wear baggy jeans, the big T-shirts.
and the bandana around the front of my head like this, okay?.
Yeah, that was exciting, right?.
That was, okay, let me tell you about this, okay?.
So I thought I was Tupac, okay?.
But I wasn't quite cool enough to be Tupac,.
so my friends, true story, used to call me half-pack, okay?.
Because I was trying so hard to be like Tupac..
But you know, clothing is important..
That's what I was trying to say..
Actually, sometimes I think it gets a bit ridiculous..
We take it too far..
Right, you see a Wan Chai Market T-shirt,.
\$50, you can get a nice white T-shirt,.
but you put Balenciaga on it, okay?.
And suddenly that T-shirt is worth \$500, all right?.
They are shoes, handbags.
that cost literally more than a house..
I say all this just to make the point..
Clothing is significant..
It's significant because it marks us out.
and defines us in some sort of way..
As you think about new clothes and the Chinese New Year,.
the big question for us this morning is this..
How have we clothed ourselves?.
Like how does our clothing define us as a family?.
And when I say family, I mean us here today.
as followers of Jesus, that divine church in this community..
What is it that marks us out?.
How do we clothe ourselves?.
Like when people look at us from the outside,.

$^{161}$how do these people clothe themselves?.
What defines them?.
What do you think people are saying?.
This is an important question to ask.
as we step into the new year,.
because we've been talking,.
we've come out of our January fast,.
we've just come out of the Exodus series not too long ago,.
and so we've been talking about a lot of concepts.
and all these kind of things,.
but it's time, I think, just from going,.
we need to go from thinking and hearing.
to actually start doing and being, right?.
In this season where we've been talking about intimacy,.
we actually have to start living lives of intimacy.
rather than just hearing about it..
Lives are actually deeply connected and intimate.
with God and the people around us..
'Cause what I hope is that when people look at us.
from the outside, they're saying to themselves,.
you know what, this group looks good..
I like what they're wearing..
I want to be a part of this community..
I wanna be like those people too..
So you could say this passage we're studying today.
is like the ultimate fashion guide for the church, okay?.
For followers of Jesus, this is the ultimate fashion guide..
Because in the passage we just read,.
Paul uses clothes and fashion as a metaphor.
to describe the certain virtues and character traits.
that we should put on as followers of Jesus..
He says this, first of all..
Therefore, as God's chosen people, holy and dearly loved..
Let's pause right there for a minute..
Paul starts by reminding us about who we are..
So let's remind ourselves about who we are..
If you're sitting here today,.
or if you're listening online or in a podcast later on,.
and you've given your life to Jesus,.
you are a part of God's chosen people..

$^{201}$You are holy, which means you have been set apart..
This is what the word holy means..
You have been wholly set apart by God,.
and he loves you dearly..
You are special to him..
This is who you are..
This is your identity..
So it has to mean something to you.
if that's what you want to embrace..
You are holy and dearly loved..
Pastor Andrew mentioned this last week.
when he reminded us it's out of God's intimate relationship.
with us that we can truly find our identity.
in who he's created us to be..
He loves you, and he wants you to love him..
You're set apart, which means you are different.
to who you were before you gave your life to Jesus..
You are different to those who don't know Jesus yet..
We're like people that have taken off.
the shabby clothes that we used to wear..
Paul defines this as rage, anger, malice,.
slander, filthy language..
These are things he says in the verses before this..
We've taken off those old clothes.
so that we can put on new clothes..
This is our starting point,.
because it reminds us that we are being called.
into something not that we do.
with our own ability or power..
God starts this process of clothing us..
We are called into this not by our own ability,.
not by our own power..
We have been chosen by God not because we are good,.
but because God has been good to us, amen?.
We have been called holy not because we are perfect.
and sinless, but because God has given us his holiness..
We are loved not because we're lovable,.
but because God chooses to love us no matter what..
This is the gospel..
This is the good news,.

$^{241}$and this is our starting point for today..
We are chosen, holy and dearly loved,.
changed and renewed, which is why we must do this..
We must clothe ourselves with compassion,.
kindness, humility, gentleness, and patience..
Just one more illustration..
Maybe you can think about it this way..
Imagine you've done some exercise,.
you've gone to the gym, gone for a run.
or something like that, you're all sweaty and gross,.
so you get cleaned up, you have a nice shower, okay?.
You're clean..
You're not gonna put on those sweaty clothes again, right?.
You're gonna go to your closet.
and put on new, fresh, clean clothes..
And so in the spirit of Chinese New Year,.
these are the clothes that I feel like God's trying.
to encourage us, all of us, to put on.
as we step into the year of the dragon..
First thing he says is this..
We have to put on clothes of compassion..
Now, if compassion was an item of clothing,.
it might be a soft, really soft, well-fitting T-shirt,.
something like that, something that covers your body, okay?.
The word compassion actually translates.
as like the bowels of compassion, okay?.
It's something a bit more passionate than just compassion.
because in ancient thought, right,.
they thought that it's the bowels.
that's where emotion and feelings came from, right?.
And it sort of makes sense, right?.
Think about it this way..
When you have a strong emotional reaction to something,.
right, you feel it right here, right, in your gut,.
in the pit of your stomach..
That's why we say something, it's like gut-wrenching.
when we see it, right?.
You feel it right here..
So compassion is then when we're able.
to sympathize with someone..

$^{281}$We get that feeling..
We're able to sympathize with them so much.
that it affects our innermost being,.
that we have to do something about it..
You ever had that when you see something,.
see someone, you just get that feeling,.
you know, it's like, this isn't right..
I'm feeling something and I have to do something about it..
It's what the Gospels tell us when it says.
Jesus was moved with compassion..
When he saw people who were lost,.
when he saw people who were sick,.
when we saw people who were hungry,.
it was more than just that he felt the feeling,.
but it was more than that, right?.
This compassion led him to reach out,.
to love them, to heal them, and to feed them..
So church, have we put on compassion?.
Like when we look at the world around us,.
when we see people today who are lost and sick.
and hungry and helpless, mistreated,.
are we so affected by it that we feel that.
and we actually have to do something about it.
to try and make it right?.
The first thing we need to do, church,.
is put on compassion together..
We put on compassion so we're not just loving.
in word and tongue, but in deed and truth..
For this is what true compassion is, love in action..
Next thing Paul tells us, encourages us to put on is this,.
kindness..
Kindness as an item of clothing.
could be like a pair of gloves, right?.
Because kindness has to do how you reach out.
and react to others..
You could define kindness as having the attitude of Jesus.
towards people around you..
And again, throughout scripture we see that Jesus.
is always kind to people,.
always ready to help in times of need,.

$^{321}$showing grace and mercy when he could have chosen.
wrath or anger instead..
It's interesting, right?.
The Bible tells us that it's the kindness of God.
that leads people to change..
It's the kindness of God that leads people to repentance,.
to recognize their sin and their need for him..
So in the same way, church,.
we've been encouraged to put on kindness..
We've been commanded even to put on kindness..
Because when we put on kindness, it's like saying.
we now have the mind of Christ towards others..
Because Jesus has been kind to us,.
we are called to be kind to everybody..
So when that person is on the MTR and is crowded.
and they bump into you, your reaction is not just.
you step on the back of the shoe a little bit.
so they trip as they fall, as they get up..
I don't do that, by the way, okay?.
I'm not teaching you guys anything, okay?.
But instead, you choose kindness..
And if kindness that truly leads people to repentance,.
then kindness is actually the number one thing,.
one of the best things we can do.
to reach people with the gospel..
When the world that's hurting and lost in sin comes to us,.
when people approach us with their messiness and their lies,.
how do we respond?.
Right?.
Do we bash them with the Bible, with rules and regulations?.
Yeah, I told you so..
It's because you haven't been doing this, this, this..
Now look, you're a mess..
Is that our reaction?.
Do we shame them for their sin?.
Or which do you think will be more effective?.
Maybe it's reaching out in kindness.
to sit with them, to understand them..
I would say that being kind is the most powerful thing,.
one of the most powerful things we can do..

$^{361}$When the world reaches out to us, we reach out in kindness..
Next, Paul tells us to put on humility..
So humility I see as like a sturdy pair of socks, right?.
Nothing fancy, but without them,.
your feet get stinky and sweaty, right?.
You know the Cantonese saying,.
(speaking in foreign language).
If you wear socks without shoes, right,.
it makes your feet stinky and it's really gross, okay?.
That's basically what it translates to..
But I say this in the same way, okay?.
If we do anything without humility, it's gonna stink..
If we do anything without humility,.
no one wants to come near it..
People don't like hanging out.
with arrogant and proud people, right?.
Okay, maybe you can think about it this way..
If kindness is looking at others in a Christlike way,.
like I just said,.
humility is looking at yourself in a Christlike way..
C.S. Lewis puts it this way,.
"Humility is not thinking less about yourself,.
"but thinking about yourself less.".
So humility means that we're willing.
to give up our own rights.
so that others can have their rights..
But after all, this is what Jesus did for us..
Leaving his place in heaven,.
he humbled himself, becoming obedient to death,.
even death on a cross..
And humility is essential.
for your relationship with Jesus, first of all,.
because how are you gonna follow somebody.
if you think that you know better than them?.
But humility is also essential.
for our relationship with each other,.
because how are you gonna sit and listen to each other.
when you're already looking down at them?.
So church, have we put on humility.
so we can walk out our relationship with Jesus.

$^{401}$and walk out our relationship with others?.
When we put on humility, we're saying,.
"There's no judgment here..
"You are welcome to be with us.".
We put on humility..
And then we have gentleness..
You know, in the ancient world where Paul lived,.
gentleness was sometimes seen as something.
to be looked down upon,.
because it was seen as a weakness..
Nobody ever won wars and battles by being gentle..
Alexander the Great wasn't known for his gentleness..
He was known for his brutality..
And in those days, and even now to some extent,.
brutality, force, anger is how you showed strength and power..
That's what gets things done..
But if you look at gentleness as like a scarf,.
it might help us to put this in context a little bit..
Scarves are like a soft, fluffy piece of material,.
but when it's wrapped around you right,.
it protects you from the cold..
Even in the middle of a really cold snowstorm,.
if you have a scarf around you, it makes a big difference..
Helps you feel warm and safe and protected..
So gentleness then you can see,.
it's like strength in disguise..
It's like a wild horse with all that power.
that's been tamed so you can control it..
Think about it this way..
When you get angry, when you get triggered,.
what's the easiest thing to do?.
The easiest thing to do is to react right then, then..
I'm learning gentle parenting at the moment, right?.
And it always talks about regulate yourself.
before you speak to your child, okay?.
The easy thing to do for me though is just shout, right?.
Just to react..
Shut up, be quiet, go to your room, things like that..
Match your volume, their volume with your volume..
Match their anger with rage..

$^{441}$When you get hit in the face, the easier thing to do.
is to hit that person right back..
So you have to understand when Jesus says.
if someone hits you, you're supposed to.
turn the other cheek, right?.
Think about it..
That takes a lot more strength, self-control..
To restrain yourself..
That's not an easy thing to do..
That's not a weak thing to do..
But it is the gentle thing to do..
And so gentleness is essential within this community.
for without it, we would constantly be.
at war with each other..
Without gentleness, we would be dealing.
with each other harshly all the time..
Without gentleness, we would just bully our way,.
not considering others, but just doing whatever we wanted..
But the strength of gentleness,.
it's going to bring people together..
A community that shows gentleness to each other.
is one that reflects God's heart towards us..
A church that's known for being gentle.
is where people are gonna come and feel safe.
with their feelings, with their emotions,.
with their mess that we just talked about..
Because they know they're gonna be met.
with gentleness for their burdens..
But don't hear me wrong, church..
It's not a weakness..
It doesn't mean we just let people run over us.
and do whatever they want..
That's not the point..
Gentleness is about showing power and strength.
that does not condemn and put down and belittle..
But rather when we reach out in gentleness,.
we're able to reach people with love,.
draw people in so then they can receive.
the help that they need..
So we put on gentleness..

$^{481}$Next four says we have to put on patience..
Patience is like a pair of sunglasses perhaps..
You could think about it that way..
Right, when you put on sunglasses,.
it cuts out the harshness of the light.
so you can fully open your eyes.
and clearly see what's in front of you..
In the same way, right, we need patience.
in order to see things and people.
for what and who they truly are..
Patience is what's gonna help you stop,.
maybe seek advice if that's needed,.
get into a good headspace.
before making a decision about something..
Patience is what's gonna help you stop.
and listen to yourself before simply just blurting out.
whatever comes to mind..
Patience is what's gonna help you wait on the Lord.
instead of just going ahead with whatever plans.
that you had for yourself..
Patience is what's gonna allow you.
to walk through seasons of endurance,.
refinement, maybe even suffering sometimes..
If it's for your own good..
So that's for you..
But patience is also a great gift and blessing.
that you can offer to others as well..
For if you have patience,.
it's gonna allow you to be able to sit in the valley.
with those who are hurt..
Patience allows you to never stop,.
cease praying for those around you.
as scripture calls us to do..
Patience will allow you to be able to hold a calm.
and civil conversation with someone.
even if you fundamentally disagree with them..
We are called to put on patience.
because God has been patient to us..
Scripture says this,.
the Lord is not slow in keeping his promise.

$^{521}$as some understand slowness..
Instead, he has been patient with you..
We're called to put on patience.
because the Lord God is compassionate and gracious,.
slow to anger and abounding in love..
So church, let's put on patience..
Finally, we have love..
Love is what binds everything together..
Love is like the big overcoat.
that protects and covers everything..
Paul talks about love often..
He describes love as the greatest of all things..
He says when all things have faded away,.
the only thing remaining will be love..
Love is the one thing that he guarantees will never fail..
This is also important because here's the truth..
As much effort as we do to put on these new clothes,.
we are not perfect people..
So we're gonna need to forgive each other sometimes..
If you take the illustration,.
sometimes we're gonna have a wardrobe malfunction, okay?.
Our patience runs out..
We might not be as kind as we hope to be..
We may respond in a harsh word.
instead of just taking the time to listen to someone..
We may fail to truly understand each other..
We act out of anger rather than kindness..
When this happens, it's our love for each other.
that's gonna allow us to forgive..
It's our love for each other that's what's gonna ensure.
that we don't break apart when the hard times come,.
but rather seek to honor and encourage each other.
to strive forward to be better..
So love, therefore, is like the ultimate protector..
And without love, we would be nothing..
Finally then, Paul tells us this,.
that the peace of Christ rule in your hearts..
The peace of Christ you can see maybe as a crown.
that sits above the whole outfit..
Because the important thing is this, church..

$^{561}$These aren't just virtues we follow because they're good..
Good rules to be a good human being, right?.
That's not the point..
The point is not for us just to become good people, okay?.
Rather, we aim for these things.
because we know who we belong to..
We put on these things, like I said in the beginning,.
because the peace of Christ, we are God's people..
The peace of Christ rules in our hearts..
We do these things because our hearts.
have been so captivated by Jesus..
He is the one we're trying to live like..
He is the one that rules and directs how we live..
That's what's gonna make the ultimate difference..
It's not just about being good people..
It's about being Jesus to other people..
That's what's gonna keep us together as one body, right?.
It's not just about looking good..
We wear these clothes because we are ruled by Jesus..
We belong to him, we follow him..
If you've noticed, all these things are things.
that Jesus lived out as he was on this earth..
They're simply imitating who Jesus was..
He is the one who rules our hearts,.
so he is the one we live for..
Because ultimately, it's this..
Fashion trends will come and go, right?.
Cargo shorts aren't cool anymore, guys, okay?.
Because if we're trying to look like,.
if what we're trying to do is just look good, right,.
that's gonna go out of fashion someday..
But what we're trying to do, if we look like Jesus,.
if we're trying to be like Jesus,.
that's never gonna go old..
That's never gonna run out of style..
So, that let the peace of Christ rule in our hearts..
And then Paul says this, be thankful..
Now, on a day like today, I wanted specifically.
to end with these words because I think it helps us.
to keep a good perspective on Chinese New Year..

$^{601}$Like I said, this is when I started,.
Chinese New Year, we step into this season,.
and it's very much about blessings..
Actually, it's more than just blessings..
It's really about getting rich, right?.
Gong yei fat choy, you know?.
I hope you get rich, I hope I get rich,.
and we hope everybody gets rich, okay?.
That's the sort of general mindset.
we step into this season upon..
And therefore, when we just keep thinking.
about blessings all the time,.
when we keep on thinking about blessings.
and blessings and blessings, if we're not careful,.
as Pastor Andrew reminded us a few weeks ago,.
it's easy for this to interfere.
our relationship with God and others..
We start to want things of God more than God himself..
And then when it comes to these things,.
we feel like we wanna look good sometimes,.
just so we look like, oh, you know,.
I've been doing all these things,.
I've been a good Christian, God, bless me more..
Give me more, I've been following all the rules..
But like I said, it's not about that..
And one of the best ways we can guard ourselves.
in that mindset is to be thankful..
It's impossible to be resentful and thankful.
at the same time..
And as followers of Jesus, we don't need to go.
to Wang Dai Hsin or Tze Kung Miu to spin the wheels,.
to ask for those gods, to burn incense.
for good fame and for good fortune, right?.
We don't need someone else to tell us.
what our fortune is gonna look like for the upcoming year,.
because we follow Jesus, and that's enough..
Because we are thankful for the things.
that God has given to us, and we can be content,.
and that's enough..
And if you wanna remind of how thankful we need to be,.

$^{641}$just take a look around you, church..
These are your brothers and sisters in Christ,.
your family, your friends in Christ,.
sat around you, gathered here today..
This building, this space we have right here,.
is a sign of God's faithfulness throughout the years.
that he's blessed us with already..
We just need to stop and think for a minute,.
and help us to be thankful..
Do we see fellow children of God gathered together today,.
dressed in the best of God's clothing,.
not just literally, but in the ways.
that we talked about just now?.
So as a family, church, I wonder if we could spend some time.
blessing each other in this way this morning..
One of the things we do at Chinese New Year.
is to visit friends and family, to bless them,.
to pray with them, to spend time with them..
So we're actually gonna set aside some time.
to do that right now..
Now, there's a lot of people in this room,.
and if you're out in the overflow watching online,.
you can try and do this too..
But maybe you came in with friends or family today,.
maybe there's people around you..
Stop and take a moment right now..
Remember those things that we just talked about,.
that you're clothed in..
And then as you look around you,.
ask God, okay, how can I bless this person today?.
Maybe there's a word of encouragement,.
maybe it's a prayer, maybe it's a blessing.
you can say over them, something God places in your heart,.
or something that you think would be a blessing.
to your family in Christ right now..
Okay, the band's gonna play softly..
We're gonna set aside about five, 10 minutes to do this..
Okay, so church, just take a moment, think about it..
All right, and then start interacting.
with the people around you..

$^{681}$Spend some time with your church family this morning.
in order to bless them, to pray for them, to encourage them..
Let's bless each other by praying for them..
Let's bless each other by showing compassion,.
kindness, humility, gentleness, and patience,.
and see what God has to say to us.
as a community this morning..
So come on, church, I encourage you..
The band's gonna play..
We're gonna spend a little bit of time doing this,.
and then I'm gonna come and pray for us.
in just a little while, okay?.
\newpage



\section{}
\label{sec:plgL9V4r1NI}
\textbf{2024-02-19 The Subversive Act of Generosity [plgL9V4r1NI].mp3}
\newline
\newline
連結: \href{https://youtube.com/watch?v=plgL9V4r1NI}{\texttt{ https://youtube.com/watch?v=plgL9V4r1NI}} ~~~~ 語音日期: 2024-02-19 
\newline
\newline
\hyperref[sec:qcPJWbnjxGQ]{\small{< < < PREV SERMON < < <}}
~
\hyperref[sec:index]{\small{[返主目錄]}}
~
\hyperref[sec:tf_Wm93QZHI]{\small{> > > NEXT SERMON > > >}}
\newline
\newline
$^{1}$Have a seat..
Well, I wanna continue our conversation today on intimacy,.
this theme that we've been having over us.
as a church community for this past year..
And I wanna talk about what I would consider.
perhaps one of the most critical things that is involved,.
not just in your intimacy with God,.
but in particular, your intimacy with one another..
I wanna talk a little bit today.
about what that intimacy can look like with one another..
And the topic that I'm talking about,.
I think has probably the greatest power.
to impact a church community.
with the community in which it's planted in..
So what we're talking about today.
has a significant impact into what it is to serve.
and to bless Hong Kong and the society around it..
And not only that,.
but when you do what I'm talking about today well,.
when you do it well in your own life,.
it also has the power to bring you.
into deeper moments of freedom and joy.
that you perhaps can't experience in anything else..
This topic is really important..
The topic is what I would call.
the subversive act of generosity..
Now I call generosity a subversive act.
because I think that's exactly what generosity does..
When we're generous,.
whether that's with our resources like our time,.
with the resources like our emotions,.
resources like our possessions, our material goods,.
our resources like our finances,.
whatever those resources might be,.
when we're generous with them,.
that has the power to basically subvert, uproot, overturn,.
what I think is our natural human tendency.
to hold on to what we have.
at the expense of what others do not have..
Are you following this?.

$^{41}$That generosity has this characteristic to it.
where it has the power to subvert this tendency.
that I think all of us humans struggle with.
to hold on to the things that we have.
at the expense of what others do not have..
And when we move in the spirit.
and in the movement of generosity,.
what we're doing is we're subverting.
that tendency in our lives.
and we're then opening ourselves and those around us.
to some of the most profound and beautiful encounters.
and moments with the character of God.
that they can experience.
'cause God's character is to be generous..
So when his people are generous,.
they're revealing something of the nature.
and the character of God to the world..
And that can happen.
through some of the smallest acts of generosity.
as well as some of the largest..
Give me an example..
About 30 years ago, I became a Christian..
So about 30 years ago, I gave my life to Jesus..
And then shortly after that,.
about a year into being a Christian,.
I was mentored by somebody.
who was a part of our youth ministry at that time..
The youth ministry was called Saturday Night Alive..
And I was being mentored by a guy called Doggan..
And Doggan, for me in those days, and he still is today,.
he's like one of the most radical Christians.
I have ever met, okay?.
Doggan like lives, eats, sleeps, drinks, and breathes Jesus..
There's Jesus and there's Doggan..
And Andrew in those days was like way down the list, okay?.
I was a new Christian, didn't really know too much..
One day he reached out to me..
He said, "Hey, let's hang out..
"Let's get together and hang out.".
And I thought that'd be great..

$^{81}$I met him in one chai, MTR..
And as I met him, he said,.
"Okay, we're gonna go into Southern Playground.
"and we're going to actually minister.
"to the drug addicts and the street sleepers.
"that hang out in Southern Playground..
"I'm a new Christian, I'm freaking out..
"I don't know how to do that..
"I've never gone out and done something like that.".
He's like, "Don't worry, it's fine..
"Everybody will speak Cantonese..
"So I will speak to everybody in Cantonese.".
He's Chinese..
He's like, "I'm gonna speak to everybody in Cantonese..
"Andrew, your role is just to pray.".
I'm like, "Cool, I can pray.".
He's like, "Pray in tongues, pray in the spirit, whatever..
"Just pray whilst I share in Cantonese.".
So we go out to Southern Playground just along here..
And sure enough, we meet a couple of these guys..
And it's quite obvious that they're either street sleepers.
or they're drug addicts just by the way they looked.
and the way they were dressed and their face and everything..
It was quite a sad situation..
Doggan begins to share in Cantonese.
the good hope of Jesus with these people..
And I'm there, I'm praying..
And as soon as I start to pray,.
I hear God say something very specifically to me..
He said, "Andrew, I want you to take off the cross.
"that you're wearing,.
"and I want you to give it to this man.".
Now this is way before I had any idea.
that he knows anything about Jesus,.
is a Christian or anything like that..
I feel like God is saying,.
"Take off your cross and give it to this man.".
Now that might sound like a small thing,.
but this cross had been given to me by Chris..
We weren't even married at the time..

$^{121}$I think we were engaged at the time..
And she'd given me this cross.
as like a gift of our engagement..
I had given her a ring..
She'd given me a cross..
This was a very, very beautiful thing..
It was a very precious, personal thing for me..
And I was feeling like God was saying,.
"Give this to this man, this random man,.
"who I don't even think knows.
"or has accepted Jesus into his life.".
And I'm like, I don't know if I can give up.
something that is so sentimental and so precious to me.
to some random drug addict..
I don't think that's what I'm gonna do..
But as I'm praying,.
God is compressing this on my heart over and over again..
And so eventually I'm kinda like,.
"Okay, well, I'm a new Christian..
"I guess I better be obedient to God.".
And I said, "Okay, I'll do this.".
So now Dorgan's been speaking to this guy over this time.
for about five minutes..
And after about five minutes, he turns to me and he says,.
"Have you been sensing anything in prayer?".
And so I said to him, "Well, actually, funnily enough, yes..
"I feel like God wants me to give this man my cross.".
Now, as soon as I said that, Dorgan's face goes white..
And he's like, "Oh, dude," he goes,.
"you'll never guess what's happened..
"I'm speaking to this guy in Cantonese..
"I'm preaching the gospel..
"I'm sharing Jesus with him..
"But the whole time, God's been speaking to me.
"about giving my cross to him.".
And he said, "You don't understand, Andrew,.
"but my wife gave me this cross..
"And this cross means a lot to me.".
And he's like, "I said to God,.
"no, I'm not gonna give him my cross..

$^{161}$"And now you're telling me.
"that God told you to give your cross..
"You should give your cross to this man.".
(audience laughing).
I'm like, "That's so generous of you, Dorgan..
"What a giver.".
So I figured I better be obedient..
So I took off my cross, my precious cross.
that my wife had given me, or fiance..
And as I held it out to this man,.
Dorgan's explaining what this is..
The man unbuttons his shirt and he's wearing this,.
one of those kind of jade and gold Buddha,.
a big Buddha around his neck..
And as the man decides to take off the Buddha from his neck,.
as he gets literally over his head like this,.
he starts to shake..
And this was the first time.
that I'd ever seen anyone do a demonic manifestation..
And he starts to shake and he starts to shout in Cantonese..
And Dorgan begins to gently talk back with him..
And then Dorgan takes my cross.
and puts it over this guy's neck..
And honestly, as it went over his neck,.
almost at the same place that the medallion came off,.
as the cross went over, peace came over him..
It was one of the most profoundly beautiful things.
I had ever witnessed..
And Dorgan begins to pray for this man..
And the man's in tears, I'm in tears, Dorgan's in tears..
Dorgan's feeling slightly guilty.
'cause he didn't give up his cross,.
but I was the good Christian in that moment..
So I was in tears..
And I learned that day that there's a subversive act.
that happens when we're generous..
That there's something that gets dislodged in us.
when we do it..
And I think it's this tendency of moving against idolatry..
It's this tendency of, I think we all have,.

$^{201}$with the things that we have in our lives,.
the resources we have, we wanna hold onto those things.
and we wanna grasp a hold of them really tightly..
And when we release them,.
we release the grasp that those things have over us..
Sometimes we don't realise that when we grasp a hold.
of our possessions really tightly,.
what's really grasping is not us with them,.
it's them with us..
That's the idolatrous nature of these things on our lives..
And our act of generosity is a way of doing two things..
It releases that idolatry from us..
It releases that grip that those things can have on us,.
but it also releases the gospel outwards into the world..
Does that make sense?.
So I wanna talk to you about generosity a little bit today.
and link that a bit to our intimacy with God..
And I wanna start by telling you that there's no surprise.
that I'm doing this after Chinese New Year last week..
Chinese New Year, where Ellison so eloquently told us.
that the main thing that we say at Chinese New Year,.
Gong Hei Fa Chui, is actually hope you get rich..
Are you with me, church?.
It's not a surprise, therefore, that God would say,.
"Let's talk about generosity the next Sunday..
"I hope you get rich if it means you'll be more generous.".
Amen?.
Here's something I've noticed, though..
Oftentimes when we get richer, we get less generous..
And I wanna talk a little bit about that.
with you here today as well..
Let me give you some thoughts, biblically,.
a theology of generosity to start with..
The Bible's very, very clear.
that every single resource in this world.
comes and flows from the same source, and that is God..
In other words, everything that we have ever received,.
creation itself, all the way down to every dollar.
that is in your bank account,.
every time that you have the emotions that you have,.

$^{241}$everything that you have in your life,.
the Bible's very clear,.
that all comes from the same source..
It all comes from God..
Now, I want you to think about that this way..
If God, if everything that you have,.
all the resources in your life have come from God,.
then God is the most generous being there has ever been..
If all of our resources are from God,.
then God, in his very nature, is generous, right?.
His heart is to give..
This is a really important thing..
His heart is not to consume and hoard..
His heart is to release and give..
That's the very nature of who God is..
If everything comes from him and we have received,.
that means God has released and he has given..
His generosity means that we have something.
that we can therefore be generous with..
This is really important because a lot of Christians think.
that generosity is a moral imperative..
What I mean by that is moral imperative means.
it's an ethically good thing to do..
We think as Christians, we should be generous,.
whether that's to a charity, whether that's to our church,.
with our tithes and our giving..
We think generosity is linked.
to some morally good thing to do..
It's morally right or ethically right to give stuff.
to the people who are in need..
The Bible never says that that is at the heart.
of what generosity is..
The Bible talks about that,.
but it's not the heart of generosity..
The heart of generosity doesn't flow.
from some moral imperative..
It flows from the nature and the character of God..
In other words, we are generous because God is generous,.
not because it's some ethically right thing to do..
We are generous because as Christians,.

$^{281}$we want to be and live and reveal the nature.
and character of Jesus to the world..
Our generosity is part of our revealing.
and understanding the character and the nature of God..
Therefore, generosity is not an act, it is a response..
Think of it this way, generosity is worship..
'Cause worship is a response out of what God has done for us..
So when we are generous,.
we're not just fulfilling some moral imperative.
in the world, although of course,.
it's good to give to those in need..
We are actually worshiping in a response to who God is..
In this way, generosity is never an act..
It is always a response..
Therefore, generosity is a lifestyle..
It should be something that flows out of us all the time..
But again, what happens so often for us Christians.
is that we think about generosity as an act,.
a one-off moment..
We think that generosity is a mission,.
one thing we should do,.
rather than a mindset, something of who we are..
And if you keep generosity as a mission,.
a one-off moment, an act,.
there's always gonna be a beginning and an end.
to your generosity..
But if we flip that and think that generosity.
instead is a mindset out of the response of the character.
and the mindset of God,.
then generosity can begin to flow.
in every aspect of our lives..
We're not thinking, oh, I haven't given my tithe this month..
We're thinking, how do I give of everything that I have.
to flourish what God has given around me?.
Whether that's my time, my energy, my resources,.
my talent, my possessions,.
and yes, of course, my tithe to the church,.
whatever it is, I want to give.
out of an overflow of a mindset.
that is committed to revealing the nature.

$^{321}$and the character of God in this world..
In this way, God calls us to be a generous people,.
not a people who are generous..
There's a big difference between those two things..
The call is to be a generous people,.
not a people who are occasionally, sometimes generous..
There are too many Christians in this world.
who are occasionally generous..
When the kind of church that I think Christ died for.
is a generous church, a generous people,.
a people who have made that part of their lifestyle.
and part of their mindset.
and are living out of the overflow of everything being,.
how can I be generous in this moment?.
Whatever that moment might be, are you tracking with this?.
Now, if this is all true,.
if that's a theology of generosity,.
what is the thing that enables us to do this?.
Or perhaps a better way of saying that is,.
what is the thing that stops us.
from actually being generous?.
Well, I think to understand.
how we live out a lifestyle of generosity.
rather than just moments of generosity,.
we have to understand the role that we have.
in the resources God has given us..
I've already said, haven't I,.
that God has everything in His hands.
and God is the one who gives..
Everything is flowing from the source of God..
Therefore, God's role is ownership..
That is not our role..
But here's the number one mistake Christians make..
We think we're owners of the things that God has given us..
You are not an owner of it..
I'm gonna say that one more time..
You are not an owner of what God has given you..
Your role is not ownership..
That's God's role..
And as soon as you take ownership.

$^{361}$of the things that God has given you, you're playing God..
That's a very dangerous thing..
What will happen is you will end up controlling.
what you have 'cause you think you're an owner of it..
You're not an owner..
God is very clear..
The very first thing He says to humanity.
after He creates them, I talked about this two weeks ago.
when we talked about sex..
In Genesis 1 and 2, He creates humanity.
and He gives them His creation..
And He says, "Go and do what?.
"Go and steward what I have given you.".
So our role is not owner of all the things.
that God has created, all the resources at His disposal..
Our role is steward..
You are to be a steward of God's resources..
That's your role..
That's what you've been called to..
And stewardship is really, really important..
Stewardship, it's funny, we think ownership is important.
and stewardship is, stewardship is actually.
a huge responsibility..
Think about this..
If you are a steward of all of God's resources,.
here's the crazy thing..
Your stewardship will either release more generosity.
into this world, the overabundant flow of God's resources,.
or it'll release more poverty in this world,.
the inequitable distribution of God's resources..
- Yeah..
- I'm gonna say that one more time.
'cause I don't think anybody's getting it in this room..
It's really important..
Your stewardship, how you think about yourself as a steward,.
will either release more generosity into this world,.
i.e. the flow of God's resources,.
or you will create more poverty in this world.
because there will be more inequitable distribution.
of God's resources..

$^{401}$So that stewardship is an incredibly important thing..
How we steward will release or withhold..
How we steward will mean that people.
have an equitable distribution or not..
How we steward will mean that things flourish around us.
or they won't..
That's all around us,.
and so stewardship is the gateway to generosity..
And if you think that your role as owner,.
you're gonna be way more frugal.
because here's what you'll think if you're an owner..
If you're an owner, you'll think,.
what should I do with this?.
If you're a steward, you'll say,.
what would God do with this?.
Come on, church..
If you're an owner, you're like,.
okay, what do I do with this?.
'Cause this is my resource to do what I want to do with it..
If you're a steward, you're like,.
everything I have is not mine, it's God's..
What is God's heart?.
What is God's character?.
What is his nature?.
How should I spend?.
How should I save?.
How should I provide for my family?.
That's a good thing..
God wants me to do that..
How do I provide for my family?.
How do I put money away for savings?.
How do I have nice holidays?.
That's all fine..
There's no problems with that..
But then equally, what does God say about those things?.
What does God say about my possession?.
What does God say about my spending habits?.
'Cause everything I'm doing here is on behalf of him..
This is the parable of the talents..
This is God, Jesus, describing this thing..

$^{441}$Oh, there's this wealthy landowner,.
and he gives talent to his individual servants,.
and those that know him well take those talents.
and put them to good use and produce an abundance..
Those that don't know him well and are afraid of him,.
they bury it..
Are you with me?.
And he comes and he says,.
you who stewarded well 'cause you knew me well,.
you get to enjoy my kingdom..
You who stewarded poorly 'cause you don't know me very well,.
you bury it..
And this is the connection between intimacy and generosity..
The more intimate we are with God,.
the better stewards of his resources you'll become..
When we understand his nature and his character.
and who he is, the better it is.
that we are to release those things into the world..
We begin to understand how God would treat.
the money in my bank account,.
how God would treat the time that I have each week,.
how God would treat the talents.
and the gifts God's given me,.
how God would treat my emotional capacity.
that I have towards those around me..
When I understand that I'm a steward,.
God resources can flow,.
and my relationship with him, my intimacy with him.
is the gateway to whether that's gonna flow well.
or whether it's not..
Intimacy and generosity are linked..
Helpful so far?.
All right, I've got about five minutes left,.
which is really bad,.
'cause I have about 15 minutes more..
Is that okay?.
Can you hang with me for a few more minutes?.
'Cause I want you to wanna go to scripture.
'cause you're probably thinking like,.
is Andrew ever gonna talk about scripture ever?.

$^{481}$So let me open up the word of God,.
and this will just take about 15 minutes,.
but I want you to see here from Ecclesiastes.
something that is really important about stewardship..
It says here in Ecclesiastes 4, verses five and six,.
"The fool folds his hands and ruins himself..
"But a one-handful with tranquility,.
"than two handfuls with toil and chasing after the wind.".
Ecclesiastes is this beautiful writing..
It's a wisdom literature in the Hebrew scriptures,.
and it's a wisdom tradition.
where it provides provocative imagery.
and very extreme ways of talking.
to be able to explain or to understand.
the character and the nature of God..
Let me read this to you again..
"The fool folds his hands and ruins himself..
"But a one-handful with tranquility,.
"than two handfuls with toil and chasing after the wind.".
When we read this in the English,.
we see that there are three mentions.
of the word hand in this passage..
There is the fool who folds their hands,.
there is the one handful with tranquility,.
there are the two handfuls with toil.
and chasing after the wind..
In the English, we only have one word for hand,.
and that's hand, and so the translators translate it.
as, funnily enough, hand, okay?.
In the Hebrew, though, what's going on here.
is really powerful, because in the Hebrew,.
there's about five different words.
for the word hand in English..
In the Hebrew, three of those words.
are being used in this passage,.
and each of them is trying to communicate.
something really important about stewardship..
So let me show you this in the Hebrew,.
and I want you, I've highlighted the three words there.
in the Hebrew that are used here.

$^{521}$and translated in the English hands,.
but each of those words, as you can just tell.
by looking at them, are very different words,.
they have very different meanings,.
and they help us to unpack.
what is being said here about stewardship..
So let me unpack each of these really quickly..
The first thing, oh, and by the way,.
to help you illustrate this,.
I've got this little box full of rice here,.
but I want you to assume that this box.
is like a picture of all the generous resources.
and all of the things that God has at His disposal..
They're all the resources that are exposed for us at life,.
all the things that we could have in this life.
out of the generosity of God..
That's what this represents..
Now, the wisdom writer says this,.
first of all, "The fool folds his hands.".
The word there for hand is the word yad,.
and the word yad is used throughout the Old Testament.
as the word that means the mighty right hand..
The word yad means might or power or energy..
It's the word used throughout the Old Testament.
to talk about God's right hand at work for His people,.
His mighty right hand to save us..
So the wisdom writer takes that idea of might and power,.
and he says, "If you have might and power and resource.
"and energy and gifts and talents,.
"and you fold yourself and don't do anything with it,.
"you're a fool," he says..
So the fool takes the yad that they have..
Let's say this is all of their resources,.
all of your talent, all of your gifts,.
all the things that God has given you,.
the great stuff, the energy that God has put inside of you,.
and you just decide to check out of life,.
that is a fool, the Bible says..
You've been given so much,.
and yet your posture is to fold your hands,.

$^{561}$and he says here, "You will ruin yourself.".
And I think, although it doesn't say it here,.
I think what he's also saying is.
you will ruin those around you as well,.
'cause you're not flowing any of that equitable distribution.
of God's resource..
You have yad your hands..
You've folded them and given up on being generous,.
given up on anything..
Maybe the modern term for this is the idea of lying flat..
Have you heard about that recently?.
It's like, "I got this great education,.
"I've got all this stuff going on, blah, blah, blah,.
"but I'm just not gonna bother..
"I'm just gonna kinda check out of life..
"I'm gonna fold my yad.".
Now, you might be sitting there thinking,.
"I never do that..
"I work really hard, I'm exhausted all the time..
"I'm working really hard..
"I'm using my talents and my gifts,".
and you're thinking, "I don't really do this very often..
"I tell you what, every single one of us,.
"at some point, we will do this.
"to the opportunities that are around us to be generous.".
And I think there's a couple of reasons.
why we fold our yad..
The first is that I think maybe we've taken a step out.
in faith in the past with our yad,.
with our power or our talent or our gift.
or the energy God has given us..
We've taken a step out in faith with it.
and it's not worked out, or maybe people laughed at us..
Maybe we tried that talent.
or we tried to be generous with a gift.
and people didn't accept it..
Maybe they laughed at it, maybe they turned away from it,.
maybe they didn't embrace it.
how we thought they were gonna embrace it,.
and we felt shame for that, we felt some embarrassment..

$^{601}$Or maybe we gifted somebody with something.
and they just treated it like anything and that hurt us..
Don't you know how much I prepared that gift?.
Don't you know how much I slaved for that gift.
and I gave that to you and you didn't want it.
or you just didn't treat it.
how I thought you were gonna treat it?.
I'm hurt by that, are you with me?.
Hurt, pain, disillusionment,.
that's gonna cause you to fold your yard..
It's like, well, I tried that before,.
I'm not trying that again..
Well, I gave to that person, but they didn't say thank you,.
so I'm never giving to them again, are you with me?.
So hurt and disappointment and failure.
will cause us to fold our yard..
I think here's the other thing that we'll often do it,.
and I struggle with this one..
We think that what we have is not worth giving..
We think like, well, I mean,.
if you compare me to these other people,.
if you looked at my bank compared to their bank,.
I mean, they're the ones that should be giving..
God knows how much I've got,.
and God knows I can't give..
They should give, but not me..
Or we maybe think, you know,.
like their talent is way better than my talent..
I'm not gonna sign up to worship,.
have you heard Emma sing?.
I can't sing like Emma..
I'm just gonna fold my arms and sit in the chairs..
By the way, I'm now recruiting for the worship team..
Just saying, I'm just gonna sit in the pew.
and fold my arms because have you heard her?.
Are you with me?.
So not only do we fold our hands.
because of pain and disillusionment and hurt,.
we fold our hands because we compare ourselves to others.
and we have insecurity.

$^{641}$and we think we have nothing to offer,.
and we fold our yard..
Following this?.
Then the Bible says this..
It says, "Better to have one handful with tranquility.
than two handfuls with toil and chasing after wind.".
Two handfuls here is the word kofen,.
and kofen basically means to grasp as to a fist..
The idea would be like,.
this is all of God's resources that God has,.
and we as his people in the stewardship of kofen,.
we grab a hold of what God has given us..
Now, on the surface, kofen looks great.
because we've taken responsibility..
God has, we've not laid flat and crossed our arms.
and gone, "Who cares?".
We've taken as much as we could,.
and we're holding on for dear life..
I mean, we've got God's resources in our hands,.
and occasionally those resources come out.
through those moments of generosity,.
but we've got a lot still in our hands..
And we've got a lot in our hands.
because we wanna make sure that we keep it in our hands,.
'cause we think that it's a good steward.
to grasp a hold of what God's given us,.
and we think that this is the right way to go..
Let me tell you this..
We think that this is an abundance mentality, kofen,.
but kofen is actually a poverty mentality..
It's a poverty mentality.
because although we think we've got a lot,.
and theoretically we do have a lot,.
we've got two fistfuls of all of those resources..
The problem is we don't want to let go.
of those resources very much..
And second of all, we can't really put more resources in..
If you keep your hands fisted,.
it's really hard to get more rice in the hand..
Are you with me?.

$^{681}$And we think that this is actually the right thing..
God's blessed me, and I need to take responsibility.
for the things that God's blessed me with..
Absolutely we should..
But when we're kofen, when our stewardship posture.
is this gripping and this grasping,.
all it really does is cause us.
to become owners and not stewards..
It causes us to control the things..
And every once in a while, we go to that event and we give,.
or every once in a while, we put something in the offering,.
and there's a little bit of generosity that comes out,.
but we've still got most of it for us..
The wisdom writer says,.
"This is toil and chasing after the wind..
"This is exhausting," the wisdom writer is saying..
He said, "This is what 21st century life is like..
"This is like the grasping and the getting hold of more.
"and trying to achieve more and getting more,.
"because we're actually afraid that we won't have enough..
"And our fear of poverty drives us to grasp holder.
"and tighter and more upon the resources that we've got.".
And the Bible says,.
"That's just gonna cause stress and anxiety..
"You're gonna be even more stressed than you ever were.".
And I tell you, this is interesting..
My experience is the more money I've gotten in my life,.
the more stressed I've become..
And we often think that if we had more,.
we'd be less stressed..
And the Bible says,.
"Two handfuls is toil and chasing after wind..
"If your posture and your attitude is kofen,.
"to grasp a hold.".
You with me?.
So if it's foolish,.
if it's foolish to fold your yard.
when you've been given all this great opportunity,.
but it's also foolish to grab a hold of it.
as tightly as possible,.

$^{721}$and that's just stress and anxiety..
Is there a better way?.
Well, the Bible says this..
It says,.
that's like one of those miracles,.
mercy is coming out of the Bible..
It says here,.
"Better one handful with tranquility.".
I love this..
"Better one rather than two..
"Better less than more and have tranquility,".
the Bible says..
Now, the one handful here is the word kof..
And kof is the picture of a hand open.
and palm facing upwards..
The idea is this..
The idea is that you have the resource of God in your hands,.
but that resource,.
because it's in an open palm like this,.
can flow out really, really, really easily..
Now, if you're anything like me,.
if that's your posture,.
you're worried because if it keeps flowing,.
I'm not gonna be able to have much left to flow with..
But here's the amazing thing..
When you're postured like this,.
God can come along and add more..
And look at this..
There's the flow that happens..
We're gonna have to vacuum between services..
There's a flow that happens..
Are you with me?.
The Bible says this is peace and tranquility..
This is what it's like to have inner peace.
when you are kof with the resources God has given you..
Your palm is open,.
and God will keep blessing you..
And as God pours into you.
and your hand posture remains open,.
you will continue to pour out..

$^{761}$This is a river..
This is a lake..
This becomes stagnant and doesn't flow..
But the Bible says this is what generosity looks like.
through the stewardship of kof..
Better to have one hand open than two.
with toil and grasping after the wind..
So as I bring it all to a close,.
here's the question for you..
Which of those postures is you?.
And if you're anything like me,.
you're saying all three..
That there are moments where you know you've just given up..
And you've just folded your hands and said,.
can't be bothered anymore..
And there are moments where you have grasped tightly.
and you're holding on.
and you feel that stress and that anxiety..
And there are moments perhaps where you are open-handed.
and you feel the inner peace and the beauty.
and the wonder of God when you're doing so..
And yet you find yourself just going back and forth.
between the two..
Are you folded?.
Are you fist?.
Or are you free?.
If you find yourself folded,.
here's what I wanna encourage you to think about..
What areas of pain or hurt or insecurities.
are there for you that would cause you.
to fold your hands so often?.
If you're fist, the question you should ask yourself is,.
why don't I trust God enough?.
That I feel like I need to hold onto all these resources.
as tightly as possible.
'cause I don't think I'm going to get again..
Why is trust an issue for me?.
And if you're open-handed and beautiful before Him.
and you're seeing the flow of resources,.
perhaps for you it's just to be grateful and thankful.

$^{801}$that God is using you and can use you.
and continue to use you to bless those around you..
Are you Yad, Kofen, or Kof?.
'Cause either or any of those.
will be whether you're a generous person.
or just simply a person who is occasionally generous..
Helpful?.
Can I pray for you?.
Let's pray..
Father, I'm so grateful for each person here.
and Lord for what you're doing in their lives.
and for the generosity that is flowing from them..
Lord, I just really wanna honor the vine.
'cause as the senior pastor here over the last 10 years.
and been on staff for 15,.
I've seen the generosity of this church..
I've seen this church be generous people.
time and time again..
And Lord, it is an incredible thing to see..
And so Lord, I just honor each person here.
who has faithfully given to this church over so many years..
Maybe those who are faithfully giving in this season..
Lord, we are grateful for that..
And Lord, we recognize that generosity.
that is already flowing here..
Father, we also recognize that in our own lives,.
we have a tendency to move.
between different forms of stewardship..
Lord, I know that I have the tendency to fold my hand,.
I have the tendency to clench, hold tightly..
And so Father, I wanna pray that as we reflect.
around generosity in our lives,.
we would first remember, as I've spoken about earlier,.
that it starts with you, that intimacy with you.
has a direct impact on our work of stewardship and generosity.
and that as the wisdom writer in Ecclesiastes shows us.
that it's better to have less but be open.
than to have more and be closed..
That the path of peace is found in being open and generous..
The path of stress is when we shift.

$^{841}$from a steward to an owner..
And Father, I wanna pray for people here.
where that resonates with them..
Perhaps there are some people here this morning.
where their posture has largely been one of folding.
in this season..
Maybe there's been some hurt there..
Maybe they've given and not been thanked..
Maybe they've done something generous.
and it hasn't produced the fruit.
that they were hoping it would produce..
Maybe they've even given to the church.
because they thought that that would mean.
that you would answer their prayer and you didn't do that.
and they've folded their hands..
Father, would you just come for any of us.
where that's the case?.
Would you comfort us and speak to us and heal us?.
For those of us that have compared our gifts.
or our bank accounts or what we have to others.
and thought that's their responsibility, not mine..
Father, I pray you would shake us out of our complacency..
Father, for those that are clenching tightly.
and you of course want us to be sensible.
and sensible with our money,.
sensible with our spending, of course,.
but Father, for those that are perhaps clenching too tightly.
and there's stress..
Lord, I wanna pray that your inner peace would flow,.
the tranquility that the writer speaks about here would flow..
Father, I pray that you would give us the strength.
to be able to be more open to the Lord..
I pray that you would give us the strength.
to be able to be more open to the Lord..
Father, I pray that you would shake us out of our complacency.
and comfort us..
Lord, I pray that you would shake us out of our complacency.
and comfort us..
Father, I pray that you would shake us out of our complacency.
and comfort us..

$^{881}$Lord, I pray that you would shake us out of our complacency.
and comfort us..
Lord, I pray that you would shake us out of our complacency.
and comfort us..
Lord, I pray that you would shake us out of our complacency.
and comfort us..
Lord, I pray that you would shake us out of our complacency.
\newpage



\section{}
\label{sec:GMknFcIDczo}
\textbf{2024-03-03 Vision Sunday 2024 [GMknFcIDczo].mp3}
\newline
\newline
連結: \href{https://youtube.com/watch?v=GMknFcIDczo}{\texttt{ https://youtube.com/watch?v=GMknFcIDczo}} ~~~~ 語音日期: 2024-03-03 
\newline
\newline
\hyperref[sec:W1x4kQ0_ys]]{\small{< < < PREV SERMON < < <}}
~
\hyperref[sec:index]{\small{[返主目錄]}}
~
\hyperref[sec:E2_iBrxfKKc]{\small{> > > NEXT SERMON > > >}}
\newline
\newline
$^{1}$The past is here at the Vine and we are so glad you're in this room online with us right now watching from wherever you might be watching.
Around the world around Hong Kong also in our overflow as well. We're so glad that you guys are with us, too.
It is vision Sunday.
And vision Sunday is one of the Sundays of this year that we see is.
most important most dear to the direction in the future of us as a.
Community if you've not been here at the vine for long, you're probably wondering what is this?.
Vision Sunday is a time where we share.
What we see as God's vision for us as a church.
what we see is God's heart for us as the community that's called the vine and.
We talk about what God wants to see happen in and through us in the year ahead and that as part of that.
We ask you boldly.
We ask you boldly to use your resources.
superficially towards the vision that God has for his church and for us here at the vine in Hong Kong and that's.
What we're gonna do today.
But as I was praying over the last number of months in preparation for today.
As we as a leadership here at the vine were praying and seeking God's heart.
We felt like God was actually calling a shift for us.
Calling us actually to approach things today a little bit differently.
As I was praying about two months ago asking God for his vision for the vine. I.
Hope God say this he said Andrew. What is my intent for my church?.
Kind of a simple question, right Andrew.
What is my intent for my church and I figured as a pastor of a church?.
I should have an answer to this question, right?.
Have you ever had God ask you a question and then you're like, I'm gonna now teach God something, right?.
I'm gonna now theologize with God, right? So he's like Angie. What is my intent for my church?.
And I'm like, I know this question. I've given up my life to build and serve the church.
I said it's this that that we as the the leaders of the church need to get around the people in the church and and.
Equip them for works of service.
So that together we would be built up to become more and more like Christ Jesus.
I basically started preaching Ephesians 4 at God, right? And I said like this is what it's all about God, right?.
This is about us gathering together in a room or gathering wherever we're sitting online watching this and and spending time speaking life into each.
Other so that we can disciple each other so that we can grow up to more maturity and become more like Christ Jesus.
I felt really good about myself preaching to God, you know, and.
It's so funny, you know.
God has such a gentle but also very direct way of speaking back when we're arrogant with him.
He said hmm.
Have I heard God say to you? Hmm.
That's when you know, you're in trouble. Hmm.
He said I want you to read something because you've forgotten.
What my intent for the church really is and he led me to a passage in Ephesians chapter 3.

$^{41}$I want to read this to you because this has changed so much of my life in the last few months.
He writes this this is Paul writing to the church in Ephesus.
He's speaking of God. He says his intent was that now through the church the manifold wisdom of God.
Should be made known to the rulers and the authorities in the heavenly realms.
According to his eternal purpose, which he has accomplished in Christ Jesus our Lord.
Paul is writing to a small group of people.
who at this time gathered in house churches in the city of Ephesus a.
City that was filled with people who worshipped sacred and magic cults.
It had one of the largest temples of worship to Diana in the whole of the Greek and Roman Empire.
It was a place that was famous for people to go to to do idol worship to speak about this.
Magic cult that they're a part of and Paul picks up a little bit of that language and he says here's my intent.
My intent is that the manifold wisdom of God.
Would be revealed to the authorities and the powers in the heavenly realms.
Which is Paul using magical language to speak of the reality of God wanting to say something to the world.
And he said here's God's intent. He wants the manifold wisdom of God to be revealed to the world.
And so we might be wondering what is this idea of the manifold wisdom of God?.
Well Paul in chapter 1 in the book of Ephesians actually explains what that is.
And he said here's the will of God. Here's what all of God's wisdom is about.
He wants to see everything on heaven and earth come under one person. That is Christ Jesus.
Paul Paul pulls out his beautiful image of what the whole of God's intent and his will and his desire and his wisdom is for.
the world that everything all of the broken things all of the separate things all the divided things all of the things that hate all of.
The sin all of the stuff that's in us as humanity that all of it would find its rightful place its renewed place.
It's restored place under Christ Jesus.
It's this beautiful vision that God's wisdom is reconciliation.
God's wisdom is.
Restorative life for people it is so that in him and through him we would become new.
Creations the old would go would go and the new would come.
So it's a pool lays out this incredible picture of what God's will and desire in the world is that everything that is currently broken and separated.
would actually come to new unity and life and.
reconciliation and restoration under the person of Jesus Christ and.
Then he says this it is his intent now that through the church.
That manifold wisdom would be known to the world.
Here this guy's through the church.
Now remember he's speaking to a bunch of people who are hiding basically in house churches.
Who are probably 20 30 40 people at that time?.
This is 300 years before the institution of church where church becomes more of an organized religion.
He's writing to a bunch of people in a house.
And he's saying here's God's vision that through the 40 of you who meet regularly in house churches.
That you through you would come the manifold wisdom of God to see.
Reconciliation and love and and forgiveness and grace be seen and heard in the world.

$^{81}$That should blow our minds.
through the church.
That actually what we do when we gather in these rooms is not actually primarily about us feeling better about us. Oh.
I'm gonna preach on this one.
That actually what this is about when we gather together whether that's in larger rooms like this whether it's in homes like those watching online.
Right now whether it's in workplaces wherever it might be where where God's people gather together.
Through that God wants to release a manifold vision of his wisdom in the world. He wants to release restoration.
Reconciliation and love to the world which means something and this is gonna sound very controversial.
But stick with me for a second, which means that the church is not actually about Christians.
Come on, the church is not actually about Christians.
That the church gathers together on behalf of the broken world around it.
That we get to be the manifold wisdom of God to the world.
That's exciting.
That's what it's supposed to be all about.
and so here I am preaching at God about the fact that we're all gonna equip ourselves for works of service so that we might.
Become mature and to the wholeness and the wholeness of Christ and that is absolutely what God does also through the church.
But then God says but you've forgotten something.
You've forgotten actually a hidden imperative that is so critical that actually what I want to do through you is release my love my wisdom.
My reconciliation my will to the world out there.
So when you gather what better be in your minds what better be in your heart is everything out there and not just what's on in here.
Hmm so.
When I was praying and saying God, what is the vision that you have for the vine God quite rightly flip the script.
And he said to me Andrew your question should not be what is my vision for the vine?.
Your question actually should be what is my vision for Hong Kong?.
Because if the church is.
the manifold wisdom of God to the world and if the vine is.
Founded and it's growing and it's part of the city of Hong Kong what you should really be asking me Andrew is what is my?.
Heart for Hong Kong because I love Hong Kong.
I love the people of this city.
And I see Hong Kong and I have so much for the city and really if you want to be a church that has a vision.
Here's what your vision should be your vision should be in line with what my vision is for the world.
What my vision is for the brokenness that's out there what my vision is for the fact that there are so many people who are.
Scrambling trying to find the key to open the lock that would actually set them free and the church is silent. Oh.
I'm gonna preach today. Are you ready?.
but the church has been given the access code and.
It's our season.
to release our voice.
And if you think about it.
Surely now more than ever. I.
Mean this is why I think that we have been called into intimacy as a church in this year.

$^{121}$You see I think so often when we think about intimacy we actually think about ourselves again.
We think about intimacy being primarily about my own personal relationship with God and what that can do for me.
But I actually think this that intimacy when a church is called to intimacy.
It's about drawing closer to the heartbeat of Jesus.
closer to the.
Relationship we have with the Holy Spirit closer to the power that is found in the love of the Father.
So that yes, we may hear about ourselves and grow and change and disciple and become more and more like Christ Jesus.
But also so that we could hear what God has in his heart for Hong Kong.
And I believe over the year ahead as we gather together.
Individually and corporately on Sundays and connect into intimacy with Christ. It's because he wants to draw us in to send us out.
He's not drawing us in for in sake.
This is really important. We actually saw this last year throughout the whole of the Exodus series.
Didn't we in that series?.
We saw that God took his people who were enslaved in in slavery in Egypt drew them out of that drew them into himself.
Around Mount Sinai gave them the law so they could understand his heart.
Why so that he could then send them out into the promised land so that they could reveal the manifold wisdom of God in the world.
Could it be that our year of intimacy is really not so much about just us and our own personal growth.
But maybe it's about Hong Kong and its growth. I.
Don't know about you.
But most of my foreign friends right now. The main question they asked me is this.
Do you really still want to live in Hong Kong?.
Anyone else heard that recently like do you really still want to live in Hong Kong?.
Like have you seen Hong Kong right now? I mean, I mean surely there is also.
Singapore and Taylor Swift like maybe you can move to Singapore and you would have Taylor Swift you have messy.
They have Taylor Swift. Okay, they they have Ed Sheeran. They have Coldplay.
They have a great economy like really do you still really want to live in Hong Kong?.
And if we're honest with ourselves.
There's actually some pretty big challenges in Hong Kong right now.
Hong Kong's been through a rough time over the last five years or so.
the social unrest and changes that happened off the back of that.
With Kovac coming into our city and some of the challenges that were found in Kovac for all of us.
Hong Kong's had a pretty rough time.
And although now we're about a year and a half out of some of that really hard stuff.
Hong Kong's still struggling I think to recover and.
There's still some pretty major issues that are right here on our doorstep.
One of things we have to realize right now is that Hong Kong is facing a mental health crisis.
Recent study was done by the Hong Kong Federation of Youth Groups and they looked at and interviewed personally.
4,000 secondary school students and out of all of these.
4,000 secondary school students from a whole cross-spectrum of local schools in Hong Kong. Here's what they found.
51.9\% of them had early emotional symptoms of depression.

$^{161}$51.9\%.
They then found within the 51.9\% that about.
48.6\% of them were actually registering the highest levels possible for anxiety and stress as a scale that you can.
Basically discover somebody's anxiety and stress levels.
48.6 had the highest anxiety and stress levels that are possible to be recorded.
Save the Children Hong Kong did a recent study and they wanted to look at youth suicide and.
They looked at the ages of 15 to 24.
And they looked at a hundred thousand out of a hundred thousand people in that age range in Hong Kong.
How many have actually committed suicide? So what is the suicide rate?.
Here's what they found. They found that the rate was six point two people out of a hundred thousand back in 2014.
But in 2020 that had risen to ten point four percent or ten point four people out of a hundred thousand.
Now that was in 2020. I wonder what that statistic would be if they did it now in 2024.
I would argue it's jumped up even more from that point and here's something else they found in that report.
And this blows my mind.
They also looked at university students in our city and they discovered that ten.
percent of the people they spoke to.
Had either tried to kill themselves in the past year or had thought about killing themselves in the past year.
ten percent.
We have a mental health crisis in our city.
Outside of that you just need to look economically and politically at some of the challenges we have in the city.
Hong Kong has always had one of the highest.
Gini coefficients in Hong in the world.
That's the gap that there is between the wealthy and the poor.
Hong Kong's always had one of the more higher of those around the world that has actually only.
Increased in the last number of years over Kovat the wealthy got wealthier and the poor got poorer.
One of the great challenges in our city is affordable housing.
affordable housing for the poor the people that really need it and out of some of the affordable housing that we do have the.
Quality of that affordable housing is also a challenge for us.
add on to that some of the thoughts around.
kind of the.
Population decline that's happened in Hong Kong over the last little while.
The government's very aware and is working really hard to try to attract foreign investment back into the city.
Track foreign talent also back into our city.
Talk about some of the frustrations that are there for young people under the age of 30.
a lot of them are worried about their.
Ability for freedom and their belief of freedom of speech and freedom of expression.
Many of our young people under the age of 30 are asking is the one country two systems really gonna work.
Is that going to change more dramatically in the near future a number of them are also wrestling?.
With where their future is going in the report by save the children.
Here's some quotes that were found in that report. Let me read this to you.

$^{201}$There is no future anymore in Hong Kong.
I have too much pressure to provide for my family. I have no dreams for life.
These are representative of what is happening to this generation in our city.
And so when we actually take a look at some of the challenges that we have in Hong Kong.
Maybe the question is do you still really want to live here?.
Talk about somebody taking away my my main preaching point.
I.
Mean I mean do you do you really still want to live here and if it's true if it's true that the church.
through the church the manifold wisdom of.
reconciliation.
restoration love power forgiveness.
Grace is seen through the church if that's true, then surely this is the most important time to be in Hong Kong.
Surely this is the time where we would want to double down and invest in our city at this time.
Surely this is the time where we realize as the church.
It is no more important moment in history than right now for us to release our voice.
For us not to stand next to the people who are desperately searching for the key to open the lock.
But actually to say let me tell you about the manifold wisdom of God because through the church that will be seen.
It's not beautiful that God would want to release through this group of people us.
And all of our weaknesses and all of our flaws that manifold wisdom of God think about this the church.
matters.
The church matters.
Because together through the grace of Christ we are God's vessels in all of our flaws and weaknesses.
To reveal the manifestation of the wisdom power and love of God to the world.
That's why we do what we do.
That's why you gather here every week if you come here.
That's why you join us online from wherever it is that you're joining us right now. We do it.
Yes, so that we can grow in Christ. Of course. Yes so that we can heal and be renewed. Absolutely.
Yes, so that we can become more like the the wholeness of the wholeness of Jesus, of course.
But here's the core reason why you gather it is so that you through you.
Through me through our weaknesses and our brokenness.
We would still be a manifestation of the reconciling and restoring love of Christ to the world.
Isn't that beautiful? That's what church is about.
Now here's the interesting thing.
This is not the first time that God has had a passion for a city.
In fact scripture is filled with a picture of God's passion for cities.
And cities are always on God's heart. And if that's true, then we have to say that God so loves Hong Kong.
And perhaps if we can take a look at God's heart for a city in scripture.
We might be able to connect that to God's heart for our city as well.
And in that might get a better understanding of why we do what we do. I.
Want to take you actually to one example of this from the prophetic book of Zechariah.

$^{241}$Zechariah is important for us to go to you because of some of the connections that I think are to the season that Hong Kong.
Is in at the moment.
Zechariah is prophesying to a group of people that have returned from exile.
You probably know the story in the Old Testament.
Jerusalem is there. It's a thriving city, but it's broken and in the sin of idolatry.
God brings up Babylon and Babylon comes and actually breaks down the walls of Jerusalem.
actually kills many of the people in Jerusalem and takes a remnant of them into captivity in Babylon and.
For 70 years in captivity.
God's people.
suffer in Babylon and.
After the end of that 70 years God in his grace raises up Persia and Persia comes and actually releases.
all of the Jewish people back to Jerusalem and.
Zechariah is writing 20 years after they've returned to Jerusalem.
So they've come out of this really traumatic time and this traumatic season.
They've come back into the city of Jerusalem and for 20 years later Zechariah now begins to speak and here's what's happened.
They are still finding themselves holding on to the trauma of their past.
And I think if we're honest with ourselves Hong Kong and we are still processing a lot of the trauma.
That we've experienced over the last four or five years.
And we're still holding on to so much of that. And like I said earlier.
there are some significant issues that we should be concerned about and.
Here is a group of people who have returned to Jerusalem being set up in God's city and 20 years later.
God has to raise up Zechariah to bring a word of hope and a word of life.
To a bunch of people who needed a new vision.
They needed to understand what is God's heart for Jerusalem? I think we need to understand. What is God's heart for Hong Kong?.
And so I want to show you some of the things that Zechariah said, is this helping anyone? Is this alright?.
Good all right.
Zechariah chapter 8.
Starting in verse 18. Let me read this to you again. The word of the Lord Almighty came to me.
This is what the Lord Almighty says the fast of the fourth fifth seventh and tenth months will become joyful and glad.
occasions and happy festivals for Judah.
therefore love truth and peace.
This is what the Lord Almighty says many peoples and the inhabitants of many cities will yet come.
and the inhabitants of one city will go to another and say let us go at once to.
Entreat the Lord and seek the Lord Almighty for I myself am going and many peoples and powerful nations will come to.
Jerusalem to seek the Lord Almighty and to entreat him.
This is what the Lord Almighty says in those days ten men from all languages and nations.
Will take firm hold of one Jew by the hem of his robe and say let us go with you.
Because we have heard that God is with you. This is a glorious vision.
That God is unwrapping through Zechariah to a bunch of people in Jerusalem who are still wondering. What is the hope of this place?.
They're still asking the big question. Do we still really want to live here?.

$^{281}$And God through Zacharias says oh, yes you do and let me tell you why.
The first thing he says is okay. First of all, your fasts are about to change now.
Let me explain why this is important when Israel was in captivity in Babylon for 70 years.
It was a very important time for them to lament there had been so much that they had done.
before.
Babylon had come into Jerusalem and it was so much that they were still wrestling with and trying to process and.
During those 70 years in captivity.
they actually wrote so many of the lament Psalms that we have in the book of Psalms and those it's a chance for.
Actually, actually the Jewish people to to basically process their trauma together in a place of lament and the fast.
What happened was that the Jewish leadership at the time decided we're gonna fast every single month and in those fasts.
It's gonna be a place for us like to tear our garments to put on sackcloth.
to realize that the world is not how it's how it should be to understand that we're under trauma and grief and to be able.
To try to process some of that in a place of lament and what had happened was although God had now released them back.
Into Jerusalem. They were still acting in those ways of captivity back in the freedom of Jerusalem.
And so God shows up and he does something really quite powerful.
He says I'm gonna take those fasts that you've instituted and here's what I'm gonna do.
I'm gonna make them joyful events and happy festivals.
In other words what you were doing once to lament and process trauma now.
I'm gonna flip the script around and those are gonna be times of hope for you times of joy for you times of happiness.
For you and you're gonna be able to move forward in a new way a new way of thinking a new way of believing.
Before yes, you needed to process your trauma, but now you're back in my city and I've got hope for that city.
I've got a future for that city. I'm about to tell you all the things I'm about to do.
but in order for you to capture that you need to start shifting from a place of lament and trauma and grief towards a place of.
Hope and life and and courage again.
Notice that he doesn't say that your problems are gonna go away.
He doesn't remove the fasts because the issues are still gonna be there.
I'd be silly to stand in front of you today and tell you just because the church gets its voice.
It means all the issues in Hong Kong is gonna go away.
The issues will probably still remain those facts that I've talked about.
Those are big issues and complex issues that a lot of hard work will need to go in to see a shift and a change.
But here's what I am saying.
The time is coming the time is now to shift our approach to all that issue from oh.
I'm not sure if I want to be here anymore - oh my gosh.
I'm so glad I'm planted here because I can see what God is doing in this.
I can see that there could be a better future that there could be a place of hope that we could move forward and see.
the manifold wisdom of God's reconciliation and restoration.
Come for the university students and the young people and everybody in this city who doesn't have a vision. We have a vision.
And so we're gonna take the thing that we used to lament about and now we're gonna look at it and we're gonna say.
All right, there are issues but we're gonna approach those issues from a place of joy of hope.
That isn't it a privilege that we get to walk into broken people and show them something of the light of Jesus. I.

$^{321}$Think we need this in our city.
And I think God is signaling a shift in our city a shift in the spirit of our city.
And if the church doesn't do it who else is gonna do it?.
I mean think about this if it is through the church that the manifold wisdom of God is revealed in the world.
Then if the church doesn't rise up with a fresh vision of joy in this.
Season, how could we ever expect the national consciousness to change?.
Come on church you with me.
Like how could we ever expect the national consciousness?.
to change if the church is not rising up with a fresh vision a fresh hope of joy and.
Courage and what God sees in this place if we can't do that.
How could we ever expect the world around us to change because it is through the church.
That the manifold wisdom of God is revealed in this world. Oh.
Man, that should pump our veins more than anything else.
Bible is very clear. The Bible says that it is the joy of the Lord. That is our strength.
the joy of the Lord.
Not our smarts not our finances.
not our.
Great things that we do is the joy of the Lord. That is our strength.
Which means that when we're in a season where that joy has been taken from us and again.
perhaps there's some right reasons because we needed to lament and we needed to face and we needed to grieve and at the vine over the.
Last five years we've created space many times for us to be able to do that.
But we have to realize that when that happens in a city or in a church community the ultimate.
Reality will be that that will result in weakness.
when joy is removed people and cities and societies governments and churches get weak and.
It's really important that we understand this Jesus did not go to the cross for a weak church.
He went to the cross so that we would trade our sorrow for joy.
So that the joy of the Lord would be our strength and notice that when the Bible says the joy of the Lord is your strength.
He's not just talking about Christians.
That's the formula for every human being.
That when every human person comes into the recognition of the joy of the Lord.
They will find renewed strength and vigor and they won't be saying I have no vision for my future.
They were saying I am so glad that I am Hong Kong Chinese that I am in Hong Kong planted in this place.
Oh, I struggled. Oh, I wrestled but guess what the joy of the Lord is in me.
And because the joy of the Lord is in me, I want to open my mouth and share the hope share the courage share the future.
I want to speak now because I had this.
Overflowing sense of the peace and the love of Christ in me.
the joy of the Lord is.
my strength I.
Wonder whether.
we as a church need to regain the practice of the joy in the sacred and.

$^{361}$I thought it was really powerful earlier because we didn't orchestrate this when at the end of worship.
We just shouted for like I don't know how long it was 30 seconds or so. Some of you were like, okay.
I'm kind of done shouting now.
Others of us were just wrapped up in this incredible praise of God. Amen. I.
Think we need to restore joy in the sacred again, and I think the joy of the Lord will be our strength. Amen.
and in that joy.
That joy is not just so that we would feel good about ourselves.
That joy is the fuel for the manifold wisdom of God in the world.
Now here's the next thing he does.
He then says this this is what the Lord Almighty says.
Many people's from the inhabitants of many other cities will yet come and the inhabitants of one city will go to another and say let.
Us at once go to entreat the Lord and seek the Lord Almighty. I myself am going I love this.
This is the opposite to what we've seen over the last number of years here in Hong Kong.
God's like here's my vision for my city if people could just understand this.
I'm gonna rise up the church to be the manifold wisdom of God in society.
So that people actually see that society and go I want to be a part of that.
That society is better than my society that city is better than my city.
There's something that's going on. Those people are more joyful when they're under persecution.
Those people seem to be more joyful, even though things are hard.
Wouldn't it be amazing if people didn't say to us? Hey, do you really still want to live in Hong Kong?.
But actually said to us we're moving to Hong Kong.
Come on church.
Wouldn't that be awesome?.
Wouldn't it be so cool if people were so aware that the church is on fire in Hong Kong and that there's something different in.
The spirit of our city that people would want to come here.
Would want to actually move themselves here anytime I meet a new family at the vine who have just moved to Hong Kong.
Here's the first thought that goes through my mind every single time.
I think isn't Hong Kong so privileged that God would send this family to us.
Because there's something in this family that our city obviously needs.
Because we are the manifold wisdom of God in the world. And if this family's just moved to Hong Kong. Oh.
Watch out Hong Kong.
Because there's something that's about to happen. Amen.
Notice what he says next this is what the Lord Almighty says in those ten days in those days ten people from all languages and.
Nations will take firm hold of one Jew by the hem of his robe and say let us go with you.
Because we have heard that God is with you. I love this picture.
Zachariah sees this picture of a Jewish person standing in the city of Jerusalem.
And in those days they had these beautiful flowing robes and at the end of the rows with the tassels that mark them as a Jewish.
believer and.
Here's the vision that God has for Jerusalem and he speaks with Zachariah over his people.
He says here's what's gonna happen in the future.

$^{401}$Ten foreigners are gonna come around you and they're gonna reach out and grab a hold of the hem of your garment.
They're literally gonna grab one of the tassels that are hanging off your garment because they're gonna say this.
I know that God is with you. I don't have all the answers. I don't have it all figured out.
I don't even know if I fully believe in this God, but I can see it in you.
I can see it in the people that gather with you and I want to have something to do with that because I'm desperately in.
Need I can't find the code, but I will grab a hold of the one who has it.
When you grabbed a hem of the robe in those days it meant three things it meant protection.
refuge and security I.
Wonder if protection refuge and security could be.
Vital things for the generation of Hong Kong in this time.
The ones who are asking what is our future?.
Well, what all these changes really mean?.
The ones who are wondering about the economy and whether now might be the time to go.
Could you imagine if the church rised up in this time with a renewed?.
Hope and a renewed vision and a sense of joy that is their strength and was confident in saying that God is.
Here he has not moved to Singapore. He is right here and.
we know it and we see it and we feel it and it's good and.
People around us go. I want to hold on to something like that.
I want to grab a hold of something like that. Here's what you need to understand.
This is really important. The church is the hem of the robe of Jesus in a city.
church.
Church is the hem of his robe in a city. It's the thing that people should look to for refuge.
for protection.
for security inside their soul.
Where they feel like they've been so torn aside within themselves.
The church should be the place where they go. That's a place of refuge for me. That's a place where I can rebuild my life.
That's a place where I can find renewed hope. That's the place where when I meet those people they speak confidently boldly.
They love the city. They love what's happening in Hong Kong there. They're involved their need deep in it. They're not leaving.
They're here to stay. I want to be a part of that.
That's the hem of the robe and here's the thing you are the church.
This is why here at the vine we have a river model vision.
Our river model vision is not about trying to get us all in a building all the time and just making this about church.
The river model is about you being the church in your sphere of influence.
Because follow the analogy that Zechariah is bringing. It's no point being the hem of the robe if nobody can see the robe.
There's no point in that if people can't see that robe and grab a hold of it.
The picture he's bringing is actually a picture of the marketplace in.
Jerusalem and the Jewish people walking through the marketplace and those on the outside.
Looking at that and going I see that and I want to be a part of that.
and.
so now more than ever the river model is part of who we are at the vine because if God is saying it is through.

$^{441}$the church of the manifold wisdom of God might be seen in the world the reality is people are not.
Busting the door down to come in and find out what's going on in here.
But what they might do is see you in your workplace.
See you in your sphere of influence.
See you in the place that God has planted you and see you act different you speak different.
You have hope where no one else does you have courage when everyone's afraid you see a vision for Hong Kong.
When nobody else seems to and they see that in their workplace and they go that's different and I want to know more about that. I.
Want to grab a hold of the hem of that robe because that means something.
Let me tell you about some of the robes that are flowing here at the vine in our city.
I can tell you about a doctor a.
Doctor from within our community who's offering surgery pro bono to people who can't afford it.
I can tell you about a teacher a.
Teacher who's running afternoon classes out of their own for free out of their own time because they want to compete.
Against the the tutor schools that charge far too much money.
And they want to offer quality education free as a way of standing against some of the profiteering that's happening in that sector. I.
Could tell you about the banker who is.
Incredibly brave and trying to stand up against some unethical practices in their business even knowing it might cost.
Them their job and there are many others.
You are the hem of the robe of Christ in your workplace in your school and your family and wherever it is that you have.
a sphere of influence.
Be present.
Be present in that place because that's where people will go. I want some of that and.
Then they get to experience something like this and and find out what church gatherings are about on Sundays. Yeah, that's all good.
That's all important. We continue to do this. We love this, but they're not gonna knock the door down.
But they are gonna see you.
Being someone of joy whilst everyone else is still lamenting.
We as a church need to stop saying things like man, Hong Kong is just fill in the blank.
how amazing would it be if the church actually could change the whole conversation of the city and.
Rather than joining in to the conversation of despair actually bring a fresh vision of hope.
That's what church is about.
That's us and if that's the case.
If that's what God wants to do in the city of Hong Kong then how do we as a church become that let me let me.
Read this to you.
Just that big long quote thing that's there.
If God's vision for Hong Kong is to turn lament into joy and bring back to this city.
Hope for a good future.
Then this is his vision for the vine to be a church present in every sphere of society.
Filled with people who know God and want to make him known who are visible hopeful.
joyful courageous and loving.
That's what I'm about. That's what our leadership here at the vine is about. That's what the vine is actually.

$^{481}$ultimately going to always be about.
So I stand before you today and I make no excuses to ask you to give towards that vision.
Because what I'm really doing is asking you to give an investment into the church that you're planted in on behalf of the city.
That you're living in.
I'm asking you to give to this glorious bigger vision that I think sits there.
That's through the church. The manifold wisdom of God is made known in the world.
I'm asking you to support us because church actually matters.
It matters more than just doing the things that we do. It actually matters.
Significantly spiritually in the culture that it's planted in.
Let me just briefly tell you a little bit about that. I want to show you our finances.
This is really important just for you to know transparently where we are with our finances.
Of the white box is the white line is income. The grayer line is our expenses.
This is the last 10 months.
What you'll notice is that one two, three, four, five six seven eight of the last ten months.
We have been under in terms of our income over expenses in some of the worst months.
We've been over a million dollars under where we need to be. The average deficit is about.
\$2,000 a month over the last 10 months.
We've had a full deficit of two million dollars in the last 10 months.
And I'm sharing that with you just because we are a family together.
And that's the current state of our finances and I want you to know that that's got nothing to do a vision Sunday.
this is just what happens every month and.
Every month we're not quite making it where we need to be.
But today I stand before you because I'm actually calling you to give beyond your tithes and your offerings.
To actually stand on this bigger and this larger vision that we have and so at the vine.
We're investing in multiple ways to try to address the joy that we want to see come back to our city Oasis.
Which is our psychotherapy center here.
We're continuing to invest in that so there's good mental health.
That's happening through the vine into the city.
The majority of clients now at Oasis are actually outside of the vine community them within which we think is a wonderful thing.
We're investing in our restoration ministry because we recognize that there's so much inner healing that's needed in our city at this time.
And we're taking materials for one of the global.
schools of restoration teaching and we're actually we've paid for and we're currently doing a.
Translation into Chinese of all of that material so that we can have something that could really help and impact those within our city.
Our K4C and flight ministries, which is our generation ministries.
We're trying to double down there as well so that we're really trying to impact and influence the next generation of people.
Here in Hong Kong there are many ways our asylum seekers refugees through branches of hope those caught in trafficking.
Multiple things that we're doing to say that we're here to stay and that we love this city.
God has a heart for this city and we are the manifestation of his love through the church.
and so.
When we see that together when we hold that together.

$^{521}$We actually become the church that he truly is calling us to be my heart for you.
Is that you would feel invigorated by that vision?.
So you would feel cool to be a voice of hope in the city that God has planted you in.
He loves Hong Kong so much that he called you here. It's time to rise.
Amen.
All right. Can we stand together? I'd love to pray for you.
Father I'm just so grateful.
I'm so grateful for each person here in this room and.
Lord as we about to move into the most important part of this service.
And whether that's for people in this room watching online in our overflow.
The most important thing we do now is we stand together as a community.
sacrificially on behalf of this big vision that you have for your church.
Lord we are here because we love Hong Kong.
We have a vision in our hearts for this city and we believe that you are the answer to the locks.
That people are desperately trying to open.
we believe that you are the hope to a city and.
So Lord, we believe that you are at work here.
So Lord today we give on behalf of that vision.
We give on behalf of the fact that we believe the vine matters.
that church in this city.
matters and.
As we do that together.
Lord we do it in a place of worship a.
place of openness and a place of joy.
Lord, I thank you that you're turning our sorrow into joy. I.
thank you that you are repairing the broken walls and.
You are using the church to be a place of refuge and hope.
And Lord, I pray that each person here would feel that they are the hem of the robe of Jesus.
Wherever it is that you have planted them and I pray that today they would leave here inspired.
To live out that vision more and more in Jesus name we say amen. Let me hand back over to Ellison.
He's gonna lead us in this time.
\newpage



\section{}
\label{sec:xe7hssHqQdQ}
\textbf{2024-03-20 Conversations of Life [xe7hssHqQdQ].mp3}
\newline
\newline
連結: \href{https://youtube.com/watch?v=xe7hssHqQdQ}{\texttt{ https://youtube.com/watch?v=xe7hssHqQdQ}} ~~~~ 語音日期: 2024-03-20 
\newline
\newline
\hyperref[sec:E2_iBrxfKKc]{\small{< < < PREV SERMON < < <}}
~
\hyperref[sec:index]{\small{[返主目錄]}}
~
\hyperref[sec:eHh561juIA4]{\small{> > > NEXT SERMON > > >}}
\newline
\newline
$^{1}$>> I think the only solution is people doing more..
And it's better for the next generation to do more..
And I'm confident in the process of doing this..
>> Okay, so I'm talking about, and I'm talking about the role of the union tax..
And I'm talking about the union tax and the balance for it..
Because I think two of the most difficult things that I've talked about, that are still in the plan, is for.
the class to be fine, so that they can finish it, and that the child and the students, which is a massive number of people as well, can split as much as they want to do it..
And we're seeing a lot of students, and we have a class that's doing a very good job of it..
It's been delayed in the last couple of years, but it's been a lot of fun..
And we've got a lot of students that are doing it for themselves..
But when you're not, I know we can see, we can predict, to be confident,.
that this union is going to be good, because we're going to have some balance, and we're going to have a good union..
And we're going to have a form of growth, of a lot of growth..
But I also want to talk a little bit about the role of the union tax..
I think it's a completely different thing from the union tax..
The union tax is a class that's going to be successful..
So, when you're working on a lot of the things that are going on, and you're working on a lot of the things that are going on, so the union tax is a class that's going to be able to get a major..
And it is a class that is going to be able to work in a big way, to just get a major, and all of those are the things that are really important to all of us..
But it's also going to be a class that's going to be able to do a lot of things..
It's going to be a class that's going to be able to solve a lot of these problems, and it's going to be a class that's going to be able to make a big impact on a lot of people..
And so, you know, I think the union tax is going to be a class that's going to be able to do a lot of things..
So, we're going to have some restrictions, and we're going to have to be very careful..
And we've got to be careful, and we're going to have to be careful, and we're going to have to be careful, because I think we're going to have to be careful..
So, the union tax is a class that's going to be able to solve a lot of problems..
And it's a class that's going to be able to do a lot of things..
And it's going to be a class that's going to be able to do a lot of things..
So, I think we're going to have to be careful..
And there's a lot of people who are going to be careful..
And we're going to have to be careful..
[BLANK AUDIO].
\newpage



\section{}
\label{sec:NdzV_39RUA4}
\textbf{2024-03-25 Intimacy: Remain In Love [NdzV-39RUA4].mp3}
\newline
\newline
連結: \href{https://youtube.com/watch?v=NdzV-39RUA4}{\texttt{ https://youtube.com/watch?v=NdzV-39RUA4}} ~~~~ 語音日期: 2024-03-25 
\newline
\newline
\hyperref[sec:eHh561juIA4]{\small{< < < PREV SERMON < < <}}
~
\hyperref[sec:index]{\small{[返主目錄]}}
~
\hyperref[sec:86KwvyIr2XE]{\small{> > > NEXT SERMON > > >}}
\newline
\newline
$^{1}$Pastor Ellison, we thank you that you have put him here.
within our community..
We thank you for his leadership..
We thank you for his love for everyone..
We thank you for him, for his passion to share your love.
and your word with us..
So anoint his words now we pray,.
and I pray for each of us here that we will receive.
what you have for each of us..
In Jesus' name we pray, amen..
Amen, over to you..
- Thank you Jess, and thank you worship team..
That was amazing, good grief..
Okay, well, morning everyone..
If you're joining us online, if you're sitting over,.
good morning to you as well..
Really glad to be back in the,.
you know, we've had a bit of a renovation,.
sort of like a home edit Vine Church version..
Thank you, Tim..
Want to swap mics?.
Okay, good..
All right, so fresh building..
I hope you guys feel welcomed back..
And as Jess said, today is Palm Sunday,.
the day where we remember Jesus riding in triumphantly.
into Jerusalem, crowds worshiping with shouts of hosanna..
So it's only appropriate that we are welcoming Jesus today.
in our midst as well..
And so it really is, each time I get to share,.
it's a great honor to be able to share..
And as Jess said, we've been,.
the theme for this year is intimacy..
And so hopefully today we're gonna dig a little bit deeper.
into that word..
We really feel like God's drawing us into intimacy.
with each other, into a relationship with him,.
but that should change,.
that should affect the way we live our life..
And this passage there really hope will allow us to do that..

$^{41}$So if we're ready, let's get to it..
John chapter 15, nine to 17.
is what we're gonna be looking at today..
You could gather with me or just follow up and listen along..
John 15, nine..
As the Father has loved me, so I have loved you..
Now remain in my love..
If you keep my commands, you will remain in my love.
just as I have kept my Father's commands.
and remain in his love..
I have told you this so that my joy may be in you.
and that your joy may be complete..
My command is this..
Love each other as I have loved you..
Greater love has no one than this,.
to lay down one's life for one's friend..
If you are my friends, you will do what I command..
I no longer call you servants.
because a servant does not know his master's business..
Instead, I have called you friends.
for everything that I learned from my Father,.
I have made known to you..
You did not choose me, but I chose you.
and appointed you so that you might go and bear fruit,.
fruit that will last..
And so whatever you ask in my name,.
the Father will give to you..
This is my command..
Love each other..
So "Forrest Gump," one of my favorite movies of all time..
I shared in "Yunlong," I like "Karate Kid,".
it was the most nostalgic movie I felt like..
But "Forrest Gump," one of my favorite movies of all time..
I just love the story, right?.
A very simple man who went on to do great things..
An all-American football champion, war hero,.
international ping pong star,.
multimillionaire, right?.
The list is endless..
He did so many amazing things..

$^{81}$But despite all his achievements,.
the one thing he couldn't get right was his love life.
and his relationship with his childhood sweetheart, Jenny..
Right, on and off, he keeps on showing love to Jenny.
and she keeps running away..
And the whole story is based upon.
his relationship with Jenny..
And there's one particular scene..
Jenny has come back after running away again for a while.
and she's come back and she's been living in his house.
and they've been having a really good time together..
It's really sweet..
They get to reconnect in lots of sweet, different ways..
And so, he feels like, okay, maybe this is going somewhere..
And so he decides to propose to Jenny.
so that she will stay this time and let him take care of her..
But then Jenny rejects the proposal..
And so Forrest is, of course, very heartbroken, right?.
He doesn't understand..
Why won't she just stay?.
Why won't this woman that I love so much.
just stay and let me look after her?.
And he says this famous line, okay?.
I'm gonna give you my best Forrest Gump impression..
Are you ready?.
Okay..
(clears throat).
Why don't you love me, Jenny?.
I'm not a smart man, but I love you..
All right?.
You remember watching that scene?.
Like tears pouring down my eyes, right?.
Crying, crying, crying, all right?.
You felt the pain, right?.
The effort this man had put into loving this woman.
only to be faced with rejection..
But I also remember thinking this.
when I was watching this..
Actually, he is a smart guy, actually..
You know why?.

$^{121}$Because Forrest Gump says he's figured out what love is..
So he must be a really smart guy,.
because love is actually a really confusing thing..
Right, we say we love pizza,.
and we say we love our moms, right?.
You don't love pizza the same way you love your mom..
That'd be really weird..
So what are we talking about when we use the word love?.
It's hard to define..
I've always been confused about what this word means..
I think I was most confused.
just before I got married, I think,.
was when I was most confused about what the word love was,.
which is inconvenient, right, to say the least..
'Cause that was when I was finding out, you know,.
there's, in our premarital stuff, right,.
there's five love languages, right?.
First time I heard about five love languages..
Let me remind you, you probably all familiar, right?.
Physical touch, words of affirmation,.
gifts, acts of service, and quality time, right?.
So the basic idea is, you know,.
each one of us receives and presents love.
in different ways, right?.
It's how we show and receive love..
But this was really weird for me, right?.
Maybe a lot of you guys can relate..
I know you might have heard this before, right?.
I had no idea there was five love languages..
And maybe I wasn't the easiest kid growing up, right?.
But the language used at my home.
was threats of beating, right?.
You wanna do this?.
Okay, stop doing that, okay?.
That was the language I was used to when I was growing up..
So I had no idea there was different languages..
And when I got married, I thought,.
well, my wife is going to massage me.
while I eat my favorite meal,.
watching the TV that she bought for me.

$^{161}$because I'm such an amazing husband, right?.
That was, we love to eat, amen..
Did someone amen that?.
Do not amen that, okay?.
No, no, okay?.
What does love really mean?.
Right, it's a question countless songs, movies, poems,.
reality TV shows have tried to answer,.
but we still can't seem to figure it out..
So in this season of intimacy,.
it's important we get a good understanding.
of what the word love is,.
especially because in church we love to say,.
oh, Jesus loves you..
Oh, God loves you so much..
But what does that mean?.
How does it affect us?.
And how does it affect the way we treat each other?.
And so this passage we just read gives us an explanation.
of what we're talking about..
Jesus loves us and how it should make a difference.
in how we treat each other..
And I might not be a smart guy either,.
but hopefully by the end,.
we'll be able to see what we mean..
So let's go back to the passage and see what it has to say..
And just give us a second..
Timothy to the rescue..
This is a brand new sound system,.
which is why we're having some teething issues,.
but hopefully..
Check, check, check..
No, okay, hang on..
Back to this one..
We'll be okay, right?.
Are we okay?.
We're good, okay..
Let's just keep going, okay?.
If you need Tim, just come back anytime, all right?.
[audience laughing].

$^{201}$Verse nine says this..
As the Father has loved me, so I have loved you..
Now remain in my love..
If you keep my commands, you will remain in my love.
just as I have kept my Father's commands.
and remain in his love..
In the lead up to Good Friday and the lead up to Easter,.
this passage actually records some of the last words.
Jesus spoke to his disciples..
Before he went to the cross,.
he wants to make sure they understand who he is.
and why he came to this earth..
He wants to remind them also of who they are.
and reassure them of how he feels about them..
And the number one thing he wants to reassure them of.
is his love for them..
He wants them to understand what this love is..
See, when it comes to love, Jesus is never confused..
Jesus was always confidently talking about love..
And Jesus was always confidently showing love to people..
If you read the Gospels, he was always ready to love..
During his time on earth, Jesus loved people..
And the way he loved was radically different from the norm..
This is why the religious leaders didn't like him very much.
because he loved people that others had deemed.
unworthy of love..
But I was thinking about this,.
like who taught Jesus to be this way?.
Of course he's the son of God, right?.
But who taught him to be this way?.
Was reaching out to the poor and oppressed,.
maybe growing up with Mary and Joseph,.
something that they taught him, that they instilled in him?.
Did he grow up with his parents,.
going on family mission trips and stuff like that?.
With his brothers and sisters, reaching out to people?.
Did they spend time in the Old Testament.
reading about stories about how God redeemed.
all people into his name?.
We don't really know about Jesus' childhood..

$^{241}$But it tells us very clearly why Jesus is so confident.
he knows what love is..
He knows what love is because he has felt.
and experienced it firsthand..
From the source of love itself in the most purest form..
Jesus knows he learned how to love.
because he has been loved by the Father..
And this love that Jesus talks about.
has a very clear definition, right?.
There's no confusion..
The word that the Bible translators use for love,.
the word that Jesus used for love, okay,.
is the word agape..
You see, agape is the word that the Bible writers.
chose to represent what Jesus was talking about..
But this agape love is more than just a feeling..
Agape love is what Jesus meant when he said,.
"Love the Lord your God with all your heart,.
"with all your mind, and with all your strength.".
Agape love is the kind of love that Jesus was constantly.
demonstrating to his disciples..
It was the kind of love that moved him into action,.
healing, embracing, teaching those that no one else would..
It was the kind of love to say crazy things like,.
"Love your enemy, pray for those who persecute you.".
And it was the kind of love that eventually led him.
to the cross to die for humanity..
In other words, if we wanna know what love is,.
if we wanna know the definition of love,.
the answer is found when we look at Jesus..
And so when Jesus tells his disciples,.
"So I have loved you," it's not a joke..
He's not pizza, right?.
It's not something that he's just saying,.
coming to the end of his life,.
and he's getting all sentimental and things like that..
Jesus loves you..
And he has done everything possible to show us.
how much he loves you..
That's our starting point for today..

$^{281}$And even though it might sound cliche to say this.
on a Sunday service, we have to remind ourselves of this,.
because the truth never goes out of style..
So let me ask you again, church,.
do you know that Jesus loves you?.
Do you know that he loves you so, so much?.
I hope the answer is yes..
'Cause if the answer is yes,.
there's a way for you to respond to that love..
And Jesus says this, "If you know that I love you,.
"remain in my love.".
Okay, wow, that sounds pretty good..
In fact, it sounds great..
Do you mean all I have to do is stay in one place?.
I don't have to do anything, just remain in this place,.
and I get loved on all day, right?.
Like a baby being snuggled by his mama..
Who doesn't want that kind of love?.
But remember this, right?.
Agape is more than just a feeling..
Agape is love in action..
Which is why Jesus follows up by saying,.
"If you keep my commands, you will remain in my love,.
"just as I have kept my father's commands,.
"and remain in his love.".
Okay, now, hold on a minute..
Now it's beginning to sound a bit more like.
what we bargained for, right?.
Because at first it sounded like.
we just have to stay in one place, soak up all the love..
But now it seems like you're telling me.
that I have to work for it..
And besides, don't we always say.
that Jesus' love is unconditional?.
So what does he mean by he says, "Remain in my love"?.
Maybe we can explain it this way..
Yes, it's true..
God does love you no matter what..
His love for you is unconditional..
Nothing you can do will ever change Jesus' love for you..

$^{321}$But the point is not his love for us..
The point is whether or not you will choose.
to remain in his love..
That's the key word, right, remain..
The word remain is actually written in an active sense,.
okay, an active tense, which is something,.
that means it's happening at the same time..
This means we have to make a choice..
We can choose whether or not we want to remain.
in Jesus' love..
Loving Jesus has to be a conscious choice you make..
Because if it wasn't, it wouldn't be true love..
If it wasn't, it wouldn't be true intimacy..
And so what Jesus is teaching us is if we want to remain,.
like stay in the place of love, then yes,.
there are conditions to that..
To remain in Jesus' love means we must keep his commandment..
But understand this, church, right,.
this isn't some kind of manipulation.
that Jesus is trying to control us with..
If we think about healthy relationships,.
actually all healthy relationships have a sense.
of boundaries, rules, commands that we have to follow..
Think about a marriage or a friendship.
or a relationship of some kind..
Even if you're not married, you can probably understand.
that within marriage, right, there are certain rules,.
there are certain commands that we have to obey..
And if you want to stay in a healthy, loving relationship.
with your spouse or your partner,.
then within that relationship, right,.
you cannot have that relationship with anyone else..
If you want your marriage to thrive,.
to be all it was meant to be,.
then you have to remain in the commands of your marriage,.
right, to make it a safe place, to show love and honor.
and respect to each other..
Within this relationship, there should be healthy boundaries,.
rules, and commandments that both partners have to follow..
Otherwise, trust is broken.

$^{361}$and there wouldn't be any kind of relationship.
in the first place..
And so it is in our relationship with Jesus..
Like I said, he's already done everything he can.
to show you how much he loves you..
But if we want that thriving, healthy relationship with him,.
then we have to keep his commands..
Now, I think the uncomfortable place,.
at least for me when I was reading this here, right,.
the trigger word also is command..
Because when we think of command,.
to me, two things come to mind, right?.
First, it sounds a bit like, you know, you're programmed,.
right, I'm old enough to remember.
when I had that first MS-Docs computer at home..
Did you guys remember that?.
With the black screen, you had to actually type in.
a D-I-R slash W, whatever it was, right,.
and a whole bunch of text would come out on the screen,.
get the floppy, actual floppy disk.
that you've like put into the computer, right?.
But you put in a command into the computer,.
like you're coding these days,.
and then it comes out with a result..
So it sounds a bit like maybe that Jesus.
is just forcing us to do it..
We have no choice, he's programmed us this way,.
so we respond in a certain way..
But on the other hand, also, command sounds a bit like.
something that an evil or an abusive person,.
something like they're trying to force you to do, right?.
Because kind people don't command..
Kind people use their words and ask nicely, right?.
Command is what your 3-year-old tells you.
in the middle of the night when he says,.
"I'm thirsty, bring me water.".
Okay, that sounds more like a command to me, right?.
We don't have a choice, we're just forced to do it,.
otherwise you'll be punished, right?.
So is that what Jesus is saying?.

$^{401}$No..
Jesus's commands are not like this..
He's not trying to program you into intimacy with him,.
nor is he using his commands to be evil or abusive,.
because remember the source of where Jesus's commands.
are coming from..
It's from a place of love..
Jesus loves us, he commands us based upon the love.
that he has received from the Father..
And so whatever he tells us to do,.
it's always gonna be from a place of love..
And because of this, we can trust, we should trust,.
that following what Jesus has called us to do.
is the best way to live our lives..
In fact, I would go even further and say this..
Obeying Jesus is what you were created for..
Because, check this out,.
the outcome of obeying Jesus's commands,.
the outcome of remaining in his love,.
it's not something that feels you abused,.
manipulated, resentful, right?.
Jesus doesn't force you like a robot,.
but the outcome of following Jesus's command is joy..
I have told you this, he says,.
so that my joy may be in you.
and that your joy may be complete..
If you want to live a life that's full of meaning,.
if you want to know what true joy is,.
then follow his commands..
And the reason why following Jesus's commands.
bring us so much joy is because, like I said,.
this is, I believe, how we,.
this is what God made us for..
This is what Jesus created us for..
The best way to live, I'll say it again,.
is to obey Jesus's commands so that our joy be full..
Notice this, though, church, it says joy..
Doesn't say happiness, doesn't say excitement,.
doesn't mean the most luxurious, most comfortable life,.
but joy, and joy is different from those things..

$^{441}$You see, when we hear the phrase joy might be complete,.
it sounds a bit like the most comfortable life.
you can imagine, maybe, right?.
So, complete life of joy means no worries, no cares,.
chilling on a beach with a cold beer,.
in a library with a book, whatever your,.
you know, your picture of serenity is..
It sounds like it's saying that, but that isn't it..
Complete joys come, complete joy comes when we live.
how we are called to live, obeying Jesus..
And when that's the case,.
then there's all kinds of possibilities..
And yes, there might be some really good, of course,.
but it will almost most definitely include.
walking through some dark and hard and rough places..
It will most definitely include being rejected by the world..
In fact, a bit later on, Jesus goes to say,.
"Don't be surprised if the world hates you.
"because it's hated me.".
It will most definitely include making an effort.
to fight back against the darkness and the evil.
and the sicknesses of this world..
The point is that our joy is not based upon situation.
or circumstances or anything on this earth..
Our joy is based on the fact that we are doing.
what we were created to do, which is obeying Jesus..
And if you wanna experience true joy,.
that will never come until we learn what it means.
to follow Jesus' command, to live as he has called us to live..
Now, luckily for us, he doesn't leave us in the dark..
He tells us exactly what the expectations are..
Jesus says this, "My command is,.
"love each other as I have loved you.".
Intimacy with God will always result.
in intimacy with each other..
To remain in Jesus' love, to keep his commandments,.
the next logical step is that we will love each other..
See, it all links together..
Jesus knows what love is because he has felt it.
from the Father, he's experienced it from the Father..

$^{481}$We know what love is because we have experienced it.
from Jesus..
Now, our task is to go and love others with that same love..
To love Jesus is to love each other..
And the way Jesus has called us to love each other.
is through friendship..
We are friends of each other..
We are brothers and sisters of each other..
We have been called to be friends with each other..
But this isn't a friendship as the world might define it..
This isn't friendship that's just based upon.
getting together with some people with similar interests,.
that surface-level relationship we might be used to..
If this was a way to measure true friendship,.
then I'm friends with the whole of Hong Kong..
You could literally sit me any time, anywhere,.
I will talk to anybody, okay?.
I love talking to people..
Okay, but it's a deeper connection than that..
It means more than that..
The definition that Jesus gives for friendship.
is far greater..
Jesus says this, he defines just how deep and fierce.
this friendship that he expects for us to have each other is..
Greater love, he says, has no one than this,.
to lay down one's life for one's friends..
So, just to be clear, Jesus is saying that your friends.
are the people that you're willing to die for..
In that case, our list just got pretty small, right?.
Nathan DeLista, Andrew Gardiner,.
Promise Armstrong, okay?.
I don't think there's many people.
I would be willing to do this for..
So that Jesus really expect us.
to uphold something so extreme,.
but it shouldn't surprise us that Jesus said this,.
actually, like I said, these are his last words.
to his disciples and his friends..
He is literally hours away from laying down his life.
for their sake..

$^{521}$He's hours away from demonstrating.
the most radical, extravagant form of love,.
dying on the cross for his friends..
So yes, on one hand, I do think this is something.
that Jesus really expects from his followers..
If we are really to love each other as Jesus has loved us,.
if we mean what we say that we're gonna be Jesus.
to each other, then that means that we love each other.
enough to literally give up our lives..
The greatest way, Jesus says, to show love for each other.
is to lay your life down for your friends..
However, and I would even say, thankfully for us,.
here at the Vine Church today,.
in the church in Hong Kong, in our current context,.
hopefully, none of us will ever be able to ask.
to do something drastic..
I hope we're never placed with that horrendous choice.
of being forced to give up our lives.
and die on behalf of someone else..
But just because it doesn't have to go.
to that extreme right now,.
we still can apply this to our lives..
So what does that mean?.
In the everyday way we live our lives,.
what does it mean to lay your life down for one's friends?.
The art behind laying your life down for your friends.
for the sake of others is more about having.
a self-sacrificing nature..
Laying your life down for your friends in this light.
means that you're willing, perhaps sometimes,.
to deny yourself so that others can have..
Laying your life down for your friends.
means that you think of others before you think of yourself..
Laying your life down for your friends.
means that you sacrifice so that others.
might have what they need..
It's about an attitude, it's about a perspective change.
on how you live your life..
It's about saying, I'm willing to give up my own rights,.
to give up my own comfort zone,.

$^{561}$to be stretched in my generosity so that others can be free..
Let me give us some Hong Kong examples..
Perhaps on a very, this is just like.
basic basement level entry, right?.
Laying your life down at your friends is,.
maybe it's not pretending to be asleep or distracted.
when you're sitting on the bus on the MTR.
and someone more in need of a seat walks on..
Oh, this is a very Hong Kong problem, right?.
We, a lot of us have selected blindness.
when it comes to this church, right?.
We sit there on the seat and then,.
oh, hopefully no one in this room does that,.
but other people we see, right?.
Someone in need gets onto the bus.
and they just pretend like, suddenly they're sleepy.
and they fall asleep..
Suddenly they can't get off their phone, okay?.
Give up your seat, you see someone in need..
Sacrifice a little something for the sake of your friends..
Or maybe you can even take it a step further.
and be proactive and act out injustice.
and be like, excuse me, don't you have a scene?.
There's a paw paw standing right here..
Would you like to get up your seat.
so she can have a seat, right?.
It might be a bit scary to do that,.
but that's one way we show our friendship to each other..
Being friends to each other might be participating.
in some of the neighborhood conversations.
that we've been having as a church..
Reaching out to the community in Wan Chai,.
telling, getting to know them,.
asking them what their needs are.
and seeing if we can meet them in some practical way..
Maybe it means volunteering some of your time.
to serve those who are most vulnerable..
Maybe you could consider to be a foster family.
to care for children who are in need..
Join our Arise community, join our volunteer team..

$^{601}$Maybe laying your life down for your friends.
means refusing to let the busyness of the office.
dictate how you live your life..
But set up healthy boundaries.
so you can set aside time for your family.
and your friends and the people that you love..
Are we willing to live like this church?.
To live in a self-sacrificial way.
so we're not just living for ourselves,.
we're living to show Jesus' love for other people..
It's counter-cultural..
It flies in the face of a city like Hong Kong.
that says gain as much as you can,.
do as much as you can for yourself..
Do whatever it takes for yourself to succeed..
But if you really want to remain in Jesus' love,.
if you want Jesus to be your friend,.
if you want intimacy with Him,.
then this is the way we've been called to live..
Jesus is saying the way I define love and intimacy.
is that you are willing to obey my commands..
And my command is that you're willing.
to live sacrificially for each other..
Those who do not are not my friends,.
nor are they friends of each other..
Verse 14 makes it very clear..
You are my friend if you do what I command..
Now again, it sounds very one way right now..
You know, just follow Jesus,.
just follow Jesus, follow Jesus..
Well yes, that's part of it..
But there's also great benefits that come with this too..
When we are in that deep intimate friendship with Jesus,.
when we are being His love to others around us,.
it also results in honor..
And benefits, listen to what Jesus says..
I no longer call you my servants,.
because a servant does not know his master's business..
Instead, I have called you friends..
For everything I've learned from my Father,.

$^{641}$I have made known to you..
This is beautiful..
Right, if we were just workers and servants of Jesus,.
He wouldn't owe us any kind of apology..
Yes, He would just simply command us to do things,.
and we would just simply follow..
But this is a relationship..
When you enter into that friendship with Jesus,.
He speaks to you, He talks to you,.
He guides you, explains to you,.
He tells you what it means to follow Him..
He wants to reveal things to you,.
to help you to understand why He's called you to live.
in the way that He's called you to live..
And this should blow our minds,.
because Jesus chose you to be His friend..
You did not choose me, but I choose you,.
and appointed you so that you might go and bear fruit,.
fruit that will last..
Think back to your school days..
Was there ever a time where, you know,.
or maybe you are the popular person,.
congratulations to you if you are, okay,.
but if not, did you ever think, you know,.
that really popular person.
that you wanted to be friends with, right,.
you just wanted that person to choose you.
to be your friend so much..
Or maybe you were in PE, right,.
and that dreaded moment where you split off into two teams,.
and there's two captains,.
and they slowly pick one person in each team,.
and the team gets dwindled down until, you know,.
if you're me, I always pick last,.
'cause we're playing basketball,.
and no one wants the short guy on their team, right?.
And so I was never actually picked, I felt like..
They just ended up with me, okay?.
Like, they were stuck with Ellison, okay?.
Oh, don't cry for me, it's okay..

$^{681}$But that's not how God chooses people, right?.
This isn't some kind of schoolyard,.
childish popularity contest..
Jesus has chosen you,.
despite of your insecurities,.
despite of your lack of ability..
Jesus has chosen you..
In fact, throughout the Bible, throughout Scripture,.
Jesus always chooses the kind of people.
that no one else would choose..
God chose Abraham, a pagan, to become a great nation.
that he would bless other nations out of..
God chose Moses, a coward, to go in front of Pharaoh,.
the most powerful human being on the earth at that time,.
to bring freedom to the Israelites..
God chose David, a shepherd boy,.
to become one of the greatest kings.
that Israel had ever seen..
And God chose the prophets..
If you read through the prophets,.
all of them were messed up..
So many issues, but yet he chose them.
to proclaim God's word, to bring warning,.
correction, healing, and hope to the nations..
And so it shouldn't be surprising, right?.
Even when we look at Jesus' disciples,.
they weren't the sharpest tools in the shed either..
Peter, who was always hot-headed, abrasive..
Thomas, who doubted Jesus..
Judas, who even betrayed him for some coins..
But all of his disciples, excluding Judas,.
went on to establish the church,.
to continue the work that Jesus had called them to do..
They went on to bear fruit,.
fruit that has lasted to this day..
Because of what Jesus' disciples did,.
because they remained in his love,.
because they followed his commands.
and were obedient to him,.
because they chose to love each other,.

$^{721}$we have the church today..
Listen to me, church..
This is what I'm trying to say..
Today, now, Jesus continues to choose people.
to be his friends..
And we have all been chosen to be friends of Jesus..
Not because of your ability,.
not because of who you are..
We are not chosen based on some kind of privilege..
We have been chosen because he has chosen you.
for a purpose..
We have to cherish this church..
You have been chosen by Jesus for a purpose..
And that purpose is to bear fruit..
You have been chosen by Jesus to bear fruit..
And if Jesus has chosen you as a friend,.
there is no way your life can be insignificant or pointless..
He has something for you..
Your purpose is to bear fruit..
That's your calling..
And again, the amazing thing is that can happen.
in so many different ways..
The gifts, the talents, the things that God has given to you,.
the ways that Jesus has created you,.
you have been chosen to do those things.
so that you will go and bear fruit..
Fruit that will last..
That might mean faithfully praying for somebody,.
sharing the gospel with somebody..
It might mean refusing to give up on someone else,.
that's everyone else has given up on..
It might mean visiting and connecting with people.
that other people have forgotten about..
Opening a home, whatever that is..
The point is that connection and intimacy.
with Jesus always bears fruit in the way we love each other..
Whether this is in terms of a single life changed.
because we have loved somebody as Jesus has loved us,.
in terms of a single tough decision that you had to make,.
a single task which you had to perform,.

$^{761}$and at that time you might not have seen.
the significance of it,.
but you know that the world is a better place.
because you chose to remain in Jesus' love,.
obey Him, and love others in the same way..
This is the fruit that will last,.
not just in this age,.
but have an eternal impact on this world..
This is what we're called to, church..
And so these should become like the prayers of our heart..
Finally, Jesus says this,.
if you do this, whatever you ask in my name,.
the Father will give to you..
Whatever you ask in my name,.
the Father will give to you..
The point here is not that Jesus is some sort of.
thought machine that if we just have enough faith,.
then we pull the lever and the prizes.
of answered prayers will come out..
That's not what Jesus is saying..
But what Jesus is saying,.
because you've been so connected to me,.
because we've been walking intimately.
with love in each other,.
because you've remained in my love,.
and because you've now stepped out in faith.
and chose to love others in the same way,.
we have the same heart..
The things that break God's heart breaks your heart..
We're praying for the right things..
We wanna see the same things that Jesus wants to see happen..
And so when we're doing these things,.
we have to realize that this is what God wants..
And so our prayers will be answered.
because we are following in His commandments..
So when you do these things,.
maybe some questions to consider..
Are we praying for the right things?.
When we talk to Jesus, have we asked Him,.
what do you want me to pray for, Lord?.

$^{801}$How can I pray for this situation?.
Do we have the mind of Christ when it comes to prayer?.
Are the things that break Jesus' heart.
the same things that break our heart?.
And when we're answering these questions,.
when we're living in His love,.
when we're following Him in intimacy,.
our prayers are answered.
because we're working in connection.
and the power of who Jesus is..
It's all about yielding your heart to Him,.
allowing Him to work through you at all times,.
for Him to take control of what you say,.
what you think, and what you do,.
to remain in His love..
And when we do this, you will see your prayers answered..
You will see Jesus move and work in this world..
So it's simple, but it's challenging..
And Jesus wasn't joking when He said,.
all the laws of the prophets hang on this..
Love the Lord your God with all your heart,.
mind, and strength..
Love your neighbor as yourself..
As a community, friends, I pray,.
our prayer is that we would all remain in Jesus' love,.
to be obedient to His commandments,.
so that we can change a world that needs it so badly..
Remain in His love, reach out to others in that same love,.
so that we can bear fruit,.
fruit that will never lose its purpose,.
fruit that will last from now until all eternity..
I wonder if you pray with me, church..
(audience member coughing).
Perhaps for some of us, like I said earlier in this room,.
that just the first step of remaining in Jesus' love.
has been a big challenge..
And maybe you found yourself, you know,.
oh yeah, I had that one time,.
or I've never really understood what it was..
And so it seems scary and seems foreign,.

$^{841}$and you ran away from it, or you've stayed away from it..
And the first step, like Jess was saying just now,.
is just to take that step toward Jesus..
He has already chosen you..
He has already loved you..
You don't have to be worried if you approach Him,.
He's gonna reject you or say no, or say,.
ah, not for me, not on my team..
But Jesus has a purpose for you..
He created you to follow what He wants for you to do..
And there is so much fruit.
that you're gonna see in your life.
if you choose to remain in Jesus' love.
and be obedient to Him..
So if you found yourself feeling lost and restless.
and joyless and meaningless with your life,.
come back into His love and stay in that place..
And as we stay in that place, Jesus, I pray that.
you would help us see the joy in following your commands..
Even through the toughest of times,.
even when it seems like it doesn't make sense to claim joy,.
Jesus, I pray that we would remain in your love,.
that we would never stray away from it..
And as we do that, Lord,.
help us to love others in the same way..
Help us to be friends to each other..
Help us to be friends in a world.
that so desperately needs friendship and community and love,.
that is so lost and so hungering for this..
And as we do that, Lord, we bear fruit..
Lives are changed..
People are brought from a place of hopelessness to hope,.
of death into life, of darkness into light,.
of sorrow into joy, because we have been obedient to you,.
because we have been friends of you.
and friends to each other..
So Jesus, continue to walk with us..
Thank you for being our friend..
May we be your friends in return.
and be friends to those around us..

$^{881}$In your name we pray..
- Church, let's all stand as we respond in worship..
(gentle music).
♪ So lonely, there is no one like you ♪.
♪ There is none beside you ♪.
♪ Open up my eyes in wonder ♪.
♪ Show me who you are and fill me with your heart ♪.
♪ And lead me in your love to the world around ♪.
♪ Lonely, there is no one like you ♪.
♪ There is none beside you ♪.
♪ Open up my eyes in wonder ♪.
♪ Show me who you are and fill me with your heart ♪.
♪ And lead me in your love to the world around ♪.
♪ Holy, there is no one like you ♪.
♪ There is none beside you ♪.
♪ Open up my eyes in wonder ♪.
♪ Show me who you are and fill me with your heart ♪.
♪ And lead me in your love to those around me ♪.
♪ And I will build my life upon your love ♪.
♪ It is a firm foundation ♪.
♪ And I will put my trust in you alone ♪.
♪ And I will build my life ♪.
♪ And I will build my life upon your love ♪.
♪ It is a firm foundation ♪.
♪ And I will put my trust in you alone ♪.
♪ And I will not be shaken ♪.
♪ Holy, there is no one like you ♪.
\newpage



\section{}
\label{sec:bacTl9XHWhY}
\textbf{2024-04-08 Ephesians: Grace \ and  Peace [bacTl9XHWhY].mp3}
\newline
\newline
連結: \href{https://youtube.com/watch?v=bacTl9XHWhY}{\texttt{ https://youtube.com/watch?v=bacTl9XHWhY}} ~~~~ 語音日期: 2024-04-08 
\newline
\newline
\hyperref[sec:86KwvyIr2XE]{\small{< < < PREV SERMON < < <}}
~
\hyperref[sec:index]{\small{[返主目錄]}}
~
\hyperref[sec:qcO4mIHWxfY]{\small{> > > NEXT SERMON > > >}}
\newline
\newline
$^{1}$Amen..
You know, when I stop for a moment and think about the incredible things, the blessings.
that have been brought into our lives because of Christ, I can't help but stop in my busy.
life and actually just worship Him like we were doing just a moment ago, just giving.
Him honor and giving Him praise..
I mean, stop for a moment and think about what He has done in your life..
You have been handpicked before the creation of this world, handpicked by God to be in.
relationship with Him..
I mean, think about that for a moment..
He has chosen you as an act of grace, chosen you not because you earned it, not because.
you deserve it, but because out of His great love, He wanted to be in relationship with.
you..
I think of this a little bit like my wife and I and our daughter, Mia..
God has adopted you into His family..
You know, Mia, our daughter, she didn't choose Chris and I..
Chris and I chose her, and in choosing her, she ended up coming into relationship with.
us and choosing us..
And in the same way, you didn't choose God..
God chose you, and in choosing you, you have now come into a relationship where you are.
now choosing Him and walking with Him, and that's a profound thing, not because you've.
earned it or deserved it, because He acted first..
And here is how He chose you..
He sent His only Son to die on a cross for all of your sin and all of your brokenness,.
to stand in the gap to create unity between you and reconciliation between you and God.
once again..
He did that out of just a place of love and grace..
And the reason why He's done that is He wants you to understand what His greater purpose,.
what His greater will is in this world..
Here's His will, that everything on heaven and on earth would all come together under.
one person, that person being Jesus Christ..
Again, think about that for a second, that everything in this world, everything that.
is being created would come under one authority, the person of Jesus Christ..
I mean, that should just blow your mind..
Think about all the things He has done personally for you..
He has chosen you, He has adopted you, He has saved you, He has filled you with His.
power of His Spirit, and He sent you out to be His hands and feet in the world..
That, my friends, is purely grace..
And because of all of that, here's how I pray for you..
I pray that you wouldn't just believe in God, but that you would actually live a life.
that would show the world what that belief truly means to you..

$^{41}$I pray that you would come to a knowledge of God, or knowledge of all of these incredible.
things that He has done for you on your behalf by His grace..
I pray that the eyes of your heart would be opened so you would come to a realization.
of the inheritance that you have as a believer in Jesus..
You would come to understand, because the eyes of your heart have been opened, that.
you would see the hope to which you have been called as a Christian..
This hope that everything that God has created in the world, all authorities and powers and.
principalities, that all of these things sit under Christ Jesus..
That there is nothing above Jesus..
That His name is the highest name, and that at His name every knee will bow..
That God has placed everything under Christ's authority for the power and the glory of the.
church..
So think about the transformation that has come into your life..
Because it's true to say that at one point you were spiritually dead..
You were dead in your sin..
And being dead in your sin, you were separated to God..
That's what sin does..
It separates us from God..
And God hates sin, but He looks down on us..
And out of His mercy and His grace, He reached before us..
And in the work of Jesus on the cross and the power of the resurrection, Jesus has taken.
the place of sin in us and provided mercy..
Has taken the place of sin in our lives and now given us grace and forgiveness..
And everything that He's done is not because we prayed for it..
It's not because we asked it..
It's not because we worked really hard to achieve something..
He's done it all out of the wonder of His grace..
Think about this..
Your salvation is by grace and grace alone..
Nothing that you have done has accomplished what Christ has done on the cross..
He's done it purely out of His grace for you..
You are quite literally a poem of God that God has written so that the world can read.
about that grace..
You are God's workmanship..
And you've been created in Christ Jesus for these incredible works that He has for you..
Think about that..
You are His work so that you can go and do His work..
That is the beauty of what Christ has done in His death and His resurrection for you..
Once you were separated from God, but you're no longer separated from Him..
He's actually brought you near..

$^{81}$This is what Jesus' whole ministry and purpose was when He was here..
It was to reconcile the things that were separated from God, to draw the things that perhaps.
shouldn't have been brought together..
Now bring them together under the reconciliation of God's forgiveness and love..
That, my friends, is a picture of what the church is all about..
I mean, if we're honest with ourselves, many of us in here probably shouldn't be sitting.
in the same room as each other..
The only way that we can do this is because of the reconciling unity and love that Christ.
has done on the cross for us..
I mean, look around this room right now..
I mean, I love this about the vine, that there are local Chinese here in this room..
There are overseas Chinese..
There are Asians from almost every Asian country..
There are foreigners from all over the world all gathered in this room right now..
There are wealthy people..
There are poor people in this room..
There are those that have great influence and power and leadership in our city in this.
room..
And there are some of you who live a very quiet life here..
All of us are able to be together because of the reconciling love of Jesus..
That is the church..
Think about it this way..
We're like an apartment building in Hong Kong, okay?.
We're like an apartment building..
And our apartment building is rooted deep into the Word of God and the teachings of.
the early church..
And we'd be constructed with the bamboo and the cements, if you will, of the power of.
the Holy Spirit..
And our apartment building has one management team that runs it, the Father, the Son, and.
the Holy Spirit..
And each one of us, we have an apartment in that building..
And all together, we share in the same utilities and the same power, but we have our own home.
and we have our own lives..
That is the picture of the church..
That my friends, is the gospel..
And this gospel, our city of Hong Kong so desperately needs to hear..
I have given my life on behalf of this gospel..
And I don't say that because I'm some perfect person..
No, I stand here today recognizing the sin that is in my own life, the brokenness that.
I own carry..

$^{121}$And sometimes that sin tells me that I am not worthy to be doing this..
Sometimes I think because of my sin that I shouldn't be a pastor, I shouldn't be doing.
what I do..
And then I'm reminded that it's not about my abilities, it's not about my effort, it's.
not about what I do, it's about the call of God that is on my life..
He has called me and He that has called me is faithful..
And He has also called you..
He has called you to be wherever it is that He has planted you, to release something of.
His wisdom in the world..
That's what the church does..
We get to represent, to show the world a manifest expression of the wisdom of God..
What an honor that is for us as the church to do that..
And let's face it, sometimes as a church, we're pretty messed up..
Sometimes we're pretty broken..
Our apartment building, if you will, is a little bit run down and in need of renovations..
We're the kind of apartment building that most people do not want to rent a flat in,.
okay?.
We're the kind of apartment building that when a typhoon comes, the windows rattle and.
it gets a little bit leaky..
But guess what?.
God has created the weak things in this world to confound the strong..
God has made those things that don't on the surface appear like they're gonna have any.
value at all to be the ones that give hope and value into the world..
That's the church..
So here is how we pray..
We pray that God would fill us with such strength in our inner being by the power of the Holy.
Spirit..
And we pray that for you who already have this idea of the grandeur and the wonder of.
love in your life, my prayer is that you would come to see the height and the depth and the.
length and the width of that love like never before..
And in seeing that kind of love, it would set your heart alight because you would realize.
that He is able to do immeasurably more than you could ever ask or imagine so that His.
glory is seen not just through Christ Jesus as beautiful as that is, but also through.
the church for His glory..
Now because of all of that, here is what I plead of you..
Do not waste your life..
Do not treat God like a hobby, like you might go hiking or running or going to the gym..
God is not a hobby to be segmented into some moment of your life..
God is your whole life..
And because He is your whole life, you are swept up in this wonder and this beauty of.

$^{161}$what it is to live His life in this world..
And it will look different for every single one of us, but it will hold these things that.
are values in the kingdom, things like grace and humility and love and kindness and goodness..
And as we hold to these things together, one another, we will find unity amongst us in.
a beautiful way because we serve one God, one spirit, one church, one family, one faith,.
one hope..
And in that unity, God moves powerfully..
You know what He's actually done in the church?.
He's actually raised up apostles and prophets and teachers and pastors and those who are.
able to serve in beautiful ways to help and equip you so that you might become the fullness.
of who you've been created to be..
God sends these leaders within a local church body to enable you to be prepared for those.
works that He has called you to work in whatever sphere of influence that you have power in.
this city..
He wants you to be His hands and feet in a beautiful way..
And guess what?.
When you work that way, and when those apostles and those prophets and those pastors and those.
teachers and those leaders, evangelists, they also work that way, the whole church is built.
up into this maturity, this maturity in Christ Jesus and in the knowledge of Him so that.
we would come to the fullness of the fullness of Christ..
What a picture..
I wonder if you could imagine that..
The fullness of the fullness of Christ..
And when that happens, no longer are we susceptible to being swayed back and forth by every whim.
of teaching, by whatever is trendy and cool..
No, instead, we're rooted in the reality that Jesus is our head..
Think of it here in Hong Kong..
Think of it like a lion dance, okay?.
God is our head..
Jesus, our head..
We're the ones who get to be in the tail..
We're the ones that get to run alongside of Him..
We're the ones that actually are in the dance..
We get to make the dance beautiful..
So let me challenge you, put your dancing shoes on..
You're in the dance..
You're not a bystander..
You're not a tourist with some camera and some fleeting memories..
No, you're actually a local in the house of God..
You're a part of the team..

$^{201}$You're on the dance team..
Put your dance shoes on and get yourself excited..
Put on the new self that Christ has paid the price for..
Put on the new attitudes that Christ has given you..
It's like literally putting on new clothes that you can show the world..
Now if you're going to put on some new clothes, that probably means that you might have to.
take off some old clothes..
Clothes like anger and deceit and lying..
You should wear clothes that represent what is truly actually in your heart..
And so often we don't do that..
We think we can wear clothes of humility and truth and falsehood and pride at the same.
time..
If you try that, you will look pretty stupid..
My suggestion to you is to ensure that everything that comes out of your mouth actually represents.
what you believe in your heart..
Those of you who have been stealing before, you shouldn't steal anymore..
Those of you who have lied before, you should stop lying..
You should correlate the words of your mouth to the beauty that is within you..
This is the simplest way I could say it..
Imitate God..
How do you imitate God?.
You receive the fullness of His love and then you show the world that love..
How has Christ shown His love to you?.
He has sent Himself to the cross to die for you..
The question then you should ask yourself is this, what act of self-sacrificial love.
could I show somebody today?.
Do that and you imitate God..
But if you're anything like me, you can end up imitating other things..
I imitate things like anger at times, bitterness..
I imitate at times lust, things that I struggle with..
And maybe there are some things that you also imitate..
The reality is we can so end up imitating Satan rather than God and that has no place.
in the kingdom..
Instead, you are children of light..
So live in the light..
Shine your light brightly in the city of Hong Kong so people would know the gospel that.
sets them free..
And if you see any darkness around you, you should know that you're bringing the light.
of Christ into that place of darkness..
If you see any injustice around you, you should know that you're living as a child of light.

$^{241}$to show gospel and truth and hope and justice in the very places where we see the greatest.
brokenness..
That's what it is to be the people of light..
And Christ has called us to do that..
So if you want to act wisely, live in the light..
If you want to live in the light, well then you should imitate God..
And if you want to imitate God, you should love like God loves..
You should probably write that down somewhere..
If you want to act and live wisely, you should live in the light..
If you want to live in the light, you should imitate God..
And if you imitate God, that is to love like God loves..
Do that and this world truly changes..
What I want you to do every single morning when you wake up is I want you to think about.
this in your relationships because it is living as children of light in the areas of our relationships.
that gets transformed the most..
I mean you should be speaking words of love and encouragement over one another..
You should even be singing spiritual songs over one another as strange as that might.
sound..
Think about the relationships in your lives..
If you're a wife here, you should be thinking about how do you best support and grow and.
flourish your family..
How do you support your husband and lift him up?.
If you're a husband here, you should be thinking about how you can love your wife like Christ.
loved the church and gave himself up for the church..
You should be thinking about what it is to love your wife like you would love your own.
body, caring for her like you would care for yourself, putting her first..
If you're a child here, you need to honor and be obedient to your parents..
If you're a domestic worker here, you ought to work with an open and loving heart..
If you're an employer here or somebody in leadership or government here, then you should.
govern with humility and dignity and love..
When we treat one another like this, we actually show the world something of what Christ has.
done in his death and his resurrection..
So every morning when you wake up, do this spiritual checklist and do this in order to.
protect yourself from the work of the enemy in your life..
Here's the first thing you should do..
Get truth into you like a daily breakfast..
Eat up God's Word first thing in the morning..
Before you jump online, get on email, do Twitter or anything like that, Instagram, get the.
Bible open, read his Word, listen to it, put it in your heart..
His Word first, the truth, like a daily breakfast..

$^{281}$Here's the second thing..
You should put on your new identity and wear it proudly..
Maybe like a comfortable t-shirt that you love, your favorite t-shirt, putting on that.
t-shirt to show the world who you are..
That's the new identity Christ has given you, and you should live that out in the world.
proudly..
You should think every day thirdly about what it is to walk in peace..
Think about the most comfortable shoes that you have and what it feels like to walk in.
those comfortable shoes..
That's how you should feel every second of the day..
And then have faith that God will protect you..
Faith is like your skin, protecting and covering up your organs so they're resisting from decay..
That's what faith will do in your life..
It'll enable you to know that God is with you no matter what it is that you are facing..
And then think about this..
I want you to know that God is with you above everything else..
That's like your mobile phone..
Your mobile phone is always with you..
Some of you, it's glued to your hand..
Think about God's presence in your life like a mobile phone, always with you..
And I want you to pray..
I want you to pray in every occasion with every kind of prayer..
And here's the beautiful thing..
When you pray for one another, you will be astounded at the healing you will receive..
And if I had one final request, it would be this..
I would love it if you would pray for me..
Pray that I would declare the gospel of Jesus Christ in Hong Kong fearlessly as God has.
called me to do it..
I pray that you would live in grace and peace..
And as you live in grace and peace, you will find the glory of the Lord around you and.
you will find yourself safe and comfortable and alive like never before..
And I pray this in Jesus' name, everyone says, amen..
Well in case you didn't realize, that's the whole book of Ephesians from start to finish..
Yeah..
So I wanted to....
We're starting a whole study, a whole sermon series, 15 weeks on the book of Ephesians..
But before we jump into studying it passage by passage, I actually wanted to bring it.
to you today just like Paul originally intended it to..
Because when he wrote this letter to the church in Ephesus, a small church meeting in some.
houses in the city of Ephesus, a church under great persecution, a church that was struggling.

$^{321}$to understand its identity, Paul writes them a letter not to be dissected and studied and.
all of this kind of stuff..
He writes them basically a conversation that he's trying to have with them to help them.
to understand who they are in Christ Jesus, what Christ has done, and then in that to.
live in a way that would honor and glorify Christ in the world..
That's why he wrote the book of Ephesians..
And I think so often as Christians, we open the book and we go to our favorite section.
and we read that little bit, but actually the whole letter was designed to be read in.
one sitting..
It was designed to be heard just like you just heard it..
And of course I changed it and paraphrased it to make it like Hong Kong because I think.
that's what Paul would want me to do..
Because he wrote the letter to people in Ephesus for specific issues that were going on in.
Ephesus..
I translated it for us in Hong Kong for the specific issues that we're seeing, but wasn't.
it glorious just to sit under the beautiful Word of God..
This idea that there is so much that Christ is doing in our lives, so much that he's done.
that we have not earned..
And when we understand the beauty of that, we're then able to walk out as his representatives.
in the world around us..
That's what Ephesians is all about..
And in the first two verses of the letter, just the first two verses, Paul actually introduces.
the two main pivot points of the whole book..
So I'm actually now going to read you the actual book of Ephesians..
I'm not going to read the whole thing..
You'll be glad to hear it because I just told you the whole thing..
So let me read you just the first two verses..
Is that okay?.
All right..
Some of you were like, "What is Andrew doing today?.
Did he have a weird curry last night?".
You were thinking like, "This is not like he normally does it.".
Anybody felt that way?.
Did I turn you off?.
Hopefully you stuck with me..
Is that okay?.
All right..
Ephesians 1, first two verses, "Paul, an apostle of Christ Jesus by the will of God,.
to the saints in Ephesus, the faithful in Christ Jesus, grace and peace to you from.
God our Father and the Lord Jesus Christ.".

$^{361}$I love how Paul begins..
He says, "Grace and peace to you.".
This was not just a standard greeting in those days..
This was actually something that Paul formulated and used in almost all of his writings, in.
almost all of the letters that he wrote to the church, he would often start with this.
idea of grace and peace..
And here in the book of Ephesians, the whole book is actually structured around these two.
themes of grace and peace..
This is not Paul just going, "Hey, how are you?.
Good to see you.".
This is Paul trying to explain something that is central to the heart of God and something.
that should be central to every church community, not just in Ephesus, but I think for all time,.
grace and peace..
Let me break these down real quick..
Grace is the Greek word here, charis, which comes from the root word kario..
And that word in the Greek essentially means this, an unmerited favor that causes you to.
celebrate, rejoice, or be glad, unmerited favor..
In other words, what grace is defined is as a gift, something that you've never earned,.
something you don't deserve, even something that you've never asked for..
It's like a surprise that is given to you out of the blue, and you're just blown away.
that somebody would be so generous..
That's the idea of grace..
And what's really important for you to grasp just right at the start of the book of Ephesians.
is that this word grace in the Greek comes in the passive form, which makes sense because.
it is a passive thing..
Grace is not something you do..
Grace is something you receive..
Grace is not something you earn..
Grace is something you understand..
And as you receive and you understand it, it is something that comes into you, which.
causes you to do nothing other than to rejoice and be glad..
That's grace..
And that's why Paul spends all that time in this letter saying you need to understand.
that your salvation is not because you did something..
Your salvation is a gift of grace from God to you, unmerited, because He loves you, and.
that should cause you to celebrate, to worship, and to praise Him..
Grace is a surprise of the generosity of God..
Peace is different..
Peace actually has two words that are associated to it..
In the Greek, it's this eriane..

$^{401}$In the Hebrew, though, which is what it draws from, is the word shalom..
When we think of peace in the modern context, we think peace is the absence of something,.
usually the absence of conflict..
So we might say, "Oh, that home is very peaceful.".
What we're meaning is they're not throwing things at each other in that house, right?.
Or we might pray for peace in the Middle East, and what we mean is that war would cease there..
But in the Hebrew thinking, peace is not about the absence of something..
It's actually about the presence of something..
And it's actually about working towards the presence of something, the presence of wholeness,.
of completeness, of fullness, of how something ought to be, the presence of safety, of security..
Another word associated to it means shalom, which is reconciliation..
There's another word, shalomat, that represents perfection..
These ideas of wholeness, reconciliation, the way things ought to be, things flourishing.
and being just as God created them to, the presence of that is peace..
So interestingly, grace is this idea of passivity, "I receive," but peace in the Greek and in.
the Hebrew is an active word..
It's about a doing, it's about a being, and a way we are..
We act in peace..
So we pick up the phone and we call that person that we haven't spoken to for ages because.
we had an argument we never reconciled, and we get on the phone and we try to reconcile.
it..
That's an act of peace..
Or we send that email that we've been meaning to send because we saw somebody was mistreated.
by this company and we want to point that out and stand in the gap for someone who was.
mistreated..
That is a work, an act of peace..
Peace has action and movement towards it..
And I love the way that Paul begins his letter..
He says, "Here's the whole point of this..
There is grace and there is peace.".
In other words, there is something for you to receive and there is something for you.
to do..
There is something for you to understand in revelation, and as you understand what God.
has done in revelation for you, you then want to live a certain way in the world..
There is grace and there is peace..
Does that make sense to you?.
Whoo, good, you're still there..
So grace and peace is how Paul actually breaks down the whole of the letter of Ephesians..
Let me show this to you..
The first three chapters are all the focus on what Christ has done..

$^{441}$So in there he talks about how all things are under Christ..
We've been adopted into his family..
Sin is being dealt with..
We've been given a new self..
We can know God now..
We have a gift of salvation..
We are his workmanship..
We have this unity together..
You probably heard me talk about that sort of stuff in the opening of this message..
That's the first three chapters of Ephesians..
That stuff that you are to receive, not stuff for you to do..
But then in chapters four to six, he then speaks on this idea of peace..
So because of all the grace on the left side, live a life worthy of this grace..
Build one another up..
Strive for maturity in Christ..
Live as light of Christ in the world..
Imitate Christ in everything..
Build healthy marriages and families..
Protect yourself from temptation..
Live in the power of the Spirit..
He's like, you've got grace, all this stuff Christ has done for you, now in peace, live.
this way..
You should be inspired, he's saying to his church..
Once you understand what Christ has done, how can you not then?.
I said this in my little opening bit..
I said, here's what I plead with you..
Don't waste your life..
There's too many Christians that die in old age with a belief in Jesus who wasted their.
life..
You haven't been saved just so that you can get to heaven..
Ephesians is about being saved so you can live the new humanity in the world today so.
that others might see the light and be drawn to it..
Ephesians is ascending gospel..
Ephesians is about receiving and giving..
And if there's one thing that we have as a leadership here at the Vine for you over the.
next 15 weeks, as we break down every little segment of this book, you're going to be so.
full of Ephesians by the end of it, trust me..
As we go through it, my prayer for you right at the start is grace and peace..
It was designed so that you would hear the whole thing in one go and see the wonder that.
there is in a God who has done all of this stuff and now calls us to live in a radical.

$^{481}$new way..
So as I began, let me finish..
This is from me, a disciple of Christ Jesus by the call of God to you, my family and friends.
who call the Vine their home church in Hong Kong, who are faithful and trying to live.
a faithful life..
Grace and peace to you in the name of the Lord Jesus Christ now and forever..
And everyone says, "Amen.".
Let's pray..
Father, I'm so humbled by the swept beauty of your Word, the grandeur of it..
We thank you that Paul wrote a letter to a church that wasn't supposed to be just dissected.
line by line, but he created a letter that was to be read in the church in one sitting.
so that people could hear the glory of the gospel of Jesus..
Lord, my prayer is that that is something that may have happened for some of us in this.
room here today, that as we hear the whole of the story in one, we would be people called.
to grace and peace..
Father, I know that in this room there are some people in the season ahead of these next.
15 weeks who are going to need the grace..
Maybe some of you in this room, you are new to the faith..
Maybe you came to faith last week at Easter or maybe recently through an Alpha we've done.
or in other ways..
And you're wondering, "What is it that Christ has really done for me?".
My prayer for you over the next 15 weeks is that you would come into this beautiful understanding.
of the grace of God in your life, that you would truly know what He's done..
And if you've been a Christian for many, many years, if you're anything like me, it's easy.
to settle into the habit of being a Christian and lose something of the wonder of what it.
is to be a Christian..
So my prayer is if you've been a Christian for a long time, that as we open up Ephesians.
together you would also be blown away once again by His grace..
His grace to you, not because you deserve it, but because He loves you..
And Lord, I know that there'll be people in this room where peace is the season you're.
moving them into as we do this series together..
That you're going to be calling them to pick up the phone, to write that email, to have.
a coffee, to actually stretch out and try to reconcile a broken relationship, or perhaps.
to stand in the gap where they see an injustice..
There'll be some people in this room, Lord, that peace is going to be the season you're.
moving into, and they have a voice and an authority and an influence, and you want them.
now to use it in a new way..
So Lord, for those in this room where peace, the bringing forth of the flourishing of Christ.
in this world, Lord, if that's their season, I pray you would empower them in the Holy.
Spirit to live that way boldly, Lord..

$^{521}$Whether it's grace or peace, all of it we submit ourselves under the authority of Christ.
Jesus, and we look forward, Lord, to being that apartment building we spoke about, the.
lion dance that we talked about, being in the game, living life that matters..
I pray you would release this over the vine in these weeks, and we thank you for this..
In Jesus' name, everyone says, amen, amen..
Would you stand with me, and we're going to respond together as a church in words..
\newpage



\section{}
\label{sec:5M17mPHN4_U}
\textbf{2024-04-15 Ephesians: The Will of God – Unity in Christ [5M17mPHN4\_U].mp3}
\newline
\newline
連結: \href{https://youtube.com/watch?v=5M17mPHN4_U}{\texttt{ https://youtube.com/watch?v=5M17mPHN4\_U}} ~~~~ 語音日期: 2024-04-15 
\newline
\newline
\hyperref[sec:qcO4mIHWxfY]{\small{< < < PREV SERMON < < <}}
~
\hyperref[sec:index]{\small{[返主目錄]}}
~
\hyperref[sec:C343Yf1MgVA]{\small{> > > NEXT SERMON > > >}}
\newline
\newline
$^{1}$Well, thank you so much, Alison..
And I also want to thank the worship team, Waymaker..
Indeed, he's our waymaker and he continues to work..
That's actually in line with what we're going to share today, that God has a grand.
plan and he's constantly working..
So welcome once again and good morning to all of you and all those also watching us.
over online..
Well, I will share a bit of my own story just to help us understand that God is indeed having.
a great and grand plan for each and every one of us..
So by now you would recognize that I do have some Singapore accent..
Well, I'm actually born and bred here in Hong Kong..
I studied in local universities..
I worked in one of the most busy hospitals here in Hong Kong for 10 years until I really.
got stuck professionally..
I yearned for a breakthrough and we made a very difficult decision, a family of four..
We actually decided to emigrate over to Singapore..
And that was 27 years ago and to be exact, it is 1st of July, 1997..
Whoa, anyway, we did fairly well over there as a family and as well as professionally..
But most significantly was that I actually rededicated my life to Jesus..
And my wife first came to know Jesus as personal Lord and Savior over there..
So over there we enter into a new reality..
Well, of course, it's a different location, but there's a new reality with a new identity..
Now while we settled fairly well over there by God's grace and divine purposes, we actually.
got a chance to relocate back here to Hong Kong..
So 10 years down the road, we moved back and here we are attending divine worshipping and.
I also got to serve in the Cantonese community as well as in the eldership..
So over the years, as I reflect, God has been graceful and He has a great plan for me..
We just celebrate our 60th birthday..
Wow, I look too young..
Yes, Botox works..
Yeah, but I also realized that while the blessing actually extend to over the next generation,.
my daughter and my son-in-law, while they were here in the 9/15 service, so they tied.
their knots just a few months ago while they also planning to move over to London to start.
a new season..
So I'm actually quite amazed that how come this wonderful young Chinese gentleman from.
Australia got to work here in Hong Kong and met my, well, also quite wonderful daughter..
But the thing is, God has a grand plan for each and every one of us..
In due time, He reveals His purposes in us and it's actually for us to give thanks to.
Him and even to praise Him for what He has done in our life in the past, in the present,.
and that would continue into the future..

$^{41}$But that's what Paul wants to share with the congregation in Ephesians, that while.
indeed God is doing great and wonderful things..
So let's read the Bible text for today..
That's in chapter 1..
Praise be to God, Father of our Lord Jesus Christ, who has blessed us in the heavenly.
realms with every single spiritual blessing in Christ..
For He chose us in Him before the creation of the world to be holy and blameless..
In love, He predestined us for adoption to sonship through Jesus Christ in accordance.
to His pleasure and will, to the praise of His glorious grace..
In Him, we have redemption through His blood, the forgiveness of sins in accordance with.
the riches of God's grace that He lavished on us..
With all wisdom and understanding, He made known to us the mystery of His will according.
to His goodwill and pleasure, which He purposed in Christ to be put into full effect when.
the times reach the fulfillment, to bring unity to all things in Christ for all things.
in heaven and on earth to be under Him..
In Him, we were also chosen, predestined according to the plan of Him who works out everything.
in conformity with the purpose of His will in order that we, the first to put our hope.
in Christ, may be for the praise of His glory..
And you also were included in Christ when you heard the gospel of your salvation, when.
you believed, you were marked with a seal, the promised Holy Spirit who is the deposit.
guaranteeing our inheritance until the redemption of those who are God's possession to the praise.
of His glory..
So you can see purposes, goodwill, plans, all come from God..
Well, indeed, it is a lengthy passage that really originates from one sentence in the.
original Greek..
I'll share with you in the next 20 minutes or so three things..
One, how God or what God has done in Christ..
Number two, we try to find the distinction between in Christ versus Christ in us..
And lastly, we talk about Christ being the head and that should bring comfort..
Let's come back to verse one, verse 10, which is the core message..
The whole passage is about spiritual blessings planned by the Father, accomplished by the.
Son and applied by the Holy Spirit..
So indeed, it actually says, "Blessed is the God who bless us with every spiritual blessing.".
Heavenly blessings..
But why a doxology?.
Doxology actually literally means glory..
Okay, why a doxology up front?.
Well, many of us, pre-believers and believers included, wrongly assume that the reality.
is the physical world around which we feel, we touch, or we experience..
Yet, the doxology asserts that the reality is much, much bigger..

$^{81}$It includes God, his actions, and what takes place in Christ and in the heavenly realms..
We must expand our thinking to do justice to this larger reality..
Because in our contemporary practice, we view ourselves as the primary actor on the stage.
of history..
But Paul's doxology reminded that God is the primary actor..
Just the other day, a few of the fellow elders went to visit a church member and pray for.
God's healing..
And that is beyond what the doctors may say and mention about his physical reality and.
challenges..
We must see the bigger reality of God's grand plan for this individual and for each and.
every one of us..
Now paradoxically, I actually worked in the medical line..
And more often than not, I actually would slip into the mode of viewing myself as the.
healer..
I have neglected the bigger reality..
The bigger reality is that I'm only there to manage or to treat certain medical conditions..
Ultimately, only God heals..
And that is the bigger reality..
In Christ, well, let's zoom in a bit more to the prominent theme in this passage, "In.
Christ.".
The expression "In Christ, in the Lord, in Him" appear 164 times in Paul's writing..
Paul's writing really flows from his understanding of how we are to be connected and unified.
with him..
So we can read from this passage, "We enjoy every spiritual blessing in Christ..
We are chosen in Him and adopted through Him..
Everything is accomplished in Him as the revelation of the mystery of His will so that God's purpose.
is to unite all things in Him..
We enjoy an inheritance in Him and also we are sealed with the Spirit in Him.".
Among all the possible other meanings or translations, let us concentrate on the key understanding.
of "In Christ" in this context..
It's more what we call a location, a location sense..
We think of Christ, the location, as the vast repository or treasury holding the gifts of.
God..
And Christ is the source of all these ritual blessings..
And while we reside in Him, we enjoy all these blessings..
And mind you, our individuality is not lost when we are fully in Christ..
It is not some Eastern religious thought of absorption into the deity..
Rather, Christ and we abide in the unity whereby Christ sets the parameters for life and makes.
available God's provision for life..
What are the implications of our being in Christ?.

$^{121}$When we are in Christ, we are united with Him and participate in His death, resurrection,.
and life..
Now, this interplay of death and life is dynamic and I would say is mind-blogging..
Let me illustrate by citing a not too uncommon scenario, chemotherapy..
How it works?.
Now, the toxic chemotherapy would literally kill the cancer cells and part of it, well,.
part of the individual who received the chemotherapy also dies in some extent..
Well, that is how exactly new life begins..
The normal regenerate cell would begin to repopulate and replace all that which was.
lost..
The chemotherapy concept applies to various cancers and many a times the chemotherapy.
is actually designed with what we call an intent to cure..
That means we can cure a person from this disease..
That purposeful death and renewal concept is particularly true for some blood-borne.
cancer or we call it leukemia..
Now, I have this understanding as a personal first-hand experience..
Now, for all of you here, if you do your health check, your blood count, your white cell count.
will be somewhere like 5 to 10..
That is normal..
My cell count as my leukemia progressed went up to 100, 150 and I panicked..
I thought about death..
Well, I bite the bullet and go ahead with the chemotherapy and specifically it is what.
we call a targeted therapy..
The day of the treatment, I was so ill, so sick, I literally thought I was going to die..
In fact, my cell count dropped to the level of below 1 on day 1 post-treatment which was.
dangerously low to the extent that my oncologist has to stop all the scheduled treatment altogether.
so that I will be allowed time to recover for the cell to replenish..
And thereafter, in a sense, I died and I died again as I received multiple causes of the.
same treatment along the way..
That was almost 7 or 8 years ago..
I lost count but I'm happy to share with all of you that I am treatment-free and I'm disease-free..
New life has crept into my life even as part of me died..
Just remember his death is our death and his life is also our life..
It is in this solidarity achieved by a double identification via the incarnation and faith..
What it means?.
In incarnation, Christ identifies with us and by faith, we respond and we identify with.
him..
To be in Christ does not mean to be inside Christ as tools are in a box or clothes are.
in a closet but rather we are organically united to Christ as a limb is in the body.
or a branch is in the tree..

$^{161}$We are all familiar with this..
Christ, I am the vine, you are the branches..
If you remain in me and I in you, you will bear much fruit..
You will bear much fruit..
It is this personal relationship with Christ that is the distinctive hallmark of his authentic.
followers..
As we live in Christ, our awareness of the presence of God and of living in Christ becomes.
the keys to all aspects of our life..
How strange would it be if we are to forget the place that we live?.
We lived in different places here in Hong Kong over the 16 years we are back here and.
my wife Vivian is superbly efficient in arranging for every time that we move house..
I had the experience of actually for a number of times that I left home in the morning,.
went to work, then when I left work I go to a new place, new house altogether..
I just need to be reminded, well, you are to move to a new place..
Well we need to be reminded because for in him we live and move and have our being..
I do want to highlight some important distinction between the notion Christ in us and we in.
Christ..
We often find comfort and indeed we should find the comfort as we embrace that Christ.
is in us..
His presence and his company should be reassuring..
However, when we only emphasise that Christ is in us, there may be a danger that we become.
me-focused..
We try to identify or define reality and Christ would be about one inch tall..
On the other hand, if we realise that we are in Christ or under Christ, he determines reality.
and he encompasses all that we are..
Indeed some of us might struggle with too much of a focus that Christ is in us..
Therefore having that danger of Christ almost submitting to our will, our control, as if.
Christ is only there to meet our own needs..
On the other hand, if we have a distorted understanding of being in Christ, then we.
might think well we need to do nothing more than whatever he has already done..
We take for granted all that he has done for us and we live in a passive way in this world..
Well both dangers are common in the church and perhaps you have experienced one way or.
the other..
I know I have..
Let me share my story to illustrate the distinction..
Christ in me or me in Christ..
I mentioned that I decided to move or relocate back here to Hong Kong..
It was largely due to an offer to take up a leading academic position in one of the.
local universities..
Personally it is a major career advancement and I really had that dream of leaving a legacy.

$^{201}$professionally to be remembered..
We did not take this offer lightly..
We did all the good things that a good Christian is supposed to do..
We prayed, we discerned, we shared, we waited..
It was quite some time before we finally took up the offer..
I was confident that Christ was in me and with me..
I was also committed that I would bring his presence to the workplace..
But my honest reflection and confession was that the focus was still me..
Those few years working back in Hong Kong were most challenging..
Not that I didn't anticipate all those challenges in the first place..
As it turned out, probably partly due to my shortcomings, I did not secure the tenural.
position..
I lost my job..
I had to leave the prestigious position and the title and I have to journey on in the.
private practice setting, starting all over again..
It's painful..
My dream died..
But then I realized Jesus' dream for me is alive and vibrant..
His dream is alive and vibrant..
I come to understand I am in a bigger than myself reality..
I am in Christ..
For in him we live and move and have our being..
First and foremost, I would say I'm really proud that I was once part of a team and that.
they have since ascended to even greater heights professionally and academically..
But for myself, that I'm in private practice allows me a lot of more time and space to.
ponder on, well, how I can serve him better..
And that is not necessarily via my profession nor the title that I bear..
Rather, I am in Christ and therefore he should dictate how best to entrust me in whatever.
arena or circumstances..
So I had the inspiration from my wife, who actually finished her seminary studies in.
her 50s, and I thought, well, I could do likewise..
So I completed my part-time seminary studies while I was in full-time medical practice..
That was despite the health care challenge of having to go through multiple courses of.
chemotherapy..
Studying in the seminary teach me to humble, that there is far more that I do not know.
than the very little that I study and got to know..
And that stirred up my yearning to know him more..
On serving, I had the joy of serving the Cantonese community, which again had a humble beginning..
We started as one of the earliest Cantonese-speaking small groups..
That was 10 years ago..

$^{241}$We started at home, but it soon burst..
God blessed the community tremendously..
We moved back to the premise in this church building, and it continued to grow into a.
Chinese ministry..
Along the way, we also, this church saw the birth of a Mandarin-speaking community..
Now the very little Mandarin that we picked up when we were attending church in Singapore.
became useful and handy because we were able to reach out to others, especially those who.
are Mandarin-speaking..
So, in that sense, God prepared us, or this church, to reach out to others in interesting.
and creative ways..
Of course, I'm also proud and had the privilege to serve with a wonderful bunch of elders.
in this church in the past nine years..
Going through firsthand the tremendous ups and downs for this church, we went through.
all the trials and triumphs together, including during the social movement period and the.
COVID restriction..
But we are thankful, thankful to you because you gave generously and faithfully through.
the difficult times..
And we are so encouraged because you make every single effort to come and worship and.
to gather in this place..
We can testify the great purpose of God, including his seeing us through the tremendous hurdles.
of setting up the Yuen Long Church Plant..
Now I'm not recounting all this as Christ in me..
It's not about me, but rather as an individual in this congregation..
I witnessed the hands of God moving amongst us, all of us..
And this is only a mini account of the many good things God is working in this church.
through the many of you seated here today to serve and love each other in your small.
groups, in your K4C classes, in the language ministries, in the prayer team, in alpha,.
in worship team, in your outreach efforts, in your smile when you meet someone in this.
congregation and in your hard work up front or behind the scenes..
There are people who begin to ask me how I would like to serve as I step down from the.
eldership board later this year..
Good question..
Lots of wrestling..
Well, but I believe the best is yet to be..
Well, no solid answer yet..
However, I will say on behalf of my fellow elders that they are committed to serve all.
of you diligently in days and years to come..
But actually each and every one of us should be asking these questions..
How we can best serve our God and how God wants His best for this church and how He.
wants to entrust each and every one of us here..

$^{281}$I do have a glimpse of the possible future for this church, that each congregant here.
is called into a community intimately rooted in His Word and in Him and ever ready to reach.
out to others..
Well, after all, God's plan for us remains a mystery but will be revealed in the fulfillment.
of times..
Indeed, verse 10 says, "It would be put into effect when the times reached their fulfillment.
to bring unity to all things in heaven and on earth under Christ.".
Note that Paul actually expands this use of language to a cosmic scale..
Everything, creation and redemption of the cosmos is to be in Christ, under Christ..
The word translated as "bring unity" is interesting, "grit rith anna kafalio.".
Kafali means the head..
Okay, so we have medical conditions like micro-cavillous, very small brain, with hydro-cavillous,.
water-filled up brain..
But it's a different head we are talking about..
It's the majestic, royal head, Christ in the heavenlies..
So it is significant because it emphasizes on three things..
One, Christ is the head, the ruler..
Christ sums up or brings things to a coherent and meaningful whole..
And number three, Christ restores harmony to a universe that was chaotic because of.
sin..
So what would be the implication?.
Now we have fellowship not only with God but also with one another..
Horizontally, Christ connects God and the church here..
Horizontally, Christ connects all of us, even those who come from diverse social economic.
backgrounds..
Christ is ruling over us..
His fulfillment has begun through his life, death, and resurrection..
He's already Lord of all times, just as redemption is both now, present, and future..
And yet the future has moved into the present and changes how we should live..
I do understand sometimes we doubt because we do not see obvious actions from him..
And then we wrongly assume that God is not active..
Well God is not necessarily public in all his actions..
But surely God does reveal himself in his own time..
He's active in making his purposes in Christ known..
So we almost need a sign that reads, "Slow down, God is at work.".
"Slow down, God is at work.".
This should be comforting because Christ is supreme over everything in this world..
This could speak to our fears for the future or worries that we hold for the present..
It should help us freed from the oppression or the opposition that we are facing..
We have Jesus who is head over everything..

$^{321}$Therefore we have peace in our present..
I really love the school motto of my son when he studied in Anglo-Chinese school in Singapore..
The motto is, "The best is yet to be.".
The best is yet to be..
As my son-in-law and my daughter about to plan to move to the UK in just two weeks'.
time, well my encouragement for them is, "The best is yet to be.".
But they are to discover for themselves the bigger reality of God's grand plan in their.
life..
Likewise, young men, young women, and indeed everyone in this congregation, we are assured.
and promised that the best is yet to be..
However, for this church, planting a church in Yuen Long or growing in numbers here in.
Wan Chai are not by themselves the ultimate fulfillment of God's purposes for the vine..
There is a bigger reality, but His grand plan for this church will continue to unfold..
So I paraphrase the whole prayer in this biblical text by Paul..
I would like to invite all of us to stand so that we may read this prayer together in.
a paraphrased manner..
So we shall see on the screen the prayer that I wrote, "How Marvelous God is.".
And let us read together..
His Spirit has provided everything needed for life and for the vine..
For every good thing has been made available in Christ..
We praise such a God..
Right from the first, God has been busy devising a way to draw us home to Himself so that we.
may live with Him and for Him..
Through Jesus Christ, He has made us family..
As a result, the vine prays God for the way He freely gave Himself to us in Christ..
In Christ's death, God's abundant care for us is known..
God gave Himself for us to bring us back and make us His people..
What lavish love He has for us..
We honor you, God..
In His unfathomable wisdom, God has made known His plan and desire to bring all things together.
in Christ..
This includes everything in our world and everything in God's world..
Amazingly, God's plan includes the vine and gives us a share of what He is doing..
For this, the vine prays God for the hope that is ours in Christ..
When we heard about the truth from God and believe the good news about His plan, God.
marked us as His own by giving us His Spirit..
The Spirit dwelling in us is a pledge from God that He will complete His plan and that.
one day we will truly live with God..
For this, we, the vine, praise you, our God, we do worship you..
May I invite the band?.

$^{361}$And I want to challenge all of us while I give you this encouragement..
And in a short moment's time, you're most encouraged to join us in a prayer line as.
you commit yourself..
Number one, we all have a dual identity on earth and in heaven, which is the bigger reality..
You might be from local or you may be from a foreign land..
Dare to believe that you are here in Hong Kong in this season as part of God's grand.
plan for Hong Kong and you matter to Him..
You might be born and bred from the ministry of this church or you have moved in from other.
congregation to try to settle here in the vine..
Dare to believe that you are part of God's grand plan in this church..
You matter to us..
Last, allow His comforting presence to rest in you always as indeed He is in you..
But also be mindful that He encompasses all that we are..
We are in Christ and under Christ..
Let Him take lead in your life and that His dream becomes your dream..
Be truly blessed by our blessed, worthy God..
Amen..
(soft music).
(gentle music).
\newpage



\section{}
\label{sec:PkcSALXT_yw}
\textbf{2024-04-22 Ephesians: Incomparably Great Power [PkcSALXT\_yw].mp3}
\newline
\newline
連結: \href{https://youtube.com/watch?v=PkcSALXT_yw}{\texttt{ https://youtube.com/watch?v=PkcSALXT\_yw}} ~~~~ 語音日期: 2024-04-22 
\newline
\newline
\hyperref[sec:C343Yf1MgVA]{\small{< < < PREV SERMON < < <}}
~
\hyperref[sec:index]{\small{[返主目錄]}}
~
\hyperref[sec:_afRGwFbmt4]{\small{> > > NEXT SERMON > > >}}
\newline
\newline
$^{1}$Everyone says amen. Amen. Amen. Can we thank our worship team? So, so beautiful..
Thanks guys so much. Have a seat. Have a seat. It's so good to have you here. My.
name is Andrew. I'm one of the pastors here at the Vine. If you're relatively.
new to us, new to the church, we're really glad that you're with us and you.
know everything we've just been sharing about and singing about is for you. This.
is a place of acceptance, a place of love, a place where you can come and just be.
who you are. Be who you are in this season of life that you're in and my.
prayer today is that you would know and encounter God who accepts and loves you.
and fills you. You know, I'm really grateful that I have an amazing woman in.
my life, my wife, who loves me dearly. My wife and I started dating, by the way,.
back in 1996. I know, for like most of the people in this part of the room right.
now, that was before you were born, okay? 1996. I know, that ages me tremendously..
But my wife and I, we met at Mother's Choice, an organization here in Hong Kong,.
and we started to date and as we started to date, my wife began to complain about.
a pain that she had in her side and she said the pain would come and go, but there.
was this kind of like niggly sort of pain in her almost all the time. I tried.
my best not to assume that that pain was because I was now in her life and we.
were dating together, but she complained about this pain and so one night we.
were praying together and as we were praying, I saw a picture, a vision..
It was almost like this prophetic kind of vision and I saw a dagger stuck in my.
girlfriend's, at the time, my girlfriend's side and the blade was deeply inside of.
her, but the handle of the dagger was on the outside and I could see that every.
once in a while, because it was always there, the enemy would like to come along.
and just grab a hold of it and twist it a little bit and cause that pain into.
her. So as I'm seeing this in prayer, I share it with her, "Hey honey, I see that.
there's this dagger inside of you, the handle's on the outside, I can see it's.
there, that the enemy's manipulating it, it's causing you great pain." And she kind.
of looks up at me and she goes, "Well pull it out then." I'm like, "What do you mean?".
She's like, "Well pull it out! If you see it, you should just pull it out, I don't.
want it in me." And I'm like, "This is weird, okay." But I'm trying to like date her,.
right? So I'm like, "Okay, sure honey, I can be the man, I will pull out the dagger.
from your side, this is very weird." So I leant forward and I put my hand on the.
kind of, the dagger as I could kind of see where it was, I put my hand on there.
and I pulled it out and she never felt a pain in her side ever again. A couple of.
years later, a couple of years later, I'm on a missions trip in the Philippines..
I'm a youth leader here at the Vine and we're leading a group of youth to the.
Philippines and we're working in the slums in Manila. And there's a man and.
his wife that we're ministering to, the man had cataracts on his eyes. We'd been.
told by the people that we were serving with that he'd been blind for over 20.
years, full cataracts, couldn't see out of his eyes at all. And as we're talking.

$^{41}$with him and just sharing the gospel with him, one of the 14-year-old youth on.
this trip kind of tugs my shirt and I look down and she's like, "I believe Jesus.
wants to heal him right now." Now I don't know about you, but when you're the youth.
pastor and you're leading all these youth to a moment like this and somebody.
goes, "I think Jesus is gonna heal him and do this miracle." You're kind of like, "Hmm,.
really? I hope he does, but I got to prepare myself for when that doesn't.
happen and then I have all these disappointed youth and Jesus, right?" So I.
unfortunately was not feeling much faith in that moment, so my response there was.
like, "Oh, cool. Okay, well then you pray for him." I'm a terrible Christian, by the.
way. So I'm like, "Why don't you pray for him?" So she comes forward and she.
stretches out her hands and she puts her hands on his eyes and she prays like the.
simplest prayer you could ever imagine. Now I'm watching this with my eyes open.
and I see something that I had never seen before. The best way to describe it.
to you is like windshield wipers, just you know like if you have a really dirty.
windscreen in your car and the windshield wipers kind of wipe it away..
Immediately as she starts to pray, his cataracts are wiped away from his eyes..
He starts screaming his head off because for 20 years he's not had light coming.
into his eyes. He starts screaming, his wife starts crying, all the youth start.
screaming, I start screaming because everybody's screaming. We're all just.
there in this amazing moment because God in the power of prayer has set this man.
free. And suddenly he could see. A few years later I'm here at the Vine on a.
Sunday. It's at our four o'clock service and we're all worshiping away and the.
power of the Spirit is in the room and suddenly this guy starts shouting and.
screaming and thrashing around. He falls on the floor, he's thrashing around. It's.
super awkward. Everybody's like what is going on? Myself and a few of the elders,.
we went over to see him and we sort of knelt down and we realized that he's.
having this demonic kind of manifestation. This demon is screaming.
and he's swearing and it's all pretty gross and ugly and horrible. And one of.
the elders just reaches out his hand and just gently in prayer says, "In the name.
of Jesus be still." And this man immediately went still. In fact it was.
like what the Bible describes. He was suddenly dressed in his right mind. A few.
years later I'm traveling on a bus here in Hong Kong. Just traveling on a bus,.
minding my own business. Got some music going on in my ears, trying to be.
anonymous. Some guy starts freaking out on the bus, starts screaming.
and shouting and immediately I know it's the same kind of thing. He's probably.
going through some demonic manifestation right now and I'm thinking I don't want.
to be a Christian in this moment. So I'm like staring out of the window of.
the bus, kind of trying to pretend like I'm not a pastor, I'm not a Christian..
This is my day off, okay? I'm on a day off from being a Christian right now, okay?.
And as I'm staring out the window this guy's freaking out. Everybody's kind of.

$^{81}$freaking out on the bus and the Holy Spirit says to me, "Pray for him right now.".
And so out loud in front of everybody I just say, "In the name of Jesus be still.".
And immediately he falls asleep. Immediately he falls asleep right there.
and then everybody starts staring at me like, "What is this guy?" Immediately he.
falls asleep. A few years later I'm sitting with this mother as she holds.
her dead son in her arms and you can imagine the pain and the trauma of that.
moment. And I'm sitting there like I have nothing to say to this. I have no words.
of comfort. I don't even know how to approach this pastoral moment. I'm.
sitting with her on the bed as she's holding her dead son in her arms and I'm.
like I have nothing to say. And she, in the middle of her morning and trauma, she.
makes a prayer. And in her prayer she asks God to come and I have never seen.
anything like this. She's like crying and weeping as you could imagine that she.
should in her morning. But as she prays a peace that comes upon her, as Scripture.
says, a peace that surpasses understanding comes over her. And.
although she's still in a place of deep pain and deep mourning, I can see in her.
eyes that she's filled with the peace of Christ. And why am I sharing all these.
incredible stories with you? I think it's because right now in the time that we're.
in, in the season that we're in, in the world right now, I don't know if you saw.
this week but there was newspaper reports that were talking about the.
possibility of World War III. I think in the time where there's a rising sense of.
panic and fear across the world for what is happening in our world, now more than.
ever, I think the global church needs to rise up and regain a passion for power.
that is only felt and seen through prayer. See, I think there's a work of.
God's Spirit that he wants to do through the power of prayer like he has never.
done before. You see, in a time where the world is doubling down on the grasping.
of power through violent wars, through the kind of overestimation of the.
false freedom of individualism, through the inequitable distribution of.
resources around the world, all these ways that we see a narrative in the.
world to grasp a hold of power, now more than ever, the church needs to remember.
the subversive and revolutionary power that there is in the humble posture of.
bending a knee. You see, I think the narrative in the world right now about.
power is those who are powerful are the ones who have the greatest army, the most.
amount of resources, and the ones who have the loudest voice. And we need to.
remember that that is not how Scripture defines power. We need to understand that.
that is not the way in which the kingdom of God operates in power. And that.
actually, I'm a hundred percent convinced that the only true power in this world.
is found in God by the Spirit through Jesus Christ. And if that is true, if that.
is true, then a church's ability to impact and change the city that it's.
planted in is not found in how large we are, how influential we are, it's not.
found in how relevant we might be, it's not found even in how great the worship.

$^{121}$might be or how clear the teaching might be. The only thing that can move the.
needle in this city is the unbridled power of the Spirit of God that is.
released in prayer. Show me a church that is passionate for prayer and I will show.
you a church that is humble enough to understand that only God can change the.
world, not us. See, here's the thing that I think we really need to understand. When.
a church lacks a culture of dependency on prayer, it is usually because that.
church has become satisfied being dependent on itself. I'm gonna say that.
again, that's super important. I think when a church lacks a culture of.
dependency on prayer, it is because, quite likely, that that church has become.
satisfied being dependent on itself, being dependent on its own resources. And.
if I was to say this, I want to put it this way, I think the number one danger.
that faces the vine in the season that we're in, in the world that we're in.
right now, is that we would, because of this incredible building we have, because.
of the money that's in our bank account, because of the crazy talents that we see.
on our staff team and our volunteers, because the services we have and the.
ministries that we have, that all of those things which are a gift of the.
working of the Spirit of God, that all of those things would so amplify in us a.
sense of self-reliance that it would dampen our desperation for prayer. Is.
anybody here? Are you okay? That we would so see all this great stuff that God has.
given us, but that would seduce us into a place of self-reliance that we would.
actually dampen the enthusiasm and the desperation that we have for prayer,.
because the kind of power that the world needs to see today is not more warring.
armies, it's not more resources in the bank account, surely it is this unbridled.
passion and power of the Spirit of God, and that my friends is released always.
in prayer. What might it look like that the church would truly believe that.
power is not found in the hands of the outwardly powerful, but in the scarred.
hands of the inwardly crucified? Now when Paul writes to the church in Ephesus, all.
of that is on his heart for those people in that church. He sees what they've been.
saved out of. He understands that Ephesus was a place of great power. In fact,.
Ephesus in those days was one of the most powerful cities that there was in.
the Greek or Roman Empire. It was a port city, very much like Hong Kong, and because.
it was a port city, it was a center of trade and commerce. People, particularly.
entrepreneurial people, were moving in the Greek or Roman Empire to Ephesus to.
make a life for themselves, to make a name for themselves, to gain that wealth.
and that status, and to have generational wealth for the generations of their.
family to come. This was what Ephesus was known for. And Paul has shared the.
gospel in that city, and he's helped people to understand that power in the.
kingdom is actually seen in a different way, that it's not about what resources.
you have, what's in your bank account, all these other ways that power is seen in.
the world, that it's seen through the life, death, and the resurrection of Jesus..

$^{161}$But Paul knows that his church is susceptible to falling back under the.
seduction of the power of its city. He knows that the church is caught at times.
with its feet in two camps, in the camp of being seduced by that power seen in.
the world, and the camp of this newfound relationship that they have in Jesus, and.
they're struggling between the two. And so he writes this letter to help the.
church understand where true power really sits. And Paul understands that.
that's in prayer. And so interestingly, in the book of Ephesians, more than any of.
his letters in the New Testament, he has a number of prayers in the book. One in.
the first chapter, one in the third chapter, we're looking at the one in the.
first chapter here today. Now, why is prayer and power so important for Paul.
in Ephesus? It's because not only was there this great commerce and this great.
economic power that was seen in the city of Ephesus, there was also a powerful.
spiritual entity that was at work. See, Ephesus was famous for the worship of.
Artemis, the goddess Artemis, in Greek, the goddess Diana. And Diana and Artemis, the.
same person, same God, this goddess was worshipped in Ephesus. It had one of the.
largest, most glorious temples to Artemis that you can see. In fact, you can go to.
Ephesus today and still see the ruins of this temple. It truly is an incredible.
thing to see. But in those days, that temple was massive. And that temple was.
filled with people worshipping this Artemis for a number of very significant.
reasons. Artemis was believed to be the goddess of the underworld. So she was the.
goddess, actually, of evil forces and dark forces and curses. She was also seen as.
the goddess of fertility, which is why when you see statues of her, if you jump.
back to that statue just a moment ago, you'll see that she's covered in all of.
these things. These are actually breasts that she's covered in. She was the goddess.
of fertility, so she's covered in these breasts. That was how she was actually.
depicted to the people. And if you wanted fertility in that day, here's how you.
would do it. You would go to the temple and you would hire a temple prostitute..
And you would have sex with the temple prostitute, and that was your way of.
worshipping Artemis in order for her to have favor to bless you with fertility..
The other thing that Artemis was known for was the giver, the cosmic savior and.
giver of wisdom. If you wanted wisdom to solve a problem in your life, if you.
wanted to get some revelation to help sort out something in your life, you.
would go to the temple and you would pray to Artemis. The other thing that.
you would do, or the other thing that took place in Artemis worship, was.
also this idea of the opening of the mind. And actually, ancient sources have.
shown that there was the use of drugs in the temple to help people to have.
ecstatic experiences, so their minds were open to see things in different ways..
This was all happening in the temple in Ephesus. And then, if you wanted Artemis.
or one of the gods that were under her control to do something for you, here's.
what you would also do. You would go to the priest of the priestess in the.

$^{201}$temple and you would say, "Hey, I want this person..." So, give me an example out of my.
own life. I have this amazing girl that I meet. She's blonde, from New Zealand. She.
became my wife, okay? So, hang with me. But she's blonde, she's from New Zealand. I.
fall in love with her, but she hasn't fallen in love with me yet. This is a true.
story, by the way. She hadn't fallen in love with me. So, if I was in the first.
century Ephesus, here's what I would do. I would go to the temple of Artemis and.
I would say to the priest and the priestess, "Can you give me a prayer to.
pray to a specific God that would enable this woman to fall in love with me?" And.
the priest and priestess would say, "Well, tell us a little bit more about this.
person." And I would describe, "Well, she's from New Zealand. She's blonde. She's.
beautiful. Her last name's Ho. I think she's awesome." And I would go through all.
the things about her that I liked about her. And then, the priestess would go, "I've.
got just the prayer for you." And they would go to a chest and they would pull.
out a scroll, a magic scroll. And they would basically say, "This magic scroll.
contains the name of the exact God that you need to pray to in order for that.
woman to fall in love with you." And they kind of give you the scroll and you're.
like, "Give me the scroll." And then they take the scroll away and they say, "You.
have to buy the scroll. The scroll is gonna cost X, Y, Z amount of money." And.
you're like, "Of course I will pay that because I need to know the name of the.
specific God that can help me to have this woman fall in love with me. You following.
this?" So, you buy it and you take the scroll. You open it up at home. It has the.
name of that God. You pray to that God, believing that this woman is then gonna.
fall in love. This was the way that worship was done in the temple at this.
time. And so, here's Paul writing to the church, to a bunch of people who've been.
saved out of that context, whose feet are still kind of in both camps. They're.
attracted still to the power that is seen in the world with finance and.
commerce and making a life for themselves. Equally, they're attracted to.
the temple and to Artemis and to knowing the names of all these different gods..
And at the same time, they're trying to also now live on behalf of Christ and.
try to navigate all of that. And Paul writes them and says, "You need to.
understand what true power comes from. You need to understand that it doesn't.
come from this temple. It doesn't come from this goddess Artemis. You need to.
understand that the only power you need, the only answer that you'll ever need to.
every prayer that you will ever pray is Jesus Christ." And he helps the church to.
re-center themselves in the power that comes through prayer. And if there's one.
thing I want to do with you tonight, no matter how you felt coming in here today,.
no matter what is going on in your life, I want to pray that what we do together.
today will help to revolutionize your prayer life. Because I think when we.
truly understand what it is that Christ has done for us, we truly will then.
begin to lean into the praying that I think Paul longs for his church to pray..

$^{241}$Let me read this to you, this incredible prayer that he has in Ephesians chapter.
1. Everybody okay still? All right, here we go. Starting in verse 15. Listen to this.
incredible prayer. This is Paul. He says, "For this reason, ever since I heard about.
your faith in the Lord Jesus and your love for all the saints, I have not.
stopped giving thanks for you, remembering you in my prayers." And here's.
what I pray. "I keep asking that the God of our Lord Jesus Christ, the glorious.
Father, may give you the spirit of wisdom and revelation so that you may know Him.
better." Notice that not Artemis, that the Father, the glorious Father, would give.
you a spirit of wisdom and revelation so you would know Him better. "I pray also.
that the eyes of your heart would be opened." Not your mind through some drugs,.
but the eyes of your heart would be opened so that you would be enlightened.
to know the hope that there is to which He has called you, the riches of His.
glorious inheritance in the saints, and you would know this, His incomparably.
great power for us who believe. Oh, that power is like the working of His mighty.
strength which He exerted in Christ when He raised Him from the dead and seated.
Him at the right hand in the heavenly realms. Oh, far above all other powers, all.
other authorities, all other dominion, over every name that could ever be named,.
not only in the present age, but in the one to come. And God placed everything,.
all things, under His feet and appointed Him to be the head over everything for.
the church, which is His body, the fullness of Him who fills everything in.
every way. Paul is saying, "Church, don't you understand the power that is before.
you? Don't you understand who Christ is and what He's done for you? My prayer is.
that you would have wisdom and revelation so that you would know more.
of what it is that He's done with you. I pray that the eyes of your heart would.
be opened up, not just your mind so you can think better, your heart so that you.
would have more compassion, more grace, more forgiveness, so you would know the.
hope by which it is that you've been called." He says, "Not only that, but I want.
you to see the incomparably great power that there is for those who believe. That.
power is greater than anything Artemis could ever offer you. That power is.
greater than anything economics or commerce could ever offer you. That power,.
everything else sits under that. Every name has to submit to that name. Guess.
what? You don't need to buy magic scrolls anymore. You can get rid of all of that..
If you've got the name of Jesus, everything else submits under that power.".
And He's basically saying, "That is all you need to know. Jesus." And when you come.
to a relationship with Him, all power is available to you. I love how He starts.
the prayer. He says, "For this reason." In other words, Paul is saying, "Okay, for.
everything I've already said so far in the letter, the first 14 verses, for the.
reason of all of that, here's now how I pray for you." I love this. I'm very.
grateful for Sydney last week for taking us so beautifully through those 14.
verses and showing us what it means for us to live in Christ and to have Christ.

$^{281}$living in us. And we saw last week some of the things that Paul is.
presenting to the church in Ephesus about who they are in Christ because he.
wants them to understand their identity in Him. Let me just show you a bit of a.
summary of these things. This is what he has said in the first 14 verses of the.
letter. "You are blessed with every spiritual blessing. You've been chosen by.
Him, adopted to His family, freely given grace. You're redeemed by His blood. You've.
got wisdom and understanding. You've been made known God's will. You have hope in.
Christ. You are the praise of His glory. You've received the word of truth. You've.
been given salvation. You are marked by the Holy Spirit. You have an inheritance..
And because of all of this, for this reason, here's how you pray." Which is so.
powerful. See, for Paul, his prayers are not coming out of thin air. For Paul, he's.
not praying some rote prayers that he's memorized to learn. He's like, "Because of.
all that Christ has done for me, because of all that Christ has done for you, you.
haven't earned it, you don't deserve it, He's just done it out of His grace..
Because of all of that, this is how we pray. We pray from a posture and from a.
place of the power for which God has done stuff for us. The power for which.
Christ has changed our lives. That's the fuel for our prayers," he's saying. Which is.
deeply challenging, because I realize that's not how I pray. And I wonder.
whether that's really how you pray. Because if I'm honest with you tonight,.
here's how I pray. I don't pray from a Christ-centered perspective. I pray from.
an Andrew-centered perspective. "Oh, my ankle hurts a little bit. I'm gonna pray..
I've had an argument with my wife. I better pray. I've had a really stressful.
week at work. I'm gonna pray." In other words, my approach to prayer is so often.
based on what is happening for me, and is based on my perception of reality. Now,.
don't hear me wrong. If you're in a season right now in your life where.
things are really challenging, really difficult, things are really hard, and.
you're going to Christ out of that for prayer, that's absolutely awesome. That's.
totally fine. Totally great. Please don't hear me wrong. What Paul is saying, though,.
is that the true power that fuels our prayers shouldn't be what is going on.
for us. It should be what Christ has done for us. And when we center our prayer.
life in the perspective of what Christ has done for us, for this reason, our.
prayers take on a different focus. Our prayers take on a different power to.
them. I mean, could you imagine what that would be like? You see, when the self is.
the starting point of prayer, we very quickly become people who are.
overwhelmed by our need and our circumstance. But when Christ is the.
starting point of prayer, we are very quickly comforted by His authority and.
His power. And from that place, we are then able to pray prayers that.
truly mean something. My challenge for you in the week ahead is to do this, to.
find some time in your diary, to take out the book of Ephesians from Scripture,.
read the first 14 verses, and then afterwards begin to pray. I think you'll.

$^{321}$be amazed at how different your prayers are off the back of all of the things.
that Christ has done for you. It will shape and transform your prayers in a.
very different way. Paul's like, "For this reason, then I pray." And my hope for you.
this week ahead is that you will find your prayers coming out of the overflow.
of what Christ has done for you, so that you're finding yourself more bold, more.
courageous, more direct, and more hopeful in the things that you're bringing to Him..
And you may still have challenging circumstances and situations going on in.
your life, and it's absolutely no problem to pray for those things. You should pray.
for those things. But they're flowing not out of a power that you're trying to.
find in yourself, they're flowing out of the power of the cross, the death, and the.
resurrection of Jesus. Are you with me? Then he starts to pray, and he does this.
incredible prayer. He first starts off by thanking them. He's like, "I thank you, and I.
thank God for who you are. You're awesome." Then he says, "I hope that you will have.
wisdom and revelation, that God will give you what you've been trying to seek from.
Artemis." The ultimate wisdom and revelation, he's saying, comes from Him,.
that you might be able to know Him better. Not that you would.
get answers to all the problems in your life, but that you would know Him better..
He then says, "I pray that the eyes of your heart would be opened, so that.
you know more compassion, and forgiveness, and love, and ultimately know the hope by.
which you have been called, the great inheritance of all the saints." And then.
he gets to the heart of his prayer. In fact, from this part forward, from verse.
19 onwards, the rest of the prayer is focused on one theme and one theme only,.
and that's power. He says, "Here's what I pray for you, too. I pray that you would.
come to know the incomparably great power that there is for those who.
believe." And he wants his church to be so rooted in a right understanding of.
power, that they're able to let go of all of that seduction that they see in the.
world, and put their full trust and faith in Christ and Christ alone. Let me read.
you, actually, verse 19 in its full. It says this, it says, "And His incomparably.
great power for us who believe." That power is like the working of His mighty.
strength. Now, in the English, when we read that, it seems like pretty.
straightforward, but actually in the Greek, this is very.
interesting. In fact, what Paul does in the Greek here, which is what he's.
writing in, is something that he hasn't done and doesn't do in any of the other.
places of his letters. He actually uses six different words for the word "power.".
Look at this. You don't need to know what the words are, but these six words are.
all words which we translate into our English as "power." Which is really funny,.
because if you were to read this from the Greek into a transliterated English,.
here's how it would read. It would read this, "God wants you to know the power of.
the power of the power, for those who believe according to the power of the.
power's power." In other words, no power! Are you with me? Can you feel like.

$^{361}$Paul's passion? He's like, "I really want you to understand that the power of the.
power of the power, even the power's power has got power, right?" Like, he's really.
trying to help his church to understand the fullness of what Christ has done for.
them. So he uses six Greek words that we translate to "power," but mean slightly.
different things. He starts with the word "hupoballon." "Hupoballon" means this.
idea of a power that is beyond anything else, a power that cannot be undone or.
outdone by anything else. In other words, this power is incomparable, which is the.
English word we use, than any other power. "Hupoballon." He then uses this other.
word "maitheus." "Maitheus" is this idea that the power is of the greatest size.
that you could ever imagine, that the scope of that power is bigger than any.
other scope. He then uses the word "dunamis." "Dunamis" means a power that is.
in the possessive form, the idea of a power that is kind of the capability for.
action, the capability of this thing for its power. The picture is like a.
crossbow that an archer might pull back, and when they've got the crossbow pulled.
back, just before they're about to let go, their fingers and the arrow shoots.
forward, the tension in the bow, that's "dunamis." It's the potential power that is.
there. He then uses this other word "henephia." "Henephia" means actual power,.
not just a potential of the power, but the actual power that's at work in the.
world. He then uses another word, it's this kind of interesting word "kratos.".
"Kratos" essentially means that there is a mastery over every dominion and.
overplace. So this power is not just potential, it's not just actual, it's.
also a power that has dominion over every other kind of power that you could.
ever imagine. Every underpower is gonna sit under this. This is what "kratos" means..
And then he finishes using this word that I think sounds like a Korean boy.
band name, "hysun." "Hysun" sounds Korean, doesn't it? "Hysun" is a Greek word.
which traditionally means like a possessive power or a power that's unique to the.
individual. So I want you to see this, these six words, and they capture a.
slightly different emphasis of what power is about. And when Paul writes.
Ephesians 119, he's wanting his church to understand, they would have known the.
Greek, they would understand what it is. So let me now translate this from what I.
think Paul is trying to say back into English for us. Let me read this to you..
"I pray also that you would know the hope there is in the knowledge that there is.
no power in all the world that can go beyond the power found in God. No power.
that has the scope to outdo the power of God, and that this power sits in its full.
potential in the heart of God in every moment and is ready to be released in.
this world with a moment's notice. This is an actual power already at work all.
around us, not a power in theory but one in reality. And this power has dominion.
and mastery over all of the pretensions to power, for it is the personal power of.
the one true God, and he has uniquely revealed this power to us in the life,.
death, and resurrection of Jesus." That's what he's saying. That should blow your.

$^{401}$mind, that he would say in this prayer, "This is what I want you to believe about.
what Christ has done. This is the power that we have access to as Christians, a.
God who is able to do all of this." And so maybe you came in earlier and you were.
feeling defeated and overwhelmed. You have a God who is all of this for you,.
right here, right now. The fullness of that power right here and right now. This.
is why he actually says in verse 21, I love how he puts it here, he's getting.
like really sort of fired up at this point, he says, "Far above all the rulers.
and authorities, powers and dominions, over every name that can be named, not.
only in the present age but in the one to come." He's saying, "You don't need magic.
scrolls anymore, Ephesus. Stop spending your money on these stupid scrolls that.
are supposed to be giving you the name of some God that you think you need to.
pray to make that girl fall in love with you. If you want the girl to fall in love.
with you, you only need one name." Well, maybe that's not true. I don't know. But.
he said, "There's only one name for you. It's the name of Jesus, and if you get.
the name of Jesus, every other name is under that name. You don't need to do.
anything else. That becomes the one thing that you have. You can let go of all of.
the other pretensions to power, and you can declaim that Jesus and Jesus alone.
is Lord." This is a message for us in Hong Kong. I can't tell you the number of.
times where I've led somebody to the Lord here in our city, and I know that.
what's happened is they've taken Jesus, and it's the new deity that's on their.
shelf, next to a bunch of other deities that they've still got on their shelf..
Jesus has become the latest guru to help with their problems in life, but they're.
still holding on to a lot of other superstitions, a lot of other cultural.
deities, maybe a lot of deities that their family has worshipped in the past,.
and they're still there too. And it's like Jesus has become added to a whole.
pantheon of deities for that person or for their family. And Paul leans into.
that context, and he says, "You just need one. In fact, there is only one, and it's.
Jesus Christ. And when you have Him, everything else has to be removed from.
the shelf. And if you're trying to hedge your bets across a whole bunch of.
different gods, that's not going to work for you. There's only one true power in.
this world, the life, death, resurrection of Jesus. And in relationship to Him,.
every other pretension to power submits itself unto Jesus Christ." This is why he.
says, and I think this is really, really important, in verse 19 he says this,.
"And this incomparably great power for us who believe." In other words, if you.
want to know, "Well, how do I get this power? How do I align myself to this power?" I.
hear you, Andrew, talking about all this great power. I kind of want to be a part.
of that too. I want to know a God who's going to be over me and going to move in.
that path. How do I get that power in my life? Paul says it simply. You want to get.
that power? Believe in Jesus. It's available for all who believe. Which,.
funnily enough for us Christians, sometimes grates against us, because.

$^{441}$we're kind of like, "How do I get that power? What do I need to do in order to.
get that power? Because I want that power. Are there more Bible studies that I can.
join to get that power? Or maybe it's power only reserved for the senior.
pastor. So after the service, I'm going to ask him to pray for me because his.
prayers are way more powerful than anybody else's, of course. Or maybe that.
power comes if I attend church every week for the next six months. If I do.
that, do I get that power?" And Paul says, "If you're still thinking that way, you're.
still purchasing magic scrolls." He's saying, "Actually, all you need to do is.
believe in Jesus. And if you believe in Jesus, you have access." The fullness of.
that. The only qualification that Paul puts on the access to the power of God.
is belief in Jesus. That's the only thing. Which I think is such a fundamentally.
important thing for us to remember. Because Christians, we can be so.
judgmental of others at times. "Oh, look at that person over there. Have you seen.
the way they dress? They're not really filled with the power of God. Because if.
they were, they would certainly not dress like that. Are you with me, too?" We are so.
judgmental in the church. Paul writes this to his church to say, "Let's stop.
judging one another. The only qualification for the power of God to be.
at work in your life, for you to be in Christ and Christ in you, is belief in.
Him." That's it. See, Paul wants you to know and accept that the fullness of the.
power of God is in those who believe in Him. And that the fullness of that power.
is potentially working through them as well. That's a wonderful thing. That we.
would all know a God who can fully meet us in all of that power that Paul has.
been talking about. So it comes back to our prayer lives. And it comes back to.
those moments where we wonder, "Is God really going to come through?" And I tell.
you what, in this time and in this hour, where we see all the stuff that's.
happening in the world around us, it's very easy for the church to feel.
defeated. I don't think Jesus died on the cross for a defeated church. I think He.
died on the cross so that we would know the exact thing that Paul's been talking.
about in this passage. He died on the cross so His church would not be.
defeated but victorious. He died on the cross so His church would know that when.
they pray, they have access to the fullness of the power of God. That.
power means that the name of Jesus, every other name, falls under that. It means.
that there is no dominion or mastery that is greater than that. It means that.
no matter what the world might tell us about the narratives of power, none of.
that makes sense because Jesus is the only one in true power. And when we put.
our allegiance and our obedience and our faith to Him, we know that we are.
walking in the only power that can sustain things for eternal life. That's.
our gospel. That's our story. That's the church God needs in the world today..
See, a church that lacks a culture of the dependency on prayer has become a church.
that's ultimately satisfied in being dependent on itself. And that's not the.

$^{481}$church we need right now. My prayer for you is that you, as you see a Christ-.
centered approach to prayer, you would transform your prayers to be prayers of.
boldness and courage and hope and that you would pray with new unbridled power.
that comes from the Spirit of God through the activity of prayer. Can I.
pray for you? Let's pray. Why don't we stand together? I'm gonna pray for us..
Maybe just posture your heart into a place of prayer now. Just maybe close.
your eyes. Just come before Him. You've been saved for a reason and a purpose..
He loves you so much. Next week we're gonna talk about what it means to be the.
workmanship of God. What it means to be fashioned and shaped for purpose in this.
world. And it starts here in this moment because prayer is our access point to.
everything we've been talking about here today..
And I believe that you're gonna see the power of God through prayer in your life.
like never before. It doesn't mean He's gonna answer every prayer you pray. The.
answer of prayer is always in the hands of God. It's not in the hands of man..
It doesn't mean He's gonna answer every prayer. It doesn't mean He's gonna come.
through for every dream. It doesn't mean that your life's suddenly gonna be.
perfect. But I believe that you will have a revolutionized prayer life as you come.
to remember and reflect on the power of Christ and as you pray from a Christ.
centered perspective. From all of the things that He has done for you out of.
the gratitude of that we pray. So Lord I want to pray this week that each person.
in this room would come to know your power in a new way. Father I thank you.
that you went to the cross and you died and took on the fullness of our sin so.
that we would know the fullness of life. I thank you that every single person in.
this room who confesses you as their Lord and Savior has access to the.
fullness of the Spirit. That the same Spirit that raised you from the dead now.
lives in them. I pray Lord that the power of God would move in our hearts and our.
lives and our mouths as we pray. Lord I pray that you would birth a fresh.
desperation for prayer at the vine like never before. Lord we know in the season.
where people are talking about wars and more wars and world wars and all this.
crazy stuff that we don't need to retreat. We don't need to be fearful. We.
don't need to run in the opposite direction. We can take our stand because.
we are filled with the power of Christ. We are filled with the hope that comes.
from the gospel. We are secure in our feet that are rooted in peace and we can.
be His hands and feet in this time. So Lord I pray, I pray that that all-.
encompassing power would fall on each person right now. I wonder whether you.
just be willing to open your hands as I pray. Father that power is in your spirit.
and it is released through prayer. And so I want to pray right now in the name of.
Jesus for the releasing of the power of the Spirit of God over each person here..
Fear be gone in the name of Jesus. Worry be gone in the name of Jesus. Pain be gone.
in the name of Jesus. Shame, regrets gone in the name of Jesus. Father I pray that.

$^{521}$you would release over your people love and goodness and mercy. Lord I pray that.
every person would walk out of here today knowing the hope to which you have.
called them. The great inheritance that is there in the saints and the.
incomparably great power for those who believe. For that power is like the.
working of your mighty strength that has come and raised Jesus from the dead and.
now sits him at the right hand of the Father. That every power and dominion and.
authority and every name that can be named now sits under the authority in.
the name of Jesus Christ. And that he has done that for the church, for you. So that.
you would not live a defeated life. You would not live a life where you're.
overwhelmed and struck down. But you would live a life that is full of.
purpose, full of hope, full of light and full of love. And I pray that you would.
live that life to the fullness of the Spirit of God that is inside of you. That.
you would live a life that is worthy of the calling that you have received,.
filled with the power that is only found in Jesus. And Father we thank you for.
this in Jesus name. Everyone says amen. Hey we're gonna we're gonna worship.
together so let's just spend some time together as a community responding in.
worship now..
you.
[MUSIC].
\newpage



\section{}
\label{sec:AKMiqQYiQTY}
\textbf{2024-04-29 Ephesians: The Work of Grace - God's Poem [AKMiqQYiQTY].mp3}
\newline
\newline
連結: \href{https://youtube.com/watch?v=AKMiqQYiQTY}{\texttt{ https://youtube.com/watch?v=AKMiqQYiQTY}} ~~~~ 語音日期: 2024-04-29 
\newline
\newline
\hyperref[sec:RDtMm_QR3U4]{\small{< < < PREV SERMON < < <}}
~
\hyperref[sec:index]{\small{[返主目錄]}}
~
\hyperref[sec:is1ogDlCd70]{\small{> > > NEXT SERMON > > >}}
\newline
\newline
$^{1}$Oh man, it's so good to be here with you..
If you're new to the Vine, my name's Andrew..
I'm one of the pastors here, and it's just great that we get to fellowship..
It's great that we get to come together in a city like we live in and worship God in.
that way and just for that resurrection power to be at work in that..
And that's my prayer for you as we open this time of just hearing from God's Word that.
you would continue to know and to understand and to feel that resurrection power in you..
A number of years ago, I was in a work meeting..
I was in a work meeting where I was feeling way out of my depth, way, way out of my comfort.
zone..
I wonder if you've ever been in a work meeting like that where you realize that you're the.
dumbest person in the room..
Anyone ever been in that moment?.
If not, you need a new job, okay?.
You need a new job..
I was in a meeting..
At the time, I was working for a US investment bank, and I was in this meeting room in our.
offices in Tokyo..
Now I wasn't actually supposed to be in this meeting, but one of my colleagues had called.
in sick that morning, and my boss had asked me to represent them in this meeting..
And the reason why I was incredibly nervous, sweaty and shaky kind of nervous, was that.
this meeting was with four of the top bankers, investment bankers in our bank in the whole.
of Asia..
These four people were well known to everybody who worked in the bank..
They were revered, and they were feared..
I realized just before the meeting, I was thinking about this..
I was thinking their combined net worth was probably higher than the GDP of some small.
Asian countries..
I mean, these were very powerful people..
I was just 24 years old at the time..
I didn't really know anything, and I was subbing in for a colleague into this meeting that.
I really had no preparation for..
I made sure that I got to that meeting super early..
That was a meeting that I was not going to be late to, right?.
So I get there super early..
I go into this room, big, huge conference room, massive conference table..
And I'm trying to go, "Okay, where do I sit on this table that would be the best for when.
these people arrive?".
And I'm looking at the table, and I decide I'm going to sit right down in the middle.
of this long conference table..

$^{41}$That'll give people the chance to kind of sit around me when they come in, and we'll.
all be gathered in the center of the table..
Well about five minutes of me just sitting there, kind of sweaty and a bit shaky, these.
four men walk into the room..
They all come in together about five minutes late, and they're chatting kind of like in.
loud voices together, and that kind of confident tone that people who have incredible power.
and incredible wealth sort of hold..
And I'm sitting there thinking, "Oh my gosh, this is literally the worst day of my life.".
And as they come in, they sit on the far other end of the conference table, and they sit.
down just the four of them at the head of the table, and I could realize that they had.
no idea there was somebody else in the room..
Or if they knew it, they were just plain ignoring me..
And they keep on chatting together, and after a little while, one of them looks up, and.
he stares at me, and perhaps for the first time, they realize that there's somebody else.
in the room, and he says this..
He goes, "Who are you?".
It's a great way to start a meeting, right?.
"Who are you?".
And I'm like, "I'm Andrew..
I'm Miko's replacement..
She's sick today, so I'm in this meeting with you.".
And they kind of just stared at me, all four of them..
And then they turned around and started chatting together again..
And they weren't even chatting about what we were supposed to be talking about in the.
meeting..
They were just chatting amongst themselves..
And as I sat there, really awkward, I listened in to what they were talking about..
They were talking about the state of the world, and most specifically, actually, they were.
talking about the state of humanity, which was really quite fascinating..
These four bankers were talking about the state of humanity, and I realized after a.
little while that they were having a debate amongst themselves, and two of them were arguing.
one position, and the other two were arguing the other..
Two of them were basically saying humanity is essentially sick, that the human condition.
is essentially sick, and humanity is in need of a doctor..
And if humanity can work out what it needs to help itself, then humanity can get better,.
and perhaps in the future, it might be able to thrive..
The other two bankers were arguing the opposite..
They were saying, "No, no, humanity is not sick..
Humanity is well..
Humanity essentially is good..

$^{81}$Look at our intelligence..
Look at our wisdom..
Look at the gifts and the talents that humans have..
Look through human history and the amazing ways that humans have done things in this.
world.".
They were just saying, "Look, humanity is well..
It doesn't need a doctor..
It needs opportunity..
And the more opportunity that's created for humanity, the best, the fittest, and the best.
will emerge to the top, and they'll lead the world into a better future.".
And they were arguing back and forth..
"Humanity is sick..
Humanity is well..
Humanity is sick..
Humanity is well.".
And I'm sitting there, and after a while, about five minutes of this, one of them turns.
to me and he goes, "Miko's replacement.".
Doesn't even say Andrew, right?.
He's like, "Miko's replacement..
What do you think?.
Do you think humanity is essentially sick and in need of a doctor, or do you think humanity.
is essentially well and just needs some opportunity?".
All four of them are staring at me to see what I say..
And this is like the worst position you can be in, right?.
Because whatever I answer, whichever side I take, I'm going to offend at least two of.
these powerful people in the room, right?.
So I'm like, "Is there a right answer here?".
And I'm trying to desperately work out, "What's the answer that they are wanting me to say?".
And as I'm trying to think about that, the Holy Spirit speaks to me..
And the Holy Spirit, out of nowhere, says this to me..
He says, "What would the Apostle Paul say to that question?".
I wonder if you've ever been in a situation where you didn't want God to speak to you?.
Anyone been in that kind of, anyone been in that sort of situation where God says something.
to your spirit and you're like, "Not now, God..
I'll get back to being a Christian after this meeting, okay?.
I don't want to interact with you right now..
I'm trying to save my career with these four very powerful people.".
And God said again, "What would the Apostle Paul say to that question?".
And I remember this real quick prayer as these guys are watching me..
I remember saying, "Lord, if I say that, they will probably fire me..

$^{121}$I will lose my job and my wife will leave me.".
I'm catastrophizing at this point..
And here's what the Holy Spirit said, "If you believe what the Apostle Paul taught,.
then you should say it.".
So I look up at these four men who are staring at me, and I say, "I actually think there's.
a third option..
I actually don't think that humanity is sick and in need of a doctor..
I don't think humanity is well and in need of opportunity..
I think actually humanity's dead and in need of resurrection.".
They did not clap..
In fact, the four of them just stared at me for what felt like a really long time..
And then they started to talk together and ignored me the rest of the meeting..
They obviously thought I was some religious nut..
And I suddenly in that moment realized that for the first time in my life, I truly resonated.
with what Paul says..
I knew in that moment that my career was dead and desperately in need of resurrection..
You may understand now that I am no longer a banker..
I'm a pastor..
That is my story..
It's an interesting thing, though..
This idea that humanity is spiritually dead and in need of resurrection, it's not a very.
popular one in modern cultural society..
Fascinatingly, actually, it's not a very popular one even in a lot of churches..
Because it's easier to tell people that humanity's just sick, a little bit sick and in need of.
a doctor..
And if you get the right amount of self-help and the right amount of exercise and the right.
amount of positive thinking, if you read my latest book that I've written on leadership.
abilities to you, you'll be able to thrive and flourish as best as possible..
It's easier to tell people that..
It's easier, in fact, to tell them that, "No, actually, you're well..
You're fantastic..
That everything you should ever need for future prosperity is found within you..
You just need to try harder..
You just need to do this..
You need to read this," or whatever it might be..
It's actually easy to tell people that they're essentially well and they just need better.
opportunity..
"Oh, if you had better opportunity, if you had better circumstances in life, everything.
would be fine for you.".
It's not easy to actually represent biblical truth, to actually represent how the scriptures.

$^{161}$speak about the human condition because it's challenging, because it puts us in a bit of.
an awkward space and it asks some really big questions about us..
And when the Apostle Paul is writing to his church in Ephesus and he's trying to communicate.
to them the heart of the Father, he's trying to communicate what God's will is in the world,.
he's trying to pray for them and show them the power that is put aside for them in a.
God who has been raised from death to life and now is the name that is above every name.
and that every knee is going to bow under that as he's talking about all these incredible.
things..
He comes to the second chapter and he says, "Now I want to talk about you..
I want you to understand who you are and what your condition is.".
Because if you can understand the human condition, then you can actually understand in a new.
way the profundity and the power and the wonder and the majesty of the gospel..
Because when we understand who we are, we begin to get a different revelation of the.
work that Christ has done for us and we cannot step back and say, "Hallelujah!".
Because we realize that it's not about us trying to fix ourselves, but we are literally.
dead and in need of resurrection..
And when the one who comes to bring resurrection comes, we find ourselves in a posture of humble.
gratitude and thankfulness and an infused life so that we can actually begin to live.
in the way that we were truly created to be..
We have to journey through death to receive life..
And Paul writes to the church about this..
These are some of the most famous words that Paul writes in any of his letters in the New.
Testament..
I'm going to read this to you in a minute and you're going to recognize a bunch of these.
things, themes and ideas and a theology that is very familiar to us if you've been coming.
to church for a while..
It's so familiar to you..
But I pray that as we unpack this a little bit together as a community today, you'll.
come to a deeper understanding about the reality of what it is that Christ has done for you.
and that those dead things that we've been talking about so far on the surface, that.
in those dead areas you might believe again that they could come to resurrected life..
Let me read this to you from Ephesians chapter 2, verse 1..
As for you, notice this, you were dead, not sick, not well, you were dead in your transgressions.
and sins in which you used to live when you followed the ways of this world and of the.
ruler of the kingdom of the air, the spirit who is now at work in those who are disobedient..
All of us lived among them at one time, gratifying the cravings of our sinful nature and following.
its desires and thoughts..
Like the rest, we were by nature, notice that, by nature objects of wrath..
But because of his great love for us, God who is rich in mercy made us now alive with.
Christ even when we were dead in our transgressions, for it is by grace that you have been saved..

$^{201}$And God has raised us up with Christ and now has seated us in the heavenly realms in Christ.
Jesus in order that in the coming ages he might show the incomparable riches of his.
grace expressed in his kindness to us in Christ Jesus..
For it is by grace that you have been saved, through faith..
And this not for a work in yourself, for it is the gift of God, not by works so that you.
cannot boast..
For we are God's workmanship, created in Christ Jesus to do good works which God prepared.
in advance for us to do..
Father, I just pray that as we open your word here, you would move by your spirit, that.
there would be a demonstration of your spirit's power and that for some people in this room.
who right now feel they are dead, that they would be brought to life, Lord..
And I pray for areas in our lives where we're looking for you to move, you would, through.
your word..
And we pray this in Jesus' name..
Everyone says, "Paul doesn't mince his words..
He speaks direct to his church because he wants them to understand who they were before.
they met Christ and now what Christ has done in and through them.".
And what I want to do with us just over the next number of minutes here is I want to actually.
map out for you how Paul presents this..
Because the gospel presented in this way is such a profound realization for his people.
and he wants them to journey through these 10 verses from a place of death to resurrection..
So the first three verses, he starts with this profound thing..
He says, "We have to understand that we were dead.".
Again, not sick, in a need of a doctor, not well, in a need of some opportunity..
Literally when it came to the spirit, when it comes to our soul, humanity is dead..
The word that he uses in the Greek here in these three verses, the way that it's phrased,.
he's talking about absolute and universal..
He's not just saying, "Hey, some people are dead.".
He's not just saying the really bad people, the really evil people, the people like Hitler,.
they're the dead ones..
He's saying all of us..
No one escapes the reality that because of what happened in the garden, because although.
Christ created humanity well and good in Genesis 1 and 2 and blessed them, in Genesis 3 they.
chose to take of that fruit that they were told not to..
In that, the disobedience of sin and rebellion came into the human spirit..
And basically from that point onwards, humanity has been dead in its soul..
And it's dead and crying out for new life..
And so Paul's starting point for his church is to understand that there was a time where.
in the transgressions and their sin, they found themselves dead..
In fact, in unpacking this idea of death, he says there are three things in these three.

$^{241}$verses, three things that have been at work to cause us to die..
He says it's the world, he said it's the devil, and he said it's flesh..
He says we are in the world and the ways of the world have been acting upon us..
For Paul, this is the area of something that is outside of us, something that is without.
us, outside of us, acting upon us..
And he's basically saying that the world and its cultures and its brokenness and this depravity.
of sin that's working through the world's cultures and the world's systems is keeping.
you dead..
It's wanting you to follow its way of thinking, but that is not the right way of thinking..
There is a third way..
And he's saying that there is this death that is there..
He then speaks about the devil..
He says the ruler of the kingdom of air..
It's a beautiful phrase from him, a ruler of the kingdom of the air..
This is the reality of evil..
He says evil is present in this world and we can't shy away from the fact that there.
is good, yes, but there is also evil and this evil is beyond us..
It's beyond anything that we could ever think or imagine..
It's out there in the world and there is a ruler of the kingdom of the air and that is.
at work..
And his spirit is also causing us to become disobedient, he teaches..
He then says that all of us at some point, we have had these cravings inside of us to.
satisfy, to gratify our sinful nature..
In other places he uses this word flesh..
It's just the idea of our sinful brokenness and this is within us..
So he's saying without us, in other words, what is around, he's saying beyond us is evil.
that is there in the world..
He's saying within us are brokenness and our sin and transgressions that are within us..
These three things mean that we in our soul, the thing that is eternal in our spirit, we're.
dead..
He's not saying, "Hey, humans can't do anything good.".
He's not saying that humans don't have the ability to be kind or to be loving or to be.
selfless or to be generous..
He's not speaking about just the behavioral stuff we do..
He's talking about the soul..
He's talking about our spiritual soul and he's saying that is dead..
Before you met Christ, that's the only way to think about it..
Are you following this?.
That's the first three verses..
He's like, "You've got to make sure you're starting in the right place.".

$^{281}$But then he actually goes and says in verses four to seven, but he says, "You, we are now.
alive..
We've been made alive in Christ.".
The beautiful phrase..
He's like, "Even when we were dead.".
Notice this..
"Even when we were dead in our transgressions, Christ came and paid the price, life, death,.
and resurrection, so that we now are made alive," is the phrase he uses, "made alive.
in Christ.".
That's the same phrase he uses in other places..
Sometimes it's translated, "Resurrected in Christ.".
It's profound what Paul's doing here because he's just spent a bunch of verses in chapter.
one, unpacking the idea of what God did in Christ..
That Christ, in submitting himself to the will of his Father, went to the cross, paid.
the price for sin, went into death, but was raised from death to life, then became the.
name that is above every name that every ruler and power and authority needs to submit now.
under the authority of Christ..
Christ is now in this place..
And actually, Paul has just written in chapter one that Christ now sits at the right hand.
of the Father in the heavenly realms, and that at the name of Jesus, every knee will.
bow on heaven and earth..
And all of this is happening for the glory of the church..
So he's just unpacked this beautiful picture of Jesus' journey from death to resurrection..
Now at the beginning of chapter two, he flips it onto us..
He says, "You need to understand the journey you've been in, because you were once dead,.
just like Christ..
But you have, through his Spirit, been raised from death to life.".
And then he takes it a step further..
This blows my mind..
He says, "And not only that, not only has he raised you, but you are now seated," he.
says, "in this area that he calls the heavenly realms.".
I want you to just get your head around this for a moment..
We're not just resurrected in Christ, but we are now raised up to the heavenly realms,.
and we sit with Christ in the heavenly realms..
The picture is one of relationship, one of friendship, one of camaraderie, sitting together..
It's a picture of relaxation and not stress..
When you're sitting with the one who has all authority over everything in the world, we.
find ourselves in shalom, peace..
And Paul's saying, "This is what those who are in Christ Jesus, you are spiritually sitting.
with Christ in the heavenly realms, at the right hand of the Father with Christ.".

$^{321}$In other words, all of the power that he had spoken about in the prayer that we looked.
at last week and that I was preaching about last week, all of that power is now found.
in us spiritually because we are at the right hand of the Father as well, seated with Christ..
It blows our minds..
Now I want you to see what Paul's doing here in saying this..
He wants to draw this contrast between the lowest possible thing, you are literally dead,.
to the highest possible thing, you are now seated in Christ Jesus on the right hand of.
the Father in the Spirit..
Isn't that beautiful?.
And for Paul, the reason why he's saying this, because he wants us to understand that there.
are only two types of human beings, those who are dead and those who are seated with.
Christ in the heavenly realms, just two..
Those who are spiritually dead and those who are alive and been made alive in Christ and.
are seated at the right hand of the Father..
This is hope..
But in this, Paul is saying, "Do you want to know how this happens?.
Do you want to know?".
Because guess what?.
Like Ezekiel, when God says to him, "Can these bones live?".
Ezekiel's response is, "Sovereign Lord, only you know.".
And it's the same in humanity..
Can we live when we've been soaked in the travesty of our transgressions and sins?.
Only you, Lord, know..
Only you can bring this dead thing to life..
This is why in verses 8 and 9, Paul explains how that process takes place, how new life.
comes..
And he says, "This is why you've been saved by grace through faith, not by your own works,.
so that no one can boast.".
He talks about a negative and a positive as he's trying to speak about this beauty of.
the gospel..
First of all, he starts with the negative..
He says there's this thing called works, our human works, the things that we are trying.
to do to maybe try to be better people..
He's saying, "You've got to understand that this cannot ever save you, that salvation.
is not about what we can do..
And this is countercultural to what we see in society.".
I want to challenge us that it's countercultural to what we see in Hong Kong, because Hong.
Kong is literally built on a culture of meritocracy, on a culture of if you try really hard, if.
you work really hard, if you become the best student in your class, you can get certain.
things, if you go to the right kind of universities, you can get the right kind of jobs..

$^{361}$And our thinking always in the world around us is, "If I do more, I will get more.".
Are you with me?.
And Paul doesn't want his church in Ephesus to bring that into their thinking about their.
spirit or into their thinking about their relationship with God..
If I do more, I'm going to get more..
And he cuts it out right at the start..
And he says, "You're not saved by anything you do..
You cannot ever do enough to earn your salvation," he's saying..
And this is important because he's basically said to us, "You're dead..
Dead things don't do anything, by the way.".
You want to know why works doesn't save you?.
It's because you're dead..
And if you're dead, you can't do anything..
That's number one..
Number two is this, even if you could do something, your greatest moral acts will always pale.
into insignificance to the righteousness of Christ..
No matter how great we could be, no matter how much we could do, no matter how much we.
could do for charity, how nice of a person we could be, how much love we could give,.
it would always pale in significance to the righteousness that is found in Christ..
When we try to compare our righteousness with the righteousness of Christ, pride fills us.
and we're heading towards more and more sin..
And Paul's saying, "Your works, both because you're dead, you can't do in any way, but.
even if you could do them, they wouldn't even compare to the righteousness found in Christ..
None of that is going to get you anywhere, so no one can boast.".
This is really important..
In the Greek and Roman culture, boasting was a huge thing..
We'll talk more about that in another week, but even here in Hong Kong, boasting is a.
big thing still, isn't it?.
Oh, no, it's not..
Okay..
It's not about the car I drive, the watch I wear, the clothes I have, the university.
that my kids have gone to..
We love to boast, I think, all the time..
Could you imagine what church would be like if we actually earned our salvation?.
Have you ever stopped and thought about that for a moment?.
Can you imagine what church would be like if all of us earned our way into heaven?.
Number one, most of us would not be in this room, myself included..
Number two, those of us that did get in, we'd be like, "Whoa, we got in.".
And no one else did..
We're the cool ones..

$^{401}$We're the lucky ones..
We're the great ones..
We're the awesome ones..
Paul's like, "There is no room for that kind of mental thinking or that attitude within.
the church..
No, you are all dead, even in your transgressions and sin..
But when you were dead, Christ raised you from death to life, not because you did anything.
so that no one could boast, so that no other name other than the name of Jesus would be.
heard in the world..
Not a name of a church, a name of a preacher, a name of a worship leader, a name of a song,.
not anything like that..
The only thing that gets to be projected in this world is Christ..
That's why our works don't save us," Paul's saying..
So you're not saved by works..
He says this, "You're saved simply by this incredible idea of grace.".
Grace, this Greek word means unmerited favor, this idea of a gift..
This idea of a gift that we have, and we talked about this quite a lot at the Vine, we have.
not earned and we have not deserved..
We actually sing about that quite a bit as well..
We have not earned it..
We have not deserved it..
But we have this incredible grace from God..
And here's the reason Paul is saying that it is not deserved or earned..
It is not deserved because you're dead..
It is not earned because none of your works could ever get you there..
That's grace..
And he's saying that this grace is at work in this world..
And that you, because you have experienced that grace in you, you are now living examples.
of that grace..
And Paul is mapping out this beautiful picture because he wants his church to understand.
the journey that they've been on, but also the journey that others could go on as well..
Because Christ has gone to the cross, Paul is saying, and he's paid the price for sin,.
and he's been raised from death to life, and he now sits at the right hand of the Father..
And anyone who comes to believe in him, anyone who asks for the forgiveness of sins, who.
confesses the name of Jesus, they also can be saved..
Anyone..
Now this is not universalism..
Christ has done it on the cross, therefore everybody's saved, and it doesn't matter what.
you do in life..
No, Paul breaks the idea of universalism with one simple phrase in the passage here in verse.

$^{441}$8 and 9..
He says, "Through faith.".
It's a really important little phrase he says..
We are saved by grace, not by our works, through faith..
In other words, there is something we as humanity do to align ourselves to the work of the cross..
It's not some universalism where Christ has died for all, and it doesn't matter what you.
do..
No, we align ourselves..
We receive the power of the work of the cross..
We receive the power of resurrection life through faith..
Through recognizing that Christ has done this, and moving in that..
For Paul, the definition of faith is two things..
It is both belief and trust..
Is this helping anyone so far?.
Both belief and trust..
And both of those things are important..
Sometimes we think faith is just about belief..
As long as I can believe the right things, then everything's going to be okay..
For Paul, if you look throughout his letters, it's belief and trust..
Those two things work hand in hand..
Give you an example of this..
A number of years ago, I was at university in a place called Durham in the northeast.
of England..
Durham has this beautiful river that runs through the center of the town, and there.
are some high, maybe 40 feet or so cliffs that are around it, and then the river weaves.
through the town..
At university, there's this thing called Freshers' Week..
It probably happened at your university too, where all the new people come to the university,.
and parties are held, and bands play..
In my time, weird as it sounds, there was a circus..
They had a circus at the university to welcome in all the new people..
One of the circus acts was the tightrope walking thing..
On one day, this guy strung a metal wire across the two sides of the cliff face over the river,.
and he walked from one side to the other on this small little tightrope..
It was amazing..
There were many of us in the crowd that day watching this guy walk from one side to the.
other..
We thought halfway across he was going to fall..
He would shake, but no..
He got from one side to the other..

$^{481}$We were cheering him when he got to the other side..
When he got to the other side, he did this crazy thing..
He took a colleague from the circus, a light person..
They weren't that heavy..
He put him up on his shoulders, and then he walked back the other way with this guy on.
his shoulders..
It was one of the most amazing things I've ever seen..
I can't believe that he's doing this with a guy on his shoulders..
We surely thought he was going to die every step of the way, but he got there to the other.
side..
When he got to the other side, we were cheering..
We were screaming..
This was the coolest thing we'd ever seen..
As the guy gets off his shoulders, the guy who just walked across, he points to somebody.
in the crowd..
He goes, "Do you believe that I could do this with you?".
The guy in the crowd was trying to shrink away..
He said what I think all of us were feeling in that moment..
He was like, "Well, I believe you can do it because I've just seen you do it, but I don't.
know if I would trust you to do it.".
This is faith..
It's more than just the belief that Jesus died and rose again..
It's trusting your life to him and entrusting, trusting that he will carry you from here.
to here..
So, Paul, what he's trying to do is help everybody to go, "Okay, this is the journey..
You are dead in your sin-induced transgressions because of the world, because of the devil,.
because of the flesh, but you have been raised to new life in Christ through faith..
That new life leads you into the heavenly realms where you're now seated at the right.
hand of the Father..
This is not done for you by works..
No, it's done for you only by grace that is found in Christ Jesus..
And in that grace, undeserved, unmerited by you, through faith, your belief and trust.
that Jesus indeed died for your sins and rose again, you are able to also now no longer.
be stuck in the death of your soul, but you can find new life through Christ Jesus..
For it is by grace that you have been saved in faith.".
And this, not something that you have accomplished..
"For if it was by your works, you would boast..
This is a work of the Spirit," Paul says..
Isn't that amazing?.
Nobody thinks that that's amazing..

$^{521}$Is that amazing?.
That's the gospel..
Now, here's the crazy thing..
Paul doesn't stop there..
He has one more verse..
I want you to read this because this is one of the most famous verses that we have in.
Ephesians..
"For we are God's workmanship," Paul says, "created in Christ Jesus to do good works,.
which God prepared in advance for us to do.".
This is how he wraps it all up..
He says, "Now that you understand the journey that there is in going from death to the fullness.
of life through Christ by His grace, you have to realize now that you, therefore, are God's.
workmanship," is the phrase he uses..
The Greek word is the word poiema..
It actually is the word that we translate into the English word poem..
Paul's like, "You are God's poem," that He's written something for the world to see in.
and through you..
When he says, "You are God's workmanship," he's not talking about that first moment of.
your creation..
Elsewhere in the Bible, it says that you were knitted and made in your mother's womb, and.
that's an amazing thing, but that's not what Paul's speaking about here..
When he says that you are a workmanship of God, what he's speaking about is everything.
that he's just said in the passage..
He's talking about the fact that you have been transformed and renewed..
Elsewhere, he says that when you're in Christ Jesus, you're a new creation..
The old is gone, the new has come..
This is his phrase for it here in Ephesians..
You are God's workmanship..
Another way of saying it, you are a masterpiece of art, that Christ has changed your life,.
literally taken you from death to life, and He has done that for you by His grace, and.
you now are a living embodiment of the grace of Christ in your life..
That's awesome, he's saying..
You're a masterpiece, a masterpiece of the grace of God, and because you're a masterpiece,.
you shouldn't shy yourself away..
You should hang yourselves in the galleries of this world..
You should be in the marketplace..
You should be in society..
You should be in your families..
You should be in your workplaces..
You should be influencing everything out there because you are a work of art..

$^{561}$You're a masterpiece on display for all the world to see, he's saying..
Michelangelo, one of the great sculptors, once was chipping away at a shapeless rock,.
and somebody asked him what he was doing, and his reply was this, "I am liberating an.
angel from this stone.".
And I think if you were to say to Paul, "How would you help people to understand what Christ.
has done in them?".
He would say, "He has liberated you from the stone of death, and He has shaped and formed.
you into someone who can now live, and live in a way that would show the world the glory.
of Christ.".
So he says, "There is a work that Christ has done in you so that you can now be a work.
in the world.".
It's not the works of salvation that we've just talked about, because we know that those.
works are not what it's about..
This is a new work that we can now do, not one that earns us salvation, but one that.
shows the world the grace of salvation..
So he says, "You go now and do that..
You go be Christ in the world, His hands and His feet and His grace, living as this embodiment.
of a different narrative..
You are Miko's replacement, sat in a room, and when the narratives of the world are presented,.
you have the courage to say that there's a third way, that you have a testimony in your.
life of a life that has been changed by the grace of Christ.".
If we were to summarize Paul, we'd say it this way, "What you are is a gift of God..
What you now do is your gift to Him..
You're a gift by grace, because He loves you, and out of love and mercy, He has changed.
and transformed you..
Now how you live matters.".
Just a little bit further on, we'll see this in a few weeks' time, Paul will say, "Live.
a life worthy of that calling.".
He challenges his church..
Live a life worthy..
That's what he's saying here too..
You're God's workmanship, created in Christ Jesus for good works..
I love the way that Erwin McManus puts it..
He says, "You're not just a work of art, but you are also artist at work.".
So if you're a painter, paint..
If you're a writer, write..
If you're a teacher, educate and teach..
If you're a banker, bank..
If you're a lawyer, if you're a lawyer, I pray for you, if you're a lawyer, judge fairly..
If you work for the government, lead humbly and diligently..

$^{601}$Whatever it is that is before you, do it in a way that honors and glorifies Christ, because.
you are no longer dead..
He can bring the dead things and bring them to life..
My prayer for you is that you would go from here today filled with this joy in your heart,.
that you have been one who has been chosen by Him in grace..
And in your faith and trust and belief in Him, you have aligned yourselves to the forgiveness.
that was there on the cross, a forgiveness that sets your soul free, so you can now live.
as you were truly created to be..
May you feel the wind in the Spirit upon you, as you be who you have been created to be,.
in the spirit of influence He has placed you in..
For you have been saved by grace, not by works, so that you cannot boast, but through faith,.
so that you as God's workmanship may be released to good works..
Can I pray for you?.
Let's pray..
Father, we're so grateful for the people in this room, people that you love deeply and.
dearly, people that you have saved by a gracious act..
Father, I pray for anyone in this room who perhaps is just exploring faith in this time..
Maybe spirituality is a new thing for them..
And Lord, maybe they're exploring what it is to explore the truths and the beliefs of.
Scripture..
Father, I pray for anyone here that is in that camp, I just pray, Lord, that today they.
would feel the encouragement and the challenge of your Word..
You love them so much..
It is out of your love and your mercy, as Paul writes in this passage, that you can.
take them from death to life, and the kind of life that they have never felt or never.
lived before..
Lord, I pray for anyone in this room where that's resonating with them now, for the first.
time, I pray, Lord, that they would reach out to you today..
Maybe they would talk with the person that they've come from, or maybe they would come.
up and chat with us as a team after the service, and we can help you..
But if you want to give your life to Jesus, it's really a simple thing..
You just, in a simple prayer, just come before Him and ask Him to be your Lord and Savior..
You confess those sins in you that have kept you dead spiritually, as we've been talking.
about today, and you, through belief and trust in Him, thank Him for forgiving your sin and.
giving you new life..
And then you can just simply pray and ask for His Spirit to help you, to walk with you.
and strengthen you..
I want to encourage you to do that..
Do that with the person you came with..
Come up to us afterwards..

$^{641}$We'll do that with you personally..
But I guess that there are many of us in this room where we are already Christian..
I pray that today would be an important reminder for you of the grace that has been in your.
story..
And I pray out of that grace, you would leave here today with a bounce in your step, wanting.
to live that grace life publicly before a broken and tired and needy world..
That you would realize again that you are God's workmanship..
You've been created in Christ Jesus for good works..
My prayer for you is that you would walk out those works in whatever spirit of influence,.
whatever gifts and talents that He has placed upon you, would come to the fore in your life.
so that Jesus' grace would win many others from death to life..
Lord, we thank you for this..
In Jesus' name, everyone says..
Would you stand with me and we're just going to respond together in a time of worship as.
we just let the Holy Spirit continue to just minister with us..
(gentle music).
[BLANK AUDIO].
\newpage



\section{}
\label{sec:Ps22iab4JXU}
\textbf{2024-05-06 Ephesians: Making Peace [Ps22iab4JXU].mp3}
\newline
\newline
連結: \href{https://youtube.com/watch?v=Ps22iab4JXU}{\texttt{ https://youtube.com/watch?v=Ps22iab4JXU}} ~~~~ 語音日期: 2024-05-06 
\newline
\newline
\hyperref[sec:is1ogDlCd70]{\small{< < < PREV SERMON < < <}}
~
\hyperref[sec:index]{\small{[返主目錄]}}
~
\hyperref[sec:jk_FZgF1kQs]{\small{> > > NEXT SERMON > > >}}
\newline
\newline
$^{1}$Amen..
Thank you worship team for leading us..
That was beautiful..
Thank you church..
You guys can have a seat..
Get yourselves comfortable..
Good afternoon, my name's Ellison..
It's an honor to be sharing God's word with you today..
And yeah, the sun's shining,.
so everybody got here nice and dry..
That's really good..
Glad to be here..
Well I'm not sure if you've noticed.
in the songs that we sing,.
in the popular songs that we sing,.
there is this sentiment, right?.
For some reason, there's this idea out there.
that to go a long distance for the one you love.
is really, really romantic..
Okay, it's something really beautifully romantic about that..
As the saying goes,.
distance makes the heart grow fonder, right?.
Technically, I guess what that means.
is the further you are apart,.
the more romantic it is, right?.
So I can testify to this a little bit..
There was a period for about nine months, right,.
when Brittany and I were dating long distance,.
nine, 10 months..
I was here in Hong Kong, she was back in Texas..
And I have the receipts, okay?.
I'm not lying, okay?.
And she's sitting right there, so you can ask her..
For nine months, every single day,.
I wrote a handwritten letter,.
put it in the mail, and sent it over to Texas, right?.
Sometimes it'd just be one word,.
like hello, or whatever, right?.
Sometimes it'd be like a little doodle I did during the day..
But pretty much every single day,.

$^{41}$I would send a letter by mail to Brittany, right?.
And it was a good thing, right?.
Yeah, we still have those letters..
If you ever wanna come over for dinner, I'll show you, okay?.
But there's that idea out there..
We sing about it, there's popular songs about it..
This is kinda old now, I'm dating myself..
But do you remember the soundtrack.
for the first Space Jam movie,.
the original Space Jam movie with Michael Jordan, right?.
There was a song there called "For You I Will".
by a pop star called Monica, right?.
And the lyrics go like this,.
I would cross the ocean for you..
I would go and bring you the moon, all right?.
I would be your hero, your strength, anything you need..
Right, a song that's made popular again,.
or actually I don't think it's ever been not popular,.
the Proclaimers, right?.
It's really famous, right?.
I would walk 500 miles, and I would walk 500 more, right?.
To be the man who walks 1,000 miles,.
1,000 miles to end up at your door..
Right, Vanessa Carlton, you know, the do-do-do-do-do-do,.
you know, she also walks 1,000 miles, right?.
To just be with the person that she loves..
Tonight, right, that's how she thinks..
But it's kinda silly, right,.
when you think about these things..
Because, okay, I'm sorry, Monica,.
but you are an R and B pop star, you're not an astronaut,.
you're not a nautical expert, okay?.
Do you know how even the smallest ocean.
is 14,000, 140,000 square kilometers wide?.
The moon is more than 380,000 kilometers away from Earth..
I'm not sure she's capable of going to the moon and back..
The average walking pace of someone,.
unless you're Carla Roscoe, who walks like lightning speed,.
is about three to four miles per hour..
If you were to walk 1,000 miles without taking any breaks,.

$^{81}$the song should really go,.
I would walk 1,000 miles to see you in about two weeks..
Not tonight, right?.
So, it's ridiculous, right?.
These things are just hyperboles,.
these things are just imagination, right?.
It's romanticized way of expressing our love for someone..
But what would you be willing to do?.
Like, how far would you be willing to go.
for the one you love?.
These days, if you ask me,.
my distance and romance is a bit more limited, right?.
I get annoyed these days when after I sit down.
and Brittany goes, can you go and get me a glass of water?.
I'm like, all the way to the kitchen and back?.
Like, really?.
Like, I don't really have the energy for that, right?.
So, maybe I could do with being a bit more romantic.
these days as well..
But we like the idea of this, right?.
We like the idea of someone doing something.
spectacular for us, of someone going all the lengths.
to the end of the world and back..
It's romantic, it's good, but no one actually expects.
their loved ones to do these kind of things for them,.
do they?.
But here's the thing, and as cheesy as this sounds,.
I have to tell you, somebody has..
All right, this is one of the key points.
we like to emphasize when it comes to the Christian faith..
We have a savior who left his home in heaven,.
came down to earth, took on sin and darkness.
and shame and evil and injustice and all this other way.
the world's been messed up, died the death.
that was meant for us just because he loved us.
and he wanted to be near to us..
And so in this process, we've been brought near to Jesus,.
he has also made us new..
And we have been made new to carry out the many.
wonderful things that he has prepared for his children.

$^{121}$to do..
This was a beautiful reminder, this was the message.
that Pastor Andrew said to us last week..
But this is just the starting point..
And if we want to know his love, his grace,.
his mercy and forgiveness, if we want to know.
the nearness of Christ, then there is a part.
for us to play as well, as his children,.
as his church, as his people..
Being close to Jesus isn't just about soaking it all in.
and being in that space, but there's a challenge.
for us too, when we receive his love,.
there is something we have to do with it..
And so we come to a point in our journey.
through the book of Ephesians, because like we said,.
yes, we've been blessed by God..
Yes, our hope is in Christ..
Yes, we have this resurrection power at work within us..
We've been raised to life by God, but there's one reminder..
God didn't do it just for you, the individual..
He did it for everybody..
And so if he did it for everybody, it means everybody.
is being brought near to Christ, which means.
there has to be unity as we're being brought together.
as one body..
The question is, how?.
How do we have this unity in Christ?.
And the passage we're looking at today.
is gonna tell us a little bit about that..
So I'm gonna read it, and we're gonna go through it.
and see what God has to say to us..
Ephesians chapter two, verses 11 to 22..
It says this..
Therefore, remember that formerly you who were Gentiles.
by birth and called uncircumcised by those.
who call themselves the circumcision,.
which is done in the body by human hands..
Remember that at that time you were separate from Christ,.
excluded from citizenship in heaven,.
and foreigners to the covenants of the promise,.

$^{161}$without hope and without God in the world..
But now in Christ Jesus, you who were once far away.
have been brought near by the blood of Christ..
For he himself is our peace, who has made the two groups one.
and has destroyed the barrier, the dividing wall of hostility.
by setting aside in his flesh the law.
with his commandments and regulations..
His purpose was to create in himself one new humanity.
out of the two, thus making peace..
And in one body to reconcile both of them to God.
through the cross by which he put to death their hostility..
He came and preached peace to you who were far away.
and preached to those who were near..
For through him we have both access.
to the Father by one spirit..
Consequently, you are no longer foreigners and strangers,.
but fellow citizens with God's people,.
and also members of his household,.
built on the foundation of the apostles and prophets.
with Jesus Christ himself as the chief cornerstone..
In him the whole building is joined together.
and rises to become a holy temple in the Lord..
And in him you too are being built together.
to become a dwelling in which God lives by his spirit..
So let's go back to Paul..
Let's go back to this time in Ephesus..
As we know, it's not an easy situation.
that Paul is speaking to when this letter that he wrote.
reached the people of Ephesus..
Apart from the culture of the pantheon of gods.
that different people worshipped,.
there was also the very open and hostile division.
between Jews and Gentiles..
The Jewish people stood firm on their faith.
in the part of God's history as his chosen people..
And there were a few things that marked them out as such..
And circumcision was one of these main things..
That's the main emphasis of what Paul's trying to say here..
Now the act of circumcision itself, of course,.
is rooted in their history..

$^{201}$Given to Abraham by God, Genesis 17,.
the point of circumcision was a physical sign.
of the covenant that God had given his people..
It was like a physical marking of you are separated.
to be a part of God's people..
And so therefore circumcision became one of those things.
that was used against non-Jews to keep them separate..
If a Gentile man wanted to enter into the community of God,.
the people of God, it was expected.
that he would have to be circumcised..
This was the established mindset at the time..
Which is why the point Paul is making here.
sounds actually really, really shocking..
It would have sounded really shocking..
Because Paul makes it obvious to the readers,.
to the hearers, to the listeners..
It's obvious that circumcision, he says,.
no longer carries the importance that it used to..
And one of the reasons he's saying that.
is because now it's actually being used to exclude people.
from entering into the community.
that God has called them to be..
Right, Galatians 5, 6, Paul says this in a different way..
For in Christ Jesus, neither circumcision.
nor non-circumcision has any value..
And why is this?.
Because Paul tells us circumcision is something.
that is done in the body by human hands..
Now when Paul says by human hands,.
that's also significant here..
Because against the backdrop of the idol worship of Ephesus,.
using the phrase by human hands,.
that's the same phrase that people would use.
to describe the idols that were being made.
by the people in Ephesus at the time..
Those too were made by human hands..
Right, lifeless, powerless objects..
Not made by God, not designed by God,.
but only by humans, by their hands..
And so therefore, by talking about circumcision.

$^{241}$in the same way, you would say that Paul's talking,.
this is circumcision, it's just like these idols..
It's become a little bit irrelevant, right?.
It's got no power, it's no longer a critical part.
of your identity as God's chosen people..
Now, what has this got to do with us?.
I just said circumcision 13 times, okay,.
in the past two minutes..
What does this have to do with us?.
Well, let me ask you this way..
What practices do we currently hold.
to keep people out of community with us?.
But yet, actually, this is something that's worthless.
in the eyes of God..
We might not have something as physical,.
as outward as circumcision,.
so it might not be as obvious to us,.
but it's something we got to be asking ourselves,.
because if there are such barriers,.
then we need to name them for what they are,.
and name them for the worthless things they are,.
because they have no place in the community of God..
Because as we're about to find out,.
there ought to be no barriers for those.
who want to enter into the community of Christ..
No man-made barriers for those who want to enter.
into the community of Christ..
So let's think about it this way, right?.
If you're here listening today,.
if you would call yourself a follower of Jesus,.
think back..
Think back to when, before you became a Christian,.
before you came to know Jesus..
What was your life like back then?.
Like most people, like myself,.
you probably wrestled with a lot of doubt, right?.
Questions and challenges as you began to come to grips.
with what it means to be a life,.
to live a life as a Christian, as a Christ follower..
Right, there's probably already enough obstacles.

$^{281}$that you had to overcome without people.
putting more barriers up in the way..
In fact, being separated from Christ,.
it's a terrible place to be, Paul says..
There's five things that you are at a disadvantage at.
before you knew Jesus, okay?.
These are, okay, these are the things..
You are separated from Christ,.
you are excluded from citizenship,.
foreigners to the covenant, without hope, without God..
This is the way Paul is describing the Gentiles, okay?.
But it's also relevant to us..
Before you come to know Jesus,.
before you are in God's people,.
these are the five things that you are.
sort of at a disadvantage for..
And they come together to make a point,.
so let's talk about it..
Firstly, being separated from Christ, right?.
Means you probably didn't even have the idea.
that there was a salvation that you needed, right?.
That there was a savior out there.
that would come to rescue humanity,.
that you were a sinner, indeed, of rescuing from your sins..
Right, because like what Pastor Andrew said.
last week as well, right,.
back then you were spiritually dead, right?.
Dead people do not have the ability.
to raise themselves from the dead..
Dead people need a savior, need someone to intervene,.
to give them the resurrection power.
to raise them from the dead..
You cannot save yourself..
You needed rescue..
So that's what it means to be separated from Christ..
Now, because the Gentiles weren't part.
of God's original people as well,.
they also didn't share in the first covenant,.
the blessings and the promises.
that God had originally planned for his people, right?.

$^{321}$The Abrahamic, the Mosaic, the Davidic covenants,.
they weren't a part of that..
They were outsiders, which means back then.
they couldn't participate in God's promises..
And therefore, that means they also had no hope.
because they had no God..
Well, maybe they had a little bit of fake sense of hope.
and a little bit of idea of some gods, right?.
As we have discovered, there were so many gods.
that people, the Gentiles of Ephesus,.
before they came to know Jesus,.
were probably worshiping all sorts of different gods.
and idols that were a part of the culture..
And in the same way, it probably was for us too..
Before you came to Jesus,.
think about the things you used to put your hope in..
Right, think about the things you might use to worship..
Some of us might have come from homes and households.
that literally worship idols..
For others, it might be idols of power and prestige.
and all these things in culture.
that we tend to put our hopes and dreams in..
Whatever it is, this isn't true hope,.
and this isn't putting your hope in the true God..
We lack connection to the one true God..
We lack connection to hope,.
God who is the source of all life..
So it's not a good picture..
It's pretty clear what Paul's trying to say..
We, like the Gentiles, were once about as far away.
from God as one could be..
So let me ask you again, think about it..
Do you remember a time when you were far away from Jesus,.
when you did not know him as your savior?.
Or perhaps if you're listening right now.
and you don't quite know Jesus as your savior yet,.
maybe you're sitting here listening,.
you're just exploring,.
you've heard a little bit about this Christianity thing,.
but you're not sure how legit it all is,.

$^{361}$and you're wondering, what is this talking about?.
What is this hope, what is this God?.
Let me say this very clearly, and I say this with love..
At one point, all of us have been separated from Christ..
All of us, right?.
And because we were separated,.
we weren't part of his promises..
Of course God loves all people,.
but there's a certain benefit, there's certain promises,.
there's certain function as God's children.
that you can only know once you enter.
into a relationship with him..
The point is this, being separated from Christ.
is not a good place to be..
In fact, being separated from Jesus.
is probably the worst state that you could find yourself in..
And I know that sounds drastic,.
but I don't say it lightly either, right?.
Because I know amongst us, in this church,.
in our city, in our world,.
there are people in so many different situations.
walking through some terrible things,.
walking through pain, trauma,.
stuff that you can't even think of,.
things you wouldn't wish upon your worst enemy..
These things are tough, and yes,.
when we see people walking through these things,.
we need to walk alongside them, fight for them,.
pray for the healing, bring justice.
to those who are suffering..
But I would say this, a person who is suffering,.
who is in pain, but who knows and is connected to Jesus.
is ultimately in a better place.
than someone who is comfortable,.
but is living without hope and without God..
To be living in this world without hope and without God.
is a true tragedy, and it's as true then as it is today..
But this is the good news..
But now, Paul says, but now,.
we don't have to live this way..

$^{401}$We don't have to live a life separated from God..
We don't have to live a life separate from Jesus..
We can enjoy his blessings..
We can live in the calling and the purpose.
that he gives to us..
We have that hope..
We have God because in Christ Jesus,.
you who were once far away have been brought near.
by the blood of Christ..
Jesus has done that for us..
We have been brought near by Christ.
through his sacrifice, through his blood..
And amen and praise God for this..
This is the part that we could have never done for ourselves..
Jesus has given us life..
His incomparably great power has been at work within us,.
and this is nothing we can do for ourselves..
And we can't do anything else but say thank you.
and accept his grace and his love and his mercy..
But like I said, that's just the starting point..
And being brought near to Jesus means a new era,.
a new season, a new direction in your life has begun..
A new way to live is on offer..
And this new way to live is not just about.
a certain group of people now..
It's not just about individuals being brought close.
to Christ, but this is talking about all people,.
Paul is saying Jews and Gentiles alike..
Two people that were once enemies with each other.
have now come together to become one,.
and this is where the challenge is..
Because this new way to live involves peace..
And it involves peace not just the peace.
that you have with God, but peace with each other..
It involves everyone living as if there were.
no barriers between us, living as if nothing.
would be able to separate us from each other..
Paul goes on to say this..
For he himself is our peace, who has made the two groups one.
and has destroyed the barrier, the dividing wall.

$^{441}$of hostility by setting aside in his flesh.
the law with its commands and regulations..
His purpose was to create in himself one new humanity.
out of the two, thus making peace..
And in one body to reconcile both of them to God.
through the cross, to which he put to death,.
by which he put to death their hostility..
Peace..
Right, peace, this is the word that the world.
is crying out for at the moment..
Apart from the bigger, more obvious wars.
that we see in our news feeds every day,.
Ukraine and Russia, Israel and Palestine,.
dominating the news 24/7 these days,.
but actually there's about 110 other armed conflicts.
going on in the world right now..
We need peace..
We want peace in this world..
And indeed, we could do a whole sermon series.
on war and violence and how it breaks God's heart,.
the injustice of war, and when we see these things,.
we get moved to pray, right, we want these conflicts to end.
and we get passionate, we start social media campaigns,.
all that kind of stuff, and this is good..
We want to raise awareness about these things..
Great, that's fine..
But perhaps before we wanna do something.
about these major wars we see around us,.
we need to address the conflicts and hostilities.
that lie closer to home..
And this is what I wanna focus on today..
Right, because for Paul and the Ephesians,.
these hostilities were close to home, right?.
They were between the Jews and the Gentiles..
The prejudice, the hatred, the racism.
between the two groups was intense for many reasons..
And so Paul's saying, we need to deal with this..
We need to break down any hostility,.
any barriers that are stopping us.
from coming together as one right now..

$^{481}$And so circumcision was one of those things,.
but the other thing that was keeping them separated.
was the Old Testament law..
The Old Testament regulations that God had once.
commanded them to live by..
Now here's the thing..
For a season, the Old Testament laws.
were given to the Israelites were good, right?.
They were used to protect them.
from becoming like the nations around them..
You see what happens when you read through the Old Testament..
Every time the Israelites step away from those boundaries,.
right, they get carried away, they start worshiping idols,.
then God has to send all sorts of ways.
to let them know they've gone wrong, right?.
And then they start the whole cycle over again, right?.
Wait, first kings, second kings,.
all these kind of things keep happening.
because the Israelites fail to live in the regulation,.
in the protection of the laws that God has given to them..
Now for a season, those were good..
But very quickly, these laws also became.
a source of pride and arrogance..
Meaning instead of being a fence to protect the Israelites,.
they started using it as a wall, as a barrier,.
as a sign of hostility to prevent people from coming in,.
to exclude people, to say, we're the people of God,.
we don't want you to be a part of us..
Right, people became so focused.
on following these rules and regulations,.
they forgot that the purpose of these rules.
was to keep them close to God in the first place,.
not to make themselves look good as rule followers.
and as regulation followers..
This is why Jesus gets angry.
with some of the religious leaders and Pharisees.
of his time..
He calls them things like brooder vipers.
and whitewashed tombs..
You look good on the outside, but in reality,.

$^{521}$your hearts are far away from God..
Remember this, the point is that we have been brought near.
by the blood of Christ..
And on the cross as Jesus died,.
his blood fulfills the old covenant..
And so now, those who want to worship God.
don't have to do through rules and rituals and sacrifices,.
they simply have to come..
In doing so, Jesus' death killed the hostility.
between Jews and Gentiles..
The one who was slain also slayed the hate.
between two groups that were at enmity with each other..
And so the result is this, something beautiful,.
even something surprising..
See, the plan all along wasn't to make Gentiles.
become more Jewish so they could join.
into the nation of Israel..
His plan was to create in himself one new humanity.
out of the two, thus making peace..
This was never about conforming one people group.
to become more like the other,.
not about Gentiles making them Jewish.
or Jewish into Gentiles..
It was always about creating something new,.
which is the body of Christ..
It's almost romantic, the union of two groups.
coming together in one humanity..
It sounds a bit like what Jesus says about marriage..
The two become one flesh..
They are no longer two, but one flesh..
And so good, right?.
Bravo, we cheer at this..
This is amazing..
Two groups that were once at conflict with each other.
coming together to create something even more beautiful..
It's like a fairy tale come true..
Everyone likes that story..
In fact, if you might say, actually, you know what?.
We've done this as a church..
We're good..

$^{561}$I mean, look around us..
There's people from all over the world.
sitting with each other..
We have peace..
There's no fighting within our worship services..
I don't really hear people cursing at each other..
We're super friendly in this church..
We're friendly, we invite new people..
We even do the awkward thing sometimes.
when we sit down and say hello to your neighbor.
and no one ever goes, "Ee, not this person," right?.
Yeah, we're good here..
We even pray with each other sometimes, right?.
Isn't that great?.
Okay, yeah, I will give us that..
We don't have the very obvious Gentile Jewish divisions.
that Paul was talking about..
We might not be openly fighting with each other, right?.
It's not the USC every time we gather together to worship..
At the same time, I don't think we should be so naive.
to say that we've got everything right.
and there's no work to be done in creating unity and peace.
within this family, this body of Christ..
I would say we shouldn't be so blind.
that in fact there might be some wars and hostilities.
that we hold against each other..
But here's the problem..
It's hard to see where these hostilities lay.
when you're not at the receiving end of them..
It's hard to see where people do not feel at peace.
when we don't put ourselves in the shoes.
of those on the receiving end of injustice..
Something I experienced myself really opened my eyes to this..
So a few years ago, we were visiting my wife's wife..
I say that three times, I said that now..
I said it every time..
My wife's wife, who's that, Kayla?.
I don't know, right?.
My wife's family in Texas..
We took a road trip..

$^{601}$And one of the stops we had on this road trip.
was in Nashville..
We were going from Texas all the way up to Buffalo,.
into Niagara..
So we looked at a nice, long road trip..
And one of the stops was Tennessee..
And I was super excited..
Tennessee, the home of country music..
I have to admit, I love country music..
I might not look like it, but I love country music..
Country music and hip hop..
That's what dominates my Spotify playlist..
Ask Emma and Promise, one of these days.
I'm gonna lead worship doing a country song.
and we'll see how it goes..
But Nashville, is there anyone from Nashville.
or been to Nashville?.
Beautiful city, right?.
Amazing food, great atmosphere..
But you walk down the high street, the main street,.
and there's about a thousand cowboy hats,.
aspiring musicians, a beautiful place..
And so we decided to go to one of the honky tonks.
that they have..
A honky tonk, by the way, is one of those bars.
that you go, it's like a dance hall..
Live band, play music, you can have a beer,.
sit with people and have a good time, dance,.
and do all that kind of stuff, okay?.
So, imagine the scene..
Here I am, a Chinese guy, walking into this honky tonk.
with my white Texan wife, hoping to have a good time..
But the moment I walk in, right, and it wasn't everybody,.
but I just felt this two pair of eyes.
just staring at me immediately, the moment I walk in..
Right, and I see that the man kind of leans over.
to the woman, whispers something in her ear,.
and having a conversation back and forth with each other,.
there's visibly pointing at me now..
Okay, I have to admit, I didn't hear what they said,.

$^{641}$right, they could have been, whoa, look at that guy,.
he's super cool, right?.
(audience laughing).
They could have been saying that, okay,.
but that wasn't the vibe I got from them,.
that wasn't the sense I got from them, okay?.
What they were communicating actually was,.
what are you doing here?.
You see that guy that just walked in?.
Why did he just walk in here?.
He doesn't belong in here..
Obviously, they didn't want me there..
Now, apart from Brittany, who was with me,.
no one else noticed, right?.
The rest of the people that were with didn't notice, right?.
Why should they?.
No one's ever looked at them in that way,.
no one's ever judged them in that way..
They've been to a thousand honky tonks..
Every time they go in, they have a fun,.
they have a good time, all right, it's great..
But for me, the night was over, all right?.
Then the moment that happened, I just wanted.
to get out of there as quickly as possible,.
and so we left..
Okay, let's bring it a bit closer to home..
When I walked into the building this morning,.
as I do every Sunday, when I walk in, I feel good..
I feel confident, right?.
I feel like I belong here..
I don't worry about how people are gonna treat me..
I don't feel like people are gonna say,.
"You're not supposed to be here.".
I've never experienced anything like that here..
People are always kind and friendly and respectful..
For some strange reason, some of you even want.
to come and speak to me, okay?.
It's great for me..
But I wonder if we could confidently say.
that this is the same experience everyone has.

$^{681}$when they come into this building..
I wonder if this is the same experience.
that people can say they have when they go.
into other places, other establishments here in Hong Kong..
I wonder for some if coming even here to the Vine,.
walking into this church building for worship on Sunday.
might feel a bit like me walking.
to the honky-tonk that night..
Perhaps for some of us, coming here every Sunday,.
there's just a bit of nervousness,.
a little bit of worry about whether or not.
people are gonna treat me with respect..
Maybe some of us come into this space feeling.
there are walls, there are unspoken rules.
and hostilities between us as a body of Christ..
Just because you haven't felt it or seen it.
doesn't mean it's not happening..
In fact, one of the most powerful barriers.
that happen without us even realizing it,.
and it's one of the ways it happens most frequently,.
is with our language and the way we speak.
about each other and other people..
And I'm speaking specifically to Cantonese,.
Hong Kong Chinese, Cantonese speakers here..
But even if you're not a Cantonese speaker,.
you can probably contextualize it to apply it.
to your own identity, your own culture.
from what your sphere is..
Because I would say this, a lot of Cantonese language.
is very casually racist, not even low-key racist,.
like properly racist..
Common racial slurs in Cantonese,.
terms like hak kwai when we're referring to black people..
Things like a-tah or a-sing when we're referring.
to Indian, Pakistani, and South Asian people groups..
These are terms that Chinese Cantonese speakers.
say behind the backs of these people.
as a derogatory way of referring to them..
You could even say guai lo is sort of like.
embraced right now as a term, but if you think about it,.

$^{721}$the origins of that aren't very friendly either..
Or what about the way we refer to our Filipino sisters?.
The term ban muy when we're talking about.
our Filipino foreign domestic workers..
In fact, when it comes to them, right,.
we forget that they are human beings sometimes..
Some of the conversations over here are shameful.
when it comes to the way we refer to our Filipino sisters.
who live and work in the city, serving our families..
Often they're talked about like they're commodities..
Oh, this one's not good, this one's not good at that,.
this one, and then they talk about them.
like they're commodities, things that they can get rid of.
when it's no longer convenient for them..
This is the kind of language we use very frequently..
What about attitudes towards Hong Kongers.
used towards our mainland brothers and sisters?.
Now I struggle with this..
When I see those tour groups getting off the bus,.
when I'm being crowded on the Star Ferry,.
when I'm just trying to relax or take in the views.
or enjoy the Star Ferry ride and there's like 500 people.
walking in at the same time, it's all loud and crazy,.
I admit that the feelings that come out in my heart,.
in my mind, are not very pleasant sometimes..
They're ugly and they're wrong..
I wrestle with this..
We need to do better than this, church, right?.
We need to do better than this..
We can do better than this..
Because if this is the way we talk,.
then how are we gonna relate to each other.
if this is the sort of like, in the back of our minds,.
this is the prejudice that we've already set up..
These are walls of hostility.
that we've already set up against other people..
So we need to do better..
This means watching your own language,.
being aware of the words that come out of your mouth..
This might even mean sticking out a little bit.

$^{761}$when you overhear conversations.
or when you're amongst your friends.
and they start throwing out those terms sort of freely..
We need to call it out for the sin it is.
and refuse to participate in it..
The point is this..
If the church is just a reflection.
of what our culture is like,.
especially when it comes to racial and cultural prejudice,.
then I would say it shows we haven't fully understood.
the message of the gospel even..
If we do not have unity that Jesus has called us to,.
then maybe we haven't fully embraced.
the power of the cross..
We fail to see God's heart..
His desire was to bring all people unity,.
to reconcile all people together..
Yes, Jesus died for your sins on the cross,.
so you could have a transforming, powerful, meaningful life,.
all that good stuff..
We know that..
But let me read verse 15 and 16 once again..
His purpose then was to create in himself.
one new humanity out of the two, thus making peace..
And in one body to reconcile both of them to God.
through the cross by which he put to death.
their hostility, all hostility..
So if we claim to love Jesus, we claim to love his word,.
we claim to love worshiping him,.
then we must take seriously the demands.
that he has called us to live in unity.
and in peace with each other..
This means we have work to do, church..
We have to be serious about continuing.
to tear down the barriers, both seen and unseen,.
starting with this family right here,.
because that is exactly what Jesus has done..
And if we don't do that, or if we even worse,.
continue to put up walls,.
then we're literally trying to undo what God has done..

$^{801}$It's literally swapping the truth of God's word for a lie..
It's rebelling against what God has, which is sin..
We don't wanna be a church, right,.
that says peace, peace, when there is no peace..
Right, like I said, there's no conflict happening here,.
but peace isn't just the absence of conflict..
You can squash conflict in many different ways..
You could get an autocratic leader or someone like that.
to come and just dominate with his power..
Nobody's, everybody's too afraid to make a move, right?.
That could look like peace on the surface,.
but that's not true peace..
Peace is where everybody can flourish..
Peace is when everybody can be what God created them to be,.
regardless of age, gender, color, social status,.
whatever other barriers that we might have put up..
The point is this, Jesus longs for his church.
to overcome the cultural barriers of racism, nationalism,.
economic pride, and to embody a practical way.
which he created it to be,.
an attractive yet countercultural family of people.
very different from one another,.
but who love each other deeply.
and display the presence of God who is near..
The presence of God is best displayed in a church.
that is at true peace with each other,.
that has true unity with each other..
We want this place to be a place.
where the presence of God is pleased to dwell..
That's the whole point of coming together as a church,.
isn't it?.
If the presence of God isn't here, then we're not a church..
The building, this isn't what makes the church..
The presence of God with us, that's what makes us a church..
And so if we're gathering without God's presence,.
we're not a church..
This is like a TED Talk every Sunday,.
motivational speaker, good band, okay, at best..
At worst, we're like a cult that gets together every Sunday..
That's not who we want to be..

$^{841}$We want to be a place where Jesus sits at the heart,.
at the foundation of everything that we do,.
because it's only through Jesus, Paul says,.
that the whole building is joined together..
In Christ, the whole building is joined together..
He becomes the dwelling place which God lives by his spirit..
This is the temple of the Lord..
And as we have said, when this happens,.
then there is true peace..
Then there is true unity, when there is reconciliation.
between all people..
So it's a big calling..
It's a big calling..
But we have to keep working at it..
And God knows this, right?.
It's going to be a journey..
Because this is what Paul says..
In him, you too are being built together.
to become a dwelling in which God lives by his spirit..
When he says being built, right,.
this means it's a work in progress..
We are not gonna get perfect at this..
Jesus knows that..
But it reminds us that we need to keep going..
We need to be willing to do the hard work,.
excuse me, to go through the process of continuing.
to figure out what it means to be built together..
This is an ongoing work and it's not going to end.
until Jesus returns again..
This isn't a project that we take on for a while.
and put on the back burner, right?.
It's, you know, Social Justice Awareness Week..
So we take on this work and then we put it.
on the back burner again..
This is something we have to commit and promise each other.
that we're going to do..
Always to work on peace..
Always to seek unity..
Always to be reconcilers..
Because this is the type of community.

$^{881}$that God is delighted to dwell in..
In fact, I would say this is the kind of peace.
that the world needs to see..
This is how we become a reflection of what's in heaven.
as done on earth..
We can't just be a gathering of people if God is not here..
But it has to be done under the church..
The only way that we ever will see true peace in the world,.
again, this is why the church is so important..
The only thing that brings people together.
that will last forever, the only thing that everyone.
can be fully unified under is Jesus..
Because he is our source of life..
He is everyone's source of life..
So church, let's build ourselves up..
Let's be that temple, let's be that space.
that God is pleased to dwell in.
where he sees his children gathering together.
no matter who they are and where they're from,.
breaking down these walls of hostility..
I'm gonna give you three tips, three action steps,.
three things you can take home and consider,.
practical things you can do to start doing this..
These steps are taken from the Difference course..
And if you Google Difference course,.
you'll find it's a fantastic resource.
to learn more about what peacemaking actually means..
So three things we can do..
Firstly, it's this, be curious..
Be curious..
This means listening to each other..
This means learning to see the world.
through other people's eyes..
Sitting down with someone that you're not familiar with,.
sitting down with someone who has a different background.
to you and hearing their story.
and allow them to hear yours as well..
Ask each other the questions..
The moment we get to know each other,.
so many barriers will naturally come down.

$^{921}$just as we get to know each other..
Be present..
Now don't just do it for the sake of doing it,.
but put your heart into it..
Be present with each other..
Meet each other with authenticity and confidence..
Stick up for each other when things come up..
Be advocates for each other..
Be present..
And finally, re-imagine..
This means that we have to have that faith,.
that hope that this work can be done..
The work of peace can be done,.
despite how bleak things look at times..
Be reassured that this is God's heart..
And if the Lord desires to see this happen,.
then nothing can stop it from happening..
And of course, finally, we have to pray..
We have to pray..
We pray because the enemy hates unity..
Well, this whole mess with sin.
started when the enemy tried to put distance,.
successfully, between humans and God..
Right, this is why Paul tells us,.
later in Ephesians,.
do not give the devil any space to work in this..
We have to keep working on our unity..
Don't let the sun go down.
while you're angry with your brother and sister..
This is how the devil gets a foothold..
In other words, all the barriers that we have put up,.
all the barriers that have been threatened to put up,.
we need to deal with, and we need to deal with it now..
And so, church, we must pray..
The task is large,.
and it's gonna take everybody working together.
to put effort into this..
But know that as we do this,.
this is what pleases God's heart..
This is the kind of church that God wants to see..

$^{961}$This is what the church is called to be..
This is what the world needs to see in answer.
to the fighting and the violence.
and the wars that are going on out there,.
that if we're unified under the one true God,.
under the source of all life,.
we can truly have peace,.
no matter who we are, where we come from..
We can truly be that place where we live.
without any barriers, without any prejudice,.
where we come together, get to know each other,.
because that's how God has drawn us together under his name..
You have been called to be a peacemaker..
So, church, let's go and make peace in the name of Jesus..
Would you pray with me?.
[silence].
Yeah, Lord, we just,.
yeah, we acknowledge that this is a great challenge.
that you set for us,.
and there's a lot of homework we can do on this,.
a lot of ways we can reflect on.
how we've participated in this,.
the own barriers that we've put up in our own hearts..
Even within this body,.
perhaps there's some of us who have felt judged.
and mistreated,.
discriminated against..
If that's where we've been, if that's our story,.
we pray for comfort,.
and we pray for the truth to speak louder than the lies,.
to say that God loves you and recognizes you.
and welcomes you, no matter what other people have said..
And for all of us, we need to look at the prejudices.
that we hold in our own hearts,.
to see the ways we've put up walls and barriers.
that might have stopped people from entering.
into knowing who you are,.
where we haven't been a reflection of your love..
Jesus, in those ways, work in our hearts,.
help us to understand the other,.

$^{1001}$people who are different to us..
And culturally, yes, it's a cultural, it's a race thing,.
but also through the generations,.
through class and socioeconomic barriers,.
through gender barriers,.
through sexual orientation barriers,.
through social status barriers, whatever it may be, Lord,.
you have come so we can have true peace and unity.
under your name..
So help us be a church that reflects that..
Help us be a church that always prays for your peace,.
for true peace and true unity,.
so that we can demonstrate to the world.
that when we come under the all-powerful,.
almighty, all-loving name of Jesus,.
he creates us to be something more beautiful,.
a new humanity where we can all live and thrive.
and be who God has called us to be,.
the unique gifts that you've given everybody,.
where we can truly fit together as one body.
and function for the glory of the name of Jesus..
So keep working in us, Lord, and keep us near to you,.
keep us near to each other..
Grant us your peace, we pray, in Jesus' name..
Amen..
(gentle music).
[MUSIC].
\newpage

\allsectionsfont{\centering}

\setlength\parindent{0pt}
\setlength{\columnsep}{1.25em}
\setlength{\parfillskip}{0pt}
\setlength{\tabcolsep}{1em}
\raggedbottom

\pagenumbering{gobble}


\newfontfamily\leftfont[Path=../fonts/fell_french_canon/, Ligatures=TeX, ItalicFont=IMFeFCit29C.otf, BoldFont=AveriaLibre-Bold.ttf]{IMFeFCrm29C.otf}
\newfontfamily\leftcitationfont[Path=../fonts/frankruehl/]{FrankRuehlCLM-Medium.ttf}
\newfontfamily\centerfont[Path=../fonts/garamond/, Ligatures=TeX, ItalicFont=EBGaramond-SemiBoldItalic.ttf]{EBGaramond-SemiBold.ttf}
\newfontfamily\rightfont[Path=../fonts/averia/, Ligatures=TeX, ItalicFont=AveriaLibre-RegularItalic.ttf, BoldFont=AveriaLibre-Bold.ttf, BoldItalicFont=AveriaLibre-BoldItalic.ttf]{AveriaLibre-Light.ttf}
\newfontfamily\rightcitationfont[Path=../fonts/rashi/]{Mekorot-Rashi.ttf}
\definecolor{hcolor}{HTML}{D3230C}
\definecolor{rcolor}{HTML}{D36F0C}
\newcommand{\chfont}[1]{\centerfont{\huge\textcolor{hcolor}{#1}}}
\newcommand{\leftcitation}[1]{\leftcitationfont{\Large\textcolor{hcolor}{#1}}}
\newcommand{\rightcitation}[1]{\rightcitationfont{\normalsize\textcolor{rcolor}{#1}}}
\newfontfamily\flowerfont[Path=../fonts/fell_flowers/]{IMFeFlow2.otf}

\begin{sloppypar}

\chapter*{\chfont{編按結語}}

\columnratio{0.5,0.5}\begin{paracol}{2}

\fontsize{11}{13}\leftfont \Large \leftcitation{א} \leftfont 余少好文.宏志博覽群書而不忘.善存藏經籍文獻備後時之用。\leftcitation{ב} \leftfont 歸主年時.受友所薦.聞道網海.\switchcolumn\fontsize{11}{13}\rightfont \Large \leftcitation{ח} \rightfont 有見粵道之危.國之封講道千言亦將就至.急之何則為?\leftcitation{ט} \rightfont 嘗聞猶太者之傳承.在其力守口述之

\end{paracol}


\columnratio{0.32,0.32,0.32}\begin{paracol}{3}

\fontsize{11}{13}\leftfont \Large 尤以吳約翰遜者 \switchcolumn[2]\fontsize{11}{13}\rightfont \Large 統.以煉千載不

\end{paracol}

\columnratio{0.32,0.32,0.32}
\begin{paracol}{3}\fontsize{11}{13}\leftfont \Large 為重.其載上之粵語講道緩緩入耳.收之藏其音頻.善妥整存.反復而嚼.受益無窮。\leftcitation{ג} \leftfont 我城我國既限.歷一四一九之不測.肺疫延年.信徒靈長屢受圍創.神州燈臺數盡指日可待.粵道之求與日俱增。\leftcitation{ד} \leftfont 觀乎社、經、法、媒、言、信、網之地.愈趨受鋤.自翔不果.授受壓力.粵道聖言亦愈漸艱難。\leftcitation{ה} \leftfont 況崇基例乎.學苑講道屢逆權勢者.其言末強受壓.舊章盡刪以存其身。講道釋數失傳.徒嘆奈何。

\switchcolumn

\fontsize{11}{13}\centerfont 
\begin{tikzpicture}
    \node (0,0) [xshift=-0.10cm, yshift=-1.0cm, opacity=0.10]{\includegraphics[width=0.30\textwidth]{../ot_frontcover.png}} ;
    \node (0,0) [xshift=+0.20cm, yshift=+2.0cm, opacity=0.10]{\includegraphics[width=0.20\textwidth]{../christ_on_cross.png}} ;
\end{tikzpicture}
\Large 

\leftcitation{ס} \centerfont 詩百又廿七載:
\leftcitation{ע} \centerfont 非耶和華建屋宇.則匠人之經營徒.
\leftcitation{פ} \centerfont 非耶和華衛城邑.則守者之儆醒徒.
\leftcitation{צ} \centerfont 余獻是卷予華人社區.願為福音流通之器.願獻斯微材為祭榮耀上帝.
\leftcitation{ק} \centerfont 阿門

\switchcolumn

\fontsize{11}{13}\rightfont \Large 滅.時越次聖殿期及當今。\leftcitation{י} \rightfont 猶太者力廣納之.筆錄以卷軸.便以傳、閱、頌、攜、守、鎖、抄、譯、釋、編,得書塔木德、密示拿等經傳.家喻戶曉.傳流若芳。\leftcitation{כ} \rightfont 猶太者文以載道.傳其口述.今我輩粵道之傳應當作如是.遂力行粵音識辨之法.載言載道.以盡忠傳粵道以待興。\leftcitation{ל} \rightfont 蒙下賜恩惠.無畏海量字音文書.既馭上帝之道.今廣及粵語講道.重駛編程之技.匯導粵音遂字稿.重塑講道現場.以傚猶太卷軸之舉便以傳流。\leftcitation{מ} \rightfont 是卷乃粵音口述傳之屬.莫通華文白話之語.

\end{paracol}

\columnratio{0.5,0.5}
\begin{paracol}{2}\fontsize{11}{13}\leftfont \Large \leftcitation{ו} \leftfont 斯殺一違儆百逆.既禁壓之.我輩聞風無奈.在所難免。\leftcitation{ז} \leftfont 另有異人例乎.以版權之名.脅網絡頻道之舉.同授礙予粵道之存流。

\switchcolumn

\fontsize{11}{13}\rightfont \Large 惟待後繼來者之傚.以譯釋傳之於神州華文地。\leftcitation{נ} \rightfont 今能排程驅馭圖靈以編彙文檔,其碼長共數千千亦無逢大礙.全蒙上帝保守。

\end{paracol}



\columnratio{1}\begin{paracol}{1}

\fontsize{11}{13}\rightfont \Large
~~~~~~~~~~~~~~~~~~~~~~~~~~~~~~~~~~~~~~~~~~~~~~~~~~~~~~~~~~~~~~~~~~~~~~~~~~~~~~~\leftcitation{ר} \rightfont 二零二三年二月一日

~~~~~~~~~~~~~~~~~~~~~~~~~~~~~~~~~~~~~~~~~~~~~~~~~~~~~~~~~~~~~~~~~~~~~~~~~~~~~~~\leftcitation{ש} \rightfont 米迦勒

~~~~~~~~~~~~~~~~~~~~~~~~~~~~~~~~~~~~~~~~~~~~~~~~~~~~~~~~~~~~~~~~~~~~~~~~~~~~~~~\leftcitation{ת} \rightfont 書於香港

\end{paracol}

\end{sloppypar}
\end{document}
